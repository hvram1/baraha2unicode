\documentclass[17pt]{extarticle}
\usepackage{babel}
\usepackage{fontspec}
\usepackage{polyglossia}
\usepackage{extsizes}

\usepackage{color}   %May be necessary if you want to color links
\usepackage{hyperref}
\hypersetup{
    colorlinks=true, %set true if you want colored links
    linktoc=all,     %set to all if you want both sections and subsections linked
    linkcolor=black,  %choose some color if you want links to stand out
}

\setmainlanguage{sanskrit}
\setotherlanguages{english} %% or other languages
\setlength{\parindent}{0pt}
\pagestyle{myheadings}
\newfontfamily\devanagarifont[Script=Devanagari]{AdishilaVedic}
\renewcommand{\theHsection}{\thepart.section.\thesection}

\newcommand{\VAR}[1]{}
\newcommand{\BLOCK}[1]{}




\begin{document}
\begin{titlepage}
    \begin{center}
 
\begin{sanskrit}
    { \Large
    कृष्ण यजुर्वेदीय तैत्तिरीय संहिता,पद,जटा,घन पाठः 
    }
    \\
    \vspace{2.5cm}
    \mbox{ \Large
    4.4      चतुर्थकाण्डे चतुर्थः प्रश्नः - पञ्चमचितिशेषनिरूपणं   }
\end{sanskrit}
\end{center}

\end{titlepage}
\tableofcontents
\phantomsection
\pagebreak

\markright{ TS 4.4.1.1  \hfill https://www.vedavms.in \hfill}

\section{ TS 4.4.1.1 }

\textbf{TS 4.4.1.1 } \newline
\textbf{Samhita Paata} \newline

र॒श्मिर॑सि॒ क्षया॑य त्वा॒ क्षयं॑ जिन्व॒ प्रेति॑रसि॒ धर्मा॑य त्वा॒ धर्मं॑ जि॒न्वान्वि॑तिरसि दि॒वे त्वा॒ दिवं॑ जिन्व स॒न्धिर॑स्य॒न्तरि॑क्षाय त्वा॒ऽन्तरि॑क्षं जिन्व प्रति॒धिर॑सि पृथि॒व्यै त्वा॑ पृथि॒वीं जि॑न्व विष्ट॒भों॑ऽसि॒ वृष्ट्यै᳚ त्वा॒ वृष्टिं॑ जिन्व प्र॒वाऽस्यह्ने॒ त्वाऽह॑र्जिन्वानु॒ वाऽसि॒ रात्रि॑यै त्वा॒ रात्रिं॑ जिन्वो॒ शिग॑सि॒ - [  ] \newline

\textbf{Pada Paata} \newline

र॒श्मिः । अ॒सि॒ । क्षया॑य । त्वा॒ । क्षय᳚म् । जि॒न्व॒ । प्रेति॒रिति॒ प्र - इ॒तिः॒ । अ॒सि॒ । धर्मा॑य । त्वा॒ । धर्म᳚म् । जि॒न्व॒ । अन्वि॑ति॒रित्यनु॑ - इ॒तिः॒ । अ॒सि॒ । दि॒वे । त्वा॒ । दिव᳚म् । जि॒न्व॒ । स॒न्धिरिति॑ सं - धिः । अ॒सि॒ । अ॒न्तरि॑क्षाय । त्वा॒ । अ॒न्तरि॑क्षम् । जि॒न्व॒ । प्र॒ति॒धिरिति॑ प्रति-धिः । अ॒सि॒ । पृ॒थि॒व्यै । त्वा॒ । पृ॒थि॒वीम् । जि॒न्व॒ । वि॒ष्ट॒भं इति॑ वि-स्त॒भंः । अ॒सि॒ । वृष्ट्यै᳚ । त्वा॒ । वृष्टि᳚म् । जि॒न्व॒ । प्र॒वेति॑ प्र - वा । अ॒सि॒ । अह्ने᳚ । त्वा॒ । अहः॑ । जि॒न्व॒ । अ॒नु॒वेत्य॑नु - वा । अ॒सि॒ । रात्रि॑यै । त्वा॒ । रात्रि᳚म् । जि॒न्व॒ । उ॒शिक् । अ॒सि॒ ।  \newline


\textbf{Krama Paata} \newline

र॒श्मिर॑सि । अ॒सि॒ क्षया॑य । क्षया॑य त्वा । त्वा॒ क्षय᳚म् । क्षय॑म् जिन्व । जि॒न्व॒ प्रेतिः॑ । प्रेति॑रसि । प्रेति॒रिति॒ प्र - इ॒तिः॒ । अ॒सि॒ धर्मा॑य । धर्मा॑य त्वा । त्वा॒ धर्म᳚म् । धर्म॑म् जिन्व । जि॒न्वान्वि॑तिः । अन्वि॑तिरसि । अन्वि॑ति॒रित्यनु॑ - इ॒तिः॒ । अ॒सि॒ दि॒वे । दि॒वे त्वा᳚ । त्वा॒ दिव᳚म् । दिव॑म् जिन्व । जि॒न्व॒ स॒न्धिः । स॒न्धिर॑सि । स॒न्धिरिति॑ सम् - धिः । अ॒स्य॒न्तरि॑क्षाय । अ॒न्तरि॑क्षाय त्वा । त्वा॒ऽन्तरि॑क्षम् । अ॒न्तरि॑क्षम् जिन्व । जि॒न्व॒ प्र॒ति॒धिः । प्र॒ति॒धिर॑सि । प्र॒ति॒धिरिति॑ प्रति - धिः । अ॒सि॒ पृ॒थि॒व्यै । पृ॒थि॒व्यै त्वा᳚ । त्वा॒ पृ॒थि॒वीम् । पृ॒थि॒वीम् जि॑न्व । जि॒न्व॒ वि॒ष्ट॒म्भः । वि॒ष्ट॒म्भो॑ऽसि । वि॒ष्ट॒म्भ इति॑ वि - स्त॒म्भः । अ॒सि॒ वृष्ट्यै᳚ । वृष्ट्यै᳚ त्वा । त्वा॒ वृष्टि᳚म् । वृष्टि॑म् जिन्व । जि॒न्व॒ प्र॒वा । प्र॒वाऽसि॑ । प्र॒वेति॑ प्र - वा । अ॒स्यह्ने᳚ । अह्ने᳚ त्वा । त्वाऽहः॑ । अह॑र् जिन्व । जि॒न्वा॒नु॒वा । अ॒नु॒वाऽसि॑ । अ॒नु॒वेत्य॑नु - वा । 
अ॒सि॒ रात्रि॑यै । रात्रि॑यै त्वा । त्वा॒ रात्रि᳚म् । रात्रि॑म् जिन्व । जि॒न्वो॒शिक् । उ॒शिग॑सि । अ॒सि॒ वसु॑भ्यः \newline

\textbf{Jatai Paata} \newline

1. र॒श्मि र॑स्यसि र॒श्मी र॒श्मि र॑सि । \newline
2. अ॒सि॒ क्षया॑य॒ क्षया॑या स्यसि॒ क्षया॑य । \newline
3. क्षया॑य त्वा त्वा॒ क्षया॑य॒ क्षया॑य त्वा । \newline
4. त्वा॒ क्षय॒म् क्षय॑म् त्वा त्वा॒ क्षय᳚म् । \newline
5. क्षय॑म् जिन्व जिन्व॒ क्षय॒म् क्षय॑म् जिन्व । \newline
6. जि॒न्व॒ प्रेतिः॒ प्रेति॑र् जिन्व जिन्व॒ प्रेतिः॑ । \newline
7. प्रेति॑ रस्यसि॒ प्रेतिः॒ प्रेति॑ रसि । \newline
8. प्रेति॒रिति॒ प्र - इ॒तिः॒ । \newline
9. अ॒सि॒ धर्मा॑य॒ धर्मा॑या स्यसि॒ धर्मा॑य । \newline
10. धर्मा॑य त्वा त्वा॒ धर्मा॑य॒ धर्मा॑य त्वा । \newline
11. त्वा॒ धर्म॒म् धर्म॑म् त्वा त्वा॒ धर्म᳚म् । \newline
12. धर्म॑म् जिन्व जिन्व॒ धर्म॒म् धर्म॑म् जिन्व । \newline
13. जि॒न्वान्वि॑ति॒ रन्वि॑तिर् जिन्व जि॒न्वान्वि॑तिः । \newline
14. अन्वि॑ति रस्य॒स्य न्वि॑ति॒ रन्वि॑ति रसि । \newline
15. अन्वि॑ति॒रित्यनु॑ - इ॒तिः॒ । \newline
16. अ॒सि॒ दि॒वे दि॒वे᳚ ऽस्यसि दि॒वे । \newline
17. दि॒वे त्वा᳚ त्वा दि॒वे दि॒वे त्वा᳚ । \newline
18. त्वा॒ दिव॒म् दिव॑म् त्वा त्वा॒ दिव᳚म् । \newline
19. दिव॑म् जिन्व जिन्व॒ दिव॒म् दिव॑म् जिन्व । \newline
20. जि॒न्व॒ स॒न्धिः स॒न्धिर् जि॑न्व जिन्व स॒न्धिः । \newline
21. स॒न्धि र॑स्यसि स॒न्धिः स॒न्धि र॑सि । \newline
22. स॒न्धिरिति॑ सं - धिः । \newline
23. अ॒स्य॒ न्तरि॑क्षाया॒ न्तरि॑क्षाया स्यस्य॒ न्तरि॑क्षाय । \newline
24. अ॒न्तरि॑क्षाय त्वा त्वा॒ ऽन्तरि॑क्षाया॒ न्तरि॑क्षाय त्वा । \newline
25. त्वा॒ ऽन्तरि॑क्ष म॒न्तरि॑क्षम् त्वा त्वा॒ ऽन्तरि॑क्षम् । \newline
26. अ॒न्तरि॑क्षम् जिन्व जिन्वा॒ न्तरि॑क्ष म॒न्तरि॑क्षम् जिन्व । \newline
27. जि॒न्व॒ प्र॒ति॒धिः प्र॑ति॒धिर् जि॑न्व जिन्व प्रति॒धिः । \newline
28. प्र॒ति॒धि र॑स्यसि प्रति॒धिः प्र॑ति॒धि र॑सि । \newline
29. प्र॒ति॒धिरिति॑ प्रति - धिः । \newline
30. अ॒सि॒ पृ॒थि॒व्यै पृ॑थि॒व्या अ॑स्यसि पृथि॒व्यै । \newline
31. पृ॒थि॒व्यै त्वा᳚ त्वा पृथि॒व्यै पृ॑थि॒व्यै त्वा᳚ । \newline
32. त्वा॒ पृ॒थि॒वीम् पृ॑थि॒वीम् त्वा᳚ त्वा पृथि॒वीम् । \newline
33. पृ॒थि॒वीम् जि॑न्व जिन्व पृथि॒वीम् पृ॑थि॒वीम् जि॑न्व । \newline
34. जि॒न्व॒ वि॒ष्टं॒भो वि॑ष्टं॒भो जि॑न्व जिन्व विष्टं॒भः । \newline
35. वि॒ष्टं॒भो᳚ ऽस्यसि विष्टं॒भो वि॑ष्टं॒भो॑ ऽसि । \newline
36. वि॒ष्टं॒भ इति॑ वि - स्तं॒भः । \newline
37. अ॒सि॒ वृष्ट्यै॒ वृष्ट्या॑ अस्यसि॒ वृष्ट्यै᳚ । \newline
38. वृष्ट्यै᳚ त्वा त्वा॒ वृष्ट्यै॒ वृष्ट्यै᳚ त्वा । \newline
39. त्वा॒ वृष्टिं॒ ॅवृष्टि॑म् त्वा त्वा॒ वृष्टि᳚म् । \newline
40. वृष्टि॑म् जिन्व जिन्व॒ वृष्टिं॒ ॅवृष्टि॑म् जिन्व । \newline
41. जि॒न्व॒ प्र॒वा प्र॒वा जि॑न्व जिन्व प्र॒वा । \newline
42. प्र॒वा ऽस्य॑सि प्र॒वा प्र॒वा ऽसि॑ । \newline
43. प्र॒वेति॑ प्र - वा । \newline
44. अ॒स्यह्ने ऽह्ने᳚ ऽस्य॒ स्यह्ने᳚ । \newline
45. अह्ने᳚ त्वा॒ त्वा ऽह्ने ऽह्ने᳚ त्वा । \newline
46. त्वा ऽह॒ रह॑ स्त्वा॒ त्वा ऽहः॑ । \newline
47. अह॑र् जिन्व जि॒न्वाह॒ रह॑र् जिन्व । \newline
48. जि॒न्वा॒ नु॒वा ऽनु॒वा जि॑न्व जिन्वा नु॒वा । \newline
49. अ॒नु॒वा ऽस्य॑स्य नु॒वा ऽनु॒वा ऽसि॑ । \newline
50. अ॒नु॒वेत्य॑नु - वा । \newline
51. अ॒सि॒ रात्रि॑यै॒ रात्रि॑या अस्यसि॒ रात्रि॑यै । \newline
52. रात्रि॑यै त्वा त्वा॒ रात्रि॑यै॒ रात्रि॑यै त्वा । \newline
53. त्वा॒ रात्रिꣳ॒॒ रात्रि॑म् त्वा त्वा॒ रात्रि᳚म् । \newline
54. रात्रि॑म् जिन्व जिन्व॒ रात्रिꣳ॒॒ रात्रि॑म् जिन्व । \newline
55. जि॒न्वो॒शि गु॒शिग् जि॑न्व जिन्वो॒शिक् । \newline
56. उ॒शि ग॑स्य स्यु॒शि गु॒शि ग॑सि । \newline
57. अ॒सि॒ वसु॑भ्यो॒ वसु॑भ्यो ऽस्यसि॒ वसु॑भ्यः । \newline

\textbf{Ghana Paata } \newline

1. र॒श्मि र॑स्यसि र॒श्मी र॒श्मि र॑सि॒ क्षया॑य॒ क्षया॑यासि र॒श्मी र॒श्मि र॑सि॒ क्षया॑य । \newline
2. अ॒सि॒ क्षया॑य॒ क्षया॑ यास्यसि॒ क्षया॑य त्वा त्वा॒ क्षया॑ यास्यसि॒ क्षया॑य त्वा । \newline
3. क्षया॑य त्वा त्वा॒ क्षया॑य॒ क्षया॑य त्वा॒ क्षय॒म् क्षय॑म् त्वा॒ क्षया॑य॒ क्षया॑य त्वा॒ क्षय᳚म् । \newline
4. त्वा॒ क्षय॒म् क्षय॑म् त्वा त्वा॒ क्षय॑म् जिन्व जिन्व॒ क्षय॑म् त्वा त्वा॒ क्षय॑म् जिन्व । \newline
5. क्षय॑म् जिन्व जिन्व॒ क्षय॒म् क्षय॑म् जिन्व॒ प्रेतिः॒ प्रेति॑र् जिन्व॒ क्षय॒म् क्षय॑म् जिन्व॒ प्रेतिः॑ । \newline
6. जि॒न्व॒ प्रेतिः॒ प्रेति॑र् जिन्व जिन्व॒ प्रेति॑ रस्यसि॒ प्रेति॑र् जिन्व जिन्व॒ प्रेति॑रसि । \newline
7. प्रेति॑ रस्यसि॒ प्रेतिः॒ प्रेति॑रसि॒ धर्मा॑य॒ धर्मा॑यासि॒ प्रेतिः॒ प्रेति॑रसि॒ धर्मा॑य । \newline
8. प्रेति॒रिति॒ प्र - इ॒तिः॒ । \newline
9. अ॒सि॒ धर्मा॑य॒ धर्मा॑ यास्यसि॒ धर्मा॑य त्वा त्वा॒ धर्मा॑ यास्यसि॒ धर्मा॑य त्वा । \newline
10. धर्मा॑य त्वा त्वा॒ धर्मा॑य॒ धर्मा॑य त्वा॒ धर्म॒म् धर्म॑म् त्वा॒ धर्मा॑य॒ धर्मा॑य त्वा॒ धर्म᳚म् । \newline
11. त्वा॒ धर्म॒म् धर्म॑म् त्वा त्वा॒ धर्म॑म् जिन्व जिन्व॒ धर्म॑म् त्वा त्वा॒ धर्म॑म् जिन्व । \newline
12. धर्म॑म् जिन्व जिन्व॒ धर्म॒म् धर्म॑म् जि॒न्वान्‌वि॑ति॒ रन्वि॑तिर् जिन्व॒ धर्म॒म् धर्म॑म् जि॒न्वान्‌वि॑तिः । \newline
13. जि॒न्वान्‌ वि॑ति॒ रन्वि॑तिर् जिन्व जि॒न्वान्‌ वि॑ति रस्य॒ स्यन् वि॑तिर् जिन्व जि॒न्वान् वि॑तिरसि । \newline
14. अन्वि॑ति रस्य॒ स्यन्वि॑ति॒ रन्वि॑ति रसि दि॒वे दि॒वे᳚ ऽस्यन्वि॑ति॒ रन्वि॑ति रसि दि॒वे । \newline
15. अन्वि॑ति॒रित्यनु॑ - इ॒तिः॒ । \newline
16. अ॒सि॒ दि॒वे दि॒वे᳚ ऽस्यसि दि॒वे त्वा᳚ त्वा दि॒वे᳚ ऽस्यसि दि॒वे त्वा᳚ । \newline
17. दि॒वे त्वा᳚ त्वा दि॒वे दि॒वे त्वा॒ दिव॒म् दिव॑म् त्वा दि॒वे दि॒वे त्वा॒ दिव᳚म् । \newline
18. त्वा॒ दिव॒म् दिव॑म् त्वा त्वा॒ दिव॑म् जिन्व जिन्व॒ दिव॑म् त्वा त्वा॒ दिव॑म् जिन्व । \newline
19. दिव॑म् जिन्व जिन्व॒ दिव॒म् दिव॑म् जिन्व स॒न्धिः स॒न्धिर् जि॑न्व॒ दिव॒म् दिव॑म् जिन्व स॒न्धिः । \newline
20. जि॒न्व॒ स॒न्धिः स॒न्धिर् जि॑न्व जिन्व स॒न्धि र॑स्यसि स॒न्धिर् जि॑न्व जिन्व स॒न्धि र॑सि । \newline
21. स॒न्धि र॑स्यसि स॒न्धिः स॒न्धि र॑स्य॒ न्तरि॑क्षाया॒ न्तरि॑क्षायासि स॒न्धिः स॒न्धि र॑स्य॒ न्तरि॑क्षाय । \newline
22. स॒न्धिरिति॑ सं - धिः । \newline
23. अ॒स्य॒ न्तरि॑क्षाया॒ न्तरि॑क्षाया स्यस्य॒ न्तरि॑क्षाय त्वा त्वा॒ ऽन्तरि॑क्षाया स्यस्य॒ न्तरि॑क्षाय त्वा । \newline
24. अ॒न्तरि॑क्षाय त्वा त्वा॒ ऽन्तरि॑क्षाया॒ न्तरि॑क्षाय त्वा॒ ऽन्तरि॑क्ष म॒न्तरि॑क्षम् त्वा॒ ऽन्तरि॑क्षाया॒ न्तरि॑क्षाय त्वा॒ ऽन्तरि॑क्षम् । \newline
25. त्वा॒ ऽन्तरि॑क्ष म॒न्तरि॑क्षम् त्वा त्वा॒ ऽन्तरि॑क्षम् जिन्व जिन्वा॒न्तरि॑क्षम् त्वा त्वा॒ ऽन्तरि॑क्षम् जिन्व । \newline
26. अ॒न्तरि॑क्षम् जिन्व जिन्वा॒न्तरि॑क्ष म॒न्तरि॑क्षम् जिन्व प्रति॒धिः प्र॑ति॒धिर् जि॑न्वा॒न्तरि॑क्ष म॒न्तरि॑क्षम् जिन्व प्रति॒धिः । \newline
27. जि॒न्व॒ प्र॒ति॒धिः प्र॑ति॒धिर् जि॑न्व जिन्व प्रति॒धि र॑स्यसि प्रति॒धिर् जि॑न्व जिन्व प्रति॒धिर॑सि । \newline
28. प्र॒ति॒धि र॑स्यसि प्रति॒धिः प्र॑ति॒धि र॑सि पृथि॒व्यै पृ॑थि॒व्या अ॑सि प्रति॒धिः प्र॑ति॒धि र॑सि पृथि॒व्यै । \newline
29. प्र॒ति॒धिरिति॑ प्रति - धिः । \newline
30. अ॒सि॒ पृ॒थि॒व्यै पृ॑थि॒व्या अ॑स्यसि पृथि॒व्यै त्वा᳚ त्वा पृथि॒व्या अ॑स्यसि पृथि॒व्यै त्वा᳚ । \newline
31. पृ॒थि॒व्यै त्वा᳚ त्वा पृथि॒व्यै पृ॑थि॒व्यै त्वा॑ पृथि॒वीम् पृ॑थि॒वीम् त्वा॑ पृथि॒व्यै पृ॑थि॒व्यै त्वा॑ पृथि॒वीम् । \newline
32. त्वा॒ पृ॒थि॒वीम् पृ॑थि॒वीम् त्वा᳚ त्वा पृथि॒वीम् जि॑न्व जिन्व पृथि॒वीम् त्वा᳚ त्वा पृथि॒वीम् जि॑न्व । \newline
33. पृ॒थि॒वीम् जि॑न्व जिन्व पृथि॒वीम् पृ॑थि॒वीम् जि॑न्व विष्टं॒भो वि॑ष्टं॒भो जि॑न्व पृथि॒वीम् पृ॑थि॒वीम् जि॑न्व विष्टं॒भः । \newline
34. जि॒न्व॒ वि॒ष्टं॒भो वि॑ष्टं॒भो जि॑न्व जिन्व विष्टं॒भो᳚ ऽस्यसि विष्टं॒भो जि॑न्व जिन्व विष्टं॒भो॑ ऽसि । \newline
35. वि॒ष्टं॒भो᳚ ऽस्यसि विष्टं॒भो वि॑ष्टं॒भो॑ ऽसि॒ वृष्ट्यै॒ वृष्ट्या॑ असि विष्टं॒भो वि॑ष्टं॒भो॑ ऽसि॒ वृष्ट्यै᳚ । \newline
36. वि॒ष्टं॒भ इति॑ वि - स्तं॒भः । \newline
37. अ॒सि॒ वृष्ट्यै॒ वृष्ट्या॑ अस्यसि॒ वृष्ट्यै᳚ त्वा त्वा॒ वृष्ट्या॑ अस्यसि॒ वृष्ट्यै᳚ त्वा । \newline
38. वृष्ट्यै᳚ त्वा त्वा॒ वृष्ट्यै॒ वृष्ट्यै᳚ त्वा॒ वृष्टिं॒ ॅवृष्टि॑म् त्वा॒ वृष्ट्यै॒ वृष्ट्यै᳚ त्वा॒ वृष्टि᳚म् । \newline
39. त्वा॒ वृष्टिं॒ ॅवृष्टि॑म् त्वा त्वा॒ वृष्टि॑म् जिन्व जिन्व॒ वृष्टि॑म् त्वा त्वा॒ वृष्टि॑म् जिन्व । \newline
40. वृष्टि॑म् जिन्व जिन्व॒ वृष्टिं॒ ॅवृष्टि॑म् जिन्व प्र॒वा प्र॒वा जि॑न्व॒ वृष्टिं॒ ॅवृष्टि॑म् जिन्व प्र॒वा । \newline
41. जि॒न्व॒ प्र॒वा प्र॒वा जि॑न्व जिन्व प्र॒वा ऽस्य॑सि प्र॒वा जि॑न्व जिन्व प्र॒वा ऽसि॑ । \newline
42. प्र॒वा ऽस्य॑सि प्र॒वा प्र॒वा ऽस्यह्ने ऽह्ने॑ ऽसि प्र॒वा प्र॒वा ऽस्यह्ने᳚ । \newline
43. प्र॒वेति॑ प्र - वा । \newline
44. अ॒स्यह्ने ऽह्ने᳚ ऽस्य॒ स्यह्ने᳚ त्वा॒ त्वा ऽह्ने᳚ ऽस्य॒ स्यह्ने᳚ त्वा । \newline
45. अह्ने᳚ त्वा॒ त्वा ऽह्ने ऽह्ने॒ त्वा ऽह॒ रह॒ स्त्वा ऽह्ने ऽह्ने॒ त्वा ऽहः॑ । \newline
46. त्वा ऽह॒ रह॑ स्त्वा॒ त्वा ऽह॑र् जिन्व जि॒न्वाह॑ स्त्वा॒ त्वा ऽह॑र् जिन्व । \newline
47. अह॑र् जिन्व जि॒न्वाह॒ रह॑र् जिन्वानु॒वा ऽनु॒वा जि॒न्वाह॒ रह॑र् जिन्वानु॒वा । \newline
48. जि॒न्वा॒नु॒वा ऽनु॒वा जि॑न्व जिन्वानु॒वा ऽस्य॑ स्यनु॒वा जि॑न्व जिन्वानु॒वा ऽसि॑ । \newline
49. अ॒नु॒वा ऽस्य॑ स्यनु॒वा ऽनु॒वा ऽसि॒ रात्रि॑यै॒ रात्रि॑या अस्यनु॒वा ऽनु॒वा ऽसि॒ रात्रि॑यै । \newline
50. अ॒नु॒वेत्य॑नु - वा । \newline
51. अ॒सि॒ रात्रि॑यै॒ रात्रि॑या अस्यसि॒ रात्रि॑यै त्वा त्वा॒ रात्रि॑या अस्यसि॒ रात्रि॑यै त्वा । \newline
52. रात्रि॑यै त्वा त्वा॒ रात्रि॑यै॒ रात्रि॑यै त्वा॒ रात्रिꣳ॒॒ रात्रि॑म् त्वा॒ रात्रि॑यै॒ रात्रि॑यै त्वा॒ रात्रि᳚म् । \newline
53. त्वा॒ रात्रिꣳ॒॒ रात्रि॑म् त्वा त्वा॒ रात्रि॑म् जिन्व जिन्व॒ रात्रि॑म् त्वा त्वा॒ रात्रि॑म् जिन्व । \newline
54. रात्रि॑म् जिन्व जिन्व॒ रात्रिꣳ॒॒ रात्रि॑म् जिन्वो॒शि गु॒शिग् जि॑न्व॒ रात्रिꣳ॒॒ रात्रि॑म् जिन्वो॒शिक् । \newline
55. जि॒न्वो॒शि गु॒शिग् जि॑न्व जिन्वो॒शि ग॑स्य स्यु॒शिग् जि॑न्व जिन्वो॒शि ग॑सि । \newline
56. उ॒शि ग॑स्यस्यु॒शि गु॒शि ग॑सि॒ वसु॑भ्यो॒ वसु॑भ्यो ऽस्यु॒शि गु॒शि ग॑सि॒ वसु॑भ्यः । \newline
57. अ॒सि॒ वसु॑भ्यो॒ वसु॑भ्यो ऽस्यसि॒ वसु॑भ्य स्त्वा त्वा॒ वसु॑भ्यो ऽस्यसि॒ वसु॑भ्य स्त्वा । \newline
\pagebreak
\markright{ TS 4.4.1.2  \hfill https://www.vedavms.in \hfill}

\section{ TS 4.4.1.2 }

\textbf{TS 4.4.1.2 } \newline
\textbf{Samhita Paata} \newline

वसु॑भ्यस्त्वा॒ वसू᳚ञ्जिन्व प्रके॒तो॑ऽसि रु॒द्रेभ्य॑स्त्वा रु॒द्राञ्जि॑न्व सुदी॒तिर॑स्यादि॒त्येभ्य॑स्त्वा ऽऽदि॒त्याञ्जि॒न्वौजो॑ऽसि पि॒तृभ्य॑स्त्वा पि॒तॄञ्जि॑न्व॒ तन्तु॑रसि प्र॒जाभ्य॑स्त्वा प्र॒जा जि॑न्व पृतना॒षाड॑सि प॒शुभ्य॑स्त्वा प॒शूञ्जि॑न्व रे॒वद॒स्योष॑धीभ्य॒-स्त्वौष॑धी-र्जिन्वाभि॒जिद॑सि यु॒क्तग्रा॒वेन्द्रा॑य॒ त्वेन्द्रं॑ जि॒न्वाधि॑पतिरसि प्रा॒णाय॑ - [  ] \newline

\textbf{Pada Paata} \newline

वसु॑भ्य॒ इति॒ वसु॑ - भ्यः॒ । त्वा॒ । वसून्॑ । जि॒न्व॒ । प्र॒के॒त इति॑ प्र - के॒तः । अ॒सि॒ । रु॒द्रेभ्यः॑ । त्वा॒ । रु॒द्रान् । जि॒न्व॒ । सु॒दी॒तिरिति॑ सु- दी॒तिः । अ॒सि॒ । आ॒दि॒त्येभ्यः॑ । त्वा॒ । आ॒दि॒त्यान् । जि॒न्व॒ । ओजः॑ । अ॒सि॒ । पि॒तृभ्य॒ इति॑ पि॒तृ - भ्यः॒ । त्वा॒ । पि॒तॄन् । जि॒न्व॒ । तन्तुः॑ । अ॒सि॒ । प्र॒जाभ्य॒ इति॑ प्र - जाभ्यः॑ । त्वा॒ । प्र॒जा इति॑ प्र - जाः । जि॒न्व॒ । पृ॒त॒ना॒षाट् । अ॒सि॒ । प॒शुभ्य॒ इति॑ प॒शु - भ्यः॒ । त्वा॒ । प॒शून् । जि॒न्व॒ । रे॒वत् । अ॒सि॒ । ओष॑धीभ्य॒ इत्योष॑धि - भ्यः॒ । त्वा॒ । ओष॑धीः । जि॒न्व॒ । अ॒भि॒जिदित्य॑भि - जित् । अ॒सि॒ । यु॒क्तग्रा॒वेति॑ यु॒क्त - ग्रा॒वा॒ । इन्द्रा॑य । त्वा॒ । इन्द्र᳚म् । जि॒न्व॒ । अधि॑पति॒रित्यधि॑-प॒तिः॒ । अ॒सि॒ । प्रा॒णायेति॑ प्र-अ॒नाय॑ ।  \newline


\textbf{Krama Paata} \newline

वसु॑भ्यस्त्वा । वसु॑भ्य॒ इति॒ वसु॑ - भ्यः॒ । त्वा॒ वसून्॑ । वसू᳚न् जिन्व । जि॒न्व॒ प्र॒के॒तः । प्र॒के॒तो॑ऽसि । प्र॒के॒त इति॑ प्र - के॒तः । अ॒सि॒ रु॒द्रेभ्यः॑ । रु॒द्रेभ्य॑स्त्वा । त्वा॒ रु॒द्रान् । रु॒द्रान् जि॑न्व । जि॒न्व॒ सु॒दी॒तिः । सु॒दी॒तिर॑सि । सु॒दी॒तिरिति॑ सु - दी॒तिः । अ॒स्या॒दि॒त्येभ्यः॑ । आ॒दि॒त्येभ्य॑स्त्वा । त्वा॒ऽऽदि॒त्यान् । आ॒दि॒त्यान् जि॑न्व । जि॒न्वौजः॑ । ओजो॑ऽसि । अ॒सि॒ पि॒तृभ्यः॑ । पि॒तृभ्य॑स्त्वा । पि॒तृभ्य॒ इति॑ पि॒तृ - भ्यः॒ । त्वा॒ पि॒तॄन् । पि॒तॄन् जि॑न्व । जि॒न्व॒ तन्तुः॑ । तन्तु॑रसि । अ॒सि॒ प्र॒जाभ्यः॑ । प्र॒जाभ्य॑स्त्वा । प्र॒जाभ्य॒ इति॑ प्र - जाभ्यः॑ । त्वा॒ प्र॒जाः । प्र॒जा जि॑न्व । प्र॒जा इति॑ प्र - जाः । जि॒न्व॒ पृ॒त॒ना॒षाट् । पृ॒त॒ना॒षाड॑सि । अ॒सि॒ प॒शुभ्यः॑ । प॒शुभ्य॑स्त्वा । प॒शुभ्य॒ इति॑ प॒शु - भ्यः॒ । त्वा॒ प॒शून् । प॒शून् जि॑न्व । जि॒न्व॒ रे॒वत् । रे॒वद॑सि । अ॒स्योष॑धीभ्यः । ओष॑धीभ्यस्त्वा । ओष॑धीभ्य॒ इत्योष॑धि - भ्यः॒ । त्वौष॑धीः । ओष॑धीर् जिन्व । जि॒न्वा॒भि॒जित् । अ॒भि॒जिद॑सि । अ॒भि॒जिदित्य॑भि - जित् । अ॒सि॒ यु॒क्तग्रा॑वा । यु॒क्तग्रा॒वेन्द्रा॑य । यु॒क्तग्रा॒वेति॑ यु॒क्त - ग्रा॒वा॒ । इन्द्रा॑य त्वा । त्वेन्द्र᳚म् । इन्द्र॑म् जिन्व । जि॒न्वाधि॑पतिः । अधि॑पतिरसि । अधि॑पति॒रित्यधि॑ - प॒तिः॒ । अ॒सि॒ प्रा॒णाय॑ ( ) । प्रा॒णाय॑ त्वा । प्रा॒णायेति॑ प्र - अ॒नाय॑ \newline

\textbf{Jatai Paata} \newline

1. वसु॑भ्य स्त्वा त्वा॒ वसु॑भ्यो॒ वसु॑भ्य स्त्वा । \newline
2. वसु॑भ्य॒ इति॒ वसु॑ - भ्यः॒ । \newline
3. त्वा॒ वसू॒न्॒. वसू᳚न् त्वा त्वा॒ वसून्॑ । \newline
4. वसू᳚न् जिन्व जिन्व॒ वसू॒न्॒. वसू᳚न् जिन्व । \newline
5. जि॒न्व॒ प्र॒के॒तः प्र॑के॒तो जि॑न्व जिन्व प्रके॒तः । \newline
6. प्र॒के॒तो᳚ ऽस्यसि प्रके॒तः प्र॑के॒तो॑ ऽसि । \newline
7. प्र॒के॒त इति॑ प्र - के॒तः । \newline
8. अ॒सि॒ रु॒द्रेभ्यो॑ रु॒द्रेभ्यो᳚ ऽस्यसि रु॒द्रेभ्यः॑ । \newline
9. रु॒द्रेभ्य॑ स्त्वा त्वा रु॒द्रेभ्यो॑ रु॒द्रेभ्य॑ स्त्वा । \newline
10. त्वा॒ रु॒द्रान् रु॒द्रान् त्वा᳚ त्वा रु॒द्रान् । \newline
11. रु॒द्रान् जि॑न्व जिन्व रु॒द्रान् रु॒द्रान् जि॑न्व । \newline
12. जि॒न्व॒ सु॒दी॒तिः सु॑दी॒तिर् जि॑न्व जिन्व सुदी॒तिः । \newline
13. सु॒दी॒ति र॑स्यसि सुदी॒तिः सु॑दी॒ति र॑सि । \newline
14. सु॒दी॒तिरिति॑ सु - दी॒तिः । \newline
15. अ॒स्या॒ दि॒त्येभ्य॑ आदि॒त्येभ्यो᳚ ऽस्यस्या दि॒त्येभ्यः॑ । \newline
16. आ॒दि॒त्येभ्य॑ स्त्वा त्वा ऽऽदि॒त्येभ्य॑ आदि॒त्येभ्य॑ स्त्वा । \newline
17. त्वा॒ ऽऽदि॒त्या-ना॑दि॒त्यान् त्वा᳚ त्वा ऽऽदि॒त्यान् । \newline
18. आ॒दि॒त्यान् जि॑न्व जिन्वादि॒त्या-ना॑दि॒त्यान् जि॑न्व । \newline
19. जि॒न्वौज॒ ओजो॑ जिन्व जि॒न्वौजः॑ । \newline
20. ओजो᳚ ऽस्य॒ स्योज॒ ओजो॑ ऽसि । \newline
21. अ॒सि॒ पि॒तृभ्यः॑ पि॒तृभ्यो᳚ ऽस्यसि पि॒तृभ्यः॑ । \newline
22. पि॒तृभ्य॑ स्त्वा त्वा पि॒तृभ्यः॑ पि॒तृभ्य॑ स्त्वा । \newline
23. पि॒तृभ्य॒ इति॑ पि॒तृ - भ्यः॒ । \newline
24. त्वा॒ पि॒तॄन् पि॒तॄꣳस् त्वा᳚ त्वा पि॒तॄन् । \newline
25. पि॒तॄन् जि॑न्व जिन्व पि॒तॄन् पि॒तॄन् जि॑न्व । \newline
26. जि॒न्व॒ तन्तु॒ स्तन्तु॑र् जिन्व जिन्व॒ तन्तुः॑ । \newline
27. तन्तु॑र स्यसि॒ तन्तु॒ स्तन्तु॑ रसि । \newline
28. अ॒सि॒ प्र॒जाभ्यः॑ प्र॒जाभ्यो᳚ ऽस्यसि प्र॒जाभ्यः॑ । \newline
29. प्र॒जाभ्य॑ स्त्वा त्वा प्र॒जाभ्यः॑ प्र॒जाभ्य॑ स्त्वा । \newline
30. प्र॒जाभ्य॒ इति॑ प्र - जाभ्यः॑ । \newline
31. त्वा॒ प्र॒जाः प्र॒जा स्त्वा᳚ त्वा प्र॒जाः । \newline
32. प्र॒जा जि॑न्व जिन्व प्र॒जाः प्र॒जा जि॑न्व । \newline
33. प्र॒जा इति॑ प्र - जाः । \newline
34. जि॒न्व॒ पृ॒त॒ना॒षाट् पृ॑तना॒षाड् जि॑न्व जिन्व पृतना॒षाट् । \newline
35. पृ॒त॒ना॒षा ड॑स्यसि पृतना॒षाट् पृ॑तना॒षा ड॑सि । \newline
36. अ॒सि॒ प॒शुभ्यः॑ प॒शुभ्यो᳚ ऽस्यसि प॒शुभ्यः॑ । \newline
37. प॒शुभ्य॑ स्त्वा त्वा प॒शुभ्यः॑ प॒शुभ्य॑ स्त्वा । \newline
38. प॒शुभ्य॒ इति॑ प॒शु - भ्यः॒ । \newline
39. त्वा॒ प॒शून् प॒शून् त्वा᳚ त्वा प॒शून् । \newline
40. प॒शून् जि॑न्व जिन्व प॒शून् प॒शून् जि॑न्व । \newline
41. जि॒न्व॒ रे॒वद् रे॒वज् जि॑न्व जिन्व रे॒वत् । \newline
42. रे॒व द॑स्यसि रे॒वद् रे॒व द॑सि । \newline
43. अ॒स्योष॑धीभ्य॒ ओष॑धीभ्यो ऽस्य॒ स्योष॑धीभ्यः । \newline
44. ओष॑धीभ्य स्त्वा॒ त्वौष॑धीभ्य॒ ओष॑धीभ्य स्त्वा । \newline
45. ओष॑धीभ्य॒ इत्योष॑धि - भ्यः॒ । \newline
46. त्वौष॑धी॒ रोष॑धी स्त्वा॒ त्वौष॑धीः । \newline
47. ओष॑धीर् जिन्व जि॒न्वौ ष॑धी॒ रोष॑धीर् जिन्व । \newline
48. जि॒न्वा॒ भि॒जि द॑भि॒जिज् जि॑न्व जिन्वा भि॒जित् । \newline
49. अ॒भि॒जि द॑स्यस्य भि॒जिद॑ भि॒जिद॑सि । \newline
50. अ॒भि॒जिदित्य॑भि - जित् । \newline
51. अ॒सि॒ यु॒क्तग्रा॑वा यु॒क्तग्रा॑वा ऽस्यसि यु॒क्तग्रा॑वा । \newline
52. यु॒क्तग्रा॒ वेन्द्रा॒ येन्द्रा॑य यु॒क्तग्रा॑वा यु॒क्तग्रा॒ वेन्द्रा॑य । \newline
53. यु॒क्तग्रा॒वेति॑ यु॒क्त - ग्रा॒वा॒ । \newline
54. इन्द्रा॑य त्वा॒ त्वेन्द्रा॒ येन्द्रा॑य त्वा । \newline
55. त्वेन्द्र॒ मिन्द्र॑म् त्वा॒ त्वेन्द्र᳚म् । \newline
56. इन्द्र॑म् जिन्व जि॒न्वेन्द्र॒ मिन्द्र॑म् जिन्व । \newline
57. जि॒न्वा धि॑पति॒ रधि॑पतिर् जिन्व जि॒न्वा धि॑पतिः । \newline
58. अधि॑पति रस्य॒स्य धि॑पति॒ रधि॑पति रसि । \newline
59. अधि॑पति॒रित्यधि॑ - प॒तिः॒ । \newline
60. अ॒सि॒ प्रा॒णाय॑ प्रा॒णाया᳚स्यसि प्रा॒णाय॑ । \newline
61. प्रा॒णाय॑ त्वा त्वा प्रा॒णाय॑ प्रा॒णाय॑ त्वा । \newline
62. प्रा॒णायेति॑ प्र - अ॒नाय॑ । \newline

\textbf{Ghana Paata } \newline

1. वसु॑भ्य स्त्वा त्वा॒ वसु॑भ्यो॒ वसु॑भ्य स्त्वा॒ वसू॒न्॒. वसू᳚न् त्वा॒ वसु॑भ्यो॒ वसु॑भ्य स्त्वा॒ वसून्॑ । \newline
2. वसु॑भ्य॒ इति॒ वसु॑ - भ्यः॒ । \newline
3. त्वा॒ वसू॒न्॒. वसू᳚न् त्वा त्वा॒ वसू᳚न् जिन्व जिन्व॒ वसू᳚न् त्वा त्वा॒ वसू᳚न् जिन्व । \newline
4. वसू᳚न् जिन्व जिन्व॒ वसू॒न्॒. वसू᳚न् जिन्व प्रके॒तः प्र॑के॒तो जि॑न्व॒ वसू॒न्॒. वसू᳚न् जिन्व प्रके॒तः । \newline
5. जि॒न्व॒ प्र॒के॒तः प्र॑के॒तो जि॑न्व जिन्व प्रके॒तो᳚ ऽस्यसि प्रके॒तो जि॑न्व जिन्व प्रके॒तो॑ ऽसि । \newline
6. प्र॒के॒तो᳚ ऽस्यसि प्रके॒तः प्र॑के॒तो॑ ऽसि रु॒द्रेभ्यो॑ रु॒द्रेभ्यो॑ ऽसि प्रके॒तः प्र॑के॒तो॑ ऽसि रु॒द्रेभ्यः॑ । \newline
7. प्र॒के॒त इति॑ प्र - के॒तः । \newline
8. अ॒सि॒ रु॒द्रेभ्यो॑ रु॒द्रेभ्यो᳚ ऽस्यसि रु॒द्रेभ्य॑ स्त्वा त्वा रु॒द्रेभ्यो᳚ ऽस्यसि रु॒द्रेभ्य॑ स्त्वा । \newline
9. रु॒द्रेभ्य॑ स्त्वा त्वा रु॒द्रेभ्यो॑ रु॒द्रेभ्य॑ स्त्वा रु॒द्रान् रु॒द्रान् त्वा॑ रु॒द्रेभ्यो॑ रु॒द्रेभ्य॑ स्त्वा रु॒द्रान् । \newline
10. त्वा॒ रु॒द्रान् रु॒द्रान् त्वा᳚ त्वा रु॒द्रान् जि॑न्व जिन्व रु॒द्रान् त्वा᳚ त्वा रु॒द्रान् जि॑न्व । \newline
11. रु॒द्रान् जि॑न्व जिन्व रु॒द्रान् रु॒द्रान् जि॑न्व सुदी॒तिः सु॑दी॒तिर् जि॑न्व रु॒द्रान् रु॒द्रान् जि॑न्व सुदी॒तिः । \newline
12. जि॒न्व॒ सु॒दी॒तिः सु॑दी॒तिर् जि॑न्व जिन्व सुदी॒ति र॑स्यसि सुदी॒तिर् जि॑न्व जिन्व सुदी॒तिर॑सि । \newline
13. सु॒दी॒ति र॑स्यसि सुदी॒तिः सु॑दी॒ति र॑स्या दि॒त्येभ्य॑ आदि॒त्येभ्यो॑ ऽसि सुदी॒तिः सु॑दी॒ति र॑स्या दि॒त्येभ्यः॑ । \newline
14. सु॒दी॒तिरिति॑ सु - दी॒तिः । \newline
15. अ॒स्या॒ दि॒त्येभ्य॑ आदि॒त्येभ्यो᳚ ऽस्यस्या दि॒त्येभ्य॑ स्त्वा त्वा ऽऽदि॒त्येभ्यो᳚ ऽस्यस्या दि॒त्येभ्य॑ स्त्वा । \newline
16. आ॒दि॒त्येभ्य॑ स्त्वा त्वा ऽऽदि॒त्येभ्य॑ आदि॒त्येभ्य॑ स्त्वा ऽऽदि॒त्या ना॑दि॒त्यान् त्वा॑ ऽऽदि॒त्येभ्य॑ आदि॒त्येभ्य॑ स्त्वा ऽऽदि॒त्यान् । \newline
17. त्वा॒ ऽऽदि॒त्या ना॑दि॒त्यान् त्वा᳚ त्वा ऽऽदि॒त्यान् जि॑न्व जिन्वादि॒त्यान् त्वा᳚ त्वा ऽऽदि॒त्यान् जि॑न्व । \newline
18. आ॒दि॒त्यान् जि॑न्व जिन्वादि॒त्या ना॑दि॒त्यान् जि॒न्वौज॒ ओजो॑ जिन्वादि॒त्या ना॑दि॒त्यान् जि॒न्वौजः॑ । \newline
19. जि॒न्वौज॒ ओजो॑ जिन्व जि॒न्वौजो᳚ ऽस्य॒ स्योजो॑ जिन्व जि॒न्वौजो॑ ऽसि । \newline
20. ओजो᳚ ऽस्य॒ स्योज॒ ओजो॑ ऽसि पि॒तृभ्यः॑ पि॒तृभ्यो॒ ऽस्योज॒ ओजो॑ ऽसि पि॒तृभ्यः॑ । \newline
21. अ॒सि॒ पि॒तृभ्यः॑ पि॒तृभ्यो᳚ ऽस्यसि पि॒तृभ्य॑ स्त्वा त्वा पि॒तृभ्यो᳚ ऽस्यसि पि॒तृभ्य॑ स्त्वा । \newline
22. पि॒तृभ्य॑ स्त्वा त्वा पि॒तृभ्यः॑ पि॒तृभ्य॑ स्त्वा पि॒तॄन् पि॒तॄꣳ स्त्वा॑ पि॒तृभ्यः॑ पि॒तृभ्य॑ स्त्वा पि॒तॄन् । \newline
23. पि॒तृभ्य॒ इति॑ पि॒तृ - भ्यः॒ । \newline
24. त्वा॒ पि॒तॄन् पि॒तॄꣳ स्त्वा᳚ त्वा पि॒तॄन् जि॑न्व जिन्व पि॒तॄꣳ स्त्वा᳚ त्वा पि॒तॄन् जि॑न्व । \newline
25. पि॒तॄन् जि॑न्व जिन्व पि॒तॄन् पि॒तॄन् जि॑न्व॒ तन्तु॒ स्तन्तु॑र् जिन्व पि॒तॄन् पि॒तॄन् जि॑न्व॒ तन्तुः॑ । \newline
26. जि॒न्व॒ तन्तु॒ स्तन्तु॑र् जिन्व जिन्व॒ तन्तु॑ रस्यसि॒ तन्तु॑र् जिन्व जिन्व॒ तन्तु॑ रसि । \newline
27. तन्तु॑ रस्यसि॒ तन्तु॒ स्तन्तु॑ रसि प्र॒जाभ्यः॑ प्र॒जाभ्यो॑ ऽसि॒ तन्तु॒ स्तन्तु॑ रसि प्र॒जाभ्यः॑ । \newline
28. अ॒सि॒ प्र॒जाभ्यः॑ प्र॒जाभ्यो᳚ ऽस्यसि प्र॒जाभ्य॑ स्त्वा त्वा प्र॒जाभ्यो᳚ ऽस्यसि प्र॒जाभ्य॑ स्त्वा । \newline
29. प्र॒जाभ्य॑ स्त्वा त्वा प्र॒जाभ्यः॑ प्र॒जाभ्य॑ स्त्वा प्र॒जाः प्र॒जा स्त्वा᳚ प्र॒जाभ्यः॑ प्र॒जाभ्य॑ स्त्वा प्र॒जाः । \newline
30. प्र॒जाभ्य॒ इति॑ प्र - जाभ्यः॑ । \newline
31. त्वा॒ प्र॒जाः प्र॒जा स्त्वा᳚ त्वा प्र॒जा जि॑न्व जिन्व प्र॒जा स्त्वा᳚ त्वा प्र॒जा जि॑न्व । \newline
32. प्र॒जा जि॑न्व जिन्व प्र॒जाः प्र॒जा जि॑न्व पृतना॒षाट् पृ॑तना॒षाड् जि॑न्व प्र॒जाः प्र॒जा जि॑न्व पृतना॒षाट् । \newline
33. प्र॒जा इति॑ प्र - जाः । \newline
34. जि॒न्व॒ पृ॒त॒ना॒षाट् पृ॑तना॒षाड् जि॑न्व जिन्व पृतना॒षा ड॑स्यसि पृतना॒षाड् जि॑न्व जिन्व पृतना॒षाड॑सि । \newline
35. पृ॒त॒ना॒षा ड॑स्यसि पृतना॒षाट् पृ॑तना॒षाड॑सि प॒शुभ्यः॑ प॒शुभ्यो॑ ऽसि पृतना॒षाट् पृ॑तना॒षाड॑सि प॒शुभ्यः॑ । \newline
36. अ॒सि॒ प॒शुभ्यः॑ प॒शुभ्यो᳚ ऽस्यसि प॒शुभ्य॑ स्त्वा त्वा प॒शुभ्यो᳚ ऽस्यसि प॒शुभ्य॑ स्त्वा । \newline
37. प॒शुभ्य॑ स्त्वा त्वा प॒शुभ्यः॑ प॒शुभ्य॑ स्त्वा प॒शून् प॒शून् त्वा॑ प॒शुभ्यः॑ प॒शुभ्य॑ स्त्वा प॒शून् । \newline
38. प॒शुभ्य॒ इति॑ प॒शु - भ्यः॒ । \newline
39. त्वा॒ प॒शून् प॒शून् त्वा᳚ त्वा प॒शून् जि॑न्व जिन्व प॒शून् त्वा᳚ त्वा प॒शून् जि॑न्व । \newline
40. प॒शून् जि॑न्व जिन्व प॒शून् प॒शून् जि॑न्व रे॒वद् रे॒वज् जि॑न्व प॒शून् प॒शून् जि॑न्व रे॒वत् । \newline
41. जि॒न्व॒ रे॒वद् रे॒वज् जि॑न्व जिन्व रे॒व द॑स्यसि रे॒वज् जि॑न्व जिन्व रे॒वद॑सि । \newline
42. रे॒व द॑स्यसि रे॒वद् रे॒व द॒स्योष॑धीभ्य॒ ओष॑धीभ्यो ऽसि रे॒वद् रे॒व द॒स्योष॑धीभ्यः । \newline
43. अ॒स्योष॑धीभ्य॒ ओष॑धीभ्यो ऽस्य॒ स्योष॑धीभ्य स्त्वा॒ त्वौष॑धीभ्यो ऽस्य॒स्योष॑धीभ्य स्त्वा । \newline
44. ओष॑धीभ्य स्त्वा॒ त्वौष॑धीभ्य॒ ओष॑धीभ्य॒ स्त्वौष॑धी॒ रोष॑धी॒ स्त्वौष॑धीभ्य॒ ओष॑धीभ्य॒ 
स्त्वौष॑धीः । \newline
45. ओष॑धीभ्य॒ इत्योष॑धि - भ्यः॒ । \newline
46. त्वौष॑धी॒ रोष॑धी स्त्वा॒ त्वौष॑धीर् जिन्व जि॒न्वौष॑धी स्त्वा॒ त्वौष॑धीर् जिन्व । \newline
47. ओष॑धीर् जिन्व जि॒न्वौष॑धी॒ रोष॑धीर् जिन्वा भि॒जि द॑भि॒जिज् जि॒न्वौष॑धी॒ रोष॑धीर् जिन्वा भि॒जित् । \newline
48. जि॒न्वा॒ भि॒जि द॑भि॒जिज् जि॑न्व जिन्वा भि॒जिद॑ स्यस्य भि॒जिज् जि॑न्व जिन्वा भि॒जि द॑सि । \newline
49. अ॒भि॒जि द॑स्यस्य भि॒जि द॑भि॒जिद॑सि यु॒क्तग्रा॑वा यु॒क्तग्रा॑वा ऽस्यभि॒जि द॑भि॒जि द॑सि यु॒क्तग्रा॑वा । \newline
50. अ॒भि॒जिदित्य॑भि - जित् । \newline
51. अ॒सि॒ यु॒क्तग्रा॑वा यु॒क्तग्रा॑वा ऽस्यसि यु॒क्तग्रा॒ वेन्द्रा॒ येन्द्रा॑य यु॒क्तग्रा॑वा ऽस्यसि यु॒क्तग्रा॒ वेन्द्रा॑य । \newline
52. यु॒क्तग्रा॒ वेन्द्रा॒ येन्द्रा॑य यु॒क्तग्रा॑वा यु॒क्तग्रा॒ वेन्द्रा॑य त्वा॒ त्वेन्द्रा॑य यु॒क्तग्रा॑वा यु॒क्तग्रा॒ वेन्द्रा॑य त्वा । \newline
53. यु॒क्तग्रा॒वेति॑ यु॒क्त - ग्रा॒वा॒ । \newline
54. इन्द्रा॑य त्वा॒ त्वेन्द्रा॒ येन्द्रा॑य॒ त्वेन्द्र॒ मिन्द्र॒म् त्वेन्द्रा॒ येन्द्रा॑य॒ त्वेन्द्र᳚म् । \newline
55. वेन्द्र॒ मिन्द्र॑म् त्वा॒ त्वेन्द्र॑म् जिन्व जि॒न्वे न्द्र॑म् त्वा॒ त्वेन्द्र॑म् जिन्व । \newline
56. इन्द्र॑म् जिन्व जि॒न्वेन्द्र॒ मिन्द्र॑म् जि॒न्वा धि॑पति॒ रधि॑पतिर् जि॒न्वेन्द्र॒ मिन्द्र॑म् जि॒न्वा धि॑पतिः । \newline
57. जि॒न्वा धि॑पति॒ रधि॑पतिर् जिन्व जि॒न्वा धि॑पति रस्य॒ स्यधि॑पतिर् जिन्व जि॒न्वा धि॑पति रसि । \newline
58. अधि॑पति रस्य॒ स्यधि॑पति॒ रधि॑पति रसि प्रा॒णाय॑ प्रा॒णाया॒ स्यधि॑पति॒ रधि॑पति रसि प्रा॒णाय॑ । \newline
59. अधि॑पति॒रित्यधि॑ - प॒तिः॒ । \newline
60. अ॒सि॒ प्रा॒णाय॑ प्रा॒णाया᳚ स्यसि प्रा॒णाय॑ त्वा त्वा प्रा॒णाया᳚ स्यसि प्रा॒णाय॑ त्वा । \newline
61. प्रा॒णाय॑ त्वा त्वा प्रा॒णाय॑ प्रा॒णाय॑ त्वा प्रा॒णम् प्रा॒णम् त्वा᳚ प्रा॒णाय॑ प्रा॒णाय॑ त्वा प्रा॒णम् । \newline
62. प्रा॒णायेति॑ प्र - अ॒नाय॑ । \newline
\pagebreak
\markright{ TS 4.4.1.3  \hfill https://www.vedavms.in \hfill}

\section{ TS 4.4.1.3 }

\textbf{TS 4.4.1.3 } \newline
\textbf{Samhita Paata} \newline

त्वा प्रा॒णं जि॑न्व य॒न्ताऽस्य॑पा॒नाय॑ त्वाऽपा॒नं जि॑न्व सꣳ॒॒सर्पो॑ऽसि॒ चक्षु॑षे त्वा॒ चक्षु॑र्जिन्व वयो॒धा अ॑सि॒ श्रोत्रा॑य त्वा॒ श्रोत्रं॑ जिन्व त्रि॒वृद॑सि प्र॒वृद॑सि सं॒ॅवृद॑सि वि॒वृद॑सि सꣳरो॒हो॑ऽसि नीरो॒हो॑ऽसि प्ररो॒हो᳚ऽस्यनुरो॒हो॑ऽसि वसु॒को॑ऽसि॒ वेष॑श्रिरसि॒ वस्य॑ष्टिरसि ॥ \newline

\textbf{Pada Paata} \newline

त्वा॒ । प्रा॒णमिति॑ प्र - अ॒नम् । जि॒न्व॒ । य॒न्ता । अ॒सि॒ । अ॒पा॒नायेत्य॑प - अ॒नाय॑ । त्वा॒ । अ॒पा॒नमित्य॑प - अ॒नम् । जि॒न्व॒ । सꣳ॒॒सर्प॒ इति॑ सं - सर्पः॑ । अ॒सि॒ । चक्षु॑षे । त्वा॒ । चक्षुः॑ । जि॒न्व॒ । व॒यो॒धा इति॑ वयो - धाः । अ॒सि॒ । श्रोत्रा॑य । त्वा॒ । श्रोत्र᳚म् । जि॒न्व॒ । त्रि॒वृदिति॑ त्रि - वृत् । अ॒सि॒ । प्र॒वृदिति॑ प्र - वृत् । अ॒सि॒ । सं॒ॅवृदिति॑ सं - वृत् । अ॒सि॒ । वि॒वृदिति वि - वृत् । अ॒सि॒ । सꣳ॒॒रो॒ह इति॑ सं - रो॒हः । अ॒सि॒ । नी॒रो॒ह इति॑ निः - रो॒हः । अ॒सि॒ । प्र॒रो॒ह इति॑ प्र - रो॒हः । अ॒सि॒ । अ॒नु॒रो॒ह इत्य॑नु - रो॒हः । अ॒सि॒ । व॒सु॒कः । अ॒सि॒ । वेष॑श्रि॒रिति॒ वेष॑ - श्रिः॒ । अ॒सि॒ । वस्य॑ष्टिः । अ॒सि॒ ॥  \newline


\textbf{Krama Paata} \newline

त्वा॒ प्रा॒णम् । प्रा॒णम् जि॑न्व । प्रा॒णमिति॑ प्र - अ॒नम् । जि॒न्व॒ य॒न्ता । य॒न्ताऽसि॑ । अ॒स्य॒पा॒नाय॑ । अ॒पा॒नाय॑ त्वा । अ॒पा॒नायेत्य॑प - अ॒नाय॑ । त्वा॒ऽपा॒नम् । अ॒पा॒नम् जि॑न्व । अ॒पा॒नमित्य॑प - अ॒नम् । जि॒न्व॒ सꣳ॒॒सर्पः॑ । सꣳ॒॒सर्पो॑ऽसि । सꣳ॒॒सर्प॒ इति॑ सम् - सर्पः॑ । अ॒सि॒ चक्षु॑षे । चक्षु॑षे त्वा । त्वा॒ चक्षुः॑ । चक्षु॑र् जिन्व । जि॒न्व॒ व॒यो॒धाः । व॒यो॒धा अ॑सि । व॒यो॒धा इति॑ वयः - धाः । अ॒सि॒ श्रोत्रा॑य । श्रोत्रा॑य त्वा । त्वा॒ श्रोत्र᳚म् । श्रोत्र॑म् जिन्व । जि॒न्व॒ त्रि॒वृत् । त्रि॒वृद॑सि । त्रि॒वृदिति॑ त्रि - वृत् । अ॒सि॒ प्र॒वृत् । प्र॒वृद॑सि । प्र॒वृदि॑ति प्र - वृत् । अ॒सि॒ स॒म्ॅवृत् । स॒म्ॅवृद॑सि । स॒म्ॅवृदि॑ति सम् - वृत् । अ॒सि॒ वि॒वृत् । वि॒वृद॑सि । वि॒वृदिति॑ वि - वृत् । अ॒सि॒ सꣳ॒॒रो॒हः । सꣳ॒॒रो॒हो॑ऽसि । सꣳ॒॒रो॒ह इति॑ सम् - रो॒हः । अ॒सि॒ नी॒रो॒हः । नी॒रो॒हो॑ऽसि । नी॒रो॒ह इति॑ निः - रो॒हः । अ॒सि॒ प्र॒रो॒हः । प्र॒रो॒हो॑ऽसि । प्र॒रो॒ह इति॑ प्र - रो॒हः । अ॒स्य॒नु॒रो॒हः । अ॒नु॒रो॒हो॑ऽसि । अ॒नु॒रो॒ह इत्य॑नु - रो॒हः । अ॒सि॒ व॒सु॒कः । व॒सु॒को॑ऽसि । अ॒सि॒ वेष॑श्रिः । वेष॑श्रिरसि । वेष॑श्रि॒रिति॒ वेष॑ - श्रिः॒ । अ॒सि॒ वस्य॑ष्टिः । वस्य॑ष्टिरसि । 
अ॒सीत्य॑सि । \newline

\textbf{Jatai Paata} \newline

1. त्वा॒ प्रा॒णम् प्रा॒णम् त्वा᳚ त्वा प्रा॒णम् । \newline
2. प्रा॒णम् जि॑न्व जिन्व प्रा॒णम् प्रा॒णम् जि॑न्व । \newline
3. प्रा॒णमिति॑ प्र - अ॒नम् । \newline
4. जि॒न्व॒ य॒न्ता य॒न्ता जि॑न्व जिन्व य॒न्ता । \newline
5. य॒न्ता ऽस्य॑सि य॒न्ता य॒न्ता ऽसि॑ । \newline
6. अ॒स्य॒ पा॒नाया॑ पा॒नाया᳚ स्यस्य पा॒नाय॑ । \newline
7. अ॒पा॒नाय॑ त्वा त्वा ऽपा॒नाया॑ पा॒नाय॑ त्वा । \newline
8. अ॒पा॒नायेत्य॑प - अ॒नाय॑ । \newline
9. त्वा॒ ऽपा॒न म॑पा॒नम् त्वा᳚ त्वा ऽपा॒नम् । \newline
10. अ॒पा॒नम् जि॑न्व जिन्वा पा॒न म॑पा॒नम् जि॑न्व । \newline
11. अ॒पा॒नमित्य॑प - अ॒नम् । \newline
12. जि॒न्व॒ सꣳ॒॒सर्पः॑ सꣳ॒॒सर्पो॑ जिन्व जिन्व सꣳ॒॒सर्पः॑ । \newline
13. सꣳ॒॒सर्पो᳚ ऽस्यसि सꣳ॒॒सर्पः॑ सꣳ॒॒सर्पो॑ ऽसि । \newline
14. सꣳ॒॒सर्प॒ इति॑ सं - सर्पः॑ । \newline
15. अ॒सि॒ चक्षु॑षे॒ चक्षु॑षे ऽस्यसि॒ चक्षु॑षे । \newline
16. चक्षु॑षे त्वा त्वा॒ चक्षु॑षे॒ चक्षु॑षे त्वा । \newline
17. त्वा॒ चक्षु॒ श्चक्षु॑ स्त्वा त्वा॒ चक्षुः॑ । \newline
18. चक्षु॑र् जिन्व जिन्व॒ चक्षु॒ श्चक्षु॑र् जिन्व । \newline
19. जि॒न्व॒ व॒यो॒धा व॑यो॒धा जि॑न्व जिन्व वयो॒धाः । \newline
20. व॒यो॒धा अ॑स्यसि वयो॒धा व॑यो॒धा अ॑सि । \newline
21. व॒यो॒धा इति॑ वयः - धाः । \newline
22. अ॒सि॒ श्रोत्रा॑य॒ श्रोत्रा॑या स्यसि॒ श्रोत्रा॑य । \newline
23. श्रोत्रा॑य त्वा त्वा॒ श्रोत्रा॑य॒ श्रोत्रा॑य त्वा । \newline
24. त्वा॒ श्रोत्रꣳ॒॒ श्रोत्र॑म् त्वा त्वा॒ श्रोत्र᳚म् । \newline
25. श्रोत्र॑म् जिन्व जिन्व॒ श्रोत्रꣳ॒॒ श्रोत्र॑म् जिन्व । \newline
26. जि॒न्व॒ त्रि॒वृत् त्रि॒वृज् जि॑न्व जिन्व त्रि॒वृत् । \newline
27. त्रि॒वृ द॑स्यसि त्रि॒वृत् त्रि॒वृ द॑सि । \newline
28. त्रि॒वृदिति॑ त्रि - वृत् । \newline
29. अ॒सि॒ प्र॒वृत् प्र॒वृ द॑स्यसि प्र॒वृत् । \newline
30. प्र॒वृ द॑स्यसि प्र॒वृत् प्र॒वृ द॑सि । \newline
31. प्र॒वृदिति॑ प्र - वृत् । \newline
32. अ॒सि॒ सं॒ॅवृथ् सं॒ॅवृ द॑स्यसि सं॒ॅवृत् । \newline
33. सं॒ॅवृ द॑स्यसि सं॒ॅवृथ् सं॒ॅवृ द॑सि । \newline
34. सं॒ॅवृदिति॑ सं - वृत् । \newline
35. अ॒सि॒ वि॒वृद् वि॒वृ द॑स्यसि वि॒वृत् । \newline
36. वि॒वृ द॑स्यसि वि॒वृद् वि॒वृ द॑सि । \newline
37. वि॒वृदिति॑ वि - वृत् । \newline
38. अ॒सि॒ सꣳ॒॒रो॒हः सꣳ॑रो॒हो᳚ ऽस्यसि सꣳरो॒हः । \newline
39. सꣳ॒॒रो॒हो᳚ ऽस्यसि सꣳरो॒हः सꣳ॑रो॒हो॑ ऽसि । \newline
40. सꣳ॒॒रो॒ह इति॑ सं - रो॒हः । \newline
41. अ॒सि॒ नी॒रो॒हो नी॑रो॒हो᳚ ऽस्यसि नीरो॒हः । \newline
42. नी॒रो॒हो᳚ ऽस्यसि नीरो॒हो नी॑रो॒हो॑ ऽसि । \newline
43. नी॒रो॒ह इति॑ निः - रो॒हः । \newline
44. अ॒सि॒ प्र॒रो॒हः प्र॑रो॒हो᳚ ऽस्यसि प्ररो॒हः । \newline
45. प्र॒रो॒हो᳚ ऽस्यसि प्ररो॒हः प्र॑रो॒हो॑ ऽसि । \newline
46. प्र॒रो॒ह इति॑ प्र - रो॒हः । \newline
47. अ॒स्य॒नु॒रो॒हो॑ ऽनुरो॒हो᳚ ऽस्यस्य-नुरो॒हः । \newline
48. अ॒नु॒रो॒हो᳚ ऽस्यस्य-नुरो॒हो॑ ऽनुरो॒हो॑ ऽसि । \newline
49. अ॒नु॒रो॒ह इत्य॑नु - रो॒हः । \newline
50. अ॒सि॒ व॒सु॒को व॑सु॒को᳚ ऽस्यसि वसु॒कः । \newline
51. व॒सु॒को᳚ ऽस्यसि वसु॒को व॑सु॒को॑ ऽसि । \newline
52. अ॒सि॒ वेष॑श्रि॒र् वेष॑श्रि रस्यसि॒ वेष॑श्रिः । \newline
53. वेष॑श्रि रस्यसि॒ वेष॑श्रि॒र् वेष॑श्रि रसि । \newline
54. वेष॑श्रि॒रिति॒ वेष॑ - श्रिः॒ । \newline
55. अ॒सि॒ वस्य॑ष्टि॒र् वस्य॑ष्टि रस्यसि॒ वस्य॑ष्टिः । \newline
56. वस्य॑ष्टि रस्यसि॒ वस्य॑ष्टि॒र् वस्य॑ष्टि रसि । \newline
57. अ॒सीत्य॑सि । \newline

\textbf{Ghana Paata } \newline

1. त्वा॒ प्रा॒णम् प्रा॒णम् त्वा᳚ त्वा प्रा॒णम् जि॑न्व जिन्व प्रा॒णम् त्वा᳚ त्वा प्रा॒णम् जि॑न्व । \newline
2. प्रा॒णम् जि॑न्व जिन्व प्रा॒णम् प्रा॒णम् जि॑न्व य॒न्ता य॒न्ता जि॑न्व प्रा॒णम् प्रा॒णम् जि॑न्व य॒न्ता । \newline
3. प्रा॒णमिति॑ प्र - अ॒नम् । \newline
4. जि॒न्व॒ य॒न्ता य॒न्ता जि॑न्व जिन्व य॒न्ता ऽस्य॑सि य॒न्ता जि॑न्व जिन्व य॒न्ता ऽसि॑ । \newline
5. य॒न्ता ऽस्य॑सि य॒न्ता य॒न्ता ऽस्य॑ पा॒नाया॑ पा॒नाया॑सि य॒न्ता य॒न्ता ऽस्य॑ पा॒नाय॑ । \newline
6. अ॒स्य॒ पा॒नाया॑ पा॒नाया᳚ स्यस्य पा॒नाय॑ त्वा त्वा ऽपा॒नाया᳚ स्यस्य पा॒नाय॑ त्वा । \newline
7. अ॒पा॒नाय॑ त्वा त्वा ऽपा॒नाया॑ पा॒नाय॑ त्वा ऽपा॒न म॑पा॒नम् त्वा॑ ऽपा॒नाया॑ पा॒नाय॑ त्वा ऽपा॒नम् । \newline
8. अ॒पा॒नायेत्य॑प - अ॒नाय॑ । \newline
9. त्वा॒ ऽपा॒न म॑पा॒नम् त्वा᳚ त्वा ऽपा॒नम् जि॑न्व जिन्वापा॒नम् त्वा᳚ त्वा ऽपा॒नम् जि॑न्व । \newline
10. अ॒पा॒नम् जि॑न्व जिन्वापा॒न म॑पा॒नम् जि॑न्व सꣳ॒॒सर्पः॑ सꣳ॒॒सर्पो॑ जिन्वापा॒न म॑पा॒नम् जि॑न्व सꣳ॒॒सर्पः॑ । \newline
11. अ॒पा॒नमित्य॑प - अ॒नम् । \newline
12. जि॒न्व॒ सꣳ॒॒सर्पः॑ सꣳ॒॒सर्पो॑ जिन्व जिन्व सꣳ॒॒सर्पो᳚ ऽस्यसि सꣳ॒॒सर्पो॑ जिन्व जिन्व सꣳ॒॒सर्पो॑ ऽसि । \newline
13. सꣳ॒॒सर्पो᳚ ऽस्यसि सꣳ॒॒सर्पः॑ सꣳ॒॒सर्पो॑ ऽसि॒ चक्षु॑षे॒ चक्षु॑षे ऽसि सꣳ॒॒सर्पः॑ सꣳ॒॒सर्पो॑ ऽसि॒ चक्षु॑षे । \newline
14. सꣳ॒॒सर्प॒ इति॑ सं - सर्पः॑ । \newline
15. अ॒सि॒ चक्षु॑षे॒ चक्षु॑षे ऽस्यसि॒ चक्षु॑षे त्वा त्वा॒ चक्षु॑षे ऽस्यसि॒ चक्षु॑षे त्वा । \newline
16. चक्षु॑षे त्वा त्वा॒ चक्षु॑षे॒ चक्षु॑षे त्वा॒ चक्षु॒ श्चक्षु॑ स्त्वा॒ चक्षु॑षे॒ चक्षु॑षे त्वा॒ चक्षुः॑ । \newline
17. त्वा॒ चक्षु॒ श्चक्षु॑ स्त्वा त्वा॒ चक्षु॑र् जिन्व जिन्व॒ चक्षु॑ स्त्वा त्वा॒ चक्षु॑र् जिन्व । \newline
18. चक्षु॑र् जिन्व जिन्व॒ चक्षु॒ श्चक्षु॑र् जिन्व वयो॒धा व॑यो॒धा जि॑न्व॒ चक्षु॒ श्चक्षु॑र् जिन्व वयो॒धाः । \newline
19. जि॒न्व॒ व॒यो॒धा व॑यो॒धा जि॑न्व जिन्व वयो॒धा अ॑स्यसि वयो॒धा जि॑न्व जिन्व वयो॒धा अ॑सि । \newline
20. व॒यो॒धा अ॑स्यसि वयो॒धा व॑यो॒धा अ॑सि॒ श्रोत्रा॑य॒ श्रोत्रा॑यासि वयो॒धा व॑यो॒धा अ॑सि॒ श्रोत्रा॑य । \newline
21. व॒यो॒धा इति॑ वयः - धाः । \newline
22. अ॒सि॒ श्रोत्रा॑य॒ श्रोत्रा॑या स्यसि॒ श्रोत्रा॑य त्वा त्वा॒ श्रोत्रा॑या स्यसि॒ श्रोत्रा॑य त्वा । \newline
23. श्रोत्रा॑य त्वा त्वा॒ श्रोत्रा॑य॒ श्रोत्रा॑य त्वा॒ श्रोत्रꣳ॒॒ श्रोत्र॑म् त्वा॒ श्रोत्रा॑य॒ श्रोत्रा॑य त्वा॒ श्रोत्र᳚म् । \newline
24. त्वा॒ श्रोत्रꣳ॒॒ श्रोत्र॑म् त्वा त्वा॒ श्रोत्र॑म् जिन्व जिन्व॒ श्रोत्र॑म् त्वा त्वा॒ श्रोत्र॑म् जिन्व । \newline
25. श्रोत्र॑म् जिन्व जिन्व॒ श्रोत्रꣳ॒॒ श्रोत्र॑म् जिन्व त्रि॒वृत् त्रि॒वृज् जि॑न्व॒ श्रोत्रꣳ॒॒ श्रोत्र॑म् जिन्व त्रि॒वृत् । \newline
26. जि॒न्व॒ त्रि॒वृत् त्रि॒वृज् जि॑न्व जिन्व त्रि॒वृ द॑स्यसि त्रि॒वृज् जि॑न्व जिन्व त्रि॒वृद॑सि । \newline
27. त्रि॒वृ द॑स्यसि त्रि॒वृत् त्रि॒वृ द॑सि प्र॒वृत् प्र॒वृ द॑सि त्रि॒वृत् त्रि॒वृ द॑सि प्र॒वृत् । \newline
28. त्रि॒वृदिति॑ त्रि - वृत् । \newline
29. अ॒सि॒ प्र॒वृत् प्र॒वृ द॑स्यसि प्र॒वृ द॑स्यसि प्र॒वृ द॑स्यसि प्र॒वृ द॑सि । \newline
30. प्र॒वृ द॑स्यसि प्र॒वृत् प्र॒वृ द॑सि सं॒ॅवृथ् सं॒ॅवृ द॑सि प्र॒वृत् प्र॒वृ द॑सि सं॒ॅवृत् । \newline
31. प्र॒वृदिति॑ प्र - वृत् । \newline
32. अ॒सि॒ सं॒ॅवृथ् सं॒ॅवृ द॑स्यसि सं॒ॅवृ द॑स्यसि सं॒ॅवृ द॑स्यसि सं॒ॅवृद॑सि । \newline
33. सं॒ॅवृ द॑स्यसि सं॒ॅवृथ् सं॒ॅवृद॑सि वि॒वृद् वि॒वृद॑सि सं॒ॅवृथ् सं॒ॅवृद॑सि वि॒वृत् । \newline
34. सं॒ॅवृदिति॑ सं - वृत् । \newline
35. अ॒सि॒ वि॒वृद् वि॒वृ द॑स्यसि वि॒वृ द॑स्यसि वि॒वृ द॑स्यसि वि॒वृद॑सि । \newline
36. वि॒वृ द॑स्यसि वि॒वृद् वि॒वृ द॑सि सꣳरो॒हः सꣳ॑रो॒हो॑ ऽसि वि॒वृद् वि॒वृ द॑सि सꣳरो॒हः । \newline
37. वि॒वृदिति॑ वि - वृत् । \newline
38. अ॒सि॒ सꣳ॒॒रो॒हः सꣳ॑रो॒हो᳚ ऽस्यसि सꣳरो॒हो᳚ ऽस्यसि सꣳरो॒हो᳚ ऽस्यसि सꣳरो॒हो॑ ऽसि । \newline
39. सꣳ॒॒रो॒हो᳚ ऽस्यसि सꣳरो॒हः सꣳ॑रो॒हो॑ ऽसि नीरो॒हो नी॑रो॒हो॑ ऽसि सꣳरो॒हः सꣳ॑रो॒हो॑ ऽसि नीरो॒हः । \newline
40. सꣳ॒॒रो॒ह इति॑ सं - रो॒हः । \newline
41. अ॒सि॒ नी॒रो॒हो नी॑रो॒हो᳚ ऽस्यसि नीरो॒हो᳚ ऽस्यसि नीरो॒हो᳚ ऽस्यसि नीरो॒हो॑ ऽसि । \newline
42. नी॒रो॒हो᳚ ऽस्यसि नीरो॒हो नी॑रो॒हो॑ ऽसि प्ररो॒हः प्र॑रो॒हो॑ ऽसि नीरो॒हो नी॑रो॒हो॑ ऽसि प्ररो॒हः । \newline
43. नी॒रो॒ह इति॑ निः - रो॒हः । \newline
44. अ॒सि॒ प्र॒रो॒हः प्र॑रो॒हो᳚ ऽस्यसि प्ररो॒हो᳚ ऽस्यसि प्ररो॒हो᳚ ऽस्यसि प्ररो॒हो॑ ऽसि । \newline
45. प्र॒रो॒हो᳚ ऽस्यसि प्ररो॒हः प्र॑रो॒हो᳚ ऽस्यनुरो॒हो॑ ऽनुरो॒हो॑ ऽसि प्ररो॒हः प्र॑रो॒हो᳚ ऽस्यनुरो॒हः । \newline
46. प्र॒रो॒ह इति॑ प्र - रो॒हः । \newline
47. अ॒स्य॒नु॒रो॒हो॑ ऽनुरो॒हो᳚ ऽस्य स्यनुरो॒हो᳚ ऽस्य स्यनुरो॒हो᳚ ऽस्य स्यनुरो॒हो॑ ऽसि । \newline
48. अ॒नु॒रो॒हो᳚ ऽस्य स्यनुरो॒हो॑ ऽनुरो॒हो॑ ऽसि वसु॒को व॑सु॒को᳚ ऽस्यनुरो॒हो॑ ऽनुरो॒हो॑ ऽसि वसु॒कः । \newline
49. अ॒नु॒रो॒ह इत्य॑नु - रो॒हः । \newline
50. अ॒सि॒ व॒सु॒को व॑सु॒को᳚ ऽस्यसि वसु॒को᳚ ऽस्यसि वसु॒को᳚ ऽस्यसि वसु॒को॑ ऽसि । \newline
51. व॒सु॒को᳚ ऽस्यसि वसु॒को व॑सु॒को॑ ऽसि॒ वेष॑श्रि॒र् वेष॑श्रि रसि वसु॒को व॑सु॒को॑ ऽसि॒ वेष॑श्रिः । \newline
52. अ॒सि॒ वेष॑श्रि॒र् वेष॑श्रि रस्यसि॒ वेष॑श्रि रस्यसि॒ वेष॑श्रि रस्यसि॒ वेष॑श्रिरसि । \newline
53. वेष॑श्रि रस्यसि॒ वेष॑श्रि॒र् वेष॑श्रि रसि॒ वस्य॑ष्टि॒र् वस्य॑ष्टि रसि॒ वेष॑श्रि॒र् वेष॑श्रि रसि॒ वस्य॑ष्टिः । \newline
54. वेष॑श्रि॒रिति॒ वेष॑ - श्रिः॒ । \newline
55. अ॒सि॒ वस्य॑ष्टि॒र् वस्य॑ष्टि रस्यसि॒ वस्य॑ष्टि रस्यसि॒ वस्य॑ष्टि रस्यसि॒ वस्य॑ष्टिरसि । \newline
56. वस्य॑ष्टि रस्यसि॒ वस्य॑ष्टि॒र् वस्य॑ष्टिरसि । \newline
57. अ॒सीत्य॑सि । \newline
\pagebreak
\markright{ TS 4.4.2.1  \hfill https://www.vedavms.in \hfill}

\section{ TS 4.4.2.1 }

\textbf{TS 4.4.2.1 } \newline
\textbf{Samhita Paata} \newline

राज्ञ्य॑सि॒ प्राची॒ दिग्वस॑वस्ते दे॒वा अधि॑पतयो॒ऽग्निर्.हे॑ती॒नां प्र॑तिध॒र्ता॑ त्रि॒वृत् त्वा॒ स्तोमः॑ पृथि॒व्याꣳ श्र॑य॒त्वाज्य॑-मु॒क्थमव्य॑थयथ् स्तभ्नातु रथन्त॒रꣳ साम॒ प्रति॑ष्ठित्यै वि॒राड॑सि दक्षि॒णा दिग्रु॒द्रास्ते॑ दे॒वा अधि॑पतय॒ इन्द्रो॑ हेती॒नां प्र॑तिध॒र्ता प॑ञ्चद॒शस्त्वा॒ स्तोमः॑ पृथि॒व्याꣳ श्र॑यतु॒ प्र-उ॑गमु॒क्थमव्य॑थयथ् स्तभ्नातु बृ॒हथ् साम॒ प्रति॑ष्ठित्यै स॒म्राड॑सि प्र॒तीची॒ दि- [  ] \newline

\textbf{Pada Paata} \newline

राज्ञी᳚ । अ॒सि॒ । प्राची᳚ । दिक् । वस॑वः । ते॒ । दे॒वाः । अधि॑पतय॒ इत्यधि॑ - प॒त॒यः॒ । अ॒ग्निः । हे॒ती॒नाम् । प्र॒ति॒ध॒र्तेति॑ प्रति - ध॒र्ता । त्रि॒वृदिति॑ त्रि-वृत् । त्वा॒ । स्तोमः॑ । पृ॒थि॒व्याम् । श्र॒य॒तु॒ । आज्य᳚म् । उ॒क्थम् । अव्य॑थयत् । स्त॒भ्ना॒तु॒ । र॒थ॒न्त॒रमिति॑ रथं - त॒रम् । साम॑ । प्रति॑ष्ठित्या॒ इति॒ प्रति॑ - स्थि॒त्यै॒ । वि॒राडिति॑ वि - राट् । अ॒सि॒ । द॒क्षि॒णा । दिक् । रु॒द्राः । ते॒ । दे॒वाः । अधि॑पतय॒ इत्यधि॑ - प॒त॒यः॒ । इन्द्रः॑ । हे॒ती॒नाम् । प्र॒ति॒ध॒र्तेति॑ प्रति - ध॒र्ता । प॒ञ्च॒द॒श इति॑ पञ्च - द॒शः । त्वा॒ । स्तोमः॑ । पृ॒थि॒व्याम् । श्र॒य॒तु॒ । प्र उ॑गम् । उ॒क्थम् । अव्य॑थयत् । स्त॒भ्ना॒तु॒ । बृ॒हत् । साम॑ । प्रति॑ष्ठित्या॒ इति॒ प्रति॑ - स्थि॒त्यै॒ । स॒म्राडिति॑ सं - राट् । अ॒सि॒ । प्र॒तीची᳚ । दिक् ।  \newline


\textbf{Krama Paata} \newline

राज्ञ्य॑सि । अ॒सि॒ प्राची᳚ । प्राची॒ दिक् । दिग् वस॑वः । वस॑वस्ते । ते॒ दे॒वाः । दे॒वा अधि॑पतयः । अधि॑पतयो॒ऽग्निः । अधि॑पतय॒ इत्यधि॑ - प॒त॒यः॒ । अ॒ग्निर्. हे॑ती॒नाम् । हे॒ती॒नाम् प्र॑तिध॒र्ता । प्र॒ति॒ध॒र्ता त्रि॒वृत् । प्र॒ति॒ध॒र्तेति॑ प्रति - ध॒र्ता । त्रि॒वृत् त्वा᳚ । त्रि॒वृदिति॑ त्रि - वृत् । त्वा॒ स्तोमः॑ । स्तोमः॑ पृथि॒व्याम् । पृ॒थि॒व्याꣳ श्र॑यतु । श्र॒य॒त्वाज्य᳚म् । आज्य॑मु॒क्थम् । उ॒क्थमव्य॑थयत् । अव्य॑थयथ् स्तभ्नातु । स्त॒भ्ना॒तु॒ र॒थ॒न्त॒रम् । र॒थ॒न्त॒रꣳ साम॑ । र॒थ॒न्त॒रमिति॑ रथम् - त॒रम् । साम॒ प्रति॑ष्ठित्यै । प्रति॑ष्ठित्यै वि॒राट् । प्रति॑ष्ठित्या॒ इति॒ प्रति॑ - स्थि॒त्यै॒ । वि॒राड॑सि । वि॒राडिति॑ वि - राट् । अ॒सि॒ द॒क्षि॒णा । द॒क्षि॒णा दिक् । दिग् रु॒द्राः । रु॒द्रास्ते᳚ । ते॒ दे॒वाः । दे॒वा अधि॑पतयः । अधि॑पतय॒ इन्द्रः॑ । अधि॑पतय॒ इत्यधि॑ - प॒त॒यः॒ । इन्द्रो॑ हेती॒नाम् । हे॒ती॒नाम् प्र॑तिध॒र्ता । प्र॒ति॒ध॒र्ता प॑ञ्चद॒शः । प्र॒ति॒ध॒र्तेति॑ प्रति - ध॒र्ता । प॒ञ्च॒द॒शस्त्वा᳚ । प॒ञ्च॒द॒श इति॑ पञ्च - द॒शः । त्वा॒ स्तोमः॑ । स्तोमः॑ पृथि॒व्याम् । पृ॒थि॒व्याꣳ श्र॑यतु । श्र॒य॒तु॒ प्र,उ॑गम् । प्र,उ॑गमु॒क्थम् । उ॒क्थमव्य॑थयत् । अव्य॑थयथ् स्तभ्नातु । स्त॒भ्ना॒तु॒ बृ॒हत् । बृ॒हथ् साम॑ । साम॒ प्रति॑ष्ठित्यै । प्रति॑ष्ठित्यै स॒म्राट् । प्रति॑ष्ठित्या॒ इति॒ प्रति॑ - स्थि॒त्यै॒ । स॒म्राड॑सि । स॒म्राडिति॑ सम् - राट् । अ॒सि॒ प्र॒तीची᳚ । प्र॒तीची॒ दिक् । दिगा॑दि॒त्याः \newline

\textbf{Jatai Paata} \newline

1. राज्ञ्य॑ स्यसि॒ राज्ञी॒ राज्ञ्य॑सि । \newline
2. अ॒सि॒ प्राची॒ प्राच्य॑ स्यसि॒ प्राची᳚ । \newline
3. प्राची॒ दिग् दिक् प्राची॒ प्राची॒ दिक् । \newline
4. दिग् वस॑वो॒ वस॑वो॒ दिग् दिग् वस॑वः । \newline
5. वस॑व स्ते ते॒ वस॑वो॒ वस॑व स्ते । \newline
6. ते॒ दे॒वा दे॒वा स्ते॑ ते दे॒वाः । \newline
7. दे॒वा अधि॑पत॒यो ऽधि॑पतयो दे॒वा दे॒वा अधि॑पतयः । \newline
8. अधि॑पतयो॒ ऽग्नि र॒ग्नि रधि॑पत॒यो ऽधि॑पतयो॒ ऽग्निः । \newline
9. अधि॑पतय॒ इत्यधि॑ - प॒त॒यः॒ । \newline
10. अ॒ग्निर्. हे॑ती॒नाꣳ हे॑ती॒ना म॒ग्नि र॒ग्निर्. हे॑ती॒नाम् । \newline
11. हे॒ती॒नाम् प्र॑तिध॒र्ता प्र॑तिध॒र्ता हे॑ती॒नाꣳ हे॑ती॒नाम् प्र॑तिध॒र्ता । \newline
12. प्र॒ति॒ध॒र्ता त्रि॒वृत् त्रि॒वृत् प्र॑तिध॒र्ता प्र॑तिध॒र्ता त्रि॒वृत् । \newline
13. प्र॒ति॒ध॒र्तेति॑ प्रति - ध॒र्ता । \newline
14. त्रि॒वृत् त्वा᳚ त्वा त्रि॒वृत् त्रि॒वृत् त्वा᳚ । \newline
15. त्रि॒वृदिति॑ त्रि - वृत् । \newline
16. त्वा॒ स्तोमः॒ स्तोम॑ स्त्वा त्वा॒ स्तोमः॑ । \newline
17. स्तोमः॑ पृथि॒व्याम् पृ॑थि॒व्याꣳ स्तोमः॒ स्तोमः॑ पृथि॒व्याम् । \newline
18. पृ॒थि॒व्याꣳ श्र॑यतु श्रयतु पृथि॒व्याम् पृ॑थि॒व्याꣳ श्र॑यतु । \newline
19. श्र॒य॒त्वाज्य॒ माज्यꣳ॑ श्रयतु श्रय॒त्वाज्य᳚म् । \newline
20. आज्य॑ मु॒क्थ मु॒क्थ माज्य॒ माज्य॑ मु॒क्थम् । \newline
21. उ॒क्थ मव्य॑थय॒द व्य॑थयद् उ॒क्थ मु॒क्थ मव्य॑थयत् । \newline
22. अव्य॑थयथ् स्तभ्नातु स्तभ्ना॒ त्वव्य॑थय॒ दव्य॑थयथ् स्तभ्नातु । \newline
23. स्त॒भ्ना॒तु॒ र॒थ॒न्त॒रꣳ र॑थन्त॒रꣳ स्त॑भ्नातु स्तभ्नातु रथन्त॒रम् । \newline
24. र॒थ॒न्त॒रꣳ साम॒ साम॑ रथन्त॒रꣳ र॑थन्त॒रꣳ साम॑ । \newline
25. र॒थ॒न्त॒रमिति॑ रथं - त॒रम् । \newline
26. साम॒ प्रति॑ष्ठित्यै॒ प्रति॑ष्ठित्यै॒ साम॒ साम॒ प्रति॑ष्ठित्यै । \newline
27. प्रति॑ष्ठित्यै वि॒राड् वि॒राट् प्रति॑ष्ठित्यै॒ प्रति॑ष्ठित्यै वि॒राट् । \newline
28. प्रति॑ष्ठित्या॒ इति॒ प्रति॑ - स्थि॒त्यै॒ । \newline
29. वि॒रा ड॑स्यसि वि॒राड् वि॒रा ड॑सि । \newline
30. वि॒राडिति॑ वि - राट् । \newline
31. अ॒सि॒ द॒क्षि॒णा द॑क्षि॒णा ऽस्य॑सि दक्षि॒णा । \newline
32. द॒क्षि॒णा दिग् दिग् द॑क्षि॒णा द॑क्षि॒णा दिक् । \newline
33. दिग् रु॒द्रा रु॒द्रा दिग् दिग् रु॒द्राः । \newline
34. रु॒द्रा स्ते॑ ते रु॒द्रा रु॒द्रा स्ते᳚ । \newline
35. ते॒ दे॒वा दे॒वा स्ते॑ ते दे॒वाः । \newline
36. दे॒वा अधि॑पत॒यो ऽधि॑पतयो दे॒वा दे॒वा अधि॑पतयः । \newline
37. अधि॑पतय॒ इन्द्र॒ इन्द्रो ऽधि॑पत॒यो ऽधि॑पतय॒ इन्द्रः॑ । \newline
38. अधि॑पतय॒ इत्यधि॑ - प॒त॒यः॒ । \newline
39. इन्द्रो॑ हेती॒नाꣳ हे॑ती॒ना मिन्द्र॒ इन्द्रो॑ हेती॒नाम् । \newline
40. हे॒ती॒नाम् प्र॑तिध॒र्ता प्र॑तिध॒र्ता हे॑ती॒नाꣳ हे॑ती॒नाम् प्र॑तिध॒र्ता । \newline
41. प्र॒ति॒ध॒र्ता प॑ञ्चद॒शः प॑ञ्चद॒शः प्र॑तिध॒र्ता प्र॑तिध॒र्ता प॑ञ्चद॒शः । \newline
42. प्र॒ति॒ध॒र्तेति॑ प्रति - ध॒र्ता । \newline
43. प॒ञ्च॒द॒श स्त्वा᳚ त्वा पञ्चद॒शः प॑ञ्चद॒श स्त्वा᳚ । \newline
44. प॒ञ्च॒द॒श इति॑ पञ्च - द॒शः । \newline
45. त्वा॒ स्तोमः॒ स्तोम॑ स्त्वा त्वा॒ स्तोमः॑ । \newline
46. स्तोमः॑ पृथि॒व्याम् पृ॑थि॒व्याꣳ स्तोमः॒ स्तोमः॑ पृथि॒व्याम् । \newline
47. पृ॒थि॒व्याꣳ श्र॑यतु श्रयतु पृथि॒व्याम् पृ॑थि॒व्याꣳ श्र॑यतु । \newline
48. श्र॒य॒तु॒ प्र‌उ॑ग॒म् प्र‌उ॑गꣳ श्रयतु श्रयतु॒ प्र‌उ॑गम् । \newline
49. प्र‌उ॑ग मु॒क्थ मु॒क्थम् प्र‌उ॑ग॒म् प्र‌उ॑ग मु॒क्थम् । \newline
50. उ॒क्थ मव्य॑थय॒ दव्य॑थय दु॒क्थ मु॒क्थ मव्य॑थयत् । \newline
51. अव्य॑थयथ् स्तभ्नातु स्तभ्ना॒ त्वव्य॑थय॒ दव्य॑थयथ् स्तभ्नातु । \newline
52. स्त॒भ्ना॒तु॒ बृ॒हद् बृ॒हथ् स्त॑भ्नातु स्तभ्नातु बृ॒हत् । \newline
53. बृ॒हथ् साम॒ साम॑ बृ॒हद् बृ॒हथ् साम॑ । \newline
54. साम॒ प्रति॑ष्ठित्यै॒ प्रति॑ष्ठित्यै॒ साम॒ साम॒ प्रति॑ष्ठित्यै । \newline
55. प्रति॑ष्ठित्यै स॒म्राट् थ्स॒म्राट् प्रति॑ष्ठित्यै॒ प्रति॑ष्ठित्यै स॒म्राट् । \newline
56. प्रति॑ष्ठित्या॒ इति॒ प्रति॑ - स्थि॒त्यै॒ । \newline
57. स॒म्रा ड॑स्यसि स॒म्राट् थ्स॒म्रा ड॑सि । \newline
58. स॒म्राडिति॑ सं - राट् । \newline
59. अ॒सि॒ प्र॒तीची᳚ प्र॒तीच्य॑ स्यसि प्र॒तीची᳚ । \newline
60. प्र॒तीची॒ दिग् दिक् प्र॒तीची᳚ प्र॒तीची॒ दिक् । \newline
61. दिगा॑दि॒त्या आ॑दि॒त्या दिग् दिगा॑दि॒त्याः । \newline

\textbf{Ghana Paata } \newline

1. राज्ञ्य॑स्यसि॒ राज्ञी॒ राज्ञ्य॑सि॒ प्राची॒ प्राच्य॑सि॒ राज्ञी॒ राज्ञ्य॑सि॒ प्राची᳚ । \newline
2. अ॒सि॒ प्राची॒ प्राच्य॑स्यसि॒ प्राची॒ दिग् दिक् प्राच्य॑स्यसि॒ प्राची॒ दिक् । \newline
3. प्राची॒ दिग् दिक् प्राची॒ प्राची॒ दिग् वस॑वो॒ वस॑वो॒ दिक् प्राची॒ प्राची॒ दिग् वस॑वः । \newline
4. दिग् वस॑वो॒ वस॑वो॒ दिग् दिग् वस॑व स्ते ते॒ वस॑वो॒ दिग् दिग् वस॑व स्ते । \newline
5. वस॑व स्ते ते॒ वस॑वो॒ वस॑व स्ते दे॒वा दे॒वा स्ते॒ वस॑वो॒ वस॑व स्ते दे॒वाः । \newline
6. ते॒ दे॒वा दे॒वा स्ते॑ ते दे॒वा अधि॑पत॒यो ऽधि॑पतयो दे॒वा स्ते॑ ते दे॒वा अधि॑पतयः । \newline
7. दे॒वा अधि॑पत॒यो ऽधि॑पतयो दे॒वा दे॒वा अधि॑पतयो॒ ऽग्नि र॒ग्नि रधि॑पतयो दे॒वा दे॒वा अधि॑पतयो॒ ऽग्निः । \newline
8. अधि॑पतयो॒ ऽग्नि र॒ग्नि रधि॑पत॒यो ऽधि॑पतयो॒ ऽग्निर्. हे॑ती॒नाꣳ हे॑ती॒ना म॒ग्नि रधि॑पत॒यो ऽधि॑पतयो॒ ऽग्निर्. हे॑ती॒नाम् । \newline
9. अधि॑पतय॒ इत्यधि॑ - प॒त॒यः॒ । \newline
10. अ॒ग्निर्. हे॑ती॒नाꣳ हे॑ती॒ना म॒ग्नि र॒ग्निर्. हे॑ती॒नाम् प्र॑तिध॒र्ता प्र॑तिध॒र्ता हे॑ती॒ना म॒ग्नि र॒ग्निर्. हे॑ती॒नाम् प्र॑तिध॒र्ता । \newline
11. हे॒ती॒नाम् प्र॑तिध॒र्ता प्र॑तिध॒र्ता हे॑ती॒नाꣳ हे॑ती॒नाम् प्र॑तिध॒र्ता त्रि॒वृत् त्रि॒वृत् प्र॑तिध॒र्ता हे॑ती॒नाꣳ हे॑ती॒नाम् प्र॑तिध॒र्ता त्रि॒वृत् । \newline
12. प्र॒ति॒ध॒र्ता त्रि॒वृत् त्रि॒वृत् प्र॑तिध॒र्ता प्र॑तिध॒र्ता त्रि॒वृत् त्वा᳚ त्वा त्रि॒वृत् प्र॑तिध॒र्ता प्र॑तिध॒र्ता त्रि॒वृत् त्वा᳚ । \newline
13. प्र॒ति॒ध॒र्तेति॑ प्रति - ध॒र्ता । \newline
14. त्रि॒वृत् त्वा᳚ त्वा त्रि॒वृत् त्रि॒वृत् त्वा॒ स्तोमः॒ स्तोम॑ स्त्वा त्रि॒वृत् त्रि॒वृत् त्वा॒ स्तोमः॑ । \newline
15. त्रि॒वृदिति॑ त्रि - वृत् । \newline
16. त्वा॒ स्तोमः॒ स्तोम॑ स्त्वा त्वा॒ स्तोमः॑ पृथि॒व्याम् पृ॑थि॒व्याꣳ स्तोम॑ स्त्वा त्वा॒ स्तोमः॑ पृथि॒व्याम् । \newline
17. स्तोमः॑ पृथि॒व्याम् पृ॑थि॒व्याꣳ स्तोमः॒ स्तोमः॑ पृथि॒व्याꣳ श्र॑यतु श्रयतु पृथि॒व्याꣳ स्तोमः॒ स्तोमः॑ पृथि॒व्याꣳ श्र॑यतु । \newline
18. पृ॒थि॒व्याꣳ श्र॑यतु श्रयतु पृथि॒व्याम् पृ॑थि॒व्याꣳ श्र॑य॒त्वाज्य॒ माज्यꣳ॑ श्रयतु पृथि॒व्याम् पृ॑थि॒व्याꣳ श्र॑य॒त्वाज्य᳚म् । \newline
19. श्र॒य॒त्वाज्य॒ माज्यꣳ॑ श्रयतु श्रय॒त्वाज्य॑ मु॒क्थ मु॒क्थ माज्यꣳ॑ श्रयतु श्रय॒त्वाज्य॑ मु॒क्थम् । \newline
20. आज्य॑ मु॒क्थ मु॒क्थ माज्य॒ माज्य॑ मु॒क्थ मव्य॑थय॒ दव्य॑थय दु॒क्थ माज्य॒ माज्य॑ मु॒क्थ मव्य॑थयत् । \newline
21. उ॒क्थ मव्य॑थय॒ दव्य॑थय दु॒क्थ मु॒क्थ मव्य॑थयथ् स्तभ्नातु स्तभ्ना॒त्व व्य॑थय दु॒क्थ मु॒क्थ मव्य॑थयथ् स्तभ्नातु । \newline
22. अव्य॑थयथ् स्तभ्नातु स्तभ्ना॒त्वव्य॑ थय॒ दव्य॑थयथ् स्तभ्नातु रथन्त॒रꣳ र॑थन्त॒रꣳ स्त॑भ्ना॒त्वव्य॑ थय॒ दव्य॑थयथ् स्तभ्नातु रथन्त॒रम् । \newline
23. स्त॒भ्ना॒तु॒ र॒थ॒न्त॒रꣳ र॑थन्त॒रꣳ स्त॑भ्नातु स्तभ्नातु रथन्त॒रꣳ साम॒ साम॑ रथन्त॒रꣳ स्त॑भ्नातु स्तभ्नातु रथन्त॒रꣳ साम॑ । \newline
24. र॒थ॒न्त॒रꣳ साम॒ साम॑ रथन्त॒रꣳ र॑थन्त॒रꣳ साम॒ प्रति॑ष्ठित्यै॒ प्रति॑ष्ठित्यै॒ साम॑ रथन्त॒रꣳ र॑थन्त॒रꣳ साम॒ प्रति॑ष्ठित्यै । \newline
25. र॒थ॒न्त॒रमिति॑ रथं - त॒रम् । \newline
26. साम॒ प्रति॑ष्ठित्यै॒ प्रति॑ष्ठित्यै॒ साम॒ साम॒ प्रति॑ष्ठित्यै वि॒राड् वि॒राट् प्रति॑ष्ठित्यै॒ साम॒ साम॒ प्रति॑ष्ठित्यै वि॒राट् । \newline
27. प्रति॑ष्ठित्यै वि॒राड् वि॒राट् प्रति॑ष्ठित्यै॒ प्रति॑ष्ठित्यै वि॒रा ड॑स्यसि वि॒राट् प्रति॑ष्ठित्यै॒ प्रति॑ष्ठित्यै वि॒रा ड॑सि । \newline
28. प्रति॑ष्ठित्या॒ इति॒ प्रति॑ - स्थि॒त्यै॒ । \newline
29. वि॒रा ड॑स्यसि वि॒राड् वि॒रा ड॑सि दक्षि॒णा द॑क्षि॒णा ऽसि॑ वि॒राड् वि॒रा ड॑सि दक्षि॒णा । \newline
30. वि॒राडिति॑ वि - राट् । \newline
31. अ॒सि॒ द॒क्षि॒णा द॑क्षि॒णा ऽस्य॑सि दक्षि॒णा दिग् दिग् द॑क्षि॒णा ऽस्य॑सि दक्षि॒णा दिक् । \newline
32. द॒क्षि॒णा दिग् दिग् द॑क्षि॒णा द॑क्षि॒णा दिग् रु॒द्रा रु॒द्रा दिग् द॑क्षि॒णा द॑क्षि॒णा दिग् रु॒द्राः । \newline
33. दिग् रु॒द्रा रु॒द्रा दिग् दिग् रु॒द्रा स्ते॑ ते रु॒द्रा दिग् दिग् रु॒द्रा स्ते᳚ । \newline
34. रु॒द्रा स्ते॑ ते रु॒द्रा रु॒द्रा स्ते॑ दे॒वा दे॒वा स्ते॑ रु॒द्रा रु॒द्रा स्ते॑ दे॒वाः । \newline
35. ते॒ दे॒वा दे॒वा स्ते॑ ते दे॒वा अधि॑पत॒यो ऽधि॑पतयो दे॒वा स्ते॑ ते दे॒वा अधि॑पतयः । \newline
36. दे॒वा अधि॑पत॒यो ऽधि॑पतयो दे॒वा दे॒वा अधि॑पतय॒ इन्द्र॒ इन्द्रो ऽधि॑पतयो दे॒वा दे॒वा अधि॑पतय॒ इन्द्रः॑ । \newline
37. अधि॑पतय॒ इन्द्र॒ इन्द्रो ऽधि॑पत॒यो ऽधि॑पतय॒ इन्द्रो॑ हेती॒नाꣳ हे॑ती॒ना मिन्द्रो ऽधि॑पत॒यो ऽधि॑पतय॒ इन्द्रो॑ हेती॒नाम् । \newline
38. अधि॑पतय॒ इत्यधि॑ - प॒त॒यः॒ । \newline
39. इन्द्रो॑ हेती॒नाꣳ हे॑ती॒ना मिन्द्र॒ इन्द्रो॑ हेती॒नाम् प्र॑तिध॒र्ता प्र॑तिध॒र्ता हे॑ती॒ना मिन्द्र॒ इन्द्रो॑ हेती॒नाम् प्र॑तिध॒र्ता । \newline
40. हे॒ती॒नाम् प्र॑तिध॒र्ता प्र॑तिध॒र्ता हे॑ती॒नाꣳ हे॑ती॒नाम् प्र॑तिध॒र्ता प॑ञ्चद॒शः प॑ञ्चद॒शः प्र॑तिध॒र्ता हे॑ती॒नाꣳ हे॑ती॒नाम् प्र॑तिध॒र्ता प॑ञ्चद॒शः । \newline
41. प्र॒ति॒ध॒र्ता प॑ञ्चद॒शः प॑ञ्चद॒शः प्र॑तिध॒र्ता प्र॑तिध॒र्ता प॑ञ्चद॒श स्त्वा᳚ त्वा पञ्चद॒शः प्र॑तिध॒र्ता प्र॑तिध॒र्ता प॑ञ्चद॒श स्त्वा᳚ । \newline
42. प्र॒ति॒ध॒र्तेति॑ प्रति - ध॒र्ता । \newline
43. प॒ञ्च॒द॒श स्त्वा᳚ त्वा पञ्चद॒शः प॑ञ्चद॒श स्त्वा॒ स्तोमः॒ स्तोम॑ स्त्वा पञ्चद॒शः प॑ञ्चद॒श स्त्वा॒ स्तोमः॑ । \newline
44. प॒ञ्च॒द॒श इति॑ पञ्च - द॒शः । \newline
45. त्वा॒ स्तोमः॒ स्तोम॑ स्त्वा त्वा॒ स्तोमः॑ पृथि॒व्याम् पृ॑थि॒व्याꣳ स्तोम॑ स्त्वा त्वा॒ स्तोमः॑ पृथि॒व्याम् । \newline
46. स्तोमः॑ पृथि॒व्याम् पृ॑थि॒व्याꣳ स्तोमः॒ स्तोमः॑ पृथि॒व्याꣳ श्र॑यतु श्रयतु पृथि॒व्याꣳ स्तोमः॒ स्तोमः॑ पृथि॒व्याꣳ श्र॑यतु । \newline
47. पृ॒थि॒व्याꣳ श्र॑यतु श्रयतु पृथि॒व्याम् पृ॑थि॒व्याꣳ श्र॑यतु॒ प्र‌उ॑ग॒म् प्र‌उ॑गꣳ श्रयतु पृथि॒व्याम् पृ॑थि॒व्याꣳ श्र॑यतु॒ प्र‌उ॑गम् । \newline
48. श्र॒य॒तु॒ प्र‌उ॑ग॒म् प्र‌उ॑गꣳ श्रयतु श्रयतु॒ प्र‌उ॑ग मु॒क्थ मु॒क्थम् प्र‌उ॑गꣳ श्रयतु श्रयतु॒ 
प्र‌उ॑ग मु॒क्थम् । \newline
49. प्र‌उ॑ग मु॒क्थ मु॒क्थम् प्र‌उ॑ग॒म् प्र‌उ॑ग मु॒क्थ मव्य॑थय॒ दव्य॑थय दु॒क्थम् प्र‌उ॑ग॒म् 
प्र‌उ॑ग मु॒क्थ मव्य॑थयत् । \newline
50. उ॒क्थ मव्य॑थय॒ दव्य॑थय दु॒क्थ मु॒क्थ मव्य॑थयथ् स्तभ्नातु स्तभ्ना॒त्व व्य॑थय दु॒क्थ मु॒क्थ मव्य॑थयथ् स्तभ्नातु । \newline
51. अव्य॑थयथ् स्तभ्नातु स्तभ्ना॒त्व व्य॑थय॒ दव्य॑थयथ् स्तभ्नातु बृ॒हद् बृ॒हथ् स्त॑भ्ना॒त्व व्य॑थय॒ दव्य॑थयथ् स्तभ्नातु बृ॒हत् । \newline
52. स्त॒भ्ना॒तु॒ बृ॒हद् बृ॒हथ् स्त॑भ्नातु स्तभ्नातु बृ॒हथ् साम॒ साम॑ बृ॒हथ् स्त॑भ्नातु स्तभ्नातु बृ॒हथ् साम॑ । \newline
53. बृ॒हथ् साम॒ साम॑ बृ॒हद् बृ॒हथ् साम॒ प्रति॑ष्ठित्यै॒ प्रति॑ष्ठित्यै॒ साम॑ बृ॒हद् बृ॒हथ् साम॒ प्रति॑ष्ठित्यै । \newline
54. साम॒ प्रति॑ष्ठित्यै॒ प्रति॑ष्ठित्यै॒ साम॒ साम॒ प्रति॑ष्ठित्यै स॒म्राट् थ्स॒म्राट् प्रति॑ष्ठित्यै॒ साम॒ साम॒ प्रति॑ष्ठित्यै स॒म्राट् । \newline
55. प्रति॑ष्ठित्यै स॒म्राट् थ्स॒म्राट् प्रति॑ष्ठित्यै॒ प्रति॑ष्ठित्यै स॒म्रा ड॑स्यसि स॒म्राट् प्रति॑ष्ठित्यै॒ प्रति॑ष्ठित्यै स॒म्रा ड॑सि । \newline
56. प्रति॑ष्ठित्या॒ इति॒ प्रति॑ - स्थि॒त्यै॒ । \newline
57. स॒म्रा ड॑स्यसि स॒म्राट् थ्स॒म्राड॑सि प्र॒तीची᳚ प्र॒तीच्य॑सि स॒म्राट् थ्स॒म्राड॑सि प्र॒तीची᳚ । \newline
58. स॒म्राडिति॑ सं - राट् । \newline
59. अ॒सि॒ प्र॒तीची᳚ प्र॒ती च्य॑स्यसि प्र॒तीची॒ दिग् दिक् प्र॒ती च्य॑स्यसि प्र॒तीची॒ दिक् । \newline
60. प्र॒तीची॒ दिग् दिक् प्र॒तीची᳚ प्र॒तीची॒ दिगा॑दि॒त्या आ॑दि॒त्या दिक् प्र॒तीची᳚ प्र॒तीची॒ दिगा॑दि॒त्याः । \newline
61. दिगा॑दि॒त्या आ॑दि॒त्या दिग् दिगा॑ दि॒त्या स्ते॑ त आदि॒त्या दिग् दिगा॑ दि॒त्या स्ते᳚ । \newline
\pagebreak
\markright{ TS 4.4.2.2  \hfill https://www.vedavms.in \hfill}

\section{ TS 4.4.2.2 }

\textbf{TS 4.4.2.2 } \newline
\textbf{Samhita Paata} \newline

गा॑दि॒त्यास्ते॑ दे॒वा अधि॑पतयः॒ सोमो॑ हेती॒नां प्र॑तिध॒र्ता स॑प्तद॒शस्त्वा॒ स्तोमः॑ पृथि॒व्याꣳ श्र॑यतु मरुत्व॒तीय॑मु॒क्थ-मव्य॑थयथ् स्तभ्नातु वैरू॒पꣳ साम॒ प्रति॑ष्ठित्यायै स्व॒राड॒स्युदी॑ची॒ दिग् विश्वे॑ ते दे॒वा अधि॑पतयो॒ वरु॑णो हेती॒नाम् प्र॑तिध॒र्तैक॑विꣳ॒॒श स्त्वा॒ स्तोमः॑ पृथि॒व्याꣳ श्र॑यतु॒ निष्के॑वल्य-मु॒क्थमव्य॑थयथ् स्तभ्नातु वैरा॒जꣳ साम॒ प्रति॑ष्ठित्या॒ अधि॑पत्न्यसि बृह॒ती दिङ्म॒रुत॑स्ते दे॒वा अधि॑पतयो॒ - [  ] \newline

\textbf{Pada Paata} \newline

आ॒दि॒त्याः । ते॒ । दे॒वाः । अधि॑पतय॒ इत्यधि॑ - प॒त॒यः॒ । सोमः॑ । हे॒ती॒नाम् । प्र॒ति॒ध॒र्तेति॑ प्रति - ध॒र्ता । स॒प्त॒द॒श इति॑ सप्त - द॒शः । त्वा॒ । स्तोमः॑ । पृ॒थि॒व्याम् । श्र॒य॒तु॒ । म॒रु॒त्व॒तीय᳚म् । उ॒क्थम् । अव्य॑थयत् । स्त॒भ्ना॒तु॒ । वै॒रू॒पम् । साम॑ । प्रति॑ष्ठित्या॒ इति॒ प्रति॑ - स्थि॒त्यै॒ । स्व॒राडिति॑ स्व - राट् । अ॒सि॒ । उदी॑ची । दिक् । विश्वे᳚ । ते॒ । दे॒वाः । अधि॑पतय॒ इत्यधि॑ - प॒त॒यः॒ । वरु॑णः । हे॒ती॒नाम् । प्र॒ति॒ध॒र्तेति॑ प्रति - ध॒र्ता । ए॒क॒विꣳ॒॒श इत्ये॑क-विꣳ॒॒शः । त्वा॒ । स्तोमः॑ । पृ॒थि॒व्याम् । श्र॒य॒तु॒ । निष्के॑वल्यम् । उ॒क्थम् । अव्य॑थयत् । स्त॒भ्ना॒तु॒ । वै॒रा॒जम् । साम॑ । प्रति॑ष्ठित्या॒ इति॒ प्रति॑ - स्थि॒त्यै॒ । अधि॑प॒त्नीत्यधि॑ - प॒त्नी॒ । अ॒सि॒ । बृ॒ह॒ती । दिक् । म॒रुतः॑ । ते॒ । दे॒वाः । अधि॑पतय॒ इत्यधि॑ - प॒त॒यः॒ ।  \newline


\textbf{Krama Paata} \newline

आ॒दि॒त्यास्ते᳚ । ते॒ दे॒वाः । दे॒वा अधि॑पतयः । अधि॑पतयः॒ सोमः॑ । अधि॑पतय॒ इत्यधि॑ - प॒त॒यः॒ । सोमो॑ हेती॒नाम् । हे॒ती॒नाम् प्र॑तिध॒र्ता । प्र॒ति॒ध॒र्ता स॑प्तद॒शः । प्र॒ति॒ध॒र्तेति॑ प्रति - ध॒र्ता । स॒प्त॒द॒शस्त्वा᳚ । स॒प्त॒द॒श इति॑ सप्त - द॒शः । त्वा॒ स्तोमः॑ । स्तोमः॑ पृथि॒व्याम् । पृ॒थि॒व्याꣳ श्र॑यतु । श्र॒य॒तु॒ म॒रु॒त्व॒तीय᳚म् । म॒रु॒त्व॒तीय॑मु॒क्थम् । उ॒क्थमव्य॑थयत् । अव्य॑थयथ् स्तभ्नातु । स्त॒भ्ना॒तु॒ वै॒रू॒पम् । वै॒रू॒पꣳ साम॑ । साम॒ प्रति॑ष्ठित्यै । प्रति॑ष्ठित्यै स्व॒राट् । प्रति॑ष्ठित्या॒ इति॒ प्रति॑ - स्थि॒त्यै॒ । स्व॒राड॑सि । स्व॒राडिति॑ स्व - राट् । अ॒स्युदी॑ची । उदी॑ची॒ दिक् । दिग् विश्वे᳚ । विश्वे॑ ते । ते॒ दे॒वाः । दे॒वा अधि॑पतयः । अधि॑पतयो॒ वरु॑णः । अधि॑पतय॒ इत्यधि॑ - प॒त॒यः॒ । वरु॑णो हेती॒नाम् । हे॒ती॒नाम् प्र॑तिध॒र्ता । प्र॒ति॒ध॒र्तैक॑विꣳ॒॒शः । प्र॒ति॒ध॒र्तेति॑ प्रति - ध॒र्ता । ए॒क॒विꣳ॒॒शस्त्वा᳚ । ए॒क॒विꣳ॒॒श इत्ये॑क - विꣳ॒॒शः । त्वा॒ स्तोमः॑ । स्तोमः॑ पृथि॒व्याम् । पृ॒थि॒व्याꣳ श्र॑यतु । श्र॒य॒तु॒ निष्के॑वल्यम् । निष्के॑वल्यमु॒क्थम् । उ॒क्थमव्य॑थयत् । अव्य॑थयथ् स्तभ्नातु । स्त॒भ्ना॒तु॒ वै॒रा॒जम् । वै॒रा॒जꣳ साम॑ । साम॒ प्रति॑ष्ठित्यै । प्रति॑ष्ठित्या॒ अधि॑पत्नी । प्रति॑ष्ठित्या॒ इति॒ प्रति॑ - स्थि॒त्यै॒ । अधि॑पत्न्यसि । अधि॑प॒त्नीत्यधि॑ - प॒त्नी॒ । अ॒सि॒ बृ॒ह॒ती । बृ॒ह॒ती दिक् । दिङ्म॒रुतः॑ । म॒रुत॑स्ते । ते॒ दे॒वाः । दे॒वा अधि॑पतयः ( ) । अधि॑पतयो॒ बृह॒स्पतिः॑ । अधि॑पतय॒ इत्यधि॑ - प॒त॒यः॒ \newline

\textbf{Jatai Paata} \newline

1. आ॒दि॒त्या स्ते॑ त आदि॒त्या आ॑दि॒त्या स्ते᳚ । \newline
2. ते॒ दे॒वा दे॒वा स्ते॑ ते दे॒वाः । \newline
3. दे॒वा अधि॑पत॒यो ऽधि॑पतयो दे॒वा दे॒वा अधि॑पतयः । \newline
4. अधि॑पतयः॒ सोमः॒ सोमो ऽधि॑पत॒यो ऽधि॑पतयः॒ सोमः॑ । \newline
5. अधि॑पतय॒ इत्यधि॑ - प॒त॒यः॒ । \newline
6. सोमो॑ हेती॒नाꣳ हे॑ती॒नाꣳ सोमः॒ सोमो॑ हेती॒नाम् । \newline
7. हे॒ती॒नाम् प्र॑तिध॒र्ता प्र॑तिध॒र्ता हे॑ती॒नाꣳ हे॑ती॒नाम् प्र॑तिध॒र्ता । \newline
8. प्र॒ति॒ध॒र्ता स॑प्तद॒शः स॑प्तद॒शः प्र॑तिध॒र्ता प्र॑तिध॒र्ता स॑प्तद॒शः । \newline
9. प्र॒ति॒ध॒र्तेति॑ प्रति - ध॒र्ता । \newline
10. स॒प्त॒द॒श स्त्वा᳚ त्वा सप्तद॒शः स॑प्तद॒श स्त्वा᳚ । \newline
11. स॒प्त॒द॒श इति॑ सप्त - द॒शः । \newline
12. त्वा॒ स्तोमः॒ स्तोम॑ स्त्वा त्वा॒ स्तोमः॑ । \newline
13. स्तोमः॑ पृथि॒व्याम् पृ॑थि॒व्याꣳ स्तोमः॒ स्तोमः॑ पृथि॒व्याम् । \newline
14. पृ॒थि॒व्याꣳ श्र॑यतु श्रयतु पृथि॒व्याम् पृ॑थि॒व्याꣳ श्र॑यतु । \newline
15. श्र॒य॒तु॒ म॒रु॒त्व॒तीय॑म् मरुत्व॒तीयꣳ॑ श्रयतु श्रयतु मरुत्व॒तीय᳚म् । \newline
16. म॒रु॒त्व॒तीय॑ मु॒क्थ मु॒क्थम् म॑रुत्व॒तीय॑म् मरुत्व॒तीय॑ मु॒क्थम् । \newline
17. उ॒क्थ मव्य॑थय॒ दव्य॑थय दु॒क्थ मु॒क्थ मव्य॑थयत् । \newline
18. अव्य॑थयथ् स्तभ्नातु स्तभ्ना॒ त्वव्य॑थय॒ दव्य॑थयथ् स्तभ्नातु । \newline
19. स्त॒भ्ना॒तु॒ वै॒रू॒पं ॅवै॑रू॒पꣳ स्त॑भ्नातु स्तभ्नातु वैरू॒पम् । \newline
20. वै॒रू॒पꣳ साम॒ साम॑ वैरू॒पं ॅवै॑रू॒पꣳ साम॑ । \newline
21. साम॒ प्रति॑ष्ठित्यै॒ प्रति॑ष्ठित्यै॒ साम॒ साम॒ प्रति॑ष्ठित्यै । \newline
22. प्रति॑ष्ठित्यै स्व॒राट् थ्स्व॒राट् प्रति॑ष्ठित्यै॒ प्रति॑ष्ठित्यै स्व॒राट् । \newline
23. प्रति॑ष्ठित्या॒ इति॒ प्रति॑ - स्थि॒त्यै॒ । \newline
24. स्व॒रा ड॑स्यसि स्व॒राट् थ्स्व॒रा ड॑सि । \newline
25. स्व॒राडिति॑ स्व - राट् । \newline
26. अ॒स्यु दी॒च्यु दी᳚च्य स्य॒ स्युदी॑ची । \newline
27. उदी॑ची॒ दिग् दिगुदी॒ च्युदी॑ची॒ दिक् । \newline
28. दिग् विश्वे॒ विश्वे॒ दिग् दिग् विश्वे᳚ । \newline
29. विश्वे॑ ते ते॒ विश्वे॒ विश्वे॑ ते । \newline
30. ते॒ दे॒वा दे॒वा स्ते॑ ते दे॒वाः । \newline
31. दे॒वा अधि॑पत॒यो ऽधि॑पतयो दे॒वा दे॒वा अधि॑पतयः । \newline
32. अधि॑पतयो॒ वरु॑णो॒ वरु॒णो ऽधि॑पत॒यो ऽधि॑पतयो॒ वरु॑णः । \newline
33. अधि॑पतय॒ इत्यधि॑ - प॒त॒यः॒ । \newline
34. वरु॑णो हेती॒नाꣳ हे॑ती॒नां ॅवरु॑णो॒ वरु॑णो हेती॒नाम् । \newline
35. हे॒ती॒नाम् प्र॑तिध॒र्ता प्र॑तिध॒र्ता हे॑ती॒नाꣳ हे॑ती॒नाम् प्र॑तिध॒र्ता । \newline
36. प्र॒ति॒ध॒र्तै क॑विꣳ॒॒श ए॑कविꣳ॒॒शः प्र॑तिध॒र्ता प्र॑तिध॒र्तै क॑विꣳ॒॒शः । \newline
37. प्र॒ति॒ध॒र्तेति॑ प्रति - ध॒र्ता । \newline
38. ए॒क॒विꣳ॒॒श स्त्वा᳚ त्वैकविꣳ॒॒श ए॑कविꣳ॒॒श स्त्वा᳚ । \newline
39. ए॒क॒विꣳ॒॒श इत्ये॑क - विꣳ॒॒शः । \newline
40. त्वा॒ स्तोमः॒ स्तोम॑ स्त्वा त्वा॒ स्तोमः॑ । \newline
41. स्तोमः॑ पृथि॒व्याम् पृ॑थि॒व्याꣳ स्तोमः॒ स्तोमः॑ पृथि॒व्याम् । \newline
42. पृ॒थि॒व्याꣳ श्र॑यतु श्रयतु पृथि॒व्याम् पृ॑थि॒व्याꣳ श्र॑यतु । \newline
43. श्र॒य॒तु॒ निष्के॑वल्य॒न् निष्के॑वल्यꣳ श्रयतु श्रयतु॒ निष्के॑वल्यम् । \newline
44. निष्के॑वल्य मु॒क्थ मु॒क्थन् निष्के॑वल्य॒म् निष्के॑वल्य मु॒क्थम् । \newline
45. उ॒क्थ मव्य॑थय॒ दव्य॑थय दु॒क्थ मु॒क्थ मव्य॑थयत् । \newline
46. अव्य॑थयथ् स्तभ्नातु स्तभ्ना॒ त्वव्य॑थय॒ दव्य॑थयथ् स्तभ्नातु । \newline
47. स्त॒भ्ना॒तु॒ वै॒रा॒जं ॅवै॑रा॒जꣳ स्त॑भ्नातु स्तभ्नातु वैरा॒जम् । \newline
48. वै॒रा॒जꣳ साम॒ साम॑ वैरा॒जं ॅवै॑रा॒जꣳ साम॑ । \newline
49. साम॒ प्रति॑ष्ठित्यै॒ प्रति॑ष्ठित्यै॒ साम॒ साम॒ प्रति॑ष्ठित्यै । \newline
50. प्रति॑ष्ठित्या॒ अधि॑प॒त्न्य धि॑पत्नी॒ प्रति॑ष्ठित्यै॒ प्रति॑ष्ठित्या॒ अधि॑पत्नी । \newline
51. प्रति॑ष्ठित्या॒ इति॒ प्रति॑ - स्थि॒त्यै॒ । \newline
52. अधि॑पत् न्यस्य॒ स्यधि॑प॒ त्न्यधि॑पत्न्यसि । \newline
53. अधि॑प॒त्नीत्यधि॑ - प॒त्नी॒ । \newline
54. अ॒सि॒ बृ॒ह॒ती बृ॑ह॒ त्य॑स्यसि बृह॒ती । \newline
55. बृ॒ह॒ती दिग् दिग् बृ॑ह॒ती बृ॑ह॒ती दिक् । \newline
56. दिङ् म॒रुतो॑ म॒रुतो॒ दिग् दिङ् म॒रुतः॑ । \newline
57. म॒रुत॑ स्ते ते म॒रुतो॑ म॒रुत॑ स्ते । \newline
58. ते॒ दे॒वा दे॒वा स्ते॑ ते दे॒वाः । \newline
59. दे॒वा अधि॑पत॒यो ऽधि॑पतयो दे॒वा दे॒वा अधि॑पतयः । \newline
60. अधि॑पतयो॒ बृह॒स्पति॒र् बृह॒स्पति॒ रधि॑पत॒यो ऽधि॑पतयो॒ बृह॒स्पतिः॑ । \newline
61. अधि॑पतय॒ इत्यधि॑ - प॒त॒यः॒ । \newline

\textbf{Ghana Paata } \newline

1. आ॒दि॒त्या स्ते॑ त आदि॒त्या आ॑दि॒त्या स्ते॑ दे॒वा दे॒वा स्त॑ आदि॒त्या आ॑दि॒त्या स्ते॑ दे॒वाः । \newline
2. ते॒ दे॒वा दे॒वा स्ते॑ ते दे॒वा अधि॑पत॒यो ऽधि॑पतयो दे॒वा स्ते॑ ते दे॒वा अधि॑पतयः । \newline
3. दे॒वा अधि॑पत॒यो ऽधि॑पतयो दे॒वा दे॒वा अधि॑पतयः॒ सोमः॒ सोमो ऽधि॑पतयो दे॒वा दे॒वा अधि॑पतयः॒ सोमः॑ । \newline
4. अधि॑पतयः॒ सोमः॒ सोमो ऽधि॑पत॒यो ऽधि॑पतयः॒ सोमो॑ हेती॒नाꣳ हे॑ती॒नाꣳ सोमो ऽधि॑पत॒यो ऽधि॑पतयः॒ सोमो॑ हेती॒नाम् । \newline
5. अधि॑पतय॒ इत्यधि॑ - प॒त॒यः॒ । \newline
6. सोमो॑ हेती॒नाꣳ हे॑ती॒नाꣳ सोमः॒ सोमो॑ हेती॒नाम् प्र॑तिध॒र्ता प्र॑तिध॒र्ता हे॑ती॒नाꣳ सोमः॒ सोमो॑ हेती॒नाम् प्र॑तिध॒र्ता । \newline
7. हे॒ती॒नाम् प्र॑तिध॒र्ता प्र॑तिध॒र्ता हे॑ती॒नाꣳ हे॑ती॒नाम् प्र॑तिध॒र्ता स॑प्तद॒शः स॑प्तद॒शः प्र॑तिध॒र्ता हे॑ती॒नाꣳ हे॑ती॒नाम् प्र॑तिध॒र्ता स॑प्तद॒शः । \newline
8. प्र॒ति॒ध॒र्ता स॑प्तद॒शः स॑प्तद॒शः प्र॑तिध॒र्ता प्र॑तिध॒र्ता स॑प्तद॒श स्त्वा᳚ त्वा सप्तद॒शः प्र॑तिध॒र्ता प्र॑तिध॒र्ता स॑प्तद॒श स्त्वा᳚ । \newline
9. प्र॒ति॒ध॒र्तेति॑ प्रति - ध॒र्ता । \newline
10. स॒प्त॒द॒श स्त्वा᳚ त्वा सप्तद॒शः स॑प्तद॒श स्त्वा॒ स्तोमः॒ स्तोम॑ स्त्वा सप्तद॒शः स॑प्तद॒श स्त्वा॒ स्तोमः॑ । \newline
11. स॒प्त॒द॒श इति॑ सप्त - द॒शः । \newline
12. त्वा॒ स्तोमः॒ स्तोम॑ स्त्वा त्वा॒ स्तोमः॑ पृथि॒व्याम् पृ॑थि॒व्याꣳ स्तोम॑ स्त्वा त्वा॒ स्तोमः॑ पृथि॒व्याम् । \newline
13. स्तोमः॑ पृथि॒व्याम् पृ॑थि॒व्याꣳ स्तोमः॒ स्तोमः॑ पृथि॒व्याꣳ श्र॑यतु श्रयतु पृथि॒व्याꣳ स्तोमः॒ स्तोमः॑ पृथि॒व्याꣳ श्र॑यतु । \newline
14. पृ॒थि॒व्याꣳ श्र॑यतु श्रयतु पृथि॒व्याम् पृ॑थि॒व्याꣳ श्र॑यतु मरुत्व॒तीय॑म् मरुत्व॒तीयꣳ॑ श्रयतु पृथि॒व्याम् पृ॑थि॒व्याꣳ श्र॑यतु मरुत्व॒तीय᳚म् । \newline
15. श्र॒य॒तु॒ म॒रु॒त्व॒तीय॑म् मरुत्व॒तीयꣳ॑ श्रयतु श्रयतु मरुत्व॒तीय॑ मु॒क्थ मु॒क्थम् म॑रुत्व॒तीयꣳ॑ श्रयतु श्रयतु मरुत्व॒तीय॑ मु॒क्थम् । \newline
16. म॒रु॒त्व॒तीय॑ मु॒क्थ मु॒क्थम् म॑रुत्व॒तीय॑म् मरुत्व॒तीय॑ मु॒क्थ मव्य॑थय॒ दव्य॑थय दु॒क्थम् म॑रुत्व॒तीय॑म् मरुत्व॒तीय॑ मु॒क्थ मव्य॑थयत् । \newline
17. उ॒क्थ मव्य॑थय॒ दव्य॑थय दु॒क्थ मु॒क्थ मव्य॑थयथ् स्तभ्नातु स्तभ्ना॒ त्वव्य॑थय दु॒क्थ मु॒क्थ मव्य॑थयथ् स्तभ्नातु । \newline
18. अव्य॑थयथ् स्तभ्नातु स्तभ्ना॒ त्वव्य॑थय॒ दव्य॑थयथ् स्तभ्नातु वैरू॒पं ॅवै॑रू॒पꣳ 
स्त॑भ्ना॒ त्वव्य॑थय॒ दव्य॑थयथ् स्तभ्नातु वैरू॒पम् । \newline
19. स्त॒भ्ना॒तु॒ वै॒रू॒पं ॅवै॑रू॒पꣳ स्त॑भ्नातु स्तभ्नातु वैरू॒पꣳ साम॒ साम॑ वैरू॒पꣳ स्त॑भ्नातु 
स्तभ्नातु वैरू॒पꣳ साम॑ । \newline
20. वै॒रू॒पꣳ साम॒ साम॑ वैरू॒पं ॅवै॑रू॒पꣳ साम॒ प्रति॑ष्ठित्यै॒ प्रति॑ष्ठित्यै॒ साम॑ वैरू॒पं ॅवै॑रू॒पꣳ साम॒ प्रति॑ष्ठित्यै । \newline
21. साम॒ प्रति॑ष्ठित्यै॒ प्रति॑ष्ठित्यै॒ साम॒ साम॒ प्रति॑ष्ठित्यै स्व॒राट् थ्स्व॒राट् प्रति॑ष्ठित्यै॒ साम॒ साम॒ प्रति॑ष्ठित्यै स्व॒राट् । \newline
22. प्रति॑ष्ठित्यै स्व॒राट् थ्स्व॒राट् प्रति॑ष्ठित्यै॒ प्रति॑ष्ठित्यै स्व॒रा ड॑स्यसि स्व॒राट् प्रति॑ष्ठित्यै॒ प्रति॑ष्ठित्यै स्व॒राड॑सि । \newline
23. प्रति॑ष्ठित्या॒ इति॒ प्रति॑ - स्थि॒त्यै॒ । \newline
24. स्व॒रा ड॑स्यसि स्व॒राट् थ्स्व॒रा ड॒स्युदी॒ च्युदी᳚ च्यसि स्व॒राट् थ्स्व॒राड॒ स्युदी॑ची । \newline
25. स्व॒राडिति॑ स्व - राट् । \newline
26. अ॒स्युदी॒ च्युदी᳚ च्यस्य॒ स्युदी॑ची॒ दिग् दिगुदी᳚ च्यस्य॒ स्युदी॑ची॒ दिक् । \newline
27. उदी॑ची॒ दिग् दिगुदी॒ च्युदी॑ची॒ दिग् विश्वे॒ विश्वे॒ दिगुदी॒ च्युदी॑ची॒ दिग् विश्वे᳚ । \newline
28. दिग् विश्वे॒ विश्वे॒ दिग् दिग् विश्वे॑ ते ते॒ विश्वे॒ दिग् दिग् विश्वे॑ ते । \newline
29. विश्वे॑ ते ते॒ विश्वे॒ विश्वे॑ ते दे॒वा दे॒वा स्ते॒ विश्वे॒ विश्वे॑ ते दे॒वाः । \newline
30. ते॒ दे॒वा दे॒वा स्ते॑ ते दे॒वा अधि॑पत॒यो ऽधि॑पतयो दे॒वा स्ते॑ ते दे॒वा अधि॑पतयः । \newline
31. दे॒वा अधि॑पत॒यो ऽधि॑पतयो दे॒वा दे॒वा अधि॑पतयो॒ वरु॑णो॒ वरु॒णो ऽधि॑पतयो दे॒वा दे॒वा अधि॑पतयो॒ वरु॑णः । \newline
32. अधि॑पतयो॒ वरु॑णो॒ वरु॒णो ऽधि॑पत॒यो ऽधि॑पतयो॒ वरु॑णो हेती॒नाꣳ हे॑ती॒नां ॅवरु॒णो ऽधि॑पत॒यो ऽधि॑पतयो॒ वरु॑णो हेती॒नाम् । \newline
33. अधि॑पतय॒ इत्यधि॑ - प॒त॒यः॒ । \newline
34. वरु॑णो हेती॒नाꣳ हे॑ती॒नां ॅवरु॑णो॒ वरु॑णो हेती॒नाम् प्र॑तिध॒र्ता प्र॑तिध॒र्ता हे॑ती॒नां ॅवरु॑णो॒ वरु॑णो हेती॒नाम् प्र॑तिध॒र्ता । \newline
35. हे॒ती॒नाम् प्र॑तिध॒र्ता प्र॑तिध॒र्ता हे॑ती॒नाꣳ हे॑ती॒नाम् प्र॑तिध॒र्तैक॑विꣳ॒॒श ए॑कविꣳ॒॒शः प्र॑तिध॒र्ता हे॑ती॒नाꣳ हे॑ती॒नाम् प्र॑तिध॒र्तैक॑विꣳ॒॒शः । \newline
36. प्र॒ति॒ध॒र्तै क॑विꣳ॒॒श ए॑कविꣳ॒॒शः प्र॑तिध॒र्ता प्र॑तिध॒र्तैक॑विꣳ॒॒श स्त्वा᳚ त्वैकविꣳ॒॒शः प्र॑तिध॒र्ता प्र॑तिध॒र्तैक॑विꣳ॒॒श स्त्वा᳚ । \newline
37. प्र॒ति॒ध॒र्तेति॑ प्रति - ध॒र्ता । \newline
38. ए॒क॒विꣳ॒॒श स्त्वा᳚ त्वैकविꣳ॒॒श ए॑कविꣳ॒॒श स्त्वा॒ स्तोमः॒ स्तोम॑ स्त्वैकविꣳ॒॒श ए॑कविꣳ॒॒श स्त्वा॒ स्तोमः॑ । \newline
39. ए॒क॒विꣳ॒॒श इत्ये॑क - विꣳ॒॒शः । \newline
40. त्वा॒ स्तोमः॒ स्तोम॑ स्त्वा त्वा॒ स्तोमः॑ पृथि॒व्याम् पृ॑थि॒व्याꣳ स्तोम॑ स्त्वा त्वा॒ स्तोमः॑ पृथि॒व्याम् । \newline
41. स्तोमः॑ पृथि॒व्याम् पृ॑थि॒व्याꣳ स्तोमः॒ स्तोमः॑ पृथि॒व्याꣳ श्र॑यतु श्रयतु पृथि॒व्याꣳ स्तोमः॒ स्तोमः॑ पृथि॒व्याꣳ श्र॑यतु । \newline
42. पृ॒थि॒व्याꣳ श्र॑यतु श्रयतु पृथि॒व्याम् पृ॑थि॒व्याꣳ श्र॑यतु॒ निष्के॑वल्य॒म् निष्के॑वल्यꣳ श्रयतु पृथि॒व्याम् पृ॑थि॒व्याꣳ श्र॑यतु॒ निष्के॑वल्यम् । \newline
43. श्र॒य॒तु॒ निष्के॑वल्य॒म् निष्के॑वल्यꣳ श्रयतु श्रयतु॒ निष्के॑वल्य मु॒क्थ मु॒क्थम् निष्के॑वल्यꣳ श्रयतु श्रयतु॒ निष्के॑वल्य मु॒क्थम् । \newline
44. निष्के॑वल्य मु॒क्थ मु॒क्थम् निष्के॑वल्य॒म् निष्के॑वल्य मु॒क्थ मव्य॑थय॒ दव्य॑थय दु॒क्थम् निष्के॑वल्य॒म् निष्के॑वल्य मु॒क्थ मव्य॑थयत् । \newline
45. उ॒क्थ मव्य॑थय॒ दव्य॑थय दु॒क्थ मु॒क्थ मव्य॑थयथ् स्तभ्नातु स्तभ्ना॒ त्वव्य॑थय दु॒क्थ मु॒क्थ मव्य॑थयथ् स्तभ्नातु । \newline
46. अव्य॑थयथ् स्तभ्नातु स्तभ्ना॒ त्वव्य॑थय॒ दव्य॑थयथ् स्तभ्नातु वैरा॒जं ॅवै॑रा॒जꣳ स्त॑भ्ना॒ त्वव्य॑थय॒ दव्य॑थयथ् स्तभ्नातु वैरा॒जम् । \newline
47. स्त॒भ्ना॒तु॒ वै॒रा॒जं ॅवै॑रा॒जꣳ स्त॑भ्नातु स्तभ्नातु वैरा॒जꣳ साम॒ साम॑ वैरा॒जꣳ स्त॑भ्नातु स्तभ्नातु वैरा॒जꣳ साम॑ । \newline
48. वै॒रा॒जꣳ साम॒ साम॑ वैरा॒जं ॅवै॑रा॒जꣳ साम॒ प्रति॑ष्ठित्यै॒ प्रति॑ष्ठित्यै॒ साम॑ वैरा॒जं ॅवै॑रा॒जꣳ साम॒ प्रति॑ष्ठित्यै । \newline
49. साम॒ प्रति॑ष्ठित्यै॒ प्रति॑ष्ठित्यै॒ साम॒ साम॒ प्रति॑ष्ठित्या॒ अधि॑प॒त् न्यधि॑पत्नी॒ प्रति॑ष्ठित्यै॒ साम॒ साम॒ प्रति॑ष्ठित्या॒ अधि॑पत्नी । \newline
50. प्रति॑ष्ठित्या॒ अधि॑प॒त् न्यधि॑पत्नी॒ प्रति॑ष्ठित्यै॒ प्रति॑ष्ठित्या॒ अधि॑पत् न्यस्य॒स्य धि॑पत्नी॒ प्रति॑ष्ठित्यै॒ प्रति॑ष्ठित्या॒ अधि॑पत्न्यसि । \newline
51. प्रति॑ष्ठित्या॒ इति॒ प्रति॑ - स्थि॒त्यै॒ । \newline
52. अधि॑पत् न्यस्य॒स्य धि॑प॒त् न्यधि॑पत् न्यसि बृह॒ती बृ॑ह॒ त्य॑स्यधि॑प॒त् न्यधि॑पत् न्यसि बृह॒ती । \newline
53. अधि॑प॒त्नीत्यधि॑ - प॒त्नी॒ । \newline
54. अ॒सि॒ बृ॒ह॒ती बृ॑ह॒ त्य॑स्यसि बृह॒ती दिग् दिग् बृ॑ह॒ त्य॑स्यसि बृह॒ती दिक् । \newline
55. बृ॒ह॒ती दिग् दिग् बृ॑ह॒ती बृ॑ह॒ती दिङ् म॒रुतो॑ म॒रुतो॒ दिग् बृ॑ह॒ती बृ॑ह॒ती दिङ् म॒रुतः॑ । \newline
56. दिङ् म॒रुतो॑ म॒रुतो॒ दिग् दिङ् म॒रुत॑ स्ते ते म॒रुतो॒ दिग् दिङ् म॒रुत॑ स्ते । \newline
57. म॒रुत॑ स्ते ते म॒रुतो॑ म॒रुत॑ स्ते दे॒वा दे॒वा स्ते॑ म॒रुतो॑ म॒रुत॑ स्ते दे॒वाः । \newline
58. ते॒ दे॒वा दे॒वा स्ते॑ ते दे॒वा अधि॑पत॒यो ऽधि॑पतयो दे॒वा स्ते॑ ते दे॒वा अधि॑पतयः । \newline
59. दे॒वा अधि॑पत॒यो ऽधि॑पतयो दे॒वा दे॒वा अधि॑पतयो॒ बृह॒स्पति॒र् बृह॒स्पति॒ रधि॑पतयो दे॒वा दे॒वा अधि॑पतयो॒ बृह॒स्पतिः॑ । \newline
60. अधि॑पतयो॒ बृह॒स्पति॒र् बृह॒स्पति॒ रधि॑पत॒यो ऽधि॑पतयो॒ बृह॒स्पति॑र्. हेती॒नाꣳ हे॑ती॒नाम् बृह॒स्पति॒ रधि॑पत॒यो ऽधि॑पतयो॒ बृह॒स्पति॑र्. हेती॒नाम् । \newline
61. अधि॑पतय॒ इत्यधि॑ - प॒त॒यः॒ । \newline
\pagebreak
\markright{ TS 4.4.2.3  \hfill https://www.vedavms.in \hfill}

\section{ TS 4.4.2.3 }

\textbf{TS 4.4.2.3 } \newline
\textbf{Samhita Paata} \newline

बृह॒स्पति॑र्.हेती॒नां प्र॑तिध॒र्ता त्रि॑णवत्रयस्त्रिꣳ॒॒शौ त्वा॒ स्तोमौ॑ पृथि॒व्याꣳ श्र॑यतां ॅवैश्वदेवाग्निमारु॒ते उ॒क्थे अव्य॑थयन्ती स्तभ्नीताꣳ शाक्वररैव॒ते साम॑नी॒ प्रति॑ष्ठित्या अ॒न्तरि॑क्षा॒यर्.ष॑यस्त्वा प्रथम॒जा दे॒वेषु॑ दि॒वो मात्र॑या वरि॒णा प्र॑थन्तु विध॒र्ता चा॒यमधि॑पतिश्च॒ ते त्वा॒ सर्वे॑ संॅविदा॒ना नाक॑स्य पृ॒ष्ठे सु॑व॒र्गे लो॒के यज॑मानं च सादयन्तु ॥ \newline

\textbf{Pada Paata} \newline

बृह॒स्पतिः॑ । हे॒ती॒नाम् । प्र॒ति॒ध॒र्तेति॑ प्रति - ध॒र्ता । त्रि॒ण॒व॒त्र॒य॒स्त्रिꣳ॒॒शाविति॑ त्रिणव - त्र॒य॒स्त्रिꣳ॒॒शौ । त्वा॒ । स्तोमौ᳚ । पृ॒थि॒व्याम् । श्र॒य॒ता॒म् । वै॒श्व॒दे॒वा॒ग्नि॒मा॒रु॒ते इति॑ वैश्वदेव-अ॒ग्नि॒मा॒रु॒ते । उ॒क्थे इति॑ । अव्य॑थयन्ती॒ इति॑ । स्त॒भ्नी॒ता॒म् । शा॒क्व॒र॒रै॒व॒ते इति॑ शाक्वर - रै॒व॒ते । साम॑नी॒ इति॑ । प्रति॑ष्ठित्या॒ इति॒ प्रति॑ - स्थि॒त्यै॒ । अ॒न्तरि॑क्षाय । ऋष॑यः । त्वा॒ । प्र॒थ॒म॒जा इति॑ प्रथम - जाः । दे॒वेषु॑ । दि॒वः । मात्र॑या । व॒रि॒णा । प्र॒थ॒न्तु॒ । वि॒ध॒र्तेति॑ वि - ध॒र्ता । च॒ । अ॒यम् । अधि॑पति॒रित्यधि॑-प॒तिः॒ । च॒ । ते । त्वा॒ । सर्वे᳚ । सं॒ॅवि॒दा॒ना इति॑ सं - वि॒दा॒नाः । नाक॑स्य । पृ॒ष्ठे । सु॒व॒र्ग इति॑ सुवः - गे । लो॒के । यज॑मानम् । च॒ । सा॒द॒य॒न्तु॒ ॥  \newline


\textbf{Krama Paata} \newline

बृह॒स्पति॑र्. हेती॒नाम् । हे॒ती॒नाम् प्र॑तिध॒र्ता । प्र॒ति॒ध॒र्ता त्रि॑णवत्रयस्त्रिꣳ॒॒शौ । प्र॒ति॒ध॒र्तेति॑ प्रति - ध॒र्ता । त्रि॒ण॒व॒त्र॒य॒स्त्रिꣳ॒॒शौ त्वा᳚ । त्रि॒ण॒व॒त्र॒य॒स्त्रिꣳ॒॒शाविति॑ त्रिणव - त्र॒य॒स्त्रिꣳ॒॒शौ । त्वा॒ स्तोमौ᳚ । स्तोमौ॑ पृथि॒व्याम् । पृ॒थि॒व्याꣳ श्र॑यताम् । श्र॒य॒तां॒ ॅवै॒श्व॒दे॒वा॒ग्नि॒मा॒रु॒ते । वै॒श्व॒दे॒वा॒ग्नि॒मा॒रु॒ते उ॒क्थे । वै॒श्व॒दे॒वा॒ग्नि॒मा॒रु॒ते इति॑ वैश्वदेव - आ॒ग्नि॒मा॒रु॒ते । उ॒क्थे अव्य॑थयन्ती । उ॒क्थे इत्यु॒क्थे । अव्य॑थयन्ती स्तभ्नीताम् । अव्य॑थयन्ती॒ इत्यव्य॑थयन्ती । स्त॒भ्नी॒ताꣳ॒॒ शा॒क्व॒र॒रै॒व॒ते । शा॒क्व॒र॒रै॒व॒ते साम॑नी । शा॒क्व॒र॒रै॒व॒ते इति॑ शाक्वर - रै॒व॒ते । साम॑नी॒ प्रति॑ष्ठित्यै । साम॑नी॒ इति॒ साम॑नी । प्रति॑ष्ठित्या अ॒न्तरि॑क्षाय । प्रति॑ष्ठित्या॒ इति॒ प्रति॑ - स्थि॒त्यै॒ । अ॒न्तरि॑क्षा॒यर्.ष॑यः । ऋष॑यस्त्वा । त्वा॒ प्र॒थ॒म॒जाः । प्र॒थ॒म॒जा दे॒वेषु॑ । प्र॒थ॒म॒जा इति॑ प्रथम - जाः । दे॒वेषु॑ दि॒वः । दि॒वो मात्र॑या । मात्र॑या वरि॒णा । व॒रि॒णा प्र॑थन्तु । प्र॒थ॒न्तु॒ वि॒ध॒र्ता । वि॒ध॒र्ता च॑ । वि॒ध॒र्तेति॑ वि - ध॒र्ता । चा॒यम् । अ॒यमधि॑पतिः । अधि॑पतिश्च । अधि॑पति॒रित्यधि॑ - प॒तिः॒ । च॒ ते । ते त्वा᳚ । त्वा॒ सर्वे᳚ । सर्वे॑ सम्ॅविदा॒नाः । स॒म्ॅवि॒दा॒ना नाक॑स्य । स॒म्ॅवि॒दा॒ना इति॑ सम् - वि॒दा॒नाः । नाक॑स्य पृ॒ष्ठे । पृ॒ष्ठे सु॑व॒र्गे । सु॒व॒र्गे लो॒के । सु॒व॒र्ग इति॑ सुवः - गे । लो॒के यज॑मानम् । यज॑मानम् च । च॒ सा॒द॒य॒न्तु॒ । सा॒द॒य॒न्त्विति॑ सादयन्तु । \newline

\textbf{Jatai Paata} \newline

1. बृह॒स्पति॑र्. हेती॒नाꣳ हे॑ती॒नाम् बृह॒स्पति॒र् बृह॒स्पति॑र्. हेती॒नाम् । \newline
2. हे॒ती॒नाम् प्र॑तिध॒र्ता प्र॑तिध॒र्ता हे॑ती॒नाꣳ हे॑ती॒नाम् प्र॑तिध॒र्ता । \newline
3. प्र॒ति॒ध॒र्ता त्रि॑णवत्रयस्त्रिꣳ॒॒शौ त्रि॑णवत्रयस्त्रिꣳ॒॒शौ प्र॑तिध॒र्ता प्र॑तिध॒र्ता त्रि॑णवत्रयस्त्रिꣳ॒॒शौ । \newline
4. प्र॒ति॒ध॒र्तेति॑ प्रति - ध॒र्ता । \newline
5. त्रि॒ण॒व॒त्र॒य॒स्त्रिꣳ॒॒शौ त्वा᳚ त्वा त्रिणवत्रयस्त्रिꣳ॒॒शौ त्रि॑णवत्रयस्त्रिꣳ॒॒शौ त्वा᳚ । \newline
6. त्रि॒ण॒व॒त्र॒य॒स्त्रिꣳ॒॒शाविति॑ त्रिणव - त्र॒य॒स्त्रिꣳ॒॒शौ । \newline
7. त्वा॒ स्तोमौ॒ स्तोमौ᳚ त्वा त्वा॒ स्तोमौ᳚ । \newline
8. स्तोमौ॑ पृथि॒व्याम् पृ॑थि॒व्याꣳ स्तोमौ॒ स्तोमौ॑ पृथि॒व्याम् । \newline
9. पृ॒थि॒व्याꣳ श्र॑यताꣳ श्रयताम् पृथि॒व्याम् पृ॑थि॒व्याꣳ श्र॑यताम् । \newline
10. श्र॒य॒तां॒ ॅवै॒श्व॒दे॒वा॒ग्नि॒मा॒रु॒ते वै᳚श्वदेवाग्निमारु॒ते श्र॑यताꣳ श्रयतां ॅवैश्वदेवाग्निमारु॒ते । \newline
11. वै॒श्व॒दे॒वा॒ग्नि॒मा॒रु॒ते उ॒क्थे उ॒क्थे वै᳚श्वदेवाग्निमारु॒ते वै᳚श्वदेवाग्निमारु॒ते उ॒क्थे । \newline
12. वै॒श्व॒दे॒वा॒ग्नि॒मा॒रु॒ते इति॑ वैश्वदेव - आ॒ग्नि॒मा॒रु॒ते । \newline
13. उ॒क्थे अव्य॑थयन्ती॒ अव्य॑थयन्ती उ॒क्थे उ॒क्थे अव्य॑थयन्ती । \newline
14. उ॒क्थे इत्यु॒क्थे । \newline
15. अव्य॑थयन्ती स्तभ्नीताꣳ स्तभ्नीता॒ मव्य॑थयन्ती॒ अव्य॑थयन्ती स्तभ्नीताम् । \newline
16. अव्य॑थयन्ती॒ इतियव्य॑थयन्ती । \newline
17. स्त॒भ्नी॒ताꣳ॒॒ शा॒क्व॒र॒रै॒व॒ते शा᳚क्वररैव॒ते स्त॑भ्नीताꣳ स्तभ्नीताꣳ शाक्वररैव॒ते । \newline
18. शा॒क्व॒र॒रै॒व॒ते साम॑नी॒ साम॑नी शाक्वररैव॒ते शा᳚क्वररैव॒ते साम॑नी । \newline
19. शा॒क्व॒र॒रै॒व॒ते इति॑ शाक्वर - रै॒व॒ते । \newline
20. साम॑नी॒ प्रति॑ष्ठित्यै॒ प्रति॑ष्ठित्यै॒ साम॑नी॒ साम॑नी॒ प्रति॑ष्ठित्यै । \newline
21. साम॑नी॒ इति॒ साम॑नी । \newline
22. प्रति॑ष्ठित्या अ॒न्तरि॑क्षाया॒ न्तरि॑क्षाय॒ प्रति॑ष्ठित्यै॒ प्रति॑ष्ठित्या अ॒न्तरि॑क्षाय । \newline
23. प्रति॑ष्ठित्या॒ इति॒ प्रति॑ - स्थि॒त्यै॒ । \newline
24. अ॒न्तरि॑क्षा॒य र्.ष॑य॒ ऋष॑यो॒ ऽन्तरि॑क्षाया॒ न्तरि॑क्षा॒य र्.ष॑यः । \newline
25. ऋष॑य स्त्वा॒ त्वर्.ष॑य॒ ऋष॑य स्त्वा । \newline
26. त्वा॒ प्र॒थ॒म॒जाः प्र॑थम॒जा स्त्वा᳚ त्वा प्रथम॒जाः । \newline
27. प्र॒थ॒म॒जा दे॒वेषु॑ दे॒वेषु॑ प्रथम॒जाः प्र॑थम॒जा दे॒वेषु॑ । \newline
28. प्र॒थ॒म॒जा इति॑ प्रथम - जाः । \newline
29. दे॒वेषु॑ दि॒वो दि॒वो दे॒वेषु॑ दे॒वेषु॑ दि॒वः । \newline
30. दि॒वो मात्र॑या॒ मात्र॑या दि॒वो दि॒वो मात्र॑या । \newline
31. मात्र॑या वरि॒णा व॑रि॒णा मात्र॑या॒ मात्र॑या वरि॒णा । \newline
32. व॒रि॒णा प्र॑थन्तु प्रथन् त्वरि॒णा व॑रि॒णा प्र॑थन्तु । \newline
33. प्र॒थ॒न्तु॒ वि॒ध॒र्ता वि॑ध॒र्ता प्र॑थन्तु प्रथन्तु विध॒र्ता । \newline
34. वि॒ध॒र्ता च॑ च विध॒र्ता वि॑ध॒र्ता च॑ । \newline
35. वि॒ध॒र्तेति॑ वि - ध॒र्ता । \newline
36. चा॒य म॒यम् च॑ चा॒यम् । \newline
37. अ॒य मधि॑पति॒ रधि॑पति र॒य म॒य मधि॑पतिः । \newline
38. अधि॑पतिश्च॒ चाधि॑पति॒ रधि॑पतिश्च । \newline
39. अधि॑पति॒रित्यधि॑ - प॒तिः॒ । \newline
40. च॒ ते ते च॑ च॒ ते । \newline
41. ते त्वा᳚ त्वा॒ ते ते त्वा᳚ । \newline
42. त्वा॒ सर्वे॒ सर्वे᳚ त्वा त्वा॒ सर्वे᳚ । \newline
43. सर्वे॑ संॅविदा॒नाः सं॑ॅविदा॒नाः सर्वे॒ सर्वे॑ संॅविदा॒नाः । \newline
44. सं॒ॅवि॒दा॒ना नाक॑स्य॒ नाक॑स्य संॅविदा॒नाः सं॑ॅविदा॒ना नाक॑स्य । \newline
45. सं॒ॅवि॒दा॒ना इति॑ सं - वि॒दा॒नाः । \newline
46. नाक॑स्य पृ॒ष्ठे पृ॒ष्ठे नाक॑स्य॒ नाक॑स्य पृ॒ष्ठे । \newline
47. पृ॒ष्ठे सु॑व॒र्गे सु॑व॒र्गे पृ॒ष्ठे पृ॒ष्ठे सु॑व॒र्गे । \newline
48. सु॒व॒र्गे लो॒के लो॒के सु॑व॒र्गे सु॑व॒र्गे लो॒के । \newline
49. सु॒व॒र्ग इति॑ सुवः - गे । \newline
50. लो॒के यज॑मानं॒ ॅयज॑मानम् ॅलो॒के लो॒के यज॑मानम् । \newline
51. यज॑मानम् च च॒ यज॑मानं॒ ॅयज॑मानम् च । \newline
52. च॒ सा॒द॒य॒न्तु॒ सा॒द॒य॒न्तु॒ च॒ च॒ सा॒द॒य॒न्तु॒ । \newline
53. सा॒द॒य॒न्त्विति॑ सादयन्तु । \newline

\textbf{Ghana Paata } \newline

1. बृह॒स्पति॑र्. हेती॒नाꣳ हे॑ती॒नाम् बृह॒स्पति॒र् बृह॒स्पति॑र्. हेती॒नाम् प्र॑तिध॒र्ता प्र॑तिध॒र्ता हे॑ती॒नाम् बृह॒स्पति॒र् बृह॒स्पति॑र्. हेती॒नाम् प्र॑तिध॒र्ता । \newline
2. हे॒ती॒नाम् प्र॑तिध॒र्ता प्र॑तिध॒र्ता हे॑ती॒नाꣳ हे॑ती॒नाम् प्र॑तिध॒र्ता त्रि॑णवत्रयस्त्रिꣳ॒॒शौ त्रि॑णवत्रयस्त्रिꣳ॒॒शौ प्र॑तिध॒र्ता हे॑ती॒नाꣳ हे॑ती॒नाम् प्र॑तिध॒र्ता त्रि॑णवत्रयस्त्रिꣳ॒॒शौ । \newline
3. प्र॒ति॒ध॒र्ता त्रि॑णवत्रयस्त्रिꣳ॒॒शौ त्रि॑णवत्रयस्त्रिꣳ॒॒शौ प्र॑तिध॒र्ता प्र॑तिध॒र्ता त्रि॑णवत्रयस्त्रिꣳ॒॒शौ त्वा᳚ त्वा त्रिणवत्रयस्त्रिꣳ॒॒शौ प्र॑तिध॒र्ता प्र॑तिध॒र्ता त्रि॑णवत्रयस्त्रिꣳ॒॒शौ त्वा᳚ । \newline
4. प्र॒ति॒ध॒र्तेति॑ प्रति - ध॒र्ता । \newline
5. त्रि॒ण॒व॒त्र॒य॒स्त्रिꣳ॒॒शौ त्वा᳚ त्वा त्रिणवत्रयस्त्रिꣳ॒॒शौ त्रि॑णवत्रयस्त्रिꣳ॒॒शौ त्वा॒ स्तोमौ॒ स्तोमौ᳚ त्वा त्रिणवत्रयस्त्रिꣳ॒॒शौ त्रि॑णवत्रयस्त्रिꣳ॒॒शौ त्वा॒ स्तोमौ᳚ । \newline
6. त्रि॒ण॒व॒त्र॒य॒स्त्रिꣳ॒॒शाविति॑ त्रिणव - त्र॒य॒स्त्रिꣳ॒॒शौ । \newline
7. त्वा॒ स्तोमौ॒ स्तोमौ᳚ त्वा त्वा॒ स्तोमौ॑ पृथि॒व्याम् पृ॑थि॒व्याꣳ स्तोमौ᳚ त्वा त्वा॒ स्तोमौ॑ पृथि॒व्याम् । \newline
8. स्तोमौ॑ पृथि॒व्याम् पृ॑थि॒व्याꣳ स्तोमौ॒ स्तोमौ॑ पृथि॒व्याꣳ श्र॑यताꣳ श्रयताम् पृथि॒व्याꣳ स्तोमौ॒ 
स्तोमौ॑ पृथि॒व्याꣳ श्र॑यताम् । \newline
9. पृ॒थि॒व्याꣳ श्र॑यताꣳ श्रयताम् पृथि॒व्याम् पृ॑थि॒व्याꣳ श्र॑यतां ॅवैश्वदेवाग्निमारु॒ते 
वै᳚श्वदेवाग्निमारु॒ते श्र॑यताम् पृथि॒व्याम् पृ॑थि॒व्याꣳ श्र॑यतां ॅवैश्वदेवाग्निमारु॒ते । \newline
10. श्र॒य॒तां॒ ॅवै॒श्व॒दे॒वा॒ग्नि॒मा॒रु॒ते वै᳚श्वदेवाग्निमारु॒ते श्र॑यताꣳ श्रयतां ॅवैश्वदेवाग्निमारु॒ते उ॒क्थे उ॒क्थे वै᳚श्वदेवाग्निमारु॒ते श्र॑यताꣳ श्रयतां ॅवैश्वदेवाग्निमारु॒ते उ॒क्थे । \newline
11. वै॒श्व॒दे॒वा॒ग्नि॒मा॒रु॒ते उ॒क्थे उ॒क्थे वै᳚श्वदेवाग्निमारु॒ते वै᳚श्वदेवाग्निमारु॒ते उ॒क्थे अव्य॑थयन्ती॒ अव्य॑थयन्ती उ॒क्थे वै᳚श्वदेवाग्निमारु॒ते वै᳚श्वदेवाग्निमारु॒ते उ॒क्थे अव्य॑थयन्ती । \newline
12. वै॒श्व॒दे॒वा॒ग्नि॒मा॒रु॒ते इति॑ वैश्वदेव - आ॒ग्नि॒मा॒रु॒ते । \newline
13. उ॒क्थे अव्य॑थयन्ती॒ अव्य॑थयन्ती उ॒क्थे उ॒क्थे अव्य॑थयन्ती स्तभ्नीताꣳ स्तभ्नीता॒ मव्य॑थयन्ती उ॒क्थे उ॒क्थे अव्य॑थयन्ती स्तभ्नीताम् । \newline
14. उ॒क्थे इत्यु॒क्थे । \newline
15. अव्य॑थयन्ती स्तभ्नीताꣳ स्तभ्नीता॒ मव्य॑थयन्ती॒ अव्य॑थयन्ती स्तभ्नीताꣳ शाक्वररैव॒ते शा᳚क्वररैव॒ते स्त॑भ्नीता॒ मव्य॑थयन्ती॒ अव्य॑थयन्ती स्तभ्नीताꣳ शाक्वररैव॒ते । \newline
16. अव्य॑थयन्ती॒ इत्यव्य॑थयन्ती । \newline
17. स्त॒भ्नी॒ताꣳ॒॒ शा॒क्व॒र॒रै॒व॒ते शा᳚क्वररैव॒ते स्त॑भ्नीताꣳ स्तभ्नीताꣳ शाक्वररैव॒ते साम॑नी॒ साम॑नी शाक्वररैव॒ते स्त॑भ्नीताꣳ स्तभ्नीताꣳ शाक्वररैव॒ते साम॑नी । \newline
18. शा॒क्व॒र॒रै॒व॒ते साम॑नी॒ साम॑नी शाक्वररैव॒ते शा᳚क्वररैव॒ते साम॑नी॒ प्रति॑ष्ठित्यै॒ प्रति॑ष्ठित्यै॒ साम॑नी शाक्वररैव॒ते शा᳚क्वररैव॒ते साम॑नी॒ प्रति॑ष्ठित्यै । \newline
19. शा॒क्व॒र॒रै॒व॒ते इति॑ शाक्वर - रै॒व॒ते । \newline
20. साम॑नी॒ प्रति॑ष्ठित्यै॒ प्रति॑ष्ठित्यै॒ साम॑नी॒ साम॑नी॒ प्रति॑ष्ठित्या अ॒न्तरि॑क्षाया॒ न्तरि॑क्षाय॒ प्रति॑ष्ठित्यै॒ साम॑नी॒ साम॑नी॒ प्रति॑ष्ठित्या अ॒न्तरि॑क्षाय । \newline
21. साम॑नी॒ इति॒ साम॑नी । \newline
22. प्रति॑ष्ठित्या अ॒न्तरि॑क्षाया॒ न्तरि॑क्षाय॒ प्रति॑ष्ठित्यै॒ प्रति॑ष्ठित्या अ॒न्तरि॑क्षा॒य र्.ष॑य॒ ऋष॑यो॒ ऽन्तरि॑क्षाय॒ प्रति॑ष्ठित्यै॒ प्रति॑ष्ठित्या अ॒न्तरि॑क्षा॒य र्.ष॑यः । \newline
23. प्रति॑ष्ठित्या॒ इति॒ प्रति॑ - स्थि॒त्यै॒ । \newline
24. अ॒न्तरि॑क्षा॒य र्.ष॑य॒ ऋष॑यो॒ ऽन्तरि॑क्षाया॒ न्तरि॑क्षा॒य र्.ष॑य स्त्वा॒ त्वर्.ष॑यो॒ ऽन्तरि॑क्षाया॒ न्तरि॑क्षा॒य र्.ष॑यस्त्वा । \newline
25. ऋष॑य स्त्वा॒ त्वर्. ष॑य॒ ऋष॑य स्त्वा प्रथम॒जाः प्र॑थम॒जा स्त्व र्.ष॑य॒ ऋष॑य स्त्वा प्रथम॒जाः । \newline
26. त्वा॒ प्र॒थ॒म॒जाः प्र॑थम॒जा स्त्वा᳚ त्वा प्रथम॒जा दे॒वेषु॑ दे॒वेषु॑ प्रथम॒जा स्त्वा᳚ त्वा प्रथम॒जा दे॒वेषु॑ । \newline
27. प्र॒थ॒म॒जा दे॒वेषु॑ दे॒वेषु॑ प्रथम॒जाः प्र॑थम॒जा दे॒वेषु॑ दि॒वो दि॒वो दे॒वेषु॑ प्रथम॒जाः प्र॑थम॒जा दे॒वेषु॑ दि॒वः । \newline
28. प्र॒थ॒म॒जा इति॑ प्रथम - जाः । \newline
29. दे॒वेषु॑ दि॒वो दि॒वो दे॒वेषु॑ दे॒वेषु॑ दि॒वो मात्र॑या॒ मात्र॑या दि॒वो दे॒वेषु॑ दे॒वेषु॑ दि॒वो मात्र॑या । \newline
30. दि॒वो मात्र॑या॒ मात्र॑या दि॒वो दि॒वो मात्र॑या वरि॒णा व॑रि॒णा मात्र॑या दि॒वो दि॒वो मात्र॑या वरि॒णा । \newline
31. मात्र॑या वरि॒णा व॑रि॒णा मात्र॑या॒ मात्र॑या वरि॒णा प्र॑थन्तु प्रथन्तु वरि॒णा मात्र॑या॒ मात्र॑या वरि॒णा प्र॑थन्तु । \newline
32. व॒रि॒णा प्र॑थन्तु प्रथन्तु वरि॒णा व॑रि॒णा प्र॑थन्तु विध॒र्ता वि॑ध॒र्ता प्र॑थन्तु वरि॒णा व॑रि॒णा 
प्र॑थन्तु विध॒र्ता । \newline
33. प्र॒थ॒न्तु॒ वि॒ध॒र्ता वि॑ध॒र्ता प्र॑थन्तु प्रथन्तु विध॒र्ता च॑ च विध॒र्ता प्र॑थन्तु प्रथन्तु विध॒र्ता च॑ । \newline
34. वि॒ध॒र्ता च॑ च विध॒र्ता वि॑ध॒र्ता चा॒य म॒यम् च॑ विध॒र्ता वि॑ध॒र्ता चा॒यम् । \newline
35. वि॒ध॒र्तेति॑ वि - ध॒र्ता । \newline
36. चा॒य म॒यम् च॑ चा॒य मधि॑पति॒ रधि॑पति र॒यम् च॑ चा॒य मधि॑पतिः । \newline
37. अ॒य मधि॑पति॒ रधि॑पति र॒य म॒य मधि॑पतिश्च॒ चाधि॑पति र॒य म॒य मधि॑पतिश्च । \newline
38. अधि॑पतिश्च॒ चाधि॑पति॒ रधि॑पतिश्च॒ ते ते चाधि॑पति॒ रधि॑पतिश्च॒ ते । \newline
39. अधि॑पति॒रित्यधि॑ - प॒तिः॒ । \newline
40. च॒ ते ते च॑ च॒ ते त्वा᳚ त्वा॒ ते च॑ च॒ ते त्वा᳚ । \newline
41. ते त्वा᳚ त्वा॒ ते ते त्वा॒ सर्वे॒ सर्वे᳚ त्वा॒ ते ते त्वा॒ सर्वे᳚ । \newline
42. त्वा॒ सर्वे॒ सर्वे᳚ त्वा त्वा॒ सर्वे॑ संॅविदा॒नाः सं॑ॅविदा॒नाः सर्वे᳚ त्वा त्वा॒ सर्वे॑ संॅविदा॒नाः । \newline
43. सर्वे॑ संॅविदा॒नाः सं॑ॅविदा॒नाः सर्वे॒ सर्वे॑ संॅविदा॒ना नाक॑स्य॒ नाक॑स्य संॅविदा॒नाः सर्वे॒ सर्वे॑ संॅविदा॒ना नाक॑स्य । \newline
44. सं॒ॅवि॒दा॒ना नाक॑स्य॒ नाक॑स्य संॅविदा॒नाः सं॑ॅविदा॒ना नाक॑स्य पृ॒ष्ठे पृ॒ष्ठे नाक॑स्य संॅविदा॒नाः सं॑ॅविदा॒ना नाक॑स्य पृ॒ष्ठे । \newline
45. सं॒ॅवि॒दा॒ना इति॑ सं - वि॒दा॒नाः । \newline
46. नाक॑स्य पृ॒ष्ठे पृ॒ष्ठे नाक॑स्य॒ नाक॑स्य पृ॒ष्ठे सु॑व॒र्गे सु॑व॒र्गे पृ॒ष्ठे नाक॑स्य॒ नाक॑स्य पृ॒ष्ठे सु॑व॒र्गे । \newline
47. पृ॒ष्ठे सु॑व॒र्गे सु॑व॒र्गे पृ॒ष्ठे पृ॒ष्ठे सु॑व॒र्गे लो॒के लो॒के सु॑व॒र्गे पृ॒ष्ठे पृ॒ष्ठे सु॑व॒र्गे लो॒के । \newline
48. सु॒व॒र्गे लो॒के लो॒के सु॑व॒र्गे सु॑व॒र्गे लो॒के यज॑मानं॒ ॅयज॑मानम् ॅलो॒के सु॑व॒र्गे सु॑व॒र्गे लो॒के यज॑मानम् । \newline
49. सु॒व॒र्ग इति॑ सुवः - गे । \newline
50. लो॒के यज॑मानं॒ ॅयज॑मानम् ॅलो॒के लो॒के यज॑मानम् च च॒ यज॑मानम् ॅलो॒के लो॒के यज॑मानम् च । \newline
51. यज॑मानम् च च॒ यज॑मानं॒ ॅयज॑मानम् च सादयन्तु सादयन्तु च॒ यज॑मानं॒ ॅयज॑मानम् च सादयन्तु । \newline
52. च॒ सा॒द॒य॒न्तु॒ सा॒द॒य॒न्तु॒ च॒ च॒ सा॒द॒य॒न्तु॒ । \newline
53. सा॒द॒य॒न्त्विति॑ सादयन्तु । \newline
\pagebreak
\markright{ TS 4.4.3.1  \hfill https://www.vedavms.in \hfill}

\section{ TS 4.4.3.1 }

\textbf{TS 4.4.3.1 } \newline
\textbf{Samhita Paata} \newline

अ॒यं पु॒रो हरि॑केशः॒ सूर्य॑रश्मि॒स्तस्य॑ रथगृ॒थ्सश्च॒ रथौ॑जाश्च सेनानि ग्राम॒ण्यौ॑ पुञ्जिकस्थ॒ला च॑ कृतस्थ॒ला चा᳚फ्स॒रसौ॑ यातु॒धाना॑ हे॒ती रक्षाꣳ॑सि॒ प्रहे॑ति र॒यं द॑क्षि॒णा वि॒श्व क॑र्मा॒ तस्य॑ रथस्व॒नश्च॒ रथे॑चित्रश्च सेनानि ग्राम॒ण्यौ॑ मेन॒का च॑ सहज॒न्या चा᳚फ्स॒रसौ॑ द॒ङ्णवः॑ प॒शवो॑ हे॒तिः पौरु॑षेयो व॒धः प्रहे॑ति र॒यं प॒श्चाद् वि॒श्वव्य॑चा॒ स्तस्य॒ रथ॑ प्रोत॒श्चा-स॑मरथश्च सेनानि ग्राम॒ण्यौ᳚ प्र॒म्लोच॑न्ती चा -[  ] \newline

\textbf{Pada Paata} \newline

अ॒यम् । पु॒रः । हरि॑केश॒ इति॒ हरि॑ - के॒शः॒ । सूर्य॑रश्मि॒रिति॒ सूर्य॑ - र॒श्मिः॒ । तस्य॑ । र॒थ॒गृ॒थ्स इति॑ रथ - गृ॒थ्सः । च॒ । रथौ॑जा॒ इति॒ रथ॑ - ओ॒जाः॒ । च॒ । से॒ना॒नि॒ग्रा॒म॒ण्या॑विति॑ सेनानि - ग्रा॒म॒ण्यौ᳚ । पु॒ञ्जि॒क॒स्थ॒लेति॑ पुञ्जिक - स्थ॒ला । च॒ । कृ॒त॒स्थ॒लेति॑ कृत - स्थ॒ला । च॒ । अ॒फ्स॒रसौ᳚ । या॒तु॒धाना॒ इति॑ यातु - धानाः᳚ । हे॒तिः । रक्षाꣳ॑सि । प्रहे॑ति॒रिति॒ प्र-हे॒तिः॒ । अ॒यम् । द॒क्षि॒णा । वि॒श्वक॒र्मेति॑ वि॒श्व - क॒र्मा॒ । तस्य॑ । र॒थ॒स्व॒न इति॑ रथ - स्व॒नः । च॒ । रथे॑चित्र॒ इति॒ रथे᳚ - चि॒त्रः॒ । च॒ । से॒ना॒नि॒ग्रा॒म॒ण्या॑विति॑ सेनानि - ग्रा॒म॒ण्यौ᳚ । मे॒न॒का । च॒ । स॒ह॒ज॒न्येति॑ सह - ज॒न्या । च॒ । अ॒फ्स॒रसौ᳚ । द॒ङ्णवः॑ । प॒शवः॑ । हे॒तिः । पौरु॑षेयः । व॒धः । प्रहे॑ति॒रिति॒ प्र - हे॒तिः॒ । अ॒यम् । प॒श्चात् । वि॒श्वव्य॑चा॒ इति॑ वि॒श्व - व्य॒चाः॒ । तस्य॑ । रथ॑प्रोत॒ इति॒ रथ॑-प्रो॒तः॒ । च॒ । अस॑मरथ॒ इत्यस॑म - र॒थः॒ । च॒ । से॒ना॒नि॒ग्रा॒म॒ण्या॑विति॑ सेनानि - ग्रा॒म॒ण्यौ᳚ । प्र॒म्लोच॒न्तीति॑ प्र - म्लोच॑न्ती । च॒ ।  \newline


\textbf{Krama Paata} \newline

अ॒यम् पु॒रः । पु॒रो हरि॑केशः । हरि॑केशः॒ सूर्य॑रश्मिः । हरि॑केश॒ इति॒ हरि॑ - के॒शः॒ । सूर्य॑रश्मि॒स्तस्य॑ । सूर्य॑रश्मि॒रिति॒ सूर्य॑ - र॒श्मिः॒ । तस्य॑ रथगृ॒थ्सः । र॒थ॒गृ॒थ्सश्च॑ । र॒थ॒गृ॒थ्स इति॑ रथ - गृ॒थ्सः । च॒ रथौ॑जाः । रथौ॑जाश्च । रथौ॑जा॒ इति॒ रथ॑ - ओ॒जाः॒ । च॒ से॒ना॒नि॒ग्रा॒म॒ण्यौ᳚ । से॒ना॒नि॒गा॒म॒ण्यौ॑ पुञ्जिकस्थ॒ला । से॒ना॒नि॒ग्रा॒म॒ण्या॑विति॑ सेनानि - ग्रा॒म॒ण्यौ᳚ । पु॒ञ्जि॒क॒स्थ॒ला च॑ । पु॒ञ्जि॒क॒स्थ॒लेति॑ पुञ्जिक - स्थ॒ला । च॒ कृ॒त॒स्थ॒ला । कृ॒त॒स्थ॒ला च॑ । कृ॒त॒स्थ॒लेति॑ कृत - स्थ॒ला । चा॒फ्स॒रसौ᳚ । अ॒फ्स॒रसौ॑ यातु॒धानाः᳚ । या॒तु॒धाना॑ हे॒तिः । या॒तु॒धाना॒ इति॑ यातु - धानाः᳚ । हे॒ती रक्षाꣳ॑सि । रक्षाꣳ॑सि॒ प्रहे॑तिः । प्रहे॑तिर॒यम् । प्रहे॑ति॒रिति॒ प्र - हे॒तिः॒ । अ॒यम् द॑क्षि॒णा । द॒क्षि॒णा वि॒श्वक॑र्मा । वि॒श्वक॑र्मा॒ तस्य॑ । वि॒श्वक॒र्मेति॑ वि॒श्व - क॒र्मा॒ । तस्य॑ रथस्व॒नः । र॒थ॒स्व॒नश्च॑ । र॒थ॒स्व॒न इति॑ रथ - स्व॒नः । च॒ रथे॑चित्रः । रथे॑चित्रश्च । रथे॑चित्र॒ इति॒ रथे᳚ - चि॒त्रः॒ । च॒ से॒ना॒नि॒ग्रा॒म॒ण्यौ᳚ । से॒ना॒नि॒ग्रा॒म॒ण्यौ॑ मेन॒का । से॒ना॒नि॒ग्रा॒म॒ण्या॑विति॑ सेनानि - ग्रा॒म॒ण्यौ᳚ । मे॒न॒का च॑ । च॒ स॒ह॒ज॒न्या । स॒ह॒ज॒न्या च॑ । स॒ह॒ज॒न्येति॑ सह - ज॒न्या । चा॒फ्स॒रसौ᳚ । अ॒फ्स॒रसौ॑ द॒ङ्क्ष्णवः॑ । द॒ङ्क्ष्णवः॑ प॒शवः॑ । प॒शवो॑ हे॒तिः । हे॒तिः पौरु॑षेयः । पौरु॑षेयो व॒धः । व॒धः प्रहे॑तिः । प्रहे॑तिर॒यम् । प्रहे॑ति॒रिति॒ प्र - हे॒तिः॒ । अ॒यम् प॒श्चात् । प॒श्चाद् वि॒श्वव्य॑चाः । वि॒श्वव्य॑चा॒स्तस्य॑ । वि॒श्वव्य॑चा॒ इति॑ वि॒श्व - व्य॒चाः॒ । तस्य॒ रथ॑प्रोतः । रथ॑प्रोतश्च । रथ॑प्रोत॒ इति॒ रथ॑ - प्रो॒तः॒ । चास॑मरथः । अस॑मरथश्च । अस॑मरथ॒ इत्यस॑म - र॒थः॒ । च॒ से॒ना॒नि॒ग्रा॒म॒ण्यौ᳚ । से॒ना॒नि॒ग्रा॒म॒ण्यौ᳚ प्र॒म्लोच॑न्ती । से॒ना॒नि॒ग्रा॒म॒ण्या॑विति॑ सेनानि - ग्रा॒म॒ण्यौ᳚ । प्र॒म्लोच॑न्ती च । प्र॒म्लोच॒न्तीति॑ प्र - म्लोच॑न्ती । चा॒नु॒म्लोच॑न्ती \newline

\textbf{Jatai Paata} \newline

1. अ॒यम् पु॒रः पु॒रो॑ ऽय म॒यम् पु॒रः । \newline
2. पु॒रो हरि॑केशो॒ हरि॑केशः पु॒रः पु॒रो हरि॑केशः । \newline
3. हरि॑केशः॒ सूर्य॑रश्मिः॒ सूर्य॑रश्मि॒र्॒. हरि॑केशो॒ हरि॑केशः॒ सूर्य॑रश्मिः । \newline
4. हरि॑केश॒ इति॒ हरि॑ - के॒शः॒ । \newline
5. सूर्य॑रश्मि॒ स्तस्य॒ तस्य॒ सूर्य॑रश्मिः॒ सूर्य॑रश्मि॒ स्तस्य॑ । \newline
6. सूर्य॑रश्मि॒रिति॒ सूर्य॑ - र॒श्मिः॒ । \newline
7. तस्य॑ रथगृ॒थ्सो र॑थगृ॒थ्स स्तस्य॒ तस्य॑ रथगृ॒थ्सः । \newline
8. र॒थ॒गृ॒थ्सश्च॑ च रथगृ॒थ्सो र॑थगृ॒थ्सश्च॑ । \newline
9. र॒थ॒गृ॒थ्स इति॑ रथ - गृ॒थ्सः । \newline
10. च॒ रथौ॑जा॒ रथौ॑जाश्च च॒ रथौ॑जाः । \newline
11. रथौ॑जाश्च च॒ रथौ॑जा॒ रथौ॑जाश्च । \newline
12. रथौ॑जा॒ इति॒ रथ॑ - ओ॒जाः॒ । \newline
13. च॒ से॒ना॒नि॒ग्रा॒म॒ण्यौ॑ सेनानिग्राम॒ण्यौ॑ च च सेनानिग्राम॒ण्यौ᳚ । \newline
14. से॒ना॒नि॒ग्रा॒म॒ण्यौ॑ पुञ्जिकस्थ॒ला पु॑ञ्जिकस्थ॒ला से॑नानिग्राम॒ण्यौ॑ सेनानिग्राम॒ण्यौ॑ पुञ्जिकस्थ॒ला । \newline
15. से॒ना॒नि॒ग्रा॒म॒ण्या॑विति॑ सेनानि - ग्रा॒म॒ण्यौ᳚ । \newline
16. पु॒ञ्जि॒क॒स्थ॒ला च॑ च पुञ्जिकस्थ॒ला पु॑ञ्जिकस्थ॒ला च॑ । \newline
17. पु॒ञ्जि॒क॒स्थ॒लेति॑ पुञ्जिक - स्थ॒ला । \newline
18. च॒ कृ॒त॒स्थ॒ला कृ॑तस्थ॒ला च॑ च कृतस्थ॒ला । \newline
19. कृ॒त॒स्थ॒ला च॑ च कृतस्थ॒ला कृ॑तस्थ॒ला च॑ । \newline
20. कृ॒त॒स्थ॒लेति॑ कृत - स्थ॒ला । \newline
21. चा॒फ्स॒रसा॑ वफ्स॒रसौ॑ च चाफ्स॒रसौ᳚ । \newline
22. अ॒फ्स॒रसौ॑ यातु॒धाना॑ यातु॒धाना॑ अफ्स॒रसा॑ वफ्स॒रसौ॑ यातु॒धानाः᳚ । \newline
23. या॒तु॒धाना॑ हे॒तिर्. हे॒तिर् या॑तु॒धाना॑ यातु॒धाना॑ हे॒तिः । \newline
24. या॒तु॒धाना॒ इति॑ यातु - धानाः᳚ । \newline
25. हे॒ती रक्षाꣳ॑सि॒ रक्षाꣳ॑सि हे॒तिर्. हे॒ती रक्षाꣳ॑सि । \newline
26. रक्षाꣳ॑सि॒ प्रहे॑तिः॒ प्रहे॑ती॒ रक्षाꣳ॑सि॒ रक्षाꣳ॑सि॒ प्रहे॑तिः । \newline
27. प्रहे॑ति र॒य म॒यम् प्रहे॑तिः॒ प्रहे॑ति र॒यम् । \newline
28. प्रहे॑ति॒रिति॒ प्र - हे॒तिः॒ । \newline
29. अ॒यम् द॑क्षि॒णा द॑क्षि॒णा ऽय म॒यम् द॑क्षि॒णा । \newline
30. द॒क्षि॒णा वि॒श्वक॑र्मा वि॒श्वक॑र्मा दक्षि॒णा द॑क्षि॒णा वि॒श्वक॑र्मा । \newline
31. वि॒श्वक॑र्मा॒ तस्य॒ तस्य॑ वि॒श्वक॑र्मा वि॒श्वक॑र्मा॒ तस्य॑ । \newline
32. वि॒श्वक॒र्मेति॑ वि॒श्व - क॒र्मा॒ । \newline
33. तस्य॑ रथस्व॒नो र॑थस्व॒न स्तस्य॒ तस्य॑ रथस्व॒नः । \newline
34. र॒थ॒स्व॒नश्च॑ च रथस्व॒नो र॑थस्व॒नश्च॑ । \newline
35. र॒थ॒स्व॒न इति॑ रथ - स्व॒नः । \newline
36. च॒ रथे॑चित्रो॒ रथे॑चित्रश्च च॒ रथे॑चित्रः । \newline
37. रथे॑चित्रश्च च॒ रथे॑चित्रो॒ रथे॑चित्रश्च । \newline
38. रथे॑चित्र॒ इति॒ रथे᳚ - चि॒त्रः॒ । \newline
39. च॒ से॒ना॒नि॒ग्रा॒म॒ण्यौ॑ सेनानिग्राम॒ण्यौ॑ च च सेनानिग्राम॒ण्यौ᳚ । \newline
40. से॒ना॒नि॒ग्रा॒म॒ण्यौ॑ मेन॒का मे॑न॒का से॑नानिग्राम॒ण्यौ॑ सेनानिग्राम॒ण्यौ॑ मेन॒का । \newline
41. से॒ना॒नि॒ग्रा॒म॒ण्या॑विति॑ सेनानि - ग्रा॒म॒ण्यौ᳚ । \newline
42. मे॒न॒का च॑ च मेन॒का मे॑न॒का च॑ । \newline
43. च॒ स॒ह॒ज॒न्या स॑हज॒न्या च॑ च सहज॒न्या । \newline
44. स॒ह॒ज॒न्या च॑ च सहज॒न्या स॑हज॒न्या च॑ । \newline
45. स॒ह॒ज॒न्येति॑ सह - ज॒न्या । \newline
46. चा॒फ्स॒रसा॑ वफ्स॒रसौ॑ च चाफ्स॒रसौ᳚ । \newline
47. अ॒फ्स॒रसौ॑ द॒ङ्क्ष्णवो॑ द॒ङ्क्ष्णवो᳚ ऽफ्स॒रसा॑ वफ्स॒रसौ॑ द॒ङ्क्ष्णवः॑ । \newline
48. द॒ङ्क्ष्णवः॑ प॒शवः॑ प॒शवो॑ द॒ङ्क्ष्णवो॑ द॒ङ्क्ष्णवः॑ प॒शवः॑ । \newline
49. प॒शवो॑ हे॒तिर्. हे॒तिः प॒शवः॑ प॒शवो॑ हे॒तिः । \newline
50. हे॒तिः पौरु॑षेयः॒ पौरु॑षेयो हे॒तिर्. हे॒तिः पौरु॑षेयः । \newline
51. पौरु॑षेयो व॒धो व॒धः पौरु॑षेयः॒ पौरु॑षेयो व॒धः । \newline
52. व॒धः प्रहे॑तिः॒ प्रहे॑तिर् व॒धो व॒धः प्रहे॑तिः । \newline
53. प्रहे॑ति र॒य म॒यम् प्रहे॑तिः॒ प्रहे॑ति र॒यम् । \newline
54. प्रहे॑ति॒रिति॒ प्र - हे॒तिः॒ । \newline
55. अ॒यम् प॒श्चात् प॒श्चा द॒य म॒यम् प॒श्चात् । \newline
56. प॒श्चाद् वि॒श्वव्य॑चा वि॒श्वव्य॑चाः प॒श्चात् प॒श्चाद् वि॒श्वव्य॑चाः । \newline
57. वि॒श्वव्य॑चा॒ स्तस्य॒ तस्य॑ वि॒श्वव्य॑चा वि॒श्वव्य॑चा॒ स्तस्य॑ । \newline
58. वि॒श्वव्य॑चा॒ इति॑ वि॒श्व - व्य॒चाः॒ । \newline
59. तस्य॒ रथ॑प्रोतो॒ रथ॑प्रोत॒ स्तस्य॒ तस्य॒ रथ॑प्रोतः । \newline
60. रथ॑प्रोतश्च च॒ रथ॑प्रोतो॒ रथ॑प्रोतश्च । \newline
61. रथ॑प्रोत॒ इति॒ रथ॑ - प्रो॒तः॒ । \newline
62. चास॑मर॒थो ऽस॑मरथश्च॒ चास॑मरथः । \newline
63. अस॑मरथश्च॒ चास॑मर॒थो ऽस॑मरथश्च । \newline
64. अस॑मरथ॒ इत्यस॑म - र॒थः॒ । \newline
65. च॒ से॒ना॒नि॒ग्रा॒म॒ण्यौ॑ सेनानिग्राम॒ण्यौ॑ च च सेनानिग्राम॒ण्यौ᳚ । \newline
66. से॒ना॒नि॒ग्रा॒म॒ण्यौ᳚ प्र॒म्लोच॑न्ती प्र॒म्लोच॑न्ती सेनानिग्राम॒ण्यौ॑ सेनानिग्राम॒ण्यौ᳚ प्र॒म्लोच॑न्ती । \newline
67. से॒ना॒नि॒ग्रा॒म॒ण्या॑विति॑ सेनानि - ग्रा॒म॒ण्यौ᳚ । \newline
68. प्र॒म्लोच॑न्ती च च प्र॒म्लोच॑न्ती प्र॒म्लोच॑न्ती च । \newline
69. प्र॒म्लोच॒न्तीति॑ प्र - म्लोच॑न्ती । \newline
70. चा॒नु॒म्लोच॑न्त्य नु॒म्लोच॑न्ती च चानु॒म्लोच॑न्ती । \newline

\textbf{Ghana Paata } \newline

1. अ॒यम् पु॒रः पु॒रो॑ ऽय म॒यम् पु॒रो हरि॑केशो॒ हरि॑केशः पु॒रो॑ ऽयम॒यम् पु॒रो हरि॑केशः । \newline
2. पु॒रो हरि॑केशो॒ हरि॑केशः पु॒रः पु॒रो हरि॑केशः॒ सूर्य॑रश्मिः॒ सूर्य॑रश्मि॒र्॒. हरि॑केशः पु॒रः पु॒रो हरि॑केशः॒ सूर्य॑रश्मिः । \newline
3. हरि॑केशः॒ सूर्य॑रश्मिः॒ सूर्य॑रश्मि॒र्॒. हरि॑केशो॒ हरि॑केशः॒ सूर्य॑रश्मि॒ स्तस्य॒ तस्य॒ सूर्य॑रश्मि॒र्॒. हरि॑केशो॒ हरि॑केशः॒ सूर्य॑रश्मि॒ स्तस्य॑ । \newline
4. हरि॑केश॒ इति॒ हरि॑ - के॒शः॒ । \newline
5. सूर्य॑रश्मि॒ स्तस्य॒ तस्य॒ सूर्य॑रश्मिः॒ सूर्य॑रश्मि॒ स्तस्य॑ रथगृ॒थ्सो र॑थगृ॒थ्स स्तस्य॒ सूर्य॑रश्मिः॒ सूर्य॑रश्मि॒ स्तस्य॑ रथगृ॒थ्सः । \newline
6. सूर्य॑रश्मि॒रिति॒ सूर्य॑ - र॒श्मिः॒ । \newline
7. तस्य॑ रथगृ॒थ्सो र॑थगृ॒थ्स स्तस्य॒ तस्य॑ रथगृ॒थ्सश्च॑ च रथगृ॒थ्स स्तस्य॒ तस्य॑ रथगृ॒थ्सश्च॑ । \newline
8. र॒थ॒गृ॒थ्सश्च॑ च रथगृ॒थ्सो र॑थगृ॒थ्सश्च॒ रथौ॑जा॒ रथौ॑जाश्च रथगृ॒थ्सो र॑थगृ॒थ्सश्च॒ रथौ॑जाः । \newline
9. र॒थ॒गृ॒थ्स इति॑ रथ - गृ॒थ्सः । \newline
10. च॒ रथौ॑जा॒ रथौ॑जाश्च च॒ रथौ॑जाश्च च॒ रथौ॑जाश्च च॒ रथौ॑जाश्च । \newline
11. रथौ॑जाश्च च॒ रथौ॑जा॒ रथौ॑जाश्च सेनानिग्राम॒ण्यौ॑ सेनानिग्राम॒ण्यौ॑ च॒ रथौ॑जा॒ रथौ॑जाश्च सेनानिग्राम॒ण्यौ᳚ । \newline
12. रथौ॑जा॒ इति॒ रथ॑ - ओ॒जाः॒ । \newline
13. च॒ से॒ना॒नि॒ग्रा॒म॒ण्यौ॑ सेनानिग्राम॒ण्यौ॑ च च सेनानिग्राम॒ण्यौ॑ पुञ्जिकस्थ॒ला पु॑ञ्जिकस्थ॒ला से॑नानिग्राम॒ण्यौ॑ च च सेनानिग्राम॒ण्यौ॑ पुञ्जिकस्थ॒ला । \newline
14. से॒ना॒नि॒ग्रा॒म॒ण्यौ॑ पुञ्जिकस्थ॒ला पु॑ञ्जिकस्थ॒ला से॑नानिग्राम॒ण्यौ॑ सेनानिग्राम॒ण्यौ॑ पुञ्जिकस्थ॒ला च॑ च पुञ्जिकस्थ॒ला से॑नानिग्राम॒ण्यौ॑ सेनानिग्राम॒ण्यौ॑ पुञ्जिकस्थ॒ला च॑ । \newline
15. से॒ना॒नि॒ग्रा॒म॒ण्या॑विति॑ सेनानि - ग्रा॒म॒ण्यौ᳚ । \newline
16. पु॒ञ्जि॒क॒स्थ॒ला च॑ च पुञ्जिकस्थ॒ला पु॑ञ्जिकस्थ॒ला च॑ कृतस्थ॒ला कृ॑तस्थ॒ला च॑ पुञ्जिकस्थ॒ला पु॑ञ्जिकस्थ॒ला च॑ कृतस्थ॒ला । \newline
17. पु॒ञ्जि॒क॒स्थ॒लेति॑ पुञ्जिक - स्थ॒ला । \newline
18. च॒ कृ॒त॒स्थ॒ला कृ॑तस्थ॒ला च॑ च कृतस्थ॒ला च॑ च कृतस्थ॒ला च॑ च कृतस्थ॒ला च॑ । \newline
19. कृ॒त॒स्थ॒ला च॑ च कृतस्थ॒ला कृ॑तस्थ॒ला चा᳚फ्स॒रसा॑ वफ्स॒रसौ॑ च कृतस्थ॒ला कृ॑तस्थ॒ला चा᳚फ्स॒रसौ᳚ । \newline
20. कृ॒त॒स्थ॒लेति॑ कृत - स्थ॒ला । \newline
21. चा॒फ्स॒रसा॑ वफ्स॒रसौ॑ च चाफ्स॒रसौ॑ यातु॒धाना॑ यातु॒धाना॑ अफ्स॒रसौ॑ च चाफ्स॒रसौ॑ यातु॒धानाः᳚ । \newline
22. अ॒फ्स॒रसौ॑ यातु॒धाना॑ यातु॒धाना॑ अफ्स॒रसा॑ वफ्स॒रसौ॑ यातु॒धाना॑ हे॒तिर्. हे॒तिर् या॑तु॒धाना॑ अफ्स॒रसा॑ वफ्स॒रसौ॑ यातु॒धाना॑ हे॒तिः । \newline
23. या॒तु॒धाना॑ हे॒तिर्. हे॒तिर् या॑तु॒धाना॑ यातु॒धाना॑ हे॒ती रक्षाꣳ॑सि॒ रक्षाꣳ॑सि हे॒तिर् या॑तु॒धाना॑ यातु॒धाना॑ हे॒ती रक्षाꣳ॑सि । \newline
24. या॒तु॒धाना॒ इति॑ यातु - धानाः᳚ । \newline
25. हे॒ती रक्षाꣳ॑सि॒ रक्षाꣳ॑सि हे॒तिर्. हे॒ती रक्षाꣳ॑सि॒ प्रहे॑तिः॒ प्रहे॑ती॒ रक्षाꣳ॑सि हे॒तिर्. हे॒ती रक्षाꣳ॑सि॒ प्रहे॑तिः । \newline
26. रक्षाꣳ॑सि॒ प्रहे॑तिः॒ प्रहे॑ती॒ रक्षाꣳ॑सि॒ रक्षाꣳ॑सि॒ प्रहे॑ति र॒य म॒यम् प्रहे॑ती॒ रक्षाꣳ॑सि॒ रक्षाꣳ॑सि॒ प्रहे॑ति र॒यम् । \newline
27. प्रहे॑ति र॒य म॒यम् प्रहे॑तिः॒ प्रहे॑ति र॒यम् द॑क्षि॒णा द॑क्षि॒णा ऽयम् प्रहे॑तिः॒ प्रहे॑ति र॒यम् द॑क्षि॒णा । \newline
28. प्रहे॑ति॒रिति॒ प्र - हे॒तिः॒ । \newline
29. अ॒यम् द॑क्षि॒णा द॑क्षि॒णा ऽय म॒यम् द॑क्षि॒णा वि॒श्वक॑र्मा वि॒श्वक॑र्मा दक्षि॒णा ऽय म॒यम् द॑क्षि॒णा वि॒श्वक॑र्मा । \newline
30. द॒क्षि॒णा वि॒श्वक॑र्मा वि॒श्वक॑र्मा दक्षि॒णा द॑क्षि॒णा वि॒श्वक॑र्मा॒ तस्य॒ तस्य॑ वि॒श्वक॑र्मा दक्षि॒णा द॑क्षि॒णा वि॒श्वक॑र्मा॒ तस्य॑ । \newline
31. वि॒श्वक॑र्मा॒ तस्य॒ तस्य॑ वि॒श्वक॑र्मा वि॒श्वक॑र्मा॒ तस्य॑ रथस्व॒नो र॑थस्व॒न स्तस्य॑ वि॒श्वक॑र्मा वि॒श्वक॑र्मा॒ तस्य॑ रथस्व॒नः । \newline
32. वि॒श्वक॒र्मेति॑ वि॒श्व - क॒र्मा॒ । \newline
33. तस्य॑ रथस्व॒नो र॑थस्व॒न स्तस्य॒ तस्य॑ रथस्व॒नश्च॑ च रथस्व॒न स्तस्य॒ तस्य॑ रथस्व॒नश्च॑ । \newline
34. र॒थ॒स्व॒नश्च॑ च रथस्व॒नो र॑थस्व॒नश्च॒ रथे॑चित्रो॒ रथे॑चित्रश्च रथस्व॒नो र॑थस्व॒नश्च॒ रथे॑चित्रः । \newline
35. र॒थ॒स्व॒न इति॑ रथ - स्व॒नः । \newline
36. च॒ रथे॑चित्रो॒ रथे॑चित्रश्च च॒ रथे॑चित्रश्च च॒ रथे॑चित्रश्च च॒ रथे॑चित्रश्च । \newline
37. रथे॑चित्रश्च च॒ रथे॑चित्रो॒ रथे॑चित्रश्च सेनानिग्राम॒ण्यौ॑ सेनानिग्राम॒ण्यौ॑ च॒ रथे॑चित्रो॒ रथे॑चित्रश्च सेनानिग्राम॒ण्यौ᳚ । \newline
38. रथे॑चित्र॒ इति॒ रथे᳚ - चि॒त्रः॒ । \newline
39. च॒ से॒ना॒नि॒ग्रा॒म॒ण्यौ॑ सेनानिग्राम॒ण्यौ॑ च च सेनानिग्राम॒ण्यौ॑ मेन॒का मे॑न॒का से॑नानिग्राम॒ण्यौ॑ च च सेनानिग्राम॒ण्यौ॑ मेन॒का । \newline
40. से॒ना॒नि॒ग्रा॒म॒ण्यौ॑ मेन॒का मे॑न॒का से॑नानिग्राम॒ण्यौ॑ सेनानिग्राम॒ण्यौ॑ मेन॒का च॑ च मेन॒का से॑नानिग्राम॒ण्यौ॑ सेनानिग्राम॒ण्यौ॑ मेन॒का च॑ । \newline
41. से॒ना॒नि॒ग्रा॒म॒ण्या॑विति॑ सेनानि - ग्रा॒म॒ण्यौ᳚ । \newline
42. मे॒न॒का च॑ च मेन॒का मे॑न॒का च॑ सहज॒न्या स॑हज॒न्या च॑ मेन॒का मे॑न॒का च॑ सहज॒न्या । \newline
43. च॒ स॒ह॒ज॒न्या स॑हज॒न्या च॑ च सहज॒न्या च॑ च सहज॒न्या च॑ च सहज॒न्या च॑ । \newline
44. स॒ह॒ज॒न्या च॑ च सहज॒न्या स॑हज॒न्या चा᳚फ्स॒रसा॑ वफ्स॒रसौ॑ च सहज॒न्या स॑हज॒न्या चा᳚फ्स॒रसौ᳚ । \newline
45. स॒ह॒ज॒न्येति॑ सह - ज॒न्या । \newline
46. चा॒फ्स॒रसा॑ वफ्स॒रसौ॑ च चाफ्स॒रसौ॑ द॒ङ्क्ष्णवो॑ द॒ङ्क्ष्णवो᳚ ऽफ्स॒रसौ॑ च चाफ्स॒रसौ॑ द॒ङ्क्ष्णवः॑ । \newline
47. अ॒फ्स॒रसौ॑ द॒ङ्क्ष्णवो॑ द॒ङ्क्ष्णवो᳚ ऽफ्स॒रसा॑ वफ्स॒रसौ॑ द॒ङ्क्ष्णवः॑ प॒शवः॑ प॒शवो॑ द॒ङ्क्ष्णवो᳚ ऽफ्स॒रसा॑ वफ्स॒रसौ॑ द॒ङ्क्ष्णवः॑ प॒शवः॑ । \newline
48. द॒ङ्क्ष्णवः॑ प॒शवः॑ प॒शवो॑ द॒ङ्क्ष्णवो॑ द॒ङ्क्ष्णवः॑ प॒शवो॑ हे॒तिर्. हे॒तिः प॒शवो॑ द॒ङ्क्ष्णवो॑ द॒ङ्क्ष्णवः॑ प॒शवो॑ हे॒तिः । \newline
49. प॒शवो॑ हे॒तिर्. हे॒तिः प॒शवः॑ प॒शवो॑ हे॒तिः पौरु॑षेयः॒ पौरु॑षेयो हे॒तिः प॒शवः॑ प॒शवो॑ हे॒तिः पौरु॑षेयः । \newline
50. हे॒तिः पौरु॑षेयः॒ पौरु॑षेयो हे॒तिर्. हे॒तिः पौरु॑षेयो व॒धो व॒धः पौरु॑षेयो हे॒तिर्. हे॒तिः पौरु॑षेयो व॒धः । \newline
51. पौरु॑षेयो व॒धो व॒धः पौरु॑षेयः॒ पौरु॑षेयो व॒धः प्रहे॑तिः॒ प्रहे॑तिर् व॒धः पौरु॑षेयः॒ पौरु॑षेयो व॒धः प्रहे॑तिः । \newline
52. व॒धः प्रहे॑तिः॒ प्रहे॑तिर् व॒धो व॒धः प्रहे॑ति र॒य म॒यम् प्रहे॑तिर् व॒धो व॒धः प्रहे॑ति र॒यम् । \newline
53. प्रहे॑ति र॒य म॒यम् प्रहे॑तिः॒ प्रहे॑ति र॒यम् प॒श्चात् प॒श्चा द॒यम् प्रहे॑तिः॒ प्रहे॑ति र॒यम् प॒श्चात् । \newline
54. प्रहे॑ति॒रिति॒ प्र - हे॒तिः॒ । \newline
55. अ॒यम् प॒श्चात् प॒श्चा द॒य म॒यम् प॒श्चाद् वि॒श्वव्य॑चा वि॒श्वव्य॑चाः प॒श्चा द॒य म॒यम् प॒श्चाद् वि॒श्वव्य॑चाः । \newline
56. प॒श्चाद् वि॒श्वव्य॑चा वि॒श्वव्य॑चाः प॒श्चात् प॒श्चाद् वि॒श्वव्य॑चा॒ स्तस्य॒ तस्य॑ वि॒श्वव्य॑चाः प॒श्चात् प॒श्चाद् वि॒श्वव्य॑चा॒ स्तस्य॑ । \newline
57. वि॒श्वव्य॑चा॒ स्तस्य॒ तस्य॑ वि॒श्वव्य॑चा वि॒श्वव्य॑चा॒ स्तस्य॒ रथ॑प्रोतो॒ रथ॑प्रोत॒ स्तस्य॑ वि॒श्वव्य॑चा वि॒श्वव्य॑चा॒ स्तस्य॒ रथ॑प्रोतः । \newline
58. वि॒श्वव्य॑चा॒ इति॑ वि॒श्व - व्य॒चाः॒ । \newline
59. तस्य॒ रथ॑प्रोतो॒ रथ॑प्रोत॒ स्तस्य॒ तस्य॒ रथ॑प्रोतश्च च॒ रथ॑प्रोत॒ स्तस्य॒ तस्य॒ रथ॑प्रोतश्च । \newline
60. रथ॑प्रोतश्च च॒ रथ॑प्रोतो॒ रथ॑प्रोत॒ श्चास॑मर॒थो ऽस॑मरथश्च॒ रथ॑प्रोतो॒ रथ॑प्रोत॒ श्चास॑मरथः । \newline
61. रथ॑प्रोत॒ इति॒ रथ॑ - प्रो॒तः॒ । \newline
62. चास॑मर॒थो ऽस॑मरथश्च॒ चास॑मरथश्च॒ चास॑मरथश्च॒ चास॑मरथश्च । \newline
63. अस॑मरथश्च॒ चास॑मर॒थो ऽस॑मरथश्च सेनानिग्राम॒ण्यौ॑ सेनानिग्राम॒ण्यौ॑ चास॑मर॒थो ऽस॑मरथश्च सेनानिग्राम॒ण्यौ᳚ । \newline
64. अस॑मरथ॒ इत्यस॑म - र॒थः॒ । \newline
65. च॒ से॒ना॒नि॒ग्रा॒म॒ण्यौ॑ सेनानिग्राम॒ण्यौ॑ च च सेनानिग्राम॒ण्यौ᳚ प्र॒म्लोच॑न्ती प्र॒म्लोच॑न्ती सेनानिग्राम॒ण्यौ॑ च च सेनानिग्राम॒ण्यौ᳚ प्र॒म्लोच॑न्ती । \newline
66. से॒ना॒नि॒ग्रा॒म॒ण्यौ᳚ प्र॒म्लोच॑न्ती प्र॒म्लोच॑न्ती सेनानिग्राम॒ण्यौ॑ सेनानिग्राम॒ण्यौ᳚ प्र॒म्लोच॑न्ती च च प्र॒म्लोच॑न्ती सेनानिग्राम॒ण्यौ॑ सेनानिग्राम॒ण्यौ᳚ प्र॒म्लोच॑न्ती च । \newline
67. से॒ना॒नि॒ग्रा॒म॒ण्या॑विति॑ सेनानि - ग्रा॒म॒ण्यौ᳚ । \newline
68. प्र॒म्लोच॑न्ती च च प्र॒म्लोच॑न्ती प्र॒म्लोच॑न्ती चानु॒म्लोच॑न् त्यनु॒म्लोच॑न्ती च प्र॒म्लोच॑न्ती प्र॒म्लोच॑न्ती चानु॒म्लोच॑न्ती । \newline
69. प्र॒म्लोच॒न्तीति॑ प्र - म्लोच॑न्ती । \newline
70. चा॒नु॒म्लोच॑न् त्यनु॒म्लोच॑न्ती च चानु॒म्लोच॑न्ती च चानु॒म्लोच॑न्ती च चानु॒म्लोच॑न्ती च । \newline
\pagebreak
\markright{ TS 4.4.3.2  \hfill https://www.vedavms.in \hfill}

\section{ TS 4.4.3.2 }

\textbf{TS 4.4.3.2 } \newline
\textbf{Samhita Paata} \newline

नु॒म्लोच॑न्ती-चाफ्स॒रसौ॑ स॒र्पा हे॒ति र्व्या॒घ्राः प्रहे॑ति र॒य मु॑त्त॒राथ् सं॒यॅद्- व॑सु॒स्तस्य॑ सेन॒जिच्च॑ सु॒षेण॑श्च सेनानि ग्राम॒ण्यौ॑ वि॒श्वाची॑ च घृ॒ताची॑ चाफ्स॒रसा॒ वापो॑ हे॒ति र्वातः॒ प्रहे॑ति र॒यमु॒पर्य॒ र्वाग्व॑-सु॒स्तस्य॒ तार्क्ष्य॒-श्चारि॑ष्ट-नेमिश्च सेनानि ग्राम॒ण्या॑ वु॒र्वशी॑ च पू॒र्वचि॑त्तिश्चा-फ्स॒रसौ॑ वि॒द्युद्धे॒तिर॑-व॒स्फूर्ज॒न् प्रहे॑ति॒ स्तेभ्यो॒ नम॒स्ते नो॑ मृडयन्तु॒ ते यं - [  ] \newline

\textbf{Pada Paata} \newline

अ॒नु॒म्लोच॒न्तीत्य॑नु - म्लोच॑न्ती । च॒ । अ॒फ्स॒रसौ᳚ । स॒र्पाः । हे॒तिः । व्या॒घ्राः । प्रहे॑ति॒रिति॒ प्र - हे॒तिः॒ । अ॒यम् । उ॒त्त॒रादित्यु॑त्-त॒रात् । सं॒ॅयद्व॑सु॒रिति॑ सं॒ॅयत् - व॒सुः॒ । तस्य॑ । से॒न॒जिदिति॑ सेन-जित् । च॒ । सु॒षेण॒ इति॑ सु - सेनः॑ । च॒ । से॒ना॒नि॒ग्रा॒म॒ण्या॑विति॑ सेनानि - ग्रा॒म॒ण्यौ᳚ । वि॒श्वाची᳚ । च॒ । घृ॒ताची᳚ । च॒ । अ॒फ्स॒रसौ᳚ । आपः॑ । हे॒ति । वातः॑ । प्रहे॑ति॒रिति॒ प्र - हे॒तिः॒ । अ॒यम् । उ॒परि॑ । अ॒र्वाग्व॑सु॒रित्य॒र्वाक् - व॒सुः॒ । तस्य॑ । तार्क्ष्यः॑ । च॒ । अरि॑ष्टनेमि॒रित्यरि॑ष्ट - ने॒मिः॒ । च॒ । से॒ना॒नि॒ग्रा॒म॒ण्या॑विति॑ सेनानि - ग्रा॒म॒ण्यौ᳚ । उ॒र्वशी᳚ । च॒ । पू॒र्वचि॑त्ति॒रिति॑ पू॒र्व - चि॒त्तिः॒ । च॒ । अ॒फ्स॒रसौ᳚ । वि॒द्युदिति॑ वित् - युत् । हे॒तिः । अ॒व॒स्फूर्ज॒न्नित्य॑व - स्फूर्जन्न्॑ । प्रहे॑ति॒रिति॒ प्र - हे॒तिः॒ । तेभ्यः॑ । नमः॑ । ते । नः॒ । मृ॒ड॒यन्तु॒ । ते । यम् ।  \newline


\textbf{Krama Paata} \newline

अ॒नु॒म्लोच॑न्ती च । अ॒नु॒म्लोच॒न्तीत्य॑नु - म्लोच॑न्ती । चा॒फ्स॒रसौ᳚ । अ॒फ्स॒रसौ॑ स॒र्पाः । स॒र्पा हे॒तिः । हे॒तिर् व्या॒घ्राः । व्या॒घ्राः प्रहे॑तिः । प्रहे॑तिर॒यम् । प्रहे॑ति॒रिति॑ प्र - हे॒तिः॒ । अ॒यमु॑त्त॒रात् । उ॒त्त॒राथ् स॒म्ॅयद्व॑सुः । उ॒त्त॒रादित्यु॑त् - त॒रात् । स॒म्ॅयुद्व॑सु॒स्तस्य॑ । स॒म्ॅयद्व॑सु॒रिति॑ स॒म्ॅयत् - व॒सुः॒ । तस्य॑ सेन॒जित् । से॒न॒जिच् च॑ । से॒न॒जिदिति॑ सेन - जित् । च॒ सु॒षेणः॑ । सु॒षेण॑श्च । सु॒षेण॒ इति॑ सु - सेनः॑ । च॒ से॒ना॒नि॒ग्रा॒म॒ण्यौ᳚ । से॒ना॒नि॒गा॒म॒ण्यौ॑ वि॒श्वाची᳚ । से॒ना॒नि॒ग्रा॒म॒ण्या॑विति॑ सेनानि - ग्रा॒म॒ण्यौ᳚ । वि॒श्वाची॑ च । च॒ घृ॒ताची᳚ । घृ॒ताची॑ च । चा॒फ्स॒रसौ᳚ । अ॒फ्स॒रसा॒वापः॑ । आपो॑ हे॒तिः । हे॒तिर् वातः॑ । वातः॒ प्रहे॑तिः । प्रहे॑तिर॒यम् । प्रहे॑ति॒रिति॒ प्र - हे॒तिः॒ । अ॒यमु॒परि॑ । उ॒पर्य॒र्वाग्व॑सुः । अ॒र्वाग्व॑सु॒स्तस्य॑ । अ॒र्वाग्व॑सु॒रित्य॒र्वाक् - व॒सुः॒ । तस्य॒ तार्क्ष्यः॑ । तार्क्ष्य॑श्च । चारि॑ष्टनेमिः । अरि॑ष्टनेमिश्च । अरि॑ष्टनेमि॒रित्यरि॑ष्ट - ने॒मिः॒ । च॒ से॒ना॒नि॒ग्रा॒म॒ण्यौ᳚ । से॒ना॒नि॒ग्रा॒म॒ण्या॑वु॒र्वशी᳚ । से॒ना॒नि॒ग्रा॒म॒ण्या॑विति॑ सेनानि - ग्रा॒म॒ण्यौ᳚ । उ॒र्वशी॑ च । च॒ पू॒र्वचि॑त्तिः । पू॒र्वचि॑त्तिश्च । पू॒र्वचि॑त्ति॒रिति॑ पू॒र्व - चि॒त्तिः॒ । चा॒फ्स॒रसौ᳚ । अ॒फ्स॒रसौ॑ वि॒द्युत् । वि॒द्युद्धे॒तिः । वि॒द्युदिति॑ वि - द्युत् । हे॒तिर॑व॒स्फूर्जन्न्॑ । अ॒व॒स्फूर्ज॒न् प्रहे॑तिः । अ॒व॒स्फूर्ज॒न्नित्य॑व - स्पूर्जन्न्॑ । प्रहे॑ति॒स्तेभ्यः॑ । प्रहे॑ति॒रिति॒ प्र - हे॒तिः॒ । तेभ्यो॒ नमः॑ । नम॒स्ते । ते नः॑ । नो॒ मृ॒ड॒य॒न्तु॒ । मृ॒ड॒य॒न्तु॒ ते । ते यम् । यम् द्वि॒ष्मः \newline

\textbf{Jatai Paata} \newline

1. अ॒नु॒म्लोच॑न्ती च चानु॒म्लोच॑न् त्यनु॒म्लोच॑न्ती च । \newline
2. अ॒नु॒म्लोच॒न्तीत्य॑नु - म्लोच॑न्ती । \newline
3. चा॒फ्स॒रसा॑ वफ्स॒रसौ॑ च चाफ्स॒रसौ᳚ । \newline
4. अ॒फ्स॒रसौ॑ स॒र्पाः स॒र्पा अ॑फ्स॒रसा॑ वफ्स॒रसौ॑ स॒र्पाः । \newline
5. स॒र्पा हे॒तिर्. हे॒तिः स॒र्पाः स॒र्पा हे॒तिः । \newline
6. हे॒तिर् व्या॒घ्रा व्या॒घ्रा हे॒तिर्. हे॒तिर् व्या॒घ्राः । \newline
7. व्या॒घ्राः प्रहे॑तिः॒ प्रहे॑तिर् व्या॒घ्रा व्या॒घ्राः प्रहे॑तिः । \newline
8. प्रहे॑ति र॒य म॒यम् प्रहे॑तिः॒ प्रहे॑ति र॒यम् । \newline
9. प्रहे॑ति॒रिति॒ प्र - हे॒तिः॒ । \newline
10. अ॒य मु॑त्त॒रा दु॑त्त॒रा द॒य म॒य मु॑त्त॒रात् । \newline
11. उ॒त्त॒राथ् सं॒ॅयद्व॑सुः सं॒ॅयद्व॑सु रुत्त॒रा दु॑त्त॒राथ् सं॒ॅयद्व॑सुः । \newline
12. उ॒त्त॒रादित्यु॑त् - त॒रात् । \newline
13. सं॒ॅयद्व॑सु॒ स्तस्य॒ तस्य॑ सं॒ॅयद्व॑सुः॒ सं॒ॅयद्व॑सु॒ स्तस्य॑ । \newline
14. सं॒ॅयद्व॑सु॒रिति॑ सं॒ॅयत् - व॒सुः॒ । \newline
15. तस्य॑ सेन॒जिथ् से॑न॒जित् तस्य॒ तस्य॑ सेन॒जित् । \newline
16. से॒न॒जिच् च॑ च सेन॒जिथ् से॑न॒जिच् च॑ । \newline
17. से॒न॒जिदिति॑ सेन - जित् । \newline
18. च॒ सु॒षेणः॑ सु॒षेण॑श्च च सु॒षेणः॑ । \newline
19. सु॒षेण॑श्च च सु॒षेणः॑ सु॒षेण॑श्च । \newline
20. सु॒षेण॒ इति॑ सु - सेनः॑ । \newline
21. च॒ से॒ना॒नि॒ग्रा॒म॒ण्यौ॑ सेनानिग्राम॒ण्यौ॑ च च सेनानिग्राम॒ण्यौ᳚ । \newline
22. से॒ना॒नि॒ग्रा॒म॒ण्यौ॑ वि॒श्वाची॑ वि॒श्वाची॑ सेनानिग्राम॒ण्यौ॑ सेनानिग्राम॒ण्यौ॑ वि॒श्वाची᳚ । \newline
23. से॒ना॒नि॒ग्रा॒म॒ण्या॑विति॑ सेनानि - ग्रा॒म॒ण्यौ᳚ । \newline
24. वि॒श्वाची॑ च च वि॒श्वाची॑ वि॒श्वाची॑ च । \newline
25. च॒ घृ॒ताची॑ घृ॒ताची॑ च च घृ॒ताची᳚ । \newline
26. घृ॒ताची॑ च च घृ॒ताची॑ घृ॒ताची॑ च । \newline
27. चा॒फ्स॒रसा॑ वफ्स॒रसौ॑ च चाफ्स॒रसौ᳚ । \newline
28. अ॒फ्स॒रसा॒ वाप॒ आपो᳚ ऽफ्स॒रसा॑ वफ्स॒रसा॒ वापः॑ । \newline
29. आपो॑ हे॒तिर्. हे॒ति राप॒ आपो॑ हे॒तिः । \newline
30. हे॒ति र्वातो॒ वातो॑ हे॒तिर्. हे॒तिर् वातः॑ । \newline
31. वातः॒ प्रहे॑तिः॒ प्रहे॑ति॒र् वातो॒ वातः॒ प्रहे॑तिः । \newline
32. प्रहे॑ति र॒य म॒यम् प्रहे॑तिः॒ प्रहे॑ति र॒यम् । \newline
33. प्रहे॑ति॒रिति॒ प्र - हे॒तिः॒ । \newline
34. अ॒य मु॒पर् यु॒पर् य॒य म॒य मु॒परि॑ । \newline
35. उ॒पर्य॒र् वाग्व॑सु र॒र्वाग्व॑सु रु॒पर् यु॒पर्य॒र् वाग्व॑सुः । \newline
36. अ॒र्वाग्व॑सु॒ स्तस्य॒ तस्या॒ र्वाग्व॑सु र॒र्वाग्व॑सु॒ स्तस्य॑ । \newline
37. अ॒र्वाग्व॑सु॒रित्य॒र्वाक् - व॒सुः॒ । \newline
38. तस्य॒ तार्क्ष्य॒ स्तार्क्ष्य॒ स्तस्य॒ तस्य॒ तार्क्ष्यः॑ । \newline
39. तार्क्ष्य॑श्च च॒ तार्क्ष्य॒ स्तार्क्ष्य॑श्च । \newline
40. चारि॑ष्टनेमि॒ ररि॑ष्टनेमिश्च॒ चारि॑ष्टनेमिः । \newline
41. अरि॑ष्टनेमिश्च॒ चारि॑ष्टनेमि॒ ररि॑ष्टनेमिश्च । \newline
42. अरि॑ष्टनेमि॒रित्यरि॑ष्ट - ने॒मिः॒ । \newline
43. च॒ से॒ना॒नि॒ग्रा॒म॒ण्यौ॑ सेनानिग्राम॒ण्यौ॑ च च सेनानिग्राम॒ण्यौ᳚ । \newline
44. से॒ना॒नि॒ग्रा॒म॒ण्या॑ वु॒र्व श्यु॒र्वशी॑ सेनानिग्राम॒ण्यौ॑ सेनानिग्राम॒ण्या॑ वु॒र्वशी᳚ । \newline
45. से॒ना॒नि॒ग्रा॒म॒ण्या॑विति॑ सेनानि - ग्रा॒म॒ण्यौ᳚ । \newline
46. उ॒र्वशी॑ च चो॒र्व श्यु॒र्वशी॑ च । \newline
47. च॒ पू॒र्वचि॑त्तिः पू॒र्वचि॑त्तिश्च च पू॒र्वचि॑त्तिः । \newline
48. पू॒र्वचि॑त्तिश्च च पू॒र्वचि॑त्तिः पू॒र्वचि॑त्तिश्च । \newline
49. पू॒र्वचि॑त्ति॒रिति॑ पू॒र्व - चि॒त्तिः॒ । \newline
50. चा॒फ्स॒रसा॑ वफ्स॒रसौ॑ च चाफ्स॒रसौ᳚ । \newline
51. अ॒फ्स॒रसौ॑ वि॒द्युद् वि॒द्यु द॑फ्स॒रसा॑ वफ्स॒रसौ॑ वि॒द्युत् । \newline
52. वि॒द्यु द्धे॒तिर्. हे॒तिर् वि॒द्युद् वि॒द्यु द्धे॒तिः । \newline
53. वि॒द्युदिति॑ वि - द्युत् । \newline
54. हे॒ति र॑व॒स्फूर्ज॑न्-नव॒स्फूर्जन्॑. हे॒तिर्. हे॒ति र॑व॒स्फूर्जन्न्॑ । \newline
55. अ॒व॒स्फूर्ज॒न् प्रहे॑तिः॒ प्रहे॑ति रव॒स्फूर्ज॑न्-नव॒स्फूर्ज॒न् प्रहे॑तिः । \newline
56. अ॒व॒स्फूर्ज॒न्नित्य॑व - स्फूर्जन्न्॑ । \newline
57. प्रहे॑ति॒ स्तेभ्य॒ स्तेभ्यः॒ प्रहे॑तिः॒ प्रहे॑ति॒ स्तेभ्यः॑ । \newline
58. प्रहे॑ति॒रिति॒ प्र - हे॒तिः॒ । \newline
59. तेभ्यो॒ नमो॒ नम॒ स्तेभ्य॒ स्तेभ्यो॒ नमः॑ । \newline
60. नम॒ स्ते ते नमो॒ नम॒ स्ते । \newline
61. ते नो॑ न॒ स्ते ते नः॑ । \newline
62. नो॒ मृ॒ड॒य॒न्तु॒ मृ॒ड॒य॒न्तु॒ नो॒ नो॒ मृ॒ड॒य॒न्तु॒ । \newline
63. मृ॒ड॒य॒न्तु॒ ते ते मृ॑डयन्तु मृडयन्तु॒ ते । \newline
64. ते यं ॅयम् ते ते यम् । \newline
65. यम् द्वि॒ष्मो द्वि॒ष्मो यं ॅयम् द्वि॒ष्मः । \newline

\textbf{Ghana Paata } \newline

1. अ॒नु॒म्लोच॑न्ती च चानु॒म्लोच॑न् त्यनु॒म्लोच॑न्ती चाफ्स॒रसा॑ वफ्स॒रसौ॑ चानु॒म्लोच॑न् त्यनु॒म्लोच॑न्ती चाफ्स॒रसौ᳚ । \newline
2. अ॒नु॒म्लोच॒न्तीत्य॑नु - म्लोच॑न्ती । \newline
3. चा॒फ्स॒रसा॑ वफ्स॒रसौ॑ च चाफ्स॒रसौ॑ स॒र्पाः स॒र्पा अ॑फ्स॒रसौ॑ च चाफ्स॒रसौ॑ स॒र्पाः । \newline
4. अ॒फ्स॒रसौ॑ स॒र्पाः स॒र्पा अ॑फ्स॒रसा॑ वफ्स॒रसौ॑ स॒र्पा हे॒तिर्. हे॒तिः स॒र्पा अ॑फ्स॒रसा॑ वफ्स॒रसौ॑ स॒र्पा हे॒तिः । \newline
5. स॒र्पा हे॒तिर्. हे॒तिः स॒र्पाः स॒र्पा हे॒तिर् व्या॒घ्रा व्या॒घ्रा हे॒तिः स॒र्पाः स॒र्पा हे॒तिर् व्या॒घ्राः । \newline
6. हे॒तिर् व्या॒घ्रा व्या॒घ्रा हे॒तिर्. हे॒तिर् व्या॒घ्राः प्रहे॑तिः॒ प्रहे॑तिर् व्या॒घ्रा हे॒तिर्. हे॒तिर् व्या॒घ्राः प्रहे॑तिः । \newline
7. व्या॒घ्राः प्रहे॑तिः॒ प्रहे॑तिर् व्या॒घ्रा व्या॒घ्राः प्रहे॑ति र॒य म॒यम् प्रहे॑तिर् व्या॒घ्रा व्या॒घ्राः प्रहे॑ति र॒यम् । \newline
8. प्रहे॑ति र॒य म॒यम् प्रहे॑तिः॒ प्रहे॑ति र॒य मु॑त्त॒रा दु॑त्त॒रा द॒यम् प्रहे॑तिः॒ प्रहे॑ति र॒य मु॑त्त॒रात् । \newline
9. प्रहे॑ति॒रिति॒ प्र - हे॒तिः॒ । \newline
10. अ॒य मु॑त्त॒रा दु॑त्त॒रा द॒य म॒य मु॑त्त॒राथ् सं॒ॅयद्व॑सुः सं॒ॅयद्व॑सु रुत्त॒रा द॒य म॒य मु॑त्त॒राथ् सं॒ॅयद्व॑सुः । \newline
11. उ॒त्त॒राथ् सं॒ॅयद्व॑सुः सं॒ॅयद्व॑सु रुत्त॒रा दु॑त्त॒राथ् सं॒ॅयद्व॑सु॒ स्तस्य॒ तस्य॑ सं॒ॅयद्व॑सु रुत्त॒रा दु॑त्त॒राथ् सं॒ॅयद्व॑सु॒ स्तस्य॑ । \newline
12. उ॒त्त॒रादित्यु॑त् - त॒रात् । \newline
13. सं॒ॅयद्व॑सु॒ स्तस्य॒ तस्य॑ सं॒ॅयद्व॑सुः सं॒ॅयद्व॑सु॒ स्तस्य॑ सेन॒जिथ् से॑न॒जित् तस्य॑ सं॒ॅयद्व॑सुः सं॒ॅयद्व॑सु॒ स्तस्य॑ सेन॒जित् । \newline
14. सं॒ॅयद्व॑सु॒रिति॑ सं॒ॅयत् - व॒सुः॒ । \newline
15. तस्य॑ सेन॒जिथ् से॑न॒जित् तस्य॒ तस्य॑ सेन॒जिच् च॑ च सेन॒जित् तस्य॒ तस्य॑ सेन॒जिच् च॑ । \newline
16. से॒न॒जिच् च॑ च सेन॒जिथ् से॑न॒जिच् च॑ सु॒षेणः॑ सु॒षेण॑श्च सेन॒जिथ् से॑न॒जिच् च॑ सु॒षेणः॑ । \newline
17. से॒न॒जिदिति॑ सेन - जित् । \newline
18. च॒ सु॒षेणः॑ सु॒षेण॑श्च च सु॒षेण॑श्च च सु॒षेण॑श्च च सु॒षेण॑श्च । \newline
19. सु॒षेण॑श्च च सु॒षेणः॑ सु॒षेण॑श्च सेनानिग्राम॒ण्यौ॑ सेनानिग्राम॒ण्यौ॑ च सु॒षेणः॑ सु॒षेण॑श्च सेनानिग्राम॒ण्यौ᳚ । \newline
20. सु॒षेण॒ इति॑ सु - सेनः॑ । \newline
21. च॒ से॒ना॒नि॒ग्रा॒म॒ण्यौ॑ सेनानिग्राम॒ण्यौ॑ च च सेनानिग्राम॒ण्यौ॑ वि॒श्वाची॑ वि॒श्वाची॑ सेनानिग्राम॒ण्यौ॑ च च सेनानिग्राम॒ण्यौ॑ वि॒श्वाची᳚ । \newline
22. से॒ना॒नि॒ग्रा॒म॒ण्यौ॑ वि॒श्वाची॑ वि॒श्वाची॑ सेनानिग्राम॒ण्यौ॑ सेनानिग्राम॒ण्यौ॑ वि॒श्वाची॑ च च वि॒श्वाची॑ सेनानिग्राम॒ण्यौ॑ सेनानिग्राम॒ण्यौ॑ वि॒श्वाची॑ च । \newline
23. से॒ना॒नि॒ग्रा॒म॒ण्या॑विति॑ सेनानि - ग्रा॒म॒ण्यौ᳚ । \newline
24. वि॒श्वाची॑ च च वि॒श्वाची॑ वि॒श्वाची॑ च घृ॒ताची॑ घृ॒ताची॑ च वि॒श्वाची॑ वि॒श्वाची॑ च घृ॒ताची᳚ । \newline
25. च॒ घृ॒ताची॑ घृ॒ताची॑ च च घृ॒ताची॑ च च घृ॒ताची॑ च च घृ॒ताची॑ च । \newline
26. घृ॒ताची॑ च च घृ॒ताची॑ घृ॒ताची॑ चाफ्स॒रसा॑ वफ्स॒रसौ॑ च घृ॒ताची॑ घृ॒ताची॑ चाफ्स॒रसौ᳚ । \newline
27. चा॒फ्स॒रसा॑ वफ्स॒रसौ॑ च चाफ्स॒रसा॒ वाप॒ आपो᳚ ऽफ्स॒रसौ॑ च चाफ्स॒रसा॒ वापः॑ । \newline
28. अ॒फ्स॒रसा॒ वाप॒ आपो᳚ ऽफ्स॒रसा॑ वफ्स॒रसा॒ वापो॑ हे॒तिर्. हे॒ति रापो᳚ ऽफ्स॒रसा॑ वफ्स॒रसा॒ वापो॑ हे॒तिः । \newline
29. आपो॑ हे॒तिर्. हे॒ति राप॒ आपो॑ हे॒तिर् वातो॒ वातो॑ हे॒ति राप॒ आपो॑ हे॒तिर् वातः॑ । \newline
30. हे॒तिर् वातो॒ वातो॑ हे॒तिर्. हे॒तिर् वातः॒ प्रहे॑तिः॒ प्रहे॑ति॒र् वातो॑ हे॒तिर्. हे॒तिर् वातः॒ प्रहे॑तिः । \newline
31. वातः॒ प्रहे॑तिः॒ प्रहे॑ति॒र् वातो॒ वातः॒ प्रहे॑ति र॒य म॒यम् प्रहे॑ति॒र् वातो॒ वातः॒ प्रहे॑ति र॒यम् । \newline
32. प्रहे॑ति र॒य म॒यम् प्रहे॑तिः॒ प्रहे॑ति र॒य मु॒पर् यु॒पर् य॒यम् प्रहे॑तिः॒ प्रहे॑ति र॒य मु॒परि॑ । \newline
33. प्रहे॑ति॒रिति॒ प्र - हे॒तिः॒ । \newline
34. आ॒य मु॒पर् यु॒पर् य॒य म॒य मु॒पर् य॒र्वाग्व॑सु र॒र्वाग्व॑सु रु॒प र्य॒य म॒य मु॒पर् य॒र्वाग्व॑सुः । \newline
35. उ॒प र्य॒र्वाग्व॑सु र॒र्वाग्व॑सु रु॒पर्यु॒प र्य॒र्वाग्व॑सु॒ स्तस्य॒ तस्या॒ र्वाग्व॑सु रु॒प र्यु॒प र्य॒र्वाग्व॑सु॒ स्तस्य॑ । \newline
36. अ॒र्वाग्व॑सु॒ स्तस्य॒ तस्या॒ र्वाग्व॑सु र॒र्वाग्व॑सु॒ स्तस्य॒ तार्क्ष्य॒ स्तार्क्ष्य॒ स्तस्या॒ र्वाग्व॑सु र॒र्वाग्व॑सु॒ स्तस्य॒ तार्क्ष्यः॑ । \newline
37. अ॒र्वाग्व॑सु॒रित्य॒र्वाक् - व॒सुः॒ । \newline
38. तस्य॒ तार्क्ष्य॒ स्तार्क्ष्य॒ स्तस्य॒ तस्य॒ तार्क्ष्य॑श्च च॒ तार्क्ष्य॒ स्तस्य॒ तस्य॒ तार्क्ष्य॑श्च । \newline
39. तार्क्ष्य॑श्च च॒ तार्क्ष्य॒ स्तार्क्ष्य॒ श्चारि॑ष्टनेमि॒ ररि॑ष्टनेमिश्च॒ तार्क्ष्य॒ स्तार्क्ष्य॒श्चा रि॑ष्टनेमिः । \newline
40. चारि॑ष्टनेमि॒ ररि॑ष्टनेमिश्च॒ चारि॑ष्टनेमिश्च॒ चारि॑ष्टनेमिश्च॒ चारि॑ष्टनेमिश्च । \newline
41. अरि॑ष्टनेमिश्च॒ चारि॑ष्टनेमि॒ ररि॑ष्टनेमिश्च सेनानिग्राम॒ण्यौ॑ सेनानिग्राम॒ण्यौ॑ चारि॑ष्टनेमि॒ ररि॑ष्टनेमिश्च सेनानिग्राम॒ण्यौ᳚ । \newline
42. अरि॑ष्टनेमि॒रित्यरि॑ष्ट - ने॒मिः॒ । \newline
43. च॒ से॒ना॒नि॒ग्रा॒म॒ण्यौ॑ सेनानिग्राम॒ण्यौ॑ च च सेनानिग्राम॒ण्या॑ वु॒र्व श्यु॒र्वशी॑ सेनानिग्राम॒ण्यौ॑ च च सेनानिग्राम॒ण्या॑ वु॒र्वशी᳚ । \newline
44. से॒ना॒नि॒ग्रा॒म॒ण्या॑ वु॒र्व श्यु॒र्वशी॑ सेनानिग्राम॒ण्यौ॑ सेनानिग्राम॒ण्या॑ वु॒र्वशी॑ च चो॒र्वशी॑ सेनानिग्राम॒ण्यौ॑ सेनानिग्राम॒ण्या॑ वु॒र्वशी॑ च । \newline
45. से॒ना॒नि॒ग्रा॒म॒ण्या॑विति॑ सेनानि - ग्रा॒म॒ण्यौ᳚ । \newline
46. उ॒र्वशी॑ च चो॒र्व श्यु॒र्वशी॑ च पू॒र्वचि॑त्तिः पू॒र्वचि॑त्ति श्चो॒र्व श्यु॒र्वशी॑ च पू॒र्वचि॑त्तिः । \newline
47. च॒ पू॒र्वचि॑त्तिः पू॒र्वचि॑त्तिश्च च पू॒र्वचि॑त्तिश्च च पू॒र्वचि॑त्तिश्च च पू॒र्वचि॑त्तिश्च । \newline
48. पू॒र्वचि॑त्तिश्च च पू॒र्वचि॑त्तिः पू॒र्वचि॑त्ति श्चाफ्स॒रसा॑ वफ्स॒रसौ॑ च पू॒र्वचि॑त्तिः पू॒र्वचि॑त्ति श्चाफ्स॒रसौ᳚ । \newline
49. पू॒र्वचि॑त्ति॒रिति॑ पू॒र्व - चि॒त्तिः॒ । \newline
50. चा॒फ्स॒रसा॑ वफ्स॒रसौ॑ च चाफ्स॒रसौ॑ वि॒द्युद् वि॒द्यु द॑फ्स॒रसौ॑ च चाफ्स॒रसौ॑ वि॒द्युत् । \newline
51. अ॒फ्स॒रसौ॑ वि॒द्युद् वि॒द्यु द॑फ्स॒रसा॑ वफ्स॒रसौ॑ वि॒द्यु द्धे॒तिर्. हे॒तिर् वि॒द्यु द॑फ्स॒रसा॑ वफ्स॒रसौ॑ वि॒द्यु द्धे॒तिः । \newline
52. वि॒द्यु द्धे॒तिर्. हे॒तिर् वि॒द्युद् वि॒द्यु द्धे॒ति र॑व॒स्फूर्ज॑न् नव॒स्फूर्जन्॑. हे॒तिर् वि॒द्युद् वि॒द्यु द्धे॒ति र॑व॒स्फूर्जन्न्॑ । \newline
53. वि॒द्युदिति॑ वि - द्युत् । \newline
54. हे॒ति र॑व॒स्फूर्ज॑न् नव॒स्फूर्जन्॑. हे॒तिर्. हे॒ति र॑व॒स्फूर्ज॒न् प्रहे॑तिः॒ प्रहे॑ति रव॒स्फूर्जन्॑. हे॒तिर्. हे॒ति र॑व॒स्फूर्ज॒न् प्रहे॑तिः । \newline
55. अ॒व॒स्फूर्ज॒न् प्रहे॑तिः॒ प्रहे॑ति रव॒स्फूर्ज॑न् नव॒स्फूर्ज॒न् प्रहे॑ति॒ स्तेभ्य॒ स्तेभ्यः॒ प्रहे॑ति रव॒स्फूर्ज॑न् नव॒स्फूर्ज॒न् प्रहे॑ति॒ स्तेभ्यः॑ । \newline
56. अ॒व॒स्फूर्ज॒न्नित्य॑व - स्फूर्जन्न्॑ । \newline
57. प्रहे॑ति॒ स्तेभ्य॒ स्तेभ्यः॒ प्रहे॑तिः॒ प्रहे॑ति॒ स्तेभ्यो॒ नमो॒ नम॒ स्तेभ्यः॒ प्रहे॑तिः॒ प्रहे॑ति॒ स्तेभ्यो॒ नमः॑ । \newline
58. प्रहे॑ति॒रिति॒ प्र - हे॒तिः॒ । \newline
59. तेभ्यो॒ नमो॒ नम॒ स्तेभ्य॒ स्तेभ्यो॒ नम॒ स्ते ते नम॒ स्तेभ्य॒ स्तेभ्यो॒ नम॒ स्ते । \newline
60. नम॒ स्ते ते नमो॒ नम॒ स्ते नो॑ न॒ स्ते नमो॒ नम॒ स्ते नः॑ । \newline
61. ते नो॑ न॒ स्ते ते नो॑ मृडयन्तु मृडयन्तु न॒ स्ते ते नो॑ मृडयन्तु । \newline
62. नो॒ मृ॒ड॒य॒न्तु॒ मृ॒ड॒य॒न्तु॒ नो॒ नो॒ मृ॒ड॒य॒न्तु॒ ते ते मृ॑डयन्तु नो नो मृडयन्तु॒ ते । \newline
63. मृ॒ड॒य॒न्तु॒ ते ते मृ॑डयन्तु मृडयन्तु॒ ते यं ॅयम् ते मृ॑डयन्तु मृडयन्तु॒ ते यम् । \newline
64. ते यं ॅयम् ते ते यम् द्वि॒ष्मो द्वि॒ष्मो यम् ते ते यम् द्वि॒ष्मः । \newline
65. यम् द्वि॒ष्मो द्वि॒ष्मो यं ॅयम् द्वि॒ष्मो यो यो द्वि॒ष्मो यं ॅयम् द्वि॒ष्मो यः । \newline
\pagebreak
\markright{ TS 4.4.3.3  \hfill https://www.vedavms.in \hfill}

\section{ TS 4.4.3.3 }

\textbf{TS 4.4.3.3 } \newline
\textbf{Samhita Paata} \newline

द्वि॒ष्मो यश्च॑ नो॒ द्वेष्टि॒ तं ॅवो॒ जंभे॑ दधाम्या॒योस्त्वा॒ सद॑ने सादया॒म्यव॑त श्छा॒यायां॒ नमः॑ समु॒द्राय॒ नमः॑ समु॒द्रस्य॒ चक्ष॑से परमे॒ष्ठी त्वा॑ सादयतु दि॒वः पृ॒ष्ठे व्यच॑स्वतीं॒ प्रथ॑स्वतीं ॅवि॒भूम॑तीं प्र॒भूम॑तीं परि॒भूम॑तीं॒ दिवं॑ ॅयच्छ॒ दिवं॑ दृꣳह॒ दिवं॒ मा हिꣳ॑सी॒र्विश्व॑स्मै प्रा॒णाया॑पा॒नाय॑ व्या॒नायो॑दा॒नाय॑ प्रति॒ष्ठायै॑ च॒रित्रा॑य॒ सूर्य॑स्त्वा॒ऽभि पा॑तु म॒ह्या स्व॒स्त्या ( ) छ॒र्दिषा॒ शन्त॑मेन॒ तया॑ दे॒वत॑याऽङ्गिर॒स्वद्-ध्रु॒वा सी॑द ॥ प्रोथ॒दश्वो॒ न यव॑से अवि॒ष्यन्. य॒दा म॒हः स॒ॅवर॑णा॒द् व्यस्था᳚त् । आद॑स्य॒ वातो॒ अनु॑ वाति शो॒चिरध॑ स्म ते॒ व्रज॑नं कृ॒ष्णम॑स्ति ॥ \newline

\textbf{Pada Paata} \newline

द्वि॒ष्मः । यः । च॒ । नः॒ । द्वेष्टि॑ । तम् । वः॒ । जंभे᳚ । द॒धा॒मि॒ । आ॒योः । त्वा॒ । सद॑ने । सा॒द॒या॒मि॒ । अव॑तः । छा॒याया᳚म् । नमः॑ । स॒मु॒द्राय॑ । नमः॑ । स॒मु॒द्रस्य॑ । चक्ष॑से । प॒र॒मे॒ष्ठी । त्वा॒ । सा॒द॒य॒तु॒ । दि॒वः । पृ॒ष्ठे । व्यच॑स्वतीम् । प्रथ॑स्वतीम् । वि॒भूम॑ती॒मिति॑ वि - भूम॑तीम् । प्र॒भूम॑ती॒मिति॑ प्र - भूम॑तीम् । प॒रि॒भूम॑ती॒मिति॑ परि - भूम॑तीम् । दिव᳚म् । य॒च्छ॒ । दिव᳚म् । दृꣳ॒॒ह॒ । दिव᳚म् । मा । हिꣳ॒॒सीः॒ । विश्व॑स्मै । प्रा॒णायेति॑ प्र - अ॒नाय॑ । अ॒पा॒नायेत्य॑प - अ॒नाय॑ । व्या॒नायेति॑ वि - अ॒नाय॑ । उ॒दा॒नायेत्यु॑त् - अ॒नाय॑ । प्र॒ति॒ष्ठाया॒ इति॑ प्रति - स्थायै᳚ । च॒रित्रा॑य । सूर्यः॑ । त्वा॒ । अ॒भीति॑ । पा॒तु॒ । म॒ह्या । स्व॒स्त्या ( ) । छ॒र्दिषा᳚ । शन्त॑मे॒नेति॒ शं - त॒मे॒न॒ । तया᳚ । दे॒वत॑या । अ॒ङ्गि॒र॒स्वत् । ध्रु॒वा । सी॒द॒ ॥ प्रोथ॑त् । अश्वः॑ । न । यव॑से । अ॒वि॒ष्यन्न् । य॒दा । म॒हः । सं॒ॅवर॑णा॒दिति॑ सं - वर॑णात् । व्यस्था॒दिति॑ वि - अस्था᳚त् ॥ आत् । अ॒स्य॒ । वातः॑ । अन्विति॑ । वा॒ति॒ । शो॒चिः । अध॑ । स्म॒ । ते॒ । व्रज॑नम् । कृ॒ष्णम् । अ॒स्ति॒ ॥  \newline


\textbf{Krama Paata} \newline

द्वि॒ष्मो यः । यश्च॑ । च॒ नः॒ । नो॒ द्वेष्टि॑ । द्वेष्टि॒ तम् । तं ॅवः॑ । वो॒ जम्भे᳚ । जम्भे॑ दधामि । द॒धा॒म्या॒योः । आ॒योस्त्वा᳚ । त्वा॒ सद॑ने । सद॑ने सादयामि । सा॒द॒या॒म्यव॑तः । अव॑तश्छा॒याया᳚म् । छा॒याया॒म् नमः॑ । नमः॑ समु॒द्राय॑ । स॒मु॒द्राय॒ नमः॑ । नमः॑ समु॒द्रस्य॑ । स॒मु॒द्रस्य॒ चक्ष॑से । चक्ष॑से परमे॒ष्ठी । प॒र॒मे॒ष्ठी त्वा᳚ । त्वा॒ सा॒द॒य॒तु॒ । सा॒द॒य॒तु॒ दि॒वः । दि॒वः पृ॒ष्ठे । पृ॒ष्ठे व्यच॑स्वतीम् । व्यच॑स्वती॒म् प्रथ॑स्वतीम् । प्रथ॑स्वतीं ॅवि॒भूम॑तीम् । वि॒भूम॑तीम् प्र॒भूम॑तीम् । वि॒भूम॑ती॒मिति॑ वि - भूम॑तीम् । प्र॒भूम॑तीम् परि॒भूम॑तीम् । प्र॒भूम॑ती॒मिति॑ प्र - भूम॑तीम् । प॒रि॒भूम॑ती॒म् दिव᳚म् । प॒रि॒भूम॑ती॒मिति॑ परि - भूम॑तीम् । दिवं॑ ॅयच्छ । य॒च्छ॒ दिव᳚म् । दिव॑म् दृꣳह । दृꣳ॒॒ह॒ दिव᳚म् । दिव॒म् मा । मा हिꣳ॑सीः । हिꣳ॒॒सी॒र् विश्व॑स्मै । विश्व॑स्मै प्रा॒णाय॑ । प्रा॒णाया॑पा॒नाय॑ । प्रा॒णायेति॑ प्र - अ॒नाय॑ । अ॒पा॒नाय॑ व्या॒नाय॑ । अ॒पा॒नायेत्य॑प - अ॒नाय॑ । व्या॒नायो॑दा॒नाय॑ । व्या॒नायेति॑ वि - अ॒नाय॑ । उ॒दा॒नाय॑ प्रति॒ष्ठायै᳚ । उ॒दा॒नायेत्यु॑त् - अ॒नाय॑ । प्र॒ति॒ष्ठायै॑ च॒रित्रा॑य । प्र॒ति॒ष्ठाया॒ इति॑ प्रति - स्थायै᳚ । च॒रित्रा॑य॒ सूर्यः॑ । सूर्य॑स्त्वा । त्वा॒ऽभि । अ॒भि पा॑तु । पा॒तु॒ म॒ह्या । म॒ह्या स्व॒स्त्या ( ) । स्व॒स्त्या छ॒र्दिषा᳚ । छ॒र्दिषा॒ शन्त॑मेन । शन्त॑मेन॒ तया᳚ । शन्त॑मे॒नेति॒ शम् - त॒मे॒न॒ । तया॑ दे॒वत॑या । दे॒वत॑याऽङ्गिर॒स्वत् । अ॒ङ्गि॒र॒स्वद् ध्रु॒वा । ध्रु॒वा सी॑द । सी॒देति॑ सीद ॥ प्रोथ॒दश्वः॑ । अश्वो॒ न । न यव॑से । यव॑से अवि॒ष्यन्न् । अ॒वि॒ष्यन्. य॒दा । य॒दा म॒हः । म॒हः स॒म्ॅवर॑णात् । स॒म्ॅवर॑णा॒द् व्यस्था᳚त् । स॒म्ॅवर॑णा॒दिति॑ सम् - वर॑णात् । व्यस्था॒दिति॑ वि - अस्था᳚त् ॥ आद॑स्य । अ॒स्य॒ वातः॑ । वातो॒ अनु॑ । अनु॑ वाति । वा॒ति॒ शो॒चिः । शो॒चिरध॑ । अध॑ स्म । स्म॒ ते॒ । ते॒ व्रज॑नम् । व्रज॑नम् कृ॒ष्णम् । कृ॒ष्णम॑स्ति । अ॒स्तीत्य॑स्ति । \newline

\textbf{Jatai Paata} \newline

1. द्वि॒ष्मो यो यो द्वि॒ष्मो द्वि॒ष्मो यः । \newline
2. यश्च॑ च॒ यो यश्च॑ । \newline
3. च॒ नो॒ न॒श्च॒ च॒ नः॒ । \newline
4. नो॒ द्वेष्टि॒ द्वेष्टि॑ नो नो॒ द्वेष्टि॑ । \newline
5. द्वेष्टि॒ तम् तम् द्वेष्टि॒ द्वेष्टि॒ तम् । \newline
6. तं ॅवो॑ व॒ स्तम् तं ॅवः॑ । \newline
7. वो॒ जंभे॒ जंभे॑ वो वो॒ जंभे᳚ । \newline
8. जंभे॑ दधामि दधामि॒ जंभे॒ जंभे॑ दधामि । \newline
9. द॒धा॒ म्या॒यो रा॒योर् द॑धामि दधा म्या॒योः । \newline
10. आ॒यो स्त्वा᳚ त्वा॒ ऽऽयो रा॒यो स्त्वा᳚ । \newline
11. त्वा॒ सद॑ने॒ सद॑ने त्वा त्वा॒ सद॑ने । \newline
12. सद॑ने सादयामि सादयामि॒ सद॑ने॒ सद॑ने सादयामि । \newline
13. सा॒द॒या॒ म्यव॒तो ऽव॑तः सादयामि सादया॒ म्यव॑तः । \newline
14. अव॑त श्छा॒याया᳚म् छा॒याया॒ मव॒तो ऽव॑त श्छा॒याया᳚म् । \newline
15. छा॒याया॒न् नमो॒ नम॑ श्छा॒याया᳚म् छा॒याया॒न् नमः॑ । \newline
16. नमः॑ समु॒द्राय॑ समु॒द्राय॒ नमो॒ नमः॑ समु॒द्राय॑ । \newline
17. स॒मु॒द्राय॒ नमो॒ नमः॑ समु॒द्राय॑ समु॒द्राय॒ नमः॑ । \newline
18. नमः॑ समु॒द्रस्य॑ समु॒द्रस्य॒ नमो॒ नमः॑ समु॒द्रस्य॑ । \newline
19. स॒मु॒द्रस्य॒ चक्ष॑से॒ चक्ष॑से समु॒द्रस्य॑ समु॒द्रस्य॒ चक्ष॑से । \newline
20. चक्ष॑से परमे॒ष्ठी प॑रमे॒ष्ठी चक्ष॑से॒ चक्ष॑से परमे॒ष्ठी । \newline
21. प॒र॒मे॒ष्ठी त्वा᳚ त्वा परमे॒ष्ठी प॑रमे॒ष्ठी त्वा᳚ । \newline
22. त्वा॒ सा॒द॒य॒तु॒ सा॒द॒य॒तु॒ त्वा॒ त्वा॒ सा॒द॒य॒तु॒ । \newline
23. सा॒द॒य॒तु॒ दि॒वो दि॒वः सा॑दयतु सादयतु दि॒वः । \newline
24. दि॒वः पृ॒ष्ठे पृ॒ष्ठे दि॒वो दि॒वः पृ॒ष्ठे । \newline
25. पृ॒ष्ठे व्यच॑स्वतीं॒ ॅव्यच॑स्वतीम् पृ॒ष्ठे पृ॒ष्ठे व्यच॑स्वतीम् । \newline
26. व्यच॑स्वती॒म् प्रथ॑स्वती॒म् प्रथ॑स्वतीं॒ ॅव्यच॑स्वतीं॒ ॅव्यच॑स्वती॒म् प्रथ॑स्वतीम् । \newline
27. प्रथ॑स्वतीं ॅवि॒भूम॑तीं ॅवि॒भूम॑ती॒म् प्रथ॑स्वती॒म् प्रथ॑स्वतीं ॅवि॒भूम॑तीम् । \newline
28. वि॒भूम॑तीम् प्र॒भूम॑तीम् प्र॒भूम॑तीं ॅवि॒भूम॑तीं ॅवि॒भूम॑तीम् प्र॒भूम॑तीम् । \newline
29. वि॒भूम॑ती॒मिति॑ वि - भूम॑तीम् । \newline
30. प्र॒भूम॑तीम् परि॒भूम॑तीम् परि॒भूम॑तीम् प्र॒भूम॑तीम् प्र॒भूम॑तीम् परि॒भूम॑तीम् । \newline
31. प्र॒भूम॑ती॒मिति॑ प्र - भूम॑तीम् । \newline
32. प॒रि॒भूम॑ती॒म् दिव॒म् दिव॑म् परि॒भूम॑तीम् परि॒भूम॑ती॒म् दिव᳚म् । \newline
33. प॒रि॒भूम॑ती॒मिति॑ परि - भूम॑तीम् । \newline
34. दिवं॑ ॅयच्छ यच्छ॒ दिव॒म् दिवं॑ ॅयच्छ । \newline
35. य॒च्छ॒ दिव॒म् दिवं॑ ॅयच्छ यच्छ॒ दिव᳚म् । \newline
36. दिव॑म् दृꣳह दृꣳह॒ दिव॒म् दिव॑म् दृꣳह । \newline
37. दृꣳ॒॒ह॒ दिव॒म् दिव॑म् दृꣳह दृꣳह॒ दिव᳚म् । \newline
38. दिव॒म् मा मा दिव॒म् दिव॒म् मा । \newline
39. मा हिꣳ॑सीर्. हिꣳसी॒र् मा मा हिꣳ॑सीः । \newline
40. हिꣳ॒॒सी॒र् विश्व॑स्मै॒ विश्व॑स्मै हिꣳसीर्. हिꣳसी॒र् विश्व॑स्मै । \newline
41. विश्व॑स्मै प्रा॒णाय॑ प्रा॒णाय॒ विश्व॑स्मै॒ विश्व॑स्मै प्रा॒णाय॑ । \newline
42. प्रा॒णाया॑ पा॒नाया॑ पा॒नाय॑ प्रा॒णाय॑ प्रा॒णाया॑ पा॒नाय॑ । \newline
43. प्रा॒णायेति॑ प्र - अ॒नाय॑ । \newline
44. अ॒पा॒नाय॑ व्या॒नाय॑ व्या॒नाया॑ पा॒नाया॑ पा॒नाय॑ व्या॒नाय॑ । \newline
45. अ॒पा॒नायेत्य॑प - अ॒नाय॑ । \newline
46. व्या॒ना यो॑दा॒ना यो॑दा॒नाय॑ व्या॒नाय॑ व्या॒ना यो॑दा॒नाय॑ । \newline
47. व्या॒नायेति॑ वि - अ॒नाय॑ । \newline
48. उ॒दा॒नाय॑ प्रति॒ष्ठायै᳚ प्रति॒ष्ठाया॑ उदा॒ना यो॑दा॒नाय॑ प्रति॒ष्ठायै᳚ । \newline
49. उ॒दा॒नायेत्यु॑त् - अ॒नाय॑ । \newline
50. प्र॒ति॒ष्ठायै॑ च॒रित्रा॑य च॒रित्रा॑य प्रति॒ष्ठायै᳚ प्रति॒ष्ठायै॑ च॒रित्रा॑य । \newline
51. प्र॒ति॒ष्ठाया॒ इति॑ प्रति - स्थायै᳚ । \newline
52. च॒रित्रा॑य॒ सूर्यः॒ सूर्य॑ श्च॒रित्रा॑य च॒रित्रा॑य॒ सूर्यः॑ । \newline
53. सूर्य॑ स्त्वा त्वा॒ सूर्यः॒ सूर्य॑ स्त्वा । \newline
54. त्वा॒ ऽभ्य॑भि त्वा᳚ त्वा॒ ऽभि । \newline
55. अ॒भि पा॑तु पात्व॒भ्य॑भि पा॑तु । \newline
56. पा॒तु॒ म॒ह्या म॒ह्या पा॑तु पातु म॒ह्या । \newline
57. म॒ह्या स्व॒स्त्या स्व॒स्त्या म॒ह्या म॒ह्या स्व॒स्त्या । \newline
58. स्व॒स्त्या छ॒र्दिषा॑ छ॒र्दिषा᳚ स्व॒स्त्या स्व॒स्त्या छ॒र्दिषा᳚ । \newline
59. छ॒र्दिषा॒ शन्त॑मेन॒ शन्त॑मेन छ॒र्दिषा॑ छ॒र्दिषा॒ शन्त॑मेन । \newline
60. शन्त॑मेन॒ तया॒ तया॒ शन्त॑मेन॒ शन्त॑मेन॒ तया᳚ । \newline
61. शन्त॑मे॒नेति॒ शं - त॒मे॒न॒ । \newline
62. तया॑ दे॒वत॑या दे॒वत॑या॒ तया॒ तया॑ दे॒वत॑या । \newline
63. दे॒वत॑या ऽङ्गिर॒स्व द॑ङ्गिर॒स्वद् दे॒वत॑या दे॒वत॑या ऽङ्गिर॒स्वत् । \newline
64. अ॒ङ्गि॒र॒स्वद् ध्रु॒वा ध्रु॒वा ऽङ्गि॑र॒स्व द॑ङ्गिर॒स्वद् ध्रु॒वा । \newline
65. ध्रु॒वा सी॑द सीद ध्रु॒वा ध्रु॒वा सी॑द । \newline
66. सी॒देति॑ सीद । \newline
67. प्रोथ॒ दश्वो ऽश्वः॒ प्रोथ॒त् प्रोथ॒ दश्वः॑ । \newline
68. अश्वो॒ न नाश्वो ऽश्वो॒ न । \newline
69. न यव॑से॒ यव॑से॒ न न यव॑से । \newline
70. यव॑से अवि॒ष्यन् न॑वि॒ष्यन्. यव॑से॒ यव॑से अवि॒ष्यन्न् । \newline
71. अ॒वि॒ष्यन्. य॒दा य॒दा ऽवि॒ष्यन् न॑वि॒ष्यन्. य॒दा । \newline
72. य॒दा म॒हो म॒हो य॒दा य॒दा म॒हः । \newline
73. म॒हः सं॒ॅवर॑णाथ् सं॒ॅवर॑णान् म॒हो म॒हः सं॒ॅवर॑णात् । \newline
74. सं॒ॅवर॑णा॒द् व्यस्था॒द् व्यस्था᳚थ् सं॒ॅवर॑णाथ् सं॒ॅवर॑णा॒द् व्यस्था᳚त् । \newline
75. सं॒ॅवर॑णा॒दिति॑ सं - वर॑णात् । \newline
76. व्यस्था॒दिति॑ वि - अस्था᳚त् । \newline
77. आद॑स्या॒ स्यादा द॑स्य । \newline
78. अ॒स्य॒ वातो॒ वातो᳚ ऽस्यास्य॒ वातः॑ । \newline
79. वातो॒ अन्वनु॒ वातो॒ वातो॒ अनु॑ । \newline
80. अनु॑ वाति वा॒त्यन् वनु॑ वाति । \newline
81. वा॒ति॒ शो॒चिः शो॒चिर् वा॑ति वाति शो॒चिः । \newline
82. शो॒चि रधाध॑ शो॒चिः शो॒चि रध॑ । \newline
83. अध॑ स्म॒ स्मा धाध॑ स्म । \newline
84. स्म॒ ते॒ ते॒ स्म॒ स्म॒ ते॒ । \newline
85. ते॒ व्रज॑नं॒ ॅव्रज॑नम् ते ते॒ व्रज॑नम् । \newline
86. व्रज॑नम् कृ॒ष्णम् कृ॒ष्णं ॅव्रज॑नं॒ ॅव्रज॑नम् कृ॒ष्णम् । \newline
87. कृ॒ष्ण म॑स्त्यस्ति कृ॒ष्णम् कृ॒ष्ण म॑स्ति । \newline
88. अ॒स्तीत्य॑स्ति । \newline

\textbf{Ghana Paata } \newline

1. द्वि॒ष्मो यो यो द्वि॒ष्मो द्वि॒ष्मो यश्च॑ च॒ यो द्वि॒ष्मो द्वि॒ष्मो यश्च॑ । \newline
2. यश्च॑ च॒ यो यश्च॑ नो नश्च॒ यो यश्च॑ नः । \newline
3. च॒ नो॒ न॒श्च॒ च॒ नो॒ द्वेष्टि॒ द्वेष्टि॑ नश्च च नो॒ द्वेष्टि॑ । \newline
4. नो॒ द्वेष्टि॒ द्वेष्टि॑ नो नो॒ द्वेष्टि॒ तम् तम् द्वेष्टि॑ नो नो॒ द्वेष्टि॒ तम् । \newline
5. द्वेष्टि॒ तम् तम् द्वेष्टि॒ द्वेष्टि॒ तं ॅवो॑ व॒ स्तम् द्वेष्टि॒ द्वेष्टि॒ तं ॅवः॑ । \newline
6. तं ॅवो॑ व॒ स्तम् तं ॅवो॒ जंभे॒ जंभे॑ व॒ स्तम् तं ॅवो॒ जंभे᳚ । \newline
7. वो॒ जंभे॒ जंभे॑ वो वो॒ जंभे॑ दधामि दधामि॒ जंभे॑ वो वो॒ जंभे॑ दधामि । \newline
8. जंभे॑ दधामि दधामि॒ जंभे॒ जंभे॑ दधा म्या॒यो रा॒योर् द॑धामि॒ जंभे॒ जंभे॑ दधा म्या॒योः । \newline
9. द॒धा॒ म्या॒यो रा॒योर् द॑धामि दधा म्या॒यो स्त्वा᳚ त्वा॒ ऽऽयोर् द॑धामि दधा म्या॒यो स्त्वा᳚ । \newline
10. आ॒यो स्त्वा᳚ त्वा॒ ऽऽयो रा॒यो स्त्वा॒ सद॑ने॒ सद॑ने त्वा॒ ऽऽयो रा॒यो स्त्वा॒ सद॑ने । \newline
11. त्वा॒ सद॑ने॒ सद॑ने त्वा त्वा॒ सद॑ने सादयामि सादयामि॒ सद॑ने त्वा त्वा॒ सद॑ने सादयामि । \newline
12. सद॑ने सादयामि सादयामि॒ सद॑ने॒ सद॑ने सादया॒ म्यव॒तो ऽव॑तः सादयामि॒ सद॑ने॒ सद॑ने सादया॒ म्यव॑तः । \newline
13. सा॒द॒या॒ म्यव॒तो ऽव॑तः सादयामि सादया॒ म्यव॑त श्छा॒याया᳚म् छा॒याया॒ मव॑तः सादयामि सादया॒ म्यव॑त श्छा॒याया᳚म् । \newline
14. अव॑त श्छा॒याया᳚म् छा॒याया॒ मव॒तो ऽव॑त श्छा॒याया॒म् नमो॒ नम॑ श्छा॒याया॒ मव॒तो ऽव॑त श्छा॒याया॒म् नमः॑ । \newline
15. छा॒याया॒म् नमो॒ नम॑ श्छा॒याया᳚म् छा॒याया॒म् नमः॑ समु॒द्राय॑ समु॒द्राय॒ नम॑ श्छा॒याया᳚म् छा॒याया॒म् नमः॑ समु॒द्राय॑ । \newline
16. नमः॑ समु॒द्राय॑ समु॒द्राय॒ नमो॒ नमः॑ समु॒द्राय॒ नमो॒ नमः॑ समु॒द्राय॒ नमो॒ नमः॑ समु॒द्राय॒ नमः॑ । \newline
17. स॒मु॒द्राय॒ नमो॒ नमः॑ समु॒द्राय॑ समु॒द्राय॒ नमः॑ समु॒द्रस्य॑ समु॒द्रस्य॒ नमः॑ समु॒द्राय॑ समु॒द्राय॒ नमः॑ समु॒द्रस्य॑ । \newline
18. नमः॑ समु॒द्रस्य॑ समु॒द्रस्य॒ नमो॒ नमः॑ समु॒द्रस्य॒ चक्ष॑से॒ चक्ष॑से समु॒द्रस्य॒ नमो॒ 
नमः॑ समु॒द्रस्य॒ चक्ष॑से । \newline
19. स॒मु॒द्रस्य॒ चक्ष॑से॒ चक्ष॑से समु॒द्रस्य॑ समु॒द्रस्य॒ चक्ष॑से परमे॒ष्ठी प॑रमे॒ष्ठी चक्ष॑से समु॒द्रस्य॑ समु॒द्रस्य॒ चक्ष॑से परमे॒ष्ठी । \newline
20. चक्ष॑से परमे॒ष्ठी प॑रमे॒ष्ठी चक्ष॑से॒ चक्ष॑से परमे॒ष्ठी त्वा᳚ त्वा परमे॒ष्ठी चक्ष॑से॒ चक्ष॑से परमे॒ष्ठी त्वा᳚ । \newline
21. प॒र॒मे॒ष्ठी त्वा᳚ त्वा परमे॒ष्ठी प॑रमे॒ष्ठी त्वा॑ सादयतु सादयतु त्वा परमे॒ष्ठी प॑रमे॒ष्ठी त्वा॑ सादयतु । \newline
22. त्वा॒ सा॒द॒य॒तु॒ सा॒द॒य॒तु॒ त्वा॒ त्वा॒ सा॒द॒य॒तु॒ दि॒वो दि॒वः सा॑दयतु त्वा त्वा सादयतु दि॒वः । \newline
23. सा॒द॒य॒तु॒ दि॒वो दि॒वः सा॑दयतु सादयतु दि॒वः पृ॒ष्ठे पृ॒ष्ठे दि॒वः सा॑दयतु सादयतु दि॒वः पृ॒ष्ठे । \newline
24. दि॒वः पृ॒ष्ठे पृ॒ष्ठे दि॒वो दि॒वः पृ॒ष्ठे व्यच॑स्वतीं॒ ॅव्यच॑स्वतीम् पृ॒ष्ठे दि॒वो दि॒वः पृ॒ष्ठे व्यच॑स्वतीम् । \newline
25. पृ॒ष्ठे व्यच॑स्वतीं॒ ॅव्यच॑स्वतीम् पृ॒ष्ठे पृ॒ष्ठे व्यच॑स्वती॒म् प्रथ॑स्वती॒म् प्रथ॑स्वतीं॒ ॅव्यच॑स्वतीम् पृ॒ष्ठे पृ॒ष्ठे व्यच॑स्वती॒म् प्रथ॑स्वतीम् । \newline
26. व्यच॑स्वती॒म् प्रथ॑स्वती॒म् प्रथ॑स्वतीं॒ ॅव्यच॑स्वतीं॒ ॅव्यच॑स्वती॒म् प्रथ॑स्वतीं ॅवि॒भूम॑तीं ॅवि॒भूम॑ती॒म् प्रथ॑स्वतीं॒ ॅव्यच॑स्वतीं॒ ॅव्यच॑स्वती॒म् प्रथ॑स्वतीं ॅवि॒भूम॑तीम् । \newline
27. प्रथ॑स्वतीं ॅवि॒भूम॑तीं ॅवि॒भूम॑ती॒म् प्रथ॑स्वती॒म् प्रथ॑स्वतीं ॅवि॒भूम॑तीम् प्र॒भूम॑तीम् प्र॒भूम॑तीं ॅवि॒भूम॑ती॒म् प्रथ॑स्वती॒म् प्रथ॑स्वतीं ॅवि॒भूम॑तीम् प्र॒भूम॑तीम् । \newline
28. वि॒भूम॑तीम् प्र॒भूम॑तीम् प्र॒भूम॑तीं ॅवि॒भूम॑तीं ॅवि॒भूम॑तीम् प्र॒भूम॑तीम् परि॒भूम॑तीम् परि॒भूम॑तीम् प्र॒भूम॑तीं ॅवि॒भूम॑तीं ॅवि॒भूम॑तीम् प्र॒भूम॑तीम् परि॒भूम॑तीम् । \newline
29. वि॒भूम॑ती॒मिति॑ वि - भूम॑तीम् । \newline
30. प्र॒भूम॑तीम् परि॒भूम॑तीम् परि॒भूम॑तीम् प्र॒भूम॑तीम् प्र॒भूम॑तीम् परि॒भूम॑ती॒म् दिव॒म् दिव॑म् परि॒भूम॑तीम् प्र॒भूम॑तीम् प्र॒भूम॑तीम् परि॒भूम॑ती॒म् दिव᳚म् । \newline
31. प्र॒भूम॑ती॒मिति॑ प्र - भूम॑तीम् । \newline
32. प॒रि॒भूम॑ती॒म् दिव॒म् दिव॑म् परि॒भूम॑तीम् परि॒भूम॑ती॒म् दिवं॑ ॅयच्छ यच्छ॒ दिव॑म् परि॒भूम॑तीम् परि॒भूम॑ती॒म् दिवं॑ ॅयच्छ । \newline
33. प॒रि॒भूम॑ती॒मिति॑ परि - भूम॑तीम् । \newline
34. दिवं॑ ॅयच्छ यच्छ॒ दिव॒म् दिवं॑ ॅयच्छ॒ दिव॒म् दिवं॑ ॅयच्छ॒ दिव॒म् दिवं॑ ॅयच्छ॒ दिव᳚म् । \newline
35. य॒च्छ॒ दिव॒म् दिवं॑ ॅयच्छ यच्छ॒ दिव॑म् दृꣳह दृꣳह॒ दिवं॑ ॅयच्छ यच्छ॒ दिव॑म् दृꣳह । \newline
36. दिव॑म् दृꣳह दृꣳह॒ दिव॒म् दिव॑म् दृꣳह॒ दिव॒म् दिव॑म् दृꣳह॒ दिव॒म् दिव॑म् दृꣳह॒ दिव᳚म् । \newline
37. दृꣳ॒॒ह॒ दिव॒म् दिव॑म् दृꣳह दृꣳह॒ दिव॒म् मा मा दिव॑म् दृꣳह दृꣳह॒ दिव॒म् मा । \newline
38. दिव॒म् मा मा दिव॒म् दिव॒म् मा हिꣳ॑सीर्. हिꣳसी॒र् मा दिव॒म् दिव॒म् मा हिꣳ॑सीः । \newline
39. मा हिꣳ॑सीर्. हिꣳसी॒र् मा मा हिꣳ॑सी॒र् विश्व॑स्मै॒ विश्व॑स्मै हिꣳसी॒र् मा मा हिꣳ॑सी॒र् विश्व॑स्मै । \newline
40. हिꣳ॒॒सी॒र् विश्व॑स्मै॒ विश्व॑स्मै हिꣳसीर्. हिꣳसी॒र् विश्व॑स्मै प्रा॒णाय॑ प्रा॒णाय॒ विश्व॑स्मै हिꣳसीर्. हिꣳसी॒र् विश्व॑स्मै प्रा॒णाय॑ । \newline
41. विश्व॑स्मै प्रा॒णाय॑ प्रा॒णाय॒ विश्व॑स्मै॒ विश्व॑स्मै प्रा॒णाया॑ पा॒नाया॑ पा॒नाय॑ प्रा॒णाय॒ विश्व॑स्मै॒ विश्व॑स्मै प्रा॒णाया॑ पा॒नाय॑ । \newline
42. प्रा॒णाया॑ पा॒नाया॑ पा॒नाय॑ प्रा॒णाय॑ प्रा॒णाया॑ पा॒नाय॑ व्या॒नाय॑ व्या॒नाया॑ पा॒नाय॑ प्रा॒णाय॑ प्रा॒णाया॑ पा॒नाय॑ व्या॒नाय॑ । \newline
43. प्रा॒णायेति॑ प्र - अ॒नाय॑ । \newline
44. अ॒पा॒नाय॑ व्या॒नाय॑ व्या॒नाया॑ पा॒नाया॑ पा॒नाय॑ व्या॒ना यो॑दा॒ना यो॑दा॒नाय॑ व्या॒नाया॑ पा॒नाया॑ पा॒नाय॑ व्या॒ना यो॑दा॒नाय॑ । \newline
45. अ॒पा॒नायेत्य॑प - अ॒नाय॑ । \newline
46. व्या॒ना यो॑दा॒ना यो॑दा॒नाय॑ व्या॒नाय॑ व्या॒ना यो॑दा॒नाय॑ प्रति॒ष्ठायै᳚ प्रति॒ष्ठाया॑ उदा॒नाय॑ व्या॒नाय॑ व्या॒ना यो॑दा॒नाय॑ प्रति॒ष्ठायै᳚ । \newline
47. व्या॒नायेति॑ वि - अ॒नाय॑ । \newline
48. उ॒दा॒नाय॑ प्रति॒ष्ठायै᳚ प्रति॒ष्ठाया॑ उदा॒ना यो॑दा॒नाय॑ प्रति॒ष्ठायै॑ च॒रित्रा॑य च॒रित्रा॑य प्रति॒ष्ठाया॑ उदा॒ना यो॑दा॒नाय॑ प्रति॒ष्ठायै॑ च॒रित्रा॑य । \newline
49. उ॒दा॒नायेत्यु॑त् - अ॒नाय॑ । \newline
50. प्र॒ति॒ष्ठायै॑ च॒रित्रा॑य च॒रित्रा॑य प्रति॒ष्ठायै᳚ प्रति॒ष्ठायै॑ च॒रित्रा॑य॒ सूर्यः॒ सूर्य॑ श्च॒रित्रा॑य प्रति॒ष्ठायै᳚ प्रति॒ष्ठायै॑ च॒रित्रा॑य॒ सूर्यः॑ । \newline
51. प्र॒ति॒ष्ठाया॒ इति॑ प्रति - स्थायै᳚ । \newline
52. च॒रित्रा॑य॒ सूर्यः॒ सूर्य॑ श्च॒रित्रा॑य च॒रित्रा॑य॒ सूर्य॑ स्त्वा त्वा॒ सूर्य॑ श्च॒रित्रा॑य च॒रित्रा॑य॒ सूर्य॑ स्त्वा । \newline
53. सूर्य॑ स्त्वा त्वा॒ सूर्यः॒ सूर्य॑ स्त्वा॒ ऽभ्य॑भि त्वा॒ सूर्यः॒ सूर्य॑ स्त्वा॒ ऽभि । \newline
54. त्वा॒ ऽभ्य॑भि त्वा᳚ त्वा॒ ऽभि पा॑तु पात्व॒भि त्वा᳚ त्वा॒ ऽभि पा॑तु । \newline
55. अ॒भि पा॑तु पात्व॒ भ्य॑भि पा॑तु म॒ह्या म॒ह्या पा᳚त्व॒ भ्य॑भि पा॑तु म॒ह्या । \newline
56. पा॒तु॒ म॒ह्या म॒ह्या पा॑तु पातु म॒ह्या स्व॒स्त्या स्व॒स्त्या म॒ह्या पा॑तु पातु म॒ह्या स्व॒स्त्या । \newline
57. म॒ह्या स्व॒स्त्या स्व॒स्त्या म॒ह्या म॒ह्या स्व॒स्त्या छ॒र्दिषा॑ छ॒र्दिषा᳚ स्व॒स्त्या म॒ह्या म॒ह्या स्व॒स्त्या छ॒र्दिषा᳚ । \newline
58. स्व॒स्त्या छ॒र्दिषा॑ छ॒र्दिषा᳚ स्व॒स्त्या स्व॒स्त्या छ॒र्दिषा॒ शन्त॑मेन॒ शन्त॑मेन छ॒र्दिषा᳚ स्व॒स्त्या स्व॒स्त्या छ॒र्दिषा॒ शन्त॑मेन । \newline
59. छ॒र्दिषा॒ शन्त॑मेन॒ शन्त॑मेन छ॒र्दिषा॑ छ॒र्दिषा॒ शन्त॑मेन॒ तया॒ तया॒ शन्त॑मेन छ॒र्दिषा॑ छ॒र्दिषा॒ शन्त॑मेन॒ तया᳚ । \newline
60. शन्त॑मेन॒ तया॒ तया॒ शन्त॑मेन॒ शन्त॑मेन॒ तया॑ दे॒वत॑या दे॒वत॑या॒ तया॒ शन्त॑मेन॒ शन्त॑मेन॒ तया॑ दे॒वत॑या । \newline
61. शन्त॑मे॒नेति॒ शं - त॒मे॒न॒ । \newline
62. तया॑ दे॒वत॑या दे॒वत॑या॒ तया॒ तया॑ दे॒वत॑या ऽङ्गिर॒स्व द॑ङ्गिर॒स्वद् दे॒वत॑या॒ तया॒ तया॑ दे॒वत॑या ऽङ्गिर॒स्वत् । \newline
63. दे॒वत॑या ऽङ्गिर॒स्व द॑ङ्गिर॒स्वद् दे॒वत॑या दे॒वत॑या ऽङ्गिर॒स्वद् ध्रु॒वा ध्रु॒वा ऽङ्गि॑र॒स्वद् दे॒वत॑या दे॒वत॑या ऽङ्गिर॒स्वद् ध्रु॒वा । \newline
64. अ॒ङ्गि॒र॒स्वद् ध्रु॒वा ध्रु॒वा ऽङ्गि॑र॒स्व द॑ङ्गिर॒स्वद् ध्रु॒वा सी॑द सीद ध्रु॒वा ऽङ्गि॑र॒स्व द॑ङ्गिर॒स्वद् ध्रु॒वा सी॑द । \newline
65. ध्रु॒वा सी॑द सीद ध्रु॒वा ध्रु॒वा सी॑द । \newline
66. सी॒देति॑ सीद । \newline
67. प्रोथ॒ दश्वो ऽश्वः॒ प्रोथ॒त् प्रोथ॒ दश्वो॒ न नाश्वः॒ प्रोथ॒त् प्रोथ॒ दश्वो॒ न । \newline
68. अश्वो॒ न नाश्वो ऽश्वो॒ न यव॑से॒ यव॑से॒ नाश्वो ऽश्वो॒ न यव॑से । \newline
69. न यव॑से॒ यव॑से॒ न न यव॑से अवि॒ष्यन् न॑वि॒ष्यन्. यव॑से॒ न न यव॑से अवि॒ष्यन्न् । \newline
70. यव॑से अवि॒ष्यन् न॑वि॒ष्यन्. यव॑से॒ यव॑से अवि॒ष्यन्. य॒दा य॒दा ऽवि॒ष्यन्. यव॑से॒ यव॑से अवि॒ष्यन्. य॒दा । \newline
71. अ॒वि॒ष्यन्. य॒दा य॒दा ऽवि॒ष्यन् न॑वि॒ष्यन्. य॒दा म॒हो म॒हो य॒दा ऽवि॒ष्यन् न॑वि॒ष्यन्. य॒दा म॒हः । \newline
72. य॒दा म॒हो म॒हो य॒दा य॒दा म॒हः सं॒ॅवर॑णाथ् सं॒ॅवर॑णान् म॒हो य॒दा य॒दा म॒हः सं॒ॅवर॑णात् । \newline
73. म॒हः सं॒ॅवर॑णाथ् सं॒ॅवर॑णान् म॒हो म॒हः सं॒ॅवर॑णा॒द् व्यस्था॒द् व्यस्था᳚थ् सं॒ॅवर॑णान् म॒हो म॒हः सं॒ॅवर॑णा॒द् व्यस्था᳚त् । \newline
74. सं॒ॅवर॑णा॒द् व्यस्था॒द् व्यस्था᳚थ् सं॒ॅवर॑णाथ् सं॒ॅवर॑णा॒द् व्यस्था᳚त् । \newline
75. सं॒ॅवर॑णा॒दिति॑ सं - वर॑णात् । \newline
76. व्यस्था॒दिति॑ वि - अस्था᳚त् । \newline
77. आद॑स्या॒ स्यादा द॑स्य॒ वातो॒ वातो॒ ऽस्यादा द॑स्य॒ वातः॑ । \newline
78. अ॒स्य॒ वातो॒ वातो᳚ ऽस्यास्य॒ वातो॒ अन्वनु॒ वातो᳚ ऽस्यास्य॒ वातो॒ अनु॑ । \newline
79. वातो॒ अन्वनु॒ वातो॒ वातो॒ अनु॑ वाति वा॒त्यनु॒ वातो॒ वातो॒ अनु॑ वाति । \newline
80. अनु॑ वाति वा॒त्यन् वनु॑ वाति शो॒चिः शो॒चिर् वा॒त्यन् वनु॑ वाति शो॒चिः । \newline
81. वा॒ति॒ शो॒चिः शो॒चिर् वा॑ति वाति शो॒चि रधाध॑ शो॒चिर् वा॑ति वाति शो॒चि रध॑ । \newline
82. शो॒चि रधाध॑ शो॒चिः शो॒चि रध॑ स्म॒ स्माध॑ शो॒चिः शो॒चि रध॑ स्म । \newline
83. अध॑ स्म॒ स्मा धाध॑ स्म ते ते॒ स्मा धाध॑ स्म ते । \newline
84. स्म॒ ते॒ ते॒ स्म॒ स्म॒ ते॒ व्रज॑नं॒ ॅव्रज॑नम् ते स्म स्म ते॒ व्रज॑नम् । \newline
85. ते॒ व्रज॑नं॒ ॅव्रज॑नम् ते ते॒ व्रज॑नम् कृ॒ष्णम् कृ॒ष्णं ॅव्रज॑नम् ते ते॒ व्रज॑नम् कृ॒ष्णम् । \newline
86. व्रज॑नम् कृ॒ष्णम् कृ॒ष्णं ॅव्रज॑नं॒ ॅव्रज॑नम् कृ॒ष्ण म॑स्त्यस्ति कृ॒ष्णं ॅव्रज॑नं॒ ॅव्रज॑नम् कृ॒ष्ण म॑स्ति । \newline
87. कृ॒ष्ण म॑स्त्यस्ति कृ॒ष्णम् कृ॒ष्ण म॑स्ति । \newline
88. अ॒स्तीत्य॑स्ति । \newline
\pagebreak
\markright{ TS 4.4.4.1  \hfill https://www.vedavms.in \hfill}

\section{ TS 4.4.4.1 }

\textbf{TS 4.4.4.1 } \newline
\textbf{Samhita Paata} \newline

अ॒ग्निर्मू॒र्द्धा दि॒वः क॒कुत् पतिः॑ पृथि॒व्या अ॒यं । अ॒पाꣳ रेताꣳ॑सि जिन्वति ॥ त्वाम॑ग्ने॒ पुष्क॑रा॒दद्ध्यथ॑र्वा॒ निर॑मन्थत । मू॒र्द्ध्नो विश्व॑स्य वा॒घतः॑ ॥ अ॒यम॒ग्निः स॑ह॒स्रिणो॒ वाज॑स्य श॒तिन॒स्पतिः॑ । मू॒र्द्धा क॒वी र॑यी॒णां ॥ भुवो॑ य॒ज्ञ्स्य॒ रज॑सश्च ने॒ता यत्रा॑ नि॒युद्भिः॒ सच॑से शि॒वाभिः॑ । दि॒वि मू॒र्द्धानं॑ दधिषे सुव॒र्॒.षां जि॒ह्वाम॑ग्ने चकृषे हव्य॒वाहं᳚ ॥ अबो᳚द्ध्य॒ग्निः स॒मिधा॒ जना॑नां॒ - [  ] \newline

\textbf{Pada Paata} \newline

अ॒ग्निः । मू॒द्‌र्धा । दि॒वः । क॒कुत् । पतिः॑ । पृ॒थि॒व्याः । अ॒यम् ॥ अ॒पाम् । रेताꣳ॑सि । जि॒न्व॒ति॒ ॥ त्वाम् । अ॒ग्ने॒ । पुष्क॑रात् । अधीति॑ । अथ॑र्वा । निरिति॑ । अ॒म॒न्थ॒त॒ ॥ मू॒द्‌र्ध्नः । विश्व॑स्य । वा॒घतः॑ ॥ अ॒यम् । अ॒ग्निः । स॒ह॒स्रिणः॑ । वाज॑स्य । श॒तिनः॑ । पतिः॑ ॥ मू॒द्‌र्धा । क॒विः । र॒यी॒णाम् ॥ भुवः॑ । य॒ज्ञ्स्य॑ । रज॑सः । च॒ । ने॒ता । यत्र॑ । नि॒युद्भि॒रिति॑ नि॒युत् - भिः॒ । सच॑से । शि॒वाभिः॑ ॥ दि॒वि । मू॒द्‌र्धान᳚म् । द॒धि॒षे॒ । सु॒व॒र्॒.षामिति॑ सुवः - साम् । जि॒ह्वाम् । अ॒ग्ने॒ । च॒कृ॒षे॒ । ह॒व्य॒वाह॒मिति॑ हव्य - वाह᳚म् ॥ अबा॑धि । अ॒ग्निः । स॒मिधेति॑ सम् - इधा᳚ । जना॑नाम् ।  \newline


\textbf{Krama Paata} \newline

अ॒ग्निर् मू॒र्द्धा । मू॒र्द्धा दि॒वः । दि॒वः क॒कुत् । क॒कुत् पतिः॑ । पतिः॑ पृथि॒व्याः । पृ॒थि॒व्या अ॒यम् । अ॒यमित्य॒यम् ॥ अ॒पाꣳ रेताꣳ॑सि । रेताꣳ॑सि जिन्वति । जि॒न्व॒तीति॑ जिन्वति ॥ त्वाम॑ग्ने । अ॒ग्ने॒ पुष्क॑रात् । पुष्क॑रा॒दधि॑ । अध्यथ॑र्वा । अथ॑र्वा॒ निः । निर॑मन्थत । अ॒म॒न्थ॒तेत्य॑मन्थत ॥ मू॒र्द्ध्नो विश्व॑स्य । विश्व॑स्य वा॒घतः॑ । वा॒घत॒ इति॑ वा॒घतः॑ ॥ अ॒यम॒ग्निः । अ॒ग्निः स॑ह॒स्रिणः॑ । स॒ह॒स्रिणो॒ वाज॑स्य । वाज॑स्य श॒तिनः॑ । श॒तिन॒स्पतिः॑ । पति॒रिति॒ पतिः॑ ॥ मू॒र्द्धा क॒विः । क॒वीर॑यी॒णाम् । र॒यी॒णामिति॑ रयी॒णाम् ॥ भुवो॑ य॒ज्ञ्स्य॑ । य॒ज्ञ्स्य॒ रज॑सः । रज॑सश्च । च॒ ने॒ता । ने॒ता यत्र॑ । यत्रा॑ नि॒युद्भिः॑ । नि॒युद्भिः॒ सच॑से । नि॒युद्भि॒रिति॑ नि॒युत् - भिः॒ । सच॑से शि॒वाभिः॑ । शि॒वाभि॒रिति॑ शि॒वाभिः॑ ॥ दि॒वि मू॒र्द्धान᳚म् । मू॒र्द्धान॑म् दधिषे । द॒धि॒षे॒ सु॒व॒र्.॒षाम् । सु॒व॒र्.॒षाम् जि॒ह्वाम् । सु॒व॒र्.॒षामिति॑ सुवः - साम् । जि॒ह्वाम॑ग्ने । अ॒ग्ने॒ च॒कृ॒षे॒ । च॒कृ॒षे॒ ह॒व्य॒वाह᳚म् । ह॒व्य॒वाह॒मिति॑ हव्य - वाह᳚म् ॥ अबो᳚द्ध्य॒ग्निः । अ॒ग्निः स॒मिधा᳚ । स॒मिधा॒ जना॑नाम् । स॒मिधेति॑ सम् - इधा᳚ । जना॑ना॒म् प्रति॑ \newline

\textbf{Jatai Paata} \newline

1. अ॒ग्निर् मू॒र्द्धा मू॒र्द्धा ऽग्नि र॒ग्निर् मू॒र्द्धा । \newline
2. मू॒र्द्धा दि॒वो दि॒वो मू॒र्द्धा मू॒र्द्धा दि॒वः । \newline
3. दि॒वः क॒कुत् क॒कुद् दि॒वो दि॒वः क॒कुत् । \newline
4. क॒कुत् पति॒ष् पतिः॑ क॒कुत् क॒कुत् पतिः॑ । \newline
5. पतिः॑ पृथि॒व्याः पृ॑थि॒व्या स्पति॒ष् पतिः॑ पृथि॒व्याः । \newline
6. पृ॒थि॒व्या अ॒य म॒यम् पृ॑थि॒व्याः पृ॑थि॒व्या अ॒यम् । \newline
7. अ॒यमित्य॒यम् । \newline
8. अ॒पाꣳ रेताꣳ॑सि॒ रेताꣳ॑ स्य॒पा म॒पाꣳ रेताꣳ॑सि । \newline
9. रेताꣳ॑सि जिन्वति जिन्वति॒ रेताꣳ॑सि॒ रेताꣳ॑सि जिन्वति । \newline
10. जि॒न्व॒तीति॑ जिन्वति । \newline
11. त्वा म॑ग्ने अग्ने॒ त्वाम् त्वा म॑ग्ने । \newline
12. अ॒ग्ने॒ पुष्क॑रा॒त् पुष्क॑रा दग्ने अग्ने॒ पुष्क॑रात् । \newline
13. पुष्क॑रा॒ दध्यधि॒ पुष्क॑रा॒त् पुष्क॑रा॒ दधि॑ । \newline
14. अध्य थ॒र्वा ऽथ॒र्वा ऽध्यध्य थ॑र्वा । \newline
15. अथ॑र्वा॒ निर् णि रथ॒र्वा ऽथ॑र्वा॒ निः । \newline
16. निर॑मन्थता मन्थत॒ निर् णिर॑मन्थत । \newline
17. अ॒म॒न्थ॒तेत्य॑मन्थत । \newline
18. मू॒र्द्ध्नो विश्व॑स्य॒ विश्व॑स्य मू॒र्द्ध्नो मू॒र्द्ध्नो विश्व॑स्य । \newline
19. विश्व॑स्य वा॒घतो॑ वा॒घतो॒ विश्व॑स्य॒ विश्व॑स्य वा॒घतः॑ । \newline
20. वा॒घत॒ इति॑ वा॒घतः॑ । \newline
21. अ॒य म॒ग्नि र॒ग्नि र॒य म॒य म॒ग्निः । \newline
22. अ॒ग्निः स॑ह॒स्रिणः॑ सह॒स्रिणो॑ अ॒ग्नि र॒ग्निः स॑ह॒स्रिणः॑ । \newline
23. स॒ह॒स्रिणो॒ वाज॑स्य॒ वाज॑स्य सह॒स्रिणः॑ सह॒स्रिणो॒ वाज॑स्य । \newline
24. वाज॑स्य श॒तिनः॑ श॒तिनो॒ वाज॑स्य॒ वाज॑स्य श॒तिनः॑ । \newline
25. श॒तिन॒ स्पति॒ष् पतिः॑ श॒तिनः॑ श॒तिन॒ स्पतिः॑ । \newline
26. पति॒रिति॒ पतिः॑ । \newline
27. मू॒र्द्धा क॒विः क॒विर् मू॒र्द्धा मू॒र्द्धा क॒विः । \newline
28. क॒वी र॑यी॒णाꣳ र॑यी॒णाम् क॒विः क॒वी र॑यी॒णाम् । \newline
29. र॒यी॒णामिति॑ रयी॒णाम् । \newline
30. भुवो॑ य॒ज्ञ्स्य॑ य॒ज्ञ्स्य॒ भुवो॒ भुवो॑ य॒ज्ञ्स्य॑ । \newline
31. य॒ज्ञ्स्य॒ रज॑सो॒ रज॑सो य॒ज्ञ्स्य॑ य॒ज्ञ्स्य॒ रज॑सः । \newline
32. रज॑सश्च च॒ रज॑सो॒ रज॑सश्च । \newline
33. च॒ ने॒ता ने॒ता च॑ च ने॒ता । \newline
34. ने॒ता यत्र॒ यत्र॑ ने॒ता ने॒ता यत्र॑ । \newline
35. यत्रा॑ नि॒युद्भि॑र् नि॒युद्भि॒र् यत्र॒ यत्रा॑ नि॒युद्भिः॑ । \newline
36. नि॒युद्भिः॒ सच॑से॒ सच॑से नि॒युद्भि॑र् नि॒युद्भिः॒ सच॑से । \newline
37. नि॒युद्भि॒रिति॑ नि॒युत् - भिः॒ । \newline
38. सच॑से शि॒वाभिः॑ शि॒वाभिः॒ सच॑से॒ सच॑से शि॒वाभिः॑ । \newline
39. शि॒वाभि॒रिति॑ शि॒वाभिः॑ । \newline
40. दि॒वि मू॒र्द्धान॑म् मू॒र्द्धान॑म् दि॒वि दि॒वि मू॒र्द्धान᳚म् । \newline
41. मू॒र्द्धान॑म् दधिषे दधिषे मू॒र्द्धान॑म् मू॒र्द्धान॑म् दधिषे । \newline
42. द॒धि॒षे॒ सु॒व॒र्॒.षाꣳ सु॑व॒र्॒.षाम् द॑धिषे दधिषे सुव॒र्॒.षाम् । \newline
43. सु॒व॒र्॒.षाम् जि॒ह्वाम् जि॒ह्वाꣳ सु॑व॒र्॒.षाꣳ सु॑व॒र्॒.षाम् जि॒ह्वाम् । \newline
44. सु॒व॒र्॒.षामिति॑ सुवः - साम् । \newline
45. जि॒ह्वा म॑ग्ने अग्ने जि॒ह्वाम् जि॒ह्वा म॑ग्ने । \newline
46. अ॒ग्ने॒ च॒कृ॒षे॒ च॒कृ॒षे॒ अ॒ग्ने॒ अ॒ग्ने॒ च॒कृ॒षे॒ । \newline
47. च॒कृ॒षे॒ ह॒व्य॒वाहꣳ॑ हव्य॒वाह॑म् चकृषे चकृषे हव्य॒वाह᳚म् । \newline
48. ह॒व्य॒वाह॒मिति॑ हव्य - वाह᳚म् । \newline
49. अबो᳚ध्य॒ ग्नि र॒ग्नि रबो॒ध्य बो᳚ध्य॒ ग्निः । \newline
50. अ॒ग्निः स॒मिधा॑ स॒मिधा॒ ऽग्नि र॒ग्निः स॒मिधा᳚ । \newline
51. स॒मिधा॒ जना॑ना॒म् जना॑नाꣳ स॒मिधा॑ स॒मिधा॒ जना॑नाम् । \newline
52. स॒मिधेति॑ सम् - इधा᳚ । \newline
53. जना॑ना॒म् प्रति॒ प्रति॒ जना॑ना॒म् जना॑ना॒म् प्रति॑ । \newline

\textbf{Ghana Paata } \newline

1. अ॒ग्निर् मू॒र्द्धा मू॒र्द्धा ऽग्नि र॒ग्निर् मू॒र्द्धा दि॒वो दि॒वो मू॒र्द्धा ऽग्नि र॒ग्निर् मू॒र्द्धा दि॒वः । \newline
2. मू॒र्द्धा दि॒वो दि॒वो मू॒र्द्धा मू॒र्द्धा दि॒वः क॒कुत् क॒कुद् दि॒वो मू॒र्द्धा मू॒र्द्धा दि॒वः क॒कुत् । \newline
3. दि॒वः क॒कुत् क॒कुद् दि॒वो दि॒वः क॒कुत् पति॒ष् पतिः॑ क॒कुद् दि॒वो दि॒वः क॒कुत् पतिः॑ । \newline
4. क॒कुत् पति॒ष् पतिः॑ क॒कुत् क॒कुत् पतिः॑ पृथि॒व्याः पृ॑थि॒व्या स्पतिः॑ क॒कुत् क॒कुत् पतिः॑ पृथि॒व्याः । \newline
5. पतिः॑ पृथि॒व्याः पृ॑थि॒व्या स्पति॒ष् पतिः॑ पृथि॒व्या अ॒य म॒यम् पृ॑थि॒व्या स्पति॒ष् पतिः॑ पृथि॒व्या अ॒यम् । \newline
6. पृ॒थि॒व्या अ॒य म॒यम् पृ॑थि॒व्याः पृ॑थि॒व्या अ॒यम् । \newline
7. अ॒यमित्य॒यम् । \newline
8. अ॒पाꣳ रेताꣳ॑सि॒ रेताꣳ॑ स्य॒पा म॒पाꣳ रेताꣳ॑सि जिन्वति जिन्वति॒ रेताꣳ॑ स्य॒पा म॒पाꣳ रेताꣳ॑सि जिन्वति । \newline
9. रेताꣳ॑सि जिन्वति जिन्वति॒ रेताꣳ॑सि॒ रेताꣳ॑सि जिन्वति । \newline
10. जि॒न्व॒तीति॑ जिन्वति । \newline
11. त्वा म॑ग्ने अग्ने॒ त्वाम् त्वा म॑ग्ने॒ पुष्क॑रा॒त् पुष्क॑रा दग्ने॒ त्वाम् त्वा म॑ग्ने॒ पुष्क॑रात् । \newline
12. अ॒ग्ने॒ पुष्क॑रा॒त् पुष्क॑रा दग्ने अग्ने॒ पुष्क॑रा॒ दध्यधि॒ पुष्क॑रा दग्ने अग्ने॒ पुष्क॑रा॒ दधि॑ । \newline
13. पुष्क॑रा॒ दध्यधि॒ पुष्क॑रा॒त् पुष्क॑रा॒ दध्य थ॒र्वा ऽथ॒र्वा ऽधि॒ पुष्क॑रा॒त् पुष्क॑रा॒द् अध्यथ॑र्वा । \newline
14. अध्यथ॒र्वा ऽथ॒र्वा ऽध्यध्य थ॑र्वा॒ निर् णिरथ॒र्वा ऽध्यध्य थ॑र्वा॒ निः । \newline
15. अथ॑र्वा॒ निर् णिरथ॒र्वा ऽथ॑र्वा॒ निर॑मन्थता मन्थत॒ निरथ॒र्वा ऽथ॑र्वा॒ निर॑मन्थत । \newline
16. निर॑मन्थता मन्थत॒ निर् णिर॑मन्थत । \newline
17. अ॒म॒न्थ॒तेत्य॑मन्थत । \newline
18. मू॒र्द्ध्नो विश्व॑स्य॒ विश्व॑स्य मू॒र्द्ध्नो मू॒र्द्ध्नो विश्व॑स्य वा॒घतो॑ वा॒घतो॒ विश्व॑स्य मू॒र्द्ध्नो मू॒र्द्ध्नो विश्व॑स्य वा॒घतः॑ । \newline
19. विश्व॑स्य वा॒घतो॑ वा॒घतो॒ विश्व॑स्य॒ विश्व॑स्य वा॒घतः॑ । \newline
20. वा॒घत॒ इति॑ वा॒घतः॑ । \newline
21. अ॒य म॒ग्नि र॒ग्नि र॒य म॒य म॒ग्निः स॑ह॒स्रिणः॑ सह॒स्रिणो॑ अ॒ग्नि र॒य म॒य म॒ग्निः स॑ह॒स्रिणः॑ । \newline
22. अ॒ग्निः स॑ह॒स्रिणः॑ सह॒स्रिणो॑ अ॒ग्नि र॒ग्निः स॑ह॒स्रिणो॒ वाज॑स्य॒ वाज॑स्य सह॒स्रिणो॑ अ॒ग्नि र॒ग्निः स॑ह॒स्रिणो॒ वाज॑स्य । \newline
23. स॒ह॒स्रिणो॒ वाज॑स्य॒ वाज॑स्य सह॒स्रिणः॑ सह॒स्रिणो॒ वाज॑स्य श॒तिनः॑ श॒तिनो॒ वाज॑स्य सह॒स्रिणः॑ सह॒स्रिणो॒ वाज॑स्य श॒तिनः॑ । \newline
24. वाज॑स्य श॒तिनः॑ श॒तिनो॒ वाज॑स्य॒ वाज॑स्य श॒तिन॒ स्पति॒ष् पतिः॑ श॒तिनो॒ वाज॑स्य॒ वाज॑स्य श॒तिन॒ स्पतिः॑ । \newline
25. श॒तिन॒ स्पति॒ष् पतिः॑ श॒तिनः॑ श॒तिन॒ स्पतिः॑ । \newline
26. पति॒रिति॒ पतिः॑ । \newline
27. मू॒र्द्धा क॒विः क॒विर् मू॒र्द्धा मू॒र्द्धा क॒वी र॑यी॒णाꣳ र॑यी॒णाम् क॒विर् मू॒र्द्धा मू॒र्द्धा क॒वी र॑यी॒णाम् । \newline
28. क॒वी र॑यी॒णाꣳ र॑यी॒णाम् क॒विः क॒वी र॑यी॒णाम् । \newline
29. र॒यी॒णामिति॑ रयी॒णाम् । \newline
30. भुवो॑ य॒ज्ञ्स्य॑ य॒ज्ञ्स्य॒ भुवो॒ भुवो॑ य॒ज्ञ्स्य॒ रज॑सो॒ रज॑सो य॒ज्ञ्स्य॒ भुवो॒ भुवो॑ य॒ज्ञ्स्य॒ रज॑सः । \newline
31. य॒ज्ञ्स्य॒ रज॑सो॒ रज॑सो य॒ज्ञ्स्य॑ य॒ज्ञ्स्य॒ रज॑सश्च च॒ रज॑सो य॒ज्ञ्स्य॑ य॒ज्ञ्स्य॒ रज॑सश्च । \newline
32. रज॑सश्च च॒ रज॑सो॒ रज॑सश्च ने॒ता ने॒ता च॒ रज॑सो॒ रज॑सश्च ने॒ता । \newline
33. च॒ ने॒ता ने॒ता च॑ च ने॒ता यत्र॒ यत्र॑ ने॒ता च॑ च ने॒ता यत्र॑ । \newline
34. ने॒ता यत्र॒ यत्र॑ ने॒ता ने॒ता यत्रा॑ नि॒युद्भि॑र् नि॒युद्भि॒र् यत्र॑ ने॒ता ने॒ता यत्रा॑ नि॒युद्भिः॑ । \newline
35. यत्रा॑ नि॒युद्भि॑र् नि॒युद्भि॒र् यत्र॒ यत्रा॑ नि॒युद्भिः॒ सच॑से॒ सच॑से नि॒युद्भि॒र् यत्र॒ यत्रा॑ नि॒युद्भिः॒ सच॑से । \newline
36. नि॒युद्भिः॒ सच॑से॒ सच॑से नि॒युद्भि॑र् नि॒युद्भिः॒ सच॑से शि॒वाभिः॑ शि॒वाभिः॒ सच॑से नि॒युद्भि॑र् नि॒युद्भिः॒ सच॑से शि॒वाभिः॑ । \newline
37. नि॒युद्भि॒रिति॑ नि॒युत् - भिः॒ । \newline
38. सच॑से शि॒वाभिः॑ शि॒वाभिः॒ सच॑से॒ सच॑से शि॒वाभिः॑ । \newline
39. शि॒वाभि॒रिति॑ शि॒वाभिः॑ । \newline
40. दि॒वि मू॒र्द्धान॑म् मू॒र्द्धान॑म् दि॒वि दि॒वि मू॒र्द्धान॑म् दधिषे दधिषे मू॒र्द्धान॑म् दि॒वि दि॒वि मू॒र्द्धान॑म् दधिषे । \newline
41. मू॒र्द्धान॑म् दधिषे दधिषे मू॒र्द्धान॑म् मू॒र्द्धान॑म् दधिषे सुव॒र्॒.षाꣳ सु॑व॒र्॒.षाम् द॑धिषे मू॒र्द्धान॑म् मू॒र्द्धान॑म् दधिषे सुव॒र्॒.षाम् । \newline
42. द॒धि॒षे॒ सु॒व॒र्॒.षाꣳ सु॑व॒र्॒.षाम् द॑धिषे दधिषे सुव॒र्॒.षाम् जि॒ह्वाम् जि॒ह्वाꣳ सु॑व॒र्॒.षाम् द॑धिषे दधिषे सुव॒र्॒.षाम् जि॒ह्वाम् । \newline
43. सु॒व॒र्॒.षाम् जि॒ह्वाम् जि॒ह्वाꣳ सु॑व॒र्॒.षाꣳ सु॑व॒र्॒.षाम् जि॒ह्वा म॑ग्ने अग्ने जि॒ह्वाꣳ सु॑व॒र्॒.षाꣳ सु॑व॒र्॒.षाम् जि॒ह्वा म॑ग्ने । \newline
44. सु॒व॒र्॒.षामिति॑ सुवः - साम् । \newline
45. जि॒ह्वा म॑ग्ने अग्ने जि॒ह्वाम् जि॒ह्वा म॑ग्ने चकृषे चकृषे अग्ने जि॒ह्वाम् जि॒ह्वा म॑ग्ने चकृषे । \newline
46. अ॒ग्ने॒ च॒कृ॒षे॒ च॒कृ॒षे॒ अ॒ग्ने॒ अ॒ग्ने॒ च॒कृ॒षे॒ ह॒व्य॒वाहꣳ॑ हव्य॒वाह॑म् 
चकृषे अग्ने अग्ने चकृषे हव्य॒वाह᳚म् । \newline
47. च॒कृ॒षे॒ ह॒व्य॒वाहꣳ॑ हव्य॒वाह॑म् चकृषे चकृषे हव्य॒वाह᳚म् । \newline
48. ह॒व्य॒वाह॒मिति॑ हव्य - वाह᳚म् । \newline
49. अबो᳚ध्य॒ ग्निर॒ग्नि रबो॒ध्य बो᳚ध्य॒ ग्निः स॒मिधा॑ स॒मिधा॒ ऽग्नि रबो॒ध्य बो᳚ध्य॒ग्निः स॒मिधा᳚ । \newline
50. अ॒ग्निः स॒मिधा॑ स॒मिधा॒ ऽग्नि र॒ग्निः स॒मिधा॒ जना॑ना॒म् जना॑नाꣳ स॒मिधा॒ ऽग्नि र॒ग्निः स॒मिधा॒ जना॑नाम् । \newline
51. स॒मिधा॒ जना॑ना॒म् जना॑नाꣳ स॒मिधा॑ स॒मिधा॒ जना॑ना॒म् प्रति॒ प्रति॒ जना॑नाꣳ स॒मिधा॑ स॒मिधा॒ जना॑ना॒म् प्रति॑ । \newline
52. स॒मिधेति॑ सम् - इधा᳚ । \newline
53. जना॑ना॒म् प्रति॒ प्रति॒ जना॑ना॒म् जना॑ना॒म् प्रति॑ धे॒नुम् धे॒नुम् प्रति॒ जना॑ना॒म् जना॑ना॒म् प्रति॑ धे॒नुम् । \newline
\pagebreak
\markright{ TS 4.4.4.2  \hfill https://www.vedavms.in \hfill}

\section{ TS 4.4.4.2 }

\textbf{TS 4.4.4.2 } \newline
\textbf{Samhita Paata} \newline

प्रति॑ धे॒नुमि॑वा य॒तीमु॒षासं᳚ । य॒ह्वा इ॑व॒ प्रव॒या मु॒ज्जिहा॑नाः॒ प्र भा॒नवः॑ सिस्रते॒ नाक॒मच्छ॑ ॥ अवो॑चाम क॒वये॒ मेद्ध्या॑य॒ वचो॑ व॒न्दारु॑ वृष॒भाय॒ वृष्णे᳚ । गवि॑ष्ठिरो॒ नम॑सा॒ स्तोम॑म॒ग्नौ दि॒वीव॑ रु॒क्ममु॒र्व्यञ्च॑मश्रेत् ॥ जन॑स्य गो॒पा अ॑जनिष्ट॒ जागृ॑विर॒ग्निः सु॒दक्षः॑ सुवि॒ताय॒ नव्य॑से । घृ॒तप्र॑तीको बृह॒ता दि॑वि॒स्पृशा᳚ द्यु॒मद्वि भा॑ति भर॒तेभ्यः॒ शुचिः॑ ॥ त्वाम॑ग्ने॒ अङ्गि॑रसो॒ - [  ] \newline

\textbf{Pada Paata} \newline

प्रतीति॑ । धे॒नुम् । इ॒व । आ॒य॒तीमित्या᳚ - य॒तीम् । उ॒षास᳚म् ॥ य॒ह्वाः । इ॒व॒ । प्रेति॑ । व॒याम् । उ॒ज्जिहा॑ना॒ इत्यु॑त् - जिहा॑नाः । प्रेति॑ । भा॒नवः॑ । सि॒स्र॒ते॒ । नाक᳚म् । अच्छ॑ ॥ अवो॑चाम । क॒वये᳚ । मेद्ध्या॑य । वचः॑ । व॒न्दारु॑ । वृ॒ष॒भाय॑ । वृष्णे᳚ ॥ गवि॑ष्ठिरः । नम॑सा । स्तोम᳚म् । अ॒ग्नौ । दि॒वि । इ॒व॒ । रु॒क्मम् । उ॒र्व्यञ्च᳚म् । अ॒श्रे॒त् ॥ जन॑स्य । गो॒पा इति॑ गो - पाः । अ॒ज॒नि॒ष्ट॒। जागृ॑विः । अ॒ग्निः । सु॒दक्ष॒ इति॑ सु - दक्षः॑ । सु॒वि॒ताय॑ । नव्य॑से ॥ घृ॒तप्र॑तीक॒ इति॑ घृ॒त - प्र॒ती॒कः॒ । बृ॒ह॒ता । दि॒वि॒स्पृशेति॑ दिवि - स्पृशा᳚ । द्यु॒मदिति॑ द्यु - मत् । वीति॑ । भा॒ति॒ । भ॒र॒तेभ्यः॑ । शुचिः॑ ॥ त्वाम् । अ॒ग्ने॒ । अङ्गि॑रसः ।  \newline


\textbf{Krama Paata} \newline

प्रति॑ धे॒नुम् । धे॒नुमि॑व । इ॒वा॒य॒तीम् । आ॒य॒तीमु॒षास᳚म् । आ॒य॒तीमित्या᳚ - य॒तीम् । उ॒षास॒मित्यु॒षास᳚म् ॥ य॒ह्वा इ॑व । इ॒व॒ प्र । प्र व॒याम् । व॒यामु॒ज्जिहा॑नाः । उ॒ज्जिहा॑नाः॒ प्र । उ॒ज्जिहा॑ना॒ इत्यु॑त् - जिहा॑नाः । प्र भा॒नवः॑ । भा॒नवः॑ सिस्रते । सि॒स्र॒ते॒ नाक᳚म् । नाक॒मच्छ॑ । अच्छेत्यच्छ॑ ॥ अवो॑चाम क॒वये᳚ । क॒वये॒ मेद्ध्या॑य । मेद्ध्या॑य॒ वचः॑ । वचो॑ व॒न्दारु॑ । व॒न्दारु॑ वृष॒भाय॑ । वृ॒ष॒भाय॒ वृष्णे᳚ । वृष्ण॒ इति॒ वृष्णे᳚ ॥ गवि॑ष्ठिरो॒ नम॑सा । नम॑सा॒ स्तोम᳚म् । स्तोम॑म॒ग्नौ । अ॒ग्नौ दि॒वि । दि॒वीव॑ । इ॒व॒ रु॒क्मम् । रु॒क्ममु॒र्व्यञ्च᳚म् । उ॒र्व्यञ्च॑मश्रेत् । अ॒श्रे॒दित्य॑श्रेत् ॥ जन॑स्य गो॒पाः । गो॒पा अ॑जनिष्ट । गो॒पा इति॑ गो - पाः । अ॒ज॒नि॒ष्ट॒ जागृ॑विः । जागृ॑विर॒ग्निः । अ॒ग्निः सु॒दक्षः॑ । सु॒दक्षः॑ सुवि॒ताय॑ । सु॒दक्ष॒ इति॑ सु - दक्षः॑ । सु॒वि॒ताय॒ नव्य॑से । नव्य॑स॒ इति॒ नव्य॑से ॥ घृ॒तप्र॑तीको बृह॒ता । घृ॒तप्र॑तीक॒ इति॑ घृ॒त - प्र॒ती॒कः॒ । बृ॒ह॒ता दि॑वि॒स्पृशा᳚ । दि॒वि॒स्पृशा᳚ द्यु॒मत् । दि॒वि॒स्पृशेति॑ दिवि - स्पृशा᳚ । द्यु॒मद् वि । द्यु॒मदिति॑ द्यु - मत् । वि भा॑ति । भा॒ति॒ भ॒र॒तेभ्यः॑ । भ॒र॒तेभ्यः॒ शुचिः॑ । शुचि॒रिति॒ शुचिः॑ ॥ त्वाम॑ग्ने । अ॒ग्ने॒ अङ्गि॑रसः । अङ्गि॑रसो॒ गुहा᳚ \newline

\textbf{Jatai Paata} \newline

1. प्रति॑ धे॒नुम् धे॒नुम् प्रति॒ प्रति॑ धे॒नुम् । \newline
2. धे॒नु मि॒वेव धे॒नुम् धे॒नु मि॒व । \newline
3. इ॒वाय॒ती मा॑य॒ती मि॒वेवा य॒तीम् । \newline
4. आ॒य॒ती मु॒षास॑ मु॒षास॑ माय॒ती मा॑य॒ती मु॒षास᳚म् । \newline
5. आ॒य॒तीमित्या᳚ - य॒तीम् । \newline
6. उ॒षास॒मित्यु॒षास᳚म् । \newline
7. य॒ह्वा इ॑वेव य॒ह्वा य॒ह्वा इ॑व । \newline
8. इ॒व॒ प्र प्रेवे॑व॒ प्र । \newline
9. प्र व॒यां ॅव॒याम् प्र प्र व॒याम् । \newline
10. व॒या मु॒ज्जिहा॑ना उ॒ज्जिहा॑ना व॒यां ॅव॒या मु॒ज्जिहा॑नाः । \newline
11. उ॒ज्जिहा॑नाः॒ प्र प्रोज्जिहा॑ना उ॒ज्जिहा॑नाः॒ प्र । \newline
12. उ॒ज्जिहा॑ना॒ इत्यु॑त् - जिहा॑नाः । \newline
13. प्र भा॒नवो॑ भा॒नवः॒ प्र प्र भा॒नवः॑ । \newline
14. भा॒नवः॑ सिस्रते सिस्रते भा॒नवो॑ भा॒नवः॑ सिस्रते । \newline
15. सि॒स्र॒ते॒ नाक॒म् नाकꣳ॑ सिस्रते सिस्रते॒ नाक᳚म् । \newline
16. नाक॒ मच्छाच्छ॒ नाक॒म् नाक॒ मच्छ॑ । \newline
17. अच्छेत्यच्छ॑ । \newline
18. अवो॑चाम क॒वये॑ क॒वये ऽवो॑चा॒मा वो॑चाम क॒वये᳚ । \newline
19. क॒वये॒ मेद्ध्या॑य॒ मेद्ध्या॑य क॒वये॑ क॒वये॒ मेद्ध्या॑य । \newline
20. मेद्ध्या॑य॒ वचो॒ वचो॒ मेद्ध्या॑य॒ मेद्ध्या॑य॒ वचः॑ । \newline
21. वचो॑ व॒न्दारु॑ व॒न्दारु॒ वचो॒ वचो॑ व॒न्दारु॑ । \newline
22. व॒न्दारु॑ वृष॒भाय॑ वृष॒भाय॑ व॒न्दारु॑ व॒न्दारु॑ वृष॒भाय॑ । \newline
23. वृ॒ष॒भाय॒ वृष्णे॒ वृष्णे॑ वृष॒भाय॑ वृष॒भाय॒ वृष्णे᳚ । \newline
24. वृष्ण॒ इति॒ वृष्णे᳚ । \newline
25. गवि॑ष्ठिरो॒ नम॑सा॒ नम॑सा॒ गवि॑ष्ठिरो॒ गवि॑ष्ठिरो॒ नम॑सा । \newline
26. नम॑सा॒ स्तोमꣳ॒॒ स्तोम॒म् नम॑सा॒ नम॑सा॒ स्तोम᳚म् । \newline
27. स्तोम॑ म॒ग्ना व॒ग्नौ स्तोमꣳ॒॒ स्तोम॑ म॒ग्नौ । \newline
28. अ॒ग्नौ दि॒वि दि॒व्य॑ग्ना व॒ग्नौ दि॒वि । \newline
29. दि॒वीवे॑व दि॒वि दि॒वीव॑ । \newline
30. इ॒व॒ रु॒क्मꣳ रु॒क्म मि॑वेव रु॒क्मम् । \newline
31. रु॒क्म मु॒र्व्यञ्च॑ मु॒र्व्यञ्चꣳ॑ रु॒क्मꣳ रु॒क्म मु॒र्व्यञ्च᳚म् । \newline
32. उ॒र्व्यञ्च॑ मश्रे दश्रे दु॒र्व्यञ्च॑ मु॒र्व्यञ्च॑ मश्रेत् । \newline
33. अ॒श्रे॒दित्य॑श्रेत् । \newline
34. जन॑स्य गो॒पा गो॒पा जन॑स्य॒ जन॑स्य गो॒पाः । \newline
35. गो॒पा अ॑जनिष्टा जनिष्ट गो॒पा गो॒पा अ॑जनिष्ट । \newline
36. गो॒पा इति॑ गो - पाः । \newline
37. अ॒ज॒नि॒ष्ट॒ जागृ॑वि॒र् जागृ॑वि रजनिष्टा जनिष्ट॒ जागृ॑विः । \newline
38. जागृ॑वि र॒ग्नि र॒ग्निर् जागृ॑वि॒र् जागृ॑वि र॒ग्निः । \newline
39. अ॒ग्निः सु॒दक्षः॑ सु॒दक्षो॑ अ॒ग्नि र॒ग्निः सु॒दक्षः॑ । \newline
40. सु॒दक्षः॑ सुवि॒ताय॑ सुवि॒ताय॑ सु॒दक्षः॑ सु॒दक्षः॑ सुवि॒ताय॑ । \newline
41. सु॒दक्ष॒ इति॑ सु - दक्षः॑ । \newline
42. सु॒वि॒ताय॒ नव्य॑से॒ नव्य॑से सुवि॒ताय॑ सुवि॒ताय॒ नव्य॑से । \newline
43. नव्य॑स॒ इति॒ नव्य॑से । \newline
44. घृ॒तप्र॑तीको बृह॒ता बृ॑ह॒ता घृ॒तप्र॑तीको घृ॒तप्र॑तीको बृह॒ता । \newline
45. घृ॒तप्र॑तीक॒ इति॑ घृ॒त - प्र॒ती॒कः॒ । \newline
46. बृ॒ह॒ता दि॑वि॒स्पृशा॑ दिवि॒स्पृशा॑ बृह॒ता बृ॑ह॒ता दि॑वि॒स्पृशा᳚ । \newline
47. दि॒वि॒स्पृशा᳚ द्यु॒मद् द्यु॒मद् दि॑वि॒स्पृशा॑ दिवि॒स्पृशा᳚ द्यु॒मत् । \newline
48. दि॒वि॒स्पृशेति॑ दिवि - स्पृशा᳚ । \newline
49. द्यु॒मद् वि वि द्यु॒मद् द्यु॒मद् वि । \newline
50. द्यु॒मदिति॑ द्यु - मत् । \newline
51. वि भा॑ति भाति॒ वि वि भा॑ति । \newline
52. भा॒ति॒ भ॒र॒तेभ्यो॑ भर॒तेभ्यो॑ भाति भाति भर॒तेभ्यः॑ । \newline
53. भ॒र॒तेभ्यः॒ शुचिः॒ शुचि॑र् भर॒तेभ्यो॑ भर॒तेभ्यः॒ शुचिः॑ । \newline
54. शुचि॒रिति॒ शुचिः॑ । \newline
55. त्वा म॑ग्ने अग्ने॒ त्वाम् त्वा म॑ग्ने । \newline
56. अ॒ग्ने॒ अङ्गि॑रसो॒ अङ्गि॑रसो अग्ने अग्ने॒ अङ्गि॑रसः । \newline
57. अङ्गि॑रसो॒ गुहा॒ गुहा ऽङ्गि॑रसो॒ अङ्गि॑रसो॒ गुहा᳚ । \newline

\textbf{Ghana Paata } \newline

1. प्रति॑ धे॒नुम् धे॒नुम् प्रति॒ प्रति॑ धे॒नु मि॒वेव धे॒नुम् प्रति॒ प्रति॑ धे॒नु मि॒व । \newline
2. धे॒नु मि॒वेव धे॒नुम् धे॒नु मि॒वाय॒ती मा॑य॒ती मि॒व धे॒नुम् धे॒नु मि॒वाय॒तीम् । \newline
3. इ॒वाय॒ती मा॑य॒ती मि॒वे वाय॒ती मु॒षास॑ मु॒षास॑ माय॒ती मि॒वे वाय॒ती मु॒षास᳚म् । \newline
4. आ॒य॒ती मु॒षास॑ मु॒षास॑ माय॒ती मा॑य॒ती मु॒षास᳚म् । \newline
5. आ॒य॒तीमित्या᳚ - य॒तीम् । \newline
6. उ॒षास॒मित्यु॒षास᳚म् । \newline
7. य॒ह्वा इ॑वेव य॒ह्वा य॒ह्वा इ॑व॒ प्र प्रेव॑ य॒ह्वा य॒ह्वा इ॑व॒ प्र । \newline
8. इ॒व॒ प्र प्रेवे॑व॒ प्र व॒यां ॅव॒याम् प्रेवे॑व॒ प्र व॒याम् । \newline
9. प्र व॒यां ॅव॒याम् प्र प्र व॒या मु॒ज्जिहा॑ना उ॒ज्जिहा॑ना व॒याम् प्र प्र व॒या मु॒ज्जिहा॑नाः । \newline
10. व॒या मु॒ज्जिहा॑ना उ॒ज्जिहा॑ना व॒यां ॅव॒या मु॒ज्जिहा॑नाः॒ प्र प्रोज्जिहा॑ना व॒यां ॅव॒या मु॒ज्जिहा॑नाः॒ प्र । \newline
11. उ॒ज्जिहा॑नाः॒ प्र प्रोज्जिहा॑ना उ॒ज्जिहा॑नाः॒ प्र भा॒नवो॑ भा॒नवः॒ प्रोज्जिहा॑ना उ॒ज्जिहा॑नाः॒ प्र भा॒नवः॑ । \newline
12. उ॒ज्जिहा॑ना॒ इत्यु॑त् - जिहा॑नाः । \newline
13. प्र भा॒नवो॑ भा॒नवः॒ प्र प्र भा॒नवः॑ सिस्रते सिस्रते भा॒नवः॒ प्र प्र भा॒नवः॑ सिस्रते । \newline
14. भा॒नवः॑ सिस्रते सिस्रते भा॒नवो॑ भा॒नवः॑ सिस्रते॒ नाक॒म् नाकꣳ॑ सिस्रते भा॒नवो॑ भा॒नवः॑ सिस्रते॒ नाक᳚म् । \newline
15. सि॒स्र॒ते॒ नाक॒म् नाकꣳ॑ सिस्रते सिस्रते॒ नाक॒ मच्छाच्छ॒ नाकꣳ॑ सिस्रते सिस्रते॒ नाक॒ मच्छ॑ । \newline
16. नाक॒ मच्छाच्छ॒ नाक॒म् नाक॒ मच्छ॑ । \newline
17. अच्छेत्यच्छ॑ । \newline
18. अवो॑चाम क॒वये॑ क॒वये ऽवो॑चा॒मा वो॑चाम क॒वये॒ मेद्ध्या॑य॒ मेद्ध्या॑य क॒वये ऽवो॑चा॒मा वो॑चाम क॒वये॒ मेद्ध्या॑य । \newline
19. क॒वये॒ मेद्ध्या॑य॒ मेद्ध्या॑य क॒वये॑ क॒वये॒ मेद्ध्या॑य॒ वचो॒ वचो॒ मेद्ध्या॑य क॒वये॑ क॒वये॒ मेद्ध्या॑य॒ वचः॑ । \newline
20. मेद्ध्या॑य॒ वचो॒ वचो॒ मेद्ध्या॑य॒ मेद्ध्या॑य॒ वचो॑ व॒न्दारु॑ व॒न्दारु॒ वचो॒ मेद्ध्या॑य॒ मेद्ध्या॑य॒ वचो॑ व॒न्दारु॑ । \newline
21. वचो॑ व॒न्दारु॑ व॒न्दारु॒ वचो॒ वचो॑ व॒न्दारु॑ वृष॒भाय॑ वृष॒भाय॑ व॒न्दारु॒ वचो॒ वचो॑ 
व॒न्दारु॑ वृष॒भाय॑ । \newline
22. व॒न्दारु॑ वृष॒भाय॑ वृष॒भाय॑ व॒न्दारु॑ व॒न्दारु॑ वृष॒भाय॒ वृष्णे॒ वृष्णे॑ वृष॒भाय॑ व॒न्दारु॑ व॒न्दारु॑ वृष॒भाय॒ वृष्णे᳚ । \newline
23. वृ॒ष॒भाय॒ वृष्णे॒ वृष्णे॑ वृष॒भाय॑ वृष॒भाय॒ वृष्णे᳚ । \newline
24. वृष्ण॒ इति॒ वृष्णे᳚ । \newline
25. गवि॑ष्ठिरो॒ नम॑सा॒ नम॑सा॒ गवि॑ष्ठिरो॒ गवि॑ष्ठिरो॒ नम॑सा॒ स्तोमꣳ॒॒ स्तोम॒म् नम॑सा॒ गवि॑ष्ठिरो॒ गवि॑ष्ठिरो॒ नम॑सा॒ स्तोम᳚म् । \newline
26. नम॑सा॒ स्तोमꣳ॒॒ स्तोम॒म् नम॑सा॒ नम॑सा॒ स्तोम॑ म॒ग्ना व॒ग्नौ स्तोम॒म् नम॑सा॒ नम॑सा॒ स्तोम॑ म॒ग्नौ । \newline
27. स्तोम॑ म॒ग्ना व॒ग्नौ स्तोमꣳ॒॒ स्तोम॑ म॒ग्नौ दि॒वि दि॒व्य॑ग्नौ स्तोमꣳ॒॒ स्तोम॑ म॒ग्नौ दि॒वि । \newline
28. अ॒ग्नौ दि॒वि दि॒व्य॑ग्ना व॒ग्नौ दि॒वीवे॑व दि॒व्य॑ग्ना व॒ग्नौ दि॒वीव॑ । \newline
29. दि॒वीवे॑व दि॒वि दि॒वीव॑ रु॒क्मꣳ रु॒क्म मि॑व दि॒वि दि॒वीव॑ रु॒क्मम् । \newline
30. इ॒व॒ रु॒क्मꣳ रु॒क्म मि॑वेव रु॒क्म मु॒र्व्यञ्च॑ मु॒र्व्यञ्चꣳ॑ रु॒क्म मि॑वेव रु॒क्म मु॒र्व्यञ्च᳚म् । \newline
31. रु॒क्म मु॒र्व्यञ्च॑ मु॒र्व्यञ्चꣳ॑ रु॒क्मꣳ रु॒क्म मु॒र्व्यञ्च॑ मश्रे दश्रे दु॒र्व्यञ्चꣳ॑ रु॒क्मꣳ रु॒क्म मु॒र्व्यञ्च॑ मश्रेत् । \newline
32. उ॒र्व्यञ्च॑ मश्रे दश्रे दु॒र्व्यञ्च॑ मु॒र्व्यञ्च॑ मश्रेत् । \newline
33. अ॒श्रे॒दित्य॑श्रेत् । \newline
34. जन॑स्य गो॒पा गो॒पा जन॑स्य॒ जन॑स्य गो॒पा अ॑जनिष्टा जनिष्ट गो॒पा जन॑स्य॒ जन॑स्य गो॒पा अ॑जनिष्ट । \newline
35. गो॒पा अ॑जनिष्टा जनिष्ट गो॒पा गो॒पा अ॑जनिष्ट॒ जागृ॑वि॒र् जागृ॑वि रजनिष्ट गो॒पा गो॒पा अ॑जनिष्ट॒ जागृ॑विः । \newline
36. गो॒पा इति॑ गो - पाः । \newline
37. अ॒ज॒नि॒ष्ट॒ जागृ॑वि॒र् जागृ॑वि रजनिष्टा जनिष्ट॒ जागृ॑वि र॒ग्निर॒ग्निर् जागृ॑वि रजनिष्टा जनिष्ट॒ जागृ॑वि र॒ग्निः । \newline
38. जागृ॑वि र॒ग्नि र॒ग्निर् जागृ॑वि॒र् जागृ॑वि र॒ग्निः सु॒दक्षः॑ सु॒दक्षो॑ अ॒ग्निर् जागृ॑वि॒र् जागृ॑वि र॒ग्निः सु॒दक्षः॑ । \newline
39. अ॒ग्निः सु॒दक्षः॑ सु॒दक्षो॑ अ॒ग्नि र॒ग्निः सु॒दक्षः॑ सुवि॒ताय॑ सुवि॒ताय॑ सु॒दक्षो॑ अ॒ग्नि र॒ग्निः सु॒दक्षः॑ सुवि॒ताय॑ । \newline
40. सु॒दक्षः॑ सुवि॒ताय॑ सुवि॒ताय॑ सु॒दक्षः॑ सु॒दक्षः॑ सुवि॒ताय॒ नव्य॑से॒ नव्य॑से सुवि॒ताय॑ सु॒दक्षः॑ सु॒दक्षः॑ सुवि॒ताय॒ नव्य॑से । \newline
41. सु॒दक्ष॒ इति॑ सु - दक्षः॑ । \newline
42. सु॒वि॒ताय॒ नव्य॑से॒ नव्य॑से सुवि॒ताय॑ सुवि॒ताय॒ नव्य॑से । \newline
43. नव्य॑स इति॒ नव्य॑से । \newline
44. घृ॒तप्र॑तीको बृह॒ता बृ॑ह॒ता घृ॒तप्र॑तीको घृ॒तप्र॑तीको बृह॒ता दि॑वि॒स्पृशा॑ दिवि॒स्पृशा॑ बृह॒ता घृ॒तप्र॑तीको घृ॒तप्र॑तीको बृह॒ता दि॑वि॒स्पृशा᳚ । \newline
45. घृ॒तप्र॑तीक॒ इति॑ घृ॒त - प्र॒ती॒कः॒ । \newline
46. बृ॒ह॒ता दि॑वि॒स्पृशा॑ दिवि॒स्पृशा॑ बृह॒ता बृ॑ह॒ता दि॑वि॒स्पृशा᳚ द्यु॒मद् द्यु॒मद् दि॑वि॒स्पृशा॑ बृह॒ता बृ॑ह॒ता दि॑वि॒स्पृशा᳚ द्यु॒मत् । \newline
47. दि॒वि॒स्पृशा᳚ द्यु॒मद् द्यु॒मद् दि॑वि॒स्पृशा॑ दिवि॒स्पृशा᳚ द्यु॒मद् वि वि द्यु॒मद् दि॑वि॒स्पृशा॑ दिवि॒स्पृशा᳚ द्यु॒मद् वि । \newline
48. दि॒वि॒स्पृशेति॑ दिवि - स्पृशा᳚ । \newline
49. द्यु॒मद् वि वि द्यु॒मद् द्यु॒मद् वि भा॑ति भाति॒ वि द्यु॒मद् द्यु॒मद् वि भा॑ति । \newline
50. द्यु॒मदिति॑ द्यु - मत् । \newline
51. वि भा॑ति भाति॒ वि वि भा॑ति भर॒तेभ्यो॑ भर॒तेभ्यो॑ भाति॒ वि वि भा॑ति भर॒तेभ्यः॑ । \newline
52. भा॒ति॒ भ॒र॒तेभ्यो॑ भर॒तेभ्यो॑ भाति भाति भर॒तेभ्यः॒ शुचिः॒ शुचि॑र् भर॒तेभ्यो॑ भाति भाति भर॒तेभ्यः॒ शुचिः॑ । \newline
53. भ॒र॒तेभ्यः॒ शुचिः॒ शुचि॑र् भर॒तेभ्यो॑ भर॒तेभ्यः॒ शुचिः॑ । \newline
54. शुचि॒रिति॒ शुचिः॑ । \newline
55. त्वा म॑ग्ने अग्ने॒ त्वाम् त्वा म॑ग्ने॒ अङ्गि॑रसो॒ अङ्गि॑रसो अग्ने॒ त्वाम् त्वा म॑ग्ने॒ अङ्गि॑रसः । \newline
56. अ॒ग्ने॒ अङ्गि॑रसो॒ अङ्गि॑रसो अग्ने अग्ने॒ अङ्गि॑रसो॒ गुहा॒ गुहा ऽङ्गि॑रसो अग्ने अग्ने॒ अङ्गि॑रसो॒ गुहा᳚ । \newline
57. अङ्गि॑रसो॒ गुहा॒ गुहा ऽङ्गि॑रसो॒ अङ्गि॑रसो॒ गुहा॑ हि॒तꣳ हि॒तम् गुहा ऽङ्गि॑रसो॒ अङ्गि॑रसो॒ गुहा॑ हि॒तम् । \newline
\pagebreak
\markright{ TS 4.4.4.3  \hfill https://www.vedavms.in \hfill}

\section{ TS 4.4.4.3 }

\textbf{TS 4.4.4.3 } \newline
\textbf{Samhita Paata} \newline

गुहा॑ हि॒तमन्व॑-विन्दञ्छिश्रिया॒णं ॅवने॑वने । स जा॑यसे म॒थ्यमा॑नः॒ सहो॑ म॒हत् त्वामा॑हुः॒ सह॑सस्पु॒त्रम॑ङ्गिरः ॥ य॒ज्ञ्स्य॑ के॒तुं प्र॑थ॒मं पु॒रोहि॑तम॒ग्निं नर॑स्त्रिषध॒स्थे समि॑न्धते । इन्द्रे॑ण दे॒वैः स॒रथꣳ॒॒ स ब॒र्॒.हिषि॒ सीद॒न्नि होता॑ य॒जथा॑य सु॒क्रतुः॑ ॥ त्वां चि॑त्रश्रवस्तम॒ हव॑न्ते वि॒क्षु ज॒न्तवः॑ । शो॒चिष्के॑शं पुरुप्रि॒याग्ने॑ ह॒व्याय॒ वोढ॑वे ॥ सखा॑यः॒ संॅवः॑ स॒म्यञ्च॒मिषꣳ॒॒ - [  ] \newline

\textbf{Pada Paata} \newline

गुहा᳚ । हि॒तम् । अन्विति॑ । अ॒वि॒न्द॒न्न् । शि॒श्रि॒या॒णम् । वने॑वन॒ इति॒ वने᳚ - व॒ने॒ ॥ सः । जा॒य॒से॒ । म॒थ्यमा॑नः । सहः॑ । म॒हत् । त्वाम् । आ॒हुः॒ । सह॑सः । पु॒त्रम् । अ॒ङ्गि॒रः॒ ॥ य॒ज्ञ्स्य॑ । के॒तुम् । प्र॒थ॒मम् । पु॒रोहि॑त॒मिति॑ पु॒रः - हि॒त॒म् । अ॒ग्निम् । नरः॑ । त्रि॒ष॒ध॒स्थ इति॑ त्रि - स॒ध॒स्थे । समिति॑ । इ॒न्ध॒ते॒ ॥ इन्द्रे॑ण । दे॒वैः । स॒रथ॒मिति॑ स - रथ᳚म् । सः । ब॒र्॒.हिषि॑ । सीद॑त् । नीति॑ । होता᳚ । य॒जथा॑य । सु॒क्रतु॒रिति॑ सु - क्रतुः॑ ॥ त्वाम् । चि॒त्र॒श्र॒व॒स्त॒मेति॑ चित्रश्रवः-त॒म॒ । हव॑न्ते । वि॒क्षु । ज॒न्तवः॑ ॥ शो॒चिष्के॑श॒मिति॑ शो॒चिः - के॒श॒म् । पु॒रु॒प्रि॒येति॑ पुरु - प्रि॒य॒ । अग्ने᳚ । ह॒व्याय॑ । वोढ॑वे ॥ सखा॑यः । समिति॑ । वः॒ । स॒म्यञ्च᳚म् । इष᳚म् ।  \newline


\textbf{Krama Paata} \newline

गुहा॑ हि॒तम् । हि॒तमनु॑ । अन्व॑विन्दन्न् । अ॒वि॒न्द॒ञ्छि॒श्रि॒या॒णम् । शि॒श्रि॒या॒णम् ॅवने॑वने । वने॑वन॒ इति॒ वने᳚ - व॒ने॒ ॥ स जा॑यसे । जा॒य॒से॒ म॒थ्यमा॑नः । म॒थ्यमा॑नः॒ सहः॑ । सहो॑ म॒हत् । म॒हत् त्वाम् । त्वामा॑हुः । आ॒हुः॒ सह॑सः । सह॑सस्पु॒त्रम् । पु॒त्रम॑ङ्गिरः । अ॒ङ्गि॒र॒ इत्य॑ङ्गिरः ॥ य॒ज्ञ्स्य॑ के॒तुम् । के॒तुम् प्र॑थ॒मम् । प्र॒थ॒मम् पु॒रोहि॑तम् । पु॒रोहि॑तम॒ग्निम् । पु॒रोहि॑त॒मिति॑ पु॒रः - हि॒त॒म् । अ॒ग्निम् नरः॑ । नर॑स्त्रिषध॒स्थे । त्रि॒ष॒ध॒स्थे सम् । त्रि॒ष॒ध॒स्थ इति॑ त्रि - स॒ध॒स्थे । समि॑न्धते । इ॒न्ध॒त॒ इती᳚न्धते ॥ इन्द्रे॑ण दे॒वैः । दे॒वैः स॒रथ᳚म् । स॒रथꣳ॒॒ सः । स॒रथ॒मिति॑ स - रथ᳚म् । स ब॒र्॒.हिषि॑ । ब॒र्॒.हिषि॒ सीद॑त् । सीद॒न्नि । नि होता᳚ । होता॑ य॒जथा॑य । य॒जथा॑य सु॒क्रतुः॑ । सु॒क्रतु॒रिति॑ सु - क्रतुः॑ ॥ त्वाम् चि॑त्रश्रवस्तम । चि॒त्र॒श्र॒व॒स्त॒म॒ हव॑न्ते । चि॒त्र॒श्र॒व॒स्त॒मेति॑ चित्रश्रवः - त॒म॒ । हव॑न्ते वि॒क्षु । वि॒क्षु ज॒न्तवः॑ । ज॒न्तव॒ इति॑ ज॒न्तवः॑ ॥ शो॒चिष्के॑शम् पुरुप्रि॒य । शो॒चिष्के॑श॒मिति॑ शो॒चिः - के॒श॒म् । पु॒रु॒प्रि॒याग्ने᳚ । पु॒रु॒प्रि॒येति॑ पुरु - प्रि॒य॒ । अग्ने॑ ह॒व्याय॑ । ह॒व्याय॒ वोढ॑वे । वोढ॑व॒ इति॒ वोढ॑वे ॥ सखा॑यः॒ सम् । सं ॅवः॑ । वः स॒म्यञ्च᳚म् । स॒म्यञ्च॒मिष᳚म् । इषꣳ॒॒ स्तोम᳚म् \newline

\textbf{Jatai Paata} \newline

1. गुहा॑ हि॒तꣳ हि॒तम् गुहा॒ गुहा॑ हि॒तम् । \newline
2. हि॒त मन्वनु॑ हि॒तꣳ हि॒त मनु॑ । \newline
3. अन्व॑विन्दन् नविन्द॒न् नन्वन्व॑ विन्दन्न् । \newline
4. अ॒वि॒न्द॒ञ् छि॒श्रि॒या॒णꣳ शि॑श्रिया॒ण म॑विन्दन् नविन्दञ् छिश्रिया॒णम् । \newline
5. शि॒श्रि॒या॒णं ॅवने॑वने॒ वने॑वने शिश्रिया॒णꣳ शि॑श्रिया॒णं ॅवने॑वने । \newline
6. वने॑वन॒ इति॒ वने᳚ - व॒ने॒ । \newline
7. स जा॑यसे जायसे॒ स स जा॑यसे । \newline
8. जा॒य॒से॒ म॒थ्यमा॑नो म॒थ्यमा॑नो जायसे जायसे म॒थ्यमा॑नः । \newline
9. म॒थ्यमा॑नः॒ सहः॒ सहो॑ म॒थ्यमा॑नो म॒थ्यमा॑नः॒ सहः॑ । \newline
10. सहो॑ म॒हन् म॒हथ् सहः॒ सहो॑ म॒हत् । \newline
11. म॒हत् त्वाम् त्वाम् म॒हन् म॒हत् त्वाम् । \newline
12. त्वा मा॑हु राहु॒ स्त्वाम् त्वा मा॑हुः । \newline
13. आ॒हुः॒ सह॑सः॒ सह॑स आहु राहुः॒ सह॑सः । \newline
14. सह॑स स्पु॒त्रम् पु॒त्रꣳ सह॑सः॒ सह॑स स्पु॒त्रम् । \newline
15. पु॒त्र म॑ङ्गिरो अङ्गिरः पु॒त्रम् पु॒त्र म॑ङ्गिरः । \newline
16. अ॒ङ्गि॒र॒ इत्य॑ङ्गिरः । \newline
17. य॒ज्ञ्स्य॑ के॒तुम् के॒तुं ॅय॒ज्ञ्स्य॑ य॒ज्ञ्स्य॑ के॒तुम् । \newline
18. के॒तुम् प्र॑थ॒मम् प्र॑थ॒मम् के॒तुम् के॒तुम् प्र॑थ॒मम् । \newline
19. प्र॒थ॒मम् पु॒रोहि॑तम् पु॒रोहि॑तम् प्रथ॒मम् प्र॑थ॒मम् पु॒रोहि॑तम् । \newline
20. पु॒रोहि॑त म॒ग्नि म॒ग्निम् पु॒रोहि॑तम् पु॒रोहि॑त म॒ग्निम् । \newline
21. पु॒रोहि॑त॒मिति॑ पु॒रः - हि॒त॒म् । \newline
22. अ॒ग्निन् नरो॒ नरो॑ अ॒ग्नि म॒ग्निन् नरः॑ । \newline
23. नर॑ स्त्रिषध॒स्थे त्रि॑षध॒स्थे नरो॒ नर॑ स्त्रिषध॒स्थे । \newline
24. त्रि॒ष॒ध॒स्थे सꣳ सम् त्रि॑षध॒स्थे त्रि॑षध॒स्थे सम् । \newline
25. त्रि॒ष॒ध॒स्थ इति॑ त्रि - स॒ध॒स्थे । \newline
26. स मि॑न्धत इन्धते॒ सꣳ स मि॑न्धते । \newline
27. इ॒न्ध॒त॒ इती᳚न्धते । \newline
28. इन्द्रे॑ण दे॒वैर् दे॒वै रिन्द्रे॒ णेन्द्रे॑ण दे॒वैः । \newline
29. दे॒वैः स॒रथꣳ॑ स॒रथ॑म् दे॒वैर् दे॒वैः स॒रथ᳚म् । \newline
30. स॒रथꣳ॒॒ स स स॒रथꣳ॑ स॒रथꣳ॒॒ सः । \newline
31. स॒रथ॒मिति॑ स - रथ᳚म् । \newline
32. स ब॒र्॒.हिषि॑ ब॒र्॒.हिषि॒ स स ब॒र्॒.हिषि॑ । \newline
33. ब॒र्॒.हिषि॒ सीद॒थ् सीद॑द् ब॒र्॒.हिषि॑ ब॒र्॒.हिषि॒ सीद॑त् । \newline
34. सीद॒न् नि नि सीद॒थ् सीद॒न् नि । \newline
35. नि होता॒ होता॒ नि नि होता᳚ । \newline
36. होता॑ य॒जथा॑य य॒जथा॑य॒ होता॒ होता॑ य॒जथा॑य । \newline
37. य॒जथा॑य सु॒क्रतुः॑ सु॒क्रतु॑र् य॒जथा॑य य॒जथा॑य सु॒क्रतुः॑ । \newline
38. सु॒क्रतु॒रिति॑ सु - क्रतुः॑ । \newline
39. त्वाम् चि॑त्रश्रवस्तम चित्रश्रवस्तम॒ त्वाम् त्वाम् चि॑त्रश्रवस्तम । \newline
40. चि॒त्र॒श्र॒व॒स्त॒म॒ हव॑न्ते॒ हव॑न्ते चित्रश्रवस्तम चित्रश्रवस्तम॒ हव॑न्ते । \newline
41. चि॒त्र॒श्र॒व॒स्त॒मेति॑ चित्रश्रवः - त॒म॒ । \newline
42. हव॑न्ते वि॒क्षु वि॒क्षु हव॑न्ते॒ हव॑न्ते वि॒क्षु । \newline
43. वि॒क्षु ज॒न्तवो॑ ज॒न्तवो॑ वि॒क्षु वि॒क्षु ज॒न्तवः॑ । \newline
44. ज॒न्तव॒ इति॑ ज॒न्तवः॑ । \newline
45. शो॒चिष्के॑शम् पुरुप्रिय पुरुप्रिय शो॒चिष्के॑शꣳ शो॒चिष्के॑शम् पुरुप्रिय । \newline
46. शो॒चिष्के॑श॒मिति॑ शो॒चिः - के॒श॒म् । \newline
47. पु॒रु॒प्रि॒ याग्ने ऽग्ने॑ पुरुप्रिय पुरुप्रि॒ याग्ने᳚ । \newline
48. पु॒रु॒प्रि॒येति॑ पुरु - प्रि॒य॒ । \newline
49. अग्ने॑ ह॒व्याय॑ ह॒व्यायाग्ने ऽग्ने॑ ह॒व्याय॑ । \newline
50. ह॒व्याय॒ वोढ॑वे॒ वोढ॑वे ह॒व्याय॑ ह॒व्याय॒ वोढ॑वे । \newline
51. वोढ॑व॒ इति॒ वोढ॑वे । \newline
52. सखा॑यः॒ सꣳ सꣳ सखा॑यः॒ सखा॑यः॒ सम् । \newline
53. सं ॅवो॑ वः॒ सꣳ सं ॅवः॑ । \newline
54. वः॒ स॒म्यञ्चꣳ॑ स॒म्यञ्चं॑ ॅवो वः स॒म्यञ्च᳚म् । \newline
55. स॒म्यञ्च॒ मिष॒ मिषꣳ॑ स॒म्यञ्चꣳ॑ स॒म्यञ्च॒ मिष᳚म् । \newline
56. इषꣳ॒॒ स्तोमꣳ॒॒ स्तोम॒ मिष॒ मिषꣳ॒॒ स्तोम᳚म् । \newline

\textbf{Ghana Paata } \newline

1. गुहा॑ हि॒तꣳ हि॒तम् गुहा॒ गुहा॑ हि॒त मन्वनु॑ हि॒तम् गुहा॒ गुहा॑ हि॒त मनु॑ । \newline
2. हि॒त मन्वनु॑ हि॒तꣳ हि॒त मन्व॑विन्दन् नविन्द॒न् ननु॑ हि॒तꣳ हि॒त मन्व॑विन्दन्न् । \newline
3. अन्व॑विन्दन् नविन्द॒न् नन्वन्व॑विन्दञ् छिश्रिया॒णꣳ शि॑श्रिया॒ण म॑विन्द॒न् नन्वन्व॑विन्दञ् छिश्रिया॒णम् । \newline
4. अ॒वि॒न्द॒ञ् छि॒श्रि॒या॒णꣳ शि॑श्रिया॒ण म॑विन्दन् नविन्दञ् छिश्रिया॒णं ॅवने॑वने॒ वने॑वने शिश्रिया॒ण म॑विन्दन् नविन्दञ् छिश्रिया॒णं ॅवने॑वने । \newline
5. शि॒श्रि॒या॒णं ॅवने॑वने॒ वने॑वने शिश्रिया॒णꣳ शि॑श्रिया॒णं ॅवने॑वने । \newline
6. वने॑वन॒ इति॒ वने᳚ - व॒ने॒ । \newline
7. स जा॑यसे जायसे॒ स स जा॑यसे म॒थ्यमा॑नो म॒थ्यमा॑नो जायसे॒ स स जा॑यसे म॒थ्यमा॑नः । \newline
8. जा॒य॒से॒ म॒थ्यमा॑नो म॒थ्यमा॑नो जायसे जायसे म॒थ्यमा॑नः॒ सहः॒ सहो॑ म॒थ्यमा॑नो जायसे जायसे म॒थ्यमा॑नः॒ सहः॑ । \newline
9. म॒थ्यमा॑नः॒ सहः॒ सहो॑ म॒थ्यमा॑नो म॒थ्यमा॑नः॒ सहो॑ म॒हन् म॒हथ् सहो॑ म॒थ्यमा॑नो म॒थ्यमा॑नः॒ सहो॑ म॒हत् । \newline
10. सहो॑ म॒हन् म॒हथ् सहः॒ सहो॑ म॒हत् त्वाम् त्वाम् म॒हथ् सहः॒ सहो॑ म॒हत् त्वाम् । \newline
11. म॒हत् त्वाम् त्वाम् म॒हन् म॒हत् त्वा मा॑हु राहु॒ स्त्वाम् म॒हन् म॒हत् त्वा मा॑हुः । \newline
12. त्वा मा॑हु राहु॒ स्त्वाम् त्वा मा॑हुः॒ सह॑सः॒ सह॑स आहु॒ स्त्वाम् त्वा मा॑हुः॒ सह॑सः । \newline
13. आ॒हुः॒ सह॑सः॒ सह॑स आहु राहुः॒ सह॑स स्पु॒त्रम् पु॒त्रꣳ सह॑स आहु राहुः॒ सह॑स स्पु॒त्रम् । \newline
14. सह॑स स्पु॒त्रम् पु॒त्रꣳ सह॑सः॒ सह॑स स्पु॒त्र म॑ङ्गिरो अङ्गिरः पु॒त्रꣳ सह॑सः॒ सह॑स स्पु॒त्र म॑ङ्गिरः । \newline
15. पु॒त्र म॑ङ्गिरो अङ्गिरः पु॒त्रम् पु॒त्र म॑ङ्गिरः । \newline
16. अ॒ङ्गि॒र॒ इत्य॑ङ्गिरः । \newline
17. य॒ज्ञ्स्य॑ के॒तुम् के॒तुं ॅय॒ज्ञ्स्य॑ य॒ज्ञ्स्य॑ के॒तुम् प्र॑थ॒मम् प्र॑थ॒मम् के॒तुं ॅय॒ज्ञ्स्य॑ य॒ज्ञ्स्य॑ के॒तुम् प्र॑थ॒मम् । \newline
18. के॒तुम् प्र॑थ॒मम् प्र॑थ॒मम् के॒तुम् के॒तुम् प्र॑थ॒मम् पु॒रोहि॑तम् पु॒रोहि॑तम् प्रथ॒मम् के॒तुम् के॒तुम् प्र॑थ॒मम् पु॒रोहि॑तम् । \newline
19. प्र॒थ॒मम् पु॒रोहि॑तम् पु॒रोहि॑तम् प्रथ॒मम् प्र॑थ॒मम् पु॒रोहि॑त म॒ग्नि म॒ग्निम् पु॒रोहि॑तम् प्रथ॒मम् प्र॑थ॒मम् पु॒रोहि॑त म॒ग्निम् । \newline
20. पु॒रोहि॑त म॒ग्नि म॒ग्निम् पु॒रोहि॑तम् पु॒रोहि॑त म॒ग्निम् नरो॒ नरो॑ अ॒ग्निम् पु॒रोहि॑तम् पु॒रोहि॑त म॒ग्निम् नरः॑ । \newline
21. पु॒रोहि॑त॒मिति॑ पु॒रः - हि॒त॒म् । \newline
22. अ॒ग्निम् नरो॒ नरो॑ अ॒ग्नि म॒ग्निम् नर॑ स्त्रिषध॒स्थे त्रि॑षध॒स्थे नरो॑ अ॒ग्नि म॒ग्निम् नर॑ स्त्रिषध॒स्थे । \newline
23. नर॑ स्त्रिषध॒स्थे त्रि॑षध॒स्थे नरो॒ नर॑ स्त्रिषध॒स्थे सꣳ सम् त्रि॑षध॒स्थे नरो॒ नर॑ स्त्रिषध॒स्थे सम् । \newline
24. त्रि॒ष॒ध॒स्थे सꣳ सम् त्रि॑षध॒स्थे त्रि॑षध॒स्थे स मि॑न्धत इन्धते॒ सम् त्रि॑षध॒स्थे त्रि॑षध॒स्थे स मि॑न्धते । \newline
25. त्रि॒ष॒ध॒स्थ इति॑ त्रि - स॒ध॒स्थे । \newline
26. स मि॑न्धत इन्धते॒ सꣳ स मि॑न्धते । \newline
27. इ॒न्ध॒त॒ इती᳚न्धते । \newline
28. इन्द्रे॑ण दे॒वैर् दे॒वै रिन्द्रे॒ णेन्द्रे॑ण दे॒वैः स॒रथꣳ॑ स॒रथ॑म् दे॒वै रिन्द्रे॒ णेन्द्रे॑ण दे॒वैः स॒रथ᳚म् । \newline
29. दे॒वैः स॒रथꣳ॑ स॒रथ॑म् दे॒वैर् दे॒वैः स॒रथꣳ॒॒ स स स॒रथ॑म् दे॒वैर् दे॒वैः स॒रथꣳ॒॒ सः । \newline
30. स॒रथꣳ॒॒ स स स॒रथꣳ॑ स॒रथꣳ॒॒ स ब॒र्॒.हिषि॑ ब॒र्॒.हिषि॒ स स॒रथꣳ॑ स॒रथꣳ॒॒ स ब॒र्॒.हिषि॑ । \newline
31. स॒रथ॒मिति॑ स - रथ᳚म् । \newline
32. स ब॒र्॒.हिषि॑ ब॒र्॒.हिषि॒ स स ब॒र्॒.हिषि॒ सीद॒थ् सीद॑द् ब॒र्॒.हिषि॒ स स ब॒र्॒.हिषि॒ सीद॑त् । \newline
33. ब॒र्॒.हिषि॒ सीद॒थ् सीद॑द् ब॒र्॒.हिषि॑ ब॒र्॒.हिषि॒ सीद॒न् नि नि सीद॑द् ब॒र्॒.हिषि॑ ब॒र्॒.हिषि॒ सीद॒न् नि । \newline
34. सीद॒न् नि नि सीद॒थ् सीद॒न् नि होता॒ होता॒ नि सीद॒थ् सीद॒न् नि होता᳚ । \newline
35. नि होता॒ होता॒ नि नि होता॑ य॒जथा॑य य॒जथा॑य॒ होता॒ नि नि होता॑ य॒जथा॑य । \newline
36. होता॑ य॒जथा॑य य॒जथा॑य॒ होता॒ होता॑ य॒जथा॑य सु॒क्रतुः॑ सु॒क्रतु॑र् य॒जथा॑य॒ होता॒ होता॑ य॒जथा॑य सु॒क्रतुः॑ । \newline
37. य॒जथा॑य सु॒क्रतुः॑ सु॒क्रतु॑र् य॒जथा॑य य॒जथा॑य सु॒क्रतुः॑ । \newline
38. सु॒क्रतु॒रिति॑ सु - क्रतुः॑ । \newline
39. त्वाम् चि॑त्रश्रवस्तम चित्रश्रवस्तम॒ त्वाम् त्वाम् चि॑त्रश्रवस्तम॒ हव॑न्ते॒ हव॑न्ते चित्रश्रवस्तम॒ त्वाम् त्वाम् चि॑त्रश्रवस्तम॒ हव॑न्ते । \newline
40. चि॒त्र॒श्र॒व॒स्त॒म॒ हव॑न्ते॒ हव॑न्ते चित्रश्रवस्तम चित्रश्रवस्तम॒ हव॑न्ते वि॒क्षु वि॒क्षु हव॑न्ते चित्रश्रवस्तम चित्रश्रवस्तम॒ हव॑न्ते वि॒क्षु । \newline
41. चि॒त्र॒श्र॒व॒स्त॒मेति॑ चित्रश्रवः - त॒म॒ । \newline
42. हव॑न्ते वि॒क्षु वि॒क्षु हव॑न्ते॒ हव॑न्ते वि॒क्षु ज॒न्तवो॑ ज॒न्तवो॑ वि॒क्षु हव॑न्ते॒ हव॑न्ते वि॒क्षु ज॒न्तवः॑ । \newline
43. वि॒क्षु ज॒न्तवो॑ ज॒न्तवो॑ वि॒क्षु वि॒क्षु ज॒न्तवः॑ । \newline
44. ज॒न्तव॒ इति॑ ज॒न्तवः॑ । \newline
45. शो॒चिष्के॑शम् पुरुप्रिय पुरुप्रिय शो॒चिष्के॑शꣳ शो॒चिष्के॑शम् पुरुप्रि॒याग्ने ऽग्ने॑ पुरुप्रिय शो॒चिष्के॑शꣳ शो॒चिष्के॑शम् पुरुप्रि॒याग्ने᳚ । \newline
46. शो॒चिष्के॑श॒मिति॑ शो॒चिः - के॒श॒म् । \newline
47. पु॒रु॒प्रि॒याग्ने ऽग्ने॑ पुरुप्रिय पुरुप्रि॒याग्ने॑ ह॒व्याय॑ ह॒व्यायाग्ने॑ पुरुप्रिय पुरुप्रि॒याग्ने॑ ह॒व्याय॑ । \newline
48. पु॒रु॒प्रि॒येति॑ पुरु - प्रि॒य॒ । \newline
49. अग्ने॑ ह॒व्याय॑ ह॒व्यायाग्ने ऽग्ने॑ ह॒व्याय॒ वोढ॑वे॒ वोढ॑वे ह॒व्यायाग्ने ऽग्ने॑ ह॒व्याय॒ वोढ॑वे । \newline
50. ह॒व्याय॒ वोढ॑वे॒ वोढ॑वे ह॒व्याय॑ ह॒व्याय॒ वोढ॑वे । \newline
51. वोढ॑व॒ इति॒ वोढ॑वे । \newline
52. सखा॑यः॒ सꣳ सꣳ सखा॑यः॒ सखा॑यः॒ सं ॅवो॑ वः॒ सꣳ सखा॑यः॒ सखा॑यः॒ सं ॅवः॑ । \newline
53. सं ॅवो॑ वः॒ सꣳ सं ॅवः॑ स॒म्यञ्चꣳ॑ स॒म्यञ्चं॑ ॅवः॒ सꣳ सं ॅवः॑ स॒म्यञ्च᳚म् । \newline
54. वः॒ स॒म्यञ्चꣳ॑ स॒म्यञ्चं॑ ॅवो वः स॒म्यञ्च॒ मिष॒ मिषꣳ॑ स॒म्यञ्चं॑ ॅवो वः स॒म्यञ्च॒ मिष᳚म् । \newline
55. स॒म्यञ्च॒ मिष॒ मिषꣳ॑ स॒म्यञ्चꣳ॑ स॒म्यञ्च॒ मिषꣳ॒॒ स्तोमꣳ॒॒ स्तोम॒ मिषꣳ॑ स॒म्यञ्चꣳ॑ स॒म्यञ्च॒ मिषꣳ॒॒ स्तोम᳚म् । \newline
56. इषꣳ॒॒ स्तोमꣳ॒॒ स्तोम॒ मिष॒ मिषꣳ॒॒ स्तोम॑म् च च॒ स्तोम॒ मिष॒ मिषꣳ॒॒ स्तोम॑म् च । \newline
\pagebreak
\markright{ TS 4.4.4.4  \hfill https://www.vedavms.in \hfill}

\section{ TS 4.4.4.4 }

\textbf{TS 4.4.4.4 } \newline
\textbf{Samhita Paata} \newline

स्तोमं॑ चा॒ग्नये᳚ । वर्.षि॑ष्ठाय क्षिती॒नामू॒र्जो नप्त्रे॒ सह॑स्वते ॥ सꣳस॒मिद्यु॑वसे वृष॒न्नग्ने॒ विश्वा᳚न्य॒र्य आ । इ॒डस्प॒दे समि॑द्ध्यसे॒ स नो॒ वसू॒न्या भ॑र ॥ ए॒ना वो॑ अ॒ग्निं नम॑सो॒र्जो नपा॑त॒मा हु॑वे । प्रि॒यं चेति॑ष्ठमर॒तिꣳ स्व॑द्ध्व॒रं ॅविश्व॑स्य दू॒तम॒मृतं᳚ ॥ स यो॑जते अरु॒षो वि॒श्वभो॑जसा॒ स दु॑द्रव॒थ् स्वा॑हुतः । सु॒ब्रह्मा॑ य॒ज्ञ्ः सु॒शमी॒ - [  ] \newline

\textbf{Pada Paata} \newline

स्तोम᳚म् । च॒ । अ॒ग्नये᳚ ॥ वर्.षि॑ष्ठाय । क्षि॒ती॒नाम् । ऊ॒र्जः । नप्त्रे᳚ । सह॑स्वते ॥ सꣳस॒मिति॒ सम् - स॒म् । इत् । यु॒व॒से॒ । वृ॒ष॒न्न् । अग्ने᳚ । विश्वा॑नि । अ॒र्यः । आ ॥ इ॒डः । प॒दे । समिति॑ । इ॒द्ध्य॒से॒ । सः । नः॒ । वसू॑नि । एति॑ । भ॒र॒ ॥ ए॒ना । वः॒ । अ॒ग्निम् । नम॑सा । ऊ॒र्जः । नपा॑तम् । एति॑ । हु॒वे॒ ॥ प्रि॒यम् । चेति॑ष्ठम् । अ॒र॒तिम् । स्व॒द्ध्व॒रमिति॑ सु - अ॒द्ध्व॒रम् । विश्व॑स्य । दू॒तम् । अ॒मृत᳚म् ॥ सः । यो॒ज॒ते॒ । अ॒रु॒षः । वि॒श्वभो॑ज॒सेति॑ वि॒श्व - भो॒ज॒सा॒ । सः । दु॒द्र॒व॒त् । स्वा॑हुत॒ इति॒ सु - आ॒हु॒तः॒ ॥ सु॒ब्रह्मेति॑ सु - ब्रह्मा᳚ । य॒ज्ञ्ः । सु॒शमीति॑ सु - शमी᳚ ।  \newline


\textbf{Krama Paata} \newline

स्तोम॑म् च । चा॒ग्नये᳚ । अ॒ग्नय॒ इत्य॒ग्नये᳚ ॥ वर्.षि॑ष्ठाय क्षिती॒नाम् । क्षि॒ती॒नामू॒र्जः । ऊ॒र्जो नप्त्रे᳚ । नप्त्रे॒ सह॑स्वते । सह॑स्वत॒ इति॒ सह॑स्वते ॥ सꣳस॒मित् । सꣳस॒मिति॒ सम् - स॒म् । इद् यु॑वसे । यु॒व॒से॒ वृ॒ष॒न्न्॒ । वृ॒ष॒न्नग्ने᳚ । अग्ने॒ विश्वा॑नि । विश्वा᳚न्य॒र्यः । अ॒र्य आ । एत्या ॥ इ॒डस्प॒दे । प॒दे सम् । समि॑द्ध्यसे । इ॒द्ध्य॒से॒ सः । स नः॑ । नो॒ वसू॑नि । वसू॒न्या । आ भ॑र । भ॒रेति॑ भर ॥ ए॒ना वः॑ । वो॒ अ॒ग्निम् । अ॒ग्निम् नम॑सा । नम॑सो॒र्जः । ऊ॒र्जो नपा॑तम् । नपा॑त॒मा । आ हु॑वे । हु॒व॒ इति॑ हुवे ॥ प्रि॒यम् चेति॑ष्ठम् । चेति॑ष्ठमर॒तिम् । अ॒र॒तिꣳ स्व॑द्ध्व॒रम् । स्व॒द्ध्व॒रं ॅविश्व॑स्य । स्व॒द्ध्व॒रमिति॑ सु - अ॒द्ध्व॒रम् । विश्व॑स्य दू॒तम् । दू॒तम॒मृत᳚म् । अ॒मृत॒मित्य॒मृत᳚म् ॥ स यो॑जते । यो॒ज॒ते॒ अ॒रु॒षः । अ॒रु॒षो वि॒श्वभो॑जसा । वि॒श्वभो॑जसा॒ सः । वि॒श्वभो॑ज॒सेति॑ वि॒श्व - भो॒ज॒सा॒ । स दु॑द्रवत् । दु॒द्र॒व॒थ् स्वा॑हुतः । स्वा॑हुत॒ इति॒ सु - आ॒हु॒तः॒ ॥ सु॒ब्रह्मा॑ य॒ज्ञ्ः । सु॒ब्रह्मेति॑ सु - ब्रह्मा᳚ । य॒ज्ञ्ः सु॒शमी᳚ । सु॒शमी॒ वसू॑नाम् । सु॒शमीति॑ सु - शमी᳚ \newline

\textbf{Jatai Paata} \newline

1. स्तोम॑म् च च॒ स्तोमꣳ॒॒ स्तोम॑म् च । \newline
2. चा॒ग्नये॑ अ॒ग्नये॑ च चा॒ग्नये᳚ । \newline
3. अ॒ग्नय॒ इत्य॒ग्नये᳚ । \newline
4. वर्.षि॑ष्ठाय क्षिती॒नाम् क्षि॑ती॒नां ॅवर्.षि॑ष्ठाय॒ वर्.षि॑ष्ठाय क्षिती॒नाम् । \newline
5. क्षि॒ती॒ना मू॒र्ज ऊ॒र्जः क्षि॑ती॒नाम् क्षि॑ती॒ना मू॒र्जः । \newline
6. ऊ॒र्जो नप्त्रे॒ नप्त्र॑ ऊ॒र्ज ऊ॒र्जो नप्त्रे᳚ । \newline
7. नप्त्रे॒ सह॑स्वते॒ सह॑स्वते॒ नप्त्रे॒ नप्त्रे॒ सह॑स्वते । \newline
8. सह॑स्वत॒ इति॒ सह॑स्वते । \newline
9. सꣳस॒ मिदिथ् सꣳसꣳ॒॒ सꣳस॒ मित् । \newline
10. सꣳस॒मिति॒ सम् - स॒म् । \newline
11. इद् यु॑वसे युवस॒ इदिद् यु॑वसे । \newline
12. यु॒व॒से॒ वृ॒ष॒न् वृ॒ष॒न्॒. यु॒व॒से॒ यु॒व॒से॒ वृ॒ष॒न्न् । \newline
13. वृ॒ष॒न् नग्ने ऽग्ने॑ वृषन् वृष॒न् नग्ने᳚ । \newline
14. अग्ने॒ विश्वा॑नि॒ विश्वा॒ न्यग्ने ऽग्ने॒ विश्वा॑नि । \newline
15. विश्वा᳚ न्य॒र्यो अ॒र्यो विश्वा॑नि॒ विश्वा᳚ न्य॒र्यः । \newline
16. अ॒र्य आ ऽर्यो अ॒र्य आ । \newline
17. एत्या । \newline
18. इ॒ड स्प॒दे प॒द इ॒ड इ॒ड स्प॒दे । \newline
19. प॒दे सꣳ सम् प॒दे प॒दे सम् । \newline
20. स मि॑द्ध्यस इद्ध्यसे॒ सꣳ स मि॑द्ध्यसे । \newline
21. इ॒द्ध्य॒से॒ स स इ॑द्ध्यस इद्ध्यसे॒ सः । \newline
22. स नो॑ नः॒ स स नः॑ । \newline
23. नो॒ वसू॑नि॒ वसू॑नि नो नो॒ वसू॑नि । \newline
24. वसू॒न्या वसू॑नि॒ वसू॒न्या । \newline
25. आ भ॑र भ॒रा भ॑र । \newline
26. भ॒रेति॑ भर । \newline
27. ए॒ना वो॑ व ए॒नैना वः॑ । \newline
28. वो॒ अ॒ग्नि म॒ग्निं ॅवो॑ वो अ॒ग्निम् । \newline
29. अ॒ग्निन् नम॑सा॒ नम॑सा॒ ऽग्नि म॒ग्निन् नम॑सा । \newline
30. नम॑ सो॒र्ज ऊ॒र्जो नम॑सा॒ नम॑ सो॒र्जः । \newline
31. ऊ॒र्जो नपा॑त॒म् नपा॑त मू॒र्ज ऊ॒र्जो नपा॑तम् । \newline
32. नपा॑त॒ मा नपा॑त॒म् नपा॑त॒ मा । \newline
33. आ हु॑वे हुव॒ आ हु॑वे । \newline
34. हु॒व॒ इति॑ हुवे । \newline
35. प्रि॒यम् चेति॑ष्ठ॒म् चेति॑ष्ठम् प्रि॒यम् प्रि॒यम् चेति॑ष्ठम् । \newline
36. चेति॑ष्ठ मर॒ति म॑र॒तिम् चेति॑ष्ठ॒म् चेति॑ष्ठ मर॒तिम् । \newline
37. अ॒र॒तिꣳ स्व॑द्ध्व॒रꣳ स्व॑द्ध्व॒र म॑र॒ति म॑र॒तिꣳ स्व॑द्ध्व॒रम् । \newline
38. स्व॒द्ध्व॒रं ॅविश्व॑स्य॒ विश्व॑स्य स्वद्ध्व॒रꣳ स्व॑द्ध्व॒रं ॅविश्व॑स्य । \newline
39. स्व॒द्ध्व॒रमिति॑ सु - अ॒द्ध्व॒रम् । \newline
40. विश्व॑स्य दू॒तम् दू॒तं ॅविश्व॑स्य॒ विश्व॑स्य दू॒तम् । \newline
41. दू॒त म॒मृत॑ म॒मृत॑म् दू॒तम् दू॒त म॒मृत᳚म् । \newline
42. अ॒मृत॒मित्य॒मृत᳚म् । \newline
43. स यो॑जते योजते॒ स स यो॑जते । \newline
44. यो॒ज॒ते॒ अ॒रु॒षो अ॑रु॒षो यो॑जते योजते अरु॒षः । \newline
45. अ॒रु॒षो वि॒श्वभो॑जसा वि॒श्वभो॑जसा ऽरु॒षो अ॑रु॒षो वि॒श्वभो॑जसा । \newline
46. वि॒श्वभो॑जसा॒ स स वि॒श्वभो॑जसा वि॒श्वभो॑जसा॒ सः । \newline
47. वि॒श्वभो॑ज॒सेति॑ वि॒श्व - भो॒ज॒सा॒ । \newline
48. स दु॑द्रवद् दुद्रव॒थ् स स दु॑द्रवत् । \newline
49. दु॒द्र॒व॒थ् स्वा॑हुतः॒ स्वा॑हुतो दुद्रवद् दुद्रव॒थ् स्वा॑हुतः । \newline
50. स्वा॑हुत॒ इति॒ सु - आ॒हु॒तः॒ । \newline
51. सु॒ब्रह्मा॑ य॒ज्ञो य॒ज्ञ्ः सु॒ब्रह्मा॑ सु॒ब्रह्मा॑ य॒ज्ञ्ः । \newline
52. सु॒ब्रह्मेति॑ सु - ब्रह्मा᳚ । \newline
53. य॒ज्ञ्ः सु॒शमी॑ सु॒शमी॑ य॒ज्ञो य॒ज्ञ्ः सु॒शमी᳚ । \newline
54. सु॒शमी॒ वसू॑नां॒ ॅवसू॑नाꣳ सु॒शमी॑ सु॒शमी॒ वसू॑नाम् । \newline
55. सु॒शमीति॑ सु - शमी᳚ । \newline

\textbf{Ghana Paata } \newline

1. स्तोम॑म् च च॒ स्तोमꣳ॒॒ स्तोम॑म् चा॒ग्नये॑ अ॒ग्नये॑ च॒ स्तोमꣳ॒॒ स्तोम॑म् चा॒ग्नये᳚ । \newline
2. चा॒ग्नये॑ अ॒ग्नये॑ च चा॒ग्नये᳚ । \newline
3. अ॒ग्नय॒ इत्य॒ग्नये᳚ । \newline
4. वर्.षि॑ष्ठाय क्षिती॒नाम् क्षि॑ती॒नां ॅवर्.षि॑ष्ठाय॒ वर्.षि॑ष्ठाय क्षिती॒ना मू॒र्ज ऊ॒र्जः क्षि॑ती॒नां 
ॅवर्.षि॑ष्ठाय॒ वर्.षि॑ष्ठाय क्षिती॒ना मू॒र्जः । \newline
5. क्षि॒ती॒ना मू॒र्ज ऊ॒र्जः क्षि॑ती॒नाम् क्षि॑ती॒ना मू॒र्जो नप्त्रे॒ नप्त्र॑ ऊ॒र्जः क्षि॑ती॒नाम् क्षि॑ती॒ना मू॒र्जो नप्त्रे᳚ । \newline
6. ऊ॒र्जो नप्त्रे॒ नप्त्र॑ ऊ॒र्ज ऊ॒र्जो नप्त्रे॒ सह॑स्वते॒ सह॑स्वते॒ नप्त्र॑ ऊ॒र्ज ऊ॒र्जो नप्त्रे॒ सह॑स्वते । \newline
7. नप्त्रे॒ सह॑स्वते॒ सह॑स्वते॒ नप्त्रे॒ नप्त्रे॒ सह॑स्वते । \newline
8. सह॑स्वत॒ इति॒ सह॑स्वते । \newline
9. सꣳस॒ मिदिथ् सꣳसꣳ॒॒ सꣳस॒ मिद् यु॑वसे युवस॒ इथ् सꣳसꣳ॒॒ सꣳस॒ मिद् यु॑वसे । \newline
10. सꣳस॒मिति॒ सम् - स॒म् । \newline
11. इद् यु॑वसे युवस॒ इदिद् यु॑वसे वृषन् वृषन्. युवस॒ इदिद् यु॑वसे वृषन्न् । \newline
12. यु॒व॒से॒ वृ॒ष॒न् वृ॒ष॒न्॒. यु॒व॒से॒ यु॒व॒से॒ वृ॒ष॒न् नग्ने ऽग्ने॑ वृषन्. युवसे युवसे वृष॒न् नग्ने᳚ । \newline
13. वृ॒ष॒न् नग्ने ऽग्ने॑ वृषन् वृष॒न् नग्ने॒ विश्वा॑नि॒ विश्वा॒न्यग्ने॑ वृषन् वृष॒न् नग्ने॒ विश्वा॑नि । \newline
14. अग्ने॒ विश्वा॑नि॒ विश्वा॒न्यग्ने ऽग्ने॒ विश्वा᳚ न्य॒र्यो अ॒र्यो विश्वा॒न्यग्ने ऽग्ने॒ विश्वा᳚ न्य॒र्यः । \newline
15. विश्वा᳚ न्य॒र्यो अ॒र्यो विश्वा॑नि॒ विश्वा᳚ न्य॒र्य आ ऽर्यो विश्वा॑नि॒ विश्वा᳚ न्य॒र्य आ । \newline
16. अ॒र्य आ ऽर्यो अ॒र्य आ । \newline
17. एत्या । \newline
18. इ॒ड स्प॒दे प॒द इ॒ड इ॒ड स्प॒दे सꣳ सम् प॒द इ॒ड इ॒ड स्प॒दे सम् । \newline
19. प॒दे सꣳ सम् प॒दे प॒दे स मि॑द्ध्यस इद्ध्यसे॒ सम् प॒दे प॒दे स मि॑द्ध्यसे । \newline
20. स मि॑द्ध्यस इद्ध्यसे॒ सꣳ स मि॑द्ध्यसे॒ स स इ॑द्ध्यसे॒ सꣳ स मि॑द्ध्यसे॒ सः । \newline
21. इ॒द्ध्य॒से॒ स स इ॑द्ध्यस इद्ध्यसे॒ स नो॑ नः॒ स इ॑द्ध्यस इद्ध्यसे॒ स नः॑ । \newline
22. स नो॑ नः॒ स स नो॒ वसू॑नि॒ वसू॑नि नः॒ स स नो॒ वसू॑नि । \newline
23. नो॒ वसू॑नि॒ वसू॑नि नो नो॒ वसू॒न्या वसू॑नि नो नो॒ वसू॒न्या । \newline
24. वसू॒न्या वसू॑नि॒ वसू॒न्या भ॑र भ॒रा वसू॑नि॒ वसू॒न्या भ॑र । \newline
25. आ भ॑र भ॒रा भ॑र । \newline
26. भ॒रेति॑ भर । \newline
27. ए॒ना वो॑ व ए॒नैना वो॑ अ॒ग्नि म॒ग्निं ॅव॑ ए॒नैना वो॑ अ॒ग्निम् । \newline
28. वो॒ अ॒ग्नि म॒ग्निं ॅवो॑ वो अ॒ग्निम् नम॑सा॒ नम॑सा॒ ऽग्निं ॅवो॑ वो अ॒ग्निम् नम॑सा । \newline
29. अ॒ग्निम् नम॑सा॒ नम॑सा॒ ऽग्नि म॒ग्निम् नम॑सो॒र्ज ऊ॒र्जो नम॑सा॒ ऽग्नि म॒ग्निम् नम॑सो॒र्जः । \newline
30. नम॑सो॒र्ज ऊ॒र्जो नम॑सा॒ नम॑सो॒र्जो नपा॑त॒म् नपा॑त मू॒र्जो नम॑सा॒ नम॑सो॒र्जो नपा॑तम् । \newline
31. ऊ॒र्जो नपा॑त॒म् नपा॑त मू॒र्ज ऊ॒र्जो नपा॑त॒ मा नपा॑त मू॒र्ज ऊ॒र्जो नपा॑त॒ मा । \newline
32. नपा॑त॒ मा नपा॑त॒म् नपा॑त॒ मा हु॑वे हुव॒ आ नपा॑त॒म् नपा॑त॒ मा हु॑वे । \newline
33. आ हु॑वे हुव॒ आ हु॑वे । \newline
34. हु॒व॒ इति॑ हुवे । \newline
35. प्रि॒यम् चेति॑ष्ठ॒म् चेति॑ष्ठम् प्रि॒यम् प्रि॒यम् चेति॑ष्ठ मर॒ति म॑र॒तिम् चेति॑ष्ठम् प्रि॒यम् प्रि॒यम् चेति॑ष्ठ मर॒तिम् । \newline
36. चेति॑ष्ठ मर॒ति म॑र॒तिम् चेति॑ष्ठ॒म् चेति॑ष्ठ मर॒तिꣳ स्व॑द्ध्व॒रꣳ स्व॑द्ध्व॒र म॑र॒तिम् चेति॑ष्ठ॒म् चेति॑ष्ठ मर॒तिꣳ स्व॑द्ध्व॒रम् । \newline
37. अ॒र॒तिꣳ स्व॑द्ध्व॒रꣳ स्व॑द्ध्व॒र म॑र॒ति म॑र॒तिꣳ स्व॑द्ध्व॒रं ॅविश्व॑स्य॒ विश्व॑स्य स्वद्ध्व॒र म॑र॒ति म॑र॒तिꣳ स्व॑द्ध्व॒रं ॅविश्व॑स्य । \newline
38. स्व॒द्ध्व॒रं ॅविश्व॑स्य॒ विश्व॑स्य स्वद्ध्व॒रꣳ स्व॑द्ध्व॒रं ॅविश्व॑स्य दू॒तम् दू॒तं ॅविश्व॑स्य स्वद्ध्व॒रꣳ स्व॑द्ध्व॒रं ॅविश्व॑स्य दू॒तम् । \newline
39. स्व॒द्ध्व॒रमिति॑ सु - अ॒द्ध्व॒रम् । \newline
40. विश्व॑स्य दू॒तम् दू॒तं ॅविश्व॑स्य॒ विश्व॑स्य दू॒त म॒मृत॑ म॒मृत॑म् दू॒तं ॅविश्व॑स्य॒ विश्व॑स्य दू॒त म॒मृत᳚म् । \newline
41. दू॒त म॒मृत॑ म॒मृत॑म् दू॒तम् दू॒त म॒मृत᳚म् । \newline
42. अ॒मृत॒मित्य॒मृत᳚म् । \newline
43. स यो॑जते योजते॒ स स यो॑जते अरु॒षो अ॑रु॒षो यो॑जते॒ स स यो॑जते अरु॒षः । \newline
44. यो॒ज॒ते॒ अ॒रु॒षो अ॑रु॒षो यो॑जते योजते अरु॒षो वि॒श्वभो॑जसा वि॒श्वभो॑जसा ऽरु॒षो यो॑जते योजते अरु॒षो वि॒श्वभो॑जसा । \newline
45. अ॒रु॒षो वि॒श्वभो॑जसा वि॒श्वभो॑जसा ऽरु॒षो अ॑रु॒षो वि॒श्वभो॑जसा॒ स स वि॒श्वभो॑जसा ऽरु॒षो अ॑रु॒षो वि॒श्वभो॑जसा॒ सः । \newline
46. वि॒श्वभो॑जसा॒ स स वि॒श्वभो॑जसा वि॒श्वभो॑जसा॒ स दु॑द्रवद् दुद्रव॒थ् स वि॒श्वभो॑जसा वि॒श्वभो॑जसा॒ स दु॑द्रवत् । \newline
47. वि॒श्वभो॑ज॒सेति॑ वि॒श्व - भो॒ज॒सा॒ । \newline
48. स दु॑द्रवद् दुद्रव॒थ् स स दु॑द्रव॒थ् स्वा॑हुतः॒ स्वा॑हुतो दुद्रव॒थ् स स दु॑द्रव॒थ् स्वा॑हुतः । \newline
49. दु॒द्र॒व॒थ् स्वा॑हुतः॒ स्वा॑हुतो दुद्रवद् दुद्रव॒थ् स्वा॑हुतः । \newline
50. स्वा॑हुत॒ इति॒ सु - आ॒हु॒तः॒ । \newline
51. सु॒ब्रह्मा॑ य॒ज्ञो य॒ज्ञ्ः सु॒ब्रह्मा॑ सु॒ब्रह्मा॑ य॒ज्ञ्ः सु॒शमी॑ सु॒शमी॑ य॒ज्ञ्ः सु॒ब्रह्मा॑ सु॒ब्रह्मा॑ य॒ज्ञ्ः सु॒शमी᳚ । \newline
52. सु॒ब्रह्मेति॑ सु - ब्रह्मा᳚ । \newline
53. य॒ज्ञ्ः सु॒शमी॑ सु॒शमी॑ य॒ज्ञो य॒ज्ञ्ः सु॒शमी॒ वसू॑नां॒ ॅवसू॑नाꣳ सु॒शमी॑ य॒ज्ञो य॒ज्ञ्ः सु॒शमी॒ वसू॑नाम् । \newline
54. सु॒शमी॒ वसू॑नां॒ ॅवसू॑नाꣳ सु॒शमी॑ सु॒शमी॒ वसू॑नाम् दे॒वम् दे॒वं ॅवसू॑नाꣳ सु॒शमी॑ सु॒शमी॒ वसू॑नाम् दे॒वम् । \newline
55. सु॒शमीति॑ सु - शमी᳚ । \newline
\pagebreak
\markright{ TS 4.4.4.5  \hfill https://www.vedavms.in \hfill}

\section{ TS 4.4.4.5 }

\textbf{TS 4.4.4.5 } \newline
\textbf{Samhita Paata} \newline

वसू॑नां दे॒वꣳ राधो॒ जना॑नां ॥ उद॑स्य शो॒चिर॑स्थादा॒-जुह्वा॑नस्य मी॒ढुषः॑ । उद्ध॒मासो॑ अरु॒षासो॑ दिवि॒स्पृशः॒ सम॒ग्निमि॑न्धते॒ नरः॑ ॥ अग्ने॒ वाज॑स्य॒ गोम॑त॒ ईशा॑नः सहसो यहो । अ॒स्मे धे॑हि जातवेदो॒ महि॒ श्रवः॑ ॥ स इ॑धा॒नो वसु॑ष्क॒वि-र॒ग्निरी॒डेन्यो॑ गि॒रा । रे॒वद॒स्मभ्यं॑ पुर्वणीक दीदिहि ॥ क्ष॒पो रा॑जन्नु॒त त्मनाऽग्ने॒ वस्तो॑रु॒तोषसः॑ । स ति॑ग्मजंभ - [  ] \newline

\textbf{Pada Paata} \newline

वसू॑नाम् । दे॒वम् । राधः॑ । जना॑नाम् ॥ उदिति॑ । अ॒स्य॒ । शो॒चिः । अ॒स्था॒त् । आ॒जुह्वा॑न॒स्येत्या᳚ - जुह्वा॑नस्य । मी॒ढुषः॑ ॥ उदिति॑ । धू॒मासः॑ । अ॒रु॒षासः॑ । दि॒वि॒स्पृश॒ इति॑ दिवि - स्पृशः॑ । समिति॑ । अ॒ग्निम् । इ॒न्ध॒ते॒ । नरः॑ ॥ अग्ने᳚ । वाज॑स्य । गोम॑त॒ इति॒ गो-म॒तः॒ । ईशा॑नः । स॒ह॒सः॒ । य॒हो॒ इति॑ ॥ अ॒स्मे इति॑ । धे॒हि॒ । जा॒त॒वे॒द॒ इति॑ जात - वे॒दः॒ । महि॑ । श्रवः॑ ॥ सः । इ॒धा॒नः । वसुः॑ । क॒विः । अ॒ग्निः । ई॒डेन्यः॑ । गि॒रा ॥ रे॒वत् । अ॒स्मभ्य॒मित्य॒स्म - भ्य॒म् । पु॒र्व॒णी॒केति॑ पुरु-अ॒नी॒क॒ । दी॒दि॒हि॒ ॥ क्ष॒पः । रा॒ज॒न्न् । उ॒त । त्मना᳚ । अग्ने᳚ । वस्तोः᳚ । उ॒त । उ॒षसः॑ ॥ सः । ति॒ग्म॒ज॒भेंति॑ तिग्म - ज॒भं॒ ।  \newline


\textbf{Krama Paata} \newline

वसू॑नाम् दे॒वम् । दे॒वꣳ राधः॑ । राधो॒ जना॑नाम् । जना॑ना॒मिति॒ जना॑नाम् ॥ उद॑स्य । अ॒स्य॒ शो॒चिः । शो॒चिर॑स्थात् । अ॒स्था॒दा॒जुह्वा॑नस्य । आ॒जुह्वा॑नस्य मी॒ढुषः॑ । आ॒जुह्वा॑न॒स्येत्या᳚ - जुह्वा॑नस्य । मी॒ढुष॒ इति॑ मि॒ढुषः॑ ॥ उद् धू॒मासः॑ । धू॒मासो॑ अरु॒षासः॑ । अ॒रु॒षासो॑ दिवि॒स्पृशः॑ । दि॒वि॒स्पृशः॒ सम् । दि॒वि॒स्पृश॒ इति॑ दिवि - स्पृशः॑ । सम॒ग्निम् । अ॒ग्निमि॑न्धते । इ॒न्ध॒ते॒ नरः॑ । नर॒ इति॒ नरः॑ ॥ अग्ने॒ वाज॑स्य । वाज॑स्य॒ गोम॑तः । गोम॑त॒ ईशा॑नः । गोम॑त॒ इति॒ गो - म॒तः॒ । ईशा॑नः सहसः । स॒ह॒सो॒ य॒हो॒ । य॒हो॒ इति॑ यहो ॥ अ॒स्मे धे॑हि । अ॒स्मे इत्य॒स्मे । धे॒हि॒ जा॒त॒वे॒दः॒ । जा॒त॒वे॒दो॒ महि॑ । जा॒त॒वे॒द॒ इति॑ जात - वे॒दः॒ । महि॒ श्रवः॑ । श्रव॒ इति॒ श्रवः॑ ॥ स इ॑धा॒नः । इ॒धा॒नो वसुः॑ । वसु॑ष्क॒विः । क॒विर॒ग्निः । अ॒ग्निरी॒डेन्यः॑ । ई॒डेन्यो॑ गि॒रा । गि॒रेति॑ गि॒रा ॥ रे॒वद॒स्मभ्य᳚म् । अ॒स्मभ्य॑म् पुर्वणीक । अ॒स्मभ्य॒मित्य॒स्म - भ्य॒म् । पु॒र्व॒णी॒क॒ दी॒दि॒हि॒ । पु॒र्व॒णी॒केति॑ पुरु - अ॒नी॒क॒ । दी॒दि॒हीति॑ दीदिहि ॥ क्ष॒पो रा॑जन्न् । रा॒ज॒न्नु॒त । उ॒त त्मना᳚ । त्मनाऽग्ने᳚ । अग्ने॒ वस्तोः᳚ । वस्तो॑रु॒त । उ॒तोषसः॑ । उ॒षस॒ इत्यु॒षसः॑ ॥ स ति॑ग्मजम्भ । ति॒ग्म॒ज॒म्भ॒ र॒क्षसः॑ । ति॒ग्म॒ज॒म्भेति॑ तिग्म - ज॒म्भ॒ \newline

\textbf{Jatai Paata} \newline

1. वसू॑नाम् दे॒वम् दे॒वं ॅवसू॑नां॒ ॅवसू॑नाम् दे॒वम् । \newline
2. दे॒वꣳ राधो॒ राधो॑ दे॒वम् दे॒वꣳ राधः॑ । \newline
3. राधो॒ जना॑ना॒म् जना॑नाꣳ॒॒ राधो॒ राधो॒ जना॑नाम् । \newline
4. जना॑ना॒मिति॒ जना॑नाम् । \newline
5. उद॑स्या॒ स्योदु द॑स्य । \newline
6. अ॒स्य॒ शो॒चिः शो॒चिर॑ स्यास्य शो॒चिः । \newline
7. शो॒चि र॑स्था दस्था च्छो॒चिः शो॒चि र॑स्थात् । \newline
8. अ॒स्था॒ दा॒जुह्वा॑नस्या॒ जुह्वा॑नस्या स्था दस्था दा॒जुह्वा॑नस्य । \newline
9. आ॒जुह्वा॑नस्य मी॒ढुषो॑ मी॒ढुष॑ आ॒जुह्वा॑नस्या॒ जुह्वा॑नस्य मी॒ढुषः॑ । \newline
10. आ॒जुह्वा॑न॒स्येत्या᳚ - जुह्वा॑नस्य । \newline
11. मी॒ढुष॒ इति॑ मी॒ढुषः॑ । \newline
12. उद् धू॒मासो॑ धू॒मास॒ उदुद् धू॒मासः॑ । \newline
13. धू॒मासो॑ अरु॒षासो॑ अरु॒षासो॑ धू॒मासो॑ धू॒मासो॑ अरु॒षासः॑ । \newline
14. अ॒रु॒षासो॑ दिवि॒स्पृशो॑ दिवि॒स्पृशो॑ अरु॒षासो॑ अरु॒षासो॑ दिवि॒स्पृशः॑ । \newline
15. दि॒वि॒स्पृशः॒ सꣳ सम् दि॑वि॒स्पृशो॑ दिवि॒स्पृशः॒ सम् । \newline
16. दि॒वि॒स्पृश॒ इति॑ दिवि - स्पृशः॑ । \newline
17. स म॒ग्नि म॒ग्निꣳ सꣳ स म॒ग्निम् । \newline
18. अ॒ग्नि मि॑न्धत इन्धते अ॒ग्नि म॒ग्नि मि॑न्धते । \newline
19. इ॒न्ध॒ते॒ नरो॒ नर॑ इन्धत इन्धते॒ नरः॑ । \newline
20. नर॒ इति॒ नरः॑ । \newline
21. अग्ने॒ वाज॑स्य॒ वाज॒ स्याग्ने ऽग्ने॒ वाज॑स्य । \newline
22. वाज॑स्य॒ गोम॑तो॒ गोम॑तो॒ वाज॑स्य॒ वाज॑स्य॒ गोम॑तः । \newline
23. गोम॑त॒ ईशा॑न॒ ईशा॑नो॒ गोम॑तो॒ गोम॑त॒ ईशा॑नः । \newline
24. गोम॑त॒ इति॒ गो - म॒तः॒ । \newline
25. ईशा॑नः सहसः सहस॒ ईशा॑न॒ ईशा॑नः सहसः । \newline
26. स॒ह॒सो॒ य॒हो॒ य॒हो॒ स॒ह॒सः॒ स॒ह॒सो॒ य॒हो॒ । \newline
27. य॒हो॒ इति॑ यहो । \newline
28. अ॒स्मे धे॑हि धेह्य॒स्मे अ॒स्मे धे॑हि । \newline
29. अ॒स्मे इत्य॒स्मे । \newline
30. धे॒हि॒ जा॒त॒वे॒दो॒ जा॒त॒वे॒दो॒ धे॒हि॒ धे॒हि॒ जा॒त॒वे॒दः॒ । \newline
31. जा॒त॒वे॒दो॒ महि॒ महि॑ जातवेदो जातवेदो॒ महि॑ । \newline
32. जा॒त॒वे॒द॒ इति॑ जात - वे॒दः॒ । \newline
33. महि॒ श्रवः॒ श्रवो॒ महि॒ महि॒ श्रवः॑ । \newline
34. श्रव॒ इति॒ श्रवः॑ । \newline
35. स इ॑धा॒न इ॑धा॒नः स स इ॑धा॒नः । \newline
36. इ॒धा॒नो वसु॒र् वसु॑ रिधा॒न इ॑धा॒नो वसुः॑ । \newline
37. वसु॑ष् क॒विः क॒विर् वसु॒र् वसु॑ष् क॒विः । \newline
38. क॒वि र॒ग्नि र॒ग्निः क॒विः क॒वि र॒ग्निः । \newline
39. अ॒ग्नि री॒डेन्य॑ ई॒डेन्यो॑ अ॒ग्नि र॒ग्नि री॒डेन्यः॑ । \newline
40. ई॒डेन्यो॑ गि॒रा गि॒रे डेन्य॑ ई॒डेन्यो॑ गि॒रा । \newline
41. गि॒रेति॑ गि॒रा । \newline
42. रे॒व द॒स्मभ्य॑ म॒स्मभ्यꣳ॑ रे॒वद् रे॒व द॒स्मभ्य᳚म् । \newline
43. अ॒स्मभ्य॑म् पुर्वणीक पुर्वणीका॒ स्मभ्य॑ म॒स्मभ्य॑म् पुर्वणीक । \newline
44. अ॒स्मभ्य॒मित्य॒स्म - भ्य॒म् । \newline
45. पु॒र्व॒णी॒क॒ दी॒दि॒हि॒ दी॒दि॒हि॒ पु॒र्व॒णी॒क॒ पु॒र्व॒णी॒क॒ दी॒दि॒हि॒ । \newline
46. पु॒र्व॒णी॒केति॑ पुरु - अ॒नी॒क॒ । \newline
47. दी॒दि॒हीति॑ दीदिहि । \newline
48. क्ष॒पो रा॑जन् राजन् क्ष॒पः क्ष॒पो रा॑जन्न् । \newline
49. रा॒ज॒न् नु॒तोत रा॑जन् राजन् नु॒त । \newline
50. उ॒त त्मना॒ त्मनो॒ तोत त्मना᳚ । \newline
51. त्मना ऽग्ने ऽग्ने॒ त्मना॒ त्मना ऽग्ने᳚ । \newline
52. अग्ने॒ वस्तो॒र् वस्तो॒ रग्ने ऽग्ने॒ वस्तोः᳚ । \newline
53. वस्तो॑ रु॒तोत वस्तो॒र् वस्तो॑ रु॒त । \newline
54. उ॒तोषस॑ उ॒षस॑ उ॒तोतोषसः॑ । \newline
55. उ॒षस॒ इत्यु॒षसः॑ । \newline
56. स ति॑ग्मजंभ तिग्मजंभ॒ स स ति॑ग्मजंभ । \newline
57. ति॒ग्म॒जं॒भ॒ र॒क्षसो॑ र॒क्षस॑ स्तिग्मजंभ तिग्मजंभ र॒क्षसः॑ । \newline
58. ति॒ग्म॒जं॒भेति॑ तिग्म - जं॒भ॒ । \newline

\textbf{Ghana Paata } \newline

1. वसू॑नाम् दे॒वम् दे॒वं ॅवसू॑नां॒ ॅवसू॑नाम् दे॒वꣳ राधो॒ राधो॑ दे॒वं ॅवसू॑नां॒ ॅवसू॑नाम् दे॒वꣳ राधः॑ । \newline
2. दे॒वꣳ राधो॒ राधो॑ दे॒वम् दे॒वꣳ राधो॒ जना॑ना॒म् जना॑नाꣳ॒॒ राधो॑ दे॒वम् दे॒वꣳ राधो॒ जना॑नाम् । \newline
3. राधो॒ जना॑ना॒म् जना॑नाꣳ॒॒ राधो॒ राधो॒ जना॑नाम् । \newline
4. जना॑ना॒मिति॒ जना॑नाम् । \newline
5. उद॑स्या॒ स्योदु द॑स्य शो॒चिः शो॒चि र॒स्यो दुद॑स्य शो॒चिः । \newline
6. अ॒स्य॒ शो॒चिः शो॒चिर॑ स्यास्य शो॒चि र॑स्था दस्था च्छो॒चि र॑स्यास्य शो॒चि र॑स्थात् । \newline
7. शो॒चि र॑स्था दस्था च्छो॒चिः शो॒चि र॑स्था दा॒जुह्वा॑नस्या॒ जुह्वा॑नस्या स्था च्छो॒चिः शो॒चि र॑स्था दा॒जुह्वा॑नस्य । \newline
8. अ॒स्था॒ दा॒जुह्वा॑नस्या॒ जुह्वा॑नस्या स्था दस्था दा॒जुह्वा॑नस्य मी॒ढुषो॑ मी॒ढुष॑ आ॒जुह्वा॑नस्या 
स्था दस्था दा॒जुह्वा॑नस्य मी॒ढुषः॑ । \newline
9. आ॒जुह्वा॑नस्य मी॒ढुषो॑ मी॒ढुष॑ आ॒जुह्वा॑नस्या॒ जुह्वा॑नस्य मी॒ढुषः॑ । \newline
10. आ॒जुह्वा॑न॒स्येत्या᳚ - जुह्वा॑नस्य । \newline
11. मी॒ढुष॒ इति॑ मी॒ढुषः॑ । \newline
12. उद् धू॒मासो॑ धू॒मास॒ उदुद् धू॒मासो॑ अरु॒षासो॑ अरु॒षासो॑ धू॒मास॒ उदुद् धू॒मासो॑ अरु॒षासः॑ । \newline
13. धू॒मासो॑ अरु॒षासो॑ अरु॒षासो॑ धू॒मासो॑ धू॒मासो॑ अरु॒षासो॑ दिवि॒स्पृशो॑ दिवि॒स्पृशो॑ अरु॒षासो॑ धू॒मासो॑ धू॒मासो॑ अरु॒षासो॑ दिवि॒स्पृशः॑ । \newline
14. अ॒रु॒षासो॑ दिवि॒स्पृशो॑ दिवि॒स्पृशो॑ अरु॒षासो॑ अरु॒षासो॑ दिवि॒स्पृशः॒ सꣳ सम् 
दि॑वि॒स्पृशो॑ अरु॒षासो॑ अरु॒षासो॑ दिवि॒स्पृशः॒ सम् । \newline
15. दि॒वि॒स्पृशः॒ सꣳ सम् दि॑वि॒स्पृशो॑ दिवि॒स्पृशः॒ स म॒ग्नि म॒ग्निꣳ सम् दि॑वि॒स्पृशो॑ दिवि॒स्पृशः॒ स म॒ग्निम् । \newline
16. दि॒वि॒स्पृश॒ इति॑ दिवि - स्पृशः॑ । \newline
17. स म॒ग्नि म॒ग्निꣳ सꣳ स म॒ग्नि मि॑न्धत इन्धते अ॒ग्निꣳ सꣳ स म॒ग्नि मि॑न्धते । \newline
18. अ॒ग्नि मि॑न्धत इन्धते अ॒ग्नि म॒ग्नि मि॑न्धते॒ नरो॒ नर॑ इन्धते अ॒ग्नि म॒ग्नि मि॑न्धते॒ नरः॑ । \newline
19. इ॒न्ध॒ते॒ नरो॒ नर॑ इन्धत इन्धते॒ नरः॑ । \newline
20. नर॒ इति॒ नरः॑ । \newline
21. अग्ने॒ वाज॑स्य॒ वाज॒स्याग्ने ऽग्ने॒ वाज॑स्य॒ गोम॑तो॒ गोम॑तो॒ वाज॒स्याग्ने ऽग्ने॒ वाज॑स्य॒ गोम॑तः । \newline
22. वाज॑स्य॒ गोम॑तो॒ गोम॑तो॒ वाज॑स्य॒ वाज॑स्य॒ गोम॑त॒ ईशा॑न॒ ईशा॑नो॒ गोम॑तो॒ वाज॑स्य॒ वाज॑स्य॒ गोम॑त॒ ईशा॑नः । \newline
23. गोम॑त॒ ईशा॑न॒ ईशा॑नो॒ गोम॑तो॒ गोम॑त॒ ईशा॑नः सहसः सहस॒ ईशा॑नो॒ गोम॑तो॒ गोम॑त॒ ईशा॑नः सहसः । \newline
24. गोम॑त॒ इति॒ गो - म॒तः॒ । \newline
25. ईशा॑नः सहसः सहस॒ ईशा॑न॒ ईशा॑नः सहसो यहो यहो सहस॒ ईशा॑न॒ ईशा॑नः सहसो यहो । \newline
26. स॒ह॒सो॒ य॒हो॒ य॒हो॒ स॒ह॒सः॒ स॒ह॒सो॒ य॒हो॒ । \newline
27. य॒हो॒ इति॑ यहो । \newline
28. अ॒स्मे धे॑हि धेह्य॒स्मे अ॒स्मे धे॑हि जातवेदो जातवेदो धेह्य॒स्मे अ॒स्मे धे॑हि जातवेदः । \newline
29. अ॒स्मे इत्य॒स्मे । \newline
30. धे॒हि॒ जा॒त॒वे॒दो॒ जा॒त॒वे॒दो॒ धे॒हि॒ धे॒हि॒ जा॒त॒वे॒दो॒ महि॒ महि॑ जातवेदो धेहि धेहि जातवेदो॒ महि॑ । \newline
31. जा॒त॒वे॒दो॒ महि॒ महि॑ जातवेदो जातवेदो॒ महि॒ श्रवः॒ श्रवो॒ महि॑ जातवेदो जातवेदो॒ महि॒ श्रवः॑ । \newline
32. जा॒त॒वे॒द॒ इति॑ जात - वे॒दः॒ । \newline
33. महि॒ श्रवः॒ श्रवो॒ महि॒ महि॒ श्रवः॑ । \newline
34. श्रव॒ इति॒ श्रवः॑ । \newline
35. स इ॑धा॒न इ॑धा॒नः स स इ॑धा॒नो वसु॒र् वसु॑ रिधा॒नः स स इ॑धा॒नो वसुः॑ । \newline
36. इ॒धा॒नो वसु॒र् वसु॑ रिधा॒न इ॑धा॒नो वसु॑ष् क॒विः क॒विर् वसु॑ रिधा॒न इ॑धा॒नो वसु॑ष् क॒विः । \newline
37. वसु॑ष् क॒विः क॒विर् वसु॒र् वसु॑ष् क॒वि र॒ग्नि र॒ग्निः क॒विर् वसु॒र् वसु॑ष् क॒वि र॒ग्निः । \newline
38. क॒वि र॒ग्नि र॒ग्निः क॒विः क॒वि र॒ग्नि री॒डेन्य॑ ई॒डेन्यो॑ अ॒ग्निः क॒विः क॒वि र॒ग्नि री॒डेन्यः॑ । \newline
39. अ॒ग्नि री॒डेन्य॑ ई॒डेन्यो॑ अ॒ग्नि र॒ग्नि री॒डेन्यो॑ गि॒रा गि॒रेडेन्यो॑ अ॒ग्नि र॒ग्नि री॒डेन्यो॑ गि॒रा । \newline
40. ई॒डेन्यो॑ गि॒रा गि॒रेडेन्य॑ ई॒डेन्यो॑ गि॒रा । \newline
41. गि॒रेति॑ गि॒रा । \newline
42. रे॒व द॒स्मभ्य॑ म॒स्मभ्यꣳ॑ रे॒वद् रे॒वद॒ स्मभ्य॑म् पुर्वणीक पुर्वणीका॒ स्मभ्यꣳ॑ रे॒वद् रे॒व द॒स्मभ्य॑म् पुर्वणीक । \newline
43. अ॒स्मभ्य॑म् पुर्वणीक पुर्वणीका॒ स्मभ्य॑ म॒स्मभ्य॑म् पुर्वणीक दीदिहि दीदिहि पुर्वणीका॒ स्मभ्य॑ म॒स्मभ्य॑म् पुर्वणीक दीदिहि । \newline
44. अ॒स्मभ्य॒मित्य॒स्म - भ्य॒म् । \newline
45. पु॒र्व॒णी॒क॒ दी॒दि॒हि॒ दी॒दि॒हि॒ पु॒र्व॒णी॒क॒ पु॒र्व॒णी॒क॒ दी॒दि॒हि॒ । \newline
46. पु॒र्व॒णी॒केति॑ पुरु - अ॒नी॒क॒ । \newline
47. दी॒दि॒हीति॑ दीदिहि । \newline
48. क्ष॒पो रा॑जन् राजन् क्ष॒पः क्ष॒पो रा॑जन् नु॒तोत रा॑जन् क्ष॒पः क्ष॒पो रा॑जन् नु॒त । \newline
49. रा॒ज॒न् नु॒तोत रा॑जन् राजन् नु॒त त्मना॒ त्मनो॒त रा॑जन् राजन् नु॒त त्मना᳚ । \newline
50. उ॒त त्मना॒ त्म नो॒तोत त्मना ऽग्ने ऽग्ने॒ त्मनो॒तोत त्मना ऽग्ने᳚ । \newline
51. त्मना ऽग्ने ऽग्ने॒ त्मना॒ त्मना ऽग्ने॒ वस्तो॒र् वस्तो॒ रग्ने॒ त्मना॒ त्मना ऽग्ने॒ वस्तोः᳚ । \newline
52. अग्ने॒ वस्तो॒र् वस्तो॒ रग्ने ऽग्ने॒ वस्तो॑ रु॒तोत वस्तो॒ रग्ने ऽग्ने॒ वस्तो॑ रु॒त । \newline
53. वस्तो॑ रु॒तोत वस्तो॒र् वस्तो॑ रु॒तोषस॑ उ॒षस॑ उ॒त वस्तो॒र् वस्तो॑ रु॒तोषसः॑ । \newline
54. उ॒तोषस॑ उ॒षस॑ उ॒तोतोषसः॑ । \newline
55. उ॒षस॒ इत्यु॒षसः॑ । \newline
56. स ति॑ग्मजंभ तिग्मजंभ॒ स स ति॑ग्मजंभ र॒क्षसो॑ र॒क्षस॑ स्तिग्मजंभ॒ स स ति॑ग्मजंभ र॒क्षसः॑ । \newline
57. ति॒ग्म॒जं॒भ॒ र॒क्षसो॑ र॒क्षस॑ स्तिग्मजंभ तिग्मजंभ र॒क्षसो॑ दह दह र॒क्षस॑ स्तिग्मजंभ तिग्मजंभ र॒क्षसो॑ दह । \newline
58. ति॒ग्म॒जं॒भेति॑ तिग्म - जं॒भ॒ । \newline
\pagebreak
\markright{ TS 4.4.4.6  \hfill https://www.vedavms.in \hfill}

\section{ TS 4.4.4.6 }

\textbf{TS 4.4.4.6 } \newline
\textbf{Samhita Paata} \newline

र॒क्षसो॑ दह॒ प्रति॑ ॥ आ ते॑ अग्न इधीमहि द्यु॒मन्तं॑ देवा॒जरं᳚ । यद्ध॒ स्या ते॒ पनी॑यसी स॒मिद्-दी॒दय॑ति॒ द्यवीषꣳ॑ स्तो॒तृभ्य॒ आ भ॑र ॥ आ ते॑ अग्न ऋ॒चा ह॒विः शु॒क्रस्य॑ ज्योतिषस्पते । सुश्च॑न्द्र॒ दस्म॒ विश्प॑ते॒ हव्य॑वा॒ट् तुभ्यꣳ॑ हूयत॒ इषꣳ॑ स्तो॒तृभ्य॒ आ भ॑र ॥ उ॒भे सु॑श्चन्द्र स॒र्पिषो॒ दर्वी᳚ श्रीणीष आ॒सनि॑ । उ॒तो न॒ उत् पु॑पूर्या- [  ] \newline

\textbf{Pada Paata} \newline

र॒क्षसः॑ । द॒ह॒ । प्रति॑ ॥ एति॑ । ते॒ । अ॒ग्ने॒ । इ॒धी॒म॒हि॒ । द्यु॒मन्त॒मिति॑ द्यु - मन्त᳚म् । दे॒व॒ । अ॒जर᳚म् ॥ यत् । ह॒ । स्या । ते॒ । पनी॑यसी । स॒मिदिति॑ सं - इत् । दी॒दय॑ति । द्यवि॑ । इष᳚म् । स्तो॒तृभ्य॒ इति॑ स्तो॒तृ - भ्यः॒ । एति॑ । भ॒र॒ ॥ एति॑ । ते॒ । अ॒ग्ने॒ । ऋ॒चा । ह॒विः । शु॒क्रस्य॑ । ज्यो॒ति॒षः॒ । प॒ते॒ ॥ सुश्च॒न्द्रेति॒ सु-च॒न्द्र॒ । दस्म॑ । विश्प॑ते । हव्य॑वा॒डिति॒ हव्य॑ - वा॒ट् । तुभ्य᳚म् । हू॒य॒ते॒ । इष᳚म् । स्तो॒तृभ्य॒ इति॑ स्तो॒तृ - भ्यः॒ । एति॑ । भ॒र॒ ॥ उ॒भे इति॑ । सु॒श्च॒न्द्रेति॑ सु - च॒न्द्र॒ । स॒र्पिषः॑ । दर्वी॒ इति॑ । श्री॒णी॒षे॒ । आ॒सनि॑ ॥ उ॒तो इति॑ । नः॒ । उदिति॑ । पु॒पू॒र्याः॒ ।  \newline


\textbf{Krama Paata} \newline

र॒क्षसो॑ दह । द॒ह॒ प्रति॑ । प्रतीति॒ प्रति॑ ॥ आ ते᳚ । ते॒ अ॒ग्ने॒ । अ॒ग्न॒ इ॒धी॒म॒हि॒ । इ॒धी॒म॒हि॒ द्यु॒मन्त᳚म् । द्यु॒मन्त॑म् देव । द्यु॒मन्त॒मिति॑ द्यु - मन्त᳚म् । दे॒वा॒जर᳚म् । अ॒जर॒मित्य॒जर᳚म् ॥ यद्ध॑ । ह॒ स्या । स्या ते᳚ । ते॒ पनी॑यसी । पनी॑यसी स॒मित् । स॒मिद् दी॒दय॑ति । स॒मिदिति॑ सम् - इत् । दी॒दय॑ति॒ द्यवि॑ । द्यवीष᳚म् । इषꣳ॑ स्तो॒तृभ्यः॑ । स्तो॒तृभ्य॒ आ । स्तो॒तृभ्य॒ इति॑ स्तो॒तृ - भ्यः॒ । आ भ॑र । भ॒रेति॑ भर ॥ आ ते᳚ । ते॒ अ॒ग्ने॒ । अ॒ग्न॒ ऋ॒चा । ऋ॒चा ह॒विः । ह॒विः शु॒क्रस्य॑ । शु॒क्रस्य॑ ज्योतिषः । ज्यो॒ति॒ष॒स्प॒ते॒ । प॒त॒ इति॑ पते ॥ सुश्च॑न्द्र॒ दस्म॑ । सुश्च॒न्द्रेति॒ सु - च॒न्द्र॒ । दस्म॒ विश्प॑ते । विश्प॑ते॒ हव्य॑वाट् । हव्य॑वा॒ट् तुभ्य᳚म् । हव्य॑वा॒डिति॒ हव्य॑ - वा॒ट्॒ । तुभ्यꣳ॑ हूयते । हू॒य॒त॒ इष᳚म् । इषꣳ॑ स्तो॒तृभ्यः॑ । स्तो॒तृभ्य॒ आ । स्तो॒तृभ्य॒ इति॑ स्तो॒तृ - भ्यः॒ । आ भ॑र । भ॒रेति॑ भर ॥ उ॒भे सु॑श्चन्द्र । उ॒भे इत्यु॒भे । सु॒श्च॒न्द्र॒ स॒र्पिषः॑ । सु॒श्च॒न्द्रेति॑ सु - च॒न्द्र॒ । स॒र्पिषो॒ दर्वी᳚ । दर्वी᳚ श्रीणीषे । दर्वी॒ इति॒ दर्वी᳚ । श्री॒णी॒ष॒ आ॒सनि॑ । आ॒सनीत्या॒सनि॑ ॥ उ॒तो नः॑ । उ॒तो इत्यु॒तो । न॒ उत् । उत् पु॑पूर्याः । पु॒पू॒र्या॒ उ॒क्थेषु॑ \newline

\textbf{Jatai Paata} \newline

1. र॒क्षसो॑ दह दह र॒क्षसो॑ र॒क्षसो॑ दह । \newline
2. द॒ह॒ प्रति॒ प्रति॑ दह दह॒ प्रति॑ । \newline
3. प्रतीति॒ प्रति॑ । \newline
4. आ ते॑ त॒ आ ते᳚ । \newline
5. ते॒ अ॒ग्ने॒ अ॒ग्ने॒ ते॒ ते॒ अ॒ग्ने॒ । \newline
6. अ॒ग्न॒ इ॒धी॒म॒ही॒ धी॒म॒ ह्य॒ग्ने॒ अ॒ग्न॒ इ॒धी॒म॒हि॒ । \newline
7. इ॒धी॒म॒हि॒ द्यु॒मन्त॑म् द्यु॒मन्त॑ मिधीमही धीमहि द्यु॒मन्त᳚म् । \newline
8. द्यु॒मन्त॑म् देव देव द्यु॒मन्त॑म् द्यु॒मन्त॑म् देव । \newline
9. द्यु॒मन्त॒मिति॑ द्यु - मन्त᳚म् । \newline
10. दे॒वा॒ जर॑ म॒जर॑म् देव देवा॒ जर᳚म् । \newline
11. अ॒जर॒मित्य॒जर᳚म् । \newline
12. यद्ध॑ ह॒ यद् यद्ध॑ । \newline
13. ह॒ स्या स्या ह॑ ह॒ स्या । \newline
14. स्या ते॑ ते॒ स्या स्या ते᳚ । \newline
15. ते॒ पनी॑यसी॒ पनी॑यसी ते ते॒ पनी॑यसी । \newline
16. पनी॑यसी स॒मिथ् स॒मित् पनी॑यसी॒ पनी॑यसी स॒मित् । \newline
17. स॒मिद् दी॒दय॑ति दी॒दय॑ति स॒मिथ् स॒मिद् दी॒दय॑ति । \newline
18. स॒मिदिति॑ सं - इत् । \newline
19. दी॒दय॑ति॒ द्यवि॒ द्यवि॑ दी॒दय॑ति दी॒दय॑ति॒ द्यवि॑ । \newline
20. द्यवीष॒ मिष॒म् द्यवि॒ द्यवीष᳚म् । \newline
21. इषꣳ॑ स्तो॒तृभ्यः॑ स्तो॒तृभ्य॒ इष॒ मिषꣳ॑ स्तो॒तृभ्यः॑ । \newline
22. स्तो॒तृभ्य॒ आ स्तो॒तृभ्यः॑ स्तो॒तृभ्य॒ आ । \newline
23. स्तो॒तृभ्य॒ इति॑ स्तो॒तृ - भ्यः॒ । \newline
24. आ भ॑र भ॒रा भ॑र । \newline
25. भ॒रेति॑ भर । \newline
26. आ ते॑ त॒ आ ते᳚ । \newline
27. ते॒ अ॒ग्ने॒ अ॒ग्ने॒ ते॒ ते॒ अ॒ग्ने॒ । \newline
28. अ॒ग्न॒ ऋ॒च र्‌चा ऽग्ने॑ अग्न ऋ॒चा । \newline
29. ऋ॒चा ह॒विर्. ह॒विर्. ऋ॒च र्‌चा ह॒विः । \newline
30. ह॒विः शु॒क्रस्य॑ शु॒क्रस्य॑ ह॒विर्. ह॒विः शु॒क्रस्य॑ । \newline
31. शु॒क्रस्य॑ ज्योतिषो ज्योतिषः शु॒क्रस्य॑ शु॒क्रस्य॑ ज्योतिषः । \newline
32. ज्यो॒ति॒ष॒ स्प॒ते॒ प॒ते॒ ज्यो॒ति॒षो॒ ज्यो॒ति॒ष॒ स्प॒ते॒ । \newline
33. प॒त॒ इति॑ पते । \newline
34. सुश्च॑न्द्र॒ दस्म॒ दस्म॒ सुश्च॑न्द्र॒ सुश्च॑न्द्र॒ दस्म॑ । \newline
35. सुश्च॒न्द्रेति॒ सु - च॒न्द्र॒ । \newline
36. दस्म॒ विश्प॑ते॒ विश्प॑ते॒ दस्म॒ दस्म॒ विश्प॑ते । \newline
37. विश्प॑ते॒ हव्य॑वा॒ड्-ढव्य॑वा॒ड् विश्प॑ते॒ विश्प॑ते॒ हव्य॑वाट् । \newline
38. हव्य॑वा॒ट् तुभ्य॒म् तुभ्यꣳ॒॒ हव्य॑वा॒ड्-ढव्य॑वा॒ट् तुभ्य᳚म् । \newline
39. हव्य॑वा॒डिति॒ हव्य॑ - वा॒ट् । \newline
40. तुभ्यꣳ॑ हूयते हूयते॒ तुभ्य॒म् तुभ्यꣳ॑ हूयते । \newline
41. हू॒य॒त॒ इष॒ मिषꣳ॑ हूयते हूयत॒ इष᳚म् । \newline
42. इषꣳ॑ स्तो॒तृभ्यः॑ स्तो॒तृभ्य॒ इष॒ मिषꣳ॑ स्तो॒तृभ्यः॑ । \newline
43. स्तो॒तृभ्य॒ आ स्तो॒तृभ्यः॑ स्तो॒तृभ्य॒ आ । \newline
44. स्तो॒तृभ्य॒ इति॑ स्तो॒तृ - भ्यः॒ । \newline
45. आ भ॑र भ॒रा भ॑र । \newline
46. भ॒रेति॑ भर । \newline
47. उ॒भे सु॑श्चन्द्र सुश्चन्द् रो॒भे उ॒भे सु॑श्चन्द्र । \newline
48. उ॒भे इत्यु॒भे । \newline
49. सु॒श्च॒न्द्र॒ स॒र्पिषः॑ स॒र्पिषः॑ सुश्चन्द्र सुश्चन्द्र स॒र्पिषः॑ । \newline
50. सु॒श्च॒न्द्रेति॑ सु - च॒न्द्र॒ । \newline
51. स॒र्पिषो॒ दर्वी॒ दर्वी॑ स॒र्पिषः॑ स॒र्पिषो॒ दर्वी᳚ । \newline
52. दर्वी᳚ श्रीणीषे श्रीणीषे॒ दर्वी॒ दर्वी᳚ श्रीणीषे । \newline
53. दर्वी॒ इति॒ दर्वी᳚ । \newline
54. श्री॒णी॒ष॒ आ॒स न्या॒सनि॑ श्रीणीषे श्रीणीष आ॒सनि॑ । \newline
55. आ॒सनीत्या॒सनि॑ । \newline
56. उ॒तो नो॑ न उ॒तो उ॒तो नः॑ । \newline
57. उ॒तो इत्यु॒तो । \newline
58. न॒ उदुन् नो॑ न॒ उत् । \newline
59. उत् पु॑पूर्याः पुपूर्या॒ उदुत् पु॑पूर्याः । \newline
60. पु॒पू॒र्या॒ उ॒क्थे षू॒क्थेषु॑ पुपूर्याः पुपूर्या उ॒क्थेषु॑ । \newline

\textbf{Ghana Paata } \newline

1. र॒क्षसो॑ दह दह र॒क्षसो॑ र॒क्षसो॑ दह॒ प्रति॒ प्रति॑ दह र॒क्षसो॑ र॒क्षसो॑ दह॒ प्रति॑ । \newline
2. द॒ह॒ प्रति॒ प्रति॑ दह दह॒ प्रति॑ । \newline
3. प्रतीति॒ प्रति॑ । \newline
4. आ ते॑ त॒ आ ते॑ अग्ने अग्ने त॒ आ ते॑ अग्ने । \newline
5. ते॒ अ॒ग्ने॒ अ॒ग्ने॒ ते॒ ते॒ अ॒ग्न॒ इ॒धी॒म॒ही॒ धी॒म॒ ह्य॒ग्ने॒ ते॒ ते॒ अ॒ग्न॒ इ॒धी॒म॒हि॒ । \newline
6. अ॒ग्न॒ इ॒धी॒म॒ही॒ धी॒म॒ ह्य॒ग्ने॒ अ॒ग्न॒ इ॒धी॒म॒हि॒ द्यु॒मन्त॑म् द्यु॒मन्त॑ मिधीमह्यग्ने अग्न इधीमहि द्यु॒मन्त᳚म् । \newline
7. इ॒धी॒म॒हि॒ द्यु॒मन्त॑म् द्यु॒मन्त॑ मिधीमही धीमहि द्यु॒मन्त॑म् देव देव द्यु॒मन्त॑ मिधीमही धीमहि द्यु॒मन्त॑म् देव । \newline
8. द्यु॒मन्त॑म् देव देव द्यु॒मन्त॑म् द्यु॒मन्त॑म् देवा॒जर॑ म॒जर॑म् देव द्यु॒मन्त॑म् द्यु॒मन्त॑म् देवा॒जर᳚म् । \newline
9. द्यु॒मन्त॒मिति॑ द्यु - मन्त᳚म् । \newline
10. दे॒वा॒जर॑ म॒जर॑म् देव देवा॒जर᳚म् । \newline
11. अ॒जर॒मित्य॒जर᳚म् । \newline
12. यद्ध॑ ह॒ यद् यद्ध॒ स्या स्या ह॒ यद् यद्ध॒ स्या । \newline
13. ह॒ स्या स्या ह॑ ह॒ स्या ते॑ ते॒ स्या ह॑ ह॒ स्या ते᳚ । \newline
14. स्या ते॑ ते॒ स्या स्या ते॒ पनी॑यसी॒ पनी॑यसी ते॒ स्या स्या ते॒ पनी॑यसी । \newline
15. ते॒ पनी॑यसी॒ पनी॑यसी ते ते॒ पनी॑यसी स॒मिथ् स॒मित् पनी॑यसी ते ते॒ पनी॑यसी स॒मित् । \newline
16. पनी॑यसी स॒मिथ् स॒मित् पनी॑यसी॒ पनी॑यसी स॒मिद् दी॒दय॑ति दी॒दय॑ति स॒मित् पनी॑यसी॒ पनी॑यसी स॒मिद् दी॒दय॑ति । \newline
17. स॒मिद् दी॒दय॑ति दी॒दय॑ति स॒मिथ् स॒मिद् दी॒दय॑ति॒ द्यवि॒ द्यवि॑ दी॒दय॑ति स॒मिथ् स॒मिद् दी॒दय॑ति॒ द्यवि॑ । \newline
18. स॒मिदिति॑ सं - इत् । \newline
19. दी॒दय॑ति॒ द्यवि॒ द्यवि॑ दी॒दय॑ति दी॒दय॑ति॒ द्यवीष॒ मिष॒म् द्यवि॑ दी॒दय॑ति दी॒दय॑ति॒ द्यवीष᳚म् । \newline
20. द्यवीष॒ मिष॒म् द्यवि॒ द्यवीषꣳ॑ स्तो॒तृभ्यः॑ स्तो॒तृभ्य॒ इष॒म् द्यवि॒ द्यवीषꣳ॑ स्तो॒तृभ्यः॑ । \newline
21. इषꣳ॑ स्तो॒तृभ्यः॑ स्तो॒तृभ्य॒ इष॒ मिषꣳ॑ स्तो॒तृभ्य॒ आ स्तो॒तृभ्य॒ इष॒ मिषꣳ॑ स्तो॒तृभ्य॒ आ । \newline
22. स्तो॒तृभ्य॒ आ स्तो॒तृभ्यः॑ स्तो॒तृभ्य॒ आ भ॑र भ॒रा स्तो॒तृभ्यः॑ स्तो॒तृभ्य॒ आ भ॑र । \newline
23. स्तो॒तृभ्य॒ इति॑ स्तो॒तृ - भ्यः॒ । \newline
24. आ भ॑र भ॒रा भ॑र । \newline
25. भ॒रेति॑ भर । \newline
26. आ ते॑ त॒ आ ते॑ अग्ने अग्ने त॒ आ ते॑ अग्ने । \newline
27. ते॒ अ॒ग्ने॒ अ॒ग्ने॒ ते॒ ते॒ अ॒ग्न॒ ऋ॒च र्‌चा ऽग्ने॑ ते ते अग्न ऋ॒चा । \newline
28. अ॒ग्न॒ ऋ॒च र्‌चा ऽग्ने॑ अग्न ऋ॒चा ह॒विर्. ह॒विर्. ऋ॒चा ऽग्ने॑ अग्न ऋ॒चा ह॒विः । \newline
29. ऋ॒चा ह॒विर्. ह॒विर्. ऋ॒च र्चा ह॒विः शु॒क्रस्य॑ शु॒क्रस्य॑ ह॒विर्. ऋ॒चर्चा ह॒विः शु॒क्रस्य॑ । \newline
30. ह॒विः शु॒क्रस्य॑ शु॒क्रस्य॑ ह॒विर्. ह॒विः शु॒क्रस्य॑ ज्योतिषो ज्योतिषः शु॒क्रस्य॑ ह॒विर्. ह॒विः शु॒क्रस्य॑ ज्योतिषः । \newline
31. शु॒क्रस्य॑ ज्योतिषो ज्योतिषः शु॒क्रस्य॑ शु॒क्रस्य॑ ज्योतिष स्पते पते ज्योतिषः शु॒क्रस्य॑ शु॒क्रस्य॑ ज्योतिष स्पते । \newline
32. ज्यो॒ति॒ष॒ स्प॒ते॒ प॒ते॒ ज्यो॒ति॒षो॒ ज्यो॒ति॒ष॒ स्प॒ते॒ । \newline
33. प॒त॒ इति॑ पते । \newline
34. सुश्च॑न्द्र॒ दस्म॒ दस्म॒ सुश्च॑न्द्र॒ सुश्च॑न्द्र॒ दस्म॒ विश्प॑ते॒ विश्प॑ते॒ दस्म॒ सुश्च॑न्द्र॒ सुश्च॑न्द्र॒ दस्म॒ विश्प॑ते । \newline
35. सुश्च॒न्द्रेति॒ सु - च॒न्द्र॒ । \newline
36. दस्म॒ विश्प॑ते॒ विश्प॑ते॒ दस्म॒ दस्म॒ विश्प॑ते॒ हव्य॑वा॒ड् ढव्य॑वा॒ड् विश्प॑ते॒ दस्म॒ दस्म॒ विश्प॑ते॒ हव्य॑वाट् । \newline
37. विश्प॑ते॒ हव्य॑वा॒ड् ढव्य॑वा॒ड् विश्प॑ते॒ विश्प॑ते॒ हव्य॑वा॒ट् तुभ्य॒म् तुभ्यꣳ॒॒ हव्य॑वा॒ड् विश्प॑ते॒ विश्प॑ते॒ हव्य॑वा॒ट् तुभ्य᳚म् । \newline
38. हव्य॑वा॒ट् तुभ्य॒म् तुभ्यꣳ॒॒ हव्य॑वा॒ड् ढव्य॑वा॒ट् तुभ्यꣳ॑ हूयते हूयते॒ तुभ्यꣳ॒॒ हव्य॑वा॒ड् ढव्य॑वा॒ट् तुभ्यꣳ॑ हूयते । \newline
39. हव्य॑वा॒डिति॒ हव्य॑ - वा॒ट् । \newline
40. तुभ्यꣳ॑ हूयते हूयते॒ तुभ्य॒म् तुभ्यꣳ॑ हूयत॒ इष॒ मिषꣳ॑ हूयते॒ तुभ्य॒म् तुभ्यꣳ॑ हूयत॒ इष᳚म् । \newline
41. हू॒य॒त॒ इष॒ मिषꣳ॑ हूयते हूयत॒ इषꣳ॑ स्तो॒तृभ्यः॑ स्तो॒तृभ्य॒ इषꣳ॑ हूयते हूयत॒ इषꣳ॑ स्तो॒तृभ्यः॑ । \newline
42. इषꣳ॑ स्तो॒तृभ्यः॑ स्तो॒तृभ्य॒ इष॒ मिषꣳ॑ स्तो॒तृभ्य॒ आ स्तो॒तृभ्य॒ इष॒ मिषꣳ॑ स्तो॒तृभ्य॒ आ । \newline
43. स्तो॒तृभ्य॒ आ स्तो॒तृभ्यः॑ स्तो॒तृभ्य॒ आ भ॑र भ॒रा स्तो॒तृभ्यः॑ स्तो॒तृभ्य॒ आ भ॑र । \newline
44. स्तो॒तृभ्य॒ इति॑ स्तो॒तृ - भ्यः॒ । \newline
45. आ भ॑र भ॒रा भ॑र । \newline
46. भ॒रेति॑ भर । \newline
47. उ॒भे सु॑श्चन्द्र सुश्चन्द्रो॒भे उ॒भे सु॑श्चन्द्र स॒र्पिषः॑ स॒र्पिषः॑ सुश्चन्द्रो॒भे उ॒भे सु॑श्चन्द्र स॒र्पिषः॑ । \newline
48. उ॒भे इत्यु॒भे । \newline
49. सु॒श्च॒न्द्र॒ स॒र्पिषः॑ स॒र्पिषः॑ सुश्चन्द्र सुश्चन्द्र स॒र्पिषो॒ दर्वी॒ दर्वी॑ स॒र्पिषः॑ सुश्चन्द्र सुश्चन्द्र स॒र्पिषो॒ दर्वी᳚ । \newline
50. सु॒श्च॒न्द्रेति॑ सु - च॒न्द्र॒ । \newline
51. स॒र्पिषो॒ दर्वी॒ दर्वी॑ स॒र्पिषः॑ स॒र्पिषो॒ दर्वी᳚ श्रीणीषे श्रीणीषे॒ दर्वी॑ स॒र्पिषः॑ स॒र्पिषो॒ दर्वी᳚ श्रीणीषे । \newline
52. दर्वी᳚ श्रीणीषे श्रीणीषे॒ दर्वी॒ दर्वी᳚ श्रीणीष आ॒स न्या॒सनि॑ श्रीणीषे॒ दर्वी॒ दर्वी᳚ श्रीणीष आ॒सनि॑ । \newline
53. दर्वी॒ इति॒ दर्वी᳚ । \newline
54. श्री॒णी॒ष॒ आ॒स न्या॒सनि॑ श्रीणीषे श्रीणीष आ॒सनि॑ । \newline
55. आ॒सनीत्या॒सनि॑ । \newline
56. उ॒तो नो॑ न उ॒तो उ॒तो न॒ उदुन् न॑ उ॒तो उ॒तो न॒ उत् । \newline
57. उ॒तो इत्यु॒तो । \newline
58. न॒ उदुन् नो॑ न॒ उत् पु॑पूर्याः पुपूर्या॒ उन् नो॑ न॒ उत् पु॑पूर्याः । \newline
59. उत् पु॑पूर्याः पुपूर्या॒ उदुत् पु॑पूर्या उ॒क्थे षू॒क्थेषु॑ पुपूर्या॒ उदुत् पु॑पूर्या उ॒क्थेषु॑ । \newline
60. पु॒पू॒र्या॒ उ॒क्थे षू॒क्थेषु॑ पुपूर्याः पुपूर्या उ॒क्थेषु॑ शवसः शवस उ॒क्थेषु॑ पुपूर्याः पुपूर्या उ॒क्थेषु॑ शवसः । \newline
\pagebreak
\markright{ TS 4.4.4.7  \hfill https://www.vedavms.in \hfill}

\section{ TS 4.4.4.7 }

\textbf{TS 4.4.4.7 } \newline
\textbf{Samhita Paata} \newline

उ॒क्थेषु॑ शवसस्पत॒ इषꣳ॑ स्तो॒तृभ्य॒ आ भ॑र ॥ अग्ने॒ तम॒द्याश्वं॒ न स्तोमैः॒ क्रतुं॒ न भ॒द्रꣳ हृ॑दि॒स्पृशं᳚ । ऋ॒द्ध्यामा॑ त॒ ओहैः᳚ ॥ अधा॒ ह्य॑ग्ने॒ क्रतो᳚र्भ॒द्रस्य॒ दक्ष॑स्य सा॒धोः । र॒थीर्.ऋ॒तस्य॑ बृह॒तो ब॒भूथ॑ ॥ आ॒भिष्टे॑ अ॒द्य गी॒र्भिर्गृ॒णन्तोऽग्ने॒ दाशे॑म । प्र ते॑ दि॒वो न स्त॑नयन्ति॒ शुष्माः᳚ ॥ ए॒भिर्नो॑ अ॒र्कैर्भवा॑ नो अ॒र्वाङ्ख् - [  ] \newline

\textbf{Pada Paata} \newline

उ॒क्थेषु॑ । श॒व॒सः॒ । प॒ते॒ । इष᳚म् । स्तो॒तृभ्य॒ इति॑ स्तो॒तृ - भ्यः॒ । एति॑ । भ॒र॒ ॥ अग्ने᳚ । तम् । अ॒द्य । अश्व᳚म् । न । स्तोमैः᳚ । क्रतु᳚म् । न । भ॒द्रम् । हृ॒दि॒स्पृश॒मिति॑ हृदि - स्पृश᳚म् ॥ ऋ॒द्ध्याम॑ । ते॒ । ओहैः᳚ ॥ अध॑ । हि । अ॒ग्ने॒ । क्रतोः᳚ । भ॒द्रस्य॑ । दक्ष॑स्य । सा॒धोः ॥ र॒थीः । ऋ॒तस्य॑ । बृ॒ह॒तः । ब॒भूथ॑ ॥ आ॒भिः । ते॒ । अ॒द्य । गी॒र्भिः । गृ॒णन्तः॑ । अग्ने᳚ । दाशे॑म ॥ प्रेति॑ । ते॒ । दि॒वः । न । स्त॒न॒य॒न्ति॒ । शुष्माः᳚ ॥ ए॒भिः । नः॒ । अ॒र्कैः । भव॑ । नः॒ । अ॒र्वाङ् ।  \newline


\textbf{Krama Paata} \newline

उ॒क्थेषु॑ शवसः । श॒व॒स॒स्प॒ते॒ । प॒त॒ इष᳚म् । इषꣳ॑ स्तो॒तृभ्यः॑ । स्तो॒तृभ्य॒ आ । स्तो॒तृभ्य॒ इति॑ स्तो॒तृ - भ्यः॒ । आ भ॑र । भ॒रेति॑ भर ॥ अग्ने॒ तम् । तम॒द्य । अ॒द्याश्व᳚म् । अश्व॒म् न । न स्तोमैः᳚ । स्तोमैः॒ क्रतु᳚म् । क्रतु॒म् न । न भ॒द्रम् । भ॒द्रꣳ हृ॑दि॒स्पृश᳚म् । हृ॒दि॒स्पृश॒मिति॑ हृदि - स्पृश᳚म् ॥ ऋ॒द्ध्यामा॑ ते । त॒ ओहैः᳚ । ओहै॒रित्योहैः᳚ ॥ अधा॒ हि । ह्य॑ग्ने । अ॒ग्ने॒ क्रतोः᳚ । क्रतो᳚र् भ॒द्रस्य॑ । भ॒द्रस्य॒ दक्ष॑स्य । दक्ष॑स्य सा॒धोः । सा॒धोरिति॑ सा॒धोः ॥ र॒थीर्. ऋ॒तस्य॑ । ऋ॒तस्य॑ बृह॒तः । बृ॒ह॒तो ब॒भूथ॑ । ब॒भूथेति॑ ब॒भूथ॑ ॥ आ॒भिष्टे᳚ । ते॒ अ॒द्य । अ॒द्य गी॒र्भिः । गी॒र्भिर् गृ॒णन्तः॑ । गृ॒णन्तोऽग्ने᳚ । अग्ने॒ दाशे॑म । दाशे॒मेति॒ दाशे॑म ॥ प्र ते᳚ । ते॒ दि॒वः । दि॒वो न । न स्त॑नयन्ति । स्त॒न॒य॒न्ति॒ शुष्माः᳚ । शुष्मा॒ इति॒ शुष्माः᳚ ॥ ए॒भिर् नः॑ । नो॒ अ॒र्कैः । अ॒र्कैर् भव॑ । भवा॑ नः । नो॒ अ॒र्वाङ् । अ॒र्वाङ्ख् सुवः॑ \newline

\textbf{Jatai Paata} \newline

1. उ॒क्थेषु॑ शवसः शवस उ॒क्थे षू॒क्थेषु॑ शवसः । \newline
2. श॒व॒स॒ स्प॒ते॒ प॒ते॒ श॒व॒सः॒ श॒व॒स॒ स्प॒ते॒ । \newline
3. प॒त॒ इष॒ मिष॑म् पते पत॒ इष᳚म् । \newline
4. इषꣳ॑ स्तो॒तृभ्यः॑ स्तो॒तृभ्य॒ इष॒ मिषꣳ॑ स्तो॒तृभ्यः॑ । \newline
5. स्तो॒तृभ्य॒ आ स्तो॒तृभ्यः॑ स्तो॒तृभ्य॒ आ । \newline
6. स्तो॒तृभ्य॒ इति॑ स्तो॒तृ - भ्यः॒ । \newline
7. आ भ॑र भ॒रा भ॑र । \newline
8. भ॒रेति॑ भर । \newline
9. अग्ने॒ तम् त मग्ने ऽग्ने॒ तम् । \newline
10. त म॒द्याद्य तम् त म॒द्य । \newline
11. अ॒द्याश्व॒ मश्व॑ म॒द्या द्याश्व᳚म् । \newline
12. अश्व॒म् न नाश्व॒ मश्व॒म् न । \newline
13. न स्तोमैः॒ स्तोमै॒र् न न स्तोमैः᳚ । \newline
14. स्तोमैः॒ क्रतु॒म् क्रतुꣳ॒॒ स्तोमैः॒ स्तोमैः॒ क्रतु᳚म् । \newline
15. क्रतु॒म् न न क्रतु॒म् क्रतु॒म् न । \newline
16. न भ॒द्रम् भ॒द्रम् न न भ॒द्रम् । \newline
17. भ॒द्रꣳ हृ॑दि॒स्पृशꣳ॑ हृदि॒स्पृश॑म् भ॒द्रम् भ॒द्रꣳ हृ॑दि॒स्पृश᳚म् । \newline
18. हृ॒दि॒स्पृश॒मिति॑ हृदि - स्पृश᳚म् । \newline
19. ऋ॒द्ध्यामा॑ ते त ऋ॒द्ध्याम॒ र्‌द्ध्यामा॑ ते । \newline
20. त॒ ओहै॒ रोहै᳚ स्ते त॒ ओहैः᳚ । \newline
21. ओहै॒रित्योहैः᳚ । \newline
22. अधा॒ हि ह्य धाधा॒ हि । \newline
23. ह्य॑ग्ने अग्ने॒ हि ह्य॑ग्ने । \newline
24. अ॒ग्ने॒ क्रतोः॒ क्रतो॑ रग्ने अग्ने॒ क्रतोः᳚ । \newline
25. क्रतो᳚र् भ॒द्रस्य॑ भ॒द्रस्य॒ क्रतोः॒ क्रतो᳚र् भ॒द्रस्य॑ । \newline
26. भ॒द्रस्य॒ दक्ष॑स्य॒ दक्ष॑स्य भ॒द्रस्य॑ भ॒द्रस्य॒ दक्ष॑स्य । \newline
27. दक्ष॑स्य सा॒धोः सा॒धोर् दक्ष॑स्य॒ दक्ष॑स्य सा॒धोः । \newline
28. सा॒धोरिति॑ सा॒धोः । \newline
29. र॒थीर्. ऋ॒तस्य॒ र्‌तस्य॑ र॒थी र॒थीर्. ऋ॒तस्य॑ । \newline
30. ऋ॒तस्य॑ बृह॒तो बृ॑ह॒त ऋ॒तस्य॒ र्‌तस्य॑ बृह॒तः । \newline
31. बृ॒ह॒तो ब॒भूथ॑ ब॒भूथ॑ बृह॒तो बृ॑ह॒तो ब॒भूथ॑ । \newline
32. ब॒भूथेति॑ ब॒भूथ॑ । \newline
33. आ॒भिष्टे॑ त आ॒भि रा॒भिष्टे᳚ । \newline
34. ते॒ अ॒द्याद्य ते॑ ते अ॒द्य । \newline
35. अ॒द्य गी॒र्भिर् गी॒र्भि र॒द्याद्य गी॒र्भिः । \newline
36. गी॒र्भिर् गृ॒णन्तो॑ गृ॒णन्तो॑ गी॒र्भिर् गी॒र्भिर् गृ॒णन्तः॑ । \newline
37. गृ॒णन्तो ऽग्ने ऽग्ने॑ गृ॒णन्तो॑ गृ॒णन्तो ऽग्ने᳚ । \newline
38. अग्ने॒ दाशे॑म॒ दाशे॒माग्ने ऽग्ने॒ दाशे॑म । \newline
39. दाशे॒मेति॒ दाशे॑म । \newline
40. प्र ते॑ ते॒ प्र प्र ते᳚ । \newline
41. ते॒ दि॒वो दि॒व स्ते॑ ते दि॒वः । \newline
42. दि॒वो न न दि॒वो दि॒वो न । \newline
43. न स्त॑नयन्ति स्तनयन्ति॒ न न स्त॑नयन्ति । \newline
44. स्त॒न॒य॒न्ति॒ शुष्माः॒ शुष्माः᳚ स्तनयन्ति स्तनयन्ति॒ शुष्माः᳚ । \newline
45. शुष्मा॒ इति॒ शुष्माः᳚ । \newline
46. ए॒भिर् नो॑ न ए॒भि रे॒भिर् नः॑ । \newline
47. नो॒ अ॒र्कै र॒र्कैर् नो॑ नो अ॒र्कैः । \newline
48. अ॒र्कैर् भव॒ भवा॒र्कै र॒र्कैर् भव॑ । \newline
49. भवा॑ नो नो॒ भव॒ भवा॑ नः । \newline
50. नो॒ अ॒र्वाङ् ङ॒र्वाङ् नो॑ नो अ॒र्वाङ् । \newline
51. अ॒र्वाङ् ख्सुवः॒ सुव॑ र॒र्वाङ् ङ॒र्वाङ् ख्सुवः॑ । \newline

\textbf{Ghana Paata } \newline

1. उ॒क्थेषु॑ शवसः शवस उ॒क्थे षू॒क्थेषु॑ शवस स्पते पते शवस उ॒क्थे षू॒क्थेषु॑ शवस स्पते । \newline
2. श॒व॒स॒ स्प॒ते॒ प॒ते॒ श॒व॒सः॒ श॒व॒स॒ स्प॒त॒ इष॒ मिष॑म् पते शवसः शवस स्पत॒ इष᳚म् । \newline
3. प॒त॒ इष॒ मिष॑म् पते पत॒ इषꣳ॑ स्तो॒तृभ्यः॑ स्तो॒तृभ्य॒ इष॑म् पते पत॒ इषꣳ॑ स्तो॒तृभ्यः॑ । \newline
4. इषꣳ॑ स्तो॒तृभ्यः॑ स्तो॒तृभ्य॒ इष॒ मिषꣳ॑ स्तो॒तृभ्य॒ आ स्तो॒तृभ्य॒ इष॒ मिषꣳ॑ स्तो॒तृभ्य॒ आ । \newline
5. स्तो॒तृभ्य॒ आ स्तो॒तृभ्यः॑ स्तो॒तृभ्य॒ आ भ॑र भ॒रा स्तो॒तृभ्यः॑ स्तो॒तृभ्य॒ आ भ॑र । \newline
6. स्तो॒तृभ्य॒ इति॑ स्तो॒तृ - भ्यः॒ । \newline
7. आ भ॑र भ॒रा भ॑र । \newline
8. भ॒रेति॑ भर । \newline
9. अग्ने॒ तम् त मग्ने ऽग्ने॒ त म॒द्याद्य त मग्ने ऽग्ने॒ त म॒द्य । \newline
10. त म॒द्याद्य तम् त म॒द्याश्व॒ मश्व॑ म॒द्य तम् त म॒द्याश्व᳚म् । \newline
11. अ॒द्याश्व॒ मश्व॑ म॒द्याद्या श्व॒म् न नाश्व॑ म॒द्याद्या श्व॒म् न । \newline
12. अश्व॒म् न नाश्व॒ मश्व॒म् न स्तोमैः॒ स्तोमै॒र् नाश्व॒ मश्व॒म् न स्तोमैः᳚ । \newline
13. न स्तोमैः॒ स्तोमै॒र् न न स्तोमैः॒ क्रतु॒म् क्रतुꣳ॒॒ स्तोमै॒र् न न स्तोमैः॒ क्रतु᳚म् । \newline
14. स्तोमैः॒ क्रतु॒म् क्रतुꣳ॒॒ स्तोमैः॒ स्तोमैः॒ क्रतु॒म् न न क्रतुꣳ॒॒ स्तोमैः॒ स्तोमैः॒ क्रतु॒म् न । \newline
15. क्रतु॒म् न न क्रतु॒म् क्रतु॒म् न भ॒द्रम् भ॒द्रम् न क्रतु॒म् क्रतु॒म् न भ॒द्रम् । \newline
16. न भ॒द्रम् भ॒द्रम् न न भ॒द्रꣳ हृ॑दि॒स्पृशꣳ॑ हृदि॒स्पृश॑म् भ॒द्रम् न न भ॒द्रꣳ हृ॑दि॒स्पृश᳚म् । \newline
17. भ॒द्रꣳ हृ॑दि॒स्पृशꣳ॑ हृदि॒स्पृश॑म् भ॒द्रम् भ॒द्रꣳ हृ॑दि॒स्पृश᳚म् । \newline
18. हृ॒दि॒स्पृश॒मिति॑ हृदि - स्पृश᳚म् । \newline
19. ऋ॒द्ध्यामा॑ ते त ऋ॒द्ध्याम॒ र्‌द्ध्यामा॑ त॒ ओहै॒ रोहै᳚ स्त ऋ॒द्ध्याम॒ र्‌द्ध्यामा॑ त॒ ओहैः᳚ । \newline
20. त॒ ओहै॒ रोहै᳚ स्ते त॒ ओहैः᳚ । \newline
21. ओहै॒रित्योहैः᳚ । \newline
22. अधा॒ हि ह्यधाधा॒ ह्य॑ग्ने अग्ने॒ ह्यधाधा॒ ह्य॑ग्ने । \newline
23. ह्य॑ग्ने अग्ने॒ हि ह्य॑ग्ने॒ क्रतोः॒ क्रतो॑ रग्ने॒ हि ह्य॑ग्ने॒ क्रतोः᳚ । \newline
24. अ॒ग्ने॒ क्रतोः॒ क्रतो॑ रग्ने अग्ने॒ क्रतो᳚र् भ॒द्रस्य॑ भ॒द्रस्य॒ क्रतो॑ रग्ने अग्ने॒ क्रतो᳚र् भ॒द्रस्य॑ । \newline
25. क्रतो᳚र् भ॒द्रस्य॑ भ॒द्रस्य॒ क्रतोः॒ क्रतो᳚र् भ॒द्रस्य॒ दक्ष॑स्य॒ दक्ष॑स्य भ॒द्रस्य॒ क्रतोः॒ क्रतो᳚र् भ॒द्रस्य॒ दक्ष॑स्य । \newline
26. भ॒द्रस्य॒ दक्ष॑स्य॒ दक्ष॑स्य भ॒द्रस्य॑ भ॒द्रस्य॒ दक्ष॑स्य सा॒धोः सा॒धोर् दक्ष॑स्य भ॒द्रस्य॑ भ॒द्रस्य॒ दक्ष॑स्य सा॒धोः । \newline
27. दक्ष॑स्य सा॒धोः सा॒धोर् दक्ष॑स्य॒ दक्ष॑स्य सा॒धोः । \newline
28. सा॒धोरिति॑ सा॒धोः । \newline
29. र॒थीर्. ऋ॒तस्य॒ र्‌‍तस्य॑ र॒थी र॒थीर्. ऋ॒तस्य॑ बृह॒तो बृ॑ह॒त ऋ॒तस्य॑ र॒थी र॒थीर्. ऋ॒तस्य॑ बृह॒तः । \newline
30. ऋ॒तस्य॑ बृह॒तो बृ॑ह॒त ऋ॒तस्य॒ र्‌तस्य॑ बृह॒तो ब॒भूथ॑ ब॒भूथ॑ बृह॒त ऋ॒तस्य॒ र्‌तस्य॑ बृह॒तो ब॒भूथ॑ । \newline
31. बृ॒ह॒तो ब॒भूथ॑ ब॒भूथ॑ बृह॒तो बृ॑ह॒तो ब॒भूथ॑ । \newline
32. ब॒भूथेति॑ ब॒भूथ॑ । \newline
33. आ॒भिष्टे॑ त आ॒भि रा॒भिष्टे॑ अ॒द्याद्य त॑ आ॒भि रा॒भिष्टे॑ अ॒द्य । \newline
34. ते॒ अ॒द्याद्य ते॑ ते अ॒द्य गी॒र्भिर् गी॒र्भि र॒द्य ते॑ ते अ॒द्य गी॒र्भिः । \newline
35. अ॒द्य गी॒र्भिर् गी॒र्भि र॒द्याद्य गी॒र्भिर् गृ॒णन्तो॑ गृ॒णन्तो॑ गी॒र्भि र॒द्याद्य गी॒र्भिर् गृ॒णन्तः॑ । \newline
36. गी॒र्भिर् गृ॒णन्तो॑ गृ॒णन्तो॑ गी॒र्भिर् गी॒र्भिर् गृ॒णन्तो ऽग्ने ऽग्ने॑ गृ॒णन्तो॑ गी॒र्भिर् गी॒र्भिर् गृ॒णन्तो ऽग्ने᳚ । \newline
37. गृ॒णन्तो ऽग्ने ऽग्ने॑ गृ॒णन्तो॑ गृ॒णन्तो ऽग्ने॒ दाशे॑म॒ दाशे॒माग्ने॑ गृ॒णन्तो॑ गृ॒णन्तो ऽग्ने॒ दाशे॑म । \newline
38. अग्ने॒ दाशे॑म॒ दाशे॒माग्ने ऽग्ने॒ दाशे॑म । \newline
39. दाशे॒मेति॒ दाशे॑म । \newline
40. प्र ते॑ ते॒ प्र प्र ते॑ दि॒वो दि॒व स्ते॒ प्र प्र ते॑ दि॒वः । \newline
41. ते॒ दि॒वो दि॒व स्ते॑ ते दि॒वो न न दि॒व स्ते॑ ते दि॒वो न । \newline
42. दि॒वो न न दि॒वो दि॒वो न स्त॑नयन्ति स्तनयन्ति॒ न दि॒वो दि॒वो न स्त॑नयन्ति । \newline
43. न स्त॑नयन्ति स्तनयन्ति॒ न न स्त॑नयन्ति॒ शुष्माः॒ शुष्माः᳚ स्तनयन्ति॒ न न स्त॑नयन्ति॒ शुष्माः᳚ । \newline
44. स्त॒न॒य॒न्ति॒ शुष्माः॒ शुष्माः᳚ स्तनयन्ति स्तनयन्ति॒ शुष्माः᳚ । \newline
45. शुष्मा॒ इति॒ शुष्माः᳚ । \newline
46. ए॒भिर् नो॑ न ए॒भि रे॒भिर् नो॑ अ॒र्कै र॒र्कैर् न॑ ए॒भि रे॒भिर् नो॑ अ॒र्कैः । \newline
47. नो॒ अ॒र्कै र॒र्कैर् नो॑ नो अ॒र्कैर् भव॒ भवा॒र्कैर् नो॑ नो अ॒र्कैर् भव॑ । \newline
48. अ॒र्कैर् भव॒ भवा॒र्कै र॒र्कैर् भवा॑ नो नो॒ भवा॒र्कै र॒र्कैर् भवा॑ नः । \newline
49. भवा॑ नो नो॒ भव॒ भवा॑ नो अ॒र्वाङ्ङ॒र्वाङ् नो॒ भव॒ भवा॑ नो अ॒र्वाङ् । \newline
50. नो॒ अ॒र्वाङ्ङ॒र्वाङ् नो॑ नो अ॒र्वाङ् ख्सुवः॒ सुव॑ र॒र्वाङ् नो॑ नो अ॒र्वाङ् ख्सुवः॑ । \newline
51. अ॒र्वाङ् ख्सुवः॒ सुव॑ र॒र्वाङ्ङ॒र्वाङ् ख्सुव॒र् न न सुव॑ र॒र्वाङ्ङ॒र्वाङ् ख्सुव॒र् न । \newline
\pagebreak
\markright{ TS 4.4.4.8  \hfill https://www.vedavms.in \hfill}

\section{ TS 4.4.4.8 }

\textbf{TS 4.4.4.8 } \newline
\textbf{Samhita Paata} \newline

सुव॒र्न ज्योतिः॑ । अग्ने॒ विश्वे॑भिः सु॒मना॒ अनी॑कैः ॥ अ॒ग्निꣳ होता॑रं मन्ये॒ दास्व॑न्तं॒ ॅवसोः᳚ सू॒नुꣳ सह॑सो जा॒तवे॑दसं । विप्रं॒ न जा॒तवे॑दसं । य ऊ॒र्द्ध्वया᳚ स्वद्ध्व॒रो दे॒वो दे॒वाच्या॑ कृ॒पा । घृ॒तस्य॒ विभ्रा᳚ष्टि॒मनु॑ शु॒क्रशो॑चिष आ॒जुह्वा॑नस्य स॒र्पिषः॑ ॥ अग्ने॒ त्वं नो॒ अन्त॑मः । उ॒त त्रा॒ता शि॒वो भ॑व वरू॒थ्यः॑ ॥ तं त्वा॑ शोचिष्ठ दीदिवः । सु॒म्नाय॑ नू॒नमी॑महे॒ सखि॑भ्यः ॥ वसु॑र॒ग्निर्वसु॑श्रवाः ( ) । अच्छा॑ नक्षि द्यु॒मत्त॑मो र॒यिं दाः᳚ ॥ \newline

\textbf{Pada Paata} \newline

सुवः॑ । न । ज्योतिः॑ ॥ अग्ने᳚ । विश्वे॑भिः । सु॒मना॒ इति॑ सु - मनाः᳚ । अनी॑कैः ॥ अ॒ग्निम् । होता॑रम् । म॒न्ये॒ । दास्व॑न्तम् । वसोः᳚ । सू॒नुम् । सह॑सः । जा॒तवे॑दस॒मिति॑ जा॒त - वे॒द॒स॒म् ॥ विप्र᳚म् । न । जा॒तवे॑दस॒मिति॑ जा॒त - वे॒द॒स॒म् ॥ यः । ऊ॒द्‌र्ध्वया᳚ । स्व॒द्ध्व॒र इति॑ सु - अ॒द्ध्व॒रः । दे॒वः । दे॒वाच्या᳚ । कृ॒पा ॥ घृ॒तस्य॑ । विभ्रा᳚ष्टि॒मिति॒ वि - भ्रा॒ष्टि॒म् । अन्विति॑ । शु॒क्रशो॑चिष॒ इति॑ शु॒क्र - शो॒चि॒षः॒ । आ॒जुह्वा॑न॒स्येत्या᳚ - जुह्वा॑नस्य । स॒र्पिषः॑ ॥ अग्ने᳚ । त्वम् । नः॒ । अन्त॑मः ॥ उ॒त । त्रा॒ता । शि॒वः । भ॒व॒ । व॒रू॒थ्यः॑ ॥ तम् । त्वा॒ । शो॒चि॒ष्ठ॒ । दी॒दि॒वः॒ ॥ सु॒म्नाय॑ । नू॒नम् । ई॒म॒हे॒ । सखि॑भ्य॒ इति॒ सखि॑ - भ्यः॒ ॥ वसुः॑ । अ॒ग्निः । वसु॑श्रवा॒ इति॒ वसु॑ - श्र॒वाः॒ ( ) ॥ अच्छ॑ । न॒क्षि॒ । द्यु॒मत्त॑म॒ इति॑ द्यु॒मत् - त॒मः॒ । र॒यिम् । दाः॒ ॥  \newline


\textbf{Krama Paata} \newline

सुव॒र् न । न ज्योतिः॑ । ज्योति॒रिति॒ ज्योतिः॑ ॥ अग्ने॒ विश्वे॑भिः । विश्वे॑भिः सु॒मनाः᳚ । सु॒मना॒ अनी॑कैः । सु॒मना॒ इति॑ सु - मनाः᳚ । अनी॑कै॒रित्यनी॑कैः ॥ अ॒ग्निꣳ होता॑रम् । होता॑रम् मन्ये । म॒न्ये॒ दास्व॑न्तम् । दास्व॑न्तं॒ ॅवसोः᳚ । वसोः᳚ सू॒नुम् । सू॒नुꣳ सह॑सः । सह॑सो जा॒तवे॑दसम् । जा॒तवे॑दस॒मिति॑ जा॒त - वे॒द॒स॒म् ॥ विप्र॒म् न । न जा॒तवे॑दसम् । जा॒तवे॑दस॒मिति॑ जा॒त - वे॒द॒स॒म् ॥ य ऊ॒र्द्ध्वया᳚ । ऊ॒र्द्ध्वया᳚ स्वद्ध्व॒रः । स्व॒द्ध्व॒रो दे॒वः । स्व॒द्ध्व॒र इति॑ सु - अ॒द्ध्व॒रः । दे॒वो दे॒वाच्या᳚ । दे॒वाच्या॑ कृ॒पा । कृ॒पेति॑ कृ॒पा ॥ घृ॒तस्य॒ विभ्रा᳚ष्टिम् । विभ्रा᳚ष्टि॒मनु॑ । विभ्रा᳚ष्टि॒मिति॒ वि - भ्रा॒ष्टि॒म् । अनु॑ शु॒क्रशो॑चिषः । शु॒क्रशो॑चिष आ॒जुह्वा॑नस्य । शु॒क्रशो॑चिष॒ इति॑ शु॒क्र - शो॒चि॒षः॒ । आ॒जुह्वा॑नस्य स॒र्पिषः॑ । आ॒जुह्वा॑न॒स्येत्या᳚ - जुह्वा॑नस्य । स॒र्पिष॒ इति॑ स॒र्पिषः॑ ॥ अग्ने॒ त्वम् । त्वम् नः॑ । नो॒ अन्त॑मः । अन्त॑म॒ इत्यन्त॑मः ॥ उ॒त त्रा॒ता । त्रा॒ता शि॒वः । शि॒वो भ॑व । भ॒व॒ व॒रू॒थ्यः॑ । व॒रू॒थ्य॑ इति॑ वरू॒थ्यः॑ ॥ तम् त्वा᳚ । त्वा॒ शो॒चि॒ष्ठ॒ । शो॒चि॒ष्ठ॒ दी॒दि॒वः॒ । दी॒दि॒व॒ इति॑ दीदिवः ॥ सु॒म्नाय॑ नू॒नम् । नू॒नमी॑महे । ई॒म॒हे॒ सखि॑भ्यः । सखि॑भ्य॒ इति॒ सखि॑ - भ्यः॒ ॥ वसु॑र॒ग्निः । अ॒ग्निर् वसु॑श्रवाः ( ) । वसु॑श्रवा॒ इति॒ वसु॑ - श्र॒वाः॒ ॥ अच्छा॑ नक्षि । न॒क्षि॒ द्यु॒मत्त॑मः । द्यु॒मत्त॑मो र॒यिम् । द्यु॒मत्त॑म॒ इति॑ द्यु॒मत् - त॒मः॒ । र॒यिम् दाः᳚ । दा॒ इति॑ दाः । \newline

\textbf{Jatai Paata} \newline

1. सुव॒र् न न सुवः॒ सुव॒र् न । \newline
2. न ज्योति॒र् ज्योति॒र् न न ज्योतिः॑ । \newline
3. ज्योति॒रिति॒ ज्योतिः॑ । \newline
4. अग्ने॒ विश्वे॑भि॒र् विश्वे॑भि॒ रग्ने ऽग्ने॒ विश्वे॑भिः । \newline
5. विश्वे॑भिः सु॒मनाः᳚ सु॒मना॒ विश्वे॑भि॒र् विश्वे॑भिः सु॒मनाः᳚ । \newline
6. सु॒मना॒ अनी॑कै॒ रनी॑कैः सु॒मनाः᳚ सु॒मना॒ अनी॑कैः । \newline
7. सु॒मना॒ इति॑ सु - मनाः᳚ । \newline
8. अनी॑कै॒रित्यनी॑कैः । \newline
9. अ॒ग्निꣳ होता॑रꣳ॒॒ होता॑र म॒ग्नि म॒ग्निꣳ होता॑रम् । \newline
10. होता॑रम् मन्ये मन्ये॒ होता॑रꣳ॒॒ होता॑रम् मन्ये । \newline
11. म॒न्ये॒ दास्व॑न्त॒म् दास्व॑न्तम् मन्ये मन्ये॒ दास्व॑न्तम् । \newline
12. दास्व॑न्तं॒ ॅवसो॒र् वसो॒र् दास्व॑न्त॒म् दास्व॑न्तं॒ ॅवसोः᳚ । \newline
13. वसोः᳚ सू॒नुꣳ सू॒नुं ॅवसो॒र् वसोः᳚ सू॒नुम् । \newline
14. सू॒नुꣳ सह॑सः॒ सह॑सः सू॒नुꣳ सू॒नुꣳ सह॑सः । \newline
15. सह॑सो जा॒तवे॑दसम् जा॒तवे॑दसꣳ॒॒ सह॑सः॒ सह॑सो जा॒तवे॑दसम् । \newline
16. जा॒तवे॑दस॒मिति॑ जा॒त - वे॒द॒स॒म् । \newline
17. विप्र॒म् न न विप्रं॒ ॅविप्र॒म् न । \newline
18. न जा॒तवे॑दसम् जा॒तवे॑दस॒न् न न जा॒तवे॑दसम् । \newline
19. जा॒तवे॑दस॒मिति॑ जा॒त - वे॒द॒स॒म् । \newline
20. य ऊ॒र्द्ध्व यो॒र्द्ध्वया॒ यो य ऊ॒र्द्ध्वया᳚ । \newline
21. ऊ॒र्द्ध्वया᳚ स्वद्ध्व॒रः स्व॑द्ध्व॒र ऊ॒र्द्ध्व यो॒र्द्ध्वया᳚ स्वद्ध्व॒रः । \newline
22. स्व॒द्ध्व॒रो दे॒वो दे॒वः स्व॑द्ध्व॒रः स्व॑द्ध्व॒रो दे॒वः । \newline
23. स्व॒द्ध्व॒र इति॑ सु - अ॒द्ध्व॒रः । \newline
24. दे॒वो दे॒वाच्या॑ दे॒वाच्या॑ दे॒वो दे॒वो दे॒वाच्या᳚ । \newline
25. दे॒वाच्या॑ कृ॒पा कृ॒पा दे॒वाच्या॑ दे॒वाच्या॑ कृ॒पा । \newline
26. कृ॒पेति॑ कृ॒पा । \newline
27. घृ॒तस्य॒ विभ्रा᳚ष्टिं॒ ॅविभ्रा᳚ष्टिम् घृ॒तस्य॑ घृ॒तस्य॒ विभ्रा᳚ष्टिम् । \newline
28. विभ्रा᳚ष्टि॒ मन्वनु॒ विभ्रा᳚ष्टिं॒ ॅविभ्रा᳚ष्टि॒ मनु॑ । \newline
29. विभ्रा᳚ष्टि॒मिति॒ वि - भ्रा॒ष्टि॒म् । \newline
30. अनु॑ शु॒क्रशो॑चिषः शु॒क्रशो॑चिषो॒ अन्वनु॑ शु॒क्रशो॑चिषः । \newline
31. शु॒क्रशो॑चिष आ॒जुह्वा॑नस्या॒ जुह्वा॑नस्य शु॒क्रशो॑चिषः शु॒क्रशो॑चिष आ॒जुह्वा॑नस्य । \newline
32. शु॒क्रशो॑चिष॒ इति॑ शु॒क्र - शो॒चि॒षः॒ । \newline
33. आ॒जुह्वा॑नस्य स॒र्पिषः॑ स॒र्पिष॑ आ॒जुह्वा॑नस्या॒ जुह्वा॑नस्य स॒र्पिषः॑ । \newline
34. आ॒जुह्वा॑न॒स्येत्या᳚ - जुह्वा॑नस्य । \newline
35. स॒र्पिष॒ इति॑ स॒र्पिषः॑ । \newline
36. अग्ने॒ त्वम् त्व मग्ने ऽग्ने॒ त्वम् । \newline
37. त्वम् नो॑ न॒ स्त्वम् त्वम् नः॑ । \newline
38. नो॒ अन्त॒मो ऽन्त॑मो नो नो॒ अन्त॑मः । \newline
39. अन्त॑म॒ इत्यन्त॑मः । \newline
40. उ॒त त्रा॒ता त्रा॒तो तोत त्रा॒ता । \newline
41. त्रा॒ता शि॒वः शि॒व स्त्रा॒ता त्रा॒ता शि॒वः । \newline
42. शि॒वो भ॑व भव शि॒वः शि॒वो भ॑व । \newline
43. भ॒व॒ व॒रू॒थ्यो॑ वरू॒थ्यो॑ भव भव वरू॒थ्यः॑ । \newline
44. व॒रू॒थ्य॑ इति॑ वरू॒थ्यः॑ । \newline
45. तम् त्वा᳚ त्वा॒ तम् तम् त्वा᳚ । \newline
46. त्वा॒ शो॒चि॒ष्ठ॒ शो॒चि॒ष्ठ॒ त्वा॒ त्वा॒ शो॒चि॒ष्ठ॒ । \newline
47. शो॒चि॒ष्ठ॒ दी॒दि॒वो॒ दी॒दि॒वः॒ शो॒चि॒ष्ठ॒ शो॒चि॒ष्ठ॒ दी॒दि॒वः॒ । \newline
48. दी॒दि॒व॒ इति॑ दीदिवः । \newline
49. सु॒म्नाय॑ नू॒नन् नू॒नꣳ सु॒म्नाय॑ सु॒म्नाय॑ नू॒नम् । \newline
50. नू॒न मी॑मह ईमहे नू॒नन् नू॒न मी॑महे । \newline
51. ई॒म॒हे॒ सखि॑भ्यः॒ सखि॑भ्य ईमह ईमहे॒ सखि॑भ्यः । \newline
52. सखि॑भ्य॒ इति॒ सखि॑ - भ्यः॒ । \newline
53. वसु॑ र॒ग्नि र॒ग्निर् वसु॒र् वसु॑ र॒ग्निः । \newline
54. अ॒ग्निर् वसु॑श्रवा॒ वसु॑श्रवा अ॒ग्नि र॒ग्निर् वसु॑श्रवाः । \newline
55. वसु॑श्रवा॒ इति॒ वसु॑ - श्र॒वाः॒ । \newline
56. अच्छा॑ नक्षि न॒क्ष्य च्छाच्छा॑ नक्षि । \newline
57. न॒क्षि॒ द्यु॒मत्त॑मो द्यु॒मत्त॑मो नक्षि नक्षि द्यु॒मत्त॑मः । \newline
58. द्यु॒मत्त॑मो र॒यिꣳ र॒यिम् द्यु॒मत्त॑मो द्यु॒मत्त॑मो र॒यिम् । \newline
59. द्यु॒मत्त॑म॒ इति॑ द्यु॒मत् - त॒मः॒ । \newline
60. र॒यिम् दा॑ दा र॒यिꣳ र॒यिम् दाः᳚ । \newline
61. दा॒ इति॑ दाः । \newline

\textbf{Ghana Paata } \newline

1. सुव॒र् न न सुवः॒ सुव॒र् न ज्योति॒र् ज्योति॒र् न सुवः॒ सुव॒र् न ज्योतिः॑ । \newline
2. न ज्योति॒र् ज्योति॒र् न न ज्योतिः॑ । \newline
3. ज्योति॒रिति॒ ज्योतिः॑ । \newline
4. अग्ने॒ विश्वे॑भि॒र् विश्वे॑भि॒ रग्ने ऽग्ने॒ विश्वे॑भिः सु॒मनाः᳚ सु॒मना॒ विश्वे॑भि॒ रग्ने ऽग्ने॒ विश्वे॑भिः सु॒मनाः᳚ । \newline
5. विश्वे॑भिः सु॒मनाः᳚ सु॒मना॒ विश्वे॑भि॒र् विश्वे॑भिः सु॒मना॒ अनी॑कै॒ रनी॑कैः सु॒मना॒ विश्वे॑भि॒र् विश्वे॑भिः सु॒मना॒ अनी॑कैः । \newline
6. सु॒मना॒ अनी॑कै॒ रनी॑कैः सु॒मनाः᳚ सु॒मना॒ अनी॑कैः । \newline
7. सु॒मना॒ इति॑ सु - मनाः᳚ । \newline
8. अनी॑कै॒रित्यनी॑कैः । \newline
9. अ॒ग्निꣳ होता॑रꣳ॒॒ होता॑र म॒ग्नि म॒ग्निꣳ होता॑रम् मन्ये मन्ये॒ होता॑र म॒ग्नि म॒ग्निꣳ होता॑रम् मन्ये । \newline
10. होता॑रम् मन्ये मन्ये॒ होता॑रꣳ॒॒ होता॑रम् मन्ये॒ दास्व॑न्त॒म् दास्व॑न्तम् मन्ये॒ होता॑रꣳ॒॒ होता॑रम् मन्ये॒ दास्व॑न्तम् । \newline
11. म॒न्ये॒ दास्व॑न्त॒म् दास्व॑न्तम् मन्ये मन्ये॒ दास्व॑न्तं॒ ॅवसो॒र् वसो॒र् दास्व॑न्तम् मन्ये मन्ये॒ दास्व॑न्तं॒ ॅवसोः᳚ । \newline
12. दास्व॑न्तं॒ ॅवसो॒र् वसो॒र् दास्व॑न्त॒म् दास्व॑न्तं॒ ॅवसोः᳚ सू॒नुꣳ सू॒नुं ॅवसो॒र् दास्व॑न्त॒म् 
दास्व॑न्तं॒ ॅवसोः᳚ सू॒नुम् । \newline
13. वसोः᳚ सू॒नुꣳ सू॒नुं ॅवसो॒र् वसोः᳚ सू॒नुꣳ सह॑सः॒ सह॑सः सू॒नुं ॅवसो॒र् वसोः᳚ सू॒नुꣳ सह॑सः । \newline
14. सू॒नुꣳ सह॑सः॒ सह॑सः सू॒नुꣳ सू॒नुꣳ सह॑सो जा॒तवे॑दसम् जा॒तवे॑दसꣳ॒॒ सह॑सः सू॒नुꣳ 
सू॒नुꣳ सह॑सो जा॒तवे॑दसम् । \newline
15. सह॑सो जा॒तवे॑दसम् जा॒तवे॑दसꣳ॒॒ सह॑सः॒ सह॑सो जा॒तवे॑दसम् । \newline
16. जा॒तवे॑दस॒मिति॑ जा॒त - वे॒द॒स॒म् । \newline
17. विप्र॒म् न न विप्रं॒ ॅविप्र॒म् न जा॒तवे॑दसम् जा॒तवे॑दस॒म् न विप्रं॒ ॅविप्र॒म् न जा॒तवे॑दसम् । \newline
18. न जा॒तवे॑दसम् जा॒तवे॑दस॒म् न न जा॒तवे॑दसम् । \newline
19. जा॒तवे॑दस॒मिति॑ जा॒त - वे॒द॒स॒म् । \newline
20. य ऊ॒र्द्ध्वयो॒ र्द्ध्वया॒ यो य ऊ॒र्द्ध्वया᳚ स्वद्ध्व॒रः स्व॑द्ध्व॒र ऊ॒र्द्ध्वया॒ यो य ऊ॒र्द्ध्वया᳚ स्वद्ध्व॒रः । \newline
21. ऊ॒र्द्ध्वया᳚ स्वद्ध्व॒रः स्व॑द्ध्व॒र ऊ॒र्द्ध्व यो॒र्द्ध्वया᳚ स्वद्ध्व॒रो दे॒वो दे॒वः स्व॑द्ध्व॒र 
ऊ॒र्द्ध्व यो॒र्द्ध्वया᳚ स्वद्ध्व॒रो दे॒वः । \newline
22. स्व॒द्ध्व॒रो दे॒वो दे॒वः स्व॑द्ध्व॒रः स्व॑द्ध्व॒रो दे॒वो दे॒वाच्या॑ दे॒वाच्या॑ दे॒वः स्व॑द्ध्व॒रः स्व॑द्ध्व॒रो दे॒वो दे॒वाच्या᳚ । \newline
23. स्व॒द्ध्व॒र इति॑ सु - अ॒द्ध्व॒रः । \newline
24. दे॒वो दे॒वाच्या॑ दे॒वाच्या॑ दे॒वो दे॒वो दे॒वाच्या॑ कृ॒पा कृ॒पा दे॒वाच्या॑ दे॒वो दे॒वो दे॒वाच्या॑ कृ॒पा । \newline
25. दे॒वाच्या॑ कृ॒पा कृ॒पा दे॒वाच्या॑ दे॒वाच्या॑ कृ॒पा । \newline
26. कृ॒पेति॑ कृ॒पा । \newline
27. घृ॒तस्य॒ विभ्रा᳚ष्टिं॒ ॅविभ्रा᳚ष्टिम् घृ॒तस्य॑ घृ॒तस्य॒ विभ्रा᳚ष्टि॒ मन्वनु॒ विभ्रा᳚ष्टिम् घृ॒तस्य॑ 
घृ॒तस्य॒ विभ्रा᳚ष्टि॒ मनु॑ । \newline
28. विभ्रा᳚ष्टि॒ मन्वनु॒ विभ्रा᳚ष्टिं॒ ॅविभ्रा᳚ष्टि॒ मनु॑ शु॒क्रशो॑चिषः शु॒क्रशो॑चि॒षो ऽनु॒ विभ्रा᳚ष्टिं॒ ॅविभ्रा᳚ष्टि॒ मनु॑ शु॒क्रशो॑चिषः । \newline
29. विभ्रा᳚ष्टि॒मिति॒ वि - भ्रा॒ष्टि॒म् । \newline
30. अनु॑ शु॒क्रशो॑चिषः शु॒क्रशो॑चिषो॒ अन्वनु॑ शु॒क्रशो॑चिष आ॒जुह्वा॑नस्या॒ जुह्वा॑नस्य शु॒क्रशो॑चिषो॒ अन्वनु॑ शु॒क्रशो॑चिष आ॒जुह्वा॑नस्य । \newline
31. शु॒क्रशो॑चिष आ॒जुह्वा॑नस्या॒ जुह्वा॑नस्य शु॒क्रशो॑चिषः शु॒क्रशो॑चिष आ॒जुह्वा॑नस्य स॒र्पिषः॑ स॒र्पिष॑ आ॒जुह्वा॑नस्य शु॒क्रशो॑चिषः शु॒क्रशो॑चिष आ॒जुह्वा॑नस्य स॒र्पिषः॑ । \newline
32. शु॒क्रशो॑चिष॒ इति॑ शु॒क्र - शो॒चि॒षः॒ । \newline
33. आ॒जुह्वा॑नस्य स॒र्पिषः॑ स॒र्पिष॑ आ॒जुह्वा॑नस्या॒ जुह्वा॑नस्य स॒र्पिषः॑ । \newline
34. आ॒जुह्वा॑न॒स्येत्या᳚ - जुह्वा॑नस्य । \newline
35. स॒र्पिष॒ इति॑ स॒र्पिषः॑ । \newline
36. अग्ने॒ त्वम् त्व मग्ने ऽग्ने॒ त्वम् नो॑ न॒ स्त्व मग्ने ऽग्ने॒ त्वम् नः॑ । \newline
37. त्वम् नो॑ न॒ स्त्वम् त्वम् नो॒ अन्त॒मो ऽन्त॑मो न॒ स्त्वम् त्वम् नो॒ अन्त॑मः । \newline
38. नो॒ अन्त॒मो ऽन्त॑मो नो नो॒ अन्त॑मः । \newline
39. अन्त॑म॒ इत्यन्त॑मः । \newline
40. उ॒त त्रा॒ता त्रा॒तोतोत त्रा॒ता शि॒वः शि॒व स्त्रा॒ तोतोत त्रा॒ता शि॒वः । \newline
41. त्रा॒ता शि॒वः शि॒व स्त्रा॒ता त्रा॒ता शि॒वो भ॑व भव शि॒व स्त्रा॒ता त्रा॒ता शि॒वो भ॑व । \newline
42. शि॒वो भ॑व भव शि॒वः शि॒वो भ॑व वरू॒थ्यो॑ वरू॒थ्यो॑ भव शि॒वः शि॒वो भ॑व वरू॒थ्यः॑ । \newline
43. भ॒व॒ व॒रू॒थ्यो॑ वरू॒थ्यो॑ भव भव वरू॒थ्यः॑ । \newline
44. व॒रू॒थ्य॑ इति॑ वरू॒थ्यः॑ । \newline
45. तम् त्वा᳚ त्वा॒ तम् तम् त्वा॑ शोचिष्ठ शोचिष्ठ त्वा॒ तम् तम् त्वा॑ शोचिष्ठ । \newline
46. त्वा॒ शो॒चि॒ष्ठ॒ शो॒चि॒ष्ठ॒ त्वा॒ त्वा॒ शो॒चि॒ष्ठ॒ दी॒दि॒वो॒ दी॒दि॒वः॒ शो॒चि॒ष्ठ॒ त्वा॒ त्वा॒ शो॒चि॒ष्ठ॒ दी॒दि॒वः॒ । \newline
47. शो॒चि॒ष्ठ॒ दी॒दि॒वो॒ दी॒दि॒वः॒ शो॒चि॒ष्ठ॒ शो॒चि॒ष्ठ॒ दी॒दि॒वः॒ । \newline
48. दी॒दि॒व॒ इति॑ दीदिवः । \newline
49. सु॒म्नाय॑ नू॒नम् नू॒नꣳ सु॒म्नाय॑ सु॒म्नाय॑ नू॒न मी॑मह ईमहे नू॒नꣳ सु॒म्नाय॑ सु॒म्नाय॑ नू॒न मी॑महे । \newline
50. नू॒न मी॑मह ईमहे नू॒नम् नू॒न मी॑महे॒ सखि॑भ्यः॒ सखि॑भ्य ईमहे नू॒नम् नू॒न मी॑महे॒ सखि॑भ्यः । \newline
51. ई॒म॒हे॒ सखि॑भ्यः॒ सखि॑भ्य ईमह ईमहे॒ सखि॑भ्यः । \newline
52. सखि॑भ्य॒ इति॒ सखि॑ - भ्यः॒ । \newline
53. वसु॑ र॒ग्नि र॒ग्निर् वसु॒र् वसु॑ र॒ग्निर् वसु॑श्रवा॒ वसु॑श्रवा अ॒ग्निर् वसु॒र् वसु॑ र॒ग्निर् वसु॑श्रवाः । \newline
54. अ॒ग्निर् वसु॑श्रवा॒ वसु॑श्रवा अ॒ग्नि र॒ग्निर् वसु॑श्रवाः । \newline
55. वसु॑श्रवा॒ इति॒ वसु॑ - श्र॒वाः॒ । \newline
56. अच्छा॑ नक्षि न॒क्ष्य च्छाच्छा॑ नक्षि द्यु॒मत्त॑मो द्यु॒मत्त॑मो न॒क्ष्य च्छाच्छा॑ नक्षि द्यु॒मत्त॑मः । \newline
57. न॒क्षि॒ द्यु॒मत्त॑मो द्यु॒मत्त॑मो नक्षि नक्षि द्यु॒मत्त॑मो र॒यिꣳ र॒यिम् द्यु॒मत्त॑मो नक्षि नक्षि द्यु॒मत्त॑मो र॒यिम् । \newline
58. द्यु॒मत्त॑मो र॒यिꣳ र॒यिम् द्यु॒मत्त॑मो द्यु॒मत्त॑मो र॒यिम् दा॑ दा र॒यिम् द्यु॒मत्त॑मो द्यु॒मत्त॑मो र॒यिम् दाः᳚ । \newline
59. द्यु॒मत्त॑म॒ इति॑ द्यु॒मत् - त॒मः॒ । \newline
60. र॒यिम् दा॑ दा र॒यिꣳ र॒यिम् दाः᳚ । \newline
61. दा॒ इति॑ दाः । \newline
\pagebreak
\markright{ TS 4.4.5.1  \hfill https://www.vedavms.in \hfill}

\section{ TS 4.4.5.1 }

\textbf{TS 4.4.5.1 } \newline
\textbf{Samhita Paata} \newline

इ॒न्द्रा॒ग्निभ्यां᳚ त्वा स॒युजा॑ यु॒जा यु॑नज्म्या घा॒राभ्यां॒ तेज॑सा॒ वर्च॑सो॒ क्थेभिः॒ स्तोमे॑भि॒ श्छन्दो॑भी र॒य्यै पोषा॑य सजा॒तानां᳚ मद्ध्यम॒स्थेया॑य॒ मया᳚ त्वा स॒युजा॑ यु॒जा यु॑नज्म्य॒बां दु॒ला नि॑त॒त्नि र॒भ्रय॑न्ती मे॒घय॑न्ती व॒र्॒.षय॑न्ती चुपु॒णीका॒ नामा॑सि प्र॒जाप॑तिना त्वा॒ विश्वा॑भिर्द्धी॒भिरुप॑ दधामि पृथि॒व्यु॑दपु॒रमन्ने॑न वि॒ष्टा म॑नु॒ष्या᳚स्ते गो॒प्तारो॒ ऽग्निर्विय॑त्तोऽस्यां॒ ताम॒हं प्र॑ पद्ये॒ सा - [  ] \newline

\textbf{Pada Paata} \newline

इ॒न्द्रा॒ग्निभ्या॒मिती᳚न्द्रा॒ग्नि - भ्या॒म् । त्वा॒ । स॒युजेति॑ स - युजा᳚ । यु॒जा । यु॒न॒ज्मि॒ । आ॒घा॒राभ्या॒मित्या᳚-घा॒राभ्या᳚म् । तेज॑सा । वर्च॑सा । उ॒क्थेभिः॑ । स्तोमे॑भिः । छन्दो॑भि॒रिति॒ छन्दः॑ - भिः॒ । र॒य्यै । पोषा॑य । स॒जा॒ताना॒मिति॑ स-जा॒ताना᳚म् । म॒द्ध्य॒म॒स्थेया॒येति॑ मद्ध्यम-स्थेया॑य । मया᳚ । त्वा॒ । स॒युजेति॑ स-युजा᳚ । यु॒जा । यु॒न॒ज्मि॒ । अ॒बां । दु॒ला । नि॒त॒त्निरिति॑ नि - त॒त्निः । अ॒भ्रय॑न्ती । मे॒घय॑न्ती । व॒र्॒.षय॑न्ती । चु॒पु॒णीका᳚ । नाम॑ । अ॒सि॒ । प्र॒जाप॑ति॒नेति॑ प्र॒जा - प॒ति॒ना॒ । त्वा॒ । विश्वा॑भिः । धी॒भिः । उपेति॑ । द॒धा॒मि॒ । पृ॒थि॒वी । उ॒द॒पु॒रमित्यु॑द - पु॒रम् । अन्ने॑न । वि॒ष्टा । म॒नु॒ष्याः᳚ । ते॒ । गो॒प्तारः॑ । अ॒ग्निः । विय॑त्त॒ इति॑ वि - य॒त्तः॒ । अ॒स्या॒म् । ताम् । अ॒हम् । प्रेति॑ । प॒द्ये॒ । सा ।  \newline


\textbf{Krama Paata} \newline

इ॒न्द्रा॒ग्निभ्या᳚म् त्वा । इ॒न्द्रा॒ग्निभ्या॒मिती᳚न्द्रा॒ग्नि - भ्या॒म् । त्वा॒ स॒युजा᳚ । स॒युजा॑ यु॒जा । स॒युजेति॑ स - युजा᳚ । यु॒जा यु॑नज्मि । यु॒न॒ज्म्या॒घा॒राभ्या᳚म् । आ॒घा॒राभ्या॒म् तेज॑सा । आ॒घा॒राभ्या॒मित्या᳚ - घा॒राभ्या᳚म् । तेज॑सा॒ वर्च॑सा । वर्च॑सो॒क्थेभिः॑ । उ॒क्थेभिः॒ स्तोमे॑भिः । स्तोमे॑भि॒श्छन्दो॑भिः । छन्दो॑भी र॒य्यै । छन्दो॑भि॒रिति॒ छन्दः॑ - भिः॒ । र॒य्यै पोषा॑य । पोषा॑य सजा॒ताना᳚म् । स॒जा॒ताना᳚म् मद्ध्यम॒स्थेया॑य । स॒जा॒ताना॒मिति॑ स - जा॒ताना᳚म् । म॒द्ध्य॒म॒स्थेया॑य॒ मया᳚ । म॒द्ध्य॒म॒स्थेया॒येति॑ मद्ध्यम - स्थेया॑य । मया᳚ त्वा । त्वा॒ स॒युजा᳚ । स॒युजा॑ यु॒जा । स॒युजेति॑ स - युजा᳚ । यु॒जा यु॑नज्मि । यु॒न॒ज्म्य॒म्बा । अ॒म्बा दु॒ला । दु॒ला नि॑त॒त्निः । नि॒त॒त्निर॒भ्रय॑न्ती । नि॒त॒त्निरिति॑ नि - त॒त्निः । अ॒भ्रय॑न्ती मे॒घय॑न्ती । मे॒घय॑न्ती व॒र्.॒षय॑न्ती । व॒र्.॒षय॑न्ती चुपु॒णीका᳚ । चु॒पु॒णीका॒ नाम॑ । नामा॑सि । अ॒सि॒ प्र॒जाप॑तिना । प्र॒जाप॑तिना त्वा । प्र॒जाप॑ति॒नेति॑ प्र॒जा - प॒ति॒ना॒ । त्वा॒ विश्वा॑भिः । विश्वा॑भिर् धी॒भिः । धी॒भिरुप॑ । उप॑ दधामि । द॒धा॒मि॒ पृ॒थि॒वी । पृ॒थि॒व्यु॑दपु॒रम् । उ॒द॒पु॒रमन्ने॑न । उ॒द॒पु॒रमित्यु॑द - पु॒रम् । अन्ने॑न वि॒ष्टा । वि॒ष्टा म॑नु॒ष्याः᳚ । म॒नु॒ष्या᳚स्ते । ते॒ गो॒प्तारः॑ । गो॒प्तारो॒ऽग्निः । अ॒ग्निर् विय॑त्तः । विय॑त्तोऽस्याम् । विय॑त्त॒ इति॒ वि - य॒त्तः॒ । 
अ॒स्या॒म् ताम् । ताम॒हम् । अ॒हम् प्र । प्र प॑द्ये । प॒द्ये॒ सा ( ) । सा मे᳚ \newline

\textbf{Jatai Paata} \newline

1. इ॒न्द्रा॒ग्निभ्या᳚म् त्वा त्वेन्द्रा॒ ग्निभ्या॑ मिन्द्रा॒ग्निभ्या᳚म् त्वा । \newline
2. इ॒न्द्रा॒ग्निभ्या॒मिती᳚न्द्रा॒ग्नि - भ्या॒म् । \newline
3. त्वा॒ स॒युजा॑ स॒युजा᳚ त्वा त्वा स॒युजा᳚ । \newline
4. स॒युजा॑ यु॒जा यु॒जा स॒युजा॑ स॒युजा॑ यु॒जा । \newline
5. स॒युजेति॑ स - युजा᳚ । \newline
6. यु॒जा यु॑नज्मि युनज्मि यु॒जा यु॒जा यु॑नज्मि । \newline
7. यु॒न॒ज्म्या॒ घा॒राभ्या॑ माघा॒राभ्यां᳚ ॅयुनज्मि युनज्म्या घा॒राभ्या᳚म् । \newline
8. आ॒घा॒राभ्या॒म् तेज॑सा॒ तेज॑सा ऽऽघा॒राभ्या॑ माघा॒राभ्या॒म् तेज॑सा । \newline
9. आ॒घा॒राभ्या॒मित्या᳚ - घा॒राभ्या᳚म् । \newline
10. तेज॑सा॒ वर्च॑सा॒ वर्च॑सा॒ तेज॑सा॒ तेज॑सा॒ वर्च॑सा । \newline
11. वर्च॑ सो॒क्थेभि॑ रु॒क्थेभि॒र् वर्च॑सा॒ वर्च॑ सो॒क्थेभिः॑ । \newline
12. उ॒क्थेभिः॒ स्तोमे॑भिः॒ स्तोमे॑भि रु॒क्थेभि॑ रु॒क्थेभिः॒ स्तोमे॑भिः । \newline
13. स्तोमे॑भि॒ श्छन्दो॑भि॒ श्छन्दो॑भिः॒ स्तोमे॑भिः॒ स्तोमे॑भि॒ श्छन्दो॑भिः । \newline
14. छन्दो॑भी र॒य्यै र॒य्यै छन्दो॑भि॒ श्छन्दो॑भी र॒य्यै । \newline
15. छन्दो॑भि॒रिति॒ छन्दः॑ - भिः॒ । \newline
16. र॒य्यै पोषा॑य॒ पोषा॑य र॒य्यै र॒य्यै पोषा॑य । \newline
17. पोषा॑य सजा॒तानाꣳ॑ सजा॒ताना॒म् पोषा॑य॒ पोषा॑य सजा॒ताना᳚म् । \newline
18. स॒जा॒ताना᳚म् मद्ध्यम॒स्थेया॑य मद्ध्यम॒स्थेया॑य सजा॒तानाꣳ॑ सजा॒ताना᳚म् मद्ध्यम॒स्थेया॑य । \newline
19. स॒जा॒ताना॒मिति॑ स - जा॒ताना᳚म् । \newline
20. म॒द्ध्य॒म॒स्थेया॑य॒ मया॒ मया॑ मद्ध्यम॒स्थेया॑य मद्ध्यम॒स्थेया॑य॒ मया᳚ । \newline
21. म॒द्ध्य॒म॒स्थेया॒येति॑ मद्ध्यम - स्थेया॑य । \newline
22. मया᳚ त्वा त्वा॒ मया॒ मया᳚ त्वा । \newline
23. त्वा॒ स॒युजा॑ स॒युजा᳚ त्वा त्वा स॒युजा᳚ । \newline
24. स॒युजा॑ यु॒जा यु॒जा स॒युजा॑ स॒युजा॑ यु॒जा । \newline
25. स॒युजेति॑ स - युजा᳚ । \newline
26. यु॒जा यु॑नज्मि युनज्मि यु॒जा यु॒जा यु॑नज्मि । \newline
27. यु॒न॒ज् म्य॒म्बाम्बा यु॑नज्मि युनज्म्य॒म्बा । \newline
28. अ॒म्बा दु॒ला दु॒ला म्बा म्बा दु॒ला । \newline
29. दु॒ला नि॑त॒त्निर् नि॑त॒त्निर् दु॒ला दु॒ला नि॑त॒त्निः । \newline
30. नि॒त॒त्नि र॒भ्रय॑ न्त्य॒भ्रय॑न्ती नित॒त्निर् नि॑त॒त्नि र॒भ्रय॑न्ती । \newline
31. नि॒त॒त्निरिति॑ नि - त॒त्निः । \newline
32. अ॒भ्रय॑न्ती मे॒घय॑न्ती मे॒घय॑ न्त्य॒भ्रय॑ न्त्य॒भ्रय॑न्ती मे॒घय॑न्ती । \newline
33. मे॒घय॑न्ती व॒र्॒.षय॑न्ती व॒र्॒.षय॑न्ती मे॒घय॑न्ती मे॒घय॑न्ती व॒र्॒.षय॑न्ती । \newline
34. व॒र्॒.षय॑न्ती चुपु॒णीका॑ चुपु॒णीका॑ व॒र्॒.षय॑न्ती व॒र्॒.षय॑न्ती चुपु॒णीका᳚ । \newline
35. चु॒पु॒णीका॒ नाम॒ नाम॑ चुपु॒णीका॑ चुपु॒णीका॒ नाम॑ । \newline
36. नामा᳚स्यसि॒ नाम॒ नामा॑सि । \newline
37. अ॒सि॒ प्र॒जाप॑तिना प्र॒जाप॑तिना ऽस्यसि प्र॒जाप॑तिना । \newline
38. प्र॒जाप॑तिना त्वा त्वा प्र॒जाप॑तिना प्र॒जाप॑तिना त्वा । \newline
39. प्र॒जाप॑ति॒नेति॑ प्र॒जा - प॒ति॒ना॒ । \newline
40. त्वा॒ विश्वा॑भि॒र् विश्वा॑भि स्त्वा त्वा॒ विश्वा॑भिः । \newline
41. विश्वा॑भिर् धी॒भिर् धी॒भिर् विश्वा॑भि॒र् विश्वा॑भिर् धी॒भिः । \newline
42. धी॒भि रुपोप॑ धी॒भिर् धी॒भि रुप॑ । \newline
43. उप॑ दधामि दधा॒ म्युपोप॑ दधामि । \newline
44. द॒धा॒मि॒ पृ॒थि॒वी पृ॑थि॒वी द॑धामि दधामि पृथि॒वी । \newline
45. पृ॒थि॒ व्यु॑दपु॒र मु॑दपु॒रम् पृ॑थि॒वी पृ॑थि॒ व्यु॑दपु॒रम् । \newline
46. उ॒द॒पु॒र मन्ने॒-नान्ने॑नो दपु॒र मु॑दपु॒र मन्ने॑न । \newline
47. उ॒द॒पु॒रमित्यु॑द - पु॒रम् । \newline
48. अन्ने॑न वि॒ष्टा वि॒ष्टा ऽन्ने॒ना-न्ने॑न वि॒ष्टा । \newline
49. वि॒ष्टा म॑नु॒ष्या॑ मनु॒ष्या॑ वि॒ष्टा वि॒ष्टा म॑नु॒ष्याः᳚ । \newline
50. म॒नु॒ष्या᳚ स्ते ते मनु॒ष्या॑ मनु॒ष्या᳚ स्ते । \newline
51. ते॒ गो॒प्तारो॑ गो॒प्तार॑ स्ते ते गो॒प्तारः॑ । \newline
52. गो॒प्तारो॒ ऽग्नि र॒ग्निर् गो॒प्तारो॑ गो॒प्तारो॒ ऽग्निः । \newline
53. अ॒ग्निर् विय॑त्तो॒ विय॑त्तो॒ ऽग्नि र॒ग्निर् विय॑त्तः । \newline
54. विय॑त्तो ऽस्या मस्यां॒ ॅविय॑त्तो॒ विय॑त्तो ऽस्याम् । \newline
55. विय॑त्त॒ इति॒ वि - य॒त्तः॒ । \newline
56. अ॒स्या॒म् ताम् ता म॑स्या मस्या॒म् ताम् । \newline
57. ता म॒ह म॒हम् ताम् ता म॒हम् । \newline
58. अ॒हम् प्र प्राह म॒हम् प्र । \newline
59. प्र प॑द्ये पद्ये॒ प्र प्र प॑द्ये । \newline
60. प॒द्ये॒ सा सा प॑द्ये पद्ये॒ सा । \newline
61. सा मे॑ मे॒ सा सा मे᳚ । \newline

\textbf{Ghana Paata } \newline

1. इ॒न्द्रा॒ग्निभ्या᳚म् त्वा त्वेन्द्रा॒ग्निभ्या॑ मिन्द्रा॒ग्निभ्या᳚म् त्वा स॒युजा॑ स॒युजा᳚ त्वेन्द्रा॒ग्निभ्या॑ मिन्द्रा॒ग्निभ्या᳚म् त्वा स॒युजा᳚ । \newline
2. इ॒न्द्रा॒ग्निभ्या॒मिती᳚न्द्रा॒ग्नि - भ्या॒म् । \newline
3. त्वा॒ स॒युजा॑ स॒युजा᳚ त्वा त्वा स॒युजा॑ यु॒जा यु॒जा स॒युजा᳚ त्वा त्वा स॒युजा॑ यु॒जा । \newline
4. स॒युजा॑ यु॒जा यु॒जा स॒युजा॑ स॒युजा॑ यु॒जा यु॑नज्मि युनज्मि यु॒जा स॒युजा॑ स॒युजा॑ यु॒जा यु॑नज्मि । \newline
5. स॒युजेति॑ स - युजा᳚ । \newline
6. यु॒जा यु॑नज्मि युनज्मि यु॒जा यु॒जा यु॑नज्म्या घा॒राभ्या॑ माघा॒राभ्यां᳚ ॅयुनज्मि यु॒जा यु॒जा यु॑नज्म्या घा॒राभ्या᳚म् । \newline
7. यु॒न॒ज्म्या॒ घा॒राभ्या॑ माघा॒राभ्यां᳚ ॅयुनज्मि युनज्म्या घा॒राभ्या॒म् तेज॑सा॒ तेज॑सा ऽऽघा॒राभ्यां᳚ ॅयुनज्मि युनज्म्या घा॒राभ्या॒म् तेज॑सा । \newline
8. आ॒घा॒राभ्या॒म् तेज॑सा॒ तेज॑सा ऽऽघा॒राभ्या॑ माघा॒राभ्या॒म् तेज॑सा॒ वर्च॑सा॒ वर्च॑सा॒ तेज॑सा ऽऽघा॒राभ्या॑ माघा॒राभ्या॒म् तेज॑सा॒ वर्च॑सा । \newline
9. आ॒घा॒राभ्या॒मित्या᳚ - घा॒राभ्या᳚म् । \newline
10. तेज॑सा॒ वर्च॑सा॒ वर्च॑सा॒ तेज॑सा॒ तेज॑सा॒ वर्च॑सो॒क्थेभि॑ रु॒क्थेभि॒र् वर्च॑सा॒ तेज॑सा॒ तेज॑सा॒ 
वर्च॑सो॒क्थेभिः॑ । \newline
11. वर्च॑सो॒क्थेभि॑ रु॒क्थेभि॒र् वर्च॑सा॒ वर्च॑सो॒क्थेभिः॒ स्तोमे॑भिः॒ स्तोमे॑भि रु॒क्थेभि॒र् वर्च॑सा॒ 
वर्च॑सो॒क्थेभिः॒ स्तोमे॑भिः । \newline
12. उ॒क्थेभिः॒ स्तोमे॑भिः॒ स्तोमे॑भि रु॒क्थेभि॑ रु॒क्थेभिः॒ स्तोमे॑भि॒ श्छन्दो॑भि॒ श्छन्दो॑भिः॒ स्तोमे॑भि रु॒क्थेभि॑ रु॒क्थेभिः॒ स्तोमे॑भि॒ श्छन्दो॑भिः । \newline
13. स्तोमे॑भि॒ श्छन्दो॑भि॒ श्छन्दो॑भिः॒ स्तोमे॑भिः॒ स्तोमे॑भि॒ श्छन्दो॑भी र॒य्यै र॒य्यै छन्दो॑भिः॒ स्तोमे॑भिः॒ स्तोमे॑भि॒ श्छन्दो॑भी र॒य्यै । \newline
14. छन्दो॑भी र॒य्यै र॒य्यै छन्दो॑भि॒ श्छन्दो॑भी र॒य्यै पोषा॑य॒ पोषा॑य र॒य्यै छन्दो॑भि॒ श्छन्दो॑भी र॒य्यै पोषा॑य । \newline
15. छन्दो॑भि॒रिति॒ छन्दः॑ - भिः॒ । \newline
16. र॒य्यै पोषा॑य॒ पोषा॑य र॒य्यै र॒य्यै पोषा॑य सजा॒तानाꣳ॑ सजा॒ताना॒म् पोषा॑य र॒य्यै र॒य्यै पोषा॑य सजा॒ताना᳚म् । \newline
17. पोषा॑य सजा॒तानाꣳ॑ सजा॒ताना॒म् पोषा॑य॒ पोषा॑य सजा॒ताना᳚म् मद्ध्यम॒स्थेया॑य मद्ध्यम॒स्थेया॑य सजा॒ताना॒म् पोषा॑य॒ पोषा॑य सजा॒ताना᳚म् मद्ध्यम॒स्थेया॑य । \newline
18. स॒जा॒ताना᳚म् मद्ध्यम॒स्थेया॑य मद्ध्यम॒स्थेया॑य सजा॒तानाꣳ॑ सजा॒ताना᳚म् मद्ध्यम॒स्थेया॑य॒ मया॒ मया॑ मद्ध्यम॒स्थेया॑य सजा॒तानाꣳ॑ सजा॒ताना᳚म् मद्ध्यम॒स्थेया॑य॒ मया᳚ । \newline
19. स॒जा॒ताना॒मिति॑ स - जा॒ताना᳚म् । \newline
20. म॒द्ध्य॒म॒स्थेया॑य॒ मया॒ मया॑ मद्ध्यम॒स्थेया॑य मद्ध्यम॒स्थेया॑य॒ मया᳚ त्वा त्वा॒ मया॑ मद्ध्यम॒स्थेया॑य मद्ध्यम॒स्थेया॑य॒ मया᳚ त्वा । \newline
21. म॒द्ध्य॒म॒स्थेया॒येति॑ मद्ध्यम - स्थेया॑य । \newline
22. मया᳚ त्वा त्वा॒ मया॒ मया᳚ त्वा स॒युजा॑ स॒युजा᳚ त्वा॒ मया॒ मया᳚ त्वा स॒युजा᳚ । \newline
23. त्वा॒ स॒युजा॑ स॒युजा᳚ त्वा त्वा स॒युजा॑ यु॒जा यु॒जा स॒युजा᳚ त्वा त्वा स॒युजा॑ यु॒जा । \newline
24. स॒युजा॑ यु॒जा यु॒जा स॒युजा॑ स॒युजा॑ यु॒जा यु॑नज्मि युनज्मि यु॒जा स॒युजा॑ स॒युजा॑ यु॒जा यु॑नज्मि । \newline
25. स॒युजेति॑ स - युजा᳚ । \newline
26. यु॒जा यु॑नज्मि युनज्मि यु॒जा यु॒जा यु॑नज् म्य॒म्बांबा यु॑नज्मि यु॒जा यु॒जा यु॑नज्म्यं॒बा । \newline
27. यु॒न॒ज्म्य॒ म्बांबा यु॑नज्मि युनज्म्यं॒बा दु॒ला दु॒लांबा यु॑नज्मि युनज्म्यं॒बा दु॒ला । \newline
28. अं॒बा दु॒ला दु॒लां बांबा दु॒ला नि॑त॒त्निर् नि॑त॒त्निर् दु॒लां बांबा दु॒ला नि॑त॒त्निः । \newline
29. दु॒ला नि॑त॒त्निर् नि॑त॒त्निर् दु॒ला दु॒ला नि॑त॒त्नि र॒भ्रय॑ न्त्य॒भ्रय॑न्ती नित॒त्निर् दु॒ला दु॒ला नि॑त॒त्नि र॒भ्रय॑न्ती । \newline
30. नि॒त॒त्नि र॒भ्रय॑ न्त्य॒भ्रय॑न्ती नित॒त्निर् नि॑त॒त्नि र॒भ्रय॑न्ती मे॒घय॑न्ती मे॒घय॑ न्त्य॒भ्रय॑न्ती नित॒त्निर् नि॑त॒त्नि र॒भ्रय॑न्ती मे॒घय॑न्ती । \newline
31. नि॒त॒त्निरिति॑ नि - त॒त्निः । \newline
32. अ॒भ्रय॑न्ती मे॒घय॑न्ती मे॒घय॑ न्त्य॒भ्रय॑ न्त्य॒भ्रय॑न्ती मे॒घय॑न्ती व॒र्॒.षय॑न्ती व॒र्॒.षय॑न्ती मे॒घय॑ न्त्य॒भ्रय॑न्त्य॒ भ्रय॑न्ती मे॒घय॑न्ती व॒र्॒.षय॑न्ती । \newline
33. मे॒घय॑न्ती व॒र्॒.षय॑न्ती व॒र्॒.षय॑न्ती मे॒घय॑न्ती मे॒घय॑न्ती व॒र्॒.षय॑न्ती चुपु॒णीका॑ चुपु॒णीका॑ व॒र्॒.षय॑न्ती मे॒घय॑न्ती मे॒घय॑न्ती व॒र्॒.षय॑न्ती चुपु॒णीका᳚ । \newline
34. व॒र्॒.षय॑न्ती चुपु॒णीका॑ चुपु॒णीका॑ व॒र्॒.षय॑न्ती व॒र्॒.षय॑न्ती चुपु॒णीका॒ नाम॒ नाम॑ चुपु॒णीका॑ व॒र्॒.षय॑न्ती व॒र्॒.षय॑न्ती चुपु॒णीका॒ नाम॑ । \newline
35. चु॒पु॒णीका॒ नाम॒ नाम॑ चुपु॒णीका॑ चुपु॒णीका॒ नामा᳚स्यसि॒ नाम॑ चुपु॒णीका॑ चुपु॒णीका॒ नामा॑सि । \newline
36. नामा᳚स्यसि॒ नाम॒ नामा॑सि प्र॒जाप॑तिना प्र॒जाप॑तिना ऽसि॒ नाम॒ नामा॑सि प्र॒जाप॑तिना । \newline
37. अ॒सि॒ प्र॒जाप॑तिना प्र॒जाप॑तिना ऽस्यसि प्र॒जाप॑तिना त्वा त्वा प्र॒जाप॑तिना ऽस्यसि प्र॒जाप॑तिना त्वा । \newline
38. प्र॒जाप॑तिना त्वा त्वा प्र॒जाप॑तिना प्र॒जाप॑तिना त्वा॒ विश्वा॑भि॒र् विश्वा॑भि स्त्वा प्र॒जाप॑तिना प्र॒जाप॑तिना त्वा॒ विश्वा॑भिः । \newline
39. प्र॒जाप॑ति॒नेति॑ प्र॒जा - प॒ति॒ना॒ । \newline
40. त्वा॒ विश्वा॑भि॒र् विश्वा॑भि स्त्वा त्वा॒ विश्वा॑भिर् धी॒भिर् धी॒भिर् विश्वा॑भि स्त्वा त्वा॒ विश्वा॑भिर् धी॒भिः । \newline
41. विश्वा॑भिर् धी॒भिर् धी॒भिर् विश्वा॑भि॒र् विश्वा॑भिर् धी॒भि रुपोप॑ धी॒भिर् विश्वा॑भि॒र् विश्वा॑भिर् धी॒भिरुप॑ । \newline
42. धी॒भि रुपोप॑ धी॒भिर् धी॒भिरुप॑ दधामि दधा॒ म्युप॑ धी॒भिर् धी॒भिरुप॑ दधामि । \newline
43. उप॑ दधामि दधा॒ म्युपोप॑ दधामि पृथि॒वी पृ॑थि॒वी द॑धा॒ म्युपोप॑ दधामि पृथि॒वी । \newline
44. द॒धा॒मि॒ पृ॒थि॒वी पृ॑थि॒वी द॑धामि दधामि पृथि॒ व्यु॑दपु॒र मु॑दपु॒रम् पृ॑थि॒वी द॑धामि दधामि पृथि॒ व्यु॑दपु॒रम् । \newline
45. पृ॒थि॒ व्यु॑दपु॒र मु॑दपु॒रम् पृ॑थि॒वी पृ॑थि॒ व्यु॑दपु॒र मन्ने॒ना न्ने॑नोदपु॒रम् पृ॑थि॒वी पृ॑थि॒ व्यु॑दपु॒र मन्ने॑न । \newline
46. उ॒द॒पु॒र मन्ने॒ना न्ने॑नो दपु॒र मु॑दपु॒र मन्ने॑न वि॒ष्टा वि॒ष्टा ऽन्ने॑नोदपु॒र मु॑दपु॒र मन्ने॑न वि॒ष्टा । \newline
47. उ॒द॒पु॒रमित्यु॑द - पु॒रम् । \newline
48. अन्ने॑न वि॒ष्टा वि॒ष्टा ऽन्ने॒ना न्ने॑न वि॒ष्टा म॑नु॒ष्या॑ मनु॒ष्या॑ वि॒ष्टा ऽन्ने॒ना न्ने॑न वि॒ष्टा म॑नु॒ष्याः᳚ । \newline
49. वि॒ष्टा म॑नु॒ष्या॑ मनु॒ष्या॑ वि॒ष्टा वि॒ष्टा म॑नु॒ष्या᳚ स्ते ते मनु॒ष्या॑ वि॒ष्टा वि॒ष्टा म॑नु॒ष्या᳚ स्ते । \newline
50. म॒नु॒ष्या᳚ स्ते ते मनु॒ष्या॑ मनु॒ष्या᳚ स्ते गो॒प्तारो॑ गो॒प्तार॑ स्ते मनु॒ष्या॑ मनु॒ष्या᳚ स्ते गो॒प्तारः॑ । \newline
51. ते॒ गो॒प्तारो॑ गो॒प्तार॑ स्ते ते गो॒प्तारो॒ ऽग्नि र॒ग्निर् गो॒प्तार॑ स्ते ते गो॒प्तारो॒ ऽग्निः । \newline
52. गो॒प्तारो॒ ऽग्नि र॒ग्निर् गो॒प्तारो॑ गो॒प्तारो॒ ऽग्निर् विय॑त्तो॒ विय॑त्तो॒ ऽग्निर् गो॒प्तारो॑ गो॒प्तारो॒ ऽग्निर् विय॑त्तः । \newline
53. अ॒ग्निर् विय॑त्तो॒ विय॑त्तो॒ ऽग्नि र॒ग्निर् विय॑त्तो ऽस्या मस्यां॒ ॅविय॑त्तो॒ ऽग्नि र॒ग्निर् विय॑त्तो ऽस्याम् । \newline
54. विय॑त्तो ऽस्या मस्यां॒ ॅविय॑त्तो॒ विय॑त्तो ऽस्या॒म् ताम् ता म॑स्यां॒ ॅविय॑त्तो॒ विय॑त्तो ऽस्या॒म् ताम् । \newline
55. विय॑त्त॒ इति॒ वि - य॒त्तः॒ । \newline
56. अ॒स्या॒म् ताम् ता म॑स्या मस्या॒म् ता म॒ह म॒हम् ता म॑स्या मस्या॒म् ता म॒हम् । \newline
57. ता म॒ह म॒हम् ताम् ता म॒हम् प्र प्राहम् ताम् ता म॒हम् प्र । \newline
58. अ॒हम् प्र प्राह म॒हम् प्र प॑द्ये पद्ये॒ प्राह म॒हम् प्र प॑द्ये । \newline
59. प्र प॑द्ये पद्ये॒ प्र प्र प॑द्ये॒ सा सा प॑द्ये॒ प्र प्र प॑द्ये॒ सा । \newline
60. प॒द्ये॒ सा सा प॑द्ये पद्ये॒ सा मे॑ मे॒ सा प॑द्ये पद्ये॒ सा मे᳚ । \newline
61. सा मे॑ मे॒ सा सा मे॒ शर्म॒ शर्म॑ मे॒ सा सा मे॒ शर्म॑ । \newline
\pagebreak
\markright{ TS 4.4.5.2  \hfill https://www.vedavms.in \hfill}

\section{ TS 4.4.5.2 }

\textbf{TS 4.4.5.2 } \newline
\textbf{Samhita Paata} \newline

मे॒ शर्म॑ च॒ वर्म॑ चा॒स्त्वधि॑ द्यौर॒न्तरि॑क्षं॒ ब्रह्म॑णा वि॒ष्टा म॒रुत॑स्ते गो॒प्तारो॑ वा॒युर्विय॑त्तोऽस्यां॒ ताम॒हं प्र प॑द्ये॒ सा मे॒ शर्म॑ च॒ वर्म॑ चास्तु॒ द्यौरप॑राजिता॒ऽमृते॑न वि॒ष्टाऽऽदि॒त्यास्ते॑ गो॒प्तारः॒ सूर्यो॒ विय॑त्तोऽस्यां॒ ताम॒हं प्र प॑द्ये॒ सा मे॒ शर्म॑ च॒ वर्म॑ चास्तु ॥ \newline

\textbf{Pada Paata} \newline

मे॒ । शर्म॑ । च॒ । वर्म॑ । च॒ । अ॒स्तु॒ । अधि॑द्यौ॒रित्यधि॑ - द्यौः॒ । अ॒न्तरि॑क्षम् । ब्रह्म॑णा । वि॒ष्टा । म॒रुतः॑ । ते॒ । गो॒प्तारः॑ । वा॒युः । विय॑त्त॒ इति॒ वि - य॒त्तः॒ । अ॒स्या॒म् । ताम् । अ॒हम् । प्रेति॑ । प॒द्ये॒ । सा । मे॒ । शर्म॑ । च॒ । वर्म॑ । च॒ । अ॒स्तु॒ । द्यौः । अप॑राजि॒तेत्यप॑रा - जि॒ता॒ । अ॒मृते॑न । वि॒ष्टा । आ॒दि॒त्याः । ते॒ । गो॒प्तारः॑ । सूर्यः॑ । विय॑त्त॒ इति॒ वि-य॒त्तः॒ । अ॒स्या॒म् । ताम् । अ॒हम् । प्रेति॑ । प॒द्ये॒ । सा । मे॒ । शर्म॑ । च॒ । वर्म॑ । च॒ । अ॒स्तु॒ ॥  \newline


\textbf{Krama Paata} \newline

मे॒ शर्म॑ । शर्म॑ च । च॒ वर्म॑ । वर्म॑ च । चा॒स्तु॒ । अ॒स्त्वधि॑द्यौः । अधि॑द्यौर॒न्तरि॑क्षम् । अधि॑द्यौ॒रित्यधि॑ - द्यौः॒ । अ॒न्तरि॑क्ष॒म् ब्रह्म॑णा । ब्रह्म॑णा वि॒ष्टा । वि॒ष्टा म॒रुतः॑ । म॒रुत॑स्ते । ते॒ गो॒प्तारः॑ । गो॒प्तारो॑ वा॒युः । वा॒युर् विय॑त्तः । विय॑त्तोऽस्याम् । विय॑त्त॒ इति॒ वि - य॒त्तः॒ । अ॒स्या॒म् ताम् । ताम॒हम् । अ॒हम् प्र । 
प्र प॑द्ये । प॒द्ये॒ सा । सा मे᳚ । मे॒ शर्म॑ । शर्म॑ च । च॒ वर्म॑ । वर्म॑ च । चा॒स्तु॒ । अ॒स्तु॒ द्यौः । द्यौरप॑राजिता । अप॑राजिता॒ऽमृते॑न । अप॑राजि॒तेत्यप॑रा - जि॒ता॒ । अ॒मृते॑न वि॒ष्टा । वि॒ष्टाऽऽदि॒त्याः । आ॒दि॒त्यास्ते᳚ । ते॒ गो॒प्तारः॑ । गो॒प्तारः॒ सूर्यः॑ । सूर्यो॒ विय॑त्तः । विय॑त्तोऽस्याम् । विय॑त्त॒ इति॒ वि - य॒त्तः॒ । अ॒स्या॒म् ताम् । ताम॒हम् । अ॒हम् प्र । प्र प॑द्ये । प॒द्ये॒ सा । सा मे᳚ । मे॒ शर्म॑ । शर्म॑ च । च॒ वर्म॑ । वर्म॑ च । चा॒स्तु॒ । अ॒स्त्वित्य॑स्तु । \newline

\textbf{Jatai Paata} \newline

1. मे॒ शर्म॒ शर्म॑ मे मे॒ शर्म॑ । \newline
2. शर्म॑ च च॒ शर्म॒ शर्म॑ च । \newline
3. च॒ वर्म॒ वर्म॑ च च॒ वर्म॑ । \newline
4. वर्म॑ च च॒ वर्म॒ वर्म॑ च । \newline
5. चा॒ स्त्व॒स्तु॒ च॒ चा॒स्तु॒ । \newline
6. अ॒स्त्व धि॑द्यौ॒ रधि॑द्यौ रस्त्व॒ स्त्व धि॑द्यौः । \newline
7. अधि॑द्यौ र॒न्तरि॑क्ष म॒न्तरि॑क्ष॒ मधि॑द्यौ॒ रधि॑द्यौ र॒न्तरि॑क्षम् । \newline
8. अधि॑द्यौ॒रित्यधि॑ - द्यौः॒ । \newline
9. अ॒न्तरि॑क्ष॒म् ब्रह्म॑णा॒ ब्रह्म॑णा॒ ऽन्तरि॑क्ष म॒न्तरि॑क्ष॒म् ब्रह्म॑णा । \newline
10. ब्रह्म॑णा वि॒ष्टा वि॒ष्टा ब्रह्म॑णा॒ ब्रह्म॑णा वि॒ष्टा । \newline
11. वि॒ष्टा म॒रुतो॑ म॒रुतो॑ वि॒ष्टा वि॒ष्टा म॒रुतः॑ । \newline
12. म॒रुत॑ स्ते ते म॒रुतो॑ म॒रुत॑ स्ते । \newline
13. ते॒ गो॒प्तारो॑ गो॒प्तार॑ स्ते ते गो॒प्तारः॑ । \newline
14. गो॒प्तारो॑ वा॒युर् वा॒युर् गो॒प्तारो॑ गो॒प्तारो॑ वा॒युः । \newline
15. वा॒युर् विय॑त्तो॒ विय॑त्तो वा॒युर् वा॒युर् विय॑त्तः । \newline
16. विय॑त्तो ऽस्या मस्यां॒ ॅविय॑त्तो॒ विय॑त्तो ऽस्याम् । \newline
17. विय॑त्त॒ इति॒ वि - य॒त्तः॒ । \newline
18. अ॒स्या॒म् ताम् ता म॑स्या मस्या॒म् ताम् । \newline
19. ता म॒ह म॒हम् ताम् ता म॒हम् । \newline
20. अ॒हम् प्र प्राह म॒हम् प्र । \newline
21. प्र प॑द्ये पद्ये॒ प्र प्र प॑द्ये । \newline
22. प॒द्ये॒ सा सा प॑द्ये पद्ये॒ सा । \newline
23. सा मे॑ मे॒ सा सा मे᳚ । \newline
24. मे॒ शर्म॒ शर्म॑ मे मे॒ शर्म॑ । \newline
25. शर्म॑ च च॒ शर्म॒ शर्म॑ च । \newline
26. च॒ वर्म॒ वर्म॑ च च॒ वर्म॑ । \newline
27. वर्म॑ च च॒ वर्म॒ वर्म॑ च । \newline
28. चा॒ स्त्व॒ स्तु॒ च॒ चा॒ स्तु॒ । \newline
29. अ॒स्तु॒ द्यौर् द्यौ र॑स्त्व स्तु॒ द्यौः । \newline
30. द्यौ रप॑राजि॒ता ऽप॑राजिता॒ द्यौर् द्यौ रप॑राजिता । \newline
31. अप॑राजिता॒ ऽमृते॑ना॒ मृते॒ना प॑राजि॒ता ऽप॑राजिता॒ ऽमृते॑न । \newline
32. अप॑राजि॒तेत्यप॑रा - जि॒ता॒ । \newline
33. अ॒मृते॑न वि॒ष्टा वि॒ष्टा ऽमृते॑ना॒ मृते॑न वि॒ष्टा । \newline
34. वि॒ष्टा ऽऽदि॒त्या आ॑दि॒त्या वि॒ष्टा वि॒ष्टा ऽऽदि॒त्याः । \newline
35. आ॒दि॒त्या स्ते॑ त आदि॒त्या आ॑दि॒त्या स्ते᳚ । \newline
36. ते॒ गो॒प्तारो॑ गो॒प्तार॑ स्ते ते गो॒प्तारः॑ । \newline
37. गो॒प्तारः॒ सूर्यः॒ सूर्यो॑ गो॒प्तारो॑ गो॒प्तारः॒ सूर्यः॑ । \newline
38. सूर्यो॒ विय॑त्तो॒ विय॑त्तः॒ सूर्यः॒ सूर्यो॒ विय॑त्तः । \newline
39. विय॑त्तो ऽस्या मस्यां॒ ॅविय॑त्तो॒ विय॑त्तो ऽस्याम् । \newline
40. विय॑त्त॒ इति॒ वि - य॒त्तः॒ । \newline
41. अ॒स्या॒म् ताम् ता म॑स्या मस्या॒म् ताम् । \newline
42. ता म॒ह म॒हम् ताम् ता म॒हम् । \newline
43. अ॒हम् प्र प्राह म॒हम् प्र । \newline
44. प्र प॑द्ये पद्ये॒ प्र प्र प॑द्ये । \newline
45. प॒द्ये॒ सा सा प॑द्ये पद्ये॒ सा । \newline
46. सा मे॑ मे॒ सा सा मे᳚ । \newline
47. मे॒ शर्म॒ शर्म॑ मे मे॒ शर्म॑ । \newline
48. शर्म॑ च च॒ शर्म॒ शर्म॑ च । \newline
49. च॒ वर्म॒ वर्म॑ च च॒ वर्म॑ । \newline
50. वर्म॑ च च॒ वर्म॒ वर्म॑ च । \newline
51. चा॒ स्त्व॒ स्तु॒ च॒ चा॒स्तु॒ । \newline
52. अ॒स्त्वित्य॑स्तु । \newline

\textbf{Ghana Paata } \newline

1. मे॒ शर्म॒ शर्म॑ मे मे॒ शर्म॑ च च॒ शर्म॑ मे मे॒ शर्म॑ च । \newline
2. शर्म॑ च च॒ शर्म॒ शर्म॑ च॒ वर्म॒ वर्म॑ च॒ शर्म॒ शर्म॑ च॒ वर्म॑ । \newline
3. च॒ वर्म॒ वर्म॑ च च॒ वर्म॑ च च॒ वर्म॑ च च॒ वर्म॑ च । \newline
4. वर्म॑ च च॒ वर्म॒ वर्म॑ चा स्त्वस्तु च॒ वर्म॒ वर्म॑ चास्तु । \newline
5. चा॒स्त्व॒स्तु॒ च॒ चा॒ स्त्वधि॑द्यौ॒ रधि॑द्यौ रस्तु च चा॒ स्त्वधि॑द्यौः । \newline
6. अ॒स्त्वधि॑द्यौ॒ रधि॑द्यौ रस्त्व॒ स्त्वधि॑द्यौ र॒न्तरि॑क्ष म॒न्तरि॑क्ष॒ मधि॑द्यौ रस्त्व॒ स्त्वधि॑द्यौ र॒न्तरि॑क्षम् । \newline
7. अधि॑द्यौ र॒न्तरि॑क्ष म॒न्तरि॑क्ष॒ मधि॑द्यौ॒ रधि॑द्यौ र॒न्तरि॑क्ष॒म् ब्रह्म॑णा॒ ब्रह्म॑णा॒ ऽन्तरि॑क्ष॒ मधि॑द्यौ॒ रधि॑द्यौ र॒न्तरि॑क्ष॒म् ब्रह्म॑णा । \newline
8. अधि॑द्यौ॒रित्यधि॑ - द्यौः॒ । \newline
9. अ॒न्तरि॑क्ष॒म् ब्रह्म॑णा॒ ब्रह्म॑णा॒ ऽन्तरि॑क्ष म॒न्तरि॑क्ष॒म् ब्रह्म॑णा वि॒ष्टा वि॒ष्टा ब्रह्म॑णा॒ ऽन्तरि॑क्ष म॒न्तरि॑क्ष॒म् ब्रह्म॑णा वि॒ष्टा । \newline
10. ब्रह्म॑णा वि॒ष्टा वि॒ष्टा ब्रह्म॑णा॒ ब्रह्म॑णा वि॒ष्टा म॒रुतो॑ म॒रुतो॑ वि॒ष्टा ब्रह्म॑णा॒ ब्रह्म॑णा वि॒ष्टा म॒रुतः॑ । \newline
11. वि॒ष्टा म॒रुतो॑ म॒रुतो॑ वि॒ष्टा वि॒ष्टा म॒रुत॑ स्ते ते म॒रुतो॑ वि॒ष्टा वि॒ष्टा म॒रुत॑ स्ते । \newline
12. म॒रुत॑ स्ते ते म॒रुतो॑ म॒रुत॑ स्ते गो॒प्तारो॑ गो॒प्तार॑ स्ते म॒रुतो॑ म॒रुत॑ स्ते गो॒प्तारः॑ । \newline
13. ते॒ गो॒प्तारो॑ गो॒प्तार॑ स्ते ते गो॒प्तारो॑ वा॒युर् वा॒युर् गो॒प्तार॑ स्ते ते गो॒प्तारो॑ वा॒युः । \newline
14. गो॒प्तारो॑ वा॒युर् वा॒युर् गो॒प्तारो॑ गो॒प्तारो॑ वा॒युर् विय॑त्तो॒ विय॑त्तो वा॒युर् गो॒प्तारो॑ गो॒प्तारो॑ वा॒युर् विय॑त्तः । \newline
15. वा॒युर् विय॑त्तो॒ विय॑त्तो वा॒युर् वा॒युर् विय॑त्तो ऽस्या मस्यां॒ ॅविय॑त्तो वा॒युर् वा॒युर् विय॑त्तो ऽस्याम् । \newline
16. विय॑त्तो ऽस्या मस्यां॒ ॅविय॑त्तो॒ विय॑त्तो ऽस्या॒म् ताम् ता म॑स्यां॒ ॅविय॑त्तो॒ विय॑त्तो ऽस्या॒म् ताम् । \newline
17. विय॑त्त॒ इति॒ वि - य॒त्तः॒ । \newline
18. अ॒स्या॒म् ताम् ता म॑स्या मस्या॒म् ता म॒ह म॒हम् ता म॑स्या मस्या॒म् ता म॒हम् । \newline
19. ता म॒ह म॒हम् ताम् ता म॒हम् प्र प्राहम् ताम् ता म॒हम् प्र । \newline
20. अ॒हम् प्र प्राह म॒हम् प्र प॑द्ये पद्ये॒ प्राह म॒हम् प्र प॑द्ये । \newline
21. प्र प॑द्ये पद्ये॒ प्र प्र प॑द्ये॒ सा सा प॑द्ये॒ प्र प्र प॑द्ये॒ सा । \newline
22. प॒द्ये॒ सा सा प॑द्ये पद्ये॒ सा मे॑ मे॒ सा प॑द्ये पद्ये॒ सा मे᳚ । \newline
23. सा मे॑ मे॒ सा सा मे॒ शर्म॒ शर्म॑ मे॒ सा सा मे॒ शर्म॑ । \newline
24. मे॒ शर्म॒ शर्म॑ मे मे॒ शर्म॑ च च॒ शर्म॑ मे मे॒ शर्म॑ च । \newline
25. शर्म॑ च च॒ शर्म॒ शर्म॑ च॒ वर्म॒ वर्म॑ च॒ शर्म॒ शर्म॑ च॒ वर्म॑ । \newline
26. च॒ वर्म॒ वर्म॑ च च॒ वर्म॑ च च॒ वर्म॑ च च॒ वर्म॑ च । \newline
27. वर्म॑ च च॒ वर्म॒ वर्म॑ चा स्त्वस्तु च॒ वर्म॒ वर्म॑ चास्तु । \newline
28. चा॒ स्त्व॒स्तु॒ च॒ चा॒स्तु॒ द्यौर् द्यौ र॑स्तु च चास्तु॒ द्यौः । \newline
29. अ॒स्तु॒ द्यौर् द्यौर॑ स्त्वस्तु॒ द्यौ रप॑राजि॒ता ऽप॑राजिता॒ द्यौ र॑स्त्वस्तु॒ द्यौ रप॑राजिता । \newline
30. द्यौ रप॑राजि॒ता ऽप॑राजिता॒ द्यौर् द्यौ रप॑राजिता॒ ऽमृते॑ना॒ मृते॒ना प॑राजिता॒ द्यौर् द्यौरप॑राजिता॒ ऽमृते॑न । \newline
31. अप॑राजिता॒ ऽमृते॑ना॒ मृते॒ना प॑राजि॒ता ऽप॑राजिता॒ ऽमृते॑न वि॒ष्टा वि॒ष्टा ऽमृते॒ना प॑राजि॒ता ऽप॑राजिता॒ ऽमृते॑न वि॒ष्टा । \newline
32. अप॑राजि॒तेत्यप॑रा - जि॒ता॒ । \newline
33. अ॒मृते॑न वि॒ष्टा वि॒ष्टा ऽमृते॑ना॒ मृते॑न वि॒ष्टा ऽऽदि॒त्या आ॑दि॒त्या वि॒ष्टा ऽमृते॑ना॒ मृते॑न वि॒ष्टा ऽऽदि॒त्याः । \newline
34. वि॒ष्टा ऽऽदि॒त्या आ॑दि॒त्या वि॒ष्टा वि॒ष्टा ऽऽदि॒त्या स्ते॑ त आदि॒त्या वि॒ष्टा वि॒ष्टा ऽऽदि॒त्या स्ते᳚ । \newline
35. आ॒दि॒त्या स्ते॑ त आदि॒त्या आ॑दि॒त्या स्ते॑ गो॒प्तारो॑ गो॒प्तार॑ स्त आदि॒त्या आ॑दि॒त्या स्ते॑ गो॒प्तारः॑ । \newline
36. ते॒ गो॒प्तारो॑ गो॒प्तार॑ स्ते ते गो॒प्तारः॒ सूर्यः॒ सूर्यो॑ गो॒प्तार॑ स्ते ते गो॒प्तारः॒ सूर्यः॑ । \newline
37. गो॒प्तारः॒ सूर्यः॒ सूर्यो॑ गो॒प्तारो॑ गो॒प्तारः॒ सूर्यो॒ विय॑त्तो॒ विय॑त्तः॒ सूर्यो॑ गो॒प्तारो॑ गो॒प्तारः॒ सूर्यो॒ विय॑त्तः । \newline
38. सूर्यो॒ विय॑त्तो॒ विय॑त्तः॒ सूर्यः॒ सूर्यो॒ विय॑त्तो ऽस्या मस्यां॒ ॅविय॑त्तः॒ सूर्यः॒ सूर्यो॒ विय॑त्तो ऽस्याम् । \newline
39. विय॑त्तो ऽस्या मस्यां॒ ॅविय॑त्तो॒ विय॑त्तो ऽस्या॒म् ताम् ता म॑स्यां॒ ॅविय॑त्तो॒ विय॑त्तो ऽस्या॒म् ताम् । \newline
40. विय॑त्त॒ इति॒ वि - य॒त्तः॒ । \newline
41. अ॒स्या॒म् ताम् ता म॑स्या मस्या॒म् ता म॒ह म॒हम् ता म॑स्या मस्या॒म् ता म॒हम् । \newline
42. ता म॒ह म॒हम् ताम् ता म॒हम् प्र प्राहम् ताम् ता म॒हम् प्र । \newline
43. अ॒हम् प्र प्राह म॒हम् प्र प॑द्ये पद्ये॒ प्राह म॒हम् प्र प॑द्ये । \newline
44. प्र प॑द्ये पद्ये॒ प्र प्र प॑द्ये॒ सा सा प॑द्ये॒ प्र प्र प॑द्ये॒ सा । \newline
45. प॒द्ये॒ सा सा प॑द्ये पद्ये॒ सा मे॑ मे॒ सा प॑द्ये पद्ये॒ सा मे᳚ । \newline
46. सा मे॑ मे॒ सा सा मे॒ शर्म॒ शर्म॑ मे॒ सा सा मे॒ शर्म॑ । \newline
47. मे॒ शर्म॒ शर्म॑ मे मे॒ शर्म॑ च च॒ शर्म॑ मे मे॒ शर्म॑ च । \newline
48. शर्म॑ च च॒ शर्म॒ शर्म॑ च॒ वर्म॒ वर्म॑ च॒ शर्म॒ शर्म॑ च॒ वर्म॑ । \newline
49. च॒ वर्म॒ वर्म॑ च च॒ वर्म॑ च च॒ वर्म॑ च च॒ वर्म॑ च । \newline
50. वर्म॑ च च॒ वर्म॒ वर्म॑ चा स्त्वस्तु च॒ वर्म॒ वर्म॑ चास्तु । \newline
51. चा॒ स्त्व॒स्तु॒ च॒ चा॒स्तु॒ । \newline
52. अ॒स्त्वित्य॑स्तु । \newline
\pagebreak
\markright{ TS 4.4.6.1  \hfill https://www.vedavms.in \hfill}

\section{ TS 4.4.6.1 }

\textbf{TS 4.4.6.1 } \newline
\textbf{Samhita Paata} \newline

बृह॒स्पति॑स्त्वा सादयतु पृथि॒व्याः पृ॒ष्ठे ज्योति॑ष्मतीं॒ ॅविश्व॑स्मै प्रा॒णाया॑पा॒नाय॒ विश्वं॒ ज्योति॑र्यच्छा॒- ग्निस्तेऽधि॑पति र्वि॒श्वक॑र्मा त्वा सादयत्व॒न्तरि॑क्षस्य पृ॒ष्ठे ज्योति॑ष्मतीं॒ ॅविश्व॑स्मै प्रा॒णाया॑पा॒नाय॒ विश्वं॒ ज्योति॑र्यच्छ वा॒युस्तेऽधि॑पतिः प्र॒जाप॑तिस्त्वा सादयतु दि॒वः पृ॒ष्ठे ज्योति॑ष्मतीं॒ ॅविश्व॑स्मै प्रा॒णाया॑पा॒नाय॒ विश्वं॒ ज्योति॑र्यच्छ परमे॒ष्ठी तेऽधि॑पतिः पुरोवात॒सनि॑रस्य भ्र॒सनि॑रसि विद्यु॒थ्सनि॑ - [  ] \newline

\textbf{Pada Paata} \newline

बृह॒स्पतिः॑ । त्वा॒ । सा॒द॒य॒तु॒ । पृ॒थि॒व्याः । पृ॒ष्ठे । ज्योति॑ष्मतीम् । विश्व॑स्मै । प्रा॒णायेति॑ प्र-अ॒नाय॑ । अ॒पा॒नायेत्य॑प - अ॒नाय॑ । विश्व᳚म् । ज्योतिः॑ । य॒च्छ॒ । अ॒ग्निः । ते॒ । अधि॑पति॒रित्यधि॑ - प॒तिः॒ । वि॒श्वक॒र्मेति॑ वि॒श्व - क॒र्मा॒ । त्वा॒ । सा॒द॒य॒तु॒ । अ॒न्तरि॑क्षस्य । पृ॒ष्ठे । ज्योति॑ष्मतीम् । विश्व॑स्मै । प्रा॒णायेति॑ प्र - अ॒नाय॑ । अ॒पा॒नायेत्य॑प - अ॒नाय॑ । विश्व᳚म् । ज्योतिः॑ । य॒च्छ॒ । वा॒युः । ते॒ । अधि॑पति॒रित्यधि॑ - प॒तिः॒ । प्र॒जाप॑ति॒रिति॑ प्र॒जा - प॒तिः॒ । त्वा॒ । सा॒द॒य॒तु॒ । दि॒वः । पृ॒ष्ठे । ज्योति॑ष्मतीम् । विश्व॑स्मै । प्रा॒णायेति॑ प्र - अ॒नाय॑ । अ॒पा॒नायेत्य॑प - अ॒नाय॑ । विश्व᳚म् । ज्योतिः॑ । य॒च्छ॒ । प॒र॒मे॒ष्ठी । ते॒ । अधि॑पति॒रित्यधि॑ - प॒तिः॒ । पु॒रो॒वा॒त॒सनि॒रिति॑ पुरोवात - सनिः॑ । अ॒स्य॒ । अ॒भ्र॒सनि॒रित्य॑भ्र - सनिः॑ । अ॒सि॒ । वि॒द्यु॒थ्सनि॒रिति॑ विद्युत् - सनिः॑ ।  \newline


\textbf{Krama Paata} \newline

बृह॒स्पति॑स्त्वा । त्वा॒ सा॒द॒य॒तु॒ । सा॒द॒य॒तु॒ पृ॒थि॒व्याः । पृ॒थि॒व्याः पृ॒ष्ठे । पृ॒ष्ठे ज्योति॑ष्मतीम् । ज्योति॑ष्मती॒म् ॅविश्व॑स्मै । विश्व॑स्मै प्रा॒णाय॑ । प्रा॒णाया॑पा॒नाय॑ । प्रा॒णायेति॑ प्र - अ॒नाय॑ । अ॒पा॒नाय॒ विश्व᳚म् । अ॒पा॒नायेत्य॑प - अ॒नाय॑ । विश्वं॒ ज्योतिः॑ । ज्योति॑र् यच्छ । य॒च्छा॒ग्निः । अ॒ग्निस्ते᳚ । तेऽधि॑पतिः । अधि॑पतिर् वि॒श्वक॑र्मा । अधि॑पति॒रित्यधि॑ - प॒तिः॒ । वि॒श्वक॑र्मा त्वा । वि॒श्वक॒र्मेति॑ वि॒श्व - क॒र्मा॒ । त्वा॒ सा॒द॒य॒तु॒ । सा॒द॒य॒त्व॒न्तरि॑क्षस्य । अ॒न्तरि॑क्षस्य पृ॒ष्ठे । पृ॒ष्ठे ज्योति॑ष्मतीम् । ज्योति॑ष्मती॒म् ॅविश्व॑स्मै । विश्व॑स्मै प्रा॒णाय॑ । प्रा॒णाया॑पा॒नाय॑ । प्रा॒णायेति॑ प्र - अ॒नाय॑ । अ॒पा॒नाय॒ विश्व᳚म् । अ॒पा॒नायेत्य॑प - अ॒नाय॑ । विश्व॒म् ज्योतिः॑ । ज्योति॑र् यच्छ । य॒च्छ॒ वा॒युः । वा॒युस्ते᳚ । तेऽधि॑पतिः । अधि॑पतिः प्र॒जाप॑तिः । अधि॑पति॒रित्यधि॑ - प॒तिः॒ । प्र॒जाप॑तिस्त्वा । प्र॒जाप॑ति॒रिति॑ प्र॒जा - प॒तिः॒ । त्वा॒ सा॒द॒य॒तु॒ । सा॒द॒य॒तु॒ दि॒वः । दि॒वः पृ॒ष्ठे । पृ॒ष्ठे ज्योति॑ष्मतीम् । ज्योति॑ष्मती॒म् ॅविश्व॑स्मै । विश्व॑स्मै प्रा॒णाय॑ । प्रा॒णाया॑पा॒नाय॑ । प्रा॒णायेति॑ प्र - अ॒नाय॑ । अ॒पा॒नाय॒ विश्व᳚म् । अ॒पा॒नायेत्य॑प - अ॒नाय॑ । विश्व॒म् ज्योतिः॑ । ज्योति॑र् यच्छ । य॒च्छ॒ प॒र॒मे॒ष्ठी । प॒र॒मे॒ष्ठी ते᳚ । तेऽधि॑पतिः । अधि॑पतिः पुरोवात॒सनिः॑ । अधि॑पति॒रित्यधि॑ - प॒तिः॒ । पु॒रो॒वा॒त॒सनि॑रसि । पु॒रो॒वा॒त॒सनि॒रिति॑ पुरोवात - सनिः॑ । अ॒स्य॒भ्र॒सनिः॑ । अ॒भ्र॒सनि॑रसि । अ॒भ्र॒सनि॒रित्य॑भ्र - सनिः॑ । अ॒सि॒ वि॒द्यु॒थ्सनिः॑ । वि॒द्यु॒थ्सनि॑रसि । वि॒द्यु॒थ्सनि॒रिति॑ विद्युत् - सनिः॑ \newline

\textbf{Jatai Paata} \newline

1. बृह॒स्पति॑ स्त्वा त्वा॒ बृह॒स्पति॒र् बृह॒स्पति॑ स्त्वा । \newline
2. त्वा॒ सा॒द॒य॒तु॒ सा॒द॒य॒तु॒ त्वा॒ त्वा॒ सा॒द॒य॒तु॒ । \newline
3. सा॒द॒य॒तु॒ पृ॒थि॒व्याः पृ॑थि॒व्याः सा॑दयतु सादयतु पृथि॒व्याः । \newline
4. पृ॒थि॒व्याः पृ॒ष्ठे पृ॒ष्ठे पृ॑थि॒व्याः पृ॑थि॒व्याः पृ॒ष्ठे । \newline
5. पृ॒ष्ठे ज्योति॑ष्मती॒म् ज्योति॑ष्मतीम् पृ॒ष्ठे पृ॒ष्ठे ज्योति॑ष्मतीम् । \newline
6. ज्योति॑ष्मतीं॒ ॅविश्व॑स्मै॒ विश्व॑स्मै॒ ज्योति॑ष्मती॒म् ज्योति॑ष्मतीं॒ ॅविश्व॑स्मै । \newline
7. विश्व॑स्मै प्रा॒णाय॑ प्रा॒णाय॒ विश्व॑स्मै॒ विश्व॑स्मै प्रा॒णाय॑ । \newline
8. प्रा॒णाया॑ पा॒नाया॑ पा॒नाय॑ प्रा॒णाय॑ प्रा॒णाया॑ पा॒नाय॑ । \newline
9. प्रा॒णायेति॑ प्र - अ॒नाय॑ । \newline
10. अ॒पा॒नाय॒ विश्वं॒ ॅविश्व॑ मपा॒नाया॑ पा॒नाय॒ विश्व᳚म् । \newline
11. अ॒पा॒नायेत्य॑प - अ॒नाय॑ । \newline
12. विश्व॒म् ज्योति॒र् ज्योति॒र् विश्वं॒ ॅविश्व॒म् ज्योतिः॑ । \newline
13. ज्योति॑र् यच्छ यच्छ॒ ज्योति॒र् ज्योति॑र् यच्छ । \newline
14. य॒च्छा॒ग्नि र॒ग्निर् य॑च्छ यच्छा॒ग्निः । \newline
15. अ॒ग्नि स्ते॑ ते॒ ऽग्नि र॒ग्नि स्ते᳚ । \newline
16. ते ऽधि॑पति॒ रधि॑पति स्ते॒ ते ऽधि॑पतिः । \newline
17. अधि॑पतिर् वि॒श्वक॑र्मा वि॒श्वक॒र्मा ऽधि॑पति॒ रधि॑पतिर् वि॒श्वक॑र्मा । \newline
18. अधि॑पति॒रित्यधि॑ - प॒तिः॒ । \newline
19. वि॒श्वक॑र्मा त्वा त्वा वि॒श्वक॑र्मा वि॒श्वक॑र्मा त्वा । \newline
20. वि॒श्वक॒र्मेति॑ वि॒श्व - क॒र्मा॒ । \newline
21. त्वा॒ सा॒द॒य॒तु॒ सा॒द॒य॒तु॒ त्वा॒ त्वा॒ सा॒द॒य॒तु॒ । \newline
22. सा॒द॒य॒ त्व॒न्तरि॑क्षस्या॒ न्तरि॑क्षस्य सादयतु सादय त्व॒न्तरि॑क्षस्य । \newline
23. अ॒न्तरि॑क्षस्य पृ॒ष्ठे पृ॒ष्ठे अ॒न्तरि॑क्षस्या॒ न्तरि॑क्षस्य पृ॒ष्ठे । \newline
24. पृ॒ष्ठे ज्योति॑ष्मती॒म् ज्योति॑ष्मतीम् पृ॒ष्ठे पृ॒ष्ठे ज्योति॑ष्मतीम् । \newline
25. ज्योति॑ष्मतीं॒ ॅविश्व॑स्मै॒ विश्व॑स्मै॒ ज्योति॑ष्मती॒म् ज्योति॑ष्मतीं॒ ॅविश्व॑स्मै । \newline
26. विश्व॑स्मै प्रा॒णाय॑ प्रा॒णाय॒ विश्व॑स्मै॒ विश्व॑स्मै प्रा॒णाय॑ । \newline
27. प्रा॒णाया॑ पा॒नाया॑ पा॒नाय॑ प्रा॒णाय॑ प्रा॒णाया॑ पा॒नाय॑ । \newline
28. प्रा॒णायेति॑ प्र - अ॒नाय॑ । \newline
29. अ॒पा॒नाय॒ विश्वं॒ ॅविश्व॑ मपा॒नाया॑ पा॒नाय॒ विश्व᳚म् । \newline
30. अ॒पा॒नायेत्य॑प - अ॒नाय॑ । \newline
31. विश्व॒म् ज्योति॒र् ज्योति॒र् विश्वं॒ ॅविश्व॒म् ज्योतिः॑ । \newline
32. ज्योति॑र् यच्छ यच्छ॒ ज्योति॒र् ज्योति॑र् यच्छ । \newline
33. य॒च्छ॒ वा॒युर् वा॒युर् य॑च्छ यच्छ वा॒युः । \newline
34. वा॒यु स्ते॑ ते वा॒युर् वा॒यु स्ते᳚ । \newline
35. ते ऽधि॑पति॒ रधि॑पति स्ते॒ ते ऽधि॑पतिः । \newline
36. अधि॑पतिः प्र॒जाप॑तिः प्र॒जाप॑ति॒ रधि॑पति॒ रधि॑पतिः प्र॒जाप॑तिः । \newline
37. अधि॑पति॒रित्यधि॑ - प॒तिः॒ । \newline
38. प्र॒जाप॑ति स्त्वा त्वा प्र॒जाप॑तिः प्र॒जाप॑ति स्त्वा । \newline
39. प्र॒जाप॑ति॒रिति॑ प्र॒जा - प॒तिः॒ । \newline
40. त्वा॒ सा॒द॒य॒तु॒ सा॒द॒य॒तु॒ त्वा॒ त्वा॒ सा॒द॒य॒तु॒ । \newline
41. सा॒द॒य॒तु॒ दि॒वो दि॒वः सा॑दयतु सादयतु दि॒वः । \newline
42. दि॒वः पृ॒ष्ठे पृ॒ष्ठे दि॒वो दि॒वः पृ॒ष्ठे । \newline
43. पृ॒ष्ठे ज्योति॑ष्मती॒म् ज्योति॑ष्मतीम् पृ॒ष्ठे पृ॒ष्ठे ज्योति॑ष्मतीम् । \newline
44. ज्योति॑ष्मतीं॒ ॅविश्व॑स्मै॒ विश्व॑स्मै॒ ज्योति॑ष्मती॒म् ज्योति॑ष्मतीं॒ ॅविश्व॑स्मै । \newline
45. विश्व॑स्मै प्रा॒णाय॑ प्रा॒णाय॒ विश्व॑स्मै॒ विश्व॑स्मै प्रा॒णाय॑ । \newline
46. प्रा॒णाया॑ पा॒नाया॑ पा॒नाय॑ प्रा॒णाय॑ प्रा॒णाया॑ पा॒नाय॑ । \newline
47. प्रा॒णायेति॑ प्र - अ॒नाय॑ । \newline
48. अ॒पा॒नाय॒ विश्वं॒ ॅविश्व॑ मपा॒नाया॑ पा॒नाय॒ विश्व᳚म् । \newline
49. अ॒पा॒नायेत्य॑प - अ॒नाय॑ । \newline
50. विश्व॒म् ज्योति॒र् ज्योति॒र् विश्वं॒ ॅविश्व॒म् ज्योतिः॑ । \newline
51. ज्योति॑र् यच्छ यच्छ॒ ज्योति॒र् ज्योति॑र् यच्छ । \newline
52. य॒च्छ॒ प॒र॒मे॒ष्ठी प॑रमे॒ष्ठी य॑च्छ यच्छ परमे॒ष्ठी । \newline
53. प॒र॒मे॒ष्ठी ते॑ ते परमे॒ष्ठी प॑रमे॒ष्ठी ते᳚ । \newline
54. ते ऽधि॑पति॒ रधि॑पति स्ते॒ ते ऽधि॑पतिः । \newline
55. अधि॑पतिः पुरोवात॒सनिः॑ पुरोवात॒सनि॒ रधि॑पति॒ रधि॑पतिः पुरोवात॒सनिः॑ । \newline
56. अधि॑पति॒रित्यधि॑ - प॒तिः॒ । \newline
57. पु॒रो॒वा॒त॒सनि॑ रस्यसि पुरोवात॒सनिः॑ पुरोवात॒सनि॑ रसि । \newline
58. पु॒रो॒वा॒त॒सनि॒रिति॑ पुरोवात - सनिः॑ । \newline
59. अ॒स्य॒भ्र॒सनि॑ रभ्र॒सनि॑ रस्य स्यभ्र॒सनिः॑ । \newline
60. अ॒भ्र॒सनि॑ रस्यस्य भ्र॒सनि॑ रभ्र॒सनि॑ रसि । \newline
61. अ॒भ्र॒सनि॒रित्य॑भ्र - सनिः॑ । \newline
62. अ॒सि॒ वि॒द्यु॒थ्सनि॑र् विद्यु॒थ्सनि॑ रस्यसि विद्यु॒थ्सनिः॑ । \newline
63. वि॒द्यु॒थ्सनि॑ रस्यसि विद्यु॒थ्सनि॑र् विद्यु॒थ्सनि॑ रसि । \newline
64. वि॒द्यु॒थ्सनि॒रिति॑ विद्युत् - सनिः॑ । \newline

\textbf{Ghana Paata } \newline

1. बृह॒स्पति॑ स्त्वा त्वा॒ बृह॒स्पति॒र् बृह॒स्पति॑ स्त्वा सादयतु सादयतु त्वा॒ बृह॒स्पति॒र् बृह॒स्पति॑ स्त्वा सादयतु । \newline
2. त्वा॒ सा॒द॒य॒तु॒ सा॒द॒य॒तु॒ त्वा॒ त्वा॒ सा॒द॒य॒तु॒ पृ॒थि॒व्याः पृ॑थि॒व्याः सा॑दयतु त्वा त्वा सादयतु पृथि॒व्याः । \newline
3. सा॒द॒य॒तु॒ पृ॒थि॒व्याः पृ॑थि॒व्याः सा॑दयतु सादयतु पृथि॒व्याः पृ॒ष्ठे पृ॒ष्ठे पृ॑थि॒व्याः सा॑दयतु सादयतु पृथि॒व्याः पृ॒ष्ठे । \newline
4. पृ॒थि॒व्याः पृ॒ष्ठे पृ॒ष्ठे पृ॑थि॒व्याः पृ॑थि॒व्याः पृ॒ष्ठे ज्योति॑ष्मती॒म् ज्योति॑ष्मतीम् पृ॒ष्ठे पृ॑थि॒व्याः पृ॑थि॒व्याः पृ॒ष्ठे ज्योति॑ष्मतीम् । \newline
5. पृ॒ष्ठे ज्योति॑ष्मती॒म् ज्योति॑ष्मतीम् पृ॒ष्ठे पृ॒ष्ठे ज्योति॑ष्मतीं॒ ॅविश्व॑स्मै॒ विश्व॑स्मै॒ ज्योति॑ष्मतीम् पृ॒ष्ठे पृ॒ष्ठे ज्योति॑ष्मतीं॒ ॅविश्व॑स्मै । \newline
6. ज्योति॑ष्मतीं॒ ॅविश्व॑स्मै॒ विश्व॑स्मै॒ ज्योति॑ष्मती॒म् ज्योति॑ष्मतीं॒ ॅविश्व॑स्मै प्रा॒णाय॑ प्रा॒णाय॒ विश्व॑स्मै॒ ज्योति॑ष्मती॒म् ज्योति॑ष्मतीं॒ ॅविश्व॑स्मै प्रा॒णाय॑ । \newline
7. विश्व॑स्मै प्रा॒णाय॑ प्रा॒णाय॒ विश्व॑स्मै॒ विश्व॑स्मै प्रा॒णाया॑ पा॒नाया॑ पा॒नाय॑ प्रा॒णाय॒ विश्व॑स्मै॒ विश्व॑स्मै प्रा॒णाया॑ पा॒नाय॑ । \newline
8. प्रा॒णाया॑ पा॒नाया॑ पा॒नाय॑ प्रा॒णाय॑ प्रा॒णाया॑ पा॒नाय॒ विश्वं॒ ॅविश्व॑ मपा॒नाय॑ प्रा॒णाय॑ प्रा॒णाया॑ पा॒नाय॒ विश्व᳚म् । \newline
9. प्रा॒णायेति॑ प्र - अ॒नाय॑ । \newline
10. अ॒पा॒नाय॒ विश्वं॒ ॅविश्व॑ मपा॒नाया॑ पा॒नाय॒ विश्व॒म् ज्योति॒र् ज्योति॒र् विश्व॑ मपा॒नाया॑ पा॒नाय॒ विश्व॒म् ज्योतिः॑ । \newline
11. अ॒पा॒नायेत्य॑प - अ॒नाय॑ । \newline
12. विश्व॒म् ज्योति॒र् ज्योति॒र् विश्वं॒ ॅविश्व॒म् ज्योति॑र् यच्छ यच्छ॒ ज्योति॒र् विश्वं॒ ॅविश्व॒म् ज्योति॑र् यच्छ । \newline
13. ज्योति॑र् यच्छ यच्छ॒ ज्योति॒र् ज्योति॑र् यच्छा॒ग्नि र॒ग्निर् य॑च्छ॒ ज्योति॒र् ज्योति॑र् यच्छा॒ग्निः । \newline
14. य॒च्छा॒ग्नि र॒ग्निर् य॑च्छ यच्छा॒ ग्नि स्ते॑ ते॒ ऽग्निर् य॑च्छ यच्छा॒ ग्नि स्ते᳚ । \newline
15. अ॒ग्नि स्ते॑ ते॒ ऽग्नि र॒ग्नि स्ते ऽधि॑पति॒ रधि॑पति स्ते॒ ऽग्नि र॒ग्नि स्ते ऽधि॑पतिः । \newline
16. ते ऽधि॑पति॒ रधि॑पति स्ते॒ ते ऽधि॑पतिर् वि॒श्वक॑र्मा वि॒श्वक॒र्मा ऽधि॑पति स्ते॒ ते ऽधि॑पतिर् वि॒श्वक॑र्मा । \newline
17. अधि॑पतिर् वि॒श्वक॑र्मा वि॒श्वक॒र्मा ऽधि॑पति॒ रधि॑पतिर् वि॒श्वक॑र्मा त्वा त्वा वि॒श्वक॒र्मा ऽधि॑पति॒ रधि॑पतिर् वि॒श्वक॑र्मा त्वा । \newline
18. अधि॑पति॒रित्यधि॑ - प॒तिः॒ । \newline
19. वि॒श्वक॑र्मा त्वा त्वा वि॒श्वक॑र्मा वि॒श्वक॑र्मा त्वा सादयतु सादयतु त्वा वि॒श्वक॑र्मा वि॒श्वक॑र्मा त्वा सादयतु । \newline
20. वि॒श्वक॒र्मेति॑ वि॒श्व - क॒र्मा॒ । \newline
21. त्वा॒ सा॒द॒य॒तु॒ सा॒द॒य॒तु॒ त्वा॒ त्वा॒ सा॒द॒य॒त्व॒ न्तरि॑क्षस्या॒ न्तरि॑क्षस्य सादयतु त्वा त्वा सादयत्व॒ न्तरि॑क्षस्य । \newline
22. सा॒द॒य॒त्व॒ न्तरि॑क्षस्या॒ न्तरि॑क्षस्य सादयतु सादयत्व॒ न्तरि॑क्षस्य पृ॒ष्ठे पृ॒ष्ठे अ॒न्तरि॑क्षस्य सादयतु सादयत्व॒ न्तरि॑क्षस्य पृ॒ष्ठे । \newline
23. अ॒न्तरि॑क्षस्य पृ॒ष्ठे पृ॒ष्ठे अ॒न्तरि॑क्षस्या॒ न्तरि॑क्षस्य पृ॒ष्ठे ज्योति॑ष्मती॒म् ज्योति॑ष्मतीम् पृ॒ष्ठे अ॒न्तरि॑क्षस्या॒ न्तरि॑क्षस्य पृ॒ष्ठे ज्योति॑ष्मतीम् । \newline
24. पृ॒ष्ठे ज्योति॑ष्मती॒म् ज्योति॑ष्मतीम् पृ॒ष्ठे पृ॒ष्ठे ज्योति॑ष्मतीं॒ ॅविश्व॑स्मै॒ विश्व॑स्मै॒ ज्योति॑ष्मतीम् पृ॒ष्ठे पृ॒ष्ठे ज्योति॑ष्मतीं॒ ॅविश्व॑स्मै । \newline
25. ज्योति॑ष्मतीं॒ ॅविश्व॑स्मै॒ विश्व॑स्मै॒ ज्योति॑ष्मती॒म् ज्योति॑ष्मतीं॒ ॅविश्व॑स्मै प्रा॒णाय॑ प्रा॒णाय॒ विश्व॑स्मै॒ ज्योति॑ष्मती॒म् ज्योति॑ष्मतीं॒ ॅविश्व॑स्मै प्रा॒णाय॑ । \newline
26. विश्व॑स्मै प्रा॒णाय॑ प्रा॒णाय॒ विश्व॑स्मै॒ विश्व॑स्मै प्रा॒णाया॑ पा॒नाया॑ पा॒नाय॑ प्रा॒णाय॒ विश्व॑स्मै॒ विश्व॑स्मै प्रा॒णाया॑ पा॒नाय॑ । \newline
27. प्रा॒णाया॑ पा॒नाया॑ पा॒नाय॑ प्रा॒णाय॑ प्रा॒णाया॑ पा॒नाय॒ विश्वं॒ ॅविश्व॑ मपा॒नाय॑ प्रा॒णाय॑ प्रा॒णाया॑ पा॒नाय॒ विश्व᳚म् । \newline
28. प्रा॒णायेति॑ प्र - अ॒नाय॑ । \newline
29. अ॒पा॒नाय॒ विश्वं॒ ॅविश्व॑ मपा॒नाया॑ पा॒नाय॒ विश्व॒म् ज्योति॒र् ज्योति॒र् विश्व॑ मपा॒नाया॑ पा॒नाय॒ विश्व॒म् ज्योतिः॑ । \newline
30. अ॒पा॒नायेत्य॑प - अ॒नाय॑ । \newline
31. विश्व॒म् ज्योति॒र् ज्योति॒र् विश्वं॒ ॅविश्व॒म् ज्योति॑र् यच्छ यच्छ॒ ज्योति॒र् विश्वं॒ ॅविश्व॒म् ज्योति॑र् यच्छ । \newline
32. ज्योति॑र् यच्छ यच्छ॒ ज्योति॒र् ज्योति॑र् यच्छ वा॒युर् वा॒युर् य॑च्छ॒ ज्योति॒र् ज्योति॑र् यच्छ वा॒युः । \newline
33. य॒च्छ॒ वा॒युर् वा॒युर् य॑च्छ यच्छ वा॒यु स्ते॑ ते वा॒युर् य॑च्छ यच्छ वा॒यु स्ते᳚ । \newline
34. वा॒यु स्ते॑ ते वा॒युर् वा॒यु स्ते ऽधि॑पति॒ रधि॑पति स्ते वा॒युर् वा॒यु स्ते ऽधि॑पतिः । \newline
35. ते ऽधि॑पति॒ रधि॑पति स्ते॒ ते ऽधि॑पतिः प्र॒जाप॑तिः प्र॒जाप॑ति॒ रधि॑पति स्ते॒ ते ऽधि॑पतिः प्र॒जाप॑तिः । \newline
36. अधि॑पतिः प्र॒जाप॑तिः प्र॒जाप॑ति॒ रधि॑पति॒ रधि॑पतिः प्र॒जाप॑ति स्त्वा त्वा प्र॒जाप॑ति॒ रधि॑पति॒ रधि॑पतिः प्र॒जाप॑ति स्त्वा । \newline
37. अधि॑पति॒रित्यधि॑ - प॒तिः॒ । \newline
38. प्र॒जाप॑ति स्त्वा त्वा प्र॒जाप॑तिः प्र॒जाप॑ति स्त्वा सादयतु सादयतु त्वा प्र॒जाप॑तिः प्र॒जाप॑ति स्त्वा सादयतु । \newline
39. प्र॒जाप॑ति॒रिति॑ प्र॒जा - प॒तिः॒ । \newline
40. त्वा॒ सा॒द॒य॒तु॒ सा॒द॒य॒तु॒ त्वा॒ त्वा॒ सा॒द॒य॒तु॒ दि॒वो दि॒वः सा॑दयतु त्वा त्वा सादयतु दि॒वः । \newline
41. सा॒द॒य॒तु॒ दि॒वो दि॒वः सा॑दयतु सादयतु दि॒वः पृ॒ष्ठे पृ॒ष्ठे दि॒वः सा॑दयतु सादयतु दि॒वः पृ॒ष्ठे । \newline
42. दि॒वः पृ॒ष्ठे पृ॒ष्ठे दि॒वो दि॒वः पृ॒ष्ठे ज्योति॑ष्मती॒म् ज्योति॑ष्मतीम् पृ॒ष्ठे दि॒वो दि॒वः पृ॒ष्ठे ज्योति॑ष्मतीम् । \newline
43. पृ॒ष्ठे ज्योति॑ष्मती॒म् ज्योति॑ष्मतीम् पृ॒ष्ठे पृ॒ष्ठे ज्योति॑ष्मतीं॒ ॅविश्व॑स्मै॒ विश्व॑स्मै॒ ज्योति॑ष्मतीम् पृ॒ष्ठे पृ॒ष्ठे ज्योति॑ष्मतीं॒ ॅविश्व॑स्मै । \newline
44. ज्योति॑ष्मतीं॒ ॅविश्व॑स्मै॒ विश्व॑स्मै॒ ज्योति॑ष्मती॒म् ज्योति॑ष्मतीं॒ ॅविश्व॑स्मै प्रा॒णाय॑ प्रा॒णाय॒ विश्व॑स्मै॒ ज्योति॑ष्मती॒म् ज्योति॑ष्मतीं॒ ॅविश्व॑स्मै प्रा॒णाय॑ । \newline
45. विश्व॑स्मै प्रा॒णाय॑ प्रा॒णाय॒ विश्व॑स्मै॒ विश्व॑स्मै प्रा॒णाया॑ पा॒नाया॑ पा॒नाय॑ प्रा॒णाय॒ विश्व॑स्मै॒ विश्व॑स्मै प्रा॒णाया॑ पा॒नाय॑ । \newline
46. प्रा॒णाया॑ पा॒नाया॑ पा॒नाय॑ प्रा॒णाय॑ प्रा॒णाया॑ पा॒नाय॒ विश्वं॒ ॅविश्व॑ मपा॒नाय॑ प्रा॒णाय॑ प्रा॒णाया॑ पा॒नाय॒ विश्व᳚म् । \newline
47. प्रा॒णायेति॑ प्र - अ॒नाय॑ । \newline
48. अ॒पा॒नाय॒ विश्वं॒ ॅविश्व॑ मपा॒नाया॑ पा॒नाय॒ विश्व॒म् ज्योति॒र् ज्योति॒र् विश्व॑ मपा॒नाया॑ पा॒नाय॒ विश्व॒म् ज्योतिः॑ । \newline
49. अ॒पा॒नायेत्य॑प - अ॒नाय॑ । \newline
50. विश्व॒म् ज्योति॒र् ज्योति॒र् विश्वं॒ ॅविश्व॒म् ज्योति॑र् यच्छ यच्छ॒ ज्योति॒र् विश्वं॒ ॅविश्व॒म् ज्योति॑र् यच्छ । \newline
51. ज्योति॑र् यच्छ यच्छ॒ ज्योति॒र् ज्योति॑र् यच्छ परमे॒ष्ठी प॑रमे॒ष्ठी य॑च्छ॒ ज्योति॒र् ज्योति॑र् यच्छ परमे॒ष्ठी । \newline
52. य॒च्छ॒ प॒र॒मे॒ष्ठी प॑रमे॒ष्ठी य॑च्छ यच्छ परमे॒ष्ठी ते॑ ते परमे॒ष्ठी य॑च्छ यच्छ परमे॒ष्ठी ते᳚ । \newline
53. प॒र॒मे॒ष्ठी ते॑ ते परमे॒ष्ठी प॑रमे॒ष्ठी ते ऽधि॑पति॒ रधि॑पति स्ते परमे॒ष्ठी प॑रमे॒ष्ठी ते ऽधि॑पतिः । \newline
54. ते ऽधि॑पति॒ रधि॑पति स्ते॒ ते ऽधि॑पतिः पुरोवात॒सनिः॑ पुरोवात॒सनि॒ रधि॑पति स्ते॒ ते ऽधि॑पतिः पुरोवात॒सनिः॑ । \newline
55. अधि॑पतिः पुरोवात॒सनिः॑ पुरोवात॒सनि॒ रधि॑पति॒ रधि॑पतिः पुरोवात॒सनि॑ रस्यसि पुरोवात॒सनि॒ रधि॑पति॒ रधि॑पतिः पुरोवात॒सनि॑ रसि । \newline
56. अधि॑पति॒रित्यधि॑ - प॒तिः॒ । \newline
57. पु॒रो॒वा॒त॒सनि॑ रस्यसि पुरोवात॒सनिः॑ पुरोवात॒सनि॑ रस्यभ्र॒सनि॑ रभ्र॒सनि॑ रसि पुरोवात॒सनिः॑ पुरोवात॒सनि॑ रस्यभ्र॒सनिः॑ । \newline
58. पु॒रो॒वा॒त॒सनि॒रिति॑ पुरोवात - सनिः॑ । \newline
59. अ॒स्य॒भ्र॒सनि॑ रभ्र॒सनि॑ रस्य स्यभ्र॒सनि॑ रस्य स्यभ्र॒सनि॑ रस्य स्यभ्र॒सनि॑ रसि । \newline
60. अ॒भ्र॒सनि॑ रस्यस्य भ्र॒सनि॑ रभ्र॒सनि॑ रसि विद्यु॒थ्सनि॑र् विद्यु॒थ्सनि॑ रस्य भ्र॒सनि॑ रभ्र॒सनि॑ रसि विद्यु॒थ्सनिः॑ । \newline
61. अ॒भ्र॒सनि॒रित्य॑भ्र - सनिः॑ । \newline
62. अ॒सि॒ वि॒द्यु॒थ्सनि॑र् विद्यु॒थ्सनि॑ रस्यसि विद्यु॒थ्सनि॑ रस्यसि विद्यु॒थ्सनि॑ रस्यसि विद्यु॒थ्सनि॑ रसि । \newline
63. वि॒द्यु॒थ्सनि॑ रस्यसि विद्यु॒थ्सनि॑र् विद्यु॒थ्सनि॑ रसि स्तनयित्नु॒सनिः॑ स्तनयित्नु॒सनि॑ रसि विद्यु॒थ्सनि॑र् विद्यु॒थ्सनि॑रसि स्तनयित्नु॒सनिः॑ । \newline
64. वि॒द्यु॒थ्सनि॒रिति॑ विद्युत् - सनिः॑ । \newline
\pagebreak
\markright{ TS 4.4.6.2  \hfill https://www.vedavms.in \hfill}

\section{ TS 4.4.6.2 }

\textbf{TS 4.4.6.2 } \newline
\textbf{Samhita Paata} \newline

-रसि स्तनयित्नु॒सनि॑रसि वृष्टि॒सनि॑रस्य॒-ग्नेर्यान्य॑सि दे॒वाना॑मग्ने॒ यान्य॑सि वा॒योर्यान्य॑सि दे॒वानां᳚ ॅवायो॒यान्य॑स्य॒न्तरि॑क्षस्य॒ यान्य॑सि दे॒वाना॑- मन्तरिक्ष॒यान्य॑स्य॒-न्तरि॑क्षमस्य॒न्तरि॑क्षाय त्वा सलि॒लाय॑ त्वा॒ सर्णी॑काय त्वा॒ सती॑काय त्वा॒ केता॑य त्वा॒ प्रचे॑तसे त्वा॒ विव॑स्वते त्वा दि॒वस्त्वा॒ ज्योति॑ष आदि॒त्येभ्य॑स्त्व॒र्चे त्वा॑ रु॒चे त्वा᳚ द्यु॒ते त्वा॑ ( ) भा॒से त्वा॒ ज्योति॑षे त्वा यशो॒दां त्वा॒ यश॑सि तेजो॒दां त्वा॒ तेज॑सि पयो॒दां त्वा॒ पय॑सि वर्चो॒दां त्वा॒ वर्च॑सि द्रविणो॒दां त्वा॒ द्रवि॑णे सादयामि॒ तेनर्.षि॑णा॒ तेन॒ ब्रह्म॑णा॒ तया॑ दे॒वत॑याऽङ्गिर॒स्वद् ध्रु॒वा सी॑द ॥ \newline

\textbf{Pada Paata} \newline

अ॒सि॒ । स्त॒न॒यि॒त्नु॒सनि॒रिति॑ स्तनयित्नु - सनिः॑ । अ॒सि॒ । वृ॒ष्टि॒सनि॒रिति॑ वृष्टि - सनिः॑ । अ॒सि॒ । अ॒ग्नेः । यानी᳚ । अ॒सि॒ । दे॒वाना᳚म् । अ॒ग्ने॒यानीत्य॑ग्ने-यानी᳚ । अ॒सि॒ । वा॒योः । यानी᳚ । अ॒सि॒ । दे॒वाना᳚म् । वा॒यो॒यानीति॑ वायो - यानी᳚ । अ॒सि॒ । अ॒न्तरि॑क्षस्य । यानी᳚ । अ॒सि॒ । दे॒वाना᳚म् । अ॒न्त॒रि॒क्ष॒यानीत्य॑न्तरिक्ष - यानी᳚ । अ॒सि॒ । अ॒न्तरि॑क्षम् । अ॒सि॒ । अ॒न्तरि॑क्षाय । त्वा॒ । स॒लि॒लाय॑ । त्वा॒ । सर्णी॑काय । त्वा॒ । सती॑का॒येति॒ स - ती॒का॒य॒ । त्वा॒ । केता॑य । त्वा॒ । प्रचे॑तस॒ इति॒ प्र - चे॒त॒से॒ । त्वा॒ । विव॑स्वते । त्वा॒ । दि॒वः । त्वा॒ । ज्योति॑षे । आ॒दि॒त्येभ्यः॑ । त्वा॒ । ऋ॒चे । त्वा॒ । रु॒चे । त्वा॒ । द्यु॒ते । त्वा॒ ( ) । भा॒से । त्वा॒ । ज्योति॑षे । त्वा॒ । य॒शो॒दामिति॑ यशः-दाम् । त्वा॒ । यश॑सि । ते॒जो॒दामिति॑ तेजः - दाम् । त्वा॒ । तेज॑सि । प॒यो॒दामिति॑ पयः - दाम् । त्वा॒ ।पय॑सि । व॒र्चो॒दामिति॑ वर्चः-दाम् । त्वा॒ । वर्च॑सि । द्र॒वि॒णो॒दामिति॑ द्रविणः - दाम् । त्वा॒ । द्रवि॑णे । सा॒द॒या॒मि॒ । तेन॑ । ऋषि॑णा । तेन॑ । ब्रह्म॑णा । तया᳚ । दे॒वत॑या । अ॒ङ्गि॒र॒स्वत् । ध्रु॒वा । सी॒द॒ ॥  \newline


\textbf{Krama Paata} \newline

अ॒सि॒ स्त॒न॒यि॒त्नु॒सनिः॑ । स्त॒न॒यि॒त्नु॒सनि॑रसि । स्त॒न॒यि॒त्नु॒सनि॒रिति॑ स्तनयित्नु - सनिः॑ । अ॒सि॒ वृ॒ष्टि॒सनिः॑ । वृ॒ष्टि॒सनि॑रसि । वृ॒ष्टि॒सनि॒रिति॑ वृष्टि - सनिः॑ । अ॒स्य॒ग्नेः । अ॒ग्नेर् यानी᳚ । यान्य॑सि । अ॒सि॒ दे॒वाना᳚म् । दे॒वाना॑मग्ने॒यानी᳚ । अ॒ग्ने॒यान्य॑सि । अ॒ग्ने॒यानीत्य॑ग्ने - यानी᳚ । अ॒सि॒ वा॒योः । वा॒योर् यानी᳚ । यान्य॑सि । अ॒सि॒ दे॒वाना᳚म् । दे॒वानां᳚ ॅवायो॒यानी᳚ । वा॒यो॒यान्य॑सि । वा॒यो॒यानीति॑ वायो - यानी᳚ । अ॒स्य॒न्तरि॑क्षस्य । अ॒न्तरि॑क्षस्य॒ यानी᳚ । यान्य॑सि । अ॒सि॒ दे॒वाना᳚म् । दे॒वाना॑मन्तरिक्ष॒यानी᳚ । अ॒न्त॒रि॒क्ष॒यान्य॑सि । अ॒न्त॒रि॒क्ष॒यानीत्य॑न्तरिक्ष - यानी᳚ । अ॒स्य॒न्तरि॑क्षम् । अ॒न्तरि॑क्षमसि । अ॒स्य॒न्तरि॑क्षाय । अ॒न्तरि॑क्षाय त्वा । 
त्वा॒ स॒लि॒लाय॑ । स॒लि॒लाय॑ त्वा । त्वा॒ सर्णी॑काय । सर्णी॑काय त्वा । त्वा॒ सती॑काय । सती॑काय त्वा । सती॑का॒येति॒ स - ती॒का॒य॒ । त्वा॒ केता॑य । केता॑य त्वा । त्वा॒ प्रचे॑तसे । प्रचे॑तसे त्वा । प्रचे॑तस॒ इति॒ प्र - चे॒त॒से॒ । त्वा॒ विव॑स्वते । विव॑स्वते त्वा । त्वा॒ दि॒वः । दि॒वस्त्वा᳚ । त्वा॒ ज्योति॑षे । ज्योति॑ष आदि॒त्येभ्यः॑ । आ॒दि॒त्येभ्य॑स्त्वा । त्व॒र्चे । 
ऋ॒चे त्वा᳚ । त्वा॒ रु॒चे । रु॒चे त्वा᳚ । त्वा॒ द्यु॒ते । द्यु॒ते त्वा᳚ ( ) । 
त्वा॒ भा॒से । भा॒से त्वा᳚ । त्वा॒ ज्योति॑षे । ज्योति॑षे त्वा । त्वा॒ य॒शो॒दाम् । य॒शो॒दाम् त्वा᳚ । य॒शो॒दामिति॑ यशः - दाम् । त्वा॒ यश॑सि । यश॑सि तेजो॒दाम् । ते॒जो॒दां त्वा᳚ । ते॒जो॒दामिति॑ तेजः - दाम् । त्वा॒ तेज॑सि । तेज॑सि पयो॒दाम् । प॒यो॒दाम् त्वा᳚ । प॒यो॒दामिति॑ पयः - दाम् । त्वा॒ पय॑सि । पय॑सि वर्चो॒दाम् । व॒र्चो॒दाम् त्वा᳚ । व॒र्चो॒दामिति॑ वर्चः - दाम् । त्वा॒ वर्च॑सि । वर्च॑सि द्रविणो॒दाम् । द्र॒वि॒णो॒दाम् त्वा᳚ । द्र॒वि॒णो॒दामिति॑ द्रविणः - दाम् । त्वा॒ द्रवि॑णे । द्रवि॑णे सादयामि । सा॒द॒या॒मि॒ तेन॑ । तेनर्षि॑णा । ऋषि॑णा॒ तेन॑ । तेन॒ ब्रह्म॑णा । ब्रह्म॑णा॒ तया᳚ । तया॑ दे॒वत॑या । दे॒वत॑याऽङ्गिर॒स्वत् । अ॒ङ्गि॒र॒स्वद् ध्रु॒वा । ध्रु॒वा सी॑द । सी॒देति॑ सीद । \newline

\textbf{Jatai Paata} \newline

1. अ॒सि॒ स्त॒न॒यि॒त्नु॒सनिः॑ स्तनयित्नु॒सनि॑ रस्यसि स्तनयित्नु॒सनिः॑ । \newline
2. स्त॒न॒यि॒त्नु॒सनि॑ रस्यसि स्तनयित्नु॒सनिः॑ स्तनयित्नु॒सनि॑ रसि । \newline
3. स्त॒न॒यि॒त्नु॒सनि॒रिति॑ स्तनयित्नु - सनिः॑ । \newline
4. अ॒सि॒ वृ॒ष्टि॒सनि॑र् वृष्टि॒सनि॑ रस्यसि वृष्टि॒सनिः॑ । \newline
5. वृ॒ष्टि॒सनि॑ रस्यसि वृष्टि॒सनि॑र् वृष्टि॒सनि॑ रसि । \newline
6. वृ॒ष्टि॒सनि॒रिति॑ वृष्टि - सनिः॑ । \newline
7. अ॒स्य॒ ग्ने र॒ग्ने र॑स्य स्य॒ग्नेः । \newline
8. अ॒ग्नेर् यानी॒ यान्य॒ग्ने र॒ग्नेर् यानी᳚ । \newline
9. यान्य॑ स्यसि॒ यानी॒ यान्य॑सि । \newline
10. अ॒सि॒ दे॒वाना᳚म् दे॒वाना॑ मस्यसि दे॒वाना᳚म् । \newline
11. दे॒वाना॑ मग्ने॒यान्य॑ ग्ने॒यानी॑ दे॒वाना᳚म् दे॒वाना॑ मग्ने॒यानी᳚ । \newline
12. अ॒ग्ने॒यान्य॑ स्यस्य ग्ने॒यान्य॑ ग्ने॒यान्य॑सि । \newline
13. अ॒ग्ने॒यानीत्य॑ग्ने - यानी᳚ । \newline
14. अ॒सि॒ वा॒योर् वा॒यो र॑स्यसि वा॒योः । \newline
15. वा॒योर् यानी॒ यानी॑ वा॒योर् वा॒योर् यानी᳚ । \newline
16. यान्य॑स्यसि॒ यानी॒ यान्य॑सि । \newline
17. अ॒सि॒ दे॒वाना᳚म् दे॒वाना॑ मस्यसि दे॒वाना᳚म् । \newline
18. दे॒वानां᳚ ॅवायो॒यानी॑ वायो॒यानी॑ दे॒वाना᳚म् दे॒वानां᳚ ॅवायो॒यानी᳚ । \newline
19. वा॒यो॒यान्य॑ स्यसि वायो॒यानी॑ वायो॒यान्य॑सि । \newline
20. वा॒यो॒यानीति॑ वायो - यानी᳚ । \newline
21. अ॒स्य॒न्तरि॑क्षस्या॒ न्तरि॑क्षस्या स्यस्य॒ न्तरि॑क्षस्य । \newline
22. अ॒न्तरि॑क्षस्य॒ यानी॒ यान्य॒न्तरि॑क्षस्या॒ न्तरि॑क्षस्य॒ यानी᳚ । \newline
23. यान्य॑स्यसि॒ यानी॒ यान्य॑सि । \newline
24. अ॒सि॒ दे॒वाना᳚म् दे॒वाना॑ मस्यसि दे॒वाना᳚म् । \newline
25. दे॒वाना॑ मन्तरिक्ष॒यान्य॑ न्तरिक्ष॒यानी॑ दे॒वाना᳚म् दे॒वाना॑ मन्तरिक्ष॒यानी᳚ । \newline
26. अ॒न्त॒रि॒क्ष॒या न्य॑स्यस्य न्तरिक्ष॒या न्य॑न्तरिक्ष॒या न्य॑सि । \newline
27. अ॒न्त॒रि॒क्ष॒यानीत्य॑न्तरिक्ष - यानी᳚ । \newline
28. अ॒स्य॒न्तरि॑क्ष म॒न्तरि॑क्ष मस्यस्य॒ न्तरि॑क्षम् । \newline
29. अ॒न्तरि॑क्ष मस्यस्य॒ न्तरि॑क्ष म॒न्तरि॑क्ष मसि । \newline
30. अ॒स्य॒ न्तरि॑क्षाया॒ न्तरि॑क्षाया स्यस्य॒ न्तरि॑क्षाय । \newline
31. अ॒न्तरि॑क्षाय त्वा त्वा॒ ऽन्तरि॑क्षाया॒ न्तरि॑क्षाय त्वा । \newline
32. त्वा॒ स॒लि॒लाय॑ सलि॒लाय॑ त्वा त्वा सलि॒लाय॑ । \newline
33. स॒लि॒लाय॑ त्वा त्वा सलि॒लाय॑ सलि॒लाय॑ त्वा । \newline
34. त्वा॒ सर्णी॑काय॒ सर्णी॑काय त्वा त्वा॒ सर्णी॑काय । \newline
35. सर्णी॑काय त्वा त्वा॒ सर्णी॑काय॒ सर्णी॑काय त्वा । \newline
36. त्वा॒ सती॑काय॒ सती॑काय त्वा त्वा॒ सती॑काय । \newline
37. सती॑काय त्वा त्वा॒ सती॑काय॒ सती॑काय त्वा । \newline
38. सती॑का॒येति॒ स - ती॒का॒य॒ । \newline
39. त्वा॒ केता॑य॒ केता॑य त्वा त्वा॒ केता॑य । \newline
40. केता॑य त्वा त्वा॒ केता॑य॒ केता॑य त्वा । \newline
41. त्वा॒ प्रचे॑तसे॒ प्रचे॑तसे त्वा त्वा॒ प्रचे॑तसे । \newline
42. प्रचे॑तसे त्वा त्वा॒ प्रचे॑तसे॒ प्रचे॑तसे त्वा । \newline
43. प्रचे॑तस॒ इति॒ प्र - चे॒त॒से॒ । \newline
44. त्वा॒ विव॑स्वते॒ विव॑स्वते त्वा त्वा॒ विव॑स्वते । \newline
45. विव॑स्वते त्वा त्वा॒ विव॑स्वते॒ विव॑स्वते त्वा । \newline
46. त्वा॒ दि॒वो दि॒व स्त्वा᳚ त्वा दि॒वः । \newline
47. दि॒व स्त्वा᳚ त्वा दि॒वो दि॒व स्त्वा᳚ । \newline
48. त्वा॒ ज्योति॑षे॒ ज्योति॑षे त्वा त्वा॒ ज्योति॑षे । \newline
49. ज्योति॑ष आदि॒त्येभ्य॑ आदि॒त्येभ्यो॒ ज्योति॑षे॒ ज्योति॑ष आदि॒त्येभ्यः॑ । \newline
50. आ॒दि॒त्येभ्य॑ स्त्वा त्वा ऽऽदि॒त्येभ्य॑ आदि॒त्येभ्य॑ स्त्वा । \newline
51. त्व॒ र्‌च ऋ॒चे त्वा᳚ त्व॒ र्‌चे । \newline
52. ऋ॒चे त्वा᳚ त्व॒ र्‌च ऋ॒चे त्वा᳚ । \newline
53. त्वा॒ रु॒चे रु॒चे त्वा᳚ त्वा रु॒चे । \newline
54. रु॒चे त्वा᳚ त्वा रु॒चे रु॒चे त्वा᳚ । \newline
55. त्वा॒ द्यु॒ते द्यु॒ते त्वा᳚ त्वा द्यु॒ते । \newline
56. द्यु॒ते त्वा᳚ त्वा द्यु॒ते द्यु॒ते त्वा᳚ । \newline
57. त्वा॒ भा॒से भा॒से त्वा᳚ त्वा भा॒से । \newline
58. भा॒से त्वा᳚ त्वा भा॒से भा॒से त्वा᳚ । \newline
59. त्वा॒ ज्योति॑षे॒ ज्योति॑षे त्वा त्वा॒ ज्योति॑षे । \newline
60. ज्योति॑षे त्वा त्वा॒ ज्योति॑षे॒ ज्योति॑षे त्वा । \newline
61. त्वा॒ य॒शो॒दां ॅय॑शो॒दाम् त्वा᳚ त्वा यशो॒दाम् । \newline
62. य॒शो॒दाम् त्वा᳚ त्वा यशो॒दां ॅय॑शो॒दाम् त्वा᳚ । \newline
63. य॒शो॒दामिति॑ यशः - दाम् । \newline
64. त्वा॒ यश॑सि॒ यश॑सि त्वा त्वा॒ यश॑सि । \newline
65. यश॑सि तेजो॒दाम् ते॑जो॒दां ॅयश॑सि॒ यश॑सि तेजो॒दाम् । \newline
66. ते॒जो॒दाम् त्वा᳚ त्वा तेजो॒दाम् ते॑जो॒दाम् त्वा᳚ । \newline
67. ते॒जो॒दामिति॑ तेजः - दाम् । \newline
68. त्वा॒ तेज॑सि॒ तेज॑सि त्वा त्वा॒ तेज॑सि । \newline
69. तेज॑सि पयो॒दाम् प॑यो॒दाम् तेज॑सि॒ तेज॑सि पयो॒दाम् । \newline
70. प॒यो॒दाम् त्वा᳚ त्वा पयो॒दाम् प॑यो॒दाम् त्वा᳚ । \newline
71. प॒यो॒दामिति॑ पयः - दाम् । \newline
72. त्वा॒ पय॑सि॒ पय॑सि त्वा त्वा॒ पय॑सि । \newline
73. पय॑सि वर्चो॒दां ॅव॑र्चो॒दाम् पय॑सि॒ पय॑सि वर्चो॒दाम् । \newline
74. व॒र्चो॒दाम् त्वा᳚ त्वा वर्चो॒दां ॅव॑र्चो॒दाम् त्वा᳚ । \newline
75. व॒र्चो॒दामिति॑ वर्चः - दाम् । \newline
76. त्वा॒ वर्च॑सि॒ वर्च॑सि त्वा त्वा॒ वर्च॑सि । \newline
77. वर्च॑सि द्रविणो॒दाम् द्र॑विणो॒दां ॅवर्च॑सि॒ वर्च॑सि द्रविणो॒दाम् । \newline
78. द्र॒वि॒णो॒दाम् त्वा᳚ त्वा द्रविणो॒दाम् द्र॑विणो॒दाम् त्वा᳚ । \newline
79. द्र॒वि॒णो॒दामिति॑ द्रविणः - दाम् । \newline
80. त्वा॒ द्रवि॑णे॒ द्रवि॑णे त्वा त्वा॒ द्रवि॑णे । \newline
81. द्रवि॑णे सादयामि सादयामि॒ द्रवि॑णे॒ द्रवि॑णे सादयामि । \newline
82. सा॒द॒या॒मि॒ तेन॒ तेन॑ सादयामि सादयामि॒ तेन॑ । \newline
83. तेन र्.षि॒ण र्.षि॑णा॒ तेन॒ तेन र्.षि॑णा । \newline
84. ऋषि॑णा॒ तेन॒ तेन र्.षि॒ण र्.षि॑णा॒ तेन॑ । \newline
85. तेन॒ ब्रह्म॑णा॒ ब्रह्म॑णा॒ तेन॒ तेन॒ ब्रह्म॑णा । \newline
86. ब्रह्म॑णा॒ तया॒ तया॒ ब्रह्म॑णा॒ ब्रह्म॑णा॒ तया᳚ । \newline
87. तया॑ दे॒वत॑या दे॒वत॑या॒ तया॒ तया॑ दे॒वत॑या । \newline
88. दे॒वत॑या ऽङ्गिर॒स्व द॑ङ्गिर॒स्वद् दे॒वत॑या दे॒वत॑या ऽङ्गिर॒स्वत् । \newline
89. अ॒ङ्गि॒र॒स्वद् ध्रु॒वा ध्रु॒वा ऽङ्गि॑र॒स्व द॑ङ्गिर॒स्वद् ध्रु॒वा । \newline
90. ध्रु॒वा सी॑द सीद ध्रु॒वा ध्रु॒वा सी॑द । \newline
91. सी॒देति॑ सीद । \newline

\textbf{Ghana Paata } \newline

1. अ॒सि॒ स्त॒न॒यि॒त्नु॒सनिः॑ स्तनयित्नु॒सनि॑ रस्यसि स्तनयित्नु॒सनि॑ रस्यसि स्तनयित्नु॒सनि॑ रस्यसि स्तनयित्नु॒सनि॑ रसि । \newline
2. स्त॒न॒यि॒त्नु॒सनि॑ रस्यसि स्तनयित्नु॒सनिः॑ स्तनयित्नु॒सनि॑ रसि वृष्टि॒सनि॑र् वृष्टि॒सनि॑ रसि स्तनयित्नु॒सनिः॑ स्तनयित्नु॒सनि॑ रसि वृष्टि॒सनिः॑ । \newline
3. स्त॒न॒यि॒त्नु॒सनि॒रिति॑ स्तनयित्नु - सनिः॑ । \newline
4. अ॒सि॒ वृ॒ष्टि॒सनि॑र् वृष्टि॒सनि॑ रस्यसि वृष्टि॒सनि॑ रस्यसि वृष्टि॒सनि॑ रस्यसि वृष्टि॒सनि॑ रसि । \newline
5. वृ॒ष्टि॒सनि॑ रस्यसि वृष्टि॒सनि॑र् वृष्टि॒सनि॑ रस्य॒ग्ने र॒ग्ने र॑सि वृष्टि॒सनि॑र् वृष्टि॒सनि॑ रस्य॒ग्नेः । \newline
6. वृ॒ष्टि॒सनि॒रिति॑ वृष्टि - सनिः॑ । \newline
7. अ॒स्य॒ग्ने र॒ग्ने र॑स्य स्य॒ग्नेर् यानी॒ यान्य॒ग्ने र॑स्य स्य॒ग्नेर् यानी᳚ । \newline
8. अ॒ग्नेर् यानी॒ यान्य॒ग्ने र॒ग्नेर् यान्य॑ स्यसि॒ यान्य॒ग्ने र॒ग्नेर् यान्य॑सि । \newline
9. यान्य॑ स्यसि॒ यानी॒ यान्य॑सि दे॒वाना᳚म् दे॒वाना॑ मसि॒ यानी॒ यान्य॑सि दे॒वाना᳚म् । \newline
10. अ॒सि॒ दे॒वाना᳚म् दे॒वाना॑ मस्यसि दे॒वाना॑ मग्ने॒ यान्य॑ ग्ने॒यानी॑ दे॒वाना॑ मस्यसि दे॒वाना॑ मग्ने॒यानी᳚ । \newline
11. दे॒वाना॑ मग्ने॒या न्य॑ग्ने॒यानी॑ दे॒वाना᳚म् दे॒वाना॑ मग्ने॒या न्य॑स्यस्य ग्ने॒यानी॑ दे॒वाना᳚म् दे॒वाना॑ मग्ने॒यान्य॑सि । \newline
12. अ॒ग्ने॒या न्य॑स्यस्य ग्ने॒या न्य॑ग्ने॒या न्य॑सि वा॒योर् वा॒यो र॑स्यग्ने॒या न्य॑ग्ने॒या न्य॑सि वा॒योः । \newline
13. अ॒ग्ने॒यानीत्य॑ग्ने - यानी᳚ । \newline
14. अ॒सि॒ वा॒योर् वा॒योर॑ स्यसि वा॒योर् यानी॒ यानी॑ वा॒यो र॑स्यसि वा॒योर् यानी᳚ । \newline
15. वा॒योर् यानी॒ यानी॑ वा॒योर् वा॒योर् यान्य॑ स्यसि॒ यानी॑ वा॒योर् वा॒योर् यान्य॑सि । \newline
16. यान्य॑ स्यसि॒ यानी॒ यान्य॑सि दे॒वाना᳚म् दे॒वाना॑ मसि॒ यानी॒ यान्य॑सि दे॒वाना᳚म् । \newline
17. अ॒सि॒ दे॒वाना᳚म् दे॒वाना॑ मस्यसि दे॒वानां᳚ ॅवायो॒यानी॑ वायो॒यानी॑ दे॒वाना॑ मस्यसि दे॒वानां᳚ ॅवायो॒यानी᳚ । \newline
18. दे॒वानां᳚ ॅवायो॒यानी॑ वायो॒यानी॑ दे॒वाना᳚म् दे॒वानां᳚ ॅवायो॒या न्य॑स्यसि वायो॒यानी॑ दे॒वाना᳚म् दे॒वानां᳚ ॅवायो॒यान् य॑सि । \newline
19. वा॒यो॒यान्य॑ स्यसि वायो॒यानी॑ वायो॒यान्य॑ स्य॒न्तरि॑क्षस्या॒ न्तरि॑क्षस्यासि वायो॒यानी॑ वायो॒या
न्य॑स्य॒न्तरि॑क्षस्य । \newline
20. वा॒यो॒यानीति॑ वायो - यानी᳚ । \newline
21. अ॒स्य॒ न्तरि॑क्षस्या॒ न्तरि॑क्षस्या स्यस्य॒ न्तरि॑क्षस्य॒ यानी॒ यान्य॒ न्तरि॑क्षस्या स्यस्य॒ न्तरि॑क्षस्य॒ यानी᳚ । \newline
22. अ॒न्तरि॑क्षस्य॒ यानी॒ यान्य॒ न्तरि॑क्षस्या॒ न्तरि॑क्षस्य॒ यान्य॑ स्यसि॒ यान्य॒ न्तरि॑क्षस्या॒ न्तरि॑क्षस्य॒ यान्य॑सि । \newline
23. यान्य॑स्यसि॒ यानी॒ यान्य॑सि दे॒वाना᳚म् दे॒वाना॑ मसि॒ यानी॒ यान्य॑सि दे॒वाना᳚म् । \newline
24. अ॒सि॒ दे॒वाना᳚म् दे॒वाना॑ मस्यसि दे॒वाना॑ मन्तरिक्ष॒या न्य॑न्तरिक्ष॒यानी॑ दे॒वाना॑ मस्यसि दे॒वाना॑ मन्तरिक्ष॒यानी᳚ । \newline
25. दे॒वाना॑ मन्तरिक्ष॒या न्य॑न्तरिक्ष॒यानी॑ दे॒वाना᳚म् दे॒वाना॑ मन्तरिक्ष॒या न्य॑स्यस्य न्तरिक्ष॒यानी॑ दे॒वाना᳚म् दे॒वाना॑ मन्तरिक्ष॒यान्य॑सि । \newline
26. अ॒न्त॒रि॒क्ष॒या न्य॑स्यस्य न्तरिक्ष॒यान्य॑ न्तरिक्ष॒या न्य॑स्य॒न्तरि॑क्ष म॒न्तरि॑क्ष मस्यन्तरिक्ष॒या
न्य॑न्तरिक्ष॒यान्य॑ स्य॒न्तरि॑क्षम् । \newline
27. अ॒न्त॒रि॒क्ष॒यानीत्य॑न्तरिक्ष - यानी᳚ । \newline
28. अ॒स्य॒ न्तरि॑क्ष म॒न्तरि॑क्ष मस्यस्य॒ न्तरि॑क्ष मस्यस्य॒ न्तरि॑क्ष मस्यस्य॒ न्तरि॑क्ष मसि । \newline
29. अ॒न्तरि॑क्ष मस्यस्य॒ न्तरि॑क्ष म॒न्तरि॑क्ष मस्य॒ न्तरि॑क्षाया॒ न्तरि॑क्षाया स्य॒न्तरि॑क्ष म॒न्तरि॑क्ष मस्य॒ न्तरि॑क्षाय । \newline
30. अ॒स्य॒ न्तरि॑क्षाया॒ न्तरि॑क्षाया स्यस्य॒ न्तरि॑क्षाय त्वा त्वा॒ ऽन्तरि॑क्षाया स्यस्य॒ न्तरि॑क्षाय त्वा । \newline
31. अ॒न्तरि॑क्षाय त्वा त्वा॒ ऽन्तरि॑क्षाया॒ न्तरि॑क्षाय त्वा सलि॒लाय॑ सलि॒लाय॑ त्वा॒ ऽन्तरि॑क्षाया॒ न्तरि॑क्षाय त्वा सलि॒लाय॑ । \newline
32. त्वा॒ स॒लि॒लाय॑ सलि॒लाय॑ त्वा त्वा सलि॒लाय॑ त्वा त्वा सलि॒लाय॑ त्वा त्वा सलि॒लाय॑ त्वा । \newline
33. स॒लि॒लाय॑ त्वा त्वा सलि॒लाय॑ सलि॒लाय॑ त्वा॒ सर्णी॑काय॒ सर्णी॑काय त्वा सलि॒लाय॑ सलि॒लाय॑ त्वा॒ सर्णी॑काय । \newline
34. त्वा॒ सर्णी॑काय॒ सर्णी॑काय त्वा त्वा॒ सर्णी॑काय त्वा त्वा॒ सर्णी॑काय त्वा त्वा॒ सर्णी॑काय त्वा । \newline
35. सर्णी॑काय त्वा त्वा॒ सर्णी॑काय॒ सर्णी॑काय त्वा॒ सती॑काय॒ सती॑काय त्वा॒ सर्णी॑काय॒ सर्णी॑काय त्वा॒ सती॑काय । \newline
36. त्वा॒ सती॑काय॒ सती॑काय त्वा त्वा॒ सती॑काय त्वा त्वा॒ सती॑काय त्वा त्वा॒ सती॑काय त्वा । \newline
37. सती॑काय त्वा त्वा॒ सती॑काय॒ सती॑काय त्वा॒ केता॑य॒ केता॑य त्वा॒ सती॑काय॒ सती॑काय त्वा॒ केता॑य । \newline
38. सती॑का॒येति॒ स - ती॒का॒य॒ । \newline
39. त्वा॒ केता॑य॒ केता॑य त्वा त्वा॒ केता॑य त्वा त्वा॒ केता॑य त्वा त्वा॒ केता॑य त्वा । \newline
40. केता॑य त्वा त्वा॒ केता॑य॒ केता॑य त्वा॒ प्रचे॑तसे॒ प्रचे॑तसे त्वा॒ केता॑य॒ केता॑य त्वा॒ प्रचे॑तसे । \newline
41. त्वा॒ प्रचे॑तसे॒ प्रचे॑तसे त्वा त्वा॒ प्रचे॑तसे त्वा त्वा॒ प्रचे॑तसे त्वा त्वा॒ प्रचे॑तसे त्वा । \newline
42. प्रचे॑तसे त्वा त्वा॒ प्रचे॑तसे॒ प्रचे॑तसे त्वा॒ विव॑स्वते॒ विव॑स्वते त्वा॒ प्रचे॑तसे॒ प्रचे॑तसे त्वा॒ विव॑स्वते । \newline
43. प्रचे॑तस॒ इति॒ प्र - चे॒त॒से॒ । \newline
44. त्वा॒ विव॑स्वते॒ विव॑स्वते त्वा त्वा॒ विव॑स्वते त्वा त्वा॒ विव॑स्वते त्वा त्वा॒ विव॑स्वते त्वा । \newline
45. विव॑स्वते त्वा त्वा॒ विव॑स्वते॒ विव॑स्वते त्वा दि॒वो दि॒व स्त्वा॒ विव॑स्वते॒ विव॑स्वते त्वा दि॒वः । \newline
46. त्वा॒ दि॒वो दि॒व स्त्वा᳚ त्वा दि॒व स्त्वा᳚ त्वा दि॒व स्त्वा᳚ त्वा दि॒व स्त्वा᳚ । \newline
47. दि॒व स्त्वा᳚ त्वा दि॒वो दि॒व स्त्वा॒ ज्योति॑षे॒ ज्योति॑षे त्वा दि॒वो दि॒व स्त्वा॒ ज्योति॑षे । \newline
48. त्वा॒ ज्योति॑षे॒ ज्योति॑षे त्वा त्वा॒ ज्योति॑ष आदि॒त्येभ्य॑ आदि॒त्येभ्यो॒ ज्योति॑षे त्वा त्वा॒ ज्योति॑ष आदि॒त्येभ्यः॑ । \newline
49. ज्योति॑ष आदि॒त्येभ्य॑ आदि॒त्येभ्यो॒ ज्योति॑षे॒ ज्योति॑ष आदि॒त्येभ्य॑ स्त्वा त्वा ऽऽदि॒त्येभ्यो॒ ज्योति॑षे॒ ज्योति॑ष आदि॒त्येभ्य॑ स्त्वा । \newline
50. आ॒दि॒त्येभ्य॑ स्त्वा त्वा ऽऽदि॒त्येभ्य॑ आदि॒त्येभ्य॑ स्त्व॒ र्‌च ऋ॒चे त्वा॑ ऽऽदि॒त्येभ्य॑ आदि॒त्येभ्य॑ स्त्व॒ र्‌चे । \newline
51. त्व॒ र्‌च ऋ॒चे त्वा᳚ त्व॒ र्‌चे त्वा᳚ त्व॒ र्‌चे त्वा᳚ त्व॒ र्‌चे त्वा᳚ । \newline
52. ऋ॒चे त्वा᳚ त्व॒ र्‌च ऋ॒चे त्वा॑ रु॒चे रु॒चे त्व॒ र्‌च ऋ॒चे त्वा॑ रु॒चे । \newline
53. त्वा॒ रु॒चे रु॒चे त्वा᳚ त्वा रु॒चे त्वा᳚ त्वा रु॒चे त्वा᳚ त्वा रु॒चे त्वा᳚ । \newline
54. रु॒चे त्वा᳚ त्वा रु॒चे रु॒चे त्वा᳚ द्यु॒ते द्यु॒ते त्वा॑ रु॒चे रु॒चे त्वा᳚ द्यु॒ते । \newline
55. त्वा॒ द्यु॒ते द्यु॒ते त्वा᳚ त्वा द्यु॒ते त्वा᳚ त्वा द्यु॒ते त्वा᳚ त्वा द्यु॒ते त्वा᳚ । \newline
56. द्यु॒ते त्वा᳚ त्वा द्यु॒ते द्यु॒ते त्वा॑ भा॒से भा॒से त्वा᳚ द्यु॒ते द्यु॒ते त्वा॑ भा॒से । \newline
57. त्वा॒ भा॒से भा॒से त्वा᳚ त्वा भा॒से त्वा᳚ त्वा भा॒से त्वा᳚ त्वा भा॒से त्वा᳚ । \newline
58. भा॒से त्वा᳚ त्वा भा॒से भा॒से त्वा॒ ज्योति॑षे॒ ज्योति॑षे त्वा भा॒से भा॒से त्वा॒ ज्योति॑षे । \newline
59. त्वा॒ ज्योति॑षे॒ ज्योति॑षे त्वा त्वा॒ ज्योति॑षे त्वा त्वा॒ ज्योति॑षे त्वा त्वा॒ ज्योति॑षे त्वा । \newline
60. ज्योति॑षे त्वा त्वा॒ ज्योति॑षे॒ ज्योति॑षे त्वा यशो॒दां ॅय॑शो॒दाम् त्वा॒ ज्योति॑षे॒ ज्योति॑षे त्वा यशो॒दाम् । \newline
61. त्वा॒ य॒शो॒दां ॅय॑शो॒दाम् त्वा᳚ त्वा यशो॒दाम् त्वा᳚ त्वा यशो॒दाम् त्वा᳚ त्वा यशो॒दाम् त्वा᳚ । \newline
62. य॒शो॒दाम् त्वा᳚ त्वा यशो॒दां ॅय॑शो॒दाम् त्वा॒ यश॑सि॒ यश॑सि त्वा यशो॒दां ॅय॑शो॒दाम् त्वा॒ यश॑सि । \newline
63. य॒शो॒दामिति॑ यशः - दाम् । \newline
64. त्वा॒ यश॑सि॒ यश॑सि त्वा त्वा॒ यश॑सि तेजो॒दाम् ते॑जो॒दां ॅयश॑सि त्वा त्वा॒ यश॑सि तेजो॒दाम् । \newline
65. यश॑सि तेजो॒दाम् ते॑जो॒दां ॅयश॑सि॒ यश॑सि तेजो॒दाम् त्वा᳚ त्वा तेजो॒दां ॅयश॑सि॒ यश॑सि तेजो॒दाम् त्वा᳚ । \newline
66. ते॒जो॒दाम् त्वा᳚ त्वा तेजो॒दाम् ते॑जो॒दाम् त्वा॒ तेज॑सि॒ तेज॑सि त्वा तेजो॒दाम् ते॑जो॒दाम् त्वा॒ तेज॑सि । \newline
67. ते॒जो॒दामिति॑ तेजः - दाम् । \newline
68. त्वा॒ तेज॑सि॒ तेज॑सि त्वा त्वा॒ तेज॑सि पयो॒दाम् प॑यो॒दाम् तेज॑सि त्वा त्वा॒ तेज॑सि पयो॒दाम् । \newline
69. तेज॑सि पयो॒दाम् प॑यो॒दाम् तेज॑सि॒ तेज॑सि पयो॒दाम् त्वा᳚ त्वा पयो॒दाम् तेज॑सि॒ तेज॑सि पयो॒दाम् त्वा᳚ । \newline
70. प॒यो॒दाम् त्वा᳚ त्वा पयो॒दाम् प॑यो॒दाम् त्वा॒ पय॑सि॒ पय॑सि त्वा पयो॒दाम् प॑यो॒दाम् त्वा॒ पय॑सि । \newline
71. प॒यो॒दामिति॑ पयः - दाम् । \newline
72. त्वा॒ पय॑सि॒ पय॑सि त्वा त्वा॒ पय॑सि वर्चो॒दां ॅव॑र्चो॒दाम् पय॑सि त्वा त्वा॒ पय॑सि वर्चो॒दाम् । \newline
73. पय॑सि वर्चो॒दां ॅव॑र्चो॒दाम् पय॑सि॒ पय॑सि वर्चो॒दाम् त्वा᳚ त्वा वर्चो॒दाम् पय॑सि॒ पय॑सि वर्चो॒दाम् त्वा᳚ । \newline
74. व॒र्चो॒दाम् त्वा᳚ त्वा वर्चो॒दां ॅव॑र्चो॒दाम् त्वा॒ वर्च॑सि॒ वर्च॑सि त्वा वर्चो॒दां ॅव॑र्चो॒दाम् त्वा॒ वर्च॑सि । \newline
75. व॒र्चो॒दामिति॑ वर्चः - दाम् । \newline
76. त्वा॒ वर्च॑सि॒ वर्च॑सि त्वा त्वा॒ वर्च॑सि द्रविणो॒दाम् द्र॑विणो॒दां ॅवर्च॑सि त्वा त्वा॒ वर्च॑सि द्रविणो॒दाम् । \newline
77. वर्च॑सि द्रविणो॒दाम् द्र॑विणो॒दां ॅवर्च॑सि॒ वर्च॑सि द्रविणो॒दाम् त्वा᳚ त्वा द्रविणो॒दां ॅवर्च॑सि॒ वर्च॑सि द्रविणो॒दाम् त्वा᳚ । \newline
78. द्र॒वि॒णो॒दाम् त्वा᳚ त्वा द्रविणो॒दाम् द्र॑विणो॒दाम् त्वा॒ द्रवि॑णे॒ द्रवि॑णे त्वा द्रविणो॒दाम् द्र॑विणो॒दाम् त्वा॒ द्रवि॑णे । \newline
79. द्र॒वि॒णो॒दामिति॑ द्रविणः - दाम् । \newline
80. त्वा॒ द्रवि॑णे॒ द्रवि॑णे त्वा त्वा॒ द्रवि॑णे सादयामि सादयामि॒ द्रवि॑णे त्वा त्वा॒ द्रवि॑णे सादयामि । \newline
81. द्रवि॑णे सादयामि सादयामि॒ द्रवि॑णे॒ द्रवि॑णे सादयामि॒ तेन॒ तेन॑ सादयामि॒ द्रवि॑णे॒ द्रवि॑णे सादयामि॒ तेन॑ । \newline
82. सा॒द॒या॒मि॒ तेन॒ तेन॑ सादयामि सादयामि॒ तेन र्.षि॒ण र्.षि॑णा॒ तेन॑ सादयामि सादयामि॒ तेन र्.षि॑णा । \newline
83. तेन र्.षि॒ण र्.षि॑णा॒ तेन॒ तेन र्.षि॑णा॒ तेन॒ तेन र्.षि॑णा॒ तेन॒ तेन र्.षि॑णा॒ तेन॑ । \newline
84. ऋषि॑णा॒ तेन॒ तेन र्.षि॒ण र्.षि॑णा॒ तेन॒ ब्रह्म॑णा॒ ब्रह्म॑णा॒ तेन र्.षि॒ण र्.षि॑णा॒ तेन॒ ब्रह्म॑णा । \newline
85. तेन॒ ब्रह्म॑णा॒ ब्रह्म॑णा॒ तेन॒ तेन॒ ब्रह्म॑णा॒ तया॒ तया॒ ब्रह्म॑णा॒ तेन॒ तेन॒ ब्रह्म॑णा॒ तया᳚ । \newline
86. ब्रह्म॑णा॒ तया॒ तया॒ ब्रह्म॑णा॒ ब्रह्म॑णा॒ तया॑ दे॒वत॑या दे॒वत॑या॒ तया॒ ब्रह्म॑णा॒ ब्रह्म॑णा॒ तया॑ दे॒वत॑या । \newline
87. तया॑ दे॒वत॑या दे॒वत॑या॒ तया॒ तया॑ दे॒वत॑या ऽङ्गिर॒स्व द॑ङ्गिर॒स्वद् दे॒वत॑या॒ तया॒ तया॑ दे॒वत॑या ऽङ्गिर॒स्वत् । \newline
88. दे॒वत॑या ऽङ्गिर॒स्व द॑ङ्गिर॒स्वद् दे॒वत॑या दे॒वत॑या ऽङ्गिर॒स्वद् ध्रु॒वा ध्रु॒वा ऽङ्गि॑र॒स्वद् दे॒वत॑या दे॒वत॑या ऽङ्गिर॒स्वद् ध्रु॒वा । \newline
89. अ॒ङ्गि॒र॒स्वद् ध्रु॒वा ध्रु॒वा ऽङ्गि॑र॒स्व द॑ङ्गिर॒स्वद् ध्रु॒वा सी॑द सीद ध्रु॒वा ऽङ्गि॑र॒स्व द॑ङ्गिर॒स्वद् ध्रु॒वा सी॑द । \newline
90. ध्रु॒वा सी॑द सीद ध्रु॒वा ध्रु॒वा सी॑द । \newline
91. सी॒देति॑ सीद । \newline
\pagebreak
\markright{ TS 4.4.7.1  \hfill https://www.vedavms.in \hfill}

\section{ TS 4.4.7.1 }

\textbf{TS 4.4.7.1 } \newline
\textbf{Samhita Paata} \newline

भू॒य॒स्कृद॑सि वरिव॒स्कृद॑सि॒ प्राच्य॑स्यू॒र्द्ध्वाऽस्य॑-न्तरिक्ष॒सद॑स्य॒-न्तरि॑क्षे सीदा-फ्सु॒षद॑सि श्येन॒सद॑सि गृद्ध्र॒सद॑सि सुपर्ण॒सद॑सि नाक॒सद॑सि पृथि॒व्यास्त्वा॒ द्रवि॑णे सादयाम्य॒-न्तरि॑क्षस्य त्वा॒ द्रवि॑णे सादयामि दि॒वस्त्वा॒ द्रवि॑णे सादयामि दि॒शां त्वा॒ द्रवि॑णे सादयामि द्रविणो॒दां त्वा॒ द्रवि॑णे सादयामि प्रा॒णं मे॑ पाह्य-पा॒नं मे॑ पाहि व्या॒नं मे॑ - [  ] \newline

\textbf{Pada Paata} \newline

भू॒य॒स्कृदिति॑ भूयः - कृत् । अ॒सि॒ । व॒रि॒व॒स्कृदिति॑ वरिवः - कृत् । अ॒सि॒ । प्राची᳚ । अ॒सि॒ । ऊ॒द्‌र्ध्वा । अ॒सि॒ । अ॒न्त॒रि॒क्ष॒सदित्य॑न्तरिक्ष - सत् । अ॒सि॒ । अ॒न्तरि॑क्षे । सी॒द॒ । अ॒फ्सु॒षदित्य॑फ्सु - सत् । अ॒सि॒ । श्ये॒न॒सदिति॑ श्येन-सत् । अ॒सि॒ । गृ॒द्ध्र॒सदिति॑ गृद्ध्र-सत् । अ॒सि॒ । सु॒प॒र्ण॒सदिति॑ सुपर्ण - सत् । अ॒सि॒ । ना॒क॒सदिति॑ नाक - सत् । अ॒सि॒ । पृ॒थि॒व्याः । त्वा॒ । द्रवि॑णे । सा॒द॒या॒मि॒ । अ॒न्तरि॑क्षस्य । त्वा॒ । द्रवि॑णे । सा॒द॒या॒मि॒ । दि॒वः । त्वा॒ । द्रवि॑णे । सा॒द॒या॒मि॒ । दि॒शाम् । त्वा॒ । द्रवि॑णे । सा॒द॒या॒मि॒ । द्र॒वि॒णो॒दामिति॑ द्रविणः - दाम् । त्वा॒ । द्रवि॑णे । सा॒द॒या॒मि॒ । प्रा॒णमिति॑ प्र - अ॒नम् । मे॒ । पा॒हि॒ । आ॒पा॒नमित्य॑प - अ॒नम् । मे॒ । पा॒हि॒ । व्या॒नमिति॑ वि - अ॒नम् । मे॒ ।  \newline


\textbf{Krama Paata} \newline

भू॒य॒स्कृद॑सि । भू॒य॒स्कृदिति॑ भूयः - कृत् । अ॒सि॒ व॒रि॒व॒स्कृत् । व॒रि॒व॒स्कृद॑सि । व॒रि॒व॒स्कृदिति॑ वरिवः - कृत् । अ॒सि॒ प्राची᳚ । प्राच्य॑सि । अ॒स्यू॒र्द्ध्वा । ऊ॒र्द्ध्वाऽसि॑ । अ॒स्य॒न्त॒रि॒क्ष॒सत् । अ॒न्त॒रि॒क्ष॒सद॑सि । अ॒न्त॒रि॒क्ष॒सदित्य॑न्तरिक्ष - सत् । अ॒स्य॒न्तरि॑क्षे । अ॒न्तरि॑क्षे सीद । सी॒दा॒फ्सु॒षत् । अ॒फ्सु॒षद॑सि । अ॒फ्सु॒षदित्य॑फ्सु - सत् । अ॒सि॒ श्ये॒न॒सत् । श्ये॒न॒सद॑सि । श्ये॒न॒सदिति॑ श्येन - सत् । अ॒सि॒ गृ॒द्ध्र॒सत् । गृ॒द्ध्र॒सद॑सि । गृ॒द्ध्र॒सदिति॑ गृद्ध्र - सत् । अ॒सि॒ सु॒प॒र्ण॒सत् । सु॒प॒र्ण॒सद॑सि । सु॒प॒र्ण॒सदिति॑ सुपर्ण - सत् । अ॒सि॒ ना॒क॒सत् । ना॒क॒सद॑सि । ना॒क॒सदिति॑ नाक - सत् । अ॒सि॒ पृ॒थि॒व्याः । पृ॒थि॒व्यास्त्वा᳚ । त्वा॒ द्रवि॑णे । द्रवि॑णे सादयामि । सा॒द॒या॒म्य॒न्तरिक्ष॑स्य । अ॒न्तरि॑क्षस्य त्वा । त्वा॒ द्रवि॑णे । द्रवि॑णे सादयामि । सा॒द॒या॒मि॒ दि॒वः । दि॒वस्त्वा᳚ । त्वा॒ द्रवि॑णे । द्रवि॑णे सादयामि । सा॒द॒या॒मि॒ दि॒शाम् । दि॒शाम् त्वा᳚ । त्वा॒ द्रवि॑णे । द्रवि॑णे सादयामि । सा॒द॒या॒मि॒ द्र॒वि॒णो॒दाम् । द्र॒वि॒णो॒दाम् त्वा᳚ । द्र॒वि॒णो॒दामिति॑ द्रविणः - दाम् । त्वा॒ द्रवि॑णे । द्रवि॑णे सादयामि । सा॒द॒या॒मि॒ प्रा॒णम् । प्रा॒णम् मे᳚ । प्रा॒णमिति॑ प्र - अ॒नम् । मे॒ पा॒हि॒ । पा॒ह्य॒पा॒नम् । अ॒पा॒नम् मे᳚ । अ॒पा॒नमित्य॑प - अ॒नम् । मे॒ पा॒हि॒ । पा॒हि॒ व्या॒नम् । व्या॒नम् मे᳚ ( ) । व्या॒नमिति॑ वि - अ॒नम् । मे॒ पा॒हि॒ \newline

\textbf{Jatai Paata} \newline

1. भु॒य॒स्कृ द॑स्यसि भूय॒स्कृद् भू॑य॒स्कृ द॑सि । \newline
2. भू॒य॒स्कृदिति॑ भूयः - कृत् । \newline
3. अ॒सि॒ व॒रि॒व॒स्कृद् व॑रिव॒स्कृ द॑स्यसि वरिव॒स्कृत् । \newline
4. व॒रि॒व॒स्कृ द॑स्यसि वरिव॒स्कृद् व॑रिव॒स्कृ द॑सि । \newline
5. व॒रि॒व॒स्कृदिति॑ वरिवः - कृत् । \newline
6. अ॒सि॒ प्राची॒ प्राच्य॑ स्यसि॒ प्राची᳚ । \newline
7. प्राच्य॑ स्यसि॒ प्राची॒ प्राच्य॑सि । \newline
8. अ॒स्यू॒र्द्ध्वोर्द्ध्वा ऽस्य॑ स्यू॒र्द्ध्वा । \newline
9. ऊ॒र्द्ध्वा ऽस्य॑ स्यू॒र्द्ध्वोर्द्ध्वा ऽसि॑ । \newline
10. अ॒स्य॒ न्त॒रि॒क्ष॒स द॑न्तरिक्ष॒स द॑स्यस्य न्तरिक्ष॒सत् । \newline
11. अ॒न्त॒रि॒क्ष॒स द॑स्यस्य न्तरिक्ष॒स द॑न्तरिक्ष॒स द॑सि । \newline
12. अ॒न्त॒रि॒क्ष॒सदित्य॑न्तरिक्ष - सत् । \newline
13. अ॒स्य॒ न्तरि॑क्षे॒ ऽन्तरि॑क्षे ऽस्यस्य॒ न्तरि॑क्षे । \newline
14. अ॒न्तरि॑क्षे सीद सीदा॒ न्तरि॑क्षे॒ ऽन्तरि॑क्षे सीद । \newline
15. सी॒दा॒ फ्सु॒षद॑ फ्सु॒षथ् सी॑द सीदाफ्सु॒षत् । \newline
16. अ॒फ्सु॒ष द॑स्यस्य फ्सु॒ष द॑फ्सु॒ष द॑सि । \newline
17. अ॒फ्सु॒षदित्य॑फ्सु - सत् । \newline
18. अ॒सि॒ श्ये॒न॒स च्छ्ये॑न॒स द॑स्यसि श्येन॒सत् । \newline
19. श्ये॒न॒स द॑स्यसि श्येन॒स च्छ्ये॑न॒स द॑सि । \newline
20. श्ये॒न॒सदिति॑ श्येन - सत् । \newline
21. अ॒सि॒ गृ॒द्ध्र॒सद् गृ॑द्ध्र॒स द॑स्यसि गृद्ध्र॒सत् । \newline
22. गृ॒द्ध्र॒स द॑स्यसि गृद्ध्र॒सद् गृ॑द्ध्र॒स द॑सि । \newline
23. गृ॒द्ध्र॒सदिति॑ गृद्ध्र - सत् । \newline
24. अ॒सि॒ सु॒प॒र्ण॒सथ् सु॑पर्ण॒स द॑स्यसि सुपर्ण॒सत् । \newline
25. सु॒प॒र्ण॒स द॑स्यसि सुपर्ण॒सथ् सु॑पर्ण॒स द॑सि । \newline
26. सु॒प॒र्ण॒सदिति॑ सुपर्ण - सत् । \newline
27. अ॒सि॒ ना॒क॒सन् ना॑क॒स द॑स्यसि नाक॒सत् । \newline
28. ना॒क॒स द॑स्यसि नाक॒सन् ना॑क॒स द॑सि । \newline
29. ना॒क॒सदिति॑ नाक - सत् । \newline
30. अ॒सि॒ पृ॒थि॒व्याः पृ॑थि॒व्या अ॑स्यसि पृथि॒व्याः । \newline
31. पृ॒थि॒व्या स्त्वा᳚ त्वा पृथि॒व्याः पृ॑थि॒व्या स्त्वा᳚ । \newline
32. त्वा॒ द्रवि॑णे॒ द्रवि॑णे त्वा त्वा॒ द्रवि॑णे । \newline
33. द्रवि॑णे सादयामि सादयामि॒ द्रवि॑णे॒ द्रवि॑णे सादयामि । \newline
34. सा॒द॒या॒म्य॒ न्तरि॑क्षस्या॒ न्तरि॑क्षस्य सादयामि सादयाम्य॒ न्तरि॑क्षस्य । \newline
35. अ॒न्तरि॑क्षस्य त्वा त्वा॒ ऽन्तरि॑क्षस्या॒ न्तरि॑क्षस्य त्वा । \newline
36. त्वा॒ द्रवि॑णे॒ द्रवि॑णे त्वा त्वा॒ द्रवि॑णे । \newline
37. द्रवि॑णे सादयामि सादयामि॒ द्रवि॑णे॒ द्रवि॑णे सादयामि । \newline
38. सा॒द॒या॒मि॒ दि॒वो दि॒वः सा॑दयामि सादयामि दि॒वः । \newline
39. दि॒व स्त्वा᳚ त्वा दि॒वो दि॒व स्त्वा᳚ । \newline
40. त्वा॒ द्रवि॑णे॒ द्रवि॑णे त्वा त्वा॒ द्रवि॑णे । \newline
41. द्रवि॑णे सादयामि सादयामि॒ द्रवि॑णे॒ द्रवि॑णे सादयामि । \newline
42. सा॒द॒या॒मि॒ दि॒शाम् दि॒शाꣳ सा॑दयामि सादयामि दि॒शाम् । \newline
43. दि॒शाम् त्वा᳚ त्वा दि॒शाम् दि॒शाम् त्वा᳚ । \newline
44. त्वा॒ द्रवि॑णे॒ द्रवि॑णे त्वा त्वा॒ द्रवि॑णे । \newline
45. द्रवि॑णे सादयामि सादयामि॒ द्रवि॑णे॒ द्रवि॑णे सादयामि । \newline
46. सा॒द॒या॒मि॒ द्र॒वि॒णो॒दाम् द्र॑विणो॒दाꣳ सा॑दयामि सादयामि द्रविणो॒दाम् । \newline
47. द्र॒वि॒णो॒दाम् त्वा᳚ त्वा द्रविणो॒दाम् द्र॑विणो॒दाम् त्वा᳚ । \newline
48. द्र॒वि॒णो॒दामिति॑ द्रविणः - दाम् । \newline
49. त्वा॒ द्रवि॑णे॒ द्रवि॑णे त्वा त्वा॒ द्रवि॑णे । \newline
50. द्रवि॑णे सादयामि सादयामि॒ द्रवि॑णे॒ द्रवि॑णे सादयामि । \newline
51. सा॒द॒या॒मि॒ प्रा॒णम् प्रा॒णꣳ सा॑दयामि सादयामि प्रा॒णम् । \newline
52. प्रा॒णम् मे॑ मे प्रा॒णम् प्रा॒णम् मे᳚ । \newline
53. प्रा॒णमिति॑ प्र - अ॒नम् । \newline
54. मे॒ पा॒हि॒ पा॒हि॒ मे॒ मे॒ पा॒हि॒ । \newline
55. पा॒ह्य॒ पा॒न म॑पा॒नम् पा॑हि पाह्य पा॒नम् । \newline
56. अ॒पा॒नम् मे॑ मे अपा॒न म॑पा॒नम् मे᳚ । \newline
57. अ॒पा॒नमित्य॑प - अ॒नम् । \newline
58. मे॒ पा॒हि॒ पा॒हि॒ मे॒ मे॒ पा॒हि॒ । \newline
59. पा॒हि॒ व्या॒नं ॅव्या॒नम् पा॑हि पाहि व्या॒नम् । \newline
60. व्या॒नम् मे॑ मे व्या॒नं ॅव्या॒नम् मे᳚ । \newline
61. व्या॒नमिति॑ वि - अ॒नम् । \newline
62. मे॒ पा॒हि॒ पा॒हि॒ मे॒ मे॒ पा॒हि॒ । \newline

\textbf{Ghana Paata } \newline

1. भु॒य॒स्कृ द॑स्यसि भूय॒स्कृद् भू॑य॒स्कृ द॑सि वरिव॒स्कृद् व॑रिव॒स्कृ द॑सि भूय॒स्कृद् भू॑य॒स्कृ द॑सि वरिव॒स्कृत् । \newline
2. भू॒य॒स्कृदिति॑ भूयः - कृत् । \newline
3. अ॒सि॒ व॒रि॒व॒स्कृद् व॑रिव॒स्कृ द॑स्यसि वरिव॒स्कृ द॑स्यसि वरिव॒स्कृ द॑स्यसि वरिव॒स्कृ द॑सि । \newline
4. व॒रि॒व॒स्कृ द॑स्यसि वरिव॒स्कृद् व॑रिव॒स्कृ द॑सि॒ प्राची॒ प्राच्य॑सि वरिव॒स्कृद् व॑रिव॒स्कृ द॑सि॒ प्राची᳚ । \newline
5. व॒रि॒व॒स्कृदिति॑ वरिवः - कृत् । \newline
6. अ॒सि॒ प्राची॒ प्राच्य॑ स्यसि॒ प्राच्य॑ स्यसि॒ प्राच्य॑ स्यसि॒ प्राच्य॑सि । \newline
7. प्राच्य॑ स्यसि॒ प्राची॒ प्राच्य॑ स्यू॒र्द्ध्वो र्द्ध्वा ऽसि॒ प्राची॒ प्राच्य॑ स्यू॒र्द्ध्वा । \newline
8. अ॒स्यू॒र्द्ध्वो र्द्ध्वा ऽस्य॑ स्यू॒र्द्ध्वा ऽस्य॑ स्यू॒र्द्ध्वा ऽस्य॑ स्यू॒र्द्ध्वा ऽसि॑ । \newline
9. ऊ॒र्द्ध्वा ऽस्य॑ स्यू॒र्द्ध्वो र्द्ध्वा ऽस्य॑न्तरिक्ष॒स द॑न्तरिक्ष॒स द॑स्यू॒र्द्ध्वो र्द्ध्वा ऽस्य॑न्तरिक्ष॒सत् । \newline
10. अ॒स्य॒न्त॒रि॒क्ष॒स द॑न्तरिक्ष॒स द॑स्यस्य न्तरिक्ष॒स द॑स्यस्य न्तरिक्ष॒स द॑स्यस्य न्तरिक्ष॒सद॑सि । \newline
11. अ॒न्त॒रि॒क्ष॒सद॑ स्यस्य न्तरिक्ष॒स द॑न्तरिक्ष॒स द॑स्य॒ न्तरि॑क्षे॒ ऽन्तरि॑क्षे ऽस्यन्तरिक्ष॒स द॑न्तरिक्ष॒स द॑स्य॒ न्तरि॑क्षे । \newline
12. अ॒न्त॒रि॒क्ष॒सदित्य॑न्तरिक्ष - सत् । \newline
13. अ॒स्य॒ न्तरि॑क्षे॒ ऽन्तरि॑क्षे ऽस्यस्य॒ न्तरि॑क्षे सीद सीदा॒ न्तरि॑क्षे ऽस्यस्य॒ न्तरि॑क्षे सीद । \newline
14. अ॒न्तरि॑क्षे सीद सीदा॒न्तरि॑क्षे॒ ऽन्तरि॑क्षे सीदा फ्सु॒ष द॑फ्सु॒षथ् सी॑दा॒ न्तरि॑क्षे॒ ऽन्तरि॑क्षे सीदाफ्सु॒षत् । \newline
15. सी॒दा॒फ्सु॒ष द॑फ्सु॒षथ् सी॑द सीदाफ्सु॒ष द॑स्यस्य फ्सु॒षथ् सी॑द सीदाफ्सु॒ष द॑सि । \newline
16. अ॒फ्सु॒षद् अ॑स्यस्य फ्सु॒ष द॑फ्सु॒ष द॑सि श्येन॒ सच्छ्ये॑न॒स द॑स्यफ्सु॒ष द॑फ्सु॒षद॑सि श्येन॒सत् । \newline
17. अ॒फ्सु॒षदित्य॑फ्सु - सत् । \newline
18. अ॒सि॒ श्ये॒न॒ सच्छ्ये॑न॒ सद॑स्यसि श्येन॒ सद॑स्यसि श्येन॒ सद॑स्यसि श्येन॒ सद॑सि । \newline
19. श्ये॒न॒सद॑स्यसि श्येन॒स च्छ्ये॑न॒ सद॑सि गृद्ध्र॒सद् गृ॑द्ध्र॒सद॑सि श्येन॒स च्छ्ये॑न॒सद॑सि गृद्ध्र॒सत् । \newline
20. श्ये॒न॒सदिति॑ श्येन - सत् । \newline
21. अ॒सि॒ गृ॒द्ध्र॒सद् गृ॑द्ध्र॒स द॑स्यसि गृद्ध्र॒स द॑स्यसि गृद्ध्र॒स द॑स्यसि गृद्ध्र॒स द॑सि । \newline
22. गृ॒द्ध्र॒स द॑स्यसि गृद्ध्र॒सद् गृ॑द्ध्र॒स द॑सि सुपर्ण॒सथ् सु॑पर्ण॒स द॑सि गृद्ध्र॒सद् 
गृ॑द्ध्र॒स द॑सि सुपर्ण॒सत् । \newline
23. गृ॒द्ध्र॒सदिति॑ गृद्ध्र - सत् । \newline
24. अ॒सि॒ सु॒प॒र्ण॒सथ् सु॑पर्ण॒स द॑स्यसि सुपर्ण॒स द॑स्यसि सुपर्ण॒स द॑स्यसि सुपर्ण॒स द॑सि । \newline
25. सु॒प॒र्ण॒स द॑स्यसि सुपर्ण॒सथ् सु॑पर्ण॒सद॑सि नाक॒सन् ना॑क॒सद॑सि सुपर्ण॒सथ् सु॑पर्ण॒सद॑सि नाक॒सत् । \newline
26. सु॒प॒र्ण॒सदिति॑ सुपर्ण - सत् । \newline
27. अ॒सि॒ ना॒क॒सन् ना॑क॒सद॑स्यसि नाक॒सद॑स्यसि नाक॒सद॑स्यसि नाक॒सद॑सि । \newline
28. ना॒क॒सद॑स्यसि नाक॒सन् ना॑क॒सद॑सि पृथि॒व्याः पृ॑थि॒व्या अ॑सि नाक॒सन् ना॑क॒सद॑सि पृथि॒व्याः । \newline
29. ना॒क॒सदिति॑ नाक - सत् । \newline
30. अ॒सि॒ पृ॒थि॒व्याः पृ॑थि॒व्या अ॑स्यसि पृथि॒व्या स्त्वा᳚ त्वा पृथि॒व्या अ॑स्यसि पृथि॒व्या स्त्वा᳚ । \newline
31. पृ॒थि॒व्या स्त्वा᳚ त्वा पृथि॒व्याः पृ॑थि॒व्या स्त्वा॒ द्रवि॑णे॒ द्रवि॑णे त्वा पृथि॒व्याः पृ॑थि॒व्या स्त्वा॒ द्रवि॑णे । \newline
32. त्वा॒ द्रवि॑णे॒ द्रवि॑णे त्वा त्वा॒ द्रवि॑णे सादयामि सादयामि॒ द्रवि॑णे त्वा त्वा॒ द्रवि॑णे सादयामि । \newline
33. द्रवि॑णे सादयामि सादयामि॒ द्रवि॑णे॒ द्रवि॑णे सादया म्य॒न्तरि॑क्षस्या॒ न्तरि॑क्षस्य सादयामि॒ द्रवि॑णे॒ द्रवि॑णे सादया म्य॒न्तरि॑क्षस्य । \newline
34. सा॒द॒या॒ म्य॒न्तरि॑क्षस्या॒ न्तरि॑क्षस्य सादयामि सादया म्य॒न्तरि॑क्षस्य त्वा त्वा॒ ऽन्तरि॑क्षस्य सादयामि सादया म्य॒न्तरि॑क्षस्य त्वा । \newline
35. अ॒न्तरि॑क्षस्य त्वा त्वा॒ ऽन्तरि॑क्षस्या॒ न्तरि॑क्षस्य त्वा॒ द्रवि॑णे॒ द्रवि॑णे त्वा॒ ऽन्तरि॑क्षस्या॒ न्तरि॑क्षस्य त्वा॒ द्रवि॑णे । \newline
36. त्वा॒ द्रवि॑णे॒ द्रवि॑णे त्वा त्वा॒ द्रवि॑णे सादयामि सादयामि॒ द्रवि॑णे त्वा त्वा॒ द्रवि॑णे सादयामि । \newline
37. द्रवि॑णे सादयामि सादयामि॒ द्रवि॑णे॒ द्रवि॑णे सादयामि दि॒वो दि॒वः सा॑दयामि॒ द्रवि॑णे॒ द्रवि॑णे सादयामि दि॒वः । \newline
38. सा॒द॒या॒मि॒ दि॒वो दि॒वः सा॑दयामि सादयामि दि॒व स्त्वा᳚ त्वा दि॒वः सा॑दयामि सादयामि दि॒व स्त्वा᳚ । \newline
39. दि॒व स्त्वा᳚ त्वा दि॒वो दि॒व स्त्वा॒ द्रवि॑णे॒ द्रवि॑णे त्वा दि॒वो दि॒व स्त्वा॒ द्रवि॑णे । \newline
40. त्वा॒ द्रवि॑णे॒ द्रवि॑णे त्वा त्वा॒ द्रवि॑णे सादयामि सादयामि॒ द्रवि॑णे त्वा त्वा॒ द्रवि॑णे सादयामि । \newline
41. द्रवि॑णे सादयामि सादयामि॒ द्रवि॑णे॒ द्रवि॑णे सादयामि दि॒शाम् दि॒शाꣳ सा॑दयामि॒ द्रवि॑णे॒ द्रवि॑णे सादयामि दि॒शाम् । \newline
42. सा॒द॒या॒मि॒ दि॒शाम् दि॒शाꣳ सा॑दयामि सादयामि दि॒शाम् त्वा᳚ त्वा दि॒शाꣳ सा॑दयामि सादयामि दि॒शाम् त्वा᳚ । \newline
43. दि॒शाम् त्वा᳚ त्वा दि॒शाम् दि॒शाम् त्वा॒ द्रवि॑णे॒ द्रवि॑णे त्वा दि॒शाम् दि॒शाम् त्वा॒ द्रवि॑णे । \newline
44. त्वा॒ द्रवि॑णे॒ द्रवि॑णे त्वा त्वा॒ द्रवि॑णे सादयामि सादयामि॒ द्रवि॑णे त्वा त्वा॒ द्रवि॑णे सादयामि । \newline
45. द्रवि॑णे सादयामि सादयामि॒ द्रवि॑णे॒ द्रवि॑णे सादयामि द्रविणो॒दाम् द्र॑विणो॒दाꣳ सा॑दयामि॒ द्रवि॑णे॒ द्रवि॑णे सादयामि द्रविणो॒दाम् । \newline
46. सा॒द॒या॒मि॒ द्र॒वि॒णो॒दाम् द्र॑विणो॒दाꣳ सा॑दयामि सादयामि द्रविणो॒दाम् त्वा᳚ त्वा द्रविणो॒दाꣳ सा॑दयामि सादयामि द्रविणो॒दाम् त्वा᳚ । \newline
47. द्र॒वि॒णो॒दाम् त्वा᳚ त्वा द्रविणो॒दाम् द्र॑विणो॒दाम् त्वा॒ द्रवि॑णे॒ द्रवि॑णे त्वा द्रविणो॒दाम् द्र॑विणो॒दाम् त्वा॒ द्रवि॑णे । \newline
48. द्र॒वि॒णो॒दामिति॑ द्रविणः - दाम् । \newline
49. त्वा॒ द्रवि॑णे॒ द्रवि॑णे त्वा त्वा॒ द्रवि॑णे सादयामि सादयामि॒ द्रवि॑णे त्वा त्वा॒ द्रवि॑णे सादयामि । \newline
50. द्रवि॑णे सादयामि सादयामि॒ द्रवि॑णे॒ द्रवि॑णे सादयामि प्रा॒णम् प्रा॒णꣳ सा॑दयामि॒ द्रवि॑णे॒ द्रवि॑णे सादयामि प्रा॒णम् । \newline
51. सा॒द॒या॒मि॒ प्रा॒णम् प्रा॒णꣳ सा॑दयामि सादयामि प्रा॒णम् मे॑ मे प्रा॒णꣳ सा॑दयामि सादयामि प्रा॒णम् मे᳚ । \newline
52. प्रा॒णम् मे॑ मे प्रा॒णम् प्रा॒णम् मे॑ पाहि पाहि मे प्रा॒णम् प्रा॒णम् मे॑ पाहि । \newline
53. प्रा॒णमिति॑ प्र - अ॒नम् । \newline
54. मे॒ पा॒हि॒ पा॒हि॒ मे॒ मे॒ पा॒ह्य॒पा॒न म॑पा॒नम् पा॑हि मे मे पाह्यपा॒नम् । \newline
55. पा॒ह्य॒पा॒न म॑पा॒नम् पा॑हि पाह्यपा॒नम् मे॑ मे अपा॒नम् पा॑हि पाह्यपा॒नम् मे᳚ । \newline
56. अ॒पा॒नम् मे॑ मे अपा॒न म॑पा॒नम् मे॑ पाहि पाहि मे अपा॒न म॑पा॒नम् मे॑ पाहि । \newline
57. अ॒पा॒नमित्य॑प - अ॒नम् । \newline
58. मे॒ पा॒हि॒ पा॒हि॒ मे॒ मे॒ पा॒हि॒ व्या॒नं ॅव्या॒नम् पा॑हि मे मे पाहि व्या॒नम् । \newline
59. पा॒हि॒ व्या॒नं ॅव्या॒नम् पा॑हि पाहि व्या॒नम् मे॑ मे व्या॒नम् पा॑हि पाहि व्या॒नम् मे᳚ । \newline
60. व्या॒नम् मे॑ मे व्या॒नं ॅव्या॒नम् मे॑ पाहि पाहि मे व्या॒नं ॅव्या॒नम् मे॑ पाहि । \newline
61. व्या॒नमिति॑ वि - अ॒नम् । \newline
62. मे॒ पा॒हि॒ पा॒हि॒ मे॒ मे॒ पा॒ह्यायु॒ रायुः॑ पाहि मे मे पा॒ह्यायुः॑ । \newline
\pagebreak
\markright{ TS 4.4.7.2  \hfill https://www.vedavms.in \hfill}

\section{ TS 4.4.7.2 }

\textbf{TS 4.4.7.2 } \newline
\textbf{Samhita Paata} \newline

पा॒ह्यायु॑र्मे पाहि वि॒श्वायु॑र्मे पाहि स॒र्वायु॑र्मे पा॒ह्यग्ने॒ यत् ते॒ परꣳ॒॒ हृन्नाम॒ तावेहि॒ सꣳ र॑भावहै॒ पाञ्च॑ जन्ये॒ष्व-प्ये᳚द्ध्यग्ने॒ यावा॒ अया॑वा॒ एवा॒ ऊमाः॒ सब्दः॒ सग॑रः सु॒मेकः॑ ॥ \newline

\textbf{Pada Paata} \newline

पा॒हि॒ । आयुः॑ । मे॒ । पा॒हि॒ । वि॒श्वायु॒रिति॑ वि॒श्व-आ॒युः॒ । मे॒ । पा॒हि॒ । स॒र्वायु॒रिति॑ स॒र्व - आ॒युः॒ । मे॒ । पा॒हि॒ । अग्ने᳚ । यत् । ते॒ । पर᳚म् । हृत् । नाम॑ । तौ । एति॑ । इ॒हि॒ । समिति॑ । र॒भा॒व॒है॒ । पाञ्च॑जन्ये॒ष्विति॒ पाञ्च॑ - ज॒न्ये॒षु॒ । अपीति॑ । ए॒धि॒ । अ॒ग्ने॒ । यावाः᳚ । अया॑वाः । एवाः᳚ । ऊमाः᳚ । सब्दः॑ । सग॑रः । सु॒मेक॒ इति॑ सु - मेकः॑ ॥  \newline


\textbf{Krama Paata} \newline

पा॒ह्यायुः॑ । आयु॑र् मे । मे॒ पा॒हि॒ । पा॒हि॒ वि॒श्वायुः॑ । वि॒श्वायु॑र् मे । वि॒श्वायु॒रिति॑ वि॒श्व - आ॒युः॒ । मे॒ पा॒हि॒ । पा॒हि॒ स॒र्वायुः॑ । स॒र्वायु॑र् मे । स॒र्वायु॒रिति॑ स॒र्व - आ॒युः॒ । मे॒ पा॒हि॒ । पा॒ह्यग्ने᳚ । अग्ने॒ यत् । यत् ते᳚ । ते॒ पर᳚म् । परꣳ॒॒ हृत् । हृन् नाम॑ । नाम॒ तौ । तावा । एहि॑ । इ॒हि॒ सम् । सꣳ र॑भावहै । र॒भा॒व॒है॒ पाञ्च॑जन्येषु । पाञ्च॑जन्ये॒ष्वपि॑ । पाञ्च॑जन्ये॒ष्विति॒ पाञ्च॑ - ज॒न्ये॒षु॒ । अप्ये॑धि । ए॒द्ध्य॒ग्ने॒ । अ॒ग्ने॒ यावाः᳚ । यावा॒ अया॑वाः । अया॑वा॒ एवाः᳚ । एवा॒ ऊमाः᳚ । ऊमाः॒ सब्दः॑ । सब्दः॒ सग॑रः । सग॑रः सु॒मेकः॑ । सु॒मेक॒ इति॑ सु - मेकः॑ । \newline

\textbf{Jatai Paata} \newline

1. पा॒ह्या यु॒रायुः॑ पाहि पा॒ह्यायुः॑ । \newline
2. आयु॑र् मे म॒ आयु॒ रायु॑र् मे । \newline
3. मे॒ पा॒हि॒ पा॒हि॒ मे॒ मे॒ पा॒हि॒ । \newline
4. पा॒हि॒ वि॒श्वायु॑र् वि॒श्वायुः॑ पाहि पाहि वि॒श्वायुः॑ । \newline
5. वि॒श्वायु॑र् मे मे वि॒श्वायु॑र् वि॒श्वायु॑र् मे । \newline
6. वि॒श्वायु॒रिति॑ वि॒श्व - आ॒युः॒ । \newline
7. मे॒ पा॒हि॒ पा॒हि॒ मे॒ मे॒ पा॒हि॒ । \newline
8. पा॒हि॒ स॒र्वायुः॑ स॒र्वायुः॑ पाहि पाहि स॒र्वायुः॑ । \newline
9. स॒र्वायु॑र् मे मे स॒र्वायुः॑ स॒र्वायु॑र् मे । \newline
10. स॒र्वायु॒रिति॑ स॒र्व - आ॒युः॒ । \newline
11. मे॒ पा॒हि॒ पा॒हि॒ मे॒ मे॒ पा॒हि॒ । \newline
12. पा॒ह्यग्ने ऽग्ने॑ पाहि पा॒ह्यग्ने᳚ । \newline
13. अग्ने॒ यद् यदग्ने ऽग्ने॒ यत् । \newline
14. यत् ते॑ ते॒ यद् यत् ते᳚ । \newline
15. ते॒ पर॒म् पर॑म् ते ते॒ पर᳚म् । \newline
16. परꣳ॒॒ हृद् धृत् पर॒म् परꣳ॒॒ हृत् । \newline
17. हृन् नाम॒ नाम॒ हृ द्धृन् नाम॑ । \newline
18. नाम॒ तौ तौ नाम॒ नाम॒ तौ । \newline
19. ता वा तौ ता वा । \newline
20. एही॒ ह्ये हि॑ । \newline
21. इ॒हि॒ सꣳ स मि॑हीहि॒ सम् । \newline
22. सꣳ र॑भावहै रभावहै॒ सꣳ सꣳ र॑भावहै । \newline
23. र॒भा॒व॒है॒ पाञ्च॑जन्येषु॒ पाञ्च॑जन्येषु रभावहै रभावहै॒ पाञ्च॑जन्येषु । \newline
24. पाञ्च॑जन्ये॒ ष्वप्यपि॒ पाञ्च॑जन्येषु॒ पाञ्च॑जन्ये॒ ष्वपि॑ । \newline
25. पाञ्च॑जन्ये॒ष्विति॒ पाञ्च॑ - ज॒न्ये॒षु॒ । \newline
26. अप्ये᳚ ध्ये॒ध्य प्यप्ये॑धि । \newline
27. ए॒ध्य॒ग्ने॒ ऽग्न॒ ए॒ध्ये॒ ध्य॒ग्ने॒ । \newline
28. अ॒ग्ने॒ यावा॒ यावा॑ अग्ने ऽग्ने॒ यावाः᳚ । \newline
29. यावा॒ अया॑वा॒ अया॑वा॒ यावा॒ यावा॒ अया॑वाः । \newline
30. अया॑वा॒ एवा॒ एवा॒ अया॑वा॒ अया॑वा॒ एवाः᳚ । \newline
31. एवा॒ ऊमा॒ ऊमा॒ एवा॒ एवा॒ ऊमाः᳚ । \newline
32. ऊमाः॒ सब्दः॒ सब्द॒ ऊमा॒ ऊमाः॒ सब्दः॑ । \newline
33. सब्दः॒ सग॑रः॒ सग॑रः॒ सब्दः॒ सब्दः॒ सग॑रः । \newline
34. सग॑रः सु॒मेकः॑ सु॒मेकः॒ सग॑रः॒ सग॑रः सु॒मेकः॑ । \newline
35. सु॒मेक॒ इति॑ सु - मेकः॑ । \newline

\textbf{Ghana Paata } \newline

1. पा॒ह्यायु॒ रायुः॑ पाहि पा॒ह्यायु॑र् मे म॒ आयुः॑ पाहि पा॒ह्यायु॑र् मे । \newline
2. आयु॑र् मे म॒ आयु॒ रायु॑र् मे पाहि पाहि म॒ आयु॒ रायु॑र् मे पाहि । \newline
3. मे॒ पा॒हि॒ पा॒हि॒ मे॒ मे॒ पा॒हि॒ वि॒श्वायु॑र् वि॒श्वायुः॑ पाहि मे मे पाहि वि॒श्वायुः॑ । \newline
4. पा॒हि॒ वि॒श्वायु॑र् वि॒श्वायुः॑ पाहि पाहि वि॒श्वायु॑र् मे मे वि॒श्वायुः॑ पाहि पाहि वि॒श्वायु॑र् मे । \newline
5. वि॒श्वायु॑र् मे मे वि॒श्वायु॑र् वि॒श्वायु॑र् मे पाहि पाहि मे वि॒श्वायु॑र् वि॒श्वायु॑र् मे पाहि । \newline
6. वि॒श्वायु॒रिति॑ वि॒श्व - आ॒युः॒ । \newline
7. मे॒ पा॒हि॒ पा॒हि॒ मे॒ मे॒ पा॒हि॒ स॒र्वायुः॑ स॒र्वायुः॑ पाहि मे मे पाहि स॒र्वायुः॑ । \newline
8. पा॒हि॒ स॒र्वायुः॑ स॒र्वायुः॑ पाहि पाहि स॒र्वायु॑र् मे मे स॒र्वायुः॑ पाहि पाहि स॒र्वायु॑र् मे । \newline
9. स॒र्वायु॑र् मे मे स॒र्वायुः॑ स॒र्वायु॑र् मे पाहि पाहि मे स॒र्वायुः॑ स॒र्वायु॑र् मे पाहि । \newline
10. स॒र्वायु॒रिति॑ स॒र्व - आ॒युः॒ । \newline
11. मे॒ पा॒हि॒ पा॒हि॒ मे॒ मे॒ पा॒ह्यग्ने ऽग्ने॑ पाहि मे मे पा॒ह्यग्ने᳚ । \newline
12. पा॒ह्यग्ने ऽग्ने॑ पाहि पा॒ह्यग्ने॒ यद् यदग्ने॑ पाहि पा॒ह्यग्ने॒ यत् । \newline
13. अग्ने॒ यद् यदग्ने ऽग्ने॒ यत् ते॑ ते॒ यदग्ने ऽग्ने॒ यत् ते᳚ । \newline
14. यत् ते॑ ते॒ यद् यत् ते॒ पर॒म् पर॑म् ते॒ यद् यत् ते॒ पर᳚म् । \newline
15. ते॒ पर॒म् पर॑म् ते ते॒ परꣳ॒॒ हृ द्धृत् पर॑म् ते ते॒ परꣳ॒॒ हृत् । \newline
16. परꣳ॒॒ हृ द्धृत् पर॒म् परꣳ॒॒ हृन् नाम॒ नाम॒ हृत् पर॒म् परꣳ॒॒ हृन् नाम॑ । \newline
17. हृन् नाम॒ नाम॒ हृ द्धृन् नाम॒ तौ तौ नाम॒ हृ द्धृन् नाम॒ तौ । \newline
18. नाम॒ तौ तौ नाम॒ नाम॒ ता वा तौ नाम॒ नाम॒ ता वा । \newline
19. ता वा तौ ता वेही॒ह्या तौ ता वेहि॑ । \newline
20. एही॒ ह्येहि॒ सꣳ स मि॒ह्येहि॒ सम् । \newline
21. इ॒हि॒ सꣳ स मि॑हीहि॒ सꣳ र॑भावहै रभावहै॒ स मि॑हीहि॒ सꣳ र॑भावहै । \newline
22. सꣳ र॑भावहै रभावहै॒ सꣳ सꣳ र॑भावहै॒ पाञ्च॑जन्येषु॒ पाञ्च॑जन्येषु रभावहै॒ सꣳ सꣳ र॑भावहै॒ पाञ्च॑जन्येषु । \newline
23. र॒भा॒व॒है॒ पाञ्च॑जन्येषु॒ पाञ्च॑जन्येषु रभावहै रभावहै॒ पाञ्च॑जन्ये॒ ष्वप्यपि॒ पाञ्च॑जन्येषु रभावहै रभावहै॒ पाञ्च॑जन्ये॒ ष्वपि॑ । \newline
24. पाञ्च॑जन्ये॒ ष्वप्यपि॒ पाञ्च॑जन्येषु॒ पाञ्च॑जन्ये॒ ष्वप्ये᳚ ध्ये॒ ध्यपि॒ पाञ्च॑जन्येषु॒ पाञ्च॑जन्ये॒ ष्वप्ये॑धि । \newline
25. पाञ्च॑जन्ये॒ष्विति॒ पाञ्च॑ - ज॒न्ये॒षु॒ । \newline
26. अप्ये᳚ ध्ये॒ध्य प्यप्ये᳚ ध्यग्ने ऽग्न ए॒ध्य प्यप्ये᳚ ध्यग्ने । \newline
27. ए॒ध्य॒ग्ने॒ ऽग्न॒ ए॒ध्ये॒ध्य॒ग्ने॒ यावा॒ यावा॑ अग्न एध्येध्यग्ने॒ यावाः᳚ । \newline
28. अ॒ग्ने॒ यावा॒ यावा॑ अग्ने ऽग्ने॒ यावा॒ अया॑वा॒ अया॑वा॒ यावा॑ अग्ने ऽग्ने॒ यावा॒ अया॑वाः । \newline
29. यावा॒ अया॑वा॒ अया॑वा॒ यावा॒ यावा॒ अया॑वा॒ एवा॒ एवा॒ अया॑वा॒ यावा॒ यावा॒ अया॑वा॒ एवाः᳚ । \newline
30. अया॑वा॒ एवा॒ एवा॒ अया॑वा॒ अया॑वा॒ एवा॒ ऊमा॒ ऊमा॒ एवा॒ अया॑वा॒ अया॑वा॒ एवा॒ ऊमाः᳚ । \newline
31. एवा॒ ऊमा॒ ऊमा॒ एवा॒ एवा॒ ऊमाः॒ सब्दः॒ सब्द॒ ऊमा॒ एवा॒ एवा॒ ऊमाः॒ सब्दः॑ । \newline
32. ऊमाः॒ सब्दः॒ सब्द॒ ऊमा॒ ऊमाः॒ सब्दः॒ सग॑रः॒ सग॑रः॒ सब्द॒ ऊमा॒ ऊमाः॒ सब्दः॒ सग॑रः । \newline
33. सब्दः॒ सग॑रः॒ सग॑रः॒ सब्दः॒ सब्दः॒ सग॑रः सु॒मेकः॑ सु॒मेकः॒ सग॑रः॒ सब्दः॒ सब्दः॒ सग॑रः सु॒मेकः॑ । \newline
34. सग॑रः सु॒मेकः॑ सु॒मेकः॒ सग॑रः॒ सग॑रः सु॒मेकः॑ । \newline
35. सु॒मेक॒ इति॑ सु - मेकः॑ । \newline
\pagebreak
\markright{ TS 4.4.8.1  \hfill https://www.vedavms.in \hfill}

\section{ TS 4.4.8.1 }

\textbf{TS 4.4.8.1 } \newline
\textbf{Samhita Paata} \newline

अ॒ग्निना॑ विश्वा॒षाट् सूर्ये॑ण स्व॒राट् क्रत्वा॒ शची॒पति॑र्. ऋष॒भेण॒ त्वष्टा॑ य॒ज्ञेन॑ म॒घवा॒न् दक्षि॑णया सुव॒र्गो म॒न्युना॑ वृत्र॒हा सौहा᳚र्द्येन तनू॒धा अन्ने॑न॒ गयः॑ पृथि॒व्याऽस॑नो दृ॒ग्भिर॑न्ना॒दो व॑षट्का॒रेण॒र्द्धः साम्ना॑ तनू॒पा वि॒राजा॒ ज्योति॑ष्मा॒न् ब्रह्म॑णा सोम॒पा गोभि॑र्य॒ज्ञ्ं दा॑धार क्ष॒त्रेण॑ मनु॒ष्या॑-नश्वे॑न च॒ रथे॑न च व॒ज्र्यृ॑तुभिः॑ प्र॒भुः सं॑ॅवथ्स॒रेण॑ परि॒भू स्तप॒साऽना॑धृष्टः॒ सूर्यः॒ सन् त॒नूभिः॑ ॥ \newline

\textbf{Pada Paata} \newline

अ॒ग्निना᳚ । वि॒श्वा॒षाट् । सूर्ये॑ण । स्व॒राडिति॑ स्व - राट् । क्रत्वा᳚ । शची॒पतिः॑ । ऋ॒ष॒भेण॑ । त्वष्टा᳚ । य॒ज्ञेन॑ । म॒घवा॒निति॑ म॒घ-वा॒न् । दक्षि॑णया । सु॒व॒र्ग इति॑ सुवः - गः । म॒न्युना᳚ । वृ॒त्र॒हेति॑ वृत्र - हा । सौहा᳚र्द्येन । त॒नू॒धा इति॑ तनू - धाः । अन्ने॑न । गयः॑ । पृ॒थि॒व्या । अ॒स॒नो॒त् । ऋ॒ग्भिरित्यृ॑क् - भिः । अ॒न्ना॒द इत्य॑न्न - अ॒दः । व॒ष॒ट्का॒रेणेति॑ वषट् - का॒रेण॑ । ऋ॒द्धः । साम्ना᳚ । त॒नू॒पा इति॑ तनू - पाः । वि॒राजेति॑ वि - राजा᳚ । ज्योति॑ष्मान् । ब्रह्म॑णा । सो॒म॒पा इति॑ सोम - पाः । गोभिः॑ । य॒ज्ञ्म् । दा॒धा॒र॒ । क्ष॒त्रेण॑ । म॒नु॒ष्यान्॑ । अश्वे॑न । च॒ । रथे॑न । च॒ । व॒ज्री । ऋ॒तुभि॒रित्यृ॒तु - भिः॒ । प्र॒भुरिति॑ प्र - भुः । सं॒ॅव॒थ्स॒रेणेति॑ सं - व॒थ्स॒रेण॑ । प॒रि॒भूरिति॑ परि - भूः । तप॑सा । अना॑धृष्ट॒ इत्यना᳚ - धृ॒ष्टः॒ । सूर्यः॑ । सन्न् । त॒नूभिः॑ ॥  \newline


\textbf{Krama Paata} \newline

अ॒ग्निना॑ विश्वा॒षाट् । वि॒श्वा॒षाट्थ् सूर्ये॑ण । सूर्ये॑ण स्व॒राट् । स्व॒राट् क्रत्वा᳚ । स्व॒राडिति॑ स्व - राट् । क्रत्वा॒ शची॒पतिः॑ । शची॒पति॑र्. ऋष॒भेण॑ । ऋ॒ष॒भेण॒ त्वष्टा᳚ । त्वष्टा॑ य॒ज्ञेन॑ । य॒ज्ञेन॑ म॒घवान्॑ । म॒घवा॒न् दक्षि॑णया । म॒घवा॒निति॑ म॒घ - वा॒न्॒ । दक्षि॑णया सुव॒र्गः । सु॒व॒र्गो म॒न्युना᳚ । सु॒व॒र्ग इति॑ सुवः - गः । म॒न्युना॑ वृत्र॒हा । वृ॒त्र॒हा सौहा᳚र्द्येन । वृ॒त्र॒हेति॑ वृत्र - हा । सौहा᳚र्द्येन तनू॒धाः । त॒नू॒धा अन्ने॑न । त॒नू॒धा इति॑ तनू - धाः । अन्ने॑न॒ गयः॑ । गयः॑ पृथि॒व्या । पृ॒थि॒व्याऽस॑नोत् । अ॒स॒नो॒दृ॒ग्भिः । ऋ॒ग्भिर॑न्ना॒दः । ऋ॒ग्भिरित्यृ॑क् - भिः । अ॒न्ना॒दो व॑षट्का॒रेण॑ । अ॒न्ना॒द इत्य॑न्न - अ॒दः । व॒ष॒ट्का॒रेण॒र्द्धः । व॒ष॒ट्का॒रेणेति॑ वषट् - का॒रेण॑ । ऋ॒द्धः साम्ना᳚ । साम्ना॑ तनू॒पाः । त॒नू॒पा वि॒राजा᳚ । त॒नू॒पा इति॑ तनू - पाः । वि॒राजा॒ ज्योति॑ष्मान् । वि॒राजेति॑ वि - राजा᳚ । ज्योति॑ष्मा॒न् ब्रह्म॑णा । ब्रह्म॑णा सोम॒पाः । सो॒म॒पा गोभिः॑ । सो॒म॒पा इति॑ सोम - पाः । गोभि॑र् य॒ज्ञ्म् । य॒ज्ञ्म् दा॑धार । दा॒धा॒र॒ क्ष॒त्रेण॑ । क्ष॒त्रेण॑ मनु॒ष्यान्॑ । म॒नु॒ष्या॑नश्वे॑न । अश्वे॑न च । च॒ रथे॑न । रथे॑न च । च॒ व॒ज्री । व॒ज्र्यृ॑तुभिः॑ । ऋ॒तुभिः॑ प्र॒भुः । ऋ॒तुभि॒रित्यृ॒तु - भिः॒ । प्र॒भुः स॑म्ॅवथ्स॒रेण॑ । प्र॒भुरिति॑ प्र - भुः । स॒म्ॅव॒थ्स॒रेण॑ परि॒भूः । स॒म्ॅव॒थ्स॒रेणेति॑ सं - व॒थ्स॒रेण॑ । प॒रि॒भूस्तप॑सा । प॒रि॒भूरिति॑ परि - भूः । तप॒साऽना॑धृष्टः । अना॑धृष्टः॒ सूर्यः॑ । अना॑धृष्ट॒ इत्यना᳚ - धृ॒ष्टः॒ । सूर्यः॒ सन्न् । सन् त॒नूभिः॑ । त॒नूभि॒रिति॑ त॒नूभिः॑ । \newline

\textbf{Jatai Paata} \newline

1. अ॒ग्निना॑ विश्वा॒षाड् वि॑श्वा॒षा ड॒ग्निना॒ ऽग्निना॑ विश्वा॒षाट् । \newline
2. वि॒श्वा॒षाट् थ्सूर्ये॑ण॒ सूर्ये॑ण विश्वा॒षाड् वि॑श्वा॒षाट् थ्सूर्ये॑ण । \newline
3. सूर्ये॑ण स्व॒राट् थ्स्व॒राट् थ्सूर्ये॑ण॒ सूर्ये॑ण स्व॒राट् । \newline
4. स्व॒राट् क्रत्वा॒ क्रत्वा᳚ स्व॒राट् थ्स्व॒राट् क्रत्वा᳚ । \newline
5. स्व॒राडिति॑ स्व - राट् । \newline
6. क्रत्वा॒ शची॒पतिः॒ शची॒पतिः॒ क्रत्वा॒ क्रत्वा॒ शची॒पतिः॑ । \newline
7. शची॒पति॑र्. ऋष॒भेण॑ र्.ष॒भेण॒ शची॒पतिः॒ शची॒पति॑र्. ऋष॒भेण॑ । \newline
8. ऋ॒ष॒भेण॒ त्वष्टा॒ त्वष्ट॑ र्.ष॒भेण॑ र्.ष॒भेण॒ त्वष्टा᳚ । \newline
9. त्वष्टा॑ य॒ज्ञेन॑ य॒ज्ञेन॒ त्वष्टा॒ त्वष्टा॑ य॒ज्ञेन॑ । \newline
10. य॒ज्ञेन॑ म॒घवा᳚न् म॒घवान्॑. य॒ज्ञेन॑ य॒ज्ञेन॑ म॒घवान्॑ । \newline
11. म॒घवा॒न् दक्षि॑णया॒ दक्षि॑णया म॒घवा᳚न् म॒घवा॒न् दक्षि॑णया । \newline
12. म॒घवा॒निति॑ म॒घ - वा॒न् । \newline
13. दक्षि॑णया सुव॒र्गः सु॑व॒र्गो दक्षि॑णया॒ दक्षि॑णया सुव॒र्गः । \newline
14. सु॒व॒र्गो म॒न्युना॑ म॒न्युना॑ सुव॒र्गः सु॑व॒र्गो म॒न्युना᳚ । \newline
15. सु॒व॒र्ग इति॑ सुवः - गः । \newline
16. म॒न्युना॑ वृत्र॒हा वृ॑त्र॒हा म॒न्युना॑ म॒न्युना॑ वृत्र॒हा । \newline
17. वृ॒त्र॒हा सौहा᳚र्द्येन॒ सौहा᳚र्द्येन वृत्र॒हा वृ॑त्र॒हा सौहा᳚र्द्येन । \newline
18. वृ॒त्र॒हेति॑ वृत्र - हा । \newline
19. सौहा᳚र्द्येन तनू॒धा स्त॑नू॒धाः सौहा᳚र्द्येन॒ सौहा᳚र्द्येन तनू॒धाः । \newline
20. त॒नू॒धा अन्ने॒नान्ने॑न तनू॒धा स्त॑नू॒धा अन्ने॑न । \newline
21. त॒नू॒धा इति॑ तनू - धाः । \newline
22. अन्ने॑न॒ गयो॒ गयो ऽन्ने॒ना न्ने॑न॒ गयः॑ । \newline
23. गयः॑ पृथि॒व्या पृ॑थि॒व्या गयो॒ गयः॑ पृथि॒व्या । \newline
24. पृ॒थि॒व्या ऽस॑नो दसनोत् पृथि॒व्या पृ॑थि॒व्या ऽस॑नोत् । \newline
25. अ॒स॒नो॒ दृ॒ग्भिर्. ऋ॒ग्भि र॑सनो दसनो दृ॒ग्भिः । \newline
26. ऋ॒ग्भि र॑न्ना॒दो᳚ ऽन्ना॒द ऋ॒ग्भिर्.ऋ॒ग्भि र॑न्ना॒दः । \newline
27. ऋ॒ग्भिरित्यृ॑क् - भिः । \newline
28. अ॒न्ना॒दो व॑षट्का॒रेण॑ वषट्का॒रेणा᳚ न्ना॒दो᳚ ऽन्ना॒दो व॑षट्का॒रेण॑ । \newline
29. अ॒न्ना॒द इत्य॑न्न - अ॒दः । \newline
30. व॒ष॒ट्का॒रेण॒ र्द्ध ऋ॒द्धो व॑षट्का॒रेण॑ वषट्का॒रेण॒ र्द्धः । \newline
31. व॒ष॒ट्का॒रेणेति॑ वषट् - का॒रेण॑ । \newline
32. ऋ॒द्धः साम्ना॒ साम्न॒ र्द्ध ऋ॒द्धः साम्ना᳚ । \newline
33. साम्ना॑ तनू॒पा स्त॑नू॒पाः साम्ना॒ साम्ना॑ तनू॒पाः । \newline
34. त॒नू॒पा वि॒राजा॑ वि॒राजा॑ तनू॒पा स्त॑नू॒पा वि॒राजा᳚ । \newline
35. त॒नू॒पा इति॑ तनू - पाः । \newline
36. वि॒राजा॒ ज्योति॑ष्मा॒न् ज्योति॑ष्मान्. वि॒राजा॑ वि॒राजा॒ ज्योति॑ष्मान् । \newline
37. वि॒राजेति॑ वि - राजा᳚ । \newline
38. ज्योति॑ष्मा॒न् ब्रह्म॑णा॒ ब्रह्म॑णा॒ ज्योति॑ष्मा॒न् ज्योति॑ष्मा॒न् ब्रह्म॑णा । \newline
39. ब्रह्म॑णा सोम॒पाः सो॑म॒पा ब्रह्म॑णा॒ ब्रह्म॑णा सोम॒पाः । \newline
40. सो॒म॒पा गोभि॒र् गोभिः॑ सोम॒पाः सो॑म॒पा गोभिः॑ । \newline
41. सो॒म॒पा इति॑ सोम - पाः । \newline
42. गोभि॑र् य॒ज्ञ्ं ॅय॒ज्ञ्म् गोभि॒र् गोभि॑र् य॒ज्ञ्म् । \newline
43. य॒ज्ञ्म् दा॑धार दाधार य॒ज्ञ्ं ॅय॒ज्ञ्म् दा॑धार । \newline
44. दा॒धा॒र॒ क्ष॒त्रेण॑ क्ष॒त्रेण॑ दाधार दाधार क्ष॒त्रेण॑ । \newline
45. क्ष॒त्रेण॑ मनु॒ष्या᳚न् मनु॒ष्या᳚न् क्ष॒त्रेण॑ क्ष॒त्रेण॑ मनु॒ष्यान्॑ । \newline
46. म॒नु॒ष्या॑ नश्वे॒नाश्वे॑न मनु॒ष्या᳚न् मनु॒ष्या॑ नश्वे॑न । \newline
47. अश्वे॑न च॒ चाश्वे॒ना श्वे॑न च । \newline
48. च॒ रथे॑न॒ रथे॑न च च॒ रथे॑न । \newline
49. रथे॑न च च॒ रथे॑न॒ रथे॑न च । \newline
50. च॒ व॒ज्री व॒ज्री च॑ च व॒ज्री । \newline
51. व॒ज्‌र्यृ॑तुभि॑र्. ऋ॒तुभि॑र् व॒ज्री व॒ज्‌र्यृ॑तुभिः॑ । \newline
52. ऋ॒तुभिः॑ प्र॒भुः प्र॒भुर्. ऋ॒तुभि॑र्. ऋ॒तुभिः॑ प्र॒भुः । \newline
53. ऋ॒तुभि॒रित्यृ॒तु - भिः॒ । \newline
54. प्र॒भुः सं॑ॅवथ्स॒रेण॑ संॅवथ्स॒रेण॑ प्र॒भुः प्र॒भुः सं॑ॅवथ्स॒रेण॑ । \newline
55. प्र॒भुरिति॑ प्र - भुः । \newline
56. सं॒ॅव॒थ्स॒रेण॑ परि॒भूः प॑रि॒भूः सं॑ॅवथ्स॒रेण॑ संॅवथ्स॒रेण॑ परि॒भूः । \newline
57. सं॒ॅव॒थ्स॒रेणेति॑ सं - व॒थ्स॒रेण॑ । \newline
58. प॒रि॒भू स्तप॑सा॒ तप॑सा परि॒भूः प॑रि॒भू स्तप॑सा । \newline
59. प॒रि॒भूरिति॑ परि - भूः । \newline
60. तप॒सा ऽना॑धृ॒ष्टो ऽना॑धृष्ट॒ स्तप॑सा॒ तप॒सा ऽना॑धृष्टः । \newline
61. अना॑धृष्टः॒ सूर्यः॒ सूर्यो ऽना॑धृ॒ष्टो ऽना॑धृष्टः॒ सूर्यः॑ । \newline
62. अना॑धृष्ट॒ इत्यना᳚ - धृ॒ष्टः॒ । \newline
63. सूर्यः॒ सन् थ्सन् थ्सूर्यः॒ सूर्यः॒ सन्न् । \newline
64. सन् त॒नूभि॑ स्त॒नूभिः॒ सन् थ्सन् त॒नूभिः॑ । \newline
65. त॒नूभि॒रिति॑ त॒नूभिः॑ । \newline

\textbf{Ghana Paata } \newline

1. अ॒ग्निना॑ विश्वा॒षाड् वि॑श्वा॒षा ड॒ग्निना॒ ऽग्निना॑ विश्वा॒षाट् थ्सूर्ये॑ण॒ सूर्ये॑ण विश्वा॒षा ड॒ग्निना॒ ऽग्निना॑ विश्वा॒षाट् थ्सूर्ये॑ण । \newline
2. वि॒श्वा॒षाट् थ्सूर्ये॑ण॒ सूर्ये॑ण विश्वा॒षाड् वि॑श्वा॒षाट् थ्सूर्ये॑ण स्व॒राट् थ्स्व॒राट् थ्सूर्ये॑ण विश्वा॒षाड् वि॑श्वा॒षाट् थ्सूर्ये॑ण स्व॒राट् । \newline
3. सूर्ये॑ण स्व॒राट् थ्स्व॒राट् थ्सूर्ये॑ण॒ सूर्ये॑ण स्व॒राट् क्रत्वा॒ क्रत्वा᳚ स्व॒राट् थ्सूर्ये॑ण॒ सूर्ये॑ण स्व॒राट् क्रत्वा᳚ । \newline
4. स्व॒राट् क्रत्वा॒ क्रत्वा᳚ स्व॒राट् थ्स्व॒राट् क्रत्वा॒ शची॒पतिः॒ शची॒पतिः॒ क्रत्वा᳚ स्व॒राट् थ्स्व॒राट् क्रत्वा॒ शची॒पतिः॑ । \newline
5. स्व॒राडिति॑ स्व - राट् । \newline
6. क्रत्वा॒ शची॒पतिः॒ शची॒पतिः॒ क्रत्वा॒ क्रत्वा॒ शची॒पति॑र्. ऋष॒भेण॑ र्.ष॒भेण॒ शची॒पतिः॒ क्रत्वा॒ क्रत्वा॒ शची॒पति॑र्. ऋष॒भेण॑ । \newline
7. शची॒पति॑र्. ऋष॒भेण॑र्.ष॒भेण॒ शची॒पतिः॒ शची॒पति॑र्. ऋष॒भेण॒ त्वष्टा॒ त्वष्ट॑र्.ष॒भेण॒ शची॒पतिः॒ शची॒पति॑र्. ऋष॒भेण॒ त्वष्टा᳚ । \newline
8. ऋ॒ष॒भेण॒ त्वष्टा॒ त्वष्ट॑ र्.ष॒भेण॑ र्.ष॒भेण॒ त्वष्टा॑ य॒ज्ञेन॑ य॒ज्ञेन॒ त्वष्ट॑ र्.ष॒भेण॑ र्.ष॒भेण॒ त्वष्टा॑ य॒ज्ञेन॑ । \newline
9. त्वष्टा॑ य॒ज्ञेन॑ य॒ज्ञेन॒ त्वष्टा॒ त्वष्टा॑ य॒ज्ञेन॑ म॒घवा᳚न् म॒घवान्॑. य॒ज्ञेन॒ त्वष्टा॒ त्वष्टा॑ य॒ज्ञेन॑ म॒घवान्॑ । \newline
10. य॒ज्ञेन॑ म॒घवा᳚न् म॒घवान्॑. य॒ज्ञेन॑ य॒ज्ञेन॑ म॒घवा॒न् दक्षि॑णया॒ दक्षि॑णया म॒घवान्॑. य॒ज्ञेन॑ य॒ज्ञेन॑ म॒घवा॒न् दक्षि॑णया । \newline
11. म॒घवा॒न् दक्षि॑णया॒ दक्षि॑णया म॒घवा᳚न् म॒घवा॒न् दक्षि॑णया सुव॒र्गः सु॑व॒र्गो दक्षि॑णया म॒घवा᳚न् म॒घवा॒न् दक्षि॑णया सुव॒र्गः । \newline
12. म॒घवा॒निति॑ म॒घ - वा॒न् । \newline
13. दक्षि॑णया सुव॒र्गः सु॑व॒र्गो दक्षि॑णया॒ दक्षि॑णया सुव॒र्गो म॒न्युना॑ म॒न्युना॑ सुव॒र्गो दक्षि॑णया॒ दक्षि॑णया सुव॒र्गो म॒न्युना᳚ । \newline
14. सु॒व॒र्गो म॒न्युना॑ म॒न्युना॑ सुव॒र्गः सु॑व॒र्गो म॒न्युना॑ वृत्र॒हा वृ॑त्र॒हा म॒न्युना॑ सुव॒र्गः 
सु॑व॒र्गो म॒न्युना॑ वृत्र॒हा । \newline
15. सु॒व॒र्ग इति॑ सुवः - गः । \newline
16. म॒न्युना॑ वृत्र॒हा वृ॑त्र॒हा म॒न्युना॑ म॒न्युना॑ वृत्र॒हा सौहा᳚र्द्येन॒ सौहा᳚र्द्येन वृत्र॒हा म॒न्युना॑ 
म॒न्युना॑ वृत्र॒हा सौहा᳚र्द्येन । \newline
17. वृ॒त्र॒हा सौहा᳚र्द्येन॒ सौहा᳚र्द्येन वृत्र॒हा वृ॑त्र॒हा सौहा᳚र्द्येन तनू॒धा स्त॑नू॒धाः सौहा᳚र्द्येन वृत्र॒हा वृ॑त्र॒हा सौहा᳚र्द्येन तनू॒धाः । \newline
18. वृ॒त्र॒हेति॑ वृत्र - हा । \newline
19. सौहा᳚र्द्येन तनू॒धा स्त॑नू॒धाः सौहा᳚र्द्येन॒ सौहा᳚र्द्येन तनू॒धा अन्ने॒ना न्ने॑न तनू॒धाः सौहा᳚र्द्येन॒ सौहा᳚र्द्येन तनू॒धा अन्ने॑न । \newline
20. त॒नू॒धा अन्ने॒ना न्ने॑न तनू॒धा स्त॑नू॒धा अन्ने॑न॒ गयो॒ गयो ऽन्ने॑न तनू॒धा स्त॑नू॒धा अन्ने॑न॒ गयः॑ । \newline
21. त॒नू॒धा इति॑ तनू - धाः । \newline
22. अन्ने॑न॒ गयो॒ गयो ऽन्ने॒ना न्ने॑न॒ गयः॑ पृथि॒व्या पृ॑थि॒व्या गयो ऽन्ने॒ना न्ने॑न॒ गयः॑ पृथि॒व्या । \newline
23. गयः॑ पृथि॒व्या पृ॑थि॒व्या गयो॒ गयः॑ पृथि॒व्या ऽस॑नो दसनोत् पृथि॒व्या गयो॒ गयः॑ पृथि॒व्या ऽस॑नोत् । \newline
24. पृ॒थि॒व्या ऽस॑नो दसनोत् पृथि॒व्या पृ॑थि॒व्या ऽस॑नो दृ॒ग्भिर्. ऋ॒ग्भि र॑सनोत् पृथि॒व्या पृ॑थि॒व्या ऽस॑नो दृ॒ग्भिः । \newline
25. अ॒स॒नो॒ दृ॒ग्भिर्. ऋ॒ग्भि र॑सनो दसनो दृ॒ग्भि र॑न्ना॒दो᳚ ऽन्ना॒द ऋ॒ग्भि र॑सनो दसनो दृ॒ग्भि र॑न्ना॒दः । \newline
26. ऋ॒ग्भि र॑न्ना॒दो᳚ ऽन्ना॒द ऋ॒ग्भिर्. ऋ॒ग्भि र॑न्ना॒दो व॑षट्का॒रेण॑ वषट्का॒रेणा᳚ न्ना॒द ऋ॒ग्भिर्. ऋ॒ग्भि र॑न्ना॒दो व॑षट्का॒रेण॑ । \newline
27. ऋ॒ग्भिरित्यृ॑क् - भिः । \newline
28. अ॒न्ना॒दो व॑षट्का॒रेण॑ वषट्का॒रेणा᳚ न्ना॒दो᳚ ऽन्ना॒दो व॑षट्का॒रेण॒ र्‌द्ध ऋ॒द्धो व॑षट्का॒रेणा᳚ न्ना॒दो᳚ ऽन्ना॒दो व॑षट्का॒रेण॒ र्‌द्धः । \newline
29. अ॒न्ना॒द इत्य॑न्न - अ॒दः । \newline
30. व॒ष॒ट्का॒रेण॒ र्‌द्ध ऋ॒द्धो व॑षट्का॒रेण॑ वषट्का॒रेण॒ र्‌द्धः साम्ना॒ साम्न॒ र्‌द्धो व॑षट्का॒रेण॑ वषट्का॒रेण॒ र्‌द्धः साम्ना᳚ । \newline
31. व॒ष॒ट्का॒रेणेति॑ वषट् - का॒रेण॑ । \newline
32. ऋ॒द्धः साम्ना॒ साम्न॒ र्‌द्ध ऋ॒द्धः साम्ना॑ तनू॒पा स्त॑नू॒पाः साम्न॒ र्‌द्ध ऋ॒द्धः साम्ना॑ तनू॒पाः । \newline
33. साम्ना॑ तनू॒पा स्त॑नू॒पाः साम्ना॒ साम्ना॑ तनू॒पा वि॒राजा॑ वि॒राजा॑ तनू॒पाः साम्ना॒ साम्ना॑ तनू॒पा वि॒राजा᳚ । \newline
34. त॒नू॒पा वि॒राजा॑ वि॒राजा॑ तनू॒पा स्त॑नू॒पा वि॒राजा॒ ज्योति॑ष्मा॒न् ज्योति॑ष्मान्. वि॒राजा॑ तनू॒पा स्त॑नू॒पा वि॒राजा॒ ज्योति॑ष्मान् । \newline
35. त॒नू॒पा इति॑ तनू - पाः । \newline
36. वि॒राजा॒ ज्योति॑ष्मा॒न् ज्योति॑ष्मान्. वि॒राजा॑ वि॒राजा॒ ज्योति॑ष्मा॒न् ब्रह्म॑णा॒ ब्रह्म॑णा॒ ज्योति॑ष्मान्. वि॒राजा॑ वि॒राजा॒ ज्योति॑ष्मा॒न् ब्रह्म॑णा । \newline
37. वि॒राजेति॑ वि - राजा᳚ । \newline
38. ज्योति॑ष्मा॒न् ब्रह्म॑णा॒ ब्रह्म॑णा॒ ज्योति॑ष्मा॒न् ज्योति॑ष्मा॒न् ब्रह्म॑णा सोम॒पाः सो॑म॒पा ब्रह्म॑णा॒ ज्योति॑ष्मा॒न् ज्योति॑ष्मा॒न् ब्रह्म॑णा सोम॒पाः । \newline
39. ब्रह्म॑णा सोम॒पाः सो॑म॒पा ब्रह्म॑णा॒ ब्रह्म॑णा सोम॒पा गोभि॒र् गोभिः॑ सोम॒पा ब्रह्म॑णा॒ ब्रह्म॑णा सोम॒पा गोभिः॑ । \newline
40. सो॒म॒पा गोभि॒र् गोभिः॑ सोम॒पाः सो॑म॒पा गोभि॑र् य॒ज्ञ्ं ॅय॒ज्ञ्म् गोभिः॑ सोम॒पाः सो॑म॒पा गोभि॑र् य॒ज्ञ्म् । \newline
41. सो॒म॒पा इति॑ सोम - पाः । \newline
42. गोभि॑र् य॒ज्ञ्ं ॅय॒ज्ञ्म् गोभि॒र् गोभि॑र् य॒ज्ञ्म् दा॑धार दाधार य॒ज्ञ्म् गोभि॒र् गोभि॑र् य॒ज्ञ्म् दा॑धार । \newline
43. य॒ज्ञ्म् दा॑धार दाधार य॒ज्ञ्ं ॅय॒ज्ञ्म् दा॑धार क्ष॒त्रेण॑ क्ष॒त्रेण॑ दाधार य॒ज्ञ्ं ॅय॒ज्ञ्म् दा॑धार क्ष॒त्रेण॑ । \newline
44. दा॒धा॒र॒ क्ष॒त्रेण॑ क्ष॒त्रेण॑ दाधार दाधार क्ष॒त्रेण॑ मनु॒ष्या᳚न् मनु॒ष्या᳚न् क्ष॒त्रेण॑ दाधार 
दाधार क्ष॒त्रेण॑ मनु॒ष्यान्॑ । \newline
45. क्ष॒त्रेण॑ मनु॒ष्या᳚न् मनु॒ष्या᳚न् क्ष॒त्रेण॑ क्ष॒त्रेण॑ मनु॒ष्या॑ नश्वे॒ना श्वे॑न मनु॒ष्या᳚न् क्ष॒त्रेण॑ क्ष॒त्रेण॑ मनु॒ष्या॑ नश्वे॑न । \newline
46. म॒नु॒ष्या॑ नश्वे॒ना श्वे॑न मनु॒ष्या᳚न् मनु॒ष्या॑ नश्वे॑न च॒ चाश्वे॑न मनु॒ष्या᳚न् मनु॒ष्या॑ नश्वे॑न च । \newline
47. अश्वे॑न च॒ चाश्वे॒ना श्वे॑न च॒ रथे॑न॒ रथे॑न॒ चाश्वे॒ना श्वे॑न च॒ रथे॑न । \newline
48. च॒ रथे॑न॒ रथे॑न च च॒ रथे॑न च च॒ रथे॑न च च॒ रथे॑न च । \newline
49. रथे॑न च च॒ रथे॑न॒ रथे॑न च व॒ज्री व॒ज्री च॒ रथे॑न॒ रथे॑न च व॒ज्री । \newline
50. च॒ व॒ज्री व॒ज्री च॑ च व॒ज्‌र्यृ॑तुभि॑र्. ऋ॒तुभि॑र् व॒ज्री च॑ च व॒ज्‌र्यृ॑तुभिः॑ । \newline
51. व॒ज्‌र्यृ॑तुभि॑र्. ऋ॒तुभि॑र् व॒ज्री व॒ज्‌र्यृ॑तुभिः॑ प्र॒भुः प्र॒भुर्. ऋ॒तुभि॑र् व॒ज्री व॒ज्‌र्यृ॑तुभिः॑ प्र॒भुः । \newline
52. ऋ॒तुभिः॑ प्र॒भुः प्र॒भुर्. ऋ॒तुभिर्॑. ऋ॒तुभिः॑ प्र॒भुः सं॑ॅवथ्स॒रेण॑ संॅवथ्स॒रेण॑ प्र॒भुर्. ऋ॒तुभि॑र्. 
ऋ॒तुभिः॑ प्र॒भुः सं॑ॅवथ्स॒रेण॑ । \newline
53. ऋ॒तुभि॒रित्यृ॒तु - भिः॒ । \newline
54. प्र॒भुः सं॑ॅवथ्स॒रेण॑ संॅवथ्स॒रेण॑ प्र॒भुः प्र॒भुः सं॑ॅवथ्स॒रेण॑ परि॒भूः प॑रि॒भूः सं॑ॅवथ्स॒रेण॑ प्र॒भुः प्र॒भुः सं॑ॅवथ्स॒रेण॑ परि॒भूः । \newline
55. प्र॒भुरिति॑ प्र - भुः । \newline
56. सं॒ॅव॒थ्स॒रेण॑ परि॒भूः प॑रि॒भूः सं॑ॅवथ्स॒रेण॑ संॅवथ्स॒रेण॑ परि॒भू स्तप॑सा॒ तप॑सा परि॒भूः सं॑ॅवथ्स॒रेण॑ संॅवथ्स॒रेण॑ परि॒भू स्तप॑सा । \newline
57. सं॒ॅव॒थ्स॒रेणेति॑ सं - व॒थ्स॒रेण॑ । \newline
58. प॒रि॒भू स्तप॑सा॒ तप॑सा परि॒भूः प॑रि॒भू स्तप॒सा ऽना॑धृ॒ष्टो ऽना॑धृष्ट॒ स्तप॑सा परि॒भूः प॑रि॒भू स्तप॒सा ऽना॑धृष्टः । \newline
59. प॒रि॒भूरिति॑ परि - भूः । \newline
60. तप॒सा ऽना॑धृ॒ष्टो ऽना॑धृष्ट॒ स्तप॑सा॒ तप॒सा ऽना॑धृष्टः॒ सूर्यः॒ सूर्यो ऽना॑धृष्ट॒ स्तप॑सा॒ तप॒सा ऽना॑धृष्टः॒ सूर्यः॑ । \newline
61. अना॑धृष्टः॒ सूर्यः॒ सूर्यो ऽना॑धृ॒ष्टो ऽना॑धृष्टः॒ सूर्यः॒ सन् थ्सन् थ्सूर्यो ऽना॑धृ॒ष्टो ऽना॑धृष्टः॒ सूर्यः॒ सन्न् । \newline
62. अना॑धृष्ट॒ इत्यना᳚ - धृ॒ष्टः॒ । \newline
63. सूर्यः॒ सन् थ्सन् थ्सूर्यः॒ सूर्यः॒ सन् त॒नूभि॑ स्त॒नूभिः॒ सन् थ्सूर्यः॒ सूर्यः॒ सन् त॒नूभिः॑ । \newline
64. सन् त॒नूभि॑ स्त॒नूभिः॒ सन् थ्सन् त॒नूभिः॑ । \newline
65. त॒नूभि॒रिति॑ त॒नूभिः॑ । \newline
\pagebreak
\markright{ TS 4.4.9.1  \hfill https://www.vedavms.in \hfill}

\section{ TS 4.4.9.1 }

\textbf{TS 4.4.9.1 } \newline
\textbf{Samhita Paata} \newline

प्र॒जाप॑ति॒र्मन॒सा ऽन्धोऽच्छे॑तो धा॒ता दी॒क्षायाꣳ॑ सवि॒ता भृ॒त्यां पू॒षा सो॑म॒क्रय॑ण्यां॒ ॅवरु॑ण॒ उप॑न॒द्धो ऽसु॑रः क्री॒यमा॑णो मि॒त्रः क्री॒तः शि॑पिवि॒ष्ट आसा॑दितो न॒रंधि॑षः प्रो॒ह्यमा॒णो ऽधि॑पति॒राग॑तः प्र॒जाप॑तिः प्रणी॒यमा॑नो॒ ऽग्निराग्नी᳚द्ध्रे॒ बृह॒स्पति॒राग्नी᳚द्ध्रात् प्रणी॒यमा॑न॒ इन्द्रो॑ हवि॒र्द्धाने ऽदि॑ति॒रासा॑दितो॒ विष्णु॑रुपावह्रि॒यमा॒णो ऽथ॒र्वोपो᳚त्तो य॒मो॑ऽभिषु॑तो ऽपूत॒पा आ॑धू॒यमा॑नो वा॒युः पू॒यमा॑नो मि॒त्रः क्षी॑र॒श्रीर्म॒न्थी स॑क्तु॒श्रीर्वै᳚श्वदे॒व उन्नी॑तो रु॒द्र ( ) आहु॑तो वा॒युरावृ॑त्तो नृ॒चक्षाः॒ प्रति॑ख्यातो भ॒क्ष आग॑तः पितृ॒णां ना॑राशꣳ॒॒सो ऽसु॒रात्तः॒ सिन्धु॑र-वभृ॒थम॑वप्र॒यन्थ् स॑मु॒द्रो ऽव॑गतः सलि॒लः प्रप्लु॑तः॒ सुव॑रु॒दृचं॑ ग॒तः ॥ \newline

\textbf{Pada Paata} \newline

प्र॒जाप॑ति॒रिति॑ प्र॒जा - प॒तिः॒ । मन॑सा । अन्धः॑ । अच्छे॑त॒ इत्यच्छ॑ - इ॒तः॒ । धा॒ता । दी॒क्षाया᳚म् । स॒वि॒ता । भृ॒त्याम् । पू॒षा । सो॒म॒क्रय॑ण्या॒मिति॑ सोम - क्रय॑ण्याम् । वरु॑णः । उप॑नद्ध॒ इत्युप॑ - न॒द्धः॒ । असु॑रः । क्री॒यमा॑णः । मि॒त्रः । क्री॒तः । शि॒पि॒वि॒ष्ट इति॑ शिपि - वि॒ष्टः । आसा॑दित॒ इत्या - सा॒दि॒तः॒ । न॒रंधि॑षः । प्रो॒ह्यमा॑ण॒ इति॑ प्र - उ॒ह्यमा॑णः । अधि॑पति॒रित्यधि॑-प॒तिः॒ । आग॑त॒ इत्या - ग॒तः॒ । प्र॒जाप॑ति॒रिति॑ प्र॒जा - प॒तिः॒ । प्र॒णी॒यमा॑न॒ इति॑ प्र-नी॒यमा॑नः । अ॒ग्निः । आग्नी᳚द्ध्र॒ इत्याग्नि॑-इ॒द्ध्रे॒ । बृह॒स्पतिः॑ । आग्नी᳚द्ध्रा॒दित्याग्नि॑ - इ॒द्ध्रा॒त् । प्र॒णी॒यमा॑न॒ इति॑ प्र - नी॒यमा॑नः । इन्द्रः॑ । ह॒वि॒द्‌र्धान॒ इति॑ हविः - धाने᳚ । अदि॑तिः । आसा॑दित॒ इत्या - सा॒दि॒तः॒ । विष्णुः॑ । उ॒पा॒व॒ह्रि॒यमा॑ण॒ इत्यु॑प - अ॒व॒ह्रि॒यमा॑णः । अथ॑र्वा । उपो᳚त्त॒ इत्युप॑ - उ॒त्तः॒ । य॒मः । अ॒भिषु॑त॒ इत्य॒भि - सु॒तः॒ । अ॒पू॒त॒पा इत्य॑पूत - पाः । आ॒धू॒यमा॑न॒ इत्या᳚ - धू॒यमा॑नः । वा॒युः । पू॒यमा॑नः । मि॒त्रः । क्षी॒र॒श्रीरिति॑ क्षीर-श्रीः । म॒न्थी । स॒क्तु॒श्रीरिति॑ सक्तु - श्रीः । वै॒श्व॒दे॒व इति॑ वैश्व - दे॒वः । उन्नी॑त॒ इत्युत् - नी॒तः॒ । रु॒द्रः ( ) । आहु॑त॒ इत्या - हु॒तः॒ । वा॒युः । आवृ॑त्त॒ इत्या - वृ॒त्तः॒ । नृ॒चक्षा॒ इति॑ नृ - चक्षाः᳚ । प्रति॑ख्यात॒ इति॒ प्रति॑ - ख्या॒तः॒ । भ॒क्षः । आग॑त॒ इत्या - ग॒तः॒ । पि॒तृ॒णाम् । ना॒रा॒शꣳ॒॒सः । असुः॑ । आत्तः॑ । सिन्धुः॑ । अ॒व॒भृ॒थमित्य॑व - भृ॒थम् । अ॒व॒प्र॒यन्नित्य॑व - प्र॒यन्न् । स॒मु॒द्रः । अव॑गत॒ इत्यव॑ - ग॒तः॒ । स॒लि॒लः । प्रप्लु॑त॒ इति॒ प्र - प्लु॒तः॒ । सुवः॑ । उ॒दृच॒मित्यु॑त् - ऋच᳚म् । ग॒तः ॥  \newline


\textbf{Krama Paata} \newline

प्र॒जाप॑ति॒र् मन॑सा । प्र॒जाप॑ति॒रिति॑ प्र॒जा - प॒तिः॒ । मन॒साऽन्धः॑ । अन्धोऽच्छे॑तः । अच्छे॑तो धा॒ता । अच्छे॑त॒ इत्यच्छ॑ - इ॒तः॒ । धा॒ता दी॒क्षाया᳚म् । दी॒क्षायाꣳ॑ सवि॒ता । स॒वि॒ता भृ॒त्याम् । भृ॒त्याम् पू॒षा । पू॒षा सो॑म॒क्रय॑ण्याम् । सो॒म॒क्रय॑ण्यां॒ ॅवरु॑णः । सो॒म॒क्रय॑ण्या॒मिति॑ सोम - क्रय॑ण्याम् । वरु॑ण॒ उप॑नद्धः । उप॑न॒द्धोऽसु॑रः । उप॑नद्ध॒ इत्युप॑ - न॒द्धः॒ । असु॑रः क्री॒यमा॑णः । क्री॒यमा॑णो मि॒त्रः । मि॒त्रः क्री॒तः । क्री॒तः शि॑पिवि॒ष्टः । शि॒पि॒वि॒ष्ट आसा॑दितः । शि॒पि॒वि॒ष्ट इति॑ शिपि - वि॒ष्टः । आसा॑दितो न॒रन्धि॑षः । आसा॑दित॒ इत्या - सा॒दि॒तः॒ । न॒रन्धि॑षः प्रो॒ह्यमा॑णः । प्रो॒ह्यमा॒णोऽधि॑पतिः । प्रो॒ह्यमा॑ण॒ इति॑ प्र - उ॒ह्यमा॑नः । अधि॑पति॒राग॑तः । अधि॑पति॒रित्यधि॑ - प॒तिः॒ । आग॑तः प्र॒जाप॑तिः । आग॑त॒ इत्या - ग॒तः॒ । प्र॒जाप॑तिः प्रणी॒यमा॑नः । प्र॒जाप॑ति॒रिति॑ प्र॒जा - प॒तिः॒ । प्र॒णी॒यमा॑नो॒ऽग्निः । प्र॒णी॒यमा॑न॒ इति॑ प्र - नी॒यमा॑नः । अ॒ग्निराग्नी᳚द्ध्रे । आग्नी᳚द्ध्रे॒ बृह॒स्पतिः॑ । आग्नी᳚द्ध्र॒ इत्याग्नि॑ - इ॒द्ध्रे॒ । बृह॒स्पति॒राग्नी᳚द्ध्रात् । आग्नी᳚द्ध्रात् प्रणी॒यमा॑नः । आग्नी᳚द्ध्रा॒दित्याग्नि॑ - इ॒द्ध्रा॒त्॒ । प्र॒णी॒यमा॑न॒ इन्द्रः॑ । प्र॒णी॒यमा॑न॒ इति॑ प्र - नी॒यमा॑नः । इन्द्रो॑ हवि॒र्द्धाने᳚ । ह॒वि॒र्द्धानेऽदि॑तिः । ह॒वि॒र्द्धान॒ इति॑ हविः - धाने᳚ । अदि॑ति॒रासा॑दितः । आसा॑दितो॒ विष्णुः॑ । आसा॑दित॒ इत्या - सा॒दि॒तः॒ । विष्णु॑रुपावह्रि॒यमा॑णः । उ॒पा॒व॒ह्रि॒यमा॒णोऽथ॑र्वा । उ॒पा॒व॒ह्रि॒यमा॑ण॒ इत्यु॑प - अ॒व॒ह्रि॒यमा॑णः । अथ॒र्वोपो᳚त्तः । उपो᳚त्तो य॒मः । उपो᳚त्त॒ इत्युप॑ - उ॒त्तः॒ । य॒मो॑ऽभिषु॑तः । अ॒भिषु॑तोऽपूत॒पाः । अ॒भि॑षुत॒ इत्य॒भि - सु॒तः॒ । अ॒पू॒त॒पा आ॑धू॒यमा॑नः । अ॒पू॒त॒पा इत्य॑पूत - पाः । आ॒धू॒यमा॑नो वा॒युः । आ॒धू॒यमा॑न॒ इत्या᳚ - धू॒यमा॑नः । वा॒युः पू॒यमा॑नः । पू॒यमा॑नो मि॒त्रः । मि॒त्रः क्षी॑र॒श्रीः । क्षी॒र॒श्रीर् म॒न्थी । क्षी॒र॒श्रीरिति॑ क्षीर - श्रीः । म॒न्थी स॑क्तु॒श्रीः । स॒क्तु॒श्रीर् वै᳚श्वदे॒वः । स॒क्तु॒श्रीरिति॑ सक्तु - श्रीः । वै॒श्व॒दे॒व उन्नी॑तः । वै॒श्व॒दे॒व इति॑ वैश्व - दे॒वः । उन्नी॑तो रु॒द्रः ( ) । उन्नी॑त॒ इत्युत् - नी॒तः॒ । रु॒द्र आहु॑तः । आहु॑तो वा॒युः । आहु॑त॒ इत्या - हु॒तः॒ । वा॒युरावृ॑त्तः । आवृ॑त्तो नृ॒चक्षाः᳚ । आवृ॑त्त॒ इत्या - वृ॒त्तः॒ । नृ॒चक्षाः॒ प्रति॑ख्यातः । नृ॒चक्षा॒ इति॑ नृ - चक्षाः᳚ । प्रति॑ख्यातो भ॒क्षः । प्रति॑ख्यात॒ इति॒ प्रति॑ - ख्या॒तः॒ । भ॒क्ष आग॑तः । आग॑तः पितृ॒णाम् । आग॑त॒ इत्या - ग॒तः॒ । पि॒तृ॒णाम् ना॑राशꣳ॒॒सः । ना॒रा॒शꣳ॒॒सोऽसुः॑ । असु॒रात्तः॑ । आत्तः॒ सिन्धुः॑ । सिन्धु॑रवभृ॒थम् । अ॒व॒भृ॒थम॑वप्र॒यन्न् । अ॒व॒भृ॒थमित्य॑व - भृ॒थम् । अ॒व॒प्र॒यन्थ् स॑मु॒द्रः । अ॒व॒प्र॒यन्नित्य॑व - प्र॒यन्न् । स॒मु॒द्रोऽव॑गतः । अव॑गतः सलि॒लः । अव॑गत॒ इत्यव॑ - ग॒तः॒ । स॒लि॒लः प्रप्लु॑तः । प्रप्लु॑तः॒ सुवः॑ । प्रप्लु॑त॒ इति॒ प्र - प्लु॒तः॒ । सुव॑रु॒दृच᳚म् । उ॒दृच॑म् ग॒तः । उ॒दृच॒मित्यु॑त् - ऋच᳚म् । ग॒त इति॑ ग॒तः । \newline

\textbf{Jatai Paata} \newline

1. प्र॒जाप॑ति॒र् मन॑सा॒ मन॑सा प्र॒जाप॑तिः प्र॒जाप॑ति॒र् मन॑सा । \newline
2. प्र॒जाप॑ति॒रिति॑ प्र॒जा - प॒तिः॒ । \newline
3. मन॒सा ऽन्धो ऽन्धो॒ मन॑सा॒ मन॒सा ऽन्धः॑ । \newline
4. अन्धो ऽच्छे॒तो ऽच्छे॒तो ऽन्धो ऽन्धो ऽच्छे॑तः । \newline
5. अच्छे॑तो धा॒ता धा॒ता ऽच्छे॒तो ऽच्छे॑तो धा॒ता । \newline
6. अच्छे॑त॒ इत्यच्छ॑ - इ॒तः॒ । \newline
7. धा॒ता दी॒क्षाया᳚म् दी॒क्षाया᳚म् धा॒ता धा॒ता दी॒क्षाया᳚म् । \newline
8. दी॒क्षायाꣳ॑ सवि॒ता स॑वि॒ता दी॒क्षाया᳚म् दी॒क्षायाꣳ॑ सवि॒ता । \newline
9. स॒वि॒ता भृ॒त्याम् भृ॒त्याꣳ स॑वि॒ता स॑वि॒ता भृ॒त्याम् । \newline
10. भृ॒त्याम् पू॒षा पू॒षा भृ॒त्याम् भृ॒त्याम् पू॒षा । \newline
11. पू॒षा सो॑म॒क्रय॑ण्याꣳ सोम॒क्रय॑ण्याम् पू॒षा पू॒षा सो॑म॒क्रय॑ण्याम् । \newline
12. सो॒म॒क्रय॑ण्यां॒ ॅवरु॑णो॒ वरु॑णः सोम॒क्रय॑ण्याꣳ सोम॒क्रय॑ण्यां॒ ॅवरु॑णः । \newline
13. सो॒म॒क्रय॑ण्या॒मिति॑ सोम - क्रय॑ण्याम् । \newline
14. वरु॑ण॒ उप॑नद्ध॒ उप॑नद्धो॒ वरु॑णो॒ वरु॑ण॒ उप॑नद्धः । \newline
15. उप॑न॒द्धो ऽसु॒रो ऽसु॑र॒ उप॑नद्ध॒ उप॑न॒द्धो ऽसु॑रः । \newline
16. उप॑नद्ध॒ इत्युप॑ - न॒द्धः॒ । \newline
17. असु॑रः क्री॒यमा॑णः क्री॒यमा॒णो ऽसु॒रो ऽसु॑रः क्री॒यमा॑णः । \newline
18. क्री॒यमा॑णो मि॒त्रो मि॒त्रः क्री॒यमा॑णः क्री॒यमा॑णो मि॒त्रः । \newline
19. मि॒त्रः क्री॒तः क्री॒तो मि॒त्रो मि॒त्रः क्री॒तः । \newline
20. क्री॒तः शि॑पिवि॒ष्टः शि॑पिवि॒ष्टः क्री॒तः क्री॒तः शि॑पिवि॒ष्टः । \newline
21. शि॒पि॒वि॒ष्ट आसा॑दित॒ आसा॑दितः शिपिवि॒ष्टः शि॑पिवि॒ष्ट आसा॑दितः । \newline
22. शि॒पि॒वि॒ष्ट इति॑ शिपि - वि॒ष्टः । \newline
23. आसा॑दितो न॒रन्धि॑षो न॒रन्धि॑ष॒ आसा॑दित॒ आसा॑दितो न॒रन्धि॑षः । \newline
24. आसा॑दित॒ इत्या - सा॒दि॒तः॒ । \newline
25. न॒रन्धि॑षः प्रो॒ह्यमा॑णः प्रो॒ह्यमा॑णो न॒रन्धि॑षो न॒रन्धि॑षः प्रो॒ह्यमा॑णः । \newline
26. प्रो॒ह्यमा॒णो ऽधि॑पति॒ रधि॑पतिः प्रो॒ह्यमा॑णः प्रो॒ह्यमा॒णो ऽधि॑पतिः । \newline
27. प्रो॒ह्यमा॑ण॒ इति॑ प्र - उ॒ह्यमा॑नः । \newline
28. अधि॑पति॒ राग॑त॒ आग॒तो ऽधि॑पति॒ रधि॑पति॒ राग॑तः । \newline
29. अधि॑पति॒रित्यधि॑ - प॒तिः॒ । \newline
30. आग॑तः प्र॒जाप॑तिः प्र॒जाप॑ति॒ राग॑त॒ आग॑तः प्र॒जाप॑तिः । \newline
31. आग॑त॒ इत्या - ग॒तः॒ । \newline
32. प्र॒जाप॑तिः प्रणी॒यमा॑नः प्रणी॒यमा॑नः प्र॒जाप॑तिः प्र॒जाप॑तिः प्रणी॒यमा॑नः । \newline
33. प्र॒जाप॑ति॒रिति॑ प्र॒जा - प॒तिः॒ । \newline
34. प्र॒णी॒यमा॑नो॒ ऽग्नि र॒ग्निः प्र॑णी॒यमा॑नः प्रणी॒यमा॑नो॒ ऽग्निः । \newline
35. प्र॒णी॒यमा॑न॒ इति॑ प्र - नी॒यमा॑नः । \newline
36. अ॒ग्नि राग्नी᳚द्ध्र॒ आग्नी᳚द्ध्रे॒ ऽग्नि र॒ग्नि राग्नी᳚द्ध्रे । \newline
37. आग्नी᳚द्ध्रे॒ बृह॒स्पति॒र् बृह॒स्पति॒ राग्नी᳚द्ध्र॒ आग्नी᳚द्ध्रे॒ बृह॒स्पतिः॑ । \newline
38. आग्नी᳚द्ध्र॒ इत्याग्नि॑ - इ॒द्ध्रे॒ । \newline
39. बृह॒स्पति॒ राग्नी᳚द्ध्रा॒ दाग्नी᳚द्ध्रा॒द् बृह॒स्पति॒र् बृह॒स्पति॒ राग्नी᳚द्ध्रात् । \newline
40. आग्नी᳚द्ध्रात् प्रणी॒यमा॑नः प्रणी॒यमा॑न॒ आग्नी᳚द्ध्रा॒ दाग्नी᳚द्ध्रात् प्रणी॒यमा॑नः । \newline
41. आग्नी᳚द्ध्रा॒दित्याग्नि॑ - इ॒द्ध्रा॒त् । \newline
42. प्र॒णी॒यमा॑न॒ इन्द्र॒ इन्द्रः॑ प्रणी॒यमा॑नः प्रणी॒यमा॑न॒ इन्द्रः॑ । \newline
43. प्र॒णी॒यमा॑न॒ इति॑ प्र - नी॒यमा॑नः । \newline
44. इन्द्रो॑ हवि॒र्द्धाने॑ हवि॒र्द्धान॒ इन्द्र॒ इन्द्रो॑ हवि॒र्द्धाने᳚ । \newline
45. ह॒वि॒र्द्धाने ऽदि॑ति॒ रदि॑तिर्. हवि॒र्द्धाने॑ हवि॒र्द्धाने ऽदि॑तिः । \newline
46. ह॒वि॒र्द्धान॒ इति॑ हविः - धाने᳚ । \newline
47. अदि॑ति॒ रासा॑दित॒ आसा॑दि॒तो ऽदि॑ति॒ रदि॑ति॒ रासा॑दितः । \newline
48. आसा॑दितो॒ विष्णु॒र् विष्णु॒ रासा॑दित॒ आसा॑दितो॒ विष्णुः॑ । \newline
49. आसा॑दित॒ इत्या - सा॒दि॒तः॒ । \newline
50. विष्णु॑ रुपावह्रि॒यमा॑ण उपावह्रि॒यमा॑णो॒ विष्णु॒र् विष्णु॑ रुपावह्रि॒यमा॑णः । \newline
51. उ॒पा॒व॒ह्रि॒यमा॒णो ऽथ॒र्वा ऽथ॑र् वोपावह्रि॒यमा॑ण उपावह्रि॒यमा॒णो ऽथ॑र्वा । \newline
52. उ॒पा॒व॒ह्रि॒यमा॑ण॒ इत्यु॑प - अ॒व॒ह्रि॒यमा॑णः । \newline
53. अथ॒र्वो पो᳚त्त॒ उपो॒त्तो ऽथ॒र्वा ऽथ॒र्वो पो᳚त्तः । \newline
54. उपो᳚त्तो य॒मो य॒म उपो᳚त्त॒ उपो᳚त्तो य॒मः । \newline
55. उपो᳚त्त॒ इत्युप॑ - उ॒त्तः॒ । \newline
56. य॒मो॑ ऽभिषु॑तो॒ ऽभिषु॑तो य॒मो य॒मो॑ ऽभिषु॑तः । \newline
57. अ॒भिषु॑तो ऽपूत॒पा अ॑पूत॒पा अ॒भिषु॑तो॒ ऽभिषु॑तो ऽपूत॒पाः । \newline
58. अ॒भिषु॑त॒ इत्य॒भि - सु॒तः॒ । \newline
59. अ॒पू॒त॒पा आ॑धू॒यमा॑न आधू॒यमा॑नो ऽपूत॒पा अ॑पूत॒पा आ॑धू॒यमा॑नः । \newline
60. अ॒पू॒त॒पा इत्य॑पूत - पाः । \newline
61. आ॒धू॒यमा॑नो वा॒युर् वा॒यु रा॑धू॒यमा॑न आधू॒यमा॑नो वा॒युः । \newline
62. आ॒धू॒यमा॑न॒ इत्या᳚ - धू॒यमा॑नः । \newline
63. वा॒युः पू॒यमा॑नः पू॒यमा॑नो वा॒युर् वा॒युः पू॒यमा॑नः । \newline
64. पू॒यमा॑नो मि॒त्रो मि॒त्रः पू॒यमा॑नः पू॒यमा॑नो मि॒त्रः । \newline
65. मि॒त्रः क्षी॑र॒श्रीः क्षी॑र॒श्रीर् मि॒त्रो मि॒त्रः क्षी॑र॒श्रीः । \newline
66. क्षी॒र॒श्रीर् म॒न्थी म॒न्थी क्षी॑र॒श्रीः क्षी॑र॒श्रीर् म॒न्थी । \newline
67. क्षी॒र॒श्रीरिति॑ क्षीर - श्रीः । \newline
68. म॒न्थी स॑क्तु॒श्रीः स॑क्तु॒श्रीर् म॒न्थी म॒न्थी स॑क्तु॒श्रीः । \newline
69. स॒क्तु॒श्रीर् वै᳚श्वदे॒वो वै᳚श्वदे॒वः स॑क्तु॒श्रीः स॑क्तु॒श्रीर् वै᳚श्वदे॒वः । \newline
70. स॒क्तु॒श्रीरिति॑ सक्तु - श्रीः । \newline
71. वै॒श्व॒दे॒व उन्नी॑त॒ उन्नी॑तो वैश्वदे॒वो वै᳚श्वदे॒व उन्नी॑तः । \newline
72. वै॒श्व॒दे॒व इति॑ वैश्व - दे॒वः । \newline
73. उन्नी॑तो रु॒द्रो रु॒द्र उन्नी॑त॒ उन्नी॑तो रु॒द्रः । \newline
74. उन्नी॑त॒ इत्युत् - नी॒तः॒ । \newline
75. रु॒द्र आहु॑त॒ आहु॑तो रु॒द्रो रु॒द्र आहु॑तः । \newline
76. आहु॑तो वा॒युर् वा॒यु राहु॑त॒ आहु॑तो वा॒युः । \newline
77. आहु॑त॒ इत्या - हु॒तः॒ । \newline
78. वा॒यु रावृ॑त्त॒ आवृ॑त्तो वा॒युर् वा॒यु रावृ॑त्तः । \newline
79. आवृ॑त्तो नृ॒चक्षा॑ नृ॒चक्षा॒ आवृ॑त्त॒ आवृ॑त्तो नृ॒चक्षाः᳚ । \newline
80. आवृ॑त्त॒ इत्या - वृ॒त्तः॒ । \newline
81. नृ॒चक्षाः॒ प्रति॑ख्यातः॒ प्रति॑ख्यातो नृ॒चक्षा॑ नृ॒चक्षाः॒ प्रति॑ख्यातः । \newline
82. नृ॒चक्षा॒ इति॑ नृ - चक्षाः᳚ । \newline
83. प्रति॑ख्यातो भ॒क्षो भ॒क्षः प्रति॑ख्यातः॒ प्रति॑ख्यातो भ॒क्षः । \newline
84. प्रति॑ख्यात॒ इति॒ प्रति॑ - ख्या॒तः॒ । \newline
85. भ॒क्ष आग॑त॒ आग॑तो भ॒क्षो भ॒क्ष आग॑तः । \newline
86. आग॑तः पितृ॒णाम् पि॑तृ॒णा माग॑त॒ आग॑तः पितृ॒णाम् । \newline
87. आग॑त॒ इत्या - ग॒तः॒ । \newline
88. पि॒तृ॒णान् ना॑राशꣳ॒॒सो ना॑राशꣳ॒॒सः पि॑तृ॒णाम् पि॑तृ॒णान् ना॑राशꣳ॒॒सः । \newline
89. ना॒रा॒शꣳ॒॒सो ऽसु॒ रसु॑र् नाराशꣳ॒॒सो ना॑राशꣳ॒॒सो ऽसुः॑ । \newline
90. असु॒ रात्त॒ आत्तो ऽसु॒ रसु॒ रात्तः॑ । \newline
91. आत्तः॒ सिन्धुः॒ सिन्धु॒ रात्त॒ आत्तः॒ सिन्धुः॑ । \newline
92. सिन्धु॑ रवभृ॒थ म॑वभृ॒थꣳ सिन्धुः॒ सिन्धु॑ रवभृ॒थम् । \newline
93. अ॒व॒भृ॒थ म॑वप्र॒यन्-न॑वप्र॒यन्-न॑वभृ॒थ म॑वभृ॒थ म॑वप्र॒यन्न् । \newline
94. अ॒व॒भृ॒थमित्य॑व - भृ॒थम् । \newline
95. अ॒व॒प्र॒यन् थ्स॑मु॒द्रः स॑मु॒द्रो॑ ऽवप्र॒यन्-न॑वप्र॒यन् थ्स॑मु॒द्रः । \newline
96. अ॒व॒प्र॒यन्नित्य॑व - प्र॒यन्न् । \newline
97. स॒मु॒द्रो ऽव॑ग॒तो ऽव॑गतः समु॒द्रः स॑मु॒द्रो ऽव॑गतः । \newline
98. अव॑गतः सलि॒लः स॑लि॒लो ऽव॑ग॒तो ऽव॑गतः सलि॒लः । \newline
99. अव॑गत॒ इत्यव॑ - ग॒तः॒ । \newline
100. स॒लि॒लः प्रप्लु॑तः॒ प्रप्लु॑तः सलि॒लः स॑लि॒लः प्रप्लु॑तः । \newline
101. प्रप्लु॑तः॒ सुवः॒ सुवः॒ प्रप्लु॑तः॒ प्रप्लु॑तः॒ सुवः॑ । \newline
102. प्रप्लु॑त॒ इति॒ प्र - प्लु॒तः॒ । \newline
103. सुव॑ रु॒दृच॑ मु॒दृचꣳ॒॒ सुवः॒ सुव॑ रु॒दृच᳚म् । \newline
104. उ॒दृच॑म् ग॒तो ग॒त उ॒दृच॑ मु॒दृच॑म् ग॒तः । \newline
105. उ॒दृच॒मित्यु॑त् - ऋच᳚म् । \newline
106. ग॒त इति॑ ग॒तः । \newline

\textbf{Ghana Paata } \newline

1. प्र॒जाप॑ति॒र् मन॑सा॒ मन॑सा प्र॒जाप॑तिः प्र॒जाप॑ति॒र् मन॒सा ऽन्धो ऽन्धो॒ मन॑सा प्र॒जाप॑तिः प्र॒जाप॑ति॒र् मन॒सा ऽन्धः॑ । \newline
2. प्र॒जाप॑ति॒रिति॑ प्र॒जा - प॒तिः॒ । \newline
3. मन॒सा ऽन्धो ऽन्धो॒ मन॑सा॒ मन॒सा ऽन्धो ऽच्छे॒तो ऽच्छे॒तो ऽन्धो॒ मन॑सा॒ मन॒सा ऽन्धो ऽच्छे॑तः । \newline
4. अन्धो ऽच्छे॒तो ऽच्छे॒तो ऽन्धो ऽन्धो ऽच्छे॑तो धा॒ता धा॒ता ऽच्छे॒तो ऽन्धो ऽन्धो ऽच्छे॑तो धा॒ता । \newline
5. अच्छे॑तो धा॒ता धा॒ता ऽच्छे॒तो ऽच्छे॑तो धा॒ता दी॒क्षाया᳚म् दी॒क्षाया᳚म् धा॒ता ऽच्छे॒तो ऽच्छे॑तो धा॒ता दी॒क्षाया᳚म् । \newline
6. अच्छे॑त॒ इत्यच्छ॑ - इ॒तः॒ । \newline
7. धा॒ता दी॒क्षाया᳚म् दी॒क्षाया᳚म् धा॒ता धा॒ता दी॒क्षायाꣳ॑ सवि॒ता स॑वि॒ता दी॒क्षाया᳚म् धा॒ता धा॒ता दी॒क्षायाꣳ॑ सवि॒ता । \newline
8. दी॒क्षायाꣳ॑ सवि॒ता स॑वि॒ता दी॒क्षाया᳚म् दी॒क्षायाꣳ॑ सवि॒ता भृ॒त्याम् भृ॒त्याꣳ स॑वि॒ता दी॒क्षाया᳚म् दी॒क्षायाꣳ॑ सवि॒ता भृ॒त्याम् । \newline
9. स॒वि॒ता भृ॒त्याम् भृ॒त्याꣳ स॑वि॒ता स॑वि॒ता भृ॒त्याम् पू॒षा पू॒षा भृ॒त्याꣳ स॑वि॒ता स॑वि॒ता भृ॒त्याम् पू॒षा । \newline
10. भृ॒त्याम् पू॒षा पू॒षा भृ॒त्याम् भृ॒त्याम् पू॒षा सो॑म॒क्रय॑ण्याꣳ सोम॒क्रय॑ण्याम् पू॒षा भृ॒त्याम् भृ॒त्याम् पू॒षा सो॑म॒क्रय॑ण्याम् । \newline
11. पू॒षा सो॑म॒क्रय॑ण्याꣳ सोम॒क्रय॑ण्याम् पू॒षा पू॒षा सो॑म॒क्रय॑ण्यां॒ ॅवरु॑णो॒ वरु॑णः सोम॒क्रय॑ण्याम् पू॒षा पू॒षा सो॑म॒क्रय॑ण्यां॒ ॅवरु॑णः । \newline
12. सो॒म॒क्रय॑ण्यां॒ ॅवरु॑णो॒ वरु॑णः सोम॒क्रय॑ण्याꣳ सोम॒क्रय॑ण्यां॒ ॅवरु॑ण॒ उप॑नद्ध॒ उप॑नद्धो॒ वरु॑णः सोम॒क्रय॑ण्याꣳ सोम॒क्रय॑ण्यां॒ ॅवरु॑ण॒ उप॑नद्धः । \newline
13. सो॒म॒क्रय॑ण्या॒मिति॑ सोम - क्रय॑ण्याम् । \newline
14. वरु॑ण॒ उप॑नद्ध॒ उप॑नद्धो॒ वरु॑णो॒ वरु॑ण॒ उप॑न॒द्धो ऽसु॒रो ऽसु॑र॒ उप॑नद्धो॒ वरु॑णो॒ वरु॑ण॒ उप॑न॒द्धो ऽसु॑रः । \newline
15. उप॑न॒द्धो ऽसु॒रो ऽसु॑र॒ उप॑नद्ध॒ उप॑न॒द्धो ऽसु॑रः क्री॒यमा॑णः क्री॒यमा॒णो ऽसु॑र॒ उप॑नद्ध॒ उप॑न॒द्धो ऽसु॑रः क्री॒यमा॑णः । \newline
16. उप॑नद्ध॒ इत्युप॑ - न॒द्धः॒ । \newline
17. असु॑रः क्री॒यमा॑णः क्री॒यमा॒णो ऽसु॒रो ऽसु॑रः क्री॒यमा॑णो मि॒त्रो मि॒त्रः क्री॒यमा॒णो ऽसु॒रो ऽसु॑रः क्री॒यमा॑णो मि॒त्रः । \newline
18. क्री॒यमा॑णो मि॒त्रो मि॒त्रः क्री॒यमा॑णः क्री॒यमा॑णो मि॒त्रः क्री॒तः क्री॒तो मि॒त्रः क्री॒यमा॑णः क्री॒यमा॑णो मि॒त्रः क्री॒तः । \newline
19. मि॒त्रः क्री॒तः क्री॒तो मि॒त्रो मि॒त्रः क्री॒तः शि॑पिवि॒ष्टः शि॑पिवि॒ष्टः क्री॒तो मि॒त्रो मि॒त्रः क्री॒तः शि॑पिवि॒ष्टः । \newline
20. क्री॒तः शि॑पिवि॒ष्टः शि॑पिवि॒ष्टः क्री॒तः क्री॒तः शि॑पिवि॒ष्ट आसा॑दित॒ आसा॑दितः शिपिवि॒ष्टः क्री॒तः क्री॒तः शि॑पिवि॒ष्ट आसा॑दितः । \newline
21. शि॒पि॒वि॒ष्ट आसा॑दित॒ आसा॑दितः शिपिवि॒ष्टः शि॑पिवि॒ष्ट आसा॑दितो न॒रन्धि॑षो न॒रन्धि॑ष॒ आसा॑दितः शिपिवि॒ष्टः शि॑पिवि॒ष्ट आसा॑दितो न॒रन्धि॑षः । \newline
22. शि॒पि॒वि॒ष्ट इति॑ शिपि - वि॒ष्टः । \newline
23. आसा॑दितो न॒रन्धि॑षो न॒रन्धि॑ष॒ आसा॑दित॒ आसा॑दितो न॒रन्धि॑षः प्रो॒ह्यमा॑णः प्रो॒ह्यमा॑णो न॒रन्धि॑ष॒ आसा॑दित॒ आसा॑दितो न॒रन्धि॑षः प्रो॒ह्यमा॑णः । \newline
24. आसा॑दित॒ इत्या - सा॒दि॒तः॒ । \newline
25. न॒रन्धि॑षः प्रो॒ह्यमा॑णः प्रो॒ह्यमा॑णो न॒रन्धि॑षो न॒रन्धि॑षः प्रो॒ह्यमा॒णो ऽधि॑पति॒ रधि॑पतिः प्रो॒ह्यमा॑णो न॒रन्धि॑षो न॒रन्धि॑षः प्रो॒ह्यमा॒णो ऽधि॑पतिः । \newline
26. प्रो॒ह्यमा॒णो ऽधि॑पति॒ रधि॑पतिः प्रो॒ह्यमा॑णः प्रो॒ह्यमा॒णो ऽधि॑पति॒ राग॑त॒ आग॒तो ऽधि॑पतिः प्रो॒ह्यमा॑णः प्रो॒ह्यमा॒णो ऽधि॑पति॒ राग॑तः । \newline
27. प्रो॒ह्यमा॑ण॒ इति॑ प्र - उ॒ह्यमा॑नः । \newline
28. अधि॑पति॒ राग॑त॒ आग॒तो ऽधि॑पति॒ रधि॑पति॒ राग॑तः प्र॒जाप॑तिः प्र॒जाप॑ति॒ राग॒तो ऽधि॑पति॒ रधि॑पति॒ राग॑तः प्र॒जाप॑तिः । \newline
29. अधि॑पति॒रित्यधि॑ - प॒तिः॒ । \newline
30. आग॑तः प्र॒जाप॑तिः प्र॒जाप॑ति॒ राग॑त॒ आग॑तः प्र॒जाप॑तिः प्रणी॒यमा॑नः प्रणी॒यमा॑नः प्र॒जाप॑ति॒ राग॑त॒ आग॑तः प्र॒जाप॑तिः प्रणी॒यमा॑नः । \newline
31. आग॑त॒ इत्या - ग॒तः॒ । \newline
32. प्र॒जाप॑तिः प्रणी॒यमा॑नः प्रणी॒यमा॑नः प्र॒जाप॑तिः प्र॒जाप॑तिः प्रणी॒यमा॑नो॒ ऽग्नि र॒ग्निः प्र॑णी॒यमा॑नः प्र॒जाप॑तिः प्र॒जाप॑तिः प्रणी॒यमा॑नो॒ ऽग्निः । \newline
33. प्र॒जाप॑ति॒रिति॑ प्र॒जा - प॒तिः॒ । \newline
34. प्र॒णी॒यमा॑नो॒ ऽग्नि र॒ग्निः प्र॑णी॒यमा॑नः प्रणी॒यमा॑नो॒ ऽग्नि राग्नी᳚द्ध्र॒ आग्नी᳚द्ध्रे॒ ऽग्निः प्र॑णी॒यमा॑नः प्रणी॒यमा॑नो॒ ऽग्नि राग्नी᳚द्ध्रे । \newline
35. प्र॒णी॒यमा॑न॒ इति॑ प्र - नी॒यमा॑नः । \newline
36. अ॒ग्नि राग्नी᳚द्ध्र॒ आग्नी᳚द्ध्रे॒ ऽग्नि र॒ग्नि राग्नी᳚द्ध्रे॒ बृह॒स्पति॒र् बृह॒स्पति॒ राग्नी᳚द्ध्रे॒ ऽग्नि र॒ग्नि राग्नी᳚द्ध्रे॒ बृह॒स्पतिः॑ । \newline
37. आग्नी᳚द्ध्रे॒ बृह॒स्पति॒र् बृह॒स्पति॒ राग्नी᳚द्ध्र॒ आग्नी᳚द्ध्रे॒ बृह॒स्पति॒ राग्नी᳚द्ध्रा॒ दाग्नी᳚द्ध्रा॒द् बृह॒स्पति॒ राग्नी᳚द्ध्र॒ आग्नी᳚द्ध्रे॒ बृह॒स्पति॒ राग्नी᳚द्ध्रात् । \newline
38. आग्नी᳚द्ध्र॒ इत्याग्नि॑ - इ॒द्ध्रे॒ । \newline
39. बृह॒स्पति॒ राग्नी᳚द्ध्रा॒ दाग्नी᳚द्ध्रा॒द् बृह॒स्पति॒र् बृह॒स्पति॒ राग्नी᳚द्ध्रात् प्रणी॒यमा॑नः प्रणी॒यमा॑न॒ आग्नी᳚द्ध्रा॒द् बृह॒स्पति॒र् बृह॒स्पति॒ राग्नी᳚द्ध्रात् प्रणी॒यमा॑नः । \newline
40. आग्नी᳚द्ध्रात् प्रणी॒यमा॑नः प्रणी॒यमा॑न॒ आग्नी᳚द्ध्रा॒ दाग्नी᳚द्ध्रात् प्रणी॒यमा॑न॒ इन्द्र॒ इन्द्रः॑ प्रणी॒यमा॑न॒ आग्नी᳚द्ध्रा॒ दाग्नी᳚द्ध्रात् प्रणी॒यमा॑न॒ इन्द्रः॑ । \newline
41. आग्नी᳚द्ध्रा॒दित्याग्नि॑ - इ॒द्ध्रा॒त् । \newline
42. प्र॒णी॒यमा॑न॒ इन्द्र॒ इन्द्रः॑ प्रणी॒यमा॑नः प्रणी॒यमा॑न॒ इन्द्रो॑ हवि॒र्द्धाने॑ हवि॒र्द्धान॒ इन्द्रः॑ प्रणी॒यमा॑नः प्रणी॒यमा॑न॒ इन्द्रो॑ हवि॒र्द्धाने᳚ । \newline
43. प्र॒णी॒यमा॑न॒ इति॑ प्र - नी॒यमा॑नः । \newline
44. इन्द्रो॑ हवि॒र्द्धाने॑ हवि॒र्द्धान॒ इन्द्र॒ इन्द्रो॑ हवि॒र्द्धाने ऽदि॑ति॒ रदि॑तिर्. हवि॒र्द्धान॒ इन्द्र॒ इन्द्रो॑ हवि॒र्द्धाने ऽदि॑तिः । \newline
45. ह॒वि॒र्द्धाने ऽदि॑ति॒ रदि॑तिर्. हवि॒र्द्धाने॑ हवि॒र्द्धाने ऽदि॑ति॒ रासा॑दित॒ आसा॑दि॒तो ऽदि॑तिर्. हवि॒र्द्धाने॑ हवि॒र्द्धाने ऽदि॑ति॒ रासा॑दितः । \newline
46. ह॒वि॒र्द्धान॒ इति॑ हविः - धाने᳚ । \newline
47. अदि॑ति॒ रासा॑दित॒ आसा॑दि॒तो ऽदि॑ति॒ रदि॑ति॒ रासा॑दितो॒ विष्णु॒र् विष्णु॒ रासा॑दि॒तो ऽदि॑ति॒ रदि॑ति॒ रासा॑दितो॒ विष्णुः॑ । \newline
48. आसा॑दितो॒ विष्णु॒र् विष्णु॒ रासा॑दित॒ आसा॑दितो॒ विष्णु॑ रुपावह्रि॒यमा॑ण उपावह्रि॒यमा॑णो॒ विष्णु॒ रासा॑दित॒ आसा॑दितो॒ विष्णु॑ रुपावह्रि॒यमा॑णः । \newline
49. आसा॑दित॒ इत्या - सा॒दि॒तः॒ । \newline
50. विष्णु॑ रुपावह्रि॒यमा॑ण उपावह्रि॒यमा॑णो॒ विष्णु॒र् विष्णु॑ रुपावह्रि॒यमा॒णो ऽथ॒र्वा ऽथ॑र्वो पावह्रि॒यमा॑णो॒ विष्णु॒र् विष्णु॑ रुपावह्रि॒यमा॒णो ऽथ॑र्वा । \newline
51. उ॒पा॒व॒ह्रि॒यमा॒णो ऽथ॒र्वा ऽथ॑र्वो पावह्रि॒यमा॑ण उपावह्रि॒यमा॒णो ऽथ॒र्वोपो᳚त्त॒ उपो॒त्तो ऽथ॑र्वोपावह्रि॒यमा॑ण उपावह्रि॒यमा॒णो ऽथ॒र्वोपो᳚त्तः । \newline
52. उ॒पा॒व॒ह्रि॒यमा॑ण॒ इत्यु॑प - अ॒व॒ह्रि॒यमा॑णः । \newline
53. अथ॒र्वो पो᳚त्त॒ उपो॒त्तो ऽथ॒र्वा ऽथ॒र्वो पो᳚त्तो य॒मो य॒म उपो॒त्तो ऽथ॒र्वा ऽथ॒र्वो पो᳚त्तो य॒मः । \newline
54. उपो᳚त्तो य॒मो य॒म उपो᳚त्त॒ उपो᳚त्तो य॒मो॑ ऽभिषु॑तो॒ ऽभिषु॑तो य॒म उपो᳚त्त॒ उपो᳚त्तो य॒मो॑ ऽभिषु॑तः । \newline
55. उपो᳚त्त॒ इत्युप॑ - उ॒त्तः॒ । \newline
56. य॒मो॑ ऽभिषु॑तो॒ ऽभिषु॑तो य॒मो य॒मो॑ ऽभिषु॑तो ऽपूत॒पा अ॑पूत॒पा अ॒भिषु॑तो य॒मो य॒मो॑ ऽभिषु॑तो ऽपूत॒पाः । \newline
57. अ॒भिषु॑तो ऽपूत॒पा अ॑पूत॒पा अ॒भिषु॑तो॒ ऽभिषु॑तो ऽपूत॒पा आ॑धू॒यमा॑न आधू॒यमा॑नो ऽपूत॒पा अ॒भिषु॑तो॒ ऽभिषु॑तो ऽपूत॒पा आ॑धू॒यमा॑नः । \newline
58. अ॒भिषु॑त॒ इत्य॒भि - सु॒तः॒ । \newline
59. अ॒पू॒त॒पा आ॑धू॒यमा॑न आधू॒यमा॑नो ऽपूत॒पा अ॑पूत॒पा आ॑धू॒यमा॑नो वा॒युर् वा॒यु रा॑धू॒यमा॑नो ऽपूत॒पा अ॑पूत॒पा आ॑धू॒यमा॑नो वा॒युः । \newline
60. अ॒पू॒त॒पा इत्य॑पूत - पाः । \newline
61. आ॒धू॒यमा॑नो वा॒युर् वा॒यु रा॑धू॒यमा॑न आधू॒यमा॑नो वा॒युः पू॒यमा॑नः पू॒यमा॑नो वा॒यु रा॑धू॒यमा॑न आधू॒यमा॑नो वा॒युः पू॒यमा॑नः । \newline
62. आ॒धू॒यमा॑न॒ इत्या᳚ - धू॒यमा॑नः । \newline
63. वा॒युः पू॒यमा॑नः पू॒यमा॑नो वा॒युर् वा॒युः पू॒यमा॑नो मि॒त्रो मि॒त्रः पू॒यमा॑नो वा॒युर् वा॒युः पू॒यमा॑नो मि॒त्रः । \newline
64. पू॒यमा॑नो मि॒त्रो मि॒त्रः पू॒यमा॑नः पू॒यमा॑नो मि॒त्रः क्षी॑र॒श्रीः क्षी॑र॒श्रीर् मि॒त्रः पू॒यमा॑नः पू॒यमा॑नो मि॒त्रः क्षी॑र॒श्रीः । \newline
65. मि॒त्रः क्षी॑र॒श्रीः क्षी॑र॒श्रीर् मि॒त्रो मि॒त्रः क्षी॑र॒श्रीर् म॒न्थी म॒न्थी क्षी॑र॒श्रीर् मि॒त्रो मि॒त्रः क्षी॑र॒श्रीर् म॒न्थी । \newline
66. क्षी॒र॒श्रीर् म॒न्थी म॒न्थी क्षी॑र॒श्रीः क्षी॑र॒श्रीर् म॒न्थी स॑क्तु॒श्रीः स॑क्तु॒श्रीर् म॒न्थी क्षी॑र॒श्रीः क्षी॑र॒श्रीर् म॒न्थी स॑क्तु॒श्रीः । \newline
67. क्षी॒र॒श्रीरिति॑ क्षीर - श्रीः । \newline
68. म॒न्थी स॑क्तु॒श्रीः स॑क्तु॒श्रीर् म॒न्थी म॒न्थी स॑क्तु॒श्रीर् वै᳚श्वदे॒वो वै᳚श्वदे॒वः स॑क्तु॒श्रीर् म॒न्थी म॒न्थी स॑क्तु॒श्रीर् वै᳚श्वदे॒वः । \newline
69. स॒क्तु॒श्रीर् वै᳚श्वदे॒वो वै᳚श्वदे॒वः स॑क्तु॒श्रीः स॑क्तु॒श्रीर् वै᳚श्वदे॒व उन्नी॑त॒ उन्नी॑तो वैश्वदे॒वः स॑क्तु॒श्रीः स॑क्तु॒श्रीर् वै᳚श्वदे॒व उन्नी॑तः । \newline
70. स॒क्तु॒श्रीरिति॑ सक्तु - श्रीः । \newline
71. वै॒श्व॒दे॒व उन्नी॑त॒ उन्नी॑तो वैश्वदे॒वो वै᳚श्वदे॒व उन्नी॑तो रु॒द्रो रु॒द्र उन्नी॑तो वैश्वदे॒वो वै᳚श्वदे॒व उन्नी॑तो रु॒द्रः । \newline
72. वै॒श्व॒दे॒व इति॑ वैश्व - दे॒वः । \newline
73. उन्नी॑तो रु॒द्रो रु॒द्र उन्नी॑त॒ उन्नी॑तो रु॒द्र आहु॑त॒ आहु॑तो रु॒द्र उन्नी॑त॒ उन्नी॑तो रु॒द्र आहु॑तः । \newline
74. उन्नी॑त॒ इत्युत् - नी॒तः॒ । \newline
75. रु॒द्र आहु॑त॒ आहु॑तो रु॒द्रो रु॒द्र आहु॑तो वा॒युर् वा॒यु राहु॑तो रु॒द्रो रु॒द्र आहु॑तो वा॒युः । \newline
76. आहु॑तो वा॒युर् वा॒यु राहु॑त॒ आहु॑तो वा॒यु रावृ॑त्त॒ आवृ॑त्तो वा॒यु राहु॑त॒ आहु॑तो वा॒यु रावृ॑त्तः । \newline
77. आहु॑त॒ इत्या - हु॒तः॒ । \newline
78. वा॒यु रावृ॑त्त॒ आवृ॑त्तो वा॒युर् वा॒यु रावृ॑त्तो नृ॒चक्षा॑ नृ॒चक्षा॒ आवृ॑त्तो वा॒युर् वा॒यु रावृ॑त्तो नृ॒चक्षाः᳚ । \newline
79. आवृ॑त्तो नृ॒चक्षा॑ नृ॒चक्षा॒ आवृ॑त्त॒ आवृ॑त्तो नृ॒चक्षाः॒ प्रति॑ख्यातः॒ प्रति॑ख्यातो नृ॒चक्षा॒ आवृ॑त्त॒ आवृ॑त्तो नृ॒चक्षाः॒ प्रति॑ख्यातः । \newline
80. आवृ॑त्त॒ इत्या - वृ॒त्तः॒ । \newline
81. नृ॒चक्षाः॒ प्रति॑ख्यातः॒ प्रति॑ख्यातो नृ॒चक्षा॑ नृ॒चक्षाः॒ प्रति॑ख्यातो भ॒क्षो भ॒क्षः प्रति॑ख्यातो 
नृ॒चक्षा॑ नृ॒चक्षाः॒ प्रति॑ख्यातो भ॒क्षः । \newline
82. नृ॒चक्षा॒ इति॑ नृ - चक्षाः᳚ । \newline
83. प्रति॑ख्यातो भ॒क्षो भ॒क्षः प्रति॑ख्यातः॒ प्रति॑ख्यातो भ॒क्ष आग॑त॒ आग॑तो भ॒क्षः प्रति॑ख्यातः॒ प्रति॑ख्यातो भ॒क्ष आग॑तः । \newline
84. प्रति॑ख्यात॒ इति॒ प्रति॑ - ख्या॒तः॒ । \newline
85. भ॒क्ष आग॑त॒ आग॑तो भ॒क्षो भ॒क्ष आग॑तः पितृ॒णाम् पि॑तृ॒णा माग॑तो भ॒क्षो भ॒क्ष आग॑तः पितृ॒णाम् । \newline
86. आग॑तः पितृ॒णाम् पि॑तृ॒णा माग॑त॒ आग॑तः पितृ॒णाम् ना॑राशꣳ॒॒सो ना॑राशꣳ॒॒सः पि॑तृ॒णा माग॑त॒ आग॑तः पितृ॒णाम् ना॑राशꣳ॒॒सः । \newline
87. आग॑त॒ इत्या - ग॒तः॒ । \newline
88. पि॒तृ॒णाम् ना॑राशꣳ॒॒सो ना॑राशꣳ॒॒सः पि॑तृ॒णाम् पि॑तृ॒णाम् ना॑राशꣳ॒॒सो ऽसु॒ रसु॑र् नाराशꣳ॒॒सः पि॑तृ॒णाम् पि॑तृ॒णाम् ना॑राशꣳ॒॒सो ऽसुः॑ । \newline
89. ना॒रा॒शꣳ॒॒सो ऽसु॒ रसु॑र् नाराशꣳ॒॒सो ना॑राशꣳ॒॒सो ऽसु॒ रात्त॒ आत्तो ऽसु॑र् नाराशꣳ॒॒सो ना॑राशꣳ॒॒सो ऽसु॒ रात्तः॑ । \newline
90. असु॒ रात्त॒ आत्तो ऽसु॒रसु॒ रात्तः॒ सिन्धुः॒ सिन्धु॒ रात्तो ऽसु॒ रसु॒ रात्तः॒ सिन्धुः॑ । \newline
91. आत्तः॒ सिन्धुः॒ सिन्धु॒ रात्त॒ आत्तः॒ सिन्धु॑ रवभृ॒थ म॑वभृ॒थꣳ सिन्धु॒ रात्त॒ आत्तः॒ सिन्धु॑ रवभृ॒थम् । \newline
92. सिन्धु॑ रवभृ॒थ म॑वभृ॒थꣳ सिन्धुः॒ सिन्धु॑ रवभृ॒थ म॑वप्र॒यन् न॑वप्र॒यन् न॑वभृ॒थꣳ सिन्धुः॒ सिन्धु॑ रवभृ॒थ म॑वप्र॒यन्न् । \newline
93. अ॒व॒भृ॒थ म॑वप्र॒यन् न॑वप्र॒यन् न॑वभृ॒थ म॑वभृ॒थ म॑वप्र॒यन् थ्स॑मु॒द्रः स॑मु॒द्रो॑ ऽवप्र॒यन् न॑वभृ॒थ म॑वभृ॒थ म॑वप्र॒यन् थ्स॑मु॒द्रः । \newline
94. अ॒व॒भृ॒थमित्य॑व - भृ॒थम् । \newline
95. अ॒व॒प्र॒यन् थ्स॑मु॒द्रः स॑मु॒द्रो॑ ऽवप्र॒यन् न॑वप्र॒यन् थ्स॑मु॒द्रो ऽव॑ग॒तो ऽव॑गतः समु॒द्रो॑ ऽवप्र॒यन् न॑वप्र॒यन् थ्स॑मु॒द्रो ऽव॑गतः । \newline
96. अ॒व॒प्र॒यन्नित्य॑व - प्र॒यन्न् । \newline
97. स॒मु॒द्रो ऽव॑ग॒तो ऽव॑गतः समु॒द्रः स॑मु॒द्रो ऽव॑गतः सलि॒लः स॑लि॒लो ऽव॑गतः समु॒द्रः स॑मु॒द्रो ऽव॑गतः सलि॒लः । \newline
98. अव॑गतः सलि॒लः स॑लि॒लो ऽव॑ग॒तो ऽव॑गतः सलि॒लः प्रप्लु॑तः॒ प्रप्लु॑तः सलि॒लो ऽव॑ग॒तो ऽव॑गतः सलि॒लः प्रप्लु॑तः । \newline
99. अव॑गत॒ इत्यव॑ - ग॒तः॒ । \newline
100. स॒लि॒लः प्रप्लु॑तः॒ प्रप्लु॑तः सलि॒लः स॑लि॒लः प्रप्लु॑तः॒ सुवः॒ सुवः॒ प्रप्लु॑तः सलि॒लः स॑लि॒लः प्रप्लु॑तः॒ सुवः॑ । \newline
101. प्रप्लु॑तः॒ सुवः॒ सुवः॒ प्रप्लु॑तः॒ प्रप्लु॑तः॒ सुव॑ रु॒दृच॑ मु॒दृचꣳ॒॒ सुवः॒ प्रप्लु॑तः॒ प्रप्लु॑तः॒ सुव॑ रु॒दृच᳚म् । \newline
102. प्रप्लु॑त॒ इति॒ प्र - प्लु॒तः॒ । \newline
103. सुव॑ रु॒दृच॑ मु॒दृचꣳ॒॒ सुवः॒ सुव॑ रु॒दृच॑म् ग॒तो ग॒त उ॒दृचꣳ॒॒ सुवः॒ सुव॑ रु॒दृच॑म् ग॒तः । \newline
104. उ॒दृच॑म् ग॒तो ग॒त उ॒दृच॑ मु॒दृच॑म् ग॒तः । \newline
105. उ॒दृच॒मित्यु॑त् - ऋच᳚म् । \newline
106. ग॒त इति॑ ग॒तः । \newline
\pagebreak
\markright{ TS 4.4.10.1  \hfill https://www.vedavms.in \hfill}

\section{ TS 4.4.10.1 }

\textbf{TS 4.4.10.1 } \newline
\textbf{Samhita Paata} \newline

कृत्ति॑का॒ नक्ष॑त्र-म॒ग्निर्दे॒वता॒ऽग्ने रुचः॑ स्थ प्र॒जाप॑तेर्द्धा॒तुः सोम॑स्य॒र्चे त्वा॑ रु॒चे त्वा᳚ द्यु॒ते त्वा॑ भा॒से त्वा॒ ज्योति॑षे त्वा रोहि॒णी नक्ष॑त्रं प्र॒जाप॑तिर्दे॒वता॑ मृगशी॒र्॒.षं॑ नक्ष॑त्रꣳ॒॒ सोमो॑ दे॒वता॒ ऽऽर्द्रा नक्ष॑त्रꣳ रु॒द्रो दे॒वता॒ पुन॑र्वसू॒ नक्ष॑त्र॒मदि॑तिर्दे॒वता॑- ति॒ष्यो॑ नक्ष॑त्रं॒ बृह॒स्पति॑र्दे॒वता᳚ ऽऽश्रे॒षा नक्ष॑त्रꣳ स॒र्पा दे॒वता॑ म॒घा नक्ष॑त्रं पि॒तरो॑ दे॒वता॒ फल्गु॑नी॒ नक्ष॑त्र - [  ] \newline

\textbf{Pada Paata} \newline

कृत्ति॑काः । नक्ष॑त्रम् । अ॒ग्निः । दे॒वता᳚ । अ॒ग्नेः । रुचः॑ । स्थ॒ । प्र॒जाप॑ते॒रिति॑ प्र॒जा - प॒तेः॒ । धा॒तुः । सोम॑स्य । ऋ॒चे । त्वा॒ । रु॒चे । त्वा॒ । द्यु॒ते । त्वा॒ । भा॒से । त्वा॒ । ज्योति॑षे । त्वा॒ । रो॒हि॒णी । नक्ष॑त्रम् । प्र॒जाप॑ति॒रिति॑ प्र॒जा - प॒तिः॒ । दे॒वता᳚ । मृ॒ग॒शी॒र्॒.षमिति॑ मृग - शी॒र्॒.षम् । नक्ष॑त्रम् । सोमः॑ । दे॒वता᳚ । आ॒र्द्रा । नक्ष॑त्रम् । रु॒द्रः । दे॒वता᳚ । पुन॑र्वसू॒ इति॒ पुनः॑ - व॒सू॒ । नक्ष॑त्रम् । अदि॑तिः । दे॒वता᳚ । ति॒ष्यः॑ । नक्ष॑त्रम् । बृह॒स्पतिः॑ । दे॒वता᳚ । आ॒श्रे॒षा इत्या᳚- श्रे॒षाः । नक्ष॑त्रम् । स॒र्पाः । दे॒वता᳚ । म॒घाः । नक्ष॑त्रम् । पि॒तरः॑ । दे॒वता᳚ । फल्गु॑नी॒ इति॑ । नक्ष॑त्रम् ।  \newline


\textbf{Krama Paata} \newline

कृत्ति॑का॒ नक्ष॑त्रम् । नक्ष॑त्रम॒ग्निः । अ॒ग्निर् दे॒वता᳚ । दे॒वता॒ऽग्नेः । अ॒ग्ने रुचः॑ । रुचः॑ स्थ । स्थ॒ प्र॒जाप॑तेः । प्र॒जाप॑तेर् धा॒तुः । प्र॒जाप॑ते॒रिति॑ प्र॒जा - प॒तेः॒ । धा॒तुः सोम॑स्य । सोम॑स्य॒र्चे । ऋ॒चे त्वा᳚ । त्वा॒ रु॒चे । रु॒चे त्वा᳚ । त्वा॒ द्यु॒ते । द्यु॒ते त्वा᳚ । त्वा॒ भा॒से । भा॒से त्वा᳚ । त्वा॒ ज्योति॑षे । ज्योति॑षे त्वा । त्वा॒ रो॒हि॒णी । रो॒हि॒णी नक्ष॑त्रम् । नक्ष॑त्रम् प्र॒जाप॑तिः । प्र॒जाप॑तिर् दे॒वता᳚ । प्र॒जाप॑ति॒रिति॑ प्र॒जा - प॒तिः॒ । दे॒वता॑ मृगशी॒र्॒.षम् । मृ॒ग॒शी॒र्॒.षम् नक्ष॑त्रम् । मृ॒ग॒शी॒र्.॒षमिति॑ मृग - शी॒र्॒.षम् । नक्ष॑त्रꣳ॒॒ सोमः॑ । सोमो॑ दे॒वता᳚ । दे॒वता॒ऽऽर्द्रा । आ॒र्द्रा नक्ष॑त्रम् । नक्ष॑त्रꣳ रु॒द्रः । रु॒द्रो दे॒वता᳚ । दे॒वता॒ पुन॑र्वसू । पुन॑र्वसू॒ नक्ष॑त्रम् । पुन॑र्वसू॒ इति॒ पुनः॑ - व॒सू॒ । नक्ष॑त्र॒मदि॑तिः । अदि॑तिर् दे॒वता᳚ । दे॒वता॑ ति॒ष्यः॑ । ति॒ष्यो॑ नक्ष॑त्रम् । नक्ष॑त्र॒म् बृह॒स्पतिः॑ । बृह॒स्पति॑र् दे॒वता᳚ । दे॒वता᳚ऽऽश्रे॒षाः । आ॒श्रे॒षा नक्ष॑त्रम् । आ॒श्रे॒षा इत्या᳚ - श्रे॒षाः । नक्ष॑त्रꣳ स॒र्पाः । स॒र्पा दे॒वता᳚ । दे॒वता॑ म॒घाः । म॒घा नक्ष॑त्रम् । नक्ष॑त्रम् पि॒तरः॑ । पि॒तरो॑ दे॒वता᳚ । दे॒वता॒ फल्गु॑नी ( ) । फल्गु॑नी॒ नक्ष॑त्रम् । फल्गु॑नी॒ इति॒ फल्गु॑नी । नक्ष॑त्रमर्य॒मा \newline

\textbf{Jatai Paata} \newline

1. कृत्ति॑का॒ नक्ष॑त्र॒म् नक्ष॑त्र॒म् कृत्ति॑काः॒ कृत्ति॑का॒ नक्ष॑त्रम् । \newline
2. नक्ष॑त्र म॒ग्नि र॒ग्निर् नक्ष॑त्र॒म् नक्ष॑त्र म॒ग्निः । \newline
3. अ॒ग्निर् दे॒वता॑ दे॒वता॒ ऽग्नि र॒ग्निर् दे॒वता᳚ । \newline
4. दे॒वता॒ ऽग्ने र॒ग्नेर् दे॒वता॑ दे॒वता॒ ऽग्नेः । \newline
5. अ॒ग्ने रुचो॒ रुचो॒ ऽग्ने र॒ग्ने रुचः॑ । \newline
6. रुचः॑ स्थ स्थ॒ रुचो॒ रुचः॑ स्थ । \newline
7. स्थ॒ प्र॒जाप॑तेः प्र॒जाप॑तेः स्थ स्थ प्र॒जाप॑तेः । \newline
8. प्र॒जाप॑तेर् धा॒तुर् धा॒तुः प्र॒जाप॑तेः प्र॒जाप॑तेर् धा॒तुः । \newline
9. प्र॒जाप॑ते॒रिति॑ प्र॒जा - प॒तेः॒ । \newline
10. धा॒तुः सोम॑स्य॒ सोम॑स्य धा॒तुर् धा॒तुः सोम॑स्य । \newline
11. सोम॑स्य॒ र्‌च ऋ॒चे सोम॑स्य॒ सोम॑स्य॒ र्‌चे । \newline
12. ऋ॒चे त्वा᳚ त्व॒ र्‌च ऋ॒चे त्वा᳚ । \newline
13. त्वा॒ रु॒चे रु॒चे त्वा᳚ त्वा रु॒चे । \newline
14. रु॒चे त्वा᳚ त्वा रु॒चे रु॒चे त्वा᳚ । \newline
15. त्वा॒ द्यु॒ते द्यु॒ते त्वा᳚ त्वा द्यु॒ते । \newline
16. द्यु॒ते त्वा᳚ त्वा द्यु॒ते द्यु॒ते त्वा᳚ । \newline
17. त्वा॒ भा॒से भा॒से त्वा᳚ त्वा भा॒से । \newline
18. भा॒से त्वा᳚ त्वा भा॒से भा॒से त्वा᳚ । \newline
19. त्वा॒ ज्योति॑षे॒ ज्योति॑षे त्वा त्वा॒ ज्योति॑षे । \newline
20. ज्योति॑षे त्वा त्वा॒ ज्योति॑षे॒ ज्योति॑षे त्वा । \newline
21. त्वा॒ रो॒हि॒णी रो॑हि॒णी त्वा᳚ त्वा रोहि॒णी । \newline
22. रो॒हि॒णी नक्ष॑त्र॒म् नक्ष॑त्रꣳ रोहि॒णी रो॑हि॒णी नक्ष॑त्रम् । \newline
23. नक्ष॑त्रम् प्र॒जाप॑तिः प्र॒जाप॑ति॒र् नक्ष॑त्र॒म् नक्ष॑त्रम् प्र॒जाप॑तिः । \newline
24. प्र॒जाप॑तिर् दे॒वता॑ दे॒वता᳚ प्र॒जाप॑तिः प्र॒जाप॑तिर् दे॒वता᳚ । \newline
25. प्र॒जाप॑ति॒रिति॑ प्र॒जा - प॒तिः॒ । \newline
26. दे॒वता॑ मृगशी॒र्॒.षम् मृ॑गशी॒र्॒.षम् दे॒वता॑ दे॒वता॑ मृगशी॒र्॒.षम् । \newline
27. मृ॒ग॒शी॒र्॒.षम् नक्ष॑त्र॒म् नक्ष॑त्रम् मृगशी॒र्॒.षम् मृ॑गशी॒र्॒.षम् नक्ष॑त्रम् । \newline
28. मृ॒ग॒शी॒र्॒.षमिति॑ मृग - शी॒र्॒.षम् । \newline
29. नक्ष॑त्रꣳ॒॒ सोमः॒ सोमो॒ नक्ष॑त्र॒म् नक्ष॑त्रꣳ॒॒ सोमः॑ । \newline
30. सोमो॑ दे॒वता॑ दे॒वता॒ सोमः॒ सोमो॑ दे॒वता᳚ । \newline
31. दे॒वता॒ ऽऽर्द्रा ऽऽर्द्रा दे॒वता॑ दे॒वता॒ ऽऽर्द्रा । \newline
32. आ॒र्द्रा नक्ष॑त्र॒म् नक्ष॑त्र मा॒र्द्रा ऽऽर्द्रा नक्ष॑त्रम् । \newline
33. नक्ष॑त्रꣳ रु॒द्रो रु॒द्रो नक्ष॑त्र॒म् नक्ष॑त्रꣳ रु॒द्रः । \newline
34. रु॒द्रो दे॒वता॑ दे॒वता॑ रु॒द्रो रु॒द्रो दे॒वता᳚ । \newline
35. दे॒वता॒ पुन॑र्वसू॒ पुन॑र्वसू दे॒वता॑ दे॒वता॒ पुन॑र्वसू । \newline
36. पुन॑र्वसू॒ नक्ष॑त्र॒म् नक्ष॑त्र॒म् पुन॑र्वसू॒ पुन॑र्वसू॒ नक्ष॑त्रम् । \newline
37. पुन॑र्वसू॒ इति॒ पुनः॑ - व॒सू॒ । \newline
38. नक्ष॑त्र॒ मदि॑ति॒ रदि॑ति॒र् नक्ष॑त्र॒म् नक्ष॑त्र॒ मदि॑तिः । \newline
39. अदि॑तिर् दे॒वता॑ दे॒वता ऽदि॑ति॒ रदि॑तिर् दे॒वता᳚ । \newline
40. दे॒वता॑ ति॒ष्य॑ स्ति॒ष्यो॑ दे॒वता॑ दे॒वता॑ ति॒ष्यः॑ । \newline
41. ति॒ष्यो॑ नक्ष॑त्र॒म् नक्ष॑त्रम् ति॒ष्य॑ स्ति॒ष्यो॑ नक्ष॑त्रम् । \newline
42. नक्ष॑त्र॒म् बृह॒स्पति॒र् बृह॒स्पति॒र् नक्ष॑त्र॒म् नक्ष॑त्र॒म् बृह॒स्पतिः॑ । \newline
43. बृह॒स्पति॑र् दे॒वता॑ दे॒वता॒ बृह॒स्पति॒र् बृह॒स्पति॑र् दे॒वता᳚ । \newline
44. दे॒वता᳚ ऽऽश्रे॒षा आ᳚श्रे॒षा दे॒वता॑ दे॒वता᳚ ऽऽश्रे॒षाः । \newline
45. आ॒श्रे॒षा नक्ष॑त्र॒म् नक्ष॑त्र माश्रे॒षा आ᳚श्रे॒षा नक्ष॑त्रम् । \newline
46. आ॒श्रे॒षा इत्या᳚ - श्रे॒षाः । \newline
47. नक्ष॑त्रꣳ स॒र्पाः स॒र्पा नक्ष॑त्र॒म् नक्ष॑त्रꣳ स॒र्पाः । \newline
48. स॒र्पा दे॒वता॑ दे॒वता॑ स॒र्पाः स॒र्पा दे॒वता᳚ । \newline
49. दे॒वता॑ म॒घा म॒घा दे॒वता॑ दे॒वता॑ म॒घाः । \newline
50. म॒घा नक्ष॑त्र॒म् नक्ष॑त्रम् म॒घा म॒घा नक्ष॑त्रम् । \newline
51. नक्ष॑त्रम् पि॒तरः॑ पि॒तरो॒ नक्ष॑त्र॒म् नक्ष॑त्रम् पि॒तरः॑ । \newline
52. पि॒तरो॑ दे॒वता॑ दे॒वता॑ पि॒तरः॑ पि॒तरो॑ दे॒वता᳚ । \newline
53. दे॒वता॒ फल्गु॑नी॒ फल्गु॑नी दे॒वता॑ दे॒वता॒ फल्गु॑नी । \newline
54. फल्गु॑नी॒ नक्ष॑त्र॒म् नक्ष॑त्र॒म् फल्गु॑नी॒ फल्गु॑नी॒ नक्ष॑त्रम् । \newline
55. फल्गु॑नी॒ इति॒ फल्गु॑नी । \newline
56. नक्ष॑त्र मर्य॒मा ऽर्य॒मा नक्ष॑त्र॒म् नक्ष॑त्र मर्य॒मा । \newline

\textbf{Ghana Paata } \newline

1. कृत्ति॑का॒ नक्ष॑त्र॒म् नक्ष॑त्र॒म् कृत्ति॑काः॒ कृत्ति॑का॒ नक्ष॑त्र म॒ग्नि र॒ग्निर् नक्ष॑त्र॒म् कृत्ति॑काः॒ कृत्ति॑का॒ नक्ष॑त्र म॒ग्निः । \newline
2. नक्ष॑त्र म॒ग्नि र॒ग्निर् नक्ष॑त्र॒म् नक्ष॑त्र म॒ग्निर् दे॒वता॑ दे॒वता॒ ऽग्निर् नक्ष॑त्र॒म् नक्ष॑त्र म॒ग्निर् दे॒वता᳚ । \newline
3. अ॒ग्निर् दे॒वता॑ दे॒वता॒ ऽग्नि र॒ग्निर् दे॒वता॒ ऽग्ने र॒ग्नेर् दे॒वता॒ ऽग्नि र॒ग्निर् दे॒वता॒ ऽग्नेः । \newline
4. दे॒वता॒ ऽग्ने र॒ग्नेर् दे॒वता॑ दे॒वता॒ ऽग्ने रुचो॒ रुचो॒ ऽग्नेर् दे॒वता॑ दे॒वता॒ ऽग्ने रुचः॑ । \newline
5. अ॒ग्ने रुचो॒ रुचो॒ ऽग्ने र॒ग्ने रुचः॑ स्थ स्थ॒ रुचो॒ ऽग्ने र॒ग्ने रुचः॑ स्थ । \newline
6. रुचः॑ स्थ स्थ॒ रुचो॒ रुचः॑ स्थ प्र॒जाप॑तेः प्र॒जाप॑तेः स्थ॒ रुचो॒ रुचः॑ स्थ प्र॒जाप॑तेः । \newline
7. स्थ॒ प्र॒जाप॑तेः प्र॒जाप॑तेः स्थ स्थ प्र॒जाप॑तेर् धा॒तुर् धा॒तुः प्र॒जाप॑तेः स्थ स्थ प्र॒जाप॑तेर् धा॒तुः । \newline
8. प्र॒जाप॑तेर् धा॒तुर् धा॒तुः प्र॒जाप॑तेः प्र॒जाप॑तेर् धा॒तुः सोम॑स्य॒ सोम॑स्य धा॒तुः प्र॒जाप॑तेः प्र॒जाप॑तेर् धा॒तुः सोम॑स्य । \newline
9. प्र॒जाप॑ते॒रिति॑ प्र॒जा - प॒तेः॒ । \newline
10. धा॒तुः सोम॑स्य॒ सोम॑स्य धा॒तुर् धा॒तुः सोम॑स्य॒ र्‌च ऋ॒चे सोम॑स्य धा॒तुर् धा॒तुः सोम॑स्य॒ र्‌चे । \newline
11. सोम॑स्य॒ र्‌च ऋ॒चे सोम॑स्य॒ सोम॑स्य॒ र्‌चे त्वा᳚ त्व॒ र्‌चे सोम॑स्य॒ सोम॑स्य॒ र्‌चे त्वा᳚ । \newline
12. ऋ॒चे त्वा᳚ त्व॒ र्‌च ऋ॒चे त्वा॑ रु॒चे रु॒चे त्व॒ र्‌च ऋ॒चे त्वा॑ रु॒चे । \newline
13. त्वा॒ रु॒चे रु॒चे त्वा᳚ त्वा रु॒चे त्वा᳚ त्वा रु॒चे त्वा᳚ त्वा रु॒चे त्वा᳚ । \newline
14. रु॒चे त्वा᳚ त्वा रु॒चे रु॒चे त्वा᳚ द्यु॒ते द्यु॒ते त्वा॑ रु॒चे रु॒चे त्वा᳚ द्यु॒ते । \newline
15. त्वा॒ द्यु॒ते द्यु॒ते त्वा᳚ त्वा द्यु॒ते त्वा᳚ त्वा द्यु॒ते त्वा᳚ त्वा द्यु॒ते त्वा᳚ । \newline
16. द्यु॒ते त्वा᳚ त्वा द्यु॒ते द्यु॒ते त्वा॑ भा॒से भा॒से त्वा᳚ द्यु॒ते द्यु॒ते त्वा॑ भा॒से । \newline
17. त्वा॒ भा॒से भा॒से त्वा᳚ त्वा भा॒से त्वा᳚ त्वा भा॒से त्वा᳚ त्वा भा॒से त्वा᳚ । \newline
18. भा॒से त्वा᳚ त्वा भा॒से भा॒से त्वा॒ ज्योति॑षे॒ ज्योति॑षे त्वा भा॒से भा॒से त्वा॒ ज्योति॑षे । \newline
19. त्वा॒ ज्योति॑षे॒ ज्योति॑षे त्वा त्वा॒ ज्योति॑षे त्वा त्वा॒ ज्योति॑षे त्वा त्वा॒ ज्योति॑षे त्वा । \newline
20. ज्योति॑षे त्वा त्वा॒ ज्योति॑षे॒ ज्योति॑षे त्वा रोहि॒णी रो॑हि॒णी त्वा॒ ज्योति॑षे॒ ज्योति॑षे त्वा रोहि॒णी । \newline
21. त्वा॒ रो॒हि॒णी रो॑हि॒णी त्वा᳚ त्वा रोहि॒णी नक्ष॑त्र॒म् नक्ष॑त्रꣳ रोहि॒णी त्वा᳚ त्वा रोहि॒णी नक्ष॑त्रम् । \newline
22. रो॒हि॒णी नक्ष॑त्र॒म् नक्ष॑त्रꣳ रोहि॒णी रो॑हि॒णी नक्ष॑त्रम् प्र॒जाप॑तिः प्र॒जाप॑ति॒र् नक्ष॑त्रꣳ रोहि॒णी रो॑हि॒णी नक्ष॑त्रम् प्र॒जाप॑तिः । \newline
23. नक्ष॑त्रम् प्र॒जाप॑तिः प्र॒जाप॑ति॒र् नक्ष॑त्र॒म् नक्ष॑त्रम् प्र॒जाप॑तिर् दे॒वता॑ दे॒वता᳚ प्र॒जाप॑ति॒र् नक्ष॑त्र॒म् नक्ष॑त्रम् प्र॒जाप॑तिर् दे॒वता᳚ । \newline
24. प्र॒जाप॑तिर् दे॒वता॑ दे॒वता᳚ प्र॒जाप॑तिः प्र॒जाप॑तिर् दे॒वता॑ मृगशी॒र्॒.षम् मृ॑गशी॒र्॒.षम् दे॒वता᳚ प्र॒जाप॑तिः प्र॒जाप॑तिर् दे॒वता॑ मृगशी॒र्॒.षम् । \newline
25. प्र॒जाप॑ति॒रिति॑ प्र॒जा - प॒तिः॒ । \newline
26. दे॒वता॑ मृगशी॒र्॒.षम् मृ॑गशी॒र्॒.षम् दे॒वता॑ दे॒वता॑ मृगशी॒र्॒.षम् नक्ष॑त्र॒म् नक्ष॑त्रम् मृगशी॒र्॒.षम् दे॒वता॑ दे॒वता॑ मृगशी॒र्॒.षम् नक्ष॑त्रम् । \newline
27. मृ॒ग॒शी॒र्॒.षम् नक्ष॑त्र॒म् नक्ष॑त्रम् मृगशी॒र्॒.षम् मृ॑गशी॒र्॒.षम् नक्ष॑त्रꣳ॒॒ सोमः॒ सोमो॒ नक्ष॑त्रम् मृगशी॒र्॒.षम् मृ॑गशी॒र्॒.षम् नक्ष॑त्रꣳ॒॒ सोमः॑ । \newline
28. मृ॒ग॒शी॒र्॒.षमिति॑ मृग - शी॒र्॒.षम् । \newline
29. नक्ष॑त्रꣳ॒॒ सोमः॒ सोमो॒ नक्ष॑त्र॒म् नक्ष॑त्रꣳ॒॒ सोमो॑ दे॒वता॑ दे॒वता॒ सोमो॒ नक्ष॑त्र॒म् नक्ष॑त्रꣳ॒॒ सोमो॑ दे॒वता᳚ । \newline
30. सोमो॑ दे॒वता॑ दे॒वता॒ सोमः॒ सोमो॑ दे॒वता॒ ऽऽर्द्रा ऽऽर्द्रा दे॒वता॒ सोमः॒ सोमो॑ दे॒वता॒ ऽऽर्द्रा । \newline
31. दे॒वता॒ ऽऽर्द्रा ऽऽर्द्रा दे॒वता॑ दे॒वता॒ ऽऽर्द्रा नक्ष॑त्र॒म् नक्ष॑त्र मा॒र्द्रा दे॒वता॑ दे॒वता॒ ऽऽर्द्रा नक्ष॑त्रम् । \newline
32. आ॒र्द्रा नक्ष॑त्र॒म् नक्ष॑त्र मा॒र्द्रा ऽऽर्द्रा नक्ष॑त्रꣳ रु॒द्रो रु॒द्रो नक्ष॑त्र मा॒र्द्रा ऽऽर्द्रा नक्ष॑त्रꣳ रु॒द्रः । \newline
33. नक्ष॑त्रꣳ रु॒द्रो रु॒द्रो नक्ष॑त्र॒म् नक्ष॑त्रꣳ रु॒द्रो दे॒वता॑ दे॒वता॑ रु॒द्रो नक्ष॑त्र॒म् नक्ष॑त्रꣳ रु॒द्रो दे॒वता᳚ । \newline
34. रु॒द्रो दे॒वता॑ दे॒वता॑ रु॒द्रो रु॒द्रो दे॒वता॒ पुन॑र्वसू॒ पुन॑र्वसू दे॒वता॑ रु॒द्रो रु॒द्रो दे॒वता॒ पुन॑र्वसू । \newline
35. दे॒वता॒ पुन॑र्वसू॒ पुन॑र्वसू दे॒वता॑ दे॒वता॒ पुन॑र्वसू॒ नक्ष॑त्र॒म् नक्ष॑त्र॒म् पुन॑र्वसू दे॒वता॑ दे॒वता॒ पुन॑र्वसू॒ नक्ष॑त्रम् । \newline
36. पुन॑र्वसू॒ नक्ष॑त्र॒म् नक्ष॑त्र॒म् पुन॑र्वसू॒ पुन॑र्वसू॒ नक्ष॑त्र॒ मदि॑ति॒ रदि॑ति॒र् नक्ष॑त्र॒म् पुन॑र्वसू॒ पुन॑र्वसू॒ नक्ष॑त्र॒ मदि॑तिः । \newline
37. पुन॑र्वसू॒ इति॒ पुनः॑ - व॒सू॒ । \newline
38. नक्ष॑त्र॒ मदि॑ति॒ रदि॑ति॒र् नक्ष॑त्र॒म् नक्ष॑त्र॒ मदि॑तिर् दे॒वता॑ दे॒वता ऽदि॑ति॒र् नक्ष॑त्र॒म् नक्ष॑त्र॒ मदि॑तिर् दे॒वता᳚ । \newline
39. अदि॑तिर् दे॒वता॑ दे॒वता ऽदि॑ति॒ रदि॑तिर् दे॒वता॑ ति॒ष्य॑ स्ति॒ष्यो॑ दे॒वता ऽदि॑ति॒ रदि॑तिर् दे॒वता॑ ति॒ष्यः॑ । \newline
40. दे॒वता॑ ति॒ष्य॑ स्ति॒ष्यो॑ दे॒वता॑ दे॒वता॑ ति॒ष्यो॑ नक्ष॑त्र॒म् नक्ष॑त्रम् ति॒ष्यो॑ दे॒वता॑ दे॒वता॑ ति॒ष्यो॑ नक्ष॑त्रम् । \newline
41. ति॒ष्यो॑ नक्ष॑त्र॒म् नक्ष॑त्रम् ति॒ष्य॑ स्ति॒ष्यो॑ नक्ष॑त्र॒म् बृह॒स्पति॒र् बृह॒स्पति॒र् नक्ष॑त्रम् ति॒ष्य॑ स्ति॒ष्यो॑ नक्ष॑त्र॒म् बृह॒स्पतिः॑ । \newline
42. नक्ष॑त्र॒म् बृह॒स्पति॒र् बृह॒स्पति॒र् नक्ष॑त्र॒म् नक्ष॑त्र॒म् बृह॒स्पति॑र् दे॒वता॑ दे॒वता॒ बृह॒स्पति॒र् नक्ष॑त्र॒म् नक्ष॑त्र॒म् बृह॒स्पति॑र् दे॒वता᳚ । \newline
43. बृह॒स्पति॑र् दे॒वता॑ दे॒वता॒ बृह॒स्पति॒र् बृह॒स्पति॑र् दे॒वता᳚ ऽऽश्रे॒षा आ᳚श्रे॒षा दे॒वता॒ बृह॒स्पति॒र् बृह॒स्पति॑र् दे॒वता᳚ ऽऽश्रे॒षाः । \newline
44. दे॒वता᳚ ऽऽश्रे॒षा आ᳚श्रे॒षा दे॒वता॑ दे॒वता᳚ ऽऽश्रे॒षा नक्ष॑त्र॒म् नक्ष॑त्र माश्रे॒षा दे॒वता॑ दे॒वता᳚ ऽऽश्रे॒षा नक्ष॑त्रम् । \newline
45. आ॒श्रे॒षा नक्ष॑त्र॒म् नक्ष॑त्र माश्रे॒षा आ᳚श्रे॒षा नक्ष॑त्रꣳ स॒र्पाः स॒र्पा नक्ष॑त्र माश्रे॒षा आ᳚श्रे॒षा नक्ष॑त्रꣳ स॒र्पाः । \newline
46. आ॒श्रे॒षा इत्या᳚ - श्रे॒षाः । \newline
47. नक्ष॑त्रꣳ स॒र्पाः स॒र्पा नक्ष॑त्र॒म् नक्ष॑त्रꣳ स॒र्पा दे॒वता॑ दे॒वता॑ स॒र्पा नक्ष॑त्र॒म् नक्ष॑त्रꣳ स॒र्पा दे॒वता᳚ । \newline
48. स॒र्पा दे॒वता॑ दे॒वता॑ स॒र्पाः स॒र्पा दे॒वता॑ म॒घा म॒घा दे॒वता॑ स॒र्पाः स॒र्पा दे॒वता॑ म॒घाः । \newline
49. दे॒वता॑ म॒घा म॒घा दे॒वता॑ दे॒वता॑ म॒घा नक्ष॑त्र॒म् नक्ष॑त्रम् म॒घा दे॒वता॑ दे॒वता॑ म॒घा नक्ष॑त्रम् । \newline
50. म॒घा नक्ष॑त्र॒म् नक्ष॑त्रम् म॒घा म॒घा नक्ष॑त्रम् पि॒तरः॑ पि॒तरो॒ नक्ष॑त्रम् म॒घा म॒घा नक्ष॑त्रम् पि॒तरः॑ । \newline
51. नक्ष॑त्रम् पि॒तरः॑ पि॒तरो॒ नक्ष॑त्र॒म् नक्ष॑त्रम् पि॒तरो॑ दे॒वता॑ दे॒वता॑ पि॒तरो॒ नक्ष॑त्र॒म् नक्ष॑त्रम् पि॒तरो॑ दे॒वता᳚ । \newline
52. पि॒तरो॑ दे॒वता॑ दे॒वता॑ पि॒तरः॑ पि॒तरो॑ दे॒वता॒ फल्गु॑नी॒ फल्गु॑नी दे॒वता॑ पि॒तरः॑ पि॒तरो॑ दे॒वता॒ फल्गु॑नी । \newline
53. दे॒वता॒ फल्गु॑नी॒ फल्गु॑नी दे॒वता॑ दे॒वता॒ फल्गु॑नी॒ नक्ष॑त्र॒म् नक्ष॑त्र॒म् फल्गु॑नी दे॒वता॑ दे॒वता॒ फल्गु॑नी॒ नक्ष॑त्रम् । \newline
54. फल्गु॑नी॒ नक्ष॑त्र॒म् नक्ष॑त्र॒म् फल्गु॑नी॒ फल्गु॑नी॒ नक्ष॑त्र मर्य॒मा ऽर्य॒मा नक्ष॑त्र॒म् फल्गु॑नी॒ फल्गु॑नी॒ नक्ष॑त्र मर्य॒मा । \newline
55. फल्गु॑नी॒ इति॒ फल्गु॑नी । \newline
56. नक्ष॑त्र मर्य॒मा ऽर्य॒मा नक्ष॑त्र॒म् नक्ष॑त्र मर्य॒मा दे॒वता॑ दे॒वता᳚ ऽर्य॒मा नक्ष॑त्र॒म् नक्ष॑त्र मर्य॒मा दे॒वता᳚ । \newline
\pagebreak
\markright{ TS 4.4.10.2  \hfill https://www.vedavms.in \hfill}

\section{ TS 4.4.10.2 }

\textbf{TS 4.4.10.2 } \newline
\textbf{Samhita Paata} \newline

-मर्य॒मा दे॒वता॒ फल्गु॑नी॒ नक्ष॑त्रं॒ भगो॑ दे॒वता॒ हस्तो॒ नक्ष॑त्रꣳ सवि॒ता दे॒वता॑ चि॒त्रा नक्ष॑त्र॒मिन्द्रो॑ दे॒वता᳚ स्वा॒ती नक्ष॑त्रं ॅवा॒युर्दे॒वता॒ विशा॑खे॒ नक्ष॑त्रमिन्द्रा॒ग्नी दे॒वता॑ ऽनूरा॒धा नक्ष॑त्रं मि॒त्रो दे॒वता॑ रोहि॒णी नक्ष॑त्र॒मिन्द्रो॑ दे॒वता॑ वि॒चृतौ॒ नक्ष॑त्रं पि॒तरो॑ दे॒वता॑ ऽषा॒ढा नक्ष॑त्र॒मापो॑ दे॒वता॑ ऽषा॒ढा नक्ष॑त्रं॒ ॅविश्वे॑ दे॒वा दे॒वता᳚ श्रो॒णा नक्ष॑त्रं॒ ॅविष्णु॑र्दे॒वता॒ श्रवि॑ष्ठा॒ नक्ष॑त्रं॒ ॅवस॑वो - [  ] \newline

\textbf{Pada Paata} \newline

अ॒र्य॒मा । दे॒वता᳚ । फल्गु॑नी॒ इति॑ । नक्ष॑त्रम् । भगः॑ । दे॒वता᳚ । हस्तः॑ । नक्ष॑त्रम् । स॒वि॒ता । दे॒वता᳚ । चि॒त्रा । नक्ष॑त्रम् । इन्द्रः॑ । दे॒वता᳚ । स्वा॒ती । नक्ष॑त्रम् । वा॒युः । दे॒वता᳚ । विशा॑खे॒ इति॒ वि - शा॒खे॒ । नक्ष॑त्रम् । इ॑006छ्;॒द्रा॒ग्नी इती᳚न्द्र - अ॒ग्नी । दे॒वता॑ । अ॒नू॒रा॒धा इत्य॑नु - रा॒धाः । नक्ष॑त्रम् । मि॒त्रः । दे॒वता᳚ । रो॒हि॒णी । नक्ष॑त्रम् । इन्द्रः॑ । दे॒वता᳚ । वि॒चृता॒विति॑ वि - चृतौ᳚ । नक्ष॑त्रम् । पि॒तरः॑ । दे॒वता᳚ । अ॒षा॒ढाः । नक्ष॑त्रम् । आपः॑ । दे॒वता᳚ । अ॒षा॒ढाः । नक्ष॑त्रम् । विश्वे᳚ । दे॒वाः । दे॒वता᳚ । श्रो॒णा । नक्ष॑त्रम् । विष्णुः॑ । दे॒वता᳚ । श्रवि॑ष्ठाः । नक्ष॑त्रम् । वस॑वः ।  \newline


\textbf{Krama Paata} \newline

अ॒र्य॒मा दे॒वता᳚ । दे॒वता॒ फल्गु॑नी । फल्गु॑नी॒ नक्ष॑त्रम् । फल्गु॑नी॒ इति॒ फल्गु॑नी । नक्ष॑त्र॒म् भगः॑ । भगो॑ दे॒वता᳚ । दे॒वता॒ हस्तः॑ । हस्तो॒ नक्ष॑त्रम् । नक्ष॑त्रꣳ सवि॒ता । स॒वि॒ता दे॒वता᳚ । दे॒वता॑ चि॒त्रा । चि॒त्रा नक्ष॑त्रम् । नक्ष॑त्र॒मिन्द्रः॑ । इन्द्रो॑ दे॒वता᳚ । दे॒वता᳚ स्वा॒ती । स्वा॒ती नक्ष॑त्रम् । नक्ष॑त्रम् ॅवा॒युः । वा॒युर् दे॒वता᳚ । दे॒वता॒ विशा॑खे । विशा॑खे॒ नक्ष॑त्रम् । विशा॑खे॒ इति॒ वि - शा॒खे॒ । नक्ष॑त्रमिन्द्रा॒ग्नी । इ॒न्द्रा॒ग्नी दे॒वता᳚ । इ॒न्द्रा॒ग्नी इती᳚न्द्र - अ॒ग्नी । दे॒वता॑ऽनूरा॒धाः । अ॒नू॒रा॒धा नक्ष॑त्रम् । अ॒नू॒रा॒धा इत्य॑नु - रा॒धाः । नक्ष॑त्रम् मि॒त्रः । मि॒त्रो दे॒वता᳚ । दे॒वता॑ रोहि॒णी । रो॒हि॒णी नक्ष॑त्रम् । नक्ष॑त्र॒मिन्द्रः॑ । इन्द्रो॑ दे॒वता᳚ । दे॒वता॑ वि॒चृतौ᳚ । वि॒चृतौ॒ नक्ष॑त्रम् । वि॒चृता॒विति॑ वि - चृतौ᳚ । नक्ष॑त्रम् पि॒तरः॑ । पि॒तरो॑ दे॒वता᳚ । दे॒वता॑ऽषा॒ढाः । अ॒षा॒ढा नक्ष॑त्रम् । नक्ष॑त्र॒मापः॑ । आपो॑ दे॒वता᳚ । दे॒वता॑ऽषा॒ढाः । अ॒षा॒ढा नक्ष॑त्रम् । नक्ष॑त्रं॒ ॅविश्वे᳚ । विश्वे॑ दे॒वाः । दे॒वा दे॒वता᳚ । दे॒वता᳚ श्रो॒णा । श्रो॒णा नक्ष॑त्रम् । नक्ष॑त्रं॒ ॅविष्णुः॑ । विष्णु॑र् दे॒वता᳚ । दे॒वता॒ श्रवि॑ष्ठाः । श्रवि॑ष्ठा॒ नक्ष॑त्रम् । नक्ष॑त्रं॒ ॅवस॑वः ( ) । वस॑वो दे॒वता᳚ \newline

\textbf{Jatai Paata} \newline

1. अ॒र्य॒मा दे॒वता॑ दे॒वता᳚ ऽर्य॒मा ऽर्य॒मा दे॒वता᳚ । \newline
2. दे॒वता॒ फल्गु॑नी॒ फल्गु॑नी दे॒वता॑ दे॒वता॒ फल्गु॑नी । \newline
3. फल्गु॑नी॒ नक्ष॑त्र॒म् नक्ष॑त्र॒म् फल्गु॑नी॒ फल्गु॑नी॒ नक्ष॑त्रम् । \newline
4. फल्गु॑नी॒ इति॒ फल्गु॑नी । \newline
5. नक्ष॑त्र॒म् भगो॒ भगो॒ नक्ष॑त्र॒म् नक्ष॑त्र॒म् भगः॑ । \newline
6. भगो॑ दे॒वता॑ दे॒वता॒ भगो॒ भगो॑ दे॒वता᳚ । \newline
7. दे॒वता॒ हस्तो॒ हस्तो॑ दे॒वता॑ दे॒वता॒ हस्तः॑ । \newline
8. हस्तो॒ नक्ष॑त्र॒म् नक्ष॑त्रꣳ॒॒ हस्तो॒ हस्तो॒ नक्ष॑त्रम् । \newline
9. नक्ष॑त्रꣳ सवि॒ता स॑वि॒ता नक्ष॑त्र॒म् नक्ष॑त्रꣳ सवि॒ता । \newline
10. स॒वि॒ता दे॒वता॑ दे॒वता॑ सवि॒ता स॑वि॒ता दे॒वता᳚ । \newline
11. दे॒वता॑ चि॒त्रा चि॒त्रा दे॒वता॑ दे॒वता॑ चि॒त्रा । \newline
12. चि॒त्रा नक्ष॑त्र॒म् नक्ष॑त्रम् चि॒त्रा चि॒त्रा नक्ष॑त्रम् । \newline
13. नक्ष॑त्र॒ मिन्द्र॒ इन्द्रो॒ नक्ष॑त्र॒म् नक्ष॑त्र॒ मिन्द्रः॑ । \newline
14. इन्द्रो॑ दे॒वता॑ दे॒वतेन्द्र॒ इन्द्रो॑ दे॒वता᳚ । \newline
15. दे॒वता᳚ स्वा॒ती स्वा॒ती दे॒वता॑ दे॒वता᳚ स्वा॒ती । \newline
16. स्वा॒ती नक्ष॑त्र॒म् नक्ष॑त्रꣳ स्वा॒ती स्वा॒ती नक्ष॑त्रम् । \newline
17. नक्ष॑त्रं ॅवा॒युर् वा॒युर् नक्ष॑त्र॒म् नक्ष॑त्रं ॅवा॒युः । \newline
18. वा॒युर् दे॒वता॑ दे॒वता॑ वा॒युर् वा॒युर् दे॒वता᳚ । \newline
19. दे॒वता॒ विशा॑खे॒ विशा॑खे दे॒वता॑ दे॒वता॒ विशा॑खे । \newline
20. विशा॑खे॒ नक्ष॑त्र॒म् नक्ष॑त्रं॒ ॅविशा॑खे॒ विशा॑खे॒ नक्ष॑त्रम् । \newline
21. विशा॑खे॒ इति॒ वि - शा॒खे॒ । \newline
22. नक्ष॑त्र मिन्द्रा॒ग्नी इ॑न्द्रा॒ग्नी नक्ष॑त्र॒म् नक्ष॑त्र मिन्द्रा॒ग्नी । \newline
23. इ॒न्द्रा॒ग्नी दे॒वता॑ दे॒वते᳚न्द्रा॒ग्नी इ॑न्द्रा॒ग्नी दे॒वता᳚ । \newline
24. इ॒न्द्रा॒ग्नी इती᳚न्द्र - अ॒ग्नी । \newline
25. दे॒वता॑ ऽनूरा॒धा अ॑नूरा॒धा दे॒वता॑ दे॒वता॑ ऽनूरा॒धाः । \newline
26. अ॒नू॒रा॒धा नक्ष॑त्र॒म् नक्ष॑त्र मनूरा॒धा अ॑नूरा॒धा नक्ष॑त्रम् । \newline
27. अ॒नू॒रा॒धा इत्य॑नु - रा॒धाः । \newline
28. नक्ष॑त्रम् मि॒त्रो मि॒त्रो नक्ष॑त्र॒म् नक्ष॑त्रम् मि॒त्रः । \newline
29. मि॒त्रो दे॒वता॑ दे॒वता॑ मि॒त्रो मि॒त्रो दे॒वता᳚ । \newline
30. दे॒वता॑ रोहि॒णी रो॑हि॒णी दे॒वता॑ दे॒वता॑ रोहि॒णी । \newline
31. रो॒हि॒णी नक्ष॑त्र॒म् नक्ष॑त्रꣳ रोहि॒णी रो॑हि॒णी नक्ष॑त्रम् । \newline
32. नक्ष॑त्र॒ मिन्द्र॒ इन्द्रो॒ नक्ष॑त्र॒म् नक्ष॑त्र॒ मिन्द्रः॑ । \newline
33. इन्द्रो॑ दे॒वता॑ दे॒वतेन्द्र॒ इन्द्रो॑ दे॒वता᳚ । \newline
34. दे॒वता॑ वि॒चृतौ॑ वि॒चृतौ॑ दे॒वता॑ दे॒वता॑ वि॒चृतौ᳚ । \newline
35. वि॒चृतौ॒ नक्ष॑त्र॒म् नक्ष॑त्रं ॅवि॒चृतौ॑ वि॒चृतौ॒ नक्ष॑त्रम् । \newline
36. वि॒चृता॒विति॑ वि - चृतौ᳚ । \newline
37. नक्ष॑त्रम् पि॒तरः॑ पि॒तरो॒ नक्ष॑त्र॒म् नक्ष॑त्रम् पि॒तरः॑ । \newline
38. पि॒तरो॑ दे॒वता॑ दे॒वता॑ पि॒तरः॑ पि॒तरो॑ दे॒वता᳚ । \newline
39. दे॒वता॑ ऽषा॒ढा अ॑षा॒ढा दे॒वता॑ दे॒वता॑ ऽषा॒ढाः । \newline
40. अ॒षा॒ढा नक्ष॑त्र॒म् नक्ष॑त्र मषा॒ढा अ॑षा॒ढा नक्ष॑त्रम् । \newline
41. नक्ष॑त्र॒ माप॒ आपो॒ नक्ष॑त्र॒म् नक्ष॑त्र॒ मापः॑ । \newline
42. आपो॑ दे॒वता॑ दे॒वता ऽऽप॒ आपो॑ दे॒वता᳚ । \newline
43. दे॒वता॑ ऽषा॒ढा अ॑षा॒ढा दे॒वता॑ दे॒वता॑ ऽषा॒ढाः । \newline
44. अ॒षा॒ढा नक्ष॑त्र॒म् नक्ष॑त्र मषा॒ढा अ॑षा॒ढा नक्ष॑त्रम् । \newline
45. नक्ष॑त्रं॒ ॅविश्वे॒ विश्वे॒ नक्ष॑त्र॒म् नक्ष॑त्रं॒ ॅविश्वे᳚ । \newline
46. विश्वे॑ दे॒वा दे॒वा विश्वे॒ विश्वे॑ दे॒वाः । \newline
47. दे॒वा दे॒वता॑ दे॒वता॑ दे॒वा दे॒वा दे॒वता᳚ । \newline
48. दे॒वता᳚ श्रो॒णा श्रो॒णा दे॒वता॑ दे॒वता᳚ श्रो॒णा । \newline
49. श्रो॒णा नक्ष॑त्र॒म् नक्ष॑त्रꣳ श्रो॒णा श्रो॒णा नक्ष॑त्रम् । \newline
50. नक्ष॑त्रं॒ ॅविष्णु॒र् विष्णु॒र् नक्ष॑त्र॒म् नक्ष॑त्रं॒ ॅविष्णुः॑ । \newline
51. विष्णु॑र् दे॒वता॑ दे॒वता॒ विष्णु॒र् विष्णु॑र् दे॒वता᳚ । \newline
52. दे॒वता॒ श्रवि॑ष्ठाः॒ श्रवि॑ष्ठा दे॒वता॑ दे॒वता॒ श्रवि॑ष्ठाः । \newline
53. श्रवि॑ष्ठा॒ नक्ष॑त्र॒म् नक्ष॑त्रꣳ॒॒ श्रवि॑ष्ठाः॒ श्रवि॑ष्ठा॒ नक्ष॑त्रम् । \newline
54. नक्ष॑त्रं॒ ॅवस॑वो॒ वस॑वो॒ नक्ष॑त्र॒म् नक्ष॑त्रं॒ ॅवस॑वः । \newline
55. वस॑वो दे॒वता॑ दे॒वता॒ वस॑वो॒ वस॑वो दे॒वता᳚ । \newline

\textbf{Ghana Paata } \newline

1. अ॒र्य॒मा दे॒वता॑ दे॒वता᳚ ऽर्य॒मा ऽर्य॒मा दे॒वता॒ फल्गु॑नी॒ फल्गु॑नी दे॒वता᳚ ऽर्य॒मा ऽर्य॒मा दे॒वता॒ फल्गु॑नी । \newline
2. दे॒वता॒ फल्गु॑नी॒ फल्गु॑नी दे॒वता॑ दे॒वता॒ फल्गु॑नी॒ नक्ष॑त्र॒म् नक्ष॑त्र॒म् फल्गु॑नी दे॒वता॑ दे॒वता॒ फल्गु॑नी॒ नक्ष॑त्रम् । \newline
3. फल्गु॑नी॒ नक्ष॑त्र॒म् नक्ष॑त्र॒म् फल्गु॑नी॒ फल्गु॑नी॒ नक्ष॑त्र॒म् भगो॒ भगो॒ नक्ष॑त्र॒म् फल्गु॑नी॒ फल्गु॑नी॒ नक्ष॑त्र॒म् भगः॑ । \newline
4. फल्गु॑नी॒ इति॒ फल्गु॑नी । \newline
5. नक्ष॑त्र॒म् भगो॒ भगो॒ नक्ष॑त्र॒म् नक्ष॑त्र॒म् भगो॑ दे॒वता॑ दे॒वता॒ भगो॒ नक्ष॑त्र॒म् नक्ष॑त्र॒म् भगो॑ दे॒वता᳚ । \newline
6. भगो॑ दे॒वता॑ दे॒वता॒ भगो॒ भगो॑ दे॒वता॒ हस्तो॒ हस्तो॑ दे॒वता॒ भगो॒ भगो॑ दे॒वता॒ हस्तः॑ । \newline
7. दे॒वता॒ हस्तो॒ हस्तो॑ दे॒वता॑ दे॒वता॒ हस्तो॒ नक्ष॑त्र॒म् नक्ष॑त्रꣳ॒॒ हस्तो॑ दे॒वता॑ दे॒वता॒ हस्तो॒ नक्ष॑त्रम् । \newline
8. हस्तो॒ नक्ष॑त्र॒म् नक्ष॑त्रꣳ॒॒ हस्तो॒ हस्तो॒ नक्ष॑त्रꣳ सवि॒ता स॑वि॒ता नक्ष॑त्रꣳ॒॒ हस्तो॒ हस्तो॒ नक्ष॑त्रꣳ सवि॒ता । \newline
9. नक्ष॑त्रꣳ सवि॒ता स॑वि॒ता नक्ष॑त्र॒म् नक्ष॑त्रꣳ सवि॒ता दे॒वता॑ दे॒वता॑ सवि॒ता नक्ष॑त्र॒म् नक्ष॑त्रꣳ सवि॒ता दे॒वता᳚ । \newline
10. स॒वि॒ता दे॒वता॑ दे॒वता॑ सवि॒ता स॑वि॒ता दे॒वता॑ चि॒त्रा चि॒त्रा दे॒वता॑ सवि॒ता स॑वि॒ता दे॒वता॑ चि॒त्रा । \newline
11. दे॒वता॑ चि॒त्रा चि॒त्रा दे॒वता॑ दे॒वता॑ चि॒त्रा नक्ष॑त्र॒म् नक्ष॑त्रम् चि॒त्रा दे॒वता॑ दे॒वता॑ चि॒त्रा नक्ष॑त्रम् । \newline
12. चि॒त्रा नक्ष॑त्र॒म् नक्ष॑त्रम् चि॒त्रा चि॒त्रा नक्ष॑त्र॒ मिन्द्र॒ इन्द्रो॒ नक्ष॑त्रम् चि॒त्रा चि॒त्रा नक्ष॑त्र॒ मिन्द्रः॑ । \newline
13. नक्ष॑त्र॒ मिन्द्र॒ इन्द्रो॒ नक्ष॑त्र॒म् नक्ष॑त्र॒ मिन्द्रो॑ दे॒वता॑ दे॒वतेन्द्रो॒ नक्ष॑त्र॒म् नक्ष॑त्र॒ मिन्द्रो॑ दे॒वता᳚ । \newline
14. इन्द्रो॑ दे॒वता॑ दे॒वतेन्द्र॒ इन्द्रो॑ दे॒वता᳚ स्वा॒ती स्वा॒ती दे॒वतेन्द्र॒ इन्द्रो॑ दे॒वता᳚ स्वा॒ती । \newline
15. दे॒वता᳚ स्वा॒ती स्वा॒ती दे॒वता॑ दे॒वता᳚ स्वा॒ती नक्ष॑त्र॒म् नक्ष॑त्रꣳ स्वा॒ती दे॒वता॑ दे॒वता᳚ स्वा॒ती नक्ष॑त्रम् । \newline
16. स्वा॒ती नक्ष॑त्र॒म् नक्ष॑त्रꣳ स्वा॒ती स्वा॒ती नक्ष॑त्रं ॅवा॒युर् वा॒युर् नक्ष॑त्रꣳ स्वा॒ती स्वा॒ती नक्ष॑त्रं ॅवा॒युः । \newline
17. नक्ष॑त्रं ॅवा॒युर् वा॒युर् नक्ष॑त्र॒म् नक्ष॑त्रं ॅवा॒युर् दे॒वता॑ दे॒वता॑ वा॒युर् नक्ष॑त्र॒म् नक्ष॑त्रं ॅवा॒युर् दे॒वता᳚ । \newline
18. वा॒युर् दे॒वता॑ दे॒वता॑ वा॒युर् वा॒युर् दे॒वता॒ विशा॑खे॒ विशा॑खे दे॒वता॑ वा॒युर् वा॒युर् दे॒वता॒ विशा॑खे । \newline
19. दे॒वता॒ विशा॑खे॒ विशा॑खे दे॒वता॑ दे॒वता॒ विशा॑खे॒ नक्ष॑त्र॒म् नक्ष॑त्रं॒ ॅविशा॑खे दे॒वता॑ दे॒वता॒ विशा॑खे॒ नक्ष॑त्रम् । \newline
20. विशा॑खे॒ नक्ष॑त्र॒म् नक्ष॑त्रं॒ ॅविशा॑खे॒ विशा॑खे॒ नक्ष॑त्र मिन्द्रा॒ग्नी इ॑न्द्रा॒ग्नी नक्ष॑त्रं॒ ॅविशा॑खे॒ विशा॑खे॒ नक्ष॑त्र मिन्द्रा॒ग्नी । \newline
21. विशा॑खे॒ इति॒ वि - शा॒खे॒ । \newline
22. नक्ष॑त्र मिन्द्रा॒ग्नी इ॑न्द्रा॒ग्नी नक्ष॑त्र॒म् नक्ष॑त्र मिन्द्रा॒ग्नी दे॒वता॑ दे॒वते᳚न्द्रा॒ग्नी नक्ष॑त्र॒म् नक्ष॑त्र मिन्द्रा॒ग्नी दे॒वता᳚ । \newline
23. इ॒न्द्रा॒ग्नी दे॒वता॑ दे॒वते᳚न्द्रा॒ग्नी इ॑न्द्रा॒ग्नी दे॒वता॑ ऽनूरा॒धा अ॑नूरा॒धा दे॒वते᳚न्द्रा॒ग्नी इ॑न्द्रा॒ग्नी दे॒वता॑ ऽनूरा॒धाः । \newline
24. इ॒न्द्रा॒ग्नी इती᳚न्द्र - अ॒ग्नी । \newline
25. दे॒वता॑ ऽनूरा॒धा अ॑नूरा॒धा दे॒वता॑ दे॒वता॑ ऽनूरा॒धा नक्ष॑त्र॒म् नक्ष॑त्र मनूरा॒धा दे॒वता॑ दे॒वता॑ ऽनूरा॒धा नक्ष॑त्रम् । \newline
26. अ॒नू॒रा॒धा नक्ष॑त्र॒म् नक्ष॑त्र मनूरा॒धा अ॑नूरा॒धा नक्ष॑त्रम् मि॒त्रो मि॒त्रो नक्ष॑त्र मनूरा॒धा अ॑नूरा॒धा नक्ष॑त्रम् मि॒त्रः । \newline
27. अ॒नू॒रा॒धा इत्य॑नु - रा॒धाः । \newline
28. नक्ष॑त्रम् मि॒त्रो मि॒त्रो नक्ष॑त्र॒म् नक्ष॑त्रम् मि॒त्रो दे॒वता॑ दे॒वता॑ मि॒त्रो नक्ष॑त्र॒म् नक्ष॑त्रम् मि॒त्रो दे॒वता᳚ । \newline
29. मि॒त्रो दे॒वता॑ दे॒वता॑ मि॒त्रो मि॒त्रो दे॒वता॑ रोहि॒णी रो॑हि॒णी दे॒वता॑ मि॒त्रो मि॒त्रो दे॒वता॑ रोहि॒णी । \newline
30. दे॒वता॑ रोहि॒णी रो॑हि॒णी दे॒वता॑ दे॒वता॑ रोहि॒णी नक्ष॑त्र॒म् नक्ष॑त्रꣳ रोहि॒णी दे॒वता॑ दे॒वता॑ रोहि॒णी नक्ष॑त्रम् । \newline
31. रो॒हि॒णी नक्ष॑त्र॒म् नक्ष॑त्रꣳ रोहि॒णी रो॑हि॒णी नक्ष॑त्र॒ मिन्द्र॒ इन्द्रो॒ नक्ष॑त्रꣳ रोहि॒णी रो॑हि॒णी नक्ष॑त्र॒ मिन्द्रः॑ । \newline
32. नक्ष॑त्र॒ मिन्द्र॒ इन्द्रो॒ नक्ष॑त्र॒म् नक्ष॑त्र॒ मिन्द्रो॑ दे॒वता॑ दे॒वतेन्द्रो॒ नक्ष॑त्र॒म् नक्ष॑त्र॒ मिन्द्रो॑ दे॒वता᳚ । \newline
33. इन्द्रो॑ दे॒वता॑ दे॒वतेन्द्र॒ इन्द्रो॑ दे॒वता॑ वि॒चृतौ॑ वि॒चृतौ॑ दे॒वतेन्द्र॒ इन्द्रो॑ दे॒वता॑ वि॒चृतौ᳚ । \newline
34. दे॒वता॑ वि॒चृतौ॑ वि॒चृतौ॑ दे॒वता॑ दे॒वता॑ वि॒चृतौ॒ नक्ष॑त्र॒म् नक्ष॑त्रं ॅवि॒चृतौ॑ दे॒वता॑ दे॒वता॑ वि॒चृतौ॒ नक्ष॑त्रम् । \newline
35. वि॒चृतौ॒ नक्ष॑त्र॒म् नक्ष॑त्रं ॅवि॒चृतौ॑ वि॒चृतौ॒ नक्ष॑त्रम् पि॒तरः॑ पि॒तरो॒ नक्ष॑त्रं ॅवि॒चृतौ॑ वि॒चृतौ॒ नक्ष॑त्रम् पि॒तरः॑ । \newline
36. वि॒चृता॒विति॑ वि - चृतौ᳚ । \newline
37. नक्ष॑त्रम् पि॒तरः॑ पि॒तरो॒ नक्ष॑त्र॒म् नक्ष॑त्रम् पि॒तरो॑ दे॒वता॑ दे॒वता॑ पि॒तरो॒ नक्ष॑त्र॒म् नक्ष॑त्रम् पि॒तरो॑ दे॒वता᳚ । \newline
38. पि॒तरो॑ दे॒वता॑ दे॒वता॑ पि॒तरः॑ पि॒तरो॑ दे॒वता॑ ऽषा॒ढा अ॑षा॒ढा दे॒वता॑ पि॒तरः॑ पि॒तरो॑ दे॒वता॑ ऽषा॒ढाः । \newline
39. दे॒वता॑ ऽषा॒ढा अ॑षा॒ढा दे॒वता॑ दे॒वता॑ ऽषा॒ढा नक्ष॑त्र॒म् नक्ष॑त्र मषा॒ढा दे॒वता॑ दे॒वता॑ ऽषा॒ढा नक्ष॑त्रम् । \newline
40. अ॒षा॒ढा नक्ष॑त्र॒म् नक्ष॑त्र मषा॒ढा अ॑षा॒ढा नक्ष॑त्र॒ माप॒ आपो॒ नक्ष॑त्र मषा॒ढा अ॑षा॒ढा नक्ष॑त्र॒ मापः॑ । \newline
41. नक्ष॑त्र॒ माप॒ आपो॒ नक्ष॑त्र॒म् नक्ष॑त्र॒ मापो॑ दे॒वता॑ दे॒वता ऽऽपो॒ नक्ष॑त्र॒म् नक्ष॑त्र॒ मापो॑ दे॒वता᳚ । \newline
42. आपो॑ दे॒वता॑ दे॒वता ऽऽप॒ आपो॑ दे॒वता॑ ऽषा॒ढा अ॑षा॒ढा दे॒वता ऽऽप॒ आपो॑ दे॒वता॑ ऽषा॒ढाः । \newline
43. दे॒वता॑ ऽषा॒ढा अ॑षा॒ढा दे॒वता॑ दे॒वता॑ ऽषा॒ढा नक्ष॑त्र॒म् नक्ष॑त्र मषा॒ढा दे॒वता॑ दे॒वता॑ ऽषा॒ढा नक्ष॑त्रम् । \newline
44. अ॒षा॒ढा नक्ष॑त्र॒म् नक्ष॑त्र मषा॒ढा अ॑षा॒ढा नक्ष॑त्रं॒ ॅविश्वे॒ विश्वे॒ नक्ष॑त्र मषा॒ढा अ॑षा॒ढा नक्ष॑त्रं॒ ॅविश्वे᳚ । \newline
45. नक्ष॑त्रं॒ ॅविश्वे॒ विश्वे॒ नक्ष॑त्र॒म् नक्ष॑त्रं॒ ॅविश्वे॑ दे॒वा दे॒वा विश्वे॒ नक्ष॑त्र॒म् नक्ष॑त्रं॒ ॅविश्वे॑ दे॒वाः । \newline
46. विश्वे॑ दे॒वा दे॒वा विश्वे॒ विश्वे॑ दे॒वा दे॒वता॑ दे॒वता॑ दे॒वा विश्वे॒ विश्वे॑ दे॒वा दे॒वता᳚ । \newline
47. दे॒वा दे॒वता॑ दे॒वता॑ दे॒वा दे॒वा दे॒वता᳚ श्रो॒णा श्रो॒णा दे॒वता॑ दे॒वा दे॒वा दे॒वता᳚ श्रो॒णा । \newline
48. दे॒वता᳚ श्रो॒णा श्रो॒णा दे॒वता॑ दे॒वता᳚ श्रो॒णा नक्ष॑त्र॒म् नक्ष॑त्रꣳ श्रो॒णा दे॒वता॑ दे॒वता᳚ श्रो॒णा नक्ष॑त्रम् । \newline
49. श्रो॒णा नक्ष॑त्र॒म् नक्ष॑त्रꣳ श्रो॒णा श्रो॒णा नक्ष॑त्रं॒ ॅविष्णु॒र् विष्णु॒र् नक्ष॑त्रꣳ श्रो॒णा 
श्रो॒णा नक्ष॑त्रं॒ ॅविष्णुः॑ । \newline
50. नक्ष॑त्रं॒ ॅविष्णु॒र् विष्णु॒र् नक्ष॑त्र॒म् नक्ष॑त्रं॒ ॅविष्णु॑र् दे॒वता॑ दे॒वता॒ विष्णु॒र् नक्ष॑त्र॒म् नक्ष॑त्रं॒ ॅविष्णु॑र् दे॒वता᳚ । \newline
51. विष्णु॑र् दे॒वता॑ दे॒वता॒ विष्णु॒र् विष्णु॑र् दे॒वता॒ श्रवि॑ष्ठाः॒ श्रवि॑ष्ठा दे॒वता॒ विष्णु॒र् विष्णु॑र् दे॒वता॒ श्रवि॑ष्ठाः । \newline
52. दे॒वता॒ श्रवि॑ष्ठाः॒ श्रवि॑ष्ठा दे॒वता॑ दे॒वता॒ श्रवि॑ष्ठा॒ नक्ष॑त्र॒म् नक्ष॑त्रꣳ॒॒ श्रवि॑ष्ठा दे॒वता॑ दे॒वता॒ श्रवि॑ष्ठा॒ नक्ष॑त्रम् । \newline
53. श्रवि॑ष्ठा॒ नक्ष॑त्र॒म् नक्ष॑त्रꣳ॒॒ श्रवि॑ष्ठाः॒ श्रवि॑ष्ठा॒ नक्ष॑त्रं॒ ॅवस॑वो॒ वस॑वो॒ नक्ष॑त्रꣳ॒॒ श्रवि॑ष्ठाः॒ श्रवि॑ष्ठा॒ नक्ष॑त्रं॒ ॅवस॑वः । \newline
54. नक्ष॑त्रं॒ ॅवस॑वो॒ वस॑वो॒ नक्ष॑त्र॒म् नक्ष॑त्रं॒ ॅवस॑वो दे॒वता॑ दे॒वता॒ वस॑वो॒ नक्ष॑त्र॒म् नक्ष॑त्रं॒ ॅवस॑वो दे॒वता᳚ । \newline
55. वस॑वो दे॒वता॑ दे॒वता॒ वस॑वो॒ वस॑वो दे॒वता॑ श॒तभि॑षख् छ॒तभि॑षग् दे॒वता॒ वस॑वो॒ वस॑वो दे॒वता॑ श॒तभि॑षक् । \newline
\pagebreak
\markright{ TS 4.4.10.3  \hfill https://www.vedavms.in \hfill}

\section{ TS 4.4.10.3 }

\textbf{TS 4.4.10.3 } \newline
\textbf{Samhita Paata} \newline

दे॒वता॑ श॒तभि॑ष॒ङ् नक्ष॑त्र॒मिन्द्रो॑ दे॒वता᳚ प्रोष्ठप॒दा नक्ष॑त्रम॒ज एक॑पाद् दे॒वता᳚ प्रोष्ठप॒दा नक्ष॑त्र॒महि॑र्बु॒द्ध्नियो॑ दे॒वता॑ रे॒वती॒ नक्ष॑त्रं पू॒षा दे॒वता᳚ ऽश्व॒युजौ॒ नक्ष॑त्रम॒श्विनौ॑ दे॒वता॑ ऽप॒भर॑णी॒र्नक्ष॑त्रं ॅय॒मो दे॒वता॑, पू॒र्णा प॒श्चाद्>1, यत् ते॑ दे॒वा अद॑धुः >2 ॥ \newline

\textbf{Pada Paata} \newline

दे॒वता᳚ । श॒तभि॑ष॒गिति॑ श॒त - भि॒ष॒क् । नक्ष॑त्रम् । इन्द्रः॑ । दे॒वता᳚ । प्रो॒ष्ठ॒प॒दा इति॑ प्रोष्ठ - प॒दाः । नक्ष॑त्रम् । अ॒जः । एक॑पा॒दित्येक॑ - पा॒त् । दे॒वता᳚ । प्रो॒ष्ठ॒प॒दा इति॑ प्रोष्ठ - प॒दाः । नक्ष॑त्रम् । अहिः॑ । बु॒द्ध्नियः॑ । दे॒वता᳚ । रे॒वती᳚ । नक्ष॑त्रम् । पू॒षा । दे॒वता᳚ । अ॒श्व॒युजा॒वित्य॑श्व - युजौ᳚ । नक्ष॑त्रम् । अ॒श्विनौ᳚ । दे॒वता᳚ । अ॒प॒भर॑णी॒रित्य॑प - भर॑णीः । नक्ष॑त्रम् । य॒मः । दे॒वता᳚ । पू॒र्णा । प॒श्चात् । यत् । ते॒ । दे॒वाः । अद॑धुः ॥  \newline


\textbf{Krama Paata} \newline

दे॒वता॑ श॒तभि॑षक् । श॒तभि॑ष॒ङ् नक्ष॑त्रम् । श॒तभि॑ष॒गिति॑ श॒त - भि॒ष॒क्॒ । नक्ष॑त्र॒मिन्द्रः॑ । इन्द्रो॑ दे॒वता᳚ । दे॒वता᳚ प्रोष्ठप॒दाः । प्रो॒ष्ठ॒प॒दा नक्ष॑त्रम् । प्रो॒ष्ठ॒प॒दा इति॑ प्रोष्ठ - प॒दाः । नक्ष॑त्रम॒जः । अ॒ज एक॑पात् । एक॑पाद् दे॒वता᳚ । एक॑पा॒दित्येक॑ - पा॒त्॒ । दे॒वता᳚ प्रोष्ठप॒दाः । प्रो॒ष्ठ॒प॒दा नक्ष॑त्रम् । प्रो॒ष्ठ॒प॒दा इति॑ प्रोष्ठ - प॒दाः । नक्ष॑त्र॒महिः॑ । अहि॑र् बु॒ध्नियः॑ । बु॒ध्नियो॑ दे॒वता᳚ । दे॒वता॑ रे॒वती᳚ । रे॒वती॒ नक्ष॑त्रम् । नक्ष॑त्रं पू॒षा । पू॒षा दे॒वता᳚ । दे॒वता᳚ऽश्व॒युजौ᳚ । अ॒श्व॒युजौ॒ नक्ष॑त्रम् । अ॒श्व॒युजा॒वित्य॑श्व - युजौ᳚ । नक्ष॑त्रम॒श्विनौ᳚ । अ॒श्विनौ॑ दे॒वता᳚ । दे॒वता॑ऽप॒भर॑णीः । अ॒प॒भर॑णी॒र् नक्ष॑त्रम् । अ॒प॒भर॑णी॒रित्य॑प - भर॑णीः । नक्ष॑त्रं ॅय॒मः । य॒मो दे॒वता᳚ । दे॒वता॑ पू॒र्णा । पू॒र्णा प॒श्चात् । प॒श्चाद् यत् । यत् ते᳚ । ते॒ दे॒वाः । दे॒वा अद॑धुः । अद॑धु॒रित्यद॑धुः । \newline

\textbf{Jatai Paata} \newline

1. दे॒वता॑ श॒तभि॑षख् छ॒तभि॑षग् दे॒वता॑ दे॒वता॑ श॒तभि॑षक् । \newline
2. श॒तभि॑ष॒ङ् नक्ष॑त्र॒म् नक्ष॑त्रꣳ श॒तभि॑षख् छ॒तभि॑ष॒ङ् नक्ष॑त्रम् । \newline
3. श॒तभि॑ष॒गिति॑ श॒त - भि॒ष॒क् । \newline
4. नक्ष॑त्र॒ मिन्द्र॒ इन्द्रो॒ नक्ष॑त्र॒म् नक्ष॑त्र॒ मिन्द्रः॑ । \newline
5. इन्द्रो॑ दे॒वता॑ दे॒व तेन्द्र॒ इन्द्रो॑ दे॒वता᳚ । \newline
6. दे॒वता᳚ प्रोष्ठप॒दाः प्रो᳚ष्ठप॒दा दे॒वता॑ दे॒वता᳚ प्रोष्ठप॒दाः । \newline
7. प्रो॒ष्ठ॒प॒दा नक्ष॑त्र॒म् नक्ष॑त्रम् प्रोष्ठप॒दाः प्रो᳚ष्ठप॒दा नक्ष॑त्रम् । \newline
8. प्रो॒ष्ठ॒प॒दा इति॑ प्रोष्ठ - प॒दाः । \newline
9. नक्ष॑त्र म॒जो॑ ऽजो नक्ष॑त्र॒म् नक्ष॑त्र म॒जः । \newline
10. अ॒ज एक॑पा॒ देक॑पा द॒जो॑ ऽज एक॑पात् । \newline
11. एक॑पाद् दे॒वता॑ दे॒वतैक॑पा॒ देक॑पाद् दे॒वता᳚ । \newline
12. एक॑पा॒दित्येक॑ - पा॒त् । \newline
13. दे॒वता᳚ प्रोष्ठप॒दाः प्रो᳚ष्ठप॒दा दे॒वता॑ दे॒वता᳚ प्रोष्ठप॒दाः । \newline
14. प्रो॒ष्ठ॒प॒दा नक्ष॑त्र॒म् नक्ष॑त्रम् प्रोष्ठप॒दाः प्रो᳚ष्ठप॒दा नक्ष॑त्रम् । \newline
15. प्रो॒ष्ठ॒प॒दा इति॑ प्रोष्ठ - प॒दाः । \newline
16. नक्ष॑त्र॒ महि॒ रहि॒र् नक्ष॑त्र॒म् नक्ष॑त्र॒ महिः॑ । \newline
17. अहि॑र् बु॒द्ध्नियो॑ बु॒द्ध्नियो ऽहि॒रहि॑र् बु॒द्ध्नियः॑ । \newline
18. बु॒द्ध्नियो॑ दे॒वता॑ दे॒वता॑ बु॒द्ध्नियो॑ बु॒द्ध्नियो॑ दे॒वता᳚ । \newline
19. दे॒वता॑ रे॒वती॑ रे॒वती॑ दे॒वता॑ दे॒वता॑ रे॒वती᳚ । \newline
20. रे॒वती॒ नक्ष॑त्र॒म् नक्ष॑त्रꣳ रे॒वती॑ रे॒वती॒ नक्ष॑त्रम् । \newline
21. नक्ष॑त्रम् पू॒षा पू॒षा नक्ष॑त्र॒म् नक्ष॑त्रम् पू॒षा । \newline
22. पू॒षा दे॒वता॑ दे॒वता॑ पू॒षा पू॒षा दे॒वता᳚ । \newline
23. दे॒वता᳚ ऽश्व॒युजा॑ वश्व॒युजौ॑ दे॒वता॑ दे॒वता᳚ ऽश्व॒युजौ᳚ । \newline
24. अ॒श्व॒युजौ॒ नक्ष॑त्र॒म् नक्ष॑त्र मश्व॒युजा॑ वश्व॒युजौ॒ नक्ष॑त्रम् । \newline
25. अ॒श्व॒युजा॒वित्य॑श्व - युजौ᳚ । \newline
26. नक्ष॑त्र म॒श्विना॑ व॒श्विनौ॒ नक्ष॑त्र॒म् नक्ष॑त्र म॒श्विनौ᳚ । \newline
27. अ॒श्विनौ॑ दे॒वता॑ दे॒वता॒ ऽश्विना॑ व॒श्विनौ॑ दे॒वता᳚ । \newline
28. दे॒वता॑ ऽप॒भर॑णी रप॒भर॑णीर् दे॒वता॑ दे॒वता॑ ऽप॒भर॑णीः । \newline
29. अ॒प॒भर॑णी॒र् नक्ष॑त्र॒म् नक्ष॑त्र मप॒भर॑णी रप॒भर॑णी॒र् नक्ष॑त्रम् । \newline
30. अ॒प॒भर॑णी॒रित्य॑प - भर॑णीः । \newline
31. नक्ष॑त्रं ॅय॒मो य॒मो नक्ष॑त्र॒म् नक्ष॑त्रं ॅय॒मः । \newline
32. य॒मो दे॒वता॑ दे॒वता॑ य॒मो य॒मो दे॒वता᳚ । \newline
33. दे॒वता॑ पू॒र्णा पू॒र्णा दे॒वता॑ दे॒वता॑ पू॒र्णा । \newline
34. पू॒र्णा प॒श्चात् प॒श्चात् पू॒र्णा पू॒र्णा प॒श्चात् । \newline
35. प॒श्चाद् यद् यत् प॒श्चात् प॒श्चाद् यत् । \newline
36. यत् ते॑ ते॒ यद् यत् ते᳚ । \newline
37. ते॒ दे॒वा दे॒वा स्ते॑ ते दे॒वाः । \newline
38. दे॒वा अद॑धु॒ रद॑धुर् दे॒वा दे॒वा अद॑धुः । \newline
39. अद॑धु॒रित्यद॑धुः । \newline

\textbf{Ghana Paata } \newline

1. दे॒वता॑ श॒तभि॑षख् छ॒तभि॑षग् दे॒वता॑ दे॒वता॑ श॒तभि॑ष॒ङ् नक्ष॑त्र॒म् नक्ष॑त्रꣳ श॒तभि॑षग् दे॒वता॑ दे॒वता॑ श॒तभि॑ष॒ङ् नक्ष॑त्रम् । \newline
2. श॒तभि॑ष॒ङ् नक्ष॑त्र॒म् नक्ष॑त्रꣳ श॒तभि॑षख् छ॒तभि॑ष॒ङ् नक्ष॑त्र॒ मिन्द्र॒ इन्द्रो॒ नक्ष॑त्रꣳ श॒तभि॑षख् छ॒तभि॑ष॒ङ् नक्ष॑त्र॒ मिन्द्रः॑ । \newline
3. श॒तभि॑ष॒गिति॑ श॒त - भि॒ष॒क् । \newline
4. नक्ष॑त्र॒ मिन्द्र॒ इन्द्रो॒ नक्ष॑त्र॒म् नक्ष॑त्र॒ मिन्द्रो॑ दे॒वता॑ दे॒वतेन्द्रो॒ नक्ष॑त्र॒म् नक्ष॑त्र॒ मिन्द्रो॑ दे॒वता᳚ । \newline
5. इन्द्रो॑ दे॒वता॑ दे॒वतेन्द्र॒ इन्द्रो॑ दे॒वता᳚ प्रोष्ठप॒दाः प्रो᳚ष्ठप॒दा दे॒वतेन्द्र॒ इन्द्रो॑ दे॒वता᳚ प्रोष्ठप॒दाः । \newline
6. दे॒वता᳚ प्रोष्ठप॒दाः प्रो᳚ष्ठप॒दा दे॒वता॑ दे॒वता᳚ प्रोष्ठप॒दा नक्ष॑त्र॒म् नक्ष॑त्रम् प्रोष्ठप॒दा दे॒वता॑ दे॒वता᳚ प्रोष्ठप॒दा नक्ष॑त्रम् । \newline
7. प्रो॒ष्ठ॒प॒दा नक्ष॑त्र॒म् नक्ष॑त्रम् प्रोष्ठप॒दाः प्रो᳚ष्ठप॒दा नक्ष॑त्र म॒जो॑ ऽजो नक्ष॑त्रम् प्रोष्ठप॒दाः प्रो᳚ष्ठप॒दा नक्ष॑त्र म॒जः । \newline
8. प्रो॒ष्ठ॒प॒दा इति॑ प्रोष्ठ - प॒दाः । \newline
9. नक्ष॑त्र म॒जो॑ ऽजो नक्ष॑त्र॒म् नक्ष॑त्र म॒ज एक॑पा॒ देक॑पा द॒जो नक्ष॑त्र॒म् नक्ष॑त्र म॒ज एक॑पात् । \newline
10. अ॒ज एक॑पा॒ देक॑पा द॒जो॑ ऽज एक॑पाद् दे॒वता॑ दे॒वतैक॑पा द॒जो॑ ऽज एक॑पाद् दे॒वता᳚ । \newline
11. एक॑पाद् दे॒वता॑ दे॒वतैक॑पा॒ देक॑पाद् दे॒वता᳚ प्रोष्ठप॒दाः प्रो᳚ष्ठप॒दा दे॒वतैक॑पा॒ देक॑पाद् दे॒वता᳚ प्रोष्ठप॒दाः । \newline
12. एक॑पा॒दित्येक॑ - पा॒त् । \newline
13. दे॒वता᳚ प्रोष्ठप॒दाः प्रो᳚ष्ठप॒दा दे॒वता॑ दे॒वता᳚ प्रोष्ठप॒दा नक्ष॑त्र॒म् नक्ष॑त्रम् प्रोष्ठप॒दा दे॒वता॑ दे॒वता᳚ प्रोष्ठप॒दा नक्ष॑त्रम् । \newline
14. प्रो॒ष्ठ॒प॒दा नक्ष॑त्र॒म् नक्ष॑त्रम् प्रोष्ठप॒दाः प्रो᳚ष्ठप॒दा नक्ष॑त्र॒ महि॒ रहि॒र् नक्ष॑त्रम् प्रोष्ठप॒दाः प्रो᳚ष्ठप॒दा नक्ष॑त्र॒ महिः॑ । \newline
15. प्रो॒ष्ठ॒प॒दा इति॑ प्रोष्ठ - प॒दाः । \newline
16. नक्ष॑त्र॒ महि॒ रहि॒र् नक्ष॑त्र॒म् नक्ष॑त्र॒ महि॑र् बु॒द्ध्नियो॑ बु॒द्ध्नियो ऽहि॒र् नक्ष॑त्र॒म् नक्ष॑त्र॒ महि॑र् बु॒द्ध्नियः॑ । \newline
17. अहि॑र् बु॒द्ध्नियो॑ बु॒द्ध्नियो ऽहि॒ रहि॑र् बु॒द्ध्नियो॑ दे॒वता॑ दे॒वता॑ बु॒द्ध्नियो ऽहि॒ रहि॑र् बु॒द्ध्नियो॑ दे॒वता᳚ । \newline
18. बु॒द्ध्नियो॑ दे॒वता॑ दे॒वता॑ बु॒द्ध्नियो॑ बु॒द्ध्नियो॑ दे॒वता॑ रे॒वती॑ रे॒वती॑ दे॒वता॑ बु॒द्ध्नियो॑ बु॒द्ध्नियो॑ दे॒वता॑ रे॒वती᳚ । \newline
19. दे॒वता॑ रे॒वती॑ रे॒वती॑ दे॒वता॑ दे॒वता॑ रे॒वती॒ नक्ष॑त्र॒म् नक्ष॑त्रꣳ रे॒वती॑ दे॒वता॑ दे॒वता॑ रे॒वती॒ नक्ष॑त्रम् । \newline
20. रे॒वती॒ नक्ष॑त्र॒म् नक्ष॑त्रꣳ रे॒वती॑ रे॒वती॒ नक्ष॑त्रम् पू॒षा पू॒षा नक्ष॑त्रꣳ रे॒वती॑ रे॒वती॒ नक्ष॑त्रम् पू॒षा । \newline
21. नक्ष॑त्रम् पू॒षा पू॒षा नक्ष॑त्र॒म् नक्ष॑त्रम् पू॒षा दे॒वता॑ दे॒वता॑ पू॒षा नक्ष॑त्र॒म् नक्ष॑त्रम् पू॒षा दे॒वता᳚ । \newline
22. पू॒षा दे॒वता॑ दे॒वता॑ पू॒षा पू॒षा दे॒वता᳚ ऽश्व॒युजा॑ वश्व॒युजौ॑ दे॒वता॑ पू॒षा पू॒षा दे॒वता᳚ ऽश्व॒युजौ᳚ । \newline
23. दे॒वता᳚ ऽश्व॒युजा॑ वश्व॒युजौ॑ दे॒वता॑ दे॒वता᳚ ऽश्व॒युजौ॒ नक्ष॑त्र॒म् नक्ष॑त्र मश्व॒युजौ॑ दे॒वता॑ दे॒वता᳚ ऽश्व॒युजौ॒ नक्ष॑त्रम् । \newline
24. अ॒श्व॒युजौ॒ नक्ष॑त्र॒म् नक्ष॑त्र मश्व॒युजा॑ वश्व॒युजौ॒ नक्ष॑त्र म॒श्विना॑ व॒श्विनौ॒ नक्ष॑त्र मश्व॒युजा॑ वश्व॒युजौ॒ नक्ष॑त्र म॒श्विनौ᳚ । \newline
25. अ॒श्व॒युजा॒वित्य॑श्व - युजौ᳚ । \newline
26. नक्ष॑त्र म॒श्विना॑ व॒श्विनौ॒ नक्ष॑त्र॒म् नक्ष॑त्र म॒श्विनौ॑ दे॒वता॑ दे॒वता॒ ऽश्विनौ॒ नक्ष॑त्र॒न् 
नक्ष॑त्र म॒श्विनौ॑ दे॒वता᳚ । \newline
27. अ॒श्विनौ॑ दे॒वता॑ दे॒वता॒ ऽश्विना॑ व॒श्विनौ॑ दे॒वता॑ ऽप॒भर॑णी रप॒भर॑णीर् दे॒वता॒ ऽश्विना॑ व॒श्विनौ॑ दे॒वता॑ ऽप॒भर॑णीः । \newline
28. दे॒वता॑ ऽप॒भर॑णी रप॒भर॑णीर् दे॒वता॑ दे॒वता॑ ऽप॒भर॑णी॒र् नक्ष॑त्र॒म् नक्ष॑त्र मप॒भर॑णीर् दे॒वता॑ दे॒वता॑ ऽप॒भर॑णी॒र् नक्ष॑त्रम् । \newline
29. अ॒प॒भर॑णी॒र् नक्ष॑त्र॒म् नक्ष॑त्र मप॒भर॑णी रप॒भर॑णी॒र् नक्ष॑त्रं ॅय॒मो य॒मो नक्ष॑त्र मप॒भर॑णी रप॒भर॑णी॒र् नक्ष॑त्रं ॅय॒मः । \newline
30. अ॒प॒भर॑णी॒रित्य॑प - भर॑णीः । \newline
31. नक्ष॑त्रं ॅय॒मो य॒मो नक्ष॑त्र॒म् नक्ष॑त्रं ॅय॒मो दे॒वता॑ दे॒वता॑ य॒मो नक्ष॑त्र॒म् नक्ष॑त्रं ॅय॒मो दे॒वता᳚ । \newline
32. य॒मो दे॒वता॑ दे॒वता॑ य॒मो य॒मो दे॒वता॑ पू॒र्णा पू॒र्णा दे॒वता॑ य॒मो य॒मो दे॒वता॑ पू॒र्णा । \newline
33. दे॒वता॑ पू॒र्णा पू॒र्णा दे॒वता॑ दे॒वता॑ पू॒र्णा प॒श्चात् प॒श्चात् पू॒र्णा दे॒वता॑ दे॒वता॑ पू॒र्णा प॒श्चात् । \newline
34. पू॒र्णा प॒श्चात् प॒श्चात् पू॒र्णा पू॒र्णा प॒श्चाद् यद् यत् प॒श्चात् पू॒र्णा पू॒र्णा प॒श्चाद् यत् । \newline
35. प॒श्चाद् यद् यत् प॒श्चात् प॒श्चाद् यत् ते॑ ते॒ यत् प॒श्चात् प॒श्चाद् यत् ते᳚ । \newline
36. यत् ते॑ ते॒ यद् यत् ते॑ दे॒वा दे॒वा स्ते॒ यद् यत् ते॑ दे॒वाः । \newline
37. ते॒ दे॒वा दे॒वा स्ते॑ ते दे॒वा अद॑धु॒ रद॑धुर् दे॒वा स्ते॑ ते दे॒वा अद॑धुः । \newline
38. दे॒वा अद॑धु॒ रद॑धुर् दे॒वा दे॒वा अद॑धुः । \newline
39. अद॑धु॒रित्यद॑धुः । \newline
\pagebreak
\markright{ TS 4.4.11.1  \hfill https://www.vedavms.in \hfill}

\section{ TS 4.4.11.1 }

\textbf{TS 4.4.11.1 } \newline
\textbf{Samhita Paata} \newline

मधु॑श्च॒ माध॑वश्च॒ वास॑न्तिकावृ॒तू शु॒क्रश्च॒ शुचि॑श्च॒ ग्रैष्मा॑वृ॒तू नभ॑श्च नभ॒स्य॑श्च॒ वार्.षि॑कावृ॒तू इ॒षश्चो॒र्जश्च॑ शार॒दावृ॒तू सह॑श्च सह॒स्य॑श्च॒ हैम॑न्तिकावृ॒तू तप॑श्च तप॒स्य॑श्च शैशि॒रावृ॒तू अ॒ग्नेर॑न्तः श्ले॒षो॑ऽसि॒ कल्पे॑तां॒ द्यावा॑पृथि॒वी कल्प॑न्ता॒माप॒ ओष॑धीः॒ कल्प॑न्ताम॒ग्नयः॒ पृथ॒ङ्मम॒ ज्यैष्ठ्य॑य॒ सव्र॑ता॒- [  ] \newline

\textbf{Pada Paata} \newline

मधुः॑ । च॒ । माध॑वः । च॒ । वास॑न्तिकौ । ऋ॒तू इति॑ । शु॒क्रः । च॒ । शुचिः॑ । च॒ । ग्रैष्मौ᳚ । ऋ॒तू इति॑ । नभः॑ । च॒ । न॒भ॒स्यः॑ । च॒ । वार्.षि॑कौ । ऋ॒तू इति॑ । इ॒षः । च॒ । ऊ॒र्जः । च॒ । शा॒र॒दौ । ऋ॒तू इति॑ । सहः॑ । च॒ । स॒ह॒स्यः॑ । च॒ । हैम॑न्तिकौ । ऋ॒तू इति॑ । तपः॑ । च॒ । त॒प॒स्यः॑ । च॒ । शै॒शि॒रौ । ऋ॒तू इति॑ । अ॒ग्नेः । अ॒न्तः॒श्ले॒ष इत्य॑न्तः - श्ले॒षः । अ॒सि॒ । कल्पे॑ताम् । द्यावा॑पृथि॒वी इति॒ द्यावा᳚ - पृ॒थि॒वी । कल्प॑न्ताम् । आपः॑ । ओष॑धीः । कल्प॑न्ताम् । अ॒ग्नयः॑ । पृथ॑क् । मम॑ । ज्यैष्ठ्य॑य । सव्र॑ता॒ इति॒ स-व्र॒ताः॒ ।  \newline


\textbf{Krama Paata} \newline

मधु॑श्च । च॒ माध॑वः । माध॑वश्च । च॒ वास॑न्तिकौ । वास॑न्तिकावृ॒तू । ऋ॒तू शु॒क्रः । ऋ॒तू इत्यृ॒तू । शु॒क्रश्च॑ । च॒ शुचिः॑ । शुचि॑श्च । च॒ ग्रैष्मौ᳚ । ग्रैष्मा॑वृ॒तू । ऋ॒तू नभः॑ । ऋ॒तू इत्यृ॒तू । नभ॑श्च । च॒ न॒भ॒स्यः॑ । न॒भ॒स्य॑श्च । च॒ वार्.षि॑कौ । वार्.षि॑कावृ॒तू । ऋ॒तू इ॒षः । ऋ॒तू इत्यृ॒तू । इ॒षश्च॑ । चो॒र्जः । ऊ॒र्जश्च॑ । च॒ शा॒र॒दौ । शा॒र॒दावृ॒तू । ऋ॒तू सहः॑ । ऋ॒तू इत्यृ॒तू । सह॑श्च । च॒ स॒ह॒स्यः॑ । स॒ह॒स्य॑श्च । च॒ हैम॑न्तिकौ । हैम॑न्तिकावृ॒तू । ऋ॒तू तपः॑ । ऋ॒तू इत्यृ॒तू । तप॑श्च । च॒ त॒प॒स्यः॑ । त॒प॒स्य॑श्च । च॒ शै॒शि॒रौ । शै॒शि॒रावृ॒तू । ऋ॒तू अ॒ग्नेः । ऋ॒तू इत्यृ॒तू । अ॒ग्नेर॑न्तःश्ले॒षः । अ॒न्तः॒श्ले॒षो॑ऽसि । अ॒न्तः॒श्ले॒ष इत्य॑न्तः - श्ले॒षः । अ॒सि॒ कल्पे॑ताम् । कल्पे॑ता॒म् द्यावा॑पृथि॒वी । द्यावा॑पृथि॒वी कल्प॑न्ताम् । द्यावा॑पृथि॒वी इति॒ द्यावा᳚ - पृ॒थि॒वी । कल्प॑न्ता॒मापः॑ । आप॒ ओष॑धीः । ओष॑धीः॒ कल्प॑न्ताम् । कल्प॑न्ताम॒ग्नयः॑ । अ॒ग्नयः॒ पृथ॑क् । पृथ॒ङ् मम॑ । मम॒ ज्यैष्ठ्या॑य । ज्यैष्ठ्या॑य॒ सव्र॑ताः । सव्र॑ता॒ ये । सव्र॑ता॒ इति॒ स - व्र॒ताः॒ \newline

\textbf{Jatai Paata} \newline

1. मधु॑श्च च॒ मधु॒र् मधु॑श्च । \newline
2. च॒ माध॑वो॒ माध॑वश्च च॒ माध॑वः । \newline
3. माध॑वश्च च॒ माध॑वो॒ माध॑वश्च । \newline
4. च॒ वास॑न्तिकौ॒ वास॑न्तिकौ च च॒ वास॑न्तिकौ । \newline
5. वास॑न्तिका वृ॒तू ऋ॒तू वास॑न्तिकौ॒ वास॑न्तिका वृ॒तू । \newline
6. ऋ॒तू शु॒क्रः शु॒क्र ऋ॒तू ऋ॒तू शु॒क्रः । \newline
7. ऋ॒तू इत्यृ॒तू । \newline
8. शु॒क्रश्च॑ च शु॒क्रः शु॒क्रश्च॑ । \newline
9. च॒ शुचिः॒ शुचि॑श्च च॒ शुचिः॑ । \newline
10. शुचि॑श्च च॒ शुचिः॒ शुचि॑श्च । \newline
11. च॒ ग्रैष्मौ॒ ग्रैष्मौ॑ च च॒ ग्रैष्मौ᳚ । \newline
12. ग्रैष्मा॑ वृ॒तू ऋ॒तू ग्रैष्मौ॒ ग्रैष्मा॑ वृ॒तू । \newline
13. ऋ॒तू नभो॒ नभ॑ ऋ॒तू ऋ॒तू नभः॑ । \newline
14. ऋ॒तू इत्यृ॒तू । \newline
15. नभ॑श्च च॒ नभो॒ नभ॑श्च । \newline
16. च॒ न॒भ॒स्यो॑ नभ॒स्य॑श्च च नभ॒स्यः॑ । \newline
17. न॒भ॒स्य॑श्च च नभ॒स्यो॑ नभ॒स्य॑श्च । \newline
18. च॒ वार्.षि॑कौ॒ वार्.षि॑कौ च च॒ वार्.षि॑कौ । \newline
19. वार्.षि॑का वृ॒तू ऋ॒तू वार्.षि॑कौ॒ वार्.षि॑का वृ॒तू । \newline
20. ऋ॒तू इ॒ष इ॒ष ऋ॒तू ऋ॒तू इ॒षः । \newline
21. ऋ॒तू इत्यृ॒तू । \newline
22. इ॒षश्च॑ चे॒ ष इ॒षश्च॑ । \newline
23. चो॒र्ज ऊ॒र्जश्च॑ चो॒र्जः । \newline
24. ऊ॒र्जश्च॑ चो॒र्ज ऊ॒र्जश्च॑ । \newline
25. च॒ शा॒र॒दौ शा॑र॒दौ च॑ च शार॒दौ । \newline
26. शा॒र॒दा वृ॒तू ऋ॒तू शा॑र॒दौ शा॑र॒दा वृ॒तू । \newline
27. ऋ॒तू सहः॒ सह॑ ऋ॒तू ऋ॒तू सहः॑ । \newline
28. ऋ॒तू इत्यृ॒तू । \newline
29. सह॑श्च च॒ सहः॒ सह॑श्च । \newline
30. च॒ स॒ह॒स्यः॑ सह॒स्य॑श्च च सह॒स्यः॑ । \newline
31. स॒ह॒स्य॑श्च च सह॒स्यः॑ सह॒स्य॑श्च । \newline
32. च॒ हैम॑न्तिकौ॒ हैम॑न्तिकौ च च॒ हैम॑न्तिकौ । \newline
33. हैम॑न्तिका वृ॒तू ऋ॒तू हैम॑न्तिकौ॒ हैम॑न्तिका वृ॒तू । \newline
34. ऋ॒तू तप॒ स्तप॑ ऋ॒तू ऋ॒तू तपः॑ । \newline
35. ऋ॒तू इत्यृ॒तू । \newline
36. तप॑श्च च॒ तप॒ स्तप॑श्च । \newline
37. च॒ त॒प॒स्य॑ स्तप॒स्य॑ श्च च तप॒स्यः॑ । \newline
38. त॒प॒स्य॑श्च च तप॒स्य॑ स्तप॒स्य॑श्च । \newline
39. च॒ शै॒शि॒रौ शै॑शि॒रौ च॑ च शैशि॒रौ । \newline
40. शै॒शि॒रा वृ॒तू ऋ॒तू शै॑शि॒रौ शै॑शि॒रा वृ॒तू । \newline
41. ऋ॒तू अ॒ग्ने र॒ग्नेर्. ऋ॒तू ऋ॒तू अ॒ग्नेः । \newline
42. ऋ॒तू इत्यृ॒तू । \newline
43. अ॒ग्ने र॑न्तःश्ले॒षो᳚ ऽन्तःश्ले॒षो᳚ ऽग्ने र॒ग्ने र॑न्तःश्ले॒षः । \newline
44. अ॒न्तः॒श्ले॒षो᳚ ऽस्य स्यन्तःश्ले॒षो᳚ ऽन्तःश्ले॒षो॑ ऽसि । \newline
45. अ॒न्तः॒श्ले॒ष इत्य॑न्तः - श्ले॒षः । \newline
46. अ॒सि॒ कल्पे॑ता॒म् कल्पे॑ता मस्यसि॒ कल्पे॑ताम् । \newline
47. कल्पे॑ता॒म् द्यावा॑पृथि॒वी द्यावा॑पृथि॒वी कल्पे॑ता॒म् कल्पे॑ता॒म् द्यावा॑पृथि॒वी । \newline
48. द्यावा॑पृथि॒वी कल्प॑न्ता॒म् कल्प॑न्ता॒म् द्यावा॑पृथि॒वी द्यावा॑पृथि॒वी कल्प॑न्ताम् । \newline
49. द्यावा॑पृथि॒वी इति॒ द्यावा᳚ - पृ॒थि॒वी । \newline
50. कल्प॑न्ता॒ माप॒ आपः॒ कल्प॑न्ता॒म् कल्प॑न्ता॒ मापः॑ । \newline
51. आप॒ ओष॑धी॒ रोष॑धी॒ राप॒ आप॒ ओष॑धीः । \newline
52. ओष॑धीः॒ कल्प॑न्ता॒म् कल्प॑न्ता॒ मोष॑धी॒ रोष॑धीः॒ कल्प॑न्ताम् । \newline
53. कल्प॑न्ता म॒ग्नयो॒ ऽग्नयः॒ कल्प॑न्ता॒म् कल्प॑न्ता म॒ग्नयः॑ । \newline
54. अ॒ग्नयः॒ पृथ॒क् पृथ॑ ग॒ग्नयो॒ ऽग्नयः॒ पृथ॑क् । \newline
55. पृथ॒ङ् मम॒ मम॒ पृथ॒क् पृथ॒ङ् मम॑ । \newline
56. मम॒ ज्यैष्ठ्या॑य॒ ज्यैष्ठ्या॑य॒ मम॒ मम॒ ज्यैष्ठ्या॑य । \newline
57. ज्यैष्ठ्या॑य॒ सव्र॑ताः॒ सव्र॑ता॒ ज्यैष्ठ्या॑य॒ ज्यैष्ठ्या॑य॒ सव्र॑ताः । \newline
58. सव्र॑ता॒ ये ये सव्र॑ताः॒ सव्र॑ता॒ ये । \newline
59. सव्र॑ता॒ इति॒ स - व्र॒ताः॒ । \newline

\textbf{Ghana Paata } \newline

1. मधु॑श्च च॒ मधु॒र् मधु॑श्च॒ माध॑वो॒ माध॑वश्च॒ मधु॒र् मधु॑श्च॒ माध॑वः । \newline
2. च॒ माध॑वो॒ माध॑वश्च च॒ माध॑वश्च च॒ माध॑वश्च च॒ माध॑वश्च । \newline
3. माध॑वश्च च॒ माध॑वो॒ माध॑वश्च॒ वास॑न्तिकौ॒ वास॑न्तिकौ च॒ माध॑वो॒ माध॑वश्च॒ वास॑न्तिकौ । \newline
4. च॒ वास॑न्तिकौ॒ वास॑न्तिकौ च च॒ वास॑न्तिका वृ॒तू ऋ॒तू वास॑न्तिकौ च च॒ वास॑न्तिका वृ॒तू । \newline
5. वास॑न्तिका वृ॒तू ऋ॒तू वास॑न्तिकौ॒ वास॑न्तिका वृ॒तू शु॒क्रः शु॒क्र ऋ॒तू वास॑न्तिकौ॒ वास॑न्तिका वृ॒तू शु॒क्रः । \newline
6. ऋ॒तू शु॒क्रः शु॒क्र ऋ॒तू ऋ॒तू शु॒क्रश्च॑ च शु॒क्र ऋ॒तू ऋ॒तू शु॒क्रश्च॑ । \newline
7. ऋ॒तू इत्यृ॒तू । \newline
8. शु॒क्रश्च॑ च शु॒क्रः शु॒क्रश्च॒ शुचिः॒ शुचि॑श्च शु॒क्रः शु॒क्रश्च॒ शुचिः॑ । \newline
9. च॒ शुचिः॒ शुचि॑श्च च॒ शुचि॑श्च च॒ शुचि॑श्च च॒ शुचि॑श्च । \newline
10. शुचि॑श्च च॒ शुचिः॒ शुचि॑श्च॒ ग्रैष्मौ॒ ग्रैष्मौ॑ च॒ शुचिः॒ शुचि॑श्च॒ ग्रैष्मौ᳚ । \newline
11. च॒ ग्रैष्मौ॒ ग्रैष्मौ॑ च च॒ ग्रैष्मा॑ वृ॒तू ऋ॒तू ग्रैष्मौ॑ च च॒ ग्रैष्मा॑ वृ॒तू । \newline
12. ग्रैष्मा॑ वृ॒तू ऋ॒तू ग्रैष्मौ॒ ग्रैष्मा॑ वृ॒तू नभो॒ नभ॑ ऋ॒तू ग्रैष्मौ॒ ग्रैष्मा॑ वृ॒तू नभः॑ । \newline
13. ऋ॒तू नभो॒ नभ॑ ऋ॒तू ऋ॒तू नभ॑श्च च॒ नभ॑ ऋ॒तू ऋ॒तू नभ॑श्च । \newline
14. ऋ॒तू इत्यृ॒तू । \newline
15. नभ॑श्च च॒ नभो॒ नभ॑श्च नभ॒स्यो॑ नभ॒स्य॑श्च॒ नभो॒ नभ॑श्च नभ॒स्यः॑ । \newline
16. च॒ न॒भ॒स्यो॑ नभ॒स्य॑श्च च नभ॒स्य॑श्च च नभ॒स्य॑श्च च नभ॒स्य॑श्च । \newline
17. न॒भ॒स्य॑श्च च नभ॒स्यो॑ नभ॒स्य॑श्च॒ वार्.षि॑कौ॒ वार्.षि॑कौ च नभ॒स्यो॑ नभ॒स्य॑श्च॒ वार्.षि॑कौ । \newline
18. च॒ वार्.षि॑कौ॒ वार्.षि॑कौ च च॒ वार्.षि॑का वृ॒तू ऋ॒तू वार्.षि॑कौ च च॒ वार्.षि॑का वृ॒तू । \newline
19. वार्.षि॑का वृ॒तू ऋ॒तू वार्.षि॑कौ॒ वार्.षि॑का वृ॒तू इ॒ष इ॒ष ऋ॒तू वार्.षि॑कौ॒ वार्.षि॑का वृ॒तू इ॒षः । \newline
20. ऋ॒तू इ॒ष इ॒ष ऋ॒तू ऋ॒तू इ॒षश्च॑ चे॒ ष ऋ॒तू ऋ॒तू इ॒षश्च॑ । \newline
21. ऋ॒तू इत्यृ॒तू । \newline
22. इ॒षश्च॑ चे॒ ष इ॒ष श्चो॒र्ज ऊ॒र्ज श्चे॒ष इ॒ष श्चो॒र्जः । \newline
23. चो॒र्ज ऊ॒र्जश्च॑ चो॒र्जश्च॑ चो॒र्जश्च॑ चो॒र्जश्च॑ । \newline
24. ऊ॒र्जश्च॑ चो॒र्ज ऊ॒र्जश्च॑ शार॒दौ शा॑र॒दौ चो॒र्ज ऊ॒र्जश्च॑ शार॒दौ । \newline
25. च॒ शा॒र॒दौ शा॑र॒दौ च॑ च शार॒दा वृ॒तू ऋ॒तू शा॑र॒दौ च॑ च शार॒दा वृ॒तू । \newline
26. शा॒र॒दा वृ॒तू ऋ॒तू शा॑र॒दौ शा॑र॒दा वृ॒तू सहः॒ सह॑ ऋ॒तू शा॑र॒दौ शा॑र॒दा वृ॒तू सहः॑ । \newline
27. ऋ॒तू सहः॒ सह॑ ऋ॒तू ऋ॒तू सह॑श्च च॒ सह॑ ऋ॒तू ऋ॒तू सह॑श्च । \newline
28. ऋ॒तू इत्यृ॒तू । \newline
29. सह॑श्च च॒ सहः॒ सह॑श्च सह॒स्यः॑ सह॒स्य॑श्च॒ सहः॒ सह॑श्च सह॒स्यः॑ । \newline
30. च॒ स॒ह॒स्यः॑ सह॒स्य॑श्च च सह॒स्य॑श्च च सह॒स्य॑श्च च सह॒स्य॑श्च । \newline
31. स॒ह॒स्य॑श्च च सह॒स्यः॑ सह॒स्य॑श्च॒ हैम॑न्तिकौ॒ हैम॑न्तिकौ च सह॒स्यः॑ सह॒स्य॑श्च॒ हैम॑न्तिकौ । \newline
32. च॒ हैम॑न्तिकौ॒ हैम॑न्तिकौ च च॒ हैम॑न्तिका वृ॒तू ऋ॒तू हैम॑न्तिकौ च च॒ हैम॑न्तिका वृ॒तू । \newline
33. हैम॑न्तिका वृ॒तू ऋ॒तू हैम॑न्तिकौ॒ हैम॑न्तिका वृ॒तू तप॒ स्तप॑ ऋ॒तू हैम॑न्तिकौ॒ हैम॑न्तिका वृ॒तू तपः॑ । \newline
34. ऋ॒तू तप॒ स्तप॑ ऋ॒तू ऋ॒तू तप॑श्च च॒ तप॑ ऋ॒तू ऋ॒तू तप॑श्च । \newline
35. ऋ॒तू इत्यृ॒तू । \newline
36. तप॑श्च च॒ तप॒ स्तप॑श्च तप॒स्य॑ स्तप॒स्य॑श्च॒ तप॒ स्तप॑श्च तप॒स्यः॑ । \newline
37. च॒ त॒प॒स्य॑ स्तप॒स्य॑श्च च तप॒स्य॑श्च च तप॒स्य॑श्च च तप॒स्य॑श्च । \newline
38. त॒प॒स्य॑श्च च तप॒स्य॑ स्तप॒स्य॑श्च शैशि॒रौ शै॑शि॒रौ च॑ तप॒स्य॑ स्तप॒स्य॑श्च शैशि॒रौ । \newline
39. च॒ शै॒शि॒रौ शै॑शि॒रौ च॑ च शैशि॒रा वृ॒तू ऋ॒तू शै॑शि॒रौ च॑ च शैशि॒रा वृ॒तू । \newline
40. शै॒शि॒रा वृ॒तू ऋ॒तू शै॑शि॒रौ शै॑शि॒रा वृ॒तू अ॒ग्ने र॒ग्नेर्. ऋ॒तू शै॑शि॒रौ शै॑शि॒रा वृ॒तू अ॒ग्नेः । \newline
41. ऋ॒तू अ॒ग्ने र॒ग्नेर्. ऋ॒तू ऋ॒तू अ॒ग्ने र॑न्तःश्ले॒षो᳚ ऽन्तःश्ले॒षो᳚ ऽग्नेर्. ऋ॒तू ऋ॒तू अ॒ग्ने र॑न्तःश्ले॒षः । \newline
42. ऋ॒तू इत्यृ॒तू । \newline
43. अ॒ग्ने र॑न्तःश्ले॒षो᳚ ऽन्तःश्ले॒षो᳚ ऽग्ने र॒ग्ने र॑न्तःश्ले॒षो᳚ ऽस्य स्यन्तःश्ले॒षो᳚ ऽग्ने र॒ग्ने र॑न्तःश्ले॒षो॑ ऽसि । \newline
44. अ॒न्तः॒श्ले॒षो᳚ ऽस्य स्यन्तःश्ले॒षो᳚ ऽन्तःश्ले॒षो॑ ऽसि॒ कल्पे॑ता॒म् कल्पे॑ता मस्यन्तःश्ले॒षो᳚ ऽन्तःश्ले॒षो॑ ऽसि॒ कल्पे॑ताम् । \newline
45. अ॒न्तः॒श्ले॒ष इत्य॑न्तः - श्ले॒षः । \newline
46. अ॒सि॒ कल्पे॑ता॒म् कल्पे॑ता मस्यसि॒ कल्पे॑ता॒म् द्यावा॑पृथि॒वी द्यावा॑पृथि॒वी कल्पे॑ता मस्यसि॒ कल्पे॑ता॒म् द्यावा॑पृथि॒वी । \newline
47. कल्पे॑ता॒म् द्यावा॑पृथि॒वी द्यावा॑पृथि॒वी कल्पे॑ता॒म् कल्पे॑ता॒म् द्यावा॑पृथि॒वी कल्प॑न्ता॒म् कल्प॑न्ता॒म् द्यावा॑पृथि॒वी कल्पे॑ता॒म् कल्पे॑ता॒म् द्यावा॑पृथि॒वी कल्प॑न्ताम् । \newline
48. द्यावा॑पृथि॒वी कल्प॑न्ता॒म् कल्प॑न्ता॒म् द्यावा॑पृथि॒वी द्यावा॑पृथि॒वी कल्प॑न्ता॒ माप॒ आपः॒ कल्प॑न्ता॒म् 
द्यावा॑पृथि॒वी द्यावा॑पृथि॒वी कल्प॑न्ता॒ मापः॑ । \newline
49. द्यावा॑पृथि॒वी इति॒ द्यावा᳚ - पृ॒थि॒वी । \newline
50. कल्प॑न्ता॒ माप॒ आपः॒ कल्प॑न्ता॒म् कल्प॑न्ता॒ माप॒ ओष॑धी॒ रोष॑धी॒ रापः॒ कल्प॑न्ता॒म् कल्प॑न्ता॒ माप॒ ओष॑धीः । \newline
51. आप॒ ओष॑धी॒ रोष॑धी॒ राप॒ आप॒ ओष॑धीः॒ कल्प॑न्ता॒म् कल्प॑न्ता॒ मोष॑धी॒ राप॒ आप॒ ओष॑धीः॒ कल्प॑न्ताम् । \newline
52. ओष॑धीः॒ कल्प॑न्ता॒म् कल्प॑न्ता॒ मोष॑धी॒ रोष॑धीः॒ कल्प॑न्ता म॒ग्नयो॒ ऽग्नयः॒ कल्प॑न्ता॒ मोष॑धी॒ रोष॑धीः॒ कल्प॑न्ता म॒ग्नयः॑ । \newline
53. कल्प॑न्ता म॒ग्नयो॒ ऽग्नयः॒ कल्प॑न्ता॒म् कल्प॑न्ता म॒ग्नयः॒ पृथ॒क् पृथ॑ ग॒ग्नयः॒ कल्प॑न्ता॒म् कल्प॑न्ता म॒ग्नयः॒ पृथ॑क् । \newline
54. अ॒ग्नयः॒ पृथ॒क् पृथ॑ ग॒ग्नयो॒ ऽग्नयः॒ पृथ॒ङ् मम॒ मम॒ पृथ॑ ग॒ग्नयो॒ ऽग्नयः॒ पृथ॒ङ् मम॑ । \newline
55. पृथ॒ङ् मम॒ मम॒ पृथ॒क् पृथ॒ङ् मम॒ ज्यैष्ठ्या॑य॒ ज्यैष्ठ्या॑य॒ मम॒ पृथ॒क् पृथ॒ङ् मम॒ ज्यैष्ठ्या॑य । \newline
56. मम॒ ज्यैष्ठ्या॑य॒ ज्यैष्ठ्या॑य॒ मम॒ मम॒ ज्यैष्ठ्या॑य॒ सव्र॑ताः॒ सव्र॑ता॒ ज्यैष्ठ्या॑य॒ मम॒ मम॒ ज्यैष्ठ्या॑य॒ सव्र॑ताः । \newline
57. ज्यैष्ठ्या॑य॒ सव्र॑ताः॒ सव्र॑ता॒ ज्यैष्ठ्या॑य॒ ज्यैष्ठ्या॑य॒ सव्र॑ता॒ ये ये सव्र॑ता॒ ज्यैष्ठ्या॑य॒ ज्यैष्ठ्या॑य॒ सव्र॑ता॒ ये । \newline
58. सव्र॑ता॒ ये ये सव्र॑ताः॒ सव्र॑ता॒ ये᳚ ऽग्नयो॒ ऽग्नयो॒ ये सव्र॑ताः॒ सव्र॑ता॒ ये᳚ ऽग्नयः॑ । \newline
59. सव्र॑ता॒ इति॒ स - व्र॒ताः॒ । \newline
\pagebreak
\markright{ TS 4.4.11.2  \hfill https://www.vedavms.in \hfill}

\section{ TS 4.4.11.2 }

\textbf{TS 4.4.11.2 } \newline
\textbf{Samhita Paata} \newline

ये᳚ऽग्नयः॒ सम॑नसोऽन्त॒रा द्यावा॑पृथि॒वी शै॑शि॒रावृ॒तू अ॒भि कल्प॑माना॒ इन्द्र॑मिव दे॒वा अ॒भि संॅवि॑शन्तु सं॒ॅयच्च॒ प्रचे॑ताश्चा॒ग्नेः सोम॑स्य॒ सूर्य॑स्यो॒-ग्रा च॑ भी॒मा च॑ पितृ॒णां ॅय॒मस्येन्द्र॑स्य ध्रु॒वा च॑ पृथि॒वी च॑ दे॒वस्य॑ सवि॒तुर्म॒रुतां॒ ॅवरु॑णस्य ध॒र्त्री च॒ धरि॑त्री च मि॒त्रावरु॑णयो र्मि॒त्रस्य॑ धा॒तुः प्राची॑ च प्र॒तीची॑ च॒ वसू॑नाꣳ रु॒द्राणा॑ - [  ] \newline

\textbf{Pada Paata} \newline

ये । अ॒ग्नयः॑ । सम॑नस॒ इति॒ स - म॒न॒सः॒ । अ॒न्त॒रा । द्यावा॑पृथि॒वी इति॒ द्यावा᳚ - पृ॒थि॒वी । शै॒शि॒रौ । ऋ॒तू इति॑ । अ॒भीति॑ । कल्प॑मानाः । इन्द्र᳚म् । इ॒व॒ । दे॒वाः । अ॒भि । समिति॑ । वि॒श॒न्तु॒ । सं॒ॅयदिति॑ सं - यत् । च॒ । प्रचे॑ता॒ इति॒ प्र - चे॒ताः॒ । च॒ । अ॒ग्नेः । सोम॑स्य । सूर्य॑स्य । उ॒ग्रा । च॒ । भी॒मा । च॒ । पि॒तृ॒णाम् । य॒मस्य॑ । इन्द्र॑स्य । ध्रु॒वा । च॒ । पृ॒थि॒वी । च॒ । दे॒वस्य॑ । स॒वि॒तुः । म॒रुता᳚म् । वरु॑णस्य । ध॒र्त्री । च॒ । धरि॑त्री । च॒ । मि॒त्रावरु॑णयो॒रिति॑ मि॒त्रा - वरु॑णयोः । मि॒त्रस्य॑ । धा॒तुः । प्राची᳚ । च॒ । प्र॒तीची᳚ । च॒ । वसू॑नाम् । रु॒द्राणां᳚ ।  \newline


\textbf{Krama Paata} \newline

ये᳚ऽग्नयः॑ । अ॒ग्नयः॒ सम॑नसः । सम॑नसोऽन्त॒रा । सम॑नस॒ इति॒ स - म॒न॒सः॒ । अ॒न्त॒रा द्यावा॑पृथि॒वी । द्यावा॑पृथि॒वी शै॑शि॒रौ । द्यावा॑पृथि॒वी इति॒ द्यावा᳚ - पृ॒थि॒वी । शै॒शि॒रावृ॒तू । ऋ॒तू अ॒भि । ऋ॒तू इत्यृ॒तू । अ॒भि कल्प॑मानाः । कल्प॑माना॒ इन्द्र᳚म् । इन्द्र॑मिव । इ॒व॒ दे॒वाः । दे॒वा अ॒भि । अ॒भि सम् । सं ॅवि॑शन्तु । वि॒श॒न्तु॒ स॒म्ॅयत् । स॒म्ॅयच् च॑ । स॒म्ॅयदिति॑ सम् - यत् । च॒ प्रचे॑ताः । प्रचे॑ताश्च । प्रचे॑ता॒ इति॒ प्र - चे॒ताः॒ । चा॒ग्नेः । अ॒ग्नेः सोम॑स्य । सोम॑स्य॒ सूर्य॑स्य । सूर्य॑स्यो॒ग्रा । उ॒ग्रा च॑ । च॒ भी॒मा । भी॒मा च॑ । च॒ पि॒तृ॒णाम् । पि॒तृ॒णां ॅय॒मस्य॑ । य॒मस्येन्द्र॑स्य । इन्द्र॑स्य ध्रु॒वा । ध्रु॒वा च॑ । च॒ पृ॒थि॒वी । पृ॒थि॒वी च॑ । च॒ दे॒वस्य॑ । दे॒वस्य॑ सवि॒तुः । स॒वि॒तुर् म॒रुता᳚म् । म॒रुता॒म् ॅवरु॑णस्य । वरु॑णस्य ध॒र्त्री । ध॒र्त्री च॑ । च॒ धरि॑त्री । धरि॑त्री च । च॒ मि॒त्रावरु॑णयोः । मि॒त्रावरु॑णयोर् मि॒त्रस्य॑ । मि॒त्रावरु॑णयो॒रिति॑ मि॒त्रा - वरु॑णयोः । मि॒त्रस्य॑ धा॒तुः । धा॒तुः प्राची᳚ । प्राची॑ च । च॒ प्र॒तीची᳚ । प्र॒तीची॑ च । च॒ वसू॑नाम् । वसू॑नाꣳ रु॒द्राणा᳚म् । रु॒द्राणा॑मादि॒त्याना᳚म् \newline

\textbf{Jatai Paata} \newline

1. ये᳚ ऽग्नयो॒ ऽग्नयो॒ ये ये᳚ ऽग्नयः॑ । \newline
2. अ॒ग्नयः॒ सम॑नसः॒ सम॑नसो॒ ऽग्नयो॒ ऽग्नयः॒ सम॑नसः । \newline
3. सम॑नसो ऽन्त॒रा ऽन्त॒रा सम॑नसः॒ सम॑नसो ऽन्त॒रा । \newline
4. सम॑नस॒ इति॒ स - म॒न॒सः॒ । \newline
5. अ॒न्त॒रा द्यावा॑पृथि॒वी द्यावा॑पृथि॒वी अ॑न्त॒रा ऽन्त॒रा द्यावा॑पृथि॒वी । \newline
6. द्यावा॑पृथि॒वी शै॑शि॒रौ शै॑शि॒रौ द्यावा॑पृथि॒वी द्यावा॑पृथि॒वी शै॑शि॒रौ । \newline
7. द्यावा॑पृथि॒वी इति॒ द्यावा᳚ - पृ॒थि॒वी । \newline
8. शै॒शि॒रा वृ॒तू ऋ॒तू शै॑शि॒रौ शै॑शि॒रा वृ॒तू । \newline
9. ऋ॒तू अ॒भ्या᳚(1॒)भ्यृ॑तू ऋ॒तू अ॒भि । \newline
10. ऋ॒तू इत्यृ॒तू । \newline
11. अ॒भि कल्प॑मानाः॒ कल्प॑माना अ॒भ्य॑भि कल्प॑मानाः । \newline
12. कल्प॑माना॒ इन्द्र॒ मिन्द्र॒म् कल्प॑मानाः॒ कल्प॑माना॒ इन्द्र᳚म् । \newline
13. इन्द्र॑ मिवे॒ वेन्द्र॒ मिन्द्र॑ मिव । \newline
14. इ॒व॒ दे॒वा दे॒वा इ॑वेव दे॒वाः । \newline
15. दे॒वा अ॒भ्य॑भि दे॒वा दे॒वा अ॒भि । \newline
16. अ॒भि सꣳ स म॒भ्य॑भि सम् । \newline
17. सं ॅवि॑शन्तु विशन्तु॒ सꣳ सं ॅवि॑शन्तु । \newline
18. वि॒श॒न्तु॒ सं॒ॅयथ् सं॒ॅयद् वि॑शन्तु विशन्तु सं॒ॅयत् । \newline
19. सं॒ॅयच् च॑ च सं॒ॅयथ् सं॒ॅयच् च॑ । \newline
20. सं॒ॅयदिति॑ सं - यत् । \newline
21. च॒ प्रचे॑ताः॒ प्रचे॑ताश्च च॒ प्रचे॑ताः । \newline
22. प्रचे॑ताश्च च॒ प्रचे॑ताः॒ प्रचे॑ताश्च । \newline
23. प्रचे॑ता॒ इति॒ प्र - चे॒ताः॒ । \newline
24. चा॒ग्ने र॒ग्नेश्च॑ चा॒ग्नेः । \newline
25. अ॒ग्नेः सोम॑स्य॒ सोम॑ स्या॒ग्ने र॒ग्नेः सोम॑स्य । \newline
26. सोम॑स्य॒ सूर्य॑स्य॒ सूर्य॑स्य॒ सोम॑स्य॒ सोम॑स्य॒ सूर्य॑स्य । \newline
27. सूर्य॑ स्यो॒ग्रोग्रा सूर्य॑स्य॒ सूर्य॑ स्यो॒ग्रा । \newline
28. उ॒ग्रा च॑ चो॒ ग्रोग्रा च॑ । \newline
29. च॒ भी॒मा भी॒मा च॑ च भी॒मा । \newline
30. भी॒मा च॑ च भी॒मा भी॒मा च॑ । \newline
31. च॒ पि॒तृ॒णाम् पि॑तृ॒णाम् च॑ च पितृ॒णाम् । \newline
32. पि॒तृ॒णां ॅय॒मस्य॑ य॒मस्य॑ पितृ॒णाम् पि॑तृ॒णां ॅय॒मस्य॑ । \newline
33. य॒मस्येन्द्र॒ स्येन्द्र॑स्य य॒मस्य॑ य॒मस्येन्द्र॑स्य । \newline
34. इन्द्र॑स्य ध्रु॒वा ध्रु॒वेन्द्र॒ स्येन्द्र॑स्य ध्रु॒वा । \newline
35. ध्रु॒वा च॑ च ध्रु॒वा ध्रु॒वा च॑ । \newline
36. च॒ पृ॒थि॒वी पृ॑थि॒वी च॑ च पृथि॒वी । \newline
37. पृ॒थि॒वी च॑ च पृथि॒वी पृ॑थि॒वी च॑ । \newline
38. च॒ दे॒वस्य॑ दे॒वस्य॑ च च दे॒वस्य॑ । \newline
39. दे॒वस्य॑ सवि॒तुः स॑वि॒तुर् दे॒वस्य॑ दे॒वस्य॑ सवि॒तुः । \newline
40. स॒वि॒तुर् म॒रुता᳚म् म॒रुताꣳ॑ सवि॒तुः स॑वि॒तुर् म॒रुता᳚म् । \newline
41. म॒रुतां॒ ॅवरु॑णस्य॒ वरु॑णस्य म॒रुता᳚म् म॒रुतां॒ ॅवरु॑णस्य । \newline
42. वरु॑णस्य ध॒र्त्री ध॒र्त्री वरु॑णस्य॒ वरु॑णस्य ध॒र्त्री । \newline
43. ध॒र्त्री च॑ च ध॒र्त्री ध॒र्त्री च॑ । \newline
44. च॒ धरि॑त्री॒ धरि॑त्री च च॒ धरि॑त्री । \newline
45. धरि॑त्री च च॒ धरि॑त्री॒ धरि॑त्री च । \newline
46. च॒ मि॒त्रावरु॑णयोर् मि॒त्रावरु॑णयोश्च च मि॒त्रावरु॑णयोः । \newline
47. मि॒त्रावरु॑णयोर् मि॒त्रस्य॑ मि॒त्रस्य॑ मि॒त्रावरु॑णयोर् मि॒त्रावरु॑णयोर् मि॒त्रस्य॑ । \newline
48. मि॒त्रावरु॑णयो॒रिति॑ मि॒त्रा - वरु॑णयोः । \newline
49. मि॒त्रस्य॑ धा॒तुर् धा॒तुर् मि॒त्रस्य॑ मि॒त्रस्य॑ धा॒तुः । \newline
50. धा॒तुः प्राची॒ प्राची॑ धा॒तुर् धा॒तुः प्राची᳚ । \newline
51. प्राची॑ च च॒ प्राची॒ प्राची॑ च । \newline
52. च॒ प्र॒तीची᳚ प्र॒तीची॑ च च प्र॒तीची᳚ । \newline
53. प्र॒तीची॑ च च प्र॒तीची᳚ प्र॒तीची॑ च । \newline
54. च॒ वसू॑नां॒ ॅवसू॑नाम् च च॒ वसू॑नाम् । \newline
55. वसू॑नाꣳ रु॒द्राणाꣳ॑ रु॒द्राणां॒ वसू॑नां॒ ॅवसू॑नाꣳ रु॒द्राणां᳚ । \newline
56. रु॒द्राणा॑ मादि॒त्याना॑ मादि॒त्यानाꣳ॑ रु॒द्राणाꣳ॑ रु॒द्राणा॑ मादि॒त्याना᳚म् । \newline

\textbf{Ghana Paata } \newline

1. ये᳚ ऽग्नयो॒ ऽग्नयो॒ ये ये᳚ ऽग्नयः॒ सम॑नसः॒ सम॑नसो॒ ऽग्नयो॒ ये ये᳚ ऽग्नयः॒ सम॑नसः । \newline
2. अ॒ग्नयः॒ सम॑नसः॒ सम॑नसो॒ ऽग्नयो॒ ऽग्नयः॒ सम॑नसो ऽन्त॒रा ऽन्त॒रा सम॑नसो॒ ऽग्नयो॒ ऽग्नयः॒ सम॑नसो ऽन्त॒रा । \newline
3. सम॑नसो ऽन्त॒रा ऽन्त॒रा सम॑नसः॒ सम॑नसो ऽन्त॒रा द्यावा॑पृथि॒वी द्यावा॑पृथि॒वी अ॑न्त॒रा सम॑नसः॒ सम॑नसो ऽन्त॒रा द्यावा॑पृथि॒वी । \newline
4. सम॑नस॒ इति॒ स - म॒न॒सः॒ । \newline
5. अ॒न्त॒रा द्यावा॑पृथि॒वी द्यावा॑पृथि॒वी अ॑न्त॒रा ऽन्त॒रा द्यावा॑पृथि॒वी शै॑शि॒रौ शै॑शि॒रौ द्यावा॑पृथि॒वी अ॑न्त॒रा ऽन्त॒रा द्यावा॑पृथि॒वी शै॑शि॒रौ । \newline
6. द्यावा॑पृथि॒वी शै॑शि॒रौ शै॑शि॒रौ द्यावा॑पृथि॒वी द्यावा॑पृथि॒वी शै॑शि॒रा वृ॒तू ऋ॒तू शै॑शि॒रौ द्यावा॑पृथि॒वी द्यावा॑पृथि॒वी शै॑शि॒रा वृ॒तू । \newline
7. द्यावा॑पृथि॒वी इति॒ द्यावा᳚ - पृ॒थि॒वी । \newline
8. शै॒शि॒रा वृ॒तू ऋ॒तू शै॑शि॒रौ शै॑शि॒रा वृ॒तू अ॒भ्या᳚(1॒)भ्यृ॑तू शै॑शि॒रौ शै॑शि॒रा वृ॒तू अ॒भि । \newline
9. ऋ॒तू अ॒भ्या᳚(1॒)भ्यृ॑तू ऋ॒तू अ॒भि कल्प॑मानाः॒ कल्प॑माना अ॒भ्यृ॑तू ऋ॒तू अ॒भि कल्प॑मानाः । \newline
10. ऋ॒तू इत्यृ॒तू । \newline
11. अ॒भि कल्प॑मानाः॒ कल्प॑माना अ॒भ्य॑भि कल्प॑माना॒ इन्द्र॒ मिन्द्र॒म् कल्प॑माना अ॒भ्य॑भि कल्प॑माना॒ इन्द्र᳚म् । \newline
12. कल्प॑माना॒ इन्द्र॒ मिन्द्र॒म् कल्प॑मानाः॒ कल्प॑माना॒ इन्द्र॑ मिवे॒वेन्द्र॒म् कल्प॑मानाः॒ कल्प॑माना॒ इन्द्र॑ मिव । \newline
13. इन्द्र॑ मिवे॒वेन्द्र॒ मिन्द्र॑ मिव दे॒वा दे॒वा इ॒वेन्द्र॒ मिन्द्र॑ मिव दे॒वाः । \newline
14. इ॒व॒ दे॒वा दे॒वा इ॑वेव दे॒वा अ॒भ्य॑भि दे॒वा इ॑वेव दे॒वा अ॒भि । \newline
15. दे॒वा अ॒भ्य॑भि दे॒वा दे॒वा अ॒भि सꣳ स म॒भि दे॒वा दे॒वा अ॒भि सम् । \newline
16. अ॒भि सꣳ स म॒भ्य॑भि सं ॅवि॑शन्तु विशन्तु॒ स म॒भ्य॑भि सं ॅवि॑शन्तु । \newline
17. सं ॅवि॑शन्तु विशन्तु॒ सꣳ सं ॅवि॑शन्तु सं॒ॅयथ् सं॒ॅयद् वि॑शन्तु॒ सꣳ सं ॅवि॑शन्तु सं॒ॅयत् । \newline
18. वि॒श॒न्तु॒ सं॒ॅयथ् सं॒ॅयद् वि॑शन्तु विशन्तु सं॒ॅयच् च॑ च सं॒ॅयद् वि॑शन्तु विशन्तु सं॒ॅयच् च॑ । \newline
19. सं॒ॅयच् च॑ च सं॒ॅयथ् सं॒ॅयच् च॒ प्रचे॑ताः॒ प्रचे॑ताश्च सं॒ॅयथ् सं॒ॅयच् च॒ प्रचे॑ताः । \newline
20. सं॒ॅयदिति॑ सं - यत् । \newline
21. च॒ प्रचे॑ताः॒ प्रचे॑ताश्च च॒ प्रचे॑ताश्च च॒ प्रचे॑ताश्च च॒ प्रचे॑ताश्च । \newline
22. प्रचे॑ता श्च च॒ प्रचे॑ताः॒ प्रचे॑ता श्चा॒ग्ने र॒ग्ने श्च॒ प्रचे॑ताः॒ प्रचे॑ता श्चा॒ग्नेः । \newline
23. प्रचे॑ता॒ इति॒ प्र - चे॒ताः॒ । \newline
24. चा॒ग्ने र॒ग्ने श्च॑ चा॒ग्नेः सोम॑स्य॒ सोम॑स्या॒ ग्नेश्च॑ चा॒ग्नेः सोम॑स्य । \newline
25. अ॒ग्नेः सोम॑स्य॒ सोम॑ स्या॒ग्ने र॒ग्नेः सोम॑स्य॒ सूर्य॑स्य॒ सूर्य॑स्य॒ सोम॑ स्या॒ग्ने र॒ग्नेः सोम॑स्य॒ सूर्य॑स्य । \newline
26. सोम॑स्य॒ सूर्य॑स्य॒ सूर्य॑स्य॒ सोम॑स्य॒ सोम॑स्य॒ सूर्य॑स्यो॒ ग्रोग्रा सूर्य॑स्य॒ सोम॑स्य॒ सोम॑स्य॒ 
सूर्य॑स्यो॒ग्रा । \newline
27. सूर्य॑स्यो॒ ग्रोग्रा सूर्य॑स्य॒ सूर्य॑स्यो॒ग्रा च॑ चो॒ग्रा सूर्य॑स्य॒ सूर्य॑ स्यो॒ग्रा च॑ । \newline
28. उ॒ग्रा च॑ चो॒ग्रोग्रा च॑ भी॒मा भी॒मा चो॒ग्रोग्रा च॑ भी॒मा । \newline
29. च॒ भी॒मा भी॒मा च॑ च भी॒मा च॑ च भी॒मा च॑ च भी॒मा च॑ । \newline
30. भी॒मा च॑ च भी॒मा भी॒मा च॑ पितृ॒णाम् पि॑तृ॒णाम् च॑ भी॒मा भी॒मा च॑ पितृ॒णाम् । \newline
31. च॒ पि॒तृ॒णाम् पि॑तृ॒णाम् च॑ च पितृ॒णां ॅय॒मस्य॑ य॒मस्य॑ पितृ॒णाम् च॑ च पितृ॒णां ॅय॒मस्य॑ । \newline
32. पि॒तृ॒णां ॅय॒मस्य॑ य॒मस्य॑ पितृ॒णाम् पि॑तृ॒णां ॅय॒मस्येन्द्र॒ स्येन्द्र॑स्य य॒मस्य॑ पितृ॒णाम् पि॑तृ॒णां ॅय॒म स्येन्द्र॑स्य । \newline
33. य॒मस्येन्द्र॒ स्येन्द्र॑स्य य॒मस्य॑ य॒म स्येन्द्र॑स्य ध्रु॒वा ध्रु॒वेन्द्र॑स्य य॒मस्य॑ य॒मस्येन्द्र॑स्य ध्रु॒वा । \newline
34. इन्द्र॑स्य ध्रु॒वा ध्रु॒वेन्द्र॒ स्येन्द्र॑स्य ध्रु॒वा च॑ च ध्रु॒वेन्द्र॒ स्येन्द्र॑स्य ध्रु॒वा च॑ । \newline
35. ध्रु॒वा च॑ च ध्रु॒वा ध्रु॒वा च॑ पृथि॒वी पृ॑थि॒वी च॑ ध्रु॒वा ध्रु॒वा च॑ पृथि॒वी । \newline
36. च॒ पृ॒थि॒वी पृ॑थि॒वी च॑ च पृथि॒वी च॑ च पृथि॒वी च॑ च पृथि॒वी च॑ । \newline
37. पृ॒थि॒वी च॑ च पृथि॒वी पृ॑थि॒वी च॑ दे॒वस्य॑ दे॒वस्य॑ च पृथि॒वी पृ॑थि॒वी च॑ दे॒वस्य॑ । \newline
38. च॒ दे॒वस्य॑ दे॒वस्य॑ च च दे॒वस्य॑ सवि॒तुः स॑वि॒तुर् दे॒वस्य॑ च च दे॒वस्य॑ सवि॒तुः । \newline
39. दे॒वस्य॑ सवि॒तुः स॑वि॒तुर् दे॒वस्य॑ दे॒वस्य॑ सवि॒तुर् म॒रुता᳚म् म॒रुताꣳ॑ सवि॒तुर् दे॒वस्य॑ दे॒वस्य॑ सवि॒तुर् म॒रुता᳚म् । \newline
40. स॒वि॒तुर् म॒रुता᳚म् म॒रुताꣳ॑ सवि॒तुः स॑वि॒तुर् म॒रुतां॒ ॅवरु॑णस्य॒ वरु॑णस्य म॒रुताꣳ॑ सवि॒तुः 
स॑वि॒तुर् म॒रुतां॒ ॅवरु॑णस्य । \newline
41. म॒रुतां॒ ॅवरु॑णस्य॒ वरु॑णस्य म॒रुता᳚म् म॒रुतां॒ ॅवरु॑णस्य ध॒र्त्री ध॒र्त्री वरु॑णस्य म॒रुता᳚म् म॒रुतां॒ ॅवरु॑णस्य ध॒र्त्री । \newline
42. वरु॑णस्य ध॒र्त्री ध॒र्त्री वरु॑णस्य॒ वरु॑णस्य ध॒र्त्री च॑ च ध॒र्त्री वरु॑णस्य॒ वरु॑णस्य ध॒र्त्री च॑ । \newline
43. ध॒र्त्री च॑ च ध॒र्त्री ध॒र्त्री च॒ धरि॑त्री॒ धरि॑त्री च ध॒र्त्री ध॒र्त्री च॒ धरि॑त्री । \newline
44. च॒ धरि॑त्री॒ धरि॑त्री च च॒ धरि॑त्री च च॒ धरि॑त्री च च॒ धरि॑त्री च । \newline
45. धरि॑त्री च च॒ धरि॑त्री॒ धरि॑त्री च मि॒त्रावरु॑णयोर् मि॒त्रावरु॑णयोश्च॒ धरि॑त्री॒ धरि॑त्री च मि॒त्रावरु॑णयोः । \newline
46. च॒ मि॒त्रावरु॑णयोर् मि॒त्रावरु॑णयो श्च च मि॒त्रावरु॑णयोर् मि॒त्रस्य॑ मि॒त्रस्य॑ मि॒त्रावरु॑णयो श्च च मि॒त्रावरु॑णयोर् मि॒त्रस्य॑ । \newline
47. मि॒त्रावरु॑णयोर् मि॒त्रस्य॑ मि॒त्रस्य॑ मि॒त्रावरु॑णयोर् मि॒त्रावरु॑णयोर् मि॒त्रस्य॑ धा॒तुर् धा॒तुर् मि॒त्रस्य॑ मि॒त्रावरु॑णयोर् मि॒त्रावरु॑णयोर् मि॒त्रस्य॑ धा॒तुः । \newline
48. मि॒त्रावरु॑णयो॒रिति॑ मि॒त्रा - वरु॑णयोः । \newline
49. मि॒त्रस्य॑ धा॒तुर् धा॒तुर् मि॒त्रस्य॑ मि॒त्रस्य॑ धा॒तुः प्राची॒ प्राची॑ धा॒तुर् मि॒त्रस्य॑ मि॒त्रस्य॑ धा॒तुः प्राची᳚ । \newline
50. धा॒तुः प्राची॒ प्राची॑ धा॒तुर् धा॒तुः प्राची॑ च च॒ प्राची॑ धा॒तुर् धा॒तुः प्राची॑ च । \newline
51. प्राची॑ च च॒ प्राची॒ प्राची॑ च प्र॒तीची᳚ प्र॒तीची॑ च॒ प्राची॒ प्राची॑ च प्र॒तीची᳚ । \newline
52. च॒ प्र॒तीची᳚ प्र॒तीची॑ च च प्र॒तीची॑ च च प्र॒तीची॑ च च प्र॒तीची॑ च । \newline
53. प्र॒तीची॑ च च प्र॒तीची᳚ प्र॒तीची॑ च॒ वसू॑नां॒ ॅवसू॑नाम् च प्र॒तीची᳚ प्र॒तीची॑ च॒ वसू॑नाम् । \newline
54. च॒ वसू॑नां॒ ॅवसू॑नाम् च च॒ वसू॑नाꣳ रु॒द्राणाꣳ॑ रु॒द्राणां॒ वसू॑नाम् च च॒ वसू॑नाꣳ रु॒द्राणां᳚ । \newline
55. वसू॑नाꣳ रु॒द्राणाꣳ॑ रु॒द्राणां॒ वसू॑नां॒ ॅवसू॑नाꣳ रु॒द्राणा॑ मादि॒त्याना॑ मादि॒त्यानाꣳ॑ रु॒द्राणां॒ वसू॑नां॒ ॅवसू॑नाꣳ रु॒द्राणा॑ मादि॒त्याना᳚म् । \newline
56. रु॒द्राणा॑ मादि॒त्याना॑ मादि॒त्यानाꣳ॑ रु॒द्राणाꣳ॑ रु॒द्राणा॑ मादि॒त्याना॒म् ते त आ॑दि॒त्यानाꣳ॑ 
रु॒द्राणाꣳ॑ रु॒द्राणा॑ मादि॒त्याना॒म् ते । \newline
\pagebreak
\markright{ TS 4.4.11.3  \hfill https://www.vedavms.in \hfill}

\section{ TS 4.4.11.3 }

\textbf{TS 4.4.11.3 } \newline
\textbf{Samhita Paata} \newline

-मादि॒त्यानां॒ ते तेऽधि॑पतय॒स्तेभ्यो॒ नम॒स्ते नो॑ मृडयन्तु॒ ते यं द्वि॒ष्मो यश्च॑ नो॒ द्वेष्टि॒ तं ॅवो॒ जंभे॑ दधामि स॒हस्र॑स्य प्र॒मा अ॑सि स॒हस्र॑स्य प्रति॒मा अ॑सि स॒हस्र॑स्य वि॒मा अ॑सि स॒हस्र॑स्यो॒न्मा अ॑सि साह॒स्रो॑ऽसि स॒हस्रा॑य त्वे॒मा मे॑ अग्न॒ इष्ट॑का धे॒नवः॑ स॒न्त्वेका॑ च श॒तं च॑ स॒हस्रं॑ चा॒युतं॑ च - [  ] \newline

\textbf{Pada Paata} \newline

आ॒दि॒त्याना᳚म् । ते । ते॒ । अधि॑पतय॒ इत्यधि॑-प॒त॒यः॒ । तेभ्यः॑ । नमः॑ । ते । नः॒ । मृ॒ड॒य॒न्तु॒ । ते । यम् । द्वि॒ष्मः । यः । च॒ । नः॒ । द्वेष्टि॑ । तम् । वः॒ । जंभे᳚ । द॒धा॒मि॒ । स॒हस्र॑स्य । प्र॒मेति॑ प्र - मा । अ॒सि॒ । स॒हस्र॑स्य । प्र॒ति॒मेति॑ प्रति - मा । अ॒सि॒ । स॒हस्र॑स्य । वि॒मेति॑ वि - मा । अ॒सि॒ । स॒हस्र॑स्य । उ॒न्मेत्यु॑त् - मा । अ॒सि॒ । सा॒ह॒स्रः । अ॒सि॒ । स॒हस्रा॑य । त्वा॒ । इ॒माः । मे॒ । अ॒ग्ने॒ । इष्ट॑काः । धे॒नवः॑ । स॒न्तु॒ । एका᳚ । च॒ । श॒तम् । च॒ । स॒हस्र᳚म् । च॒ । अ॒युत᳚म् । च॒ ।  \newline


\textbf{Krama Paata} \newline

आ॒दि॒त्याना॒म् ते । ते ते᳚ । तेऽधि॑पतयः । अधि॑पतय॒स्तेभ्यः॑ । अधि॑पतय॒ इत्यधि॑ - प॒त॒यः॒ । तेभ्यो॒ नमः॑ । नम॒स्ते । ते नः॑ । नो॒ मृ॒ड॒य॒न्तु॒ । मृ॒ड॒य॒न्तु॒ ते । ते यम् । यम् द्वि॒ष्मः । द्वि॒ष्मो यः । यश्च॑ । च॒ नः॒ । नो॒ द्वेष्टि॑ । द्वेष्टि॒ तम् । तं ॅवः॑ । वो॒ जम्भे᳚ । जम्भे॑ दधामि । द॒धा॒मि॒ स॒हस्र॑स्य । स॒हस्र॑स्य प्र॒मा । प्र॒मा अ॑सि । प्र॒मेति॑ प्र - मा । अ॒सि॒ स॒हस्र॑स्य । स॒हस्र॑स्य प्रति॒मा । प्र॒ति॒मा अ॑सि । प्र॒ति॒मेति॑ प्रति - मा । अ॒सि॒ स॒हस्र॑स्य । स॒हस्र॑स्य वि॒मा । वि॒मा अ॑सि । वि॒मेति॑ वि - मा । अ॒सि॒ स॒हस्र॑स्य । स॒हस्र॑स्यो॒न्मा । उ॒न्मा अ॑सि । उ॒न्मेत्यु॑त् - मा । अ॒सि॒ सा॒ह॒स्रः । सा॒ह॒स्रो॑ऽसि । अ॒सि॒ स॒हस्रा॑य । स॒हस्रा॑य त्वा । त्वे॒माः । इ॒मा मे᳚ । मे॒ अ॒ग्ने॒ । अ॒ग्न॒ इष्ट॑काः । इष्ट॑का धे॒नवः॑ । धे॒नवः॑ सन्तु । स॒न्त्वेका᳚ । एका॑ च । च॒ श॒तम् । श॒तम् च॑ । च॒ स॒हस्र᳚म् । स॒हस्र॑म् च । चा॒युत᳚म् । अ॒युत॑म् च ( ) । च॒ नि॒युत᳚म् \newline

\textbf{Jatai Paata} \newline

1. आ॒दि॒त्याना॒म् ते त आ॑दि॒त्याना॑ मादि॒त्याना॒म् ते । \newline
2. ते ते॑ ते॒ ते ते ते᳚ । \newline
3. ते ऽधि॑पत॒यो ऽधि॑पतय स्ते॒ ते ऽधि॑पतयः । \newline
4. अधि॑पतय॒ स्तेभ्य॒ स्तेभ्यो ऽधि॑पत॒यो ऽधि॑पतय॒ स्तेभ्यः॑ । \newline
5. अधि॑पतय॒ इत्यधि॑ - प॒त॒यः॒ । \newline
6. तेभ्यो॒ नमो॒ नम॒ स्तेभ्य॒ स्तेभ्यो॒ नमः॑ । \newline
7. नम॒ स्ते ते नमो॒ नम॒ स्ते । \newline
8. ते नो॑ न॒ स्ते ते नः॑ । \newline
9. नो॒ मृ॒ड॒य॒न्तु॒ मृ॒ड॒य॒न्तु॒ नो॒ नो॒ मृ॒ड॒य॒न्तु॒ । \newline
10. मृ॒ड॒य॒न्तु॒ ते ते मृ॑डयन्तु मृडयन्तु॒ ते । \newline
11. ते यं ॅयम् ते ते यम् । \newline
12. यम् द्वि॒ष्मो द्वि॒ष्मो यं ॅयम् द्वि॒ष्मः । \newline
13. द्वि॒ष्मो यो यो द्वि॒ष्मो द्वि॒ष्मो यः । \newline
14. यश्च॑ च॒ यो यश्च॑ । \newline
15. च॒ नो॒ न॒श्च॒ च॒ नः॒ । \newline
16. नो॒ द्वेष्टि॒ द्वेष्टि॑ नो नो॒ द्वेष्टि॑ । \newline
17. द्वेष्टि॒ तम् तम् द्वेष्टि॒ द्वेष्टि॒ तम् । \newline
18. तं ॅवो॑ व॒ स्तम् तं ॅवः॑ । \newline
19. वो॒ जंभे॒ जंभे॑ वो वो॒ जंभे᳚ । \newline
20. जंभे॑ दधामि दधामि॒ जंभे॒ जंभे॑ दधामि । \newline
21. द॒धा॒मि॒ स॒हस्र॑स्य स॒हस्र॑स्य दधामि दधामि स॒हस्र॑स्य । \newline
22. स॒हस्र॑स्य प्र॒मा प्र॒मा स॒हस्र॑स्य स॒हस्र॑स्य प्र॒मा । \newline
23. प्र॒मा अ॑स्यसि प्र॒मा प्र॒मा अ॑सि । \newline
24. प्र॒मेति॑ प्र - मा । \newline
25. अ॒सि॒ स॒हस्र॑स्य स॒हस्र॑ स्यास्यसि स॒हस्र॑स्य । \newline
26. स॒हस्र॑स्य प्रति॒मा प्र॑ति॒मा स॒हस्र॑स्य स॒हस्र॑स्य प्रति॒मा । \newline
27. प्र॒ति॒मा अ॑स्यसि प्रति॒मा प्र॑ति॒मा अ॑सि । \newline
28. प्र॒ति॒मेति॑ प्रति - मा । \newline
29. अ॒सि॒ स॒हस्र॑स्य स॒हस्र॑ स्यास्यसि स॒हस्र॑स्य । \newline
30. स॒हस्र॑स्य वि॒मा वि॒मा स॒हस्र॑स्य स॒हस्र॑स्य वि॒मा । \newline
31. वि॒मा अ॑स्यसि वि॒मा वि॒मा अ॑सि । \newline
32. वि॒मेति॑ वि - मा । \newline
33. अ॒सि॒ स॒हस्र॑स्य स॒हस्र॑स्यास्यसि स॒हस्र॑स्य । \newline
34. स॒हस्र॑ स्यो॒न्मोन्मा स॒हस्र॑स्य स॒हस्र॑ स्यो॒न्मा । \newline
35. उ॒न्मा अ॑स्य स्यु॒न्मोन्मा अ॑सि । \newline
36. उ॒न्मेत्यु॑त् - मा । \newline
37. अ॒सि॒ सा॒ह॒स्रः सा॑ह॒स्रो᳚ ऽस्यसि साह॒स्रः । \newline
38. सा॒ह॒स्रो᳚ ऽस्यसि साह॒स्रः सा॑ह॒स्रो॑ ऽसि । \newline
39. अ॒सि॒ स॒हस्रा॑य स॒हस्रा॑यास्यसि स॒हस्रा॑य । \newline
40. स॒हस्रा॑य त्वा त्वा स॒हस्रा॑य स॒हस्रा॑य त्वा । \newline
41. त्वे॒मा इ॒मा स्त्वा᳚ त्वे॒माः । \newline
42. इ॒मा मे॑ म इ॒मा इ॒मा मे᳚ । \newline
43. मे॒ अ॒ग्ने॒ ऽग्ने॒ मे॒ मे॒ अ॒ग्ने॒ । \newline
44. अ॒ग्न॒ इष्ट॑का॒ इष्ट॑का अग्ने ऽग्न॒ इष्ट॑काः । \newline
45. इष्ट॑का धे॒नवो॑ धे॒नव॒ इष्ट॑का॒ इष्ट॑का धे॒नवः॑ । \newline
46. धे॒नवः॑ सन्तु सन्तु धे॒नवो॑ धे॒नवः॑ सन्तु । \newline
47. स॒न्त्वे कैका॑ सन्तु स॒न्त्वेका᳚ । \newline
48. एका॑ च॒ चैकैका॑ च । \newline
49. च॒ श॒तꣳ श॒तम् च॑ च श॒तम् । \newline
50. श॒तम् च॑ च श॒तꣳ श॒तम् च॑ । \newline
51. च॒ स॒हस्रꣳ॑ स॒हस्र॑म् च च स॒हस्र᳚म् । \newline
52. स॒हस्र॑म् च च स॒हस्रꣳ॑ स॒हस्र॑म् च । \newline
53. चा॒युत॑ म॒युत॑म् च चा॒युत᳚म् । \newline
54. अ॒युत॑म् च चा॒युत॑ म॒युत॑म् च । \newline
55. च॒ नि॒युत॑न् नि॒युत॑म् च च नि॒युत᳚म् । \newline

\textbf{Ghana Paata } \newline

1. आ॒दि॒त्याना॒म् ते त आ॑दि॒त्याना॑ मादि॒त्याना॒म् ते ते॑ ते॒ त आ॑दि॒त्याना॑ मादि॒त्याना॒म् ते ते᳚ । \newline
2. ते ते॑ ते॒ ते ते ते ऽधि॑पत॒यो ऽधि॑पतय स्ते॒ ते ते ते ऽधि॑पतयः । \newline
3. ते ऽधि॑पत॒यो ऽधि॑पतय स्ते॒ ते ऽधि॑पतय॒ स्तेभ्य॒ स्तेभ्यो ऽधि॑पतय स्ते॒ ते ऽधि॑पतय॒ स्तेभ्यः॑ । \newline
4. अधि॑पतय॒ स्तेभ्य॒ स्तेभ्यो ऽधि॑पत॒यो ऽधि॑पतय॒ स्तेभ्यो॒ नमो॒ नम॒ स्तेभ्यो ऽधि॑पत॒यो ऽधि॑पतय॒ स्तेभ्यो॒ नमः॑ । \newline
5. अधि॑पतय॒ इत्यधि॑ - प॒त॒यः॒ । \newline
6. तेभ्यो॒ नमो॒ नम॒ स्तेभ्य॒ स्तेभ्यो॒ नम॒ स्ते ते नम॒ स्तेभ्य॒ स्तेभ्यो॒ नम॒ स्ते । \newline
7. नम॒ स्ते ते नमो॒ नम॒ स्ते नो॑ न॒ स्ते नमो॒ नम॒ स्ते नः॑ । \newline
8. ते नो॑ न॒स्ते ते नो॑ मृडयन्तु मृडयन्तु न॒स्ते ते नो॑ मृडयन्तु । \newline
9. नो॒ मृ॒ड॒य॒न्तु॒ मृ॒ड॒य॒न्तु॒ नो॒ नो॒ मृ॒ड॒य॒न्तु॒ ते ते मृ॑डयन्तु नो नो मृडयन्तु॒ ते । \newline
10. मृ॒ड॒य॒न्तु॒ ते ते मृ॑डयन्तु मृडयन्तु॒ ते यं ॅयम् ते मृ॑डयन्तु मृडयन्तु॒ ते यम् । \newline
11. ते यं ॅयम् ते ते यम् द्वि॒ष्मो द्वि॒ष्मो यम् ते ते यम् द्वि॒ष्मः । \newline
12. यम् द्वि॒ष्मो द्वि॒ष्मो यं ॅयम् द्वि॒ष्मो यो यो द्वि॒ष्मो यं ॅयम् द्वि॒ष्मो यः । \newline
13. द्वि॒ष्मो यो यो द्वि॒ष्मो द्वि॒ष्मो यश्च॑ च॒ यो द्वि॒ष्मो द्वि॒ष्मो यश्च॑ । \newline
14. यश्च॑ च॒ यो यश्च॑ नो नश्च॒ यो यश्च॑ नः । \newline
15. च॒ नो॒ न॒श्च॒ च॒ नो॒ द्वेष्टि॒ द्वेष्टि॑ नश्च च नो॒ द्वेष्टि॑ । \newline
16. नो॒ द्वेष्टि॒ द्वेष्टि॑ नो नो॒ द्वेष्टि॒ तम् तम् द्वेष्टि॑ नो नो॒ द्वेष्टि॒ तम् । \newline
17. द्वेष्टि॒ तम् तम् द्वेष्टि॒ द्वेष्टि॒ तं ॅवो॑ व॒ स्तम् द्वेष्टि॒ द्वेष्टि॒ तं ॅवः॑ । \newline
18. तं ॅवो॑ व॒ स्तम् तं ॅवो॒ जंभे॒ जंभे॑ व॒ स्तम् तं ॅवो॒ जंभे᳚ । \newline
19. वो॒ जंभे॒ जंभे॑ वो वो॒ जंभे॑ दधामि दधामि॒ जंभे॑ वो वो॒ जंभे॑ दधामि । \newline
20. जंभे॑ दधामि दधामि॒ जंभे॒ जंभे॑ दधामि स॒हस्र॑स्य स॒हस्र॑स्य दधामि॒ जंभे॒ जंभे॑ दधामि स॒हस्र॑स्य । \newline
21. द॒धा॒मि॒ स॒हस्र॑स्य स॒हस्र॑स्य दधामि दधामि स॒हस्र॑स्य प्र॒मा प्र॒मा स॒हस्र॑स्य दधामि दधामि स॒हस्र॑स्य प्र॒मा । \newline
22. स॒हस्र॑स्य प्र॒मा प्र॒मा स॒हस्र॑स्य स॒हस्र॑स्य प्र॒मा अ॑स्यसि प्र॒मा स॒हस्र॑स्य स॒हस्र॑स्य प्र॒मा अ॑सि । \newline
23. प्र॒मा अ॑स्यसि प्र॒मा प्र॒मा अ॑सि स॒हस्र॑स्य स॒हस्र॑ स्यासि प्र॒मा प्र॒मा अ॑सि स॒हस्र॑स्य । \newline
24. प्र॒मेति॑ प्र - मा । \newline
25. अ॒सि॒ स॒हस्र॑स्य स॒हस्र॑ स्यास्यसि स॒हस्र॑स्य प्रति॒मा प्र॑ति॒मा स॒हस्र॑ स्यास्यसि स॒हस्र॑स्य प्रति॒मा । \newline
26. स॒हस्र॑स्य प्रति॒मा प्र॑ति॒मा स॒हस्र॑स्य स॒हस्र॑स्य प्रति॒मा अ॑स्यसि प्रति॒मा स॒हस्र॑स्य स॒हस्र॑स्य प्रति॒मा अ॑सि । \newline
27. प्र॒ति॒मा अ॑स्यसि प्रति॒मा प्र॑ति॒मा अ॑सि स॒हस्र॑स्य स॒हस्र॑ स्यासि प्रति॒मा प्र॑ति॒मा अ॑सि स॒हस्र॑स्य । \newline
28. प्र॒ति॒मेति॑ प्रति - मा । \newline
29. अ॒सि॒ स॒हस्र॑स्य स॒हस्र॑ स्यास्यसि स॒हस्र॑स्य वि॒मा वि॒मा स॒हस्र॑ स्यास्यसि स॒हस्र॑स्य वि॒मा । \newline
30. स॒हस्र॑स्य वि॒मा वि॒मा स॒हस्र॑स्य स॒हस्र॑स्य वि॒मा अ॑स्यसि वि॒मा स॒हस्र॑स्य स॒हस्र॑स्य वि॒मा अ॑सि । \newline
31. वि॒मा अ॑स्यसि वि॒मा वि॒मा अ॑सि स॒हस्र॑स्य स॒हस्र॑स्यासि वि॒मा वि॒मा अ॑सि स॒हस्र॑स्य । \newline
32. वि॒मेति॑ वि - मा । \newline
33. अ॒सि॒ स॒हस्र॑स्य स॒हस्र॑स्यास्यसि स॒हस्र॑ स्यो॒न्मोन्मा स॒हस्र॑स्यास्यसि स॒हस्र॑ स्यो॒न्मा । \newline
34. स॒हस्र॑ स्यो॒न्मोन्मा स॒हस्र॑स्य स॒हस्र॑ स्यो॒न्मा अ॑स्यस्यु॒न्मा स॒हस्र॑स्य स॒हस्र॑ स्यो॒न्मा अ॑सि । \newline
35. उ॒न्मा अ॑स्यस्यु॒न् मोन्मा अ॑सि साह॒स्रः सा॑ह॒स्रो᳚ ऽस्यु॒न् मोन्मा अ॑सि साह॒स्रः । \newline
36. उ॒न्मेत्यु॑त् - मा । \newline
37. अ॒सि॒ सा॒ह॒स्रः सा॑ह॒स्रो᳚ ऽस्यसि साह॒स्रो᳚ ऽस्यसि साह॒स्रो᳚ ऽस्यसि साह॒स्रो॑ ऽसि । \newline
38. सा॒ह॒स्रो᳚ ऽस्यसि साह॒स्रः सा॑ह॒स्रो॑ ऽसि स॒हस्रा॑य स॒हस्रा॑यासि साह॒स्रः सा॑ह॒स्रो॑ ऽसि स॒हस्रा॑य । \newline
39. अ॒सि॒ स॒हस्रा॑य स॒हस्रा॑ यास्यसि स॒हस्रा॑य त्वा त्वा स॒हस्रा॑ यास्यसि स॒हस्रा॑य त्वा । \newline
40. स॒हस्रा॑य त्वा त्वा स॒हस्रा॑य स॒हस्रा॑य त्वे॒मा इ॒मा स्त्वा॑ स॒हस्रा॑य स॒हस्रा॑य त्वे॒माः । \newline
41. त्वे॒मा इ॒मा स्त्वा᳚ त्वे॒मा मे॑ म इ॒मा स्त्वा᳚ त्वे॒मा मे᳚ । \newline
42. इ॒मा मे॑ म इ॒मा इ॒मा मे॑ अग्ने ऽग्ने म इ॒मा इ॒मा मे॑ अग्ने । \newline
43. मे॒ अ॒ग्ने॒ ऽग्ने॒ मे॒ मे॒ अ॒ग्न॒ इष्ट॑का॒ इष्ट॑का अग्ने मे मे अग्न॒ इष्ट॑काः । \newline
44. अ॒ग्न॒ इष्ट॑का॒ इष्ट॑का अग्ने ऽग्न॒ इष्ट॑का धे॒नवो॑ धे॒नव॒ इष्ट॑का अग्ने ऽग्न॒ इष्ट॑का धे॒नवः॑ । \newline
45. इष्ट॑का धे॒नवो॑ धे॒नव॒ इष्ट॑का॒ इष्ट॑का धे॒नवः॑ सन्तु सन्तु धे॒नव॒ इष्ट॑का॒ इष्ट॑का धे॒नवः॑ सन्तु । \newline
46. धे॒नवः॑ सन्तु सन्तु धे॒नवो॑ धे॒नवः॑ स॒न्त्वेकैका॑ सन्तु धे॒नवो॑ धे॒नवः॑ स॒न्त्वेका᳚ । \newline
47. स॒न्त्वेकैका॑ सन्तु स॒न्त्वेका॑ च॒ चैका॑ सन्तु स॒न्त्वेका॑ च । \newline
48. एका॑ च॒ चैकैका॑ च श॒तꣳ श॒तम् चैकैका॑ च श॒तम् । \newline
49. च॒ श॒तꣳ श॒तम् च॑ च श॒तम् च॑ च श॒तम् च॑ च श॒तम् च॑ । \newline
50. श॒तम् च॑ च श॒तꣳ श॒तम् च॑ स॒हस्रꣳ॑ स॒हस्र॑म् च श॒तꣳ श॒तम् च॑ स॒हस्र᳚म् । \newline
51. च॒ स॒हस्रꣳ॑ स॒हस्र॑म् च च स॒हस्र॑म् च च स॒हस्र॑म् च च स॒हस्र॑म् च । \newline
52. स॒हस्र॑म् च च स॒हस्रꣳ॑ स॒हस्र॑म् चा॒युत॑ म॒युत॑म् च स॒हस्रꣳ॑ स॒हस्र॑म् चा॒युत᳚म् । \newline
53. चा॒युत॑ म॒युत॑म् च चा॒युत॑म् च चा॒युत॑म् च चा॒युत॑म् च । \newline
54. अ॒युत॑म् च चा॒युत॑ म॒युत॑म् च नि॒युत॑म् नि॒युत॑म् चा॒युत॑ म॒युत॑म् च नि॒युत᳚म् । \newline
55. च॒ नि॒युत॑म् नि॒युत॑म् च च नि॒युत॑म् च च नि॒युत॑म् च च नि॒युत॑म् च । \newline
\pagebreak
\markright{ TS 4.4.11.4  \hfill https://www.vedavms.in \hfill}

\section{ TS 4.4.11.4 }

\textbf{TS 4.4.11.4 } \newline
\textbf{Samhita Paata} \newline

नि॒युतं॑ च प्र॒युतं॒ चार्बु॑दं च॒ न्य॑र्बुदं च समु॒द्रश्च॒ मद्ध्यं॒ चान्त॑श्च परा॒र्द्धश्चे॒मा मे॑ अग्न॒ इष्ट॑का धे॒नवः॑ सन्तु ष॒ष्ठिः स॒हस्र॑म॒युत॒-मक्षी॑यमाणा ऋत॒स्था स्थ॑र्ता॒वृधो॑ घृत॒श्चुतो॑ मधु॒श्चुत॒ ऊर्ज॑स्वतीः स्वधा॒विनी॒स्ता मे॑ अग्न॒ इष्ट॑का धे॒नवः॑ सन्तु वि॒राजो॒ नाम॑ काम॒दुघा॑ अ॒मुत्रा॒मुष्मि॑न् ॅलो॒के ॥ \newline

\textbf{Pada Paata} \newline

नि॒युत॒मिति॑ नि - युत᳚म् । च॒ । प्र॒युत॒मिति॑ प्र-युत᳚म् । च॒ । अर्बु॑दम् । च॒ । न्य॑र्बुद॒मिति॒ नि - अ॒र्बु॒द॒म् । च॒ । स॒मु॒द्रः । च॒ । मद्ध्य᳚म् । च॒ । अन्तः॑ । च॒ । प॒रा॒द्‌र्ध इति॑ पर- अ॒द्‌र्धः । च॒ । इ॒माः । मे॒ । अ॒ग्ने॒ । इष्ट॑काः । धे॒नवः॑ । स॒न्तु॒ । ष॒ष्टिः । स॒हस्र᳚म् । अ॒युत᳚म् । अक्षी॑यमाणाः । ऋ॒त॒स्था इत्यृ॑त - स्थाः । स्थ॒ । ऋ॒ता॒वृध॒ इत्यृ॑त - वृधः॑ । घृ॒त॒श्चुत॒ इति॑ घृत - श्चुतः॑ । म॒धु॒श्चुत॒ इति॑ मधु - श्चुतः॑ । ऊर्ज॑स्वतीः । स्व॒धा॒विनी॒रिति॑ स्वधा - विनीः᳚ । ताः । मे॒ । अ॒ग्ने॒ । इष्ट॑काः । धे॒नवः॑ । स॒न्तु॒ । वि॒राज॒ इति॑ वि - राजः॑ । नाम॑ । का॒म॒दुघा॒ इति॑ काम - दुघाः᳚ । अ॒मुत्र॑ । अ॒मुष्मिन्न्॑ । लो॒के ॥  \newline


\textbf{Krama Paata} \newline

नि॒युत॑म् च । नि॒युत॒मिति॑ नि - युत᳚म् । च॒ प्र॒युत᳚म् । प्र॒युत॑म् च । प्र॒युत॒मिति॑ प्र - युत᳚म् । चार्बु॑दम् । अर्बु॑दम् च । च॒ न्य॑र्बुदम् । न्य॑र्बुदम् च । न्य॑र्बुद॒मिति॒ नि - अ॒र्बु॒द॒म् । च॒ स॒मु॒द्रः । स॒मु॒द्रश्च॑ । च॒ मद्ध्य᳚म् । मद्ध्य॑म् च । चान्तः॑ । अन्त॑श्च । च॒ प॒रा॒र्द्धः । प॒रा॒र्द्धश्च॑ । प॒रा॒र्द्ध इति॑ पर - अ॒र्द्धः । चे॒माः । इ॒मा मे᳚ । मे॒ अ॒ग्ने॒ । अ॒ग्न॒ इष्ट॑काः । इष्ट॑का धे॒नवः॑ । धे॒नवः॑ सन्तु । स॒न्तु॒ ष॒ष्टिः । ष॒ष्टिः स॒हस्र᳚म् । स॒हस्र॑म॒युत᳚म् । अ॒युत॒मक्षी॑यमाणाः । अक्षी॑यमाणा ऋत॒स्थाः । ऋ॒त॒स्थाः स्थ॑ । ऋ॒त॒स्था इत्यृ॑त - स्थाः । स्थ॒र्ता॒वृधः॑ । ऋ॒ता॒वृधो॑ घृत॒श्चुतः॑ । ऋ॒ता॒वृध॒ इत्यृ॑त - वृधः॑ । घृ॒त॒श्चुतो॑ मधु॒श्चुतः॑ । घृ॒त॒श्चुत॒ इति॑ घृत - श्चुतः॑ । म॒धु॒श्चुत॒ ऊर्ज॑स्वतीः । म॒धु॒श्चुत॒ इति॑ मधु - श्चुतः॑ । ऊर्ज॑स्वतीः स्वधा॒विनीः᳚ । स्व॒धा॒विनी॒स्ताः । स्व॒धा॒विनी॒रिति॑ स्वधा - विनीः᳚ । ता मे᳚ । मे॒ अ॒ग्ने॒ । अ॒ग्न॒ इष्ट॑काः । इष्ट॑का धे॒नवः॑ । धे॒नवः॑ सन्तु । स॒न्तु॒ वि॒राजः॑ । वि॒राजो॒ नाम॑ । वि॒राज॒ इति॑ वि - राजः॑ । नाम॑ काम॒दुघाः᳚ । का॒म॒दुघा॑ अ॒मुत्र॑ । का॒म॒दुघा॒ इति॑ काम - दुघाः᳚ । अ॒मुत्रा॒मुष्मिन्न्॑ । अ॒मुष्मि॑न् ॅलो॒के । लो॒क इति॑ लो॒के । \newline

\textbf{Jatai Paata} \newline

1. नि॒युत॑म् च च नि॒युत॑न् नि॒युत॑म् च । \newline
2. नि॒युत॒मिति॑ नि - युत᳚म् । \newline
3. च॒ प्र॒युत॑म् प्र॒युत॑म् च च प्र॒युत᳚म् । \newline
4. प्र॒युत॑म् च च प्र॒युत॑म् प्र॒युत॑म् च । \newline
5. प्र॒युत॒मिति॑ प्र - युत᳚म् । \newline
6. चार्बु॑द॒ मर्बु॑दम् च॒ चार्बु॑दम् । \newline
7. अर्बु॑दम् च॒ चार्बु॑द॒ मर्बु॑दम् च । \newline
8. च॒ न्य॑र्बुद॒न् न्य॑र्बुदम् च च॒ न्य॑र्बुदम् । \newline
9. न्य॑र्बुदम् च च॒ न्य॑र्बुद॒न् न्य॑र्बुदम् च । \newline
10. न्य॑र्बुद॒मिति॒ नि - अ॒र्बु॒द॒म् । \newline
11. च॒ स॒मु॒द्रः स॑मु॒द्रश्च॑ च समु॒द्रः । \newline
12. स॒मु॒द्रश्च॑ च समु॒द्रः स॑मु॒द्रश्च॑ । \newline
13. च॒ मद्ध्य॒म् मद्ध्य॑म् च च॒ मद्ध्य᳚म् । \newline
14. मद्ध्य॑म् च च॒ मद्ध्य॒म् मद्ध्य॑म् च । \newline
15. चान्तो ऽन्त॑श्च॒ चान्तः॑ । \newline
16. अन्त॑श्च॒ चान्तो ऽन्त॑श्च । \newline
17. च॒ प॒रा॒र्द्धः प॑रा॒र्द्धश्च॑ च परा॒र्द्धः । \newline
18. प॒रा॒र्द्धश्च॑ च परा॒र्द्धः प॑रा॒र्द्धश्च॑ । \newline
19. प॒रा॒र्द्ध इति॑ पर - अ॒र्द्धः । \newline
20. चे॒मा इ॒माश्च॑ चे॒माः । \newline
21. इ॒मा मे॑ म इ॒मा इ॒मा मे᳚ । \newline
22. मे॒ अ॒ग्ने॒ ऽग्ने॒ मे॒ मे॒ अ॒ग्ने॒ । \newline
23. अ॒ग्न॒ इष्ट॑का॒ इष्ट॑का अग्ने ऽग्न॒ इष्ट॑काः । \newline
24. इष्ट॑का धे॒नवो॑ धे॒नव॒ इष्ट॑का॒ इष्ट॑का धे॒नवः॑ । \newline
25. धे॒नवः॑ सन्तु सन्तु धे॒नवो॑ धे॒नवः॑ सन्तु । \newline
26. स॒न्तु॒ ष॒ष्टि ष्ष॒ष्टिः स॑न्तु सन्तु ष॒ष्टिः । \newline
27. ष॒ष्टिः स॒हस्रꣳ॑ स॒हस्रꣳ॑ ष॒ष्टि ष्ष॒ष्टिः स॒हस्र᳚म् । \newline
28. स॒हस्र॑ म॒युत॑ म॒युतꣳ॑ स॒हस्रꣳ॑ स॒हस्र॑ म॒युत᳚म् । \newline
29. अ॒युत॒ मक्षी॑यमाणा॒ अक्षी॑यमाणा अ॒युत॑ म॒युत॒ मक्षी॑यमाणाः । \newline
30. अक्षी॑यमाणा ऋत॒स्था ऋ॑त॒स्था अक्षी॑यमाणा॒ अक्षी॑यमाणा ऋत॒स्थाः । \newline
31. ऋ॒त॒स्थाः स्थ॑ स्थ र्‌त॒स्था ऋ॑त॒स्थाः स्थ॑ । \newline
32. ऋ॒त॒स्था इत्यृ॑त - स्थाः । \newline
33. स्थ॒ र्‌ता॒वृध॑ ऋता॒वृधः॑ स्थ स्थ र्‌ता॒वृधः॑ । \newline
34. ऋ॒ता॒वृधो॑ घृत॒श्चुतो॑ घृत॒श्चुत॑ ऋता॒वृध॑ ऋता॒वृधो॑ घृत॒श्चुतः॑ । \newline
35. ऋ॒ता॒वृध॒ इत्यृ॑त - वृधः॑ । \newline
36. घृ॒त॒श्चुतो॑ मधु॒श्चुतो॑ मधु॒श्चुतो॑ घृत॒श्चुतो॑ घृत॒श्चुतो॑ मधु॒श्चुतः॑ । \newline
37. घृ॒त॒श्चुत॒ इति॑ घृत - श्चुतः॑ । \newline
38. म॒धु॒श्चुत॒ ऊर्ज॑स्वती॒ रूर्ज॑स्वतीर् मधु॒श्चुतो॑ मधु॒श्चुत॒ ऊर्ज॑स्वतीः । \newline
39. म॒धु॒श्चुत॒ इति॑ मधु - श्चुतः॑ । \newline
40. ऊर्ज॑स्वतीः स्वधा॒विनीः᳚ स्वधा॒विनी॒ रूर्ज॑स्वती॒ रूर्ज॑स्वतीः स्वधा॒विनीः᳚ । \newline
41. स्व॒धा॒विनी॒ स्ता स्ताः स्व॑धा॒विनीः᳚ स्वधा॒विनी॒ स्ताः । \newline
42. स्व॒धा॒विनी॒रिति॑ स्वधा - विनीः᳚ । \newline
43. ता मे॑ मे॒ ता स्ता मे᳚ । \newline
44. मे॒ अ॒ग्ने॒ ऽग्ने॒ मे॒ मे॒ अ॒ग्ने॒ । \newline
45. अ॒ग्न॒ इष्ट॑का॒ इष्ट॑का अग्ने ऽग्न॒ इष्ट॑काः । \newline
46. इष्ट॑का धे॒नवो॑ धे॒नव॒ इष्ट॑का॒ इष्ट॑का धे॒नवः॑ । \newline
47. धे॒नवः॑ सन्तु सन्तु धे॒नवो॑ धे॒नवः॑ सन्तु । \newline
48. स॒न्तु॒ वि॒राजो॑ वि॒राजः॑ सन्तु सन्तु वि॒राजः॑ । \newline
49. वि॒राजो॒ नाम॒ नाम॑ वि॒राजो॑ वि॒राजो॒ नाम॑ । \newline
50. वि॒राज॒ इति॑ वि - राजः॑ । \newline
51. नाम॑ काम॒दुघाः᳚ काम॒दुघा॒ नाम॒ नाम॑ काम॒दुघाः᳚ । \newline
52. का॒म॒दुघा॑ अ॒मुत्रा॒ मुत्र॑ काम॒दुघाः᳚ काम॒दुघा॑ अ॒मुत्र॑ । \newline
53. का॒म॒दुघा॒ इति॑ काम - दुघाः᳚ । \newline
54. अ॒मुत्रा॒ मुष्मि॑न् न॒मुष्मि॑न् न॒मुत्रा॒ मुत्रा॒ मुष्मिन्न्॑ । \newline
55. अ॒मुष्मि॑न् ॅलो॒के लो॒के॑ ऽमुष्मि॑न् न॒मुष्मि॑न् ॅलो॒के । \newline
56. लो॒क इति॑ लो॒के । \newline

\textbf{Ghana Paata } \newline

1. नि॒युत॑म् च च नि॒युत॑म् नि॒युत॑म् च प्र॒युत॑म् प्र॒युत॑म् च नि॒युत॑म् नि॒युत॑म् च प्र॒युत᳚म् । \newline
2. नि॒युत॒मिति॑ नि - युत᳚म् । \newline
3. च॒ प्र॒युत॑म् प्र॒युत॑म् च च प्र॒युत॑म् च च प्र॒युत॑म् च च प्र॒युत॑म् च । \newline
4. प्र॒युत॑म् च च प्र॒युत॑म् प्र॒युत॒म् चार्बु॑द॒ मर्बु॑दम् च प्र॒युत॑म् प्र॒युत॒म् चार्बु॑दम् । \newline
5. प्र॒युत॒मिति॑ प्र - युत᳚म् । \newline
6. चार्बु॑द॒ मर्बु॑दम् च॒ चार्बु॑दम् च॒ चार्बु॑दम् च॒ चार्बु॑दम् च । \newline
7. अर्बु॑दम् च॒ चार्बु॑द॒ मर्बु॑दम् च॒ न्य॑र्बुद॒म् न्य॑र्बुद॒म् चार्बु॑द॒ मर्बु॑दम् च॒ न्य॑र्बुदम् । \newline
8. च॒ न्य॑र्बुद॒म् न्य॑र्बुदम् च च॒ न्य॑र्बुदम् च च॒ न्य॑र्बुदम् च च॒ न्य॑र्बुदम् च । \newline
9. न्य॑र्बुदम् च च॒ न्य॑र्बुद॒म् न्य॑र्बुदम् च समु॒द्रः स॑मु॒द्रश्च॒ न्य॑र्बुद॒म् न्य॑र्बुदम् च समु॒द्रः । \newline
10. न्य॑र्बुद॒मिति॒ नि - अ॒र्बु॒द॒म् । \newline
11. च॒ स॒मु॒द्रः स॑मु॒द्रश्च॑ च समु॒द्रश्च॑ च समु॒द्रश्च॑ च समु॒द्रश्च॑ । \newline
12. स॒मु॒द्रश्च॑ च समु॒द्रः स॑मु॒द्रश्च॒ मद्ध्य॒म् मद्ध्य॑म् च समु॒द्रः स॑मु॒द्रश्च॒ मद्ध्य᳚म् । \newline
13. च॒ मद्ध्य॒म् मद्ध्य॑म् च च॒ मद्ध्य॑म् च च॒ मद्ध्य॑म् च च॒ मद्ध्य॑म् च । \newline
14. मद्ध्य॑म् च च॒ मद्ध्य॒म् मद्ध्य॒म् चान्तो ऽन्त॑श्च॒ मद्ध्य॒म् मद्ध्य॒म् चान्तः॑ । \newline
15. चान्तो ऽन्त॑श्च॒ चान्त॑श्च॒ चान्त॑श्च॒ चान्त॑श्च । \newline
16. अन्त॑श्च॒ चान्तो ऽन्त॑श्च परा॒र्द्धः प॑रा॒र्द्ध श्चान्तो ऽन्त॑श्च परा॒र्द्धः । \newline
17. च॒ प॒रा॒र्द्धः प॑रा॒र्द्धश्च॑ च परा॒र्द्धश्च॑ च परा॒र्द्धश्च॑ च परा॒र्द्धश्च॑ । \newline
18. प॒रा॒र्द्धश्च॑ च परा॒र्द्धः प॑रा॒र्द्ध श्चे॒मा इ॒माश्च॑ परा॒र्द्धः प॑रा॒र्द्ध श्चे॒माः । \newline
19. प॒रा॒र्द्ध इति॑ पर - अ॒र्द्धः । \newline
20. चे॒ मा इ॒माश्च॑ चे॒ मा मे॑ म इ॒माश्च॑ चे॒ मा मे᳚ । \newline
21. इ॒मा मे॑ म इ॒मा इ॒मा मे॑ अग्ने ऽग्ने म इ॒मा इ॒मा मे॑ अग्ने । \newline
22. मे॒ अ॒ग्ने॒ ऽग्ने॒ मे॒ मे॒ अ॒ग्न॒ इष्ट॑का॒ इष्ट॑का अग्ने मे मे अग्न॒ इष्ट॑काः । \newline
23. अ॒ग्न॒ इष्ट॑का॒ इष्ट॑का अग्ने ऽग्न॒ इष्ट॑का धे॒नवो॑ धे॒नव॒ इष्ट॑का अग्ने ऽग्न॒ इष्ट॑का धे॒नवः॑ । \newline
24. इष्ट॑का धे॒नवो॑ धे॒नव॒ इष्ट॑का॒ इष्ट॑का धे॒नवः॑ सन्तु सन्तु धे॒नव॒ इष्ट॑का॒ इष्ट॑का धे॒नवः॑ सन्तु । \newline
25. धे॒नवः॑ सन्तु सन्तु धे॒नवो॑ धे॒नवः॑ सन्तु ष॒ष्टि ष्ष॒ष्टिः स॑न्तु धे॒नवो॑ धे॒नवः॑ सन्तु ष॒ष्टिः । \newline
26. स॒न्तु॒ ष॒ष्टि ष्ष॒ष्टिः स॑न्तु सन्तु ष॒ष्टिः स॒हस्रꣳ॑ स॒हस्रꣳ॑ ष॒ष्टिः स॑न्तु सन्तु ष॒ष्टिः स॒हस्र᳚म् । \newline
27. ष॒ष्टिः स॒हस्रꣳ॑ स॒हस्रꣳ॑ ष॒ष्टि ष्ष॒ष्टिः स॒हस्र॑ म॒युत॑ म॒युतꣳ॑ स॒हस्रꣳ॑ ष॒ष्टि ष्ष॒ष्टिः स॒हस्र॑ म॒युत᳚म् । \newline
28. स॒हस्र॑ म॒युत॑ म॒युतꣳ॑ स॒हस्रꣳ॑ स॒हस्र॑ म॒युत॒ मक्षी॑यमाणा॒ अक्षी॑यमाणा अ॒युतꣳ॑ स॒हस्रꣳ॑ स॒हस्र॑ म॒युत॒ मक्षी॑यमाणाः । \newline
29. अ॒युत॒ मक्षी॑यमाणा॒ अक्षी॑यमाणा अ॒युत॑ म॒युत॒ मक्षी॑यमाणा ऋत॒स्था ऋ॑त॒स्था अक्षी॑यमाणा अ॒युत॑ म॒युत॒ मक्षी॑यमाणा ऋत॒स्थाः । \newline
30. अक्षी॑यमाणा ऋत॒स्था ऋ॑त॒स्था अक्षी॑यमाणा॒ अक्षी॑यमाणा ऋत॒स्थाः स्थ॑ स्थ र्‌त॒स्था अक्षी॑यमाणा॒ अक्षी॑यमाणा ऋत॒स्थाः स्थ॑ । \newline
31. ऋ॒त॒स्थाः स्थ॑ स्थ र्‌त॒स्था ऋ॑त॒स्थाः स्थ॑ र्‌ता॒वृध॑ ऋता॒वृधः॑ स्थ र्‌त॒स्था ऋ॑त॒स्थाः स्थ॑ र्‌ता॒वृधः॑ । \newline
32. ऋ॒त॒स्था इत्यृ॑त - स्थाः । \newline
33. स्थ॒ र्‌ता॒वृध॑ ऋता॒वृधः॑ स्थ स्थ र्‌ता॒वृधो॑ घृत॒श्चुतो॑ घृत॒श्चुत॑ ऋता॒वृधः॑ स्थ स्थ र्‌ता॒वृधो॑ घृत॒श्चुतः॑ । \newline
34. ऋ॒ता॒वृधो॑ घृत॒श्चुतो॑ घृत॒श्चुत॑ ऋता॒वृध॑ ऋता॒वृधो॑ घृत॒श्चुतो॑ मधु॒श्चुतो॑ मधु॒श्चुतो॑ घृत॒श्चुत॑ ऋता॒वृध॑ ऋता॒वृधो॑ घृत॒श्चुतो॑ मधु॒श्चुतः॑ । \newline
35. ऋ॒ता॒वृध॒ इत्यृ॑त - वृधः॑ । \newline
36. घृ॒त॒श्चुतो॑ मधु॒श्चुतो॑ मधु॒श्चुतो॑ घृत॒श्चुतो॑ घृत॒श्चुतो॑ मधु॒श्चुत॒ ऊर्ज॑स्वती॒ रूर्ज॑स्वतीर् मधु॒श्चुतो॑ घृत॒श्चुतो॑ घृत॒श्चुतो॑ मधु॒श्चुत॒ ऊर्ज॑स्वतीः । \newline
37. घृ॒त॒श्चुत॒ इति॑ घृत - श्चुतः॑ । \newline
38. म॒धु॒श्चुत॒ ऊर्ज॑स्वती॒ रूर्ज॑स्वतीर् मधु॒श्चुतो॑ मधु॒श्चुत॒ ऊर्ज॑स्वतीः स्वधा॒विनीः᳚ स्वधा॒विनी॒ रूर्ज॑स्वतीर् मधु॒श्चुतो॑ मधु॒श्चुत॒ ऊर्ज॑स्वतीः स्वधा॒विनीः᳚ । \newline
39. म॒धु॒श्चुत॒ इति॑ मधु - श्चुतः॑ । \newline
40. ऊर्ज॑स्वतीः स्वधा॒विनीः᳚ स्वधा॒विनी॒ रूर्ज॑स्वती॒ रूर्ज॑स्वतीः स्वधा॒विनी॒ स्ता स्ताः स्व॑धा॒विनी॒ रूर्ज॑स्वती॒ रूर्ज॑स्वतीः स्वधा॒विनी॒ स्ताः । \newline
41. स्व॒धा॒विनी॒ स्ता स्ताः स्व॑धा॒विनीः᳚ स्वधा॒विनी॒ स्ता मे॑ मे॒ ताः स्व॑धा॒विनीः᳚ स्वधा॒विनी॒ स्ता मे᳚ । \newline
42. स्व॒धा॒विनी॒रिति॑ स्वधा - विनीः᳚ । \newline
43. ता मे॑ मे॒ ता स्ता मे॑ अग्ने ऽग्ने मे॒ ता स्ता मे॑ अग्ने । \newline
44. मे॒ अ॒ग्ने॒ ऽग्ने॒ मे॒ मे॒ अ॒ग्न॒ इष्ट॑का॒ इष्ट॑का अग्ने मे मे अग्न॒ इष्ट॑काः । \newline
45. अ॒ग्न॒ इष्ट॑का॒ इष्ट॑का अग्ने ऽग्न॒ इष्ट॑का धे॒नवो॑ धे॒नव॒ इष्ट॑का अग्ने ऽग्न॒ इष्ट॑का धे॒नवः॑ । \newline
46. इष्ट॑का धे॒नवो॑ धे॒नव॒ इष्ट॑का॒ इष्ट॑का धे॒नवः॑ सन्तु सन्तु धे॒नव॒ इष्ट॑का॒ इष्ट॑का धे॒नवः॑ सन्तु । \newline
47. धे॒नवः॑ सन्तु सन्तु धे॒नवो॑ धे॒नवः॑ सन्तु वि॒राजो॑ वि॒राजः॑ सन्तु धे॒नवो॑ धे॒नवः॑ सन्तु वि॒राजः॑ । \newline
48. स॒न्तु॒ वि॒राजो॑ वि॒राजः॑ सन्तु सन्तु वि॒राजो॒ नाम॒ नाम॑ वि॒राजः॑ सन्तु सन्तु वि॒राजो॒ नाम॑ । \newline
49. वि॒राजो॒ नाम॒ नाम॑ वि॒राजो॑ वि॒राजो॒ नाम॑ काम॒दुघाः᳚ काम॒दुघा॒ नाम॑ वि॒राजो॑ वि॒राजो॒ नाम॑ काम॒दुघाः᳚ । \newline
50. वि॒राज॒ इति॑ वि - राजः॑ । \newline
51. नाम॑ काम॒दुघाः᳚ काम॒दुघा॒ नाम॒ नाम॑ काम॒दुघा॑ अ॒मुत्रा॒ मुत्र॑ काम॒दुघा॒ नाम॒ नाम॑ काम॒दुघा॑ अ॒मुत्र॑ । \newline
52. का॒म॒दुघा॑ अ॒मुत्रा॒ मुत्र॑ काम॒दुघाः᳚ काम॒दुघा॑ अ॒मुत्रा॒ मुष्मि॑न् न॒मुष्मि॑न् न॒मुत्र॑ काम॒दुघाः᳚ 
काम॒दुघा॑ अ॒मुत्रा॒ मुष्मिन्न्॑ । \newline
53. का॒म॒दुघा॒ इति॑ काम - दुघाः᳚ । \newline
54. अ॒मुत्रा॒ मुष्मि॑न् न॒मुष्मि॑न् न॒मुत्रा॒ मुत्रा मुष्मि॑न् ॅलो॒के लो॒के॑ ऽमुष्मि॑न् न॒मुत्रा॒ मुत्रा॒ मुष्मि॑न् ॅलो॒के । \newline
55. अ॒मुष्मि॑न् ॅलो॒के लो॒के॑ ऽमुष्मि॑न् न॒मुष्मि॑न् ॅलो॒के । \newline
56. लो॒क इति॑ लो॒के । \newline
\pagebreak
\markright{ TS 4.4.12.1  \hfill https://www.vedavms.in \hfill}

\section{ TS 4.4.12.1 }

\textbf{TS 4.4.12.1 } \newline
\textbf{Samhita Paata} \newline

स॒मिद्-दि॒शामा॒शया॑ नः सुव॒र्विन्मधो॒रतो॒ माध॑वः पात्व॒स्मान् । अ॒ग्निर्दे॒वो दु॒ष्टरी॑तु॒रदा᳚भ्य इ॒दं क्ष॒त्रꣳ र॑क्षतु॒ पात्व॒स्मान् ॥ र॒थ॒न्त॒रꣳ साम॑भिः पात्व॒स्मान् गा॑य॒त्री छन्द॑सां ॅवि॒श्वरू॑पा । त्रि॒वृन्नो॑ वि॒ष्ठया॒ स्तोमो॒ अह्नाꣳ॑ समु॒द्रो वात॑ इ॒दमोजः॑ पिपर्तु ॥ उ॒ग्रा दि॒शाम॒भि-भू॑तिर्वयो॒धाः शुचिः॑ शु॒क्रे अह॑न्योज॒सीना᳚ ।इन्द्राधि॑पतिः पिपृता॒दतो॑ नो॒ महि॑ - [  ] \newline

\textbf{Pada Paata} \newline

स॒मिदिति॑ सम् - इत् । दि॒शाम् । आ॒शया᳚ । नः॒ । सु॒व॒र्विदिति॑ सुवः - वित् । मधोः᳚ । अतः॑ । माध॑वः । पा॒तु॒ । अ॒स्मान् ॥ अ॒ग्निः । दे॒वः । दु॒ष्टरी॑तुः । अदा᳚भ्यः । इ॒दम् । क्ष॒त्रम् । र॒क्ष॒तु॒ । पातु॑ । अ॒स्मान् ॥ र॒थ॒न्त॒रमिति॑ रथं - त॒रम् । साम॑भि॒रिति॒ साम॑ - भिः॒ । पा॒तु॒ । अ॒स्मान् । गा॒य॒त्री । छन्द॑साम् । वि॒श्वरू॒पेति॑ वि॒श्व - रू॒पा॒ ॥ त्रि॒वृदिति॑ त्रि-वृत् । नः॒ । वि॒ष्ठयेति॑ वि - स्थया᳚ । स्तोमः॑ । अह्ना᳚म् । स॒मु॒द्रः । वातः॑ । इ॒दम् । ओजः॑ । पि॒प॒र्तु॒ ॥ उ॒ग्रा । दि॒शाम् । अ॒भिभू॑ति॒रित्य॒भि - भू॒तिः॒ । व॒यो॒धा इति॑ वयः- धाः । शुचिः॑ । शु॒क्रे । अह॑नि । ओ॒ज॒सीना᳚ ॥ इन्द्र॑ । अधि॑पति॒रित्यधि॑ - प॒तिः॒ । पि॒पृ॒ता॒त् । अतः॑ । नः॒ । महि॑ ।  \newline


\textbf{Krama Paata} \newline

स॒मिद् दि॒शाम् । स॒मिदिति॑ सम् - इत् । दि॒शामा॒शयाः᳚ । आ॒शया॑ नः । नः॒ सु॒व॒र्वित् । सु॒व॒र्विन् मधोः᳚ । सु॒व॒र्विदिति॑ सुवः - वित् । मधो॒रतः॑ । अतो॒ माध॑वः । माध॑वः पातु । पा॒त्व॒स्मान् । अ॒स्मानित्य॒स्मान् ॥ अ॒ग्निर् दे॒वः । दे॒वो दु॒ष्टरी॑तुः । दु॒ष्टरी॑तु॒रदा᳚भ्यः । अदा᳚भ्य इ॒दम् । इ॒दम् क्ष॒त्रम् । क्ष॒त्रꣳ र॑क्षतु । र॒क्ष॒तु॒ पातु॑ । पात्व॒स्मान् । अ॒स्मानित्य॒स्मान् ॥ र॒थ॒न्त॒रꣳ साम॑भिः । र॒थ॒न्त॒रमिति॑ रथम् - त॒रम् । साम॑भिः पातु । साम॑भि॒रिति॒ साम॑ - भिः॒ । पा॒त्व॒स्मान् । अ॒स्मान् गा॑य॒त्री । गा॒य॒त्री छन्द॑साम् । छन्द॑सां ॅवि॒श्वरू॑पा । वि॒श्वरू॒पेति॑ वि॒श्व - रू॒पा॒ ॥ त्रि॒वृन् नः॑ । त्रि॒वृदिति॑ त्रि - वृत् । नो॒ वि॒ष्ठया᳚ । वि॒ष्ठया॒ स्तोमः॑ । वि॒ष्ठयेति॑ वि - स्थया᳚ । स्तोमो॒ अह्ना᳚म् । अह्नाꣳ॑ समु॒द्रः । स॒मु॒द्रो वातः॑ । वात॑ इ॒दम् । इ॒दमोजः॑ । ओजः॑ पिपर्तु । पि॒प॒र्त्विति॑ पिपर्तु ॥ उ॒ग्रा दि॒शाम् । दि॒शाम॒भिभू॑तिः । अ॒भिभू॑तिर् वयो॒धाः । अ॒भिभू॑ति॒रित्य॒भि - भू॒तिः॒ । व॒यो॒धाः शुचिः॑ । व॒यो॒धा इति॑ वयः - धाः । शुचिः॑ शु॒क्रे । शु॒क्रे अह॑नि । अह॑न्योज॒सीना᳚ । ओ॒ज॒सीनेत्यो॑ज॒सीना᳚ ॥ इन्द्राधि॑पतिः । अधि॑पतिः पिपृतात् । अधि॑पति॒रित्यधि॑ - प॒तिः॒ । पि॒पृ॒ता॒दतः॑ । अतो॑ नः । नो॒ महि॑ । महि॑ क्ष॒त्रम् \newline

\textbf{Jatai Paata} \newline

1. स॒मिद् दि॒शाम् दि॒शाꣳ स॒मिथ् स॒मिद् दि॒शाम् । \newline
2. स॒मिदिति॑ सम् - इत् । \newline
3. दि॒शा मा॒शया॒ ऽऽशया॑ दि॒शाम् दि॒शा मा॒शया᳚ । \newline
4. आ॒शया॑ नो न आ॒शया॒ ऽऽशया॑ नः । \newline
5. नः॒ सु॒व॒र्विथ् सु॑व॒र्विन् नो॑ नः सुव॒र्वित् । \newline
6. सु॒व॒र्विन् मधो॒र् मधोः᳚ सुव॒र्विथ् सु॑व॒र्विन् मधोः᳚ । \newline
7. सु॒व॒र्विदिति॑ सुवः - वित् । \newline
8. मधो॒ रतो॒ अतो॒ मधो॒र् मधो॒ रतः॑ । \newline
9. अतो॒ माध॑वो॒ माध॑वो॒ अतो॒ अतो॒ माध॑वः । \newline
10. माध॑वः पातु पातु॒ माध॑वो॒ माध॑वः पातु । \newline
11. पा॒त्व॒स्मा न॒स्मान् पा॑तु पात्व॒स्मान् । \newline
12. अ॒स्मानित्य॒स्मान् । \newline
13. अ॒ग्निर् दे॒वो दे॒वो अ॒ग्नि र॒ग्निर् दे॒वः । \newline
14. दे॒वो दु॒ष्टरी॑तुर् दु॒ष्टरी॑तुर् दे॒वो दे॒वो दु॒ष्टरी॑तुः । \newline
15. दु॒ष्टरी॑तु॒ रदा᳚भ्यो॒ अदा᳚भ्यो दु॒ष्टरी॑तुर् दु॒ष्टरी॑तु॒ रदा᳚भ्यः । \newline
16. अदा᳚भ्य इ॒द मि॒द मदा᳚भ्यो॒ अदा᳚भ्य इ॒दम् । \newline
17. इ॒दम् क्ष॒त्रम् क्ष॒त्र मि॒द मि॒दम् क्ष॒त्रम् । \newline
18. क्ष॒त्रꣳ र॑क्षतु रक्षतु क्ष॒त्रम् क्ष॒त्रꣳ र॑क्षतु । \newline
19. र॒क्ष॒तु॒ पातु॒ पातु॑ रक्षतु रक्षतु॒ पातु॑ । \newline
20. पात्व॒स्मा न॒स्मान् पातु॒ पात्व॒स्मान् । \newline
21. अ॒स्मानित्य॒स्मान् । \newline
22. र॒थ॒न्त॒रꣳ साम॑भिः॒ साम॑भी रथन्त॒रꣳ र॑थन्त॒रꣳ साम॑भिः । \newline
23. र॒थ॒न्त॒रमिति॑ रथं - त॒रम् । \newline
24. साम॑भिः पातु पातु॒ साम॑भिः॒ साम॑भिः पातु । \newline
25. साम॑भि॒रिति॒ साम॑ - भिः॒ । \newline
26. पा॒त्व॒स्मा न॒स्मान् पा॑तु पात्व॒स्मान् । \newline
27. अ॒स्मान् गा॑य॒त्री गा॑य॒त्र्य॑स्मा न॒स्मान् गा॑य॒त्री । \newline
28. गा॒य॒त्री छन्द॑सा॒म् छन्द॑साम् गाय॒त्री गा॑य॒त्री छन्द॑साम् । \newline
29. छन्द॑सां ॅवि॒श्वरू॑पा वि॒श्वरू॑पा॒ छन्द॑सा॒म् छन्द॑सां ॅवि॒श्वरू॑पा । \newline
30. वि॒श्वरू॒पेति॑ वि॒श्व - रू॒पा॒ । \newline
31. त्रि॒वृन् नो॑ न स्त्रि॒वृत् त्रि॒वृन् नः॑ । \newline
32. त्रि॒वृदिति॑ त्रि - वृत् । \newline
33. नो॒ वि॒ष्ठया॑ वि॒ष्ठया॑ नो नो वि॒ष्ठया᳚ । \newline
34. वि॒ष्ठया॒ स्तोमः॒ स्तोमो॑ वि॒ष्ठया॑ वि॒ष्ठया॒ स्तोमः॑ । \newline
35. वि॒ष्ठयेति॑ वि - स्थया᳚ । \newline
36. स्तोमो॒ अह्ना॒ मह्नाꣳ॒॒ स्तोमः॒ स्तोमो॒ अह्ना᳚म् । \newline
37. अह्नाꣳ॑ समु॒द्रः स॑मु॒द्रो अह्ना॒ मह्नाꣳ॑ समु॒द्रः । \newline
38. स॒मु॒द्रो वातो॒ वातः॑ समु॒द्रः स॑मु॒द्रो वातः॑ । \newline
39. वात॑ इ॒द मि॒दं ॅवातो॒ वात॑ इ॒दम् । \newline
40. इ॒द मोज॒ ओज॑ इ॒द मि॒द मोजः॑ । \newline
41. ओजः॑ पिपर्तु पिप॒र्त्वोज॒ ओजः॑ पिपर्तु । \newline
42. पि॒प॒र्त्विति॑ पिपर्तु । \newline
43. उ॒ग्रा दि॒शाम् दि॒शा मु॒ग्रोग्रा दि॒शाम् । \newline
44. दि॒शा म॒भिभू॑ति र॒भिभू॑तिर् दि॒शाम् दि॒शा म॒भिभू॑तिः । \newline
45. अ॒भिभू॑तिर् वयो॒धा व॑यो॒धा अ॒भिभू॑ति र॒भिभू॑तिर् वयो॒धाः । \newline
46. अ॒भिभू॑ति॒रित्य॒भि - भू॒तिः॒ । \newline
47. व॒यो॒धाः शुचिः॒ शुचि॑र् वयो॒धा व॑यो॒धाः शुचिः॑ । \newline
48. व॒यो॒धा इति॑ वयः - धाः । \newline
49. शुचिः॑ शु॒क्रे शु॒क्रे शुचिः॒ शुचिः॑ शु॒क्रे । \newline
50. शु॒क्रे अह॒ न्यह॑नि शु॒क्रे शु॒क्रे अह॑नि । \newline
51. अह॑ न्योज॒सीनौ॑ ज॒सीना ऽह॒ न्यह॑ न्योज॒सीना᳚ । \newline
52. ओ॒ज॒सीनेत्यो॑ज॒सीना᳚ । \newline
53. इन्द्राधि॑पति॒ रधि॑पति॒ रिन्द्रेन्द्रा धि॑पतिः । \newline
54. अधि॑पतिः पिपृतात् पिपृता॒ दधि॑पति॒ रधि॑पतिः पिपृतात् । \newline
55. अधि॑पति॒रित्यधि॑ - प॒तिः॒ । \newline
56. पि॒पृ॒ता॒ दतो॒ अतः॑ पिपृतात् पिपृता॒ दतः॑ । \newline
57. अतो॑ नो नो॒ अतो॒ अतो॑ नः । \newline
58. नो॒ महि॒ महि॑ नो नो॒ महि॑ । \newline
59. महि॑ क्ष॒त्रम् क्ष॒त्रम् महि॒ महि॑ क्ष॒त्रम् । \newline

\textbf{Ghana Paata } \newline

1. स॒मिद् दि॒शाम् दि॒शाꣳ स॒मिथ् स॒मिद् दि॒शा मा॒शया॒ ऽऽशया॑ दि॒शाꣳ स॒मिथ् स॒मिद् दि॒शा मा॒शया᳚ । \newline
2. स॒मिदिति॑ सम् - इत् । \newline
3. दि॒शा मा॒शया॒ ऽऽशया॑ दि॒शाम् दि॒शा मा॒शया॑ नो न आ॒शया॑ दि॒शाम् दि॒शा मा॒शया॑ नः । \newline
4. आ॒शया॑ नो न आ॒शया॒ ऽऽशया॑ नः सुव॒र्विथ् सु॑व॒र्विन् न॑ आ॒शया॒ ऽऽशया॑ नः सुव॒र्वित् । \newline
5. नः॒ सु॒व॒र्विथ् सु॑व॒र्विन् नो॑ नः सुव॒र्विन् मधो॒र् मधोः᳚ सुव॒र्विन् नो॑ नः सुव॒र्विन् मधोः᳚ । \newline
6. सु॒व॒र्विन् मधो॒र् मधोः᳚ सुव॒र्विथ् सु॑व॒र्विन् मधो॒ रतो॒ अतो॒ मधोः᳚ सुव॒र्विथ् सु॑व॒र्विन् मधो॒ रतः॑ । \newline
7. सु॒व॒र्विदिति॑ सुवः - वित् । \newline
8. मधो॒ रतो॒ अतो॒ मधो॒र् मधो॒ रतो॒ माध॑वो॒ माध॑वो॒ अतो॒ मधो॒र् मधो॒ रतो॒ माध॑वः । \newline
9. अतो॒ माध॑वो॒ माध॑वो॒ अतो॒ अतो॒ माध॑वः पातु पातु॒ माध॑वो॒ अतो॒ अतो॒ माध॑वः पातु । \newline
10. माध॑वः पातु पातु॒ माध॑वो॒ माध॑वः पात्व॒स्मा न॒स्मान् पा॑तु॒ माध॑वो॒ माध॑वः पात्व॒स्मान् । \newline
11. पा॒त्व॒स्मा न॒स्मान् पा॑तु पात्व॒स्मान् । \newline
12. अ॒स्मानित्य॒स्मान् । \newline
13. अ॒ग्निर् दे॒वो दे॒वो अ॒ग्नि र॒ग्निर् दे॒वो दु॒ष्टरी॑तुर् दु॒ष्टरी॑तुर् दे॒वो अ॒ग्नि र॒ग्निर् दे॒वो दु॒ष्टरी॑तुः । \newline
14. दे॒वो दु॒ष्टरी॑तुर् दु॒ष्टरी॑तुर् दे॒वो दे॒वो दु॒ष्टरी॑तु॒ रदा᳚भ्यो॒ अदा᳚भ्यो दु॒ष्टरी॑तुर् दे॒वो दे॒वो दु॒ष्टरी॑तु॒ रदा᳚भ्यः । \newline
15. दु॒ष्टरी॑तु॒ रदा᳚भ्यो॒ अदा᳚भ्यो दु॒ष्टरी॑तुर् दु॒ष्टरी॑तु॒ रदा᳚भ्य इ॒द मि॒द मदा᳚भ्यो दु॒ष्टरी॑तुर् दु॒ष्टरी॑तु॒ रदा᳚भ्य इ॒दम् । \newline
16. अदा᳚भ्य इ॒द मि॒द मदा᳚भ्यो॒ अदा᳚भ्य इ॒दम् क्ष॒त्रम् क्ष॒त्र मि॒द मदा᳚भ्यो॒ अदा᳚भ्य इ॒दम् क्ष॒त्रम् । \newline
17. इ॒दम् क्ष॒त्रम् क्ष॒त्र मि॒द मि॒दम् क्ष॒त्रꣳ र॑क्षतु रक्षतु क्ष॒त्र मि॒द मि॒दम् क्ष॒त्रꣳ र॑क्षतु । \newline
18. क्ष॒त्रꣳ र॑क्षतु रक्षतु क्ष॒त्रम् क्ष॒त्रꣳ र॑क्षतु॒ पातु॒ पातु॑ रक्षतु क्ष॒त्रम् क्ष॒त्रꣳ र॑क्षतु॒ पातु॑ । \newline
19. र॒क्ष॒तु॒ पातु॒ पातु॑ रक्षतु रक्षतु॒ पात्व॒स्मा न॒स्मान् पातु॑ रक्षतु रक्षतु॒ पात्व॒स्मान् । \newline
20. पात्व॒स्मा न॒स्मान् पातु॒ पात्व॒स्मान् । \newline
21. अ॒स्मानित्य॒स्मान् । \newline
22. र॒थ॒न्त॒रꣳ साम॑भिः॒ साम॑भी रथन्त॒रꣳ र॑थन्त॒रꣳ साम॑भिः पातु पातु॒ साम॑भी रथन्त॒रꣳ र॑थन्त॒रꣳ साम॑भिः पातु । \newline
23. र॒थ॒न्त॒रमिति॑ रथं - त॒रम् । \newline
24. साम॑भिः पातु पातु॒ साम॑भिः॒ साम॑भिः पात्व॒स्मा न॒स्मान् पा॑तु॒ साम॑भिः॒ साम॑भिः पात्व॒स्मान् । \newline
25. साम॑भि॒रिति॒ साम॑ - भिः॒ । \newline
26. पा॒त्व॒स्मा न॒स्मान् पा॑तु पात्व॒स्मान् गा॑य॒त्री गा॑य॒त्र्य॑स्मान् पा॑तु पात्व॒स्मान् गा॑य॒त्री । \newline
27. अ॒स्मान् गा॑य॒त्री गा॑य॒त्र्य॑स्मा न॒स्मान् गा॑य॒त्री छन्द॑सा॒म् छन्द॑साम् गाय॒त्र्य॑स्मा न॒स्मान् गा॑य॒त्री छन्द॑साम् । \newline
28. गा॒य॒त्री छन्द॑सा॒म् छन्द॑साम् गाय॒त्री गा॑य॒त्री छन्द॑सां ॅवि॒श्वरू॑पा वि॒श्वरू॑पा॒ छन्द॑साम् गाय॒त्री गा॑य॒त्री छन्द॑सां ॅवि॒श्वरू॑पा । \newline
29. छन्द॑सां ॅवि॒श्वरू॑पा वि॒श्वरू॑पा॒ छन्द॑सा॒म् छन्द॑सां ॅवि॒श्वरू॑पा । \newline
30. वि॒श्वरू॒पेति॑ वि॒श्व - रू॒पा॒ । \newline
31. त्रि॒वृन् नो॑ नस्त्रि॒वृत् त्रि॒वृन् नो॑ वि॒ष्ठया॑ वि॒ष्ठया॑ न स्त्रि॒वृत् त्रि॒वृन् नो॑ वि॒ष्ठया᳚ । \newline
32. त्रि॒वृदिति॑ त्रि - वृत् । \newline
33. नो॒ वि॒ष्ठया॑ वि॒ष्ठया॑ नो नो वि॒ष्ठया॒ स्तोमः॒ स्तोमो॑ वि॒ष्ठया॑ नो नो वि॒ष्ठया॒ स्तोमः॑ । \newline
34. वि॒ष्ठया॒ स्तोमः॒ स्तोमो॑ वि॒ष्ठया॑ वि॒ष्ठया॒ स्तोमो॒ अह्ना॒ मह्नाꣳ॒॒ स्तोमो॑ वि॒ष्ठया॑ वि॒ष्ठया॒ स्तोमो॒ अह्ना᳚म् । \newline
35. वि॒ष्ठयेति॑ वि - स्थया᳚ । \newline
36. स्तोमो॒ अह्ना॒ मह्नाꣳ॒॒ स्तोमः॒ स्तोमो॒ अह्नाꣳ॑ समु॒द्रः स॑मु॒द्रो अह्नाꣳ॒॒ स्तोमः॒ स्तोमो॒ अह्नाꣳ॑ समु॒द्रः । \newline
37. अह्नाꣳ॑ समु॒द्रः स॑मु॒द्रो अह्ना॒ मह्नाꣳ॑ समु॒द्रो वातो॒ वातः॑ समु॒द्रो अह्ना॒ मह्नाꣳ॑ समु॒द्रो वातः॑ । \newline
38. स॒मु॒द्रो वातो॒ वातः॑ समु॒द्रः स॑मु॒द्रो वात॑ इ॒द मि॒दं ॅवातः॑ समु॒द्रः स॑मु॒द्रो वात॑ इ॒दम् । \newline
39. वात॑ इ॒द मि॒दं ॅवातो॒ वात॑ इ॒द मोज॒ ओज॑ इ॒दं ॅवातो॒ वात॑ इ॒द मोजः॑ । \newline
40. इ॒द मोज॒ ओज॑ इ॒द मि॒द मोजः॑ पिपर्तु पिप॒र्त्वोज॑ इ॒द मि॒द मोजः॑ पिपर्तु । \newline
41. ओजः॑ पिपर्तु पिप॒र्त्वोज॒ ओजः॑ पिपर्तु । \newline
42. पि॒प॒र्त्विति॑ पिपर्तु । \newline
43. उ॒ग्रा दि॒शाम् दि॒शा मु॒ग्रोग्रा दि॒शा म॒भिभू॑ति र॒भिभू॑तिर् दि॒शा मु॒ग्रोग्रा दि॒शा म॒भिभू॑तिः । \newline
44. दि॒शा म॒भिभू॑ति र॒भिभू॑तिर् दि॒शाम् दि॒शा म॒भिभू॑तिर् वयो॒धा व॑यो॒धा अ॒भिभू॑तिर् दि॒शाम् दि॒शा म॒भिभू॑तिर् वयो॒धाः । \newline
45. अ॒भिभू॑तिर् वयो॒धा व॑यो॒धा अ॒भिभू॑ति र॒भिभू॑तिर् वयो॒धाः शुचिः॒ शुचि॑र् वयो॒धा अ॒भिभू॑ति र॒भिभू॑तिर् वयो॒धाः शुचिः॑ । \newline
46. अ॒भिभू॑ति॒रित्य॒भि - भू॒तिः॒ । \newline
47. व॒यो॒धाः शुचिः॒ शुचि॑र् वयो॒धा व॑यो॒धाः शुचिः॑ शु॒क्रे शु॒क्रे शुचि॑र् वयो॒धा व॑यो॒धाः शुचिः॑ शु॒क्रे । \newline
48. व॒यो॒धा इति॑ वयः - धाः । \newline
49. शुचिः॑ शु॒क्रे शु॒क्रे शुचिः॒ शुचिः॑ शु॒क्रे अह॒ न्यह॑नि शु॒क्रे शुचिः॒ शुचिः॑ शु॒क्रे अह॑नि । \newline
50. शु॒क्रे अह॒ न्यह॑नि शु॒क्रे शु॒क्रे अह॑ न्योज॒सीनौ॑ ज॒सीना ऽह॑नि शु॒क्रे शु॒क्रे अह॑ न्योज॒सीना᳚ । \newline
51. अह॑ न्योज॒सीनौ॑ ज॒सीना ऽह॒ न्यह॑ न्योज॒सीना᳚ । \newline
52. ओ॒ज॒सीनेत्यो॑ज॒सीना᳚ । \newline
53. इन्द्राधि॑पति॒ रधि॑पति॒ रिन्द्रेन्द्राधि॑पतिः पिपृतात् पिपृता॒ दधि॑पति॒ रिन्द्रेन्द्राधि॑पतिः पिपृतात् । \newline
54. अधि॑पतिः पिपृतात् पिपृता॒ दधि॑पति॒ रधि॑पतिः पिपृता॒ दतो॒ अतः॑ पिपृता॒ दधि॑पति॒ रधि॑पतिः पिपृता॒ दतः॑ । \newline
55. अधि॑पति॒रित्यधि॑ - प॒तिः॒ । \newline
56. पि॒पृ॒ता॒ दतो॒ अतः॑ पिपृतात् पिपृता॒ दतो॑ नो नो॒ अतः॑ पिपृतात् पिपृता॒ दतो॑ नः । \newline
57. अतो॑ नो नो॒ अतो॒ अतो॑ नो॒ महि॒ महि॑ नो॒ अतो॒ अतो॑ नो॒ महि॑ । \newline
58. नो॒ महि॒ महि॑ नो नो॒ महि॑ क्ष॒त्रम् क्ष॒त्रम् महि॑ नो नो॒ महि॑ क्ष॒त्रम् । \newline
59. महि॑ क्ष॒त्रम् क्ष॒त्रम् महि॒ महि॑ क्ष॒त्रं ॅवि॒श्वतो॑ वि॒श्वतः॑ क्ष॒त्रम् महि॒ महि॑ क्ष॒त्रं ॅवि॒श्वतः॑ । \newline
\pagebreak
\markright{ TS 4.4.12.2  \hfill https://www.vedavms.in \hfill}

\section{ TS 4.4.12.2 }

\textbf{TS 4.4.12.2 } \newline
\textbf{Samhita Paata} \newline

क्ष॒त्रं ॅवि॒श्वतो॑ धारये॒दं ॥ बृ॒हत् साम॑ क्षत्र॒भृद्-वृ॒द्ध वृ॑ष्णियं त्रि॒ष्टुभौजः॑ शुभि॒त मु॒ग्रवी॑रं । इन्द्र॒ स्तोमे॑न पञ्चद॒शेन॒ मद्ध्य॑मि॒दं ॅवाते॑न॒ सग॑रेण रक्ष ॥ प्राची॑ दि॒शाꣳ स॒हय॑शा॒ यश॑स्वती॒ विश्वे॑ देवाः प्रा॒वृषा ऽह्नाꣳ॒॒ सुव॑र्वती । इ॒दं क्ष॒त्रं दु॒ष्टर॑म॒स्त्वोजो ऽना॑धृष्टꣳ सह॒स्रियꣳ॒॒ सह॑स्वत् ॥ वै॒रू॒पे साम॑न्नि॒ह तच्छ॑केम॒ जग॑त्यैनं ॅवि॒क्ष्वा वे॑शयामः । विश्वे॑ देवाः सप्तद॒शेन॒ - [  ] \newline

\textbf{Pada Paata} \newline

क्ष॒त्रम् । वि॒श्वतः॑ । धा॒र॒य॒ । इ॒दम् ॥ बृ॒हत् । साम॑ । क्ष॒त्र॒भृदिति॑ क्षत्र - भृत् । वृ॒द्धवृ॑ष्णिय॒मिति॑ वृ॒द्ध-वृ॒ष्णि॒य॒म् । त्रि॒ष्टुभा᳚ । ओजः॑ । शु॒भि॒तम् । उ॒ग्रवी॑र॒मित्यु॒ग्र - वी॒र॒म् ॥ इन्द्र॑ । स्तोमे॑न । प॒ञ्च॒द॒शेनेति॑ पञ्च - द॒शेन॑ । मद्ध्य᳚म् । इ॒दम् । वाते॑न । सग॑रेण । र॒क्ष॒ ॥ प्राची᳚ । दि॒शाम् । स॒हय॑शा॒ इति॑ स॒ह - य॒शाः॒ । यश॑स्वती । विश्वे᳚ । दे॒वाः॒ । प्रा॒वृषा᳚ । अह्ना᳚म् । सुव॑र्व॒तीति॒ सुवः॑ - व॒ती॒ ॥ इ॒दम् । क्ष॒त्रम् । दु॒ष्टर᳚म् । अ॒स्तु॒ । ओजः॑ । अना॑धृष्ट॒मित्यना᳚ - धृ॒ष्ट॒म् । स॒ह॒स्रिय᳚म् । सह॑स्वत् ॥ वै॒रू॒पे । सामन्न्॑ । इ॒ह । तत् । श॒के॒म॒ । जग॑त्या । ए॒न॒म् । वि॒क्षु । एति॑ । वे॒श॒या॒मः॒ ॥ विश्वे᳚ । दे॒वाः॒ । स॒प्त॒द॒शेनेति॑ सप्त - द॒शेन॑ ।  \newline


\textbf{Krama Paata} \newline

क्ष॒त्रं ॅवि॒श्वतः॑ । वि॒श्वतो॑ धारय । धा॒र॒ये॒दम् । इ॒दमिती॒दम् ॥ बृ॒हथ् साम॑ । साम॑ क्षत्र॒भृत् । क्ष॒त्र॒भृद् वृ॒द्धवृ॑ष्णियम् । क्ष॒त्र॒भृदिति॑ क्षत्र - भृत् । वृ॒द्धवृ॑ष्णियम् त्रि॒ष्टुभा᳚ । वृ॒द्धवृ॑ष्णिय॒मिति॑ वृ॒द्ध - वृ॒ष्णि॒य॒म् । त्रि॒ष्टुभौजः॑ । ओजः॑ शुभि॒तम् । शु॒भि॒तमु॒ग्रवी॑रम् । उ॒ग्रवी॑र॒मित्यु॒ग्र - वी॒र॒म् ॥ इन्द्र॒ स्तोमे॑न । स्तोमे॑न पञ्चद॒शेन॑ । प॒ञ्च॒द॒शेन॒ मद्ध्य᳚म् । प॒ञ्च॒द॒शेनेति॑ पञ्च - द॒शेन॑ । मद्ध्य॑मि॒दम् । इ॒दं ॅवाते॑न । वाते॑न॒ सग॑रेण । सग॑रेण रक्ष । र॒क्षेति॑ रक्ष ॥ प्राची॑ दि॒शाम् । दि॒शाꣳ स॒हय॑शाः । स॒हय॑शा॒ यश॑स्वती । स॒हय॑शा॒ इति॑ स॒ह - य॒शाः॒ । यश॑स्वती॒ विश्वे᳚ । विश्वे॑ देवाः । दे॒वाः॒ प्रा॒वृषा᳚ । प्रा॒वृषाऽह्ना᳚म् । अह्नाꣳ॒॒ सुव॑र्वती । सुव॑र्व॒तीति॒ सुवः॑ - व॒ती॒ ॥ इ॒दम् क्ष॒त्रम् । क्ष॒त्रम् दु॒ष्टर᳚म् । दु॒ष्टर॑मस्तु । अ॒स्त्वोजः॑ । ओजोऽना॑धृष्टम् । अना॑धृष्टꣳ सह॒स्रिय᳚म् । अना॑धृष्ट॒मित्यना᳚ - धृ॒ष्ट॒म् । स॒ह॒स्रियꣳ॒॒ सह॑स्वत् । सह॑स्व॒दिति॒ सह॑स्वत् ॥ वै॒रू॒पे सामन्न्॑ । साम॑न्नि॒ह । इ॒ह तत् । तच्छ॑केम । श॒के॒म॒ जग॑त्या । जग॑त्यैनम् । ए॒नं॒ ॅवि॒क्षु । वि॒क्ष्वा । आ वे॑शयामः । वे॒श॒या॒म॒ इति॑ वेशयामः ॥ विश्वे॑ देवाः । दे॒वाः॒ स॒प्त॒द॒शेन॑ । स॒प्त॒द॒शेन॒ वर्चः॑ । स॒प्त॒द॒शेनेति॑ सप्त - द॒शेन॑ \newline

\textbf{Jatai Paata} \newline

1. क्ष॒त्रं ॅवि॒श्वतो॑ वि॒श्वतः॑ क्ष॒त्रम् क्ष॒त्रं ॅवि॒श्वतः॑ । \newline
2. वि॒श्वतो॑ धारय धारय वि॒श्वतो॑ वि॒श्वतो॑ धारय । \newline
3. धा॒र॒ये॒ द मि॒दम् धा॑रय धारये॒ दम् । \newline
4. इ॒दमिती॒दम् । \newline
5. बृ॒हथ् साम॒ साम॑ बृ॒हद् बृ॒हथ् साम॑ । \newline
6. साम॑ क्षत्र॒भृत् क्ष॑त्र॒भृथ् साम॒ साम॑ क्षत्र॒भृत् । \newline
7. क्ष॒त्र॒भृद् वृ॒द्धवृ॑ष्णियं ॅवृ॒द्धवृ॑ष्णियम् क्षत्र॒भृत् क्ष॑त्र॒भृद् वृ॒द्धवृ॑ष्णियम् । \newline
8. क्ष॒त्र॒भृदिति॑ क्षत्र - भृत् । \newline
9. वृ॒द्धवृ॑ष्णियम् त्रि॒ष्टुभा᳚ त्रि॒ष्टुभा॑ वृ॒द्धवृ॑ष्णियं ॅवृ॒द्धवृ॑ष्णियम् त्रि॒ष्टुभा᳚ । \newline
10. वृ॒द्धवृ॑ष्णिय॒मिति॑ वृ॒द्ध - वृ॒ष्णि॒य॒म् । \newline
11. त्रि॒ष्टु भौज॒ ओज॑ स्त्रि॒ष्टुभा᳚ त्रि॒ष्टु भौजः॑ । \newline
12. ओजः॑ शुभि॒तꣳ शु॑भि॒त मोज॒ ओजः॑ शुभि॒तम् । \newline
13. शु॒भि॒त मु॒ग्रवी॑र मु॒ग्रवी॑रꣳ शुभि॒तꣳ शु॑भि॒त मु॒ग्रवी॑रम् । \newline
14. उ॒ग्रवी॑र॒मित्यु॒ग्र - वी॒र॒म् । \newline
15. इन्द्र॒ स्तोमे॑न॒ स्तोमे॒ नेन्द्रे न्द्र॒ स्तोमे॑न । \newline
16. स्तोमे॑न पञ्चद॒शेन॑ पञ्चद॒शेन॒ स्तोमे॑न॒ स्तोमे॑न पञ्चद॒शेन॑ । \newline
17. प॒ञ्च॒द॒शेन॒ मद्ध्य॒म् मद्ध्य॑म् पञ्चद॒शेन॑ पञ्चद॒शेन॒ मद्ध्य᳚म् । \newline
18. प॒ञ्च॒द॒शेनेति॑ पञ्च - द॒शेन॑ । \newline
19. मद्ध्य॑ मि॒द मि॒दम् मद्ध्य॒म् मद्ध्य॑ मि॒दम् । \newline
20. इ॒दं ॅवाते॑न॒ वाते॑ने॒ द मि॒दं ॅवाते॑न । \newline
21. वाते॑न॒ सग॑रेण॒ सग॑रेण॒ वाते॑न॒ वाते॑न॒ सग॑रेण । \newline
22. सग॑रेण रक्ष रक्ष॒ सग॑रेण॒ सग॑रेण रक्ष । \newline
23. र॒क्षेति॑ रक्ष । \newline
24. प्राची॑ दि॒शाम् दि॒शाम् प्राची॒ प्राची॑ दि॒शाम् । \newline
25. दि॒शाꣳ स॒हय॑शाः स॒हय॑शा दि॒शाम् दि॒शाꣳ स॒हय॑शाः । \newline
26. स॒हय॑शा॒ यश॑स्वती॒ यश॑स्वती स॒हय॑शाः स॒हय॑शा॒ यश॑स्वती । \newline
27. स॒हय॑शा॒ इति॑ स॒ह - य॒शाः॒ । \newline
28. यश॑स्वती॒ विश्वे॒ विश्वे॒ यश॑स्वती॒ यश॑स्वती॒ विश्वे᳚ । \newline
29. विश्वे॑ देवा देवा॒ विश्वे॒ विश्वे॑ देवाः । \newline
30. दे॒वाः॒ प्रा॒वृषा᳚ प्रा॒वृषा॑ देवा देवाः प्रा॒वृषा᳚ । \newline
31. प्रा॒वृषा ऽह्ना॒ मह्ना᳚म् प्रा॒वृषा᳚ प्रा॒वृषा ऽह्ना᳚म् । \newline
32. अह्नाꣳ॒॒ सुव॑र्वती॒ सुव॑र्व॒ त्यह्ना॒ मह्नाꣳ॒॒ सुव॑र्वती । \newline
33. सुव॑र्व॒तीति॒ सुवः॑ - व॒ती॒ । \newline
34. इ॒दम् क्ष॒त्रम् क्ष॒त्र मि॒द मि॒दम् क्ष॒त्रम् । \newline
35. क्ष॒त्रम् दु॒ष्टर॑म् दु॒ष्टर॑म् क्ष॒त्रम् क्ष॒त्रम् दु॒ष्टर᳚म् । \newline
36. दु॒ष्टर॑ मस्त्वस्तु दु॒ष्टर॑म् दु॒ष्टर॑ मस्तु । \newline
37. अ॒स्त्वोज॒ ओजो॑ अस्त्व॒ स्त्वोजः॑ । \newline
38. ओजो ऽना॑धृष्ट॒ मना॑धृष्ट॒ मोज॒ ओजो ऽना॑धृष्टम् । \newline
39. अना॑धृष्टꣳ सह॒स्रियꣳ॑ सह॒स्रिय॒ मना॑धृष्ट॒ मना॑धृष्टꣳ सह॒स्रिय᳚म् । \newline
40. अना॑धृष्ट॒मित्यना᳚ - धृ॒ष्ट॒म् । \newline
41. स॒ह॒स्रियꣳ॒॒ सह॑स्व॒थ् सह॑स्वथ् सह॒स्रियꣳ॑ सह॒स्रियꣳ॒॒ सह॑स्वत् । \newline
42. सह॑स्व॒दिति॒ सह॑स्वत् । \newline
43. वै॒रू॒पे साम॒न् थ्साम॑न्. वैरू॒पे वै॑रू॒पे सामन्न्॑ । \newline
44. साम॑न् नि॒हेह साम॒न् थ्साम॑न् नि॒ह । \newline
45. इ॒ह तत् तदि॒ हेह तत् । \newline
46. तच्छ॑केम शकेम॒ तत् तच्छ॑केम । \newline
47. श॒के॒म॒ जग॑त्या॒ जग॑त्या शकेम शकेम॒ जग॑त्या । \newline
48. जग॑त्यैन मेन॒म् जग॑त्या॒ जग॑त्यैनम् । \newline
49. ए॒नं॒ ॅवि॒क्षु वि॒क्ष्वे॑न मेनं ॅवि॒क्षु । \newline
50. वि॒क्ष्वा वि॒क्षु वि॒क्ष्वा । \newline
51. आ वे॑शयामो वेशयाम॒ आ वे॑शयामः । \newline
52. वे॒श॒या॒म॒ इति॑ वेशयामः । \newline
53. विश्वे॑ देवा देवा॒ विश्वे॒ विश्वे॑ देवाः । \newline
54. दे॒वाः॒ स॒प्त॒द॒शेन॑ सप्तद॒शेन॑ देवा देवाः सप्तद॒शेन॑ । \newline
55. स॒प्त॒द॒शेन॒ वर्चो॒ वर्चः॑ सप्तद॒शेन॑ सप्तद॒शेन॒ वर्चः॑ । \newline
56. स॒प्त॒द॒शेनेति॑ सप्त - द॒शेन॑ । \newline

\textbf{Ghana Paata } \newline

1. क्ष॒त्रं ॅवि॒श्वतो॑ वि॒श्वतः॑ क्ष॒त्रम् क्ष॒त्रं ॅवि॒श्वतो॑ धारय धारय वि॒श्वतः॑ क्ष॒त्रम् क्ष॒त्रं ॅवि॒श्वतो॑ धारय । \newline
2. वि॒श्वतो॑ धारय धारय वि॒श्वतो॑ वि॒श्वतो॑ धारये॒द मि॒दम् धा॑रय वि॒श्वतो॑ वि॒श्वतो॑ धारये॒दम् । \newline
3. धा॒र॒ये॒द मि॒दम् धा॑रय धारये॒दम् । \newline
4. इ॒दमिती॒दम् । \newline
5. बृ॒हथ् साम॒ साम॑ बृ॒हद् बृ॒हथ् साम॑ क्षत्र॒भृत् क्ष॑त्र॒भृथ् साम॑ बृ॒हद् बृ॒हथ् साम॑ क्षत्र॒भृत् । \newline
6. साम॑ क्षत्र॒भृत् क्ष॑त्र॒भृथ् साम॒ साम॑ क्षत्र॒भृद् वृ॒द्धवृ॑ष्णियं ॅवृ॒द्धवृ॑ष्णियम् क्षत्र॒भृथ् साम॒ साम॑ क्षत्र॒भृद् वृ॒द्धवृ॑ष्णियम् । \newline
7. क्ष॒त्र॒भृद् वृ॒द्धवृ॑ष्णियं ॅवृ॒द्धवृ॑ष्णियम् क्षत्र॒भृत् क्ष॑त्र॒भृद् वृ॒द्धवृ॑ष्णियम् त्रि॒ष्टुभा᳚ त्रि॒ष्टुभा॑ वृ॒द्धवृ॑ष्णियम् क्षत्र॒भृत् क्ष॑त्र॒भृद् वृ॒द्धवृ॑ष्णियम् त्रि॒ष्टुभा᳚ । \newline
8. क्ष॒त्र॒भृदिति॑ क्षत्र - भृत् । \newline
9. वृ॒द्धवृ॑ष्णियम् त्रि॒ष्टुभा᳚ त्रि॒ष्टुभा॑ वृ॒द्धवृ॑ष्णियं ॅवृ॒द्धवृ॑ष्णियम् त्रि॒ष्टुभौज॒ ओज॑ स्त्रि॒ष्टुभा॑ वृ॒द्धवृ॑ष्णियं ॅवृ॒द्धवृ॑ष्णियम् त्रि॒ष्टुभौजः॑ । \newline
10. वृ॒द्धवृ॑ष्णिय॒मिति॑ वृ॒द्ध - वृ॒ष्णि॒य॒म् । \newline
11. त्रि॒ष्टु भौज॒ ओज॑ स्त्रि॒ष्टुभा᳚ त्रि॒ष्टु भौजः॑ शुभि॒तꣳ शु॑भि॒त मोज॑ स्त्रि॒ष्टुभा᳚ त्रि॒ष्टु भौजः॑ शुभि॒तम् । \newline
12. ओजः॑ शुभि॒तꣳ शु॑भि॒त मोज॒ ओजः॑ शुभि॒त मु॒ग्रवी॑र मु॒ग्रवी॑रꣳ॒॒ शुभि॒त मोज॒ ओजः॑ शुभि॒त मु॒ग्रवी॑रम् । \newline
13. शु॒भि॒त मु॒ग्रवी॑र मु॒ग्रवी॑रꣳ शुभि॒तꣳ शु॑भि॒त मु॒ग्रवी॑रम् । \newline
14. उ॒ग्रवी॑र॒मित्यु॒ग्र - वी॒र॒म् । \newline
15. इन्द्र॒ स्तोमे॑न॒ स्तोमे॒ नेन्द्रेन्द्र॒ स्तोमे॑न पञ्चद॒शेन॑ पञ्चद॒शेन॒ स्तोमे॒ नेन्द्रेन्द्र॒ स्तोमे॑न पञ्चद॒शेन॑ । \newline
16. स्तोमे॑न पञ्चद॒शेन॑ पञ्चद॒शेन॒ स्तोमे॑न॒ स्तोमे॑न पञ्चद॒शेन॒ मद्ध्य॒म् मद्ध्य॑म् पञ्चद॒शेन॒ स्तोमे॑न॒ स्तोमे॑न पञ्चद॒शेन॒ मद्ध्य᳚म् । \newline
17. प॒ञ्च॒द॒शेन॒ मद्ध्य॒म् मद्ध्य॑म् पञ्चद॒शेन॑ पञ्चद॒शेन॒ मद्ध्य॑ मि॒द मि॒दम् मद्ध्य॑म् पञ्चद॒शेन॑ पञ्चद॒शेन॒ मद्ध्य॑ मि॒दम् । \newline
18. प॒ञ्च॒द॒शेनेति॑ पञ्च - द॒शेन॑ । \newline
19. मद्ध्य॑ मि॒द मि॒दम् मद्ध्य॒म् मद्ध्य॑ मि॒दं ॅवाते॑न॒ वाते॑ने॒दम् मद्ध्य॒म् मद्ध्य॑ मि॒दं ॅवाते॑न । \newline
20. इ॒दं ॅवाते॑न॒ वाते॑ने॒द मि॒दं ॅवाते॑न॒ सग॑रेण॒ सग॑रेण॒ वाते॑ने॒द मि॒दं ॅवाते॑न॒ सग॑रेण । \newline
21. वाते॑न॒ सग॑रेण॒ सग॑रेण॒ वाते॑न॒ वाते॑न॒ सग॑रेण रक्ष रक्ष॒ सग॑रेण॒ वाते॑न॒ वाते॑न॒ सग॑रेण रक्ष । \newline
22. सग॑रेण रक्ष रक्ष॒ सग॑रेण॒ सग॑रेण रक्ष । \newline
23. र॒क्षेति॑ रक्ष । \newline
24. प्राची॑ दि॒शाम् दि॒शाम् प्राची॒ प्राची॑ दि॒शाꣳ स॒हय॑शाः स॒हय॑शा दि॒शाम् प्राची॒ प्राची॑ दि॒शाꣳ स॒हय॑शाः । \newline
25. दि॒शाꣳ स॒हय॑शाः स॒हय॑शा दि॒शाम् दि॒शाꣳ स॒हय॑शा॒ यश॑स्वती॒ यश॑स्वती स॒हय॑शा दि॒शाम् दि॒शाꣳ स॒हय॑शा॒ यश॑स्वती । \newline
26. स॒हय॑शा॒ यश॑स्वती॒ यश॑स्वती स॒हय॑शाः स॒हय॑शा॒ यश॑स्वती॒ विश्वे॒ विश्वे॒ यश॑स्वती स॒हय॑शाः स॒हय॑शा॒ यश॑स्वती॒ विश्वे᳚ । \newline
27. स॒हय॑शा॒ इति॑ स॒ह - य॒शाः॒ । \newline
28. यश॑स्वती॒ विश्वे॒ विश्वे॒ यश॑स्वती॒ यश॑स्वती॒ विश्वे॑ देवा देवा॒ विश्वे॒ यश॑स्वती॒ यश॑स्वती॒ विश्वे॑ देवाः । \newline
29. विश्वे॑ देवा देवा॒ विश्वे॒ विश्वे॑ देवाः प्रा॒वृषा᳚ प्रा॒वृषा॑ देवा॒ विश्वे॒ विश्वे॑ देवाः प्रा॒वृषा᳚ । \newline
30. दे॒वाः॒ प्रा॒वृषा᳚ प्रा॒वृषा॑ देवा देवाः प्रा॒वृषा ऽह्ना॒ मह्ना᳚म् प्रा॒वृषा॑ देवा देवाः प्रा॒वृषा ऽह्ना᳚म् । \newline
31. प्रा॒वृषा ऽह्ना॒ मह्ना᳚म् प्रा॒वृषा᳚ प्रा॒वृषा ऽह्नाꣳ॒॒ सुव॑र्वती॒ सुव॑र्व॒ त्यह्ना᳚म् प्रा॒वृषा᳚ प्रा॒वृषा ऽह्नाꣳ॒॒ सुव॑र्वती । \newline
32. अह्नाꣳ॒॒ सुव॑र्वती॒ सुव॑र्व॒ त्यह्ना॒ मह्नाꣳ॒॒ सुव॑र्वती । \newline
33. सुव॑र्व॒तीति॒ सुवः॑ - व॒ती॒ । \newline
34. इ॒दम् क्ष॒त्रम् क्ष॒त्र मि॒द मि॒दम् क्ष॒त्रम् दु॒ष्टर॑म् दु॒ष्टर॑म् क्ष॒त्र मि॒द मि॒दम् क्ष॒त्रम् दु॒ष्टर᳚म् । \newline
35. क्ष॒त्रम् दु॒ष्टर॑म् दु॒ष्टर॑म् क्ष॒त्रम् क्ष॒त्रम् दु॒ष्टर॑ मस्त्वस्तु दु॒ष्टर॑म् क्ष॒त्रम् क्ष॒त्रम् दु॒ष्टर॑ मस्तु । \newline
36. दु॒ष्टर॑ मस्त्वस्तु दु॒ष्टर॑म् दु॒ष्टर॑ म॒स्त्वोज॒ ओजो॑ अस्तु दु॒ष्टर॑म् दु॒ष्टर॑ म॒स्त्वोजः॑ । \newline
37. अ॒स्त्वोज॒ ओजो॑ अस्त्व॒ स्त्वोजो ऽना॑धृष्ट॒ मना॑धृष्ट॒ मोजो॑ अस्त्व॒ स्त्वोजो ऽना॑धृष्टम् । \newline
38. ओजो ऽना॑धृष्ट॒ मना॑धृष्ट॒ मोज॒ ओजो ऽना॑धृष्टꣳ सह॒स्रियꣳ॑ सह॒स्रिय॒ मना॑धृष्ट॒ मोज॒ ओजो ऽना॑धृष्टꣳ सह॒स्रिय᳚म् । \newline
39. अना॑धृष्टꣳ सह॒स्रियꣳ॑ सह॒स्रिय॒ मना॑धृष्ट॒ मना॑धृष्टꣳ सह॒स्रियꣳ॒॒ सह॑स्व॒थ् सह॑स्वथ् सह॒स्रिय॒ मना॑धृष्ट॒ मना॑धृष्टꣳ सह॒स्रियꣳ॒॒ सह॑स्वत् । \newline
40. अना॑धृष्ट॒मित्यना᳚ - धृ॒ष्ट॒म् । \newline
41. स॒ह॒स्रियꣳ॒॒ सह॑स्व॒थ् सह॑स्वथ् सह॒स्रियꣳ॑ सह॒स्रियꣳ॒॒ सह॑स्वत् । \newline
42. सह॑स्व॒दिति॒ सह॑स्वत् । \newline
43. वै॒रू॒पे साम॒न् थ्साम॑न्. वैरू॒पे वै॑रू॒पे साम॑न् नि॒हेह साम॑न्. वैरू॒पे वै॑रू॒पे साम॑न् नि॒ह । \newline
44. साम॑न् नि॒हेह साम॒न् थ्साम॑न् नि॒ह तत् तदि॒ह साम॒न् थ्साम॑न् नि॒ह तत् । \newline
45. इ॒ह तत् तदि॒हेह तच्छ॑केम शकेम॒ तदि॒हेह तच्छ॑केम । \newline
46. तच्छ॑केम शकेम॒ तत् तच्छ॑केम॒ जग॑त्या॒ जग॑त्या शकेम॒ तत् तच्छ॑केम॒ जग॑त्या । \newline
47. श॒के॒म॒ जग॑त्या॒ जग॑त्या शकेम शकेम॒ जग॑त्यैन मेन॒म् जग॑त्या शकेम शकेम॒ जग॑त्यैनम् । \newline
48. जग॑त्यैन मेन॒म् जग॑त्या॒ जग॑त्यैनं ॅवि॒क्षु वि॒क्ष्वे॑न॒म् जग॑त्या॒ जग॑त्यैनं ॅवि॒क्षु । \newline
49. ए॒नं॒ ॅवि॒क्षु वि॒क्ष्वे॑न मेनं ॅवि॒क्ष्वा वि॒क्ष्वे॑न मेनं ॅवि॒क्ष्वा । \newline
50. वि॒क्ष्वा वि॒क्षु वि॒क्ष्वा वे॑शयामो वेशयाम॒ आ वि॒क्षु वि॒क्ष्वा वे॑शयामः । \newline
51. आ वे॑शयामो वेशयाम॒ आ वे॑शयामः । \newline
52. वे॒श॒या॒म॒ इति॑ वेशयामः । \newline
53. विश्वे॑ देवा देवा॒ विश्वे॒ विश्वे॑ देवाः सप्तद॒शेन॑ सप्तद॒शेन॑ देवा॒ विश्वे॒ विश्वे॑ देवाः सप्तद॒शेन॑ । \newline
54. दे॒वाः॒ स॒प्त॒द॒शेन॑ सप्तद॒शेन॑ देवा देवाः सप्तद॒शेन॒ वर्चो॒ वर्चः॑ सप्तद॒शेन॑ देवा देवाः सप्तद॒शेन॒ वर्चः॑ । \newline
55. स॒प्त॒द॒शेन॒ वर्चो॒ वर्चः॑ सप्तद॒शेन॑ सप्तद॒शेन॒ वर्च॑ इ॒द मि॒दं ॅवर्चः॑ सप्तद॒शेन॑ सप्तद॒शेन॒ वर्च॑ इ॒दम् । \newline
56. स॒प्त॒द॒शेनेति॑ सप्त - द॒शेन॑ । \newline
\pagebreak
\markright{ TS 4.4.12.3  \hfill https://www.vedavms.in \hfill}

\section{ TS 4.4.12.3 }

\textbf{TS 4.4.12.3 } \newline
\textbf{Samhita Paata} \newline

वर्च॑ इ॒दं क्ष॒त्रꣳ स॑लि॒लवा॑तमु॒ग्रं ॥ ध॒र्त्री दि॒शां क्ष॒त्रमि॒दं दा॑धारोप॒स्थाऽऽशा॑नां मि॒त्रव॑द॒स्त्वोजः॑ । मित्रा॑वरुणा श॒रदाऽह्नां᳚ चिकित्नू अ॒स्मै रा॒ष्ट्राय॒ महि॒ शर्म॑ यच्छतं ॥ वै॒रा॒जे साम॒न्नधि॑ मे मनी॒षाऽनु॒ष्टुभा॒ संभृ॑तं ॅवी॒र्यꣳ॑ सहः॑ । इ॒दं क्ष॒त्रं मि॒त्रव॑दा॒र्द्रदा॑नु॒ मित्रा॑वरुणा॒ रक्ष॑त॒-माधि॑पत्यैः ॥ स॒म्राड् दि॒शाꣳ स॒हसा᳚म्नी॒ सह॑स्वत्यृ॒तुर्.हे॑म॒न्तो वि॒ष्ठया॑ नः पिपर्तु । अ॒व॒स्युवा॑ता - [  ] \newline

\textbf{Pada Paata} \newline

वर्चः॑ । इ॒दम् । क्ष॒त्रम् । स॒लि॒लवा॑त॒मिति॑ सलि॒ल - वा॒त॒म् । उ॒ग्रम् ॥ ध॒र्त्री । दि॒शाम् । क्ष॒त्रम् । इ॒दम् । दा॒धा॒र॒ । उ॒प॒स्थेत्यु॑प - स्था । आशा॑नाम् । मि॒त्रव॒दिति॑ मि॒त्र - व॒त् । अ॒स्तु॒ । ओजः॑ ॥ मित्रा॑वरु॒णेति॒ मित्रा᳚ - व॒रु॒णा॒ । श॒रदा᳚ । अह्ना᳚म् । चि॒कि॒त्नू॒ इति॑ । अ॒स्मै । रा॒ष्ट्राय॑ । महि॑ । शर्म॑ । य॒च्छ॒त॒म् ॥ वै॒रा॒जे । सामन्न्॑ । अधीति॑ । मे॒ । म॒नी॒षा । अ॒नु॒ष्टुभेत्य॑नु - स्तुभा᳚ । संभृ॑त॒मिति॑ सं - भृ॒त॒म् । वी॒र्य᳚म् । सहः॑ ॥ इ॒दम् । क्ष॒त्रम् । मि॒त्रव॒दिति॑ मि॒त्र - व॒त् । आ॒र्द्रदा॒न्वित्या॒र्द्र - दा॒नु॒ । मित्रा॑वरु॒णेति॒ मित्रा᳚ - व॒रु॒णा॒ । रक्ष॑तम् । आधि॑पत्यै॒रित्याधि॑-प॒त्यैः॒ ॥ स॒म्राडिति॑ सम् - राट् । दि॒शाम् । स॒हसा॒म्नीति॑ स॒ह - सा॒म्नी॒ । सह॑स्वती । ऋ॒तुः । हे॒म॒न्तः । वि॒ष्ठयेति॑ वि - स्थया᳚ । नः॒ । पि॒प॒र्तु॒ ॥ अ॒व॒स्युवा॑ता॒ इत्य॑व॒स्यु - वा॒ताः॒ ।  \newline


\textbf{Krama Paata} \newline

वर्च॑ इ॒दम् । इ॒दम् क्ष॒त्रम् । क्ष॒त्रꣳ स॑लि॒लवा॑तम् । स॒लि॒लवा॑तमु॒ग्रम् । स॒लि॒लवा॑त॒मिति॑ सलि॒ल - वा॒त॒म् । उ॒ग्रमित्यु॒ग्रम् ॥ ध॒र्त्री दि॒शाम् । दि॒शाम् क्ष॒त्रम् । क्ष॒त्रमि॒दम् । इ॒दम् दा॑धार । दा॒धा॒रो॒प॒स्था । उ॒प॒स्थाऽऽशा॑नाम् । उ॒प॒स्थेत्यु॑प - स्था । आशा॑नाम् मि॒त्रव॑त् । मि॒त्रव॑दस्तु । मि॒त्रव॒दिति॑ मि॒त्र - व॒त्॒ । अ॒स्त्वोजः॑ । ओज॒ इत्योजः॑ ॥ मित्रा॑वरुणा श॒रदा᳚ । मित्रा॑वरु॒णेति॒ मित्रा᳚ - व॒रु॒णा॒ । श॒रदाऽह्ना᳚म् । अह्ना᳚म् चिकित्नू । चि॒कि॒त्नू॒ अ॒स्मै । चि॒कि॒त्नू॒ इति॑ चिकित्नू । अ॒स्मै रा॒ष्ट्राय॑ । रा॒ष्ट्राय॒ महि॑ । महि॒ शर्म॑ । शर्म॑ यच्छतम् । य॒च्छ॒त॒मिति॑ यच्छतम् ॥ वै॒रा॒जे सामन्न्॑ । साम॒न्नधि॑ । अधि॑ मे । मे॒ म॒नी॒षा । म॒नी॒षाऽनु॒ष्टुभा᳚ । अ॒नु॒ष्टुभा॒ सम्भृ॑तम् । अ॒नु॒ष्टुभेत्य॑नु - स्तुभा᳚ । सम्भृ॑तं ॅवी॒र्य᳚म् । सम्भृ॑त॒मिति॒ सम् - भृ॒त॒म् । वी॒र्यꣳ॑ सहः॑ । सह॒ इति॒ सहः॑ ॥ इ॒दम् क्ष॒त्रम् । क्ष॒त्रम् मि॒त्रव॑त् । मि॒त्रव॑दा॒र्द्रदा॑नु । मि॒त्रव॒दिति॑ मि॒त्र - व॒त्॒ । आ॒र्द्रदा॑नु॒ मित्रा॑वरुणा । आ॒र्द्रदा॒न्वित्या॒र्द्र - दा॒नु॒ । मित्रा॑वरुणा॒ रक्ष॑तम् । मित्रा॑वरु॒णेति॒ मित्रा᳚ - व॒रु॒णा॒ । रक्ष॑त॒माधि॑पत्यैः । आधि॑पत्यै॒रित्याधि॑ - प॒त्यैः॒ ॥ स॒म्राड् दि॒शाम् । स॒म्राडिति॑ सम् - राट् । दि॒शाꣳ स॒हसा᳚म्नी । स॒हसा᳚म्नी॒ सह॑स्वती । स॒हसा॒म्नीति॑ स॒ह - सा॒म्नी॒ । सह॑स्वत्यृ॒तुः । ऋ॒तुर्. हे॑म॒न्तः । हे॒म॒न्तो वि॒ष्ठया᳚ । वि॒ष्ठया॑ नः । वि॒ष्ठयेति॑ वि - स्थया᳚ । नः॒ पि॒प॒र्तु॒ । पि॒प॒र्त्विति॑ पिपर्तु ॥ अ॒व॒स्युवा॑ता बृह॒तीः । अ॒व॒स्युवा॑ता॒ इत्य॑व॒स्यु - वा॒ताः॒ \newline

\textbf{Jatai Paata} \newline

1. वर्च॑ इ॒द मि॒दं ॅवर्चो॒ वर्च॑ इ॒दम् । \newline
2. इ॒दम् क्ष॒त्रम् क्ष॒त्र मि॒द मि॒दम् क्ष॒त्रम् । \newline
3. क्ष॒त्रꣳ स॑लि॒लवा॑तꣳ सलि॒लवा॑तम् क्ष॒त्रम् क्ष॒त्रꣳ स॑लि॒लवा॑तम् । \newline
4. स॒लि॒लवा॑त मु॒ग्र मु॒ग्रꣳ स॑लि॒लवा॑तꣳ सलि॒लवा॑त मु॒ग्रम् । \newline
5. स॒लि॒लवा॑त॒मिति॑ सलि॒ल - वा॒त॒म् । \newline
6. उ॒ग्रमित्यु॒ग्रम् । \newline
7. ध॒र्त्री दि॒शाम् दि॒शाम् ध॒र्त्री ध॒र्त्री दि॒शाम् । \newline
8. दि॒शाम् क्ष॒त्रम् क्ष॒त्रम् दि॒शाम् दि॒शाम् क्ष॒त्रम् । \newline
9. क्ष॒त्र मि॒द मि॒दम् क्ष॒त्रम् क्ष॒त्र मि॒दम् । \newline
10. इ॒दम् दा॑धार दाधारे॒द मि॒दम् दा॑धार । \newline
11. दा॒धा॒ रो॒प॒स्थो प॒स्था दा॑धार दाधा रोप॒स्था । \newline
12. उ॒प॒स्था ऽऽशा॑ना॒ माशा॑ना मुप॒स्थो प॒स्था ऽऽशा॑नाम् । \newline
13. उ॒प॒स्थेत्यु॑प - स्था । \newline
14. आशा॑नाम् मि॒त्रव॑न् मि॒त्रव॒ दाशा॑ना॒ माशा॑नाम् मि॒त्रव॑त् । \newline
15. मि॒त्रव॑ दस्त्वस्तु मि॒त्रव॑न् मि॒त्रव॑ दस्तु । \newline
16. मि॒त्रव॒दिति॑ मि॒त्र - व॒त् । \newline
17. अ॒स्त्वोज॒ ओजो॑ अस्त्व॒ स्त्वोजः॑ । \newline
18. ओज॒ इत्योजः॑ । \newline
19. मित्रा॑वरुणा श॒रदा॑ श॒रदा॒ मित्रा॑वरुणा॒ मित्रा॑वरुणा श॒रदा᳚ । \newline
20. मित्रा॑वरु॒णेति॒ मित्रा᳚ - व॒रु॒णा॒ । \newline
21. श॒रदा ऽह्ना॒ मह्नाꣳ॑ श॒रदा॑ श॒रदा ऽह्ना᳚म् । \newline
22. अह्ना᳚म् चिकित्नू चिकित्नू॒ अह्ना॒ मह्ना᳚म् चिकित्नू । \newline
23. चि॒कि॒त्नू॒ अ॒स्मा अ॒स्मै चि॑कित्नू चिकित्नू अ॒स्मै । \newline
24. चि॒कि॒त्नू॒ इति॑ चिकित्नू । \newline
25. अ॒स्मै रा॒ष्ट्राय॑ रा॒ष्ट्रा या॒स्मा अ॒स्मै रा॒ष्ट्राय॑ । \newline
26. रा॒ष्ट्राय॒ महि॒ महि॑ रा॒ष्ट्राय॑ रा॒ष्ट्राय॒ महि॑ । \newline
27. महि॒ शर्म॒ शर्म॒ महि॒ महि॒ शर्म॑ । \newline
28. शर्म॑ यच्छतं ॅयच्छतꣳ॒॒ शर्म॒ शर्म॑ यच्छतम् । \newline
29. य॒च्छ॒त॒मिति॑ यच्छतम् । \newline
30. वै॒रा॒जे साम॒न् थ्साम॑न्. वैरा॒जे वै॑रा॒जे सामन्न्॑ । \newline
31. साम॒न् नध्यधि॒ साम॒न् थ्साम॒न् नधि॑ । \newline
32. अधि॑ मे मे॒ अध्यधि॑ मे । \newline
33. मे॒ म॒नी॒षा म॑नी॒षा मे॑ मे मनी॒षा । \newline
34. म॒नी॒षा ऽनु॒ष्टुभा॑ ऽनु॒ष्टुभा॑ मनी॒षा म॑नी॒षा ऽनु॒ष्टुभा᳚ । \newline
35. अ॒नु॒ष्टुभा॒ संभृ॑तꣳ॒॒ संभृ॑त मनु॒ष्टुभा॑ ऽनु॒ष्टुभा॒ संभृ॑तम् । \newline
36. अ॒नु॒ष्टुभेत्य॑नु - स्तुभा᳚ । \newline
37. संभृ॑तं ॅवी॒र्यं॑ ॅवी॒र्यꣳ॑ संभृ॑तꣳ॒॒ संभृ॑तं ॅवी॒र्य᳚म् । \newline
38. संभृ॑त॒मिति॒ सं - भृ॒त॒म् । \newline
39. वी॒र्यꣳ॑ सहः॒ सहो॑ वी॒र्यं॑ ॅवी॒र्यꣳ॑ सहः॑ । \newline
40. सह॒ इति॒ सहः॑ । \newline
41. इ॒दम् क्ष॒त्रम् क्ष॒त्र मि॒द मि॒दम् क्ष॒त्रम् । \newline
42. क्ष॒त्रम् मि॒त्रव॑न् मि॒त्रव॑त् क्ष॒त्रम् क्ष॒त्रम् मि॒त्रव॑त् । \newline
43. मि॒त्रव॑ दा॒र्द्रदा᳚न् वा॒र्द्रदा॑नु मि॒त्रव॑न् मि॒त्रव॑ दा॒र्द्रदा॑नु । \newline
44. मि॒त्रव॒दिति॑ मि॒त्र - व॒त् । \newline
45. आ॒र्द्रदा॑नु॒ मित्रा॑वरुणा॒ मित्रा॑वरुणा॒ ऽऽर्द्रदा᳚न् वा॒र्द्रदा॑नु॒ मित्रा॑वरुणा । \newline
46. आ॒र्द्रदा॒न्वित्या॒र्द्र - दा॒नु॒ । \newline
47. मित्रा॑वरुणा॒ रक्ष॑तꣳ॒॒ रक्ष॑त॒म् मित्रा॑वरुणा॒ मित्रा॑वरुणा॒ रक्ष॑तम् । \newline
48. मित्रा॑वरु॒णेति॒ मित्रा᳚ - व॒रु॒णा॒ । \newline
49. रक्ष॑त॒ माधि॑पत्यै॒ राधि॑पत्यै॒ रक्ष॑तꣳ॒॒ रक्ष॑त॒ माधि॑पत्यैः । \newline
50. आधि॑पत्यै॒रित्याधि॑ - प॒त्यैः॒ । \newline
51. स॒म्राड् दि॒शाम् दि॒शाꣳ स॒म्राट् थ्स॒म्राड् दि॒शाम् । \newline
52. स॒म्राडिति॑ सम् - राट् । \newline
53. दि॒शाꣳ स॒हसा᳚म्नी स॒हसा᳚म्नी दि॒शाम् दि॒शाꣳ स॒हसा᳚म्नी । \newline
54. स॒हसा᳚म्नी॒ सह॑स्वती॒ सह॑स्वती स॒हसा᳚म्नी स॒हसा᳚म्नी॒ सह॑स्वती । \newline
55. स॒हसा॒म्नीति॑ स॒ह - सा॒म्नी॒ । \newline
56. सह॑स्व त्यृ॒तुर्. ऋ॒तुः सह॑स्वती॒ सह॑स्व त्यृ॒तुः । \newline
57. ऋ॒तुर्. हे॑म॒न्तो हे॑म॒न्त ऋ॒तुर्. ऋ॒तुर्. हे॑म॒न्तः । \newline
58. हे॒म॒न्तो वि॒ष्ठया॑ वि॒ष्ठया॑ हेम॒न्तो हे॑म॒न्तो वि॒ष्ठया᳚ । \newline
59. वि॒ष्ठया॑ नो नो वि॒ष्ठया॑ वि॒ष्ठया॑ नः । \newline
60. वि॒ष्ठयेति॑ वि - स्थया᳚ । \newline
61. नः॒ पि॒प॒र्तु॒ पि॒प॒र्तु॒ नो॒ नः॒ पि॒प॒र्तु॒ । \newline
62. पि॒प॒र्त्विति॑ पिपर्तु । \newline
63. अ॒व॒स्युवा॑ता बृह॒तीर् बृ॑ह॒ती र॑व॒स्युवा॑ता अव॒स्युवा॑ता बृह॒तीः । \newline
64. अ॒व॒स्युवा॑ता॒ इत्य॑व॒स्यु - वा॒ताः॒ । \newline

\textbf{Ghana Paata } \newline

1. वर्च॑ इ॒द मि॒दं ॅवर्चो॒ वर्च॑ इ॒दम् क्ष॒त्रम् क्ष॒त्र मि॒दं ॅवर्चो॒ वर्च॑ इ॒दम् क्ष॒त्रम् । \newline
2. इ॒दम् क्ष॒त्रम् क्ष॒त्र मि॒द मि॒दम् क्ष॒त्रꣳ स॑लि॒लवा॑तꣳ सलि॒लवा॑तम् क्ष॒त्र मि॒द मि॒दम् क्ष॒त्रꣳ स॑लि॒लवा॑तम् । \newline
3. क्ष॒त्रꣳ स॑लि॒लवा॑तꣳ सलि॒लवा॑तम् क्ष॒त्रम् क्ष॒त्रꣳ स॑लि॒लवा॑त मु॒ग्र मु॒ग्रꣳ स॑लि॒लवा॑तम् 
क्ष॒त्रम् क्ष॒त्रꣳ स॑लि॒लवा॑त मु॒ग्रम् । \newline
4. स॒लि॒लवा॑त मु॒ग्र मु॒ग्रꣳ स॑लि॒लवा॑तꣳ सलि॒लवा॑त मु॒ग्रम् । \newline
5. स॒लि॒लवा॑त॒मिति॑ सलि॒ल - वा॒त॒म् । \newline
6. उ॒ग्रमित्यु॒ग्रम् । \newline
7. ध॒र्त्री दि॒शाम् दि॒शाम् ध॒र्त्री ध॒र्त्री दि॒शाम् क्ष॒त्रम् क्ष॒त्रम् दि॒शाम् ध॒र्त्री ध॒र्त्री दि॒शाम् क्ष॒त्रम् । \newline
8. दि॒शाम् क्ष॒त्रम् क्ष॒त्रम् दि॒शाम् दि॒शाम् क्ष॒त्र मि॒द मि॒दम् क्ष॒त्रम् दि॒शाम् दि॒शाम् क्ष॒त्र मि॒दम् । \newline
9. क्ष॒त्र मि॒द मि॒दम् क्ष॒त्रम् क्ष॒त्र मि॒दम् दा॑धार दाधारे॒दम् क्ष॒त्रम् क्ष॒त्र मि॒दम् दा॑धार । \newline
10. इ॒दम् दा॑धार दाधारे॒द मि॒दम् दा॑धा रोप॒स्थो प॒स्था दा॑धारे॒द मि॒दम् दा॑धारोप॒स्था । \newline
11. दा॒धा॒ रो॒प॒स्थो प॒स्था दा॑धार दाधा रोप॒स्था ऽऽशा॑ना॒ माशा॑ना मुप॒स्था दा॑धार दाधा रोप॒स्था ऽऽशा॑नाम् । \newline
12. उ॒प॒स्था ऽऽशा॑ना॒ माशा॑ना मुप॒स्थो प॒स्था ऽऽशा॑नाम् मि॒त्रव॑न् मि॒त्रव॒ दाशा॑ना मुप॒स्थो प॒स्था ऽऽशा॑नाम् मि॒त्रव॑त् । \newline
13. उ॒प॒स्थेत्यु॑प - स्था । \newline
14. आशा॑नाम् मि॒त्रव॑न् मि॒त्रव॒ दाशा॑ना॒ माशा॑नाम् मि॒त्रव॑ दस्त्वस्तु मि॒त्रव॒ दाशा॑ना॒ माशा॑नाम् मि॒त्रव॑ दस्तु । \newline
15. मि॒त्रव॑ दस्त्वस्तु मि॒त्रव॑न् मि॒त्रव॑ द॒स्त्वोज॒ ओजो॑ अस्तु मि॒त्रव॑न् मि॒त्रव॑ द॒स्त्वोजः॑ । \newline
16. मि॒त्रव॒दिति॑ मि॒त्र - व॒त् । \newline
17. अ॒स्त्वोज॒ ओजो॑ अस्त्व॒ स्त्वोजः॑ । \newline
18. ओज॒ इत्योजः॑ । \newline
19. मित्रा॑वरुणा श॒रदा॑ श॒रदा॒ मित्रा॑वरुणा॒ मित्रा॑वरुणा श॒रदा ऽह्ना॒ मह्नाꣳ॑ श॒रदा॒ मित्रा॑वरुणा॒ मित्रा॑वरुणा श॒रदा ऽह्ना᳚म् । \newline
20. मित्रा॑वरु॒णेति॒ मित्रा᳚ - व॒रु॒णा॒ । \newline
21. श॒रदा ऽह्ना॒ मह्नाꣳ॑ श॒रदा॑ श॒रदा ऽह्ना᳚म् चिकित्नू चिकित्नू॒ अह्नाꣳ॑ श॒रदा॑ श॒रदा ऽह्ना᳚म् चिकित्नू । \newline
22. अह्ना᳚म् चिकित्नू चिकित्नू॒ अह्ना॒ मह्ना᳚म् चिकित्नू अ॒स्मा अ॒स्मै चि॑कित्नू॒ अह्ना॒ मह्ना᳚म् चिकित्नू अ॒स्मै । \newline
23. चि॒कि॒त्नू॒ अ॒स्मा अ॒स्मै चि॑कित्नू चिकित्नू अ॒स्मै रा॒ष्ट्राय॑ रा॒ष्ट्राया॒स्मै चि॑कित्नू चिकित्नू अ॒स्मै रा॒ष्ट्राय॑ । \newline
24. चि॒कि॒त्नू॒ इति॑ चिकित्नू । \newline
25. अ॒स्मै रा॒ष्ट्राय॑ रा॒ष्ट्राया॒स्मा अ॒स्मै रा॒ष्ट्राय॒ महि॒ महि॑ रा॒ष्ट्राया॒स्मा अ॒स्मै रा॒ष्ट्राय॒ महि॑ । \newline
26. रा॒ष्ट्राय॒ महि॒ महि॑ रा॒ष्ट्राय॑ रा॒ष्ट्राय॒ महि॒ शर्म॒ शर्म॒ महि॑ रा॒ष्ट्राय॑ रा॒ष्ट्राय॒ महि॒ शर्म॑ । \newline
27. महि॒ शर्म॒ शर्म॒ महि॒ महि॒ शर्म॑ यच्छतं ॅयच्छतꣳ॒॒ शर्म॒ महि॒ महि॒ शर्म॑ यच्छतम् । \newline
28. शर्म॑ यच्छतं ॅयच्छतꣳ॒॒ शर्म॒ शर्म॑ यच्छतम् । \newline
29. य॒च्छ॒त॒मिति॑ यच्छतम् । \newline
30. वै॒रा॒जे साम॒न् थ्साम॑न्. वैरा॒जे वै॑रा॒जे साम॒न् नध्यधि॒ साम॑न्. वैरा॒जे वै॑रा॒जे साम॒न् नधि॑ । \newline
31. साम॒न् नध्यधि॒ साम॒न् थ्साम॒न् नधि॑ मे मे॒ अधि॒ साम॒न् थ्साम॒न् नधि॑ मे । \newline
32. अधि॑ मे मे॒ अध्यधि॑ मे मनी॒षा म॑नी॒षा मे॒ अध्यधि॑ मे मनी॒षा । \newline
33. मे॒ म॒नी॒षा म॑नी॒षा मे॑ मे मनी॒षा ऽनु॒ष्टुभा॑ ऽनु॒ष्टुभा॑ मनी॒षा मे॑ मे मनी॒षा ऽनु॒ष्टुभा᳚ । \newline
34. म॒नी॒षा ऽनु॒ष्टुभा॑ ऽनु॒ष्टुभा॑ मनी॒षा म॑नी॒षा ऽनु॒ष्टुभा॒ संभृ॑तꣳ॒॒ संभृ॑त मनु॒ष्टुभा॑ मनी॒षा म॑नी॒षा ऽनु॒ष्टुभा॒ संभृ॑तम् । \newline
35. अ॒नु॒ष्टुभा॒ संभृ॑तꣳ॒॒ संभृ॑त मनु॒ष्टुभा॑ ऽनु॒ष्टुभा॒ संभृ॑तं ॅवी॒र्यं॑ ॅवी॒र्यꣳ॑ संभृ॑त मनु॒ष्टुभा॑ ऽनु॒ष्टुभा॒ संभृ॑तं ॅवी॒र्य᳚म् । \newline
36. अ॒नु॒ष्टुभेत्य॑नु - स्तुभा᳚ । \newline
37. संभृ॑तं ॅवी॒र्यं॑ ॅवी॒र्यꣳ॑ संभृ॑तꣳ॒॒ संभृ॑तं ॅवी॒र्यꣳ॑ सहः॒ सहो॑ वी॒र्यꣳ॑ 
संभृ॑तꣳ॒॒ संभृ॑तं ॅवी॒र्यꣳ॑ सहः॑ । \newline
38. संभृ॑त॒मिति॒ सं - भृ॒त॒म् । \newline
39. वी॒र्यꣳ॑ सहः॒ सहो॑ वी॒र्यं॑ ॅवी॒र्यꣳ॑ सहः॑ । \newline
40. सह॒ इति॒ सहः॑ । \newline
41. इ॒दम् क्ष॒त्रम् क्ष॒त्र मि॒द मि॒दम् क्ष॒त्रम् मि॒त्रव॑न् मि॒त्रव॑त् क्ष॒त्र मि॒द मि॒दम् क्ष॒त्रम् मि॒त्रव॑त् । \newline
42. क्ष॒त्रम् मि॒त्रव॑न् मि॒त्रव॑त् क्ष॒त्रम् क्ष॒त्रम् मि॒त्रव॑ दा॒र्द्रदा᳚न् वा॒र्द्रदा॑नु मि॒त्रव॑त् क्ष॒त्रम् क्ष॒त्रम् मि॒त्रव॑ दा॒र्द्रदा॑नु । \newline
43. मि॒त्रव॑ दा॒र्द्रदा᳚न् वा॒र्द्रदा॑नु मि॒त्रव॑न् मि॒त्रव॑ दा॒र्द्रदा॑नु॒ मित्रा॑वरुणा॒ मित्रा॑वरुणा॒ ऽऽर्द्रदा॑नु मि॒त्रव॑न् मि॒त्रव॑ दा॒र्द्रदा॑नु॒ मित्रा॑वरुणा । \newline
44. मि॒त्रव॒दिति॑ मि॒त्र - व॒त् । \newline
45. आ॒र्द्रदा॑नु॒ मित्रा॑वरुणा॒ मित्रा॑वरुणा॒ ऽऽर्द्रदा᳚न् वा॒र्द्रदा॑नु॒ मित्रा॑वरुणा॒ रक्ष॑तꣳ॒॒ रक्ष॑त॒म् मित्रा॑वरुणा॒ ऽऽर्द्रदा᳚न् वा॒र्द्रदा॑नु॒ मित्रा॑वरुणा॒ रक्ष॑तम् । \newline
46. आ॒र्द्रदा॒न्वित्या॒र्द्र - दा॒नु॒ । \newline
47. मित्रा॑वरुणा॒ रक्ष॑तꣳ॒॒ रक्ष॑त॒म् मित्रा॑वरुणा॒ मित्रा॑वरुणा॒ रक्ष॑त॒ माधि॑पत्यै॒ राधि॑पत्यै॒ रक्ष॑त॒म् मित्रा॑वरुणा॒ मित्रा॑वरुणा॒ रक्ष॑त॒ माधि॑पत्यैः । \newline
48. मित्रा॑वरु॒णेति॒ मित्रा᳚ - व॒रु॒णा॒ । \newline
49. रक्ष॑त॒ माधि॑पत्यै॒ राधि॑पत्यै॒ रक्ष॑तꣳ॒॒ रक्ष॑त॒ माधि॑पत्यैः । \newline
50. आधि॑पत्यै॒रित्याधि॑ - प॒त्यैः॒ । \newline
51. स॒म्राड् दि॒शाम् दि॒शाꣳ स॒म्राट् थ्स॒म्राड् दि॒शाꣳ स॒हसा᳚म्नी स॒हसा᳚म्नी दि॒शाꣳ स॒म्राट् थ्स॒म्राड् दि॒शाꣳ स॒हसा᳚म्नी । \newline
52. स॒म्राडिति॑ सम् - राट् । \newline
53. दि॒शाꣳ स॒हसा᳚म्नी स॒हसा᳚म्नी दि॒शाम् दि॒शाꣳ स॒हसा᳚म्नी॒ सह॑स्वती॒ सह॑स्वती स॒हसा᳚म्नी दि॒शाम् 
दि॒शाꣳ स॒हसा᳚म्नी॒ सह॑स्वती । \newline
54. स॒हसा᳚म्नी॒ सह॑स्वती॒ सह॑स्वती स॒हसा᳚म्नी स॒हसा᳚म्नी॒ सह॑स्व त्यृ॒तुर्. ऋ॒तुः सह॑स्वती स॒हसा᳚म्नी स॒हसा᳚म्नी॒ सह॑स्व त्यृ॒तुः । \newline
55. स॒हसा॒म्नीति॑ स॒ह - सा॒म्नी॒ । \newline
56. सह॑स्व त्यृ॒तुर्. ऋ॒तुः सह॑स्वती॒ सह॑स्व त्यृ॒तुर्. हे॑म॒न्तो हे॑म॒न्त ऋ॒तुः सह॑स्वती॒ सह॑स्व त्यृ॒तुर्. हे॑म॒न्तः । \newline
57. ऋ॒तुर्. हे॑म॒न्तो हे॑म॒न्त ऋ॒तुर्. ऋ॒तुर्. हे॑म॒न्तो वि॒ष्ठया॑ वि॒ष्ठया॑ हेम॒न्त ऋ॒तुर्. ऋ॒तुर्. हे॑म॒न्तो वि॒ष्ठया᳚ । \newline
58. हे॒म॒न्तो वि॒ष्ठया॑ वि॒ष्ठया॑ हेम॒न्तो हे॑म॒न्तो वि॒ष्ठया॑ नो नो वि॒ष्ठया॑ हेम॒न्तो हे॑म॒न्तो वि॒ष्ठया॑ नः । \newline
59. वि॒ष्ठया॑ नो नो वि॒ष्ठया॑ वि॒ष्ठया॑ नः पिपर्तु पिपर्तु नो वि॒ष्ठया॑ वि॒ष्ठया॑ नः पिपर्तु । \newline
60. वि॒ष्ठयेति॑ वि - स्थया᳚ । \newline
61. नः॒ पि॒प॒र्तु॒ पि॒प॒र्तु॒ नो॒ नः॒ पि॒प॒र्तु॒ । \newline
62. पि॒प॒र्त्विति॑ पिपर्तु । \newline
63. अ॒व॒स्युवा॑ता बृह॒तीर् बृ॑ह॒ती र॑व॒स्युवा॑ता अव॒स्युवा॑ता बृह॒तीर् नु नु बृ॑ह॒ती र॑व॒स्युवा॑ता अव॒स्युवा॑ता बृह॒तीर् नु । \newline
64. अ॒व॒स्युवा॑ता॒ इत्य॑व॒स्यु - वा॒ताः॒ । \newline
\pagebreak
\markright{ TS 4.4.12.4  \hfill https://www.vedavms.in \hfill}

\section{ TS 4.4.12.4 }

\textbf{TS 4.4.12.4 } \newline
\textbf{Samhita Paata} \newline

बृह॒तीर्नु शक्व॑रीरि॒मं ॅय॒ज्ञ्म॑वन्तु नो घृ॒ताचीः᳚ ॥ सुव॑र्वती सु॒दुघा॑ नः॒ पय॑स्वती दि॒शां दे॒व्य॑वतु नो घृ॒ताची᳚ । त्वं गो॒पाः पु॑रए॒तोत प॒श्चाद् बृह॑स्पते॒ याम्यां᳚ ॅयुङ्ग्धि॒ वाचं᳚ ॥ऊ॒र्द्ध्वा दि॒शाꣳ रन्ति॒राशौष॑धीनाꣳ संॅवथ्स॒रेण॑ सवि॒ता नो॒ अह्नां᳚ । रे॒वथ् सामाति॑च्छन्दा उ॒ छन्दोऽजा॑त शत्रुः स्यो॒ना नो॑ अस्तु ॥ स्तोम॑त्रयस्त्रिꣳशे॒ भुव॑नस्य पत्नि॒ विव॑स्वद्वाते अ॒भि नो॑ - [  ] \newline

\textbf{Pada Paata} \newline

बृ॒ह॒तीः । नु । शक्व॑रीः । इ॒मम् । य॒ज्ञ्म् । अ॒व॒न्तु॒ । नः॒ । घृ॒ताचीः᳚ । सुव॑र्व॒तीति॒ सुवः॑ - व॒ती॒ । सु॒दुघेति॑ सु - दुघा᳚ । नः॒ । पय॑स्वती । दि॒शाम् । दे॒वी । अ॒व॒तु॒ । नः॒ । घृ॒ताची᳚ ॥ त्वम् । गो॒पा इति॑ गो - पाः । पु॒र॒ ए॒तेति॑ पुरः - ए॒ता । उ॒त । प॒श्चात् । बृह॑स्पते । याम्या᳚म् । यु॒ङ्ग्धि॒ । वाच᳚म् ॥ ऊ॒द्‌र्ध्वा । दि॒शाम् । रन्तिः॑ । आशा᳚ । ओष॑धीनाम् । सं॒ॅव॒थ्स॒रेणेति॑ सं-व॒थ्स॒रेण॑ । स॒वि॒ता । नः॒ । अह्ना᳚म् ॥ रे॒वत् । साम॑ । अति॑॑च्छन्दा॒ इत्याति॑ - छ॒न्दाः॒ । उ॒ । छन्दः॑ । अजा॑तशत्रु॒रित्यजा॑त - श॒त्रुः॒ । स्यो॒ना । नः॒ । अ॒स्तु॒ ॥ स्तोम॑त्रयस्त्रिꣳश॒ इति॒ स्तोम॑ - त्र॒य॒स्त्रिꣳ॒॒शे॒ । भुव॑नस्य । प॒त्नि॒ । विव॑स्वद्वात॒ इति॒ विव॑स्वत् - वा॒ते॒ । अ॒भीति॑ । नः॒ ।  \newline


\textbf{Krama Paata} \newline

बृ॒ह॒तीर् नु । नु शक्व॑रीः । शक्व॑रीरि॒मम् । इ॒मं ॅय॒ज्ञ्म् । य॒ज्ञ्म॑वन्तु । अ॒व॒न्तु॒ नः॒ । नो॒ घृ॒ताचीः᳚ । घृ॒ताची॒रिति॑ घृ॒ताचीः᳚ ॥ सुव॑र्वती सु॒दुघा᳚ । सुव॑र्व॒तीति॒ सुवः॑ - व॒ती॒ । सु॒दुघा॑ नः । सु॒दुघेति॑ सु - दुघा᳚ । नः॒ पय॑स्वती । पय॑स्वती दि॒शाम् । दि॒शाम् दे॒वी । दे॒व्य॑वतु । अ॒व॒तु॒ नः॒ । नो॒ घृ॒ताची᳚ । घृ॒ताचीति॑ घृ॒ताची᳚ ॥ त्वम् गो॒पाः । गो॒पाः पु॑रए॒ता । गो॒पा इति॑ गो - पाः । पु॒र॒ए॒तोत । पु॒र॒ए॒तेति॑ पुरः - ए॒ता । उ॒त प॒श्चात् । प॒श्चाद् बृह॑स्पते । बृह॑स्पते॒ याम्या᳚म् । याम्या᳚म् ॅयुङ्ग्धि । यु॒ङ्॒ग्धि॒ वाचम्᳚ । वाच॒मिति॒ वाच᳚म् ॥ ऊ॒र्द्ध्वा दि॒शाम् । दि॒शाꣳ रन्तिः॑ । रन्ति॒राशा᳚ । आशौष॑धीनाम् । ओष॑धीनाꣳ सम्ॅवथ्स॒रेण॑ । स॒म्ॅव॒थ्स॒रेण॑ सवि॒ता । स॒म्ॅव॒थ्स॒रेणेति॑ सं - व॒थ्स॒रेण॑ । स॒वि॒ता नः॑ । नो॒ अह्नाम्᳚ । अह्ना॒मित्यह्ना᳚म् ॥ रे॒वथ् साम॑ । सामाति॑छन्दाः । अति॑छन्दा उ । 
अति॑छन्दा॒ इत्यति॑ - छ॒न्दाः॒ । उ॒ छन्दः॑ । छन्दोऽजा॑तशत्रुः । अजा॑तशत्रुः स्यो॒ना । अजा॑तशत्रु॒रित्यजा॑त - श॒त्रुः॒ । स्यो॒ना नः॑ । नो॒ अ॒स्तु॒ । अ॒स्त्वित्य॑स्तु ॥ स्तोम॑त्रयस्त्रिꣳशे॒ भुव॑नस्य । स्तोम॑त्रयस्त्रिꣳश॒ इति॒ स्तोम॑ - त्र॒य॒स्त्रिꣳ॒॒शे॒ । भुव॑नस्य पत्नि । प॒त्नि॒ विव॑स्वद्वाते । विव॑स्वद्वाते अ॒भि । विव॑स्वद्वात॒ इति॒ विव॑स्वत् - वा॒ते॒ । अ॒भि नः॑ । नो॒ गृ॒णा॒हि॒ \newline

\textbf{Jatai Paata} \newline

1. बृ॒ह॒तीर् नु नु बृ॑ह॒तीर् बृ॑ह॒तीर् नु । \newline
2. नु शक्व॑रीः॒ शक्व॑री॒र् नु नु शक्व॑रीः । \newline
3. शक्व॑री रि॒म मि॒मꣳ शक्व॑रीः॒ शक्व॑री रि॒मम् । \newline
4. इ॒मं ॅय॒ज्ञ्ं ॅय॒ज्ञ् मि॒म मि॒मं ॅय॒ज्ञ्म् । \newline
5. य॒ज्ञ् म॑वन् त्ववन्तु य॒ज्ञ्ं ॅय॒ज्ञ् म॑वन्तु । \newline
6. अ॒व॒न्तु॒ नो॒ नो॒ ऽव॒न्त्व॒ व॒न्तु॒ नः॒ । \newline
7. नो॒ घृ॒ताची᳚र् घृ॒ताची᳚र् नो नो घृ॒ताचीः᳚ । \newline
8. घृ॒ताची॒रिति॑ घृ॒ताचीः᳚ । \newline
9. सुव॑र्वती सु॒दुघा॑ सु॒दुघा॒ सुव॑र्वती॒ सुव॑र्वती सु॒दुघा᳚ । \newline
10. सुव॑र्व॒तीति॒ सुवः॑ - व॒ती॒ । \newline
11. सु॒दुघा॑ नो नः सु॒दुघा॑ सु॒दुघा॑ नः । \newline
12. सु॒दुघेति॑ सु - दुघा᳚ । \newline
13. नः॒ पय॑स्वती॒ पय॑स्वती नो नः॒ पय॑स्वती । \newline
14. पय॑स्वती दि॒शाम् दि॒शाम् पय॑स्वती॒ पय॑स्वती दि॒शाम् । \newline
15. दि॒शाम् दे॒वी दे॒वी दि॒शाम् दि॒शाम् दे॒वी । \newline
16. दे॒व्य॑व त्ववतु दे॒वी दे॒व्य॑वतु । \newline
17. अ॒व॒तु॒ नो॒ नो॒ ऽव॒ त्व॒व॒तु॒ नः॒ । \newline
18. नो॒ घृ॒ताची॑ घृ॒ताची॑ नो नो घृ॒ताची᳚ । \newline
19. घृ॒ताची॒रिति॑ घृ॒ताचीः᳚ । \newline
20. त्वम् गो॒पा गो॒पा स्त्वम् त्वम् गो॒पाः । \newline
21. गो॒पाः पु॑र‌ए॒ता पु॑र‌ए॒ता गो॒पा गो॒पाः पु॑र‌ए॒ता । \newline
22. गो॒पा इति॑ गो - पाः । \newline
23. पु॒र॒‌ए॒तोतोत पु॑र‌ए॒ता पु॑र‌ए॒तोत । \newline
24. पु॒र॒‌ए॒तेति॑ पुरः - ए॒ता । \newline
25. उ॒त प॒श्चात् प॒श्चा दु॒तोत प॒श्चात् । \newline
26. प॒श्चाद् बृह॑स्पते॒ बृह॑स्पते प॒श्चात् प॒श्चाद् बृह॑स्पते । \newline
27. बृह॑स्पते॒ याम्यां॒ ॅयाम्या॒म् बृह॑स्पते॒ बृह॑स्पते॒ याम्या᳚म् । \newline
28. याम्यां᳚ ॅयुङ्ग्धि युङ्ग्धि॒ याम्यां॒ ॅयाम्यां᳚ ॅयुङ्ग्धि । \newline
29. यु॒ङ्ग्धि॒ वाचं॒ ॅवाचं॑ ॅयुङ्ग्धि युङ्ग्धि॒ वाच᳚म् । \newline
30. वाच॒मिति॒ वाच᳚म् । \newline
31. ऊ॒र्द्ध्वा दि॒शाम् दि॒शा मू॒र्द्ध्वोर्द्ध्वा दि॒शाम् । \newline
32. दि॒शाꣳ रन्ती॒ रन्ति॑र् दि॒शाम् दि॒शाꣳ रन्तिः॑ । \newline
33. रन्ति॒ राशा ऽऽशा॒ रन्ती॒ रन्ति॒ राशा᳚ । \newline
34. आशौष॑धीना॒ मोष॑धीना॒ माशा ऽऽशौष॑धीनाम् । \newline
35. ओष॑धीनाꣳ संॅवथ्स॒रेण॑ संॅवथ्स॒रेणौ ष॑धीना॒ मोष॑धीनाꣳ संॅवथ्स॒रेण॑ । \newline
36. सं॒ॅव॒थ्स॒रेण॑ सवि॒ता स॑वि॒ता सं॑ॅवथ्स॒रेण॑ संॅवथ्स॒रेण॑ सवि॒ता । \newline
37. सं॒ॅव॒थ्स॒रेणेति॑ सं - व॒थ्स॒रेण॑ । \newline
38. स॒वि॒ता नो॑ नः सवि॒ता स॑वि॒ता नः॑ । \newline
39. नो॒ अह्ना॒ मह्ना᳚म् नो नो॒ अह्ना᳚म् । \newline
40. अह्ना॒मित्यह्ना᳚म् । \newline
41. रे॒वथ् साम॒ साम॑ रे॒वद् रे॒वथ् साम॑ । \newline
42. सामा ति॑च्छन्दा॒ अति॑च्छन्दाः॒ साम॒ सामा ति॑च्छन्दाः । \newline
43. अति॑च्छन्दा उ वु॒ वति॑च्छन्दा॒ अति॑च्छन्दा उ । \newline
44. अति॑॑च्छन्दा॒ इत्यति॑ - छ॒न्दाः॒ । \newline
45. उ॒ छन्द॒ श्छन्द॑ उ वु॒ छन्दः॑ । \newline
46. छन्दो ऽजा॑तशत्रु॒ रजा॑तशत्रु॒ श्छन्द॒ श्छन्दो ऽजा॑तशत्रुः । \newline
47. अजा॑तशत्रुः स्यो॒ना स्यो॒ना ऽजा॑तशत्रु॒ रजा॑तशत्रुः स्यो॒ना । \newline
48. अजा॑तशत्रु॒रित्यजा॑त - श॒त्रुः॒ । \newline
49. स्यो॒ना नो॑ नः स्यो॒ना स्यो॒ना नः॑ । \newline
50. नो॒ अ॒स्त्व॒स्तु॒ नो॒ नो॒ अ॒स्तु॒ । \newline
51. अ॒स्त्वित्य॑स्तु । \newline
52. स्तोम॑त्रयस्त्रिꣳशे॒ भुव॑नस्य॒ भुव॑नस्य॒ स्तोम॑त्रयस्त्रिꣳशे॒ स्तोम॑त्रयस्त्रिꣳशे॒ भुव॑नस्य । \newline
53. स्तोम॑त्रयस्त्रिꣳश॒ इति॒ स्तोम॑ - त्र॒य॒स्त्रिꣳ॒॒शे॒ । \newline
54. भुव॑नस्य पत्नि पत्नि॒ भुव॑नस्य॒ भुव॑नस्य पत्नि । \newline
55. प॒त्नि॒ विव॑स्वद्वाते॒ विव॑स्वद्वाते पत्नि पत्नि॒ विव॑स्वद्वाते । \newline
56. विव॑स्वद्वाते अ॒भ्य॑भि विव॑स्वद्वाते॒ विव॑स्वद्वाते अ॒भि । \newline
57. विव॑स्वद्वात॒ इति॒ विव॑स्वत् - वा॒ते॒ । \newline
58. अ॒भि नो॑ नो अ॒भ्य॑भि नः॑ । \newline
59. नो॒ गृ॒णा॒हि॒ गृ॒णा॒हि॒ नो॒ नो॒ गृ॒णा॒हि॒ । \newline

\textbf{Ghana Paata } \newline

1. बृ॒ह॒तीर् नु नु बृ॑ह॒तीर् बृ॑ह॒तीर् नु शक्व॑रीः॒ शक्व॑री॒र् नु बृ॑ह॒तीर् बृ॑ह॒तीर् नु शक्व॑रीः । \newline
2. नु शक्व॑रीः॒ शक्व॑री॒र् नु नु शक्व॑री रि॒म मि॒मꣳ शक्व॑री॒र् नु नु शक्व॑री रि॒मम् । \newline
3. शक्व॑री रि॒म मि॒मꣳ शक्व॑रीः॒ शक्व॑री रि॒मं ॅय॒ज्ञ्ं ॅय॒ज्ञ् मि॒मꣳ शक्व॑रीः॒ शक्व॑री रि॒मं ॅय॒ज्ञ्म् । \newline
4. इ॒मं ॅय॒ज्ञ्ं ॅय॒ज्ञ् मि॒म मि॒मं ॅय॒ज्ञ् म॑वन् त्ववन्तु य॒ज्ञ् मि॒म मि॒मं ॅय॒ज्ञ् म॑वन्तु । \newline
5. य॒ज्ञ् म॑वन् त्ववन्तु य॒ज्ञ्ं ॅय॒ज्ञ् म॑वन्तु नो नो ऽवन्तु य॒ज्ञ्ं ॅय॒ज्ञ् म॑वन्तु नः । \newline
6. अ॒व॒न्तु॒ नो॒ नो॒ ऽव॒न् त्व॒व॒न्तु॒ नो॒ घृ॒ताची᳚र् घृ॒ताची᳚र् नो ऽवन् त्ववन्तु नो घृ॒ताचीः᳚ । \newline
7. नो॒ घृ॒ताची᳚र् घृ॒ताची᳚र् नो नो घृ॒ताचीः᳚ । \newline
8. घृ॒ताची॒रिति॑ घृ॒ताचीः᳚ । \newline
9. सुव॑र्वती सु॒दुघा॑ सु॒दुघा॒ सुव॑र्वती॒ सुव॑र्वती सु॒दुघा॑ नो नः सु॒दुघा॒ सुव॑र्वती॒ सुव॑र्वती सु॒दुघा॑ नः । \newline
10. सुव॑र्व॒तीति॒ सुवः॑ - व॒ती॒ । \newline
11. सु॒दुघा॑ नो नः सु॒दुघा॑ सु॒दुघा॑ नः॒ पय॑स्वती॒ पय॑स्वती नः सु॒दुघा॑ सु॒दुघा॑ नः॒ पय॑स्वती । \newline
12. सु॒दुघेति॑ सु - दुघा᳚ । \newline
13. नः॒ पय॑स्वती॒ पय॑स्वती नो नः॒ पय॑स्वती दि॒शाम् दि॒शाम् पय॑स्वती नो नः॒ पय॑स्वती दि॒शाम् । \newline
14. पय॑स्वती दि॒शाम् दि॒शाम् पय॑स्वती॒ पय॑स्वती दि॒शाम् दे॒वी दे॒वी दि॒शाम् पय॑स्वती॒ पय॑स्वती दि॒शाम् दे॒वी । \newline
15. दि॒शाम् दे॒वी दे॒वी दि॒शाम् दि॒शाम् दे॒व्य॑व त्ववतु दे॒वी दि॒शाम् दि॒शाम् दे॒व्य॑वतु । \newline
16. दे॒व्य॑व त्ववतु दे॒वी दे॒व्य॑वतु नो नो ऽवतु दे॒वी दे॒व्य॑वतु नः । \newline
17. अ॒व॒तु॒ नो॒ नो॒ ऽव॒त्व॒ व॒तु॒ नो॒ घृ॒ताची॑ घृ॒ताची॑ नो ऽवत्ववतु नो घृ॒ताची᳚ । \newline
18. नो॒ घृ॒ताची॑ घृ॒ताची॑ नो नो घृ॒ताची᳚ । \newline
19. घृ॒ताचीति॑ घृ॒ताचीः᳚ । \newline
20. त्वम् गो॒पा गो॒पा स्त्वम् त्वम् गो॒पाः पु॑र‌ए॒ता पु॑र‌ए॒ता गो॒पा स्त्वम् त्वम् गो॒पाः पु॑र‌ए॒ता । \newline
21. गो॒पाः पु॑र‌ए॒ता पु॑र‌ए॒ता गो॒पा गो॒पाः पु॑र‌ए॒तोतोत पु॑र‌ए॒ता गो॒पा गो॒पाः पु॑र‌ए॒तोत । \newline
22. गो॒पा इति॑ गो - पाः । \newline
23. पु॒र॒‌ए॒तोतोत पु॑र‌ए॒ता पु॑र‌ए॒तोत प॒श्चात् प॒श्चादु॒त पु॑र‌ए॒ता पु॑र‌ए॒तोत प॒श्चात् । \newline
24. पु॒र॒‌ए॒तेति॑ पुरः - ए॒ता । \newline
25. उ॒त प॒श्चात् प॒श्चा दु॒तोत प॒श्चाद् बृह॑स्पते॒ बृह॑स्पते प॒श्चा दु॒तोत प॒श्चाद् बृह॑स्पते । \newline
26. प॒श्चाद् बृह॑स्पते॒ बृह॑स्पते प॒श्चात् प॒श्चाद् बृह॑स्पते॒ याम्यां॒ ॅयाम्या॒म् बृह॑स्पते प॒श्चात् 
प॒श्चाद् बृह॑स्पते॒ याम्या᳚म् । \newline
27. बृह॑स्पते॒ याम्यां॒ ॅयाम्या॒म् बृह॑स्पते॒ बृह॑स्पते॒ याम्यां᳚ ॅयुङ्ग्धि युङ्ग्धि॒ याम्या॒म् बृह॑स्पते॒ बृह॑स्पते॒ याम्यां᳚ ॅयुङ्ग्धि । \newline
28. याम्यां᳚ ॅयुङ्ग्धि युङ्ग्धि॒ याम्यां॒ ॅयाम्यां᳚ ॅयुङ्ग्धि॒ वाचं॒ ॅवाचं॑ ॅयुङ्ग्धि॒ याम्यां॒ ॅयाम्यां᳚ ॅयुङ्ग्धि॒ वाच᳚म् । \newline
29. यु॒ङ्ग्धि॒ वाचं॒ ॅवाचं॑ ॅयुङ्ग्धि युङ्ग्धि॒ वाच᳚म् । \newline
30. वाच॒मिति॒ वाच᳚म् । \newline
31. ऊ॒र्द्ध्वा दि॒शाम् दि॒शा मू॒र्द्ध्वोर्द्ध्वा दि॒शाꣳ रन्ती॒ रन्ति॑र् दि॒शा मू॒र्द्ध्वोर्द्ध्वा दि॒शाꣳ रन्तिः॑ । \newline
32. दि॒शाꣳ रन्ती॒ रन्ति॑र् दि॒शाम् दि॒शाꣳ रन्ति॒ राशा ऽऽशा॒ रन्ति॑र् दि॒शाम् दि॒शाꣳ रन्ति॒ राशा᳚ । \newline
33. रन्ति॒राशा ऽऽशा॒ रन्ती॒ रन्ति॒ राशौष॑धीना॒ मोष॑धीना॒ माशा॒ रन्ती॒ रन्ति॒ राशौष॑धीनाम् । \newline
34. आशौष॑धीना॒ मोष॑धीना॒ माशा ऽऽशौष॑धीनाꣳ संॅवथ्स॒रेण॑ संॅवथ्स॒रे णौष॑धीना॒ माशा ऽऽशौष॑धीनाꣳ संॅवथ्स॒रेण॑ । \newline
35. ओष॑धीनाꣳ संॅवथ्स॒रेण॑ संॅवथ्स॒रे णौष॑धीना॒ मोष॑धीनाꣳ संॅवथ्स॒रेण॑ सवि॒ता स॑वि॒ता सं॑ॅवथ्स॒रे णौष॑धीना॒ मोष॑धीनाꣳ संॅवथ्स॒रेण॑ सवि॒ता । \newline
36. सं॒ॅव॒थ्स॒रेण॑ सवि॒ता स॑वि॒ता सं॑ॅवथ्स॒रेण॑ संॅवथ्स॒रेण॑ सवि॒ता नो॑ नः सवि॒ता सं॑ॅवथ्स॒रेण॑ संॅवथ्स॒रेण॑ सवि॒ता नः॑ । \newline
37. सं॒ॅव॒थ्स॒रेणेति॑ सं - व॒थ्स॒रेण॑ । \newline
38. स॒वि॒ता नो॑ नः सवि॒ता स॑वि॒ता नो॒ अह्ना॒ मह्ना᳚म् नः सवि॒ता स॑वि॒ता नो॒ अह्ना᳚म् । \newline
39. नो॒ अह्ना॒ मह्ना᳚म् नो नो॒ अह्ना᳚म् । \newline
40. अह्ना॒मित्यह्ना᳚म् । \newline
41. रे॒वथ् साम॒ साम॑ रे॒वद् रे॒वथ् सामाति॑च्छन्दा॒ अति॑च्छन्दाः॒ साम॑ रे॒वद् रे॒वथ् सामाति॑च्छन्दाः । \newline
42. सामाति॑च्छन्दा॒ अति॑च्छन्दाः॒ साम॒ सामाति॑च्छन्दा उ वु॒ वति॑च्छन्दाः॒ साम॒ सामाति॑च्छन्दा उ । \newline
43. अति॑च्छन्दा उ वु॒ वति॑च्छन्दा॒ अति॑च्छन्दा उ॒ छन्द॒ श्छन्द॑ उ॒ वति॑च्छन्दा॒ अति॑च्छन्दा उ॒ छन्दः॑ । \newline
44. अति॑॑च्छन्दा॒ इत्यति॑ - छ॒न्दाः॒ । \newline
45. उ॒ छन्द॒ श्छन्द॑ उ वु॒ छन्दो ऽजा॑तशत्रु॒ रजा॑तशत्रु॒ श्छन्द॑ उ वु॒ छन्दो ऽजा॑तशत्रुः । \newline
46. छन्दो ऽजा॑तशत्रु॒ रजा॑तशत्रु॒ श्छन्द॒ श्छन्दो ऽजा॑तशत्रुः स्यो॒ना स्यो॒ना ऽजा॑तशत्रु॒ श्छन्द॒ श्छन्दो ऽजा॑तशत्रुः स्यो॒ना । \newline
47. अजा॑तशत्रुः स्यो॒ना स्यो॒ना ऽजा॑तशत्रु॒ रजा॑तशत्रुः स्यो॒ना नो॑ नः स्यो॒ना ऽजा॑तशत्रु॒ रजा॑तशत्रुः स्यो॒ना नः॑ । \newline
48. अजा॑तशत्रु॒रित्यजा॑त - श॒त्रुः॒ । \newline
49. स्यो॒ना नो॑ नः स्यो॒ना स्यो॒ना नो॑ अस्त्वस्तु नः स्यो॒ना स्यो॒ना नो॑ अस्तु । \newline
50. नो॒ अ॒स्त्व॒स्तु॒ नो॒ नो॒ अ॒स्तु॒ । \newline
51. अ॒स्त्वित्य॑स्तु । \newline
52. स्तोम॑त्रयस्त्रिꣳशे॒ भुव॑नस्य॒ भुव॑नस्य॒ स्तोम॑त्रयस्त्रिꣳशे॒ स्तोम॑त्रयस्त्रिꣳशे॒ भुव॑नस्य पत्नि पत्नि॒ भुव॑नस्य॒ स्तोम॑त्रयस्त्रिꣳशे॒ स्तोम॑त्रयस्त्रिꣳशे॒ भुव॑नस्य पत्नि । \newline
53. स्तोम॑त्रयस्त्रिꣳश॒ इति॒ स्तोम॑ - त्र॒य॒स्त्रिꣳ॒॒शे॒ । \newline
54. भुव॑नस्य पत्नि पत्नि॒ भुव॑नस्य॒ भुव॑नस्य पत्नि॒ विव॑स्वद्वाते॒ विव॑स्वद्वाते पत्नि॒ भुव॑नस्य॒ भुव॑नस्य पत्नि॒ विव॑स्वद्वाते । \newline
55. प॒त्नि॒ विव॑स्वद्वाते॒ विव॑स्वद्वाते पत्नि पत्नि॒ विव॑स्वद्वाते अ॒भ्य॑भि विव॑स्वद्वाते पत्नि पत्नि॒ विव॑स्वद्वाते अ॒भि । \newline
56. विव॑स्वद्वाते अ॒भ्य॑भि विव॑स्वद्वाते॒ विव॑स्वद्वाते अ॒भि नो॑ नो अ॒भि विव॑स्वद्वाते॒ विव॑स्वद्वाते अ॒भि नः॑ । \newline
57. विव॑स्वद्वात॒ इति॒ विव॑स्वत् - वा॒ते॒ । \newline
58. अ॒भि नो॑ नो अ॒भ्य॑भि नो॑ गृणाहि गृणाहि नो अ॒भ्य॑भि नो॑ गृणाहि । \newline
59. नो॒ गृ॒णा॒हि॒ गृ॒णा॒हि॒ नो॒ नो॒ गृ॒णा॒हि॒ । \newline
\pagebreak
\markright{ TS 4.4.12.5  \hfill https://www.vedavms.in \hfill}

\section{ TS 4.4.12.5 }

\textbf{TS 4.4.12.5 } \newline
\textbf{Samhita Paata} \newline

गृणाहि । घृ॒तव॑ती सवित॒राधि॑पत्यैः॒ पय॑स्वती॒ रन्ति॒राशा॑ नो अस्तु ॥ ध्रु॒वा दि॒शां ॅविष्णु॑प॒त्न्यघो॑रा॒ऽस्येशा॑ना॒ सह॑सो॒ या म॒नोता᳚ । बृह॒स्पति॑ र्मात॒रिश्वो॒त वा॒युः स॑न्धुवा॒ना वाता॑ अ॒भि नो॑ गृणन्तु ॥ वि॒ष्ट॒भ्ॐ दि॒वो ध॒रुणः॑ पृथि॒व्या अ॒स्येशा॑ना॒ जग॑तो॒ विष्णु॑पत्नी । वि॒श्वव्य॑चा इ॒षय॑न्ती॒ सुभू॑तिः शि॒वा नो॑ अ॒स्त्वदि॑तिरु॒पस्थे᳚ ॥ वै॒श्वा॒न॒रो न॑ ऊ॒त्या>3, पृ॒ष्टो दि॒व्य>4, नु॑ नो॒ ( ) ऽद्यानु॑मति॒>5, रन्विद॑नुमते॒ त्वं >6, कया॑ नश्चि॒त्र आभु॑व॒त्>7, को अ॒द्य यु॑ङ्क्ते >8 ॥ \newline

\textbf{Pada Paata} \newline

गृ॒णा॒हि॒ ॥ घृ॒तव॒तीति॑ घृ॒त - व॒ती॒ । स॒वि॒तः॒ । आधि॑पत्यै॒रित्याधि॑-प॒त्यैः॒ । पय॑स्वती । रन्तिः॑ । आशा᳚ । नः॒ । अ॒स्तु॒ ॥ ध्रु॒वा । दि॒शाम् । विष्णु॑प॒त्नीति॒ विष्णु॑ - प॒त्नी॒ । अघो॑रा । अ॒स्य । ईशा॑ना । सह॑सः । या । म॒नोता᳚ ॥ बृह॒स्पतिः॑ । मा॒त॒रिश्वा᳚ । उ॒त । वा॒युः । स॒न्ध॒वा॒ना इति॑ सम् - धु॒वा॒नाः । वाताः᳚ । अ॒भीति॑ । नः॒ । गृ॒ण॒न्तु॒ ॥ वि॒ष्ट॒भं इति॑ वि-स्त॒भंः । दि॒वः । ध॒रुणः॑ । पृ॒थि॒व्याः । अ॒स्य । ईशा॑ना । जग॑तः । विष्णु॑प॒त्नीति॒ विष्णु॑ - प॒त्नी॒ ॥ वि॒श्वव्य॑चा॒ इति॑ वि॒श्व - व्य॒चाः॒ । इ॒षय॑न्ती । सुभू॑ति॒रिति॒ सु-भू॒तिः॒ । शि॒वा । नः॒ । अ॒स्तु॒ । अदि॑तिः । उ॒पस्थ॒ इत्यु॒प - स्थे॒ ॥ वै॒श्वा॒न॒रः । नः॒ । ऊ॒त्या । पृ॒ष्टः । दि॒वि । अन्विति॑ । नः॒ ( ) । अ॒द्य । अनु॑मति॒रित्यनु॑ - म॒तिः॒ । अन्विति॑ । इत् । अ॒नु॒म॒त॒ इत्य॑नु - म॒ते॒ । त्वम् । कया᳚ । नः॒ । चि॒त्रः । एति॑ । भु॒व॒त् । कः । अ॒द्य । यु॒ङ्क्ते॒ ॥  \newline


\textbf{Krama Paata} \newline

गृ॒णा॒हीति॑ गृणाहि ॥ घृ॒तव॑ती सवितः । घृ॒तव॒तीति॑ घृ॒त - व॒ती॒ । स॒वि॒त॒राधि॑पत्यैः । आधि॑पत्यैः॒ पय॑स्वती । आधि॑पत्यै॒रित्याधि॑ - प॒त्यैः॒ । पय॑स्वती॒ रन्तिः॑ । रन्ति॒राशा᳚ । आशा॑ नः । नो॒ अ॒स्तु॒ । अ॒स्त्वित्य॑स्तु ॥ ध्रु॒वा दि॒शाम् । दि॒शां ॅविष्णु॑पत्नी । विष्णु॑प॒त्न्यघो॑रा । विष्णु॑प॒त्नीति॒ विष्णु॑ - प॒त्नी॒ । अघो॑रा॒ऽस्य । अ॒स्येशा॑ना । ईशा॑ना॒ सह॑सः । सह॑सो॒ या । या म॒नोता᳚ । म॒नोतेति॑ म॒नोता᳚ ॥ बृह॒स्पति॑र् मात॒रिश्वा᳚ । मा॒त॒रिश्वो॒त । उ॒त वा॒युः । वा॒युः स॑न्धुवा॒नाः । स॒न्धु॒वा॒ना वाताः᳚ । स॒न्धु॒वा॒ना इति॑ सम् - धु॒वा॒नाः । वाता॑ अ॒भि । अ॒भि नः॑ । नो॒ गृ॒ण॒न्तु॒ । गृ॒ण॒न्त्विति॑ गृणन्तु ॥ वि॒ष्ट॒म्भो दि॒वः । वि॒ष्ट॒म्भ इति॑ वि - स्त॒म्भः । दि॒वो ध॒रुणः॑ । ध॒रुणः॑ पृथि॒व्याः । पृ॒थि॒व्या अ॒स्य । अ॒स्येशा॑ना । ईशा॑ना॒ जग॑तः । जग॑तो॒ विष्णु॑पत्नी । विष्णु॑प॒त्नीति॒ विष्णु॑ - प॒त्नी॒ ॥ वि॒श्वव्य॑चा इ॒षय॑न्ती । वि॒श्वव्य॑चा॒ इति॑ वि॒श्व - व्य॒चाः॒ । इ॒षय॑न्ती॒ सुभू॑तिः । सुभू॑तिः शि॒वा । सुभू॑ति॒रिति॒ सु - भू॒तिः॒ । शि॒वा नः॑ । नो॒ अ॒स्तु॒ । अ॒स्त्वदि॑तिः । अदि॑तिरु॒पस्थे᳚ । उ॒पस्थ॒ इत्यु॒प - स्थे॒ ॥ वै॒श्वा॒न॒रो नः॑ । न॒ ऊ॒त्या । ऊ॒त्या पृ॒ष्टः । पृ॒ष्टो दि॒वि । दि॒व्यनु॑ । अनु॑ नः ( ) । नो॒ऽद्य । अ॒द्यानु॑मतिः । अनु॑मति॒रनु॑ । अनु॑मति॒रित्यनु॑ - म॒तिः॒ । अन्वित् । इद॑नुमते । अ॒नु॒म॒ते॒ त्वम् । अ॒नु॒म॒त॒ इत्य॑नु - म॒ते॒ । त्वम् कया᳚ । कया॑ नः । न॒श्चि॒त्रः । चि॒त्र आ । आ भु॑वत् । भु॒व॒त् कः । को अ॒द्य । अ॒द्य यु॑ङ्क्ते । यु॒ङ्क्त॒ इति॑ युङ्क्ते । \newline

\textbf{Jatai Paata} \newline

1. गृ॒णा॒हीति॑ गृणाहि । \newline
2. घृ॒तव॑ती सवितः सवितर् घृ॒तव॑ती घृ॒तव॑ती सवितः । \newline
3. घृ॒तव॒तीति॑ घृ॒त - व॒ती॒ । \newline
4. स॒वि॒त॒ राधि॑पत्यै॒ राधि॑पत्यैः सवितः सवित॒ राधि॑पत्यैः । \newline
5. आधि॑पत्यैः॒ पय॑स्वती॒ पय॑स्व॒ त्याधि॑पत्यै॒ राधि॑पत्यैः॒ पय॑स्वती । \newline
6. आधि॑पत्यै॒रित्याधि॑ - प॒त्यैः॒ । \newline
7. पय॑स्वती॒ रन्ती॒ रन्तिः॒ पय॑स्वती॒ पय॑स्वती॒ रन्तिः॑ । \newline
8. रन्ति॒ राशा ऽऽशा॒ रन्ती॒ रन्ति॒ राशा᳚ । \newline
9. आशा॑ नो न॒ आशा ऽऽशा॑ नः । \newline
10. नो॒ अ॒स्त्व॒स्तु॒ नो॒ नो॒ अ॒स्तु॒ । \newline
11. अ॒स्त्वित्य॑स्तु । \newline
12. ध्रु॒वा दि॒शाम् दि॒शाम् ध्रु॒वा ध्रु॒वा दि॒शाम् । \newline
13. दि॒शां ॅविष्णु॑पत्नी॒ विष्णु॑पत्नी दि॒शाम् दि॒शां ॅविष्णु॑पत्नी । \newline
14. विष्णु॑प॒त् न्यघो॒रा ऽघो॑रा॒ विष्णु॑पत्नी॒ विष्णु॑प॒त् न्यघो॑रा । \newline
15. विष्णु॑प॒त्नीति॒ विष्णु॑ - प॒त्नी॒ । \newline
16. अघो॑रा॒ ऽस्यास्या घो॒रा ऽघो॑रा॒ ऽस्य । \newline
17. अ॒स्येशा॒ नेशा॑ना॒ ऽस्या स्येशा॑ना । \newline
18. ईशा॑ना॒ सह॑सः॒ सह॑स॒ ईशा॒ने शा॑ना॒ सह॑सः । \newline
19. सह॑सो॒ या या सह॑सः॒ सह॑सो॒ या । \newline
20. या म॒नोता॑ म॒नोता॒ या या म॒नोता᳚ । \newline
21. म॒नोतेति॑ म॒नोता᳚ । \newline
22. बृह॒स्पति॑र् मात॒रिश्वा॑ मात॒रिश्वा॒ बृह॒स्पति॒र् बृह॒स्पति॑र् मात॒रिश्वा᳚ । \newline
23. मा॒त॒रिश्वो॒तोत मा॑त॒रिश्वा॑ मात॒रिश्वो॒त । \newline
24. उ॒त वा॒युर् वा॒यु रु॒तोत वा॒युः । \newline
25. वा॒युः स॑न्धुवा॒नाः स॑न्धुवा॒ना वा॒युर् वा॒युः स॑न्धुवा॒नाः । \newline
26. स॒न्धु॒वा॒ना वाता॒ वाताः᳚ सन्धुवा॒नाः स॑न्धुवा॒ना वाताः᳚ । \newline
27. स॒न्धु॒वा॒ना इति॑ सम् - धु॒वा॒नाः । \newline
28. वाता॑ अ॒भ्य॑भि वाता॒ वाता॑ अ॒भि । \newline
29. अ॒भि नो॑ नो अ॒भ्य॑भि नः॑ । \newline
30. नो॒ गृ॒ण॒न्तु॒ गृ॒ण॒न्तु॒ नो॒ नो॒ गृ॒ण॒न्तु॒ । \newline
31. गृ॒ण॒न्त्विति॑ गृणन्तु । \newline
32. वि॒ष्टं॒भो दि॒वो दि॒वो वि॑ष्टं॒भो वि॑ष्टं॒भो दि॒वः । \newline
33. वि॒ष्टं॒भ इति॑ वि - स्तं॒भः । \newline
34. दि॒वो ध॒रुणो॑ ध॒रुणो॑ दि॒वो दि॒वो ध॒रुणः॑ । \newline
35. ध॒रुणः॑ पृथि॒व्याः पृ॑थि॒व्या ध॒रुणो॑ ध॒रुणः॑ पृथि॒व्याः । \newline
36. पृ॒थि॒व्या अ॒स्यास्य पृ॑थि॒व्याः पृ॑थि॒व्या अ॒स्य । \newline
37. अ॒स्येशा॒ नेशा॑ना॒ ऽस्यास्ये शा॑ना । \newline
38. ईशा॑ना॒ जग॑तो॒ जग॑त॒ ईशा॒ने शा॑ना॒ जग॑तः । \newline
39. जग॑तो॒ विष्णु॑पत्नी॒ विष्णु॑पत्नी॒ जग॑तो॒ जग॑तो॒ विष्णु॑पत्नी । \newline
40. विष्णु॑प॒त्नीति॒ विष्णु॑ - प॒त्नी॒ । \newline
41. वि॒श्वव्य॑चा इ॒षय॑न्ती॒ षय॑न्ती वि॒श्वव्य॑चा वि॒श्वव्य॑चा इ॒षय॑न्ती । \newline
42. वि॒श्वव्य॑चा॒ इति॑ वि॒श्व - व्य॒चाः॒ । \newline
43. इ॒षय॑न्ती॒ सुभू॑तिः॒ सुभू॑ति रि॒षय॑न्ती॒ षय॑न्ती॒ सुभू॑तिः । \newline
44. सुभू॑तिः शि॒वा शि॒वा सुभू॑तिः॒ सुभू॑तिः शि॒वा । \newline
45. सुभू॑ति॒रिति॒ सु - भू॒तिः॒ । \newline
46. शि॒वा नो॑ नः शि॒वा शि॒वा नः॑ । \newline
47. नो॒ अ॒स्त्व॒स्तु॒ नो॒ नो॒ अ॒स्तु॒ । \newline
48. अ॒स्त्व दि॑ति॒ रदि॑ति रस्त्व॒ स्त्वदि॑तिः । \newline
49. अदि॑ति रु॒पस्थ॑ उ॒पस्थे॒ अदि॑ति॒ रदि॑ति रु॒पस्थे᳚ । \newline
50. उ॒पस्थ॒ इत्यु॒प - स्थे॒ । \newline
51. वै॒श्वा॒न॒रो नो॑ नो वैश्वान॒रो वै᳚श्वान॒रो नः॑ । \newline
52. न॒ ऊ॒त्योत्या नो॑ न ऊ॒त्या । \newline
53. ऊ॒त्या पृ॒ष्टः पृ॒ष्ट ऊ॒त्योत्या पृ॒ष्टः । \newline
54. पृ॒ष्टो दि॒वि दि॒वि पृ॒ष्टः पृ॒ष्टो दि॒वि । \newline
55. दि॒व्यन् वनु॑ दि॒वि दि॒व्यनु॑ । \newline
56. अनु॑ नो नो॒ अन्वनु॑ नः । \newline
57. नो॒ ऽद्याद्य नो॑ नो॒ ऽद्य । \newline
58. अ॒द्या नु॑मति॒ रनु॑मति र॒द्याद्या नु॑मतिः । \newline
59. अनु॑मति॒ रन्वन्व नु॑मति॒ रनु॑मति॒ रनु॑ । \newline
60. अनु॑मति॒रित्यनु॑ - म॒तिः॒ । \newline
61. अन्वि दिदन् वन्वित् । \newline
62. इद॑नुमते ऽनुमत॒ इदि द॑नुमते । \newline
63. अ॒नु॒म॒ते॒ त्वम् त्व म॑नुमते ऽनुमते॒ त्वम् । \newline
64. अ॒नु॒म॒त॒ इत्य॑नु - म॒ते॒ । \newline
65. त्वम् कया॒ कया॒ त्वम् त्वम् कया᳚ । \newline
66. कया॑ नो नः॒ कया॒ कया॑ नः । \newline
67. न॒ श्चि॒त्र श्चि॒त्रो नो॑ नश्चि॒त्रः । \newline
68. चि॒त्र आ चि॒त्र श्चि॒त्र आ । \newline
69. आ भु॑वद् भुव॒दा भु॑वत् । \newline
70. भु॒व॒त् कः को भु॑वद् भुव॒त् कः । \newline
71. को अ॒द्याद्य कः को अ॒द्य । \newline
72. अ॒द्य यु॑ङ्क्ते युङ्क्ते अ॒द्याद्य यु॑ङ्क्ते । \newline
73. यु॒ङ्क्त॒ इति॑ युङ्क्ते । \newline

\textbf{Ghana Paata } \newline

1. गृ॒णा॒हीति॑ गृणाहि । \newline
2. घृ॒तव॑ती सवितः सवितर् घृ॒तव॑ती घृ॒तव॑ती सवित॒ राधि॑पत्यै॒ राधि॑पत्यैः सवितर् घृ॒तव॑ती घृ॒तव॑ती सवित॒ राधि॑पत्यैः । \newline
3. घृ॒तव॒तीति॑ घृ॒त - व॒ती॒ । \newline
4. स॒वि॒त॒ राधि॑पत्यै॒ राधि॑पत्यैः सवितः सवित॒ राधि॑पत्यैः॒ पय॑स्वती॒ पय॑स्व॒ त्याधि॑पत्यैः सवितः सवित॒ राधि॑पत्यैः॒ पय॑स्वती । \newline
5. आधि॑पत्यैः॒ पय॑स्वती॒ पय॑स्व॒ त्याधि॑पत्यै॒ राधि॑पत्यैः॒ पय॑स्वती॒ रन्ती॒ रन्तिः॒ पय॑स्व॒ त्याधि॑पत्यै॒ राधि॑पत्यैः॒ पय॑स्वती॒ रन्तिः॑ । \newline
6. आधि॑पत्यै॒रित्याधि॑ - प॒त्यैः॒ । \newline
7. पय॑स्वती॒ रन्ती॒ रन्तिः॒ पय॑स्वती॒ पय॑स्वती॒ रन्ति॒ राशा ऽऽशा॒ रन्तिः॒ पय॑स्वती॒ पय॑स्वती॒ रन्ति॒ राशा᳚ । \newline
8. रन्ति॒ राशा ऽऽशा॒ रन्ती॒ रन्ति॒ राशा॑ नो न॒ आशा॒ रन्ती॒ रन्ति॒ राशा॑ नः । \newline
9. आशा॑ नो न॒ आशा ऽऽशा॑ नो अस्त्वस्तु न॒ आशा ऽऽशा॑ नो अस्तु । \newline
10. नो॒ अ॒स्त्व॒स्तु॒ नो॒ नो॒ अ॒स्तु॒ । \newline
11. अ॒स्त्वित्य॑स्तु । \newline
12. ध्रु॒वा दि॒शाम् दि॒शाम् ध्रु॒वा ध्रु॒वा दि॒शां ॅविष्णु॑पत्नी॒ विष्णु॑पत्नी दि॒शाम् ध्रु॒वा ध्रु॒वा दि॒शां ॅविष्णु॑पत्नी । \newline
13. दि॒शां ॅविष्णु॑पत्नी॒ विष्णु॑पत्नी दि॒शाम् दि॒शां ॅविष्णु॑प॒त्न्यघो॒रा ऽघो॑रा॒ विष्णु॑पत्नी दि॒शाम् दि॒शां ॅविष्णु॑प॒त्न्यघो॑रा । \newline
14. विष्णु॑प॒त्न्यघो॒रा ऽघो॑रा॒ विष्णु॑पत्नी॒ विष्णु॑प॒त्न्यघो॑रा॒ ऽस्यास्या घो॑रा॒ विष्णु॑पत्नी॒ 
विष्णु॑प॒त्न्यघो॑रा॒ ऽस्य । \newline
15. विष्णु॑प॒त्नीति॒ विष्णु॑ - प॒त्नी॒ । \newline
16. अघो॑रा॒ ऽस्यास्या घो॒रा ऽघो॑रा॒ ऽस्येशा॒ नेशा॑ना॒ ऽस्याघो॒रा ऽघो॑रा॒ ऽस्येशा॑ना । \newline
17. अ॒स्येशा॒ नेशा॑ना॒ ऽस्यास्ये शा॑ना॒ सह॑सः॒ सह॑स॒ ईशा॑ना॒ ऽस्या स्येशा॑ना॒ सह॑सः । \newline
18. ईशा॑ना॒ सह॑सः॒ सह॑स॒ ईशा॒ नेशा॑ना॒ सह॑सो॒ या या सह॑स॒ ईशा॒ नेशा॑ना॒ सह॑सो॒ या । \newline
19. सह॑सो॒ या या सह॑सः॒ सह॑सो॒ या म॒नोता॑ म॒नोता॒ या सह॑सः॒ सह॑सो॒ या म॒नोता᳚ । \newline
20. या म॒नोता॑ म॒नोता॒ या या म॒नोता᳚ । \newline
21. म॒नोतेति॑ म॒नोता᳚ । \newline
22. बृह॒स्पति॑र् मात॒रिश्वा॑ मात॒रिश्वा॒ बृह॒स्पति॒र् बृह॒स्पति॑र् मात॒रिश्वो॒तोत मा॑त॒रिश्वा॒ बृह॒स्पति॒र् बृह॒स्पति॑र् मात॒रिश्वो॒त । \newline
23. मा॒त॒रिश्वो॒तोत मा॑त॒रिश्वा॑ मात॒रिश्वो॒त वा॒युर् वा॒यु रु॒त मा॑त॒रिश्वा॑ मात॒रिश्वो॒त वा॒युः । \newline
24. उ॒त वा॒युर् वा॒यु रु॒तोत वा॒युः स॑न्धुवा॒नाः स॑न्धुवा॒ना वा॒यु रु॒तोत वा॒युः स॑न्धुवा॒नाः । \newline
25. वा॒युः स॑न्धुवा॒नाः स॑न्धुवा॒ना वा॒युर् वा॒युः स॑न्धुवा॒ना वाता॒ वाताः᳚ सन्धुवा॒ना वा॒युर् वा॒युः स॑न्धुवा॒ना वाताः᳚ । \newline
26. स॒न्धु॒वा॒ना वाता॒ वाताः᳚ सन्धुवा॒नाः स॑न्धुवा॒ना वाता॑ अ॒भ्य॑भि वाताः᳚ सन्धुवा॒नाः स॑न्धुवा॒ना वाता॑ अ॒भि । \newline
27. स॒न्धु॒वा॒ना इति॑ सम् - धु॒वा॒नाः । \newline
28. वाता॑ अ॒भ्य॑भि वाता॒ वाता॑ अ॒भि नो॑ नो अ॒भि वाता॒ वाता॑ अ॒भि नः॑ । \newline
29. अ॒भि नो॑ नो अ॒भ्य॑भि नो॑ गृणन्तु गृणन्तु नो अ॒भ्य॑भि नो॑ गृणन्तु । \newline
30. नो॒ गृ॒ण॒न्तु॒ गृ॒ण॒न्तु॒ नो॒ नो॒ गृ॒ण॒न्तु॒ । \newline
31. गृ॒ण॒न्त्विति॑ गृणन्तु । \newline
32. वि॒ष्टं॒भो दि॒वो दि॒वो वि॑ष्टं॒भो वि॑ष्टं॒भो दि॒वो ध॒रुणो॑ ध॒रुणो॑ दि॒वो वि॑ष्टं॒भो वि॑ष्टं॒भो दि॒वो ध॒रुणः॑ । \newline
33. वि॒ष्टं॒भ इति॑ वि - स्तं॒भः । \newline
34. दि॒वो ध॒रुणो॑ ध॒रुणो॑ दि॒वो दि॒वो ध॒रुणः॑ पृथि॒व्याः पृ॑थि॒व्या ध॒रुणो॑ दि॒वो दि॒वो ध॒रुणः॑ पृथि॒व्याः । \newline
35. ध॒रुणः॑ पृथि॒व्याः पृ॑थि॒व्या ध॒रुणो॑ ध॒रुणः॑ पृथि॒व्या अ॒स्यास्य पृ॑थि॒व्या ध॒रुणो॑ ध॒रुणः॑ पृथि॒व्या अ॒स्य । \newline
36. पृ॒थि॒व्या अ॒स्यास्य पृ॑थि॒व्याः पृ॑थि॒व्या अ॒स्येशा॒ नेशा॑ना॒ ऽस्य पृ॑थि॒व्याः पृ॑थि॒व्या अ॒स्येशा॑ना । \newline
37. अ॒स्येशा॒ नेशा॑ना॒ ऽस्या स्येशा॑ना॒ जग॑तो॒ जग॑त॒ ईशा॑ना॒ ऽस्या स्येशा॑ना॒ जग॑तः । \newline
38. ईशा॑ना॒ जग॑तो॒ जग॑त॒ ईशा॒नेशा॑ना॒ जग॑तो॒ विष्णु॑पत्नी॒ विष्णु॑पत्नी॒ जग॑त॒ ईशा॒नेशा॑ना॒ जग॑तो॒ विष्णु॑पत्नी । \newline
39. जग॑तो॒ विष्णु॑पत्नी॒ विष्णु॑पत्नी॒ जग॑तो॒ जग॑तो॒ विष्णु॑पत्नी । \newline
40. विष्णु॑प॒त्नीति॒ विष्णु॑ - प॒त्नी॒ । \newline
41. वि॒श्वव्य॑चा इ॒षय॑न्ती॒ षय॑न्ती वि॒श्वव्य॑चा वि॒श्वव्य॑चा इ॒षय॑न्ती॒ सुभू॑तिः॒ सुभू॑ति रि॒षय॑न्ती वि॒श्वव्य॑चा वि॒श्वव्य॑चा इ॒षय॑न्ती॒ सुभू॑तिः । \newline
42. वि॒श्वव्य॑चा॒ इति॑ वि॒श्व - व्य॒चाः॒ । \newline
43. इ॒षय॑न्ती॒ सुभू॑तिः॒ सुभू॑ति रि॒षय॑न्ती॒ षय॑न्ती॒ सुभू॑तिः शि॒वा शि॒वा सुभू॑ति रि॒षय॑न्ती॒ षय॑न्ती॒ सुभू॑तिः शि॒वा । \newline
44. सुभू॑तिः शि॒वा शि॒वा सुभू॑तिः॒ सुभू॑तिः शि॒वा नो॑ नः शि॒वा सुभू॑तिः॒ सुभू॑तिः शि॒वा नः॑ । \newline
45. सुभू॑ति॒रिति॒ सु - भू॒तिः॒ । \newline
46. शि॒वा नो॑ नः शि॒वा शि॒वा नो॑ अस्त्वस्तु नः शि॒वा शि॒वा नो॑ अस्तु । \newline
47. नो॒ अ॒स्त्व॒स्तु॒ नो॒ नो॒ अ॒स्त्वदि॑ति॒ रदि॑ति रस्तु नो नो अ॒स्त्वदि॑तिः । \newline
48. अ॒स्त्वदि॑ति॒ रदि॑ति रस्त्व॒ स्त्वदि॑ति रु॒पस्थ॑ उ॒पस्थे॒ अदि॑ति रस्त्व॒ स्त्वदि॑ति रु॒पस्थे᳚ । \newline
49. अदि॑ति रु॒पस्थ॑ उ॒पस्थे॒ अदि॑ति॒ रदि॑ति रु॒पस्थे᳚ । \newline
50. उ॒पस्थ॒ इत्यु॒प - स्थे॒ । \newline
51. वै॒श्वा॒न॒रो नो॑ नो वैश्वान॒रो वै᳚श्वान॒रो न॑ ऊ॒त्योत्या नो॑ वैश्वान॒रो वै᳚श्वान॒रो न॑ ऊ॒त्या । \newline
52. न॒ ऊ॒त्योत्या नो॑ न ऊ॒त्या पृ॒ष्टः पृ॒ष्ट ऊ॒त्या नो॑ न ऊ॒त्या पृ॒ष्टः । \newline
53. ऊ॒त्या पृ॒ष्टः पृ॒ष्ट ऊ॒त्योत्या पृ॒ष्टो दि॒वि दि॒वि पृ॒ष्ट ऊ॒त्योत्या पृ॒ष्टो दि॒वि । \newline
54. पृ॒ष्टो दि॒वि दि॒वि पृ॒ष्टः पृ॒ष्टो दि॒व्यन् वनु॑ दि॒वि पृ॒ष्टः पृ॒ष्टो दि॒व्यनु॑ । \newline
55. दि॒व्य न्वनु॑ दि॒वि दि॒व्यनु॑ नो नो ऽनु दि॒वि दि॒व्यनु॑ नः । \newline
56. अनु॑ नो नो॒ अन्वनु॑ नो॒ ऽद्याद्य नो॒ अन्वनु॑ नो॒ ऽद्य । \newline
57. नो॒ ऽद्याद्य नो॑ नो॒ ऽद्या नु॑मति॒ रनु॑मति र॒द्य नो॑ नो॒ ऽद्या नु॑मतिः । \newline
58. अ॒द्या नु॑मति॒ रनु॑मति र॒द्याद्या नु॑मति॒ रन्वन् वनु॑ मति र॒द्याद्या नु॑मति॒ रनु॑ । \newline
59. अनु॑मति॒ रन्वन् वनु॑ मति॒ रनु॑मति॒ रन्वि दिदन् वनु॑मति॒ रनु॑मति॒ रन्वित् । \newline
60. अनु॑मति॒रित्यनु॑ - म॒तिः॒ । \newline
61. अन्विदि दन् वन् वि द॑नुमते ऽनुमत॒ इदन् वन् वि द॑नुमते । \newline
62. इद॑नुमते ऽनुमत॒ इदिद॑ नुमते॒ त्वम् त्व म॑नुमत॒ इदि द॑नुमते॒ त्वम् । \newline
63. अ॒नु॒म॒ते॒ त्वम् त्व म॑नुमते ऽनुमते॒ त्वम् कया॒ कया॒ त्व म॑नुमते ऽनुमते॒ त्वम् कया᳚ । \newline
64. अ॒नु॒म॒त॒ इत्य॑नु - म॒ते॒ । \newline
65. त्वम् कया॒ कया॒ त्वम् त्वम् कया॑ नो नः॒ कया॒ त्वम् त्वम् कया॑ नः । \newline
66. कया॑ नो नः॒ कया॒ कया॑ नश्चि॒त्र श्चि॒त्रो नः॒ कया॒ कया॑ न श्चि॒त्रः । \newline
67. न॒ श्चि॒त्र श्चि॒त्रो नो॑ न श्चि॒त्र आ चि॒त्रो नो॑ न श्चि॒त्र आ । \newline
68. चि॒त्र आ चि॒त्र श्चि॒त्र आ भु॑वद् भुव॒दा चि॒त्र श्चि॒त्र आ भु॑वत् । \newline
69. आ भु॑वद् भुव॒दा भु॑व॒त् कः को भु॑व॒दा भु॑व॒त् कः । \newline
70. भु॒व॒त् कः को भु॑वद् भुव॒त् को अ॒द्याद्य को भु॑वद् भुव॒त् को अ॒द्य । \newline
71. को अ॒द्याद्य कः को अ॒द्य यु॑ङ्क्ते युङ्क्ते अ॒द्य कः को अ॒द्य यु॑ङ्क्ते । \newline
72. अ॒द्य यु॑ङ्क्ते युङ्क्ते अ॒द्याद्य यु॑ङ्क्ते । \newline
73. यु॒ङ्क्त॒ इति॑ युङ्क्ते । \newline
\pagebreak


\end{document}