\documentclass[17pt]{extarticle}
\usepackage{babel}
\usepackage{fontspec}
\usepackage{polyglossia}
\usepackage{extsizes}

\usepackage{color}   %May be necessary if you want to color links
\usepackage{hyperref}
\hypersetup{
    colorlinks=true, %set true if you want colored links
    linktoc=all,     %set to all if you want both sections and subsections linked
    linkcolor=black,  %choose some color if you want links to stand out
}

\setmainlanguage{sanskrit}
\setotherlanguages{english} %% or other languages
\setlength{\parindent}{0pt}
\pagestyle{myheadings}
\newfontfamily\devanagarifont[Script=Devanagari]{AdishilaVedic}
\renewcommand{\theHsection}{\thepart.section.\thesection}

\newcommand{\VAR}[1]{}
\newcommand{\BLOCK}[1]{}




\begin{document}
\begin{titlepage}
    \begin{center}
 
\begin{sanskrit}
    { \Large
    कृष्ण यजुर्वेदीय तैत्तिरीय संहिता,पद,जटा,घन पाठः 
    }
    \\
    \vspace{2.5cm}
    \mbox{ \Large
    4.4      चतुर्थकाण्डे चतुर्थः प्रश्नः - पञ्चमचितिशेषनिरूपणं   }
\end{sanskrit}
\end{center}

\end{titlepage}
\tableofcontents
\phantomsection
\pagebreak

\markright{ TS 4.4.1.1  \hfill https://www.vedavms.in \hfill}

\section{ TS 4.4.1.1 }

\textbf{TS 4.4.1.1 } \newline
\textbf{Samhita Paata} \newline

र॒श्मिर॑सि॒ क्षया॑य त्वा॒ क्षयं॑ जिन्व॒ प्रेति॑रसि॒ धर्मा॑य त्वा॒ धर्मं॑ जि॒न्वान्वि॑तिरसि दि॒वे त्वा॒ दिवं॑ जिन्व स॒न्धिर॑स्य॒न्तरि॑क्षाय त्वा॒ऽन्तरि॑क्षं जिन्व प्रति॒धिर॑सि पृथि॒व्यै त्वा॑ पृथि॒वीं जि॑न्व विष्ट॒भों॑ऽसि॒ वृष्ट्यै᳚ त्वा॒ वृष्टिं॑ जिन्व प्र॒वाऽस्यह्ने॒ त्वाऽह॑र्जिन्वानु॒ वाऽसि॒ रात्रि॑यै त्वा॒ रात्रिं॑ जिन्वो॒ शिग॑सि॒ - [  ] \newline

\textbf{Pada Paata} \newline

र॒श्मिः । अ॒सि॒ । क्षया॑य । त्वा॒ । क्षय᳚म् । जि॒न्व॒ । प्रेति॒रिति॒ प्र - इ॒तिः॒ । अ॒सि॒ । धर्मा॑य । त्वा॒ । धर्म᳚म् । जि॒न्व॒ । अन्वि॑ति॒रित्यनु॑ - इ॒तिः॒ । अ॒सि॒ । दि॒वे । त्वा॒ । दिव᳚म् । जि॒न्व॒ । स॒न्धिरिति॑ सं - धिः । अ॒सि॒ । अ॒न्तरि॑क्षाय । त्वा॒ । अ॒न्तरि॑क्षम् । जि॒न्व॒ । प्र॒ति॒धिरिति॑ प्रति-धिः । अ॒सि॒ । पृ॒थि॒व्यै । त्वा॒ । पृ॒थि॒वीम् । जि॒न्व॒ । वि॒ष्ट॒भं इति॑ वि-स्त॒भंः । अ॒सि॒ । वृष्ट्यै᳚ । त्वा॒ । वृष्टि᳚म् । जि॒न्व॒ । प्र॒वेति॑ प्र - वा । अ॒सि॒ । अह्ने᳚ । त्वा॒ । अहः॑ । जि॒न्व॒ । अ॒नु॒वेत्य॑नु - वा । अ॒सि॒ । रात्रि॑यै । त्वा॒ । रात्रि᳚म् । जि॒न्व॒ । उ॒शिक् । अ॒सि॒ ।  \newline




\markright{ TS 4.4.1.2  \hfill https://www.vedavms.in \hfill}

\section{ TS 4.4.1.2 }

\textbf{TS 4.4.1.2 } \newline
\textbf{Samhita Paata} \newline

वसु॑भ्यस्त्वा॒ वसू᳚ञ्जिन्व प्रके॒तो॑ऽसि रु॒द्रेभ्य॑स्त्वा रु॒द्राञ्जि॑न्व सुदी॒तिर॑स्यादि॒त्येभ्य॑स्त्वा ऽऽदि॒त्याञ्जि॒न्वौजो॑ऽसि पि॒तृभ्य॑स्त्वा पि॒तॄञ्जि॑न्व॒ तन्तु॑रसि प्र॒जाभ्य॑स्त्वा प्र॒जा जि॑न्व पृतना॒षाड॑सि प॒शुभ्य॑स्त्वा प॒शूञ्जि॑न्व रे॒वद॒स्योष॑धीभ्य॒-स्त्वौष॑धी-र्जिन्वाभि॒जिद॑सि यु॒क्तग्रा॒वेन्द्रा॑य॒ त्वेन्द्रं॑ जि॒न्वाधि॑पतिरसि प्रा॒णाय॑ - [  ] \newline

\textbf{Pada Paata} \newline

वसु॑भ्य॒ इति॒ वसु॑ - भ्यः॒ । त्वा॒ । वसून्॑ । जि॒न्व॒ । प्र॒के॒त इति॑ प्र - के॒तः । अ॒सि॒ । रु॒द्रेभ्यः॑ । त्वा॒ । रु॒द्रान् । जि॒न्व॒ । सु॒दी॒तिरिति॑ सु- दी॒तिः । अ॒सि॒ । आ॒दि॒त्येभ्यः॑ । त्वा॒ । आ॒दि॒त्यान् । जि॒न्व॒ । ओजः॑ । अ॒सि॒ । पि॒तृभ्य॒ इति॑ पि॒तृ - भ्यः॒ । त्वा॒ । पि॒तॄन् । जि॒न्व॒ । तन्तुः॑ । अ॒सि॒ । प्र॒जाभ्य॒ इति॑ प्र - जाभ्यः॑ । त्वा॒ । प्र॒जा इति॑ प्र - जाः । जि॒न्व॒ । पृ॒त॒ना॒षाट् । अ॒सि॒ । प॒शुभ्य॒ इति॑ प॒शु - भ्यः॒ । त्वा॒ । प॒शून् । जि॒न्व॒ । रे॒वत् । अ॒सि॒ । ओष॑धीभ्य॒ इत्योष॑धि - भ्यः॒ । त्वा॒ । ओष॑धीः । जि॒न्व॒ । अ॒भि॒जिदित्य॑भि - जित् । अ॒सि॒ । यु॒क्तग्रा॒वेति॑ यु॒क्त - ग्रा॒वा॒ । इन्द्रा॑य । त्वा॒ । इन्द्र᳚म् । जि॒न्व॒ । अधि॑पति॒रित्यधि॑-प॒तिः॒ । अ॒सि॒ । प्रा॒णायेति॑ प्र-अ॒नाय॑ ।  \newline




\markright{ TS 4.4.1.3  \hfill https://www.vedavms.in \hfill}

\section{ TS 4.4.1.3 }

\textbf{TS 4.4.1.3 } \newline
\textbf{Samhita Paata} \newline

त्वा प्रा॒णं जि॑न्व य॒न्ताऽस्य॑पा॒नाय॑ त्वाऽपा॒नं जि॑न्व सꣳ॒॒सर्पो॑ऽसि॒ चक्षु॑षे त्वा॒ चक्षु॑र्जिन्व वयो॒धा अ॑सि॒ श्रोत्रा॑य त्वा॒ श्रोत्रं॑ जिन्व त्रि॒वृद॑सि प्र॒वृद॑सि सं॒ॅवृद॑सि वि॒वृद॑सि सꣳरो॒हो॑ऽसि नीरो॒हो॑ऽसि प्ररो॒हो᳚ऽस्यनुरो॒हो॑ऽसि वसु॒को॑ऽसि॒ वेष॑श्रिरसि॒ वस्य॑ष्टिरसि ॥ \newline

\textbf{Pada Paata} \newline

त्वा॒ । प्रा॒णमिति॑ प्र - अ॒नम् । जि॒न्व॒ । य॒न्ता । अ॒सि॒ । अ॒पा॒नायेत्य॑प - अ॒नाय॑ । त्वा॒ । अ॒पा॒नमित्य॑प - अ॒नम् । जि॒न्व॒ । सꣳ॒॒सर्प॒ इति॑ सं - सर्पः॑ । अ॒सि॒ । चक्षु॑षे । त्वा॒ । चक्षुः॑ । जि॒न्व॒ । व॒यो॒धा इति॑ वयो - धाः । अ॒सि॒ । श्रोत्रा॑य । त्वा॒ । श्रोत्र᳚म् । जि॒न्व॒ । त्रि॒वृदिति॑ त्रि - वृत् । अ॒सि॒ । प्र॒वृदिति॑ प्र - वृत् । अ॒सि॒ । सं॒ॅवृदिति॑ सं - वृत् । अ॒सि॒ । वि॒वृदिति वि - वृत् । अ॒सि॒ । सꣳ॒॒रो॒ह इति॑ सं - रो॒हः । अ॒सि॒ । नी॒रो॒ह इति॑ निः - रो॒हः । अ॒सि॒ । प्र॒रो॒ह इति॑ प्र - रो॒हः । अ॒सि॒ । अ॒नु॒रो॒ह इत्य॑नु - रो॒हः । अ॒सि॒ । व॒सु॒कः । अ॒सि॒ । वेष॑श्रि॒रिति॒ वेष॑ - श्रिः॒ । अ॒सि॒ । वस्य॑ष्टिः । अ॒सि॒ ॥  \newline




\markright{ TS 4.4.2.1  \hfill https://www.vedavms.in \hfill}

\section{ TS 4.4.2.1 }

\textbf{TS 4.4.2.1 } \newline
\textbf{Samhita Paata} \newline

राज्ञ्य॑सि॒ प्राची॒ दिग्वस॑वस्ते दे॒वा अधि॑पतयो॒ऽग्निर्.हे॑ती॒नां प्र॑तिध॒र्ता॑ त्रि॒वृत् त्वा॒ स्तोमः॑ पृथि॒व्याꣳ श्र॑य॒त्वाज्य॑-मु॒क्थमव्य॑थयथ् स्तभ्नातु रथन्त॒रꣳ साम॒ प्रति॑ष्ठित्यै वि॒राड॑सि दक्षि॒णा दिग्रु॒द्रास्ते॑ दे॒वा अधि॑पतय॒ इन्द्रो॑ हेती॒नां प्र॑तिध॒र्ता प॑ञ्चद॒शस्त्वा॒ स्तोमः॑ पृथि॒व्याꣳ श्र॑यतु॒ प्र-उ॑गमु॒क्थमव्य॑थयथ् स्तभ्नातु बृ॒हथ् साम॒ प्रति॑ष्ठित्यै स॒म्राड॑सि प्र॒तीची॒ दि- [  ] \newline

\textbf{Pada Paata} \newline

राज्ञी᳚ । अ॒सि॒ । प्राची᳚ । दिक् । वस॑वः । ते॒ । दे॒वाः । अधि॑पतय॒ इत्यधि॑ - प॒त॒यः॒ । अ॒ग्निः । हे॒ती॒नाम् । प्र॒ति॒ध॒र्तेति॑ प्रति - ध॒र्ता । त्रि॒वृदिति॑ त्रि-वृत् । त्वा॒ । स्तोमः॑ । पृ॒थि॒व्याम् । श्र॒य॒तु॒ । आज्य᳚म् । उ॒क्थम् । अव्य॑थयत् । स्त॒भ्ना॒तु॒ । र॒थ॒न्त॒रमिति॑ रथं - त॒रम् । साम॑ । प्रति॑ष्ठित्या॒ इति॒ प्रति॑ - स्थि॒त्यै॒ । वि॒राडिति॑ वि - राट् । अ॒सि॒ । द॒क्षि॒णा । दिक् । रु॒द्राः । ते॒ । दे॒वाः । अधि॑पतय॒ इत्यधि॑ - प॒त॒यः॒ । इन्द्रः॑ । हे॒ती॒नाम् । प्र॒ति॒ध॒र्तेति॑ प्रति - ध॒र्ता । प॒ञ्च॒द॒श इति॑ पञ्च - द॒शः । त्वा॒ । स्तोमः॑ । पृ॒थि॒व्याम् । श्र॒य॒तु॒ । प्र उ॑गम् । उ॒क्थम् । अव्य॑थयत् । स्त॒भ्ना॒तु॒ । बृ॒हत् । साम॑ । प्रति॑ष्ठित्या॒ इति॒ प्रति॑ - स्थि॒त्यै॒ । स॒म्राडिति॑ सं - राट् । अ॒सि॒ । प्र॒तीची᳚ । दिक् ।  \newline




\markright{ TS 4.4.2.2  \hfill https://www.vedavms.in \hfill}

\section{ TS 4.4.2.2 }

\textbf{TS 4.4.2.2 } \newline
\textbf{Samhita Paata} \newline

गा॑दि॒त्यास्ते॑ दे॒वा अधि॑पतयः॒ सोमो॑ हेती॒नां प्र॑तिध॒र्ता स॑प्तद॒शस्त्वा॒ स्तोमः॑ पृथि॒व्याꣳ श्र॑यतु मरुत्व॒तीय॑मु॒क्थ-मव्य॑थयथ् स्तभ्नातु वैरू॒पꣳ साम॒ प्रति॑ष्ठित्यायै स्व॒राड॒स्युदी॑ची॒ दिग् विश्वे॑ ते दे॒वा अधि॑पतयो॒ वरु॑णो हेती॒नाम् प्र॑तिध॒र्तैक॑विꣳ॒॒श स्त्वा॒ स्तोमः॑ पृथि॒व्याꣳ श्र॑यतु॒ निष्के॑वल्य-मु॒क्थमव्य॑थयथ् स्तभ्नातु वैरा॒जꣳ साम॒ प्रति॑ष्ठित्या॒ अधि॑पत्न्यसि बृह॒ती दिङ्म॒रुत॑स्ते दे॒वा अधि॑पतयो॒ - [  ] \newline

\textbf{Pada Paata} \newline

आ॒दि॒त्याः । ते॒ । दे॒वाः । अधि॑पतय॒ इत्यधि॑ - प॒त॒यः॒ । सोमः॑ । हे॒ती॒नाम् । प्र॒ति॒ध॒र्तेति॑ प्रति - ध॒र्ता । स॒प्त॒द॒श इति॑ सप्त - द॒शः । त्वा॒ । स्तोमः॑ । पृ॒थि॒व्याम् । श्र॒य॒तु॒ । म॒रु॒त्व॒तीय᳚म् । उ॒क्थम् । अव्य॑थयत् । स्त॒भ्ना॒तु॒ । वै॒रू॒पम् । साम॑ । प्रति॑ष्ठित्या॒ इति॒ प्रति॑ - स्थि॒त्यै॒ । स्व॒राडिति॑ स्व - राट् । अ॒सि॒ । उदी॑ची । दिक् । विश्वे᳚ । ते॒ । दे॒वाः । अधि॑पतय॒ इत्यधि॑ - प॒त॒यः॒ । वरु॑णः । हे॒ती॒नाम् । प्र॒ति॒ध॒र्तेति॑ प्रति - ध॒र्ता । ए॒क॒विꣳ॒॒श इत्ये॑क-विꣳ॒॒शः । त्वा॒ । स्तोमः॑ । पृ॒थि॒व्याम् । श्र॒य॒तु॒ । निष्के॑वल्यम् । उ॒क्थम् । अव्य॑थयत् । स्त॒भ्ना॒तु॒ । वै॒रा॒जम् । साम॑ । प्रति॑ष्ठित्या॒ इति॒ प्रति॑ - स्थि॒त्यै॒ । अधि॑प॒त्नीत्यधि॑ - प॒त्नी॒ । अ॒सि॒ । बृ॒ह॒ती । दिक् । म॒रुतः॑ । ते॒ । दे॒वाः । अधि॑पतय॒ इत्यधि॑ - प॒त॒यः॒ ।  \newline




\markright{ TS 4.4.2.3  \hfill https://www.vedavms.in \hfill}

\section{ TS 4.4.2.3 }

\textbf{TS 4.4.2.3 } \newline
\textbf{Samhita Paata} \newline

बृह॒स्पति॑र्.हेती॒नां प्र॑तिध॒र्ता त्रि॑णवत्रयस्त्रिꣳ॒॒शौ त्वा॒ स्तोमौ॑ पृथि॒व्याꣳ श्र॑यतां ॅवैश्वदेवाग्निमारु॒ते उ॒क्थे अव्य॑थयन्ती स्तभ्नीताꣳ शाक्वररैव॒ते साम॑नी॒ प्रति॑ष्ठित्या अ॒न्तरि॑क्षा॒यर्.ष॑यस्त्वा प्रथम॒जा दे॒वेषु॑ दि॒वो मात्र॑या वरि॒णा प्र॑थन्तु विध॒र्ता चा॒यमधि॑पतिश्च॒ ते त्वा॒ सर्वे॑ संॅविदा॒ना नाक॑स्य पृ॒ष्ठे सु॑व॒र्गे लो॒के यज॑मानं च सादयन्तु ॥ \newline

\textbf{Pada Paata} \newline

बृह॒स्पतिः॑ । हे॒ती॒नाम् । प्र॒ति॒ध॒र्तेति॑ प्रति - ध॒र्ता । त्रि॒ण॒व॒त्र॒य॒स्त्रिꣳ॒॒शाविति॑ त्रिणव - त्र॒य॒स्त्रिꣳ॒॒शौ । त्वा॒ । स्तोमौ᳚ । पृ॒थि॒व्याम् । श्र॒य॒ता॒म् । वै॒श्व॒दे॒वा॒ग्नि॒मा॒रु॒ते इति॑ वैश्वदेव-अ॒ग्नि॒मा॒रु॒ते । उ॒क्थे इति॑ । अव्य॑थयन्ती॒ इति॑ । स्त॒भ्नी॒ता॒म् । शा॒क्व॒र॒रै॒व॒ते इति॑ शाक्वर - रै॒व॒ते । साम॑नी॒ इति॑ । प्रति॑ष्ठित्या॒ इति॒ प्रति॑ - स्थि॒त्यै॒ । अ॒न्तरि॑क्षाय । ऋष॑यः । त्वा॒ । प्र॒थ॒म॒जा इति॑ प्रथम - जाः । दे॒वेषु॑ । दि॒वः । मात्र॑या । व॒रि॒णा । प्र॒थ॒न्तु॒ । वि॒ध॒र्तेति॑ वि - ध॒र्ता । च॒ । अ॒यम् । अधि॑पति॒रित्यधि॑-प॒तिः॒ । च॒ । ते । त्वा॒ । सर्वे᳚ । सं॒ॅवि॒दा॒ना इति॑ सं - वि॒दा॒नाः । नाक॑स्य । पृ॒ष्ठे । सु॒व॒र्ग इति॑ सुवः - गे । लो॒के । यज॑मानम् । च॒ । सा॒द॒य॒न्तु॒ ॥  \newline




\markright{ TS 4.4.3.1  \hfill https://www.vedavms.in \hfill}

\section{ TS 4.4.3.1 }

\textbf{TS 4.4.3.1 } \newline
\textbf{Samhita Paata} \newline

अ॒यं पु॒रो हरि॑केशः॒ सूर्य॑रश्मि॒स्तस्य॑ रथगृ॒थ्सश्च॒ रथौ॑जाश्च सेनानि ग्राम॒ण्यौ॑ पुञ्जिकस्थ॒ला च॑ कृतस्थ॒ला चा᳚फ्स॒रसौ॑ यातु॒धाना॑ हे॒ती रक्षाꣳ॑सि॒ प्रहे॑ति र॒यं द॑क्षि॒णा वि॒श्व क॑र्मा॒ तस्य॑ रथस्व॒नश्च॒ रथे॑चित्रश्च सेनानि ग्राम॒ण्यौ॑ मेन॒का च॑ सहज॒न्या चा᳚फ्स॒रसौ॑ द॒ङ्णवः॑ प॒शवो॑ हे॒तिः पौरु॑षेयो व॒धः प्रहे॑ति र॒यं प॒श्चाद् वि॒श्वव्य॑चा॒ स्तस्य॒ रथ॑ प्रोत॒श्चा-स॑मरथश्च सेनानि ग्राम॒ण्यौ᳚ प्र॒म्लोच॑न्ती चा -[  ] \newline

\textbf{Pada Paata} \newline

अ॒यम् । पु॒रः । हरि॑केश॒ इति॒ हरि॑ - के॒शः॒ । सूर्य॑रश्मि॒रिति॒ सूर्य॑ - र॒श्मिः॒ । तस्य॑ । र॒थ॒गृ॒थ्स इति॑ रथ - गृ॒थ्सः । च॒ । रथौ॑जा॒ इति॒ रथ॑ - ओ॒जाः॒ । च॒ । से॒ना॒नि॒ग्रा॒म॒ण्या॑विति॑ सेनानि - ग्रा॒म॒ण्यौ᳚ । पु॒ञ्जि॒क॒स्थ॒लेति॑ पुञ्जिक - स्थ॒ला । च॒ । कृ॒त॒स्थ॒लेति॑ कृत - स्थ॒ला । च॒ । अ॒फ्स॒रसौ᳚ । या॒तु॒धाना॒ इति॑ यातु - धानाः᳚ । हे॒तिः । रक्षाꣳ॑सि । प्रहे॑ति॒रिति॒ प्र-हे॒तिः॒ । अ॒यम् । द॒क्षि॒णा । वि॒श्वक॒र्मेति॑ वि॒श्व - क॒र्मा॒ । तस्य॑ । र॒थ॒स्व॒न इति॑ रथ - स्व॒नः । च॒ । रथे॑चित्र॒ इति॒ रथे᳚ - चि॒त्रः॒ । च॒ । से॒ना॒नि॒ग्रा॒म॒ण्या॑विति॑ सेनानि - ग्रा॒म॒ण्यौ᳚ । मे॒न॒का । च॒ । स॒ह॒ज॒न्येति॑ सह - ज॒न्या । च॒ । अ॒फ्स॒रसौ᳚ । द॒ङ्णवः॑ । प॒शवः॑ । हे॒तिः । पौरु॑षेयः । व॒धः । प्रहे॑ति॒रिति॒ प्र - हे॒तिः॒ । अ॒यम् । प॒श्चात् । वि॒श्वव्य॑चा॒ इति॑ वि॒श्व - व्य॒चाः॒ । तस्य॑ । रथ॑प्रोत॒ इति॒ रथ॑-प्रो॒तः॒ । च॒ । अस॑मरथ॒ इत्यस॑म - र॒थः॒ । च॒ । से॒ना॒नि॒ग्रा॒म॒ण्या॑विति॑ सेनानि - ग्रा॒म॒ण्यौ᳚ । प्र॒म्लोच॒न्तीति॑ प्र - म्लोच॑न्ती । च॒ ।  \newline




\markright{ TS 4.4.3.2  \hfill https://www.vedavms.in \hfill}

\section{ TS 4.4.3.2 }

\textbf{TS 4.4.3.2 } \newline
\textbf{Samhita Paata} \newline

नु॒म्लोच॑न्ती-चाफ्स॒रसौ॑ स॒र्पा हे॒ति र्व्या॒घ्राः प्रहे॑ति र॒य मु॑त्त॒राथ् सं॒यॅद्- व॑सु॒स्तस्य॑ सेन॒जिच्च॑ सु॒षेण॑श्च सेनानि ग्राम॒ण्यौ॑ वि॒श्वाची॑ च घृ॒ताची॑ चाफ्स॒रसा॒ वापो॑ हे॒ति र्वातः॒ प्रहे॑ति र॒यमु॒पर्य॒ र्वाग्व॑-सु॒स्तस्य॒ तार्क्ष्य॒-श्चारि॑ष्ट-नेमिश्च सेनानि ग्राम॒ण्या॑ वु॒र्वशी॑ च पू॒र्वचि॑त्तिश्चा-फ्स॒रसौ॑ वि॒द्युद्धे॒तिर॑-व॒स्फूर्ज॒न् प्रहे॑ति॒ स्तेभ्यो॒ नम॒स्ते नो॑ मृडयन्तु॒ ते यं - [  ] \newline

\textbf{Pada Paata} \newline

अ॒नु॒म्लोच॒न्तीत्य॑नु - म्लोच॑न्ती । च॒ । अ॒फ्स॒रसौ᳚ । स॒र्पाः । हे॒तिः । व्या॒घ्राः । प्रहे॑ति॒रिति॒ प्र - हे॒तिः॒ । अ॒यम् । उ॒त्त॒रादित्यु॑त्-त॒रात् । सं॒ॅयद्व॑सु॒रिति॑ सं॒ॅयत् - व॒सुः॒ । तस्य॑ । से॒न॒जिदिति॑ सेन-जित् । च॒ । सु॒षेण॒ इति॑ सु - सेनः॑ । च॒ । से॒ना॒नि॒ग्रा॒म॒ण्या॑विति॑ सेनानि - ग्रा॒म॒ण्यौ᳚ । वि॒श्वाची᳚ । च॒ । घृ॒ताची᳚ । च॒ । अ॒फ्स॒रसौ᳚ । आपः॑ । हे॒ति । वातः॑ । प्रहे॑ति॒रिति॒ प्र - हे॒तिः॒ । अ॒यम् । उ॒परि॑ । अ॒र्वाग्व॑सु॒रित्य॒र्वाक् - व॒सुः॒ । तस्य॑ । तार्क्ष्यः॑ । च॒ । अरि॑ष्टनेमि॒रित्यरि॑ष्ट - ने॒मिः॒ । च॒ । से॒ना॒नि॒ग्रा॒म॒ण्या॑विति॑ सेनानि - ग्रा॒म॒ण्यौ᳚ । उ॒र्वशी᳚ । च॒ । पू॒र्वचि॑त्ति॒रिति॑ पू॒र्व - चि॒त्तिः॒ । च॒ । अ॒फ्स॒रसौ᳚ । वि॒द्युदिति॑ वित् - युत् । हे॒तिः । अ॒व॒स्फूर्ज॒न्नित्य॑व - स्फूर्जन्न्॑ । प्रहे॑ति॒रिति॒ प्र - हे॒तिः॒ । तेभ्यः॑ । नमः॑ । ते । नः॒ । मृ॒ड॒यन्तु॒ । ते । यम् ।  \newline




\markright{ TS 4.4.3.3  \hfill https://www.vedavms.in \hfill}

\section{ TS 4.4.3.3 }

\textbf{TS 4.4.3.3 } \newline
\textbf{Samhita Paata} \newline

द्वि॒ष्मो यश्च॑ नो॒ द्वेष्टि॒ तं ॅवो॒ जंभे॑ दधाम्या॒योस्त्वा॒ सद॑ने सादया॒म्यव॑त श्छा॒यायां॒ नमः॑ समु॒द्राय॒ नमः॑ समु॒द्रस्य॒ चक्ष॑से परमे॒ष्ठी त्वा॑ सादयतु दि॒वः पृ॒ष्ठे व्यच॑स्वतीं॒ प्रथ॑स्वतीं ॅवि॒भूम॑तीं प्र॒भूम॑तीं परि॒भूम॑तीं॒ दिवं॑ ॅयच्छ॒ दिवं॑ दृꣳह॒ दिवं॒ मा हिꣳ॑सी॒र्विश्व॑स्मै प्रा॒णाया॑पा॒नाय॑ व्या॒नायो॑दा॒नाय॑ प्रति॒ष्ठायै॑ च॒रित्रा॑य॒ सूर्य॑स्त्वा॒ऽभि पा॑तु म॒ह्या स्व॒स्त्या ( ) छ॒र्दिषा॒ शन्त॑मेन॒ तया॑ दे॒वत॑याऽङ्गिर॒स्वद्-ध्रु॒वा सी॑द ॥ प्रोथ॒दश्वो॒ न यव॑से अवि॒ष्यन्. य॒दा म॒हः स॒ॅवर॑णा॒द् व्यस्था᳚त् । आद॑स्य॒ वातो॒ अनु॑ वाति शो॒चिरध॑ स्म ते॒ व्रज॑नं कृ॒ष्णम॑स्ति ॥ \newline

\textbf{Pada Paata} \newline

द्वि॒ष्मः । यः । च॒ । नः॒ । द्वेष्टि॑ । तम् । वः॒ । जंभे᳚ । द॒धा॒मि॒ । आ॒योः । त्वा॒ । सद॑ने । सा॒द॒या॒मि॒ । अव॑तः । छा॒याया᳚म् । नमः॑ । स॒मु॒द्राय॑ । नमः॑ । स॒मु॒द्रस्य॑ । चक्ष॑से । प॒र॒मे॒ष्ठी । त्वा॒ । सा॒द॒य॒तु॒ । दि॒वः । पृ॒ष्ठे । व्यच॑स्वतीम् । प्रथ॑स्वतीम् । वि॒भूम॑ती॒मिति॑ वि - भूम॑तीम् । प्र॒भूम॑ती॒मिति॑ प्र - भूम॑तीम् । प॒रि॒भूम॑ती॒मिति॑ परि - भूम॑तीम् । दिव᳚म् । य॒च्छ॒ । दिव᳚म् । दृꣳ॒॒ह॒ । दिव᳚म् । मा । हिꣳ॒॒सीः॒ । विश्व॑स्मै । प्रा॒णायेति॑ प्र - अ॒नाय॑ । अ॒पा॒नायेत्य॑प - अ॒नाय॑ । व्या॒नायेति॑ वि - अ॒नाय॑ । उ॒दा॒नायेत्यु॑त् - अ॒नाय॑ । प्र॒ति॒ष्ठाया॒ इति॑ प्रति - स्थायै᳚ । च॒रित्रा॑य । सूर्यः॑ । त्वा॒ । अ॒भीति॑ । पा॒तु॒ । म॒ह्या । स्व॒स्त्या ( ) । छ॒र्दिषा᳚ । शन्त॑मे॒नेति॒ शं - त॒मे॒न॒ । तया᳚ । दे॒वत॑या । अ॒ङ्गि॒र॒स्वत् । ध्रु॒वा । सी॒द॒ ॥ प्रोथ॑त् । अश्वः॑ । न । यव॑से । अ॒वि॒ष्यन्न् । य॒दा । म॒हः । सं॒ॅवर॑णा॒दिति॑ सं - वर॑णात् । व्यस्था॒दिति॑ वि - अस्था᳚त् ॥ आत् । अ॒स्य॒ । वातः॑ । अन्विति॑ । वा॒ति॒ । शो॒चिः । अध॑ । स्म॒ । ते॒ । व्रज॑नम् । कृ॒ष्णम् । अ॒स्ति॒ ॥  \newline




\markright{ TS 4.4.4.1  \hfill https://www.vedavms.in \hfill}

\section{ TS 4.4.4.1 }

\textbf{TS 4.4.4.1 } \newline
\textbf{Samhita Paata} \newline

अ॒ग्निर्मू॒र्द्धा दि॒वः क॒कुत् पतिः॑ पृथि॒व्या अ॒यं । अ॒पाꣳ रेताꣳ॑सि जिन्वति ॥ त्वाम॑ग्ने॒ पुष्क॑रा॒दद्ध्यथ॑र्वा॒ निर॑मन्थत । मू॒र्द्ध्नो विश्व॑स्य वा॒घतः॑ ॥ अ॒यम॒ग्निः स॑ह॒स्रिणो॒ वाज॑स्य श॒तिन॒स्पतिः॑ । मू॒र्द्धा क॒वी र॑यी॒णां ॥ भुवो॑ य॒ज्ञ्स्य॒ रज॑सश्च ने॒ता यत्रा॑ नि॒युद्भिः॒ सच॑से शि॒वाभिः॑ । दि॒वि मू॒र्द्धानं॑ दधिषे सुव॒र्॒.षां जि॒ह्वाम॑ग्ने चकृषे हव्य॒वाहं᳚ ॥ अबो᳚द्ध्य॒ग्निः स॒मिधा॒ जना॑नां॒ - [  ] \newline

\textbf{Pada Paata} \newline

अ॒ग्निः । मू॒द्‌र्धा । दि॒वः । क॒कुत् । पतिः॑ । पृ॒थि॒व्याः । अ॒यम् ॥ अ॒पाम् । रेताꣳ॑सि । जि॒न्व॒ति॒ ॥ त्वाम् । अ॒ग्ने॒ । पुष्क॑रात् । अधीति॑ । अथ॑र्वा । निरिति॑ । अ॒म॒न्थ॒त॒ ॥ मू॒द्‌र्ध्नः । विश्व॑स्य । वा॒घतः॑ ॥ अ॒यम् । अ॒ग्निः । स॒ह॒स्रिणः॑ । वाज॑स्य । श॒तिनः॑ । पतिः॑ ॥ मू॒द्‌र्धा । क॒विः । र॒यी॒णाम् ॥ भुवः॑ । य॒ज्ञ्स्य॑ । रज॑सः । च॒ । ने॒ता । यत्र॑ । नि॒युद्भि॒रिति॑ नि॒युत् - भिः॒ । सच॑से । शि॒वाभिः॑ ॥ दि॒वि । मू॒द्‌र्धान᳚म् । द॒धि॒षे॒ । सु॒व॒र्॒.षामिति॑ सुवः - साम् । जि॒ह्वाम् । अ॒ग्ने॒ । च॒कृ॒षे॒ । ह॒व्य॒वाह॒मिति॑ हव्य - वाह᳚म् ॥ अबा॑धि । अ॒ग्निः । स॒मिधेति॑ सम् - इधा᳚ । जना॑नाम् ।  \newline




\markright{ TS 4.4.4.2  \hfill https://www.vedavms.in \hfill}

\section{ TS 4.4.4.2 }

\textbf{TS 4.4.4.2 } \newline
\textbf{Samhita Paata} \newline

प्रति॑ धे॒नुमि॑वा य॒तीमु॒षासं᳚ । य॒ह्वा इ॑व॒ प्रव॒या मु॒ज्जिहा॑नाः॒ प्र भा॒नवः॑ सिस्रते॒ नाक॒मच्छ॑ ॥ अवो॑चाम क॒वये॒ मेद्ध्या॑य॒ वचो॑ व॒न्दारु॑ वृष॒भाय॒ वृष्णे᳚ । गवि॑ष्ठिरो॒ नम॑सा॒ स्तोम॑म॒ग्नौ दि॒वीव॑ रु॒क्ममु॒र्व्यञ्च॑मश्रेत् ॥ जन॑स्य गो॒पा अ॑जनिष्ट॒ जागृ॑विर॒ग्निः सु॒दक्षः॑ सुवि॒ताय॒ नव्य॑से । घृ॒तप्र॑तीको बृह॒ता दि॑वि॒स्पृशा᳚ द्यु॒मद्वि भा॑ति भर॒तेभ्यः॒ शुचिः॑ ॥ त्वाम॑ग्ने॒ अङ्गि॑रसो॒ - [  ] \newline

\textbf{Pada Paata} \newline

प्रतीति॑ । धे॒नुम् । इ॒व । आ॒य॒तीमित्या᳚ - य॒तीम् । उ॒षास᳚म् ॥ य॒ह्वाः । इ॒व॒ । प्रेति॑ । व॒याम् । उ॒ज्जिहा॑ना॒ इत्यु॑त् - जिहा॑नाः । प्रेति॑ । भा॒नवः॑ । सि॒स्र॒ते॒ । नाक᳚म् । अच्छ॑ ॥ अवो॑चाम । क॒वये᳚ । मेद्ध्या॑य । वचः॑ । व॒न्दारु॑ । वृ॒ष॒भाय॑ । वृष्णे᳚ ॥ गवि॑ष्ठिरः । नम॑सा । स्तोम᳚म् । अ॒ग्नौ । दि॒वि । इ॒व॒ । रु॒क्मम् । उ॒र्व्यञ्च᳚म् । अ॒श्रे॒त् ॥ जन॑स्य । गो॒पा इति॑ गो - पाः । अ॒ज॒नि॒ष्ट॒। जागृ॑विः । अ॒ग्निः । सु॒दक्ष॒ इति॑ सु - दक्षः॑ । सु॒वि॒ताय॑ । नव्य॑से ॥ घृ॒तप्र॑तीक॒ इति॑ घृ॒त - प्र॒ती॒कः॒ । बृ॒ह॒ता । दि॒वि॒स्पृशेति॑ दिवि - स्पृशा᳚ । द्यु॒मदिति॑ द्यु - मत् । वीति॑ । भा॒ति॒ । भ॒र॒तेभ्यः॑ । शुचिः॑ ॥ त्वाम् । अ॒ग्ने॒ । अङ्गि॑रसः ।  \newline




\markright{ TS 4.4.4.3  \hfill https://www.vedavms.in \hfill}

\section{ TS 4.4.4.3 }

\textbf{TS 4.4.4.3 } \newline
\textbf{Samhita Paata} \newline

गुहा॑ हि॒तमन्व॑-विन्दञ्छिश्रिया॒णं ॅवने॑वने । स जा॑यसे म॒थ्यमा॑नः॒ सहो॑ म॒हत् त्वामा॑हुः॒ सह॑सस्पु॒त्रम॑ङ्गिरः ॥ य॒ज्ञ्स्य॑ के॒तुं प्र॑थ॒मं पु॒रोहि॑तम॒ग्निं नर॑स्त्रिषध॒स्थे समि॑न्धते । इन्द्रे॑ण दे॒वैः स॒रथꣳ॒॒ स ब॒र्॒.हिषि॒ सीद॒न्नि होता॑ य॒जथा॑य सु॒क्रतुः॑ ॥ त्वां चि॑त्रश्रवस्तम॒ हव॑न्ते वि॒क्षु ज॒न्तवः॑ । शो॒चिष्के॑शं पुरुप्रि॒याग्ने॑ ह॒व्याय॒ वोढ॑वे ॥ सखा॑यः॒ संॅवः॑ स॒म्यञ्च॒मिषꣳ॒॒ - [  ] \newline

\textbf{Pada Paata} \newline

गुहा᳚ । हि॒तम् । अन्विति॑ । अ॒वि॒न्द॒न्न् । शि॒श्रि॒या॒णम् । वने॑वन॒ इति॒ वने᳚ - व॒ने॒ ॥ सः । जा॒य॒से॒ । म॒थ्यमा॑नः । सहः॑ । म॒हत् । त्वाम् । आ॒हुः॒ । सह॑सः । पु॒त्रम् । अ॒ङ्गि॒रः॒ ॥ य॒ज्ञ्स्य॑ । के॒तुम् । प्र॒थ॒मम् । पु॒रोहि॑त॒मिति॑ पु॒रः - हि॒त॒म् । अ॒ग्निम् । नरः॑ । त्रि॒ष॒ध॒स्थ इति॑ त्रि - स॒ध॒स्थे । समिति॑ । इ॒न्ध॒ते॒ ॥ इन्द्रे॑ण । दे॒वैः । स॒रथ॒मिति॑ स - रथ᳚म् । सः । ब॒र्॒.हिषि॑ । सीद॑त् । नीति॑ । होता᳚ । य॒जथा॑य । सु॒क्रतु॒रिति॑ सु - क्रतुः॑ ॥ त्वाम् । चि॒त्र॒श्र॒व॒स्त॒मेति॑ चित्रश्रवः-त॒म॒ । हव॑न्ते । वि॒क्षु । ज॒न्तवः॑ ॥ शो॒चिष्के॑श॒मिति॑ शो॒चिः - के॒श॒म् । पु॒रु॒प्रि॒येति॑ पुरु - प्रि॒य॒ । अग्ने᳚ । ह॒व्याय॑ । वोढ॑वे ॥ सखा॑यः । समिति॑ । वः॒ । स॒म्यञ्च᳚म् । इष᳚म् ।  \newline




\markright{ TS 4.4.4.4  \hfill https://www.vedavms.in \hfill}

\section{ TS 4.4.4.4 }

\textbf{TS 4.4.4.4 } \newline
\textbf{Samhita Paata} \newline

स्तोमं॑ चा॒ग्नये᳚ । वर्.षि॑ष्ठाय क्षिती॒नामू॒र्जो नप्त्रे॒ सह॑स्वते ॥ सꣳस॒मिद्यु॑वसे वृष॒न्नग्ने॒ विश्वा᳚न्य॒र्य आ । इ॒डस्प॒दे समि॑द्ध्यसे॒ स नो॒ वसू॒न्या भ॑र ॥ ए॒ना वो॑ अ॒ग्निं नम॑सो॒र्जो नपा॑त॒मा हु॑वे । प्रि॒यं चेति॑ष्ठमर॒तिꣳ स्व॑द्ध्व॒रं ॅविश्व॑स्य दू॒तम॒मृतं᳚ ॥ स यो॑जते अरु॒षो वि॒श्वभो॑जसा॒ स दु॑द्रव॒थ् स्वा॑हुतः । सु॒ब्रह्मा॑ य॒ज्ञ्ः सु॒शमी॒ - [  ] \newline

\textbf{Pada Paata} \newline

स्तोम᳚म् । च॒ । अ॒ग्नये᳚ ॥ वर्.षि॑ष्ठाय । क्षि॒ती॒नाम् । ऊ॒र्जः । नप्त्रे᳚ । सह॑स्वते ॥ सꣳस॒मिति॒ सम् - स॒म् । इत् । यु॒व॒से॒ । वृ॒ष॒न्न् । अग्ने᳚ । विश्वा॑नि । अ॒र्यः । आ ॥ इ॒डः । प॒दे । समिति॑ । इ॒द्ध्य॒से॒ । सः । नः॒ । वसू॑नि । एति॑ । भ॒र॒ ॥ ए॒ना । वः॒ । अ॒ग्निम् । नम॑सा । ऊ॒र्जः । नपा॑तम् । एति॑ । हु॒वे॒ ॥ प्रि॒यम् । चेति॑ष्ठम् । अ॒र॒तिम् । स्व॒द्ध्व॒रमिति॑ सु - अ॒द्ध्व॒रम् । विश्व॑स्य । दू॒तम् । अ॒मृत᳚म् ॥ सः । यो॒ज॒ते॒ । अ॒रु॒षः । वि॒श्वभो॑ज॒सेति॑ वि॒श्व - भो॒ज॒सा॒ । सः । दु॒द्र॒व॒त् । स्वा॑हुत॒ इति॒ सु - आ॒हु॒तः॒ ॥ सु॒ब्रह्मेति॑ सु - ब्रह्मा᳚ । य॒ज्ञ्ः । सु॒शमीति॑ सु - शमी᳚ ।  \newline




\markright{ TS 4.4.4.5  \hfill https://www.vedavms.in \hfill}

\section{ TS 4.4.4.5 }

\textbf{TS 4.4.4.5 } \newline
\textbf{Samhita Paata} \newline

वसू॑नां दे॒वꣳ राधो॒ जना॑नां ॥ उद॑स्य शो॒चिर॑स्थादा॒-जुह्वा॑नस्य मी॒ढुषः॑ । उद्ध॒मासो॑ अरु॒षासो॑ दिवि॒स्पृशः॒ सम॒ग्निमि॑न्धते॒ नरः॑ ॥ अग्ने॒ वाज॑स्य॒ गोम॑त॒ ईशा॑नः सहसो यहो । अ॒स्मे धे॑हि जातवेदो॒ महि॒ श्रवः॑ ॥ स इ॑धा॒नो वसु॑ष्क॒वि-र॒ग्निरी॒डेन्यो॑ गि॒रा । रे॒वद॒स्मभ्यं॑ पुर्वणीक दीदिहि ॥ क्ष॒पो रा॑जन्नु॒त त्मनाऽग्ने॒ वस्तो॑रु॒तोषसः॑ । स ति॑ग्मजंभ - [  ] \newline

\textbf{Pada Paata} \newline

वसू॑नाम् । दे॒वम् । राधः॑ । जना॑नाम् ॥ उदिति॑ । अ॒स्य॒ । शो॒चिः । अ॒स्था॒त् । आ॒जुह्वा॑न॒स्येत्या᳚ - जुह्वा॑नस्य । मी॒ढुषः॑ ॥ उदिति॑ । धू॒मासः॑ । अ॒रु॒षासः॑ । दि॒वि॒स्पृश॒ इति॑ दिवि - स्पृशः॑ । समिति॑ । अ॒ग्निम् । इ॒न्ध॒ते॒ । नरः॑ ॥ अग्ने᳚ । वाज॑स्य । गोम॑त॒ इति॒ गो-म॒तः॒ । ईशा॑नः । स॒ह॒सः॒ । य॒हो॒ इति॑ ॥ अ॒स्मे इति॑ । धे॒हि॒ । जा॒त॒वे॒द॒ इति॑ जात - वे॒दः॒ । महि॑ । श्रवः॑ ॥ सः । इ॒धा॒नः । वसुः॑ । क॒विः । अ॒ग्निः । ई॒डेन्यः॑ । गि॒रा ॥ रे॒वत् । अ॒स्मभ्य॒मित्य॒स्म - भ्य॒म् । पु॒र्व॒णी॒केति॑ पुरु-अ॒नी॒क॒ । दी॒दि॒हि॒ ॥ क्ष॒पः । रा॒ज॒न्न् । उ॒त । त्मना᳚ । अग्ने᳚ । वस्तोः᳚ । उ॒त । उ॒षसः॑ ॥ सः । ति॒ग्म॒ज॒भेंति॑ तिग्म - ज॒भं॒ ।  \newline




\markright{ TS 4.4.4.6  \hfill https://www.vedavms.in \hfill}

\section{ TS 4.4.4.6 }

\textbf{TS 4.4.4.6 } \newline
\textbf{Samhita Paata} \newline

र॒क्षसो॑ दह॒ प्रति॑ ॥ आ ते॑ अग्न इधीमहि द्यु॒मन्तं॑ देवा॒जरं᳚ । यद्ध॒ स्या ते॒ पनी॑यसी स॒मिद्-दी॒दय॑ति॒ द्यवीषꣳ॑ स्तो॒तृभ्य॒ आ भ॑र ॥ आ ते॑ अग्न ऋ॒चा ह॒विः शु॒क्रस्य॑ ज्योतिषस्पते । सुश्च॑न्द्र॒ दस्म॒ विश्प॑ते॒ हव्य॑वा॒ट् तुभ्यꣳ॑ हूयत॒ इषꣳ॑ स्तो॒तृभ्य॒ आ भ॑र ॥ उ॒भे सु॑श्चन्द्र स॒र्पिषो॒ दर्वी᳚ श्रीणीष आ॒सनि॑ । उ॒तो न॒ उत् पु॑पूर्या- [  ] \newline

\textbf{Pada Paata} \newline

र॒क्षसः॑ । द॒ह॒ । प्रति॑ ॥ एति॑ । ते॒ । अ॒ग्ने॒ । इ॒धी॒म॒हि॒ । द्यु॒मन्त॒मिति॑ द्यु - मन्त᳚म् । दे॒व॒ । अ॒जर᳚म् ॥ यत् । ह॒ । स्या । ते॒ । पनी॑यसी । स॒मिदिति॑ सं - इत् । दी॒दय॑ति । द्यवि॑ । इष᳚म् । स्तो॒तृभ्य॒ इति॑ स्तो॒तृ - भ्यः॒ । एति॑ । भ॒र॒ ॥ एति॑ । ते॒ । अ॒ग्ने॒ । ऋ॒चा । ह॒विः । शु॒क्रस्य॑ । ज्यो॒ति॒षः॒ । प॒ते॒ ॥ सुश्च॒न्द्रेति॒ सु-च॒न्द्र॒ । दस्म॑ । विश्प॑ते । हव्य॑वा॒डिति॒ हव्य॑ - वा॒ट् । तुभ्य᳚म् । हू॒य॒ते॒ । इष᳚म् । स्तो॒तृभ्य॒ इति॑ स्तो॒तृ - भ्यः॒ । एति॑ । भ॒र॒ ॥ उ॒भे इति॑ । सु॒श्च॒न्द्रेति॑ सु - च॒न्द्र॒ । स॒र्पिषः॑ । दर्वी॒ इति॑ । श्री॒णी॒षे॒ । आ॒सनि॑ ॥ उ॒तो इति॑ । नः॒ । उदिति॑ । पु॒पू॒र्याः॒ ।  \newline




\markright{ TS 4.4.4.7  \hfill https://www.vedavms.in \hfill}

\section{ TS 4.4.4.7 }

\textbf{TS 4.4.4.7 } \newline
\textbf{Samhita Paata} \newline

उ॒क्थेषु॑ शवसस्पत॒ इषꣳ॑ स्तो॒तृभ्य॒ आ भ॑र ॥ अग्ने॒ तम॒द्याश्वं॒ न स्तोमैः॒ क्रतुं॒ न भ॒द्रꣳ हृ॑दि॒स्पृशं᳚ । ऋ॒द्ध्यामा॑ त॒ ओहैः᳚ ॥ अधा॒ ह्य॑ग्ने॒ क्रतो᳚र्भ॒द्रस्य॒ दक्ष॑स्य सा॒धोः । र॒थीर्.ऋ॒तस्य॑ बृह॒तो ब॒भूथ॑ ॥ आ॒भिष्टे॑ अ॒द्य गी॒र्भिर्गृ॒णन्तोऽग्ने॒ दाशे॑म । प्र ते॑ दि॒वो न स्त॑नयन्ति॒ शुष्माः᳚ ॥ ए॒भिर्नो॑ अ॒र्कैर्भवा॑ नो अ॒र्वाङ्ख् - [  ] \newline

\textbf{Pada Paata} \newline

उ॒क्थेषु॑ । श॒व॒सः॒ । प॒ते॒ । इष᳚म् । स्तो॒तृभ्य॒ इति॑ स्तो॒तृ - भ्यः॒ । एति॑ । भ॒र॒ ॥ अग्ने᳚ । तम् । अ॒द्य । अश्व᳚म् । न । स्तोमैः᳚ । क्रतु᳚म् । न । भ॒द्रम् । हृ॒दि॒स्पृश॒मिति॑ हृदि - स्पृश᳚म् ॥ ऋ॒द्ध्याम॑ । ते॒ । ओहैः᳚ ॥ अध॑ । हि । अ॒ग्ने॒ । क्रतोः᳚ । भ॒द्रस्य॑ । दक्ष॑स्य । सा॒धोः ॥ र॒थीः । ऋ॒तस्य॑ । बृ॒ह॒तः । ब॒भूथ॑ ॥ आ॒भिः । ते॒ । अ॒द्य । गी॒र्भिः । गृ॒णन्तः॑ । अग्ने᳚ । दाशे॑म ॥ प्रेति॑ । ते॒ । दि॒वः । न । स्त॒न॒य॒न्ति॒ । शुष्माः᳚ ॥ ए॒भिः । नः॒ । अ॒र्कैः । भव॑ । नः॒ । अ॒र्वाङ् ।  \newline




\markright{ TS 4.4.4.8  \hfill https://www.vedavms.in \hfill}

\section{ TS 4.4.4.8 }

\textbf{TS 4.4.4.8 } \newline
\textbf{Samhita Paata} \newline

सुव॒र्न ज्योतिः॑ । अग्ने॒ विश्वे॑भिः सु॒मना॒ अनी॑कैः ॥ अ॒ग्निꣳ होता॑रं मन्ये॒ दास्व॑न्तं॒ ॅवसोः᳚ सू॒नुꣳ सह॑सो जा॒तवे॑दसं । विप्रं॒ न जा॒तवे॑दसं । य ऊ॒र्द्ध्वया᳚ स्वद्ध्व॒रो दे॒वो दे॒वाच्या॑ कृ॒पा । घृ॒तस्य॒ विभ्रा᳚ष्टि॒मनु॑ शु॒क्रशो॑चिष आ॒जुह्वा॑नस्य स॒र्पिषः॑ ॥ अग्ने॒ त्वं नो॒ अन्त॑मः । उ॒त त्रा॒ता शि॒वो भ॑व वरू॒थ्यः॑ ॥ तं त्वा॑ शोचिष्ठ दीदिवः । सु॒म्नाय॑ नू॒नमी॑महे॒ सखि॑भ्यः ॥ वसु॑र॒ग्निर्वसु॑श्रवाः ( ) । अच्छा॑ नक्षि द्यु॒मत्त॑मो र॒यिं दाः᳚ ॥ \newline

\textbf{Pada Paata} \newline

सुवः॑ । न । ज्योतिः॑ ॥ अग्ने᳚ । विश्वे॑भिः । सु॒मना॒ इति॑ सु - मनाः᳚ । अनी॑कैः ॥ अ॒ग्निम् । होता॑रम् । म॒न्ये॒ । दास्व॑न्तम् । वसोः᳚ । सू॒नुम् । सह॑सः । जा॒तवे॑दस॒मिति॑ जा॒त - वे॒द॒स॒म् ॥ विप्र᳚म् । न । जा॒तवे॑दस॒मिति॑ जा॒त - वे॒द॒स॒म् ॥ यः । ऊ॒द्‌र्ध्वया᳚ । स्व॒द्ध्व॒र इति॑ सु - अ॒द्ध्व॒रः । दे॒वः । दे॒वाच्या᳚ । कृ॒पा ॥ घृ॒तस्य॑ । विभ्रा᳚ष्टि॒मिति॒ वि - भ्रा॒ष्टि॒म् । अन्विति॑ । शु॒क्रशो॑चिष॒ इति॑ शु॒क्र - शो॒चि॒षः॒ । आ॒जुह्वा॑न॒स्येत्या᳚ - जुह्वा॑नस्य । स॒र्पिषः॑ ॥ अग्ने᳚ । त्वम् । नः॒ । अन्त॑मः ॥ उ॒त । त्रा॒ता । शि॒वः । भ॒व॒ । व॒रू॒थ्यः॑ ॥ तम् । त्वा॒ । शो॒चि॒ष्ठ॒ । दी॒दि॒वः॒ ॥ सु॒म्नाय॑ । नू॒नम् । ई॒म॒हे॒ । सखि॑भ्य॒ इति॒ सखि॑ - भ्यः॒ ॥ वसुः॑ । अ॒ग्निः । वसु॑श्रवा॒ इति॒ वसु॑ - श्र॒वाः॒ ( ) ॥ अच्छ॑ । न॒क्षि॒ । द्यु॒मत्त॑म॒ इति॑ द्यु॒मत् - त॒मः॒ । र॒यिम् । दाः॒ ॥  \newline




\markright{ TS 4.4.5.1  \hfill https://www.vedavms.in \hfill}

\section{ TS 4.4.5.1 }

\textbf{TS 4.4.5.1 } \newline
\textbf{Samhita Paata} \newline

इ॒न्द्रा॒ग्निभ्यां᳚ त्वा स॒युजा॑ यु॒जा यु॑नज्म्या घा॒राभ्यां॒ तेज॑सा॒ वर्च॑सो॒ क्थेभिः॒ स्तोमे॑भि॒ श्छन्दो॑भी र॒य्यै पोषा॑य सजा॒तानां᳚ मद्ध्यम॒स्थेया॑य॒ मया᳚ त्वा स॒युजा॑ यु॒जा यु॑नज्म्य॒बां दु॒ला नि॑त॒त्नि र॒भ्रय॑न्ती मे॒घय॑न्ती व॒र्॒.षय॑न्ती चुपु॒णीका॒ नामा॑सि प्र॒जाप॑तिना त्वा॒ विश्वा॑भिर्द्धी॒भिरुप॑ दधामि पृथि॒व्यु॑दपु॒रमन्ने॑न वि॒ष्टा म॑नु॒ष्या᳚स्ते गो॒प्तारो॒ ऽग्निर्विय॑त्तोऽस्यां॒ ताम॒हं प्र॑ पद्ये॒ सा - [  ] \newline

\textbf{Pada Paata} \newline

इ॒न्द्रा॒ग्निभ्या॒मिती᳚न्द्रा॒ग्नि - भ्या॒म् । त्वा॒ । स॒युजेति॑ स - युजा᳚ । यु॒जा । यु॒न॒ज्मि॒ । आ॒घा॒राभ्या॒मित्या᳚-घा॒राभ्या᳚म् । तेज॑सा । वर्च॑सा । उ॒क्थेभिः॑ । स्तोमे॑भिः । छन्दो॑भि॒रिति॒ छन्दः॑ - भिः॒ । र॒य्यै । पोषा॑य । स॒जा॒ताना॒मिति॑ स-जा॒ताना᳚म् । म॒द्ध्य॒म॒स्थेया॒येति॑ मद्ध्यम-स्थेया॑य । मया᳚ । त्वा॒ । स॒युजेति॑ स-युजा᳚ । यु॒जा । यु॒न॒ज्मि॒ । अ॒बां । दु॒ला । नि॒त॒त्निरिति॑ नि - त॒त्निः । अ॒भ्रय॑न्ती । मे॒घय॑न्ती । व॒र्॒.षय॑न्ती । चु॒पु॒णीका᳚ । नाम॑ । अ॒सि॒ । प्र॒जाप॑ति॒नेति॑ प्र॒जा - प॒ति॒ना॒ । त्वा॒ । विश्वा॑भिः । धी॒भिः । उपेति॑ । द॒धा॒मि॒ । पृ॒थि॒वी । उ॒द॒पु॒रमित्यु॑द - पु॒रम् । अन्ने॑न । वि॒ष्टा । म॒नु॒ष्याः᳚ । ते॒ । गो॒प्तारः॑ । अ॒ग्निः । विय॑त्त॒ इति॑ वि - य॒त्तः॒ । अ॒स्या॒म् । ताम् । अ॒हम् । प्रेति॑ । प॒द्ये॒ । सा ।  \newline




\markright{ TS 4.4.5.2  \hfill https://www.vedavms.in \hfill}

\section{ TS 4.4.5.2 }

\textbf{TS 4.4.5.2 } \newline
\textbf{Samhita Paata} \newline

मे॒ शर्म॑ च॒ वर्म॑ चा॒स्त्वधि॑ द्यौर॒न्तरि॑क्षं॒ ब्रह्म॑णा वि॒ष्टा म॒रुत॑स्ते गो॒प्तारो॑ वा॒युर्विय॑त्तोऽस्यां॒ ताम॒हं प्र प॑द्ये॒ सा मे॒ शर्म॑ च॒ वर्म॑ चास्तु॒ द्यौरप॑राजिता॒ऽमृते॑न वि॒ष्टाऽऽदि॒त्यास्ते॑ गो॒प्तारः॒ सूर्यो॒ विय॑त्तोऽस्यां॒ ताम॒हं प्र प॑द्ये॒ सा मे॒ शर्म॑ च॒ वर्म॑ चास्तु ॥ \newline

\textbf{Pada Paata} \newline

मे॒ । शर्म॑ । च॒ । वर्म॑ । च॒ । अ॒स्तु॒ । अधि॑द्यौ॒रित्यधि॑ - द्यौः॒ । अ॒न्तरि॑क्षम् । ब्रह्म॑णा । वि॒ष्टा । म॒रुतः॑ । ते॒ । गो॒प्तारः॑ । वा॒युः । विय॑त्त॒ इति॒ वि - य॒त्तः॒ । अ॒स्या॒म् । ताम् । अ॒हम् । प्रेति॑ । प॒द्ये॒ । सा । मे॒ । शर्म॑ । च॒ । वर्म॑ । च॒ । अ॒स्तु॒ । द्यौः । अप॑राजि॒तेत्यप॑रा - जि॒ता॒ । अ॒मृते॑न । वि॒ष्टा । आ॒दि॒त्याः । ते॒ । गो॒प्तारः॑ । सूर्यः॑ । विय॑त्त॒ इति॒ वि-य॒त्तः॒ । अ॒स्या॒म् । ताम् । अ॒हम् । प्रेति॑ । प॒द्ये॒ । सा । मे॒ । शर्म॑ । च॒ । वर्म॑ । च॒ । अ॒स्तु॒ ॥  \newline




\markright{ TS 4.4.6.1  \hfill https://www.vedavms.in \hfill}

\section{ TS 4.4.6.1 }

\textbf{TS 4.4.6.1 } \newline
\textbf{Samhita Paata} \newline

बृह॒स्पति॑स्त्वा सादयतु पृथि॒व्याः पृ॒ष्ठे ज्योति॑ष्मतीं॒ ॅविश्व॑स्मै प्रा॒णाया॑पा॒नाय॒ विश्वं॒ ज्योति॑र्यच्छा॒- ग्निस्तेऽधि॑पति र्वि॒श्वक॑र्मा त्वा सादयत्व॒न्तरि॑क्षस्य पृ॒ष्ठे ज्योति॑ष्मतीं॒ ॅविश्व॑स्मै प्रा॒णाया॑पा॒नाय॒ विश्वं॒ ज्योति॑र्यच्छ वा॒युस्तेऽधि॑पतिः प्र॒जाप॑तिस्त्वा सादयतु दि॒वः पृ॒ष्ठे ज्योति॑ष्मतीं॒ ॅविश्व॑स्मै प्रा॒णाया॑पा॒नाय॒ विश्वं॒ ज्योति॑र्यच्छ परमे॒ष्ठी तेऽधि॑पतिः पुरोवात॒सनि॑रस्य भ्र॒सनि॑रसि विद्यु॒थ्सनि॑ - [  ] \newline

\textbf{Pada Paata} \newline

बृह॒स्पतिः॑ । त्वा॒ । सा॒द॒य॒तु॒ । पृ॒थि॒व्याः । पृ॒ष्ठे । ज्योति॑ष्मतीम् । विश्व॑स्मै । प्रा॒णायेति॑ प्र-अ॒नाय॑ । अ॒पा॒नायेत्य॑प - अ॒नाय॑ । विश्व᳚म् । ज्योतिः॑ । य॒च्छ॒ । अ॒ग्निः । ते॒ । अधि॑पति॒रित्यधि॑ - प॒तिः॒ । वि॒श्वक॒र्मेति॑ वि॒श्व - क॒र्मा॒ । त्वा॒ । सा॒द॒य॒तु॒ । अ॒न्तरि॑क्षस्य । पृ॒ष्ठे । ज्योति॑ष्मतीम् । विश्व॑स्मै । प्रा॒णायेति॑ प्र - अ॒नाय॑ । अ॒पा॒नायेत्य॑प - अ॒नाय॑ । विश्व᳚म् । ज्योतिः॑ । य॒च्छ॒ । वा॒युः । ते॒ । अधि॑पति॒रित्यधि॑ - प॒तिः॒ । प्र॒जाप॑ति॒रिति॑ प्र॒जा - प॒तिः॒ । त्वा॒ । सा॒द॒य॒तु॒ । दि॒वः । पृ॒ष्ठे । ज्योति॑ष्मतीम् । विश्व॑स्मै । प्रा॒णायेति॑ प्र - अ॒नाय॑ । अ॒पा॒नायेत्य॑प - अ॒नाय॑ । विश्व᳚म् । ज्योतिः॑ । य॒च्छ॒ । प॒र॒मे॒ष्ठी । ते॒ । अधि॑पति॒रित्यधि॑ - प॒तिः॒ । पु॒रो॒वा॒त॒सनि॒रिति॑ पुरोवात - सनिः॑ । अ॒स्य॒ । अ॒भ्र॒सनि॒रित्य॑भ्र - सनिः॑ । अ॒सि॒ । वि॒द्यु॒थ्सनि॒रिति॑ विद्युत् - सनिः॑ ।  \newline




\markright{ TS 4.4.6.2  \hfill https://www.vedavms.in \hfill}

\section{ TS 4.4.6.2 }

\textbf{TS 4.4.6.2 } \newline
\textbf{Samhita Paata} \newline

-रसि स्तनयित्नु॒सनि॑रसि वृष्टि॒सनि॑रस्य॒-ग्नेर्यान्य॑सि दे॒वाना॑मग्ने॒ यान्य॑सि वा॒योर्यान्य॑सि दे॒वानां᳚ ॅवायो॒यान्य॑स्य॒न्तरि॑क्षस्य॒ यान्य॑सि दे॒वाना॑- मन्तरिक्ष॒यान्य॑स्य॒-न्तरि॑क्षमस्य॒न्तरि॑क्षाय त्वा सलि॒लाय॑ त्वा॒ सर्णी॑काय त्वा॒ सती॑काय त्वा॒ केता॑य त्वा॒ प्रचे॑तसे त्वा॒ विव॑स्वते त्वा दि॒वस्त्वा॒ ज्योति॑ष आदि॒त्येभ्य॑स्त्व॒र्चे त्वा॑ रु॒चे त्वा᳚ द्यु॒ते त्वा॑ ( ) भा॒से त्वा॒ ज्योति॑षे त्वा यशो॒दां त्वा॒ यश॑सि तेजो॒दां त्वा॒ तेज॑सि पयो॒दां त्वा॒ पय॑सि वर्चो॒दां त्वा॒ वर्च॑सि द्रविणो॒दां त्वा॒ द्रवि॑णे सादयामि॒ तेनर्.षि॑णा॒ तेन॒ ब्रह्म॑णा॒ तया॑ दे॒वत॑याऽङ्गिर॒स्वद् ध्रु॒वा सी॑द ॥ \newline

\textbf{Pada Paata} \newline

अ॒सि॒ । स्त॒न॒यि॒त्नु॒सनि॒रिति॑ स्तनयित्नु - सनिः॑ । अ॒सि॒ । वृ॒ष्टि॒सनि॒रिति॑ वृष्टि - सनिः॑ । अ॒सि॒ । अ॒ग्नेः । यानी᳚ । अ॒सि॒ । दे॒वाना᳚म् । अ॒ग्ने॒यानीत्य॑ग्ने-यानी᳚ । अ॒सि॒ । वा॒योः । यानी᳚ । अ॒सि॒ । दे॒वाना᳚म् । वा॒यो॒यानीति॑ वायो - यानी᳚ । अ॒सि॒ । अ॒न्तरि॑क्षस्य । यानी᳚ । अ॒सि॒ । दे॒वाना᳚म् । अ॒न्त॒रि॒क्ष॒यानीत्य॑न्तरिक्ष - यानी᳚ । अ॒सि॒ । अ॒न्तरि॑क्षम् । अ॒सि॒ । अ॒न्तरि॑क्षाय । त्वा॒ । स॒लि॒लाय॑ । त्वा॒ । सर्णी॑काय । त्वा॒ । सती॑का॒येति॒ स - ती॒का॒य॒ । त्वा॒ । केता॑य । त्वा॒ । प्रचे॑तस॒ इति॒ प्र - चे॒त॒से॒ । त्वा॒ । विव॑स्वते । त्वा॒ । दि॒वः । त्वा॒ । ज्योति॑षे । आ॒दि॒त्येभ्यः॑ । त्वा॒ । ऋ॒चे । त्वा॒ । रु॒चे । त्वा॒ । द्यु॒ते । त्वा॒ ( ) । भा॒से । त्वा॒ । ज्योति॑षे । त्वा॒ । य॒शो॒दामिति॑ यशः-दाम् । त्वा॒ । यश॑सि । ते॒जो॒दामिति॑ तेजः - दाम् । त्वा॒ । तेज॑सि । प॒यो॒दामिति॑ पयः - दाम् । त्वा॒ ।पय॑सि । व॒र्चो॒दामिति॑ वर्चः-दाम् । त्वा॒ । वर्च॑सि । द्र॒वि॒णो॒दामिति॑ द्रविणः - दाम् । त्वा॒ । द्रवि॑णे । सा॒द॒या॒मि॒ । तेन॑ । ऋषि॑णा । तेन॑ । ब्रह्म॑णा । तया᳚ । दे॒वत॑या । अ॒ङ्गि॒र॒स्वत् । ध्रु॒वा । सी॒द॒ ॥  \newline




\markright{ TS 4.4.7.1  \hfill https://www.vedavms.in \hfill}

\section{ TS 4.4.7.1 }

\textbf{TS 4.4.7.1 } \newline
\textbf{Samhita Paata} \newline

भू॒य॒स्कृद॑सि वरिव॒स्कृद॑सि॒ प्राच्य॑स्यू॒र्द्ध्वाऽस्य॑-न्तरिक्ष॒सद॑स्य॒-न्तरि॑क्षे सीदा-फ्सु॒षद॑सि श्येन॒सद॑सि गृद्ध्र॒सद॑सि सुपर्ण॒सद॑सि नाक॒सद॑सि पृथि॒व्यास्त्वा॒ द्रवि॑णे सादयाम्य॒-न्तरि॑क्षस्य त्वा॒ द्रवि॑णे सादयामि दि॒वस्त्वा॒ द्रवि॑णे सादयामि दि॒शां त्वा॒ द्रवि॑णे सादयामि द्रविणो॒दां त्वा॒ द्रवि॑णे सादयामि प्रा॒णं मे॑ पाह्य-पा॒नं मे॑ पाहि व्या॒नं मे॑ - [  ] \newline

\textbf{Pada Paata} \newline

भू॒य॒स्कृदिति॑ भूयः - कृत् । अ॒सि॒ । व॒रि॒व॒स्कृदिति॑ वरिवः - कृत् । अ॒सि॒ । प्राची᳚ । अ॒सि॒ । ऊ॒द्‌र्ध्वा । अ॒सि॒ । अ॒न्त॒रि॒क्ष॒सदित्य॑न्तरिक्ष - सत् । अ॒सि॒ । अ॒न्तरि॑क्षे । सी॒द॒ । अ॒फ्सु॒षदित्य॑फ्सु - सत् । अ॒सि॒ । श्ये॒न॒सदिति॑ श्येन-सत् । अ॒सि॒ । गृ॒द्ध्र॒सदिति॑ गृद्ध्र-सत् । अ॒सि॒ । सु॒प॒र्ण॒सदिति॑ सुपर्ण - सत् । अ॒सि॒ । ना॒क॒सदिति॑ नाक - सत् । अ॒सि॒ । पृ॒थि॒व्याः । त्वा॒ । द्रवि॑णे । सा॒द॒या॒मि॒ । अ॒न्तरि॑क्षस्य । त्वा॒ । द्रवि॑णे । सा॒द॒या॒मि॒ । दि॒वः । त्वा॒ । द्रवि॑णे । सा॒द॒या॒मि॒ । दि॒शाम् । त्वा॒ । द्रवि॑णे । सा॒द॒या॒मि॒ । द्र॒वि॒णो॒दामिति॑ द्रविणः - दाम् । त्वा॒ । द्रवि॑णे । सा॒द॒या॒मि॒ । प्रा॒णमिति॑ प्र - अ॒नम् । मे॒ । पा॒हि॒ । आ॒पा॒नमित्य॑प - अ॒नम् । मे॒ । पा॒हि॒ । व्या॒नमिति॑ वि - अ॒नम् । मे॒ ।  \newline




\markright{ TS 4.4.7.2  \hfill https://www.vedavms.in \hfill}

\section{ TS 4.4.7.2 }

\textbf{TS 4.4.7.2 } \newline
\textbf{Samhita Paata} \newline

पा॒ह्यायु॑र्मे पाहि वि॒श्वायु॑र्मे पाहि स॒र्वायु॑र्मे पा॒ह्यग्ने॒ यत् ते॒ परꣳ॒॒ हृन्नाम॒ तावेहि॒ सꣳ र॑भावहै॒ पाञ्च॑ जन्ये॒ष्व-प्ये᳚द्ध्यग्ने॒ यावा॒ अया॑वा॒ एवा॒ ऊमाः॒ सब्दः॒ सग॑रः सु॒मेकः॑ ॥ \newline

\textbf{Pada Paata} \newline

पा॒हि॒ । आयुः॑ । मे॒ । पा॒हि॒ । वि॒श्वायु॒रिति॑ वि॒श्व-आ॒युः॒ । मे॒ । पा॒हि॒ । स॒र्वायु॒रिति॑ स॒र्व - आ॒युः॒ । मे॒ । पा॒हि॒ । अग्ने᳚ । यत् । ते॒ । पर᳚म् । हृत् । नाम॑ । तौ । एति॑ । इ॒हि॒ । समिति॑ । र॒भा॒व॒है॒ । पाञ्च॑जन्ये॒ष्विति॒ पाञ्च॑ - ज॒न्ये॒षु॒ । अपीति॑ । ए॒धि॒ । अ॒ग्ने॒ । यावाः᳚ । अया॑वाः । एवाः᳚ । ऊमाः᳚ । सब्दः॑ । सग॑रः । सु॒मेक॒ इति॑ सु - मेकः॑ ॥  \newline




\markright{ TS 4.4.8.1  \hfill https://www.vedavms.in \hfill}

\section{ TS 4.4.8.1 }

\textbf{TS 4.4.8.1 } \newline
\textbf{Samhita Paata} \newline

अ॒ग्निना॑ विश्वा॒षाट् सूर्ये॑ण स्व॒राट् क्रत्वा॒ शची॒पति॑र्. ऋष॒भेण॒ त्वष्टा॑ य॒ज्ञेन॑ म॒घवा॒न् दक्षि॑णया सुव॒र्गो म॒न्युना॑ वृत्र॒हा सौहा᳚र्द्येन तनू॒धा अन्ने॑न॒ गयः॑ पृथि॒व्याऽस॑नो दृ॒ग्भिर॑न्ना॒दो व॑षट्का॒रेण॒र्द्धः साम्ना॑ तनू॒पा वि॒राजा॒ ज्योति॑ष्मा॒न् ब्रह्म॑णा सोम॒पा गोभि॑र्य॒ज्ञ्ं दा॑धार क्ष॒त्रेण॑ मनु॒ष्या॑-नश्वे॑न च॒ रथे॑न च व॒ज्र्यृ॑तुभिः॑ प्र॒भुः सं॑ॅवथ्स॒रेण॑ परि॒भू स्तप॒साऽना॑धृष्टः॒ सूर्यः॒ सन् त॒नूभिः॑ ॥ \newline

\textbf{Pada Paata} \newline

अ॒ग्निना᳚ । वि॒श्वा॒षाट् । सूर्ये॑ण । स्व॒राडिति॑ स्व - राट् । क्रत्वा᳚ । शची॒पतिः॑ । ऋ॒ष॒भेण॑ । त्वष्टा᳚ । य॒ज्ञेन॑ । म॒घवा॒निति॑ म॒घ-वा॒न् । दक्षि॑णया । सु॒व॒र्ग इति॑ सुवः - गः । म॒न्युना᳚ । वृ॒त्र॒हेति॑ वृत्र - हा । सौहा᳚र्द्येन । त॒नू॒धा इति॑ तनू - धाः । अन्ने॑न । गयः॑ । पृ॒थि॒व्या । अ॒स॒नो॒त् । ऋ॒ग्भिरित्यृ॑क् - भिः । अ॒न्ना॒द इत्य॑न्न - अ॒दः । व॒ष॒ट्का॒रेणेति॑ वषट् - का॒रेण॑ । ऋ॒द्धः । साम्ना᳚ । त॒नू॒पा इति॑ तनू - पाः । वि॒राजेति॑ वि - राजा᳚ । ज्योति॑ष्मान् । ब्रह्म॑णा । सो॒म॒पा इति॑ सोम - पाः । गोभिः॑ । य॒ज्ञ्म् । दा॒धा॒र॒ । क्ष॒त्रेण॑ । म॒नु॒ष्यान्॑ । अश्वे॑न । च॒ । रथे॑न । च॒ । व॒ज्री । ऋ॒तुभि॒रित्यृ॒तु - भिः॒ । प्र॒भुरिति॑ प्र - भुः । सं॒ॅव॒थ्स॒रेणेति॑ सं - व॒थ्स॒रेण॑ । प॒रि॒भूरिति॑ परि - भूः । तप॑सा । अना॑धृष्ट॒ इत्यना᳚ - धृ॒ष्टः॒ । सूर्यः॑ । सन्न् । त॒नूभिः॑ ॥  \newline




\markright{ TS 4.4.9.1  \hfill https://www.vedavms.in \hfill}

\section{ TS 4.4.9.1 }

\textbf{TS 4.4.9.1 } \newline
\textbf{Samhita Paata} \newline

प्र॒जाप॑ति॒र्मन॒सा ऽन्धोऽच्छे॑तो धा॒ता दी॒क्षायाꣳ॑ सवि॒ता भृ॒त्यां पू॒षा सो॑म॒क्रय॑ण्यां॒ ॅवरु॑ण॒ उप॑न॒द्धो ऽसु॑रः क्री॒यमा॑णो मि॒त्रः क्री॒तः शि॑पिवि॒ष्ट आसा॑दितो न॒रंधि॑षः प्रो॒ह्यमा॒णो ऽधि॑पति॒राग॑तः प्र॒जाप॑तिः प्रणी॒यमा॑नो॒ ऽग्निराग्नी᳚द्ध्रे॒ बृह॒स्पति॒राग्नी᳚द्ध्रात् प्रणी॒यमा॑न॒ इन्द्रो॑ हवि॒र्द्धाने ऽदि॑ति॒रासा॑दितो॒ विष्णु॑रुपावह्रि॒यमा॒णो ऽथ॒र्वोपो᳚त्तो य॒मो॑ऽभिषु॑तो ऽपूत॒पा आ॑धू॒यमा॑नो वा॒युः पू॒यमा॑नो मि॒त्रः क्षी॑र॒श्रीर्म॒न्थी स॑क्तु॒श्रीर्वै᳚श्वदे॒व उन्नी॑तो रु॒द्र ( ) आहु॑तो वा॒युरावृ॑त्तो नृ॒चक्षाः॒ प्रति॑ख्यातो भ॒क्ष आग॑तः पितृ॒णां ना॑राशꣳ॒॒सो ऽसु॒रात्तः॒ सिन्धु॑र-वभृ॒थम॑वप्र॒यन्थ् स॑मु॒द्रो ऽव॑गतः सलि॒लः प्रप्लु॑तः॒ सुव॑रु॒दृचं॑ ग॒तः ॥ \newline

\textbf{Pada Paata} \newline

प्र॒जाप॑ति॒रिति॑ प्र॒जा - प॒तिः॒ । मन॑सा । अन्धः॑ । अच्छे॑त॒ इत्यच्छ॑ - इ॒तः॒ । धा॒ता । दी॒क्षाया᳚म् । स॒वि॒ता । भृ॒त्याम् । पू॒षा । सो॒म॒क्रय॑ण्या॒मिति॑ सोम - क्रय॑ण्याम् । वरु॑णः । उप॑नद्ध॒ इत्युप॑ - न॒द्धः॒ । असु॑रः । क्री॒यमा॑णः । मि॒त्रः । क्री॒तः । शि॒पि॒वि॒ष्ट इति॑ शिपि - वि॒ष्टः । आसा॑दित॒ इत्या - सा॒दि॒तः॒ । न॒रंधि॑षः । प्रो॒ह्यमा॑ण॒ इति॑ प्र - उ॒ह्यमा॑णः । अधि॑पति॒रित्यधि॑-प॒तिः॒ । आग॑त॒ इत्या - ग॒तः॒ । प्र॒जाप॑ति॒रिति॑ प्र॒जा - प॒तिः॒ । प्र॒णी॒यमा॑न॒ इति॑ प्र-नी॒यमा॑नः । अ॒ग्निः । आग्नी᳚द्ध्र॒ इत्याग्नि॑-इ॒द्ध्रे॒ । बृह॒स्पतिः॑ । आग्नी᳚द्ध्रा॒दित्याग्नि॑ - इ॒द्ध्रा॒त् । प्र॒णी॒यमा॑न॒ इति॑ प्र - नी॒यमा॑नः । इन्द्रः॑ । ह॒वि॒द्‌र्धान॒ इति॑ हविः - धाने᳚ । अदि॑तिः । आसा॑दित॒ इत्या - सा॒दि॒तः॒ । विष्णुः॑ । उ॒पा॒व॒ह्रि॒यमा॑ण॒ इत्यु॑प - अ॒व॒ह्रि॒यमा॑णः । अथ॑र्वा । उपो᳚त्त॒ इत्युप॑ - उ॒त्तः॒ । य॒मः । अ॒भिषु॑त॒ इत्य॒भि - सु॒तः॒ । अ॒पू॒त॒पा इत्य॑पूत - पाः । आ॒धू॒यमा॑न॒ इत्या᳚ - धू॒यमा॑नः । वा॒युः । पू॒यमा॑नः । मि॒त्रः । क्षी॒र॒श्रीरिति॑ क्षीर-श्रीः । म॒न्थी । स॒क्तु॒श्रीरिति॑ सक्तु - श्रीः । वै॒श्व॒दे॒व इति॑ वैश्व - दे॒वः । उन्नी॑त॒ इत्युत् - नी॒तः॒ । रु॒द्रः ( ) । आहु॑त॒ इत्या - हु॒तः॒ । वा॒युः । आवृ॑त्त॒ इत्या - वृ॒त्तः॒ । नृ॒चक्षा॒ इति॑ नृ - चक्षाः᳚ । प्रति॑ख्यात॒ इति॒ प्रति॑ - ख्या॒तः॒ । भ॒क्षः । आग॑त॒ इत्या - ग॒तः॒ । पि॒तृ॒णाम् । ना॒रा॒शꣳ॒॒सः । असुः॑ । आत्तः॑ । सिन्धुः॑ । अ॒व॒भृ॒थमित्य॑व - भृ॒थम् । अ॒व॒प्र॒यन्नित्य॑व - प्र॒यन्न् । स॒मु॒द्रः । अव॑गत॒ इत्यव॑ - ग॒तः॒ । स॒लि॒लः । प्रप्लु॑त॒ इति॒ प्र - प्लु॒तः॒ । सुवः॑ । उ॒दृच॒मित्यु॑त् - ऋच᳚म् । ग॒तः ॥  \newline




\markright{ TS 4.4.10.1  \hfill https://www.vedavms.in \hfill}

\section{ TS 4.4.10.1 }

\textbf{TS 4.4.10.1 } \newline
\textbf{Samhita Paata} \newline

कृत्ति॑का॒ नक्ष॑त्र-म॒ग्निर्दे॒वता॒ऽग्ने रुचः॑ स्थ प्र॒जाप॑तेर्द्धा॒तुः सोम॑स्य॒र्चे त्वा॑ रु॒चे त्वा᳚ द्यु॒ते त्वा॑ भा॒से त्वा॒ ज्योति॑षे त्वा रोहि॒णी नक्ष॑त्रं प्र॒जाप॑तिर्दे॒वता॑ मृगशी॒र्॒.षं॑ नक्ष॑त्रꣳ॒॒ सोमो॑ दे॒वता॒ ऽऽर्द्रा नक्ष॑त्रꣳ रु॒द्रो दे॒वता॒ पुन॑र्वसू॒ नक्ष॑त्र॒मदि॑तिर्दे॒वता॑- ति॒ष्यो॑ नक्ष॑त्रं॒ बृह॒स्पति॑र्दे॒वता᳚ ऽऽश्रे॒षा नक्ष॑त्रꣳ स॒र्पा दे॒वता॑ म॒घा नक्ष॑त्रं पि॒तरो॑ दे॒वता॒ फल्गु॑नी॒ नक्ष॑त्र - [  ] \newline

\textbf{Pada Paata} \newline

कृत्ति॑काः । नक्ष॑त्रम् । अ॒ग्निः । दे॒वता᳚ । अ॒ग्नेः । रुचः॑ । स्थ॒ । प्र॒जाप॑ते॒रिति॑ प्र॒जा - प॒तेः॒ । धा॒तुः । सोम॑स्य । ऋ॒चे । त्वा॒ । रु॒चे । त्वा॒ । द्यु॒ते । त्वा॒ । भा॒से । त्वा॒ । ज्योति॑षे । त्वा॒ । रो॒हि॒णी । नक्ष॑त्रम् । प्र॒जाप॑ति॒रिति॑ प्र॒जा - प॒तिः॒ । दे॒वता᳚ । मृ॒ग॒शी॒र्॒.षमिति॑ मृग - शी॒र्॒.षम् । नक्ष॑त्रम् । सोमः॑ । दे॒वता᳚ । आ॒र्द्रा । नक्ष॑त्रम् । रु॒द्रः । दे॒वता᳚ । पुन॑र्वसू॒ इति॒ पुनः॑ - व॒सू॒ । नक्ष॑त्रम् । अदि॑तिः । दे॒वता᳚ । ति॒ष्यः॑ । नक्ष॑त्रम् । बृह॒स्पतिः॑ । दे॒वता᳚ । आ॒श्रे॒षा इत्या᳚- श्रे॒षाः । नक्ष॑त्रम् । स॒र्पाः । दे॒वता᳚ । म॒घाः । नक्ष॑त्रम् । पि॒तरः॑ । दे॒वता᳚ । फल्गु॑नी॒ इति॑ । नक्ष॑त्रम् ।  \newline




\markright{ TS 4.4.10.2  \hfill https://www.vedavms.in \hfill}

\section{ TS 4.4.10.2 }

\textbf{TS 4.4.10.2 } \newline
\textbf{Samhita Paata} \newline

-मर्य॒मा दे॒वता॒ फल्गु॑नी॒ नक्ष॑त्रं॒ भगो॑ दे॒वता॒ हस्तो॒ नक्ष॑त्रꣳ सवि॒ता दे॒वता॑ चि॒त्रा नक्ष॑त्र॒मिन्द्रो॑ दे॒वता᳚ स्वा॒ती नक्ष॑त्रं ॅवा॒युर्दे॒वता॒ विशा॑खे॒ नक्ष॑त्रमिन्द्रा॒ग्नी दे॒वता॑ ऽनूरा॒धा नक्ष॑त्रं मि॒त्रो दे॒वता॑ रोहि॒णी नक्ष॑त्र॒मिन्द्रो॑ दे॒वता॑ वि॒चृतौ॒ नक्ष॑त्रं पि॒तरो॑ दे॒वता॑ ऽषा॒ढा नक्ष॑त्र॒मापो॑ दे॒वता॑ ऽषा॒ढा नक्ष॑त्रं॒ ॅविश्वे॑ दे॒वा दे॒वता᳚ श्रो॒णा नक्ष॑त्रं॒ ॅविष्णु॑र्दे॒वता॒ श्रवि॑ष्ठा॒ नक्ष॑त्रं॒ ॅवस॑वो - [  ] \newline

\textbf{Pada Paata} \newline

अ॒र्य॒मा । दे॒वता᳚ । फल्गु॑नी॒ इति॑ । नक्ष॑त्रम् । भगः॑ । दे॒वता᳚ । हस्तः॑ । नक्ष॑त्रम् । स॒वि॒ता । दे॒वता᳚ । चि॒त्रा । नक्ष॑त्रम् । इन्द्रः॑ । दे॒वता᳚ । स्वा॒ती । नक्ष॑त्रम् । वा॒युः । दे॒वता᳚ । विशा॑खे॒ इति॒ वि - शा॒खे॒ । नक्ष॑त्रम् । इ॑006छ्;॒द्रा॒ग्नी इती᳚न्द्र - अ॒ग्नी । दे॒वता॑ । अ॒नू॒रा॒धा इत्य॑नु - रा॒धाः । नक्ष॑त्रम् । मि॒त्रः । दे॒वता᳚ । रो॒हि॒णी । नक्ष॑त्रम् । इन्द्रः॑ । दे॒वता᳚ । वि॒चृता॒विति॑ वि - चृतौ᳚ । नक्ष॑त्रम् । पि॒तरः॑ । दे॒वता᳚ । अ॒षा॒ढाः । नक्ष॑त्रम् । आपः॑ । दे॒वता᳚ । अ॒षा॒ढाः । नक्ष॑त्रम् । विश्वे᳚ । दे॒वाः । दे॒वता᳚ । श्रो॒णा । नक्ष॑त्रम् । विष्णुः॑ । दे॒वता᳚ । श्रवि॑ष्ठाः । नक्ष॑त्रम् । वस॑वः ।  \newline




\markright{ TS 4.4.10.3  \hfill https://www.vedavms.in \hfill}

\section{ TS 4.4.10.3 }

\textbf{TS 4.4.10.3 } \newline
\textbf{Samhita Paata} \newline

दे॒वता॑ श॒तभि॑ष॒ङ् नक्ष॑त्र॒मिन्द्रो॑ दे॒वता᳚ प्रोष्ठप॒दा नक्ष॑त्रम॒ज एक॑पाद् दे॒वता᳚ प्रोष्ठप॒दा नक्ष॑त्र॒महि॑र्बु॒द्ध्नियो॑ दे॒वता॑ रे॒वती॒ नक्ष॑त्रं पू॒षा दे॒वता᳚ ऽश्व॒युजौ॒ नक्ष॑त्रम॒श्विनौ॑ दे॒वता॑ ऽप॒भर॑णी॒र्नक्ष॑त्रं ॅय॒मो दे॒वता॑, पू॒र्णा प॒श्चाद्>1, यत् ते॑ दे॒वा अद॑धुः >2 ॥ \newline

\textbf{Pada Paata} \newline

दे॒वता᳚ । श॒तभि॑ष॒गिति॑ श॒त - भि॒ष॒क् । नक्ष॑त्रम् । इन्द्रः॑ । दे॒वता᳚ । प्रो॒ष्ठ॒प॒दा इति॑ प्रोष्ठ - प॒दाः । नक्ष॑त्रम् । अ॒जः । एक॑पा॒दित्येक॑ - पा॒त् । दे॒वता᳚ । प्रो॒ष्ठ॒प॒दा इति॑ प्रोष्ठ - प॒दाः । नक्ष॑त्रम् । अहिः॑ । बु॒द्ध्नियः॑ । दे॒वता᳚ । रे॒वती᳚ । नक्ष॑त्रम् । पू॒षा । दे॒वता᳚ । अ॒श्व॒युजा॒वित्य॑श्व - युजौ᳚ । नक्ष॑त्रम् । अ॒श्विनौ᳚ । दे॒वता᳚ । अ॒प॒भर॑णी॒रित्य॑प - भर॑णीः । नक्ष॑त्रम् । य॒मः । दे॒वता᳚ । पू॒र्णा । प॒श्चात् । यत् । ते॒ । दे॒वाः । अद॑धुः ॥  \newline




\markright{ TS 4.4.11.1  \hfill https://www.vedavms.in \hfill}

\section{ TS 4.4.11.1 }

\textbf{TS 4.4.11.1 } \newline
\textbf{Samhita Paata} \newline

मधु॑श्च॒ माध॑वश्च॒ वास॑न्तिकावृ॒तू शु॒क्रश्च॒ शुचि॑श्च॒ ग्रैष्मा॑वृ॒तू नभ॑श्च नभ॒स्य॑श्च॒ वार्.षि॑कावृ॒तू इ॒षश्चो॒र्जश्च॑ शार॒दावृ॒तू सह॑श्च सह॒स्य॑श्च॒ हैम॑न्तिकावृ॒तू तप॑श्च तप॒स्य॑श्च शैशि॒रावृ॒तू अ॒ग्नेर॑न्तः श्ले॒षो॑ऽसि॒ कल्पे॑तां॒ द्यावा॑पृथि॒वी कल्प॑न्ता॒माप॒ ओष॑धीः॒ कल्प॑न्ताम॒ग्नयः॒ पृथ॒ङ्मम॒ ज्यैष्ठ्य॑य॒ सव्र॑ता॒- [  ] \newline

\textbf{Pada Paata} \newline

मधुः॑ । च॒ । माध॑वः । च॒ । वास॑न्तिकौ । ऋ॒तू इति॑ । शु॒क्रः । च॒ । शुचिः॑ । च॒ । ग्रैष्मौ᳚ । ऋ॒तू इति॑ । नभः॑ । च॒ । न॒भ॒स्यः॑ । च॒ । वार्.षि॑कौ । ऋ॒तू इति॑ । इ॒षः । च॒ । ऊ॒र्जः । च॒ । शा॒र॒दौ । ऋ॒तू इति॑ । सहः॑ । च॒ । स॒ह॒स्यः॑ । च॒ । हैम॑न्तिकौ । ऋ॒तू इति॑ । तपः॑ । च॒ । त॒प॒स्यः॑ । च॒ । शै॒शि॒रौ । ऋ॒तू इति॑ । अ॒ग्नेः । अ॒न्तः॒श्ले॒ष इत्य॑न्तः - श्ले॒षः । अ॒सि॒ । कल्पे॑ताम् । द्यावा॑पृथि॒वी इति॒ द्यावा᳚ - पृ॒थि॒वी । कल्प॑न्ताम् । आपः॑ । ओष॑धीः । कल्प॑न्ताम् । अ॒ग्नयः॑ । पृथ॑क् । मम॑ । ज्यैष्ठ्य॑य । सव्र॑ता॒ इति॒ स-व्र॒ताः॒ ।  \newline




\markright{ TS 4.4.11.2  \hfill https://www.vedavms.in \hfill}

\section{ TS 4.4.11.2 }

\textbf{TS 4.4.11.2 } \newline
\textbf{Samhita Paata} \newline

ये᳚ऽग्नयः॒ सम॑नसोऽन्त॒रा द्यावा॑पृथि॒वी शै॑शि॒रावृ॒तू अ॒भि कल्प॑माना॒ इन्द्र॑मिव दे॒वा अ॒भि संॅवि॑शन्तु सं॒ॅयच्च॒ प्रचे॑ताश्चा॒ग्नेः सोम॑स्य॒ सूर्य॑स्यो॒-ग्रा च॑ भी॒मा च॑ पितृ॒णां ॅय॒मस्येन्द्र॑स्य ध्रु॒वा च॑ पृथि॒वी च॑ दे॒वस्य॑ सवि॒तुर्म॒रुतां॒ ॅवरु॑णस्य ध॒र्त्री च॒ धरि॑त्री च मि॒त्रावरु॑णयो र्मि॒त्रस्य॑ धा॒तुः प्राची॑ च प्र॒तीची॑ च॒ वसू॑नाꣳ रु॒द्राणा॑ - [  ] \newline

\textbf{Pada Paata} \newline

ये । अ॒ग्नयः॑ । सम॑नस॒ इति॒ स - म॒न॒सः॒ । अ॒न्त॒रा । द्यावा॑पृथि॒वी इति॒ द्यावा᳚ - पृ॒थि॒वी । शै॒शि॒रौ । ऋ॒तू इति॑ । अ॒भीति॑ । कल्प॑मानाः । इन्द्र᳚म् । इ॒व॒ । दे॒वाः । अ॒भि । समिति॑ । वि॒श॒न्तु॒ । सं॒ॅयदिति॑ सं - यत् । च॒ । प्रचे॑ता॒ इति॒ प्र - चे॒ताः॒ । च॒ । अ॒ग्नेः । सोम॑स्य । सूर्य॑स्य । उ॒ग्रा । च॒ । भी॒मा । च॒ । पि॒तृ॒णाम् । य॒मस्य॑ । इन्द्र॑स्य । ध्रु॒वा । च॒ । पृ॒थि॒वी । च॒ । दे॒वस्य॑ । स॒वि॒तुः । म॒रुता᳚म् । वरु॑णस्य । ध॒र्त्री । च॒ । धरि॑त्री । च॒ । मि॒त्रावरु॑णयो॒रिति॑ मि॒त्रा - वरु॑णयोः । मि॒त्रस्य॑ । धा॒तुः । प्राची᳚ । च॒ । प्र॒तीची᳚ । च॒ । वसू॑नाम् । रु॒द्राणां᳚ ।  \newline




\markright{ TS 4.4.11.3  \hfill https://www.vedavms.in \hfill}

\section{ TS 4.4.11.3 }

\textbf{TS 4.4.11.3 } \newline
\textbf{Samhita Paata} \newline

-मादि॒त्यानां॒ ते तेऽधि॑पतय॒स्तेभ्यो॒ नम॒स्ते नो॑ मृडयन्तु॒ ते यं द्वि॒ष्मो यश्च॑ नो॒ द्वेष्टि॒ तं ॅवो॒ जंभे॑ दधामि स॒हस्र॑स्य प्र॒मा अ॑सि स॒हस्र॑स्य प्रति॒मा अ॑सि स॒हस्र॑स्य वि॒मा अ॑सि स॒हस्र॑स्यो॒न्मा अ॑सि साह॒स्रो॑ऽसि स॒हस्रा॑य त्वे॒मा मे॑ अग्न॒ इष्ट॑का धे॒नवः॑ स॒न्त्वेका॑ च श॒तं च॑ स॒हस्रं॑ चा॒युतं॑ च - [  ] \newline

\textbf{Pada Paata} \newline

आ॒दि॒त्याना᳚म् । ते । ते॒ । अधि॑पतय॒ इत्यधि॑-प॒त॒यः॒ । तेभ्यः॑ । नमः॑ । ते । नः॒ । मृ॒ड॒य॒न्तु॒ । ते । यम् । द्वि॒ष्मः । यः । च॒ । नः॒ । द्वेष्टि॑ । तम् । वः॒ । जंभे᳚ । द॒धा॒मि॒ । स॒हस्र॑स्य । प्र॒मेति॑ प्र - मा । अ॒सि॒ । स॒हस्र॑स्य । प्र॒ति॒मेति॑ प्रति - मा । अ॒सि॒ । स॒हस्र॑स्य । वि॒मेति॑ वि - मा । अ॒सि॒ । स॒हस्र॑स्य । उ॒न्मेत्यु॑त् - मा । अ॒सि॒ । सा॒ह॒स्रः । अ॒सि॒ । स॒हस्रा॑य । त्वा॒ । इ॒माः । मे॒ । अ॒ग्ने॒ । इष्ट॑काः । धे॒नवः॑ । स॒न्तु॒ । एका᳚ । च॒ । श॒तम् । च॒ । स॒हस्र᳚म् । च॒ । अ॒युत᳚म् । च॒ ।  \newline




\markright{ TS 4.4.11.4  \hfill https://www.vedavms.in \hfill}

\section{ TS 4.4.11.4 }

\textbf{TS 4.4.11.4 } \newline
\textbf{Samhita Paata} \newline

नि॒युतं॑ च प्र॒युतं॒ चार्बु॑दं च॒ न्य॑र्बुदं च समु॒द्रश्च॒ मद्ध्यं॒ चान्त॑श्च परा॒र्द्धश्चे॒मा मे॑ अग्न॒ इष्ट॑का धे॒नवः॑ सन्तु ष॒ष्ठिः स॒हस्र॑म॒युत॒-मक्षी॑यमाणा ऋत॒स्था स्थ॑र्ता॒वृधो॑ घृत॒श्चुतो॑ मधु॒श्चुत॒ ऊर्ज॑स्वतीः स्वधा॒विनी॒स्ता मे॑ अग्न॒ इष्ट॑का धे॒नवः॑ सन्तु वि॒राजो॒ नाम॑ काम॒दुघा॑ अ॒मुत्रा॒मुष्मि॑न् ॅलो॒के ॥ \newline

\textbf{Pada Paata} \newline

नि॒युत॒मिति॑ नि - युत᳚म् । च॒ । प्र॒युत॒मिति॑ प्र-युत᳚म् । च॒ । अर्बु॑दम् । च॒ । न्य॑र्बुद॒मिति॒ नि - अ॒र्बु॒द॒म् । च॒ । स॒मु॒द्रः । च॒ । मद्ध्य᳚म् । च॒ । अन्तः॑ । च॒ । प॒रा॒द्‌र्ध इति॑ पर- अ॒द्‌र्धः । च॒ । इ॒माः । मे॒ । अ॒ग्ने॒ । इष्ट॑काः । धे॒नवः॑ । स॒न्तु॒ । ष॒ष्टिः । स॒हस्र᳚म् । अ॒युत᳚म् । अक्षी॑यमाणाः । ऋ॒त॒स्था इत्यृ॑त - स्थाः । स्थ॒ । ऋ॒ता॒वृध॒ इत्यृ॑त - वृधः॑ । घृ॒त॒श्चुत॒ इति॑ घृत - श्चुतः॑ । म॒धु॒श्चुत॒ इति॑ मधु - श्चुतः॑ । ऊर्ज॑स्वतीः । स्व॒धा॒विनी॒रिति॑ स्वधा - विनीः᳚ । ताः । मे॒ । अ॒ग्ने॒ । इष्ट॑काः । धे॒नवः॑ । स॒न्तु॒ । वि॒राज॒ इति॑ वि - राजः॑ । नाम॑ । का॒म॒दुघा॒ इति॑ काम - दुघाः᳚ । अ॒मुत्र॑ । अ॒मुष्मिन्न्॑ । लो॒के ॥  \newline




\markright{ TS 4.4.12.1  \hfill https://www.vedavms.in \hfill}

\section{ TS 4.4.12.1 }

\textbf{TS 4.4.12.1 } \newline
\textbf{Samhita Paata} \newline

स॒मिद्-दि॒शामा॒शया॑ नः सुव॒र्विन्मधो॒रतो॒ माध॑वः पात्व॒स्मान् । अ॒ग्निर्दे॒वो दु॒ष्टरी॑तु॒रदा᳚भ्य इ॒दं क्ष॒त्रꣳ र॑क्षतु॒ पात्व॒स्मान् ॥ र॒थ॒न्त॒रꣳ साम॑भिः पात्व॒स्मान् गा॑य॒त्री छन्द॑सां ॅवि॒श्वरू॑पा । त्रि॒वृन्नो॑ वि॒ष्ठया॒ स्तोमो॒ अह्नाꣳ॑ समु॒द्रो वात॑ इ॒दमोजः॑ पिपर्तु ॥ उ॒ग्रा दि॒शाम॒भि-भू॑तिर्वयो॒धाः शुचिः॑ शु॒क्रे अह॑न्योज॒सीना᳚ ।इन्द्राधि॑पतिः पिपृता॒दतो॑ नो॒ महि॑ - [  ] \newline

\textbf{Pada Paata} \newline

स॒मिदिति॑ सम् - इत् । दि॒शाम् । आ॒शया᳚ । नः॒ । सु॒व॒र्विदिति॑ सुवः - वित् । मधोः᳚ । अतः॑ । माध॑वः । पा॒तु॒ । अ॒स्मान् ॥ अ॒ग्निः । दे॒वः । दु॒ष्टरी॑तुः । अदा᳚भ्यः । इ॒दम् । क्ष॒त्रम् । र॒क्ष॒तु॒ । पातु॑ । अ॒स्मान् ॥ र॒थ॒न्त॒रमिति॑ रथं - त॒रम् । साम॑भि॒रिति॒ साम॑ - भिः॒ । पा॒तु॒ । अ॒स्मान् । गा॒य॒त्री । छन्द॑साम् । वि॒श्वरू॒पेति॑ वि॒श्व - रू॒पा॒ ॥ त्रि॒वृदिति॑ त्रि-वृत् । नः॒ । वि॒ष्ठयेति॑ वि - स्थया᳚ । स्तोमः॑ । अह्ना᳚म् । स॒मु॒द्रः । वातः॑ । इ॒दम् । ओजः॑ । पि॒प॒र्तु॒ ॥ उ॒ग्रा । दि॒शाम् । अ॒भिभू॑ति॒रित्य॒भि - भू॒तिः॒ । व॒यो॒धा इति॑ वयः- धाः । शुचिः॑ । शु॒क्रे । अह॑नि । ओ॒ज॒सीना᳚ ॥ इन्द्र॑ । अधि॑पति॒रित्यधि॑ - प॒तिः॒ । पि॒पृ॒ता॒त् । अतः॑ । नः॒ । महि॑ ।  \newline




\markright{ TS 4.4.12.2  \hfill https://www.vedavms.in \hfill}

\section{ TS 4.4.12.2 }

\textbf{TS 4.4.12.2 } \newline
\textbf{Samhita Paata} \newline

क्ष॒त्रं ॅवि॒श्वतो॑ धारये॒दं ॥ बृ॒हत् साम॑ क्षत्र॒भृद्-वृ॒द्ध वृ॑ष्णियं त्रि॒ष्टुभौजः॑ शुभि॒त मु॒ग्रवी॑रं । इन्द्र॒ स्तोमे॑न पञ्चद॒शेन॒ मद्ध्य॑मि॒दं ॅवाते॑न॒ सग॑रेण रक्ष ॥ प्राची॑ दि॒शाꣳ स॒हय॑शा॒ यश॑स्वती॒ विश्वे॑ देवाः प्रा॒वृषा ऽह्नाꣳ॒॒ सुव॑र्वती । इ॒दं क्ष॒त्रं दु॒ष्टर॑म॒स्त्वोजो ऽना॑धृष्टꣳ सह॒स्रियꣳ॒॒ सह॑स्वत् ॥ वै॒रू॒पे साम॑न्नि॒ह तच्छ॑केम॒ जग॑त्यैनं ॅवि॒क्ष्वा वे॑शयामः । विश्वे॑ देवाः सप्तद॒शेन॒ - [  ] \newline

\textbf{Pada Paata} \newline

क्ष॒त्रम् । वि॒श्वतः॑ । धा॒र॒य॒ । इ॒दम् ॥ बृ॒हत् । साम॑ । क्ष॒त्र॒भृदिति॑ क्षत्र - भृत् । वृ॒द्धवृ॑ष्णिय॒मिति॑ वृ॒द्ध-वृ॒ष्णि॒य॒म् । त्रि॒ष्टुभा᳚ । ओजः॑ । शु॒भि॒तम् । उ॒ग्रवी॑र॒मित्यु॒ग्र - वी॒र॒म् ॥ इन्द्र॑ । स्तोमे॑न । प॒ञ्च॒द॒शेनेति॑ पञ्च - द॒शेन॑ । मद्ध्य᳚म् । इ॒दम् । वाते॑न । सग॑रेण । र॒क्ष॒ ॥ प्राची᳚ । दि॒शाम् । स॒हय॑शा॒ इति॑ स॒ह - य॒शाः॒ । यश॑स्वती । विश्वे᳚ । दे॒वाः॒ । प्रा॒वृषा᳚ । अह्ना᳚म् । सुव॑र्व॒तीति॒ सुवः॑ - व॒ती॒ ॥ इ॒दम् । क्ष॒त्रम् । दु॒ष्टर᳚म् । अ॒स्तु॒ । ओजः॑ । अना॑धृष्ट॒मित्यना᳚ - धृ॒ष्ट॒म् । स॒ह॒स्रिय᳚म् । सह॑स्वत् ॥ वै॒रू॒पे । सामन्न्॑ । इ॒ह । तत् । श॒के॒म॒ । जग॑त्या । ए॒न॒म् । वि॒क्षु । एति॑ । वे॒श॒या॒मः॒ ॥ विश्वे᳚ । दे॒वाः॒ । स॒प्त॒द॒शेनेति॑ सप्त - द॒शेन॑ ।  \newline




\markright{ TS 4.4.12.3  \hfill https://www.vedavms.in \hfill}

\section{ TS 4.4.12.3 }

\textbf{TS 4.4.12.3 } \newline
\textbf{Samhita Paata} \newline

वर्च॑ इ॒दं क्ष॒त्रꣳ स॑लि॒लवा॑तमु॒ग्रं ॥ ध॒र्त्री दि॒शां क्ष॒त्रमि॒दं दा॑धारोप॒स्थाऽऽशा॑नां मि॒त्रव॑द॒स्त्वोजः॑ । मित्रा॑वरुणा श॒रदाऽह्नां᳚ चिकित्नू अ॒स्मै रा॒ष्ट्राय॒ महि॒ शर्म॑ यच्छतं ॥ वै॒रा॒जे साम॒न्नधि॑ मे मनी॒षाऽनु॒ष्टुभा॒ संभृ॑तं ॅवी॒र्यꣳ॑ सहः॑ । इ॒दं क्ष॒त्रं मि॒त्रव॑दा॒र्द्रदा॑नु॒ मित्रा॑वरुणा॒ रक्ष॑त॒-माधि॑पत्यैः ॥ स॒म्राड् दि॒शाꣳ स॒हसा᳚म्नी॒ सह॑स्वत्यृ॒तुर्.हे॑म॒न्तो वि॒ष्ठया॑ नः पिपर्तु । अ॒व॒स्युवा॑ता - [  ] \newline

\textbf{Pada Paata} \newline

वर्चः॑ । इ॒दम् । क्ष॒त्रम् । स॒लि॒लवा॑त॒मिति॑ सलि॒ल - वा॒त॒म् । उ॒ग्रम् ॥ ध॒र्त्री । दि॒शाम् । क्ष॒त्रम् । इ॒दम् । दा॒धा॒र॒ । उ॒प॒स्थेत्यु॑प - स्था । आशा॑नाम् । मि॒त्रव॒दिति॑ मि॒त्र - व॒त् । अ॒स्तु॒ । ओजः॑ ॥ मित्रा॑वरु॒णेति॒ मित्रा᳚ - व॒रु॒णा॒ । श॒रदा᳚ । अह्ना᳚म् । चि॒कि॒त्नू॒ इति॑ । अ॒स्मै । रा॒ष्ट्राय॑ । महि॑ । शर्म॑ । य॒च्छ॒त॒म् ॥ वै॒रा॒जे । सामन्न्॑ । अधीति॑ । मे॒ । म॒नी॒षा । अ॒नु॒ष्टुभेत्य॑नु - स्तुभा᳚ । संभृ॑त॒मिति॑ सं - भृ॒त॒म् । वी॒र्य᳚म् । सहः॑ ॥ इ॒दम् । क्ष॒त्रम् । मि॒त्रव॒दिति॑ मि॒त्र - व॒त् । आ॒र्द्रदा॒न्वित्या॒र्द्र - दा॒नु॒ । मित्रा॑वरु॒णेति॒ मित्रा᳚ - व॒रु॒णा॒ । रक्ष॑तम् । आधि॑पत्यै॒रित्याधि॑-प॒त्यैः॒ ॥ स॒म्राडिति॑ सम् - राट् । दि॒शाम् । स॒हसा॒म्नीति॑ स॒ह - सा॒म्नी॒ । सह॑स्वती । ऋ॒तुः । हे॒म॒न्तः । वि॒ष्ठयेति॑ वि - स्थया᳚ । नः॒ । पि॒प॒र्तु॒ ॥ अ॒व॒स्युवा॑ता॒ इत्य॑व॒स्यु - वा॒ताः॒ ।  \newline




\markright{ TS 4.4.12.4  \hfill https://www.vedavms.in \hfill}

\section{ TS 4.4.12.4 }

\textbf{TS 4.4.12.4 } \newline
\textbf{Samhita Paata} \newline

बृह॒तीर्नु शक्व॑रीरि॒मं ॅय॒ज्ञ्म॑वन्तु नो घृ॒ताचीः᳚ ॥ सुव॑र्वती सु॒दुघा॑ नः॒ पय॑स्वती दि॒शां दे॒व्य॑वतु नो घृ॒ताची᳚ । त्वं गो॒पाः पु॑रए॒तोत प॒श्चाद् बृह॑स्पते॒ याम्यां᳚ ॅयुङ्ग्धि॒ वाचं᳚ ॥ऊ॒र्द्ध्वा दि॒शाꣳ रन्ति॒राशौष॑धीनाꣳ संॅवथ्स॒रेण॑ सवि॒ता नो॒ अह्नां᳚ । रे॒वथ् सामाति॑च्छन्दा उ॒ छन्दोऽजा॑त शत्रुः स्यो॒ना नो॑ अस्तु ॥ स्तोम॑त्रयस्त्रिꣳशे॒ भुव॑नस्य पत्नि॒ विव॑स्वद्वाते अ॒भि नो॑ - [  ] \newline

\textbf{Pada Paata} \newline

बृ॒ह॒तीः । नु । शक्व॑रीः । इ॒मम् । य॒ज्ञ्म् । अ॒व॒न्तु॒ । नः॒ । घृ॒ताचीः᳚ । सुव॑र्व॒तीति॒ सुवः॑ - व॒ती॒ । सु॒दुघेति॑ सु - दुघा᳚ । नः॒ । पय॑स्वती । दि॒शाम् । दे॒वी । अ॒व॒तु॒ । नः॒ । घृ॒ताची᳚ ॥ त्वम् । गो॒पा इति॑ गो - पाः । पु॒र॒ ए॒तेति॑ पुरः - ए॒ता । उ॒त । प॒श्चात् । बृह॑स्पते । याम्या᳚म् । यु॒ङ्ग्धि॒ । वाच᳚म् ॥ ऊ॒द्‌र्ध्वा । दि॒शाम् । रन्तिः॑ । आशा᳚ । ओष॑धीनाम् । सं॒ॅव॒थ्स॒रेणेति॑ सं-व॒थ्स॒रेण॑ । स॒वि॒ता । नः॒ । अह्ना᳚म् ॥ रे॒वत् । साम॑ । अति॑॑च्छन्दा॒ इत्याति॑ - छ॒न्दाः॒ । उ॒ । छन्दः॑ । अजा॑तशत्रु॒रित्यजा॑त - श॒त्रुः॒ । स्यो॒ना । नः॒ । अ॒स्तु॒ ॥ स्तोम॑त्रयस्त्रिꣳश॒ इति॒ स्तोम॑ - त्र॒य॒स्त्रिꣳ॒॒शे॒ । भुव॑नस्य । प॒त्नि॒ । विव॑स्वद्वात॒ इति॒ विव॑स्वत् - वा॒ते॒ । अ॒भीति॑ । नः॒ ।  \newline




\markright{ TS 4.4.12.5  \hfill https://www.vedavms.in \hfill}

\section{ TS 4.4.12.5 }

\textbf{TS 4.4.12.5 } \newline
\textbf{Samhita Paata} \newline

गृणाहि । घृ॒तव॑ती सवित॒राधि॑पत्यैः॒ पय॑स्वती॒ रन्ति॒राशा॑ नो अस्तु ॥ ध्रु॒वा दि॒शां ॅविष्णु॑प॒त्न्यघो॑रा॒ऽस्येशा॑ना॒ सह॑सो॒ या म॒नोता᳚ । बृह॒स्पति॑ र्मात॒रिश्वो॒त वा॒युः स॑न्धुवा॒ना वाता॑ अ॒भि नो॑ गृणन्तु ॥ वि॒ष्ट॒भ्ॐ दि॒वो ध॒रुणः॑ पृथि॒व्या अ॒स्येशा॑ना॒ जग॑तो॒ विष्णु॑पत्नी । वि॒श्वव्य॑चा इ॒षय॑न्ती॒ सुभू॑तिः शि॒वा नो॑ अ॒स्त्वदि॑तिरु॒पस्थे᳚ ॥ वै॒श्वा॒न॒रो न॑ ऊ॒त्या>3, पृ॒ष्टो दि॒व्य>4, नु॑ नो॒ ( ) ऽद्यानु॑मति॒>5, रन्विद॑नुमते॒ त्वं >6, कया॑ नश्चि॒त्र आभु॑व॒त्>7, को अ॒द्य यु॑ङ्क्ते >8 ॥ \newline

\textbf{Pada Paata} \newline

गृ॒णा॒हि॒ ॥ घृ॒तव॒तीति॑ घृ॒त - व॒ती॒ । स॒वि॒तः॒ । आधि॑पत्यै॒रित्याधि॑-प॒त्यैः॒ । पय॑स्वती । रन्तिः॑ । आशा᳚ । नः॒ । अ॒स्तु॒ ॥ ध्रु॒वा । दि॒शाम् । विष्णु॑प॒त्नीति॒ विष्णु॑ - प॒त्नी॒ । अघो॑रा । अ॒स्य । ईशा॑ना । सह॑सः । या । म॒नोता᳚ ॥ बृह॒स्पतिः॑ । मा॒त॒रिश्वा᳚ । उ॒त । वा॒युः । स॒न्ध॒वा॒ना इति॑ सम् - धु॒वा॒नाः । वाताः᳚ । अ॒भीति॑ । नः॒ । गृ॒ण॒न्तु॒ ॥ वि॒ष्ट॒भं इति॑ वि-स्त॒भंः । दि॒वः । ध॒रुणः॑ । पृ॒थि॒व्याः । अ॒स्य । ईशा॑ना । जग॑तः । विष्णु॑प॒त्नीति॒ विष्णु॑ - प॒त्नी॒ ॥ वि॒श्वव्य॑चा॒ इति॑ वि॒श्व - व्य॒चाः॒ । इ॒षय॑न्ती । सुभू॑ति॒रिति॒ सु-भू॒तिः॒ । शि॒वा । नः॒ । अ॒स्तु॒ । अदि॑तिः । उ॒पस्थ॒ इत्यु॒प - स्थे॒ ॥ वै॒श्वा॒न॒रः । नः॒ । ऊ॒त्या । पृ॒ष्टः । दि॒वि । अन्विति॑ । नः॒ ( ) । अ॒द्य । अनु॑मति॒रित्यनु॑ - म॒तिः॒ । अन्विति॑ । इत् । अ॒नु॒म॒त॒ इत्य॑नु - म॒ते॒ । त्वम् । कया᳚ । नः॒ । चि॒त्रः । एति॑ । भु॒व॒त् । कः । अ॒द्य । यु॒ङ्क्ते॒ ॥  \newline






\end{document}