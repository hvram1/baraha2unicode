\documentclass[17pt]{extarticle}
\usepackage{babel}
\usepackage{fontspec}
\usepackage{polyglossia}
\usepackage{extsizes}

\usepackage{color}   %May be necessary if you want to color links
\usepackage{hyperref}
\hypersetup{
    colorlinks=true, %set true if you want colored links
    linktoc=all,     %set to all if you want both sections and subsections linked
    linkcolor=black,  %choose some color if you want links to stand out
}

\setmainlanguage{sanskrit}
\setotherlanguages{english} %% or other languages
\setlength{\parindent}{0pt}
\pagestyle{myheadings}
\newfontfamily\devanagarifont[Script=Devanagari]{AdishilaVedic}
\renewcommand{\theHsection}{\thepart.section.\thesection}

\newcommand{\VAR}[1]{}
\newcommand{\BLOCK}[1]{}




\begin{document}
\begin{titlepage}
    \begin{center}
 
\begin{sanskrit}
    { \Large
    कृष्ण यजुर्वेदीय तैत्तिरीय संहिता,पद,जटा,घन पाठः 
    }
    \\
    \vspace{2.5cm}
    \mbox{ \Large
    7.2      सप्तमकाण्डे द्वितीयः प्रश्नः - षड् रात्राद्यानां निरूपणं   }
\end{sanskrit}
\end{center}

\end{titlepage}
\tableofcontents
\phantomsection
\pagebreak

\markright{ TS 7.2.1.1  \hfill https://www.vedavms.in \hfill}

\section{ TS 7.2.1.1 }

\textbf{TS 7.2.1.1 } \newline
\textbf{Samhita Paata} \newline

सा॒द्ध्या वै दे॒वाः सु॑व॒र्गका॑मा ए॒तꣳ ष॑ड्-रा॒त्रम॑पश्य॒न् तमाऽह॑र॒न् तेना॑यजन्त॒ ततो॒ वै ते सु॑व॒र्गं ॅलो॒कमा॑य॒न॒. य ए॒वं ॅवि॒द्वाꣳसः॑ षड्-रा॒त्रमास॑ते सुव॒र्गमे॒व लो॒कं ॅय॑न्ति देवस॒त्रं ॅवै ष॑ड्-रा॒त्रः प्र॒त्यक्षꣳ॒॒ ह्ये॑तानि॑ पृ॒ष्ठानि॒ य ए॒वं ॅवि॒द्वाꣳसः॑ षड्-रा॒त्रमास॑ते सा॒क्षादे॒व दे॒वता॑ अ॒भ्यारो॑हन्ति॒ षड्-रा॒त्रो भ॑वति॒ षड् वा ऋ॒तवः॒ षट् पृ॒ष्ठानि॑ - [  ] \newline

\textbf{Pada Paata} \newline

सा॒द्ध्याः । वै । दे॒वाः । सु॒व॒र्गका॑मा॒ इति॑ सुव॒र्ग - का॒माः॒ । ए॒तम् । ष॒ड्रा॒त्रमिति॑ षट् - रा॒त्रम् । अ॒प॒श्य॒न्न् । तम् । एति॑ । अ॒ह॒र॒न्न् । तेन॑ । अ॒य॒ज॒न्त॒ । ततः॑ । वै । ते । सु॒व॒र्गमिति॑ सुवः-गम् । लो॒कम् । आ॒य॒न्न् । ये । ए॒वम् । वि॒द्वाꣳसः॑ । ष॒ड्रा॒त्रमिति॑ षट् - रा॒त्रम् । आस॑ते । सु॒व॒र्गमिति॑ सुवः - गम् । ए॒व । लो॒कम् । य॒न्ति॒ । दे॒व॒स॒त्रमिति॑ देव - स॒त्रम् । वै । ष॒ड्रा॒त्र इति॑ षट् - रा॒त्रः । प्र॒त्यक्ष॒मिति॑ प्रति - अक्ष᳚म् । हि । ए॒तानि॑ । पृ॒ष्ठानि॑ । ये । ए॒वम् । वि॒द्वाꣳसः॑ । ष॒ड्रा॒त्रमिति॑ षट् - रा॒त्रम् । आस॑ते । सा॒क्षादिति॑ स - अ॒क्षात् । ए॒व । दे॒वताः᳚ । अ॒भ्यारो॑ह॒न्तीत्य॑भि - आरो॑हन्ति । ष॒ड्रा॒त्र इति॑ षट् - रा॒त्रः । भ॒व॒ति॒ । षट् । वै । ऋ॒तवः॑ । षट् । पृ॒ष्ठानि॑ ।  \newline




\markright{ TS 7.2.1.2  \hfill https://www.vedavms.in \hfill}

\section{ TS 7.2.1.2 }

\textbf{TS 7.2.1.2 } \newline
\textbf{Samhita Paata} \newline

पृ॒ष्ठैरे॒वर्तून॒-न्वारो॑हन्त्यृ॒तुभिः॑ संॅवथ्स॒रं ते सं॑ॅवथ्स॒र ए॒व प्रति॑ तिष्ठन्ति बृहद्-रथन्त॒राभ्यां᳚ ॅयन्ती॒यं ॅवाव र॑थंत॒रम॒सौ बृ॒हदा॒भ्यामे॒व य॒न्त्यथो॑ अ॒नयो॑रे॒व प्रति॑ तिष्ठन्त्ये॒ते वै य॒ज्ञ्स्या᳚ञ्ज॒साय॑नी स्रु॒ती ताभ्या॑मे॒व सु॑व॒र्गं ॅलो॒कं ॅय॑न्ति त्रि॒वृद॑ग्निष्टो॒मो भ॑वति॒ तेज॑ ए॒वाव॑ रुन्धते पञ्चद॒शो भ॑वतीन्द्रि॒यमे॒वाव॑ रुन्धते सप्तद॒शो- [  ] \newline

\textbf{Pada Paata} \newline

पृ॒ष्ठैः । ए॒व । ऋ॒तून् । अ॒न्वारो॑ह॒न्तीत्य॑नु - आरो॑हन्ति । ऋ॒तुभि॒रित्यृ॒तु -भिः॒ । सं॒ॅव॒थ्स॒रमिति॑ सं - व॒थ्स॒रम् । ते । सं॒ॅव॒थ्स॒र इति॑ सं - व॒थ्स॒रे । ए॒व । प्रतीति॑ । ति॒ष्ठ॒न्ति॒ । बृ॒ह॒द्र॒थ॒न्त॒राभ्या॒मिति॑ बृहत् - र॒थ॒न्त॒राभ्या᳚म् । य॒न्ति॒ । इ॒यम् । वाव । र॒थ॒न्त॒रमिति॑ रथं - त॒रम् । अ॒सौ । बृ॒हत् । आ॒भ्याम् । ए॒व । य॒न्ति॒ । अथो॒ इति॑ । अ॒नयोः᳚ । ए॒व । प्रतीति॑ । ति॒ष्ठ॒न्ति॒ । ए॒ते इति॑ । वै । य॒ज्ञ्स्य॑ । अ॒ञ्ज॒साय॑नी॒ इत्य॑ञ्जसा - अय॑नी । स्रु॒ती इति॑ । ताभ्या᳚म् । ए॒व । सु॒व॒र्गमिति॑ सुवः - गम् । लो॒कम् । य॒न्ति॒ । त्रि॒वृदिति॑ त्रि - वृत् । अ॒ग्नि॒ष्टो॒म इत्य॑ग्नि - स्तो॒मः । भ॒व॒ति॒ । तेजः॑ । ए॒व । अवेति॑ । रु॒न्ध॒ते॒ । प॒ञ्च॒द॒श इति॑ पञ्च - द॒शः । भ॒व॒ति॒ । इ॒न्द्रि॒यम् । ए॒व । अवेति॑ । रु॒न्ध॒ते॒ । स॒प्त॒द॒श इति॑ सप्त - द॒शः ।  \newline




\markright{ TS 7.2.1.3  \hfill https://www.vedavms.in \hfill}

\section{ TS 7.2.1.3 }

\textbf{TS 7.2.1.3 } \newline
\textbf{Samhita Paata} \newline

भ॑व-त्य॒न्नाद्य॒स्या-व॑रुद्ध्या॒ अथो॒ प्रैव तेन॑ जायन्त एकविꣳ॒॒शो भ॑वति॒ प्रति॑ष्ठित्या॒ अथो॒ रुच॑मे॒वाऽऽत्मन् द॑धते त्रिण॒वो भ॑वति॒ विजि॑त्यै त्रयस्त्रिꣳ॒॒शो भ॑वति॒ प्रति॑ष्ठित्यै सदोहविर्द्धा॒निन॑ ए॒तेन॑ षड्-रा॒त्रेण॑ यजेर॒न्नाश्व॑त्थी हवि॒र्द्धानं॒ चाऽऽ*ग्नी᳚द्ध्रं च भवत॒स्तद्धि सु॑व॒र्ग्यं॑ च॒क्रीव॑ती भवतः सुव॒र्गस्य॑ लो॒कस्य॒ सम॑ष्ट्या उ॒लूख॑लबुद्ध्नो॒ यूपो॑ भवति॒ प्रति॑ष्ठित्यै॒ प्राञ्चो॑ यान्ति॒ प्राङि॑व॒ हि सु॑व॒र्गो - [  ] \newline

\textbf{Pada Paata} \newline

भ॒व॒ति॒ । अ॒न्नाद्य॒स्येत्य॑न्न - अद्य॑स्य । अव॑रुद्ध्या॒ इत्यव॑ - रु॒द्ध्यै॒ । अथो॒ इति॑ । प्रेति॑ । ए॒व । तेन॑ । जा॒य॒न्ते॒ । ए॒क॒विꣳ॒॒श इत्ये॑क - विꣳ॒॒शः । भ॒व॒ति॒ । प्रति॑ष्ठित्या॒ इति॒ प्रति॑ - स्थि॒त्यै॒ । अथो॒ इति॑ । रुच᳚म् । ए॒व । आ॒त्मन्न् । द॒ध॒ते॒ । त्रि॒ण॒व इति॑ त्रि - न॒वः । भ॒व॒ति॒ । विजि॑त्या॒ इति॒ वि - जि॒त्यै॒ । त्र॒य॒स्त्रिꣳ॒॒श इति॑ त्रयः - त्रिꣳ॒॒शः । भ॒व॒ति॒ । प्रति॑ष्ठित्या॒ इति॒ प्रति॑ - स्थि॒त्यै॒ । स॒दो॒ह॒वि॒द्‌र्धा॒निन॒ इति॑ सदः - ह॒वि॒द्‌र्धा॒निनः॑ । ए॒तेन॑ । ष॒ड्रा॒त्रेणेति॑ षट् - रा॒त्रेण॑ । य॒जे॒र॒न्न् । आश्व॑त्थी॒ इति॑ । ह॒वि॒द्‌र्धान॒मिति॑ हविः - धान᳚म् । च॒ । आग्नी᳚द्ध्र॒मित्याग्नि॑-इ॒द्ध्र॒म् । च॒ । भ॒व॒तः॒ । तत् । हि । सु॒व॒र्ग्य॑मिति॑ सुवः - ग्य᳚म् । च॒क्रीव॑ती॒ इति॑ । भ॒व॒तः॒ । सु॒व॒र्गस्येति॑ सुवः - गस्य॑ । लो॒कस्य॑ । सम॑ष्ट्या॒ इति॒ सं-अ॒ष्ट्यै॒ । उ॒लूख॑लबुद्ध्न॒ इत्यु॒लूख॑ल - बु॒द्ध्नः॒ । यूपः॑ । भ॒व॒ति॒ । प्रति॑ष्ठित्या॒ इति॒ प्रति॑ - स्थि॒त्यै॒ । प्राञ्चः॑ । या॒न्ति॒ । प्राङ् । इ॒व॒ । हि । सु॒व॒र्ग इति॑ सुवः - गः ।  \newline




\markright{ TS 7.2.1.4  \hfill https://www.vedavms.in \hfill}

\section{ TS 7.2.1.4 }

\textbf{TS 7.2.1.4 } \newline
\textbf{Samhita Paata} \newline

लो॒कः सर॑स्वत्या यान्त्ये॒ष वै दे॑व॒यानः॒ पन्था॒स्त-मे॒वा-न्वारो॑हन्त्या॒क्रोश॑न्तो या॒न्त्यव॑र्ति-मे॒वान्यस्मि॑न् प्रति॒षज्य॑ प्रति॒ष्ठां ग॑च्छन्ति य॒दा दश॑ श॒तं कु॒र्वन्त्यथैक॑-मु॒त्थानꣳ॑ श॒तायुः॒ पुरु॑षः श॒तेन्द्रि॑य॒ आयु॑ष्ये॒वेन्द्रि॒ये प्रति॑ तिष्ठन्ति य॒दा श॒तꣳ स॒हस्रं॑ कु॒र्वन्त्यथैक॑-मु॒त्थानꣳ॑ स॒हस्र॑संमितो॒ वा अ॒सौ लो॒को॑ऽमुमे॒व लो॒कम॒भि ज॑यन्ति य॒दै ( ) -षां᳚ प्र॒मीये॑त य॒दा वा॒ जीये॑र॒न्नथैक॑-मु॒त्थानं॒ तद्धि ती॒र्थं ॥ \newline

\textbf{Pada Paata} \newline

लो॒कः । सर॑स्वत्या । या॒न्ति॒ । ए॒षः । वै । दे॒व॒यान॒ इति॑ देव-यानः॑ । पन्थाः᳚ । तम् । ए॒व । अ॒न्वारो॑ह॒न्तीत्य॑नु - आरो॑हन्ति । आ॒क्रोश॑न्त॒ इत्या᳚ - क्रोश॑न्तः । या॒न्ति॒ । अव॑र्तिम् । ए॒व । अ॒न्यस्मिन्न्॑ । प्र॒ति॒षज्येति॑ प्रति - सज्य॑ । प्र॒ति॒ष्ठामिति॑ प्रति-स्थाम् । ग॒च्छ॒न्ति॒ । य॒दा । दश॑ । श॒तम् । कु॒र्वन्ति॑ । अथ॑ । एक᳚म् । उ॒त्थान॒मित्यु॑त्-स्थान᳚म् । श॒तायु॒रिति॑ श॒त - आ॒युः॒ । पुरु॑षः । श॒तेन्द्रि॑य॒ इति॑ श॒त-इ॒न्द्रि॒यः॒ । आयु॑षि । ए॒व । इ॒न्द्रि॒ये । प्रतीति॑ । ति॒ष्ठ॒न्ति॒ । य॒दा । श॒तम् । स॒हस्र᳚म् । कु॒र्वन्ति॑ । अथ॑ । एक᳚म् । उ॒त्थान॒मित्यु॑त्- स्थान᳚म् । स॒हस्र॑संमित॒ इति॑ स॒हस्र॑-स॒म्मि॒तः॒ । वै । अ॒सौ । लो॒कः । अ॒मुम् । ए॒व । लो॒कम् । अ॒भीति॑ । ज॒य॒न्ति॒ । य॒दा ( ) । ए॒षा॒म् । प्र॒मीये॒तेति॑ प्र - मीये॑त । य॒दा । वा॒ । जीये॑रन्न् । अथ॑ । एक᳚म् । उ॒त्थान॒मित्यु॑त् - स्थान᳚म् । तत् । हि । ती॒र्थम् ॥  \newline




\markright{ TS 7.2.2.1  \hfill https://www.vedavms.in \hfill}

\section{ TS 7.2.2.1 }

\textbf{TS 7.2.2.1 } \newline
\textbf{Samhita Paata} \newline

कु॒सु॒रु॒बिन्द॒ औद्दा॑लकिरकामयत पशु॒मान्थ् स्या॒मिति॒ स ए॒तꣳ स॑प्तरा॒त्रमाऽह॑र॒त् तेना॑यजत॒ तेन॒ वै स याव॑न्तो ग्रा॒म्याः प॒शव॒स्तानवा॑-रुन्ध॒ य ए॒वं ॅवि॒द्वान्थ् स॑प्तरा॒त्रेण॒ यज॑ते॒ याव॑न्त ए॒व ग्रा॒म्याः प॒शव॒स्ताने॒वाव॑ रुन्धे सप्तरा॒त्रो भ॑वति स॒प्त ग्रा॒म्याः प॒शवः॑ स॒प्ताऽऽ*र॒ण्याः स॒प्त छन्दाꣳ॑-स्यु॒भय॒स्या-व॑रुद्ध्यै त्रि॒वृद॑ग्निष्टो॒मो भ॑वति॒ तेज॑ - [  ] \newline

\textbf{Pada Paata} \newline

कु॒सु॒रु॒बिन्दः॑ । औद्दा॑लकि॒रित्यौत् - दा॒ल॒किः॒ । अ॒का॒म॒य॒त॒ । प॒शु॒मानिति॑ पशु - मान् । स्या॒म् । इति॑ । सः । ए॒तम् । स॒प्त॒रा॒त्रमिति॑ सप्त-रा॒त्रम् । एति॑ । अ॒ह॒र॒त् । तेन॑ । अ॒य॒ज॒त॒ । तेन॑ । वै । सः । याव॑न्तः । ग्रा॒म्याः । प॒शवः॑ । तान् । अवेति॑ । अ॒रु॒न्ध॒ । यः । ए॒वम् । वि॒द्वान् । स॒प्त॒रा॒त्रेणेति॑ सप्त - रा॒त्रेण॑ । यज॑ते । याव॑न्तः । ए॒व । ग्रा॒म्याः । प॒शवः॑ । तान् । ए॒व । अवेति॑ । रु॒न्धे॒ । स॒प्त॒रा॒त्र इति॑ सप्त-रा॒त्रः । भ॒व॒ति॒ । स॒प्त । ग्रा॒म्याः । प॒शवः॑ । स॒प्त । आ॒र॒ण्याः । स॒प्त । छन्दाꣳ॑सि । उ॒भय॑स्य । अव॑रुद्ध्या॒ इत्यव॑-रु॒द्ध्यै॒ । त्रि॒वृदिति॑ त्रि - वृत् । अ॒ग्नि॒ष्टो॒म इत्य॑ग्नि-स्तो॒मः । भ॒व॒ति॒ । तेजः॑ ।  \newline




\markright{ TS 7.2.2.2  \hfill https://www.vedavms.in \hfill}

\section{ TS 7.2.2.2 }

\textbf{TS 7.2.2.2 } \newline
\textbf{Samhita Paata} \newline

ए॒वाव॑ रुन्धे पञ्चद॒शो भ॑वतीन्द्रि॒यमे॒वाव॑ रुन्धे सप्तद॒शो भ॑वत्य॒न्नाद्य॒स्या व॑रुद्ध्या॒ अथो॒ प्रैव तेन॑ जायत एकविꣳ॒॒शो भ॑वति॒ प्रति॑ष्ठित्या॒ अथो॒ रुच॑मे॒वाऽऽ*त्मन् ध॑त्ते त्रिण॒वो भ॑वति॒ विजि॑त्यै पञ्चविꣳ॒॒शो᳚-ऽग्निष्टो॒मो भ॑वति प्र॒जाप॑ते॒राप्त्यै॑ महाव्र॒तवा॑-न॒न्नाद्य॒स्या व॑रुद्ध्यै विश्व॒जिथ् सर्व॑पृष्ठो ऽतिरा॒त्रो भ॑वति॒ सर्व॑स्या॒भिजि॑त्यै॒ यत् प्र॒त्यक्षं॒ पूर्वे॒ष्वह॑स्सु पृ॒ष्ठान्यु॑पे॒युः प्र॒त्यक्षं॑ -[  ] \newline

\textbf{Pada Paata} \newline

ए॒व । अवेति॑ । रु॒न्धे॒ । प॒ञ्च॒द॒श इति॑ पञ्च - द॒शः । भ॒व॒ति॒ । इ॒न्द्रि॒यम् । ए॒व । अवेति॑ । रु॒न्धे॒ । स॒प्त॒द॒श इति॑ सप्त - द॒शः । भ॒व॒ति॒ । अ॒न्नाद्य॒स्येत्य॑न्न - अद्य॑स्य । अव॑रुद्ध्या॒ इत्यव॑-रु॒द्ध्यै॒ । अथो॒ इति॑ । प्रेति॑ । ए॒व । तेन॑ । जा॒य॒ते॒ । ए॒क॒विꣳ॒॒श इत्ये॑क - विꣳ॒॒शः । भ॒व॒ति॒ । प्रति॑ष्ठित्या॒ इति॒ प्रति॑ - स्थि॒त्यै॒ । अथो॒ इति॑ । रुच᳚म् । ए॒व । आ॒त्मन्न् । ध॒त्ते॒ । त्रि॒ण॒व इति॑ त्रि-न॒वः । भ॒व॒ति॒ । विजि॑त्या॒ इति॒ वि - जि॒त्यै॒ । प॒ञ्च॒विꣳ॒॒श इति॑ पञ्च - विꣳ॒॒शः । अ॒ग्नि॒ष्टो॒म इत्य॑ग्नि - स्तो॒मः । भ॒व॒ति॒ । प्र॒जाप॑ते॒रिति॑ प्र॒जा-प॒तेः॒ । आप्त्यै᳚ । म॒हा॒व्र॒तवा॒निति॑ महाव्र॒त-वा॒न् । अ॒न्नाद्य॒स्येत्य॑न्न-अद्य॑स्य । अव॑रुद्ध्या॒ इत्यव॑ - रु॒द्ध्यै॒ । वि॒श्व॒जिदिति॑ विश्व - जित् । सर्व॑पृष्ठ॒ इति॒ सर्व॑ - पृ॒ष्ठः॒ । अ॒ति॒रा॒त्र इत्य॑ति - रा॒त्रः । भ॒व॒ति॒ । सर्व॑स्य । अ॒भिजि॑त्या॒ इत्य॒भि - जि॒त्यै॒ । यत् । प्र॒त्यक्ष॒मिति॑ प्रति - अक्ष᳚म् । पूर्वे॑षु । अह॒स्स्वित्यहः॑ - सु॒ । पृ॒ष्ठानि॑ । उ॒पे॒युरित्यु॑प - इ॒युः । प्र॒त्यक्ष॒मिति॑ प्रति - अक्ष᳚म् ।  \newline




\markright{ TS 7.2.2.3  \hfill https://www.vedavms.in \hfill}

\section{ TS 7.2.2.3 }

\textbf{TS 7.2.2.3 } \newline
\textbf{Samhita Paata} \newline

ॅविश्व॒जिति॒ यथा॑ दु॒ग्धा-मु॑प॒सीद॑त्ये॒वमु॑त्त॒म-महः॑ स्या॒न्नैक॑रा॒त्रश्च॒न स्या᳚द् बृहद्-रथन्त॒रे पूर्वे॒ष्वह॒स्सूप॑ यन्ती॒यं ॅवाव र॑थन्त॒रम॒सौ बृ॒हदा॒भ्यामे॒व न य॒न्त्यथो॑ अ॒नयो॑रे॒व प्रति॑ तिष्ठन्ति॒ यत् प्र॒त्यक्षं॑ ॅविश्व॒जिति॑ पृ॒ष्ठान्यु॑प॒यन्ति॒ यथा॒ प्रत्तां᳚ दु॒हे ता॒दृगे॒व तत् ॥ \newline

\textbf{Pada Paata} \newline

वि॒श्व॒जितीति॑ विश्व - जिति॑ । यथा᳚ । दु॒ग्धाम् । उ॒प॒सीद॒तीत्यु॑प- सीद॑ति । ए॒वम् । उ॒त्त॒ममित्यु॑त् - त॒मम् । अहः॑ । स्या॒त् । न । ए॒क॒रा॒त्र इत्ये॑क - रा॒त्रः । च॒न । स्या॒त् । बृ॒ह॒द्र॒थ॒न्त॒रे इति॑ बृहत्-र॒थ॒न्त॒रे । पूर्वे॑षु । अह॒स्स्वित्यहः॑ - सु॒ । उपेति॑ । य॒न्ति॒ । इ॒यम् । वाव । र॒थ॒न्त॒रमिति॑ रथं - त॒रम् । अ॒सौ । बृ॒हत् । आ॒भ्याम् । ए॒व । न । य॒न्ति॒ । अथो॒ इति॑ । अ॒नयोः᳚ । ए॒व । प्रतीति॑ । ति॒ष्ठ॒न्ति॒ । यत् । प्र॒त्यक्ष॒मिति॑ प्रति - अक्ष᳚म् । वि॒श्व॒जितीति॑ विश्व - जिति॑ । पृ॒ष्ठानि॑ । उ॒प॒यन्तीत्यु॑प - यन्ति॑ । यथा᳚ । प्रत्ता᳚म् । दु॒हे । ता॒दृक् । ए॒व । तत् ॥  \newline




\markright{ TS 7.2.3.1  \hfill https://www.vedavms.in \hfill}

\section{ TS 7.2.3.1 }

\textbf{TS 7.2.3.1 } \newline
\textbf{Samhita Paata} \newline

बृह॒स्पति॑रकामयत ब्रह्मवर्च॒सी स्या॒मिति॒ स ए॒त-म॑ष्टरा॒त्र-म॑पश्य॒त् तमाऽह॑र॒त् तेना॑यजत॒ ततो॒ वै स ब्र॑ह्मवर्च॒स्य॑भव॒द्य ए॒वं ॅवि॒द्वान॑ष्टरा॒त्रेण॒ यज॑ते ब्रह्मवर्च॒स्ये॑व भ॑वत्यष्टरा॒त्रो भ॑वत्य॒ष्टाक्ष॑रा गाय॒त्री गा॑य॒त्री ब्र॑ह्मवर्च॒सं गा॑यत्रि॒यैव ब्र॑ह्मवर्च॒समव॑ रुन्धेऽष्टरा॒त्रो भ॑वति॒ चत॑स्रो॒ वै दिश॒श्चत॑स्रो ऽवान्तरदि॒शा दि॒ग्भ्य ए॒व ब्र॑ह्मवर्च॒समव॑ रुन्धे- [  ] \newline

\textbf{Pada Paata} \newline

बृह॒स्पतिः॑ । अ॒का॒म॒य॒त॒ । ब्र॒ह्म॒व॒र्च॒सीति॑ ब्रह्म - व॒र्च॒सी । स्या॒म् । इति॑ । सः । ए॒तम् । अ॒ष्ट॒रा॒त्रमित्य॑ष्ट-रा॒त्रम् । अ॒प॒श्य॒त् । तम् । एति॑ । अ॒ह॒र॒त् । तेन॑ । अ॒य॒ज॒त॒ । ततः॑ । वै । सः । ब्र॒ह्म॒व॒र्च॒सीति॑ ब्रह्म-व॒र्च॒सी । अ॒भ॒व॒त् । यः । ए॒वम् । वि॒द्वान् । अ॒ष्ट॒रा॒त्रेणेत्य॑ष्ट - रा॒त्रेण॑ । यज॑ते । ब्र॒ह्म॒व॒र्च॒सीति॑ ब्रह्म - व॒र्च॒सी । ए॒व । भ॒व॒ति॒ । अ॒ष्ट॒रा॒त्र इत्य॑ष्ट - रा॒त्रः । भ॒व॒ति॒ । अ॒ष्टाक्ष॒रेत्य॒ष्टा - अ॒क्ष॒रा॒ । गा॒य॒त्री । गा॒य॒त्री । ब्र॒ह्म॒व॒र्च॒समिति॑ ब्रह्म - व॒र्च॒सम् । गा॒य॒त्रि॒या । ए॒व । ब्र॒ह्म॒व॒र्च॒समिति॑ ब्रह्म - व॒र्च॒सम् । अवेति॑ । रु॒न्धे॒ । अ॒ष्ट॒रा॒त्र इत्य॑ष्ट-रा॒त्रः । भ॒व॒ति॒ । चत॑स्रः । वै । दिशः॑ । चत॑स्रः । अ॒वा॒न्त॒र॒दि॒शा इत्य॑वान्तर-दि॒शाः । दि॒ग्भ्य इति॑ दिक् - भ्यः । ए॒व । ब्र॒ह्म॒व॒र्च॒समिति॑ ब्रह्म - व॒र्च॒सम् । अवेति॑ । रु॒न्धे॒ ।  \newline




\markright{ TS 7.2.3.2  \hfill https://www.vedavms.in \hfill}

\section{ TS 7.2.3.2 }

\textbf{TS 7.2.3.2 } \newline
\textbf{Samhita Paata} \newline

त्रि॒वृद॑ग्निष्टो॒मो भ॑वति॒ तेज॑ ए॒वाव॑ रुन्धे पञ्चद॒शो भ॑वतीन्द्रि॒यमे॒वाव॑ रुन्धे सप्तद॒शो भ॑वत्य॒न्नाद्य॒स्या-व॑रुद्ध्या॒ अथो॒ प्रैव तेन॑ जायत एकविꣳ॒॒शो भ॑वति॒ प्रति॑ष्ठित्या॒ अथो॒ रुच॑मे॒वाऽऽ*त्मन् ध॑त्ते त्रिण॒वो भ॑वति॒ विजि॑त्यै त्रयस्त्रिꣳ॒॒शो भ॑वति॒ प्रति॑ष्ठित्यै पञ्चविꣳ॒॒शो᳚ऽग्निष्टो॒मो भ॑वति प्र॒जाप॑ते॒राप्त्यै॑ महाव्र॒तवा॑न॒न्नाद्य॒स्या व॑रुद्ध्यै विश्व॒जिथ् सर्व॑पृष्ठोऽतिरा॒त्रो भ॑वति॒ सर्व॑स्या॒भिजि॑त्यै ( ) ॥ \newline

\textbf{Pada Paata} \newline

त्रि॒वृदिति॑ त्रि - वृत् । अ॒ग्नि॒ष्टो॒म इत्य॑ग्नि-स्तो॒मः । भ॒व॒ति॒ । तेजः॑ । ए॒व । अवेति॑ । रु॒न्धे॒ । प॒ञ्च॒द॒श इति॑ पञ्च - द॒शः । भ॒व॒ति॒ । इ॒न्द्रि॒यम् । ए॒व । अवेति॑ । रु॒न्धे॒ । स॒प्त॒द॒श इति॑ सप्त - द॒शः । भ॒व॒ति॒ । अ॒न्नाद्य॒स्येत्य॑न्न - अद्य॑स्य । अव॑रुद्ध्या॒ इत्यव॑ - रु॒द्ध्यै॒ । अथो॒ इति॑ । प्रेति॑ । ए॒व । तेन॑ । जा॒य॒ते॒ । ए॒क॒विꣳ॒॒श इत्ये॑क - विꣳ॒॒शः । भ॒व॒ति॒ । प्रति॑ष्ठित्या॒ इति॒ प्रति॑ - स्थि॒त्यै॒ । अथो॒ इति॑ । रुच᳚म् । ए॒व । आ॒त्मन्न् । ध॒त्ते॒ । त्रि॒ण॒व इति॑ त्रि-न॒वः । भ॒व॒ति॒ । विजि॑त्या॒ इति॒ वि - जि॒त्यै॒ । त्र॒य॒स्त्रिꣳ॒॒श इति॑ त्रयः - त्रिꣳ॒॒शः । भ॒व॒ति॒ । प्रति॑ष्ठित्या॒ इति॒ प्रति॑ - स्थि॒त्यै॒ । प॒ञ्च॒विꣳ॒॒श इति॑ पञ्च - विꣳ॒॒शः । अ॒ग्नि॒ष्टो॒म इत्य॑ग्नि - स्तो॒मः । भ॒व॒ति॒ । प्र॒जाप॑ते॒रिति॑ प्र॒जा-प॒तेः॒ । आप्त्यै᳚ । म॒हा॒व्र॒तवा॒निति॑ महाव्र॒त-वा॒न् । अ॒न्नाद्य॒स्येत्य॑न्न - अद्य॑स्य । अव॑रुद्ध्या॒ इत्यव॑ - रु॒द्ध्यै॒ । वि॒श्व॒जिदिति॑ विश्व - जित् । सर्व॑पृष्ठ॒ इति॒ सर्व॑ - पृ॒ष्ठः॒ । अ॒ति॒रा॒त्र इत्य॑ति - रा॒त्रः । भ॒व॒ति॒ । सर्व॑स्य । अ॒भिजि॑त्या॒ इत्य॒भि - जि॒त्यै॒ ( ) ॥  \newline




\markright{ TS 7.2.4.1  \hfill https://www.vedavms.in \hfill}

\section{ TS 7.2.4.1 }

\textbf{TS 7.2.4.1 } \newline
\textbf{Samhita Paata} \newline

प्र॒जाप॑तिः प्र॒जा अ॑सृजत॒ ताः सृ॒ष्टाः क्षुधं॒ न्या॑य॒न्थ्स ए॒तं न॑वरा॒त्रप॑श्य॒त् तमाऽह॑र॒त् तेना॑यजत॒ ततो॒ वै प्र॒जाभ्यो॑ऽकल्पत॒ यर्.हि॑ प्र॒जाः क्षुधं॑ नि॒गच्छे॑यु॒स्तर्.हि॑ नवरा॒त्रेण॑ यजेते॒मे हि वा ए॒तासां᳚ ॅलो॒का अक्लृ॑प्ता॒ अथै॒ताः क्षुधं॒ नि ग॑च्छन्ती॒माने॒वाऽऽ*भ्यो॑ लो॒कान् क॑ल्पयति॒ तान् कल्प॑मानान् प्र॒जाभ्योऽनु॑ कल्पते॒ कल्प॑न्ते - [  ] \newline

\textbf{Pada Paata} \newline

प्र॒जाप॑ति॒रिति॑ प्र॒जा - प॒तिः॒ । प्र॒जा इति॑ प्र-जाः । अ॒सृ॒ज॒त॒ । ताः । सृ॒ष्टाः । क्षुध᳚म् । नीति॑ । आ॒य॒न्न् । सः । ए॒तम् । न॒व॒रा॒त्रमिति॑ नव - रा॒त्रम् । अ॒प॒श्य॒त् । तम् । एति॑ । अ॒ह॒र॒त् । तेन॑ । अ॒य॒ज॒त॒ । ततः॑ । वै । प्र॒जाभ्य॒ इति॑ प्र - जाभ्यः॑ । अ॒क॒ल्प॒त॒ । यर्.हि॑ । प्र॒जा इति॑ प्र - जाः । क्षुध᳚म् । नि॒गच्छे॑यु॒रिति॑ नि - गच्छे॑युः । तर्.हि॑ । न॒व॒रा॒त्रेणेति॑ नव - रा॒त्रेण॑ । य॒जे॒त॒ । इ॒मे । हि । वै । ए॒तासा᳚म् । लो॒काः । अक्लृ॑प्ताः । अथ॑ । ए॒ताः । क्षुध᳚म् । नीति॑ । ग॒च्छ॒न्ति॒ । इ॒मान् । ए॒व । आ॒भ्यः॒ । लो॒कान् । क॒ल्प॒य॒ति॒ । तान् । कल्प॑मानान् । प्र॒जाभ्य॒ इति॑ प्र - जाभ्यः॑ । अन्विति॑ । क॒ल्प॒ते॒ । कल्प॑न्ते ।  \newline




\markright{ TS 7.2.4.2  \hfill https://www.vedavms.in \hfill}

\section{ TS 7.2.4.2 }

\textbf{TS 7.2.4.2 } \newline
\textbf{Samhita Paata} \newline

ऽस्मा इ॒मे लो॒का ऊर्जं॑ प्र॒जासु॑ दधाति त्रिरा॒त्रेणै॒वेमं ॅलो॒कं क॑ल्पयति त्रिरा॒त्रेणा॒न्तरि॑क्षं त्रिरा॒त्रेणा॒मुं ॅलो॒कं ॅयथा॑ गु॒णे ग॒णम॒न्वस्य॑त्ये॒वमे॒व तल्लो॒के लो॒कमन्व॑स्यति॒ धृत्या॒ अशि॑थिलंभावाय॒ ज्योति॒र्गौरायु॒रिति॑ ज्ञा॒ताः स्तोमा॑ भवन्ती॒यं ॅवाव ज्योति॑र॒न्तरि॑क्षं॒ गौर॒सावायु॑रे॒ष्वे॑व लो॒केषु॒ प्रति॑ तिष्ठन्ति॒ ज्ञात्रं॑ प्र॒जानां᳚-[  ] \newline

\textbf{Pada Paata} \newline

अ॒स्मै॒ । इ॒मे । लो॒काः । ऊर्ज᳚म् । प्र॒जास्विति॑ प्र - जासु॑ । द॒धा॒ति॒ । त्रि॒रा॒त्रेणेति॑ त्रि - रा॒त्रेण॑ । ए॒व । इ॒मम् । लो॒कम् । क॒ल्प॒य॒ति॒ । त्रि॒रा॒त्रेणेति॑ त्रि - रा॒त्रेण॑ । अ॒न्तरि॑क्षम् । त्रि॒रा॒त्रेणेति॑ त्रि - रा॒त्रेण॑ । अ॒मुम् । लो॒कम् । यथा᳚ । गु॒णे । गु॒णम् । अ॒न्वस्य॒तीत्य॑नु-अस्य॑ति । ए॒वम् । ए॒व । तत् । लो॒के । लो॒कम् । अन्विति॑ । अ॒स्य॒ति॒ । धृत्यै᳚ । अशि॑थिलंभावा॒येत्यशि॑थिलं - भा॒वा॒य॒ । ज्योतिः॑ । गौः । आयुः॑ । इति॑ । ज्ञा॒ताः । स्तोमाः᳚ । भ॒व॒न्ति॒ । इ॒यम् । वाव । ज्योतिः॑ । अ॒न्तरि॑क्षम् । गौः । अ॒सौ । आयुः॑ । ए॒षु । ए॒व । लो॒केषु॑ । प्रतीति॑ । ति॒ष्ठ॒न्ति॒ । ज्ञात्र᳚म् । प्र॒जाना॒मिति॑ प्र - जाना᳚म् ।  \newline




\markright{ TS 7.2.4.3  \hfill https://www.vedavms.in \hfill}

\section{ TS 7.2.4.3 }

\textbf{TS 7.2.4.3 } \newline
\textbf{Samhita Paata} \newline

गच्छति नवरा॒त्रो भ॑वत्यभिपू॒र्वमे॒वाऽस्मि॒न् तेजो॑ दधाति॒ यो ज्योगा॑मयावी॒ स्याथ् स न॑वरा॒त्रेण॑ यजेत प्रा॒णा हि वा ए॒तस्या धृ॑ता॒ अथै॒तस्य॒ ज्योगा॑मयति प्रा॒णाने॒वास्मि॑न् दाधारो॒त यदी॒तासु॒र्भव॑ति॒ जीव॑त्ये॒व ॥ \newline

\textbf{Pada Paata} \newline

ग॒च्छ॒ति॒ । न॒व॒रा॒त्र इति॑ नव - रा॒त्रः । भ॒व॒ति॒ । अ॒भि॒पू॒र्वमित्य॑भि-पू॒र्वम् । ए॒व । अ॒स्मि॒न्न् । तेजः॑ । द॒धा॒ति॒ । यः । ज्योगा॑मया॒वीति॒ ज्योक् - आ॒म॒या॒वी॒ । स्यात् । सः । न॒व॒रा॒त्रेणेति॑ नव - रा॒त्रेण॑ । य॒जे॒त॒ । प्रा॒णा इति॑ प्र-अ॒नाः । हि । वै । ए॒तस्य॑ । अधृ॑ताः । अथ॑ । ए॒तस्य॑ । ज्योक् । आ॒म॒य॒ति॒ । प्रा॒णानिति॑ प्र - अ॒नान् । ए॒व । अ॒स्मि॒न्न् । दा॒धा॒र॒ । उ॒त । यदि॑ । इ॒तासु॒रिती॒त - अ॒सुः॒ । भव॑ति । जीव॑ति । ए॒व ॥  \newline




\markright{ TS 7.2.5.1  \hfill https://www.vedavms.in \hfill}

\section{ TS 7.2.5.1 }

\textbf{TS 7.2.5.1 } \newline
\textbf{Samhita Paata} \newline

प्र॒जाप॑तिरकामयत॒ प्र जा॑ये॒येति॒ स ए॒तं दश॑होतारमपश्य॒त् तम॑जुहो॒त् तेन॑ दशरा॒त्रम॑सृजत॒ तेन॑ दशरा॒त्रेण॒ प्रा जा॑यत दशरा॒त्राय॑ दीक्षि॒ष्यमा॑णो॒ दश॑होतारं जुहुया॒द्-दश॑होत्रै॒व द॑शरा॒त्रꣳ सृ॑जते॒ तेन॑ दशरा॒त्रेण॒ प्र जा॑यते वैरा॒जो वा ए॒ष य॒ज्ञो यद्द॑शरा॒त्रो य ए॒वं ॅवि॒द्वान्-द॑शरा॒त्रेण॒ यज॑ते वि॒राज॑मे॒व ग॑च्छति प्राजाप॒त्यो वा ए॒ष य॒ज्ञो यद्-द॑शरा॒त्रो-[  ] \newline

\textbf{Pada Paata} \newline

प्र॒जाप॑ति॒रिति॑ प्र॒जा - प॒तिः॒ । अ॒का॒म॒य॒त॒ । प्रेति॑ । जा॒ये॒य॒ । इति॑ । सः । ए॒तम् । दश॑होतार॒मिति॒ दश॑ - हो॒ता॒र॒म् । अ॒प॒श्य॒त् । तम् । अ॒जु॒हो॒त् । तेन॑ । द॒श॒रा॒त्रमिति॑ दश - रा॒त्रम् । अ॒सृ॒ज॒त॒ । तेन॑ । द॒श॒रा॒त्रेणेति॑ दश - रा॒त्रेण॑ । प्रेति॑ । अ॒जा॒य॒त॒ । द॒श॒रा॒त्रायेति॑ दश-रा॒त्राय॑ । दी॒क्षि॒ष्यमा॑णः । दश॑होतार॒मिति॒ दश॑ - हो॒ता॒र॒म् । जु॒हु॒या॒त् । दश॑हो॒त्रेति॒ दश॑-हो॒त्रा॒ । ए॒व । द॒श॒रा॒त्रमिति॑ दश-रा॒त्रम् । सृ॒ज॒ते॒ । तेन॑ । द॒श॒रा॒त्रेणेति॑ दश-रा॒त्रेण॑ । प्रेति॑ । जा॒य॒ते॒ । वै॒रा॒जः । वै । ए॒षः । य॒ज्ञ्ः । यत् । द॒श॒रा॒त्र इति॑ दश - रा॒त्रः । यः । ए॒वम् । वि॒द्वान् । द॒श॒रा॒त्रेणेति॑ दश - रा॒त्रेण॑ । यज॑ते । वि॒राज॒मिति॑ वि - राज᳚म् । ए॒व । ग॒च्छ॒ति॒ । प्रा॒जा॒प॒त्य इति॑ प्राजा - प॒त्यः । वै । ए॒षः । य॒ज्ञ्ः । यत् । द॒श॒रा॒त्र इति॑ दश - रा॒त्रः ।  \newline




\markright{ TS 7.2.5.2  \hfill https://www.vedavms.in \hfill}

\section{ TS 7.2.5.2 }

\textbf{TS 7.2.5.2 } \newline
\textbf{Samhita Paata} \newline

य ए॒वं ॅवि॒द्वान्-द॑शरा॒त्रेण॒ यज॑ते॒ प्रैव जा॑यत॒ इन्द्रो॒ वै स॒दृङ् दे॒वता॑भिरासी॒थ् स न व्या॒वृत॑मगच्छ॒थ् स प्र॒जाप॑ति॒मुपा॑धाव॒त् तस्मा॑ ए॒तं द॑शरा॒त्रं प्राय॑च्छ॒त् तमाऽह॑र॒त् तेना॑यजत॒ ततो॒ वै सो᳚ऽन्याभि॑-र्दे॒वता॑भि-र्व्या॒वृत॑मगच्छ॒द्य ए॒वं ॅवि॒द्वान् द॑शरा॒त्रेण॒ यज॑ते व्या॒वृत॑मे॒व पा॒प्मना॒ भ्रातृ॑व्येण गच्छति त्रिक॒कुद्वा - [  ] \newline

\textbf{Pada Paata} \newline

यः । ए॒वम् । वि॒द्वान् । द॒श॒रा॒त्रेणेति॑ दश - रा॒त्रेण॑ । यज॑ते । प्रेति॑ । ए॒व । जा॒य॒ते॒ । इन्द्रः॑ । वै । स॒दृङ्ङिति॑ स - दृङ् । दे॒वता॑भिः । आ॒सी॒त् । सः । न । व्या॒वृत॒मिति॑ वि - आ॒वृत᳚म् । अ॒ग॒च्छ॒त् । सः । प्र॒जाप॑ति॒मिति॑ प्र॒जा - प॒ति॒म् । उपेति॑ । अ॒धा॒व॒त् । तस्मै᳚ । ए॒तम् । द॒श॒रा॒त्रमिति॑ दश-रा॒त्रम् । प्रेति॑ । अ॒य॒च्छ॒त् । तम् । एति॑ । अ॒ह॒र॒त् । तेन॑ । अ॒य॒ज॒त॒ । ततः॑ । वै । सः । अ॒न्याभिः॑ । दे॒वता॑भिः । व्या॒वृत॒मिति॑ वि - आ॒वृत᳚म् । अ॒ग॒च्छ॒त् । यः । ए॒वम् । वि॒द्वान् । द॒श॒रा॒त्रेणेति॑ दश - रा॒त्रेण॑ । यज॑ते । व्या॒वृत॒मिति॑ वि - आ॒वृत᳚म् । ए॒व । पा॒प्मना᳚ । भ्रातृ॑व्येण । ग॒च्छ॒ति॒ । त्रि॒क॒कुदिति॑ त्रि - क॒कुत् । वै ।  \newline




\markright{ TS 7.2.5.3  \hfill https://www.vedavms.in \hfill}

\section{ TS 7.2.5.3 }

\textbf{TS 7.2.5.3 } \newline
\textbf{Samhita Paata} \newline

ए॒ष य॒ज्ञो यद्-द॑शरा॒त्रः क॒कुत् प॑ञ्चद॒शः क॒कुदे॑कविꣳ॒॒शः क॒कुत् त्र॑यस्त्रिꣳ॒॒शो य ए॒वं ॅवि॒द्वान् द॑शरा॒त्रेण॒ यज॑ते त्रिक॒कुदे॒व स॑मा॒नानां᳚ भवति॒ यज॑मानः पञ्चद॒शो यज॑मान एकविꣳ॒॒शो यज॑मानस्त्रयस्त्रिꣳ॒॒शः पुर॒ इत॑रा अभिच॒र्यमा॑णो दशरा॒त्रेण॑ यजेत देवपु॒रा ए॒व पर्यू॑हते॒ तस्य॒ न कुत॑श्च॒नोपा᳚व्या॒धो भ॑वति॒ नैन॑मभि॒चरन्᳚थ् स्तृणुते देवासु॒राः संॅय॑त्ता आस॒न् ते दे॒वा ए॒ता - [  ] \newline

\textbf{Pada Paata} \newline

ए॒षः । य॒ज्ञ्ः । यत् । द॒श॒रा॒त्र इति॑ दश - रा॒त्रः । क॒कुत् । प॒ञ्च॒द॒श इति॑ पञ्च - द॒शः । क॒कुत् । ए॒क॒विꣳ॒॒श इत्ये॑क - विꣳ॒॒शः । क॒कुत् । त्र॒य॒स्त्रिꣳ॒॒श इति॑ त्रयः - त्रिꣳ॒॒शः । यः । ए॒वम् । वि॒द्वान् । द॒श॒रा॒त्रेणेति॑ दश - रा॒त्रेण॑ । यज॑ते । त्रि॒क॒कुदिति॑ त्रि - क॒कुत् । ए॒व । स॒मा॒नाना᳚म् । भ॒व॒ति॒ । यज॑मानः । प॒ञ्च॒द॒श इति॑ पञ्च -द॒शः । यज॑मानः । ए॒क॒विꣳ॒॒श इत्ये॑क - विꣳ॒॒शः । यज॑मानः । त्र॒य॒स्त्रिꣳ॒॒श इति॑ त्रयः - त्रिꣳ॒॒शः । पुरः॑ । इत॑राः । अ॒भि॒च॒र्यमा॑ण॒ इत्य॑भि - च॒र्यमा॑णः । द॒श॒रा॒त्रेणेति॑ दश - रा॒त्रेण॑ । य॒जे॒त॒ । दे॒व॒पु॒रा इति॑ देव-पु॒राः । ए॒व । परीति॑ । ऊ॒ह॒ते॒ । तस्य॑ । न । कुतः॑ । च॒न । उ॒पा॒व्या॒ध इत्यु॑प - आ॒व्या॒धः । भ॒व॒ति॒ । न । ए॒न॒म् । अ॒भि॒चर॒न्नित्य॑भि - चरन्न्॑ । स्तृ॒णु॒ते॒ । दे॒वा॒सु॒रा इति॑ देव-अ॒सु॒राः । संॅय॑त्ता॒ इति॒ सं - य॒त्ताः॒ । आ॒स॒न्न् । ते । दे॒वाः । ए॒ताः ।  \newline




\markright{ TS 7.2.5.4  \hfill https://www.vedavms.in \hfill}

\section{ TS 7.2.5.4 }

\textbf{TS 7.2.5.4 } \newline
\textbf{Samhita Paata} \newline

दे॑वपु॒रा अ॑पश्य॒न्॒ यद्-द॑शरा॒त्रस्ताः पर्यौ॑हन्त॒ तेषां॒ न कुत॑श्च॒नोपा᳚व्या॒धो॑ ऽभव॒त् ततो॑ दे॒वा अभ॑व॒न् पराऽसु॑रा॒ यो भ्रातृ॑व्यवा॒न्थ् स्याथ् स द॑शरा॒त्रेण॑ यजेत देवपु॒रा ए॒व पर्यू॑हते॒ तस्य॒ न कुत॑श्च॒नोपा᳚व्या॒धो भ॑वति॒ भव॑त्या॒त्मना॒ परा᳚ऽस्य॒ भ्रातृ॑व्यो भवति॒ स्तोमः॒ स्तोम॒स्योप॑स्तिर्भवति॒ भ्रातृ॑व्यमे॒वोप॑स्तिं कुरुते जा॒मि वा - [  ] \newline

\textbf{Pada Paata} \newline

दे॒व॒पु॒रा इति॑ देव-पु॒राः । अ॒प॒श्य॒न्न् । यत् । द॒श॒रा॒त्र इति॑ दश-रा॒त्रः । ताः । परीति॑ । औ॒ह॒न्त॒ । तेषा᳚म् । न । कुतः॑ । च॒न । उ॒पा॒व्या॒ध इत्यु॑प - आ॒व्या॒धः । अ॒भ॒व॒त् । ततः॑ । दे॒वाः । अभ॑वन्न् । परेति॑ । असु॑राः । यः । भ्रातृ॑व्यवा॒निति॒ भ्रातृ॑व्य - वा॒न् । स्यात् । सः । द॒श॒रा॒त्रेणेति॑ दश - रा॒त्रेण॑ । य॒जे॒त॒ । दे॒व॒पु॒रा इति॑ देव - पु॒राः । ए॒व । परीति॑ । ऊ॒ह॒ते॒ । तस्य॑ । न । कुतः॑ । च॒न । उ॒पा॒व्या॒ध इत्यु॑प- आ॒व्या॒धः । भ॒व॒ति॒ । भव॑ति । आ॒त्मना᳚ । परेति॑ । अ॒स्य॒ । भ्रातृ॑व्यः । भ॒व॒ति॒ । स्तोमः॑ । स्तोम॑स्य । उप॑स्तिः । भ॒व॒ति॒ । भ्रातृ॑व्यम् । ए॒व । उप॑स्तिम् । कु॒रु॒ते॒ । जा॒मि । वै ।  \newline




\markright{ TS 7.2.5.5  \hfill https://www.vedavms.in \hfill}

\section{ TS 7.2.5.5 }

\textbf{TS 7.2.5.5 } \newline
\textbf{Samhita Paata} \newline

ए॒तत् कु॑र्वन्ति॒ यज्ज्यायाꣳ॑सꣳ॒॒ स्तोम॑मु॒पेत्य॒ कनी॑याꣳसमुप॒यन्ति॒ यद॑ग्निष्टो-मसा॒मान्य॒वस्ता᳚च्च प॒रस्ता᳚च्च॒ भव॒न्त्यजा॑मित्वाय त्रि॒वृद॑ग्निष्टो॒मो᳚ ऽग्नि॒ष्टुदा᳚ग्ने॒यीषु॑ भवति॒ तेज॑ ए॒वाव॑ रुन्धे पञ्चद॒श उ॒क्थ्य॑ ऐ॒न्द्रीष्वि॑न्द्रि॒यमे॒वाव॑ रुन्धे त्रि॒वृद॑ग्निष्टो॒मो वै᳚श्वदे॒वीषु॒ पुष्टि॑मे॒वाव॑ रुन्धे सप्तद॒शो᳚ऽग्निष्टो॒मः प्रा॑जाप॒त्यासु॑ तीव्रसो॒मो᳚ ऽन्नाद्य॒स्या-व॑रुद्ध्या॒ अथो॒ प्रैव तेन॑ जायत - [  ] \newline

\textbf{Pada Paata} \newline

ए॒तत् । कु॒र्व॒न्ति॒ । यत् । ज्यायाꣳ॑सम् । स्तोम᳚म् । उ॒पेत्येत्यु॑प-इत्य॑ । कनी॑याꣳसम् । उ॒प॒यन्तीत्यु॑प - यन्ति॑ । यत् । अ॒ग्नि॒ष्टो॒म॒सा॒मानीत्य॑ग्निष्टोम-सा॒मानि॑ । अ॒वस्ता᳚त् । च॒ । प॒रस्ता᳚त् । च॒ । भव॑न्ति । अजा॑मित्वा॒येत्यजा॑मि - त्वा॒य॒ । त्रि॒वृदिति॑ त्रि - वृत् । अ॒ग्नि॒ष्टो॒म इत्य॑ग्नि - स्तो॒मः । अ॒ग्नि॒ष्टुदित्य॑ग्नि-स्तुत् । आ॒ग्ने॒यीषु॑ । भ॒व॒ति॒ । तेजः॑ । ए॒व । अवेति॑ । रु॒न्धे॒ । प॒ञ्च॒द॒श इति॑ पञ्च-द॒शः । उ॒क्थ्यः॑ । ऐ॒न्द्रीषु॑ । इ॒न्द्रि॒यम् । ए॒व । अवेति॑ । रु॒न्धे॒ । त्रि॒व॒दिति॑ त्रि - वृत् । अ॒ग्नि॒ष्टो॒म इत्य॑ग्नि - स्तो॒मः । वै॒श्व॒दे॒वीष्विति॑ वैश्व - दे॒वीषु॑ । पुष्टि᳚म् । ए॒व । अवेति॑ । रु॒न्धे॒ । स॒प्त॒द॒श इति॑ सप्त - द॒शः । अ॒ग्नि॒ष्टो॒म इत्य॑ग्नि - स्तो॒मः । प्रा॒जा॒प॒त्यास्विति॑ प्राजा - प॒त्यासु॑ । ती॒व्र॒सो॒म इति॑ तीव्र-सो॒मः । अ॒न्नाद्य॒स्येत्य॑न्न - अद्य॑स्य । अव॑रुद्ध्या॒ इत्यव॑-रुद्॒ध्यै॒ । अथो॒ इति॑ । प्रेति॑ । ए॒व । तेन॑ । जा॒य॒ते॒ ।  \newline




\markright{ TS 7.2.5.6  \hfill https://www.vedavms.in \hfill}

\section{ TS 7.2.5.6 }

\textbf{TS 7.2.5.6 } \newline
\textbf{Samhita Paata} \newline

एकविꣳ॒॒श उ॒क्थ्यः॑ सौ॒रीषु॒ प्रति॑ष्ठित्या॒ अथो॒ रुच॑मे॒वाऽऽ*त्मन् ध॑त्ते सप्तद॒शो᳚ऽग्निष्टो॒मः प्रा॑जाप॒त्यासू॑पह॒व्य॑ उपह॒वमे॒व ग॑च्छति त्रिण॒वाव॑ग्निष्टो॒माव॒भित॑ ऐ॒न्द्रीषु॒ विजि॑त्यै त्रयस्त्रिꣳ॒॒श उ॒क्थ्यो॑ वैश्वदे॒वीषु॒ प्रति॑ष्ठित्यै विश्व॒जिथ् सर्व॑पृष्ठोऽतिरा॒त्रो भ॑वति॒ सर्व॑स्या॒भिजि॑त्यै ॥ \newline

\textbf{Pada Paata} \newline

ए॒क॒विꣳ॒॒श इत्ये॑क - विꣳ॒॒शः । उ॒क्थ्यः॑ । सौ॒रीषु॑ । प्रति॑ष्ठित्या॒ इति॒ प्रति॑ - स्थि॒त्यै॒ । अथो॒ इति॑ । रुच᳚म् । ए॒व । आ॒त्मन्न् । ध॒त्ते॒ । स॒प्त॒द॒श इति॑ सप्त - द॒शः । अ॒ग्नि॒ष्टो॒म इत्य॑ग्नि - स्तो॒मः । प्रा॒जा॒प॒त्यास्विति॑ प्राजा - प॒त्यासु॑ । उ॒प॒ह॒व्य॑ इत्यु॑प - ह॒व्यः॑ । उ॒प॒ह॒वमित्यु॑प - ह॒वम् । ए॒व । ग॒च्छ॒ति॒ । त्रि॒ण॒वाविति॑ त्रि - न॒वौ । अ॒ग्नि॒ष्टो॒मावित्य॑ग्नि - स्तो॒मौ । अ॒भितः॑ । ऐ॒न्द्रीषु॑ । विजि॑त्या॒ इति॒ वि - जि॒त्यै॒ । त्र॒य॒स्त्रिꣳ॒॒श इति॑ त्रयः - त्रिꣳ॒॒शः । उ॒क्थ्यः॑ । वै॒श्व॒दे॒वीष्विति॑ वैश्व - दे॒वीषु॑ । प्रति॑ष्ठित्या॒ इति॒ प्रति॑ - स्थि॒त्यै॒ । वि॒श्व॒जिदिति॑ विश्व - जित् । सर्व॑पृष्ठ॒ इति॒ सर्व॑ - पृ॒ष्ठः॒ । अ॒ति॒रा॒त्र इत्य॑ति - रा॒त्रः । भ॒व॒ति॒ । सर्व॑स्य । अ॒भिजि॑त्या॒ इत्य॒भि - जि॒त्यै॒ ॥  \newline




\markright{ TS 7.2.6.1  \hfill https://www.vedavms.in \hfill}

\section{ TS 7.2.6.1 }

\textbf{TS 7.2.6.1 } \newline
\textbf{Samhita Paata} \newline

ऋ॒तवो॒ वै प्र॒जाका॑माः प्र॒जां नाऽवि॑न्दन्त॒ ते॑ऽकामयन्त प्र॒जाꣳ सृ॑जेमहि प्र॒जामव॑ रुन्धीमहि प्र॒जां ॅवि॑न्देमहि प्र॒जाव॑न्तः स्या॒मेति॒ त ए॒तमे॑कादशरा॒त्रम॑पश्य॒न् तमाऽह॑र॒न् तेना॑यजन्त॒ ततो॒ वै ते प्र॒जाम॑सृजन्त प्र॒जामवा॑रुन्धत प्र॒जाम॑विन्दन्त प्र॒जाव॑न्तोऽभव॒न्त ऋ॒तवो॑ऽभव॒न् तदा᳚र्त॒वाना॑-मार्तव॒त्व-मृ॑तू॒नां ॅवा ए॒ते पु॒त्रास्तस्मा॑ - [  ] \newline

\textbf{Pada Paata} \newline

ऋ॒तवः॑ । वै । प्र॒जाका॑मा॒ इति॑ प्र॒जा - का॒माः॒ । प्र॒जामिति॑ प्र - जाम् । न । अ॒वि॒न्द॒न्त॒ । ते । अ॒का॒म॒य॒न्त॒ । प्र॒जामिति॑ प्र - जाम् । सृ॒जे॒म॒हि॒ । प्र॒जामिति॑ प्र-जाम् । अवेति॑ । रु॒न्धी॒म॒हि॒ । प्र॒जामिति॑ प्र - जाम् । वि॒न्दे॒म॒हि॒ । प्र॒जाव॑न्त॒ इति॑ प्र॒जा - व॒न्तः॒ । स्या॒म॒ । इति॑ । ते । ए॒तम् । ए॒का॒द॒श॒रा॒त्रमित्ये॑कादश - रा॒त्रम् । अ॒प॒श्य॒न्न् । तम् । एति॑ । अ॒ह॒र॒न्न् । तेन॑ । अ॒य॒ज॒न्त॒ । ततः॑ । वै । ते । प्र॒जामिति॑ प्र - जाम् । अ॒सृ॒ज॒न्त॒ । प्र॒जामिति॑ प्र - जाम् । अवेति॑ । अ॒रु॒न्ध॒त॒ । प्र॒जामिति॑ प्र - जाम् । अ॒वि॒न्द॒न्त॒ । प्र॒जाव॑न्त॒ इति॑ प्र॒जा - व॒न्तः॒ । अ॒भ॒व॒न्न् । ते । ऋ॒तवः॑ । अ॒भ॒व॒न्न् । तत् । आ॒र्त॒वाना᳚म् । आ॒र्त॒व॒त्वमित्या᳚र्तव - त्वम् । ऋ॒तू॒नाम् । वै । ए॒ते । पु॒त्राः । तस्मा᳚त् ।  \newline




\markright{ TS 7.2.6.2  \hfill https://www.vedavms.in \hfill}

\section{ TS 7.2.6.2 }

\textbf{TS 7.2.6.2 } \newline
\textbf{Samhita Paata} \newline

-दार्त॒वा उ॑च्यन्ते॒ य ए॒वं ॅवि॒द्वाꣳस॑ एकादशरा॒त्रमास॑ते प्र॒जामे॒व सृ॑जन्ते प्र॒जामव॑ रुन्धते प्र॒जां ॅवि॑न्दन्ते प्र॒जाव॑न्तो भवन्ति॒ ज्योति॑रतिरा॒त्रो भ॑वति॒ ज्योति॑रे॒व पु॒रस्ता᳚द्-दधते सुव॒र्गस्य॑ लो॒कस्या-नु॑ख्यात्यै॒ पृष्ठ्यः॑ षड॒हो भ॑वति॒ षड् वा ऋ॒तवः॒ षट् पृ॒ष्ठानि॑ पृ॒ष्ठैरे॒वर्तून॒-न्वारो॑हन्त्यृ॒तुभिः॑ संॅवथ्स॒रं ते सं॑ॅवथ्स॒र ए॒व प्रति॑ तिष्ठन्ति चतुर्विꣳ॒॒शो भ॑वति॒ चतु॑र्विꣳशत्यक्षरा गाय॒त्री - [  ] \newline

\textbf{Pada Paata} \newline

आ॒र्त॒वाः । उ॒च्य॒न्ते॒ । ये । ए॒वम् । वि॒द्वाꣳसः॑ । ए॒का॒द॒श॒रा॒त्रमित्ये॑कादश - रा॒त्रम् । आस॑ते । प्र॒जामिति॑ प्र - जाम् । ए॒व । सृ॒ज॒न्ते॒ । प्र॒जामिति॑ प्र - जाम् । अवेति॑ । रु॒न्ध॒ते॒ । प्र॒जामिति॑ प्र - जाम् । वि॒न्द॒न्ते॒ । प्र॒जाव॑न्त॒ इति॑ प्र॒जा - व॒न्तः॒ । भ॒व॒न्ति॒ । ज्योतिः॑ । अ॒ति॒रा॒त्र इत्य॑ति - रा॒त्रः । भ॒व॒ति॒ । ज्योतिः॑ । ए॒व । पु॒रस्ता᳚त् । द॒ध॒ते॒ । सु॒व॒र्गस्येति॑ सुवः - गस्य॑ । लो॒कस्य॑ । अनु॑ख्यात्या॒ इत्यनु॑ - ख्या॒त्यै॒ । पृष्ठ्यः॑ । ष॒ड॒ह इति॑ षट् - अ॒हः । भ॒व॒ति॒ । षट् । वै । ऋ॒तवः॑ । षट् । पृ॒ष्ठानि॑ । पृ॒ष्ठैः । ए॒व । ऋ॒तून् । अ॒न्वारो॑ह॒न्तीत्य॑नु - आरो॑हन्ति । ऋ॒तुभि॒रित्यृ॒तु - भिः॒ । सं॒ॅव॒थ्स॒रमिति॑ सं - व॒थ्स॒रम् । ते । सं॒ॅव॒थ्स॒र इति॑ सं - व॒थ्स॒रे । ए॒व । प्रतीति॑ । ति॒ष्ठ॒न्ति॒ । च॒तु॒र्विꣳ॒॒श इति॑ चतुः-विꣳ॒॒शः । भ॒व॒ति॒ । चतु॑र्विꣳशत्यक्ष॒रेति॒ चतु॑विꣳशति - अ॒क्ष॒रा॒ । गा॒य॒त्री ।  \newline




\markright{ TS 7.2.6.3  \hfill https://www.vedavms.in \hfill}

\section{ TS 7.2.6.3 }

\textbf{TS 7.2.6.3 } \newline
\textbf{Samhita Paata} \newline

गा॑य॒त्रं ब्र॑ह्मवर्च॒सं गा॑यत्रि॒यामे॒व ब्र॑ह्मवर्च॒से प्रति॑ तिष्ठन्ति चतुश्चत्वारिꣳ॒॒शो भ॑वति॒ चतु॑ष्चत्वारिꣳशदक्षरा त्रि॒ष्टुगि॑न्द्रि॒यं त्रि॒ष्टुप् त्रि॒ष्टुभ्ये॒वेन्द्रि॒ये प्रति॑ तिष्ठन्त्यष्टाचत्वारिꣳ॒॒शो भ॑वत्य॒ष्टाच॑त्वारिꣳशदक्षरा॒ जग॑ती॒ जाग॑ताः प॒शवो॒ जग॑त्यामे॒व प॒शुषु॒ प्रति॑ तिष्ठन्त्ये-कादशरा॒त्रो भ॑वति॒ पञ्च॒ वा ऋ॒तव॑ आर्त॒वाः पञ्च॒र्तुष्वे॒वाऽऽ*र्त॒वेषु॑ संॅवथ्स॒रे प्र॑ति॒ष्ठाय॑ प्र॒जामव॑ रुन्धते ऽतिरा॒त्राव॒भितो॑ भवतः प्र॒जायै॒ परि॑गृहीत्यै ॥ \newline

\textbf{Pada Paata} \newline

गा॒य॒त्रम् । ब्र॒ह्म॒व॒र्च॒समिति॑ ब्रह्म - व॒र्च॒सम् । गा॒य॒त्रि॒याम् । ए॒व । ब्र॒ह्म॒व॒र्च॒स इति॑ ब्रह्म - व॒र्च॒से । प्रतीति॑ । ति॒ष्ठ॒न्ति॒ । च॒तु॒श्च॒त्वा॒रिꣳ॒॒श इति॑ चतुः - च॒त्वा॒रिꣳ॒॒शः । भ॒व॒ति॒ । चतु॑श्चत्वारिꣳशदक्ष॒रेति॒ चतु॑श्चत्वारिꣳशत् - अ॒क्ष॒रा॒ । त्रि॒ष्टुक् । इ॒न्द्रि॒यम् । त्रि॒ष्टुप् । त्रि॒ष्टुभि॑ । ए॒व । इ॒न्द्रि॒ये । प्रतीति॑ । ति॒ष्ठ॒न्ति॒ । अ॒ष्टा॒च॒त्वा॒रिꣳ॒॒श इत्य॑ष्टा - च॒त्वा॒रिꣳ॒॒शः । भ॒व॒ति॒ । अ॒ष्टाच॑त्वारिꣳशदक्ष॒रेत्य॒ष्टाच॑त्वारिꣳशत् - अ॒क्ष॒रा॒ । जग॑ती । जाग॑ताः । प॒शवः॑ । जग॑त्याम् । ए॒व । प॒शुषु॑ । प्रतीति॑ । ति॒ष्ठ॒न्ति॒ । ए॒का॒द॒श॒रा॒त्र इत्ये॑कादश-रा॒त्रः । भ॒व॒ति॒ । पञ्च॑ । वै । ऋ॒तवः॑ । आ॒र्त॒वाः । पञ्च॑ । ऋ॒तुषु॑ । ए॒व । आ॒र्त॒वेषु॑ । सं॒ॅव॒थ्स॒र इति॑ सं - व॒थ्स॒रे । प्र॒ति॒ष्ठायेति॑ प्रति - स्थाय॑ । प्र॒जामिति॑ प्र - जाम् । अवेति॑ । रु॒न्ध॒ते॒ । अ॒ति॒रा॒त्रावित्य॑ति - रा॒त्रौ । अ॒भितः॑ । भ॒व॒तः॒ । प्र॒जाया॒ इति॑ प्र - जायै᳚ । परि॑गृहीत्या॒ इति॒ परि॑-गृ॒ही॒त्यै॒ ॥  \newline




\markright{ TS 7.2.7.1  \hfill https://www.vedavms.in \hfill}

\section{ TS 7.2.7.1 }

\textbf{TS 7.2.7.1 } \newline
\textbf{Samhita Paata} \newline

ऐ॒न्द्र॒वा॒य॒वाग्रा᳚न् गृह्णीया॒द्यः का॒मये॑त यथा पू॒र्वं प्र॒जाः क॑ल्पेर॒न्निति॑ य॒ज्ञ्स्य॒ वै क्लृप्ति॒मनु॑ प्र॒जाः क॑ल्पन्ते य॒ज्ञ्स्या-क्लृ॑प्ति॒मनु॒ न क॑ल्पन्ते यथा पू॒र्वमे॒व प्र॒जाः क॑ल्पयति॒ न ज्यायाꣳ॑सं॒ कनी॑या॒नति॑ क्रामत्यैन्द्रवाय॒वाग्रा᳚न् गृह्णीयादामया॒विनः॑ प्रा॒णेन॒ वा ए॒ष व्यृ॑द्ध्यते॒ यस्या॒ऽऽ*मय॑ति प्रा॒ण ऐ᳚न्द्रवाय॒वः प्रा॒णेनै॒वैनꣳ॒॒ सम॑र्द्धयति मैत्रावरु॒णाग्रा᳚न् गृह्णीर॒न्॒ येषां᳚ दीक्षि॒तानां᳚ प्र॒मीये॑त-[  ] \newline

\textbf{Pada Paata} \newline

ऐ॒न्द्र॒वा॒य॒वाग्रा॒नित्यै᳚न्द्रवाय॒व - अ॒ग्रा॒न् । गृ॒ह्णी॒या॒त् । यः । का॒मये॑त । य॒था॒पू॒र्वमिति॑ यथा - पू॒र्वम् । प्र॒जा इति॑ प्र - जाः । क॒ल्पे॒र॒न्न् । इति॑ । य॒ज्ञ्स्य॑ । वै । क्लृप्ति᳚म् । अन्विति॑ । प्र॒जा इति॑ प्र - जाः । क॒ल्प॒न्ते॒ । य॒ज्ञ्स्य॑ । अक्लृ॑प्तिम् । अन्विति॑ । न । क॒ल्प॒न्ते॒ । य॒था॒पू॒र्वमिति॑ यथा - पू॒र्वम् । ए॒व । प्र॒जा इति॑ प्र - जाः । क॒ल्प॒य॒ति॒ । न । ज्यायाꣳ॑सम् । कनी॑यान् । अतीति॑ । क्रा॒म॒ति॒ । ऐ॒न्द्र॒वा॒य॒वाग्रा॒नित्यै᳚न्द्रवाय॒व - अ॒ग्रा॒न् । गृ॒ह्णी॒या॒त् । आ॒म॒या॒विनः॑ । प्रा॒णेनेति॑ प्र - अ॒नेन॑ । वै । ए॒षः । वीति॑ । ऋ॒द्ध्य॒ते॒ । यस्य॑ । आ॒मय॑ति । प्रा॒ण इति॑ प्र - अ॒नः । ऐ॒न्द्र॒वा॒य॒व इत्यै᳚न्द्र - वा॒य॒वः । प्रा॒णेनेति॑ प्र - अ॒नेन॑ । ए॒व । ए॒न॒म् । समिति॑ । अ॒द्‌र्ध॒य॒ति॒ । मै॒त्रा॒व॒रु॒णाग्रा॒निति॑ मैत्रावरु॒ण - अ॒ग्रा॒न् । गृ॒ह्णी॒र॒न्न् । येषा᳚म् । दी॒क्षि॒ताना᳚म् । प्र॒मीये॒तेति॑ प्र - मीये॑त ।  \newline




\markright{ TS 7.2.7.2  \hfill https://www.vedavms.in \hfill}

\section{ TS 7.2.7.2 }

\textbf{TS 7.2.7.2 } \newline
\textbf{Samhita Paata} \newline

प्राणापा॒नाभ्यां॒ ॅवा ए॒ते व्यृ॑द्ध्यन्ते॒ येषां᳚ दीक्षि॒तानां᳚ प्र॒मीय॑ते प्राणापा॒नौ मि॒त्रावरु॑णौ प्राणापा॒नावे॒व मु॑ख॒तः परि॑ हरन्त आश्वि॒नाग्रा᳚न् गृह्णीता ऽऽ*नुजाव॒रो᳚ऽश्विनौ॒ वै दे॒वाना॑मानुजाव॒रौ प॒श्चेवाग्रं॒ पर्यै॑ता-म॒श्विना॑वे॒तस्य॑ दे॒वता॒ य आ॑नुजाव॒र-स्तावे॒वैन॒मग्रं॒ परि॑ णयतः शु॒क्राग्रा᳚न् गृह्णीत ग॒तश्रीः᳚ प्रति॒ष्ठाका॑मो॒ऽसौ वा आ॑दि॒त्यः शु॒क्र ए॒षोऽन्तोऽन्तं॑ मनु॒ष्यः॑ - [  ] \newline

\textbf{Pada Paata} \newline

प्रा॒णा॒पा॒नाभ्या॒मिति॑ प्राण - अ॒पा॒नाभ्या᳚म् । वै । ए॒ते । वीति॑ । ऋ॒द्ध्य॒न्ते॒ । येषा᳚म् । दी॒क्षि॒ताना᳚म् । प्र॒मीय॑त॒ इति॑ प्र - मीय॑ते । प्रा॒णा॒पा॒नाविति॑ प्राण - अ॒पा॒नौ । मि॒त्रावरु॑णा॒विति॑ मि॒त्रा - वरु॑णौ । प्रा॒णा॒पा॒नाविति॑ प्राण - अ॒पा॒नौ । ए॒व । मु॒ख॒तः । परीति॑ । ह॒र॒न्ते॒ । आ॒श्वि॒नाग्रा॒नित्या᳚श्वि॒न-अ॒ग्रा॒न् । गृ॒ह्णी॒त॒ । आ॒नु॒जा॒व॒र इत्या॑नु-जा॒व॒रः । अ॒श्विनौ᳚ । वै । दे॒वाना᳚म् । आ॒नु॒जा॒व॒रावित्या॑नु - जा॒व॒रौ । प॒श्चा । इ॒व॒ । अग्र᳚म् । परीति॑ । ऐ॒ता॒म् । अ॒श्विनौ᳚ । ए॒तस्य॑ । दे॒वता᳚ । यः । आ॒नु॒जा॒व॒र इत्या॑नु - जा॒व॒रः । तौ । ए॒व । ए॒न॒म् । अग्र᳚म् । परीति॑ । न॒य॒तः॒ । शु॒क्राग्रा॒निति॑ शु॒क्र - अ॒ग्रा॒न् । गृ॒ह्णी॒त॒ । ग॒तश्री॒रिति॑ ग॒त - श्रीः॒ । प्र॒ति॒ष्ठाका॑म॒ इति॑ प्रति॒ष्ठा - का॒मः॒ । अ॒सौ । वै । आ॒दि॒त्यः । शु॒क्रः । ए॒षः । अन्तः॑ । अन्त᳚म् । म॒नु॒ष्यः॑ ।  \newline




\markright{ TS 7.2.7.3  \hfill https://www.vedavms.in \hfill}

\section{ TS 7.2.7.3 }

\textbf{TS 7.2.7.3 } \newline
\textbf{Samhita Paata} \newline

श्रि॒यै ग॒त्वा नि व॑र्त॒ते ऽन्ता॑दे॒वाऽन्त॒मा र॑भते॒ न ततः॒ पापी॑यान् भवति मन्थ्य॑ग्रान् गृह्णीता-भि॒चर॑-न्नार्तपा॒त्रं ॅवा ए॒तद्-यन्-म॑न्थिपा॒त्रं मृ॒त्युनै॒वैनं॑ ग्राहयति ता॒जगार्ति॒मार्च्छ॑त्या-ग्रय॒णाग्रा᳚न् गृह्णीत॒ यस्य॑ पि॒ता पि॑ताम॒हः पुण्यः॒ स्यादथ॒ तन्न प्रा᳚प्नु॒याद्-वा॒चा वा ए॒ष इ॑न्द्रि॒येण॒ व्यृ॑द्ध्यते॒ यस्य॑ पि॒ता पि॑ताम॒हः पुण्यो॒ - [  ] \newline

\textbf{Pada Paata} \newline

श्रि॒यै । ग॒त्वा । नीति॑ । व॒र्त॒ते॒ । अन्ता᳚त् । ए॒व । अन्त᳚म् । एति॑ । र॒भ॒ते॒ । न । ततः॑ । पापी॑यान् । भ॒व॒ति॒ । म॒न्थ्य॑ग्रा॒निति॑ म॒न्थि-अ॒ग्रा॒न् । गृ॒ह्णी॒त॒ । अ॒भि॒चर॒न्नित्य॑भि - चरन्न्॑ । आ॒र्त॒पा॒त्रमित्या᳚र्त - पा॒त्रम् । वै । ए॒तत् । यत् । म॒न्थि॒पा॒त्रमिति॑ मन्थि - पा॒त्रम् । मृ॒त्युना᳚ । ए॒व । ए॒न॒म् । ग्रा॒ह॒य॒ति॒ । ता॒जक् । आर्ति᳚म् । एति॑ । ऋ॒च्छ॒ति॒ । आ॒ग्र॒य॒णाग्रा॒नित्या᳚ग्रय॒ण-अ॒ग्रा॒न् । गृ॒ह्णी॒त॒ । यस्य॑ । पि॒ता । पि॒ता॒म॒हः । पुण्यः॑ । स्यात् । अथ॑ । तत् । न । प्रा॒प्नु॒यादिति॑ प्र - आ॒प्नु॒यात् । वा॒चा । वै । ए॒षः । इ॒न्द्रि॒येण॑ । वीति॑ । ऋ॒द्ध्य॒ते॒ । यस्य॑ । पि॒ता । पि॒ता॒म॒हः । पुण्यः॑ ।  \newline




\markright{ TS 7.2.7.4  \hfill https://www.vedavms.in \hfill}

\section{ TS 7.2.7.4 }

\textbf{TS 7.2.7.4 } \newline
\textbf{Samhita Paata} \newline

भ॒वत्यथ॒ तन्न प्रा॒प्नोत्युर॑ इवै॒तद्-य॒ज्ञ्स्य॒ वागि॑व॒ यदा᳚ग्रय॒णो वा॒चैवैन॑मिन्द्रि॒येण॒ सम॑र्द्धयति॒ न ततः॒ पापी॑यान् भवत्यु॒क्थ्या᳚ग्रान् गृह्णीताभिच॒र्यमा॑णः॒ सर्वे॑षां॒ ॅवा ए॒तत् पात्रा॑णामिन्द्रि॒यं ॅयदु॑क्थ्यपा॒त्रꣳ सर्वे॑णै॒वैन॑मिन्द्रि॒येणाति॒ प्रयु॑ङ्क्ते॒ सर॑स्वत्य॒भि नो॑ नेषि॒ वस्य॒ इति॑ पुरो॒रुचं॑ कुर्या॒द्-वाग्वै - [  ] \newline

\textbf{Pada Paata} \newline

भव॑ति । अथ॑ । तत् । न । प्रा॒प्नोतीति॑ प्र - आ॒प्नोति॑ । उरः॑ । इ॒व॒ । ए॒तत् । य॒ज्ञ्स्य॑ । वाक् । इ॒व॒ । यत् । आ॒ग्र॒य॒णः । वा॒चा । ए॒व । ए॒न॒म् । इ॒न्द्रि॒येण॑ । समिति॑ । अ॒द्‌र्ध॒य॒ति॒ । न । ततः॑ । पापी॑यान् । भ॒व॒ति॒ । उ॒क्थ्या᳚ग्रा॒नित्यु॒क्थ्य॑ - अ॒ग्रा॒न् । गृ॒ह्णी॒त॒ । अ॒भि॒च॒र्यमा॑ण॒ इत्य॑भि - च॒र्यमा॑णः । सर्वे॑षाम् । वै । ए॒तत् । पात्रा॑णाम् । इ॒न्द्रि॒यम् । यत् । उ॒क्थ्य॒पा॒त्रमित्यु॑क्थ्य - पा॒त्रम् । सर्वे॑ण । ए॒व । ए॒न॒म् । इ॒न्द्रि॒येण॑ । अति॑ । प्रेति॑ । यु॒ङ्क्ते॒ । सर॑स्वति । अ॒भीति॑ । नः॒ । ने॒षि॒ । वस्यः॑ । इति॑ । पु॒रो॒रुच॒मिति॑ पुरः - रुच᳚म् । कु॒र्या॒त् । वाक् । वै ।  \newline




\markright{ TS 7.2.7.5  \hfill https://www.vedavms.in \hfill}

\section{ TS 7.2.7.5 }

\textbf{TS 7.2.7.5 } \newline
\textbf{Samhita Paata} \newline

सर॑स्वती वा॒चैवैन॒मति॒ प्रयु॑ङ्क्ते॒ मा त्वत् क्षेत्रा॒ण्यर॑णानि ग॒न्मेत्या॑ह मृ॒त्योर्वै क्षेत्रा॒ण्यर॑णानि॒ तेनै॒व मृ॒त्योः क्षेत्रा॑णि॒ न ग॑च्छति पू॒र्णान् ग्रहा᳚न् गृह्णीयादामया॒विनः॑ प्रा॒णान् वा ए॒तस्य॒ शुगृ॑च्छति॒ यस्या॒ ऽऽ*मय॑ति प्रा॒णा ग्रहाः᳚ प्रा॒णाने॒वास्य॑ शु॒चो मु॑ञ्चत्यु॒त यदी॒तासु॒र्भव॑ति॒ जीव॑त्ये॒व पू॒र्णान् ग्रहा᳚न् ( ) गृह्णीया॒द्-यर्.हि॑ प॒र्जन्यो॒ न वर्.षे᳚त् प्रा॒णान् वा ए॒तर्.हि॑ प्र॒जानाꣳ॒॒ शुगृ॑च्छति॒ यर्.हि॑ प॒र्जन्यो॒ न वर्.ष॑ति प्रा॒णा ग्रहाः᳚ प्रा॒णाने॒व प्र॒जानाꣳ॑ शु॒चो मु॑ञ्चति ता॒जक् प्र व॑र्.षति ॥ \newline

\textbf{Pada Paata} \newline

सर॑स्वती । वा॒चा । ए॒व । ए॒न॒म् । अति॑ । प्रेति॑ । यु॒ङ्क्ते॒ । मा । त्वत् । क्षेत्रा॑णि । अर॑णानि । ग॒न्म॒ । इति॑ । आ॒ह॒ । मृ॒त्योः । वै । क्षेत्रा॑णि । अर॑णानि । तेन॑ । ए॒व । मृ॒त्योः । क्षेत्रा॑णि । न । ग॒च्छ॒ति॒ । पू॒र्णान् । ग्रहान्॑ । गृ॒ह्णी॒या॒त् । आ॒म॒या॒विनः॑ । प्रा॒णानिति॑ प्र - अ॒नान् । वै । ए॒तस्य॑ । शुक् । ऋ॒च्छ॒ति॒ । यस्य॑ । आ॒मय॑ति । प्रा॒णा इति॑ प्र - अ॒नाः । ग्रहाः᳚ । प्रा॒णानिति॑ प्र-अ॒नान् । ए॒व । अ॒स्य॒ । शु॒चः । मु॒ञ्च॒ति॒ । उ॒त । यदि॑ । इ॒तासु॒रिती॒त - अ॒सुः॒ । भव॑ति । जीव॑ति । ए॒व । पू॒र्णान् । ग्रहान्॑ ( ) । गृ॒ह्णी॒या॒त् । यर्.हि॑ । प॒र्जन्यः॑ । न । वर्.ष᳚त् । प्रा॒णानिति॑ प्र - अ॒नान् । वै । ए॒तर्.हि॑ । प्र॒जाना॒मिति॑ प्र - जाना᳚म् । शुक् । ऋ॒च्छ॒ति॒ । यर्.हि॑ । प॒र्जन्यः॑ । न । वर्.ष॑ति । प्रा॒णा इति॑ प्र - अ॒नाः । ग्रहाः᳚ । प्रा॒णानिति॑ प्र - अ॒नान् । ए॒व । प्र॒जाना॒मिति॑ प्र - जाना᳚म् । शु॒चः । मु॒ञ्च॒ति॒ । ता॒जक् । प्रेति॑ । व॒र्.ष॒ति॒ ॥  \newline




\markright{ TS 7.2.8.1  \hfill https://www.vedavms.in \hfill}

\section{ TS 7.2.8.1 }

\textbf{TS 7.2.8.1 } \newline
\textbf{Samhita Paata} \newline

गा॒य॒त्रो वा ऐ᳚न्द्रवाय॒वो गा॑य॒त्रं प्रा॑य॒णीय॒-मह॒स्तस्मा᳚त् प्राय॒णीये-ऽह॑न्नैन्द्रवाय॒वो गृ॑ह्यते॒ स्व ए॒वैन॑मा॒यत॑ने गृह्णाति॒ त्रैष्टु॑भो॒ वै शु॒क्रस्त्रैष्टु॑भं द्वि॒तीय॒-मह॒स्तस्मा᳚द् द्वि॒तीयेऽह॑ञ्छु॒क्रो गृ॑ह्यते॒ स्व ए॒वैन॑मा॒यत॑ने गृह्णाति॒ जाग॑तो॒ वा आ᳚ग्रय॒णो जाग॑तं तृ॒तीय॒-मह॒स्तस्मा᳚त् तृ॒तीयेऽह॑न्नाग्रय॒णो गृ॑ह्यते॒ स्व ए॒वैन॑मा॒यत॑ने गृह्णात्ये॒तद्वै - [  ] \newline

\textbf{Pada Paata} \newline

गा॒य॒त्रः । वै । ऐ॒न्द्र॒वा॒य॒व इत्यै᳚न्द्र - वा॒य॒वः । गा॒य॒त्रम् । प्रा॒य॒णीय॒मिति॑ प्र - अ॒य॒नीय᳚म् । अहः॑ । तस्मा᳚त् । प्रा॒य॒णीय॒ इति॑ प्र - अ॒य॒नीये᳚ । अहन्न्॑ । ऐ॒न्द्र॒वा॒य॒व इत्यै᳚न्द्र - वा॒य॒वः । गृ॒ह्य॒ते॒ । स्वे । ए॒व । ए॒न॒म् । आ॒यत॑न॒ इत्या᳚ - यत॑ने । गृ॒ह्णा॒ति॒ । त्रैष्टु॑भः । वै । शु॒क्रः । त्रैष्टु॑भम् । द्वि॒तीय᳚म् । अहः॑ । तस्मा᳚त् । द्वि॒तीये᳚ । अहन्न्॑ । शु॒क्रः । गृ॒ह्य॒ते॒ । स्वे । ए॒व । ए॒न॒म् । आ॒यत॑न॒ इत्या᳚ - यत॑ने । गृ॒ह्णा॒ति॒ । जाग॑तः । वै । आ॒ग्र॒य॒णः । जाग॑तम् । तृ॒तीय᳚म् । अहः॑ । तस्मा᳚त् । तृ॒तीये᳚ । अहन्न्॑ । आ॒ग्र॒य॒णः । गृ॒ह्य॒ते॒ । स्वे । ए॒व । ए॒न॒म् । आ॒यत॑न॒ इत्या᳚ - यत॑ने । गृ॒ह्णा॒ति॒ । ए॒तत् । वै ।  \newline




\markright{ TS 7.2.8.2  \hfill https://www.vedavms.in \hfill}

\section{ TS 7.2.8.2 }

\textbf{TS 7.2.8.2 } \newline
\textbf{Samhita Paata} \newline

य॒ज्ञ्मा॑प॒द्-यच्छन्दाꣳ॑स्या॒प्नोति॒ यदा᳚ग्रय॒णः श्वो गृ॒ह्यते॒ यत्रै॒व य॒ज्ञ्मदृ॑श॒न् तत॑ ए॒वैनं॒ पुनः॒ प्रयु॑ङ्क्ते॒ जग॑न्मुखो॒ वै द्वि॒तीय॑स्त्रिरा॒त्रो जाग॑त आग्रय॒णो यच्च॑तु॒र्थेऽह॑न्नाग्रय॒णो गृ॒ह्यते॒ स्व ए॒वैन॑मा॒यत॑ने गृह्णा॒त्यथो॒ स्वमे॒व छन्दोऽनु॑ प॒र्याव॑र्तन्ते॒ राथ॑न्तरो॒ वा ऐ᳚न्द्रवाय॒वो राथ॑न्तरं पञ्च॒ममह॒स्तस्मा᳚त् पञ्च॒मेऽह॑ - [  ] \newline

\textbf{Pada Paata} \newline

य॒ज्ञ्म् । आ॒प॒त् । यत् । छन्दाꣳ॑सि । आ॒प्नोति॑ । यत् । आ॒ग्र॒य॒णः । श्वः । गृ॒ह्यते᳚ । यत्र॑ । ए॒व । य॒ज्ञ्म् । अदृ॑शन्न् । ततः॑ । ए॒व । ए॒न॒म् । पुनः॑ । प्रेति॑ । यु॒ङ्क्ते॒ । जग॑न्मुख॒ इति॒ जग॑त् - मु॒खः॒ । वै । द्वि॒तीयः॑ । त्रि॒रा॒त्र इति॑ त्रि - रा॒त्रः । जाग॑तः । आ॒ग्र॒य॒णः । यत् । च॒तु॒र्थे । अहन्न्॑ । आ॒ग्र॒य॒णः । गृ॒ह्यते᳚ । स्वे । ए॒व । ए॒न॒म् । आ॒यत॑न॒ इत्या᳚ - यत॑ने । गृ॒ह्णा॒ति॒ । अथो॒ इति॑ । स्वम् । ए॒व । छन्दः॑ । अन्विति॑ । प॒र्याव॑र्तन्त॒ इति॑ परि-आव॑र्तन्ते । राथ॑न्तर॒ इति॒ राथं᳚-त॒रः॒ । वै । ऐ॒न्द्र॒वा॒य॒व इत्यै᳚न्द्र - वा॒य॒वः । राथ॑न्तर॒मिति॒ राथं᳚ - त॒र॒म् । प॒ञ्च॒मम् । अहः॑ । तस्मा᳚त् । प॒ञ्च॒मे । अहन्न्॑ ।  \newline




\markright{ TS 7.2.8.3  \hfill https://www.vedavms.in \hfill}

\section{ TS 7.2.8.3 }

\textbf{TS 7.2.8.3 } \newline
\textbf{Samhita Paata} \newline

-न्नैन्द्रवाय॒वो गृ॑ह्यते॒ स्व ए॒वैन॑मा॒यत॑ने गृह्णाति॒ बार्.ह॑तो॒ वै शु॒क्रो बार्.ह॑तꣳ ष॒ष्ठमह॒स्तस्मा᳚थ् ष॒ष्ठेऽह॑ञ्छु॒क्रो गृ॑ह्यते॒ स्व ए॒वैन॑मा॒यत॑ने गृह्णात्ये॒तद्वै द्वि॒तीयं॑ ॅय॒ज्ञ्मा॑प॒द्-यच्छन्दाꣳ॑स्या॒प्नोति॒ यच्छु॒क्रः श्वो गृ॒ह्यते॒ यत्रै॒व य॒ज्ञ्मदृ॑श॒न् तत॑ ए॒वैनं॒ पुनः॒ प्रयु॑ङ्क्ते त्रि॒ष्टुङ्मु॑खो॒ वै तृ॒तीय॑स्त्रिरा॒त्रस्त्रैष्टु॑भः - [  ] \newline

\textbf{Pada Paata} \newline

ऐ॒न्द्र॒वा॒य॒व इत्यै᳚न्द्र - वा॒य॒वः । गृ॒ह्य॒ते॒ । स्वे । ए॒व । ए॒न॒म् । आ॒यत॑न॒ इत्या᳚ - यत॑ने । गृ॒ह्णा॒ति॒ । बार्.ह॑तः । वै । शु॒क्रः । बार्.ह॑तम् । ष॒ष्ठम् । अहः॑ । तस्मा᳚त् । ष॒ष्ठे । अहन्न्॑ । शु॒क्रः । गृ॒ह्य॒ते॒ । स्वे । ए॒व । ए॒न॒म् । आ॒यत॑न॒ इत्या᳚ - यत॑ने । गृ॒ह्णा॒ति॒ । ए॒तत् । वै । द्वि॒तीय᳚म् । य॒ज्ञ्म् । आ॒प॒त् । यत् । छन्दाꣳ॑सि । आ॒प्नोति॑ । यत् । शु॒क्रः । श्वः । गृ॒ह्यते᳚ । यत्र॑ । ए॒व । य॒ज्ञ्म् । अदृ॑शन्न् । ततः॑ । ए॒व । ए॒न॒म् । पुनः॑ । प्रेति॑ । यु॒ङ्क्ते॒ । त्रि॒ष्टुङ्मु॑ख॒ इति॑ त्रि॒ष्टुक् - मु॒खः॒ । वै । तृ॒तीयः॑ । त्रि॒रा॒त्र इति॑ त्रि - रा॒त्रः । त्रैष्टु॑भः ।  \newline




\markright{ TS 7.2.8.4  \hfill https://www.vedavms.in \hfill}

\section{ TS 7.2.8.4 }

\textbf{TS 7.2.8.4 } \newline
\textbf{Samhita Paata} \newline

शु॒क्रो यथ् स॑प्त॒मेऽह॑ञ्छु॒क्रो गृ॒ह्यते॒ स्व ए॒वैन॑मा॒यत॑ने गृह्णा॒त्यथो॒ स्वमे॒व छन्दोऽनु॑ प॒र्याव॑र्तन्ते॒ वाग्वा आ᳚ग्रय॒णो वाग॑ष्ट॒ममह॒-स्तस्मा॑दष्ट॒मेऽह॑न्नाग्रय॒णो गृ॑ह्यते॒ स्व ए॒वैन॑मा॒यत॑ने गृह्णाति प्रा॒णो वा ऐ᳚न्द्रवाय॒वः प्रा॒णो न॑व॒म-मह॒स्तस्मा᳚न्नव॒मे ऽह॑न्नैन्द्रवाय॒वो गृ॑ह्यते॒ स्व ए॒वैन॑मा॒यत॑ने गृह्णात्ये॒त - [  ] \newline

\textbf{Pada Paata} \newline

शु॒क्रः । यत् । स॒प्त॒मे । अहन्न्॑ । शु॒क्रः । गृ॒ह्यते᳚ । स्वे । ए॒व । ए॒न॒म् । आ॒यत॑न॒ इत्या᳚ - यत॑ने । गृ॒ह्णा॒ति॒ । अथो॒ इति॑ । स्वम् । ए॒व । छन्दः॑ । अन्विति॑ । प॒र्याव॑र्तन्त॒ इति॑ परि - आव॑र्तन्ते । वाक् । वै । आ॒ग्र॒य॒णः । वाक् । अ॒ष्ट॒मम् । अहः॑ । तस्मा᳚त् । अ॒ष्ट॒मे । अहन्न्॑ । आ॒ग्र॒य॒णः । गृ॒ह्य॒ते॒ । स्वे । ए॒व । ए॒न॒म् । आ॒यत॑न॒ इत्या᳚ - यत॑ने । गृ॒ह्णा॒ति॒ । प्रा॒ण इति॑ प्र - अ॒नः । वै । ऐ॒न्द्र॒वा॒य॒व इत्यै᳚न्द्र - वा॒य॒वः । प्रा॒ण इति॑ प्र - अ॒नः । न॒व॒मम् । अहः॑ । तस्मा᳚त् । न॒व॒मे । अहन्न्॑ । ऐ॒न्द्र॒वा॒य॒व इत्यै᳚न्द्र - वा॒य॒वः । गृ॒ह्य॒ते॒ । स्वे । ए॒व । ए॒न॒म् । आ॒यत॑न॒ इत्या᳚ - यत॑ने । गृ॒ह्णा॒ति॒ । ए॒तत् ।  \newline




\markright{ TS 7.2.8.5  \hfill https://www.vedavms.in \hfill}

\section{ TS 7.2.8.5 }

\textbf{TS 7.2.8.5 } \newline
\textbf{Samhita Paata} \newline

-द्वै तृ॒तीयं॑ ॅय॒ज्ञ्मा॑प॒द्-यच्छन्दाꣳ॑स्या॒प्नोति॒ यदै᳚न्द्रवाय॒वः श्वो गृ॒ह्यते॒ यत्रै॒व य॒ज्ञ्मदृ॑श॒न् तत॑ ए॒वैनं॒ पुनः॒ प्रयु॒ङ्क्ते ऽथो॒ स्वमे॒व छन्दोऽनु॑ प॒र्याव॑र्तन्ते प॒थो वा ए॒तेऽद्ध्यप॑थेन यन्ति॒ ये᳚ऽन्येनै᳚न्द्रवाय॒वात् प्र॑ति॒पद्य॒न्तेऽन्तः॒ खलु॒ वा ए॒ष य॒ज्ञ्स्य॒ यद्-द॑श॒म-मह॑र्दश॒मे ऽह॑न्नैन्द्रवाय॒वो गृ॑ह्यते य॒ज्ञ्स्यै॒-[  ] \newline

\textbf{Pada Paata} \newline

वै । तृ॒तीय᳚म् । य॒ज्ञ्म् । आ॒प॒त् । यत् । छन्दाꣳ॑सि । आ॒प्नोति॑ । यत् । ऐ॒न्द्र॒वा॒य॒व इत्यै᳚न्द्र - वा॒य॒वः । श्वः । गृ॒ह्यते᳚ । यत्र॑ । ए॒व । य॒ज्ञ्म् । अदृ॑शन्न् । ततः॑ । ए॒व । ए॒न॒म् । पुनः॑ । प्रेति॑ । यु॒ङ्क्ते॒ । अथो॒ इति॑ । स्वम् । ए॒व । छन्दः॑ । अन्विति॑ । प॒र्याव॑र्तन्त॒ इति॑ परि - आव॑र्तन्ते । प॒थः । वै । ए॒ते । अधीति॑ । अप॑थेन । य॒न्ति॒ । ये । अ॒न्येन॑ । ऐ॒न्द्र॒वा॒य॒वादित्यै᳚न्द्र - वा॒य॒वात् । प्र॒ति॒पद्य॑न्त॒ इति॑ प्रति - पद्य॑न्ते । अन्तः॑ । खलु॑ । वै । ए॒षः । य॒ज्ञ्स्य॑ । यत् । द॒श॒मम् । अहः॑ । द॒श॒मे । अहन्न्॑ । ऐ॒न्द्र॒वा॒य॒व इत्यै᳚न्द्र - वा॒य॒वः । गृ॒ह्य॒ते॒ । य॒ज्ञ्स्य॑ ।  \newline




\markright{ TS 7.2.8.6  \hfill https://www.vedavms.in \hfill}

\section{ TS 7.2.8.6 }

\textbf{TS 7.2.8.6 } \newline
\textbf{Samhita Paata} \newline

-वान्तं॑ ग॒त्वा ऽप॑था॒त् पन्था॒मपि॑ य॒न्त्यथो॒ यथा॒ वही॑यसा प्रति॒सारं॒ ॅवह॑न्ति ता॒दृगे॒व तच्छन्दाꣳ॑स्य॒न्यो᳚ऽन्यस्य॑ लो॒कम॒भ्य॑द्ध्याय॒न् तान्ये॒तेनै॒व दे॒वा व्य॑वाहयन्नैन्द्रवाय॒वस्य॒ वा ए॒तदा॒यत॑नं॒ ॅयच्च॑तु॒र्थमह॒स्तस्मि॑-न्नाग्रय॒णो गृ॑ह्यते॒ तस्मा॑दाग्रय॒णस्या॒ ऽऽ*यत॑ने नव॒मेऽह॑न्नैन्द्रवाय॒वो गृ॑ह्यते शु॒क्रस्य॒ वा ए॒तदा॒यत॑नं॒ ॅयत् प॑ञ्च॒म - [  ] \newline

\textbf{Pada Paata} \newline

ए॒व । अन्त᳚म् । ग॒त्वा । अप॑थात् । पन्था᳚म् । अपीति॑ । य॒न्ति॒ । अथो॒ इति॑ । यथा᳚ । वही॑यसा । प्र॒ति॒सार॒मिति॑ प्रति-सार᳚म् । वह॑न्ति । ता॒दृक् । ए॒व । तत् । छन्दाꣳ॑सि । अ॒न्यः । अ॒न्यस्य॑ । लो॒कम् । अ॒भीति॑ । अ॒द्ध्या॒य॒न्न् । तानि॑ । ए॒तेन॑ । ए॒व । दे॒वाः । वीति॑ । अ॒वा॒ह॒य॒न्न् । ऐ॒न्द्र॒वा॒य॒वस्येत्यै᳚न्द्र - वा॒य॒वस्य॑ । वै । ए॒तत् । आ॒यत॑न॒मित्या᳚ - यत॑नम् । यत् । च॒तु॒र्थम् । अहः॑ । तस्मिन्न्॑ । आ॒ग्र॒य॒णः । गृ॒ह्य॒ते॒ । तस्मा᳚त् । आ॒ग्र॒य॒णस्य॑ । आ॒यत॑न॒ इत्या᳚-यत॑ने । न॒व॒मे । अहन्न्॑ । ऐ॒न्द्र॒वा॒य॒व इत्यै᳚न्द्र - वा॒य॒वः । गृ॒ह्य॒ते॒ । शु॒क्रस्य॑ । वै । ए॒तत् । आ॒यत॑न॒मित्या᳚ - यत॑नम् । यत् । प॒ञ्च॒मम् ।  \newline




\markright{ TS 7.2.8.7  \hfill https://www.vedavms.in \hfill}

\section{ TS 7.2.8.7 }

\textbf{TS 7.2.8.7 } \newline
\textbf{Samhita Paata} \newline

-मह॒स्तस्मि॑न्नैन्द्रवाय॒वो गृ॑ह्यते॒ तस्मा॑दैन्द्रवाय॒वस्या॒ऽऽ*यत॑ने सप्त॒मेऽह॑ञ्छु॒क्रो गृ॑ह्यत आग्रय॒णस्य॒ वा ए॒तदा॒यत॑नं ॅयथ् ष॒ष्ठमह॒स्तस्मि॑ञ्छु॒क्रो गृ॑ह्यते॒ तस्मा᳚च्छु॒क्रस्या॒ ऽऽ*यत॑नेऽष्ट॒मेऽह॑न्नाग्रय॒णो गृ॑ह्यते॒ छन्दाꣳ॑स्ये॒व तद्वि वा॑हयति॒ प्र वस्य॑सो विवा॒हमा᳚प्नोति॒ य ए॒वं ॅवेदाथो॑ दे॒वता᳚भ्य ए॒व य॒ज्ञे सं॒ॅविदं॑ दधाति॒ तस्मा॑दि॒द-म॒न्यो᳚ऽन्यस्मै॑ ( ) ददाति ॥ \newline

\textbf{Pada Paata} \newline

अहः॑ । तस्मिन्न्॑ । ऐ॒न्द्र॒वा॒य॒व इत्यै᳚न्द्र - वा॒य॒वः । गृ॒ह्य॒ते॒ । तस्मा᳚त् । ऐ॒न्द्र॒वा॒य॒वस्येत्यै᳚न्द्र - वा॒य॒वस्य॑ । आ॒यत॑न॒ इत्या᳚-यत॑ने । स॒प्त॒मे । अहन्न्॑ । शु॒क्रः । गृ॒ह्य॒ते॒ । आ॒ग्र॒य॒णस्य॑ । वै । ए॒तत् । आ॒यत॑न॒मित्या᳚ - यत॑नम् । यत् । ष॒ष्ठम् । अहः॑ । तस्मिन्न्॑ । शु॒क्रः । गृ॒ह्य॒ते॒ । तस्मा᳚त् । शु॒क्रस्य॑ । आ॒यत॑न॒ इत्या᳚ - यत॑ने । अ॒ष्ट॒मे । अहन्न्॑ । आ॒ग्र॒य॒णः । गृ॒ह्य॒ते॒ । छन्दाꣳ॑सि । ए॒व । तत् । वीति॑ । वा॒ह॒य॒ति॒ । प्रेति॑ । वस्य॑सः । वि॒वा॒हमिति॑ वि-वा॒हम् । आ॒प्नो॒ति॒ । यः । ए॒वम् । वेद॑ । अथो॒ इति॑ । दे॒वता᳚भ्यः । ए॒व । य॒ज्ञे । सं॒ॅविद॒मिति॑ सं - विद᳚म् । द॒धा॒ति॒ । तस्मा᳚त् । इ॒दम् । अ॒न्यः । अ॒न्यस्मै᳚ ( ) । द॒दा॒ति॒ ॥  \newline




\markright{ TS 7.2.9.1  \hfill https://www.vedavms.in \hfill}

\section{ TS 7.2.9.1 }

\textbf{TS 7.2.9.1 } \newline
\textbf{Samhita Paata} \newline

प्र॒जाप॑तिरकामयत॒ प्र जा॑ये॒येति॒ स ए॒तं द्वा॑दशरा॒त्रम॑पश्य॒त् तमाऽह॑र॒त् तेना॑यजत॒ ततो॒ वै स प्राजा॑यत॒ यः का॒मये॑त॒ प्र जा॑ये॒येति॒ स द्वा॑दशरा॒त्रेण॑ यजेत॒ प्रैव जा॑यते ब्रह्मवा॒दिनो॑ वदन्त्यग्निष्टो॒मप्रा॑यणा य॒ज्ञा अथ॒ कस्मा॑दतिरा॒त्रः पूर्वः॒ प्र यु॑ज्यत॒ इति॒ चक्षु॑षी॒ वा ए॒ते य॒ज्ञ्स्य॒ यद॑तिरा॒त्रौ क॒नीनि॑के अग्निष्टो॒मौ य - [  ] \newline

\textbf{Pada Paata} \newline

प्र॒जाप॑ति॒रिति॑ प्र॒जा - प॒तिः॒ । अ॒का॒म॒य॒त॒ । प्रेति॑ । जा॒ये॒य॒ । इति॑ । सः । ए॒तम् । द्वा॒द॒श॒रा॒त्रमिति॑ द्वादश - रा॒त्रम् । अ॒प॒श्य॒त् । तम् । एति॑ । अ॒ह॒र॒त् । तेन॑ । अ॒य॒ज॒त॒ । ततः॑ । वै । सः । प्रेति॑ । अ॒जा॒य॒त॒ । यः । का॒मये॑त । प्रेति॑ । जा॒ये॒य॒ । इति॑ । सः । द्वा॒द॒श॒रा॒त्रेणेति॑ द्वादश - रा॒त्रेण॑ । य॒जे॒त॒ । प्रेति॑ । ए॒व । जा॒य॒ते॒ । ब्र॒ह्म॒वा॒दिन॒ इति॑ ब्रह्म-वा॒दिनः॑ । व॒द॒न्ति॒ । अ॒ग्नि॒ष्टो॒मप्रा॑यणा॒ इत्य॑ग्निष्टो॒म - प्रा॒य॒णाः॒ । य॒ज्ञाः । अथ॑ । कस्मा᳚त् । अ॒ति॒रा॒त्र इत्य॑ति - रा॒त्रः । पूर्वः॑ । प्रेति॑ । यु॒ज्य॒ते॒ । इति॑ । चक्षु॑षी॒ इति॑ । वै । ए॒ते इति॑ । य॒ज्ञ्स्य॑ । यत् । अ॒ति॒रा॒त्रावित्य॑ति - रा॒त्रौ । क॒नीनि॑के॒ इति॑ । अ॒ग्नि॒ष्टो॒मावित्य॑ग्नि- स्तो॒मौ । यत् ।  \newline




\markright{ TS 7.2.9.2  \hfill https://www.vedavms.in \hfill}

\section{ TS 7.2.9.2 }

\textbf{TS 7.2.9.2 } \newline
\textbf{Samhita Paata} \newline

-द॑ग्निष्टो॒मं पूर्वं॑ प्रयुञ्जी॒रन् ब॑हि॒र्द्धा क॒नीनि॑के दद्ध्यु॒स्तस्मा॑दतिरा॒त्रः पूर्वः॒ प्र यु॑ज्यते॒ चक्षु॑षी ए॒व य॒ज्ञे धि॒त्वा म॑द्ध्य॒तः क॒नीनि॑के॒ प्रति॑ दधति॒ यो वै गा॑य॒त्रीं ज्योतिः॑पक्षां॒ ॅवेद॒ ज्योति॑षा भा॒सा सु॑व॒र्गं ॅलो॒कमे॑ति॒ याव॑ग्निष्टो॒मौ तौ प॒क्षौ येऽन्त॑रे॒ऽष्टावु॒क्थ्याः᳚ स आ॒त्मैषा वै गा॑य॒त्री ज्योतिः॑पक्षा॒ य ए॒वं ॅवेद॒ ज्योति॑षा भा॒सा सु॑व॒र्गं ॅलो॒क - [  ] \newline

\textbf{Pada Paata} \newline

अ॒ग्नि॒ष्टो॒ममित्य॑ग्नि-स्तो॒मम् । पूर्व᳚म् । प्र॒यु॒ञ्जी॒रन्निति॑ प्र-यु॒ञ्जी॒रन्न् । ब॒हि॒द्‌र्धेति॑ बहिः - धा । क॒नीनि॑के॒ इति॑ । द॒द्ध्युः॒ । तस्मा᳚त् । अ॒ति॒रा॒त्र इत्य॑ति - रा॒त्रः । पूर्वः॑ । प्रेति॑ । यु॒ज्य॒ते॒ । चक्षु॑षी॒ इति॑ । ए॒व । य॒ज्ञे । धि॒त्वा । म॒द्ध्य॒तः । क॒नीनि॑के॒ इति॑ । प्रतीति॑ । द॒ध॒ति॒ । यः । वै । गा॒य॒त्रीम् । ज्योतिः॑पक्षा॒मिति॒ ज्योतिः॑ - प॒क्षा॒म् । वेद॑ । ज्योति॑षा । भा॒सा । सु॒व॒र्गमिति॑ सुवः - गम् । लो॒कम् । ए॒ति॒ । यौ । अ॒ग्नि॒ष्टो॒मावित्य॑ग्नि - स्तो॒मौ । तौ । प॒क्षौ । ये । अन्त॑रे । अ॒ष्टौ । उ॒क्थ्याः᳚ । सः । आ॒त्मा । ए॒षा । वै । गा॒य॒त्री । ज्योतिः॑प॒क्षेति॒ ज्योतिः॑ - प॒क्षा॒ । यः । ए॒वम् । वेद॑ । ज्योति॑षा । भा॒सा । सु॒व॒र्गमिति॑ सुवः - गम् । लो॒कम् ।  \newline




\markright{ TS 7.2.9.3  \hfill https://www.vedavms.in \hfill}

\section{ TS 7.2.9.3 }

\textbf{TS 7.2.9.3 } \newline
\textbf{Samhita Paata} \newline

-मे॑ति प्र॒जाप॑ति॒र्वा ए॒ष द्वा॑दश॒धा विहि॑तो॒ यद् द्वा॑दशरा॒त्रो याव॑तिरा॒त्रो तौ प॒क्षौ येऽन्त॑रे॒ऽष्टावु॒क्थ्याः᳚ स आ॒त्मा प्र॒जाप॑ति॒र्वावैष सन्थ्सद्ध॒ वै स॒त्रेण॑ स्पृणोति प्रा॒णा वै सत् प्रा॒णाने॒व स्पृ॑णोति॒ सर्वा॑सां॒ ॅवा ए॒ते प्र॒जानां᳚ प्रा॒णैरा॑सते॒ ये स॒त्रमास॑ते॒ तस्मा᳚त् पृच्छन्ति॒ किमे॒ते स॒त्रिण॒ इति॑ प्रि॒यः प्र॒जाना॒ ( ) मुत्थि॑तो भवति॒ य ए॒वं ॅवेद॑ ॥ \newline

\textbf{Pada Paata} \newline

ए॒ति॒ । प्र॒जाप॑ति॒रिति॑ प्र॒जा - प॒तिः॒ । वै । ए॒षः । द्वा॒द॒श॒धेति॑ द्वादश - धा । विहि॑त॒ इति॒ वि - हि॒तः॒ । यत् । द्वा॒द॒श॒रा॒त्र इति॑ द्वादश - रा॒त्रः । यौ । अ॒ति॒रा॒त्रावित्य॑ति - रा॒त्रौ । तौ । प॒क्षौ । ये । अन्त॑रे । अ॒ष्टौ । उ॒क्थ्याः᳚ । सः । आ॒त्मा । प्र॒जाप॑ति॒रिति॑ प्र॒जा-प॒तिः॒ । वाव । ए॒षः । सन्न् । सत् । ह॒ । वै । स॒त्रेण॑ । स्पृ॒णो॒ति॒ । प्रा॒णा इति॑ प्र - अ॒नाः । वै । सत् । प्रा॒णानिति॑ प्र - अ॒नान् । ए॒व । स्पृ॒णो॒ति॒ । सर्वा॑साम् । वै । ए॒ते । प्र॒जाना॒मिति॑ प्र - जाना᳚म् । प्रा॒णैरिति॑ प्र - अ॒नैः । आ॒स॒ते॒ । ये । स॒त्रम् । आस॑ते । तस्मा᳚त् । पृ॒च्छ॒न्ति॒ । किम् । ए॒ते । स॒त्रिणः॑ । इति॑ । प्रि॒यः । प्र॒जाना॒मिति॑ प्र-जाना᳚म् ( ) । उत्थि॑त॒ इत्युत् - स्थि॒तः॒ । भ॒व॒ति॒ । यः । ए॒वम् । वेद॑ ॥  \newline




\markright{ TS 7.2.10.1  \hfill https://www.vedavms.in \hfill}

\section{ TS 7.2.10.1 }

\textbf{TS 7.2.10.1 } \newline
\textbf{Samhita Paata} \newline

न वा ए॒षो᳚ऽन्यतो॑वैश्वानरः सुव॒र्गाय॑ लो॒काय॒ प्राभ॑वदू॒र्द्ध्वो ह॒ वा ए॒ष आत॑त आसी॒त् ते दे॒वा ए॒तं ॅवै᳚श्वान॒रं पर्यौ॑हन्थ् सुव॒र्गस्य॑ लो॒कस्य॒ प्रभू᳚त्या ऋ॒तवो॒ वा ए॒तेन॑ प्र॒जाप॑तिमयाजय॒न् तेष्वा᳚र्द्ध्नो॒दधि॒ तदृ॒द्ध्नोति॑ ह॒ वा ऋ॒त्विक्षु॒ य ए॒वं ॅवि॒द्वान् द्वा॑दशा॒हेन॒ यज॑ते॒ ते᳚ऽस्मिन्नैच्छन्त॒ स रस॒मह॑ वस॒न्ताय॒ प्राय॑च्छ॒ - [  ] \newline

\textbf{Pada Paata} \newline

न । वै । ए॒षः । अ॒न्यतो॑वैश्वानर॒ इत्य॒न्यतः॑ - वै॒श्वा॒न॒रः॒ । सु॒व॒र्गायेति॑ सुवः - गाय॑ । लो॒काय॑ । प्रेति॑ । अ॒भ॒व॒त् । ऊ॒द्‌र्ध्वः । ह॒ । वै । ए॒षः । आत॑त॒ इत्या - त॒तः॒ । आ॒सी॒त् । ते । दे॒वाः । ए॒तम् । वै॒श्वा॒न॒रम् । परीति॑ । औ॒ह॒न्न् । सु॒व॒र्गस्येति॑ सुवः - गस्य॑ । लो॒कस्य॑ । प्रभू᳚त्या॒ इति॒ प्र-भू॒त्यै॒ । ऋ॒तवः॑ । वै । ए॒तेन॑ । प्र॒जाप॑ति॒मिति॑ प्र॒जा-प॒ति॒म् । अ॒या॒ज॒य॒न्न् । तेषु॑ । आ॒द्‌र्ध्नो॒त् । अधीति॑ । तत् । ऋ॒द्ध्नोति॑ । ह॒ । वै । ऋ॒त्विक्षु॑ । यः । ए॒वम् । वि॒द्वान् । द्वा॒द॒शा॒हेनेति॑ द्वादश - अ॒हेन॑ । यज॑ते । ते । अ॒स्मि॒न्न् । ऐ॒च्छ॒न्त॒ । सः । रस᳚म् । अह॑ । व॒स॒न्ताय॑ । प्रेति॑ । अय॑च्छत् ।  \newline




\markright{ TS 7.2.10.2  \hfill https://www.vedavms.in \hfill}

\section{ TS 7.2.10.2 }

\textbf{TS 7.2.10.2 } \newline
\textbf{Samhita Paata} \newline

द्यवं॑ ग्री॒ष्मायौष॑धीर्व॒र्॒.षाभ्यो᳚ व्री॒हीञ्छ॒रदे॑ माषति॒लौ हे॑मन्तशिशि॒राभ्यां॒ तेनेन्द्रं॑ प्र॒जाप॑तिरयाजय॒त् ततो॒ वा इन्द्र॒ इन्द्रो॑ऽभव॒त् तस्मा॑दाहुरानुजाव॒रस्य॑ य॒ज्ञ् इति॒ स ह्ये॑तेनाऽग्रेऽय॑जतै॒ष ह॒ वै कु॒णप॑मत्ति॒ यः स॒त्रे प्र॑तिगृ॒ह्णाति॑ पुरुषकुण॒पम॑श्वकुण॒पं गौर्वा अन्नं॒ ॅयेन॒ पात्रे॒णान्नं॒ बिभ्र॑ति॒ यत् तन्न नि॒र्णेनि॑जति॒ ततोऽधि॒ - [  ] \newline

\textbf{Pada Paata} \newline

यव᳚म् । ग्री॒ष्माय॑ । ओष॑धीः । व॒र्॒.षाभ्यः॑ । व्री॒हीन् । श॒रदे᳚ । मा॒ष॒ति॒लाविति॑ माष - ति॒लौ । हे॒म॒न्त॒शि॒शि॒राभ्या॒मिति॑ हेमन्त - शि॒शि॒राभ्या᳚म् । तेन॑ । इन्द्र᳚म् । प्र॒जाप॑ति॒रिति॑ प्र॒जा-प॒तिः॒ । अ॒या॒ज॒य॒त् । ततः॑ । वै । इन्द्रः॑ । इन्द्रः॑ । अ॒भ॒व॒त् । तस्मा᳚त् । आ॒हुः॒ । आ॒नु॒जा॒व॒रस्येत्या॑नु - जा॒व॒रस्य॑ । य॒ज्ञ्ः । इति॑ । सः । हि । ए॒तेन॑ । अग्रे᳚ । अय॑जत । ए॒षः । ह॒ । वै । कु॒णप᳚म् । अ॒त्ति॒ । यः । स॒त्रे । प्र॒ति॒गृ॒ह्णातीति॑ प्रति - गृ॒ह्णाति॑ । पु॒रु॒ष॒कु॒ण॒पमिति॑ पुरुष - कु॒ण॒पम् । अ॒श्व॒कु॒ण॒पमित्य॑श्व - कु॒ण॒पम् । गौः । वै । अन्न᳚म् । येन॑ । पात्रे॑ण । अन्न᳚म् । बिभ्र॑ति । यत् । तत् । न । नि॒र्णेनि॑ज॒तीति॑ निः - नेनि॑जति । ततः॑ । अधीति॑ ।  \newline




\markright{ TS 7.2.10.3  \hfill https://www.vedavms.in \hfill}

\section{ TS 7.2.10.3 }

\textbf{TS 7.2.10.3 } \newline
\textbf{Samhita Paata} \newline

मलं॑ जायत॒ एक॑ ए॒व य॑जे॒तैको॒ हि प्र॒जाप॑ति॒रार्द्ध्नो॒द् द्वाद॑श॒ रात्री᳚र्दीक्षि॒तः स्या॒द् द्वाद॑श॒ मासाः᳚ संॅवथ्स॒रः सं॑ॅवथ्स॒रः प्र॒जाप॑तिः प्र॒जाप॑ति॒र्वावैष ए॒ष ह॒ त्वै जा॑यते॒ यस्तप॒सोऽधि॒ जाय॑ते चतु॒र्द्धा वा ए॒तास्ति॒स्रस्ति॑स्रो॒ रात्र॑यो॒ यद् द्वाद॑शोप॒सदो॒ याः प्र॑थ॒मा य॒ज्ञ्ं ताभिः॒ सं भ॑रति॒ या द्वि॒तीया॑ य॒ज्ञ्ं ताभि॒रा र॑भते॒ - [  ] \newline

\textbf{Pada Paata} \newline

मल᳚म् । जा॒य॒ते॒ । एकः॑ । ए॒व । य॒जे॒त॒ । एकः॑ । हि । प्र॒जाप॑ति॒रिति॑ प्र॒जा - प॒तिः॒ । आद्‌र्ध्नो᳚त् । द्वाद॑श । रात्रीः᳚ । दी॒क्षि॒तः । स्या॒त् । द्वाद॑श । मासाः᳚ । सं॒ॅव॒थ्स॒र इति॑ सं - व॒थ्स॒रः । सं॒ॅव॒थ्स॒र इति॑ सं - व॒थ्स॒रः । प्र॒जाप॑ति॒रिति॑ प्र॒जा - प॒तिः॒ । प्र॒जाप॑ति॒रिति॑ प्र॒जा - प॒तिः॒ । वाव । ए॒षः । ए॒षः । ह॒ । तु । वै । जा॒य॒ते॒ । यः । तप॑सः । अधीति॑ । जाय॑ते । च॒तु॒द्‌र्धेति॑ चतुः - धा । वै । ए॒ताः । ति॒स्रस्ति॑स्र॒ इति॑ ति॒स्रः-ति॒स्रः॒ । रात्र॑यः । यत् । द्वाद॑श । उ॒प॒सद॒ इत्यु॑प - सदः॑ । याः । प्र॒थ॒माः । य॒ज्ञ्म् । ताभिः॑ । समिति॑ । भ॒र॒ति॒ । याः । द्वि॒तीयाः᳚ । य॒ज्ञ्म् । ताभिः॑ । एति॑ । र॒भ॒ते॒ ।  \newline




\markright{ TS 7.2.10.4  \hfill https://www.vedavms.in \hfill}

\section{ TS 7.2.10.4 }

\textbf{TS 7.2.10.4 } \newline
\textbf{Samhita Paata} \newline

यास्तृ॒तीयाः॒ पात्रा॑णि॒ ताभि॒र्निर्णे॑निक्ते॒ याश्च॑तु॒र्थीरपि॒ ताभि॑रा॒त्मान॑मन्तर॒तः शु॑न्धते॒ यो वा अ॑स्य प॒शुमत्ति॑ माꣳ॒॒सꣳ सो᳚ऽत्ति॒ यः पु॑रो॒डाशं॑ म॒स्तिष्कꣳ॒॒ स यः प॑रिवा॒पं पुरी॑षꣳ॒॒ स य आज्यं॑ म॒ज्जानꣳ॒॒ स यः सोमꣳ॒॒ स्वेदꣳ॒॒ सोऽपि॑ ह॒ वा अ॑स्य शीर्.ष॒ण्या॑ नि॒ष्पदः॒ प्रति॑ गृह्णाति॒ यो द्वा॑दशा॒हे प्र॑तिगृ॒ह्णाति॒ तस्मा᳚द् द्वादशा॒हेन॒ ( ) न याज्यं॑ पा॒प्मनो॒ व्यावृ॑त्त्यै ॥ \newline

\textbf{Pada Paata} \newline

याः । तृ॒तीयाः᳚ । पात्रा॑णि । ताभिः॑ । निरिति॑ । ने॒नि॒क्ते॒ । याः । च॒तु॒र्थीः । अपीति॑ । ताभिः॑ । आ॒त्मान᳚म् । अ॒न्त॒र॒तः । शु॒न्ध॒ते॒ । यः । वै । अ॒स्य॒ । प॒शुम् । अत्ति॑ । माꣳ॒॒सम् । सः । अ॒त्ति॒ । यः । पु॒रो॒डाश᳚म् । म॒स्तिष्क᳚म् । सः । यः । प॒रि॒वा॒पमिति॑ परि - वा॒पम् । पुरी॑षम् । सः । यः । आज्य᳚म् । म॒ज्जान᳚म् । सः । यः । सोम᳚म् । स्वेद᳚म् । सः । अपीति॑ । ह॒ । वै । अ॒स्य॒ । शी॒र्.॒ष॒ण्याः᳚ । नि॒ष्पद॒ इति॑ निः - पदः॑ । प्रतीति॑ । गृ॒ह्णा॒ति॒ । यः । द्वा॒द॒शा॒ह इति॑ द्वादश-अ॒हे । प्र॒ति॒गृ॒ह्णातीति॑ प्रति - गृ॒ह्णाति॑ । तस्मा᳚त् । द्वा॒द॒शा॒हेनेति॑ द्वादश - अ॒हेन॑ ( ) । न । याज्य᳚म् । पा॒प्मनः॑ । व्यावृ॑त्त्या॒ इति॑ वि - आवृ॑त्त्यै ॥  \newline




\markright{ TS 7.2.11.1  \hfill https://www.vedavms.in \hfill}

\section{ TS 7.2.11.1 }

\textbf{TS 7.2.11.1 } \newline
\textbf{Samhita Paata} \newline

एक॑स्मै॒ स्वाहा॒ द्वाभ्याꣳ॒॒ स्वाहा᳚ त्रि॒भ्यः स्वाहा॑ च॒तुर्भ्यः॒ स्वाहा॑ प॒ञ्चभ्यः॒ स्वाहा॑ ष॒ड्भ्यः स्वाहा॑ स॒प्तभ्यः॒ स्वाहा᳚ ऽष्टा॒भ्यः स्वाहा॑ न॒वभ्यः॒ स्वाहा॑ द॒शभ्यः॒ स्वाहै॑ -काद॒शभ्यः॒ स्वाहा᳚ द्वाद॒शभ्यः॒ स्वाहा᳚ त्रयोद॒शभ्यः॒ स्वाहा॑ चतुर्द॒शभ्यः॒ स्वाहा॑ पञ्चद॒शभ्यः॒ स्वाहा॑ षोड॒शभ्यः॒ स्वाहा॑ सप्तद॒शभ्यः॒ स्वाहा᳚ ऽष्टाद॒शभ्यः॒ स्वाहै-का॒न्न विꣳ॑श॒त्यै स्वाहा॒ नव॑विꣳशत्यै॒ स्वाहै-का॒न्न च॑त्वारिꣳ॒॒शते॒ स्वाहा॒ नव॑चत्वारिꣳशते॒ स्वाहै-का॒न्न ( ) ष॒ष्ट्यै स्वाहा॒ नव॑षष्ट्यै॒ स्वाहै -का॒न्नाशी॒त्यै स्वाहा॒ नवा॑शीत्यै॒ स्वाहैका॒न्न श॒ताय॒ स्वाहा॑ श॒ताय॒ स्वाहा॒ द्वाभ्याꣳ॑ श॒ताभ्याꣳ॒॒ स्वाहा॒ सर्व॑स्मै॒ स्वाहा᳚ ॥ \newline

\textbf{Pada Paata} \newline

एक॑स्मै । स्वाहा᳚ । द्वाभ्या᳚म् । स्वाहा᳚ । त्रि॒भ्य इति॑ त्रि - भ्यः । स्वाहा᳚ । च॒तुर्भ्य॒ इति॑ च॒तुः - भ्यः॒ । स्वाहा᳚ । प॒ञ्चभ्य॒ इति॑ प॒ञ्च - भ्यः॒ । स्वाहा᳚ । ष॒ड्भ्य इति॑ षट्- भ्यः । स्वाहा᳚ । स॒प्तभ्य॒ इति॑ स॒प्त - भ्यः॒ । स्वाहा᳚ । अ॒ष्टा॒भ्यः । स्वाहा᳚ । न॒वभ्य॒ इति॑ न॒व - भ्यः॒ । स्वाहा᳚ । द॒शभ्य॒ इति॑ द॒श - भ्यः॒ । स्वाहा᳚ । ए॒का॒द॒शभ्य॒ इत्ये॑काद॒श - भ्यः॒ । स्वाहा᳚ । द्वा॒द॒शभ्य॒ इति॑ द्वाद॒श - भ्यः॒ । स्वाहा᳚ । त्र॒यो॒द॒शभ्य॒ इति॑ त्रयोद॒श-भ्यः॒ । स्वाहा᳚ । च॒तु॒र्द॒शभ्य॒ इति॑ चतुर्द॒श - भ्यः॒ । स्वाहा᳚ । प॒ञ्च॒द॒शभ्य॒ इति॑ पञ्चद॒श - भ्यः॒ । स्वाहा᳚ । षो॒ड॒शभ्य॒ इति॑ षोड॒श - भ्यः॒ । स्वाहा᳚ । स॒प्त॒द॒शभ्य॒ इति॑ सप्तद॒श - भ्यः॒ । स्वाहा᳚ । अ॒ष्टा॒द॒शभ्य॒ इत्य॑ष्टाद॒श - भ्यः॒ । स्वाहा᳚ । एका᳚त् । न । विꣳ॒॒श॒त्यै । स्वाहा᳚ । नव॑विꣳशत्या॒ इति॒ नव॑ - विꣳ॒॒श॒त्यै॒ । स्वाहा᳚ । एका᳚त् । न । च॒त्वा॒रिꣳ॒॒शते᳚ । स्वाहा᳚ । नव॑चत्वारिꣳशत॒ इति॒ नव॑ - च॒त्वा॒रिꣳ॒॒श॒ते॒ । स्वाहा᳚ । एका᳚त् । न ( ) । ष॒ष्ट्यै । स्वाहा᳚ । नव॑षष्ट्या॒ इति॒ नव॑ - ष॒ष्ट्यै॒ । स्वाहा᳚ । एका᳚त् । न । अ॒शी॒त्यै । स्वाहा᳚ । नवा॑शीत्या॒ इति॒ नव॑ - अ॒शी॒त्यै॒ । स्वाहा᳚ । एका᳚त् । न । श॒ताय॑ । स्वाहा᳚ । श॒ताय॑ । स्वाहा᳚ । द्वाभ्या᳚म् । श॒ताभ्या᳚म् । स्वाहा᳚ । सर्व॑स्मै । स्वाहा᳚ ॥  \newline




\markright{ TS 7.2.12.1  \hfill https://www.vedavms.in \hfill}

\section{ TS 7.2.12.1 }

\textbf{TS 7.2.12.1 } \newline
\textbf{Samhita Paata} \newline

एक॑स्मै॒ स्वाहा᳚ त्रि॒भ्यः स्वाहा॑ प॒ञ्चभ्यः॒ स्वाहा॑ स॒प्तभ्यः॒ स्वाहा॑ न॒वभ्यः॒ स्वाहै॑- काद॒शभ्यः॒ स्वाहा᳚ त्रयोद॒शभ्यः॒ स्वाहा॑ पञ्चद॒शभ्यः॒ स्वाहा॑ सप्तद॒शभ्यः॒ स्वाहैका॒न्न विꣳ॑श॒त्यै स्वाहा॒ नव॑विꣳशत्यै॒ स्वाहैका॒न्न च॑त्वारिꣳ॒॒शते॒ स्वाहा॒ नव॑चत्वारिꣳशते॒ स्वाहैका॒न्न ष॒ष्ट्यै स्वाहा॒ नव॑षष्ट्यै॒ स्वाहैका॒न्ना शी॒त्यै स्वाहा॒ नवा॑शीत्यै॒ स्वाहैका॒न्न श॒ताय॒ स्वाहा॑ श॒ताय॒ स्वाहा॒ सर्व॑स्मै॒ स्वाहा᳚ ॥ \newline

\textbf{Pada Paata} \newline

एक॑स्मै । स्वाहा᳚ । त्रि॒भ्य इति॑ त्रि - भ्यः । स्वाहा᳚ । प॒ञ्चभ्य॒ इति॑ प॒ञ्च - भ्यः॒ । स्वाहा᳚ । स॒प्तभ्य॒ इति॑ स॒प्त - भ्यः॒ । स्वाहा᳚ । न॒वभ्य॒ इति॑ न॒व - भ्यः॒ । स्वाहा᳚ । ए॒का॒द॒शभ्य॒ इत्ये॑काद॒श-भ्यः॒ । स्वाहा᳚ । त्र॒यो॒द॒शभ्य॒ इति॑ त्रयोद॒श - भ्यः॒ । स्वाहा᳚ । प॒ञ्च॒द॒शभ्य॒ इति॑ पञ्चद॒श - भ्यः॒ । स्वाहा᳚ । स॒प्त॒द॒शभ्य॒ इति॑ सप्तद॒श - भ्यः॒ । स्वाहा᳚ । एका᳚त् । न । विꣳ॒॒श॒त्यै । स्वाहा᳚ । नव॑विꣳशत्या॒ इति॒ नव॑ - विꣳ॒॒श॒त्यै॒ । स्वाहा᳚ । एका᳚त् । न । च॒त्वा॒रिꣳ॒॒शते᳚ । स्वाहा᳚ । नव॑चत्वारिꣳशत॒ इति॒ नव॑-च॒त्वा॒रिꣳ॒॒श॒ते॒ । स्वाहा᳚ । एका᳚त् । न । ष॒ष्ट्यै । स्वाहा᳚ । नव॑षष्ट्या॒ इति॒ नव॑-ष॒ष्ट्यै॒ । स्वाहा᳚ । एका᳚त् । न । अ॒शी॒त्यै । स्वाहा᳚ । नवा॑शीत्या॒ इति॒ नव॑ - अ॒शी॒त्यै॒ । स्वाहा᳚ । एका᳚त् । न । श॒ताय॑ । स्वाहा᳚ । श॒ताय॑ । स्वाहा᳚ । सर्व॑स्मै । स्वाहा᳚ ॥  \newline




\markright{ TS 7.2.13.1  \hfill https://www.vedavms.in \hfill}

\section{ TS 7.2.13.1 }

\textbf{TS 7.2.13.1 } \newline
\textbf{Samhita Paata} \newline

द्वाभ्याꣳ॒॒ स्वाहा॑ च॒तुर्भ्यः॒ स्वाहा॑ ष॒ड्भ्यः स्वाहा᳚ ऽष्टा॒भ्यः स्वाहा॑ द॒शभ्यः॒ स्वाहा᳚ द्वाद॒शभ्यः॒ स्वाहा॑ चतुर्द॒शभ्यः॒ स्वाहा॑ षोड॒शभ्यः॒ स्वाहा᳚ ऽष्टाद॒शभ्यः॒ स्वाहा॑ विꣳश॒त्यै स्वाहा॒ ऽष्टान॑वत्यै॒ स्वाहा॑ श॒ताय॒ स्वाहा॒ सर्व॑स्मै॒ स्वाहा᳚ ॥ \newline

\textbf{Pada Paata} \newline

द्वाभ्या᳚म् । स्वाहा᳚ । च॒तुर्भ्य॒ इति॑ च॒तुः - भ्यः॒ । स्वाहा᳚ । ष॒ड्भ्य इति॑ षट्- भ्यः । स्वाहा᳚ । अ॒ष्टा॒भ्यः । स्वाहा᳚ । द॒शभ्य॒ इति॑ द॒श-भ्यः॒ । स्वाहा᳚ । द्वा॒द॒शभ्य॒ इति॑ द्वाद॒श - भ्यः॒ । स्वाहा᳚ । च॒तु॒र्द॒शभ्य॒ इति॑ चतुर्द॒श - भ्यः॒ । स्वाहा᳚ । षो॒ड॒शभ्य॒ इति॑ षोड॒श - भ्यः॒ । स्वाहा᳚ । अ॒ष्टा॒द॒शभ्य॒ इत्य॑ष्टाद॒श - भ्यः॒ । स्वाहा᳚ । विꣳ॒॒श॒त्यै । स्वाहा᳚ । अ॒ष्टान॑वत्या॒ इत्य॒ष्टा - न॒व॒त्यै॒ । स्वाहा᳚ । श॒ताय॑ । स्वाहा᳚ । सर्व॑स्मै । स्वाहा᳚ ॥  \newline




\markright{ TS 7.2.14.1  \hfill https://www.vedavms.in \hfill}

\section{ TS 7.2.14.1 }

\textbf{TS 7.2.14.1 } \newline
\textbf{Samhita Paata} \newline

त्रि॒भ्यः स्वाहा॑ प॒ञ्चभ्यः॒ स्वाहा॑ स॒प्तभ्यः॒ स्वाहा॑ न॒वभ्यः॒ स्वाहै॑-काद॒शभ्यः॒ स्वाहा᳚ त्रयोद॒शभ्यः॒ स्वाहा॑ पञ्चद॒शभ्यः॒ स्वाहा॑ सप्तद॒शभ्यः॒ स्वाहैका॒न्न विꣳ॑श॒त्यै स्वाहा॒ नव॑विꣳशत्यै॒ स्वाहैका॒न्न च॑त्वारिꣳ॒॒शते॒ स्वाहा॒ नव॑चत्वारिꣳशते॒ स्वाहैका॒न्न ष॒ष्ट्यै स्वाहा॒ नव॑षष्ट्यै॒ स्वाहैका॒न्ना ऽशी॒त्यै स्वाहा॒ नवा॑शीत्यै॒ स्वाहैका॒न्न श॒ताय॒ स्वाहा॑ श॒ताय॒ स्वाहा॒ सर्व॑स्मै॒ स्वाहा᳚ ॥ \newline

\textbf{Pada Paata} \newline

त्रि॒भ्य इति॑ त्रि - भ्यः । स्वाहा᳚ । प॒ञ्चभ्य॒ इति॑ प॒ञ्च - भ्यः॒ । स्वाहा᳚ । स॒प्तभ्य॒ इति॑ स॒प्त-भ्यः॒ । स्वाहा᳚ । न॒वभ्य॒ इति॑ न॒व-भ्यः॒ । स्वाहा᳚ । ए॒का॒द॒शभ्य॒ इत्ये॑काद॒श - भ्यः॒ । स्वाहा᳚ । त्र॒यो॒द॒शभ्य॒ इति॑ त्रयोद॒श - भ्यः॒ । स्वाहा᳚ । प॒ञ्च॒द॒शभ्य॒ इति॑ पञ्चद॒श - भ्यः॒ । स्वाहा᳚ । स॒प्त॒द॒शभ्य॒ इति॑ सप्तद॒श - भ्यः॒ । स्वाहा᳚ । एका᳚त् । न । विꣳ॒॒श॒त्यै । स्वाहा᳚ । नव॑विꣳशत्या॒ इति॒ नव॑ - विꣳ॒॒श॒त्यै॒ । स्वाहा᳚ । एका᳚त् । न । च॒त्वा॒रिꣳ॒॒शते᳚ । स्वाहा᳚ । नव॑चत्वारिꣳशत॒ इति॒ नव॑ - च॒त्वा॒रिꣳ॒॒श॒ते॒ । स्वाहा᳚ । एका᳚त् । न । ष॒ष्ट्यै । स्वाहा᳚ । नव॑षष्ट्या॒ इति॒ नव॑ - ष॒ष्ट्यै॒ । स्वाहा᳚ । एका᳚त् । न । अ॒शी॒त्यै । स्वाहा᳚ । नवा॑शीत्या॒ इति॒ नव॑ - अ॒शी॒त्यै॒ । स्वाहा᳚ । एका᳚त् । न । श॒ताय॑ । स्वाहा᳚ । श॒ताय॑ । स्वाहा᳚ । सर्व॑स्मै । स्वाहा᳚ ॥  \newline




\markright{ TS 7.2.15.1  \hfill https://www.vedavms.in \hfill}

\section{ TS 7.2.15.1 }

\textbf{TS 7.2.15.1 } \newline
\textbf{Samhita Paata} \newline

च॒तुर्भ्यः॒ स्वाहा᳚ ऽष्टा॒भ्यः स्वाहा᳚ द्वाद॒शभ्यः॒ स्वाहा॑ षोड॒शभ्यः॒ स्वाहा॑ विꣳश॒त्यै स्वाहा॒ षण्ण॑वत्यै॒ स्वाहा॑ श॒ताय॒ स्वाहा॒ सर्व॑स्मै॒ स्वाहा᳚ ॥ \newline

\textbf{Pada Paata} \newline

च॒तुर्भ्य॒ इति॑ च॒तुः - भ्यः॒ । स्वाहा᳚ । अ॒ष्टा॒भ्यः । स्वाहा᳚ । द्वा॒द॒शभ्य॒ इति॑ द्वाद॒श - भ्यः॒ । स्वाहा᳚ । षो॒ड॒शभ्य॒ इति॑ षोड॒श - भ्यः॒ । स्वाहा᳚ । विꣳ॒॒श॒त्यै । स्वाहा᳚ । षण्ण॑वत्या॒ इति॒ षट्-न॒व॒त्यै॒ । स्वाहा᳚ । श॒ताय॑ । स्वाहा᳚ । सर्व॑स्मै । स्वाहा᳚ ॥  \newline




\markright{ TS 7.2.16.1  \hfill https://www.vedavms.in \hfill}

\section{ TS 7.2.16.1 }

\textbf{TS 7.2.16.1 } \newline
\textbf{Samhita Paata} \newline

प॒ञ्चभ्यः॒ स्वाहा॑ द॒शभ्यः॒ स्वाहा॑ पञ्चद॒शभ्यः॒ स्वाहा॑ विꣳश॒त्यै स्वाहा॒ पञ्च॑नवत्यै॒ स्वाहा॑ श॒ताय॒ स्वाहा॒ सर्व॑स्मै॒ स्वाहा᳚ ॥ \newline

\textbf{Pada Paata} \newline

प॒ञ्चभ्य॒ इति॑ प॒ञ्च - भ्यः॒ । स्वाहा᳚ । द॒शभ्य॒ इति॑ द॒श - भ्यः॒ । स्वाहा᳚ । प॒ञ्च॒द॒शभ्य॒ इति॑ पञ्चद॒श - भ्यः॒ । स्वाहा᳚ । विꣳ॒॒श॒त्यै । स्वाहा᳚ । पञ्च॑नवत्या॒ इति॒ पञ्च॑ - न॒व॒त्यै॒ । स्वाहा᳚ । श॒ताय॑ । स्वाहा᳚ । सर्व॑स्मै । स्वाहा᳚ ॥  \newline




\markright{ TS 7.2.17.1  \hfill https://www.vedavms.in \hfill}

\section{ TS 7.2.17.1 }

\textbf{TS 7.2.17.1 } \newline
\textbf{Samhita Paata} \newline

द॒शभ्यः॒ स्वाहा॑ विꣳश॒त्यै स्वाहा᳚ त्रिꣳ॒॒शते॒ स्वाहा॑ चत्वारिꣳ॒॒शते॒ स्वाहा॑ पञ्चा॒शते॒ स्वाहा॑ ष॒ष्ट्यै स्वाहा॑ सप्त॒त्यै स्वाहा॑ ऽशी॒त्यै स्वाहा॑ नव॒त्यै स्वाहा॑ श॒ताय॒ स्वाहा॒ सर्व॑स्मै॒ स्वाहा᳚ ॥ \newline

\textbf{Pada Paata} \newline

द॒शभ्य॒ इति॑ द॒श - भ्यः॒ । स्वाहा᳚ । विꣳ॒॒श॒त्यै । स्वाहा᳚ । त्रिꣳ॒॒शते᳚ । स्वाहा᳚ । च॒त्वा॒रिꣳ॒॒शते᳚ । स्वाहा᳚ । प॒ञ्चा॒शते᳚ । स्वाहा᳚ । ष॒ष्ट्यै । स्वाहा᳚ । स॒प्त॒त्यै । स्वाहा᳚ । अ॒शी॒त्यै । स्वाहा᳚ । न॒व॒त्यै । स्वाहा᳚ । श॒ताय॑ । स्वाहा᳚ । सर्व॑स्मै । स्वाहा᳚ ॥  \newline




\markright{ TS 7.2.18.1  \hfill https://www.vedavms.in \hfill}

\section{ TS 7.2.18.1 }

\textbf{TS 7.2.18.1 } \newline
\textbf{Samhita Paata} \newline

विꣳ॒॒श॒त्यै स्वाहा॑ चत्वारिꣳ॒॒शते॒ स्वाहा॑ ष॒ष्ट्यै स्वाहा॑ ऽशी॒त्यै स्वाहा॑ श॒ताय॒ स्वाहा॒ सर्व॑स्मै॒ स्वाहा᳚ ॥ \newline

\textbf{Pada Paata} \newline

विꣳ॒॒श॒त्यै । स्वाहा᳚ । च॒त्वा॒रिꣳ॒॒शते᳚ । स्वाहा᳚ । ष॒ष्ट्यै । स्वाहा᳚ । अ॒शी॒त्यै । स्वाहा᳚ । श॒ताय॑ । स्वाहा᳚ । सर्व॑स्मै । स्वाहा᳚ ॥  \newline




\markright{ TS 7.2.19.1  \hfill https://www.vedavms.in \hfill}

\section{ TS 7.2.19.1 }

\textbf{TS 7.2.19.1 } \newline
\textbf{Samhita Paata} \newline

प॒ञ्चा॒शते॒ स्वाहा॑ श॒ताय॒ स्वाहा॒ द्वाभ्याꣳ॑ श॒ताभ्याꣳ॒॒ स्वाहा᳚ त्रि॒भ्यः श॒तेभ्यः॒ स्वाहा॑ च॒तुर्भ्यः॑ श॒तेभ्यः॒ स्वाहा॑ प॒ञ्चभ्यः॑ श॒तेभ्यः॒ स्वाहा॑ ष॒ड्भ्यः श॒तेभ्यः॒ स्वाहा॑ स॒प्तभ्यः॑ श॒तेभ्यः॒ स्वाहा᳚ ऽष्टा॒भ्यः श॒तेभ्यः॒ स्वाहा॑ न॒वभ्यः॑ श॒तेभ्यः॒ स्वाहा॑ स॒हस्रा॑य॒ स्वाहा॒ सर्व॑स्मै॒ स्वाहा᳚ ॥ \newline

\textbf{Pada Paata} \newline

प॒ञ्चा॒शते᳚ । स्वाहा᳚ । श॒ताय॑ । स्वाहा᳚ । द्वाभ्या᳚म् । श॒ताभ्या᳚म् । स्वाहा᳚ । त्रि॒भ्य इति॑ त्रि - भ्यः । श॒तेभ्यः॑ । स्वाहा᳚ । च॒तुर्भ्य॒ इति॑ च॒तुः - भ्यः॒ । श॒तेभ्यः॑ । स्वाहा᳚ । प॒ञ्चभ्य॒ इति॑ प॒ञ्च-भ्यः॒ । श॒तेभ्यः॑ । स्वाहा᳚ । ष॒ड्भ्य इति॑ षट्- भ्यः । श॒तेभ्यः॑ । स्वाहा᳚ । स॒प्तभ्य॒ इति॑ स॒प्त-भ्यः॒ । श॒तेभ्यः॑ । स्वाहा᳚ । अ॒ष्टा॒भ्यः । श॒तेभ्यः॑ । स्वाहा᳚ । न॒वभ्य॒ इति॑ न॒व - भ्यः॒ । श॒तेभ्यः॑ । स्वाहा᳚ । स॒हस्रा॑य । स्वाहा᳚ । सर्व॑स्मै । स्वाहा᳚ ॥  \newline




\markright{ TS 7.2.20.1  \hfill https://www.vedavms.in \hfill}

\section{ TS 7.2.20.1 }

\textbf{TS 7.2.20.1 } \newline
\textbf{Samhita Paata} \newline

श॒ताय॒ स्वाहा॑ स॒हस्रा॑य॒ स्वाहा॒ ऽयुता॑य॒ स्वाहा॑ नि॒युता॑य॒ स्वाहा᳚ प्र॒युता॑य॒ स्वाहा ऽर्बु॑दाय॒ स्वाहा॒ न्य॑र्बुदाय॒ स्वाहा॑ समु॒द्राय॒ स्वाहा॒ मद्ध्या॑य॒ स्वाहा ऽन्ता॑य॒ स्वाहा॑ परा॒र्द्धाय॒ स्वाहो॒षसे॒ स्वाहा॒ व्यु॑ष्ट्यै॒ स्वाहो॑देष्य॒ते स्वाहो᳚द्य॒ते स्वाहोदि॑ताय॒ स्वाहा॑ सुव॒र्गाय॒ स्वाहा॑ लो॒काय॒ स्वाहा॒ सर्व॑स्मै॒ स्वाहा᳚ ॥ \newline

\textbf{Pada Paata} \newline

श॒ताय॑ । स्वाहा᳚ । स॒हस्रा॑य । स्वाहा᳚ । अ॒युता॑य । स्वाहा᳚ । नि॒युता॒येति॑ नि - युता॑य । स्वाहा᳚ । प्र॒युता॒येति॑ प्र - युता॑य । स्वाहा᳚ । अर्बु॑दाय । स्वाहा᳚ । न्य॑र्बुदा॒येति॒ नि - अ॒र्बु॒दा॒य॒ । स्वाहा᳚ । स॒मु॒द्राय॑ । स्वाहा᳚ । मद्ध्या॑य । स्वाहा᳚ । अन्ता॑य । स्वाहा᳚ । प॒रा॒द्‌र्धायेति॑ पर-अ॒द्‌र्धाय॑ । स्वाहा᳚ । उ॒षसे᳚ । स्वाहा᳚ । व्यु॑ष्ट्या॒ इति॒ वि - उ॒ष्ट्यै॒ । स्वाहा᳚ । उ॒दे॒ष्य॒त इत्यु॑त् - ए॒ष्य॒ते । स्वाहा᳚ । उ॒द्य॒त इत्यु॑त् - य॒ते । स्वाहा᳚ । उदि॑ता॒येत्युत् - इ॒ता॒य॒ । स्वाहा᳚ । सु॒व॒र्गायेति॑ सुवः-गाय॑ । स्वाहा᳚ । लो॒काय॑ । स्वाहा᳚ । सर्व॑स्मै । स्वाहा᳚ ॥  \newline






\end{document}