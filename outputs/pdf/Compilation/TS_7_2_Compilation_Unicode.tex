\documentclass[17pt]{extarticle}
\usepackage{babel}
\usepackage{fontspec}
\usepackage{polyglossia}
\usepackage{extsizes}

\usepackage{color}   %May be necessary if you want to color links
\usepackage{hyperref}
\hypersetup{
    colorlinks=true, %set true if you want colored links
    linktoc=all,     %set to all if you want both sections and subsections linked
    linkcolor=black,  %choose some color if you want links to stand out
}

\setmainlanguage{sanskrit}
\setotherlanguages{english} %% or other languages
\setlength{\parindent}{0pt}
\pagestyle{myheadings}
\newfontfamily\devanagarifont[Script=Devanagari]{AdishilaVedic}
\renewcommand{\theHsection}{\thepart.section.\thesection}

\newcommand{\VAR}[1]{}
\newcommand{\BLOCK}[1]{}




\begin{document}
\begin{titlepage}
    \begin{center}
 
\begin{sanskrit}
    { \Large
    कृष्ण यजुर्वेदीय तैत्तिरीय संहिता,पद,जटा,घन पाठः 
    }
    \\
    \vspace{2.5cm}
    \mbox{ \Large
    7.2      सप्तमकाण्डे द्वितीयः प्रश्नः - षड् रात्राद्यानां निरूपणं   }
\end{sanskrit}
\end{center}

\end{titlepage}
\tableofcontents
\phantomsection
\pagebreak

\markright{ TS 7.2.1.1  \hfill https://www.vedavms.in \hfill}

\section{ TS 7.2.1.1 }

\textbf{TS 7.2.1.1 } \newline
\textbf{Samhita Paata} \newline

सा॒द्ध्या वै दे॒वाः सु॑व॒र्गका॑मा ए॒तꣳ ष॑ड्-रा॒त्रम॑पश्य॒न् तमाऽह॑र॒न् तेना॑यजन्त॒ ततो॒ वै ते सु॑व॒र्गं ॅलो॒कमा॑य॒न॒. य ए॒वं ॅवि॒द्वाꣳसः॑ षड्-रा॒त्रमास॑ते सुव॒र्गमे॒व लो॒कं ॅय॑न्ति देवस॒त्रं ॅवै ष॑ड्-रा॒त्रः प्र॒त्यक्षꣳ॒॒ ह्ये॑तानि॑ पृ॒ष्ठानि॒ य ए॒वं ॅवि॒द्वाꣳसः॑ षड्-रा॒त्रमास॑ते सा॒क्षादे॒व दे॒वता॑ अ॒भ्यारो॑हन्ति॒ षड्-रा॒त्रो भ॑वति॒ षड् वा ऋ॒तवः॒ षट् पृ॒ष्ठानि॑ - [  ] \newline

\textbf{Pada Paata} \newline

सा॒द्ध्याः । वै । दे॒वाः । सु॒व॒र्गका॑मा॒ इति॑ सुव॒र्ग - का॒माः॒ । ए॒तम् । ष॒ड्रा॒त्रमिति॑ षट् - रा॒त्रम् । अ॒प॒श्य॒न्न् । तम् । एति॑ । अ॒ह॒र॒न्न् । तेन॑ । अ॒य॒ज॒न्त॒ । ततः॑ । वै । ते । सु॒व॒र्गमिति॑ सुवः-गम् । लो॒कम् । आ॒य॒न्न् । ये । ए॒वम् । वि॒द्वाꣳसः॑ । ष॒ड्रा॒त्रमिति॑ षट् - रा॒त्रम् । आस॑ते । सु॒व॒र्गमिति॑ सुवः - गम् । ए॒व । लो॒कम् । य॒न्ति॒ । दे॒व॒स॒त्रमिति॑ देव - स॒त्रम् । वै । ष॒ड्रा॒त्र इति॑ षट् - रा॒त्रः । प्र॒त्यक्ष॒मिति॑ प्रति - अक्ष᳚म् । हि । ए॒तानि॑ । पृ॒ष्ठानि॑ । ये । ए॒वम् । वि॒द्वाꣳसः॑ । ष॒ड्रा॒त्रमिति॑ षट् - रा॒त्रम् । आस॑ते । सा॒क्षादिति॑ स - अ॒क्षात् । ए॒व । दे॒वताः᳚ । अ॒भ्यारो॑ह॒न्तीत्य॑भि - आरो॑हन्ति । ष॒ड्रा॒त्र इति॑ षट् - रा॒त्रः । भ॒व॒ति॒ । षट् । वै । ऋ॒तवः॑ । षट् । पृ॒ष्ठानि॑ ।  \newline


\textbf{Krama Paata} \newline

सा॒द्ध्या वै । वै दे॒वाः । दे॒वाः सु॑व॒र्गका॑माः । सु॒व॒र्गका॑मा ए॒तम् । सु॒व॒र्गका॑मा॒ इति॑ सुव॒र्ग - का॒माः॒ । ए॒तꣳ ष॑ड्‍रा॒त्रम् । ष॒ड्‍रा॒त्रम॑पश्यन्न् । ष॒ड्‍रा॒त्रमिति॑ षट् - रा॒त्रम् । अ॒प॒श्य॒न् तम् । तमा । आऽह॑रन्न् । अ॒ह॒र॒न् तेन॑ । तेना॑यजन्त । अ॒ज॒य॒न्त॒ ततः॑ । ततो॒ वै । वै ते । ते सु॑व॒र्गम् । सु॒व॒र्गम् ॅलो॒कम् । सु॒व॒र्गमिति॑ सुवः - गम् । लो॒कमा॑यन्न् । आ॒य॒न्.॒ ये । य ए॒वम् । ए॒वम् ॅवि॒द्वाꣳसः॑ । वि॒द्वाꣳसः॑ षड्‍रा॒त्रम् । ष॒ड्‍रा॒त्रमास॑ते । ष॒ड्‍रा॒त्रमिति॑ षट् - रा॒त्रम् । आस॑ते सुव॒र्गम् । सु॒व॒र्गमे॒व । सु॒व॒र्गमिति॑ सुवः - गम् । ए॒व लो॒कम् । लो॒कम् ॅय॑न्ति । य॒न्ति॒ दे॒व॒स॒त्रम् । दे॒व॒स॒त्रम् ॅवै । दे॒व॒स॒त्रमिति॑ देव - स॒त्रम् । वै ष॑ड्‍रा॒त्रः । ष॒ड्‍रा॒त्रः प्र॒त्यक्ष᳚म् । ष॒ड्‍रा॒त्र इति॑ षट् - रा॒त्रः । प्र॒त्यक्षꣳ॒॒ हि । प्र॒त्यक्ष॒मिति॑ प्रति - अक्ष᳚म् । ह्ये॑तानि॑ । ए॒तानि॑ पृ॒ष्ठानि॑ । पृ॒ष्ठानि॒ ये । य ए॒वम् । ए॒वम् ॅवि॒द्वाꣳसः॑ । वि॒द्वाꣳसः॑ षड्‍रा॒त्रम् । ष॒ड्‍रा॒त्रमास॑ते । ष॒ड्‍रा॒त्रमिति॑ षट् - रा॒त्रम् । आस॑ते सा॒क्षात् । सा॒क्षादे॒व । सा॒क्षादिति॑ स - अ॒क्षात् । ए॒व दे॒वताः᳚ । दे॒वता॑ अ॒भ्यारो॑हन्ति । अ॒भ्यारो॑हन्ति षड्‍रा॒त्रः । अ॒भ्यारो॑ह॒न्तीत्य॑भि - आरो॑हन्ति । ष॒ड्‍रा॒त्रो भ॑वति । ष॒ड्‍रा॒त्र इति॑ षट् - रा॒त्रः । भ॒व॒ति॒ षट् । षड् वै । वा ऋ॒तवः॑ । ऋ॒तवः॒ षट् । षट् पृ॒ष्ठानि॑ \newline

\textbf{Jatai Paata} \newline

1. सा॒द्ध्या वै वै सा॒द्ध्याः सा॒द्ध्या वै । \newline
2. वै दे॒वा दे॒वा वै वै दे॒वाः । \newline
3. दे॒वाः सु॑व॒र्गका॑माः सुव॒र्गका॑मा दे॒वा दे॒वाः सु॑व॒र्गका॑माः । \newline
4. सु॒व॒र्गका॑मा ए॒त मे॒तꣳ सु॑व॒र्गका॑माः सुव॒र्गका॑मा ए॒तम् । \newline
5. सु॒व॒र्गका॑मा॒ इति॑ सुव॒र्ग - का॒माः॒ । \newline
6. ए॒तꣳ ष॑ड्रा॒त्रꣳ ष॑ड्रा॒त्र मे॒त मे॒तꣳ ष॑ड्रा॒त्रम् । \newline
7. ष॒ड्रा॒त्र म॑पश्यन् नपश्यन् थ्षड्रा॒त्रꣳ ष॑ड्रा॒त्र म॑पश्यन्न् । \newline
8. ष॒ड्रा॒त्रमिति॑ षट् - रा॒त्रम् । \newline
9. अ॒प॒श्य॒न् तम् त म॑पश्यन् नपश्य॒न् तम् । \newline
10. त मा तम् त मा । \newline
11. आ ऽह॑रन् नहर॒न् ना ऽह॑रन्न् । \newline
12. अ॒ह॒र॒न् तेन॒ तेना॑ हरन् नहर॒न् तेन॑ । \newline
13. तेना॑ यजन्ता यजन्त॒ तेन॒ तेना॑ यजन्त । \newline
14. अ॒य॒ज॒न्त॒ तत॒ स्ततो॑ ऽयजन्ता यजन्त॒ ततः॑ । \newline
15. ततो॒ वै वै तत॒ स्ततो॒ वै । \newline
16. वै ते ते वै वै ते । \newline
17. ते सु॑व॒र्गꣳ सु॑व॒र्गम् ते ते सु॑व॒र्गम् । \newline
18. सु॒व॒र्गम् ॅलो॒कम् ॅलो॒कꣳ सु॑व॒र्गꣳ सु॑व॒र्गम् ॅलो॒कम् । \newline
19. सु॒व॒र्गमिति॑ सुवः - गम् । \newline
20. लो॒क मा॑यन् नायन् ॅलो॒कम् ॅलो॒क मा॑यन्न् । \newline
21. आ॒य॒न्॒. ये य आ॑यन् नाय॒न्॒. ये । \newline
22. य ए॒व मे॒वं ॅये य ए॒वम् । \newline
23. ए॒वं ॅवि॒द्वाꣳसो॑ वि॒द्वाꣳस॑ ए॒व मे॒वं ॅवि॒द्वाꣳसः॑ । \newline
24. वि॒द्वाꣳस॑ ष्षड्रा॒त्रꣳ ष॑ड्रा॒त्रं ॅवि॒द्वाꣳसो॑ वि॒द्वाꣳस॑ ष्षड्रा॒त्रम् । \newline
25. ष॒ड्रा॒त्र मास॑त॒ आस॑ते षड्रा॒त्रꣳ ष॑ड्रा॒त्र मास॑ते । \newline
26. ष॒ड्रा॒त्रमिति॑ षट् - रा॒त्रम् । \newline
27. आस॑ते सुव॒र्गꣳ सु॑व॒र्ग मास॑त॒ आस॑ते सुव॒र्गम् । \newline
28. सु॒व॒र्ग मे॒वैव सु॑व॒र्गꣳ सु॑व॒र्ग मे॒व । \newline
29. सु॒व॒र्गमिति॑ सुवः - गम् । \newline
30. ए॒व लो॒कम् ॅलो॒क मे॒वैव लो॒कम् । \newline
31. लो॒कं ॅय॑न्ति यन्ति लो॒कम् ॅलो॒कं ॅय॑न्ति । \newline
32. य॒न्ति॒ दे॒व॒स॒त्रम् दे॑वस॒त्रं ॅय॑न्ति यन्ति देवस॒त्रम् । \newline
33. दे॒व॒स॒त्रं ॅवै वै दे॑वस॒त्रम् दे॑वस॒त्रं ॅवै । \newline
34. दे॒व॒स॒त्रमिति॑ देव - स॒त्रम् । \newline
35. वै ष॑ड्रा॒त्र ष्ष॑ड्रा॒त्रो वै वै ष॑ड्रा॒त्रः । \newline
36. ष॒ड्रा॒त्रः प्र॒त्यक्ष॑म् प्र॒त्यक्षꣳ॑ षड्रा॒त्र ष्ष॑ड्रा॒त्रः प्र॒त्यक्ष᳚म् । \newline
37. ष॒ड्रा॒त्र इति॑ षट् - रा॒त्रः । \newline
38. प्र॒त्यक्षꣳ॒॒ हि हि प्र॒त्यक्ष॑म् प्र॒त्यक्षꣳ॒॒ हि । \newline
39. प्र॒त्यक्ष॒मिति॑ प्रति - अक्ष᳚म् । \newline
40. ह्ये॑ता न्ये॒तानि॒ हि ह्ये॑तानि॑ । \newline
41. ए॒तानि॑ पृ॒ष्ठानि॑ पृ॒ष्ठा न्ये॒ता न्ये॒तानि॑ पृ॒ष्ठानि॑ । \newline
42. पृ॒ष्ठानि॒ ये ये पृ॒ष्ठानि॑ पृ॒ष्ठानि॒ ये । \newline
43. य ए॒व मे॒वं ॅये य ए॒वम् । \newline
44. ए॒वं ॅवि॒द्वाꣳसो॑ वि॒द्वाꣳस॑ ए॒व मे॒वं ॅवि॒द्वाꣳसः॑ । \newline
45. वि॒द्वाꣳस॑ ष्षड्रा॒त्रꣳ ष॑ड्रा॒त्रं ॅवि॒द्वाꣳसो॑ वि॒द्वाꣳस॑ ष्षड्रा॒त्रम् । \newline
46. ष॒ड्रा॒त्र मास॑त॒ आस॑ते षड्रा॒त्रꣳ ष॑ड्रा॒त्र मास॑ते । \newline
47. ष॒ड्रा॒त्रमिति॑ षट् - रा॒त्रम् । \newline
48. आस॑ते सा॒क्षाथ् सा॒क्षा दास॑त॒ आस॑ते सा॒क्षात् । \newline
49. सा॒क्षा दे॒वैव सा॒क्षाथ् सा॒क्षा दे॒व । \newline
50. सा॒क्षादिति॑ स - अ॒क्षात् । \newline
51. ए॒व दे॒वता॑ दे॒वता॑ ए॒वैव दे॒वताः᳚ । \newline
52. दे॒वता॑ अ॒भ्यारो॑ह न्त्य॒भ्यारो॑हन्ति दे॒वता॑ दे॒वता॑ अ॒भ्यारो॑हन्ति । \newline
53. अ॒भ्यारो॑हन्ति षड्रा॒त्र ष्ष॑ड्रा॒त्रो᳚ ऽभ्यारो॑ह न्त्य॒भ्यारो॑हन्ति षड्रा॒त्रः । \newline
54. अ॒भ्यारो॑ह॒न्तीत्य॑भि - आरो॑हन्ति । \newline
55. ष॒ड्रा॒त्रो भ॑वति भवति षड्रा॒त्र ष्ष॑ड्रा॒त्रो भ॑वति । \newline
56. ष॒ड्रा॒त्र इति॑ षट् - रा॒त्रः । \newline
57. भ॒व॒ति॒ षट् थ्षड् भ॑वति भवति॒ षट् । \newline
58. षड् वै वै षट् थ्षड् वै । \newline
59. वा ऋ॒तव॑ ऋ॒तवो॒ वै वा ऋ॒तवः॑ । \newline
60. ऋ॒तव॒ ष्षट् थ्षडृ॒तव॑ ऋ॒तव॒ ष्षट् । \newline
61. षट् पृ॒ष्ठानि॑ पृ॒ष्ठानि॒ षट् थ्षट् पृ॒ष्ठानि॑ । \newline
62. पृ॒ष्ठानि॑ पृ॒ष्ठैः पृ॒ष्ठैः पृ॒ष्ठानि॑ पृ॒ष्ठानि॑ पृ॒ष्ठैः । \newline

\textbf{Ghana Paata } \newline

1. सा॒द्ध्या वै वै सा॒द्ध्याः सा॒द्ध्या वै दे॒वा दे॒वा वै सा॒द्ध्याः सा॒द्ध्या वै दे॒वाः । \newline
2. वै दे॒वा दे॒वा वै वै दे॒वाः सु॑व॒र्गका॑माः सुव॒र्गका॑मा दे॒वा वै वै दे॒वाः सु॑व॒र्गका॑माः । \newline
3. दे॒वाः सु॑व॒र्गका॑माः सुव॒र्गका॑मा दे॒वा दे॒वाः सु॑व॒र्गका॑मा ए॒त मे॒तꣳ सु॑व॒र्गका॑मा दे॒वा दे॒वाः सु॑व॒र्गका॑मा ए॒तम् । \newline
4. सु॒व॒र्गका॑मा ए॒त मे॒तꣳ सु॑व॒र्गका॑माः सुव॒र्गका॑मा ए॒तꣳ ष॑ड्रा॒त्रꣳ ष॑ड्रा॒त्र मे॒तꣳ सु॑व॒र्गका॑माः सुव॒र्गका॑मा ए॒तꣳ ष॑ड्रा॒त्रम् । \newline
5. सु॒व॒र्गका॑मा॒ इति॑ सुव॒र्ग - का॒माः॒ । \newline
6. ए॒तꣳ ष॑ड्रा॒त्रꣳ ष॑ड्रा॒त्र मे॒त मे॒तꣳ ष॑ड्रा॒त्र म॑पश्यन् नपश्यन् थ्षड्रा॒त्र मे॒त मे॒तꣳ ष॑ड्रा॒त्र म॑पश्यन्न् । \newline
7. ष॒ड्रा॒त्र म॑पश्यन् नपश्यन् थ्षड्रा॒त्रꣳ ष॑ड्रा॒त्र म॑पश्य॒न् तम् तम॑पश्यन् थ्षड्रा॒त्रꣳ ष॑ड्रा॒त्र म॑पश्य॒न् तम् । \newline
8. ष॒ड्रा॒त्रमिति॑ षट् - रा॒त्रम् । \newline
9. अ॒प॒श्य॒न् तम् त म॑पश्यन् नपश्य॒न् त मा त म॑पश्यन् नपश्य॒न् त मा । \newline
10. त मा तम् त मा ऽह॑रन् नहर॒न् ना तम् त मा ऽह॑रन्न् । \newline
11. आ ऽह॑रन् नहर॒न् ना ऽह॑र॒न् तेन॒ तेना॑ हर॒न् ना ऽह॑र॒न् तेन॑ । \newline
12. अ॒ह॒र॒न् तेन॒ तेना॑ हरन् नहर॒न् तेना॑ यजन्ता यजन्त॒ तेना॑ हरन् नहर॒न् तेना॑ यजन्त । \newline
13. तेना॑ यजन्ता यजन्त॒ तेन॒ तेना॑ यजन्त॒ तत॒ स्ततो॑ ऽयजन्त॒ तेन॒ तेना॑ यजन्त॒ ततः॑ । \newline
14. अ॒य॒ज॒न्त॒ तत॒ स्ततो॑ ऽयजन्ता यजन्त॒ ततो॒ वै वै ततो॑ ऽयजन्ता यजन्त॒ ततो॒ वै । \newline
15. ततो॒ वै वै तत॒ स्ततो॒ वै ते ते वै तत॒ स्ततो॒ वै ते । \newline
16. वै ते ते वै वै ते सु॑व॒र्गꣳ सु॑व॒र्गम् ते वै वै ते सु॑व॒र्गम् । \newline
17. ते सु॑व॒र्गꣳ सु॑व॒र्गम् ते ते सु॑व॒र्गम् ॅलो॒कम् ॅलो॒कꣳ सु॑व॒र्गम् ते ते सु॑व॒र्गम् ॅलो॒कम् । \newline
18. सु॒व॒र्गम् ॅलो॒कम् ॅलो॒कꣳ सु॑व॒र्गꣳ सु॑व॒र्गम् ॅलो॒क मा॑यन् नायन् ॅलो॒कꣳ सु॑व॒र्गꣳ सु॑व॒र्गम् ॅलो॒क मा॑यन्न् । \newline
19. सु॒व॒र्गमिति॑ सुवः - गम् । \newline
20. लो॒क मा॑यन् नायन् ॅलो॒कम् ॅलो॒क मा॑य॒न्॒. ये य आ॑यन् ॅलो॒कम् ॅलो॒क मा॑य॒न्॒. ये । \newline
21. आ॒य॒न्॒. ये य आ॑यन् नाय॒न्॒. य ए॒व मे॒वं ॅय आ॑यन् नाय॒न्॒. य ए॒वम् । \newline
22. य ए॒व मे॒वं ॅये य ए॒वं ॅवि॒द्वाꣳसो॑ वि॒द्वाꣳस॑ ए॒वं ॅये य ए॒वं ॅवि॒द्वाꣳसः॑ । \newline
23. ए॒वं ॅवि॒द्वाꣳसो॑ वि॒द्वाꣳस॑ ए॒व मे॒वं ॅवि॒द्वाꣳस॑ ष्षड्रा॒त्रꣳ ष॑ड्रा॒त्रं ॅवि॒द्वाꣳस॑ ए॒व मे॒वं ॅवि॒द्वाꣳस॑ ष्षड्रा॒त्रम् । \newline
24. वि॒द्वाꣳस॑ ष्षड्रा॒त्रꣳ ष॑ड्रा॒त्रं ॅवि॒द्वाꣳसो॑ वि॒द्वाꣳस॑ ष्षड्रा॒त्र मास॑त॒ आस॑ते षड्रा॒त्रं ॅवि॒द्वाꣳसो॑ वि॒द्वाꣳस॑ ष्षड्रा॒त्र मास॑ते । \newline
25. ष॒ड्रा॒त्र मास॑त॒ आस॑ते षड्रा॒त्रꣳ ष॑ड्रा॒त्र मास॑ते सुव॒र्गꣳ सु॑व॒र्ग मास॑ते षड्रा॒त्रꣳ ष॑ड्रा॒त्र मास॑ते सुव॒र्गम् । \newline
26. ष॒ड्रा॒त्रमिति॑ षट् - रा॒त्रम् । \newline
27. आस॑ते सुव॒र्गꣳ सु॑व॒र्ग मास॑त॒ आस॑ते सुव॒र्ग मे॒वैव सु॑व॒र्ग मास॑त॒ आस॑ते सुव॒र्ग मे॒व । \newline
28. सु॒व॒र्ग मे॒वैव सु॑व॒र्गꣳ सु॑व॒र्ग मे॒व लो॒कम् ॅलो॒क मे॒व सु॑व॒र्गꣳ सु॑व॒र्ग मे॒व लो॒कम् । \newline
29. सु॒व॒र्गमिति॑ सुवः - गम् । \newline
30. ए॒व लो॒कम् ॅलो॒क मे॒वैव लो॒कं ॅय॑न्ति यन्ति लो॒क मे॒वैव लो॒कं ॅय॑न्ति । \newline
31. लो॒कं ॅय॑न्ति यन्ति लो॒कम् ॅलो॒कं ॅय॑न्ति देवस॒त्रम् दे॑वस॒त्रं ॅय॑न्ति लो॒कम् ॅलो॒कं ॅय॑न्ति देवस॒त्रम् । \newline
32. य॒न्ति॒ दे॒व॒स॒त्रम् दे॑वस॒त्रं ॅय॑न्ति यन्ति देवस॒त्रं ॅवै वै दे॑वस॒त्रं ॅय॑न्ति यन्ति देवस॒त्रं ॅवै । \newline
33. दे॒व॒स॒त्रं ॅवै वै दे॑वस॒त्रम् दे॑वस॒त्रं ॅवै ष॑ड्रा॒त्र ष्ष॑ड्रा॒त्रो वै दे॑वस॒त्रम् दे॑वस॒त्रं ॅवै ष॑ड्रा॒त्रः । \newline
34. दे॒व॒स॒त्रमिति॑ देव - स॒त्रम् । \newline
35. वै ष॑ड्रा॒त्र ष्ष॑ड्रा॒त्रो वै वै ष॑ड्रा॒त्रः प्र॒त्यक्ष॑म् प्र॒त्यक्षꣳ॑ षड्रा॒त्रो वै वै ष॑ड्रा॒त्रः प्र॒त्यक्ष᳚म् । \newline
36. ष॒ड्रा॒त्रः प्र॒त्यक्ष॑म् प्र॒त्यक्षꣳ॑ षड्रा॒त्र ष्ष॑ड्रा॒त्रः प्र॒त्यक्षꣳ॒॒ हि हि प्र॒त्यक्षꣳ॑ षड्रा॒त्र ष्ष॑ड्रा॒त्रः प्र॒त्यक्षꣳ॒॒ हि । \newline
37. ष॒ड्रा॒त्र इति॑ षट् - रा॒त्रः । \newline
38. प्र॒त्यक्षꣳ॒॒ हि हि प्र॒त्यक्ष॑म् प्र॒त्यक्षꣳ॒॒ ह्ये॑ता न्ये॒तानि॒ हि प्र॒त्यक्ष॑म् प्र॒त्यक्षꣳ॒॒ ह्ये॑तानि॑ । \newline
39. प्र॒त्यक्ष॒मिति॑ प्रति - अक्ष᳚म् । \newline
40. ह्ये॑ता न्ये॒तानि॒ हि ह्ये॑तानि॑ पृ॒ष्ठानि॑ पृ॒ष्ठा न्ये॒तानि॒ हि ह्ये॑तानि॑ पृ॒ष्ठानि॑ । \newline
41. ए॒तानि॑ पृ॒ष्ठानि॑ पृ॒ष्ठा न्ये॒ता न्ये॒तानि॑ पृ॒ष्ठानि॒ ये ये पृ॒ष्ठा न्ये॒ता न्ये॒तानि॑ पृ॒ष्ठानि॒ ये । \newline
42. पृ॒ष्ठानि॒ ये ये पृ॒ष्ठानि॑ पृ॒ष्ठानि॒ य ए॒व मे॒वं ॅये पृ॒ष्ठानि॑ पृ॒ष्ठानि॒ य ए॒वम् । \newline
43. य ए॒व मे॒वं ॅये य ए॒वं ॅवि॒द्वाꣳसो॑ वि॒द्वाꣳस॑ ए॒वं ॅये य ए॒वं ॅवि॒द्वाꣳसः॑ । \newline
44. ए॒वं ॅवि॒द्वाꣳसो॑ वि॒द्वाꣳस॑ ए॒व मे॒वं ॅवि॒द्वाꣳस॑ ष्षड्रा॒त्रꣳ ष॑ड्रा॒त्रं ॅवि॒द्वाꣳस॑ ए॒व मे॒वं ॅवि॒द्वाꣳस॑ ष्षड्रा॒त्रम् । \newline
45. वि॒द्वाꣳस॑ ष्षड्रा॒त्रꣳ ष॑ड्रा॒त्रं ॅवि॒द्वाꣳसो॑ वि॒द्वाꣳस॑ ष्षड्रा॒त्र मास॑त॒ आस॑ते षड्रा॒त्रं ॅवि॒द्वाꣳसो॑ वि॒द्वाꣳस॑ ष्षड्रा॒त्र मास॑ते । \newline
46. ष॒ड्रा॒त्र मास॑त॒ आस॑ते षड्रा॒त्रꣳ ष॑ड्रा॒त्र मास॑ते सा॒क्षाथ् सा॒क्षा दास॑ते षड्रा॒त्रꣳ ष॑ड्रा॒त्र मास॑ते सा॒क्षात् । \newline
47. ष॒ड्रा॒त्रमिति॑ षट् - रा॒त्रम् । \newline
48. आस॑ते सा॒क्षाथ् सा॒क्षा दास॑त॒ आस॑ते सा॒क्षा दे॒वैव सा॒क्षा दास॑त॒ आस॑ते सा॒क्षा दे॒व । \newline
49. सा॒क्षा दे॒वैव सा॒क्षाथ् सा॒क्षा दे॒व दे॒वता॑ दे॒वता॑ ए॒व सा॒क्षाथ् सा॒क्षा दे॒व दे॒वताः᳚ । \newline
50. सा॒क्षादिति॑ स - अ॒क्षात् । \newline
51. ए॒व दे॒वता॑ दे॒वता॑ ए॒वैव दे॒वता॑ अ॒भ्यारो॑ह न्त्य॒भ्यारो॑हन्ति दे॒वता॑ ए॒वैव दे॒वता॑ अ॒भ्यारो॑हन्ति । \newline
52. दे॒वता॑ अ॒भ्यारो॑ह न्त्य॒भ्यारो॑हन्ति दे॒वता॑ दे॒वता॑ अ॒भ्यारो॑हन्ति षड्रा॒त्र ष्ष॑ड्रा॒त्रो᳚ ऽभ्यारो॑हन्ति दे॒वता॑ दे॒वता॑ अ॒भ्यारो॑हन्ति षड्रा॒त्रः । \newline
53. अ॒भ्यारो॑हन्ति षड्रा॒त्र ष्ष॑ड्रा॒त्रो᳚ ऽभ्यारो॑ह न्त्य॒भ्यारो॑हन्ति षड्रा॒त्रो भ॑वति भवति षड्रा॒त्रो᳚ ऽभ्यारो॑ह न्त्य॒भ्यारो॑हन्ति षड्रा॒त्रो भ॑वति । \newline
54. अ॒भ्यारो॑ह॒न्तीत्य॑भि - आरो॑हन्ति । \newline
55. ष॒ड्रा॒त्रो भ॑वति भवति षड्रा॒त्र ष्ष॑ड्रा॒त्रो भ॑वति॒ षट् थ्षड् भ॑वति षड्रा॒त्र ष्ष॑ड्रा॒त्रो भ॑वति॒ षट् । \newline
56. ष॒ड्रा॒त्र इति॑ षट् - रा॒त्रः । \newline
57. भ॒व॒ति॒ षट् थ्षड् भ॑वति भवति॒ षड् वै वै षड् भ॑वति भवति॒ षड् वै । \newline
58. षड् वै वै षट् थ्षड् वा ऋ॒तव॑ ऋ॒तवो॒ वै षट् थ्षड् वा ऋ॒तवः॑ । \newline
59. वा ऋ॒तव॑ ऋ॒तवो॒ वै वा ऋ॒तव॒ ष्षट् थ्षडृ॒तवो॒ वै वा ऋ॒तव॒ ष्षट् । \newline
60. ऋ॒तव॒ ष्षट् थ्षडृ॒तव॑ ऋ॒तव॒ ष्षट् पृ॒ष्ठानि॑ पृ॒ष्ठानि॒ षडृ॒तव॑ ऋ॒तव॒ ष्षट् पृ॒ष्ठानि॑ । \newline
61. षट् पृ॒ष्ठानि॑ पृ॒ष्ठानि॒ षट् थ्षट् पृ॒ष्ठानि॑ पृ॒ष्ठैः पृ॒ष्ठैः पृ॒ष्ठानि॒ षट् थ्षट् पृ॒ष्ठानि॑ पृ॒ष्ठैः । \newline
62. पृ॒ष्ठानि॑ पृ॒ष्ठैः पृ॒ष्ठैः पृ॒ष्ठानि॑ पृ॒ष्ठानि॑ पृ॒ष्ठै रे॒वैव पृ॒ष्ठैः पृ॒ष्ठानि॑ पृ॒ष्ठानि॑ पृ॒ष्ठै रे॒व । \newline
\pagebreak
\markright{ TS 7.2.1.2  \hfill https://www.vedavms.in \hfill}

\section{ TS 7.2.1.2 }

\textbf{TS 7.2.1.2 } \newline
\textbf{Samhita Paata} \newline

पृ॒ष्ठैरे॒वर्तून॒-न्वारो॑हन्त्यृ॒तुभिः॑ संॅवथ्स॒रं ते सं॑ॅवथ्स॒र ए॒व प्रति॑ तिष्ठन्ति बृहद्-रथन्त॒राभ्यां᳚ ॅयन्ती॒यं ॅवाव र॑थंत॒रम॒सौ बृ॒हदा॒भ्यामे॒व य॒न्त्यथो॑ अ॒नयो॑रे॒व प्रति॑ तिष्ठन्त्ये॒ते वै य॒ज्ञ्स्या᳚ञ्ज॒साय॑नी स्रु॒ती ताभ्या॑मे॒व सु॑व॒र्गं ॅलो॒कं ॅय॑न्ति त्रि॒वृद॑ग्निष्टो॒मो भ॑वति॒ तेज॑ ए॒वाव॑ रुन्धते पञ्चद॒शो भ॑वतीन्द्रि॒यमे॒वाव॑ रुन्धते सप्तद॒शो- [  ] \newline

\textbf{Pada Paata} \newline

पृ॒ष्ठैः । ए॒व । ऋ॒तून् । अ॒न्वारो॑ह॒न्तीत्य॑नु - आरो॑हन्ति । ऋ॒तुभि॒रित्यृ॒तु -भिः॒ । सं॒ॅव॒थ्स॒रमिति॑ सं - व॒थ्स॒रम् । ते । सं॒ॅव॒थ्स॒र इति॑ सं - व॒थ्स॒रे । ए॒व । प्रतीति॑ । ति॒ष्ठ॒न्ति॒ । बृ॒ह॒द्र॒थ॒न्त॒राभ्या॒मिति॑ बृहत् - र॒थ॒न्त॒राभ्या᳚म् । य॒न्ति॒ । इ॒यम् । वाव । र॒थ॒न्त॒रमिति॑ रथं - त॒रम् । अ॒सौ । बृ॒हत् । आ॒भ्याम् । ए॒व । य॒न्ति॒ । अथो॒ इति॑ । अ॒नयोः᳚ । ए॒व । प्रतीति॑ । ति॒ष्ठ॒न्ति॒ । ए॒ते इति॑ । वै । य॒ज्ञ्स्य॑ । अ॒ञ्ज॒साय॑नी॒ इत्य॑ञ्जसा - अय॑नी । स्रु॒ती इति॑ । ताभ्या᳚म् । ए॒व । सु॒व॒र्गमिति॑ सुवः - गम् । लो॒कम् । य॒न्ति॒ । त्रि॒वृदिति॑ त्रि - वृत् । अ॒ग्नि॒ष्टो॒म इत्य॑ग्नि - स्तो॒मः । भ॒व॒ति॒ । तेजः॑ । ए॒व । अवेति॑ । रु॒न्ध॒ते॒ । प॒ञ्च॒द॒श इति॑ पञ्च - द॒शः । भ॒व॒ति॒ । इ॒न्द्रि॒यम् । ए॒व । अवेति॑ । रु॒न्ध॒ते॒ । स॒प्त॒द॒श इति॑ सप्त - द॒शः ।  \newline


\textbf{Krama Paata} \newline

पृ॒ष्ठानि॑ पृ॒ष्ठैः । पृ॒ष्ठैरे॒व । ए॒वर्तून् । ऋ॒तून॒न्वारो॑हन्ति । अ॒न्वारो॑हन्त्यृ॒तुभिः॑ । अ॒न्वारो॑ह॒न्तीत्य॑नु - आरो॑हन्ति । ऋ॒तुभिः॑ सम्ॅवथ्स॒रम् । ऋ॒तुभि॒रित्यृ॒तु - भिः॒ । स॒म्ॅव॒थ्स॒रम् ते । स॒म्ॅव॒थ्स॒रमिति॑ सम् - व॒थ्स॒रम् । ते स॑म्ॅवथ्स॒रे । स॒म्ॅव॒थ्स॒र ए॒व । स॒म्ॅव॒थ्स॒र इति॑ सम् - व॒थ्स॒रे । ए॒व प्रति॑ । प्रति॑ तिष्ठन्ति । ति॒ष्ठ॒न्ति॒ बृ॒ह॒द्‍र॒थ॒न्त॒राभ्या᳚म् । बृ॒ह॒द्‍र॒थ॒न्त॒राभ्या᳚म् ॅयन्ति । बृ॒ह॒द्‍र॒थ॒न्त॒राभ्या॒मिति॑ बृहत् - र॒थ॒न्त॒राभ्या᳚म् । य॒न्ती॒यम् । इ॒यम् ॅवाव । वाव र॑थन्त॒रम् । र॒थ॒न्त॒रम॒सौ । र॒थ॒न्त॒रमिति॑ रथम् - त॒रम् । अ॒सौ बृ॒हत् । बृ॒हदा॒भ्याम् । आ॒भ्यामे॒व । ए॒व य॑न्ति । य॒न्त्यथो᳚ । अथो॑ अ॒नयोः᳚ । अथो॒ इत्यथो᳚ । अ॒नयो॑रे॒व । ए॒व प्रति॑ । प्रति॑ तिष्ठन्ति । ति॒ष्ठ॒न्त्ये॒ते । ए॒ते वै । ए॒ते इत्ये॒ते । वै य॒ज्ञ्स्य॑ । य॒ज्ञ्स्या᳚ञ्ज॒साय॑नी । अ॒ञ्ज॒साय॑नी स्रु॒ती । अ॒ञ्ज॒साय॑नी॒ इत्य॑ञ्जसा - अय॑नी । स्रु॒ती ताभ्या᳚म् । स्रु॒ती इति॑ स्रु॒ती । ताभ्या॑मे॒व । ए॒व सु॑व॒र्गम् । सु॒व॒र्गम् ॅलो॒कम् । सु॒व॒र्गमिति॑ सुवः - गम् । लो॒कम् ॅय॑न्ति । य॒न्ति॒ त्रि॒वृत् । त्रि॒वृद॑ग्निष्टो॒मः । त्रि॒वृदिति॑ त्रि - वृत् । अ॒ग्नि॒ष्टो॒मो भ॑वति । अ॒ग्नि॒ष्टो॒म इत्य॑ग्नि - स्तो॒मः । भ॒व॒ति॒ तेजः॑ । तेज॑ ए॒व । ए॒वाव॑ । अव॑ रुन्धते । रु॒न्ध॒ते॒ प॒ञ्च॒द॒शः । प॒ञ्च॒द॒शो भ॑वति । प॒ञ्च॒द॒श इति॑ पञ्च - द॒शः । भ॒व॒ती॒न्द्रि॒यम् । इ॒न्द्रि॒यमे॒व । ए॒वाव॑ । अव॑ रुन्धते । रु॒न्ध॒ते॒ स॒प्त॒द॒शः । स॒प्त॒द॒शो भ॑वति । स॒प्त॒द॒श इति॑ सप्त - द॒शः \newline

\textbf{Jatai Paata} \newline

1. पृ॒ष्ठै रे॒वैव पृ॒ष्ठैः पृ॒ष्ठै रे॒व । \newline
2. ए॒व र्‌तू नृ॒तूने॒ वैव र्‌तून् । \newline
3. ऋ॒तून॒ न्वारो॑ह न्त्य॒न्वारो॑ह न्त्यृ॒तू नृ॒तून॒ न्वारो॑हन्ति । \newline
4. अ॒न्वारो॑ह न्त्यृ॒तुभिर्॑. ऋ॒तुभि॑ र॒न्वारो॑ह न्त्य॒न्वारो॑ह न्त्यृ॒तुभिः॑ । \newline
5. अ॒न्वारो॑ह॒न्तीत्य॑नु - आरो॑हन्ति । \newline
6. ऋ॒तुभिः॑ संॅवथ्स॒रꣳ सं॑ॅवथ्स॒र मृ॒तुभिर्॑. ऋ॒तुभिः॑ संॅवथ्स॒रम् । \newline
7. ऋ॒तुभि॒रित्यृ॒तु - भिः॒ । \newline
8. सं॒ॅव॒थ्स॒रम् ते ते सं॑ॅवथ्स॒रꣳ सं॑ॅवथ्स॒रम् ते । \newline
9. सं॒ॅव॒थ्स॒रमिति॑ सं - व॒थ्स॒रम् । \newline
10. ते सं॑ॅवथ्स॒रे सं॑ॅवथ्स॒रे ते ते सं॑ॅवथ्स॒रे । \newline
11. सं॒ॅव॒थ्स॒र ए॒वैव सं॑ॅवथ्स॒रे सं॑ॅवथ्स॒र ए॒व । \newline
12. सं॒ॅव॒थ्स॒र इति॑ सं - व॒थ्स॒रे । \newline
13. ए॒व प्रति॒ प्रत्ये॒वैव प्रति॑ । \newline
14. प्रति॑ तिष्ठन्ति तिष्ठन्ति॒ प्रति॒ प्रति॑ तिष्ठन्ति । \newline
15. ति॒ष्ठ॒न्ति॒ बृ॒ह॒द्र॒थ॒न्त॒राभ्या᳚म् बृहद्रथन्त॒राभ्या᳚म् तिष्ठन्ति तिष्ठन्ति बृहद्रथन्त॒राभ्या᳚म् । \newline
16. बृ॒ह॒द्र॒थ॒न्त॒राभ्यां᳚ ॅयन्ति यन्ति बृहद्रथन्त॒राभ्या᳚म् बृहद्रथन्त॒राभ्यां᳚ ॅयन्ति । \newline
17. बृ॒ह॒द्र॒थ॒न्त॒राभ्या॒मिति॑ बृहत् - र॒थ॒न्त॒राभ्या᳚म् । \newline
18. य॒न्ती॒य मि॒यं ॅय॑न्ति यन्ती॒यम् । \newline
19. इ॒यं ॅवाव वावे य मि॒यं ॅवाव । \newline
20. वाव र॑थन्त॒रꣳ र॑थन्त॒रं ॅवाव वाव र॑थन्त॒रम् । \newline
21. र॒थ॒न्त॒र म॒सा व॒सौ र॑थन्त॒रꣳ र॑थन्त॒र म॒सौ । \newline
22. र॒थ॒न्त॒रमिति॑ रथं - त॒रम् । \newline
23. अ॒सौ बृ॒हद् बृ॒ह द॒सा व॒सौ बृ॒हत् । \newline
24. बृ॒ह दा॒भ्या मा॒भ्याम् बृ॒हद् बृ॒ह दा॒भ्याम् । \newline
25. आ॒भ्या मे॒वै वाभ्या मा॒भ्या मे॒व । \newline
26. ए॒व य॑न्ति यन्त्ये॒वैव य॑न्ति । \newline
27. य॒न्त्यथो॒ अथो॑ यन्ति य॒न्त्यथो᳚ । \newline
28. अथो॑ अ॒नयो॑ र॒नयो॒ रथो॒ अथो॑ अ॒नयोः᳚ । \newline
29. अथो॒ इत्यथो᳚ । \newline
30. अ॒नयो॑ रे॒वै वानयो॑ र॒नयो॑ रे॒व । \newline
31. ए॒व प्रति॒ प्रत्ये॒वैव प्रति॑ । \newline
32. प्रति॑ तिष्ठन्ति तिष्ठन्ति॒ प्रति॒ प्रति॑ तिष्ठन्ति । \newline
33. ति॒ष्ठ॒ न्त्ये॒ते ए॒ते ति॑ष्ठन्ति तिष्ठ न्त्ये॒ते । \newline
34. ए॒ते वै वा ए॒ते ए॒ते वै । \newline
35. ए॒ते इत्ये॒ते । \newline
36. वै य॒ज्ञ्स्य॑ य॒ज्ञ्स्य॒ वै वै य॒ज्ञ्स्य॑ । \newline
37. य॒ज्ञ्स्या᳚ञ्ज॒साय॑नी अञ्ज॒साय॑नी य॒ज्ञ्स्य॑ य॒ज्ञ्स्या᳚ञ्ज॒साय॑नी । \newline
38. अ॒ञ्ज॒साय॑नी स्रु॒ती स्रु॒ती अ॑ञ्ज॒साय॑नी अञ्ज॒साय॑नी स्रु॒ती । \newline
39. अ॒ञ्ज॒साय॑नी॒ इत्य॑ञ्जसा - अय॑नी । \newline
40. स्रु॒ती ताभ्या॒म् ताभ्याꣳ॑ स्रु॒ती स्रु॒ती ताभ्या᳚म् । \newline
41. स्रु॒ती इति॑ स्रु॒ती । \newline
42. ताभ्या॑ मे॒वैव ताभ्या॒म् ताभ्या॑ मे॒व । \newline
43. ए॒व सु॑व॒र्गꣳ सु॑व॒र्ग मे॒वैव सु॑व॒र्गम् । \newline
44. सु॒व॒र्गम् ॅलो॒कम् ॅलो॒कꣳ सु॑व॒र्गꣳ सु॑व॒र्गम् ॅलो॒कम् । \newline
45. सु॒व॒र्गमिति॑ सुवः - गम् । \newline
46. लो॒कं ॅय॑न्ति यन्ति लो॒कम् ॅलो॒कं ॅय॑न्ति । \newline
47. य॒न्ति॒ त्रि॒वृत् त्रि॒वृद् य॑न्ति यन्ति त्रि॒वृत् । \newline
48. त्रि॒वृ द॑ग्निष्टो॒मो᳚ ऽग्निष्टो॒म स्त्रि॒वृत् त्रि॒वृ द॑ग्निष्टो॒मः । \newline
49. त्रि॒वृदिति॑ त्रि - वृत् । \newline
50. अ॒ग्नि॒ष्टो॒मो भ॑वति भव त्यग्निष्टो॒मो᳚ ऽग्निष्टो॒मो भ॑वति । \newline
51. अ॒ग्नि॒ष्टो॒म इत्य॑ग्नि - स्तो॒मः । \newline
52. भ॒व॒ति॒ तेज॒ स्तेजो॑ भवति भवति॒ तेजः॑ । \newline
53. तेज॑ ए॒वैव तेज॒ स्तेज॑ ए॒व । \newline
54. ए॒वावा वै॒वै वाव॑ । \newline
55. अव॑ रुन्धते रुन्ध॒ते ऽवाव॑ रुन्धते । \newline
56. रु॒न्ध॒ते॒ प॒ञ्च॒द॒शः प॑ञ्चद॒शो रु॑न्धते रुन्धते पञ्चद॒शः । \newline
57. प॒ञ्च॒द॒शो भ॑वति भवति पञ्चद॒शः प॑ञ्चद॒शो भ॑वति । \newline
58. प॒ञ्च॒द॒श इति॑ पञ्च - द॒शः । \newline
59. भ॒व॒ती॒न्द्रि॒य मि॑न्द्रि॒यम् भ॑वति भवतीन्द्रि॒यम् । \newline
60. इ॒न्द्रि॒य मे॒वैवेन्द्रि॒य मि॑न्द्रि॒य मे॒व । \newline
61. ए॒वावा वै॒वै वाव॑ । \newline
62. अव॑ रुन्धते रुन्ध॒ते ऽवाव॑ रुन्धते । \newline
63. रु॒न्ध॒ते॒ स॒प्त॒द॒शः स॑प्तद॒शो रु॑न्धते रुन्धते सप्तद॒शः । \newline
64. स॒प्त॒द॒शो भ॑वति भवति सप्तद॒शः स॑प्तद॒शो भ॑वति । \newline
65. स॒प्त॒द॒श इति॑ सप्त - द॒शः । \newline

\textbf{Ghana Paata } \newline

1. पृ॒ष्ठै रे॒वैव पृ॒ष्ठैः पृ॒ष्ठै रे॒व र्‌तू नृ॒तू ने॒व पृ॒ष्ठैः पृ॒ष्ठै रे॒व र्‌तून् । \newline
2. ए॒व र्‌तू नृ॒तू ने॒वैव र्‌तू न॒न्वारो॑ह न्त्य॒न्वारो॑ह न्त्यृ॒तू ने॒वैव र्‌तू न॒न्वारो॑हन्ति । \newline
3. ऋ॒तू न॒न्वारो॑ह न्त्य॒न्वारो॑ह न्त्यृ॒तू नृ॒तू न॒न्वारो॑ह न्त्यृ॒तुभिर्॑. ऋ॒तुभि॑ र॒न्वारो॑ह न्त्यृ॒तू नृ॒तून॒ न्वारो॑ह न्त्यृ॒तुभिः॑ । \newline
4. अ॒न्वारो॑ह न्त्यृ॒तुभिर्॑. ऋ॒तुभि॑ र॒न्वारो॑ह न्त्य॒न्वारो॑ह न्त्यृ॒तुभिः॑ संॅवथ्स॒रꣳ सं॑ॅवथ्स॒र मृ॒तुभि॑ र॒न्वारो॑ह न्त्य॒न्वारो॑ह न्त्यृ॒तुभिः॑ संॅवथ्स॒रम् । \newline
5. अ॒न्वारो॑ह॒न्तीत्य॑नु - आरो॑हन्ति । \newline
6. ऋ॒तुभिः॑ संॅवथ्स॒रꣳ सं॑ॅवथ्स॒र मृ॒तुभिर्॑. ऋ॒तुभिः॑ संॅवथ्स॒रम् ते ते सं॑ॅवथ्स॒र मृ॒तुभिर्॑. ऋ॒तुभिः॑ संॅवथ्स॒रम् ते । \newline
7. ऋ॒तुभि॒रित्यृ॒तु - भिः॒ । \newline
8. सं॒ॅव॒थ्स॒रम् ते ते सं॑ॅवथ्स॒रꣳ सं॑ॅवथ्स॒रम् ते सं॑ॅवथ्स॒रे सं॑ॅवथ्स॒रे ते सं॑ॅवथ्स॒रꣳ सं॑ॅवथ्स॒रम् ते सं॑ॅवथ्स॒रे । \newline
9. सं॒ॅव॒थ्स॒रमिति॑ सं - व॒थ्स॒रम् । \newline
10. ते सं॑ॅवथ्स॒रे सं॑ॅवथ्स॒रे ते ते सं॑ॅवथ्स॒र ए॒वैव सं॑ॅवथ्स॒रे ते ते सं॑ॅवथ्स॒र ए॒व । \newline
11. सं॒ॅव॒थ्स॒र ए॒वैव सं॑ॅवथ्स॒रे सं॑ॅवथ्स॒र ए॒व प्रति॒ प्रत्ये॒व सं॑ॅवथ्स॒रे सं॑ॅवथ्स॒र ए॒व प्रति॑ । \newline
12. सं॒ॅव॒थ्स॒र इति॑ सं - व॒थ्स॒रे । \newline
13. ए॒व प्रति॒ प्रत्ये॒वैव प्रति॑ तिष्ठन्ति तिष्ठन्ति॒ प्रत्ये॒वैव प्रति॑ तिष्ठन्ति । \newline
14. प्रति॑ तिष्ठन्ति तिष्ठन्ति॒ प्रति॒ प्रति॑ तिष्ठन्ति बृहद्रथन्त॒राभ्या᳚म् बृहद्रथन्त॒राभ्या᳚म् तिष्ठन्ति॒ प्रति॒ प्रति॑ तिष्ठन्ति बृहद्रथन्त॒राभ्या᳚म् । \newline
15. ति॒ष्ठ॒न्ति॒ बृ॒ह॒द्र॒थ॒न्त॒राभ्या᳚म् बृहद्रथन्त॒राभ्या᳚म् तिष्ठन्ति तिष्ठन्ति बृहद्रथन्त॒राभ्यां᳚ ॅयन्ति यन्ति बृहद्रथन्त॒राभ्या᳚म् तिष्ठन्ति तिष्ठन्ति बृहद्रथन्त॒राभ्यां᳚ ॅयन्ति । \newline
16. बृ॒ह॒द्र॒थ॒न्त॒राभ्यां᳚ ॅयन्ति यन्ति बृहद्रथन्त॒राभ्या᳚म् बृहद्रथन्त॒राभ्यां᳚ ॅयन्ती॒य मि॒यं ॅय॑न्ति बृहद्रथन्त॒राभ्या᳚म् बृहद्रथन्त॒राभ्यां᳚ ॅयन्ती॒यम् । \newline
17. बृ॒ह॒द्र॒थ॒न्त॒राभ्या॒मिति॑ बृहत् - र॒थ॒न्त॒राभ्या᳚म् । \newline
18. य॒न्ती॒य मि॒यं ॅय॑न्ति यन्ती॒यं ॅवाव वावेयं ॅय॑न्ति यन्ती॒यं ॅवाव । \newline
19. इ॒यं ॅवाव वावेय मि॒यं ॅवाव र॑थन्त॒रꣳ र॑थन्त॒रं ॅवावेय मि॒यं ॅवाव र॑थन्त॒रम् । \newline
20. वाव र॑थन्त॒रꣳ र॑थन्त॒रं ॅवाव वाव र॑थन्त॒र म॒सा व॒सौ र॑थन्त॒रं ॅवाव वाव र॑थन्त॒र म॒सौ । \newline
21. र॒थ॒न्त॒र म॒सा व॒सौ र॑थन्त॒रꣳ र॑थन्त॒र म॒सौ बृ॒हद् बृ॒ह द॒सौ र॑थन्त॒रꣳ र॑थन्त॒र म॒सौ बृ॒हत् । \newline
22. र॒थ॒न्त॒रमिति॑ रथं - त॒रम् । \newline
23. अ॒सौ बृ॒हद् बृ॒ह द॒सा व॒सौ बृ॒ह दा॒भ्या मा॒भ्याम् बृ॒ह द॒सा व॒सौ बृ॒ह दा॒भ्याम् । \newline
24. बृ॒ह दा॒भ्या मा॒भ्याम् बृ॒हद् बृ॒ह दा॒भ्या मे॒वै वाभ्याम् बृ॒हद् बृ॒ह दा॒भ्या मे॒व । \newline
25. आ॒भ्या मे॒वै वाभ्या मा॒भ्या मे॒व य॑न्ति यन्त्ये॒वाभ्या मा॒भ्या मे॒व य॑न्ति । \newline
26. ए॒व य॑न्ति यन्त्ये॒वैव य॒न्त्यथो॒ अथो॑ यन्त्ये॒ वैव य॒न्त्यथो᳚ । \newline
27. य॒न्त्यथो॒ अथो॑ यन्ति य॒न्त्यथो॑ अ॒नयो॑ र॒नयो॒ रथो॑ यन्ति य॒न्त्यथो॑ अ॒नयोः᳚ । \newline
28. अथो॑ अ॒नयो॑ र॒नयो॒ रथो॒ अथो॑ अ॒नयो॑ रे॒वैवा नयो॒ रथो॒ अथो॑ अ॒नयो॑ रे॒व । \newline
29. अथो॒ इत्यथो᳚ । \newline
30. अ॒नयो॑ रे॒वै वानयो॑ र॒नयो॑ रे॒व प्रति॒ प्रत्ये॒ वानयो॑ र॒नयो॑ रे॒व प्रति॑ । \newline
31. ए॒व प्रति॒ प्रत्ये॒ वैव प्रति॑ तिष्ठन्ति तिष्ठन्ति॒ प्रत्ये॒ वैव प्रति॑ तिष्ठन्ति । \newline
32. प्रति॑ तिष्ठन्ति तिष्ठन्ति॒ प्रति॒ प्रति॑ तिष्ठ न्त्ये॒ते ए॒ते ति॑ष्ठन्ति॒ प्रति॒ प्रति॑ तिष्ठ न्त्ये॒ते । \newline
33. ति॒ष्ठ॒ न्त्ये॒ते ए॒ते ति॑ष्ठन्ति तिष्ठ न्त्ये॒ते वै वा ए॒ते ति॑ष्ठन्ति तिष्ठ न्त्ये॒ते वै । \newline
34. ए॒ते वै वा ए॒ते ए॒ते वै य॒ज्ञ्स्य॑ य॒ज्ञ्स्य॒ वा ए॒ते ए॒ते वै य॒ज्ञ्स्य॑ । \newline
35. ए॒ते इत्ये॒ते । \newline
36. वै य॒ज्ञ्स्य॑ य॒ज्ञ्स्य॒ वै वै य॒ज्ञ्स्या᳚ ञ्ज॒साय॑नी अञ्ज॒साय॑नी य॒ज्ञ्स्य॒ वै वै य॒ज्ञ्स्या ᳚ञ्ज॒साय॑नी । \newline
37. य॒ज्ञ्स्या᳚ ञ्ज॒साय॑नी अञ्ज॒साय॑नी य॒ज्ञ्स्य॑ य॒ज्ञ्स्या᳚ ञ्ज॒साय॑नी स्रु॒ती स्रु॒ती अ॑ञ्ज॒साय॑नी य॒ज्ञ्स्य॑ य॒ज्ञ्स्या᳚ ञ्ज॒साय॑नी स्रु॒ती । \newline
38. अ॒ञ्ज॒साय॑नी स्रु॒ती स्रु॒ती अ॑ञ्ज॒साय॑नी अञ्ज॒साय॑नी स्रु॒ती ताभ्या॒म् ताभ्याꣳ॑ स्रु॒ती अ॑ञ्ज॒साय॑नी अञ्ज॒साय॑नी स्रु॒ती ताभ्या᳚म् । \newline
39. अ॒ञ्ज॒साय॑नी॒ इत्य॑ञ्जसा - अय॑नी । \newline
40. स्रु॒ती ताभ्या॒म् ताभ्याꣳ॑ स्रु॒ती स्रु॒ती ताभ्या॑ मे॒वैव ताभ्याꣳ॑ स्रु॒ती स्रु॒ती ताभ्या॑ मे॒व । \newline
41. स्रु॒ती इति॑ स्रु॒ती । \newline
42. ताभ्या॑ मे॒वैव ताभ्या॒म् ताभ्या॑ मे॒व सु॑व॒र्गꣳ सु॑व॒र्ग मे॒व ताभ्या॒म् ताभ्या॑ मे॒व सु॑व॒र्गम् । \newline
43. ए॒व सु॑व॒र्गꣳ सु॑व॒र्ग मे॒वैव सु॑व॒र्गम् ॅलो॒कम् ॅलो॒कꣳ सु॑व॒र्ग मे॒वैव सु॑व॒र्गम् ॅलो॒कम् । \newline
44. सु॒व॒र्गम् ॅलो॒कम् ॅलो॒कꣳ सु॑व॒र्गꣳ सु॑व॒र्गम् ॅलो॒कं ॅय॑न्ति यन्ति लो॒कꣳ सु॑व॒र्गꣳ सु॑व॒र्गम् ॅलो॒कं ॅय॑न्ति । \newline
45. सु॒व॒र्गमिति॑ सुवः - गम् । \newline
46. लो॒कं ॅय॑न्ति यन्ति लो॒कम् ॅलो॒कं ॅय॑न्ति त्रि॒वृत् त्रि॒वृद् य॑न्ति लो॒कम् ॅलो॒कं ॅय॑न्ति त्रि॒वृत् । \newline
47. य॒न्ति॒ त्रि॒वृत् त्रि॒वृद् य॑न्ति यन्ति त्रि॒वृ द॑ग्निष्टो॒मो᳚ ऽग्निष्टो॒म स्त्रि॒वृद् य॑न्ति यन्ति त्रि॒वृ द॑ग्निष्टो॒मः । \newline
48. त्रि॒वृ द॑ग्निष्टो॒मो᳚ ऽग्निष्टो॒म स्त्रि॒वृत् त्रि॒वृ द॑ग्निष्टो॒मो भ॑वति भव त्यग्निष्टो॒म स्त्रि॒वृत् त्रि॒वृ द॑ग्निष्टो॒मो भ॑वति । \newline
49. त्रि॒वृदिति॑ त्रि - वृत् । \newline
50. अ॒ग्नि॒ष्टो॒मो भ॑वति भव त्यग्निष्टो॒मो᳚ ऽग्निष्टो॒मो भ॑वति॒ तेज॒ स्तेजो॑ भव त्यग्निष्टो॒मो᳚ ऽग्निष्टो॒मो भ॑वति॒ तेजः॑ । \newline
51. अ॒ग्नि॒ष्टो॒म इत्य॑ग्नि - स्तो॒मः । \newline
52. भ॒व॒ति॒ तेज॒ स्तेजो॑ भवति भवति॒ तेज॑ ए॒वैव तेजो॑ भवति भवति॒ तेज॑ ए॒व । \newline
53. तेज॑ ए॒वैव तेज॒ स्तेज॑ ए॒वावा वै॒व तेज॒ स्तेज॑ ए॒वाव॑ । \newline
54. ए॒वावा वै॒वै वाव॑ रुन्धते रुन्ध॒ते ऽवै॒वै वाव॑ रुन्धते । \newline
55. अव॑ रुन्धते रुन्ध॒ते ऽवाव॑ रुन्धते पञ्चद॒शः प॑ञ्चद॒शो रु॑न्ध॒ते ऽवाव॑ रुन्धते पञ्चद॒शः । \newline
56. रु॒न्ध॒ते॒ प॒ञ्च॒द॒शः प॑ञ्चद॒शो रु॑न्धते रुन्धते पञ्चद॒शो भ॑वति भवति पञ्चद॒शो रु॑न्धते रुन्धते पञ्चद॒शो भ॑वति । \newline
57. प॒ञ्च॒द॒शो भ॑वति भवति पञ्चद॒शः प॑ञ्चद॒शो भ॑वतीन्द्रि॒य मि॑न्द्रि॒यम् भ॑वति पञ्चद॒शः प॑ञ्चद॒शो भ॑वतीन्द्रि॒यम् । \newline
58. प॒ञ्च॒द॒श इति॑ पञ्च - द॒शः । \newline
59. भ॒व॒ती॒न्द्रि॒य मि॑न्द्रि॒यम् भ॑वति भवतीन्द्रि॒य मे॒वै वेन्द्रि॒यम् भ॑वति भवतीन्द्रि॒य मे॒व । \newline
60. इ॒न्द्रि॒य मे॒वै वेन्द्रि॒य मि॑न्द्रि॒य मे॒वावा वै॒वेन्द्रि॒य मि॑न्द्रि॒य मे॒वाव॑ । \newline
61. ए॒वावा वै॒वै वाव॑ रुन्धते रुन्ध॒ते ऽवै॒वै वाव॑ रुन्धते । \newline
62. अव॑ रुन्धते रुन्ध॒ते ऽवाव॑ रुन्धते सप्तद॒शः स॑प्तद॒शो रु॑न्ध॒ते ऽवाव॑ रुन्धते सप्तद॒शः । \newline
63. रु॒न्ध॒ते॒ स॒प्त॒द॒शः स॑प्तद॒शो रु॑न्धते रुन्धते सप्तद॒शो भ॑वति भवति सप्तद॒शो रु॑न्धते रुन्धते सप्तद॒शो भ॑वति । \newline
64. स॒प्त॒द॒शो भ॑वति भवति सप्तद॒शः स॑प्तद॒शो भ॑व त्य॒न्नाद्य॑स्या॒ न्नाद्य॑स्य भवति सप्तद॒शः स॑प्तद॒शो भ॑व त्य॒न्नाद्य॑स्य । \newline
65. स॒प्त॒द॒श इति॑ सप्त - द॒शः । \newline
\pagebreak
\markright{ TS 7.2.1.3  \hfill https://www.vedavms.in \hfill}

\section{ TS 7.2.1.3 }

\textbf{TS 7.2.1.3 } \newline
\textbf{Samhita Paata} \newline

भ॑व-त्य॒न्नाद्य॒स्या-व॑रुद्ध्या॒ अथो॒ प्रैव तेन॑ जायन्त एकविꣳ॒॒शो भ॑वति॒ प्रति॑ष्ठित्या॒ अथो॒ रुच॑मे॒वाऽऽत्मन् द॑धते त्रिण॒वो भ॑वति॒ विजि॑त्यै त्रयस्त्रिꣳ॒॒शो भ॑वति॒ प्रति॑ष्ठित्यै सदोहविर्द्धा॒निन॑ ए॒तेन॑ षड्-रा॒त्रेण॑ यजेर॒न्नाश्व॑त्थी हवि॒र्द्धानं॒ चाऽऽ*ग्नी᳚द्ध्रं च भवत॒स्तद्धि सु॑व॒र्ग्यं॑ च॒क्रीव॑ती भवतः सुव॒र्गस्य॑ लो॒कस्य॒ सम॑ष्ट्या उ॒लूख॑लबुद्ध्नो॒ यूपो॑ भवति॒ प्रति॑ष्ठित्यै॒ प्राञ्चो॑ यान्ति॒ प्राङि॑व॒ हि सु॑व॒र्गो - [  ] \newline

\textbf{Pada Paata} \newline

भ॒व॒ति॒ । अ॒न्नाद्य॒स्येत्य॑न्न - अद्य॑स्य । अव॑रुद्ध्या॒ इत्यव॑ - रु॒द्ध्यै॒ । अथो॒ इति॑ । प्रेति॑ । ए॒व । तेन॑ । जा॒य॒न्ते॒ । ए॒क॒विꣳ॒॒श इत्ये॑क - विꣳ॒॒शः । भ॒व॒ति॒ । प्रति॑ष्ठित्या॒ इति॒ प्रति॑ - स्थि॒त्यै॒ । अथो॒ इति॑ । रुच᳚म् । ए॒व । आ॒त्मन्न् । द॒ध॒ते॒ । त्रि॒ण॒व इति॑ त्रि - न॒वः । भ॒व॒ति॒ । विजि॑त्या॒ इति॒ वि - जि॒त्यै॒ । त्र॒य॒स्त्रिꣳ॒॒श इति॑ त्रयः - त्रिꣳ॒॒शः । भ॒व॒ति॒ । प्रति॑ष्ठित्या॒ इति॒ प्रति॑ - स्थि॒त्यै॒ । स॒दो॒ह॒वि॒द्‌र्धा॒निन॒ इति॑ सदः - ह॒वि॒द्‌र्धा॒निनः॑ । ए॒तेन॑ । ष॒ड्रा॒त्रेणेति॑ षट् - रा॒त्रेण॑ । य॒जे॒र॒न्न् । आश्व॑त्थी॒ इति॑ । ह॒वि॒द्‌र्धान॒मिति॑ हविः - धान᳚म् । च॒ । आग्नी᳚द्ध्र॒मित्याग्नि॑-इ॒द्ध्र॒म् । च॒ । भ॒व॒तः॒ । तत् । हि । सु॒व॒र्ग्य॑मिति॑ सुवः - ग्य᳚म् । च॒क्रीव॑ती॒ इति॑ । भ॒व॒तः॒ । सु॒व॒र्गस्येति॑ सुवः - गस्य॑ । लो॒कस्य॑ । सम॑ष्ट्या॒ इति॒ सं-अ॒ष्ट्यै॒ । उ॒लूख॑लबुद्ध्न॒ इत्यु॒लूख॑ल - बु॒द्ध्नः॒ । यूपः॑ । भ॒व॒ति॒ । प्रति॑ष्ठित्या॒ इति॒ प्रति॑ - स्थि॒त्यै॒ । प्राञ्चः॑ । या॒न्ति॒ । प्राङ् । इ॒व॒ । हि । सु॒व॒र्ग इति॑ सुवः - गः ।  \newline


\textbf{Krama Paata} \newline

भ॒व॒त्य॒न्नाद्य॑स्य । अ॒न्नाद्य॒स्या,व॑रुद्ध्यै । अ॒न्नाद्य॒स्येत्य॑न्न - अद्य॑स्य । अव॑रुद्ध्या॒ अथो᳚ । अव॑रुद्ध्या॒ इत्यव॑ - रु॒द्ध्यै॒ । अथो॒ प्र । अथो॒ इत्यथो᳚ । प्रैव । ए॒व तेन॑ । तेन॑ जायन्ते । जा॒य॒न्त॒ ए॒क॒विꣳ॒॒शः । ए॒क॒विꣳ॒॒शो भ॑वति । ए॒क॒विꣳ॒॒श इत्ये॑क - विꣳ॒॒शः । भ॒व॒ति॒ प्रति॑ष्ठित्यै । प्रति॑ष्ठित्या॒ अथो᳚ । प्रति॑ष्ठित्या॒ इति॒ प्रति॑ - स्थि॒त्यै॒ । अथो॒ रुच᳚म् । अथो॒ इत्यथो᳚ । रुच॑मे॒व । ए॒वात्मन्न् । आ॒त्मन् द॑धते । द॒ध॒ते॒ त्रि॒ण॒वः । त्रि॒ण॒वो भ॑वति । त्रि॒ण॒व इति॑ त्रि - न॒वः । भ॒व॒ति॒ विजि॑त्यै । विजि॑त्यै त्रयस्त्रिꣳ॒॒शः । विजि॑त्या॒ इति॒ वि - जि॒त्यै॒ । त्र॒य॒स्त्रिꣳ॒॒शो भ॑वति । त्र॒य॒स्त्रिꣳ॒॒श इति॑ त्रयः - त्रिꣳ॒॒शः । भ॒व॒ति॒ प्रति॑ष्ठित्यै । प्रति॑ष्ठित्यै सदोहविर्द्धा॒निनः॑ । प्रति॑ष्ठित्या॒ इति॒ प्रति॑ - स्थि॒त्यै॒ । स॒दो॒ह॒वि॒र्द्धा॒निन॑ ए॒तेन॑ । स॒दो॒ह॒वि॒र्द्धा॒निन॒ इति॑ सदः - ह॒वि॒र्द्धा॒निनः॑ । ए॒तेन॑ षड्‍रा॒त्रेण॑ । ष॒ड्‍रा॒त्रेण॑ यजेरन्न् । ष॒ड्‍रा॒त्रेणेति॑ षट् - रा॒त्रेण॑ । य॒जे॒र॒न्नाश्व॑त्थी । आश्व॑त्थी हवि॒र्द्धान᳚म् । आश्व॑त्थी॒ इत्याश्व॑त्थी । ह॒वि॒र्द्धान॑म् च । ह॒वि॒र्द्धान॒मिति॑ हविः - धान᳚म् । चाग्नी᳚द्ध्रम् । आग्नी᳚द्ध्रम् च । आग्नी᳚द्ध्र॒मित्याग्नि॑ - इ॒द्ध्र॒म् । च॒ भ॒व॒तः॒ । भ॒व॒त॒स्तत् । तद्‌धि । हि सु॑व॒र्ग्य᳚म् । सु॒व॒र्ग्य॑म् च॒क्रीव॑ती । सु॒व॒र्ग्य॑मिति॑ सुवः - ग्य᳚म् । च॒क्रीव॑ती भवतः । च॒क्रीव॑ती॒ इति॑ च॒क्रीव॑ती । भ॒व॒तः॒ सु॒व॒र्गस्य॑ । सु॒व॒र्गस्य॑ लो॒कस्य॑ । सु॒व॒र्गस्येति॑ सुवः - गस्य॑ । लो॒कस्य॒ सम॑ष्ट्‍यै । सम॑ष्ट्‍या उ॒लूख॑लबुद्ध्नः । सम॑ष्ट्‍या॒ इति॒ सम् - अ॒ष्ट्‍यै॒ । उ॒लूख॑लबुद्ध्नो॒ यूपः॑ । उ॒लूख॑लबुद्ध्न॒ इत्यु॒लूख॑ल - बु॒द्ध्नः॒ । यूपो॑ भवति । भ॒व॒ति॒ प्रति॑ष्ठित्यै । प्रति॑ष्ठित्यै॒ प्राञ्चः॑ । प्रति॑ष्ठित्या॒ इति॒ प्रति॑ - स्थि॒त्यै॒ । प्राञ्चो॑ यान्ति । या॒न्ति॒ प्राङ्‍ । प्राङि॑व । इ॒व॒ हि । हि सु॑व॒र्गः । सु॒व॒र्गो लो॒कः । सु॒व॒र्ग इति॑ सुवः - गः \newline

\textbf{Jatai Paata} \newline

1. भ॒व॒ त्य॒न्नाद्य॑स्या॒ न्नाद्य॑स्य भवति भव त्य॒न्नाद्य॑स्य । \newline
2. अ॒न्नाद्य॒स्या व॑रुद्ध्या॒ अव॑रुद्ध्या अ॒न्नाद्य॑स्या॒ न्नाद्य॒स्या व॑रुद्ध्यै । \newline
3. अ॒न्नाद्य॒स्येत्य॑न्न - अद्य॑स्य । \newline
4. अव॑रुद्ध्या॒ अथो॒ अथो॒ अव॑रुद्ध्या॒ अव॑रुद्ध्या॒ अथो᳚ । \newline
5. अव॑रुद्ध्या॒ इत्यव॑ - रु॒द्ध्यै॒ । \newline
6. अथो॒ प्र प्राथो॒ अथो॒ प्र । \newline
7. अथो॒ इत्यथो᳚ । \newline
8. प्रैवैव प्र प्रैव । \newline
9. ए॒व तेन॒ तेनै॒वैव तेन॑ । \newline
10. तेन॑ जायन्ते जायन्ते॒ तेन॒ तेन॑ जायन्ते । \newline
11. जा॒य॒न्त॒ ए॒क॒विꣳ॒॒श ए॑कविꣳ॒॒शो जा॑यन्ते जायन्त एकविꣳ॒॒शः । \newline
12. ए॒क॒विꣳ॒॒शो भ॑वति भव त्येकविꣳ॒॒श ए॑कविꣳ॒॒शो भ॑वति । \newline
13. ए॒क॒विꣳ॒॒श इत्ये॑क - विꣳ॒॒शः । \newline
14. भ॒व॒ति॒ प्रति॑ष्ठित्यै॒ प्रति॑ष्ठित्यै भवति भवति॒ प्रति॑ष्ठित्यै । \newline
15. प्रति॑ष्ठित्या॒ अथो॒ अथो॒ प्रति॑ष्ठित्यै॒ प्रति॑ष्ठित्या॒ अथो᳚ । \newline
16. प्रति॑ष्ठित्या॒ इति॒ प्रति॑ - स्थि॒त्यै॒ । \newline
17. अथो॒ रुचꣳ॒॒ रुच॒ मथो॒ अथो॒ रुच᳚म् । \newline
18. अथो॒ इत्यथो᳚ । \newline
19. रुच॑ मे॒वैव रुचꣳ॒॒ रुच॑ मे॒व । \newline
20. ए॒वात्मन् ना॒त्मन् ने॒वै वात्मन्न् । \newline
21. आ॒त्मन् द॑धते दधत आ॒त्मन् ना॒त्मन् द॑धते । \newline
22. द॒ध॒ते॒ त्रि॒ण॒व स्त्रि॑ण॒वो द॑धते दधते त्रिण॒वः । \newline
23. त्रि॒ण॒वो भ॑वति भवति त्रिण॒व स्त्रि॑ण॒वो भ॑वति । \newline
24. त्रि॒ण॒व इति॑ त्रि - न॒वः । \newline
25. भ॒व॒ति॒ विजि॑त्यै॒ विजि॑त्यै भवति भवति॒ विजि॑त्यै । \newline
26. विजि॑त्यै त्रयस्त्रिꣳ॒॒श स्त्र॑यस्त्रिꣳ॒॒शो विजि॑त्यै॒ विजि॑त्यै त्रयस्त्रिꣳ॒॒शः । \newline
27. विजि॑त्या॒ इति॒ वि - जि॒त्यै॒ । \newline
28. त्र॒य॒स्त्रिꣳ॒॒शो भ॑वति भवति त्रयस्त्रिꣳ॒॒श स्त्र॑यस्त्रिꣳ॒॒शो भ॑वति । \newline
29. त्र॒य॒स्त्रिꣳ॒॒श इति॑ त्रयः - त्रिꣳ॒॒शः । \newline
30. भ॒व॒ति॒ प्रति॑ष्ठित्यै॒ प्रति॑ष्ठित्यै भवति भवति॒ प्रति॑ष्ठित्यै । \newline
31. प्रति॑ष्ठित्यै सदोहविर्द्धा॒निनः॑ सदोहविर्द्धा॒निनः॒ प्रति॑ष्ठित्यै॒ प्रति॑ष्ठित्यै सदोहविर्द्धा॒निनः॑ । \newline
32. प्रति॑ष्ठित्या॒ इति॒ प्रति॑ - स्थि॒त्यै॒ । \newline
33. स॒दो॒ह॒वि॒र्द्धा॒निन॑ ए॒ते नै॒तेन॑ सदोहविर्द्धा॒निनः॑ सदोहविर्द्धा॒निन॑ ए॒तेन॑ । \newline
34. स॒दो॒ह॒वि॒र्द्धा॒निन॒ इति॑ सदः - ह॒वि॒र्द्धा॒निनः॑ । \newline
35. ए॒तेन॑ षड्रा॒त्रेण॑ षड्रा॒त्रे णै॒ते नै॒तेन॑ षड्रा॒त्रेण॑ । \newline
36. ष॒ड्रा॒त्रेण॑ यजेरन्. यजेरन् थ्षड्रा॒त्रेण॑ षड्रा॒त्रेण॑ यजेरन्न् । \newline
37. ष॒ड्रा॒त्रेणेति॑ षट् - रा॒त्रेण॑ । \newline
38. य॒जे॒र॒न् नाश्व॑त्थी॒ आश्व॑त्थी यजेरन्. यजेर॒न् नाश्व॑त्थी । \newline
39. आश्व॑त्थी हवि॒र्द्धानꣳ॑ हवि॒र्द्धान॒ माश्व॑त्थी॒ आश्व॑त्थी हवि॒र्द्धान᳚म् । \newline
40. आश्व॑त्थी॒ इत्याश्व॑त्थी । \newline
41. ह॒वि॒र्द्धान॑म् च च हवि॒र्द्धानꣳ॑ हवि॒र्द्धान॑म् च । \newline
42. ह॒वि॒र्द्धान॒मिति॑ हविः - धान᳚म् । \newline
43. चाग्नी᳚द्ध्र॒ माग्नी᳚द्ध्रम् च॒ चाग्नी᳚द्ध्रम् । \newline
44. आग्नी᳚द्ध्रम् च॒ चाग्नी᳚द्ध्र॒ माग्नी᳚द्ध्रम् च । \newline
45. आग्नी᳚द्ध्र॒मित्याग्नि॑ - इ॒द्ध्र॒म् । \newline
46. च॒ भ॒व॒तो॒ भ॒व॒त॒ श्च॒ च॒ भ॒व॒तः॒ । \newline
47. भ॒व॒त॒ स्तत् तद् भ॑वतो भवत॒ स्तत् । \newline
48. तद्धि हि तत् तद्धि । \newline
49. हि सु॑व॒र्ग्यꣳ॑ सुव॒र्ग्यꣳ॑ हि हि सु॑व॒र्ग्य᳚म् । \newline
50. सु॒व॒र्ग्य॑म् च॒क्रीव॑ती च॒क्रीव॑ती सुव॒र्ग्यꣳ॑ सुव॒र्ग्य॑म् च॒क्रीव॑ती । \newline
51. सु॒व॒र्ग्य॑मिति॑ सुवः - ग्य᳚म् । \newline
52. च॒क्रीव॑ती भवतो भवत श्च॒क्रीव॑ती च॒क्रीव॑ती भवतः । \newline
53. च॒क्रीव॑ती॒ इति॑ च॒क्रीव॑ती । \newline
54. भ॒व॒तः॒ सु॒व॒र्गस्य॑ सुव॒र्गस्य॑ भवतो भवतः सुव॒र्गस्य॑ । \newline
55. सु॒व॒र्गस्य॑ लो॒कस्य॑ लो॒कस्य॑ सुव॒र्गस्य॑ सुव॒र्गस्य॑ लो॒कस्य॑ । \newline
56. सु॒व॒र्गस्येति॑ सुवः - गस्य॑ । \newline
57. लो॒कस्य॒ सम॑ष्ट्यै॒ सम॑ष्ट्यै लो॒कस्य॑ लो॒कस्य॒ सम॑ष्ट्यै । \newline
58. सम॑ष्ट्या उ॒लूख॑लबुद्ध्न उ॒लूख॑लबुद्ध्नः॒ सम॑ष्ट्यै॒ सम॑ष्ट्या उ॒लूख॑लबुद्ध्नः । \newline
59. सम॑ष्ट्या॒ इति॒ सं - अ॒ष्ट्यै॒ । \newline
60. उ॒लूख॑लबुद्ध्नो॒ यूपो॒ यूप॑ उ॒लूख॑लबुद्ध्न उ॒लूख॑लबुद्ध्नो॒ यूपः॑ । \newline
61. उ॒लूख॑लबुद्ध्न॒ इत्यु॒लूख॑ल - बु॒द्ध्नः॒ । \newline
62. यूपो॑ भवति भवति॒ यूपो॒ यूपो॑ भवति । \newline
63. भ॒व॒ति॒ प्रति॑ष्ठित्यै॒ प्रति॑ष्ठित्यै भवति भवति॒ प्रति॑ष्ठित्यै । \newline
64. प्रति॑ष्ठित्यै॒ प्राञ्चः॒ प्राञ्चः॒ प्रति॑ष्ठित्यै॒ प्रति॑ष्ठित्यै॒ प्राञ्चः॑ । \newline
65. प्रति॑ष्ठित्या॒ इति॒ प्रति॑ - स्थि॒त्यै॒ । \newline
66. प्राञ्चो॑ यान्ति यान्ति॒ प्राञ्चः॒ प्राञ्चो॑ यान्ति । \newline
67. या॒न्ति॒ प्राङ् प्राङ् या᳚न्ति यान्ति॒ प्राङ् । \newline
68. प्राङ् ङि॑वेव॒ प्राङ् प्राङ् ङि॑व । \newline
69. इ॒व॒ हि हीवे॑व॒ हि । \newline
70. हि सु॑व॒र्गः सु॑व॒र्गो हि हि सु॑व॒र्गः । \newline
71. सु॒व॒र्गो लो॒को लो॒कः सु॑व॒र्गः सु॑व॒र्गो लो॒कः । \newline
72. सु॒व॒र्ग इति॑ सुवः - गः । \newline

\textbf{Ghana Paata } \newline

1. भ॒व॒ त्य॒न्नाद्य॑स्या॒ न्नाद्य॑स्य भवति भव त्य॒न्नाद्य॒स्या व॑रुद्ध्या॒ अव॑रुद्ध्या अ॒न्नाद्य॑स्य भवति भव त्य॒न्नाद्य॒स्या व॑रुद्ध्यै । \newline
2. अ॒न्नाद्य॒स्या व॑रुद्ध्या॒ अव॑रुद्ध्या अ॒न्नाद्य॑ स्या॒न्नाद्य॒स्या व॑रुद्ध्या॒ अथो॒ अथो॒ अव॑रुद्ध्या अ॒न्नाद्य॑ स्या॒न्नाद्य॒स्या व॑रुद्ध्या॒ अथो᳚ । \newline
3. अ॒न्नाद्य॒स्येत्य॑न्न - अद्य॑स्य । \newline
4. अव॑रुद्ध्या॒ अथो॒ अथो॒ अव॑रुद्ध्या॒ अव॑रुद्ध्या॒ अथो॒ प्र प्राथो॒ अव॑रुद्ध्या॒ अव॑रुद्ध्या॒ अथो॒ प्र । \newline
5. अव॑रुद्ध्या॒ इत्यव॑ - रु॒द्ध्यै॒ । \newline
6. अथो॒ प्र प्राथो॒ अथो॒ प्रैवैव प्राथो॒ अथो॒ प्रैव । \newline
7. अथो॒ इत्यथो᳚ । \newline
8. प्रैवैव प्र प्रैव तेन॒ तेनै॒व प्र प्रैव तेन॑ । \newline
9. ए॒व तेन॒ तेनै॒ वैव तेन॑ जायन्ते जायन्ते॒ तेनै॒ वैव तेन॑ जायन्ते । \newline
10. तेन॑ जायन्ते जायन्ते॒ तेन॒ तेन॑ जायन्त एकविꣳ॒॒श ए॑कविꣳ॒॒शो जा॑यन्ते॒ तेन॒ तेन॑ जायन्त एकविꣳ॒॒शः । \newline
11. जा॒य॒न्त॒ ए॒क॒विꣳ॒॒श ए॑कविꣳ॒॒शो जा॑यन्ते जायन्त एकविꣳ॒॒शो भ॑वति भव त्येकविꣳ॒॒शो जा॑यन्ते जायन्त एकविꣳ॒॒शो भ॑वति । \newline
12. ए॒क॒विꣳ॒॒शो भ॑वति भव त्येकविꣳ॒॒श ए॑कविꣳ॒॒शो भ॑वति॒ प्रति॑ष्ठित्यै॒ प्रति॑ष्ठित्यै भव त्येकविꣳ॒॒श ए॑कविꣳ॒॒शो भ॑वति॒ प्रति॑ष्ठित्यै । \newline
13. ए॒क॒विꣳ॒॒श इत्ये॑क - विꣳ॒॒शः । \newline
14. भ॒व॒ति॒ प्रति॑ष्ठित्यै॒ प्रति॑ष्ठित्यै भवति भवति॒ प्रति॑ष्ठित्या॒ अथो॒ अथो॒ प्रति॑ष्ठित्यै भवति भवति॒ प्रति॑ष्ठित्या॒ अथो᳚ । \newline
15. प्रति॑ष्ठित्या॒ अथो॒ अथो॒ प्रति॑ष्ठित्यै॒ प्रति॑ष्ठित्या॒ अथो॒ रुचꣳ॒॒ रुच॒ मथो॒ प्रति॑ष्ठित्यै॒ प्रति॑ष्ठित्या॒ अथो॒ रुच᳚म् । \newline
16. प्रति॑ष्ठित्या॒ इति॒ प्रति॑ - स्थि॒त्यै॒ । \newline
17. अथो॒ रुचꣳ॒॒ रुच॒ मथो॒ अथो॒ रुच॑ मे॒वैव रुच॒ मथो॒ अथो॒ रुच॑ मे॒व । \newline
18. अथो॒ इत्यथो᳚ । \newline
19. रुच॑ मे॒वैव रुचꣳ॒॒ रुच॑ मे॒वात्मन् ना॒त्मन् ने॒व रुचꣳ॒॒ रुच॑ मे॒वात्मन्न् । \newline
20. ए॒वात्मन् ना॒त्मन् ने॒वै वात्मन् द॑धते दधत आ॒त्मन् ने॒वै वात्मन् द॑धते । \newline
21. आ॒त्मन् द॑धते दधत आ॒त्मन् ना॒त्मन् द॑धते त्रिण॒व स्त्रि॑ण॒वो द॑धत आ॒त्मन् ना॒त्मन् द॑धते त्रिण॒वः । \newline
22. द॒ध॒ते॒ त्रि॒ण॒व स्त्रि॑ण॒वो द॑धते दधते त्रिण॒वो भ॑वति भवति त्रिण॒वो द॑धते दधते त्रिण॒वो भ॑वति । \newline
23. त्रि॒ण॒वो भ॑वति भवति त्रिण॒व स्त्रि॑ण॒वो भ॑वति॒ विजि॑त्यै॒ विजि॑त्यै भवति त्रिण॒व स्त्रि॑ण॒वो भ॑वति॒ विजि॑त्यै । \newline
24. त्रि॒ण॒व इति॑ त्रि - न॒वः । \newline
25. भ॒व॒ति॒ विजि॑त्यै॒ विजि॑त्यै भवति भवति॒ विजि॑त्यै त्रयस्त्रिꣳ॒॒श स्त्र॑यस्त्रिꣳ॒॒शो विजि॑त्यै भवति भवति॒ विजि॑त्यै त्रयस्त्रिꣳ॒॒शः । \newline
26. विजि॑त्यै त्रयस्त्रिꣳ॒॒श स्त्र॑यस्त्रिꣳ॒॒शो विजि॑त्यै॒ विजि॑त्यै त्रयस्त्रिꣳ॒॒शो भ॑वति भवति त्रयस्त्रिꣳ॒॒शो विजि॑त्यै॒ विजि॑त्यै त्रयस्त्रिꣳ॒॒शो भ॑वति । \newline
27. विजि॑त्या॒ इति॒ वि - जि॒त्यै॒ । \newline
28. त्र॒य॒स्त्रिꣳ॒॒शो भ॑वति भवति त्रयस्त्रिꣳ॒॒श स्त्र॑यस्त्रिꣳ॒॒शो भ॑वति॒ प्रति॑ष्ठित्यै॒ प्रति॑ष्ठित्यै भवति त्रयस्त्रिꣳ॒॒श स्त्र॑यस्त्रिꣳ॒॒शो भ॑वति॒ प्रति॑ष्ठित्यै । \newline
29. त्र॒य॒स्त्रिꣳ॒॒श इति॑ त्रयः - त्रिꣳ॒॒शः । \newline
30. भ॒व॒ति॒ प्रति॑ष्ठित्यै॒ प्रति॑ष्ठित्यै भवति भवति॒ प्रति॑ष्ठित्यै सदोहविर्द्धा॒निनः॑ सदोहविर्द्धा॒निनः॒ प्रति॑ष्ठित्यै भवति भवति॒ प्रति॑ष्ठित्यै सदोहविर्द्धा॒निनः॑ । \newline
31. प्रति॑ष्ठित्यै सदोहविर्द्धा॒निनः॑ सदोहविर्द्धा॒निनः॒ प्रति॑ष्ठित्यै॒ प्रति॑ष्ठित्यै सदोहविर्द्धा॒निन॑ 
ए॒ते नै॒तेन॑ सदोहविर्द्धा॒निनः॒ प्रति॑ष्ठित्यै॒ प्रति॑ष्ठित्यै सदोहविर्द्धा॒निन॑ ए॒तेन॑ । \newline
32. प्रति॑ष्ठित्या॒ इति॒ प्रति॑ - स्थि॒त्यै॒ । \newline
33. स॒दो॒ह॒वि॒र्द्धा॒निन॑ ए॒ते नै॒तेन॑ सदोहविर्द्धा॒निनः॑ सदोहविर्द्धा॒निन॑ ए॒तेन॑ षड्रा॒त्रेण॑ षड्रा॒त्रे णै॒तेन॑ सदोहविर्द्धा॒निनः॑ सदोहविर्द्धा॒निन॑ ए॒तेन॑ षड्रा॒त्रेण॑ । \newline
34. स॒दो॒ह॒वि॒र्द्धा॒निन॒ इति॑ सदः - ह॒वि॒र्द्धा॒निनः॑ । \newline
35. ए॒तेन॑ षड्रा॒त्रेण॑ षड्रा॒त्रे णै॒ते नै॒तेन॑ षड्रा॒त्रेण॑ यजेरन्. यजेरन् थ्षड्रा॒त्रे णै॒ते नै॒तेन॑ षड्रा॒त्रेण॑ यजेरन्न् । \newline
36. ष॒ड्रा॒त्रेण॑ यजेरन्. यजेरन् थ्षड्रा॒त्रेण॑ षड्रा॒त्रेण॑ यजेर॒न् नाश्व॑त्थी॒ आश्व॑त्थी यजेरन् थ्षड्रा॒त्रेण॑ षड्रा॒त्रेण॑ यजेर॒न् नाश्व॑त्थी । \newline
37. ष॒ड्रा॒त्रेणेति॑ षट् - रा॒त्रेण॑ । \newline
38. य॒जे॒र॒न् नाश्व॑त्थी॒ आश्व॑त्थी यजेरन्. यजेर॒न् नाश्व॑त्थी हवि॒र्द्धानꣳ॑ हवि॒र्द्धान॒ माश्व॑त्थी यजेरन्. यजेर॒न् नाश्व॑त्थी हवि॒र्द्धान᳚म् । \newline
39. आश्व॑त्थी हवि॒र्द्धानꣳ॑ हवि॒र्द्धान॒ माश्व॑त्थी॒ आश्व॑त्थी हवि॒र्द्धान॑म् च च हवि॒र्द्धान॒ माश्व॑त्थी॒ आश्व॑त्थी हवि॒र्द्धान॑म् च । \newline
40. आश्व॑त्थी॒ इत्याश्व॑त्थी । \newline
41. ह॒वि॒र्द्धान॑म् च च हवि॒र्द्धानꣳ॑ हवि॒र्द्धान॒म् चाग्नी᳚द्ध्र॒ माग्नी᳚द्ध्रम् च हवि॒र्द्धानꣳ॑ हवि॒र्द्धान॒म् चाग्नी᳚द्ध्रम् । \newline
42. ह॒वि॒र्द्धान॒मिति॑ हविः - धान᳚म् । \newline
43. चाग्नी᳚द्ध्र॒ माग्नी᳚द्ध्रम् च॒ चाग्नी᳚द्ध्रम् च॒ चाग्नी᳚द्ध्रम् च॒ चाग्नी᳚द्ध्रम् च । \newline
44. आग्नी᳚द्ध्रम् च॒ चाग्नी᳚द्ध्र॒ माग्नी᳚द्ध्रम् च भवतो भवत॒ श्चाग्नी᳚द्ध्र॒ माग्नी᳚द्ध्रम् च भवतः । \newline
45. आग्नी᳚द्ध्र॒मित्याग्नि॑ - इ॒द्ध्र॒म् । \newline
46. च॒ भ॒व॒तो॒ भ॒व॒त॒ श्च॒ च॒ भ॒व॒त॒ स्तत् तद् भ॑वत श्च च भवत॒ स्तत् । \newline
47. भ॒व॒त॒ स्तत् तद् भ॑वतो भवत॒ स्तद्धि हि तद् भ॑वतो भवत॒ स्तद्धि । \newline
48. तद्धि हि तत् तद्धि सु॑व॒र्ग्यꣳ॑ सुव॒र्ग्यꣳ॑ हि तत् तद्धि सु॑व॒र्ग्य᳚म् । \newline
49. हि सु॑व॒र्ग्यꣳ॑ सुव॒र्ग्यꣳ॑ हि हि सु॑व॒र्ग्य॑म् च॒क्रीव॑ती च॒क्रीव॑ती सुव॒र्ग्यꣳ॑ हि हि सु॑व॒र्ग्य॑म् च॒क्रीव॑ती । \newline
50. सु॒व॒र्ग्य॑म् च॒क्रीव॑ती च॒क्रीव॑ती सुव॒र्ग्यꣳ॑ सुव॒र्ग्य॑म् च॒क्रीव॑ती भवतो भवत श्च॒क्रीव॑ती सुव॒र्ग्यꣳ॑ सुव॒र्ग्य॑म् च॒क्रीव॑ती भवतः । \newline
51. सु॒व॒र्ग्य॑मिति॑ सुवः - ग्य᳚म् । \newline
52. च॒क्रीव॑ती भवतो भवत श्च॒क्रीव॑ती च॒क्रीव॑ती भवतः सुव॒र्गस्य॑ सुव॒र्गस्य॑ भवत श्च॒क्रीव॑ती च॒क्रीव॑ती भवतः सुव॒र्गस्य॑ । \newline
53. च॒क्रीव॑ती॒ इति॑ च॒क्रीव॑ती । \newline
54. भ॒व॒तः॒ सु॒व॒र्गस्य॑ सुव॒र्गस्य॑ भवतो भवतः सुव॒र्गस्य॑ लो॒कस्य॑ लो॒कस्य॑ सुव॒र्गस्य॑ भवतो भवतः सुव॒र्गस्य॑ लो॒कस्य॑ । \newline
55. सु॒व॒र्गस्य॑ लो॒कस्य॑ लो॒कस्य॑ सुव॒र्गस्य॑ सुव॒र्गस्य॑ लो॒कस्य॒ सम॑ष्ट्यै॒ सम॑ष्ट्यै लो॒कस्य॑ सुव॒र्गस्य॑ सुव॒र्गस्य॑ लो॒कस्य॒ सम॑ष्ट्यै । \newline
56. सु॒व॒र्गस्येति॑ सुवः - गस्य॑ । \newline
57. लो॒कस्य॒ सम॑ष्ट्यै॒ सम॑ष्ट्यै लो॒कस्य॑ लो॒कस्य॒ सम॑ष्ट्या उ॒लूख॑लबुद्ध्न उ॒लूख॑लबुद्ध्नः॒ सम॑ष्ट्यै लो॒कस्य॑ लो॒कस्य॒ सम॑ष्ट्या उ॒लूख॑लबुद्ध्नः । \newline
58. सम॑ष्ट्या उ॒लूख॑लबुद्ध्न उ॒लूख॑लबुद्ध्नः॒ सम॑ष्ट्यै॒ सम॑ष्ट्या उ॒लूख॑लबुद्ध्नो॒ यूपो॒ यूप॑ उ॒लूख॑लबुद्ध्नः॒ सम॑ष्ट्यै॒ सम॑ष्ट्या उ॒लूख॑लबुद्ध्नो॒ यूपः॑ । \newline
59. सम॑ष्ट्या॒ इति॒ सं - अ॒ष्ट्यै॒ । \newline
60. उ॒लूख॑लबुद्ध्नो॒ यूपो॒ यूप॑ उ॒लूख॑लबुद्ध्न उ॒लूख॑लबुद्ध्नो॒ यूपो॑ भवति भवति॒ यूप॑ उ॒लूख॑लबुद्ध्न उ॒लूख॑लबुद्ध्नो॒ यूपो॑ भवति । \newline
61. उ॒लूख॑लबुद्ध्न॒ इत्यु॒लूख॑ल - बु॒द्ध्नः॒ । \newline
62. यूपो॑ भवति भवति॒ यूपो॒ यूपो॑ भवति॒ प्रति॑ष्ठित्यै॒ प्रति॑ष्ठित्यै भवति॒ यूपो॒ यूपो॑ भवति॒ प्रति॑ष्ठित्यै । \newline
63. भ॒व॒ति॒ प्रति॑ष्ठित्यै॒ प्रति॑ष्ठित्यै भवति भवति॒ प्रति॑ष्ठित्यै॒ प्राञ्चः॒ प्राञ्चः॒ प्रति॑ष्ठित्यै भवति भवति॒ प्रति॑ष्ठित्यै॒ प्राञ्चः॑ । \newline
64. प्रति॑ष्ठित्यै॒ प्राञ्चः॒ प्राञ्चः॒ प्रति॑ष्ठित्यै॒ प्रति॑ष्ठित्यै॒ प्राञ्चो॑ यान्ति यान्ति॒ प्राञ्चः॒ प्रति॑ष्ठित्यै॒ प्रति॑ष्ठित्यै॒ प्राञ्चो॑ यान्ति । \newline
65. प्रति॑ष्ठित्या॒ इति॒ प्रति॑ - स्थि॒त्यै॒ । \newline
66. प्राञ्चो॑ यान्ति यान्ति॒ प्राञ्चः॒ प्राञ्चो॑ यान्ति॒ प्राङ् प्राङ् या᳚न्ति॒ प्राञ्चः॒ प्राञ्चो॑ यान्ति॒ प्राङ् । \newline
67. या॒न्ति॒ प्राङ् प्राङ् या᳚न्ति यान्ति॒ प्राङ् ङि॑वेव॒ प्राङ् या᳚न्ति यान्ति॒ प्राङ् ङि॑व । \newline
68. प्राङ् ङि॑वेव॒ प्राङ् प्राङ् ङि॑व॒ हि हीव॒ प्राङ् प्राङ् ङि॑व॒ हि । \newline
69. इ॒व॒ हि हीवे॑व॒ हि सु॑व॒र्गः सु॑व॒र्गो हीवे॑व॒ हि सु॑व॒र्गः । \newline
70. हि सु॑व॒र्गः सु॑व॒र्गो हि हि सु॑व॒र्गो लो॒को लो॒कः सु॑व॒र्गो हि हि सु॑व॒र्गो लो॒कः । \newline
71. सु॒व॒र्गो लो॒को लो॒कः सु॑व॒र्गः सु॑व॒र्गो लो॒कः सर॑स्वत्या॒ सर॑स्वत्या लो॒कः सु॑व॒र्गः सु॑व॒र्गो लो॒कः सर॑स्वत्या । \newline
72. सु॒व॒र्ग इति॑ सुवः - गः । \newline
\pagebreak
\markright{ TS 7.2.1.4  \hfill https://www.vedavms.in \hfill}

\section{ TS 7.2.1.4 }

\textbf{TS 7.2.1.4 } \newline
\textbf{Samhita Paata} \newline

लो॒कः सर॑स्वत्या यान्त्ये॒ष वै दे॑व॒यानः॒ पन्था॒स्त-मे॒वा-न्वारो॑हन्त्या॒क्रोश॑न्तो या॒न्त्यव॑र्ति-मे॒वान्यस्मि॑न् प्रति॒षज्य॑ प्रति॒ष्ठां ग॑च्छन्ति य॒दा दश॑ श॒तं कु॒र्वन्त्यथैक॑-मु॒त्थानꣳ॑ श॒तायुः॒ पुरु॑षः श॒तेन्द्रि॑य॒ आयु॑ष्ये॒वेन्द्रि॒ये प्रति॑ तिष्ठन्ति य॒दा श॒तꣳ स॒हस्रं॑ कु॒र्वन्त्यथैक॑-मु॒त्थानꣳ॑ स॒हस्र॑संमितो॒ वा अ॒सौ लो॒को॑ऽमुमे॒व लो॒कम॒भि ज॑यन्ति य॒दै ( ) -षां᳚ प्र॒मीये॑त य॒दा वा॒ जीये॑र॒न्नथैक॑-मु॒त्थानं॒ तद्धि ती॒र्थं ॥ \newline

\textbf{Pada Paata} \newline

लो॒कः । सर॑स्वत्या । या॒न्ति॒ । ए॒षः । वै । दे॒व॒यान॒ इति॑ देव-यानः॑ । पन्थाः᳚ । तम् । ए॒व । अ॒न्वारो॑ह॒न्तीत्य॑नु - आरो॑हन्ति । आ॒क्रोश॑न्त॒ इत्या᳚ - क्रोश॑न्तः । या॒न्ति॒ । अव॑र्तिम् । ए॒व । अ॒न्यस्मिन्न्॑ । प्र॒ति॒षज्येति॑ प्रति - सज्य॑ । प्र॒ति॒ष्ठामिति॑ प्रति-स्थाम् । ग॒च्छ॒न्ति॒ । य॒दा । दश॑ । श॒तम् । कु॒र्वन्ति॑ । अथ॑ । एक᳚म् । उ॒त्थान॒मित्यु॑त्-स्थान᳚म् । श॒तायु॒रिति॑ श॒त - आ॒युः॒ । पुरु॑षः । श॒तेन्द्रि॑य॒ इति॑ श॒त-इ॒न्द्रि॒यः॒ । आयु॑षि । ए॒व । इ॒न्द्रि॒ये । प्रतीति॑ । ति॒ष्ठ॒न्ति॒ । य॒दा । श॒तम् । स॒हस्र᳚म् । कु॒र्वन्ति॑ । अथ॑ । एक᳚म् । उ॒त्थान॒मित्यु॑त्- स्थान᳚म् । स॒हस्र॑संमित॒ इति॑ स॒हस्र॑-स॒म्मि॒तः॒ । वै । अ॒सौ । लो॒कः । अ॒मुम् । ए॒व । लो॒कम् । अ॒भीति॑ । ज॒य॒न्ति॒ । य॒दा ( ) । ए॒षा॒म् । प्र॒मीये॒तेति॑ प्र - मीये॑त । य॒दा । वा॒ । जीये॑रन्न् । अथ॑ । एक᳚म् । उ॒त्थान॒मित्यु॑त् - स्थान᳚म् । तत् । हि । ती॒र्थम् ॥  \newline


\textbf{Krama Paata} \newline

लो॒कः सर॑स्वत्या । सर॑स्वत्या यान्ति । या॒न्त्ये॒षः । ए॒ष वै । वै दे॑व॒यानः॑ । दे॒व॒यानः॒ पन्थाः᳚ । दे॒व॒यान॒ इति॑ देव - यानः॑ । पन्था॒स्तम् । तमे॒व । ए॒वान्वारो॑हन्ति । अ॒न्वारो॑हन्त्या॒क्रोश॑न्तः । अ॒न्वारो॑ह॒न्तीत्य॑नु - आरो॑हन्ति । आ॒क्रोश॑न्तो यान्ति । आ॒क्रोश॑न्त॒ इत्या᳚ - क्रोश॑न्तः । या॒न्त्यव॑र्तिम् । अव॑र्तिमे॒व । ए॒वान्यस्मिन्न्॑ । अ॒न्यस्मि॑न् प्रति॒षज्य॑ । प्र॒ति॒षज्य॑ प्रति॒ष्ठाम् । प्र॒ति॒षज्येति॑ प्रति - सज्य॑ । प्र॒ति॒ष्ठाम् ग॑च्छन्ति । प्र॒ति॒ष्ठामिति॑ प्रति - स्थाम् । ग॒च्छ॒न्ति॒ य॒दा । य॒दा दश॑ । दश॑ श॒तम् । श॒तम् कु॒र्वन्ति॑ । कु॒र्वन्त्यथ॑ । अथैक᳚म् । एक॑मु॒त्थान᳚म् । उ॒त्थानꣳ॑ श॒तायुः॑ । उ॒त्थान॒मित्यु॑त् - स्थान᳚म् । श॒तायुः॒ पुरु॑षः । श॒तायु॒रिति॑ श॒त - आ॒युः॒ । पुरु॑षः श॒तेन्द्रि॑यः । श॒तेन्द्रि॑य॒ आयु॑षि । श॒तेन्द्रि॑य॒ इति॑ श॒त - इ॒न्द्रि॒यः॒ । आयु॑ष्ये॒व । ए॒वेन्द्रि॒ये । इ॒न्द्रि॒ये प्रति॑ । प्रति॑ तिष्ठन्ति । ति॒ष्ठ॒न्ति॒ य॒दा । य॒दा श॒तम् । श॒तꣳ स॒हस्र᳚म् । स॒हस्र॑म् कु॒र्वन्ति॑ । कु॒र्वन्त्यथ॑ । अथैक᳚म् । एक॑मु॒त्थान᳚म् । उ॒त्थानꣳ॑ स॒हस्र॑सम्मितः । उ॒त्थान॒मित्यु॑त् - स्थान᳚म् । स॒हस्र॑सम्मितो॒ वै । स॒हस्र॑सम्मित॒ इति॑ स॒हस्र॑ - स॒म्मि॒तः॒ । वा अ॒सौ । अ॒सौ लो॒कः । लो॒को॑ऽमुम् । अ॒मुमे॒व । ए॒व लो॒कम् । लो॒कम॒भि । अ॒भि ज॑यन्ति । ज॒य॒न्ति॒ य॒दा ( ) । य॒दैषा᳚म् । ए॒षा॒म् प्र॒मीये॑त । प्र॒मीये॑त य॒दा । प्र॒मीये॒तेति॑ प्र - मीये॑त । य॒दा वा᳚ । वा॒ जीये॑रन्न् । जीये॑र॒न्नथ॑ । अथैक᳚म् । एक॑मु॒त्थान᳚म् । उ॒त्थान॒म् तत् । उ॒त्थान॒मित्यु॑त् - स्थान᳚म् । तद्‌धि । हि ती॒र्थम् । ती॒र्त्थमिति॑ ती॒र्त्थम् । \newline

\textbf{Jatai Paata} \newline

1. लो॒कः सर॑स्वत्या॒ सर॑स्वत्या लो॒को लो॒कः सर॑स्वत्या । \newline
2. सर॑स्वत्या यान्ति यान्ति॒ सर॑स्वत्या॒ सर॑स्वत्या यान्ति । \newline
3. या॒न्त्ये॒ष ए॒ष या᳚न्ति यान्त्ये॒षः । \newline
4. ए॒ष वै वा ए॒ष ए॒ष वै । \newline
5. वै दे॑व॒यानो॑ देव॒यानो॒ वै वै दे॑व॒यानः॑ । \newline
6. दे॒व॒यानः॒ पन्थाः॒ पन्था॑ देव॒यानो॑ देव॒यानः॒ पन्थाः᳚ । \newline
7. दे॒व॒यान॒ इति॑ देव - यानः॑ । \newline
8. पन्था॒ स्तम् तम् पन्थाः॒ पन्था॒ स्तम् । \newline
9. त मे॒वैव तम् त मे॒व । \newline
10. ए॒वा न्वारो॑ह न्त्य॒न्वारो॑ह न्त्ये॒वैवा न्वारो॑हन्ति । \newline
11. अ॒न्वारो॑ह न्त्या॒क्रोश॑न्त आ॒क्रोश॑न्तो॒ ऽन्वारो॑ह न्त्य॒न्वारो॑ह न्त्या॒क्रोश॑न्तः । \newline
12. अ॒न्वारो॑ह॒न्तीत्य॑नु - आरो॑हन्ति । \newline
13. आ॒क्रोश॑न्तो यान्ति यान्त्या॒क्रोश॑न्त आ॒क्रोश॑न्तो यान्ति । \newline
14. आ॒क्रोश॑न्त॒ इत्या᳚ - क्रोश॑न्तः । \newline
15. या॒न्त्यव॑र्ति॒ मव॑र्तिं ॅयान्ति या॒न्त्यव॑र्तिम् । \newline
16. अव॑र्ति मे॒वैवाव॑र्ति॒ मव॑र्ति मे॒व । \newline
17. ए॒वा न्यस्मि॑न् न॒न्यस्मि॑न् ने॒वैवा न्यस्मिन्न्॑ । \newline
18. अ॒न्यस्मि॑न् प्रति॒षज्य॑ प्रति॒षज्या॒ न्यस्मि॑न् न॒न्यस्मि॑न् प्रति॒षज्य॑ । \newline
19. प्र॒ति॒षज्य॑ प्रति॒ष्ठाम् प्र॑ति॒ष्ठाम् प्र॑ति॒षज्य॑ प्रति॒षज्य॑ प्रति॒ष्ठाम् । \newline
20. प्र॒ति॒षज्येति॑ प्रति - सज्य॑ । \newline
21. प्र॒ति॒ष्ठाम् ग॑च्छन्ति गच्छन्ति प्रति॒ष्ठाम् प्र॑ति॒ष्ठाम् ग॑च्छन्ति । \newline
22. प्र॒ति॒ष्ठामिति॑ प्रति - स्थाम् । \newline
23. ग॒च्छ॒न्ति॒ य॒दा य॒दा ग॑च्छन्ति गच्छन्ति य॒दा । \newline
24. य॒दा दश॒ दश॑ य॒दा य॒दा दश॑ । \newline
25. दश॑ श॒तꣳ श॒तम् दश॒ दश॑ श॒तम् । \newline
26. श॒तम् कु॒र्वन्ति॑ कु॒र्वन्ति॑ श॒तꣳ श॒तम् कु॒र्वन्ति॑ । \newline
27. कु॒र्व न्त्यथाथ॑ कु॒र्वन्ति॑ कु॒र्व न्त्यथ॑ । \newline
28. अथैक॒ मेक॒ मथाथैक᳚म् । \newline
29. एक॑ मु॒त्थान॑ मु॒त्थान॒ मेक॒ मेक॑ मु॒त्थान᳚म् । \newline
30. उ॒त्थानꣳ॑ श॒तायुः॑ श॒तायु॑ रु॒त्थान॑ मु॒त्थानꣳ॑ श॒तायुः॑ । \newline
31. उ॒त्थान॒मित्यु॑त् - स्थान᳚म् । \newline
32. श॒तायुः॒ पुरु॑षः॒ पुरु॑षः श॒तायुः॑ श॒तायुः॒ पुरु॑षः । \newline
33. श॒तायु॒रिति॑ श॒त - आ॒युः॒ । \newline
34. पुरु॑षः श॒तेन्द्रि॑यः श॒तेन्द्रि॑यः॒ पुरु॑षः॒ पुरु॑षः श॒तेन्द्रि॑यः । \newline
35. श॒तेन्द्रि॑य॒ आयु॒ ष्यायु॑षि श॒तेन्द्रि॑यः श॒तेन्द्रि॑य॒ आयु॑षि । \newline
36. श॒तेन्द्रि॑य॒ इति॑ श॒त - इ॒न्द्रि॒यः॒ । \newline
37. आयु॑ ष्ये॒वैवायु॒ ष्यायु॑ ष्ये॒व । \newline
38. ए॒वेन्द्रि॒य इ॑न्द्रि॒य ए॒वैवेन्द्रि॒ये । \newline
39. इ॒न्द्रि॒ये प्रति॒ प्रती᳚न्द्रि॒य इ॑न्द्रि॒ये प्रति॑ । \newline
40. प्रति॑ तिष्ठन्ति तिष्ठन्ति॒ प्रति॒ प्रति॑ तिष्ठन्ति । \newline
41. ति॒ष्ठ॒न्ति॒ य॒दा य॒दा ति॑ष्ठन्ति तिष्ठन्ति य॒दा । \newline
42. य॒दा श॒तꣳ श॒तं ॅय॒दा य॒दा श॒तम् । \newline
43. श॒तꣳ स॒हस्रꣳ॑ स॒हस्रꣳ॑ श॒तꣳ श॒तꣳ स॒हस्र᳚म् । \newline
44. स॒हस्र॑म् कु॒र्वन्ति॑ कु॒र्वन्ति॑ स॒हस्रꣳ॑ स॒हस्र॑म् कु॒र्वन्ति॑ । \newline
45. कु॒र्व न्त्यथाथ॑ कु॒र्वन्ति॑ कु॒र्व न्त्यथ॑ । \newline
46. अथैक॒ मेक॒ मथाथैक᳚म् । \newline
47. एक॑ मु॒त्थान॑ मु॒त्थान॒ मेक॒ मेक॑ मु॒त्थान᳚म् । \newline
48. उ॒त्थानꣳ॑ स॒हस्र॑सम्मितः स॒हस्र॑सम्मित उ॒त्थान॑ मु॒त्थानꣳ॑ स॒हस्र॑सम्मितः । \newline
49. उ॒त्थान॒मित्यु॑त् - स्थान᳚म् । \newline
50. स॒हस्र॑सम्मितो॒ वै वै स॒हस्र॑सम्मितः स॒हस्र॑सम्मितो॒ वै । \newline
51. स॒हस्र॑सम्मित॒ इति॑ स॒हस्र॑ - स॒म्मि॒तः॒ । \newline
52. वा अ॒सा व॒सौ वै वा अ॒सौ । \newline
53. अ॒सौ लो॒को लो॒को॑ ऽसा व॒सौ लो॒कः । \newline
54. लो॒को॑ ऽमु म॒मुम् ॅलो॒को लो॒को॑ ऽमुम् । \newline
55. अ॒मु मे॒वै वामु म॒मु मे॒व । \newline
56. ए॒व लो॒कम् ॅलो॒क मे॒वैव लो॒कम् । \newline
57. लो॒क म॒भ्य॑भि लो॒कम् ॅलो॒क म॒भि । \newline
58. अ॒भि ज॑यन्ति जय न्त्य॒भ्य॑भि ज॑यन्ति । \newline
59. ज॒य॒न्ति॒ य॒दा य॒दा ज॑यन्ति जयन्ति य॒दा । \newline
60. य॒दैषा॑ मेषां ॅय॒दा य॒दैषा᳚म् । \newline
61. ए॒षा॒म् प्र॒मीये॑त प्र॒मीये॑ तैषा मेषाम् प्र॒मीये॑त । \newline
62. प्र॒मीये॑त य॒दा य॒दा प्र॒मीये॑त प्र॒मीये॑त य॒दा । \newline
63. प्र॒मीये॒तेति॑ प्र - मीये॑त । \newline
64. य॒दा वा॑ वा य॒दा य॒दा वा᳚ । \newline
65. वा॒ जीये॑र॒न् जीये॑रन्. वा वा॒ जीये॑रन्न् । \newline
66. जीये॑र॒न् नथाथ॒ जीये॑र॒न् जीये॑र॒न् नथ॑ । \newline
67. अथैक॒ मेक॒ मथा थैक᳚म् । \newline
68. एक॑ मु॒त्थान॑ मु॒त्थान॒ मेक॒ मेक॑ मु॒त्थान᳚म् । \newline
69. उ॒त्थान॒म् तत् तदु॒त्थान॑ मु॒त्थान॒म् तत् । \newline
70. उ॒त्थान॒मित्यु॑त् - स्थान᳚म् । \newline
71. तद्धि हि तत् तद्धि । \newline
72. हि ती॒र्थम् ती॒र्थꣳ हि हि ती॒र्थम् । \newline
73. ती॒र्त्थमिति॑ ती॒र्त्थम् । \newline

\textbf{Ghana Paata } \newline

1. लो॒कः सर॑स्वत्या॒ सर॑स्वत्या लो॒को लो॒कः सर॑स्वत्या यान्ति यान्ति॒ सर॑स्वत्या लो॒को लो॒कः सर॑स्वत्या यान्ति । \newline
2. सर॑स्वत्या यान्ति यान्ति॒ सर॑स्वत्या॒ सर॑स्वत्या यान्त्ये॒ष ए॒ष या᳚न्ति॒ सर॑स्वत्या॒ सर॑स्वत्या यान्त्ये॒षः । \newline
3. या॒न्त्ये॒ष ए॒ष या᳚न्ति यान्त्ये॒ष वै वा ए॒ष या᳚न्ति यान्त्ये॒ष वै । \newline
4. ए॒ष वै वा ए॒ष ए॒ष वै दे॑व॒यानो॑ देव॒यानो॒ वा ए॒ष ए॒ष वै दे॑व॒यानः॑ । \newline
5. वै दे॑व॒यानो॑ देव॒यानो॒ वै वै दे॑व॒यानः॒ पन्थाः॒ पन्था॑ देव॒यानो॒ वै वै दे॑व॒यानः॒ पन्थाः᳚ । \newline
6. दे॒व॒यानः॒ पन्थाः॒ पन्था॑ देव॒यानो॑ देव॒यानः॒ पन्था॒ स्तम् तम् पन्था॑ देव॒यानो॑ देव॒यानः॒ पन्था॒ स्तम् । \newline
7. दे॒व॒यान॒ इति॑ देव - यानः॑ । \newline
8. पन्था॒ स्तम् तम् पन्थाः॒ पन्था॒ स्त मे॒वैव तम् पन्थाः॒ पन्था॒ स्त मे॒व । \newline
9. त मे॒वैव तम् त मे॒वान्वारो॑ह न्त्य॒न्वारो॑ह न्त्ये॒व तम् त मे॒वा न्वारो॑हन्ति । \newline
10. ए॒वा न्वारो॑ह न्त्य॒न्वारो॑ह न्त्ये॒वैवान्वारो॑ह न्त्या॒क्रोश॑न्त आ॒क्रोश॑न्तो॒ ऽन्वारो॑ह न्त्ये॒वैवा न्वारो॑ह न्त्या॒क्रोश॑न्तः । \newline
11. अ॒न्वारो॑ह न्त्या॒क्रोश॑न्त आ॒क्रोश॑न्तो॒ ऽन्वारो॑ह न्त्य॒न्वारो॑ह न्त्या॒क्रोश॑न्तो यान्ति यान्त्या॒क्रोश॑न्तो॒ ऽन्वारो॑ह न्त्य॒न्वारो॑ह न्त्या॒क्रोश॑न्तो यान्ति । \newline
12. अ॒न्वारो॑ह॒न्तीत्य॑नु - आरो॑हन्ति । \newline
13. आ॒क्रोश॑न्तो यान्ति यान्त्या॒क्रोश॑न्त आ॒क्रोश॑न्तो या॒न्त्यव॑र्ति॒ मव॑र्तिं ॅयान्त्या॒क्रोश॑न्त आ॒क्रोश॑न्तो या॒न्त्यव॑र्तिम् । \newline
14. आ॒क्रोश॑न्त॒ इत्या᳚ - क्रोश॑न्तः । \newline
15. या॒न्त्यव॑र्ति॒ मव॑र्तिं ॅयान्ति या॒न्त्यव॑र्ति मे॒वै वाव॑र्तिं ॅयान्ति या॒न्त्यव॑र्ति मे॒व । \newline
16. अव॑र्ति मे॒वै वाव॑र्ति॒ मव॑र्ति मे॒वा न्यस्मि॑न् न॒न्यस्मि॑न् ने॒वा व॑र्ति॒ मव॑र्ति मे॒वा न्यस्मिन्न्॑ । \newline
17. ए॒वा न्यस्मि॑न् न॒न्यस्मि॑न् ने॒वैवा न्यस्मि॑न् प्रति॒षज्य॑ प्रति॒षज्या॒ न्यस्मि॑न् ने॒वैवा न्यस्मि॑न् प्रति॒षज्य॑ । \newline
18. अ॒न्यस्मि॑न् प्रति॒षज्य॑ प्रति॒षज्या॒ न्यस्मि॑न् न॒न्यस्मि॑न् प्रति॒षज्य॑ प्रति॒ष्ठाम् प्र॑ति॒ष्ठाम् प्र॑ति॒षज्या॒ न्यस्मि॑न् न॒न्यस्मि॑न् प्रति॒षज्य॑ प्रति॒ष्ठाम् । \newline
19. प्र॒ति॒षज्य॑ प्रति॒ष्ठाम् प्र॑ति॒ष्ठाम् प्र॑ति॒षज्य॑ प्रति॒षज्य॑ प्रति॒ष्ठाम् ग॑च्छन्ति गच्छन्ति प्रति॒ष्ठाम् प्र॑ति॒षज्य॑ प्रति॒षज्य॑ प्रति॒ष्ठाम् ग॑च्छन्ति । \newline
20. प्र॒ति॒षज्येति॑ प्रति - सज्य॑ । \newline
21. प्र॒ति॒ष्ठाम् ग॑च्छन्ति गच्छन्ति प्रति॒ष्ठाम् प्र॑ति॒ष्ठाम् ग॑च्छन्ति य॒दा य॒दा ग॑च्छन्ति प्रति॒ष्ठाम् प्र॑ति॒ष्ठाम् ग॑च्छन्ति य॒दा । \newline
22. प्र॒ति॒ष्ठामिति॑ प्रति - स्थाम् । \newline
23. ग॒च्छ॒न्ति॒ य॒दा य॒दा ग॑च्छन्ति गच्छन्ति य॒दा दश॒ दश॑ य॒दा ग॑च्छन्ति गच्छन्ति य॒दा दश॑ । \newline
24. य॒दा दश॒ दश॑ य॒दा य॒दा दश॑ श॒तꣳ श॒तम् दश॑ य॒दा य॒दा दश॑ श॒तम् । \newline
25. दश॑ श॒तꣳ श॒तम् दश॒ दश॑ श॒तम् कु॒र्वन्ति॑ कु॒र्वन्ति॑ श॒तम् दश॒ दश॑ श॒तम् कु॒र्वन्ति॑ । \newline
26. श॒तम् कु॒र्वन्ति॑ कु॒र्वन्ति॑ श॒तꣳ श॒तम् कु॒र्व न्त्यथाथ॑ कु॒र्वन्ति॑ श॒तꣳ श॒तम् कु॒र्व न्त्यथ॑ । \newline
27. कु॒र्व न्त्यथाथ॑ कु॒र्वन्ति॑ कु॒र्व न्त्यथैक॒ मेक॒ मथ॑ कु॒र्वन्ति॑ कु॒र्व न्त्यथैक᳚म् । \newline
28. अथैक॒ मेक॒ मथा थैक॑ मु॒त्थान॑ मु॒त्थान॒ मेक॒ मथा थैक॑ मु॒त्थान᳚म् । \newline
29. एक॑ मु॒त्थान॑ मु॒त्थान॒ मेक॒ मेक॑ मु॒त्थानꣳ॑ श॒तायुः॑ श॒तायु॑ रु॒त्थान॒ मेक॒ मेक॑ मु॒त्थानꣳ॑ श॒तायुः॑ । \newline
30. उ॒त्थानꣳ॑ श॒तायुः॑ श॒तायु॑ रु॒त्थान॑ मु॒त्थानꣳ॑ श॒तायुः॒ पुरु॑षः॒ पुरु॑षः श॒तायु॑ रु॒त्थान॑ मु॒त्थानꣳ॑ श॒तायुः॒ पुरु॑षः । \newline
31. उ॒त्थान॒मित्यु॑त् - स्थान᳚म् । \newline
32. श॒तायुः॒ पुरु॑षः॒ पुरु॑षः श॒तायुः॑ श॒तायुः॒ पुरु॑षः श॒तेन्द्रि॑यः श॒तेन्द्रि॑यः॒ पुरु॑षः श॒तायुः॑ श॒तायुः॒ पुरु॑षः श॒तेन्द्रि॑यः । \newline
33. श॒तायु॒रिति॑ श॒त - आ॒युः॒ । \newline
34. पुरु॑षः श॒तेन्द्रि॑यः श॒तेन्द्रि॑यः॒ पुरु॑षः॒ पुरु॑षः श॒तेन्द्रि॑य॒ आयु॒ ष्यायु॑षि श॒तेन्द्रि॑यः॒ पुरु॑षः॒ पुरु॑षः श॒तेन्द्रि॑य॒ आयु॑षि । \newline
35. श॒तेन्द्रि॑य॒ आयु॒ ष्यायु॑षि श॒तेन्द्रि॑यः श॒तेन्द्रि॑य॒ आयु॑ ष्ये॒वै वायु॑षि श॒तेन्द्रि॑यः श॒तेन्द्रि॑य॒ आयु॑ष्ये॒व । \newline
36. श॒तेन्द्रि॑य॒ इति॑ श॒त - इ॒न्द्रि॒यः॒ । \newline
37. आयु॑ ष्ये॒वैवायु॒ ष्यायु॑ ष्ये॒वेन्द्रि॒य इ॑न्द्रि॒य ए॒वायु॒ ष्यायु॑ ष्ये॒वेन्द्रि॒ये । \newline
38. ए॒वेन्द्रि॒य इ॑न्द्रि॒य ए॒वैवेन्द्रि॒ये प्रति॒ प्रती᳚ न्द्रि॒य ए॒वैवेन्द्रि॒ये प्रति॑ । \newline
39. इ॒न्द्रि॒ये प्रति॒ प्रती᳚ न्द्रि॒य इ॑न्द्रि॒ये प्रति॑ तिष्ठन्ति तिष्ठन्ति॒ प्रती᳚ न्द्रि॒य इ॑न्द्रि॒ये प्रति॑ तिष्ठन्ति । \newline
40. प्रति॑ तिष्ठन्ति तिष्ठन्ति॒ प्रति॒ प्रति॑ तिष्ठन्ति य॒दा य॒दा ति॑ष्ठन्ति॒ प्रति॒ प्रति॑ तिष्ठन्ति य॒दा । \newline
41. ति॒ष्ठ॒न्ति॒ य॒दा य॒दा ति॑ष्ठन्ति तिष्ठन्ति य॒दा श॒तꣳ श॒तं ॅय॒दा ति॑ष्ठन्ति तिष्ठन्ति य॒दा श॒तम् । \newline
42. य॒दा श॒तꣳ श॒तं ॅय॒दा य॒दा श॒तꣳ स॒हस्रꣳ॑ स॒हस्रꣳ॑ श॒तं ॅय॒दा य॒दा श॒तꣳ स॒हस्र᳚म् । \newline
43. श॒तꣳ स॒हस्रꣳ॑ स॒हस्रꣳ॑ श॒तꣳ श॒तꣳ स॒हस्र॑म् कु॒र्वन्ति॑ कु॒र्वन्ति॑ स॒हस्रꣳ॑ श॒तꣳ श॒तꣳ स॒हस्र॑म् कु॒र्वन्ति॑ । \newline
44. स॒हस्र॑म् कु॒र्वन्ति॑ कु॒र्वन्ति॑ स॒हस्रꣳ॑ स॒हस्र॑म् कु॒र्व न्त्यथाथ॑ कु॒र्वन्ति॑ स॒हस्रꣳ॑ स॒हस्र॑म् कु॒र्व न्त्यथ॑ । \newline
45. कु॒र्व न्त्यथाथ॑ कु॒र्वन्ति॑ कु॒र्व न्त्यथैक॒ मेक॒ मथ॑ कु॒र्वन्ति॑ कु॒र्व न्त्यथैक᳚म् । \newline
46. अथैक॒ मेक॒ मथा थैक॑ मु॒त्थान॑ मु॒त्थान॒ मेक॒ मथा थैक॑ मु॒त्थान᳚म् । \newline
47. एक॑ मु॒त्थान॑ मु॒त्थान॒ मेक॒ मेक॑ मु॒त्थानꣳ॑ स॒हस्र॑सम्मितः स॒हस्र॑सम्मित उ॒त्थान॒ मेक॒ मेक॑ मु॒त्थानꣳ॑ स॒हस्र॑सम्मितः । \newline
48. उ॒त्थानꣳ॑ स॒हस्र॑सम्मितः स॒हस्र॑सम्मित उ॒त्थान॑ मु॒त्थानꣳ॑ स॒हस्र॑सम्मितो॒ वै वै स॒हस्र॑सम्मित उ॒त्थान॑ मु॒त्थानꣳ॑ स॒हस्र॑सम्मितो॒ वै । \newline
49. उ॒त्थान॒मित्यु॑त् - स्थान᳚म् । \newline
50. स॒हस्र॑सम्मितो॒ वै वै स॒हस्र॑सम्मितः स॒हस्र॑सम्मितो॒ वा अ॒सा व॒सौ वै स॒हस्र॑सम्मितः स॒हस्र॑सम्मितो॒ वा अ॒सौ । \newline
51. स॒हस्र॑सम्मित॒ इति॑ स॒हस्र॑ - स॒म्मि॒तः॒ । \newline
52. वा अ॒सा व॒सौ वै वा अ॒सौ लो॒को लो॒को॑ ऽसौ वै वा अ॒सौ लो॒कः । \newline
53. अ॒सौ लो॒को लो॒को॑ ऽसा व॒सौ लो॒को॑ ऽमु म॒मुम् ॅलो॒को॑ ऽसा व॒सौ लो॒को॑ ऽमुम् । \newline
54. लो॒को॑ ऽमु म॒मुम् ॅलो॒को लो॒को॑ ऽमु मे॒वै वामुम् ॅलो॒को लो॒को॑ ऽमु मे॒व । \newline
55. अ॒मु मे॒वै वामु म॒मु मे॒व लो॒कम् ॅलो॒क मे॒वामु म॒मु मे॒व लो॒कम् । \newline
56. ए॒व लो॒कम् ॅलो॒क मे॒वैव लो॒क म॒भ्य॑भि लो॒क मे॒वैव लो॒क म॒भि । \newline
57. लो॒क म॒भ्य॑भि लो॒कम् ॅलो॒क म॒भि ज॑यन्ति जयन्त्य॒भि लो॒कम् ॅलो॒क म॒भि ज॑यन्ति । \newline
58. अ॒भि ज॑यन्ति जय न्त्य॒भ्य॑भि ज॑यन्ति य॒दा य॒दा ज॑य न्त्य॒भ्य॑भि ज॑यन्ति य॒दा । \newline
59. ज॒य॒न्ति॒ य॒दा य॒दा ज॑यन्ति जयन्ति य॒दैषा॑ मेषां ॅय॒दा ज॑यन्ति जयन्ति य॒दैषा᳚म् । \newline
60. य॒दैषा॑ मेषां ॅय॒दा य॒दैषा᳚म् प्र॒मीये॑त प्र॒मीये॑तैषां ॅय॒दा य॒दैषा᳚म् प्र॒मीये॑त । \newline
61. ए॒षा॒म् प्र॒मीये॑त प्र॒मीये॑तैषा मेषाम् प्र॒मीये॑त य॒दा य॒दा प्र॒मीये॑तैषा मेषाम् प्र॒मीये॑त य॒दा । \newline
62. प्र॒मीये॑त य॒दा य॒दा प्र॒मीये॑त प्र॒मीये॑त य॒दा वा॑ वा य॒दा प्र॒मीये॑त प्र॒मीये॑त य॒दा वा᳚ । \newline
63. प्र॒मीये॒तेति॑ प्र - मीये॑त । \newline
64. य॒दा वा॑ वा य॒दा य॒दा वा॒ जीये॑र॒न् जीये॑रन्. वा य॒दा य॒दा वा॒ जीये॑रन्न् । \newline
65. वा॒ जीये॑र॒न् जीये॑रन्. वा वा॒ जीये॑र॒न् नथाथ॒ जीये॑रन्. वा वा॒ जीये॑र॒न् नथ॑ । \newline
66. जीये॑र॒न् नथाथ॒ जीये॑र॒न् जीये॑र॒न् नथैक॒ मेक॒ मथ॒ जीये॑र॒न् जीये॑र॒न् नथैक᳚म् । \newline
67. अथैक॒ मेक॒ मथाथैक॑ मु॒त्थान॑ मु॒त्थान॒ मेक॒ मथाथैक॑ मु॒त्थान᳚म् । \newline
68. एक॑ मु॒त्थान॑ मु॒त्थान॒ मेक॒ मेक॑ मु॒त्थान॒म् तत् तदु॒त्थान॒ मेक॒ मेक॑ मु॒त्थान॒म् तत् । \newline
69. उ॒त्थान॒म् तत् तदु॒त्थान॑ मु॒त्थान॒म् तद्धि हि तदु॒त्थान॑ मु॒त्थान॒म् तद्धि । \newline
70. उ॒त्थान॒मित्यु॑त् - स्थान᳚म् । \newline
71. तद्धि हि तत् तद्धि ती॒र्थम् ती॒र्थꣳ हि तत् तद्धि ती॒र्थम् । \newline
72. हि ती॒र्थम् ती॒र्थꣳ हि हि ती॒र्थम् । \newline
73. ती॒र्त्थमिति॑ ती॒र्त्थम् । \newline
\pagebreak
\markright{ TS 7.2.2.1  \hfill https://www.vedavms.in \hfill}

\section{ TS 7.2.2.1 }

\textbf{TS 7.2.2.1 } \newline
\textbf{Samhita Paata} \newline

कु॒सु॒रु॒बिन्द॒ औद्दा॑लकिरकामयत पशु॒मान्थ् स्या॒मिति॒ स ए॒तꣳ स॑प्तरा॒त्रमाऽह॑र॒त् तेना॑यजत॒ तेन॒ वै स याव॑न्तो ग्रा॒म्याः प॒शव॒स्तानवा॑-रुन्ध॒ य ए॒वं ॅवि॒द्वान्थ् स॑प्तरा॒त्रेण॒ यज॑ते॒ याव॑न्त ए॒व ग्रा॒म्याः प॒शव॒स्ताने॒वाव॑ रुन्धे सप्तरा॒त्रो भ॑वति स॒प्त ग्रा॒म्याः प॒शवः॑ स॒प्ताऽऽ*र॒ण्याः स॒प्त छन्दाꣳ॑-स्यु॒भय॒स्या-व॑रुद्ध्यै त्रि॒वृद॑ग्निष्टो॒मो भ॑वति॒ तेज॑ - [  ] \newline

\textbf{Pada Paata} \newline

कु॒सु॒रु॒बिन्दः॑ । औद्दा॑लकि॒रित्यौत् - दा॒ल॒किः॒ । अ॒का॒म॒य॒त॒ । प॒शु॒मानिति॑ पशु - मान् । स्या॒म् । इति॑ । सः । ए॒तम् । स॒प्त॒रा॒त्रमिति॑ सप्त-रा॒त्रम् । एति॑ । अ॒ह॒र॒त् । तेन॑ । अ॒य॒ज॒त॒ । तेन॑ । वै । सः । याव॑न्तः । ग्रा॒म्याः । प॒शवः॑ । तान् । अवेति॑ । अ॒रु॒न्ध॒ । यः । ए॒वम् । वि॒द्वान् । स॒प्त॒रा॒त्रेणेति॑ सप्त - रा॒त्रेण॑ । यज॑ते । याव॑न्तः । ए॒व । ग्रा॒म्याः । प॒शवः॑ । तान् । ए॒व । अवेति॑ । रु॒न्धे॒ । स॒प्त॒रा॒त्र इति॑ सप्त-रा॒त्रः । भ॒व॒ति॒ । स॒प्त । ग्रा॒म्याः । प॒शवः॑ । स॒प्त । आ॒र॒ण्याः । स॒प्त । छन्दाꣳ॑सि । उ॒भय॑स्य । अव॑रुद्ध्या॒ इत्यव॑-रु॒द्ध्यै॒ । त्रि॒वृदिति॑ त्रि - वृत् । अ॒ग्नि॒ष्टो॒म इत्य॑ग्नि-स्तो॒मः । भ॒व॒ति॒ । तेजः॑ ।  \newline


\textbf{Krama Paata} \newline

कु॒सु॒रु॒बिन्द॒ औद्‍दा॑लकिः । औद्‍दा॑लकिरकामयत । औद्‍दा॑लकि॒रित्यौत् - दा॒ल॒किः॒ । अ॒का॒म॒य॒त॒ प॒शु॒मान् । प॒शु॒मान्थ् स्या᳚म् । प॒शु॒मानिति॑ पशु - मान् । स्या॒मिति॑ । इति॒ सः । स ए॒तम् । ए॒तꣳ स॑प्तरा॒त्रम् । स॒प्त॒रा॒त्रमा । स॒प्त॒रा॒त्रमिति॑ सप्त - रा॒त्रम् । आऽह॑रत् । अ॒ह॒र॒त् तेन॑ । तेना॑यजत । अ॒य॒ज॒त॒ तेन॑ । तेन॒ वै । वै सः । स याव॑न्तः । याव॑न्तो ग्रा॒म्याः । ग्रा॒म्याः प॒शवः॑ । प॒शव॒स्तान् । तानव॑ । अवा॑रुन्ध । अ॒रु॒न्ध॒ यः । य ए॒वम् । ए॒वम् ॅवि॒द्वान् । वि॒द्वान्थ् स॑प्तरा॒त्रेण॑ । स॒प्त॒रा॒त्रेण॒ यज॑ते । स॒प्त॒रा॒त्रेणेति॑ सप्त - रा॒त्रेण॑ । यज॑ते॒ याव॑न्तः । याव॑न्त ए॒व । ए॒व ग्रा॒म्याः । ग्रा॒म्याः प॒शवः॑ । प॒शव॒स्तान् । ताने॒व । ए॒वाव॑ । अव॑ रुन्धे । रु॒न्धे॒ स॒प्त॒रा॒त्रः । स॒प्त॒रा॒त्रो भ॑वति । स॒प्त॒रा॒त्र इति॑ सप्त - रा॒त्रः । भ॒व॒ति॒ स॒प्त । स॒प्त ग्रा॒म्याः । ग्रा॒म्याः प॒शवः॑ । प॒शवः॑ स॒प्त । स॒प्तार॒ण्याः । आ॒र॒ण्याः स॒प्त । स॒प्त छन्दाꣳ॑सि । छन्दाꣳ॑स्यु॒भय॑स्य । उ॒भय॒स्याव॑रुद्ध्यै । अव॑रुद्ध्यै त्रि॒वृत् । अव॑रुद्ध्या॒ इत्यव॑ - रु॒द्ध्यै॒ । त्रि॒वृद॑ग्निष्टो॒मः । त्रि॒वृदिति॑ त्रि - वृत् । अ॒ग्नि॒ष्टो॒मो भ॑वति । अ॒ग्नि॒ष्टो॒म इत्य॑ग्नि - स्तो॒मः । भ॒व॒ति॒ तेजः॑ । तेज॑ ए॒व \newline

\textbf{Jatai Paata} \newline

1. कु॒सु॒रु॒बिन्द॒ औद्दा॑लकि॒ रौद्दा॑लकिः कुसुरु॒बिन्दः॑ कुसुरु॒बिन्द॒ औद्दा॑लकिः । \newline
2. औद्दा॑लकि रकामयता कामय॒ तौद्दा॑लकि॒ रौद्दा॑लकि रकामयत । \newline
3. औद्दा॑लकि॒रित्यौत् - दा॒ल॒किः॒ । \newline
4. अ॒का॒म॒य॒त॒ प॒शु॒मान् प॑शु॒मा न॑कामयता कामयत पशु॒मान् । \newline
5. प॒शु॒मान् थ्स्याꣳ॑ स्याम् पशु॒मान् प॑शु॒मान् थ्स्या᳚म् । \newline
6. प॒शु॒मानिति॑ पशु - मान् । \newline
7. स्या॒ मितीति॑ स्याꣳ स्या॒ मिति॑ । \newline
8. इति॒ स स इतीति॒ सः । \newline
9. स ए॒त मे॒तꣳ स स ए॒तम् । \newline
10. ए॒तꣳ स॑प्तरा॒त्रꣳ स॑प्तरा॒त्र मे॒त मे॒तꣳ स॑प्तरा॒त्रम् । \newline
11. स॒प्त॒रा॒त्र मा स॑प्तरा॒त्रꣳ स॑प्तरा॒त्र मा । \newline
12. स॒प्त॒रा॒त्रमिति॑ सप्त - रा॒त्रम् । \newline
13. आ ऽह॑र दहर॒दा ऽह॑रत् । \newline
14. अ॒ह॒र॒त् तेन॒ तेना॑ हर दहर॒त् तेन॑ । \newline
15. तेना॑ यजता यजत॒ तेन॒ तेना॑ यजत । \newline
16. अ॒य॒ज॒त॒ तेन॒ तेना॑ यजता यजत॒ तेन॑ । \newline
17. तेन॒ वै वै तेन॒ तेन॒ वै । \newline
18. वै स स वै वै सः । \newline
19. स याव॑न्तो॒ याव॑न्तः॒ स स याव॑न्तः । \newline
20. याव॑न्तो ग्रा॒म्या ग्रा॒म्या याव॑न्तो॒ याव॑न्तो ग्रा॒म्याः । \newline
21. ग्रा॒म्याः प॒शवः॑ प॒शवो᳚ ग्रा॒म्या ग्रा॒म्याः प॒शवः॑ । \newline
22. प॒शव॒ स्ताꣳ स्तान् प॒शवः॑ प॒शव॒ स्तान् । \newline
23. तान वाव॒ ताꣳ स्तानव॑ । \newline
24. अवा॑ रुन्धा रु॒न्धा वावा॑ रुन्ध । \newline
25. अ॒रु॒न्ध॒ यो यो॑ ऽरुन्धा रुन्ध॒ यः । \newline
26. य ए॒व मे॒वं ॅयो य ए॒वम् । \newline
27. ए॒वं ॅवि॒द्वान्. वि॒द्वा ने॒व मे॒वं ॅवि॒द्वान् । \newline
28. वि॒द्वान् थ्स॑प्तरा॒त्रेण॑ सप्तरा॒त्रेण॑ वि॒द्वान्. वि॒द्वान् थ्स॑प्तरा॒त्रेण॑ । \newline
29. स॒प्त॒रा॒त्रेण॒ यज॑ते॒ यज॑ते सप्तरा॒त्रेण॑ सप्तरा॒त्रेण॒ यज॑ते । \newline
30. स॒प्त॒रा॒त्रेणेति॑ सप्त - रा॒त्रेण॑ । \newline
31. यज॑ते॒ याव॑न्तो॒ याव॑न्तो॒ यज॑ते॒ यज॑ते॒ याव॑न्तः । \newline
32. याव॑न्त ए॒वैव याव॑न्तो॒ याव॑न्त ए॒व । \newline
33. ए॒व ग्रा॒म्या ग्रा॒म्या ए॒वैव ग्रा॒म्याः । \newline
34. ग्रा॒म्याः प॒शवः॑ प॒शवो᳚ ग्रा॒म्या ग्रा॒म्याः प॒शवः॑ । \newline
35. प॒शव॒स्ताꣳ स्तान् प॒शवः॑ प॒शव॒ स्तान् । \newline
36. ताने॒वैव ताꣳ स्ताने॒व । \newline
37. ए॒वावा वै॒वै वाव॑ । \newline
38. अव॑ रुन्धे रु॒न्धे ऽवाव॑ रुन्धे । \newline
39. रु॒न्धे॒ स॒प्त॒रा॒त्रः स॑प्तरा॒त्रो रु॑न्धे रुन्धे सप्तरा॒त्रः । \newline
40. स॒प्त॒रा॒त्रो भ॑वति भवति सप्तरा॒त्रः स॑प्तरा॒त्रो भ॑वति । \newline
41. स॒प्त॒रा॒त्र इति॑ सप्त - रा॒त्रः । \newline
42. भ॒व॒ति॒ स॒प्त स॒प्त भ॑वति भवति स॒प्त । \newline
43. स॒प्त ग्रा॒म्या ग्रा॒म्याः स॒प्त स॒प्त ग्रा॒म्याः । \newline
44. ग्रा॒म्याः प॒शवः॑ प॒शवो᳚ ग्रा॒म्या ग्रा॒म्याः प॒शवः॑ । \newline
45. प॒शवः॑ स॒प्त स॒प्त प॒शवः॑ प॒शवः॑ स॒प्त । \newline
46. स॒प्ता र॒ण्या आ॑र॒ण्याः स॒प्त स॒प्ता र॒ण्याः । \newline
47. आ॒र॒ण्याः स॒प्त स॒प्ता र॒ण्या आ॑र॒ण्याः स॒प्त । \newline
48. स॒प्त छन्दाꣳ॑सि॒ छन्दाꣳ॑सि स॒प्त स॒प्त छन्दाꣳ॑सि । \newline
49. छन्दाꣳ॑ स्यु॒भय॑ स्यो॒भय॑स्य॒ छन्दाꣳ॑सि॒ छन्दाꣳ॑ स्यु॒भय॑स्य । \newline
50. उ॒भय॒स्या व॑रुद्ध्या॒ अव॑रुद्ध्या उ॒भय॑ स्यो॒भय॒स्या व॑रुद्ध्यै । \newline
51. अव॑रुद्ध्यै त्रि॒वृत् त्रि॒वृ दव॑रुद्ध्या॒ अव॑रुद्ध्यै त्रि॒वृत् । \newline
52. अव॑रुद्ध्या॒ इत्यव॑ - रु॒द्ध्यै॒ । \newline
53. त्रि॒वृ द॑ग्निष्टो॒मो᳚ ऽग्निष्टो॒म स्त्रि॒वृत् त्रि॒वृ द॑ग्निष्टो॒मः । \newline
54. त्रि॒वृदिति॑ त्रि - वृत् । \newline
55. अ॒ग्नि॒ष्टो॒मो भ॑वति भव त्यग्निष्टो॒मो᳚ ऽग्निष्टो॒मो भ॑वति । \newline
56. अ॒ग्नि॒ष्टो॒म इत्य॑ग्नि - स्तो॒मः । \newline
57. भ॒व॒ति॒ तेज॒ स्तेजो॑ भवति भवति॒ तेजः॑ । \newline
58. तेज॑ ए॒वैव तेज॒ स्तेज॑ ए॒व । \newline

\textbf{Ghana Paata } \newline

1. कु॒सु॒रु॒बिन्द॒ औद्दा॑लकि॒ रौद्दा॑लकिः कुसुरु॒बिन्दः॑ कुसुरु॒बिन्द॒ औद्दा॑लकि रकामयता कामय॒ तौद्दा॑लकिः कुसुरु॒बिन्दः॑ कुसुरु॒बिन्द॒ औद्दा॑लकि रकामयत । \newline
2. औद्दा॑लकि रकामयता कामय॒ तौद्दा॑लकि॒ रौद्दा॑लकि रकामयत पशु॒मान् प॑शु॒मा न॑कामय॒ तौद्दा॑लकि॒ रौद्दा॑लकि रकामयत पशु॒मान् । \newline
3. औद्दा॑लकि॒रित्यौत् - दा॒ल॒किः॒ । \newline
4. अ॒का॒म॒य॒त॒ प॒शु॒मान् प॑शु॒मा न॑कामयता कामयत पशु॒मान् थ्स्याꣳ॑ स्याम् पशु॒मा न॑कामयता कामयत पशु॒मान् थ्स्या᳚म् । \newline
5. प॒शु॒मान् थ्स्याꣳ॑ स्याम् पशु॒मान् प॑शु॒मान् थ्स्या॒ मितीति॑ स्याम् पशु॒मान् प॑शु॒मान् थ्स्या॒ मिति॑ । \newline
6. प॒शु॒मानिति॑ पशु - मान् । \newline
7. स्या॒ मितीति॑ स्याꣳ स्या॒ मिति॒ स स इति॑ स्याꣳ स्या॒ मिति॒ सः । \newline
8. इति॒ स स इतीति॒ स ए॒त मे॒तꣳ स इतीति॒ स ए॒तम् । \newline
9. स ए॒त मे॒तꣳ स स ए॒तꣳ स॑प्तरा॒त्रꣳ स॑प्तरा॒त्र मे॒तꣳ स स ए॒तꣳ स॑प्तरा॒त्रम् । \newline
10. ए॒तꣳ स॑प्तरा॒त्रꣳ स॑प्तरा॒त्र मे॒त मे॒तꣳ स॑प्तरा॒त्र मा स॑प्तरा॒त्र मे॒त मे॒तꣳ स॑प्तरा॒त्र मा । \newline
11. स॒प्त॒रा॒त्र मा स॑प्तरा॒त्रꣳ स॑प्तरा॒त्र मा ऽह॑र दहर॒दा स॑प्तरा॒त्रꣳ स॑प्तरा॒त्र मा ऽह॑रत् । \newline
12. स॒प्त॒रा॒त्रमिति॑ सप्त - रा॒त्रम् । \newline
13. आ ऽह॑र दहर॒दा ऽह॑र॒त् तेन॒ तेना॑ हर॒दा ऽह॑र॒त् तेन॑ । \newline
14. अ॒ह॒र॒त् तेन॒ तेना॑ हर दहर॒त् तेना॑ यजता यजत॒ तेना॑ हर दहर॒त् तेना॑ यजत । \newline
15. तेना॑ यजता यजत॒ तेन॒ तेना॑ यजत॒ तेन॒ तेना॑ यजत॒ तेन॒ तेना॑ यजत॒ तेन॑ । \newline
16. अ॒य॒ज॒त॒ तेन॒ तेना॑ यजता यजत॒ तेन॒ वै वै तेना॑ यजता यजत॒ तेन॒ वै । \newline
17. तेन॒ वै वै तेन॒ तेन॒ वै स स वै तेन॒ तेन॒ वै सः । \newline
18. वै स स वै वै स याव॑न्तो॒ याव॑न्तः॒ स वै वै स याव॑न्तः । \newline
19. स याव॑न्तो॒ याव॑न्तः॒ स स याव॑न्तो ग्रा॒म्या ग्रा॒म्या याव॑न्तः॒ स स याव॑न्तो ग्रा॒म्याः । \newline
20. याव॑न्तो ग्रा॒म्या ग्रा॒म्या याव॑न्तो॒ याव॑न्तो ग्रा॒म्याः प॒शवः॑ प॒शवो᳚ ग्रा॒म्या याव॑न्तो॒ याव॑न्तो ग्रा॒म्याः प॒शवः॑ । \newline
21. ग्रा॒म्याः प॒शवः॑ प॒शवो᳚ ग्रा॒म्या ग्रा॒म्याः प॒शव॒ स्ताꣳ स्तान् प॒शवो᳚ ग्रा॒म्या ग्रा॒म्याः प॒शव॒ स्तान् । \newline
22. प॒शव॒ स्ताꣳ स्तान् प॒शवः॑ प॒शव॒ स्ता नवाव॒ तान् प॒शवः॑ प॒शव॒ स्तानव॑ । \newline
23. तान वाव॒ ताꣳ स्ता नवा॑रुन्धा रु॒न्धाव॒ ताꣳ स्ता नवा॑रुन्ध । \newline
24. अवा॑रुन्धा रु॒न्धा वावा॑ रुन्ध॒ यो यो॑ ऽरु॒न्धा वावा॑ रुन्ध॒ यः । \newline
25. अ॒रु॒न्ध॒ यो यो॑ ऽरुन्धा रुन्ध॒ य ए॒व मे॒वं ॅयो॑ ऽरुन्धा रुन्ध॒ य ए॒वम् । \newline
26. य ए॒व मे॒वं ॅयो य ए॒वं ॅवि॒द्वान्. वि॒द्वा ने॒वं ॅयो य ए॒वं ॅवि॒द्वान् । \newline
27. ए॒वं ॅवि॒द्वान्. वि॒द्वा ने॒व मे॒वं ॅवि॒द्वान् थ्स॑प्तरा॒त्रेण॑ सप्तरा॒त्रेण॑ वि॒द्वा ने॒व मे॒वं ॅवि॒द्वान् थ्स॑प्तरा॒त्रेण॑ । \newline
28. वि॒द्वान् थ्स॑प्तरा॒त्रेण॑ सप्तरा॒त्रेण॑ वि॒द्वान्. वि॒द्वान् थ्स॑प्तरा॒त्रेण॒ यज॑ते॒ यज॑ते सप्तरा॒त्रेण॑ वि॒द्वान्. वि॒द्वान् थ्स॑प्तरा॒त्रेण॒ यज॑ते । \newline
29. स॒प्त॒रा॒त्रेण॒ यज॑ते॒ यज॑ते सप्तरा॒त्रेण॑ सप्तरा॒त्रेण॒ यज॑ते॒ याव॑न्तो॒ याव॑न्तो॒ यज॑ते सप्तरा॒त्रेण॑ सप्तरा॒त्रेण॒ यज॑ते॒ याव॑न्तः । \newline
30. स॒प्त॒रा॒त्रेणेति॑ सप्त - रा॒त्रेण॑ । \newline
31. यज॑ते॒ याव॑न्तो॒ याव॑न्तो॒ यज॑ते॒ यज॑ते॒ याव॑न्त ए॒वैव याव॑न्तो॒ यज॑ते॒ यज॑ते॒ याव॑न्त ए॒व । \newline
32. याव॑न्त ए॒वैव याव॑न्तो॒ याव॑न्त ए॒व ग्रा॒म्या ग्रा॒म्या ए॒व याव॑न्तो॒ याव॑न्त ए॒व ग्रा॒म्याः । \newline
33. ए॒व ग्रा॒म्या ग्रा॒म्या ए॒वैव ग्रा॒म्याः प॒शवः॑ प॒शवो᳚ ग्रा॒म्या ए॒वैव ग्रा॒म्याः प॒शवः॑ । \newline
34. ग्रा॒म्याः प॒शवः॑ प॒शवो᳚ ग्रा॒म्या ग्रा॒म्याः प॒शव॒ स्ताꣳ स्तान् प॒शवो᳚ ग्रा॒म्या ग्रा॒म्याः प॒शव॒ स्तान् । \newline
35. प॒शव॒ स्ताꣳ स्तान् प॒शवः॑ प॒शव॒ स्ताने॒वैव तान् प॒शवः॑ प॒शव॒ स्ताने॒व । \newline
36. ताने॒ वैव ताꣳ स्ताने॒ वावा वै॒व ताꣳ स्ताने॒ वाव॑ । \newline
37. ए॒वावा वै॒वै वाव॑ रुन्धे रु॒न्धे ऽवै॒वै वाव॑ रुन्धे । \newline
38. अव॑ रुन्धे रु॒न्धे ऽवाव॑ रुन्धे सप्तरा॒त्रः स॑प्तरा॒त्रो रु॒न्धे ऽवाव॑ रुन्धे सप्तरा॒त्रः । \newline
39. रु॒न्धे॒ स॒प्त॒रा॒त्रः स॑प्तरा॒त्रो रु॑न्धे रुन्धे सप्तरा॒त्रो भ॑वति भवति सप्तरा॒त्रो रु॑न्धे रुन्धे सप्तरा॒त्रो भ॑वति । \newline
40. स॒प्त॒रा॒त्रो भ॑वति भवति सप्तरा॒त्रः स॑प्तरा॒त्रो भ॑वति स॒प्त स॒प्त भ॑वति सप्तरा॒त्रः स॑प्तरा॒त्रो भ॑वति स॒प्त । \newline
41. स॒प्त॒रा॒त्र इति॑ सप्त - रा॒त्रः । \newline
42. भ॒व॒ति॒ स॒प्त स॒प्त भ॑वति भवति स॒प्त ग्रा॒म्या ग्रा॒म्याः स॒प्त भ॑वति भवति स॒प्त ग्रा॒म्याः । \newline
43. स॒प्त ग्रा॒म्या ग्रा॒म्याः स॒प्त स॒प्त ग्रा॒म्याः प॒शवः॑ प॒शवो᳚ ग्रा॒म्याः स॒प्त स॒प्त ग्रा॒म्याः प॒शवः॑ । \newline
44. ग्रा॒म्याः प॒शवः॑ प॒शवो᳚ ग्रा॒म्या ग्रा॒म्याः प॒शवः॑ स॒प्त स॒प्त प॒शवो᳚ ग्रा॒म्या ग्रा॒म्याः प॒शवः॑ स॒प्त । \newline
45. प॒शवः॑ स॒प्त स॒प्त प॒शवः॑ प॒शवः॑ स॒प्तार॒ण्या आ॑र॒ण्याः स॒प्त प॒शवः॑ प॒शवः॑ स॒प्ता र॒ण्याः । \newline
46. स॒प्ता र॒ण्या आ॑र॒ण्याः स॒प्त स॒प्ता र॒ण्याः स॒प्त स॒प्ता र॒ण्याः स॒प्त स॒प्ता र॒ण्याः स॒प्त । \newline
47. आ॒र॒ण्याः स॒प्त स॒प्ता र॒ण्या आ॑र॒ण्याः स॒प्त छन्दाꣳ॑सि॒ छन्दाꣳ॑सि स॒प्ता र॒ण्या आ॑र॒ण्याः स॒प्त छन्दाꣳ॑सि । \newline
48. स॒प्त छन्दाꣳ॑सि॒ छन्दाꣳ॑सि स॒प्त स॒प्त छन्दाꣳ॑ स्यु॒भय॑ स्यो॒भय॑स्य॒ छन्दाꣳ॑सि स॒प्त स॒प्त छन्दाꣳ॑ स्यु॒भय॑स्य । \newline
49. छन्दाꣳ॑ स्यु॒भय॑ स्यो॒भय॑स्य॒ छन्दाꣳ॑सि॒ छन्दाꣳ॑ स्यु॒भय॒स्या व॑रुद्ध्या॒ अव॑रुद्ध्या उ॒भय॑स्य॒ छन्दाꣳ॑सि॒ छन्दाꣳ॑ स्यु॒भय॒स्या व॑रुद्ध्यै । \newline
50. उ॒भय॒स्या व॑रुद्ध्या॒ अव॑रुद्ध्या उ॒भय॑स्यो॒ भय॒स्या व॑रुद्ध्यै त्रि॒वृत् त्रि॒वृ दव॑रुद्ध्या उ॒भय॑
स्यो॒भय॒स्या व॑रुद्ध्यै त्रि॒वृत् । \newline
51. अव॑रुद्ध्यै त्रि॒वृत् त्रि॒वृ दव॑रुद्ध्या॒ अव॑रुद्ध्यै त्रि॒वृ द॑ग्निष्टो॒मो᳚ ऽग्निष्टो॒म स्त्रि॒वृ दव॑रुद्ध्या॒ अव॑रुद्ध्यै त्रि॒वृ द॑ग्निष्टो॒मः । \newline
52. अव॑रुद्ध्या॒ इत्यव॑ - रु॒द्ध्यै॒ । \newline
53. त्रि॒वृ द॑ग्निष्टो॒मो᳚ ऽग्निष्टो॒म स्त्रि॒वृत् त्रि॒वृ द॑ग्निष्टो॒मो भ॑वति भव त्यग्निष्टो॒म स्त्रि॒वृत् त्रि॒वृ द॑ग्निष्टो॒मो भ॑वति । \newline
54. त्रि॒वृदिति॑ त्रि - वृत् । \newline
55. अ॒ग्नि॒ष्टो॒मो भ॑वति भव त्यग्निष्टो॒मो᳚ ऽग्निष्टो॒मो भ॑वति॒ तेज॒ स्तेजो॑ भव त्यग्निष्टो॒मो᳚ ऽग्निष्टो॒मो भ॑वति॒ तेजः॑ । \newline
56. अ॒ग्नि॒ष्टो॒म इत्य॑ग्नि - स्तो॒मः । \newline
57. भ॒व॒ति॒ तेज॒ स्तेजो॑ भवति भवति॒ तेज॑ ए॒वैव तेजो॑ भवति भवति॒ तेज॑ ए॒व । \newline
58. तेज॑ ए॒वैव तेज॒ स्तेज॑ ए॒वावा वै॒व तेज॒ स्तेज॑ ए॒वाव॑ । \newline
\pagebreak
\markright{ TS 7.2.2.2  \hfill https://www.vedavms.in \hfill}

\section{ TS 7.2.2.2 }

\textbf{TS 7.2.2.2 } \newline
\textbf{Samhita Paata} \newline

ए॒वाव॑ रुन्धे पञ्चद॒शो भ॑वतीन्द्रि॒यमे॒वाव॑ रुन्धे सप्तद॒शो भ॑वत्य॒न्नाद्य॒स्या व॑रुद्ध्या॒ अथो॒ प्रैव तेन॑ जायत एकविꣳ॒॒शो भ॑वति॒ प्रति॑ष्ठित्या॒ अथो॒ रुच॑मे॒वाऽऽ*त्मन् ध॑त्ते त्रिण॒वो भ॑वति॒ विजि॑त्यै पञ्चविꣳ॒॒शो᳚-ऽग्निष्टो॒मो भ॑वति प्र॒जाप॑ते॒राप्त्यै॑ महाव्र॒तवा॑-न॒न्नाद्य॒स्या व॑रुद्ध्यै विश्व॒जिथ् सर्व॑पृष्ठो ऽतिरा॒त्रो भ॑वति॒ सर्व॑स्या॒भिजि॑त्यै॒ यत् प्र॒त्यक्षं॒ पूर्वे॒ष्वह॑स्सु पृ॒ष्ठान्यु॑पे॒युः प्र॒त्यक्षं॑ -[  ] \newline

\textbf{Pada Paata} \newline

ए॒व । अवेति॑ । रु॒न्धे॒ । प॒ञ्च॒द॒श इति॑ पञ्च - द॒शः । भ॒व॒ति॒ । इ॒न्द्रि॒यम् । ए॒व । अवेति॑ । रु॒न्धे॒ । स॒प्त॒द॒श इति॑ सप्त - द॒शः । भ॒व॒ति॒ । अ॒न्नाद्य॒स्येत्य॑न्न - अद्य॑स्य । अव॑रुद्ध्या॒ इत्यव॑-रु॒द्ध्यै॒ । अथो॒ इति॑ । प्रेति॑ । ए॒व । तेन॑ । जा॒य॒ते॒ । ए॒क॒विꣳ॒॒श इत्ये॑क - विꣳ॒॒शः । भ॒व॒ति॒ । प्रति॑ष्ठित्या॒ इति॒ प्रति॑ - स्थि॒त्यै॒ । अथो॒ इति॑ । रुच᳚म् । ए॒व । आ॒त्मन्न् । ध॒त्ते॒ । त्रि॒ण॒व इति॑ त्रि-न॒वः । भ॒व॒ति॒ । विजि॑त्या॒ इति॒ वि - जि॒त्यै॒ । प॒ञ्च॒विꣳ॒॒श इति॑ पञ्च - विꣳ॒॒शः । अ॒ग्नि॒ष्टो॒म इत्य॑ग्नि - स्तो॒मः । भ॒व॒ति॒ । प्र॒जाप॑ते॒रिति॑ प्र॒जा-प॒तेः॒ । आप्त्यै᳚ । म॒हा॒व्र॒तवा॒निति॑ महाव्र॒त-वा॒न् । अ॒न्नाद्य॒स्येत्य॑न्न-अद्य॑स्य । अव॑रुद्ध्या॒ इत्यव॑ - रु॒द्ध्यै॒ । वि॒श्व॒जिदिति॑ विश्व - जित् । सर्व॑पृष्ठ॒ इति॒ सर्व॑ - पृ॒ष्ठः॒ । अ॒ति॒रा॒त्र इत्य॑ति - रा॒त्रः । भ॒व॒ति॒ । सर्व॑स्य । अ॒भिजि॑त्या॒ इत्य॒भि - जि॒त्यै॒ । यत् । प्र॒त्यक्ष॒मिति॑ प्रति - अक्ष᳚म् । पूर्वे॑षु । अह॒स्स्वित्यहः॑ - सु॒ । पृ॒ष्ठानि॑ । उ॒पे॒युरित्यु॑प - इ॒युः । प्र॒त्यक्ष॒मिति॑ प्रति - अक्ष᳚म् ।  \newline


\textbf{Krama Paata} \newline

ए॒वाव॑ । अव॑ रुन्धे । रु॒न्धे॒ प॒ञ्च॒द॒शः । प॒ञ्च॒द॒शो भ॑वति । प॒ञ्च॒द॒श इति॑ पञ्च - द॒शः । भ॒व॒ती॒न्द्रि॒यम् । इ॒न्द्रि॒यमे॒व । ए॒वाव॑ । अव॑ रुन्धे । रु॒न्धे॒ स॒प्त॒द॒शः । स॒प्त॒द॒शो भ॑वति । स॒प्त॒द॒श इति॑ सप्त - द॒शः । भ॒व॒त्य॒न्नाद्य॑स्य । अ॒न्नाद्य॒स्याव॑रुद्ध्यै । अ॒न्नाद्य॒स्येत्य॑न्न - अद्य॑स्य । अव॑रुद्ध्या॒ अथो᳚ । अव॑रुद्ध्या॒ इत्यव॑ - रु॒द्ध्यै॒ । अथो॒ प्र । अथो॒ इत्यथो᳚ । प्रैव । ए॒व तेन॑ । तेन॑ जायते । जा॒य॒त॒ ए॒क॒विꣳ॒॒शः । ए॒क॒विꣳ॒॒शो भ॑वति । ए॒क॒विꣳ॒॒श इत्ये॑क - विꣳ॒॒शः । भ॒व॒ति॒ प्रति॑ष्ठित्यै । प्रति॑ष्ठित्या॒ अथो᳚ । प्रति॑ष्ठित्या॒ इति॒ प्रति॑ - स्थि॒त्यै॒ । अथो॒ रुच᳚म् । अथो॒ इत्यथो᳚ । रुच॑मे॒व । ए॒वात्मन्न् । आ॒त्मन् ध॑त्ते । ध॒त्ते॒ त्रि॒ण॒वः । त्रि॒ण॒वो भ॑वति । त्रि॒ण॒व इति॑ त्रि - न॒वः । भ॒व॒ति॒ विजि॑त्यै । विजि॑त्यै पञ्चविꣳ॒॒शः । विजि॑त्या॒ इति॒ वि - जि॒त्यै॒ । प॒ञ्च॒विꣳ॒॒शो᳚ऽग्निष्टो॒मः । प॒ञ्च॒विꣳ॒॒श इति॑ पञ्च - विꣳ॒॒शः । अ॒ग्नि॒ष्टो॒मो भ॑वति । अ॒ग्नि॒ष्टो॒म इत्य॑ग्नि - स्तो॒मः । भ॒व॒ति॒ प्र॒जाप॑तेः । प्र॒जाप॑ते॒राप्त्यै᳚ । प्र॒जाप॑ते॒रिति॑ प्र॒जा - प॒तेः॒ । आप्त्यै॑ महाव्र॒तवान्॑ । म॒हा॒व्र॒त,वा॑न॒न्नाद्य॑स्य । म॒हा॒व्र॒तवा॒निति॑ महाव्र॒त - वा॒न्॒ । अ॒न्नाद्य॒स्या,व॑रुद्ध्यै । अ॒न्नाद्य॒स्येत्य॑न्न - अद्य॑स्य । अव॑रुद्ध्यै विश्व॒जित् । अव॑रुद्ध्या॒ इत्यव॑ - रु॒द्ध्यै॒ । वि॒श्व॒जिथ् सर्व॑पृष्ठः । वि॒श्व॒जिदिति॑ विश्व - जित् । सर्व॑पृष्ठोऽतिरा॒त्रः । सर्व॑पृष्ठ॒ इति॒ सर्व॑ - पृ॒ष्ठः॒ । अ॒ति॒रा॒त्रो भ॑वति । अ॒ति॒रा॒त्र इत्य॑ति - रा॒त्रः । भ॒व॒ति॒ सर्व॑स्य । सर्व॑स्या॒भिजि॑त्यै । अ॒भिजि॑त्यै॒ यत् । अ॒भिजि॑त्या॒ इत्य॒भि - जि॒त्यै॒ । यत् प्र॒त्यक्ष᳚म् । प्र॒त्यक्ष॒म् पूर्वे॑षु । प्र॒त्यक्ष॒मिति॑ प्रति - अक्ष᳚म् । पूर्वे॒ष्वह॑स्सु । अह॑स्सु पृ॒ष्ठानि॑ । अह॒स्स्वित्यहः॑ - सु॒ । पृ॒ष्ठान्यु॑पे॒युः । उ॒पे॒युः प्र॒त्यक्ष᳚म् ( ) । उ॒पे॒युरित्यु॑प - इ॒युः । प्र॒त्यक्ष॑म् ॅविश्व॒जिति॑ । प्र॒त्यक्ष॒मिति॑ प्रति - अक्ष᳚म् \newline

\textbf{Jatai Paata} \newline

1. ए॒वावा वै॒वै वाव॑ । \newline
2. अव॑ रुन्धे रु॒न्धे ऽवाव॑ रुन्धे । \newline
3. रु॒न्धे॒ प॒ञ्च॒द॒शः प॑ञ्चद॒शो रु॑न्धे रुन्धे पञ्चद॒शः । \newline
4. प॒ञ्च॒द॒शो भ॑वति भवति पञ्चद॒शः प॑ञ्चद॒शो भ॑वति । \newline
5. प॒ञ्च॒द॒श इति॑ पञ्च - द॒शः । \newline
6. भ॒व॒ ती॒न्द्रि॒य मि॑न्द्रि॒यम् भ॑वति भव तीन्द्रि॒यम् । \newline
7. इ॒न्द्रि॒य मे॒वैवेन्द्रि॒य मि॑न्द्रि॒य मे॒व । \newline
8. ए॒वावा वै॒वै वाव॑ । \newline
9. अव॑ रुन्धे रु॒न्धे ऽवाव॑ रुन्धे । \newline
10. रु॒न्धे॒ स॒प्त॒द॒शः स॑प्तद॒शो रु॑न्धे रुन्धे सप्तद॒शः । \newline
11. स॒प्त॒द॒शो भ॑वति भवति सप्तद॒शः स॑प्तद॒शो भ॑वति । \newline
12. स॒प्त॒द॒श इति॑ सप्त - द॒शः । \newline
13. भ॒व॒ त्य॒न्नाद्य॑स्या॒ न्नाद्य॑स्य भवति भव त्य॒न्नाद्य॑स्य । \newline
14. अ॒न्नाद्य॒ स्याव॑रुद्ध्या॒ अव॑रुद्ध्या अ॒न्नाद्य॑स्या॒ न्नाद्य॒ स्याव॑रुद्ध्यै । \newline
15. अ॒न्नाद्य॒स्येत्य॑न्न - अद्य॑स्य । \newline
16. अव॑रुद्ध्या॒ अथो॒ अथो॒ अव॑रुद्ध्या॒ अव॑रुद्ध्या॒ अथो᳚ । \newline
17. अव॑रुद्ध्या॒ इत्यव॑ - रु॒द्ध्यै॒ । \newline
18. अथो॒ प्र प्राथो॒ अथो॒ प्र । \newline
19. अथो॒ इत्यथो᳚ । \newline
20. प्रैवैव प्र प्रैव । \newline
21. ए॒व तेन॒ तेनै॒वैव तेन॑ । \newline
22. तेन॑ जायते जायते॒ तेन॒ तेन॑ जायते । \newline
23. जा॒य॒त॒ ए॒क॒विꣳ॒॒श ए॑कविꣳ॒॒शो जा॑यते जायत एकविꣳ॒॒शः । \newline
24. ए॒क॒विꣳ॒॒शो भ॑वति भव त्येकविꣳ॒॒श ए॑कविꣳ॒॒शो भ॑वति । \newline
25. ए॒क॒विꣳ॒॒श इत्ये॑क - विꣳ॒॒शः । \newline
26. भ॒व॒ति॒ प्रति॑ष्ठित्यै॒ प्रति॑ष्ठित्यै भवति भवति॒ प्रति॑ष्ठित्यै । \newline
27. प्रति॑ष्ठित्या॒ अथो॒ अथो॒ प्रति॑ष्ठित्यै॒ प्रति॑ष्ठित्या॒ अथो᳚ । \newline
28. प्रति॑ष्ठित्या॒ इति॒ प्रति॑ - स्थि॒त्यै॒ । \newline
29. अथो॒ रुचꣳ॒॒ रुच॒ मथो॒ अथो॒ रुच᳚म् । \newline
30. अथो॒ इत्यथो᳚ । \newline
31. रुच॑ मे॒वैव रुचꣳ॒॒ रुच॑ मे॒व । \newline
32. ए॒वात्मन् ना॒त्मन् ने॒वै वात्मन्न् । \newline
33. आ॒त्मन् ध॑त्ते धत्त आ॒त्मन् ना॒त्मन् ध॑त्ते । \newline
34. ध॒त्ते॒ त्रि॒ण॒व स्त्रि॑ण॒वो ध॑त्ते धत्ते त्रिण॒वः । \newline
35. त्रि॒ण॒वो भ॑वति भवति त्रिण॒व स्त्रि॑ण॒वो भ॑वति । \newline
36. त्रि॒ण॒व इति॑ त्रि - न॒वः । \newline
37. भ॒व॒ति॒ विजि॑त्यै॒ विजि॑त्यै भवति भवति॒ विजि॑त्यै । \newline
38. विजि॑त्यै पञ्चविꣳ॒॒शः प॑ञ्चविꣳ॒॒शो विजि॑त्यै॒ विजि॑त्यै पञ्चविꣳ॒॒शः । \newline
39. विजि॑त्या॒ इति॒ वि - जि॒त्यै॒ । \newline
40. प॒ञ्च॒विꣳ॒॒शो᳚ ऽग्निष्टो॒मो᳚ ऽग्निष्टो॒मः प॑ञ्चविꣳ॒॒शः प॑ञ्चविꣳ॒॒शो᳚ ऽग्निष्टो॒मः । \newline
41. प॒ञ्च॒विꣳ॒॒श इति॑ पञ्च - विꣳ॒॒शः । \newline
42. अ॒ग्नि॒ष्टो॒मो भ॑वति भव त्यग्निष्टो॒मो᳚ ऽग्निष्टो॒मो भ॑वति । \newline
43. अ॒ग्नि॒ष्टो॒म इत्य॑ग्नि - स्तो॒मः । \newline
44. भ॒व॒ति॒ प्र॒जाप॑तेः प्र॒जाप॑तेर् भवति भवति प्र॒जाप॑तेः । \newline
45. प्र॒जाप॑ते॒ राप्त्या॒ आप्त्यै᳚ प्र॒जाप॑तेः प्र॒जाप॑ते॒ राप्त्यै᳚ । \newline
46. प्र॒जाप॑ते॒रिति॑ प्र॒जा - प॒तेः॒ । \newline
47. आप्त्यै॑ महाव्र॒तवा᳚न् महाव्र॒तवा॒ नाप्त्या॒ आप्त्यै॑ महाव्र॒तवान्॑ । \newline
48. म॒हा॒व्र॒तवा॑ न॒न्नाद्य॑स्या॒ न्नाद्य॑स्य महाव्र॒तवा᳚न् महाव्र॒तवा॑ न॒न्नाद्य॑स्य । \newline
49. म॒हा॒व्र॒तवा॒निति॑ महाव्र॒त - वा॒न् । \newline
50. अ॒न्नाद्य॒स्या व॑रुद्ध्या॒ अव॑रुद्ध्या अ॒न्नाद्य॑स्या॒ न्नाद्य॒स्या व॑रुद्ध्यै । \newline
51. अ॒न्नाद्य॒स्येत्य॑न्न - अद्य॑स्य । \newline
52. अव॑रुद्ध्यै विश्व॒जिद् वि॑श्व॒जि दव॑रुद्ध्या॒ अव॑रुद्ध्यै विश्व॒जित् । \newline
53. अव॑रुद्ध्या॒ इत्यव॑ - रु॒द्ध्यै॒ । \newline
54. वि॒श्व॒जिथ् सर्व॑पृष्ठः॒ सर्व॑पृष्ठो विश्व॒जिद् वि॑श्व॒जिथ् सर्व॑पृष्ठः । \newline
55. वि॒श्व॒जिदिति॑ विश्व - जित् । \newline
56. सर्व॑पृष्ठो ऽतिरा॒त्रो॑ ऽतिरा॒त्रः सर्व॑पृष्ठः॒ सर्व॑पृष्ठो ऽतिरा॒त्रः । \newline
57. सर्व॑पृष्ठ॒ इति॒ सर्व॑ - पृ॒ष्ठः॒ । \newline
58. अ॒ति॒रा॒त्रो भ॑वति भव त्यतिरा॒त्रो॑ ऽतिरा॒त्रो भ॑वति । \newline
59. अ॒ति॒रा॒त्र इत्य॑ति - रा॒त्रः । \newline
60. भ॒व॒ति॒ सर्व॑स्य॒ सर्व॑स्य भवति भवति॒ सर्व॑स्य । \newline
61. सर्व॑स्या॒ भिजि॑त्या अ॒भिजि॑त्यै॒ सर्व॑स्य॒ सर्व॑स्या॒ भिजि॑त्यै । \newline
62. अ॒भिजि॑त्यै॒ यद् यद॒भिजि॑त्या अ॒भिजि॑त्यै॒ यत् । \newline
63. अ॒भिजि॑त्या॒ इत्य॒भि - जि॒त्यै॒ । \newline
64. यत् प्र॒त्यक्ष॑म् प्र॒त्यक्षं॒ ॅयद् यत् प्र॒त्यक्ष᳚म् । \newline
65. प्र॒त्यक्ष॒म् पूर्वे॑षु॒ पूर्वे॑षु प्र॒त्यक्ष॑म् प्र॒त्यक्ष॒म् पूर्वे॑षु । \newline
66. प्र॒त्यक्ष॒मिति॑ प्रति - अक्ष᳚म् । \newline
67. पूर्वे॒ ष्वह॒ स्स्वह॑स्सु॒ पूर्वे॑षु॒ पूर्वे॒ ष्वह॑स्सु । \newline
68. अह॑स्सु पृ॒ष्ठानि॑ पृ॒ष्ठा न्यह॒ स्स्वह॑स्सु पृ॒ष्ठानि॑ । \newline
69. अह॒स्स्वित्यहः॑ - सु॒ । \newline
70. पृ॒ष्ठा न्यु॑पे॒यु रु॑पे॒युः पृ॒ष्ठानि॑ पृ॒ष्ठा न्यु॑पे॒युः । \newline
71. उ॒पे॒युः प्र॒त्यक्ष॑म् प्र॒त्यक्ष॑ मुपे॒यु रु॑पे॒युः प्र॒त्यक्ष᳚म् । \newline
72. उ॒पे॒युरित्यु॑प - इ॒युः । \newline
73. प्र॒त्यक्षं॑ ॅविश्व॒जिति॑ विश्व॒जिति॑ प्र॒त्यक्ष॑म् प्र॒त्यक्षं॑ ॅविश्व॒जिति॑ । \newline
74. प्र॒त्यक्ष॒मिति॑ प्रति - अक्ष᳚म् । \newline

\textbf{Ghana Paata } \newline

1. ए॒वावा वै॒वै वाव॑ रुन्धे रु॒न्धे ऽवै॒वै वाव॑ रुन्धे । \newline
2. अव॑ रुन्धे रु॒न्धे ऽवाव॑ रुन्धे पञ्चद॒शः प॑ञ्चद॒शो रु॒न्धे ऽवाव॑ रुन्धे पञ्चद॒शः । \newline
3. रु॒न्धे॒ प॒ञ्च॒द॒शः प॑ञ्चद॒शो रु॑न्धे रुन्धे पञ्चद॒शो भ॑वति भवति पञ्चद॒शो रु॑न्धे रुन्धे पञ्चद॒शो भ॑वति । \newline
4. प॒ञ्च॒द॒शो भ॑वति भवति पञ्चद॒शः प॑ञ्चद॒शो भ॑व तीन्द्रि॒य मि॑न्द्रि॒यम् भ॑वति पञ्चद॒शः प॑ञ्चद॒शो भ॑व तीन्द्रि॒यम् । \newline
5. प॒ञ्च॒द॒श इति॑ पञ्च - द॒शः । \newline
6. भ॒व॒ ती॒न्द्रि॒य मि॑न्द्रि॒यम् भ॑वति भव तीन्द्रि॒य मे॒वैवेन्द्रि॒यम् भ॑वति भव तीन्द्रि॒य मे॒व । \newline
7. इ॒न्द्रि॒य मे॒वैवेन्द्रि॒य मि॑न्द्रि॒य मे॒वावा वै॒वेन्द्रि॒य मि॑न्द्रि॒य मे॒वाव॑ । \newline
8. ए॒वावा वै॒वै वाव॑ रुन्धे रु॒न्धे ऽवै॒वै वाव॑ रुन्धे । \newline
9. अव॑ रुन्धे रु॒न्धे ऽवाव॑ रुन्धे सप्तद॒शः स॑प्तद॒शो रु॒न्धे ऽवाव॑ रुन्धे सप्तद॒शः । \newline
10. रु॒न्धे॒ स॒प्त॒द॒शः स॑प्तद॒शो रु॑न्धे रुन्धे सप्तद॒शो भ॑वति भवति सप्तद॒शो रु॑न्धे रुन्धे सप्तद॒शो भ॑वति । \newline
11. स॒प्त॒द॒शो भ॑वति भवति सप्तद॒शः स॑प्तद॒शो भ॑व त्य॒न्नाद्य॑स्या॒ न्नाद्य॑स्य भवति सप्तद॒शः स॑प्तद॒शो भ॑व त्य॒न्नाद्य॑स्य । \newline
12. स॒प्त॒द॒श इति॑ सप्त - द॒शः । \newline
13. भ॒व॒ त्य॒न्नाद्य॑स्या॒ न्नाद्य॑स्य भवति भव त्य॒न्नाद्य॒स्या व॑रुद्ध्या॒ अव॑रुद्ध्या अ॒न्नाद्य॑स्य भवति भव त्य॒न्नाद्य॒स्या व॑रुद्ध्यै । \newline
14. अ॒न्नाद्य॒स्या व॑रुद्ध्या॒ अव॑रुद्ध्या अ॒न्नाद्य॑स्या॒ न्नाद्य॒स्या व॑रुद्ध्या॒ अथो॒ अथो॒ अव॑रुद्ध्या 
अ॒न्नाद्य॑स्या॒ न्नाद्य॒स्या व॑रुद्ध्या॒ अथो᳚ । \newline
15. अ॒न्नाद्य॒स्येत्य॑न्न - अद्य॑स्य । \newline
16. अव॑रुद्ध्या॒ अथो॒ अथो॒ अव॑रुद्ध्या॒ अव॑रुद्ध्या॒ अथो॒ प्र प्राथो॒ अव॑रुद्ध्या॒ अव॑रुद्ध्या॒ अथो॒ प्र । \newline
17. अव॑रुद्ध्या॒ इत्यव॑ - रु॒द्ध्यै॒ । \newline
18. अथो॒ प्र प्राथो॒ अथो॒ प्रैवैव प्राथो॒ अथो॒ प्रैव । \newline
19. अथो॒ इत्यथो᳚ । \newline
20. प्रैवैव प्र प्रैव तेन॒ तेनै॒व प्र प्रैव तेन॑ । \newline
21. ए॒व तेन॒ तेनै॒ वैव तेन॑ जायते जायते॒ तेनै॒ वैव तेन॑ जायते । \newline
22. तेन॑ जायते जायते॒ तेन॒ तेन॑ जायत एकविꣳ॒॒श ए॑कविꣳ॒॒शो जा॑यते॒ तेन॒ तेन॑ जायत एकविꣳ॒॒शः । \newline
23. जा॒य॒त॒ ए॒क॒विꣳ॒॒श ए॑कविꣳ॒॒शो जा॑यते जायत एकविꣳ॒॒शो भ॑वति भव त्येकविꣳ॒॒शो जा॑यते जायत एकविꣳ॒॒शो भ॑वति । \newline
24. ए॒क॒विꣳ॒॒शो भ॑वति भव त्येकविꣳ॒॒श ए॑कविꣳ॒॒शो भ॑वति॒ प्रति॑ष्ठित्यै॒ प्रति॑ष्ठित्यै भव त्येकविꣳ॒॒श ए॑कविꣳ॒॒शो भ॑वति॒ प्रति॑ष्ठित्यै । \newline
25. ए॒क॒विꣳ॒॒श इत्ये॑क - विꣳ॒॒शः । \newline
26. भ॒व॒ति॒ प्रति॑ष्ठित्यै॒ प्रति॑ष्ठित्यै भवति भवति॒ प्रति॑ष्ठित्या॒ अथो॒ अथो॒ प्रति॑ष्ठित्यै भवति भवति॒ प्रति॑ष्ठित्या॒ अथो᳚ । \newline
27. प्रति॑ष्ठित्या॒ अथो॒ अथो॒ प्रति॑ष्ठित्यै॒ प्रति॑ष्ठित्या॒ अथो॒ रुचꣳ॒॒ रुच॒ मथो॒ प्रति॑ष्ठित्यै॒ प्रति॑ष्ठित्या॒ अथो॒ रुच᳚म् । \newline
28. प्रति॑ष्ठित्या॒ इति॒ प्रति॑ - स्थि॒त्यै॒ । \newline
29. अथो॒ रुचꣳ॒॒ रुच॒ मथो॒ अथो॒ रुच॑ मे॒वैव रुच॒ मथो॒ अथो॒ रुच॑ मे॒व । \newline
30. अथो॒ इत्यथो᳚ । \newline
31. रुच॑ मे॒वैव रुचꣳ॒॒ रुच॑ मे॒वात्मन् ना॒त्मन् ने॒व रुचꣳ॒॒ रुच॑ मे॒वात्मन्न् । \newline
32. ए॒वात्मन् ना॒त्मन् ने॒वै वात्मन् ध॑त्ते धत्त आ॒त्मन् ने॒वै वात्मन् ध॑त्ते । \newline
33. आ॒त्मन् ध॑त्ते धत्त आ॒त्मन् ना॒त्मन् ध॑त्ते त्रिण॒व स्त्रि॑ण॒वो ध॑त्त आ॒त्मन् ना॒त्मन् ध॑त्ते त्रिण॒वः । \newline
34. ध॒त्ते॒ त्रि॒ण॒व स्त्रि॑ण॒वो ध॑त्ते धत्ते त्रिण॒वो भ॑वति भवति त्रिण॒वो ध॑त्ते धत्ते त्रिण॒वो भ॑वति । \newline
35. त्रि॒ण॒वो भ॑वति भवति त्रिण॒व स्त्रि॑ण॒वो भ॑वति॒ विजि॑त्यै॒ विजि॑त्यै भवति त्रिण॒व स्त्रि॑ण॒वो भ॑वति॒ विजि॑त्यै । \newline
36. त्रि॒ण॒व इति॑ त्रि - न॒वः । \newline
37. भ॒व॒ति॒ विजि॑त्यै॒ विजि॑त्यै भवति भवति॒ विजि॑त्यै पञ्चविꣳ॒॒शः प॑ञ्चविꣳ॒॒शो विजि॑त्यै भवति भवति॒ विजि॑त्यै पञ्चविꣳ॒॒शः । \newline
38. विजि॑त्यै पञ्चविꣳ॒॒शः प॑ञ्चविꣳ॒॒शो विजि॑त्यै॒ विजि॑त्यै पञ्चविꣳ॒॒शो᳚ ऽग्निष्टो॒मो᳚ ऽग्निष्टो॒मः प॑ञ्चविꣳ॒॒शो विजि॑त्यै॒ विजि॑त्यै पञ्चविꣳ॒॒शो᳚ ऽग्निष्टो॒मः । \newline
39. विजि॑त्या॒ इति॒ वि - जि॒त्यै॒ । \newline
40. प॒ञ्च॒विꣳ॒॒शो᳚ ऽग्निष्टो॒मो᳚ ऽग्निष्टो॒मः प॑ञ्चविꣳ॒॒शः प॑ञ्चविꣳ॒॒शो᳚ ऽग्निष्टो॒मो भ॑वति भव त्यग्निष्टो॒मः प॑ञ्चविꣳ॒॒शः प॑ञ्चविꣳ॒॒शो᳚ ऽग्निष्टो॒मो भ॑वति । \newline
41. प॒ञ्च॒विꣳ॒॒श इति॑ पञ्च - विꣳ॒॒शः । \newline
42. अ॒ग्नि॒ष्टो॒मो भ॑वति भव त्यग्निष्टो॒मो᳚ ऽग्निष्टो॒मो भ॑वति प्र॒जाप॑तेः प्र॒जाप॑तेर् भव त्यग्निष्टो॒मो᳚ ऽग्निष्टो॒मो भ॑वति प्र॒जाप॑तेः । \newline
43. अ॒ग्नि॒ष्टो॒म इत्य॑ग्नि - स्तो॒मः । \newline
44. भ॒व॒ति॒ प्र॒जाप॑तेः प्र॒जाप॑तेर् भवति भवति प्र॒जाप॑ते॒ राप्त्या॒ आप्त्यै᳚ प्र॒जाप॑तेर् भवति भवति प्र॒जाप॑ते॒ राप्त्यै᳚ । \newline
45. प्र॒जाप॑ते॒ राप्त्या॒ आप्त्यै᳚ प्र॒जाप॑तेः प्र॒जाप॑ते॒ राप्त्यै॑ महाव्र॒तवा᳚न् महाव्र॒तवा॒ नाप्त्यै᳚ प्र॒जाप॑तेः प्र॒जाप॑ते॒ राप्त्यै॑ महाव्र॒तवान्॑ । \newline
46. प्र॒जाप॑ते॒रिति॑ प्र॒जा - प॒तेः॒ । \newline
47. आप्त्यै॑ महाव्र॒तवा᳚न् महाव्र॒तवा॒ नाप्त्या॒ आप्त्यै॑ महाव्र॒तवा॑ न॒न्नाद्य॑स्या॒ न्नाद्य॑स्य महाव्र॒तवा॒ नाप्त्या॒ आप्त्यै॑ महाव्र॒तवा॑ न॒न्नाद्य॑स्य । \newline
48. म॒हा॒व्र॒तवा॑ न॒न्नाद्य॑स्या॒ न्नाद्य॑स्य महाव्र॒तवा᳚न् महाव्र॒तवा॑ न॒न्नाद्य॒स्या व॑रुद्ध्या॒ अव॑रुद्ध्या अ॒न्नाद्य॑स्य महाव्र॒तवा᳚न् महाव्र॒तवा॑ न॒न्नाद्य॒स्या व॑रुद्ध्यै । \newline
49. म॒हा॒व्र॒तवा॒निति॑ महाव्र॒त - वा॒न् । \newline
50. अ॒न्नाद्य॒स्या व॑रुद्ध्या॒ अव॑रुद्ध्या अ॒न्नाद्य॑स्या॒ न्नाद्य॒स्या व॑रुद्ध्यै विश्व॒जिद् वि॑श्व॒जि दव॑रुद्ध्या अ॒न्नाद्य॑स्या॒ न्नाद्य॒स्या व॑रुद्ध्यै विश्व॒जित् । \newline
51. अ॒न्नाद्य॒स्येत्य॑न्न - अद्य॑स्य । \newline
52. अव॑रुद्ध्यै विश्व॒जिद् वि॑श्व॒जि दव॑रुद्ध्या॒ अव॑रुद्ध्यै विश्व॒जिथ् सर्व॑पृष्ठः॒ सर्व॑पृष्ठो विश्व॒जि दव॑रुद्ध्या॒ अव॑रुद्ध्यै विश्व॒जिथ् सर्व॑पृष्ठः । \newline
53. अव॑रुद्ध्या॒ इत्यव॑ - रु॒द्ध्यै॒ । \newline
54. वि॒श्व॒जिथ् सर्व॑पृष्ठः॒ सर्व॑पृष्ठो विश्व॒जिद् वि॑श्व॒जिथ् सर्व॑पृष्ठो ऽतिरा॒त्रो॑ ऽतिरा॒त्रः सर्व॑पृष्ठो विश्व॒जिद् वि॑श्व॒जिथ् सर्व॑पृष्ठो ऽतिरा॒त्रः । \newline
55. वि॒श्व॒जिदिति॑ विश्व - जित् । \newline
56. सर्व॑पृष्ठो ऽतिरा॒त्रो॑ ऽतिरा॒त्रः सर्व॑पृष्ठः॒ सर्व॑पृष्ठो ऽतिरा॒त्रो भ॑वति भव त्यतिरा॒त्रः सर्व॑पृष्ठः॒ सर्व॑पृष्ठो ऽतिरा॒त्रो भ॑वति । \newline
57. सर्व॑पृष्ठ॒ इति॒ सर्व॑ - पृ॒ष्ठः॒ । \newline
58. अ॒ति॒रा॒त्रो भ॑वति भव त्यतिरा॒त्रो॑ ऽतिरा॒त्रो भ॑वति॒ सर्व॑स्य॒ सर्व॑स्य भव त्यतिरा॒त्रो॑ ऽतिरा॒त्रो भ॑वति॒ सर्व॑स्य । \newline
59. अ॒ति॒रा॒त्र इत्य॑ति - रा॒त्रः । \newline
60. भ॒व॒ति॒ सर्व॑स्य॒ सर्व॑स्य भवति भवति॒ सर्व॑स्या॒ भिजि॑त्या अ॒भिजि॑त्यै॒ सर्व॑स्य भवति भवति॒ सर्व॑स्या॒ भिजि॑त्यै । \newline
61. सर्व॑स्या॒ भिजि॑त्या अ॒भिजि॑त्यै॒ सर्व॑स्य॒ सर्व॑स्या॒ भिजि॑त्यै॒ यद् यद॒भिजि॑त्यै॒ सर्व॑स्य॒ सर्व॑स्या॒ भिजि॑त्यै॒ यत् । \newline
62. अ॒भिजि॑त्यै॒ यद् यद॒भिजि॑त्या अ॒भिजि॑त्यै॒ यत् प्र॒त्यक्ष॑म् प्र॒त्यक्षं॒ ॅयद॒भिजि॑त्या अ॒भिजि॑त्यै॒ यत् प्र॒त्यक्ष᳚म् । \newline
63. अ॒भिजि॑त्या॒ इत्य॒भि - जि॒त्यै॒ । \newline
64. यत् प्र॒त्यक्ष॑म् प्र॒त्यक्षं॒ ॅयद् यत् प्र॒त्यक्ष॒म् पूर्वे॑षु॒ पूर्वे॑षु प्र॒त्यक्षं॒ ॅयद् यत् प्र॒त्यक्ष॒म् पूर्वे॑षु । \newline
65. प्र॒त्यक्ष॒म् पूर्वे॑षु॒ पूर्वे॑षु प्र॒त्यक्ष॑म् प्र॒त्यक्ष॒म् पूर्वे॒ ष्वह॒ स्स्वह॑स्सु॒ पूर्वे॑षु प्र॒त्यक्ष॑म् प्र॒त्यक्ष॒म् पूर्वे॒ ष्वह॑स्सु । \newline
66. प्र॒त्यक्ष॒मिति॑ प्रति - अक्ष᳚म् । \newline
67. पूर्वे॒ ष्वह॒ स्स्वह॑स्सु॒ पूर्वे॑षु॒ पूर्वे॒ ष्वह॑स्सु पृ॒ष्ठानि॑ पृ॒ष्ठा न्यह॑स्सु॒ पूर्वे॑षु॒ पूर्वे॒ ष्वह॑स्सु पृ॒ष्ठानि॑ । \newline
68. अह॑स्सु पृ॒ष्ठानि॑ पृ॒ष्ठा न्यह॒ स्स्वह॑स्सु पृ॒ष्ठा न्यु॑पे॒यु रु॑पे॒युः पृ॒ष्ठा न्यह॒ स्स्वह॑स्सु पृ॒ष्ठा न्यु॑पे॒युः । \newline
69. अह॒स्स्वित्यहः॑ - सु॒ । \newline
70. पृ॒ष्ठा न्यु॑पे॒यु रु॑पे॒युः पृ॒ष्ठानि॑ पृ॒ष्ठा न्यु॑पे॒युः प्र॒त्यक्ष॑म् प्र॒त्यक्ष॑ मुपे॒युः पृ॒ष्ठानि॑ पृ॒ष्ठा न्यु॑पे॒युः प्र॒त्यक्ष᳚म् । \newline
71. उ॒पे॒युः प्र॒त्यक्ष॑म् प्र॒त्यक्ष॑ मुपे॒यु रु॑पे॒युः प्र॒त्यक्षं॑ ॅविश्व॒जिति॑ विश्व॒जिति॑ प्र॒त्यक्ष॑ मुपे॒यु रु॑पे॒युः प्र॒त्यक्षं॑ ॅविश्व॒जिति॑ । \newline
72. उ॒पे॒युरित्यु॑प - इ॒युः । \newline
73. प्र॒त्यक्षं॑ ॅविश्व॒जिति॑ विश्व॒जिति॑ प्र॒त्यक्ष॑म् प्र॒त्यक्षं॑ ॅविश्व॒जिति॒ यथा॒ यथा॑ विश्व॒जिति॑ प्र॒त्यक्ष॑म् प्र॒त्यक्षं॑ ॅविश्व॒जिति॒ यथा᳚ । \newline
74. प्र॒त्यक्ष॒मिति॑ प्रति - अक्ष᳚म् । \newline
\pagebreak
\markright{ TS 7.2.2.3  \hfill https://www.vedavms.in \hfill}

\section{ TS 7.2.2.3 }

\textbf{TS 7.2.2.3 } \newline
\textbf{Samhita Paata} \newline

ॅविश्व॒जिति॒ यथा॑ दु॒ग्धा-मु॑प॒सीद॑त्ये॒वमु॑त्त॒म-महः॑ स्या॒न्नैक॑रा॒त्रश्च॒न स्या᳚द् बृहद्-रथन्त॒रे पूर्वे॒ष्वह॒स्सूप॑ यन्ती॒यं ॅवाव र॑थन्त॒रम॒सौ बृ॒हदा॒भ्यामे॒व न य॒न्त्यथो॑ अ॒नयो॑रे॒व प्रति॑ तिष्ठन्ति॒ यत् प्र॒त्यक्षं॑ ॅविश्व॒जिति॑ पृ॒ष्ठान्यु॑प॒यन्ति॒ यथा॒ प्रत्तां᳚ दु॒हे ता॒दृगे॒व तत् ॥ \newline

\textbf{Pada Paata} \newline

वि॒श्व॒जितीति॑ विश्व - जिति॑ । यथा᳚ । दु॒ग्धाम् । उ॒प॒सीद॒तीत्यु॑प- सीद॑ति । ए॒वम् । उ॒त्त॒ममित्यु॑त् - त॒मम् । अहः॑ । स्या॒त् । न । ए॒क॒रा॒त्र इत्ये॑क - रा॒त्रः । च॒न । स्या॒त् । बृ॒ह॒द्र॒थ॒न्त॒रे इति॑ बृहत्-र॒थ॒न्त॒रे । पूर्वे॑षु । अह॒स्स्वित्यहः॑ - सु॒ । उपेति॑ । य॒न्ति॒ । इ॒यम् । वाव । र॒थ॒न्त॒रमिति॑ रथं - त॒रम् । अ॒सौ । बृ॒हत् । आ॒भ्याम् । ए॒व । न । य॒न्ति॒ । अथो॒ इति॑ । अ॒नयोः᳚ । ए॒व । प्रतीति॑ । ति॒ष्ठ॒न्ति॒ । यत् । प्र॒त्यक्ष॒मिति॑ प्रति - अक्ष᳚म् । वि॒श्व॒जितीति॑ विश्व - जिति॑ । पृ॒ष्ठानि॑ । उ॒प॒यन्तीत्यु॑प - यन्ति॑ । यथा᳚ । प्रत्ता᳚म् । दु॒हे । ता॒दृक् । ए॒व । तत् ॥  \newline


\textbf{Krama Paata} \newline

वि॒श्व॒जिति॒ यथा᳚ । वि॒श्व॒जितीति॑ विश्व - जिति॑ । यथा॑ दु॒ग्धाम् । दु॒ग्धामु॑प॒सीद॑ति । उ॒प॒सीद॑त्ये॒वम् । उ॒प॒सीद॒तीत्यु॑प - सीद॑ति । ए॒वमु॑त्त॒मम् । उ॒त्त॒ममहः॑ । उ॒त्त॒ममित्यु॑त् - त॒मम् । अहः॑ स्यात् । स्या॒न् न । नैक॑रा॒त्रः । ए॒क॒रा॒त्रश्च॒न । ए॒क॒रा॒त्र इत्ये॑क - रा॒त्रः । च॒न स्या᳚त् । स्या॒द् बृ॒ह॒द्‍र॒थ॒न्त॒रे । बृ॒ह॒द्‍र॒थ॒न्त॒रे पूर्वे॑षु । बृ॒ह॒द्‍र॒थ॒न्त॒रे इति॑ बृहत् - र॒थ॒न्त॒रे । पूर्वे॒ष्वह॑स्सु । अह॒स्सूप॑ । अह॒स्स्वित्यहः॑ - सु॒ । उप॑ यन्ति । य॒न्ती॒यम् । इ॒यम् ॅवाव । वाव र॑थन्त॒रम् । र॒थ॒न्त॒रम॒सौ । र॒थ॒न्त॒रमिति॑ रथम् - त॒रम् । अ॒सौ बृ॒हत् । बृ॒हदा॒भ्याम् । आ॒भ्यामे॒व । ए॒व न । न य॑न्ति । य॒न्त्यथो᳚ । अथो॑ अ॒नयोः᳚ । अथो॒ इत्यथो᳚ । अ॒नयो॑रे॒व । ए॒व प्रति॑ । प्रति॑ तिष्ठन्ति । ति॒ष्ठ॒न्ति॒ यत् । यत् प्र॒त्यक्ष᳚म् । प्र॒त्यक्ष॑म् ॅविश्व॒जिति॑ । प्र॒त्यक्ष॒मिति॑ प्रति - अक्ष᳚म् । वि॒श्व॒जिति॑ पृ॒ष्ठानि॑ । वि॒श्व॒जितीति॑ विश्व - जिति॑ । पृ॒ष्ठान्यु॑प॒यन्ति॑ । उ॒प॒यन्ति॒ यथा᳚ । उ॒प॒यन्तीत्यु॑प - यन्ति॑ । यथा॒ प्रत्ता᳚म् । प्रत्ता᳚म् दु॒हे । दु॒हे ता॒दृक् । ता॒दृगे॒व । ए॒व तत् । तदिति॒ तत् । \newline

\textbf{Jatai Paata} \newline

1. वि॒श्व॒जिति॒ यथा॒ यथा॑ विश्व॒जिति॑ विश्व॒जिति॒ यथा᳚ । \newline
2. वि॒श्व॒जितीति॑ विश्व - जिति॑ । \newline
3. यथा॑ दु॒ग्धाम् दु॒ग्धां ॅयथा॒ यथा॑ दु॒ग्धाम् । \newline
4. दु॒ग्धा मु॑प॒सीद॑ त्युप॒सीद॑ति दु॒ग्धाम् दु॒ग्धा मु॑प॒सीद॑ति । \newline
5. उ॒प॒सीद॑ त्ये॒व मे॒व मु॑प॒सीद॑ त्युप॒सीद॑ त्ये॒वम् । \newline
6. उ॒प॒सीद॒तीत्यु॑प - सीद॑ति । \newline
7. ए॒व मु॑त्त॒म मु॑त्त॒म मे॒व मे॒व मु॑त्त॒मम् । \newline
8. उ॒त्त॒म मह॒ रह॑ रुत्त॒म मु॑त्त॒म महः॑ । \newline
9. उ॒त्त॒ममित्यु॑त् - त॒मम् । \newline
10. अहः॑ स्याथ् स्या॒ दह॒ रहः॑ स्यात् । \newline
11. स्या॒न् न न स्या᳚थ् स्या॒न् न । \newline
12. नैक॑रा॒त्र ए॑करा॒त्रो न नैक॑रा॒त्रः । \newline
13. ए॒क॒रा॒त्र श्च॒न च॒नैक॑रा॒त्र ए॑करा॒त्र श्च॒न । \newline
14. ए॒क॒रा॒त्र इत्ये॑क - रा॒त्रः । \newline
15. च॒न स्या᳚थ् स्याच् च॒न च॒न स्या᳚त् । \newline
16. स्या॒द् बृ॒ह॒द्र॒थ॒न्त॒रे बृ॑हद्रथन्त॒रे स्या᳚थ् स्याद् बृहद्रथन्त॒रे । \newline
17. बृ॒ह॒द्र॒थ॒न्त॒रे पूर्वे॑षु॒ पूर्वे॑षु बृहद्रथन्त॒रे बृ॑हद्रथन्त॒रे पूर्वे॑षु । \newline
18. बृ॒ह॒द्र॒थ॒न्त॒रे इति॑ बृहत् - र॒थ॒न्त॒रे । \newline
19. पूर्वे॒ ष्वह॒ स्स्वह॑स्सु॒ पूर्वे॑षु॒ पूर्वे॒ ष्वह॑स्सु । \newline
20. अह॒ स्सूपोपा ह॒ स्स्वह॒ स्सूप॑ । \newline
21. अह॒स्स्वित्यहः॑ - सु॒ । \newline
22. उप॑ यन्ति य॒ न्त्युपोप॑ यन्ति । \newline
23. य॒न्ती॒य मि॒यं ॅय॑न्ति यन्ती॒यम् । \newline
24. इ॒यं ॅवाव वावे य मि॒यं ॅवाव । \newline
25. वाव र॑थन्त॒रꣳ र॑थन्त॒रं ॅवाव वाव र॑थन्त॒रम् । \newline
26. र॒थ॒न्त॒र म॒सा व॒सौ र॑थन्त॒रꣳ र॑थन्त॒र म॒सौ । \newline
27. र॒थ॒न्त॒रमिति॑ रथं - त॒रम् । \newline
28. अ॒सौ बृ॒हद् बृ॒ह द॒सा व॒सौ बृ॒हत् । \newline
29. बृ॒ह दा॒भ्या मा॒भ्याम् बृ॒हद् बृ॒ह दा॒भ्याम् । \newline
30. आ॒भ्या मे॒वै वाभ्या मा॒भ्या मे॒व । \newline
31. ए॒व न नैवैव न । \newline
32. न य॑न्ति यन्ति॒ न न य॑न्ति । \newline
33. य॒न्त्यथो॒ अथो॑ यन्ति य॒न्त्यथो᳚ । \newline
34. अथो॑ अ॒नयो॑ र॒नयो॒ रथो॒ अथो॑ अ॒नयोः᳚ । \newline
35. अथो॒ इत्यथो᳚ । \newline
36. अ॒नयो॑ रे॒वै वानयो॑ र॒नयो॑ रे॒व । \newline
37. ए॒व प्रति॒ प्रत्ये॒ वैव प्रति॑ । \newline
38. प्रति॑ तिष्ठन्ति तिष्ठन्ति॒ प्रति॒ प्रति॑ तिष्ठन्ति । \newline
39. ति॒ष्ठ॒न्ति॒ यद् यत् ति॑ष्ठन्ति तिष्ठन्ति॒ यत् । \newline
40. यत् प्र॒त्यक्ष॑म् प्र॒त्यक्षं॒ ॅयद् यत् प्र॒त्यक्ष᳚म् । \newline
41. प्र॒त्यक्षं॑ ॅविश्व॒जिति॑ विश्व॒जिति॑ प्र॒त्यक्ष॑म् प्र॒त्यक्षं॑ ॅविश्व॒जिति॑ । \newline
42. प्र॒त्यक्ष॒मिति॑ प्रति - अक्ष᳚म् । \newline
43. वि॒श्व॒जिति॑ पृ॒ष्ठानि॑ पृ॒ष्ठानि॑ विश्व॒जिति॑ विश्व॒जिति॑ पृ॒ष्ठानि॑ । \newline
44. वि॒श्व॒जितीति॑ विश्व - जिति॑ । \newline
45. पृ॒ष्ठा न्यु॑प॒य न्त्यु॑प॒यन्ति॑ पृ॒ष्ठानि॑ पृ॒ष्ठा न्यु॑प॒यन्ति॑ । \newline
46. उ॒प॒यन्ति॒ यथा॒ यथो॑प॒य न्त्यु॑प॒यन्ति॒ यथा᳚ । \newline
47. उ॒प॒यन्तीत्यु॑प - यन्ति॑ । \newline
48. यथा॒ प्रत्ता॒म् प्रत्तां॒ ॅयथा॒ यथा॒ प्रत्ता᳚म् । \newline
49. प्रत्ता᳚म् दु॒हे दु॒हे प्रत्ता॒म् प्रत्ता᳚म् दु॒हे । \newline
50. दु॒हे ता॒दृक् ता॒दृग् दु॒हे दु॒हे ता॒दृक् । \newline
51. ता॒दृ गे॒वैव ता॒दृक् ता॒दृ गे॒व । \newline
52. ए॒व तत् तदे॒ वैव तत् । \newline
53. तदिति॒ तत् । \newline

\textbf{Ghana Paata } \newline

1. वि॒श्व॒जिति॒ यथा॒ यथा॑ विश्व॒जिति॑ विश्व॒जिति॒ यथा॑ दु॒ग्धाम् दु॒ग्धां ॅयथा॑ विश्व॒जिति॑ विश्व॒जिति॒ यथा॑ दु॒ग्धाम् । \newline
2. वि॒श्व॒जितीति॑ विश्व - जिति॑ । \newline
3. यथा॑ दु॒ग्धाम् दु॒ग्धां ॅयथा॒ यथा॑ दु॒ग्धा मु॑प॒सीद॑ त्युप॒सीद॑ति दु॒ग्धां ॅयथा॒ यथा॑ दु॒ग्धा मु॑प॒सीद॑ति । \newline
4. दु॒ग्धा मु॑प॒सीद॑ त्युप॒सीद॑ति दु॒ग्धाम् दु॒ग्धा मु॑प॒सीद॑ त्ये॒व मे॒व मु॑प॒सीद॑ति दु॒ग्धाम् दु॒ग्धा मु॑प॒सीद॑ त्ये॒वम् । \newline
5. उ॒प॒सीद॑ त्ये॒व मे॒व मु॑प॒सीद॑ त्युप॒सीद॑ त्ये॒व मु॑त्त॒म मु॑त्त॒म मे॒व मु॑प॒सीद॑ त्युप॒सीद॑ त्ये॒व मु॑त्त॒मम् । \newline
6. उ॒प॒सीद॒तीत्यु॑प - सीद॑ति । \newline
7. ए॒व मु॑त्त॒म मु॑त्त॒म मे॒व मे॒व मु॑त्त॒म मह॒ रह॑ रुत्त॒म मे॒व मे॒व मु॑त्त॒म महः॑ । \newline
8. उ॒त्त॒म मह॒ रह॑ रुत्त॒म मु॑त्त॒म महः॑ स्याथ् स्या॒ दह॑ रुत्त॒म मु॑त्त॒म महः॑ स्यात् । \newline
9. उ॒त्त॒ममित्यु॑त् - त॒मम् । \newline
10. अहः॑ स्याथ् स्या॒ दह॒ रहः॑ स्या॒न् न न स्या॒ दह॒ रहः॑ स्या॒न् न । \newline
11. स्या॒न् न न स्या᳚थ् स्या॒न् नैक॑रा॒त्र ए॑करा॒त्रो न स्या᳚थ् स्या॒न् नैक॑रा॒त्रः । \newline
12. नैक॑रा॒त्र ए॑करा॒त्रो न नैक॑रा॒त्र श्च॒न च॒नैक॑रा॒त्रो न नैक॑रा॒त्र श्च॒न । \newline
13. ए॒क॒रा॒त्र श्च॒न च॒नैक॑रा॒त्र ए॑करा॒त्र श्च॒न स्या᳚थ् स्याच् च॒नैक॑रा॒त्र ए॑करा॒त्र श्च॒न स्या᳚त् । \newline
14. ए॒क॒रा॒त्र इत्ये॑क - रा॒त्रः । \newline
15. च॒न स्या᳚थ् स्याच् च॒न च॒न स्या᳚द् बृहद्रथन्त॒रे बृ॑हद्रथन्त॒रे स्या᳚च् च॒न च॒न स्या᳚द् बृहद्रथन्त॒रे । \newline
16. स्या॒द् बृ॒ह॒द्र॒थ॒न्त॒रे बृ॑हद्रथन्त॒रे स्या᳚थ् स्याद् बृहद्रथन्त॒रे पूर्वे॑षु॒ पूर्वे॑षु बृहद्रथन्त॒रे स्या᳚थ् स्याद् बृहद्रथन्त॒रे पूर्वे॑षु । \newline
17. बृ॒ह॒द्र॒थ॒न्त॒रे पूर्वे॑षु॒ पूर्वे॑षु बृहद्रथन्त॒रे बृ॑हद्रथन्त॒रे पूर्वे॒ ष्वह॒ स्स्वह॑स्सु॒ पूर्वे॑षु बृहद्रथन्त॒रे बृ॑हद्रथन्त॒रे पूर्वे॒ ष्वह॑स्सु । \newline
18. बृ॒ह॒द्र॒थ॒न्त॒रे इति॑ बृहत् - र॒थ॒न्त॒रे । \newline
19. पूर्वे॒ ष्वह॒ स्स्वह॑स्सु॒ पूर्वे॑षु॒ पूर्वे॒ ष्वह॒स्सू पोपा ह॑स्सु॒ पूर्वे॑षु॒ पूर्वे॒ ष्वह॒स्सूप॑ । \newline
20. अह॒स्सूपोपा ह॒ स्स्वह॒ स्सूप॑ यन्ति य॒ न्त्युपाह॒ स्स्वह॒ स्सूप॑ यन्ति । \newline
21. अह॒स्स्वित्यहः॑ - सु॒ । \newline
22. उप॑ यन्ति य॒ न्त्युपोप॑ यन्ती॒य मि॒यं ॅय॒ न्त्युपोप॑ यन्ती॒यम् । \newline
23. य॒न्ती॒य मि॒यं ॅय॑न्ति यन्ती॒यं ॅवाव वावेयं ॅय॑न्ति यन्ती॒यं ॅवाव । \newline
24. इ॒यं ॅवाव वावेय मि॒यं ॅवाव र॑थन्त॒रꣳ र॑थन्त॒रं ॅवावेय मि॒यं ॅवाव र॑थन्त॒रम् । \newline
25. वाव र॑थन्त॒रꣳ र॑थन्त॒रं ॅवाव वाव र॑थन्त॒र म॒सा व॒सौ र॑थन्त॒रं ॅवाव वाव र॑थन्त॒र म॒सौ । \newline
26. र॒थ॒न्त॒र म॒सा व॒सौ र॑थन्त॒रꣳ र॑थन्त॒र म॒सौ बृ॒हद् बृ॒ह द॒सौ र॑थन्त॒रꣳ र॑थन्त॒र म॒सौ बृ॒हत् । \newline
27. र॒थ॒न्त॒रमिति॑ रथं - त॒रम् । \newline
28. अ॒सौ बृ॒हद् बृ॒ह द॒सा व॒सौ बृ॒ह दा॒भ्या मा॒भ्याम् बृ॒ह द॒सा व॒सौ बृ॒ह दा॒भ्याम् । \newline
29. बृ॒ह दा॒भ्या मा॒भ्याम् बृ॒हद् बृ॒ह दा॒भ्या मे॒वै वाभ्याम् बृ॒हद् बृ॒ह दा॒भ्या मे॒व । \newline
30. आ॒भ्या मे॒वै वाभ्या मा॒भ्या मे॒व न नैवाभ्या मा॒भ्या मे॒व न । \newline
31. ए॒व न नैवैव न य॑न्ति यन्ति॒ नैवैव न य॑न्ति । \newline
32. न य॑न्ति यन्ति॒ न न य॒ न्त्यथो॒ अथो॑ यन्ति॒ न न य॒ न्त्यथो᳚ । \newline
33. य॒न्त्यथो॒ अथो॑ यन्ति य॒न्त्यथो॑ अ॒नयो॑ र॒नयो॒ रथो॑ यन्ति य॒न्त्यथो॑ अ॒नयोः᳚ । \newline
34. अथो॑ अ॒नयो॑ र॒नयो॒ रथो॒ अथो॑ अ॒नयो॑ रे॒वै वानयो॒ रथो॒ अथो॑ अ॒नयो॑ रे॒व । \newline
35. अथो॒ इत्यथो᳚ । \newline
36. अ॒नयो॑ रे॒वै वानयो॑ र॒नयो॑ रे॒व प्रति॒ प्रत्ये॒वा नयो॑ र॒नयो॑ रे॒व प्रति॑ । \newline
37. ए॒व प्रति॒ प्रत्ये॒ वैव प्रति॑ तिष्ठन्ति तिष्ठन्ति॒ प्रत्ये॒ वैव प्रति॑ तिष्ठन्ति । \newline
38. प्रति॑ तिष्ठन्ति तिष्ठन्ति॒ प्रति॒ प्रति॑ तिष्ठन्ति॒ यद् यत् ति॑ष्ठन्ति॒ प्रति॒ प्रति॑ तिष्ठन्ति॒ यत् । \newline
39. ति॒ष्ठ॒न्ति॒ यद् यत् ति॑ष्ठन्ति तिष्ठन्ति॒ यत् प्र॒त्यक्ष॑म् प्र॒त्यक्षं॒ ॅयत् ति॑ष्ठन्ति तिष्ठन्ति॒ यत् प्र॒त्यक्ष᳚म् । \newline
40. यत् प्र॒त्यक्ष॑म् प्र॒त्यक्षं॒ ॅयद् यत् प्र॒त्यक्षं॑ ॅविश्व॒जिति॑ विश्व॒जिति॑ प्र॒त्यक्षं॒ ॅयद् यत् प्र॒त्यक्षं॑ ॅविश्व॒जिति॑ । \newline
41. प्र॒त्यक्षं॑ ॅविश्व॒जिति॑ विश्व॒जिति॑ प्र॒त्यक्ष॑म् प्र॒त्यक्षं॑ ॅविश्व॒जिति॑ पृ॒ष्ठानि॑ पृ॒ष्ठानि॑ विश्व॒जिति॑ प्र॒त्यक्ष॑म् प्र॒त्यक्षं॑ ॅविश्व॒जिति॑ पृ॒ष्ठानि॑ । \newline
42. प्र॒त्यक्ष॒मिति॑ प्रति - अक्ष᳚म् । \newline
43. वि॒श्व॒जिति॑ पृ॒ष्ठानि॑ पृ॒ष्ठानि॑ विश्व॒जिति॑ विश्व॒जिति॑ पृ॒ष्ठा न्यु॑प॒य न्त्यु॑प॒यन्ति॑ पृ॒ष्ठानि॑ विश्व॒जिति॑ विश्व॒जिति॑ पृ॒ष्ठा न्यु॑प॒यन्ति॑ । \newline
44. वि॒श्व॒जितीति॑ विश्व - जिति॑ । \newline
45. पृ॒ष्ठा न्यु॑प॒य न्त्यु॑प॒यन्ति॑ पृ॒ष्ठानि॑ पृ॒ष्ठा न्यु॑प॒यन्ति॒ यथा॒ यथो॑ प॒यन्ति॑ पृ॒ष्ठानि॑ पृ॒ष्ठा न्यु॑प॒यन्ति॒ यथा᳚ । \newline
46. उ॒प॒यन्ति॒ यथा॒ यथो॑ प॒य न्त्यु॑प॒यन्ति॒ यथा॒ प्रत्ता॒म् प्रत्तां॒ ॅयथो॑ प॒य न्त्यु॑प॒यन्ति॒ यथा॒ प्रत्ता᳚म् । \newline
47. उ॒प॒यन्तीत्यु॑प - यन्ति॑ । \newline
48. यथा॒ प्रत्ता॒म् प्रत्तां॒ ॅयथा॒ यथा॒ प्रत्ता᳚म् दु॒हे दु॒हे प्रत्तां॒ ॅयथा॒ यथा॒ प्रत्ता᳚म् दु॒हे । \newline
49. प्रत्ता᳚म् दु॒हे दु॒हे प्रत्ता॒म् प्रत्ता᳚म् दु॒हे ता॒दृक् ता॒दृग् दु॒हे प्रत्ता॒म् प्रत्ता᳚म् दु॒हे ता॒दृक् । \newline
50. दु॒हे ता॒दृक् ता॒दृग् दु॒हे दु॒हे ता॒दृ गे॒वैव ता॒दृग् दु॒हे दु॒हे ता॒दृ गे॒व । \newline
51. ता॒दृ गे॒वैव ता॒दृक् ता॒दृ गे॒व तत् तदे॒व ता॒दृक् ता॒दृ गे॒व तत् । \newline
52. ए॒व तत् तदे॒वैव तत् । \newline
53. तदिति॒ तत् । \newline
\pagebreak
\markright{ TS 7.2.3.1  \hfill https://www.vedavms.in \hfill}

\section{ TS 7.2.3.1 }

\textbf{TS 7.2.3.1 } \newline
\textbf{Samhita Paata} \newline

बृह॒स्पति॑रकामयत ब्रह्मवर्च॒सी स्या॒मिति॒ स ए॒त-म॑ष्टरा॒त्र-म॑पश्य॒त् तमाऽह॑र॒त् तेना॑यजत॒ ततो॒ वै स ब्र॑ह्मवर्च॒स्य॑भव॒द्य ए॒वं ॅवि॒द्वान॑ष्टरा॒त्रेण॒ यज॑ते ब्रह्मवर्च॒स्ये॑व भ॑वत्यष्टरा॒त्रो भ॑वत्य॒ष्टाक्ष॑रा गाय॒त्री गा॑य॒त्री ब्र॑ह्मवर्च॒सं गा॑यत्रि॒यैव ब्र॑ह्मवर्च॒समव॑ रुन्धेऽष्टरा॒त्रो भ॑वति॒ चत॑स्रो॒ वै दिश॒श्चत॑स्रो ऽवान्तरदि॒शा दि॒ग्भ्य ए॒व ब्र॑ह्मवर्च॒समव॑ रुन्धे- [  ] \newline

\textbf{Pada Paata} \newline

बृह॒स्पतिः॑ । अ॒का॒म॒य॒त॒ । ब्र॒ह्म॒व॒र्च॒सीति॑ ब्रह्म - व॒र्च॒सी । स्या॒म् । इति॑ । सः । ए॒तम् । अ॒ष्ट॒रा॒त्रमित्य॑ष्ट-रा॒त्रम् । अ॒प॒श्य॒त् । तम् । एति॑ । अ॒ह॒र॒त् । तेन॑ । अ॒य॒ज॒त॒ । ततः॑ । वै । सः । ब्र॒ह्म॒व॒र्च॒सीति॑ ब्रह्म-व॒र्च॒सी । अ॒भ॒व॒त् । यः । ए॒वम् । वि॒द्वान् । अ॒ष्ट॒रा॒त्रेणेत्य॑ष्ट - रा॒त्रेण॑ । यज॑ते । ब्र॒ह्म॒व॒र्च॒सीति॑ ब्रह्म - व॒र्च॒सी । ए॒व । भ॒व॒ति॒ । अ॒ष्ट॒रा॒त्र इत्य॑ष्ट - रा॒त्रः । भ॒व॒ति॒ । अ॒ष्टाक्ष॒रेत्य॒ष्टा - अ॒क्ष॒रा॒ । गा॒य॒त्री । गा॒य॒त्री । ब्र॒ह्म॒व॒र्च॒समिति॑ ब्रह्म - व॒र्च॒सम् । गा॒य॒त्रि॒या । ए॒व । ब्र॒ह्म॒व॒र्च॒समिति॑ ब्रह्म - व॒र्च॒सम् । अवेति॑ । रु॒न्धे॒ । अ॒ष्ट॒रा॒त्र इत्य॑ष्ट-रा॒त्रः । भ॒व॒ति॒ । चत॑स्रः । वै । दिशः॑ । चत॑स्रः । अ॒वा॒न्त॒र॒दि॒शा इत्य॑वान्तर-दि॒शाः । दि॒ग्भ्य इति॑ दिक् - भ्यः । ए॒व । ब्र॒ह्म॒व॒र्च॒समिति॑ ब्रह्म - व॒र्च॒सम् । अवेति॑ । रु॒न्धे॒ ।  \newline


\textbf{Krama Paata} \newline

बृह॒स्पति॑रकामयत । अ॒का॒म॒य॒त॒ ब्र॒ह्म॒व॒र्च॒सी । ब्र॒ह्म॒व॒र्च॒सी स्या᳚म् । ब्र॒ह्म॒व॒र्च॒सीति॑ ब्रह्म - व॒र्च॒सी । स्या॒मिति॑ । इति॒ सः । स ए॒तम् । ए॒तम॑ष्टरा॒त्रम् । अ॒ष्ट॒रा॒त्रम॑पश्यत् । अ॒ष्ट॒रा॒त्रमित्य॑ष्ट - रा॒त्रम् । अ॒प॒श्य॒त् तम् । तमा । आऽह॑रत् । अ॒ह॒र॒त् तेन॑ । तेना॑यजत । अ॒य॒ज॒त॒ ततः॑ । ततो॒ वै । वै सः । स ब्र॑ह्मवर्च॒सी । ब्र॒ह्म॒व॒र्च॒स्य॑भवत् । ब्र॒ह्म॒व॒र्च॒सीति॑ ब्रह्म - व॒र्च॒सी । अ॒भ॒व॒द् यः । य ए॒वम् । ए॒वम् ॅवि॒द्वान् । वि॒द्वान॑ष्टरा॒त्रेण॑ । अ॒ष्ट॒रा॒त्रेण॒ यज॑ते । अ॒ष्ट॒रा॒त्रेणेत्य॑ष्ट - रा॒त्रेण॑ । यज॑ते ब्रह्मवर्च॒सी । ब्र॒ह्म॒व॒र्च॒स्ये॑व । ब्र॒ह्म॒व॒र्च॒सीति॑ ब्रह्म - व॒र्च॒सी । ए॒व भ॑वति । भ॒व॒त्य॒ष्ट॒रा॒त्रः । अ॒ष्ट॒रा॒त्रो भ॑वति । अ॒ष्ट॒रा॒त्र इत्य॑ष्ट - रा॒त्रः । भ॒व॒त्य॒ष्टाक्ष॑रा । अ॒ष्टाक्ष॑रा गाय॒त्री । अ॒ष्टाक्ष॒रेत्य॒ष्टा - अ॒क्ष॒रा॒ । गा॒य॒त्री गा॑य॒त्री । गा॒य॒त्री ब्र॑ह्मवर्च॒सम् । ब्र॒ह्म॒व॒र्च॒सम् गा॑यत्रि॒या । ब्र॒ह्म॒व॒र्च॒समिति॑ ब्रह्म - व॒र्च॒सम् । गा॒य॒त्रि॒यैव । ए॒व ब्र॑ह्मवर्च॒सम् । ब्र॒ह्म॒व॒र्च॒समव॑ । ब्र॒ह्म॒व॒र्च॒समिति॑ ब्रह्म - व॒र्च॒सम् । अव॑ रुन्धे । रु॒न्धे॒ऽष्ट॒रा॒त्रः । अ॒ष्ट॒रा॒त्रो भ॑वति । अ॒ष्ट॒रा॒त्र इत्य॑ष्ट - रा॒त्रः । भ॒व॒ति॒ चत॑स्रः । चत॑स्रो॒ वै । वै दिशः॑ । दिश॒श्चत॑स्रः । चत॑स्रोऽवान्तरदि॒शाः । अ॒वा॒न्त॒र॒दि॒शा दि॒ग्भ्यः । अ॒वा॒न्त॒र॒दि॒शा इत्य॑वान्तर - दि॒शाः । दि॒ग्भ्य ए॒व । दि॒ग्भ्य इति॑ दिक् - भ्यः । ए॒व ब्र॑ह्मवर्च॒सम् । ब्र॒ह्म॒व॒र्च॒समव॑ । ब्र॒ह्म॒व॒र्च॒समिति॑ ब्रह्म - व॒र्च॒सम् । अव॑ रुन्धे । रु॒न्धे॒ त्रि॒वृत् \newline

\textbf{Jatai Paata} \newline

1. बृह॒स्पति॑ रकामयता कामयत॒ बृह॒स्पति॒र् बृह॒स्पति॑ रकामयत । \newline
2. अ॒का॒म॒य॒त॒ ब्र॒ह्म॒व॒र्च॒सी ब्र॑ह्मवर्च॒स्य॑ कामयता कामयत ब्रह्मवर्च॒सी । \newline
3. ब्र॒ह्म॒व॒र्च॒सी स्याꣳ॑ स्याम् ब्रह्मवर्च॒सी ब्र॑ह्मवर्च॒सी स्या᳚म् । \newline
4. ब्र॒ह्म॒व॒र्च॒सीति॑ ब्रह्म - व॒र्च॒सी । \newline
5. स्या॒ मितीति॑ स्याꣳ स्या॒ मिति॑ । \newline
6. इति॒ स स इतीति॒ सः । \newline
7. स ए॒त मे॒तꣳ स स ए॒तम् । \newline
8. ए॒त म॑ष्टरा॒त्र म॑ष्टरा॒त्र मे॒त मे॒त म॑ष्टरा॒त्रम् । \newline
9. अ॒ष्ट॒रा॒त्र म॑पश्य दपश्य दष्टरा॒त्र म॑ष्टरा॒त्र म॑पश्यत् । \newline
10. अ॒ष्ट॒रा॒त्रमित्य॑ष्ट - रा॒त्रम् । \newline
11. अ॒प॒श्य॒त् तम् त म॑पश्य दपश्य॒त् तम् । \newline
12. त मा तम् त मा । \newline
13. आ ऽह॑र दहर॒दा ऽह॑रत् । \newline
14. अ॒ह॒र॒त् तेन॒ तेना॑ हर दहर॒त् तेन॑ । \newline
15. तेना॑ यजता यजत॒ तेन॒ तेना॑ यजत । \newline
16. अ॒य॒ज॒त॒ तत॒ स्ततो॑ ऽयजता यजत॒ ततः॑ । \newline
17. ततो॒ वै वै तत॒ स्ततो॒ वै । \newline
18. वै स स वै वै सः । \newline
19. स ब्र॑ह्मवर्च॒सी ब्र॑ह्मवर्च॒सी स स ब्र॑ह्मवर्च॒सी । \newline
20. ब्र॒ह्म॒व॒र्च॒ स्य॑भव दभवद् ब्रह्मवर्च॒सी ब्र॑ह्मवर्च॒ स्य॑भवत् । \newline
21. ब्र॒ह्म॒व॒र्च॒सीति॑ ब्रह्म - व॒र्च॒सी । \newline
22. अ॒भ॒व॒द् यो यो॑ ऽभव दभव॒द् यः । \newline
23. य ए॒व मे॒वं ॅयो य ए॒वम् । \newline
24. ए॒वं ॅवि॒द्वान्. वि॒द्वा ने॒व मे॒वं ॅवि॒द्वान् । \newline
25. वि॒द्वा न॑ष्टरा॒त्रेणा᳚ ष्टरा॒त्रेण॑ वि॒द्वान्. वि॒द्वा न॑ष्टरा॒त्रेण॑ । \newline
26. अ॒ष्ट॒रा॒त्रेण॒ यज॑ते॒ यज॑ते ऽष्टरा॒त्रेणा᳚ ष्टरा॒त्रेण॒ यज॑ते । \newline
27. अ॒ष्ट॒रा॒त्रेणेत्य॑ष्ट - रा॒त्रेण॑ । \newline
28. यज॑ते ब्रह्मवर्च॒सी ब्र॑ह्मवर्च॒सी यज॑ते॒ यज॑ते ब्रह्मवर्च॒सी । \newline
29. ब्र॒ह्म॒व॒र्च॒ स्ये॑वैव ब्र॑ह्मवर्च॒सी ब्र॑ह्मवर्च॒ स्ये॑व । \newline
30. ब्र॒ह्म॒व॒र्च॒सीति॑ ब्रह्म - व॒र्च॒सी । \newline
31. ए॒व भ॑वति भव त्ये॒वैव भ॑वति । \newline
32. भ॒व॒ त्य॒ष्ट॒रा॒त्रो᳚ ऽष्टरा॒त्रो भ॑वति भव त्यष्टरा॒त्रः । \newline
33. अ॒ष्ट॒रा॒त्रो भ॑वति भव त्यष्टरा॒त्रो᳚ ऽष्टरा॒त्रो भ॑वति । \newline
34. अ॒ष्ट॒रा॒त्र इत्य॑ष्ट - रा॒त्रः । \newline
35. भ॒व॒ त्य॒ष्टाक्ष॑रा॒ ऽष्टाक्ष॑रा भवति भव त्य॒ष्टाक्ष॑रा । \newline
36. अ॒ष्टाक्ष॑रा गाय॒त्री गा॑य॒ त्र्य॑ष्टाक्ष॑रा॒ ऽष्टाक्ष॑रा गाय॒त्री । \newline
37. अ॒ष्टाक्ष॒रेत्य॒ष्टा - अ॒क्ष॒रा॒ । \newline
38. गा॒य॒त्री गा॑य॒त्री । \newline
39. गा॒य॒त्री ब्र॑ह्मवर्च॒सम् ब्र॑ह्मवर्च॒सम् गा॑य॒त्री गा॑य॒त्री ब्र॑ह्मवर्च॒सम् । \newline
40. ब्र॒ह्म॒व॒र्च॒सम् गा॑यत्रि॒या गा॑यत्रि॒या ब्र॑ह्मवर्च॒सम् ब्र॑ह्मवर्च॒सम् गा॑यत्रि॒या । \newline
41. ब्र॒ह्म॒व॒र्च॒समिति॑ ब्रह्म - व॒र्च॒सम् । \newline
42. गा॒य॒त्रि॒ यैवैव गा॑यत्रि॒या गा॑यत्रि॒ यैव । \newline
43. ए॒व ब्र॑ह्मवर्च॒सम् ब्र॑ह्मवर्च॒स मे॒वैव ब्र॑ह्मवर्च॒सम् । \newline
44. ब्र॒ह्म॒व॒र्च॒स मवाव॑ ब्रह्मवर्च॒सम् ब्र॑ह्मवर्च॒स मव॑ । \newline
45. ब्र॒ह्म॒व॒र्च॒समिति॑ ब्रह्म - व॒र्च॒सम् । \newline
46. अव॑ रुन्धे रु॒न्धे ऽवाव॑ रुन्धे । \newline
47. रु॒न्धे॒ ऽष्ट॒रा॒त्रो᳚ ऽष्टरा॒त्रो रु॑न्धे रुन्धे ऽष्टरा॒त्रः । \newline
48. अ॒ष्ट॒रा॒त्रो भ॑वति भव त्यष्टरा॒त्रो᳚ ऽष्टरा॒त्रो भ॑वति । \newline
49. अ॒ष्ट॒रा॒त्र इत्य॑ष्ट - रा॒त्रः । \newline
50. भ॒व॒ति॒ चत॑स्र॒ श्चत॑स्रो भवति भवति॒ चत॑स्रः । \newline
51. चत॑स्रो॒ वै वै चत॑स्र॒ श्चत॑स्रो॒ वै । \newline
52. वै दिशो॒ दिशो॒ वै वै दिशः॑ । \newline
53. दिश॒ श्चत॑स्र॒ श्चत॑स्रो॒ दिशो॒ दिश॒ श्चत॑स्रः । \newline
54. चत॑स्रो ऽवान्तरदि॒शा अ॑वान्तरदि॒शा श्चत॑स्र॒ श्चत॑स्रो ऽवान्तरदि॒शाः । \newline
55. अ॒वा॒न्त॒र॒दि॒शा दि॒ग्भ्यो दि॒ग्भ्यो॑ ऽवान्तरदि॒शा अ॑वान्तरदि॒शा दि॒ग्भ्यः । \newline
56. अ॒वा॒न्त॒र॒दि॒शा इत्य॑वान्तर - दि॒शाः । \newline
57. दि॒ग्भ्य ए॒वैव दि॒ग्भ्यो दि॒ग्भ्य ए॒व । \newline
58. दि॒ग्भ्य इति॑ दिक् - भ्यः । \newline
59. ए॒व ब्र॑ह्मवर्च॒सम् ब्र॑ह्मवर्च॒स मे॒वैव ब्र॑ह्मवर्च॒सम् । \newline
60. ब्र॒ह्म॒व॒र्च॒स मवाव॑ ब्रह्मवर्च॒सम् ब्र॑ह्मवर्च॒स मव॑ । \newline
61. ब्र॒ह्म॒व॒र्च॒समिति॑ ब्रह्म - व॒र्च॒सम् । \newline
62. अव॑ रुन्धे रु॒न्धे ऽवाव॑ रुन्धे । \newline
63. रु॒न्धे॒ त्रि॒वृत् त्रि॒वृद् रु॑न्धे रुन्धे त्रि॒वृत् । \newline

\textbf{Ghana Paata } \newline

1. बृह॒स्पति॑ रकामयता कामयत॒ बृह॒स्पति॒र् बृह॒स्पति॑ रकामयत ब्रह्मवर्च॒सी ब्र॑ह्मवर्च॒ स्य॑कामयत॒ बृह॒स्पति॒र् बृह॒स्पति॑ रकामयत ब्रह्मवर्च॒सी । \newline
2. अ॒का॒म॒य॒त॒ ब्र॒ह्म॒व॒र्च॒सी ब्र॑ह्मवर्च॒ स्य॑कामयता कामयत ब्रह्मवर्च॒सी स्याꣳ॑ स्याम् ब्रह्मवर्च॒ स्य॑कामयता कामयत ब्रह्मवर्च॒सी स्या᳚म् । \newline
3. ब्र॒ह्म॒व॒र्च॒सी स्याꣳ॑ स्याम् ब्रह्मवर्च॒सी ब्र॑ह्मवर्च॒सी स्या॒ मितीति॑ स्याम् ब्रह्मवर्च॒सी ब्र॑ह्मवर्च॒सी स्या॒ मिति॑ । \newline
4. ब्र॒ह्म॒व॒र्च॒सीति॑ ब्रह्म - व॒र्च॒सी । \newline
5. स्या॒ मितीति॑ स्याꣳ स्या॒ मिति॒ स स इति॑ स्याꣳ स्या॒ मिति॒ सः । \newline
6. इति॒ स स इतीति॒ स ए॒त मे॒तꣳ स इतीति॒ स ए॒तम् । \newline
7. स ए॒त मे॒तꣳ स स ए॒त म॑ष्टरा॒त्र म॑ष्टरा॒त्र मे॒तꣳ स स ए॒त म॑ष्टरा॒त्रम् । \newline
8. ए॒त म॑ष्टरा॒त्र म॑ष्टरा॒त्र मे॒त मे॒त म॑ष्टरा॒त्र म॑पश्य दपश्य दष्टरा॒त्र मे॒त मे॒त म॑ष्टरा॒त्र म॑पश्यत् । \newline
9. अ॒ष्ट॒रा॒त्र म॑पश्य दपश्य दष्टरा॒त्र म॑ष्टरा॒त्र म॑पश्य॒त् तम् त म॑पश्य दष्टरा॒त्र म॑ष्टरा॒त्र म॑पश्य॒त् तम् । \newline
10. अ॒ष्ट॒रा॒त्रमित्य॑ष्ट - रा॒त्रम् । \newline
11. अ॒प॒श्य॒त् तम् त म॑पश्य दपश्य॒त् त मा त म॑पश्य दपश्य॒त् त मा । \newline
12. त मा तम् त मा ऽह॑र दहर॒दा तम् त मा ऽह॑रत् । \newline
13. आ ऽह॑र दहर॒दा ऽह॑र॒त् तेन॒ तेना॑ हर॒दा ऽह॑र॒त् तेन॑ । \newline
14. अ॒ह॒र॒त् तेन॒ तेना॑ हर दहर॒त् तेना॑ यजता यजत॒ तेना॑ हर दहर॒त् तेना॑ यजत । \newline
15. तेना॑ यजता यजत॒ तेन॒ तेना॑ यजत॒ तत॒ स्ततो॑ ऽयजत॒ तेन॒ तेना॑ यजत॒ ततः॑ । \newline
16. अ॒य॒ज॒त॒ तत॒ स्ततो॑ ऽयजता यजत॒ ततो॒ वै वै ततो॑ ऽयजता यजत॒ ततो॒ वै । \newline
17. ततो॒ वै वै तत॒ स्ततो॒ वै स स वै तत॒ स्ततो॒ वै सः । \newline
18. वै स स वै वै स ब्र॑ह्मवर्च॒सी ब्र॑ह्मवर्च॒सी स वै वै स ब्र॑ह्मवर्च॒सी । \newline
19. स ब्र॑ह्मवर्च॒सी ब्र॑ह्मवर्च॒सी स स ब्र॑ह्मवर्च॒ स्य॑भव दभवद् ब्रह्मवर्च॒सी स स ब्र॑ह्मवर्च॒ स्य॑भवत् । \newline
20. ब्र॒ह्म॒व॒र्च॒ स्य॑भव दभवद् ब्रह्मवर्च॒सी ब्र॑ह्मवर्च॒ स्य॑भव॒द् यो यो॑ ऽभवद् ब्रह्मवर्च॒सी ब्र॑ह्मवर्च॒ स्य॑भव॒द् यः । \newline
21. ब्र॒ह्म॒व॒र्च॒सीति॑ ब्रह्म - व॒र्च॒सी । \newline
22. अ॒भ॒व॒द् यो यो॑ ऽभव दभव॒द् य ए॒व मे॒वं ॅयो॑ ऽभव दभव॒द् य ए॒वम् । \newline
23. य ए॒व मे॒वं ॅयो य ए॒वं ॅवि॒द्वान्. वि॒द्वा ने॒वं ॅयो य ए॒वं ॅवि॒द्वान् । \newline
24. ए॒वं ॅवि॒द्वान्. वि॒द्वा ने॒व मे॒वं ॅवि॒द्वा न॑ष्टरा॒त्रेणा᳚ ष्टरा॒त्रेण॑ वि॒द्वा ने॒व मे॒वं ॅवि॒द्वा न॑ष्टरा॒त्रेण॑ । \newline
25. वि॒द्वा न॑ष्टरा॒त्रेणा᳚ ष्टरा॒त्रेण॑ वि॒द्वान्. वि॒द्वा न॑ष्टरा॒त्रेण॒ यज॑ते॒ यज॑ते ऽष्टरा॒त्रेण॑ वि॒द्वान्. वि॒द्वा न॑ष्टरा॒त्रेण॒ यज॑ते । \newline
26. अ॒ष्ट॒रा॒त्रेण॒ यज॑ते॒ यज॑ते ऽष्टरा॒त्रेणा᳚ ष्टरा॒त्रेण॒ यज॑ते ब्रह्मवर्च॒सी ब्र॑ह्मवर्च॒सी यज॑ते ऽष्टरा॒त्रेणा᳚ ष्टरा॒त्रेण॒ यज॑ते ब्रह्मवर्च॒सी । \newline
27. अ॒ष्ट॒रा॒त्रेणेत्य॑ष्ट - रा॒त्रेण॑ । \newline
28. यज॑ते ब्रह्मवर्च॒सी ब्र॑ह्मवर्च॒सी यज॑ते॒ यज॑ते ब्रह्मवर्च॒ स्ये॑वैव ब्र॑ह्मवर्च॒सी यज॑ते॒ यज॑ते ब्रह्मवर्च॒ स्ये॑व । \newline
29. ब्र॒ह्म॒व॒र्च॒ स्ये॑वैव ब्र॑ह्मवर्च॒सी ब्र॑ह्मवर्च॒ स्ये॑व भ॑वति भव त्ये॒व ब्र॑ह्मवर्च॒सी ब्र॑ह्मवर्च॒ स्ये॑व भ॑वति । \newline
30. ब्र॒ह्म॒व॒र्च॒सीति॑ ब्रह्म - व॒र्च॒सी । \newline
31. ए॒व भ॑वति भव त्ये॒वैव भ॑व त्यष्टरा॒त्रो᳚ ऽष्टरा॒त्रो भ॑व त्ये॒वैव भ॑व त्यष्टरा॒त्रः । \newline
32. भ॒व॒ त्य॒ष्ट॒रा॒त्रो᳚ ऽष्टरा॒त्रो भ॑वति भव त्यष्टरा॒त्रो भ॑वति भव त्यष्टरा॒त्रो भ॑वति भव त्यष्टरा॒त्रो भ॑वति । \newline
33. अ॒ष्ट॒रा॒त्रो भ॑वति भव त्यष्टरा॒त्रो᳚ ऽष्टरा॒त्रो भ॑व त्य॒ष्टाक्ष॑रा॒ ऽष्टाक्ष॑रा भव त्यष्टरा॒त्रो᳚ ऽष्टरा॒त्रो भ॑व त्य॒ष्टाक्ष॑रा । \newline
34. अ॒ष्ट॒रा॒त्र इत्य॑ष्ट - रा॒त्रः । \newline
35. भ॒व॒ त्य॒ष्टाक्ष॑रा॒ ऽष्टाक्ष॑रा भवति भव त्य॒ष्टाक्ष॑रा गाय॒त्री गा॑य॒ त्र्य॑ष्टाक्ष॑रा भवति भव त्य॒ष्टाक्ष॑रा गाय॒त्री । \newline
36. अ॒ष्टाक्ष॑रा गाय॒त्री गा॑य॒ त्र्य॑ष्टाक्ष॑रा॒ ऽष्टाक्ष॑रा गाय॒त्री । \newline
37. अ॒ष्टाक्ष॒रेत्य॒ष्टा - अ॒क्ष॒रा॒ । \newline
38. गा॒य॒त्री गा॑य॒त्री । \newline
39. गा॒य॒त्री ब्र॑ह्मवर्च॒सम् ब्र॑ह्मवर्च॒सम् गा॑य॒त्री गा॑य॒त्री ब्र॑ह्मवर्च॒सम् गा॑यत्रि॒या गा॑यत्रि॒या ब्र॑ह्मवर्च॒सम् गा॑य॒त्री गा॑य॒त्री ब्र॑ह्मवर्च॒सम् गा॑यत्रि॒या । \newline
40. ब्र॒ह्म॒व॒र्च॒सम् गा॑यत्रि॒या गा॑यत्रि॒या ब्र॑ह्मवर्च॒सम् ब्र॑ह्मवर्च॒सम् गा॑यत्रि॒यै वैव गा॑यत्रि॒या ब्र॑ह्मवर्च॒सम् ब्र॑ह्मवर्च॒सम् गा॑यत्रि॒यैव । \newline
41. ब्र॒ह्म॒व॒र्च॒समिति॑ ब्रह्म - व॒र्च॒सम् । \newline
42. गा॒य॒त्रि॒यै वैव गा॑यत्रि॒या गा॑यत्रि॒यैव ब्र॑ह्मवर्च॒सम् ब्र॑ह्मवर्च॒स मे॒व गा॑यत्रि॒या गा॑यत्रि॒यैव ब्र॑ह्मवर्च॒सम् । \newline
43. ए॒व ब्र॑ह्मवर्च॒सम् ब्र॑ह्मवर्च॒स मे॒वैव ब्र॑ह्मवर्च॒स मवाव॑ ब्रह्मवर्च॒स मे॒वैव ब्र॑ह्मवर्च॒स मव॑ । \newline
44. ब्र॒ह्म॒व॒र्च॒स मवाव॑ ब्रह्मवर्च॒सम् ब्र॑ह्मवर्च॒स मव॑ रुन्धे रु॒न्धे ऽव॑ ब्रह्मवर्च॒सम् ब्र॑ह्मवर्च॒स मव॑ रुन्धे । \newline
45. ब्र॒ह्म॒व॒र्च॒समिति॑ ब्रह्म - व॒र्च॒सम् । \newline
46. अव॑ रुन्धे रु॒न्धे ऽवाव॑ रुन्धे ऽष्टरा॒त्रो᳚ ऽष्टरा॒त्रो रु॒न्धे ऽवाव॑ रुन्धे ऽष्टरा॒त्रः । \newline
47. रु॒न्धे॒ ऽष्ट॒रा॒त्रो᳚ ऽष्टरा॒त्रो रु॑न्धे रुन्धे ऽष्टरा॒त्रो भ॑वति भव त्यष्टरा॒त्रो रु॑न्धे रुन्धे ऽष्टरा॒त्रो भ॑वति । \newline
48. अ॒ष्ट॒रा॒त्रो भ॑वति भव त्यष्टरा॒त्रो᳚ ऽष्टरा॒त्रो भ॑वति॒ चत॑स्र॒ श्चत॑स्रो भव त्यष्टरा॒त्रो᳚ ऽष्टरा॒त्रो भ॑वति॒ चत॑स्रः । \newline
49. अ॒ष्ट॒रा॒त्र इत्य॑ष्ट - रा॒त्रः । \newline
50. भ॒व॒ति॒ चत॑स्र॒ श्चत॑स्रो भवति भवति॒ चत॑स्रो॒ वै वै चत॑स्रो भवति भवति॒ चत॑स्रो॒ वै । \newline
51. चत॑स्रो॒ वै वै चत॑स्र॒ श्चत॑स्रो॒ वै दिशो॒ दिशो॒ वै चत॑स्र॒ श्चत॑स्रो॒ वै दिशः॑ । \newline
52. वै दिशो॒ दिशो॒ वै वै दिश॒ श्चत॑स्र॒ श्चत॑स्रो॒ दिशो॒ वै वै दिश॒ श्चत॑स्रः । \newline
53. दिश॒ श्चत॑स्र॒ श्चत॑स्रो॒ दिशो॒ दिश॒ श्चत॑स्रो ऽवान्तरदि॒शा अ॑वान्तरदि॒शा श्चत॑स्रो॒ दिशो॒ दिश॒ श्चत॑स्रो ऽवान्तरदि॒शाः । \newline
54. चत॑स्रो ऽवान्तरदि॒शा अ॑वान्तरदि॒शा श्चत॑स्र॒ श्चत॑स्रो ऽवान्तरदि॒शा दि॒ग्भ्यो दि॒ग्भ्यो॑ ऽवान्तरदि॒शा श्चत॑स्र॒ श्चत॑स्रो ऽवान्तरदि॒शा दि॒ग्भ्यः । \newline
55. अ॒वा॒न्त॒र॒दि॒शा दि॒ग्भ्यो दि॒ग्भ्यो॑ ऽवान्तरदि॒शा अ॑वान्तरदि॒शा दि॒ग्भ्य ए॒वैव दि॒ग्भ्यो॑ ऽवान्तरदि॒शा अ॑वान्तरदि॒शा दि॒ग्भ्य ए॒व । \newline
56. अ॒वा॒न्त॒र॒दि॒शा इत्य॑वान्तर - दि॒शाः । \newline
57. दि॒ग्भ्य ए॒वैव दि॒ग्भ्यो दि॒ग्भ्य ए॒व ब्र॑ह्मवर्च॒सम् ब्र॑ह्मवर्च॒स मे॒व दि॒ग्भ्यो दि॒ग्भ्य ए॒व ब्र॑ह्मवर्च॒सम् । \newline
58. दि॒ग्भ्य इति॑ दिक् - भ्यः । \newline
59. ए॒व ब्र॑ह्मवर्च॒सम् ब्र॑ह्मवर्च॒स मे॒वैव ब्र॑ह्मवर्च॒स मवाव॑ ब्रह्मवर्च॒स मे॒वैव ब्र॑ह्मवर्च॒स मव॑ । \newline
60. ब्र॒ह्म॒व॒र्च॒स मवाव॑ ब्रह्मवर्च॒सम् ब्र॑ह्मवर्च॒स मव॑ रुन्धे रु॒न्धे ऽव॑ ब्रह्मवर्च॒सम् ब्र॑ह्मवर्च॒स मव॑ रुन्धे । \newline
61. ब्र॒ह्म॒व॒र्च॒समिति॑ ब्रह्म - व॒र्च॒सम् । \newline
62. अव॑ रुन्धे रु॒न्धे ऽवाव॑ रुन्धे त्रि॒वृत् त्रि॒वृद् रु॒न्धे ऽवाव॑ रुन्धे त्रि॒वृत् । \newline
63. रु॒न्धे॒ त्रि॒वृत् त्रि॒वृद् रु॑न्धे रुन्धे त्रि॒वृ द॑ग्निष्टो॒मो᳚ ऽग्निष्टो॒म स्त्रि॒वृद् रु॑न्धे रुन्धे त्रि॒वृ द॑ग्निष्टो॒मः । \newline
\pagebreak
\markright{ TS 7.2.3.2  \hfill https://www.vedavms.in \hfill}

\section{ TS 7.2.3.2 }

\textbf{TS 7.2.3.2 } \newline
\textbf{Samhita Paata} \newline

त्रि॒वृद॑ग्निष्टो॒मो भ॑वति॒ तेज॑ ए॒वाव॑ रुन्धे पञ्चद॒शो भ॑वतीन्द्रि॒यमे॒वाव॑ रुन्धे सप्तद॒शो भ॑वत्य॒न्नाद्य॒स्या-व॑रुद्ध्या॒ अथो॒ प्रैव तेन॑ जायत एकविꣳ॒॒शो भ॑वति॒ प्रति॑ष्ठित्या॒ अथो॒ रुच॑मे॒वाऽऽ*त्मन् ध॑त्ते त्रिण॒वो भ॑वति॒ विजि॑त्यै त्रयस्त्रिꣳ॒॒शो भ॑वति॒ प्रति॑ष्ठित्यै पञ्चविꣳ॒॒शो᳚ऽग्निष्टो॒मो भ॑वति प्र॒जाप॑ते॒राप्त्यै॑ महाव्र॒तवा॑न॒न्नाद्य॒स्या व॑रुद्ध्यै विश्व॒जिथ् सर्व॑पृष्ठोऽतिरा॒त्रो भ॑वति॒ सर्व॑स्या॒भिजि॑त्यै ( ) ॥ \newline

\textbf{Pada Paata} \newline

त्रि॒वृदिति॑ त्रि - वृत् । अ॒ग्नि॒ष्टो॒म इत्य॑ग्नि-स्तो॒मः । भ॒व॒ति॒ । तेजः॑ । ए॒व । अवेति॑ । रु॒न्धे॒ । प॒ञ्च॒द॒श इति॑ पञ्च - द॒शः । भ॒व॒ति॒ । इ॒न्द्रि॒यम् । ए॒व । अवेति॑ । रु॒न्धे॒ । स॒प्त॒द॒श इति॑ सप्त - द॒शः । भ॒व॒ति॒ । अ॒न्नाद्य॒स्येत्य॑न्न - अद्य॑स्य । अव॑रुद्ध्या॒ इत्यव॑ - रु॒द्ध्यै॒ । अथो॒ इति॑ । प्रेति॑ । ए॒व । तेन॑ । जा॒य॒ते॒ । ए॒क॒विꣳ॒॒श इत्ये॑क - विꣳ॒॒शः । भ॒व॒ति॒ । प्रति॑ष्ठित्या॒ इति॒ प्रति॑ - स्थि॒त्यै॒ । अथो॒ इति॑ । रुच᳚म् । ए॒व । आ॒त्मन्न् । ध॒त्ते॒ । त्रि॒ण॒व इति॑ त्रि-न॒वः । भ॒व॒ति॒ । विजि॑त्या॒ इति॒ वि - जि॒त्यै॒ । त्र॒य॒स्त्रिꣳ॒॒श इति॑ त्रयः - त्रिꣳ॒॒शः । भ॒व॒ति॒ । प्रति॑ष्ठित्या॒ इति॒ प्रति॑ - स्थि॒त्यै॒ । प॒ञ्च॒विꣳ॒॒श इति॑ पञ्च - विꣳ॒॒शः । अ॒ग्नि॒ष्टो॒म इत्य॑ग्नि - स्तो॒मः । भ॒व॒ति॒ । प्र॒जाप॑ते॒रिति॑ प्र॒जा-प॒तेः॒ । आप्त्यै᳚ । म॒हा॒व्र॒तवा॒निति॑ महाव्र॒त-वा॒न् । अ॒न्नाद्य॒स्येत्य॑न्न - अद्य॑स्य । अव॑रुद्ध्या॒ इत्यव॑ - रु॒द्ध्यै॒ । वि॒श्व॒जिदिति॑ विश्व - जित् । सर्व॑पृष्ठ॒ इति॒ सर्व॑ - पृ॒ष्ठः॒ । अ॒ति॒रा॒त्र इत्य॑ति - रा॒त्रः । भ॒व॒ति॒ । सर्व॑स्य । अ॒भिजि॑त्या॒ इत्य॒भि - जि॒त्यै॒ ( ) ॥  \newline


\textbf{Krama Paata} \newline

त्रि॒वृद॑ग्निष्टो॒मः । त्रि॒वृदिति॑ त्रि - वृत् । अ॒ग्नि॒ष्टो॒मो भ॑वति । अ॒ग्नि॒ष्टो॒म इत्य॑ग्नि - स्तो॒मः । भ॒व॒ति॒ तेजः॑ । तेज॑ ए॒व । ए॒वाव॑ । अव॑ रुन्धे । रु॒न्धे॒ प॒ञ्च॒द॒शः । प॒ञ्च॒द॒शो भ॑वति । प॒ञ्च॒द॒श इति॑ पञ्च - द॒शः । भ॒व॒ती॒न्द्रि॒यम् । इ॒न्द्रि॒यमे॒व । ए॒वाव॑ । अव॑ रुन्धे । रु॒न्धे॒ स॒प्त॒द॒शः । स॒प्त॒द॒शो भ॑वति । स॒प्त॒द॒श इति॑ सप्त - द॒शः । भ॒व॒त्य॒न्नाद्य॑स्य । अ॒न्नाद्य॒स्याव॑रुद्ध्यै । अ॒न्नाद्य॒स्येत्य॑न्न - अद्य॑स्य । अव॑रुद्ध्या॒ अथो᳚ । अव॑रुद्ध्या॒ इत्यव॑ - रु॒द्ध्यै॒ । अथो॒ प्र । अथो॒ इत्यथो᳚ । प्रैव । ए॒व तेन॑ । तेन॑ जायते । जा॒य॒त॒ ए॒क॒विꣳ॒॒शः । ए॒क॒विꣳ॒॒शो भ॑वति । ए॒क॒विꣳ॒॒श इत्ये॑क - विꣳ॒॒शः । भ॒व॒ति॒ प्रति॑ष्ठित्यै । प्रति॑ष्ठित्या॒ अथो᳚ । प्रति॑ष्ठित्या॒ इति॒ प्रति॑ - स्थि॒त्यै॒ । अथो॒ रुच᳚म् । अथो॒ इत्यथो᳚ । रुच॑मे॒व । ए॒वात्मन्न् । आ॒त्मन् ध॑त्ते । ध॒त्ते॒ त्रि॒ण॒वः । त्रि॒ण॒वो भ॑वति । त्रि॒ण॒व इति॑ त्रि - न॒वः । भ॒व॒ति॒ विजि॑त्यै । विजि॑त्यै त्रयस्त्रिꣳ॒॒शः । विजि॑त्या॒ इति॒ वि - जि॒त्यै॒ । त्र॒य॒स्त्रिꣳ॒॒शो भ॑वति । त्र॒य॒स्त्रिꣳ॒॒श इति॑ त्रयः - त्रिꣳ॒॒शः । भ॒व॒ति॒ प्रति॑ष्ठित्यै । प्रति॑ष्ठित्यै पञ्चविꣳ॒॒शः । प्रति॑ष्ठित्या॒ इति॒ प्रति॑ - स्थि॒त्यै॒ । प॒ञ्च॒विꣳ॒॒शो᳚ऽग्निष्टो॒मः । प॒ञ्च॒विꣳ॒॒श इति॑ पञ्च - विꣳ॒॒शः । अ॒ग्नि॒ष्टो॒मो भ॑वति । अ॒ग्नि॒ष्टो॒म इत्य॑ग्नि - स्तो॒मः । भ॒व॒ति॒ प्र॒जाप॑तेः । प्र॒जाप॑ते॒राप्त्यै᳚ । प्र॒जाप॑ते॒रिति॑ प्र॒जा - प॒तेः॒ । आप्त्यै॑ महाव्र॒तवान्॑ । म॒हा॒व्र॒तवा॑न॒न्नाद्य॑स्य । म॒हा॒व्र॒तवा॒निति॑ महाव्र॒त - वा॒न्॒ । अ॒न्नाद्य॒स्याव॑रुद्ध्यै । अ॒न्नाद्य॒स्येत्य॑न्न - अद्य॑स्य । अव॑रुद्ध्यै विश्व॒जित् । अव॑रुद्ध्या॒ इत्यव॑ - रु॒द्ध्यै॒ । वि॒श्व॒जिथ् सर्व॑पृष्ठः । वि॒श्व॒जिदिति॑ विश्व - जित् । सर्व॑पृष्ठोऽतिरा॒त्रः । सर्व॑पृष्ठ॒ इति॒ सर्व॑ - पृ॒ष्ठः॒ । अ॒ति॒रा॒त्रो भ॑वति । अ॒ति॒रा॒त्र इत्य॑ति - रा॒त्रः । भ॒व॒ति॒ सर्व॑स्य । सर्व॑स्या॒भिजि॑त्यै ( ) । अ॒भिजि॑त्या॒ इत्य॒भि - जि॒त्यै॒ । \newline

\textbf{Jatai Paata} \newline

1. त्रि॒वृ द॑ग्निष्टो॒मो᳚ ऽग्निष्टो॒म स्त्रि॒वृत् त्रि॒वृ द॑ग्निष्टो॒मः । \newline
2. त्रि॒वृदिति॑ त्रि - वृत् । \newline
3. अ॒ग्नि॒ष्टो॒मो भ॑वति भव त्यग्निष्टो॒मो᳚ ऽग्निष्टो॒मो भ॑वति । \newline
4. अ॒ग्नि॒ष्टो॒म इत्य॑ग्नि - स्तो॒मः । \newline
5. भ॒व॒ति॒ तेज॒ स्तेजो॑ भवति भवति॒ तेजः॑ । \newline
6. तेज॑ ए॒वैव तेज॒ स्तेज॑ ए॒व । \newline
7. ए॒वावा वै॒वै वाव॑ । \newline
8. अव॑ रुन्धे रु॒न्धे ऽवाव॑ रुन्धे । \newline
9. रु॒न्धे॒ प॒ञ्च॒द॒शः प॑ञ्चद॒शो रु॑न्धे रुन्धे पञ्चद॒शः । \newline
10. प॒ञ्च॒द॒शो भ॑वति भवति पञ्चद॒शः प॑ञ्चद॒शो भ॑वति । \newline
11. प॒ञ्च॒द॒श इति॑ पञ्च - द॒शः । \newline
12. भ॒व॒ती॒न्द्रि॒य मि॑न्द्रि॒यम् भ॑वति भवतीन्द्रि॒यम् । \newline
13. इ॒न्द्रि॒य मे॒वैवेन्द्रि॒य मि॑न्द्रि॒य मे॒व । \newline
14. ए॒वावा वै॒वै वाव॑ । \newline
15. अव॑ रुन्धे रु॒न्धे ऽवाव॑ रुन्धे । \newline
16. रु॒न्धे॒ स॒प्त॒द॒शः स॑प्तद॒शो रु॑न्धे रुन्धे सप्तद॒शः । \newline
17. स॒प्त॒द॒शो भ॑वति भवति सप्तद॒शः स॑प्तद॒शो भ॑वति । \newline
18. स॒प्त॒द॒श इति॑ सप्त - द॒शः । \newline
19. भ॒व॒ त्य॒न्नाद्य॑स्या॒ न्नाद्य॑स्य भवति भव त्य॒न्नाद्य॑स्य । \newline
20. अ॒न्नाद्य॒स्या व॑रुद्ध्या॒ अव॑रुद्ध्या अ॒न्नाद्य॑स्या॒ न्नाद्य॒स्या व॑रुद्ध्यै । \newline
21. अ॒न्नाद्य॒स्येत्य॑न्न - अद्य॑स्य । \newline
22. अव॑रुद्ध्या॒ अथो॒ अथो॒ अव॑रुद्ध्या॒ अव॑रुद्ध्या॒ अथो᳚ । \newline
23. अव॑रुद्ध्या॒ इत्यव॑ - रु॒द्ध्यै॒ । \newline
24. अथो॒ प्र प्राथो॒ अथो॒ प्र । \newline
25. अथो॒ इत्यथो᳚ । \newline
26. प्रैवैव प्र प्रैव । \newline
27. ए॒व तेन॒ तेनै॒ वैव तेन॑ । \newline
28. तेन॑ जायते जायते॒ तेन॒ तेन॑ जायते । \newline
29. जा॒य॒त॒ ए॒क॒विꣳ॒॒श ए॑कविꣳ॒॒शो जा॑यते जायत एकविꣳ॒॒शः । \newline
30. ए॒क॒विꣳ॒॒शो भ॑वति भव त्येकविꣳ॒॒श ए॑कविꣳ॒॒शो भ॑वति । \newline
31. ए॒क॒विꣳ॒॒श इत्ये॑क - विꣳ॒॒शः । \newline
32. भ॒व॒ति॒ प्रति॑ष्ठित्यै॒ प्रति॑ष्ठित्यै भवति भवति॒ प्रति॑ष्ठित्यै । \newline
33. प्रति॑ष्ठित्या॒ अथो॒ अथो॒ प्रति॑ष्ठित्यै॒ प्रति॑ष्ठित्या॒ अथो᳚ । \newline
34. प्रति॑ष्ठित्या॒ इति॒ प्रति॑ - स्थि॒त्यै॒ । \newline
35. अथो॒ रुचꣳ॒॒ रुच॒ मथो॒ अथो॒ रुच᳚म् । \newline
36. अथो॒ इत्यथो᳚ । \newline
37. रुच॑ मे॒वैव रुचꣳ॒॒ रुच॑ मे॒व । \newline
38. ए॒वात्मन् ना॒त्मन् ने॒वै वात्मन्न् । \newline
39. आ॒त्मन् ध॑त्ते धत्त आ॒त्मन् ना॒त्मन् ध॑त्ते । \newline
40. ध॒त्ते॒ त्रि॒ण॒व स्त्रि॑ण॒वो ध॑त्ते धत्ते त्रिण॒वः । \newline
41. त्रि॒ण॒वो भ॑वति भवति त्रिण॒व स्त्रि॑ण॒वो भ॑वति । \newline
42. त्रि॒ण॒व इति॑ त्रि - न॒वः । \newline
43. भ॒व॒ति॒ विजि॑त्यै॒ विजि॑त्यै भवति भवति॒ विजि॑त्यै । \newline
44. विजि॑त्यै त्रयस्त्रिꣳ॒॒श स्त्र॑यस्त्रिꣳ॒॒शो विजि॑त्यै॒ विजि॑त्यै त्रयस्त्रिꣳ॒॒शः । \newline
45. विजि॑त्या॒ इति॒ वि - जि॒त्यै॒ । \newline
46. त्र॒य॒स्त्रिꣳ॒॒शो भ॑वति भवति त्रयस्त्रिꣳ॒॒श स्त्र॑यस्त्रिꣳ॒॒शो भ॑वति । \newline
47. त्र॒य॒स्त्रिꣳ॒॒श इति॑ त्रयः - त्रिꣳ॒॒शः । \newline
48. भ॒व॒ति॒ प्रति॑ष्ठित्यै॒ प्रति॑ष्ठित्यै भवति भवति॒ प्रति॑ष्ठित्यै । \newline
49. प्रति॑ष्ठित्यै पञ्चविꣳ॒॒शः प॑ञ्चविꣳ॒॒शः प्रति॑ष्ठित्यै॒ प्रति॑ष्ठित्यै पञ्चविꣳ॒॒शः । \newline
50. प्रति॑ष्ठित्या॒ इति॒ प्रति॑ - स्थि॒त्यै॒ । \newline
51. प॒ञ्च॒विꣳ॒॒शो᳚ ऽग्निष्टो॒मो᳚ ऽग्निष्टो॒मः प॑ञ्चविꣳ॒॒शः प॑ञ्चविꣳ॒॒शो᳚ ऽग्निष्टो॒मः । \newline
52. प॒ञ्च॒विꣳ॒॒श इति॑ पञ्च - विꣳ॒॒शः । \newline
53. अ॒ग्नि॒ष्टो॒मो भ॑वति भव त्यग्निष्टो॒मो᳚ ऽग्निष्टो॒मो भ॑वति । \newline
54. अ॒ग्नि॒ष्टो॒म इत्य॑ग्नि - स्तो॒मः । \newline
55. भ॒व॒ति॒ प्र॒जाप॑तेः प्र॒जाप॑तेर् भवति भवति प्र॒जाप॑तेः । \newline
56. प्र॒जाप॑ते॒ राप्त्या॒ आप्त्यै᳚ प्र॒जाप॑तेः प्र॒जाप॑ते॒ राप्त्यै᳚ । \newline
57. प्र॒जाप॑ते॒रिति॑ प्र॒जा - प॒तेः॒ । \newline
58. आप्त्यै॑ महाव्र॒तवा᳚न् महाव्र॒तवा॒ नाप्त्या॒ आप्त्यै॑ महाव्र॒तवान्॑ । \newline
59. म॒हा॒व्र॒तवा॑ न॒न्नाद्य॑स्या॒ न्नाद्य॑स्य महाव्र॒तवा᳚न् महाव्र॒तवा॑ न॒न्नाद्य॑स्य । \newline
60. म॒हा॒व्र॒तवा॒निति॑ महाव्र॒त - वा॒न् । \newline
61. अ॒न्नाद्य॒स्या व॑रुद्ध्या॒ अव॑रुद्ध्या अ॒न्नाद्य॑स्या॒ न्नाद्य॒स्या व॑रुद्ध्यै । \newline
62. अ॒न्नाद्य॒स्येत्य॑न्न - अद्य॑स्य । \newline
63. अव॑रुद्ध्यै विश्व॒जिद् वि॑श्व॒जि दव॑रुद्ध्या॒ अव॑रुद्ध्यै विश्व॒जित् । \newline
64. अव॑रुद्ध्या॒ इत्यव॑ - रु॒द्ध्यै॒ । \newline
65. वि॒श्व॒जिथ् सर्व॑पृष्ठः॒ सर्व॑पृष्ठो विश्व॒जिद् वि॑श्व॒जिथ् सर्व॑पृष्ठः । \newline
66. वि॒श्व॒जिदिति॑ विश्व - जित् । \newline
67. सर्व॑पृष्ठो ऽतिरा॒त्रो॑ ऽतिरा॒त्रः सर्व॑पृष्ठः॒ सर्व॑पृष्ठो ऽतिरा॒त्रः । \newline
68. सर्व॑पृष्ठ॒ इति॒ सर्व॑ - पृ॒ष्ठः॒ । \newline
69. अ॒ति॒रा॒त्रो भ॑वति भव त्यतिरा॒त्रो॑ ऽतिरा॒त्रो भ॑वति । \newline
70. अ॒ति॒रा॒त्र इत्य॑ति - रा॒त्रः । \newline
71. भ॒व॒ति॒ सर्व॑स्य॒ सर्व॑स्य भवति भवति॒ सर्व॑स्य । \newline
72. सर्व॑स्या॒ भिजि॑त्या अ॒भिजि॑त्यै॒ सर्व॑स्य॒ सर्व॑स्या॒ भिजि॑त्यै । \newline
73. अ॒भिजि॑त्या॒ इत्य॒भि - जि॒त्यै॒ । \newline

\textbf{Ghana Paata } \newline

1. त्रि॒वृ द॑ग्निष्टो॒मो᳚ ऽग्निष्टो॒म स्त्रि॒वृत् त्रि॒वृ द॑ग्निष्टो॒मो भ॑वति भव त्यग्निष्टो॒म स्त्रि॒वृत् त्रि॒वृ द॑ग्निष्टो॒मो भ॑वति । \newline
2. त्रि॒वृदिति॑ त्रि - वृत् । \newline
3. अ॒ग्नि॒ष्टो॒मो भ॑वति भव त्यग्निष्टो॒मो᳚ ऽग्निष्टो॒मो भ॑वति॒ तेज॒ स्तेजो॑ भव त्यग्निष्टो॒मो᳚ ऽग्निष्टो॒मो भ॑वति॒ तेजः॑ । \newline
4. अ॒ग्नि॒ष्टो॒म इत्य॑ग्नि - स्तो॒मः । \newline
5. भ॒व॒ति॒ तेज॒ स्तेजो॑ भवति भवति॒ तेज॑ ए॒वैव तेजो॑ भवति भवति॒ तेज॑ ए॒व । \newline
6. तेज॑ ए॒वैव तेज॒ स्तेज॑ ए॒वावा वै॒व तेज॒ स्तेज॑ ए॒वाव॑ । \newline
7. ए॒वावा वै॒वै वाव॑ रुन्धे रु॒न्धे ऽवै॒वै वाव॑ रुन्धे । \newline
8. अव॑ रुन्धे रु॒न्धे ऽवाव॑ रुन्धे पञ्चद॒शः प॑ञ्चद॒शो रु॒न्धे ऽवाव॑ रुन्धे पञ्चद॒शः । \newline
9. रु॒न्धे॒ प॒ञ्च॒द॒शः प॑ञ्चद॒शो रु॑न्धे रुन्धे पञ्चद॒शो भ॑वति भवति पञ्चद॒शो रु॑न्धे रुन्धे पञ्चद॒शो भ॑वति । \newline
10. प॒ञ्च॒द॒शो भ॑वति भवति पञ्चद॒शः प॑ञ्चद॒शो भ॑व तीन्द्रि॒य मि॑न्द्रि॒यम् भ॑वति पञ्चद॒शः प॑ञ्चद॒शो भ॑व तीन्द्रि॒यम् । \newline
11. प॒ञ्च॒द॒श इति॑ पञ्च - द॒शः । \newline
12. भ॒व॒ ती॒न्द्रि॒य मि॑न्द्रि॒यम् भ॑वति भव तीन्द्रि॒य मे॒वैवेन्द्रि॒यम् भ॑वति भव तीन्द्रि॒य मे॒व । \newline
13. इ॒न्द्रि॒य मे॒वैवेन्द्रि॒य मि॑न्द्रि॒य मे॒वावा वै॒वेन्द्रि॒य मि॑न्द्रि॒य मे॒वाव॑ । \newline
14. ए॒वावा वै॒वै वाव॑ रुन्धे रु॒न्धे ऽवै॒वै वाव॑ रुन्धे । \newline
15. अव॑ रुन्धे रु॒न्धे ऽवाव॑ रुन्धे सप्तद॒शः स॑प्तद॒शो रु॒न्धे ऽवाव॑ रुन्धे सप्तद॒शः । \newline
16. रु॒न्धे॒ स॒प्त॒द॒शः स॑प्तद॒शो रु॑न्धे रुन्धे सप्तद॒शो भ॑वति भवति सप्तद॒शो रु॑न्धे रुन्धे सप्तद॒शो भ॑वति । \newline
17. स॒प्त॒द॒शो भ॑वति भवति सप्तद॒शः स॑प्तद॒शो भ॑व त्य॒न्नाद्य॑स्या॒ न्नाद्य॑स्य भवति सप्तद॒शः स॑प्तद॒शो भ॑व त्य॒न्नाद्य॑स्य । \newline
18. स॒प्त॒द॒श इति॑ सप्त - द॒शः । \newline
19. भ॒व॒ त्य॒न्नाद्य॑स्या॒ न्नाद्य॑स्य भवति भव त्य॒न्नाद्य॒स्या व॑रुद्ध्या॒ अव॑रुद्ध्या अ॒न्नाद्य॑स्य भवति भव त्य॒न्नाद्य॒स्या व॑रुद्ध्यै । \newline
20. अ॒न्नाद्य॒स्या व॑रुद्ध्या॒ अव॑रुद्ध्या अ॒न्नाद्य॑स्या॒ न्नाद्य॒स्या व॑रुद्ध्या॒ अथो॒ अथो॒ अव॑रुद्ध्या अ॒न्नाद्य॑स्या॒ न्नाद्य॒स्या व॑रुद्ध्या॒ अथो᳚ । \newline
21. अ॒न्नाद्य॒स्येत्य॑न्न - अद्य॑स्य । \newline
22. अव॑रुद्ध्या॒ अथो॒ अथो॒ अव॑रुद्ध्या॒ अव॑रुद्ध्या॒ अथो॒ प्र प्राथो॒ अव॑रुद्ध्या॒ अव॑रुद्ध्या॒ अथो॒ प्र । \newline
23. अव॑रुद्ध्या॒ इत्यव॑ - रु॒द्ध्यै॒ । \newline
24. अथो॒ प्र प्राथो॒ अथो॒ प्रैवैव प्राथो॒ अथो॒ प्रैव । \newline
25. अथो॒ इत्यथो᳚ । \newline
26. प्रैवैव प्र प्रैव तेन॒ तेनै॒व प्र प्रैव तेन॑ । \newline
27. ए॒व तेन॒ तेनै॒ वैव तेन॑ जायते जायते॒ तेनै॒ वैव तेन॑ जायते । \newline
28. तेन॑ जायते जायते॒ तेन॒ तेन॑ जायत एकविꣳ॒॒श ए॑कविꣳ॒॒शो जा॑यते॒ तेन॒ तेन॑ जायत एकविꣳ॒॒शः । \newline
29. जा॒य॒त॒ ए॒क॒विꣳ॒॒श ए॑कविꣳ॒॒शो जा॑यते जायत एकविꣳ॒॒शो भ॑वति भव त्येकविꣳ॒॒शो जा॑यते जायत एकविꣳ॒॒शो भ॑वति । \newline
30. ए॒क॒विꣳ॒॒शो भ॑वति भव त्येकविꣳ॒॒श ए॑कविꣳ॒॒शो भ॑वति॒ प्रति॑ष्ठित्यै॒ प्रति॑ष्ठित्यै भव त्येकविꣳ॒॒श ए॑कविꣳ॒॒शो भ॑वति॒ प्रति॑ष्ठित्यै । \newline
31. ए॒क॒विꣳ॒॒श इत्ये॑क - विꣳ॒॒शः । \newline
32. भ॒व॒ति॒ प्रति॑ष्ठित्यै॒ प्रति॑ष्ठित्यै भवति भवति॒ प्रति॑ष्ठित्या॒ अथो॒ अथो॒ प्रति॑ष्ठित्यै भवति भवति॒ प्रति॑ष्ठित्या॒ अथो᳚ । \newline
33. प्रति॑ष्ठित्या॒ अथो॒ अथो॒ प्रति॑ष्ठित्यै॒ प्रति॑ष्ठित्या॒ अथो॒ रुचꣳ॒॒ रुच॒ मथो॒ प्रति॑ष्ठित्यै॒ प्रति॑ष्ठित्या॒ अथो॒ रुच᳚म् । \newline
34. प्रति॑ष्ठित्या॒ इति॒ प्रति॑ - स्थि॒त्यै॒ । \newline
35. अथो॒ रुचꣳ॒॒ रुच॒ मथो॒ अथो॒ रुच॑ मे॒वैव रुच॒ मथो॒ अथो॒ रुच॑ मे॒व । \newline
36. अथो॒ इत्यथो᳚ । \newline
37. रुच॑ मे॒वैव रुचꣳ॒॒ रुच॑ मे॒वात्मन् ना॒त्मन् ने॒व रुचꣳ॒॒ रुच॑ मे॒वात्मन्न् । \newline
38. ए॒वात्मन् ना॒त्मन् ने॒वै वात्मन् ध॑त्ते धत्त आ॒त्मन् ने॒वै वात्मन् ध॑त्ते । \newline
39. आ॒त्मन् ध॑त्ते धत्त आ॒त्मन् ना॒त्मन् ध॑त्ते त्रिण॒व स्त्रि॑ण॒वो ध॑त्त आ॒त्मन् ना॒त्मन् ध॑त्ते त्रिण॒वः । \newline
40. ध॒त्ते॒ त्रि॒ण॒व स्त्रि॑ण॒वो ध॑त्ते धत्ते त्रिण॒वो भ॑वति भवति त्रिण॒वो ध॑त्ते धत्ते त्रिण॒वो भ॑वति । \newline
41. त्रि॒ण॒वो भ॑वति भवति त्रिण॒व स्त्रि॑ण॒वो भ॑वति॒ विजि॑त्यै॒ विजि॑त्यै भवति त्रिण॒व स्त्रि॑ण॒वो भ॑वति॒ विजि॑त्यै । \newline
42. त्रि॒ण॒व इति॑ त्रि - न॒वः । \newline
43. भ॒व॒ति॒ विजि॑त्यै॒ विजि॑त्यै भवति भवति॒ विजि॑त्यै त्रयस्त्रिꣳ॒॒श स्त्र॑यस्त्रिꣳ॒॒शो विजि॑त्यै भवति भवति॒ विजि॑त्यै त्रयस्त्रिꣳ॒॒शः । \newline
44. विजि॑त्यै त्रयस्त्रिꣳ॒॒श स्त्र॑यस्त्रिꣳ॒॒शो विजि॑त्यै॒ विजि॑त्यै त्रयस्त्रिꣳ॒॒शो भ॑वति भवति त्रयस्त्रिꣳ॒॒शो विजि॑त्यै॒ विजि॑त्यै त्रयस्त्रिꣳ॒॒शो भ॑वति । \newline
45. विजि॑त्या॒ इति॒ वि - जि॒त्यै॒ । \newline
46. त्र॒य॒स्त्रिꣳ॒॒शो भ॑वति भवति त्रयस्त्रिꣳ॒॒श स्त्र॑यस्त्रिꣳ॒॒शो भ॑वति॒ प्रति॑ष्ठित्यै॒ प्रति॑ष्ठित्यै भवति त्रयस्त्रिꣳ॒॒श स्त्र॑यस्त्रिꣳ॒॒शो भ॑वति॒ प्रति॑ष्ठित्यै । \newline
47. त्र॒य॒स्त्रिꣳ॒॒श इति॑ त्रयः - त्रिꣳ॒॒शः । \newline
48. भ॒व॒ति॒ प्रति॑ष्ठित्यै॒ प्रति॑ष्ठित्यै भवति भवति॒ प्रति॑ष्ठित्यै पञ्चविꣳ॒॒शः प॑ञ्चविꣳ॒॒शः प्रति॑ष्ठित्यै भवति भवति॒ प्रति॑ष्ठित्यै पञ्चविꣳ॒॒शः । \newline
49. प्रति॑ष्ठित्यै पञ्चविꣳ॒॒शः प॑ञ्चविꣳ॒॒शः प्रति॑ष्ठित्यै॒ प्रति॑ष्ठित्यै पञ्चविꣳ॒॒शो᳚ ऽग्निष्टो॒मो᳚ ऽग्निष्टो॒मः प॑ञ्चविꣳ॒॒शः प्रति॑ष्ठित्यै॒ प्रति॑ष्ठित्यै पञ्चविꣳ॒॒शो᳚ ऽग्निष्टो॒मः । \newline
50. प्रति॑ष्ठित्या॒ इति॒ प्रति॑ - स्थि॒त्यै॒ । \newline
51. प॒ञ्च॒विꣳ॒॒शो᳚ ऽग्निष्टो॒मो᳚ ऽग्निष्टो॒मः प॑ञ्चविꣳ॒॒शः प॑ञ्चविꣳ॒॒शो᳚ ऽग्निष्टो॒मो भ॑वति भव त्यग्निष्टो॒मः प॑ञ्चविꣳ॒॒शः प॑ञ्चविꣳ॒॒शो᳚ ऽग्निष्टो॒मो भ॑वति । \newline
52. प॒ञ्च॒विꣳ॒॒श इति॑ पञ्च - विꣳ॒॒शः । \newline
53. अ॒ग्नि॒ष्टो॒मो भ॑वति भव त्यग्निष्टो॒मो᳚ ऽग्निष्टो॒मो भ॑वति प्र॒जाप॑तेः प्र॒जाप॑तेर् भव त्यग्निष्टो॒मो᳚ ऽग्निष्टो॒मो भ॑वति प्र॒जाप॑तेः । \newline
54. अ॒ग्नि॒ष्टो॒म इत्य॑ग्नि - स्तो॒मः । \newline
55. भ॒व॒ति॒ प्र॒जाप॑तेः प्र॒जाप॑तेर् भवति भवति प्र॒जाप॑ते॒ राप्त्या॒ आप्त्यै᳚ प्र॒जाप॑तेर् भवति भवति प्र॒जाप॑ते॒ राप्त्यै᳚ । \newline
56. प्र॒जाप॑ते॒ राप्त्या॒ आप्त्यै᳚ प्र॒जाप॑तेः प्र॒जाप॑ते॒ राप्त्यै॑ महाव्र॒तवा᳚न् महाव्र॒तवा॒ नाप्त्यै᳚ प्र॒जाप॑तेः प्र॒जाप॑ते॒ राप्त्यै॑ महाव्र॒तवान्॑ । \newline
57. प्र॒जाप॑ते॒रिति॑ प्र॒जा - प॒तेः॒ । \newline
58. आप्त्यै॑ महाव्र॒तवा᳚न् महाव्र॒तवा॒ नाप्त्या॒ आप्त्यै॑ महाव्र॒तवा॑ न॒न्नाद्य॑स्या॒ न्नाद्य॑स्य महाव्र॒तवा॒ नाप्त्या॒ आप्त्यै॑ महाव्र॒तवा॑ न॒न्नाद्य॑स्य । \newline
59. म॒हा॒व्र॒तवा॑ न॒न्नाद्य॑स्या॒ न्नाद्य॑स्य महाव्र॒तवा᳚न् महाव्र॒तवा॑ न॒न्नाद्य॒स्या व॑रुद्ध्या॒ अव॑रुद्ध्या अ॒न्नाद्य॑स्य महाव्र॒तवा᳚न् महाव्र॒तवा॑ न॒न्नाद्य॒स्या व॑रुद्ध्यै । \newline
60. म॒हा॒व्र॒तवा॒निति॑ महाव्र॒त - वा॒न् । \newline
61. अ॒न्नाद्य॒स्या व॑रुद्ध्या॒ अव॑रुद्ध्या अ॒न्नाद्य॑स्या॒ न्नाद्य॒स्या व॑रुद्ध्यै विश्व॒जिद् वि॑श्व॒जि दव॑रुद्ध्या अ॒न्नाद्य॑स्या॒ न्नाद्य॒स्या व॑रुद्ध्यै विश्व॒जित् । \newline
62. अ॒न्नाद्य॒स्येत्य॑न्न - अद्य॑स्य । \newline
63. अव॑रुद्ध्यै विश्व॒जिद् वि॑श्व॒जि दव॑रुद्ध्या॒ अव॑रुद्ध्यै विश्व॒जिथ् सर्व॑पृष्ठः॒ सर्व॑पृष्ठो विश्व॒जि दव॑रुद्ध्या॒ अव॑रुद्ध्यै विश्व॒जिथ् सर्व॑पृष्ठः । \newline
64. अव॑रुद्ध्या॒ इत्यव॑ - रु॒द्ध्यै॒ । \newline
65. वि॒श्व॒जिथ् सर्व॑पृष्ठः॒ सर्व॑पृष्ठो विश्व॒जिद् वि॑श्व॒जिथ् सर्व॑पृष्ठो ऽतिरा॒त्रो॑ ऽतिरा॒त्रः सर्व॑पृष्ठो विश्व॒जिद् वि॑श्व॒जिथ् सर्व॑पृष्ठो ऽतिरा॒त्रः । \newline
66. वि॒श्व॒जिदिति॑ विश्व - जित् । \newline
67. सर्व॑पृष्ठो ऽतिरा॒त्रो॑ ऽतिरा॒त्रः सर्व॑पृष्ठः॒ सर्व॑पृष्ठो ऽतिरा॒त्रो भ॑वति भव त्यतिरा॒त्रः सर्व॑पृष्ठः॒ सर्व॑पृष्ठो ऽतिरा॒त्रो भ॑वति । \newline
68. सर्व॑पृष्ठ॒ इति॒ सर्व॑ - पृ॒ष्ठः॒ । \newline
69. अ॒ति॒रा॒त्रो भ॑वति भव त्यतिरा॒त्रो॑ ऽतिरा॒त्रो भ॑वति॒ सर्व॑स्य॒ सर्व॑स्य भव त्यतिरा॒त्रो॑ ऽतिरा॒त्रो भ॑वति॒ सर्व॑स्य । \newline
70. अ॒ति॒रा॒त्र इत्य॑ति - रा॒त्रः । \newline
71. भ॒व॒ति॒ सर्व॑स्य॒ सर्व॑स्य भवति भवति॒ सर्व॑स्या॒ भिजि॑त्या अ॒भिजि॑त्यै॒ सर्व॑स्य भवति भवति॒ सर्व॑स्या॒ भिजि॑त्यै । \newline
72. सर्व॑स्या॒ भिजि॑त्या अ॒भिजि॑त्यै॒ सर्व॑स्य॒ सर्व॑स्या॒ भिजि॑त्यै । \newline
73. अ॒भिजि॑त्या॒ इत्य॒भि - जि॒त्यै॒ । \newline
\pagebreak
\markright{ TS 7.2.4.1  \hfill https://www.vedavms.in \hfill}

\section{ TS 7.2.4.1 }

\textbf{TS 7.2.4.1 } \newline
\textbf{Samhita Paata} \newline

प्र॒जाप॑तिः प्र॒जा अ॑सृजत॒ ताः सृ॒ष्टाः क्षुधं॒ न्या॑य॒न्थ्स ए॒तं न॑वरा॒त्रप॑श्य॒त् तमाऽह॑र॒त् तेना॑यजत॒ ततो॒ वै प्र॒जाभ्यो॑ऽकल्पत॒ यर्.हि॑ प्र॒जाः क्षुधं॑ नि॒गच्छे॑यु॒स्तर्.हि॑ नवरा॒त्रेण॑ यजेते॒मे हि वा ए॒तासां᳚ ॅलो॒का अक्लृ॑प्ता॒ अथै॒ताः क्षुधं॒ नि ग॑च्छन्ती॒माने॒वाऽऽ*भ्यो॑ लो॒कान् क॑ल्पयति॒ तान् कल्प॑मानान् प्र॒जाभ्योऽनु॑ कल्पते॒ कल्प॑न्ते - [  ] \newline

\textbf{Pada Paata} \newline

प्र॒जाप॑ति॒रिति॑ प्र॒जा - प॒तिः॒ । प्र॒जा इति॑ प्र-जाः । अ॒सृ॒ज॒त॒ । ताः । सृ॒ष्टाः । क्षुध᳚म् । नीति॑ । आ॒य॒न्न् । सः । ए॒तम् । न॒व॒रा॒त्रमिति॑ नव - रा॒त्रम् । अ॒प॒श्य॒त् । तम् । एति॑ । अ॒ह॒र॒त् । तेन॑ । अ॒य॒ज॒त॒ । ततः॑ । वै । प्र॒जाभ्य॒ इति॑ प्र - जाभ्यः॑ । अ॒क॒ल्प॒त॒ । यर्.हि॑ । प्र॒जा इति॑ प्र - जाः । क्षुध᳚म् । नि॒गच्छे॑यु॒रिति॑ नि - गच्छे॑युः । तर्.हि॑ । न॒व॒रा॒त्रेणेति॑ नव - रा॒त्रेण॑ । य॒जे॒त॒ । इ॒मे । हि । वै । ए॒तासा᳚म् । लो॒काः । अक्लृ॑प्ताः । अथ॑ । ए॒ताः । क्षुध᳚म् । नीति॑ । ग॒च्छ॒न्ति॒ । इ॒मान् । ए॒व । आ॒भ्यः॒ । लो॒कान् । क॒ल्प॒य॒ति॒ । तान् । कल्प॑मानान् । प्र॒जाभ्य॒ इति॑ प्र - जाभ्यः॑ । अन्विति॑ । क॒ल्प॒ते॒ । कल्प॑न्ते ।  \newline


\textbf{Krama Paata} \newline

प्र॒जाप॑तिः प्र॒जाः । प्र॒जाप॑ति॒रिति॑ प्र॒जा - प॒तिः॒ । प्र॒जा अ॑सृजत । प्र॒जा इति॑ प्र - जाः । अ॒सृ॒ज॒त॒ ताः । ताः सृ॒ष्टाः । सृ॒ष्टाः क्षुध᳚म् । क्षुध॒म् नि । न्या॑यन्न् । आ॒य॒न्थ् सः । स ए॒तम् । ए॒तम् न॑वरा॒त्रम् । न॒व॒रा॒त्रम॑पश्यत् । न॒व॒रा॒त्रमिति॑ नव - रा॒त्रम् । अ॒प॒श्य॒त् तम् । तमा । आऽह॑रत् । अ॒ह॒र॒त् तेन॑ । तेना॑यजत । अ॒य॒ज॒त॒ ततः॑ । ततो॒ वै । वै प्र॒जाभ्यः॑ । प्र॒जाभ्यो॑ऽकल्पत । प्र॒जाभ्य॒ इति॑ प्र - जाभ्यः॑ । अ॒क॒ल्प॒त॒ यर्.हि॑ । यर्.हि॑ प्र॒जाः । प्र॒जाः क्षुध᳚म् । प्र॒जा इति॑ प्र - जाः । क्षुध॑म् नि॒गच्छे॑युः । नि॒गच्छे॑यु॒स्तर्.हि॑ । नि॒गच्छे॑यु॒रिति॑ नि - गच्छे॑युः । तर्.हि॑ नवरा॒त्रेण॑ । न॒व॒रा॒त्रेण॑ यजेत । न॒व॒रा॒त्रेणेति॑ नव - रा॒त्रेण॑ । य॒जे॒ते॒मे । इ॒मे हि । हि वै । वा ए॒तासा᳚म् । ए॒तासा᳚म् ॅलो॒काः । लो॒का अक्लृ॑प्ताः । अक्लृ॑प्ता॒ अथ॑ । अथै॒ताः । ए॒ताः क्षुध᳚म् । क्षुध॒म् नि । नि ग॑च्छन्ति । ग॒च्छ॒न्ती॒मान् । इ॒माने॒व । ए॒वाभ्यः॑ । आ॒भ्यो॒ लो॒कान् । लो॒कान् क॑ल्पयति । क॒ल्प॒य॒ति॒ तान् । तान् कल्प॑मानान् । कल्प॑मानान् प्र॒जाभ्यः॑ । प्र॒जाभ्योऽनु॑ । प्र॒जाभ्य॒ इति॑ प्र - जाभ्यः॑ । अनु॑ कल्पते । क॒ल्प॒ते॒ कल्प॑न्ते । कल्प॑न्तेऽस्मै \newline

\textbf{Jatai Paata} \newline

1. प्र॒जाप॑तिः प्र॒जाः प्र॒जाः प्र॒जाप॑तिः प्र॒जाप॑तिः प्र॒जाः । \newline
2. प्र॒जाप॑ति॒रिति॑ प्र॒जा - प॒तिः॒ । \newline
3. प्र॒जा अ॑सृजता सृजत प्र॒जाः प्र॒जा अ॑सृजत । \newline
4. प्र॒जा इति॑ प्र - जाः । \newline
5. अ॒सृ॒ज॒त॒ ता स्ता अ॑सृजता सृजत॒ ताः । \newline
6. ताः सृ॒ष्टाः सृ॒ष्टा स्ता स्ताः सृ॒ष्टाः । \newline
7. सृ॒ष्टाः क्षुध॒म् क्षुधꣳ॑ सृ॒ष्टाः सृ॒ष्टाः क्षुध᳚म् । \newline
8. क्षुध॒न् नि नि क्षुध॒म् क्षुध॒न् नि । \newline
9. न्या॑यन् नाय॒न् नि न्या॑यन्न् । \newline
10. आ॒य॒न् थ्स स आ॑यन् नाय॒न् थ्सः । \newline
11. स ए॒त मे॒तꣳ स स ए॒तम् । \newline
12. ए॒तन् न॑वरा॒त्रन् न॑वरा॒त्र मे॒त मे॒तन् न॑वरा॒त्रम् । \newline
13. न॒व॒रा॒त्र म॑पश्य दपश्यन् नवरा॒त्रन् न॑वरा॒त्र म॑पश्यत् । \newline
14. न॒व॒रा॒त्रमिति॑ नव - रा॒त्रम् । \newline
15. अ॒प॒श्य॒त् तम् त म॑पश्य दपश्य॒त् तम् । \newline
16. त मा तम् त मा । \newline
17. आ ऽह॑र दहर॒दा ऽह॑रत् । \newline
18. अ॒ह॒र॒त् तेन॒ तेना॑ हर दहर॒त् तेन॑ । \newline
19. तेना॑ यजता यजत॒ तेन॒ तेना॑ यजत । \newline
20. अ॒य॒ज॒त॒ तत॒ स्ततो॑ ऽयजता यजत॒ ततः॑ । \newline
21. ततो॒ वै वै तत॒ स्ततो॒ वै । \newline
22. वै प्र॒जाभ्यः॑ प्र॒जाभ्यो॒ वै वै प्र॒जाभ्यः॑ । \newline
23. प्र॒जाभ्यो॑ ऽकल्पता कल्पत प्र॒जाभ्यः॑ प्र॒जाभ्यो॑ ऽकल्पत । \newline
24. प्र॒जाभ्य॒ इति॑ प्र - जाभ्यः॑ । \newline
25. अ॒क॒ल्प॒त॒ यर्.हि॒ यर्ह्य॑कल्पता कल्पत॒ यर्.हि॑ । \newline
26. यर्.हि॑ प्र॒जाः प्र॒जा यर्.हि॒ यर्.हि॑ प्र॒जाः । \newline
27. प्र॒जाः क्षुध॒म् क्षुध॑म् प्र॒जाः प्र॒जाः क्षुध᳚म् । \newline
28. प्र॒जा इति॑ प्र - जाः । \newline
29. क्षुध॑न् नि॒गच्छे॑युर् नि॒गच्छे॑युः॒ क्षुध॒म् क्षुध॑न् नि॒गच्छे॑युः । \newline
30. नि॒गच्छे॑यु॒ स्तर्.हि॒ तर्.हि॑ नि॒गच्छे॑युर् नि॒गच्छे॑यु॒ स्तर्.हि॑ । \newline
31. नि॒गच्छे॑यु॒रिति॑ नि - गच्छे॑युः । \newline
32. तर्.हि॑ नवरा॒त्रेण॑ नवरा॒त्रेण॒ तर्.हि॒ तर्.हि॑ नवरा॒त्रेण॑ । \newline
33. न॒व॒रा॒त्रेण॑ यजेत यजेत नवरा॒त्रेण॑ नवरा॒त्रेण॑ यजेत । \newline
34. न॒व॒रा॒त्रेणेति॑ नव - रा॒त्रेण॑ । \newline
35. य॒जे॒ते॒म इ॒मे य॑जेत यजेते॒मे । \newline
36. इ॒मे हि हीम इ॒मे हि । \newline
37. हि वै वै हि हि वै । \newline
38. वा ए॒तासा॑ मे॒तासां॒ ॅवै वा ए॒तासा᳚म् । \newline
39. ए॒तासा᳚म् ॅलो॒का लो॒का ए॒तासा॑ मे॒तासा᳚म् ॅलो॒काः । \newline
40. लो॒का अक्लृ॑प्ता॒ अक्लृ॑प्ता लो॒का लो॒का अक्लृ॑प्ताः । \newline
41. अक्लृ॑प्ता॒ अथाथा क्लृ॑प्ता॒ अक्लृ॑प्ता॒ अथ॑ । \newline
42. अथै॒ता ए॒ता अथा थै॒ताः । \newline
43. ए॒ताः क्षुध॒म् क्षुध॑ मे॒ता ए॒ताः क्षुध᳚म् । \newline
44. क्षुध॒न् नि नि क्षुध॒म् क्षुध॒न् नि । \newline
45. नि ग॑च्छन्ति गच्छन्ति॒ नि नि ग॑च्छन्ति । \newline
46. ग॒च्छ॒न्ती॒मा नि॒मान् ग॑च्छन्ति गच्छन्ती॒मान् । \newline
47. इ॒मा ने॒वैवेमा नि॒मा ने॒व । \newline
48. ए॒वाभ्य॑ आभ्य ए॒वै वाभ्यः॑ । \newline
49. आ॒भ्यो॒ लो॒कान् ॅलो॒का ना᳚भ्य आभ्यो लो॒कान् । \newline
50. लो॒कान् क॑ल्पयति कल्पयति लो॒कान् ॅलो॒कान् क॑ल्पयति । \newline
51. क॒ल्प॒य॒ति॒ ताꣳ स्तान् क॑ल्पयति कल्पयति॒ तान् । \newline
52. तान् कल्प॑माना॒न् कल्प॑माना॒न् ताꣳ स्तान् कल्प॑मानान् । \newline
53. कल्प॑मानान् प्र॒जाभ्यः॑ प्र॒जाभ्यः॒ कल्प॑माना॒न् कल्प॑मानान् प्र॒जाभ्यः॑ । \newline
54. प्र॒जाभ्यो ऽन्वनु॑ प्र॒जाभ्यः॑ प्र॒जाभ्यो ऽनु॑ । \newline
55. प्र॒जाभ्य॒ इति॑ प्र - जाभ्यः॑ । \newline
56. अनु॑ कल्पते कल्प॒ते ऽन्वनु॑ कल्पते । \newline
57. क॒ल्प॒ते॒ कल्प॑न्ते॒ कल्प॑न्ते कल्पते कल्पते॒ कल्प॑न्ते । \newline
58. कल्प॑न्ते ऽस्मा अस्मै॒ कल्प॑न्ते॒ कल्प॑न्ते ऽस्मै । \newline

\textbf{Ghana Paata } \newline

1. प्र॒जाप॑तिः प्र॒जाः प्र॒जाः प्र॒जाप॑तिः प्र॒जाप॑तिः प्र॒जा अ॑सृजता सृजत प्र॒जाः प्र॒जाप॑तिः प्र॒जाप॑तिः प्र॒जा अ॑सृजत । \newline
2. प्र॒जाप॑ति॒रिति॑ प्र॒जा - प॒तिः॒ । \newline
3. प्र॒जा अ॑सृजता सृजत प्र॒जाः प्र॒जा अ॑सृजत॒ ता स्ता अ॑सृजत प्र॒जाः प्र॒जा अ॑सृजत॒ ताः । \newline
4. प्र॒जा इति॑ प्र - जाः । \newline
5. अ॒सृ॒ज॒त॒ ता स्ता अ॑सृजता सृजत॒ ताः सृ॒ष्टाः सृ॒ष्टा स्ता अ॑सृजता सृजत॒ ताः सृ॒ष्टाः । \newline
6. ताः सृ॒ष्टाः सृ॒ष्टा स्ता स्ताः सृ॒ष्टाः क्षुध॒म् क्षुधꣳ॑ सृ॒ष्टा स्ता स्ताः सृ॒ष्टाः क्षुध᳚म् । \newline
7. सृ॒ष्टाः क्षुध॒म् क्षुधꣳ॑ सृ॒ष्टाः सृ॒ष्टाः क्षुध॒न् नि नि क्षुधꣳ॑ सृ॒ष्टाः सृ॒ष्टाः क्षुध॒न् नि । \newline
8. क्षुध॒न् नि नि क्षुध॒म् क्षुध॒न् न्या॑यन् नाय॒न् नि क्षुध॒म् क्षुध॒न् न्या॑यन्न् । \newline
9. न्या॑यन् नाय॒न् नि न्या॑य॒न् थ्स स आ॑य॒न् नि न्या॑य॒न् थ्सः । \newline
10. आ॒य॒न् थ्स स आ॑यन् नाय॒न् थ्स ए॒त मे॒तꣳ स आ॑यन् नाय॒न् थ्स ए॒तम् । \newline
11. स ए॒त मे॒तꣳ स स ए॒तम् न॑वरा॒त्रन् न॑वरा॒त्र मे॒तꣳ स स ए॒तम् न॑वरा॒त्रम् । \newline
12. ए॒तम् न॑वरा॒त्रम् न॑वरा॒त्र मे॒त मे॒तम् न॑वरा॒त्र म॑पश्य दपश्यन् नवरा॒त्र मे॒त मे॒तम् न॑वरा॒त्र म॑पश्यत् । \newline
13. न॒व॒रा॒त्र म॑पश्य दपश्यन् नवरा॒त्रन् न॑वरा॒त्र म॑पश्य॒त् तम् त म॑पश्यन् नवरा॒त्रन् न॑वरा॒त्र म॑पश्य॒त् तम् । \newline
14. न॒व॒रा॒त्रमिति॑ नव - रा॒त्रम् । \newline
15. अ॒प॒श्य॒त् तम् त म॑पश्य दपश्य॒त् त मा त म॑पश्य दपश्य॒त् त मा । \newline
16. त मा तम् त मा ऽह॑र दहर॒दा तम् त मा ऽह॑रत् । \newline
17. आ ऽह॑र दहर॒दा ऽह॑र॒त् तेन॒ तेना॑ हर॒दा ऽह॑र॒त् तेन॑ । \newline
18. अ॒ह॒र॒त् तेन॒ तेना॑ हर दहर॒त् तेना॑ यजता यजत॒ तेना॑ हर दहर॒त् तेना॑ यजत । \newline
19. तेना॑ यजता यजत॒ तेन॒ तेना॑ यजत॒ तत॒ स्ततो॑ ऽयजत॒ तेन॒ तेना॑ यजत॒ ततः॑ । \newline
20. अ॒य॒ज॒त॒ तत॒ स्ततो॑ ऽयजता यजत॒ ततो॒ वै वै ततो॑ ऽयजता यजत॒ ततो॒ वै । \newline
21. ततो॒ वै वै तत॒ स्ततो॒ वै प्र॒जाभ्यः॑ प्र॒जाभ्यो॒ वै तत॒ स्ततो॒ वै प्र॒जाभ्यः॑ । \newline
22. वै प्र॒जाभ्यः॑ प्र॒जाभ्यो॒ वै वै प्र॒जाभ्यो॑ ऽकल्पता कल्पत प्र॒जाभ्यो॒ वै वै प्र॒जाभ्यो॑ ऽकल्पत । \newline
23. प्र॒जाभ्यो॑ ऽकल्पता कल्पत प्र॒जाभ्यः॑ प्र॒जाभ्यो॑ ऽकल्पत॒ यर्.हि॒ यर्ह्य॑कल्पत प्र॒जाभ्यः॑ प्र॒जाभ्यो॑ ऽकल्पत॒ यर्.हि॑ । \newline
24. प्र॒जाभ्य॒ इति॑ प्र - जाभ्यः॑ । \newline
25. अ॒क॒ल्प॒त॒ यर्.हि॒ यर्ह्य॑कल्पता कल्पत॒ यर्.हि॑ प्र॒जाः प्र॒जा यर्ह्य॑कल्पता कल्पत॒ यर्.हि॑ प्र॒जाः । \newline
26. यर्.हि॑ प्र॒जाः प्र॒जा यर्.हि॒ यर्.हि॑ प्र॒जाः क्षुध॒म् क्षुध॑म् प्र॒जा यर्.हि॒ यर्.हि॑ प्र॒जाः क्षुध᳚म् । \newline
27. प्र॒जाः क्षुध॒म् क्षुध॑म् प्र॒जाः प्र॒जाः क्षुध॑म् नि॒गच्छे॑युर् नि॒गच्छे॑युः॒ क्षुध॑म् प्र॒जाः प्र॒जाः क्षुध॑म् नि॒गच्छे॑युः । \newline
28. प्र॒जा इति॑ प्र - जाः । \newline
29. क्षुध॑म् नि॒गच्छे॑युर् नि॒गच्छे॑युः॒ क्षुध॒म् क्षुध॑म् नि॒गच्छे॑यु॒ स्तर्.हि॒ तर्.हि॑ नि॒गच्छे॑युः॒ क्षुध॒म् क्षुध॑म् नि॒गच्छे॑यु॒ स्तर्.हि॑ । \newline
30. नि॒गच्छे॑यु॒ स्तर्.हि॒ तर्.हि॑ नि॒गच्छे॑युर् नि॒गच्छे॑यु॒ स्तर्.हि॑ नवरा॒त्रेण॑ नवरा॒त्रेण॒ तर्.हि॑ नि॒गच्छे॑युर् नि॒गच्छे॑यु॒ स्तर्.हि॑ नवरा॒त्रेण॑ । \newline
31. नि॒गच्छे॑यु॒रिति॑ नि - गच्छे॑युः । \newline
32. तर्.हि॑ नवरा॒त्रेण॑ नवरा॒त्रेण॒ तर्.हि॒ तर्.हि॑ नवरा॒त्रेण॑ यजेत यजेत नवरा॒त्रेण॒ तर्.हि॒ तर्.हि॑ नवरा॒त्रेण॑ यजेत । \newline
33. न॒व॒रा॒त्रेण॑ यजेत यजेत नवरा॒त्रेण॑ नवरा॒त्रेण॑ यजेते॒म इ॒मे य॑जेत नवरा॒त्रेण॑ नवरा॒त्रेण॑ 
यजेते॒मे । \newline
34. न॒व॒रा॒त्रेणेति॑ नव - रा॒त्रेण॑ । \newline
35. य॒जे॒ते॒म इ॒मे य॑जेत यजेते॒मे हि हीमे य॑जेत यजेते॒मे हि । \newline
36. इ॒मे हि हीम इ॒मे हि वै वै हीम इ॒मे हि वै । \newline
37. हि वै वै हि हि वा ए॒तासा॑ मे॒तासां॒ ॅवै हि हि वा ए॒तासा᳚म् । \newline
38. वा ए॒तासा॑ मे॒तासां॒ ॅवै वा ए॒तासा᳚म् ॅलो॒का लो॒का ए॒तासां॒ ॅवै वा ए॒तासा᳚म् ॅलो॒काः । \newline
39. ए॒तासा᳚म् ॅलो॒का लो॒का ए॒तासा॑ मे॒तासा᳚म् ॅलो॒का अक्लृ॑प्ता॒ अक्लृ॑प्ता लो॒का ए॒तासा॑ मे॒तासा᳚म् ॅलो॒का अक्लृ॑प्ताः । \newline
40. लो॒का अक्लृ॑प्ता॒ अक्लृ॑प्ता लो॒का लो॒का अक्लृ॑प्ता॒ अथाथा क्लृ॑प्ता लो॒का लो॒का अक्लृ॑प्ता॒ अथ॑ । \newline
41. अक्लृ॑प्ता॒ अथाथा क्लृ॑प्ता॒ अक्लृ॑प्ता॒ अथै॒ता ए॒ता अथाक्लृ॑प्ता॒ अक्लृ॑प्ता॒ अथै॒ताः । \newline
42. अथै॒ता ए॒ता अथा थै॒ताः क्षुध॒म् क्षुध॑ मे॒ता अथा थै॒ताः क्षुध᳚म् । \newline
43. ए॒ताः क्षुध॒म् क्षुध॑ मे॒ता ए॒ताः क्षुध॒न् नि नि क्षुध॑ मे॒ता ए॒ताः क्षुध॒न् नि । \newline
44. क्षुध॒न् नि नि क्षुध॒म् क्षुध॒न् नि ग॑च्छन्ति गच्छन्ति॒ नि क्षुध॒म् क्षुध॒न् नि ग॑च्छन्ति । \newline
45. नि ग॑च्छन्ति गच्छन्ति॒ नि नि ग॑च्छन्ती॒मा नि॒मान् ग॑च्छन्ति॒ नि नि ग॑च्छन्ती॒मान् । \newline
46. ग॒च्छ॒न् ती॒मा नि॒मान् ग॑च्छन्ति गच्छन्ती॒मा ने॒वैवेमान् ग॑च्छन्ति गच्छन्ती॒मा ने॒व । \newline
47. इ॒मा ने॒वैवेमा नि॒मा ने॒वाभ्य॑ आभ्य ए॒वेमा नि॒मा ने॒वाभ्यः॑ । \newline
48. ए॒वाभ्य॑ आभ्य ए॒वै वाभ्यो॑ लो॒कान् ॅलो॒का ना᳚भ्य ए॒वै वाभ्यो॑ लो॒कान् । \newline
49. आ॒भ्यो॒ लो॒कान् ॅलो॒का ना᳚भ्य आभ्यो लो॒कान् क॑ल्पयति कल्पयति लो॒का ना᳚भ्य आभ्यो लो॒कान् क॑ल्पयति । \newline
50. लो॒कान् क॑ल्पयति कल्पयति लो॒कान् ॅलो॒कान् क॑ल्पयति॒ ताꣳ स्तान् क॑ल्पयति लो॒कान् ॅलो॒कान् क॑ल्पयति॒ तान् । \newline
51. क॒ल्प॒य॒ति॒ ताꣳ स्तान् क॑ल्पयति कल्पयति॒ तान् कल्प॑माना॒न् कल्प॑माना॒न् तान् क॑ल्पयति कल्पयति॒ तान् कल्प॑मानान् । \newline
52. तान् कल्प॑माना॒न् कल्प॑मा॒न् ताꣳ स्तान् कल्प॑मानान् प्र॒जाभ्यः॑ प्र॒जाभ्यः॒ कल्प॑माना॒न् ताꣳ स्तान् कल्प॑मानान् प्र॒जाभ्यः॑ । \newline
53. कल्प॑मानान् प्र॒जाभ्यः॑ प्र॒जाभ्यः॒ कल्प॑माना॒न् कल्प॑मानान् प्र॒जाभ्यो ऽन्वनु॑ प्र॒जाभ्यः॒ कल्प॑माना॒न् कल्प॑मानान् प्र॒जाभ्यो ऽनु॑ । \newline
54. प्र॒जाभ्यो ऽन्वनु॑ प्र॒जाभ्यः॑ प्र॒जाभ्यो ऽनु॑ कल्पते कल्प॒ते ऽनु॑ प्र॒जाभ्यः॑ प्र॒जाभ्यो ऽनु॑ कल्पते । \newline
55. प्र॒जाभ्य॒ इति॑ प्र - जाभ्यः॑ । \newline
56. अनु॑ कल्पते कल्प॒ते ऽन्वनु॑ कल्पते॒ कल्प॑न्ते॒ कल्प॑न्ते कल्प॒ते ऽन्वनु॑ कल्पते॒ कल्प॑न्ते । \newline
57. क॒ल्प॒ते॒ कल्प॑न्ते॒ कल्प॑न्ते कल्पते कल्पते॒ कल्प॑न्ते ऽस्मा अस्मै॒ कल्प॑न्ते कल्पते कल्पते॒ कल्प॑न्ते ऽस्मै । \newline
58. कल्प॑न्ते ऽस्मा अस्मै॒ कल्प॑न्ते॒ कल्प॑न्ते ऽस्मा इ॒म इ॒मे᳚ ऽस्मै॒ कल्प॑न्ते॒ कल्प॑न्ते ऽस्मा इ॒मे । \newline
\pagebreak
\markright{ TS 7.2.4.2  \hfill https://www.vedavms.in \hfill}

\section{ TS 7.2.4.2 }

\textbf{TS 7.2.4.2 } \newline
\textbf{Samhita Paata} \newline

ऽस्मा इ॒मे लो॒का ऊर्जं॑ प्र॒जासु॑ दधाति त्रिरा॒त्रेणै॒वेमं ॅलो॒कं क॑ल्पयति त्रिरा॒त्रेणा॒न्तरि॑क्षं त्रिरा॒त्रेणा॒मुं ॅलो॒कं ॅयथा॑ गु॒णे ग॒णम॒न्वस्य॑त्ये॒वमे॒व तल्लो॒के लो॒कमन्व॑स्यति॒ धृत्या॒ अशि॑थिलंभावाय॒ ज्योति॒र्गौरायु॒रिति॑ ज्ञा॒ताः स्तोमा॑ भवन्ती॒यं ॅवाव ज्योति॑र॒न्तरि॑क्षं॒ गौर॒सावायु॑रे॒ष्वे॑व लो॒केषु॒ प्रति॑ तिष्ठन्ति॒ ज्ञात्रं॑ प्र॒जानां᳚-[  ] \newline

\textbf{Pada Paata} \newline

अ॒स्मै॒ । इ॒मे । लो॒काः । ऊर्ज᳚म् । प्र॒जास्विति॑ प्र - जासु॑ । द॒धा॒ति॒ । त्रि॒रा॒त्रेणेति॑ त्रि - रा॒त्रेण॑ । ए॒व । इ॒मम् । लो॒कम् । क॒ल्प॒य॒ति॒ । त्रि॒रा॒त्रेणेति॑ त्रि - रा॒त्रेण॑ । अ॒न्तरि॑क्षम् । त्रि॒रा॒त्रेणेति॑ त्रि - रा॒त्रेण॑ । अ॒मुम् । लो॒कम् । यथा᳚ । गु॒णे । गु॒णम् । अ॒न्वस्य॒तीत्य॑नु-अस्य॑ति । ए॒वम् । ए॒व । तत् । लो॒के । लो॒कम् । अन्विति॑ । अ॒स्य॒ति॒ । धृत्यै᳚ । अशि॑थिलंभावा॒येत्यशि॑थिलं - भा॒वा॒य॒ । ज्योतिः॑ । गौः । आयुः॑ । इति॑ । ज्ञा॒ताः । स्तोमाः᳚ । भ॒व॒न्ति॒ । इ॒यम् । वाव । ज्योतिः॑ । अ॒न्तरि॑क्षम् । गौः । अ॒सौ । आयुः॑ । ए॒षु । ए॒व । लो॒केषु॑ । प्रतीति॑ । ति॒ष्ठ॒न्ति॒ । ज्ञात्र᳚म् । प्र॒जाना॒मिति॑ प्र - जाना᳚म् ।  \newline


\textbf{Krama Paata} \newline

अ॒स्मा॒ इ॒मे । इ॒मे लो॒काः । लो॒का ऊर्ज᳚म् । ऊर्ज॑म् प्र॒जासु॑ । प्र॒जासु॑ दधाति । प्र॒जास्विति॑ प्र - जासु॑ । द॒धा॒ति॒ त्रि॒रा॒त्रेण॑ । त्रि॒रा॒त्रेणै॒व । त्रि॒रा॒त्रेणेति॑ त्रि - रा॒त्रेण॑ । ए॒वेमम् । इ॒मम् ॅलो॒कम् । लो॒कम् क॑ल्पयति । क॒ल्प॒य॒ति॒ त्रि॒रा॒त्रेण॑ । त्रि॒रा॒त्रेणा॒न्तरि॑क्षम् । त्रि॒रा॒त्रेणेति॑ त्रि - रा॒त्रेण॑ । अ॒न्तरि॑क्षम् त्रिरा॒त्रेण॑ । त्रि॒रा॒त्रेणा॒मुम् । त्रि॒रा॒त्रेणेति॑ त्रि - रा॒त्रेण॑ । अ॒मुम् ॅलो॒कम् । लो॒कम् ॅयथा᳚ । यथा॑ गु॒णे । गु॒णे गु॒णम् । गु॒णम॒न्वस्य॑ति । अ॒न्वस्य॑त्ये॒वम् । अ॒न्वस्य॒तीत्य॑नु - अस्य॑ति । ए॒वमे॒व । ए॒व तत् । तल्लो॒के । लो॒के लो॒कम् । लो॒कमनु॑ । अन्व॑स्यति । अ॒स्य॒ति॒ धृत्यै᳚ । धृत्या॒ अशि॑थिलम्भावाय । अशि॑थिलम्भावाय॒ ज्योतिः॑ । अशि॑थिलम्भावा॒येत्यशि॑थिलम् - भा॒वा॒य॒ । ज्योति॒र् गौः । गौरायुः॑ । आयु॒रिति॑ । इति॑ ज्ञा॒ताः । ज्ञा॒ताः स्तोमाः᳚ । स्तोमा॑ भवन्ति । भ॒व॒न्ती॒यम् । इ॒यम् ॅवाव । वाव ज्योतिः॑ । ज्योति॑र॒न्तरि॑क्षम् । अ॒न्तरि॑क्ष॒म् गौः । गौर॒सौ । अ॒सावायुः॑ । आयु॑रे॒षु । ए॒ष्वे॑व । ए॒व लो॒केषु॑ । लो॒केषु॒ प्रति॑ । प्रति॑ तिष्ठन्ति । ति॒ष्ठ॒न्ति॒ ज्ञात्र᳚म् । ज्ञात्र॑म् प्र॒जाना᳚म् ( ) । प्र॒जाना᳚म् गच्छति । प्र॒जाना॒मिति॑ प्र - जाना᳚म् \newline

\textbf{Jatai Paata} \newline

1. अ॒स्मा॒ इ॒म इ॒मे᳚ ऽस्मा अस्मा इ॒मे । \newline
2. इ॒मे लो॒का लो॒का इ॒म इ॒मे लो॒काः । \newline
3. लो॒का ऊर्ज॒ मूर्ज॑म् ॅलो॒का लो॒का ऊर्ज᳚म् । \newline
4. ऊर्ज॑म् प्र॒जासु॑ प्र॒जासूर्ज॒ मूर्ज॑म् प्र॒जासु॑ । \newline
5. प्र॒जासु॑ दधाति दधाति प्र॒जासु॑ प्र॒जासु॑ दधाति । \newline
6. प्र॒जास्विति॑ प्र - जासु॑ । \newline
7. द॒धा॒ति॒ त्रि॒रा॒त्रेण॑ त्रिरा॒त्रेण॑ दधाति दधाति त्रिरा॒त्रेण॑ । \newline
8. त्रि॒रा॒त्रे णै॒वैव त्रि॑रा॒त्रेण॑ त्रिरा॒त्रे णै॒व । \newline
9. त्रि॒रा॒त्रेणेति॑ त्रि - रा॒त्रेण॑ । \newline
10. ए॒वेम मि॒म मे॒वैवेमम् । \newline
11. इ॒मम् ॅलो॒कम् ॅलो॒क मि॒म मि॒मम् ॅलो॒कम् । \newline
12. लो॒कम् क॑ल्पयति कल्पयति लो॒कम् ॅलो॒कम् क॑ल्पयति । \newline
13. क॒ल्प॒य॒ति॒ त्रि॒रा॒त्रेण॑ त्रिरा॒त्रेण॑ कल्पयति कल्पयति त्रिरा॒त्रेण॑ । \newline
14. त्रि॒रा॒त्रेणा॒ न्तरि॑क्ष म॒न्तरि॑क्षम् त्रिरा॒त्रेण॑ त्रिरा॒त्रेणा॒ न्तरि॑क्षम् । \newline
15. त्रि॒रा॒त्रेणेति॑ त्रि - रा॒त्रेण॑ । \newline
16. अ॒न्तरि॑क्षम् त्रिरा॒त्रेण॑ त्रिरा॒त्रेणा॒ न्तरि॑क्ष म॒न्तरि॑क्षम् त्रिरा॒त्रेण॑ । \newline
17. त्रि॒रा॒त्रे णा॒मु म॒मुम् त्रि॑रा॒त्रेण॑ त्रिरा॒त्रे णा॒मुम् । \newline
18. त्रि॒रा॒त्रेणेति॑ त्रि - रा॒त्रेण॑ । \newline
19. अ॒मुम् ॅलो॒कम् ॅलो॒क म॒मु म॒मुम् ॅलो॒कम् । \newline
20. लो॒कं ॅयथा॒ यथा॑ लो॒कम् ॅलो॒कं ॅयथा᳚ । \newline
21. यथा॑ गु॒णे गु॒णे यथा॒ यथा॑ गु॒णे । \newline
22. गु॒णे गु॒णम् गु॒णम् गु॒णे गु॒णे गु॒णम् । \newline
23. गु॒ण म॒न्वस्य॑ त्य॒न्वस्य॑ति गु॒णम् गु॒ण म॒न्वस्य॑ति । \newline
24. अ॒न्वस्य॑ त्ये॒व मे॒व म॒न्वस्य॑ त्य॒न्वस्य॑ त्ये॒वम् । \newline
25. अ॒न्वस्य॒तीत्य॑नु - अस्य॑ति । \newline
26. ए॒व मे॒वै वैव मे॒व मे॒व । \newline
27. ए॒व तत् तदे॒ वैव तत् । \newline
28. तल्लो॒के लो॒के तत् तल्लो॒के । \newline
29. लो॒के लो॒कम् ॅलो॒कम् ॅलो॒के लो॒के लो॒कम् । \newline
30. लो॒क मन्वनु॑ लो॒कम् ॅलो॒क मनु॑ । \newline
31. अन्व॑स्य त्यस्य॒ त्यन् वन् व॑स्यति । \newline
32. अ॒स्य॒ति॒ धृत्यै॒ धृत्या॑ अस्यत्य स्यति॒ धृत्यै᳚ । \newline
33. धृत्या॒ अशि॑थिलंभावा॒या शि॑थिलंभावाय॒ धृत्यै॒ धृत्या॒ अशि॑थिलंभावाय । \newline
34. अशि॑थिलंभावाय॒ ज्योति॒र् ज्योति॒ रशि॑थिलंभावा॒या शि॑थिलंभावाय॒ ज्योतिः॑ । \newline
35. अशि॑थिलंभावा॒येत्यशि॑थिलं - भा॒वा॒य॒ । \newline
36. ज्योति॒र् गौर् गौर् ज्योति॒र् ज्योति॒र् गौः । \newline
37. गौरायु॒ रायु॒र् गौर् गौरायुः॑ । \newline
38. आयु॒ रिती त्यायु॒ रायु॒ रिति॑ । \newline
39. इति॑ ज्ञा॒ता ज्ञा॒ता इतीति॑ ज्ञा॒ताः । \newline
40. ज्ञा॒ताः स्तोमाः॒ स्तोमा᳚ ज्ञा॒ता ज्ञा॒ताः स्तोमाः᳚ । \newline
41. स्तोमा॑ भवन्ति भवन्ति॒ स्तोमाः॒ स्तोमा॑ भवन्ति । \newline
42. भ॒व॒न्ती॒य मि॒यम् भ॑वन्ति भवन्ती॒यम् । \newline
43. इ॒यं ॅवाव वावेय मि॒यं ॅवाव । \newline
44. वाव ज्योति॒र् ज्योति॒र् वाव वाव ज्योतिः॑ । \newline
45. ज्योति॑ र॒न्तरि॑क्ष म॒न्तरि॑क्ष॒म् ज्योति॒र् ज्योति॑ र॒न्तरि॑क्षम् । \newline
46. अ॒न्तरि॑क्ष॒म् गौर् गौ र॒न्तरि॑क्ष म॒न्तरि॑क्ष॒म् गौः । \newline
47. गौ र॒सा व॒सौ गौर् गौ र॒सौ । \newline
48. अ॒सा वायु॒ रायु॑ र॒सा व॒सा वायुः॑ । \newline
49. आयु॑ रे॒ष्वे᳚ ष्वायु॒ रायु॑ रे॒षु । \newline
50. ए॒ष्वे॑वै वैष्वे᳚(1॒)ष्वे॑व । \newline
51. ए॒व लो॒केषु॑ लो॒के ष्वे॒वैव लो॒केषु॑ । \newline
52. लो॒केषु॒ प्रति॒ प्रति॑ लो॒केषु॑ लो॒केषु॒ प्रति॑ । \newline
53. प्रति॑ तिष्ठन्ति तिष्ठन्ति॒ प्रति॒ प्रति॑ तिष्ठन्ति । \newline
54. ति॒ष्ठ॒न्ति॒ ज्ञात्र॒म्. ज्ञात्र॑म् तिष्ठन्ति तिष्ठन्ति॒ ज्ञात्र᳚म् । \newline
55. ज्ञात्र॑म् प्र॒जाना᳚म् प्र॒जाना॒म्. ज्ञात्र॒म्. ज्ञात्र॑म् प्र॒जाना᳚म् । \newline
56. प्र॒जाना᳚म् गच्छति गच्छति प्र॒जाना᳚म् प्र॒जाना᳚म् गच्छति । \newline
57. प्र॒जाना॒मिति॑ प्र - जाना᳚म् । \newline

\textbf{Ghana Paata } \newline

1. अ॒स्मा॒ इ॒म इ॒मे᳚ ऽस्मा अस्मा इ॒मे लो॒का लो॒का इ॒मे᳚ ऽस्मा अस्मा इ॒मे लो॒काः । \newline
2. इ॒मे लो॒का लो॒का इ॒म इ॒मे लो॒का ऊर्ज॒ मूर्ज॑म् ॅलो॒का इ॒म इ॒मे लो॒का ऊर्ज᳚म् । \newline
3. लो॒का ऊर्ज॒ मूर्ज॑म् ॅलो॒का लो॒का ऊर्ज॑म् प्र॒जासु॑ प्र॒जासूर्ज॑म् ॅलो॒का लो॒का ऊर्ज॑म् प्र॒जासु॑ । \newline
4. ऊर्ज॑म् प्र॒जासु॑ प्र॒जासूर्ज॒ मूर्ज॑म् प्र॒जासु॑ दधाति दधाति प्र॒जासूर्ज॒ मूर्ज॑म् प्र॒जासु॑ दधाति । \newline
5. प्र॒जासु॑ दधाति दधाति प्र॒जासु॑ प्र॒जासु॑ दधाति त्रिरा॒त्रेण॑ त्रिरा॒त्रेण॑ दधाति प्र॒जासु॑ प्र॒जासु॑ दधाति त्रिरा॒त्रेण॑ । \newline
6. प्र॒जास्विति॑ प्र - जासु॑ । \newline
7. द॒धा॒ति॒ त्रि॒रा॒त्रेण॑ त्रिरा॒त्रेण॑ दधाति दधाति त्रिरा॒त्रे णै॒वैव त्रि॑रा॒त्रेण॑ दधाति दधाति त्रिरा॒त्रे णै॒व । \newline
8. त्रि॒रा॒त्रे णै॒वैव त्रि॑रा॒त्रेण॑ त्रिरा॒त्रे णै॒वेम मि॒म मे॒व त्रि॑रा॒त्रेण॑ त्रिरा॒त्रे णै॒वेमम् । \newline
9. त्रि॒रा॒त्रेणेति॑ त्रि - रा॒त्रेण॑ । \newline
10. ए॒वेम मि॒म मे॒वैवे मम् ॅलो॒कम् ॅलो॒क मि॒म मे॒वैवेमम् ॅलो॒कम् । \newline
11. इ॒मम् ॅलो॒कम् ॅलो॒क मि॒म मि॒मम् ॅलो॒कम् क॑ल्पयति कल्पयति लो॒क मि॒म मि॒मम् ॅलो॒कम् क॑ल्पयति । \newline
12. लो॒कम् क॑ल्पयति कल्पयति लो॒कम् ॅलो॒कम् क॑ल्पयति त्रिरा॒त्रेण॑ त्रिरा॒त्रेण॑ कल्पयति लो॒कम् ॅलो॒कम् क॑ल्पयति त्रिरा॒त्रेण॑ । \newline
13. क॒ल्प॒य॒ति॒ त्रि॒रा॒त्रेण॑ त्रिरा॒त्रेण॑ कल्पयति कल्पयति त्रिरा॒त्रेणा॒ न्तरि॑क्ष म॒न्तरि॑क्षम् त्रिरा॒त्रेण॑ कल्पयति कल्पयति त्रिरा॒त्रेणा॒ न्तरि॑क्षम् । \newline
14. त्रि॒रा॒त्रेणा॒ न्तरि॑क्ष म॒न्तरि॑क्षम् त्रिरा॒त्रेण॑ त्रिरा॒त्रेणा॒ न्तरि॑क्षम् त्रिरा॒त्रेण॑ त्रिरा॒त्रेणा॒ न्तरि॑क्षम् त्रिरा॒त्रेण॑ त्रिरा॒त्रेणा॒ न्तरि॑क्षम् त्रिरा॒त्रेण॑ । \newline
15. त्रि॒रा॒त्रेणेति॑ त्रि - रा॒त्रेण॑ । \newline
16. अ॒न्तरि॑क्षम् त्रिरा॒त्रेण॑ त्रिरा॒त्रेणा॒ न्तरि॑क्ष म॒न्तरि॑क्षम् त्रिरा॒त्रे णा॒मु म॒मुम् त्रि॑रा॒त्रेणा॒ न्तरि॑क्ष म॒न्तरि॑क्षम् त्रिरा॒त्रे णा॒मुम् । \newline
17. त्रि॒रा॒त्रे णा॒मु म॒मुम् त्रि॑रा॒त्रेण॑ त्रिरा॒त्रे णा॒मुम् ॅलो॒कम् ॅलो॒क म॒मुम् त्रि॑रा॒त्रेण॑ त्रिरा॒त्रे णा॒मुम् ॅलो॒कम् । \newline
18. त्रि॒रा॒त्रेणेति॑ त्रि - रा॒त्रेण॑ । \newline
19. अ॒मुम् ॅलो॒कम् ॅलो॒क म॒मु म॒मुम् ॅलो॒कं ॅयथा॒ यथा॑ लो॒क म॒मु म॒मुम् ॅलो॒कं ॅयथा᳚ । \newline
20. लो॒कं ॅयथा॒ यथा॑ लो॒कम् ॅलो॒कं ॅयथा॑ गु॒णे गु॒णे यथा॑ लो॒कम् ॅलो॒कं ॅयथा॑ गु॒णे । \newline
21. यथा॑ गु॒णे गु॒णे यथा॒ यथा॑ गु॒णे गु॒णम् गु॒णम् गु॒णे यथा॒ यथा॑ गु॒णे गु॒णम् । \newline
22. गु॒णे गु॒णम् गु॒णम् गु॒णे गु॒णे गु॒ण म॒न्वस्य॑ त्य॒न्वस्य॑ति गु॒णम् गु॒णे गु॒णे गु॒ण म॒न्वस्य॑ति । \newline
23. गु॒ण म॒न्वस्य॑ त्य॒न्वस्य॑ति गु॒णम् गु॒ण म॒न्वस्य॑ त्ये॒व मे॒व म॒न्वस्य॑ति गु॒णम् गु॒ण म॒न्वस्य॑ त्ये॒वम् । \newline
24. अ॒न्वस्य॑ त्ये॒व मे॒व म॒न्वस्य॑ त्य॒न्वस्य॑ त्ये॒व मे॒वै वैव म॒न्वस्य॑ त्य॒न्वस्य॑ त्ये॒व मे॒व । \newline
25. अ॒न्वस्य॒तीत्य॑नु - अस्य॑ति । \newline
26. ए॒व मे॒वै वैव मे॒व मे॒व तत् तदे॒ वैव मे॒व मे॒व तत् । \newline
27. ए॒व तत् तदे॒ वैव तल्लो॒के लो॒के तदे॒ वैव तल्लो॒के । \newline
28. तल्लो॒के लो॒के तत् तल्लो॒के लो॒कम् ॅलो॒कम् ॅलो॒के तत् तल्लो॒के लो॒कम् । \newline
29. लो॒के लो॒कम् ॅलो॒कम् ॅलो॒के लो॒के लो॒क मन्वनु॑ लो॒कम् ॅलो॒के लो॒के लो॒क मनु॑ । \newline
30. लो॒क मन्वनु॑ लो॒कम् ॅलो॒क मन्व॑ स्य त्यस्य॒ त्यनु॑ लो॒कम् ॅलो॒क मन्व॑स्यति । \newline
31. अन्व॑ स्य त्यस्य॒ त्यन् वन् व॑स्यति॒ धृत्यै॒ धृत्या॑ अस्य॒त्यन् वन् व॑स्यति॒ धृत्यै᳚ । \newline
32. अ॒स्य॒ति॒ धृत्यै॒ धृत्या॑ अस्य त्यस्यति॒ धृत्या॒ अशि॑थिलंभावा॒या शि॑थिलंभावाय॒ धृत्या॑ अस्य त्यस्यति॒ धृत्या॒ अशि॑थिलंभावाय । \newline
33. धृत्या॒ अशि॑थिलंभावा॒या शि॑थिलंभावाय॒ धृत्यै॒ धृत्या॒ अशि॑थिलंभावाय॒ ज्योति॒र् ज्योति॒ रशि॑थिलंभावाय॒ धृत्यै॒ धृत्या॒ अशि॑थिलंभावाय॒ ज्योतिः॑ । \newline
34. अशि॑थिलंभावाय॒ ज्योति॒र् ज्योति॒ रशि॑थिलंभावा॒या शि॑थिलंभावाय॒ ज्योति॒र् गौर् गौर् ज्योति॒ रशि॑थिलंभावा॒या शि॑थिलंभावाय॒ ज्योति॒र् गौः । \newline
35. अशि॑थिलंभावा॒येत्यशि॑थिलं - भा॒वा॒य॒ । \newline
36. ज्योति॒र् गौर् गौर् ज्योति॒र् ज्योति॒र् गौरायु॒ रायु॒र् गौर् ज्योति॒र् ज्योति॒र् गौरायुः॑ । \newline
37. गौरायु॒ रायु॒र् गौर् गौ रायु॒ रिती त्यायु॒र् गौर् गौ रायु॒ रिति॑ । \newline
38. आयु॒ रिती त्यायु॒ रायु॒ रिति॑ ज्ञा॒ता ज्ञा॒ता इत्यायु॒ रायु॒ रिति॑ ज्ञा॒ताः । \newline
39. इति॑ ज्ञा॒ता ज्ञा॒ता इतीति॑ ज्ञा॒ताः स्तोमाः॒ स्तोमा᳚ ज्ञा॒ता इतीति॑ ज्ञा॒ताः स्तोमाः᳚ । \newline
40. ज्ञा॒ताः स्तोमाः॒ स्तोमा᳚ ज्ञा॒ता ज्ञा॒ताः स्तोमा॑ भवन्ति भवन्ति॒ स्तोमा᳚ ज्ञा॒ता ज्ञा॒ताः स्तोमा॑ भवन्ति । \newline
41. स्तोमा॑ भवन्ति भवन्ति॒ स्तोमाः॒ स्तोमा॑ भवन्ती॒य मि॒यम् भ॑वन्ति॒ स्तोमाः॒ स्तोमा॑ भवन्ती॒यम् । \newline
42. भ॒व॒न्ती॒य मि॒यम् भ॑वन्ति भवन्ती॒यं ॅवाव वावेयम् भ॑वन्ति भवन्ती॒यं ॅवाव । \newline
43. इ॒यं ॅवाव वावेय मि॒यं ॅवाव ज्योति॒र् ज्योति॒र् वावेय मि॒यं ॅवाव ज्योतिः॑ । \newline
44. वाव ज्योति॒र् ज्योति॒र् वाव वाव ज्योति॑ र॒न्तरि॑क्ष म॒न्तरि॑क्ष॒म् ज्योति॒र् वाव वाव ज्योति॑ र॒न्तरि॑क्षम् । \newline
45. ज्योति॑ र॒न्तरि॑क्ष म॒न्तरि॑क्ष॒म् ज्योति॒र् ज्योति॑ र॒न्तरि॑क्ष॒म् गौर् गौ र॒न्तरि॑क्ष॒म् ज्योति॒र् ज्योति॑ र॒न्तरि॑क्ष॒म् गौः । \newline
46. अ॒न्तरि॑क्ष॒म् गौर् गौ र॒न्तरि॑क्ष म॒न्तरि॑क्ष॒म् गौर॒सा व॒सौ गौ र॒न्तरि॑क्ष म॒न्तरि॑क्ष॒म् गौर॒सौ । \newline
47. गौ र॒सा व॒सौ गौर् गौ र॒सा वायु॒ रायु॑ र॒सौ गौर् गौ र॒सा वायुः॑ । \newline
48. अ॒सा वायु॒ रायु॑ र॒सा व॒सा वायु॑ रे॒ष्वे᳚ ष्वायु॑ र॒सा व॒सा वायु॑ रे॒षु । \newline
49. आयु॑ रे॒ष्वे᳚ ष्वायु॒ रायु॑ रे॒ष्वे॑वै वैष्वायु॒ रायु॑ रे॒ष्वे॑व । \newline
50. ए॒ष्वे॑वै वैष्वे᳚(1॒)ष्वे॑व लो॒केषु॑ लो॒के ष्वे॒वैष्वे᳚(1॒)ष्वे॑व लो॒केषु॑ । \newline
51. ए॒व लो॒केषु॑ लो॒के ष्वे॒वैव लो॒केषु॒ प्रति॒ प्रति॑ लो॒के ष्वे॒वैव लो॒केषु॒ प्रति॑ । \newline
52. लो॒केषु॒ प्रति॒ प्रति॑ लो॒केषु॑ लो॒केषु॒ प्रति॑ तिष्ठन्ति तिष्ठन्ति॒ प्रति॑ लो॒केषु॑ लो॒केषु॒ प्रति॑ तिष्ठन्ति । \newline
53. प्रति॑ तिष्ठन्ति तिष्ठन्ति॒ प्रति॒ प्रति॑ तिष्ठन्ति॒ ज्ञात्र॒म्. ज्ञात्र॑म् तिष्ठन्ति॒ प्रति॒ प्रति॑ तिष्ठन्ति॒ ज्ञात्र᳚म् । \newline
54. ति॒ष्ठ॒न्ति॒ ज्ञात्र॒म्. ज्ञात्र॑म् तिष्ठन्ति तिष्ठन्ति॒ ज्ञात्र॑म् प्र॒जाना᳚म् प्र॒जाना॒म्. ज्ञात्र॑म् तिष्ठन्ति तिष्ठन्ति॒ ज्ञात्र॑म् प्र॒जाना᳚म् । \newline
55. ज्ञात्र॑म् प्र॒जाना᳚म् प्र॒जाना॒म्. ज्ञात्र॒म्. ज्ञात्र॑म् प्र॒जाना᳚म् गच्छति गच्छति प्र॒जाना॒म्. ज्ञात्र॒म्. ज्ञात्र॑म् प्र॒जाना᳚म् गच्छति । \newline
56. प्र॒जाना᳚म् गच्छति गच्छति प्र॒जाना᳚म् प्र॒जाना᳚म् गच्छति नवरा॒त्रो न॑वरा॒त्रो ग॑च्छति प्र॒जाना᳚म् प्र॒जाना᳚म् गच्छति नवरा॒त्रः । \newline
57. प्र॒जाना॒मिति॑ प्र - जाना᳚म् । \newline
\pagebreak
\markright{ TS 7.2.4.3  \hfill https://www.vedavms.in \hfill}

\section{ TS 7.2.4.3 }

\textbf{TS 7.2.4.3 } \newline
\textbf{Samhita Paata} \newline

गच्छति नवरा॒त्रो भ॑वत्यभिपू॒र्वमे॒वाऽस्मि॒न् तेजो॑ दधाति॒ यो ज्योगा॑मयावी॒ स्याथ् स न॑वरा॒त्रेण॑ यजेत प्रा॒णा हि वा ए॒तस्या धृ॑ता॒ अथै॒तस्य॒ ज्योगा॑मयति प्रा॒णाने॒वास्मि॑न् दाधारो॒त यदी॒तासु॒र्भव॑ति॒ जीव॑त्ये॒व ॥ \newline

\textbf{Pada Paata} \newline

ग॒च्छ॒ति॒ । न॒व॒रा॒त्र इति॑ नव - रा॒त्रः । भ॒व॒ति॒ । अ॒भि॒पू॒र्वमित्य॑भि-पू॒र्वम् । ए॒व । अ॒स्मि॒न्न् । तेजः॑ । द॒धा॒ति॒ । यः । ज्योगा॑मया॒वीति॒ ज्योक् - आ॒म॒या॒वी॒ । स्यात् । सः । न॒व॒रा॒त्रेणेति॑ नव - रा॒त्रेण॑ । य॒जे॒त॒ । प्रा॒णा इति॑ प्र-अ॒नाः । हि । वै । ए॒तस्य॑ । अधृ॑ताः । अथ॑ । ए॒तस्य॑ । ज्योक् । आ॒म॒य॒ति॒ । प्रा॒णानिति॑ प्र - अ॒नान् । ए॒व । अ॒स्मि॒न्न् । दा॒धा॒र॒ । उ॒त । यदि॑ । इ॒तासु॒रिती॒त - अ॒सुः॒ । भव॑ति । जीव॑ति । ए॒व ॥  \newline


\textbf{Krama Paata} \newline

ग॒च्छ॒ति॒ न॒व॒रा॒त्रः । न॒व॒रा॒त्रो भ॑वति । न॒व॒रा॒त्र इति॑ नव - रा॒त्रः । भ॒व॒त्य॒भि॒पू॒र्वम् । अ॒भि॒पू॒र्वमे॒व । अ॒भि॒पू॒र्वमित्य॑भि - पू॒र्वम् । ए॒वास्मिन्न्॑ । अ॒स्मि॒न् तेजः॑ । तेजो॑ दधाति । द॒धा॒ति॒ यः । यो ज्योगा॑मयावी । ज्योगा॑मयावी॒ स्यात् । ज्योगा॑मया॒वीति॒ ज्योक् - आ॒म॒या॒वी॒ । स्याथ् सः । स न॑वरा॒त्रेण॑ । न॒व॒रा॒त्रेण॑ यजेत । न॒व॒रा॒त्रेणेति॑ नव - रा॒त्रेण॑ । य॒जे॒त॒ प्रा॒णाः । प्रा॒णा हि । प्रा॒णा इति॑ प्र - अ॒नाः । हि वै । वा ए॒तस्य॑ । ए॒तस्याधृ॑ताः । अधृ॑ता॒ अथ॑ । अथै॒तस्य॑ । ए॒तस्य॒ ज्योक् । ज्योगा॑मयति । आ॒म॒य॒ति॒ प्रा॒णान् । प्रा॒णाने॒व । प्रा॒णानिति॑ प्र - अ॒नान् । ए॒वास्मिन्न्॑ । अ॒स्मि॒न् दा॒धा॒र॒ । दा॒धा॒रो॒त । उ॒त यदि॑ । यदी॒तासुः॑ । इ॒तासु॒र् भव॑ति । इ॒तासु॒रिती॒त - अ॒सुः॒ । भव॑ति॒ जीव॑ति । जीव॑त्ये॒व । ए॒वेत्ये॒व । \newline

\textbf{Jatai Paata} \newline

1. ग॒च्छ॒ति॒ न॒व॒रा॒त्रो न॑वरा॒त्रो ग॑च्छति गच्छति नवरा॒त्रः । \newline
2. न॒व॒रा॒त्रो भ॑वति भवति नवरा॒त्रो न॑वरा॒त्रो भ॑वति । \newline
3. न॒व॒रा॒त्र इति॑ नव - रा॒त्रः । \newline
4. भ॒व॒ त्य॒भि॒पू॒र्व म॑भिपू॒र्वम् भ॑वति भव त्यभिपू॒र्वम् । \newline
5. अ॒भि॒पू॒र्व मे॒वै वाभि॑पू॒र्व म॑भिपू॒र्व मे॒व । \newline
6. अ॒भि॒पू॒र्वमित्य॑भि - पू॒र्वम् । \newline
7. ए॒वास्मि॑न् नस्मिन् ने॒वै वास्मिन्न्॑ । \newline
8. अ॒स्मि॒न् तेज॒ स्तेजो᳚ ऽस्मिन् नस्मि॒न् तेजः॑ । \newline
9. तेजो॑ दधाति दधाति॒ तेज॒ स्तेजो॑ दधाति । \newline
10. द॒धा॒ति॒ यो यो द॑धाति दधाति॒ यः । \newline
11. यो ज्योगा॑मयावी॒ ज्योगा॑मयावी॒ यो यो ज्योगा॑मयावी । \newline
12. ज्योगा॑मयावी॒ स्याथ् स्याज् ज्योगा॑मयावी॒ ज्योगा॑मयावी॒ स्यात् । \newline
13. ज्योगा॑मया॒वीति॒ ज्योक् - आ॒म॒या॒वी॒ । \newline
14. स्याथ् स स स्याथ् स्याथ् सः । \newline
15. स न॑वरा॒त्रेण॑ नवरा॒त्रेण॒ स स न॑वरा॒त्रेण॑ । \newline
16. न॒व॒रा॒त्रेण॑ यजेत यजेत नवरा॒त्रेण॑ नवरा॒त्रेण॑ यजेत । \newline
17. न॒व॒रा॒त्रेणेति॑ नव - रा॒त्रेण॑ । \newline
18. य॒जे॒त॒ प्रा॒णाः प्रा॒णा य॑जेत यजेत प्रा॒णाः । \newline
19. प्रा॒णा हि हि प्रा॒णाः प्रा॒णा हि । \newline
20. प्रा॒णा इति॑ प्र - अ॒नाः । \newline
21. हि वै वै हि हि वै । \newline
22. वा ए॒त स्यै॒तस्य॒ वै वा ए॒तस्य॑ । \newline
23. ए॒तस्या धृ॑ता॒ अधृ॑ता ए॒त स्यै॒तस्या धृ॑ताः । \newline
24. अधृ॑ता॒ अथाथा धृ॑ता॒ अधृ॑ता॒ अथ॑ । \newline
25. अथै॒ तस्यै॒तस्या थाथै॒तस्य॑ । \newline
26. ए॒तस्य॒ ज्योग् ज्योगे॒ तस्यै॒तस्य॒ ज्योक् । \newline
27. ज्योगा॑मय त्यामयति॒ ज्योग् ज्योगा॑मयति । \newline
28. आ॒म॒य॒ति॒ प्रा॒णान् प्रा॒णा ना॑मय त्यामयति प्रा॒णान् । \newline
29. प्रा॒णा ने॒वैव प्रा॒णान् प्रा॒णा ने॒व । \newline
30. प्रा॒णानिति॑ प्र - अ॒नान् । \newline
31. ए॒वास्मि॑न् नस्मिन् ने॒वै वास्मिन्न्॑ । \newline
32. अ॒स्मि॒न् दा॒धा॒र॒ दा॒धा॒ रा॒स्मि॒न् न॒स्मि॒न् दा॒धा॒र॒ । \newline
33. दा॒धा॒ रो॒तोत दा॑धार दाधा रो॒त । \newline
34. उ॒त यदि॒ यद्यु॒तोत यदि॑ । \newline
35. यदी॒तासु॑ रि॒तासु॒र् यदि॒ यदी॒तासुः॑ । \newline
36. इ॒तासु॒र् भव॑ति॒ भव॑ती॒तासु॑ रि॒तासु॒र् भव॑ति । \newline
37. इ॒तासु॒रिती॒त - अ॒सुः॒ । \newline
38. भव॑ति॒ जीव॑ति॒ जीव॑ति॒ भव॑ति॒ भव॑ति॒ जीव॑ति । \newline
39. जीव॑ त्ये॒वैव जीव॑ति॒ जीव॑त्ये॒व । \newline
40. ए॒वेत्ये॒व । \newline

\textbf{Ghana Paata } \newline

1. ग॒च्छ॒ति॒ न॒व॒रा॒त्रो न॑वरा॒त्रो ग॑च्छति गच्छति नवरा॒त्रो भ॑वति भवति नवरा॒त्रो ग॑च्छति गच्छति नवरा॒त्रो भ॑वति । \newline
2. न॒व॒रा॒त्रो भ॑वति भवति नवरा॒त्रो न॑वरा॒त्रो भ॑व त्यभिपू॒र्व म॑भिपू॒र्वम् भ॑वति नवरा॒त्रो न॑वरा॒त्रो भ॑व त्यभिपू॒र्वम् । \newline
3. न॒व॒रा॒त्र इति॑ नव - रा॒त्रः । \newline
4. भ॒व॒ त्य॒भि॒पू॒र्व म॑भिपू॒र्वम् भ॑वति भव त्यभिपू॒र्व मे॒वैवा भि॑पू॒र्वम् भ॑वति भव त्यभिपू॒र्व मे॒व । \newline
5. अ॒भि॒पू॒र्व मे॒वैवा भि॑पू॒र्व म॑भिपू॒र्व मे॒वास्मि॑न् नस्मिन् ने॒वा भि॑पू॒र्व म॑भिपू॒र्व मे॒वास्मिन्न्॑ । \newline
6. अ॒भि॒पू॒र्वमित्य॑भि - पू॒र्वम् । \newline
7. ए॒वास्मि॑न् नस्मिन् ने॒वै वास्मि॒न् तेज॒ स्तेजो᳚ ऽस्मिन् ने॒वै वास्मि॒न् तेजः॑ । \newline
8. अ॒स्मि॒न् तेज॒ स्तेजो᳚ ऽस्मिन् नस्मि॒न् तेजो॑ दधाति दधाति॒ तेजो᳚ ऽस्मिन् नस्मि॒न् तेजो॑ दधाति । \newline
9. तेजो॑ दधाति दधाति॒ तेज॒ स्तेजो॑ दधाति॒ यो यो द॑धाति॒ तेज॒ स्तेजो॑ दधाति॒ यः । \newline
10. द॒धा॒ति॒ यो यो द॑धाति दधाति॒ यो ज्योगा॑मयावी॒ ज्योगा॑मयावी॒ यो द॑धाति दधाति॒ यो ज्योगा॑मयावी । \newline
11. यो ज्योगा॑मयावी॒ ज्योगा॑मयावी॒ यो यो ज्योगा॑मयावी॒ स्याथ् स्याज् ज्योगा॑मयावी॒ यो यो ज्योगा॑मयावी॒ स्यात् । \newline
12. ज्योगा॑मयावी॒ स्याथ् स्याज् ज्योगा॑मयावी॒ ज्योगा॑मयावी॒ स्याथ् स स स्याज् ज्योगा॑मयावी॒ ज्योगा॑मयावी॒ स्याथ् सः । \newline
13. ज्योगा॑मया॒वीति॒ ज्योक् - आ॒म॒या॒वी॒ । \newline
14. स्याथ् स स स्याथ् स्याथ् स न॑वरा॒त्रेण॑ नवरा॒त्रेण॒ स स्याथ् स्याथ् स न॑वरा॒त्रेण॑ । \newline
15. स न॑वरा॒त्रेण॑ नवरा॒त्रेण॒ स स न॑वरा॒त्रेण॑ यजेत यजेत नवरा॒त्रेण॒ स स न॑वरा॒त्रेण॑ यजेत । \newline
16. न॒व॒रा॒त्रेण॑ यजेत यजेत नवरा॒त्रेण॑ नवरा॒त्रेण॑ यजेत प्रा॒णाः प्रा॒णा य॑जेत नवरा॒त्रेण॑ नवरा॒त्रेण॑ यजेत प्रा॒णाः । \newline
17. न॒व॒रा॒त्रेणेति॑ नव - रा॒त्रेण॑ । \newline
18. य॒जे॒त॒ प्रा॒णाः प्रा॒णा य॑जेत यजेत प्रा॒णा हि हि प्रा॒णा य॑जेत यजेत प्रा॒णा हि । \newline
19. प्रा॒णा हि हि प्रा॒णाः प्रा॒णा हि वै वै हि प्रा॒णाः प्रा॒णा हि वै । \newline
20. प्रा॒णा इति॑ प्र - अ॒नाः । \newline
21. हि वै वै हि हि वा ए॒त स्यै॒तस्य॒ वै हि हि वा ए॒तस्य॑ । \newline
22. वा ए॒त स्यै॒तस्य॒ वै वा ए॒तस्या धृ॑ता॒ अधृ॑ता ए॒तस्य॒ वै वा ए॒तस्या धृ॑ताः । \newline
23. ए॒तस्या धृ॑ता॒ अधृ॑ता ए॒त स्यै॒तस्या धृ॑ता॒ अथाथा धृ॑ता ए॒त स्यै॒तस्या धृ॑ता॒ अथ॑ । \newline
24. अधृ॑ता॒ अथाथा धृ॑ता॒ अधृ॑ता॒ अथै॒त स्यै॒तस्याथा धृ॑ता॒ अधृ॑ता॒ अथै॒तस्य॑ । \newline
25. अथै॒तस्यै॒ तस्या थाथै॒तस्य॒ ज्योग् ज्यो गे॒तस्या थाथै॒तस्य॒ ज्योक् । \newline
26. ए॒तस्य॒ ज्योग् ज्योगे॒त स्यै॒तस्य॒ ज्योगा॑मय त्यामयति॒ ज्योगे॒त स्यै॒तस्य॒ ज्योगा॑मयति । \newline
27. ज्योगा॑मय त्यामयति॒ ज्योग् ज्योगा॑मयति प्रा॒णान् प्रा॒णा ना॑मयति॒ ज्योग् ज्योगा॑मयति प्रा॒णान् । \newline
28. आ॒म॒य॒ति॒ प्रा॒णान् प्रा॒णा ना॑मय त्यामयति प्रा॒णा ने॒वैव प्रा॒णा ना॑मय त्यामयति प्रा॒णा ने॒व । \newline
29. प्रा॒णा ने॒वैव प्रा॒णान् प्रा॒णा ने॒वास्मि॑न् नस्मिन् ने॒व प्रा॒णान् प्रा॒णा ने॒वास्मिन्न्॑ । \newline
30. प्रा॒णानिति॑ प्र - अ॒नान् । \newline
31. ए॒वास्मि॑न् नस्मिन् ने॒वै वास्मि॑न् दाधार दाधारा स्मिन् ने॒वै वास्मि॑न् दाधार । \newline
32. अ॒स्मि॒न् दा॒धा॒र॒ दा॒धा॒ रा॒स्मि॒न् न॒स्मि॒न् दा॒धा॒ रो॒तोत दा॑धा रास्मिन् नस्मिन् दाधा रो॒त । \newline
33. दा॒धा॒ रो॒तोत दा॑धार दाधा रो॒त यदि॒ यद्यु॒त दा॑धार दाधा रो॒त यदि॑ । \newline
34. उ॒त यदि॒ यद्यु॒तोत यदी॒तासु॑ रि॒तासु॒र् यद्यु॒तोत यदी॒ तासुः॑ । \newline
35. यदी॒ तासु॑ रि॒तासु॒र् यदि॒ यदी॒ तासु॒र् भव॑ति॒ भव॑ती॒ तासु॒र् यदि॒ यदी॒ तासु॒र् भव॑ति । \newline
36. इ॒तासु॒र् भव॑ति॒ भव॑ती॒ तासु॑ रि॒तासु॒र् भव॑ति॒ जीव॑ति॒ जीव॑ति॒ भव॑ती॒ तासु॑ रि॒तासु॒र् भव॑ति॒ जीव॑ति । \newline
37. इ॒तासु॒रिती॒त - अ॒सुः॒ । \newline
38. भव॑ति॒ जीव॑ति॒ जीव॑ति॒ भव॑ति॒ भव॑ति॒ जीव॑ त्ये॒वैव जीव॑ति॒ भव॑ति॒ भव॑ति॒ जीव॑ त्ये॒व । \newline
39. जीव॑ त्ये॒वैव जीव॑ति॒ जीव॑ त्ये॒व । \newline
40. ए॒वेत्ये॒व । \newline
\pagebreak
\markright{ TS 7.2.5.1  \hfill https://www.vedavms.in \hfill}

\section{ TS 7.2.5.1 }

\textbf{TS 7.2.5.1 } \newline
\textbf{Samhita Paata} \newline

प्र॒जाप॑तिरकामयत॒ प्र जा॑ये॒येति॒ स ए॒तं दश॑होतारमपश्य॒त् तम॑जुहो॒त् तेन॑ दशरा॒त्रम॑सृजत॒ तेन॑ दशरा॒त्रेण॒ प्रा जा॑यत दशरा॒त्राय॑ दीक्षि॒ष्यमा॑णो॒ दश॑होतारं जुहुया॒द्-दश॑होत्रै॒व द॑शरा॒त्रꣳ सृ॑जते॒ तेन॑ दशरा॒त्रेण॒ प्र जा॑यते वैरा॒जो वा ए॒ष य॒ज्ञो यद्द॑शरा॒त्रो य ए॒वं ॅवि॒द्वान्-द॑शरा॒त्रेण॒ यज॑ते वि॒राज॑मे॒व ग॑च्छति प्राजाप॒त्यो वा ए॒ष य॒ज्ञो यद्-द॑शरा॒त्रो-[  ] \newline

\textbf{Pada Paata} \newline

प्र॒जाप॑ति॒रिति॑ प्र॒जा - प॒तिः॒ । अ॒का॒म॒य॒त॒ । प्रेति॑ । जा॒ये॒य॒ । इति॑ । सः । ए॒तम् । दश॑होतार॒मिति॒ दश॑ - हो॒ता॒र॒म् । अ॒प॒श्य॒त् । तम् । अ॒जु॒हो॒त् । तेन॑ । द॒श॒रा॒त्रमिति॑ दश - रा॒त्रम् । अ॒सृ॒ज॒त॒ । तेन॑ । द॒श॒रा॒त्रेणेति॑ दश - रा॒त्रेण॑ । प्रेति॑ । अ॒जा॒य॒त॒ । द॒श॒रा॒त्रायेति॑ दश-रा॒त्राय॑ । दी॒क्षि॒ष्यमा॑णः । दश॑होतार॒मिति॒ दश॑ - हो॒ता॒र॒म् । जु॒हु॒या॒त् । दश॑हो॒त्रेति॒ दश॑-हो॒त्रा॒ । ए॒व । द॒श॒रा॒त्रमिति॑ दश-रा॒त्रम् । सृ॒ज॒ते॒ । तेन॑ । द॒श॒रा॒त्रेणेति॑ दश-रा॒त्रेण॑ । प्रेति॑ । जा॒य॒ते॒ । वै॒रा॒जः । वै । ए॒षः । य॒ज्ञ्ः । यत् । द॒श॒रा॒त्र इति॑ दश - रा॒त्रः । यः । ए॒वम् । वि॒द्वान् । द॒श॒रा॒त्रेणेति॑ दश - रा॒त्रेण॑ । यज॑ते । वि॒राज॒मिति॑ वि - राज᳚म् । ए॒व । ग॒च्छ॒ति॒ । प्रा॒जा॒प॒त्य इति॑ प्राजा - प॒त्यः । वै । ए॒षः । य॒ज्ञ्ः । यत् । द॒श॒रा॒त्र इति॑ दश - रा॒त्रः ।  \newline


\textbf{Krama Paata} \newline

प्र॒जाप॑तिरकामयत । प्र॒जाप॑ति॒रिति॑ प्र॒जा - प॒तिः॒ । अ॒का॒म॒य॒त॒ प्र । प्र जा॑येय । जा॒ये॒येति॑ । इति॒ सः । स ए॒तम् । ए॒तम् दश॑होतारम् । दश॑होतारमपश्यत् । दश॑होतार॒मिति॒ दश॑ - हो॒ता॒र॒म् । अ॒प॒श्य॒त् तम् । तम॑जुहोत् । अ॒जु॒हो॒त् तेन॑ । तेन॑ दशरा॒त्रम् । द॒श॒रा॒त्रम॑सृजत । द॒श॒रा॒त्रमिति॑ दश - रा॒त्रम् । अ॒सृ॒ज॒त॒ तेन॑ । तेन॑ दशरा॒त्रेण॑ । द॒श॒रा॒त्रेण॒ प्र । द॒श॒रा॒त्रेणेति॑ दश - रा॒त्रेण॑ । प्राजा॑यत । अ॒जा॒य॒त॒ द॒श॒रा॒त्राय॑ । द॒श॒रा॒त्राय॑ दीक्षि॒ष्यमा॑णः । द॒श॒रा॒त्रायेति॑ दश - रा॒त्राय॑ । दी॒क्षि॒ष्यमा॑णो॒ दश॑होतारम् । दश॑होतारम् जुहुयात् । दश॑होतार॒मिति॒ दश॑ - हो॒ता॒र॒म् । जु॒हु॒या॒द् दश॑होत्रा । दश॑होत्रै॒व । दश॑हो॒त्रेति॒ दश॑ - हो॒त्रा॒ । ए॒व द॑शरा॒त्रम् । द॒श॒रा॒त्रꣳ सृ॑जते । द॒श॒रा॒त्रमिति॑ दश - रा॒त्रम् । सृ॒ज॒ते॒ तेन॑ । तेन॑ दशरा॒त्रेण॑ । द॒श॒रा॒त्रेण॒ प्र । द॒श॒रा॒त्रेणेति॑ दश - रा॒त्रेण॑ । प्र जा॑यते । जा॒य॒ते॒ वै॒रा॒जः । वै॒रा॒जो वै । वा ए॒षः । ए॒ष य॒ज्ञ्ः । य॒ज्ञो यत् । यद् द॑शरा॒त्रः । द॒श॒रा॒त्रो यः । द॒श॒रा॒त्र इति॑ दश - रा॒त्रः । य ए॒वम् । ए॒वम् ॅवि॒द्वान् । वि॒द्वान् द॑शरा॒त्रेण॑ । द॒श॒रा॒त्रेण॒ यज॑ते । द॒श॒रा॒त्रेणेति॑ दश - रा॒त्रेण॑ । यज॑ते वि॒राज᳚म् । वि॒राज॑मे॒व । वि॒राज॒मिति॑ वि - राज᳚म् । ए॒व ग॑च्छति । ग॒च्छ॒ति॒ प्रा॒जा॒प॒त्यः । प्रा॒जा॒प॒त्यो वै । प्रा॒जा॒प॒त्य इति॑ प्राजा - प॒त्यः । वा ए॒षः । ए॒ष य॒ज्ञ्ः । य॒ज्ञो यत् । यद् द॑शरा॒त्रः । द॒श॒रा॒त्रो यः । द॒श॒रा॒त्र इति॑ दश - रा॒त्रः \newline

\textbf{Jatai Paata} \newline

1. प्र॒जाप॑ति रकामयता कामयत प्र॒जाप॑तिः प्र॒जाप॑ति रकामयत । \newline
2. प्र॒जाप॑ति॒रिति॑ प्र॒जा - प॒तिः॒ । \newline
3. अ॒का॒म॒य॒त॒ प्र प्रा का॑मयता कामयत॒ प्र । \newline
4. प्र जा॑येय जायेय॒ प्र प्र जा॑येय । \newline
5. जा॒ये॒येतीति॑ जायेय जाये॒येति॑ । \newline
6. इति॒ स स इतीति॒ सः । \newline
7. स ए॒त मे॒तꣳ स स ए॒तम् । \newline
8. ए॒तम् दश॑होतार॒म् दश॑होतार मे॒त मे॒तम् दश॑होतारम् । \newline
9. दश॑होतार मपश्य दपश्य॒द् दश॑होतार॒म् दश॑होतार मपश्यत् । \newline
10. दश॑होतार॒मिति॒ दश॑ - हो॒ता॒र॒म् । \newline
11. अ॒प॒श्य॒त् तम् त म॑पश्य दपश्य॒त् तम् । \newline
12. त म॑जुहो दजुहो॒त् तम् त म॑जुहोत् । \newline
13. अ॒जु॒हो॒त् तेन॒ तेना॑ जुहो दजुहो॒त् तेन॑ । \newline
14. तेन॑ दशरा॒त्रम् द॑शरा॒त्रम् तेन॒ तेन॑ दशरा॒त्रम् । \newline
15. द॒श॒रा॒त्र म॑सृजता सृजत दशरा॒त्रम् द॑शरा॒त्र म॑सृजत । \newline
16. द॒श॒रा॒त्रमिति॑ दश - रा॒त्रम् । \newline
17. अ॒सृ॒ज॒त॒ तेन॒ तेना॑ सृजता सृजत॒ तेन॑ । \newline
18. तेन॑ दशरा॒त्रेण॑ दशरा॒त्रेण॒ तेन॒ तेन॑ दशरा॒त्रेण॑ । \newline
19. द॒श॒रा॒त्रेण॒ प्र प्र द॑शरा॒त्रेण॑ दशरा॒त्रेण॒ प्र । \newline
20. द॒श॒रा॒त्रेणेति॑ दश - रा॒त्रेण॑ । \newline
21. प्रा जा॑यता जायत॒ प्र प्रा जा॑यत । \newline
22. अ॒जा॒य॒त॒ द॒श॒रा॒त्राय॑ दशरा॒त्राया॑ जायता जायत दशरा॒त्राय॑ । \newline
23. द॒श॒रा॒त्राय॑ दीक्षि॒ष्यमा॑णो दीक्षि॒ष्यमा॑णो दशरा॒त्राय॑ दशरा॒त्राय॑ दीक्षि॒ष्यमा॑णः । \newline
24. द॒श॒रा॒त्रायेति॑ दश - रा॒त्राय॑ । \newline
25. दी॒क्षि॒ष्यमा॑णो॒ दश॑होतार॒म् दश॑होतारम् दीक्षि॒ष्यमा॑णो दीक्षि॒ष्यमा॑णो॒ दश॑होतारम् । \newline
26. दश॑होतारम् जुहुयाज् जुहुया॒द् दश॑होतार॒म् दश॑होतारम् जुहुयात् । \newline
27. दश॑होतार॒मिति॒ दश॑ - हो॒ता॒र॒म् । \newline
28. जु॒हु॒या॒द् दश॑होत्रा॒ दश॑होत्रा जुहुयाज् जुहुया॒द् दश॑होत्रा । \newline
29. दश॑होत्रै॒वैव दश॑होत्रा॒ दश॑होत्रै॒व । \newline
30. दश॑हो॒त्रेति॒ दश॑ - हो॒त्रा॒ । \newline
31. ए॒व द॑शरा॒त्रम् द॑शरा॒त्र मे॒वैव द॑शरा॒त्रम् । \newline
32. द॒श॒रा॒त्रꣳ सृ॑जते सृजते दशरा॒त्रम् द॑शरा॒त्रꣳ सृ॑जते । \newline
33. द॒श॒रा॒त्रमिति॑ दश - रा॒त्रम् । \newline
34. सृ॒ज॒ते॒ तेन॒ तेन॑ सृजते सृजते॒ तेन॑ । \newline
35. तेन॑ दशरा॒त्रेण॑ दशरा॒त्रेण॒ तेन॒ तेन॑ दशरा॒त्रेण॑ । \newline
36. द॒श॒रा॒त्रेण॒ प्र प्र द॑शरा॒त्रेण॑ दशरा॒त्रेण॒ प्र । \newline
37. द॒श॒रा॒त्रेणेति॑ दश - रा॒त्रेण॑ । \newline
38. प्र जा॑यते जायते॒ प्र प्र जा॑यते । \newline
39. जा॒य॒ते॒ वै॒रा॒जो वै॑रा॒जो जा॑यते जायते वैरा॒जः । \newline
40. वै॒रा॒जो वै वै वै॑रा॒जो वै॑रा॒जो वै । \newline
41. वा ए॒ष ए॒ष वै वा ए॒षः । \newline
42. ए॒ष य॒ज्ञो य॒ज्ञ् ए॒ष ए॒ष य॒ज्ञ्ः । \newline
43. य॒ज्ञो यद् यद् य॒ज्ञो य॒ज्ञो यत् । \newline
44. यद् द॑शरा॒त्रो द॑शरा॒त्रो यद् यद् द॑शरा॒त्रः । \newline
45. द॒श॒रा॒त्रो यो यो द॑शरा॒त्रो द॑शरा॒त्रो यः । \newline
46. द॒श॒रा॒त्र इति॑ दश - रा॒त्रः । \newline
47. य ए॒व मे॒वं ॅयो य ए॒वम् । \newline
48. ए॒वं ॅवि॒द्वान्. वि॒द्वा ने॒व मे॒वं ॅवि॒द्वान् । \newline
49. वि॒द्वान् द॑शरा॒त्रेण॑ दशरा॒त्रेण॑ वि॒द्वान्. वि॒द्वान् द॑शरा॒त्रेण॑ । \newline
50. द॒श॒रा॒त्रेण॒ यज॑ते॒ यज॑ते दशरा॒त्रेण॑ दशरा॒त्रेण॒ यज॑ते । \newline
51. द॒श॒रा॒त्रेणेति॑ दश - रा॒त्रेण॑ । \newline
52. यज॑ते वि॒राजं॑ ॅवि॒राजं॒ ॅयज॑ते॒ यज॑ते वि॒राज᳚म् । \newline
53. वि॒राज॑ मे॒वैव वि॒राजं॑ ॅवि॒राज॑ मे॒व । \newline
54. वि॒राज॒मिति॑ वि - राज᳚म् । \newline
55. ए॒व ग॑च्छति गच्छ त्ये॒वैव ग॑च्छति । \newline
56. ग॒च्छ॒ति॒ प्रा॒जा॒प॒त्यः प्रा॑जाप॒त्यो ग॑च्छति गच्छति प्राजाप॒त्यः । \newline
57. प्रा॒जा॒प॒त्यो वै वै प्रा॑जाप॒त्यः प्रा॑जाप॒त्यो वै । \newline
58. प्रा॒जा॒प॒त्य इति॑ प्राजा - प॒त्यः । \newline
59. वा ए॒ष ए॒ष वै वा ए॒षः । \newline
60. ए॒ष य॒ज्ञो य॒ज्ञ् ए॒ष ए॒ष य॒ज्ञ्ः । \newline
61. य॒ज्ञो यद् यद् य॒ज्ञो य॒ज्ञो यत् । \newline
62. यद् द॑शरा॒त्रो द॑शरा॒त्रो यद् यद् द॑शरा॒त्रः । \newline
63. द॒श॒रा॒त्रो यो यो द॑शरा॒त्रो द॑शरा॒त्रो यः । \newline
64. द॒श॒रा॒त्र इति॑ दश - रा॒त्रः । \newline

\textbf{Ghana Paata } \newline

1. प्र॒जाप॑ति रकामयता कामयत प्र॒जाप॑तिः प्र॒जाप॑ति रकामयत॒ प्र प्राका॑मयत प्र॒जाप॑तिः प्र॒जाप॑ति रकामयत॒ प्र । \newline
2. प्र॒जाप॑ति॒रिति॑ प्र॒जा - प॒तिः॒ । \newline
3. अ॒का॒म॒य॒त॒ प्र प्राका॑मयता कामयत॒ प्र जा॑येय जायेय॒ प्राका॑मयता कामयत॒ प्र जा॑येय । \newline
4. प्र जा॑येय जायेय॒ प्र प्र जा॑ये॒येतीति॑ जायेय॒ प्र प्र जा॑ये॒येति॑ । \newline
5. जा॒ये॒येतीति॑ जायेय जाये॒येति॒ स स इति॑ जायेय जाये॒येति॒ सः । \newline
6. इति॒ स स इतीति॒ स ए॒त मे॒तꣳ स इतीति॒ स ए॒तम् । \newline
7. स ए॒त मे॒तꣳ स स ए॒तम् दश॑होतार॒म् दश॑होतार मे॒तꣳ स स ए॒तम् दश॑होतारम् । \newline
8. ए॒तम् दश॑होतार॒म् दश॑होतार मे॒त मे॒तम् दश॑होतार मपश्य दपश्य॒द् दश॑होतार मे॒त मे॒तम् दश॑होतार मपश्यत् । \newline
9. दश॑होतार मपश्य दपश्य॒द् दश॑होतार॒म् दश॑होतार मपश्य॒त् तम् त म॑पश्य॒द् दश॑होतार॒म् दश॑होतार मपश्य॒त् तम् । \newline
10. दश॑होतार॒मिति॒ दश॑ - हो॒ता॒र॒म् । \newline
11. अ॒प॒श्य॒त् तम् त म॑पश्य दपश्य॒त् त म॑जुहो दजुहो॒त् त म॑पश्य दपश्य॒त् त म॑जुहोत् । \newline
12. त म॑जुहो दजुहो॒त् तम् त म॑जुहो॒त् तेन॒ तेना॑ जुहो॒त् तम् त म॑जुहो॒त् तेन॑ । \newline
13. अ॒जु॒हो॒त् तेन॒ तेना॑जुहो दजुहो॒त् तेन॑ दशरा॒त्रम् द॑शरा॒त्रम् तेना॑ जुहो दजुहो॒त् तेन॑ दशरा॒त्रम् । \newline
14. तेन॑ दशरा॒त्रम् द॑शरा॒त्रम् तेन॒ तेन॑ दशरा॒त्र म॑सृजता सृजत दशरा॒त्रम् तेन॒ तेन॑ दशरा॒त्र म॑सृजत । \newline
15. द॒श॒रा॒त्र म॑सृजता सृजत दशरा॒त्रम् द॑शरा॒त्र म॑सृजत॒ तेन॒ तेना॑ सृजत दशरा॒त्रम् द॑शरा॒त्र म॑सृजत॒ तेन॑ । \newline
16. द॒श॒रा॒त्रमिति॑ दश - रा॒त्रम् । \newline
17. अ॒सृ॒ज॒त॒ तेन॒ तेना॑ सृजता सृजत॒ तेन॑ दशरा॒त्रेण॑ दशरा॒त्रेण॒ तेना॑ सृजता सृजत॒ तेन॑ दशरा॒त्रेण॑ । \newline
18. तेन॑ दशरा॒त्रेण॑ दशरा॒त्रेण॒ तेन॒ तेन॑ दशरा॒त्रेण॒ प्र प्र द॑शरा॒त्रेण॒ तेन॒ तेन॑ दशरा॒त्रेण॒ प्र । \newline
19. द॒श॒रा॒त्रेण॒ प्र प्र द॑शरा॒त्रेण॑ दशरा॒त्रेण॒ प्राजा॑यता जायत॒ प्र द॑शरा॒त्रेण॑ दशरा॒त्रेण॒ प्राजा॑यत । \newline
20. द॒श॒रा॒त्रेणेति॑ दश - रा॒त्रेण॑ । \newline
21. प्रा जा॑यता जायत॒ प्र प्राजा॑यत दशरा॒त्राय॑ दशरा॒त्राया॑ जायत॒ प्र प्राजा॑यत दशरा॒त्राय॑ । \newline
22. अ॒जा॒य॒त॒ द॒श॒रा॒त्राय॑ दशरा॒त्राया॑ जायता जायत दशरा॒त्राय॑ दीक्षि॒ष्यमा॑णो दीक्षि॒ष्यमा॑णो दशरा॒त्राया॑ जायता जायत दशरा॒त्राय॑ दीक्षि॒ष्यमा॑णः । \newline
23. द॒श॒रा॒त्राय॑ दीक्षि॒ष्यमा॑णो दीक्षि॒ष्यमा॑णो दशरा॒त्राय॑ दशरा॒त्राय॑ दीक्षि॒ष्यमा॑णो॒ दश॑होतार॒म् दश॑होतारम् दीक्षि॒ष्यमा॑णो दशरा॒त्राय॑ दशरा॒त्राय॑ दीक्षि॒ष्यमा॑णो॒ दश॑होतारम् । \newline
24. द॒श॒रा॒त्रायेति॑ दश - रा॒त्राय॑ । \newline
25. दी॒क्षि॒ष्यमा॑णो॒ दश॑होतार॒म् दश॑होतारम् दीक्षि॒ष्यमा॑णो दीक्षि॒ष्यमा॑णो॒ दश॑होतारम् जुहुयाज् जुहुया॒द् दश॑होतारम् दीक्षि॒ष्यमा॑णो दीक्षि॒ष्यमा॑णो॒ दश॑होतारम् जुहुयात् । \newline
26. दश॑होतारम् जुहुयाज् जुहुया॒द् दश॑होतार॒म् दश॑होतारम् जुहुया॒द् दश॑होत्रा॒ दश॑होत्रा जुहुया॒द् दश॑होतार॒म् दश॑होतारम् जुहुया॒द् दश॑होत्रा । \newline
27. दश॑होतार॒मिति॒ दश॑ - हो॒ता॒र॒म् । \newline
28. जु॒हु॒या॒द् दश॑होत्रा॒ दश॑होत्रा जुहुयाज् जुहुया॒द् दश॑होत्रै॒वैव दश॑होत्रा जुहुयाज् जुहुया॒द् दश॑होत्रै॒व । \newline
29. दश॑होत्रै॒वैव दश॑होत्रा॒ दश॑होत्रै॒व द॑शरा॒त्रम् द॑शरा॒त्र मे॒व दश॑होत्रा॒ दश॑होत्रै॒व द॑शरा॒त्रम् । \newline
30. दश॑हो॒त्रेति॒ दश॑ - हो॒त्रा॒ । \newline
31. ए॒व द॑शरा॒त्रम् द॑शरा॒त्र मे॒वैव द॑शरा॒त्रꣳ सृ॑जते सृजते दशरा॒त्र मे॒वैव द॑शरा॒त्रꣳ सृ॑जते । \newline
32. द॒श॒रा॒त्रꣳ सृ॑जते सृजते दशरा॒त्रम् द॑शरा॒त्रꣳ सृ॑जते॒ तेन॒ तेन॑ सृजते दशरा॒त्रम् द॑शरा॒त्रꣳ सृ॑जते॒ तेन॑ । \newline
33. द॒श॒रा॒त्रमिति॑ दश - रा॒त्रम् । \newline
34. सृ॒ज॒ते॒ तेन॒ तेन॑ सृजते सृजते॒ तेन॑ दशरा॒त्रेण॑ दशरा॒त्रेण॒ तेन॑ सृजते सृजते॒ तेन॑ दशरा॒त्रेण॑ । \newline
35. तेन॑ दशरा॒त्रेण॑ दशरा॒त्रेण॒ तेन॒ तेन॑ दशरा॒त्रेण॒ प्र प्र द॑शरा॒त्रेण॒ तेन॒ तेन॑ दशरा॒त्रेण॒ प्र । \newline
36. द॒श॒रा॒त्रेण॒ प्र प्र द॑शरा॒त्रेण॑ दशरा॒त्रेण॒ प्र जा॑यते जायते॒ प्र द॑शरा॒त्रेण॑ दशरा॒त्रेण॒ प्र जा॑यते । \newline
37. द॒श॒रा॒त्रेणेति॑ दश - रा॒त्रेण॑ । \newline
38. प्र जा॑यते जायते॒ प्र प्र जा॑यते वैरा॒जो वै॑रा॒जो जा॑यते॒ प्र प्र जा॑यते वैरा॒जः । \newline
39. जा॒य॒ते॒ वै॒रा॒जो वै॑रा॒जो जा॑यते जायते वैरा॒जो वै वै वै॑रा॒जो जा॑यते जायते वैरा॒जो वै । \newline
40. वै॒रा॒जो वै वै वै॑रा॒जो वै॑रा॒जो वा ए॒ष ए॒ष वै वै॑रा॒जो वै॑रा॒जो वा ए॒षः । \newline
41. वा ए॒ष ए॒ष वै वा ए॒ष य॒ज्ञो य॒ज्ञ् ए॒ष वै वा ए॒ष य॒ज्ञ्ः । \newline
42. ए॒ष य॒ज्ञो य॒ज्ञ् ए॒ष ए॒ष य॒ज्ञो यद् यद् य॒ज्ञ् ए॒ष ए॒ष य॒ज्ञो यत् । \newline
43. य॒ज्ञो यद् यद् य॒ज्ञो य॒ज्ञो यद् द॑शरा॒त्रो द॑शरा॒त्रो यद् य॒ज्ञो य॒ज्ञो यद् द॑शरा॒त्रः । \newline
44. यद् द॑शरा॒त्रो द॑शरा॒त्रो यद् यद् द॑शरा॒त्रो यो यो द॑शरा॒त्रो यद् यद् द॑शरा॒त्रो यः । \newline
45. द॒श॒रा॒त्रो यो यो द॑शरा॒त्रो द॑शरा॒त्रो य ए॒व मे॒वं ॅयो द॑शरा॒त्रो द॑शरा॒त्रो य ए॒वम् । \newline
46. द॒श॒रा॒त्र इति॑ दश - रा॒त्रः । \newline
47. य ए॒व मे॒वं ॅयो य ए॒वं ॅवि॒द्वान्. वि॒द्वा ने॒वं ॅयो य ए॒वं ॅवि॒द्वान् । \newline
48. ए॒वं ॅवि॒द्वान्. वि॒द्वा ने॒व मे॒वं ॅवि॒द्वान् द॑शरा॒त्रेण॑ दशरा॒त्रेण॑ वि॒द्वा ने॒व मे॒वं ॅवि॒द्वान् द॑शरा॒त्रेण॑ । \newline
49. वि॒द्वान् द॑शरा॒त्रेण॑ दशरा॒त्रेण॑ वि॒द्वान्. वि॒द्वान् द॑शरा॒त्रेण॒ यज॑ते॒ यज॑ते दशरा॒त्रेण॑ वि॒द्वान्. वि॒द्वान् द॑शरा॒त्रेण॒ यज॑ते । \newline
50. द॒श॒रा॒त्रेण॒ यज॑ते॒ यज॑ते दशरा॒त्रेण॑ दशरा॒त्रेण॒ यज॑ते वि॒राजं॑ ॅवि॒राजं॒ ॅयज॑ते दशरा॒त्रेण॑ दशरा॒त्रेण॒ यज॑ते वि॒राज᳚म् । \newline
51. द॒श॒रा॒त्रेणेति॑ दश - रा॒त्रेण॑ । \newline
52. यज॑ते वि॒राजं॑ ॅवि॒राजं॒ ॅयज॑ते॒ यज॑ते वि॒राज॑ मे॒वैव वि॒राजं॒ ॅयज॑ते॒ यज॑ते वि॒राज॑ मे॒व । \newline
53. वि॒राज॑ मे॒वैव वि॒राजं॑ ॅवि॒राज॑ मे॒व ग॑च्छति गच्छ त्ये॒व वि॒राजं॑ ॅवि॒राज॑ मे॒व ग॑च्छति । \newline
54. वि॒राज॒मिति॑ वि - राज᳚म् । \newline
55. ए॒व ग॑च्छति गच्छ त्ये॒वैव ग॑च्छति प्राजाप॒त्यः प्रा॑जाप॒त्यो ग॑च्छ त्ये॒वैव ग॑च्छति प्राजाप॒त्यः । \newline
56. ग॒च्छ॒ति॒ प्रा॒जा॒प॒त्यः प्रा॑जाप॒त्यो ग॑च्छति गच्छति प्राजाप॒त्यो वै वै प्रा॑जाप॒त्यो ग॑च्छति गच्छति प्राजाप॒त्यो वै । \newline
57. प्रा॒जा॒प॒त्यो वै वै प्रा॑जाप॒त्यः प्रा॑जाप॒त्यो वा ए॒ष ए॒ष वै प्रा॑जाप॒त्यः प्रा॑जाप॒त्यो वा ए॒षः । \newline
58. प्रा॒जा॒प॒त्य इति॑ प्राजा - प॒त्यः । \newline
59. वा ए॒ष ए॒ष वै वा ए॒ष य॒ज्ञो य॒ज्ञ् ए॒ष वै वा ए॒ष य॒ज्ञ्ः । \newline
60. ए॒ष य॒ज्ञो य॒ज्ञ् ए॒ष ए॒ष य॒ज्ञो यद् यद् य॒ज्ञ् ए॒ष ए॒ष य॒ज्ञो यत् । \newline
61. य॒ज्ञो यद् यद् य॒ज्ञो य॒ज्ञो यद् द॑शरा॒त्रो द॑शरा॒त्रो यद् य॒ज्ञो य॒ज्ञो यद् द॑शरा॒त्रः । \newline
62. यद् द॑शरा॒त्रो द॑शरा॒त्रो यद् यद् द॑शरा॒त्रो यो यो द॑शरा॒त्रो यद् यद् द॑शरा॒त्रो यः । \newline
63. द॒श॒रा॒त्रो यो यो द॑शरा॒त्रो द॑शरा॒त्रो य ए॒व मे॒वं ॅयो द॑शरा॒त्रो द॑शरा॒त्रो य ए॒वम् । \newline
64. द॒श॒रा॒त्र इति॑ दश - रा॒त्रः । \newline
\pagebreak
\markright{ TS 7.2.5.2  \hfill https://www.vedavms.in \hfill}

\section{ TS 7.2.5.2 }

\textbf{TS 7.2.5.2 } \newline
\textbf{Samhita Paata} \newline

य ए॒वं ॅवि॒द्वान्-द॑शरा॒त्रेण॒ यज॑ते॒ प्रैव जा॑यत॒ इन्द्रो॒ वै स॒दृङ् दे॒वता॑भिरासी॒थ् स न व्या॒वृत॑मगच्छ॒थ् स प्र॒जाप॑ति॒मुपा॑धाव॒त् तस्मा॑ ए॒तं द॑शरा॒त्रं प्राय॑च्छ॒त् तमाऽह॑र॒त् तेना॑यजत॒ ततो॒ वै सो᳚ऽन्याभि॑-र्दे॒वता॑भि-र्व्या॒वृत॑मगच्छ॒द्य ए॒वं ॅवि॒द्वान् द॑शरा॒त्रेण॒ यज॑ते व्या॒वृत॑मे॒व पा॒प्मना॒ भ्रातृ॑व्येण गच्छति त्रिक॒कुद्वा - [  ] \newline

\textbf{Pada Paata} \newline

यः । ए॒वम् । वि॒द्वान् । द॒श॒रा॒त्रेणेति॑ दश - रा॒त्रेण॑ । यज॑ते । प्रेति॑ । ए॒व । जा॒य॒ते॒ । इन्द्रः॑ । वै । स॒दृङ्ङिति॑ स - दृङ् । दे॒वता॑भिः । आ॒सी॒त् । सः । न । व्या॒वृत॒मिति॑ वि - आ॒वृत᳚म् । अ॒ग॒च्छ॒त् । सः । प्र॒जाप॑ति॒मिति॑ प्र॒जा - प॒ति॒म् । उपेति॑ । अ॒धा॒व॒त् । तस्मै᳚ । ए॒तम् । द॒श॒रा॒त्रमिति॑ दश-रा॒त्रम् । प्रेति॑ । अ॒य॒च्छ॒त् । तम् । एति॑ । अ॒ह॒र॒त् । तेन॑ । अ॒य॒ज॒त॒ । ततः॑ । वै । सः । अ॒न्याभिः॑ । दे॒वता॑भिः । व्या॒वृत॒मिति॑ वि - आ॒वृत᳚म् । अ॒ग॒च्छ॒त् । यः । ए॒वम् । वि॒द्वान् । द॒श॒रा॒त्रेणेति॑ दश - रा॒त्रेण॑ । यज॑ते । व्या॒वृत॒मिति॑ वि - आ॒वृत᳚म् । ए॒व । पा॒प्मना᳚ । भ्रातृ॑व्येण । ग॒च्छ॒ति॒ । त्रि॒क॒कुदिति॑ त्रि - क॒कुत् । वै ।  \newline


\textbf{Krama Paata} \newline

य ए॒वम् । ए॒वम् ॅवि॒द्वान् । वि॒द्वान् द॑शरा॒त्रेण॑ । द॒श॒रा॒त्रेण॒ यज॑ते । द॒श॒रा॒त्रेणेति॑ दश - रा॒त्रेण॑ । यज॑ते॒ प्र । प्रैव । ए॒व जा॑यते । जा॒य॒त॒ इन्द्रः॑ । इन्द्रो॒ वै । वै स॒दृङ्‍ङ् । स॒दृङ् दे॒वता॑भिः । स॒दृङ्ङिति॑  स - दृङ्ङ् । दे॒वता॑भिरासीत् । आ॒सी॒थ् सः । स न । न व्या॒वृत᳚म् । व्या॒वृत॑मगच्छत् । व्या॒वृत॒मिति॑ वि - आ॒वृत᳚म् । अ॒ग॒च्छ॒थ् सः । स प्र॒जाप॑तिम् । प्र॒जाप॑ति॒मुप॑ । प्र॒जाप॑ति॒मिति॑ प्र॒जा - प॒ति॒म् । उपा॑धावत् । अ॒धा॒व॒त् तस्मै᳚ । तस्मा॑ ए॒तम् । ए॒तम् द॑शरा॒त्रम् । द॒श॒रा॒त्रम् प्र । द॒श॒रा॒त्रमिति॑ दश - रा॒त्रम् । प्राय॑च्छत् । अ॒य॒च्छ॒त् तम् । तमा । आऽह॑रत् । अ॒ह॒र॒त् तेन॑ । तेना॑यजत । अ॒य॒ज॒त॒ ततः॑ । ततो॒ वै । वै सः । सो᳚ऽन्याभिः॑ । अ॒न्याभि॑र् दे॒वता॑भिः । दे॒वता॑भिर् व्या॒वृत᳚म् । व्या॒वृत॑मगच्छत् । व्या॒वृत॒मिति॑ वि - आ॒वृत᳚म् । अ॒ग॒च्छ॒द् यः । य ए॒वम् । ए॒वम् ॅवि॒द्वान् । वि॒द्वान् द॑शरा॒त्रेण॑ । द॒श॒रा॒त्रेण॒ यज॑ते । द॒श॒रा॒त्रेणेति॑ दश - रा॒त्रेण॑ । यज॑ते व्या॒वृत᳚म् । व्या॒वृत॑मे॒व । व्या॒वृत॒मिति॑ वि - आ॒वृत᳚म् । ए॒व पा॒प्मना᳚ । पा॒प्मना॒ भ्रातृ॑व्येण । भ्रातृ॑व्येण गच्छति । ग॒च्छ॒ति॒ त्रि॒क॒कुत् । त्रि॒क॒कुद् वै । त्रि॒क॒कुदिति॑ त्रि - क॒कुत् । वा ए॒षः \newline

\textbf{Jatai Paata} \newline

1. य ए॒व मे॒वं ॅयो य ए॒वम् । \newline
2. ए॒वं ॅवि॒द्वान्. वि॒द्वा ने॒व मे॒वं ॅवि॒द्वान् । \newline
3. वि॒द्वान् द॑शरा॒त्रेण॑ दशरा॒त्रेण॑ वि॒द्वान्. वि॒द्वान् द॑शरा॒त्रेण॑ । \newline
4. द॒श॒रा॒त्रेण॒ यज॑ते॒ यज॑ते दशरा॒त्रेण॑ दशरा॒त्रेण॒ यज॑ते । \newline
5. द॒श॒रा॒त्रेणेति॑ दश - रा॒त्रेण॑ । \newline
6. यज॑ते॒ प्र प्र यज॑ते॒ यज॑ते॒ प्र । \newline
7. प्रैवैव प्र प्रैव । \newline
8. ए॒व जा॑यते जायत ए॒वैव जा॑यते । \newline
9. जा॒य॒त॒ इन्द्र॒ इन्द्रो॑ जायते जायत॒ इन्द्रः॑ । \newline
10. इन्द्रो॒ वै वा इन्द्र॒ इन्द्रो॒ वै । \newline
11. वै स॒दृङ् ख्स॒दृङ्. वै वै स॒दृङ् । \newline
12. स॒दृङ् दे॒वता॑भिर् दे॒वता॑भिः स॒दृङ् ख्स॒दृङ् दे॒वता॑भिः । \newline
13. स॒दृङ्ङिति॑ स - दृङ् । \newline
14. दे॒वता॑भि रासी दासीद् दे॒वता॑भिर् दे॒वता॑भि रासीत् । \newline
15. आ॒सी॒थ् स स आ॑सी दासी॒थ् सः । \newline
16. स न न स स न । \newline
17. न व्या॒वृतं॑ ॅव्या॒वृत॒न् न न व्या॒वृत᳚म् । \newline
18. व्या॒वृत॑ मगच्छ दगच्छद् व्या॒वृतं॑ ॅव्या॒वृत॑ मगच्छत् । \newline
19. व्या॒वृत॒मिति॑ वि - आ॒वृत᳚म् । \newline
20. अ॒ग॒च्छ॒थ् स सो॑ ऽगच्छ दगच्छ॒थ् सः । \newline
21. स प्र॒जाप॑तिम् प्र॒जाप॑तिꣳ॒॒ स स प्र॒जाप॑तिम् । \newline
22. प्र॒जाप॑ति॒ मुपोप॑ प्र॒जाप॑तिम् प्र॒जाप॑ति॒ मुप॑ । \newline
23. प्र॒जाप॑ति॒मिति॑ प्र॒जा - प॒ति॒म् । \newline
24. उपा॑ धाव दधाव॒ दुपोपा॑ धावत् । \newline
25. अ॒धा॒व॒त् तस्मै॒ तस्मा॑ अधावद धाव॒त् तस्मै᳚ । \newline
26. तस्मा॑ ए॒त मे॒तम् तस्मै॒ तस्मा॑ ए॒तम् । \newline
27. ए॒तम् द॑शरा॒त्रम् द॑शरा॒त्र मे॒त मे॒तम् द॑शरा॒त्रम् । \newline
28. द॒श॒रा॒त्रम् प्र प्र द॑शरा॒त्रम् द॑शरा॒त्रम् प्र । \newline
29. द॒श॒रा॒त्रमिति॑ दश - रा॒त्रम् । \newline
30. प्राय॑च्छ दयच्छ॒त् प्र प्राय॑च्छत् । \newline
31. अ॒य॒च्छ॒त् तम् तम॑यच्छ दयच्छ॒त् तम् । \newline
32. त मा तम् त मा । \newline
33. आ ऽह॑र दहर॒दा ऽह॑रत् । \newline
34. अ॒ह॒र॒त् तेन॒ तेना॑ हर दहर॒त् तेन॑ । \newline
35. तेना॑ यजता यजत॒ तेन॒ तेना॑ यजत । \newline
36. अ॒य॒ज॒त॒ तत॒ स्ततो॑ ऽयजता यजत॒ ततः॑ । \newline
37. ततो॒ वै वै तत॒ स्ततो॒ वै । \newline
38. वै स स वै वै सः । \newline
39. सो᳚ ऽन्याभि॑ र॒न्याभिः॒ स सो᳚ ऽन्याभिः॑ । \newline
40. अ॒न्याभि॑र् दे॒वता॑भिर् दे॒वता॑भि र॒न्याभि॑ र॒न्याभि॑र् दे॒वता॑भिः । \newline
41. दे॒वता॑भिर् व्या॒वृतं॑ ॅव्या॒वृत॑म् दे॒वता॑भिर् दे॒वता॑भिर् व्या॒वृत᳚म् । \newline
42. व्या॒वृत॑ मगच्छ दगच्छद् व्या॒वृतं॑ ॅव्या॒वृत॑ मगच्छत् । \newline
43. व्या॒वृत॒मिति॑ वि - आ॒वृत᳚म् । \newline
44. अ॒ग॒च्छ॒द् यो यो॑ ऽगच्छ दगच्छ॒द् यः । \newline
45. य ए॒व मे॒वं ॅयो य ए॒वम् । \newline
46. ए॒वं ॅवि॒द्वान्. वि॒द्वा ने॒व मे॒वं ॅवि॒द्वान् । \newline
47. वि॒द्वान् द॑शरा॒त्रेण॑ दशरा॒त्रेण॑ वि॒द्वान्. वि॒द्वान् द॑शरा॒त्रेण॑ । \newline
48. द॒श॒रा॒त्रेण॒ यज॑ते॒ यज॑ते दशरा॒त्रेण॑ दशरा॒त्रेण॒ यज॑ते । \newline
49. द॒श॒रा॒त्रेणेति॑ दश - रा॒त्रेण॑ । \newline
50. यज॑ते व्या॒वृतं॑ ॅव्या॒वृतं॒ ॅयज॑ते॒ यज॑ते व्या॒वृत᳚म् । \newline
51. व्या॒वृत॑ मे॒वैव व्या॒वृतं॑ ॅव्या॒वृत॑ मे॒व । \newline
52. व्या॒वृत॒मिति॑ वि - आ॒वृत᳚म् । \newline
53. ए॒व पा॒प्मना॑ पा॒प्म नै॒वैव पा॒प्मना᳚ । \newline
54. पा॒प्मना॒ भ्रातृ॑व्येण॒ भ्रातृ॑व्येण पा॒प्मना॑ पा॒प्मना॒ भ्रातृ॑व्येण । \newline
55. भ्रातृ॑व्येण गच्छति गच्छति॒ भ्रातृ॑व्येण॒ भ्रातृ॑व्येण गच्छति । \newline
56. ग॒च्छ॒ति॒ त्रि॒क॒कुत् त्रि॑क॒कुद् ग॑च्छति गच्छति त्रिक॒कुत् । \newline
57. त्रि॒क॒कुद् वै वै त्रि॑क॒कुत् त्रि॑क॒कुद् वै । \newline
58. त्रि॒क॒कुदिति॑ त्रि - क॒कुत् । \newline
59. वा ए॒ष ए॒ष वै वा ए॒षः । \newline

\textbf{Ghana Paata } \newline

1. य ए॒व मे॒वं ॅयो य ए॒वं ॅवि॒द्वान्. वि॒द्वा ने॒वं ॅयो य ए॒वं ॅवि॒द्वान् । \newline
2. ए॒वं ॅवि॒द्वान्. वि॒द्वा ने॒व मे॒वं ॅवि॒द्वान् द॑शरा॒त्रेण॑ दशरा॒त्रेण॑ वि॒द्वा ने॒व मे॒वं ॅवि॒द्वान् द॑शरा॒त्रेण॑ । \newline
3. वि॒द्वान् द॑शरा॒त्रेण॑ दशरा॒त्रेण॑ वि॒द्वान्. वि॒द्वान् द॑शरा॒त्रेण॒ यज॑ते॒ यज॑ते दशरा॒त्रेण॑ वि॒द्वान्. वि॒द्वान् द॑शरा॒त्रेण॒ यज॑ते । \newline
4. द॒श॒रा॒त्रेण॒ यज॑ते॒ यज॑ते दशरा॒त्रेण॑ दशरा॒त्रेण॒ यज॑ते॒ प्र प्र यज॑ते दशरा॒त्रेण॑ दशरा॒त्रेण॒ यज॑ते॒ प्र । \newline
5. द॒श॒रा॒त्रेणेति॑ दश - रा॒त्रेण॑ । \newline
6. यज॑ते॒ प्र प्र यज॑ते॒ यज॑ते॒ प्रैवैव प्र यज॑ते॒ यज॑ते॒ प्रैव । \newline
7. प्रैवैव प्र प्रैव जा॑यते जायत ए॒व प्र प्रैव जा॑यते । \newline
8. ए॒व जा॑यते जायत ए॒वैव जा॑यत॒ इन्द्र॒ इन्द्रो॑ जायत ए॒वैव जा॑यत॒ इन्द्रः॑ । \newline
9. जा॒य॒त॒ इन्द्र॒ इन्द्रो॑ जायते जायत॒ इन्द्रो॒ वै वा इन्द्रो॑ जायते जायत॒ इन्द्रो॒ वै । \newline
10. इन्द्रो॒ वै वा इन्द्र॒ इन्द्रो॒ वै स॒दृङ् ख्स॒दृङ्. वा इन्द्र॒ इन्द्रो॒ वै स॒दृङ् । \newline
11. वै स॒दृङ् ख्स॒दृङ्. वै वै स॒दृङ् दे॒वता॑भिर् दे॒वता॑भिः स॒दृङ्. वै वै स॒दृङ् दे॒वता॑भिः । \newline
12. स॒दृङ् दे॒वता॑भिर् दे॒वता॑भिः स॒दृङ् ख्स॒दृङ् दे॒वता॑भि रासी दासीद् दे॒वता॑भिः स॒दृङ् ख्स॒दृङ् दे॒वता॑भि रासीत् । \newline
13. स॒दृङ्ङिति॑ स - दृङ् । \newline
14. दे॒वता॑भि रासी दासीद् दे॒वता॑भिर् दे॒वता॑भि रासी॒थ् स स आ॑सीद् दे॒वता॑भिर् दे॒वता॑भि रासी॒थ् सः । \newline
15. आ॒सी॒थ् स स आ॑सी दासी॒थ् स न न स आ॑सी दासी॒थ् स न । \newline
16. स न न स स न व्या॒वृतं॑ ॅव्या॒वृत॒न् न स स न व्या॒वृत᳚म् । \newline
17. न व्या॒वृतं॑ ॅव्या॒वृत॒न् न न व्या॒वृत॑ मगच्छ दगच्छद् व्या॒वृत॒न् न न व्या॒वृत॑ मगच्छत् । \newline
18. व्या॒वृत॑ मगच्छ दगच्छद् व्या॒वृतं॑ ॅव्या॒वृत॑ मगच्छ॒थ् स सो॑ ऽगच्छद् व्या॒वृतं॑ ॅव्या॒वृत॑ मगच्छ॒थ् सः । \newline
19. व्या॒वृत॒मिति॑ वि - आ॒वृत᳚म् । \newline
20. अ॒ग॒च्छ॒थ् स सो॑ ऽगच्छ दगच्छ॒थ् स प्र॒जाप॑तिम् प्र॒जाप॑तिꣳ॒॒ सो॑ ऽगच्छ दगच्छ॒थ् स प्र॒जाप॑तिम् । \newline
21. स प्र॒जाप॑तिम् प्र॒जाप॑तिꣳ॒॒ स स प्र॒जाप॑ति॒ मुपोप॑ प्र॒जाप॑तिꣳ॒॒ स स प्र॒जाप॑ति॒ मुप॑ । \newline
22. प्र॒जाप॑ति॒ मुपोप॑ प्र॒जाप॑तिम् प्र॒जाप॑ति॒ मुपा॑धाव दधाव॒ दुप प्र॒जाप॑तिम् प्र॒जाप॑ति॒ मुपा॑धावत् । \newline
23. प्र॒जाप॑ति॒मिति॑ प्र॒जा - प॒ति॒म् । \newline
24. उपा॑ धाव दधाव॒ दुपोपा॑ धाव॒त् तस्मै॒ तस्मा॑ अधाव॒ दुपोपा॑ धाव॒त् तस्मै᳚ । \newline
25. अ॒धा॒व॒त् तस्मै॒ तस्मा॑ अधाव दधाव॒त् तस्मा॑ ए॒त मे॒तम् तस्मा॑ अधाव दधाव॒त् तस्मा॑ ए॒तम् । \newline
26. तस्मा॑ ए॒त मे॒तम् तस्मै॒ तस्मा॑ ए॒तम् द॑शरा॒त्रम् द॑शरा॒त्र मे॒तम् तस्मै॒ तस्मा॑ ए॒तम् द॑शरा॒त्रम् । \newline
27. ए॒तम् द॑शरा॒त्रम् द॑शरा॒त्र मे॒त मे॒तम् द॑शरा॒त्रम् प्र प्र द॑शरा॒त्र मे॒त मे॒तम् द॑शरा॒त्रम् प्र । \newline
28. द॒श॒रा॒त्रम् प्र प्र द॑शरा॒त्रम् द॑शरा॒त्रम् प्राय॑च्छ दयच्छ॒त् प्र द॑शरा॒त्रम् द॑शरा॒त्रम् प्राय॑च्छत् । \newline
29. द॒श॒रा॒त्रमिति॑ दश - रा॒त्रम् । \newline
30. प्रा य॑च्छ दयच्छ॒त् प्र प्राय॑च्छ॒त् तम् त म॑यच्छ॒त् प्र प्राय॑च्छ॒त् तम् । \newline
31. अ॒य॒च्छ॒त् तम् त म॑यच्छ दयच्छ॒त् त मा त म॑यच्छ दयच्छ॒त् त मा । \newline
32. त मा तम् त मा ऽह॑र दहर॒दा तम् त मा ऽह॑रत् । \newline
33. आ ऽह॑र दहर॒दा ऽह॑र॒त् तेन॒ तेना॑ हर॒दा ऽह॑र॒त् तेन॑ । \newline
34. अ॒ह॒र॒त् तेन॒ तेना॑ हर दहर॒त् तेना॑ यजता यजत॒ तेना॑ हर दहर॒त् तेना॑ यजत । \newline
35. तेना॑ यजता यजत॒ तेन॒ तेना॑ यजत॒ तत॒ स्ततो॑ ऽयजत॒ तेन॒ तेना॑ यजत॒ ततः॑ । \newline
36. अ॒य॒ज॒त॒ तत॒ स्ततो॑ ऽयजता यजत॒ ततो॒ वै वै ततो॑ ऽयजता यजत॒ ततो॒ वै । \newline
37. ततो॒ वै वै तत॒ स्ततो॒ वै स स वै तत॒ स्ततो॒ वै सः । \newline
38. वै स स वै वै सो᳚ ऽन्याभि॑ र॒न्याभिः॒ स वै वै सो᳚ ऽन्याभिः॑ । \newline
39. सो᳚ ऽन्याभि॑ र॒न्याभिः॒ स सो᳚ ऽन्याभि॑र् दे॒वता॑भिर् दे॒वता॑भि र॒न्याभिः॒ स सो᳚ ऽन्याभि॑र् दे॒वता॑भिः । \newline
40. अ॒न्याभि॑र् दे॒वता॑भिर् दे॒वता॑भि र॒न्याभि॑ र॒न्याभि॑र् दे॒वता॑भिर् व्या॒वृतं॑ ॅव्या॒वृत॑म् दे॒वता॑भि र॒न्याभि॑ र॒न्याभि॑र् दे॒वता॑भिर् व्या॒वृत᳚म् । \newline
41. दे॒वता॑भिर् व्या॒वृतं॑ ॅव्या॒वृत॑म् दे॒वता॑भिर् दे॒वता॑भिर् व्या॒वृत॑ मगच्छ दगच्छद् व्या॒वृत॑म् दे॒वता॑भिर् दे॒वता॑भिर् व्या॒वृत॑ मगच्छत् । \newline
42. व्या॒वृत॑ मगच्छ दगच्छद् व्या॒वृतं॑ ॅव्या॒वृत॑ मगच्छ॒द् यो यो॑ ऽगच्छद् व्या॒वृतं॑ ॅव्या॒वृत॑ मगच्छ॒द् यः । \newline
43. व्या॒वृत॒मिति॑ वि - आ॒वृत᳚म् । \newline
44. अ॒ग॒च्छ॒द् यो यो॑ ऽगच्छ दगच्छ॒द् य ए॒व मे॒वं ॅयो॑ ऽगच्छ दगच्छ॒द् य ए॒वम् । \newline
45. य ए॒व मे॒वं ॅयो य ए॒वं ॅवि॒द्वान्. वि॒द्वा ने॒वं ॅयो य ए॒वं ॅवि॒द्वान् । \newline
46. ए॒वं ॅवि॒द्वान्. वि॒द्वा ने॒व मे॒वं ॅवि॒द्वान् द॑शरा॒त्रेण॑ दशरा॒त्रेण॑ वि॒द्वा ने॒व मे॒वं ॅवि॒द्वान् द॑शरा॒त्रेण॑ । \newline
47. वि॒द्वान् द॑शरा॒त्रेण॑ दशरा॒त्रेण॑ वि॒द्वान्. वि॒द्वान् द॑शरा॒त्रेण॒ यज॑ते॒ यज॑ते दशरा॒त्रेण॑ वि॒द्वान्. वि॒द्वान् द॑शरा॒त्रेण॒ यज॑ते । \newline
48. द॒श॒रा॒त्रेण॒ यज॑ते॒ यज॑ते दशरा॒त्रेण॑ दशरा॒त्रेण॒ यज॑ते व्या॒वृतं॑ ॅव्या॒वृतं॒ ॅयज॑ते दशरा॒त्रेण॑ दशरा॒त्रेण॒ यज॑ते व्या॒वृत᳚म् । \newline
49. द॒श॒रा॒त्रेणेति॑ दश - रा॒त्रेण॑ । \newline
50. यज॑ते व्या॒वृतं॑ ॅव्या॒वृतं॒ ॅयज॑ते॒ यज॑ते व्या॒वृत॑ मे॒वैव व्या॒वृतं॒ ॅयज॑ते॒ यज॑ते व्या॒वृत॑ मे॒व । \newline
51. व्या॒वृत॑ मे॒वैव व्या॒वृतं॑ ॅव्या॒वृत॑ मे॒व पा॒प्मना॑ पा॒प्म नै॒व व्या॒वृतं॑ ॅव्या॒वृत॑ मे॒व पा॒प्मना᳚ । \newline
52. व्या॒वृत॒मिति॑ वि - आ॒वृत᳚म् । \newline
53. ए॒व पा॒प्मना॑ पा॒प्मनै॒वैव पा॒प्मना॒ भ्रातृ॑व्येण॒ भ्रातृ॑व्येण पा॒प्मनै॒वैव पा॒प्मना॒ भ्रातृ॑व्येण । \newline
54. पा॒प्मना॒ भ्रातृ॑व्येण॒ भ्रातृ॑व्येण पा॒प्मना॑ पा॒प्मना॒ भ्रातृ॑व्येण गच्छति गच्छति॒ भ्रातृ॑व्येण पा॒प्मना॑ पा॒प्मना॒ भ्रातृ॑व्येण गच्छति । \newline
55. भ्रातृ॑व्येण गच्छति गच्छति॒ भ्रातृ॑व्येण॒ भ्रातृ॑व्येण गच्छति त्रिक॒कुत् त्रि॑क॒कुद् ग॑च्छति॒ भ्रातृ॑व्येण॒ भ्रातृ॑व्येण गच्छति त्रिक॒कुत् । \newline
56. ग॒च्छ॒ति॒ त्रि॒क॒कुत् त्रि॑क॒कुद् ग॑च्छति गच्छति त्रिक॒कुद् वै वै त्रि॑क॒कुद् ग॑च्छति गच्छति त्रिक॒कुद् वै । \newline
57. त्रि॒क॒कुद् वै वै त्रि॑क॒कुत् त्रि॑क॒कुद् वा ए॒ष ए॒ष वै त्रि॑क॒कुत् त्रि॑क॒कुद् वा ए॒षः । \newline
58. त्रि॒क॒कुदिति॑ त्रि - क॒कुत् । \newline
59. वा ए॒ष ए॒ष वै वा ए॒ष य॒ज्ञो य॒ज्ञ् ए॒ष वै वा ए॒ष य॒ज्ञ्ः । \newline
\pagebreak
\markright{ TS 7.2.5.3  \hfill https://www.vedavms.in \hfill}

\section{ TS 7.2.5.3 }

\textbf{TS 7.2.5.3 } \newline
\textbf{Samhita Paata} \newline

ए॒ष य॒ज्ञो यद्-द॑शरा॒त्रः क॒कुत् प॑ञ्चद॒शः क॒कुदे॑कविꣳ॒॒शः क॒कुत् त्र॑यस्त्रिꣳ॒॒शो य ए॒वं ॅवि॒द्वान् द॑शरा॒त्रेण॒ यज॑ते त्रिक॒कुदे॒व स॑मा॒नानां᳚ भवति॒ यज॑मानः पञ्चद॒शो यज॑मान एकविꣳ॒॒शो यज॑मानस्त्रयस्त्रिꣳ॒॒शः पुर॒ इत॑रा अभिच॒र्यमा॑णो दशरा॒त्रेण॑ यजेत देवपु॒रा ए॒व पर्यू॑हते॒ तस्य॒ न कुत॑श्च॒नोपा᳚व्या॒धो भ॑वति॒ नैन॑मभि॒चरन्᳚थ् स्तृणुते देवासु॒राः संॅय॑त्ता आस॒न् ते दे॒वा ए॒ता - [  ] \newline

\textbf{Pada Paata} \newline

ए॒षः । य॒ज्ञ्ः । यत् । द॒श॒रा॒त्र इति॑ दश - रा॒त्रः । क॒कुत् । प॒ञ्च॒द॒श इति॑ पञ्च - द॒शः । क॒कुत् । ए॒क॒विꣳ॒॒श इत्ये॑क - विꣳ॒॒शः । क॒कुत् । त्र॒य॒स्त्रिꣳ॒॒श इति॑ त्रयः - त्रिꣳ॒॒शः । यः । ए॒वम् । वि॒द्वान् । द॒श॒रा॒त्रेणेति॑ दश - रा॒त्रेण॑ । यज॑ते । त्रि॒क॒कुदिति॑ त्रि - क॒कुत् । ए॒व । स॒मा॒नाना᳚म् । भ॒व॒ति॒ । यज॑मानः । प॒ञ्च॒द॒श इति॑ पञ्च -द॒शः । यज॑मानः । ए॒क॒विꣳ॒॒श इत्ये॑क - विꣳ॒॒शः । यज॑मानः । त्र॒य॒स्त्रिꣳ॒॒श इति॑ त्रयः - त्रिꣳ॒॒शः । पुरः॑ । इत॑राः । अ॒भि॒च॒र्यमा॑ण॒ इत्य॑भि - च॒र्यमा॑णः । द॒श॒रा॒त्रेणेति॑ दश - रा॒त्रेण॑ । य॒जे॒त॒ । दे॒व॒पु॒रा इति॑ देव-पु॒राः । ए॒व । परीति॑ । ऊ॒ह॒ते॒ । तस्य॑ । न । कुतः॑ । च॒न । उ॒पा॒व्या॒ध इत्यु॑प - आ॒व्या॒धः । भ॒व॒ति॒ । न । ए॒न॒म् । अ॒भि॒चर॒न्नित्य॑भि - चरन्न्॑ । स्तृ॒णु॒ते॒ । दे॒वा॒सु॒रा इति॑ देव-अ॒सु॒राः । संॅय॑त्ता॒ इति॒ सं - य॒त्ताः॒ । आ॒स॒न्न् । ते । दे॒वाः । ए॒ताः ।  \newline


\textbf{Krama Paata} \newline

ए॒ष य॒ज्ञ्ः । य॒ज्ञो यत् । यद् द॑शरा॒त्रः । द॒श॒रा॒त्रः क॒कुत् । द॒श॒रा॒त्र इति॑ दश - रा॒त्रः । क॒कुत् प॑ञ्चद॒शः । प॒ञ्च॒द॒शः क॒कुत् । प॒ञ्च॒द॒श इति॑ पञ्च - द॒शः । क॒कुदे॑कविꣳ॒॒शः । ए॒क॒विꣳ॒॒शः क॒कुत् । ए॒क॒विꣳ॒॒श इत्ये॑क - विꣳ॒॒शः । क॒कुत् त्र॑यस्त्रिꣳ॒॒शः । त्र॒य॒स्त्रिꣳ॒॒शो यः । त्र॒य॒स्त्रिꣳ॒॒श इति॑ त्रयः - त्रिꣳ॒॒शः । य ए॒वम् । ए॒वम् ॅवि॒द्वान् । वि॒द्वान् द॑शरा॒त्रेण॑ । द॒श॒रा॒त्रेण॒ यज॑ते । द॒श॒रा॒त्रेणेति॑ दश - रा॒त्रेण॑ । यज॑ते त्रिक॒कुत् । त्रि॒क॒कुदे॒व । त्रि॒क॒कुदिति॑ त्रि - क॒कुत् । ए॒व स॑मा॒नाना᳚म् । स॒मा॒नाना᳚म् भवति । भ॒व॒ति॒ यज॑मानः । यज॑मानः पञ्चद॒शः । प॒ञ्च॒द॒शो यज॑मानः । प॒ञ्च॒द॒श इति॑ पञ्च - द॒शः । यज॑मान एकविꣳ॒॒शः । ए॒क॒विꣳ॒॒शो यज॑मानः । ए॒क॒विꣳ॒॒श इत्ये॑क - विꣳ॒॒शः । यज॑मान स्त्रयस्त्रिꣳ॒॒शः । त्र॒य॒स्त्रिꣳ॒॒शः पुरः॑ । त्र॒य॒स्त्रिꣳ॒॒श इति॑ त्रयः - त्रिꣳ॒॒शः । पुर॒ इत॑राः । इत॑रा अभिच॒र्यमा॑णः । अ॒भि॒च॒र्यमा॑णो दशरा॒त्रेण॑ । अ॒भि॒च॒र्यमा॑ण॒ इत्य॑भि - च॒र्यमा॑णः । द॒श॒रा॒त्रेण॑ यजेत । द॒श॒रा॒त्रेणेति॑ दश - रा॒त्रेण॑ । य॒जे॒त॒ दे॒व॒पु॒राः । दे॒व॒पु॒रा ए॒व । दे॒व॒पु॒रा इति॑ देव - पु॒राः । ए॒व परि॑ । पर्यू॑हते । ऊ॒ह॒ते॒ तस्य॑ । तस्य॒ न । न कुतः॑ । कुत॑श्च॒न । च॒नोपा᳚व्या॒धः । उ॒पा॒व्या॒धो भ॑वति । उ॒पा॒व्या॒ध इत्यु॑प - आ॒व्या॒धः । भ॒व॒ति॒ न । नैन᳚म् । ए॒न॒म॒भि॒चरन्न्॑ । अ॒भि॒चर᳚न्थ् सृणुते । अ॒भि॒चर॒न्नित्य॑भि - चरन्न्॑ । सृ॒णु॒ते॒ दे॒वा॒सु॒राः । दे॒वा॒सु॒राः सम्ॅय॑त्ताः । दे॒वा॒सु॒रा इति॑ देव - अ॒सु॒राः । सम्ॅय॑त्ता आसन्न् । सम्ॅय॑त्ता॒ इति॒ सम् - य॒त्ताः॒ । आ॒स॒न् ते । ते दे॒वाः । दे॒वा ए॒ताः । ए॒ता दे॑वपु॒राः \newline

\textbf{Jatai Paata} \newline

1. ए॒ष य॒ज्ञो य॒ज्ञ् ए॒ष ए॒ष य॒ज्ञ्ः । \newline
2. य॒ज्ञो यद् यद् य॒ज्ञो य॒ज्ञो यत् । \newline
3. यद् द॑शरा॒त्रो द॑शरा॒त्रो यद् यद् द॑शरा॒त्रः । \newline
4. द॒श॒रा॒त्रः क॒कुत् क॒कुद् द॑शरा॒त्रो द॑शरा॒त्रः क॒कुत् । \newline
5. द॒श॒रा॒त्र इति॑ दश - रा॒त्रः । \newline
6. क॒कुत् प॑ञ्चद॒शः प॑ञ्चद॒शः क॒कुत् क॒कुत् प॑ञ्चद॒शः । \newline
7. प॒ञ्च॒द॒शः क॒कुत् क॒कुत् प॑ञ्चद॒शः प॑ञ्चद॒शः क॒कुत् । \newline
8. प॒ञ्च॒द॒श इति॑ पञ्च - द॒शः । \newline
9. क॒कु दे॑कविꣳ॒॒श ए॑कविꣳ॒॒शः क॒कुत् क॒कु दे॑कविꣳ॒॒शः । \newline
10. ए॒क॒विꣳ॒॒शः क॒कुत् क॒कु दे॑कविꣳ॒॒श ए॑कविꣳ॒॒शः क॒कुत् । \newline
11. ए॒क॒विꣳ॒॒श इत्ये॑क - विꣳ॒॒शः । \newline
12. क॒कुत् त्र॑यस्त्रिꣳ॒॒श स्त्र॑यस्त्रिꣳ॒॒शः क॒कुत् क॒कुत् त्र॑यस्त्रिꣳ॒॒शः । \newline
13. त्र॒य॒स्त्रिꣳ॒॒शो यो यस्त्र॑यस्त्रिꣳ॒॒श स्त्र॑यस्त्रिꣳ॒॒शो यः । \newline
14. त्र॒य॒स्त्रिꣳ॒॒श इति॑ त्रयः - त्रिꣳ॒॒शः । \newline
15. य ए॒व मे॒वं ॅयो य ए॒वम् । \newline
16. ए॒वं ॅवि॒द्वान्. वि॒द्वा ने॒व मे॒वं ॅवि॒द्वान् । \newline
17. वि॒द्वान् द॑शरा॒त्रेण॑ दशरा॒त्रेण॑ वि॒द्वान्. वि॒द्वान् द॑शरा॒त्रेण॑ । \newline
18. द॒श॒रा॒त्रेण॒ यज॑ते॒ यज॑ते दशरा॒त्रेण॑ दशरा॒त्रेण॒ यज॑ते । \newline
19. द॒श॒रा॒त्रेणेति॑ दश - रा॒त्रेण॑ । \newline
20. यज॑ते त्रिक॒कुत् त्रि॑क॒कुद् यज॑ते॒ यज॑ते त्रिक॒कुत् । \newline
21. त्रि॒क॒कु दे॒वैव त्रि॑क॒कुत् त्रि॑क॒कु दे॒व । \newline
22. त्रि॒क॒कुदिति॑ त्रि - क॒कुत् । \newline
23. ए॒व स॑मा॒नानाꣳ॑ समा॒नाना॑ मे॒वैव स॑मा॒नाना᳚म् । \newline
24. स॒मा॒नाना᳚म् भवति भवति समा॒नानाꣳ॑ समा॒नाना᳚म् भवति । \newline
25. भ॒व॒ति॒ यज॑मानो॒ यज॑मानो भवति भवति॒ यज॑मानः । \newline
26. यज॑मानः पञ्चद॒शः प॑ञ्चद॒शो यज॑मानो॒ यज॑मानः पञ्चद॒शः । \newline
27. प॒ञ्च॒द॒शो यज॑मानो॒ यज॑मानः पञ्चद॒शः प॑ञ्चद॒शो यज॑मानः । \newline
28. प॒ञ्च॒द॒श इति॑ पञ्च - द॒शः । \newline
29. यज॑मान एकविꣳ॒॒श ए॑कविꣳ॒॒शो यज॑मानो॒ यज॑मान एकविꣳ॒॒शः । \newline
30. ए॒क॒विꣳ॒॒शो यज॑मानो॒ यज॑मान एकविꣳ॒॒श ए॑कविꣳ॒॒शो यज॑मानः । \newline
31. ए॒क॒विꣳ॒॒श इत्ये॑क - विꣳ॒॒शः । \newline
32. यज॑मान स्त्रयस्त्रिꣳ॒॒श स्त्र॑यस्त्रिꣳ॒॒शो यज॑मानो॒ यज॑मान स्त्रयस्त्रिꣳ॒॒शः । \newline
33. त्र॒य॒स्त्रिꣳ॒॒शः पुरः॒ पुर॑ स्त्रयस्त्रिꣳ॒॒श स्त्र॑यस्त्रिꣳ॒॒शः पुरः॑ । \newline
34. त्र॒य॒स्त्रिꣳ॒॒श इति॑ त्रयः - त्रिꣳ॒॒शः । \newline
35. पुर॒ इत॑रा॒ इत॑राः॒ पुरः॒ पुर॒ इत॑राः । \newline
36. इत॑रा अभिच॒र्यमा॑णो ऽभिच॒र्यमा॑ण॒ इत॑रा॒ इत॑रा अभिच॒र्यमा॑णः । \newline
37. अ॒भि॒च॒र्यमा॑णो दशरा॒त्रेण॑ दशरा॒त्रेणा॑ भिच॒र्यमा॑णो ऽभिच॒र्यमा॑णो दशरा॒त्रेण॑ । \newline
38. अ॒भि॒च॒र्यमा॑ण॒ इत्य॑भि - च॒र्यमा॑णः । \newline
39. द॒श॒रा॒त्रेण॑ यजेत यजेत दशरा॒त्रेण॑ दशरा॒त्रेण॑ यजेत । \newline
40. द॒श॒रा॒त्रेणेति॑ दश - रा॒त्रेण॑ । \newline
41. य॒जे॒त॒ दे॒व॒पु॒रा दे॑वपु॒रा य॑जेत यजेत देवपु॒राः । \newline
42. दे॒व॒पु॒रा ए॒वैव दे॑वपु॒रा दे॑वपु॒रा ए॒व । \newline
43. दे॒व॒पु॒रा इति॑ देव - पु॒राः । \newline
44. ए॒व परि॒ पर्ये॒वैव परि॑ । \newline
45. पर्यू॑हत ऊहते॒ परि॒ पर्यू॑हते । \newline
46. ऊ॒ह॒ते॒ तस्य॒ तस्यो॑हत ऊहते॒ तस्य॑ । \newline
47. तस्य॒ न न तस्य॒ तस्य॒ न । \newline
48. न कुतः॒ कुतो॒ न न कुतः॑ । \newline
49. कुत॑ श्च॒न च॒न कुतः॒ कुत॑ श्च॒न । \newline
50. च॒नोपा᳚व्या॒ध उ॑पाव्या॒ध श्च॒न च॒नोपा᳚व्या॒धः । \newline
51. उ॒पा॒व्या॒धो भ॑वति भव त्युपाव्या॒ध उ॑पाव्या॒धो भ॑वति । \newline
52. उ॒पा॒व्या॒ध इत्यु॑प - आ॒व्या॒धः । \newline
53. भ॒व॒ति॒ न न भ॑वति भवति॒ न । \newline
54. नैन॑ मेन॒न् न नैन᳚म् । \newline
55. ए॒न॒ म॒भि॒चर॑न् नभि॒चर॑न् नेन मेन मभि॒चरन्न्॑ । \newline
56. अ॒भि॒चरन्᳚ थ्स्तृणुते स्तृणुते ऽभि॒चर॑न् नभि॒चरन्᳚ थ्स्तृणुते । \newline
57. अ॒भि॒चर॒न्नित्य॑भि - चरन्न्॑ । \newline
58. स्तृ॒णु॒ते॒ दे॒वा॒सु॒रा दे॑वासु॒राः स्तृ॑णुते स्तृणुते देवासु॒राः । \newline
59. दे॒वा॒सु॒राः संॅय॑त्ताः॒ संॅय॑त्ता देवासु॒रा दे॑वासु॒राः संॅय॑त्ताः । \newline
60. दे॒वा॒सु॒रा इति॑ देव - अ॒सु॒राः । \newline
61. संॅय॑त्ता आसन् नास॒न् थ्संॅय॑त्ताः॒ संॅय॑त्ता आसन्न् । \newline
62. संॅय॑त्ता॒ इति॒ सं - य॒त्ताः॒ । \newline
63. आ॒स॒न् ते त आ॑सन् नास॒न् ते । \newline
64. ते दे॒वा दे॒वा स्ते ते दे॒वाः । \newline
65. दे॒वा ए॒ता ए॒ता दे॒वा दे॒वा ए॒ताः । \newline
66. ए॒ता दे॑वपु॒रा दे॑वपु॒रा ए॒ता ए॒ता दे॑वपु॒राः । \newline

\textbf{Ghana Paata } \newline

1. ए॒ष य॒ज्ञो य॒ज्ञ् ए॒ष ए॒ष य॒ज्ञो यद् यद् य॒ज्ञ् ए॒ष ए॒ष य॒ज्ञो यत् । \newline
2. य॒ज्ञो यद् यद् य॒ज्ञो य॒ज्ञो यद् द॑शरा॒त्रो द॑शरा॒त्रो यद् य॒ज्ञो य॒ज्ञो यद् द॑शरा॒त्रः । \newline
3. यद् द॑शरा॒त्रो द॑शरा॒त्रो यद् यद् द॑शरा॒त्रः क॒कुत् क॒कुद् द॑शरा॒त्रो यद् यद् द॑शरा॒त्रः क॒कुत् । \newline
4. द॒श॒रा॒त्रः क॒कुत् क॒कुद् द॑शरा॒त्रो द॑शरा॒त्रः क॒कुत् प॑ञ्चद॒शः प॑ञ्चद॒शः क॒कुद् द॑शरा॒त्रो द॑शरा॒त्रः क॒कुत् प॑ञ्चद॒शः । \newline
5. द॒श॒रा॒त्र इति॑ दश - रा॒त्रः । \newline
6. क॒कुत् प॑ञ्चद॒शः प॑ञ्चद॒शः क॒कुत् क॒कुत् प॑ञ्चद॒शः क॒कुत् क॒कुत् प॑ञ्चद॒शः क॒कुत् क॒कुत् प॑ञ्चद॒शः क॒कुत् । \newline
7. प॒ञ्च॒द॒शः क॒कुत् क॒कुत् प॑ञ्चद॒शः प॑ञ्चद॒शः क॒कु दे॑कविꣳ॒॒श ए॑कविꣳ॒॒शः क॒कुत् प॑ञ्चद॒शः प॑ञ्चद॒शः क॒कु दे॑कविꣳ॒॒शः । \newline
8. प॒ञ्च॒द॒श इति॑ पञ्च - द॒शः । \newline
9. क॒कु दे॑कविꣳ॒॒श ए॑कविꣳ॒॒शः क॒कुत् क॒कु दे॑कविꣳ॒॒शः क॒कुत् क॒कु दे॑कविꣳ॒॒शः क॒कुत् क॒कु दे॑कविꣳ॒॒शः क॒कुत् । \newline
10. ए॒क॒विꣳ॒॒शः क॒कुत् क॒कु दे॑कविꣳ॒॒श ए॑कविꣳ॒॒शः क॒कुत् त्र॑यस्त्रिꣳ॒॒श स्त्र॑यस्त्रिꣳ॒॒शः क॒कु दे॑कविꣳ॒॒श ए॑कविꣳ॒॒शः क॒कुत् त्र॑यस्त्रिꣳ॒॒शः । \newline
11. ए॒क॒विꣳ॒॒श इत्ये॑क - विꣳ॒॒शः । \newline
12. क॒कुत् त्र॑यस्त्रिꣳ॒॒श स्त्र॑यस्त्रिꣳ॒॒शः क॒कुत् क॒कुत् त्र॑यस्त्रिꣳ॒॒शो यो यस्त्र॑यस्त्रिꣳ॒॒शः क॒कुत् क॒कुत् त्र॑यस्त्रिꣳ॒॒शो यः । \newline
13. त्र॒य॒स्त्रिꣳ॒॒शो यो यस्त्र॑यस्त्रिꣳ॒॒श स्त्र॑यस्त्रिꣳ॒॒शो य ए॒व मे॒वं ॅयस्त्र॑यस्त्रिꣳ॒॒श स्त्र॑यस्त्रिꣳ॒॒शो य ए॒वम् । \newline
14. त्र॒य॒स्त्रिꣳ॒॒श इति॑ त्रयः - त्रिꣳ॒॒शः । \newline
15. य ए॒व मे॒वं ॅयो य ए॒वं ॅवि॒द्वान्. वि॒द्वा ने॒वं ॅयो य ए॒वं ॅवि॒द्वान् । \newline
16. ए॒वं ॅवि॒द्वान्. वि॒द्वा ने॒व मे॒वं ॅवि॒द्वान् द॑शरा॒त्रेण॑ दशरा॒त्रेण॑ वि॒द्वा ने॒व मे॒वं ॅवि॒द्वान् द॑शरा॒त्रेण॑ । \newline
17. वि॒द्वान् द॑शरा॒त्रेण॑ दशरा॒त्रेण॑ वि॒द्वान्. वि॒द्वान् द॑शरा॒त्रेण॒ यज॑ते॒ यज॑ते दशरा॒त्रेण॑ वि॒द्वान्. वि॒द्वान् द॑शरा॒त्रेण॒ यज॑ते । \newline
18. द॒श॒रा॒त्रेण॒ यज॑ते॒ यज॑ते दशरा॒त्रेण॑ दशरा॒त्रेण॒ यज॑ते त्रिक॒कुत् त्रि॑क॒कुद् यज॑ते दशरा॒त्रेण॑ दशरा॒त्रेण॒ यज॑ते त्रिक॒कुत् । \newline
19. द॒श॒रा॒त्रेणेति॑ दश - रा॒त्रेण॑ । \newline
20. यज॑ते त्रिक॒कुत् त्रि॑क॒कुद् यज॑ते॒ यज॑ते त्रिक॒कु दे॒वैव त्रि॑क॒कुद् यज॑ते॒ यज॑ते त्रिक॒कु दे॒व । \newline
21. त्रि॒क॒कु दे॒वैव त्रि॑क॒कुत् त्रि॑क॒कु दे॒व स॑मा॒नानाꣳ॑ समा॒नाना॑ मे॒व त्रि॑क॒कुत् त्रि॑क॒कु दे॒व स॑मा॒नाना᳚म् । \newline
22. त्रि॒क॒कुदिति॑ त्रि - क॒कुत् । \newline
23. ए॒व स॑मा॒नानाꣳ॑ समा॒नाना॑ मे॒वैव स॑मा॒नाना᳚म् भवति भवति समा॒नाना॑ मे॒वैव स॑मा॒नाना᳚म् भवति । \newline
24. स॒मा॒नाना᳚म् भवति भवति समा॒नानाꣳ॑ समा॒नाना᳚म् भवति॒ यज॑मानो॒ यज॑मानो भवति समा॒नानाꣳ॑ समा॒नाना᳚म् भवति॒ यज॑मानः । \newline
25. भ॒व॒ति॒ यज॑मानो॒ यज॑मानो भवति भवति॒ यज॑मानः पञ्चद॒शः प॑ञ्चद॒शो यज॑मानो भवति भवति॒ यज॑मानः पञ्चद॒शः । \newline
26. यज॑मानः पञ्चद॒शः प॑ञ्चद॒शो यज॑मानो॒ यज॑मानः पञ्चद॒शो यज॑मानो॒ यज॑मानः पञ्चद॒शो यज॑मानो॒ यज॑मानः पञ्चद॒शो यज॑मानः । \newline
27. प॒ञ्च॒द॒शो यज॑मानो॒ यज॑मानः पञ्चद॒शः प॑ञ्चद॒शो यज॑मान एकविꣳ॒॒श ए॑कविꣳ॒॒शो यज॑मानः पञ्चद॒शः प॑ञ्चद॒शो यज॑मान एकविꣳ॒॒शः । \newline
28. प॒ञ्च॒द॒श इति॑ पञ्च - द॒शः । \newline
29. यज॑मान एकविꣳ॒॒श ए॑कविꣳ॒॒शो यज॑मानो॒ यज॑मान एकविꣳ॒॒शो यज॑मानो॒ यज॑मान एकविꣳ॒॒शो यज॑मानो॒ यज॑मान एकविꣳ॒॒शो यज॑मानः । \newline
30. ए॒क॒विꣳ॒॒शो यज॑मानो॒ यज॑मान एकविꣳ॒॒श ए॑कविꣳ॒॒शो यज॑मान स्त्रयस्त्रिꣳ॒॒श स्त्र॑यस्त्रिꣳ॒॒शो यज॑मान एकविꣳ॒॒श ए॑कविꣳ॒॒शो यज॑मान स्त्रयस्त्रिꣳ॒॒शः । \newline
31. ए॒क॒विꣳ॒॒श इत्ये॑क - विꣳ॒॒शः । \newline
32. यज॑मान स्त्रयस्त्रिꣳ॒॒श स्त्र॑यस्त्रिꣳ॒॒शो यज॑मानो॒ यज॑मान स्त्रयस्त्रिꣳ॒॒शः पुरः॒ पुर॑ स्त्रयस्त्रिꣳ॒॒शो यज॑मानो॒ यज॑मान स्त्रयस्त्रिꣳ॒॒शः पुरः॑ । \newline
33. त्र॒य॒स्त्रिꣳ॒॒शः पुरः॒ पुर॑ स्त्रयस्त्रिꣳ॒॒श स्त्र॑यस्त्रिꣳ॒॒शः पुर॒ इत॑रा॒ इत॑राः॒ पुर॑ स्त्रयस्त्रिꣳ॒॒श स्त्र॑यस्त्रिꣳ॒॒शः पुर॒ इत॑राः । \newline
34. त्र॒य॒स्त्रिꣳ॒॒श इति॑ त्रयः - त्रिꣳ॒॒शः । \newline
35. पुर॒ इत॑रा॒ इत॑राः॒ पुरः॒ पुर॒ इत॑रा अभिच॒र्यमा॑णो ऽभिच॒र्यमा॑ण॒ इत॑राः॒ पुरः॒ पुर॒ इत॑रा अभिच॒र्यमा॑णः । \newline
36. इत॑रा अभिच॒र्यमा॑णो ऽभिच॒र्यमा॑ण॒ इत॑रा॒ इत॑रा अभिच॒र्यमा॑णो दशरा॒त्रेण॑ दशरा॒त्रेणा॑ भिच॒र्यमा॑ण॒ इत॑रा॒ इत॑रा अभिच॒र्यमा॑णो दशरा॒त्रेण॑ । \newline
37. अ॒भि॒च॒र्यमा॑णो दशरा॒त्रेण॑ दशरा॒त्रेणा॑ भिच॒र्यमा॑णो ऽभिच॒र्यमा॑णो दशरा॒त्रेण॑ यजेत यजेत दशरा॒त्रेणा॑ भिच॒र्यमा॑णो ऽभिच॒र्यमा॑णो दशरा॒त्रेण॑ यजेत । \newline
38. अ॒भि॒च॒र्यमा॑ण॒ इत्य॑भि - च॒र्यमा॑णः । \newline
39. द॒श॒रा॒त्रेण॑ यजेत यजेत दशरा॒त्रेण॑ दशरा॒त्रेण॑ यजेत देवपु॒रा दे॑वपु॒रा य॑जेत दशरा॒त्रेण॑ दशरा॒त्रेण॑ यजेत देवपु॒राः । \newline
40. द॒श॒रा॒त्रेणेति॑ दश - रा॒त्रेण॑ । \newline
41. य॒जे॒त॒ दे॒व॒पु॒रा दे॑वपु॒रा य॑जेत यजेत देवपु॒रा ए॒वैव दे॑वपु॒रा य॑जेत यजेत देवपु॒रा ए॒व । \newline
42. दे॒व॒पु॒रा ए॒वैव दे॑वपु॒रा दे॑वपु॒रा ए॒व परि॒ पर्ये॒व दे॑वपु॒रा दे॑वपु॒रा ए॒व परि॑ । \newline
43. दे॒व॒पु॒रा इति॑ देव - पु॒राः । \newline
44. ए॒व परि॒ पर्ये॒वैव पर्यू॑हत ऊहते॒ पर्ये॒वैव पर्यू॑हते । \newline
45. पर्यू॑हत ऊहते॒ परि॒ पर्यू॑हते॒ तस्य॒ तस्यो॑हते॒ परि॒ पर्यू॑हते॒ तस्य॑ । \newline
46. ऊ॒ह॒ते॒ तस्य॒ तस्यो॑हत ऊहते॒ तस्य॒ न न तस्यो॑हत ऊहते॒ तस्य॒ न । \newline
47. तस्य॒ न न तस्य॒ तस्य॒ न कुतः॒ कुतो॒ न तस्य॒ तस्य॒ न कुतः॑ । \newline
48. न कुतः॒ कुतो॒ न न कुत॑ श्च॒न च॒न कुतो॒ न न कुत॑ श्च॒न । \newline
49. कुत॑ श्च॒न च॒न कुतः॒ कुत॑ श्च॒नोपा᳚व्या॒ध उ॑पाव्या॒ध श्च॒न कुतः॒ कुत॑ श्च॒नोपा᳚व्या॒धः । \newline
50. च॒नोपा᳚व्या॒ध उ॑पाव्या॒ध श्च॒न च॒नोपा᳚व्या॒धो भ॑वति भव त्युपाव्या॒ध श्च॒न च॒नोपा᳚व्या॒धो भ॑वति । \newline
51. उ॒पा॒व्या॒धो भ॑वति भव त्युपाव्या॒ध उ॑पाव्या॒धो भ॑वति॒ न न भ॑व त्युपाव्या॒ध उ॑पाव्या॒धो भ॑वति॒ न । \newline
52. उ॒पा॒व्या॒ध इत्यु॑प - आ॒व्या॒धः । \newline
53. भ॒व॒ति॒ न न भ॑वति भवति॒ नैन॑ मेन॒न् न भ॑वति भवति॒ नैन᳚म् । \newline
54. नैन॑ मेन॒न् न नैन॑ मभि॒चर॑न् नभि॒चर॑न् नेन॒न् न नैन॑ मभि॒चरन्न्॑ । \newline
55. ए॒न॒ म॒भि॒चर॑न् नभि॒चर॑न् नेन मेन मभि॒चरन्᳚ थ्स्तृणुते स्तृणुते ऽभि॒चर॑न् नेन मेन मभि॒चरन्᳚ थ्स्तृणुते । \newline
56. अ॒भि॒चरन्᳚ थ्स्तृणुते स्तृणुते ऽभि॒चर॑न् नभि॒चरन्᳚ थ्स्तृणुते देवासु॒रा दे॑वासु॒राः स्तृ॑णुते ऽभि॒चर॑न् नभि॒चरन्᳚ थ्स्तृणुते देवासु॒राः । \newline
57. अ॒भि॒चर॒न्नित्य॑भि - चरन्न्॑ । \newline
58. स्तृ॒णु॒ते॒ दे॒वा॒सु॒रा दे॑वासु॒राः स्तृ॑णुते स्तृणुते देवासु॒राः संॅय॑त्ताः॒ संॅय॑त्ता देवासु॒राः स्तृ॑णुते स्तृणुते देवासु॒राः संॅय॑त्ताः । \newline
59. दे॒वा॒सु॒राः संॅय॑त्ताः॒ संॅय॑त्ता देवासु॒रा दे॑वासु॒राः संॅय॑त्ता आसन् नास॒न् थ्संॅय॑त्ता देवासु॒रा दे॑वासु॒राः संॅय॑त्ता आसन्न् । \newline
60. दे॒वा॒सु॒रा इति॑ देव - अ॒सु॒राः । \newline
61. संॅय॑त्ता आसन् नास॒न् थ्संॅय॑त्ताः॒ संॅय॑त्ता आस॒न् ते त आ॑स॒न् थ्संॅय॑त्ताः॒ संॅय॑त्ता आस॒न् ते । \newline
62. संॅय॑त्ता॒ इति॒ सं - य॒त्ताः॒ । \newline
63. आ॒स॒न् ते त आ॑सन् नास॒न् ते दे॒वा दे॒वा स्त आ॑सन् नास॒न् ते दे॒वाः । \newline
64. ते दे॒वा दे॒वा स्ते ते दे॒वा ए॒ता ए॒ता दे॒वा स्ते ते दे॒वा ए॒ताः । \newline
65. दे॒वा ए॒ता ए॒ता दे॒वा दे॒वा ए॒ता दे॑वपु॒रा दे॑वपु॒रा ए॒ता दे॒वा दे॒वा ए॒ता दे॑वपु॒राः । \newline
66. ए॒ता दे॑वपु॒रा दे॑वपु॒रा ए॒ता ए॒ता दे॑वपु॒रा अ॑पश्यन् नपश्यन् देवपु॒रा ए॒ता ए॒ता दे॑वपु॒रा अ॑पश्यन्न् । \newline
\pagebreak
\markright{ TS 7.2.5.4  \hfill https://www.vedavms.in \hfill}

\section{ TS 7.2.5.4 }

\textbf{TS 7.2.5.4 } \newline
\textbf{Samhita Paata} \newline

दे॑वपु॒रा अ॑पश्य॒न्॒ यद्-द॑शरा॒त्रस्ताः पर्यौ॑हन्त॒ तेषां॒ न कुत॑श्च॒नोपा᳚व्या॒धो॑ ऽभव॒त् ततो॑ दे॒वा अभ॑व॒न् पराऽसु॑रा॒ यो भ्रातृ॑व्यवा॒न्थ् स्याथ् स द॑शरा॒त्रेण॑ यजेत देवपु॒रा ए॒व पर्यू॑हते॒ तस्य॒ न कुत॑श्च॒नोपा᳚व्या॒धो भ॑वति॒ भव॑त्या॒त्मना॒ परा᳚ऽस्य॒ भ्रातृ॑व्यो भवति॒ स्तोमः॒ स्तोम॒स्योप॑स्तिर्भवति॒ भ्रातृ॑व्यमे॒वोप॑स्तिं कुरुते जा॒मि वा - [  ] \newline

\textbf{Pada Paata} \newline

दे॒व॒पु॒रा इति॑ देव-पु॒राः । अ॒प॒श्य॒न्न् । यत् । द॒श॒रा॒त्र इति॑ दश-रा॒त्रः । ताः । परीति॑ । औ॒ह॒न्त॒ । तेषा᳚म् । न । कुतः॑ । च॒न । उ॒पा॒व्या॒ध इत्यु॑प - आ॒व्या॒धः । अ॒भ॒व॒त् । ततः॑ । दे॒वाः । अभ॑वन्न् । परेति॑ । असु॑राः । यः । भ्रातृ॑व्यवा॒निति॒ भ्रातृ॑व्य - वा॒न् । स्यात् । सः । द॒श॒रा॒त्रेणेति॑ दश - रा॒त्रेण॑ । य॒जे॒त॒ । दे॒व॒पु॒रा इति॑ देव - पु॒राः । ए॒व । परीति॑ । ऊ॒ह॒ते॒ । तस्य॑ । न । कुतः॑ । च॒न । उ॒पा॒व्या॒ध इत्यु॑प- आ॒व्या॒धः । भ॒व॒ति॒ । भव॑ति । आ॒त्मना᳚ । परेति॑ । अ॒स्य॒ । भ्रातृ॑व्यः । भ॒व॒ति॒ । स्तोमः॑ । स्तोम॑स्य । उप॑स्तिः । भ॒व॒ति॒ । भ्रातृ॑व्यम् । ए॒व । उप॑स्तिम् । कु॒रु॒ते॒ । जा॒मि । वै ।  \newline


\textbf{Krama Paata} \newline

दे॒व॒पु॒रा अ॑पश्यन्न् । दे॒व॒पु॒रा इति॑ देव - पु॒राः । अ॒प॒श्य॒न्॒. यत् । यद् द॑शरा॒त्रः । द॒श॒रा॒त्रस्ताः । द॒श॒रा॒त्र इति॑ दश - रा॒त्रः । ताः परि॑ । पर्यौ॑हन्त । औ॒ह॒न्त॒ तेषा᳚म् । तेषा॒म् न । न कुतः॑ । कुत॑श्च॒न । च॒नोपा᳚व्या॒धः । उ॒पा॒व्या॒धो॑ऽभवत् । उ॒पा॒व्या॒ध इत्यु॑प - आ॒व्या॒धः । अ॒भ॒व॒त् ततः॑ । ततो॑ दे॒वाः । दे॒वा अभ॑वन्न् । अभ॑व॒न् परा᳚ । पराऽसु॑राः । असु॑रा॒ यः । यो भ्रातृ॑व्यवान्न् । भ्रातृ॑व्यवा॒न्थ् स्यात् । भ्रातृ॑व्यवा॒निति॒ भ्रातृ॑व्य - वा॒न्॒ । स्याथ् सः । स द॑शरा॒त्रेण॑ । द॒श॒रा॒त्रेण॑ यजेत । द॒श॒रा॒त्रेणेति॑ दश - रा॒त्रेण॑ । य॒जे॒त॒ दे॒व॒पु॒राः । दे॒व॒पु॒रा ए॒व । दे॒व॒पु॒रा इति॑ देव - पु॒राः । ए॒व परि॑ । पर्यू॑हते । ऊ॒ह॒ते॒ तस्य॑ । तस्य॒ न । न कुतः॑ । कुत॑श्च॒न । च॒नोपा᳚व्या॒धः । उ॒पा॒व्या॒धो भ॑वति । उ॒पा॒व्या॒ध इत्यु॑प - आ॒व्या॒धः । भ॒व॒ति॒ भव॑ति । भव॑त्या॒त्मना᳚ । आ॒त्मना॒ परा᳚ । परा᳚ऽस्य । अ॒स्य॒ भ्रातृ॑व्यः । भ्रातृ॑व्यो भवति । भ॒व॒ति॒ स्तोमः॑ । स्तोमः॒ स्तोम॑स्य । स्तोम॒स्योप॑स्तिः । उप॑स्तिर् भवति । भ॒व॒ति॒ भ्रातृ॑व्यम् । भ्रातृ॑व्यमे॒व । ए॒वोप॑स्तिम् । उप॑स्तिम् कुरुते । कु॒रु॒ते॒ जा॒मि । जा॒मि वै । वा ए॒तत् \newline

\textbf{Jatai Paata} \newline

1. दे॒व॒पु॒रा अ॑पश्यन् नपश्यन् देवपु॒रा दे॑वपु॒रा अ॑पश्यन्न् । \newline
2. दे॒व॒पु॒रा इति॑ देव - पु॒राः । \newline
3. अ॒प॒श्य॒न्॒. यद् यद॑पश्यन् नपश्य॒न्॒. यत् । \newline
4. यद् द॑शरा॒त्रो द॑शरा॒त्रो यद् यद् द॑शरा॒त्रः । \newline
5. द॒श॒रा॒त्र स्ता स्ता द॑शरा॒त्रो द॑शरा॒त्र स्ताः । \newline
6. द॒श॒रा॒त्र इति॑ दश - रा॒त्रः । \newline
7. ताः परि॒ परि॒ ता स्ताः परि॑ । \newline
8. पर्यौ॑हन्तौ हन्त॒ परि॒ पर्यौ॑हन्त । \newline
9. औ॒ह॒न्त॒ तेषा॒म् तेषा॑ मौहन्तौहन्त॒ तेषा᳚म् । \newline
10. तेषा॒न् न न तेषा॒म् तेषा॒न् न । \newline
11. न कुतः॒ कुतो॒ न न कुतः॑ । \newline
12. कुत॑ श्च॒न च॒न कुतः॒ कुत॑ श्च॒न । \newline
13. च॒नोपा᳚व्या॒ध उ॑पाव्या॒ध श्च॒न च॒नोपा᳚व्या॒धः । \newline
14. उ॒पा॒व्या॒धो॑ ऽभव दभव दुपाव्या॒ध उ॑पाव्या॒धो॑ ऽभवत् । \newline
15. उ॒पा॒व्या॒ध इत्यु॑प - आ॒व्या॒धः । \newline
16. अ॒भ॒व॒त् तत॒ स्ततो॑ ऽभव दभव॒त् ततः॑ । \newline
17. ततो॑ दे॒वा दे॒वा स्तत॒ स्ततो॑ दे॒वाः । \newline
18. दे॒वा अभ॑व॒न् नभ॑वन् दे॒वा दे॒वा अभ॑वन्न् । \newline
19. अभ॑व॒न् परा॒ परा ऽभ॑व॒न् नभ॑व॒न् परा᳚ । \newline
20. परा ऽसु॑रा॒ असु॑राः॒ परा॒ परा ऽसु॑राः । \newline
21. असु॑रा॒ यो यो ऽसु॑रा॒ असु॑रा॒ यः । \newline
22. यो भ्रातृ॑व्यवा॒न् भ्रातृ॑व्यवा॒न्॒. यो यो भ्रातृ॑व्यवान् । \newline
23. भ्रातृ॑व्यवा॒न् थ्स्याथ् स्याद् भ्रातृ॑व्यवा॒न् भ्रातृ॑व्यवा॒न् थ्स्यात् । \newline
24. भ्रातृ॑व्यवा॒निति॒ भ्रातृ॑व्य - वा॒न् । \newline
25. स्याथ् स स स्याथ् स्याथ् सः । \newline
26. स द॑शरा॒त्रेण॑ दशरा॒त्रेण॒ स स द॑शरा॒त्रेण॑ । \newline
27. द॒श॒रा॒त्रेण॑ यजेत यजेत दशरा॒त्रेण॑ दशरा॒त्रेण॑ यजेत । \newline
28. द॒श॒रा॒त्रेणेति॑ दश - रा॒त्रेण॑ । \newline
29. य॒जे॒त॒ दे॒व॒पु॒रा दे॑वपु॒रा य॑जेत यजेत देवपु॒राः । \newline
30. दे॒व॒पु॒रा ए॒वैव दे॑वपु॒रा दे॑वपु॒रा ए॒व । \newline
31. दे॒व॒पु॒रा इति॑ देव - पु॒राः । \newline
32. ए॒व परि॒ पर्ये॒वैव परि॑ । \newline
33. पर्यू॑हत ऊहते॒ परि॒ पर्यू॑हते । \newline
34. ऊ॒ह॒ते॒ तस्य॒ तस्यो॑हत ऊहते॒ तस्य॑ । \newline
35. तस्य॒ न न तस्य॒ तस्य॒ न । \newline
36. न कुतः॒ कुतो॒ न न कुतः॑ । \newline
37. कुत॑ श्च॒न च॒न कुतः॒ कुत॑ श्च॒न । \newline
38. च॒नोपा᳚व्या॒ध उ॑पाव्या॒ध श्च॒न च॒नोपा᳚व्या॒धः । \newline
39. उ॒पा॒व्या॒धो भ॑वति भव त्युपाव्या॒ध उ॑पाव्या॒धो भ॑वति । \newline
40. उ॒पा॒व्या॒ध इत्यु॑प - आ॒व्या॒धः । \newline
41. भ॒व॒ति॒ भव॑ति॒ भव॑ति भवति भवति॒ भव॑ति । \newline
42. भव॑ त्या॒त्मना॒ ऽऽत्मना॒ भव॑ति॒ भव॑ त्या॒त्मना᳚ । \newline
43. आ॒त्मना॒ परा॒ परा॒ ऽऽत्मना॒ ऽऽत्मना॒ परा᳚ । \newline
44. परा᳚ ऽस्यास्य॒ परा॒ परा᳚ ऽस्य । \newline
45. अ॒स्य॒ भ्रातृ॑व्यो॒ भ्रातृ॑व्यो ऽस्यास्य॒ भ्रातृ॑व्यः । \newline
46. भ्रातृ॑व्यो भवति भवति॒ भ्रातृ॑व्यो॒ भ्रातृ॑व्यो भवति । \newline
47. भ॒व॒ति॒ स्तोमः॒ स्तोमो॑ भवति भवति॒ स्तोमः॑ । \newline
48. स्तोमः॒ स्तोम॑स्य॒ स्तोम॑स्य॒ स्तोमः॒ स्तोमः॒ स्तोम॑स्य । \newline
49. स्तोम॒ स्योप॑स्ति॒ रुप॑स्तिः॒ स्तोम॑स्य॒ स्तोम॒ स्योप॑स्तिः । \newline
50. उप॑स्तिर् भवति भव॒ त्युप॑स्ति॒ रुप॑स्तिर् भवति । \newline
51. भ॒व॒ति॒ भ्रातृ॑व्य॒म् भ्रातृ॑व्यम् भवति भवति॒ भ्रातृ॑व्यम् । \newline
52. भ्रातृ॑व्य मे॒वैव भ्रातृ॑व्य॒म् भ्रातृ॑व्य मे॒व । \newline
53. ए॒वोप॑स्ति॒ मुप॑स्ति मे॒वै वोप॑स्तिम् । \newline
54. उप॑स्तिम् कुरुते कुरुत॒ उप॑स्ति॒ मुप॑स्तिम् कुरुते । \newline
55. कु॒रु॒ते॒ जा॒मि जा॒मि कु॑रुते कुरुते जा॒मि । \newline
56. जा॒मि वै वै जा॒मि जा॒मि वै । \newline
57. वा ए॒त दे॒तद् वै वा ए॒तत् । \newline

\textbf{Ghana Paata } \newline

1. दे॒व॒पु॒रा अ॑पश्यन् नपश्यन् देवपु॒रा दे॑वपु॒रा अ॑पश्य॒न्॒. यद् यद॑पश्यन् देवपु॒रा दे॑वपु॒रा अ॑पश्य॒न्॒. यत् । \newline
2. दे॒व॒पु॒रा इति॑ देव - पु॒राः । \newline
3. अ॒प॒श्य॒न्॒. यद् यद॑पश्यन् नपश्य॒न्॒. यद् द॑शरा॒त्रो द॑शरा॒त्रो यद॑पश्यन् नपश्य॒न्॒. यद् द॑शरा॒त्रः । \newline
4. यद् द॑शरा॒त्रो द॑शरा॒त्रो यद् यद् द॑शरा॒त्र स्ता स्ता द॑शरा॒त्रो यद् यद् द॑शरा॒त्र स्ताः । \newline
5. द॒श॒रा॒त्र स्ता स्ता द॑शरा॒त्रो द॑शरा॒त्र स्ताः परि॒ परि॒ ता द॑शरा॒त्रो द॑शरा॒त्र स्ताः परि॑ । \newline
6. द॒श॒रा॒त्र इति॑ दश - रा॒त्रः । \newline
7. ताः परि॒ परि॒ ता स्ताः पर्यौ॑हन्तौ हन्त॒ परि॒ ता स्ताः पर्यौ॑हन्त । \newline
8. पर्यौ॑हन्तौ हन्त॒ परि॒ पर्यौ॑हन्त॒ तेषा॒म् तेषा॑ मौहन्त॒ परि॒ पर्यौ॑हन्त॒ तेषा᳚म् । \newline
9. औ॒ह॒न्त॒ तेषा॒म् तेषा॑ मौहन्तौ हन्त॒ तेषा॒न् न न तेषा॑ मौहन्तौ हन्त॒ तेषा॒न् न । \newline
10. तेषा॒न् न न तेषा॒म् तेषा॒न् न कुतः॒ कुतो॒ न तेषा॒म् तेषा॒न् न कुतः॑ । \newline
11. न कुतः॒ कुतो॒ न न कुत॑ श्च॒न च॒न कुतो॒ न न कुत॑ श्च॒न । \newline
12. कुत॑ श्च॒न च॒न कुतः॒ कुत॑ श्च॒नोपा᳚व्या॒ध उ॑पाव्या॒ध श्च॒न कुतः॒ कुत॑ श्च॒नोपा᳚व्या॒धः । \newline
13. च॒नोपा᳚व्या॒ध उ॑पाव्या॒ध श्च॒न च॒नोपा᳚व्या॒धो॑ ऽभव दभव दुपाव्या॒ध श्च॒न च॒नोपा᳚व्या॒धो॑ ऽभवत् । \newline
14. उ॒पा॒व्या॒धो॑ ऽभव दभव दुपाव्या॒ध उ॑पाव्या॒धो॑ ऽभव॒त् तत॒ स्ततो॑ ऽभव दुपाव्या॒ध उ॑पाव्या॒धो॑ ऽभव॒त् ततः॑ । \newline
15. उ॒पा॒व्या॒ध इत्यु॑प - आ॒व्या॒धः । \newline
16. अ॒भ॒व॒त् तत॒ स्ततो॑ ऽभव दभव॒त् ततो॑ दे॒वा दे॒वा स्ततो॑ ऽभव दभव॒त् ततो॑ दे॒वाः । \newline
17. ततो॑ दे॒वा दे॒वा स्तत॒ स्ततो॑ दे॒वा अभ॑व॒न् नभ॑वन् दे॒वा स्तत॒ स्ततो॑ दे॒वा अभ॑वन्न् । \newline
18. दे॒वा अभ॑व॒न् नभ॑वन् दे॒वा दे॒वा अभ॑व॒न् परा॒ परा ऽभ॑वन् दे॒वा दे॒वा अभ॑व॒न् परा᳚ । \newline
19. अभ॑व॒न् परा॒ परा ऽभ॑व॒न् नभ॑व॒न् परा ऽसु॑रा॒ असु॑राः॒ परा ऽभ॑व॒न् नभ॑व॒न् परा ऽसु॑राः । \newline
20. परा ऽसु॑रा॒ असु॑राः॒ परा॒ परा ऽसु॑रा॒ यो यो ऽसु॑राः॒ परा॒ परा ऽसु॑रा॒ यः । \newline
21. असु॑रा॒ यो यो ऽसु॑रा॒ असु॑रा॒ यो भ्रातृ॑व्यवा॒न् भ्रातृ॑व्यवा॒न्॒. यो ऽसु॑रा॒ असु॑रा॒ यो भ्रातृ॑व्यवान् । \newline
22. यो भ्रातृ॑व्यवा॒न् भ्रातृ॑व्यवा॒न्॒. यो यो भ्रातृ॑व्यवा॒न् थ्स्याथ् स्याद् भ्रातृ॑व्यवा॒न्॒. यो यो भ्रातृ॑व्यवा॒न् थ्स्यात् । \newline
23. भ्रातृ॑व्यवा॒न् थ्स्याथ् स्याद् भ्रातृ॑व्यवा॒न् भ्रातृ॑व्यवा॒न् थ्स्याथ् स स स्याद् भ्रातृ॑व्यवा॒न् भ्रातृ॑व्यवा॒न् थ्स्याथ् सः । \newline
24. भ्रातृ॑व्यवा॒निति॒ भ्रातृ॑व्य - वा॒न् । \newline
25. स्याथ् स स स्याथ् स्याथ् स द॑शरा॒त्रेण॑ दशरा॒त्रेण॒ स स्याथ् स्याथ् स द॑शरा॒त्रेण॑ । \newline
26. स द॑शरा॒त्रेण॑ दशरा॒त्रेण॒ स स द॑शरा॒त्रेण॑ यजेत यजेत दशरा॒त्रेण॒ स स द॑शरा॒त्रेण॑ यजेत । \newline
27. द॒श॒रा॒त्रेण॑ यजेत यजेत दशरा॒त्रेण॑ दशरा॒त्रेण॑ यजेत देवपु॒रा दे॑वपु॒रा य॑जेत दशरा॒त्रेण॑ दशरा॒त्रेण॑ यजेत देवपु॒राः । \newline
28. द॒श॒रा॒त्रेणेति॑ दश - रा॒त्रेण॑ । \newline
29. य॒जे॒त॒ दे॒व॒पु॒रा दे॑वपु॒रा य॑जेत यजेत देवपु॒रा ए॒वैव दे॑वपु॒रा य॑जेत यजेत देवपु॒रा ए॒व । \newline
30. दे॒व॒पु॒रा ए॒वैव दे॑वपु॒रा दे॑वपु॒रा ए॒व परि॒ पर्ये॒व दे॑वपु॒रा दे॑वपु॒रा ए॒व परि॑ । \newline
31. दे॒व॒पु॒रा इति॑ देव - पु॒राः । \newline
32. ए॒व परि॒ पर्ये॒वैव पर्यू॑हत ऊहते॒ पर्ये॒वैव पर्यू॑हते । \newline
33. पर्यू॑हत ऊहते॒ परि॒ पर्यू॑हते॒ तस्य॒ तस्यो॑हते॒ परि॒ पर्यू॑हते॒ तस्य॑ । \newline
34. ऊ॒ह॒ते॒ तस्य॒ तस्यो॑हत ऊहते॒ तस्य॒ न न तस्यो॑हत ऊहते॒ तस्य॒ न । \newline
35. तस्य॒ न न तस्य॒ तस्य॒ न कुतः॒ कुतो॒ न तस्य॒ तस्य॒ न कुतः॑ । \newline
36. न कुतः॒ कुतो॒ न न कुत॑ श्च॒न च॒न कुतो॒ न न कुत॑ श्च॒न । \newline
37. कुत॑ श्च॒न च॒न कुतः॒ कुत॑ श्च॒नोपा᳚व्या॒ध उ॑पाव्या॒ध श्च॒न कुतः॒ कुत॑ श्च॒नोपा᳚व्या॒धः । \newline
38. च॒नोपा᳚व्या॒ध उ॑पाव्या॒ध श्च॒न च॒नोपा᳚व्या॒धो भ॑वति भव त्युपाव्या॒ध श्च॒न च॒नोपा᳚व्या॒धो भ॑वति । \newline
39. उ॒पा॒व्या॒धो भ॑वति भव त्युपाव्या॒ध उ॑पाव्या॒धो भ॑वति॒ भव॑ति॒ भव॑ति भव त्युपाव्या॒ध उ॑पाव्या॒धो भ॑वति॒ भव॑ति । \newline
40. उ॒पा॒व्या॒ध इत्यु॑प - आ॒व्या॒धः । \newline
41. भ॒व॒ति॒ भव॑ति॒ भव॑ति भवति भवति॒ भव॑ त्या॒त्मना॒ ऽऽत्मना॒ भव॑ति भवति भवति॒ भव॑ त्या॒त्मना᳚ । \newline
42. भव॑ त्या॒त्मना॒ ऽऽत्मना॒ भव॑ति॒ भव॑ त्या॒त्मना॒ परा॒ परा॒ ऽऽत्मना॒ भव॑ति॒ भव॑ त्या॒त्मना॒ परा᳚ । \newline
43. आ॒त्मना॒ परा॒ परा॒ ऽऽत्मना॒ ऽऽत्मना॒ परा᳚ ऽस्यास्य॒ परा॒ ऽऽत्मना॒ ऽऽत्मना॒ परा᳚ ऽस्य । \newline
44. परा᳚ ऽस्यास्य॒ परा॒ परा᳚ ऽस्य॒ भ्रातृ॑व्यो॒ भ्रातृ॑व्यो ऽस्य॒ परा॒ परा᳚ ऽस्य॒ भ्रातृ॑व्यः । \newline
45. अ॒स्य॒ भ्रातृ॑व्यो॒ भ्रातृ॑व्यो ऽस्यास्य॒ भ्रातृ॑व्यो भवति भवति॒ भ्रातृ॑व्यो ऽस्यास्य॒ भ्रातृ॑व्यो भवति । \newline
46. भ्रातृ॑व्यो भवति भवति॒ भ्रातृ॑व्यो॒ भ्रातृ॑व्यो भवति॒ स्तोमः॒ स्तोमो॑ भवति॒ भ्रातृ॑व्यो॒ भ्रातृ॑व्यो भवति॒ स्तोमः॑ । \newline
47. भ॒व॒ति॒ स्तोमः॒ स्तोमो॑ भवति भवति॒ स्तोमः॒ स्तोम॑स्य॒ स्तोम॑स्य॒ स्तोमो॑ भवति भवति॒ स्तोमः॒ स्तोम॑स्य । \newline
48. स्तोमः॒ स्तोम॑स्य॒ स्तोम॑स्य॒ स्तोमः॒ स्तोमः॒ स्तोम॒स्योप॑स्ति॒ रुप॑स्तिः॒ स्तोम॑स्य॒ स्तोमः॒ स्तोमः॒ 
स्तोम॒स्योप॑स्तिः । \newline
49. स्तोम॒ स्योप॑स्ति॒ रुप॑स्तिः॒ स्तोम॑स्य॒ स्तोम॒ स्योप॑स्तिर् भवति भव॒ त्युप॑स्तिः॒ स्तोम॑स्य॒ स्तोम॒ स्योप॑स्तिर् भवति । \newline
50. उप॑स्तिर् भवति भव॒ त्युप॑स्ति॒ रुप॑स्तिर् भवति॒ भ्रातृ॑व्य॒म् भ्रातृ॑व्यम् भव॒ त्युप॑स्ति॒ रुप॑स्तिर् भवति॒ भ्रातृ॑व्यम् । \newline
51. भ॒व॒ति॒ भ्रातृ॑व्य॒म् भ्रातृ॑व्यम् भवति भवति॒ भ्रातृ॑व्य मे॒वैव भ्रातृ॑व्यम् भवति भवति॒ भ्रातृ॑व्य मे॒व । \newline
52. भ्रातृ॑व्य मे॒वैव भ्रातृ॑व्य॒म् भ्रातृ॑व्य मे॒वोप॑स्ति॒ मुप॑स्ति मे॒व भ्रातृ॑व्य॒म् भ्रातृ॑व्य मे॒वोप॑स्तिम् । \newline
53. ए॒वोप॑स्ति॒ मुप॑स्ति मे॒वैवोप॑स्तिम् कुरुते कुरुत॒ उप॑स्ति मे॒वैवोप॑स्तिम् कुरुते । \newline
54. उप॑स्तिम् कुरुते कुरुत॒ उप॑स्ति॒ मुप॑स्तिम् कुरुते जा॒मि जा॒मि कु॑रुत॒ उप॑स्ति॒ मुप॑स्तिम् कुरुते जा॒मि । \newline
55. कु॒रु॒ते॒ जा॒मि जा॒मि कु॑रुते कुरुते जा॒मि वै वै जा॒मि कु॑रुते कुरुते जा॒मि वै । \newline
56. जा॒मि वै वै जा॒मि जा॒मि वा ए॒त दे॒तद् वै जा॒मि जा॒मि वा ए॒तत् । \newline
57. वा ए॒त दे॒तद् वै वा ए॒तत् कु॑र्वन्ति कुर्व न्त्ये॒तद् वै वा ए॒तत् कु॑र्वन्ति । \newline
\pagebreak
\markright{ TS 7.2.5.5  \hfill https://www.vedavms.in \hfill}

\section{ TS 7.2.5.5 }

\textbf{TS 7.2.5.5 } \newline
\textbf{Samhita Paata} \newline

ए॒तत् कु॑र्वन्ति॒ यज्ज्यायाꣳ॑सꣳ॒॒ स्तोम॑मु॒पेत्य॒ कनी॑याꣳसमुप॒यन्ति॒ यद॑ग्निष्टो-मसा॒मान्य॒वस्ता᳚च्च प॒रस्ता᳚च्च॒ भव॒न्त्यजा॑मित्वाय त्रि॒वृद॑ग्निष्टो॒मो᳚ ऽग्नि॒ष्टुदा᳚ग्ने॒यीषु॑ भवति॒ तेज॑ ए॒वाव॑ रुन्धे पञ्चद॒श उ॒क्थ्य॑ ऐ॒न्द्रीष्वि॑न्द्रि॒यमे॒वाव॑ रुन्धे त्रि॒वृद॑ग्निष्टो॒मो वै᳚श्वदे॒वीषु॒ पुष्टि॑मे॒वाव॑ रुन्धे सप्तद॒शो᳚ऽग्निष्टो॒मः प्रा॑जाप॒त्यासु॑ तीव्रसो॒मो᳚ ऽन्नाद्य॒स्या-व॑रुद्ध्या॒ अथो॒ प्रैव तेन॑ जायत - [  ] \newline

\textbf{Pada Paata} \newline

ए॒तत् । कु॒र्व॒न्ति॒ । यत् । ज्यायाꣳ॑सम् । स्तोम᳚म् । उ॒पेत्येत्यु॑प-इत्य॑ । कनी॑याꣳसम् । उ॒प॒यन्तीत्यु॑प - यन्ति॑ । यत् । अ॒ग्नि॒ष्टो॒म॒सा॒मानीत्य॑ग्निष्टोम-सा॒मानि॑ । अ॒वस्ता᳚त् । च॒ । प॒रस्ता᳚त् । च॒ । भव॑न्ति । अजा॑मित्वा॒येत्यजा॑मि - त्वा॒य॒ । त्रि॒वृदिति॑ त्रि - वृत् । अ॒ग्नि॒ष्टो॒म इत्य॑ग्नि - स्तो॒मः । अ॒ग्नि॒ष्टुदित्य॑ग्नि-स्तुत् । आ॒ग्ने॒यीषु॑ । भ॒व॒ति॒ । तेजः॑ । ए॒व । अवेति॑ । रु॒न्धे॒ । प॒ञ्च॒द॒श इति॑ पञ्च-द॒शः । उ॒क्थ्यः॑ । ऐ॒न्द्रीषु॑ । इ॒न्द्रि॒यम् । ए॒व । अवेति॑ । रु॒न्धे॒ । त्रि॒व॒दिति॑ त्रि - वृत् । अ॒ग्नि॒ष्टो॒म इत्य॑ग्नि - स्तो॒मः । वै॒श्व॒दे॒वीष्विति॑ वैश्व - दे॒वीषु॑ । पुष्टि᳚म् । ए॒व । अवेति॑ । रु॒न्धे॒ । स॒प्त॒द॒श इति॑ सप्त - द॒शः । अ॒ग्नि॒ष्टो॒म इत्य॑ग्नि - स्तो॒मः । प्रा॒जा॒प॒त्यास्विति॑ प्राजा - प॒त्यासु॑ । ती॒व्र॒सो॒म इति॑ तीव्र-सो॒मः । अ॒न्नाद्य॒स्येत्य॑न्न - अद्य॑स्य । अव॑रुद्ध्या॒ इत्यव॑-रुद्॒ध्यै॒ । अथो॒ इति॑ । प्रेति॑ । ए॒व । तेन॑ । जा॒य॒ते॒ ।  \newline


\textbf{Krama Paata} \newline

ए॒तत् कु॑र्वन्ति । कु॒र्व॒न्ति॒ यत् । यज् ज्यायाꣳ॑सम् । ज्यायाꣳ॑सꣳ॒॒ स्तोम᳚म् । स्तोम॑मु॒पेत्य॑ । उ॒पेत्य॒ कनी॑याꣳसम् । उ॒पेत्येत्यु॑प - इत्य॑ । कनी॑याꣳसमुप॒यन्ति॑ । उ॒प॒यन्ति॒ यत् । उ॒प॒यन्तीत्यु॑प - यन्ति॑ । यद॑ग्निष्टोमसा॒मानि॑ । अ॒ग्नि॒ष्टो॒म॒सा॒मान्य॒वस्ता᳚त् । अ॒ग्नि॒ष्टो॒म॒सा॒मानीत्य॑ग्निष्टोम - सा॒मानि॑ । अ॒वस्ता᳚च् च । च॒ प॒रस्ता᳚त् । प॒रस्ता᳚च् च । च॒ भव॑न्ति । भव॒न्त्यजा॑मित्वाय । अजा॑मित्वाय त्रि॒वृत् । अजा॑मित्वा॒येत्यजा॑मि - त्वा॒य॒ । त्रि॒वृद॑ग्निष्टो॒मः । त्रि॒वृदिति॑ त्रि - वृत् । अ॒ग्नि॒ष्टो॒मो᳚ऽग्नि॒ष्टुत् । अ॒ग्नि॒ष्टो॒म इत्य॑ग्नि - स्तो॒मः । अ॒ग्नि॒ष्टुदा᳚ग्ने॒यीषु॑ । अ॒ग्नि॒ष्टुदित्य॑ग्नि - स्तुत् । आ॒ग्ने॒यीषु॑ भवति । भ॒व॒ति॒ तेजः॑ । तेज॑ ए॒व । ए॒वाव॑ । अव॑ रुन्धे । रु॒न्धे॒ प॒ञ्च॒द॒शः । प॒ञ्च॒द॒श उ॒क्थ्यः॑ । प॒ञ्च॒द॒श इति॑ पञ्च - द॒शः । उ॒क्थ्य॑ ऐ॒न्द्रीषु॑ । ऐ॒न्द्रीष्वि॑न्द्रि॒यम् । इ॒न्द्रि॒यमे॒व । ए॒वाव॑ । अव॑ रुन्धे । रु॒न्धे॒ त्रि॒वृत् । त्रि॒वृद॑ग्निष्टो॒मः । त्रि॒वृदिति॑ त्रि - वृत् । अ॒ग्नि॒ष्टो॒मो वै᳚श्वदे॒वीषु॑ । अ॒ग्नि॒ष्टो॒म इत्य॑ग्नि - स्तो॒मः । वै॒श्व॒दे॒वीषु॒ पुष्टि᳚म् । वै॒श्व॒दे॒वीष्विति॑ वैश्व - दे॒वीषु॑ । पुष्टि॑मे॒व । ए॒वाव॑ । अव॑ रुन्धे । रु॒न्धे॒ स॒प्त॒द॒शः । स॒प्त॒द॒शो᳚ऽग्निष्टो॒मः । स॒प्त॒द॒श इति॑ सप्त - द॒शः । अ॒ग्नि॒ष्टो॒मः प्रा॑जाप॒त्यासु॑ । अ॒ग्नि॒ष्टो॒म इत्य॑ग्नि - स्तो॒मः । प्रा॒जा॒प॒त्यासु॑ तीव्रसो॒मः । प्रा॒जा॒प॒त्यास्विति॑ प्राजा - प॒त्यासु॑ । ती॒व्र॒सो॒मो᳚ऽन्नाद्य॑स्य । ती॒व्र॒सो॒म इति॑ तीव्र - सो॒मः । अ॒न्नाद्य॒स्या,व॑रुद्ध्यै । अ॒न्नाद्य॒स्येत्य॑न्न - अद्य॑स्य । अव॑रुद्ध्या॒ अथो᳚ । अव॑रुद्ध्या॒ इत्यव॑ - रु॒द्ध्यै॒ । अथो॒ प्र । अथो॒ इत्यथो᳚ । प्रैव । ए॒व तेन॑ । तेन॑ जायते ( ) । जा॒य॒त॒ ए॒क॒विꣳ॒॒शः \newline

\textbf{Jatai Paata} \newline

1. ए॒तत् कु॑र्वन्ति कुर्व न्त्ये॒त दे॒तत् कु॑र्वन्ति । \newline
2. कु॒र्व॒न्ति॒ यद् यत् कु॑र्वन्ति कुर्वन्ति॒ यत् । \newline
3. यज् ज्यायाꣳ॑स॒म् ज्यायाꣳ॑सं॒ ॅयद् यज् ज्यायाꣳ॑सम् । \newline
4. ज्यायाꣳ॑सꣳ॒॒ स्तोमꣳ॒॒ स्तोम॒म् ज्यायाꣳ॑स॒म् ज्यायाꣳ॑सꣳ॒॒ स्तोम᳚म् । \newline
5. स्तोम॑ मु॒पे त्यो॒पेत्य॒ स्तोमꣳ॒॒ स्तोम॑ मु॒पेत्य॑ । \newline
6. उ॒पेत्य॒ कनी॑याꣳस॒म् कनी॑याꣳस मु॒पे त्यो॒पेत्य॒ कनी॑याꣳसम् । \newline
7. उ॒पेत्येत्यु॑प - इत्य॑ । \newline
8. कनी॑याꣳस मुप॒य न्त्यु॑प॒यन्ति॒ कनी॑याꣳस॒म् कनी॑याꣳस मुप॒यन्ति॑ । \newline
9. उ॒प॒यन्ति॒ यद् यदु॑प॒य न्त्यु॑प॒यन्ति॒ यत् । \newline
10. उ॒प॒यन्तीत्यु॑प - यन्ति॑ । \newline
11. यद॑ग्निष्टोमसा॒मा न्य॑ग्निष्टोमसा॒मानि॒ यद् यद॑ग्निष्टोमसा॒मानि॑ । \newline
12. अ॒ग्नि॒ष्टो॒म॒सा॒मा न्य॒वस्ता॑ द॒वस्ता॑ दग्निष्टोमसा॒मा न्य॑ग्निष्टोमसा॒मा न्य॒वस्ता᳚त् । \newline
13. अ॒ग्नि॒ष्टो॒म॒सा॒मानीत्य॑ग्निष्टोम - सा॒मानि॑ । \newline
14. अ॒वस्ता᳚च् च चा॒वस्ता॑ द॒वस्ता᳚च् च । \newline
15. च॒ प॒रस्ता᳚त् प॒रस्ता᳚च् च च प॒रस्ता᳚त् । \newline
16. प॒रस्ता᳚च् च च प॒रस्ता᳚त् प॒रस्ता᳚च् च । \newline
17. च॒ भव॑न्ति॒ भव॑न्ति च च॒ भव॑न्ति । \newline
18. भव॒ न्त्यजा॑मित्वा॒या जा॑मित्वाय॒ भव॑न्ति॒ भव॒ न्त्यजा॑मित्वाय । \newline
19. अजा॑मित्वाय त्रि॒वृत् त्रि॒वृ दजा॑मित्वा॒या जा॑मित्वाय त्रि॒वृत् । \newline
20. अजा॑मित्वा॒येत्यजा॑मि - त्वा॒य॒ । \newline
21. त्रि॒वृ द॑ग्निष्टो॒मो᳚ ऽग्निष्टो॒म स्त्रि॒वृत् त्रि॒वृ द॑ग्निष्टो॒मः । \newline
22. त्रि॒वृदिति॑ त्रि - वृत् । \newline
23. अ॒ग्नि॒ष्टो॒मो᳚ ऽग्नि॒ष्टु द॑ग्नि॒ष्टु द॑ग्निष्टो॒मो᳚ ऽग्निष्टो॒मो᳚ ऽग्नि॒ष्टुत् । \newline
24. अ॒ग्नि॒ष्टो॒म इत्य॑ग्नि - स्तो॒मः । \newline
25. अ॒ग्नि॒ष्टु दा᳚ग्ने॒यी ष्वा᳚ग्ने॒यी ष्व॑ग्नि॒ष्टु द॑ग्नि॒ष्टु दा᳚ग्ने॒यीषु॑ । \newline
26. अ॒ग्नि॒ष्टुदित्य॑ग्नि - स्तुत् । \newline
27. आ॒ग्ने॒यीषु॑ भवति भव त्याग्ने॒यी ष्वा᳚ग्ने॒यीषु॑ भवति । \newline
28. भ॒व॒ति॒ तेज॒ स्तेजो॑ भवति भवति॒ तेजः॑ । \newline
29. तेज॑ ए॒वैव तेज॒ स्तेज॑ ए॒व । \newline
30. ए॒वावा वै॒वै वाव॑ । \newline
31. अव॑ रुन्धे रु॒न्धे ऽवाव॑ रुन्धे । \newline
32. रु॒न्धे॒ प॒ञ्च॒द॒शः प॑ञ्चद॒शो रु॑न्धे रुन्धे पञ्चद॒शः । \newline
33. प॒ञ्च॒द॒श उ॒क्थ्य॑ उ॒क्थ्यः॑ पञ्चद॒शः प॑ञ्चद॒श उ॒क्थ्यः॑ । \newline
34. प॒ञ्च॒द॒श इति॑ पञ्च - द॒शः । \newline
35. उ॒क्थ्य॑ ऐ॒न्द्री ष्वै॒न्द्रीषू॒क्थ्य॑ उ॒क्थ्य॑ ऐ॒न्द्रीषु॑ । \newline
36. ऐ॒न्द्री ष्वि॑न्द्रि॒य मि॑न्द्रि॒य मै॒न्द्री ष्वै॒न्द्री ष्वि॑न्द्रि॒यम् । \newline
37. इ॒न्द्रि॒य मे॒वैवेन्द्रि॒य मि॑न्द्रि॒य मे॒व । \newline
38. ए॒वावा वै॒वै वाव॑ । \newline
39. अव॑ रुन्धे रु॒न्धे ऽवाव॑ रुन्धे । \newline
40. रु॒न्धे॒ त्रि॒वृत् त्रि॒वृद् रु॑न्धे रुन्धे त्रि॒वृत् । \newline
41. त्रि॒वृ द॑ग्निष्टो॒मो᳚ ऽग्निष्टो॒म स्त्रि॒वृत् त्रि॒वृ द॑ग्निष्टो॒मः । \newline
42. त्रि॒व॒दिति॑ त्रि - वृत् । \newline
43. अ॒ग्नि॒ष्टो॒मो वै᳚श्वदे॒वीषु॑ वैश्वदे॒वी ष्व॑ग्निष्टो॒मो᳚ ऽग्निष्टो॒मो वै᳚श्वदे॒वीषु॑ । \newline
44. अ॒ग्नि॒ष्टो॒म इत्य॑ग्नि - स्तो॒मः । \newline
45. वै॒श्व॒दे॒वीषु॒ पुष्टि॒म् पुष्टिं॑ ॅवैश्वदे॒वीषु॑ वैश्वदे॒वीषु॒ पुष्टि᳚म् । \newline
46. वै॒श्व॒दे॒वीष्विति॑ वैश्व - दे॒वीषु॑ । \newline
47. पुष्टि॑ मे॒वैव पुष्टि॒म् पुष्टि॑ मे॒व । \newline
48. ए॒वावा वै॒वै वाव॑ । \newline
49. अव॑ रुन्धे रु॒न्धे ऽवाव॑ रुन्धे । \newline
50. रु॒न्धे॒ स॒प्त॒द॒शः स॑प्तद॒शो रु॑न्धे रुन्धे सप्तद॒शः । \newline
51. स॒प्त॒द॒शो᳚ ऽग्निष्टो॒मो᳚ ऽग्निष्टो॒मः स॑प्तद॒शः स॑प्तद॒शो᳚ ऽग्निष्टो॒मः । \newline
52. स॒प्त॒द॒श इति॑ सप्त - द॒शः । \newline
53. अ॒ग्नि॒ष्टो॒मः प्रा॑जाप॒त्यासु॑ प्राजाप॒त्या स्व॑ग्निष्टो॒मो᳚ ऽग्निष्टो॒मः प्रा॑जाप॒त्यासु॑ । \newline
54. अ॒ग्नि॒ष्टो॒म इत्य॑ग्नि - स्तो॒मः । \newline
55. प्रा॒जा॒प॒त्यासु॑ तीव्रसो॒म स्ती᳚व्रसो॒मः प्रा॑जाप॒त्यासु॑ प्राजाप॒त्यासु॑ तीव्रसो॒मः । \newline
56. प्रा॒जा॒प॒त्यास्विति॑ प्राजा - प॒त्यासु॑ । \newline
57. ती॒व्र॒सो॒मो᳚ ऽन्नाद्य॑स्या॒ न्नाद्य॑स्य तीव्रसो॒म स्ती᳚व्रसो॒मो᳚ ऽन्नाद्य॑स्य । \newline
58. ती॒व्र॒सो॒म इति॑ तीव्र - सो॒मः । \newline
59. अ॒न्नाद्य॒स्या व॑रुद्ध्या॒ अव॑रुद्ध्या अ॒न्नाद्य॑स्या॒ न्नाद्य॒स्या व॑रुद्ध्यै । \newline
60. अ॒न्नाद्य॒स्येत्य॑न्न - अद्य॑स्य । \newline
61. अव॑रुद्ध्या॒ अथो॒ अथो॒ अव॑रुद्ध्या॒ अव॑रुद्ध्या॒ अथो᳚ । \newline
62. अव॑रुद्ध्या॒ इत्यव॑ - रुद्॒ध्यै॒ । \newline
63. अथो॒ प्र प्राथो॒ अथो॒ प्र । \newline
64. अथो॒ इत्यथो᳚ । \newline
65. प्रैवैव प्र प्रैव । \newline
66. ए॒व तेन॒ तेनै॒ वैव तेन॑ । \newline
67. तेन॑ जायते जायते॒ तेन॒ तेन॑ जायते । \newline
68. जा॒य॒त॒ ए॒क॒विꣳ॒॒श ए॑कविꣳ॒॒शो जा॑यते जायत एकविꣳ॒॒शः । \newline

\textbf{Ghana Paata } \newline

1. ए॒तत् कु॑र्वन्ति कुर्व न्त्ये॒त दे॒तत् कु॑र्वन्ति॒ यद् यत् कु॑र्व न्त्ये॒त दे॒तत् कु॑र्वन्ति॒ यत् । \newline
2. कु॒र्व॒न्ति॒ यद् यत् कु॑र्वन्ति कुर्वन्ति॒ यज् ज्यायाꣳ॑स॒म् ज्यायाꣳ॑सं॒ ॅयत् कु॑र्वन्ति कुर्वन्ति॒ यज् ज्यायाꣳ॑सम् । \newline
3. यज् ज्यायाꣳ॑स॒म् ज्यायाꣳ॑सं॒ ॅयद् यज् ज्यायाꣳ॑सꣳ॒॒ स्तोमꣳ॒॒ स्तोम॒म् ज्यायाꣳ॑सं॒ ॅयद् यज् ज्यायाꣳ॑सꣳ॒॒ स्तोम᳚म् । \newline
4. ज्यायाꣳ॑सꣳ॒॒ स्तोमꣳ॒॒ स्तोम॒म् ज्यायाꣳ॑स॒म् ज्यायाꣳ॑सꣳ॒॒ स्तोम॑ मु॒पे त्यो॒पेत्य॒ स्तोम॒म् ज्यायाꣳ॑स॒म् ज्यायाꣳ॑सꣳ॒॒ स्तोम॑ मु॒पेत्य॑ । \newline
5. स्तोम॑ मु॒पे त्यो॒पेत्य॒ स्तोमꣳ॒॒ स्तोम॑ मु॒पेत्य॒ कनी॑याꣳस॒म् कनी॑याꣳस मु॒पेत्य॒ स्तोमꣳ॒॒ स्तोम॑ मु॒पेत्य॒ कनी॑याꣳसम् । \newline
6. उ॒पेत्य॒ कनी॑याꣳस॒म् कनी॑याꣳस मु॒पे त्यो॒पेत्य॒ कनी॑याꣳस मुप॒य न्त्यु॑प॒यन्ति॒ कनी॑याꣳस मु॒पे त्यो॒पेत्य॒ कनी॑याꣳस मुप॒यन्ति॑ । \newline
7. उ॒पेत्येत्यु॑प - इत्य॑ । \newline
8. कनी॑याꣳस मुप॒य न्त्यु॑प॒यन्ति॒ कनी॑याꣳस॒म् कनी॑याꣳस मुप॒यन्ति॒ यद् यदु॑प॒यन्ति॒ कनी॑याꣳस॒म् कनी॑याꣳस मुप॒यन्ति॒ यत् । \newline
9. उ॒प॒यन्ति॒ यद् यदु॑प॒य न्त्यु॑प॒यन्ति॒ यद॑ग्निष्टोमसा॒मा न्य॑ग्निष्टोमसा॒मानि॒ यदु॑प॒य न्त्यु॑प॒यन्ति॒ यद॑ग्निष्टोमसा॒मानि॑ । \newline
10. उ॒प॒यन्तीत्यु॑प - यन्ति॑ । \newline
11. यद॑ग्निष्टोमसा॒मा न्य॑ग्निष्टोमसा॒मानि॒ यद् यद॑ग्निष्टोमसा॒मा न्य॒वस्ता॑ द॒वस्ता॑ दग्निष्टोमसा॒मानि॒ यद् यद॑ग्निष्टोमसा॒मा न्य॒वस्ता᳚त् । \newline
12. अ॒ग्नि॒ष्टो॒म॒सा॒मा न्य॒वस्ता॑ द॒वस्ता॑ दग्निष्टोमसा॒मा न्य॑ग्निष्टोमसा॒मा न्य॒वस्ता᳚च् च चा॒वस्ता॑ दग्निष्टोमसा॒मा न्य॑ग्निष्टोमसा॒मा न्य॒वस्ता᳚च् च । \newline
13. अ॒ग्नि॒ष्टो॒म॒सा॒मानीत्य॑ग्निष्टोम - सा॒मानि॑ । \newline
14. अ॒वस्ता᳚च् च चा॒वस्ता॑ द॒वस्ता᳚च् च प॒रस्ता᳚त् प॒रस्ता᳚च् चा॒वस्ता॑ द॒वस्ता᳚च् च प॒रस्ता᳚त् । \newline
15. च॒ प॒रस्ता᳚त् प॒रस्ता᳚च् च च प॒रस्ता᳚च् च च प॒रस्ता᳚च् च च प॒रस्ता᳚च् च । \newline
16. प॒रस्ता᳚च् च च प॒रस्ता᳚त् प॒रस्ता᳚च् च॒ भव॑न्ति॒ भव॑न्ति च प॒रस्ता᳚त् प॒रस्ता᳚च् च॒ भव॑न्ति । \newline
17. च॒ भव॑न्ति॒ भव॑न्ति च च॒ भव॒ न्त्यजा॑मित्वा॒या जा॑मित्वाय॒ भव॑न्ति च च॒ भव॒ न्त्यजा॑मित्वाय । \newline
18. भव॒ न्त्यजा॑मित्वा॒या जा॑मित्वाय॒ भव॑न्ति॒ भव॒ न्त्यजा॑मित्वाय त्रि॒वृत् त्रि॒वृ दजा॑मित्वाय॒ भव॑न्ति॒ भव॒ न्त्यजा॑मित्वाय त्रि॒वृत् । \newline
19. अजा॑मित्वाय त्रि॒वृत् त्रि॒वृ दजा॑मित्वा॒या जा॑मित्वाय त्रि॒वृ द॑ग्निष्टो॒मो᳚ ऽग्निष्टो॒म स्त्रि॒वृ दजा॑मित्वा॒या जा॑मित्वाय त्रि॒वृ द॑ग्निष्टो॒मः । \newline
20. अजा॑मित्वा॒येत्यजा॑मि - त्वा॒य॒ । \newline
21. त्रि॒वृ द॑ग्निष्टो॒मो᳚ ऽग्निष्टो॒म स्त्रि॒वृत् त्रि॒वृ द॑ग्निष्टो॒मो᳚ ऽग्नि॒ष्टु द॑ग्नि॒ष्टु द॑ग्निष्टो॒म स्त्रि॒वृत् त्रि॒वृ द॑ग्निष्टो॒मो᳚ ऽग्नि॒ष्टुत् । \newline
22. त्रि॒वृदिति॑ त्रि - वृत् । \newline
23. अ॒ग्नि॒ष्टो॒मो᳚ ऽग्नि॒ष्टु द॑ग्नि॒ष्टु द॑ग्निष्टो॒मो᳚ ऽग्निष्टो॒मो᳚ ऽग्नि॒ष्टु दा᳚ग्ने॒यी ष्वा᳚ग्ने॒यी ष्व॑ग्नि॒ष्टु द॑ग्निष्टो॒मो᳚ ऽग्निष्टो॒मो᳚ ऽग्नि॒ष्टु दा᳚ग्ने॒यीषु॑ । \newline
24. अ॒ग्नि॒ष्टो॒म इत्य॑ग्नि - स्तो॒मः । \newline
25. अ॒ग्नि॒ष्टु दा᳚ग्ने॒यी ष्वा᳚ग्ने॒यी ष्व॑ग्नि॒ष्टु द॑ग्नि॒ष्टु दा᳚ग्ने॒यीषु॑ भवति भव त्याग्ने॒यी ष्व॑ग्नि॒ष्टु द॑ग्नि॒ष्टु दा᳚ग्ने॒यीषु॑ भवति । \newline
26. अ॒ग्नि॒ष्टुदित्य॑ग्नि - स्तुत् । \newline
27. आ॒ग्ने॒यीषु॑ भवति भव त्याग्ने॒यी ष्वा᳚ग्ने॒यीषु॑ भवति॒ तेज॒ स्तेजो॑ भव त्याग्ने॒यी ष्वा᳚ग्ने॒यीषु॑ भवति॒ तेजः॑ । \newline
28. भ॒व॒ति॒ तेज॒ स्तेजो॑ भवति भवति॒ तेज॑ ए॒वैव तेजो॑ भवति भवति॒ तेज॑ ए॒व । \newline
29. तेज॑ ए॒वैव तेज॒ स्तेज॑ ए॒वावा वै॒व तेज॒ स्तेज॑ ए॒वाव॑ । \newline
30. ए॒वावा वै॒वै वाव॑ रुन्धे रु॒न्धे ऽवै॒वै वाव॑ रुन्धे । \newline
31. अव॑ रुन्धे रु॒न्धे ऽवाव॑ रुन्धे पञ्चद॒शः प॑ञ्चद॒शो रु॒न्धे ऽवाव॑ रुन्धे पञ्चद॒शः । \newline
32. रु॒न्धे॒ प॒ञ्च॒द॒शः प॑ञ्चद॒शो रु॑न्धे रुन्धे पञ्चद॒श उ॒क्थ्य॑ उ॒क्थ्यः॑ पञ्चद॒शो रु॑न्धे रुन्धे पञ्चद॒श उ॒क्थ्यः॑ । \newline
33. प॒ञ्च॒द॒श उ॒क्थ्य॑ उ॒क्थ्यः॑ पञ्चद॒शः प॑ञ्चद॒श उ॒क्थ्य॑ ऐ॒न्द्री ष्वै॒न्द्रीषू॒क्थ्यः॑ पञ्चद॒शः प॑ञ्चद॒श उ॒क्थ्य॑ ऐ॒न्द्रीषु॑ । \newline
34. प॒ञ्च॒द॒श इति॑ पञ्च - द॒शः । \newline
35. उ॒क्थ्य॑ ऐ॒न्द्री ष्वै॒न्द्रीषू॒क्थ्य॑ उ॒क्थ्य॑ ऐ॒न्द्री ष्वि॑न्द्रि॒य मि॑न्द्रि॒य मै॒न्द्रीषू॒क्थ्य॑ उ॒क्थ्य॑ ऐ॒न्द्री 
ष्वि॑न्द्रि॒यम् । \newline
36. ऐ॒न्द्री ष्वि॑न्द्रि॒य मि॑न्द्रि॒य मै॒न्द्री ष्वै॒न्द्री ष्वि॑न्द्रि॒य मे॒वै वेन्द्रि॒य मै॒न्द्री ष्वै॒न्द्री ष्वि॑न्द्रि॒य मे॒व । \newline
37. इ॒न्द्रि॒य मे॒वै वेन्द्रि॒य मि॑न्द्रि॒य मे॒वावा वै॒वेन्द्रि॒य मि॑न्द्रि॒य मे॒वाव॑ । \newline
38. ए॒वावा वै॒वै वाव॑ रुन्धे रु॒न्धे ऽवै॒वै वाव॑ रुन्धे । \newline
39. अव॑ रुन्धे रु॒न्धे ऽवाव॑ रुन्धे त्रि॒वृत् त्रि॒वृद् रु॒न्धे ऽवाव॑ रुन्धे त्रि॒वृत् । \newline
40. रु॒न्धे॒ त्रि॒वृत् त्रि॒वृद् रु॑न्धे रुन्धे त्रि॒वृ द॑ग्निष्टो॒मो᳚ ऽग्निष्टो॒म स्त्रि॒वृद् रु॑न्धे रुन्धे त्रि॒वृ द॑ग्निष्टो॒मः । \newline
41. त्रि॒वृ द॑ग्निष्टो॒मो᳚ ऽग्निष्टो॒म स्त्रि॒वृत् त्रि॒वृ द॑ग्निष्टो॒मो वै᳚श्वदे॒वीषु॑ वैश्वदे॒वी ष्व॑ग्निष्टो॒म स्त्रि॒वृत् त्रि॒वृ द॑ग्निष्टो॒मो वै᳚श्वदे॒वीषु॑ । \newline
42. त्रि॒व॒दिति॑ त्रि - वृत् । \newline
43. अ॒ग्नि॒ष्टो॒मो वै᳚श्वदे॒वीषु॑ वैश्वदे॒वी ष्व॑ग्निष्टो॒मो᳚ ऽग्निष्टो॒मो वै᳚श्वदे॒वीषु॒ पुष्टि॒म् पुष्टिं॑ ॅवैश्वदे॒वी ष्व॑ग्निष्टो॒मो᳚ ऽग्निष्टो॒मो वै᳚श्वदे॒वीषु॒ पुष्टि᳚म् । \newline
44. अ॒ग्नि॒ष्टो॒म इत्य॑ग्नि - स्तो॒मः । \newline
45. वै॒श्व॒दे॒वीषु॒ पुष्टि॒म् पुष्टिं॑ ॅवैश्वदे॒वीषु॑ वैश्वदे॒वीषु॒ पुष्टि॑ मे॒वैव पुष्टिं॑ ॅवैश्वदे॒वीषु॑ वैश्वदे॒वीषु॒ पुष्टि॑ मे॒व । \newline
46. वै॒श्व॒दे॒वीष्विति॑ वैश्व - दे॒वीषु॑ । \newline
47. पुष्टि॑ मे॒वैव पुष्टि॒म् पुष्टि॑ मे॒वावा वै॒व पुष्टि॒म् पुष्टि॑ मे॒वाव॑ । \newline
48. ए॒वावा वै॒वै वाव॑ रुन्धे रु॒न्धे ऽवै॒वै वाव॑ रुन्धे । \newline
49. अव॑ रुन्धे रु॒न्धे ऽवाव॑ रुन्धे सप्तद॒शः स॑प्तद॒शो रु॒न्धे ऽवाव॑ रुन्धे सप्तद॒शः । \newline
50. रु॒न्धे॒ स॒प्त॒द॒शः स॑प्तद॒शो रु॑न्धे रुन्धे सप्तद॒शो᳚ ऽग्निष्टो॒मो᳚ ऽग्निष्टो॒मः स॑प्तद॒शो रु॑न्धे रुन्धे सप्तद॒शो᳚ ऽग्निष्टो॒मः । \newline
51. स॒प्त॒द॒शो᳚ ऽग्निष्टो॒मो᳚ ऽग्निष्टो॒मः स॑प्तद॒शः स॑प्तद॒शो᳚ ऽग्निष्टो॒मः प्रा॑जाप॒त्यासु॑ प्राजाप॒त्या स्व॑ग्निष्टो॒मः स॑प्तद॒शः स॑प्तद॒शो᳚ ऽग्निष्टो॒मः प्रा॑जाप॒त्यासु॑ । \newline
52. स॒प्त॒द॒श इति॑ सप्त - द॒शः । \newline
53. अ॒ग्नि॒ष्टो॒मः प्रा॑जाप॒त्यासु॑ प्राजाप॒त्या स्व॑ग्निष्टो॒मो᳚ ऽग्निष्टो॒मः प्रा॑जाप॒त्यासु॑ तीव्रसो॒म स्ती᳚व्रसो॒मः प्रा॑जाप॒त्या स्व॑ग्निष्टो॒मो᳚ ऽग्निष्टो॒मः प्रा॑जाप॒त्यासु॑ तीव्रसो॒मः । \newline
54. अ॒ग्नि॒ष्टो॒म इत्य॑ग्नि - स्तो॒मः । \newline
55. प्रा॒जा॒प॒त्यासु॑ तीव्रसो॒म स्ती᳚व्रसो॒मः प्रा॑जाप॒त्यासु॑ प्राजाप॒त्यासु॑ तीव्रसो॒मो᳚ ऽन्नाद्य॑स्या॒ न्नाद्य॑स्य तीव्रसो॒मः प्रा॑जाप॒त्यासु॑ प्राजाप॒त्यासु॑ तीव्रसो॒मो᳚ ऽन्नाद्य॑स्य । \newline
56. प्रा॒जा॒प॒त्यास्विति॑ प्राजा - प॒त्यासु॑ । \newline
57. ती॒व्र॒सो॒मो᳚ ऽन्नाद्य॑स्या॒ न्नाद्य॑स्य तीव्रसो॒म स्ती᳚व्रसो॒मो᳚ ऽन्नाद्य॒स्या व॑रुद्ध्या॒ अव॑रुद्ध्या अ॒न्नाद्य॑स्य तीव्रसो॒म स्ती᳚व्रसो॒मो᳚ ऽन्नाद्य॒स्या व॑रुद्ध्यै । \newline
58. ती॒व्र॒सो॒म इति॑ तीव्र - सो॒मः । \newline
59. अ॒न्नाद्य॒स्या व॑रुद्ध्या॒ अव॑रुद्ध्या अ॒न्नाद्य॑स्या॒ न्नाद्य॒स्या व॑रुद्ध्या॒ अथो॒ अथो॒ अव॑रुद्ध्या अ॒न्नाद्य॑ स्या॒ न्नाद्य॒स्या व॑रुद्ध्या॒ अथो᳚ । \newline
60. अ॒न्नाद्य॒स्येत्य॑न्न - अद्य॑स्य । \newline
61. अव॑रुद्ध्या॒ अथो॒ अथो॒ अव॑रुद्ध्या॒ अव॑रुद्ध्या॒ अथो॒ प्र प्राथो॒ अव॑रुद्ध्या॒ अव॑रुद्ध्या॒ अथो॒ प्र । \newline
62. अव॑रुद्ध्या॒ इत्यव॑ - रुद्॒ध्यै॒ । \newline
63. अथो॒ प्र प्राथो॒ अथो॒ प्रैवैव प्राथो॒ अथो॒ प्रैव । \newline
64. अथो॒ इत्यथो᳚ । \newline
65. प्रैवैव प्र प्रैव तेन॒ तेनै॒व प्र प्रैव तेन॑ । \newline
66. ए॒व तेन॒ तेनै॒ वैव तेन॑ जायते जायते॒ तेनै॒ वैव तेन॑ जायते । \newline
67. तेन॑ जायते जायते॒ तेन॒ तेन॑ जायत एकविꣳ॒॒श ए॑कविꣳ॒॒शो जा॑यते॒ तेन॒ तेन॑ जायत एकविꣳ॒॒शः । \newline
68. जा॒य॒त॒ ए॒क॒विꣳ॒॒श ए॑कविꣳ॒॒शो जा॑यते जायत एकविꣳ॒॒श उ॒क्थ्य॑ उ॒क्थ्य॑ एकविꣳ॒॒शो जा॑यते जायत एकविꣳ॒॒श उ॒क्थ्यः॑ । \newline
\pagebreak
\markright{ TS 7.2.5.6  \hfill https://www.vedavms.in \hfill}

\section{ TS 7.2.5.6 }

\textbf{TS 7.2.5.6 } \newline
\textbf{Samhita Paata} \newline

एकविꣳ॒॒श उ॒क्थ्यः॑ सौ॒रीषु॒ प्रति॑ष्ठित्या॒ अथो॒ रुच॑मे॒वाऽऽ*त्मन् ध॑त्ते सप्तद॒शो᳚ऽग्निष्टो॒मः प्रा॑जाप॒त्यासू॑पह॒व्य॑ उपह॒वमे॒व ग॑च्छति त्रिण॒वाव॑ग्निष्टो॒माव॒भित॑ ऐ॒न्द्रीषु॒ विजि॑त्यै त्रयस्त्रिꣳ॒॒श उ॒क्थ्यो॑ वैश्वदे॒वीषु॒ प्रति॑ष्ठित्यै विश्व॒जिथ् सर्व॑पृष्ठोऽतिरा॒त्रो भ॑वति॒ सर्व॑स्या॒भिजि॑त्यै ॥ \newline

\textbf{Pada Paata} \newline

ए॒क॒विꣳ॒॒श इत्ये॑क - विꣳ॒॒शः । उ॒क्थ्यः॑ । सौ॒रीषु॑ । प्रति॑ष्ठित्या॒ इति॒ प्रति॑ - स्थि॒त्यै॒ । अथो॒ इति॑ । रुच᳚म् । ए॒व । आ॒त्मन्न् । ध॒त्ते॒ । स॒प्त॒द॒श इति॑ सप्त - द॒शः । अ॒ग्नि॒ष्टो॒म इत्य॑ग्नि - स्तो॒मः । प्रा॒जा॒प॒त्यास्विति॑ प्राजा - प॒त्यासु॑ । उ॒प॒ह॒व्य॑ इत्यु॑प - ह॒व्यः॑ । उ॒प॒ह॒वमित्यु॑प - ह॒वम् । ए॒व । ग॒च्छ॒ति॒ । त्रि॒ण॒वाविति॑ त्रि - न॒वौ । अ॒ग्नि॒ष्टो॒मावित्य॑ग्नि - स्तो॒मौ । अ॒भितः॑ । ऐ॒न्द्रीषु॑ । विजि॑त्या॒ इति॒ वि - जि॒त्यै॒ । त्र॒य॒स्त्रिꣳ॒॒श इति॑ त्रयः - त्रिꣳ॒॒शः । उ॒क्थ्यः॑ । वै॒श्व॒दे॒वीष्विति॑ वैश्व - दे॒वीषु॑ । प्रति॑ष्ठित्या॒ इति॒ प्रति॑ - स्थि॒त्यै॒ । वि॒श्व॒जिदिति॑ विश्व - जित् । सर्व॑पृष्ठ॒ इति॒ सर्व॑ - पृ॒ष्ठः॒ । अ॒ति॒रा॒त्र इत्य॑ति - रा॒त्रः । भ॒व॒ति॒ । सर्व॑स्य । अ॒भिजि॑त्या॒ इत्य॒भि - जि॒त्यै॒ ॥  \newline


\textbf{Krama Paata} \newline

ए॒क॒विꣳ॒॒श उ॒क्थ्यः॑ । ए॒क॒विꣳ॒॒श इत्ये॑क - विꣳ॒॒शः । उ॒क्थ्यः॑ सौ॒रीषु॑ । सौ॒रीषु॒ प्रति॑ष्ठित्यै । प्रति॑ष्ठित्या॒ अथो᳚ । प्रति॑ष्ठित्या॒ इति॒ प्रति॑ - स्थि॒त्यै॒ । अथो॒ रुच᳚म् । अथो॒ इत्यथो᳚ । रुच॑मे॒व । ए॒वात्मन्न् । आ॒त्मन् ध॑त्ते । ध॒त्ते॒ स॒प्त॒द॒शः । स॒प्त॒द॒शो᳚ऽग्निष्टो॒मः । स॒प्त॒द॒श इति॑ सप्त - द॒शः । अ॒ग्नि॒ष्टो॒मः प्रा॑जाप॒त्यासु॑ । अ॒ग्नि॒ष्टो॒म इत्य॑ग्नि - स्तो॒मः । प्रा॒जा॒प॒त्यासू॑पह॒व्यः॑ । प्रा॒जा॒प॒त्यास्विति॑ प्राजा - प॒त्यासु॑ । उ॒प॒ह॒व्य॑ उपह॒वम् । उ॒प॒ह॒व्य॑ इत्यु॑प - ह॒व्यः॑ । उ॒प॒ह॒वमे॒व । उ॒प॒ह॒वमित्यु॑प - ह॒वम् । ए॒व ग॑च्छति । ग॒च्छ॒ति॒ त्रि॒ण॒वौ । त्रि॒ण॒वाव॑ग्निष्टो॒मौ । त्रि॒ण॒वाविति॑ त्रि - न॒वौ । अ॒ग्नि॒ष्टो॒माव॒भितः॑ । अ॒ग्नि॒ष्टो॒मावित्य॑ग्नि - स्तौ॒मौ । अ॒भित॑ ऐ॒न्द्रीषु॑ । ऐ॒न्द्रीषु॒ विजि॑त्यै । विजि॑त्यै त्रयस्त्रिꣳ॒॒शः । विजि॑त्या॒ इति॒ वि - जि॒त्यै॒ । त्र॒य॒स्त्रिꣳ॒॒श उ॒क्थ्यः॑ । त्र॒य॒स्त्रिꣳ॒॒श इति॑ त्रयः - त्रिꣳ॒॒शः । उ॒क्थ्यो॑ वैश्वदे॒वीषु॑ । वै॒श्व॒दे॒वीषु॒ प्रति॑ष्ठित्यै । वै॒श्व॒दे॒वीष्विति॑ वैश्व - दे॒वीषु॑ । प्रति॑ष्ठित्यै विश्व॒जित् । प्रति॑ष्ठित्या॒ इति॒ प्रति॑ - स्थि॒त्यै॒ । वि॒श्व॒जिथ् सर्व॑पृष्ठः । वि॒श्व॒जिदिति॑ विश्व - जित् । सर्व॑पृष्ठोऽतिरा॒त्रः । सर्व॑पृष्ठ॒ इति॒ सर्व॑ - पृ॒ष्ठः॒ । अ॒ति॒रा॒त्रो भ॑वति । अ॒ति॒रा॒त्र इत्य॑ति - रा॒त्रः । भ॒व॒ति॒ सर्व॑स्य । सर्व॑स्या॒भिजि॑त्यै । अ॒भिजि॑त्या॒ इत्य॒भि - जि॒त्यै॒ । \newline

\textbf{Jatai Paata} \newline

1. ए॒क॒विꣳ॒॒श उ॒क्थ्य॑ उ॒क्थ्य॑ एकविꣳ॒॒श ए॑कविꣳ॒॒श उ॒क्थ्यः॑ । \newline
2. ए॒क॒विꣳ॒॒श इत्ये॑क - विꣳ॒॒शः । \newline
3. उ॒क्थ्यः॑ सौ॒रीषु॑ सौ॒रीषू॒क्थ्य॑ उ॒क्थ्यः॑ सौ॒रीषु॑ । \newline
4. सौ॒रीषु॒ प्रति॑ष्ठित्यै॒ प्रति॑ष्ठित्यै सौ॒रीषु॑ सौ॒रीषु॒ प्रति॑ष्ठित्यै । \newline
5. प्रति॑ष्ठित्या॒ अथो॒ अथो॒ प्रति॑ष्ठित्यै॒ प्रति॑ष्ठित्या॒ अथो᳚ । \newline
6. प्रति॑ष्ठित्या॒ इति॒ प्रति॑ - स्थि॒त्यै॒ । \newline
7. अथो॒ रुचꣳ॒॒ रुच॒ मथो॒ अथो॒ रुच᳚म् । \newline
8. अथो॒ इत्यथो᳚ । \newline
9. रुच॑ मे॒वैव रुचꣳ॒॒ रुच॑ मे॒व । \newline
10. ए॒वात्मन् ना॒त्मन् ने॒वै वात्मन्न् । \newline
11. आ॒त्मन् ध॑त्ते धत्त आ॒त्मन् ना॒त्मन् ध॑त्ते । \newline
12. ध॒त्ते॒ स॒प्त॒द॒शः स॑प्तद॒शो ध॑त्ते धत्ते सप्तद॒शः । \newline
13. स॒प्त॒द॒शो᳚ ऽग्निष्टो॒मो᳚ ऽग्निष्टो॒मः स॑प्तद॒शः स॑प्तद॒शो᳚ ऽग्निष्टो॒मः । \newline
14. स॒प्त॒द॒श इति॑ सप्त - द॒शः । \newline
15. अ॒ग्नि॒ष्टो॒मः प्रा॑जाप॒त्यासु॑ प्राजाप॒त्या स्व॑ग्निष्टो॒मो᳚ ऽग्निष्टो॒मः प्रा॑जाप॒त्यासु॑ । \newline
16. अ॒ग्नि॒ष्टो॒म इत्य॑ग्नि - स्तो॒मः । \newline
17. प्रा॒जा॒प॒त्यासू॑ पह॒व्य॑ उपह॒व्यः॑ प्राजाप॒त्यासु॑ प्राजाप॒त्यासू॑ पह॒व्यः॑ । \newline
18. प्रा॒जा॒प॒त्यास्विति॑ प्राजा - प॒त्यासु॑ । \newline
19. उ॒प॒ह॒व्य॑ उपह॒व मु॑पह॒व मु॑पह॒व्य॑ उपह॒व्य॑ उपह॒वम् । \newline
20. उ॒प॒ह॒व्य॑ इत्यु॑प - ह॒व्यः॑ । \newline
21. उ॒प॒ह॒व मे॒वैवोप॑ह॒व मु॑पह॒व मे॒व । \newline
22. उ॒प॒ह॒वमित्यु॑प - ह॒वम् । \newline
23. ए॒व ग॑च्छति गच्छ त्ये॒वैव ग॑च्छति । \newline
24. ग॒च्छ॒ति॒ त्रि॒ण॒वौ त्रि॑ण॒वौ ग॑च्छति गच्छति त्रिण॒वौ । \newline
25. त्रि॒ण॒वा व॑ग्निष्टो॒मा व॑ग्निष्टो॒मौ त्रि॑ण॒वौ त्रि॑ण॒वा व॑ग्निष्टो॒मौ । \newline
26. त्रि॒ण॒वाविति॑ त्रि - न॒वौ । \newline
27. अ॒ग्नि॒ष्टो॒मा व॒भितो॒ ऽभितो᳚ ऽग्निष्टो॒मा व॑ग्निष्टो॒मा व॒भितः॑ । \newline
28. अ॒ग्नि॒ष्टो॒मावित्य॑ग्नि - स्तो॒मौ । \newline
29. अ॒भित॑ ऐ॒न्द्री ष्वै॒न्द्री ष्व॒भितो॒ ऽभित॑ ऐ॒न्द्रीषु॑ । \newline
30. ऐ॒न्द्रीषु॒ विजि॑त्यै॒ विजि॑त्या ऐ॒न्द्री ष्वै॒न्द्रीषु॒ विजि॑त्यै । \newline
31. विजि॑त्यै त्रयस्त्रिꣳ॒॒श स्त्र॑यस्त्रिꣳ॒॒शो विजि॑त्यै॒ विजि॑त्यै त्रयस्त्रिꣳ॒॒शः । \newline
32. विजि॑त्या॒ इति॒ वि - जि॒त्यै॒ । \newline
33. त्र॒य॒स्त्रिꣳ॒॒श उ॒क्थ्य॑ उ॒क्थ्य॑ स्त्रयस्त्रिꣳ॒॒श स्त्र॑यस्त्रिꣳ॒॒श उ॒क्थ्यः॑ । \newline
34. त्र॒य॒स्त्रिꣳ॒॒श इति॑ त्रयः - त्रिꣳ॒॒शः । \newline
35. उ॒क्थ्यो॑ वैश्वदे॒वीषु॑ वैश्वदे॒वीषू॒क्थ्य॑ उ॒क्थ्यो॑ वैश्वदे॒वीषु॑ । \newline
36. वै॒श्व॒दे॒वीषु॒ प्रति॑ष्ठित्यै॒ प्रति॑ष्ठित्यै वैश्वदे॒वीषु॑ वैश्वदे॒वीषु॒ प्रति॑ष्ठित्यै । \newline
37. वै॒श्व॒दे॒वीष्विति॑ वैश्व - दे॒वीषु॑ । \newline
38. प्रति॑ष्ठित्यै विश्व॒जिद् वि॑श्व॒जित् प्रति॑ष्ठित्यै॒ प्रति॑ष्ठित्यै विश्व॒जित् । \newline
39. प्रति॑ष्ठित्या॒ इति॒ प्रति॑ - स्थि॒त्यै॒ । \newline
40. वि॒श्व॒जिथ् सर्व॑पृष्ठः॒ सर्व॑पृष्ठो विश्व॒जिद् वि॑श्व॒जिथ् सर्व॑पृष्ठः । \newline
41. वि॒श्व॒जिदिति॑ विश्व - जित् । \newline
42. सर्व॑पृष्ठो ऽतिरा॒त्रो॑ ऽतिरा॒त्रः सर्व॑पृष्ठः॒ सर्व॑पृष्ठो ऽतिरा॒त्रः । \newline
43. सर्व॑पृष्ठ॒ इति॒ सर्व॑ - पृ॒ष्ठः॒ । \newline
44. अ॒ति॒रा॒त्रो भ॑वति भव त्यतिरा॒त्रो॑ ऽतिरा॒त्रो भ॑वति । \newline
45. अ॒ति॒रा॒त्र इत्य॑ति - रा॒त्रः । \newline
46. भ॒व॒ति॒ सर्व॑स्य॒ सर्व॑स्य भवति भवति॒ सर्व॑स्य । \newline
47. सर्व॑स्या॒ भिजि॑त्या अ॒भिजि॑त्यै॒ सर्व॑स्य॒ सर्व॑स्या॒ भिजि॑त्यै । \newline
48. अ॒भिजि॑त्या॒ इत्य॒भि - जि॒त्यै॒ । \newline

\textbf{Ghana Paata } \newline

1. ए॒क॒विꣳ॒॒श उ॒क्थ्य॑ उ॒क्थ्य॑ एकविꣳ॒॒श ए॑कविꣳ॒॒श उ॒क्थ्यः॑ सौ॒रीषु॑ सौ॒रीषू॒क्थ्य॑ एकविꣳ॒॒श ए॑कविꣳ॒॒श उ॒क्थ्यः॑ सौ॒रीषु॑ । \newline
2. ए॒क॒विꣳ॒॒श इत्ये॑क - विꣳ॒॒शः । \newline
3. उ॒क्थ्यः॑ सौ॒रीषु॑ सौ॒रीषू॒क्थ्य॑ उ॒क्थ्यः॑ सौ॒रीषु॒ प्रति॑ष्ठित्यै॒ प्रति॑ष्ठित्यै सौ॒रीषू॒क्थ्य॑ उ॒क्थ्यः॑ सौ॒रीषु॒ प्रति॑ष्ठित्यै । \newline
4. सौ॒रीषु॒ प्रति॑ष्ठित्यै॒ प्रति॑ष्ठित्यै सौ॒रीषु॑ सौ॒रीषु॒ प्रति॑ष्ठित्या॒ अथो॒ अथो॒ प्रति॑ष्ठित्यै सौ॒रीषु॑ सौ॒रीषु॒ प्रति॑ष्ठित्या॒ अथो᳚ । \newline
5. प्रति॑ष्ठित्या॒ अथो॒ अथो॒ प्रति॑ष्ठित्यै॒ प्रति॑ष्ठित्या॒ अथो॒ रुचꣳ॒॒ रुच॒ मथो॒ प्रति॑ष्ठित्यै॒ प्रति॑ष्ठित्या॒ अथो॒ रुच᳚म् । \newline
6. प्रति॑ष्ठित्या॒ इति॒ प्रति॑ - स्थि॒त्यै॒ । \newline
7. अथो॒ रुचꣳ॒॒ रुच॒ मथो॒ अथो॒ रुच॑ मे॒वैव रुच॒ मथो॒ अथो॒ रुच॑ मे॒व । \newline
8. अथो॒ इत्यथो᳚ । \newline
9. रुच॑ मे॒वैव रुचꣳ॒॒ रुच॑ मे॒वात्मन् ना॒त्मन् ने॒व रुचꣳ॒॒ रुच॑ मे॒वात्मन्न् । \newline
10. ए॒वात्मन् ना॒त्मन् ने॒वै वात्मन् ध॑त्ते धत्त आ॒त्मन् ने॒वै वात्मन् ध॑त्ते । \newline
11. आ॒त्मन् ध॑त्ते धत्त आ॒त्मन् ना॒त्मन् ध॑त्ते सप्तद॒शः स॑प्तद॒शो ध॑त्त आ॒त्मन् ना॒त्मन् ध॑त्ते सप्तद॒शः । \newline
12. ध॒त्ते॒ स॒प्त॒द॒शः स॑प्तद॒शो ध॑त्ते धत्ते सप्तद॒शो᳚ ऽग्निष्टो॒मो᳚ ऽग्निष्टो॒मः स॑प्तद॒शो ध॑त्ते धत्ते सप्तद॒शो᳚ ऽग्निष्टो॒मः । \newline
13. स॒प्त॒द॒शो᳚ ऽग्निष्टो॒मो᳚ ऽग्निष्टो॒मः स॑प्तद॒शः स॑प्तद॒शो᳚ ऽग्निष्टो॒मः प्रा॑जाप॒त्यासु॑ प्राजाप॒त्या स्व॑ग्निष्टो॒मः स॑प्तद॒शः स॑प्तद॒शो᳚ ऽग्निष्टो॒मः प्रा॑जाप॒त्यासु॑ । \newline
14. स॒प्त॒द॒श इति॑ सप्त - द॒शः । \newline
15. अ॒ग्नि॒ष्टो॒मः प्रा॑जाप॒त्यासु॑ प्राजाप॒त्या स्व॑ग्निष्टो॒मो᳚ ऽग्निष्टो॒मः प्रा॑जाप॒त्यासू॑ पह॒व्य॑ उपह॒व्यः॑ प्राजाप॒त्या स्व॑ग्निष्टो॒मो᳚ ऽग्निष्टो॒मः प्रा॑जाप॒त्यासू॑ पह॒व्यः॑ । \newline
16. अ॒ग्नि॒ष्टो॒म इत्य॑ग्नि - स्तो॒मः । \newline
17. प्रा॒जा॒प॒त्यासू॑ पह॒व्य॑ उपह॒व्यः॑ प्राजाप॒त्यासु॑ प्राजाप॒त्यासू॑ पह॒व्य॑ उपह॒व मु॑पह॒व मु॑पह॒व्यः॑ प्राजाप॒त्यासु॑ प्राजाप॒त्यासू॑ पह॒व्य॑ उपह॒वम् । \newline
18. प्रा॒जा॒प॒त्यास्विति॑ प्राजा - प॒त्यासु॑ । \newline
19. उ॒प॒ह॒व्य॑ उपह॒व मु॑पह॒व मु॑पह॒व्य॑ उपह॒व्य॑ उपह॒व मे॒वैवोप॑ह॒व मु॑पह॒व्य॑ उपह॒व्य॑ उपह॒व मे॒व । \newline
20. उ॒प॒ह॒व्य॑ इत्यु॑प - ह॒व्यः॑ । \newline
21. उ॒प॒ह॒व मे॒वैवोप॑ह॒व मु॑पह॒व मे॒व ग॑च्छति गच्छ त्ये॒वोप॑ह॒व मु॑पह॒व मे॒व ग॑च्छति । \newline
22. उ॒प॒ह॒वमित्यु॑प - ह॒वम् । \newline
23. ए॒व ग॑च्छति गच्छ त्ये॒वैव ग॑च्छति त्रिण॒वौ त्रि॑ण॒वौ ग॑च्छ त्ये॒वैव ग॑च्छति त्रिण॒वौ । \newline
24. ग॒च्छ॒ति॒ त्रि॒ण॒वौ त्रि॑ण॒वौ ग॑च्छति गच्छति त्रिण॒वा व॑ग्निष्टो॒मा व॑ग्निष्टो॒मौ त्रि॑ण॒वौ ग॑च्छति गच्छति त्रिण॒वा व॑ग्निष्टो॒मौ । \newline
25. त्रि॒ण॒वा व॑ग्निष्टो॒मा व॑ग्निष्टो॒मौ त्रि॑ण॒वौ त्रि॑ण॒वा व॑ग्निष्टो॒मा व॒भितो॒ ऽभितो᳚ ऽग्निष्टो॒मौ त्रि॑ण॒वौ त्रि॑ण॒वा व॑ग्निष्टो॒मा व॒भितः॑ । \newline
26. त्रि॒ण॒वाविति॑ त्रि - न॒वौ । \newline
27. अ॒ग्नि॒ष्टो॒मा व॒भितो॒ ऽभितो᳚ ऽग्निष्टो॒मा व॑ग्निष्टो॒मा व॒भित॑ ऐ॒न्द्री ष्वै॒न्द्री ष्व॒भितो᳚ ऽग्निष्टो॒मा व॑ग्निष्टो॒मा व॒भित॑ ऐ॒न्द्रीषु॑ । \newline
28. अ॒ग्नि॒ष्टो॒मावित्य॑ग्नि - स्तो॒मौ । \newline
29. अ॒भित॑ ऐ॒न्द्री ष्वै॒न्द्री ष्व॒भितो॒ ऽभित॑ ऐ॒न्द्रीषु॒ विजि॑त्यै॒ विजि॑त्या ऐ॒न्द्री ष्व॒भितो॒ ऽभित॑ ऐ॒न्द्रीषु॒ विजि॑त्यै । \newline
30. ऐ॒न्द्रीषु॒ विजि॑त्यै॒ विजि॑त्या ऐ॒न्द्री ष्वै॒न्द्रीषु॒ विजि॑त्यै त्रयस्त्रिꣳ॒॒श स्त्र॑यस्त्रिꣳ॒॒शो विजि॑त्या ऐ॒न्द्री ष्वै॒न्द्रीषु॒ विजि॑त्यै त्रयस्त्रिꣳ॒॒शः । \newline
31. विजि॑त्यै त्रयस्त्रिꣳ॒॒श स्त्र॑यस्त्रिꣳ॒॒शो विजि॑त्यै॒ विजि॑त्यै त्रयस्त्रिꣳ॒॒श उ॒क्थ्य॑ उ॒क्थ्य॑ स्त्रयस्त्रिꣳ॒॒शो विजि॑त्यै॒ विजि॑त्यै त्रयस्त्रिꣳ॒॒श उ॒क्थ्यः॑ । \newline
32. विजि॑त्या॒ इति॒ वि - जि॒त्यै॒ । \newline
33. त्र॒य॒स्त्रिꣳ॒॒श उ॒क्थ्य॑ उ॒क्थ्य॑ स्त्रयस्त्रिꣳ॒॒श स्त्र॑यस्त्रिꣳ॒॒श उ॒क्थ्यो॑ वैश्वदे॒वीषु॑ वैश्वदे॒वीषू॒ क्थ्य॑ स्त्रयस्त्रिꣳ॒॒श स्त्र॑यस्त्रिꣳ॒॒श उ॒क्थ्यो॑ वैश्वदे॒वीषु॑ । \newline
34. त्र॒य॒स्त्रिꣳ॒॒श इति॑ त्रयः - त्रिꣳ॒॒शः । \newline
35. उ॒क्थ्यो॑ वैश्वदे॒वीषु॑ वैश्वदे॒वीषू॒क्थ्य॑ उ॒क्थ्यो॑ वैश्वदे॒वीषु॒ प्रति॑ष्ठित्यै॒ प्रति॑ष्ठित्यै वैश्वदे॒वीषू॒क्थ्य॑ उ॒क्थ्यो॑ वैश्वदे॒वीषु॒ प्रति॑ष्ठित्यै । \newline
36. वै॒श्व॒दे॒वीषु॒ प्रति॑ष्ठित्यै॒ प्रति॑ष्ठित्यै वैश्वदे॒वीषु॑ वैश्वदे॒वीषु॒ प्रति॑ष्ठित्यै विश्व॒जिद् वि॑श्व॒जित् प्रति॑ष्ठित्यै वैश्वदे॒वीषु॑ वैश्वदे॒वीषु॒ प्रति॑ष्ठित्यै विश्व॒जित् । \newline
37. वै॒श्व॒दे॒वीष्विति॑ वैश्व - दे॒वीषु॑ । \newline
38. प्रति॑ष्ठित्यै विश्व॒जिद् वि॑श्व॒जित् प्रति॑ष्ठित्यै॒ प्रति॑ष्ठित्यै विश्व॒जिथ् सर्व॑पृष्ठः॒ सर्व॑पृष्ठो विश्व॒जित् प्रति॑ष्ठित्यै॒ प्रति॑ष्ठित्यै विश्व॒जिथ् सर्व॑पृष्ठः । \newline
39. प्रति॑ष्ठित्या॒ इति॒ प्रति॑ - स्थि॒त्यै॒ । \newline
40. वि॒श्व॒जिथ् सर्व॑पृष्ठः॒ सर्व॑पृष्ठो विश्व॒जिद् वि॑श्व॒जिथ् सर्व॑पृष्ठो ऽतिरा॒त्रो॑ ऽतिरा॒त्रः सर्व॑पृष्ठो विश्व॒जिद् वि॑श्व॒जिथ् सर्व॑पृष्ठो ऽतिरा॒त्रः । \newline
41. वि॒श्व॒जिदिति॑ विश्व - जित् । \newline
42. सर्व॑पृष्ठो ऽतिरा॒त्रो॑ ऽतिरा॒त्रः सर्व॑पृष्ठः॒ सर्व॑पृष्ठो ऽतिरा॒त्रो भ॑वति भव त्यतिरा॒त्रः सर्व॑पृष्ठः॒ सर्व॑पृष्ठो ऽतिरा॒त्रो भ॑वति । \newline
43. सर्व॑पृष्ठ॒ इति॒ सर्व॑ - पृ॒ष्ठः॒ । \newline
44. अ॒ति॒रा॒त्रो भ॑वति भव त्यतिरा॒त्रो॑ ऽतिरा॒त्रो भ॑वति॒ सर्व॑स्य॒ सर्व॑स्य भव त्यतिरा॒त्रो॑ ऽतिरा॒त्रो भ॑वति॒ सर्व॑स्य । \newline
45. अ॒ति॒रा॒त्र इत्य॑ति - रा॒त्रः । \newline
46. भ॒व॒ति॒ सर्व॑स्य॒ सर्व॑स्य भवति भवति॒ सर्व॑स्या॒ भिजि॑त्या अ॒भिजि॑त्यै॒ सर्व॑स्य भवति भवति॒ सर्व॑स्या॒ भिजि॑त्यै । \newline
47. सर्व॑स्या॒ भिजि॑त्या अ॒भिजि॑त्यै॒ सर्व॑स्य॒ सर्व॑स्या॒ भिजि॑त्यै । \newline
48. अ॒भिजि॑त्या॒ इत्य॒भि - जि॒त्यै॒ । \newline
\pagebreak
\markright{ TS 7.2.6.1  \hfill https://www.vedavms.in \hfill}

\section{ TS 7.2.6.1 }

\textbf{TS 7.2.6.1 } \newline
\textbf{Samhita Paata} \newline

ऋ॒तवो॒ वै प्र॒जाका॑माः प्र॒जां नाऽवि॑न्दन्त॒ ते॑ऽकामयन्त प्र॒जाꣳ सृ॑जेमहि प्र॒जामव॑ रुन्धीमहि प्र॒जां ॅवि॑न्देमहि प्र॒जाव॑न्तः स्या॒मेति॒ त ए॒तमे॑कादशरा॒त्रम॑पश्य॒न् तमाऽह॑र॒न् तेना॑यजन्त॒ ततो॒ वै ते प्र॒जाम॑सृजन्त प्र॒जामवा॑रुन्धत प्र॒जाम॑विन्दन्त प्र॒जाव॑न्तोऽभव॒न्त ऋ॒तवो॑ऽभव॒न् तदा᳚र्त॒वाना॑-मार्तव॒त्व-मृ॑तू॒नां ॅवा ए॒ते पु॒त्रास्तस्मा॑ - [  ] \newline

\textbf{Pada Paata} \newline

ऋ॒तवः॑ । वै । प्र॒जाका॑मा॒ इति॑ प्र॒जा - का॒माः॒ । प्र॒जामिति॑ प्र - जाम् । न । अ॒वि॒न्द॒न्त॒ । ते । अ॒का॒म॒य॒न्त॒ । प्र॒जामिति॑ प्र - जाम् । सृ॒जे॒म॒हि॒ । प्र॒जामिति॑ प्र-जाम् । अवेति॑ । रु॒न्धी॒म॒हि॒ । प्र॒जामिति॑ प्र - जाम् । वि॒न्दे॒म॒हि॒ । प्र॒जाव॑न्त॒ इति॑ प्र॒जा - व॒न्तः॒ । स्या॒म॒ । इति॑ । ते । ए॒तम् । ए॒का॒द॒श॒रा॒त्रमित्ये॑कादश - रा॒त्रम् । अ॒प॒श्य॒न्न् । तम् । एति॑ । अ॒ह॒र॒न्न् । तेन॑ । अ॒य॒ज॒न्त॒ । ततः॑ । वै । ते । प्र॒जामिति॑ प्र - जाम् । अ॒सृ॒ज॒न्त॒ । प्र॒जामिति॑ प्र - जाम् । अवेति॑ । अ॒रु॒न्ध॒त॒ । प्र॒जामिति॑ प्र - जाम् । अ॒वि॒न्द॒न्त॒ । प्र॒जाव॑न्त॒ इति॑ प्र॒जा - व॒न्तः॒ । अ॒भ॒व॒न्न् । ते । ऋ॒तवः॑ । अ॒भ॒व॒न्न् । तत् । आ॒र्त॒वाना᳚म् । आ॒र्त॒व॒त्वमित्या᳚र्तव - त्वम् । ऋ॒तू॒नाम् । वै । ए॒ते । पु॒त्राः । तस्मा᳚त् ।  \newline


\textbf{Krama Paata} \newline

ऋ॒तवो॒ वै । वै प्र॒जाका॑माः । प्र॒जाका॑माः प्र॒जाम् । प्र॒जाका॑मा॒ इति॑ प्र॒जा - का॒माः॒ । प्र॒जाम् न । प्र॒जामिति॑ प्र - जाम् । नावि॑न्दन्त । अ॒वि॒न्द॒न्त॒ ते । ते॑ऽकामयन्त । अ॒का॒म॒य॒न्त॒ प्र॒जाम् । प्र॒जाꣳ सृ॑जेमहि । प्र॒जामिति॑ प्र - जाम् । सृ॒जे॒म॒हि॒ प्र॒जाम् । प्र॒जामव॑ । प्र॒जामिति॑ प्र - जाम् । अव॑ रुन्धीमहि । रु॒न्धी॒म॒हि॒ प्र॒जाम् । प्र॒जाम् ॅवि॑न्देमहि । प्र॒जामिति॑ प्र - जाम् । वि॒न्दे॒म॒हि॒ प्र॒जाव॑न्तः । प्र॒जाव॑न्तः स्याम । प्र॒जाव॑न्त॒ इति॑ प्र॒जा - व॒न्तः॒ । स्या॒मेति॑ । इति॒ ते । त ए॒तम् । ए॒तमे॑कादशरा॒त्रम् । ए॒का॒द॒श॒रा॒त्रम॑पश्यन्न् । ए॒का॒द॒श॒रा॒त्रमित्ये॑कादश - रा॒त्रम् । अ॒प॒श्य॒न् तम् । तमा । आऽह॑रन्न् । अ॒ह॒र॒न् तेन॑ । तेना॑यजन्त । अ॒य॒ज॒न्त॒ ततः॑ । ततो॒ वै । वै ते । ते प्र॒जाम् । प्र॒जाम॑सृजन्त । प्र॒जामिति॑ प्र - जाम् । अ॒सृ॒ज॒न्त॒ प्र॒जाम् । प्र॒जामव॑ । प्र॒जामिति॑ प्र - जाम् । अवा॑रुन्धत । अ॒रु॒न्ध॒त॒ प्र॒जाम् । प्र॒जाम॑विन्दन्त । प्र॒जामिति॑ प्र - जाम् । अ॒वि॒न्द॒न्त॒ प्र॒जाव॑न्तः । प्र॒जाव॑न्तोऽभवन्न्न् । प्र॒जाव॑न्त॒ इति॑ प्र॒जा - व॒न्तः॒ । अ॒भ॒व॒न् ते । त ऋ॒तवः॑ । ऋ॒तवो॑ऽभवन्न् । अ॒भ॒व॒न् तत् । तदा᳚र्त॒वाना᳚म् । आ॒र्त॒वाना॑मार्तव॒त्वम् । आ॒र्त॒व॒त्वमृ॑तू॒नाम् । आ॒र्त॒व॒त्वमित्या᳚र्तव - त्वम् । ऋ॒तू॒नाम् ॅवै । वा ए॒ते । ए॒ते पु॒त्राः । पु॒त्रास्तस्मा᳚त् । तस्मा॑दार्त॒वाः \newline

\textbf{Jatai Paata} \newline

1. ऋ॒तवो॒ वै वा ऋ॒तव॑ ऋ॒तवो॒ वै । \newline
2. वै प्र॒जाका॑माः प्र॒जाका॑मा॒ वै वै प्र॒जाका॑माः । \newline
3. प्र॒जाका॑माः प्र॒जाम् प्र॒जाम् प्र॒जाका॑माः प्र॒जाका॑माः प्र॒जाम् । \newline
4. प्र॒जाका॑मा॒ इति॑ प्र॒जा - का॒माः॒ । \newline
5. प्र॒जान् न न प्र॒जाम् प्र॒जान् न । \newline
6. प्र॒जामिति॑ प्र - जाम् । \newline
7. नावि॑न्दन्ता विन्दन्त॒ न नावि॑न्दन्त । \newline
8. अ॒वि॒न्द॒न्त॒ ते ते॑ ऽविन्दन्ता विन्दन्त॒ ते । \newline
9. ते॑ ऽकामयन्ता कामयन्त॒ ते ते॑ ऽकामयन्त । \newline
10. अ॒का॒म॒य॒न्त॒ प्र॒जाम् प्र॒जा म॑कामयन्ता कामयन्त प्र॒जाम् । \newline
11. प्र॒जाꣳ सृ॑जेमहि सृजेमहि प्र॒जाम् प्र॒जाꣳ सृ॑जेमहि । \newline
12. प्र॒जामिति॑ प्र - जाम् । \newline
13. सृ॒जे॒म॒हि॒ प्र॒जाम् प्र॒जाꣳ सृ॑जेमहि सृजेमहि प्र॒जाम् । \newline
14. प्र॒जा मवाव॑ प्र॒जाम् प्र॒जा मव॑ । \newline
15. प्र॒जामिति॑ प्र - जाम् । \newline
16. अव॑ रुन्धीमहि रुन्धीम॒ ह्यवाव॑ रुन्धीमहि । \newline
17. रु॒न्धी॒म॒हि॒ प्र॒जाम् प्र॒जाꣳ रु॑न्धीमहि रुन्धीमहि प्र॒जाम् । \newline
18. प्र॒जां ॅवि॑न्देमहि विन्देमहि प्र॒जाम् प्र॒जां ॅवि॑न्देमहि । \newline
19. प्र॒जामिति॑ प्र - जाम् । \newline
20. वि॒न्दे॒म॒हि॒ प्र॒जाव॑न्तः प्र॒जाव॑न्तो विन्देमहि विन्देमहि प्र॒जाव॑न्तः । \newline
21. प्र॒जाव॑न्तः स्याम स्याम प्र॒जाव॑न्तः प्र॒जाव॑न्तः स्याम । \newline
22. प्र॒जाव॑न्त॒ इति॑ प्र॒जा - व॒न्तः॒ । \newline
23. स्या॒मेतीति॑ स्याम स्या॒मेति॑ । \newline
24. इति॒ ते त इतीति॒ ते । \newline
25. त ए॒त मे॒तम् ते त ए॒तम् । \newline
26. ए॒त मे॑कादशरा॒त्र मे॑कादशरा॒त्र मे॒त मे॒त मे॑कादशरा॒त्रम् । \newline
27. ए॒का॒द॒श॒रा॒त्र म॑पश्यन् नपश्यन् नेकादशरा॒त्र मे॑कादशरा॒त्र म॑पश्यन्न् । \newline
28. ए॒का॒द॒श॒रा॒त्रमित्ये॑कादश - रा॒त्रम् । \newline
29. अ॒प॒श्य॒न् तम् त म॑पश्यन् नपश्य॒न् तम् । \newline
30. त मा तम् त मा । \newline
31. आ ऽह॑रन् नहर॒न् ना ऽह॑रन्न् । \newline
32. अ॒ह॒र॒न् तेन॒ तेना॑ हरन् नहर॒न् तेन॑ । \newline
33. तेना॑ यजन्ता यजन्त॒ तेन॒ तेना॑ यजन्त । \newline
34. अ॒य॒ज॒न्त॒ तत॒ स्ततो॑ ऽयजन्ता यजन्त॒ ततः॑ । \newline
35. ततो॒ वै वै तत॒ स्ततो॒ वै । \newline
36. वै ते ते वै वै ते । \newline
37. ते प्र॒जाम् प्र॒जाम् ते ते प्र॒जाम् । \newline
38. प्र॒जा म॑सृजन्ता सृजन्त प्र॒जाम् प्र॒जा म॑सृजन्त । \newline
39. प्र॒जामिति॑ प्र - जाम् । \newline
40. अ॒सृ॒ज॒न्त॒ प्र॒जाम् प्र॒जा म॑सृजन्ता सृजन्त प्र॒जाम् । \newline
41. प्र॒जा मवाव॑ प्र॒जाम् प्र॒जा मव॑ । \newline
42. प्र॒जामिति॑ प्र - जाम् । \newline
43. अवा॑ रुन्धता रुन्ध॒ता वावा॑ रुन्धत । \newline
44. अ॒रु॒न्ध॒त॒ प्र॒जाम् प्र॒जा म॑रुन्धता रुन्धत प्र॒जाम् । \newline
45. प्र॒जा म॑विन्दन्ता विन्दन्त प्र॒जाम् प्र॒जा म॑विन्दन्त । \newline
46. प्र॒जामिति॑ प्र - जाम् । \newline
47. अ॒वि॒न्द॒न्त॒ प्र॒जाव॑न्तः प्र॒जाव॑न्तो ऽविन्दन्ता विन्दन्त प्र॒जाव॑न्तः । \newline
48. प्र॒जाव॑न्तो ऽभवन् नभवन् प्र॒जाव॑न्तः प्र॒जाव॑न्तो ऽभवन्न् । \newline
49. प्र॒जाव॑न्त॒ इति॑ प्र॒जा - व॒न्तः॒ । \newline
50. अ॒भ॒व॒न् ते ते॑ ऽभवन् नभव॒न् ते । \newline
51. त ऋ॒तव॑ ऋ॒तव॒ स्ते त ऋ॒तवः॑ । \newline
52. ऋ॒तवो॑ ऽभवन् नभवन् नृ॒तव॑ ऋ॒तवो॑ ऽभवन्न् । \newline
53. अ॒भ॒व॒न् तत् तद॑भवन् नभव॒न् तत् । \newline
54. तदा᳚र्त॒वाना॑ मार्त॒वाना॒म् तत् तदा᳚र्त॒वाना᳚म् । \newline
55. आ॒र्त॒वाना॑ मार्तव॒त्व मा᳚र्तव॒त्व मा᳚र्त॒वाना॑ मार्त॒वाना॑ मार्तव॒त्वम् । \newline
56. आ॒र्त॒व॒त्व मृ॑तू॒ना मृ॑तू॒ना मा᳚र्तव॒त्व मा᳚र्तव॒त्व मृ॑तू॒नाम् । \newline
57. आ॒र्त॒व॒त्वमित्या᳚र्तव - त्वम् । \newline
58. ऋ॒तू॒नां ॅवै वा ऋ॑तू॒ना मृ॑तू॒नां ॅवै । \newline
59. वा ए॒त ए॒ते वै वा ए॒ते । \newline
60. ए॒ते पु॒त्राः पु॒त्रा ए॒त ए॒ते पु॒त्राः । \newline
61. पु॒त्रा स्तस्मा॒त् तस्मा᳚त् पु॒त्राः पु॒त्रा स्तस्मा᳚त् । \newline
62. तस्मा॑ दार्त॒वा आ᳚र्त॒वा स्तस्मा॒त् तस्मा॑ दार्त॒वाः । \newline

\textbf{Ghana Paata } \newline

1. ऋ॒तवो॒ वै वा ऋ॒तव॑ ऋ॒तवो॒ वै प्र॒जाका॑माः प्र॒जाका॑मा॒ वा ऋ॒तव॑ ऋ॒तवो॒ वै प्र॒जाका॑माः । \newline
2. वै प्र॒जाका॑माः प्र॒जाका॑मा॒ वै वै प्र॒जाका॑माः प्र॒जाम् प्र॒जाम् प्र॒जाका॑मा॒ वै वै प्र॒जाका॑माः प्र॒जाम् । \newline
3. प्र॒जाका॑माः प्र॒जाम् प्र॒जाम् प्र॒जाका॑माः प्र॒जाका॑माः प्र॒जान् न न प्र॒जाम् प्र॒जाका॑माः प्र॒जाका॑माः प्र॒जान् न । \newline
4. प्र॒जाका॑मा॒ इति॑ प्र॒जा - का॒माः॒ । \newline
5. प्र॒जान् न न प्र॒जाम् प्र॒जाम् नावि॑न्दन्ता विन्दन्त॒ न प्र॒जाम् प्र॒जाम् नावि॑न्दन्त । \newline
6. प्र॒जामिति॑ प्र - जाम् । \newline
7. नावि॑न्दन्ता विन्दन्त॒ न नावि॑न्दन्त॒ ते ते॑ ऽविन्दन्त॒ न नावि॑न्दन्त॒ ते । \newline
8. अ॒वि॒न्द॒न्त॒ ते ते॑ ऽविन्दन्ता विन्दन्त॒ ते॑ ऽकामयन्ता कामयन्त॒ ते॑ ऽविन्दन्ता विन्दन्त॒ ते॑ ऽकामयन्त । \newline
9. ते॑ ऽकामयन्ता कामयन्त॒ ते ते॑ ऽकामयन्त प्र॒जाम् प्र॒जा म॑कामयन्त॒ ते ते॑ ऽकामयन्त प्र॒जाम् । \newline
10. अ॒का॒म॒य॒न्त॒ प्र॒जाम् प्र॒जा म॑कामयन्ता कामयन्त प्र॒जाꣳ सृ॑जेमहि सृजेमहि प्र॒जा म॑कामयन्ता कामयन्त प्र॒जाꣳ सृ॑जेमहि । \newline
11. प्र॒जाꣳ सृ॑जेमहि सृजेमहि प्र॒जाम् प्र॒जाꣳ सृ॑जेमहि प्र॒जाम् प्र॒जाꣳ सृ॑जेमहि प्र॒जाम् प्र॒जाꣳ सृ॑जेमहि प्र॒जाम् । \newline
12. प्र॒जामिति॑ प्र - जाम् । \newline
13. सृ॒जे॒म॒हि॒ प्र॒जाम् प्र॒जाꣳ सृ॑जेमहि सृजेमहि प्र॒जा मवाव॑ प्र॒जाꣳ सृ॑जेमहि सृजेमहि प्र॒जा मव॑ । \newline
14. प्र॒जा मवाव॑ प्र॒जाम् प्र॒जा मव॑ रुन्धीमहि रुन्धीम॒ ह्यव॑ प्र॒जाम् प्र॒जा मव॑ रुन्धीमहि । \newline
15. प्र॒जामिति॑ प्र - जाम् । \newline
16. अव॑ रुन्धीमहि रुन्धीम॒ ह्यवाव॑ रुन्धीमहि प्र॒जाम् प्र॒जाꣳ रु॑न्धीम॒ ह्यवाव॑ रुन्धीमहि प्र॒जाम् । \newline
17. रु॒न्धी॒म॒हि॒ प्र॒जाम् प्र॒जाꣳ रु॑न्धीमहि रुन्धीमहि प्र॒जां ॅवि॑न्देमहि विन्देमहि प्र॒जाꣳ रु॑न्धीमहि रुन्धीमहि प्र॒जां ॅवि॑न्देमहि । \newline
18. प्र॒जां ॅवि॑न्देमहि विन्देमहि प्र॒जाम् प्र॒जां ॅवि॑न्देमहि प्र॒जाव॑न्तः प्र॒जाव॑न्तो विन्देमहि प्र॒जाम् प्र॒जां ॅवि॑न्देमहि प्र॒जाव॑न्तः । \newline
19. प्र॒जामिति॑ प्र - जाम् । \newline
20. वि॒न्दे॒म॒हि॒ प्र॒जाव॑न्तः प्र॒जाव॑न्तो विन्देमहि विन्देमहि प्र॒जाव॑न्तः स्याम स्याम प्र॒जाव॑न्तो विन्देमहि विन्देमहि प्र॒जाव॑न्तः स्याम । \newline
21. प्र॒जाव॑न्तः स्याम स्याम प्र॒जाव॑न्तः प्र॒जाव॑न्तः स्या॒मेतीति॑ स्याम प्र॒जाव॑न्तः प्र॒जाव॑न्तः स्या॒मेति॑ । \newline
22. प्र॒जाव॑न्त॒ इति॑ प्र॒जा - व॒न्तः॒ । \newline
23. स्या॒मेतीति॑ स्याम स्या॒मेति॒ ते त इति॑ स्याम स्या॒मेति॒ ते । \newline
24. इति॒ ते त इतीति॒ त ए॒त मे॒तम् त इतीति॒ त ए॒तम् । \newline
25. त ए॒त मे॒तम् ते त ए॒त मे॑कादशरा॒त्र मे॑कादशरा॒त्र मे॒तम् ते त ए॒त मे॑कादशरा॒त्रम् । \newline
26. ए॒त मे॑कादशरा॒त्र मे॑कादशरा॒त्र मे॒त मे॒त मे॑कादशरा॒त्र म॑पश्यन् नपश्यन् नेकादशरा॒त्र मे॒त मे॒त मे॑कादशरा॒त्र म॑पश्यन्न् । \newline
27. ए॒का॒द॒श॒रा॒त्र म॑पश्यन् नपश्यन् नेकादशरा॒त्र मे॑कादशरा॒त्र म॑पश्य॒न् तम् त म॑पश्यन् नेकादशरा॒त्र मे॑कादशरा॒त्र म॑पश्य॒न् तम् । \newline
28. ए॒का॒द॒श॒रा॒त्रमित्ये॑कादश - रा॒त्रम् । \newline
29. अ॒प॒श्य॒न् तम् त म॑पश्यन् नपश्य॒न् त मा त म॑पश्यन् नपश्य॒न् त मा । \newline
30. त मा तम् त मा ऽह॑रन् नहर॒न् ना तम् त मा ऽह॑रन्न् । \newline
31. आ ऽह॑रन् नहर॒न् ना ऽह॑र॒न् तेन॒ तेना॑ हर॒न् ना ऽह॑र॒न् तेन॑ । \newline
32. अ॒ह॒र॒न् तेन॒ तेना॑ हरन् नहर॒न् तेना॑ यजन्ता यजन्त॒ तेना॑ हरन् नहर॒न् तेना॑ यजन्त । \newline
33. तेना॑ यजन्ता यजन्त॒ तेन॒ तेना॑ यजन्त॒ तत॒ स्ततो॑ ऽयजन्त॒ तेन॒ तेना॑ यजन्त॒ ततः॑ । \newline
34. अ॒य॒ज॒न्त॒ तत॒ स्ततो॑ ऽयजन्ता यजन्त॒ ततो॒ वै वै ततो॑ ऽयजन्ता यजन्त॒ ततो॒ वै । \newline
35. ततो॒ वै वै तत॒ स्ततो॒ वै ते ते वै तत॒ स्ततो॒ वै ते । \newline
36. वै ते ते वै वै ते प्र॒जाम् प्र॒जाम् ते वै वै ते प्र॒जाम् । \newline
37. ते प्र॒जाम् प्र॒जाम् ते ते प्र॒जा म॑सृजन्ता सृजन्त प्र॒जाम् ते ते प्र॒जा म॑सृजन्त । \newline
38. प्र॒जा म॑सृजन्ता सृजन्त प्र॒जाम् प्र॒जा म॑सृजन्त प्र॒जाम् प्र॒जा म॑सृजन्त प्र॒जाम् प्र॒जा म॑सृजन्त प्र॒जाम् । \newline
39. प्र॒जामिति॑ प्र - जाम् । \newline
40. अ॒सृ॒ज॒न्त॒ प्र॒जाम् प्र॒जा म॑सृजन्ता सृजन्त प्र॒जा मवाव॑ प्र॒जा म॑सृजन्ता सृजन्त प्र॒जा मव॑ । \newline
41. प्र॒जा मवाव॑ प्र॒जाम् प्र॒जा मवा॑ रुन्धता रुन्ध॒ताव॑ प्र॒जाम् प्र॒जा मवा॑ रुन्धत । \newline
42. प्र॒जामिति॑ प्र - जाम् । \newline
43. अवा॑ रुन्धता रुन्ध॒ता वावा॑ रुन्धत प्र॒जाम् प्र॒जा म॑रुन्ध॒ता वावा॑ रुन्धत प्र॒जाम् । \newline
44. अ॒रु॒न्ध॒त॒ प्र॒जाम् प्र॒जा म॑रुन्धता रुन्धत प्र॒जा म॑विन्दन्ता विन्दन्त प्र॒जा म॑रुन्धता रुन्धत प्र॒जा म॑विन्दन्त । \newline
45. प्र॒जा म॑विन्दन्ता विन्दन्त प्र॒जाम् प्र॒जा म॑विन्दन्त प्र॒जाव॑न्तः प्र॒जाव॑न्तो ऽविन्दन्त प्र॒जाम् प्र॒जा म॑विन्दन्त प्र॒जाव॑न्तः । \newline
46. प्र॒जामिति॑ प्र - जाम् । \newline
47. अ॒वि॒न्द॒न्त॒ प्र॒जाव॑न्तः प्र॒जाव॑न्तो ऽविन्दन्ता विन्दन्त प्र॒जाव॑न्तो ऽभवन् नभवन् प्र॒जाव॑न्तो ऽविन्दन्ता विन्दन्त प्र॒जाव॑न्तो ऽभवन्न् । \newline
48. प्र॒जाव॑न्तो ऽभवन् नभवन् प्र॒जाव॑न्तः प्र॒जाव॑न्तो ऽभव॒न् ते ते॑ ऽभवन् प्र॒जाव॑न्तः प्र॒जाव॑न्तो ऽभव॒न् ते । \newline
49. प्र॒जाव॑न्त॒ इति॑ प्र॒जा - व॒न्तः॒ । \newline
50. अ॒भ॒व॒न् ते ते॑ ऽभवन् नभव॒न् त ऋ॒तव॑ ऋ॒तव॒ स्ते॑ ऽभवन् नभव॒न् त ऋ॒तवः॑ । \newline
51. त ऋ॒तव॑ ऋ॒तव॒ स्ते त ऋ॒तवो॑ ऽभवन् नभवन् नृ॒तव॒ स्ते त ऋ॒तवो॑ ऽभवन्न् । \newline
52. ऋ॒तवो॑ ऽभवन् नभवन् नृ॒तव॑ ऋ॒तवो॑ ऽभव॒न् तत् तद॑भवन् नृ॒तव॑ ऋ॒तवो॑ ऽभव॒न् तत् । \newline
53. अ॒भ॒व॒न् तत् तद॑भवन् नभव॒न् तदा᳚र्त॒वाना॑ मार्त॒वाना॒म् तद॑भवन् नभव॒न् तदा᳚र्त॒वाना᳚म् । \newline
54. तदा᳚र्त॒वाना॑ मार्त॒वाना॒म् तत् तदा᳚र्त॒वाना॑ मार्तव॒त्व मा᳚र्तव॒त्व मा᳚र्त॒वाना॒म् तत् तदा᳚र्त॒वाना॑ मार्तव॒त्वम् । \newline
55. आ॒र्त॒वाना॑ मार्तव॒त्व मा᳚र्तव॒त्व मा᳚र्त॒वाना॑ मार्त॒वाना॑ मार्तव॒त्व मृ॑तू॒ना मृ॑तू॒ना मा᳚र्तव॒त्व मा᳚र्त॒वाना॑ मार्त॒वाना॑ मार्तव॒त्व मृ॑तू॒नाम् । \newline
56. आ॒र्त॒व॒त्व मृ॑तू॒ना मृ॑तू॒ना मा᳚र्तव॒त्व मा᳚र्तव॒त्व मृ॑तू॒नां ॅवै वा ऋ॑तू॒ना मा᳚र्तव॒त्व मा᳚र्तव॒त्व मृ॑तू॒नां ॅवै । \newline
57. आ॒र्त॒व॒त्वमित्या᳚र्तव - त्वम् । \newline
58. ऋ॒तू॒नां ॅवै वा ऋ॑तू॒ना मृ॑तू॒नां ॅवा ए॒त ए॒ते वा ऋ॑तू॒ना मृ॑तू॒नां ॅवा ए॒ते । \newline
59. वा ए॒त ए॒ते वै वा ए॒ते पु॒त्राः पु॒त्रा ए॒ते वै वा ए॒ते पु॒त्राः । \newline
60. ए॒ते पु॒त्राः पु॒त्रा ए॒त ए॒ते पु॒त्रा स्तस्मा॒त् तस्मा᳚त् पु॒त्रा ए॒त ए॒ते पु॒त्रा स्तस्मा᳚त् । \newline
61. पु॒त्रा स्तस्मा॒त् तस्मा᳚त् पु॒त्राः पु॒त्रा स्तस्मा॑ दार्त॒वा आ᳚र्त॒वा स्तस्मा᳚त् पु॒त्राः पु॒त्रा स्तस्मा॑ दार्त॒वाः । \newline
62. तस्मा॑ दार्त॒वा आ᳚र्त॒वा स्तस्मा॒त् तस्मा॑ दार्त॒वा उ॑च्यन्त उच्यन्त आर्त॒वा स्तस्मा॒त् तस्मा॑ दार्त॒वा उ॑च्यन्ते । \newline
\pagebreak
\markright{ TS 7.2.6.2  \hfill https://www.vedavms.in \hfill}

\section{ TS 7.2.6.2 }

\textbf{TS 7.2.6.2 } \newline
\textbf{Samhita Paata} \newline

-दार्त॒वा उ॑च्यन्ते॒ य ए॒वं ॅवि॒द्वाꣳस॑ एकादशरा॒त्रमास॑ते प्र॒जामे॒व सृ॑जन्ते प्र॒जामव॑ रुन्धते प्र॒जां ॅवि॑न्दन्ते प्र॒जाव॑न्तो भवन्ति॒ ज्योति॑रतिरा॒त्रो भ॑वति॒ ज्योति॑रे॒व पु॒रस्ता᳚द्-दधते सुव॒र्गस्य॑ लो॒कस्या-नु॑ख्यात्यै॒ पृष्ठ्यः॑ षड॒हो भ॑वति॒ षड् वा ऋ॒तवः॒ षट् पृ॒ष्ठानि॑ पृ॒ष्ठैरे॒वर्तून॒-न्वारो॑हन्त्यृ॒तुभिः॑ संॅवथ्स॒रं ते सं॑ॅवथ्स॒र ए॒व प्रति॑ तिष्ठन्ति चतुर्विꣳ॒॒शो भ॑वति॒ चतु॑र्विꣳशत्यक्षरा गाय॒त्री - [  ] \newline

\textbf{Pada Paata} \newline

आ॒र्त॒वाः । उ॒च्य॒न्ते॒ । ये । ए॒वम् । वि॒द्वाꣳसः॑ । ए॒का॒द॒श॒रा॒त्रमित्ये॑कादश - रा॒त्रम् । आस॑ते । प्र॒जामिति॑ प्र - जाम् । ए॒व । सृ॒ज॒न्ते॒ । प्र॒जामिति॑ प्र - जाम् । अवेति॑ । रु॒न्ध॒ते॒ । प्र॒जामिति॑ प्र - जाम् । वि॒न्द॒न्ते॒ । प्र॒जाव॑न्त॒ इति॑ प्र॒जा - व॒न्तः॒ । भ॒व॒न्ति॒ । ज्योतिः॑ । अ॒ति॒रा॒त्र इत्य॑ति - रा॒त्रः । भ॒व॒ति॒ । ज्योतिः॑ । ए॒व । पु॒रस्ता᳚त् । द॒ध॒ते॒ । सु॒व॒र्गस्येति॑ सुवः - गस्य॑ । लो॒कस्य॑ । अनु॑ख्यात्या॒ इत्यनु॑ - ख्या॒त्यै॒ । पृष्ठ्यः॑ । ष॒ड॒ह इति॑ षट् - अ॒हः । भ॒व॒ति॒ । षट् । वै । ऋ॒तवः॑ । षट् । पृ॒ष्ठानि॑ । पृ॒ष्ठैः । ए॒व । ऋ॒तून् । अ॒न्वारो॑ह॒न्तीत्य॑नु - आरो॑हन्ति । ऋ॒तुभि॒रित्यृ॒तु - भिः॒ । सं॒ॅव॒थ्स॒रमिति॑ सं - व॒थ्स॒रम् । ते । सं॒ॅव॒थ्स॒र इति॑ सं - व॒थ्स॒रे । ए॒व । प्रतीति॑ । ति॒ष्ठ॒न्ति॒ । च॒तु॒र्विꣳ॒॒श इति॑ चतुः-विꣳ॒॒शः । भ॒व॒ति॒ । चतु॑र्विꣳशत्यक्ष॒रेति॒ चतु॑विꣳशति - अ॒क्ष॒रा॒ । गा॒य॒त्री ।  \newline


\textbf{Krama Paata} \newline

आ॒र्त॒वा उ॑च्यन्ते । उ॒च्य॒न्ते॒ ये । य ए॒वम् । ए॒वम् ॅवि॒द्वाꣳसः॑ । वि॒द्वाꣳस॑ एकादशरा॒त्रम् । ए॒का॒द॒श॒रा॒त्रमास॑ते । ए॒का॒द॒श॒रा॒त्रमित्ये॑कादश - रा॒त्रम् । आस॑ते प्र॒जाम् । प्र॒जामे॒व । प्र॒जामिति॑ प्र - जाम् । ए॒व सृ॑जन्ते । सृ॒ज॒न्ते॒ प्र॒जाम् । प्र॒जामव॑ । प्र॒जामिति॑ प्र - जाम् । अव॑ रुन्धते । रु॒न्ध॒ते॒ प्र॒जाम् । प्र॒जाम् ॅवि॑न्दन्ते । प्र॒जामिति॑ प्र - जाम् । वि॒न्द॒न्ते॒ प्र॒जाव॑न्तः । प्र॒जाव॑न्तो भवन्ति । प्र॒जाव॑न्त॒ इति॑ प्र॒जा - व॒न्तः॒ । भ॒व॒न्ति॒ ज्योतिः॑ । ज्योति॑रतिरा॒त्रः । अ॒ति॒रा॒त्रो भ॑वति । अ॒ति॒रा॒त्र इत्य॑ति - रा॒त्रः । भ॒व॒ति॒ ज्योतिः॑ । ज्योति॑रे॒व । ए॒व पु॒रस्ता᳚त् । पु॒रस्ता᳚द् दधते । द॒ध॒ते॒ सु॒व॒र्गस्य॑ । सु॒व॒र्गस्य॑ लो॒कस्य॑ । सु॒व॒र्गस्येति॑ सुवः - गस्य॑ । लो॒कस्यानु॑ख्यात्यै । अनु॑ख्यात्यै॒ पृष्ठ्‍यः॑ । अनु॑ख्यात्या॒ इत्यनु॑ - ख्या॒त्यै॒ । पृष्ठ्‍यः॑ षड॒हः । ष॒ड॒हो भ॑वति । ष॒ड॒ह इति॑ षट् - अ॒हः । भ॒व॒ति॒ षट् । षड् वै । वा ऋ॒तवः॑ । ऋ॒तवः॒ षट् । षट् पृ॒ष्ठानि॑ । पृ॒ष्ठानि॑ पृ॒ष्ठैः । पृ॒ष्ठैरे॒व । ए॒वर्तून् । ऋ॒तून॒न्वारो॑हन्ति । अ॒न्वारो॑हन्त्यृ॒तुभिः॑ । अ॒न्वारो॑ह॒न्तीत्य॑नु - आरो॑हन्ति । ऋ॒तुभिः॑ सम्ॅवथ्स॒रम् । ऋ॒तुभि॒रित्यृ॒तु - भिः॒ । स॒म्ॅव॒थ्स॒रम् ते । स॒म्ॅव॒थ्स॒रमिति॑ सम् - व॒थ्स॒रम् । ते स॑म्ॅवथ्स॒रे । स॒म्ॅव॒थ्स॒र ए॒व । स॒म्ॅव॒थ्स॒र इति॑ सम् - व॒थ्स॒रे । ए॒व प्रति॑ । प्रति॑ तिष्ठन्ति । ति॒ष्ठ॒न्ति॒ च॒तु॒र्विꣳ॒॒शः । च॒तु॒र्विꣳ॒॒शो भ॑वति । च॒तु॒र्विꣳ॒॒श इति॑ चतुः - विꣳ॒॒शः । भ॒व॒ति॒ चतु॑र्विꣳशत्यक्षरा । चतु॑र्विꣳशत्यक्षरा गाय॒त्री ( ) । चतु॑र्विꣳशत्यक्ष॒रेति॒ चतु॑र्विꣳशति - अ॒क्ष॒रा॒ । गा॒य॒त्री गा॑य॒त्रम् \newline

\textbf{Jatai Paata} \newline

1. आ॒र्त॒वा उ॑च्यन्त उच्यन्त आर्त॒वा आ᳚र्त॒वा उ॑च्यन्ते । \newline
2. उ॒च्य॒न्ते॒ ये य उ॑च्यन्त उच्यन्ते॒ ये । \newline
3. य ए॒व मे॒वं ॅये य ए॒वम् । \newline
4. ए॒वं ॅवि॒द्वाꣳसो॑ वि॒द्वाꣳस॑ ए॒व मे॒वं ॅवि॒द्वाꣳसः॑ । \newline
5. वि॒द्वाꣳस॑ एकादशरा॒त्र मे॑कादशरा॒त्रं ॅवि॒द्वाꣳसो॑ वि॒द्वाꣳस॑ एकादशरा॒त्रम् । \newline
6. ए॒का॒द॒श॒रा॒त्र मास॑त॒ आस॑त एकादशरा॒त्र मे॑कादशरा॒त्र मास॑ते । \newline
7. ए॒का॒द॒श॒रा॒त्रमित्ये॑कादश - रा॒त्रम् । \newline
8. आस॑ते प्र॒जाम् प्र॒जा मास॑त॒ आस॑ते प्र॒जाम् । \newline
9. प्र॒जा मे॒वैव प्र॒जाम् प्र॒जा मे॒व । \newline
10. प्र॒जामिति॑ प्र - जाम् । \newline
11. ए॒व सृ॑जन्ते सृजन्त ए॒वैव सृ॑जन्ते । \newline
12. सृ॒ज॒न्ते॒ प्र॒जाम् प्र॒जाꣳ सृ॑जन्ते सृजन्ते प्र॒जाम् । \newline
13. प्र॒जा मवाव॑ प्र॒जाम् प्र॒जा मव॑ । \newline
14. प्र॒जामिति॑ प्र - जाम् । \newline
15. अव॑ रुन्धते रुन्ध॒ते ऽवाव॑ रुन्धते । \newline
16. रु॒न्ध॒ते॒ प्र॒जाम् प्र॒जाꣳ रु॑न्धते रुन्धते प्र॒जाम् । \newline
17. प्र॒जां ॅवि॑न्दन्ते विन्दन्ते प्र॒जाम् प्र॒जां ॅवि॑न्दन्ते । \newline
18. प्र॒जामिति॑ प्र - जाम् । \newline
19. वि॒न्द॒न्ते॒ प्र॒जाव॑न्तः प्र॒जाव॑न्तो विन्दन्ते विन्दन्ते प्र॒जाव॑न्तः । \newline
20. प्र॒जाव॑न्तो भवन्ति भवन्ति प्र॒जाव॑न्तः प्र॒जाव॑न्तो भवन्ति । \newline
21. प्र॒जाव॑न्त॒ इति॑ प्र॒जा - व॒न्तः॒ । \newline
22. भ॒व॒न्ति॒ ज्योति॒र् ज्योति॑र् भवन्ति भवन्ति॒ ज्योतिः॑ । \newline
23. ज्योति॑ रतिरा॒त्रो॑ ऽतिरा॒त्रो ज्योति॒र् ज्योति॑ रतिरा॒त्रः । \newline
24. अ॒ति॒रा॒त्रो भ॑वति भव त्यतिरा॒त्रो॑ ऽतिरा॒त्रो भ॑वति । \newline
25. अ॒ति॒रा॒त्र इत्य॑ति - रा॒त्रः । \newline
26. भ॒व॒ति॒ ज्योति॒र् ज्योति॑र् भवति भवति॒ ज्योतिः॑ । \newline
27. ज्योति॑ रे॒वैव ज्योति॒र् ज्योति॑ रे॒व । \newline
28. ए॒व पु॒रस्ता᳚त् पु॒रस्ता॑ दे॒वैव पु॒रस्ता᳚त् । \newline
29. पु॒रस्ता᳚द् दधते दधते पु॒रस्ता᳚त् पु॒रस्ता᳚द् दधते । \newline
30. द॒ध॒ते॒ सु॒व॒र्गस्य॑ सुव॒र्गस्य॑ दधते दधते सुव॒र्गस्य॑ । \newline
31. सु॒व॒र्गस्य॑ लो॒कस्य॑ लो॒कस्य॑ सुव॒र्गस्य॑ सुव॒र्गस्य॑ लो॒कस्य॑ । \newline
32. सु॒व॒र्गस्येति॑ सुवः - गस्य॑ । \newline
33. लो॒कस्या नु॑ख्यात्या॒ अनु॑ख्यात्यै लो॒कस्य॑ लो॒कस्या नु॑ख्यात्यै । \newline
34. अनु॑ख्यात्यै॒ पृष्ठ्यः॒ पृष्ठ्यो ऽनु॑ख्यात्या॒ अनु॑ख्यात्यै॒ पृष्ठ्यः॑ । \newline
35. अनु॑ख्यात्या॒ इत्यनु॑ - ख्या॒त्यै॒ । \newline
36. पृष्ठ्य॑ ष्षड॒ह ष्ष॑ड॒हः पृष्ठ्यः॒ पृष्ठ्य॑ ष्षड॒हः । \newline
37. ष॒ड॒हो भ॑वति भवति षड॒ह ष्ष॑ड॒हो भ॑वति । \newline
38. ष॒ड॒ह इति॑ षट् - अ॒हः । \newline
39. भ॒व॒ति॒ षट् थ्षड् भ॑वति भवति॒ षट् । \newline
40. षड् वै वै षट् थ्षड् वै । \newline
41. वा ऋ॒तव॑ ऋ॒तवो॒ वै वा ऋ॒तवः॑ । \newline
42. ऋ॒तव॒ ष्षट् थ्षडृ॒तव॑ ऋ॒तव॒ ष्षट् । \newline
43. षट् पृ॒ष्ठानि॑ पृ॒ष्ठानि॒ षट् थ्षट् पृ॒ष्ठानि॑ । \newline
44. पृ॒ष्ठानि॑ पृ॒ष्ठैः पृ॒ष्ठैः पृ॒ष्ठानि॑ पृ॒ष्ठानि॑ पृ॒ष्ठैः । \newline
45. पृ॒ष्ठै रे॒वैव पृ॒ष्ठैः पृ॒ष्ठै रे॒व । \newline
46. ए॒व र्‌तू नृ॒तू ने॒वैव र्‌तून् । \newline
47. ऋ॒तू न॒न्वारो॑ह न्त्य॒न्वारो॑ह न्त्यृ॒तू नृ॒तू न॒न्वारो॑हन्ति । \newline
48. अ॒न्वारो॑ह न्त्यृ॒तुभिर्॑. ऋ॒तुभि॑ र॒न्वारो॑ह न्त्य॒न्वारो॑ह न्त्यृ॒तुभिः॑ । \newline
49. अ॒न्वारो॑ह॒न्तीत्य॑नु - आरो॑हन्ति । \newline
50. ऋ॒तुभिः॑ संॅवथ्स॒रꣳ सं॑ॅवथ्स॒र मृ॒तुभिर्॑. ऋ॒तुभिः॑ संॅवथ्स॒रम् । \newline
51. ऋ॒तुभि॒रित्यृ॒तु - भिः॒ । \newline
52. सं॒ॅव॒थ्स॒रम् ते ते सं॑ॅवथ्स॒रꣳ सं॑ॅवथ्स॒रम् ते । \newline
53. सं॒ॅव॒थ्स॒रमिति॑ सं - व॒थ्स॒रम् । \newline
54. ते सं॑ॅवथ्स॒रे सं॑ॅवथ्स॒रे ते ते सं॑ॅवथ्स॒रे । \newline
55. सं॒ॅव॒थ्स॒र ए॒वैव सं॑ॅवथ्स॒रे सं॑ॅवथ्स॒र ए॒व । \newline
56. सं॒ॅव॒थ्स॒र इति॑ सं - व॒थ्स॒रे । \newline
57. ए॒व प्रति॒ प्रत्ये॒वैव प्रति॑ । \newline
58. प्रति॑ तिष्ठन्ति तिष्ठन्ति॒ प्रति॒ प्रति॑ तिष्ठन्ति । \newline
59. ति॒ष्ठ॒न्ति॒ च॒तु॒र्विꣳ॒॒श श्च॑तुर्विꣳ॒॒श स्ति॑ष्ठन्ति तिष्ठन्ति चतुर्विꣳ॒॒शः । \newline
60. च॒तु॒र्विꣳ॒॒शो भ॑वति भवति चतुर्विꣳ॒॒श श्च॑तुर्विꣳ॒॒शो भ॑वति । \newline
61. च॒तु॒र्विꣳ॒॒श इति॑ चतुः - विꣳ॒॒शः । \newline
62. भ॒व॒ति॒ चतु॑र्विꣳशत्यक्षरा॒ चतु॑र्विꣳशत्यक्षरा भवति भवति॒ चतु॑र्विꣳशत्यक्षरा । \newline
63. चतु॑र्विꣳशत्यक्षरा गाय॒त्री गा॑य॒त्री चतु॑र्विꣳशत्यक्षरा॒ चतु॑र्विꣳशत्यक्षरा गाय॒त्री । \newline
64. चतु॑र्विꣳशत्यक्ष॒रेति॒ चतु॑विꣳशति - अ॒क्ष॒रा॒ । \newline
65. गा॒य॒त्री गा॑य॒त्रम् गा॑य॒त्रम् गा॑य॒त्री गा॑य॒त्री गा॑य॒त्रम् । \newline

\textbf{Ghana Paata } \newline

1. आ॒र्त॒वा उ॑च्यन्त उच्यन्त आर्त॒वा आ᳚र्त॒वा उ॑च्यन्ते॒ ये य उ॑च्यन्त आर्त॒वा आ᳚र्त॒वा उ॑च्यन्ते॒ ये । \newline
2. उ॒च्य॒न्ते॒ ये य उ॑च्यन्त उच्यन्ते॒ य ए॒व मे॒वं ॅय उ॑च्यन्त उच्यन्ते॒ य ए॒वम् । \newline
3. य ए॒व मे॒वं ॅये य ए॒वं ॅवि॒द्वाꣳसो॑ वि॒द्वाꣳस॑ ए॒वं ॅये य ए॒वं ॅवि॒द्वाꣳसः॑ । \newline
4. ए॒वं ॅवि॒द्वाꣳसो॑ वि॒द्वाꣳस॑ ए॒व मे॒वं ॅवि॒द्वाꣳस॑ एकादशरा॒त्र मे॑कादशरा॒त्रं ॅवि॒द्वाꣳस॑ ए॒व मे॒वं ॅवि॒द्वाꣳस॑ एकादशरा॒त्रम् । \newline
5. वि॒द्वाꣳस॑ एकादशरा॒त्र मे॑कादशरा॒त्रं ॅवि॒द्वाꣳसो॑ वि॒द्वाꣳस॑ एकादशरा॒त्र मास॑त॒ आस॑त एकादशरा॒त्रं ॅवि॒द्वाꣳसो॑ वि॒द्वाꣳस॑ एकादशरा॒त्र मास॑ते । \newline
6. ए॒का॒द॒श॒रा॒त्र मास॑त॒ आस॑त एकादशरा॒त्र मे॑कादशरा॒त्र मास॑ते प्र॒जाम् प्र॒जा मास॑त एकादशरा॒त्र मे॑कादशरा॒त्र मास॑ते प्र॒जाम् । \newline
7. ए॒का॒द॒श॒रा॒त्रमित्ये॑कादश - रा॒त्रम् । \newline
8. आस॑ते प्र॒जाम् प्र॒जा मास॑त॒ आस॑ते प्र॒जा मे॒वैव प्र॒जा मास॑त॒ आस॑ते प्र॒जा मे॒व । \newline
9. प्र॒जा मे॒वैव प्र॒जाम् प्र॒जा मे॒व सृ॑जन्ते सृजन्त ए॒व प्र॒जाम् प्र॒जा मे॒व सृ॑जन्ते । \newline
10. प्र॒जामिति॑ प्र - जाम् । \newline
11. ए॒व सृ॑जन्ते सृजन्त ए॒वैव सृ॑जन्ते प्र॒जाम् प्र॒जाꣳ सृ॑जन्त ए॒वैव सृ॑जन्ते प्र॒जाम् । \newline
12. सृ॒ज॒न्ते॒ प्र॒जाम् प्र॒जाꣳ सृ॑जन्ते सृजन्ते प्र॒जा मवाव॑ प्र॒जाꣳ सृ॑जन्ते सृजन्ते प्र॒जा मव॑ । \newline
13. प्र॒जा मवाव॑ प्र॒जाम् प्र॒जा मव॑ रुन्धते रुन्ध॒ते ऽव॑ प्र॒जाम् प्र॒जा मव॑ रुन्धते । \newline
14. प्र॒जामिति॑ प्र - जाम् । \newline
15. अव॑ रुन्धते रुन्ध॒ते ऽवाव॑ रुन्धते प्र॒जाम् प्र॒जाꣳ रु॑न्ध॒ते ऽवाव॑ रुन्धते प्र॒जाम् । \newline
16. रु॒न्ध॒ते॒ प्र॒जाम् प्र॒जाꣳ रु॑न्धते रुन्धते प्र॒जां ॅवि॑न्दन्ते विन्दन्ते प्र॒जाꣳ रु॑न्धते रुन्धते प्र॒जां ॅवि॑न्दन्ते । \newline
17. प्र॒जां ॅवि॑न्दन्ते विन्दन्ते प्र॒जाम् प्र॒जां ॅवि॑न्दन्ते प्र॒जाव॑न्तः प्र॒जाव॑न्तो विन्दन्ते प्र॒जाम् प्र॒जां ॅवि॑न्दन्ते प्र॒जाव॑न्तः । \newline
18. प्र॒जामिति॑ प्र - जाम् । \newline
19. वि॒न्द॒न्ते॒ प्र॒जाव॑न्तः प्र॒जाव॑न्तो विन्दन्ते विन्दन्ते प्र॒जाव॑न्तो भवन्ति भवन्ति प्र॒जाव॑न्तो विन्दन्ते विन्दन्ते प्र॒जाव॑न्तो भवन्ति । \newline
20. प्र॒जाव॑न्तो भवन्ति भवन्ति प्र॒जाव॑न्तः प्र॒जाव॑न्तो भवन्ति॒ ज्योति॒र् ज्योति॑र् भवन्ति प्र॒जाव॑न्तः प्र॒जाव॑न्तो भवन्ति॒ ज्योतिः॑ । \newline
21. प्र॒जाव॑न्त॒ इति॑ प्र॒जा - व॒न्तः॒ । \newline
22. भ॒व॒न्ति॒ ज्योति॒र् ज्योति॑र् भवन्ति भवन्ति॒ ज्योति॑ रतिरा॒त्रो॑ ऽतिरा॒त्रो ज्योति॑र् भवन्ति भवन्ति॒ ज्योति॑ रतिरा॒त्रः । \newline
23. ज्योति॑ रतिरा॒त्रो॑ ऽतिरा॒त्रो ज्योति॒र् ज्योति॑ रतिरा॒त्रो भ॑वति भव त्यतिरा॒त्रो ज्योति॒र् ज्योति॑ रतिरा॒त्रो भ॑वति । \newline
24. अ॒ति॒रा॒त्रो भ॑वति भव त्यतिरा॒त्रो॑ ऽतिरा॒त्रो भ॑वति॒ ज्योति॒र् ज्योति॑र् भव त्यतिरा॒त्रो॑ ऽतिरा॒त्रो भ॑वति॒ ज्योतिः॑ । \newline
25. अ॒ति॒रा॒त्र इत्य॑ति - रा॒त्रः । \newline
26. भ॒व॒ति॒ ज्योति॒र् ज्योति॑र् भवति भवति॒ ज्योति॑ रे॒वैव ज्योति॑र् भवति भवति॒ ज्योति॑ रे॒व । \newline
27. ज्योति॑ रे॒वैव ज्योति॒र् ज्योति॑ रे॒व पु॒रस्ता᳚त् पु॒रस्ता॑ दे॒व ज्योति॒र् ज्योति॑ रे॒व पु॒रस्ता᳚त् । \newline
28. ए॒व पु॒रस्ता᳚त् पु॒रस्ता॑ दे॒वैव पु॒रस्ता᳚द् दधते दधते पु॒रस्ता॑ दे॒वैव पु॒रस्ता᳚द् दधते । \newline
29. पु॒रस्ता᳚द् दधते दधते पु॒रस्ता᳚त् पु॒रस्ता᳚द् दधते सुव॒र्गस्य॑ सुव॒र्गस्य॑ दधते पु॒रस्ता᳚त् पु॒रस्ता᳚द् दधते सुव॒र्गस्य॑ । \newline
30. द॒ध॒ते॒ सु॒व॒र्गस्य॑ सुव॒र्गस्य॑ दधते दधते सुव॒र्गस्य॑ लो॒कस्य॑ लो॒कस्य॑ सुव॒र्गस्य॑ दधते दधते सुव॒र्गस्य॑ लो॒कस्य॑ । \newline
31. सु॒व॒र्गस्य॑ लो॒कस्य॑ लो॒कस्य॑ सुव॒र्गस्य॑ सुव॒र्गस्य॑ लो॒कस्या नु॑ख्यात्या॒ अनु॑ख्यात्यै लो॒कस्य॑ सुव॒र्गस्य॑ सुव॒र्गस्य॑ लो॒कस्या नु॑ख्यात्यै । \newline
32. सु॒व॒र्गस्येति॑ सुवः - गस्य॑ । \newline
33. लो॒कस्या नु॑ख्यात्या॒ अनु॑ख्यात्यै लो॒कस्य॑ लो॒कस्या नु॑ख्यात्यै॒ पृष्ठ्यः॒ पृष्ठ्यो ऽनु॑ख्यात्यै लो॒कस्य॑ लो॒कस्या नु॑ख्यात्यै॒ पृष्ठ्यः॑ । \newline
34. अनु॑ख्यात्यै॒ पृष्ठ्यः॒ पृष्ठ्यो ऽनु॑ख्यात्या॒ अनु॑ख्यात्यै॒ पृष्ठ्य॑ ष्षड॒ह ष्ष॑ड॒हः पृष्ठ्यो ऽनु॑ख्यात्या॒ अनु॑ख्यात्यै॒ पृष्ठ्य॑ ष्षड॒हः । \newline
35. अनु॑ख्यात्या॒ इत्यनु॑ - ख्या॒त्यै॒ । \newline
36. पृष्ठ्य॑ ष्षड॒ह ष्ष॑ड॒हः पृष्ठ्यः॒ पृष्ठ्य॑ ष्षड॒हो भ॑वति भवति षड॒हः पृष्ठ्यः॒ पृष्ठ्य॑ ष्षड॒हो भ॑वति । \newline
37. ष॒ड॒हो भ॑वति भवति षड॒ह ष्ष॑ड॒हो भ॑वति॒ षट् थ्षड् भ॑वति षड॒ह ष्ष॑ड॒हो भ॑वति॒ षट् । \newline
38. ष॒ड॒ह इति॑ षट् - अ॒हः । \newline
39. भ॒व॒ति॒ षट् थ्षड् भ॑वति भवति॒ षड् वै वै षड् भ॑वति भवति॒ षड् वै । \newline
40. षड् वै वै षट् थ्षड् वा ऋ॒तव॑ ऋ॒तवो॒ वै षट् थ्षड् वा ऋ॒तवः॑ । \newline
41. वा ऋ॒तव॑ ऋ॒तवो॒ वै वा ऋ॒तव॒ ष्षट् थ्षडृ॒तवो॒ वै वा ऋ॒तव॒ ष्षट् । \newline
42. ऋ॒तव॒ ष्षट् थ्षडृ॒तव॑ ऋ॒तव॒ ष्षट् पृ॒ष्ठानि॑ पृ॒ष्ठानि॒ षडृ॒तव॑ ऋ॒तव॒ ष्षट् पृ॒ष्ठानि॑ । \newline
43. षट् पृ॒ष्ठानि॑ पृ॒ष्ठानि॒ षट् थ्षट् पृ॒ष्ठानि॑ पृ॒ष्ठैः पृ॒ष्ठैः पृ॒ष्ठानि॒ षट् थ्षट् पृ॒ष्ठानि॑ पृ॒ष्ठैः । \newline
44. पृ॒ष्ठानि॑ पृ॒ष्ठैः पृ॒ष्ठैः पृ॒ष्ठानि॑ पृ॒ष्ठानि॑ पृ॒ष्ठै रे॒वैव पृ॒ष्ठैः पृ॒ष्ठानि॑ पृ॒ष्ठानि॑ पृ॒ष्ठै रे॒व । \newline
45. पृ॒ष्ठै रे॒वैव पृ॒ष्ठैः पृ॒ष्ठै रे॒व र्‌तू नृ॒तू ने॒व पृ॒ष्ठैः पृ॒ष्ठै रे॒व र्‌तून् । \newline
46. ए॒व र्‌तू नृ॒तू ने॒वैव र्‌तू न॒न्वारो॑ह न्त्य॒न्वारो॑ह न्त्यृ॒तू ने॒वैव र्‌तू न॒न्वारो॑हन्ति । \newline
47. ऋ॒तू न॒न्वारो॑ह न्त्य॒न्वारो॑ह न्त्यृ॒तू नृ॒तून॒ न्वारो॑ह न्त्यृ॒तुभिर्॑. ऋ॒तुभि॑ र॒न्वारो॑ह न्त्यृ॒तू नृ॒तून॒ न्वारो॑ह न्त्यृ॒तुभिः॑ । \newline
48. अ॒न्वारो॑ह न्त्यृ॒तुभिर्॑. ऋ॒तुभि॑ र॒न्वारो॑ह न्त्य॒न्वारो॑ह न्त्यृ॒तुभिः॑ संॅवथ्स॒रꣳ सं॑ॅवथ्स॒र मृ॒तुभि॑ र॒न्वारो॑ह न्त्य॒न्वारो॑ह न्त्यृ॒तुभिः॑ संॅवथ्स॒रम् । \newline
49. अ॒न्वारो॑ह॒न्तीत्य॑नु - आरो॑हन्ति । \newline
50. ऋ॒तुभिः॑ संॅवथ्स॒रꣳ सं॑ॅवथ्स॒र मृ॒तुभिर्॑. ऋ॒तुभिः॑ संॅवथ्स॒रम् ते ते सं॑ॅवथ्स॒र मृ॒तुभिर्॑. ऋ॒तुभिः॑ संॅवथ्स॒रम् ते । \newline
51. ऋ॒तुभि॒रित्यृ॒तु - भिः॒ । \newline
52. सं॒ॅव॒थ्स॒रम् ते ते सं॑ॅवथ्स॒रꣳ सं॑ॅवथ्स॒रम् ते सं॑ॅवथ्स॒रे सं॑ॅवथ्स॒रे ते सं॑ॅवथ्स॒रꣳ सं॑ॅवथ्स॒रम् ते सं॑ॅवथ्स॒रे । \newline
53. सं॒ॅव॒थ्स॒रमिति॑ सं - व॒थ्स॒रम् । \newline
54. ते सं॑ॅवथ्स॒रे सं॑ॅवथ्स॒रे ते ते सं॑ॅवथ्स॒र ए॒वैव सं॑ॅवथ्स॒रे ते ते सं॑ॅवथ्स॒र ए॒व । \newline
55. सं॒ॅव॒थ्स॒र ए॒वैव सं॑ॅवथ्स॒रे सं॑ॅवथ्स॒र ए॒व प्रति॒ प्रत्ये॒व सं॑ॅवथ्स॒रे सं॑ॅवथ्स॒र ए॒व प्रति॑ । \newline
56. सं॒ॅव॒थ्स॒र इति॑ सं - व॒थ्स॒रे । \newline
57. ए॒व प्रति॒ प्रत्ये॒ वैव प्रति॑ तिष्ठन्ति तिष्ठन्ति॒ प्रत्ये॒ वैव प्रति॑ तिष्ठन्ति । \newline
58. प्रति॑ तिष्ठन्ति तिष्ठन्ति॒ प्रति॒ प्रति॑ तिष्ठन्ति चतुर्विꣳ॒॒श श्च॑तुर्विꣳ॒॒श स्ति॑ष्ठन्ति॒ प्रति॒ प्रति॑ तिष्ठन्ति चतुर्विꣳ॒॒शः । \newline
59. ति॒ष्ठ॒न्ति॒ च॒तु॒र्विꣳ॒॒श श्च॑तुर्विꣳ॒॒श स्ति॑ष्ठन्ति तिष्ठन्ति चतुर्विꣳ॒॒शो भ॑वति भवति चतुर्विꣳ॒॒श स्ति॑ष्ठन्ति तिष्ठन्ति चतुर्विꣳ॒॒शो भ॑वति । \newline
60. च॒तु॒र्विꣳ॒॒शो भ॑वति भवति चतुर्विꣳ॒॒श श्च॑तुर्विꣳ॒॒शो भ॑वति॒ चतु॑र्विꣳशत्यक्षरा॒ चतु॑र्विꣳशत्यक्षरा भवति चतुर्विꣳ॒॒श श्च॑तुर्विꣳ॒॒शो भ॑वति॒ चतु॑र्विꣳशत्यक्षरा । \newline
61. च॒तु॒र्विꣳ॒॒श इति॑ चतुः - विꣳ॒॒शः । \newline
62. भ॒व॒ति॒ चतु॑र्विꣳशत्यक्षरा॒ चतु॑र्विꣳशत्यक्षरा भवति भवति॒ चतु॑र्विꣳशत्यक्षरा गाय॒त्री गा॑य॒त्री चतु॑र्विꣳशत्यक्षरा भवति भवति॒ चतु॑र्विꣳशत्यक्षरा गाय॒त्री । \newline
63. चतु॑र्विꣳशत्यक्षरा गाय॒त्री गा॑य॒त्री चतु॑र्विꣳशत्यक्षरा॒ चतु॑र्विꣳशत्यक्षरा गाय॒त्री गा॑य॒त्रम् गा॑य॒त्रम् गा॑य॒त्री चतु॑र्विꣳशत्यक्षरा॒ चतु॑र्विꣳशत्यक्षरा गाय॒त्री गा॑य॒त्रम् । \newline
64. चतु॑र्विꣳशत्यक्ष॒रेति॒ चतु॑विꣳशति - अ॒क्ष॒रा॒ । \newline
65. गा॒य॒त्री गा॑य॒त्रम् गा॑य॒त्रम् गा॑य॒त्री गा॑य॒त्री गा॑य॒त्रम् ब्र॑ह्मवर्च॒सम् ब्र॑ह्मवर्च॒सम् गा॑य॒त्रम् गा॑य॒त्री गा॑य॒त्री गा॑य॒त्रम् ब्र॑ह्मवर्च॒सम् । \newline
\pagebreak
\markright{ TS 7.2.6.3  \hfill https://www.vedavms.in \hfill}

\section{ TS 7.2.6.3 }

\textbf{TS 7.2.6.3 } \newline
\textbf{Samhita Paata} \newline

गा॑य॒त्रं ब्र॑ह्मवर्च॒सं गा॑यत्रि॒यामे॒व ब्र॑ह्मवर्च॒से प्रति॑ तिष्ठन्ति चतुश्चत्वारिꣳ॒॒शो भ॑वति॒ चतु॑ष्चत्वारिꣳशदक्षरा त्रि॒ष्टुगि॑न्द्रि॒यं त्रि॒ष्टुप् त्रि॒ष्टुभ्ये॒वेन्द्रि॒ये प्रति॑ तिष्ठन्त्यष्टाचत्वारिꣳ॒॒शो भ॑वत्य॒ष्टाच॑त्वारिꣳशदक्षरा॒ जग॑ती॒ जाग॑ताः प॒शवो॒ जग॑त्यामे॒व प॒शुषु॒ प्रति॑ तिष्ठन्त्ये-कादशरा॒त्रो भ॑वति॒ पञ्च॒ वा ऋ॒तव॑ आर्त॒वाः पञ्च॒र्तुष्वे॒वाऽऽ*र्त॒वेषु॑ संॅवथ्स॒रे प्र॑ति॒ष्ठाय॑ प्र॒जामव॑ रुन्धते ऽतिरा॒त्राव॒भितो॑ भवतः प्र॒जायै॒ परि॑गृहीत्यै ॥ \newline

\textbf{Pada Paata} \newline

गा॒य॒त्रम् । ब्र॒ह्म॒व॒र्च॒समिति॑ ब्रह्म - व॒र्च॒सम् । गा॒य॒त्रि॒याम् । ए॒व । ब्र॒ह्म॒व॒र्च॒स इति॑ ब्रह्म - व॒र्च॒से । प्रतीति॑ । ति॒ष्ठ॒न्ति॒ । च॒तु॒श्च॒त्वा॒रिꣳ॒॒श इति॑ चतुः - च॒त्वा॒रिꣳ॒॒शः । भ॒व॒ति॒ । चतु॑श्चत्वारिꣳशदक्ष॒रेति॒ चतु॑श्चत्वारिꣳशत् - अ॒क्ष॒रा॒ । त्रि॒ष्टुक् । इ॒न्द्रि॒यम् । त्रि॒ष्टुप् । त्रि॒ष्टुभि॑ । ए॒व । इ॒न्द्रि॒ये । प्रतीति॑ । ति॒ष्ठ॒न्ति॒ । अ॒ष्टा॒च॒त्वा॒रिꣳ॒॒श इत्य॑ष्टा - च॒त्वा॒रिꣳ॒॒शः । भ॒व॒ति॒ । अ॒ष्टाच॑त्वारिꣳशदक्ष॒रेत्य॒ष्टाच॑त्वारिꣳशत् - अ॒क्ष॒रा॒ । जग॑ती । जाग॑ताः । प॒शवः॑ । जग॑त्याम् । ए॒व । प॒शुषु॑ । प्रतीति॑ । ति॒ष्ठ॒न्ति॒ । ए॒का॒द॒श॒रा॒त्र इत्ये॑कादश-रा॒त्रः । भ॒व॒ति॒ । पञ्च॑ । वै । ऋ॒तवः॑ । आ॒र्त॒वाः । पञ्च॑ । ऋ॒तुषु॑ । ए॒व । आ॒र्त॒वेषु॑ । सं॒ॅव॒थ्स॒र इति॑ सं - व॒थ्स॒रे । प्र॒ति॒ष्ठायेति॑ प्रति - स्थाय॑ । प्र॒जामिति॑ प्र - जाम् । अवेति॑ । रु॒न्ध॒ते॒ । अ॒ति॒रा॒त्रावित्य॑ति - रा॒त्रौ । अ॒भितः॑ । भ॒व॒तः॒ । प्र॒जाया॒ इति॑ प्र - जायै᳚ । परि॑गृहीत्या॒ इति॒ परि॑-गृ॒ही॒त्यै॒ ॥  \newline


\textbf{Krama Paata} \newline

गा॒य॒त्रम् ब्र॑ह्मवर्च॒सम् । ब्र॒ह्म॒व॒र्च॒सम् गा॑यत्रि॒याम् । ब्र॒ह्म॒व॒र्च॒समिति॑ ब्रह्म - व॒र्च॒सम् । गा॒य॒त्रि॒यामे॒व । ए॒व ब्र॑ह्मवर्च॒से । ब्र॒ह्म॒व॒र्च॒से प्रति॑ । ब्र॒ह्म॒व॒र्च॒स इति॑ ब्रह्म - व॒र्च॒से । प्रति॑ तिष्ठन्ति । ति॒ष्ठ॒न्ति॒ च॒तु॒श्च॒त्वा॒रिꣳ॒॒शः । च॒तु॒श्च॒त्वा॒रिꣳ॒॒शो भ॑वति । च॒तु॒श्च॒त्वा॒रिꣳ॒॒श इति॑ चतुः - च॒त्वा॒रिꣳ॒॒शः । भ॒व॒ति॒ चतु॑श्चत्वारिꣳशदक्षरा । चतु॑श्चत्वारिꣳशदक्षरा त्रि॒ष्टुक् । चतु॑श्चत्वारिꣳशदक्ष॒रेति॒ चतु॑श्चत्वारिꣳशत् - अ॒क्ष॒रा॒ । त्रि॒ष्टुगि॑न्द्रि॒यम् । इ॒न्द्रि॒यम् त्रि॒ष्टुप् । त्रि॒ष्टुप् त्रि॒ष्टुभि॑ । त्रि॒ष्टुभ्ये॒व । ए॒वेन्द्रि॒ये । इ॒न्द्रि॒ये प्रति॑ । प्रति॑ तिष्ठन्ति । ति॒ष्ठ॒न्त्य॒ष्टा॒च॒त्वा॒रिꣳ॒॒शः । अ॒ष्टा॒च॒त्वा॒रिꣳ॒॒शो भ॑वति । अ॒ष्टा॒च॒त्वा॒रिꣳ॒॒श इत्य॑ष्टा - च॒त्वा॒रिꣳ॒॒शः । भ॒व॒त्य॒ष्टा,च॑त्वारिꣳशदक्षरा । अ॒ष्टाच॑त्वारिꣳशदक्षरा॒ जग॑ती । अ॒ष्टाच॑त्वारिꣳश,दक्ष॒रेत्य॒ष्टाच॑त्वारिꣳशत् - अ॒क्ष॒रा॒ । जग॑ती॒ जाग॑ताः । जाग॑ताः प॒शवः॑ । प॒शवो॒ जग॑त्याम् । जग॑त्यामे॒व । ए॒व प॒शुषु॑ । प॒शुषु॒ प्रति॑ । प्रति॑ तिष्ठन्ति । ति॒ष्ठ॒न्त्ये॒का॒द॒श॒रा॒त्रः । ए॒का॒द॒श॒रा॒त्रो भ॑वति । ए॒का॒द॒श॒रा॒त्र इत्ये॑काश - रा॒त्रः । भ॒व॒ति॒ पञ्च॑ । पञ्च॒ वै । वा ऋ॒तवः॑ । ऋ॒तव॑ आर्त॒वाः । आ॒र्त॒वाः पञ्च॑ । पञ्च॒र्तुषु॑ । ऋ॒तुष्वे॒व । ए॒वार्त॒वेषु॑ । आ॒र्त॒वेषु॑ सम्ॅवथ्स॒रे । स॒म्ॅव॒थ्स॒रे प्र॑ति॒ष्ठाय॑ । स॒म्ॅव॒थ्स॒र इति॑ सम् - व॒थ्स॒रे । प्र॒ति॒ष्ठाय॑ प्र॒जाम् । प्र॒ति॒ष्ठायेति॑ प्रति - स्थाय॑ । प्र॒जामव॑ । प्र॒जामिति॑ प्र - जाम् । अव॑ रुन्धते । रु॒न्ध॒ते॒ऽति॒रा॒त्रौ । अ॒ति॒रा॒त्राव॒भितः॑ । अ॒ति॒रा॒त्रावित्य॑ति - रा॒त्रौ । अ॒भितो॑ भवतः । भ॒व॒तः॒ प्र॒जायै᳚ । प्र॒जायै॒ परि॑गृहीत्यै । प्र॒जाया॒ इति॑ प्र - जायै᳚ । परि॑गृहीत्या॒ इति॒ परि॑ - गृ॒ही॒त्यै॒ । \newline

\textbf{Jatai Paata} \newline

1. गा॒य॒त्रम् ब्र॑ह्मवर्च॒सम् ब्र॑ह्मवर्च॒सम् गा॑य॒त्रम् गा॑य॒त्रम् ब्र॑ह्मवर्च॒सम् । \newline
2. ब्र॒ह्म॒व॒र्च॒सम् गा॑यत्रि॒याम् गा॑यत्रि॒याम् ब्र॑ह्मवर्च॒सम् ब्र॑ह्मवर्च॒सम् गा॑यत्रि॒याम् । \newline
3. ब्र॒ह्म॒व॒र्च॒समिति॑ ब्रह्म - व॒र्च॒सम् । \newline
4. गा॒य॒त्रि॒या मे॒वैव गा॑यत्रि॒याम् गा॑यत्रि॒या मे॒व । \newline
5. ए॒व ब्र॑ह्मवर्च॒से ब्र॑ह्मवर्च॒स ए॒वैव ब्र॑ह्मवर्च॒से । \newline
6. ब्र॒ह्म॒व॒र्च॒से प्रति॒ प्रति॑ ब्रह्मवर्च॒से ब्र॑ह्मवर्च॒से प्रति॑ । \newline
7. ब्र॒ह्म॒व॒र्च॒स इति॑ ब्रह्म - व॒र्च॒से । \newline
8. प्रति॑ तिष्ठन्ति तिष्ठन्ति॒ प्रति॒ प्रति॑ तिष्ठन्ति । \newline
9. ति॒ष्ठ॒न्ति॒ च॒तु॒श्च॒त्वा॒रिꣳ॒॒श श्च॑तुश्चत्वारिꣳ॒॒श स्ति॑ष्ठन्ति तिष्ठन्ति चतुश्चत्वारिꣳ॒॒शः । \newline
10. च॒तु॒श्च॒त्वा॒रिꣳ॒॒शो भ॑वति भवति चतुश्चत्वारिꣳ॒॒श श्च॑तुश्चत्वारिꣳ॒॒शो भ॑वति । \newline
11. च॒तु॒श्च॒त्वा॒रिꣳ॒॒श इति॑ चतुः - च॒त्वा॒रिꣳ॒॒शः । \newline
12. भ॒व॒ति॒ चतु॑श्चत्वारिꣳशदक्षरा॒ चतु॑श्चत्वारिꣳशदक्षरा भवति भवति॒ चतु॑श्चत्वारिꣳशदक्षरा । \newline
13. चतु॑श्चत्वारिꣳशदक्षरा त्रि॒ष्टुक् त्रि॒ष्टुक् चतु॑श्चत्वारिꣳशदक्षरा॒ चतु॑श्चत्वारिꣳशदक्षरा त्रि॒ष्टुक् । \newline
14. चतु॑श्चत्वारिꣳशदक्ष॒रेति॒ चतु॑श्चत्वारिꣳशत् - अ॒क्ष॒रा॒ । \newline
15. त्रि॒ष्टु गि॑न्द्रि॒य मि॑न्द्रि॒यम् त्रि॒ष्टुक् त्रि॒ष्टु गि॑न्द्रि॒यम् । \newline
16. इ॒न्द्रि॒यम् त्रि॒ष्टुप् त्रि॒ष्टु बि॑न्द्रि॒य मि॑न्द्रि॒यम् त्रि॒ष्टुप् । \newline
17. त्रि॒ष्टुप् त्रि॒ष्टुभि॑ त्रि॒ष्टुभि॑ त्रि॒ष्टुप् त्रि॒ष्टुप् त्रि॒ष्टुभि॑ । \newline
18. त्रि॒ष्टु भ्ये॒वैव त्रि॒ष्टुभि॑ त्रि॒ष्टु भ्ये॒व । \newline
19. ए॒वेन्द्रि॒य इ॑न्द्रि॒य ए॒वैवेन्द्रि॒ये । \newline
20. इ॒न्द्रि॒ये प्रति॒ प्रती᳚ न्द्रि॒य इ॑न्द्रि॒ये प्रति॑ । \newline
21. प्रति॑ तिष्ठन्ति तिष्ठन्ति॒ प्रति॒ प्रति॑ तिष्ठन्ति । \newline
22. ति॒ष्ठ॒ न्त्य॒ष्टा॒च॒त्वा॒रिꣳ॒॒शो᳚ ऽष्टाचत्वारिꣳ॒॒श स्ति॑ष्ठन्ति तिष्ठ न्त्यष्टाचत्वारिꣳ॒॒शः । \newline
23. अ॒ष्टा॒च॒त्वा॒रिꣳ॒॒शो भ॑वति भव त्यष्टाचत्वारिꣳ॒॒शो᳚ ऽष्टाचत्वारिꣳ॒॒शो भ॑वति । \newline
24. अ॒ष्टा॒च॒त्वा॒रिꣳ॒॒श इत्य॑ष्टा - च॒त्वा॒रिꣳ॒॒शः । \newline
25. भ॒व॒ त्य॒ष्टाच॑त्वारिꣳशदक्षरा॒ ऽष्टाच॑त्वारिꣳशदक्षरा भवति भव त्य॒ष्टाच॑त्वारिꣳशदक्षरा । \newline
26. अ॒ष्टाच॑त्वारिꣳशदक्षरा॒ जग॑ती॒ जग॑ त्य॒ष्टाच॑त्वारिꣳशदक्षरा॒ ऽष्टाच॑त्वारिꣳशदक्षरा॒ जग॑ती । \newline
27. अ॒ष्टाच॑त्वारिꣳशदक्ष॒रेत्य॒ष्टाच॑त्वारिꣳशत् - अ॒क्ष॒रा॒ । \newline
28. जग॑ती॒ जाग॑ता॒ जाग॑ता॒ जग॑ती॒ जग॑ती॒ जाग॑ताः । \newline
29. जाग॑ताः प॒शवः॑ प॒शवो॒ जाग॑ता॒ जाग॑ताः प॒शवः॑ । \newline
30. प॒शवो॒ जग॑त्या॒म् जग॑त्याम् प॒शवः॑ प॒शवो॒ जग॑त्याम् । \newline
31. जग॑त्या मे॒वैव जग॑त्या॒म् जग॑त्या मे॒व । \newline
32. ए॒व प॒शुषु॑ प॒शु ष्वे॒वैव प॒शुषु॑ । \newline
33. प॒शुषु॒ प्रति॒ प्रति॑ प॒शुषु॑ प॒शुषु॒ प्रति॑ । \newline
34. प्रति॑ तिष्ठन्ति तिष्ठन्ति॒ प्रति॒ प्रति॑ तिष्ठन्ति । \newline
35. ति॒ष्ठ॒ न्त्ये॒का॒द॒श॒रा॒त्र ए॑कादशरा॒त्र स्ति॑ष्ठन्ति तिष्ठ न्त्येकादशरा॒त्रः । \newline
36. ए॒का॒द॒श॒रा॒त्रो भ॑वति भव त्येकादशरा॒त्र ए॑कादशरा॒त्रो भ॑वति । \newline
37. ए॒का॒द॒श॒रा॒त्र इत्ये॑कादश - रा॒त्रः । \newline
38. भ॒व॒ति॒ पञ्च॒ पञ्च॑ भवति भवति॒ पञ्च॑ । \newline
39. पञ्च॒ वै वै पञ्च॒ पञ्च॒ वै । \newline
40. वा ऋ॒तव॑ ऋ॒तवो॒ वै वा ऋ॒तवः॑ । \newline
41. ऋ॒तव॑ आर्त॒वा आ᳚र्त॒वा ऋ॒तव॑ ऋ॒तव॑ आर्त॒वाः । \newline
42. आ॒र्त॒वाः पञ्च॒ पञ्चा᳚र्त॒वा आ᳚र्त॒वाः पञ्च॑ । \newline
43. पञ्च॒ र्‌तुष् वृ॒तुषु॒ पञ्च॒ पञ्च॒ र्तुषु॑ । \newline
44. ऋ॒तुष् वे॒वैव र्‌तुष् वृ॒तु ष्वे॒व । \newline
45. ए॒वार्त॒वे ष्वा᳚र्त॒वे ष्वे॒वै वार्त॒वेषु॑ । \newline
46. आ॒र्त॒वेषु॑ संॅवथ्स॒रे सं॑ॅवथ्स॒र आ᳚र्त॒वे ष्वा᳚र्त॒वेषु॑ संॅवथ्स॒रे । \newline
47. सं॒ॅव॒थ्स॒रे प्र॑ति॒ष्ठाय॑ प्रति॒ष्ठाय॑ संॅवथ्स॒रे सं॑ॅवथ्स॒रे प्र॑ति॒ष्ठाय॑ । \newline
48. सं॒ॅव॒थ्स॒र इति॑ सं - व॒थ्स॒रे । \newline
49. प्र॒ति॒ष्ठाय॑ प्र॒जाम् प्र॒जाम् प्र॑ति॒ष्ठाय॑ प्रति॒ष्ठाय॑ प्र॒जाम् । \newline
50. प्र॒ति॒ष्ठायेति॑ प्रति - स्थाय॑ । \newline
51. प्र॒जा मवाव॑ प्र॒जाम् प्र॒जा मव॑ । \newline
52. प्र॒जामिति॑ प्र - जाम् । \newline
53. अव॑ रुन्धते रुन्ध॒ते ऽवाव॑ रुन्धते । \newline
54. रु॒न्ध॒ते॒ ऽति॒रा॒त्रा व॑तिरा॒त्रौ रु॑न्धते रुन्धते ऽतिरा॒त्रौ । \newline
55. अ॒ति॒रा॒त्रा व॒भितो॒ ऽभितो॑ ऽतिरा॒त्रा व॑तिरा॒त्रा व॒भितः॑ । \newline
56. अ॒ति॒रा॒त्रावित्य॑ति - रा॒त्रौ । \newline
57. अ॒भितो॑ भवतो भवतो॒ ऽभितो॒ ऽभितो॑ भवतः । \newline
58. भ॒व॒तः॒ प्र॒जायै᳚ प्र॒जायै॑ भवतो भवतः प्र॒जायै᳚ । \newline
59. प्र॒जायै॒ परि॑गृहीत्यै॒ परि॑गृहीत्यै प्र॒जायै᳚ प्र॒जायै॒ परि॑गृहीत्यै । \newline
60. प्र॒जाया॒ इति॑ प्र - जायै᳚ । \newline
61. परि॑गृहीत्या॒ इति॒ परि॑ - गृ॒ही॒त्यै॒ । \newline

\textbf{Ghana Paata } \newline

1. गा॒य॒त्रम् ब्र॑ह्मवर्च॒सम् ब्र॑ह्मवर्च॒सम् गा॑य॒त्रम् गा॑य॒त्रम् ब्र॑ह्मवर्च॒सम् गा॑यत्रि॒याम् गा॑यत्रि॒याम् ब्र॑ह्मवर्च॒सम् गा॑य॒त्रम् गा॑य॒त्रम् ब्र॑ह्मवर्च॒सम् गा॑यत्रि॒याम् । \newline
2. ब्र॒ह्म॒व॒र्च॒सम् गा॑यत्रि॒याम् गा॑यत्रि॒याम् ब्र॑ह्मवर्च॒सम् ब्र॑ह्मवर्च॒सम् गा॑यत्रि॒या मे॒वैव गा॑यत्रि॒याम् ब्र॑ह्मवर्च॒सम् ब्र॑ह्मवर्च॒सम् गा॑यत्रि॒या मे॒व । \newline
3. ब्र॒ह्म॒व॒र्च॒समिति॑ ब्रह्म - व॒र्च॒सम् । \newline
4. गा॒य॒त्रि॒या मे॒वैव गा॑यत्रि॒याम् गा॑यत्रि॒या मे॒व ब्र॑ह्मवर्च॒से ब्र॑ह्मवर्च॒स ए॒व गा॑यत्रि॒याम् गा॑यत्रि॒या मे॒व ब्र॑ह्मवर्च॒से । \newline
5. ए॒व ब्र॑ह्मवर्च॒से ब्र॑ह्मवर्च॒स ए॒वैव ब्र॑ह्मवर्च॒से प्रति॒ प्रति॑ ब्रह्मवर्च॒स ए॒वैव ब्र॑ह्मवर्च॒से प्रति॑ । \newline
6. ब्र॒ह्म॒व॒र्च॒से प्रति॒ प्रति॑ ब्रह्मवर्च॒से ब्र॑ह्मवर्च॒से प्रति॑ तिष्ठन्ति तिष्ठन्ति॒ प्रति॑ ब्रह्मवर्च॒से ब्र॑ह्मवर्च॒से प्रति॑ तिष्ठन्ति । \newline
7. ब्र॒ह्म॒व॒र्च॒स इति॑ ब्रह्म - व॒र्च॒से । \newline
8. प्रति॑ तिष्ठन्ति तिष्ठन्ति॒ प्रति॒ प्रति॑ तिष्ठन्ति चतुश्चत्वारिꣳ॒॒श श्च॑तुश्चत्वारिꣳ॒॒श स्ति॑ष्ठन्ति॒ प्रति॒ प्रति॑ तिष्ठन्ति चतुश्चत्वारिꣳ॒॒शः । \newline
9. ति॒ष्ठ॒न्ति॒ च॒तु॒श्च॒त्वा॒रिꣳ॒॒श श्च॑तुश्चत्वारिꣳ॒॒श स्ति॑ष्ठन्ति तिष्ठन्ति चतुश्चत्वारिꣳ॒॒शो भ॑वति भवति चतुश्चत्वारिꣳ॒॒श स्ति॑ष्ठन्ति तिष्ठन्ति चतुश्चत्वारिꣳ॒॒शो भ॑वति । \newline
10. च॒तु॒श्च॒त्वा॒रिꣳ॒॒शो भ॑वति भवति चतुश्चत्वारिꣳ॒॒श श्च॑तुश्चत्वारिꣳ॒॒शो भ॑वति॒ चतु॑श्चत्वारिꣳशदक्षरा॒ चतु॑श्चत्वारिꣳशदक्षरा भवति चतुश्चत्वारिꣳ॒॒श श्च॑तुश्चत्वारिꣳ॒॒शो भ॑वति॒ चतु॑श्चत्वारिꣳशदक्षरा । \newline
11. च॒तु॒श्च॒त्वा॒रिꣳ॒॒श इति॑ चतुः - च॒त्वा॒रिꣳ॒॒शः । \newline
12. भ॒व॒ति॒ चतु॑श्चत्वारिꣳशदक्षरा॒ चतु॑श्चत्वारिꣳशदक्षरा भवति भवति॒ चतु॑श्चत्वारिꣳशदक्षरा त्रि॒ष्टुक् त्रि॒ष्टुक् चतु॑श्चत्वारिꣳशदक्षरा भवति भवति॒ चतु॑श्चत्वारिꣳशदक्षरा त्रि॒ष्टुक् । \newline
13. चतु॑श्चत्वारिꣳशदक्षरा त्रि॒ष्टुक् त्रि॒ष्टुक् चतु॑श्चत्वारिꣳशदक्षरा॒ चतु॑श्चत्वारिꣳशदक्षरा त्रि॒ष्टु गि॑न्द्रि॒य मि॑न्द्रि॒यम् त्रि॒ष्टुक् चतु॑श्चत्वारिꣳशदक्षरा॒ चतु॑श्चत्वारिꣳशदक्षरा त्रि॒ष्टु गि॑न्द्रि॒यम् । \newline
14. चतु॑श्चत्वारिꣳशदक्ष॒रेति॒ चतु॑श्चत्वारिꣳशत् - अ॒क्ष॒रा॒ । \newline
15. त्रि॒ष्टु गि॑न्द्रि॒य मि॑न्द्रि॒यम् त्रि॒ष्टुक् त्रि॒ष्टु गि॑न्द्रि॒यम् त्रि॒ष्टुप् त्रि॒ष्टु बि॑न्द्रि॒यम् त्रि॒ष्टुक् त्रि॒ष्टु गि॑न्द्रि॒यम् त्रि॒ष्टुप् । \newline
16. इ॒न्द्रि॒यम् त्रि॒ष्टुप् त्रि॒ष्टु बि॑न्द्रि॒य मि॑न्द्रि॒यम् त्रि॒ष्टुप् त्रि॒ष्टुभि॑ त्रि॒ष्टुभि॑ त्रि॒ष्टु बि॑न्द्रि॒य मि॑न्द्रि॒यम् त्रि॒ष्टुप् त्रि॒ष्टुभि॑ । \newline
17. त्रि॒ष्टुप् त्रि॒ष्टुभि॑ त्रि॒ष्टुभि॑ त्रि॒ष्टुप् त्रि॒ष्टुप् त्रि॒ष्टुभ्ये॒ वैव त्रि॒ष्टुभि॑ त्रि॒ष्टुप् त्रि॒ष्टुप् त्रि॒ष्टु भ्ये॒व । \newline
18. त्रि॒ष्टुभ्ये॒ वैव त्रि॒ष्टुभि॑ त्रि॒ष्टु भ्ये॒वेन्द्रि॒य इ॑न्द्रि॒य ए॒व त्रि॒ष्टुभि॑ त्रि॒ष्टु भ्ये॒वेन्द्रि॒ये । \newline
19. ए॒वेन्द्रि॒य इ॑न्द्रि॒य ए॒वैवेन्द्रि॒ये प्रति॒ प्रती᳚न्द्रि॒य ए॒वैवेन्द्रि॒ये प्रति॑ । \newline
20. इ॒न्द्रि॒ये प्रति॒ प्रती᳚न्द्रि॒य इ॑न्द्रि॒ये प्रति॑ तिष्ठन्ति तिष्ठन्ति॒ प्रती᳚न्द्रि॒य इ॑न्द्रि॒ये प्रति॑ तिष्ठन्ति । \newline
21. प्रति॑ तिष्ठन्ति तिष्ठन्ति॒ प्रति॒ प्रति॑ तिष्ठ न्त्यष्टाचत्वारिꣳ॒॒शो᳚ ऽष्टाचत्वारिꣳ॒॒श स्ति॑ष्ठन्ति॒ प्रति॒ प्रति॑ तिष्ठ न्त्यष्टाचत्वारिꣳ॒॒शः । \newline
22. ति॒ष्ठ॒ न्त्य॒ष्टा॒च॒त्वा॒रिꣳ॒॒शो᳚ ऽष्टाचत्वारिꣳ॒॒श स्ति॑ष्ठन्ति तिष्ठ न्त्यष्टाचत्वारिꣳ॒॒शो भ॑वति भव त्यष्टाचत्वारिꣳ॒॒श स्ति॑ष्ठन्ति तिष्ठ न्त्यष्टाचत्वारिꣳ॒॒शो भ॑वति । \newline
23. अ॒ष्टा॒च॒त्वा॒रिꣳ॒॒शो भ॑वति भव त्यष्टाचत्वारिꣳ॒॒शो᳚ ऽष्टाचत्वारिꣳ॒॒शो भ॑व त्य॒ष्टाच॑त्वारिꣳशदक्षरा॒ ऽष्टाच॑त्वारिꣳशदक्षरा भव त्यष्टाचत्वारिꣳ॒॒शो᳚ ऽष्टाचत्वारिꣳ॒॒शो भ॑व त्य॒ष्टाच॑त्वारिꣳशदक्षरा । \newline
24. अ॒ष्टा॒च॒त्वा॒रिꣳ॒॒श इत्य॑ष्टा - च॒त्वा॒रिꣳ॒॒शः । \newline
25. भ॒व॒ त्य॒ष्टाच॑त्वारिꣳशदक्षरा॒ ऽष्टाच॑त्वारिꣳशदक्षरा भवति भव त्य॒ष्टाच॑त्वारिꣳशदक्षरा॒ जग॑ती॒ जग॑ त्य॒ष्टाच॑त्वारिꣳशदक्षरा भवति भव त्य॒ष्टाच॑त्वारिꣳशदक्षरा॒ जग॑ती । \newline
26. अ॒ष्टाच॑त्वारिꣳशदक्षरा॒ जग॑ती॒ जग॑ त्य॒ष्टाच॑त्वारिꣳशदक्षरा॒ ऽष्टाच॑त्वारिꣳशदक्षरा॒ जग॑ती॒ जाग॑ता॒ जाग॑ता॒ जग॑ त्य॒ष्टाच॑त्वारिꣳशदक्षरा॒ ऽष्टाच॑त्वारिꣳशदक्षरा॒ जग॑ती॒ जाग॑ताः । \newline
27. अ॒ष्टाच॑त्वारिꣳशदक्ष॒रेत्य॒ष्टाच॑त्वारिꣳशत् - अ॒क्ष॒रा॒ । \newline
28. जग॑ती॒ जाग॑ता॒ जाग॑ता॒ जग॑ती॒ जग॑ती॒ जाग॑ताः प॒शवः॑ प॒शवो॒ जाग॑ता॒ जग॑ती॒ जग॑ती॒ जाग॑ताः प॒शवः॑ । \newline
29. जाग॑ताः प॒शवः॑ प॒शवो॒ जाग॑ता॒ जाग॑ताः प॒शवो॒ जग॑त्या॒म् जग॑त्याम् प॒शवो॒ जाग॑ता॒ जाग॑ताः प॒शवो॒ जग॑त्याम् । \newline
30. प॒शवो॒ जग॑त्या॒म् जग॑त्याम् प॒शवः॑ प॒शवो॒ जग॑त्या मे॒वैव जग॑त्याम् प॒शवः॑ प॒शवो॒ जग॑त्या मे॒व । \newline
31. जग॑त्या मे॒वैव जग॑त्या॒म् जग॑त्या मे॒व प॒शुषु॑ प॒शुष्वे॒व जग॑त्या॒म् जग॑त्या मे॒व प॒शुषु॑ । \newline
32. ए॒व प॒शुषु॑ प॒शुष्वे॒वैव प॒शुषु॒ प्रति॒ प्रति॑ प॒शुष्वे॒वैव प॒शुषु॒ प्रति॑ । \newline
33. प॒शुषु॒ प्रति॒ प्रति॑ प॒शुषु॑ प॒शुषु॒ प्रति॑ तिष्ठन्ति तिष्ठन्ति॒ प्रति॑ प॒शुषु॑ प॒शुषु॒ प्रति॑ तिष्ठन्ति । \newline
34. प्रति॑ तिष्ठन्ति तिष्ठन्ति॒ प्रति॒ प्रति॑ तिष्ठ न्त्येकादशरा॒त्र ए॑कादशरा॒त्र स्ति॑ष्ठन्ति॒ प्रति॒ प्रति॑ तिष्ठ न्त्येकादशरा॒त्रः । \newline
35. ति॒ष्ठ॒ न्त्ये॒का॒द॒श॒रा॒त्र ए॑कादशरा॒त्र स्ति॑ष्ठन्ति तिष्ठ न्त्येकादशरा॒त्रो भ॑वति भव त्येकादशरा॒त्र स्ति॑ष्ठन्ति तिष्ठ न्त्येकादशरा॒त्रो भ॑वति । \newline
36. ए॒का॒द॒श॒रा॒त्रो भ॑वति भव त्येकादशरा॒त्र ए॑कादशरा॒त्रो भ॑वति॒ पञ्च॒ पञ्च॑ भव त्येकादशरा॒त्र ए॑कादशरा॒त्रो भ॑वति॒ पञ्च॑ । \newline
37. ए॒का॒द॒श॒रा॒त्र इत्ये॑कादश - रा॒त्रः । \newline
38. भ॒व॒ति॒ पञ्च॒ पञ्च॑ भवति भवति॒ पञ्च॒ वै वै पञ्च॑ भवति भवति॒ पञ्च॒ वै । \newline
39. पञ्च॒ वै वै पञ्च॒ पञ्च॒ वा ऋ॒तव॑ ऋ॒तवो॒ वै पञ्च॒ पञ्च॒ वा ऋ॒तवः॑ । \newline
40. वा ऋ॒तव॑ ऋ॒तवो॒ वै वा ऋ॒तव॑ आर्त॒वा आ᳚र्त॒वा ऋ॒तवो॒ वै वा ऋ॒तव॑ आर्त॒वाः । \newline
41. ऋ॒तव॑ आर्त॒वा आ᳚र्त॒वा ऋ॒तव॑ ऋ॒तव॑ आर्त॒वाः पञ्च॒ पञ्चा᳚र्त॒वा ऋ॒तव॑ ऋ॒तव॑ आर्त॒वाः पञ्च॑ । \newline
42. आ॒र्त॒वाः पञ्च॒ पञ्चा᳚र्त॒वा आ᳚र्त॒वाः पञ्च॒ र्‌तुष् वृ॒तुषु॒ पञ्चा᳚र्त॒वा आ᳚र्त॒वाः पञ्च॒ र्‌तुषु॑ । \newline
43. पञ्च॒ र्‌तु ष्वृ॒तुषु॒ पञ्च॒ पञ्च॒ र्‌तुष्वे॒वैव र्‌तुषु॒ पञ्च॒ पञ्च॒ र्‌तुष्वे॒व । \newline
44. ऋ॒तु ष्वे॒वैव र्‌तुष् वृ॒तु ष्वे॒वार्त॒वे ष्वा᳚र्त॒वे ष्वे॒व र्‌तुष् वृ॒तु ष्वे॒वार्त॒वेषु॑ । \newline
45. ए॒वार्त॒वे ष्वा᳚र्त॒ वेष्वे॒ वैवार्त॒वेषु॑ संॅवथ्स॒रे सं॑ॅवथ्स॒र आ᳚र्त॒वे ष्वे॒वै वार्त॒वेषु॑ संॅवथ्स॒रे । \newline
46. आ॒र्त॒वेषु॑ संॅवथ्स॒रे सं॑ॅवथ्स॒र आ᳚र्त॒ वेष्वा᳚र्त॒वेषु॑ संॅवथ्स॒रे प्र॑ति॒ष्ठाय॑ प्रति॒ष्ठाय॑ संॅवथ्स॒र आ᳚र्त॒ वेष्वा᳚र्त॒वेषु॑ संॅवथ्स॒रे प्र॑ति॒ष्ठाय॑ । \newline
47. सं॒ॅव॒थ्स॒रे प्र॑ति॒ष्ठाय॑ प्रति॒ष्ठाय॑ संॅवथ्स॒रे सं॑ॅवथ्स॒रे प्र॑ति॒ष्ठाय॑ प्र॒जाम् प्र॒जाम् प्र॑ति॒ष्ठाय॑ संॅवथ्स॒रे सं॑ॅवथ्स॒रे प्र॑ति॒ष्ठाय॑ प्र॒जाम् । \newline
48. सं॒ॅव॒थ्स॒र इति॑ सं - व॒थ्स॒रे । \newline
49. प्र॒ति॒ष्ठाय॑ प्र॒जाम् प्र॒जाम् प्र॑ति॒ष्ठाय॑ प्रति॒ष्ठाय॑ प्र॒जा मवाव॑ प्र॒जाम् प्र॑ति॒ष्ठाय॑ प्रति॒ष्ठाय॑ प्र॒जा मव॑ । \newline
50. प्र॒ति॒ष्ठायेति॑ प्रति - स्थाय॑ । \newline
51. प्र॒जा मवाव॑ प्र॒जाम् प्र॒जा मव॑ रुन्धते रुन्ध॒ते ऽव॑ प्र॒जाम् प्र॒जा मव॑ रुन्धते । \newline
52. प्र॒जामिति॑ प्र - जाम् । \newline
53. अव॑ रुन्धते रुन्ध॒ते ऽवाव॑ रुन्धते ऽतिरा॒त्रा व॑तिरा॒त्रौ रु॑न्ध॒ते ऽवाव॑ रुन्धते ऽतिरा॒त्रौ । \newline
54. रु॒न्ध॒ते॒ ऽति॒रा॒त्रा व॑तिरा॒त्रौ रु॑न्धते रुन्धते ऽतिरा॒त्रा व॒भितो॒ ऽभितो॑ ऽतिरा॒त्रौ रु॑न्धते रुन्धते ऽतिरा॒त्रा व॒भितः॑ । \newline
55. अ॒ति॒रा॒त्रा व॒भितो॒ ऽभितो॑ ऽतिरा॒त्रा व॑तिरा॒त्रा व॒भितो॑ भवतो भवतो॒ ऽभितो॑ ऽतिरा॒त्रा व॑तिरा॒त्रा व॒भितो॑ भवतः । \newline
56. अ॒ति॒रा॒त्रावित्य॑ति - रा॒त्रौ । \newline
57. अ॒भितो॑ भवतो भवतो॒ ऽभितो॒ ऽभितो॑ भवतः प्र॒जायै᳚ प्र॒जायै॑ भवतो॒ ऽभितो॒ ऽभितो॑ भवतः प्र॒जायै᳚ । \newline
58. भ॒व॒तः॒ प्र॒जायै᳚ प्र॒जायै॑ भवतो भवतः प्र॒जायै॒ परि॑गृहीत्यै॒ परि॑गृहीत्यै प्र॒जायै॑ भवतो भवतः प्र॒जायै॒ परि॑गृहीत्यै । \newline
59. प्र॒जायै॒ परि॑गृहीत्यै॒ परि॑गृहीत्यै प्र॒जायै᳚ प्र॒जायै॒ परि॑गृहीत्यै । \newline
60. प्र॒जाया॒ इति॑ प्र - जायै᳚ । \newline
61. परि॑गृहीत्या॒ इति॒ परि॑ - गृ॒ही॒त्यै॒ । \newline
\pagebreak
\markright{ TS 7.2.7.1  \hfill https://www.vedavms.in \hfill}

\section{ TS 7.2.7.1 }

\textbf{TS 7.2.7.1 } \newline
\textbf{Samhita Paata} \newline

ऐ॒न्द्र॒वा॒य॒वाग्रा᳚न् गृह्णीया॒द्यः का॒मये॑त यथा पू॒र्वं प्र॒जाः क॑ल्पेर॒न्निति॑ य॒ज्ञ्स्य॒ वै क्लृप्ति॒मनु॑ प्र॒जाः क॑ल्पन्ते य॒ज्ञ्स्या-क्लृ॑प्ति॒मनु॒ न क॑ल्पन्ते यथा पू॒र्वमे॒व प्र॒जाः क॑ल्पयति॒ न ज्यायाꣳ॑सं॒ कनी॑या॒नति॑ क्रामत्यैन्द्रवाय॒वाग्रा᳚न् गृह्णीयादामया॒विनः॑ प्रा॒णेन॒ वा ए॒ष व्यृ॑द्ध्यते॒ यस्या॒ऽऽ*मय॑ति प्रा॒ण ऐ᳚न्द्रवाय॒वः प्रा॒णेनै॒वैनꣳ॒॒ सम॑र्द्धयति मैत्रावरु॒णाग्रा᳚न् गृह्णीर॒न्॒ येषां᳚ दीक्षि॒तानां᳚ प्र॒मीये॑त-[  ] \newline

\textbf{Pada Paata} \newline

ऐ॒न्द्र॒वा॒य॒वाग्रा॒नित्यै᳚न्द्रवाय॒व - अ॒ग्रा॒न् । गृ॒ह्णी॒या॒त् । यः । का॒मये॑त । य॒था॒पू॒र्वमिति॑ यथा - पू॒र्वम् । प्र॒जा इति॑ प्र - जाः । क॒ल्पे॒र॒न्न् । इति॑ । य॒ज्ञ्स्य॑ । वै । क्लृप्ति᳚म् । अन्विति॑ । प्र॒जा इति॑ प्र - जाः । क॒ल्प॒न्ते॒ । य॒ज्ञ्स्य॑ । अक्लृ॑प्तिम् । अन्विति॑ । न । क॒ल्प॒न्ते॒ । य॒था॒पू॒र्वमिति॑ यथा - पू॒र्वम् । ए॒व । प्र॒जा इति॑ प्र - जाः । क॒ल्प॒य॒ति॒ । न । ज्यायाꣳ॑सम् । कनी॑यान् । अतीति॑ । क्रा॒म॒ति॒ । ऐ॒न्द्र॒वा॒य॒वाग्रा॒नित्यै᳚न्द्रवाय॒व - अ॒ग्रा॒न् । गृ॒ह्णी॒या॒त् । आ॒म॒या॒विनः॑ । प्रा॒णेनेति॑ प्र - अ॒नेन॑ । वै । ए॒षः । वीति॑ । ऋ॒द्ध्य॒ते॒ । यस्य॑ । आ॒मय॑ति । प्रा॒ण इति॑ प्र - अ॒नः । ऐ॒न्द्र॒वा॒य॒व इत्यै᳚न्द्र - वा॒य॒वः । प्रा॒णेनेति॑ प्र - अ॒नेन॑ । ए॒व । ए॒न॒म् । समिति॑ । अ॒द्‌र्ध॒य॒ति॒ । मै॒त्रा॒व॒रु॒णाग्रा॒निति॑ मैत्रावरु॒ण - अ॒ग्रा॒न् । गृ॒ह्णी॒र॒न्न् । येषा᳚म् । दी॒क्षि॒ताना᳚म् । प्र॒मीये॒तेति॑ प्र - मीये॑त ।  \newline


\textbf{Krama Paata} \newline

ऐ॒न्द्र॒वा॒य॒वाग्रा᳚न् गृह्णीयात् । ऐ॒न्द्र॒वा॒य॒वाग्रा॒नित्यै᳚न्द्रवाय॒व - अ॒ग्रा॒न्॒ । गृ॒ह्णी॒या॒द् यः । यः का॒मये॑त । का॒मये॑त यथापू॒र्वम् । य॒था॒पू॒र्वम् प्र॒जाः । य॒था॒पू॒र्वमिति॑ यथा - पू॒र्वम् । प्र॒जाः क॑ल्पेरन्न् । प्र॒जा इति॑ प्र - जाः । क॒ल्पे॒र॒न्निति॑ । इति॑ य॒ज्ञ्स्य॑ । य॒ज्ञ्स्य॒ वै । वै क्लृप्ति᳚म् । क्लृप्ति॒मनु॑ । अनु॑ प्र॒जाः । प्र॒जाः क॑ल्पन्ते । प्र॒जा इति॑ प्र - जाः । क॒ल्प॒न्ते॒ य॒ज्ञ्स्य॑ । य॒ज्ञ्स्याक्लृ॑प्तिम् । अक्लृ॑प्ति॒मनु॑ । अनु॒ न । न क॑ल्पन्ते । क॒ल्प॒न्ते॒ य॒था॒पू॒र्वम् । य॒था॒पू॒र्वमे॒व । य॒था॒पू॒र्वमिति॑ यथा - पू॒र्वम् । ए॒व प्र॒जाः । प्र॒जाः क॑ल्पयति । प्र॒जा इति॑ प्र - जाः । क॒ल्प॒य॒ति॒ न । न ज्यायाꣳ॑सम् । ज्यायाꣳ॑स॒म् कनी॑यान् । कनी॑या॒नति॑ । अति॑ क्रामति । क्रा॒म॒त्यै॒न्द्र॒वा॒य॒वाग्रान्॑ । ऐ॒न्द्र॒वा॒य॒वाग्रा᳚न् गृह्णीयात् । ऐ॒न्द्र॒वा॒य॒वाग्रा॒नित्यै᳚न्द्रवाय॒व - अ॒ग्रा॒न्॒ । गृ॒ह्णि॒या॒दा॒म॒या॒विनः॑ । आ॒म॒या॒विनः॑ प्रा॒णेन॑ । प्रा॒णेन॒ वै । प्रा॒णेनेति॑ प्र - अ॒नेन॑ । वा ए॒षः । ए॒ष वि । व्यृ॑द्ध्यते । ऋ॒द्ध्य॒ते॒ यस्य॑ । यस्या॒मय॑ति । आ॒मय॑ति प्रा॒णः । प्रा॒ण ऐ᳚न्द्रवाय॒वः । प्रा॒ण इति॑ प्र - अ॒नः । ऐ॒न्द्र॒वा॒य॒वः प्रा॒णेन॑ । ऐ॒न्द्र॒वा॒य॒व इत्यै᳚न्द्र - वा॒य॒वः । प्रा॒णेनै॒व । प्रा॒णेनेति॑ प्र - अ॒नेन॑ । ए॒वैन᳚म् । ए॒नꣳ॒॒ सम् । सम॑र्द्धयति । अ॒र्द्ध॒य॒ति॒ मै॒त्रा॒व॒रु॒णाग्रान्॑ । मै॒त्रा॒व॒रु॒णाग्रा᳚न् गृह्णीरन्न् । मै॒त्रा॒व॒रु॒णाग्रा॒निति॑ मैत्रावरु॒ण - अ॒ग्रा॒न्॒ । गृ॒ह्णी॒र॒न्॒. येषा᳚म् । येषा᳚म् दीक्षि॒ताना᳚म् । दी॒क्षि॒ताना᳚म् प्र॒मीये॑त । प्र॒मीये॑त प्राणापा॒नाभ्या᳚म् । प्र॒मीये॒तेति॑ प्र - मीये॑त \newline

\textbf{Jatai Paata} \newline

1. ऐ॒न्द्र॒वा॒य॒वाग्रा᳚न् गृह्णीयाद् गृह्णीया दैन्द्रवाय॒वाग्रा॑ नैन्द्रवाय॒वाग्रा᳚न् गृह्णीयात् । \newline
2. ऐ॒न्द्र॒वा॒य॒वाग्रा॒नित्यै᳚न्द्रवाय॒व - अ॒ग्रा॒न् । \newline
3. गृ॒ह्णी॒या॒द् यो यो गृ॑ह्णीयाद् गृह्णीया॒द् यः । \newline
4. यः का॒मये॑त का॒मये॑त॒ यो यः का॒मये॑त । \newline
5. का॒मये॑त यथापू॒र्वं ॅय॑थापू॒र्वम् का॒मये॑त का॒मये॑त यथापू॒र्वम् । \newline
6. य॒था॒पू॒र्वम् प्र॒जाः प्र॒जा य॑थापू॒र्वं ॅय॑थापू॒र्वम् प्र॒जाः । \newline
7. य॒था॒पू॒र्वमिति॑ यथा - पू॒र्वम् । \newline
8. प्र॒जाः क॑ल्पेरन् कल्पेरन् प्र॒जाः प्र॒जाः क॑ल्पेरन्न् । \newline
9. प्र॒जा इति॑ प्र - जाः । \newline
10. क॒ल्पे॒र॒न् नितीति॑ कल्पेरन् कल्पेर॒न् निति॑ । \newline
11. इति॑ य॒ज्ञ्स्य॑ य॒ज्ञ् स्येतीति॑ य॒ज्ञ्स्य॑ । \newline
12. य॒ज्ञ्स्य॒ वै वै य॒ज्ञ्स्य॑ य॒ज्ञ्स्य॒ वै । \newline
13. वै क्लृप्ति॒म् क्लृप्तिं॒ ॅवै वै क्लृप्ति᳚म् । \newline
14. क्लृप्ति॒ मन्वनु॒ क्लृप्ति॒म् क्लृप्ति॒ मनु॑ । \newline
15. अनु॑ प्र॒जाः प्र॒जा अन्वनु॑ प्र॒जाः । \newline
16. प्र॒जाः क॑ल्पन्ते कल्पन्ते प्र॒जाः प्र॒जाः क॑ल्पन्ते । \newline
17. प्र॒जा इति॑ प्र - जाः । \newline
18. क॒ल्प॒न्ते॒ य॒ज्ञ्स्य॑ य॒ज्ञ्स्य॑ कल्पन्ते कल्पन्ते य॒ज्ञ्स्य॑ । \newline
19. य॒ज्ञ्स्या क्लृ॑प्ति॒ मक्लृ॑प्तिं ॅय॒ज्ञ्स्य॑ य॒ज्ञ्स्या क्लृ॑प्तिम् । \newline
20. अक्लृ॑प्ति॒ मन्वन्व क्लृ॑प्ति॒ मक्लृ॑प्ति॒ मनु॑ । \newline
21. अनु॒ न नान्वनु॒ न । \newline
22. न क॑ल्पन्ते कल्पन्ते॒ न न क॑ल्पन्ते । \newline
23. क॒ल्प॒न्ते॒ य॒था॒पू॒र्वं ॅय॑थापू॒र्वम् क॑ल्पन्ते कल्पन्ते यथापू॒र्वम् । \newline
24. य॒था॒पू॒र्व मे॒वैव य॑थापू॒र्वं ॅय॑थापू॒र्व मे॒व । \newline
25. य॒था॒पू॒र्वमिति॑ यथा - पू॒र्वम् । \newline
26. ए॒व प्र॒जाः प्र॒जा ए॒वैव प्र॒जाः । \newline
27. प्र॒जाः क॑ल्पयति कल्पयति प्र॒जाः प्र॒जाः क॑ल्पयति । \newline
28. प्र॒जा इति॑ प्र - जाः । \newline
29. क॒ल्प॒य॒ति॒ न न क॑ल्पयति कल्पयति॒ न । \newline
30. न ज्यायाꣳ॑स॒म् ज्यायाꣳ॑स॒न् न न ज्यायाꣳ॑सम् । \newline
31. ज्यायाꣳ॑स॒म् कनी॑या॒न् कनी॑या॒न् ज्यायाꣳ॑स॒म् ज्यायाꣳ॑स॒म् कनी॑यान् । \newline
32. कनी॑या॒ नत्यति॒ कनी॑या॒न् कनी॑या॒ नति॑ । \newline
33. अति॑ क्रामति क्राम॒ त्यत्यति॑ क्रामति । \newline
34. क्रा॒म॒ त्यै॒न्द्र॒वा॒य॒वाग्रा॑ नैन्द्रवाय॒वाग्रा᳚न् क्रामति क्राम त्यैन्द्रवाय॒वाग्रान्॑ । \newline
35. ऐ॒न्द्र॒वा॒य॒वाग्रा᳚न् गृह्णीयाद् गृह्णीया दैन्द्रवाय॒वाग्रा॑ नैन्द्रवाय॒वाग्रा᳚न् गृह्णीयात् । \newline
36. ऐ॒न्द्र॒वा॒य॒वाग्रा॒नित्यै᳚न्द्रवाय॒व - अ॒ग्रा॒न् । \newline
37. गृ॒ह्णी॒या॒ दा॒म॒या॒विन॑ आमया॒विनो॑ गृह्णीयाद् गृह्णीया दामया॒विनः॑ । \newline
38. आ॒म॒या॒विनः॑ प्रा॒णेन॑ प्रा॒णेना॑ मया॒विन॑ आमया॒विनः॑ प्रा॒णेन॑ । \newline
39. प्रा॒णेन॒ वै वै प्रा॒णेन॑ प्रा॒णेन॒ वै । \newline
40. प्रा॒णेनेति॑ प्र - अ॒नेन॑ । \newline
41. वा ए॒ष ए॒ष वै वा ए॒षः । \newline
42. ए॒ष वि व्ये॑ष ए॒ष वि । \newline
43. व्यृ॑द्ध्यत ऋद्ध्यते॒ वि व्यृ॑द्ध्यते । \newline
44. ऋ॒द्ध्य॒ते॒ यस्य॒ यस्य॑ र्‌द्ध्यत ऋद्ध्यते॒ यस्य॑ । \newline
45. यस्या॒मय॑ त्या॒मय॑ति॒ यस्य॒ यस्या॒मय॑ति । \newline
46. आ॒मय॑ति प्रा॒णः प्रा॒ण आ॒मय॑ त्या॒मय॑ति प्रा॒णः । \newline
47. प्रा॒ण ऐ᳚न्द्रवाय॒व ऐ᳚न्द्रवाय॒वः प्रा॒णः प्रा॒ण ऐ᳚न्द्रवाय॒वः । \newline
48. प्रा॒ण इति॑ प्र - अ॒नः । \newline
49. ऐ॒न्द्र॒वा॒य॒वः प्रा॒णेन॑ प्रा॒णे नै᳚न्द्रवाय॒व ऐ᳚न्द्रवाय॒वः प्रा॒णेन॑ । \newline
50. ऐ॒न्द्र॒वा॒य॒व इत्यै᳚न्द्र - वा॒य॒वः । \newline
51. प्रा॒णे नै॒वैव प्रा॒णेन॑ प्रा॒णे नै॒व । \newline
52. प्रा॒णेनेति॑ प्र - अ॒नेन॑ । \newline
53. ए॒वैन॑ मेन मे॒वै वैन᳚म् । \newline
54. ए॒नꣳ॒॒ सꣳ स मे॑न मेनꣳ॒॒ सम् । \newline
55. स म॑र्द्धय त्यर्द्धयति॒ सꣳ स म॑र्द्धयति । \newline
56. अ॒र्द्ध॒य॒ति॒ मै॒त्रा॒व॒रु॒णाग्रा᳚न् मैत्रावरु॒णाग्रा॑ नर्द्धय त्यर्द्धयति मैत्रावरु॒णाग्रान्॑ । \newline
57. मै॒त्रा॒व॒रु॒णाग्रा᳚न् गृह्णीरन् गृह्णीरन् मैत्रावरु॒णाग्रा᳚न् मैत्रावरु॒णाग्रा᳚न् गृह्णीरन्न् । \newline
58. मै॒त्रा॒व॒रु॒णाग्रा॒निति॑ मैत्रावरु॒ण - अ॒ग्रा॒न् । \newline
59. गृ॒ह्णी॒र॒न्॒. येषां॒ ॅयेषा᳚म् गृह्णीरन् गृह्णीर॒न्॒. येषा᳚म् । \newline
60. येषा᳚म् दीक्षि॒ताना᳚म् दीक्षि॒तानां॒ ॅयेषां॒ ॅयेषा᳚म् दीक्षि॒ताना᳚म् । \newline
61. दी॒क्षि॒ताना᳚म् प्र॒मीये॑त प्र॒मीये॑त दीक्षि॒ताना᳚म् दीक्षि॒ताना᳚म् प्र॒मीये॑त । \newline
62. प्र॒मीये॑त प्राणापा॒नाभ्या᳚म् प्राणापा॒नाभ्या᳚म् प्र॒मीये॑त प्र॒मीये॑त प्राणापा॒नाभ्या᳚म् । \newline
63. प्र॒मीये॒तेति॑ प्र - मीये॑त । \newline

\textbf{Ghana Paata } \newline

1. ऐ॒न्द्र॒वा॒य॒वाग्रा᳚न् गृह्णीयाद् गृह्णीया दैन्द्रवाय॒वाग्रा॑ नैन्द्रवाय॒वाग्रा᳚न् गृह्णीया॒द् यो यो गृ॑ह्णीया दैन्द्रवाय॒वाग्रा॑ नैन्द्रवाय॒वाग्रा᳚न् गृह्णीया॒द् यः । \newline
2. ऐ॒न्द्र॒वा॒य॒वाग्रा॒नित्यै᳚न्द्रवाय॒व - अ॒ग्रा॒न् । \newline
3. गृ॒ह्णी॒या॒द् यो यो गृ॑ह्णीयाद् गृह्णीया॒द् यः का॒मये॑त का॒मये॑त॒ यो गृ॑ह्णीयाद् गृह्णीया॒द् यः का॒मये॑त । \newline
4. यः का॒मये॑त का॒मये॑त॒ यो यः का॒मये॑त यथापू॒र्वं ॅय॑थापू॒र्वम् का॒मये॑त॒ यो यः का॒मये॑त यथापू॒र्वम् । \newline
5. का॒मये॑त यथापू॒र्वं ॅय॑थापू॒र्वम् का॒मये॑त का॒मये॑त यथापू॒र्वम् प्र॒जाः प्र॒जा य॑थापू॒र्वम् का॒मये॑त का॒मये॑त यथापू॒र्वम् प्र॒जाः । \newline
6. य॒था॒पू॒र्वम् प्र॒जाः प्र॒जा य॑थापू॒र्वं ॅय॑थापू॒र्वम् प्र॒जाः क॑ल्पेरन् कल्पेरन् प्र॒जा य॑थापू॒र्वं ॅय॑थापू॒र्वम् प्र॒जाः क॑ल्पेरन्न् । \newline
7. य॒था॒पू॒र्वमिति॑ यथा - पू॒र्वम् । \newline
8. प्र॒जाः क॑ल्पेरन् कल्पेरन् प्र॒जाः प्र॒जाः क॑ल्पेर॒न् नितीति॑ कल्पेरन् प्र॒जाः प्र॒जाः क॑ल्पेर॒न् निति॑ । \newline
9. प्र॒जा इति॑ प्र - जाः । \newline
10. क॒ल्पे॒र॒न् नितीति॑ कल्पेरन् कल्पेर॒न् निति॑ य॒ज्ञ्स्य॑ य॒ज्ञ्स्येति॑ कल्पेरन् कल्पेर॒न् निति॑ य॒ज्ञ्स्य॑ । \newline
11. इति॑ य॒ज्ञ्स्य॑ य॒ज्ञ्स्येतीति॑ य॒ज्ञ्स्य॒ वै वै य॒ज्ञ्स्येतीति॑ य॒ज्ञ्स्य॒ वै । \newline
12. य॒ज्ञ्स्य॒ वै वै य॒ज्ञ्स्य॑ य॒ज्ञ्स्य॒ वै क्लृप्ति॒म् क्लृप्तिं॒ ॅवै य॒ज्ञ्स्य॑ य॒ज्ञ्स्य॒ वै क्लृप्ति᳚म् । \newline
13. वै क्लृप्ति॒म् क्लृप्तिं॒ ॅवै वै क्लृप्ति॒ मन्वनु॒ क्लृप्तिं॒ ॅवै वै क्लृप्ति॒ मनु॑ । \newline
14. क्लृप्ति॒ मन्वनु॒ क्लृप्ति॒म् क्लृप्ति॒ मनु॑ प्र॒जाः प्र॒जा अनु॒ क्लृप्ति॒म् क्लृप्ति॒ मनु॑ प्र॒जाः । \newline
15. अनु॑ प्र॒जाः प्र॒जा अन्वनु॑ प्र॒जाः क॑ल्पन्ते कल्पन्ते प्र॒जा अन्वनु॑ प्र॒जाः क॑ल्पन्ते । \newline
16. प्र॒जाः क॑ल्पन्ते कल्पन्ते प्र॒जाः प्र॒जाः क॑ल्पन्ते य॒ज्ञ्स्य॑ य॒ज्ञ्स्य॑ कल्पन्ते प्र॒जाः प्र॒जाः क॑ल्पन्ते य॒ज्ञ्स्य॑ । \newline
17. प्र॒जा इति॑ प्र - जाः । \newline
18. क॒ल्प॒न्ते॒ य॒ज्ञ्स्य॑ य॒ज्ञ्स्य॑ कल्पन्ते कल्पन्ते य॒ज्ञ्स्या क्लृ॑प्ति॒ मक्लृ॑प्तिं ॅय॒ज्ञ्स्य॑ कल्पन्ते कल्पन्ते य॒ज्ञ्स्या क्लृ॑प्तिम् । \newline
19. य॒ज्ञ्स्या क्लृ॑प्ति॒ मक्लृ॑प्तिं ॅय॒ज्ञ्स्य॑ य॒ज्ञ्स्या क्लृ॑प्ति॒ मन् वन् वक्लृ॑प्तिं ॅय॒ज्ञ्स्य॑ य॒ज्ञ्स्या क्लृ॑प्ति॒ मनु॑ । \newline
20. अक्लृ॑प्ति॒ मन् वन् वक्लृ॑प्ति॒ मक्लृ॑प्ति॒ मनु॒ न नान् वक्लृ॑प्ति॒ मक्लृ॑प्ति॒ मनु॒ न । \newline
21. अनु॒ न नान् वनु॒ न क॑ल्पन्ते कल्पन्ते॒ नान् वनु॒ न क॑ल्पन्ते । \newline
22. न क॑ल्पन्ते कल्पन्ते॒ न न क॑ल्पन्ते यथापू॒र्वं ॅय॑थापू॒र्वम् क॑ल्पन्ते॒ न न क॑ल्पन्ते यथापू॒र्वम् । \newline
23. क॒ल्प॒न्ते॒ य॒था॒पू॒र्वं ॅय॑थापू॒र्वम् क॑ल्पन्ते कल्पन्ते यथापू॒र्व मे॒वैव य॑थापू॒र्वम् क॑ल्पन्ते कल्पन्ते यथापू॒र्व मे॒व । \newline
24. य॒था॒पू॒र्व मे॒वैव य॑थापू॒र्वं ॅय॑थापू॒र्व मे॒व प्र॒जाः प्र॒जा ए॒व य॑थापू॒र्वं ॅय॑थापू॒र्व मे॒व प्र॒जाः । \newline
25. य॒था॒पू॒र्वमिति॑ यथा - पू॒र्वम् । \newline
26. ए॒व प्र॒जाः प्र॒जा ए॒वैव प्र॒जाः क॑ल्पयति कल्पयति प्र॒जा ए॒वैव प्र॒जाः क॑ल्पयति । \newline
27. प्र॒जाः क॑ल्पयति कल्पयति प्र॒जाः प्र॒जाः क॑ल्पयति॒ न न क॑ल्पयति प्र॒जाः प्र॒जाः क॑ल्पयति॒ न । \newline
28. प्र॒जा इति॑ प्र - जाः । \newline
29. क॒ल्प॒य॒ति॒ न न क॑ल्पयति कल्पयति॒ न ज्यायाꣳ॑स॒म् ज्यायाꣳ॑स॒न् न क॑ल्पयति कल्पयति॒ न ज्यायाꣳ॑सम् । \newline
30. न ज्यायाꣳ॑स॒म् ज्यायाꣳ॑स॒न् न न ज्यायाꣳ॑स॒म् कनी॑या॒न् कनी॑या॒न् ज्यायाꣳ॑स॒न् न न ज्यायाꣳ॑स॒म् कनी॑यान् । \newline
31. ज्यायाꣳ॑स॒म् कनी॑या॒न् कनी॑या॒न् ज्यायाꣳ॑स॒म् ज्यायाꣳ॑स॒म् कनी॑या॒ नत्यति॒ कनी॑या॒न् ज्यायाꣳ॑स॒म् ज्यायाꣳ॑स॒म् कनी॑या॒ नति॑ । \newline
32. कनी॑या॒ नत्यति॒ कनी॑या॒न् कनी॑या॒ नति॑ क्रामति क्राम॒ त्यति॒ कनी॑या॒न् कनी॑या॒ नति॑ क्रामति । \newline
33. अति॑ क्रामति क्राम॒ त्यत्यति॑ क्राम त्यैन्द्रवाय॒वाग्रा॑ नैन्द्रवाय॒वाग्रा᳚न् क्राम॒ त्यत्यति॑ क्राम त्यैन्द्रवाय॒वाग्रान्॑ । \newline
34. क्रा॒म॒ त्यै॒न्द्र॒वा॒य॒वाग्रा॑ नैन्द्रवाय॒वाग्रा᳚न् क्रामति क्राम त्यैन्द्रवाय॒वाग्रा᳚न् गृह्णीयाद् गृह्णीया दैन्द्रवाय॒वाग्रा᳚न् क्रामति क्राम त्यैन्द्रवाय॒वाग्रा᳚न् गृह्णीयात् । \newline
35. ऐ॒न्द्र॒वा॒य॒वाग्रा᳚न् गृह्णीयाद् गृह्णीया दैन्द्रवाय॒वाग्रा॑ नैन्द्रवाय॒वाग्रा᳚न् गृह्णीया दामया॒विन॑ आमया॒विनो॑ गृह्णीया दैन्द्रवाय॒वाग्रा॑ नैन्द्रवाय॒वाग्रा᳚न् गृह्णीया दामया॒विनः॑ । \newline
36. ऐ॒न्द्र॒वा॒य॒वाग्रा॒नित्यै᳚न्द्रवाय॒व - अ॒ग्रा॒न् । \newline
37. गृ॒ह्णी॒या॒ दा॒म॒या॒विन॑ आमया॒विनो॑ गृह्णीयाद् गृह्णीया दामया॒विनः॑ प्रा॒णेन॑ प्रा॒णेना॑ मया॒विनो॑ गृह्णीयाद् गृह्णीया दामया॒विनः॑ प्रा॒णेन॑ । \newline
38. आ॒म॒या॒विनः॑ प्रा॒णेन॑ प्रा॒णेना॑ मया॒विन॑ आमया॒विनः॑ प्रा॒णेन॒ वै वै प्रा॒णेना॑ मया॒विन॑ आमया॒विनः॑ प्रा॒णेन॒ वै । \newline
39. प्रा॒णेन॒ वै वै प्रा॒णेन॑ प्रा॒णेन॒ वा ए॒ष ए॒ष वै प्रा॒णेन॑ प्रा॒णेन॒ वा ए॒षः । \newline
40. प्रा॒णेनेति॑ प्र - अ॒नेन॑ । \newline
41. वा ए॒ष ए॒ष वै वा ए॒ष वि व्ये॑ष वै वा ए॒ष वि । \newline
42. ए॒ष वि व्ये॑ष ए॒ष व्यृ॑द्ध्यत ऋद्ध्यते॒ व्ये॑ष ए॒ष व्यृ॑द्ध्यते । \newline
43. व्यृ॑द्ध्यत ऋद्ध्यते॒ वि व्यृ॑द्ध्यते॒ यस्य॒ यस्य॑ र्‌द्ध्यते॒ वि व्यृ॑द्ध्यते॒ यस्य॑ । \newline
44. ऋ॒द्ध्य॒ते॒ यस्य॒ यस्य॑ र्द्ध्यत ऋद्ध्यते॒ यस्या॒मय॑ त्या॒मय॑ति॒ यस्य॑ र्‌द्ध्यत ऋद्ध्यते॒ यस्या॒मय॑ति । \newline
45. यस्या॒मय॑ त्या॒मय॑ति॒ यस्य॒ यस्या॒मय॑ति प्रा॒णः प्रा॒ण आ॒मय॑ति॒ यस्य॒ यस्या॒मय॑ति प्रा॒णः । \newline
46. आ॒मय॑ति प्रा॒णः प्रा॒ण आ॒मय॑ त्या॒मय॑ति प्रा॒ण ऐ᳚न्द्रवाय॒व ऐ᳚न्द्रवाय॒वः प्रा॒ण आ॒मय॑ त्या॒मय॑ति प्रा॒ण ऐ᳚न्द्रवाय॒वः । \newline
47. प्रा॒ण ऐ᳚न्द्रवाय॒व ऐ᳚न्द्रवाय॒वः प्रा॒णः प्रा॒ण ऐ᳚न्द्रवाय॒वः प्रा॒णेन॑ प्रा॒णे नै᳚न्द्रवाय॒वः प्रा॒णः प्रा॒ण ऐ᳚न्द्रवाय॒वः प्रा॒णेन॑ । \newline
48. प्रा॒ण इति॑ प्र - अ॒नः । \newline
49. ऐ॒न्द्र॒वा॒य॒वः प्रा॒णेन॑ प्रा॒णे नै᳚न्द्रवाय॒व ऐ᳚न्द्रवाय॒वः प्रा॒णेनै॒वैव प्रा॒णे नै᳚न्द्रवाय॒व ऐ᳚न्द्रवाय॒वः प्रा॒णे नै॒व । \newline
50. ऐ॒न्द्र॒वा॒य॒व इत्यै᳚न्द्र - वा॒य॒वः । \newline
51. प्रा॒णे नै॒वैव प्रा॒णेन॑ प्रा॒णे नै॒वैन॑ मेन मे॒व प्रा॒णेन॑ प्रा॒णे नै॒वैन᳚म् । \newline
52. प्रा॒णेनेति॑ प्र - अ॒नेन॑ । \newline
53. ए॒वैन॑ मेन मे॒वै वैनꣳ॒॒ सꣳ स मे॑न मे॒वै वैनꣳ॒॒ सम् । \newline
54. ए॒नꣳ॒॒ सꣳ स मे॑न मेनꣳ॒॒ स म॑र्द्धय त्यर्द्धयति॒ स मे॑न मेनꣳ॒॒ स म॑र्द्धयति । \newline
55. स म॑र्द्धय त्यर्द्धयति॒ सꣳ स म॑र्द्धयति मैत्रावरु॒णाग्रा᳚न् मैत्रावरु॒णाग्रा॑ नर्द्धयति॒ सꣳ स म॑र्द्धयति मैत्रावरु॒णाग्रान्॑ । \newline
56. अ॒र्द्ध॒य॒ति॒ मै॒त्रा॒व॒रु॒णाग्रा᳚न् मैत्रावरु॒णाग्रा॑ नर्द्धय त्यर्द्धयति मैत्रावरु॒णाग्रा᳚न् गृह्णीरन् गृह्णीरन् मैत्रावरु॒णाग्रा॑ नर्द्धय त्यर्द्धयति मैत्रावरु॒णाग्रा᳚न् गृह्णीरन्न् । \newline
57. मै॒त्रा॒व॒रु॒णाग्रा᳚न् गृह्णीरन् गृह्णीरन् मैत्रावरु॒णाग्रा᳚न् मैत्रावरु॒णाग्रा᳚न् गृह्णीर॒न्॒. येषां॒ ॅयेषा᳚म् गृह्णीरन् मैत्रावरु॒णाग्रा᳚न् मैत्रावरु॒णाग्रा᳚न् गृह्णीर॒न्॒. येषा᳚म् । \newline
58. मै॒त्रा॒व॒रु॒णाग्रा॒निति॑ मैत्रावरु॒ण - अ॒ग्रा॒न् । \newline
59. गृ॒ह्णी॒र॒न्॒. येषां॒ ॅयेषा᳚म् गृह्णीरन् गृह्णीर॒न्॒. येषा᳚म् दीक्षि॒ताना᳚म् दीक्षि॒तानां॒ ॅयेषा᳚म् गृह्णीरन् गृह्णीर॒न्॒. येषा᳚म् दीक्षि॒ताना᳚म् । \newline
60. येषा᳚म् दीक्षि॒ताना᳚म् दीक्षि॒तानां॒ ॅयेषां॒ ॅयेषा᳚म् दीक्षि॒ताना᳚म् प्र॒मीये॑त प्र॒मीये॑त दीक्षि॒तानां॒ ॅयेषां॒ ॅयेषा᳚म् दीक्षि॒ताना᳚म् प्र॒मीये॑त । \newline
61. दी॒क्षि॒ताना᳚म् प्र॒मीये॑त प्र॒मीये॑त दीक्षि॒ताना᳚म् दीक्षि॒ताना᳚म् प्र॒मीये॑त प्राणापा॒नाभ्या᳚म् प्राणापा॒नाभ्या᳚म् प्र॒मीये॑त दीक्षि॒ताना᳚म् दीक्षि॒ताना᳚म् प्र॒मीये॑त प्राणापा॒नाभ्या᳚म् । \newline
62. प्र॒मीये॑त प्राणापा॒नाभ्या᳚म् प्राणापा॒नाभ्या᳚म् प्र॒मीये॑त प्र॒मीये॑त प्राणापा॒नाभ्यां॒ ॅवै वै प्रा॑णापा॒नाभ्या᳚म् प्र॒मीये॑त प्र॒मीये॑त प्राणापा॒नाभ्यां॒ ॅवै । \newline
63. प्र॒मीये॒तेति॑ प्र - मीये॑त । \newline
\pagebreak
\markright{ TS 7.2.7.2  \hfill https://www.vedavms.in \hfill}

\section{ TS 7.2.7.2 }

\textbf{TS 7.2.7.2 } \newline
\textbf{Samhita Paata} \newline

प्राणापा॒नाभ्यां॒ ॅवा ए॒ते व्यृ॑द्ध्यन्ते॒ येषां᳚ दीक्षि॒तानां᳚ प्र॒मीय॑ते प्राणापा॒नौ मि॒त्रावरु॑णौ प्राणापा॒नावे॒व मु॑ख॒तः परि॑ हरन्त आश्वि॒नाग्रा᳚न् गृह्णीता ऽऽ*नुजाव॒रो᳚ऽश्विनौ॒ वै दे॒वाना॑मानुजाव॒रौ प॒श्चेवाग्रं॒ पर्यै॑ता-म॒श्विना॑वे॒तस्य॑ दे॒वता॒ य आ॑नुजाव॒र-स्तावे॒वैन॒मग्रं॒ परि॑ णयतः शु॒क्राग्रा᳚न् गृह्णीत ग॒तश्रीः᳚ प्रति॒ष्ठाका॑मो॒ऽसौ वा आ॑दि॒त्यः शु॒क्र ए॒षोऽन्तोऽन्तं॑ मनु॒ष्यः॑ - [  ] \newline

\textbf{Pada Paata} \newline

प्रा॒णा॒पा॒नाभ्या॒मिति॑ प्राण - अ॒पा॒नाभ्या᳚म् । वै । ए॒ते । वीति॑ । ऋ॒द्ध्य॒न्ते॒ । येषा᳚म् । दी॒क्षि॒ताना᳚म् । प्र॒मीय॑त॒ इति॑ प्र - मीय॑ते । प्रा॒णा॒पा॒नाविति॑ प्राण - अ॒पा॒नौ । मि॒त्रावरु॑णा॒विति॑ मि॒त्रा - वरु॑णौ । प्रा॒णा॒पा॒नाविति॑ प्राण - अ॒पा॒नौ । ए॒व । मु॒ख॒तः । परीति॑ । ह॒र॒न्ते॒ । आ॒श्वि॒नाग्रा॒नित्या᳚श्वि॒न-अ॒ग्रा॒न् । गृ॒ह्णी॒त॒ । आ॒नु॒जा॒व॒र इत्या॑नु-जा॒व॒रः । अ॒श्विनौ᳚ । वै । दे॒वाना᳚म् । आ॒नु॒जा॒व॒रावित्या॑नु - जा॒व॒रौ । प॒श्चा । इ॒व॒ । अग्र᳚म् । परीति॑ । ऐ॒ता॒म् । अ॒श्विनौ᳚ । ए॒तस्य॑ । दे॒वता᳚ । यः । आ॒नु॒जा॒व॒र इत्या॑नु - जा॒व॒रः । तौ । ए॒व । ए॒न॒म् । अग्र᳚म् । परीति॑ । न॒य॒तः॒ । शु॒क्राग्रा॒निति॑ शु॒क्र - अ॒ग्रा॒न् । गृ॒ह्णी॒त॒ । ग॒तश्री॒रिति॑ ग॒त - श्रीः॒ । प्र॒ति॒ष्ठाका॑म॒ इति॑ प्रति॒ष्ठा - का॒मः॒ । अ॒सौ । वै । आ॒दि॒त्यः । शु॒क्रः । ए॒षः । अन्तः॑ । अन्त᳚म् । म॒नु॒ष्यः॑ ।  \newline


\textbf{Krama Paata} \newline

प्रा॒णा॒पा॒नाभ्या॒म् ॅवै । प्रा॒णा॒पा॒नाभ्या॒मिति॑ प्राण - अ॒पा॒नाभ्या᳚म् । वा ए॒ते । ए॒ते वि । व्यृ॑द्ध्यन्ते । ऋ॒द्ध्य॒न्ते॒ येषा᳚म् । येषा᳚म् दीक्षि॒ताना᳚म् । दी॒क्षि॒ताना᳚म् प्र॒मीय॑ते । प्र॒मीय॑ते प्राणापा॒नौ । प्र॒मिय॑त॒ इति॑ प्र - मीय॑ते । प्रा॒णा॒पा॒नौ मि॒त्रावरु॑णौ । प्रा॒णा॒पा॒नाविति॑ प्राण - अ॒पा॒नौ । मि॒त्रावरु॑णौ प्राणापा॒नौ । मि॒त्रावरु॑णा॒विति॑ मि॒त्रा - वरु॑णौ । प्रा॒णा॒पा॒नावे॒व । प्रा॒णा॒पा॒नाविति॑ प्राण - अ॒पा॒नौ । ए॒व मु॑ख॒तः । मु॒ख॒तः परि॑ । परि॑ हरन्ते । ह॒र॒न्त॒ आ॒श्वि॒नाग्रान्॑ । आ॒श्वि॒नाग्रा᳚न् गृह्णीत । आ॒श्वि॒नाग्रा॒नित्या᳚श्वि॒न - अ॒ग्रा॒न्॒ । गृ॒ह्णी॒ता॒नु॒जा॒व॒रः । आ॒नु॒जा॒व॒रो᳚ऽश्विनौ᳚ । आ॒नु॒जा॒व॒र इत्या॑नु - जा॒व॒रः । अ॒श्विनौ॒ वै । वै दे॒वाना᳚म् । दे॒वाना॑मानुजाव॒रौ । आ॒नु॒जा॒व॒रौ प॒श्चा । आ॒नु॒जा॒व॒रावित्या॑नु - जा॒व॒रौ । प॒श्चेव॑ । इ॒वाग्र᳚म् । अग्र॒म् परि॑ । पर्यै॑ताम् । ऐ॒ता॒म॒श्विनौ᳚ । अ॒श्विना॑वे॒तस्य॑ । ए॒तस्य॑ दे॒वता᳚ । दे॒वता॒ यः । य आ॑नुजाव॒रः । आ॒नु॒जा॒व॒रस्तौ । आ॒नु॒जा॒व॒र इत्या॑नु - जा॒व॒रः । तावे॒व । ए॒वैन᳚म् । ए॒न॒मग्र᳚म् । अग्र॒म् परि॑ । परि॑ णयतः । न॒य॒तः॒ शु॒क्राग्रान्॑ । शु॒क्राग्रा᳚न् गृह्णीत । शु॒क्राग्रा॒निति॑ शु॒क्र - अ॒ग्रा॒न्॒ । गृ॒ह्णी॒त॒ ग॒तश्रीः᳚ । ग॒तश्रीः᳚ प्रति॒ष्ठाका॑मः । ग॒तश्री॒रिति॑ ग॒त - श्रीः॒ । प्र॒ति॒ष्ठाका॑मो॒ऽसौ । प्र॒ति॒ष्ठाका॑म॒ इति॑ प्रति॒ष्ठा - का॒मः॒ । अ॒सौ वै । वा आ॑दि॒त्यः । आ॒दि॒त्यः शु॒क्रः । शु॒क्र ए॒षः । ए॒षोऽन्तः॑ । अन्तोऽन्त᳚म् । अन्त॑म् मनु॒ष्यः॑ । म॒नु॒ष्यः॑ श्रि॒यै \newline

\textbf{Jatai Paata} \newline

1. प्रा॒णा॒पा॒नाभ्यां॒ ॅवै वै प्रा॑णापा॒नाभ्या᳚म् प्राणापा॒नाभ्यां॒ ॅवै । \newline
2. प्रा॒णा॒पा॒नाभ्या॒मिति॑ प्राण - अ॒पा॒नाभ्या᳚म् । \newline
3. वा ए॒त ए॒ते वै वा ए॒ते । \newline
4. ए॒ते वि व्ये॑त ए॒ते वि । \newline
5. व्यृ॑द्ध्यन्त ऋद्ध्यन्ते॒ वि व्यृ॑द्ध्यन्ते । \newline
6. ऋ॒द्ध्य॒न्ते॒ येषां॒ ॅयेषा॑ मृद्ध्यन्त ऋद्ध्यन्ते॒ येषा᳚म् । \newline
7. येषा᳚म् दीक्षि॒ताना᳚म् दीक्षि॒तानां॒ ॅयेषां॒ ॅयेषा᳚म् दीक्षि॒ताना᳚म् । \newline
8. दी॒क्षि॒ताना᳚म् प्र॒मीय॑ते प्र॒मीय॑ते दीक्षि॒ताना᳚म् दीक्षि॒ताना᳚म् प्र॒मीय॑ते । \newline
9. प्र॒मीय॑ते प्राणापा॒नौ प्रा॑णापा॒नौ प्र॒मीय॑ते प्र॒मीय॑ते प्राणापा॒नौ । \newline
10. प्र॒मीय॑त॒ इति॑ प्र - मीय॑ते । \newline
11. प्रा॒णा॒पा॒नौ मि॒त्रावरु॑णौ मि॒त्रावरु॑णौ प्राणापा॒नौ प्रा॑णापा॒नौ मि॒त्रावरु॑णौ । \newline
12. प्रा॒णा॒पा॒नाविति॑ प्राण - अ॒पा॒नौ । \newline
13. मि॒त्रावरु॑णौ प्राणापा॒नौ प्रा॑णापा॒नौ मि॒त्रावरु॑णौ मि॒त्रावरु॑णौ प्राणापा॒नौ । \newline
14. मि॒त्रावरु॑णा॒विति॑ मि॒त्रा - वरु॑णौ । \newline
15. प्रा॒णा॒पा॒ना वे॒वैव प्रा॑णापा॒नौ प्रा॑णापा॒ना वे॒व । \newline
16. प्रा॒णा॒पा॒नाविति॑ प्राण - अ॒पा॒नौ । \newline
17. ए॒व मु॑ख॒तो मु॑ख॒त ए॒वैव मु॑ख॒तः । \newline
18. मु॒ख॒तः परि॒ परि॑ मुख॒तो मु॑ख॒तः परि॑ । \newline
19. परि॑ हरन्ते हरन्ते॒ परि॒ परि॑ हरन्ते । \newline
20. ह॒र॒न्त॒ आ॒श्वि॒नाग्रा॑ नाश्वि॒नाग्रान्॑. हरन्ते हरन्त आश्वि॒नाग्रान्॑ । \newline
21. आ॒श्वि॒नाग्रा᳚न् गृह्णीत गृह्णीता श्वि॒नाग्रा॑ नाश्वि॒नाग्रा᳚न् गृह्णीत । \newline
22. आ॒श्वि॒नाग्रा॒नित्या᳚श्वि॒न - अ॒ग्रा॒न् । \newline
23. गृ॒ह्णी॒ता॒ नु॒जा॒व॒र आ॑नुजाव॒रो गृ॑ह्णीत गृह्णीता नुजाव॒रः । \newline
24. आ॒नु॒जा॒व॒रो᳚ ऽश्विना॑ व॒श्विना॑ वानुजाव॒र आ॑नुजाव॒रो᳚ ऽश्विनौ᳚ । \newline
25. आ॒नु॒जा॒व॒र इत्या॑नु - जा॒व॒रः । \newline
26. अ॒श्विनौ॒ वै वा अ॒श्विना॑ व॒श्विनौ॒ वै । \newline
27. वै दे॒वाना᳚म् दे॒वानां॒ ॅवै वै दे॒वाना᳚म् । \newline
28. दे॒वाना॑ मानुजाव॒रा वा॑नुजाव॒रौ दे॒वाना᳚म् दे॒वाना॑ मानुजाव॒रौ । \newline
29. आ॒नु॒जा॒व॒रौ प॒श्चा प॒श्चा ऽऽनु॑जाव॒रा वा॑नुजाव॒रौ प॒श्चा । \newline
30. आ॒नु॒जा॒व॒रावित्या॑नु - जा॒व॒रौ । \newline
31. प॒श्चेवे॑ व प॒श्चा प॒श्चेव॑ । \newline
32. इ॒वाग्र॒ मग्र॑ मिवे॒ वाग्र᳚म् । \newline
33. अग्र॒म् परि॒ पर्यग्र॒ मग्र॒म् परि॑ । \newline
34. पर्यै॑ता मैता॒म् परि॒ पर्यै॑ताम् । \newline
35. ऐ॒ता॒ म॒श्विना॑ व॒श्विना॑ वैता मैता म॒श्विनौ᳚ । \newline
36. अ॒श्विना॑ वे॒त स्यै॒त स्या॒श्विना॑ व॒श्विना॑ वे॒तस्य॑ । \newline
37. ए॒तस्य॑ दे॒वता॑ दे॒व तै॒त स्यै॒तस्य॑ दे॒वता᳚ । \newline
38. दे॒वता॒ यो यो दे॒वता॑ दे॒वता॒ यः । \newline
39. य आ॑नुजाव॒र आ॑नुजाव॒रो यो य आ॑नुजाव॒रः । \newline
40. आ॒नु॒जा॒व॒र स्तौ ता वा॑नुजाव॒र आ॑नुजाव॒र स्तौ । \newline
41. आ॒नु॒जा॒व॒र इत्या॑नु - जा॒व॒रः । \newline
42. ता वे॒वैव तौ ता वे॒व । \newline
43. ए॒वैन॑ मेन मे॒वै वैन᳚म् । \newline
44. ए॒न॒ मग्र॒ मग्र॑ मेन मेन॒ मग्र᳚म् । \newline
45. अग्र॒म् परि॒ पर्यग्र॒ मग्र॒म् परि॑ । \newline
46. परि॑ णयतो नयतः॒ परि॒ परि॑ णयतः । \newline
47. न॒य॒तः॒ शु॒क्राग्रा᳚ञ् छु॒क्राग्रा᳚न् नयतो नयतः शु॒क्राग्रान्॑ । \newline
48. शु॒क्राग्रा᳚न् गृह्णीत गृह्णीत शु॒क्राग्रा᳚ञ् छु॒क्राग्रा᳚न् गृह्णीत । \newline
49. शु॒क्राग्रा॒निति॑ शु॒क्र - अ॒ग्रा॒न् । \newline
50. गृ॒ह्णी॒त॒ ग॒तश्री᳚र् ग॒तश्री᳚र् गृह्णीत गृह्णीत ग॒तश्रीः᳚ । \newline
51. ग॒तश्रीः᳚ प्रति॒ष्ठाका॑मः प्रति॒ष्ठाका॑मो ग॒तश्री᳚र् ग॒तश्रीः᳚ प्रति॒ष्ठाका॑मः । \newline
52. ग॒तश्री॒रिति॑ ग॒त - श्रीः॒ । \newline
53. प्र॒ति॒ष्ठाका॑मो॒ ऽसा व॒सौ प्र॑ति॒ष्ठाका॑मः प्रति॒ष्ठाका॑मो॒ ऽसौ । \newline
54. प्र॒ति॒ष्ठाका॑म॒ इति॑ प्रति॒ष्ठा - का॒मः॒ । \newline
55. अ॒सौ वै वा अ॒सा व॒सौ वै । \newline
56. वा आ॑दि॒त्य आ॑दि॒त्यो वै वा आ॑दि॒त्यः । \newline
57. आ॒दि॒त्यः शु॒क्रः शु॒क्र आ॑दि॒त्य आ॑दि॒त्यः शु॒क्रः । \newline
58. शु॒क्र ए॒ष ए॒ष शु॒क्रः शु॒क्र ए॒षः । \newline
59. ए॒षो ऽन्तो ऽन्त॑ ए॒ष ए॒षो ऽन्तः॑ । \newline
60. अन्तो ऽन्त॒ मन्त॒ मन्तो ऽन्तो ऽन्त᳚म् । \newline
61. अन्त॑म् मनु॒ष्यो॑ मनु॒ष्यो ऽन्त॒ मन्त॑म् मनु॒ष्यः॑ । \newline
62. म॒नु॒ष्यः॑ श्रि॒यै श्रि॒यै म॑नु॒ष्यो॑ मनु॒ष्यः॑ श्रि॒यै । \newline

\textbf{Ghana Paata } \newline

1. प्रा॒णा॒पा॒नाभ्यां॒ ॅवै वै प्रा॑णापा॒नाभ्या᳚म् प्राणापा॒नाभ्यां॒ ॅवा ए॒त ए॒ते वै प्रा॑णापा॒नाभ्या᳚म् प्राणापा॒नाभ्यां॒ ॅवा ए॒ते । \newline
2. प्रा॒णा॒पा॒नाभ्या॒मिति॑ प्राण - अ॒पा॒नाभ्या᳚म् । \newline
3. वा ए॒त ए॒ते वै वा ए॒ते वि व्ये॑ते वै वा ए॒ते वि । \newline
4. ए॒ते वि व्ये॑त ए॒ते व्यृ॑द्ध्यन्त ऋद्ध्यन्ते॒ व्ये॑त ए॒ते व्यृ॑द्ध्यन्ते । \newline
5. व्यृ॑द्ध्यन्त ऋद्ध्यन्ते॒ वि व्यृ॑द्ध्यन्ते॒ येषां॒ ॅयेषा॑ मृद्ध्यन्ते॒ वि व्यृ॑द्ध्यन्ते॒ येषा᳚म् । \newline
6. ऋ॒द्ध्य॒न्ते॒ येषां॒ ॅयेषा॑ मृद्ध्यन्त ऋद्ध्यन्ते॒ येषा᳚म् दीक्षि॒ताना᳚म् दीक्षि॒तानां॒ ॅयेषा॑ मृद्ध्यन्त ऋद्ध्यन्ते॒ येषा᳚म् दीक्षि॒ताना᳚म् । \newline
7. येषा᳚म् दीक्षि॒ताना᳚म् दीक्षि॒तानां॒ ॅयेषां॒ ॅयेषा᳚म् दीक्षि॒ताना᳚म् प्र॒मीय॑ते प्र॒मीय॑ते दीक्षि॒तानां॒ ॅयेषां॒ ॅयेषा᳚म् दीक्षि॒ताना᳚म् प्र॒मीय॑ते । \newline
8. दी॒क्षि॒ताना᳚म् प्र॒मीय॑ते प्र॒मीय॑ते दीक्षि॒ताना᳚म् दीक्षि॒ताना᳚म् प्र॒मीय॑ते प्राणापा॒नौ प्रा॑णापा॒नौ प्र॒मीय॑ते दीक्षि॒ताना᳚म् दीक्षि॒ताना᳚म् प्र॒मीय॑ते प्राणापा॒नौ । \newline
9. प्र॒मीय॑ते प्राणापा॒नौ प्रा॑णापा॒नौ प्र॒मीय॑ते प्र॒मीय॑ते प्राणापा॒नौ मि॒त्रावरु॑णौ मि॒त्रावरु॑णौ प्राणापा॒नौ प्र॒मीय॑ते प्र॒मीय॑ते प्राणापा॒नौ मि॒त्रावरु॑णौ । \newline
10. प्र॒मीय॑त॒ इति॑ प्र - मीय॑ते । \newline
11. प्रा॒णा॒पा॒नौ मि॒त्रावरु॑णौ मि॒त्रावरु॑णौ प्राणापा॒नौ प्रा॑णापा॒नौ मि॒त्रावरु॑णौ प्राणापा॒नौ प्रा॑णापा॒नौ मि॒त्रावरु॑णौ प्राणापा॒नौ प्रा॑णापा॒नौ मि॒त्रावरु॑णौ प्राणापा॒नौ । \newline
12. प्रा॒णा॒पा॒नाविति॑ प्राण - अ॒पा॒नौ । \newline
13. मि॒त्रावरु॑णौ प्राणापा॒नौ प्रा॑णापा॒नौ मि॒त्रावरु॑णौ मि॒त्रावरु॑णौ प्राणापा॒ना वे॒वैव प्रा॑णापा॒नौ मि॒त्रावरु॑णौ मि॒त्रावरु॑णौ प्राणापा॒ना वे॒व । \newline
14. मि॒त्रावरु॑णा॒विति॑ मि॒त्रा - वरु॑णौ । \newline
15. प्रा॒णा॒पा॒ना वे॒वैव प्रा॑णापा॒नौ प्रा॑णापा॒ना वे॒व मु॑ख॒तो मु॑ख॒त ए॒व प्रा॑णापा॒नौ प्रा॑णापा॒ना वे॒व मु॑ख॒तः । \newline
16. प्रा॒णा॒पा॒नाविति॑ प्राण - अ॒पा॒नौ । \newline
17. ए॒व मु॑ख॒तो मु॑ख॒त ए॒वैव मु॑ख॒तः परि॒ परि॑ मुख॒त ए॒वैव मु॑ख॒तः परि॑ । \newline
18. मु॒ख॒तः परि॒ परि॑ मुख॒तो मु॑ख॒तः परि॑ हरन्ते हरन्ते॒ परि॑ मुख॒तो मु॑ख॒तः परि॑ हरन्ते । \newline
19. परि॑ हरन्ते हरन्ते॒ परि॒ परि॑ हरन्त आश्वि॒नाग्रा॑ नाश्वि॒नाग्रान्॑. हरन्ते॒ परि॒ परि॑ हरन्त आश्वि॒नाग्रान्॑ । \newline
20. ह॒र॒न्त॒ आ॒श्वि॒नाग्रा॑ नाश्वि॒नाग्रान्॑. हरन्ते हरन्त आश्वि॒नाग्रा᳚न् गृह्णीत गृह्णीता श्वि॒नाग्रान्॑. हरन्ते हरन्त आश्वि॒नाग्रा᳚न् गृह्णीत । \newline
21. आ॒श्वि॒नाग्रा᳚न् गृह्णीत गृह्णीता श्वि॒नाग्रा॑ नाश्वि॒नाग्रा᳚न् गृह्णीता नुजाव॒र आ॑नुजाव॒रो गृ॑ह्णीता श्वि॒नाग्रा॑ नाश्वि॒नाग्रा᳚न् गृह्णीता नुजाव॒रः । \newline
22. आ॒श्वि॒नाग्रा॒नित्या᳚श्वि॒न - अ॒ग्रा॒न् । \newline
23. गृ॒ह्णी॒ता॒ नु॒जा॒व॒र आ॑नुजाव॒रो गृ॑ह्णीत गृह्णीता नुजाव॒रो᳚ ऽश्विना॑ व॒श्विना॑ वानुजाव॒रो गृ॑ह्णीत गृह्णीता नुजाव॒रो᳚ ऽश्विनौ᳚ । \newline
24. आ॒नु॒जा॒व॒रो᳚ ऽश्विना॑ व॒श्विना॑ वानुजाव॒र आ॑नुजाव॒रो᳚ ऽश्विनौ॒ वै वा अ॒श्विना॑ वानुजाव॒र आ॑नुजाव॒रो᳚ ऽश्विनौ॒ वै । \newline
25. आ॒नु॒जा॒व॒र इत्या॑नु - जा॒व॒रः । \newline
26. अ॒श्विनौ॒ वै वा अ॒श्विना॑ व॒श्विनौ॒ वै दे॒वाना᳚म् दे॒वानां॒ ॅवा अ॒श्विना॑ व॒श्विनौ॒ वै दे॒वाना᳚म् । \newline
27. वै दे॒वाना᳚म् दे॒वानां॒ ॅवै वै दे॒वाना॑ मानुजाव॒रा वा॑नुजाव॒रौ दे॒वानां॒ ॅवै वै दे॒वाना॑ मानुजाव॒रौ । \newline
28. दे॒वाना॑ मानुजाव॒रा वा॑नुजाव॒रौ दे॒वाना᳚म् दे॒वाना॑ मानुजाव॒रौ प॒श्चा प॒श्चा ऽऽनु॑जाव॒रौ दे॒वाना᳚म् दे॒वाना॑ मानुजाव॒रौ प॒श्चा । \newline
29. आ॒नु॒जा॒व॒रौ प॒श्चा प॒श्चा ऽऽनु॑जाव॒रा वा॑नुजाव॒रौ प॒श्चेवे॑व प॒श्चा ऽऽनु॑जाव॒रा वा॑नुजाव॒रौ प॒श्चेव॑ । \newline
30. आ॒नु॒जा॒व॒रावित्या॑नु - जा॒व॒रौ । \newline
31. प॒श्चेवे॑व प॒श्चा प॒श्चे वाग्र॒ मग्र॑ मिव प॒श्चा प॒श्चे वाग्र᳚म् । \newline
32. इ॒वाग्र॒ मग्र॑ मिवे॒ वाग्र॒म् परि॒ पर्यग्र॑ मिवे॒ वाग्र॒म् परि॑ । \newline
33. अग्र॒म् परि॒ पर्यग्र॒ मग्र॒म् पर्यै॑ता मैता॒म् पर्यग्र॒ मग्र॒म् पर्यै॑ताम् । \newline
34. पर्यै॑ता मैता॒म् परि॒ पर्यै॑ता म॒श्विना॑ व॒श्विना॑ वैता॒म् परि॒ पर्यै॑ता म॒श्विनौ᳚ । \newline
35. ऐ॒ता॒ म॒श्विना॑ व॒श्विना॑ वैता मैता म॒श्विना॑ वे॒त स्यै॒तस्या॒ श्विना॑ वैता मैता म॒श्विना॑ वे॒तस्य॑ । \newline
36. अ॒श्विना॑ वे॒त स्यै॒तस्या॒ श्विना॑ व॒श्विना॑ वे॒तस्य॑ दे॒वता॑ दे॒व तै॒तस्या॒ श्विना॑ व॒श्विना॑ वे॒तस्य॑ दे॒वता᳚ । \newline
37. ए॒तस्य॑ दे॒वता॑ दे॒व तै॒त स्यै॒तस्य॑ दे॒वता॒ यो यो दे॒व तै॒त स्यै॒तस्य॑ दे॒वता॒ यः । \newline
38. दे॒वता॒ यो यो दे॒वता॑ दे॒वता॒ य आ॑नुजाव॒र आ॑नुजाव॒रो यो दे॒वता॑ दे॒वता॒ य आ॑नुजाव॒रः । \newline
39. य आ॑नुजाव॒र आ॑नुजाव॒रो यो य आ॑नुजाव॒र स्तौ ता वा॑नुजाव॒रो यो य आ॑नुजाव॒र स्तौ । \newline
40. आ॒नु॒जा॒व॒र स्तौ ता वा॑नुजाव॒र आ॑नुजाव॒र स्ता वे॒वैव ता वा॑नुजाव॒र आ॑नुजाव॒र स्ता वे॒व । \newline
41. आ॒नु॒जा॒व॒र इत्या॑नु - जा॒व॒रः । \newline
42. ता वे॒वैव तौ ता वे॒वैन॑ मेन मे॒व तौ ता वे॒वैन᳚म् । \newline
43. ए॒वैन॑ मेन मे॒वै वैन॒ मग्र॒ मग्र॑ मेन मे॒वै वैन॒ मग्र᳚म् । \newline
44. ए॒न॒ मग्र॒ मग्र॑ मेन मेन॒ मग्र॒म् परि॒ पर्यग्र॑ मेन मेन॒ मग्र॒म् परि॑ । \newline
45. अग्र॒म् परि॒ पर्यग्र॒ मग्र॒म् परि॑ णयतो नयतः॒ पर्यग्र॒ मग्र॒म् परि॑ णयतः । \newline
46. परि॑ णयतो नयतः॒ परि॒ परि॑ णयतः शु॒क्राग्रा᳚ञ् छु॒क्राग्रा᳚न् नयतः॒ परि॒ परि॑ णयतः शु॒क्राग्रान्॑ । \newline
47. न॒य॒तः॒ शु॒क्राग्रा᳚ञ् छु॒क्राग्रा᳚न् नयतो नयतः शु॒क्राग्रा᳚न् गृह्णीत गृह्णीत शु॒क्राग्रा᳚न् नयतो नयतः शु॒क्राग्रा᳚न् गृह्णीत । \newline
48. शु॒क्राग्रा᳚न् गृह्णीत गृह्णीत शु॒क्राग्रा᳚ञ् छु॒क्राग्रा᳚न् गृह्णीत ग॒तश्री᳚र् ग॒तश्री᳚र् गृह्णीत शु॒क्राग्रा᳚ञ् छु॒क्राग्रा᳚न् गृह्णीत ग॒तश्रीः᳚ । \newline
49. शु॒क्राग्रा॒निति॑ शु॒क्र - अ॒ग्रा॒न् । \newline
50. गृ॒ह्णी॒त॒ ग॒तश्री᳚र् ग॒तश्री᳚र् गृह्णीत गृह्णीत ग॒तश्रीः᳚ प्रति॒ष्ठाका॑मः प्रति॒ष्ठाका॑मो ग॒तश्री᳚र् गृह्णीत गृह्णीत ग॒तश्रीः᳚ प्रति॒ष्ठाका॑मः । \newline
51. ग॒तश्रीः᳚ प्रति॒ष्ठाका॑मः प्रति॒ष्ठाका॑मो ग॒तश्री᳚र् ग॒तश्रीः᳚ प्रति॒ष्ठाका॑मो॒ ऽसा व॒सौ प्र॑ति॒ष्ठाका॑मो ग॒तश्री᳚र् ग॒तश्रीः᳚ प्रति॒ष्ठाका॑मो॒ ऽसौ । \newline
52. ग॒तश्री॒रिति॑ ग॒त - श्रीः॒ । \newline
53. प्र॒ति॒ष्ठाका॑मो॒ ऽसा व॒सौ प्र॑ति॒ष्ठाका॑मः प्रति॒ष्ठाका॑मो॒ ऽसौ वै वा अ॒सौ प्र॑ति॒ष्ठाका॑मः प्रति॒ष्ठाका॑मो॒ ऽसौ वै । \newline
54. प्र॒ति॒ष्ठाका॑म॒ इति॑ प्रति॒ष्ठा - का॒मः॒ । \newline
55. अ॒सौ वै वा अ॒सा व॒सौ वा आ॑दि॒त्य आ॑दि॒त्यो वा अ॒सा व॒सौ वा आ॑दि॒त्यः । \newline
56. वा आ॑दि॒त्य आ॑दि॒त्यो वै वा आ॑दि॒त्यः शु॒क्रः शु॒क्र आ॑दि॒त्यो वै वा आ॑दि॒त्यः शु॒क्रः । \newline
57. आ॒दि॒त्यः शु॒क्रः शु॒क्र आ॑दि॒त्य आ॑दि॒त्यः शु॒क्र ए॒ष ए॒ष शु॒क्र आ॑दि॒त्य आ॑दि॒त्यः शु॒क्र ए॒षः । \newline
58. शु॒क्र ए॒ष ए॒ष शु॒क्रः शु॒क्र ए॒षो ऽन्तो ऽन्त॑ ए॒ष शु॒क्रः शु॒क्र ए॒षो ऽन्तः॑ । \newline
59. ए॒षो ऽन्तो ऽन्त॑ ए॒ष ए॒षो ऽन्तो ऽन्त॒ मन्त॒ मन्त॑ ए॒ष ए॒षो ऽन्तो ऽन्त᳚म् । \newline
60. अन्तो ऽन्त॒ मन्त॒ मन्तो ऽन्तो ऽन्त॑म् मनु॒ष्यो॑ मनु॒ष्यो ऽन्त॒ मन्तो ऽन्तो ऽन्त॑म् मनु॒ष्यः॑ । \newline
61. अन्त॑म् मनु॒ष्यो॑ मनु॒ष्यो ऽन्त॒ मन्त॑म् मनु॒ष्यः॑ श्रि॒यै श्रि॒यै म॑नु॒ष्यो ऽन्त॒ मन्त॑म् मनु॒ष्यः॑ श्रि॒यै । \newline
62. म॒नु॒ष्यः॑ श्रि॒यै श्रि॒यै म॑नु॒ष्यो॑ मनु॒ष्यः॑ श्रि॒यै ग॒त्वा ग॒त्वा श्रि॒यै म॑नु॒ष्यो॑ मनु॒ष्यः॑ श्रि॒यै ग॒त्वा । \newline
\pagebreak
\markright{ TS 7.2.7.3  \hfill https://www.vedavms.in \hfill}

\section{ TS 7.2.7.3 }

\textbf{TS 7.2.7.3 } \newline
\textbf{Samhita Paata} \newline

श्रि॒यै ग॒त्वा नि व॑र्त॒ते ऽन्ता॑दे॒वाऽन्त॒मा र॑भते॒ न ततः॒ पापी॑यान् भवति मन्थ्य॑ग्रान् गृह्णीता-भि॒चर॑-न्नार्तपा॒त्रं ॅवा ए॒तद्-यन्-म॑न्थिपा॒त्रं मृ॒त्युनै॒वैनं॑ ग्राहयति ता॒जगार्ति॒मार्च्छ॑त्या-ग्रय॒णाग्रा᳚न् गृह्णीत॒ यस्य॑ पि॒ता पि॑ताम॒हः पुण्यः॒ स्यादथ॒ तन्न प्रा᳚प्नु॒याद्-वा॒चा वा ए॒ष इ॑न्द्रि॒येण॒ व्यृ॑द्ध्यते॒ यस्य॑ पि॒ता पि॑ताम॒हः पुण्यो॒ - [  ] \newline

\textbf{Pada Paata} \newline

श्रि॒यै । ग॒त्वा । नीति॑ । व॒र्त॒ते॒ । अन्ता᳚त् । ए॒व । अन्त᳚म् । एति॑ । र॒भ॒ते॒ । न । ततः॑ । पापी॑यान् । भ॒व॒ति॒ । म॒न्थ्य॑ग्रा॒निति॑ म॒न्थि-अ॒ग्रा॒न् । गृ॒ह्णी॒त॒ । अ॒भि॒चर॒न्नित्य॑भि - चरन्न्॑ । आ॒र्त॒पा॒त्रमित्या᳚र्त - पा॒त्रम् । वै । ए॒तत् । यत् । म॒न्थि॒पा॒त्रमिति॑ मन्थि - पा॒त्रम् । मृ॒त्युना᳚ । ए॒व । ए॒न॒म् । ग्रा॒ह॒य॒ति॒ । ता॒जक् । आर्ति᳚म् । एति॑ । ऋ॒च्छ॒ति॒ । आ॒ग्र॒य॒णाग्रा॒नित्या᳚ग्रय॒ण-अ॒ग्रा॒न् । गृ॒ह्णी॒त॒ । यस्य॑ । पि॒ता । पि॒ता॒म॒हः । पुण्यः॑ । स्यात् । अथ॑ । तत् । न । प्रा॒प्नु॒यादिति॑ प्र - आ॒प्नु॒यात् । वा॒चा । वै । ए॒षः । इ॒न्द्रि॒येण॑ । वीति॑ । ऋ॒द्ध्य॒ते॒ । यस्य॑ । पि॒ता । पि॒ता॒म॒हः । पुण्यः॑ ।  \newline


\textbf{Krama Paata} \newline

श्रि॒यै ग॒त्वा । ग॒त्वा नि । नि व॑र्तते । व॒र्त॒तेऽन्ता᳚त् । अन्ता॑दे॒व । ए॒वान्त᳚म् । अन्त॒मा । आ र॑भते । र॒भ॒ते॒ न । न ततः॑ । ततः॒ पापी॑यान् । पापी॑यान् भवति । भ॒व॒ति॒ म॒न्थ्य॑ग्रान् । म॒न्थ्य॑ग्रान् गृह्णीत । म॒न्थ्य॑ग्रा॒निति॑ म॒न्थि - अ॒ग्रा॒न्॒ । गृ॒ह्णी॒ता॒भि॒चरन्न्॑ । अ॒भि॒चर॑न्नार्तपा॒त्रम् । अ॒भि॒चर॒न्नित्य॑भि - चरन्न्॑ । आ॒र्त॒पा॒त्रम् ॅवै । आ॒र्त॒पा॒त्रमित्या᳚र्त - पा॒त्रम् । वा ए॒तत् । ए॒तद् यत् । यन् म॑न्थिपा॒त्रम् । म॒न्थि॒पा॒त्रम् मृ॒त्युना᳚ । म॒न्थि॒पा॒त्रमिति॑ मन्थि - पा॒त्रम् । मृ॒त्युनै॒व । ए॒वैन᳚म् । ए॒न॒म् ग्रा॒ह॒य॒ति॒ । ग्रा॒ह॒य॒ति॒ ता॒जक् । ता॒जगार्ति᳚म् । आर्ति॒मा । आर्च्छ॑ति । ऋ॒च्छ॒त्या॒ग्र॒य॒णाग्रान्॑ । आ॒ग्र॒य॒णाग्रा᳚न् गृह्णीत । आ॒ग्र॒य॒णाग्रा॒नित्या᳚ग्रय॒ण - अ॒ग्रा॒न्॒ । गृ॒ह्णी॒त॒ यस्य॑ । यस्य॑ पि॒ता । पि॒ता पि॑ताम॒हः । पि॒ता॒म॒हः पुण्यः॑ । पुण्यः॒ स्यात् । स्यादथ॑ । अथ॒ तत् । तन् न । न प्रा᳚प्नु॒यात् । प्रा॒प्नु॒याद् वा॒चा । प्रा॒प्नु॒यादिति॑ प्र - आ॒प्नु॒यात् । वा॒चा वै । वा ए॒षः । ए॒ष इ॑न्द्रि॒येण॑ । इ॒न्द्रि॒येण॒ वि । व्यृ॑द्ध्यते । ऋ॒द्ध्य॒ते॒ यस्य॑ । यस्य॑ पि॒ता । पि॒ता पि॑ताम॒हः । पि॒ता॒म॒हः पुण्यः॑ । पुण्यो॒ भव॑ति \newline

\textbf{Jatai Paata} \newline

1. श्रि॒यै ग॒त्वा ग॒त्वा श्रि॒यै श्रि॒यै ग॒त्वा । \newline
2. ग॒त्वा नि नि ग॒त्वा ग॒त्वा नि । \newline
3. नि व॑र्तते वर्तते॒ नि नि व॑र्तते । \newline
4. व॒र्त॒ते ऽन्ता॒ दन्ता᳚द् वर्तते वर्त॒ते ऽन्ता᳚त् । \newline
5. अन्ता॑ दे॒वै वान्ता॒ दन्ता॑ दे॒व । \newline
6. ए॒वान्त॒ मन्त॑ मे॒वै वान्त᳚म् । \newline
7. अन्त॒ मा ऽन्त॒ मन्त॒ मा । \newline
8. आ र॑भते रभत॒ आ र॑भते । \newline
9. र॒भ॒ते॒ न न र॑भते रभते॒ न । \newline
10. न तत॒ स्ततो॒ न न ततः॑ । \newline
11. ततः॒ पापी॑या॒न् पापी॑या॒न् तत॒ स्ततः॒ पापी॑यान् । \newline
12. पापी॑यान् भवति भवति॒ पापी॑या॒न् पापी॑यान् भवति । \newline
13. भ॒व॒ति॒ म॒न्थ्य॑ग्रान् म॒न्थ्य॑ग्रान् भवति भवति म॒न्थ्य॑ग्रान् । \newline
14. म॒न्थ्य॑ग्रान् गृह्णीत गृह्णीत म॒न्थ्य॑ग्रान् म॒न्थ्य॑ग्रान् गृह्णीत । \newline
15. म॒न्थ्य॑ग्रा॒निति॑ म॒न्थि - अ॒ग्रा॒न् । \newline
16. गृ॒ह्णी॒ता॒ भि॒चर॑न् नभि॒चर॑न् गृह्णीत गृह्णीता भि॒चरन्न्॑ । \newline
17. अ॒भि॒चर॑न् नार्तपा॒त्र मा᳚र्तपा॒त्र म॑भि॒चर॑न् नभि॒चर॑न् नार्तपा॒त्रम् । \newline
18. अ॒भि॒चर॒न्नित्य॑भि - चरन्न्॑ । \newline
19. आ॒र्त॒पा॒त्रं ॅवै वा आ᳚र्तपा॒त्र मा᳚र्तपा॒त्रं ॅवै । \newline
20. आ॒र्त॒पा॒त्रमित्या᳚र्त - पा॒त्रम् । \newline
21. वा ए॒त दे॒तद् वै वा ए॒तत् । \newline
22. ए॒तद् यद् यदे॒त दे॒तद् यत् । \newline
23. यन् म॑न्थिपा॒त्रम् म॑न्थिपा॒त्रं ॅयद् यन् म॑न्थिपा॒त्रम् । \newline
24. म॒न्थि॒पा॒त्रम् मृ॒त्युना॑ मृ॒त्युना॑ मन्थिपा॒त्रम् म॑न्थिपा॒त्रम् मृ॒त्युना᳚ । \newline
25. म॒न्थि॒पा॒त्रमिति॑ मन्थि - पा॒त्रम् । \newline
26. मृ॒त्यु नै॒वैव मृ॒त्युना॑ मृ॒त्यु नै॒व । \newline
27. ए॒वैन॑ मेन मे॒वै वैन᳚म् । \newline
28. ए॒न॒म् ग्रा॒ह॒य॒ति॒ ग्रा॒ह॒य॒ त्ये॒न॒ मे॒न॒म् ग्रा॒ह॒य॒ति॒ । \newline
29. ग्रा॒ह॒य॒ति॒ ता॒जक् ता॒जग् ग्रा॑हयति ग्राहयति ता॒जक् । \newline
30. ता॒जगार्ति॒ मार्ति॑म् ता॒जक् ता॒जगार्ति᳚म् । \newline
31. आर्ति॒ मा ऽऽर्ति॒ मार्ति॒ मा । \newline
32. आर्च्छ॑ त्यृच्छ॒ त्यार्च्छ॑ति । \newline
33. ऋ॒च्छ॒ त्या॒ग्र॒य॒णाग्रा॑ नाग्रय॒णाग्रा॑ नृच्छ त्यृच्छ त्याग्रय॒णाग्रान्॑ । \newline
34. आ॒ग्र॒य॒णाग्रा᳚न् गृह्णीत गृह्णी ताग्रय॒णाग्रा॑ नाग्रय॒णाग्रा᳚न् गृह्णीत । \newline
35. आ॒ग्र॒य॒णाग्रा॒नित्या᳚ग्रय॒ण - अ॒ग्रा॒न् । \newline
36. गृ॒ह्णी॒त॒ यस्य॒ यस्य॑ गृह्णीत गृह्णीत॒ यस्य॑ । \newline
37. यस्य॑ पि॒ता पि॒ता यस्य॒ यस्य॑ पि॒ता । \newline
38. पि॒ता पि॑ताम॒हः पि॑ताम॒हः पि॒ता पि॒ता पि॑ताम॒हः । \newline
39. पि॒ता॒म॒हः पुण्यः॒ पुण्यः॑ पिताम॒हः पि॑ताम॒हः पुण्यः॑ । \newline
40. पुण्यः॒ स्याथ् स्यात् पुण्यः॒ पुण्यः॒ स्यात् । \newline
41. स्या दथाथ॒ स्याथ् स्या दथ॑ । \newline
42. अथ॒ तत् तद थाथ॒ तत् । \newline
43. तन् न न तत् तन् न । \newline
44. न प्रा᳚प्नु॒यात् प्रा᳚प्नु॒यान् न न प्रा᳚प्नु॒यात् । \newline
45. प्रा॒प्नु॒याद् वा॒चा वा॒चा प्रा᳚प्नु॒यात् प्रा᳚प्नु॒याद् वा॒चा । \newline
46. प्रा॒प्नु॒यादिति॑ प्र - आ॒प्नु॒यात् । \newline
47. वा॒चा वै वै वा॒चा वा॒चा वै । \newline
48. वा ए॒ष ए॒ष वै वा ए॒षः । \newline
49. ए॒ष इ॑न्द्रि॒ये णे᳚न्द्रि॒ये णै॒ष ए॒ष इ॑न्द्रि॒येण॑ । \newline
50. इ॒न्द्रि॒येण॒ वि वीन्द्रि॒ये णे᳚न्द्रि॒येण॒ वि । \newline
51. व्यृ॑द्ध्यत ऋद्ध्यते॒ वि व्यृ॑द्ध्यते । \newline
52. ऋ॒द्ध्य॒ते॒ यस्य॒ यस्य॑ र्‌द्ध्यत ऋद्ध्यते॒ यस्य॑ । \newline
53. यस्य॑ पि॒ता पि॒ता यस्य॒ यस्य॑ पि॒ता । \newline
54. पि॒ता पि॑ताम॒हः पि॑ताम॒हः पि॒ता पि॒ता पि॑ताम॒हः । \newline
55. पि॒ता॒म॒हः पुण्यः॒ पुण्यः॑ पिताम॒हः पि॑ताम॒हः पुण्यः॑ । \newline
56. पुण्यो॒ भव॑ति॒ भव॑ति॒ पुण्यः॒ पुण्यो॒ भव॑ति । \newline

\textbf{Ghana Paata } \newline

1. श्रि॒यै ग॒त्वा ग॒त्वा श्रि॒यै श्रि॒यै ग॒त्वा नि नि ग॒त्वा श्रि॒यै श्रि॒यै ग॒त्वा नि । \newline
2. ग॒त्वा नि नि ग॒त्वा ग॒त्वा नि व॑र्तते वर्तते॒ नि ग॒त्वा ग॒त्वा नि व॑र्तते । \newline
3. नि व॑र्तते वर्तते॒ नि नि व॑र्त॒ते ऽन्ता॒ दन्ता᳚द् वर्तते॒ नि नि व॑र्त॒ते ऽन्ता᳚त् । \newline
4. व॒र्त॒ते ऽन्ता॒ दन्ता᳚द् वर्तते वर्त॒ते ऽन्ता॑ दे॒वै वान्ता᳚द् वर्तते वर्त॒ते ऽन्ता॑दे॒व । \newline
5. अन्ता॑ दे॒वै वान्ता॒ दन्ता॑ दे॒वान्त॒ मन्त॑ मे॒वान्ता॒ दन्ता॑ दे॒वान्त᳚म् । \newline
6. ए॒वान्त॒ मन्त॑ मे॒वै वान्त॒ मा ऽन्त॑ मे॒वै वान्त॒ मा । \newline
7. अन्त॒ मा ऽन्त॒ मन्त॒ मा र॑भते रभत॒ आ ऽन्त॒ मन्त॒ मा र॑भते । \newline
8. आ र॑भते रभत॒ आ र॑भते॒ न न र॑भत॒ आ र॑भते॒ न । \newline
9. र॒भ॒ते॒ न न र॑भते रभते॒ न तत॒ स्ततो॒ न र॑भते रभते॒ न ततः॑ । \newline
10. न तत॒ स्ततो॒ न न ततः॒ पापी॑या॒न् पापी॑या॒न् ततो॒ न न ततः॒ पापी॑यान् । \newline
11. ततः॒ पापी॑या॒न् पापी॑या॒न् तत॒ स्ततः॒ पापी॑यान् भवति भवति॒ पापी॑या॒न् तत॒ स्ततः॒ पापी॑यान् भवति । \newline
12. पापी॑यान् भवति भवति॒ पापी॑या॒न् पापी॑यान् भवति म॒न्थ्य॑ग्रान् म॒न्थ्य॑ग्रान् भवति॒ पापी॑या॒न् पापी॑यान् भवति म॒न्थ्य॑ग्रान् । \newline
13. भ॒व॒ति॒ म॒न्थ्य॑ग्रान् म॒न्थ्य॑ग्रान् भवति भवति म॒न्थ्य॑ग्रान् गृह्णीत गृह्णीत म॒न्थ्य॑ग्रान् भवति भवति म॒न्थ्य॑ग्रान् गृह्णीत । \newline
14. म॒न्थ्य॑ग्रान् गृह्णीत गृह्णीत म॒न्थ्य॑ग्रान् म॒न्थ्य॑ग्रान् गृह्णीता भि॒चर॑न् नभि॒चर॑न् गृह्णीत म॒न्थ्य॑ग्रान् म॒न्थ्य॑ग्रान् गृह्णीता भि॒चरन्न्॑ । \newline
15. म॒न्थ्य॑ग्रा॒निति॑ म॒न्थि - अ॒ग्रा॒न् । \newline
16. गृ॒ह्णी॒ता॒ भि॒चर॑न् नभि॒चर॑न् गृह्णीत गृह्णीता भि॒चर॑न् नार्तपा॒त्र मा᳚र्तपा॒त्र म॑भि॒चर॑न् गृह्णीत गृह्णीता भि॒चर॑न् नार्तपा॒त्रम् । \newline
17. अ॒भि॒चर॑न् नार्तपा॒त्र मा᳚र्तपा॒त्र म॑भि॒चर॑न् नभि॒चर॑न् नार्तपा॒त्रं ॅवै वा आ᳚र्तपा॒त्र म॑भि॒चर॑न् नभि॒चर॑न् नार्तपा॒त्रं ॅवै । \newline
18. अ॒भि॒चर॒न्नित्य॑भि - चरन्न्॑ । \newline
19. आ॒र्त॒पा॒त्रं ॅवै वा आ᳚र्तपा॒त्र मा᳚र्तपा॒त्रं ॅवा ए॒त दे॒तद् वा आ᳚र्तपा॒त्र मा᳚र्तपा॒त्रं ॅवा ए॒तत् । \newline
20. आ॒र्त॒पा॒त्रमित्या᳚र्त - पा॒त्रम् । \newline
21. वा ए॒त दे॒तद् वै वा ए॒तद् यद् यदे॒तद् वै वा ए॒तद् यत् । \newline
22. ए॒तद् यद् यदे॒त दे॒तद् यन् म॑न्थिपा॒त्रम् म॑न्थिपा॒त्रं ॅयदे॒त दे॒तद् यन् म॑न्थिपा॒त्रम् । \newline
23. यन् म॑न्थिपा॒त्रम् म॑न्थिपा॒त्रं ॅयद् यन् म॑न्थिपा॒त्रम् मृ॒त्युना॑ मृ॒त्युना॑ मन्थिपा॒त्रं ॅयद् यन् म॑न्थिपा॒त्रम् मृ॒त्युना᳚ । \newline
24. म॒न्थि॒पा॒त्रम् मृ॒त्युना॑ मृ॒त्युना॑ मन्थिपा॒त्रम् म॑न्थिपा॒त्रम् मृ॒त्यु नै॒वैव मृ॒त्युना॑ मन्थिपा॒त्रम् म॑न्थिपा॒त्रम् मृ॒त्यु नै॒व । \newline
25. म॒न्थि॒पा॒त्रमिति॑ मन्थि - पा॒त्रम् । \newline
26. मृ॒त्युनै॒वैव मृ॒त्युना॑ मृ॒त्यु नै॒वैन॑ मेन मे॒व मृ॒त्युना॑ मृ॒त्यु नै॒वैन᳚म् । \newline
27. ए॒वैन॑ मेन मे॒वैवैन॑म् ग्राहयति ग्राहय त्येन मे॒वैवैन॑म् ग्राहयति । \newline
28. ए॒न॒म् ग्रा॒ह॒य॒ति॒ ग्रा॒ह॒य॒ त्ये॒न॒ मे॒न॒म् ग्रा॒ह॒य॒ति॒ ता॒जक् ता॒जग् ग्रा॑हय त्येन मेनम् ग्राहयति ता॒जक् । \newline
29. ग्रा॒ह॒य॒ति॒ ता॒जक् ता॒जग् ग्रा॑हयति ग्राहयति ता॒जगार्ति॒ मार्ति॑म् ता॒जग् ग्रा॑हयति ग्राहयति ता॒जगार्ति᳚म् । \newline
30. ता॒जगार्ति॒ मार्ति॑म् ता॒जक् ता॒जगार्ति॒ मा ऽऽर्ति॑म् ता॒जक् ता॒जगार्ति॒ मा । \newline
31. आर्ति॒ मा ऽऽर्ति॒ मार्ति॒ मार्च्छ॑ त्यृच्छ॒त्या ऽऽर्ति॒ मार्ति॒ मार्च्छ॑ति । \newline
32. आर्च्छ॑ त्यृच्छ॒ त्यार्च्छ॑ त्याग्रय॒णाग्रा॑ नाग्रय॒णाग्रा॑ नृच्छ॒ त्यार्च्छ॑ त्याग्रय॒णाग्रान्॑ । \newline
33. ऋ॒च्छ॒ त्या॒ग्र॒य॒णाग्रा॑ नाग्रय॒णाग्रा॑ नृच्छ त्यृच्छ त्याग्रय॒णाग्रा᳚न् गृह्णीत गृह्णीता ग्रय॒णाग्रा॑ नृच्छ त्यृच्छ त्याग्रय॒णाग्रा᳚न् गृह्णीत । \newline
34. आ॒ग्र॒य॒णाग्रा᳚न् गृह्णीत गृह्णीता ग्रय॒णाग्रा॑ नाग्रय॒णाग्रा᳚न् गृह्णीत॒ यस्य॒ यस्य॑ गृह्णीता ग्रय॒णाग्रा॑ नाग्रय॒णाग्रा᳚न् गृह्णीत॒ यस्य॑ । \newline
35. आ॒ग्र॒य॒णाग्रा॒नित्या᳚ग्रय॒ण - अ॒ग्रा॒न् । \newline
36. गृ॒ह्णी॒त॒ यस्य॒ यस्य॑ गृह्णीत गृह्णीत॒ यस्य॑ पि॒ता पि॒ता यस्य॑ गृह्णीत गृह्णीत॒ यस्य॑ पि॒ता । \newline
37. यस्य॑ पि॒ता पि॒ता यस्य॒ यस्य॑ पि॒ता पि॑ताम॒हः पि॑ताम॒हः पि॒ता यस्य॒ यस्य॑ पि॒ता पि॑ताम॒हः । \newline
38. पि॒ता पि॑ताम॒हः पि॑ताम॒हः पि॒ता पि॒ता पि॑ताम॒हः पुण्यः॒ पुण्यः॑ पिताम॒हः पि॒ता पि॒ता पि॑ताम॒हः पुण्यः॑ । \newline
39. पि॒ता॒म॒हः पुण्यः॒ पुण्यः॑ पिताम॒हः पि॑ताम॒हः पुण्यः॒ स्याथ् स्यात् पुण्यः॑ पिताम॒हः पि॑ताम॒हः पुण्यः॒ स्यात् । \newline
40. पुण्यः॒ स्याथ् स्यात् पुण्यः॒ पुण्यः॒ स्या दथाथ॒ स्यात् पुण्यः॒ पुण्यः॒ स्या दथ॑ । \newline
41. स्या दथाथ॒ स्याथ् स्या दथ॒ तत् तदथ॒ स्याथ् स्या दथ॒ तत् । \newline
42. अथ॒ तत् त दथाथ॒ तन् न न त दथाथ॒ तन् न । \newline
43. तन् न न तत् तन् न प्रा᳚प्नु॒यात् प्रा᳚प्नु॒यान् न तत् तन् न प्रा᳚प्नु॒यात् । \newline
44. न प्रा᳚प्नु॒यात् प्रा᳚प्नु॒यान् न न प्रा᳚प्नु॒याद् वा॒चा वा॒चा प्रा᳚प्नु॒यान् न न प्रा᳚प्नु॒याद् वा॒चा । \newline
45. प्रा॒प्नु॒याद् वा॒चा वा॒चा प्रा᳚प्नु॒यात् प्रा᳚प्नु॒याद् वा॒चा वै वै वा॒चा प्रा᳚प्नु॒यात् प्रा᳚प्नु॒याद् वा॒चा वै । \newline
46. प्रा॒प्नु॒यादिति॑ प्र - आ॒प्नु॒यात् । \newline
47. वा॒चा वै वै वा॒चा वा॒चा वा ए॒ष ए॒ष वै वा॒चा वा॒चा वा ए॒षः । \newline
48. वा ए॒ष ए॒ष वै वा ए॒ष इ॑न्द्रि॒ये णे᳚न्द्रि॒ये णै॒ष वै वा ए॒ष इ॑न्द्रि॒येण॑ । \newline
49. ए॒ष इ॑न्द्रि॒ये णे᳚न्द्रि॒ये णै॒ष ए॒ष इ॑न्द्रि॒येण॒ वि वीन्द्रि॒ये णै॒ष ए॒ष इ॑न्द्रि॒येण॒ वि । \newline
50. इ॒न्द्रि॒येण॒ वि वीन्द्रि॒ये णे᳚न्द्रि॒येण॒ व्यृ॑द्ध्यत ऋद्ध्यते॒ वीन्द्रि॒ये णे᳚न्द्रि॒येण॒ व्यृ॑द्ध्यते । \newline
51. व्यृ॑द्ध्यत ऋद्ध्यते॒ वि व्यृ॑द्ध्यते॒ यस्य॒ यस्य॑ र्‌द्ध्यते॒ वि व्यृ॑द्ध्यते॒ यस्य॑ । \newline
52. ऋ॒द्ध्य॒ते॒ यस्य॒ यस्य॑ र्‌द्ध्यत ऋद्ध्यते॒ यस्य॑ पि॒ता पि॒ता यस्य॑ र्‌द्ध्यत ऋद्ध्यते॒ यस्य॑ पि॒ता । \newline
53. यस्य॑ पि॒ता पि॒ता यस्य॒ यस्य॑ पि॒ता पि॑ताम॒हः पि॑ताम॒हः पि॒ता यस्य॒ यस्य॑ पि॒ता पि॑ताम॒हः । \newline
54. पि॒ता पि॑ताम॒हः पि॑ताम॒हः पि॒ता पि॒ता पि॑ताम॒हः पुण्यः॒ पुण्यः॑ पिताम॒हः पि॒ता पि॒ता पि॑ताम॒हः पुण्यः॑ । \newline
55. पि॒ता॒म॒हः पुण्यः॒ पुण्यः॑ पिताम॒हः पि॑ताम॒हः पुण्यो॒ भव॑ति॒ भव॑ति॒ पुण्यः॑ पिताम॒हः पि॑ताम॒हः पुण्यो॒ भव॑ति । \newline
56. पुण्यो॒ भव॑ति॒ भव॑ति॒ पुण्यः॒ पुण्यो॒ भव॒ त्यथाथ॒ भव॑ति॒ पुण्यः॒ पुण्यो॒ भव॒ त्यथ॑ । \newline
\pagebreak
\markright{ TS 7.2.7.4  \hfill https://www.vedavms.in \hfill}

\section{ TS 7.2.7.4 }

\textbf{TS 7.2.7.4 } \newline
\textbf{Samhita Paata} \newline

भ॒वत्यथ॒ तन्न प्रा॒प्नोत्युर॑ इवै॒तद्-य॒ज्ञ्स्य॒ वागि॑व॒ यदा᳚ग्रय॒णो वा॒चैवैन॑मिन्द्रि॒येण॒ सम॑र्द्धयति॒ न ततः॒ पापी॑यान् भवत्यु॒क्थ्या᳚ग्रान् गृह्णीताभिच॒र्यमा॑णः॒ सर्वे॑षां॒ ॅवा ए॒तत् पात्रा॑णामिन्द्रि॒यं ॅयदु॑क्थ्यपा॒त्रꣳ सर्वे॑णै॒वैन॑मिन्द्रि॒येणाति॒ प्रयु॑ङ्क्ते॒ सर॑स्वत्य॒भि नो॑ नेषि॒ वस्य॒ इति॑ पुरो॒रुचं॑ कुर्या॒द्-वाग्वै - [  ] \newline

\textbf{Pada Paata} \newline

भव॑ति । अथ॑ । तत् । न । प्रा॒प्नोतीति॑ प्र - आ॒प्नोति॑ । उरः॑ । इ॒व॒ । ए॒तत् । य॒ज्ञ्स्य॑ । वाक् । इ॒व॒ । यत् । आ॒ग्र॒य॒णः । वा॒चा । ए॒व । ए॒न॒म् । इ॒न्द्रि॒येण॑ । समिति॑ । अ॒द्‌र्ध॒य॒ति॒ । न । ततः॑ । पापी॑यान् । भ॒व॒ति॒ । उ॒क्थ्या᳚ग्रा॒नित्यु॒क्थ्य॑ - अ॒ग्रा॒न् । गृ॒ह्णी॒त॒ । अ॒भि॒च॒र्यमा॑ण॒ इत्य॑भि - च॒र्यमा॑णः । सर्वे॑षाम् । वै । ए॒तत् । पात्रा॑णाम् । इ॒न्द्रि॒यम् । यत् । उ॒क्थ्य॒पा॒त्रमित्यु॑क्थ्य - पा॒त्रम् । सर्वे॑ण । ए॒व । ए॒न॒म् । इ॒न्द्रि॒येण॑ । अति॑ । प्रेति॑ । यु॒ङ्क्ते॒ । सर॑स्वति । अ॒भीति॑ । नः॒ । ने॒षि॒ । वस्यः॑ । इति॑ । पु॒रो॒रुच॒मिति॑ पुरः - रुच᳚म् । कु॒र्या॒त् । वाक् । वै ।  \newline


\textbf{Krama Paata} \newline

भव॒त्यथ॑ । अथ॒ तत् । तन् न । न प्रा॒प्नोति॑ । प्रा॒प्नोत्युरः॑ । प्रा॒प्नोतीति॑ प्र - आ॒प्नोति॑ । उर॑ इव । इ॒वै॒तत् । ए॒तद् य॒ज्ञ्स्य॑ । य॒ज्ञ्स्य॒ वाक् । वागि॑व । इ॒व॒ यत् । यदा᳚ग्रय॒णः । आ॒ग्र॒य॒णो वा॒चा । वा॒चैव । ए॒वैन᳚म् । ए॒न॒मि॒न्द्रि॒येण॑ । इ॒न्द्रि॒येण॒ सम् । सम॑र्द्धयति । अ॒र्द्ध॒य॒ति॒ न । न ततः॑ । ततः॒ पापी॑यान् । पापी॑यान् भवति । भ॒व॒त्यु॒क्थ्या᳚ग्रान् । उ॒क्थ्या᳚ग्रान् गृह्णीत । उ॒क्थ्या᳚ग्रा॒नित्यु॒क्थ्य॑ - अ॒ग्रा॒न्॒ । गृ॒ह्णी॒ता॒भि॒च॒र्यमा॑णः । अ॒भि॒च॒र्यमा॑णः॒ सर्वे॑षाम् । अ॒भि॒च॒र्यमा॑ण॒ इत्य॑भि - च॒र्यमा॑णः । सर्वे॑षा॒म् ॅवै । वा ए॒तत् । ए॒तत् पात्रा॑णाम् । पात्रा॑णामिन्द्रि॒यम् । इ॒न्द्रि॒यम् ॅयत् । यदु॑क्थ्यपा॒त्रम् । उ॒क्थ्य॒पा॒त्रꣳ सर्वे॑ण । उ॒क्थ्य॒पा॒त्रमित्यु॑क्थ्य - पा॒त्रम् । सर्वे॑णै॒व । ए॒वैन᳚म् । ए॒न॒मि॒न्द्रि॒येण॑ । इ॒न्द्रि॒येणाति॑ । अति॒ प्र । प्र यु॑ङ्‍क्ते । यु॒ङ्‍क्ते॒ सर॑स्वति । सर॑स्वत्य॒भि । अ॒भि नः॑ । नो॒ ने॒षि॒ । ने॒षि॒ वस्यः॑ । वस्य॒ इति॑ । इति॑ पुरो॒रुच᳚म् । पु॒रो॒रुच॑म् कुर्यात् । पु॒रो॒रुच॒मिति॑ पुरः - रुच᳚म् । कु॒र्या॒द् वाक् । वाग् वै । वै सर॑स्वती \newline

\textbf{Jatai Paata} \newline

1. भव॒ त्यथाथ॒ भव॑ति॒ भव॒ त्यथ॑ । \newline
2. अथ॒ तत् तद थाथ॒ तत् । \newline
3. तन् न न तत् तन् न । \newline
4. न प्रा॒प्नोति॑ प्रा॒प्नोति॒ न न प्रा॒प्नोति॑ । \newline
5. प्रा॒प्नो त्युर॒ उरः॑ प्रा॒प्नोति॑ प्रा॒प्नो त्युरः॑ । \newline
6. प्रा॒प्नोतीति॑ प्र - आ॒प्नोति॑ । \newline
7. उर॑ इवे॒ वोर॒ उर॑ इव । \newline
8. इ॒वै॒त दे॒त दि॑वे वै॒तत् । \newline
9. ए॒तद् य॒ज्ञ्स्य॑ य॒ज्ञ् स्यै॒त दे॒तद् य॒ज्ञ्स्य॑ । \newline
10. य॒ज्ञ्स्य॒ वाग् वाग् य॒ज्ञ्स्य॑ य॒ज्ञ्स्य॒ वाक् । \newline
11. वागि॑वेव॒ वाग् वागि॑व । \newline
12. इ॒व॒ यद् यदि॑वेव॒ यत् । \newline
13. यदा᳚ग्रय॒ण आ᳚ग्रय॒णो यद् यदा᳚ग्रय॒णः । \newline
14. आ॒ग्र॒य॒णो वा॒चा वा॒चा ऽऽग्र॑य॒ण आ᳚ग्रय॒णो वा॒चा । \newline
15. वा॒चै वैव वा॒चा वा॒चैव । \newline
16. ए॒वैन॑ मेन मे॒वै वैन᳚म् । \newline
17. ए॒न॒ मि॒न्द्रि॒ये णे᳚न्द्रि॒येणै॑न मेन मिन्द्रि॒येण॑ । \newline
18. इ॒न्द्रि॒येण॒ सꣳ स मि॑न्द्रि॒ये णे᳚न्द्रि॒येण॒ सम् । \newline
19. स म॑र्द्धय त्यर्द्धयति॒ सꣳ स म॑र्द्धयति । \newline
20. अ॒र्द्ध॒य॒ति॒ न नार्द्ध॑य त्यर्द्धयति॒ न । \newline
21. न तत॒ स्ततो॒ न न ततः॑ । \newline
22. ततः॒ पापी॑या॒न् पापी॑या॒न् तत॒ स्ततः॒ पापी॑यान् । \newline
23. पापी॑यान् भवति भवति॒ पापी॑या॒न् पापी॑यान् भवति । \newline
24. भ॒व॒ त्यु॒क्थ्या᳚ग्रा नु॒क्थ्या᳚ग्रान् भवति भव त्यु॒क्थ्या᳚ग्रान् । \newline
25. उ॒क्थ्या᳚ग्रान् गृह्णीत गृह्णीतो॒क्थ्या᳚ग्रा नु॒क्थ्या᳚ग्रान् गृह्णीत । \newline
26. उ॒क्थ्या᳚ग्रा॒नित्यु॒क्थ्य॑ - अ॒ग्रा॒न् । \newline
27. गृ॒ह्णी॒ता॒ भि॒च॒र्यमा॑णो ऽभिच॒र्यमा॑णो गृह्णीत गृह्णीता भिच॒र्यमा॑णः । \newline
28. अ॒भि॒च॒र्यमा॑णः॒ सर्वे॑षाꣳ॒॒ सर्वे॑षा मभिच॒र्यमा॑णो ऽभिच॒र्यमा॑णः॒ सर्वे॑षाम् । \newline
29. अ॒भि॒च॒र्यमा॑ण॒ इत्य॑भि - च॒र्यमा॑णः । \newline
30. सर्वे॑षां॒ ॅवै वै सर्वे॑षाꣳ॒॒ सर्वे॑षां॒ ॅवै । \newline
31. वा ए॒त दे॒तद् वै वा ए॒तत् । \newline
32. ए॒तत् पात्रा॑णा॒म् पात्रा॑णा मे॒त दे॒तत् पात्रा॑णाम् । \newline
33. पात्रा॑णा मिन्द्रि॒य मि॑न्द्रि॒यम् पात्रा॑णा॒म् पात्रा॑णा मिन्द्रि॒यम् । \newline
34. इ॒न्द्रि॒यं ॅयद् यदि॑न्द्रि॒य मि॑न्द्रि॒यं ॅयत् । \newline
35. यदु॑क्थ्यपा॒त्र मु॑क्थ्यपा॒त्रं ॅयद् यदु॑क्थ्यपा॒त्रम् । \newline
36. उ॒क्थ्य॒पा॒त्रꣳ सर्वे॑ण॒ सर्वे॑णोक्थ्यपा॒त्र मु॑क्थ्यपा॒त्रꣳ सर्वे॑ण । \newline
37. उ॒क्थ्य॒पा॒त्रमित्यु॑क्थ्य - पा॒त्रम् । \newline
38. सर्वे॑ णै॒वैव सर्वे॑ण॒ सर्वे॑ णै॒व । \newline
39. ए॒वैन॑ मेन मे॒वै वैन᳚म् । \newline
40. ए॒न॒ मि॒न्द्रि॒ये णे᳚न्द्रि॒ये णै॑न मेन मिन्द्रि॒येण॑ । \newline
41. इ॒न्द्रि॒येणा त्यती᳚न्द्रि॒ये णे᳚न्द्रि॒ये णाति॑ । \newline
42. अति॒ प्र प्रात्यति॒ प्र । \newline
43. प्र यु॑ङ्क्ते युङ्क्ते॒ प्र प्र यु॑ङ्क्ते । \newline
44. यु॒ङ्क्ते॒ सर॑स्वति॒ सर॑स्वति युङ्क्ते युङ्क्ते॒ सर॑स्वति । \newline
45. सर॑स्व त्य॒भ्य॑भि सर॑स्वति॒ सर॑स्व त्य॒भि । \newline
46. अ॒भि नो॑ नो अ॒भ्य॑भि नः॑ । \newline
47. नो॒ ने॒षि॒ ने॒षि॒ नो॒ नो॒ ने॒षि॒ । \newline
48. ने॒षि॒ वस्यो॒ वस्यो॑ नेषि नेषि॒ वस्यः॑ । \newline
49. वस्य॒ इतीति॒ वस्यो॒ वस्य॒ इति॑ । \newline
50. इति॑ पुरो॒रुच॑म् पुरो॒रुच॒ मितीति॑ पुरो॒रुच᳚म् । \newline
51. पु॒रो॒रुच॑म् कुर्यात् कुर्यात् पुरो॒रुच॑म् पुरो॒रुच॑म् कुर्यात् । \newline
52. पु॒रो॒रुच॒मिति॑ पुरः - रुच᳚म् । \newline
53. कु॒र्या॒द् वाग् वाक् कु॑र्यात् कुर्या॒द् वाक् । \newline
54. वाग् वै वै वाग् वाग् वै । \newline
55. वै सर॑स्वती॒ सर॑स्वती॒ वै वै सर॑स्वती । \newline

\textbf{Ghana Paata } \newline

1. भव॒ त्यथाथ॒ भव॑ति॒ भव॒ त्यथ॒ तत् तदथ॒ भव॑ति॒ भव॒ त्यथ॒ तत् । \newline
2. अथ॒ तत् त दथाथ॒ तन् न न त दथाथ॒ तन् न । \newline
3. तन् न न तत् तन् न प्रा॒प्नोति॑ प्रा॒प्नोति॒ न तत् तन् न प्रा॒प्नोति॑ । \newline
4. न प्रा॒प्नोति॑ प्रा॒प्नोति॒ न न प्रा॒प्नो त्युर॒ उरः॑ प्रा॒प्नोति॒ न न प्रा॒प्नो त्युरः॑ । \newline
5. प्रा॒प्नो त्युर॒ उरः॑ प्रा॒प्नोति॑ प्रा॒प्नो त्युर॑ इवे॒वोरः॑ प्रा॒प्नोति॑ प्रा॒प्नो त्युर॑ इव । \newline
6. प्रा॒प्नोतीति॑ प्र - आ॒प्नोति॑ । \newline
7. उर॑ इवे॒वोर॒ उर॑ इवै॒त दे॒त दि॒वोर॒ उर॑ इवै॒तत् । \newline
8. इ॒वै॒त दे॒त दि॑वे वै॒तद् य॒ज्ञ्स्य॑ य॒ज्ञ् स्यै॒त दि॑वे वै॒तद् य॒ज्ञ्स्य॑ । \newline
9. ए॒तद् य॒ज्ञ्स्य॑ य॒ज्ञ्स्यै॒त दे॒तद् य॒ज्ञ्स्य॒ वाग् वाग् य॒ज्ञ्स्यै॒त दे॒तद् य॒ज्ञ्स्य॒ वाक् । \newline
10. य॒ज्ञ्स्य॒ वाग् वाग् य॒ज्ञ्स्य॑ य॒ज्ञ्स्य॒ वागि॑वेव॒ वाग् य॒ज्ञ्स्य॑ य॒ज्ञ्स्य॒ वागि॑व । \newline
11. वागि॑वेव॒ वाग् वागि॑व॒ यद् यदि॑व॒ वाग् वागि॑व॒ यत् । \newline
12. इ॒व॒ यद् यदि॑वेव॒ यदा᳚ग्रय॒ण आ᳚ग्रय॒णो यदि॑वेव॒ यदा᳚ग्रय॒णः । \newline
13. यदा᳚ग्रय॒ण आ᳚ग्रय॒णो यद् यदा᳚ग्रय॒णो वा॒चा वा॒चा ऽऽग्र॑य॒णो यद् यदा᳚ग्रय॒णो वा॒चा । \newline
14. आ॒ग्र॒य॒णो वा॒चा वा॒चा ऽऽग्र॑य॒ण आ᳚ग्रय॒णो वा॒चैवैव वा॒चा ऽऽग्र॑य॒ण आ᳚ग्रय॒णो वा॒चैव । \newline
15. वा॒चै वैव वा॒चा वा॒चैवैन॑ मेन मे॒व वा॒चा वा॒चैवैन᳚म् । \newline
16. ए॒वैन॑ मेन मे॒वै वैन॑ मिन्द्रि॒ये णे᳚न्द्रि॒येणै॑न मे॒वै वैन॑ मिन्द्रि॒येण॑ । \newline
17. ए॒न॒ मि॒न्द्रि॒ये णे᳚न्द्रि॒ये णै॑न मेन मिन्द्रि॒येण॒ सꣳ स मि॑न्द्रि॒ये णै॑न मेन मिन्द्रि॒येण॒ सम् । \newline
18. इ॒न्द्रि॒येण॒ सꣳ स मि॑न्द्रि॒ये णे᳚न्द्रि॒येण॒ स म॑र्द्धय त्यर्द्धयति॒ स मि॑न्द्रि॒ये णे᳚न्द्रि॒येण॒ स म॑र्द्धयति । \newline
19. स म॑र्द्धय त्यर्द्धयति॒ सꣳ स म॑र्द्धयति॒ न नार्द्ध॑यति॒ सꣳ स म॑र्द्धयति॒ न । \newline
20. अ॒र्द्ध॒य॒ति॒ न नार्द्ध॑य त्यर्द्धयति॒ न तत॒ स्ततो॒ नार्द्ध॑य त्यर्द्धयति॒ न ततः॑ । \newline
21. न तत॒ स्ततो॒ न न ततः॒ पापी॑या॒न् पापी॑या॒न् ततो॒ न न ततः॒ पापी॑यान् । \newline
22. ततः॒ पापी॑या॒न् पापी॑या॒न् तत॒ स्ततः॒ पापी॑यान् भवति भवति॒ पापी॑या॒न् तत॒ स्ततः॒ पापी॑यान् भवति । \newline
23. पापी॑यान् भवति भवति॒ पापी॑या॒न् पापी॑यान् भव त्यु॒क्थ्या᳚ग्रा नु॒क्थ्या᳚ग्रान् भवति॒ पापी॑या॒न् पापी॑यान् भव त्यु॒क्थ्या᳚ग्रान् । \newline
24. भ॒व॒ त्यु॒क्थ्या᳚ग्रा नु॒क्थ्या᳚ग्रान् भवति भव त्यु॒क्थ्या᳚ग्रान् गृह्णीत गृह्णीतो॒क्थ्या᳚ग्रान् भवति भव त्यु॒क्थ्या᳚ग्रान् गृह्णीत । \newline
25. उ॒क्थ्या᳚ग्रान् गृह्णीत गृह्णीतो॒क्थ्या᳚ग्रा नु॒क्थ्या᳚ग्रान् गृह्णीता भिच॒र्यमा॑णो ऽभिच॒र्यमा॑णो गृह्णीतो॒क्थ्या᳚ग्रा नु॒क्थ्या᳚ग्रान् गृह्णीता भिच॒र्यमा॑णः । \newline
26. उ॒क्थ्या᳚ग्रा॒नित्यु॒क्थ्य॑ - अ॒ग्रा॒न् । \newline
27. गृ॒ह्णी॒ता॒ भि॒च॒र्यमा॑णो ऽभिच॒र्यमा॑णो गृह्णीत गृह्णीता भिच॒र्यमा॑णः॒ सर्वे॑षाꣳ॒॒ सर्वे॑षा मभिच॒र्यमा॑णो गृह्णीत गृह्णीता भिच॒र्यमा॑णः॒ सर्वे॑षाम् । \newline
28. अ॒भि॒च॒र्यमा॑णः॒ सर्वे॑षाꣳ॒॒ सर्वे॑षा मभिच॒र्यमा॑णो ऽभिच॒र्यमा॑णः॒ सर्वे॑षां॒ ॅवै वै सर्वे॑षा मभिच॒र्यमा॑णो ऽभिच॒र्यमा॑णः॒ सर्वे॑षां॒ ॅवै । \newline
29. अ॒भि॒च॒र्यमा॑ण॒ इत्य॑भि - च॒र्यमा॑णः । \newline
30. सर्वे॑षां॒ ॅवै वै सर्वे॑षाꣳ॒॒ सर्वे॑षां॒ ॅवा ए॒त दे॒तद् वै सर्वे॑षाꣳ॒॒ सर्वे॑षां॒ ॅवा ए॒तत् । \newline
31. वा ए॒त दे॒तद् वै वा ए॒तत् पात्रा॑णा॒म् पात्रा॑णा मे॒तद् वै वा ए॒तत् पात्रा॑णाम् । \newline
32. ए॒तत् पात्रा॑णा॒म् पात्रा॑णा मे॒त दे॒तत् पात्रा॑णा मिन्द्रि॒य मि॑न्द्रि॒यम् पात्रा॑णा मे॒त दे॒तत् पात्रा॑णा मिन्द्रि॒यम् । \newline
33. पात्रा॑णा मिन्द्रि॒य मि॑न्द्रि॒यम् पात्रा॑णा॒म् पात्रा॑णा मिन्द्रि॒यं ॅयद् यदि॑न्द्रि॒यम् पात्रा॑णा॒म् पात्रा॑णा मिन्द्रि॒यं ॅयत् । \newline
34. इ॒न्द्रि॒यं ॅयद् यदि॑न्द्रि॒य मि॑न्द्रि॒यं ॅयदु॑क्थ्यपा॒त्र मु॑क्थ्यपा॒त्रं ॅयदि॑न्द्रि॒य मि॑न्द्रि॒यं ॅयदु॑क्थ्यपा॒त्रम् । \newline
35. यदु॑क्थ्यपा॒त्र मु॑क्थ्यपा॒त्रं ॅयद् यदु॑क्थ्यपा॒त्रꣳ सर्वे॑ण॒ सर्वे॑णोक्थ्यपा॒त्रं ॅयद् यदु॑क्थ्यपा॒त्रꣳ सर्वे॑ण । \newline
36. उ॒क्थ्य॒पा॒त्रꣳ सर्वे॑ण॒ सर्वे॑णोक्थ्यपा॒त्र मु॑क्थ्यपा॒त्रꣳ सर्वे॑णै॒ वैव सर्वे॑णोक्थ्यपा॒त्र मु॑क्थ्यपा॒त्रꣳ सर्वे॑ णै॒व । \newline
37. उ॒क्थ्य॒पा॒त्रमित्यु॑क्थ्य - पा॒त्रम् । \newline
38. सर्वे॑णै॒ वैव सर्वे॑ण॒ सर्वे॑णै॒वैन॑ मेन मे॒व सर्वे॑ण॒ सर्वे॑णै॒वैन᳚म् । \newline
39. ए॒वैन॑ मेन मे॒वै वैन॑ मिन्द्रि॒ये णे᳚न्द्रि॒येणै॑न मे॒वै वैन॑ मिन्द्रि॒येण॑ । \newline
40. ए॒न॒ मि॒न्द्रि॒ये णे᳚न्द्रि॒येणै॑न मेन मिन्द्रि॒येणा त्यती᳚न्द्रि॒येणै॑न मेन मिन्द्रि॒येणाति॑ । \newline
41. इ॒न्द्रि॒येणा त्यती᳚न्द्रि॒ये णे᳚न्द्रि॒येणाति॒ प्र प्राती᳚न्द्रि॒ये णे᳚न्द्रि॒येणाति॒ प्र । \newline
42. अति॒ प्र प्रात्यति॒ प्र यु॑ङ्क्ते युङ्क्ते॒ प्रात्यति॒ प्र यु॑ङ्क्ते । \newline
43. प्र यु॑ङ्क्ते युङ्क्ते॒ प्र प्र यु॑ङ्क्ते॒ सर॑स्वति॒ सर॑स्वति युङ्क्ते॒ प्र प्र यु॑ङ्क्ते॒ सर॑स्वति । \newline
44. यु॒ङ्क्ते॒ सर॑स्वति॒ सर॑स्वति युङ्क्ते युङ्क्ते॒ सर॑स्व त्य॒भ्य॑भि सर॑स्वति युङ्क्ते युङ्क्ते॒ सर॑स्व त्य॒भि । \newline
45. सर॑स्व त्य॒भ्य॑भि सर॑स्वति॒ सर॑स्व त्य॒भि नो॑ नो अ॒भि सर॑स्वति॒ सर॑स्व त्य॒भि नः॑ । \newline
46. अ॒भि नो॑ नो अ॒भ्य॑भि नो॑ नेषि नेषि नो अ॒भ्य॑भि नो॑ नेषि । \newline
47. नो॒ ने॒षि॒ ने॒षि॒ नो॒ नो॒ ने॒षि॒ वस्यो॒ वस्यो॑ नेषि नो नो नेषि॒ वस्यः॑ । \newline
48. ने॒षि॒ वस्यो॒ वस्यो॑ नेषि नेषि॒ वस्य॒ इतीति॒ वस्यो॑ नेषि नेषि॒ वस्य॒ इति॑ । \newline
49. वस्य॒ इतीति॒ वस्यो॒ वस्य॒ इति॑ पुरो॒रुच॑म् पुरो॒रुच॒ मिति॒ वस्यो॒ वस्य॒ इति॑ पुरो॒रुच᳚म् । \newline
50. इति॑ पुरो॒रुच॑म् पुरो॒रुच॒ मितीति॑ पुरो॒रुच॑म् कुर्यात् कुर्यात् पुरो॒रुच॒ मितीति॑ पुरो॒रुच॑म् कुर्यात् । \newline
51. पु॒रो॒रुच॑म् कुर्यात् कुर्यात् पुरो॒रुच॑म् पुरो॒रुच॑म् कुर्या॒द् वाग् वाक् कु॑र्यात् पुरो॒रुच॑म् पुरो॒रुच॑म् कुर्या॒द् वाक् । \newline
52. पु॒रो॒रुच॒मिति॑ पुरः - रुच᳚म् । \newline
53. कु॒र्या॒द् वाग् वाक् कु॑र्यात् कुर्या॒द् वाग् वै वै वाक् कु॑र्यात् कुर्या॒द् वाग् वै । \newline
54. वाग् वै वै वाग् वाग् वै सर॑स्वती॒ सर॑स्वती॒ वै वाग् वाग् वै सर॑स्वती । \newline
55. वै सर॑स्वती॒ सर॑स्वती॒ वै वै सर॑स्वती वा॒चा वा॒चा सर॑स्वती॒ वै वै सर॑स्वती वा॒चा । \newline
\pagebreak
\markright{ TS 7.2.7.5  \hfill https://www.vedavms.in \hfill}

\section{ TS 7.2.7.5 }

\textbf{TS 7.2.7.5 } \newline
\textbf{Samhita Paata} \newline

सर॑स्वती वा॒चैवैन॒मति॒ प्रयु॑ङ्क्ते॒ मा त्वत् क्षेत्रा॒ण्यर॑णानि ग॒न्मेत्या॑ह मृ॒त्योर्वै क्षेत्रा॒ण्यर॑णानि॒ तेनै॒व मृ॒त्योः क्षेत्रा॑णि॒ न ग॑च्छति पू॒र्णान् ग्रहा᳚न् गृह्णीयादामया॒विनः॑ प्रा॒णान् वा ए॒तस्य॒ शुगृ॑च्छति॒ यस्या॒ ऽऽ*मय॑ति प्रा॒णा ग्रहाः᳚ प्रा॒णाने॒वास्य॑ शु॒चो मु॑ञ्चत्यु॒त यदी॒तासु॒र्भव॑ति॒ जीव॑त्ये॒व पू॒र्णान् ग्रहा᳚न् ( ) गृह्णीया॒द्-यर्.हि॑ प॒र्जन्यो॒ न वर्.षे᳚त् प्रा॒णान् वा ए॒तर्.हि॑ प्र॒जानाꣳ॒॒ शुगृ॑च्छति॒ यर्.हि॑ प॒र्जन्यो॒ न वर्.ष॑ति प्रा॒णा ग्रहाः᳚ प्रा॒णाने॒व प्र॒जानाꣳ॑ शु॒चो मु॑ञ्चति ता॒जक् प्र व॑र्.षति ॥ \newline

\textbf{Pada Paata} \newline

सर॑स्वती । वा॒चा । ए॒व । ए॒न॒म् । अति॑ । प्रेति॑ । यु॒ङ्क्ते॒ । मा । त्वत् । क्षेत्रा॑णि । अर॑णानि । ग॒न्म॒ । इति॑ । आ॒ह॒ । मृ॒त्योः । वै । क्षेत्रा॑णि । अर॑णानि । तेन॑ । ए॒व । मृ॒त्योः । क्षेत्रा॑णि । न । ग॒च्छ॒ति॒ । पू॒र्णान् । ग्रहान्॑ । गृ॒ह्णी॒या॒त् । आ॒म॒या॒विनः॑ । प्रा॒णानिति॑ प्र - अ॒नान् । वै । ए॒तस्य॑ । शुक् । ऋ॒च्छ॒ति॒ । यस्य॑ । आ॒मय॑ति । प्रा॒णा इति॑ प्र - अ॒नाः । ग्रहाः᳚ । प्रा॒णानिति॑ प्र-अ॒नान् । ए॒व । अ॒स्य॒ । शु॒चः । मु॒ञ्च॒ति॒ । उ॒त । यदि॑ । इ॒तासु॒रिती॒त - अ॒सुः॒ । भव॑ति । जीव॑ति । ए॒व । पू॒र्णान् । ग्रहान्॑ ( ) । गृ॒ह्णी॒या॒त् । यर्.हि॑ । प॒र्जन्यः॑ । न । वर्.ष᳚त् । प्रा॒णानिति॑ प्र - अ॒नान् । वै । ए॒तर्.हि॑ । प्र॒जाना॒मिति॑ प्र - जाना᳚म् । शुक् । ऋ॒च्छ॒ति॒ । यर्.हि॑ । प॒र्जन्यः॑ । न । वर्.ष॑ति । प्रा॒णा इति॑ प्र - अ॒नाः । ग्रहाः᳚ । प्रा॒णानिति॑ प्र - अ॒नान् । ए॒व । प्र॒जाना॒मिति॑ प्र - जाना᳚म् । शु॒चः । मु॒ञ्च॒ति॒ । ता॒जक् । प्रेति॑ । व॒र्.ष॒ति॒ ॥  \newline


\textbf{Krama Paata} \newline

सर॑स्वती वा॒चा । वा॒चैव । ए॒वैन᳚म् । ए॒न॒मति॑ । अति॒ प्र । प्र यु॑ङ्‍क्ते । यु॒ङ्‍क्ते॒ मा । मा त्वत् । त्वत् क्षेत्रा॑णि । क्षेत्रा॒ण्यर॑णानि । अर॑णानि गन्म । ग॒न्मेति॑ । इत्या॑ह । आ॒ह॒ मृ॒त्योः । मृ॒त्योर् वै । वै क्षेत्रा॑णि । क्षेत्रा॒ण्यर॑णानि । अर॑णानि॒ तेन॑ । तेनै॒व । ए॒व मृ॒त्योः । मृ॒त्योः क्षेत्रा॑णि । क्षेत्रा॑णि॒ न । न ग॑च्छति । ग॒च्छ॒ति॒ पू॒र्णान् । पू॒र्णान् ग्रहान्॑ । ग्रहा᳚न् गृह्णीयात् । गृ॒ह्णी॒या॒दा॒म॒या॒विनः॑ । आ॒म॒या॒विनः॑ प्रा॒णान् । प्रा॒णान्. वै । प्रा॒णानिति॑ प्र - अ॒नान् । वा ए॒तस्य॑ । ए॒तस्य॒ शुक् । शुगृ॑च्छति । ऋ॒च्छ॒ति॒ यस्य॑ । यस्या॒मय॑ति । आ॒मय॑ति प्रा॒णाः । प्रा॒णा ग्रहाः᳚ । प्रा॒णा इति॑ प्र - अ॒नाः । ग्रहाः᳚ प्रा॒णान् । प्रा॒णाने॒व । प्रा॒णानिति॑ प्र - अ॒नान् । ए॒वास्य॑ । अ॒स्य॒ शु॒चः । शु॒चो मु॑ञ्चति । मु॒ञ्च॒त्यु॒त । उ॒त यदि॑ । यदी॒तासुः॑ । इ॒तासु॒र् भव॑ति । इ॒तासु॒रिती॒त - अ॒सुः॒ । भव॑ति॒ जीव॑ति । जीव॑त्ये॒व । ए॒व पू॒र्णान् । पू॒र्णान् ग्रहान्॑ ( ) । ग्रहा᳚न् गृह्णीयात् । गृ॒ह्णी॒या॒द् यर्.हि॑ । यर्.हि॑ प॒र्जन्यः॑ । प॒र्जन्यो॒ न । न वर्.षे᳚त् । वर्.षे᳚त् प्रा॒णान् । प्रा॒णान्. वै । प्रा॒णानिति॑ प्र - अ॒नान् । वा ए॒तर्.हि॑ । ए॒तर्.हि॑ प्र॒जाना᳚म् । प्र॒जानाꣳ॒॒ शुक् । प्र॒जाना॒मिति॑ प्र - जाना᳚म् । शुगृ॑च्छति । ऋ॒च्छ॒ति॒ यर्.हि॑ । यर्.हि॑ प॒र्जन्यः॑ । प॒र्जन्यो॒ न । न वर्.ष॑ति । वर्.ष॑ति प्रा॒णाः । प्रा॒णा ग्रहाः᳚ । प्रा॒णा इति॑ प्र - अ॒नाः । ग्रहाः᳚ प्रा॒णान् । प्रा॒णाने॒व । प्रा॒णानिति॑ प्र - अ॒नान् । ए॒व प्र॒जाना᳚म् । प्र॒जानाꣳ॑ शु॒चः । प्र॒जाना॒मिति॑ प्र - जाना᳚म् । शु॒चो मु॑ञ्चति । मु॒ञ्च॒ति॒ ता॒जक् । ता॒जक् प्र । प्र व॑र्.षति । व॒र्.॒ष॒तीति॑ वर्.षति । \newline

\textbf{Jatai Paata} \newline

1. सर॑स्वती वा॒चा वा॒चा सर॑स्वती॒ सर॑स्वती वा॒चा । \newline
2. वा॒चै वैव वा॒चा वा॒चैव । \newline
3. ए॒वैन॑ मेन मे॒वै वैन᳚म् । \newline
4. ए॒न॒ मत्य त्ये॑न मेन॒ मति॑ । \newline
5. अति॒ प्र प्रात्यति॒ प्र । \newline
6. प्र यु॑ङ्क्ते युङ्क्ते॒ प्र प्र यु॑ङ्क्ते । \newline
7. यु॒ङ्क्ते॒ मा मा यु॑ङ्क्ते युङ्क्ते॒ मा । \newline
8. मा त्वत् त्वन् मा मा त्वत् । \newline
9. त्वत् क्षेत्रा॑णि॒ क्षेत्रा॑णि॒ त्वत् त्वत् क्षेत्रा॑णि । \newline
10. क्षेत्रा॒ ण्यर॑णा॒ न्यर॑णानि॒ क्षेत्रा॑णि॒ क्षेत्रा॒ ण्यर॑णानि । \newline
11. अर॑णानि गन्म ग॒न्मा र॑णा॒ न्यर॑णानि गन्म । \newline
12. ग॒न्मेतीति॑ गन्म ग॒न्मेति॑ । \newline
13. इत्या॑हा॒हे तीत्या॑ह । \newline
14. आ॒ह॒ मृ॒त्योर् मृ॒त्यो रा॑हाह मृ॒त्योः । \newline
15. मृ॒त्योर् वै वै मृ॒त्योर् मृ॒त्योर् वै । \newline
16. वै क्षेत्रा॑णि॒ क्षेत्रा॑णि॒ वै वै क्षेत्रा॑णि । \newline
17. क्षेत्रा॒ ण्यर॑णा॒ न्यर॑णानि॒ क्षेत्रा॑णि॒ क्षेत्रा॒ ण्यर॑णानि । \newline
18. अर॑णानि॒ तेन॒ तेना र॑णा॒ न्यर॑णानि॒ तेन॑ । \newline
19. तेनै॒ वैव तेन॒ तेनै॒व । \newline
20. ए॒व मृ॒त्योर् मृ॒त्यो रे॒वैव मृ॒त्योः । \newline
21. मृ॒त्योः क्षेत्रा॑णि॒ क्षेत्रा॑णि मृ॒त्योर् मृ॒त्योः क्षेत्रा॑णि । \newline
22. क्षेत्रा॑णि॒ न न क्षेत्रा॑णि॒ क्षेत्रा॑णि॒ न । \newline
23. न ग॑च्छति गच्छति॒ न न ग॑च्छति । \newline
24. ग॒च्छ॒ति॒ पू॒र्णान् पू॒र्णान् ग॑च्छति गच्छति पू॒र्णान् । \newline
25. पू॒र्णान् ग्रहा॒न् ग्रहा᳚न् पू॒र्णान् पू॒र्णान् ग्रहान्॑ । \newline
26. ग्रहा᳚न् गृह्णीयाद् गृह्णीया॒द् ग्रहा॒न् ग्रहा᳚न् गृह्णीयात् । \newline
27. गृ॒ह्णी॒या॒ दा॒म॒या॒विन॑ आमया॒विनो॑ गृह्णीयाद् गृह्णीया दामया॒विनः॑ । \newline
28. आ॒म॒या॒विनः॑ प्रा॒णान् प्रा॒णा ना॑मया॒विन॑ आमया॒विनः॑ प्रा॒णान् । \newline
29. प्रा॒णान्. वै वै प्रा॒णान् प्रा॒णान्. वै । \newline
30. प्रा॒णानिति॑ प्र - अ॒नान् । \newline
31. वा ए॒त स्यै॒तस्य॒ वै वा ए॒तस्य॑ । \newline
32. ए॒तस्य॒ शुख् छुगे॒त स्यै॒तस्य॒ शुक् । \newline
33. शुगृ॑च्छ त्यृच्छति॒ शुख् छुगृ॑च्छति । \newline
34. ऋ॒च्छ॒ति॒ यस्य॒ यस्य॑ र्च्छत्यृच्छति॒ यस्य॑ । \newline
35. यस्या॒ मय॑ त्या॒मय॑ति॒ यस्य॒ यस्या॒ मय॑ति । \newline
36. आ॒मय॑ति प्रा॒णाः प्रा॒णा आ॒मय॑ त्या॒मय॑ति प्रा॒णाः । \newline
37. प्रा॒णा ग्रहा॒ ग्रहाः᳚ प्रा॒णाः प्रा॒णा ग्रहाः᳚ । \newline
38. प्रा॒णा इति॑ प्र - अ॒नाः । \newline
39. ग्रहाः᳚ प्रा॒णान् प्रा॒णान् ग्रहा॒ ग्रहाः᳚ प्रा॒णान् । \newline
40. प्रा॒णा ने॒वैव प्रा॒णान् प्रा॒णाने॒व । \newline
41. प्रा॒णानिति॑ प्र - अ॒नान् । \newline
42. ए॒वास्या᳚ स्यै॒वै वास्य॑ । \newline
43. अ॒स्य॒ शु॒चः शु॒चो᳚ ऽस्यास्य शु॒चः । \newline
44. शु॒चो मु॑ञ्चति मुञ्चति शु॒चः शु॒चो मु॑ञ्चति । \newline
45. मु॒ञ्च॒ त्यु॒तोत मु॑ञ्चति मुञ्च त्यु॒त । \newline
46. उ॒त यदि॒ यद्यु॒तोत यदि॑ । \newline
47. यदी॒तासु॑ रि॒तासु॒र् यदि॒ यदी॒तासुः॑ । \newline
48. इ॒तासु॒र् भव॑ति॒ भव॑ती॒तासु॑ रि॒तासु॒र् भव॑ति । \newline
49. इ॒तासु॒रिती॒त - अ॒सुः॒ । \newline
50. भव॑ति॒ जीव॑ति॒ जीव॑ति॒ भव॑ति॒ भव॑ति॒ जीव॑ति । \newline
51. जीव॑ त्ये॒वैव जीव॑ति॒ जीव॑त्ये॒व । \newline
52. ए॒व पू॒र्णान् पू॒र्णा ने॒वैव पू॒र्णान् । \newline
53. पू॒र्णान् ग्रहा॒न् ग्रहा᳚न् पू॒र्णान् पू॒र्णान् ग्रहान्॑ । \newline
54. ग्रहा᳚न् गृह्णीयाद् गृह्णीया॒द् ग्रहा॒न् ग्रहा᳚न् गृह्णीयात् । \newline
55. गृ॒ह्णी॒या॒द् यर्.हि॒ यर्.हि॑ गृह्णीयाद् गृह्णीया॒द् यर्.हि॑ । \newline
56. यर्.हि॑ प॒र्जन्यः॑ प॒र्जन्यो॒ यर्.हि॒ यर्.हि॑ प॒र्जन्यः॑ । \newline
57. प॒र्जन्यो॒ न न प॒र्जन्यः॑ प॒र्जन्यो॒ न । \newline
58. न वर्.षे॒त् वर्.षे॒त् न न वर्.षे᳚त् । \newline
59. वर्.षे᳚त् प्रा॒णान् प्रा॒णान्. वर्.षे॒त् वर्.षे᳚त् प्रा॒णान् । \newline
60. प्रा॒णान्. वै वै प्रा॒णान् प्रा॒णान्. वै । \newline
61. प्रा॒णानिति॑ प्र - अ॒नान् । \newline
62. वा ए॒तर् ह्ये॒तर्.हि॒ वै वा ए॒तर्.हि॑ । \newline
63. ए॒तर्.हि॑ प्र॒जाना᳚म् प्र॒जाना॑ मे॒तर् ह्ये॒तर्.हि॑ प्र॒जाना᳚म् । \newline
64. प्र॒जानाꣳ॒॒ शुख् छुक् प्र॒जाना᳚म् प्र॒जानाꣳ॒॒ शुक् । \newline
65. प्र॒जाना॒मिति॑ प्र - जाना᳚म् । \newline
66. शुगृ॑च्छ त्यृच्छति॒ शुख् छुगृ॑च्छति । \newline
67. ऋ॒च्छ॒ति॒ यर्.हि॒ यर्ह्यृ॑च्छ त्यृच्छति॒ यर्.हि॑ । \newline
68. यर्.हि॑ प॒र्जन्यः॑ प॒र्जन्यो॒ यर्.हि॒ यर्.हि॑ प॒र्जन्यः॑ । \newline
69. प॒र्जन्यो॒ न न प॒र्जन्यः॑ प॒र्जन्यो॒ न । \newline
70. न वर्.ष॑ति॒ वर्.ष॑ति॒ न न वर्.ष॑ति । \newline
71. वर्.ष॑ति प्रा॒णाः प्रा॒णा वर्.ष॑ति॒ वर्.ष॑ति प्रा॒णाः । \newline
72. प्रा॒णा ग्रहा॒ ग्रहाः᳚ प्रा॒णाः प्रा॒णा ग्रहाः᳚ । \newline
73. प्रा॒णा इति॑ प्र - अ॒नाः । \newline
74. ग्रहाः᳚ प्रा॒णान् प्रा॒णान् ग्रहा॒ ग्रहाः᳚ प्रा॒णान् । \newline
75. प्रा॒णा ने॒वैव प्रा॒णान् प्रा॒णाने॒व । \newline
76. प्रा॒णानिति॑ प्र - अ॒नान् । \newline
77. ए॒व प्र॒जाना᳚म् प्र॒जाना॑ मे॒वैव प्र॒जाना᳚म् । \newline
78. प्र॒जानाꣳ॑ शु॒चः शु॒चः प्र॒जाना᳚म् प्र॒जानाꣳ॑ शु॒चः । \newline
79. प्र॒जाना॒मिति॑ प्र - जाना᳚म् । \newline
80. शु॒चो मु॑ञ्चति मुञ्चति शु॒चः शु॒चो मु॑ञ्चति । \newline
81. मु॒ञ्च॒ति॒ ता॒जक् ता॒जङ् मु॑ञ्चति मुञ्चति ता॒जक् । \newline
82. ता॒जक् प्र प्र ता॒जक् ता॒जक् प्र । \newline
83. प्र व॑र्.षति वर्.षति॒ प्र प्र व॑र्.षति । \newline
84. व॒र्.॒ष॒तीति॑ वर्.षति । \newline

\textbf{Ghana Paata } \newline

1. सर॑स्वती वा॒चा वा॒चा सर॑स्वती॒ सर॑स्वती वा॒चै वैव वा॒चा सर॑स्वती॒ सर॑स्वती वा॒चैव । \newline
2. वा॒चै वैव वा॒चा वा॒चै वैन॑ मेन मे॒व वा॒चा वा॒चै वैन᳚म् । \newline
3. ए॒वैन॑ मेन मे॒वै वैन॒ मत्य त्ये॑न मे॒वै वैन॒ मति॑ । \newline
4. ए॒न॒ मत्य त्ये॑न मेन॒ मति॒ प्र प्रात्ये॑न मेन॒ मति॒ प्र । \newline
5. अति॒ प्र प्रात्यति॒ प्र यु॑ङ्क्ते युङ्क्ते॒ प्रात्यति॒ प्र यु॑ङ्क्ते । \newline
6. प्र यु॑ङ्क्ते युङ्क्ते॒ प्र प्र यु॑ङ्क्ते॒ मा मा यु॑ङ्क्ते॒ प्र प्र यु॑ङ्क्ते॒ मा । \newline
7. यु॒ङ्क्ते॒ मा मा यु॑ङ्क्ते युङ्क्ते॒ मा त्वत् त्वन् मा यु॑ङ्क्ते युङ्क्ते॒ मा त्वत् । \newline
8. मा त्वत् त्वन् मा मा त्वत् क्षेत्रा॑णि॒ क्षेत्रा॑णि॒ त्वन् मा मा त्वत् क्षेत्रा॑णि । \newline
9. त्वत् क्षेत्रा॑णि॒ क्षेत्रा॑णि॒ त्वत् त्वत् क्षेत्रा॒ ण्यर॑णा॒ न्यर॑णानि॒ क्षेत्रा॑णि॒ त्वत् त्वत् क्षेत्रा॒ ण्यर॑णानि । \newline
10. क्षेत्रा॒ ण्यर॑णा॒ न्यर॑णानि॒ क्षेत्रा॑णि॒ क्षेत्रा॒ ण्यर॑णानि गन्म ग॒न्मा र॑णानि॒ क्षेत्रा॑णि॒ क्षेत्रा॒ ण्यर॑णानि गन्म । \newline
11. अर॑णानि गन्म ग॒न्मार॑णा॒ न्यर॑णानि ग॒न्मेतीति॑ ग॒न्मार॑णा॒ न्यर॑णानि ग॒न्मेति॑ । \newline
12. ग॒न्मेतीति॑ गन्म ग॒न्मे त्या॑हा॒ हेति॑ गन्म ग॒न्मे त्या॑ह । \newline
13. इत्या॑हा॒हे तीत्या॑ह मृ॒त्योर् मृ॒त्यो रा॒हे तीत्या॑ह मृ॒त्योः । \newline
14. आ॒ह॒ मृ॒त्योर् मृ॒त्यो रा॑हाह मृ॒त्योर् वै वै मृ॒त्यो रा॑हाह मृ॒त्योर् वै । \newline
15. मृ॒त्योर् वै वै मृ॒त्योर् मृ॒त्योर् वै क्षेत्रा॑णि॒ क्षेत्रा॑णि॒ वै मृ॒त्योर् मृ॒त्योर् वै क्षेत्रा॑णि । \newline
16. वै क्षेत्रा॑णि॒ क्षेत्रा॑णि॒ वै वै क्षेत्रा॒ ण्यर॑णा॒ न्यर॑णानि॒ क्षेत्रा॑णि॒ वै वै क्षेत्रा॒ ण्यर॑णानि । \newline
17. क्षेत्रा॒ ण्यर॑णा॒ न्यर॑णानि॒ क्षेत्रा॑णि॒ क्षेत्रा॒ ण्यर॑णानि॒ तेन॒ तेना र॑णानि॒ क्षेत्रा॑णि॒ क्षेत्रा॒ ण्यर॑णानि॒ तेन॑ । \newline
18. अर॑णानि॒ तेन॒ तेना र॑णा॒ न्यर॑णानि॒ तेनै॒ वैव तेना र॑णा॒ न्यर॑णानि॒ तेनै॒व । \newline
19. तेनै॒ वैव तेन॒ तेनै॒व मृ॒त्योर् मृ॒त्यो रे॒व तेन॒ तेनै॒व मृ॒त्योः । \newline
20. ए॒व मृ॒त्योर् मृ॒त्यो रे॒वैव मृ॒त्योः क्षेत्रा॑णि॒ क्षेत्रा॑णि मृ॒त्यो रे॒वैव मृ॒त्योः क्षेत्रा॑णि । \newline
21. मृ॒त्योः क्षेत्रा॑णि॒ क्षेत्रा॑णि मृ॒त्योर् मृ॒त्योः क्षेत्रा॑णि॒ न न क्षेत्रा॑णि मृ॒त्योर् मृ॒त्योः क्षेत्रा॑णि॒ न । \newline
22. क्षेत्रा॑णि॒ न न क्षेत्रा॑णि॒ क्षेत्रा॑णि॒ न ग॑च्छति गच्छति॒ न क्षेत्रा॑णि॒ क्षेत्रा॑णि॒ न ग॑च्छति । \newline
23. न ग॑च्छति गच्छति॒ न न ग॑च्छति पू॒र्णान् पू॒र्णान् ग॑च्छति॒ न न ग॑च्छति पू॒र्णान् । \newline
24. ग॒च्छ॒ति॒ पू॒र्णान् पू॒र्णान् ग॑च्छति गच्छति पू॒र्णान् ग्रहा॒न् ग्रहा᳚न् पू॒र्णान् ग॑च्छति गच्छति पू॒र्णान् ग्रहान्॑ । \newline
25. पू॒र्णान् ग्रहा॒न् ग्रहा᳚न् पू॒र्णान् पू॒र्णान् ग्रहा᳚न् गृह्णीयाद् गृह्णीया॒द् ग्रहा᳚न् पू॒र्णान् पू॒र्णान् ग्रहा᳚न् गृह्णीयात् । \newline
26. ग्रहा᳚न् गृह्णीयाद् गृह्णीया॒द् ग्रहा॒न् ग्रहा᳚न् गृह्णीया दामया॒विन॑ आमया॒विनो॑ गृह्णीया॒द् ग्रहा॒न् ग्रहा᳚न् गृह्णीया दामया॒विनः॑ । \newline
27. गृ॒ह्णी॒या॒ दा॒म॒या॒विन॑ आमया॒विनो॑ गृह्णीयाद् गृह्णीया दामया॒विनः॑ प्रा॒णान् प्रा॒णा ना॑मया॒विनो॑ गृह्णीयाद् गृह्णीया दामया॒विनः॑ प्रा॒णान् । \newline
28. आ॒म॒या॒विनः॑ प्रा॒णान् प्रा॒णा ना॑मया॒विन॑ आमया॒विनः॑ प्रा॒णान्. वै वै प्रा॒णा ना॑मया॒विन॑ आमया॒विनः॑ प्रा॒णान्. वै । \newline
29. प्रा॒णान्. वै वै प्रा॒णान् प्रा॒णान्. वा ए॒त स्यै॒तस्य॒ वै प्रा॒णान् प्रा॒णान्. वा ए॒तस्य॑ । \newline
30. प्रा॒णानिति॑ प्र - अ॒नान् । \newline
31. वा ए॒त स्यै॒तस्य॒ वै वा ए॒तस्य॒ शुख् छुगे॒तस्य॒ वै वा ए॒तस्य॒ शुक् । \newline
32. ए॒तस्य॒ शुख् छुगे॒त स्यै॒तस्य॒ शुगृ॑च्छ त्यृच्छति॒ शुगे॒त स्यै॒तस्य॒ शुगृ॑च्छति । \newline
33. शुगृ॑च्छ त्यृच्छति॒ शुख् छुगृ॑च्छति॒ यस्य॒ यस्य॑ र्‌च्छति॒ शुख् छुगृ॑च्छति॒ यस्य॑ । \newline
34. ऋ॒च्छ॒ति॒ यस्य॒ यस्य॑ र्‌च्छ त्यृच्छति॒ यस्या॒मय॑ त्या॒मय॑ति॒ यस्य॑ र्‌च्छ त्यृच्छति॒ यस्या॒मय॑ति । \newline
35. यस्या॒ मय॑ त्या॒मय॑ति॒ यस्य॒ यस्या॒ मय॑ति प्रा॒णाः प्रा॒णा आ॒मय॑ति॒ यस्य॒ यस्या॒ मय॑ति प्रा॒णाः । \newline
36. आ॒मय॑ति प्रा॒णाः प्रा॒णा आ॒मय॑ त्या॒मय॑ति प्रा॒णा ग्रहा॒ ग्रहाः᳚ प्रा॒णा आ॒मय॑ त्या॒मय॑ति प्रा॒णा ग्रहाः᳚ । \newline
37. प्रा॒णा ग्रहा॒ ग्रहाः᳚ प्रा॒णाः प्रा॒णा ग्रहाः᳚ प्रा॒णान् प्रा॒णान् ग्रहाः᳚ प्रा॒णाः प्रा॒णा ग्रहाः᳚ प्रा॒णान् । \newline
38. प्रा॒णा इति॑ प्र - अ॒नाः । \newline
39. ग्रहाः᳚ प्रा॒णान् प्रा॒णान् ग्रहा॒ ग्रहाः᳚ प्रा॒णा ने॒वैव प्रा॒णान् ग्रहा॒ ग्रहाः᳚ प्रा॒णा ने॒व । \newline
40. प्रा॒णा ने॒वैव प्रा॒णान् प्रा॒णा ने॒वास्या᳚ स्यै॒व प्रा॒णान् प्रा॒णा ने॒वास्य॑ । \newline
41. प्रा॒णानिति॑ प्र - अ॒नान् । \newline
42. ए॒वास्या᳚ स्यै॒वै वास्य॑ शु॒चः शु॒चो᳚ ऽस्यै॒वै वास्य॑ शु॒चः । \newline
43. अ॒स्य॒ शु॒चः शु॒चो᳚ ऽस्यास्य शु॒चो मु॑ञ्चति मुञ्चति शु॒चो᳚ ऽस्यास्य शु॒चो मु॑ञ्चति । \newline
44. शु॒चो मु॑ञ्चति मुञ्चति शु॒चः शु॒चो मु॑ञ्च त्यु॒तोत मु॑ञ्चति शु॒चः शु॒चो मु॑ञ्च त्यु॒त । \newline
45. मु॒ञ्च॒ त्यु॒तोत मु॑ञ्चति मुञ्च त्यु॒त यदि॒ यद्यु॒त मु॑ञ्चति मुञ्च त्यु॒त यदि॑ । \newline
46. उ॒त यदि॒ यद्यु॒तोत यदी॒ तासु॑ रि॒तासु॒र् यद्यु॒तोत यदी॒ तासुः॑ । \newline
47. यदी॒ तासु॑ रि॒तासु॒र् यदि॒ यदी॒ तासु॒र् भव॑ति॒ भव॑ती॒ तासु॒र् यदि॒ यदी॒ तासु॒र् भव॑ति । \newline
48. इ॒तासु॒र् भव॑ति॒ भव॑ती॒ तासु॑ रि॒तासु॒र् भव॑ति॒ जीव॑ति॒ जीव॑ति॒ भव॑ ती॒तासु॑ रि॒तासु॒र् भव॑ति॒ जीव॑ति । \newline
49. इ॒तासु॒रिती॒त - अ॒सुः॒ । \newline
50. भव॑ति॒ जीव॑ति॒ जीव॑ति॒ भव॑ति॒ भव॑ति॒ जीव॑ त्ये॒वैव जीव॑ति॒ भव॑ति॒ भव॑ति॒ जीव॑ त्ये॒व । \newline
51. जीव॑ त्ये॒वैव जीव॑ति॒ जीव॑ त्ये॒व पू॒र्णान् पू॒र्णा ने॒व जीव॑ति॒ जीव॑ त्ये॒व पू॒र्णान् । \newline
52. ए॒व पू॒र्णान् पू॒र्णा ने॒वैव पू॒र्णान् ग्रहा॒न् ग्रहा᳚न् पू॒र्णा ने॒वैव पू॒र्णान् ग्रहान्॑ । \newline
53. पू॒र्णान् ग्रहा॒न् ग्रहा᳚न् पू॒र्णान् पू॒र्णान् ग्रहा᳚न् गृह्णीयाद् गृह्णीया॒द् ग्रहा᳚न् पू॒र्णान् पू॒र्णान् ग्रहा᳚न् गृह्णीयात् । \newline
54. ग्रहा᳚न् गृह्णीयाद् गृह्णीया॒द् ग्रहा॒न् ग्रहा᳚न् गृह्णीया॒द् यर्.हि॒ यर्.हि॑ गृह्णीया॒द् ग्रहा॒न् ग्रहा᳚न् गृह्णीया॒द् यर्.हि॑ । \newline
55. गृ॒ह्णी॒या॒द् यर्.हि॒ यर्.हि॑ गृह्णीयाद् गृह्णीया॒द् यर्.हि॑ प॒र्जन्यः॑ प॒र्जन्यो॒ यर्.हि॑ गृह्णीयाद् गृह्णीया॒द् यर्.हि॑ प॒र्जन्यः॑ । \newline
56. यर्.हि॑ प॒र्जन्यः॑ प॒र्जन्यो॒ यर्.हि॒ यर्.हि॑ प॒र्जन्यो॒ न न प॒र्जन्यो॒ यर्.हि॒ यर्.हि॑ प॒र्जन्यो॒ न । \newline
57. प॒र्जन्यो॒ न न प॒र्जन्यः॑ प॒र्जन्यो॒ न वर्.षे॒त् वर्.षे॒त् न प॒र्जन्यः॑ प॒र्जन्यो॒ न वर्.षे᳚त् । \newline
58. न वर्.षे॒त् वर्.षे॒त् न न वर्.षे᳚त् प्रा॒णान् प्रा॒णान्. वर्.षे॒त् न न वर्.षे᳚त् प्रा॒णान् । \newline
59. वर्.षे᳚त् प्रा॒णान् प्रा॒णान्. वर्.षे॒त् वर्.षे᳚त् प्रा॒णान्. वै वै प्रा॒णान्. वर्.षे॒त् वर्.षे᳚त् प्रा॒णान्. वै । \newline
60. प्रा॒णान्. वै वै प्रा॒णान् प्रा॒णान्. वा ए॒तर् ह्ये॒तर्.हि॒ वै प्रा॒णान् प्रा॒णान्. वा ए॒तर्.हि॑ । \newline
61. प्रा॒णानिति॑ प्र - अ॒नान् । \newline
62. वा ए॒तर् ह्ये॒तर्.हि॒ वै वा ए॒तर्.हि॑ प्र॒जाना᳚म् प्र॒जाना॑ मे॒तर्.हि॒ वै वा ए॒तर्.हि॑ प्र॒जाना᳚म् । \newline
63. ए॒तर्.हि॑ प्र॒जाना᳚म् प्र॒जाना॑ मे॒तर् ह्ये॒तर्.हि॑ प्र॒जानाꣳ॒॒ शुख् छुक् प्र॒जाना॑ मे॒तर् ह्ये॒तर्.हि॑ प्र॒जानाꣳ॒॒ शुक् । \newline
64. प्र॒जानाꣳ॒॒ शुख् छुक् प्र॒जाना᳚म् प्र॒जानाꣳ॒॒ शुगृ॑च्छ त्यृच्छति॒ शुक् प्र॒जाना᳚म् प्र॒जानाꣳ॒॒ शुगृ॑च्छति । \newline
65. प्र॒जाना॒मिति॑ प्र - जाना᳚म् । \newline
66. शुगृ॑च्छ त्यृच्छति॒ शुख् छुगृ॑च्छति॒ यर्.हि॒ यर्.ह्यृ॑च्छति॒ शुख् छुगृ॑च्छति॒ यर्.हि॑ । \newline
67. ऋ॒च्छ॒ति॒ यर्.हि॒ यर्.ह्यृ॑च्छ त्यृच्छति॒ यर्.हि॑ प॒र्जन्यः॑ प॒र्जन्यो॒ यर्.ह्यृ॑च्छ त्यृच्छति॒ यर्.हि॑ प॒र्जन्यः॑ । \newline
68. यर्.हि॑ प॒र्जन्यः॑ प॒र्जन्यो॒ यर्.हि॒ यर्.हि॑ प॒र्जन्यो॒ न न प॒र्जन्यो॒ यर्.हि॒ यर्.हि॑ प॒र्जन्यो॒ न । \newline
69. प॒र्जन्यो॒ न न प॒र्जन्यः॑ प॒र्जन्यो॒ न वर्.ष॑ति॒ वर्.ष॑ति॒ न प॒र्जन्यः॑ प॒र्जन्यो॒ न वर्.ष॑ति । \newline
70. न वर्.ष॑ति॒ वर्.ष॑ति॒ न न वर्.ष॑ति प्रा॒णाः प्रा॒णा वर्.ष॑ति॒ न न वर्.ष॑ति प्रा॒णाः । \newline
71. वर्.ष॑ति प्रा॒णाः प्रा॒णा वर्.ष॑ति॒ वर्.ष॑ति प्रा॒णा ग्रहा॒ ग्रहाः᳚ प्रा॒णा वर्.ष॑ति॒ वर्.ष॑ति प्रा॒णा ग्रहाः᳚ । \newline
72. प्रा॒णा ग्रहा॒ ग्रहाः᳚ प्रा॒णाः प्रा॒णा ग्रहाः᳚ प्रा॒णान् प्रा॒णान् ग्रहाः᳚ प्रा॒णाः प्रा॒णा ग्रहाः᳚ प्रा॒णान् । \newline
73. प्रा॒णा इति॑ प्र - अ॒नाः । \newline
74. ग्रहाः᳚ प्रा॒णान् प्रा॒णान् ग्रहा॒ ग्रहाः᳚ प्रा॒णा ने॒वैव प्रा॒णान् ग्रहा॒ ग्रहाः᳚ प्रा॒णा ने॒व । \newline
75. प्रा॒णा ने॒वैव प्रा॒णान् प्रा॒णा ने॒व प्र॒जाना᳚म् प्र॒जाना॑ मे॒व प्रा॒णान् प्रा॒णा ने॒व प्र॒जाना᳚म् । \newline
76. प्रा॒णानिति॑ प्र - अ॒नान् । \newline
77. ए॒व प्र॒जाना᳚म् प्र॒जाना॑ मे॒वैव प्र॒जानाꣳ॑ शु॒चः शु॒चः प्र॒जाना॑ मे॒वैव प्र॒जानाꣳ॑ शु॒चः । \newline
78. प्र॒जानाꣳ॑ शु॒चः शु॒चः प्र॒जाना᳚म् प्र॒जानाꣳ॑ शु॒चो मु॑ञ्चति मुञ्चति शु॒चः प्र॒जाना᳚म् प्र॒जानाꣳ॑ शु॒चो मु॑ञ्चति । \newline
79. प्र॒जाना॒मिति॑ प्र - जाना᳚म् । \newline
80. शु॒चो मु॑ञ्चति मुञ्चति शु॒चः शु॒चो मु॑ञ्चति ता॒जक् ता॒जङ् मु॑ञ्चति शु॒चः शु॒चो मु॑ञ्चति ता॒जक् । \newline
81. मु॒ञ्च॒ति॒ ता॒जक् ता॒जङ् मु॑ञ्चति मुञ्चति ता॒जक् प्र प्र ता॒जङ् मु॑ञ्चति मुञ्चति ता॒जक् प्र । \newline
82. ता॒जक् प्र प्र ता॒जक् ता॒जक् प्र व॑र्.षति वर्.षति॒ प्र ता॒जक् ता॒जक् प्र व॑र्.षति । \newline
83. प्र व॑र्.षति वर्.षति॒ प्र प्र व॑र्.षति । \newline
84. व॒र्.॒ष॒तीति॑ वर्.षति । \newline
\pagebreak
\markright{ TS 7.2.8.1  \hfill https://www.vedavms.in \hfill}

\section{ TS 7.2.8.1 }

\textbf{TS 7.2.8.1 } \newline
\textbf{Samhita Paata} \newline

गा॒य॒त्रो वा ऐ᳚न्द्रवाय॒वो गा॑य॒त्रं प्रा॑य॒णीय॒-मह॒स्तस्मा᳚त् प्राय॒णीये-ऽह॑न्नैन्द्रवाय॒वो गृ॑ह्यते॒ स्व ए॒वैन॑मा॒यत॑ने गृह्णाति॒ त्रैष्टु॑भो॒ वै शु॒क्रस्त्रैष्टु॑भं द्वि॒तीय॒-मह॒स्तस्मा᳚द् द्वि॒तीयेऽह॑ञ्छु॒क्रो गृ॑ह्यते॒ स्व ए॒वैन॑मा॒यत॑ने गृह्णाति॒ जाग॑तो॒ वा आ᳚ग्रय॒णो जाग॑तं तृ॒तीय॒-मह॒स्तस्मा᳚त् तृ॒तीयेऽह॑न्नाग्रय॒णो गृ॑ह्यते॒ स्व ए॒वैन॑मा॒यत॑ने गृह्णात्ये॒तद्वै - [  ] \newline

\textbf{Pada Paata} \newline

गा॒य॒त्रः । वै । ऐ॒न्द्र॒वा॒य॒व इत्यै᳚न्द्र - वा॒य॒वः । गा॒य॒त्रम् । प्रा॒य॒णीय॒मिति॑ प्र - अ॒य॒नीय᳚म् । अहः॑ । तस्मा᳚त् । प्रा॒य॒णीय॒ इति॑ प्र - अ॒य॒नीये᳚ । अहन्न्॑ । ऐ॒न्द्र॒वा॒य॒व इत्यै᳚न्द्र - वा॒य॒वः । गृ॒ह्य॒ते॒ । स्वे । ए॒व । ए॒न॒म् । आ॒यत॑न॒ इत्या᳚ - यत॑ने । गृ॒ह्णा॒ति॒ । त्रैष्टु॑भः । वै । शु॒क्रः । त्रैष्टु॑भम् । द्वि॒तीय᳚म् । अहः॑ । तस्मा᳚त् । द्वि॒तीये᳚ । अहन्न्॑ । शु॒क्रः । गृ॒ह्य॒ते॒ । स्वे । ए॒व । ए॒न॒म् । आ॒यत॑न॒ इत्या᳚ - यत॑ने । गृ॒ह्णा॒ति॒ । जाग॑तः । वै । आ॒ग्र॒य॒णः । जाग॑तम् । तृ॒तीय᳚म् । अहः॑ । तस्मा᳚त् । तृ॒तीये᳚ । अहन्न्॑ । आ॒ग्र॒य॒णः । गृ॒ह्य॒ते॒ । स्वे । ए॒व । ए॒न॒म् । आ॒यत॑न॒ इत्या᳚ - यत॑ने । गृ॒ह्णा॒ति॒ । ए॒तत् । वै ।  \newline


\textbf{Krama Paata} \newline

गा॒य॒त्रो वै । वा ऐ᳚न्द्रवाय॒वः । ऐ॒न्द्र॒वा॒य॒वो गा॑य॒त्रम् । ऐ॒न्द्र॒वा॒य॒व इत्यै᳚न्द्र - वा॒य॒वः । गा॒य॒त्रम् प्रा॑य॒णीय᳚म् । प्रा॒य॒णीय॒महः॑ । प्रा॒य॒णीय॒मिति॑ प्र - अ॒य॒नीय᳚म् । अह॒स्तस्मा᳚त् । तस्मा᳚त् प्राय॒णीये᳚ । प्रा॒य॒णीयेऽहन्न्॑ । प्रा॒य॒णीय॒ इति॑ प्र - अ॒य॒नीये᳚ । अह॑न्नैन्द्रवाय॒वः । ऐ॒न्द्र॒वा॒य॒वो गृ॑ह्यते । ऐ॒न्द्र॒वा॒य॒व इत्यै᳚न्द्र - वा॒य॒वः । गृ॒ह्य॒ते॒ स्वे । स्व ए॒व । ए॒वैन᳚म् । ए॒न॒मा॒यत॑ने । आ॒यत॑ने गृह्णाति । आ॒यत॑न॒ इत्या᳚ - यत॑ने । गृ॒ह्णा॒ति॒ त्रैष्टु॑भः । त्रैष्टु॑भो॒ वै । वै शु॒क्रः । शु॒क्रस्त्रैष्टु॑भम् । त्रैष्टु॑भम् द्वि॒तीय᳚म् । द्वि॒तीय॒महः॑ । अह॒स्तस्मा᳚त् । तस्मा᳚द् द्वि॒तीये᳚ । द्वि॒तीयेऽहन्न्॑ । अह॑ञ्छु॒क्रः । शु॒क्रो गृ॑ह्यते । गृ॒ह्य॒ते॒ स्वे । स्व ए॒व । ए॒वैन᳚म् । ए॒न॒मा॒यत॑ने । आ॒यत॑ने गृह्णाति । आ॒यत॑न॒ इत्या᳚ - यत॑ने । गृ॒ह्णा॒ति॒ जाग॑तः । जाग॑तो॒ वै । वा आ᳚ग्रय॒णः । आ॒ग्र॒य॒णो जाग॑तम् । जाग॑तम् तृ॒तीय᳚म् । तृ॒तीय॒महः॑ । अह॒स्तस्मा᳚त् । तस्मा᳚त् तृ॒तीये᳚ । तृ॒तीयेऽहन्न्॑ । अह॑न्नाग्रय॒णः । आ॒ग्र॒य॒णो गृ॑ह्यते । गृ॒ह्य॒ते॒ स्वे । स्व ए॒व । ए॒वैन᳚म् । ए॒न॒मा॒यत॑ने । आ॒यत॑ने गृह्णाति । आ॒यत॑न॒ इत्या᳚ - यत॑ने । गृ॒ह्णा॒त्ये॒तत् । ए॒तद् वै । वै य॒ज्ञ्म् \newline

\textbf{Jatai Paata} \newline

1. गा॒य॒त्रो वै वै गा॑य॒त्रो गा॑य॒त्रो वै । \newline
2. वा ऐ᳚न्द्रवाय॒व ऐ᳚न्द्रवाय॒वो वै वा ऐ᳚न्द्रवाय॒वः । \newline
3. ऐ॒न्द्र॒वा॒य॒वो गा॑य॒त्रम् गा॑य॒त्र मै᳚न्द्रवाय॒व ऐ᳚न्द्रवाय॒वो गा॑य॒त्रम् । \newline
4. ऐ॒न्द्र॒वा॒य॒व इत्यै᳚न्द्र - वा॒य॒वः । \newline
5. गा॒य॒त्रम् प्रा॑य॒णीय॑म् प्राय॒णीय॑म् गाय॒त्रम् गा॑य॒त्रम् प्रा॑य॒णीय᳚म् । \newline
6. प्रा॒य॒णीय॒ मह॒ रहः॑ प्राय॒णीय॑म् प्राय॒णीय॒ महः॑ । \newline
7. प्रा॒य॒णीय॒मिति॑ प्र - अ॒य॒नीय᳚म् । \newline
8. अह॒ स्तस्मा॒त् तस्मा॒ दह॒ रह॒ स्तस्मा᳚त् । \newline
9. तस्मा᳚त् प्राय॒णीये᳚ प्राय॒णीये॒ तस्मा॒त् तस्मा᳚त् प्राय॒णीये᳚ । \newline
10. प्रा॒य॒णीये ऽह॒न् नह॑न् प्राय॒णीये᳚ प्राय॒णीये ऽहन्न्॑ । \newline
11. प्रा॒य॒णीय॒ इति॑ प्र - अ॒य॒नीये᳚ । \newline
12. अह॑न् नैन्द्रवाय॒व ऐ᳚न्द्रवाय॒वो ऽह॒न् नह॑न् नैन्द्रवाय॒वः । \newline
13. ऐ॒न्द्र॒वा॒य॒वो गृ॑ह्यते गृह्यत ऐन्द्रवाय॒व ऐ᳚न्द्रवाय॒वो गृ॑ह्यते । \newline
14. ऐ॒न्द्र॒वा॒य॒व इत्यै᳚न्द्र - वा॒य॒वः । \newline
15. गृ॒ह्य॒ते॒ स्वे स्वे गृ॑ह्यते गृह्यते॒ स्वे । \newline
16. स्व ए॒वैव स्वे स्व ए॒व । \newline
17. ए॒वैन॑ मेन मे॒वै वैन᳚म् । \newline
18. ए॒न॒ मा॒यत॑न आ॒यत॑न एन मेन मा॒यत॑ने । \newline
19. आ॒यत॑ने गृह्णाति गृह्णा त्या॒यत॑न आ॒यत॑ने गृह्णाति । \newline
20. आ॒यत॑न॒ इत्या᳚ - यत॑ने । \newline
21. गृ॒ह्णा॒ति॒ त्रैष्टु॑भ॒ स्त्रैष्टु॑भो गृह्णाति गृह्णाति॒ त्रैष्टु॑भः । \newline
22. त्रैष्टु॑भो॒ वै वै त्रैष्टु॑भ॒ स्त्रैष्टु॑भो॒ वै । \newline
23. वै शु॒क्रः शु॒क्रो वै वै शु॒क्रः । \newline
24. शु॒क्र स्त्रैष्टु॑भ॒म् त्रैष्टु॑भꣳ शु॒क्रः शु॒क्र स्त्रैष्टु॑भम् । \newline
25. त्रैष्टु॑भम् द्वि॒तीय॑म् द्वि॒तीय॒म् त्रैष्टु॑भ॒म् त्रैष्टु॑भम् द्वि॒तीय᳚म् । \newline
26. द्वि॒तीय॒ मह॒ रह॑र् द्वि॒तीय॑म् द्वि॒तीय॒ महः॑ । \newline
27. अह॒ स्तस्मा॒त् तस्मा॒ दह॒ रह॒ स्तस्मा᳚त् । \newline
28. तस्मा᳚द् द्वि॒तीये᳚ द्वि॒तीये॒ तस्मा॒त् तस्मा᳚द् द्वि॒तीये᳚ । \newline
29. द्वि॒तीये ऽह॒न् नह॑न् द्वि॒तीये᳚ द्वि॒तीये ऽहन्न्॑ । \newline
30. अह॑ञ् छु॒क्रः शु॒क्रो ऽह॒न् नह॑ञ् छु॒क्रः । \newline
31. शु॒क्रो गृ॑ह्यते गृह्यते शु॒क्रः शु॒क्रो गृ॑ह्यते । \newline
32. गृ॒ह्य॒ते॒ स्वे स्वे गृ॑ह्यते गृह्यते॒ स्वे । \newline
33. स्व ए॒वैव स्वे स्व ए॒व । \newline
34. ए॒वैन॑ मेन मे॒वै वैन᳚म् । \newline
35. ए॒न॒ मा॒यत॑न आ॒यत॑न एन मेन मा॒यत॑ने । \newline
36. आ॒यत॑ने गृह्णाति गृह्णा त्या॒यत॑न आ॒यत॑ने गृह्णाति । \newline
37. आ॒यत॑न॒ इत्या᳚ - यत॑ने । \newline
38. गृ॒ह्णा॒ति॒ जाग॑तो॒ जाग॑तो गृह्णाति गृह्णाति॒ जाग॑तः । \newline
39. जाग॑तो॒ वै वै जाग॑तो॒ जाग॑तो॒ वै । \newline
40. वा आ᳚ग्रय॒ण आ᳚ग्रय॒णो वै वा आ᳚ग्रय॒णः । \newline
41. आ॒ग्र॒य॒णो जाग॑त॒म् जाग॑त माग्रय॒ण आ᳚ग्रय॒णो जाग॑तम् । \newline
42. जाग॑तम् तृ॒तीय॑म् तृ॒तीय॒म् जाग॑त॒म् जाग॑तम् तृ॒तीय᳚म् । \newline
43. तृ॒तीय॒ मह॒ रह॑ स्तृ॒तीय॑म् तृ॒तीय॒ महः॑ । \newline
44. अह॒ स्तस्मा॒त् तस्मा॒ दह॒ रह॒ स्तस्मा᳚त् । \newline
45. तस्मा᳚त् तृ॒तीये॑ तृ॒तीये॒ तस्मा॒त् तस्मा᳚त् तृ॒तीये᳚ । \newline
46. तृ॒तीये ऽह॒न् नह॑न् तृ॒तीये॑ तृ॒तीये ऽहन्न्॑ । \newline
47. अह॑न् नाग्रय॒ण आ᳚ग्रय॒णो ऽह॒न् नह॑न् नाग्रय॒णः । \newline
48. आ॒ग्र॒य॒णो गृ॑ह्यते गृह्यत आग्रय॒ण आ᳚ग्रय॒णो गृ॑ह्यते । \newline
49. गृ॒ह्य॒ते॒ स्वे स्वे गृ॑ह्यते गृह्यते॒ स्वे । \newline
50. स्व ए॒वैव स्वे स्व ए॒व । \newline
51. ए॒वैन॑ मेन मे॒वै वैन᳚म् । \newline
52. ए॒न॒ मा॒यत॑न आ॒यत॑न एन मेन मा॒यत॑ने । \newline
53. आ॒यत॑ने गृह्णाति गृह्णा त्या॒यत॑न आ॒यत॑ने गृह्णाति । \newline
54. आ॒यत॑न॒ इत्या᳚ - यत॑ने । \newline
55. गृ॒ह्णा॒ त्ये॒त दे॒तद् गृ॑ह्णाति गृह्णा त्ये॒तत् । \newline
56. ए॒तद् वै वा ए॒त दे॒तद् वै । \newline
57. वै य॒ज्ञ्ं ॅय॒ज्ञ्ं ॅवै वै य॒ज्ञ्म् । \newline

\textbf{Ghana Paata } \newline

1. गा॒य॒त्रो वै वै गा॑य॒त्रो गा॑य॒त्रो वा ऐ᳚न्द्रवाय॒व ऐ᳚न्द्रवाय॒वो वै गा॑य॒त्रो गा॑य॒त्रो वा ऐ᳚न्द्रवाय॒वः । \newline
2. वा ऐ᳚न्द्रवाय॒व ऐ᳚न्द्रवाय॒वो वै वा ऐ᳚न्द्रवाय॒वो गा॑य॒त्रम् गा॑य॒त्र मै᳚न्द्रवाय॒वो वै वा ऐ᳚न्द्रवाय॒वो गा॑य॒त्रम् । \newline
3. ऐ॒न्द्र॒वा॒य॒वो गा॑य॒त्रम् गा॑य॒त्र मै᳚न्द्रवाय॒व ऐ᳚न्द्रवाय॒वो गा॑य॒त्रम् प्रा॑य॒णीय॑म् प्राय॒णीय॑म् गाय॒त्र मै᳚न्द्रवाय॒व ऐ᳚न्द्रवाय॒वो गा॑य॒त्रम् प्रा॑य॒णीय᳚म् । \newline
4. ऐ॒न्द्र॒वा॒य॒व इत्यै᳚न्द्र - वा॒य॒वः । \newline
5. गा॒य॒त्रम् प्रा॑य॒णीय॑म् प्राय॒णीय॑म् गाय॒त्रम् गा॑य॒त्रम् प्रा॑य॒णीय॒ मह॒ रहः॑ प्राय॒णीय॑म् गाय॒त्रम् गा॑य॒त्रम् प्रा॑य॒णीय॒ महः॑ । \newline
6. प्रा॒य॒णीय॒ मह॒ रहः॑ प्राय॒णीय॑म् प्राय॒णीय॒ मह॒ स्तस्मा॒त् तस्मा॒ दहः॑ प्राय॒णीय॑म् प्राय॒णीय॒ मह॒ स्तस्मा᳚त् । \newline
7. प्रा॒य॒णीय॒मिति॑ प्र - अ॒य॒नीय᳚म् । \newline
8. अह॒ स्तस्मा॒त् तस्मा॒ दह॒ रह॒ स्तस्मा᳚त् प्राय॒णीये᳚ प्राय॒णीये॒ तस्मा॒ दह॒ रह॒ स्तस्मा᳚त् प्राय॒णीये᳚ । \newline
9. तस्मा᳚त् प्राय॒णीये᳚ प्राय॒णीये॒ तस्मा॒त् तस्मा᳚त् प्राय॒णीये ऽह॒न् नह॑न् प्राय॒णीये॒ तस्मा॒त् तस्मा᳚त् प्राय॒णीये ऽहन्न्॑ । \newline
10. प्रा॒य॒णीये ऽह॒न् नह॑न् प्राय॒णीये᳚ प्राय॒णीये ऽह॑न् नैन्द्रवाय॒व ऐ᳚न्द्रवाय॒वो ऽह॑न् प्राय॒णीये᳚ प्राय॒णीये ऽह॑न् नैन्द्रवाय॒वः । \newline
11. प्रा॒य॒णीय॒ इति॑ प्र - अ॒य॒नीये᳚ । \newline
12. अह॑न् नैन्द्रवाय॒व ऐ᳚न्द्रवाय॒वो ऽह॒न् नह॑न् नैन्द्रवाय॒वो गृ॑ह्यते गृह्यत ऐन्द्रवाय॒वो ऽह॒न् नह॑न् नैन्द्रवाय॒वो गृ॑ह्यते । \newline
13. ऐ॒न्द्र॒वा॒य॒वो गृ॑ह्यते गृह्यत ऐन्द्रवाय॒व ऐ᳚न्द्रवाय॒वो गृ॑ह्यते॒ स्वे स्वे गृ॑ह्यत ऐन्द्रवाय॒व ऐ᳚न्द्रवाय॒वो गृ॑ह्यते॒ स्वे । \newline
14. ऐ॒न्द्र॒वा॒य॒व इत्यै᳚न्द्र - वा॒य॒वः । \newline
15. गृ॒ह्य॒ते॒ स्वे स्वे गृ॑ह्यते गृह्यते॒ स्व ए॒वैव स्वे गृ॑ह्यते गृह्यते॒ स्व ए॒व । \newline
16. स्व ए॒वैव स्वे स्व ए॒वैन॑ मेन मे॒व स्वे स्व ए॒वैन᳚म् । \newline
17. ए॒वैन॑ मेन मे॒वै वैन॑ मा॒यत॑न आ॒यत॑न एन मे॒वै वैन॑ मा॒यत॑ने । \newline
18. ए॒न॒ मा॒यत॑न आ॒यत॑न एन मेन मा॒यत॑ने गृह्णाति गृह्णा त्या॒यत॑न एन मेन मा॒यत॑ने गृह्णाति । \newline
19. आ॒यत॑ने गृह्णाति गृह्णा त्या॒यत॑न आ॒यत॑ने गृह्णाति॒ त्रैष्टु॑भ॒ स्त्रैष्टु॑भो गृह्णा त्या॒यत॑न आ॒यत॑ने गृह्णाति॒ त्रैष्टु॑भः । \newline
20. आ॒यत॑न॒ इत्या᳚ - यत॑ने । \newline
21. गृ॒ह्णा॒ति॒ त्रैष्टु॑भ॒ स्त्रैष्टु॑भो गृह्णाति गृह्णाति॒ त्रैष्टु॑भो॒ वै वै त्रैष्टु॑भो गृह्णाति गृह्णाति॒ त्रैष्टु॑भो॒ वै । \newline
22. त्रैष्टु॑भो॒ वै वै त्रैष्टु॑भ॒ स्त्रैष्टु॑भो॒ वै शु॒क्रः शु॒क्रो वै त्रैष्टु॑भ॒ स्त्रैष्टु॑भो॒ वै शु॒क्रः । \newline
23. वै शु॒क्रः शु॒क्रो वै वै शु॒क्र स्त्रैष्टु॑भ॒म् त्रैष्टु॑भꣳ शु॒क्रो वै वै शु॒क्र स्त्रैष्टु॑भम् । \newline
24. शु॒क्र स्त्रैष्टु॑भ॒म् त्रैष्टु॑भꣳ शु॒क्रः शु॒क्र स्त्रैष्टु॑भम् द्वि॒तीय॑म् द्वि॒तीय॒म् त्रैष्टु॑भꣳ शु॒क्रः शु॒क्र स्त्रैष्टु॑भम् द्वि॒तीय᳚म् । \newline
25. त्रैष्टु॑भम् द्वि॒तीय॑म् द्वि॒तीय॒म् त्रैष्टु॑भ॒म् त्रैष्टु॑भम् द्वि॒तीय॒ मह॒ रह॑र् द्वि॒तीय॒म् त्रैष्टु॑भ॒म् त्रैष्टु॑भम् द्वि॒तीय॒ महः॑ । \newline
26. द्वि॒तीय॒ मह॒ रह॑र् द्वि॒तीय॑म् द्वि॒तीय॒ मह॒ स्तस्मा॒त् तस्मा॒ दह॑र् द्वि॒तीय॑म् द्वि॒तीय॒ मह॒ स्तस्मा᳚त् । \newline
27. अह॒ स्तस्मा॒त् तस्मा॒ दह॒ रह॒ स्तस्मा᳚द् द्वि॒तीये᳚ द्वि॒तीये॒ तस्मा॒ दह॒ रह॒ स्तस्मा᳚द् द्वि॒तीये᳚ । \newline
28. तस्मा᳚द् द्वि॒तीये᳚ द्वि॒तीये॒ तस्मा॒त् तस्मा᳚द् द्वि॒तीये ऽह॒न् नह॑न् द्वि॒तीये॒ तस्मा॒त् तस्मा᳚द् द्वि॒तीये ऽहन्न्॑ । \newline
29. द्वि॒तीये ऽह॒न् नह॑न् द्वि॒तीये᳚ द्वि॒तीये ऽह॑ञ् छु॒क्रः शु॒क्रो ऽह॑न् द्वि॒तीये᳚ द्वि॒तीये ऽह॑ञ् छु॒क्रः । \newline
30. अह॑ञ् छु॒क्रः शु॒क्रो ऽह॒न् नह॑ञ् छु॒क्रो गृ॑ह्यते गृह्यते शु॒क्रो ऽह॒न् नह॑ञ् छु॒क्रो गृ॑ह्यते । \newline
31. शु॒क्रो गृ॑ह्यते गृह्यते शु॒क्रः शु॒क्रो गृ॑ह्यते॒ स्वे स्वे गृ॑ह्यते शु॒क्रः शु॒क्रो गृ॑ह्यते॒ स्वे । \newline
32. गृ॒ह्य॒ते॒ स्वे स्वे गृ॑ह्यते गृह्यते॒ स्व ए॒वैव स्वे गृ॑ह्यते गृह्यते॒ स्व ए॒व । \newline
33. स्व ए॒वैव स्वे स्व ए॒वैन॑ मेन मे॒व स्वे स्व ए॒वैन᳚म् । \newline
34. ए॒वैन॑ मेन मे॒वै वैन॑ मा॒यत॑न आ॒यत॑न एन मे॒वै वैन॑ मा॒यत॑ने । \newline
35. ए॒न॒ मा॒यत॑न आ॒यत॑न एन मेन मा॒यत॑ने गृह्णाति गृह्णा त्या॒यत॑न एन मेन मा॒यत॑ने गृह्णाति । \newline
36. आ॒यत॑ने गृह्णाति गृह्णा त्या॒यत॑न आ॒यत॑ने गृह्णाति॒ जाग॑तो॒ जाग॑तो गृह्णा त्या॒यत॑न आ॒यत॑ने गृह्णाति॒ जाग॑तः । \newline
37. आ॒यत॑न॒ इत्या᳚ - यत॑ने । \newline
38. गृ॒ह्णा॒ति॒ जाग॑तो॒ जाग॑तो गृह्णाति गृह्णाति॒ जाग॑तो॒ वै वै जाग॑तो गृह्णाति गृह्णाति॒ जाग॑तो॒ वै । \newline
39. जाग॑तो॒ वै वै जाग॑तो॒ जाग॑तो॒ वा आ᳚ग्रय॒ण आ᳚ग्रय॒णो वै जाग॑तो॒ जाग॑तो॒ वा आ᳚ग्रय॒णः । \newline
40. वा आ᳚ग्रय॒ण आ᳚ग्रय॒णो वै वा आ᳚ग्रय॒णो जाग॑त॒म् जाग॑त माग्रय॒णो वै वा आ᳚ग्रय॒णो जाग॑तम् । \newline
41. आ॒ग्र॒य॒णो जाग॑त॒म् जाग॑त माग्रय॒ण आ᳚ग्रय॒णो जाग॑तम् तृ॒तीय॑म् तृ॒तीय॒म् जाग॑त माग्रय॒ण आ᳚ग्रय॒णो जाग॑तम् तृ॒तीय᳚म् । \newline
42. जाग॑तम् तृ॒तीय॑म् तृ॒तीय॒म् जाग॑त॒म् जाग॑तम् तृ॒तीय॒ मह॒ रह॑ स्तृ॒तीय॒म् जाग॑त॒म् जाग॑तम् तृ॒तीय॒ महः॑ । \newline
43. तृ॒तीय॒ मह॒ रह॑ स्तृ॒तीय॑म् तृ॒तीय॒ मह॒ स्तस्मा॒त् तस्मा॒ दह॑ स्तृ॒तीय॑म् तृ॒तीय॒ मह॒ स्तस्मा᳚त् । \newline
44. अह॒ स्तस्मा॒त् तस्मा॒ दह॒ रह॒ स्तस्मा᳚त् तृ॒तीये॑ तृ॒तीये॒ तस्मा॒ दह॒ रह॒ स्तस्मा᳚त् तृ॒तीये᳚ । \newline
45. तस्मा᳚त् तृ॒तीये॑ तृ॒तीये॒ तस्मा॒त् तस्मा᳚त् तृ॒तीये ऽह॒न् नह॑न् तृ॒तीये॒ तस्मा॒त् तस्मा᳚त् तृ॒तीये ऽहन्न्॑ । \newline
46. तृ॒तीये ऽह॒न् नह॑न् तृ॒तीये॑ तृ॒तीये ऽह॑न् नाग्रय॒ण आ᳚ग्रय॒णो ऽह॑न् तृ॒तीये॑ तृ॒तीये ऽह॑न् नाग्रय॒णः । \newline
47. अह॑न् नाग्रय॒ण आ᳚ग्रय॒णो ऽह॒न् नह॑न् नाग्रय॒णो गृ॑ह्यते गृह्यत आग्रय॒णो ऽह॒न् नह॑न् नाग्रय॒णो गृ॑ह्यते । \newline
48. आ॒ग्र॒य॒णो गृ॑ह्यते गृह्यत आग्रय॒ण आ᳚ग्रय॒णो गृ॑ह्यते॒ स्वे स्वे गृ॑ह्यत आग्रय॒ण आ᳚ग्रय॒णो गृ॑ह्यते॒ स्वे । \newline
49. गृ॒ह्य॒ते॒ स्वे स्वे गृ॑ह्यते गृह्यते॒ स्व ए॒वैव स्वे गृ॑ह्यते गृह्यते॒ स्व ए॒व । \newline
50. स्व ए॒वैव स्वे स्व ए॒वैन॑ मेन मे॒व स्वे स्व ए॒वैन᳚म् । \newline
51. ए॒वैन॑ मेन मे॒वै वैन॑ मा॒यत॑न आ॒यत॑न एन मे॒वै वैन॑ मा॒यत॑ने । \newline
52. ए॒न॒ मा॒यत॑न आ॒यत॑न एन मेन मा॒यत॑ने गृह्णाति गृह्णा त्या॒यत॑न एन मेन मा॒यत॑ने गृह्णाति । \newline
53. आ॒यत॑ने गृह्णाति गृह्णा त्या॒यत॑न आ॒यत॑ने गृह्णा त्ये॒त दे॒तद् गृ॑ह्णा त्या॒यत॑न आ॒यत॑ने गृह्णा त्ये॒तत् । \newline
54. आ॒यत॑न॒ इत्या᳚ - यत॑ने । \newline
55. गृ॒ह्णा॒ त्ये॒त दे॒तद् गृ॑ह्णाति गृह्णा त्ये॒तद् वै वा ए॒तद् गृ॑ह्णाति गृह्णा त्ये॒तद् वै । \newline
56. ए॒तद् वै वा ए॒त दे॒तद् वै य॒ज्ञ्ं ॅय॒ज्ञ्ं ॅवा ए॒त दे॒तद् वै य॒ज्ञ्म् । \newline
57. वै य॒ज्ञ्ं ॅय॒ज्ञ्ं ॅवै वै य॒ज्ञ् मा॑प दापद् य॒ज्ञ्ं ॅवै वै य॒ज्ञ् मा॑पत् । \newline
\pagebreak
\markright{ TS 7.2.8.2  \hfill https://www.vedavms.in \hfill}

\section{ TS 7.2.8.2 }

\textbf{TS 7.2.8.2 } \newline
\textbf{Samhita Paata} \newline

य॒ज्ञ्मा॑प॒द्-यच्छन्दाꣳ॑स्या॒प्नोति॒ यदा᳚ग्रय॒णः श्वो गृ॒ह्यते॒ यत्रै॒व य॒ज्ञ्मदृ॑श॒न् तत॑ ए॒वैनं॒ पुनः॒ प्रयु॑ङ्क्ते॒ जग॑न्मुखो॒ वै द्वि॒तीय॑स्त्रिरा॒त्रो जाग॑त आग्रय॒णो यच्च॑तु॒र्थेऽह॑न्नाग्रय॒णो गृ॒ह्यते॒ स्व ए॒वैन॑मा॒यत॑ने गृह्णा॒त्यथो॒ स्वमे॒व छन्दोऽनु॑ प॒र्याव॑र्तन्ते॒ राथ॑न्तरो॒ वा ऐ᳚न्द्रवाय॒वो राथ॑न्तरं पञ्च॒ममह॒स्तस्मा᳚त् पञ्च॒मेऽह॑ - [  ] \newline

\textbf{Pada Paata} \newline

य॒ज्ञ्म् । आ॒प॒त् । यत् । छन्दाꣳ॑सि । आ॒प्नोति॑ । यत् । आ॒ग्र॒य॒णः । श्वः । गृ॒ह्यते᳚ । यत्र॑ । ए॒व । य॒ज्ञ्म् । अदृ॑शन्न् । ततः॑ । ए॒व । ए॒न॒म् । पुनः॑ । प्रेति॑ । यु॒ङ्क्ते॒ । जग॑न्मुख॒ इति॒ जग॑त् - मु॒खः॒ । वै । द्वि॒तीयः॑ । त्रि॒रा॒त्र इति॑ त्रि - रा॒त्रः । जाग॑तः । आ॒ग्र॒य॒णः । यत् । च॒तु॒र्थे । अहन्न्॑ । आ॒ग्र॒य॒णः । गृ॒ह्यते᳚ । स्वे । ए॒व । ए॒न॒म् । आ॒यत॑न॒ इत्या᳚ - यत॑ने । गृ॒ह्णा॒ति॒ । अथो॒ इति॑ । स्वम् । ए॒व । छन्दः॑ । अन्विति॑ । प॒र्याव॑र्तन्त॒ इति॑ परि-आव॑र्तन्ते । राथ॑न्तर॒ इति॒ राथं᳚-त॒रः॒ । वै । ऐ॒न्द्र॒वा॒य॒व इत्यै᳚न्द्र - वा॒य॒वः । राथ॑न्तर॒मिति॒ राथं᳚ - त॒र॒म् । प॒ञ्च॒मम् । अहः॑ । तस्मा᳚त् । प॒ञ्च॒मे । अहन्न्॑ ।  \newline


\textbf{Krama Paata} \newline

य॒ज्ञ्मा॑पत् । आ॒प॒द् यत् । यच् छन्दाꣳ॑सि । छन्दाꣳ॑स्या॒प्नोति॑ । आ॒प्नोति॒ यत् । यदा᳚ग्रय॒णः । आ॒ग्र॒य॒णः श्वः । श्वो गृ॒ह्यते᳚ । गृ॒ह्यते॒ यत्र॑ । यत्रै॒व । ए॒व य॒ज्ञ्म् । य॒ज्ञ्मदृ॑शन्न् । अदृ॑श॒न् ततः॑ । तत॑ ए॒व । ए॒वैन᳚म् । ए॒न॒म् पुनः॑ । पुनः॒ प्र । प्र यु॑ङ्‍क्ते । यु॒ङ्‍क्ते॒ जग॑न्मुखः । जग॑न्मुखो॒ वै । जग॑न्मुख॒ इति॒ जग॑त् - मु॒खः॒ । वै द्वि॒तीयः॑ । द्वि॒तीय॑स्त्रिरा॒त्रः । त्रि॒रा॒त्रो जाग॑तः । त्रि॒रा॒त्र इति॑ त्रि - रा॒त्रः । जाग॑त आग्रय॒णः । आ॒ग्र॒य॒णो यत् । यच् च॑तु॒र्थे । च॒तु॒र्थेऽहन्न्॑ । अह॑न्नाग्रय॒णः । आ॒ग्र॒य॒णो गृ॒ह्यते᳚ । गृ॒ह्यते॒ स्वे । स्व ए॒व । ए॒वैन᳚म् । ए॒न॒मा॒यत॑ने । आ॒यत॑ने गृह्णाति । आ॒यत॑न॒ इत्या᳚ - यत॑ने । गृ॒ह्णा॒त्यथो᳚ । अथो॒ स्वम् । अथो॒ इत्यथो᳚ । स्वमे॒व । ए॒व छन्दः॑ । छन्दोऽनु॑ । अनु॑ प॒र्याव॑र्तन्ते । प॒र्याव॑र्तन्ते॒ राथ॑न्तरः । प॒र्याव॑र्तन्त॒ इति॑ परि - आव॑र्तन्ते । राथ॑न्तरो॒ वै । राथ॑न्तर॒ इति॒ राथ᳚म् - त॒रः॒ । वा ऐ᳚न्द्रवाय॒वः । ऐ॒न्द्र॒वा॒य॒वो राथ॑न्तरम् । ऐ॒न्द्र॒वा॒य॒व इत्यै᳚न्द्र - वा॒य॒वः । राथ॑न्तरम् पञ्च॒मम् । राथ॑न्तर॒मिति॒ राथ᳚म् - त॒र॒म् । प॒ञ्च॒ममहः॑ । अह॒स्तस्मा᳚त् । तस्मा᳚त् पञ्च॒मे । प॒ञ्च॒मेऽहन्न्॑ । अह॑न्नैन्द्रवाय॒वः \newline

\textbf{Jatai Paata} \newline

1. य॒ज्ञ् मा॑प दापद् य॒ज्ञ्ं ॅय॒ज्ञ् मा॑पत् । \newline
2. आ॒प॒द् यद् यदा॑प दाप॒द् यत् । \newline
3. यच् छन्दाꣳ॑सि॒ छन्दाꣳ॑सि॒ यद् यच् छन्दाꣳ॑सि । \newline
4. छन्दाꣳ॑ स्या॒प्नो त्या॒प्नोति॒ छन्दाꣳ॑सि॒ छन्दाꣳ॑ स्या॒प्नोति॑ । \newline
5. आ॒प्नोति॒ यद् यदा॒प्नो त्या॒प्नोति॒ यत् । \newline
6. यदा᳚ग्रय॒ण आ᳚ग्रय॒णो यद् यदा᳚ग्रय॒णः । \newline
7. आ॒ग्र॒य॒णः श्वः श्व आ᳚ग्रय॒ण आ᳚ग्रय॒णः श्वः । \newline
8. श्वो गृ॒ह्यते॑ गृ॒ह्यते॒ श्वः श्वो गृ॒ह्यते᳚ । \newline
9. गृ॒ह्यते॒ यत्र॒ यत्र॑ गृ॒ह्यते॑ गृ॒ह्यते॒ यत्र॑ । \newline
10. यत्रै॒ वैव यत्र॒ यत्रै॒व । \newline
11. ए॒व य॒ज्ञ्ं ॅय॒ज्ञ् मे॒वैव य॒ज्ञ्म् । \newline
12. य॒ज्ञ् मदृ॑श॒न् नदृ॑शन्. य॒ज्ञ्ं ॅय॒ज्ञ् मदृ॑शन्न् । \newline
13. अदृ॑श॒न् तत॒ स्ततो ऽदृ॑श॒न् नदृ॑श॒न् ततः॑ । \newline
14. तत॑ ए॒वैव तत॒ स्तत॑ ए॒व । \newline
15. ए॒वैन॑ मेन मे॒वै वैन᳚म् । \newline
16. ए॒न॒म् पुनः॒ पुन॑ रेन मेन॒म् पुनः॑ । \newline
17. पुनः॒ प्र प्र पुनः॒ पुनः॒ प्र । \newline
18. प्र यु॑ङ्क्ते युङ्क्ते॒ प्र प्र यु॑ङ्क्ते । \newline
19. यु॒ङ्क्ते॒ जग॑न्मुखो॒ जग॑न्मुखो युङ्क्ते युङ्क्ते॒ जग॑न्मुखः । \newline
20. जग॑न्मुखो॒ वै वै जग॑न्मुखो॒ जग॑न्मुखो॒ वै । \newline
21. जग॑न्मुख॒ इति॒ जग॑त् - मु॒खः॒ । \newline
22. वै द्वि॒तीयो᳚ द्वि॒तीयो॒ वै वै द्वि॒तीयः॑ । \newline
23. द्वि॒तीय॑ स्त्रिरा॒त्र स्त्रि॑रा॒त्रो द्वि॒तीयो᳚ द्वि॒तीय॑ स्त्रिरा॒त्रः । \newline
24. त्रि॒रा॒त्रो जाग॑तो॒ जाग॑त स्त्रिरा॒त्र स्त्रि॑रा॒त्रो जाग॑तः । \newline
25. त्रि॒रा॒त्र इति॑ त्रि - रा॒त्रः । \newline
26. जाग॑त आग्रय॒ण आ᳚ग्रय॒णो जाग॑तो॒ जाग॑त आग्रय॒णः । \newline
27. आ॒ग्र॒य॒णो यद् यदा᳚ग्रय॒ण आ᳚ग्रय॒णो यत् । \newline
28. यच् च॑तु॒र्थे च॑तु॒र्थे यद् यच् च॑तु॒र्थे । \newline
29. च॒तु॒र्थे ऽह॒न् नहꣳ॑ श्चतु॒र्थे च॑तु॒र्थे ऽहन्न्॑ । \newline
30. अह॑न् नाग्रय॒ण आ᳚ग्रय॒णो ऽह॒न् नह॑न् नाग्रय॒णः । \newline
31. आ॒ग्र॒य॒णो गृ॒ह्यते॑ गृ॒ह्यत॑ आग्रय॒ण आ᳚ग्रय॒णो गृ॒ह्यते᳚ । \newline
32. गृ॒ह्यते॒ स्वे स्वे गृ॒ह्यते॑ गृ॒ह्यते॒ स्वे । \newline
33. स्व ए॒वैव स्वे स्व ए॒व । \newline
34. ए॒वैन॑ मेन मे॒वै वैन᳚म् । \newline
35. ए॒न॒ मा॒यत॑न आ॒यत॑न एन मेन मा॒यत॑ने । \newline
36. आ॒यत॑ने गृह्णाति गृह्णा त्या॒यत॑न आ॒यत॑ने गृह्णाति । \newline
37. आ॒यत॑न॒ इत्या᳚ - यत॑ने । \newline
38. गृ॒ह्णा॒ त्यथो॒ अथो॑ गृह्णाति गृह्णा॒ त्यथो᳚ । \newline
39. अथो॒ स्वꣳ स्व मथो॒ अथो॒ स्वम् । \newline
40. अथो॒ इत्यथो᳚ । \newline
41. स्व मे॒वैव स्वꣳ स्व मे॒व । \newline
42. ए॒व छन्द॒ श्छन्द॑ ए॒वैव छन्दः॑ । \newline
43. छन्दो ऽन्वनु॒ च्छन्द॒ श्छन्दो ऽनु॑ । \newline
44. अनु॑ प॒र्याव॑र्तन्ते प॒र्याव॑र्त॒न्ते ऽन्वनु॑ प॒र्याव॑र्तन्ते । \newline
45. प॒र्याव॑र्तन्ते॒ राथ॑न्तरो॒ राथ॑न्तरः प॒र्याव॑र्तन्ते प॒र्याव॑र्तन्ते॒ राथ॑न्तरः । \newline
46. प॒र्याव॑र्तन्त॒ इति॑ परि - आव॑र्तन्ते । \newline
47. राथ॑न्तरो॒ वै वै राथ॑न्तरो॒ राथ॑न्तरो॒ वै । \newline
48. राथ॑न्तर॒ इति॒ राथं᳚ - त॒रः॒ । \newline
49. वा ऐ᳚न्द्रवाय॒व ऐ᳚न्द्रवाय॒वो वै वा ऐ᳚न्द्रवाय॒वः । \newline
50. ऐ॒न्द्र॒वा॒य॒वो राथ॑न्तरꣳ॒॒ राथ॑न्तर मैन्द्रवाय॒व ऐ᳚न्द्रवाय॒वो राथ॑न्तरम् । \newline
51. ऐ॒न्द्र॒वा॒य॒व इत्यै᳚न्द्र - वा॒य॒वः । \newline
52. राथ॑न्तरम् पञ्च॒मम् प॑ञ्च॒मꣳ राथ॑न्तरꣳ॒॒ राथ॑न्तरम् पञ्च॒मम् । \newline
53. राथ॑न्तर॒मिति॒ राथं᳚ - त॒र॒म् । \newline
54. प॒ञ्च॒म मह॒ रहः॑ पञ्च॒मम् प॑ञ्च॒म महः॑ । \newline
55. अह॒ स्तस्मा॒त् तस्मा॒ दह॒ रह॒ स्तस्मा᳚त् । \newline
56. तस्मा᳚त् पञ्च॒मे प॑ञ्च॒मे तस्मा॒त् तस्मा᳚त् पञ्च॒मे । \newline
57. प॒ञ्च॒मे ऽह॒न् नह॑न् पञ्च॒मे प॑ञ्च॒मे ऽहन्न्॑ । \newline
58. अह॑न् नैन्द्रवाय॒व ऐ᳚न्द्रवाय॒वो ऽह॒न् नह॑न् नैन्द्रवाय॒वः । \newline

\textbf{Ghana Paata } \newline

1. य॒ज्ञ् मा॑प दापद् य॒ज्ञ्ं ॅय॒ज्ञ् मा॑प॒द् यद् यदा॑पद् य॒ज्ञ्ं ॅय॒ज्ञ् मा॑प॒द् यत् । \newline
2. आ॒प॒द् यद् यदा॑प दाप॒द् यच् छन्दाꣳ॑सि॒ छन्दाꣳ॑सि॒ यदा॑प दाप॒द् यच् छन्दाꣳ॑सि । \newline
3. यच् छन्दाꣳ॑सि॒ छन्दाꣳ॑सि॒ यद् यच् छन्दाꣳ॑ स्या॒प्नो त्या॒प्नोति॒ छन्दाꣳ॑सि॒ यद् यच् छन्दाꣳ॑ स्या॒प्नोति॑ । \newline
4. छन्दाꣳ॑ स्या॒प्नो त्या॒प्नोति॒ छन्दाꣳ॑सि॒ छन्दाꣳ॑ स्या॒प्नोति॒ यद् यदा॒प्नोति॒ छन्दाꣳ॑सि॒ छन्दाꣳ॑ स्या॒प्नोति॒ यत् । \newline
5. आ॒प्नोति॒ यद् यदा॒प्नो त्या॒प्नोति॒ यदा᳚ग्रय॒ण आ᳚ग्रय॒णो यदा॒प्नो त्या॒प्नोति॒ यदा᳚ग्रय॒णः । \newline
6. यदा᳚ग्रय॒ण आ᳚ग्रय॒णो यद् यदा᳚ग्रय॒णः श्वः श्व आ᳚ग्रय॒णो यद् यदा᳚ग्रय॒णः श्वः । \newline
7. आ॒ग्र॒य॒णः श्वः श्व आ᳚ग्रय॒ण आ᳚ग्रय॒णः श्वो गृ॒ह्यते॑ गृ॒ह्यते॒ श्व आ᳚ग्रय॒ण आ᳚ग्रय॒णः श्वो गृ॒ह्यते᳚ । \newline
8. श्वो गृ॒ह्यते॑ गृ॒ह्यते॒ श्वः श्वो गृ॒ह्यते॒ यत्र॒ यत्र॑ गृ॒ह्यते॒ श्वः श्वो गृ॒ह्यते॒ यत्र॑ । \newline
9. गृ॒ह्यते॒ यत्र॒ यत्र॑ गृ॒ह्यते॑ गृ॒ह्यते॒ यत्रै॒ वैव यत्र॑ गृ॒ह्यते॑ गृ॒ह्यते॒ यत्रै॒व । \newline
10. यत्रै॒ वैव यत्र॒ यत्रै॒व य॒ज्ञ्ं ॅय॒ज्ञ् मे॒व यत्र॒ यत्रै॒व य॒ज्ञ्म् । \newline
11. ए॒व य॒ज्ञ्ं ॅय॒ज्ञ् मे॒वैव य॒ज्ञ् मदृ॑श॒न् नदृ॑शन्. य॒ज्ञ् मे॒ वैव य॒ज्ञ् मदृ॑शन्न् । \newline
12. य॒ज्ञ् मदृ॑श॒न् नदृ॑शन्. य॒ज्ञ्ं ॅय॒ज्ञ् मदृ॑श॒न् तत॒ स्ततो ऽदृ॑शन्. य॒ज्ञ्ं ॅय॒ज्ञ् मदृ॑श॒न् ततः॑ । \newline
13. अदृ॑श॒न् तत॒ स्ततो ऽदृ॑श॒न् नदृ॑श॒न् तत॑ ए॒वैव ततो ऽदृ॑श॒न् नदृ॑श॒न् तत॑ ए॒व । \newline
14. तत॑ ए॒वैव तत॒ स्तत॑ ए॒वैन॑ मेन मे॒व तत॒ स्तत॑ ए॒वैन᳚म् । \newline
15. ए॒वैन॑ मेन मे॒वै वैन॒म् पुनः॒ पुन॑ रेन मे॒वै वैन॒म् पुनः॑ । \newline
16. ए॒न॒म् पुनः॒ पुन॑ रेन मेन॒म् पुनः॒ प्र प्र पुन॑ रेन मेन॒म् पुनः॒ प्र । \newline
17. पुनः॒ प्र प्र पुनः॒ पुनः॒ प्र यु॑ङ्क्ते युङ्क्ते॒ प्र पुनः॒ पुनः॒ प्र यु॑ङ्क्ते । \newline
18. प्र यु॑ङ्क्ते युङ्क्ते॒ प्र प्र यु॑ङ्क्ते॒ जग॑न्मुखो॒ जग॑न्मुखो युङ्क्ते॒ प्र प्र यु॑ङ्क्ते॒ जग॑न्मुखः । \newline
19. यु॒ङ्क्ते॒ जग॑न्मुखो॒ जग॑न्मुखो युङ्क्ते युङ्क्ते॒ जग॑न्मुखो॒ वै वै जग॑न्मुखो युङ्क्ते युङ्क्ते॒ जग॑न्मुखो॒ वै । \newline
20. जग॑न्मुखो॒ वै वै जग॑न्मुखो॒ जग॑न्मुखो॒ वै द्वि॒तीयो᳚ द्वि॒तीयो॒ वै जग॑न्मुखो॒ जग॑न्मुखो॒ वै द्वि॒तीयः॑ । \newline
21. जग॑न्मुख॒ इति॒ जग॑त् - मु॒खः॒ । \newline
22. वै द्वि॒तीयो᳚ द्वि॒तीयो॒ वै वै द्वि॒तीय॑ स्त्रिरा॒त्र स्त्रि॑रा॒त्रो द्वि॒तीयो॒ वै वै द्वि॒तीय॑ स्त्रिरा॒त्रः । \newline
23. द्वि॒तीय॑ स्त्रिरा॒त्र स्त्रि॑रा॒त्रो द्वि॒तीयो᳚ द्वि॒तीय॑ स्त्रिरा॒त्रो जाग॑तो॒ जाग॑त स्त्रिरा॒त्रो द्वि॒तीयो᳚ द्वि॒तीय॑ स्त्रिरा॒त्रो जाग॑तः । \newline
24. त्रि॒रा॒त्रो जाग॑तो॒ जाग॑त स्त्रिरा॒त्र स्त्रि॑रा॒त्रो जाग॑त आग्रय॒ण आ᳚ग्रय॒णो जाग॑त स्त्रिरा॒त्र स्त्रि॑रा॒त्रो 
जाग॑त आग्रय॒णः । \newline
25. त्रि॒रा॒त्र इति॑ त्रि - रा॒त्रः । \newline
26. जाग॑त आग्रय॒ण आ᳚ग्रय॒णो जाग॑तो॒ जाग॑त आग्रय॒णो यद् यदा᳚ग्रय॒णो जाग॑तो॒ जाग॑त आग्रय॒णो यत् । \newline
27. आ॒ग्र॒य॒णो यद् यदा᳚ग्रय॒ण आ᳚ग्रय॒णो यच् च॑तु॒र्थे च॑तु॒र्थे यदा᳚ग्रय॒ण आ᳚ग्रय॒णो यच् च॑तु॒र्थे । \newline
28. यच् च॑तु॒र्थे च॑तु॒र्थे यद् यच् च॑तु॒र्थे ऽह॒न् नहꣳ॑ श्चतु॒र्थे यद् यच् च॑तु॒र्थे ऽहन्न्॑ । \newline
29. च॒तु॒र्थे ऽह॒न् नहꣳ॑ श्चतु॒र्थे च॑तु॒र्थे ऽह॑न् नाग्रय॒ण आ᳚ग्रय॒णो ऽहꣳ॑ श्चतु॒र्थे च॑तु॒र्थे ऽह॑न् नाग्रय॒णः । \newline
30. अह॑न् नाग्रय॒ण आ᳚ग्रय॒णो ऽह॒न् नह॑न् नाग्रय॒णो गृ॒ह्यते॑ गृ॒ह्यत॑ आग्रय॒णो ऽह॒न् नह॑न् नाग्रय॒णो गृ॒ह्यते᳚ । \newline
31. आ॒ग्र॒य॒णो गृ॒ह्यते॑ गृ॒ह्यत॑ आग्रय॒ण आ᳚ग्रय॒णो गृ॒ह्यते॒ स्वे स्वे गृ॒ह्यत॑ आग्रय॒ण आ᳚ग्रय॒णो गृ॒ह्यते॒ स्वे । \newline
32. गृ॒ह्यते॒ स्वे स्वे गृ॒ह्यते॑ गृ॒ह्यते॒ स्व ए॒वैव स्वे गृ॒ह्यते॑ गृ॒ह्यते॒ स्व ए॒व । \newline
33. स्व ए॒वैव स्वे स्व ए॒वैन॑ मेन मे॒व स्वे स्व ए॒वैन᳚म् । \newline
34. ए॒वैन॑ मेन मे॒वै वैन॑ मा॒यत॑न आ॒यत॑न एन मे॒वै वैन॑ मा॒यत॑ने । \newline
35. ए॒न॒ मा॒यत॑न आ॒यत॑न एन मेन मा॒यत॑ने गृह्णाति गृह्णा त्या॒यत॑न एन मेन मा॒यत॑ने गृह्णाति । \newline
36. आ॒यत॑ने गृह्णाति गृह्णा त्या॒यत॑न आ॒यत॑ने गृह्णा॒ त्यथो॒ अथो॑ गृह्णा त्या॒यत॑न आ॒यत॑ने गृह्णा॒ त्यथो᳚ । \newline
37. आ॒यत॑न॒ इत्या᳚ - यत॑ने । \newline
38. गृ॒ह्णा॒ त्यथो॒ अथो॑ गृह्णाति गृह्णा॒ त्यथो॒ स्वꣳ स्व मथो॑ गृह्णाति गृह्णा॒ त्यथो॒ स्वम् । \newline
39. अथो॒ स्वꣳ स्व मथो॒ अथो॒ स्व मे॒वैव स्व मथो॒ अथो॒ स्व मे॒व । \newline
40. अथो॒ इत्यथो᳚ । \newline
41. स्व मे॒वैव स्वꣳ स्व मे॒व छन्द॒ श्छन्द॑ ए॒व स्वꣳ स्व मे॒व छन्दः॑ । \newline
42. ए॒व छन्द॒ श्छन्द॑ ए॒वैव छन्दो ऽन्वनु॒ च्छन्द॑ ए॒वैव छन्दो ऽनु॑ । \newline
43. छन्दो ऽन्वनु॒ च्छन्द॒ श्छन्दो ऽनु॑ प॒र्याव॑र्तन्ते प॒र्याव॑र्त॒न्ते ऽनु॒ च्छन्द॒ श्छन्दो ऽनु॑ प॒र्याव॑र्तन्ते । \newline
44. अनु॑ प॒र्याव॑र्तन्ते प॒र्याव॑र्त॒न्ते ऽन्वनु॑ प॒र्याव॑र्तन्ते॒ राथ॑न्तरो॒ राथ॑न्तरः प॒र्याव॑र्त॒न्ते ऽन्वनु॑ प॒र्याव॑र्तन्ते॒ राथ॑न्तरः । \newline
45. प॒र्याव॑र्तन्ते॒ राथ॑न्तरो॒ राथ॑न्तरः प॒र्याव॑र्तन्ते प॒र्याव॑र्तन्ते॒ राथ॑न्तरो॒ वै वै राथ॑न्तरः प॒र्याव॑र्तन्ते प॒र्याव॑र्तन्ते॒ राथ॑न्तरो॒ वै । \newline
46. प॒र्याव॑र्तन्त॒ इति॑ परि - आव॑र्तन्ते । \newline
47. राथ॑न्तरो॒ वै वै राथ॑न्तरो॒ राथ॑न्तरो॒ वा ऐ᳚न्द्रवाय॒व ऐ᳚न्द्रवाय॒वो वै राथ॑न्तरो॒ राथ॑न्तरो॒ वा ऐ᳚न्द्रवाय॒वः । \newline
48. राथ॑न्तर॒ इति॒ राथं᳚ - त॒रः॒ । \newline
49. वा ऐ᳚न्द्रवाय॒व ऐ᳚न्द्रवाय॒वो वै वा ऐ᳚न्द्रवाय॒वो राथ॑न्तरꣳ॒॒ राथ॑न्तर मैन्द्रवाय॒वो वै वा ऐ᳚न्द्रवाय॒वो राथ॑न्तरम् । \newline
50. ऐ॒न्द्र॒वा॒य॒वो राथ॑न्तरꣳ॒॒ राथ॑न्तर मैन्द्रवाय॒व ऐ᳚न्द्रवाय॒वो राथ॑न्तरम् पञ्च॒मम् प॑ञ्च॒मꣳ राथ॑न्तर मैन्द्रवाय॒व ऐ᳚न्द्रवाय॒वो राथ॑न्तरम् पञ्च॒मम् । \newline
51. ऐ॒न्द्र॒वा॒य॒व इत्यै᳚न्द्र - वा॒य॒वः । \newline
52. राथ॑न्तरम् पञ्च॒मम् प॑ञ्च॒मꣳ राथ॑न्तरꣳ॒॒ राथ॑न्तरम् पञ्च॒म मह॒ रहः॑ पञ्च॒मꣳ राथ॑न्तरꣳ॒॒ राथ॑न्तरम् पञ्च॒म महः॑ । \newline
53. राथ॑न्तर॒मिति॒ राथं᳚ - त॒र॒म् । \newline
54. प॒ञ्च॒म मह॒ रहः॑ पञ्च॒मम् प॑ञ्च॒म मह॒ स्तस्मा॒त् तस्मा॒ दहः॑ पञ्च॒मम् प॑ञ्च॒म मह॒ स्तस्मा᳚त् । \newline
55. अह॒ स्तस्मा॒त् तस्मा॒ दह॒ रह॒ स्तस्मा᳚त् पञ्च॒मे प॑ञ्च॒मे तस्मा॒ दह॒ रह॒ स्तस्मा᳚त् पञ्च॒मे । \newline
56. तस्मा᳚त् पञ्च॒मे प॑ञ्च॒मे तस्मा॒त् तस्मा᳚त् पञ्च॒मे ऽह॒न् नह॑न् पञ्च॒मे तस्मा॒त् तस्मा᳚त् पञ्च॒मे ऽहन्न्॑ । \newline
57. प॒ञ्च॒मे ऽह॒न् नह॑न् पञ्च॒मे प॑ञ्च॒मे ऽह॑न् नैन्द्रवाय॒व ऐ᳚न्द्रवाय॒वो ऽह॑न् पञ्च॒मे प॑ञ्च॒मे ऽह॑न् नैन्द्रवाय॒वः । \newline
58. अह॑न् नैन्द्रवाय॒व ऐ᳚न्द्रवाय॒वो ऽह॒न् नह॑न् नैन्द्रवाय॒वो गृ॑ह्यते गृह्यत ऐन्द्रवाय॒वो ऽह॒न् नह॑न् नैन्द्रवाय॒वो गृ॑ह्यते । \newline
\pagebreak
\markright{ TS 7.2.8.3  \hfill https://www.vedavms.in \hfill}

\section{ TS 7.2.8.3 }

\textbf{TS 7.2.8.3 } \newline
\textbf{Samhita Paata} \newline

-न्नैन्द्रवाय॒वो गृ॑ह्यते॒ स्व ए॒वैन॑मा॒यत॑ने गृह्णाति॒ बार्.ह॑तो॒ वै शु॒क्रो बार्.ह॑तꣳ ष॒ष्ठमह॒स्तस्मा᳚थ् ष॒ष्ठेऽह॑ञ्छु॒क्रो गृ॑ह्यते॒ स्व ए॒वैन॑मा॒यत॑ने गृह्णात्ये॒तद्वै द्वि॒तीयं॑ ॅय॒ज्ञ्मा॑प॒द्-यच्छन्दाꣳ॑स्या॒प्नोति॒ यच्छु॒क्रः श्वो गृ॒ह्यते॒ यत्रै॒व य॒ज्ञ्मदृ॑श॒न् तत॑ ए॒वैनं॒ पुनः॒ प्रयु॑ङ्क्ते त्रि॒ष्टुङ्मु॑खो॒ वै तृ॒तीय॑स्त्रिरा॒त्रस्त्रैष्टु॑भः - [  ] \newline

\textbf{Pada Paata} \newline

ऐ॒न्द्र॒वा॒य॒व इत्यै᳚न्द्र - वा॒य॒वः । गृ॒ह्य॒ते॒ । स्वे । ए॒व । ए॒न॒म् । आ॒यत॑न॒ इत्या᳚ - यत॑ने । गृ॒ह्णा॒ति॒ । बार्.ह॑तः । वै । शु॒क्रः । बार्.ह॑तम् । ष॒ष्ठम् । अहः॑ । तस्मा᳚त् । ष॒ष्ठे । अहन्न्॑ । शु॒क्रः । गृ॒ह्य॒ते॒ । स्वे । ए॒व । ए॒न॒म् । आ॒यत॑न॒ इत्या᳚ - यत॑ने । गृ॒ह्णा॒ति॒ । ए॒तत् । वै । द्वि॒तीय᳚म् । य॒ज्ञ्म् । आ॒प॒त् । यत् । छन्दाꣳ॑सि । आ॒प्नोति॑ । यत् । शु॒क्रः । श्वः । गृ॒ह्यते᳚ । यत्र॑ । ए॒व । य॒ज्ञ्म् । अदृ॑शन्न् । ततः॑ । ए॒व । ए॒न॒म् । पुनः॑ । प्रेति॑ । यु॒ङ्क्ते॒ । त्रि॒ष्टुङ्मु॑ख॒ इति॑ त्रि॒ष्टुक् - मु॒खः॒ । वै । तृ॒तीयः॑ । त्रि॒रा॒त्र इति॑ त्रि - रा॒त्रः । त्रैष्टु॑भः ।  \newline


\textbf{Krama Paata} \newline

ऐ॒न्द्र॒वा॒य॒वो गृ॑ह्यते । ऐ॒न्द्र॒वा॒य॒व इत्यै᳚न्द्र - वा॒य॒वः । गृ॒ह्य॒ते॒ स्वे । स्व ए॒व । ए॒वैन᳚म् । ए॒न॒मा॒यत॑ने । आ॒यत॑ने गृह्णाति । आ॒यत॑न॒ इत्या᳚ - यत॑ने । गृ॒ह्णा॒ति॒ बार्.ह॑तः । बार्.ह॑तो॒ वै । वै शु॒क्रः । शु॒क्रो बार्.ह॑तम् । बार्.ह॑तꣳ ष॒ष्ठम् । ष॒ष्ठमहः॑ । अह॒स्तस्मा᳚त् । तस्मा᳚थ् ष॒ष्ठे । ष॒ष्ठेऽहन्न्॑ । अह॑ञ्छु॒क्रः । शु॒क्रो गृ॑ह्यते । गृ॒ह्य॒ते॒ स्वे । स्व ए॒व । ए॒वैन᳚म् । ए॒न॒मा॒यत॑ने । आ॒यत॑ने गृह्णाति । आ॒यत॑न॒ इत्या᳚ - यत॑ने । गृ॒ह्णा॒त्ये॒तत् । ए॒तद् वै । वै द्वि॒तीय᳚म् । द्वि॒तीय॑म् ॅय॒ज्ञ्म् । य॒ज्ञ्मा॑पत् । आ॒प॒द् यत् । यच् छन्दाꣳ॑सि । छन्दाꣳ॑स्या॒प्नोति॑ । आ॒प्नोति॒ यत् । यच्छु॒क्रः । शु॒क्रः श्वः । श्वो गृ॒ह्यते᳚ । गृ॒ह्यते॒ यत्र॑ । यत्रै॒व । ए॒व य॒ज्ञ्म् । य॒ज्ञ्मदृ॑शन्न् । अदृ॑श॒न् ततः॑ । तत॑ ए॒व । ए॒वैन᳚म् । ए॒न॒म् पुनः॑ । पुनः॒ प्र । प्र यु॑ङ्‍क्ते । यु॒ङ्‍क्ते॒ त्रि॒ष्टुङ्‍मु॑खः । त्रि॒ष्टुङ्‍मु॑खो॒ वै । त्रि॒ष्टुङ्‍मु॑ख॒ इति॑ त्रि॒ष्टुक् - मु॒खः॒ । वै तृ॒तीयः॑ । तृ॒तीय॑स्त्रिरा॒त्रः । त्रि॒रा॒त्रस्त्रैष्टु॑भः । त्रि॒रा॒त्र इति॑ त्रि - रा॒त्रः । त्रैष्टु॑भः शु॒क्रः \newline

\textbf{Jatai Paata} \newline

1. ऐ॒न्द्र॒वा॒य॒वो गृ॑ह्यते गृह्यत ऐन्द्रवाय॒व ऐ᳚न्द्रवाय॒वो गृ॑ह्यते । \newline
2. ऐ॒न्द्र॒वा॒य॒व इत्यै᳚न्द्र - वा॒य॒वः । \newline
3. गृ॒ह्य॒ते॒ स्वे स्वे गृ॑ह्यते गृह्यते॒ स्वे । \newline
4. स्व ए॒वैव स्वे स्व ए॒व । \newline
5. ए॒वैन॑ मेन मे॒वै वैन᳚म् । \newline
6. ए॒न॒ मा॒यत॑न आ॒यत॑न एन मेन मा॒यत॑ने । \newline
7. आ॒यत॑ने गृह्णाति गृह्णा त्या॒यत॑न आ॒यत॑ने गृह्णाति । \newline
8. आ॒यत॑न॒ इत्या᳚ - यत॑ने । \newline
9. गृ॒ह्णा॒ति॒ बार्.ह॑तो॒ बार्.ह॑तो गृह्णाति गृह्णाति॒ बार्.ह॑तः । \newline
10. बार्.ह॑तो॒ वै वै बार्.ह॑तो॒ बार्.ह॑तो॒ वै । \newline
11. वै शु॒क्रः शु॒क्रो वै वै शु॒क्रः । \newline
12. शु॒क्रो बार्.ह॑त॒म् बार्.ह॑तꣳ शु॒क्रः शु॒क्रो बार्.ह॑तम् । \newline
13. बार्.ह॑तꣳ ष॒ष्ठꣳ ष॒ष्ठम् बार्.ह॑त॒म् बार्.ह॑तꣳ ष॒ष्ठम् । \newline
14. ष॒ष्ठ मह॒ रह॑ ष्ष॒ष्ठꣳ ष॒ष्ठ महः॑ । \newline
15. अह॒ स्तस्मा॒त् तस्मा॒ दह॒ रह॒ स्तस्मा᳚त् । \newline
16. तस्मा᳚ थ्ष॒ष्ठे ष॒ष्ठे तस्मा॒त् तस्मा᳚ थ्ष॒ष्ठे । \newline
17. ष॒ष्ठे ऽह॒न् नहन्᳚ थ्ष॒ष्ठे ष॒ष्ठे ऽहन्न्॑ । \newline
18. अह॑ञ् छु॒क्रः शु॒क्रो ऽह॒न् नह॑ञ् छु॒क्रः । \newline
19. शु॒क्रो गृ॑ह्यते गृह्यते शु॒क्रः शु॒क्रो गृ॑ह्यते । \newline
20. गृ॒ह्य॒ते॒ स्वे स्वे गृ॑ह्यते गृह्यते॒ स्वे । \newline
21. स्व ए॒वैव स्वे स्व ए॒व । \newline
22. ए॒वैन॑ मेन मे॒वै वैन᳚म् । \newline
23. ए॒न॒ मा॒यत॑न आ॒यत॑न एन मेन मा॒यत॑ने । \newline
24. आ॒यत॑ने गृह्णाति गृह्णा त्या॒यत॑न आ॒यत॑ने गृह्णाति । \newline
25. आ॒यत॑न॒ इत्या᳚ - यत॑ने । \newline
26. गृ॒ह्णा॒ त्ये॒त दे॒तद् गृ॑ह्णाति गृह्णा त्ये॒तत् । \newline
27. ए॒तद् वै वा ए॒त दे॒तद् वै । \newline
28. वै द्वि॒तीय॑म् द्वि॒तीयं॒ ॅवै वै द्वि॒तीय᳚म् । \newline
29. द्वि॒तीयं॑ ॅय॒ज्ञ्ं ॅय॒ज्ञ्म् द्वि॒तीय॑म् द्वि॒तीयं॑ ॅय॒ज्ञ्म् । \newline
30. य॒ज्ञ् मा॑प दापद् य॒ज्ञ्ं ॅय॒ज्ञ् मा॑पत् । \newline
31. आ॒प॒द् यद् यदा॑प दाप॒द् यत् । \newline
32. यच् छन्दाꣳ॑सि॒ छन्दाꣳ॑सि॒ यद् यच् छन्दाꣳ॑सि । \newline
33. छन्दाꣳ॑ स्या॒प्नो त्या॒प्नोति॒ छन्दाꣳ॑सि॒ छन्दाꣳ॑ स्या॒प्नोति॑ । \newline
34. आ॒प्नोति॒ यद् यदा॒प्नो त्या॒प्नोति॒ यत् । \newline
35. यच्छु॒क्रः शु॒क्रो यद् यच्छु॒क्रः । \newline
36. शु॒क्रः श्वः श्वः शु॒क्रः शु॒क्रः श्वः । \newline
37. श्वो गृ॒ह्यते॑ गृ॒ह्यते॒ श्वः श्वो गृ॒ह्यते᳚ । \newline
38. गृ॒ह्यते॒ यत्र॒ यत्र॑ गृ॒ह्यते॑ गृ॒ह्यते॒ यत्र॑ । \newline
39. यत्रै॒ वैव यत्र॒ यत्रै॒व । \newline
40. ए॒व य॒ज्ञ्ं ॅय॒ज्ञ् मे॒वैव य॒ज्ञ्म् । \newline
41. य॒ज्ञ् मदृ॑श॒न् नदृ॑शन्. य॒ज्ञ्ं ॅय॒ज्ञ् मदृ॑शन्न् । \newline
42. अदृ॑श॒न् तत॒ स्ततो ऽदृ॑श॒न् नदृ॑श॒न् ततः॑ । \newline
43. तत॑ ए॒वैव तत॒ स्तत॑ ए॒व । \newline
44. ए॒वैन॑ मेन मे॒वै वैन᳚म् । \newline
45. ए॒न॒म् पुनः॒ पुन॑ रेन मेन॒म् पुनः॑ । \newline
46. पुनः॒ प्र प्र पुनः॒ पुनः॒ प्र । \newline
47. प्र यु॑ङ्क्ते युङ्क्ते॒ प्र प्र यु॑ङ्क्ते । \newline
48. यु॒ङ्क्ते॒ त्रि॒ष्टुङ्‌मु॑ख स्त्रि॒ष्टुङ्‌मु॑खो युङ्क्ते युङ्क्ते त्रि॒ष्टुङ्‌मु॑खः । \newline
49. त्रि॒ष्टुङ्‌मु॑खो॒ वै वै त्रि॒ष्टुङ्‌मु॑ख स्त्रि॒ष्टुङ्‌मु॑खो॒ वै । \newline
50. त्रि॒ष्टुङ्‌मु॑ख॒ इति॑ त्रि॒ष्टुक् - मु॒खः॒ । \newline
51. वै तृ॒तीय॑ स्तृ॒तीयो॒ वै वै तृ॒तीयः॑ । \newline
52. तृ॒तीय॑ स्त्रिरा॒त्र स्त्रि॑रा॒त्र स्तृ॒तीय॑ स्तृ॒तीय॑ स्त्रिरा॒त्रः । \newline
53. त्रि॒रा॒त्र स्त्रैष्टु॑भ॒ स्त्रैष्टु॑भ स्त्रिरा॒त्र स्त्रि॑रा॒त्र स्त्रैष्टु॑भः । \newline
54. त्रि॒रा॒त्र इति॑ त्रि - रा॒त्रः । \newline
55. त्रैष्टु॑भः शु॒क्रः शु॒क्र स्त्रैष्टु॑भ॒ स्त्रैष्टु॑भः शु॒क्रः । \newline

\textbf{Ghana Paata } \newline

1. ऐ॒न्द्र॒वा॒य॒वो गृ॑ह्यते गृह्यत ऐन्द्रवाय॒व ऐ᳚न्द्रवाय॒वो गृ॑ह्यते॒ स्वे स्वे गृ॑ह्यत ऐन्द्रवाय॒व ऐ᳚न्द्रवाय॒वो गृ॑ह्यते॒ स्वे । \newline
2. ऐ॒न्द्र॒वा॒य॒व इत्यै᳚न्द्र - वा॒य॒वः । \newline
3. गृ॒ह्य॒ते॒ स्वे स्वे गृ॑ह्यते गृह्यते॒ स्व ए॒वैव स्वे गृ॑ह्यते गृह्यते॒ स्व ए॒व । \newline
4. स्व ए॒वैव स्वे स्व ए॒वैन॑ मेन मे॒व स्वे स्व ए॒वैन᳚म् । \newline
5. ए॒वैन॑ मेन मे॒वै वैन॑ मा॒यत॑न आ॒यत॑न एन मे॒वै वैन॑ मा॒यत॑ने । \newline
6. ए॒न॒ मा॒यत॑न आ॒यत॑न एन मेन मा॒यत॑ने गृह्णाति गृह्णा त्या॒यत॑न एन मेन मा॒यत॑ने गृह्णाति । \newline
7. आ॒यत॑ने गृह्णाति गृह्णा त्या॒यत॑न आ॒यत॑ने गृह्णाति॒ बार्.ह॑तो॒ बार्.ह॑तो गृह्णा त्या॒यत॑न आ॒यत॑ने गृह्णाति॒ बार्.ह॑तः । \newline
8. आ॒यत॑न॒ इत्या᳚ - यत॑ने । \newline
9. गृ॒ह्णा॒ति॒ बार्.ह॑तो॒ बार्.ह॑तो गृह्णाति गृह्णाति॒ बार्.ह॑तो॒ वै वै बार्.ह॑तो गृह्णाति गृह्णाति॒ बार्.ह॑तो॒ वै । \newline
10. बार्.ह॑तो॒ वै वै बार्.ह॑तो॒ बार्.ह॑तो॒ वै शु॒क्रः शु॒क्रो वै बार्.ह॑तो॒ बार्.ह॑तो॒ वै शु॒क्रः । \newline
11. वै शु॒क्रः शु॒क्रो वै वै शु॒क्रो बार्.ह॑त॒म् बार्.ह॑तꣳ शु॒क्रो वै वै शु॒क्रो बार्.ह॑तम् । \newline
12. शु॒क्रो बार्.ह॑त॒म् बार्.ह॑तꣳ शु॒क्रः शु॒क्रो बार्.ह॑तꣳ ष॒ष्ठꣳ ष॒ष्ठम् बार्.ह॑तꣳ शु॒क्रः शु॒क्रो बार्.ह॑तꣳ ष॒ष्ठम् । \newline
13. बार्.ह॑तꣳ ष॒ष्ठꣳ ष॒ष्ठम् बार्.ह॑त॒म् बार्.ह॑तꣳ ष॒ष्ठ मह॒ रह॑ ष्ष॒ष्ठम् बार्.ह॑त॒म् बार्.ह॑तꣳ ष॒ष्ठ महः॑ । \newline
14. ष॒ष्ठ मह॒ रह॑ ष्ष॒ष्ठꣳ ष॒ष्ठ मह॒ स्तस्मा॒त् तस्मा॒ दह॑ ष्ष॒ष्ठꣳ ष॒ष्ठ मह॒ स्तस्मा᳚त् । \newline
15. अह॒ स्तस्मा॒त् तस्मा॒ दह॒ रह॒ स्तस्मा᳚थ् ष॒ष्ठे ष॒ष्ठे तस्मा॒ दह॒ रह॒ स्तस्मा᳚ थ्ष॒ष्ठे । \newline
16. तस्मा᳚थ् ष॒ष्ठे ष॒ष्ठे तस्मा॒त् तस्मा᳚ थ्ष॒ष्ठे ऽह॒न् नहन्᳚ थ्ष॒ष्ठे तस्मा॒त् तस्मा᳚ थ्ष॒ष्ठे ऽहन्न्॑ । \newline
17. ष॒ष्ठे ऽह॒न् नहन्᳚ थ्ष॒ष्ठे ष॒ष्ठे ऽह॑ञ् छु॒क्रः शु॒क्रो ऽहन्᳚ थ्ष॒ष्ठे ष॒ष्ठे ऽह॑ञ् छु॒क्रः । \newline
18. अह॑ञ् छु॒क्रः शु॒क्रो ऽह॒न् नह॑ञ् छु॒क्रो गृ॑ह्यते गृह्यते शु॒क्रो ऽह॒न् नह॑ञ् छु॒क्रो गृ॑ह्यते । \newline
19. शु॒क्रो गृ॑ह्यते गृह्यते शु॒क्रः शु॒क्रो गृ॑ह्यते॒ स्वे स्वे गृ॑ह्यते शु॒क्रः शु॒क्रो गृ॑ह्यते॒ स्वे । \newline
20. गृ॒ह्य॒ते॒ स्वे स्वे गृ॑ह्यते गृह्यते॒ स्व ए॒वैव स्वे गृ॑ह्यते गृह्यते॒ स्व ए॒व । \newline
21. स्व ए॒वैव स्वे स्व ए॒वैन॑ मेन मे॒व स्वे स्व ए॒वैन᳚म् । \newline
22. ए॒वैन॑ मेन मे॒वै वैन॑ मा॒यत॑न आ॒यत॑न एन मे॒वै वैन॑ मा॒यत॑ने । \newline
23. ए॒न॒ मा॒यत॑न आ॒यत॑न एन मेन मा॒यत॑ने गृह्णाति गृह्णा त्या॒यत॑न एन मेन मा॒यत॑ने गृह्णाति । \newline
24. आ॒यत॑ने गृह्णाति गृह्णा त्या॒यत॑न आ॒यत॑ने गृह्णा त्ये॒त दे॒तद् गृ॑ह्णा त्या॒यत॑न आ॒यत॑ने गृह्णा त्ये॒तत् । \newline
25. आ॒यत॑न॒ इत्या᳚ - यत॑ने । \newline
26. गृ॒ह्णा॒ त्ये॒त दे॒तद् गृ॑ह्णाति गृह्णा त्ये॒तद् वै वा ए॒तद् गृ॑ह्णाति गृह्णा त्ये॒तद् वै । \newline
27. ए॒तद् वै वा ए॒त दे॒तद् वै द्वि॒तीय॑म् द्वि॒तीयं॒ ॅवा ए॒त दे॒तद् वै द्वि॒तीय᳚म् । \newline
28. वै द्वि॒तीय॑म् द्वि॒तीयं॒ ॅवै वै द्वि॒तीयं॑ ॅय॒ज्ञ्ं ॅय॒ज्ञ्म् द्वि॒तीयं॒ ॅवै वै द्वि॒तीयं॑ ॅय॒ज्ञ्म् । \newline
29. द्वि॒तीयं॑ ॅय॒ज्ञ्ं ॅय॒ज्ञ्म् द्वि॒तीय॑म् द्वि॒तीयं॑ ॅय॒ज्ञ् मा॑प दापद् य॒ज्ञ्म् द्वि॒तीय॑म् द्वि॒तीयं॑ ॅय॒ज्ञ् मा॑पत् । \newline
30. य॒ज्ञ् मा॑प दापद् य॒ज्ञ्ं ॅय॒ज्ञ् मा॑प॒द् यद् यदा॑पद् य॒ज्ञ्ं ॅय॒ज्ञ् मा॑प॒द् यत् । \newline
31. आ॒प॒द् यद् यदा॑प दाप॒द् यच् छन्दाꣳ॑सि॒ छन्दाꣳ॑सि॒ यदा॑प दाप॒द् यच् छन्दाꣳ॑सि । \newline
32. यच् छन्दाꣳ॑सि॒ छन्दाꣳ॑सि॒ यद् यच् छन्दाꣳ॑ स्या॒प्नो त्या॒प्नोति॒ छन्दाꣳ॑सि॒ यद् यच् छन्दाꣳ॑ स्या॒प्नोति॑ । \newline
33. छन्दाꣳ॑ स्या॒प्नो त्या॒प्नोति॒ छन्दाꣳ॑सि॒ छन्दाꣳ॑ स्या॒प्नोति॒ यद् यदा॒प्नोति॒ छन्दाꣳ॑सि॒ छन्दाꣳ॑ स्या॒प्नोति॒ यत् । \newline
34. आ॒प्नोति॒ यद् यदा॒प्नो त्या॒प्नोति॒ यच्छु॒क्रः शु॒क्रो यदा॒प्नो त्या॒प्नोति॒ यच्छु॒क्रः । \newline
35. यच्छु॒क्रः शु॒क्रो यद् यच्छु॒क्रः श्वः श्वः शु॒क्रो यद् यच्छु॒क्रः श्वः । \newline
36. शु॒क्रः श्वः श्वः शु॒क्रः शु॒क्रः श्वो गृ॒ह्यते॑ गृ॒ह्यते॒ श्वः शु॒क्रः शु॒क्रः श्वो गृ॒ह्यते᳚ । \newline
37. श्वो गृ॒ह्यते॑ गृ॒ह्यते॒ श्वः श्वो गृ॒ह्यते॒ यत्र॒ यत्र॑ गृ॒ह्यते॒ श्वः श्वो गृ॒ह्यते॒ यत्र॑ । \newline
38. गृ॒ह्यते॒ यत्र॒ यत्र॑ गृ॒ह्यते॑ गृ॒ह्यते॒ यत्रै॒ वैव यत्र॑ गृ॒ह्यते॑ गृ॒ह्यते॒ यत्रै॒व । \newline
39. यत्रै॒ वैव यत्र॒ यत्रै॒व य॒ज्ञ्ं ॅय॒ज्ञ् मे॒व यत्र॒ यत्रै॒व य॒ज्ञ्म् । \newline
40. ए॒व य॒ज्ञ्ं ॅय॒ज्ञ् मे॒वैव य॒ज्ञ् मदृ॑श॒न् नदृ॑शन्. य॒ज्ञ् मे॒वैव य॒ज्ञ् मदृ॑शन्न् । \newline
41. य॒ज्ञ् मदृ॑श॒न् नदृ॑शन्. य॒ज्ञ्ं ॅय॒ज्ञ् मदृ॑श॒न् तत॒ स्ततो ऽदृ॑शन्. य॒ज्ञ्ं ॅय॒ज्ञ् मदृ॑श॒न् ततः॑ । \newline
42. अदृ॑श॒न् तत॒ स्ततो ऽदृ॑श॒न् नदृ॑श॒न् तत॑ ए॒वैव ततो ऽदृ॑श॒न् नदृ॑श॒न् तत॑ ए॒व । \newline
43. तत॑ ए॒वैव तत॒ स्तत॑ ए॒वैन॑ मेन मे॒व तत॒ स्तत॑ ए॒वैन᳚म् । \newline
44. ए॒वैन॑ मेन मे॒वै वैन॒म् पुनः॒ पुन॑ रेन मे॒वै वैन॒म् पुनः॑ । \newline
45. ए॒न॒म् पुनः॒ पुन॑ रेन मेन॒म् पुनः॒ प्र प्र पुन॑ रेन मेन॒म् पुनः॒ प्र । \newline
46. पुनः॒ प्र प्र पुनः॒ पुनः॒ प्र यु॑ङ्क्ते युङ्क्ते॒ प्र पुनः॒ पुनः॒ प्र यु॑ङ्क्ते । \newline
47. प्र यु॑ङ्क्ते युङ्क्ते॒ प्र प्र यु॑ङ्क्ते त्रि॒ष्टुङ्‌मु॑ख स्त्रि॒ष्टुङ्‌मु॑खो युङ्क्ते॒ प्र प्र यु॑ङ्क्ते त्रि॒ष्टुङ्‌मु॑खः । \newline
48. यु॒ङ्क्ते॒ त्रि॒ष्टुङ्‌मु॑ख स्त्रि॒ष्टुङ्‌मु॑खो युङ्क्ते युङ्क्ते त्रि॒ष्टुङ्‌मु॑खो॒ वै वै त्रि॒ष्टुङ्‌मु॑खो युङ्क्ते युङ्क्ते त्रि॒ष्टुङ्‌मु॑खो॒ वै । \newline
49. त्रि॒ष्टुङ्‌मु॑खो॒ वै वै त्रि॒ष्टुङ्‌मु॑ख स्त्रि॒ष्टुङ्‌मु॑खो॒ वै तृ॒तीय॑ स्तृ॒तीयो॒ वै त्रि॒ष्टुङ्‌मु॑ख स्त्रि॒ष्टुङ्‌मु॑खो॒ वै तृ॒तीयः॑ । \newline
50. त्रि॒ष्टुङ्‌मु॑ख॒ इति॑ त्रि॒ष्टुक् - मु॒खः॒ । \newline
51. वै तृ॒तीय॑ स्तृ॒तीयो॒ वै वै तृ॒तीय॑ स्त्रिरा॒त्र स्त्रि॑रा॒त्र स्तृ॒तीयो॒ वै वै तृ॒तीय॑ स्त्रिरा॒त्रः । \newline
52. तृ॒तीय॑ स्त्रिरा॒त्र स्त्रि॑रा॒त्र स्तृ॒तीय॑ स्तृ॒तीय॑ स्त्रिरा॒त्र स्त्रैष्टु॑भ॒ स्त्रैष्टु॑भ स्त्रिरा॒त्र स्तृ॒तीय॑ स्तृ॒तीय॑ स्त्रिरा॒त्र स्त्रैष्टु॑भः । \newline
53. त्रि॒रा॒त्र स्त्रैष्टु॑भ॒ स्त्रैष्टु॑भ स्त्रिरा॒त्र स्त्रि॑रा॒त्र स्त्रैष्टु॑भः शु॒क्रः शु॒क्र स्त्रैष्टु॑भ स्त्रिरा॒त्र स्त्रि॑रा॒त्र स्त्रैष्टु॑भः शु॒क्रः । \newline
54. त्रि॒रा॒त्र इति॑ त्रि - रा॒त्रः । \newline
55. त्रैष्टु॑भः शु॒क्रः शु॒क्र स्त्रैष्टु॑भ॒ स्त्रैष्टु॑भः शु॒क्रो यद् यच्छु॒क्र स्त्रैष्टु॑भ॒ स्त्रैष्टु॑भः शु॒क्रो यत् । \newline
\pagebreak
\markright{ TS 7.2.8.4  \hfill https://www.vedavms.in \hfill}

\section{ TS 7.2.8.4 }

\textbf{TS 7.2.8.4 } \newline
\textbf{Samhita Paata} \newline

शु॒क्रो यथ् स॑प्त॒मेऽह॑ञ्छु॒क्रो गृ॒ह्यते॒ स्व ए॒वैन॑मा॒यत॑ने गृह्णा॒त्यथो॒ स्वमे॒व छन्दोऽनु॑ प॒र्याव॑र्तन्ते॒ वाग्वा आ᳚ग्रय॒णो वाग॑ष्ट॒ममह॒-स्तस्मा॑दष्ट॒मेऽह॑न्नाग्रय॒णो गृ॑ह्यते॒ स्व ए॒वैन॑मा॒यत॑ने गृह्णाति प्रा॒णो वा ऐ᳚न्द्रवाय॒वः प्रा॒णो न॑व॒म-मह॒स्तस्मा᳚न्नव॒मे ऽह॑न्नैन्द्रवाय॒वो गृ॑ह्यते॒ स्व ए॒वैन॑मा॒यत॑ने गृह्णात्ये॒त - [  ] \newline

\textbf{Pada Paata} \newline

शु॒क्रः । यत् । स॒प्त॒मे । अहन्न्॑ । शु॒क्रः । गृ॒ह्यते᳚ । स्वे । ए॒व । ए॒न॒म् । आ॒यत॑न॒ इत्या᳚ - यत॑ने । गृ॒ह्णा॒ति॒ । अथो॒ इति॑ । स्वम् । ए॒व । छन्दः॑ । अन्विति॑ । प॒र्याव॑र्तन्त॒ इति॑ परि - आव॑र्तन्ते । वाक् । वै । आ॒ग्र॒य॒णः । वाक् । अ॒ष्ट॒मम् । अहः॑ । तस्मा᳚त् । अ॒ष्ट॒मे । अहन्न्॑ । आ॒ग्र॒य॒णः । गृ॒ह्य॒ते॒ । स्वे । ए॒व । ए॒न॒म् । आ॒यत॑न॒ इत्या᳚ - यत॑ने । गृ॒ह्णा॒ति॒ । प्रा॒ण इति॑ प्र - अ॒नः । वै । ऐ॒न्द्र॒वा॒य॒व इत्यै᳚न्द्र - वा॒य॒वः । प्रा॒ण इति॑ प्र - अ॒नः । न॒व॒मम् । अहः॑ । तस्मा᳚त् । न॒व॒मे । अहन्न्॑ । ऐ॒न्द्र॒वा॒य॒व इत्यै᳚न्द्र - वा॒य॒वः । गृ॒ह्य॒ते॒ । स्वे । ए॒व । ए॒न॒म् । आ॒यत॑न॒ इत्या᳚ - यत॑ने । गृ॒ह्णा॒ति॒ । ए॒तत् ।  \newline


\textbf{Krama Paata} \newline

शु॒क्रो यत् । यथ् स॑प्त॒मे । स॒प्त॒मेऽहन्न्॑ । अह॑ञ्छु॒क्रः । शु॒क्रो गृ॒ह्यते᳚ । गृ॒ह्यते॒ स्वे । स्व ए॒व । ए॒वैन᳚म् । ए॒न॒मा॒यत॑ने । आ॒यत॑ने गृह्णाति । आ॒यत॑न॒ इत्या᳚ - यत॑ने । गृ॒ह्णा॒त्यथो᳚ । अथो॒ स्वम् । अथो॒ इत्यथो᳚ । स्वमे॒व । ए॒व छन्दः॑ । छन्दोऽनु॑ । अनु॑ प॒र्याव॑र्तन्ते । प॒र्याव॑र्तन्ते॒ वाक् । प॒र्याव॑र्तन्त॒ इति॑ परि - आव॑र्तन्ते । वाग् वै । वा आ᳚ग्रय॒णः । आ॒ग्र॒य॒णो वाक् । वाग॑ष्ट॒मम् । अ॒ष्ट॒ममहः॑ । अह॒स्तस्मा᳚त् । तस्मा॑दष्ट॒मे । अ॒ष्ट॒मेऽहन्न्॑ । अह॑न्नाग्रय॒णः । आ॒ग्र॒य॒णो गृ॑ह्यते । गृ॒ह्य॒ते॒ स्वे । स्व ए॒व । ए॒वैन᳚म् । ए॒न॒मा॒यत॑ने । आ॒यत॑ने गृह्णाति । आ॒यत॑न॒ इत्या᳚ - यत॑ने । गृ॒ह्णा॒ति॒ प्रा॒णः । प्रा॒णो वै । प्रा॒ण इति॑ प्र - अ॒नः । वा ऐ᳚न्द्रवाय॒वः । ऐ॒न्द्र॒वा॒य॒वः प्रा॒णः । ऐ॒न्द्र॒वा॒य॒व इत्यै᳚न्द्र - वा॒य॒वः । प्रा॒णो न॑व॒मम् । प्रा॒ण इति॑ प्र - अ॒नः । न॒व॒ममहः॑ । अह॒स्तस्मा᳚त् । तस्मा᳚न् नव॒मे । न॒व॒मेऽहन्न्॑ । अह॑न्नैन्द्रवाय॒वः । ऐ॒न्द्र॒वा॒य॒वो गृ॑ह्यते । ऐ॒न्द्र॒वा॒य॒व इत्यै᳚न्द्र - वा॒य॒वः । गृ॒ह्य॒ते॒ स्वे । स्व ए॒व । ए॒वैन᳚म् । ए॒न॒मा॒यत॑ने । आ॒यत॑ने गृह्णाति । आ॒यत॑न॒ इत्या᳚ - यत॑ने । गृ॒ह्णा॒त्ये॒तत् । ए॒तद् वै \newline

\textbf{Jatai Paata} \newline

1. शु॒क्रो यद् यच्छु॒क्रः शु॒क्रो यत् । \newline
2. यथ् स॑प्त॒मे स॑प्त॒मे यद् यथ् स॑प्त॒मे । \newline
3. स॒प्त॒मे ऽह॒न् नहन्᳚ थ्सप्त॒मे स॑प्त॒मे ऽहन्न्॑ । \newline
4. अह॑ञ् छु॒क्रः शु॒क्रो ऽह॒न् नह॑ञ् छु॒क्रः । \newline
5. शु॒क्रो गृ॒ह्यते॑ गृ॒ह्यते॑ शु॒क्रः शु॒क्रो गृ॒ह्यते᳚ । \newline
6. गृ॒ह्यते॒ स्वे स्वे गृ॒ह्यते॑ गृ॒ह्यते॒ स्वे । \newline
7. स्व ए॒वैव स्वे स्व ए॒व । \newline
8. ए॒वैन॑ मेन मे॒वै वैन᳚म् । \newline
9. ए॒न॒ मा॒यत॑न आ॒यत॑न एन मेन मा॒यत॑ने । \newline
10. आ॒यत॑ने गृह्णाति गृह्णा त्या॒यत॑न आ॒यत॑ने गृह्णाति । \newline
11. आ॒यत॑न॒ इत्या᳚ - यत॑ने । \newline
12. गृ॒ह्णा॒ त्यथो॒ अथो॑ गृह्णाति गृह्णा॒ त्यथो᳚ । \newline
13. अथो॒ स्वꣳ स्व मथो॒ अथो॒ स्वम् । \newline
14. अथो॒ इत्यथो᳚ । \newline
15. स्व मे॒वैव स्वꣳ स्व मे॒व । \newline
16. ए॒व छन्द॒ श्छन्द॑ ए॒वैव छन्दः॑ । \newline
17. छन्दो ऽन्वनु॒ च्छन्द॒ श्छन्दो ऽनु॑ । \newline
18. अनु॑ प॒र्याव॑र्तन्ते प॒र्याव॑र्त॒न्ते ऽन्वनु॑ प॒र्याव॑र्तन्ते । \newline
19. प॒र्याव॑र्तन्ते॒ वाग् वाक् प॒र्याव॑र्तन्ते प॒र्याव॑र्तन्ते॒ वाक् । \newline
20. प॒र्याव॑र्तन्त॒ इति॑ परि - आव॑र्तन्ते । \newline
21. वाग् वै वै वाग् वाग् वै । \newline
22. वा आ᳚ग्रय॒ण आ᳚ग्रय॒णो वै वा आ᳚ग्रय॒णः । \newline
23. आ॒ग्र॒य॒णो वाग् वागा᳚ग्रय॒ण आ᳚ग्रय॒णो वाक् । \newline
24. वाग॑ष्ट॒म म॑ष्ट॒मं ॅवाग् वाग॑ष्ट॒मम् । \newline
25. अ॒ष्ट॒म मह॒ रह॑ रष्ट॒म म॑ष्ट॒म महः॑ । \newline
26. अह॒ स्तस्मा॒त् तस्मा॒ दह॒ रह॒ स्तस्मा᳚त् । \newline
27. तस्मा॑ दष्ट॒मे᳚ ऽष्ट॒मे तस्मा॒त् तस्मा॑ दष्ट॒मे । \newline
28. अ॒ष्ट॒मे ऽह॒न् नह॑न् नष्ट॒मे᳚ ऽष्ट॒मे ऽहन्न्॑ । \newline
29. अह॑न् नाग्रय॒ण आ᳚ग्रय॒णो ऽह॒न् नह॑न् नाग्रय॒णः । \newline
30. आ॒ग्र॒य॒णो गृ॑ह्यते गृह्यत आग्रय॒ण आ᳚ग्रय॒णो गृ॑ह्यते । \newline
31. गृ॒ह्य॒ते॒ स्वे स्वे गृ॑ह्यते गृह्यते॒ स्वे । \newline
32. स्व ए॒वैव स्वे स्व ए॒व । \newline
33. ए॒वैन॑ मेन मे॒वै वैन᳚म् । \newline
34. ए॒न॒ मा॒यत॑न आ॒यत॑न एन मेन मा॒यत॑ने । \newline
35. आ॒यत॑ने गृह्णाति गृह्णा त्या॒यत॑न आ॒यत॑ने गृह्णाति । \newline
36. आ॒यत॑न॒ इत्या᳚ - यत॑ने । \newline
37. गृ॒ह्णा॒ति॒ प्रा॒णः प्रा॒णो गृ॑ह्णाति गृह्णाति प्रा॒णः । \newline
38. प्रा॒णो वै वै प्रा॒णः प्रा॒णो वै । \newline
39. प्रा॒ण इति॑ प्र - अ॒नः । \newline
40. वा ऐ᳚न्द्रवाय॒व ऐ᳚न्द्रवाय॒वो वै वा ऐ᳚न्द्रवाय॒वः । \newline
41. ऐ॒न्द्र॒वा॒य॒वः प्रा॒णः प्रा॒ण ऐ᳚न्द्रवाय॒व ऐ᳚न्द्रवाय॒वः प्रा॒णः । \newline
42. ऐ॒न्द्र॒वा॒य॒व इत्यै᳚न्द्र - वा॒य॒वः । \newline
43. प्रा॒णो न॑व॒मन् न॑व॒मम् प्रा॒णः प्रा॒णो न॑व॒मम् । \newline
44. प्रा॒ण इति॑ प्र - अ॒नः । \newline
45. न॒व॒म मह॒ रह॑र् नव॒मन् न॑व॒म महः॑ । \newline
46. अह॒ स्तस्मा॒त् तस्मा॒ दह॒ रह॒ स्तस्मा᳚त् । \newline
47. तस्मा᳚न् नव॒मे न॑व॒मे तस्मा॒त् तस्मा᳚न् नव॒मे । \newline
48. न॒व॒मे ऽह॒न् नह॑न् नव॒मे न॑व॒मे ऽहन्न्॑ । \newline
49. अह॑न् नैन्द्रवाय॒व ऐ᳚न्द्रवाय॒वो ऽह॒न् नह॑न् नैन्द्रवाय॒वः । \newline
50. ऐ॒न्द्र॒वा॒य॒वो गृ॑ह्यते गृह्यत ऐन्द्रवाय॒व ऐ᳚न्द्रवाय॒वो गृ॑ह्यते । \newline
51. ऐ॒न्द्र॒वा॒य॒व इत्यै᳚न्द्र - वा॒य॒वः । \newline
52. गृ॒ह्य॒ते॒ स्वे स्वे गृ॑ह्यते गृह्यते॒ स्वे । \newline
53. स्व ए॒वैव स्वे स्व ए॒व । \newline
54. ए॒वैन॑ मेन मे॒वै वैन᳚म् । \newline
55. ए॒न॒ मा॒यत॑न आ॒यत॑न एन मेन मा॒यत॑ने । \newline
56. आ॒यत॑ने गृह्णाति गृह्णा त्या॒यत॑न आ॒यत॑ने गृह्णाति । \newline
57. आ॒यत॑न॒ इत्या᳚ - यत॑ने । \newline
58. गृ॒ह्णा॒ त्ये॒त दे॒तद् गृ॑ह्णाति गृह्णा त्ये॒तत् । \newline
59. ए॒तद् वै वा ए॒त दे॒तद् वै । \newline

\textbf{Ghana Paata } \newline

1. शु॒क्रो यद् यच्छु॒क्रः शु॒क्रो यथ् स॑प्त॒मे स॑प्त॒मे यच्छु॒क्रः शु॒क्रो यथ् स॑प्त॒मे । \newline
2. यथ् स॑प्त॒मे स॑प्त॒मे यद् यथ् स॑प्त॒मे ऽह॒न् नहन्᳚ थ्सप्त॒मे यद् यथ् स॑प्त॒मे ऽहन्न्॑ । \newline
3. स॒प्त॒मे ऽह॒न् नहन्᳚ थ्सप्त॒मे स॑प्त॒मे ऽह॑ञ् छु॒क्रः शु॒क्रो ऽहन्᳚ थ्सप्त॒मे स॑प्त॒मे ऽह॑ञ् छु॒क्रः । \newline
4. अह॑ञ् छु॒क्रः शु॒क्रो ऽह॒न् नह॑ञ् छु॒क्रो गृ॒ह्यते॑ गृ॒ह्यते॑ शु॒क्रो ऽह॒न् नह॑ञ् छु॒क्रो गृ॒ह्यते᳚ । \newline
5. शु॒क्रो गृ॒ह्यते॑ गृ॒ह्यते॑ शु॒क्रः शु॒क्रो गृ॒ह्यते॒ स्वे स्वे गृ॒ह्यते॑ शु॒क्रः शु॒क्रो गृ॒ह्यते॒ स्वे । \newline
6. गृ॒ह्यते॒ स्वे स्वे गृ॒ह्यते॑ गृ॒ह्यते॒ स्व ए॒वैव स्वे गृ॒ह्यते॑ गृ॒ह्यते॒ स्व ए॒व । \newline
7. स्व ए॒वैव स्वे स्व ए॒वैन॑ मेन मे॒व स्वे स्व ए॒वैन᳚म् । \newline
8. ए॒वैन॑ मेन मे॒वै वैन॑ मा॒यत॑न आ॒यत॑न एन मे॒वै वैन॑ मा॒यत॑ने । \newline
9. ए॒न॒ मा॒यत॑न आ॒यत॑न एन मेन मा॒यत॑ने गृह्णाति गृह्णा त्या॒यत॑न एन मेन मा॒यत॑ने गृह्णाति । \newline
10. आ॒यत॑ने गृह्णाति गृह्णा त्या॒यत॑न आ॒यत॑ने गृह्णा॒ त्यथो॒ अथो॑ गृह्णा त्या॒यत॑न आ॒यत॑ने गृह्णा॒ त्यथो᳚ । \newline
11. आ॒यत॑न॒ इत्या᳚ - यत॑ने । \newline
12. गृ॒ह्णा॒ त्यथो॒ अथो॑ गृह्णाति गृह्णा॒ त्यथो॒ स्वꣳ स्व मथो॑ गृह्णाति गृह्णा॒ त्यथो॒ स्वम् । \newline
13. अथो॒ स्वꣳ स्व मथो॒ अथो॒ स्व मे॒वैव स्व मथो॒ अथो॒ स्व मे॒व । \newline
14. अथो॒ इत्यथो᳚ । \newline
15. स्व मे॒वैव स्वꣳ स्व मे॒व छन्द॒ श्छन्द॑ ए॒व स्वꣳ स्व मे॒व छन्दः॑ । \newline
16. ए॒व छन्द॒ श्छन्द॑ ए॒वैव छन्दो ऽन्वनु॒ च्छन्द॑ ए॒वैव छन्दो ऽनु॑ । \newline
17. छन्दो ऽन्वनु॒ च्छन्द॒ श्छन्दो ऽनु॑ प॒र्याव॑र्तन्ते प॒र्याव॑र्त॒न्ते ऽनु॒ च्छन्द॒ श्छन्दो ऽनु॑ प॒र्याव॑र्तन्ते । \newline
18. अनु॑ प॒र्याव॑र्तन्ते प॒र्याव॑र्त॒न्ते ऽन्वनु॑ प॒र्याव॑र्तन्ते॒ वाग् वाक् प॒र्याव॑र्त॒न्ते ऽन्वनु॑ प॒र्याव॑र्तन्ते॒ वाक् । \newline
19. प॒र्याव॑र्तन्ते॒ वाग् वाक् प॒र्याव॑र्तन्ते प॒र्याव॑र्तन्ते॒ वाग् वै वै वाक् प॒र्याव॑र्तन्ते प॒र्याव॑र्तन्ते॒ वाग् वै । \newline
20. प॒र्याव॑र्तन्त॒ इति॑ परि - आव॑र्तन्ते । \newline
21. वाग् वै वै वाग् वाग् वा आ᳚ग्रय॒ण आ᳚ग्रय॒णो वै वाग् वाग् वा आ᳚ग्रय॒णः । \newline
22. वा आ᳚ग्रय॒ण आ᳚ग्रय॒णो वै वा आ᳚ग्रय॒णो वाग् वागा᳚ग्रय॒णो वै वा आ᳚ग्रय॒णो वाक् । \newline
23. आ॒ग्र॒य॒णो वाग् वागा᳚ग्रय॒ण आ᳚ग्रय॒णो वाग॑ष्ट॒म म॑ष्ट॒मं ॅवागा᳚ग्रय॒ण आ᳚ग्रय॒णो वाग॑ष्ट॒मम् । \newline
24. वाग॑ष्ट॒म म॑ष्ट॒मं ॅवाग् वाग॑ष्ट॒म मह॒ रह॑ रष्ट॒मं ॅवाग् वाग॑ष्ट॒म महः॑ । \newline
25. अ॒ष्ट॒म मह॒ रह॑ रष्ट॒म म॑ष्ट॒म मह॒ स्तस्मा॒त् तस्मा॒ दह॑ रष्ट॒म म॑ष्ट॒म मह॒ स्तस्मा᳚त् । \newline
26. अह॒ स्तस्मा॒त् तस्मा॒ दह॒ रह॒ स्तस्मा॑ दष्ट॒मे᳚ ऽष्ट॒मे तस्मा॒ दह॒ रह॒ स्तस्मा॑ दष्ट॒मे । \newline
27. तस्मा॑ दष्ट॒मे᳚ ऽष्ट॒मे तस्मा॒त् तस्मा॑ दष्ट॒मे ऽह॒न् नह॑न् नष्ट॒मे तस्मा॒त् तस्मा॑ दष्ट॒मे ऽहन्न्॑ । \newline
28. अ॒ष्ट॒मे ऽह॒न् नह॑न् नष्ट॒मे᳚ ऽष्ट॒मे ऽह॑न् नाग्रय॒ण आ᳚ग्रय॒णो ऽह॑न् नष्ट॒मे᳚ ऽष्ट॒मे ऽह॑न् नाग्रय॒णः । \newline
29. अह॑न् नाग्रय॒ण आ᳚ग्रय॒णो ऽह॒न् नह॑न् नाग्रय॒णो गृ॑ह्यते गृह्यत आग्रय॒णो ऽह॒न् नह॑न् नाग्रय॒णो गृ॑ह्यते । \newline
30. आ॒ग्र॒य॒णो गृ॑ह्यते गृह्यत आग्रय॒ण आ᳚ग्रय॒णो गृ॑ह्यते॒ स्वे स्वे गृ॑ह्यत आग्रय॒ण आ᳚ग्रय॒णो गृ॑ह्यते॒ स्वे । \newline
31. गृ॒ह्य॒ते॒ स्वे स्वे गृ॑ह्यते गृह्यते॒ स्व ए॒वैव स्वे गृ॑ह्यते गृह्यते॒ स्व ए॒व । \newline
32. स्व ए॒वैव स्वे स्व ए॒वैन॑ मेन मे॒व स्वे स्व ए॒वैन᳚म् । \newline
33. ए॒वैन॑ मेन मे॒वै वैन॑ मा॒यत॑न आ॒यत॑न एन मे॒वै वैन॑ मा॒यत॑ने । \newline
34. ए॒न॒ मा॒यत॑न आ॒यत॑न एन मेन मा॒यत॑ने गृह्णाति गृह्णा त्या॒यत॑न एन मेन मा॒यत॑ने गृह्णाति । \newline
35. आ॒यत॑ने गृह्णाति गृह्णा त्या॒यत॑न आ॒यत॑ने गृह्णाति प्रा॒णः प्रा॒णो गृ॑ह्णा त्या॒यत॑न आ॒यत॑ने गृह्णाति प्रा॒णः । \newline
36. आ॒यत॑न॒ इत्या᳚ - यत॑ने । \newline
37. गृ॒ह्णा॒ति॒ प्रा॒णः प्रा॒णो गृ॑ह्णाति गृह्णाति प्रा॒णो वै वै प्रा॒णो गृ॑ह्णाति गृह्णाति प्रा॒णो वै । \newline
38. प्रा॒णो वै वै प्रा॒णः प्रा॒णो वा ऐ᳚न्द्रवाय॒व ऐ᳚न्द्रवाय॒वो वै प्रा॒णः प्रा॒णो वा ऐ᳚न्द्रवाय॒वः । \newline
39. प्रा॒ण इति॑ प्र - अ॒नः । \newline
40. वा ऐ᳚न्द्रवाय॒व ऐ᳚न्द्रवाय॒वो वै वा ऐ᳚न्द्रवाय॒वः प्रा॒णः प्रा॒ण ऐ᳚न्द्रवाय॒वो वै वा ऐ᳚न्द्रवाय॒वः प्रा॒णः । \newline
41. ऐ॒न्द्र॒वा॒य॒वः प्रा॒णः प्रा॒ण ऐ᳚न्द्रवाय॒व ऐ᳚न्द्रवाय॒वः प्रा॒णो न॑व॒मम् न॑व॒मम् प्रा॒ण ऐ᳚न्द्रवाय॒व ऐ᳚न्द्रवाय॒वः प्रा॒णो न॑व॒मम् । \newline
42. ऐ॒न्द्र॒वा॒य॒व इत्यै᳚न्द्र - वा॒य॒वः । \newline
43. प्रा॒णो न॑व॒मम् न॑व॒मम् प्रा॒णः प्रा॒णो न॑व॒म मह॒ रह॑र् नव॒मम् प्रा॒णः प्रा॒णो न॑व॒म महः॑ । \newline
44. प्रा॒ण इति॑ प्र - अ॒नः । \newline
45. न॒व॒म मह॒ रह॑र् नव॒मम् न॑व॒म मह॒ स्तस्मा॒त् तस्मा॒ दह॑र् नव॒मम् न॑व॒म मह॒ स्तस्मा᳚त् । \newline
46. अह॒ स्तस्मा॒त् तस्मा॒ दह॒ रह॒ स्तस्मा᳚न् नव॒मे न॑व॒मे तस्मा॒ दह॒ रह॒ स्तस्मा᳚न् नव॒मे । \newline
47. तस्मा᳚न् नव॒मे न॑व॒मे तस्मा॒त् तस्मा᳚न् नव॒मे ऽह॒न् नह॑न् नव॒मे तस्मा॒त् तस्मा᳚न् नव॒मे ऽहन्न्॑ । \newline
48. न॒व॒मे ऽह॒न् नह॑न् नव॒मे न॑व॒मे ऽह॑न् नैन्द्रवाय॒व ऐ᳚न्द्रवाय॒वो ऽह॑न् नव॒मे न॑व॒मे ऽह॑न् नैन्द्रवाय॒वः । \newline
49. अह॑न् नैन्द्रवाय॒व ऐ᳚न्द्रवाय॒वो ऽह॒न् नह॑न् नैन्द्रवाय॒वो गृ॑ह्यते गृह्यत ऐन्द्रवाय॒वो ऽह॒न् नह॑न् नैन्द्रवाय॒वो गृ॑ह्यते । \newline
50. ऐ॒न्द्र॒वा॒य॒वो गृ॑ह्यते गृह्यत ऐन्द्रवाय॒व ऐ᳚न्द्रवाय॒वो गृ॑ह्यते॒ स्वे स्वे गृ॑ह्यत ऐन्द्रवाय॒व ऐ᳚न्द्रवाय॒वो गृ॑ह्यते॒ स्वे । \newline
51. ऐ॒न्द्र॒वा॒य॒व इत्यै᳚न्द्र - वा॒य॒वः । \newline
52. गृ॒ह्य॒ते॒ स्वे स्वे गृ॑ह्यते गृह्यते॒ स्व ए॒वैव स्वे गृ॑ह्यते गृह्यते॒ स्व ए॒व । \newline
53. स्व ए॒वैव स्वे स्व ए॒वैन॑ मेन मे॒व स्वे स्व ए॒वैन᳚म् । \newline
54. ए॒वैन॑ मेन मे॒वै वैन॑ मा॒यत॑न आ॒यत॑न एन मे॒वै वैन॑ मा॒यत॑ने । \newline
55. ए॒न॒ मा॒यत॑न आ॒यत॑न एन मेन मा॒यत॑ने गृह्णाति गृह्णा त्या॒यत॑न एन मेन मा॒यत॑ने गृह्णाति । \newline
56. आ॒यत॑ने गृह्णाति गृह्णा त्या॒यत॑न आ॒यत॑ने गृह्णा त्ये॒त दे॒तद् गृ॑ह्णा त्या॒यत॑न आ॒यत॑ने गृह्णा त्ये॒तत् । \newline
57. आ॒यत॑न॒ इत्या᳚ - यत॑ने । \newline
58. गृ॒ह्णा॒ त्ये॒त दे॒तद् गृ॑ह्णाति गृह्णा त्ये॒तद् वै वा ए॒तद् गृ॑ह्णाति गृह्णा त्ये॒तद् वै । \newline
59. ए॒तद् वै वा ए॒त दे॒तद् वै तृ॒तीय॑म् तृ॒तीयं॒ ॅवा ए॒त दे॒तद् वै तृ॒तीय᳚म् । \newline
\pagebreak
\markright{ TS 7.2.8.5  \hfill https://www.vedavms.in \hfill}

\section{ TS 7.2.8.5 }

\textbf{TS 7.2.8.5 } \newline
\textbf{Samhita Paata} \newline

-द्वै तृ॒तीयं॑ ॅय॒ज्ञ्मा॑प॒द्-यच्छन्दाꣳ॑स्या॒प्नोति॒ यदै᳚न्द्रवाय॒वः श्वो गृ॒ह्यते॒ यत्रै॒व य॒ज्ञ्मदृ॑श॒न् तत॑ ए॒वैनं॒ पुनः॒ प्रयु॒ङ्क्ते ऽथो॒ स्वमे॒व छन्दोऽनु॑ प॒र्याव॑र्तन्ते प॒थो वा ए॒तेऽद्ध्यप॑थेन यन्ति॒ ये᳚ऽन्येनै᳚न्द्रवाय॒वात् प्र॑ति॒पद्य॒न्तेऽन्तः॒ खलु॒ वा ए॒ष य॒ज्ञ्स्य॒ यद्-द॑श॒म-मह॑र्दश॒मे ऽह॑न्नैन्द्रवाय॒वो गृ॑ह्यते य॒ज्ञ्स्यै॒-[  ] \newline

\textbf{Pada Paata} \newline

वै । तृ॒तीय᳚म् । य॒ज्ञ्म् । आ॒प॒त् । यत् । छन्दाꣳ॑सि । आ॒प्नोति॑ । यत् । ऐ॒न्द्र॒वा॒य॒व इत्यै᳚न्द्र - वा॒य॒वः । श्वः । गृ॒ह्यते᳚ । यत्र॑ । ए॒व । य॒ज्ञ्म् । अदृ॑शन्न् । ततः॑ । ए॒व । ए॒न॒म् । पुनः॑ । प्रेति॑ । यु॒ङ्क्ते॒ । अथो॒ इति॑ । स्वम् । ए॒व । छन्दः॑ । अन्विति॑ । प॒र्याव॑र्तन्त॒ इति॑ परि - आव॑र्तन्ते । प॒थः । वै । ए॒ते । अधीति॑ । अप॑थेन । य॒न्ति॒ । ये । अ॒न्येन॑ । ऐ॒न्द्र॒वा॒य॒वादित्यै᳚न्द्र - वा॒य॒वात् । प्र॒ति॒पद्य॑न्त॒ इति॑ प्रति - पद्य॑न्ते । अन्तः॑ । खलु॑ । वै । ए॒षः । य॒ज्ञ्स्य॑ । यत् । द॒श॒मम् । अहः॑ । द॒श॒मे । अहन्न्॑ । ऐ॒न्द्र॒वा॒य॒व इत्यै᳚न्द्र - वा॒य॒वः । गृ॒ह्य॒ते॒ । य॒ज्ञ्स्य॑ ।  \newline


\textbf{Krama Paata} \newline

वै तृ॒तीय᳚म् । तृ॒तीय॑म् ॅय॒ज्ञ्म् । य॒ज्ञ्मा॑पत् । आ॒प॒द् यत् । यच् छन्दाꣳ॑सि । छन्दाꣳ॑स्या॒प्नोति॑ । 
आ॒प्नोति॒ यत् । यदै᳚न्द्रवाय॒वः । ऐ॒न्द्रा॒वा॒य॒वः श्वः । ऐ॒न्द्र॒वा॒य॒व इत्यै᳚न्द्र - वा॒य॒वः । श्वो गृ॒ह्यते᳚ । गृ॒ह्यते॒ यत्र॑ । यत्रै॒व । ए॒व य॒ज्ञ्म् । य॒ज्ञ्मदृ॑शन्न् । अदृ॑श॒न् ततः॑ । तत॑ ए॒व । ए॒वैन᳚म् । ए॒न॒म् पुनः॑ । पुनः॒ प्र । प्र यु॑ङ्‍क्ते । यु॒ङ्‍क्तेऽथो᳚ । अथो॒ स्वम् । अथो॒ इत्यथो᳚ । स्वमे॒व । ए॒व छन्दः॑ । छन्दोऽनु॑ । अनु॑ प॒र्याव॑र्तन्ते । प॒र्याव॑र्तन्ते प॒थः । प॒र्याव॑र्तन्त॒ इति॑ परि - आव॑र्तन्ते । प॒थो वै । वा ए॒ते । ए॒तेऽधि॑ । अद्ध्यप॑थेन । अप॑थेन यन्ति । य॒न्ति॒ ये । ये᳚ऽन्येन॑ । अ॒न्येनै᳚न्द्रवाय॒वात् । ऐ॒न्द्र॒वा॒य॒वात् प्र॑ति॒पद्य॑न्ते । ऐ॒न्द्र॒वा॒य॒वादित्यै᳚न्द्र - वा॒य॒वात् । प्र॒ति॒पद्य॒न्तेऽन्तः॑ । प्र॒ति॒पद्य॑न्त॒ इति॑ प्रति - पद्य॑न्ते । अन्तः॒ खलु॑ । खलु॒ वै । वा ए॒षः । ए॒ष य॒ज्ञ्स्य॑ । य॒ज्ञ्स्य॒ यत् । यद् द॑श॒मम् । द॒श॒ममहः॑ । अह॑र् दश॒मे । द॒श॒मेऽहन्न्॑ । अह॑न्नैन्द्रवाय॒वः । ऐ॒न्द्र॒वा॒य॒वो गृ॑ह्यते । ऐ॒न्द्र॒वा॒य॒व इत्यै᳚न्द्र - वा॒य॒वः । गृ॒ह्य॒ते॒ य॒ज्ञ्स्य॑ । य॒ज्ञ्स्यै॒व \newline

\textbf{Jatai Paata} \newline

1. वै तृ॒तीय॑म् तृ॒तीयं॒ ॅवै वै तृ॒तीय᳚म् । \newline
2. तृ॒तीयं॑ ॅय॒ज्ञ्ं ॅय॒ज्ञ्म् तृ॒तीय॑म् तृ॒तीयं॑ ॅय॒ज्ञ्म् । \newline
3. य॒ज्ञ् मा॑प दापद् य॒ज्ञ्ं ॅय॒ज्ञ् मा॑पत् । \newline
4. आ॒प॒द् यद् यदा॑प दाप॒द् यत् । \newline
5. यच् छन्दाꣳ॑सि॒ छन्दाꣳ॑सि॒ यद् यच् छन्दाꣳ॑सि । \newline
6. छन्दाꣳ॑ स्या॒प्नो त्या॒प्नोति॒ छन्दाꣳ॑सि॒ छन्दाꣳ॑ स्या॒प्नोति॑ । \newline
7. आ॒प्नोति॒ यद् यदा॒प्नो त्या॒प्नोति॒ यत् । \newline
8. यदै᳚न्द्रवाय॒व ऐ᳚न्द्रवाय॒वो यद् यदै᳚न्द्रवाय॒वः । \newline
9. ऐ॒न्द्र॒वा॒य॒वः श्वः श्व ऐ᳚न्द्रवाय॒व ऐ᳚न्द्रवाय॒वः श्वः । \newline
10. ऐ॒न्द्र॒वा॒य॒व इत्यै᳚न्द्र - वा॒य॒वः । \newline
11. श्वो गृ॒ह्यते॑ गृ॒ह्यते॒ श्वः श्वो गृ॒ह्यते᳚ । \newline
12. गृ॒ह्यते॒ यत्र॒ यत्र॑ गृ॒ह्यते॑ गृ॒ह्यते॒ यत्र॑ । \newline
13. यत्रै॒ वैव यत्र॒ यत्रै॒व । \newline
14. ए॒व य॒ज्ञ्ं ॅय॒ज्ञ् मे॒वैव य॒ज्ञ्म् । \newline
15. य॒ज्ञ् मदृ॑श॒न् नदृ॑शन्. य॒ज्ञ्ं ॅय॒ज्ञ् मदृ॑शन्न् । \newline
16. अदृ॑श॒न् तत॒ स्ततो ऽदृ॑श॒न् नदृ॑श॒न् ततः॑ । \newline
17. तत॑ ए॒वैव तत॒ स्तत॑ ए॒व । \newline
18. ए॒वैन॑ मेन मे॒वै वैन᳚म् । \newline
19. ए॒न॒म् पुनः॒ पुन॑ रेन मेन॒म् पुनः॑ । \newline
20. पुनः॒ प्र प्र पुनः॒ पुनः॒ प्र । \newline
21. प्र यु॑ङ्क्ते युङ्क्ते॒ प्र प्र यु॑ङ्क्ते । \newline
22. यु॒ङ्क्ते ऽथो॒ अथो॑ युङ्क्ते यु॒ङ्क्ते ऽथो᳚ । \newline
23. अथो॒ स्वꣳ स्व मथो॒ अथो॒ स्वम् । \newline
24. अथो॒ इत्यथो᳚ । \newline
25. स्व मे॒वैव स्वꣳ स्व मे॒व । \newline
26. ए॒व छन्द॒ श्छन्द॑ ए॒वैव छन्दः॑ । \newline
27. छन्दो ऽन्वनु॒ च्छन्द॒ श्छन्दो ऽनु॑ । \newline
28. अनु॑ प॒र्याव॑र्तन्ते प॒र्याव॑र्त॒न्ते ऽन्वनु॑ प॒र्याव॑र्तन्ते । \newline
29. प॒र्याव॑र्तन्ते प॒थः प॒थः प॒र्याव॑र्तन्ते प॒र्याव॑र्तन्ते प॒थः । \newline
30. प॒र्याव॑र्तन्त॒ इति॑ परि - आव॑र्तन्ते । \newline
31. प॒थो वै वै प॒थः प॒थो वै । \newline
32. वा ए॒त ए॒ते वै वा ए॒ते । \newline
33. ए॒ते ऽध्य ध्ये॒त ए॒ते ऽधि॑ । \newline
34. अध्यप॑थे॒ना प॑थे॒ना ध्यध्यप॑थेन । \newline
35. अप॑थेन यन्ति य॒न्त्यप॑थे॒ना प॑थेन यन्ति । \newline
36. य॒न्ति॒ ये ये य॑न्ति यन्ति॒ ये । \newline
37. ये᳚ ऽन्येना॒ न्येन॒ ये ये᳚ ऽन्येन॑ । \newline
38. अ॒न्येनै᳚न्द्रवाय॒वा दै᳚न्द्रवाय॒वा द॒न्येना॒ न्येनै᳚न्द्रवाय॒वात् । \newline
39. ऐ॒न्द्र॒वा॒य॒वात् प्र॑ति॒पद्य॑न्ते प्रति॒पद्य॑न्त ऐन्द्रवाय॒वा दै᳚न्द्रवाय॒वात् प्र॑ति॒पद्य॑न्ते । \newline
40. ऐ॒न्द्र॒वा॒य॒वादित्यै᳚न्द्र - वा॒य॒वात् । \newline
41. प्र॒ति॒पद्य॒न्ते ऽन्तो ऽन्तः॑ प्रति॒पद्य॑न्ते प्रति॒पद्य॒न्ते ऽन्तः॑ । \newline
42. प्र॒ति॒पद्य॑न्त॒ इति॑ प्रति - पद्य॑न्ते । \newline
43. अन्तः॒ खलु॒ खल्वन्तो ऽन्तः॒ खलु॑ । \newline
44. खलु॒ वै वै खलु॒ खलु॒ वै । \newline
45. वा ए॒ष ए॒ष वै वा ए॒षः । \newline
46. ए॒ष य॒ज्ञ्स्य॑ य॒ज्ञ्स्यै॒ष ए॒ष य॒ज्ञ्स्य॑ । \newline
47. य॒ज्ञ्स्य॒ यद् यद् य॒ज्ञ्स्य॑ य॒ज्ञ्स्य॒ यत् । \newline
48. यद् द॑श॒मम् द॑श॒मं ॅयद् यद् द॑श॒मम् । \newline
49. द॒श॒म मह॒ रह॑र् दश॒मम् द॑श॒म महः॑ । \newline
50. अह॑र् दश॒मे द॑श॒मे ऽह॒ रह॑र् दश॒मे । \newline
51. द॒श॒मे ऽह॒न् नह॑न् दश॒मे द॑श॒मे ऽहन्न्॑ । \newline
52. अह॑न् नैन्द्रवाय॒व ऐ᳚न्द्रवाय॒वो ऽह॒न् नह॑न् नैन्द्रवाय॒वः । \newline
53. ऐ॒न्द्र॒वा॒य॒वो गृ॑ह्यते गृह्यत ऐन्द्रवाय॒व ऐ᳚न्द्रवाय॒वो गृ॑ह्यते । \newline
54. ऐ॒न्द्र॒वा॒य॒व इत्यै᳚न्द्र - वा॒य॒वः । \newline
55. गृ॒ह्य॒ते॒ य॒ज्ञ्स्य॑ य॒ज्ञ्स्य॑ गृह्यते गृह्यते य॒ज्ञ्स्य॑ । \newline
56. य॒ज्ञ् स्यै॒वैव य॒ज्ञ्स्य॑ य॒ज्ञ्स्यै॒व । \newline

\textbf{Ghana Paata } \newline

1. वै तृ॒तीय॑म् तृ॒तीयं॒ ॅवै वै तृ॒तीयं॑ ॅय॒ज्ञ्ं ॅय॒ज्ञ्म् तृ॒तीयं॒ ॅवै वै तृ॒तीयं॑ ॅय॒ज्ञ्म् । \newline
2. तृ॒तीयं॑ ॅय॒ज्ञ्ं ॅय॒ज्ञ्म् तृ॒तीय॑म् तृ॒तीयं॑ ॅय॒ज्ञ् मा॑प दापद् य॒ज्ञ्म् तृ॒तीय॑म् तृ॒तीयं॑ ॅय॒ज्ञ् मा॑पत् । \newline
3. य॒ज्ञ् मा॑प दापद् य॒ज्ञ्ं ॅय॒ज्ञ् मा॑प॒द् यद् यदा॑पद् य॒ज्ञ्ं ॅय॒ज्ञ् मा॑प॒द् यत् । \newline
4. आ॒प॒द् यद् यदा॑प दाप॒द् यच् छन्दाꣳ॑सि॒ छन्दाꣳ॑सि॒ यदा॑प दाप॒द् यच् छन्दाꣳ॑सि । \newline
5. यच् छन्दाꣳ॑सि॒ छन्दाꣳ॑सि॒ यद् यच् छन्दाꣳ॑ स्या॒प्नो त्या॒प्नोति॒ छन्दाꣳ॑सि॒ यद् यच् छन्दाꣳ॑ स्या॒प्नोति॑ । \newline
6. छन्दाꣳ॑ स्या॒प्नो त्या॒प्नोति॒ छन्दाꣳ॑सि॒ छन्दाꣳ॑ स्या॒प्नोति॒ यद् यदा॒प्नोति॒ छन्दाꣳ॑सि॒ छन्दाꣳ॑ स्या॒प्नोति॒ यत् । \newline
7. आ॒प्नोति॒ यद् यदा॒प्नो त्या॒प्नोति॒ यदै᳚न्द्रवाय॒व ऐ᳚न्द्रवाय॒वो यदा॒प्नो त्या॒प्नोति॒ यदै᳚न्द्रवाय॒वः । \newline
8. यदै᳚न्द्रवाय॒व ऐ᳚न्द्रवाय॒वो यद् यदै᳚न्द्रवाय॒वः श्वः श्व ऐ᳚न्द्रवाय॒वो यद् यदै᳚न्द्रवाय॒वः श्वः । \newline
9. ऐ॒न्द्र॒वा॒य॒वः श्वः श्व ऐ᳚न्द्रवाय॒व ऐ᳚न्द्रवाय॒वः श्वो गृ॒ह्यते॑ गृ॒ह्यते॒ श्व ऐ᳚न्द्रवाय॒व ऐ᳚न्द्रवाय॒वः श्वो गृ॒ह्यते᳚ । \newline
10. ऐ॒न्द्र॒वा॒य॒व इत्यै᳚न्द्र - वा॒य॒वः । \newline
11. श्वो गृ॒ह्यते॑ गृ॒ह्यते॒ श्वः श्वो गृ॒ह्यते॒ यत्र॒ यत्र॑ गृ॒ह्यते॒ श्वः श्वो गृ॒ह्यते॒ यत्र॑ । \newline
12. गृ॒ह्यते॒ यत्र॒ यत्र॑ गृ॒ह्यते॑ गृ॒ह्यते॒ यत्रै॒वैव यत्र॑ गृ॒ह्यते॑ गृ॒ह्यते॒ यत्रै॒व । \newline
13. यत्रै॒वैव यत्र॒ यत्रै॒व य॒ज्ञ्ं ॅय॒ज्ञ् मे॒व यत्र॒ यत्रै॒व य॒ज्ञ्म् । \newline
14. ए॒व य॒ज्ञ्ं ॅय॒ज्ञ् मे॒वैव य॒ज्ञ् मदृ॑श॒न् नदृ॑शन्. य॒ज्ञ् मे॒वैव य॒ज्ञ् मदृ॑शन्न् । \newline
15. य॒ज्ञ् मदृ॑श॒न् नदृ॑शन्. य॒ज्ञ्ं ॅय॒ज्ञ् मदृ॑श॒न् तत॒ स्ततो ऽदृ॑शन्. य॒ज्ञ्ं ॅय॒ज्ञ् मदृ॑श॒न् ततः॑ । \newline
16. अदृ॑श॒न् तत॒ स्ततो ऽदृ॑श॒न् नदृ॑श॒न् तत॑ ए॒वैव ततो ऽदृ॑श॒न् नदृ॑श॒न् तत॑ ए॒व । \newline
17. तत॑ ए॒वैव तत॒ स्तत॑ ए॒वैन॑ मेन मे॒व तत॒ स्तत॑ ए॒वैन᳚म् । \newline
18. ए॒वैन॑ मेन मे॒वै वैन॒म् पुनः॒ पुन॑ रेन मे॒वै वैन॒म् पुनः॑ । \newline
19. ए॒न॒म् पुनः॒ पुन॑ रेन मेन॒म् पुनः॒ प्र प्र पुन॑ रेन मेन॒म् पुनः॒ प्र । \newline
20. पुनः॒ प्र प्र पुनः॒ पुनः॒ प्र यु॑ङ्क्ते युङ्क्ते॒ प्र पुनः॒ पुनः॒ प्र यु॑ङ्क्ते । \newline
21. प्र यु॑ङ्क्ते युङ्क्ते॒ प्र प्र यु॒ङ्क्ते ऽथो॒ अथो॑ युङ्क्ते॒ प्र प्र यु॒ङ्क्ते ऽथो᳚ । \newline
22. यु॒ङ्क्ते ऽथो॒ अथो॑ युङ्क्ते यु॒ङ्क्ते ऽथो॒ स्वꣳ स्व मथो॑ युङ्क्ते यु॒ङ्क्ते ऽथो॒ स्वम् । \newline
23. अथो॒ स्वꣳ स्व मथो॒ अथो॒ स्व मे॒वैव स्व मथो॒ अथो॒ स्व मे॒व । \newline
24. अथो॒ इत्यथो᳚ । \newline
25. स्व मे॒वैव स्वꣳ स्व मे॒व छन्द॒ श्छन्द॑ ए॒व स्वꣳ स्व मे॒व छन्दः॑ । \newline
26. ए॒व छन्द॒ श्छन्द॑ ए॒वैव छन्दो ऽन्वनु॒ च्छन्द॑ ए॒वैव छन्दो ऽनु॑ । \newline
27. छन्दो ऽन्वनु॒ च्छन्द॒ श्छन्दो ऽनु॑ प॒र्याव॑र्तन्ते प॒र्याव॑र्त॒न्ते ऽनु॒ च्छन्द॒ श्छन्दो ऽनु॑ प॒र्याव॑र्तन्ते । \newline
28. अनु॑ प॒र्याव॑र्तन्ते प॒र्याव॑र्त॒न्ते ऽन्वनु॑ प॒र्याव॑र्तन्ते प॒थः प॒थः प॒र्याव॑र्त॒न्ते ऽन्वनु॑ प॒र्याव॑र्तन्ते प॒थः । \newline
29. प॒र्याव॑र्तन्ते प॒थः प॒थः प॒र्याव॑र्तन्ते प॒र्याव॑र्तन्ते प॒थो वै वै प॒थः प॒र्याव॑र्तन्ते प॒र्याव॑र्तन्ते प॒थो वै । \newline
30. प॒र्याव॑र्तन्त॒ इति॑ परि - आव॑र्तन्ते । \newline
31. प॒थो वै वै प॒थः प॒थो वा ए॒त ए॒ते वै प॒थः प॒थो वा ए॒ते । \newline
32. वा ए॒त ए॒ते वै वा ए॒ते ऽध्य ध्ये॒ते वै वा ए॒ते ऽधि॑ । \newline
33. ए॒ते ऽध्य ध्ये॒त ए॒ते ऽध्य प॑थे॒ना प॑थे॒ना ध्ये॒त ए॒ते ऽध्य प॑थेन । \newline
34. अध्य प॑थे॒ना प॑थे॒ना ध्यध्य प॑थेन यन्ति य॒न्त्य प॑थे॒ना ध्यध्य प॑थेन यन्ति । \newline
35. अप॑थेन यन्ति य॒न्त्यप॑थे॒ना प॑थेन यन्ति॒ ये ये य॒न्त्य प॑थे॒ना प॑थेन यन्ति॒ ये । \newline
36. य॒न्ति॒ ये ये य॑न्ति यन्ति॒ ये᳚ ऽन्येना॒ न्येन॒ ये य॑न्ति यन्ति॒ ये᳚ ऽन्येन॑ । \newline
37. ये᳚ ऽन्येना॒ न्येन॒ ये ये᳚ ऽन्ये नै᳚न्द्रवाय॒वा दै᳚न्द्रवाय॒वा द॒न्येन॒ ये ये᳚ ऽन्येनै᳚न्द्रवाय॒वात् । \newline
38. अ॒न्ये नै᳚न्द्रवाय॒वा दै᳚न्द्रवाय॒वा द॒न्येना॒ न्येनै᳚न्द्रवाय॒वात् प्र॑ति॒पद्य॑न्ते प्रति॒पद्य॑न्त ऐन्द्रवाय॒वा द॒न्येना॒न्ये नै᳚न्द्रवाय॒वात् प्र॑ति॒पद्य॑न्ते । \newline
39. ऐ॒न्द्र॒वा॒य॒वात् प्र॑ति॒पद्य॑न्ते प्रति॒पद्य॑न्त ऐन्द्रवाय॒वा दै᳚न्द्रवाय॒वात् प्र॑ति॒पद्य॒न्ते ऽन्तो ऽन्तः॑ प्रति॒पद्य॑न्त ऐन्द्रवाय॒वा दै᳚न्द्रवाय॒वात् प्र॑ति॒पद्य॒न्ते ऽन्तः॑ । \newline
40. ऐ॒न्द्र॒वा॒य॒वादित्यै᳚न्द्र - वा॒य॒वात् । \newline
41. प्र॒ति॒पद्य॒न्ते ऽन्तो ऽन्तः॑ प्रति॒पद्य॑न्ते प्रति॒पद्य॒न्ते ऽन्तः॒ खलु॒ खल्वन्तः॑ प्रति॒पद्य॑न्ते प्रति॒पद्य॒न्ते ऽन्तः॒ खलु॑ । \newline
42. प्र॒ति॒पद्य॑न्त॒ इति॑ प्रति - पद्य॑न्ते । \newline
43. अन्तः॒ खलु॒ खल्वन्तो ऽन्तः॒ खलु॒ वै वै खल्वन्तो ऽन्तः॒ खलु॒ वै । \newline
44. खलु॒ वै वै खलु॒ खलु॒ वा ए॒ष ए॒ष वै खलु॒ खलु॒ वा ए॒षः । \newline
45. वा ए॒ष ए॒ष वै वा ए॒ष य॒ज्ञ्स्य॑ य॒ज्ञ्स्यै॒ष वै वा ए॒ष य॒ज्ञ्स्य॑ । \newline
46. ए॒ष य॒ज्ञ्स्य॑ य॒ज्ञ् स्यै॒ष ए॒ष य॒ज्ञ्स्य॒ यद् यद् य॒ज्ञ्स्यै॒ष ए॒ष य॒ज्ञ्स्य॒ यत् । \newline
47. य॒ज्ञ्स्य॒ यद् यद् य॒ज्ञ्स्य॑ य॒ज्ञ्स्य॒ यद् द॑श॒मम् द॑श॒मं ॅयद् य॒ज्ञ्स्य॑ य॒ज्ञ्स्य॒ यद् द॑श॒मम् । \newline
48. यद् द॑श॒मम् द॑श॒मं ॅयद् यद् द॑श॒म मह॒ रह॑र् दश॒मं ॅयद् यद् द॑श॒म महः॑ । \newline
49. द॒श॒म मह॒ रह॑र् दश॒मम् द॑श॒म मह॑र् दश॒मे द॑श॒मे ऽह॑र् दश॒मम् द॑श॒म मह॑र् दश॒मे । \newline
50. अह॑र् दश॒मे द॑श॒मे ऽह॒ रह॑र् दश॒मे ऽह॒न् नह॑न् दश॒मे ऽह॒ रह॑र् दश॒मे ऽहन्न्॑ । \newline
51. द॒श॒मे ऽह॒न् नह॑न् दश॒मे द॑श॒मे ऽह॑न् नैन्द्रवाय॒व ऐ᳚न्द्रवाय॒वो ऽह॑न् दश॒मे द॑श॒मे ऽह॑न् नैन्द्रवाय॒वः । \newline
52. अह॑न् नैन्द्रवाय॒व ऐ᳚न्द्रवाय॒वो ऽह॒न् नह॑न् नैन्द्रवाय॒वो गृ॑ह्यते गृह्यत ऐन्द्रवाय॒वो ऽह॒न् नह॑न् नैन्द्रवाय॒वो गृ॑ह्यते । \newline
53. ऐ॒न्द्र॒वा॒य॒वो गृ॑ह्यते गृह्यत ऐन्द्रवाय॒व ऐ᳚न्द्रवाय॒वो गृ॑ह्यते य॒ज्ञ्स्य॑ य॒ज्ञ्स्य॑ गृह्यत ऐन्द्रवाय॒व ऐ᳚न्द्रवाय॒वो गृ॑ह्यते य॒ज्ञ्स्य॑ । \newline
54. ऐ॒न्द्र॒वा॒य॒व इत्यै᳚न्द्र - वा॒य॒वः । \newline
55. गृ॒ह्य॒ते॒ य॒ज्ञ्स्य॑ य॒ज्ञ्स्य॑ गृह्यते गृह्यते य॒ज्ञ् स्यै॒वैव य॒ज्ञ्स्य॑ गृह्यते गृह्यते य॒ज्ञ् स्यै॒व । \newline
56. य॒ज्ञ्स्यै॒ वैव य॒ज्ञ्स्य॑ य॒ज्ञ् स्यै॒वान्त॒ मन्त॑ मे॒व य॒ज्ञ्स्य॑ य॒ज्ञ्स्यै॒ वान्त᳚म् । \newline
\pagebreak
\markright{ TS 7.2.8.6  \hfill https://www.vedavms.in \hfill}

\section{ TS 7.2.8.6 }

\textbf{TS 7.2.8.6 } \newline
\textbf{Samhita Paata} \newline

-वान्तं॑ ग॒त्वा ऽप॑था॒त् पन्था॒मपि॑ य॒न्त्यथो॒ यथा॒ वही॑यसा प्रति॒सारं॒ ॅवह॑न्ति ता॒दृगे॒व तच्छन्दाꣳ॑स्य॒न्यो᳚ऽन्यस्य॑ लो॒कम॒भ्य॑द्ध्याय॒न् तान्ये॒तेनै॒व दे॒वा व्य॑वाहयन्नैन्द्रवाय॒वस्य॒ वा ए॒तदा॒यत॑नं॒ ॅयच्च॑तु॒र्थमह॒स्तस्मि॑-न्नाग्रय॒णो गृ॑ह्यते॒ तस्मा॑दाग्रय॒णस्या॒ ऽऽ*यत॑ने नव॒मेऽह॑न्नैन्द्रवाय॒वो गृ॑ह्यते शु॒क्रस्य॒ वा ए॒तदा॒यत॑नं॒ ॅयत् प॑ञ्च॒म - [  ] \newline

\textbf{Pada Paata} \newline

ए॒व । अन्त᳚म् । ग॒त्वा । अप॑थात् । पन्था᳚म् । अपीति॑ । य॒न्ति॒ । अथो॒ इति॑ । यथा᳚ । वही॑यसा । प्र॒ति॒सार॒मिति॑ प्रति-सार᳚म् । वह॑न्ति । ता॒दृक् । ए॒व । तत् । छन्दाꣳ॑सि । अ॒न्यः । अ॒न्यस्य॑ । लो॒कम् । अ॒भीति॑ । अ॒द्ध्या॒य॒न्न् । तानि॑ । ए॒तेन॑ । ए॒व । दे॒वाः । वीति॑ । अ॒वा॒ह॒य॒न्न् । ऐ॒न्द्र॒वा॒य॒वस्येत्यै᳚न्द्र - वा॒य॒वस्य॑ । वै । ए॒तत् । आ॒यत॑न॒मित्या᳚ - यत॑नम् । यत् । च॒तु॒र्थम् । अहः॑ । तस्मिन्न्॑ । आ॒ग्र॒य॒णः । गृ॒ह्य॒ते॒ । तस्मा᳚त् । आ॒ग्र॒य॒णस्य॑ । आ॒यत॑न॒ इत्या᳚-यत॑ने । न॒व॒मे । अहन्न्॑ । ऐ॒न्द्र॒वा॒य॒व इत्यै᳚न्द्र - वा॒य॒वः । गृ॒ह्य॒ते॒ । शु॒क्रस्य॑ । वै । ए॒तत् । आ॒यत॑न॒मित्या᳚ - यत॑नम् । यत् । प॒ञ्च॒मम् ।  \newline


\textbf{Krama Paata} \newline

ए॒वान्त᳚म् । अन्त॑म् ग॒त्वा । ग॒त्वाऽप॑थात् । अप॑था॒त् पन्था᳚म् । पन्था॒मपि॑ । अपि॑ यन्ति । य॒न्त्यथो᳚ । अथो॒ यथा᳚ । अथो॒ इत्यथो᳚ । यथा॒ वही॑यसा । वही॑यसा प्रति॒सार᳚म् । प्र॒ति॒सार॒म् ॅवह॑न्ति । प्र॒ति॒सार॒मिति॑ प्रति - सार᳚म् । वह॑न्ति ता॒दृक् । ता॒दृगे॒व । ए॒व तत् । तच् छन्दाꣳ॑सि । छन्दाꣳ॑स्य॒न्यः । अ॒न्यो᳚ऽन्यस्य॑ । अ॒न्यस्य॑ लो॒कम् । लो॒कम॒भि । अ॒भ्य॑द्ध्यायन्न् । अ॒द्ध्या॒य॒न् तानि॑ । तान्ये॒तेन॑ । ए॒तेनै॒व । ए॒व दे॒वाः । दे॒वा वि । व्य॑वाहयन्न् । अ॒वा॒ह॒य॒न्नै॒न्द्र॒वा॒य॒वस्य॑ । ऐ॒न्द्र॒वा॒य॒वस्य॒ वै । ऐ॒न्द्र॒वा॒य॒वस्येत्यै᳚न्द्र - वा॒य॒वस्य॑ । वा ए॒तत् । ए॒तदा॒यत॑नम् । आ॒यत॑न॒म् ॅयत् । आ॒यत॑न॒मित्या᳚ - यत॑नम् । यच् च॑तु॒र्थम् । च॒तु॒र्थमहः॑ । अह॒स्तस्मिन्न्॑ । तस्मि॑न्नाग्रय॒णः । आ॒ग्र॒य॒णो गृ॑ह्यते । गृ॒ह्य॒ते॒ तस्मा᳚त् । तस्मा॑दाग्रय॒णस्य॑ । आ॒ग्र॒य॒णस्या॒यत॑ने । आ॒यत॑ने नव॒मे । आ॒यत॑न॒ इत्या᳚ - यत॑ने । न॒व॒मेऽहन्न्॑ । अह॑न्नैन्द्रवाय॒वः । ऐ॒न्द्र॒वा॒य॒वो गृ॑ह्यते । ऐ॒न्द्र॒वा॒य॒व इत्यै᳚न्द्र - वा॒य॒वः । गृ॒ह्य॒ते॒ शु॒क्रस्य॑ । शु॒क्रस्य॒ वै । वा ए॒तत् । ए॒तदा॒यत॑नम् । आ॒यत॑न॒म् ॅयत् । आ॒यत॑न॒मित्या᳚ - यत॑नम् । यत् प॑ञ्च॒मम् । प॒ञ्च॒ममहः॑ \newline

\textbf{Jatai Paata} \newline

1. ए॒वान्त॒ मन्त॑ मे॒वै वान्त᳚म् । \newline
2. अन्त॑म् ग॒त्वा ग॒त्वा ऽन्त॒ मन्त॑म् ग॒त्वा । \newline
3. ग॒त्वा ऽप॑था॒ दप॑थाद् ग॒त्वा ग॒त्वा ऽप॑थात् । \newline
4. अप॑था॒त् पन्था॒म् पन्था॒ मप॑था॒ दप॑था॒त् पन्था᳚म् । \newline
5. पन्था॒ मप्यपि॒ पन्था॒म् पन्था॒ मपि॑ । \newline
6. अपि॑ यन्ति य॒न्त्य प्यपि॑ यन्ति । \newline
7. य॒न्त्यथो॒ अथो॑ यन्ति य॒न्त्यथो᳚ । \newline
8. अथो॒ यथा॒ यथा ऽथो॒ अथो॒ यथा᳚ । \newline
9. अथो॒ इत्यथो᳚ । \newline
10. यथा॒ वही॑यसा॒ वही॑यसा॒ यथा॒ यथा॒ वही॑यसा । \newline
11. वही॑यसा प्रति॒सार॑म् प्रति॒सारं॒ ॅवही॑यसा॒ वही॑यसा प्रति॒सार᳚म् । \newline
12. प्र॒ति॒सारं॒ ॅवह॑न्ति॒ वह॑न्ति प्रति॒सार॑म् प्रति॒सारं॒ ॅवह॑न्ति । \newline
13. प्र॒ति॒सार॒मिति॑ प्रति - सार᳚म् । \newline
14. वह॑न्ति ता॒दृक् ता॒दृग् वह॑न्ति॒ वह॑न्ति ता॒दृक् । \newline
15. ता॒दृ गे॒वैव ता॒दृक् ता॒दृ गे॒व । \newline
16. ए॒व तत् तदे॒ वैव तत् । \newline
17. तच् छन्दाꣳ॑सि॒ छन्दाꣳ॑सि॒ तत् तच् छन्दाꣳ॑सि । \newline
18. छन्दाꣳ॑ स्य॒न्यो᳚ ऽन्य श्छन्दाꣳ॑सि॒ छन्दाꣳ॑ स्य॒न्यः । \newline
19. अ॒न्यो᳚ ऽन्यस्या॒ न्यस्या॒ न्यो᳚(1॒) ऽन्यो᳚ ऽन्यस्य॑ । \newline
20. अ॒न्यस्य॑ लो॒कम् ॅलो॒क म॒न्यस्या॒ न्यस्य॑ लो॒कम् । \newline
21. लो॒क म॒भ्य॑भि लो॒कम् ॅलो॒क म॒भि । \newline
22. अ॒भ्य॑द्ध्यायन् नद्ध्यायन् न॒भ्या᳚(1॒) भ्य॑द्ध्यायन्न् । \newline
23. अ॒द्ध्या॒य॒न् तानि॒ तान्य॑द्ध्यायन् नद्ध्याय॒न् तानि॑ । \newline
24. तान्ये॒ते नै॒तेन॒ तानि॒ तान्ये॒तेन॑ । \newline
25. ए॒ते नै॒वै वैते नै॒ते नै॒व । \newline
26. ए॒व दे॒वा दे॒वा ए॒वैव दे॒वाः । \newline
27. दे॒वा वि वि दे॒वा दे॒वा वि । \newline
28. व्य॑वाहयन् नवाहय॒न्॒. वि व्य॑वाहयन्न् । \newline
29. अ॒वा॒ह॒य॒न् नै॒न्द्र॒वा॒य॒व स्यै᳚न्द्रवाय॒वस्या॑ वाहयन् नवाहयन् नैन्द्रवाय॒वस्य॑ । \newline
30. ऐ॒न्द्र॒वा॒य॒वस्य॒ वै वा ऐ᳚न्द्रवाय॒व स्यै᳚न्द्रवाय॒वस्य॒ वै । \newline
31. ऐ॒न्द्र॒वा॒य॒वस्येत्यै᳚न्द्र - वा॒य॒वस्य॑ । \newline
32. वा ए॒त दे॒तद् वै वा ए॒तत् । \newline
33. ए॒त दा॒यत॑न मा॒यत॑न मे॒त दे॒त दा॒यत॑नम् । \newline
34. आ॒यत॑नं॒ ॅयद् यदा॒यत॑न मा॒यत॑नं॒ ॅयत् । \newline
35. आ॒यत॑न॒मित्या᳚ - यत॑नम् । \newline
36. यच् च॑तु॒र्थम् च॑तु॒र्थं ॅयद् यच् च॑तु॒र्थम् । \newline
37. च॒तु॒र्थ मह॒ रह॑ श्चतु॒र्थम् च॑तु॒र्थ महः॑ । \newline
38. अह॒ स्तस्मिꣳ॒॒ स्तस्मि॒न् नह॒ रह॒ स्तस्मिन्न्॑ । \newline
39. तस्मि॑न् नाग्रय॒ण आ᳚ग्रय॒ण स्तस्मिꣳ॒॒ स्तस्मि॑न् नाग्रय॒णः । \newline
40. आ॒ग्र॒य॒णो गृ॑ह्यते गृह्यत आग्रय॒ण आ᳚ग्रय॒णो गृ॑ह्यते । \newline
41. गृ॒ह्य॒ते॒ तस्मा॒त् तस्मा᳚द् गृह्यते गृह्यते॒ तस्मा᳚त् । \newline
42. तस्मा॑ दाग्रय॒णस्या᳚ ग्रय॒णस्य॒ तस्मा॒त् तस्मा॑ दाग्रय॒णस्य॑ । \newline
43. आ॒ग्र॒य॒णस्या॒ यत॑न आ॒यत॑न आग्रय॒णस्या᳚ ग्रय॒णस्या॒ यत॑ने । \newline
44. आ॒यत॑ने नव॒मे न॑व॒म आ॒यत॑न आ॒यत॑ने नव॒मे । \newline
45. आ॒यत॑न॒ इत्या᳚ - यत॑ने । \newline
46. न॒व॒मे ऽह॒न् नह॑न् नव॒मे न॑व॒मे ऽहन्न्॑ । \newline
47. अह॑न् नैन्द्रवाय॒व ऐ᳚न्द्रवाय॒वो ऽह॒न् नह॑न् नैन्द्रवाय॒वः । \newline
48. ऐ॒न्द्र॒वा॒य॒वो गृ॑ह्यते गृह्यत ऐन्द्रवाय॒व ऐ᳚न्द्रवाय॒वो गृ॑ह्यते । \newline
49. ऐ॒न्द्र॒वा॒य॒व इत्यै᳚न्द्र - वा॒य॒वः । \newline
50. गृ॒ह्य॒ते॒ शु॒क्रस्य॑ शु॒क्रस्य॑ गृह्यते गृह्यते शु॒क्रस्य॑ । \newline
51. शु॒क्रस्य॒ वै वै शु॒क्रस्य॑ शु॒क्रस्य॒ वै । \newline
52. वा ए॒त दे॒तद् वै वा ए॒तत् । \newline
53. ए॒त दा॒यत॑न मा॒यत॑न मे॒त दे॒त दा॒यत॑नम् । \newline
54. आ॒यत॑नं॒ ॅयद् यदा॒यत॑न मा॒यत॑नं॒ ॅयत् । \newline
55. आ॒यत॑न॒मित्या᳚ - यत॑नम् । \newline
56. यत् प॑ञ्च॒मम् प॑ञ्च॒मं ॅयद् यत् प॑ञ्च॒मम् । \newline
57. प॒ञ्च॒म मह॒ रहः॑ पञ्च॒मम् प॑ञ्च॒म महः॑ । \newline

\textbf{Ghana Paata } \newline

1. ए॒वान्त॒ मन्त॑ मे॒वै वान्त॑म् ग॒त्वा ग॒त्वा ऽन्त॑ मे॒वै वान्त॑म् ग॒त्वा । \newline
2. अन्त॑म् ग॒त्वा ग॒त्वा ऽन्त॒ मन्त॑म् ग॒त्वा ऽप॑था॒ दप॑थाद् ग॒त्वा ऽन्त॒ मन्त॑म् ग॒त्वा ऽप॑थात् । \newline
3. ग॒त्वा ऽप॑था॒ दप॑थाद् ग॒त्वा ग॒त्वा ऽप॑था॒त् पन्था॒म् पन्था॒ मप॑थाद् ग॒त्वा ग॒त्वा ऽप॑था॒त् पन्था᳚म् । \newline
4. अप॑था॒त् पन्था॒म् पन्था॒ मप॑था॒ दप॑था॒त् पन्था॒ मप्यपि॒ पन्था॒ मप॑था॒ दप॑था॒त् पन्था॒ मपि॑ । \newline
5. पन्था॒ मप्यपि॒ पन्था॒म् पन्था॒ मपि॑ यन्ति य॒न्त्यपि॒ पन्था॒म् पन्था॒ मपि॑ यन्ति । \newline
6. अपि॑ यन्ति य॒न्त्यप्यपि॑ य॒न्त्यथो॒ अथो॑ य॒न्त्यप्यपि॑ य॒न्त्यथो᳚ । \newline
7. य॒न्त्यथो॒ अथो॑ यन्ति य॒न्त्यथो॒ यथा॒ यथा ऽथो॑ यन्ति य॒न्त्यथो॒ यथा᳚ । \newline
8. अथो॒ यथा॒ यथा ऽथो॒ अथो॒ यथा॒ वही॑यसा॒ वही॑यसा॒ यथा ऽथो॒ अथो॒ यथा॒ वही॑यसा । \newline
9. अथो॒ इत्यथो᳚ । \newline
10. यथा॒ वही॑यसा॒ वही॑यसा॒ यथा॒ यथा॒ वही॑यसा प्रति॒सार॑म् प्रति॒सारं॒ ॅवही॑यसा॒ यथा॒ यथा॒ वही॑यसा प्रति॒सार᳚म् । \newline
11. वही॑यसा प्रति॒सार॑म् प्रति॒सारं॒ ॅवही॑यसा॒ वही॑यसा प्रति॒सारं॒ ॅवह॑न्ति॒ वह॑न्ति प्रति॒सारं॒ ॅवही॑यसा॒ वही॑यसा प्रति॒सारं॒ ॅवह॑न्ति । \newline
12. प्र॒ति॒सारं॒ ॅवह॑न्ति॒ वह॑न्ति प्रति॒सार॑म् प्रति॒सारं॒ ॅवह॑न्ति ता॒दृक् ता॒दृग् वह॑न्ति प्रति॒सार॑म् प्रति॒सारं॒ ॅवह॑न्ति ता॒दृक् । \newline
13. प्र॒ति॒सार॒मिति॑ प्रति - सार᳚म् । \newline
14. वह॑न्ति ता॒दृक् ता॒दृग् वह॑न्ति॒ वह॑न्ति ता॒दृ गे॒वैव ता॒दृग् वह॑न्ति॒ वह॑न्ति ता॒दृ गे॒व । \newline
15. ता॒दृ गे॒वैव ता॒दृक् ता॒दृ गे॒व तत् तदे॒व ता॒दृक् ता॒दृ गे॒व तत् । \newline
16. ए॒व तत् तदे॒वैव तच् छन्दाꣳ॑सि॒ छन्दाꣳ॑सि॒ तदे॒वैव तच् छन्दाꣳ॑सि । \newline
17. तच् छन्दाꣳ॑सि॒ छन्दाꣳ॑सि॒ तत् तच् छन्दाꣳ॑ स्य॒न्यो᳚ ऽन्य श्छन्दाꣳ॑सि॒ तत् तच् छन्दाꣳ॑ स्य॒न्यः । \newline
18. छन्दाꣳ॑ स्य॒न्यो᳚ ऽन्य श्छन्दाꣳ॑सि॒ छन्दाꣳ॑ स्य॒न्यो᳚ ऽन्यस्या॒ न्यस्या॒न्य श्छन्दाꣳ॑सि॒ छन्दाꣳ॑ स्य॒न्यो᳚ ऽन्यस्य॑ । \newline
19. अ॒न्यो᳚ ऽन्यस्या॒ न्यस्या॒न्यो᳚(1॒) ऽन्यो᳚ ऽन्यस्य॑ लो॒कम् ॅलो॒क म॒न्यस्या॒ न्यो᳚(1॒) ऽन्यो᳚ ऽन्यस्य॑ लो॒कम् । \newline
20. अ॒न्यस्य॑ लो॒कम् ॅलो॒क म॒न्यस्या॒ न्यस्य॑ लो॒क म॒भ्य॑भि लो॒क म॒न्यस्या॒ न्यस्य॑ लो॒क म॒भि । \newline
21. लो॒क म॒भ्य॑भि लो॒कम् ॅलो॒क म॒भ्य॑द्ध्यायन् नद्ध्यायन् न॒भि लो॒कम् ॅलो॒क म॒भ्य॑द्ध्यायन्न् । \newline
22. अ॒भ्य॑द्ध्यायन् नद्ध्यायन् न॒भ्या᳚(1॒)भ्य॑द्ध्याय॒न् तानि॒ तान्य॑द्ध्यायन् न॒भ्या᳚(1॒)भ्य॑द्ध्याय॒न् तानि॑ । \newline
23. अ॒द्ध्या॒य॒न् तानि॒ तान्य॑द्ध्यायन् नद्ध्याय॒न् तान्ये॒ते नै॒तेन॒ तान्य॑द्ध्यायन् नद्ध्याय॒न् तान्ये॒तेन॑ । \newline
24. तान्ये॒ते नै॒तेन॒ तानि॒ तान्ये॒ते नै॒वै वैतेन॒ तानि॒ तान्ये॒ते नै॒व । \newline
25. ए॒तेनै॒वै वैते नै॒ते नै॒व दे॒वा दे॒वा ए॒वैते नै॒ते नै॒व दे॒वाः । \newline
26. ए॒व दे॒वा दे॒वा ए॒वैव दे॒वा वि वि दे॒वा ए॒वैव दे॒वा वि । \newline
27. दे॒वा वि वि दे॒वा दे॒वा व्य॑वाहयन् नवाहय॒न्॒. वि दे॒वा दे॒वा व्य॑वाहयन्न् । \newline
28. व्य॑वाहयन् नवाहय॒न्॒. वि व्य॑वाहयन् नैन्द्रवाय॒व स्यै᳚न्द्रवाय॒वस्या॑ वाहय॒न्॒. वि व्य॑वाहयन् नैन्द्रवाय॒वस्य॑ । \newline
29. अ॒वा॒ह॒य॒न् नै॒न्द्र॒वा॒य॒व स्यै᳚न्द्रवाय॒वस्या॑ वाहयन् नवाहयन् नैन्द्रवाय॒वस्य॒ वै वा ऐ᳚न्द्रवाय॒वस्या॑ वाहयन् नवाहयन् नैन्द्रवाय॒वस्य॒ वै । \newline
30. ऐ॒न्द्र॒वा॒य॒वस्य॒ वै वा ऐ᳚न्द्रवाय॒व स्यै᳚न्द्रवाय॒वस्य॒ वा ए॒त दे॒तद् वा ऐ᳚न्द्रवाय॒व
स्यै᳚न्द्रवाय॒वस्य॒ वा ए॒तत् । \newline
31. ऐ॒न्द्र॒वा॒य॒वस्येत्यै᳚न्द्र - वा॒य॒वस्य॑ । \newline
32. वा ए॒त दे॒तद् वै वा ए॒त दा॒यत॑न मा॒यत॑न मे॒तद् वै वा ए॒त दा॒यत॑नम् । \newline
33. ए॒त दा॒यत॑न मा॒यत॑न मे॒त दे॒त दा॒यत॑नं॒ ॅयद् यदा॒यत॑न मे॒त दे॒त दा॒यत॑नं॒ ॅयत् । \newline
34. आ॒यत॑नं॒ ॅयद् यदा॒यत॑न मा॒यत॑नं॒ ॅयच् च॑तु॒र्थम् च॑तु॒र्थं ॅयदा॒यत॑न मा॒यत॑नं॒ ॅयच् च॑तु॒र्थम् । \newline
35. आ॒यत॑न॒मित्या᳚ - यत॑नम् । \newline
36. यच् च॑तु॒र्थम् च॑तु॒र्थं ॅयद् यच् च॑तु॒र्थ मह॒ रह॑ श्चतु॒र्थं ॅयद् यच् च॑तु॒र्थ महः॑ । \newline
37. च॒तु॒र्थ मह॒ रह॑ श्चतु॒र्थम् च॑तु॒र्थ मह॒ स्तस्मिꣳ॒॒ स्तस्मि॒न् नह॑ श्चतु॒र्थम् च॑तु॒र्थ मह॒ स्तस्मिन्न्॑ । \newline
38. अह॒ स्तस्मिꣳ॒॒ स्तस्मि॒न् नह॒ रह॒ स्तस्मि॑न् नाग्रय॒ण आ᳚ग्रय॒ण स्तस्मि॒न् नह॒ रह॒ स्तस्मि॑न् नाग्रय॒णः । \newline
39. तस्मि॑न् नाग्रय॒ण आ᳚ग्रय॒ण स्तस्मिꣳ॒॒ स्तस्मि॑न् नाग्रय॒णो गृ॑ह्यते गृह्यत आग्रय॒ण स्तस्मिꣳ॒॒ स्तस्मि॑न् नाग्रय॒णो गृ॑ह्यते । \newline
40. आ॒ग्र॒य॒णो गृ॑ह्यते गृह्यत आग्रय॒ण आ᳚ग्रय॒णो गृ॑ह्यते॒ तस्मा॒त् तस्मा᳚द् गृह्यत आग्रय॒ण आ᳚ग्रय॒णो गृ॑ह्यते॒ तस्मा᳚त् । \newline
41. गृ॒ह्य॒ते॒ तस्मा॒त् तस्मा᳚द् गृह्यते गृह्यते॒ तस्मा॑ दाग्रय॒णस्या᳚ ग्रय॒णस्य॒ तस्मा᳚द् गृह्यते गृह्यते॒ तस्मा॑ दाग्रय॒णस्य॑ । \newline
42. तस्मा॑ दाग्रय॒णस्या᳚ ग्रय॒णस्य॒ तस्मा॒त् तस्मा॑ दाग्रय॒णस्या॒ यत॑न आ॒यत॑न आग्रय॒णस्य॒ तस्मा॒त् तस्मा॑ दाग्रय॒णस्या॒ यत॑ने । \newline
43. आ॒ग्र॒य॒णस्या॒ यत॑न आ॒यत॑न आग्रय॒णस्या᳚ ग्रय॒णस्या॒ यत॑ने नव॒मे न॑व॒म आ॒यत॑न आग्रय॒णस्या᳚ ग्रय॒णस्या॒ यत॑ने नव॒मे । \newline
44. आ॒यत॑ने नव॒मे न॑व॒म आ॒यत॑न आ॒यत॑ने नव॒मे ऽह॒न् नह॑न् नव॒म आ॒यत॑न आ॒यत॑ने नव॒मे ऽहन्न्॑ । \newline
45. आ॒यत॑न॒ इत्या᳚ - यत॑ने । \newline
46. न॒व॒मे ऽह॒न् नह॑न् नव॒मे न॑व॒मे ऽह॑न् नैन्द्रवाय॒व ऐ᳚न्द्रवाय॒वो ऽह॑न् नव॒मे न॑व॒मे ऽह॑न् नैन्द्रवाय॒वः । \newline
47. अह॑न् नैन्द्रवाय॒व ऐ᳚न्द्रवाय॒वो ऽह॒न् नह॑न् नैन्द्रवाय॒वो गृ॑ह्यते गृह्यत ऐन्द्रवाय॒वो ऽह॒न् नह॑न् नैन्द्रवाय॒वो गृ॑ह्यते । \newline
48. ऐ॒न्द्र॒वा॒य॒वो गृ॑ह्यते गृह्यत ऐन्द्रवाय॒व ऐ᳚न्द्रवाय॒वो गृ॑ह्यते शु॒क्रस्य॑ शु॒क्रस्य॑ गृह्यत ऐन्द्रवाय॒व ऐ᳚न्द्रवाय॒वो गृ॑ह्यते शु॒क्रस्य॑ । \newline
49. ऐ॒न्द्र॒वा॒य॒व इत्यै᳚न्द्र - वा॒य॒वः । \newline
50. गृ॒ह्य॒ते॒ शु॒क्रस्य॑ शु॒क्रस्य॑ गृह्यते गृह्यते शु॒क्रस्य॒ वै वै शु॒क्रस्य॑ गृह्यते गृह्यते शु॒क्रस्य॒ वै । \newline
51. शु॒क्रस्य॒ वै वै शु॒क्रस्य॑ शु॒क्रस्य॒ वा ए॒त दे॒तद् वै शु॒क्रस्य॑ शु॒क्रस्य॒ वा ए॒तत् । \newline
52. वा ए॒त दे॒तद् वै वा ए॒त दा॒यत॑न मा॒यत॑न मे॒तद् वै वा ए॒त दा॒यत॑नम् । \newline
53. ए॒त दा॒यत॑न मा॒यत॑न मे॒त दे॒त दा॒यत॑नं॒ ॅयद् यदा॒यत॑न मे॒त दे॒त दा॒यत॑नं॒ ॅयत् । \newline
54. आ॒यत॑नं॒ ॅयद् यदा॒यत॑न मा॒यत॑नं॒ ॅयत् प॑ञ्च॒मम् प॑ञ्च॒मं ॅयदा॒यत॑न मा॒यत॑नं॒ ॅयत् प॑ञ्च॒मम् । \newline
55. आ॒यत॑न॒मित्या᳚ - यत॑नम् । \newline
56. यत् प॑ञ्च॒मम् प॑ञ्च॒मं ॅयद् यत् प॑ञ्च॒म मह॒ रहः॑ पञ्च॒मं ॅयद् यत् प॑ञ्च॒म महः॑ । \newline
57. प॒ञ्च॒म मह॒ रहः॑ पञ्च॒मम् प॑ञ्च॒म मह॒ स्तस्मिꣳ॒॒ स्तस्मि॒न् नहः॑ पञ्च॒मम् प॑ञ्च॒म मह॒ स्तस्मिन्न्॑ । \newline
\pagebreak
\markright{ TS 7.2.8.7  \hfill https://www.vedavms.in \hfill}

\section{ TS 7.2.8.7 }

\textbf{TS 7.2.8.7 } \newline
\textbf{Samhita Paata} \newline

-मह॒स्तस्मि॑न्नैन्द्रवाय॒वो गृ॑ह्यते॒ तस्मा॑दैन्द्रवाय॒वस्या॒ऽऽ*यत॑ने सप्त॒मेऽह॑ञ्छु॒क्रो गृ॑ह्यत आग्रय॒णस्य॒ वा ए॒तदा॒यत॑नं ॅयथ् ष॒ष्ठमह॒स्तस्मि॑ञ्छु॒क्रो गृ॑ह्यते॒ तस्मा᳚च्छु॒क्रस्या॒ ऽऽ*यत॑नेऽष्ट॒मेऽह॑न्नाग्रय॒णो गृ॑ह्यते॒ छन्दाꣳ॑स्ये॒व तद्वि वा॑हयति॒ प्र वस्य॑सो विवा॒हमा᳚प्नोति॒ य ए॒वं ॅवेदाथो॑ दे॒वता᳚भ्य ए॒व य॒ज्ञे सं॒ॅविदं॑ दधाति॒ तस्मा॑दि॒द-म॒न्यो᳚ऽन्यस्मै॑ ( ) ददाति ॥ \newline

\textbf{Pada Paata} \newline

अहः॑ । तस्मिन्न्॑ । ऐ॒न्द्र॒वा॒य॒व इत्यै᳚न्द्र - वा॒य॒वः । गृ॒ह्य॒ते॒ । तस्मा᳚त् । ऐ॒न्द्र॒वा॒य॒वस्येत्यै᳚न्द्र - वा॒य॒वस्य॑ । आ॒यत॑न॒ इत्या᳚-यत॑ने । स॒प्त॒मे । अहन्न्॑ । शु॒क्रः । गृ॒ह्य॒ते॒ । आ॒ग्र॒य॒णस्य॑ । वै । ए॒तत् । आ॒यत॑न॒मित्या᳚ - यत॑नम् । यत् । ष॒ष्ठम् । अहः॑ । तस्मिन्न्॑ । शु॒क्रः । गृ॒ह्य॒ते॒ । तस्मा᳚त् । शु॒क्रस्य॑ । आ॒यत॑न॒ इत्या᳚ - यत॑ने । अ॒ष्ट॒मे । अहन्न्॑ । आ॒ग्र॒य॒णः । गृ॒ह्य॒ते॒ । छन्दाꣳ॑सि । ए॒व । तत् । वीति॑ । वा॒ह॒य॒ति॒ । प्रेति॑ । वस्य॑सः । वि॒वा॒हमिति॑ वि-वा॒हम् । आ॒प्नो॒ति॒ । यः । ए॒वम् । वेद॑ । अथो॒ इति॑ । दे॒वता᳚भ्यः । ए॒व । य॒ज्ञे । सं॒ॅविद॒मिति॑ सं - विद᳚म् । द॒धा॒ति॒ । तस्मा᳚त् । इ॒दम् । अ॒न्यः । अ॒न्यस्मै᳚ ( ) । द॒दा॒ति॒ ॥  \newline


\textbf{Krama Paata} \newline

अह॒स्तस्मिन्न्॑ । तस्मि॑न्नैन्द्रवाय॒वः । ऐ॒न्द्रा॒वा॒य॒वो गृ॑ह्यते । ऐ॒न्द्रा॒वा॒य॒व इत्यै᳚न्द्र - वा॒य॒वः । गृ॒ह्य॒ते॒ तस्मा᳚त् । तस्मा॑दैन्द्रावाय॒वस्य॑ । ऐ॒न्द्रा॒वा॒य॒वस्या॒यत॑ने । ऐ॒न्द्रा॒वा॒य॒वस्येत्यै᳚न्द्र - वा॒य॒वस्य॑ । आ॒यत॑ने सप्त॒मे । आ॒यत॑न॒ इत्या᳚ - यत॑ने । स॒प्त॒मेऽहन्न्॑ । अह॑ञ्छु॒क्रः । शु॒क्रो गृ॑ह्यते । गृ॒ह्य॒त॒ आ॒ग्र॒य॒णस्य॑ । आ॒ग्र॒य॒णस्य॒ वै । वा ए॒तत् । ए॒तदा॒यत॑नम् । आ॒यत॑न॒म् ॅयत् । आ॒यत॑न॒मित्या᳚ - यत॑नम् । यथ् ष॒ष्ठम् । ष॒ष्ठमहः॑ । अह॒स्तस्मिन्न्॑ । तस्मि॑ञ्छु॒क्रः । शु॒क्रो गृ॑ह्यते । गृ॒ह्य॒ते॒ तस्मा᳚त् । तस्मा᳚च्छु॒क्रस्य॑ । शु॒क्रस्या॒यत॑ने । आ॒यत॑नेऽष्ट॒मे । आ॒यत॑न॒ इत्या᳚ - यत॑ने । अ॒ष्ट॒मेऽहन्न्॑ । अह॑न्नाग्रय॒णः । आ॒ग्र॒य॒णो गृ॑ह्यते । गृ॒ह्य॒ते॒ छन्दाꣳ॑सि । छन्दाꣳ॑स्ये॒व । ए॒व तत् । तद् वि । वि वा॑हयति । वा॒ह॒य॒ति॒ प्र । प्र वस्य॑सः । वस्य॑सो विवा॒हम् । वि॒वा॒हमा᳚प्नोति । वि॒वा॒हमिति॑ वि - वा॒हम् । आ॒प्नो॒ति॒ यः । य ए॒वम् । ए॒वम् ॅवेद॑ । वेदाथो᳚ । अथो॑ दे॒वता᳚भ्यः । अथो॒ इत्यथो᳚ । दे॒वता᳚भ्य ए॒व । ए॒व य॒ज्ञे । य॒ज्ञे स॒म्ॅविद᳚म् । स॒म्ॅविद॑म् दधाति । स॒म्ॅविद॒मिति॑ सम् - विद᳚म् । द॒धा॒ति॒ तस्मा᳚त् । तस्मा॑दि॒दम् । इ॒दम॒न्यः । अ॒न्यो᳚न्यस्मै᳚ ( ) । अ॒न्यस्मै॑ ददाति । द॒दा॒तीति॑ ददाति । \newline

\textbf{Jatai Paata} \newline

1. अह॒ स्तस्मिꣳ॒॒ स्तस्मि॒न् नह॒ रह॒ स्तस्मिन्न्॑ । \newline
2. तस्मि॑न् नैन्द्रवाय॒व ऐ᳚न्द्रवाय॒व स्तस्मिꣳ॒॒ स्तस्मि॑न् नैन्द्रवाय॒वः । \newline
3. ऐ॒न्द्र॒वा॒य॒वो गृ॑ह्यते गृह्यत ऐन्द्रवाय॒व ऐ᳚न्द्रवाय॒वो गृ॑ह्यते । \newline
4. ऐ॒न्द्र॒वा॒य॒व इत्यै᳚न्द्र - वा॒य॒वः । \newline
5. गृ॒ह्य॒ते॒ तस्मा॒त् तस्मा᳚द् गृह्यते गृह्यते॒ तस्मा᳚त् । \newline
6. तस्मा॑ दैन्द्रवाय॒व स्यै᳚न्द्रवाय॒वस्य॒ तस्मा॒त् तस्मा॑ दैन्द्रवाय॒वस्य॑ । \newline
7. ऐ॒न्द्र॒वा॒य॒वस्या॒ यत॑न आ॒यत॑न ऐन्द्रवाय॒व स्यै᳚न्द्रवाय॒वस्या॒ यत॑ने । \newline
8. ऐ॒न्द्र॒वा॒य॒वस्येत्यै᳚न्द्र - वा॒य॒वस्य॑ । \newline
9. आ॒यत॑ने सप्त॒मे स॑प्त॒म आ॒यत॑न आ॒यत॑ने सप्त॒मे । \newline
10. आ॒यत॑न॒ इत्या᳚ - यत॑ने । \newline
11. स॒प्त॒मे ऽह॒न् नहन्᳚ थ्सप्त॒मे स॑प्त॒मे ऽहन्न्॑ । \newline
12. अह॑ञ् छु॒क्रः शु॒क्रो ऽह॒न् नह॑ञ् छु॒क्रः । \newline
13. शु॒क्रो गृ॑ह्यते गृह्यते शु॒क्रः शु॒क्रो गृ॑ह्यते । \newline
14. गृ॒ह्य॒त॒ आ॒ग्र॒य॒णस्या᳚ ग्रय॒णस्य॑ गृह्यते गृह्यत आग्रय॒णस्य॑ । \newline
15. आ॒ग्र॒य॒णस्य॒ वै वा आ᳚ग्रय॒णस्या᳚ ग्रय॒णस्य॒ वै । \newline
16. वा ए॒त दे॒तद् वै वा ए॒तत् । \newline
17. ए॒त दा॒यत॑न मा॒यत॑न मे॒त दे॒त दा॒यत॑नम् । \newline
18. आ॒यत॑नं॒ ॅयद् यदा॒यत॑न मा॒यत॑नं॒ ॅयत् । \newline
19. आ॒यत॑न॒मित्या᳚ - यत॑नम् । \newline
20. यथ् ष॒ष्ठꣳ ष॒ष्ठं ॅयद् यथ् ष॒ष्ठम् । \newline
21. ष॒ष्ठ मह॒ रह॑ ष्ष॒ष्ठꣳ ष॒ष्ठ महः॑ । \newline
22. अह॒ स्तस्मिꣳ॒॒ स्तस्मि॒न् नह॒ रह॒ स्तस्मिन्न्॑ । \newline
23. तस्मि॑ञ् छु॒क्रः शु॒क्र स्तस्मिꣳ॒॒ स्तस्मि॑ञ् छु॒क्रः । \newline
24. शु॒क्रो गृ॑ह्यते गृह्यते शु॒क्रः शु॒क्रो गृ॑ह्यते । \newline
25. गृ॒ह्य॒ते॒ तस्मा॒त् तस्मा᳚द् गृह्यते गृह्यते॒ तस्मा᳚त् । \newline
26. तस्मा᳚च् छु॒क्रस्य॑ शु॒क्रस्य॒ तस्मा॒त् तस्मा᳚च् छु॒क्रस्य॑ । \newline
27. शु॒क्रस्या॒ यत॑न आ॒यत॑ने शु॒क्रस्य॑ शु॒क्रस्या॒ यत॑ने । \newline
28. आ॒यत॑ने ऽष्ट॒मे᳚ ऽष्ट॒म आ॒यत॑न आ॒यत॑ने ऽष्ट॒मे । \newline
29. आ॒यत॑न॒ इत्या᳚ - यत॑ने । \newline
30. अ॒ष्ट॒मे ऽह॒न् नह॑न् नष्ट॒मे᳚ ऽष्ट॒मे ऽहन्न्॑ । \newline
31. अह॑न् नाग्रय॒ण आ᳚ग्रय॒णो ऽह॒न् नह॑न् नाग्रय॒णः । \newline
32. आ॒ग्र॒य॒णो गृ॑ह्यते गृह्यत आग्रय॒ण आ᳚ग्रय॒णो गृ॑ह्यते । \newline
33. गृ॒ह्य॒ते॒ छन्दाꣳ॑सि॒ छन्दाꣳ॑सि गृह्यते गृह्यते॒ छन्दाꣳ॑सि । \newline
34. छन्दाꣳ॑ स्ये॒वैव छन्दाꣳ॑सि॒ छन्दाꣳ॑ स्ये॒व । \newline
35. ए॒व तत् तदे॒ वैव तत् । \newline
36. तद् वि वि तत् तद् वि । \newline
37. वि वा॑हयति वाहयति॒ वि वि वा॑हयति । \newline
38. वा॒ह॒य॒ति॒ प्र प्र वा॑हयति वाहयति॒ प्र । \newline
39. प्र वस्य॑सो॒ वस्य॑सः॒ प्र प्र वस्य॑सः । \newline
40. वस्य॑सो विवा॒हं ॅवि॑वा॒हं ॅवस्य॑सो॒ वस्य॑सो विवा॒हम् । \newline
41. वि॒वा॒ह मा᳚प्नो त्याप्नोति विवा॒हं ॅवि॑वा॒ह मा᳚प्नोति । \newline
42. वि॒वा॒हमिति॑ वि - वा॒हम् । \newline
43. आ॒प्नो॒ति॒ यो य आ᳚प्नो त्याप्नोति॒ यः । \newline
44. य ए॒व मे॒वं ॅयो य ए॒वम् । \newline
45. ए॒वं ॅवेद॒ वेदै॒व मे॒वं ॅवेद॑ । \newline
46. वेदाथो॒ अथो॒ वेद॒ वेदाथो᳚ । \newline
47. अथो॑ दे॒वता᳚भ्यो दे॒वता॒भ्यो ऽथो॒ अथो॑ दे॒वता᳚भ्यः । \newline
48. अथो॒ इत्यथो᳚ । \newline
49. दे॒वता᳚भ्य ए॒वैव दे॒वता᳚भ्यो दे॒वता᳚भ्य ए॒व । \newline
50. ए॒व य॒ज्ञे य॒ज्ञ् ए॒वैव य॒ज्ञे । \newline
51. य॒ज्ञे सं॒ॅविदꣳ॑ सं॒ॅविदं॑ ॅय॒ज्ञे य॒ज्ञे सं॒ॅविद᳚म् । \newline
52. सं॒ॅविद॑म् दधाति दधाति सं॒ॅविदꣳ॑ सं॒ॅविद॑म् दधाति । \newline
53. सं॒ॅविद॒मिति॑ सं - विद᳚म् । \newline
54. द॒धा॒ति॒ तस्मा॒त् तस्मा᳚द् दधाति दधाति॒ तस्मा᳚त् । \newline
55. तस्मा॑ दि॒द मि॒दम् तस्मा॒त् तस्मा॑ दि॒दम् । \newline
56. इ॒द म॒न्यो᳚ ऽन्य इ॒द मि॒द म॒न्यः । \newline
57. अ॒न्यो᳚ ऽन्यस्मा॑ अ॒न्यस्मा॑ अ॒न्यो᳚(1॒) ऽन्यो᳚ ऽन्यस्मै᳚ । \newline
58. अ॒न्यस्मै॑ ददाति ददा त्य॒न्यस्मा॑ अ॒न्यस्मै॑ ददाति । \newline
59. द॒दा॒तीति॑ ददाति । \newline

\textbf{Ghana Paata } \newline

1. अह॒ स्तस्मिꣳ॒॒ स्तस्मि॒न् नह॒ रह॒ स्तस्मि॑न् नैन्द्रवाय॒व ऐ᳚न्द्रवाय॒व स्तस्मि॒न् नह॒ रह॒ स्तस्मि॑न् नैन्द्रवाय॒वः । \newline
2. तस्मि॑न् नैन्द्रवाय॒व ऐ᳚न्द्रवाय॒व स्तस्मिꣳ॒॒ स्तस्मि॑न् नैन्द्रवाय॒वो गृ॑ह्यते गृह्यत ऐन्द्रवाय॒व स्तस्मिꣳ॒॒ स्तस्मि॑न् नैन्द्रवाय॒वो गृ॑ह्यते । \newline
3. ऐ॒न्द्र॒वा॒य॒वो गृ॑ह्यते गृह्यत ऐन्द्रवाय॒व ऐ᳚न्द्रवाय॒वो गृ॑ह्यते॒ तस्मा॒त् तस्मा᳚द् गृह्यत ऐन्द्रवाय॒व ऐ᳚न्द्रवाय॒वो गृ॑ह्यते॒ तस्मा᳚त् । \newline
4. ऐ॒न्द्र॒वा॒य॒व इत्यै᳚न्द्र - वा॒य॒वः । \newline
5. गृ॒ह्य॒ते॒ तस्मा॒त् तस्मा᳚द् गृह्यते गृह्यते॒ तस्मा॑ दैन्द्रवाय॒व स्यै᳚न्द्रवाय॒वस्य॒ तस्मा᳚द् गृह्यते गृह्यते॒ तस्मा॑ दैन्द्रवाय॒वस्य॑ । \newline
6. तस्मा॑ दैन्द्रवाय॒व स्यै᳚न्द्रवाय॒वस्य॒ तस्मा॒त् तस्मा॑ दैन्द्रवाय॒वस्या॒ यत॑न आ॒यत॑न ऐन्द्रवाय॒वस्य॒ तस्मा॒त् तस्मा॑ दैन्द्रवाय॒वस्या॒ यत॑ने । \newline
7. ऐ॒न्द्र॒वा॒य॒वस्या॒ यत॑न आ॒यत॑न ऐन्द्रवाय॒व स्यै᳚न्द्रवाय॒वस्या॒ यत॑ने सप्त॒मे स॑प्त॒म आ॒यत॑न ऐन्द्रवाय॒व स्यै᳚न्द्रवाय॒वस्या॒ यत॑ने सप्त॒मे । \newline
8. ऐ॒न्द्र॒वा॒य॒वस्येत्यै᳚न्द्र - वा॒य॒वस्य॑ । \newline
9. आ॒यत॑ने सप्त॒मे स॑प्त॒म आ॒यत॑न आ॒यत॑ने सप्त॒मे ऽह॒न् नहन्᳚ थ्सप्त॒म आ॒यत॑न आ॒यत॑ने सप्त॒मे ऽहन्न्॑ । \newline
10. आ॒यत॑न॒ इत्या᳚ - यत॑ने । \newline
11. स॒प्त॒मे ऽह॒न् नहन्᳚ थ्सप्त॒मे स॑प्त॒मे ऽह॑ञ् छु॒क्रः शु॒क्रो ऽहन्᳚ थ्सप्त॒मे स॑प्त॒मे ऽह॑ञ् छु॒क्रः । \newline
12. अह॑ञ् छु॒क्रः शु॒क्रो ऽह॒न् नह॑ञ् छु॒क्रो गृ॑ह्यते गृह्यते शु॒क्रो ऽह॒न् नह॑ञ् छु॒क्रो गृ॑ह्यते । \newline
13. शु॒क्रो गृ॑ह्यते गृह्यते शु॒क्रः शु॒क्रो गृ॑ह्यत आग्रय॒णस्या᳚ ग्रय॒णस्य॑ गृह्यते शु॒क्रः शु॒क्रो गृ॑ह्यत आग्रय॒णस्य॑ । \newline
14. गृ॒ह्य॒त॒ आ॒ग्र॒य॒णस्या᳚ ग्रय॒णस्य॑ गृह्यते गृह्यत आग्रय॒णस्य॒ वै वा आ᳚ग्रय॒णस्य॑ गृह्यते गृह्यत आग्रय॒णस्य॒ वै । \newline
15. आ॒ग्र॒य॒णस्य॒ वै वा आ᳚ग्रय॒णस्या᳚ ग्रय॒णस्य॒ वा ए॒त दे॒तद् वा आ᳚ग्रय॒णस्या᳚ ग्रय॒णस्य॒ वा ए॒तत् । \newline
16. वा ए॒त दे॒तद् वै वा ए॒त दा॒यत॑न मा॒यत॑न मे॒तद् वै वा ए॒त दा॒यत॑नम् । \newline
17. ए॒त दा॒यत॑न मा॒यत॑न मे॒त दे॒त दा॒यत॑नं॒ ॅयद् यदा॒यत॑न मे॒त दे॒त दा॒यत॑नं॒ ॅयत् । \newline
18. आ॒यत॑नं॒ ॅयद् यदा॒यत॑न मा॒यत॑नं॒ ॅयथ्ष॒ष्ठꣳ ष॒ष्ठं ॅयदा॒यत॑न मा॒यत॑नं॒ ॅयथ्ष॒ष्ठम् । \newline
19. आ॒यत॑न॒मित्या᳚ - यत॑नम् । \newline
20. यथ्ष॒ष्ठꣳ ष॒ष्ठं ॅयद् यथ्ष॒ष्ठ मह॒ रह॑ ष्ष॒ष्ठं ॅयद् यथ्ष॒ष्ठ महः॑ । \newline
21. ष॒ष्ठ मह॒ रह॑ ष्ष॒ष्ठꣳ ष॒ष्ठ मह॒ स्तस्मिꣳ॒॒ स्तस्मि॒न् नह॑ ष्ष॒ष्ठꣳ ष॒ष्ठ मह॒ स्तस्मिन्न्॑ । \newline
22. अह॒ स्तस्मिꣳ॒॒ स्तस्मि॒न् नह॒ रह॒ स्तस्मि॑ञ् छु॒क्रः शु॒क्र स्तस्मि॒न् नह॒ रह॒ स्तस्मि॑ञ् छु॒क्रः । \newline
23. तस्मि॑ञ् छु॒क्रः शु॒क्र स्तस्मिꣳ॒॒ स्तस्मि॑ञ् छु॒क्रो गृ॑ह्यते गृह्यते शु॒क्र स्तस्मिꣳ॒॒
स्तस्मि॑ञ् छु॒क्रो गृ॑ह्यते । \newline
24. शु॒क्रो गृ॑ह्यते गृह्यते शु॒क्रः शु॒क्रो गृ॑ह्यते॒ तस्मा॒त् तस्मा᳚द् गृह्यते शु॒क्रः शु॒क्रो गृ॑ह्यते॒ तस्मा᳚त् । \newline
25. गृ॒ह्य॒ते॒ तस्मा॒त् तस्मा᳚द् गृह्यते गृह्यते॒ तस्मा᳚च् छु॒क्रस्य॑ शु॒क्रस्य॒ तस्मा᳚द् गृह्यते गृह्यते॒ तस्मा᳚च् छु॒क्रस्य॑ । \newline
26. तस्मा᳚च् छु॒क्रस्य॑ शु॒क्रस्य॒ तस्मा॒त् तस्मा᳚ च्छु॒क्रस्या॒ यत॑न आ॒यत॑ने शु॒क्रस्य॒ तस्मा॒त् तस्मा᳚ च्छु॒क्रस्या॒ यत॑ने । \newline
27. शु॒क्रस्या॒ यत॑न आ॒यत॑ने शु॒क्रस्य॑ शु॒क्रस्या॒ यत॑ने ऽष्ट॒मे᳚ ऽष्ट॒म आ॒यत॑ने शु॒क्रस्य॑ शु॒क्रस्या॒ यत॑ने ऽष्ट॒मे । \newline
28. आ॒यत॑ने ऽष्ट॒मे᳚ ऽष्ट॒म आ॒यत॑न आ॒यत॑ने ऽष्ट॒मे ऽह॒न् नह॑न् नष्ट॒म आ॒यत॑न आ॒यत॑ने ऽष्ट॒मे ऽहन्न्॑ । \newline
29. आ॒यत॑न॒ इत्या᳚ - यत॑ने । \newline
30. अ॒ष्ट॒मे ऽह॒न् नह॑न् नष्ट॒मे᳚ ऽष्ट॒मे ऽह॑न् नाग्रय॒ण आ᳚ग्रय॒णो ऽह॑न् नष्ट॒मे᳚ ऽष्ट॒मे ऽह॑न् नाग्रय॒णः । \newline
31. अह॑न् नाग्रय॒ण आ᳚ग्रय॒णो ऽह॒न् नह॑न् नाग्रय॒णो गृ॑ह्यते गृह्यत आग्रय॒णो ऽह॒न् नह॑न् नाग्रय॒णो गृ॑ह्यते । \newline
32. आ॒ग्र॒य॒णो गृ॑ह्यते गृह्यत आग्रय॒ण आ᳚ग्रय॒णो गृ॑ह्यते॒ छन्दाꣳ॑सि॒ छन्दाꣳ॑सि गृह्यत आग्रय॒ण आ᳚ग्रय॒णो गृ॑ह्यते॒ छन्दाꣳ॑सि । \newline
33. गृ॒ह्य॒ते॒ छन्दाꣳ॑सि॒ छन्दाꣳ॑सि गृह्यते गृह्यते॒ छन्दाꣳ॑ स्ये॒वैव छन्दाꣳ॑सि गृह्यते गृह्यते॒ छन्दाꣳ॑ स्ये॒व । \newline
34. छन्दाꣳ॑ स्ये॒वैव छन्दाꣳ॑सि॒ छन्दाꣳ॑ स्ये॒व तत् तदे॒व छन्दाꣳ॑सि॒ छन्दाꣳ॑ स्ये॒व तत् । \newline
35. ए॒व तत् तदे॒ वैव तद् वि वि तदे॒ वैव तद् वि । \newline
36. तद् वि वि तत् तद् वि वा॑हयति वाहयति॒ वि तत् तद् वि वा॑हयति । \newline
37. वि वा॑हयति वाहयति॒ वि वि वा॑हयति॒ प्र प्र वा॑हयति॒ वि वि वा॑हयति॒ प्र । \newline
38. वा॒ह॒य॒ति॒ प्र प्र वा॑हयति वाहयति॒ प्र वस्य॑सो॒ वस्य॑सः॒ प्र वा॑हयति वाहयति॒ प्र वस्य॑सः । \newline
39. प्र वस्य॑सो॒ वस्य॑सः॒ प्र प्र वस्य॑सो विवा॒हं ॅवि॑वा॒हं ॅवस्य॑सः॒ प्र प्र वस्य॑सो विवा॒हम् । \newline
40. वस्य॑सो विवा॒हं ॅवि॑वा॒हं ॅवस्य॑सो॒ वस्य॑सो विवा॒ह मा᳚प्नो त्याप्नोति विवा॒हं ॅवस्य॑सो॒ वस्य॑सो विवा॒ह मा᳚प्नोति । \newline
41. वि॒वा॒ह मा᳚प्नो त्याप्नोति विवा॒हं ॅवि॑वा॒ह मा᳚प्नोति॒ यो य आ᳚प्नोति विवा॒हं ॅवि॑वा॒ह मा᳚प्नोति॒ यः । \newline
42. वि॒वा॒हमिति॑ वि - वा॒हम् । \newline
43. आ॒प्नो॒ति॒ यो य आ᳚प्नो त्याप्नोति॒ य ए॒व मे॒वं ॅय आ᳚प्नो त्याप्नोति॒ य ए॒वम् । \newline
44. य ए॒व मे॒वं ॅयो य ए॒वं ॅवेद॒ वेदै॒वं ॅयो य ए॒वं ॅवेद॑ । \newline
45. ए॒वं ॅवेद॒ वेदै॒व मे॒वं ॅवेदाथो॒ अथो॒ वेदै॒व मे॒वं ॅवेदाथो᳚ । \newline
46. वेदाथो॒ अथो॒ वेद॒ वेदाथो॑ दे॒वता᳚भ्यो दे॒वता॒भ्यो ऽथो॒ वेद॒ वेदाथो॑ दे॒वता᳚भ्यः । \newline
47. अथो॑ दे॒वता᳚भ्यो दे॒वता॒भ्यो ऽथो॒ अथो॑ दे॒वता᳚भ्य ए॒वैव दे॒वता॒भ्यो ऽथो॒ अथो॑ दे॒वता᳚भ्य ए॒व । \newline
48. अथो॒ इत्यथो᳚ । \newline
49. दे॒वता᳚भ्य ए॒वैव दे॒वता᳚भ्यो दे॒वता᳚भ्य ए॒व य॒ज्ञे य॒ज्ञ् ए॒व दे॒वता᳚भ्यो दे॒वता᳚भ्य ए॒व य॒ज्ञे । \newline
50. ए॒व य॒ज्ञे य॒ज्ञ् ए॒वैव य॒ज्ञे सं॒ॅविदꣳ॑ सं॒ॅविदं॑ ॅय॒ज्ञ् ए॒वैव य॒ज्ञे सं॒ॅविद᳚म् । \newline
51. य॒ज्ञे सं॒ॅविदꣳ॑ सं॒ॅविदं॑ ॅय॒ज्ञे य॒ज्ञे सं॒ॅविद॑म् दधाति दधाति सं॒ॅविदं॑ ॅय॒ज्ञे य॒ज्ञे सं॒ॅविद॑म् दधाति । \newline
52. सं॒ॅविद॑म् दधाति दधाति सं॒ॅविदꣳ॑ सं॒ॅविद॑म् दधाति॒ तस्मा॒त् तस्मा᳚द् दधाति सं॒ॅविदꣳ॑ सं॒ॅविद॑म् दधाति॒ तस्मा᳚त् । \newline
53. सं॒ॅविद॒मिति॑ सं - विद᳚म् । \newline
54. द॒धा॒ति॒ तस्मा॒त् तस्मा᳚द् दधाति दधाति॒ तस्मा॑ दि॒द मि॒दम् तस्मा᳚द् दधाति दधाति॒ तस्मा॑ दि॒दम् । \newline
55. तस्मा॑ दि॒द मि॒दम् तस्मा॒त् तस्मा॑ दि॒द म॒न्यो᳚ ऽन्य इ॒दम् तस्मा॒त् तस्मा॑ दि॒द म॒न्यः । \newline
56. इ॒द म॒न्यो᳚ ऽन्य इ॒द मि॒द म॒न्यो᳚ ऽन्यस्मा॑ अ॒न्यस्मा॑ अ॒न्य इ॒द मि॒द म॒न्यो᳚ ऽन्यस्मै᳚ । \newline
57. अ॒न्यो᳚ ऽन्यस्मा॑ अ॒न्यस्मा॑ अ॒न्यो᳚(1॒) ऽन्यो᳚ ऽन्यस्मै॑ ददाति ददात्य॒ न्यस्मा॑ अ॒न्यो᳚(1॒) ऽन्यो᳚ ऽन्यस्मै॑ ददाति । \newline
58. अ॒न्यस्मै॑ ददाति ददा त्य॒न्यस्मा॑ अ॒न्यस्मै॑ ददाति । \newline
59. द॒दा॒तीति॑ ददाति । \newline
\pagebreak
\markright{ TS 7.2.9.1  \hfill https://www.vedavms.in \hfill}

\section{ TS 7.2.9.1 }

\textbf{TS 7.2.9.1 } \newline
\textbf{Samhita Paata} \newline

प्र॒जाप॑तिरकामयत॒ प्र जा॑ये॒येति॒ स ए॒तं द्वा॑दशरा॒त्रम॑पश्य॒त् तमाऽह॑र॒त् तेना॑यजत॒ ततो॒ वै स प्राजा॑यत॒ यः का॒मये॑त॒ प्र जा॑ये॒येति॒ स द्वा॑दशरा॒त्रेण॑ यजेत॒ प्रैव जा॑यते ब्रह्मवा॒दिनो॑ वदन्त्यग्निष्टो॒मप्रा॑यणा य॒ज्ञा अथ॒ कस्मा॑दतिरा॒त्रः पूर्वः॒ प्र यु॑ज्यत॒ इति॒ चक्षु॑षी॒ वा ए॒ते य॒ज्ञ्स्य॒ यद॑तिरा॒त्रौ क॒नीनि॑के अग्निष्टो॒मौ य - [  ] \newline

\textbf{Pada Paata} \newline

प्र॒जाप॑ति॒रिति॑ प्र॒जा - प॒तिः॒ । अ॒का॒म॒य॒त॒ । प्रेति॑ । जा॒ये॒य॒ । इति॑ । सः । ए॒तम् । द्वा॒द॒श॒रा॒त्रमिति॑ द्वादश - रा॒त्रम् । अ॒प॒श्य॒त् । तम् । एति॑ । अ॒ह॒र॒त् । तेन॑ । अ॒य॒ज॒त॒ । ततः॑ । वै । सः । प्रेति॑ । अ॒जा॒य॒त॒ । यः । का॒मये॑त । प्रेति॑ । जा॒ये॒य॒ । इति॑ । सः । द्वा॒द॒श॒रा॒त्रेणेति॑ द्वादश - रा॒त्रेण॑ । य॒जे॒त॒ । प्रेति॑ । ए॒व । जा॒य॒ते॒ । ब्र॒ह्म॒वा॒दिन॒ इति॑ ब्रह्म-वा॒दिनः॑ । व॒द॒न्ति॒ । अ॒ग्नि॒ष्टो॒मप्रा॑यणा॒ इत्य॑ग्निष्टो॒म - प्रा॒य॒णाः॒ । य॒ज्ञाः । अथ॑ । कस्मा᳚त् । अ॒ति॒रा॒त्र इत्य॑ति - रा॒त्रः । पूर्वः॑ । प्रेति॑ । यु॒ज्य॒ते॒ । इति॑ । चक्षु॑षी॒ इति॑ । वै । ए॒ते इति॑ । य॒ज्ञ्स्य॑ । यत् । अ॒ति॒रा॒त्रावित्य॑ति - रा॒त्रौ । क॒नीनि॑के॒ इति॑ । अ॒ग्नि॒ष्टो॒मावित्य॑ग्नि- स्तो॒मौ । यत् ।  \newline


\textbf{Krama Paata} \newline

प्र॒जाप॑तिरकामयत । प्र॒जाप॑ति॒रिति॑ प्र॒जा - प॒तिः॒ । अ॒का॒म॒य॒त॒ प्र । प्र जा॑येय । जा॒ये॒येति॑ । इति॒ सः । स ए॒तम् । ए॒तम् द्वा॑दशरा॒त्रम् । द्वा॒द॒श॒रा॒त्रम॑पश्यत् । द्वा॒द॒श॒रा॒त्रमिति॑ द्वादश - रा॒त्रम् । अ॒प॒श्य॒त् तम् । तमा । आऽह॑रत् । अ॒ह॒र॒त् तेन॑ । तेना॑यजत । अ॒य॒ज॒त॒ ततः॑ । ततो॒ वै । वै सः । स प्र । प्राजा॑यत । अ॒जा॒य॒त॒ यः । यः का॒मये॑त । का॒मये॑त॒ प्र । प्र जा॑येय । जा॒ये॒येति॑ । इति॒ सः । स द्वा॑दशरा॒त्रेण॑ । द्वा॒द॒श॒रा॒त्रेण॑ यजेत । द्वा॒द॒श॒रा॒त्रेणेति॑ द्वादश - रा॒त्रेण॑ । य॒जे॒त॒ प्र । प्रैव । ए॒व जा॑यते । जा॒य॒ते॒ ब्र॒ह्म॒वा॒दिनः॑ । ब्र॒ह्म॒वा॒दिनो॑ वदन्ति । ब्र॒ह्म॒वा॒दिन॒ इति॑ ब्रह्म - वा॒दिनः॑ । व॒द॒न्त्य॒ग्नि॒ष्टो॒मप्रा॑यणाः । अ॒ग्नि॒ष्टो॒मप्रा॑यणा य॒ज्ञाः । अ॒ग्नि॒ष्टो॒मप्रा॑यणा॒ इत्य॑ग्निष्टो॒म - प्रा॒य॒णाः॒ । य॒ज्ञा अथ॑ । अथ॒ कस्मा᳚त् । कस्मा॑दतिरा॒त्रः । अ॒ति॒रा॒त्रः पूर्वः॑ । अ॒ति॒रा॒त्र इत्य॑ति - रा॒त्रः । पूर्वः॒ प्र । प्र यु॑ज्यते । यु॒ज्य॒त॒ इति॑ । इति॒ चक्षु॑षी । चक्षु॑षी॒ वै । चक्षु॑षी॒ इति॒ चक्षु॑षी । वा ए॒ते । ए॒ते य॒ज्ञ्स्य॑ । ए॒ते इत्ये॒ते । य॒ज्ञ्स्य॒ यत् । यद॑तिरा॒त्रौ । अ॒ति॒रा॒त्रौ क॒नीनि॑के । अ॒ति॒रा॒त्रावित्य॑ति - रा॒त्रौ । क॒नीनि॑के अग्निष्टो॒मौ । क॒नीनि॑के॒ इति॑ क॒नीनि॑के । अ॒ग्नि॒ष्टो॒मौ यत् । अ॒ग्नि॒ष्टो॒मावित्य॑ग्नि - स्तो॒मौ । यद॑ग्निष्टो॒मम् \newline

\textbf{Jatai Paata} \newline

1. प्र॒जाप॑ति रकामयता कामयत प्र॒जाप॑तिः प्र॒जाप॑ति रकामयत । \newline
2. प्र॒जाप॑ति॒रिति॑ प्र॒जा - प॒तिः॒ । \newline
3. अ॒का॒म॒य॒त॒ प्र प्राका॑मयता कामयत॒ प्र । \newline
4. प्र जा॑येय जायेय॒ प्र प्र जा॑येय । \newline
5. जा॒ये॒येतीति॑ जायेय जाये॒येति॑ । \newline
6. इति॒ स स इतीति॒ सः । \newline
7. स ए॒त मे॒तꣳ स स ए॒तम् । \newline
8. ए॒तम् द्वा॑दशरा॒त्रम् द्वा॑दशरा॒त्र मे॒त मे॒तम् द्वा॑दशरा॒त्रम् । \newline
9. द्वा॒द॒श॒रा॒त्र म॑पश्य दपश्यद् द्वादशरा॒त्रम् द्वा॑दशरा॒त्र म॑पश्यत् । \newline
10. द्वा॒द॒श॒रा॒त्रमिति॑ द्वादश - रा॒त्रम् । \newline
11. अ॒प॒श्य॒त् तम् त म॑पश्य दपश्य॒त् तम् । \newline
12. त मा तम् त मा । \newline
13. आ ऽह॑र दहर॒दा ऽह॑रत् । \newline
14. अ॒ह॒र॒त् तेन॒ तेना॑ हर दहर॒त् तेन॑ । \newline
15. तेना॑ यजता यजत॒ तेन॒ तेना॑ यजत । \newline
16. अ॒य॒ज॒त॒ तत॒ स्ततो॑ ऽयजता यजत॒ ततः॑ । \newline
17. ततो॒ वै वै तत॒ स्ततो॒ वै । \newline
18. वै स स वै वै सः । \newline
19. स प्र प्र स स प्र । \newline
20. प्रा जा॑यता जायत॒ प्र प्रा जा॑यत । \newline
21. अ॒जा॒य॒त॒ यो यो॑ ऽजायता जायत॒ यः । \newline
22. यः का॒मये॑त का॒मये॑त॒ यो यः का॒मये॑त । \newline
23. का॒मये॑त॒ प्र प्र का॒मये॑त का॒मये॑त॒ प्र । \newline
24. प्र जा॑येय जायेय॒ प्र प्र जा॑येय । \newline
25. जा॒ये॒येतीति॑ जायेय जाये॒येति॑ । \newline
26. इति॒ स स इतीति॒ सः । \newline
27. स द्वा॑दशरा॒त्रेण॑ द्वादशरा॒त्रेण॒ स स द्वा॑दशरा॒त्रेण॑ । \newline
28. द्वा॒द॒श॒रा॒त्रेण॑ यजेत यजेत द्वादशरा॒त्रेण॑ द्वादशरा॒त्रेण॑ यजेत । \newline
29. द्वा॒द॒श॒रा॒त्रेणेति॑ द्वादश - रा॒त्रेण॑ । \newline
30. य॒जे॒त॒ प्र प्र य॑जेत यजेत॒ प्र । \newline
31. प्रैवैव प्र प्रैव । \newline
32. ए॒व जा॑यते जायत ए॒वैव जा॑यते । \newline
33. जा॒य॒ते॒ ब्र॒ह्म॒वा॒दिनो᳚ ब्रह्मवा॒दिनो॑ जायते जायते ब्रह्मवा॒दिनः॑ । \newline
34. ब्र॒ह्म॒वा॒दिनो॑ वदन्ति वदन्ति ब्रह्मवा॒दिनो᳚ ब्रह्मवा॒दिनो॑ वदन्ति । \newline
35. ब्र॒ह्म॒वा॒दिन॒ इति॑ ब्रह्म - वा॒दिनः॑ । \newline
36. व॒द॒ न्त्य॒ग्नि॒ष्टो॒मप्रा॑यणा अग्निष्टो॒मप्रा॑यणा वदन्ति वद न्त्यग्निष्टो॒मप्रा॑यणाः । \newline
37. अ॒ग्नि॒ष्टो॒मप्रा॑यणा य॒ज्ञा य॒ज्ञा अ॑ग्निष्टो॒मप्रा॑यणा अग्निष्टो॒मप्रा॑यणा य॒ज्ञाः । \newline
38. अ॒ग्नि॒ष्टो॒मप्रा॑यणा॒ इत्य॑ग्निष्टो॒म - प्रा॒य॒णाः॒ । \newline
39. य॒ज्ञा अथाथ॑ य॒ज्ञा य॒ज्ञा अथ॑ । \newline
40. अथ॒ कस्मा॒त् कस्मा॒ दथाथ॒ कस्मा᳚त् । \newline
41. कस्मा॑ दतिरा॒त्रो॑ ऽतिरा॒त्रः कस्मा॒त् कस्मा॑ दतिरा॒त्रः । \newline
42. अ॒ति॒रा॒त्रः पूर्वः॒ पूर्वो॑ ऽतिरा॒त्रो॑ ऽतिरा॒त्रः पूर्वः॑ । \newline
43. अ॒ति॒रा॒त्र इत्य॑ति - रा॒त्रः । \newline
44. पूर्वः॒ प्र प्र पूर्वः॒ पूर्वः॒ प्र । \newline
45. प्र यु॑ज्यते युज्यते॒ प्र प्र यु॑ज्यते । \newline
46. यु॒ज्य॒त॒ इतीति॑ युज्यते युज्यत॒ इति॑ । \newline
47. इति॒ चक्षु॑षी॒ चक्षु॑षी॒ इतीति॒ चक्षु॑षी । \newline
48. चक्षु॑षी॒ वै वै चक्षु॑षी॒ चक्षु॑षी॒ वै । \newline
49. चक्षु॑षी॒ इति॒ चक्षु॑षी । \newline
50. वा ए॒ते ए॒ते वै वा ए॒ते । \newline
51. ए॒ते य॒ज्ञ्स्य॑ य॒ज्ञ्स्यै॒ते ए॒ते य॒ज्ञ्स्य॑ । \newline
52. ए॒ते इत्ये॒ते । \newline
53. य॒ज्ञ्स्य॒ यद् यद् य॒ज्ञ्स्य॑ य॒ज्ञ्स्य॒ यत् । \newline
54. यद॑तिरा॒त्रा व॑तिरा॒त्रौ यद् यद॑तिरा॒त्रौ । \newline
55. अ॒ति॒रा॒त्रौ क॒नीनि॑के क॒नीनि॑के अतिरा॒त्रा व॑तिरा॒त्रौ क॒नीनि॑के । \newline
56. अ॒ति॒रा॒त्रावित्य॑ति - रा॒त्रौ । \newline
57. क॒नीनि॑के अग्निष्टो॒मा व॑ग्निष्टो॒मौ क॒नीनि॑के क॒नीनि॑के अग्निष्टो॒मौ । \newline
58. क॒नीनि॑के॒ इति॑ क॒नीनि॑के । \newline
59. अ॒ग्नि॒ष्टो॒मौ यद् यद॑ग्निष्टो॒मा व॑ग्निष्टो॒मौ यत् । \newline
60. अ॒ग्नि॒ष्टो॒मावित्य॑ग्नि - स्तो॒मौ । \newline
61. यद॑ग्निष्टो॒म म॑ग्निष्टो॒मं ॅयद् यद॑ग्निष्टो॒मम् । \newline

\textbf{Ghana Paata } \newline

1. प्र॒जाप॑ति रकामयता कामयत प्र॒जाप॑तिः प्र॒जाप॑ति रकामयत॒ प्र प्राका॑मयत प्र॒जाप॑तिः प्र॒जाप॑ति रकामयत॒ प्र । \newline
2. प्र॒जाप॑ति॒रिति॑ प्र॒जा - प॒तिः॒ । \newline
3. अ॒का॒म॒य॒त॒ प्र प्रा का॑मयता कामयत॒ प्र जा॑येय जायेय॒ प्रा का॑मयता कामयत॒ प्र जा॑येय । \newline
4. प्र जा॑येय जायेय॒ प्र प्र जा॑ये॒येतीति॑ जायेय॒ प्र प्र जा॑ये॒येति॑ । \newline
5. जा॒ये॒येतीति॑ जायेय जाये॒येति॒ स स इति॑ जायेय जाये॒येति॒ सः । \newline
6. इति॒ स स इतीति॒ स ए॒त मे॒तꣳ स इतीति॒ स ए॒तम् । \newline
7. स ए॒त मे॒तꣳ स स ए॒तम् द्वा॑दशरा॒त्रम् द्वा॑दशरा॒त्र मे॒तꣳ स स ए॒तम् द्वा॑दशरा॒त्रम् । \newline
8. ए॒तम् द्वा॑दशरा॒त्रम् द्वा॑दशरा॒त्र मे॒त मे॒तम् द्वा॑दशरा॒त्र म॑पश्य दपश्यद् द्वादशरा॒त्र मे॒त मे॒तम् द्वा॑दशरा॒त्र म॑पश्यत् । \newline
9. द्वा॒द॒श॒रा॒त्र म॑पश्य दपश्यद् द्वादशरा॒त्रम् द्वा॑दशरा॒त्र म॑पश्य॒त् तम् त म॑पश्यद् द्वादशरा॒त्रम् द्वा॑दशरा॒त्र म॑पश्य॒त् तम् । \newline
10. द्वा॒द॒श॒रा॒त्रमिति॑ द्वादश - रा॒त्रम् । \newline
11. अ॒प॒श्य॒त् तम् त म॑पश्य दपश्य॒त् त मा त म॑पश्य दपश्य॒त् त मा । \newline
12. त मा तम् त मा ऽह॑र दहर॒दा तम् त मा ऽह॑रत् । \newline
13. आ ऽह॑र दहर॒दा ऽह॑र॒त् तेन॒ तेना॑ हर॒दा ऽह॑र॒त् तेन॑ । \newline
14. अ॒ह॒र॒त् तेन॒ तेना॑ हर दहर॒त् तेना॑ यजता यजत॒ तेना॑ हर दहर॒त् तेना॑यजत । \newline
15. तेना॑ यजता यजत॒ तेन॒ तेना॑ यजत॒ तत॒ स्ततो॑ ऽयजत॒ तेन॒ तेना॑ यजत॒ ततः॑ । \newline
16. अ॒य॒ज॒त॒ तत॒ स्ततो॑ ऽयजता यजत॒ ततो॒ वै वै ततो॑ ऽयजता यजत॒ ततो॒ वै । \newline
17. ततो॒ वै वै तत॒ स्ततो॒ वै स स वै तत॒ स्ततो॒ वै सः । \newline
18. वै स स वै वै स प्र प्र स वै वै स प्र । \newline
19. स प्र प्र स स प्रा जा॑यता जायत॒ प्र स स प्रा जा॑यत । \newline
20. प्रा जा॑यता जायत॒ प्र प्रा जा॑यत॒ यो यो॑ ऽजायत॒ प्र प्रा जा॑यत॒ यः । \newline
21. अ॒जा॒य॒त॒ यो यो॑ ऽजायता जायत॒ यः का॒मये॑त का॒मये॑त॒ यो॑ ऽजायता जायत॒ यः का॒मये॑त । \newline
22. यः का॒मये॑त का॒मये॑त॒ यो यः का॒मये॑त॒ प्र प्र का॒मये॑त॒ यो यः का॒मये॑त॒ प्र । \newline
23. का॒मये॑त॒ प्र प्र का॒मये॑त का॒मये॑त॒ प्र जा॑येय जायेय॒ प्र का॒मये॑त का॒मये॑त॒ प्र जा॑येय । \newline
24. प्र जा॑येय जायेय॒ प्र प्र जा॑ये॒येतीति॑ जायेय॒ प्र प्र जा॑ये॒येति॑ । \newline
25. जा॒ये॒येतीति॑ जायेय जाये॒येति॒ स स इति॑ जायेय जाये॒येति॒ सः । \newline
26. इति॒ स स इतीति॒ स द्वा॑दशरा॒त्रेण॑ द्वादशरा॒त्रेण॒ स इतीति॒ स द्वा॑दशरा॒त्रेण॑ । \newline
27. स द्वा॑दशरा॒त्रेण॑ द्वादशरा॒त्रेण॒ स स द्वा॑दशरा॒त्रेण॑ यजेत यजेत द्वादशरा॒त्रेण॒ स स द्वा॑दशरा॒त्रेण॑ यजेत । \newline
28. द्वा॒द॒श॒रा॒त्रेण॑ यजेत यजेत द्वादशरा॒त्रेण॑ द्वादशरा॒त्रेण॑ यजेत॒ प्र प्र य॑जेत द्वादशरा॒त्रेण॑ द्वादशरा॒त्रेण॑ यजेत॒ प्र । \newline
29. द्वा॒द॒श॒रा॒त्रेणेति॑ द्वादश - रा॒त्रेण॑ । \newline
30. य॒जे॒त॒ प्र प्र य॑जेत यजेत॒ प्रैवैव प्र य॑जेत यजेत॒ प्रैव । \newline
31. प्रैवैव प्र प्रैव जा॑यते जायत ए॒व प्र प्रैव जा॑यते । \newline
32. ए॒व जा॑यते जायत ए॒वैव जा॑यते ब्रह्मवा॒दिनो᳚ ब्रह्मवा॒दिनो॑ जायत ए॒वैव जा॑यते ब्रह्मवा॒दिनः॑ । \newline
33. जा॒य॒ते॒ ब्र॒ह्म॒वा॒दिनो᳚ ब्रह्मवा॒दिनो॑ जायते जायते ब्रह्मवा॒दिनो॑ वदन्ति वदन्ति ब्रह्मवा॒दिनो॑ जायते जायते ब्रह्मवा॒दिनो॑ वदन्ति । \newline
34. ब्र॒ह्म॒वा॒दिनो॑ वदन्ति वदन्ति ब्रह्मवा॒दिनो᳚ ब्रह्मवा॒दिनो॑ वद न्त्यग्निष्टो॒मप्रा॑यणा अग्निष्टो॒मप्रा॑यणा वदन्ति ब्रह्मवा॒दिनो᳚ ब्रह्मवा॒दिनो॑ वद न्त्यग्निष्टो॒मप्रा॑यणाः । \newline
35. ब्र॒ह्म॒वा॒दिन॒ इति॑ ब्रह्म - वा॒दिनः॑ । \newline
36. व॒द॒ न्त्य॒ग्नि॒ष्टो॒मप्रा॑यणा अग्निष्टो॒मप्रा॑यणा वदन्ति वद न्त्यग्निष्टो॒मप्रा॑यणा य॒ज्ञा य॒ज्ञा अ॑ग्निष्टो॒मप्रा॑यणा वदन्ति वद न्त्यग्निष्टो॒मप्रा॑यणा य॒ज्ञाः । \newline
37. अ॒ग्नि॒ष्टो॒मप्रा॑यणा य॒ज्ञा य॒ज्ञा अ॑ग्निष्टो॒मप्रा॑यणा अग्निष्टो॒मप्रा॑यणा य॒ज्ञा अथाथ॑ य॒ज्ञा अ॑ग्निष्टो॒मप्रा॑यणा अग्निष्टो॒मप्रा॑यणा य॒ज्ञा अथ॑ । \newline
38. अ॒ग्नि॒ष्टो॒मप्रा॑यणा॒ इत्य॑ग्निष्टो॒म - प्रा॒य॒णाः॒ । \newline
39. य॒ज्ञा अथाथ॑ य॒ज्ञा य॒ज्ञा अथ॒ कस्मा॒त् कस्मा॒ दथ॑ य॒ज्ञा य॒ज्ञा अथ॒ कस्मा᳚त् । \newline
40. अथ॒ कस्मा॒त् कस्मा॒ दथाथ॒ कस्मा॑ दतिरा॒त्रो॑ ऽतिरा॒त्रः कस्मा॒ दथाथ॒ कस्मा॑ दतिरा॒त्रः । \newline
41. कस्मा॑ दतिरा॒त्रो॑ ऽतिरा॒त्रः कस्मा॒त् कस्मा॑ दतिरा॒त्रः पूर्वः॒ पूर्वो॑ ऽतिरा॒त्रः कस्मा॒त् कस्मा॑ दतिरा॒त्रः पूर्वः॑ । \newline
42. अ॒ति॒रा॒त्रः पूर्वः॒ पूर्वो॑ ऽतिरा॒त्रो॑ ऽतिरा॒त्रः पूर्वः॒ प्र प्र पूर्वो॑ ऽतिरा॒त्रो॑ ऽतिरा॒त्रः पूर्वः॒ प्र । \newline
43. अ॒ति॒रा॒त्र इत्य॑ति - रा॒त्रः । \newline
44. पूर्वः॒ प्र प्र पूर्वः॒ पूर्वः॒ प्र यु॑ज्यते युज्यते॒ प्र पूर्वः॒ पूर्वः॒ प्र यु॑ज्यते । \newline
45. प्र यु॑ज्यते युज्यते॒ प्र प्र यु॑ज्यत॒ इतीति॑ युज्यते॒ प्र प्र यु॑ज्यत॒ इति॑ । \newline
46. यु॒ज्य॒त॒ इतीति॑ युज्यते युज्यत॒ इति॒ चक्षु॑षी॒ चक्षु॑षी॒ इति॑ युज्यते युज्यत॒ इति॒ चक्षु॑षी । \newline
47. इति॒ चक्षु॑षी॒ चक्षु॑षी॒ इतीति॒ चक्षु॑षी॒ वै वै चक्षु॑षी॒ इतीति॒ चक्षु॑षी॒ वै । \newline
48. चक्षु॑षी॒ वै वै चक्षु॑षी॒ चक्षु॑षी॒ वा ए॒ते ए॒ते वै चक्षु॑षी॒ चक्षु॑षी॒ वा ए॒ते । \newline
49. चक्षु॑षी॒ इति॒ चक्षु॑षी । \newline
50. वा ए॒ते ए॒ते वै वा ए॒ते य॒ज्ञ्स्य॑ य॒ज्ञ्स्यै॒ते वै वा ए॒ते य॒ज्ञ्स्य॑ । \newline
51. ए॒ते य॒ज्ञ्स्य॑ य॒ज्ञ्स्यै॒ते ए॒ते य॒ज्ञ्स्य॒ यद् यद् य॒ज्ञ्स्यै॒ते ए॒ते य॒ज्ञ्स्य॒ यत् । \newline
52. ए॒ते इत्ये॒ते । \newline
53. य॒ज्ञ्स्य॒ यद् यद् य॒ज्ञ्स्य॑ य॒ज्ञ्स्य॒ यद॑तिरा॒त्रा व॑तिरा॒त्रौ यद् य॒ज्ञ्स्य॑ य॒ज्ञ्स्य॒ यद॑तिरा॒त्रौ । \newline
54. यद॑तिरा॒त्रा व॑तिरा॒त्रौ यद् यद॑तिरा॒त्रौ क॒नीनि॑के क॒नीनि॑के अतिरा॒त्रौ यद् यद॑तिरा॒त्रौ क॒नीनि॑के । \newline
55. अ॒ति॒रा॒त्रौ क॒नीनि॑के क॒नीनि॑के अतिरा॒त्रा व॑तिरा॒त्रौ क॒नीनि॑के अग्निष्टो॒मा व॑ग्निष्टो॒मौ क॒नीनि॑के अतिरा॒त्रा व॑तिरा॒त्रौ क॒नीनि॑के अग्निष्टो॒मौ । \newline
56. अ॒ति॒रा॒त्रावित्य॑ति - रा॒त्रौ । \newline
57. क॒नीनि॑के अग्निष्टो॒मा व॑ग्निष्टो॒मौ क॒नीनि॑के क॒नीनि॑के अग्निष्टो॒मौ यद् यद॑ग्निष्टो॒मौ क॒नीनि॑के क॒नीनि॑के अग्निष्टो॒मौ यत् । \newline
58. क॒नीनि॑के॒ इति॑ क॒नीनि॑के । \newline
59. अ॒ग्नि॒ष्टो॒मौ यद् यद॑ग्निष्टो॒मा व॑ग्निष्टो॒मौ यद॑ग्निष्टो॒म म॑ग्निष्टो॒मं ॅयद॑ग्निष्टो॒मा व॑ग्निष्टो॒मौ यद॑ग्निष्टो॒मम् । \newline
60. अ॒ग्नि॒ष्टो॒मावित्य॑ग्नि - स्तो॒मौ । \newline
61. यद॑ग्निष्टो॒म म॑ग्निष्टो॒मं ॅयद् यद॑ग्निष्टो॒मम् पूर्व॒म् पूर्व॑ मग्निष्टो॒मं ॅयद् यद॑ग्निष्टो॒मम् पूर्व᳚म् । \newline
\pagebreak
\markright{ TS 7.2.9.2  \hfill https://www.vedavms.in \hfill}

\section{ TS 7.2.9.2 }

\textbf{TS 7.2.9.2 } \newline
\textbf{Samhita Paata} \newline

-द॑ग्निष्टो॒मं पूर्वं॑ प्रयुञ्जी॒रन् ब॑हि॒र्द्धा क॒नीनि॑के दद्ध्यु॒स्तस्मा॑दतिरा॒त्रः पूर्वः॒ प्र यु॑ज्यते॒ चक्षु॑षी ए॒व य॒ज्ञे धि॒त्वा म॑द्ध्य॒तः क॒नीनि॑के॒ प्रति॑ दधति॒ यो वै गा॑य॒त्रीं ज्योतिः॑पक्षां॒ ॅवेद॒ ज्योति॑षा भा॒सा सु॑व॒र्गं ॅलो॒कमे॑ति॒ याव॑ग्निष्टो॒मौ तौ प॒क्षौ येऽन्त॑रे॒ऽष्टावु॒क्थ्याः᳚ स आ॒त्मैषा वै गा॑य॒त्री ज्योतिः॑पक्षा॒ य ए॒वं ॅवेद॒ ज्योति॑षा भा॒सा सु॑व॒र्गं ॅलो॒क - [  ] \newline

\textbf{Pada Paata} \newline

अ॒ग्नि॒ष्टो॒ममित्य॑ग्नि-स्तो॒मम् । पूर्व᳚म् । प्र॒यु॒ञ्जी॒रन्निति॑ प्र-यु॒ञ्जी॒रन्न् । ब॒हि॒द्‌र्धेति॑ बहिः - धा । क॒नीनि॑के॒ इति॑ । द॒द्ध्युः॒ । तस्मा᳚त् । अ॒ति॒रा॒त्र इत्य॑ति - रा॒त्रः । पूर्वः॑ । प्रेति॑ । यु॒ज्य॒ते॒ । चक्षु॑षी॒ इति॑ । ए॒व । य॒ज्ञे । धि॒त्वा । म॒द्ध्य॒तः । क॒नीनि॑के॒ इति॑ । प्रतीति॑ । द॒ध॒ति॒ । यः । वै । गा॒य॒त्रीम् । ज्योतिः॑पक्षा॒मिति॒ ज्योतिः॑ - प॒क्षा॒म् । वेद॑ । ज्योति॑षा । भा॒सा । सु॒व॒र्गमिति॑ सुवः - गम् । लो॒कम् । ए॒ति॒ । यौ । अ॒ग्नि॒ष्टो॒मावित्य॑ग्नि - स्तो॒मौ । तौ । प॒क्षौ । ये । अन्त॑रे । अ॒ष्टौ । उ॒क्थ्याः᳚ । सः । आ॒त्मा । ए॒षा । वै । गा॒य॒त्री । ज्योतिः॑प॒क्षेति॒ ज्योतिः॑ - प॒क्षा॒ । यः । ए॒वम् । वेद॑ । ज्योति॑षा । भा॒सा । सु॒व॒र्गमिति॑ सुवः - गम् । लो॒कम् ।  \newline


\textbf{Krama Paata} \newline

अ॒ग्नि॒ष्टो॒मम् पूर्व᳚म् । अ॒ग्नि॒ष्टो॒ममित्य॑ग्नि - स्तो॒मम् । पूर्व॑म् प्रयुञ्जी॒रन्न् । प्र॒यु॒ञ्जी॒रन् ब॑हि॒र्द्धा । प्र॒यु॒ञ्जी॒रन्निति॑ प्र - यु॒ञ्जी॒रन्न् । ब॒हि॒र्द्धा क॒नीनि॑के । ब॒हि॒र्द्धेति॑ बहिः - धा । क॒नीनि॑के दद्ध्युः । क॒नीनि॑के॒ इति॑ क॒नीनि॑के । द॒द्ध्यु॒स्तस्मा᳚त् । तस्मा॑दतिरा॒त्रः । अ॒ति॒रा॒त्रः पूर्वः॑ । अ॒ति॒रा॒त्र इत्य॑ति - रा॒त्रः । पूर्वः॒ प्र । 
प्र यु॑ज्यते । यु॒ज्य॒ते॒ चक्षु॑षी । चक्षु॑षी ए॒व । चक्षु॑षी॒ इति॒ चक्षु॑षी । ए॒व य॒ज्ञे । य॒ज्ञे धि॒त्वा । धि॒त्वा म॑द्ध्य॒तः । म॒द्ध्य॒तः क॒नीनि॑के । क॒नीनि॑के॒ प्रति॑ । क॒नीनि॑के॒ इति॑ क॒नीनि॑के । प्रति॑ दधति । द॒ध॒ति॒ यः । यो वै । वै गा॑य॒त्रीम् । गा॒य॒त्रीम् ज्योतिः॑पक्षाम् । ज्योतिः॑पक्षा॒म् ॅवेद॑ । ज्योतिः॑पक्षा॒मिति॒ ज्योतिः॑ - प॒क्षा॒म् । वेद॒ ज्योति॑षा । ज्योति॑षा भा॒सा । भा॒सा सु॑व॒र्गम् । सु॒व॒र्गम् ॅलो॒कम् । सु॒व॒र्गमिति॑ सुवः - गम् । लो॒कमे॑ति । ए॒ति॒ यौ । याव॑ग्निष्टो॒मौ । अ॒ग्नि॒ष्टो॒मौ तौ । अ॒ग्नि॒ष्टो॒मावित्य॑ग्नि - स्तो॒मौ । तौ प॒क्षौ । प॒क्षौ ये । येऽन्त॑रे । अन्त॑रे॒ऽष्टौ । अ॒ष्टावु॒क्थ्याः᳚ । उ॒क्थ्याः᳚ सः । स आ॒त्मा । आ॒त्मैषा । ए॒षा वै । वै गा॑य॒त्री । गा॒य॒त्री ज्योतिः॑पक्षा । ज्योतिः॑पक्षा॒ यः । ज्योतिः॑प॒क्षेति॒ ज्योतिः॑ - प॒क्षा॒ । य ए॒वम् । ए॒व वेद॑ । वेद॒ ज्योति॑षा । ज्योति॑षा भा॒सा । भा॒सा सु॑व॒र्गम् । सु॒व॒र्गम् ॅलो॒कम् । सु॒व॒र्गमिति॑ सुवः - गम् । लो॒कमे॑ति \newline

\textbf{Jatai Paata} \newline

1. अ॒ग्नि॒ष्टो॒मम् पूर्व॒म् पूर्व॑ मग्निष्टो॒म म॑ग्निष्टो॒मम् पूर्व᳚म् । \newline
2. अ॒ग्नि॒ष्टो॒ममित्य॑ग्नि - स्तो॒मम् । \newline
3. पूर्व॑म् प्रयुञ्जी॒रन् प्र॑युञ्जी॒रन् पूर्व॒म् पूर्व॑म् प्रयुञ्जी॒रन्न् । \newline
4. प्र॒यु॒ञ्जी॒रन् ब॑हि॒र्द्धा ब॑हि॒र्द्धा प्र॑युञ्जी॒रन् प्र॑युञ्जी॒रन् ब॑हि॒र्द्धा । \newline
5. प्र॒यु॒ञ्जी॒रन्निति॑ प्र - यु॒ञ्जी॒रन्न् । \newline
6. ब॒हि॒र्द्धा क॒नीनि॑के क॒नीनि॑के बहि॒र्द्धा ब॑हि॒र्द्धा क॒नीनि॑के । \newline
7. ब॒हि॒र्द्धेति॑ बहिः - धा । \newline
8. क॒नीनि॑के दद्ध्युर् दद्ध्युः क॒नीनि॑के क॒नीनि॑के दद्ध्युः । \newline
9. क॒नीनि॑के॒ इति॑ क॒नीनि॑के । \newline
10. द॒द्ध्यु॒ स्तस्मा॒त् तस्मा᳚द् दद्ध्युर् दद्ध्यु॒ स्तस्मा᳚त् । \newline
11. तस्मा॑ दतिरा॒त्रो॑ ऽतिरा॒त्र स्तस्मा॒त् तस्मा॑ दतिरा॒त्रः । \newline
12. अ॒ति॒रा॒त्रः पूर्वः॒ पूर्वो॑ ऽतिरा॒त्रो॑ ऽतिरा॒त्रः पूर्वः॑ । \newline
13. अ॒ति॒रा॒त्र इत्य॑ति - रा॒त्रः । \newline
14. पूर्वः॒ प्र प्र पूर्वः॒ पूर्वः॒ प्र । \newline
15. प्र यु॑ज्यते युज्यते॒ प्र प्र यु॑ज्यते । \newline
16. यु॒ज्य॒ते॒ चक्षु॑षी॒ चक्षु॑षी युज्यते युज्यते॒ चक्षु॑षी । \newline
17. चक्षु॑षी ए॒वैव चक्षु॑षी॒ चक्षु॑षी ए॒व । \newline
18. चक्षु॑षी॒ इति॒ चक्षु॑षी । \newline
19. ए॒व य॒ज्ञे य॒ज्ञ् ए॒वैव य॒ज्ञे । \newline
20. य॒ज्ञे धि॒त्वा धि॒त्वा य॒ज्ञे य॒ज्ञे धि॒त्वा । \newline
21. धि॒त्वा म॑द्ध्य॒तो म॑द्ध्य॒तो धि॒त्वा धि॒त्वा म॑द्ध्य॒तः । \newline
22. म॒द्ध्य॒तः क॒नीनि॑के क॒नीनि॑के मद्ध्य॒तो म॑द्ध्य॒तः क॒नीनि॑के । \newline
23. क॒नीनि॑के॒ प्रति॒ प्रति॑ क॒नीनि॑के क॒नीनि॑के॒ प्रति॑ । \newline
24. क॒नीनि॑के॒ इति॑ क॒नीनि॑के । \newline
25. प्रति॑ दधति दधति॒ प्रति॒ प्रति॑ दधति । \newline
26. द॒ध॒ति॒ यो यो द॑धति दधति॒ यः । \newline
27. यो वै वै यो यो वै । \newline
28. वै गा॑य॒त्रीम् गा॑य॒त्रीं ॅवै वै गा॑य॒त्रीम् । \newline
29. गा॒य॒त्रीम् ज्योतिः॑पक्षा॒म् ज्योतिः॑पक्षाम् गाय॒त्रीम् गा॑य॒त्रीम् ज्योतिः॑पक्षाम् । \newline
30. ज्योतिः॑पक्षां॒ ॅवेद॒ वेद॒ ज्योतिः॑पक्षा॒म् ज्योतिः॑पक्षां॒ ॅवेद॑ । \newline
31. ज्योतिः॑पक्षा॒मिति॒ ज्योतिः॑ - प॒क्षा॒म् । \newline
32. वेद॒ ज्योति॑षा॒ ज्योति॑षा॒ वेद॒ वेद॒ ज्योति॑षा । \newline
33. ज्योति॑षा भा॒सा भा॒सा ज्योति॑षा॒ ज्योति॑षा भा॒सा । \newline
34. भा॒सा सु॑व॒र्गꣳ सु॑व॒र्गम् भा॒सा भा॒सा सु॑व॒र्गम् । \newline
35. सु॒व॒र्गम् ॅलो॒कम् ॅलो॒कꣳ सु॑व॒र्गꣳ सु॑व॒र्गम् ॅलो॒कम् । \newline
36. सु॒व॒र्गमिति॑ सुवः - गम् । \newline
37. लो॒क मे᳚त्येति लो॒कम् ॅलो॒क मे॑ति । \newline
38. ए॒ति॒ यौ या वे᳚त्येति॒ यौ । \newline
39. या व॑ग्निष्टो॒मा व॑ग्निष्टो॒मौ यौ या व॑ग्निष्टो॒मौ । \newline
40. अ॒ग्नि॒ष्टो॒मौ तौ ता व॑ग्निष्टो॒मा व॑ग्निष्टो॒मौ तौ । \newline
41. अ॒ग्नि॒ष्टो॒मावित्य॑ग्नि - स्तो॒मौ । \newline
42. तौ प॒क्षौ प॒क्षौ तौ तौ प॒क्षौ । \newline
43. प॒क्षौ ये ये प॒क्षौ प॒क्षौ ये । \newline
44. ये ऽन्त॒रे ऽन्त॑रे॒ ये ये ऽन्त॑रे । \newline
45. अन्त॑रे॒ ऽष्टा व॒ष्टा वन्त॒रे ऽन्त॑रे॒ ऽष्टौ । \newline
46. अ॒ष्टा वु॒क्थ्या॑ उ॒क्थ्या॑ अ॒ष्टा व॒ष्टा वु॒क्थ्याः᳚ । \newline
47. उ॒क्थ्याः᳚ स स उ॒क्थ्या॑ उ॒क्थ्याः᳚ सः । \newline
48. स आ॒त्मा ऽऽत्मा स स आ॒त्मा । \newline
49. आ॒त्मै षैषा ऽऽत्मा ऽऽत्मैषा । \newline
50. ए॒षा वै वा ए॒षैषा वै । \newline
51. वै गा॑य॒त्री गा॑य॒त्री वै वै गा॑य॒त्री । \newline
52. गा॒य॒त्री ज्योतिः॑पक्षा॒ ज्योतिः॑पक्षा गाय॒त्री गा॑य॒त्री ज्योतिः॑पक्षा । \newline
53. ज्योतिः॑पक्षा॒ यो यो ज्योतिः॑पक्षा॒ ज्योतिः॑पक्षा॒ यः । \newline
54. ज्योतिः॑प॒क्षेति॒ ज्योतिः॑ - प॒क्षा॒ । \newline
55. य ए॒व मे॒वं ॅयो य ए॒वम् । \newline
56. ए॒वं ॅवेद॒ वेदै॒व मे॒वं ॅवेद॑ । \newline
57. वेद॒ ज्योति॑षा॒ ज्योति॑षा॒ वेद॒ वेद॒ ज्योति॑षा । \newline
58. ज्योति॑षा भा॒सा भा॒सा ज्योति॑षा॒ ज्योति॑षा भा॒सा । \newline
59. भा॒सा सु॑व॒र्गꣳ सु॑व॒र्गम् भा॒सा भा॒सा सु॑व॒र्गम् । \newline
60. सु॒व॒र्गम् ॅलो॒कम् ॅलो॒कꣳ सु॑व॒र्गꣳ सु॑व॒र्गम् ॅलो॒कम् । \newline
61. सु॒व॒र्गमिति॑ सुवः - गम् । \newline
62. लो॒क मे᳚त्येति लो॒कम् ॅलो॒क मे॑ति । \newline

\textbf{Ghana Paata } \newline

1. अ॒ग्नि॒ष्टो॒मम् पूर्व॒म् पूर्व॑ मग्निष्टो॒म म॑ग्निष्टो॒मम् पूर्व॑म् प्रयुञ्जी॒रन् प्र॑युञ्जी॒रन् पूर्व॑ मग्निष्टो॒म म॑ग्निष्टो॒मम् पूर्व॑म् प्रयुञ्जी॒रन्न् । \newline
2. अ॒ग्नि॒ष्टो॒ममित्य॑ग्नि - स्तो॒मम् । \newline
3. पूर्व॑म् प्रयुञ्जी॒रन् प्र॑युञ्जी॒रन् पूर्व॒म् पूर्व॑म् प्रयुञ्जी॒रन् ब॑हि॒र्द्धा ब॑हि॒र्द्धा प्र॑युञ्जी॒रन् पूर्व॒म् पूर्व॑म् प्रयुञ्जी॒रन् ब॑हि॒र्द्धा । \newline
4. प्र॒यु॒ञ्जी॒रन् ब॑हि॒र्द्धा ब॑हि॒र्द्धा प्र॑युञ्जी॒रन् प्र॑युञ्जी॒रन् ब॑हि॒र्द्धा क॒नीनि॑के क॒नीनि॑के बहि॒र्द्धा प्र॑युञ्जी॒रन् प्र॑युञ्जी॒रन् ब॑हि॒र्द्धा क॒नीनि॑के । \newline
5. प्र॒यु॒ञ्जी॒रन्निति॑ प्र - यु॒ञ्जी॒रन्न् । \newline
6. ब॒हि॒र्द्धा क॒नीनि॑के क॒नीनि॑के बहि॒र्द्धा ब॑हि॒र्द्धा क॒नीनि॑के दद्ध्युर् दद्ध्युः क॒नीनि॑के बहि॒र्द्धा ब॑हि॒र्द्धा क॒नीनि॑के दद्ध्युः । \newline
7. ब॒हि॒र्द्धेति॑ बहिः - धा । \newline
8. क॒नीनि॑के दद्ध्युर् दद्ध्युः क॒नीनि॑के क॒नीनि॑के दद्ध्यु॒ स्तस्मा॒त् तस्मा᳚द् दद्ध्युः क॒नीनि॑के क॒नीनि॑के दद्ध्यु॒ स्तस्मा᳚त् । \newline
9. क॒नीनि॑के॒ इति॑ क॒नीनि॑के । \newline
10. द॒द्ध्यु॒ स्तस्मा॒त् तस्मा᳚द् दद्ध्युर् दद्ध्यु॒ स्तस्मा॑ दतिरा॒त्रो॑ ऽतिरा॒त्र स्तस्मा᳚द् दद्ध्युर् दद्ध्यु॒ स्तस्मा॑ दतिरा॒त्रः । \newline
11. तस्मा॑ दतिरा॒त्रो॑ ऽतिरा॒त्र स्तस्मा॒त् तस्मा॑ दतिरा॒त्रः पूर्वः॒ पूर्वो॑ ऽतिरा॒त्र स्तस्मा॒त् तस्मा॑ दतिरा॒त्रः पूर्वः॑ । \newline
12. अ॒ति॒रा॒त्रः पूर्वः॒ पूर्वो॑ ऽतिरा॒त्रो॑ ऽतिरा॒त्रः पूर्वः॒ प्र प्र पूर्वो॑ ऽतिरा॒त्रो॑ ऽतिरा॒त्रः पूर्वः॒ प्र । \newline
13. अ॒ति॒रा॒त्र इत्य॑ति - रा॒त्रः । \newline
14. पूर्वः॒ प्र प्र पूर्वः॒ पूर्वः॒ प्र यु॑ज्यते युज्यते॒ प्र पूर्वः॒ पूर्वः॒ प्र यु॑ज्यते । \newline
15. प्र यु॑ज्यते युज्यते॒ प्र प्र यु॑ज्यते॒ चक्षु॑षी॒ चक्षु॑षी युज्यते॒ प्र प्र यु॑ज्यते॒ चक्षु॑षी । \newline
16. यु॒ज्य॒ते॒ चक्षु॑षी॒ चक्षु॑षी युज्यते युज्यते॒ चक्षु॑षी ए॒वैव चक्षु॑षी युज्यते युज्यते॒ चक्षु॑षी ए॒व । \newline
17. चक्षु॑षी ए॒वैव चक्षु॑षी॒ चक्षु॑षी ए॒व य॒ज्ञे य॒ज्ञ् ए॒व चक्षु॑षी॒ चक्षु॑षी ए॒व य॒ज्ञे । \newline
18. चक्षु॑षी॒ इति॒ चक्षु॑षी । \newline
19. ए॒व य॒ज्ञे य॒ज्ञ् ए॒वैव य॒ज्ञे धि॒त्वा धि॒त्वा य॒ज्ञ् ए॒वैव य॒ज्ञे धि॒त्वा । \newline
20. य॒ज्ञे धि॒त्वा धि॒त्वा य॒ज्ञे य॒ज्ञे धि॒त्वा म॑द्ध्य॒तो म॑द्ध्य॒तो धि॒त्वा य॒ज्ञे य॒ज्ञे धि॒त्वा म॑द्ध्य॒तः । \newline
21. धि॒त्वा म॑द्ध्य॒तो म॑द्ध्य॒तो धि॒त्वा धि॒त्वा म॑द्ध्य॒तः क॒नीनि॑के क॒नीनि॑के मद्ध्य॒तो धि॒त्वा धि॒त्वा म॑द्ध्य॒तः क॒नीनि॑के । \newline
22. म॒द्ध्य॒तः क॒नीनि॑के क॒नीनि॑के मद्ध्य॒तो म॑द्ध्य॒तः क॒नीनि॑के॒ प्रति॒ प्रति॑ क॒नीनि॑के मद्ध्य॒तो म॑द्ध्य॒तः क॒नीनि॑के॒ प्रति॑ । \newline
23. क॒नीनि॑के॒ प्रति॒ प्रति॑ क॒नीनि॑के क॒नीनि॑के॒ प्रति॑ दधति दधति॒ प्रति॑ क॒नीनि॑के क॒नीनि॑के॒ प्रति॑ दधति । \newline
24. क॒नीनि॑के॒ इति॑ क॒नीनि॑के । \newline
25. प्रति॑ दधति दधति॒ प्रति॒ प्रति॑ दधति॒ यो यो द॑धति॒ प्रति॒ प्रति॑ दधति॒ यः । \newline
26. द॒ध॒ति॒ यो यो द॑धति दधति॒ यो वै वै यो द॑धति दधति॒ यो वै । \newline
27. यो वै वै यो यो वै गा॑य॒त्रीम् गा॑य॒त्रीं ॅवै यो यो वै गा॑य॒त्रीम् । \newline
28. वै गा॑य॒त्रीम् गा॑य॒त्रीं ॅवै वै गा॑य॒त्रीम् ज्योतिः॑पक्षा॒म् ज्योतिः॑पक्षाम् गाय॒त्रीं ॅवै वै गा॑य॒त्रीम् ज्योतिः॑पक्षाम् । \newline
29. गा॒य॒त्रीम् ज्योतिः॑पक्षा॒म् ज्योतिः॑पक्षाम् गाय॒त्रीम् गा॑य॒त्रीम् ज्योतिः॑पक्षां॒ ॅवेद॒ वेद॒ ज्योतिः॑पक्षाम् गाय॒त्रीम् गा॑य॒त्रीम् ज्योतिः॑पक्षां॒ ॅवेद॑ । \newline
30. ज्योतिः॑पक्षां॒ ॅवेद॒ वेद॒ ज्योतिः॑पक्षा॒म् ज्योतिः॑पक्षां॒ ॅवेद॒ ज्योति॑षा॒ ज्योति॑षा॒ वेद॒ ज्योतिः॑पक्षा॒म् ज्योतिः॑पक्षां॒ ॅवेद॒ ज्योति॑षा । \newline
31. ज्योतिः॑पक्षा॒मिति॒ ज्योतिः॑ - प॒क्षा॒म् । \newline
32. वेद॒ ज्योति॑षा॒ ज्योति॑षा॒ वेद॒ वेद॒ ज्योति॑षा भा॒सा भा॒सा ज्योति॑षा॒ वेद॒ वेद॒ ज्योति॑षा भा॒सा । \newline
33. ज्योति॑षा भा॒सा भा॒सा ज्योति॑षा॒ ज्योति॑षा भा॒सा सु॑व॒र्गꣳ सु॑व॒र्गम् भा॒सा ज्योति॑षा॒ ज्योति॑षा भा॒सा सु॑व॒र्गम् । \newline
34. भा॒सा सु॑व॒र्गꣳ सु॑व॒र्गम् भा॒सा भा॒सा सु॑व॒र्गम् ॅलो॒कम् ॅलो॒कꣳ सु॑व॒र्गम् भा॒सा भा॒सा सु॑व॒र्गम् ॅलो॒कम् । \newline
35. सु॒व॒र्गम् ॅलो॒कम् ॅलो॒कꣳ सु॑व॒र्गꣳ सु॑व॒र्गम् ॅलो॒क मे᳚त्येति लो॒कꣳ सु॑व॒र्गꣳ सु॑व॒र्गम् ॅलो॒क मे॑ति । \newline
36. सु॒व॒र्गमिति॑ सुवः - गम् । \newline
37. लो॒क मे᳚त्येति लो॒कम् ॅलो॒क मे॑ति॒ यौ या वे॑ति लो॒कम् ॅलो॒क मे॑ति॒ यौ । \newline
38. ए॒ति॒ यौ या वे᳚त्येति॒ या व॑ग्निष्टो॒मा व॑ग्निष्टो॒मौ या वे᳚त्येति॒ या व॑ग्निष्टो॒मौ । \newline
39. या व॑ग्निष्टो॒मा व॑ग्निष्टो॒मौ यौ या व॑ग्निष्टो॒मौ तौ ता व॑ग्निष्टो॒मौ यौ या व॑ग्निष्टो॒मौ तौ । \newline
40. अ॒ग्नि॒ष्टो॒मौ तौ ता व॑ग्निष्टो॒मा व॑ग्निष्टो॒मौ तौ प॒क्षौ प॒क्षौ ता व॑ग्निष्टो॒मा व॑ग्निष्टो॒मौ तौ प॒क्षौ । \newline
41. अ॒ग्नि॒ष्टो॒मावित्य॑ग्नि - स्तो॒मौ । \newline
42. तौ प॒क्षौ प॒क्षौ तौ तौ प॒क्षौ ये ये प॒क्षौ तौ तौ प॒क्षौ ये । \newline
43. प॒क्षौ ये ये प॒क्षौ प॒क्षौ ये ऽन्त॒रे ऽन्त॑रे॒ ये प॒क्षौ प॒क्षौ ये ऽन्त॑रे । \newline
44. ये ऽन्त॒रे ऽन्त॑रे॒ ये ये ऽन्त॑रे॒ ऽष्टा व॒ष्टा वन्त॑रे॒ ये ये ऽन्त॑रे॒ ऽष्टौ । \newline
45. अन्त॑रे॒ ऽष्टा व॒ष्टा वन्त॒रे ऽन्त॑रे॒ ऽष्टा वु॒क्थ्या॑ उ॒क्थ्या॑ अ॒ष्टा वन्त॒रे ऽन्त॑रे॒ ऽष्टा वु॒क्थ्याः᳚ । \newline
46. अ॒ष्टा वु॒क्थ्या॑ उ॒क्थ्या॑ अ॒ष्टा व॒ष्टा वु॒क्थ्याः᳚ स स उ॒क्थ्या॑ अ॒ष्टा व॒ष्टा वु॒क्थ्याः᳚ सः । \newline
47. उ॒क्थ्याः᳚ स स उ॒क्थ्या॑ उ॒क्थ्याः᳚ स आ॒त्मा ऽऽत्मा स उ॒क्थ्या॑ उ॒क्थ्याः᳚ स आ॒त्मा । \newline
48. स आ॒त्मा ऽऽत्मा स स आ॒त्मैषैषा ऽऽत्मा स स आ॒त्मैषा । \newline
49. आ॒त्मैषैषा ऽऽत्मा ऽऽत्मैषा वै वा ए॒षा ऽऽत्मा ऽऽत्मैषा वै । \newline
50. ए॒षा वै वा ए॒षैषा वै गा॑य॒त्री गा॑य॒त्री वा ए॒षैषा वै गा॑य॒त्री । \newline
51. वै गा॑य॒त्री गा॑य॒त्री वै वै गा॑य॒त्री ज्योतिः॑पक्षा॒ ज्योतिः॑पक्षा गाय॒त्री वै वै गा॑य॒त्री ज्योतिः॑पक्षा । \newline
52. गा॒य॒त्री ज्योतिः॑पक्षा॒ ज्योतिः॑पक्षा गाय॒त्री गा॑य॒त्री ज्योतिः॑पक्षा॒ यो यो ज्योतिः॑पक्षा गाय॒त्री गा॑य॒त्री ज्योतिः॑पक्षा यः । \newline
53. ज्योतिः॑पक्षा॒ यो यो ज्योतिः॑पक्षा॒ ज्योतिः॑पक्षा॒ य ए॒व मे॒वं ॅयो ज्योतिः॑पक्षा॒ ज्योतिः॑पक्षा॒ य ए॒वम् । \newline
54. ज्योतिः॑प॒क्षेति॒ ज्योतिः॑ - प॒क्षा॒ । \newline
55. य ए॒व मे॒वं ॅयो य ए॒वं ॅवेद॒ वेदै॒वं ॅयो य ए॒वं ॅवेद॑ । \newline
56. ए॒वं ॅवेद॒ वेदै॒व मे॒वं ॅवेद॒ ज्योति॑षा॒ ज्योति॑षा॒ वेदै॒व मे॒वं ॅवेद॒ ज्योति॑षा । \newline
57. वेद॒ ज्योति॑षा॒ ज्योति॑षा॒ वेद॒ वेद॒ ज्योति॑षा भा॒सा भा॒सा ज्योति॑षा॒ वेद॒ वेद॒ ज्योति॑षा भा॒सा । \newline
58. ज्योति॑षा भा॒सा भा॒सा ज्योति॑षा॒ ज्योति॑षा भा॒सा सु॑व॒र्गꣳ सु॑व॒र्गम् भा॒सा ज्योति॑षा॒ ज्योति॑षा भा॒सा सु॑व॒र्गम् । \newline
59. भा॒सा सु॑व॒र्गꣳ सु॑व॒र्गम् भा॒सा भा॒सा सु॑व॒र्गम् ॅलो॒कम् ॅलो॒कꣳ सु॑व॒र्गम् भा॒सा भा॒सा सु॑व॒र्गम् ॅलो॒कम् । \newline
60. सु॒व॒र्गम् ॅलो॒कम् ॅलो॒कꣳ सु॑व॒र्गꣳ सु॑व॒र्गम् ॅलो॒क मे᳚त्येति लो॒कꣳ सु॑व॒र्गꣳ सु॑व॒र्गम् ॅलो॒क मे॑ति । \newline
61. सु॒व॒र्गमिति॑ सुवः - गम् । \newline
62. लो॒क मे᳚त्येति लो॒कम् ॅलो॒क मे॑ति प्र॒जाप॑तिः प्र॒जाप॑ति रेति लो॒कम् ॅलो॒क मे॑ति प्र॒जाप॑तिः । \newline
\pagebreak
\markright{ TS 7.2.9.3  \hfill https://www.vedavms.in \hfill}

\section{ TS 7.2.9.3 }

\textbf{TS 7.2.9.3 } \newline
\textbf{Samhita Paata} \newline

-मे॑ति प्र॒जाप॑ति॒र्वा ए॒ष द्वा॑दश॒धा विहि॑तो॒ यद् द्वा॑दशरा॒त्रो याव॑तिरा॒त्रो तौ प॒क्षौ येऽन्त॑रे॒ऽष्टावु॒क्थ्याः᳚ स आ॒त्मा प्र॒जाप॑ति॒र्वावैष सन्थ्सद्ध॒ वै स॒त्रेण॑ स्पृणोति प्रा॒णा वै सत् प्रा॒णाने॒व स्पृ॑णोति॒ सर्वा॑सां॒ ॅवा ए॒ते प्र॒जानां᳚ प्रा॒णैरा॑सते॒ ये स॒त्रमास॑ते॒ तस्मा᳚त् पृच्छन्ति॒ किमे॒ते स॒त्रिण॒ इति॑ प्रि॒यः प्र॒जाना॒ ( ) मुत्थि॑तो भवति॒ य ए॒वं ॅवेद॑ ॥ \newline

\textbf{Pada Paata} \newline

ए॒ति॒ । प्र॒जाप॑ति॒रिति॑ प्र॒जा - प॒तिः॒ । वै । ए॒षः । द्वा॒द॒श॒धेति॑ द्वादश - धा । विहि॑त॒ इति॒ वि - हि॒तः॒ । यत् । द्वा॒द॒श॒रा॒त्र इति॑ द्वादश - रा॒त्रः । यौ । अ॒ति॒रा॒त्रावित्य॑ति - रा॒त्रौ । तौ । प॒क्षौ । ये । अन्त॑रे । अ॒ष्टौ । उ॒क्थ्याः᳚ । सः । आ॒त्मा । प्र॒जाप॑ति॒रिति॑ प्र॒जा-प॒तिः॒ । वाव । ए॒षः । सन्न् । सत् । ह॒ । वै । स॒त्रेण॑ । स्पृ॒णो॒ति॒ । प्रा॒णा इति॑ प्र - अ॒नाः । वै । सत् । प्रा॒णानिति॑ प्र - अ॒नान् । ए॒व । स्पृ॒णो॒ति॒ । सर्वा॑साम् । वै । ए॒ते । प्र॒जाना॒मिति॑ प्र - जाना᳚म् । प्रा॒णैरिति॑ प्र - अ॒नैः । आ॒स॒ते॒ । ये । स॒त्रम् । आस॑ते । तस्मा᳚त् । पृ॒च्छ॒न्ति॒ । किम् । ए॒ते । स॒त्रिणः॑ । इति॑ । प्रि॒यः । प्र॒जाना॒मिति॑ प्र-जाना᳚म् ( ) । उत्थि॑त॒ इत्युत् - स्थि॒तः॒ । भ॒व॒ति॒ । यः । ए॒वम् । वेद॑ ॥  \newline


\textbf{Krama Paata} \newline

ए॒ति॒ प्र॒जाप॑तिः । प्र॒जाप॑ति॒र् वै । प्र॒जाप॑ति॒रिति॑ प्र॒जा - प॒तिः॒ । वा ए॒षः । ए॒ष द्वा॑दश॒धा । द्वा॒द॒श॒धा विहि॑तः । द्वा॒द॒श॒धेति॑ द्वादश - धा । विहि॑तो॒ यत् । विहि॑त॒ इति॒ वि - हि॒तः॒ । यद् द्वा॑दशरा॒त्रः । द्वा॒द॒श॒रा॒त्रो यौ । द्वा॒द॒श॒रा॒त्र इति॑ द्वादश - रा॒त्रः । याव॑तिरा॒त्रौ । अ॒ति॒रा॒त्रौ तौ । अ॒ति॒रा॒त्रावित्य॑ति - रा॒त्रौ । तौ प॒क्षौ । प॒क्षौ ये । येऽन्त॑रे । अन्त॑रे॒ऽष्टौ । अ॒ष्टावु॒क्थ्याः᳚ । उ॒क्थ्याः᳚ सः । स आ॒त्मा । आ॒त्मा प्र॒जाप॑तिः । प्र॒जाप॑ति॒र् वाव । प्र॒जाप॑ति॒रिति॑ प्र॒जा - प॒तिः॒ । वावैषः । ए॒ष सन्न् । सन्थ् सत् । सद्‌ध॑ । ह॒ वै । वै स॒त्रेण॑ । स॒त्रेण॑ स्पृणोति । स्पृ॒णो॒ति॒ प्रा॒णाः । प्रा॒णा वै । प्रा॒णा इति॑ प्र - अ॒नाः । वै सत् । सत् प्रा॒णान् । प्रा॒णाने॒व । प्रा॒णानिति॑ प्र - अ॒नान् । ए॒व स्पृ॑णोति । स्पृ॒णो॒ति॒ सर्वा॑साम् । सर्वा॑सा॒म् ॅवै । वा ए॒ते । ए॒ते प्र॒जाना᳚म् । प्र॒जाना᳚म् प्रा॒णैः । प्र॒जाना॒मिति॑ प्र - जाना᳚म् । प्रा॒णैरा॑सते । प्रा॒णैरिति॑ प्र - अ॒नैः । आ॒स॒ते॒ ये । ये स॒त्रम् । स॒त्रमास॑ते । आस॑ते॒ तस्मा᳚त् । तस्मा᳚त् पृच्छन्ति । पृ॒च्छ॒न्ति॒ किम् । किमे॒ते । ए॒ते स॒त्रिणः॑ । स॒त्रिण॒ इति॑ । इति॑ प्रि॒यः । प्रि॒यः प्र॒जाना᳚म् । प्र॒जाना॒मुत्थि॑तः ( ) । प्र॒जाना॒मिति॑ प्र - जाना᳚म् । उत्थि॑तो भवति । उत्थि॑त॒ इत्युत् - स्थि॒तः॒ । भ॒व॒ति॒ यः । य ए॒वम् । ए॒वम् ॅवेद॑ । वेदेति॒ वेद॑ । \newline

\textbf{Jatai Paata} \newline

1. ए॒ति॒ प्र॒जाप॑तिः प्र॒जाप॑ति रेत्येति प्र॒जाप॑तिः । \newline
2. प्र॒जाप॑ति॒र् वै वै प्र॒जाप॑तिः प्र॒जाप॑ति॒र् वै । \newline
3. प्र॒जाप॑ति॒रिति॑ प्र॒जा - प॒तिः॒ । \newline
4. वा ए॒ष ए॒ष वै वा ए॒षः । \newline
5. ए॒ष द्वा॑दश॒धा द्वा॑दश॒ धैष ए॒ष द्वा॑दश॒धा । \newline
6. द्वा॒द॒श॒धा विहि॑तो॒ विहि॑तो द्वादश॒धा द्वा॑दश॒धा विहि॑तः । \newline
7. द्वा॒द॒श॒धेति॑ द्वादश - धा । \newline
8. विहि॑तो॒ यद् यद् विहि॑तो॒ विहि॑तो॒ यत् । \newline
9. विहि॑त॒ इति॒ वि - हि॒तः॒ । \newline
10. यद् द्वा॑दशरा॒त्रो द्वा॑दशरा॒त्रो यद् यद् द्वा॑दशरा॒त्रः । \newline
11. द्वा॒द॒श॒रा॒त्रो यौ यौ द्वा॑दशरा॒त्रो द्वा॑दशरा॒त्रो यौ । \newline
12. द्वा॒द॒श॒रा॒त्र इति॑ द्वादश - रा॒त्रः । \newline
13. या व॑तिरा॒त्रा व॑तिरा॒त्रौ यौ या व॑तिरा॒त्रौ । \newline
14. अ॒ति॒रा॒त्रौ तौ ता व॑तिरा॒त्रा व॑तिरा॒त्रौ तौ । \newline
15. अ॒ति॒रा॒त्रावित्य॑ति - रा॒त्रौ । \newline
16. तौ प॒क्षौ प॒क्षौ तौ तौ प॒क्षौ । \newline
17. प॒क्षौ ये ये प॒क्षौ प॒क्षौ ये । \newline
18. ये ऽन्त॒रे ऽन्त॑रे॒ ये ये ऽन्त॑रे । \newline
19. अन्त॑रे॒ ऽष्टा व॒ष्टा वन्त॒रे ऽन्त॑रे॒ ऽष्टौ । \newline
20. अ॒ष्टा वु॒क्थ्या॑ उ॒क्थ्या॑ अ॒ष्टा व॒ष्टा वु॒क्थ्याः᳚ । \newline
21. उ॒क्थ्याः᳚ स स उ॒क्थ्या॑ उ॒क्थ्याः᳚ सः । \newline
22. स आ॒त्मा ऽऽत्मा स स आ॒त्मा । \newline
23. आ॒त्मा प्र॒जाप॑तिः प्र॒जाप॑ति रा॒त्मा ऽऽत्मा प्र॒जाप॑तिः । \newline
24. प्र॒जाप॑ति॒र् वाव वाव प्र॒जाप॑तिः प्र॒जाप॑ति॒र् वाव । \newline
25. प्र॒जाप॑ति॒रिति॑ प्र॒जा - प॒तिः॒ । \newline
26. वावैष ए॒ष वाव वावैषः । \newline
27. ए॒ष सन् थ्सन् ने॒ष ए॒ष सन्न् । \newline
28. सन् थ्सथ् सथ् सन् थ्सन् थ्सत् । \newline
29. सद्ध॑ ह॒ सथ् सद्ध॑ । \newline
30. ह॒ वै वै ह॑ ह॒ वै । \newline
31. वै स॒त्रेण॑ स॒त्रेण॒ वै वै स॒त्रेण॑ । \newline
32. स॒त्रेण॑ स्पृणोति स्पृणोति स॒त्रेण॑ स॒त्रेण॑ स्पृणोति । \newline
33. स्पृ॒णो॒ति॒ प्रा॒णाः प्रा॒णाः स्पृ॑णोति स्पृणोति प्रा॒णाः । \newline
34. प्रा॒णा वै वै प्रा॒णाः प्रा॒णा वै । \newline
35. प्रा॒णा इति॑ प्र - अ॒नाः । \newline
36. वै सथ् सद् वै वै सत् । \newline
37. सत् प्रा॒णान् प्रा॒णान् थ्सथ् सत् प्रा॒णान् । \newline
38. प्रा॒णा ने॒वैव प्रा॒णान् प्रा॒णा ने॒व । \newline
39. प्रा॒णानिति॑ प्र - अ॒नान् । \newline
40. ए॒व स्पृ॑णोति स्पृणो त्ये॒वैव स्पृ॑णोति । \newline
41. स्पृ॒णो॒ति॒ सर्वा॑साꣳ॒॒ सर्वा॑साꣳ स्पृणोति स्पृणोति॒ सर्वा॑साम् । \newline
42. सर्वा॑सां॒ ॅवै वै सर्वा॑साꣳ॒॒ सर्वा॑सां॒ ॅवै । \newline
43. वा ए॒त ए॒ते वै वा ए॒ते । \newline
44. ए॒ते प्र॒जाना᳚म् प्र॒जाना॑ मे॒त ए॒ते प्र॒जाना᳚म् । \newline
45. प्र॒जाना᳚म् प्रा॒णैः प्रा॒णैः प्र॒जाना᳚म् प्र॒जाना᳚म् प्रा॒णैः । \newline
46. प्र॒जाना॒मिति॑ प्र - जाना᳚म् । \newline
47. प्रा॒णै रा॑सत आसते प्रा॒णैः प्रा॒णै रा॑सते । \newline
48. प्रा॒णैरिति॑ प्र - अ॒नैः । \newline
49. आ॒स॒ते॒ ये य आ॑सत आसते॒ ये । \newline
50. ये स॒त्रꣳ स॒त्रं ॅये ये स॒त्रम् । \newline
51. स॒त्र मास॑त॒ आस॑ते स॒त्रꣳ स॒त्र मास॑ते । \newline
52. आस॑ते॒ तस्मा॒त् तस्मा॒ दास॑त॒ आस॑ते॒ तस्मा᳚त् । \newline
53. तस्मा᳚त् पृच्छन्ति पृच्छन्ति॒ तस्मा॒त् तस्मा᳚त् पृच्छन्ति । \newline
54. पृ॒च्छ॒न्ति॒ किम् किम् पृ॑च्छन्ति पृच्छन्ति॒ किम् । \newline
55. किमे॒त ए॒ते किम् किमे॒ते । \newline
56. ए॒ते स॒त्रिणः॑ स॒त्रिण॑ ए॒त ए॒ते स॒त्रिणः॑ । \newline
57. स॒त्रिण॒ इतीति॑ स॒त्रिणः॑ स॒त्रिण॒ इति॑ । \newline
58. इति॑ प्रि॒यः प्रि॒य इतीति॑ प्रि॒यः । \newline
59. प्रि॒यः प्र॒जाना᳚म् प्र॒जाना᳚म् प्रि॒यः प्रि॒यः प्र॒जाना᳚म् । \newline
60. प्र॒जाना॒ मुत्थि॑त॒ उत्थि॑तः प्र॒जाना᳚म् प्र॒जाना॒ मुत्थि॑तः । \newline
61. प्र॒जाना॒मिति॑ प्र - जाना᳚म् । \newline
62. उत्थि॑तो भवति भव॒ त्युत्थि॑त॒ उत्थि॑तो भवति । \newline
63. उत्थि॑त॒ इत्युत् - स्थि॒तः॒ । \newline
64. भ॒व॒ति॒ यो यो भ॑वति भवति॒ यः । \newline
65. य ए॒व मे॒वं ॅयो य ए॒वम् । \newline
66. ए॒वं ॅवेद॒ वेदै॒व मे॒वं ॅवेद॑ । \newline
67. वेदेति॒ वेद॑ । \newline

\textbf{Ghana Paata } \newline

1. ए॒ति॒ प्र॒जाप॑तिः प्र॒जाप॑ति रेत्येति प्र॒जाप॑ति॒र् वै वै प्र॒जाप॑ति रेत्येति प्र॒जाप॑ति॒र् वै । \newline
2. प्र॒जाप॑ति॒र् वै वै प्र॒जाप॑तिः प्र॒जाप॑ति॒र् वा ए॒ष ए॒ष वै प्र॒जाप॑तिः प्र॒जाप॑ति॒र् वा ए॒षः । \newline
3. प्र॒जाप॑ति॒रिति॑ प्र॒जा - प॒तिः॒ । \newline
4. वा ए॒ष ए॒ष वै वा ए॒ष द्वा॑दश॒धा द्वा॑दश॒धैष वै वा ए॒ष द्वा॑दश॒धा । \newline
5. ए॒ष द्वा॑दश॒धा द्वा॑दश॒ धैष ए॒ष द्वा॑दश॒धा विहि॑तो॒ विहि॑तो द्वादश॒ धैष ए॒ष द्वा॑दश॒धा विहि॑तः । \newline
6. द्वा॒द॒श॒धा विहि॑तो॒ विहि॑तो द्वादश॒धा द्वा॑दश॒धा विहि॑तो॒ यद् यद् विहि॑तो द्वादश॒धा द्वा॑दश॒धा विहि॑तो॒ यत् । \newline
7. द्वा॒द॒श॒धेति॑ द्वादश - धा । \newline
8. विहि॑तो॒ यद् यद् विहि॑तो॒ विहि॑तो॒ यद् द्वा॑दशरा॒त्रो द्वा॑दशरा॒त्रो यद् विहि॑तो॒ विहि॑तो॒ यद् द्वा॑दशरा॒त्रः । \newline
9. विहि॑त॒ इति॒ वि - हि॒तः॒ । \newline
10. यद् द्वा॑दशरा॒त्रो द्वा॑दशरा॒त्रो यद् यद् द्वा॑दशरा॒त्रो यौ यौ द्वा॑दशरा॒त्रो यद् यद् द्वा॑दशरा॒त्रो यौ । \newline
11. द्वा॒द॒श॒रा॒त्रो यौ यौ द्वा॑दशरा॒त्रो द्वा॑दशरा॒त्रो या व॑तिरा॒त्रा व॑तिरा॒त्रौ यौ द्वा॑दशरा॒त्रो द्वा॑दशरा॒त्रो या व॑तिरा॒त्रौ । \newline
12. द्वा॒द॒श॒रा॒त्र इति॑ द्वादश - रा॒त्रः । \newline
13. या व॑तिरा॒त्रा व॑तिरा॒त्रौ यौ या व॑तिरा॒त्रौ तौ ता व॑तिरा॒त्रौ यौ या व॑तिरा॒त्रौ तौ । \newline
14. अ॒ति॒रा॒त्रौ तौ ता व॑तिरा॒त्रा व॑तिरा॒त्रौ तौ प॒क्षौ प॒क्षौ ता व॑तिरा॒त्रा व॑तिरा॒त्रौ तौ प॒क्षौ । \newline
15. अ॒ति॒रा॒त्रावित्य॑ति - रा॒त्रौ । \newline
16. तौ प॒क्षौ प॒क्षौ तौ तौ प॒क्षौ ये ये प॒क्षौ तौ तौ प॒क्षौ ये । \newline
17. प॒क्षौ ये ये प॒क्षौ प॒क्षौ ये ऽन्त॒रे ऽन्त॑रे॒ ये प॒क्षौ प॒क्षौ ये ऽन्त॑रे । \newline
18. ये ऽन्त॒रे ऽन्त॑रे॒ ये ये ऽन्त॑रे॒ ऽष्टा व॒ष्टा वन्त॑रे॒ ये ये ऽन्त॑रे॒ ऽष्टौ । \newline
19. अन्त॑रे॒ ऽष्टा व॒ष्टा वन्त॒रे ऽन्त॑रे॒ ऽष्टा वु॒क्थ्या॑ उ॒क्थ्या॑ अ॒ष्टा वन्त॒रे ऽन्त॑रे॒ ऽष्टा वु॒क्थ्याः᳚ । \newline
20. अ॒ष्टा वु॒क्थ्या॑ उ॒क्थ्या॑ अ॒ष्टा व॒ष्टा वु॒क्थ्याः᳚ स स उ॒क्थ्या॑ अ॒ष्टा व॒ष्टा वु॒क्थ्याः᳚ सः । \newline
21. उ॒क्थ्याः᳚ स स उ॒क्थ्या॑ उ॒क्थ्याः᳚ स आ॒त्मा ऽऽत्मा स उ॒क्थ्या॑ उ॒क्थ्याः᳚ स आ॒त्मा । \newline
22. स आ॒त्मा ऽऽत्मा स स आ॒त्मा प्र॒जाप॑तिः प्र॒जाप॑ति रा॒त्मा स स आ॒त्मा प्र॒जाप॑तिः । \newline
23. आ॒त्मा प्र॒जाप॑तिः प्र॒जाप॑ति रा॒त्मा ऽऽत्मा प्र॒जाप॑ति॒र् वाव वाव प्र॒जाप॑ति रा॒त्मा ऽऽत्मा प्र॒जाप॑ति॒र् वाव । \newline
24. प्र॒जाप॑ति॒र् वाव वाव प्र॒जाप॑तिः प्र॒जाप॑ति॒र् वावैष ए॒ष वाव प्र॒जाप॑तिः प्र॒जाप॑ति॒र् वावैषः । \newline
25. प्र॒जाप॑ति॒रिति॑ प्र॒जा - प॒तिः॒ । \newline
26. वावैष ए॒ष वाव वावैष सन् थ्सन् ने॒ष वाव वावैष सन्न् । \newline
27. ए॒ष सन् थ्सन् ने॒ष ए॒ष सन् थ्सथ् सथ् सन् ने॒ष ए॒ष सन् थ्सत् । \newline
28. सन् थ्सथ् सथ् सन् थ्सन् थ्सद्ध॑ ह॒ सथ् सन् थ्सन् थ्सद्ध॑ । \newline
29. सद्ध॑ ह॒ सथ् सद्ध॒ वै वै ह॒ सथ् सद्ध॒ वै । \newline
30. ह॒ वै वै ह॑ ह॒ वै स॒त्रेण॑ स॒त्रेण॒ वै ह॑ ह॒ वै स॒त्रेण॑ । \newline
31. वै स॒त्रेण॑ स॒त्रेण॒ वै वै स॒त्रेण॑ स्पृणोति स्पृणोति स॒त्रेण॒ वै वै स॒त्रेण॑ स्पृणोति । \newline
32. स॒त्रेण॑ स्पृणोति स्पृणोति स॒त्रेण॑ स॒त्रेण॑ स्पृणोति प्रा॒णाः प्रा॒णाः स्पृ॑णोति स॒त्रेण॑ स॒त्रेण॑ स्पृणोति प्रा॒णाः । \newline
33. स्पृ॒णो॒ति॒ प्रा॒णाः प्रा॒णाः स्पृ॑णोति स्पृणोति प्रा॒णा वै वै प्रा॒णाः स्पृ॑णोति स्पृणोति प्रा॒णा वै । \newline
34. प्रा॒णा वै वै प्रा॒णाः प्रा॒णा वै सथ् सद् वै प्रा॒णाः प्रा॒णा वै सत् । \newline
35. प्रा॒णा इति॑ प्र - अ॒नाः । \newline
36. वै सथ् सद् वै वै सत् प्रा॒णान् प्रा॒णान् थ्सद् वै वै सत् प्रा॒णान् । \newline
37. सत् प्रा॒णान् प्रा॒णान् थ्सथ् सत् प्रा॒णाने॒वैव प्रा॒णान् थ्सथ् सत् प्रा॒णाने॒व । \newline
38. प्रा॒णाने॒वैव प्रा॒णान् प्रा॒णाने॒व स्पृ॑णोति स्पृणोत्ये॒व प्रा॒णान् प्रा॒णाने॒व स्पृ॑णोति । \newline
39. प्रा॒णानिति॑ प्र - अ॒नान् । \newline
40. ए॒व स्पृ॑णोति स्पृणो त्ये॒वैव स्पृ॑णोति॒ सर्वा॑साꣳ॒॒ सर्वा॑साꣳ स्पृणो त्ये॒वैव स्पृ॑णोति॒ सर्वा॑साम् । \newline
41. स्पृ॒णो॒ति॒ सर्वा॑साꣳ॒॒ सर्वा॑साꣳ स्पृणोति स्पृणोति॒ सर्वा॑सां॒ ॅवै वै सर्वा॑साꣳ स्पृणोति स्पृणोति॒ सर्वा॑सां॒ ॅवै । \newline
42. सर्वा॑सां॒ ॅवै वै सर्वा॑साꣳ॒॒ सर्वा॑सां॒ ॅवा ए॒त ए॒ते वै सर्वा॑साꣳ॒॒ सर्वा॑सां॒ ॅवा ए॒ते । \newline
43. वा ए॒त ए॒ते वै वा ए॒ते प्र॒जाना᳚म् प्र॒जाना॑ मे॒ते वै वा ए॒ते प्र॒जाना᳚म् । \newline
44. ए॒ते प्र॒जाना᳚म् प्र॒जाना॑ मे॒त ए॒ते प्र॒जाना᳚म् प्रा॒णैः प्रा॒णैः प्र॒जाना॑ मे॒त ए॒ते प्र॒जाना᳚म् प्रा॒णैः । \newline
45. प्र॒जाना᳚म् प्रा॒णैः प्रा॒णैः प्र॒जाना᳚म् प्र॒जाना᳚म् प्रा॒णै रा॑सत आसते प्रा॒णैः प्र॒जाना᳚म् प्र॒जाना᳚म् प्रा॒णै रा॑सते । \newline
46. प्र॒जाना॒मिति॑ प्र - जाना᳚म् । \newline
47. प्रा॒णै रा॑सत आसते प्रा॒णैः प्रा॒णै रा॑सते॒ ये य आ॑सते प्रा॒णैः प्रा॒णै रा॑सते॒ ये । \newline
48. प्रा॒णैरिति॑ प्र - अ॒नैः । \newline
49. आ॒स॒ते॒ ये य आ॑सत आसते॒ ये स॒त्रꣳ स॒त्रं ॅय आ॑सत आसते॒ ये स॒त्रम् । \newline
50. ये स॒त्रꣳ स॒त्रं ॅये ये स॒त्र मास॑त॒ आस॑ते स॒त्रं ॅये ये स॒त्र मास॑ते । \newline
51. स॒त्र मास॑त॒ आस॑ते स॒त्रꣳ स॒त्र मास॑ते॒ तस्मा॒त् तस्मा॒ दास॑ते स॒त्रꣳ स॒त्र मास॑ते॒ तस्मा᳚त् । \newline
52. आस॑ते॒ तस्मा॒त् तस्मा॒ दास॑त॒ आस॑ते॒ तस्मा᳚त् पृच्छन्ति पृच्छन्ति॒ तस्मा॒ दास॑त॒ आस॑ते॒ तस्मा᳚त् पृच्छन्ति । \newline
53. तस्मा᳚त् पृच्छन्ति पृच्छन्ति॒ तस्मा॒त् तस्मा᳚त् पृच्छन्ति॒ किम् किम् पृ॑च्छन्ति॒ तस्मा॒त् तस्मा᳚त् पृच्छन्ति॒ किम् । \newline
54. पृ॒च्छ॒न्ति॒ किम् किम् पृ॑च्छन्ति पृच्छन्ति॒ कि मे॒त ए॒ते किम् पृ॑च्छन्ति पृच्छन्ति॒ कि मे॒ते । \newline
55. कि मे॒त ए॒ते किम् कि मे॒ते स॒त्रिणः॑ स॒त्रिण॑ ए॒ते किम् कि मे॒ते स॒त्रिणः॑ । \newline
56. ए॒ते स॒त्रिणः॑ स॒त्रिण॑ ए॒त ए॒ते स॒त्रिण॒ इतीति॑ स॒त्रिण॑ ए॒त ए॒ते स॒त्रिण॒ इति॑ । \newline
57. स॒त्रिण॒ इतीति॑ स॒त्रिणः॑ स॒त्रिण॒ इति॑ प्रि॒यः प्रि॒य इति॑ स॒त्रिणः॑ स॒त्रिण॒ इति॑ प्रि॒यः । \newline
58. इति॑ प्रि॒यः प्रि॒य इतीति॑ प्रि॒यः प्र॒जाना᳚म् प्र॒जाना᳚म् प्रि॒य इतीति॑ प्रि॒यः प्र॒जाना᳚म् । \newline
59. प्रि॒यः प्र॒जाना᳚म् प्र॒जाना᳚म् प्रि॒यः प्रि॒यः प्र॒जाना॒ मुत्थि॑त॒ उत्थि॑तः प्र॒जाना᳚म् प्रि॒यः प्रि॒यः प्र॒जाना॒ मुत्थि॑तः । \newline
60. प्र॒जाना॒ मुत्थि॑त॒ उत्थि॑तः प्र॒जाना᳚म् प्र॒जाना॒ मुत्थि॑तो भवति भव॒ त्युत्थि॑तः प्र॒जाना᳚म् प्र॒जाना॒ मुत्थि॑तो भवति । \newline
61. प्र॒जाना॒मिति॑ प्र - जाना᳚म् । \newline
62. उत्थि॑तो भवति भव॒ त्युत्थि॑त॒ उत्थि॑तो भवति॒ यो यो भ॑व॒ त्युत्थि॑त॒ उत्थि॑तो भवति॒ यः । \newline
63. उत्थि॑त॒ इत्युत् - स्थि॒तः॒ । \newline
64. भ॒व॒ति॒ यो यो भ॑वति भवति॒ य ए॒व मे॒वं ॅयो भ॑वति भवति॒ य ए॒वम् । \newline
65. य ए॒व मे॒वं ॅयो य ए॒वं ॅवेद॒ वेदै॒वं ॅयो य ए॒वं ॅवेद॑ । \newline
66. ए॒वं ॅवेद॒ वेदै॒व मे॒वं ॅवेद॑ । \newline
67. वेदेति॒ वेद॑ । \newline
\pagebreak
\markright{ TS 7.2.10.1  \hfill https://www.vedavms.in \hfill}

\section{ TS 7.2.10.1 }

\textbf{TS 7.2.10.1 } \newline
\textbf{Samhita Paata} \newline

न वा ए॒षो᳚ऽन्यतो॑वैश्वानरः सुव॒र्गाय॑ लो॒काय॒ प्राभ॑वदू॒र्द्ध्वो ह॒ वा ए॒ष आत॑त आसी॒त् ते दे॒वा ए॒तं ॅवै᳚श्वान॒रं पर्यौ॑हन्थ् सुव॒र्गस्य॑ लो॒कस्य॒ प्रभू᳚त्या ऋ॒तवो॒ वा ए॒तेन॑ प्र॒जाप॑तिमयाजय॒न् तेष्वा᳚र्द्ध्नो॒दधि॒ तदृ॒द्ध्नोति॑ ह॒ वा ऋ॒त्विक्षु॒ य ए॒वं ॅवि॒द्वान् द्वा॑दशा॒हेन॒ यज॑ते॒ ते᳚ऽस्मिन्नैच्छन्त॒ स रस॒मह॑ वस॒न्ताय॒ प्राय॑च्छ॒ - [  ] \newline

\textbf{Pada Paata} \newline

न । वै । ए॒षः । अ॒न्यतो॑वैश्वानर॒ इत्य॒न्यतः॑ - वै॒श्वा॒न॒रः॒ । सु॒व॒र्गायेति॑ सुवः - गाय॑ । लो॒काय॑ । प्रेति॑ । अ॒भ॒व॒त् । ऊ॒द्‌र्ध्वः । ह॒ । वै । ए॒षः । आत॑त॒ इत्या - त॒तः॒ । आ॒सी॒त् । ते । दे॒वाः । ए॒तम् । वै॒श्वा॒न॒रम् । परीति॑ । औ॒ह॒न्न् । सु॒व॒र्गस्येति॑ सुवः - गस्य॑ । लो॒कस्य॑ । प्रभू᳚त्या॒ इति॒ प्र-भू॒त्यै॒ । ऋ॒तवः॑ । वै । ए॒तेन॑ । प्र॒जाप॑ति॒मिति॑ प्र॒जा-प॒ति॒म् । अ॒या॒ज॒य॒न्न् । तेषु॑ । आ॒द्‌र्ध्नो॒त् । अधीति॑ । तत् । ऋ॒द्ध्नोति॑ । ह॒ । वै । ऋ॒त्विक्षु॑ । यः । ए॒वम् । वि॒द्वान् । द्वा॒द॒शा॒हेनेति॑ द्वादश - अ॒हेन॑ । यज॑ते । ते । अ॒स्मि॒न्न् । ऐ॒च्छ॒न्त॒ । सः । रस᳚म् । अह॑ । व॒स॒न्ताय॑ । प्रेति॑ । अय॑च्छत् ।  \newline


\textbf{Krama Paata} \newline

न वै । वा ए॒षः । ए॒षो᳚ऽन्यतो॑वैश्वानरः । अ॒न्यतो॑वैश्वानरः सुव॒र्गाय॑ । अ॒न्यतो॑वैश्वानर॒ इत्य॒न्यतः॑ - वै॒श्वा॒न॒रः॒ । सु॒व॒र्गाय॑ लो॒काय॑ । सु॒व॒र्गायेति॑ सुवः - गाय॑ । लो॒काय॒ प्र । प्राभ॑वत् । अ॒भ॒व॒दू॒र्द्ध्वः । ऊ॒र्द्ध्वो ह॑ । ह॒ वै । वा ए॒षः । ए॒ष आत॑तः । आत॑त आसीत् । आत॑त॒ इत्या - त॒तः॒ । आ॒सी॒त् ते । ते दे॒वाः । दे॒वा ए॒तम् । ए॒तम् ॅवै᳚श्वान॒रम् । वै॒श्वा॒न॒रम् परि॑ । पर्यौ॑हन्न् । औ॒ह॒न्थ् सु॒व॒र्गस्य॑ । सु॒व॒र्गस्य॑ लो॒कस्य॑ । सु॒व॒र्गस्येति॑ सुवः - गस्य॑ । लो॒कस्य॒ प्रभू᳚त्यै । प्रभू᳚त्या ऋ॒तवः॑ । प्रभू᳚त्या॒ इति॒ प्र - भू॒त्यै॒ । ऋ॒तवो॒ वै । वा ए॒तेन॑ । ए॒तेन॑ प्र॒जाप॑तिम् । प्र॒जाप॑तिमयाजयन्न् । प्र॒जाप॑ति॒मिति॑ प्र॒जा - प॒ति॒म् । अ॒या॒ज॒य॒न् तेषु॑ । 
तेष्वा᳚र्द्ध्नोत् । आ॒र्द्ध्नो॒दधि॑ । अधि॒ तत् । तदृ॒द्ध्नोति॑ । ऋ॒द्ध्नोति॑ ह । ह॒ वै । वा ऋ॒त्विक्षु॑ । ऋ॒त्विक्षु॒ यः । य ए॒वम् । ए॒वम् ॅवि॒द्वान् । वि॒द्वान् द्वा॑दशा॒हेन॑ । द्वा॒द॒शा॒हेन॒ यज॑ते । द्वा॒द॒शा॒हेनेति॑ द्वादश - अ॒हेन॑ । यज॑ते॒ ते । ते᳚ऽस्मिन्न् । अ॒स्मि॒न्नै॒च्छ॒न्त॒ । ऐ॒च्छ॒न्त॒ सः । स रस᳚म् । रस॒मह॑ । अह॑ वस॒न्ताय॑ । व॒स॒न्ताय॒ प्र । प्राय॑च्छत् । अय॑च्छ॒द् यव᳚म् \newline

\textbf{Jatai Paata} \newline

1. न वै वै न न वै । \newline
2. वा ए॒ष ए॒ष वै वा ए॒षः । \newline
3. ए॒षो᳚ ऽन्यतो॑वैश्वानरो॒ ऽन्यतो॑वैश्वानर ए॒ष ए॒षो᳚ ऽन्यतो॑वैश्वानरः । \newline
4. अ॒न्यतो॑वैश्वानरः सुव॒र्गाय॑ सुव॒र्गाया॒ न्यतो॑वैश्वानरो॒ ऽन्यतो॑वैश्वानरः सुव॒र्गाय॑ । \newline
5. अ॒न्यतो॑वैश्वानर॒ इत्य॒न्यतः॑ - वै॒श्वा॒न॒रः॒ । \newline
6. सु॒व॒र्गाय॑ लो॒काय॑ लो॒काय॑ सुव॒र्गाय॑ सुव॒र्गाय॑ लो॒काय॑ । \newline
7. सु॒व॒र्गायेति॑ सुवः - गाय॑ । \newline
8. लो॒काय॒ प्र प्र लो॒काय॑ लो॒काय॒ प्र । \newline
9. प्राभ॑व दभव॒त् प्र प्राभ॑वत् । \newline
10. अ॒भ॒व॒ दू॒र्द्ध्व ऊ॒र्द्ध्वो॑ ऽभव दभव दू॒र्द्ध्वः । \newline
11. ऊ॒र्द्ध्वो ह॑ हो॒र्द्ध्व ऊ॒र्द्ध्वो ह॑ । \newline
12. ह॒ वै वै ह॑ ह॒ वै । \newline
13. वा ए॒ष ए॒ष वै वा ए॒षः । \newline
14. ए॒ष आत॑त॒ आत॑त ए॒ष ए॒ष आत॑तः । \newline
15. आत॑त आसी दासी॒ दात॑त॒ आत॑त आसीत् । \newline
16. आत॑त॒ इत्या - त॒तः॒ । \newline
17. आ॒सी॒त् ते त आ॑सी दासी॒त् ते । \newline
18. ते दे॒वा दे॒वा स्ते ते दे॒वाः । \newline
19. दे॒वा ए॒त मे॒तम् दे॒वा दे॒वा ए॒तम् । \newline
20. ए॒तं ॅवै᳚श्वान॒रं ॅवै᳚श्वान॒र मे॒त मे॒तं ॅवै᳚श्वान॒रम् । \newline
21. वै॒श्वा॒न॒रम् परि॒ परि॑ वैश्वान॒रं ॅवै᳚श्वान॒रम् परि॑ । \newline
22. पर्यौ॑हन् नौह॒न् परि॒ पर्यौ॑हन्न् । \newline
23. औ॒ह॒न् थ्सु॒व॒र्गस्य॑ सुव॒र्ग स्यौ॑हन् नौहन् थ्सुव॒र्गस्य॑ । \newline
24. सु॒व॒र्गस्य॑ लो॒कस्य॑ लो॒कस्य॑ सुव॒र्गस्य॑ सुव॒र्गस्य॑ लो॒कस्य॑ । \newline
25. सु॒व॒र्गस्येति॑ सुवः - गस्य॑ । \newline
26. लो॒कस्य॒ प्रभू᳚त्यै॒ प्रभू᳚त्यै लो॒कस्य॑ लो॒कस्य॒ प्रभू᳚त्यै । \newline
27. प्रभू᳚त्या ऋ॒तव॑ ऋ॒तवः॒ प्रभू᳚त्यै॒ प्रभू᳚त्या ऋ॒तवः॑ । \newline
28. प्रभू᳚त्या॒ इति॒ प्र - भू॒त्यै॒ । \newline
29. ऋ॒तवो॒ वै वा ऋ॒तव॑ ऋ॒तवो॒ वै । \newline
30. वा ए॒ते नै॒तेन॒ वै वा ए॒तेन॑ । \newline
31. ए॒तेन॑ प्र॒जाप॑तिम् प्र॒जाप॑ति मे॒ते नै॒तेन॑ प्र॒जाप॑तिम् । \newline
32. प्र॒जाप॑ति मयाजयन् नयाजयन् प्र॒जाप॑तिम् प्र॒जाप॑ति मयाजयन्न् । \newline
33. प्र॒जाप॑ति॒मिति॑ प्र॒जा - प॒ति॒म् । \newline
34. अ॒या॒ज॒य॒न् तेषु॒ तेष्व॑याजयन् नयाजय॒न् तेषु॑ । \newline
35. तेष्वा᳚ र्द्ध्नो दार्द्ध्नो॒त् तेषु॒ तेष्वा᳚ र्द्ध्नोत् । \newline
36. आ॒र्द्ध्नो॒ दध्यध्या᳚ र्द्ध्नो दार्द्ध्नो॒ दधि॑ । \newline
37. अधि॒ तत् तदध्यधि॒ तत् । \newline
38. तदृ॒द्ध्नो त्यृ॒द्ध्नोति॒ तत् तदृ॒द्ध्नोति॑ । \newline
39. ऋ॒द्ध्नोति॑ ह ह॒ र्‌द्ध्नो त्यृ॒द्ध्नोति॑ ह । \newline
40. ह॒ वै वै ह॑ ह॒ वै । \newline
41. वा ऋ॒त्विक् ष्वृ॒त्विक्षु॒ वै वा ऋ॒त्विक्षु॑ । \newline
42. ऋ॒त्विक्षु॒ यो य ऋ॒त्विक् ष्वृ॒त्विक्षु॒ यः । \newline
43. य ए॒व मे॒वं ॅयो य ए॒वम् । \newline
44. ए॒वं ॅवि॒द्वान्. वि॒द्वा ने॒व मे॒वं ॅवि॒द्वान् । \newline
45. वि॒द्वान् द्वा॑दशा॒हेन॑ द्वादशा॒हेन॑ वि॒द्वान्. वि॒द्वान् द्वा॑दशा॒हेन॑ । \newline
46. द्वा॒द॒शा॒हेन॒ यज॑ते॒ यज॑ते द्वादशा॒हेन॑ द्वादशा॒हेन॒ यज॑ते । \newline
47. द्वा॒द॒शा॒हेनेति॑ द्वादश - अ॒हेन॑ । \newline
48. यज॑ते॒ ते ते यज॑ते॒ यज॑ते॒ ते । \newline
49. ते᳚ ऽस्मिन् नस्मि॒न् ते ते᳚ ऽस्मिन्न् । \newline
50. अ॒स्मि॒न् नै॒च्छ॒न् तै॒च्छ॒न् ता॒स्मि॒न् न॒स्मि॒न् नै॒च्छ॒न्त॒ । \newline
51. ऐ॒च्छ॒न्त॒ स स ऐ᳚च्छन् तैच्छन्त॒ सः । \newline
52. स रसꣳ॒॒ रसꣳ॒॒ स स रस᳚म् । \newline
53. रस॒ महाह॒ रसꣳ॒॒ रस॒ मह॑ । \newline
54. अह॑ वस॒न्ताय॑ वस॒न्ताया हाह॑ वस॒न्ताय॑ । \newline
55. व॒स॒न्ताय॒ प्र प्र व॑स॒न्ताय॑ वस॒न्ताय॒ प्र । \newline
56. प्राय॑च्छ॒ दय॑च्छ॒त् प्र प्राय॑च्छत् । \newline
57. अय॑च्छ॒द् यवं॒ ॅयव॒ मय॑च्छ॒ दय॑च्छ॒द् यव᳚म् । \newline

\textbf{Ghana Paata } \newline

1. न वै वै न न वा ए॒ष ए॒ष वै न न वा ए॒षः । \newline
2. वा ए॒ष ए॒ष वै वा ए॒षो᳚ ऽन्यतो॑वैश्वानरो॒ ऽन्यतो॑वैश्वानर ए॒ष वै वा ए॒षो᳚ ऽन्यतो॑वैश्वानरः । \newline
3. ए॒षो᳚ ऽन्यतो॑वैश्वानरो॒ ऽन्यतो॑वैश्वानर ए॒ष ए॒षो᳚ ऽन्यतो॑वैश्वानरः सुव॒र्गाय॑ सुव॒र्गाया॒ न्यतो॑वैश्वानर ए॒ष ए॒षो᳚ ऽन्यतो॑वैश्वानरः सुव॒र्गाय॑ । \newline
4. अ॒न्यतो॑वैश्वानरः सुव॒र्गाय॑ सुव॒र्गाया॒ न्यतो॑वैश्वानरो॒ ऽन्यतो॑वैश्वानरः सुव॒र्गाय॑ लो॒काय॑ लो॒काय॑ सुव॒र्गाया॒ न्यतो॑वैश्वानरो॒ ऽन्यतो॑वैश्वानरः सुव॒र्गाय॑ लो॒काय॑ । \newline
5. अ॒न्यतो॑वैश्वानर॒ इत्य॒न्यतः॑ - वै॒श्वा॒न॒रः॒ । \newline
6. सु॒व॒र्गाय॑ लो॒काय॑ लो॒काय॑ सुव॒र्गाय॑ सुव॒र्गाय॑ लो॒काय॒ प्र प्र लो॒काय॑ सुव॒र्गाय॑ सुव॒र्गाय॑ लो॒काय॒ प्र । \newline
7. सु॒व॒र्गायेति॑ सुवः - गाय॑ । \newline
8. लो॒काय॒ प्र प्र लो॒काय॑ लो॒काय॒ प्रा भ॑व दभव॒त् प्र लो॒काय॑ लो॒काय॒ प्रा भ॑वत् । \newline
9. प्रा भ॑व दभव॒त् प्र प्रा भ॑व दू॒र्द्ध्व ऊ॒र्द्ध्वो॑ ऽभव॒त् प्र प्रा भ॑व दू॒र्द्ध्वः । \newline
10. अ॒भ॒व॒ दू॒र्द्ध्व ऊ॒र्द्ध्वो॑ ऽभव दभव दू॒र्द्ध्वो ह॑ हो॒र्द्ध्वो॑ ऽभव दभव दू॒र्द्ध्वो ह॑ । \newline
11. ऊ॒र्द्ध्वो ह॑ हो॒र्द्ध्व ऊ॒र्द्ध्वो ह॒ वै वै हो॒र्द्ध्व ऊ॒र्द्ध्वो ह॒ वै । \newline
12. ह॒ वै वै ह॑ ह॒ वा ए॒ष ए॒ष वै ह॑ ह॒ वा ए॒षः । \newline
13. वा ए॒ष ए॒ष वै वा ए॒ष आत॑त॒ आत॑त ए॒ष वै वा ए॒ष आत॑तः । \newline
14. ए॒ष आत॑त॒ आत॑त ए॒ष ए॒ष आत॑त आसी दासी॒ दात॑त ए॒ष ए॒ष आत॑त आसीत् । \newline
15. आत॑त आसी दासी॒ दात॑त॒ आत॑त आसी॒त् ते त आ॑सी॒ दात॑त॒ आत॑त आसी॒त् ते । \newline
16. आत॑त॒ इत्या - त॒तः॒ । \newline
17. आ॒सी॒त् ते त आ॑सी दासी॒त् ते दे॒वा दे॒वा स्त आ॑सी दासी॒त् ते दे॒वाः । \newline
18. ते दे॒वा दे॒वा स्ते ते दे॒वा ए॒त मे॒तम् दे॒वा स्ते ते दे॒वा ए॒तम् । \newline
19. दे॒वा ए॒त मे॒तम् दे॒वा दे॒वा ए॒तं ॅवै᳚श्वान॒रं ॅवै᳚श्वान॒र मे॒तम् दे॒वा दे॒वा ए॒तं ॅवै᳚श्वान॒रम् । \newline
20. ए॒तं ॅवै᳚श्वान॒रं ॅवै᳚श्वान॒र मे॒त मे॒तं ॅवै᳚श्वान॒रम् परि॒ परि॑ वैश्वान॒र मे॒त मे॒तं ॅवै᳚श्वान॒रम् परि॑ । \newline
21. वै॒श्वा॒न॒रम् परि॒ परि॑ वैश्वान॒रं ॅवै᳚श्वान॒रम् पर्यौ॑हन् नौह॒न् परि॑ वैश्वान॒रं ॅवै᳚श्वान॒रम् पर्यौ॑हन्न् । \newline
22. पर्यौ॑हन् नौह॒न् परि॒ पर्यौ॑हन् थ्सुव॒र्गस्य॑ सुव॒र्ग स्यौ॑ह॒न् परि॒ पर्यौ॑हन् थ्सुव॒र्गस्य॑ । \newline
23. औ॒ह॒न् थ्सु॒व॒र्गस्य॑ सुव॒र्ग स्यौ॑हन् नौहन् थ्सुव॒र्गस्य॑ लो॒कस्य॑ लो॒कस्य॑ सुव॒र्ग स्यौ॑हन् नौहन् थ्सुव॒र्गस्य॑ लो॒कस्य॑ । \newline
24. सु॒व॒र्गस्य॑ लो॒कस्य॑ लो॒कस्य॑ सुव॒र्गस्य॑ सुव॒र्गस्य॑ लो॒कस्य॒ प्रभू᳚त्यै॒ प्रभू᳚त्यै लो॒कस्य॑ सुव॒र्गस्य॑ सुव॒र्गस्य॑ लो॒कस्य॒ प्रभू᳚त्यै । \newline
25. सु॒व॒र्गस्येति॑ सुवः - गस्य॑ । \newline
26. लो॒कस्य॒ प्रभू᳚त्यै॒ प्रभू᳚त्यै लो॒कस्य॑ लो॒कस्य॒ प्रभू᳚त्या ऋ॒तव॑ ऋ॒तवः॒ प्रभू᳚त्यै लो॒कस्य॑ लो॒कस्य॒ प्रभू᳚त्या ऋ॒तवः॑ । \newline
27. प्रभू᳚त्या ऋ॒तव॑ ऋ॒तवः॒ प्रभू᳚त्यै॒ प्रभू᳚त्या ऋ॒तवो॒ वै वा ऋ॒तवः॒ प्रभू᳚त्यै॒ प्रभू᳚त्या ऋ॒तवो॒ वै । \newline
28. प्रभू᳚त्या॒ इति॒ प्र - भू॒त्यै॒ । \newline
29. ऋ॒तवो॒ वै वा ऋ॒तव॑ ऋ॒तवो॒ वा ए॒ते नै॒तेन॒ वा ऋ॒तव॑ ऋ॒तवो॒ वा ए॒तेन॑ । \newline
30. वा ए॒ते नै॒तेन॒ वै वा ए॒तेन॑ प्र॒जाप॑तिम् प्र॒जाप॑ति मे॒तेन॒ वै वा ए॒तेन॑ प्र॒जाप॑तिम् । \newline
31. ए॒तेन॑ प्र॒जाप॑तिम् प्र॒जाप॑ति मे॒ते नै॒तेन॑ प्र॒जाप॑ति मयाजयन् नयाजयन् प्र॒जाप॑ति मे॒ते नै॒तेन॑ प्र॒जाप॑ति मयाजयन्न् । \newline
32. प्र॒जाप॑ति मयाजयन् नयाजयन् प्र॒जाप॑तिम् प्र॒जाप॑ति मयाजय॒न् तेषु॒ तेष्व॑याजयन् प्र॒जाप॑तिम् प्र॒जाप॑ति मयाजय॒न् तेषु॑ । \newline
33. प्र॒जाप॑ति॒मिति॑ प्र॒जा - प॒ति॒म् । \newline
34. अ॒या॒ज॒य॒न् तेषु॒ तेष्व॑याजयन् नयाजय॒न् तेष्वा᳚ र्द्ध्नो दार्द्ध्नो॒त् तेष्व॑याजयन् नयाजय॒न् तेष्वा᳚ र्द्ध्नोत् । \newline
35. तेष्वा᳚ र्द्ध्नो दार्द्ध्नो॒त् तेषु॒ तेष्वा᳚ र्द्ध्नो॒ दध्यध्या᳚ र्द्ध्नो॒त् तेषु॒ तेष्वा᳚र्द्ध्नो॒ दधि॑ । \newline
36. आ॒र्द्ध्नो॒ दध्यध्या᳚ र्द्ध्नोदा र्द्ध्नो॒ दधि॒ तत् तदध्या᳚ र्द्ध्नोदा र्द्ध्नो॒ दधि॒ तत् । \newline
37. अधि॒ तत् तद ध्यधि॒ तदृ॒द्ध्नो त्यृ॒द्ध्नोति॒ तद ध्यधि॒ तदृ॒द्ध्नोति॑ । \newline
38. तदृ॒द्ध्नो त्यृ॒द्ध्नोति॒ तत् तदृ॒द्ध्नोति॑ ह ह॒ र्द्ध्नोति॒ तत् तदृ॒द्ध्नोति॑ ह । \newline
39. ऋ॒द्ध्नोति॑ ह ह॒ र्द्ध्नो त्यृ॒द्ध्नोति॑ ह॒ वै वै ह॒ र्द्ध्नो त्यृ॒द्ध्नोति॑ ह॒ वै । \newline
40. ह॒ वै वै ह॑ ह॒ वा ऋ॒त्विक् ष्वृ॒त्विक्षु॒ वै ह॑ ह॒ वा ऋ॒त्विक्षु॑ । \newline
41. वा ऋ॒त्विक् ष्वृ॒त्विक्षु॒ वै वा ऋ॒त्विक्षु॒ यो य ऋ॒त्विक्षु॒ वै वा ऋ॒त्विक्षु॒ यः । \newline
42. ऋ॒त्विक्षु॒ यो य ऋ॒त्विक् ष्वृ॒त्विक्षु॒ य ए॒व मे॒वं ॅय ऋ॒त्विक् ष्वृ॒त्विक्षु॒ य ए॒वम् । \newline
43. य ए॒व मे॒वं ॅयो य ए॒वं ॅवि॒द्वान्. वि॒द्वा ने॒वं ॅयो य ए॒वं ॅवि॒द्वान् । \newline
44. ए॒वं ॅवि॒द्वान्. वि॒द्वा ने॒व मे॒वं ॅवि॒द्वान् द्वा॑दशा॒हेन॑ द्वादशा॒हेन॑ वि॒द्वा ने॒व मे॒वं ॅवि॒द्वान् द्वा॑दशा॒हेन॑ । \newline
45. वि॒द्वान् द्वा॑दशा॒हेन॑ द्वादशा॒हेन॑ वि॒द्वान्. वि॒द्वान् द्वा॑दशा॒हेन॒ यज॑ते॒ यज॑ते द्वादशा॒हेन॑ वि॒द्वान्. वि॒द्वान् द्वा॑दशा॒हेन॒ यज॑ते । \newline
46. द्वा॒द॒शा॒हेन॒ यज॑ते॒ यज॑ते द्वादशा॒हेन॑ द्वादशा॒हेन॒ यज॑ते॒ ते ते यज॑ते द्वादशा॒हेन॑ द्वादशा॒हेन॒ यज॑ते॒ ते । \newline
47. द्वा॒द॒शा॒हेनेति॑ द्वादश - अ॒हेन॑ । \newline
48. यज॑ते॒ ते ते यज॑ते॒ यज॑ते॒ ते᳚ ऽस्मिन् नस्मि॒न् ते यज॑ते॒ यज॑ते॒ ते᳚ ऽस्मिन्न् । \newline
49. ते᳚ ऽस्मिन् नस्मि॒न् ते ते᳚ ऽस्मिन् नैच्छ न्तैच्छन्ता स्मि॒न् ते ते᳚ ऽस्मिन् नैच्छन्त । \newline
50. अ॒स्मि॒न् नै॒च्छ॒ न्तै॒च्छ॒न्ता॒ स्मि॒न् न॒स्मि॒न् नै॒च्छ॒न्त॒ स स ऐ᳚च्छ न्तास्मिन् नस्मिन् नैच्छन्त॒ सः । \newline
51. ऐ॒च्छ॒न्त॒ स स ऐ᳚च्छ न्तैच्छन्त॒ स रसꣳ॒॒ रसꣳ॒॒ स ऐ᳚च्छ न्तैच्छन्त॒ स रस᳚म् । \newline
52. स रसꣳ॒॒ रसꣳ॒॒ स स रस॒ महाह॒ रसꣳ॒॒ स स रस॒ मह॑ । \newline
53. रस॒ महाह॒ रसꣳ॒॒ रस॒ मह॑ वस॒न्ताय॑ वस॒न्ता याह॒ रसꣳ॒॒ रस॒ मह॑ वस॒न्ताय॑ । \newline
54. अह॑ वस॒न्ताय॑ वस॒न्ता याहाह॑ वस॒न्ताय॒ प्र प्र व॑स॒न्ता याहाह॑ वस॒न्ताय॒ प्र । \newline
55. व॒स॒न्ताय॒ प्र प्र व॑स॒न्ताय॑ वस॒न्ताय॒ प्राय॑च्छ॒ दय॑च्छ॒त् प्र व॑स॒न्ताय॑ वस॒न्ताय॒ प्राय॑च्छत् । \newline
56. प्रा य॑च्छ॒ दय॑च्छ॒त् प्र प्रा य॑च्छ॒द् यवं॒ ॅयव॒ मय॑च्छ॒त् प्र प्रा य॑च्छ॒द् यव᳚म् । \newline
57. अय॑च्छ॒द् यवं॒ ॅयव॒ मय॑च्छ॒ दय॑च्छ॒द् यव॑म् ग्री॒ष्माय॑ ग्री॒ष्माय॒ यव॒ मय॑च्छ॒ दय॑च्छ॒द् यव॑म् ग्री॒ष्माय॑ । \newline
\pagebreak
\markright{ TS 7.2.10.2  \hfill https://www.vedavms.in \hfill}

\section{ TS 7.2.10.2 }

\textbf{TS 7.2.10.2 } \newline
\textbf{Samhita Paata} \newline

द्यवं॑ ग्री॒ष्मायौष॑धीर्व॒र्॒.षाभ्यो᳚ व्री॒हीञ्छ॒रदे॑ माषति॒लौ हे॑मन्तशिशि॒राभ्यां॒ तेनेन्द्रं॑ प्र॒जाप॑तिरयाजय॒त् ततो॒ वा इन्द्र॒ इन्द्रो॑ऽभव॒त् तस्मा॑दाहुरानुजाव॒रस्य॑ य॒ज्ञ् इति॒ स ह्ये॑तेनाऽग्रेऽय॑जतै॒ष ह॒ वै कु॒णप॑मत्ति॒ यः स॒त्रे प्र॑तिगृ॒ह्णाति॑ पुरुषकुण॒पम॑श्वकुण॒पं गौर्वा अन्नं॒ ॅयेन॒ पात्रे॒णान्नं॒ बिभ्र॑ति॒ यत् तन्न नि॒र्णेनि॑जति॒ ततोऽधि॒ - [  ] \newline

\textbf{Pada Paata} \newline

यव᳚म् । ग्री॒ष्माय॑ । ओष॑धीः । व॒र्॒.षाभ्यः॑ । व्री॒हीन् । श॒रदे᳚ । मा॒ष॒ति॒लाविति॑ माष - ति॒लौ । हे॒म॒न्त॒शि॒शि॒राभ्या॒मिति॑ हेमन्त - शि॒शि॒राभ्या᳚म् । तेन॑ । इन्द्र᳚म् । प्र॒जाप॑ति॒रिति॑ प्र॒जा-प॒तिः॒ । अ॒या॒ज॒य॒त् । ततः॑ । वै । इन्द्रः॑ । इन्द्रः॑ । अ॒भ॒व॒त् । तस्मा᳚त् । आ॒हुः॒ । आ॒नु॒जा॒व॒रस्येत्या॑नु - जा॒व॒रस्य॑ । य॒ज्ञ्ः । इति॑ । सः । हि । ए॒तेन॑ । अग्रे᳚ । अय॑जत । ए॒षः । ह॒ । वै । कु॒णप᳚म् । अ॒त्ति॒ । यः । स॒त्रे । प्र॒ति॒गृ॒ह्णातीति॑ प्रति - गृ॒ह्णाति॑ । पु॒रु॒ष॒कु॒ण॒पमिति॑ पुरुष - कु॒ण॒पम् । अ॒श्व॒कु॒ण॒पमित्य॑श्व - कु॒ण॒पम् । गौः । वै । अन्न᳚म् । येन॑ । पात्रे॑ण । अन्न᳚म् । बिभ्र॑ति । यत् । तत् । न । नि॒र्णेनि॑ज॒तीति॑ निः - नेनि॑जति । ततः॑ । अधीति॑ ।  \newline


\textbf{Krama Paata} \newline

यव॑म् ग्री॒ष्माय॑ । ग्री॒ष्मायौष॑धीः । ओष॑धीर् व॒र्॒.षाभ्यः॑ । व॒र्॒.षाभ्यो᳚ व्री॒हीन् । व्री॒हीञ्छ॒रदे᳚ । श॒रदे॑ माषति॒लौ । मा॒ष॒ति॒लौ हे॑मन्तशिशि॒राभ्या᳚म् । मा॒ष॒ति॒लाविति॑ माष - ति॒लौ । हे॒म॒न्त॒शि॒शि॒राभ्या॒म् तेन॑ । हे॒म॒न्त॒शि॒शि॒राभ्या॒मिति॑ हेमन्त - शि॒शि॒राभ्या᳚म् । तेनेन्द्र᳚म् । इन्द्र॑म् प्र॒जाप॑तिः । प्र॒जाप॑तिरयाजयत् । प्र॒जाप॑ति॒रिति॑ प्र॒जा - प॒तिः॒ । अ॒या॒ज॒य॒त् ततः॑ । ततो॒ वै । वा इन्द्रः॑ । इन्द्र॒ इन्द्रः॑ । इन्द्रो॑ऽभवत् । अ॒भ॒व॒त् तस्मा᳚त् । तस्मा॑दाहुः । आ॒हु॒रा॒नु॒जा॒व॒रस्य॑ । आ॒नु॒जा॒व॒रस्य॑ य॒ज्ञ्ः । आ॒नु॒जा॒व॒रस्येत्या॑नु - जा॒व॒रस्य॑ । य॒ज्ञ् इति॑ । इति॒ सः । स हि । ह्ये॑तेन॑ । ए॒तेनाग्रे᳚ । अग्रेऽय॑जत । अय॑जतै॒षः । ए॒ष ह॑ । ह॒ वै । वै कु॒णप᳚म् । कु॒णप॑मत्ति । अ॒त्ति॒ यः । यः स॒त्रे । स॒त्रे प्र॑तिगृ॒ह्णाति॑ । प्र॒ति॒गृ॒ह्णाति॑ पुरुषकुण॒पम् । प्र॒ति॒गृ॒ह्णातीति॑ प्रति - गृ॒ह्णाति॑ । पु॒रु॒ष॒कु॒ण॒पम॑श्वकुण॒पम् । पु॒रु॒ष॒कु॒ण॒पमिति॑ पुरुष - कु॒ण॒पम् । अ॒श्व॒कु॒ण॒पम् गौः । अ॒श्व॒कु॒ण॒पमित्य॑श्व - कु॒ण॒पम् । गौर् वै । वा अन्न᳚म् । अन्न॒म् ॅयेन॑ । येन॒ पात्रे॑ण । पात्रे॒णान्न᳚म् । अन्न॒म् बिभ्र॑ति । बिभ्र॑ति॒ यत् । यत् तत् । तन् न । न नि॒र्णेनि॑जति । नि॒र्णेनि॑जति॒ ततः॑ । नि॒र्णेनि॑ज॒तीति॑ निः - नेनि॑जति । ततोऽधि॑ । अधि॒ मल᳚म् \newline

\textbf{Jatai Paata} \newline

1. यव॑म् ग्री॒ष्माय॑ ग्री॒ष्माय॒ यवं॒ ॅयव॑म् ग्री॒ष्माय॑ । \newline
2. ग्री॒ष्मा यौष॑धी॒ रोष॑धीर् ग्री॒ष्माय॑ ग्री॒ष्मा यौष॑धीः । \newline
3. ओष॑धीर् व॒र्॒.षाभ्यो॑ व॒र्॒.षाभ्य॒ ओष॑धी॒ रोष॑धीर् व॒र्॒.षाभ्यः॑ । \newline
4. व॒र्॒.षाभ्यो᳚ व्री॒हीन् व्री॒हीन्. व॒र्॒.षाभ्यो॑ व॒र्॒.षाभ्यो᳚ व्री॒हीन् । \newline
5. व्री॒हीञ् छ॒रदे॑ श॒रदे᳚ व्री॒हीन् व्री॒हीञ् छ॒रदे᳚ । \newline
6. श॒रदे॑ माषति॒लौ मा॑षति॒लौ श॒रदे॑ श॒रदे॑ माषति॒लौ । \newline
7. मा॒ष॒ति॒लौ हे॑मन्तशिशि॒राभ्याꣳ॑ हेमन्तशिशि॒राभ्या᳚म् माषति॒लौ मा॑षति॒लौ हे॑मन्तशिशि॒राभ्या᳚म् । \newline
8. मा॒ष॒ति॒लाविति॑ माष - ति॒लौ । \newline
9. हे॒म॒न्त॒शि॒शि॒राभ्या॒म् तेन॒ तेन॑ हेमन्तशिशि॒राभ्याꣳ॑ हेमन्तशिशि॒राभ्या॒म् तेन॑ । \newline
10. हे॒म॒न्त॒शि॒शि॒राभ्या॒मिति॑ हेमन्त - शि॒शि॒राभ्या᳚म् । \newline
11. तेनेन्द्र॒ मिन्द्र॒म् तेन॒ तेनेन्द्र᳚म् । \newline
12. इन्द्र॑म् प्र॒जाप॑तिः प्र॒जाप॑ति॒ रिन्द्र॒ मिन्द्र॑म् प्र॒जाप॑तिः । \newline
13. प्र॒जाप॑ति रयाजय दयाजयत् प्र॒जाप॑तिः प्र॒जाप॑ति रयाजयत् । \newline
14. प्र॒जाप॑ति॒रिति॑ प्र॒जा - प॒तिः॒ । \newline
15. अ॒या॒ज॒य॒त् तत॒ स्ततो॑ ऽयाजय दयाजय॒त् ततः॑ । \newline
16. ततो॒ वै वै तत॒ स्ततो॒ वै । \newline
17. वा इन्द्र॒ इन्द्रो॒ वै वा इन्द्रः॑ । \newline
18. इन्द्र॒ इन्द्रः॑ । \newline
19. इन्द्रो॑ ऽभव दभव॒ दिन्द्र॒ इन्द्रो॑ ऽभवत् । \newline
20. अ॒भ॒व॒त् तस्मा॒त् तस्मा॑ दभव दभव॒त् तस्मा᳚त् । \newline
21. तस्मा॑ दाहु राहु॒ स्तस्मा॒त् तस्मा॑ दाहुः । \newline
22. आ॒हु॒ रा॒नु॒जा॒व॒रस्या॑ नुजाव॒रस्या॑हु राहु रानुजाव॒रस्य॑ । \newline
23. आ॒नु॒जा॒व॒रस्य॑ य॒ज्ञो य॒ज्ञ् आ॑नुजाव॒रस्या॑ नुजाव॒रस्य॑ य॒ज्ञ्ः । \newline
24. आ॒नु॒जा॒व॒रस्येत्या॑नु - जा॒व॒रस्य॑ । \newline
25. य॒ज्ञ् इतीति॑ य॒ज्ञो य॒ज्ञ् इति॑ । \newline
26. इति॒ स स इतीति॒ सः । \newline
27. स हि हि स स हि । \newline
28. ह्ये॑ते नै॒तेन॒ हि ह्ये॑तेन॑ । \newline
29. ए॒ते नाग्रे ऽग्र॑ ए॒ते नै॒तेनाग्रे᳚ । \newline
30. अग्रे ऽय॑ज॒ता य॑ज॒ ताग्रे ऽग्रे ऽय॑जत । \newline
31. अय॑ज तै॒ष ए॒षो ऽय॑ज॒ता य॑ज तै॒षः । \newline
32. ए॒ष ह॑ है॒ष ए॒ष ह॑ । \newline
33. ह॒ वै वै ह॑ ह॒ वै । \newline
34. वै कु॒णप॑म् कु॒णपं॒ ॅवै वै कु॒णप᳚म् । \newline
35. कु॒णप॑ मत्त्यत्ति कु॒णप॑म् कु॒णप॑ मत्ति । \newline
36. अ॒त्ति॒ यो यो᳚ ऽत्त्यत्ति॒ यः । \newline
37. यः स॒त्रे स॒त्रे यो यः स॒त्रे । \newline
38. स॒त्रे प्र॑तिगृ॒ह्णाति॑ प्रतिगृ॒ह्णाति॑ स॒त्रे स॒त्रे प्र॑तिगृ॒ह्णाति॑ । \newline
39. प्र॒ति॒गृ॒ह्णाति॑ पुरुषकुण॒पम् पु॑रुषकुण॒पम् प्र॑तिगृ॒ह्णाति॑ प्रतिगृ॒ह्णाति॑ पुरुषकुण॒पम् । \newline
40. प्र॒ति॒गृ॒ह्णातीति॑ प्रति - गृ॒ह्णाति॑ । \newline
41. पु॒रु॒ष॒कु॒ण॒प म॑श्वकुण॒प म॑श्वकुण॒पम् पु॑रुषकुण॒पम् पु॑रुषकुण॒प म॑श्वकुण॒पम् । \newline
42. पु॒रु॒ष॒कु॒ण॒पमिति॑ पुरुष - कु॒ण॒पम् । \newline
43. अ॒श्व॒कु॒ण॒पम् गौर् गौ र॑श्वकुण॒प म॑श्वकुण॒पम् गौः । \newline
44. अ॒श्व॒कु॒ण॒पमित्य॑श्व - कु॒ण॒पम् । \newline
45. गौर् वै वै गौर् गौर् वै । \newline
46. वा अन्न॒ मन्नं॒ ॅवै वा अन्न᳚म् । \newline
47. अन्नं॒ ॅयेन॒ येनान्न॒ मन्नं॒ ॅयेन॑ । \newline
48. येन॒ पात्रे॑ण॒ पात्रे॑ण॒ येन॒ येन॒ पात्रे॑ण । \newline
49. पात्रे॒ णान्न॒ मन्न॒म् पात्रे॑ण॒ पात्रे॒ णान्न᳚म् । \newline
50. अन्न॒म् बिभ्र॑ति॒ बिभ्र॒ त्यन्न॒ मन्न॒म् बिभ्र॑ति । \newline
51. बिभ्र॑ति॒ यद् यद् बिभ्र॑ति॒ बिभ्र॑ति॒ यत् । \newline
52. यत् तत् तद् यद् यत् तत् । \newline
53. तन् न न तत् तन् न । \newline
54. न नि॒र्णेनि॑जति नि॒र्णेनि॑जति॒ न न नि॒र्णेनि॑जति । \newline
55. नि॒र्णेनि॑जति॒ तत॒ स्ततो॑ नि॒र्णेनि॑जति नि॒र्णेनि॑जति॒ ततः॑ । \newline
56. नि॒र्णेनि॑ज॒तीति॑ निः - नेनि॑जति । \newline
57. ततो ऽध्यधि॒ तत॒ स्ततो ऽधि॑ । \newline
58. अधि॒ मल॒म् मल॒ मध्यधि॒ मल᳚म् । \newline

\textbf{Ghana Paata } \newline

1. यव॑म् ग्री॒ष्माय॑ ग्री॒ष्माय॒ यवं॒ ॅयव॑म् ग्री॒ष्मा यौष॑धी॒ रोष॑धीर् ग्री॒ष्माय॒ यवं॒ ॅयव॑म् ग्री॒ष्मा यौष॑धीः । \newline
2. ग्री॒ष्मा यौष॑धी॒ रोष॑धीर् ग्री॒ष्माय॑ ग्री॒ष्मा यौष॑धीर् व॒र्॒.षाभ्यो॑ व॒र्॒.षाभ्य॒ ओष॑धीर् ग्री॒ष्माय॑ ग्री॒ष्मा यौष॑धीर् व॒र्॒.षाभ्यः॑ । \newline
3. ओष॑धीर् व॒र्॒.षाभ्यो॑ व॒र्॒.षाभ्य॒ ओष॑धी॒ रोष॑धीर् व॒र्॒.षाभ्यो᳚ व्री॒हीन् व्री॒हीन्. व॒र्॒.षाभ्य॒ ओष॑धी॒ रोष॑धीर् व॒र्॒.षाभ्यो᳚ व्री॒हीन् । \newline
4. व॒र्॒.षाभ्यो᳚ व्री॒हीन् व्री॒हीन्. व॒र्॒.षाभ्यो॑ व॒र्॒.षाभ्यो᳚ व्री॒हीञ् छ॒रदे॑ श॒रदे᳚ व्री॒हीन्. व॒र्॒.षाभ्यो॑ व॒र्॒.षाभ्यो᳚ व्री॒हीञ् छ॒रदे᳚ । \newline
5. व्री॒हीञ् छ॒रदे॑ श॒रदे᳚ व्री॒हीन् व्री॒हीञ् छ॒रदे॑ माषति॒लौ मा॑षति॒लौ श॒रदे᳚ व्री॒हीन् व्री॒हीञ् छ॒रदे॑ माषति॒लौ । \newline
6. श॒रदे॑ माषति॒लौ मा॑षति॒लौ श॒रदे॑ श॒रदे॑ माषति॒लौ हे॑मन्तशिशि॒राभ्याꣳ॑ हेमन्तशिशि॒राभ्या᳚म् माषति॒लौ श॒रदे॑ श॒रदे॑ माषति॒लौ हे॑मन्तशिशि॒राभ्या᳚म् । \newline
7. मा॒ष॒ति॒लौ हे॑मन्तशिशि॒राभ्याꣳ॑ हेमन्तशिशि॒राभ्या᳚म् माषति॒लौ मा॑षति॒लौ हे॑मन्तशिशि॒राभ्या॒म् तेन॒ तेन॑ हेमन्तशिशि॒राभ्या᳚म् माषति॒लौ मा॑षति॒लौ हे॑मन्तशिशि॒राभ्या॒म् तेन॑ । \newline
8. मा॒ष॒ति॒लाविति॑ माष - ति॒लौ । \newline
9. हे॒म॒न्त॒शि॒शि॒राभ्या॒म् तेन॒ तेन॑ हेमन्तशिशि॒राभ्याꣳ॑ हेमन्तशिशि॒राभ्या॒म् तेनेन्द्र॒ मिन्द्र॒म् तेन॑ हेमन्तशिशि॒राभ्याꣳ॑ हेमन्तशिशि॒राभ्या॒म् तेनेन्द्र᳚म् । \newline
10. हे॒म॒न्त॒शि॒शि॒राभ्या॒मिति॑ हेमन्त - शि॒शि॒राभ्या᳚म् । \newline
11. तेनेन्द्र॒ मिन्द्र॒म् तेन॒ तेनेन्द्र॑म् प्र॒जाप॑तिः प्र॒जाप॑ति॒ रिन्द्र॒म् तेन॒ तेनेन्द्र॑म् प्र॒जाप॑तिः । \newline
12. इन्द्र॑म् प्र॒जाप॑तिः प्र॒जाप॑ति॒ रिन्द्र॒ मिन्द्र॑म् प्र॒जाप॑ति रयाजय दयाजयत् प्र॒जाप॑ति॒ रिन्द्र॒ मिन्द्र॑म् प्र॒जाप॑ति रयाजयत् । \newline
13. प्र॒जाप॑ति रयाजय दयाजयत् प्र॒जाप॑तिः प्र॒जाप॑ति रयाजय॒त् तत॒ स्ततो॑ ऽयाजयत् प्र॒जाप॑तिः प्र॒जाप॑ति रयाजय॒त् ततः॑ । \newline
14. प्र॒जाप॑ति॒रिति॑ प्र॒जा - प॒तिः॒ । \newline
15. अ॒या॒ज॒य॒त् तत॒ स्ततो॑ ऽयाजय दयाजय॒त् ततो॒ वै वै ततो॑ ऽयाजय दयाजय॒त् ततो॒ वै । \newline
16. ततो॒ वै वै तत॒ स्ततो॒ वा इन्द्र॒ इन्द्रो॒ वै तत॒ स्ततो॒ वा इन्द्रः॑ । \newline
17. वा इन्द्र॒ इन्द्रो॒ वै वा इन्द्रः॑ । \newline
18. इन्द्र॒ इन्द्रः॑ । \newline
19. इन्द्रो॑ ऽभव दभव॒ दिन्द्र॒ इन्द्रो॑ ऽभव॒त् तस्मा॒त् तस्मा॑ दभव॒ दिन्द्र॒ इन्द्रो॑ ऽभव॒त् तस्मा᳚त् । \newline
20. अ॒भ॒व॒त् तस्मा॒त् तस्मा॑ दभव दभव॒त् तस्मा॑ दाहु राहु॒ स्तस्मा॑ दभव दभव॒त् तस्मा॑ दाहुः । \newline
21. तस्मा॑ दाहु राहु॒ स्तस्मा॒त् तस्मा॑ दाहु रानुजाव॒रस्या॑ नुजाव॒रस्या॑हु॒ स्तस्मा॒त् तस्मा॑ दाहु रानुजाव॒रस्य॑ । \newline
22. आ॒हु॒ रा॒नु॒जा॒व॒रस्या॑ नुजाव॒र स्या॑हु राहु रानुजाव॒रस्य॑ य॒ज्ञो य॒ज्ञ् आ॑नुजाव॒रस्या॑ हु राहु रानुजाव॒रस्य॑ य॒ज्ञ्ः । \newline
23. आ॒नु॒जा॒व॒रस्य॑ य॒ज्ञो य॒ज्ञ् आ॑नुजाव॒रस्या॑ नुजाव॒रस्य॑ य॒ज्ञ् इतीति॑ य॒ज्ञ् आ॑नुजाव॒रस्या॑ नुजाव॒रस्य॑ य॒ज्ञ् इति॑ । \newline
24. आ॒नु॒जा॒व॒रस्येत्या॑नु - जा॒व॒रस्य॑ । \newline
25. य॒ज्ञ् इतीति॑ य॒ज्ञो य॒ज्ञ् इति॒ स स इति॑ य॒ज्ञो य॒ज्ञ् इति॒ सः । \newline
26. इति॒ स स इतीति॒ स हि हि स इतीति॒ स हि । \newline
27. स हि हि स स ह्ये॑ते नै॒तेन॒ हि स स ह्ये॑तेन॑ । \newline
28. ह्ये॑ते नै॒तेन॒ हि ह्ये॑ते नाग्रे ऽग्र॑ ए॒तेन॒ हि ह्ये॑ते नाग्रे᳚ । \newline
29. ए॒ते नाग्रे ऽग्र॑ ए॒ते नै॒तेनाग्रे ऽय॑ज॒ता य॑ज॒ ताग्र॑ ए॒ते नै॒तेनाग्रे ऽय॑जत । \newline
30. अग्रे ऽय॑ज॒ता य॑ज॒ताग्रे ऽग्रे ऽय॑ज तै॒ष ए॒षो ऽय॑ज॒ ताग्रे ऽग्रे ऽय॑ज तै॒षः । \newline
31. अय॑ज तै॒ष ए॒षो ऽय॑ज॒ता य॑ज तै॒ष ह॑ है॒षो ऽय॑ज॒ता य॑ज तै॒ष ह॑ । \newline
32. ए॒ष ह॑ है॒ष ए॒ष ह॒ वै वै है॒ष ए॒ष ह॒ वै । \newline
33. ह॒ वै वै ह॑ ह॒ वै कु॒णप॑म् कु॒णपं॒ ॅवै ह॑ ह॒ वै कु॒णप᳚म् । \newline
34. वै कु॒णप॑म् कु॒णपं॒ ॅवै वै कु॒णप॑ मत्त्यत्ति कु॒णपं॒ ॅवै वै कु॒णप॑ मत्ति । \newline
35. कु॒णप॑ मत्त्यत्ति कु॒णप॑म् कु॒णप॑ मत्ति॒ यो यो᳚ ऽत्ति कु॒णप॑म् कु॒णप॑ मत्ति॒ यः । \newline
36. अ॒त्ति॒ यो यो᳚ ऽत्त्यत्ति॒ यः स॒त्रे स॒त्रे यो᳚ ऽत्त्यत्ति॒ यः स॒त्रे । \newline
37. यः स॒त्रे स॒त्रे यो यः स॒त्रे प्र॑तिगृ॒ह्णाति॑ प्रतिगृ॒ह्णाति॑ स॒त्रे यो यः स॒त्रे प्र॑तिगृ॒ह्णाति॑ । \newline
38. स॒त्रे प्र॑तिगृ॒ह्णाति॑ प्रतिगृ॒ह्णाति॑ स॒त्रे स॒त्रे प्र॑तिगृ॒ह्णाति॑ पुरुषकुण॒पम् पु॑रुषकुण॒पम् प्र॑तिगृ॒ह्णाति॑ स॒त्रे स॒त्रे प्र॑तिगृ॒ह्णाति॑ पुरुषकुण॒पम् । \newline
39. प्र॒ति॒गृ॒ह्णाति॑ पुरुषकुण॒पम् पु॑रुषकुण॒पम् प्र॑तिगृ॒ह्णाति॑ प्रतिगृ॒ह्णाति॑ पुरुषकुण॒प म॑श्वकुण॒प म॑श्वकुण॒पम् पु॑रुषकुण॒पम् प्र॑तिगृ॒ह्णाति॑ प्रतिगृ॒ह्णाति॑ पुरुषकुण॒प म॑श्वकुण॒पम् । \newline
40. प्र॒ति॒गृ॒ह्णातीति॑ प्रति - गृ॒ह्णाति॑ । \newline
41. पु॒रु॒ष॒कु॒ण॒प म॑श्वकुण॒प म॑श्वकुण॒पम् पु॑रुषकुण॒पम् पु॑रुषकुण॒प म॑श्वकुण॒पम् गौर् गौ र॑श्वकुण॒पम् पु॑रुषकुण॒पम् पु॑रुषकुण॒प म॑श्वकुण॒पम् गौः । \newline
42. पु॒रु॒ष॒कु॒ण॒पमिति॑ पुरुष - कु॒ण॒पम् । \newline
43. अ॒श्व॒कु॒ण॒पम् गौर् गौ र॑श्वकुण॒प म॑श्वकुण॒पम् गौर् वै वै गौ र॑श्वकुण॒प म॑श्वकुण॒पम् गौर् वै । \newline
44. अ॒श्व॒कु॒ण॒पमित्य॑श्व - कु॒ण॒पम् । \newline
45. गौर् वै वै गौर् गौर् वा अन्न॒ मन्नं॒ ॅवै गौर् गौर् वा अन्न᳚म् । \newline
46. वा अन्न॒ मन्नं॒ ॅवै वा अन्नं॒ ॅयेन॒ येनान्नं॒ ॅवै वा अन्नं॒ ॅयेन॑ । \newline
47. अन्नं॒ ॅयेन॒ येनान्न॒ मन्नं॒ ॅयेन॒ पात्रे॑ण॒ पात्रे॑ण॒ येनान्न॒ मन्नं॒ ॅयेन॒ पात्रे॑ण । \newline
48. येन॒ पात्रे॑ण॒ पात्रे॑ण॒ येन॒ येन॒ पात्रे॒णान्न॒ मन्न॒म् पात्रे॑ण॒ येन॒ येन॒ पात्रे॒णान्न᳚म् । \newline
49. पात्रे॒णान्न॒ मन्न॒म् पात्रे॑ण॒ पात्रे॒णान्न॒म् बिभ्र॑ति॒ बिभ्र॒ त्यन्न॒म् पात्रे॑ण॒ पात्रे॒णान्न॒म् बिभ्र॑ति । \newline
50. अन्न॒म् बिभ्र॑ति॒ बिभ्र॒ त्यन्न॒ मन्न॒म् बिभ्र॑ति॒ यद् यद् बिभ्र॒ त्यन्न॒ मन्न॒म् बिभ्र॑ति॒ यत् । \newline
51. बिभ्र॑ति॒ यद् यद् बिभ्र॑ति॒ बिभ्र॑ति॒ यत् तत् तद् यद् बिभ्र॑ति॒ बिभ्र॑ति॒ यत् तत् । \newline
52. यत् तत् तद् यद् यत् तन् न न तद् यद् यत् तन् न । \newline
53. तन् न न तत् तन् न नि॒र्णेनि॑जति नि॒र्णेनि॑जति॒ न तत् तन् न नि॒र्णेनि॑जति । \newline
54. न नि॒र्णेनि॑जति नि॒र्णेनि॑जति॒ न न नि॒र्णेनि॑जति॒ तत॒ स्ततो॑ नि॒र्णेनि॑जति॒ न न नि॒र्णेनि॑जति॒ ततः॑ । \newline
55. नि॒र्णेनि॑जति॒ तत॒ स्ततो॑ नि॒र्णेनि॑जति नि॒र्णेनि॑जति॒ ततो ऽध्यधि॒ ततो॑ नि॒र्णेनि॑जति नि॒र्णेनि॑जति॒ ततो ऽधि॑ । \newline
56. नि॒र्णेनि॑ज॒तीति॑ निः - नेनि॑जति । \newline
57. ततो ऽध्यधि॒ तत॒ स्ततो ऽधि॒ मल॒म् मल॒ मधि॒ तत॒ स्ततो ऽधि॒ मल᳚म् । \newline
58. अधि॒ मल॒म् मल॒ मध्यधि॒ मल॑म् जायते जायते॒ मल॒ मध्यधि॒ मल॑म् जायते । \newline
\pagebreak
\markright{ TS 7.2.10.3  \hfill https://www.vedavms.in \hfill}

\section{ TS 7.2.10.3 }

\textbf{TS 7.2.10.3 } \newline
\textbf{Samhita Paata} \newline

मलं॑ जायत॒ एक॑ ए॒व य॑जे॒तैको॒ हि प्र॒जाप॑ति॒रार्द्ध्नो॒द् द्वाद॑श॒ रात्री᳚र्दीक्षि॒तः स्या॒द् द्वाद॑श॒ मासाः᳚ संॅवथ्स॒रः सं॑ॅवथ्स॒रः प्र॒जाप॑तिः प्र॒जाप॑ति॒र्वावैष ए॒ष ह॒ त्वै जा॑यते॒ यस्तप॒सोऽधि॒ जाय॑ते चतु॒र्द्धा वा ए॒तास्ति॒स्रस्ति॑स्रो॒ रात्र॑यो॒ यद् द्वाद॑शोप॒सदो॒ याः प्र॑थ॒मा य॒ज्ञ्ं ताभिः॒ सं भ॑रति॒ या द्वि॒तीया॑ य॒ज्ञ्ं ताभि॒रा र॑भते॒ - [  ] \newline

\textbf{Pada Paata} \newline

मल᳚म् । जा॒य॒ते॒ । एकः॑ । ए॒व । य॒जे॒त॒ । एकः॑ । हि । प्र॒जाप॑ति॒रिति॑ प्र॒जा - प॒तिः॒ । आद्‌र्ध्नो᳚त् । द्वाद॑श । रात्रीः᳚ । दी॒क्षि॒तः । स्या॒त् । द्वाद॑श । मासाः᳚ । सं॒ॅव॒थ्स॒र इति॑ सं - व॒थ्स॒रः । सं॒ॅव॒थ्स॒र इति॑ सं - व॒थ्स॒रः । प्र॒जाप॑ति॒रिति॑ प्र॒जा - प॒तिः॒ । प्र॒जाप॑ति॒रिति॑ प्र॒जा - प॒तिः॒ । वाव । ए॒षः । ए॒षः । ह॒ । तु । वै । जा॒य॒ते॒ । यः । तप॑सः । अधीति॑ । जाय॑ते । च॒तु॒द्‌र्धेति॑ चतुः - धा । वै । ए॒ताः । ति॒स्रस्ति॑स्र॒ इति॑ ति॒स्रः-ति॒स्रः॒ । रात्र॑यः । यत् । द्वाद॑श । उ॒प॒सद॒ इत्यु॑प - सदः॑ । याः । प्र॒थ॒माः । य॒ज्ञ्म् । ताभिः॑ । समिति॑ । भ॒र॒ति॒ । याः । द्वि॒तीयाः᳚ । य॒ज्ञ्म् । ताभिः॑ । एति॑ । र॒भ॒ते॒ ।  \newline


\textbf{Krama Paata} \newline

मल॑म् जायते । जा॒य॒त॒ एकः॑ । एक॑ ए॒व । ए॒व य॑जेत । य॒जे॒तैकः॑ । एको॒ हि । हि प्र॒जाप॑तिः । प्र॒जाप॑ति॒रार्द्ध्नो᳚त् । प्र॒जाप॑ति॒रिति॑ प्र॒जा - प॒तिः॒ । आर्द्ध्नो॒द् द्वाद॑श । द्वाद॑श॒ रात्रीः᳚ । रात्री᳚र् दीक्षि॒तः । दी॒क्षि॒तः स्या᳚त् । 
स्या॒द् द्वाद॑श । द्वाद॑श॒ मासाः᳚ । मासाः᳚ सम्ॅवथ्स॒रः । स॒म्ॅव॒थ्स॒रः स॑म्ॅवथ्स॒रः । स॒म्ॅव॒थ्स॒र इति॑ सम् - व॒थ्स॒रः । स॒म्ॅव॒थ्स॒रः प्र॒जाप॑तिः । स॒म्ॅव॒थ्स॒र इति॑ सम् - व॒थ्स॒रः । प्र॒जाप॑तिः प्र॒जाप॑तिः । प्र॒जाप॑ति॒रिति॑ प्र॒जा - प॒तिः॒ । प्र॒जाप॑ति॒र् वाव । प्र॒जाप॑ति॒रिति॑ प्र॒जा - प॒तिः॒ । वावैषः । ए॒ष ए॒षः । ए॒ष ह॑ । ह॒ तु । त्वै । वै जा॑यते । जा॒य॒ते॒ यः । यस्तप॑सः । तप॒सोऽधि॑ । अधि॒ जाय॑ते । जाय॑ते चतु॒र्द्धा । च॒तु॒र्द्धा वै । च॒तु॒र्द्धेति॑ चतुः - धा । वा ए॒ताः । ए॒तास्ति॒स्रस्ति॑स्रः । ति॒स्रस्ति॑स्रो॒ रात्र॑यः । ति॒स्रस्ति॑स्र॒ इति॑ ति॒स्रः - ति॒स्रः॒ । रात्र॑यो॒ यत् । यद् द्वाद॑श । द्वाद॑शोप॒सदः॑ । उ॒प॒सदो॒ याः । उ॒प॒सद॒ इत्यु॑प - सदः॑ । याः प्र॑थ॒माः । प्र॒थ॒मा य॒ज्ञ्म् । य॒ज्ञ्म् ताभिः॑ । ताभिः॒ सम् । सम् भ॑रति । भ॒र॒ति॒ याः । या द्वि॒तीयाः᳚ । द्वि॒तीया॑ य॒ज्ञ्म् । य॒ज्ञ्म् ताभिः॑ । ताभि॒रा । आ र॑भते । र॒भ॒ते॒ याः \newline

\textbf{Jatai Paata} \newline

1. मल॑म् जायते जायते॒ मल॒म् मल॑म् जायते । \newline
2. जा॒य॒त॒ एक॒ एको॑ जायते जायत॒ एकः॑ । \newline
3. एक॑ ए॒वै वैक॒ एक॑ ए॒व । \newline
4. ए॒व य॑जेत यजेतै॒वैव य॑जेत । \newline
5. य॒जे॒तैक॒ एको॑ यजेत यजे॒तैकः॑ । \newline
6. एको॒ हि ह्येक॒ एको॒ हि । \newline
7. हि प्र॒जाप॑तिः प्र॒जाप॑ति॒र्॒. हि हि प्र॒जाप॑तिः । \newline
8. प्र॒जाप॑ति॒ रार्द्ध्नो॒ दार्द्ध्नो᳚त् प्र॒जाप॑तिः प्र॒जाप॑ति॒ रार्द्ध्नो᳚त् । \newline
9. प्र॒जाप॑ति॒रिति॑ प्र॒जा - प॒तिः॒ । \newline
10. आर्द्ध्नो॒द् द्वाद॑श॒ द्वाद॒शा र्द्ध्नो॒ दार्द्ध्नो॒द् द्वाद॑श । \newline
11. द्वाद॑श॒ रात्री॒ रात्री॒र् द्वाद॑श॒ द्वाद॑श॒ रात्रीः᳚ । \newline
12. रात्री᳚र् दीक्षि॒तो दी᳚क्षि॒तो रात्री॒ रात्री᳚र् दीक्षि॒तः । \newline
13. दी॒क्षि॒तः स्या᳚थ् स्याद् दीक्षि॒तो दी᳚क्षि॒तः स्या᳚त् । \newline
14. स्या॒द् द्वाद॑श॒ द्वाद॑श स्याथ् स्या॒द् द्वाद॑श । \newline
15. द्वाद॑श॒ मासा॒ मासा॒ द्वाद॑श॒ द्वाद॑श॒ मासाः᳚ । \newline
16. मासाः᳚ संॅवथ्स॒रः सं॑ॅवथ्स॒रो मासा॒ मासाः᳚ संॅवथ्स॒रः । \newline
17. सं॒ॅव॒थ्स॒रः सं॑ॅवथ्स॒रः । \newline
18. सं॒ॅव॒थ्स॒र इति॑ सं - व॒थ्स॒रः । \newline
19. सं॒ॅव॒थ्स॒रः प्र॒जाप॑तिः प्र॒जाप॑तिः संॅवथ्स॒रः सं॑ॅवथ्स॒रः प्र॒जाप॑तिः । \newline
20. सं॒ॅव॒थ्स॒र इति॑ सं - व॒थ्स॒रः । \newline
21. प्र॒जाप॑तिः प्र॒जाप॑तिः । \newline
22. प्र॒जाप॑ति॒रिति॑ प्र॒जा - प॒तिः॒ । \newline
23. प्र॒जाप॑ति॒र् वाव वाव प्र॒जाप॑तिः प्र॒जाप॑ति॒र् वाव । \newline
24. प्र॒जाप॑ति॒रिति॑ प्र॒जा - प॒तिः॒ । \newline
25. वावैष ए॒ष वाव वावैषः । \newline
26. ए॒ष ए॒षः । \newline
27. ए॒ष ह॑ है॒ष ए॒ष ह॑ । \newline
28. ह॒ तु तु ह॑ ह॒ तु । \newline
29. त्वै वै तु त्वै । \newline
30. वै जा॑यते जायते॒ वै वै जा॑यते । \newline
31. जा॒य॒ते॒ यो यो जा॑यते जायते॒ यः । \newline
32. यस्तप॑स॒ स्तप॑सो॒ यो यस्तप॑सः । \newline
33. तप॒सो ऽध्यधि॒ तप॑स॒ स्तप॒सो ऽधि॑ । \newline
34. अधि॒ जाय॑ते॒ जाय॒ते ऽध्यधि॒ जाय॑ते । \newline
35. जाय॑ते चतु॒र्द्धा च॑तु॒र्द्धा जाय॑ते॒ जाय॑ते चतु॒र्द्धा । \newline
36. च॒तु॒र्द्धा वै वै च॑तु॒र्द्धा च॑तु॒र्द्धा वै । \newline
37. च॒तु॒र्द्धेति॑ चतुः - धा । \newline
38. वा ए॒ता ए॒ता वै वा ए॒ताः । \newline
39. ए॒ता स्ति॒स्रस्ति॑स्र स्ति॒स्रस्ति॑स्र ए॒ता ए॒ता स्ति॒स्रस्ति॑स्रः । \newline
40. ति॒स्रस्ति॑स्रो॒ रात्र॑यो॒ रात्र॑य स्ति॒स्रस्ति॑स्र स्ति॒स्रस्ति॑स्रो॒ रात्र॑यः । \newline
41. ति॒स्रस्ति॑स्र॒ इति॑ ति॒स्रः - ति॒स्रः॒ । \newline
42. रात्र॑यो॒ यद् यद् रात्र॑यो॒ रात्र॑यो॒ यत् । \newline
43. यद् द्वाद॑श॒ द्वाद॑श॒ यद् यद् द्वाद॑श । \newline
44. द्वाद॑शोप॒सद॑ उप॒सदो॒ द्वाद॑श॒ द्वाद॑शोप॒सदः॑ । \newline
45. उ॒प॒सदो॒ या या उ॑प॒सद॑ उप॒सदो॒ याः । \newline
46. उ॒प॒सद॒ इत्यु॑प - सदः॑ । \newline
47. याः प्र॑थ॒माः प्र॑थ॒मा या याः प्र॑थ॒माः । \newline
48. प्र॒थ॒मा य॒ज्ञ्ं ॅय॒ज्ञ्म् प्र॑थ॒माः प्र॑थ॒मा य॒ज्ञ्म् । \newline
49. य॒ज्ञ्म् ताभि॒ स्ताभि॑र् य॒ज्ञ्ं ॅय॒ज्ञ्म् ताभिः॑ । \newline
50. ताभिः॒ सꣳ सम् ताभि॒ स्ताभिः॒ सम् । \newline
51. सम् भ॑रति भरति॒ सꣳ सम् भ॑रति । \newline
52. भ॒र॒ति॒ या या भ॑रति भरति॒ याः । \newline
53. या द्वि॒तीया᳚ द्वि॒तीया॒ या या द्वि॒तीयाः᳚ । \newline
54. द्वि॒तीया॑ य॒ज्ञ्ं ॅय॒ज्ञ्म् द्वि॒तीया᳚ द्वि॒तीया॑ य॒ज्ञ्म् । \newline
55. य॒ज्ञ्म् ताभि॒ स्ताभि॑र् य॒ज्ञ्ं ॅय॒ज्ञ्म् ताभिः॑ । \newline
56. ताभि॒रा ताभि॒ स्ताभि॒रा । \newline
57. आ र॑भते रभत॒ आ र॑भते । \newline
58. र॒भ॒ते॒ या या र॑भते रभते॒ याः । \newline

\textbf{Ghana Paata } \newline

1. मल॑म् जायते जायते॒ मल॒म् मल॑म् जायत॒ एक॒ एको॑ जायते॒ मल॒म् मल॑म् जायत॒ एकः॑ । \newline
2. जा॒य॒त॒ एक॒ एको॑ जायते जायत॒ एक॑ ए॒वै वैको॑ जायते जायत॒ एक॑ ए॒व । \newline
3. एक॑ ए॒वै वैक॒ एक॑ ए॒व य॑जेत यजे तै॒वैक॒ एक॑ ए॒व य॑जेत । \newline
4. ए॒व य॑जेत यजे तै॒वैव य॑जे॒ तैक॒ एको॑ यजे तै॒वैव य॑जे॒ तैकः॑ । \newline
5. य॒जे॒ तैक॒ एको॑ यजेत यजे॒ तैको॒ हि ह्येको॑ यजेत यजे॒ तैको॒ हि । \newline
6. एको॒ हि ह्येक॒ एको॒ हि प्र॒जाप॑तिः प्र॒जाप॑ति॒र् ह्येक॒ एको॒ हि प्र॒जाप॑तिः । \newline
7. हि प्र॒जाप॑तिः प्र॒जाप॑ति॒र्॒. हि हि प्र॒जाप॑ति॒ रार्द्ध्नो॒ दार्द्ध्नो᳚त् प्र॒जाप॑ति॒र्॒. हि हि प्र॒जाप॑ति॒ रार्द्ध्नो᳚त् । \newline
8. प्र॒जाप॑ति॒ रार्द्ध्नो॒ दार्द्ध्नो᳚त् प्र॒जाप॑तिः प्र॒जाप॑ति॒ रार्द्ध्नो॒द् द्वाद॑श॒ द्वाद॒शा र्द्ध्नो᳚त् प्र॒जाप॑तिः प्र॒जाप॑ति॒ रार्द्ध्नो॒द् द्वाद॑श । \newline
9. प्र॒जाप॑ति॒रिति॑ प्र॒जा - प॒तिः॒ । \newline
10. आर्द्ध्नो॒द् द्वाद॑श॒ द्वाद॒शा र्द्ध्नो॒ दार्द्ध्नो॒द् द्वाद॑श॒ रात्री॒ रात्री॒र् द्वाद॒शा र्द्ध्नो॒ दार्द्ध्नो॒द् द्वाद॑श॒ रात्रीः᳚ । \newline
11. द्वाद॑श॒ रात्री॒ रात्री॒र् द्वाद॑श॒ द्वाद॑श॒ रात्री᳚र् दीक्षि॒तो दी᳚क्षि॒तो रात्री॒र् द्वाद॑श॒ द्वाद॑श॒ रात्री᳚र् दीक्षि॒तः । \newline
12. रात्री᳚र् दीक्षि॒तो दी᳚क्षि॒तो रात्री॒ रात्री᳚र् दीक्षि॒तः स्या᳚थ् स्याद् दीक्षि॒तो रात्री॒ रात्री᳚र् दीक्षि॒तः स्या᳚त् । \newline
13. दी॒क्षि॒तः स्या᳚थ् स्याद् दीक्षि॒तो दी᳚क्षि॒तः स्या॒द् द्वाद॑श॒ द्वाद॑श स्याद् दीक्षि॒तो दी᳚क्षि॒तः स्या॒द् द्वाद॑श । \newline
14. स्या॒द् द्वाद॑श॒ द्वाद॑श स्याथ् स्या॒द् द्वाद॑श॒ मासा॒ मासा॒ द्वाद॑श स्याथ् स्या॒द् द्वाद॑श॒ मासाः᳚ । \newline
15. द्वाद॑श॒ मासा॒ मासा॒ द्वाद॑श॒ द्वाद॑श॒ मासाः᳚ संॅवथ्स॒रः सं॑ॅवथ्स॒रो मासा॒ द्वाद॑श॒ द्वाद॑श॒ मासाः᳚ संॅवथ्स॒रः । \newline
16. मासाः᳚ संॅवथ्स॒रः सं॑ॅवथ्स॒रो मासा॒ मासाः᳚ संॅवथ्स॒रः । \newline
17. सं॒ॅव॒थ्स॒रः सं॑ॅवथ्स॒रः । \newline
18. सं॒ॅव॒थ्स॒र इति॑ सं - व॒थ्स॒रः । \newline
19. सं॒ॅव॒थ्स॒रः प्र॒जाप॑तिः प्र॒जाप॑तिः संॅवथ्स॒रः सं॑ॅवथ्स॒रः प्र॒जाप॑तिः । \newline
20. सं॒ॅव॒थ्स॒र इति॑ सं - व॒थ्स॒रः । \newline
21. प्र॒जाप॑तिः प्र॒जाप॑तिः । \newline
22. प्र॒जाप॑ति॒रिति॑ प्र॒जा - प॒तिः॒ । \newline
23. प्र॒जाप॑ति॒र् वाव वाव प्र॒जाप॑तिः प्र॒जाप॑ति॒र् वावैष ए॒ष वाव प्र॒जाप॑तिः प्र॒जाप॑ति॒र् वावैषः । \newline
24. प्र॒जाप॑ति॒रिति॑ प्र॒जा - प॒तिः॒ । \newline
25. वावैष ए॒ष वाव वावैषः । \newline
26. ए॒ष ए॒षः । \newline
27. ए॒ष ह॑ है॒ष ए॒ष ह॒ तु तु है॒ष ए॒ष ह॒ तु । \newline
28. ह॒ तु तु ह॑ ह॒त्वै वै तु ह॑ ह॒त्वै । \newline
29. त्वै वै तु त्वै जा॑यते जायते॒ वै तु त्वै जा॑यते । \newline
30. वै जा॑यते जायते॒ वै वै जा॑यते॒ यो यो जा॑यते॒ वै वै जा॑यते॒ यः । \newline
31. जा॒य॒ते॒ यो यो जा॑यते जायते॒ यस्तप॑स॒ स्तप॑सो॒ यो जा॑यते जायते॒ यस्तप॑सः । \newline
32. यस्तप॑स॒ स्तप॑सो॒ यो यस्तप॒सो ऽध्यधि॒ तप॑सो॒ यो यस्तप॒सो ऽधि॑ । \newline
33. तप॒सो ऽध्यधि॒ तप॑स॒ स्तप॒सो ऽधि॒ जाय॑ते॒ जाय॒ते ऽधि॒ तप॑स॒ स्तप॒सो ऽधि॒ जाय॑ते । \newline
34. अधि॒ जाय॑ते॒ जाय॒ते ऽध्यधि॒ जाय॑ते चतु॒र्द्धा च॑तु॒र्द्धा जाय॒ते ऽध्यधि॒ जाय॑ते चतु॒र्द्धा । \newline
35. जाय॑ते चतु॒र्द्धा च॑तु॒र्द्धा जाय॑ते॒ जाय॑ते चतु॒र्द्धा वै वै च॑तु॒र्द्धा जाय॑ते॒ जाय॑ते चतु॒र्द्धा वै । \newline
36. च॒तु॒र्द्धा वै वै च॑तु॒र्द्धा च॑तु॒र्द्धा वा ए॒ता ए॒ता वै च॑तु॒र्द्धा च॑तु॒र्द्धा वा ए॒ताः । \newline
37. च॒तु॒र्द्धेति॑ चतुः - धा । \newline
38. वा ए॒ता ए॒ता वै वा ए॒ता स्ति॒स्रस्ति॑स्र स्ति॒स्रस्ति॑स्र ए॒ता वै वा ए॒ता स्ति॒स्रस्ति॑स्रः । \newline
39. ए॒ता स्ति॒स्रस्ति॑स्र स्ति॒स्रस्ति॑स्र ए॒ता ए॒ता स्ति॒स्रस्ति॑स्रो॒ रात्र॑यो॒ रात्र॑य स्ति॒स्रस्ति॑स्र ए॒ता ए॒ता स्ति॒स्रस्ति॑स्रो॒ रात्र॑यः । \newline
40. ति॒स्रस्ति॑स्रो॒ रात्र॑यो॒ रात्र॑य स्ति॒स्रस्ति॑स्र स्ति॒स्रस्ति॑स्रो॒ रात्र॑यो॒ यद् यद् रात्र॑य स्ति॒स्रस्ति॑स्र स्ति॒स्रस्ति॑स्रो॒ रात्र॑यो॒ यत् । \newline
41. ति॒स्रस्ति॑स्र॒ इति॑ ति॒स्रः - ति॒स्रः॒ । \newline
42. रात्र॑यो॒ यद् यद् रात्र॑यो॒ रात्र॑यो॒ यद् द्वाद॑श॒ द्वाद॑श॒ यद् रात्र॑यो॒ रात्र॑यो॒ यद् द्वाद॑श । \newline
43. यद् द्वाद॑श॒ द्वाद॑श॒ यद् यद् द्वाद॑शोप॒सद॑ उप॒सदो॒ द्वाद॑श॒ यद् यद् द्वाद॑शोप॒सदः॑ । \newline
44. द्वाद॑शोप॒सद॑ उप॒सदो॒ द्वाद॑श॒ द्वाद॑शोप॒सदो॒ या या उ॑प॒सदो॒ द्वाद॑श॒ द्वाद॑शोप॒सदो॒ याः । \newline
45. उ॒प॒सदो॒ या या उ॑प॒सद॑ उप॒सदो॒ याः प्र॑थ॒माः प्र॑थ॒मा या उ॑प॒सद॑ उप॒सदो॒ याः प्र॑थ॒माः । \newline
46. उ॒प॒सद॒ इत्यु॑प - सदः॑ । \newline
47. याः प्र॑थ॒माः प्र॑थ॒मा या याः प्र॑थ॒मा य॒ज्ञ्ं ॅय॒ज्ञ्म् प्र॑थ॒मा या याः प्र॑थ॒मा य॒ज्ञ्म् । \newline
48. प्र॒थ॒मा य॒ज्ञ्ं ॅय॒ज्ञ्म् प्र॑थ॒माः प्र॑थ॒मा य॒ज्ञ्म् ताभि॒ स्ताभि॑र् य॒ज्ञ्म् प्र॑थ॒माः प्र॑थ॒मा य॒ज्ञ्म् ताभिः॑ । \newline
49. य॒ज्ञ्म् ताभि॒ स्ताभि॑र् य॒ज्ञ्ं ॅय॒ज्ञ्म् ताभिः॒ सꣳ सम् ताभि॑र् य॒ज्ञ्ं ॅय॒ज्ञ्म् ताभिः॒ सम् । \newline
50. ताभिः॒ सꣳ सम् ताभि॒ स्ताभिः॒ सम् भ॑रति भरति॒ सम् ताभि॒ स्ताभिः॒ सम् भ॑रति । \newline
51. सम् भ॑रति भरति॒ सꣳ सम् भ॑रति॒ या या भ॑रति॒ सꣳ सम् भ॑रति॒ याः । \newline
52. भ॒र॒ति॒ या या भ॑रति भरति॒ या द्वि॒तीया᳚ द्वि॒तीया॒ या भ॑रति भरति॒ या द्वि॒तीयाः᳚ । \newline
53. या द्वि॒तीया᳚ द्वि॒तीया॒ या या द्वि॒तीया॑ य॒ज्ञ्ं ॅय॒ज्ञ्म् द्वि॒तीया॒ या या द्वि॒तीया॑ य॒ज्ञ्म् । \newline
54. द्वि॒तीया॑ य॒ज्ञ्ं ॅय॒ज्ञ्म् द्वि॒तीया᳚ द्वि॒तीया॑ य॒ज्ञ्म् ताभि॒ स्ताभि॑र् य॒ज्ञ्म् द्वि॒तीया᳚ द्वि॒तीया॑ य॒ज्ञ्म् ताभिः॑ । \newline
55. य॒ज्ञ्म् ताभि॒ स्ताभि॑र् य॒ज्ञ्ं ॅय॒ज्ञ्म् ताभि॒रा ताभि॑र् य॒ज्ञ्ं ॅय॒ज्ञ्म् ताभि॒रा । \newline
56. ताभि॒रा ताभि॒ स्ताभि॒रा र॑भते रभत॒ आ ताभि॒ स्ताभि॒रा र॑भते । \newline
57. आ र॑भते रभत॒ आ र॑भते॒ या या र॑भत॒ आ र॑भते॒ याः । \newline
58. र॒भ॒ते॒ या या र॑भते रभते॒ यास्तृ॒तीया᳚ स्तृ॒तीया॒ या र॑भते रभते॒ यास्तृ॒तीयाः᳚ । \newline
\pagebreak
\markright{ TS 7.2.10.4  \hfill https://www.vedavms.in \hfill}

\section{ TS 7.2.10.4 }

\textbf{TS 7.2.10.4 } \newline
\textbf{Samhita Paata} \newline

यास्तृ॒तीयाः॒ पात्रा॑णि॒ ताभि॒र्निर्णे॑निक्ते॒ याश्च॑तु॒र्थीरपि॒ ताभि॑रा॒त्मान॑मन्तर॒तः शु॑न्धते॒ यो वा अ॑स्य प॒शुमत्ति॑ माꣳ॒॒सꣳ सो᳚ऽत्ति॒ यः पु॑रो॒डाशं॑ म॒स्तिष्कꣳ॒॒ स यः प॑रिवा॒पं पुरी॑षꣳ॒॒ स य आज्यं॑ म॒ज्जानꣳ॒॒ स यः सोमꣳ॒॒ स्वेदꣳ॒॒ सोऽपि॑ ह॒ वा अ॑स्य शीर्.ष॒ण्या॑ नि॒ष्पदः॒ प्रति॑ गृह्णाति॒ यो द्वा॑दशा॒हे प्र॑तिगृ॒ह्णाति॒ तस्मा᳚द् द्वादशा॒हेन॒ ( ) न याज्यं॑ पा॒प्मनो॒ व्यावृ॑त्त्यै ॥ \newline

\textbf{Pada Paata} \newline

याः । तृ॒तीयाः᳚ । पात्रा॑णि । ताभिः॑ । निरिति॑ । ने॒नि॒क्ते॒ । याः । च॒तु॒र्थीः । अपीति॑ । ताभिः॑ । आ॒त्मान᳚म् । अ॒न्त॒र॒तः । शु॒न्ध॒ते॒ । यः । वै । अ॒स्य॒ । प॒शुम् । अत्ति॑ । माꣳ॒॒सम् । सः । अ॒त्ति॒ । यः । पु॒रो॒डाश᳚म् । म॒स्तिष्क᳚म् । सः । यः । प॒रि॒वा॒पमिति॑ परि - वा॒पम् । पुरी॑षम् । सः । यः । आज्य᳚म् । म॒ज्जान᳚म् । सः । यः । सोम᳚म् । स्वेद᳚म् । सः । अपीति॑ । ह॒ । वै । अ॒स्य॒ । शी॒र्.॒ष॒ण्याः᳚ । नि॒ष्पद॒ इति॑ निः - पदः॑ । प्रतीति॑ । गृ॒ह्णा॒ति॒ । यः । द्वा॒द॒शा॒ह इति॑ द्वादश-अ॒हे । प्र॒ति॒गृ॒ह्णातीति॑ प्रति - गृ॒ह्णाति॑ । तस्मा᳚त् । द्वा॒द॒शा॒हेनेति॑ द्वादश - अ॒हेन॑ ( ) । न । याज्य᳚म् । पा॒प्मनः॑ । व्यावृ॑त्त्या॒ इति॑ वि - आवृ॑त्त्यै ॥  \newline


\textbf{Krama Paata} \newline

यास्तृ॒तीयाः᳚ । तृ॒तीयाः॒ पात्रा॑णि । पात्रा॑णि॒ ताभिः॑ । ताभि॒र् निः । निर् णेनिक्ते । ने॒नि॒क्ते॒ याः । याश्च॑तु॒र्थीः । च॒तु॒र्थीरपि॑ । अपि॒ ताभिः॑ । ताभि॑रा॒त्मान᳚म् । आ॒त्मान॑मन्तर॒तः । अ॒न्त॒र॒तः शु॑न्धते । शु॒न्ध॒ते॒ यः । यो वै । वा अ॑स्य । अ॒स्य॒ प॒शुम् । प॒शुमत्ति॑ । अत्ति॑ माꣳ॒॒सम् । माꣳ॒॒सꣳ सः । सो᳚ऽत्ति । अ॒त्ति॒ यः । यः पु॑रो॒डाश᳚म् । पु॒रो॒डाश॑म् म॒स्तिष्क᳚म् । म॒स्तिष्कꣳ॒॒ सः । स यः । यः प॑रिवा॒पम् । प॒रि॒वा॒पम् पुरी॑षम् । प॒रि॒वा॒पमिति॑ परि - वा॒पम् । पुरी॑षꣳ॒॒ सः । स यः । य आज्य᳚म् । आज्य॑म् म॒ज्जान᳚म् । म॒ज्जानꣳ॒॒ सः । स यः । यः सोम᳚म् । सोमꣳ॒॒ स्वेद᳚म् । स्वेदꣳ॒॒ सः । सोऽपि॑ । अपि॑ ह । ह॒ वै । वा अ॑स्य । अ॒स्य॒ शी॒र्.॒ष॒ण्याः᳚ । शी॒र्॒.ष॒ण्या॑ नि॒ष्पदः॑ । नि॒ष्पद॒ प्रति॑ । नि॒ष्पद॒ इति॑ निः - पदः॑ । प्रति॑ गृह्णाति । गृ॒ह्णा॒ति॒ यः । यो द्वा॑दशा॒हे । द्वा॒द॒शा॒हे प्र॑तिगृ॒ह्णाति॑ । द्वा॒द॒शा॒ह इति॑ द्वादश - अ॒हे । प्र॒ति॒गृ॒ह्णाति॒ तस्मा᳚त् । प्र॒ति॒गृ॒ह्णातीति॑ प्रति - गृ॒ह्णाति॑ । तस्मा᳚द् द्वादशा॒हेन॑ ( ) । द्वा॒द॒शा॒हेन॒ न । द्वा॒द॒शा॒हेनेति॑ द्वादश - अ॒हेन॑ । न याज्य᳚म् । याज्य॑म् पा॒प्मनः॑ । पा॒प्मनो॒ व्यावृ॑त्त्यै । व्यावृ॑त्त्या॒ इति॑ वि - आवृ॑त्त्यै । \newline

\textbf{Jatai Paata} \newline

1. यास्तृ॒तीया᳚ स्तृ॒तीया॒ या यास्तृ॒तीयाः᳚ । \newline
2. तृ॒तीयाः॒ पात्रा॑णि॒ पात्रा॑णि तृ॒तीया᳚ स्तृ॒तीयाः॒ पात्रा॑णि । \newline
3. पात्रा॑णि॒ ताभि॒ स्ताभिः॒ पात्रा॑णि॒ पात्रा॑णि॒ ताभिः॑ । \newline
4. ताभि॒र् निर् णिष् टाभि॒ स्ताभि॒र् निः । \newline
5. निर् णे॑निक्ते नेनिक्ते॒ निर् णिर् णे॑निक्ते । \newline
6. ने॒नि॒क्ते॒ या या ने॑निक्ते नेनिक्ते॒ याः । \newline
7. याश्च॑तु॒र्थी श्च॑तु॒र्थीर् या याश्च॑तु॒र्थीः । \newline
8. च॒तु॒र्थी रप्यपि॑ चतु॒र्थी श्च॑तु॒र्थी रपि॑ । \newline
9. अपि॒ ताभि॒ स्ताभि॒ रप्यपि॒ ताभिः॑ । \newline
10. ताभि॑ रा॒त्मान॑ मा॒त्मान॒म् ताभि॒ स्ताभि॑ रा॒त्मान᳚म् । \newline
11. आ॒त्मान॑ मन्तर॒तो᳚ ऽन्तर॒त आ॒त्मान॑ मा॒त्मान॑ मन्तर॒तः । \newline
12. अ॒न्त॒र॒तः शु॑न्धते शुन्धते ऽन्तर॒तो᳚ ऽन्तर॒तः शु॑न्धते । \newline
13. शु॒न्ध॒ते॒ यो यः शु॑न्धते शुन्धते॒ यः । \newline
14. यो वै वै यो यो वै । \newline
15. वा अ॑स्यास्य॒ वै वा अ॑स्य । \newline
16. अ॒स्य॒ प॒शुम् प॒शु म॑स्यास्य प॒शुम् । \newline
17. प॒शु मत्त्यत्ति॑ प॒शुम् प॒शु मत्ति॑ । \newline
18. अत्ति॑ माꣳ॒॒सम् माꣳ॒॒स मत्त्यत्ति॑ माꣳ॒॒सम् । \newline
19. माꣳ॒॒सꣳ स स माꣳ॒॒सम् माꣳ॒॒सꣳ सः । \newline
20. सो᳚ ऽत्त्यत्ति॒ स सो᳚ ऽत्ति । \newline
21. अ॒त्ति॒ यो यो᳚ ऽत्त्यत्ति॒ यः । \newline
22. यः पु॑रो॒डाश॑म् पुरो॒डाशं॒ ॅयो यः पु॑रो॒डाश᳚म् । \newline
23. पु॒रो॒डाश॑म् म॒स्तिष्क॑म् म॒स्तिष्क॑म् पुरो॒डाश॑म् पुरो॒डाश॑म् म॒स्तिष्क᳚म् । \newline
24. म॒स्तिष्कꣳ॒॒ स स म॒स्तिष्क॑म् म॒स्तिष्कꣳ॒॒ सः । \newline
25. स यो यः स स यः । \newline
26. यः प॑रिवा॒पम् प॑रिवा॒पं ॅयो यः प॑रिवा॒पम् । \newline
27. प॒रि॒वा॒पम् पुरी॑ष॒म् पुरी॑षम् परिवा॒पम् प॑रिवा॒पम् पुरी॑षम् । \newline
28. प॒रि॒वा॒पमिति॑ परि - वा॒पम् । \newline
29. पुरी॑षꣳ॒॒ स स पुरी॑ष॒म् पुरी॑षꣳ॒॒ सः । \newline
30. स यो यः स स यः । \newline
31. य आज्य॒ माज्यं॒ ॅयो य आज्य᳚म् । \newline
32. आज्य॑म् म॒ज्जान॑म् म॒ज्जान॒ माज्य॒ माज्य॑म् म॒ज्जान᳚म् । \newline
33. म॒ज्जानꣳ॒॒ स स म॒ज्जान॑म् म॒ज्जानꣳ॒॒ सः । \newline
34. स यो यः स स यः । \newline
35. यः सोमꣳ॒॒ सोमं॒ ॅयो यः सोम᳚म् । \newline
36. सोमꣳ॒॒ स्वेदꣳ॒॒ स्वेदꣳ॒॒ सोमꣳ॒॒ सोमꣳ॒॒ स्वेद᳚म् । \newline
37. स्वेदꣳ॒॒ स स स्वेदꣳ॒॒ स्वेदꣳ॒॒ सः । \newline
38. सो ऽप्यपि॒ स सो ऽपि॑ । \newline
39. अपि॑ ह॒ हाप्यपि॑ ह । \newline
40. ह॒ वै वै ह॑ ह॒ वै । \newline
41. वा अ॑स्यास्य॒ वै वा अ॑स्य । \newline
42. अ॒स्य॒ शी॒र्॒.ष॒ण्याः᳚ शीर्.ष॒ण्या॑ अस्यास्य शीर्.ष॒ण्याः᳚ । \newline
43. शी॒र्॒.ष॒ण्या॑ नि॒ष्पदो॑ नि॒ष्पदः॑ शीर्.ष॒ण्याः᳚ शीर्.ष॒ण्या॑ नि॒ष्पदः॑ । \newline
44. नि॒ष्पदः॒ प्रति॒ प्रति॑ नि॒ष्पदो॑ नि॒ष्पदः॒ प्रति॑ । \newline
45. नि॒ष्पद॒ इति॑ निः - पदः॑ । \newline
46. प्रति॑ गृह्णाति गृह्णाति॒ प्रति॒ प्रति॑ गृह्णाति । \newline
47. गृ॒ह्णा॒ति॒ यो यो गृ॑ह्णाति गृह्णाति॒ यः । \newline
48. यो द्वा॑दशा॒हे द्वा॑दशा॒हे यो यो द्वा॑दशा॒हे । \newline
49. द्वा॒द॒शा॒हे प्र॑तिगृ॒ह्णाति॑ प्रतिगृ॒ह्णाति॑ द्वादशा॒हे द्वा॑दशा॒हे प्र॑तिगृ॒ह्णाति॑ । \newline
50. द्वा॒द॒शा॒ह इति॑ द्वादश - अ॒हे । \newline
51. प्र॒ति॒गृ॒ह्णाति॒ तस्मा॒त् तस्मा᳚त् प्रतिगृ॒ह्णाति॑ प्रतिगृ॒ह्णाति॒ तस्मा᳚त् । \newline
52. प्र॒ति॒गृ॒ह्णातीति॑ प्रति - गृ॒ह्णाति॑ । \newline
53. तस्मा᳚द् द्वादशा॒हेन॑ द्वादशा॒हेन॒ तस्मा॒त् तस्मा᳚द् द्वादशा॒हेन॑ । \newline
54. द्वा॒द॒शा॒हेन॒ न न द्वा॑दशा॒हेन॑ द्वादशा॒हेन॒ न । \newline
55. द्वा॒द॒शा॒हेनेति॑ द्वादश - अ॒हेन॑ । \newline
56. न याज्यं॒ ॅयाज्य॒न् न न याज्य᳚म् । \newline
57. याज्य॑म् पा॒प्मनः॑ पा॒प्मनो॒ याज्यं॒ ॅयाज्य॑म् पा॒प्मनः॑ । \newline
58. पा॒प्मनो॒ व्यावृ॑त्त्यै॒ व्यावृ॑त्त्यै पा॒प्मनः॑ पा॒प्मनो॒ व्यावृ॑त्त्यै । \newline
59. व्यावृ॑त्त्या॒ इति॑ वि - आवृ॑त्त्यै । \newline

\textbf{Ghana Paata } \newline

1. यास्तृ॒तीया᳚ स्तृ॒तीया॒ या यास्तृ॒तीयाः॒ पात्रा॑णि॒ पात्रा॑णि तृ॒तीया॒ या यास्तृ॒तीयाः॒ पात्रा॑णि । \newline
2. तृ॒तीयाः॒ पात्रा॑णि॒ पात्रा॑णि तृ॒तीया᳚ स्तृ॒तीयाः॒ पात्रा॑णि॒ ताभि॒ स्ताभिः॒ पात्रा॑णि तृ॒तीया᳚ स्तृ॒तीयाः॒ पात्रा॑णि॒ ताभिः॑ । \newline
3. पात्रा॑णि॒ ताभि॒ स्ताभिः॒ पात्रा॑णि॒ पात्रा॑णि॒ ताभि॒र् निर् णिष् टाभिः॒ पात्रा॑णि॒ पात्रा॑णि॒ ताभि॒र् निः । \newline
4. ताभि॒र् निर् णिष् टाभि॒ स्ताभि॒र् निर् णे॑निक्ते नेनिक्ते॒ निष् टाभि॒ स्ताभि॒र् निर् णे॑निक्ते । \newline
5. निर् णे॑निक्ते नेनिक्ते॒ निर् णिर् णे॑निक्ते॒ या या ने॑निक्ते॒ निर् णिर् णे॑निक्ते॒ याः । \newline
6. ने॒नि॒क्ते॒ या या ने॑निक्ते नेनिक्ते॒ याश्च॑तु॒र्थी श्च॑तु॒र्थीर् या ने॑निक्ते नेनिक्ते॒ याश्च॑तु॒र्थीः । \newline
7. याश्च॑तु॒र्थी श्च॑तु॒र्थीर् या याश्च॑तु॒र्थी रप्यपि॑ चतु॒र्थीर् या याश्च॑तु॒र्थी रपि॑ । \newline
8. च॒तु॒र्थी रप्यपि॑ चतु॒र्थी श्च॑तु॒र्थी रपि॒ ताभि॒ स्ताभि॒ रपि॑ चतु॒र्थी श्च॑तु॒र्थी रपि॒ ताभिः॑ । \newline
9. अपि॒ ताभि॒ स्ताभि॒ रप्यपि॒ ताभि॑ रा॒त्मान॑ मा॒त्मान॒म् ताभि॒ रप्यपि॒ ताभि॑ रा॒त्मान᳚म् । \newline
10. ताभि॑ रा॒त्मान॑ मा॒त्मान॒म् ताभि॒ स्ताभि॑ रा॒त्मान॑ मन्तर॒तो᳚ ऽन्तर॒त आ॒त्मान॒म् ताभि॒ स्ताभि॑ रा॒त्मान॑ मन्तर॒तः । \newline
11. आ॒त्मान॑ मन्तर॒तो᳚ ऽन्तर॒त आ॒त्मान॑ मा॒त्मान॑ मन्तर॒तः शु॑न्धते शुन्धते ऽन्तर॒त आ॒त्मान॑ मा॒त्मान॑ मन्तर॒तः शु॑न्धते । \newline
12. अ॒न्त॒र॒तः शु॑न्धते शुन्धते ऽन्तर॒तो᳚ ऽन्तर॒तः शु॑न्धते॒ यो यः शु॑न्धते ऽन्तर॒तो᳚ ऽन्तर॒तः शु॑न्धते॒ यः । \newline
13. शु॒न्ध॒ते॒ यो यः शु॑न्धते शुन्धते॒ यो वै वै यः शु॑न्धते शुन्धते॒ यो वै । \newline
14. यो वै वै यो यो वा अ॑स्यास्य॒ वै यो यो वा अ॑स्य । \newline
15. वा अ॑स्यास्य॒ वै वा अ॑स्य प॒शुम् प॒शु म॑स्य॒ वै वा अ॑स्य प॒शुम् । \newline
16. अ॒स्य॒ प॒शुम् प॒शु म॑स्यास्य प॒शु मत्त्यत्ति॑ प॒शु म॑स्यास्य प॒शु मत्ति॑ । \newline
17. प॒शु मत्त्यत्ति॑ प॒शुम् प॒शु मत्ति॑ माꣳ॒॒सम् माꣳ॒॒स मत्ति॑ प॒शुम् प॒शु मत्ति॑ माꣳ॒॒सम् । \newline
18. अत्ति॑ माꣳ॒॒सम् माꣳ॒॒स मत्त्यत्ति॑ माꣳ॒॒सꣳ स स माꣳ॒॒स मत्त्यत्ति॑ माꣳ॒॒सꣳ सः । \newline
19. माꣳ॒॒सꣳ स स माꣳ॒॒सम् माꣳ॒॒सꣳ सो᳚ ऽत्त्यत्ति॒ स माꣳ॒॒सम् माꣳ॒॒सꣳ सो᳚ ऽत्ति । \newline
20. सो᳚ ऽत्त्यत्ति॒ स सो᳚ ऽत्ति॒ यो यो᳚ ऽत्ति॒ स सो᳚ ऽत्ति॒ यः । \newline
21. अ॒त्ति॒ यो यो᳚ ऽत्त्यत्ति॒ यः पु॑रो॒डाश॑म् पुरो॒डाशं॒ ॅयो᳚ ऽत्त्यत्ति॒ यः पु॑रो॒डाश᳚म् । \newline
22. यः पु॑रो॒डाश॑म् पुरो॒डाशं॒ ॅयो यः पु॑रो॒डाश॑म् म॒स्तिष्क॑म् म॒स्तिष्क॑म् पुरो॒डाशं॒ ॅयो यः पु॑रो॒डाश॑म् म॒स्तिष्क᳚म् । \newline
23. पु॒रो॒डाश॑म् म॒स्तिष्क॑म् म॒स्तिष्क॑म् पुरो॒डाश॑म् पुरो॒डाश॑म् म॒स्तिष्कꣳ॒॒ स स म॒स्तिष्क॑म् पुरो॒डाश॑म् पुरो॒डाश॑म् म॒स्तिष्कꣳ॒॒ सः । \newline
24. म॒स्तिष्कꣳ॒॒ स स म॒स्तिष्क॑म् म॒स्तिष्कꣳ॒॒ स यो यः स म॒स्तिष्क॑म् म॒स्तिष्कꣳ॒॒ स यः । \newline
25. स यो यः स स यः प॑रिवा॒पम् प॑रिवा॒पं ॅयः स स यः प॑रिवा॒पम् । \newline
26. यः प॑रिवा॒पम् प॑रिवा॒पं ॅयो यः प॑रिवा॒पम् पुरी॑ष॒म् पुरी॑षम् परिवा॒पं ॅयो यः प॑रिवा॒पम् पुरी॑षम् । \newline
27. प॒रि॒वा॒पम् पुरी॑ष॒म् पुरी॑षम् परिवा॒पम् प॑रिवा॒पम् पुरी॑षꣳ॒॒ स स पुरी॑षम् परिवा॒पम् प॑रिवा॒पम् पुरी॑षꣳ॒॒ सः । \newline
28. प॒रि॒वा॒पमिति॑ परि - वा॒पम् । \newline
29. पुरी॑षꣳ॒॒ स स पुरी॑ष॒म् पुरी॑षꣳ॒॒ स यो यः स पुरी॑ष॒म् पुरी॑षꣳ॒॒ स यः । \newline
30. स यो यः स स य आज्य॒ माज्यं॒ ॅयः स स य आज्य᳚म् । \newline
31. य आज्य॒ माज्यं॒ ॅयो य आज्य॑म् म॒ज्जान॑म् म॒ज्जान॒ माज्यं॒ ॅयो य आज्य॑म् म॒ज्जान᳚म् । \newline
32. आज्य॑म् म॒ज्जान॑म् म॒ज्जान॒ माज्य॒ माज्य॑म् म॒ज्जानꣳ॒॒ स स म॒ज्जान॒ माज्य॒ माज्य॑म् म॒ज्जानꣳ॒॒ सः । \newline
33. म॒ज्जानꣳ॒॒ स स म॒ज्जान॑म् म॒ज्जानꣳ॒॒ स यो यः स म॒ज्जान॑म् म॒ज्जानꣳ॒॒ स यः । \newline
34. स यो यः स स यः सोमꣳ॒॒ सोमं॒ ॅयः स स यः सोम᳚म् । \newline
35. यः सोमꣳ॒॒ सोमं॒ ॅयो यः सोमꣳ॒॒ स्वेदꣳ॒॒ स्वेदꣳ॒॒ सोमं॒ ॅयो यः सोमꣳ॒॒ स्वेद᳚म् । \newline
36. सोमꣳ॒॒ स्वेदꣳ॒॒ स्वेदꣳ॒॒ सोमꣳ॒॒ सोमꣳ॒॒ स्वेदꣳ॒॒ स स स्वेदꣳ॒॒ सोमꣳ॒॒ सोमꣳ॒॒ स्वेदꣳ॒॒ सः । \newline
37. स्वेदꣳ॒॒ स स स्वेदꣳ॒॒ स्वेदꣳ॒॒ सो ऽप्यपि॒ स स्वेदꣳ॒॒ स्वेदꣳ॒॒ सो ऽपि॑ । \newline
38. सो ऽप्यपि॒ स सो ऽपि॑ ह॒ हापि॒ स सो ऽपि॑ ह । \newline
39. अपि॑ ह॒ हाप्यपि॑ ह॒ वै वै हाप्यपि॑ ह॒ वै । \newline
40. ह॒ वै वै ह॑ ह॒ वा अ॑स्यास्य॒ वै ह॑ ह॒ वा अ॑स्य । \newline
41. वा अ॑स्यास्य॒ वै वा अ॑स्य शीर्.ष॒ण्याः᳚ शीर्.ष॒ण्या॑ अस्य॒ वै वा अ॑स्य शीर्.ष॒ण्याः᳚ । \newline
42. अ॒स्य॒ शी॒र्॒.ष॒ण्याः᳚ शीर्.ष॒ण्या॑ अस्यास्य शीर्.ष॒ण्या॑ नि॒ष्पदो॑ नि॒ष्पदः॑ शीर्.ष॒ण्या॑ अस्यास्य शीर्.ष॒ण्या॑ नि॒ष्पदः॑ । \newline
43. शी॒र्॒.ष॒ण्या॑ नि॒ष्पदो॑ नि॒ष्पदः॑ शीर्.ष॒ण्याः᳚ शीर्.ष॒ण्या॑ नि॒ष्पदः॒ प्रति॒ प्रति॑ नि॒ष्पदः॑ शीर्.ष॒ण्याः᳚ शीर्.ष॒ण्या॑ नि॒ष्पदः॒ प्रति॑ । \newline
44. नि॒ष्पदः॒ प्रति॒ प्रति॑ नि॒ष्पदो॑ नि॒ष्पदः॒ प्रति॑ गृह्णाति गृह्णाति॒ प्रति॑ नि॒ष्पदो॑ नि॒ष्पदः॒ प्रति॑ गृह्णाति । \newline
45. नि॒ष्पद॒ इति॑ निः - पदः॑ । \newline
46. प्रति॑ गृह्णाति गृह्णाति॒ प्रति॒ प्रति॑ गृह्णाति॒ यो यो गृ॑ह्णाति॒ प्रति॒ प्रति॑ गृह्णाति॒ यः । \newline
47. गृ॒ह्णा॒ति॒ यो यो गृ॑ह्णाति गृह्णाति॒ यो द्वा॑दशा॒हे द्वा॑दशा॒हे यो गृ॑ह्णाति गृह्णाति॒ यो द्वा॑दशा॒हे । \newline
48. यो द्वा॑दशा॒हे द्वा॑दशा॒हे यो यो द्वा॑दशा॒हे प्र॑तिगृ॒ह्णाति॑ प्रतिगृ॒ह्णाति॑ द्वादशा॒हे यो यो द्वा॑दशा॒हे प्र॑तिगृ॒ह्णाति॑ । \newline
49. द्वा॒द॒शा॒हे प्र॑तिगृ॒ह्णाति॑ प्रतिगृ॒ह्णाति॑ द्वादशा॒हे द्वा॑दशा॒हे प्र॑तिगृ॒ह्णाति॒ तस्मा॒त् तस्मा᳚त् प्रतिगृ॒ह्णाति॑ द्वादशा॒हे द्वा॑दशा॒हे प्र॑तिगृ॒ह्णाति॒ तस्मा᳚त् । \newline
50. द्वा॒द॒शा॒ह इति॑ द्वादश - अ॒हे । \newline
51. प्र॒ति॒गृ॒ह्णाति॒ तस्मा॒त् तस्मा᳚त् प्रतिगृ॒ह्णाति॑ प्रतिगृ॒ह्णाति॒ तस्मा᳚द् द्वादशा॒हेन॑ द्वादशा॒हेन॒ तस्मा᳚त् प्रतिगृ॒ह्णाति॑ प्रतिगृ॒ह्णाति॒ तस्मा᳚द् द्वादशा॒हेन॑ । \newline
52. प्र॒ति॒गृ॒ह्णातीति॑ प्रति - गृ॒ह्णाति॑ । \newline
53. तस्मा᳚द् द्वादशा॒हेन॑ द्वादशा॒हेन॒ तस्मा॒त् तस्मा᳚द् द्वादशा॒हेन॒ न न द्वा॑दशा॒हेन॒ तस्मा॒त् तस्मा᳚द् द्वादशा॒हेन॒ न । \newline
54. द्वा॒द॒शा॒हेन॒ न न द्वा॑दशा॒हेन॑ द्वादशा॒हेन॒ न याज्यं॒ ॅयाज्य॒न् न द्वा॑दशा॒हेन॑ द्वादशा॒हेन॒ न याज्य᳚म् । \newline
55. द्वा॒द॒शा॒हेनेति॑ द्वादश - अ॒हेन॑ । \newline
56. न याज्यं॒ ॅयाज्य॒न् न न याज्य॑म् पा॒प्मनः॑ पा॒प्मनो॒ याज्य॒न् न न याज्य॑म् पा॒प्मनः॑ । \newline
57. याज्य॑म् पा॒प्मनः॑ पा॒प्मनो॒ याज्यं॒ ॅयाज्य॑म् पा॒प्मनो॒ व्यावृ॑त्त्यै॒ व्यावृ॑त्त्यै पा॒प्मनो॒ याज्यं॒ ॅयाज्य॑म् पा॒प्मनो॒ व्यावृ॑त्त्यै । \newline
58. पा॒प्मनो॒ व्यावृ॑त्त्यै॒ व्यावृ॑त्त्यै पा॒प्मनः॑ पा॒प्मनो॒ व्यावृ॑त्त्यै । \newline
59. व्यावृ॑त्त्या॒ इति॑ वि - आवृ॑त्त्यै । \newline
\pagebreak
\markright{ TS 7.2.11.1  \hfill https://www.vedavms.in \hfill}

\section{ TS 7.2.11.1 }

\textbf{TS 7.2.11.1 } \newline
\textbf{Samhita Paata} \newline

एक॑स्मै॒ स्वाहा॒ द्वाभ्याꣳ॒॒ स्वाहा᳚ त्रि॒भ्यः स्वाहा॑ च॒तुर्भ्यः॒ स्वाहा॑ प॒ञ्चभ्यः॒ स्वाहा॑ ष॒ड्भ्यः स्वाहा॑ स॒प्तभ्यः॒ स्वाहा᳚ ऽष्टा॒भ्यः स्वाहा॑ न॒वभ्यः॒ स्वाहा॑ द॒शभ्यः॒ स्वाहै॑ -काद॒शभ्यः॒ स्वाहा᳚ द्वाद॒शभ्यः॒ स्वाहा᳚ त्रयोद॒शभ्यः॒ स्वाहा॑ चतुर्द॒शभ्यः॒ स्वाहा॑ पञ्चद॒शभ्यः॒ स्वाहा॑ षोड॒शभ्यः॒ स्वाहा॑ सप्तद॒शभ्यः॒ स्वाहा᳚ ऽष्टाद॒शभ्यः॒ स्वाहै-का॒न्न विꣳ॑श॒त्यै स्वाहा॒ नव॑विꣳशत्यै॒ स्वाहै-का॒न्न च॑त्वारिꣳ॒॒शते॒ स्वाहा॒ नव॑चत्वारिꣳशते॒ स्वाहै-का॒न्न ( ) ष॒ष्ट्यै स्वाहा॒ नव॑षष्ट्यै॒ स्वाहै -का॒न्नाशी॒त्यै स्वाहा॒ नवा॑शीत्यै॒ स्वाहैका॒न्न श॒ताय॒ स्वाहा॑ श॒ताय॒ स्वाहा॒ द्वाभ्याꣳ॑ श॒ताभ्याꣳ॒॒ स्वाहा॒ सर्व॑स्मै॒ स्वाहा᳚ ॥ \newline

\textbf{Pada Paata} \newline

एक॑स्मै । स्वाहा᳚ । द्वाभ्या᳚म् । स्वाहा᳚ । त्रि॒भ्य इति॑ त्रि - भ्यः । स्वाहा᳚ । च॒तुर्भ्य॒ इति॑ च॒तुः - भ्यः॒ । स्वाहा᳚ । प॒ञ्चभ्य॒ इति॑ प॒ञ्च - भ्यः॒ । स्वाहा᳚ । ष॒ड्भ्य इति॑ षट्- भ्यः । स्वाहा᳚ । स॒प्तभ्य॒ इति॑ स॒प्त - भ्यः॒ । स्वाहा᳚ । अ॒ष्टा॒भ्यः । स्वाहा᳚ । न॒वभ्य॒ इति॑ न॒व - भ्यः॒ । स्वाहा᳚ । द॒शभ्य॒ इति॑ द॒श - भ्यः॒ । स्वाहा᳚ । ए॒का॒द॒शभ्य॒ इत्ये॑काद॒श - भ्यः॒ । स्वाहा᳚ । द्वा॒द॒शभ्य॒ इति॑ द्वाद॒श - भ्यः॒ । स्वाहा᳚ । त्र॒यो॒द॒शभ्य॒ इति॑ त्रयोद॒श-भ्यः॒ । स्वाहा᳚ । च॒तु॒र्द॒शभ्य॒ इति॑ चतुर्द॒श - भ्यः॒ । स्वाहा᳚ । प॒ञ्च॒द॒शभ्य॒ इति॑ पञ्चद॒श - भ्यः॒ । स्वाहा᳚ । षो॒ड॒शभ्य॒ इति॑ षोड॒श - भ्यः॒ । स्वाहा᳚ । स॒प्त॒द॒शभ्य॒ इति॑ सप्तद॒श - भ्यः॒ । स्वाहा᳚ । अ॒ष्टा॒द॒शभ्य॒ इत्य॑ष्टाद॒श - भ्यः॒ । स्वाहा᳚ । एका᳚त् । न । विꣳ॒॒श॒त्यै । स्वाहा᳚ । नव॑विꣳशत्या॒ इति॒ नव॑ - विꣳ॒॒श॒त्यै॒ । स्वाहा᳚ । एका᳚त् । न । च॒त्वा॒रिꣳ॒॒शते᳚ । स्वाहा᳚ । नव॑चत्वारिꣳशत॒ इति॒ नव॑ - च॒त्वा॒रिꣳ॒॒श॒ते॒ । स्वाहा᳚ । एका᳚त् । न ( ) । ष॒ष्ट्यै । स्वाहा᳚ । नव॑षष्ट्या॒ इति॒ नव॑ - ष॒ष्ट्यै॒ । स्वाहा᳚ । एका᳚त् । न । अ॒शी॒त्यै । स्वाहा᳚ । नवा॑शीत्या॒ इति॒ नव॑ - अ॒शी॒त्यै॒ । स्वाहा᳚ । एका᳚त् । न । श॒ताय॑ । स्वाहा᳚ । श॒ताय॑ । स्वाहा᳚ । द्वाभ्या᳚म् । श॒ताभ्या᳚म् । स्वाहा᳚ । सर्व॑स्मै । स्वाहा᳚ ॥  \newline


\textbf{Krama Paata} \newline

एक॑स्मै॒ स्वाहा᳚ । स्वाहा॒ द्वाभ्या᳚म् । द्वाभ्याꣳ॒॒ स्वाहा᳚ । स्वाहा᳚ त्रि॒भ्यः । त्रि॒भ्यः स्वाहा᳚ । त्रि॒भ्य इति॑ त्रि - भ्यः । स्वाहा॑ च॒तुर्भ्यः॑ । च॒तुर्भ्यः॒ स्वाहा᳚ । च॒तुर्भ्य॒ इति॑ च॒तुः - भ्यः॒ । स्वाहा॑ प॒ञ्चभ्यः॑ । प॒ञ्चभ्यः॒ स्वाहा᳚ । प॒ञ्चभ्य॒ इति॑ प॒ञ्च - भ्यः॒ । स्वाहा॑ ष॒ड्भ्यः । ष॒ड्भ्यः स्वाहा᳚ । ष॒ड्भ्य इति॑ षट् - भ्यः । 
स्वाहा॑ स॒प्तभ्यः॑ । स॒प्तभ्यः॒ स्वाहा᳚ । स॒प्तभ्य॒ इति॑ स॒प्त - भ्यः॒ । स्वाहा᳚ऽष्टा॒भ्यः । अ॒ष्टा॒भ्यः स्वाहा᳚ । स्वाहा॑ न॒वभ्यः॑ । न॒वभ्यः॒ स्वाहा᳚ । न॒वभ्य॒ इति॑ न॒व - भ्यः॒ । स्वाहा॑ द॒शभ्यः॑ । द॒शभ्यः॒ स्वाहा᳚ । द॒शभ्य॒ इति॑ द॒श - भ्यः॒ । स्वाहै॑काद॒शभ्यः॑ । ए॒का॒द॒शभ्यः॒ स्वाहा᳚ । ए॒का॒द॒शभ्य॒ इत्ये॑काद॒श - भ्यः॒ । स्वाहा᳚ द्वाद॒शभ्यः॑ । द्वा॒द॒शभ्यः॒ स्वाहा᳚ । द्वा॒द॒शभ्य॒ इति॑ द्वाद॒श - भ्यः॒ । स्वाहा᳚ त्रयोद॒शभ्यः॑ । त्र॒यो॒द॒शभ्यः॒ स्वाहा᳚ । त्र॒यो॒द॒शभ्य॒ इति॑ त्रयोद॒श - भ्यः॒ । स्वाहा॑ चतुर्द॒शभ्यः॑ । च॒तु॒र्द॒शभ्यः॒ स्वाहा᳚ । च॒तु॒र्द॒शभ्य॒ इति॑ चतुर्द॒श - भ्यः॒ । स्वाहा॑ पञ्चद॒शभ्यः॑ । प॒ञ्च॒द॒शभ्यः॒ स्वाहा᳚ । प॒ञ्च॒द॒शभ्य॒ इति॑ पञ्चद॒श - भ्यः॒ । स्वाहा॑ षोड॒शभ्यः॑ । षो॒ड॒शभ्यः॒ स्वाहा᳚ । षो॒ड॒शभ्य॒ इति॑ षोड॒श - भ्यः॒ । स्वाहा॑ सप्तद॒शभ्यः॑ । स॒प्त॒द॒शभ्यः॒ स्वाहा᳚ । स॒प्त॒द॒शभ्य॒ इति॑ सप्तद॒श - भ्यः॒ । स्वाहा᳚ऽष्टाद॒शभ्यः॑ । अ॒ष्टा॒द॒शभ्यः॒ स्वाहा᳚ । अ॒ष्टा॒द॒शभ्य॒ इत्य॑ष्टाद॒श - भ्यः॒ । स्वाहैका᳚त् । एका॒न् न । न विꣳ॑श॒त्यै । विꣳ॒॒श॒त्यै स्वाहा᳚ । स्वाहा॒ नव॑विꣳशत्यै । नव॑विꣳशत्यै॒ स्वाहा᳚ । नव॑विꣳशत्या॒ इति॒ नव॑ - विꣳ॒॒श॒त्यै॒ । स्वाहैका᳚त् । एका॒न् न । न च॑त्वारिꣳ॒॒शते᳚ । च॒त्वा॒रिꣳ॒॒शते॒ स्वाहा᳚ । स्वाहा॒ नव॑चत्वारिꣳशते । नव॑चत्वारिꣳशते॒ स्वाहा᳚ । नव॑चत्वारिꣳशत॒ इति॒ नव॑ - च॒त्वा॒रिꣳ॒॒श॒ते॒ । स्वाहैका᳚त् । एका॒न् न ( ) । न ष॒ष्ट्‍यै । ष॒ष्ट्‍यै स्वाहा᳚ । स्वाहा॒ नव॑षष्ट्‍यै । नव॑षष्ट्‍यै॒ स्वाहा᳚ । नव॑षष्ट्‍या॒ इति॒ नव॑ - ष॒ष्ट्‍यै॒ । स्वाहैका᳚त् । एका॒न् न । नाशी॒त्यै । अ॒शी॒त्यै स्वाहा᳚ । स्वाहा॒ नवा॑शीत्यै । नवा॑शीत्यै॒ स्वाहा᳚ । नवा॑शीत्या॒ इति॒ नव॑ - अ॒शी॒त्यै॒ । स्वाहैका᳚त् । एका॒न् न । न श॒ताय॑ । श॒ताय॒ स्वाहा᳚ । स्वाहा॑ श॒ताय॑ । श॒ताय॒ स्वाहा᳚ । स्वाहा॒ द्वाभ्या᳚म् । द्वाभ्याꣳ॑ श॒ताभ्या᳚म् । श॒ताभ्याꣳ॒॒ स्वाहा᳚ । स्वाहा॒ सर्व॑स्मै । सर्व॑स्मै॒ स्वाहा᳚ । स्वाहेति॒ स्वाहा᳚ । \newline

\textbf{Jatai Paata} \newline

1. एक॑स्मै॒ स्वाहा॒ स्वाहै क॑स्मा॒ एक॑स्मै॒ स्वाहा᳚ । \newline
2. स्वाहा॒ द्वाभ्या॒म् द्वाभ्याꣳ॒॒ स्वाहा॒ स्वाहा॒ द्वाभ्या᳚म् । \newline
3. द्वाभ्याꣳ॒॒ स्वाहा॒ स्वाहा॒ द्वाभ्या॒म् द्वाभ्याꣳ॒॒ स्वाहा᳚ । \newline
4. स्वाहा᳚ त्रि॒भ्य स्त्रि॒भ्यः स्वाहा॒ स्वाहा᳚ त्रि॒भ्यः । \newline
5. त्रि॒भ्यः स्वाहा॒ स्वाहा᳚ त्रि॒भ्य स्त्रि॒भ्यः स्वाहा᳚ । \newline
6. त्रि॒भ्य इति॑ त्रि - भ्यः । \newline
7. स्वाहा॑ च॒तुर्भ्य॑ श्च॒तुर्भ्यः॒ स्वाहा॒ स्वाहा॑ च॒तुर्भ्यः॑ । \newline
8. च॒तुर्भ्यः॒ स्वाहा॒ स्वाहा॑ च॒तुर्भ्य॑ श्च॒तुर्भ्यः॒ स्वाहा᳚ । \newline
9. च॒तुर्भ्य॒ इति॑ च॒तुः - भ्यः॒ । \newline
10. स्वाहा॑ प॒ञ्चभ्यः॑ प॒ञ्चभ्यः॒ स्वाहा॒ स्वाहा॑ प॒ञ्चभ्यः॑ । \newline
11. प॒ञ्चभ्यः॒ स्वाहा॒ स्वाहा॑ प॒ञ्चभ्यः॑ प॒ञ्चभ्यः॒ स्वाहा᳚ । \newline
12. प॒ञ्चभ्य॒ इति॑ प॒ञ्च - भ्यः॒ । \newline
13. स्वाहा॑ ष॒ड्भ्य ष्ष॒ड्भ्यः स्वाहा॒ स्वाहा॑ ष॒ड्भ्यः । \newline
14. ष॒ड्भ्यः स्वाहा॒ स्वाहा॑ ष॒ड्भ्य ष्ष॒ड्भ्यः स्वाहा᳚ । \newline
15. ष॒ड्भ्य इति॑ षट् - भ्यः । \newline
16. स्वाहा॑ स॒प्तभ्यः॑ स॒प्तभ्यः॒ स्वाहा॒ स्वाहा॑ स॒प्तभ्यः॑ । \newline
17. स॒प्तभ्यः॒ स्वाहा॒ स्वाहा॑ स॒प्तभ्यः॑ स॒प्तभ्यः॒ स्वाहा᳚ । \newline
18. स॒प्तभ्य॒ इति॑ स॒प्त - भ्यः॒ । \newline
19. स्वाहा᳚ ऽष्टा॒भ्यो᳚ ऽष्टा॒भ्यः स्वाहा॒ स्वाहा᳚ ऽष्टा॒भ्यः । \newline
20. अ॒ष्टा॒भ्यः स्वाहा॒ स्वाहा᳚ ऽष्टा॒भ्यो᳚ ऽष्टा॒भ्यः स्वाहा᳚ । \newline
21. स्वाहा॑ न॒वभ्यो॑ न॒वभ्यः॒ स्वाहा॒ स्वाहा॑ न॒वभ्यः॑ । \newline
22. न॒वभ्यः॒ स्वाहा॒ स्वाहा॑ न॒वभ्यो॑ न॒वभ्यः॒ स्वाहा᳚ । \newline
23. न॒वभ्य॒ इति॑ न॒व - भ्यः॒ । \newline
24. स्वाहा॑ द॒शभ्यो॑ द॒शभ्यः॒ स्वाहा॒ स्वाहा॑ द॒शभ्यः॑ । \newline
25. द॒शभ्यः॒ स्वाहा॒ स्वाहा॑ द॒शभ्यो॑ द॒शभ्यः॒ स्वाहा᳚ । \newline
26. द॒शभ्य॒ इति॑ द॒श - भ्यः॒ । \newline
27. स्वाहै॑काद॒शभ्य॑ एकाद॒शभ्यः॒ स्वाहा॒ स्वाहै॑काद॒शभ्यः॑ । \newline
28. ए॒का॒द॒शभ्यः॒ स्वाहा॒ स्वाहै॑काद॒शभ्य॑ एकाद॒शभ्यः॒ स्वाहा᳚ । \newline
29. ए॒का॒द॒शभ्य॒ इत्ये॑काद॒श - भ्यः॒ । \newline
30. स्वाहा᳚ द्वाद॒शभ्यो᳚ द्वाद॒शभ्यः॒ स्वाहा॒ स्वाहा᳚ द्वाद॒शभ्यः॑ । \newline
31. द्वा॒द॒शभ्यः॒ स्वाहा॒ स्वाहा᳚ द्वाद॒शभ्यो᳚ द्वाद॒शभ्यः॒ स्वाहा᳚ । \newline
32. द्वा॒द॒शभ्य॒ इति॑ द्वाद॒श - भ्यः॒ । \newline
33. स्वाहा᳚ त्रयोद॒शभ्य॑ स्त्रयोद॒शभ्यः॒ स्वाहा॒ स्वाहा᳚ त्रयोद॒शभ्यः॑ । \newline
34. त्र॒यो॒द॒शभ्यः॒ स्वाहा॒ स्वाहा᳚ त्रयोद॒शभ्य॑ स्त्रयोद॒शभ्यः॒ स्वाहा᳚ । \newline
35. त्र॒यो॒द॒शभ्य॒ इति॑ त्रयोद॒श - भ्यः॒ । \newline
36. स्वाहा॑ चतुर्द॒शभ्य॑ श्चतुर्द॒शभ्यः॒ स्वाहा॒ स्वाहा॑ चतुर्द॒शभ्यः॑ । \newline
37. च॒तु॒र्द॒शभ्यः॒ स्वाहा॒ स्वाहा॑ चतुर्द॒शभ्य॑ श्चतुर्द॒शभ्यः॒ स्वाहा᳚ । \newline
38. च॒तु॒र्द॒शभ्य॒ इति॑ चतुर्द॒श - भ्यः॒ । \newline
39. स्वाहा॑ पञ्चद॒शभ्यः॑ पञ्चद॒शभ्यः॒ स्वाहा॒ स्वाहा॑ पञ्चद॒शभ्यः॑ । \newline
40. प॒ञ्च॒द॒शभ्यः॒ स्वाहा॒ स्वाहा॑ पञ्चद॒शभ्यः॑ पञ्चद॒शभ्यः॒ स्वाहा᳚ । \newline
41. प॒ञ्च॒द॒शभ्य॒ इति॑ पञ्चद॒श - भ्यः॒ । \newline
42. स्वाहा॑ षोड॒शभ्य॑ ष्षोड॒शभ्यः॒ स्वाहा॒ स्वाहा॑ षोड॒शभ्यः॑ । \newline
43. षो॒ड॒शभ्यः॒ स्वाहा॒ स्वाहा॑ षोड॒शभ्य॑ ष्षोड॒शभ्यः॒ स्वाहा᳚ । \newline
44. षो॒ड॒शभ्य॒ इति॑ षोड॒श - भ्यः॒ । \newline
45. स्वाहा॑ सप्तद॒शभ्यः॑ सप्तद॒शभ्यः॒ स्वाहा॒ स्वाहा॑ सप्तद॒शभ्यः॑ । \newline
46. स॒प्त॒द॒शभ्यः॒ स्वाहा॒ स्वाहा॑ सप्तद॒शभ्यः॑ सप्तद॒शभ्यः॒ स्वाहा᳚ । \newline
47. स॒प्त॒द॒शभ्य॒ इति॑ सप्तद॒श - भ्यः॒ । \newline
48. स्वाहा᳚ ऽष्टाद॒शभ्यो᳚ ऽष्टाद॒शभ्यः॒ स्वाहा॒ स्वाहा᳚ ऽष्टाद॒शभ्यः॑ । \newline
49. अ॒ष्टा॒द॒शभ्यः॒ स्वाहा॒ स्वाहा᳚ ऽष्टाद॒शभ्यो᳚ ऽष्टाद॒शभ्यः॒ स्वाहा᳚ । \newline
50. अ॒ष्टा॒द॒शभ्य॒ इत्य॑ष्टाद॒श - भ्यः॒ । \newline
51. स्वाहैका॒ देका॒थ् स्वाहा॒ स्वाहैका᳚त् । \newline
52. एका॒न् न नैका॒ देका॒न् न । \newline
53. न विꣳ॑श॒त्यै विꣳ॑श॒त्यै न न विꣳ॑श॒त्यै । \newline
54. विꣳ॒॒श॒त्यै स्वाहा॒ स्वाहा॑ विꣳश॒त्यै विꣳ॑श॒त्यै स्वाहा᳚ । \newline
55. स्वाहा॒ नव॑विꣳशत्यै॒ नव॑विꣳशत्यै॒ स्वाहा॒ स्वाहा॒ नव॑विꣳशत्यै । \newline
56. नव॑विꣳशत्यै॒ स्वाहा॒ स्वाहा॒ नव॑विꣳशत्यै॒ नव॑विꣳशत्यै॒ स्वाहा᳚ । \newline
57. नव॑विꣳशत्या॒ इति॒ नव॑ - विꣳ॒॒श॒त्यै॒ । \newline
58. स्वाहैका॒ देका॒थ् स्वाहा॒ स्वाहैका᳚त् । \newline
59. एका॒न् न नैका॒ देका॒न् न । \newline
60. न च॑त्वारिꣳ॒॒शते॑ चत्वारिꣳ॒॒शते॒ न न च॑त्वारिꣳ॒॒शते᳚ । \newline
61. च॒त्वा॒रिꣳ॒॒शते॒ स्वाहा॒ स्वाहा॑ चत्वारिꣳ॒॒शते॑ चत्वारिꣳ॒॒शते॒ स्वाहा᳚ । \newline
62. स्वाहा॒ नव॑चत्वारिꣳशते॒ नव॑चत्वारिꣳशते॒ स्वाहा॒ स्वाहा॒ नव॑चत्वारिꣳशते । \newline
63. नव॑चत्वारिꣳशते॒ स्वाहा॒ स्वाहा॒ नव॑चत्वारिꣳशते॒ नव॑चत्वारिꣳशते॒ स्वाहा᳚ । \newline
64. नव॑चत्वारिꣳशत॒ इति॒ नव॑ - च॒त्वा॒रिꣳ॒॒श॒ते॒ । \newline
65. स्वाहैका॒ देका॒थ् स्वाहा॒ स्वाहैका᳚त् । \newline
66. एका॒न् न नैका॒ देका॒न् न । \newline
67. न ष॒ष्ट्यै ष॒ष्ट्यै न न ष॒ष्ट्यै । \newline
68. ष॒ष्ट्यै स्वाहा॒ स्वाहा॑ ष॒ष्ट्यै ष॒ष्ट्यै स्वाहा᳚ । \newline
69. स्वाहा॒ नव॑षष्ट्यै॒ नव॑षष्ट्यै॒ स्वाहा॒ स्वाहा॒ नव॑षष्ट्यै । \newline
70. नव॑षष्ट्यै॒ स्वाहा॒ स्वाहा॒ नव॑षष्ट्यै॒ नव॑षष्ट्यै॒ स्वाहा᳚ । \newline
71. नव॑षष्ट्या॒ इति॒ नव॑ - ष॒ष्ट्यै॒ । \newline
72. स्वाहैका॒ देका॒थ् स्वाहा॒ स्वाहैका᳚त् । \newline
73. एका॒न् न नैका॒ देका॒न् न । \newline
74. नाशी॒त्या अ॑शी॒त्यै न नाशी॒त्यै । \newline
75. अ॒शी॒त्यै स्वाहा॒ स्वाहा॑ ऽशी॒त्या अ॑शी॒त्यै स्वाहा᳚ । \newline
76. स्वाहा॒ नवा॑शीत्यै॒ नवा॑शीत्यै॒ स्वाहा॒ स्वाहा॒ नवा॑शीत्यै । \newline
77. नवा॑शीत्यै॒ स्वाहा॒ स्वाहा॒ नवा॑शीत्यै॒ नवा॑शीत्यै॒ स्वाहा᳚ । \newline
78. नवा॑शीत्या॒ इति॒ नव॑ - अ॒शी॒त्यै॒ । \newline
79. स्वाहैका॒ देका॒थ् स्वाहा॒ स्वाहैका᳚त् । \newline
80. एका॒न् न नैका॒ देका॒न् न । \newline
81. न श॒ताय॑ श॒ताय॒ न न श॒ताय॑ । \newline
82. श॒ताय॒ स्वाहा॒ स्वाहा॑ श॒ताय॑ श॒ताय॒ स्वाहा᳚ । \newline
83. स्वाहा॑ श॒ताय॑ श॒ताय॒ स्वाहा॒ स्वाहा॑ श॒ताय॑ । \newline
84. श॒ताय॒ स्वाहा॒ स्वाहा॑ श॒ताय॑ श॒ताय॒ स्वाहा᳚ । \newline
85. स्वाहा॒ द्वाभ्या॒म् द्वाभ्याꣳ॒॒ स्वाहा॒ स्वाहा॒ द्वाभ्या᳚म् । \newline
86. द्वाभ्याꣳ॑ श॒ताभ्याꣳ॑ श॒ताभ्या॒म् द्वाभ्या॒म् द्वाभ्याꣳ॑ श॒ताभ्या᳚म् । \newline
87. श॒ताभ्याꣳ॒॒ स्वाहा॒ स्वाहा॑ श॒ताभ्याꣳ॑ श॒ताभ्याꣳ॒॒ स्वाहा᳚ । \newline
88. स्वाहा॒ सर्व॑स्मै॒ सर्व॑स्मै॒ स्वाहा॒ स्वाहा॒ सर्व॑स्मै । \newline
89. सर्व॑स्मै॒ स्वाहा॒ स्वाहा॒ सर्व॑स्मै॒ सर्व॑स्मै॒ स्वाहा᳚ । \newline
90. स्वाहेति॒ स्वाहा᳚ । \newline

\textbf{Ghana Paata } \newline

1. एक॑स्मै॒ स्वाहा॒ स्वाहैक॑स्मा॒ एक॑स्मै॒ स्वाहा॒ द्वाभ्या॒म् द्वाभ्याꣳ॒॒ स्वाहैक॑स्मा॒ एक॑स्मै॒ स्वाहा॒ द्वाभ्या᳚म् । \newline
2. स्वाहा॒ द्वाभ्या॒म् द्वाभ्याꣳ॒॒ स्वाहा॒ स्वाहा॒ द्वाभ्याꣳ॒॒ स्वाहा॒ स्वाहा॒ द्वाभ्याꣳ॒॒ स्वाहा॒ स्वाहा॒ द्वाभ्याꣳ॒॒ स्वाहा᳚ । \newline
3. द्वाभ्याꣳ॒॒ स्वाहा॒ स्वाहा॒ द्वाभ्या॒म् द्वाभ्याꣳ॒॒ स्वाहा᳚ त्रि॒भ्य स्त्रि॒भ्यः स्वाहा॒ द्वाभ्या॒म् द्वाभ्याꣳ॒॒ स्वाहा᳚ त्रि॒भ्यः । \newline
4. स्वाहा᳚ त्रि॒भ्य स्त्रि॒भ्यः स्वाहा॒ स्वाहा᳚ त्रि॒भ्यः स्वाहा॒ स्वाहा᳚ त्रि॒भ्यः स्वाहा॒ स्वाहा᳚ त्रि॒भ्यः स्वाहा᳚ । \newline
5. त्रि॒भ्यः स्वाहा॒ स्वाहा᳚ त्रि॒भ्य स्त्रि॒भ्यः स्वाहा॑ च॒तुर्भ्य॑ श्च॒तुर्भ्यः॒ स्वाहा᳚ त्रि॒भ्य स्त्रि॒भ्यः स्वाहा॑ च॒तुर्भ्यः॑ । \newline
6. त्रि॒भ्य इति॑ त्रि - भ्यः । \newline
7. स्वाहा॑ च॒तुर्भ्य॑ श्च॒तुर्भ्यः॒ स्वाहा॒ स्वाहा॑ च॒तुर्भ्यः॒ स्वाहा॒ स्वाहा॑ च॒तुर्भ्यः॒ स्वाहा॒ स्वाहा॑ च॒तुर्भ्यः॒ स्वाहा᳚ । \newline
8. च॒तुर्भ्यः॒ स्वाहा॒ स्वाहा॑ च॒तुर्भ्य॑ श्च॒तुर्भ्यः॒ स्वाहा॑ प॒ञ्चभ्यः॑ प॒ञ्चभ्यः॒ स्वाहा॑ च॒तुर्भ्य॑ श्च॒तुर्भ्यः॒ स्वाहा॑ प॒ञ्चभ्यः॑ । \newline
9. च॒तुर्भ्य॒ इति॑ च॒तुः - भ्यः॒ । \newline
10. स्वाहा॑ प॒ञ्चभ्यः॑ प॒ञ्चभ्यः॒ स्वाहा॒ स्वाहा॑ प॒ञ्चभ्यः॒ स्वाहा॒ स्वाहा॑ प॒ञ्चभ्यः॒ स्वाहा॒ स्वाहा॑ प॒ञ्चभ्यः॒ स्वाहा᳚ । \newline
11. प॒ञ्चभ्यः॒ स्वाहा॒ स्वाहा॑ प॒ञ्चभ्यः॑ प॒ञ्चभ्यः॒ स्वाहा॑ ष॒ड्भ्य ष्ष॒ड्भ्यः स्वाहा॑ प॒ञ्चभ्यः॑ प॒ञ्चभ्यः॒ स्वाहा॑ ष॒ड्भ्यः । \newline
12. प॒ञ्चभ्य॒ इति॑ प॒ञ्च - भ्यः॒ । \newline
13. स्वाहा॑ ष॒ड्भ्य ष्ष॒ड्भ्यः स्वाहा॒ स्वाहा॑ ष॒ड्भ्यः स्वाहा॒ स्वाहा॑ ष॒ड्भ्यः स्वाहा॒ स्वाहा॑ ष॒ड्भ्यः स्वाहा᳚ । \newline
14. ष॒ड्भ्यः स्वाहा॒ स्वाहा॑ ष॒ड्भ्य ष्ष॒ड्भ्यः स्वाहा॑ स॒प्तभ्यः॑ स॒प्तभ्यः॒ स्वाहा॑ ष॒ड्भ्य ष्ष॒ड्भ्यः स्वाहा॑ स॒प्तभ्यः॑ । \newline
15. ष॒ड्भ्य इति॑ षट् - भ्यः । \newline
16. स्वाहा॑ स॒प्तभ्यः॑ स॒प्तभ्यः॒ स्वाहा॒ स्वाहा॑ स॒प्तभ्यः॒ स्वाहा॒ स्वाहा॑ स॒प्तभ्यः॒ स्वाहा॒ स्वाहा॑ स॒प्तभ्यः॒ स्वाहा᳚ । \newline
17. स॒प्तभ्यः॒ स्वाहा॒ स्वाहा॑ स॒प्तभ्यः॑ स॒प्तभ्यः॒ स्वाहा᳚ ऽष्टा॒भ्यो᳚ ऽष्टा॒भ्यः स्वाहा॑ स॒प्तभ्यः॑ स॒प्तभ्यः॒ स्वाहा᳚ ऽष्टा॒भ्यः । \newline
18. स॒प्तभ्य॒ इति॑ स॒प्त - भ्यः॒ । \newline
19. स्वाहा᳚ ऽष्टा॒भ्यो᳚ ऽष्टा॒भ्यः स्वाहा॒ स्वाहा᳚ ऽष्टा॒भ्यः स्वाहा॒ स्वाहा᳚ ऽष्टा॒भ्यः स्वाहा॒ स्वाहा᳚ ऽष्टा॒भ्यः स्वाहा᳚ । \newline
20. अ॒ष्टा॒भ्यः स्वाहा॒ स्वाहा᳚ ऽष्टा॒भ्यो᳚ ऽष्टा॒भ्यः स्वाहा॑ न॒वभ्यो॑ न॒वभ्यः॒ स्वाहा᳚ ऽष्टा॒भ्यो᳚ ऽष्टा॒भ्यः स्वाहा॑ न॒वभ्यः॑ । \newline
21. स्वाहा॑ न॒वभ्यो॑ न॒वभ्यः॒ स्वाहा॒ स्वाहा॑ न॒वभ्यः॒ स्वाहा॒ स्वाहा॑ न॒वभ्यः॒ स्वाहा॒ स्वाहा॑ न॒वभ्यः॒ स्वाहा᳚ । \newline
22. न॒वभ्यः॒ स्वाहा॒ स्वाहा॑ न॒वभ्यो॑ न॒वभ्यः॒ स्वाहा॑ द॒शभ्यो॑ द॒शभ्यः॒ स्वाहा॑ न॒वभ्यो॑ न॒वभ्यः॒ स्वाहा॑ द॒शभ्यः॑ । \newline
23. न॒वभ्य॒ इति॑ न॒व - भ्यः॒ । \newline
24. स्वाहा॑ द॒शभ्यो॑ द॒शभ्यः॒ स्वाहा॒ स्वाहा॑ द॒शभ्यः॒ स्वाहा॒ स्वाहा॑ द॒शभ्यः॒ स्वाहा॒ स्वाहा॑ द॒शभ्यः॒ स्वाहा᳚ । \newline
25. द॒शभ्यः॒ स्वाहा॒ स्वाहा॑ द॒शभ्यो॑ द॒शभ्यः॒ स्वाहै॑काद॒शभ्य॑ एकाद॒शभ्यः॒ स्वाहा॑ द॒शभ्यो॑ द॒शभ्यः॒ स्वाहै॑काद॒शभ्यः॑ । \newline
26. द॒शभ्य॒ इति॑ द॒श - भ्यः॒ । \newline
27. स्वाहै॑काद॒शभ्य॑ एकाद॒शभ्यः॒ स्वाहा॒ स्वाहै॑काद॒शभ्यः॒ स्वाहा॒ स्वाहै॑काद॒शभ्यः॒ स्वाहा॒ 
स्वाहै॑काद॒शभ्यः॒ स्वाहा᳚ । \newline
28. ए॒का॒द॒शभ्यः॒ स्वाहा॒ स्वाहै॑काद॒शभ्य॑ एकाद॒शभ्यः॒ स्वाहा᳚ द्वाद॒शभ्यो᳚ द्वाद॒शभ्यः॒ 
स्वाहै॑काद॒शभ्य॑ एकाद॒शभ्यः॒ स्वाहा᳚ द्वाद॒शभ्यः॑ । \newline
29. ए॒का॒द॒शभ्य॒ इत्ये॑काद॒श - भ्यः॒ । \newline
30. स्वाहा᳚ द्वाद॒शभ्यो᳚ द्वाद॒शभ्यः॒ स्वाहा॒ स्वाहा᳚ द्वाद॒शभ्यः॒ स्वाहा॒ स्वाहा᳚ द्वाद॒शभ्यः॒ स्वाहा॒ स्वाहा᳚ द्वाद॒शभ्यः॒ स्वाहा᳚ । \newline
31. द्वा॒द॒शभ्यः॒ स्वाहा॒ स्वाहा᳚ द्वाद॒शभ्यो᳚ द्वाद॒शभ्यः॒ स्वाहा᳚ त्रयोद॒शभ्य॑ स्त्रयोद॒शभ्यः॒ स्वाहा᳚ द्वाद॒शभ्यो᳚ द्वाद॒शभ्यः॒ स्वाहा᳚ त्रयोद॒शभ्यः॑ । \newline
32. द्वा॒द॒शभ्य॒ इति॑ द्वाद॒श - भ्यः॒ । \newline
33. स्वाहा᳚ त्रयोद॒शभ्य॑ स्त्रयोद॒शभ्यः॒ स्वाहा॒ स्वाहा᳚ त्रयोद॒शभ्यः॒ स्वाहा॒ स्वाहा᳚ त्रयोद॒शभ्यः॒ स्वाहा॒ स्वाहा᳚ त्रयोद॒शभ्यः॒ स्वाहा᳚ । \newline
34. त्र॒यो॒द॒शभ्यः॒ स्वाहा॒ स्वाहा᳚ त्रयोद॒शभ्य॑ स्त्रयोद॒शभ्यः॒ स्वाहा॑ चतुर्द॒शभ्य॑ श्चतुर्द॒शभ्यः॒ स्वाहा᳚ त्रयोद॒शभ्य॑ स्त्रयोद॒शभ्यः॒ स्वाहा॑ चतुर्द॒शभ्यः॑ । \newline
35. त्र॒यो॒द॒शभ्य॒ इति॑ त्रयोद॒श - भ्यः॒ । \newline
36. स्वाहा॑ चतुर्द॒शभ्य॑ श्चतुर्द॒शभ्यः॒ स्वाहा॒ स्वाहा॑ चतुर्द॒शभ्यः॒ स्वाहा॒ स्वाहा॑ चतुर्द॒शभ्यः॒ स्वाहा॒ स्वाहा॑ चतुर्द॒शभ्यः॒ स्वाहा᳚ । \newline
37. च॒तु॒र्द॒शभ्यः॒ स्वाहा॒ स्वाहा॑ चतुर्द॒शभ्य॑ श्चतुर्द॒शभ्यः॒ स्वाहा॑ पञ्चद॒शभ्यः॑ पञ्चद॒शभ्यः॒ स्वाहा॑ चतुर्द॒शभ्य॑ श्चतुर्द॒शभ्यः॒ स्वाहा॑ पञ्चद॒शभ्यः॑ । \newline
38. च॒तु॒र्द॒शभ्य॒ इति॑ चतुर्द॒श - भ्यः॒ । \newline
39. स्वाहा॑ पञ्चद॒शभ्यः॑ पञ्चद॒शभ्यः॒ स्वाहा॒ स्वाहा॑ पञ्चद॒शभ्यः॒ स्वाहा॒ स्वाहा॑ पञ्चद॒शभ्यः॒ स्वाहा॒ स्वाहा॑ पञ्चद॒शभ्यः॒ स्वाहा᳚ । \newline
40. प॒ञ्च॒द॒शभ्यः॒ स्वाहा॒ स्वाहा॑ पञ्चद॒शभ्यः॑ पञ्चद॒शभ्यः॒ स्वाहा॑ षोड॒शभ्य॑ ष्षोड॒शभ्यः॒ स्वाहा॑ पञ्चद॒शभ्यः॑ पञ्चद॒शभ्यः॒ स्वाहा॑ षोड॒शभ्यः॑ । \newline
41. प॒ञ्च॒द॒शभ्य॒ इति॑ पञ्चद॒श - भ्यः॒ । \newline
42. स्वाहा॑ षोड॒शभ्य॑ ष्षोड॒शभ्यः॒ स्वाहा॒ स्वाहा॑ षोड॒शभ्यः॒ स्वाहा॒ स्वाहा॑ षोड॒शभ्यः॒ स्वाहा॒ स्वाहा॑ षोड॒शभ्यः॒ स्वाहा᳚ । \newline
43. षो॒ड॒शभ्यः॒ स्वाहा॒ स्वाहा॑ षोड॒शभ्य॑ ष्षोड॒शभ्यः॒ स्वाहा॑ सप्तद॒शभ्यः॑ सप्तद॒शभ्यः॒ स्वाहा॑ षोड॒शभ्य॑ ष्षोड॒शभ्यः॒ स्वाहा॑ सप्तद॒शभ्यः॑ । \newline
44. षो॒ड॒शभ्य॒ इति॑ षोड॒श - भ्यः॒ । \newline
45. स्वाहा॑ सप्तद॒शभ्यः॑ सप्तद॒शभ्यः॒ स्वाहा॒ स्वाहा॑ सप्तद॒शभ्यः॒ स्वाहा॒ स्वाहा॑ सप्तद॒शभ्यः॒ स्वाहा॒ स्वाहा॑ सप्तद॒शभ्यः॒ स्वाहा᳚ । \newline
46. स॒प्त॒द॒शभ्यः॒ स्वाहा॒ स्वाहा॑ सप्तद॒शभ्यः॑ सप्तद॒शभ्यः॒ स्वाहा᳚ ऽष्टाद॒शभ्यो᳚ ऽष्टाद॒शभ्यः॒ स्वाहा॑ सप्तद॒शभ्यः॑ सप्तद॒शभ्यः॒ स्वाहा᳚ ऽष्टाद॒शभ्यः॑ । \newline
47. स॒प्त॒द॒शभ्य॒ इति॑ सप्तद॒श - भ्यः॒ । \newline
48. स्वाहा᳚ ऽष्टाद॒शभ्यो᳚ ऽष्टाद॒शभ्यः॒ स्वाहा॒ स्वाहा᳚ ऽष्टाद॒शभ्यः॒ स्वाहा॒ स्वाहा᳚ ऽष्टाद॒शभ्यः॒ स्वाहा॒ स्वाहा᳚ ऽष्टाद॒शभ्यः॒ स्वाहा᳚ । \newline
49. अ॒ष्टा॒द॒शभ्यः॒ स्वाहा॒ स्वाहा᳚ ऽष्टाद॒शभ्यो᳚ ऽष्टाद॒शभ्यः॒ स्वाहैका॒देका॒थ् स्वाहा᳚ ऽष्टाद॒शभ्यो᳚ ऽष्टाद॒शभ्यः॒ स्वाहैका᳚त् । \newline
50. अ॒ष्टा॒द॒शभ्य॒ इत्य॑ष्टाद॒श - भ्यः॒ । \newline
51. स्वाहैका॒ देका॒थ् स्वाहा॒ स्वाहैका॒न् न नैका॒थ् स्वाहा॒ स्वाहैका॒न् न । \newline
52. एका॒न् न नैका॒ देका॒न् न विꣳ॑श॒त्यै विꣳ॑श॒त्यै नैका॒ देका॒न् न विꣳ॑श॒त्यै । \newline
53. न विꣳ॑श॒त्यै विꣳ॑श॒त्यै न न विꣳ॑श॒त्यै स्वाहा॒ स्वाहा॑ विꣳश॒त्यै न न विꣳ॑श॒त्यै स्वाहा᳚ । \newline
54. विꣳ॒॒श॒त्यै स्वाहा॒ स्वाहा॑ विꣳश॒त्यै विꣳ॑श॒त्यै स्वाहा॒ नव॑विꣳशत्यै॒ नव॑विꣳशत्यै॒ स्वाहा॑ विꣳश॒त्यै विꣳ॑श॒त्यै स्वाहा॒ नव॑विꣳशत्यै । \newline
55. स्वाहा॒ नव॑विꣳशत्यै॒ नव॑विꣳशत्यै॒ स्वाहा॒ स्वाहा॒ नव॑विꣳशत्यै॒ स्वाहा॒ स्वाहा॒ नव॑विꣳशत्यै॒ स्वाहा॒ स्वाहा॒ नव॑विꣳशत्यै॒ स्वाहा᳚ । \newline
56. नव॑विꣳशत्यै॒ स्वाहा॒ स्वाहा॒ नव॑विꣳशत्यै॒ नव॑विꣳशत्यै॒ स्वाहैका॒ देका॒थ् स्वाहा॒ नव॑विꣳशत्यै॒ नव॑विꣳशत्यै॒ स्वाहैका᳚त् । \newline
57. नव॑विꣳशत्या॒ इति॒ नव॑ - विꣳ॒॒श॒त्यै॒ । \newline
58. स्वाहैका॒ देका॒थ् स्वाहा॒ स्वाहैका॒न् न नैका॒थ् स्वाहा॒ स्वाहैका॒न् न । \newline
59. एका॒न् न नैका॒ देका॒न् न च॑त्वारिꣳ॒॒शते॑ चत्वारिꣳ॒॒शते॒ नैका॒ देका॒न् न च॑त्वारिꣳ॒॒शते᳚ । \newline
60. न च॑त्वारिꣳ॒॒शते॑ चत्वारिꣳ॒॒शते॒ न न च॑त्वारिꣳ॒॒शते॒ स्वाहा॒ स्वाहा॑ चत्वारिꣳ॒॒शते॒ न न च॑त्वारिꣳ॒॒शते॒ स्वाहा᳚ । \newline
61. च॒त्वा॒रिꣳ॒॒शते॒ स्वाहा॒ स्वाहा॑ चत्वारिꣳ॒॒शते॑ चत्वारिꣳ॒॒शते॒ स्वाहा॒ नव॑चत्वारिꣳशते॒ नव॑चत्वारिꣳशते॒ स्वाहा॑ चत्वारिꣳ॒॒शते॑ चत्वारिꣳ॒॒शते॒ स्वाहा॒ नव॑चत्वारिꣳशते । \newline
62. स्वाहा॒ नव॑चत्वारिꣳशते॒ नव॑चत्वारिꣳशते॒ स्वाहा॒ स्वाहा॒ नव॑चत्वारिꣳशते॒ स्वाहा॒ स्वाहा॒ नव॑चत्वारिꣳशते॒ स्वाहा॒ स्वाहा॒ नव॑चत्वारिꣳशते॒ स्वाहा᳚ । \newline
63. नव॑चत्वारिꣳशते॒ स्वाहा॒ स्वाहा॒ नव॑चत्वारिꣳशते॒ नव॑चत्वारिꣳशते॒ स्वाहैका॒ देका॒थ् स्वाहा॒ नव॑चत्वारिꣳशते॒ नव॑चत्वारिꣳशते॒ स्वाहैका᳚त् । \newline
64. नव॑चत्वारिꣳशत॒ इति॒ नव॑ - च॒त्वा॒रिꣳ॒॒श॒ते॒ । \newline
65. स्वाहैका॒ देका॒थ् स्वाहा॒ स्वाहैका॒न् न नैका॒थ् स्वाहा॒ स्वाहैका॒न् न । \newline
66. एका॒न् न नैका॒ देका॒न् न ष॒ष्ट्यै ष॒ष्ट्यै नैका॒ देका॒न् न ष॒ष्ट्यै । \newline
67. न ष॒ष्ट्यै ष॒ष्ट्यै न न ष॒ष्ट्यै स्वाहा॒ स्वाहा॑ ष॒ष्ट्यै न न ष॒ष्ट्यै स्वाहा᳚ । \newline
68. ष॒ष्ट्यै स्वाहा॒ स्वाहा॑ ष॒ष्ट्यै ष॒ष्ट्यै स्वाहा॒ नव॑षष्ट्यै॒ नव॑षष्ट्यै॒ स्वाहा॑ ष॒ष्ट्यै ष॒ष्ट्यै स्वाहा॒ नव॑षष्ट्यै । \newline
69. स्वाहा॒ नव॑षष्ट्यै॒ नव॑षष्ट्यै॒ स्वाहा॒ स्वाहा॒ नव॑षष्ट्यै॒ स्वाहा॒ स्वाहा॒ नव॑षष्ट्यै॒ स्वाहा॒ स्वाहा॒ नव॑षष्ट्यै॒ स्वाहा᳚ । \newline
70. नव॑षष्ट्यै॒ स्वाहा॒ स्वाहा॒ नव॑षष्ट्यै॒ नव॑षष्ट्यै॒ स्वाहैका॒ देका॒थ् स्वाहा॒ नव॑षष्ट्यै॒ नव॑षष्ट्यै॒ स्वाहैका᳚त् । \newline
71. नव॑षष्ट्या॒ इति॒ नव॑ - ष॒ष्ट्यै॒ । \newline
72. स्वाहैका॒ देका॒थ् स्वाहा॒ स्वाहैका॒न् न नैका॒थ् स्वाहा॒ स्वाहैका॒न् न । \newline
73. एका॒न् न नैका॒ देका॒न् नाशी॒त्या अ॑शी॒त्यै नैका॒ देका॒न् नाशी॒त्यै । \newline
74. नाशी॒त्या अ॑शी॒त्यै न नाशी॒त्यै स्वाहा॒ स्वाहा॑ ऽशी॒त्यै न नाशी॒त्यै स्वाहा᳚ । \newline
75. अ॒शी॒त्यै स्वाहा॒ स्वाहा॑ ऽशी॒त्या अ॑शी॒त्यै स्वाहा॒ नवा॑शीत्यै॒ नवा॑शीत्यै॒ स्वाहा॑ ऽशी॒त्या अ॑शी॒त्यै स्वाहा॒ नवा॑शीत्यै । \newline
76. स्वाहा॒ नवा॑शीत्यै॒ नवा॑शीत्यै॒ स्वाहा॒ स्वाहा॒ नवा॑शीत्यै॒ स्वाहा॒ स्वाहा॒ नवा॑शीत्यै॒ स्वाहा॒ स्वाहा॒ नवा॑शीत्यै॒ स्वाहा᳚ । \newline
77. नवा॑शीत्यै॒ स्वाहा॒ स्वाहा॒ नवा॑शीत्यै॒ नवा॑शीत्यै॒ स्वाहैका॒ देका॒थ् स्वाहा॒ नवा॑शीत्यै॒ नवा॑शीत्यै॒ स्वाहैका᳚त् । \newline
78. नवा॑शीत्या॒ इति॒ नव॑ - अ॒शी॒त्यै॒ । \newline
79. स्वाहैका॒ देका॒थ् स्वाहा॒ स्वाहैका॒न् न नैका॒थ् स्वाहा॒ स्वाहैका॒न् न । \newline
80. एका॒न् न नैका॒ देका॒न् न श॒ताय॑ श॒ताय॒ नैका॒ देका॒न् न श॒ताय॑ । \newline
81. न श॒ताय॑ श॒ताय॒ न न श॒ताय॒ स्वाहा॒ स्वाहा॑ श॒ताय॒ न न श॒ताय॒ स्वाहा᳚ । \newline
82. श॒ताय॒ स्वाहा॒ स्वाहा॑ श॒ताय॑ श॒ताय॒ स्वाहा॑ श॒ताय॑ श॒ताय॒ स्वाहा॑ श॒ताय॑ श॒ताय॒ स्वाहा॑ श॒ताय॑ । \newline
83. स्वाहा॑ श॒ताय॑ श॒ताय॒ स्वाहा॒ स्वाहा॑ श॒ताय॒ स्वाहा॒ स्वाहा॑ श॒ताय॒ स्वाहा॒ स्वाहा॑ श॒ताय॒ स्वाहा᳚ । \newline
84. श॒ताय॒ स्वाहा॒ स्वाहा॑ श॒ताय॑ श॒ताय॒ स्वाहा॒ द्वाभ्या॒म् द्वाभ्याꣳ॒॒ स्वाहा॑ श॒ताय॑ श॒ताय॒ स्वाहा॒ द्वाभ्या᳚म् । \newline
85. स्वाहा॒ द्वाभ्या॒म् द्वाभ्याꣳ॒॒ स्वाहा॒ स्वाहा॒ द्वाभ्याꣳ॑ श॒ताभ्याꣳ॑ श॒ताभ्या॒म् द्वाभ्याꣳ॒॒ स्वाहा॒ स्वाहा॒ द्वाभ्याꣳ॑ श॒ताभ्या᳚म् । \newline
86. द्वाभ्याꣳ॑ श॒ताभ्याꣳ॑ श॒ताभ्या॒म् द्वाभ्या॒म् द्वाभ्याꣳ॑ श॒ताभ्याꣳ॒॒ स्वाहा॒ स्वाहा॑ श॒ताभ्या॒म् द्वाभ्या॒म् द्वाभ्याꣳ॑ श॒ताभ्याꣳ॒॒ स्वाहा᳚ । \newline
87. श॒ताभ्याꣳ॒॒ स्वाहा॒ स्वाहा॑ श॒ताभ्याꣳ॑ श॒ताभ्याꣳ॒॒ स्वाहा॒ सर्व॑स्मै॒ सर्व॑स्मै॒ स्वाहा॑ श॒ताभ्याꣳ॑ श॒ताभ्याꣳ॒॒ स्वाहा॒ सर्व॑स्मै । \newline
88. स्वाहा॒ सर्व॑स्मै॒ सर्व॑स्मै॒ स्वाहा॒ स्वाहा॒ सर्व॑स्मै॒ स्वाहा॒ स्वाहा॒ सर्व॑स्मै॒ स्वाहा॒ स्वाहा॒ सर्व॑स्मै॒ स्वाहा᳚ । \newline
89. सर्व॑स्मै॒ स्वाहा॒ स्वाहा॒ सर्व॑स्मै॒ सर्व॑स्मै॒ स्वाहा᳚ । \newline
90. स्वाहेति॒ स्वाहा᳚ । \newline
\pagebreak
\markright{ TS 7.2.12.1  \hfill https://www.vedavms.in \hfill}

\section{ TS 7.2.12.1 }

\textbf{TS 7.2.12.1 } \newline
\textbf{Samhita Paata} \newline

एक॑स्मै॒ स्वाहा᳚ त्रि॒भ्यः स्वाहा॑ प॒ञ्चभ्यः॒ स्वाहा॑ स॒प्तभ्यः॒ स्वाहा॑ न॒वभ्यः॒ स्वाहै॑- काद॒शभ्यः॒ स्वाहा᳚ त्रयोद॒शभ्यः॒ स्वाहा॑ पञ्चद॒शभ्यः॒ स्वाहा॑ सप्तद॒शभ्यः॒ स्वाहैका॒न्न विꣳ॑श॒त्यै स्वाहा॒ नव॑विꣳशत्यै॒ स्वाहैका॒न्न च॑त्वारिꣳ॒॒शते॒ स्वाहा॒ नव॑चत्वारिꣳशते॒ स्वाहैका॒न्न ष॒ष्ट्यै स्वाहा॒ नव॑षष्ट्यै॒ स्वाहैका॒न्ना शी॒त्यै स्वाहा॒ नवा॑शीत्यै॒ स्वाहैका॒न्न श॒ताय॒ स्वाहा॑ श॒ताय॒ स्वाहा॒ सर्व॑स्मै॒ स्वाहा᳚ ॥ \newline

\textbf{Pada Paata} \newline

एक॑स्मै । स्वाहा᳚ । त्रि॒भ्य इति॑ त्रि - भ्यः । स्वाहा᳚ । प॒ञ्चभ्य॒ इति॑ प॒ञ्च - भ्यः॒ । स्वाहा᳚ । स॒प्तभ्य॒ इति॑ स॒प्त - भ्यः॒ । स्वाहा᳚ । न॒वभ्य॒ इति॑ न॒व - भ्यः॒ । स्वाहा᳚ । ए॒का॒द॒शभ्य॒ इत्ये॑काद॒श-भ्यः॒ । स्वाहा᳚ । त्र॒यो॒द॒शभ्य॒ इति॑ त्रयोद॒श - भ्यः॒ । स्वाहा᳚ । प॒ञ्च॒द॒शभ्य॒ इति॑ पञ्चद॒श - भ्यः॒ । स्वाहा᳚ । स॒प्त॒द॒शभ्य॒ इति॑ सप्तद॒श - भ्यः॒ । स्वाहा᳚ । एका᳚त् । न । विꣳ॒॒श॒त्यै । स्वाहा᳚ । नव॑विꣳशत्या॒ इति॒ नव॑ - विꣳ॒॒श॒त्यै॒ । स्वाहा᳚ । एका᳚त् । न । च॒त्वा॒रिꣳ॒॒शते᳚ । स्वाहा᳚ । नव॑चत्वारिꣳशत॒ इति॒ नव॑-च॒त्वा॒रिꣳ॒॒श॒ते॒ । स्वाहा᳚ । एका᳚त् । न । ष॒ष्ट्यै । स्वाहा᳚ । नव॑षष्ट्या॒ इति॒ नव॑-ष॒ष्ट्यै॒ । स्वाहा᳚ । एका᳚त् । न । अ॒शी॒त्यै । स्वाहा᳚ । नवा॑शीत्या॒ इति॒ नव॑ - अ॒शी॒त्यै॒ । स्वाहा᳚ । एका᳚त् । न । श॒ताय॑ । स्वाहा᳚ । श॒ताय॑ । स्वाहा᳚ । सर्व॑स्मै । स्वाहा᳚ ॥  \newline


\textbf{Krama Paata} \newline

एक॑स्मै॒ स्वाहा᳚ । स्वाहा᳚ त्रि॒भ्यः । त्रि॒भ्यः स्वाहा᳚ । त्रि॒भ्य इति॑ त्रि - भ्यः । स्वाहा॑ प॒ञ्चभ्यः॑ । प॒ञ्चभ्यः॒ स्वाहा᳚ । प॒ञ्चभ्य॒ इति॑ प॒ञ्च - भ्यः॒ । स्वाहा॑ स॒प्तभ्यः॑ । स॒प्तभ्यः॒ स्वाहा᳚ । स॒प्तभ्य॒ इति॑ स॒प्त - भ्यः॒ । स्वाहा॑ न॒वभ्यः॑ । न॒वभ्यः॒ स्वाहा᳚ । न॒वभ्य॒ इति॑ न॒व - भ्यः॒ । स्वाहै॑काद॒शभ्यः॑ । ए॒का॒द॒शभ्यः॒ स्वाहा᳚ । ए॒का॒द॒शभ्य॒ इत्ये॑काद॒श - भ्यः॒ । स्वाहा᳚ त्रयोद॒शभ्यः॑ । त्र॒यो॒द॒शभ्यः॒ स्वाहा᳚ । त्र॒यो॒द॒शभ्य॒ इति॑ त्रयोद॒श - भ्यः॒ । स्वाहा॑ पञ्चद॒शभ्यः॑ । प॒ञ्च॒द॒शभ्यः॒ स्वाहा᳚ । प॒ञ्च॒द॒शभ्य॒ इति॑ पञ्चद॒श - भ्यः॒ । स्वाहा॑ सप्तद॒शभ्यः॑ । स॒प्त॒द॒शभ्यः॒ स्वाहा᳚ । स॒प्त॒द॒शभ्य॒ इति॑ सप्तद॒श - भ्यः॒ । स्वाहैका᳚त् । एका॒न् न । न विꣳ॑श॒त्यै । विꣳ॒॒श॒त्यै स्वाहा᳚ । स्वाहा॒ नव॑विꣳशत्यै । नव॑विꣳशत्यै॒ स्वाहा᳚ । नव॑विꣳशत्या॒ इति॒ नव॑ - विꣳ॒॒श॒त्यै॒ । स्वाहैका᳚त् । एका॒न् न । न च॑त्वारिꣳ॒॒शते᳚ । च॒त्वा॒रिꣳ॒॒शते॒ स्वाहा᳚ । स्वाहा॒ नव॑चत्वारिꣳशते । नव॑चत्वारिꣳशते॒ स्वाहा᳚ । नव॑चत्वारिꣳशत॒ इति॒ नव॑ - च॒त्वा॒रिꣳ॒॒श॒ते॒ । स्वाहैका᳚त् । एका॒न् न । न ष॒ष्ट्‍यै । ष॒ष्ट्‍यै स्वाहा᳚ । स्वाहा॒ नव॑षष्ट्‍यै । नव॑षष्ट्‍यै॒ स्वाहा᳚ । नव॑षष्ट्‍या॒ इति॒ नव॑ - ष॒ष्ट्‍यै॒ । स्वाहैका᳚त् । एका॒न् न । नाशी॒त्यै । अ॒शी॒त्यै स्वाहा᳚ । स्वाहा॒ नवा॑शीत्यै । नवा॑शीत्यै॒ स्वाहा᳚ । नवा॑शीत्या॒ इति॒ नव॑ - अ॒शी॒त्यै॒ । स्वाहैका᳚त् । एका॒न् न । न श॒ताय॑ । श॒ताय॒ स्वाहा᳚ । स्वाहा॑ श॒ताय॑ । श॒ताय॒ स्वाहा᳚ । स्वाहा॒ सर्व॑स्मै । सर्व॑स्मै॒ स्वाहा᳚ । स्वाहेति॒ स्वाहा᳚ । \newline

\textbf{Jatai Paata} \newline

1. एक॑स्मै॒ स्वाहा॒ स्वाहैक॑स्मा॒ एक॑स्मै॒ स्वाहा᳚ । \newline
2. स्वाहा᳚ त्रि॒भ्य स्त्रि॒भ्यः स्वाहा॒ स्वाहा᳚ त्रि॒भ्यः । \newline
3. त्रि॒भ्यः स्वाहा॒ स्वाहा᳚ त्रि॒भ्य स्त्रि॒भ्यः स्वाहा᳚ । \newline
4. त्रि॒भ्य इति॑ त्रि - भ्यः । \newline
5. स्वाहा॑ प॒ञ्चभ्यः॑ प॒ञ्चभ्यः॒ स्वाहा॒ स्वाहा॑ प॒ञ्चभ्यः॑ । \newline
6. प॒ञ्चभ्यः॒ स्वाहा॒ स्वाहा॑ प॒ञ्चभ्यः॑ प॒ञ्चभ्यः॒ स्वाहा᳚ । \newline
7. प॒ञ्चभ्य॒ इति॑ प॒ञ्च - भ्यः॒ । \newline
8. स्वाहा॑ स॒प्तभ्यः॑ स॒प्तभ्यः॒ स्वाहा॒ स्वाहा॑ स॒प्तभ्यः॑ । \newline
9. स॒प्तभ्यः॒ स्वाहा॒ स्वाहा॑ स॒प्तभ्यः॑ स॒प्तभ्यः॒ स्वाहा᳚ । \newline
10. स॒प्तभ्य॒ इति॑ स॒प्त - भ्यः॒ । \newline
11. स्वाहा॑ न॒वभ्यो॑ न॒वभ्यः॒ स्वाहा॒ स्वाहा॑ न॒वभ्यः॑ । \newline
12. न॒वभ्यः॒ स्वाहा॒ स्वाहा॑ न॒वभ्यो॑ न॒वभ्यः॒ स्वाहा᳚ । \newline
13. न॒वभ्य॒ इति॑ न॒व - भ्यः॒ । \newline
14. स्वाहै॑काद॒शभ्य॑ एकाद॒शभ्यः॒ स्वाहा॒ स्वाहै॑काद॒शभ्यः॑ । \newline
15. ए॒का॒द॒शभ्यः॒ स्वाहा॒ स्वाहै॑काद॒शभ्य॑ एकाद॒शभ्यः॒ स्वाहा᳚ । \newline
16. ए॒का॒द॒शभ्य॒ इत्ये॑काद॒श - भ्यः॒ । \newline
17. स्वाहा᳚ त्रयोद॒शभ्य॑ स्त्रयोद॒शभ्यः॒ स्वाहा॒ स्वाहा᳚ त्रयोद॒शभ्यः॑ । \newline
18. त्र॒यो॒द॒शभ्यः॒ स्वाहा॒ स्वाहा᳚ त्रयोद॒शभ्य॑ स्त्रयोद॒शभ्यः॒ स्वाहा᳚ । \newline
19. त्र॒यो॒द॒शभ्य॒ इति॑ त्रयोद॒श - भ्यः॒ । \newline
20. स्वाहा॑ पञ्चद॒शभ्यः॑ पञ्चद॒शभ्यः॒ स्वाहा॒ स्वाहा॑ पञ्चद॒शभ्यः॑ । \newline
21. प॒ञ्च॒द॒शभ्यः॒ स्वाहा॒ स्वाहा॑ पञ्चद॒शभ्यः॑ पञ्चद॒शभ्यः॒ स्वाहा᳚ । \newline
22. प॒ञ्च॒द॒शभ्य॒ इति॑ पञ्चद॒श - भ्यः॒ । \newline
23. स्वाहा॑ सप्तद॒शभ्यः॑ सप्तद॒शभ्यः॒ स्वाहा॒ स्वाहा॑ सप्तद॒शभ्यः॑ । \newline
24. स॒प्त॒द॒शभ्यः॒ स्वाहा॒ स्वाहा॑ सप्तद॒शभ्यः॑ सप्तद॒शभ्यः॒ स्वाहा᳚ । \newline
25. स॒प्त॒द॒शभ्य॒ इति॑ सप्तद॒श - भ्यः॒ । \newline
26. स्वाहैका॒ देका॒थ् स्वाहा॒ स्वाहैका᳚त् । \newline
27. एका॒न् न नैका॒ देका॒न् न । \newline
28. न विꣳ॑श॒त्यै विꣳ॑श॒त्यै न न विꣳ॑श॒त्यै । \newline
29. विꣳ॒॒श॒त्यै स्वाहा॒ स्वाहा॑ विꣳश॒त्यै विꣳ॑श॒त्यै स्वाहा᳚ । \newline
30. स्वाहा॒ नव॑विꣳशत्यै॒ नव॑विꣳशत्यै॒ स्वाहा॒ स्वाहा॒ नव॑विꣳशत्यै । \newline
31. नव॑विꣳशत्यै॒ स्वाहा॒ स्वाहा॒ नव॑विꣳशत्यै॒ नव॑विꣳशत्यै॒ स्वाहा᳚ । \newline
32. नव॑विꣳशत्या॒ इति॒ नव॑ - विꣳ॒॒श॒त्यै॒ । \newline
33. स्वाहैका॒ देका॒थ् स्वाहा॒ स्वाहैका᳚त् । \newline
34. एका॒न् न नैका॒ देका॒न् न । \newline
35. न च॑त्वारिꣳ॒॒शते॑ चत्वारिꣳ॒॒शते॒ न न च॑त्वारिꣳ॒॒शते᳚ । \newline
36. च॒त्वा॒रिꣳ॒॒शते॒ स्वाहा॒ स्वाहा॑ चत्वारिꣳ॒॒शते॑ चत्वारिꣳ॒॒शते॒ स्वाहा᳚ । \newline
37. स्वाहा॒ नव॑चत्वारिꣳशते॒ नव॑चत्वारिꣳशते॒ स्वाहा॒ स्वाहा॒ नव॑चत्वारिꣳशते । \newline
38. नव॑चत्वारिꣳशते॒ स्वाहा॒ स्वाहा॒ नव॑चत्वारिꣳशते॒ नव॑चत्वारिꣳशते॒ स्वाहा᳚ । \newline
39. नव॑चत्वारिꣳशत॒ इति॒ नव॑ - च॒त्वा॒रिꣳ॒॒श॒ते॒ । \newline
40. स्वाहैका॒ देका॒थ् स्वाहा॒ स्वाहैका᳚त् । \newline
41. एका॒न् न नैका॒ देका॒न् न । \newline
42. न ष॒ष्ट्यै ष॒ष्ट्यै न न ष॒ष्ट्यै । \newline
43. ष॒ष्ट्यै स्वाहा॒ स्वाहा॑ ष॒ष्ट्यै ष॒ष्ट्यै स्वाहा᳚ । \newline
44. स्वाहा॒ नव॑षष्ट्यै॒ नव॑षष्ट्यै॒ स्वाहा॒ स्वाहा॒ नव॑षष्ट्यै । \newline
45. नव॑षष्ट्यै॒ स्वाहा॒ स्वाहा॒ नव॑षष्ट्यै॒ नव॑षष्ट्यै॒ स्वाहा᳚ । \newline
46. नव॑षष्ट्या॒ इति॒ नव॑ - ष॒ष्ट्यै॒ । \newline
47. स्वाहैका॒ देका॒थ् स्वाहा॒ स्वाहैका᳚त् । \newline
48. एका॒न् न नैका॒ देका॒न् न । \newline
49. नाशी॒त्या अ॑शी॒त्यै न नाशी॒त्यै । \newline
50. अ॒शी॒त्यै स्वाहा॒ स्वाहा॑ ऽशी॒त्या अ॑शी॒त्यै स्वाहा᳚ । \newline
51. स्वाहा॒ नवा॑शीत्यै॒ नवा॑शीत्यै॒ स्वाहा॒ स्वाहा॒ नवा॑शीत्यै । \newline
52. नवा॑शीत्यै॒ स्वाहा॒ स्वाहा॒ नवा॑शीत्यै॒ नवा॑शीत्यै॒ स्वाहा᳚ । \newline
53. नवा॑शीत्या॒ इति॒ नव॑ - अ॒शी॒त्यै॒ । \newline
54. स्वाहैका॒ देका॒थ् स्वाहा॒ स्वाहैका᳚त् । \newline
55. एका॒न् न नैका॒ देका॒न् न । \newline
56. न श॒ताय॑ श॒ताय॒ न न श॒ताय॑ । \newline
57. श॒ताय॒ स्वाहा॒ स्वाहा॑ श॒ताय॑ श॒ताय॒ स्वाहा᳚ । \newline
58. स्वाहा॑ श॒ताय॑ श॒ताय॒ स्वाहा॒ स्वाहा॑ श॒ताय॑ । \newline
59. श॒ताय॒ स्वाहा॒ स्वाहा॑ श॒ताय॑ श॒ताय॒ स्वाहा᳚ । \newline
60. स्वाहा॒ सर्व॑स्मै॒ सर्व॑स्मै॒ स्वाहा॒ स्वाहा॒ सर्व॑स्मै । \newline
61. सर्व॑स्मै॒ स्वाहा॒ स्वाहा॒ सर्व॑स्मै॒ सर्व॑स्मै॒ स्वाहा᳚ । \newline
62. स्वाहेति॒ स्वाहा᳚ । \newline

\textbf{Ghana Paata } \newline

1. एक॑स्मै॒ स्वाहा॒ स्वाहैक॑स्मा॒ एक॑स्मै॒ स्वाहा᳚ त्रि॒भ्य स्त्रि॒भ्यः स्वाहैक॑स्मा॒ एक॑स्मै॒ स्वाहा᳚ त्रि॒भ्यः । \newline
2. स्वाहा᳚ त्रि॒भ्य स्त्रि॒भ्यः स्वाहा॒ स्वाहा᳚ त्रि॒भ्यः स्वाहा॒ स्वाहा᳚ त्रि॒भ्यः स्वाहा॒ स्वाहा᳚ त्रि॒भ्यः स्वाहा᳚ । \newline
3. त्रि॒भ्यः स्वाहा॒ स्वाहा᳚ त्रि॒भ्य स्त्रि॒भ्यः स्वाहा॑ प॒ञ्चभ्यः॑ प॒ञ्चभ्यः॒ स्वाहा᳚ त्रि॒भ्य स्त्रि॒भ्यः स्वाहा॑ प॒ञ्चभ्यः॑ । \newline
4. त्रि॒भ्य इति॑ त्रि - भ्यः । \newline
5. स्वाहा॑ प॒ञ्चभ्यः॑ प॒ञ्चभ्यः॒ स्वाहा॒ स्वाहा॑ प॒ञ्चभ्यः॒ स्वाहा॒ स्वाहा॑ प॒ञ्चभ्यः॒ स्वाहा॒ स्वाहा॑ प॒ञ्चभ्यः॒ स्वाहा᳚ । \newline
6. प॒ञ्चभ्यः॒ स्वाहा॒ स्वाहा॑ प॒ञ्चभ्यः॑ प॒ञ्चभ्यः॒ स्वाहा॑ स॒प्तभ्यः॑ स॒प्तभ्यः॒ स्वाहा॑ प॒ञ्चभ्यः॑ प॒ञ्चभ्यः॒ स्वाहा॑ स॒प्तभ्यः॑ । \newline
7. प॒ञ्चभ्य॒ इति॑ प॒ञ्च - भ्यः॒ । \newline
8. स्वाहा॑ स॒प्तभ्यः॑ स॒प्तभ्यः॒ स्वाहा॒ स्वाहा॑ स॒प्तभ्यः॒ स्वाहा॒ स्वाहा॑ स॒प्तभ्यः॒ स्वाहा॒ स्वाहा॑ स॒प्तभ्यः॒ स्वाहा᳚ । \newline
9. स॒प्तभ्यः॒ स्वाहा॒ स्वाहा॑ स॒प्तभ्यः॑ स॒प्तभ्यः॒ स्वाहा॑ न॒वभ्यो॑ न॒वभ्यः॒ स्वाहा॑ स॒प्तभ्यः॑ स॒प्तभ्यः॒ स्वाहा॑ न॒वभ्यः॑ । \newline
10. स॒प्तभ्य॒ इति॑ स॒प्त - भ्यः॒ । \newline
11. स्वाहा॑ न॒वभ्यो॑ न॒वभ्यः॒ स्वाहा॒ स्वाहा॑ न॒वभ्यः॒ स्वाहा॒ स्वाहा॑ न॒वभ्यः॒ स्वाहा॒ स्वाहा॑ न॒वभ्यः॒ स्वाहा᳚ । \newline
12. न॒वभ्यः॒ स्वाहा॒ स्वाहा॑ न॒वभ्यो॑ न॒वभ्यः॒ स्वाहै॑काद॒शभ्य॑ एकाद॒शभ्यः॒ स्वाहा॑ न॒वभ्यो॑ न॒वभ्यः॒ स्वाहै॑काद॒शभ्यः॑ । \newline
13. न॒वभ्य॒ इति॑ न॒व - भ्यः॒ । \newline
14. स्वाहै॑काद॒शभ्य॑ एकाद॒शभ्यः॒ स्वाहा॒ स्वाहै॑काद॒शभ्यः॒ स्वाहा॒ स्वाहै॑काद॒शभ्यः॒ स्वाहा॒ 
स्वाहै॑काद॒शभ्यः॒ स्वाहा᳚ । \newline
15. ए॒का॒द॒शभ्यः॒ स्वाहा॒ स्वाहै॑काद॒शभ्य॑ एकाद॒शभ्यः॒ स्वाहा᳚ त्रयोद॒शभ्य॑ स्त्रयोद॒शभ्यः॒ स्वाहै॑काद॒शभ्य॑ एकाद॒शभ्यः॒ स्वाहा᳚ त्रयोद॒शभ्यः॑ । \newline
16. ए॒का॒द॒शभ्य॒ इत्ये॑काद॒श - भ्यः॒ । \newline
17. स्वाहा᳚ त्रयोद॒शभ्य॑ स्त्रयोद॒शभ्यः॒ स्वाहा॒ स्वाहा᳚ त्रयोद॒शभ्यः॒ स्वाहा॒ स्वाहा᳚ त्रयोद॒शभ्यः॒ स्वाहा॒ स्वाहा᳚ त्रयोद॒शभ्यः॒ स्वाहा᳚ । \newline
18. त्र॒यो॒द॒शभ्यः॒ स्वाहा॒ स्वाहा᳚ त्रयोद॒शभ्य॑ स्त्रयोद॒शभ्यः॒ स्वाहा॑ पञ्चद॒शभ्यः॑ पञ्चद॒शभ्यः॒ स्वाहा᳚ त्रयोद॒शभ्य॑ स्त्रयोद॒शभ्यः॒ स्वाहा॑ पञ्चद॒शभ्यः॑ । \newline
19. त्र॒यो॒द॒शभ्य॒ इति॑ त्रयोद॒श - भ्यः॒ । \newline
20. स्वाहा॑ पञ्चद॒शभ्यः॑ पञ्चद॒शभ्यः॒ स्वाहा॒ स्वाहा॑ पञ्चद॒शभ्यः॒ स्वाहा॒ स्वाहा॑ पञ्चद॒शभ्यः॒ स्वाहा॒ स्वाहा॑ पञ्चद॒शभ्यः॒ स्वाहा᳚ । \newline
21. प॒ञ्च॒द॒शभ्यः॒ स्वाहा॒ स्वाहा॑ पञ्चद॒शभ्यः॑ पञ्चद॒शभ्यः॒ स्वाहा॑ सप्तद॒शभ्यः॑ सप्तद॒शभ्यः॒ स्वाहा॑ पञ्चद॒शभ्यः॑ पञ्चद॒शभ्यः॒ स्वाहा॑ सप्तद॒शभ्यः॑ । \newline
22. प॒ञ्च॒द॒शभ्य॒ इति॑ पञ्चद॒श - भ्यः॒ । \newline
23. स्वाहा॑ सप्तद॒शभ्यः॑ सप्तद॒शभ्यः॒ स्वाहा॒ स्वाहा॑ सप्तद॒शभ्यः॒ स्वाहा॒ स्वाहा॑ सप्तद॒शभ्यः॒ स्वाहा॒ स्वाहा॑ सप्तद॒शभ्यः॒ स्वाहा᳚ । \newline
24. स॒प्त॒द॒शभ्यः॒ स्वाहा॒ स्वाहा॑ सप्तद॒शभ्यः॑ सप्तद॒शभ्यः॒ स्वाहैका॒ देका॒थ् स्वाहा॑ सप्तद॒शभ्यः॑ सप्तद॒शभ्यः॒ स्वाहैका᳚त् । \newline
25. स॒प्त॒द॒शभ्य॒ इति॑ सप्तद॒श - भ्यः॒ । \newline
26. स्वाहैका॒ देका॒थ् स्वाहा॒ स्वाहैका॒न् न नैका॒थ् स्वाहा॒ स्वाहैका॒न् न । \newline
27. एका॒न् न नैका॒ देका॒न् न विꣳ॑श॒त्यै विꣳ॑श॒त्यै नैका॒ देका॒न् न विꣳ॑श॒त्यै । \newline
28. न विꣳ॑श॒त्यै विꣳ॑श॒त्यै न न विꣳ॑श॒त्यै स्वाहा॒ स्वाहा॑ विꣳश॒त्यै न न विꣳ॑श॒त्यै स्वाहा᳚ । \newline
29. विꣳ॒॒श॒त्यै स्वाहा॒ स्वाहा॑ विꣳश॒त्यै विꣳ॑श॒त्यै स्वाहा॒ नव॑विꣳशत्यै॒ नव॑विꣳशत्यै॒ स्वाहा॑ विꣳश॒त्यै विꣳ॑श॒त्यै स्वाहा॒ नव॑विꣳशत्यै । \newline
30. स्वाहा॒ नव॑विꣳशत्यै॒ नव॑विꣳशत्यै॒ स्वाहा॒ स्वाहा॒ नव॑विꣳशत्यै॒ स्वाहा॒ स्वाहा॒ नव॑विꣳशत्यै॒ स्वाहा॒ स्वाहा॒ नव॑विꣳशत्यै॒ स्वाहा᳚ । \newline
31. नव॑विꣳशत्यै॒ स्वाहा॒ स्वाहा॒ नव॑विꣳशत्यै॒ नव॑विꣳशत्यै॒ स्वाहैका॒ देका॒थ् स्वाहा॒ नव॑विꣳशत्यै॒ नव॑विꣳशत्यै॒ स्वाहैका᳚त् । \newline
32. नव॑विꣳशत्या॒ इति॒ नव॑ - विꣳ॒॒श॒त्यै॒ । \newline
33. स्वाहैका॒ देका॒थ् स्वाहा॒ स्वाहैका॒न् न नैका॒थ् स्वाहा॒ स्वाहैका॒न् न । \newline
34. एका॒न् न नैका॒ देका॒न् न च॑त्वारिꣳ॒॒शते॑ चत्वारिꣳ॒॒शते॒ नैका॒ देका॒न् न च॑त्वारिꣳ॒॒शते᳚ । \newline
35. न च॑त्वारिꣳ॒॒शते॑ चत्वारिꣳ॒॒शते॒ न न च॑त्वारिꣳ॒॒शते॒ स्वाहा॒ स्वाहा॑ चत्वारिꣳ॒॒शते॒ न न च॑त्वारिꣳ॒॒शते॒ स्वाहा᳚ । \newline
36. च॒त्वा॒रिꣳ॒॒शते॒ स्वाहा॒ स्वाहा॑ चत्वारिꣳ॒॒शते॑ चत्वारिꣳ॒॒शते॒ स्वाहा॒ नव॑चत्वारिꣳशते॒ नव॑चत्वारिꣳशते॒ स्वाहा॑ चत्वारिꣳ॒॒शते॑ चत्वारिꣳ॒॒शते॒ स्वाहा॒ नव॑चत्वारिꣳशते । \newline
37. स्वाहा॒ नव॑चत्वारिꣳशते॒ नव॑चत्वारिꣳशते॒ स्वाहा॒ स्वाहा॒ नव॑चत्वारिꣳशते॒ स्वाहा॒ स्वाहा॒ नव॑चत्वारिꣳशते॒ स्वाहा॒ स्वाहा॒ नव॑चत्वारिꣳशते॒ स्वाहा᳚ । \newline
38. नव॑चत्वारिꣳशते॒ स्वाहा॒ स्वाहा॒ नव॑चत्वारिꣳशते॒ नव॑चत्वारिꣳशते॒ स्वाहैका॒ देका॒थ् स्वाहा॒ नव॑चत्वारिꣳशते॒ नव॑चत्वारिꣳशते॒ स्वाहैका᳚त् । \newline
39. नव॑चत्वारिꣳशत॒ इति॒ नव॑ - च॒त्वा॒रिꣳ॒॒श॒ते॒ । \newline
40. स्वाहैका॒ देका॒थ् स्वाहा॒ स्वाहैका॒न् न नैका॒थ् स्वाहा॒ स्वाहैका॒न् न । \newline
41. एका॒न् न नैका॒ देका॒न् न ष॒ष्ट्यै ष॒ष्ट्यै नैका॒ देका॒न् न ष॒ष्ट्यै । \newline
42. न ष॒ष्ट्यै ष॒ष्ट्यै न न ष॒ष्ट्यै स्वाहा॒ स्वाहा॑ ष॒ष्ट्यै न न ष॒ष्ट्यै स्वाहा᳚ । \newline
43. ष॒ष्ट्यै स्वाहा॒ स्वाहा॑ ष॒ष्ट्यै ष॒ष्ट्यै स्वाहा॒ नव॑षष्ट्यै॒ नव॑षष्ट्यै॒ स्वाहा॑ ष॒ष्ट्यै ष॒ष्ट्यै स्वाहा॒ नव॑षष्ट्यै । \newline
44. स्वाहा॒ नव॑षष्ट्यै॒ नव॑षष्ट्यै॒ स्वाहा॒ स्वाहा॒ नव॑षष्ट्यै॒ स्वाहा॒ स्वाहा॒ नव॑षष्ट्यै॒ स्वाहा॒ स्वाहा॒ नव॑षष्ट्यै॒ स्वाहा᳚ । \newline
45. नव॑षष्ट्यै॒ स्वाहा॒ स्वाहा॒ नव॑षष्ट्यै॒ नव॑षष्ट्यै॒ स्वाहैका॒ देका॒थ् स्वाहा॒ नव॑षष्ट्यै॒ नव॑षष्ट्यै॒ स्वाहैका᳚त् । \newline
46. नव॑षष्ट्या॒ इति॒ नव॑ - ष॒ष्ट्यै॒ । \newline
47. स्वाहैका॒ देका॒थ् स्वाहा॒ स्वाहैका॒न् न नैका॒थ् स्वाहा॒ स्वाहैका॒न् न । \newline
48. एका॒न् न नैका॒ देका॒न् नाशी॒त्या अ॑शी॒त्यै नैका॒ देका॒न् नाशी॒त्यै । \newline
49. नाशी॒त्या अ॑शी॒त्यै न नाशी॒त्यै स्वाहा॒ स्वाहा॑ ऽशी॒त्यै न नाशी॒त्यै स्वाहा᳚ । \newline
50. अ॒शी॒त्यै स्वाहा॒ स्वाहा॑ ऽशी॒त्या अ॑शी॒त्यै स्वाहा॒ नवा॑शीत्यै॒ नवा॑शीत्यै॒ स्वाहा॑ ऽशी॒त्या अ॑शी॒त्यै स्वाहा॒ नवा॑शीत्यै । \newline
51. स्वाहा॒ नवा॑शीत्यै॒ नवा॑शीत्यै॒ स्वाहा॒ स्वाहा॒ नवा॑शीत्यै॒ स्वाहा॒ स्वाहा॒ नवा॑शीत्यै॒ स्वाहा॒ स्वाहा॒ नवा॑शीत्यै॒ स्वाहा᳚ । \newline
52. नवा॑शीत्यै॒ स्वाहा॒ स्वाहा॒ नवा॑शीत्यै॒ नवा॑शीत्यै॒ स्वाहैका॒ देका॒थ् स्वाहा॒ नवा॑शीत्यै॒ नवा॑शीत्यै॒ स्वाहैका᳚त् । \newline
53. नवा॑शीत्या॒ इति॒ नव॑ - अ॒शी॒त्यै॒ । \newline
54. स्वाहैका॒ देका॒थ् स्वाहा॒ स्वाहैका॒न् न नैका॒थ् स्वाहा॒ स्वाहैका॒न् न । \newline
55. एका॒न् न नैका॒ देका॒न् न श॒ताय॑ श॒ताय॒ नैका॒ देका॒न् न श॒ताय॑ । \newline
56. न श॒ताय॑ श॒ताय॒ न न श॒ताय॒ स्वाहा॒ स्वाहा॑ श॒ताय॒ न न श॒ताय॒ स्वाहा᳚ । \newline
57. श॒ताय॒ स्वाहा॒ स्वाहा॑ श॒ताय॑ श॒ताय॒ स्वाहा॑ श॒ताय॑ श॒ताय॒ स्वाहा॑ श॒ताय॑ श॒ताय॒ स्वाहा॑ श॒ताय॑ । \newline
58. स्वाहा॑ श॒ताय॑ श॒ताय॒ स्वाहा॒ स्वाहा॑ श॒ताय॒ स्वाहा॒ स्वाहा॑ श॒ताय॒ स्वाहा॒ स्वाहा॑ श॒ताय॒ स्वाहा᳚ । \newline
59. श॒ताय॒ स्वाहा॒ स्वाहा॑ श॒ताय॑ श॒ताय॒ स्वाहा॒ सर्व॑स्मै॒ सर्व॑स्मै॒ स्वाहा॑ श॒ताय॑ श॒ताय॒ स्वाहा॒ सर्व॑स्मै । \newline
60. स्वाहा॒ सर्व॑स्मै॒ सर्व॑स्मै॒ स्वाहा॒ स्वाहा॒ सर्व॑स्मै॒ स्वाहा॒ स्वाहा॒ सर्व॑स्मै॒ स्वाहा॒ स्वाहा॒ सर्व॑स्मै॒ स्वाहा᳚ । \newline
61. सर्व॑स्मै॒ स्वाहा॒ स्वाहा॒ सर्व॑स्मै॒ सर्व॑स्मै॒ स्वाहा᳚ । \newline
62. स्वाहेति॒ स्वाहा᳚ । \newline
\pagebreak
\markright{ TS 7.2.13.1  \hfill https://www.vedavms.in \hfill}

\section{ TS 7.2.13.1 }

\textbf{TS 7.2.13.1 } \newline
\textbf{Samhita Paata} \newline

द्वाभ्याꣳ॒॒ स्वाहा॑ च॒तुर्भ्यः॒ स्वाहा॑ ष॒ड्भ्यः स्वाहा᳚ ऽष्टा॒भ्यः स्वाहा॑ द॒शभ्यः॒ स्वाहा᳚ द्वाद॒शभ्यः॒ स्वाहा॑ चतुर्द॒शभ्यः॒ स्वाहा॑ षोड॒शभ्यः॒ स्वाहा᳚ ऽष्टाद॒शभ्यः॒ स्वाहा॑ विꣳश॒त्यै स्वाहा॒ ऽष्टान॑वत्यै॒ स्वाहा॑ श॒ताय॒ स्वाहा॒ सर्व॑स्मै॒ स्वाहा᳚ ॥ \newline

\textbf{Pada Paata} \newline

द्वाभ्या᳚म् । स्वाहा᳚ । च॒तुर्भ्य॒ इति॑ च॒तुः - भ्यः॒ । स्वाहा᳚ । ष॒ड्भ्य इति॑ षट्- भ्यः । स्वाहा᳚ । अ॒ष्टा॒भ्यः । स्वाहा᳚ । द॒शभ्य॒ इति॑ द॒श-भ्यः॒ । स्वाहा᳚ । द्वा॒द॒शभ्य॒ इति॑ द्वाद॒श - भ्यः॒ । स्वाहा᳚ । च॒तु॒र्द॒शभ्य॒ इति॑ चतुर्द॒श - भ्यः॒ । स्वाहा᳚ । षो॒ड॒शभ्य॒ इति॑ षोड॒श - भ्यः॒ । स्वाहा᳚ । अ॒ष्टा॒द॒शभ्य॒ इत्य॑ष्टाद॒श - भ्यः॒ । स्वाहा᳚ । विꣳ॒॒श॒त्यै । स्वाहा᳚ । अ॒ष्टान॑वत्या॒ इत्य॒ष्टा - न॒व॒त्यै॒ । स्वाहा᳚ । श॒ताय॑ । स्वाहा᳚ । सर्व॑स्मै । स्वाहा᳚ ॥  \newline


\textbf{Krama Paata} \newline

द्वाभ्याꣳ॒॒ स्वाहा᳚ । स्वाहा॑ च॒तुर्भ्यः॑ । च॒तुर्भ्यः॒ स्वाहा᳚ । च॒तुर्भ्य॒ इति॑ च॒तुः - भ्यः॒ । स्वाहा॑ ष॒ड्भ्यः । ष॒ड्भ्यः स्वाहा᳚ । ष॒ड्भ्य इति॑ षट् - भ्यः । स्वाहा᳚ऽष्टा॒भ्यः । अ॒ष्टा॒भ्यः स्वाहा᳚ । स्वाहा॑ द॒शभ्यः॑ । द॒शभ्यः॒ स्वाहा᳚ । द॒शभ्य॒ इति॑ द॒श - भ्यः॒ । स्वाहा᳚ द्वाद॒शभ्यः॑ । द्वा॒द॒शभ्यः॒ स्वाहा᳚ । द्वा॒द॒शभ्य॒ इति॑ द्वाद॒श - भ्यः॒ । स्वाहा॑ चतुर्द॒शभ्यः॑ । च॒तु॒र्द॒शभ्यः॒ स्वाहा᳚ । च॒तु॒र्द॒शभ्य॒ इति॑ चतुर्द॒श - भ्यः॒ । स्वाहा॑ षोड॒शभ्यः॑ । षो॒ड॒शभ्यः॒ स्वाहा᳚ । षो॒ड॒शभ्य॒ इति॑ षोड॒श - भ्यः॒ । स्वाहा᳚ऽष्टाद॒शभ्यः॑ । अ॒ष्टा॒द॒शभ्यः॒ स्वाहा᳚ । अ॒ष्टा॒द॒शभ्य॒ इत्य॑ष्टाद॒श - भ्यः॒ । स्वाहा॑ विꣳश॒त्यै । विꣳ॒॒श॒त्यै स्वाहा᳚ । स्वाहा॒ऽष्टान॑वत्यै । अ॒ष्टान॑वत्यै॒ स्वाहा᳚ । अ॒ष्टान॑वत्या॒ इत्य॒ष्टा - न॒व॒त्यै॒ । स्वाहा॑ श॒ताय॑ । श॒ताय॒ स्वाहा᳚ । स्वाहा॒ सर्व॑स्मै । सर्व॑स्मै॒ स्वाहा᳚ । स्वाहेति॒ स्वाहा᳚ । \newline

\textbf{Jatai Paata} \newline

1. द्वाभ्याꣳ॒॒ स्वाहा॒ स्वाहा॒ द्वाभ्या॒म् द्वाभ्याꣳ॒॒ स्वाहा᳚ । \newline
2. स्वाहा॑ च॒तुर्भ्य॑ श्च॒तुर्भ्यः॒ स्वाहा॒ स्वाहा॑ च॒तुर्भ्यः॑ । \newline
3. च॒तुर्भ्यः॒ स्वाहा॒ स्वाहा॑ च॒तुर्भ्य॑ श्च॒तुर्भ्यः॒ स्वाहा᳚ । \newline
4. च॒तुर्भ्य॒ इति॑ च॒तुः - भ्यः॒ । \newline
5. स्वाहा॑ ष॒ड्भ्य ष्ष॒ड्भ्यः स्वाहा॒ स्वाहा॑ ष॒ड्भ्यः । \newline
6. ष॒ड्भ्यः स्वाहा॒ स्वाहा॑ ष॒ड्भ्य ष्ष॒ड्भ्यः स्वाहा᳚ । \newline
7. ष॒ड्भ्य इति॑ षट् - भ्यः । \newline
8. स्वाहा᳚ ऽष्टा॒भ्यो᳚ ऽष्टा॒भ्यः स्वाहा॒ स्वाहा᳚ ऽष्टा॒भ्यः । \newline
9. अ॒ष्टा॒भ्यः स्वाहा॒ स्वाहा᳚ ऽष्टा॒भ्यो᳚ ऽष्टा॒भ्यः स्वाहा᳚ । \newline
10. स्वाहा॑ द॒शभ्यो॑ द॒शभ्यः॒ स्वाहा॒ स्वाहा॑ द॒शभ्यः॑ । \newline
11. द॒शभ्यः॒ स्वाहा॒ स्वाहा॑ द॒शभ्यो॑ द॒शभ्यः॒ स्वाहा᳚ । \newline
12. द॒शभ्य॒ इति॑ द॒श - भ्यः॒ । \newline
13. स्वाहा᳚ द्वाद॒शभ्यो᳚ द्वाद॒शभ्यः॒ स्वाहा॒ स्वाहा᳚ द्वाद॒शभ्यः॑ । \newline
14. द्वा॒द॒शभ्यः॒ स्वाहा॒ स्वाहा᳚ द्वाद॒शभ्यो᳚ द्वाद॒शभ्यः॒ स्वाहा᳚ । \newline
15. द्वा॒द॒शभ्य॒ इति॑ द्वाद॒श - भ्यः॒ । \newline
16. स्वाहा॑ चतुर्द॒शभ्य॑ श्चतुर्द॒शभ्यः॒ स्वाहा॒ स्वाहा॑ चतुर्द॒शभ्यः॑ । \newline
17. च॒तु॒र्द॒शभ्यः॒ स्वाहा॒ स्वाहा॑ चतुर्द॒शभ्य॑ श्चतुर्द॒शभ्यः॒ स्वाहा᳚ । \newline
18. च॒तु॒र्द॒शभ्य॒ इति॑ चतुर्द॒श - भ्यः॒ । \newline
19. स्वाहा॑ षोड॒शभ्य॑ ष्षोड॒शभ्यः॒ स्वाहा॒ स्वाहा॑ षोड॒शभ्यः॑ । \newline
20. षो॒ड॒शभ्यः॒ स्वाहा॒ स्वाहा॑ षोड॒शभ्य॑ ष्षोड॒शभ्यः॒ स्वाहा᳚ । \newline
21. षो॒ड॒शभ्य॒ इति॑ षोड॒श - भ्यः॒ । \newline
22. स्वाहा᳚ ऽष्टाद॒शभ्यो᳚ ऽष्टाद॒शभ्यः॒ स्वाहा॒ स्वाहा᳚ ऽष्टाद॒शभ्यः॑ । \newline
23. अ॒ष्टा॒द॒शभ्यः॒ स्वाहा॒ स्वाहा᳚ ऽष्टाद॒शभ्यो᳚ ऽष्टाद॒शभ्यः॒ स्वाहा᳚ । \newline
24. अ॒ष्टा॒द॒शभ्य॒ इत्य॑ष्टाद॒श - भ्यः॒ । \newline
25. स्वाहा॑ विꣳश॒त्यै विꣳ॑श॒त्यै स्वाहा॒ स्वाहा॑ विꣳश॒त्यै । \newline
26. विꣳ॒॒श॒त्यै स्वाहा॒ स्वाहा॑ विꣳश॒त्यै विꣳ॑श॒त्यै स्वाहा᳚ । \newline
27. स्वाहा॒ ऽष्टान॑वत्या अ॒ष्टान॑वत्यै॒ स्वाहा॒ स्वाहा॒ ऽष्टान॑वत्यै । \newline
28. अ॒ष्टान॑वत्यै॒ स्वाहा॒ स्वाहा॒ ऽष्टान॑वत्या अ॒ष्टान॑वत्यै॒ स्वाहा᳚ । \newline
29. अ॒ष्टान॑वत्या॒ इत्य॒ष्टा - न॒व॒त्यै॒ । \newline
30. स्वाहा॑ श॒ताय॑ श॒ताय॒ स्वाहा॒ स्वाहा॑ श॒ताय॑ । \newline
31. श॒ताय॒ स्वाहा॒ स्वाहा॑ श॒ताय॑ श॒ताय॒ स्वाहा᳚ । \newline
32. स्वाहा॒ सर्व॑स्मै॒ सर्व॑स्मै॒ स्वाहा॒ स्वाहा॒ सर्व॑स्मै । \newline
33. सर्व॑स्मै॒ स्वाहा॒ स्वाहा॒ सर्व॑स्मै॒ सर्व॑स्मै॒ स्वाहा᳚ । \newline
34. स्वाहेति॒ स्वाहा᳚ । \newline

\textbf{Ghana Paata } \newline

1. द्वाभ्याꣳ॒॒ स्वाहा॒ स्वाहा॒ द्वाभ्या॒म् द्वाभ्याꣳ॒॒ स्वाहा॑ च॒तुर्भ्य॑ श्च॒तुर्भ्यः॒ स्वाहा॒ द्वाभ्या॒म् द्वाभ्याꣳ॒॒ स्वाहा॑ च॒तुर्भ्यः॑ । \newline
2. स्वाहा॑ च॒तुर्भ्य॑ श्च॒तुर्भ्यः॒ स्वाहा॒ स्वाहा॑ च॒तुर्भ्यः॒ स्वाहा॒ स्वाहा॑ च॒तुर्भ्यः॒ स्वाहा॒ स्वाहा॑ च॒तुर्भ्यः॒ स्वाहा᳚ । \newline
3. च॒तुर्भ्यः॒ स्वाहा॒ स्वाहा॑ च॒तुर्भ्य॑ श्च॒तुर्भ्यः॒ स्वाहा॑ ष॒ड्भ्य ष्ष॒ड्भ्यः स्वाहा॑ च॒तुर्भ्य॑ श्च॒तुर्भ्यः॒ स्वाहा॑ ष॒ड्भ्यः । \newline
4. च॒तुर्भ्य॒ इति॑ च॒तुः - भ्यः॒ । \newline
5. स्वाहा॑ ष॒ड्भ्य ष्ष॒ड्भ्यः स्वाहा॒ स्वाहा॑ ष॒ड्भ्यः स्वाहा॒ स्वाहा॑ ष॒ड्भ्यः स्वाहा॒ स्वाहा॑ ष॒ड्भ्यः स्वाहा᳚ । \newline
6. ष॒ड्भ्यः स्वाहा॒ स्वाहा॑ ष॒ड्भ्य ष्ष॒ड्भ्यः स्वाहा᳚ ऽष्टा॒भ्यो᳚ ऽष्टा॒भ्यः स्वाहा॑ ष॒ड्भ्य ष्ष॒ड्भ्यः स्वाहा᳚ ऽष्टा॒भ्यः । \newline
7. ष॒ड्भ्य इति॑ षट् - भ्यः । \newline
8. स्वाहा᳚ ऽष्टा॒भ्यो᳚ ऽष्टा॒भ्यः स्वाहा॒ स्वाहा᳚ ऽष्टा॒भ्यः स्वाहा॒ स्वाहा᳚ ऽष्टा॒भ्यः स्वाहा॒ स्वाहा᳚ ऽष्टा॒भ्यः स्वाहा᳚ । \newline
9. अ॒ष्टा॒भ्यः स्वाहा॒ स्वाहा᳚ ऽष्टा॒भ्यो᳚ ऽष्टा॒भ्यः स्वाहा॑ द॒शभ्यो॑ द॒शभ्यः॒ स्वाहा᳚ ऽष्टा॒भ्यो᳚ ऽष्टा॒भ्यः स्वाहा॑ द॒शभ्यः॑ । \newline
10. स्वाहा॑ द॒शभ्यो॑ द॒शभ्यः॒ स्वाहा॒ स्वाहा॑ द॒शभ्यः॒ स्वाहा॒ स्वाहा॑ द॒शभ्यः॒ स्वाहा॒ स्वाहा॑ द॒शभ्यः॒ स्वाहा᳚ । \newline
11. द॒शभ्यः॒ स्वाहा॒ स्वाहा॑ द॒शभ्यो॑ द॒शभ्यः॒ स्वाहा᳚ द्वाद॒शभ्यो᳚ द्वाद॒शभ्यः॒ स्वाहा॑ द॒शभ्यो॑ द॒शभ्यः॒ स्वाहा᳚ द्वाद॒शभ्यः॑ । \newline
12. द॒शभ्य॒ इति॑ द॒श - भ्यः॒ । \newline
13. स्वाहा᳚ द्वाद॒शभ्यो᳚ द्वाद॒शभ्यः॒ स्वाहा॒ स्वाहा᳚ द्वाद॒शभ्यः॒ स्वाहा॒ स्वाहा᳚ द्वाद॒शभ्यः॒ स्वाहा॒ स्वाहा᳚ द्वाद॒शभ्यः॒ स्वाहा᳚ । \newline
14. द्वा॒द॒शभ्यः॒ स्वाहा॒ स्वाहा᳚ द्वाद॒शभ्यो᳚ द्वाद॒शभ्यः॒ स्वाहा॑ चतुर्द॒शभ्य॑ श्चतुर्द॒शभ्यः॒ स्वाहा᳚ द्वाद॒शभ्यो᳚ द्वाद॒शभ्यः॒ स्वाहा॑ चतुर्द॒शभ्यः॑ । \newline
15. द्वा॒द॒शभ्य॒ इति॑ द्वाद॒श - भ्यः॒ । \newline
16. स्वाहा॑ चतुर्द॒शभ्य॑ श्चतुर्द॒शभ्यः॒ स्वाहा॒ स्वाहा॑ चतुर्द॒शभ्यः॒ स्वाहा॒ स्वाहा॑ चतुर्द॒शभ्यः॒ स्वाहा॒ स्वाहा॑ चतुर्द॒शभ्यः॒ स्वाहा᳚ । \newline
17. च॒तु॒र्द॒शभ्यः॒ स्वाहा॒ स्वाहा॑ चतुर्द॒शभ्य॑ श्चतुर्द॒शभ्यः॒ स्वाहा॑ षोड॒शभ्य॑ ष्षोड॒शभ्यः॒ स्वाहा॑ चतुर्द॒शभ्य॑ श्चतुर्द॒शभ्यः॒ स्वाहा॑ षोड॒शभ्यः॑ । \newline
18. च॒तु॒र्द॒शभ्य॒ इति॑ चतुर्द॒श - भ्यः॒ । \newline
19. स्वाहा॑ षोड॒शभ्य॑ ष्षोड॒शभ्यः॒ स्वाहा॒ स्वाहा॑ षोड॒शभ्यः॒ स्वाहा॒ स्वाहा॑ षोड॒शभ्यः॒ स्वाहा॒ स्वाहा॑ षोड॒शभ्यः॒ स्वाहा᳚ । \newline
20. षो॒ड॒शभ्यः॒ स्वाहा॒ स्वाहा॑ षोड॒शभ्य॑ ष्षोड॒शभ्यः॒ स्वाहा᳚ ऽष्टाद॒शभ्यो᳚ ऽष्टाद॒शभ्यः॒ स्वाहा॑ षोड॒शभ्य॑ ष्षोड॒शभ्यः॒ स्वाहा᳚ ऽष्टाद॒शभ्यः॑ । \newline
21. षो॒ड॒शभ्य॒ इति॑ षोड॒श - भ्यः॒ । \newline
22. स्वाहा᳚ ऽष्टाद॒शभ्यो᳚ ऽष्टाद॒शभ्यः॒ स्वाहा॒ स्वाहा᳚ ऽष्टाद॒शभ्यः॒ स्वाहा॒ स्वाहा᳚ ऽष्टाद॒शभ्यः॒ स्वाहा॒ स्वाहा᳚ ऽष्टाद॒शभ्यः॒ स्वाहा᳚ । \newline
23. अ॒ष्टा॒द॒शभ्यः॒ स्वाहा॒ स्वाहा᳚ ऽष्टाद॒शभ्यो᳚ ऽष्टाद॒शभ्यः॒ स्वाहा॑ विꣳश॒त्यै विꣳ॑श॒त्यै स्वाहा᳚ ऽष्टाद॒शभ्यो᳚ ऽष्टाद॒शभ्यः॒ स्वाहा॑ विꣳश॒त्यै । \newline
24. अ॒ष्टा॒द॒शभ्य॒ इत्य॑ष्टाद॒श - भ्यः॒ । \newline
25. स्वाहा॑ विꣳश॒त्यै विꣳ॑श॒त्यै स्वाहा॒ स्वाहा॑ विꣳश॒त्यै स्वाहा॒ स्वाहा॑ विꣳश॒त्यै स्वाहा॒ स्वाहा॑ विꣳश॒त्यै स्वाहा᳚ । \newline
26. विꣳ॒॒श॒त्यै स्वाहा॒ स्वाहा॑ विꣳश॒त्यै विꣳ॑श॒त्यै स्वाहा॒ ऽष्टान॑वत्या अ॒ष्टान॑वत्यै॒ स्वाहा॑ विꣳश॒त्यै विꣳ॑श॒त्यै स्वाहा॒ ऽष्टान॑वत्यै । \newline
27. स्वाहा॒ ऽष्टान॑वत्या अ॒ष्टान॑वत्यै॒ स्वाहा॒ स्वाहा॒ ऽष्टान॑वत्यै॒ स्वाहा॒ स्वाहा॒ ऽष्टान॑वत्यै॒ स्वाहा॒ स्वाहा॒ ऽष्टान॑वत्यै॒ स्वाहा᳚ । \newline
28. अ॒ष्टान॑वत्यै॒ स्वाहा॒ स्वाहा॒ ऽष्टान॑वत्या अ॒ष्टान॑वत्यै॒ स्वाहा॑ श॒ताय॑ श॒ताय॒ स्वाहा॒ ऽष्टान॑वत्या अ॒ष्टान॑वत्यै॒ स्वाहा॑ श॒ताय॑ । \newline
29. अ॒ष्टान॑वत्या॒ इत्य॒ष्टा - न॒व॒त्यै॒ । \newline
30. स्वाहा॑ श॒ताय॑ श॒ताय॒ स्वाहा॒ स्वाहा॑ श॒ताय॒ स्वाहा॒ स्वाहा॑ श॒ताय॒ स्वाहा॒ स्वाहा॑ श॒ताय॒ स्वाहा᳚ । \newline
31. श॒ताय॒ स्वाहा॒ स्वाहा॑ श॒ताय॑ श॒ताय॒ स्वाहा॒ सर्व॑स्मै॒ सर्व॑स्मै॒ स्वाहा॑ श॒ताय॑ श॒ताय॒ स्वाहा॒ सर्व॑स्मै । \newline
32. स्वाहा॒ सर्व॑स्मै॒ सर्व॑स्मै॒ स्वाहा॒ स्वाहा॒ सर्व॑स्मै॒ स्वाहा॒ स्वाहा॒ सर्व॑स्मै॒ स्वाहा॒ स्वाहा॒ सर्व॑स्मै॒ स्वाहा᳚ । \newline
33. सर्व॑स्मै॒ स्वाहा॒ स्वाहा॒ सर्व॑स्मै॒ सर्व॑स्मै॒ स्वाहा᳚ । \newline
34. स्वाहेति॒ स्वाहा᳚ । \newline
\pagebreak
\markright{ TS 7.2.14.1  \hfill https://www.vedavms.in \hfill}

\section{ TS 7.2.14.1 }

\textbf{TS 7.2.14.1 } \newline
\textbf{Samhita Paata} \newline

त्रि॒भ्यः स्वाहा॑ प॒ञ्चभ्यः॒ स्वाहा॑ स॒प्तभ्यः॒ स्वाहा॑ न॒वभ्यः॒ स्वाहै॑-काद॒शभ्यः॒ स्वाहा᳚ त्रयोद॒शभ्यः॒ स्वाहा॑ पञ्चद॒शभ्यः॒ स्वाहा॑ सप्तद॒शभ्यः॒ स्वाहैका॒न्न विꣳ॑श॒त्यै स्वाहा॒ नव॑विꣳशत्यै॒ स्वाहैका॒न्न च॑त्वारिꣳ॒॒शते॒ स्वाहा॒ नव॑चत्वारिꣳशते॒ स्वाहैका॒न्न ष॒ष्ट्यै स्वाहा॒ नव॑षष्ट्यै॒ स्वाहैका॒न्ना ऽशी॒त्यै स्वाहा॒ नवा॑शीत्यै॒ स्वाहैका॒न्न श॒ताय॒ स्वाहा॑ श॒ताय॒ स्वाहा॒ सर्व॑स्मै॒ स्वाहा᳚ ॥ \newline

\textbf{Pada Paata} \newline

त्रि॒भ्य इति॑ त्रि - भ्यः । स्वाहा᳚ । प॒ञ्चभ्य॒ इति॑ प॒ञ्च - भ्यः॒ । स्वाहा᳚ । स॒प्तभ्य॒ इति॑ स॒प्त-भ्यः॒ । स्वाहा᳚ । न॒वभ्य॒ इति॑ न॒व-भ्यः॒ । स्वाहा᳚ । ए॒का॒द॒शभ्य॒ इत्ये॑काद॒श - भ्यः॒ । स्वाहा᳚ । त्र॒यो॒द॒शभ्य॒ इति॑ त्रयोद॒श - भ्यः॒ । स्वाहा᳚ । प॒ञ्च॒द॒शभ्य॒ इति॑ पञ्चद॒श - भ्यः॒ । स्वाहा᳚ । स॒प्त॒द॒शभ्य॒ इति॑ सप्तद॒श - भ्यः॒ । स्वाहा᳚ । एका᳚त् । न । विꣳ॒॒श॒त्यै । स्वाहा᳚ । नव॑विꣳशत्या॒ इति॒ नव॑ - विꣳ॒॒श॒त्यै॒ । स्वाहा᳚ । एका᳚त् । न । च॒त्वा॒रिꣳ॒॒शते᳚ । स्वाहा᳚ । नव॑चत्वारिꣳशत॒ इति॒ नव॑ - च॒त्वा॒रिꣳ॒॒श॒ते॒ । स्वाहा᳚ । एका᳚त् । न । ष॒ष्ट्यै । स्वाहा᳚ । नव॑षष्ट्या॒ इति॒ नव॑ - ष॒ष्ट्यै॒ । स्वाहा᳚ । एका᳚त् । न । अ॒शी॒त्यै । स्वाहा᳚ । नवा॑शीत्या॒ इति॒ नव॑ - अ॒शी॒त्यै॒ । स्वाहा᳚ । एका᳚त् । न । श॒ताय॑ । स्वाहा᳚ । श॒ताय॑ । स्वाहा᳚ । सर्व॑स्मै । स्वाहा᳚ ॥  \newline


\textbf{Krama Paata} \newline

त्रि॒भ्यः स्वाहा᳚ । त्रि॒भ्य इति॑ त्रि - भ्यः । स्वाहा॑ प॒ञ्चभ्यः॑ । प॒ञ्चभ्यः॒ स्वाहा᳚ । प॒ञ्चभ्य॒ इति॑ प॒ञ्च - भ्यः॒ । स्वाहा॑ स॒प्तभ्यः॑ । स॒प्तभ्यः॒ स्वाहा᳚ । स॒प्तभ्य॒ इति॑ स॒प्त - भ्यः॒ । स्वाहा॑ न॒वभ्यः॑ । न॒वभ्यः॒ स्वाहा᳚ । न॒वभ्य॒ इति॑ न॒व - भ्यः॒ । स्वाहै॑काद॒शभ्यः॑ । ए॒का॒द॒शभ्यः॒ स्वाहा᳚ । ए॒का॒द॒शभ्य॒ इत्ये॑काद॒श - भ्यः॒ । स्वाहा᳚ त्रयोद॒शभ्यः॑ । त्र॒यो॒द॒शभ्यः॒ स्वाहा᳚ । त्र॒यो॒द॒शभ्य॒ इति॑ त्रयोद॒श - भ्यः॒ । स्वाहा॑ पञ्चद॒शभ्यः॑ । प॒ञ्च॒द॒शभ्यः॒ स्वाहा᳚ । प॒ञ्च॒द॒शभ्य॒ इति॑ पञ्चद॒श - भ्यः॒ । स्वाहा॑ सप्तद॒शभ्यः॑ । स॒प्त॒द॒शभ्यः॒ स्वाहा᳚ । स॒प्त॒द॒शभ्य॒ इति॑ सप्तद॒श - भ्यः॒ । स्वाहैका᳚त् । एका॒न् न । न विꣳ॑श॒त्यै । विꣳ॒॒श॒त्यै स्वाहा᳚ । स्वाहा॒ नव॑विꣳशत्यै । नव॑विꣳशत्यै॒ स्वाहा᳚ । नव॑विꣳशत्या॒ इति॒ नव॑ - विꣳ॒॒श॒त्यै॒ । स्वाहैका᳚त् । एका॒न् न । न च॑त्वारिꣳ॒॒शते᳚ । च॒त्वा॒रिꣳ॒॒शते॒ स्वाहा᳚ । स्वाहा॒ नव॑चत्वारिꣳशते । नव॑चत्वारिꣳशते॒ स्वाहा᳚ । नव॑चत्वारिꣳशत॒ इति॒ नव॑ - च॒त्वा॒रिꣳ॒॒श॒ते॒ । स्वाहैका᳚त् । एका॒न् न । न ष॒ष्ट्‍यै । ष॒ष्ट्‍यै स्वाहा᳚ । स्वाहा॒ नव॑षष्ट्‍यै । नव॑षष्ट्‍यै॒ स्वाहा᳚ । नव॑षष्ट्‍या॒ इति॒ नव॑ - ष॒ष्ट्‍यै॒ । स्वाहैका᳚त् । एका॒न् न । नाशी॒त्यै । अ॒शी॒त्यै स्वाहा᳚ । स्वाहा॒ नवा॑शीत्यै । नवा॑शीत्यै॒ स्वाहा᳚ । नवा॑शीत्या॒ इति॒ नव॑ - अ॒शी॒त्यै॒ । स्वाहैका᳚त् । एका॒न् न । न श॒ताय॑ । श॒ताय॒ स्वाहा᳚ । स्वाहा॑ श॒ताय॑ । श॒ताय॒ स्वाहा᳚ । स्वाहा॒ सर्व॑स्मै । सर्व॑स्मै॒ स्वाहा᳚ । स्वाहेति॒ स्वाहा᳚ । \newline

\textbf{Jatai Paata} \newline

1. त्रि॒भ्यः स्वाहा॒ स्वाहा᳚ त्रि॒भ्य स्त्रि॒भ्यः स्वाहा᳚ । \newline
2. त्रि॒भ्य इति॑ त्रि - भ्यः । \newline
3. स्वाहा॑ प॒ञ्चभ्यः॑ प॒ञ्चभ्यः॒ स्वाहा॒ स्वाहा॑ प॒ञ्चभ्यः॑ । \newline
4. प॒ञ्चभ्यः॒ स्वाहा॒ स्वाहा॑ प॒ञ्चभ्यः॑ प॒ञ्चभ्यः॒ स्वाहा᳚ । \newline
5. प॒ञ्चभ्य॒ इति॑ प॒ञ्च - भ्यः॒ । \newline
6. स्वाहा॑ स॒प्तभ्यः॑ स॒प्तभ्यः॒ स्वाहा॒ स्वाहा॑ स॒प्तभ्यः॑ । \newline
7. स॒प्तभ्यः॒ स्वाहा॒ स्वाहा॑ स॒प्तभ्यः॑ स॒प्तभ्यः॒ स्वाहा᳚ । \newline
8. स॒प्तभ्य॒ इति॑ स॒प्त - भ्यः॒ । \newline
9. स्वाहा॑ न॒वभ्यो॑ न॒वभ्यः॒ स्वाहा॒ स्वाहा॑ न॒वभ्यः॑ । \newline
10. न॒वभ्यः॒ स्वाहा॒ स्वाहा॑ न॒वभ्यो॑ न॒वभ्यः॒ स्वाहा᳚ । \newline
11. न॒वभ्य॒ इति॑ न॒व - भ्यः॒ । \newline
12. स्वाहै॑काद॒शभ्य॑ एकाद॒शभ्यः॒ स्वाहा॒ स्वाहै॑काद॒शभ्यः॑ । \newline
13. ए॒का॒द॒शभ्यः॒ स्वाहा॒ स्वाहै॑काद॒शभ्य॑ एकाद॒शभ्यः॒ स्वाहा᳚ । \newline
14. ए॒का॒द॒शभ्य॒ इत्ये॑काद॒श - भ्यः॒ । \newline
15. स्वाहा᳚ त्रयोद॒शभ्य॑ स्त्रयोद॒शभ्यः॒ स्वाहा॒ स्वाहा᳚ त्रयोद॒शभ्यः॑ । \newline
16. त्र॒यो॒द॒शभ्यः॒ स्वाहा॒ स्वाहा᳚ त्रयोद॒शभ्य॑ स्त्रयोद॒शभ्यः॒ स्वाहा᳚ । \newline
17. त्र॒यो॒द॒शभ्य॒ इति॑ त्रयोद॒श - भ्यः॒ । \newline
18. स्वाहा॑ पञ्चद॒शभ्यः॑ पञ्चद॒शभ्यः॒ स्वाहा॒ स्वाहा॑ पञ्चद॒शभ्यः॑ । \newline
19. प॒ञ्च॒द॒शभ्यः॒ स्वाहा॒ स्वाहा॑ पञ्चद॒शभ्यः॑ पञ्चद॒शभ्यः॒ स्वाहा᳚ । \newline
20. प॒ञ्च॒द॒शभ्य॒ इति॑ पञ्चद॒श - भ्यः॒ । \newline
21. स्वाहा॑ सप्तद॒शभ्यः॑ सप्तद॒शभ्यः॒ स्वाहा॒ स्वाहा॑ सप्तद॒शभ्यः॑ । \newline
22. स॒प्त॒द॒शभ्यः॒ स्वाहा॒ स्वाहा॑ सप्तद॒शभ्यः॑ सप्तद॒शभ्यः॒ स्वाहा᳚ । \newline
23. स॒प्त॒द॒शभ्य॒ इति॑ सप्तद॒श - भ्यः॒ । \newline
24. स्वाहैका॒ देका॒थ् स्वाहा॒ स्वाहैका᳚त् । \newline
25. एका॒न् न नैका॒ देका॒न् न । \newline
26. न विꣳ॑श॒त्यै विꣳ॑श॒त्यै न न विꣳ॑श॒त्यै । \newline
27. विꣳ॒॒श॒त्यै स्वाहा॒ स्वाहा॑ विꣳश॒त्यै विꣳ॑श॒त्यै स्वाहा᳚ । \newline
28. स्वाहा॒ नव॑विꣳशत्यै॒ नव॑विꣳशत्यै॒ स्वाहा॒ स्वाहा॒ नव॑विꣳशत्यै । \newline
29. नव॑विꣳशत्यै॒ स्वाहा॒ स्वाहा॒ नव॑विꣳशत्यै॒ नव॑विꣳशत्यै॒ स्वाहा᳚ । \newline
30. नव॑विꣳशत्या॒ इति॒ नव॑ - विꣳ॒॒श॒त्यै॒ । \newline
31. स्वाहैका॒ देका॒थ् स्वाहा॒ स्वाहैका᳚त् । \newline
32. एका॒न् न नैका॒ देका॒न् न । \newline
33. न च॑त्वारिꣳ॒॒शते॑ चत्वारिꣳ॒॒शते॒ न न च॑त्वारिꣳ॒॒शते᳚ । \newline
34. च॒त्वा॒रिꣳ॒॒शते॒ स्वाहा॒ स्वाहा॑ चत्वारिꣳ॒॒शते॑ चत्वारिꣳ॒॒शते॒ स्वाहा᳚ । \newline
35. स्वाहा॒ नव॑चत्वारिꣳशते॒ नव॑चत्वारिꣳशते॒ स्वाहा॒ स्वाहा॒ नव॑चत्वारिꣳशते । \newline
36. नव॑चत्वारिꣳशते॒ स्वाहा॒ स्वाहा॒ नव॑चत्वारिꣳशते॒ नव॑चत्वारिꣳशते॒ स्वाहा᳚ । \newline
37. नव॑चत्वारिꣳशत॒ इति॒ नव॑ - च॒त्वा॒रिꣳ॒॒श॒ते॒ । \newline
38. स्वाहैका॒ देका॒थ् स्वाहा॒ स्वाहैका᳚त् । \newline
39. एका॒न् न नैका॒ देका॒न् न । \newline
40. न ष॒ष्ट्यै ष॒ष्ट्यै न न ष॒ष्ट्यै । \newline
41. ष॒ष्ट्यै स्वाहा॒ स्वाहा॑ ष॒ष्ट्यै ष॒ष्ट्यै स्वाहा᳚ । \newline
42. स्वाहा॒ नव॑षष्ट्यै॒ नव॑षष्ट्यै॒ स्वाहा॒ स्वाहा॒ नव॑षष्ट्यै । \newline
43. नव॑षष्ट्यै॒ स्वाहा॒ स्वाहा॒ नव॑षष्ट्यै॒ नव॑षष्ट्यै॒ स्वाहा᳚ । \newline
44. नव॑षष्ट्या॒ इति॒ नव॑ - ष॒ष्ट्यै॒ । \newline
45. स्वाहैका॒ देका॒थ् स्वाहा॒ स्वाहैका᳚त् । \newline
46. एका॒न् न नैका॒ देका॒न् न । \newline
47. नाशी॒त्या अ॑शी॒त्यै न नाशी॒त्यै । \newline
48. अ॒शी॒त्यै स्वाहा॒ स्वाहा॑ ऽशी॒त्या अ॑शी॒त्यै स्वाहा᳚ । \newline
49. स्वाहा॒ नवा॑शीत्यै॒ नवा॑शीत्यै॒ स्वाहा॒ स्वाहा॒ नवा॑शीत्यै । \newline
50. नवा॑शीत्यै॒ स्वाहा॒ स्वाहा॒ नवा॑शीत्यै॒ नवा॑शीत्यै॒ स्वाहा᳚ । \newline
51. नवा॑शीत्या॒ इति॒ नव॑ - अ॒शी॒त्यै॒ । \newline
52. स्वाहैका॒ देका॒थ् स्वाहा॒ स्वाहैका᳚त् । \newline
53. एका॒न् न नैका॒ देका॒न् न । \newline
54. न श॒ताय॑ श॒ताय॒ न न श॒ताय॑ । \newline
55. श॒ताय॒ स्वाहा॒ स्वाहा॑ श॒ताय॑ श॒ताय॒ स्वाहा᳚ । \newline
56. स्वाहा॑ श॒ताय॑ श॒ताय॒ स्वाहा॒ स्वाहा॑ श॒ताय॑ । \newline
57. श॒ताय॒ स्वाहा॒ स्वाहा॑ श॒ताय॑ श॒ताय॒ स्वाहा᳚ । \newline
58. स्वाहा॒ सर्व॑स्मै॒ सर्व॑स्मै॒ स्वाहा॒ स्वाहा॒ सर्व॑स्मै । \newline
59. सर्व॑स्मै॒ स्वाहा॒ स्वाहा॒ सर्व॑स्मै॒ सर्व॑स्मै॒ स्वाहा᳚ । \newline
60. स्वाहेति॒ स्वाहा᳚ । \newline

\textbf{Ghana Paata } \newline

1. त्रि॒भ्यः स्वाहा॒ स्वाहा᳚ त्रि॒भ्य स्त्रि॒भ्यः स्वाहा॑ प॒ञ्चभ्यः॑ प॒ञ्चभ्यः॒ स्वाहा᳚ त्रि॒भ्य स्त्रि॒भ्यः स्वाहा॑ प॒ञ्चभ्यः॑ । \newline
2. त्रि॒भ्य इति॑ त्रि - भ्यः । \newline
3. स्वाहा॑ प॒ञ्चभ्यः॑ प॒ञ्चभ्यः॒ स्वाहा॒ स्वाहा॑ प॒ञ्चभ्यः॒ स्वाहा॒ स्वाहा॑ प॒ञ्चभ्यः॒ स्वाहा॒ स्वाहा॑ प॒ञ्चभ्यः॒ स्वाहा᳚ । \newline
4. प॒ञ्चभ्यः॒ स्वाहा॒ स्वाहा॑ प॒ञ्चभ्यः॑ प॒ञ्चभ्यः॒ स्वाहा॑ स॒प्तभ्यः॑ स॒प्तभ्यः॒ स्वाहा॑ प॒ञ्चभ्यः॑ प॒ञ्चभ्यः॒ स्वाहा॑ स॒प्तभ्यः॑ । \newline
5. प॒ञ्चभ्य॒ इति॑ प॒ञ्च - भ्यः॒ । \newline
6. स्वाहा॑ स॒प्तभ्यः॑ स॒प्तभ्यः॒ स्वाहा॒ स्वाहा॑ स॒प्तभ्यः॒ स्वाहा॒ स्वाहा॑ स॒प्तभ्यः॒ स्वाहा॒ स्वाहा॑ स॒प्तभ्यः॒ स्वाहा᳚ । \newline
7. स॒प्तभ्यः॒ स्वाहा॒ स्वाहा॑ स॒प्तभ्यः॑ स॒प्तभ्यः॒ स्वाहा॑ न॒वभ्यो॑ न॒वभ्यः॒ स्वाहा॑ स॒प्तभ्यः॑ स॒प्तभ्यः॒ स्वाहा॑ न॒वभ्यः॑ । \newline
8. स॒प्तभ्य॒ इति॑ स॒प्त - भ्यः॒ । \newline
9. स्वाहा॑ न॒वभ्यो॑ न॒वभ्यः॒ स्वाहा॒ स्वाहा॑ न॒वभ्यः॒ स्वाहा॒ स्वाहा॑ न॒वभ्यः॒ स्वाहा॒ स्वाहा॑ न॒वभ्यः॒ स्वाहा᳚ । \newline
10. न॒वभ्यः॒ स्वाहा॒ स्वाहा॑ न॒वभ्यो॑ न॒वभ्यः॒ स्वाहै॑काद॒शभ्य॑ एकाद॒शभ्यः॒ स्वाहा॑ न॒वभ्यो॑ न॒वभ्यः॒ स्वाहै॑काद॒शभ्यः॑ । \newline
11. न॒वभ्य॒ इति॑ न॒व - भ्यः॒ । \newline
12. स्वाहै॑काद॒शभ्य॑ एकाद॒शभ्यः॒ स्वाहा॒ स्वाहै॑काद॒शभ्यः॒ स्वाहा॒ स्वाहै॑काद॒शभ्यः॒ स्वाहा॒ 
स्वाहै॑काद॒शभ्यः॒ स्वाहा᳚ । \newline
13. ए॒का॒द॒शभ्यः॒ स्वाहा॒ स्वाहै॑काद॒शभ्य॑ एकाद॒शभ्यः॒ स्वाहा᳚ त्रयोद॒शभ्य॑ स्त्रयोद॒शभ्यः॒ 
स्वाहै॑काद॒शभ्य॑ एकाद॒शभ्यः॒ स्वाहा᳚ त्रयोद॒शभ्यः॑ । \newline
14. ए॒का॒द॒शभ्य॒ इत्ये॑काद॒श - भ्यः॒ । \newline
15. स्वाहा᳚ त्रयोद॒शभ्य॑ स्त्रयोद॒शभ्यः॒ स्वाहा॒ स्वाहा᳚ त्रयोद॒शभ्यः॒ स्वाहा॒ स्वाहा᳚ त्रयोद॒शभ्यः॒ स्वाहा॒ स्वाहा᳚ त्रयोद॒शभ्यः॒ स्वाहा᳚ । \newline
16. त्र॒यो॒द॒शभ्यः॒ स्वाहा॒ स्वाहा᳚ त्रयोद॒शभ्य॑ स्त्रयोद॒शभ्यः॒ स्वाहा॑ पञ्चद॒शभ्यः॑ पञ्चद॒शभ्यः॒ स्वाहा᳚ त्रयोद॒शभ्य॑ स्त्रयोद॒शभ्यः॒ स्वाहा॑ पञ्चद॒शभ्यः॑ । \newline
17. त्र॒यो॒द॒शभ्य॒ इति॑ त्रयोद॒श - भ्यः॒ । \newline
18. स्वाहा॑ पञ्चद॒शभ्यः॑ पञ्चद॒शभ्यः॒ स्वाहा॒ स्वाहा॑ पञ्चद॒शभ्यः॒ स्वाहा॒ स्वाहा॑ पञ्चद॒शभ्यः॒ स्वाहा॒ स्वाहा॑ पञ्चद॒शभ्यः॒ स्वाहा᳚ । \newline
19. प॒ञ्च॒द॒शभ्यः॒ स्वाहा॒ स्वाहा॑ पञ्चद॒शभ्यः॑ पञ्चद॒शभ्यः॒ स्वाहा॑ सप्तद॒शभ्यः॑ सप्तद॒शभ्यः॒ स्वाहा॑ पञ्चद॒शभ्यः॑ पञ्चद॒शभ्यः॒ स्वाहा॑ सप्तद॒शभ्यः॑ । \newline
20. प॒ञ्च॒द॒शभ्य॒ इति॑ पञ्चद॒श - भ्यः॒ । \newline
21. स्वाहा॑ सप्तद॒शभ्यः॑ सप्तद॒शभ्यः॒ स्वाहा॒ स्वाहा॑ सप्तद॒शभ्यः॒ स्वाहा॒ स्वाहा॑ सप्तद॒शभ्यः॒ स्वाहा॒ स्वाहा॑ सप्तद॒शभ्यः॒ स्वाहा᳚ । \newline
22. स॒प्त॒द॒शभ्यः॒ स्वाहा॒ स्वाहा॑ सप्तद॒शभ्यः॑ सप्तद॒शभ्यः॒ स्वाहैका॒ देका॒थ् स्वाहा॑ सप्तद॒शभ्यः॑ सप्तद॒शभ्यः॒ स्वाहैका᳚त् । \newline
23. स॒प्त॒द॒शभ्य॒ इति॑ सप्तद॒श - भ्यः॒ । \newline
24. स्वाहैका॒ देका॒थ् स्वाहा॒ स्वाहैका॒न् न नैका॒थ् स्वाहा॒ स्वाहैका॒न् न । \newline
25. एका॒न् न नैका॒ देका॒न् न विꣳ॑श॒त्यै विꣳ॑श॒त्यै नैका॒ देका॒न् न विꣳ॑श॒त्यै । \newline
26. न विꣳ॑श॒त्यै विꣳ॑श॒त्यै न न विꣳ॑श॒त्यै स्वाहा॒ स्वाहा॑ विꣳश॒त्यै न न विꣳ॑श॒त्यै स्वाहा᳚ । \newline
27. विꣳ॒॒श॒त्यै स्वाहा॒ स्वाहा॑ विꣳश॒त्यै विꣳ॑श॒त्यै स्वाहा॒ नव॑विꣳशत्यै॒ नव॑विꣳशत्यै॒ स्वाहा॑ विꣳश॒त्यै विꣳ॑श॒त्यै स्वाहा॒ नव॑विꣳशत्यै । \newline
28. स्वाहा॒ नव॑विꣳशत्यै॒ नव॑विꣳशत्यै॒ स्वाहा॒ स्वाहा॒ नव॑विꣳशत्यै॒ स्वाहा॒ स्वाहा॒ नव॑विꣳशत्यै॒ स्वाहा॒ स्वाहा॒ नव॑विꣳशत्यै॒ स्वाहा᳚ । \newline
29. नव॑विꣳशत्यै॒ स्वाहा॒ स्वाहा॒ नव॑विꣳशत्यै॒ नव॑विꣳशत्यै॒ स्वाहैका॒ देका॒थ् स्वाहा॒ नव॑विꣳशत्यै॒ नव॑विꣳशत्यै॒ स्वाहैका᳚त् । \newline
30. नव॑विꣳशत्या॒ इति॒ नव॑ - विꣳ॒॒श॒त्यै॒ । \newline
31. स्वाहैका॒ देका॒थ् स्वाहा॒ स्वाहैका॒न् न नैका॒थ् स्वाहा॒ स्वाहैका॒न् न । \newline
32. एका॒न् न नैका॒ देका॒न् न च॑त्वारिꣳ॒॒शते॑ चत्वारिꣳ॒॒शते॒ नैका॒ देका॒न् न च॑त्वारिꣳ॒॒शते᳚ । \newline
33. न च॑त्वारिꣳ॒॒शते॑ चत्वारिꣳ॒॒शते॒ न न च॑त्वारिꣳ॒॒शते॒ स्वाहा॒ स्वाहा॑ चत्वारिꣳ॒॒शते॒ न न च॑त्वारिꣳ॒॒शते॒ स्वाहा᳚ । \newline
34. च॒त्वा॒रिꣳ॒॒शते॒ स्वाहा॒ स्वाहा॑ चत्वारिꣳ॒॒शते॑ चत्वारिꣳ॒॒शते॒ स्वाहा॒ नव॑चत्वारिꣳशते॒ नव॑चत्वारिꣳशते॒ स्वाहा॑ चत्वारिꣳ॒॒शते॑ चत्वारिꣳ॒॒शते॒ स्वाहा॒ नव॑चत्वारिꣳशते । \newline
35. स्वाहा॒ नव॑चत्वारिꣳशते॒ नव॑चत्वारिꣳशते॒ स्वाहा॒ स्वाहा॒ नव॑चत्वारिꣳशते॒ स्वाहा॒ स्वाहा॒ नव॑चत्वारिꣳशते॒ स्वाहा॒ स्वाहा॒ नव॑चत्वारिꣳशते॒ स्वाहा᳚ । \newline
36. नव॑चत्वारिꣳशते॒ स्वाहा॒ स्वाहा॒ नव॑चत्वारिꣳशते॒ नव॑चत्वारिꣳशते॒ स्वाहैका॒ देका॒थ् स्वाहा॒ नव॑चत्वारिꣳशते॒ नव॑चत्वारिꣳशते॒ स्वाहैका᳚त् । \newline
37. नव॑चत्वारिꣳशत॒ इति॒ नव॑ - च॒त्वा॒रिꣳ॒॒श॒ते॒ । \newline
38. स्वाहैका॒ देका॒थ् स्वाहा॒ स्वाहैका॒न् न नैका॒थ् स्वाहा॒ स्वाहैका॒न् न । \newline
39. एका॒न् न नैका॒ देका॒न् न ष॒ष्ट्यै ष॒ष्ट्यै नैका॒ देका॒न् न ष॒ष्ट्यै । \newline
40. न ष॒ष्ट्यै ष॒ष्ट्यै न न ष॒ष्ट्यै स्वाहा॒ स्वाहा॑ ष॒ष्ट्यै न न ष॒ष्ट्यै स्वाहा᳚ । \newline
41. ष॒ष्ट्यै स्वाहा॒ स्वाहा॑ ष॒ष्ट्यै ष॒ष्ट्यै स्वाहा॒ नव॑षष्ट्यै॒ नव॑षष्ट्यै॒ स्वाहा॑ ष॒ष्ट्यै ष॒ष्ट्यै स्वाहा॒ नव॑षष्ट्यै । \newline
42. स्वाहा॒ नव॑षष्ट्यै॒ नव॑षष्ट्यै॒ स्वाहा॒ स्वाहा॒ नव॑षष्ट्यै॒ स्वाहा॒ स्वाहा॒ नव॑षष्ट्यै॒ स्वाहा॒ स्वाहा॒ नव॑षष्ट्यै॒ स्वाहा᳚ । \newline
43. नव॑षष्ट्यै॒ स्वाहा॒ स्वाहा॒ नव॑षष्ट्यै॒ नव॑षष्ट्यै॒ स्वाहैका॒ देका॒थ् स्वाहा॒ नव॑षष्ट्यै॒ नव॑षष्ट्यै॒ स्वाहैका᳚त् । \newline
44. नव॑षष्ट्या॒ इति॒ नव॑ - ष॒ष्ट्यै॒ । \newline
45. स्वाहैका॒ देका॒थ् स्वाहा॒ स्वाहैका॒न् न नैका॒थ् स्वाहा॒ स्वाहैका॒न् न । \newline
46. एका॒न् न नैका॒ देका॒न् नाशी॒त्या अ॑शी॒त्यै नैका॒ देका॒न् नाशी॒त्यै । \newline
47. नाशी॒त्या अ॑शी॒त्यै न नाशी॒त्यै स्वाहा॒ स्वाहा॑ ऽशी॒त्यै न नाशी॒त्यै स्वाहा᳚ । \newline
48. अ॒शी॒त्यै स्वाहा॒ स्वाहा॑ ऽशी॒त्या अ॑शी॒त्यै स्वाहा॒ नवा॑शीत्यै॒ नवा॑शीत्यै॒ स्वाहा॑ ऽशी॒त्या अ॑शी॒त्यै स्वाहा॒ नवा॑शीत्यै । \newline
49. स्वाहा॒ नवा॑शीत्यै॒ नवा॑शीत्यै॒ स्वाहा॒ स्वाहा॒ नवा॑शीत्यै॒ स्वाहा॒ स्वाहा॒ नवा॑शीत्यै॒ स्वाहा॒ स्वाहा॒ नवा॑शीत्यै॒ स्वाहा᳚ । \newline
50. नवा॑शीत्यै॒ स्वाहा॒ स्वाहा॒ नवा॑शीत्यै॒ नवा॑शीत्यै॒ स्वाहैका॒ देका॒थ् स्वाहा॒ नवा॑शीत्यै॒ नवा॑शीत्यै॒ स्वाहैका᳚त् । \newline
51. नवा॑शीत्या॒ इति॒ नव॑ - अ॒शी॒त्यै॒ । \newline
52. स्वाहैका॒ देका॒थ् स्वाहा॒ स्वाहैका॒न् न नैका॒थ् स्वाहा॒ स्वाहैका॒न् न । \newline
53. एका॒न् न नैका॒ देका॒न् न श॒ताय॑ श॒ताय॒ नैका॒ देका॒न् न श॒ताय॑ । \newline
54. न श॒ताय॑ श॒ताय॒ न न श॒ताय॒ स्वाहा॒ स्वाहा॑ श॒ताय॒ न न श॒ताय॒ स्वाहा᳚ । \newline
55. श॒ताय॒ स्वाहा॒ स्वाहा॑ श॒ताय॑ श॒ताय॒ स्वाहा॑ श॒ताय॑ श॒ताय॒ स्वाहा॑ श॒ताय॑ श॒ताय॒ स्वाहा॑ श॒ताय॑ । \newline
56. स्वाहा॑ श॒ताय॑ श॒ताय॒ स्वाहा॒ स्वाहा॑ श॒ताय॒ स्वाहा॒ स्वाहा॑ श॒ताय॒ स्वाहा॒ स्वाहा॑ श॒ताय॒ स्वाहा᳚ । \newline
57. श॒ताय॒ स्वाहा॒ स्वाहा॑ श॒ताय॑ श॒ताय॒ स्वाहा॒ सर्व॑स्मै॒ सर्व॑स्मै॒ स्वाहा॑ श॒ताय॑ श॒ताय॒ स्वाहा॒ सर्व॑स्मै । \newline
58. स्वाहा॒ सर्व॑स्मै॒ सर्व॑स्मै॒ स्वाहा॒ स्वाहा॒ सर्व॑स्मै॒ स्वाहा॒ स्वाहा॒ सर्व॑स्मै॒ स्वाहा॒ स्वाहा॒ सर्व॑स्मै॒ स्वाहा᳚ । \newline
59. सर्व॑स्मै॒ स्वाहा॒ स्वाहा॒ सर्व॑स्मै॒ सर्व॑स्मै॒ स्वाहा᳚ । \newline
60. स्वाहेति॒ स्वाहा᳚ । \newline
\pagebreak
\markright{ TS 7.2.15.1  \hfill https://www.vedavms.in \hfill}

\section{ TS 7.2.15.1 }

\textbf{TS 7.2.15.1 } \newline
\textbf{Samhita Paata} \newline

च॒तुर्भ्यः॒ स्वाहा᳚ ऽष्टा॒भ्यः स्वाहा᳚ द्वाद॒शभ्यः॒ स्वाहा॑ षोड॒शभ्यः॒ स्वाहा॑ विꣳश॒त्यै स्वाहा॒ षण्ण॑वत्यै॒ स्वाहा॑ श॒ताय॒ स्वाहा॒ सर्व॑स्मै॒ स्वाहा᳚ ॥ \newline

\textbf{Pada Paata} \newline

च॒तुर्भ्य॒ इति॑ च॒तुः - भ्यः॒ । स्वाहा᳚ । अ॒ष्टा॒भ्यः । स्वाहा᳚ । द्वा॒द॒शभ्य॒ इति॑ द्वाद॒श - भ्यः॒ । स्वाहा᳚ । षो॒ड॒शभ्य॒ इति॑ षोड॒श - भ्यः॒ । स्वाहा᳚ । विꣳ॒॒श॒त्यै । स्वाहा᳚ । षण्ण॑वत्या॒ इति॒ षट्-न॒व॒त्यै॒ । स्वाहा᳚ । श॒ताय॑ । स्वाहा᳚ । सर्व॑स्मै । स्वाहा᳚ ॥  \newline


\textbf{Krama Paata} \newline

च॒तुर्भ्यः॒ स्वाहा᳚ । च॒तुर्भ्य॒ इति॑ च॒तुः - भ्यः॒ । स्वाहा᳚ऽष्टा॒भ्यः । अ॒ष्टा॒भ्यः स्वाहा᳚ । स्वाहा᳚ द्वाद॒शभ्यः॑ । द्वा॒द॒शाभ्यः॒ स्वाहा᳚ । द्वा॒द॒शभ्य॒ इति॑ द्वाद॒श - भ्यः॒ । स्वाहा॑ षोड॒शभ्यः॑ । षो॒ड॒शभ्यः॒ स्वाहा᳚ । षो॒ड॒शभ्य॒ इति॑ षोड॒श - भ्यः॒ । स्वाहा॑ विꣳश॒त्यै । विꣳ॒॒श॒त्यै स्वाहा᳚ । स्वाहा॒ षण्ण॑वत्यै । षण्ण॑वत्यै॒ स्वाहा᳚ । षण्ण॑वत्या॒ इति॒ षट् - न॒व॒त्यै॒ । स्वाहा॑ श॒ताय॑ । श॒ताय॒ स्वाहा᳚ । स्वाहा॒ सर्व॑स्मै । सर्व॑स्मै॒ स्वाहा᳚ । स्वाहेति॒ स्वाहा᳚ । \newline

\textbf{Jatai Paata} \newline

1. च॒तुर्भ्यः॒ स्वाहा॒ स्वाहा॑ च॒तुर्भ्य॑ श्च॒तुर्भ्यः॒ स्वाहा᳚ । \newline
2. च॒तुर्भ्य॒ इति॑ च॒तुः - भ्यः॒ । \newline
3. स्वाहा᳚ ऽष्टा॒भ्यो᳚ ऽष्टा॒भ्यः स्वाहा॒ स्वाहा᳚ ऽष्टा॒भ्यः । \newline
4. अ॒ष्टा॒भ्यः स्वाहा॒ स्वाहा᳚ ऽष्टा॒भ्यो᳚ ऽष्टा॒भ्यः स्वाहा᳚ । \newline
5. स्वाहा᳚ द्वाद॒शभ्यो᳚ द्वाद॒शभ्यः॒ स्वाहा॒ स्वाहा᳚ द्वाद॒शभ्यः॑ । \newline
6. द्वा॒द॒शभ्यः॒ स्वाहा॒ स्वाहा᳚ द्वाद॒शभ्यो᳚ द्वाद॒शभ्यः॒ स्वाहा᳚ । \newline
7. द्वा॒द॒शभ्य॒ इति॑ द्वाद॒श - भ्यः॒ । \newline
8. स्वाहा॑ षोड॒शभ्य॑ ष्षोड॒शभ्यः॒ स्वाहा॒ स्वाहा॑ षोड॒शभ्यः॑ । \newline
9. षो॒ड॒शभ्यः॒ स्वाहा॒ स्वाहा॑ षोड॒शभ्य॑ ष्षोड॒शभ्यः॒ स्वाहा᳚ । \newline
10. षो॒ड॒शभ्य॒ इति॑ षोड॒श - भ्यः॒ । \newline
11. स्वाहा॑ विꣳश॒त्यै विꣳ॑श॒त्यै स्वाहा॒ स्वाहा॑ विꣳश॒त्यै । \newline
12. विꣳ॒॒श॒त्यै स्वाहा॒ स्वाहा॑ विꣳश॒त्यै विꣳ॑श॒त्यै स्वाहा᳚ । \newline
13. स्वाहा॒ षण्ण॑वत्यै॒ षण्ण॑वत्यै॒ स्वाहा॒ स्वाहा॒ षण्ण॑वत्यै । \newline
14. षण्ण॑वत्यै॒ स्वाहा॒ स्वाहा॒ षण्ण॑वत्यै॒ षण्ण॑वत्यै॒ स्वाहा᳚ । \newline
15. षण्ण॑वत्या॒ इति॒ षट् - न॒व॒त्यै॒ । \newline
16. स्वाहा॑ श॒ताय॑ श॒ताय॒ स्वाहा॒ स्वाहा॑ श॒ताय॑ । \newline
17. श॒ताय॒ स्वाहा॒ स्वाहा॑ श॒ताय॑ श॒ताय॒ स्वाहा᳚ । \newline
18. स्वाहा॒ सर्व॑स्मै॒ सर्व॑स्मै॒ स्वाहा॒ स्वाहा॒ सर्व॑स्मै । \newline
19. सर्व॑स्मै॒ स्वाहा॒ स्वाहा॒ सर्व॑स्मै॒ सर्व॑स्मै॒ स्वाहा᳚ । \newline
20. स्वाहेति॒ स्वाहा᳚ । \newline

\textbf{Ghana Paata } \newline

1. च॒तुर्भ्यः॒ स्वाहा॒ स्वाहा॑ च॒तुर्भ्य॑ श्च॒तुर्भ्यः॒ स्वाहा᳚ ऽष्टा॒भ्यो᳚ ऽष्टा॒भ्यः स्वाहा॑ च॒तुर्भ्य॑ श्च॒तुर्भ्यः॒ स्वाहा᳚ ऽष्टा॒भ्यः । \newline
2. च॒तुर्भ्य॒ इति॑ च॒तुः - भ्यः॒ । \newline
3. स्वाहा᳚ ऽष्टा॒भ्यो᳚ ऽष्टा॒भ्यः स्वाहा॒ स्वाहा᳚ ऽष्टा॒भ्यः स्वाहा॒ स्वाहा᳚ ऽष्टा॒भ्यः स्वाहा॒ स्वाहा᳚ ऽष्टा॒भ्यः स्वाहा᳚ । \newline
4. अ॒ष्टा॒भ्यः स्वाहा॒ स्वाहा᳚ ऽष्टा॒भ्यो᳚ ऽष्टा॒भ्यः स्वाहा᳚ द्वाद॒शभ्यो᳚ द्वाद॒शभ्यः॒ स्वाहा᳚ ऽष्टा॒भ्यो᳚ ऽष्टा॒भ्यः स्वाहा᳚ द्वाद॒शभ्यः॑ । \newline
5. स्वाहा᳚ द्वाद॒शभ्यो᳚ द्वाद॒शभ्यः॒ स्वाहा॒ स्वाहा᳚ द्वाद॒शभ्यः॒ स्वाहा॒ स्वाहा᳚ द्वाद॒शभ्यः॒ स्वाहा॒ स्वाहा᳚ द्वाद॒शभ्यः॒ स्वाहा᳚ । \newline
6. द्वा॒द॒शभ्यः॒ स्वाहा॒ स्वाहा᳚ द्वाद॒शभ्यो᳚ द्वाद॒शभ्यः॒ स्वाहा॑ षोड॒शभ्य॑ ष्षोड॒शभ्यः॒ स्वाहा᳚ द्वाद॒शभ्यो᳚ द्वाद॒शभ्यः॒ स्वाहा॑ षोड॒शभ्यः॑ । \newline
7. द्वा॒द॒शभ्य॒ इति॑ द्वाद॒श - भ्यः॒ । \newline
8. स्वाहा॑ षोड॒शभ्य॑ ष्षोड॒शभ्यः॒ स्वाहा॒ स्वाहा॑ षोड॒शभ्यः॒ स्वाहा॒ स्वाहा॑ षोड॒शभ्यः॒ स्वाहा॒ स्वाहा॑ षोड॒शभ्यः॒ स्वाहा᳚ । \newline
9. षो॒ड॒शभ्यः॒ स्वाहा॒ स्वाहा॑ षोड॒शभ्य॑ ष्षोड॒शभ्यः॒ स्वाहा॑ विꣳश॒त्यै विꣳ॑श॒त्यै स्वाहा॑ षोड॒शभ्य॑ ष्षोड॒शभ्यः॒ स्वाहा॑ विꣳश॒त्यै । \newline
10. षो॒ड॒शभ्य॒ इति॑ षोड॒श - भ्यः॒ । \newline
11. स्वाहा॑ विꣳश॒त्यै विꣳ॑श॒त्यै स्वाहा॒ स्वाहा॑ विꣳश॒त्यै स्वाहा॒ स्वाहा॑ विꣳश॒त्यै स्वाहा॒ स्वाहा॑ विꣳश॒त्यै स्वाहा᳚ । \newline
12. विꣳ॒॒श॒त्यै स्वाहा॒ स्वाहा॑ विꣳश॒त्यै विꣳ॑श॒त्यै स्वाहा॒ षण्ण॑वत्यै॒ षण्ण॑वत्यै॒ स्वाहा॑ विꣳश॒त्यै विꣳ॑श॒त्यै स्वाहा॒ षण्ण॑वत्यै । \newline
13. स्वाहा॒ षण्ण॑वत्यै॒ षण्ण॑वत्यै॒ स्वाहा॒ स्वाहा॒ षण्ण॑वत्यै॒ स्वाहा॒ स्वाहा॒ षण्ण॑वत्यै॒ स्वाहा॒ स्वाहा॒ षण्ण॑वत्यै॒ स्वाहा᳚ । \newline
14. षण्ण॑वत्यै॒ स्वाहा॒ स्वाहा॒ षण्ण॑वत्यै॒ षण्ण॑वत्यै॒ स्वाहा॑ श॒ताय॑ श॒ताय॒ स्वाहा॒ षण्ण॑वत्यै॒ षण्ण॑वत्यै॒ स्वाहा॑ श॒ताय॑ । \newline
15. षण्ण॑वत्या॒ इति॒ षट् - न॒व॒त्यै॒ । \newline
16. स्वाहा॑ श॒ताय॑ श॒ताय॒ स्वाहा॒ स्वाहा॑ श॒ताय॒ स्वाहा॒ स्वाहा॑ श॒ताय॒ स्वाहा॒ स्वाहा॑ श॒ताय॒ स्वाहा᳚ । \newline
17. श॒ताय॒ स्वाहा॒ स्वाहा॑ श॒ताय॑ श॒ताय॒ स्वाहा॒ सर्व॑स्मै॒ सर्व॑स्मै॒ स्वाहा॑ श॒ताय॑ श॒ताय॒ स्वाहा॒ सर्व॑स्मै । \newline
18. स्वाहा॒ सर्व॑स्मै॒ सर्व॑स्मै॒ स्वाहा॒ स्वाहा॒ सर्व॑स्मै॒ स्वाहा॒ स्वाहा॒ सर्व॑स्मै॒ स्वाहा॒ स्वाहा॒ सर्व॑स्मै॒ स्वाहा᳚ । \newline
19. सर्व॑स्मै॒ स्वाहा॒ स्वाहा॒ सर्व॑स्मै॒ सर्व॑स्मै॒ स्वाहा᳚ । \newline
20. स्वाहेति॒ स्वाहा᳚ । \newline
\pagebreak
\markright{ TS 7.2.16.1  \hfill https://www.vedavms.in \hfill}

\section{ TS 7.2.16.1 }

\textbf{TS 7.2.16.1 } \newline
\textbf{Samhita Paata} \newline

प॒ञ्चभ्यः॒ स्वाहा॑ द॒शभ्यः॒ स्वाहा॑ पञ्चद॒शभ्यः॒ स्वाहा॑ विꣳश॒त्यै स्वाहा॒ पञ्च॑नवत्यै॒ स्वाहा॑ श॒ताय॒ स्वाहा॒ सर्व॑स्मै॒ स्वाहा᳚ ॥ \newline

\textbf{Pada Paata} \newline

प॒ञ्चभ्य॒ इति॑ प॒ञ्च - भ्यः॒ । स्वाहा᳚ । द॒शभ्य॒ इति॑ द॒श - भ्यः॒ । स्वाहा᳚ । प॒ञ्च॒द॒शभ्य॒ इति॑ पञ्चद॒श - भ्यः॒ । स्वाहा᳚ । विꣳ॒॒श॒त्यै । स्वाहा᳚ । पञ्च॑नवत्या॒ इति॒ पञ्च॑ - न॒व॒त्यै॒ । स्वाहा᳚ । श॒ताय॑ । स्वाहा᳚ । सर्व॑स्मै । स्वाहा᳚ ॥  \newline


\textbf{Krama Paata} \newline

प॒ञ्चभ्यः॒ स्वाहा᳚ । प॒ञ्चभ्य॒ इति॑ प॒ञ्च - भ्यः॒ । स्वाहा॑ द॒शभ्यः॑ । द॒शभ्यः॒ स्वाहा᳚ । द॒शभ्य॒ इति॑ द॒श - भ्यः॒ । स्वाहा॑ पञ्चद॒शभ्यः॑ । प॒ञ्च॒द॒शभ्यः॒ स्वाहा᳚ । प॒ञ्च॒द॒शभ्य॒ इति॑ पञ्चद॒श - भ्यः॒ । स्वाहा॑ विꣳश॒त्यै । विꣳ॒॒श॒त्यै स्वाहा᳚ । स्वाहा॒ पञ्च॑नवत्यै । पञ्च॑नवत्यै॒ स्वाहा᳚ । पञ्च॑नवत्या॒ इति॒ पञ्च॑ - न॒व॒त्यै॒ । स्वाहा॑ श॒ताय॑ । श॒ताय॒ स्वाहा᳚ । स्वाहा॒ सर्व॑स्मै । सर्व॑स्मै॒ स्वाहा᳚ । स्वाहेति॒ स्वाहा᳚ । \newline

\textbf{Jatai Paata} \newline

1. प॒ञ्चभ्यः॒ स्वाहा॒ स्वाहा॑ प॒ञ्चभ्यः॑ प॒ञ्चभ्यः॒ स्वाहा᳚ । \newline
2. प॒ञ्चभ्य॒ इति॑ प॒ञ्च - भ्यः॒ । \newline
3. स्वाहा॑ द॒शभ्यो॑ द॒शभ्यः॒ स्वाहा॒ स्वाहा॑ द॒शभ्यः॑ । \newline
4. द॒शभ्यः॒ स्वाहा॒ स्वाहा॑ द॒शभ्यो॑ द॒शभ्यः॒ स्वाहा᳚ । \newline
5. द॒शभ्य॒ इति॑ द॒श - भ्यः॒ । \newline
6. स्वाहा॑ पञ्चद॒शभ्यः॑ पञ्चद॒शभ्यः॒ स्वाहा॒ स्वाहा॑ पञ्चद॒शभ्यः॑ । \newline
7. प॒ञ्च॒द॒शभ्यः॒ स्वाहा॒ स्वाहा॑ पञ्चद॒शभ्यः॑ पञ्चद॒शभ्यः॒ स्वाहा᳚ । \newline
8. प॒ञ्च॒द॒शभ्य॒ इति॑ पञ्चद॒श - भ्यः॒ । \newline
9. स्वाहा॑ विꣳश॒त्यै विꣳ॑श॒त्यै स्वाहा॒ स्वाहा॑ विꣳश॒त्यै । \newline
10. विꣳ॒॒श॒त्यै स्वाहा॒ स्वाहा॑ विꣳश॒त्यै विꣳ॑श॒त्यै स्वाहा᳚ । \newline
11. स्वाहा॒ पञ्च॑नवत्यै॒ पञ्च॑नवत्यै॒ स्वाहा॒ स्वाहा॒ पञ्च॑नवत्यै । \newline
12. पञ्च॑नवत्यै॒ स्वाहा॒ स्वाहा॒ पञ्च॑नवत्यै॒ पञ्च॑नवत्यै॒ स्वाहा᳚ । \newline
13. पञ्च॑नवत्या॒ इति॒ पञ्च॑ - न॒व॒त्यै॒ । \newline
14. स्वाहा॑ श॒ताय॑ श॒ताय॒ स्वाहा॒ स्वाहा॑ श॒ताय॑ । \newline
15. श॒ताय॒ स्वाहा॒ स्वाहा॑ श॒ताय॑ श॒ताय॒ स्वाहा᳚ । \newline
16. स्वाहा॒ सर्व॑स्मै॒ सर्व॑स्मै॒ स्वाहा॒ स्वाहा॒ सर्व॑स्मै । \newline
17. सर्व॑स्मै॒ स्वाहा॒ स्वाहा॒ सर्व॑स्मै॒ सर्व॑स्मै॒ स्वाहा᳚ । \newline
18. स्वाहेति॒ स्वाहा᳚ । \newline

\textbf{Ghana Paata } \newline

1. प॒ञ्चभ्यः॒ स्वाहा॒ स्वाहा॑ प॒ञ्चभ्यः॑ प॒ञ्चभ्यः॒ स्वाहा॑ द॒शभ्यो॑ द॒शभ्यः॒ स्वाहा॑ प॒ञ्चभ्यः॑ प॒ञ्चभ्यः॒ स्वाहा॑ द॒शभ्यः॑ । \newline
2. प॒ञ्चभ्य॒ इति॑ प॒ञ्च - भ्यः॒ । \newline
3. स्वाहा॑ द॒शभ्यो॑ द॒शभ्यः॒ स्वाहा॒ स्वाहा॑ द॒शभ्यः॒ स्वाहा॒ स्वाहा॑ द॒शभ्यः॒ स्वाहा॒ स्वाहा॑ द॒शभ्यः॒ स्वाहा᳚ । \newline
4. द॒शभ्यः॒ स्वाहा॒ स्वाहा॑ द॒शभ्यो॑ द॒शभ्यः॒ स्वाहा॑ पञ्चद॒शभ्यः॑ पञ्चद॒शभ्यः॒ स्वाहा॑ द॒शभ्यो॑ द॒शभ्यः॒ स्वाहा॑ पञ्चद॒शभ्यः॑ । \newline
5. द॒शभ्य॒ इति॑ द॒श - भ्यः॒ । \newline
6. स्वाहा॑ पञ्चद॒शभ्यः॑ पञ्चद॒शभ्यः॒ स्वाहा॒ स्वाहा॑ पञ्चद॒शभ्यः॒ स्वाहा॒ स्वाहा॑ पञ्चद॒शभ्यः॒ स्वाहा॒ स्वाहा॑ पञ्चद॒शभ्यः॒ स्वाहा᳚ । \newline
7. प॒ञ्च॒द॒शभ्यः॒ स्वाहा॒ स्वाहा॑ पञ्चद॒शभ्यः॑ पञ्चद॒शभ्यः॒ स्वाहा॑ विꣳश॒त्यै विꣳ॑श॒त्यै स्वाहा॑ पञ्चद॒शभ्यः॑ पञ्चद॒शभ्यः॒ स्वाहा॑ विꣳश॒त्यै । \newline
8. प॒ञ्च॒द॒शभ्य॒ इति॑ पञ्चद॒श - भ्यः॒ । \newline
9. स्वाहा॑ विꣳश॒त्यै विꣳ॑श॒त्यै स्वाहा॒ स्वाहा॑ विꣳश॒त्यै स्वाहा॒ स्वाहा॑ विꣳश॒त्यै स्वाहा॒ स्वाहा॑ विꣳश॒त्यै स्वाहा᳚ । \newline
10. विꣳ॒॒श॒त्यै स्वाहा॒ स्वाहा॑ विꣳश॒त्यै विꣳ॑श॒त्यै स्वाहा॒ पञ्च॑नवत्यै॒ पञ्च॑नवत्यै॒ स्वाहा॑ विꣳश॒त्यै विꣳ॑श॒त्यै स्वाहा॒ पञ्च॑नवत्यै । \newline
11. स्वाहा॒ पञ्च॑नवत्यै॒ पञ्च॑नवत्यै॒ स्वाहा॒ स्वाहा॒ पञ्च॑नवत्यै॒ स्वाहा॒ स्वाहा॒ पञ्च॑नवत्यै॒ स्वाहा॒ स्वाहा॒ पञ्च॑नवत्यै॒ स्वाहा᳚ । \newline
12. पञ्च॑नवत्यै॒ स्वाहा॒ स्वाहा॒ पञ्च॑नवत्यै॒ पञ्च॑नवत्यै॒ स्वाहा॑ श॒ताय॑ श॒ताय॒ स्वाहा॒ पञ्च॑नवत्यै॒ पञ्च॑नवत्यै॒ स्वाहा॑ श॒ताय॑ । \newline
13. पञ्च॑नवत्या॒ इति॒ पञ्च॑ - न॒व॒त्यै॒ । \newline
14. स्वाहा॑ श॒ताय॑ श॒ताय॒ स्वाहा॒ स्वाहा॑ श॒ताय॒ स्वाहा॒ स्वाहा॑ श॒ताय॒ स्वाहा॒ स्वाहा॑ श॒ताय॒ स्वाहा᳚ । \newline
15. श॒ताय॒ स्वाहा॒ स्वाहा॑ श॒ताय॑ श॒ताय॒ स्वाहा॒ सर्व॑स्मै॒ सर्व॑स्मै॒ स्वाहा॑ श॒ताय॑ श॒ताय॒ स्वाहा॒ सर्व॑स्मै । \newline
16. स्वाहा॒ सर्व॑स्मै॒ सर्व॑स्मै॒ स्वाहा॒ स्वाहा॒ सर्व॑स्मै॒ स्वाहा॒ स्वाहा॒ सर्व॑स्मै॒ स्वाहा॒ स्वाहा॒ सर्व॑स्मै॒ स्वाहा᳚ । \newline
17. सर्व॑स्मै॒ स्वाहा॒ स्वाहा॒ सर्व॑स्मै॒ सर्व॑स्मै॒ स्वाहा᳚ । \newline
18. स्वाहेति॒ स्वाहा᳚ । \newline
\pagebreak
\markright{ TS 7.2.17.1  \hfill https://www.vedavms.in \hfill}

\section{ TS 7.2.17.1 }

\textbf{TS 7.2.17.1 } \newline
\textbf{Samhita Paata} \newline

द॒शभ्यः॒ स्वाहा॑ विꣳश॒त्यै स्वाहा᳚ त्रिꣳ॒॒शते॒ स्वाहा॑ चत्वारिꣳ॒॒शते॒ स्वाहा॑ पञ्चा॒शते॒ स्वाहा॑ ष॒ष्ट्यै स्वाहा॑ सप्त॒त्यै स्वाहा॑ ऽशी॒त्यै स्वाहा॑ नव॒त्यै स्वाहा॑ श॒ताय॒ स्वाहा॒ सर्व॑स्मै॒ स्वाहा᳚ ॥ \newline

\textbf{Pada Paata} \newline

द॒शभ्य॒ इति॑ द॒श - भ्यः॒ । स्वाहा᳚ । विꣳ॒॒श॒त्यै । स्वाहा᳚ । त्रिꣳ॒॒शते᳚ । स्वाहा᳚ । च॒त्वा॒रिꣳ॒॒शते᳚ । स्वाहा᳚ । प॒ञ्चा॒शते᳚ । स्वाहा᳚ । ष॒ष्ट्यै । स्वाहा᳚ । स॒प्त॒त्यै । स्वाहा᳚ । अ॒शी॒त्यै । स्वाहा᳚ । न॒व॒त्यै । स्वाहा᳚ । श॒ताय॑ । स्वाहा᳚ । सर्व॑स्मै । स्वाहा᳚ ॥  \newline


\textbf{Krama Paata} \newline

द॒शभ्यः॒ स्वाहा᳚ । द॒शभ्य॒ इति॑ द॒श - भ्यः॒ । स्वाहा॑ विꣳश॒त्यै । विꣳ॒॒श॒त्यै स्वाहा᳚ । स्वाहा᳚ त्रिꣳ॒॒शते᳚ । त्रिꣳ॒॒शते॒ स्वाहा᳚ । स्वाहा॑ चत्वारिꣳ॒॒शते᳚ । च॒त्वा॒रिꣳ॒॒शते॒ स्वाहा᳚ । स्वाहा॑ पञ्चा॒शते᳚ । प॒ञ्चा॒शते॒ स्वाहा᳚ । स्वाहा॑ ष॒ष्ट्‍यै । ष॒ष्ट्‍यै स्वाहा᳚ । स्वाहा॑ सप्त॒त्यै । स॒प्त॒त्यै स्वाहा᳚ । स्वाहा॑ऽशी॒त्यै । अ॒शी॒त्यै स्वाहा᳚ । स्वाहा॑ नव॒त्यै । न॒व॒त्यै स्वाहा᳚ । स्वाहा॑ श॒ताय॑ । श॒ताय॒ स्वाहा᳚ । स्वाहा॒ सर्व॑स्मै । सर्व॑स्मै॒ स्वाहा᳚ । स्वाहेति॒ स्वाहा᳚ । \newline

\textbf{Jatai Paata} \newline

1. द॒शभ्यः॒ स्वाहा॒ स्वाहा॑ द॒शभ्यो॑ द॒शभ्यः॒ स्वाहा᳚ । \newline
2. द॒शभ्य॒ इति॑ द॒श - भ्यः॒ । \newline
3. स्वाहा॑ विꣳश॒त्यै विꣳ॑श॒त्यै स्वाहा॒ स्वाहा॑ विꣳश॒त्यै । \newline
4. विꣳ॒॒श॒त्यै स्वाहा॒ स्वाहा॑ विꣳश॒त्यै विꣳ॑श॒त्यै स्वाहा᳚ । \newline
5. स्वाहा᳚ त्रिꣳ॒॒शते᳚ त्रिꣳ॒॒शते॒ स्वाहा॒ स्वाहा᳚ त्रिꣳ॒॒शते᳚ । \newline
6. त्रिꣳ॒॒शते॒ स्वाहा॒ स्वाहा᳚ त्रिꣳ॒॒शते᳚ त्रिꣳ॒॒शते॒ स्वाहा᳚ । \newline
7. स्वाहा॑ चत्वारिꣳ॒॒शते॑ चत्वारिꣳ॒॒शते॒ स्वाहा॒ स्वाहा॑ चत्वारिꣳ॒॒शते᳚ । \newline
8. च॒त्वा॒रिꣳ॒॒शते॒ स्वाहा॒ स्वाहा॑ चत्वारिꣳ॒॒शते॑ चत्वारिꣳ॒॒शते॒ स्वाहा᳚ । \newline
9. स्वाहा॑ पञ्चा॒शते॑ पञ्चा॒शते॒ स्वाहा॒ स्वाहा॑ पञ्चा॒शते᳚ । \newline
10. प॒ञ्चा॒शते॒ स्वाहा॒ स्वाहा॑ पञ्चा॒शते॑ पञ्चा॒शते॒ स्वाहा᳚ । \newline
11. स्वाहा॑ ष॒ष्ट्यै ष॒ष्ट्यै स्वाहा॒ स्वाहा॑ ष॒ष्ट्यै । \newline
12. ष॒ष्ट्यै स्वाहा॒ स्वाहा॑ ष॒ष्ट्यै ष॒ष्ट्यै स्वाहा᳚ । \newline
13. स्वाहा॑ सप्त॒त्यै स॑प्त॒त्यै स्वाहा॒ स्वाहा॑ सप्त॒त्यै । \newline
14. स॒प्त॒त्यै स्वाहा॒ स्वाहा॑ सप्त॒त्यै स॑प्त॒त्यै स्वाहा᳚ । \newline
15. स्वाहा॑ ऽशी॒त्या अ॑शी॒त्यै स्वाहा॒ स्वाहा॑ ऽशी॒त्यै । \newline
16. अ॒शी॒त्यै स्वाहा॒ स्वाहा॑ ऽशी॒त्या अ॑शी॒त्यै स्वाहा᳚ । \newline
17. स्वाहा॑ नव॒त्यै न॑व॒त्यै स्वाहा॒ स्वाहा॑ नव॒त्यै । \newline
18. न॒व॒त्यै स्वाहा॒ स्वाहा॑ नव॒त्यै न॑व॒त्यै स्वाहा᳚ । \newline
19. स्वाहा॑ श॒ताय॑ श॒ताय॒ स्वाहा॒ स्वाहा॑ श॒ताय॑ । \newline
20. श॒ताय॒ स्वाहा॒ स्वाहा॑ श॒ताय॑ श॒ताय॒ स्वाहा᳚ । \newline
21. स्वाहा॒ सर्व॑स्मै॒ सर्व॑स्मै॒ स्वाहा॒ स्वाहा॒ सर्व॑स्मै । \newline
22. सर्व॑स्मै॒ स्वाहा॒ स्वाहा॒ सर्व॑स्मै॒ सर्व॑स्मै॒ स्वाहा᳚ । \newline
23. स्वाहेति॒ स्वाहा᳚ । \newline

\textbf{Ghana Paata } \newline

1. द॒शभ्यः॒ स्वाहा॒ स्वाहा॑ द॒शभ्यो॑ द॒शभ्यः॒ स्वाहा॑ विꣳश॒त्यै विꣳ॑श॒त्यै स्वाहा॑ द॒शभ्यो॑ द॒शभ्यः॒ स्वाहा॑ विꣳश॒त्यै । \newline
2. द॒शभ्य॒ इति॑ द॒श - भ्यः॒ । \newline
3. स्वाहा॑ विꣳश॒त्यै विꣳ॑श॒त्यै स्वाहा॒ स्वाहा॑ विꣳश॒त्यै स्वाहा॒ स्वाहा॑ विꣳश॒त्यै स्वाहा॒ स्वाहा॑ विꣳश॒त्यै स्वाहा᳚ । \newline
4. विꣳ॒॒श॒त्यै स्वाहा॒ स्वाहा॑ विꣳश॒त्यै विꣳ॑श॒त्यै स्वाहा᳚ त्रिꣳ॒॒शते᳚ त्रिꣳ॒॒शते॒ स्वाहा॑ विꣳश॒त्यै विꣳ॑श॒त्यै स्वाहा᳚ त्रिꣳ॒॒शते᳚ । \newline
5. स्वाहा᳚ त्रिꣳ॒॒शते᳚ त्रिꣳ॒॒शते॒ स्वाहा॒ स्वाहा᳚ त्रिꣳ॒॒शते॒ स्वाहा॒ स्वाहा᳚ त्रिꣳ॒॒शते॒ स्वाहा॒ स्वाहा᳚ त्रिꣳ॒॒शते॒ स्वाहा᳚ । \newline
6. त्रिꣳ॒॒शते॒ स्वाहा॒ स्वाहा᳚ त्रिꣳ॒॒शते᳚ त्रिꣳ॒॒शते॒ स्वाहा॑ चत्वारिꣳ॒॒शते॑ चत्वारिꣳ॒॒शते॒ स्वाहा᳚ त्रिꣳ॒॒शते᳚ त्रिꣳ॒॒शते॒ स्वाहा॑ चत्वारिꣳ॒॒शते᳚ । \newline
7. स्वाहा॑ चत्वारिꣳ॒॒शते॑ चत्वारिꣳ॒॒शते॒ स्वाहा॒ स्वाहा॑ चत्वारिꣳ॒॒शते॒ स्वाहा॒ स्वाहा॑ चत्वारिꣳ॒॒शते॒ स्वाहा॒ स्वाहा॑ चत्वारिꣳ॒॒शते॒ स्वाहा᳚ । \newline
8. च॒त्वा॒रिꣳ॒॒शते॒ स्वाहा॒ स्वाहा॑ चत्वारिꣳ॒॒शते॑ चत्वारिꣳ॒॒शते॒ स्वाहा॑ पञ्चा॒शते॑ पञ्चा॒शते॒ स्वाहा॑ चत्वारिꣳ॒॒शते॑ चत्वारिꣳ॒॒शते॒ स्वाहा॑ पञ्चा॒शते᳚ । \newline
9. स्वाहा॑ पञ्चा॒शते॑ पञ्चा॒शते॒ स्वाहा॒ स्वाहा॑ पञ्चा॒शते॒ स्वाहा॒ स्वाहा॑ पञ्चा॒शते॒ स्वाहा॒ स्वाहा॑ पञ्चा॒शते॒ स्वाहा᳚ । \newline
10. प॒ञ्चा॒शते॒ स्वाहा॒ स्वाहा॑ पञ्चा॒शते॑ पञ्चा॒शते॒ स्वाहा॑ ष॒ष्ट्यै ष॒ष्ट्यै स्वाहा॑ पञ्चा॒शते॑ पञ्चा॒शते॒ स्वाहा॑ ष॒ष्ट्यै । \newline
11. स्वाहा॑ ष॒ष्ट्यै ष॒ष्ट्यै स्वाहा॒ स्वाहा॑ ष॒ष्ट्यै स्वाहा॒ स्वाहा॑ ष॒ष्ट्यै स्वाहा॒ स्वाहा॑ ष॒ष्ट्यै स्वाहा᳚ । \newline
12. ष॒ष्ट्यै स्वाहा॒ स्वाहा॑ ष॒ष्ट्यै ष॒ष्ट्यै स्वाहा॑ सप्त॒त्यै स॑प्त॒त्यै स्वाहा॑ ष॒ष्ट्यै ष॒ष्ट्यै स्वाहा॑ सप्त॒त्यै । \newline
13. स्वाहा॑ सप्त॒त्यै स॑प्त॒त्यै स्वाहा॒ स्वाहा॑ सप्त॒त्यै स्वाहा॒ स्वाहा॑ सप्त॒त्यै स्वाहा॒ स्वाहा॑ सप्त॒त्यै स्वाहा᳚ । \newline
14. स॒प्त॒त्यै स्वाहा॒ स्वाहा॑ सप्त॒त्यै स॑प्त॒त्यै स्वाहा॑ ऽशी॒त्या अ॑शी॒त्यै स्वाहा॑ सप्त॒त्यै स॑प्त॒त्यै स्वाहा॑ ऽशी॒त्यै । \newline
15. स्वाहा॑ ऽशी॒त्या अ॑शी॒त्यै स्वाहा॒ स्वाहा॑ ऽशी॒त्यै स्वाहा॒ स्वाहा॑ ऽशी॒त्यै स्वाहा॒ स्वाहा॑ ऽशी॒त्यै स्वाहा᳚ । \newline
16. अ॒शी॒त्यै स्वाहा॒ स्वाहा॑ ऽशी॒त्या अ॑शी॒त्यै स्वाहा॑ नव॒त्यै न॑व॒त्यै स्वाहा॑ ऽशी॒त्या अ॑शी॒त्यै स्वाहा॑ नव॒त्यै । \newline
17. स्वाहा॑ नव॒त्यै न॑व॒त्यै स्वाहा॒ स्वाहा॑ नव॒त्यै स्वाहा॒ स्वाहा॑ नव॒त्यै स्वाहा॒ स्वाहा॑ नव॒त्यै स्वाहा᳚ । \newline
18. न॒व॒त्यै स्वाहा॒ स्वाहा॑ नव॒त्यै न॑व॒त्यै स्वाहा॑ श॒ताय॑ श॒ताय॒ स्वाहा॑ नव॒त्यै न॑व॒त्यै स्वाहा॑ श॒ताय॑ । \newline
19. स्वाहा॑ श॒ताय॑ श॒ताय॒ स्वाहा॒ स्वाहा॑ श॒ताय॒ स्वाहा॒ स्वाहा॑ श॒ताय॒ स्वाहा॒ स्वाहा॑ श॒ताय॒ स्वाहा᳚ । \newline
20. श॒ताय॒ स्वाहा॒ स्वाहा॑ श॒ताय॑ श॒ताय॒ स्वाहा॒ सर्व॑स्मै॒ सर्व॑स्मै॒ स्वाहा॑ श॒ताय॑ श॒ताय॒ स्वाहा॒ सर्व॑स्मै । \newline
21. स्वाहा॒ सर्व॑स्मै॒ सर्व॑स्मै॒ स्वाहा॒ स्वाहा॒ सर्व॑स्मै॒ स्वाहा॒ स्वाहा॒ सर्व॑स्मै॒ स्वाहा॒ स्वाहा॒ सर्व॑स्मै॒ स्वाहा᳚ । \newline
22. सर्व॑स्मै॒ स्वाहा॒ स्वाहा॒ सर्व॑स्मै॒ सर्व॑स्मै॒ स्वाहा᳚ । \newline
23. स्वाहेति॒ स्वाहा᳚ । \newline
\pagebreak
\markright{ TS 7.2.18.1  \hfill https://www.vedavms.in \hfill}

\section{ TS 7.2.18.1 }

\textbf{TS 7.2.18.1 } \newline
\textbf{Samhita Paata} \newline

विꣳ॒॒श॒त्यै स्वाहा॑ चत्वारिꣳ॒॒शते॒ स्वाहा॑ ष॒ष्ट्यै स्वाहा॑ ऽशी॒त्यै स्वाहा॑ श॒ताय॒ स्वाहा॒ सर्व॑स्मै॒ स्वाहा᳚ ॥ \newline

\textbf{Pada Paata} \newline

विꣳ॒॒श॒त्यै । स्वाहा᳚ । च॒त्वा॒रिꣳ॒॒शते᳚ । स्वाहा᳚ । ष॒ष्ट्यै । स्वाहा᳚ । अ॒शी॒त्यै । स्वाहा᳚ । श॒ताय॑ । स्वाहा᳚ । सर्व॑स्मै । स्वाहा᳚ ॥  \newline


\textbf{Krama Paata} \newline

विꣳ॒॒श॒त्यै स्वाहा᳚ । स्वाहा॑ चत्वारिꣳ॒॒शते᳚ । च॒त्वा॒रिꣳ॒॒शते॒ स्वाहा᳚ । स्वाहा॑ ष॒ष्ट्‍यै । ष॒ष्ट्‍यै स्वाहा᳚ । स्वाहा॑ऽशी॒त्यै । अ॒शी॒त्यै स्वाहा᳚ । स्वाहा॑ श॒ताय॑ । श॒ताय॒ स्वाहा᳚ । स्वाहा॒ सर्व॑स्मै । सर्व॑स्मै॒ स्वाहा᳚ । स्वाहेति॒ स्वाहा᳚ । \newline

\textbf{Jatai Paata} \newline

1. विꣳ॒॒श॒त्यै स्वाहा॒ स्वाहा॑ विꣳश॒त्यै विꣳ॑श॒त्यै स्वाहा᳚ । \newline
2. स्वाहा॑ चत्वारिꣳ॒॒शते॑ चत्वारिꣳ॒॒शते॒ स्वाहा॒ स्वाहा॑ चत्वारिꣳ॒॒शते᳚ । \newline
3. च॒त्वा॒रिꣳ॒॒शते॒ स्वाहा॒ स्वाहा॑ चत्वारिꣳ॒॒शते॑ चत्वारिꣳ॒॒शते॒ स्वाहा᳚ । \newline
4. स्वाहा॑ ष॒ष्ट्यै ष॒ष्ट्यै स्वाहा॒ स्वाहा॑ ष॒ष्ट्यै । \newline
5. ष॒ष्ट्यै स्वाहा॒ स्वाहा॑ ष॒ष्ट्यै ष॒ष्ट्यै स्वाहा᳚ । \newline
6. स्वाहा॑ ऽशी॒त्या अ॑शी॒त्यै स्वाहा॒ स्वाहा॑ ऽशी॒त्यै । \newline
7. अ॒शी॒त्यै स्वाहा॒ स्वाहा॑ ऽशी॒त्या अ॑शी॒त्यै स्वाहा᳚ । \newline
8. स्वाहा॑ श॒ताय॑ श॒ताय॒ स्वाहा॒ स्वाहा॑ श॒ताय॑ । \newline
9. श॒ताय॒ स्वाहा॒ स्वाहा॑ श॒ताय॑ श॒ताय॒ स्वाहा᳚ । \newline
10. स्वाहा॒ सर्व॑स्मै॒ सर्व॑स्मै॒ स्वाहा॒ स्वाहा॒ सर्व॑स्मै । \newline
11. सर्व॑स्मै॒ स्वाहा॒ स्वाहा॒ सर्व॑स्मै॒ सर्व॑स्मै॒ स्वाहा᳚ । \newline
12. स्वाहेति॒ स्वाहा᳚ । \newline

\textbf{Ghana Paata } \newline

1. विꣳ॒॒श॒त्यै स्वाहा॒ स्वाहा॑ विꣳश॒त्यै विꣳ॑श॒त्यै स्वाहा॑ चत्वारिꣳ॒॒शते॑ चत्वारिꣳ॒॒शते॒ स्वाहा॑ विꣳश॒त्यै विꣳ॑श॒त्यै स्वाहा॑ चत्वारिꣳ॒॒शते᳚ । \newline
2. स्वाहा॑ चत्वारिꣳ॒॒शते॑ चत्वारिꣳ॒॒शते॒ स्वाहा॒ स्वाहा॑ चत्वारिꣳ॒॒शते॒ स्वाहा॒ स्वाहा॑ चत्वारिꣳ॒॒शते॒ स्वाहा॒ स्वाहा॑ चत्वारिꣳ॒॒शते॒ स्वाहा᳚ । \newline
3. च॒त्वा॒रिꣳ॒॒शते॒ स्वाहा॒ स्वाहा॑ चत्वारिꣳ॒॒शते॑ चत्वारिꣳ॒॒शते॒ स्वाहा॑ ष॒ष्ट्यै ष॒ष्ट्यै स्वाहा॑ चत्वारिꣳ॒॒शते॑ चत्वारिꣳ॒॒शते॒ स्वाहा॑ ष॒ष्ट्यै । \newline
4. स्वाहा॑ ष॒ष्ट्यै ष॒ष्ट्यै स्वाहा॒ स्वाहा॑ ष॒ष्ट्यै स्वाहा॒ स्वाहा॑ ष॒ष्ट्यै स्वाहा॒ स्वाहा॑ ष॒ष्ट्यै स्वाहा᳚ । \newline
5. ष॒ष्ट्यै स्वाहा॒ स्वाहा॑ ष॒ष्ट्यै ष॒ष्ट्यै स्वाहा॑ ऽशी॒त्या अ॑शी॒त्यै स्वाहा॑ ष॒ष्ट्यै ष॒ष्ट्यै स्वाहा॑ ऽशी॒त्यै । \newline
6. स्वाहा॑ ऽशी॒त्या अ॑शी॒त्यै स्वाहा॒ स्वाहा॑ ऽशी॒त्यै स्वाहा॒ स्वाहा॑ ऽशी॒त्यै स्वाहा॒ स्वाहा॑ ऽशी॒त्यै स्वाहा᳚ । \newline
7. अ॒शी॒त्यै स्वाहा॒ स्वाहा॑ ऽशी॒त्या अ॑शी॒त्यै स्वाहा॑ श॒ताय॑ श॒ताय॒ स्वाहा॑ ऽशी॒त्या अ॑शी॒त्यै स्वाहा॑ श॒ताय॑ । \newline
8. स्वाहा॑ श॒ताय॑ श॒ताय॒ स्वाहा॒ स्वाहा॑ श॒ताय॒ स्वाहा॒ स्वाहा॑ श॒ताय॒ स्वाहा॒ स्वाहा॑ श॒ताय॒ स्वाहा᳚ । \newline
9. श॒ताय॒ स्वाहा॒ स्वाहा॑ श॒ताय॑ श॒ताय॒ स्वाहा॒ सर्व॑स्मै॒ सर्व॑स्मै॒ स्वाहा॑ श॒ताय॑ श॒ताय॒ स्वाहा॒ सर्व॑स्मै । \newline
10. स्वाहा॒ सर्व॑स्मै॒ सर्व॑स्मै॒ स्वाहा॒ स्वाहा॒ सर्व॑स्मै॒ स्वाहा॒ स्वाहा॒ सर्व॑स्मै॒ स्वाहा॒ स्वाहा॒ सर्व॑स्मै॒ स्वाहा᳚ । \newline
11. सर्व॑स्मै॒ स्वाहा॒ स्वाहा॒ सर्व॑स्मै॒ सर्व॑स्मै॒ स्वाहा᳚ । \newline
12. स्वाहेति॒ स्वाहा᳚ । \newline
\pagebreak
\markright{ TS 7.2.19.1  \hfill https://www.vedavms.in \hfill}

\section{ TS 7.2.19.1 }

\textbf{TS 7.2.19.1 } \newline
\textbf{Samhita Paata} \newline

प॒ञ्चा॒शते॒ स्वाहा॑ श॒ताय॒ स्वाहा॒ द्वाभ्याꣳ॑ श॒ताभ्याꣳ॒॒ स्वाहा᳚ त्रि॒भ्यः श॒तेभ्यः॒ स्वाहा॑ च॒तुर्भ्यः॑ श॒तेभ्यः॒ स्वाहा॑ प॒ञ्चभ्यः॑ श॒तेभ्यः॒ स्वाहा॑ ष॒ड्भ्यः श॒तेभ्यः॒ स्वाहा॑ स॒प्तभ्यः॑ श॒तेभ्यः॒ स्वाहा᳚ ऽष्टा॒भ्यः श॒तेभ्यः॒ स्वाहा॑ न॒वभ्यः॑ श॒तेभ्यः॒ स्वाहा॑ स॒हस्रा॑य॒ स्वाहा॒ सर्व॑स्मै॒ स्वाहा᳚ ॥ \newline

\textbf{Pada Paata} \newline

प॒ञ्चा॒शते᳚ । स्वाहा᳚ । श॒ताय॑ । स्वाहा᳚ । द्वाभ्या᳚म् । श॒ताभ्या᳚म् । स्वाहा᳚ । त्रि॒भ्य इति॑ त्रि - भ्यः । श॒तेभ्यः॑ । स्वाहा᳚ । च॒तुर्भ्य॒ इति॑ च॒तुः - भ्यः॒ । श॒तेभ्यः॑ । स्वाहा᳚ । प॒ञ्चभ्य॒ इति॑ प॒ञ्च-भ्यः॒ । श॒तेभ्यः॑ । स्वाहा᳚ । ष॒ड्भ्य इति॑ षट्- भ्यः । श॒तेभ्यः॑ । स्वाहा᳚ । स॒प्तभ्य॒ इति॑ स॒प्त-भ्यः॒ । श॒तेभ्यः॑ । स्वाहा᳚ । अ॒ष्टा॒भ्यः । श॒तेभ्यः॑ । स्वाहा᳚ । न॒वभ्य॒ इति॑ न॒व - भ्यः॒ । श॒तेभ्यः॑ । स्वाहा᳚ । स॒हस्रा॑य । स्वाहा᳚ । सर्व॑स्मै । स्वाहा᳚ ॥  \newline


\textbf{Krama Paata} \newline

प॒ञ्चा॒शते॒ स्वाहा᳚ । स्वाहा॑ श॒ताय॑ । श॒ताय॒ स्वाहा᳚ । स्वाहा॒ द्वाभ्या᳚म् । द्वाभ्याꣳ॑ श॒ताभ्या᳚म् । श॒ताभ्याꣳ॒॒ स्वाहा᳚ । स्वाहा᳚ त्रि॒भ्यः । त्रि॒भ्यः श॒तेभ्यः॑ । त्रि॒भ्य इति॑ त्रि - भ्यः । श॒तेभ्यः॒ स्वाहा᳚ । स्वाहा॑ च॒तुर्भ्यः॑ । च॒तुर्भ्यः॑ श॒तेभ्यः॑ । च॒तुर्भ्य॒ इति॑ च॒तुः - भ्यः॒ । श॒तेभ्यः॒ स्वाहा᳚ । स्वाहा॑ प॒ञ्चभ्यः॑ । प॒ञ्चभ्यः॑ श॒तेभ्यः॑ । प॒ञ्चभ्य॒ इति॑ प॒ञ्च - भ्यः॒ । श॒तेभ्यः॒ स्वाहा᳚ । स्वाहा॑ ष॒ड्भ्यः । ष॒ड्भ्यः श॒तेभ्यः॑ । ष॒ड्भ्य इति॑ षट् - भ्यः । श॒तेभ्यः॒ स्वाहा᳚ । स्वाहा॑ स॒प्तभ्यः॑ । स॒प्तभ्यः॑ श॒तेभ्यः॑ । स॒प्तभ्य॒ इति॑ स॒प्त - भ्यः॒ । श॒तेभ्यः॒ स्वाहा᳚ । स्वाहा᳚ऽष्टा॒भ्यः । अ॒ष्टा॒भ्यः श॒तेभ्यः॑ । श॒तेभ्यः॒ स्वाहा᳚ । स्वाहा॑ न॒वभ्यः॑ । न॒वभ्यः॑ श॒तेभ्यः॑ । न॒वभ्य॒ इति॑ न॒व - भ्यः॒ । श॒तेभ्यः॒ स्वाहा᳚ । स्वाहा॑ स॒हस्रा॑य । स॒हस्रा॑य॒ स्वाहा᳚ । स्वाहा॒ सर्व॑स्मै । सर्व॑स्मै॒ स्वाहा᳚ । स्वाहेति॒ स्वाहा᳚ । \newline

\textbf{Jatai Paata} \newline

1. प॒ञ्चा॒शते॒ स्वाहा॒ स्वाहा॑ पञ्चा॒शते॑ पञ्चा॒शते॒ स्वाहा᳚ । \newline
2. स्वाहा॑ श॒ताय॑ श॒ताय॒ स्वाहा॒ स्वाहा॑ श॒ताय॑ । \newline
3. श॒ताय॒ स्वाहा॒ स्वाहा॑ श॒ताय॑ श॒ताय॒ स्वाहा᳚ । \newline
4. स्वाहा॒ द्वाभ्या॒म् द्वाभ्याꣳ॒॒ स्वाहा॒ स्वाहा॒ द्वाभ्या᳚म् । \newline
5. द्वाभ्याꣳ॑ श॒ताभ्याꣳ॑ श॒ताभ्या॒म् द्वाभ्या॒म् द्वाभ्याꣳ॑ श॒ताभ्या᳚म् । \newline
6. श॒ताभ्याꣳ॒॒ स्वाहा॒ स्वाहा॑ श॒ताभ्याꣳ॑ श॒ताभ्याꣳ॒॒ स्वाहा᳚ । \newline
7. स्वाहा᳚ त्रि॒भ्य स्त्रि॒भ्यः स्वाहा॒ स्वाहा᳚ त्रि॒भ्यः । \newline
8. त्रि॒भ्यः श॒तेभ्यः॑ श॒तेभ्य॑ स्त्रि॒भ्य स्त्रि॒भ्यः श॒तेभ्यः॑ । \newline
9. त्रि॒भ्य इति॑ त्रि - भ्यः । \newline
10. श॒तेभ्यः॒ स्वाहा॒ स्वाहा॑ श॒तेभ्यः॑ श॒तेभ्यः॒ स्वाहा᳚ । \newline
11. स्वाहा॑ च॒तुर्भ्य॑ श्च॒तुर्भ्यः॒ स्वाहा॒ स्वाहा॑ च॒तुर्भ्यः॑ । \newline
12. च॒तुर्भ्यः॑ श॒तेभ्यः॑ श॒तेभ्य॑ श्च॒तुर्भ्य॑ श्च॒तुर्भ्यः॑ श॒तेभ्यः॑ । \newline
13. च॒तुर्भ्य॒ इति॑ च॒तुः - भ्यः॒ । \newline
14. श॒तेभ्यः॒ स्वाहा॒ स्वाहा॑ श॒तेभ्यः॑ श॒तेभ्यः॒ स्वाहा᳚ । \newline
15. स्वाहा॑ प॒ञ्चभ्यः॑ प॒ञ्चभ्यः॒ स्वाहा॒ स्वाहा॑ प॒ञ्चभ्यः॑ । \newline
16. प॒ञ्चभ्यः॑ श॒तेभ्यः॑ श॒तेभ्यः॑ प॒ञ्चभ्यः॑ प॒ञ्चभ्यः॑ श॒तेभ्यः॑ । \newline
17. प॒ञ्चभ्य॒ इति॑ प॒ञ्च - भ्यः॒ । \newline
18. श॒तेभ्यः॒ स्वाहा॒ स्वाहा॑ श॒तेभ्यः॑ श॒तेभ्यः॒ स्वाहा᳚ । \newline
19. स्वाहा॑ ष॒ड्भ्य ष्ष॒ड्भ्यः स्वाहा॒ स्वाहा॑ ष॒ड्भ्यः । \newline
20. ष॒ड्भ्यः श॒तेभ्यः॑ श॒तेभ्य॑ ष्ष॒ड्भ्य ष्ष॒ड्भ्यः श॒तेभ्यः॑ । \newline
21. ष॒ड्भ्य इति॑ षट् - भ्यः । \newline
22. श॒तेभ्यः॒ स्वाहा॒ स्वाहा॑ श॒तेभ्यः॑ श॒तेभ्यः॒ स्वाहा᳚ । \newline
23. स्वाहा॑ स॒प्तभ्यः॑ स॒प्तभ्यः॒ स्वाहा॒ स्वाहा॑ स॒प्तभ्यः॑ । \newline
24. स॒प्तभ्यः॑ श॒तेभ्यः॑ श॒तेभ्यः॑ स॒प्तभ्यः॑ स॒प्तभ्यः॑ श॒तेभ्यः॑ । \newline
25. स॒प्तभ्य॒ इति॑ स॒प्त - भ्यः॒ । \newline
26. श॒तेभ्यः॒ स्वाहा॒ स्वाहा॑ श॒तेभ्यः॑ श॒तेभ्यः॒ स्वाहा᳚ । \newline
27. स्वाहा᳚ ऽष्टा॒भ्यो᳚ ऽष्टा॒भ्यः स्वाहा॒ स्वाहा᳚ ऽष्टा॒भ्यः । \newline
28. अ॒ष्टा॒भ्यः श॒तेभ्यः॑ श॒तेभ्यो᳚ ऽष्टा॒भ्यो᳚ ऽष्टा॒भ्यः श॒तेभ्यः॑ । \newline
29. श॒तेभ्यः॒ स्वाहा॒ स्वाहा॑ श॒तेभ्यः॑ श॒तेभ्यः॒ स्वाहा᳚ । \newline
30. स्वाहा॑ न॒वभ्यो॑ न॒वभ्यः॒ स्वाहा॒ स्वाहा॑ न॒वभ्यः॑ । \newline
31. न॒वभ्यः॑ श॒तेभ्यः॑ श॒तेभ्यो॑ न॒वभ्यो॑ न॒वभ्यः॑ श॒तेभ्यः॑ । \newline
32. न॒वभ्य॒ इति॑ न॒व - भ्यः॒ । \newline
33. श॒तेभ्यः॒ स्वाहा॒ स्वाहा॑ श॒तेभ्यः॑ श॒तेभ्यः॒ स्वाहा᳚ । \newline
34. स्वाहा॑ स॒हस्रा॑य स॒हस्रा॑य॒ स्वाहा॒ स्वाहा॑ स॒हस्रा॑य । \newline
35. स॒हस्रा॑य॒ स्वाहा॒ स्वाहा॑ स॒हस्रा॑य स॒हस्रा॑य॒ स्वाहा᳚ । \newline
36. स्वाहा॒ सर्व॑स्मै॒ सर्व॑स्मै॒ स्वाहा॒ स्वाहा॒ सर्व॑स्मै । \newline
37. सर्व॑स्मै॒ स्वाहा॒ स्वाहा॒ सर्व॑स्मै॒ सर्व॑स्मै॒ स्वाहा᳚ । \newline
38. स्वाहेति॒ स्वाहा᳚ । \newline

\textbf{Ghana Paata } \newline

1. प॒ञ्चा॒शते॒ स्वाहा॒ स्वाहा॑ पञ्चा॒शते॑ पञ्चा॒शते॒ स्वाहा॑ श॒ताय॑ श॒ताय॒ स्वाहा॑ पञ्चा॒शते॑ पञ्चा॒शते॒ स्वाहा॑ श॒ताय॑ । \newline
2. स्वाहा॑ श॒ताय॑ श॒ताय॒ स्वाहा॒ स्वाहा॑ श॒ताय॒ स्वाहा॒ स्वाहा॑ श॒ताय॒ स्वाहा॒ स्वाहा॑ श॒ताय॒ स्वाहा᳚ । \newline
3. श॒ताय॒ स्वाहा॒ स्वाहा॑ श॒ताय॑ श॒ताय॒ स्वाहा॒ द्वाभ्या॒म् द्वाभ्याꣳ॒॒ स्वाहा॑ श॒ताय॑ श॒ताय॒ स्वाहा॒ द्वाभ्या᳚म् । \newline
4. स्वाहा॒ द्वाभ्या॒म् द्वाभ्याꣳ॒॒ स्वाहा॒ स्वाहा॒ द्वाभ्याꣳ॑ श॒ताभ्याꣳ॑ श॒ताभ्या॒म् द्वाभ्याꣳ॒॒ स्वाहा॒ स्वाहा॒ द्वाभ्याꣳ॑ श॒ताभ्या᳚म् । \newline
5. द्वाभ्याꣳ॑ श॒ताभ्याꣳ॑ श॒ताभ्या॒म् द्वाभ्या॒म् द्वाभ्याꣳ॑ श॒ताभ्याꣳ॒॒ स्वाहा॒ स्वाहा॑ श॒ताभ्या॒म् द्वाभ्या॒म् द्वाभ्याꣳ॑ श॒ताभ्याꣳ॒॒ स्वाहा᳚ । \newline
6. श॒ताभ्याꣳ॒॒ स्वाहा॒ स्वाहा॑ श॒ताभ्याꣳ॑ श॒ताभ्याꣳ॒॒ स्वाहा᳚ त्रि॒भ्य स्त्रि॒भ्यः स्वाहा॑ श॒ताभ्याꣳ॑ श॒ताभ्याꣳ॒॒ स्वाहा᳚ त्रि॒भ्यः । \newline
7. स्वाहा᳚ त्रि॒भ्य स्त्रि॒भ्यः स्वाहा॒ स्वाहा᳚ त्रि॒भ्यः श॒तेभ्यः॑ श॒तेभ्य॑ स्त्रि॒भ्यः स्वाहा॒ स्वाहा᳚ त्रि॒भ्यः श॒तेभ्यः॑ । \newline
8. त्रि॒भ्यः श॒तेभ्यः॑ श॒तेभ्य॑ स्त्रि॒भ्य स्त्रि॒भ्यः श॒तेभ्यः॒ स्वाहा॒ स्वाहा॑ श॒तेभ्य॑ स्त्रि॒भ्य स्त्रि॒भ्यः श॒तेभ्यः॒ स्वाहा᳚ । \newline
9. त्रि॒भ्य इति॑ त्रि - भ्यः । \newline
10. श॒तेभ्यः॒ स्वाहा॒ स्वाहा॑ श॒तेभ्यः॑ श॒तेभ्यः॒ स्वाहा॑ च॒तुर्भ्य॑ श्च॒तुर्भ्यः॒ स्वाहा॑ श॒तेभ्यः॑ श॒तेभ्यः॒ स्वाहा॑ च॒तुर्भ्यः॑ । \newline
11. स्वाहा॑ च॒तुर्भ्य॑ श्च॒तुर्भ्यः॒ स्वाहा॒ स्वाहा॑ च॒तुर्भ्यः॑ श॒तेभ्यः॑ श॒तेभ्य॑ श्च॒तुर्भ्यः॒ स्वाहा॒ स्वाहा॑ च॒तुर्भ्यः॑ श॒तेभ्यः॑ । \newline
12. च॒तुर्भ्यः॑ श॒तेभ्यः॑ श॒तेभ्य॑ श्च॒तुर्भ्य॑ श्च॒तुर्भ्यः॑ श॒तेभ्यः॒ स्वाहा॒ स्वाहा॑ श॒तेभ्य॑ श्च॒तुर्भ्य॑ श्च॒तुर्भ्यः॑ श॒तेभ्यः॒ स्वाहा᳚ । \newline
13. च॒तुर्भ्य॒ इति॑ च॒तुः - भ्यः॒ । \newline
14. श॒तेभ्यः॒ स्वाहा॒ स्वाहा॑ श॒तेभ्यः॑ श॒तेभ्यः॒ स्वाहा॑ प॒ञ्चभ्यः॑ प॒ञ्चभ्यः॒ स्वाहा॑ श॒तेभ्यः॑ श॒तेभ्यः॒ स्वाहा॑ प॒ञ्चभ्यः॑ । \newline
15. स्वाहा॑ प॒ञ्चभ्यः॑ प॒ञ्चभ्यः॒ स्वाहा॒ स्वाहा॑ प॒ञ्चभ्यः॑ श॒तेभ्यः॑ श॒तेभ्यः॑ प॒ञ्चभ्यः॒ स्वाहा॒ स्वाहा॑ प॒ञ्चभ्यः॑ श॒तेभ्यः॑ । \newline
16. प॒ञ्चभ्यः॑ श॒तेभ्यः॑ श॒तेभ्यः॑ प॒ञ्चभ्यः॑ प॒ञ्चभ्यः॑ श॒तेभ्यः॒ स्वाहा॒ स्वाहा॑ श॒तेभ्यः॑ प॒ञ्चभ्यः॑ प॒ञ्चभ्यः॑ श॒तेभ्यः॒ स्वाहा᳚ । \newline
17. प॒ञ्चभ्य॒ इति॑ प॒ञ्च - भ्यः॒ । \newline
18. श॒तेभ्यः॒ स्वाहा॒ स्वाहा॑ श॒तेभ्यः॑ श॒तेभ्यः॒ स्वाहा॑ ष॒ड्भ्य ष्ष॒ड्भ्यः स्वाहा॑ श॒तेभ्यः॑ श॒तेभ्यः॒ स्वाहा॑ ष॒ड्भ्यः । \newline
19. स्वाहा॑ ष॒ड्भ्य ष्ष॒ड्भ्यः स्वाहा॒ स्वाहा॑ ष॒ड्भ्यः श॒तेभ्यः॑ श॒तेभ्य॑ ष्ष॒ड्भ्यः स्वाहा॒ स्वाहा॑ ष॒ड्भ्यः श॒तेभ्यः॑ । \newline
20. ष॒ड्भ्यः श॒तेभ्यः॑ श॒तेभ्य॑ ष्ष॒ड्भ्य ष्ष॒ड्भ्यः श॒तेभ्यः॒ स्वाहा॒ स्वाहा॑ श॒तेभ्य॑ ष्ष॒ड्भ्य ष्ष॒ड्भ्यः श॒तेभ्यः॒ स्वाहा᳚ । \newline
21. ष॒ड्भ्य इति॑ षट् - भ्यः । \newline
22. श॒तेभ्यः॒ स्वाहा॒ स्वाहा॑ श॒तेभ्यः॑ श॒तेभ्यः॒ स्वाहा॑ स॒प्तभ्यः॑ स॒प्तभ्यः॒ स्वाहा॑ श॒तेभ्यः॑ श॒तेभ्यः॒ स्वाहा॑ स॒प्तभ्यः॑ । \newline
23. स्वाहा॑ स॒प्तभ्यः॑ स॒प्तभ्यः॒ स्वाहा॒ स्वाहा॑ स॒प्तभ्यः॑ श॒तेभ्यः॑ श॒तेभ्यः॑ स॒प्तभ्यः॒ स्वाहा॒ स्वाहा॑ स॒प्तभ्यः॑ श॒तेभ्यः॑ । \newline
24. स॒प्तभ्यः॑ श॒तेभ्यः॑ श॒तेभ्यः॑ स॒प्तभ्यः॑ स॒प्तभ्यः॑ श॒तेभ्यः॒ स्वाहा॒ स्वाहा॑ श॒तेभ्यः॑ स॒प्तभ्यः॑ स॒प्तभ्यः॑ श॒तेभ्यः॒ स्वाहा᳚ । \newline
25. स॒प्तभ्य॒ इति॑ स॒प्त - भ्यः॒ । \newline
26. श॒तेभ्यः॒ स्वाहा॒ स्वाहा॑ श॒तेभ्यः॑ श॒तेभ्यः॒ स्वाहा᳚ ऽष्टा॒भ्यो᳚ ऽष्टा॒भ्यः स्वाहा॑ श॒तेभ्यः॑ श॒तेभ्यः॒ स्वाहा᳚ ऽष्टा॒भ्यः । \newline
27. स्वाहा᳚ ऽष्टा॒भ्यो᳚ ऽष्टा॒भ्यः स्वाहा॒ स्वाहा᳚ ऽष्टा॒भ्यः श॒तेभ्यः॑ श॒तेभ्यो᳚ ऽष्टा॒भ्यः स्वाहा॒ स्वाहा᳚ ऽष्टा॒भ्यः श॒तेभ्यः॑ । \newline
28. अ॒ष्टा॒भ्यः श॒तेभ्यः॑ श॒तेभ्यो᳚ ऽष्टा॒भ्यो᳚ ऽष्टा॒भ्यः श॒तेभ्यः॒ स्वाहा॒ स्वाहा॑ श॒तेभ्यो᳚ ऽष्टा॒भ्यो᳚ ऽष्टा॒भ्यः श॒तेभ्यः॒ स्वाहा᳚ । \newline
29. श॒तेभ्यः॒ स्वाहा॒ स्वाहा॑ श॒तेभ्यः॑ श॒तेभ्यः॒ स्वाहा॑ न॒वभ्यो॑ न॒वभ्यः॒ स्वाहा॑ श॒तेभ्यः॑ श॒तेभ्यः॒ स्वाहा॑ न॒वभ्यः॑ । \newline
30. स्वाहा॑ न॒वभ्यो॑ न॒वभ्यः॒ स्वाहा॒ स्वाहा॑ न॒वभ्यः॑ श॒तेभ्यः॑ श॒तेभ्यो॑ न॒वभ्यः॒ स्वाहा॒ स्वाहा॑ न॒वभ्यः॑ श॒तेभ्यः॑ । \newline
31. न॒वभ्यः॑ श॒तेभ्यः॑ श॒तेभ्यो॑ न॒वभ्यो॑ न॒वभ्यः॑ श॒तेभ्यः॒ स्वाहा॒ स्वाहा॑ श॒तेभ्यो॑ न॒वभ्यो॑ न॒वभ्यः॑ श॒तेभ्यः॒ स्वाहा᳚ । \newline
32. न॒वभ्य॒ इति॑ न॒व - भ्यः॒ । \newline
33. श॒तेभ्यः॒ स्वाहा॒ स्वाहा॑ श॒तेभ्यः॑ श॒तेभ्यः॒ स्वाहा॑ स॒हस्रा॑य स॒हस्रा॑य॒ स्वाहा॑ श॒तेभ्यः॑ श॒तेभ्यः॒ स्वाहा॑ स॒हस्रा॑य । \newline
34. स्वाहा॑ स॒हस्रा॑य स॒हस्रा॑य॒ स्वाहा॒ स्वाहा॑ स॒हस्रा॑य॒ स्वाहा॒ स्वाहा॑ स॒हस्रा॑य॒ स्वाहा॒ स्वाहा॑ स॒हस्रा॑य॒ स्वाहा᳚ । \newline
35. स॒हस्रा॑य॒ स्वाहा॒ स्वाहा॑ स॒हस्रा॑य स॒हस्रा॑य॒ स्वाहा॒ सर्व॑स्मै॒ सर्व॑स्मै॒ स्वाहा॑ स॒हस्रा॑य स॒हस्रा॑य॒ स्वाहा॒ सर्व॑स्मै । \newline
36. स्वाहा॒ सर्व॑स्मै॒ सर्व॑स्मै॒ स्वाहा॒ स्वाहा॒ सर्व॑स्मै॒ स्वाहा॒ स्वाहा॒ सर्व॑स्मै॒ स्वाहा॒ स्वाहा॒ सर्व॑स्मै॒ स्वाहा᳚ । \newline
37. सर्व॑स्मै॒ स्वाहा॒ स्वाहा॒ सर्व॑स्मै॒ सर्व॑स्मै॒ स्वाहा᳚ । \newline
38. स्वाहेति॒ स्वाहा᳚ । \newline
\pagebreak
\markright{ TS 7.2.20.1  \hfill https://www.vedavms.in \hfill}

\section{ TS 7.2.20.1 }

\textbf{TS 7.2.20.1 } \newline
\textbf{Samhita Paata} \newline

श॒ताय॒ स्वाहा॑ स॒हस्रा॑य॒ स्वाहा॒ ऽयुता॑य॒ स्वाहा॑ नि॒युता॑य॒ स्वाहा᳚ प्र॒युता॑य॒ स्वाहा ऽर्बु॑दाय॒ स्वाहा॒ न्य॑र्बुदाय॒ स्वाहा॑ समु॒द्राय॒ स्वाहा॒ मद्ध्या॑य॒ स्वाहा ऽन्ता॑य॒ स्वाहा॑ परा॒र्द्धाय॒ स्वाहो॒षसे॒ स्वाहा॒ व्यु॑ष्ट्यै॒ स्वाहो॑देष्य॒ते स्वाहो᳚द्य॒ते स्वाहोदि॑ताय॒ स्वाहा॑ सुव॒र्गाय॒ स्वाहा॑ लो॒काय॒ स्वाहा॒ सर्व॑स्मै॒ स्वाहा᳚ ॥ \newline

\textbf{Pada Paata} \newline

श॒ताय॑ । स्वाहा᳚ । स॒हस्रा॑य । स्वाहा᳚ । अ॒युता॑य । स्वाहा᳚ । नि॒युता॒येति॑ नि - युता॑य । स्वाहा᳚ । प्र॒युता॒येति॑ प्र - युता॑य । स्वाहा᳚ । अर्बु॑दाय । स्वाहा᳚ । न्य॑र्बुदा॒येति॒ नि - अ॒र्बु॒दा॒य॒ । स्वाहा᳚ । स॒मु॒द्राय॑ । स्वाहा᳚ । मद्ध्या॑य । स्वाहा᳚ । अन्ता॑य । स्वाहा᳚ । प॒रा॒द्‌र्धायेति॑ पर-अ॒द्‌र्धाय॑ । स्वाहा᳚ । उ॒षसे᳚ । स्वाहा᳚ । व्यु॑ष्ट्या॒ इति॒ वि - उ॒ष्ट्यै॒ । स्वाहा᳚ । उ॒दे॒ष्य॒त इत्यु॑त् - ए॒ष्य॒ते । स्वाहा᳚ । उ॒द्य॒त इत्यु॑त् - य॒ते । स्वाहा᳚ । उदि॑ता॒येत्युत् - इ॒ता॒य॒ । स्वाहा᳚ । सु॒व॒र्गायेति॑ सुवः-गाय॑ । स्वाहा᳚ । लो॒काय॑ । स्वाहा᳚ । सर्व॑स्मै । स्वाहा᳚ ॥  \newline


\textbf{Krama Paata} \newline

श॒ताय॒ स्वाहा᳚ । स्वाहा॑ स॒हस्रा॑य । स॒हस्रा॑य॒ स्वाहा᳚ । स्वाहा॒ऽयुता॑य । अ॒युता॑य॒ स्वाहा᳚ । स्वाहा॑ नि॒युता॑य । नि॒युता॑य॒ स्वाहा᳚ । नि॒युता॒येति॑ नि - युता॑य । स्वाहा᳚ प्र॒युता॑य । प्र॒युता॑य॒ स्वाहा᳚ । प्र॒युता॒येति॑ प्र - युता॑य । स्वाहाऽर्बु॑दाय । अर्बु॑दाय॒ स्वाहा᳚ । स्वाहा॒ न्य॑र्बुदाय । न्य॑र्बुदाय॒ स्वाहा᳚ । न्य॑र्बुदा॒येति॒ नि - अ॒र्बु॒दा॒य॒ । स्वाहा॑ समु॒द्राय॑ । स॒मु॒द्राय॒ स्वाहा᳚ । स्वाहा॒ मद्ध्या॑य । मद्ध्या॑य॒ स्वाहा᳚ । स्वाहाऽन्ता॑य । अन्ता॑य॒ स्वाहा᳚ । स्वाहा॑ परा॒र्द्धाय॑ । प॒रा॒र्द्धाय॒ स्वाहा᳚ । प॒रा॒र्द्धायेति॑ पर - अ॒र्द्धाय॑ । स्वाहो॒षसे᳚ । उ॒षसे॒ स्वाहा᳚ । स्वाहा॒ व्यु॑ष्ट्‍यै । व्यु॑ष्ट्‍यै॒ स्वाहा᳚ । व्यु॑ष्ट्‍या॒ इति॒ वि - उ॒ष्ट्‍यै॒ । स्वाहो॑देष्य॒ते । उ॒दे॒ष्य॒ते स्वाहा᳚ । उ॒दे॒ष्य॒त इत्यु॑त् - ए॒ष्य॒ते । स्वाहो᳚द्य॒ते । उ॒द्य॒ते स्वाहा᳚ । उ॒द्य॒त इत्यु॑त् - य॒ते । स्वाहोदि॑ताय । उदि॑ताय॒ स्वाहा᳚ । उदि॑ता॒येत्युत् - इ॒ता॒य॒ । स्वाहा॑ सुव॒र्गाय॑ । सु॒व॒र्गाय॒ स्वाहा᳚ । सु॒व॒र्गायेति॑ सुवः - गाय॑ । स्वाहा॑ लो॒काय॑ । लो॒काय॒ स्वाहा᳚ । स्वाहा॒ सर्व॑स्मै । सर्व॑स्मै॒ स्वाहा᳚ । स्वाहेति॒ स्वाहा᳚ । \newline

\textbf{Jatai Paata} \newline

1. श॒ताय॒ स्वाहा॒ स्वाहा॑ श॒ताय॑ श॒ताय॒ स्वाहा᳚ । \newline
2. स्वाहा॑ स॒हस्रा॑य स॒हस्रा॑य॒ स्वाहा॒ स्वाहा॑ स॒हस्रा॑य । \newline
3. स॒हस्रा॑य॒ स्वाहा॒ स्वाहा॑ स॒हस्रा॑य स॒हस्रा॑य॒ स्वाहा᳚ । \newline
4. स्वाहा॒ ऽयुता॑या॒ युता॑य॒ स्वाहा॒ स्वाहा॒ ऽयुता॑य । \newline
5. अ॒युता॑य॒ स्वाहा॒ स्वाहा॒ ऽयुता॑या॒ युता॑य॒ स्वाहा᳚ । \newline
6. स्वाहा॑ नि॒युता॑य नि॒युता॑य॒ स्वाहा॒ स्वाहा॑ नि॒युता॑य । \newline
7. नि॒युता॑य॒ स्वाहा॒ स्वाहा॑ नि॒युता॑य नि॒युता॑य॒ स्वाहा᳚ । \newline
8. नि॒युता॒येति॑ नि - युता॑य । \newline
9. स्वाहा᳚ प्र॒युता॑य प्र॒युता॑य॒ स्वाहा॒ स्वाहा᳚ प्र॒युता॑य । \newline
10. प्र॒युता॑य॒ स्वाहा॒ स्वाहा᳚ प्र॒युता॑य प्र॒युता॑य॒ स्वाहा᳚ । \newline
11. प्र॒युता॒येति॑ प्र - युता॑य । \newline
12. स्वाहा ऽर्बु॑दा॒या र्बु॑दाय॒ स्वाहा॒ स्वाहा ऽर्बु॑दाय । \newline
13. अर्बु॑दाय॒ स्वाहा॒ स्वाहा ऽर्बु॑दा॒या र्बु॑दाय॒ स्वाहा᳚ । \newline
14. स्वाहा॒ न्य॑र्बुदाय॒ न्य॑र्बुदाय॒ स्वाहा॒ स्वाहा॒ न्य॑र्बुदाय । \newline
15. न्य॑र्बुदाय॒ स्वाहा॒ स्वाहा॒ न्य॑र्बुदाय॒ न्य॑र्बुदाय॒ स्वाहा᳚ । \newline
16. न्य॑र्बुदा॒येति॒ नि - अ॒र्बु॒दा॒य॒ । \newline
17. स्वाहा॑ समु॒द्राय॑ समु॒द्राय॒ स्वाहा॒ स्वाहा॑ समु॒द्राय॑ । \newline
18. स॒मु॒द्राय॒ स्वाहा॒ स्वाहा॑ समु॒द्राय॑ समु॒द्राय॒ स्वाहा᳚ । \newline
19. स्वाहा॒ मद्ध्या॑य॒ मद्ध्या॑य॒ स्वाहा॒ स्वाहा॒ मद्ध्या॑य । \newline
20. मद्ध्या॑य॒ स्वाहा॒ स्वाहा॒ मद्ध्या॑य॒ मद्ध्या॑य॒ स्वाहा᳚ । \newline
21. स्वाहा ऽन्ता॒यान्ता॑य॒ स्वाहा॒ स्वाहा ऽन्ता॑य । \newline
22. अन्ता॑य॒ स्वाहा॒ स्वाहा ऽन्ता॒या न्ता॑य॒ स्वाहा᳚ । \newline
23. स्वाहा॑ परा॒र्द्धाय॑ परा॒र्द्धाय॒ स्वाहा॒ स्वाहा॑ परा॒र्द्धाय॑ । \newline
24. प॒रा॒र्द्धाय॒ स्वाहा॒ स्वाहा॑ परा॒र्द्धाय॑ परा॒र्द्धाय॒ स्वाहा᳚ । \newline
25. प॒रा॒र्द्धायेति॑ पर - अ॒र्द्धाय॑ । \newline
26. स्वाहो॒ षस॑ उ॒षसे॒ स्वाहा॒ स्वाहो॒ षसे᳚ । \newline
27. उ॒षसे॒ स्वाहा॒ स्वाहो॒ षस॑ उ॒षसे॒ स्वाहा᳚ । \newline
28. स्वाहा॒ व्यु॑ष्ट्यै॒ व्यु॑ष्ट्यै॒ स्वाहा॒ स्वाहा॒ व्यु॑ष्ट्यै । \newline
29. व्यु॑ष्ट्यै॒ स्वाहा॒ स्वाहा॒ व्यु॑ष्ट्यै॒ व्यु॑ष्ट्यै॒ स्वाहा᳚ । \newline
30. व्यु॑ष्ट्या॒ इति॒ वि - उ॒ष्ट्यै॒ । \newline
31. स्वाहो॑देष्य॒त उ॑देष्य॒ते स्वाहा॒ स्वाहो॑देष्य॒ते । \newline
32. उ॒दे॒ष्य॒ते स्वाहा॒ स्वाहो॑देष्य॒त उ॑देष्य॒ते स्वाहा᳚ । \newline
33. उ॒दे॒ष्य॒त इत्यु॑त् - ए॒ष्य॒ते । \newline
34. स्वाहो᳚ द्य॒त उ॑द्य॒ते स्वाहा॒ स्वाहो᳚ द्य॒ते । \newline
35. उ॒द्य॒ते स्वाहा॒ स्वाहो᳚ द्य॒त उ॑द्य॒ते स्वाहा᳚ । \newline
36. उ॒द्य॒त इत्यु॑त् - य॒ते । \newline
37. स्वाहोदि॑ता॒यो दि॑ताय॒ स्वाहा॒ स्वाहोदि॑ताय । \newline
38. उदि॑ताय॒ स्वाहा॒ स्वाहोदि॑ता॒यो दि॑ताय॒ स्वाहा᳚ । \newline
39. उदि॑ता॒येत्युत् - इ॒ता॒य॒ । \newline
40. स्वाहा॑ सुव॒र्गाय॑ सुव॒र्गाय॒ स्वाहा॒ स्वाहा॑ सुव॒र्गाय॑ । \newline
41. सु॒व॒र्गाय॒ स्वाहा॒ स्वाहा॑ सुव॒र्गाय॑ सुव॒र्गाय॒ स्वाहा᳚ । \newline
42. सु॒व॒र्गायेति॑ सुवः - गाय॑ । \newline
43. स्वाहा॑ लो॒काय॑ लो॒काय॒ स्वाहा॒ स्वाहा॑ लो॒काय॑ । \newline
44. लो॒काय॒ स्वाहा॒ स्वाहा॑ लो॒काय॑ लो॒काय॒ स्वाहा᳚ । \newline
45. स्वाहा॒ सर्व॑स्मै॒ सर्व॑स्मै॒ स्वाहा॒ स्वाहा॒ सर्व॑स्मै । \newline
46. सर्व॑स्मै॒ स्वाहा॒ स्वाहा॒ सर्व॑स्मै॒ सर्व॑स्मै॒ स्वाहा᳚ । \newline
47. स्वाहेति॒ स्वाहा᳚ । \newline

\textbf{Ghana Paata } \newline

1. श॒ताय॒ स्वाहा॒ स्वाहा॑ श॒ताय॑ श॒ताय॒ स्वाहा॑ स॒हस्रा॑य स॒हस्रा॑य॒ स्वाहा॑ श॒ताय॑ श॒ताय॒ स्वाहा॑ स॒हस्रा॑य । \newline
2. स्वाहा॑ स॒हस्रा॑य स॒हस्रा॑य॒ स्वाहा॒ स्वाहा॑ स॒हस्रा॑य॒ स्वाहा॒ स्वाहा॑ स॒हस्रा॑य॒ स्वाहा॒ स्वाहा॑ स॒हस्रा॑य॒ स्वाहा᳚ । \newline
3. स॒हस्रा॑य॒ स्वाहा॒ स्वाहा॑ स॒हस्रा॑य स॒हस्रा॑य॒ स्वाहा॒ ऽयुता॑या॒ युता॑य॒ स्वाहा॑ स॒हस्रा॑य स॒हस्रा॑य॒ स्वाहा॒ ऽयुता॑य । \newline
4. स्वाहा॒ ऽयुता॑या॒ युता॑य॒ स्वाहा॒ स्वाहा॒ ऽयुता॑य॒ स्वाहा॒ स्वाहा॒ ऽयुता॑य॒ स्वाहा॒ स्वाहा॒ ऽयुता॑य॒ स्वाहा᳚ । \newline
5. अ॒युता॑य॒ स्वाहा॒ स्वाहा॒ ऽयुता॑या॒ युता॑य॒ स्वाहा॑ नि॒युता॑य नि॒युता॑य॒ स्वाहा॒ ऽयुता॑या॒ युता॑य॒ स्वाहा॑ नि॒युता॑य । \newline
6. स्वाहा॑ नि॒युता॑य नि॒युता॑य॒ स्वाहा॒ स्वाहा॑ नि॒युता॑य॒ स्वाहा॒ स्वाहा॑ नि॒युता॑य॒ स्वाहा॒ स्वाहा॑ नि॒युता॑य॒ स्वाहा᳚ । \newline
7. नि॒युता॑य॒ स्वाहा॒ स्वाहा॑ नि॒युता॑य नि॒युता॑य॒ स्वाहा᳚ प्र॒युता॑य प्र॒युता॑य॒ स्वाहा॑ नि॒युता॑य नि॒युता॑य॒ स्वाहा᳚ प्र॒युता॑य । \newline
8. नि॒युता॒येति॑ नि - युता॑य । \newline
9. स्वाहा᳚ प्र॒युता॑य प्र॒युता॑य॒ स्वाहा॒ स्वाहा᳚ प्र॒युता॑य॒ स्वाहा॒ स्वाहा᳚ प्र॒युता॑य॒ स्वाहा॒ स्वाहा᳚ प्र॒युता॑य॒ स्वाहा᳚ । \newline
10. प्र॒युता॑य॒ स्वाहा॒ स्वाहा᳚ प्र॒युता॑य प्र॒युता॑य॒ स्वाहा ऽर्बु॑दा॒या र्बु॑दाय॒ स्वाहा᳚ प्र॒युता॑य प्र॒युता॑य॒ स्वाहा ऽर्बु॑दाय । \newline
11. प्र॒युता॒येति॑ प्र - युता॑य । \newline
12. स्वाहा ऽर्बु॑दा॒या र्बु॑दाय॒ स्वाहा॒ स्वाहा ऽर्बु॑दाय॒ स्वाहा॒ स्वाहा ऽर्बु॑दाय॒ स्वाहा॒ स्वाहा ऽर्बु॑दाय॒ स्वाहा᳚ । \newline
13. अर्बु॑दाय॒ स्वाहा॒ स्वाहा ऽर्बु॑दा॒या र्बु॑दाय॒ स्वाहा॒ न्य॑र्बुदाय॒ न्य॑र्बुदाय॒ स्वाहा ऽर्बु॑दा॒या र्बु॑दाय॒ स्वाहा॒ न्य॑र्बुदाय । \newline
14. स्वाहा॒ न्य॑र्बुदाय॒ न्य॑र्बुदाय॒ स्वाहा॒ स्वाहा॒ न्य॑र्बुदाय॒ स्वाहा॒ स्वाहा॒ न्य॑र्बुदाय॒ स्वाहा॒ स्वाहा॒ न्य॑र्बुदाय॒ स्वाहा᳚ । \newline
15. न्य॑र्बुदाय॒ स्वाहा॒ स्वाहा॒ न्य॑र्बुदाय॒ न्य॑र्बुदाय॒ स्वाहा॑ समु॒द्राय॑ समु॒द्राय॒ स्वाहा॒ न्य॑र्बुदाय॒ न्य॑र्बुदाय॒ स्वाहा॑ समु॒द्राय॑ । \newline
16. न्य॑र्बुदा॒येति॒ नि - अ॒र्बु॒दा॒य॒ । \newline
17. स्वाहा॑ समु॒द्राय॑ समु॒द्राय॒ स्वाहा॒ स्वाहा॑ समु॒द्राय॒ स्वाहा॒ स्वाहा॑ समु॒द्राय॒ स्वाहा॒ स्वाहा॑ समु॒द्राय॒ स्वाहा᳚ । \newline
18. स॒मु॒द्राय॒ स्वाहा॒ स्वाहा॑ समु॒द्राय॑ समु॒द्राय॒ स्वाहा॒ मद्ध्या॑य॒ मद्ध्या॑य॒ स्वाहा॑ समु॒द्राय॑ समु॒द्राय॒ स्वाहा॒ मद्ध्या॑य । \newline
19. स्वाहा॒ मद्ध्या॑य॒ मद्ध्या॑य॒ स्वाहा॒ स्वाहा॒ मद्ध्या॑य॒ स्वाहा॒ स्वाहा॒ मद्ध्या॑य॒ स्वाहा॒ स्वाहा॒ मद्ध्या॑य॒ स्वाहा᳚ । \newline
20. मद्ध्या॑य॒ स्वाहा॒ स्वाहा॒ मद्ध्या॑य॒ मद्ध्या॑य॒ स्वाहा ऽन्ता॒या न्ता॑य॒ स्वाहा॒ मद्ध्या॑य॒ मद्ध्या॑य॒ स्वाहा ऽन्ता॑य । \newline
21. स्वाहा ऽन्ता॒या न्ता॑य॒ स्वाहा॒ स्वाहा ऽन्ता॑य॒ स्वाहा॒ स्वाहा ऽन्ता॑य॒ स्वाहा॒ स्वाहा ऽन्ता॑य॒ स्वाहा᳚ । \newline
22. अन्ता॑य॒ स्वाहा॒ स्वाहा ऽन्ता॒या न्ता॑य॒ स्वाहा॑ परा॒र्द्धाय॑ परा॒र्द्धाय॒ स्वाहा ऽन्ता॒या न्ता॑य॒ स्वाहा॑ परा॒र्द्धाय॑ । \newline
23. स्वाहा॑ परा॒र्द्धाय॑ परा॒र्द्धाय॒ स्वाहा॒ स्वाहा॑ परा॒र्द्धाय॒ स्वाहा॒ स्वाहा॑ परा॒र्द्धाय॒ स्वाहा॒ स्वाहा॑ परा॒र्द्धाय॒ स्वाहा᳚ । \newline
24. प॒रा॒र्द्धाय॒ स्वाहा॒ स्वाहा॑ परा॒र्द्धाय॑ परा॒र्द्धाय॒ स्वाहो॒षस॑ उ॒षसे॒ स्वाहा॑ परा॒र्द्धाय॑ परा॒र्द्धाय॒ स्वाहो॒षसे᳚ । \newline
25. प॒रा॒र्द्धायेति॑ पर - अ॒र्द्धाय॑ । \newline
26. स्वाहो॒षस॑ उ॒षसे॒ स्वाहा॒ स्वाहो॒षसे॒ स्वाहा॒ स्वाहो॒षसे॒ स्वाहा॒ स्वाहो॒षसे॒ स्वाहा᳚ । \newline
27. उ॒षसे॒ स्वाहा॒ स्वाहो॒षस॑ उ॒षसे॒ स्वाहा॒ व्यु॑ष्ट्यै॒ व्यु॑ष्ट्यै॒ स्वाहो॒षस॑ उ॒षसे॒ स्वाहा॒ व्यु॑ष्ट्यै । \newline
28. स्वाहा॒ व्यु॑ष्ट्यै॒ व्यु॑ष्ट्यै॒ स्वाहा॒ स्वाहा॒ व्यु॑ष्ट्यै॒ स्वाहा॒ स्वाहा॒ व्यु॑ष्ट्यै॒ स्वाहा॒ स्वाहा॒ व्यु॑ष्ट्यै॒ स्वाहा᳚ । \newline
29. व्यु॑ष्ट्यै॒ स्वाहा॒ स्वाहा॒ व्यु॑ष्ट्यै॒ व्यु॑ष्ट्यै॒ स्वाहो॑देष्य॒त उ॑देष्य॒ते स्वाहा॒ व्यु॑ष्ट्यै॒ व्यु॑ष्ट्यै॒ स्वाहो॑देष्य॒ते । \newline
30. व्यु॑ष्ट्या॒ इति॒ वि - उ॒ष्ट्यै॒ । \newline
31. स्वाहो॑देष्य॒त उ॑देष्य॒ते स्वाहा॒ स्वाहो॑देष्य॒ते स्वाहा॒ स्वाहो॑देष्य॒ते स्वाहा॒ स्वाहो॑देष्य॒ते स्वाहा᳚ । \newline
32. उ॒दे॒ष्य॒ते स्वाहा॒ स्वाहो॑देष्य॒त उ॑देष्य॒ते स्वाहो᳚द्य॒त उ॑द्य॒ते स्वाहो॑देष्य॒त उ॑देष्य॒ते स्वाहो᳚द्य॒ते । \newline
33. उ॒दे॒ष्य॒त इत्यु॑त् - ए॒ष्य॒ते । \newline
34. स्वाहो᳚द्य॒त उ॑द्य॒ते स्वाहा॒ स्वाहो᳚द्य॒ते स्वाहा॒ स्वाहो᳚द्य॒ते स्वाहा॒ स्वाहो᳚द्य॒ते स्वाहा᳚ । \newline
35. उ॒द्य॒ते स्वाहा॒ स्वाहो᳚द्य॒त उ॑द्य॒ते स्वाहोदि॑ता॒ योदि॑ताय॒ स्वाहो᳚द्य॒त उ॑द्य॒ते स्वाहोदि॑ताय । \newline
36. उ॒द्य॒त इत्यु॑त् - य॒ते । \newline
37. स्वाहोदि॑ता॒ योदि॑ताय॒ स्वाहा॒ स्वाहोदि॑ताय॒ स्वाहा॒ स्वाहोदि॑ताय॒ स्वाहा॒ स्वाहोदि॑ताय॒ स्वाहा᳚ । \newline
38. उदि॑ताय॒ स्वाहा॒ स्वाहोदि॑ता॒ योदि॑ताय॒ स्वाहा॑ सुव॒र्गाय॑ सुव॒र्गाय॒ स्वाहोदि॑ता॒ योदि॑ताय॒ स्वाहा॑ सुव॒र्गाय॑ । \newline
39. उदि॑ता॒येत्युत् - इ॒ता॒य॒ । \newline
40. स्वाहा॑ सुव॒र्गाय॑ सुव॒र्गाय॒ स्वाहा॒ स्वाहा॑ सुव॒र्गाय॒ स्वाहा॒ स्वाहा॑ सुव॒र्गाय॒ स्वाहा॒ स्वाहा॑ सुव॒र्गाय॒ स्वाहा᳚ । \newline
41. सु॒व॒र्गाय॒ स्वाहा॒ स्वाहा॑ सुव॒र्गाय॑ सुव॒र्गाय॒ स्वाहा॑ लो॒काय॑ लो॒काय॒ स्वाहा॑ सुव॒र्गाय॑ सुव॒र्गाय॒ स्वाहा॑ लो॒काय॑ । \newline
42. सु॒व॒र्गायेति॑ सुवः - गाय॑ । \newline
43. स्वाहा॑ लो॒काय॑ लो॒काय॒ स्वाहा॒ स्वाहा॑ लो॒काय॒ स्वाहा॒ स्वाहा॑ लो॒काय॒ स्वाहा॒ स्वाहा॑ लो॒काय॒ स्वाहा᳚ । \newline
44. लो॒काय॒ स्वाहा॒ स्वाहा॑ लो॒काय॑ लो॒काय॒ स्वाहा॒ सर्व॑स्मै॒ सर्व॑स्मै॒ स्वाहा॑ लो॒काय॑ लो॒काय॒ स्वाहा॒ सर्व॑स्मै । \newline
45. स्वाहा॒ सर्व॑स्मै॒ सर्व॑स्मै॒ स्वाहा॒ स्वाहा॒ सर्व॑स्मै॒ स्वाहा॒ स्वाहा॒ सर्व॑स्मै॒ स्वाहा॒ स्वाहा॒ सर्व॑स्मै॒ स्वाहा᳚ । \newline
46. सर्व॑स्मै॒ स्वाहा॒ स्वाहा॒ सर्व॑स्मै॒ सर्व॑स्मै॒ स्वाहा᳚ । \newline
47. स्वाहेति॒ स्वाहा᳚ । \newline
\pagebreak


\end{document}