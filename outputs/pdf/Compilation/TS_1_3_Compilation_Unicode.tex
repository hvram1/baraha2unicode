\documentclass[17pt]{extarticle}
\usepackage{babel}
\usepackage{fontspec}
\usepackage{polyglossia}
\usepackage{extsizes}

\usepackage{color}   %May be necessary if you want to color links
\usepackage{hyperref}
\hypersetup{
    colorlinks=true, %set true if you want colored links
    linktoc=all,     %set to all if you want both sections and subsections linked
    linkcolor=black,  %choose some color if you want links to stand out
}

\setmainlanguage{sanskrit}
\setotherlanguages{english} %% or other languages
\setlength{\parindent}{0pt}
\pagestyle{myheadings}
\newfontfamily\devanagarifont[Script=Devanagari]{AdishilaVedic}
\renewcommand{\theHsection}{\thepart.section.\thesection}

\newcommand{\VAR}[1]{}
\newcommand{\BLOCK}[1]{}




\begin{document}
\begin{titlepage}
    \begin{center}
 
\begin{sanskrit}
    { \Large
    कृष्ण यजुर्वेदीय तैत्तिरीय संहिता,पद,जटा,घन पाठः 
    }
    \\
    \vspace{2.5cm}
    \mbox{ \Large
    1.3     प्रथमकाण्डे तृतीयः प्रश्नः -(अग्निष्टोमे पशुः)   }
\end{sanskrit}
\end{center}

\end{titlepage}
\tableofcontents
\phantomsection
\pagebreak

\markright{ TS 1.3.1.1  \hfill https://www.vedavms.in \hfill}

\section{ TS 1.3.1.1 }

\textbf{TS 1.3.1.1 } \newline
\textbf{Samhita Paata} \newline

दे॒वस्य॑ त्वा सवि॒तुः प्र॑स॒वे᳚ऽश्विनो᳚र् बा॒हुभ्यां᳚ पू॒ष्णो हस्ता᳚भ्या॒मा द॒देऽभ्रि॑रसि॒ नारि॑रसि॒ परि॑लिखितꣳ॒॒ रक्षः॒ परि॑लिखिता॒ अरा॑तय इ॒दम॒हꣳ रक्ष॑सो ग्री॒वा अपि॑ कृन्तामि॒ यो᳚ऽस्मान् द्वेष्टि॒ यं च॑ व॒यं द्वि॒ष्म इ॒दम॑स्य ग्री॒वा अपि॑ कृन्तामि दि॒वे त्वा॒ऽन्तरि॑क्षाय त्वा पृथि॒व्यै त्वा॒ शुन्ध॑तां ॅलो॒कः पि॑तृ॒षद॑नो॒ यवो॑ऽसि य॒वया॒स्मद् द्वेषो॑ - [ ] \newline

\textbf{Pada Paata} \newline

दे॒वस्य॑ । त्वा॒ । स॒वि॒तुः । प्र॒स॒व इति॑ प्र - स॒वे । अ॒श्विनोः᳚ । बा॒हुभ्या॒मिति॑ बा॒हु - भ्या॒म् । पू॒ष्णः । हस्ता᳚भ्याम् । एति॑ । द॒दे॒ । अभ्रिः॑ । अ॒सि॒ । नारिः॑ । अ॒सि॒ । परि॑लिखित॒मिति॒ परि॑ - लि॒खि॒त॒म् । रक्षः॑ । परि॑लिखिता॒ इति॒ परि॑ - लि॒खि॒ताः॒ । अरा॑तयः । इ॒दम् । अ॒हम् । रक्ष॑सः । ग्री॒वाः । अपीति॑ । कृ॒न्ता॒मि॒ । यः । अ॒स्मान् । द्वेष्टि॑ । यम् । च॒ । व॒यम् । द्वि॒ष्मः । इ॒दम् । अ॒स्य॒ । ग्री॒वाः । अपीति॑ । कृ॒न्ता॒मि॒ । दि॒वे । त्वा॒ । अ॒न्तरि॑क्षाय । त्वा॒ । पृ॒थि॒व्यै । त्वा॒ । शुन्ध॑ताम् । लो॒कः । पि॒तृ॒षद॑न॒ इति॑ पितृ - सद॑नः । यवः॑ । अ॒सि॒ । य॒वय॑ । अ॒स्मत् । द्वेषः॑ ।  \newline


\textbf{Krama Paata} \newline

दे॒वस्य॑ त्वा । त्वा॒ स॒वि॒तुः । स॒वि॒तुः प्र॑स॒वे । प्र॒स॒वे᳚ऽश्विनोः᳚ । प्र॒स॒व इति॑ प्र - स॒वे । अ॒श्विनो᳚र् बा॒हुभ्या᳚म् । बा॒हुभ्या᳚म् पू॒ष्णः । बा॒हुभ्या॒मिति॑ बा॒हु - भ्या॒म् । पू॒ष्णो हस्ता᳚भ्याम् । हस्ता᳚भ्या॒मा । 
आ द॑दे । द॒देऽभ्रिः॑ । अभ्रि॑रसि । अ॒सि॒ नारिः॑ । नारि॑रसि । अ॒सि॒ 
परि॑लिखितम् । परि॑लिखितꣳ॒॒ रक्षः॑ । परि॑लिखित॒मिति॒ परि॑ - लि॒खि॒त॒म् । रक्षः॒ परि॑लिखिताः । परि॑लिखिता॒ अरा॑तयः । परि॑लिखिता॒ इति॒ परि॑ - लि॒खि॒ताः॒ । अरा॑तय इ॒दम् । इ॒दम॒हम् । अ॒हꣳ रक्ष॑सः । 
रक्ष॑सो ग्री॒वाः । ग्री॒वा अपि॑ । अपि॑ कृन्तामि । कृ॒न्ता॒मि॒ यः । यो᳚ऽस्मान् । अ॒स्मान् द्वेष्टि॑ । द्वेष्टि॒ यम् । यम् च॑ । च॒ व॒यम् । व॒यम् द्वि॒ष्मः । द्वि॒ष्म इ॒दम् । इ॒दम॑स्य । अ॒स्य॒ ग्री॒वाः । ग्री॒वा अपि॑ । अपि॑ कृन्तामि । कृ॒न्ता॒मि॒ दि॒वे । दि॒वे त्वा᳚ । त्वा॒ऽन्तरि॑क्षाय । अ॒न्तरि॑क्षाय त्वा । त्वा॒ पृ॒थि॒व्यै । पृ॒थि॒व्यै त्वा᳚ । त्वा॒ शुन्ध॑ताम् । शुन्ध॑तां ॅलो॒कः । लो॒कः पि॑तृ॒षद॑नः । पि॒तृ॒षद॑नो॒ यवः॑ । पि॒तृ॒षद॑न॒ इति॑ पितृ - सद॑नः । यवो॑ऽसि । अ॒सि॒ य॒वय॑ । य॒वया॒स्मत् । अ॒स्मद् द्वेषः॑ । द्वेषो॑ य॒वय॑ \newline

\textbf{Jatai Paata} \newline

1. दे॒वस्य॑ त्वा त्वा दे॒वस्य॑ दे॒वस्य॑ त्वा । \newline
2. त्वा॒ स॒वि॒तुः स॑वि॒तु स्त्वा᳚ त्वा सवि॒तुः । \newline
3. स॒वि॒तुः प्र॑स॒वे प्र॑स॒वे स॑वि॒तुः स॑वि॒तुः प्र॑स॒वे । \newline
4. प्र॒स॒वे᳚ ऽश्विनो॑ र॒श्विनोः᳚ प्रस॒वे प्र॑स॒वे᳚ ऽश्विनोः᳚ । \newline
5. प्र॒स॒व इति॑ प्र - स॒वे । \newline
6. अ॒श्विनो᳚र् बा॒हुभ्या᳚म् बा॒हुभ्या॑ म॒श्विनो॑ र॒श्विनो᳚र् बा॒हुभ्या᳚म् । \newline
7. बा॒हुभ्या᳚म् पू॒ष्णः पू॒ष्णो बा॒हुभ्या᳚म् बा॒हुभ्या᳚म् पू॒ष्णः । \newline
8. बा॒हुभ्या॒मिति॑ बा॒हु - भ्या॒म् । \newline
9. पू॒ष्णो हस्ता᳚भ्या॒(ग्म्॒) हस्ता᳚भ्याम् पू॒ष्णः पू॒ष्णो हस्ता᳚भ्याम् । \newline
10. हस्ता᳚भ्या॒ मा हस्ता᳚भ्या॒(ग्म्॒) हस्ता᳚भ्या॒ मा । \newline
11. आ द॑दे दद॒ आ द॑दे । \newline
12. द॒दे ऽभ्रि॒ रभ्रि॑र् ददे द॒दे ऽभ्रिः॑ । \newline
13. अभ्रि॑ रस्य॒ स्यभ्रि॒ रभ्रि॑ रसि । \newline
14. अ॒सि॒ नारि॒र् नारि॑र स्यसि॒ नारिः॑ । \newline
15. नारि॑ रस्यसि॒ नारि॒र् नारि॑ रसि । \newline
16. अ॒सि॒ परि॑लिखित॒म् परि॑लिखित मस्यसि॒ परि॑लिखितम् । \newline
17. परि॑लिखित॒(ग्म्॒) रक्षो॒ रक्षः॒ परि॑लिखित॒म् परि॑लिखित॒(ग्म्॒) रक्षः॑ । \newline
18. परि॑लिखित॒मिति॒ परि॑ - लि॒खि॒त॒म् । \newline
19. रक्षः॒ परि॑लिखिताः॒ परि॑लिखिता॒ रक्षो॒ रक्षः॒ परि॑लिखिताः । \newline
20. परि॑लिखिता॒ अरा॑त॒यो ऽरा॑तयः॒ परि॑लिखिताः॒ परि॑लिखिता॒ अरा॑तयः । \newline
21. परि॑लिखिता॒ इति॒ परि॑ - लि॒खि॒ताः॒ । \newline
22. अरा॑तय इ॒द मि॒द मरा॑त॒यो ऽरा॑तय इ॒दम् । \newline
23. इ॒द म॒ह म॒ह मि॒द मि॒द म॒हम् । \newline
24. अ॒हꣳ रक्ष॑सो॒ रक्ष॑सो॒ ऽह म॒हꣳ रक्ष॑सः । \newline
25. रक्ष॑सो ग्री॒वा ग्री॒वा रक्ष॑सो॒ रक्ष॑सो ग्री॒वाः । \newline
26. ग्री॒वा अप्यपि॑ ग्री॒वा ग्री॒वा अपि॑ । \newline
27. अपि॑ कृन्तामि कृन्ता॒ म्यप्यपि॑ कृन्तामि । \newline
28. कृ॒न्ता॒मि॒ यो यः कृ॑न्तामि कृन्तामि॒ यः । \newline
29. यो᳚ ऽस्मा न॒स्मान्. यो यो᳚ ऽस्मान् । \newline
30. अ॒स्मान् द्वेष्टि॒ द्वेष्ट्य॒ स्मा न॒स्मान् द्वेष्टि॑ । \newline
31. द्वेष्टि॒ यं ॅयम् द्वेष्टि॒ द्वेष्टि॒ यम् । \newline
32. यम् च॑ च॒ यं ॅयम् च॑ । \newline
33. च॒ व॒यं ॅव॒यम् च॑ च व॒यम् । \newline
34. व॒यम् द्वि॒ष्मो द्वि॒ष्मो व॒यं ॅव॒यम् द्वि॒ष्मः । \newline
35. द्वि॒ष्म इ॒द मि॒दम् द्वि॒ष्मो द्वि॒ष्म इ॒दम् । \newline
36. इ॒द म॑स्यास्ये॒ द मि॒द म॑स्य । \newline
37. अ॒स्य॒ ग्री॒वा ग्री॒वा अ॑स्यास्य ग्री॒वाः । \newline
38. ग्री॒वा अप्यपि॑ ग्री॒वा ग्री॒वा अपि॑ । \newline
39. अपि॑ कृन्तामि कृन्ता॒ म्यप्यपि॑ कृन्तामि । \newline
40. कृ॒न्ता॒मि॒ दि॒वे दि॒वे कृ॑न्तामि कृन्तामि दि॒वे । \newline
41. दि॒वे त्वा᳚ त्वा दि॒वे दि॒वे त्वा᳚ । \newline
42. त्वा॒ ऽन्तरि॑क्षा या॒न्तरि॑क्षाय त्वा त्वा॒ ऽन्तरि॑क्षाय । \newline
43. अ॒न्तरि॑क्षाय त्वा त्वा॒ ऽन्तरि॑क्षा या॒न्तरि॑क्षाय त्वा । \newline
44. त्वा॒ पृ॒थि॒व्यै पृ॑थि॒व्यै त्वा᳚ त्वा पृथि॒व्यै । \newline
45. पृ॒थि॒व्यै त्वा᳚ त्वा पृथि॒व्यै पृ॑थि॒व्यै त्वा᳚ । \newline
46. त्वा॒ शुन्ध॑ता॒(ग्म्॒) शुन्ध॑ताम् त्वा त्वा॒ शुन्ध॑ताम् । \newline
47. शुन्ध॑ताम् ॅलो॒को लो॒कः शुन्ध॑ता॒(ग्म्॒) शुन्ध॑ताम् ॅलो॒कः । \newline
48. लो॒कः पि॑तृ॒षद॑नः पितृ॒षद॑नो लो॒को लो॒कः पि॑तृ॒षद॑नः । \newline
49. पि॒तृ॒षद॑नो॒ यवो॒ यवः॑ पितृ॒षद॑नः पितृ॒षद॑नो॒ यवः॑ । \newline
50. पि॒तृ॒षद॑न॒ इति॑ पितृ - सद॑नः । \newline
51. यवो᳚ ऽस्यसि॒ यवो॒ यवो॑ ऽसि । \newline
52. अ॒सि॒ य॒वय॑ य॒वया᳚ स्यसि य॒वय॑ । \newline
53. य॒वया॒स्म द॒स्मद् य॒वय॑ य॒व या॒स्मत् । \newline
54. अ॒स्मद् द्वेषो॒ द्वेषो॒ ऽस्म द॒स्मद् द्वेषः॑ । \newline
55. द्वेषो॑ य॒वय॑ य॒वय॒ द्वेषो॒ द्वेषो॑ य॒वय॑ । \newline

\textbf{Ghana Paata } \newline

1. दे॒वस्य॑ त्वा त्वा दे॒वस्य॑ दे॒वस्य॑ त्वा सवि॒तुः स॑वि॒तु स्त्वा॑ दे॒वस्य॑ दे॒वस्य॑ त्वा सवि॒तुः । \newline
2. त्वा॒ स॒वि॒तुः स॑वि॒तु स्त्वा᳚ त्वा सवि॒तुः प्र॑स॒वे प्र॑स॒वे स॑वि॒तु स्त्वा᳚ त्वा सवि॒तुः प्र॑स॒वे । \newline
3. स॒वि॒तुः प्र॑स॒वे प्र॑स॒वे स॑वि॒तुः स॑वि॒तुः प्र॑स॒वे᳚ ऽश्विनो॑ र॒श्विनोः᳚ प्रस॒वे स॑वि॒तुः स॑वि॒तुः प्र॑स॒वे᳚ ऽश्विनोः᳚ । \newline
4. प्र॒स॒वे᳚ ऽश्विनो॑ र॒श्विनोः᳚ प्रस॒वे प्र॑स॒वे᳚ ऽश्विनो᳚र् बा॒हुभ्या᳚म् बा॒हुभ्या॑ म॒श्विनोः᳚ प्रस॒वे प्र॑स॒वे᳚ ऽश्विनो᳚र् बा॒हुभ्या᳚म् । \newline
5. प्र॒स॒व इति॑ प्र - स॒वे । \newline
6. अ॒श्विनो᳚र् बा॒हुभ्या᳚म् बा॒हुभ्या॑ म॒श्विनो॑ र॒श्विनो᳚र् बा॒हुभ्या᳚म् पू॒ष्णः पू॒ष्णो बा॒हुभ्या॑ म॒श्विनो॑ र॒श्विनो᳚र् बा॒हुभ्या᳚म् पू॒ष्णः । \newline
7. बा॒हुभ्या᳚म् पू॒ष्णः पू॒ष्णो बा॒हुभ्या᳚म् बा॒हुभ्या᳚म् पू॒ष्णो हस्ता᳚भ्या॒(ग्म्॒) हस्ता᳚भ्याम् पू॒ष्णो बा॒हुभ्या᳚म् बा॒हुभ्या᳚म् पू॒ष्णो हस्ता᳚भ्याम् । \newline
8. बा॒हुभ्या॒मिति॑ बा॒हु - भ्या॒म् । \newline
9. पू॒ष्णो हस्ता᳚भ्या॒(ग्म्॒) हस्ता᳚भ्याम् पू॒ष्णः पू॒ष्णो हस्ता᳚भ्या॒ मा हस्ता᳚भ्याम् पू॒ष्णः पू॒ष्णो हस्ता᳚भ्या॒ मा । \newline
10. हस्ता᳚भ्या॒ मा हस्ता᳚भ्या॒(ग्म्॒) हस्ता᳚भ्या॒ मा द॑दे दद॒ आ हस्ता᳚भ्या॒(ग्म्॒) हस्ता᳚भ्या॒ मा द॑दे । \newline
11. आ द॑दे दद॒ आ द॒दे ऽभ्रि॒ रभ्रि॑र् दद॒ आ द॒दे ऽभ्रिः॑ । \newline
12. द॒दे ऽभ्रि॒ रभ्रि॑र् ददे द॒दे ऽभ्रि॑ रस्य॒स्यभ्रि॑र् ददे द॒दे ऽभ्रि॑रसि । \newline
13. अभ्रि॑ रस्य॒स्यभ्रि॒ रभ्रि॑रसि॒ नारि॒र् नारि॑ र॒स्यभ्रि॒ रभ्रि॑रसि॒ नारिः॑ । \newline
14. अ॒सि॒ नारि॒र् नारि॑ रस्यसि॒ नारि॑ रस्यसि॒ नारि॑ रस्यसि॒ नारि॑ रसि । \newline
15. नारि॑ रस्यसि॒ नारि॒र् नारि॑ रसि॒ परि॑लिखित॒म् परि॑लिखित मसि॒ नारि॒र् नारि॑रसि॒ परि॑लिखितम् । \newline
16. अ॒सि॒ परि॑लिखित॒म् परि॑लिखित मस्यसि॒ परि॑लिखित॒(ग्म्॒) रक्षो॒ रक्षः॒ परि॑लिखित मस्यसि॒ परि॑लिखित॒(ग्म्॒) रक्षः॑ । \newline
17. परि॑लिखित॒(ग्म्॒) रक्षो॒ रक्षः॒ परि॑लिखित॒म् परि॑लिखित॒(ग्म्॒) रक्षः॒ परि॑लिखिताः॒ परि॑लिखिता॒ रक्षः॒ परि॑लिखित॒म् परि॑लिखित॒(ग्म्॒) रक्षः॒ परि॑लिखिताः । \newline
18. परि॑लिखित॒मिति॒ परि॑ - लि॒खि॒त॒म् । \newline
19. रक्षः॒ परि॑लिखिताः॒ परि॑लिखिता॒ रक्षो॒ रक्षः॒ परि॑लिखिता॒ अरा॑त॒यो ऽरा॑तयः॒ परि॑लिखिता॒ रक्षो॒ रक्षः॒ परि॑लिखिता॒ अरा॑तयः । \newline
20. परि॑लिखिता॒ अरा॑त॒यो ऽरा॑तयः॒ परि॑लिखिताः॒ परि॑लिखिता॒ अरा॑तय इ॒द मि॒द मरा॑तयः॒ परि॑लिखिताः॒ परि॑लिखिता॒ अरा॑तय इ॒दम् । \newline
21. परि॑लिखिता॒ इति॒ परि॑ - लि॒खि॒ताः॒ । \newline
22. अरा॑तय इ॒द मि॒द मरा॑त॒यो ऽरा॑तय इ॒द म॒ह म॒ह मि॒द मरा॑त॒यो ऽरा॑तय इ॒द म॒हम् । \newline
23. इ॒द म॒ह म॒ह मि॒द मि॒द म॒हꣳ रक्ष॑सो॒ रक्ष॑सो॒ ऽह मि॒द मि॒द म॒हꣳ रक्ष॑सः । \newline
24. अ॒हꣳ रक्ष॑सो॒ रक्ष॑सो॒ ऽह म॒हꣳ रक्ष॑सो ग्री॒वा ग्री॒वा रक्ष॑सो॒ ऽह म॒हꣳ रक्ष॑सो ग्री॒वाः । \newline
25. रक्ष॑सो ग्री॒वा ग्री॒वा रक्ष॑सो॒ रक्ष॑सो ग्री॒वा अप्यपि॑ ग्री॒वा रक्ष॑सो॒ रक्ष॑सो ग्री॒वा अपि॑ । \newline
26. ग्री॒वा अप्यपि॑ ग्री॒वा ग्री॒वा अपि॑ कृन्तामि कृन्ता॒ म्यपि॑ ग्री॒वा ग्री॒वा अपि॑ कृन्तामि । \newline
27. अपि॑ कृन्तामि कृन्ता॒ म्यप्यपि॑ कृन्तामि॒ यो यः कृ॑न्ता॒ म्यप्यपि॑ कृन्तामि॒ यः । \newline
28. कृ॒न्ता॒मि॒ यो यः कृ॑न्तामि कृन्तामि॒ यो᳚ ऽस्मा न॒स्मान्. यः कृ॑न्तामि कृन्तामि॒ यो᳚ ऽस्मान् । \newline
29. यो᳚ ऽस्मा न॒स्मान्. यो यो᳚ ऽस्मान् द्वेष्टि॒ द्वेष्ट्य॒स्मान्. यो यो᳚ ऽस्मान् द्वेष्टि॑ । \newline
30. अ॒स्मान् द्वेष्टि॒ द्वेष्ट्य॒स्मा न॒स्मान् द्वेष्टि॒ यं ॅयम् द्वेष्ट्य॒स्मा न॒स्मान् द्वेष्टि॒ यम् । \newline
31. द्वेष्टि॒ यं ॅयम् द्वेष्टि॒ द्वेष्टि॒ यम् च॑ च॒ यम् द्वेष्टि॒ द्वेष्टि॒ यम् च॑ । \newline
32. यम् च॑ च॒ यं ॅयम् च॑ व॒यं ॅव॒यम् च॒ यं ॅयम् च॑ व॒यम् । \newline
33. च॒ व॒यं ॅव॒यम् च॑ च व॒यम् द्वि॒ष्मो द्वि॒ष्मो व॒यम् च॑ च व॒यम् द्वि॒ष्मः । \newline
34. व॒यम् द्वि॒ष्मो द्वि॒ष्मो व॒यं ॅव॒यम् द्वि॒ष्म इ॒द मि॒दम् द्वि॒ष्मो व॒यं ॅव॒यम् द्वि॒ष्म इ॒दम् । \newline
35. द्वि॒ष्म इ॒द मि॒दम् द्वि॒ष्मो द्वि॒ष्म इ॒द म॑स्यास्ये॒ दम् द्वि॒ष्मो द्वि॒ष्म इ॒द म॑स्य । \newline
36. इ॒द म॑स्यास्ये॒ द मि॒द म॑स्य ग्री॒वा ग्री॒वा अ॑स्ये॒ द मि॒द म॑स्य ग्री॒वाः । \newline
37. अ॒स्य॒ ग्री॒वा ग्री॒वा अ॑स्यास्य ग्री॒वा अप्यपि॑ ग्री॒वा अ॑स्यास्य ग्री॒वा अपि॑ । \newline
38. ग्री॒वा अप्यपि॑ ग्री॒वा ग्री॒वा अपि॑ कृन्तामि कृन्ता॒म्यपि॑ ग्री॒वा ग्री॒वा अपि॑ कृन्तामि । \newline
39. अपि॑ कृन्तामि कृन्ता॒ म्यप्यपि॑ कृन्तामि दि॒वे दि॒वे कृ॑न्ता॒ म्यप्यपि॑ कृन्तामि दि॒वे । \newline
40. कृ॒न्ता॒मि॒ दि॒वे दि॒वे कृ॑न्तामि कृन्तामि दि॒वे त्वा᳚ त्वा दि॒वे कृ॑न्तामि कृन्तामि दि॒वे त्वा᳚ । \newline
41. दि॒वे त्वा᳚ त्वा दि॒वे दि॒वे त्वा॒ ऽन्तरि॑क्षा या॒न्तरि॑क्षाय त्वा दि॒वे दि॒वे त्वा॒ ऽन्तरि॑क्षाय । \newline
42. त्वा॒ ऽन्तरि॑क्षा या॒न्तरि॑क्षाय त्वा त्वा॒ ऽन्तरि॑क्षाय त्वा त्वा॒ ऽन्तरि॑क्षाय त्वा त्वा॒ ऽन्तरि॑क्षाय त्वा । \newline
43. अ॒न्तरि॑क्षाय त्वा त्वा॒ ऽन्तरि॑क्षा या॒न्तरि॑क्षाय त्वा पृथि॒व्यै पृ॑थि॒व्यै त्वा॒ ऽन्तरि॑क्षा या॒न्तरि॑क्षाय त्वा पृथि॒व्यै । \newline
44. त्वा॒ पृ॒थि॒व्यै पृ॑थि॒व्यै त्वा᳚ त्वा पृथि॒व्यै त्वा᳚ त्वा पृथि॒व्यै त्वा᳚ त्वा पृथि॒व्यै त्वा᳚ । \newline
45. पृ॒थि॒व्यै त्वा᳚ त्वा पृथि॒व्यै पृ॑थि॒व्यै त्वा॒ शुन्ध॑ता॒(ग्म्॒) शुन्ध॑ताम् त्वा पृथि॒व्यै पृ॑थि॒व्यै त्वा॒ शुन्ध॑ताम् । \newline
46. त्वा॒ शुन्ध॑ता॒(ग्म्॒) शुन्ध॑ताम् त्वा त्वा॒ शुन्ध॑ताम् ॅलो॒को लो॒कः शुन्ध॑ताम् त्वा त्वा॒ शुन्ध॑ताम् ॅलो॒कः । \newline
47. शुन्ध॑ताम् ॅलो॒को लो॒कः शुन्ध॑ता॒(ग्म्॒) शुन्ध॑ताम् ॅलो॒कः पि॑तृ॒षद॑नः पितृ॒षद॑नो लो॒कः शुन्ध॑ता॒(ग्म्॒) शुन्ध॑ताम् ॅलो॒कः पि॑तृ॒षद॑नः । \newline
48. लो॒कः पि॑तृ॒षद॑नः पितृ॒षद॑नो लो॒को लो॒कः पि॑तृ॒षद॑नो॒ यवो॒ यवः॑ पितृ॒षद॑नो लो॒को लो॒कः पि॑तृ॒षद॑नो॒ यवः॑ । \newline
49. पि॒तृ॒षद॑नो॒ यवो॒ यवः॑ पितृ॒षद॑नः पितृ॒षद॑नो॒ यवो᳚ ऽस्यसि॒ यवः॑ पितृ॒षद॑नः पितृ॒षद॑नो॒ यवो॑ ऽसि । \newline
50. पि॒तृ॒षद॑न॒ इति॑ पितृ - सद॑नः । \newline
51. यवो᳚ ऽस्यसि॒ यवो॒ यवो॑ ऽसि य॒वय॑ य॒वया॑सि॒ यवो॒ यवो॑ ऽसि य॒वय॑ । \newline
52. अ॒सि॒ य॒वय॑ य॒वया᳚ स्यसि य॒वया॒ स्मद॒स्मद् य॒वया᳚ स्यसि य॒वया॒स्मत् । \newline
53. य॒वया॒स्म द॒स्मद् य॒वय॑ य॒वया॒स्मद् द्वेषो॒ द्वेषो॒ ऽस्मद् य॒वय॑ य॒वया॒स्मद् द्वेषः॑ । \newline
54. अ॒स्मद् द्वेषो॒ द्वेषो॒ ऽस्म द॒स्मद् द्वेषो॑ य॒वय॑ य॒वय॒ द्वेषो॒ ऽस्म द॒स्मद् द्वेषो॑ य॒वय॑ । \newline
55. द्वेषो॑ य॒वय॑ य॒वय॒ द्वेषो॒ द्वेषो॑ य॒वया रा॑ती॒ ररा॑तीर् य॒वय॒ द्वेषो॒ द्वेषो॑ य॒वया रा॑तीः । \newline
\pagebreak
\markright{ TS 1.3.1.2  \hfill https://www.vedavms.in \hfill}

\section{ TS 1.3.1.2 }

\textbf{TS 1.3.1.2 } \newline
\textbf{Samhita Paata} \newline

य॒वयारा॑तीः पितृ॒णाꣳ सद॑नम॒स्युद्दिवꣳ॑ स्तभा॒नाऽन्तरि॑क्षं पृण पृथि॒वीं दृꣳ॑ह द्युता॒नस्त्वा॑ मारु॒तो मि॑नोतु मि॒त्रावरु॑णयोर् ध्रु॒वेण॒ धर्म॑णा ब्रह्म॒वनिं॑ त्वा क्षत्र॒वनिꣳ॑ सुप्रजा॒वनिꣳ॑ रायस्पोष॒वनिं॒ पर्यू॑हामि॒ ब्रह्म॑ दृꣳह क्ष॒त्रं दृꣳ॑ह प्र॒जां दृꣳ॑ह रा॒यस्पोषं॑ दृꣳह घृ॒तेन॑ द्यावापृथिवी॒ आ पृ॑णेथा॒मिन्द्र॑स्य॒ सदो॑ऽसि विश्वज॒नस्य॑ छा॒या परि॑ त्वा गिर्वणो॒ गिर॑ इ॒मा ( ) भ॑वन्तु वि॒श्वतो॑ वृ॒द्धायु॒मनु॒ वृद्ध॑यो॒ जुष्टा॑ भवन्तु॒ जुष्ट॑य॒ इन्द्र॑स्य॒ स्यूर॒सीन्द्र॑स्य ध्रु॒वम॑स्यै॒न्द्रम॒सीन्द्रा॑य त्वा ॥ \newline

\textbf{Pada Paata} \newline

य॒वय॑ । अरा॑तीः । पि॒तृ॒णाम् । सद॑नम् । अ॒सि॒ । उदिति॑ । दिव᳚म् । स्त॒भा॒न॒ । एति॑ । अ॒न्तरि॑क्षम् । पृ॒ण॒ । पृ॒थि॒वीम् । दृꣳ॒॒ह॒ । द्यु॒ता॒नः । त्वा॒ । मा॒रु॒तः । मि॒नो॒तु॒ । मि॒॒त्रावरु॑णयो॒रिति॑ मि॒त्रा - वरु॑णयोः । ध्रु॒वेण॑ । धर्म॑णा । ब्र॒ह्म॒वनि॒मिति॑ ब्रह्म - वनि᳚म् । त्वा॒ । क्ष॒त्र॒वनि॒मिति॑ क्षत्र - वनि᳚म् । सु॒प्र॒जा॒वनि॒मिति॑ सुप्रजा - वनि᳚म् । रा॒य॒स्पो॒ष॒वनि॒मिति॑ रायस्योष - वनि᳚म् । परीति॑ । ऊ॒हा॒मि॒ । ब्रह्म॑ । दृꣳ॒॒ह॒ । क्ष॒त्रम् । दृꣳ॒॒ह॒ । प्र॒जामिति॑ प्र - जाम् । दृꣳ॒॒ह॒ । रा॒यः । पोष᳚म् । दृꣳ॒॒ह॒ । घृ॒तेन॑ । द्या॒वा॒ पृ॒थि॒वी॒ इति॑ द्यावा - पृ॒थि॒वी॒ । एति॑ । पृ॒णे॒था॒म् । इन्द्र॑स्य । सदः॑ । अ॒सि॒ । वि॒श्व॒ज॒नस्येति॑ विश्व - ज॒नस्य॑ । छा॒या । परीति॑ । त्वा॒ । गि॒र्व॒णः॒ । गिरः॑ । इ॒माः () । भ॒व॒न्तु॒ । वि॒श्वतः॑ । वृ॒द्धायु॒मिति॑ वृ॒द्ध - आ॒यु॒म् । अन्विति॑ । वृद्ध॑यः । जुष्टाः᳚ । भ॒व॒न्तु॒ । जुष्ट॑यः । इन्द्र॑स्य । स्यूः । अ॒सि॒ । इन्द्र॑स्य । ध्रु॒वम् । अ॒सि॒ । ऐ॒न्द्रम् । अ॒सि॒ । इन्द्रा॑य । त्वा॒ ॥  \newline


\textbf{Krama Paata} \newline

य॒वयारा॑तीः । अरा॑तीः पितृ॒णाम् । पि॒तृ॒णाꣳ सद॑नम् । सद॑नमसि । अ॒स्युत् । उद् दिव᳚म् । दिवꣳ॑ स्तभान । स्त॒भा॒ना । आऽन्तरि॑क्षम् । अ॒न्तरि॑क्षम् पृण । पृ॒ण॒ पृ॒थि॒वीम् । पृ॒थि॒वीम् दृꣳ॑ह । दृꣳ॒॒ह॒ द्यु॒ता॒नः । द्यु॒ता॒नस्त्वा᳚ । त्वा॒ मा॒रु॒तः । मा॒रु॒तो मि॑नोतु । मि॒नो॒तु॒ मि॒त्रावरु॑णयोः । मि॒त्रावरु॑णयोर् ध्रु॒वेण॑ । मि॒त्रावरु॑णयो॒रिति॑ मि॒त्रा - वरु॑णयोः । ध्रु॒वेण॒ धर्म॑णा । धर्म॑णा ब्रह्म॒वनि᳚म् । ब्र॒ह्म॒वनि॑म् त्वा । ब्र॒ह्म॒वनि॒मिति॑ ब्रह्म - वनि᳚म् । त्वा॒ क्ष॒त्र॒वनि᳚म् । क्ष॒त्र॒वनिꣳ॑ 
सुप्रजा॒वनि᳚म् । क्ष॒त्र॒वनि॒मिति॑ क्षत्र - वनि᳚म् । सु॒प्र॒जा॒वनिꣳ॑ रायस्पोष॒वनि᳚म् । सु॒प्र॒जा॒वनि॒मिति॑ सुप्रजा - वनि᳚म् । रा॒य॒स्पो॒ष॒वनि॒म् परि॑ । रा॒य॒स्पो॒ष॒वनि॒मिति॑ रायस्पोष - वनि᳚म् । पर्यू॑हामि । ऊ॒हा॒मि॒ ब्रह्म॑ । ब्रह्म॑ दृꣳह । दृꣳ॒॒ह॒ क्ष॒त्रम् । क्ष॒त्रम् दृꣳ॑ह । दृꣳ॒॒ह॒ प्र॒जाम् । प्र॒जाम् दृꣳ॑ह । प्र॒जामिति॑ प्र - जाम् । दृꣳ॒॒ह॒ रा॒यः । रा॒यस्पोष᳚म् । पोष॑म् दृꣳह । दृꣳ॒॒ह॒ घृ॒तेन॑ । घृ॒तेन॑ द्यावापृथिवी । द्या॒वा॒पृ॒थि॒वी॒ आ । द्या॒वा॒पृ॒थि॒वी॒ इति॑ द्यावा - पृ॒थि॒वी॒ । आ पृ॑णेथाम् । पृ॒णे॒था॒मिन्द्र॑स्य । इन्द्र॑स्य॒ सदः॑ । सदो॑ऽसि । अ॒सि॒ वि॒श्व॒ज॒नस्य॑ । वि॒श्व॒ज॒नस्य॑ छा॒या । वि॒श्व॒ज॒नस्येति॑ विश्व - ज॒नस्य॑ । छा॒या परि॑ । परि॑ त्वा । त्वा॒ गि॒र्व॒णः॒ । गि॒र्व॒णो॒ गिरः॑ । गिर॑ इ॒माः ( ) । इ॒मा भ॑वन्तु । भ॒व॒न्तु॒ वि॒श्वतः॑ । वि॒श्वतो॑ वृ॒द्धायु᳚म् । वृ॒द्धायु॒मनु॑ । वृ॒द्धायु॒मिति॑ वृ॒द्ध - आ॒यु॒म् । अनु॒ वृद्ध॑यः । वृद्ध॑यो॒ जुष्टाः᳚ । जुष्टा॑ भवन्तु । भ॒व॒न्तु॒ जुष्ट॑यः । जुष्ट॑य॒ इन्द्र॑स्य । इन्द्र॑स्य॒ स्यूः । स्यूर॑सि । अ॒सीन्द्र॑स्य । इन्द्र॑स्य ध्रु॒वम् । ध्रु॒वम॑सि । अ॒स्यै॒न्द्रम् । ऐ॒न्द्रम॑सि । अ॒सीन्द्रा॑य । इन्द्रा॑य त्वा । त्वेति॑ त्वा । \newline

\textbf{Jatai Paata} \newline

1. य॒वया रा॑ती॒र रा॑तीर् य॒वय॑ य॒व यारा॑तीः । \newline
2. अरा॑तीः पितृ॒णाम् पि॑तृ॒णा मरा॑ती॒ ररा॑तीः पितृ॒णाम् । \newline
3. पि॒तृ॒णाꣳ सद॑न॒(ग्म्॒) सद॑नम् पितृ॒णाम् पि॑तृ॒णाꣳ सद॑नम् । \newline
4. सद॑न मस्यसि॒ सद॑न॒(ग्म्॒) सद॑न मसि । \newline
5. अ॒स्यु दुद॑ स्य॒ स्युत् । \newline
6. उद् दिव॒म् दिव॒ मुदुद् दिव᳚म् । \newline
7. दिव(ग्ग्॑) स्तभान स्तभान॒ दिव॒म् दिव(ग्ग्॑) स्तभान । \newline
8. स्त॒भा॒ना स्त॑भान स्तभा॒ना । \newline
9. आ ऽन्तरि॑क्ष म॒न्तरि॑क्ष॒ मा ऽन्तरि॑क्षम् । \newline
10. अ॒न्तरि॑क्षम् पृण पृणा॒न्तरि॑क्ष म॒न्तरि॑क्षम् पृण । \newline
11. पृ॒ण॒ पृ॒थि॒वीम् पृ॑थि॒वीम् पृ॑ण पृण पृथि॒वीम् । \newline
12. पृ॒थि॒वीम् दृ(ग्म्॑)ह दृꣳह पृथि॒वीम् पृ॑थि॒वीम् दृ(ग्म्॑)ह । \newline
13. दृ॒(ग्म्॒)ह॒ द्यु॒ता॒नो द्यु॑ता॒नो दृ(ग्म्॑)ह दृꣳह द्युता॒नः । \newline
14. द्यु॒ता॒न स्त्वा᳚ त्वा द्युता॒नो द्यु॑ता॒न स्त्वा᳚ । \newline
15. त्वा॒ मा॒रु॒तो मा॑रु॒त स्त्वा᳚ त्वा मारु॒तः । \newline
16. मा॒रु॒तो मि॑नोतु मिनोतु मारु॒तो मा॑रु॒तो मि॑नोतु । \newline
17. मि॒नो॒तु॒ मि॒त्रावरु॑णयोर् मि॒त्रावरु॑णयोर् मिनोतु मिनोतु मि॒त्रावरु॑णयोः । \newline
18. मि॒त्रावरु॑णयोर् ध्रु॒वेण॑ ध्रु॒वेण॑ मि॒त्रावरु॑णयोर् मि॒त्रावरु॑णयोर् ध्रु॒वेण॑ । \newline
19. मि॒त्रावरु॑णयो॒रिति॑ मि॒त्रा - वरु॑णयोः । \newline
20. ध्रु॒वेण॒ धर्म॑णा॒ धर्म॑णा ध्रु॒वेण॑ ध्रु॒वेण॒ धर्म॑णा । \newline
21. धर्म॑णा ब्रह्म॒वनि॑म् ब्रह्म॒वनि॒म् धर्म॑णा॒ धर्म॑णा ब्रह्म॒वनि᳚म् । \newline
22. ब्र॒ह्म॒वनि॑म् त्वा त्वा ब्रह्म॒वनि॑म् ब्रह्म॒वनि॑म् त्वा । \newline
23. ब्र॒ह्म॒वनि॒मिति॑ ब्रह्म - वनि᳚म् । \newline
24. त्वा॒ क्ष॒त्र॒वनि॑म् क्षत्र॒वनि॑म् त्वा त्वा क्षत्र॒वनि᳚म् । \newline
25. क्ष॒त्र॒वनि(ग्म्॑) सुप्रजा॒वनि(ग्म्॑) सुप्रजा॒वनि॑म् क्षत्र॒वनि॑म् क्षत्र॒वनि(ग्म्॑) सुप्रजा॒वनि᳚म् । \newline
26. क्ष॒त्र॒वनि॒मिति॑ क्षत्र - वनि᳚म् । \newline
27. सु॒प्र॒जा॒वनि(ग्म्॑) रायस्पोष॒वनि(ग्म्॑) रायस्पोष॒वनि(ग्म्॑) सुप्रजा॒वनि(ग्म्॑) सुप्रजा॒वनि(ग्म्॑) रायस्पोष॒वनि᳚म् । \newline
28. सु॒प्र॒जा॒वनि॒मिति॑ सुप्रजा - वनि᳚म् । \newline
29. रा॒य॒स्पो॒ष॒वनि॒म् परि॒ परि॑ रायस्पोष॒वनि(ग्म्॑) रायस्पोष॒वनि॒म् परि॑ । \newline
30. रा॒य॒स्पो॒ष॒वनि॒मिति॑ रायस्पोष - वनि᳚म् । \newline
31. पर्यू॑हाम्यूहामि॒ परि॒ पर्यू॑हामि । \newline
32. ऊ॒हा॒मि॒ ब्रह्म॒ ब्रह्मो॑हा म्यूहामि॒ ब्रह्म॑ । \newline
33. ब्रह्म॑ दृꣳह दृꣳह॒ ब्रह्म॒ ब्रह्म॑ दृꣳह । \newline
34. दृ॒(ग्म्॒)ह॒ क्ष॒त्रम् क्ष॒त्रम् दृ(ग्म्॑)ह दृꣳह क्ष॒त्रम् । \newline
35. क्ष॒त्रम् दृ(ग्म्॑)ह दृꣳह क्ष॒त्रम् क्ष॒त्रम् दृ(ग्म्॑)ह । \newline
36. दृ॒(ग्म्॒)ह॒ प्र॒जाम् प्र॒जाम् दृ(ग्म्॑)ह दृꣳह प्र॒जाम् । \newline
37. प्र॒जाम् दृ(ग्म्॑)ह दृꣳह प्र॒जाम् प्र॒जाम् दृ(ग्म्॑)ह । \newline
38. प्र॒जामिति॑ प्र - जाम् । \newline
39. दृ॒(ग्म्॒)ह॒ रा॒यो रा॒यो दृ(ग्म्॑)ह दृꣳह रा॒यः । \newline
40. रा॒यस् पोष॒म् पोष(ग्म्॑) रा॒यो रा॒यस् पोष᳚म् । \newline
41. पोष॑म् दृꣳह दृꣳह॒ पोष॒म् पोष॑म् दृꣳह । \newline
42. दृ॒(ग्म्॒)ह॒ घृ॒तेन॑ घृ॒तेन॑ दृꣳह दृꣳह घृ॒तेन॑ । \newline
43. घृ॒तेन॑ द्यावापृथिवी द्यावापृथिवी घृ॒तेन॑ घृ॒तेन॑ द्यावापृथिवी । \newline
44. द्या॒वा॒पृ॒थि॒वी॒ आ द्या॑वापृथिवी द्यावापृथिवी॒ आ । \newline
45. द्या॒वा॒पृ॒थि॒वी॒ इति॑ द्यावा - पृ॒थि॒वी॒ । \newline
46. आ पृ॑णेथाम् पृणेथा॒ मा पृ॑णेथाम् । \newline
47. पृ॒णे॒था॒ मिन्द्र॒स्ये न्द्र॑स्य पृणेथाम् पृणेथा॒ मिन्द्र॑स्य । \newline
48. इन्द्र॑स्य॒ सदः॒ सद॒ इन्द्र॒स्ये न्द्र॑स्य॒ सदः॑ । \newline
49. सदो᳚ ऽस्यसि॒ सदः॒ सदो॑ ऽसि । \newline
50. अ॒सि॒ वि॒श्व॒ज॒नस्य॑ विश्वज॒न स्या᳚स्यसि विश्वज॒नस्य॑ । \newline
51. वि॒श्व॒ज॒नस्य॑ छा॒या छा॒या वि॑श्वज॒नस्य॑ विश्वज॒नस्य॑ छा॒या । \newline
52. वि॒श्व॒ज॒नस्येति॑ विश्व - ज॒नस्य॑ । \newline
53. छा॒या परि॒ परि॑च् छा॒या छा॒या परि॑ । \newline
54. परि॑ त्वा त्वा॒ परि॒ परि॑ त्वा । \newline
55. त्वा॒ गि॒र्व॒णो॒ गि॒र्व॒ण॒ स्त्वा॒ त्वा॒ गि॒र्व॒णः॒ । \newline
56. गि॒र्व॒णो॒ गिरो॒ गिरो॑ गिर्वणो गिर्वणो॒ गिरः॑ । \newline
57. गिर॑ इ॒मा इ॒मा गिरो॒ गिर॑ इ॒माः । \newline
58. इ॒मा भ॑वन्तु भव न्त्वि॒मा इ॒मा भ॑वन्तु । \newline
59. भ॒व॒न्तु॒ वि॒श्वतो॑ वि॒श्वतो॑ भवन्तु भवन्तु वि॒श्वतः॑ । \newline
60. वि॒श्वतो॑ वृ॒द्धायुं॑ ॅवृ॒द्धायुं॑ ॅवि॒श्वतो॑ वि॒श्वतो॑ वृ॒द्धायु᳚म् । \newline
61. वृ॒द्धायु॒ मन्वनु॑ वृ॒द्धायुं॑ ॅवृ॒द्धायु॒ मनु॑ । \newline
62. वृ॒द्धायु॒मिति॑ वृ॒द्ध - आ॒यु॒म् । \newline
63. अनु॒ वृद्ध॑यो॒ वृद्ध॒यो ऽन्वनु॒ वृद्ध॑यः । \newline
64. वृद्ध॑यो॒ जुष्टा॒ जुष्टा॒ वृद्ध॑यो॒ वृद्ध॑यो॒ जुष्टाः᳚ । \newline
65. जुष्टा॑ भवन्तु भवन्तु॒ जुष्टा॒ जुष्टा॑ भवन्तु । \newline
66. भ॒व॒न्तु॒ जुष्ट॑यो॒ जुष्ट॑यो भवन्तु भवन्तु॒ जुष्ट॑यः । \newline
67. जुष्ट॑य॒ इन्द्र॒स्ये न्द्र॑स्य॒ जुष्ट॑यो॒ जुष्ट॑य॒ इन्द्र॑स्य । \newline
68. इन्द्र॑स्य॒ स्यूः स्यू रिन्द्र॒स्ये न्द्र॑स्य॒ स्यूः । \newline
69. स्यू र॑स्यसि॒ स्यूः स्यू र॑सि । \newline
70. अ॒सीन्द्र॒स्ये न्द्र॑स्या स्य॒सीन्द्र॑स्य । \newline
71. इन्द्र॑स्य ध्रु॒वम् ध्रु॒व मिन्द्र॒स्ये न्द्र॑स्य ध्रु॒वम् । \newline
72. ध्रु॒व म॑स्यसि ध्रु॒वम् ध्रु॒व म॑सि । \newline
73. अ॒स्यै॒न्द्र मै॒न्द्र म॑स्य स्यै॒न्द्रम् । \newline
74. ऐ॒न्द्र म॑स्य स्यै॒न्द्र मै॒न्द्र म॑सि । \newline
75. अ॒सीन्द्रा॒ये न्द्रा॑ यास्य॒सी न्द्रा॑य । \newline
76. इन्द्रा॑य त्वा॒ त्वेन्द्रा॒ये न्द्रा॑य त्वा । \newline
77. त्वेति॑ त्वा । \newline

\textbf{Ghana Paata } \newline

1. य॒वयारा॑ती॒ ररा॑तीर् य॒वय॑ य॒वयारा॑तीः पितृ॒णाम् पि॑तृ॒णा मरा॑तीर् य॒वय॑ य॒वयारा॑तीः पितृ॒णाम् । \newline
2. अरा॑तीः पितृ॒णाम् पि॑तृ॒णा मरा॑ती॒ ररा॑तीः पितृ॒णाꣳ सद॑न॒(ग्म्॒) सद॑नम् पितृ॒णा मरा॑ती॒ ररा॑तीः पितृ॒णाꣳ सद॑नम् । \newline
3. पि॒तृ॒णाꣳ सद॑न॒(ग्म्॒) सद॑नम् पितृ॒णाम् पि॑तृ॒णाꣳ सद॑न मस्यसि॒ सद॑नम् पितृ॒णाम् पि॑तृ॒णाꣳ सद॑न मसि । \newline
4. सद॑न मस्यसि॒ सद॑न॒(ग्म्॒) सद॑न म॒स्युदुद॑सि॒ सद॑न॒(ग्म्॒) सद॑न म॒स्युत् । \newline
5. अ॒स्युदुद॑ स्य॒स्युद् दिव॒म् दिव॒ मुद॑स्य॒स्युद् दिव᳚म् । \newline
6. उद् दिव॒म् दिव॒ मुदुद् दिव(ग्ग्॑) स्तभान स्तभान॒ दिव॒ मुदुद् दिव(ग्ग्॑) स्तभान । \newline
7. दिव(ग्ग्॑) स्तभान स्तभान॒ दिव॒म् दिव(ग्ग्॑) स्तभा॒ना स्त॑भान॒ दिव॒म् दिव(ग्ग्॑) स्तभा॒ना । \newline
8. स्त॒भा॒ना स्त॑भान स्तभा॒ना ऽन्तरि॑क्ष म॒न्तरि॑क्ष॒ मा स्त॑भान स्तभा॒ना ऽन्तरि॑क्षम् । \newline
9. आ ऽन्तरि॑क्ष म॒न्तरि॑क्ष॒ मा ऽन्तरि॑क्षम् पृण पृणा॒न्तरि॑क्ष॒ मा ऽन्तरि॑क्षम् पृण । \newline
10. अ॒न्तरि॑क्षम् पृण पृणा॒न्तरि॑क्ष म॒न्तरि॑क्षम् पृण पृथि॒वीम् पृ॑थि॒वीम् पृ॑णा॒न्तरि॑क्ष म॒न्तरि॑क्षम् पृण पृथि॒वीम् । \newline
11. पृ॒ण॒ पृ॒थि॒वीम् पृ॑थि॒वीम् पृ॑ण पृण पृथि॒वीम् दृ(ग्म्॑)ह दृꣳह पृथि॒वीम् पृ॑ण पृण पृथि॒वीम् दृ(ग्म्॑)ह । \newline
12. पृ॒थि॒वीम् दृ(ग्म्॑)ह दृꣳह पृथि॒वीम् पृ॑थि॒वीम् दृ(ग्म्॑)ह द्युता॒नो द्यु॑ता॒नो दृ(ग्म्॑)ह पृथि॒वीम् पृ॑थि॒वीम् दृ(ग्म्॑)ह द्युता॒नः । \newline
13. दृ॒(ग्म्॒)ह॒ द्यु॒ता॒नो द्यु॑ता॒नो दृ(ग्म्॑)ह दृꣳह द्युता॒न स्त्वा᳚ त्वा द्युता॒नो दृ(ग्म्॑)ह दृꣳह द्युता॒न स्त्वा᳚ । \newline
14. द्यु॒ता॒न स्त्वा᳚ त्वा द्युता॒नो द्यु॑ता॒न स्त्वा॑ मारु॒तो मा॑रु॒तस्त्वा᳚ द्युता॒नो द्यु॑ता॒न स्त्वा॑ मारु॒तः । \newline
15. त्वा॒ मा॒रु॒तो मा॑रु॒त स्त्वा᳚ त्वा मारु॒तो मि॑नोतु मिनोतु मारु॒त स्त्वा᳚ त्वा मारु॒तो मि॑नोतु । \newline
16. मा॒रु॒तो मि॑नोतु मिनोतु मारु॒तो मा॑रु॒तो मि॑नोतु मि॒त्रावरु॑णयोर् मि॒त्रावरु॑णयोर् मिनोतु मारु॒तो मा॑रु॒तो मि॑नोतु मि॒त्रावरु॑णयोः । \newline
17. मि॒नो॒तु॒ मि॒त्रावरु॑णयोर् मि॒त्रावरु॑णयोर् मिनोतु मिनोतु मि॒त्रावरु॑णयोर् ध्रु॒वेण॑ ध्रु॒वेण॑ मि॒त्रावरु॑णयोर् मिनोतु मिनोतु मि॒त्रावरु॑णयोर् ध्रु॒वेण॑ । \newline
18. मि॒त्रावरु॑णयोर् ध्रु॒वेण॑ ध्रु॒वेण॑ मि॒त्रावरु॑णयोर् मि॒त्रावरु॑णयोर् ध्रु॒वेण॒ धर्म॑णा॒ धर्म॑णा ध्रु॒वेण॑ मि॒त्रावरु॑णयोर् मि॒त्रावरु॑णयोर् ध्रु॒वेण॒ धर्म॑णा । \newline
19. मि॒त्रावरु॑णयो॒रिति॑ मि॒त्रा - वरु॑णयोः । \newline
20. ध्रु॒वेण॒ धर्म॑णा॒ धर्म॑णा ध्रु॒वेण॑ ध्रु॒वेण॒ धर्म॑णा ब्रह्म॒वनि॑म् ब्रह्म॒वनि॒म् धर्म॑णा ध्रु॒वेण॑ ध्रु॒वेण॒ धर्म॑णा ब्रह्म॒वनि᳚म् । \newline
21. धर्म॑णा ब्रह्म॒वनि॑म् ब्रह्म॒वनि॒म् धर्म॑णा॒ धर्म॑णा ब्रह्म॒वनि॑म् त्वा त्वा ब्रह्म॒वनि॒म् धर्म॑णा॒ धर्म॑णा ब्रह्म॒वनि॑म् त्वा । \newline
22. ब्र॒ह्म॒वनि॑म् त्वा त्वा ब्रह्म॒वनि॑म् ब्रह्म॒वनि॑म् त्वा क्षत्र॒वनि॑म् क्षत्र॒वनि॑म् त्वा ब्रह्म॒वनि॑म् ब्रह्म॒वनि॑म् त्वा क्षत्र॒वनि᳚म् । \newline
23. ब्र॒ह्म॒वनि॒मिति॑ ब्रह्म - वनि᳚म् । \newline
24. त्वा॒ क्ष॒त्र॒वनि॑म् क्षत्र॒वनि॑म् त्वा त्वा क्षत्र॒वनि(ग्म्॑) सुप्रजा॒वनि(ग्म्॑) सुप्रजा॒वनि॑म् क्षत्र॒वनि॑म् त्वा त्वा क्षत्र॒वनि(ग्म्॑) सुप्रजा॒वनि᳚म् । \newline
25. क्ष॒त्र॒वनि(ग्म्॑) सुप्रजा॒वनि(ग्म्॑) सुप्रजा॒वनि॑म् क्षत्र॒वनि॑म् क्षत्र॒वनि(ग्म्॑) सुप्रजा॒वनि(ग्म्॑) रायस्पोष॒वनि(ग्म्॑) रायस्पोष॒वनि(ग्म्॑) सुप्रजा॒वनि॑म् क्षत्र॒वनि॑म् क्षत्र॒वनि(ग्म्॑) सुप्रजा॒वनि(ग्म्॑) रायस्पोष॒वनि᳚म् । \newline
26. क्ष॒त्र॒वनि॒मिति॑ क्षत्र - वनि᳚म् । \newline
27. सु॒प्र॒जा॒वनि(ग्म्॑) रायस्पोष॒वनि(ग्म्॑) रायस्पोष॒वनि(ग्म्॑) सुप्रजा॒वनि(ग्म्॑) सुप्रजा॒वनि(ग्म्॑) रायस्पोष॒वनि॒म् परि॒ परि॑ रायस्पोष॒वनि(ग्म्॑) सुप्रजा॒वनि(ग्म्॑) सुप्रजा॒वनि(ग्म्॑) रायस्पोष॒वनि॒म् परि॑ । \newline
28. सु॒प्र॒जा॒वनि॒मिति॑ सुप्रजा - वनि᳚म् । \newline
29. रा॒य॒स्पो॒ष॒वनि॒म् परि॒ परि॑ रायस्पोष॒वनि(ग्म्॑) रायस्पोष॒वनि॒म् पर्यू॑हा म्यूहामि॒ परि॑ रायस्पोष॒वनि(ग्म्॑) रायस्पोष॒वनि॒म् पर्यू॑हामि । \newline
30. रा॒य॒स्पो॒ष॒वनि॒मिति॑ रायस्पोष - वनि᳚म् । \newline
31. पर्यू॑हा म्यूहामि॒ परि॒ पर्यू॑हामि॒ ब्रह्म॒ ब्रह्मो॑हामि॒ परि॒ पर्यू॑हामि॒ ब्रह्म॑ । \newline
32. ऊ॒हा॒मि॒ ब्रह्म॒ ब्रह्मो॑हा म्यूहामि॒ ब्रह्म॑ दृꣳह दृꣳह॒ ब्रह्मो॑हा म्यूहामि॒ ब्रह्म॑ दृꣳह । \newline
33. ब्रह्म॑ दृꣳह दृꣳह॒ ब्रह्म॒ ब्रह्म॑ दृꣳह क्ष॒त्रम् क्ष॒त्रम् दृ(ग्म्॑)ह॒ ब्रह्म॒ ब्रह्म॑ दृꣳह क्ष॒त्रम् । \newline
34. दृ॒(ग्म्॒)ह॒ क्ष॒त्रम् क्ष॒त्रम् दृ(ग्म्॑)ह दृꣳह क्ष॒त्रम् दृ(ग्म्॑)ह दृꣳह क्ष॒त्रम् दृ(ग्म्॑)ह दृꣳह क्ष॒त्रम् दृ(ग्म्॑)ह । \newline
35. क्ष॒त्रम् दृ(ग्म्॑)ह दृꣳह क्ष॒त्रम् क्ष॒त्रम् दृ(ग्म्॑)ह प्र॒जाम् प्र॒जाम् दृ(ग्म्॑)ह क्ष॒त्रम् क्ष॒त्रम् दृ(ग्म्॑)ह प्र॒जाम् । \newline
36. दृ॒(ग्म्॒)ह॒ प्र॒जाम् प्र॒जाम् दृ(ग्म्॑)ह दृꣳह प्र॒जाम् दृ(ग्म्॑)ह दृꣳह प्र॒जाम् दृ(ग्म्॑)ह दृꣳह प्र॒जाम् दृ(ग्म्॑)ह । \newline
37. प्र॒जाम् दृ(ग्म्॑)ह दृꣳह प्र॒जाम् प्र॒जाम् दृ(ग्म्॑)ह रा॒यो रा॒यो दृ(ग्म्॑)ह प्र॒जाम् प्र॒जाम् दृ(ग्म्॑)ह रा॒यः । \newline
38. प्र॒जामिति॑ प्र - जाम् । \newline
39. दृ॒(ग्म्॒)ह॒ रा॒यो रा॒यो दृ(ग्म्॑)ह दृꣳह रा॒य स्पोष॒म् पोष(ग्म्॑) रा॒यो दृ(ग्म्॑)ह दृꣳह रा॒य स्पोष᳚म् । \newline
40. रा॒य स्पोष॒म् पोष(ग्म्॑) रा॒यो रा॒य स्पोष॑म् दृꣳह दृꣳह॒ पोष(ग्म्॑) रा॒यो रा॒य स्पोष॑म् दृꣳह । \newline
41. पोष॑म् दृꣳह दृꣳह॒ पोष॒म् पोष॑म् दृꣳह घृ॒तेन॑ घृ॒तेन॑ दृꣳह॒ पोष॒म् पोष॑म् दृꣳह घृ॒तेन॑ । \newline
42. दृ॒(ग्म्॒)ह॒ घृ॒तेन॑ घृ॒तेन॑ दृꣳह दृꣳह घृ॒तेन॑ द्यावापृथिवी द्यावापृथिवी घृ॒तेन॑ दृꣳह दृꣳह घृ॒तेन॑ द्यावापृथिवी । \newline
43. घृ॒तेन॑ द्यावापृथिवी द्यावापृथिवी घृ॒तेन॑ घृ॒तेन॑ द्यावापृथिवी॒ आ द्या॑वापृथिवी घृ॒तेन॑ घृ॒तेन॑ द्यावापृथिवी॒ आ । \newline
44. द्या॒वा॒पृ॒थि॒वी॒ आ द्या॑वापृथिवी द्यावापृथिवी॒ आ पृ॑णेथाम् पृणेथा॒ मा द्या॑वापृथिवी द्यावापृथिवी॒ आ पृ॑णेथाम् । \newline
45. द्या॒वा॒पृ॒थि॒वी॒ इति॑ द्यावा - पृ॒थि॒वी॒ । \newline
46. आ पृ॑णेथाम् पृणेथा॒ मा पृ॑णेथा॒ मिन्द्र॒स्ये न्द्र॑स्य पृणेथा॒ मा पृ॑णेथा॒ मिन्द्र॑स्य । \newline
47. पृ॒णे॒था॒ मिन्द्र॒स्ये न्द्र॑स्य पृणेथाम् पृणेथा॒ मिन्द्र॑स्य॒ सदः॒ सद॒ इन्द्र॑स्य पृणेथाम् पृणेथा॒ मिन्द्र॑स्य॒ सदः॑ । \newline
48. इन्द्र॑स्य॒ सदः॒ सद॒ इन्द्र॒स्ये न्द्र॑स्य॒ सदो᳚ ऽस्यसि॒ सद॒ इन्द्र॒स्ये न्द्र॑स्य॒ सदो॑ ऽसि । \newline
49. सदो᳚ ऽस्यसि॒ सदः॒ सदो॑ ऽसि विश्वज॒नस्य॑ विश्वज॒नस्या॑सि॒ सदः॒ सदो॑ ऽसि विश्वज॒नस्य॑ । \newline
50. अ॒सि॒ वि॒श्व॒ज॒नस्य॑ विश्वज॒नस्या᳚स्यसि विश्वज॒नस्य॑ छा॒या छा॒या वि॑श्वज॒नस्या᳚स्यसि विश्वज॒नस्य॑ छा॒या । \newline
51. वि॒श्व॒ज॒नस्य॑ छा॒या छा॒या वि॑श्वज॒नस्य॑ विश्वज॒नस्य॑ छा॒या परि॒ परि॑च् छा॒या वि॑श्वज॒नस्य॑ विश्वज॒नस्य॑ छा॒या परि॑ । \newline
52. वि॒श्व॒ज॒नस्येति॑ विश्व - ज॒नस्य॑ । \newline
53. छा॒या परि॒ परि॑च् छा॒या छा॒या परि॑ त्वा त्वा॒ परि॑च् छा॒या छा॒या परि॑ त्वा । \newline
54. परि॑ त्वा त्वा॒ परि॒ परि॑ त्वा गिर्वणो गिर्वण स्त्वा॒ परि॒ परि॑ त्वा गिर्वणः । \newline
55. त्वा॒ गि॒र्व॒णो॒ गि॒र्व॒ण॒ स्त्वा॒ त्वा॒ गि॒र्व॒णो॒ गिरो॒ गिरो॑ गिर्वण स्त्वा त्वा गिर्वणो॒ गिरः॑ । \newline
56. गि॒र्व॒णो॒ गिरो॒ गिरो॑ गिर्वणो गिर्वणो॒ गिर॑ इ॒मा इ॒मा गिरो॑ गिर्वणो गिर्वणो॒ गिर॑ इ॒माः । \newline
57. गिर॑ इ॒मा इ॒मा गिरो॒ गिर॑ इ॒मा भ॑वन्तु भवन्त्वि॒मा गिरो॒ गिर॑ इ॒मा भ॑वन्तु । \newline
58. इ॒मा भ॑वन्तु भवन्त्वि॒मा इ॒मा भ॑वन्तु वि॒श्वतो॑ वि॒श्वतो॑ भवन्त्वि॒मा इ॒मा भ॑वन्तु वि॒श्वतः॑ । \newline
59. भ॒व॒न्तु॒ वि॒श्वतो॑ वि॒श्वतो॑ भवन्तु भवन्तु वि॒श्वतो॑ वृ॒द्धायुं॑ ॅवृ॒द्धायुं॑ ॅवि॒श्वतो॑ भवन्तु भवन्तु वि॒श्वतो॑ वृ॒द्धायु᳚म् । \newline
60. वि॒श्वतो॑ वृ॒द्धायुं॑ ॅवृ॒द्धायुं॑ ॅवि॒श्वतो॑ वि॒श्वतो॑ वृ॒द्धायु॒ मन्वनु॑ वृ॒द्धायुं॑ ॅवि॒श्वतो॑ वि॒श्वतो॑ वृ॒द्धायु॒ मनु॑ । \newline
61. वृ॒द्धायु॒ मन्वनु॑ वृ॒द्धायुं॑ ॅवृ॒द्धायु॒ मनु॒ वृद्ध॑यो॒ वृद्ध॒यो ऽनु॑ वृ॒द्धायुं॑ ॅवृ॒द्धायु॒ मनु॒ वृद्ध॑यः । \newline
62. वृ॒द्धायु॒मिति॑ वृ॒द्ध - आ॒यु॒म् । \newline
63. अनु॒ वृद्ध॑यो॒ वृद्ध॒यो ऽन्वनु॒ वृद्ध॑यो॒ जुष्टा॒ जुष्टा॒ वृद्ध॒यो ऽन्वनु॒ वृद्ध॑यो॒ जुष्टाः᳚ । \newline
64. वृद्ध॑यो॒ जुष्टा॒ जुष्टा॒ वृद्ध॑यो॒ वृद्ध॑यो॒ जुष्टा॑ भवन्तु भवन्तु॒ जुष्टा॒ वृद्ध॑यो॒ वृद्ध॑यो॒ जुष्टा॑ भवन्तु । \newline
65. जुष्टा॑ भवन्तु भवन्तु॒ जुष्टा॒ जुष्टा॑ भवन्तु॒ जुष्ट॑यो॒ जुष्ट॑यो भवन्तु॒ जुष्टा॒ जुष्टा॑ भवन्तु॒ जुष्ट॑यः । \newline
66. भ॒व॒न्तु॒ जुष्ट॑यो॒ जुष्ट॑यो भवन्तु भवन्तु॒ जुष्ट॑य॒ इन्द्र॒स्ये न्द्र॑स्य॒ जुष्ट॑यो भवन्तु भवन्तु॒ जुष्ट॑य॒ इन्द्र॑स्य । \newline
67. जुष्ट॑य॒ इन्द्र॒स्ये न्द्र॑स्य॒ जुष्ट॑यो॒ जुष्ट॑य॒ इन्द्र॑स्य॒ स्यूः स्यू रिन्द्र॑स्य॒ जुष्ट॑यो॒ जुष्ट॑य॒ इन्द्र॑स्य॒ स्यूः । \newline
68. इन्द्र॑स्य॒ स्यूः स्यू रिन्द्र॒स्ये न्द्र॑स्य॒ स्यू र॑स्यसि॒ स्यू रिन्द्र॒स्ये न्द्र॑स्य॒ स्यू र॑सि । \newline
69. स्यू र॑स्यसि॒ स्यूः स्यू र॒सीन्द्र॒स्ये न्द्र॑स्यासि॒ स्यूः स्यू र॒सीन्द्र॑स्य । \newline
70. अ॒सीन्द्र॒स्ये न्द्र॑स्यास्य॒ सीन्द्र॑स्य ध्रु॒वम् ध्रु॒व मिन्द्र॑स्यास्य॒ सीन्द्र॑स्य ध्रु॒वम् । \newline
71. इन्द्र॑स्य ध्रु॒वम् ध्रु॒व मिन्द्र॒स्ये न्द्र॑स्य ध्रु॒व म॑स्यसि ध्रु॒व मिन्द्र॒स्ये न्द्र॑स्य ध्रु॒व म॑सि । \newline
72. ध्रु॒व म॑स्यसि ध्रु॒वम् ध्रु॒व म॑स्यै॒न्द्र मै॒न्द्र म॑सि ध्रु॒वम् ध्रु॒व म॑स्यै॒न्द्रम् । \newline
73. अ॒स्यै॒न्द्र मै॒न्द्र म॑स्यस्यै॒न्द्र म॑स्यस्यै॒न्द्र म॑स्यस्यै॒न्द्र म॑सि । \newline
74. ऐ॒न्द्र म॑स्यस्यै॒न्द्र मै॒न्द्र म॒सीन्द्रा॒ये न्द्रा॑यास्यै॒न्द्र मै॒न्द्र म॒सीन्द्रा॑य । \newline
75. अ॒सीन्द्रा॒ये न्द्रा॑या स्य॒सीन्द्रा॑य त्वा॒ त्वेन्द्रा॑या स्य॒सीन्द्रा॑य त्वा । \newline
76. इन्द्रा॑य त्वा॒ त्वेन्द्रा॒ये न्द्रा॑य त्वा । \newline
77. त्वेति॑ त्वा । \newline
\pagebreak
\markright{ TS 1.3.2.1  \hfill https://www.vedavms.in \hfill}

\section{ TS 1.3.2.1 }

\textbf{TS 1.3.2.1 } \newline
\textbf{Samhita Paata} \newline

र॒क्षो॒हणो॑ वलग॒हनो॑ वैष्ण॒वान् ख॑नामी॒दम॒हं तं ॅव॑ल॒गमुद्व॑पामि॒ यं नः॑ समा॒नो यमस॑मानो निच॒खाने॒दमे॑न॒मध॑रं करोमि॒ यो नः॑ समा॒नो योऽस॑मानोऽराती॒यति॑ गाय॒त्रेण॒ छन्द॒साऽव॑बाढो वल॒गः किमत्र॑ भ॒द्रं तन्नौ॑ स॒ह वि॒राड॑सि सपत्न॒हा स॒म्राड॑सि भ्रातृव्य॒हा स्व॒राड॑स्यभिमाति॒हा वि॑श्वा॒राड॑सि॒ विश्वा॑सां ना॒ष्ट्राणाꣳ॑ ह॒न्ता - [ ] \newline

\textbf{Pada Paata} \newline

र॒क्षो॒हण॒ इति॑ रक्षः - हनः॑ । व॒ल॒ग॒हन॒ इति॑ वलग - हनः॑ । वै॒ष्ण॒वान् । ख॒ना॒मि॒ । इ॒दम् । अ॒हम् । तम् । व॒ल॒गमिति॑ वल - गम् । उदिति॑ । व॒पा॒मि॒ । यम् । नः॒ । स॒मा॒नः । यम् । अस॑मानः । नि॒च॒खानेति॑ नि - च॒खान॑ । इ॒दम् । ए॒न॒म् । अद्ध॑रम् । क॒रो॒मि॒ । यः । नः॒ । स॒मा॒नः । यः । अस॑मानः । अ॒रा॒ती॒यति॑ । गा॒य॒त्रेण॑ । छन्द॑सा । अव॑बाढ॒ इत्यव॑ - बा॒ढः॒ । व॒ल॒ग इति॑ वल - गः । किम् । अत्र॑ । भ॒द्रम् । तत् । नौ॒ । स॒ह । वि॒राडिति॑ वि - राट् । अ॒सि॒ । स॒प॒त्न॒हेति॑ सपत्न - हा । स॒म्राडिति॑ सम् - राट् । अ॒सि॒ । भ्रा॒तृ॒व्य॒हेति॑ भ्रातृव्य - हा । स्व॒राडिति॑ स्व - राट् । अ॒सि॒ । अ॒भि॒मा॒ति॒हेत्य॑भिमाति - हा । वि॒श्वा॒राडिति॑ विश्व - राट् । अ॒सि॒ । विश्वा॑साम् । ना॒ष्ट्राणा᳚म् । ह॒न्ता ।  \newline


\textbf{Krama Paata} \newline

र॒क्षो॒हणो॑ वलग॒हनः॑ । र॒क्षो॒हण॒ इति॑ रक्षः - हनः॑ । व॒ल॒ग॒हनो॑ वैष्ण॒वान् । व॒ल॒ग॒हन॒ इति॑ वलग - हनः॑ । वै॒ष्ण॒वान् ख॑नामि । ख॒ना॒मी॒दम् । इ॒दम॒हम् । अ॒हम् तम् । तं ॅव॑ल॒गम् । व॒ल॒गमुत् । व॒ल॒गमिति॑ वल - गम् । उद् व॑पामि । व॒पा॒मि॒ यम् । यम् नः॑ । नः॒ स॒मा॒नः । स॒मा॒नो यम् । यमस॑मानः । अस॑मानो निच॒खान॑ । नि॒च॒खाने॒दम् । नि॒च॒खानेति॑ नि - च॒खान॑ । इ॒दमे॑नम् । ए॒न॒मध॑रम् । अध॑रम् करोमि । क॒रो॒मि॒ यः । यो नः॑ । नः॒ स॒मा॒नः । स॒मा॒नो यः । योऽस॑मानः । अस॑मानो ऽराती॒यति॑ । अ॒रा॒ती॒यति॑ गाय॒त्रेण॑ । गा॒य॒त्रेण॒ छन्द॑सा । छन्द॒सा ऽव॑बाढः । अव॑बाढो वल॒गः । अव॑बाढ॒ इत्यव॑ - बा॒ढः॒ । व॒ल॒गः किम् । व॒ल॒ग इति॑ वल - गः । किमत्र॑ । अत्र॑ भ॒द्रम् । भ॒द्रम् तत् । तन्नौ᳚ । नौ॒ स॒ह । स॒ह वि॒राट् । वि॒राड॑सि । वि॒राडिति॑ वि - राट् । अ॒सि॒ स॒प॒त्न॒हा । स॒प॒त्न॒हा स॒म्राट् । स॒प॒त्न॒हेति॑ सपत्न - हा । स॒म्राड॑सि । स॒म्राडिति॑ सम् - राट् । अ॒सि॒ भ्रा॒तृ॒व्य॒हा । भ्रा॒तृ॒व्य॒हा स्व॒राट् । भ्रा॒तृ॒व्य॒हेति॑ भ्रातृव्य - हा । स्व॒राड॑सि । स्व॒राडिति॑ स्व - राट् । अ॒स्य॒भि॒मा॒ति॒हा । अ॒भि॒मा॒ति॒हा वि॑श्वा॒राट् । 
अ॒भि॒मा॒ति॒हेत्य॑भिमाति - हा । वि॒श्वा॒राड॑सि । वि॒श्वा॒राडिति॑ विश्व - राट् । अ॒सि॒ विश्वा॑साम् । विश्वा॑साम् ना॒ष्ट्राणा᳚म् । ना॒ष्ट्राणाꣳ॑ ह॒न्ता । ह॒न्ता र॑क्षो॒हणः॑ \newline

\textbf{Jatai Paata} \newline

1. र॒क्षो॒हणो॑ वलग॒हनो॑ वलग॒हनो॑ रक्षो॒हणो॑ रक्षो॒हणो॑ वलग॒हनः॑ । \newline
2. र॒क्षो॒हण॒ इति॑ रक्षः - हनः॑ । \newline
3. व॒ल॒ग॒हनो॑ वैष्ण॒वान्. वै᳚ष्ण॒वान्. व॑लग॒हनो॑ वलग॒हनो॑ वैष्ण॒वान् । \newline
4. व॒ल॒ग॒हन॒ इति॑ वलग - हनः॑ । \newline
5. वै॒ष्ण॒वान् ख॑नामि खनामि वैष्ण॒वान्. वै᳚ष्ण॒वान् ख॑नामि । \newline
6. ख॒ना॒ मी॒द मि॒दम् ख॑नामि खना मी॒दम् । \newline
7. इ॒द म॒ह म॒ह मि॒द मि॒द म॒हम् । \newline
8. अ॒हम् तम् त म॒ह म॒हम् तम् । \newline
9. तं ॅव॑ल॒गं ॅव॑ल॒गम् तम् तं ॅव॑ल॒गम् । \newline
10. व॒ल॒ग मुदुद् व॑ल॒गं ॅव॑ल॒ग मुत् । \newline
11. व॒ल॒गमिति॑ वल - गम् । \newline
12. उद् व॑पामि वपा॒ म्युदुद् व॑पामि । \newline
13. व॒पा॒मि॒ यं ॅयं ॅव॑पामि वपामि॒ यम् । \newline
14. यम् नो॑ नो॒ यं ॅयम् नः॑ । \newline
15. नः॒ स॒मा॒नः स॑मा॒नो नो॑ नः समा॒नः । \newline
16. स॒मा॒नो यं ॅयꣳ स॑मा॒नः स॑मा॒नो यम् । \newline
17. य मस॑मा॒नो ऽस॑मानो॒ यं ॅय मस॑मानः । \newline
18. अस॑मानो निच॒खान॑ निच॒खा नास॑मा॒नो ऽस॑मानो निच॒खान॑ । \newline
19. नि॒च॒खाने॒ द मि॒दम् नि॑च॒खान॑ निच॒खाने॒ दम् । \newline
20. नि॒च॒खानेति॑ नि - च॒खान॑ । \newline
21. इ॒द मे॑न मेन मि॒द मि॒द मे॑नम् । \newline
22. ए॒न॒ मध॑र॒ मध॑र मेन मेन॒ मध॑रम् । \newline
23. अध॑रम् करोमि करो॒ म्यध॑र॒ मध॑रम् करोमि । \newline
24. क॒रो॒मि॒ यो यः क॑रोमि करोमि॒ यः । \newline
25. यो नो॑ नो॒ यो यो नः॑ । \newline
26. नः॒ स॒मा॒नः स॑मा॒नो नो॑ नः समा॒नः । \newline
27. स॒मा॒नो यो यः स॑मा॒नः स॑मा॒नो यः । \newline
28. यो ऽस॑मा॒नो ऽस॑मानो॒ यो यो ऽस॑मानः । \newline
29. अस॑मानो ऽराती॒यत्य॑ राती॒य त्यस॑मा॒नो ऽस॑मानो ऽराती॒यति॑ । \newline
30. अ॒रा॒ती॒यति॑ गाय॒त्रेण॑ गाय॒त्रेणा॑ राती॒यत्य॑ राती॒यति॑ गाय॒त्रेण॑ । \newline
31. गा॒य॒त्रेण॒ छन्द॑सा॒ छन्द॑सा गाय॒त्रेण॑ गाय॒त्रेण॒ छन्द॑सा । \newline
32. छन्द॒सा ऽव॑बा॒ढो ऽव॑बाढ॒ श्छन्द॑सा॒ छन्द॒सा ऽव॑बाढः । \newline
33. अव॑बाढो वल॒गो व॑ल॒गो ऽव॑बा॒ढो ऽव॑बाढो वल॒गः । \newline
34. अव॑बाढ॒ इत्यव॑ - बा॒ढः॒ । \newline
35. व॒ल॒गः किम् किं ॅव॑ल॒गो व॑ल॒गः किम् । \newline
36. व॒ल॒ग इति॑ वल - गः । \newline
37. कि मत्रात्र॒ किम् कि मत्र॑ । \newline
38. अत्र॑ भ॒द्रम् भ॒द्र मत्रात्र॑ भ॒द्रम् । \newline
39. भ॒द्रम् तत् तद् भ॒द्रम् भ॒द्रम् तत् । \newline
40. तन् नौ॑ नौ॒ तत् तन् नौ᳚ । \newline
41. नौ॒ स॒ह स॒ह नौ॑ नौ स॒ह । \newline
42. स॒ह वि॒राड् वि॒राट् थ् स॒ह स॒ह वि॒राट् । \newline
43. वि॒राड॑ स्यसि वि॒राड् वि॒रा ड॑सि । \newline
44. वि॒राडिति॑ वि - राट् । \newline
45. अ॒सि॒ स॒प॒त्न॒हा स॑पत्न॒हा ऽस्य॑सि सपत्न॒हा । \newline
46. स॒प॒त्न॒हा स॒म्राट् थ्स॒म्राट् थ्स॑पत्न॒हा स॑पत्न॒हा स॒म्राट् । \newline
47. स॒प॒त्न॒हेति॑ सपत्न - हा । \newline
48. स॒म्रा ड॑स्यसि स॒म्राट् थ् स॒म् राड॑सि । \newline
49. स॒म्राडिति॑ सम् - राट् । \newline
50. अ॒सि॒ भ्रा॒तृ॒व्य॒हा भ्रा॑तृव्य॒हा ऽस्य॑सि भ्रातृव्य॒हा । \newline
51. भ्रा॒तृ॒व्य॒हा स्व॒राट् थ्स्व॒राड् भ्रा॑तृव्य॒हा भ्रा॑तृव्य॒हा स्व॒राट् । \newline
52. भ्रा॒तृ॒व्य॒हेति॑ भ्रातृव्य - हा । \newline
53. स्व॒रा ड॑स्यसि स्व॒राट् थ्स्व॒रा ड॑सि । \newline
54. स्व॒राडिति॑ स्व - राट् । \newline
55. अ॒स्य॒ भि॒मा॒ति॒हा ऽभि॑माति॒हा ऽस्य॑स्य भिमाति॒हा । \newline
56. अ॒भि॒मा॒ति॒हा वि॑श्वा॒राड् वि॑श्वा॒रा ड॑भिमाति॒हा ऽभि॑माति॒हा वि॑श्वा॒राट् । \newline
57. अ॒भि॒मा॒ति॒हेत्य॑भिमाति - हा । \newline
58. वि॒श्वा॒ राड॑स्यसि विश्वा॒राड् वि॑श्वा॒ राड॑सि । \newline
59. वि॒श्वा॒राडिति॑ विश्व - राट् । \newline
60. अ॒सि॒ विश्वा॑सां॒ ॅविश्वा॑सा मस्यसि॒ विश्वा॑साम् । \newline
61. विश्वा॑साम् ना॒ष्ट्राणा᳚म् ना॒ष्ट्राणां॒ ॅविश्वा॑सां॒ ॅविश्वा॑साम् ना॒ष्ट्राणा᳚म् । \newline
62. ना॒ष्ट्राणा(ग्म्॑) ह॒न्ता ह॒न्ता ना॒ष्ट्राणा᳚म् ना॒ष्ट्राणा(ग्म्॑) ह॒न्ता । \newline
63. ह॒न्ता र॑क्षो॒हणो॑ रक्षो॒हणो॑ ह॒न्ता ह॒न्ता र॑क्षो॒हणः॑ । \newline

\textbf{Ghana Paata } \newline

1. र॒क्षो॒हणो॑ वलग॒हनो॑ वलग॒हनो॑ रक्षो॒हणो॑ रक्षो॒हणो॑ वलग॒हनो॑ वैष्ण॒वान्. वै᳚ष्ण॒वान्. व॑लग॒हनो॑ रक्षो॒हणो॑ रक्षो॒हणो॑ वलग॒हनो॑ वैष्ण॒वान् । \newline
2. र॒क्षो॒हण॒ इति॑ रक्षः - हनः॑ । \newline
3. व॒ल॒ग॒हनो॑ वैष्ण॒वान्. वै᳚ष्ण॒वान्. व॑लग॒हनो॑ वलग॒हनो॑ वैष्ण॒वान् ख॑नामि खनामि वैष्ण॒वान्. व॑लग॒हनो॑ वलग॒हनो॑ वैष्ण॒वान् ख॑नामि । \newline
4. व॒ल॒ग॒हन॒ इति॑ वलग - हनः॑ । \newline
5. वै॒ष्ण॒वान् ख॑नामि खनामि वैष्ण॒वान्. वै᳚ष्ण॒वान् ख॑नामी॒द मि॒दम् ख॑नामि वैष्ण॒वान्. वै᳚ष्ण॒वान् ख॑नामी॒दम् । \newline
6. ख॒ना॒मी॒द मि॒दम् ख॑नामि खनामी॒द म॒ह म॒ह मि॒दम् ख॑नामि खनामी॒द म॒हम् । \newline
7. इ॒द म॒ह म॒ह मि॒द मि॒द म॒हम् तम् त म॒ह मि॒द मि॒द म॒हम् तम् । \newline
8. अ॒हम् तम् त म॒ह म॒हम् तं ॅव॑ल॒गं ॅव॑ल॒गम् त म॒ह म॒हम् तं ॅव॑ल॒गम् । \newline
9. तं ॅव॑ल॒गं ॅव॑ल॒गम् तम् तं ॅव॑ल॒ग मुदुद् व॑ल॒गम् तम् तं ॅव॑ल॒ग मुत् । \newline
10. व॒ल॒ग मुदुद् व॑ल॒गं ॅव॑ल॒ग मुद् व॑पामि वपा॒म्युद् व॑ल॒गं ॅव॑ल॒ग मुद् व॑पामि । \newline
11. व॒ल॒गमिति॑ वल - गम् । \newline
12. उद् व॑पामि वपा॒म्युदुद् व॑पामि॒ यं ॅयं ॅव॑पा॒म्युदुद् व॑पामि॒ यम् । \newline
13. व॒पा॒मि॒ यं ॅयं ॅव॑पामि वपामि॒ यन्नो॑ नो॒ यं ॅव॑पामि वपामि॒ यन्नः॑ । \newline
14. यन्नो॑ नो॒ यं ॅयन्नः॑ समा॒नः स॑मा॒नो नो॒ यं ॅयन्नः॑ समा॒नः । \newline
15. नः॒ स॒मा॒नः स॑मा॒नो नो॑ नः समा॒नो यं ॅयꣳ स॑मा॒नो नो॑ नः समा॒नो यम् । \newline
16. स॒मा॒नो यं ॅयꣳ स॑मा॒नः स॑मा॒नो य मस॑मा॒नो ऽस॑मानो॒ यꣳ स॑मा॒नः स॑मा॒नो य मस॑मानः । \newline
17. य मस॑मा॒नो ऽस॑मानो॒ यं ॅय मस॑मानो निच॒खान॑ निच॒खानास॑मानो॒ यं ॅय मस॑मानो निच॒खान॑ । \newline
18. अस॑मानो निच॒खान॑ निच॒खानास॑मा॒नो ऽस॑मानो निच॒खाने॒ द मि॒दम् नि॑च॒खानास॑मा॒नो ऽस॑मानो निच॒खाने॒ दम् । \newline
19. नि॒च॒खाने॒ द मि॒दम् नि॑च॒खान॑ निच॒खाने॒ द मे॑न मेन मि॒दम् नि॑च॒खान॑ निच॒खाने॒ द मे॑नम् । \newline
20. नि॒च॒खानेति॑ नि - च॒खान॑ । \newline
21. इ॒द मे॑न मेन मि॒द मि॒द मे॑न॒ मध॑र॒ मध॑र मेन मि॒द मि॒द मे॑न॒ मध॑रम् । \newline
22. ए॒न॒ मध॑र॒ मध॑र मेन मेन॒ मध॑रम् करोमि करो॒ म्यध॑र मेन मेन॒ मध॑रम् करोमि । \newline
23. अध॑रम् करोमि करो॒ म्यध॑र॒ मध॑रम् करोमि॒ यो यः क॑रो॒ म्यध॑र॒ मध॑रम् करोमि॒ यः । \newline
24. क॒रो॒मि॒ यो यः क॑रोमि करोमि॒ यो नो॑ नो॒ यः क॑रोमि करोमि॒ यो नः॑ । \newline
25. यो नो॑ नो॒ यो यो नः॑ समा॒नः स॑मा॒नो नो॒ यो यो नः॑ समा॒नः । \newline
26. नः॒ स॒मा॒नः स॑मा॒नो नो॑ नः समा॒नो यो यः स॑मा॒नो नो॑ नः समा॒नो यः । \newline
27. स॒मा॒नो यो यः स॑मा॒नः स॑मा॒नो यो ऽस॑मा॒नो ऽस॑मानो॒ यः स॑मा॒नः स॑मा॒नो यो ऽस॑मानः । \newline
28. यो ऽस॑मा॒नो ऽस॑मानो॒ यो यो ऽस॑मानो ऽराती॒य त्य॑राती॒य त्यस॑मानो॒ यो यो ऽस॑मानो ऽराती॒यति॑ । \newline
29. अस॑मानो ऽराती॒य त्य॑राती॒य त्यस॑मा॒नो ऽस॑मानो ऽराती॒यति॑ गाय॒त्रेण॑ गाय॒त्रेणा॑ राती॒यत्यस॑मा॒नो ऽस॑मानो ऽराती॒यति॑ गाय॒त्रेण॑ । \newline
30. अ॒रा॒ती॒यति॑ गाय॒त्रेण॑ गाय॒त्रेणा॑ राती॒यत्य॑ राती॒यति॑ गाय॒त्रेण॒ छन्द॑सा॒ छन्द॑सा गाय॒त्रेणा॑ राती॒यत्य॑ राती॒यति॑ गाय॒त्रेण॒ छन्द॑सा । \newline
31. गा॒य॒त्रेण॒ छन्द॑सा॒ छन्द॑सा गाय॒त्रेण॑ गाय॒त्रेण॒ छन्द॒सा ऽव॑बा॒ढो ऽव॑बाढ॒ श्छन्द॑सा गाय॒त्रेण॑ गाय॒त्रेण॒ छन्द॒सा ऽव॑बाढः । \newline
32. छन्द॒सा ऽव॑बा॒ढो ऽव॑बाढ॒ श्छन्द॑सा॒ छन्द॒सा ऽव॑बाढो वल॒गो व॑ल॒गो ऽव॑बाढ॒ श्छन्द॑सा॒ छन्द॒सा ऽव॑बाढो वल॒गः । \newline
33. अव॑बाढो वल॒गो व॑ल॒गो ऽव॑बा॒ढो ऽव॑बाढो वल॒गः किम् किं ॅव॑ल॒गो ऽव॑बा॒ढो ऽव॑बाढो वल॒गः किम् । \newline
34. अव॑बाढ॒ इत्यव॑ - बा॒ढः॒ । \newline
35. व॒ल॒गः किम् किं ॅव॑ल॒गो व॑ल॒गः कि मत्रात्र॒ किं ॅव॑ल॒गो व॑ल॒गः कि मत्र॑ । \newline
36. व॒ल॒ग इति॑ वल - गः । \newline
37. कि मत्रात्र॒ किम् कि मत्र॑ भ॒द्रम् भ॒द्र मत्र॒ किम् कि मत्र॑ भ॒द्रम् । \newline
38. अत्र॑ भ॒द्रम् भ॒द्र मत्रात्र॑ भ॒द्रम् तत् तद् भ॒द्र मत्रात्र॑ भ॒द्रम् तत् । \newline
39. भ॒द्रम् तत् तद् भ॒द्रम् भ॒द्रम् तन् नौ॑ नौ॒ तद् भ॒द्रम् भ॒द्रम् तन् नौ᳚ । \newline
40. तन् नौ॑ नौ॒ तत् तन् नौ॑ स॒ह स॒ह नौ॒ तत् तन् नौ॑ स॒ह । \newline
41. नौ॒ स॒ह स॒ह नौ॑ नौ स॒ह वि॒राड् वि॒राट् थ्स॒ह नौ॑ नौ स॒ह वि॒राट् । \newline
42. स॒ह वि॒राड् वि॒राट् थ्स॒ह स॒ह वि॒रा ड॑स्यसि वि॒राट् थ्स॒ह स॒ह वि॒राड॑सि । \newline
43. वि॒रा ड॑स्यसि वि॒राड् वि॒राड॑सि सपत्न॒हा स॑पत्न॒हा ऽसि॑ वि॒राड् वि॒राड॑सि सपत्न॒हा । \newline
44. वि॒राडिति॑ वि - राट् । \newline
45. अ॒सि॒ स॒प॒त्न॒हा स॑पत्न॒हा ऽस्य॑सि सपत्न॒हा स॒म्राट्थ् स॒म्राट्थ् स॑पत्न॒हा ऽस्य॑सि सपत्न॒हा स॒म्राट् । \newline
46. स॒प॒त्न॒हा स॒म्राट्थ् स॒म्राट्थ् स॑पत्न॒हा स॑पत्न॒हा स॒म्राड॑स्यसि स॒म्राट्थ् स॑पत्न॒हा स॑पत्न॒हा स॒म्राड॑सि । \newline
47. स॒प॒त्न॒हेति॑ सपत्न - हा । \newline
48. स॒म्राड॑स्यसि स॒म्राट्थ् स॒म्राड॑सि भ्रातृव्य॒हा भ्रा॑तृव्य॒हा ऽसि॑ स॒म्राट्थ् स॒म्राड॑सि भ्रातृव्य॒हा । \newline
49. स॒म्राडिति॑ सम् - राट् । \newline
50. अ॒सि॒ भ्रा॒तृ॒व्य॒हा भ्रा॑तृव्य॒हा ऽस्य॑सि भ्रातृव्य॒हा स्व॒राट्थ् स्व॒राड् भ्रा॑तृव्य॒हा ऽस्य॑सि भ्रातृव्य॒हा स्व॒राट् । \newline
51. भ्रा॒तृ॒व्य॒हा स्व॒राट्थ् स्व॒राड् भ्रा॑तृव्य॒हा भ्रा॑तृव्य॒हा स्व॒राड॑स्यसि स्व॒राड् भ्रा॑तृव्य॒हा भ्रा॑तृव्य॒हा स्व॒राड॑सि । \newline
52. भ्रा॒तृ॒व्य॒हेति॑ भ्रातृव्य - हा । \newline
53. स्व॒राड॑स्यसि स्व॒राट्थ् स्व॒राड॑ स्यभिमाति॒हा ऽभि॑माति॒हा ऽसि॑ स्व॒राट्थ् स्व॒राड॑ स्यभिमाति॒हा । \newline
54. स्व॒राडिति॑ स्व - राट् । \newline
55. अ॒स्य॒भि॒मा॒ति॒हा ऽभि॑माति॒हा ऽस्य॑ स्यभिमाति॒हा वि॑श्वा॒राड् वि॑श्वा॒रा ड॑भिमाति॒हा ऽस्य॑ स्यभिमाति॒हा वि॑श्वा॒राट् । \newline
56. अ॒भि॒मा॒ति॒हा वि॑श्वा॒राड् वि॑श्वा॒रा ड॑भिमाति॒हा ऽभि॑माति॒हा वि॑श्वा॒राड॑स्यसि विश्वा॒रा ड॑भिमाति॒हा ऽभि॑माति॒हा वि॑श्वा॒राड॑सि । \newline
57. अ॒भि॒मा॒ति॒हेत्य॑भिमाति - हा । \newline
58. वि॒श्वा॒रा ड॑स्यसि विश्वा॒राड् वि॑श्वा॒राड॑सि॒ विश्वा॑सां॒ ॅविश्वा॑सा मसि विश्वा॒राड् वि॑श्वा॒राड॑सि॒ विश्वा॑साम् । \newline
59. वि॒श्वा॒राडिति॑ विश्व - राट् । \newline
60. अ॒सि॒ विश्वा॑सां॒ ॅविश्वा॑सा मस्यसि॒ विश्वा॑साम् ना॒ष्ट्राणा᳚म् ना॒ष्ट्राणां॒ ॅविश्वा॑सा मस्यसि॒ विश्वा॑साम् ना॒ष्ट्राणा᳚म् । \newline
61. विश्वा॑साम् ना॒ष्ट्राणा᳚म् ना॒ष्ट्राणां॒ ॅविश्वा॑सां॒ ॅविश्वा॑साम् ना॒ष्ट्राणा(ग्म्॑) ह॒न्ता ह॒न्ता ना॒ष्ट्राणां॒ ॅविश्वा॑सां॒ ॅविश्वा॑साम् ना॒ष्ट्राणा(ग्म्॑) ह॒न्ता । \newline
62. ना॒ष्ट्राणा(ग्म्॑) ह॒न्ता ह॒न्ता ना॒ष्ट्राणा᳚म् ना॒ष्ट्राणा(ग्म्॑) ह॒न्ता र॑क्षो॒हणो॑ रक्षो॒हणो॑ ह॒न्ता ना॒ष्ट्राणा᳚न्ना॒ष्ट्राणा(ग्म्॑) ह॒न्ता र॑क्षो॒हणः॑ । \newline
63. ह॒न्ता र॑क्षो॒हणो॑ रक्षो॒हणो॑ ह॒न्ता ह॒न्ता र॑क्षो॒हणो॑ वलग॒हनो॑ वलग॒हनो॑ रक्षो॒हणो॑ ह॒न्ता ह॒न्ता र॑क्षो॒हणो॑ वलग॒हनः॑ । \newline
\pagebreak
\markright{ TS 1.3.2.2  \hfill https://www.vedavms.in \hfill}

\section{ TS 1.3.2.2 }

\textbf{TS 1.3.2.2 } \newline
\textbf{Samhita Paata} \newline

र॑क्षो॒हणो॑ वलग॒हनः॒ प्रोक्षा॑मि वैष्ण॒वान् र॑क्षो॒हणो॑ वलग॒हनोऽव॑ नयामि वैष्ण॒वान् यवो॑ऽसि य॒वया॒स्मद् द्वेषो॑ य॒वयारा॑ती रक्षो॒हणो॑ वलग॒हनोऽव॑ स्तृणामि वैष्ण॒वान् र॑क्षो॒हणो॑ वलग॒हनो॒ऽभि जु॑होमि वैष्ण॒वान् र॑क्षो॒हणौ॑ वलग॒हना॒वुप॑ दधामि वैष्ण॒वी र॑क्षो॒हणौ॑ वलग॒हनौ॒ पर्यू॑हामि वैष्ण॒वी र॑क्षो॒हणौ॑ वलग॒हनौ॒ परि॑ स्तृणामि वैष्ण॒वी र॑क्षो॒हणौ॑ वलग॒हनौ॑ वैष्ण॒वी बृ॒हन्न॑सि बृ॒हद्ग्रा॑वा बृह॒तीमिन्द्रा॑य॒ ( ) वाचं॑ ॅवद ॥ \newline

\textbf{Pada Paata} \newline

र॒क्षो॒हण॒ इति॑ रक्षः - हनः॑ । व॒ल॒ग॒हन॒ इति॑ वलग - हनः॑ । प्रेति॑ । उ॒क्षा॒मि॒ । वै॒ष्ण॒वान् । र॒क्षो॒हण॒ इति॑ रक्षः - हनः॑ । व॒ल॒ग॒हन॒ इति॑ वलग - हनः॑ । अवेति॑ । न॒या॒मि॒ । वै॒ष्ण॒वान् । यवः॑ । अ॒सि॒ । य॒वय॑ । अ॒स्मत् । द्वेषः॑ । य॒वय॑ । अरा॑तीः । र॒क्षो॒हण॒ इति॑ रक्षः - हनः॑ । व॒ल॒ग॒हन॒ इति॑ वलग - हनः॑ । अवेति॑ । स्तृ॒णा॒मि॒ । वै॒ष्ण॒वान् । र॒क्षो॒हण॒ इति॑ रक्षः - हनः॑ । व॒ल॒ग॒हन॒ इति॑ वलग - हनः॑ । अ॒भीति॑ । जु॒हो॒मि॒ । वै॒ष्ण॒वान् । र॒क्षो॒हणा॒विति॑ रक्षः - हनौ᳚ । व॒ल॒ग॒हना॒विति॑ वलग - हनौ᳚ । उपेति॑ । द॒धा॒मि॒ । वै॒ष्ण॒वी इति॑ । र॒क्षो॒हणा॒विति॑ रक्षः - हनौ᳚ । व॒ल॒ग॒हना॒विति॑ वलग - हनौ᳚ । परीति॑ । ऊ॒हा॒मि॒ । वै॒ष्ण॒वी इति॑ । र॒क्षो॒हणा॒विति॑ रक्षः - हनौ᳚ । व॒ल॒ग॒हना॒विति॑ वलग - हनौ᳚ । परीति॑ । स्तृ॒णा॒मि॒ । वै॒ष्ण॒वी इति॑ । र॒क्षो॒हणा॒विति॑ रक्षः - हनौ᳚ । व॒ल॒ग॒हना॒विति॑ वलग - हनौ᳚ । वै॒ष्ण॒वी इति॑ । बृ॒हन्न् । अ॒सि॒ । बृ॒हद्ग्रा॒वेति॑ बृ॒हत् - ग्रा॒वा॒ । बृ॒ह॒तीम् । इन्द्रा॑य ( ) । वाच᳚म् । व॒द॒ ॥  \newline


\textbf{Krama Paata} \newline

र॒क्षो॒हणो॑ वलग॒हनः॑ । र॒क्षो॒हण॒ इति॑ रक्षः - हनः॑ । व॒ल॒ग॒हनः॒ प्र । व॒ल॒ग॒हन॒ इति॑ वलग - हनः॑ । प्रोक्षा॑मि । उ॒क्षा॒मि॒ वै॒ष्ण॒वान् । वै॒ष्ण॒वान् र॑क्षो॒हणः॑ । र॒क्षो॒हणो॑ वलग॒हनः॑ । र॒क्षो॒हण॒ इति॑ रक्षः - हनः॑ । व॒ल॒ग॒हनोऽव॑ । व॒ल॒ग॒हन॒ इति॑ वलग - हनः॑ । अव॑ नयामि । न॒या॒मि॒ वै॒ष्ण॒वान् । वै॒ष्ण॒वान्. यवः॑ । यवो॑ऽसि । अ॒सि॒ य॒वय॑ । य॒वया॒स्मत् । अ॒स्मद् द्वेषः॑ । द्वेषो॑ य॒वय॑ । य॒वयारा॑तीः । अरा॑ती रक्षो॒हणः॑ । र॒क्षो॒हणो॑ वलग॒हनः॑ । र॒क्षो॒हण॒ इति॑ रक्षः - हनः॑ । व॒ल॒ग॒हनोऽव॑ । व॒ल॒ग॒हन॒ इति॑ वलग - हनः॑ । अव॑ स्तृणामि । स्तृ॒णा॒मि॒ वै॒ष्ण॒वान् । वै॒ष्ण॒वान् र॑क्षो॒हणः॑ । र॒क्षो॒हणो॑ वलग॒हनः॑ । र॒क्षो॒हण॒ इति॑ रक्षः - हनः॑ । व॒ल॒ग॒हनो॒ऽभि । व॒ल॒ग॒हन॒ इति॑ वलग - हनः॑ । अ॒भि जु॑होमि । जु॒हो॒मि॒ वै॒ष्ण॒वान् । वै॒ष्ण॒वान् र॑क्षो॒हणौ᳚ । र॒क्षो॒हणौ॑ वलग॒हनौ᳚ । र॒क्षो॒हणा॒विति॑ रक्षः - हनौ᳚ । व॒ल॒ग॒हना॒वुप॑ । व॒ल॒ग॒हना॒विति॑ वलग - हनौ᳚ । उप॑ दधामि । द॒धा॒मि॒ वै॒ष्ण॒वी । वै॒ष्ण॒वी र॑क्षो॒हणौ᳚ । वै॒ष्ण॒वी इति॑ वैष्ण॒वी । र॒क्षो॒हणौ॑ वलग॒हनौ᳚ । र॒क्षो॒हणा॒विति॑ रक्षः - हनौ᳚ । व॒ल॒ग॒हनौ॒ परि॑ । व॒ल॒ग॒हना॒विति॑ वलग - हनौ᳚ । पर्यू॑हामि । ऊ॒हा॒मि॒ वै॒ष्ण॒वी । वै॒ष्ण॒वी र॑क्षो॒हणौ᳚ । वै॒ष्ण॒वी इति॑ वैष्ण॒वी । र॒क्षो॒हणौ॑ वलग॒हनौ᳚ । र॒क्षो॒हणा॒विति॑ रक्षः - हनौ᳚ । व॒ल॒ग॒हनौ॒ परि॑ । व॒ल॒ग॒हना॒विति॑ वलग - हनौ᳚ । परि॑ स्तृणामि । स्तृ॒णा॒मि॒ वै॒ष्ण॒वी । वै॒ष्ण॒वी र॑क्षो॒हणौ᳚ । वै॒ष्ण॒वी इति॑ वैष्ण॒वी । र॒क्षो॒हणौ॑ वलग॒हनौ᳚ । र॒क्षो॒हणा॒विति॑ रक्षः - हनौ᳚ । व॒ल॒ग॒हनौ॑ वैष्ण॒वी । व॒ल॒ग॒हना॒विति॑ वलग - हनौ᳚ । वै॒ष्ण॒वी बृ॒हन्न् । वै॒ष्ण॒वी इति॑ वैष्ण॒वी । बृ॒हन्न॑सि । अ॒सि॒ बृ॒हद्ग्रा॑वा । बृ॒हद्ग्रा॑वा बृह॒तीम् । बृ॒हद्ग्रा॒वेति॑ बृ॒हत् - ग्रा॒वा॒ । बृ॒ह॒तीमिन्द्रा॑य ( ) । इन्द्रा॑य॒ वाच᳚म् । वाचं॑ ॅवद । व॒देति॑ वद । \newline

\textbf{Jatai Paata} \newline

1. र॒क्षो॒हणो॑ वलग॒हनो॑ वलग॒हनो॑ रक्षो॒हणो॑ रक्षो॒हणो॑ वलग॒हनः॑ । \newline
2. र॒क्षो॒हण॒ इति॑ रक्षः - हनः॑ । \newline
3. व॒ल॒ग॒हनः॒ प्र प्र व॑लग॒हनो॑ वलग॒हनः॒ प्र । \newline
4. व॒ल॒ग॒हन॒ इति॑ वलग - हनः॑ । \newline
5. प्रोक्षा᳚ म्युक्षामि॒ प्र प्रोक्षा॑मि । \newline
6. उ॒क्षा॒मि॒ वै॒ष्ण॒वान्. वै᳚ष्ण॒वा नु॑क्षा म्युक्षामि वैष्ण॒वान् । \newline
7. वै॒ष्ण॒वान् र॑क्षो॒हणो॑ रक्षो॒हणो॑ वैष्ण॒वान्. वै᳚ष्ण॒वान् र॑क्षो॒हणः॑ । \newline
8. र॒क्षो॒हणो॑ वलग॒हनो॑ वलग॒हनो॑ रक्षो॒हणो॑ रक्षो॒हणो॑ वलग॒हनः॑ । \newline
9. र॒क्षो॒हण॒ इति॑ रक्षः - हनः॑ । \newline
10. व॒ल॒ग॒हनो ऽवाव॑ वलग॒हनो॑ वलग॒हनो ऽव॑ । \newline
11. व॒ल॒ग॒हन॒ इति॑ वलग - हनः॑ । \newline
12. अव॑ नयामि नया॒ म्यवाव॑ नयामि । \newline
13. न॒या॒मि॒ वै॒ष्ण॒वान्. वै᳚ष्ण॒वान् न॑यामि नयामि वैष्ण॒वान् । \newline
14. वै॒ष्ण॒वान्. यवो॒ यवो॑ वैष्ण॒वान्. वै᳚ष्ण॒वान्. यवः॑ । \newline
15. यवो᳚ ऽस्यसि॒ यवो॒ यवो॑ ऽसि । \newline
16. अ॒सि॒ य॒वय॑ य॒वया᳚ स्यसि य॒वय॑ । \newline
17. य॒व या॒स्म द॒स्मद् य॒वय॑ य॒व या॒स्मत् । \newline
18. अ॒स्मद् द्वेषो॒ द्वेषो॒ ऽस्म द॒स्मद् द्वेषः॑ । \newline
19. द्वेषो॑ य॒वय॑ य॒वय॒ द्वेषो॒ द्वेषो॑ य॒वय॑ । \newline
20. य॒व यारा॑ती॒ ररा॑तीर् य॒वय॑ य॒व यारा॑तीः । \newline
21. अरा॑ती रक्षो॒हणो॑ रक्षो॒हणो ऽरा॑ती॒ ररा॑ती रक्षो॒हणः॑ । \newline
22. र॒क्षो॒हणो॑ वलग॒हनो॑ वलग॒हनो॑ रक्षो॒हणो॑ रक्षो॒हणो॑ वलग॒हनः॑ । \newline
23. र॒क्षो॒हण॒ इति॑ रक्षः - हनः॑ । \newline
24. व॒ल॒ग॒हनो ऽवाव॑ वलग॒हनो॑ वलग॒हनो ऽव॑ । \newline
25. व॒ल॒ग॒हन॒ इति॑ वलग - हनः॑ । \newline
26. अव॑ स्तृणामि स्तृणा॒ म्यवाव॑ स्तृणामि । \newline
27. स्तृ॒णा॒मि॒ वै॒ष्ण॒वान्. वै᳚ष्ण॒वान् थ्स्तृ॑णामि स्तृणामि वैष्ण॒वान् । \newline
28. वै॒ष्ण॒वान् र॑क्षो॒हणो॑ रक्षो॒हणो॑ वैष्ण॒वान्. वै᳚ष्ण॒वान् र॑क्षो॒हणः॑ । \newline
29. र॒क्षो॒हणो॑ वलग॒हनो॑ वलग॒हनो॑ रक्षो॒हणो॑ रक्षो॒हणो॑ वलग॒हनः॑ । \newline
30. र॒क्षो॒हण॒ इति॑ रक्षः - हनः॑ । \newline
31. व॒ल॒ग॒हनो॒ ऽभ्य॑भि व॑लग॒हनो॑ वलग॒हनो॒ ऽभि । \newline
32. व॒ल॒ग॒हन॒ इति॑ वलग - हनः॑ । \newline
33. अ॒भि जु॑होमि जुहो म्य॒भ्य॑भि जु॑होमि । \newline
34. जु॒हो॒मि॒ वै॒ष्ण॒वान्. वै᳚ष्ण॒वान् जु॑होमि जुहोमि वैष्ण॒वान् । \newline
35. वै॒ष्ण॒वान् र॑क्षो॒हणौ॑ रक्षो॒हणौ॑ वैष्ण॒वान्. वै᳚ष्ण॒वान् र॑क्षो॒हणौ᳚ । \newline
36. र॒क्षो॒हणौ॑ वलग॒हनौ॑ वलग॒हनौ॑ रक्षो॒हणौ॑ रक्षो॒हणौ॑ वलग॒हनौ᳚ । \newline
37. र॒क्षो॒हणा॒विति॑ रक्षः - हनौ᳚ । \newline
38. व॒ल॒ग॒हना॒ वुपोप॑ वलग॒हनु॑ वलग॒हना॒ वुपोप॑ । \newline
39. व॒ल॒ग॒हना॒विति॑ वलग - हनौ᳚ । \newline
40. उप॑ दधामि दधा॒ म्युपोप॑ दधामि । \newline
41. द॒धा॒मि॒ वै॒ष्ण॒वी वै᳚ष्ण॒वी द॑धामि दधामि वैष्ण॒वी । \newline
42. वै॒ष्ण॒वी र॑क्षो॒हणौ॑ रक्षो॒हणौ॑ वैष्ण॒वी वै᳚ष्ण॒वी र॑क्षो॒हणौ᳚ । \newline
43. वै॒ष्ण॒वी इति॑ वैष्ण॒वी । \newline
44. र॒क्षो॒हणौ॑ वलग॒हनौ॑ वलग॒हनौ॑ रक्षो॒हणौ॑ रक्षो॒हणौ॑ वलग॒हनौ᳚ । \newline
45. र॒क्षो॒हणा॒विति॑ रक्षः - हनौ᳚ । \newline
46. व॒ल॒ग॒हनौ॒ परि॒ परि॑ वलग॒हनौ॑ वलग॒हनौ॒ परि॑ । \newline
47. व॒ल॒ग॒हना॒विति॑ वलग - हनौ᳚ । \newline
48. पर्यू॑हा म्यूहामि॒ परि॒ पर्यू॑हामि । \newline
49. ऊ॒हा॒मि॒ वै॒ष्ण॒वी वै᳚ष्ण॒वी ऊ॑हा म्यूहामि वैष्ण॒वी । \newline
50. वै॒ष्ण॒वी र॑क्षो॒हणौ॑ रक्षो॒हणौ॑ वैष्ण॒वी वै᳚ष्ण॒वी र॑क्षो॒हणौ᳚ । \newline
51. वै॒ष्ण॒वी इति॑ वैष्ण॒वी । \newline
52. र॒क्षो॒हणौ॑ वलग॒हनौ॑ वलग॒हनौ॑ रक्षो॒हणौ॑ रक्षो॒हणौ॑ वलग॒हनौ᳚ । \newline
53. र॒क्षो॒हणा॒विति॑ रक्षः - हनौ᳚ । \newline
54. व॒ल॒ग॒हनौ॒ परि॒ परि॑ वलग॒हनौ॑ वलग॒हनौ॒ परि॑ । \newline
55. व॒ल॒ग॒हना॒विति॑ वलग - हनौ᳚ । \newline
56. परि॑ स्तृणामि स्तृणामि॒ परि॒ परि॑ स्तृणामि । \newline
57. स्तृ॒णा॒मि॒ वै॒ष्ण॒वी वै᳚ष्ण॒वी स्तृ॑णामि स्तृणामि वैष्ण॒वी । \newline
58. वै॒ष्ण॒वी र॑क्षो॒हणौ॑ रक्षो॒हणौ॑ वैष्ण॒वी वै᳚ष्ण॒वी र॑क्षो॒हणौ᳚ । \newline
59. वै॒ष्ण॒वी इति॑ वैष्ण॒वी । \newline
60. र॒क्षो॒हणौ॑ वलग॒हनौ॑ वलग॒हनौ॑ रक्षो॒हणौ॑ रक्षो॒हणौ॑ वलग॒हनौ᳚ । \newline
61. र॒क्षो॒हणा॒विति॑ रक्षः - हनौ᳚ । \newline
62. व॒ल॒ग॒हनौ॑ वैष्ण॒वी वै᳚ष्ण॒वी व॑लग॒हनौ॑ वलग॒हनौ॑ वैष्ण॒वी । \newline
63. व॒ल॒ग॒हना॒विति॑ वलग - हनौ᳚ । \newline
64. वै॒ष्ण॒वी बृ॒हन् बृ॒हन्. वै᳚ष्ण॒वी वै᳚ष्ण॒वी बृ॒हन्न् । \newline
65. वै॒ष्ण॒वी इति॑ वैष्ण॒वी । \newline
66. बृ॒हन् न॑स्यसि बृ॒हन् बृ॒हन् न॑सि । \newline
67. अ॒सि॒ बृ॒हद्ग्रा॑वा बृ॒हद्ग्रा॑वा ऽस्यसि बृ॒हद्ग्रा॑वा । \newline
68. बृ॒हद्ग्रा॑वा बृह॒तीम् बृ॑ह॒तीम् बृ॒हद्ग्रा॑वा बृ॒हद्ग्रा॑वा बृह॒तीम् । \newline
69. बृ॒हद्ग्रा॒वेति॑ बृ॒हत् - ग्रा॒वा॒ । \newline
70. बृ॒ह॒ती मिन्द्रा॒ये न्द्रा॑य बृह॒तीम् बृ॑ह॒ती मिन्द्रा॑य । \newline
71. इन्द्रा॑य॒ वाचं॒ ॅवाच॒ मिन्द्रा॒ये न्द्रा॑य॒ वाच᳚म् । \newline
72. वाचं॑ ॅवद वद॒ वाचं॒ ॅवाचं॑ ॅवद । \newline
73. व॒देति॑ वद । \newline

\textbf{Ghana Paata } \newline

1. र॒क्षो॒हणो॑ वलग॒हनो॑ वलग॒हनो॑ रक्षो॒हणो॑ रक्षो॒हणो॑ वलग॒हनः॒ प्र प्र व॑लग॒हनो॑ रक्षो॒हणो॑ रक्षो॒हणो॑ वलग॒हनः॒ प्र । \newline
2. र॒क्षो॒हण॒ इति॑ रक्षः - हनः॑ । \newline
3. व॒ल॒ग॒हनः॒ प्र प्र व॑लग॒हनो॑ वलग॒हनः॒ प्रोक्षा᳚ म्युक्षामि॒ प्र व॑लग॒हनो॑ वलग॒हनः॒ प्रोक्षा॑मि । \newline
4. व॒ल॒ग॒हन॒ इति॑ वलग - हनः॑ । \newline
5. प्रोक्षा᳚म्यु क्षामि॒ प्र प्रोक्षा॑मि वैष्ण॒वान्. वै᳚ष्ण॒वा नु॑क्षामि॒ प्र प्रोक्षा॑मि वैष्ण॒वान् । \newline
6. उ॒क्षा॒मि॒ वै॒ष्ण॒वान्. वै᳚ष्ण॒वा नु॑क्षाम्युक्षामि वैष्ण॒वान् र॑क्षो॒हणो॑ रक्षो॒हणो॑ वैष्ण॒वा नु॑क्षाम्युक्षामि वैष्ण॒वान् र॑क्षो॒हणः॑ । \newline
7. वै॒ष्ण॒वान् र॑क्षो॒हणो॑ रक्षो॒हणो॑ वैष्ण॒वान्. वै᳚ष्ण॒वान् र॑क्षो॒हणो॑ वलग॒हनो॑ वलग॒हनो॑ रक्षो॒हणो॑ वैष्ण॒वान्. वै᳚ष्ण॒वान् र॑क्षो॒हणो॑ वलग॒हनः॑ । \newline
8. र॒क्षो॒हणो॑ वलग॒हनो॑ वलग॒हनो॑ रक्षो॒हणो॑ रक्षो॒हणो॑ वलग॒हनो ऽवाव॑ वलग॒हनो॑ रक्षो॒हणो॑ रक्षो॒हणो॑ वलग॒हनो ऽव॑ । \newline
9. र॒क्षो॒हण॒ इति॑ रक्षः - हनः॑ । \newline
10. व॒ल॒ग॒हनो ऽवाव॑ वलग॒हनो॑ वलग॒हनो ऽव॑ नयामि नया॒म्यव॑ वलग॒हनो॑ वलग॒हनो ऽव॑ नयामि । \newline
11. व॒ल॒ग॒हन॒ इति॑ वलग - हनः॑ । \newline
12. अव॑ नयामि नया॒ म्यवाव॑ नयामि वैष्ण॒वान्. वै᳚ष्ण॒वान् न॑या॒ म्यवाव॑ नयामि वैष्ण॒वान् । \newline
13. न॒या॒मि॒ वै॒ष्ण॒वान्. वै᳚ष्ण॒वान् न॑यामि नयामि वैष्ण॒वान्. यवो॒ यवो॑ वैष्ण॒वान् न॑यामि नयामि वैष्ण॒वान्. यवः॑ । \newline
14. वै॒ष्ण॒वान्. यवो॒ यवो॑ वैष्ण॒वान्. वै᳚ष्ण॒वान्. यवो᳚ ऽस्यसि॒ यवो॑ वैष्ण॒वान्. वै᳚ष्ण॒वान्. यवो॑ ऽसि । \newline
15. यवो᳚ ऽस्यसि॒ यवो॒ यवो॑ ऽसि य॒वय॑ य॒वया॑सि॒ यवो॒ यवो॑ ऽसि य॒वय॑ । \newline
16. अ॒सि॒ य॒वय॑ य॒वया᳚ स्यसि य॒वया॒ स्मद॒ स्मद् य॒वया᳚ स्यसि य॒वया॒स्मत् । \newline
17. य॒वया॒ स्म द॒स्मद् य॒वय॑ य॒वया॒स्मद् द्वेषो॒ द्वेषो॒ ऽस्मद् य॒वय॑ य॒वया॒स्मद् द्वेषः॑ । \newline
18. अ॒स्मद् द्वेषो॒ द्वेषो॒ ऽस्म द॒स्मद् द्वेषो॑ य॒वय॑ य॒वय॒ द्वेषो॒ ऽस्म द॒स्मद् द्वेषो॑ य॒वय॑ । \newline
19. द्वेषो॑ य॒वय॑ य॒वय॒ द्वेषो॒ द्वेषो॑ य॒वयारा॑ती॒ ररा॑तीर् य॒वय॒ द्वेषो॒ द्वेषो॑ य॒वयारा॑तीः । \newline
20. य॒वयारा॑ती॒ ररा॑तीर् य॒वय॑ य॒वयारा॑ती रक्षो॒हणो॑ रक्षो॒हणो ऽरा॑तीर् य॒वय॑ य॒वयारा॑ती रक्षो॒हणः॑ । \newline
21. अरा॑ती रक्षो॒हणो॑ रक्षो॒हणो ऽरा॑ती॒ ररा॑ती रक्षो॒हणो॑ वलग॒हनो॑ वलग॒हनो॑ रक्षो॒हणो ऽरा॑ती॒ ररा॑ती रक्षो॒हणो॑ वलग॒हनः॑ । \newline
22. र॒क्षो॒हणो॑ वलग॒हनो॑ वलग॒हनो॑ रक्षो॒हणो॑ रक्षो॒हणो॑ वलग॒हनो ऽवाव॑ वलग॒हनो॑ रक्षो॒हणो॑ रक्षो॒हणो॑ वलग॒हनो ऽव॑ । \newline
23. र॒क्षो॒हण॒ इति॑ रक्षः - हनः॑ । \newline
24. व॒ल॒ग॒हनो ऽवाव॑ वलग॒हनो॑ वलग॒हनो ऽव॑ स्तृणामि स्तृणा॒ म्यव॑ वलग॒हनो॑ वलग॒हनो ऽव॑ स्तृणामि । \newline
25. व॒ल॒ग॒हन॒ इति॑ वलग - हनः॑ । \newline
26. अव॑ स्तृणामि स्तृणा॒ म्यवाव॑ स्तृणामि वैष्ण॒वान्. वै᳚ष्ण॒वान् थ्स्तृ॑णा॒ म्यवाव॑ स्तृणामि वैष्ण॒वान् । \newline
27. स्तृ॒णा॒मि॒ वै॒ष्ण॒वान्. वै᳚ष्ण॒वान् थ्स्तृ॑णामि स्तृणामि वैष्ण॒वान् र॑क्षो॒हणो॑ रक्षो॒हणो॑ वैष्ण॒वान् थ्स्तृ॑णामि स्तृणामि वैष्ण॒वान् र॑क्षो॒हणः॑ । \newline
28. वै॒ष्ण॒वान् र॑क्षो॒हणो॑ रक्षो॒हणो॑ वैष्ण॒वान्. वै᳚ष्ण॒वान् र॑क्षो॒हणो॑ वलग॒हनो॑ वलग॒हनो॑ रक्षो॒हणो॑ वैष्ण॒वान्. वै᳚ष्ण॒वान् र॑क्षो॒हणो॑ वलग॒हनः॑ । \newline
29. र॒क्षो॒हणो॑ वलग॒हनो॑ वलग॒हनो॑ रक्षो॒हणो॑ रक्षो॒हणो॑ वलग॒हनो॒ ऽभ्य॑भि व॑लग॒हनो॑ रक्षो॒हणो॑ रक्षो॒हणो॑ वलग॒हनो॒ ऽभि । \newline
30. र॒क्षो॒हण॒ इति॑ रक्षः - हनः॑ । \newline
31. व॒ल॒ग॒हनो॒ ऽभ्य॑भि व॑लग॒हनो॑ वलग॒हनो॒ ऽभि जु॑होमि जुहोम्य॒भि व॑लग॒हनो॑ वलग॒हनो॒ ऽभि जु॑होमि । \newline
32. व॒ल॒ग॒हन॒ इति॑ वलग - हनः॑ । \newline
33. अ॒भि जु॑होमि जुहो म्य॒भ्य॑भि जु॑होमि वैष्ण॒वान्. वै᳚ष्ण॒वान् जु॑हो म्य॒भ्य॑भि जु॑होमि वैष्ण॒वान् । \newline
34. जु॒हो॒मि॒ वै॒ष्ण॒वान्. वै᳚ष्ण॒वान् जु॑होमि जुहोमि वैष्ण॒वान् र॑क्षो॒हणौ॑ रक्षो॒हणौ॑ वैष्ण॒वान् जु॑होमि जुहोमि वैष्ण॒वान् र॑क्षो॒हणौ᳚ । \newline
35. वै॒ष्ण॒वान् र॑क्षो॒हणौ॑ रक्षो॒हणौ॑ वैष्ण॒वान्. वै᳚ष्ण॒वान् र॑क्षो॒हणौ॑ वलग॒हनौ॑ वलग॒हनौ॑ रक्षो॒हणौ॑ वैष्ण॒वान्. वै᳚ष्ण॒वान् र॑क्षो॒हणौ॑ वलग॒हनौ᳚ । \newline
36. र॒क्षो॒हणौ॑ वलग॒हनौ॑ वलग॒हनौ॑ रक्षो॒हणौ॑ रक्षो॒हणौ॑ वलग॒ह ना॑पोप॑ वलग॒हनौ॑ रक्षो॒हणौ॑ रक्षो॒हणौ॑ वलग॒हना॒वुप॑ । \newline
37. र॒क्षो॒हणा॒विति॑ रक्षः - हनौ᳚ । \newline
38. व॒ल॒ग॒ह ना॒वुपोप॑ वलग॒हनु॑ वलग॒हना॒वुप॑ दधामि दधा॒म्युप॑ वलग॒हनौ॑ वलग॒हना॒वुप॑ दधामि । \newline
39. व॒ल॒ग॒हना॒विति॑ वलग - हनौ᳚ । \newline
40. उप॑ दधामि दधा॒ म्युपोप॑ दधामि वैष्ण॒वी वै᳚ष्ण॒वी द॑धा॒ म्युपोप॑ दधामि वैष्ण॒वी । \newline
41. द॒धा॒मि॒ वै॒ष्ण॒वी वै᳚ष्ण॒वी द॑धामि दधामि वैष्ण॒वी र॑क्षो॒हणौ॑ रक्षो॒हणौ॑ वैष्ण॒वी द॑धामि दधामि वैष्ण॒वी र॑क्षो॒हणौ᳚ । \newline
42. वै॒ष्ण॒वी र॑क्षो॒हणौ॑ रक्षो॒हणौ॑ वैष्ण॒वी वै᳚ष्ण॒वी र॑क्षो॒हणौ॑ वलग॒हनौ॑ वलग॒हनौ॑ रक्षो॒हणौ॑ वैष्ण॒वी वै᳚ष्ण॒वी र॑क्षो॒हणौ॑ वलग॒हनौ᳚ । \newline
43. वै॒ष्ण॒वी इति॑ वैष्ण॒वी । \newline
44. र॒क्षो॒हणौ॑ वलग॒हनौ॑ वलग॒हनौ॑ रक्षो॒हणौ॑ रक्षो॒हणौ॑ वलग॒हनौ॒ परि॒ परि॑ वलग॒हनौ॑ रक्षो॒हणौ॑ रक्षो॒हणौ॑ वलग॒हनौ॒ परि॑ । \newline
45. र॒क्षो॒हणा॒विति॑ रक्षः - हनौ᳚ । \newline
46. व॒ल॒ग॒हनौ॒ परि॒ परि॑ वलग॒हनौ॑ वलग॒हनौ॒ पर्यू॑हा म्यूहामि॒ परि॑ वलग॒हनौ॑ वलग॒हनौ॒ पर्यू॑हामि । \newline
47. व॒ल॒ग॒हना॒विति॑ वलग - हनौ᳚ । \newline
48. पर्यू॑हा म्यूहामि॒ परि॒ पर्यू॑हामि वैष्ण॒वी वै᳚ष्ण॒वी ऊ॑हामि॒ परि॒ पर्यू॑हामि वैष्ण॒वी । \newline
49. ऊ॒हा॒मि॒ वै॒ष्ण॒वी वै᳚ष्ण॒वी ऊ॑हा म्यूहामि वैष्ण॒वी र॑क्षो॒हणौ॑ रक्षो॒हणौ॑ वैष्ण॒वी ऊ॑हा म्यूहामि वैष्ण॒वी र॑क्षो॒हणौ᳚ । \newline
50. वै॒ष्ण॒वी र॑क्षो॒हणौ॑ रक्षो॒हणौ॑ वैष्ण॒वी वै᳚ष्ण॒वी र॑क्षो॒हणौ॑ वलग॒हनौ॑ वलग॒हनौ॑ रक्षो॒हणौ॑ वैष्ण॒वी वै᳚ष्ण॒वी र॑क्षो॒हणौ॑ वलग॒हनौ᳚ । \newline
51. वै॒ष्ण॒वी इति॑ वैष्ण॒वी । \newline
52. र॒क्षो॒हणौ॑ वलग॒हनौ॑ वलग॒हनौ॑ रक्षो॒हणौ॑ रक्षो॒हणौ॑ वलग॒हनौ॒ परि॒ परि॑ वलग॒हनौ॑ रक्षो॒हणौ॑ रक्षो॒हणौ॑ वलग॒हनौ॒ परि॑ । \newline
53. र॒क्षो॒हणा॒विति॑ रक्षः - हनौ᳚ । \newline
54. व॒ल॒ग॒हनौ॒ परि॒ परि॑ वलग॒हनौ॑ वलग॒हनौ॒ परि॑ स्तृणामि स्तृणामि॒ परि॑ वलग॒हनौ॑ वलग॒हनौ॒ परि॑ स्तृणामि । \newline
55. व॒ल॒ग॒हना॒विति॑ वलग - हनौ᳚ । \newline
56. परि॑ स्तृणामि स्तृणामि॒ परि॒ परि॑ स्तृणामि वैष्ण॒वी वै᳚ष्ण॒वी स्तृ॑णामि॒ परि॒ परि॑ स्तृणामि वैष्ण॒वी । \newline
57. स्तृ॒णा॒मि॒ वै॒ष्ण॒वी वै᳚ष्ण॒वी स्तृ॑णामि स्तृणामि वैष्ण॒वी र॑क्षो॒हणौ॑ रक्षो॒हणौ॑ वैष्ण॒वी स्तृ॑णामि स्तृणामि वैष्ण॒वी र॑क्षो॒हणौ᳚ । \newline
58. वै॒ष्ण॒वी र॑क्षो॒हणौ॑ रक्षो॒हणौ॑ वैष्ण॒वी वै᳚ष्ण॒वी र॑क्षो॒हणौ॑ वलग॒हनौ॑ वलग॒हनौ॑ रक्षो॒हणौ॑ वैष्ण॒वी वै᳚ष्ण॒वी र॑क्षो॒हणौ॑ वलग॒हनौ᳚ । \newline
59. वै॒ष्ण॒वी इति॑ वैष्ण॒वी । \newline
60. र॒क्षो॒हणौ॑ वलग॒हनौ॑ वलग॒हनौ॑ रक्षो॒हणौ॑ रक्षो॒हणौ॑ वलग॒हनौ॑ वैष्ण॒वी वै᳚ष्ण॒वी व॑लग॒हनौ॑ रक्षो॒हणौ॑ रक्षो॒हणौ॑ वलग॒हनौ॑ वैष्ण॒वी । \newline
61. र॒क्षो॒हणा॒विति॑ रक्षः - हनौ᳚ । \newline
62. व॒ल॒ग॒हनौ॑ वैष्ण॒वी वै᳚ष्ण॒वी व॑लग॒हनौ॑ वलग॒हनौ॑ वैष्ण॒वी बृ॒हन् बृ॒हन्. वै᳚ष्ण॒वी व॑लग॒हनौ॑ वलग॒हनौ॑ वैष्ण॒वी बृ॒हन्न् । \newline
63. व॒ल॒ग॒हना॒विति॑ वलग - हनौ᳚ । \newline
64. वै॒ष्ण॒वी बृ॒हन् बृ॒हन्. वै᳚ष्ण॒वी वै᳚ष्ण॒वी बृ॒हन् न॑स्यसि बृ॒हन्. वै᳚ष्ण॒वी वै᳚ष्ण॒वी बृ॒हन् न॑सि । \newline
65. वै॒ष्ण॒वी इति॑ वैष्ण॒वी । \newline
66. बृ॒हन् न॑स्यसि बृ॒हन् बृ॒हन् न॑सि बृ॒हद्ग्रा॑वा बृ॒हद्ग्रा॑वा ऽसि बृ॒हन् बृ॒हन् न॑सि बृ॒हद्ग्रा॑वा । \newline
67. अ॒सि॒ बृ॒हद्ग्रा॑वा बृ॒हद्ग्रा॑वा ऽस्यसि बृ॒हद्ग्रा॑वा बृह॒तीम् बृ॑ह॒तीम् बृ॒हद्ग्रा॑वा ऽस्यसि बृ॒हद्ग्रा॑वा बृह॒तीम् । \newline
68. बृ॒हद्ग्रा॑वा बृह॒तीम् बृ॑ह॒तीम् बृ॒हद्ग्रा॑वा बृ॒हद्ग्रा॑वा बृह॒ती मिन्द्रा॒ये न्द्रा॑य बृह॒तीम् बृ॒हद्ग्रा॑वा बृ॒हद्ग्रा॑वा बृह॒ती मिन्द्रा॑य । \newline
69. बृ॒हद्ग्रा॒वेति॑ बृ॒हत् - ग्रा॒वा॒ । \newline
70. बृ॒ह॒ती मिन्द्रा॒ये न्द्रा॑य बृह॒तीम् बृ॑ह॒ती मिन्द्रा॑य॒ वाचं॒ ॅवाच॒ मिन्द्रा॑य बृह॒तीम् बृ॑ह॒ती मिन्द्रा॑य॒ वाच᳚म् । \newline
71. इन्द्रा॑य॒ वाचं॒ ॅवाच॒ मिन्द्रा॒ये न्द्रा॑य॒ वाचं॑ ॅवद वद॒ वाच॒ मिन्द्रा॒ये न्द्रा॑य॒ वाचं॑ ॅवद । \newline
72. वाचं॑ ॅवद वद॒ वाचं॒ ॅवाचं॑ ॅवद । \newline
73. व॒देति॑ वद । \newline
\pagebreak
\markright{ TS 1.3.3.1  \hfill https://www.vedavms.in \hfill}

\section{ TS 1.3.3.1 }

\textbf{TS 1.3.3.1 } \newline
\textbf{Samhita Paata} \newline

वि॒भूर॑सि प्र॒वाह॑णो॒ वह्नि॑रसि हव्य॒वाह॑नः श्वा॒त्रो॑ऽसि॒ प्रचे॑तास्तु॒थो॑ऽसि वि॒श्ववे॑दा उ॒शिग॑सि क॒विरङ्घा॑रिरसि॒ बंभा॑रिरव॒स्युर॑सि॒ दुव॑स्वाञ्छु॒न्ध्यूर॑सि मार्जा॒लीयः॑ स॒म्राड॑सि कृ॒शानुः॑ परि॒षद्यो॑ऽसि॒ पव॑मानः प्र॒तक्वा॑ऽसि॒ नभ॑स्वा॒नसं॑मृष्टोऽसि हव्य॒सूद॑ ऋ॒तधा॑माऽसि॒ सुवर्ज्योति॒र् ब्रह्म॑ज्योतिरसि॒ सुव॑र्द्धामा॒ऽजो᳚ ऽस्येक॑पा॒दहि॑रसि बु॒द्ध्नियो॒ रौद्रे॒णानी॑केन ( ) पा॒हि मा᳚ऽग्ने पिपृ॒हि मा॒ मा मा॑ हिꣳसीः ॥ \newline

\textbf{Pada Paata} \newline

वि॒भूरिति॑ वि - भूः । अ॒सि॒ । प्र॒वाह॑ण॒ इति॑ प्र - वाह॑नः । वह्निः॑ । अ॒सि॒ । ह॒व्य॒वाह॑न॒ इति॑ हव्य - वाह॑नः । श्वा॒त्रः । अ॒सि॒ । प्रचे॑ता॒ इति॒ प्र - चे॒ताः॒ । तु॒थः । अ॒सि॒ । वि॒श्ववे॑दा॒ इति॑ वि॒श्व - वे॒दाः॒ । उ॒शिक् । अ॒सि॒ । क॒विः । अङ्घा॑रिः । अ॒सि॒ । बंभा॑रिः । अ॒व॒स्युः । अ॒सि॒ । दुव॑स्वान् । शु॒न्ध्यूः । अ॒सि॒ । मा॒र्जा॒लीयः॑ । स॒म्राडिति॑ सं - राट् । अ॒सि॒ । कृ॒शानु॒रिति॑ कृ॒श - अ॒नुः॒ । प॒रि॒षद्य॒ इति॑ परि - सद्यः॑ । अ॒सि॒ । पव॑मानः । प्र॒तक्वेति॑ प्र - तक्वा᳚ । अ॒सि॒ । नभ॑स्वान् । अस॑मृंष्ट॒ इत्यसं᳚ - मृ॒ष्टः॒ । अ॒सि॒ । ह॒व्य॒सूद॒ इति॑ हव्य - सूदः॑ । ऋ॒तधा॒मेत्यृ॒त - धा॒मा॒ । अ॒सि॒ । सुव॑र्ज्योति॒रिति॒ सुवः॑ - ज्यो॒तिः॒ । ब्रह्म॑ज्योति॒रिति॒ ब्रह्म॑ - ज्यो॒तिः॒ । अ॒सि॒ । सुव॑र्धा॒मेति॒ सुवः॑ - धा॒मा॒ । अ॒जः । अ॒सि॒ । एक॑पा॒दित्येक॑ - पा॒त् । अहिः॑ । अ॒सि॒ । बु॒द्ध्नियः॑ । रौद्रे॑ण । अनी॑केन ( ) । पा॒हि । मा॒ । अ॒ग्ने॒ । पि॒पृ॒हि । मा॒ । मा । मा॒ । हिꣳ॒॒सीः॒ ॥  \newline


\textbf{Krama Paata} \newline

वि॒भूर॑सि । वि॒भूरिति॑ वि - भूः । अ॒सि॒ प्र॒वाह॑णः । प्र॒वाह॑णो॒ वह्निः॑ । प्र॒वाह॑ण॒ इति॑ प्र - वाह॑नः । वह्नि॑रसि । अ॒सि॒ ह॒व्य॒वाह॑नः । ह॒व्य॒वाह॑नः श्वा॒त्रः । ह॒व्य॒वाह॑न॒ इति॑ हव्य - वाह॑नः । श्वा॒त्रो॑ऽसि । अ॒सि॒ प्रचे॑ताः । प्रचे॑तास्तु॒थः । प्रचे॑ता॒ इति॒ प्र - चे॒ताः॒ । तु॒थो॑ऽसि । अ॒सि॒ वि॒श्ववे॑दाः । वि॒श्ववे॑दा उ॒शिक् । वि॒श्ववे॑दा॒ इति॑ वि॒श्व - वे॒दाः॒ । उ॒शिग॑सि । अ॒सि॒ क॒विः । क॒विरङ्घा॑रिः । अङ्घा॑रिरसि । अ॒सि॒ बम्भा॑रिः । बम्भा॑रिरव॒स्युः । अ॒व॒स्युर॑सि । अ॒सि॒ दुव॑स्वान् । दुव॑स्वाञ्छु॒न्ध्यूः । शु॒न्ध्यूर॑सि । अ॒सि॒ मा॒र्जा॒लीयः॑ । मा॒र्जा॒लीयः॑ स॒म्राट् । स॒म्राड॑सि । स॒म्राडिति॑ सं - राट् । अ॒सि॒ कृ॒शानुः॑ । कृ॒शानुः॑ परि॒षद्यः॑ । कृ॒शानु॒रिति॑ कृ॒श - अ॒नुः॒ । प॒रि॒षद्यो॑ऽसि । प॒रि॒षद्य॒ इति॑ परि - सद्यः॑ । अ॒सि॒ पव॑मानः । पव॑मानः प्र॒तक्वा᳚ । प्र॒तक्वा॑ऽसि । प्र॒तक्वेति॑ प्र - तक्वा᳚ । अ॒सि॒ नभ॑स्वान् । नभ॑स्वा॒नस॑म्मृष्टः । अस॑म्मृष्टोऽसि । अस॑म्मृष्ट॒ इत्यस᳚म् - मृ॒ष्टः॒ । अ॒सि॒ ह॒व्य॒सूदः॑ । ह॒व्य॒सूद॑ ऋ॒तधा॑मा । ह॒व्य॒सूद॒ इति॑ हव्य - सूदः॑ । ऋ॒तधा॑माऽसि । ऋ॒तधा॒मेत्यृ॒त - धा॒मा॒ । अ॒सि॒ सुव॑र्ज्योतिः । सुव॑र्ज्योति॒र् ब्रह्म॑ज्योतिः । सुव॑र्ज्योति॒रिति॒ सुवः॑ - ज्यो॒तिः॒ । ब्रह्म॑ज्योतिरसि । बह्म॑ज्योति॒रिति॒ ब्रह्म॑ - ज्यो॒तिः॒ । अ॒सि॒ सुव॑र्द्धामा । सुव॑र्द्धामा॒ऽजः । सुव॑र्द्धा॒मेति॒ सुवः॑ - धा॒मा॒ । अ॒जो॑ऽसि । अ॒स्येक॑पात् । एक॑पा॒दहिः॑ । एक॑पा॒दित्येक॑ - पा॒त् । अहि॑रसि । अ॒सि॒ बु॒द्ध्नियः॑ । बु॒द्ध्नियो॒ रौद्रे॑ण । रौद्रे॒णानी॑केन ( ) । अनी॑केन पा॒हि । पा॒हि मा᳚ । मा॒ऽग्ने॒ । अ॒ग्ने॒ पि॒पृ॒हि । पि॒पृ॒हि मा᳚ । मा॒ मा । मा मा᳚ । मा॒ हिꣳ॒॒सीः॒ । हिꣳ॒॒सी॒रिति॑ हिꣳसीः । \newline

\textbf{Jatai Paata} \newline

1. वि॒भू र॑स्यसि वि॒भूर् वि॒भू र॑सि । \newline
2. वि॒भूरिति॑ वि - भूः । \newline
3. अ॒सि॒ प्र॒वाह॑णः प्र॒वाह॑णो ऽस्यसि प्र॒वाह॑णः । \newline
4. प्र॒वाह॑णो॒ वह्नि॒र् वह्निः॑ प्र॒वाह॑णः प्र॒वाह॑णो॒ वह्निः॑ । \newline
5. प्र॒वाह॑ण॒ इति॑ प्र - वाह॑नः । \newline
6. वह्नि॑ रस्यसि॒ वह्नि॒र् वह्नि॑ रसि । \newline
7. अ॒सि॒ ह॒व्य॒वाह॑नो हव्य॒वाह॑नो ऽस्यसि हव्य॒वाह॑नः । \newline
8. ह॒व्य॒वाह॑नः श्वा॒त्रः श्वा॒त्रो ह॑व्य॒वाह॑नो हव्य॒वाह॑नः श्वा॒त्रः । \newline
9. ह॒व्य॒वाह॑न॒ इति॑ हव्य - वाह॑नः । \newline
10. श्वा॒त्रो᳚ ऽस्यसि श्वा॒त्रः श्वा॒त्रो॑ ऽसि । \newline
11. अ॒सि॒ प्रचे॑ताः॒ प्रचे॑ता अस्यसि॒ प्रचे॑ताः । \newline
12. प्रचे॑ता स्तु॒थ स्तु॒थः प्रचे॑ताः॒ प्रचे॑ता स्तु॒थः । \newline
13. प्रचे॑ता॒ इति॒ प्र - चे॒ताः॒ । \newline
14. तु॒थो᳚ ऽस्यसि तु॒थ स्तु॒थो॑ ऽसि । \newline
15. अ॒सि॒ वि॒श्ववे॑दा वि॒श्ववे॑दा अस्यसि वि॒श्ववे॑दाः । \newline
16. वि॒श्ववे॑दा उ॒शि गु॒शिग् वि॒श्ववे॑दा वि॒श्ववे॑दा उ॒शिक् । \newline
17. वि॒श्ववे॑दा॒ इति॑ वि॒श्व - वे॒दाः॒ । \newline
18. उ॒शि ग॑स्य स्यु॒शि गु॒शिग॑सि । \newline
19. अ॒सि॒ क॒विः क॒वि र॑स्यसि क॒विः । \newline
20. क॒वि रङ्घा॑रि॒ रङ्घा॑रिः क॒विः क॒वि रङ्घा॑रिः । \newline
21. अङ्घा॑रि रस्य॒स्यङ्घा॑रि॒ रङ्घा॑रिरसि । \newline
22. अ॒सि॒ बंभा॑रि॒र् बंभा॑रि रस्यसि॒ बंभा॑रिः । \newline
23. बंभा॑रि रव॒स्यु र॑व॒स्युर् बंभा॑रि॒र् बंभा॑रि रव॒स्युः । \newline
24. अ॒व॒स्यु र॑स्य स्यव॒स्यु र॑व॒स्युर॑सि । \newline
25. अ॒सि॒ दुव॑स्वा॒न् दुव॑स्वा नस्यसि॒ दुव॑स्वान् । \newline
26. दुव॑स्वाञ् छु॒न्ध्यूः शु॒न्ध्यूर् दुव॑स्वा॒न् दुव॑स्वाञ् छु॒न्ध्यूः । \newline
27. शु॒न्ध्यू र॑स्यसि शु॒न्ध्यूः शु॒न्ध्यू र॑सि । \newline
28. अ॒सि॒ मा॒र्जा॒लीयो॑ मार्जा॒लीयो᳚ ऽस्यसि मार्जा॒लीयः॑ । \newline
29. मा॒र्जा॒लीयः॑ स॒म्राट् थ्स॒म्राण् मा᳚र्जा॒लीयो॑ मार्जा॒लीयः॑ स॒म्राट् । \newline
30. स॒म्राड॑स्यसि स॒म्राट् थ्स॒म्राड॑सि । \newline
31. स॒म्राडिति॑ सं - राट् । \newline
32. अ॒सि॒ कृ॒शानुः॑ कृ॒शानु॑ रस्यसि कृ॒शानुः॑ । \newline
33. कृ॒शानुः॑ परि॒षद्यः॑ परि॒षद्यः॑ कृ॒शानुः॑ कृ॒शानुः॑ परि॒षद्यः॑ । \newline
34. कृ॒शानु॒रिति॑ कृ॒श - अ॒नुः॒ । \newline
35. प॒रि॒षद्यो᳚ ऽस्यसि परि॒षद्यः॑ परि॒षद्यो॑ ऽसि । \newline
36. प॒रि॒षद्य॒ इति॑ परि - सद्यः॑ । \newline
37. अ॒सि॒ पव॑मानः॒ पव॑मानो ऽस्यसि॒ पव॑मानः । \newline
38. पव॑मानः प्र॒तक्वा᳚ प्र॒तक्वा॒ पव॑मानः॒ पव॑मानः प्र॒तक्वा᳚ । \newline
39. प्र॒तक्वा᳚ ऽस्यसि प्र॒तक्वा᳚ प्र॒तक्वा॑ ऽसि । \newline
40. प्र॒तक्वेति॑ प्र - तक्वा᳚ । \newline
41. अ॒सि॒ नभ॑स्वा॒न् नभ॑स्वा नस्यसि॒ नभ॑स्वान् । \newline
42. नभ॑स्वा॒ नसं॑मृ॒ष्टो ऽसं॑मृष्टो॒ नभ॑स्वा॒न् नभ॑स्वा॒ नसं॑मृष्टः । \newline
43. असं॑मृष्टो ऽस्य॒ स्यसं॑मृ॒ष्टो ऽसं॑मृष्टो ऽसि । \newline
44. असं॑मृष्ट॒ इत्यसं᳚ - मृ॒ष्टः॒ । \newline
45. अ॒सि॒ ह॒व्य॒सूदो॑ हव्य॒सूदो᳚ ऽस्यसि हव्य॒सूदः॑ । \newline
46. ह॒व्य॒सूद॑ ऋ॒तधा॑म॒र्तधा॑मा हव्य॒सूदो॑ हव्य॒सूद॑ ऋ॒तधा॑मा । \newline
47. ह॒व्य॒सूद॒ इति॑ हव्य - सूदः॑ । \newline
48. ऋ॒तधा॑मा ऽस्य स्यृ॒तधा॑म॒ र्तधा॑मा ऽसि । \newline
49. ऋ॒तधा॒मेत्यृ॒त - धा॒मा॒ । \newline
50. अ॒सि॒ सुव॑र्ज्योतिः॒ सुव॑र्ज्योति रस्यसि॒ सुव॑र्ज्योतिः । \newline
51. सुव॑र्ज्योति॒र् ब्रह्म॑ज्योति॒र् ब्रह्म॑ज्योतिः॒ सुव॑र्ज्योतिः॒ सुव॑र्ज्योति॒र् ब्रह्म॑ज्योतिः । \newline
52. सुव॑र्ज्योति॒रिति॒ सुवः॑ - ज्यो॒तिः॒ । \newline
53. ब्रह्म॑ज्योति रस्यसि॒ ब्रह्म॑ज्योति॒र् ब्रह्म॑ज्योति रसि । \newline
54. ब्रह्म॑ज्योति॒रिति॒ ब्रह्म॑ - ज्यो॒तिः॒ । \newline
55. अ॒सि॒ सुव॑र्धामा॒ सुव॑र्धामा ऽस्यसि॒ सुव॑र्धामा । \newline
56. सुव॑र्धामा॒ ऽजो॑ ऽजः सुव॑र्धामा॒ सुव॑र्धामा॒ ऽजः । \newline
57. सुव॑र्धा॒मेति॒ सुवः॑ - धा॒मा॒ । \newline
58. अ॒जो᳚ ऽस्यस्य॒जो᳚(1॒) ऽजो॑ ऽसि । \newline
59. अ॒स्ये क॑पा॒ देक॑पा दस्य॒ स्येक॑पात् । \newline
60. एक॑पा॒ दहि॒ रहि॒ रेक॑पा॒ देक॑पा॒ दहिः॑ । \newline
61. एक॑पा॒दित्येक॑ - पा॒त् । \newline
62. अहि॑ रस्य॒ स्यहि॒ रहि॑ रसि । \newline
63. अ॒सि॒ बु॒द्ध्नियो॑ बु॒द्ध्नियो᳚ ऽस्यसि बु॒द्ध्नियः॑ । \newline
64. बु॒द्ध्नियो॒ रौद्रे॑ण॒ रौद्रे॑ण बु॒द्ध्नियो॑ बु॒द्ध्नियो॒ रौद्रे॑ण । \newline
65. रौद्रे॒ णानी॑के॒नानी॑केन॒ रौद्रे॑ण॒ रौद्रे॒णानी॑केन । \newline
66. अनी॑केन पा॒हि पा॒ह्य नी॑के॒नानी॑केन पा॒हि । \newline
67. पा॒हि मा॑ मा पा॒हि पा॒हि मा᳚ । \newline
68. मा॒ ऽग्ने॒ ऽग्ने॒ मा॒ मा॒ ऽग्ने॒ । \newline
69. अ॒ग्ने॒ पि॒पृ॒हि पि॑पृ॒ह्य॑ग्ने ऽग्ने पिपृ॒हि । \newline
70. पि॒पृ॒हि मा॑ मा पिपृ॒हि पि॑पृ॒हि मा᳚ । \newline
71. मा॒ मा मा मा॑ मा॒ मा । \newline
72. मा मा॑ मा॒ मा मा मा᳚ । \newline
73. मा॒ हि॒(ग्म्॒)सी॒र्॒. हि॒(ग्म्॒)सी॒र् मा॒ मा॒ हि॒(ग्म्॒)सीः॒ । \newline
74. हि॒(ग्म्॒)सी॒रिति॑ हिꣳसीः । \newline

\textbf{Ghana Paata } \newline

1. वि॒भू र॑स्यसि वि॒भूर् वि॒भूर॑सि प्र॒वाह॑णः प्र॒वाह॑णो ऽसि वि॒भूर् वि॒भू र॑सि प्र॒वाह॑णः । \newline
2. वि॒भूरिति॑ वि - भूः । \newline
3. अ॒सि॒ प्र॒वाह॑णः प्र॒वाह॑णो ऽस्यसि प्र॒वाह॑णो॒ वह्नि॒र् वह्निः॑ प्र॒वाह॑णो ऽस्यसि प्र॒वाह॑णो॒ वह्निः॑ । \newline
4. प्र॒वाह॑णो॒ वह्नि॒र् वह्निः॑ प्र॒वाह॑णः प्र॒वाह॑णो॒ वह्नि॑ रस्यसि॒ वह्निः॑ प्र॒वाह॑णः प्र॒वाह॑णो॒ वह्नि॑रसि । \newline
5. प्र॒वाह॑ण॒ इति॑ प्र - वाह॑नः । \newline
6. वह्नि॑ रस्यसि॒ वह्नि॒र् वह्नि॑रसि हव्य॒वाह॑नो हव्य॒वाह॑नो ऽसि॒ वह्नि॒र् वह्नि॑रसि हव्य॒वाह॑नः । \newline
7. अ॒सि॒ ह॒व्य॒वाह॑नो हव्य॒वाह॑नो ऽस्यसि हव्य॒वाह॑नः श्वा॒त्रः श्वा॒त्रो ह॑व्य॒वाह॑नो ऽस्यसि हव्य॒वाह॑नः श्वा॒त्रः । \newline
8. ह॒व्य॒वाह॑नः श्वा॒त्रः श्वा॒त्रो ह॑व्य॒वाह॑नो हव्य॒वाह॑नः श्वा॒त्रो᳚ ऽस्यसि श्वा॒त्रो ह॑व्य॒वाह॑नो हव्य॒वाह॑नः श्वा॒त्रो॑ ऽसि । \newline
9. ह॒व्य॒वाह॑न॒ इति॑ हव्य - वाह॑नः । \newline
10. श्वा॒त्रो᳚ ऽस्यसि श्वा॒त्रः श्वा॒त्रो॑ ऽसि॒ प्रचे॑ताः॒ प्रचे॑ता असि श्वा॒त्रः श्वा॒त्रो॑ ऽसि॒ प्रचे॑ताः । \newline
11. अ॒सि॒ प्रचे॑ताः॒ प्रचे॑ता अस्यसि॒ प्रचे॑ता स्तु॒थ स्तु॒थः प्रचे॑ता अस्यसि॒ प्रचे॑ता स्तु॒थः । \newline
12. प्रचे॑ता स्तु॒थ स्तु॒थः प्रचे॑ताः॒ प्रचे॑ता स्तु॒थो᳚ ऽस्यसि तु॒थः प्रचे॑ताः॒ प्रचे॑ता स्तु॒थो॑ ऽसि । \newline
13. प्रचे॑ता॒ इति॒ प्र - चे॒ताः॒ । \newline
14. तु॒थो᳚ ऽस्यसि तु॒थ स्तु॒थो॑ ऽसि वि॒श्ववे॑दा वि॒श्ववे॑दा असि तु॒थ स्तु॒थो॑ ऽसि वि॒श्ववे॑दाः । \newline
15. अ॒सि॒ वि॒श्ववे॑दा वि॒श्ववे॑दा अस्यसि वि॒श्ववे॑दा उ॒शि गु॒शिग् वि॒श्ववे॑दा अस्यसि वि॒श्ववे॑दा उ॒शिक् । \newline
16. वि॒श्ववे॑दा उ॒शि गु॒शिग् वि॒श्ववे॑दा वि॒श्ववे॑दा उ॒शि ग॑स्य स्यु॒शिग् वि॒श्ववे॑दा वि॒श्ववे॑दा उ॒शिग॑सि । \newline
17. वि॒श्ववे॑दा॒ इति॑ वि॒श्व - वे॒दाः॒ । \newline
18. उ॒शि ग॑स्य स्यु॒शि गु॒शि ग॑सि क॒विः क॒वि र॑स्यु॒शि गु॒शि ग॑सि क॒विः । \newline
19. अ॒सि॒ क॒विः क॒वि र॑स्यसि क॒वि रङ्घा॑रि॒ रङ्घा॑रिः क॒वि र॑स्यसि क॒वि रङ्घा॑रिः । \newline
20. क॒वि रङ्घा॑रि॒ रङ्घा॑रिः क॒विः क॒वि रङ्घा॑रि रस्य॒ स्यङ्घा॑रिः क॒विः क॒वि रङ्घा॑रि रसि । \newline
21. अङ्घा॑रि रस्य॒ स्यङ्घा॑रि॒ रङ्घा॑रिरसि॒ बंभा॑रि॒र् बंभा॑रि र॒स्यङ्घा॑रि॒ रङ्घा॑रि रसि॒ बंभा॑रिः । \newline
22. अ॒सि॒ बंभा॑रि॒र् बंभा॑रि रस्यसि॒ बंभा॑रि रव॒स्यु र॑व॒स्युर् बंभा॑रि रस्यसि॒ बंभा॑रि रव॒स्युः । \newline
23. बंभा॑रि रव॒स्यु र॑व॒स्युर् बंभा॑रि॒र् बंभा॑रि रव॒स्यु र॑स्य स्यव॒स्युर् बंभा॑रि॒र् बंभा॑रि रव॒स्यु र॑सि । \newline
24. अ॒व॒स्यु र॑स्य स्यव॒स्यु र॑व॒स्युर॑सि॒ दुव॑स्वा॒न् दुव॑स्वा नस्यव॒स्यु र॑व॒स्यु र॑सि॒ दुव॑स्वान् । \newline
25. अ॒सि॒ दुव॑स्वा॒न् दुव॑स्वा नस्यसि॒ दुव॑स्वाञ् छु॒न्ध्यूः शु॒न्ध्यूर् दुव॑स्वा नस्यसि॒ दुव॑स्वाञ् छु॒न्ध्यूः । \newline
26. दुव॑स्वाञ् छु॒न्ध्यूः शु॒न्ध्यूर् दुव॑स्वा॒न् दुव॑स्वाञ् छु॒न्ध्यू र॑स्यसि शु॒न्ध्यूर् दुव॑स्वा॒न् दुव॑स्वाञ् छु॒न्ध्यूर॑सि । \newline
27. शु॒न्ध्यू र॑स्यसि शु॒न्ध्यूः शु॒न्ध्यूर॑सि मार्जा॒लीयो॑ मार्जा॒लीयो॑ ऽसि शु॒न्ध्यूः शु॒न्ध्यूर॑सि मार्जा॒लीयः॑ । \newline
28. अ॒सि॒ मा॒र्जा॒लीयो॑ मार्जा॒लीयो᳚ ऽस्यसि मार्जा॒लीयः॑ स॒म्राट्थ् स॒म्राण् मा᳚र्जा॒लीयो᳚ ऽस्यसि मार्जा॒लीयः॑ स॒म्राट् । \newline
29. मा॒र्जा॒लीयः॑ स॒म्राट्थ् स॒म्राण् मा᳚र्जा॒लीयो॑ मार्जा॒लीयः॑ स॒म्राड॑स्यसि स॒म्राण् मा᳚र्जा॒लीयो॑ मार्जा॒लीयः॑ स॒म्राड॑सि । \newline
30. स॒म्राड॑स्यसि स॒म्राट्थ् स॒म्राड॑सि कृ॒शानुः॑ कृ॒शानु॑रसि स॒म्राट्थ् स॒म्राड॑सि कृ॒शानुः॑ । \newline
31. स॒म्राडिति॑ सं - राट् । \newline
32. अ॒सि॒ कृ॒शानुः॑ कृ॒शानु॑ रस्यसि कृ॒शानुः॑ परि॒षद्यः॑ परि॒षद्यः॑ कृ॒शानु॑ रस्यसि कृ॒शानुः॑ परि॒षद्यः॑ । \newline
33. कृ॒शानुः॑ परि॒षद्यः॑ परि॒षद्यः॑ कृ॒शानुः॑ कृ॒शानुः॑ परि॒षद्यो᳚ ऽस्यसि परि॒षद्यः॑ कृ॒शानुः॑ कृ॒शानुः॑ परि॒षद्यो॑ ऽसि । \newline
34. कृ॒शानु॒रिति॑ कृ॒श - अ॒नुः॒ । \newline
35. प॒रि॒षद्यो᳚ ऽस्यसि परि॒षद्यः॑ परि॒षद्यो॑ ऽसि॒ पव॑मानः॒ पव॑मानो ऽसि परि॒षद्यः॑ परि॒षद्यो॑ ऽसि॒ पव॑मानः । \newline
36. प॒रि॒षद्य॒ इति॑ परि - सद्यः॑ । \newline
37. अ॒सि॒ पव॑मानः॒ पव॑मानो ऽस्यसि॒ पव॑मानः प्र॒तक्वा᳚ प्र॒तक्वा॒ पव॑मानो ऽस्यसि॒ पव॑मानः प्र॒तक्वा᳚ । \newline
38. पव॑मानः प्र॒तक्वा᳚ प्र॒तक्वा॒ पव॑मानः॒ पव॑मानः प्र॒तक्वा᳚ ऽस्यसि प्र॒तक्वा॒ पव॑मानः॒ पव॑मानः प्र॒तक्वा॑ ऽसि । \newline
39. प्र॒तक्वा᳚ ऽस्यसि प्र॒तक्वा᳚ प्र॒तक्वा॑ ऽसि॒ नभ॑स्वा॒न् नभ॑स्वा नसि प्र॒तक्वा᳚ प्र॒तक्वा॑ ऽसि॒ नभ॑स्वान् । \newline
40. प्र॒तक्वेति॑ प्र - तक्वा᳚ । \newline
41. अ॒सि॒ नभ॑स्वा॒न् नभ॑स्वा नस्यसि॒ नभ॑स्वा॒ नसं॑मृ॒ष्टो ऽसं॑मृष्टो॒ नभ॑स्वा नस्यसि॒ 
नभ॑स्वा॒ नसं॑मृष्टः । \newline
42. नभ॑स्वा॒ नसं॑मृ॒ष्टो ऽसं॑मृष्टो॒ नभ॑स्वा॒न् नभ॑स्वा॒ नसं॑मृष्टो ऽस्य॒स्यसं॑मृष्टो॒ नभ॑स्वा॒न् नभ॑स्वा॒ नसं॑मृष्टो ऽसि । \newline
43. असं॑मृष्टो ऽस्य॒स्यसं॑मृ॒ष्टो ऽसं॑मृष्टो ऽसि हव्य॒सूदो॑ हव्य॒सूदो॒ ऽस्यसं॑मृ॒ष्टो ऽसं॑मृष्टो ऽसि हव्य॒सूदः॑ । \newline
44. असं॑मृष्ट॒ इत्यसं᳚ - मृ॒ष्टः॒ । \newline
45. अ॒सि॒ ह॒व्य॒सूदो॑ हव्य॒सूदो᳚ ऽस्यसि हव्य॒सूद॑ ऋ॒तधा॑म॒र्तधा॑मा हव्य॒सूदो᳚ ऽस्यसि हव्य॒सूद॑ ऋ॒तधा॑मा । \newline
46. ह॒व्य॒सूद॑ ऋ॒तधा॑म॒र्तधा॑मा हव्य॒सूदो॑ हव्य॒सूद॑ ऋ॒तधा॑मा ऽस्यस्यृ॒तधा॑मा हव्य॒सूदो॑ हव्य॒सूद॑ ऋ॒तधा॑मा ऽसि । \newline
47. ह॒व्य॒सूद॒ इति॑ हव्य - सूदः॑ । \newline
48. ऋ॒तधा॑मा ऽस्यस्यृ॒तधा॑म॒र्तधा॑मा ऽसि॒ सुव॑र्ज्योतिः॒ सुव॑र्ज्योति रस्यृ॒तधा॑म॒र्तधा॑मा ऽसि॒ सुव॑र्ज्योतिः । \newline
49. ऋ॒तधा॒मेत्यृ॒त - धा॒मा॒ । \newline
50. अ॒सि॒ सुव॑र्ज्योतिः॒ सुव॑र्ज्योति रस्यसि॒ सुव॑र्ज्योति॒र् ब्रह्म॑ज्योति॒र् ब्रह्म॑ज्योतिः॒ सुव॑र्ज्योति रस्यसि॒ सुव॑र्ज्योति॒र् ब्रह्म॑ज्योतिः । \newline
51. सुव॑र्ज्योति॒र् ब्रह्म॑ज्योति॒र् ब्रह्म॑ज्योतिः॒ सुव॑र्ज्योतिः॒ सुव॑र्ज्योति॒र् ब्रह्म॑ज्योति रस्यसि॒ ब्रह्म॑ज्योतिः॒ सुव॑र्ज्योतिः॒ सुव॑र्ज्योति॒र् ब्रह्म॑ज्योतिरसि । \newline
52. सुव॑र्ज्योति॒रिति॒ सुवः॑ - ज्यो॒तिः॒ । \newline
53. ब्रह्म॑ज्योति रस्यसि॒ ब्रह्म॑ज्योति॒र् ब्रह्म॑ज्योतिरसि॒ सुव॑र्धामा॒ सुव॑र्धामा ऽसि॒ ब्रह्म॑ज्योति॒र् ब्रह्म॑ज्योतिरसि॒ सुव॑र्धामा । \newline
54. ब्रह्म॑ज्योति॒रिति॒ ब्रह्म॑ - ज्यो॒तिः॒ । \newline
55. अ॒सि॒ सुव॑र्धामा॒ सुव॑र्धामा ऽस्यसि॒ सुव॑र्धामा॒ ऽजो॑ ऽजः सुव॑र्धामा ऽस्यसि॒ सुव॑र्धामा॒ ऽजः । \newline
56. सुव॑र्धामा॒ ऽजो॑ ऽजः सुव॑र्धामा॒ सुव॑र्धामा॒ ऽजो᳚ ऽस्यस्य॒जः सुव॑र्धामा॒ सुव॑र्धामा॒ ऽजो॑ ऽसि । \newline
57. सुव॑र्धा॒मेति॒ सुवः॑ - धा॒मा॒ । \newline
58. अ॒जो᳚ ऽस्य स्य॒जो᳚ (1॒) ऽजो᳚ ऽस्येक॑पा॒ देक॑पा दस्य॒जो᳚ (1॒) ऽजो᳚ ऽस्येक॑पात् । \newline
59. अ॒स्येक॑पा॒ देक॑पा दस्य॒ स्येक॑पा॒ दहि॒ रहि॒रेक॑पा दस्य॒ स्येक॑पा॒ दहिः॑ । \newline
60. एक॑पा॒ दहि॒ रहि॒ रेक॑पा॒ देक॑पा॒ दहि॑ रस्य॒ स्यहि॒ रेक॑पा॒ देक॑पा॒ दहि॑ रसि । \newline
61. एक॑पा॒दित्येक॑ - पा॒त् । \newline
62. अहि॑ रस्य॒ स्यहि॒ रहि॑रसि बु॒द्ध्नियो॑ बु॒द्ध्नियो॒ ऽस्यहि॒ रहि॑रसि बु॒द्ध्नियः॑ । \newline
63. अ॒सि॒ बु॒द्ध्नियो॑ बु॒द्ध्नियो᳚ ऽस्यसि बु॒द्ध्नियो॒ रौद्रे॑ण॒ रौद्रे॑ण बु॒द्ध्नियो᳚ ऽस्यसि बु॒द्ध्नियो॒ रौद्रे॑ण । \newline
64. बु॒द्ध्नियो॒ रौद्रे॑ण॒ रौद्रे॑ण बु॒द्ध्नियो॑ बु॒द्ध्नियो॒ रौद्रे॒णानी॑ के॒नानी॑केन॒ रौद्रे॑ण बु॒द्ध्नियो॑ बु॒द्ध्नियो॒ रौद्रे॒णानी॑केन । \newline
65. रौद्रे॒णानी॑ के॒नानी॑केन॒ रौद्रे॑ण॒ रौद्रे॒णानी॑केन पा॒हि पा॒ह्यनी॑केन॒ रौद्रे॑ण॒ रौद्रे॒णानी॑केन पा॒हि । \newline
66. अनी॑केन पा॒हि पा॒ह्यनी॑के॒ नानी॑केन पा॒हि मा॑ मा पा॒ह्यनी॑के॒ नानी॑केन पा॒हि मा᳚ । \newline
67. पा॒हि मा॑ मा पा॒हि पा॒हि मा᳚ ऽग्ने ऽग्ने मा पा॒हि पा॒हि मा᳚ ऽग्ने । \newline
68. मा॒ ऽग्ने॒ ऽग्ने॒ मा॒ मा॒ ऽग्ने॒ पि॒पृ॒हि पि॑पृ॒ह्य॑ग्ने मा मा ऽग्ने पिपृ॒हि । \newline
69. अ॒ग्ने॒ पि॒पृ॒हि पि॑पृ॒ह्य॑ग्ने ऽग्ने पिपृ॒हि मा॑ मा पिपृ॒ह्य॑ग्ने ऽग्ने पिपृ॒हि मा᳚ । \newline
70. पि॒पृ॒हि मा॑ मा पिपृ॒हि पि॑पृ॒हि मा॒ मा मा मा॑ पिपृ॒हि पि॑पृ॒हि मा॒ मा । \newline
71. मा॒ मा मा मा॑ मा॒ मा मा॑ मा॒ मा मा॑ मा॒ मा मा᳚ । \newline
72. मा मा॑ मा॒ मा मा मा॑ हिꣳसीर्. हिꣳसीर् मा॒ मा मा मा॑ हिꣳसीः । \newline
73. मा॒ हि॒(ग्म्॒)सी॒र्॒. हि॒(ग्म्॒)सी॒र् मा॒ मा॒ हि॒(ग्म्॒)सीः॒ । \newline
74. हि॒(ग्म्॒)सी॒रिति॑ हिꣳसीः । \newline
\pagebreak
\markright{ TS 1.3.4.1  \hfill https://www.vedavms.in \hfill}

\section{ TS 1.3.4.1 }

\textbf{TS 1.3.4.1 } \newline
\textbf{Samhita Paata} \newline

त्वꣳ सो॑म तनू॒कृद्भ्यो॒ द्वेषो᳚भ्यो॒ऽन्यकृ॑तेभ्य उ॒रु य॒न्ताऽसि॒ वरू॑थꣳ॒॒ स्वाहा॑ जुषा॒णो अ॒प्तुराज्य॑स्य वेतु॒ स्वाहा॒ऽयन्नो॑ अ॒ग्निर्वरि॑वः कृणोत्व॒यं मृधः॑ पु॒र ए॑तु प्रभि॒न्दन्न्  । अ॒यꣳ शत्रू᳚ञ्जयतु॒ जर्.हृ॑षाणो॒ऽयं ॅवाजं॑ जयतु॒ वाज॑सातौ । उ॒रु वि॑ष्णो॒ वि क्र॑मस्वो॒रु क्षया॑य नः कृधि । घृ॒तं घृ॑तयोने पिब॒ प्रप्र॑ य॒ज्ञ्प॑तिं तिर । सोमो॑ जिगाति गातु॒विद् - [ ] \newline

\textbf{Pada Paata} \newline

त्वम् । सो॒म॒ । त॒नू॒कृद्भ्य॒ इति॑ तनू॒कृत् - भ्यः॒ । द्वेषो᳚भ्य॒ इति॒ द्वेषः॑ - भ्यः॒ । अ॒न्यकृ॑तेभ्य॒ इत्य॒न्य - कृ॒ते॒भ्यः॒ । उ॒रु । य॒न्ता । अ॒सि॒ । वरू॑थम् । स्वाहा᳚ । जु॒षा॒णः । अ॒प्तुः । आज्य॑स्य । वे॒तु॒ । स्वाहा᳚ । अ॒यं । नः॒ । अ॒ग्निः । वरि॑वः । कृ॒णो॒तु॒ । अ॒यम् । मृधः॑ । पु॒रः । ए॒तु॒ । प्र॒भि॒न्दन्निति॑ प्र - भि॒न्दन्न् ॥ अ॒यम् । शत्रून्॑ । ज॒य॒तु॒ । जर्.हृ॑षाणः । अ॒यम् । वाज᳚म् । ज॒य॒तु॒ । वाज॑साता॒विति॒ वाज॑ - सा॒तौ॒ ॥ उ॒रु । वि॒ष्णो॒ इति॑ । वीति॑ । क्र॒म॒स्व॒ । उ॒रु । क्षया॑य । नः॒ । कृ॒धि॒ ॥ घृ॒तम् । घृ॒त॒यो॒न॒ इति॑ घृत - यो॒ने॒ । पि॒ब॒ । प्रप्रेति॒ प्र - प्र॒ । य॒ज्ञ्प॑ति॒मिति॑ य॒ज्ञ् - प॒ति॒म् । ति॒र॒ ॥ सोमः॑ । जि॒गा॒ति॒ । गा॒तु॒विदिति॑ गातु - वित् ।  \newline


\textbf{Krama Paata} \newline

त्वꣳ सो॑म । सो॒म॒ त॒नू॒कृद्भ्यः॑ । त॒नू॒कृद्भ्यो॒ द्वेषो᳚भ्यः । त॒नू॒कृद्भ्य॒ इति॑ तनू॒कृत् - भ्यः॒ । द्वेषो᳚भ्यो॒ऽन्यकृ॑तेभ्यः । द्वेषो᳚भ्य॒ इति॒ द्वेषः॑ - भ्यः॒ । अ॒न्यकृ॑तेभ्य उ॒रु । अ॒न्यकृ॑तेभ्य॒ इत्य॒न्य - कृ॒ते॒भ्यः॒ । उ॒रु य॒न्ता । य॒न्ताऽसि॑ । अ॒सि॒ वरू॑थम् । वरू॑थꣳ॒॒ स्वाहा᳚ । स्वाहा॑ जुषा॒णः । जु॒षा॒णो अ॒प्तुः । अ॒प्तुराज्य॑स्य । आज्य॑स्य वेतु । वे॒तु॒ स्वाहा᳚ । स्वाहा॒ऽयम् । 
अ॒यम् नः॑ । नो॒ अ॒ग्निः । अ॒ग्निर् वरि॑वः । वरि॑वः कृणोतु । कृ॒णो॒त्व॒यम् । अ॒यम् मृधः॑ । मृधः॑ पु॒रः । पु॒र ए॑तु । ए॒तु॒ प्र॒भि॒न्दन्न् । प्र॒भि॒न्दन्निति॑ प्र - भि॒न्दन्न् ॥ अ॒यꣳ शत्रून्॑ । शत्रू᳚न् जयतु । ज॒य॒तु॒ जर्.हृ॑षाणः । जर्.हृ॑षाणो॒ऽयम् । अ॒यं ॅवाज᳚म् । वाज॑म् जयतु । ज॒य॒तु॒ वाज॑सातौ । वाज॑साता॒विति॒ वाज॑ - सा॒तौ॒ ॥ उ॒रु वि॑ष्णो । वि॒ष्णो॒ वि । वि॒ष्णो॒ इति॑ विष्णो । वि क्र॑मस्व । क्र॒म॒स्वो॒रु । उ॒रु क्षया॑य । क्षया॑य नः । नः॒ कृ॒धि॒ । कृ॒धीति॑ कृधि ॥ घृ॒तम् घृ॑तयोने । घृ॒त॒यो॒ने॒ पि॒ब॒ । घृ॒त॒यो॒न॒ इति॑ घृत - यो॒ने॒ । पि॒ब॒ प्रप्र॑ । प्रप्र॑ य॒ज्ञ्प॑तिम् । प्रप्रेति॒ प्र - प्र॒ । य॒ज्ञ्प॑तिम् तिर । य॒ज्ञ्प॑ति॒मिति॑ य॒ज्ञ् - प॒ति॒म् । ति॒रेति॑ तिर ॥ सोमो॑ जिगाति । जि॒गा॒ति॒ गा॒तु॒वित् । गा॒तु॒विद् दे॒वाना᳚म् । गा॒तु॒विदिति॑ गातु - वित् \newline

\textbf{Jatai Paata} \newline

1. त्वꣳ सो॑म सोम॒ त्वम् त्वꣳ सो॑म । \newline
2. सो॒म॒ त॒नू॒कृद्भ्य॑ स्तनू॒कृद्भ्यः॑ सोम सोम तनू॒कृद्भ्यः॑ । \newline
3. त॒नू॒कृद्भ्यो॒ द्वेषो᳚भ्यो॒ द्वेषो᳚भ्य स्तनू॒कृद्भ्य॑ स्तनू॒कृद्भ्यो॒ द्वेषो᳚भ्यः । \newline
4. त॒नू॒कृद्भ्य॒ इति॑ तनू॒कृत् - भ्यः॒ । \newline
5. द्वेषो᳚भ्यो॒ ऽन्यकृ॑तेभ्यो॒ ऽन्यकृ॑तेभ्यो॒ द्वेषो᳚भ्यो॒ द्वेषो᳚भ्यो॒ ऽन्यकृ॑तेभ्यः । \newline
6. द्वेषो᳚भ्य॒ इति॒ द्वेषः॑ - भ्यः॒ । \newline
7. अ॒न्यकृ॑तेभ्य उ॒रू᳚(1॒)र्व॑न्यकृ॑तेभ्यो॒ ऽन्यकृ॑तेभ्य उ॒रु । \newline
8. अ॒न्यकृ॑तेभ्य॒ इत्य॒न्य - कृ॒ते॒भ्यः॒ । \newline
9. उ॒रु य॒न्ता य॒न्तो रू॑रु य॒न्ता । \newline
10. य॒न्ता ऽस्य॑सि य॒न्ता य॒न्ता ऽसि॑ । \newline
11. अ॒सि॒ वरू॑थं॒ ॅवरू॑थ मस्यसि॒ वरू॑थम् । \newline
12. वरू॑थ॒(ग्ग्॒) स्वाहा॒ स्वाहा॒ वरू॑थं॒ ॅवरू॑थ॒(ग्ग्॒) स्वाहा᳚ । \newline
13. स्वाहा॑ जुषा॒णो जु॑षा॒णः स्वाहा॒ स्वाहा॑ जुषा॒णः । \newline
14. जु॒षा॒णो अ॒प्तु र॒प्तुर् जु॑षा॒णो जु॑षा॒णो अ॒प्तुः । \newline
15. अ॒प्तु राज्य॒ स्याज्य॑ स्या॒प्तु र॒प्तु राज्य॑स्य । \newline
16. आज्य॑स्य वेतु वे॒त्वाज्य॒ स्याज्य॑स्य वेतु । \newline
17. वे॒तु॒ स्वाहा॒ स्वाहा॑ वेतु वेतु॒ स्वाहा᳚ । \newline
18. स्वाहा॒ ऽय म॒यꣳ स्वाहा॒ स्वाहा॒ ऽयं । \newline
19. अ॒यम् नो॑ नो॒ ऽयं अ॒यम् नः॑ । \newline
20. नो॒ अ॒ग्नि र॒ग्निर् नो॑ नो अ॒ग्निः । \newline
21. अ॒ग्निर् वरि॑वो॒ वरि॑वो॒ ऽग्नि र॒ग्निर् वरि॑वः । \newline
22. वरि॑वः कृणोतु कृणोतु॒ वरि॑वो॒ वरि॑वः कृणोतु । \newline
23. कृ॒णो॒त्व॒य म॒यम् कृ॑णोतु कृणोत्व॒यम् । \newline
24. अ॒यम् मृधो॒ मृधो॒ ऽय म॒यम् मृधः॑ । \newline
25. मृधः॑ पु॒रः पु॒रो मृधो॒ मृधः॑ पु॒रः । \newline
26. पु॒र ए᳚त्वेतु पु॒रः पु॒र ए॑तु । \newline
27. ए॒तु॒ प्र॒भि॒न्दन् प्र॑भि॒न्दन् ने᳚त्वेतु प्रभि॒न्दन्न् ॥ \newline
28. प्र॒भि॒न्दन्निति॑ प्र - भि॒न्दन्न् ॥ \newline
29. अ॒यꣳ शत्रू॒ञ् छत्रू॑ न॒य म॒यꣳ शत्रून्॑ । \newline
30. शत्रू᳚न् जयतु जयतु॒ शत्रू॒ञ् छत्रू᳚न् जयतु । \newline
31. ज॒य॒तु॒ जर्.हृ॑षाणो॒ जर्.हृ॑षाणो जयतु जयतु॒ जर्.हृ॑षाणः । \newline
32. जर्.हृ॑षाणो॒ ऽय म॒यम् जर्.हृ॑षाणो॒ जर्.हृ॑षाणो॒ ऽयम् । \newline
33. अ॒यं ॅवाजं॒ ॅवाज॑ म॒य म॒यं ॅवाज᳚म् । \newline
34. वाज॑म् जयतु जयतु॒ वाजं॒ ॅवाज॑म् जयतु । \newline
35. ज॒य॒तु॒ वाज॑सातौ॒ वाज॑सातौ जयतु जयतु॒ वाज॑सातौ । \newline
36. वाज॑साता॒विति॒ वाज॑ - सा॒तौ॒ । \newline
37. उ॒रु वि॑ष्णो विष्णो उ॒रू॑रु वि॑ष्णो । \newline
38. वि॒ष्णो॒ वि वि वि॑ष्णो विष्णो॒ वि । \newline
39. वि॒ष्णो॒ इति॑ विष्णो । \newline
40. वि क्र॑मस्व क्रमस्व॒ वि वि क्र॑मस्व । \newline
41. क्र॒म॒स्वो॒ रू॑रु क्र॑मस्व क्रम स्वो॒रु । \newline
42. उ॒रु क्षया॑य॒ क्षया॑यो॒ रू॑रु क्षया॑य । \newline
43. क्षया॑य नो नः॒ क्षया॑य॒ क्षया॑य नः । \newline
44. नः॒ कृ॒धि॒ कृ॒धि॒ नो॒ नः॒ कृ॒धि॒ । \newline
45. कृ॒धीति॑ कृधि । \newline
46. घृ॒तम् घृ॑तयोने घृतयोने घृ॒तम् घृ॒तम् घृ॑तयोने । \newline
47. घृ॒त॒यो॒ने॒ पि॒ब॒ पि॒ब॒ घृ॒त॒यो॒ने॒ घृ॒त॒यो॒ने॒ पि॒ब॒ । \newline
48. घृ॒त॒यो॒न॒ इति॑ घृत - यो॒ने॒ । \newline
49. पि॒ब॒ प्रप्र॒ प्रप्र॑ पिब पिब॒ प्रप्र॑ । \newline
50. प्रप्र॑ य॒ज्ञ्प॑तिं ॅय॒ज्ञ्प॑ति॒म् प्रप्र॒ प्रप्र॑ य॒ज्ञ्प॑तिम् । \newline
51. प्रप्रेति॒ प्र - प्र॒ । \newline
52. य॒ज्ञ्प॑तिम् तिर तिर य॒ज्ञ्प॑तिं ॅय॒ज्ञ्प॑तिम् तिर । \newline
53. य॒ज्ञ्प॑ति॒मिति॑ य॒ज्ञ् - प॒ति॒म् । \newline
54. ति॒रेति॑ तिर । \newline
55. सोमो॑ जिगाति जिगाति॒ सोमः॒ सोमो॑ जिगाति । \newline
56. जि॒गा॒ति॒ गा॒तु॒विद् गा॑तु॒विज् जि॑गाति जिगाति गातु॒वित् । \newline
57. गा॒तु॒विद् दे॒वाना᳚म् दे॒वाना᳚म् गातु॒विद् गा॑तु॒विद् दे॒वाना᳚म् । \newline
58. गा॒तु॒विदिति॑ गातु - वित् । \newline

\textbf{Ghana Paata } \newline

1. त्वꣳ सो॑म सोम॒ त्वम् त्वꣳ सो॑म तनू॒कृद्भ्य॑ स्तनू॒कृद्भ्यः॑ सोम॒ त्वम् त्वꣳ सो॑म तनू॒कृद्भ्यः॑ । \newline
2. सो॒म॒ त॒नू॒कृद्भ्य॑ स्तनू॒कृद्भ्यः॑ सोम सोम तनू॒कृद्भ्यो॒ द्वेषो᳚भ्यो॒ द्वेषो᳚भ्य स्तनू॒कृद्भ्यः॑ सोम सोम तनू॒कृद्भ्यो॒ द्वेषो᳚भ्यः । \newline
3. त॒नू॒कृद्भ्यो॒ द्वेषो᳚भ्यो॒ द्वेषो᳚भ्य स्तनू॒कृद्भ्य॑ स्तनू॒कृद्भ्यो॒ द्वेषो᳚भ्यो॒ ऽन्यकृ॑तेभ्यो॒ ऽन्यकृ॑तेभ्यो॒ द्वेषो᳚भ्य स्तनू॒कृद्भ्य॑ स्तनू॒कृद्भ्यो॒ द्वेषो᳚भ्यो॒ ऽन्यकृ॑तेभ्यः । \newline
4. त॒नू॒कृद्भ्य॒ इति॑ तनू॒कृत् - भ्यः॒ । \newline
5. द्वेषो᳚भ्यो॒ ऽन्यकृ॑तेभ्यो॒ ऽन्यकृ॑तेभ्यो॒ द्वेषो᳚भ्यो॒ द्वेषो᳚भ्यो॒ ऽन्यकृ॑तेभ्य उ॒रू᳚(1॒)र्व॑न्यकृ॑तेभ्यो॒ द्वेषो᳚भ्यो॒ द्वेषो᳚भ्यो॒ ऽन्यकृ॑तेभ्य उ॒रु । \newline
6. द्वेषो᳚भ्य॒ इति॒ द्वेषः॑ - भ्यः॒ । \newline
7. अ॒न्यकृ॑तेभ्य उ॒रू᳚(1॒)र्व॑ न्यकृ॑तेभ्यो॒ ऽन्यकृ॑तेभ्य उ॒रु य॒न्ता य॒न्तोर्व॑ न्यकृ॑तेभ्यो॒ ऽन्यकृ॑तेभ्य उ॒रु य॒न्ता । \newline
8. अ॒न्यकृ॑तेभ्य॒ इत्य॒न्य - कृ॒ते॒भ्यः॒ । \newline
9. उ॒रु य॒न्ता य॒न्तो रू॑रु य॒न्ता ऽस्य॑सि य॒न्तो रू॑रु य॒न्ता ऽसि॑ । \newline
10. य॒न्ता ऽस्य॑सि य॒न्ता य॒न्ता ऽसि॒ वरू॑थं॒ ॅवरू॑थ मसि य॒न्ता य॒न्ता ऽसि॒ वरू॑थम् । \newline
11. अ॒सि॒ वरू॑थं॒ ॅवरू॑थ मस्यसि॒ वरू॑थ॒(ग्ग्॒) स्वाहा॒ स्वाहा॒ वरू॑थ मस्यसि॒ वरू॑थ॒(ग्ग्॒) स्वाहा᳚ । \newline
12. वरू॑थ॒(ग्ग्॒) स्वाहा॒ स्वाहा॒ वरू॑थं॒ ॅवरू॑थ॒(ग्ग्॒) स्वाहा॑ जुषा॒णो जु॑षा॒णः स्वाहा॒ वरू॑थं॒ ॅवरू॑थ॒(ग्ग्॒) स्वाहा॑ जुषा॒णः । \newline
13. स्वाहा॑ जुषा॒णो जु॑षा॒णः स्वाहा॒ स्वाहा॑ जुषा॒णो अ॒प्तु र॒प्तुर् जु॑षा॒णः स्वाहा॒ स्वाहा॑ जुषा॒णो अ॒प्तुः । \newline
14. जु॒षा॒णो अ॒प्तु र॒प्तुर् जु॑षा॒णो जु॑षा॒णो अ॒प्तु राज्य॒स्याज्य॑ स्या॒प्तुर् जु॑षा॒णो जु॑षा॒णो अ॒प्तु राज्य॑स्य । \newline
15. अ॒प्तु राज्य॒स्या ज्य॑स्या॒प्तु र॒प्तुराज्य॑स्य वेतु वे॒त्वा ज्य॑स्या॒प्तु र॒प्तु राज्य॑स्य वेतु । \newline
16. आज्य॑स्य वेतु वे॒त्वाज्य॒ स्याज्य॑स्य वेतु॒ स्वाहा॒ स्वाहा॑ वे॒त्वाज्य॒ स्याज्य॑स्य वेतु॒ स्वाहा᳚ । \newline
17. वे॒तु॒ स्वाहा॒ स्वाहा॑ वेतु वेतु॒ स्वाहा॒ ऽय म॒यꣳ स्वाहा॑ वेतु वेतु॒ स्वाहा॒ ऽयं । \newline
18. स्वाहा॒ ऽय म॒यꣳ स्वाहा॒ स्वाहा॒ ऽयम् नो॑ नो॒ ऽयꣳ स्वाहा॒ स्वाहा॒ ऽयम् नः॑ । \newline
19. अ॒यम् नो॑ नो॒ ऽय म॒यम् नो॑ अ॒ग्नि र॒ग्निर् नो॒ ऽय म॒यम् नो॑ अ॒ग्निः । \newline
20. नो॒ अ॒ग्नि र॒ग्निर् नो॑ नो अ॒ग्निर् वरि॑वो॒ वरि॑वो॒ ऽग्निर् नो॑ नो अ॒ग्निर् वरि॑वः । \newline
21. अ॒ग्निर् वरि॑वो॒ वरि॑वो॒ ऽग्नि र॒ग्निर् वरि॑वः कृणोतु कृणोतु॒ वरि॑वो॒ ऽग्नि र॒ग्निर् वरि॑वः कृणोतु । \newline
22. वरि॑वः कृणोतु कृणोतु॒ वरि॑वो॒ वरि॑वः कृणोत्व॒य म॒यम् कृ॑णोतु॒ वरि॑वो॒ वरि॑वः कृणोत्व॒यम् । \newline
23. कृ॒णो॒त्व॒य म॒यम् कृ॑णोतु कृणोत्व॒यम् मृधो॒ मृधो॒ ऽयम् कृ॑णोतु कृणोत्व॒यम् मृधः॑ । \newline
24. अ॒यम् मृधो॒ मृधो॒ ऽय म॒यम् मृधः॑ पु॒रः पु॒रो मृधो॒ ऽय म॒यम् मृधः॑ पु॒रः । \newline
25. मृधः॑ पु॒रः पु॒रो मृधो॒ मृधः॑ पु॒र ए᳚त्वेतु पु॒रो मृधो॒ मृधः॑ पु॒र ए॑तु । \newline
26. पु॒र ए᳚त्वेतु पु॒रः पु॒र ए॑तु प्रभि॒न्दन् प्र॑भि॒न्दन् ने॑तु पु॒रः पु॒र ए॑तु प्रभि॒न्दन्न् । \newline
27. ए॒तु॒ प्र॒भि॒न्दन् प्र॑भि॒न्दन् ने᳚त्वेतु प्रभि॒न्दन्न् । \newline
28. प्र॒भि॒न्दन्निति॑ प्र - भि॒न्दन्न् । \newline
29. अ॒यꣳ शत्रू॒ञ् छत्रू॑ न॒य म॒यꣳ शत्रू᳚न् जयतु जयतु॒ शत्रू॑ न॒य म॒यꣳ शत्रू᳚न् जयतु । \newline
30. शत्रू᳚न् जयतु जयतु॒ शत्रू॒ञ् छत्रू᳚न् जयतु॒ जर्.हृ॑षाणो॒ जर्.हृ॑षाणो जयतु॒ शत्रू॒ञ् छत्रू᳚न् जयतु॒ जर्.हृ॑षाणः । \newline
31. ज॒य॒तु॒ जर्.हृ॑षाणो॒ जर्.हृ॑षाणो जयतु जयतु॒ जर्.हृ॑षाणो॒ ऽय म॒यम् जर्.हृ॑षाणो जयतु जयतु॒ जर्.हृ॑षाणो॒ ऽयम् । \newline
32. जर्हृ॑षाणो॒ ऽय म॒यम् जर्.हृ॑षाणो॒ जर्.हृ॑षाणो॒ ऽयं ॅवाजं॒ ॅवाज॑ म॒यम् जर्.हृ॑षाणो॒ जर्.हृ॑षाणो॒ ऽयं ॅवाज᳚म् । \newline
33. अ॒यं ॅवाजं॒ ॅवाज॑ म॒य म॒यं ॅवाज॑म् जयतु जयतु॒ वाज॑ म॒य म॒यं ॅवाज॑म् जयतु । \newline
34. वाज॑म् जयतु जयतु॒ वाजं॒ ॅवाज॑म् जयतु॒ वाज॑सातौ॒ वाज॑सातौ जयतु॒ वाजं॒ ॅवाज॑म् जयतु॒ वाज॑सातौ । \newline
35. ज॒य॒तु॒ वाज॑सातौ॒ वाज॑सातौ जयतु जयतु॒ वाज॑सातौ । \newline
36. वाज॑साता॒विति॒ वाज॑ - सा॒तौ॒ । \newline
37. उ॒रु वि॑ष्णो विष्णो उ॒रू॑रु वि॑ष्णो॒ वि वि वि॑ष्णो उ॒रू॑रु वि॑ष्णो॒ वि । \newline
38. वि॒ष्णो॒ वि वि वि॑ष्णो विष्णो॒ वि क्र॑मस्व क्रमस्व॒ वि वि॑ष्णो विष्णो॒ वि क्र॑मस्व । \newline
39. वि॒ष्णो॒ इति॑ विष्णो । \newline
40. वि क्र॑मस्व क्रमस्व॒ वि वि क्र॑मस्वो॒ रू॑रु क्र॑मस्व॒ वि वि क्र॑मस्वो॒रु । \newline
41. क्र॒म॒स्वो॒ रू॑रु क्र॑मस्व क्रमस्वो॒रु क्षया॑य॒ क्षया॑यो॒रु क्र॑मस्व क्रमस्वो॒रु क्षया॑य । \newline
42. उ॒रु क्षया॑य॒ क्षया॑यो॒ रू॑रु क्षया॑य नो नः॒ क्षया॑यो॒ रू॑रु क्षया॑य नः । \newline
43. क्षया॑य नो नः॒ क्षया॑य॒ क्षया॑य नः कृधि कृधि नः॒ क्षया॑य॒ क्षया॑य नः कृधि । \newline
44. नः॒ कृ॒धि॒ कृ॒धि॒ नो॒ नः॒ कृ॒धि॒ । \newline
45. कृ॒धीति॑ कृधि । \newline
46. घृ॒तम् घृ॑तयोने घृतयोने घृ॒तम् घृ॒तम् घृ॑तयोने पिब पिब घृतयोने घृ॒तम् घृ॒तम् घृ॑तयोने पिब । \newline
47. घृ॒त॒यो॒ने॒ पि॒ब॒ पि॒ब॒ घृ॒त॒यो॒ने॒ घृ॒त॒यो॒ने॒ पि॒ब॒ प्रप्र॒ प्रप्र॑ पिब घृतयोने घृतयोने पिब॒ प्रप्र॑ । \newline
48. घृ॒त॒यो॒न॒ इति॑ घृत - यो॒ने॒ । \newline
49. पि॒ब॒ प्रप्र॒ प्रप्र॑ पिब पिब॒ प्रप्र॑ य॒ज्ञ्प॑तिं ॅय॒ज्ञ्प॑ति॒म् प्रप्र॑ पिब पिब॒ प्रप्र॑ य॒ज्ञ्प॑तिम् । \newline
50. प्रप्र॑ य॒ज्ञ्प॑तिं ॅय॒ज्ञ्प॑ति॒म् प्रप्र॒ प्रप्र॑ य॒ज्ञ्प॑तिम् तिर तिर य॒ज्ञ्प॑ति॒म् प्रप्र॒ प्रप्र॑ य॒ज्ञ्प॑तिम् तिर । \newline
51. प्रप्रेति॒ प्र - प्र॒ । \newline
52. य॒ज्ञ्प॑तिम् तिर तिर य॒ज्ञ्प॑तिं ॅय॒ज्ञ्प॑तिम् तिर । \newline
53. य॒ज्ञ्प॑ति॒मिति॑ य॒ज्ञ् - प॒ति॒म् । \newline
54. ति॒रेति॑ तिर । \newline
55. सोमो॑ जिगाति जिगाति॒ सोमः॒ सोमो॑ जिगाति गातु॒विद् गा॑तु॒विज् जि॑गाति॒ सोमः॒ सोमो॑ जिगाति गातु॒वित् । \newline
56. जि॒गा॒ति॒ गा॒तु॒विद् गा॑तु॒विज् जि॑गाति जिगाति गातु॒विद् दे॒वाना᳚म् दे॒वाना᳚म् गातु॒विज् जि॑गाति जिगाति गातु॒विद् दे॒वाना᳚म् । \newline
57. गा॒तु॒विद् दे॒वाना᳚म् दे॒वाना᳚म् गातु॒विद् गा॑तु॒विद् दे॒वाना॑ मेत्येति दे॒वाना᳚म् गातु॒विद् गा॑तु॒विद् दे॒वाना॑ मेति । \newline
58. गा॒तु॒विदिति॑ गातु - वित् । \newline
\pagebreak
\markright{ TS 1.3.4.2  \hfill https://www.vedavms.in \hfill}

\section{ TS 1.3.4.2 }

\textbf{TS 1.3.4.2 } \newline
\textbf{Samhita Paata} \newline

दे॒वाना॑मेति निष्कृ॒तमृ॒तस्य॒ योनि॑मा॒सद॒मदि॑त्याः॒ सदो॒ऽस्यदि॑त्याः॒ सद॒ आ सी॑दै॒ष वो॑ देव सवितः॒ सोम॒स्तꣳ र॑क्षद्ध्वं॒ मा वो॑ दभदे॒तत्त्वꣳ सो॑म दे॒वो दे॒वानुपा॑गा इ॒दम॒हं म॑नु॒ष्यो॑ मनु॒ष्या᳚न्थ् स॒ह प्र॒जया॑ स॒ह रा॒यस्पोषे॑ण॒ नमो॑ दे॒वेभ्यः॑ स्व॒धा पि॒तृभ्य॑ इ॒दम॒हं निर्वरु॑णस्य॒ पाशा॒थ् सुव॑र॒भि - [ ] \newline

\textbf{Pada Paata} \newline

दे॒वाना᳚म् । ए॒ति॒ । नि॒ष्कृ॒तमिति॑ निः - कृ॒तम् । ऋ॒तस्य॑ । योनि᳚म् । आ॒सद॒मित्या᳚ - सद᳚म् । अदि॑त्याः । सदः॑ । अ॒सि॒ । अदि॑त्याः । सदः॑ । एति॑ । सी॒द॒ । ए॒षः । वः॒ । दे॒व॒ । स॒वि॒तः॒ । सोमः॑ । तम् । र॒क्ष॒द्ध्व॒म् । मा । वः॒ । द॒भ॒॒त् । ए॒तत् । त्वम् । सो॒म॒ । दे॒वः । दे॒वान् । उपेति॑ । अ॒गाः॒ । इ॒दम् । अ॒हम् । म॒नु॒ष्यः॑ । म॒नु॒ष्यान्॑ । स॒ह । प्र॒जयेति॑ प्र - जया᳚ । स॒ह । रा॒यः । पोषे॑ण । नमः॑ । दे॒वेभ्यः॑ । स्व॒धेति॑ स्व - धा । पि॒तृभ्य॒ इति॑ पि॒तृ - भ्यः॒ । इ॒दम् । अ॒हम् । निरिति॑ । वरु॑णस्य । पाशा᳚त् । सुवः॑ । अ॒भि ।  \newline


\textbf{Krama Paata} \newline

दे॒वाना॑मेति । ए॒ति॒ नि॒ष्कृ॒तम् । नि॒ष्कृ॒तमृ॒तस्य॑ । नि॒ष्कृ॒तमिति॑ निः - कृ॒तम् । ऋ॒तस्य॒ योनि᳚म् । योनि॑मा॒सद᳚म् । आ॒सद॒मदि॑त्याः । आ॒सद॒मित्या᳚ - सद᳚म् । अदि॑त्याः॒ सदः॑ । सदो॑ऽसि । अ॒स्यदि॑त्याः । अदि॑त्याः॒ सदः॑ । सद॒ आ । आ सी॑द । सी॒दै॒षः । ए॒ष वः॑ । वो॒ दे॒व॒ । दे॒व॒ स॒वि॒तः॒ । स॒वि॒तः॒ सोमः॑ । सोम॒स्तम् । तꣳ र॑क्षद्ध्वम् । र॒क्ष॒द्ध्व॒म् मा । मा वः॑ । वो॒ द॒भ॒त् । द॒भ॒दे॒तत् । ए॒तत् त्वम् । त्वꣳ सो॑म । सो॒म॒ दे॒वः । दे॒वो दे॒वान् । दे॒वानुप॑ । उपा॑गाः । अ॒गा॒ इ॒दम् । इ॒दम॒हम् । अ॒हम् म॑नु॒ष्यः॑ । म॒नु॒ष्यो॑ मनु॒ष्यान्॑ । म॒नु॒ष्या᳚न्थ् स॒ह । स॒ह प्र॒जया᳚ । प्र॒जया॑ स॒ह । प्र॒जयेति॑ प्र - जया᳚ । स॒ह रा॒यः । रा॒यस्पोषे॑ण । पोषे॑ण॒ नमः॑ । नमो॑ दे॒वेभ्यः॑ । दे॒वेभ्यः॑ स्व॒धा । स्व॒धा पि॒तृभ्यः॑ । स्व॒धेति॑ स्व - धा । पि॒तृभ्य॑ इ॒दम् । पि॒तृभ्य॒ इति॑ पि॒तृ - भ्यः॒ । इ॒दम॒हम् । अ॒हम् निः । निर् वरु॑णस्य । वरु॑णस्य॒ पाशा᳚त् । पाशा॒थ् सुवः॑ । सुव॑र॒भि ( ) । अ॒भि वि \newline

\textbf{Jatai Paata} \newline

1. दे॒वाना॑ मेत्येति दे॒वाना᳚म् दे॒वाना॑ मेति । \newline
2. ए॒ति॒ नि॒ष्कृ॒तम् नि॑ष्कृ॒त मे᳚त्येति निष्कृ॒तम् । \newline
3. नि॒ष्कृ॒त मृ॒तस्य॒ र्तस्य॑ निष्कृ॒तम् नि॑ष्कृ॒त मृ॒तस्य॑ । \newline
4. नि॒ष्कृ॒तमिति॑ निः - कृ॒तम् । \newline
5. ऋ॒तस्य॒ योनिं॒ ॅयोनि॑ मृ॒तस्य॒ र्तस्य॒ योनि᳚म् । \newline
6. योनि॑ मा॒सद॑ मा॒सदं॒ ॅयोनिं॒ ॅयोनि॑ मा॒सद᳚म् । \newline
7. आ॒सद॒ मदि॑त्या॒ अदि॑त्या आ॒सद॑ मा॒सद॒ मदि॑त्याः । \newline
8. आ॒सद॒मित्या᳚ - सद᳚म् । \newline
9. अदि॑त्याः॒ सदः॒ सदो ऽदि॑त्या॒ अदि॑त्याः॒ सदः॑ । \newline
10. सदो᳚ ऽस्यसि॒ सदः॒ सदो॑ ऽसि । \newline
11. अ॒स्यदि॑त्या॒ अदि॑त्या अस्य॒ स्यदि॑त्याः । \newline
12. अदि॑त्याः॒ सदः॒ सदो ऽदि॑त्या॒ अदि॑त्याः॒ सदः॑ । \newline
13. सद॒ आ सदः॒ सद॒ आ । \newline
14. आ सी॑द सी॒दा सी॑द । \newline
15. सी॒दै॒ष ए॒ष सी॑द सीदै॒षः । \newline
16. ए॒ष वो॑ व ए॒ष ए॒ष वः॑ । \newline
17. वो॒ दे॒व॒ दे॒व॒ वो॒ वो॒ दे॒व॒ । \newline
18. दे॒व॒ स॒वि॒तः॒ स॒वि॒त॒र् दे॒व॒ दे॒व॒ स॒वि॒तः॒ । \newline
19. स॒वि॒तः॒ सोमः॒ सोमः॑ सवितः सवितः॒ सोमः॑ । \newline
20. सोम॒ स्तम् तꣳ सोमः॒ सोम॒ स्तम् । \newline
21. तꣳ र॑क्षद्ध्वꣳ रक्षद्ध्व॒म् तम् तꣳ र॑क्षद्ध्वम् । \newline
22. र॒क्ष॒द्ध्व॒म् मा मा र॑क्षद्ध्वꣳ रक्षद्ध्व॒म् मा । \newline
23. मा वो॑ वो॒ मा मा वः॑ । \newline
24. वो॒ द॒भ॒द् द॒भ॒द् वो॒ वो॒ द॒भ॒त् । \newline
25. द॒भ॒ दे॒त दे॒तद् द॑भद् दभ दे॒तत् । \newline
26. ए॒तत् त्वम् त्व मे॒त दे॒तत् त्वम् । \newline
27. त्वꣳ सो॑म सोम॒ त्वम् त्वꣳ सो॑म । \newline
28. सो॒म॒ दे॒वो दे॒वः सो॑म सोम दे॒वः । \newline
29. दे॒वो दे॒वान् दे॒वान् दे॒वो दे॒वो दे॒वान् । \newline
30. दे॒वा नुपोप॑ दे॒वान् दे॒वा नुप॑ । \newline
31. उपा॑गा अगा॒ उपोपा॑गाः । \newline
32. अ॒गा॒ इ॒द मि॒द म॑गा अगा इ॒दम् । \newline
33. इ॒द म॒ह म॒ह मि॒द मि॒द म॒हम् । \newline
34. अ॒हम् म॑नु॒ष्यो॑ मनु॒ष्यो॑ ऽह म॒हम् म॑नु॒ष्यः॑ । \newline
35. म॒नु॒ष्यो॑ मनु॒ष्या᳚न् मनु॒ष्या᳚न् मनु॒ष्यो॑ मनु॒ष्यो॑ मनु॒ष्यान्॑ । \newline
36. म॒नु॒ष्या᳚न् थ्स॒ह स॒ह म॑नु॒ष्या᳚न् मनु॒ष्या᳚न् थ्स॒ह । \newline
37. स॒ह प्र॒जया᳚ प्र॒जया॑ स॒ह स॒ह प्र॒जया᳚ । \newline
38. प्र॒जया॑ स॒ह स॒ह प्र॒जया᳚ प्र॒जया॑ स॒ह । \newline
39. प्र॒जयेति॑ प्र - जया᳚ । \newline
40. स॒ह रा॒यो रा॒यः स॒ह स॒ह रा॒यः । \newline
41. रा॒यस् पोषे॑ण॒ पोषे॑ण रा॒यो रा॒यस् पोषे॑ण । \newline
42. पोषे॑ण॒ नमो॒ नमः॒ पोषे॑ण॒ पोषे॑ण॒ नमः॑ । \newline
43. नमो॑ दे॒वेभ्यो॑ दे॒वेभ्यो॒ नमो॒ नमो॑ दे॒वेभ्यः॑ । \newline
44. दे॒वेभ्यः॑ स्व॒धा स्व॒धा दे॒वेभ्यो॑ दे॒वेभ्यः॑ स्व॒धा । \newline
45. स्व॒धा पि॒तृभ्यः॑ पि॒तृभ्यः॑ स्व॒धा स्व॒धा पि॒तृभ्यः॑ । \newline
46. स्व॒धेति॑ स्व - धा । \newline
47. पि॒तृभ्य॑ इ॒द मि॒दम् पि॒तृभ्यः॑ पि॒तृभ्य॑ इ॒दम् । \newline
48. पि॒तृभ्य॒ इति॑ पि॒तृ - भ्यः॒ । \newline
49. इ॒द म॒ह म॒ह मि॒द मि॒द म॒हम् । \newline
50. अ॒हम् निर् णिर॒ह म॒हम् निः । \newline
51. निर् वरु॑णस्य॒ वरु॑णस्य॒ निर् णिर् वरु॑णस्य । \newline
52. वरु॑णस्य॒ पाशा॒त् पाशा॒द् वरु॑णस्य॒ वरु॑णस्य॒ पाशा᳚त् । \newline
53. पाशा॒ थ्सुवः॒ सुवः॒ पाशा॒त् पाशा॒थ् सुवः॑ । \newline
54. सुव॑ र॒भ्य॑भि सुवः॒ सुव॑ र॒भि । \newline
55. अ॒भि वि व्या᳚(1॒)भ्य॑भि वि । \newline

\textbf{Ghana Paata } \newline

1. दे॒वाना॑ मेत्येति दे॒वाना᳚म् दे॒वाना॑ मेति निष्कृ॒तं नि॑ष्कृ॒त मे॑ति दे॒वाना᳚म् दे॒वाना॑ मेति निष्कृ॒तम् । \newline
2. ए॒ति॒ नि॒ष्कृ॒तम् नि॑ष्कृ॒त मे᳚त्येति निष्कृ॒त मृ॒तस्य॒ र्तस्य॑ निष्कृ॒त मे᳚त्येति निष्कृ॒त मृ॒तस्य॑ । \newline
3. नि॒ष्कृ॒त मृ॒तस्य॒ र्तस्य॑ निष्कृ॒तम् नि॑ष्कृ॒त मृ॒तस्य॒ योनिं॒ ॅयोनि॑ मृ॒तस्य॑ निष्कृ॒तम् नि॑ष्कृ॒त मृ॒तस्य॒ योनि᳚म् । \newline
4. नि॒ष्कृ॒तमिति॑ निः - कृ॒तम् । \newline
5. ऋ॒तस्य॒ योनिं॒ ॅयोनि॑ मृ॒तस्य॒ र्तस्य॒ योनि॑ मा॒सद॑ मा॒सदं॒ ॅयोनि॑ मृ॒तस्य॒ र्तस्य॒ योनि॑ मा॒सद᳚म् । \newline
6. योनि॑ मा॒सद॑ मा॒सदं॒ ॅयोनिं॒ ॅयोनि॑ मा॒सद॒ मदि॑त्या॒ अदि॑त्या आ॒सदं॒ ॅयोनिं॒ ॅयोनि॑ मा॒सद॒ मदि॑त्याः । \newline
7. आ॒सद॒ मदि॑त्या॒ अदि॑त्या आ॒सद॑ मा॒सद॒ मदि॑त्याः॒ सदः॒ सदो ऽदि॑त्या आ॒सद॑ मा॒सद॒ मदि॑त्याः॒ सदः॑ । \newline
8. आ॒सद॒मित्या᳚ - सद᳚म् । \newline
9. अदि॑त्याः॒ सदः॒ सदो ऽदि॑त्या॒ अदि॑त्याः॒ सदो᳚ ऽस्यसि॒ सदो ऽदि॑त्या॒ अदि॑त्याः॒ सदो॑ ऽसि । \newline
10. सदो᳚ ऽस्यसि॒ सदः॒ सदो॒ ऽस्यदि॑त्या॒ अदि॑त्या असि॒ सदः॒ सदो॒ ऽस्यदि॑त्याः । \newline
11. अ॒स्यदि॑त्या॒ अदि॑त्या अस्य॒ स्यदि॑त्याः॒ सदः॒ सदो ऽदि॑त्या अस्य॒ स्यदि॑त्याः॒ सदः॑ । \newline
12. अदि॑त्याः॒ सदः॒ सदो ऽदि॑त्या॒ अदि॑त्याः॒ सद॒ आ सदो ऽदि॑त्या॒ अदि॑त्याः॒ सद॒ आ । \newline
13. सद॒ आ सदः॒ सद॒ आ सी॑द सी॒दा सदः॒ सद॒ आ सी॑द । \newline
14. आ सी॑द सी॒दा सी॑दै॒ष ए॒ष सी॒दा सी॑दै॒षः । \newline
15. सी॒दै॒ष ए॒ष सी॑द सीदै॒ष वो॑ व ए॒ष सी॑द सीदै॒ष वः॑ । \newline
16. ए॒ष वो॑ व ए॒ष ए॒ष वो॑ देव देव व ए॒ष ए॒ष वो॑ देव । \newline
17. वो॒ दे॒व॒ दे॒व॒ वो॒ वो॒ दे॒व॒ स॒वि॒तः॒ स॒वि॒त॒र् दे॒व॒ वो॒ वो॒ दे॒व॒ स॒वि॒तः॒ । \newline
18. दे॒व॒ स॒वि॒तः॒ स॒वि॒त॒र् दे॒व॒ दे॒व॒ स॒वि॒तः॒ सोमः॒ सोमः॑ सवितर् देव देव सवितः॒ सोमः॑ । \newline
19. स॒वि॒तः॒ सोमः॒ सोमः॑ सवितः सवितः॒ सोम॒स्तम् तꣳ सोमः॑ सवितः सवितः॒ सोम॒स्तम् । \newline
20. सोम॒स्तम् तꣳ सोमः॒ सोम॒स्तꣳ र॑क्षद्ध्वꣳ रक्षद्ध्व॒म् तꣳ सोमः॒ सोम॒स्तꣳ र॑क्षद्ध्वम् । \newline
21. तꣳ र॑क्षद्ध्वꣳ रक्षद्ध्व॒म् तम् तꣳ र॑क्षद्ध्व॒म् मा मा र॑क्षद्ध्व॒म् तम् तꣳ र॑क्षद्ध्व॒म् मा । \newline
22. र॒क्ष॒द्ध्व॒म् मा मा र॑क्षद्ध्वꣳ रक्षद्ध्व॒म् मा वो॑ वो॒ मा र॑क्षद्ध्वꣳ रक्षद्ध्व॒म् मा वः॑ । \newline
23. मा वो॑ वो॒ मा मा वो॑ दभद् दभद् वो॒ मा मा वो॑ दभत् । \newline
24. वो॒ द॒भ॒द् द॒भ॒द् वो॒ वो॒ द॒भ॒ दे॒त दे॒तद् द॑भद् वो वो दभ दे॒तत् । \newline
25. द॒भ॒ दे॒त दे॒तद् द॑भद् दभ दे॒तत् त्वम् त्व मे॒तद् द॑भद् दभ दे॒तत् त्वम् । \newline
26. ए॒तत् त्वम् त्व मे॒त दे॒तत् त्वꣳ सो॑म सोम॒ त्व मे॒त दे॒तत् त्वꣳ सो॑म । \newline
27. त्वꣳ सो॑म सोम॒ त्वम् त्वꣳ सो॑म दे॒वो दे॒वः सो॑म॒ त्वम् त्वꣳ सो॑म दे॒वः । \newline
28. सो॒म॒ दे॒वो दे॒वः सो॑म सोम दे॒वो दे॒वान् दे॒वान् दे॒वः सो॑म सोम दे॒वो दे॒वान् । \newline
29. दे॒वो दे॒वान् दे॒वान् दे॒वो दे॒वो दे॒वा नुपोप॑ दे॒वान् दे॒वो दे॒वो दे॒वा नुप॑ । \newline
30. दे॒वा नुपोप॑ दे॒वान् दे॒वा नुपा॑गा अगा॒ उप॑ दे॒वान् दे॒वा नुपा॑गाः । \newline
31. उपा॑गा अगा॒ उपो पा॑गा इ॒द मि॒द म॑गा॒ उपो पा॑गा इ॒दम् । \newline
32. अ॒गा॒ इ॒द मि॒द म॑गा अगा इ॒द म॒ह म॒ह मि॒द म॑गा अगा इ॒द म॒हम् । \newline
33. इ॒द म॒ह म॒ह मि॒द मि॒द म॒हम् म॑नु॒ष्यो॑ मनु॒ष्यो॑ ऽह मि॒द मि॒द म॒हम् म॑नु॒ष्यः॑ । \newline
34. अ॒हम् म॑नु॒ष्यो॑ मनु॒ष्यो॑ ऽह म॒हम् म॑नु॒ष्यो॑ मनु॒ष्या᳚न् मनु॒ष्या᳚न् मनु॒ष्यो॑ ऽह म॒हम् म॑नु॒ष्यो॑ मनु॒ष्यान्॑ । \newline
35. म॒नु॒ष्यो॑ मनु॒ष्या᳚न् मनु॒ष्या᳚न् मनु॒ष्यो॑ मनु॒ष्यो॑ मनु॒ष्या᳚न् थ्स॒ह स॒ह म॑नु॒ष्या᳚न् मनु॒ष्यो॑ मनु॒ष्यो॑ मनु॒ष्या᳚न् थ्स॒ह । \newline
36. म॒नु॒ष्या᳚न् थ्स॒ह स॒ह म॑नु॒ष्या᳚न् मनु॒ष्या᳚न् थ्स॒ह प्र॒जया᳚ प्र॒जया॑ स॒ह म॑नु॒ष्या᳚न् मनु॒ष्या᳚न् थ्स॒ह प्र॒जया᳚ । \newline
37. स॒ह प्र॒जया᳚ प्र॒जया॑ स॒ह स॒ह प्र॒जया॑ स॒ह स॒ह प्र॒जया॑ स॒ह स॒ह प्र॒जया॑ स॒ह । \newline
38. प्र॒जया॑ स॒ह स॒ह प्र॒जया᳚ प्र॒जया॑ स॒ह रा॒यो रा॒यः स॒ह प्र॒जया᳚ प्र॒जया॑ स॒ह रा॒यः । \newline
39. प्र॒जयेति॑ प्र - जया᳚ । \newline
40. स॒ह रा॒यो रा॒यः स॒ह स॒ह रा॒य स्पोषे॑ण॒ पोषे॑ण रा॒यः स॒ह स॒ह रा॒य स्पोषे॑ण । \newline
41. रा॒य स्पोषे॑ण॒ पोषे॑ण रा॒यो रा॒य स्पोषे॑ण॒ नमो॒ नमः॒ पोषे॑ण रा॒यो रा॒य स्पोषे॑ण॒ नमः॑ । \newline
42. पोषे॑ण॒ नमो॒ नमः॒ पोषे॑ण॒ पोषे॑ण॒ नमो॑ दे॒वेभ्यो॑ दे॒वेभ्यो॒ नमः॒ पोषे॑ण॒ पोषे॑ण॒ नमो॑ दे॒वेभ्यः॑ । \newline
43. नमो॑ दे॒वेभ्यो॑ दे॒वेभ्यो॒ नमो॒ नमो॑ दे॒वेभ्यः॑ स्व॒धा स्व॒धा दे॒वेभ्यो॒ नमो॒ नमो॑ दे॒वेभ्यः॑ स्व॒धा । \newline
44. दे॒वेभ्यः॑ स्व॒धा स्व॒धा दे॒वेभ्यो॑ दे॒वेभ्यः॑ स्व॒धा पि॒तृभ्यः॑ पि॒तृभ्यः॑ स्व॒धा दे॒वेभ्यो॑ दे॒वेभ्यः॑ स्व॒धा पि॒तृभ्यः॑ । \newline
45. स्व॒धा पि॒तृभ्यः॑ पि॒तृभ्यः॑ स्व॒धा स्व॒धा पि॒तृभ्य॑ इ॒द मि॒दम् पि॒तृभ्यः॑ स्व॒धा स्व॒धा पि॒तृभ्य॑ इ॒दम् । \newline
46. स्व॒धेति॑ स्व - धा । \newline
47. पि॒तृभ्य॑ इ॒द मि॒दम् पि॒तृभ्यः॑ पि॒तृभ्य॑ इ॒द म॒ह म॒ह मि॒दम् पि॒तृभ्यः॑ पि॒तृभ्य॑ इ॒द म॒हम् । \newline
48. पि॒तृभ्य॒ इति॑ पि॒तृ - भ्यः॒ । \newline
49. इ॒द म॒ह म॒ह मि॒द मि॒द म॒हम् निर् णिर॒ह मि॒द मि॒द म॒हम् निः । \newline
50. अ॒हम् निर् णिर॒ह म॒हम् निर् वरु॑णस्य॒ वरु॑णस्य॒ निर॒ह म॒हम् निर् वरु॑णस्य । \newline
51. निर् वरु॑णस्य॒ वरु॑णस्य॒ निर् णिर् वरु॑णस्य॒ पाशा॒त् पाशा॒द् वरु॑णस्य॒ निर् णिर् वरु॑णस्य॒ पाशा᳚त् । \newline
52. वरु॑णस्य॒ पाशा॒त् पाशा॒द् वरु॑णस्य॒ वरु॑णस्य॒ पाशा॒थ् सुवः॒ सुवः॒ पाशा॒द् वरु॑णस्य॒ वरु॑णस्य॒ पाशा॒थ् सुवः॑ । \newline
53. पाशा॒थ् सुवः॒ सुवः॒ पाशा॒त् पाशा॒थ् सुव॑ र॒भ्य॑भि सुवः॒ पाशा॒त् पाशा॒थ् सुव॑ र॒भि । \newline
54. सुव॑ र॒भ्य॑भि सुवः॒ सुव॑ र॒भि वि व्य॑भि सुवः॒ सुव॑ र॒भि वि । \newline
55. अ॒भि वि व्या᳚(1॒)भ्य॑भि वि ख्ये॑षम् ख्येषं॒ ॅव्या᳚(1॒)भ्य॑भि वि ख्ये॑षम् । \newline
\pagebreak
\markright{ TS 1.3.4.3  \hfill https://www.vedavms.in \hfill}

\section{ TS 1.3.4.3 }

\textbf{TS 1.3.4.3 } \newline
\textbf{Samhita Paata} \newline

वि ख्ये॑षं ॅवैश्वान॒रं ज्योति॒रग्ने᳚ व्रतपते॒ त्वं ॅव्र॒तानां᳚ ॅव्र॒तप॑तिरसि॒ या मम॑ त॒नूस्त्वय्यभू॑दि॒यꣳ सा मयि॒ या तव॑ त॒नूर् मय्यभू॑दे॒षा सा त्वयि॑ यथाय॒थं नौ᳚ व्रतपते व्र॒तिनो᳚र् व्र॒तानि॑ ॥ \newline

\textbf{Pada Paata} \newline

वीति॑ । ख्ये॒ष॒म् । वै॒श्वा॒न॒रम् । ज्योतिः॑ । अग्ने᳚ । व्र॒त॒प॒त॒ इति॑ व्रत - प॒ते॒ । त्वम् । व्र॒ताना᳚म् । व्र॒तप॑ति॒रिति॑ व्र॒त - प॒तिः॒ । अ॒सि॒ । या । मम॑ । त॒नूः । त्वयि॑ । अभू᳚त् । इ॒यम् । सा । मयि॑ । या । तव॑ । त॒नूः । मयि॑ । अभू᳚त् । ए॒षा । सा । त्वयि॑ । य॒था॒य॒थमिति॑ यथा - य॒थम् । नौ॒ । व्र॒त॒प॒त॒ इति॑ व्रत - प॒ते॒ । व्र॒तिनोः᳚ । व्र॒तानि॑ ॥  \newline


\textbf{Krama Paata} \newline

वि ख्ये॑षम् । ख्ये॒षं॒ ॅवै॒श्वा॒न॒रम् । वै॒श्वा॒न॒रम् ज्योतिः॑ । ज्योति॒रग्ने᳚ । अग्ने᳚ व्रतपते । व्र॒त॒प॒ते॒ त्वम् । व्र॒त॒प॒त॒ इति॑ व्रत - प॒ते॒ । त्वं ॅव्र॒ताना᳚म् । व्र॒तानां᳚ ॅव्र॒तप॑तिः । व्र॒तप॑तिरसि । व्र॒तप॑ति॒रिति॑ व्र॒त - प॒तिः॒ । अ॒सि॒ या । या मम॑ । मम॑ त॒नूः । त॒नूस्त्वयि॑ । त्वय्यभू᳚त् । अभू॑दि॒यम् । इ॒यꣳ सा । सा मयि॑ । मयि॒ या । या तव॑ । तव॑ त॒नूः । त॒नूर् मयि॑ । मय्यभू᳚त् । अभू॑दे॒षा । ए॒षा सा । सा त्वयि॑ । त्वयि॑ यथाय॒थम् । य॒था॒य॒थं नौ᳚ । य॒था॒य॒थमिति॑ यथा - य॒थम् । नौ॒ व्र॒त॒प॒ते॒ । व्र॒त॒प॒ते॒ व्र॒तिनोः᳚ । व्र॒त॒प॒त॒ इति॑ व्रत - प॒ते॒ । व्र॒तिनो᳚र् व्र॒तानि॑ । व्र॒तानीति॑ व्र॒तानि॑ \newline

\textbf{Jatai Paata} \newline

1. वि ख्ये॑षम् ख्येषं॒ ॅवि वि ख्ये॑षम् । \newline
2. ख्ये॒षं॒ ॅवै॒श्वा॒न॒रं ॅवै᳚श्वान॒रम् ख्ये॑षम् ख्येषं ॅवैश्वान॒रम् । \newline
3. वै॒श्वा॒न॒रम् ज्योति॒र् ज्योति॑र् वैश्वान॒रं ॅवै᳚श्वान॒रम् ज्योतिः॑ । \newline
4. ज्योति॒ रग्ने ऽग्ने॒ ज्योति॒र् ज्योति॒ रग्ने᳚ । \newline
5. अग्ने᳚ व्रतपते व्रतप॒ते ऽग्ने ऽग्ने᳚ व्रतपते । \newline
6. व्र॒त॒प॒ते॒ त्वम् त्वं ॅव्र॑तपते व्रतपते॒ त्वम् । \newline
7. व्र॒त॒प॒त॒ इति॑ व्रत - प॒ते॒ । \newline
8. त्वं ॅव्र॒तानां᳚ ॅव्र॒ताना॒म् त्वम् त्वं ॅव्र॒ताना᳚म् । \newline
9. व्र॒तानां᳚ ॅव्र॒तप॑तिर् व्र॒तप॑तिर् व्र॒तानां᳚ ॅव्र॒तानां᳚ ॅव्र॒तप॑तिः । \newline
10. व्र॒तप॑ति रस्यसि व्र॒तप॑तिर् व्र॒तप॑ति रसि । \newline
11. व्र॒तप॑ति॒रिति॑ व्र॒त - प॒तिः॒ । \newline
12. अ॒सि॒ या या ऽस्य॑सि॒ या । \newline
13. या मम॒ मम॒ या या मम॑ । \newline
14. मम॑ त॒नू स्त॒नूर् मम॒ मम॑ त॒नूः । \newline
15. त॒नू स्त्वयि॒ त्वयि॑ त॒नू स्त॒नू स्त्वयि॑ । \newline
16. त्वय्यभू॒ दभू॒त् त्वयि॒ त्वय्यभू᳚त् । \newline
17. अभू॑ दि॒य मि॒य मभू॒ दभू॑ दि॒यम् । \newline
18. इ॒यꣳ सा सेय मि॒यꣳ सा । \newline
19. सा मयि॒ मयि॒ सा सा मयि॑ । \newline
20. मयि॒ या या मयि॒ मयि॒ या । \newline
21. या तव॒ तव॒ या या तव॑ । \newline
22. तव॑ त॒नू स्त॒नू स्तव॒ तव॑ त॒नूः । \newline
23. त॒नूर् मयि॒ मयि॑ त॒नू स्त॒नूर् मयि॑ । \newline
24. मय्यभू॒ दभू॒न् मयि॒ मय्यभू᳚त् । \newline
25. अभू॑ दे॒षैषा ऽभू॒दभू॑ दे॒षा । \newline
26. ए॒षा सा सैषैषा सा । \newline
27. सा त्वयि॒ त्वयि॒ सा सा त्वयि॑ । \newline
28. त्वयि॑ यथाय॒थं ॅय॑थाय॒थम् त्वयि॒ त्वयि॑ यथाय॒थम् । \newline
29. य॒था॒य॒थम् नौ॑ नौ यथाय॒थं ॅय॑थाय॒थम् नौ᳚ । \newline
30. य॒था॒य॒थमिति॑ यथा - य॒थम् । \newline
31. नौ॒ व्र॒त॒प॒ते॒ व्र॒त॒प॒ते॒ नौ॒ नौ॒ व्र॒त॒प॒ते॒ । \newline
32. व्र॒त॒प॒ते॒ व्र॒तिनो᳚र् व्र॒तिनो᳚र् व्रतपते व्रतपते व्र॒तिनोः᳚ । \newline
33. व्र॒त॒प॒त॒ इति॑ व्रत - प॒ते॒ । \newline
34. व्र॒तिनो᳚र् व्र॒तानि॑ व्र॒तानि॑ व्र॒तिनो᳚र् व्र॒तिनो᳚र् व्र॒तानि॑ । \newline
35. व्र॒तानीति॑ व्र॒तानि॑ । \newline

\textbf{Ghana Paata } \newline

1. वि ख्ये॑षम् ख्येषं॒ ॅवि वि ख्ये॑षं ॅवैश्वान॒रं ॅवै᳚श्वान॒रम् ख्ये॑षं॒ ॅवि वि ख्ये॑षं ॅवैश्वान॒रम् । \newline
2. ख्ये॒षं॒ ॅवै॒श्वा॒न॒रं ॅवै᳚श्वान॒रम् ख्ये॑षम् ख्येषं ॅवैश्वान॒रम् ज्योति॒र् ज्योति॑र् वैश्वान॒रम् ख्ये॑षम् ख्येषं ॅवैश्वान॒रम् ज्योतिः॑ । \newline
3. वै॒श्वा॒न॒रम् ज्योति॒र् ज्योति॑र् वैश्वान॒रं ॅवै᳚श्वान॒रम् ज्योति॒ रग्ने ऽग्ने॒ ज्योति॑र् वैश्वान॒रं ॅवै᳚श्वान॒रम् ज्योति॒रग्ने᳚ । \newline
4. ज्योति॒रग्ने ऽग्ने॒ ज्योति॒र् ज्योति॒रग्ने᳚ व्रतपते व्रतप॒ते ऽग्ने॒ ज्योति॒र् ज्योति॒रग्ने᳚ व्रतपते । \newline
5. अग्ने᳚ व्रतपते व्रतप॒ते ऽग्ने ऽग्ने᳚ व्रतपते॒ त्वम् त्वं ॅव्र॑तप॒ते ऽग्ने ऽग्ने᳚ व्रतपते॒ त्वम् । \newline
6. व्र॒त॒प॒ते॒ त्वम् त्वं ॅव्र॑तपते व्रतपते॒ त्वं ॅव्र॒तानां᳚ ॅव्र॒ताना॒म् त्वं ॅव्र॑तपते व्रतपते॒ त्वं ॅव्र॒ताना᳚म् । \newline
7. व्र॒त॒प॒त॒ इति॑ व्रत - प॒ते॒ । \newline
8. त्वं ॅव्र॒तानां᳚ ॅव्र॒ताना॒म् त्वम् त्वं ॅव्र॒तानां᳚ ॅव्र॒तप॑तिर् व्र॒तप॑तिर् व्र॒ताना॒म् त्वम् त्वं ॅव्र॒तानां᳚ ॅव्र॒तप॑तिः । \newline
9. व्र॒तानां᳚ ॅव्र॒तप॑तिर् व्र॒तप॑तिर् व्र॒तानां᳚ ॅव्र॒तानां᳚ ॅव्र॒तप॑ति रस्यसि व्र॒तप॑तिर् व्र॒तानां᳚ ॅव्र॒तानां᳚ ॅव्र॒तप॑ति रसि । \newline
10. व्र॒तप॑ति रस्यसि व्र॒तप॑तिर् व्र॒तप॑ति रसि॒ या या ऽसि॑ व्र॒तप॑तिर् व्र॒तप॑ति रसि॒ या । \newline
11. व्र॒तप॑ति॒रिति॑ व्र॒त - प॒तिः॒ । \newline
12. अ॒सि॒ या या ऽस्य॑सि॒ या मम॒ मम॒ या ऽस्य॑सि॒ या मम॑ । \newline
13. या मम॒ मम॒ या या मम॑ त॒नू स्त॒नूर् मम॒ या या मम॑ त॒नूः । \newline
14. मम॑ त॒नू स्त॒नूर् मम॒ मम॑ त॒नू स्त्वयि॒ त्वयि॑ त॒नूर् मम॒ मम॑ त॒नू स्त्वयि॑ । \newline
15. त॒नू स्त्वयि॒ त्वयि॑ त॒नू स्त॒नू स्त्वय्यभू॒ दभू॒त् त्वयि॑ त॒नू स्त॒नू स्त्वय्यभू᳚त् । \newline
16. त्वय्यभू॒ दभू॒त् त्वयि॒ त्वय्यभू॑ दि॒य मि॒य मभू॒त् त्वयि॒ त्वय्यभू॑ दि॒यम् । \newline
17. अभू॑दि॒य मि॒य मभू॒ दभू॑ दि॒यꣳ सा सेय मभू॒ दभू॑ दि॒यꣳ सा । \newline
18. इ॒यꣳ सा सेय मि॒यꣳ सा मयि॒ मयि॒ सेय मि॒यꣳ सा मयि॑ । \newline
19. सा मयि॒ मयि॒ सा सा मयि॒ या या मयि॒ सा सा मयि॒ या । \newline
20. मयि॒ या या मयि॒ मयि॒ या तव॒ तव॒ या मयि॒ मयि॒ या तव॑ । \newline
21. या तव॒ तव॒ या या तव॑ त॒नू स्त॒नू स्तव॒ या या तव॑ त॒नूः । \newline
22. तव॑ त॒नू स्त॒नू स्तव॒ तव॑ त॒नूर् मयि॒ मयि॑ त॒नू स्तव॒ तव॑ त॒नूर् मयि॑ । \newline
23. त॒नूर् मयि॒ मयि॑ त॒नू स्त॒नूर् मय्यभू॒ दभू॒न् मयि॑ त॒नू स्त॒नूर् मय्यभू᳚त् । \newline
24. मय्यभू॒ दभू॒न् मयि॒ मय्यभू॑ दे॒षैषा ऽभू॒न् मयि॒ मय्यभू॑ दे॒षा । \newline
25. अभू॑ दे॒षैषा ऽभू॒दभू॑ दे॒षा सा सैषा ऽभू॒दभू॑ दे॒षा सा । \newline
26. ए॒षा सा सैषैषा सा त्वयि॒ त्वयि॒ सैषैषा सा त्वयि॑ । \newline
27. सा त्वयि॒ त्वयि॒ सा सा त्वयि॑ यथाय॒थं ॅय॑थाय॒थम् त्वयि॒ सा सा त्वयि॑ यथाय॒थम् । \newline
28. त्वयि॑ यथाय॒थं ॅय॑थाय॒थम् त्वयि॒ त्वयि॑ यथाय॒थम् नौ॑ नौ यथाय॒थम् त्वयि॒ त्वयि॑ यथाय॒थम् नौ᳚ । \newline
29. य॒था॒य॒थम् नौ॑ नौ यथाय॒थं ॅय॑थाय॒थम् नौ᳚ व्रतपते व्रतपते नौ यथाय॒थं ॅय॑थाय॒थम् नौ᳚ व्रतपते । \newline
30. य॒था॒य॒थमिति॑ यथा - य॒थम् । \newline
31. नौ॒ व्र॒त॒प॒ते॒ व्र॒त॒प॒ते॒ नौ॒ नौ॒ व्र॒त॒प॒ते॒ व्र॒तिनो᳚र् व्र॒तिनो᳚र् व्रतपते नौ नौ व्रतपते व्र॒तिनोः᳚ । \newline
32. व्र॒त॒प॒ते॒ व्र॒तिनो᳚र् व्र॒तिनो᳚र् व्रतपते व्रतपते व्र॒तिनो᳚र् व्र॒तानि॑ व्र॒तानि॑ व्र॒तिनो᳚र् व्रतपते व्रतपते व्र॒तिनो᳚र् व्र॒तानि॑ । \newline
33. व्र॒त॒प॒त॒ इति॑ व्रत - प॒ते॒ । \newline
34. व्र॒तिनो᳚र् व्र॒तानि॑ व्र॒तानि॑ व्र॒तिनो᳚र् व्र॒तिनो᳚र् व्र॒तानि॑ । \newline
35. व्र॒तानीति॑ व्र॒तानि॑ । \newline
\pagebreak
\markright{ TS 1.3.5.1  \hfill https://www.vedavms.in \hfill}

\section{ TS 1.3.5.1 }

\textbf{TS 1.3.5.1 } \newline
\textbf{Samhita Paata} \newline

अत्य॒न्यानगां॒ नान्यानुपा॑गाम॒र्वाक्त्वा॒ परै॑रविदं प॒रोऽव॑रै॒स्तं त्वा॑ जुषे वैष्ण॒वं दे॑वय॒ज्यायै॑ दे॒वस्त्वा॑ सवि॒ता मद्ध्वा॑ऽन॒क्त्वोष॑धे॒ त्राय॑स्वैनꣳ॒॒ स्वधि॑ते॒ मैनꣳ॑ हिꣳसी॒र् दिव॒मग्रे॑ण॒ मा ले॑खीर॒न्तरि॑क्षं॒ मद्ध्ये॑न॒ मा हिꣳ॑सीः पृथि॒व्या सं भ॑व॒ वन॑स्पते श॒तव॑ल्.शो॒ वि रो॑ह स॒हस्र॑वल्.शा॒ वि व॒यꣳ रु॑हेम॒ यं ( ) त्वा॒ऽयꣳ स्वधि॑ति॒स्तेति॑जानः प्रणि॒नाय॑ मह॒ते सौभ॑गा॒याच्छि॑न्नो॒ रायः॑ सु॒वीरः॑ ॥ \newline

\textbf{Pada Paata} \newline

अतीति॑ । अ॒न्यान् । अगा᳚म् । न । अ॒न्यान् । उपेति॑ । अ॒गा॒म् । अ॒र्वाक् । त्वा॒ । परैः᳚ । अ॒वि॒द॒म् । प॒रः । अव॑रैः । तम् । त्वा॒ । जु॒षे॒ । वै॒ष्ण॒वम् । दे॒व॒य॒ज्याया॒ इति॑ देव - य॒ज्यायै᳚ । दे॒वः । त्वा॒ । स॒वि॒ता । मद्ध्वा᳚ । अ॒न॒क्तु॒ । ओष॑धे । त्राय॑स्व । ए॒न॒म् । स्वधि॑त॒ इति॒ स्व - धि॒ते॒ । मा । ए॒न॒म् । हिꣳ॒॒सीः॒ । दिव᳚म् । अग्रे॑ण । मा । ल॒खीः॒ । अ॒न्तरि॑क्षम् । मद्ध्ये॑न । मा । हिꣳ॒॒सीः॒ । पृ॒थि॒व्या । समिति॑ । भ॒व॒ । वन॑स्पते । श॒तव॑ल्.श॒ इति॑ श॒त - व॒ल्॒.शः॒ । वीति॑ । रो॒ह॒ । स॒हस्र॑वल्.शा॒ इति॑ स॒हस्र॑ - व॒ल्॒.शाः॒ । वीति॑ । व॒यम् । रु॒हे॒म॒ । यम् ( ) । त्वा॒ । अ॒यम् । स्वधि॑ति॒रिति॒ स्व - धि॒तिः॒ । तेति॑जानः । प्र॒णि॒नायेति॑ प्र - नि॒नाय॑ । म॒ह॒ते । सौभ॑गाय । अच्छि॑न्नः । रायः॑ । सु॒वीर॒ इति॑ सु - वीरः॑ ॥  \newline


\textbf{Krama Paata} \newline

अत्य॒न्यान् । अ॒न्यानगा᳚म् । अगा॒म् न । नान्यान् । अ॒न्यानुप॑ । उपा॑गाम् । अ॒गा॒म॒र्वाक् । अ॒र्वाक् त्वा᳚ । त्वा॒ परैः᳚ । परै॑रविदम् । अ॒वि॒द॒म् प॒रः । प॒रोऽव॑रैः । अव॑रै॒स्तम् । तम् त्वा᳚ । त्वा॒ जु॒षे॒ । जु॒षे॒ वै॒ष्ण॒वम् । वै॒ष्ण॒वम् दे॑वय॒ज्यायै᳚ । दे॒व॒य॒ज्यायै॑ दे॒वः । दे॒व॒य॒ज्याया॒ इति॑ देव - य॒ज्यायै᳚ । दे॒वस्त्वा᳚ । त्वा॒ स॒वि॒ता । स॒वि॒ता मद्ध्वा᳚ । मद्ध्वा॑ऽनक्तु । अ॒न॒क्त्वोष॑धे । ओष॑धे॒ त्राय॑स्व । त्राय॑स्वैनम् । ए॒नꣳ॒॒ स्वधि॑ते । स्वधि॑ते॒ मा । स्वधि॑त॒ इति॒ स्व - धि॒ते॒ । मैन᳚म् । ए॒नꣳ॒॒ हिꣳ॒॒सीः॒ । हिꣳ॒॒सी॒र् दिव᳚म् । दिव॒मग्रे॑ण । अग्रे॑ण॒ मा । मा ले॑खीः । ले॒खी॒र॒न्तरि॑क्षम् । अ॒न्तरि॑क्ष॒म् मद्ध्ये॑न । मद्ध्ये॑न॒ मा । मा हिꣳ॑सीः । हिꣳ॒॒सीः॒ पृ॒थि॒व्या । पृ॒थि॒व्या सम् । सम् भ॑व । भ॒व॒ वन॑स्पते । वन॑स्पते श॒तव॑ल्.शः । श॒तव॑ल्.शो॒ वि । श॒तव॑ल्.श॒ इति॑ श॒त - व॒ल्॒.शः॒ । वि रो॑ह । रो॒ह॒ स॒हस्र॑वल्.शाः । स॒हस्र॑वल्.शा॒ वि । स॒हस्र॑वल्.शा॒ इति॑ स॒हस्र॑ - व॒ल्॒.शाः॒ । वि व॒यम् । व॒यꣳ रु॑हेम । रु॒हे॒म॒ यम् ( ) । यम् त्वा᳚ । त्वा॒ऽयम् । अ॒यꣳ स्वधि॑तिः । स्वधि॑ति॒ स्तेति॑जानः । स्वधि॑ति॒रिति॒ स्व - धि॒तिः॒ । तेति॑जानः प्रणि॒नाय॑ । प्र॒णि॒नाय॑ मह॒ते । प्र॒णि॒नायेति॑ प्र - नि॒नाय॑ । म॒ह॒ते सौभ॑गाय । सौभ॑गा॒याच्छि॑न्नः । अच्छि॑न्नो॒ रायः॑ । रायः॑ सु॒वीरः॑ । सु॒वीर॒ इति॑ सु - वीरः॑ । \newline

\textbf{Jatai Paata} \newline

1. अत्य॒न्या न॒न्या नत्यत्य॒न्यान् । \newline
2. अ॒न्या नगा॒ मगा॑ म॒न्या न॒न्या नगा᳚म् । \newline
3. अगा॒म् न नागा॒ मगा॒म् न । \newline
4. नान्या न॒न्यान् न नान्यान् । \newline
5. अ॒न्या नुपोपा॒न्या न॒न्या नुप॑ । \newline
6. उपा॑गा मगा॒ मुपोपा॑गाम् । \newline
7. अ॒गा॒ म॒र्वा ग॒र्वाग॑गा मगा म॒र्वाक् । \newline
8. अ॒र्वाक् त्वा᳚ त्वा॒ ऽर्वाग॒र्वाक् त्वा᳚ । \newline
9. त्वा॒ परैः॒ परै᳚ स्त्वा त्वा॒ परैः᳚ । \newline
10. परै॑ रविद मविद॒म् परैः॒ परै॑ रविदम् । \newline
11. अ॒वि॒द॒म् प॒रः प॒रो॑ ऽविद मविदम् प॒रः । \newline
12. प॒रो ऽव॑ रै॒रव॑रैः प॒रः प॒रो ऽव॑रैः । \newline
13. अव॑ रै॒स्तम् त मव॑ रै॒रव॑ रै॒स्तम् । \newline
14. तम् त्वा᳚ त्वा॒ तम् तम् त्वा᳚ । \newline
15. त्वा॒ जु॒षे॒ जु॒षे॒ त्वा॒ त्वा॒ जु॒षे॒ । \newline
16. जु॒षे॒ वै॒ष्ण॒वं ॅवै᳚ष्ण॒वम् जु॑षे जुषे वैष्ण॒वम् । \newline
17. वै॒ष्ण॒वम् दे॑वय॒ज्यायै॑ देवय॒ज्यायै॑ वैष्ण॒वं ॅवै᳚ष्ण॒वम् दे॑वय॒ज्यायै᳚ । \newline
18. दे॒व॒य॒ज्यायै॑ दे॒वो दे॒वो दे॑वय॒ज्यायै॑ देवय॒ज्यायै॑ दे॒वः । \newline
19. दे॒व॒य॒ज्याया॒ इति॑ देव - य॒ज्यायै᳚ । \newline
20. दे॒व स्त्वा᳚ त्वा दे॒वो दे॒व स्त्वा᳚ । \newline
21. त्वा॒ स॒वि॒ता स॑वि॒ता त्वा᳚ त्वा सवि॒ता । \newline
22. स॒वि॒ता मद्ध्वा॒ मद्ध्वा॑ सवि॒ता स॑वि॒ता मद्ध्वा᳚ । \newline
23. मद्ध्वा॑ ऽनक्त्वनक्तु॒ मद्ध्वा॒ मद्ध्वा॑ ऽनक्तु । \newline
24. अ॒न॒क्त्वोष॑ध॒ ओष॑धे ऽनक्त्वन॒ क्त्वोष॑धे । \newline
25. ओष॑धे॒ त्राय॑स्व॒ त्राय॒ स्वौष॑ध॒ ओष॑धे॒ त्राय॑स्व । \newline
26. त्राय॑स्वैन मेन॒म् त्राय॑स्व॒ त्राय॑स्वैनम् । \newline
27. ए॒न॒(ग्ग्॒) स्वधि॑ते॒ स्वधि॑त एन मेन॒(ग्ग्॒) स्वधि॑ते । \newline
28. स्वधि॑ते॒ मा मा स्वधि॑ते॒ स्वधि॑ते॒ मा । \newline
29. स्वधि॑त॒ इति॒ स्व - धि॒ते॒ । \newline
30. मैन॑ मेन॒म् मा मैन᳚म् । \newline
31. ए॒न॒(ग्म्॒) हि॒(ग्म्॒)सी॒र्॒. हि॒(ग्म्॒)सी॒ रे॒न॒ मे॒न॒(ग्म्॒) हि॒(ग्म्॒)सीः॒ । \newline
32. हि॒(ग्म्॒)सी॒र् दिव॒म् दिव(ग्म्॑) हिꣳसीर्. हिꣳसी॒र् दिव᳚म् । \newline
33. दिव॒ मग्रे॒ णाग्रे॑ण॒ दिव॒म् दिव॒ मग्रे॑ण । \newline
34. अग्रे॑ण॒ मा मा ऽग्रे॒णाग्रे॑ण॒ मा । \newline
35. मा ले॑खीर् लेखी॒र् मा मा ले॑खीः । \newline
36. ले॒खी॒ र॒न्तरि॑क्ष म॒न्तरि॑क्षम् ॅलेखीर् लेखी र॒न्तरि॑क्षम् । \newline
37. अ॒न्तरि॑क्ष॒म् मद्ध्ये॑न॒ मद्ध्ये॑ ना॒न्तरि॑क्ष म॒न्तरि॑क्ष॒म् मद्ध्ये॑न । \newline
38. मद्ध्ये॑न॒ मा मा मद्ध्ये॑न॒ मद्ध्ये॑न॒ मा । \newline
39. मा हि(ग्म्॑)सीर्. हिꣳसी॒र् मा मा हि(ग्म्॑)सीः । \newline
40. हि॒(ग्म्॒)सीः॒ पृ॒थि॒व्या पृ॑थि॒व्या हि(ग्म्॑)सीर्. हिꣳसीः पृथि॒व्या । \newline
41. पृ॒थि॒व्या सꣳ सम् पृ॑थि॒व्या पृ॑थि॒व्या सम् । \newline
42. सम् भ॑व भव॒ सꣳ सम् भ॑व । \newline
43. भ॒व॒ वन॑स्पते॒ वन॑स्पते भव भव॒ वन॑स्पते । \newline
44. वन॑स्पते श॒तव॑ल्.शः श॒तव॑ल्.शो॒ वन॑स्पते॒ वन॑स्पते श॒तव॑ल्.शः । \newline
45. श॒तव॑ल्.शो॒ वि वि श॒तव॑ल्.शः श॒तव॑ल्.शो॒ वि । \newline
46. श॒तव॑ल्.श॒ इति॑ श॒त - व॒ल्॒.शः॒ । \newline
47. वि रो॑ह रोह॒ वि वि रो॑ह । \newline
48. रो॒ह॒ स॒हस्र॑वल्.शाः स॒हस्र॑वल्.शा रोह रोह स॒हस्र॑वल्.शाः । \newline
49. स॒हस्र॑वल्.शा॒ वि वि स॒हस्र॑वल्.शाः स॒हस्र॑वल्.शा॒ वि । \newline
50. स॒हस्र॑वल्.शा॒ इति॑ स॒हस्र॑ - व॒ल्॒.शाः॒ । \newline
51. वि व॒यं ॅव॒यं ॅवि वि व॒यम् । \newline
52. व॒यꣳ रु॑हेम रुहेम व॒यं ॅव॒यꣳ रु॑हेम । \newline
53. रु॒हे॒म॒ यं ॅयꣳ रु॑हेम रुहेम॒ यम् । \newline
54. यम् त्वा᳚ त्वा॒ यं ॅयम् त्वा᳚ । \newline
55. त्वा॒ ऽय म॒यम् त्वा᳚ त्वा॒ ऽयम् । \newline
56. अ॒यꣳ स्वधि॑तिः॒ स्वधि॑ति र॒य म॒यꣳ स्वधि॑तिः । \newline
57. स्वधि॑ति॒ स्तेति॑जान॒ स्तेति॑जानः॒ स्वधि॑तिः॒ स्वधि॑ति॒ स्तेति॑जानः । \newline
58. स्वधि॑ति॒रिति॒ स्व - धि॒तिः॒ । \newline
59. तेति॑जानः प्रणि॒नाय॑ प्रणि॒नाय॒ तेति॑जान॒ स्तेति॑जानः प्रणि॒नाय॑ । \newline
60. प्र॒णि॒नाय॑ मह॒ते म॑ह॒ते प्र॑णि॒नाय॑ प्रणि॒नाय॑ मह॒ते । \newline
61. प्र॒णि॒नायेति॑ प्र - नि॒नाय॑ । \newline
62. म॒ह॒ते सौभ॑गाय॒ सौभ॑गाय मह॒ते म॑ह॒ते सौभ॑गाय । \newline
63. सौभ॑गा॒ याच्छि॒न्नो ऽच्छि॑न्नः॒ सौभ॑गाय॒ सौभ॑गा॒ याच्छि॑न्नः । \newline
64. अच्छि॑न्नो॒ रायो॒ रायो ऽच्छि॒न्नो ऽच्छि॑न्नो॒ रायः॑ । \newline
65. रायः॑ सु॒वीरः॑ सु॒वीरो॒ रायो॒ रायः॑ सु॒वीरः॑ । \newline
66. सु॒वीर॒ इति॑ सु - वीरः॑ । \newline

\textbf{Ghana Paata } \newline

1. अत्य॒न्या न॒न्या नत्यत्य॒न्या नगा॒ मगा॑ म॒न्या नत्यत्य॒न्या नगा᳚म् । \newline
2. अ॒न्या नगा॒ मगा॑ म॒न्या न॒न्या नगा॒म् न नागा॑ म॒न्या न॒न्या नगा॒म् न । \newline
3. अगा॒म् न नागा॒ मगा॒म् नान्या न॒न्यान् नागा॒ मगा॒म् नान्यान् । \newline
4. नान्या न॒न्यान् न नान्या नुपोपा॒न्यान् न नान्या नुप॑ । \newline
5. अ॒न्या नुपोपा॒न्या न॒न्या नुपा॑गा मगा॒ मुपा॒न्या न॒न्या नुपा॑गाम् । \newline
6. उपा॑गा मगा॒ मुपोपा॑गा म॒र्वाग॒र्वाग॑गा॒ मुपोपा॑गा म॒र्वाक् । \newline
7. अ॒गा॒ म॒र्वाग॒र्वाग॑गा मगा म॒र्वाक् त्वा᳚ त्वा॒ ऽर्वाग॑गा मगा म॒र्वाक् त्वा᳚ । \newline
8. अ॒र्वाक् त्वा᳚ त्वा॒ ऽर्वाग॒र्वाक् त्वा॒ परैः॒ परै᳚ स्त्वा॒ ऽर्वाग॒र्वाक् त्वा॒ परैः᳚ । \newline
9. त्वा॒ परैः॒ परै᳚ स्त्वा त्वा॒ परै॑ रविद मविद॒म् परै᳚ स्त्वा त्वा॒ परै॑ रविदम् । \newline
10. परै॑ रविद मविद॒म् परैः॒ परै॑ रविदम् प॒रः प॒रो॑ ऽविद॒म् परैः॒ परै॑ रविदम् प॒रः । \newline
11. अ॒वि॒द॒म् प॒रः प॒रो॑ ऽविद मविदम् प॒रो ऽव॑रै॒ रव॑रैः प॒रो॑ ऽविद मविदम् प॒रो ऽव॑रैः । \newline
12. प॒रो ऽव॑रै॒ रव॑रैः प॒रः प॒रो ऽव॑रै॒ स्तम् त मव॑रैः प॒रः प॒रो ऽव॑रै॒ स्तम् । \newline
13. अव॑रै॒ स्तम् त मव॑रै॒ रव॑रै॒ स्तम् त्वा᳚ त्वा॒ त मव॑रै॒ रव॑रै॒ स्तम् त्वा᳚ । \newline
14. तम् त्वा᳚ त्वा॒ तम् तम् त्वा॑ जुषे जुषे त्वा॒ तम् तम् त्वा॑ जुषे । \newline
15. त्वा॒ जु॒षे॒ जु॒षे॒ त्वा॒ त्वा॒ जु॒षे॒ वै॒ष्ण॒वं ॅवै᳚ष्ण॒वम् जु॑षे त्वा त्वा जुषे वैष्ण॒वम् । \newline
16. जु॒षे॒ वै॒ष्ण॒वं ॅवै᳚ष्ण॒वम् जु॑षे जुषे वैष्ण॒वम् दे॑वय॒ज्यायै॑ देवय॒ज्यायै॑ वैष्ण॒वम् जु॑षे जुषे वैष्ण॒वम् दे॑वय॒ज्यायै᳚ । \newline
17. वै॒ष्ण॒वम् दे॑वय॒ज्यायै॑ देवय॒ज्यायै॑ वैष्ण॒वं ॅवै᳚ष्ण॒वम् दे॑वय॒ज्यायै॑ दे॒वो दे॒वो दे॑वय॒ज्यायै॑ वैष्ण॒वं ॅवै᳚ष्ण॒वम् दे॑वय॒ज्यायै॑ दे॒वः । \newline
18. दे॒व॒य॒ज्यायै॑ दे॒वो दे॒वो दे॑वय॒ज्यायै॑ देवय॒ज्यायै॑ दे॒वस्त्वा᳚ त्वा दे॒वो दे॑वय॒ज्यायै॑ देवय॒ज्यायै॑ दे॒वस्त्वा᳚ । \newline
19. दे॒व॒य॒ज्याया॒ इति॑ देव - य॒ज्यायै᳚ । \newline
20. दे॒व स्त्वा᳚ त्वा दे॒वो दे॒व स्त्वा॑ सवि॒ता स॑वि॒ता त्वा॑ दे॒वो दे॒व स्त्वा॑ सवि॒ता । \newline
21. त्वा॒ स॒वि॒ता स॑वि॒ता त्वा᳚ त्वा सवि॒ता मद्ध्वा॒ मद्ध्वा॑ सवि॒ता त्वा᳚ त्वा सवि॒ता मद्ध्वा᳚ । \newline
22. स॒वि॒ता मद्ध्वा॒ मद्ध्वा॑ सवि॒ता स॑वि॒ता मद्ध्वा॑ ऽनक्त्वनक्तु॒ मद्ध्वा॑ सवि॒ता स॑वि॒ता मद्ध्वा॑ ऽनक्तु । \newline
23. मद्ध्वा॑ ऽनक्त्वनक्तु॒ मद्ध्वा॒ मद्ध्वा॑ ऽन॒क्त्वोष॑ध॒ ओष॑धे ऽनक्तु॒ मद्ध्वा॒ मद्ध्वा॑ ऽन॒क्त्वोष॑धे । \newline
24. अ॒न॒क्त्वोष॑ध॒ ओष॑धे ऽनक्त्व न॒क्त्वोष॑धे॒ त्राय॑स्व॒ त्राय॒स्वौष॑धे ऽनक्त्व न॒क्त्वोष॑धे॒ त्राय॑स्व । \newline
25. ओष॑धे॒ त्राय॑स्व॒ त्राय॒स्वौष॑ध॒ ओष॑धे॒ त्राय॑स्वैन मेन॒म् त्राय॒स्वौष॑ध॒ ओष॑धे॒ त्राय॑स्वैनम् । \newline
26. त्राय॑स्वैन मेन॒म् त्राय॑स्व॒ त्राय॑स्वैन॒(ग्ग्॒) स्वधि॑ते॒ स्वधि॑त एन॒म् त्राय॑स्व॒ त्राय॑स्वैन॒(ग्ग्॒) स्वधि॑ते । \newline
27. ए॒न॒(ग्ग्॒) स्वधि॑ते॒ स्वधि॑त एन मेन॒(ग्ग्॒) स्वधि॑ते॒ मा मा स्वधि॑त एन मेन॒(ग्ग्॒) स्वधि॑ते॒ मा । \newline
28. स्वधि॑ते॒ मा मा स्वधि॑ते॒ स्वधि॑ते॒ मैन॑ मेन॒म् मा स्वधि॑ते॒ स्वधि॑ते॒ मैन᳚म् । \newline
29. स्वधि॑त॒ इति॒ स्व - धि॒ते॒ । \newline
30. मैन॑ मेन॒म् मा मैन(ग्म्॑) हिꣳसीर्. हिꣳसी रेन॒म् मा मैन(ग्म्॑) हिꣳसीः । \newline
31. ए॒न॒(ग्म्॒) हि॒(ग्म्॒)सी॒र्॒. हि॒(ग्म्॒)सी॒ रे॒न॒ मे॒न॒(ग्म्॒) हि॒(ग्म्॒)सी॒र् दिव॒म् दिव(ग्म्॑) हिꣳसी रेन मेनꣳ हिꣳसी॒र् दिव᳚म् । \newline
32. हि॒(ग्म्॒)सी॒र् दिव॒म् दिव(ग्म्॑) हिꣳसीर्. हिꣳसी॒र् दिव॒ मग्रे॒णाग्रे॑ण॒ दिव(ग्म्॑) हिꣳसीर्. हिꣳसी॒र् दिव॒ मग्रे॑ण । \newline
33. दिव॒ मग्रे॒णा ग्रे॑ण॒ दिव॒म् दिव॒ मग्रे॑ण॒ मा मा ऽग्रे॑ण॒ दिव॒म् दिव॒ मग्रे॑ण॒ मा । \newline
34. अग्रे॑ण॒ मा मा ऽग्रे॒णा ग्रे॑ण॒ मा ले॑खीर् लेखी॒र् मा ऽग्रे॒णा ग्रे॑ण॒ मा ले॑खीः । \newline
35. मा ले॑खीर् लेखी॒र् मा मा ले॑खी र॒न्तरि॑क्ष म॒न्तरि॑क्षम् ॅलेखी॒र् मा मा ले॑खी र॒न्तरि॑क्षम् । \newline
36. ले॒खी॒ र॒न्तरि॑क्ष म॒न्तरि॑क्षम् ॅलेखीर् लेखी र॒न्तरि॑क्ष॒म् मद्ध्ये॑न॒ मद्ध्ये॑ना॒न्तरि॑क्षम् ॅलेखीर् लेखी र॒न्तरि॑क्ष॒म् मद्ध्ये॑न । \newline
37. अ॒न्तरि॑क्ष॒म् मद्ध्ये॑न॒ मद्ध्ये॑ना॒न्तरि॑क्ष म॒न्तरि॑क्ष॒म् मद्ध्ये॑न॒ मा मा मद्ध्ये॑ना॒न्तरि॑क्ष म॒न्तरि॑क्ष॒म् मद्ध्ये॑न॒ मा । \newline
38. मद्ध्ये॑न॒ मा मा मद्ध्ये॑न॒ मद्ध्ये॑न॒ मा हि(ग्म्॑)सीर्. हिꣳसी॒र् मा मद्ध्ये॑न॒ मद्ध्ये॑न॒ मा हि(ग्म्॑)सीः । \newline
39. मा हि(ग्म्॑)सीर्. हिꣳसी॒र् मा मा हि(ग्म्॑)सीः पृथि॒व्या पृ॑थि॒व्या हि(ग्म्॑)सी॒र् मा मा हि(ग्म्॑)सीः पृथि॒व्या । \newline
40. हि॒(ग्म्॒)सीः॒ पृ॒थि॒व्या पृ॑थि॒व्या हि(ग्म्॑)सीर्. हिꣳसीः पृथि॒व्या सꣳ सम् पृ॑थि॒व्या हि(ग्म्॑)सीर्. हिꣳसीः पृथि॒व्या सम् । \newline
41. पृ॒थि॒व्या सꣳ सम् पृ॑थि॒व्या पृ॑थि॒व्या सम् भ॑व भव॒ सम् पृ॑थि॒व्या पृ॑थि॒व्या सम् भ॑व । \newline
42. सम् भ॑व भव॒ सꣳ सम् भ॑व॒ वन॑स्पते॒ वन॑स्पते भव॒ सꣳ सम् भ॑व॒ वन॑स्पते । \newline
43. भ॒व॒ वन॑स्पते॒ वन॑स्पते भव भव॒ वन॑स्पते श॒तव॑ल्.शः श॒तव॑ल्.शो॒ वन॑स्पते भव भव॒ वन॑स्पते श॒तव॑ल्.शः । \newline
44. वन॑स्पते श॒तव॑ल्.शः श॒तव॑ल्.शो॒ वन॑स्पते॒ वन॑स्पते श॒तव॑ल्.शो॒ वि वि श॒तव॑ल्.शो॒ वन॑स्पते॒ वन॑स्पते श॒तव॑ल्.शो॒ वि । \newline
45. श॒तव॑ल्.शो॒ वि वि श॒तव॑ल्.शः श॒तव॑ल्.शो॒ वि रो॑ह रोह॒ वि श॒तव॑ल्.शः श॒तव॑ल्.शो॒ वि रो॑ह । \newline
46. श॒तव॑ल्.श॒ इति॑ श॒त - व॒ल्॒.शः॒ । \newline
47. वि रो॑ह रोह॒ वि वि रो॑ह स॒हस्र॑वल्.शाः स॒हस्र॑वल्.शा रोह॒ वि वि रो॑ह स॒हस्र॑वल्.शाः । \newline
48. रो॒ह॒ स॒हस्र॑वल्.शाः स॒हस्र॑वल्.शा रोह रोह स॒हस्र॑वल्.शा॒ वि वि स॒हस्र॑वल्.शा रोह रोह स॒हस्र॑वल्.शा॒ वि । \newline
49. स॒हस्र॑वल्.शा॒ वि वि स॒हस्र॑वल्.शाः स॒हस्र॑वल्.शा॒ वि व॒यं ॅव॒यं ॅवि स॒हस्र॑वल्.शाः स॒हस्र॑वल्.शा॒ वि व॒यम् । \newline
50. स॒हस्र॑वल्.शा॒ इति॑ स॒हस्र॑ - व॒ल्॒.शाः॒ । \newline
51. वि व॒यं ॅव॒यं ॅवि वि व॒यꣳ रु॑हेम रुहेम व॒यं ॅवि वि व॒यꣳ रु॑हेम । \newline
52. व॒यꣳ रु॑हेम रुहेम व॒यं ॅव॒यꣳ रु॑हेम॒ यं ॅयꣳ रु॑हेम व॒यं ॅव॒यꣳ रु॑हेम॒ यम् । \newline
53. रु॒हे॒म॒ यं ॅयꣳ रु॑हेम रुहेम॒ यम् त्वा᳚ त्वा॒ यꣳ रु॑हेम रुहेम॒ यम् त्वा᳚ । \newline
54. यम् त्वा᳚ त्वा॒ यं ॅयम् त्वा॒ ऽय म॒यम् त्वा॒ यं ॅयम् त्वा॒ ऽयम् । \newline
55. त्वा॒ ऽय म॒यम् त्वा᳚ त्वा॒ ऽयꣳ स्वधि॑तिः॒ स्वधि॑ति र॒यम् त्वा᳚ त्वा॒ ऽयꣳ स्वधि॑तिः । \newline
56. अ॒यꣳ स्वधि॑तिः॒ स्वधि॑ति र॒य म॒यꣳ स्वधि॑ति॒ स्तेति॑जान॒ स्तेति॑जानः॒ स्वधि॑ति र॒य म॒यꣳ स्वधि॑ति॒ स्तेति॑जानः । \newline
57. स्वधि॑ति॒ स्तेति॑जान॒ स्तेति॑जानः॒ स्वधि॑तिः॒ स्वधि॑ति॒ स्तेति॑जानः प्रणि॒नाय॑ प्रणि॒नाय॒ तेति॑जानः॒ स्वधि॑तिः॒ स्वधि॑ति॒ स्तेति॑जानः प्रणि॒नाय॑ । \newline
58. स्वधि॑ति॒रिति॒ स्व - धि॒तिः॒ । \newline
59. तेति॑जानः प्रणि॒नाय॑ प्रणि॒नाय॒ तेति॑जान॒ स्तेति॑जानः प्रणि॒नाय॑ मह॒ते म॑ह॒ते प्र॑णि॒नाय॒ तेति॑जान॒ स्तेति॑जानः प्रणि॒नाय॑ मह॒ते । \newline
60. प्र॒णि॒नाय॑ मह॒ते म॑ह॒ते प्र॑णि॒नाय॑ प्रणि॒नाय॑ मह॒ते सौभ॑गाय॒ सौभ॑गाय मह॒ते प्र॑णि॒नाय॑ प्रणि॒नाय॑ मह॒ते सौभ॑गाय । \newline
61. प्र॒णि॒नायेति॑ प्र - नि॒नाय॑ । \newline
62. म॒ह॒ते सौभ॑गाय॒ सौभ॑गाय मह॒ते म॑ह॒ते सौभ॑गा॒याच्छि॒न्नो ऽच्छि॑न्नः॒ सौभ॑गाय मह॒ते म॑ह॒ते सौभ॑गा॒याच्छि॑न्नः । \newline
63. सौभ॑गा॒याच्छि॒न्नो ऽच्छि॑न्नः॒ सौभ॑गाय॒ सौभ॑गा॒याच्छि॑न्नो॒ रायो॒ रायो ऽच्छि॑न्नः॒ सौभ॑गाय॒ सौभ॑गा॒याच्छि॑न्नो॒ रायः॑ । \newline
64. अच्छि॑न्नो॒ रायो॒ रायो ऽच्छि॒न्नो ऽच्छि॑न्नो॒ रायः॑ सु॒वीरः॑ सु॒वीरो॒ रायो ऽच्छि॒न्नो ऽच्छि॑न्नो॒ रायः॑ सु॒वीरः॑ । \newline
65. रायः॑ सु॒वीरः॑ सु॒वीरो॒ रायो॒ रायः॑ सु॒वीरः॑ । \newline
66. सु॒वीर॒ इति॑ सु - वीरः॑ । \newline
\pagebreak
\markright{ TS 1.3.6.1  \hfill https://www.vedavms.in \hfill}

\section{ TS 1.3.6.1 }

\textbf{TS 1.3.6.1 } \newline
\textbf{Samhita Paata} \newline

पृ॒थि॒व्यै त्वा॒न्तरि॑क्षाय त्वा दि॒वे त्वा॒ शुन्ध॑तां ॅलो॒कः पि॑तृ॒षद॑नो॒ यवो॑ऽसि य॒वया॒स्मद् द्वेषो॑ य॒वयारा॑तीः पितृ॒णाꣳ सद॑नमसि स्वावे॒शो᳚-ऽस्यग्रे॒गा ने॑तृ॒णां ॅवन॒स्पति॒रधि॑ त्वा स्थास्यति॒ तस्य॑ वित्ताद् दे॒वस्त्वा॑ सवि॒ता मद्ध्वा॑ऽनक्तु सुपिप्प॒लाभ्य॒-स्त्वौष॑धीभ्य॒ उद्दिवꣳ॑ स्तभा॒नाऽन्तरि॑क्षं पृण पृथि॒वीमुप॑रेण दृꣳह॒ ते ते॒ धामा᳚न्युश्मसी - [ ] \newline

\textbf{Pada Paata} \newline

पृ॒थि॒व्यै । त्वा॒ । अ॒न्तरि॑क्षाय । त्वा॒ । दि॒वे । त्वा॒ । शुन्ध॑ताम् । लो॒कः । पि॒तृ॒षद॑न॒ इति॑ पितृ - सद॑नः । यवः॑ । अ॒सि॒ । य॒वय॑ । अ॒स्मत् । द्वेषः॑ । य॒वय॑ । अरा॑तीः । पि॒तृ॒णाम् । सद॑नम् । अ॒सि॒ । स्वा॒वे॒श इति॑ सु - आ॒वे॒शः । अ॒सि॒ । अ॒ग्रे॒गा इत्य॑ग्रे - गाः । ने॒तृ॒णाम् । वन॒स्पतिः॑ । अधीति॑ । त्वा॒ । स्था॒स्य॒ति॒ । तस्य॑ । वि॒त्ता॒त् । दे॒वः । त्वा॒ । स॒वि॒ता । मद्ध्वा᳚ । अ॒न॒क्तु॒ । सु॒पि॒प्प॒लाभ्य॒ इति॑ सु - पि॒प्प॒लाभ्यः॑ । त्वा॒ । ओष॑धीभ्य॒ इत्योष॑ध - भ्यः॒ । उदिति॑ । दिव᳚म् । स्त॒भा॒न॒ । एति॑ । अ॒न्तरि॑क्षम् । पृ॒ण॒ । पृ॒थि॒वीम् । उप॑रेण । दृꣳ॒॒ह॒ । ते । ते॒ । धामा॑नि । उ॒श्म॒सि॒ ।  \newline


\textbf{Krama Paata} \newline

पृ॒थि॒व्यै त्वा᳚ । त्वा॒ऽन्तरि॑क्षाय । अ॒न्तरि॑क्षाय त्वा । त्वा॒ दि॒वे । दि॒वे त्वा᳚ । त्वा॒ शुन्ध॑ताम् । शुन्ध॑तां ॅलो॒कः । लो॒कः पि॑तृ॒षद॑नः । पि॒तृ॒षद॑नो॒ यवः॑ । पि॒तृ॒षद॑न॒ इति॑ पितृ - सद॑नः । यवो॑ऽसि । अ॒सि॒ य॒वय॑ । य॒वया॒स्मत् । अ॒स्मद् द्वेषः॑ । द्वेषो॑ य॒वय॑ । य॒वयारा॑तीः । अरा॑तीः पितृ॒णाम् । पि॒तृ॒णाꣳ सद॑नम् । सद॑नमसि । अ॒सि॒ स्वा॒वे॒शः । स्वा॒वे॒शो॑ ऽसि । स्वा॒वे॒श इति॑ सु - आ॒वे॒शः । अ॒स्य॒ग्रे॒गाः । अ॒ग्रे॒गा ने॑तृ॒णाम् । अ॒ग्रे॒गा इत्य॑ग्रे - गाः । ने॒तृ॒णां ॅवन॒स्पतिः॑ । वन॒स्पति॒रधि॑ । अधि॑ त्वा । त्वा॒ स्था॒स्य॒ति॒ । स्था॒स्य॒ति॒ तस्य॑ । तस्य॑ वित्तात् । वि॒त्ता॒द् दे॒वः । दे॒वस्त्वा᳚ । त्वा॒ स॒वि॒ता । स॒वि॒ता मद्ध्वा᳚ । मद्ध्वा॑ऽनक्तु । अ॒न॒क्तु॒ सु॒पि॒प्प॒लाभ्यः॑ । सु॒पि॒प्प॒लाभ्य॑स्त्वा । सु॒पि॒प्प॒लाभ्य॒ इति॑ सु - पि॒प्प॒लाभ्यः॑ । त्वौष॑धीभ्यः । ओष॑धीभ्य॒ उत् । ओष॑धीभ्य॒ इत्योष॑धि - भ्यः॒ । उद् दिव᳚म् । दिवꣳ॑ स्तभान । स्त॒भा॒ना । आऽन्तरि॑क्षम् । अ॒न्तरि॑क्षम् पृण । पृ॒ण॒ पृ॒थि॒वीम् । पृ॒थि॒वीमुप॑रेण । उप॑रेण दृꣳह । दृꣳ॒॒ह॒ ते । ते ते᳚ । ते॒ धामा॑नि । धामा᳚न्युश्मसि । उ॒श्म॒सी॒ ग॒मद्ध्ये᳚ \newline

\textbf{Jatai Paata} \newline

1. पृ॒थि॒व्यै त्वा᳚ त्वा पृथि॒व्यै पृ॑थि॒व्यै त्वा᳚ । \newline
2. त्वा॒ ऽन्तरि॑क्षा या॒न्तरि॑क्षाय त्वा त्वा॒ ऽन्तरि॑क्षाय । \newline
3. अ॒न्तरि॑क्षाय त्वा त्वा॒ ऽन्तरि॑क्षा या॒न्तरि॑क्षाय त्वा । \newline
4. त्वा॒ दि॒वे दि॒वे त्वा᳚ त्वा दि॒वे । \newline
5. दि॒वे त्वा᳚ त्वा दि॒वे दि॒वे त्वा᳚ । \newline
6. त्वा॒ शुन्ध॑ता॒(ग्म्॒) शुन्ध॑ताम् त्वा त्वा॒ शुन्ध॑ताम् । \newline
7. शुन्ध॑ताम् ॅलो॒को लो॒कः शुन्ध॑ता॒(ग्म्॒) शुन्ध॑ताम् ॅलो॒कः । \newline
8. लो॒कः पि॑तृ॒षद॑नः पितृ॒षद॑नो लो॒को लो॒कः पि॑तृ॒षद॑नः । \newline
9. पि॒तृ॒षद॑नो॒ यवो॒ यवः॑ पितृ॒षद॑नः पितृ॒षद॑नो॒ यवः॑ । \newline
10. पि॒तृ॒षद॑न॒ इति॑ पितृ - सद॑नः । \newline
11. यवो᳚ ऽस्यसि॒ यवो॒ यवो॑ ऽसि । \newline
12. अ॒सि॒ य॒वय॑ य॒व या᳚स्यसि य॒वय॑ । \newline
13. य॒व या॒स्म द॒स्मद् य॒वय॑ य॒व या॒स्मत् । \newline
14. अ॒स्मद् द्वेषो॒ द्वेषो॒ ऽस्म द॒स्मद् द्वेषः॑ । \newline
15. द्वेषो॑ य॒वय॑ य॒वय॒ द्वेषो॒ द्वेषो॑ य॒वय॑ । \newline
16. य॒वयारा॑ती॒र रा॑तीर् य॒वय॑ य॒वयारा॑तीः । \newline
17. अरा॑तीः पितृ॒णाम् पि॑तृ॒णा मरा॑ती॒ ररा॑तीः पितृ॒णाम् । \newline
18. पि॒तृ॒णाꣳ सद॑न॒(ग्म्॒) सद॑नम् पितृ॒णाम् पि॑तृ॒णाꣳ सद॑नम् । \newline
19. सद॑न मस्यसि॒ सद॑न॒(ग्म्॒) सद॑न मसि । \newline
20. अ॒सि॒ स्वा॒वे॒शः स्वा॑वे॒शो᳚ ऽस्यसि स्वावे॒शः । \newline
21. स्वा॒वे॒शो᳚ ऽस्यसि स्वावे॒शः स्वा॑वे॒शो॑ ऽसि । \newline
22. स्वा॒वे॒श इति॑ सु - आ॒वे॒शः । \newline
23. अ॒स्य॒ग्रे॒गा अ॑ग्रे॒गा अ॑स्य स्यग्रे॒गाः । \newline
24. अ॒ग्रे॒गा ने॑तृ॒णाम् ने॑तृ॒णा म॑ग्रे॒गा अ॑ग्रे॒गा ने॑तृ॒णाम् । \newline
25. अ॒ग्रे॒गा इत्य॑ग्रे - गाः । \newline
26. ने॒तृ॒णां ॅवन॒स्पति॒र् वन॒स्पति॑र् नेतृ॒णाम् ने॑तृ॒णां ॅवन॒स्पतिः॑ । \newline
27. वन॒स्पति॒ रध्यधि॒ वन॒स्पति॒र् वन॒स्पति॒ रधि॑ । \newline
28. अधि॑ त्वा॒ त्वा ऽध्यधि॑ त्वा । \newline
29. त्वा॒ स्था॒स्य॒ति॒ स्था॒स्य॒ति॒ त्वा॒ त्वा॒ स्था॒स्य॒ति॒ । \newline
30. स्था॒स्य॒ति॒ तस्य॒ तस्य॑ स्थास्यति स्थास्यति॒ तस्य॑ । \newline
31. तस्य॑ वित्ताद् वित्ता॒त् तस्य॒ तस्य॑ वित्तात् । \newline
32. वि॒त्ता॒द् दे॒वो दे॒वो वि॑त्ताद् वित्ताद् दे॒वः । \newline
33. दे॒वस्त्वा᳚ त्वा दे॒वो दे॒वस्त्वा᳚ । \newline
34. त्वा॒ स॒वि॒ता स॑वि॒ता त्वा᳚ त्वा सवि॒ता । \newline
35. स॒वि॒ता मद्ध्वा॒ मद्ध्वा॑ सवि॒ता स॑वि॒ता मद्ध्वा᳚ । \newline
36. मद्ध्वा॑ ऽनक्त्वनक्तु॒ मद्ध्वा॒ मद्ध्वा॑ ऽनक्तु । \newline
37. अ॒न॒क्तु॒ सु॒पि॒प्प॒लाभ्यः॑ सुपिप्प॒लाभ्यो॑ ऽनक्त्वनक्तु सुपिप्प॒लाभ्यः॑ । \newline
38. सु॒पि॒प्प॒लाभ्य॑ स्त्वा त्वा सुपिप्प॒लाभ्यः॑ सुपिप्प॒लाभ्य॑ स्त्वा । \newline
39. सु॒पि॒प्प॒लाभ्य॒ इति॑ सु - पि॒प्प॒लाभ्यः॑ । \newline
40. त्वौष॑धीभ्य॒ ओष॑धी भ्यस्त्वा॒ त्वौष॑धीभ्यः । \newline
41. ओष॑धीभ्य॒ उदु दोष॑धीभ्य॒ ओष॑धीभ्य॒ उत् । \newline
42. ओष॑धीभ्य॒ इत्योष॑धि - भ्यः॒ । \newline
43. उद् दिव॒म् दिव॒ मुदुद् दिव᳚म् । \newline
44. दिव(ग्ग्॑) स्तभान स्तभान॒ दिव॒म् दिव(ग्ग्॑) स्तभान । \newline
45. स्त॒भा॒ना स्त॑भान स्तभा॒ना । \newline
46. आ ऽन्तरि॑क्ष म॒न्तरि॑क्ष॒ मा ऽन्तरि॑क्षम् । \newline
47. अ॒न्तरि॑क्षम् पृण पृणा॒न्तरि॑क्ष म॒न्तरि॑क्षम् पृण । \newline
48. पृ॒ण॒ पृ॒थि॒वीम् पृ॑थि॒वीम् पृ॑ण पृण पृथि॒वीम् । \newline
49. पृ॒थि॒वी मुप॑रे॒ णोप॑रेण पृथि॒वीम् पृ॑थि॒वी मुप॑रेण । \newline
50. उप॑रेण दृꣳह दृ॒(ग्म्॒) होप॑रे॒ णोप॑रेण दृꣳह । \newline
51. दृ॒(ग्म्॒)ह॒ ते ते दृ(ग्म्॑)ह दृꣳह॒ ते । \newline
52. ते ते॑ ते॒ ते ते ते᳚ । \newline
53. ते॒ धामा॑नि॒ धामा॑नि ते ते॒ धामा॑नि । \newline
54. धामा᳚ न्युश्म स्युश्मसि॒ धामा॑नि॒ धामा᳚ न्युश्मसि । \newline
55. उ॒श्म॒सी॒ ग॒मद्ध्ये॑ ग॒मद्ध्य॑ उश्म स्युश्मसी ग॒मद्ध्ये᳚ । \newline

\textbf{Ghana Paata } \newline

1. पृ॒थि॒व्यै त्वा᳚ त्वा पृथि॒व्यै पृ॑थि॒व्यै त्वा॒ ऽन्तरि॑क्षाया॒न्तरि॑क्षाय त्वा पृथि॒व्यै पृ॑थि॒व्यै त्वा॒ ऽन्तरि॑क्षाय । \newline
2. त्वा॒ ऽन्तरि॑क्षाया॒न्तरि॑क्षाय त्वा त्वा॒ ऽन्तरि॑क्षाय त्वा त्वा॒ ऽन्तरि॑क्षाय त्वा त्वा॒ ऽन्तरि॑क्षाय त्वा । \newline
3. अ॒न्तरि॑क्षाय त्वा त्वा॒ ऽन्तरि॑क्षाया॒न्तरि॑क्षाय त्वा दि॒वे दि॒वे त्वा॒ ऽन्तरि॑क्षाया॒न्तरि॑क्षाय त्वा दि॒वे । \newline
4. त्वा॒ दि॒वे दि॒वे त्वा᳚ त्वा दि॒वे त्वा᳚ त्वा दि॒वे त्वा᳚ त्वा दि॒वे त्वा᳚ । \newline
5. दि॒वे त्वा᳚ त्वा दि॒वे दि॒वे त्वा॒ शुन्ध॑ता॒(ग्म्॒) शुन्ध॑ताम् त्वा दि॒वे दि॒वे त्वा॒ शुन्ध॑ताम् । \newline
6. त्वा॒ शुन्ध॑ता॒(ग्म्॒) शुन्ध॑ताम् त्वा त्वा॒ शुन्ध॑ताम् ॅलो॒को लो॒कः शुन्ध॑ताम् त्वा त्वा॒ शुन्ध॑ताम् ॅलो॒कः । \newline
7. शुन्ध॑ताम् ॅलो॒को लो॒कः शुन्ध॑ता॒(ग्म्॒) शुन्ध॑ताम् ॅलो॒कः पि॑तृ॒षद॑नः पितृ॒षद॑नो लो॒कः शुन्ध॑ता॒(ग्म्॒) शुन्ध॑ताम् ॅलो॒कः पि॑तृ॒षद॑नः । \newline
8. लो॒कः पि॑तृ॒षद॑नः पितृ॒षद॑नो लो॒को लो॒कः पि॑तृ॒षद॑नो॒ यवो॒ यवः॑ पितृ॒षद॑नो लो॒को लो॒कः पि॑तृ॒षद॑नो॒ यवः॑ । \newline
9. पि॒तृ॒षद॑नो॒ यवो॒ यवः॑ पितृ॒षद॑नः पितृ॒षद॑नो॒ यवो᳚ ऽस्यसि॒ यवः॑ पितृ॒षद॑नः पितृ॒षद॑नो॒ यवो॑ ऽसि । \newline
10. पि॒तृ॒षद॑न॒ इति॑ पितृ - सद॑नः । \newline
11. यवो᳚ ऽस्यसि॒ यवो॒ यवो॑ ऽसि य॒वय॑ य॒वया॑सि॒ यवो॒ यवो॑ ऽसि य॒वय॑ । \newline
12. अ॒सि॒ य॒वय॑ य॒वया᳚ स्यसि य॒वया॒स्म द॒स्मद् य॒वया᳚स्यसि य॒वया॒स्मत् । \newline
13. य॒वया॒स्म द॒स्मद् य॒वय॑ य॒वया॒स्मद् द्वेषो॒ द्वेषो॒ ऽस्मद् य॒वय॑ य॒वया॒स्मद् द्वेषः॑ । \newline
14. अ॒स्मद् द्वेषो॒ द्वेषो॒ ऽस्म द॒स्मद् द्वेषो॑ य॒वय॑ य॒वय॒ द्वेषो॒ ऽस्म द॒स्मद् द्वेषो॑ य॒वय॑ । \newline
15. द्वेषो॑ य॒वय॑ य॒वय॒ द्वेषो॒ द्वेषो॑ य॒वयारा॑ती॒ ररा॑तीर् य॒वय॒ द्वेषो॒ द्वेषो॑ य॒वयारा॑तीः । \newline
16. य॒वयारा॑ती॒ ररा॑तीर् य॒वय॑ य॒वयारा॑तीः पितृ॒णाम् पि॑तृ॒णा मरा॑तीर् य॒वय॑ य॒वयारा॑तीः पितृ॒णाम् । \newline
17. अरा॑तीः पितृ॒णाम् पि॑तृ॒णा मरा॑ती॒ ररा॑तीः पितृ॒णाꣳ सद॑न॒(ग्म्॒) सद॑नम् पितृ॒णा मरा॑ती॒ ररा॑तीः पितृ॒णाꣳ सद॑नम् । \newline
18. पि॒तृ॒णाꣳ सद॑न॒(ग्म्॒) सद॑नम् पितृ॒णाम् पि॑तृ॒णाꣳ सद॑न मस्यसि॒ सद॑नम् पितृ॒णाम् पि॑तृ॒णाꣳ सद॑न मसि । \newline
19. सद॑न मस्यसि॒ सद॑न॒(ग्म्॒) सद॑न मसि स्वावे॒शः स्वा॑वे॒शो॑ ऽसि॒ सद॑न॒(ग्म्॒) सद॑न मसि स्वावे॒शः । \newline
20. अ॒सि॒ स्वा॒वे॒शः स्वा॑वे॒शो᳚ ऽस्यसि स्वावे॒शो᳚ ऽस्यसि स्वावे॒शो᳚ ऽस्यसि स्वावे॒शो॑ ऽसि । \newline
21. स्वा॒वे॒शो᳚ ऽस्यसि स्वावे॒शः स्वा॑वे॒शो᳚ ऽस्यग्रे॒गा अ॑ग्रे॒गा अ॑सि स्वावे॒शः स्वा॑वे॒शो᳚ ऽस्यग्रे॒गाः । \newline
22. स्वा॒वे॒श इति॑ सु - आ॒वे॒शः । \newline
23. अ॒स्य॒ ग्रे॒गा अ॑ग्रे॒गा अ॑स्यस्यग्रे॒गा ने॑तृ॒णाम् ने॑तृ॒णा म॑ग्रे॒गा अ॑स्यस्यग्रे॒गा ने॑तृ॒णाम् । \newline
24. अ॒ग्रे॒गा ने॑तृ॒णाम् ने॑तृ॒णा म॑ग्रे॒गा अ॑ग्रे॒गा ने॑तृ॒णां ॅवन॒स्पति॒र् वन॒स्पति॑र् नेतृ॒णा म॑ग्रे॒गा अ॑ग्रे॒गा ने॑तृ॒णां ॅवन॒स्पतिः॑ । \newline
25. अ॒ग्रे॒गा इत्य॑ग्रे - गाः । \newline
26. ने॒तृ॒णां ॅवन॒स्पति॒र् वन॒स्पति॑र् नेतृ॒णाम् ने॑तृ॒णां ॅवन॒स्पति॒ रध्यधि॒ वन॒स्पति॑र् नेतृ॒णाम् ने॑तृ॒णां ॅवन॒स्पति॒ रधि॑ । \newline
27. वन॒स्पति॒ रध्यधि॒ वन॒स्पति॒र् वन॒स्पति॒ रधि॑ त्वा॒ त्वा ऽधि॒ वन॒स्पति॒र् वन॒स्पति॒ रधि॑ त्वा । \newline
28. अधि॑ त्वा॒ त्वा ऽध्यधि॑ त्वा स्थास्यति स्थास्यति॒ त्वा ऽध्यधि॑ त्वा स्थास्यति । \newline
29. त्वा॒ स्था॒स्य॒ति॒ स्था॒स्य॒ति॒ त्वा॒ त्वा॒ स्था॒स्य॒ति॒ तस्य॒ तस्य॑ स्थास्यति त्वा त्वा स्थास्यति॒ तस्य॑ । \newline
30. स्था॒स्य॒ति॒ तस्य॒ तस्य॑ स्थास्यति स्थास्यति॒ तस्य॑ वित्ताद् वित्ता॒त् तस्य॑ स्थास्यति स्थास्यति॒ तस्य॑ वित्तात् । \newline
31. तस्य॑ वित्ताद् वित्ता॒त् तस्य॒ तस्य॑ वित्ताद् दे॒वो दे॒वो वि॑त्ता॒त् तस्य॒ तस्य॑ वित्ताद् दे॒वः । \newline
32. वि॒त्ता॒द् दे॒वो दे॒वो वि॑त्ताद् वित्ताद् दे॒वस्त्वा᳚ त्वा दे॒वो वि॑त्ताद् वित्ताद् दे॒वस्त्वा᳚ । \newline
33. दे॒वस्त्वा᳚ त्वा दे॒वो दे॒वस्त्वा॑ सवि॒ता स॑वि॒ता त्वा॑ दे॒वो दे॒वस्त्वा॑ सवि॒ता । \newline
34. त्वा॒ स॒वि॒ता स॑वि॒ता त्वा᳚ त्वा सवि॒ता मद्ध्वा॒ मद्ध्वा॑ सवि॒ता त्वा᳚ त्वा सवि॒ता मद्ध्वा᳚ । \newline
35. स॒वि॒ता मद्ध्वा॒ मद्ध्वा॑ सवि॒ता स॑वि॒ता मद्ध्वा॑ ऽनक्त्वनक्तु॒ मद्ध्वा॑ सवि॒ता स॑वि॒ता मद्ध्वा॑ ऽनक्तु । \newline
36. मद्ध्वा॑ ऽनक्त्वनक्तु॒ मद्ध्वा॒ मद्ध्वा॑ ऽनक्तु सुपिप्प॒लाभ्यः॑ सुपिप्प॒लाभ्यो॑ ऽनक्तु॒ मद्ध्वा॒ मद्ध्वा॑ ऽनक्तु सुपिप्प॒लाभ्यः॑ । \newline
37. अ॒न॒क्तु॒ सु॒पि॒प्प॒लाभ्यः॑ सुपिप्प॒लाभ्यो॑ ऽनक्त्वनक्तु सुपिप्प॒लाभ्य॑स्त्वा त्वा सुपिप्प॒लाभ्यो॑ ऽनक्त्वनक्तु सुपिप्प॒लाभ्य॑स्त्वा । \newline
38. सु॒पि॒प्प॒लाभ्य॑स्त्वा त्वा सुपिप्प॒लाभ्यः॑ सुपिप्प॒लाभ्य॒ स्त्वौष॑धीभ्य॒ ओष॑धीभ्यस्त्वा सुपिप्प॒लाभ्यः॑ सुपिप्प॒लाभ्य॒ स्त्वौष॑धीभ्यः । \newline
39. सु॒पि॒प्प॒लाभ्य॒ इति॑ सु - पि॒प्प॒लाभ्यः॑ । \newline
40. त्वौष॑धीभ्य॒ ओष॑धीभ्यस्त्वा॒ त्वौष॑धीभ्य॒ उदु दोष॑धीभ्य स्त्वा॒ त्वौष॑धीभ्य॒ उत् । \newline
41. ओष॑धीभ्य॒ उदु दोष॑धीभ्य॒ ओष॑धीभ्य॒ उद् दिव॒म् दिव॒ मुदोष॑धीभ्य॒ ओष॑धीभ्य॒ उद् दिव᳚म् । \newline
42. ओष॑धीभ्य॒ इत्योष॑धि - भ्यः॒ । \newline
43. उद् दिव॒म् दिव॒ मुदुद् दिव(ग्ग्॑) स्तभान स्तभान॒ दिव॒ मुदुद् दिव(ग्ग्॑) स्तभान । \newline
44. दिव(ग्ग्॑) स्तभान स्तभान॒ दिव॒म् दिव(ग्ग्॑) स्तभा॒ना स्त॑भान॒ दिव॒म् दिव(ग्ग्॑) स्तभा॒ना । \newline
45. स्त॒भा॒ना स्त॑भान स्तभा॒ना ऽन्तरि॑क्ष म॒न्तरि॑क्ष॒ मा स्त॑भान स्तभा॒ना ऽन्तरि॑क्षम् । \newline
46. आ ऽन्तरि॑क्ष म॒न्तरि॑क्ष॒ मा ऽन्तरि॑क्षम् पृण पृणा॒न्तरि॑क्ष॒ मा ऽन्तरि॑क्षम् पृण । \newline
47. अ॒न्तरि॑क्षम् पृण पृणा॒न्तरि॑क्ष म॒न्तरि॑क्षम् पृण पृथि॒वीम् पृ॑थि॒वीम् पृ॑णा॒न्तरि॑क्ष म॒न्तरि॑क्षम् पृण पृथि॒वीम् । \newline
48. पृ॒ण॒ पृ॒थि॒वीम् पृ॑थि॒वीम् पृ॑ण पृण पृथि॒वी मुप॑रे॒णोप॑रेण पृथि॒वीम् पृ॑ण पृण पृथि॒वी मुप॑रेण । \newline
49. पृ॒थि॒वी मुप॑रे॒णोप॑रेण पृथि॒वीम् पृ॑थि॒वी मुप॑रेण दृꣳह दृ॒(ग्म्॒)होप॑रेण पृथि॒वीम् पृ॑थि॒वी मुप॑रेण दृꣳह । \newline
50. उप॑रेण दृꣳह दृ॒(ग्म्॒)होप॑रे॒णोप॑रेण दृꣳह॒ ते ते दृ॒(ग्म्॒)होप॑रे॒णोप॑रेण दृꣳह॒ ते । \newline
51. दृ॒(ग्म्॒)ह॒ ते ते दृ(ग्म्॑)ह दृꣳह॒ ते ते॑ ते॒ ते दृ(ग्म्॑)ह दृꣳह॒ ते ते᳚ । \newline
52. ते ते॑ ते॒ ते ते ते॒ धामा॑नि॒ धामा॑नि ते॒ ते ते ते॒ धामा॑नि । \newline
53. ते॒ धामा॑नि॒ धामा॑नि ते ते॒ धामा᳚ न्युश्म स्युश्मसि॒ धामा॑नि ते ते॒ धामा᳚ न्युश्मसि । \newline
54. धामा᳚ न्युश्म स्युश्मसि॒ धामा॑नि॒ धामा᳚ न्युश्मसी ग॒मद्ध्ये॑ ग॒मद्ध्य॑ उश्मसि॒ धामा॑नि॒ धामा᳚ 
न्युश्मसी ग॒मद्ध्ये᳚ । \newline
55. उ॒श्म॒सी॒ ग॒मद्ध्ये॑ ग॒मद्ध्य॑ उश्म स्युश्मसी ग॒मद्ध्ये॒ गावो॒ गावो॑ ग॒मद्ध्य॑ उश्म स्युश्मसी ग॒मद्ध्ये॒ गावः॑ । \newline
\pagebreak
\markright{ TS 1.3.6.2  \hfill https://www.vedavms.in \hfill}

\section{ TS 1.3.6.2 }

\textbf{TS 1.3.6.2 } \newline
\textbf{Samhita Paata} \newline

ग॒मद्ध्ये॒ गावो॒ यत्र॒ भूरि॑शृङ्गा अ॒यासः॑ । अत्राह॒ तदु॑रुगा॒यस्य॒ विष्णोः᳚ प॒रमं प॒दमव॑ भाति॒ भूरेः᳚ ॥ विष्णोः॒ कर्मा॑णि पश्यत॒ यतो᳚ व्र॒तानि॑ पस्प॒शे । इन्द्र॑स्य॒ युज्यः॒ सखा᳚ ॥ तद्-विष्णोः᳚ पर॒मं प॒दꣳ सदा॑ पश्यन्ति सू॒रयः॑ । दि॒वीव॒ चक्षु॒रात॑तं ॥ ब्र॒ह्म॒वनिं॑ त्वा क्षत्र॒वनिꣳ॑ सुप्रजा॒वनिꣳ॑ रायस्पोष॒वनिं॒ पर्यू॑हामि॒ ब्रह्म॑ दृꣳह क्ष॒त्रं दृꣳ॑ह प्र॒जां दृꣳ॑ह रा॒यस्पोषं॑ ( ) दृꣳह परि॒वीर॑सि॒ परि॑ त्वा॒ दैवी॒र्विशो᳚ व्ययन्तां॒ परी॒मꣳ रा॒यस्पोषो॒ यज॑मानं मनु॒ष्या॑ अ॒न्तरि॑क्षस्य त्वा॒ साना॒वव॑ गूहामि ॥ \newline

\textbf{Pada Paata} \newline

ग॒मद्ध्ये᳚ । गावः॑ । यत्र॑ । भूरि॑शृङ्गा॒ इति॒ भूरि॑ - शृ॒ङ्गाः॒ । अ॒यासः॑ ॥ अत्र॑ । अह॑ । तत् । उ॒रु॒गा॒यस्येत्यु॑रु - गा॒यस्य॑ । विष्णोः᳚ । प॒र॒मम् । प॒दम् । अवेति॑ । भा॒ति॒ । भूरेः᳚ ॥ विष्णोः᳚ । कर्मा॑णि । प॒श्य॒त॒ । यतः॑ । व्र॒तानि॑ । प॒स्प॒शे ॥ इन्द्र॑स्य । युज्यः॑ । सखा᳚ ॥ तत् । विष्णोः᳚ । प॒र॒मम् । प॒दम् । सदा᳚ । प॒श्य॒न्ति॒ । सू॒रयः॑ ॥ दि॒वि । इ॒व॒ । चक्षुः॑ । आत॑त॒मित्या - त॒त॒म् ॥ ब्र॒ह्म॒वनि॒मिति॑ ब्रह्म - वनि᳚म् । त्वा॒ । क्ष॒त्र॒वनि॒मिति॑ क्षत्र - वनि᳚म् । सु॒प्र॒जा॒वनि॒मिति॑ सुप्रजा - वनि᳚म् । रा॒य॒स्पो॒ष॒वनि॒मिति॑ रायस्पोष - वनि᳚म् । परीति॑ । ऊ॒हा॒मि॒ । ब्रह्म॑ । दृꣳ॒॒ह॒ । क्ष॒त्रम् । दृꣳ॒॒ह॒ । प्र॒जामिति॑ प्र - जाम् । दृꣳ॒॒ह॒ । रा॒यः । पोष᳚म् ( ) । दृꣳ॒॒ह॒ । प॒रि॒वीरिति॑ परि - वीः । अ॒सि॒ । परीति॑ । त्वा॒ । दैवीः᳚ । विशः॑ । व्य॒य॒न्ता॒म् । परीति॑ । इ॒मम् । रा॒यः । पोषः॑ । यज॑मानम् । म॒नु॒ष्याः᳚ । अ॒न्तरि॑क्षस्य । त्वा॒ । सानौ᳚ । अवेति॑ । गू॒हा॒मि॒ ॥  \newline


\textbf{Krama Paata} \newline

ग॒मद्ध्ये॒ गावः॑ । गावो॒ यत्र॑ । यत्र॒ भूरि॑शृङ्गाः । भूरि॑शृङ्गा अ॒यासः॑ । भूरि॑शृङ्गा॒ इति॒ भूरि॑ - शृ॒ङ्गाः॒ । अ॒यास॒ इत्य॒यासः॑ ॥ अत्राह॑ । अह॒ तत् । तदु॑रुगा॒यस्य॑ । उ॒रु॒गा॒यस्य॒ विष्णोः᳚ । उ॒रु॒गा॒यस्येत्यु॑रु - गा॒यस्य॑ । विष्णोः᳚ पर॒मम् । प॒र॒मम् प॒दम् । प॒दमव॑ । अव॑ भाति । भा॒ति॒ भूरेः᳚ । भूरे॒रिति॒ भूरेः᳚ ॥ विष्णोः॒ कर्मा॑णि । कर्मा॑णि पश्यत । प॒श्य॒त॒ यतः॑ । यतो᳚ व्र॒तानि॑ । व्र॒तानि॑ पस्प॒शे । प॒स्प॒श इति॑ पस्प॒शे ॥ इन्द्र॑स्य॒ युज्यः॑ । युज्यः॒ सखा᳚ । सखेति॒ सखा᳚ ॥ तद् विष्णोः᳚ । विष्णोः᳚ पर॒मम् । प॒र॒मम् प॒दम् । प॒दꣳ सदा᳚ । सदा॑ पश्यन्ति । प॒श्य॒न्ति॒ सू॒रयः॑ । सू॒रय॒ इति॑ सू॒रयः॑ ॥ दि॒वीव॑ । इ॒व॒ चक्षुः॑ । चक्षु॒रात॑तम् । आत॑त॒मित्या - त॒त॒म् ॥ ब्र॒ह्म॒वनि॑म् त्वा । ब्र॒ह्म॒वनि॒मिति॑ ब्रह्म - वनि᳚म् । त्वा॒ क्ष॒त्र॒वनि᳚म् । क्ष॒त्र॒वनिꣳ॑ सुप्रजा॒वनि᳚म् । क्ष॒त्र॒वनि॒मिति॑ क्षत्र - वनि᳚म् । सु॒प्र॒जा॒वनिꣳ॑ रायस्पोष॒वनि᳚म् । सु॒प्र॒जा॒वनि॒मिति॑ सुप्रजा - वनि᳚म् । रा॒य॒स्पो॒ष॒वनि॒म् परि॑ । रा॒य॒स्पो॒ष॒वनि॒मिति॑ रायस्पोष - वनि᳚म् । पर्यू॑हामि । ऊ॒हा॒मि॒ ब्रह्म॑ । ब्रह्म॑ दृꣳह । दृꣳ॒॒ह॒ क्ष॒त्रम् । क्ष॒त्रम् दृꣳ॑ह । दृꣳ॒॒ह॒ प्र॒जाम् । प्र॒जाम् दृꣳ॑ह । प्र॒जामिति॑ प्र - जाम् । दृꣳ॒॒ह॒ रा॒यः । रा॒यस्पोष᳚म् ( ) । पोष॑म् दृꣳह । दृꣳ॒॒ह॒ प॒रि॒वीः । प॒रि॒वीर॑सि । प॒रि॒वीरिति॑ परि - वीः । अ॒सि॒ परि॑ । परि॑ त्वा । त्वा॒ दैवीः᳚ । दैवी॒र् विशः॑ । विशो᳚ व्ययन्ताम् । व्य॒य॒न्ता॒म् परि॑ । परी॒मम् । इ॒मꣳ रा॒यः । रा॒यस्पोषः॑ । पोषो॒ यज॑मानम् । यज॑मानम् मनु॒ष्याः᳚ । म॒नु॒ष्या॑ अ॒न्तरि॑क्षस्य । अ॒न्तरि॑क्षस्य त्वा । त्वा॒ सानौ᳚ । साना॒वव॑ । अव॑ गूहामि । गू॒हा॒मीति॑ गूहामि । \newline

\textbf{Jatai Paata} \newline

1. ग॒मद्ध्ये॒ गावो॒ गावो॑ ग॒मद्ध्ये॑ ग॒मद्ध्ये॒ गावः॑ । \newline
2. गावो॒ यत्र॒ यत्र॒ गावो॒ गावो॒ यत्र॑ । \newline
3. यत्र॒ भूरि॑शृङ्गा॒ भूरि॑शृङ्गा॒ यत्र॒ यत्र॒ भूरि॑शृङ्गाः । \newline
4. भूरि॑शृङ्गा अ॒यासो॒ ऽयासो॒ भूरि॑शृङ्गा॒ भूरि॑शृङ्गा अ॒यासः॑ । \newline
5. भूरि॑शृङ्गा॒ इति॒ भूरि॑ - शृ॒ङ्गाः॒ । \newline
6. अ॒यास॒ इत्य॒यासः॑ । \newline
7. अत्रा हाहा त्रा त्राह॑ । \newline
8. अह॒ तत् तदहाह॒ तत् । \newline
9. तदु॑ रुगा॒यस्यो॑ रुगा॒यस्य॒ तत् तदु॑ रुगा॒यस्य॑ । \newline
10. उ॒रु॒गा॒यस्य॒ विष्णो॒र् विष्णो॑ रुरुगा॒य स्यो॑रुगा॒यस्य॒ विष्णोः᳚ । \newline
11. उ॒रु॒गा॒यस्येत्यु॑रु - गा॒यस्य॑ । \newline
12. विष्णोः᳚ पर॒मम् प॑र॒मं ॅविष्णो॒र् विष्णोः᳚ पर॒मम् । \newline
13. प॒र॒मम् प॒दम् प॒दम् प॑र॒मम् प॑र॒मम् प॒दम् । \newline
14. प॒द मवाव॑ प॒दम् प॒द मव॑ । \newline
15. अव॑ भाति भा॒त्यवाव॑ भाति । \newline
16. भा॒ति॒ भूरे॒र् भूरे᳚र् भाति भाति॒ भूरेः᳚ । \newline
17. भूरे॒रिति॒ भुरेः᳚ । \newline
18. विष्णोः॒ कर्मा॑णि॒ कर्मा॑णि॒ विष्णो॒र् विष्णोः॒ कर्मा॑णि । \newline
19. कर्मा॑णि पश्यत पश्यत॒ कर्मा॑णि॒ कर्मा॑णि पश्यत । \newline
20. प॒श्य॒त॒ यतो॒ यतः॑ पश्यत पश्यत॒ यतः॑ । \newline
21. यतो᳚ व्र॒तानि॑ व्र॒तानि॒ यतो॒ यतो᳚ व्र॒तानि॑ । \newline
22. व्र॒तानि॑ पस्प॒शे प॑स्प॒शे व्र॒तानि॑ व्र॒तानि॑ पस्प॒शे । \newline
23. प॒स्प॒श इति॑ पस्प॒शे । \newline
24. इन्द्र॑स्य॒ युज्यो॒ युज्य॒ इन्द्र॒स्ये न्द्र॑स्य॒ युज्यः॑ । \newline
25. युज्यः॒ सखा॒ सखा॒ युज्यो॒ युज्यः॒ सखा᳚ । \newline
26. सखेति॒ सखा᳚ । \newline
27. तद् विष्णो॒र् विष्णो॒ स्तत् तद् विष्णोः᳚ । \newline
28. विष्णोः᳚ पर॒मम् प॑र॒मं ॅविष्णो॒र् विष्णोः᳚ पर॒मम् । \newline
29. प॒र॒मम् प॒दम् प॒दम् प॑र॒मम् प॑र॒मम् प॒दम् । \newline
30. प॒दꣳ सदा॒ सदा॑ प॒दम् प॒दꣳ सदा᳚ । \newline
31. सदा॑ पश्यन्ति पश्यन्ति॒ सदा॒ सदा॑ पश्यन्ति । \newline
32. प॒श्य॒न्ति॒ सू॒रयः॑ सू॒रयः॑ पश्यन्ति पश्यन्ति सू॒रयः॑ । \newline
33. सू॒रय॒ इति॑ सू॒रयः॑ । \newline
34. दि॒वीवे॑ व दि॒वि दि॒वीव॑ । \newline
35. इ॒व॒ चक्षु॒ श्चक्षु॑ रिवे व॒ चक्षुः॑ । \newline
36. चक्षु॒रात॑त॒ मात॑त॒म् चक्षु॒ श्चक्षु॒रात॑तम् । \newline
37. आत॑त॒मित्या - त॒त॒म् । \newline
38. ब्र॒ह्म॒वनि॑म् त्वा त्वा ब्रह्म॒वनि॑म् ब्रह्म॒वनि॑म् त्वा । \newline
39. ब्र॒ह्म॒वनि॒मिति॑ ब्रह्म - वनि᳚म् । \newline
40. त्वा॒ क्ष॒त्र॒वनि॑म् क्षत्र॒वनि॑म् त्वा त्वा क्षत्र॒वनि᳚म् । \newline
41. क्ष॒त्र॒वनि(ग्म्॑) सुप्रजा॒वनि(ग्म्॑) सुप्रजा॒वनि॑म् क्षत्र॒वनि॑म् क्षत्र॒वनि(ग्म्॑) सुप्रजा॒वनि᳚म् । \newline
42. क्ष॒त्र॒वनि॒मिति॑ क्षत्र - वनि᳚म् । \newline
43. सु॒प्र॒जा॒वनि(ग्म्॑) रायस्पोष॒वनि(ग्म्॑) रायस्पोष॒वनि(ग्म्॑) सुप्रजा॒वनि(ग्म्॑) सुप्रजा॒वनि(ग्म्॑) रायस्पोष॒वनि᳚म् । \newline
44. सु॒प्र॒जा॒वनि॒मिति॑ सुप्रजा - वनि᳚म् । \newline
45. रा॒य॒स्पो॒ष॒वनि॒म् परि॒ परि॑ रायस्पोष॒वनि(ग्म्॑) रायस्पोष॒वनि॒म् परि॑ । \newline
46. रा॒य॒स्पो॒ष॒वनि॒मिति॑ रायस्पोष - वनि᳚म् । \newline
47. पर् यू॑हा म्यूहामि॒ परि॒ पर्यू॑हामि । \newline
48. ऊ॒हा॒मि॒ ब्रह्म॒ ब्रह्मो॑ हाम्यूहामि॒ ब्रह्म॑ । \newline
49. ब्रह्म॑ दृꣳह दृꣳह॒ ब्रह्म॒ ब्रह्म॑ दृꣳह । \newline
50. दृ॒(ग्म्॒)ह॒ क्ष॒त्रम् क्ष॒त्रम् दृ(ग्म्॑)ह दृꣳह क्ष॒त्रम् । \newline
51. क्ष॒त्रम् दृ(ग्म्॑)ह दृꣳह क्ष॒त्रम् क्ष॒त्रम् दृ(ग्म्॑)ह । \newline
52. दृ॒(ग्म्॒)ह॒ प्र॒जाम् प्र॒जाम् दृ(ग्म्॑)ह दृꣳह प्र॒जाम् । \newline
53. प्र॒जाम् दृ(ग्म्॑)ह दृꣳह प्र॒जाम् प्र॒जाम् दृ(ग्म्॑)ह । \newline
54. प्र॒जामिति॑ प्र - जाम् । \newline
55. दृ॒(ग्म्॒)ह॒ रा॒यो रा॒यो दृ(ग्म्॑)ह दृꣳह रा॒यः । \newline
56. रा॒यस् पोष॒म् पोष(ग्म्॑) रा॒यो रा॒यस् पोष᳚म् । \newline
57. पोष॑म् दृꣳह दृꣳह॒ पोष॒म् पोष॑म् दृꣳह । \newline
58. दृ॒(ग्म्॒)ह॒ प॒रि॒वीः प॑रि॒वीर् दृ(ग्म्॑)ह दृꣳह परि॒वीः । \newline
59. प॒रि॒वी र॑स्यसि परि॒वीः प॑रि॒वी र॑सि । \newline
60. प॒रि॒वीरिति॑ परि - वीः । \newline
61. अ॒सि॒ परि॒ पर्य॑स्यसि॒ परि॑ । \newline
62. परि॑ त्वा त्वा॒ परि॒ परि॑ त्वा । \newline
63. त्वा॒ दैवी॒र् दैवी᳚ स्त्वा त्वा॒ दैवीः᳚ । \newline
64. दैवी॒र् विशो॒ विशो॒ दैवी॒र् दैवी॒र् विशः॑ । \newline
65. विशो᳚ व्ययन्तां ॅव्ययन्तां॒ ॅविशो॒ विशो᳚ व्ययन्ताम् । \newline
66. व्य॒य॒न्ता॒म् परि॒ परि॑ व्ययन्तां ॅव्ययन्ता॒म् परि॑ । \newline
67. परी॒म मि॒मम् परि॒ परी॒मम् । \newline
68. इ॒मꣳ रा॒यो रा॒य इ॒म मि॒मꣳ रा॒यः । \newline
69. रा॒य स्पोषः॒ पोषो॑ रा॒यो रा॒य स्पोषः॑ । \newline
70. पोषो॒ यज॑मानं॒ ॅयज॑मान॒म् पोषः॒ पोषो॒ यज॑मानम् । \newline
71. यज॑मानम् मनु॒ष्या॑ मनु॒ष्या॑ यज॑मानं॒ ॅयज॑मानम् मनु॒ष्याः᳚ । \newline
72. म॒नु॒ष्या॑ अ॒न्तरि॑क्ष स्या॒न्तरि॑क्षस्य मनु॒ष्या॑ मनु॒ष्या॑ अ॒न्तरि॑क्षस्य । \newline
73. अ॒न्तरि॑क्षस्य त्वा त्वा॒ ऽन्तरि॑क्ष स्या॒न्तरि॑क्षस्य त्वा । \newline
74. त्वा॒ सानौ॒ सानौ᳚ त्वा त्वा॒ सानौ᳚ । \newline
75. साना॒ ववाव॒ सानौ॒ साना॒ वव॑ । \newline
76. अव॑ गूहामि गूहा॒ म्यवाव॑ गूहामि । \newline
77. गू॒हा॒मीति॑ गूहामि । \newline

\textbf{Ghana Paata } \newline

1. ग॒मद्ध्ये॒ गावो॒ गावो॑ ग॒मद्ध्ये॑ ग॒मद्ध्ये॒ गावो॒ यत्र॒ यत्र॒ गावो॑ ग॒मद्ध्ये॑ ग॒मद्ध्ये॒ गावो॒ यत्र॑ । \newline
2. गावो॒ यत्र॒ यत्र॒ गावो॒ गावो॒ यत्र॒ भूरि॑शृङ्गा॒ भूरि॑शृङ्गा॒ यत्र॒ गावो॒ गावो॒ यत्र॒ भूरि॑शृङ्गाः । \newline
3. यत्र॒ भूरि॑शृङ्गा॒ भूरि॑शृङ्गा॒ यत्र॒ यत्र॒ भूरि॑शृङ्गा अ॒यासो॒ ऽयासो॒ भूरि॑शृङ्गा॒ यत्र॒ यत्र॒ भूरि॑शृङ्गा अ॒यासः॑ । \newline
4. भूरि॑शृङ्गा अ॒यासो॒ ऽयासो॒ भूरि॑शृङ्गा॒ भूरि॑शृङ्गा अ॒यासः॑ । \newline
5. भूरि॑शृङ्गा॒ इति॒ भूरि॑ - शृ॒ङ्गाः॒ । \newline
6. अ॒यास॒ इत्य॒यासः॑ । \newline
7. अत्रा हाहा त्रात्राह॒ तत् तदहा त्रात्राह॒ तत् । \newline
8. अह॒ तत् तदहाह॒ तदु॑रुगा॒यस्यो॑ रुगा॒यस्य॒ तदहाह॒ तदु॑रुगा॒यस्य॑ । \newline
9. तदु॑रुगा॒यस्यो॑ रुगा॒यस्य॒ तत् तदु॑रुगा॒यस्य॒ विष्णो॒र् विष्णो॑ रुरुगा॒यस्य॒ तत् तदु॑रुगा॒यस्य॒ विष्णोः᳚ । \newline
10. उ॒रु॒गा॒यस्य॒ विष्णो॒र् विष्णो॑ रुरुगा॒यस्यो॑ रुगा॒यस्य॒ विष्णोः᳚ पर॒मम् प॑र॒मं ॅविष्णो॑ रुरुगा॒यस्यो॑ रुगा॒यस्य॒ विष्णोः᳚ पर॒मम् । \newline
11. उ॒रु॒गा॒यस्येत्यु॑रु - गा॒यस्य॑ । \newline
12. विष्णोः᳚ पर॒मम् प॑र॒मं ॅविष्णो॒र् विष्णोः᳚ पर॒मम् प॒दम् प॒दम् प॑र॒मं ॅविष्णो॒र् विष्णोः᳚ पर॒मम् प॒दम् । \newline
13. प॒र॒मम् प॒दम् प॒दम् प॑र॒मम् प॑र॒मम् प॒द मवाव॑ प॒दम् प॑र॒मम् प॑र॒मम् प॒द मव॑ । \newline
14. प॒द मवाव॑ प॒दम् प॒द मव॑ भाति भा॒त्यव॑ प॒दम् प॒द मव॑ भाति । \newline
15. अव॑ भाति भा॒ त्यवाव॑ भाति॒ भूरे॒र् भूरे᳚र् भा॒ त्यवाव॑ भाति॒ भूरेः᳚ । \newline
16. भा॒ति॒ भूरे॒र् भूरे᳚र् भाति भाति॒ भूरेः᳚ । \newline
17. भूरे॒रिति॒ भुरेः᳚ । \newline
18. विष्णोः॒ कर्मा॑णि॒ कर्मा॑णि॒ विष्णो॒र् विष्णोः॒ कर्मा॑णि पश्यत पश्यत॒ कर्मा॑णि॒ विष्णो॒र् विष्णोः॒ कर्मा॑णि पश्यत । \newline
19. कर्मा॑णि पश्यत पश्यत॒ कर्मा॑णि॒ कर्मा॑णि पश्यत॒ यतो॒ यतः॑ पश्यत॒ कर्मा॑णि॒ कर्मा॑णि पश्यत॒ यतः॑ । \newline
20. प॒श्य॒त॒ यतो॒ यतः॑ पश्यत पश्यत॒ यतो᳚ व्र॒तानि॑ व्र॒तानि॒ यतः॑ पश्यत पश्यत॒ यतो᳚ व्र॒तानि॑ । \newline
21. यतो᳚ व्र॒तानि॑ व्र॒तानि॒ यतो॒ यतो᳚ व्र॒तानि॑ पस्प॒शे प॑स्प॒शे व्र॒तानि॒ यतो॒ यतो᳚ व्र॒तानि॑ पस्प॒शे । \newline
22. व्र॒तानि॑ पस्प॒शे प॑स्प॒शे व्र॒तानि॑ व्र॒तानि॑ पस्प॒शे । \newline
23. प॒स्प॒श इति॑ पस्प॒शे । \newline
24. इन्द्र॑स्य॒ युज्यो॒ युज्य॒ इन्द्र॒स्ये न्द्र॑स्य॒ युज्यः॒ सखा॒ सखा॒ युज्य॒ इन्द्र॒स्ये न्द्र॑स्य॒ युज्यः॒ सखा᳚ । \newline
25. युज्यः॒ सखा॒ सखा॒ युज्यो॒ युज्यः॒ सखा᳚ । \newline
26. सखेति॒ सखा᳚ । \newline
27. तद् विष्णो॒र् विष्णो॒ स्तत् तद् विष्णोः᳚ पर॒मम् प॑र॒मं ॅविष्णो॒ स्तत् तद् विष्णोः᳚ पर॒मम् । \newline
28. विष्णोः᳚ पर॒मम् प॑र॒मं ॅविष्णो॒र् विष्णोः᳚ पर॒मम् प॒दम् प॒दम् प॑र॒मं ॅविष्णो॒र् विष्णोः᳚ पर॒मम् प॒दम् । \newline
29. प॒र॒मम् प॒दम् प॒दम् प॑र॒मम् प॑र॒मम् प॒दꣳ सदा॒ सदा॑ प॒दम् प॑र॒मम् प॑र॒मम् प॒दꣳ सदा᳚ । \newline
30. प॒दꣳ सदा॒ सदा॑ प॒दम् प॒दꣳ सदा॑ पश्यन्ति पश्यन्ति॒ सदा॑ प॒दम् प॒दꣳ सदा॑ पश्यन्ति । \newline
31. सदा॑ पश्यन्ति पश्यन्ति॒ सदा॒ सदा॑ पश्यन्ति सू॒रयः॑ सू॒रयः॑ पश्यन्ति॒ सदा॒ सदा॑ पश्यन्ति सू॒रयः॑ । \newline
32. प॒श्य॒न्ति॒ सू॒रयः॑ सू॒रयः॑ पश्यन्ति पश्यन्ति सू॒रयः॑ । \newline
33. सू॒रय॒ इति॑ सू॒रयः॑ । \newline
34. दि॒वीवे॑ व दि॒वि दि॒वीव॒ चक्षु॒ श्चक्षु॑ रिव दि॒वि दि॒वीव॒ चक्षुः॑ । \newline
35. इ॒व॒ चक्षु॒ श्चक्षु॑ रिवे व॒ चक्षु॒ रात॑त॒ मात॑त॒म् चक्षु॑रिवे व॒ चक्षु॒ रात॑तम् । \newline
36. चक्षु॒ रात॑त॒ मात॑त॒म् चक्षु॒ श्चक्षु॒ रात॑तम् । \newline
37. आत॑त॒मित्या - त॒त॒म् । \newline
38. ब्र॒ह्म॒वनि॑म् त्वा त्वा ब्रह्म॒वनि॑म् ब्रह्म॒वनि॑म् त्वा क्षत्र॒वनि॑म् क्षत्र॒वनि॑म् त्वा ब्रह्म॒वनि॑म् ब्रह्म॒वनि॑म् त्वा क्षत्र॒वनि᳚म् । \newline
39. ब्र॒ह्म॒वनि॒मिति॑ ब्रह्म - वनि᳚म् । \newline
40. त्वा॒ क्ष॒त्र॒वनि॑म् क्षत्र॒वनि॑म् त्वा त्वा क्षत्र॒वनि(ग्म्॑) सुप्रजा॒वनि(ग्म्॑) सुप्रजा॒वनि॑म् क्षत्र॒वनि॑म् त्वा त्वा क्षत्र॒वनि(ग्म्॑) सुप्रजा॒वनि᳚म् । \newline
41. क्ष॒त्र॒वनि(ग्म्॑) सुप्रजा॒वनि(ग्म्॑) सुप्रजा॒वनि॑म् क्षत्र॒वनि॑म् क्षत्र॒वनि(ग्म्॑) सुप्रजा॒वनि(ग्म्॑) रायस्पोष॒वनि(ग्म्॑) रायस्पोष॒वनि(ग्म्॑) सुप्रजा॒वनि॑म् क्षत्र॒वनि॑म् क्षत्र॒वनि(ग्म्॑) सुप्रजा॒वनि(ग्म्॑) रायस्पोष॒वनि᳚म् । \newline
42. क्ष॒त्र॒वनि॒मिति॑ क्षत्र - वनि᳚म् । \newline
43. सु॒प्र॒जा॒वनि(ग्म्॑) रायस्पोष॒वनि(ग्म्॑) रायस्पोष॒वनि(ग्म्॑) सुप्रजा॒वनि(ग्म्॑) सुप्रजा॒वनि(ग्म्॑) रायस्पोष॒वनि॒म् परि॒ परि॑ रायस्पोष॒वनि(ग्म्॑) सुप्रजा॒वनि(ग्म्॑) सुप्रजा॒वनि(ग्म्॑) रायस्पोष॒वनि॒म् परि॑ । \newline
44. सु॒प्र॒जा॒वनि॒मिति॑ सुप्रजा - वनि᳚म् । \newline
45. रा॒य॒स्पो॒ष॒वनि॒म् परि॒ परि॑ रायस्पोष॒वनि(ग्म्॑) रायस्पोष॒वनि॒म् पर्यू॑हा म्यूहामि॒ परि॑ रायस्पोष॒वनि(ग्म्॑) रायस्पोष॒वनि॒म् पर्यू॑हामि । \newline
46. रा॒य॒स्पो॒ष॒वनि॒मिति॑ रायस्पोष - वनि᳚म् । \newline
47. पर्यू॑हा म्यूहामि॒ परि॒ पर्यू॑हामि॒ ब्रह्म॒ ब्रह्मो॑हामि॒ परि॒ पर्यू॑हामि॒ ब्रह्म॑ । \newline
48. ऊ॒हा॒मि॒ ब्रह्म॒ ब्रह्मो॑हाम्यूहामि॒ ब्रह्म॑ दृꣳह दृꣳह॒ ब्रह्मो॑हाम्यूहामि॒ ब्रह्म॑ दृꣳह । \newline
49. ब्रह्म॑ दृꣳह दृꣳह॒ ब्रह्म॒ ब्रह्म॑ दृꣳह क्ष॒त्रम् क्ष॒त्रम् दृ(ग्म्॑)ह॒ ब्रह्म॒ ब्रह्म॑ दृꣳह क्ष॒त्रम् । \newline
50. दृ॒(ग्म्॒)ह॒ क्ष॒त्रम् क्ष॒त्रम् दृ(ग्म्॑)ह दृꣳह क्ष॒त्रम् दृ(ग्म्॑)ह दृꣳह क्ष॒त्रम् दृ(ग्म्॑)ह दृꣳह क्ष॒त्रम् दृ(ग्म्॑)ह । \newline
51. क्ष॒त्रम् दृ(ग्म्॑)ह दृꣳह क्ष॒त्रम् क्ष॒त्रम् दृ(ग्म्॑)ह प्र॒जाम् प्र॒जाम् दृ(ग्म्॑)ह क्ष॒त्रम् क्ष॒त्रम् दृ(ग्म्॑)ह प्र॒जाम् । \newline
52. दृ॒(ग्म्॒)ह॒ प्र॒जाम् प्र॒जाम् दृ(ग्म्॑)ह दृꣳह प्र॒जाम् दृ(ग्म्॑)ह दृꣳह प्र॒जाम् दृ(ग्म्॑)ह दृꣳह प्र॒जाम् दृ(ग्म्॑)ह । \newline
53. प्र॒जाम् दृ(ग्म्॑)ह दृꣳह प्र॒जाम् प्र॒जाम् दृ(ग्म्॑)ह रा॒यो रा॒यो दृ(ग्म्॑)ह प्र॒जाम् प्र॒जाम् दृ(ग्म्॑)ह रा॒यः । \newline
54. प्र॒जामिति॑ प्र - जाम् । \newline
55. दृ॒(ग्म्॒)ह॒ रा॒यो रा॒यो दृ(ग्म्॑)ह दृꣳह रा॒य स्पोष॒म् पोष(ग्म्॑) रा॒यो दृ(ग्म्॑)ह दृꣳह रा॒य स्पोष᳚म् । \newline
56. रा॒य स्पोष॒म् पोष(ग्म्॑) रा॒यो रा॒य स्पोष॑म् दृꣳह दृꣳह॒ पोष(ग्म्॑) रा॒यो रा॒य स्पोष॑म् दृꣳह । \newline
57. पोष॑म् दृꣳह दृꣳह॒ पोष॒म् पोष॑म् दृꣳह परि॒वीः प॑रि॒वीर् दृ(ग्म्॑)ह॒ पोष॒म् पोष॑म् दृꣳह परि॒वीः । \newline
58. दृ॒(ग्म्॒)ह॒ प॒रि॒वीः प॑रि॒वीर् दृ(ग्म्॑)ह दृꣳह परि॒वी र॑स्यसि परि॒वीर् दृ(ग्म्॑)ह दृꣳह परि॒वी र॑सि । \newline
59. प॒रि॒वी र॑स्यसि परि॒वीः प॑रि॒वी र॑सि॒ परि॒ पर्य॑सि परि॒वीः प॑रि॒वी र॑सि॒ परि॑ । \newline
60. प॒रि॒वीरिति॑ परि - वीः । \newline
61. अ॒सि॒ परि॒ पर्य॑स्यसि॒ परि॑ त्वा त्वा॒ पर्य॑स्यसि॒ परि॑ त्वा । \newline
62. परि॑ त्वा त्वा॒ परि॒ परि॑ त्वा॒ दैवी॒र् दैवी᳚ स्त्वा॒ परि॒ परि॑ त्वा॒ दैवीः᳚ । \newline
63. त्वा॒ दैवी॒र् दैवी᳚ स्त्वा त्वा॒ दैवी॒र् विशो॒ विशो॒ दैवी᳚ स्त्वा त्वा॒ दैवी॒र् विशः॑ । \newline
64. दैवी॒र् विशो॒ विशो॒ दैवी॒र् दैवी॒र् विशो᳚ व्ययन्तां ॅव्ययन्तां॒ ॅविशो॒ दैवी॒र् दैवी॒र् विशो᳚ व्ययन्ताम् । \newline
65. विशो᳚ व्ययन्तां ॅव्ययन्तां॒ ॅविशो॒ विशो᳚ व्ययन्ता॒म् परि॒ परि॑ व्ययन्तां॒ ॅविशो॒ विशो᳚ व्ययन्ता॒म् परि॑ । \newline
66. व्य॒य॒न्ता॒म् परि॒ परि॑ व्ययन्तां ॅव्ययन्ता॒म् परी॒म मि॒मम् परि॑ व्ययन्तां ॅव्ययन्ता॒म् परी॒मम् । \newline
67. परी॒म मि॒मम् परि॒ परी॒मꣳ रा॒यो रा॒य इ॒मम् परि॒ परी॒मꣳ रा॒यः । \newline
68. इ॒मꣳ रा॒यो रा॒य इ॒म मि॒मꣳ रा॒य स्पोषः॒ पोषो॑ रा॒य इ॒म मि॒मꣳ रा॒य स्पोषः॑ । \newline
69. रा॒य स्पोषः॒ पोषो॑ रा॒यो रा॒य स्पोषो॒ यज॑मानं॒ ॅयज॑मान॒म् पोषो॑ रा॒यो रा॒य स्पोषो॒ यज॑मानम् । \newline
70. पोषो॒ यज॑मानं॒ ॅयज॑मान॒म् पोषः॒ पोषो॒ यज॑मानम् मनु॒ष्या॑ मनु॒ष्या॑ यज॑मान॒म् पोषः॒ पोषो॒ यज॑मानम् मनु॒ष्याः᳚ । \newline
71. यज॑मानम् मनु॒ष्या॑ मनु॒ष्या॑ यज॑मानं॒ ॅयज॑मानम् मनु॒ष्या॑ अ॒न्तरि॑क्षस्या॒ न्तरि॑क्षस्य मनु॒ष्या॑ यज॑मानं॒ ॅयज॑मानम् मनु॒ष्या॑ अ॒न्तरि॑क्षस्य । \newline
72. म॒नु॒ष्या॑ अ॒न्तरि॑क्षस्या॒ न्तरि॑क्षस्य मनु॒ष्या॑ मनु॒ष्या॑ अ॒न्तरि॑क्षस्य त्वा त्वा॒ ऽन्तरि॑क्षस्य मनु॒ष्या॑ मनु॒ष्या॑ अ॒न्तरि॑क्षस्य त्वा । \newline
73. अ॒न्तरि॑क्षस्य त्वा त्वा॒ ऽन्तरि॑क्षस्या॒ न्तरि॑क्षस्य त्वा॒ सानौ॒ सानौ᳚ त्वा॒ ऽन्तरि॑क्षस्या॒ न्तरि॑क्षस्य त्वा॒ सानौ᳚ । \newline
74. त्वा॒ सानौ॒ सानौ᳚ त्वा त्वा॒ साना॒ ववाव॒ सानौ᳚ त्वा त्वा॒ साना॒ वव॑ । \newline
75. साना॒ ववाव॒ सानौ॒ साना॒ वव॑ गूहामि गूहा॒ म्यव॒ सानौ॒ साना॒ वव॑ गूहामि । \newline
76. अव॑ गूहामि गूहा॒ म्यवाव॑ गूहामि । \newline
77. गू॒हा॒मीति॑ गूहामि । \newline
\pagebreak
\markright{ TS 1.3.7.1  \hfill https://www.vedavms.in \hfill}

\section{ TS 1.3.7.1 }

\textbf{TS 1.3.7.1 } \newline
\textbf{Samhita Paata} \newline

इ॒षे त्वो॑प॒वीर॒स्युपो॑ दे॒वान् दैवी॒र् विशः॒ प्रागु॒र् वह्नी॑रु॒शिजो॒ बृह॑स्पते धा॒रया॒ वसू॑नि ह॒व्या ते᳚ स्वदन्तां॒ देव॑ त्वष्ट॒र्वसु॑ रण्व॒ रेव॑ती॒ रम॑द्ध्व-म॒ग्नेर् ज॒नित्र॑मसि॒ वृष॑णौ स्थ उ॒र्वश्य॑स्या॒युर॑सि पुरू॒रवा॑ घृ॒तेना॒क्ते वृष॑णं दधाथां गाय॒त्रं छन्दोऽनु॒ प्र जा॑यस्व॒ त्रैष्टु॑भं॒ जाग॑तं॒ छन्दोऽनु॒ प्र जा॑यस्व॒ भव॑तं - [ ] \newline

\textbf{Pada Paata} \newline

इ॒षे । त्वा॒ । उ॒प॒वीरित्यु॑प - वीः । अ॒सि॒ । उपो॒ इति॑ । दे॒वान् । दैवीः᳚ । विशः॑ । प्रेति॑ । अ॒गुः॒ । वह्नीः᳚ । उ॒शिजः॑ । बृह॑स्पते । धा॒रय॑ । वसू॑नि । ह॒व्या । ते॒ । स्व॒द॒न्ता॒म् । देव॑ । त्व॒ष्टः॒ । वसु॑ । र॒ण्व॒ । रेव॑तीः । रम॑द्ध्वम् । अ॒ग्नेः । ज॒नित्र᳚म् । अ॒सि॒ । वृष॑णौ । स्थः॒ । उ॒र्वशी᳚ । अ॒सि॒ । आ॒युः । अ॒सि॒ । पु॒रू॒रवाः᳚ । घृ॒तेन॑ । अ॒क्ते इति॑ । वृष॑णम् । द॒धा॒था॒म् । गा॒य॒त्रम् । छन्दः॑ । अनु॑ । प्रेति॑ । जा॒य॒स्व॒ । त्रैष्टु॑भम् । जाग॑तम् । छन्दः॑ । अनु॑ । प्रेति॑ । जा॒य॒स्व॒ । भव॑तम् ।  \newline


\textbf{Krama Paata} \newline

इ॒षे त्वा᳚ । त्वो॒प॒वीः । उ॒प॒वीर॑सि । उ॒प॒वीरित्यु॑प - वीः । अ॒स्युपो᳚ । उपो॑ दे॒वान् । उपो॒ इत्युपो᳚ । दे॒वान् दैवीः᳚ । दैवी॒र् विशः॑ । विशः॒ प्र । प्रागुः॑ । अ॒गु॒र् वह्नीः᳚ । वह्नी॑रु॒शिजः॑ । उ॒शिजो॒ बृह॑स्पते । बृह॑स्पते धा॒रय॑ । धा॒रया॒ वसू॑नि । वसू॑नि ह॒व्या । ह॒व्या ते᳚ । ते॒ स्व॒द॒न्ता॒म् । स्व॒द॒न्ता॒म् देव॑ । देव॑ त्वष्टः । त्व॒ष्ट॒र् वसु॑ । वसु॑रण्व । र॒ण्व॒ रेव॑तीः । रेव॑ती॒ रम॑द्ध्वम् । रम॑द्ध्वम॒ग्नेः । अ॒ग्नेर् ज॒नित्र᳚म् । ज॒नित्र॑मसि । अ॒सि॒ वृष॑णौ । वृष॑णौ स्थः । स्थ॒ उ॒र्वशी᳚ । उ॒र्वश्य॑सि । अ॒स्या॒युः । आ॒युर॑सि । अ॒सि॒ पु॒रू॒रवाः᳚ । पु॒रू॒रवा॑ घृ॒तेन॑ । घृ॒तेना॒क्ते । अ॒क्ते वृष॑णम् । अ॒क्ते इत्य॒क्ते । वृष॑णम् दधाथाम् । द॒धा॒था॒म् गा॒य॒त्रम् । गा॒य॒त्रम् छन्दः॑ । छन्दोऽनु॑ । अनु॒ प्र । प्र जा॑यस्व । जा॒य॒स्व॒ त्रैष्टु॑भम् । त्रैष्टु॑भ॒म् जाग॑तम् । जाग॑त॒म् छन्दः॑ । छन्दोऽनु॑ । अनु॒ प्र । प्र जा॑यस्व । जा॒य॒स्व॒ भव॑तम् ( ) । भव॑तम् नः \newline

\textbf{Jatai Paata} \newline

1. इ॒षे त्वा᳚ त्वे॒ष इ॒षे त्वा᳚ । \newline
2. त्वो॒प॒वी रु॑प॒वी स्त्वा᳚ त्वोप॒वीः । \newline
3. उ॒प॒वी र॑स्य स्युप॒वी रु॑प॒वी र॑सि । \newline
4. उ॒प॒वीरित्यु॑प - वीः । \newline
5. अ॒स्युपो॒ उपो॑ अस्य॒ स्युपो᳚ । \newline
6. उपो॑ दे॒वान् दे॒वा नुपो॒ उपो॑ दे॒वान् । \newline
7. उपो॒ इत्युपो᳚ । \newline
8. दे॒वान् दैवी॒र् दैवी᳚र् दे॒वान् दे॒वान् दैवीः᳚ । \newline
9. दैवी॒र् विशो॒ विशो॒ दैवी॒र् दैवी॒र् विशः॑ । \newline
10. विशः॒ प्र प्र विशो॒ विशः॒ प्र । \newline
11. प्रागु॑रगुः॒ प्र प्रागुः॑ । \newline
12. अ॒गु॒र् वह्नी॒र् वह्नी॑ रगु रगु॒र् वह्नीः᳚ । \newline
13. वह्नी॑ रु॒शिज॑ उ॒शिजो॒ वह्नी॒र् वह्नी॑ रु॒शिजः॑ । \newline
14. उ॒शिजो॒ बृह॑स्पते॒ बृह॑स्पत उ॒शिज॑ उ॒शिजो॒ बृह॑स्पते । \newline
15. बृह॑स्पते धा॒रय॑ धा॒रय॒ बृह॑स्पते॒ बृह॑स्पते धा॒रय॑ । \newline
16. धा॒रया॒ वसू॑नि॒ वसू॑नि धा॒रय॑ धा॒रया॒ वसू॑नि । \newline
17. वसू॑नि ह॒व्या ह॒व्या वसू॑नि॒ वसू॑नि ह॒व्या । \newline
18. ह॒व्या ते॑ ते ह॒व्या ह॒व्या ते᳚ । \newline
19. ते॒ स्व॒द॒न्ता॒(ग्ग्॒) स्व॒द॒न्ता॒म् ते॒ ते॒ स्व॒द॒न्ता॒म् । \newline
20. स्व॒द॒न्ता॒म् देव॒ देव॑ स्वदन्ताꣳ स्वदन्ता॒म् देव॑ । \newline
21. देव॑ त्वष्टस् त्वष्ट॒र् देव॒ देव॑ त्वष्टः । \newline
22. त्व॒ष्ट॒र् वसु॒ वसु॑ त्वष्ट स्त्वष्ट॒र् वसु॑ । \newline
23. वसु॑ रण्व रण्व॒ वसु॒ वसु॑ रण्व । \newline
24. र॒ण्व॒ रेव॑ती॒ रेव॑ती रण्व रण्व॒ रेव॑तीः । \newline
25. रेव॑ती॒ रम॑द्ध्व॒(ग्म्॒) रम॑द्ध्व॒(ग्म्॒) रेव॑ती॒ रेव॑ती॒ रम॑द्ध्वम् । \newline
26. रम॑द्ध्व म॒ग्ने र॒ग्ने रम॑द्ध्व॒(ग्म्॒) रम॑द्ध्व म॒ग्नेः । \newline
27. अ॒ग्नेर् ज॒नित्र॑म् ज॒नित्र॑ म॒ग्ने र॒ग्नेर् ज॒नित्र᳚म् । \newline
28. ज॒नित्र॑ मस्यसि ज॒नित्र॑म् ज॒नित्र॑ मसि । \newline
29. अ॒सि॒ वृष॑णौ॒ वृष॑णा वस्यसि॒ वृष॑णौ । \newline
30. वृष॑णौ स्थः स्थो॒ वृष॑णौ॒ वृष॑णौ स्थः । \newline
31. स्थ॒ उ॒र्वश्यु॒र्वशी᳚ स्थः स्थ उ॒र्वशी᳚ । \newline
32. उ॒र्वश्य॑ स्य स्यु॒र्व श्यु॒र्वश्य॑सि । \newline
33. अ॒स्या॒यु रा॒यु र॑स्यस्या॒युः । \newline
34. आ॒यु र॑स्यस्या॒यु रा॒युर॑सि । \newline
35. अ॒सि॒ पु॒रू॒रवाः᳚ पुरू॒रवा॑ अस्यसि पुरू॒रवाः᳚ । \newline
36. पु॒रू॒रवा॑ घृ॒तेन॑ घृ॒तेन॑ पुरू॒रवाः᳚ पुरू॒रवा॑ घृ॒तेन॑ । \newline
37. घृ॒तेना॒क्ते अ॒क्ते घृ॒तेन॑ घृ॒तेना॒क्ते । \newline
38. अ॒क्ते वृष॑णं॒ ॅवृष॑ण म॒क्ते अ॒क्ते वृष॑णम् । \newline
39. अ॒क्ते इत्य॒क्ते । \newline
40. वृष॑णम् दधाथाम् दधाथां॒ ॅवृष॑णं॒ ॅवृष॑णम् दधाथाम् । \newline
41. द॒धा॒था॒म् गा॒य॒त्रम् गा॑य॒त्रम् द॑धाथाम् दधाथाम् गाय॒त्रम् । \newline
42. गा॒य॒त्रम् छन्द॒ श्छन्दो॑ गाय॒त्रम् गा॑य॒त्रम् छन्दः॑ । \newline
43. छन्दो ऽन्वनु॒ छन्द॒ श्छन्दो ऽनु॑ । \newline
44. अनु॒ प्र प्राण्वनु॒ प्र । \newline
45. प्र जा॑यस्व जायस्व॒ प्र प्र जा॑यस्व । \newline
46. जा॒य॒स्व॒ त्रैष्टु॑भ॒म् त्रैष्टु॑भम् जायस्व जायस्व॒ त्रैष्टु॑भम् । \newline
47. त्रैष्टु॑भ॒म् जाग॑त॒म् जाग॑त॒म् त्रैष्टु॑भ॒म् त्रैष्टु॑भ॒म् जाग॑तम् । \newline
48. जाग॑त॒म् छन्द॒ श्छन्दो॒ जाग॑त॒म् जाग॑त॒म् छन्दः॑ । \newline
49. छन्दो ऽन्वनु॒ छन्द॒ श्छन्दो ऽनु॑ । \newline
50. अनु॒ प्र प्राण्वनु॒ प्र । \newline
51. प्र जा॑यस्व जायस्व॒ प्र प्र जा॑यस्व । \newline
52. जा॒य॒स्व॒ भव॑त॒म् भव॑तम् जायस्व जायस्व॒ भव॑तम् । \newline
53. भव॑तम् नो नो॒ भव॑त॒म् भव॑तम् नः । \newline

\textbf{Ghana Paata } \newline

1. इ॒षे त्वा᳚ त्वे॒ष इ॒षे त्वो॑प॒वी रु॑प॒वी स्त्वे॒ष इ॒षे त्वो॑प॒वीः । \newline
2. त्वो॒प॒वी रु॑प॒वी स्त्वा᳚ त्वोप॒वी र॑स्य स्युप॒वी स्त्वा᳚ त्वोप॒वीर॑सि । \newline
3. उ॒प॒वी र॑स्य स्युप॒वी रु॑प॒वी र॒स्युपो॒ उपो॑ अस्युप॒वी रु॑प॒वी र॒स्युपो᳚ । \newline
4. उ॒प॒वीरित्यु॑प - वीः । \newline
5. अ॒स्युपो॒ उपो॑ अस्य॒स्युपो॑ दे॒वान् दे॒वा नुपो॑ अस्य॒स्युपो॑ दे॒वान् । \newline
6. उपो॑ दे॒वान् दे॒वा नुपो॒ उपो॑ दे॒वान् दैवी॒र् दैवी᳚र् दे॒वा नुपो॒ उपो॑ दे॒वान् दैवीः᳚ । \newline
7. उपो॒ इत्युपो᳚ । \newline
8. दे॒वान् दैवी॒र् दैवी᳚र् दे॒वान् दे॒वान् दैवी॒र् विशो॒ विशो॒ दैवी᳚र् दे॒वान् दे॒वान् दैवी॒र् विशः॑ । \newline
9. दैवी॒र् विशो॒ विशो॒ दैवी॒र् दैवी॒र् विशः॒ प्र प्र विशो॒ दैवी॒र् दैवी॒र् विशः॒ प्र । \newline
10. विशः॒ प्र प्र विशो॒ विशः॒ प्रागु॑ रगुः॒ प्र विशो॒ विशः॒ प्रागुः॑ । \newline
11. प्रागु॑ रगुः॒ प्र प्रागु॒र् वह्नी॒र् वह्नी॑रगुः॒ प्र प्रागु॒र् वह्नीः᳚ । \newline
12. अ॒गु॒र् वह्नी॒र् वह्नी॑ रगु रगु॒र् वह्नी॑ रु॒शिज॑ उ॒शिजो॒ वह्नी॑ रगु रगु॒र् वह्नी॑ रु॒शिजः॑ । \newline
13. वह्नी॑ रु॒शिज॑ उ॒शिजो॒ वह्नी॒र् वह्नी॑ रु॒शिजो॒ बृह॑स्पते॒ बृह॑स्पत उ॒शिजो॒ वह्नी॒र् वह्नी॑ रु॒शिजो॒ बृह॑स्पते । \newline
14. उ॒शिजो॒ बृह॑स्पते॒ बृह॑स्पत उ॒शिज॑ उ॒शिजो॒ बृह॑स्पते धा॒रय॑ धा॒रय॒ बृह॑स्पत उ॒शिज॑ उ॒शिजो॒ बृह॑स्पते धा॒रय॑ । \newline
15. बृह॑स्पते धा॒रय॑ धा॒रय॒ बृह॑स्पते॒ बृह॑स्पते धा॒रया॒ वसू॑नि॒ वसू॑नि धा॒रय॒ बृह॑स्पते॒ बृह॑स्पते धा॒रया॒ वसू॑नि । \newline
16. धा॒रया॒ वसू॑नि॒ वसू॑नि धा॒रय॑ धा॒रया॒ वसू॑नि ह॒व्या ह॒व्या वसू॑नि धा॒रय॑ धा॒रया॒ वसू॑नि ह॒व्या । \newline
17. वसू॑नि ह॒व्या ह॒व्या वसू॑नि॒ वसू॑नि ह॒व्या ते॑ ते ह॒व्या वसू॑नि॒ वसू॑नि ह॒व्या ते᳚ । \newline
18. ह॒व्या ते॑ ते ह॒व्या ह॒व्या ते᳚ स्वदन्ताꣳ स्वदन्ताम् ते ह॒व्या ह॒व्या ते᳚ स्वदन्ताम् । \newline
19. ते॒ स्व॒द॒न्ता॒(ग्ग्॒) स्व॒द॒न्ता॒म् ते॒ ते॒ स्व॒द॒न्ता॒म् देव॒ देव॑ स्वदन्ताम् ते ते स्वदन्ता॒म् देव॑ । \newline
20. स्व॒द॒न्ता॒म् देव॒ देव॑ स्वदन्ताꣳ स्वदन्ता॒म् देव॑ त्वष्ट स्त्वष्ट॒र् देव॑ स्वदन्ताꣳ स्वदन्ता॒म् देव॑ त्वष्टः । \newline
21. देव॑ त्वष्ट स्त्वष्ट॒र् देव॒ देव॑ त्वष्ट॒र् वसु॒ वसु॑ त्वष्ट॒र् देव॒ देव॑ त्वष्ट॒र् वसु॑ । \newline
22. त्व॒ष्ट॒र् वसु॒ वसु॑ त्वष्ट स्त्वष्ट॒र् वसु॑ रण्व रण्व॒ वसु॑ त्वष्टस् त्वष्ट॒र् वसु॑ रण्व । \newline
23. वसु॑ रण्व रण्व॒ वसु॒ वसु॑ रण्व॒ रेव॑ती॒ रेव॑ती रण्व॒ वसु॒ वसु॑ रण्व॒ रेव॑तीः । \newline
24. र॒ण्व॒ रेव॑ती॒ रेव॑ती रण्व रण्व॒ रेव॑ती॒ रम॑द्ध्व॒(ग्म्॒) रम॑द्ध्व॒(ग्म्॒) रेव॑ती रण्व रण्व॒ रेव॑ती॒ रम॑द्ध्वम् । \newline
25. रेव॑ती॒ रम॑द्ध्व॒(ग्म्॒) रम॑द्ध्व॒(ग्म्॒) रेव॑ती॒ रेव॑ती॒ रम॑द्ध्व म॒ग्नेर॒ग्ने रम॑द्ध्व॒(ग्म्॒) रेव॑ती॒ रेव॑ती॒ रम॑द्ध्व म॒ग्नेः । \newline
26. रम॑द्ध्व म॒ग्नेर॒ग्ने रम॑द्ध्व॒(ग्म्॒) रम॑द्ध्व म॒ग्नेर् ज॒नित्र॑म् ज॒नित्र॑ म॒ग्ने रम॑द्ध्व॒(ग्म्॒) रम॑द्ध्व म॒ग्नेर् ज॒नित्र᳚म् । \newline
27. अ॒ग्नेर् ज॒नित्र॑म् ज॒नित्र॑ म॒ग्ने र॒ग्नेर् ज॒नित्र॑ मस्यसि ज॒नित्र॑ म॒ग्ने र॒ग्नेर् ज॒नित्र॑ मसि । \newline
28. ज॒नित्र॑ मस्यसि ज॒नित्र॑म् ज॒नित्र॑ मसि॒ वृष॑णौ॒ वृष॑णा वसि ज॒नित्र॑म् ज॒नित्र॑ मसि॒ वृष॑णौ । \newline
29. अ॒सि॒ वृष॑णौ॒ वृष॑णा वस्यसि॒ वृष॑णौ स्थः स्थो॒ वृष॑णा वस्यसि॒ वृष॑णौ स्थः । \newline
30. वृष॑णौ स्थः स्थो॒ वृष॑णौ॒ वृष॑णौ स्थ उ॒र्व श्यु॒र्वशी᳚ स्थो॒ वृष॑णौ॒ वृष॑णौ स्थ उ॒र्वशी᳚ । \newline
31. स्थ॒ उ॒र्व श्यु॒र्वशी᳚ स्थः स्थ उ॒र्वश्य॑स्य स्यु॒र्वशी᳚ स्थः स्थ उ॒र्वश्य॑सि । \newline
32. उ॒र्वश्य॑ स्य स्यु॒र्व श्यु॒र्व श्य॑स्या॒यु रा॒यु र॑स्यु॒र्व श्यु॒र्व श्य॑स्या॒युः । \newline
33. अ॒स्या॒यु रा॒यु र॑स्यस्या॒यु र॑स्यस्या॒यु र॑स्यस्या॒यु र॑सि । \newline
34. आ॒यु र॑स्यस्या॒यु रा॒युर॑सि पुरू॒रवाः᳚ पुरू॒रवा॑ अस्या॒यु रा॒युर॑सि पुरू॒रवाः᳚ । \newline
35. अ॒सि॒ पु॒रू॒रवाः᳚ पुरू॒रवा॑ अस्यसि पुरू॒रवा॑ घृ॒तेन॑ घृ॒तेन॑ पुरू॒रवा॑ अस्यसि पुरू॒रवा॑ घृ॒तेन॑ । \newline
36. पु॒रू॒रवा॑ घृ॒तेन॑ घृ॒तेन॑ पुरू॒रवाः᳚ पुरू॒रवा॑ घृ॒तेना॒क्ते अ॒क्ते घृ॒तेन॑ पुरू॒रवाः᳚ पुरू॒रवा॑ घृ॒तेना॒क्ते । \newline
37. घृ॒तेना॒क्ते अ॒क्ते घृ॒तेन॑ घृ॒तेना॒क्ते वृष॑णं॒ ॅवृष॑ण म॒क्ते घृ॒तेन॑ घृ॒तेना॒क्ते वृष॑णम् । \newline
38. अ॒क्ते वृष॑णं॒ ॅवृष॑ण म॒क्ते अ॒क्ते वृष॑णम् दधाथाम् दधाथां॒ ॅवृष॑ण म॒क्ते अ॒क्ते वृष॑णम् दधाथाम् । \newline
39. अ॒क्ते इत्य॒क्ते । \newline
40. वृष॑णम् दधाथाम् दधाथां॒ ॅवृष॑णं॒ ॅवृष॑णम् दधाथाम् गाय॒त्रम् गा॑य॒त्रम् द॑धाथां॒ ॅवृष॑णं॒ ॅवृष॑णम् दधाथाम् गाय॒त्रम् । \newline
41. द॒धा॒था॒म् गा॒य॒त्रम् गा॑य॒त्रम् द॑धाथाम् दधाथाम् गाय॒त्रम् छन्द॒ श्छन्दो॑ गाय॒त्रम् द॑धाथाम् दधाथाम् गाय॒त्रम् छन्दः॑ । \newline
42. गा॒य॒त्रम् छन्द॒ श्छन्दो॑ गाय॒त्रम् गा॑य॒त्रम् छन्दो ऽन्वनु॒ छन्दो॑ गाय॒त्रम् गा॑य॒त्रम् छन्दो ऽनु॑ । \newline
43. छन्दो ऽन्वनु॒ छन्द॒ श्छन्दो ऽनु॒ प्र प्राणु॒ छन्द॒ श्छन्दो ऽनु॒ प्र । \newline
44. अनु॒ प्र प्राण्वनु॒ प्र जा॑यस्व जायस्व॒ प्राण्वनु॒ प्र जा॑यस्व । \newline
45. प्र जा॑यस्व जायस्व॒ प्र प्र जा॑यस्व॒ त्रैष्टु॑भ॒म् त्रैष्टु॑भम् जायस्व॒ प्र प्र जा॑यस्व॒ त्रैष्टु॑भम् । \newline
46. जा॒य॒स्व॒ त्रैष्टु॑भ॒म् त्रैष्टु॑भम् जायस्व जायस्व॒ त्रैष्टु॑भ॒म् जाग॑त॒म् जाग॑त॒म् त्रैष्टु॑भम् जायस्व जायस्व॒ त्रैष्टु॑भ॒म् जाग॑तम् । \newline
47. त्रैष्टु॑भ॒म् जाग॑त॒म् जाग॑त॒म् त्रैष्टु॑भ॒म् त्रैष्टु॑भ॒म् जाग॑त॒म् छन्द॒ श्छन्दो॒ जाग॑त॒म् त्रैष्टु॑भ॒म् त्रैष्टु॑भ॒म् जाग॑त॒म् छन्दः॑ । \newline
48. जाग॑त॒म् छन्द॒ श्छन्दो॒ जाग॑त॒म् जाग॑त॒म् छन्दो ऽन्वनु॒ छन्दो॒ जाग॑त॒म् जाग॑त॒म् छन्दो ऽनु॑ । \newline
49. छन्दो ऽन्वनु॒ छन्द॒ श्छन्दो ऽनु॒ प्र प्राणु॒ छन्द॒ श्छन्दो ऽनु॒ प्र । \newline
50. अनु॒ प्र प्राण्वनु॒ प्र जा॑यस्व जायस्व॒ प्राण्वनु॒ प्र जा॑यस्व । \newline
51. प्र जा॑यस्व जायस्व॒ प्र प्र जा॑यस्व॒ भव॑त॒म् भव॑तम् जायस्व॒ प्र प्र जा॑यस्व॒ भव॑तम् । \newline
52. जा॒य॒स्व॒ भव॑त॒म् भव॑तम् जायस्व जायस्व॒ भव॑तम् नो नो॒ भव॑तम् जायस्व जायस्व॒ भव॑तम् नः । \newline
53. भव॑तम् नो नो॒ भव॑त॒म् भव॑तम् नः॒ सम॑नसौ॒ सम॑नसौ नो॒ भव॑त॒म् भव॑तन्नः॒ सम॑नसौ । \newline
\pagebreak
\markright{ TS 1.3.7.2  \hfill https://www.vedavms.in \hfill}

\section{ TS 1.3.7.2 }

\textbf{TS 1.3.7.2 } \newline
\textbf{Samhita Paata} \newline

नः॒ सम॑नसौ॒ समो॑कसावरे॒पसौ᳚ । मा य॒ज्ञ्ꣳ हिꣳ॑सिष्टं॒ मा य॒ज्ञ्प॑तिं जातवेदसौ शि॒वौ भ॑वतम॒द्य नः॑ ॥ अ॒ग्नाव॒ग्निश्च॑रति॒ प्रवि॑ष्ट॒ ऋषी॑णां पु॒त्रो अ॑धिरा॒ज ए॒षः । स्वा॒हा॒कृत्य॒ ब्रह्म॑णा ते जुहोमि॒ मा दे॒वानां᳚ मिथु॒या क॑र्भाग॒धेयं᳚ ॥ \newline

\textbf{Pada Paata} \newline

नः॒ । सम॑नसा॒विति॒ स - म॒न॒सौ॒ । समो॑कसा॒विति॒ सं - ओ॒क॒सौ॒ । अ॒रे॒पसौ᳚ ॥ मा । य॒ज्ञ्म् । हिꣳ॒॒सि॒ष्ट॒म् । मा । य॒ज्ञ्प॑ति॒मिति॑ य॒ज्ञ् - प॒ति॒म् । जा॒त॒वे॒द॒सा॒विति॑ जात - वे॒द॒सौ॒ । शि॒वौ । भ॒व॒त॒म् । अ॒द्य । नः॒ ॥ अ॒ग्नौ । अ॒ग्निः । च॒र॒ति॒ । प्रवि॑ष्ट॒ इति॒ प्र - वि॒ष्टः॒ । ऋषी॑णाम् । पु॒त्रः । अ॒धि॒रा॒ज इत्य॑धि - रा॒जः । ए॒षः ॥ स्वा॒हा॒कृत्येति॑ स्वाहा - कृत्य॑ । ब्रह्म॑णा । ते॒ । जु॒हो॒मि॒ । मा । दे॒वाना᳚म् । मि॒थु॒या । कः॒ । भा॒ग॒धेय॒मिति॑ भाग - धेय᳚म् ॥  \newline


\textbf{Krama Paata} \newline

नः॒ सम॑नसौ । सम॑नसौ॒ समो॑कसौ । सम॑नसा॒विति॒ स - म॒न॒सौ॒ । समो॑कसावरे॒पसौ᳚ । समो॑कसा॒विति॒ सं - ओ॒क॒सौ॒ । अ॒रे॒पसा॒वित्य॑रे॒पसौ᳚ ॥ मा य॒ज्ञ्म् । य॒ज्ञ्ꣳ हिꣳ॑सिष्टम् । हिꣳ॒॒सि॒ष्ट॒म् मा । मा य॒ज्ञ्प॑तिम् । य॒ज्ञ्प॑तिम् जातवेदसौ । य॒ज्ञ्प॑ति॒मिति॑ य॒ज्ञ् - प॒ति॒म् । जा॒त॒वे॒द॒सौ॒ शि॒वौ । जा॒त॒वे॒द॒सा॒विति॑ जात - वे॒द॒सौ॒ । शि॒वौ भ॑वतम् । भ॒व॒त॒म॒द्य । अ॒द्य नः॑ । न॒ इति॑ नः ॥ अ॒ग्नाव॒ग्निः । अ॒ग्निश्च॑रति । च॒र॒ति॒ प्रवि॑ष्टः । प्रवि॑ष्ट॒ ऋषी॑णाम् । प्रवि॑ष्ट॒ इति॒ प्र - वि॒ष्टः॒ । ऋषी॑णाम् पु॒त्रः । पु॒त्रो अ॑धिरा॒जः । अ॒धि॒रा॒ज ए॒षः । अ॒धि॒रा॒ज इत्य॑धि - रा॒जः । ए॒ष इत्ये॒षः ॥ स्वा॒हा॒कृत्य॒ ब्रह्म॑णा । स्वा॒हा॒कृत्येति॑ स्वाहा - कृत्य॑ । ब्रह्म॑णा ते । ते॒ जु॒हो॒मि॒ । जु॒हो॒मि॒ मा । मा दे॒वाना᳚म् । दे॒वाना᳚म् मिथु॒या । मि॒थु॒या कः॑ । क॒र् भा॒ग॒धेय᳚म् । भा॒ग॒धेय॒मिति॑ भाग - धेय᳚म् । \newline

\textbf{Jatai Paata} \newline

1. नः॒ सम॑नसौ॒ सम॑नसौ नो नः॒ सम॑नसौ । \newline
2. सम॑नसौ॒ समो॑कसौ॒ समो॑कसौ॒ सम॑नसौ॒ सम॑नसौ॒ समो॑कसौ । \newline
3. सम॑नसा॒विति॒ स - म॒न॒सौ॒ । \newline
4. समो॑कसा वरे॒पसा॑ वरे॒पसौ॒ समो॑कसौ॒ समो॑कसा वरे॒पसौ᳚ । \newline
5. समो॑कसा॒विति॒ सं - ओ॒क॒सौ॒ । \newline
6. अ॒रे॒पसावित्य॑रे॒पसौ᳚ । \newline
7. मा य॒ज्ञ्ं ॅय॒ज्ञ्म् मा मा य॒ज्ञ्म् । \newline
8. य॒ज्ञ्ꣳ हि(ग्म्॑)सिष्टꣳ हिꣳसिष्टं ॅय॒ज्ञ्ं ॅय॒ज्ञ्ꣳ हि(ग्म्॑)सिष्टम् । \newline
9. हि॒(ग्म्॒)सि॒ष्ट॒म् मा मा हि(ग्म्॑)सिष्टꣳ हिꣳसिष्ट॒म् मा । \newline
10. मा य॒ज्ञ्प॑तिं ॅय॒ज्ञ्प॑ति॒म् मा मा य॒ज्ञ्प॑तिम् । \newline
11. य॒ज्ञ्प॑तिम् जातवेदसौ जातवेदसौ य॒ज्ञ्प॑तिं ॅय॒ज्ञ्प॑तिम् जातवेदसौ । \newline
12. य॒ज्ञ्प॑ति॒मिति॑ य॒ज्ञ् - प॒ति॒म् । \newline
13. जा॒त॒वे॒द॒सौ॒ शि॒वौ शि॒वौ जा॑तवेदसौ जातवेदसौ शि॒वौ । \newline
14. जा॒त॒वे॒द॒सा॒विति॑ जात - वे॒द॒सौ॒ । \newline
15. शि॒वौ भ॑वतम् भवतꣳ शि॒वौ शि॒वौ भ॑वतम् । \newline
16. भ॒व॒त॒ म॒द्याद्य भ॑वतम् भवत म॒द्य । \newline
17. अ॒द्य नो॑ नो॒ ऽद्याद्य नः॑ । \newline
18. न॒ इति॑ नः । \newline
19. अ॒ग्ना व॒ग्नि र॒ग्नि र॒ग्ना व॒ग्ना व॒ग्निः । \newline
20. अ॒ग्नि श्च॑रति चर त्य॒ग्नि र॒ग्नि श्च॑रति । \newline
21. च॒र॒ति॒ प्रवि॑ष्टः॒ प्रवि॑ष्ट श्चरति चरति॒ प्रवि॑ष्टः । \newline
22. प्रवि॑ष्ट॒ ऋषी॑णा॒ मृषी॑णा॒म् प्रवि॑ष्टः॒ प्रवि॑ष्ट॒ ऋषी॑णाम् । \newline
23. प्रवि॑ष्ट॒ इति॒ प्र - वि॒ष्टः॒ । \newline
24. ऋषी॑णाम् पु॒त्रः पु॒त्र ऋषी॑णा॒ मृषी॑णाम् पु॒त्रः । \newline
25. पु॒त्रो अ॑धिरा॒जो॑ ऽधिरा॒जः पु॒त्रः पु॒त्रो अ॑धिरा॒जः । \newline
26. अ॒धि॒रा॒ज ए॒ष ए॒षो॑ ऽधिरा॒जो॑ ऽधिरा॒ज ए॒षः । \newline
27. अ॒धि॒रा॒ज इत्य॑धि - रा॒जः । \newline
28. ए॒ष इत्ये॒षः । \newline
29. स्वा॒हा॒कृत्य॒ ब्रह्म॑णा॒ ब्रह्म॑णा स्वाहा॒कृत्य॑ स्वाहा॒कृत्य॒ ब्रह्म॑णा । \newline
30. स्वा॒हा॒कृत्येति॑ स्वाहा - कृत्य॑ । \newline
31. ब्रह्म॑णा ते ते॒ ब्रह्म॑णा॒ ब्रह्म॑णा ते । \newline
32. ते॒ जु॒हो॒मि॒ जु॒हो॒मि॒ ते॒ ते॒ जु॒हो॒मि॒ । \newline
33. जु॒हो॒मि॒ मा मा जु॑होमि जुहोमि॒ मा । \newline
34. मा दे॒वाना᳚म् दे॒वाना॒म् मा मा दे॒वाना᳚म् । \newline
35. दे॒वाना᳚म् मिथु॒या मि॑थु॒या दे॒वाना᳚म् दे॒वाना᳚म् मिथु॒या । \newline
36. मि॒थु॒या कः॑ कर् मिथु॒या मि॑थु॒या कः॑ । \newline
37. कर्॒ भा॒ग॒धेय॑म् भाग॒धेय॑म् कः कर् भाग॒धेय᳚म् । \newline
38. भा॒ग॒धेय॒मिति॑ भाग - धेय᳚म् । \newline

\textbf{Ghana Paata } \newline

1. नः॒ सम॑नसौ॒ सम॑नसौ नो नः॒ सम॑नसौ॒ समो॑कसौ॒ समो॑कसौ॒ सम॑नसौ नो नः॒ सम॑नसौ॒ समो॑कसौ । \newline
2. सम॑नसौ॒ समो॑कसौ॒ समो॑कसौ॒ सम॑नसौ॒ सम॑नसौ॒ समो॑कसा वरे॒पसा॑ वरे॒पसौ॒ समो॑कसौ॒ सम॑नसौ॒ सम॑नसौ॒ समो॑कसा वरे॒पसौ᳚ । \newline
3. सम॑नसा॒विति॒ स - म॒न॒सौ॒ । \newline
4. समो॑कसा वरे॒पसा॑ वरे॒पसौ॒ समो॑कसौ॒ समो॑कसा वरे॒पसौ᳚ । \newline
5. समो॑कसा॒विति॒ सं - ओ॒क॒सौ॒ । \newline
6. अ॒रे॒पसावित्य॑रे॒पसौ᳚ । \newline
7. मा य॒ज्ञ्ं ॅय॒ज्ञ्म् मा मा य॒ज्ञ्ꣳ हि(ग्म्॑)सिष्टꣳ हिꣳसिष्टं ॅय॒ज्ञ्म् मा मा य॒ज्ञ्ꣳ हि(ग्म्॑)सिष्टम् । \newline
8. य॒ज्ञ्ꣳ हि(ग्म्॑)सिष्टꣳ हिꣳसिष्टं ॅय॒ज्ञ्ं ॅय॒ज्ञ्ꣳ हि(ग्म्॑)सिष्ट॒म् मा मा हि(ग्म्॑)सिष्टं ॅय॒ज्ञ्ं ॅय॒ज्ञ्ꣳ हि(ग्म्॑)सिष्ट॒म् मा । \newline
9. हि॒(ग्म्॒)सि॒ष्ट॒म् मा मा हि(ग्म्॑)सिष्टꣳ हिꣳसिष्ट॒म् मा य॒ज्ञ्प॑तिं ॅय॒ज्ञ्प॑ति॒म् मा हि(ग्म्॑)सिष्टꣳ 
हिꣳसिष्ट॒म् मा य॒ज्ञ्प॑तिम् । \newline
10. मा य॒ज्ञ्प॑तिं ॅय॒ज्ञ्प॑ति॒म् मा मा य॒ज्ञ्प॑तिम् जातवेदसौ जातवेदसौ य॒ज्ञ्प॑ति॒म् मा मा य॒ज्ञ्प॑तिम् जातवेदसौ । \newline
11. य॒ज्ञ्प॑तिम् जातवेदसौ जातवेदसौ य॒ज्ञ्प॑तिं ॅय॒ज्ञ्प॑तिम् जातवेदसौ शि॒वौ शि॒वौ जा॑तवेदसौ य॒ज्ञ्प॑तिं ॅय॒ज्ञ्प॑तिम् जातवेदसौ शि॒वौ । \newline
12. य॒ज्ञ्प॑ति॒मिति॑ य॒ज्ञ् - प॒ति॒म् । \newline
13. जा॒त॒वे॒द॒सौ॒ शि॒वौ शि॒वौ जा॑तवेदसौ जातवेदसौ शि॒वौ भ॑वतम् भवतꣳ शि॒वौ जा॑तवेदसौ जातवेदसौ शि॒वौ भ॑वतम् । \newline
14. जा॒त॒वे॒द॒सा॒विति॑ जात - वे॒द॒सौ॒ । \newline
15. शि॒वौ भ॑वतम् भवतꣳ शि॒वौ शि॒वौ भ॑वत म॒द्याद्य भ॑वतꣳ शि॒वौ शि॒वौ भ॑वत म॒द्य । \newline
16. भ॒व॒त॒ म॒द्याद्य भ॑वतम् भवत म॒द्य नो॑ नो॒ ऽद्य भ॑वतम् भवत म॒द्य नः॑ । \newline
17. अ॒द्य नो॑ नो॒ ऽद्याद्य नः॑ । \newline
18. न॒ इति॑ नः । \newline
19. अ॒ग्ना व॒ग्नि र॒ग्नि र॒ग्ना व॒ग्ना व॒ग्नि श्च॑रति चरत्य॒ग्नि र॒ग्ना व॒ग्ना व॒ग्नि श्च॑रति । \newline
20. अ॒ग्नि श्च॑रति चरत्य॒ग्नि र॒ग्नि श्च॑रति॒ प्रवि॑ष्टः॒ प्रवि॑ष्ट श्चरत्य॒ग्नि र॒ग्नि श्च॑रति॒ प्रवि॑ष्टः । \newline
21. च॒र॒ति॒ प्रवि॑ष्टः॒ प्रवि॑ष्ट श्चरति चरति॒ प्रवि॑ष्ट॒ ऋषी॑णा॒ मृषी॑णा॒म् प्रवि॑ष्ट श्चरति चरति॒ प्रवि॑ष्ट॒ ऋषी॑णाम् । \newline
22. प्रवि॑ष्ट॒ ऋषी॑णा॒ मृषी॑णा॒म् प्रवि॑ष्टः॒ प्रवि॑ष्ट॒ ऋषी॑णाम् पु॒त्रः पु॒त्र ऋषी॑णा॒म् प्रवि॑ष्टः॒ प्रवि॑ष्ट॒ ऋषी॑णाम् पु॒त्रः । \newline
23. प्रवि॑ष्ट॒ इति॒ प्र - वि॒ष्टः॒ । \newline
24. ऋषी॑णाम् पु॒त्रः पु॒त्र ऋषी॑णा॒ मृषी॑णाम् पु॒त्रो अ॑धिरा॒जो॑ ऽधिरा॒जः पु॒त्र ऋषी॑णा॒ मृषी॑णाम् पु॒त्रो अ॑धिरा॒जः । \newline
25. पु॒त्रो अ॑धिरा॒जो॑ ऽधिरा॒जः पु॒त्रः पु॒त्रो अ॑धिरा॒ज ए॒ष ए॒षो॑ ऽधिरा॒जः पु॒त्रः पु॒त्रो अ॑धिरा॒ज ए॒षः । \newline
26. अ॒धि॒रा॒ज ए॒ष ए॒षो॑ ऽधिरा॒जो॑ ऽधिरा॒ज ए॒षः । \newline
27. अ॒धि॒रा॒ज इत्य॑धि - रा॒जः । \newline
28. ए॒ष इत्ये॒षः । \newline
29. स्वा॒हा॒कृत्य॒ ब्रह्म॑णा॒ ब्रह्म॑णा स्वाहा॒कृत्य॑ स्वाहा॒कृत्य॒ ब्रह्म॑णा ते ते॒ ब्रह्म॑णा स्वाहा॒कृत्य॑ स्वाहा॒कृत्य॒ ब्रह्म॑णा ते । \newline
30. स्वा॒हा॒कृत्येति॑ स्वाहा - कृत्य॑ । \newline
31. ब्रह्म॑णा ते ते॒ ब्रह्म॑णा॒ ब्रह्म॑णा ते जुहोमि जुहोमि ते॒ ब्रह्म॑णा॒ ब्रह्म॑णा ते जुहोमि । \newline
32. ते॒ जु॒हो॒मि॒ जु॒हो॒मि॒ ते॒ ते॒ जु॒हो॒मि॒ मा मा जु॑होमि ते ते जुहोमि॒ मा । \newline
33. जु॒हो॒मि॒ मा मा जु॑होमि जुहोमि॒ मा दे॒वाना᳚म् दे॒वाना॒म् मा जु॑होमि जुहोमि॒ मा दे॒वाना᳚म् । \newline
34. मा दे॒वाना᳚म् दे॒वाना॒म् मा मा दे॒वाना᳚म् मिथु॒या मि॑थु॒या दे॒वाना॒म् मा मा दे॒वाना᳚म् मिथु॒या । \newline
35. दे॒वाना᳚म् मिथु॒या मि॑थु॒या दे॒वाना᳚म् दे॒वाना᳚म् मिथु॒या कः॑ कर् मिथु॒या दे॒वाना᳚म् दे॒वाना᳚म् मिथु॒या कः॑ । \newline
36. मि॒थु॒या कः॑ कर् मिथु॒या मि॑थु॒या कर्॑ भाग॒धेय॑म् भाग॒धेय॑म् कर् मिथु॒या मि॑थु॒या कर्॑ भाग॒धेय᳚म् । \newline
37. कर्॒ भा॒ग॒धेय॑म् भाग॒धेय॑म् कः कर् भाग॒धेय᳚म् । \newline
38. भा॒ग॒धेय॒मिति॑ भाग - धेय᳚म् । \newline
\pagebreak
\markright{ TS 1.3.8.1  \hfill https://www.vedavms.in \hfill}

\section{ TS 1.3.8.1 }

\textbf{TS 1.3.8.1 } \newline
\textbf{Samhita Paata} \newline

आ द॑द ऋ॒तस्य॑ त्वा देवहविः॒ पाशे॒नाऽऽर॑भे॒ धर्.षा॒ मानु॑षान॒द्भ्यस्त्वौष॑धीभ्यः॒ प्रोक्षा᳚म्य॒पां पे॒रुर॑सि स्वा॒त्तं चि॒थ् सदे॑वꣳ ह॒व्यमापो॑ देवीः॒ स्वद॑तैनꣳ॒॒ सं ते᳚ प्रा॒णो वा॒युना॑ गच्छताꣳ॒॒ सं ॅयज॑त्रै॒रङ्गा॑नि॒ सं ॅय॒ज्ञ्प॑तिरा॒शिषा॑ घृ॒तेना॒क्तौ प॒शुं त्रा॑येथाꣳ॒॒ रेव॑तीर् य॒ज्ञ्प॑तिं प्रिय॒धाऽऽ* वि॑श॒तोरो॑ अन्तरिक्ष स॒जूर् दे॒वेन॒ - [ ] \newline

\textbf{Pada Paata} \newline

एति॑ । द॒दे॒ । ऋ॒तस्य॑ । त्वा॒ । दे॒व॒ह॒वि॒रिति॑ देव - ह॒विः॒ । पाशे॑न । एति॑ । र॒भे॒ । धर्.ष॑ । मानु॑षान् । अ॒द्भ्य इत्य॑त् - भ्यः । त्वा॒ । ओष॑धीभ्य॒ इत्योष॑धि - भ्यः॒ । प्रेति॑ । उ॒क्षा॒मि॒ । अ॒पाम् । पे॒रुः । अ॒सि॒ । स्वा॒त्तम् । चि॒त् । सदे॑व॒मिति॒ स - दे॒व॒म् । ह॒व्यम् । आपः॑ । दे॒वीः॒ । स्वद॑त । ए॒न॒म् । समिति॑ । ते॒ । प्रा॒ण इति॑ प्र - अ॒नः । वा॒युना᳚ । ग॒च्छ॒ता॒म् । समिति॑ । यज॑त्रैः । अङ्गा॑नि । समिति॑ । य॒ज्ञ्प॑ति॒रिति॑ य॒ज्ञ् - प॒तिः॒ । आ॒शिषेत्या᳚ - शिषा᳚ । घृ॒तेन॑ । अ॒क्तौ । प॒शुम् । त्रा॒ये॒था॒म् । रेव॑तीः । य॒ज्ञ्प॑ति॒मिति॑ य॒ज्ञ् - प॒ति॒म् । प्रि॒य॒धेति॑ प्रिय - धा । एति॑ । वि॒श॒त॒ । उरो॒ इति॑ । अ॒न्त॒रि॒क्ष॒ । स॒जूरिति॑ स - जूः । दे॒वेन॑ ।  \newline


\textbf{Krama Paata} \newline

आ द॑दे । द॒द॒ ऋ॒तस्य॑ । ऋ॒तस्य॑ त्वा । त्वा॒ दे॒व॒ह॒विः॒ । दे॒व॒ह॒विः॒ पाशे॑न । दे॒व॒ह॒वि॒रिति॑ देव - ह॒विः॒ । पाशे॒ना । आ र॑भे । र॒भे॒ धर्.ष॑ । धर्.षा॒ मानु॑षान् । मानु॑षान॒द्भ्यः । अ॒द्भ्यस्त्वा᳚ । अ॒द्भ्य इत्य॑त् - भ्यः । त्वौष॑धीभ्यः । ओष॑धीभ्यः॒ प्र । ओष॑धीभ्य॒ इत्योष॑धि - भ्यः॒ । प्रोक्षा॑मि । उ॒क्षा॒म्य॒पाम् । अ॒पाम् पे॒रुः । पे॒रुर॑सि । अ॒सि॒ स्वा॒त्तम् । स्वा॒त्तम् चि॑त् । चि॒थ् सदे॑वम् । सदे॑वꣳ ह॒व्यम् । सदे॑व॒मिति॒ स - दे॒व॒म् । ह॒व्यमापः॑ । आपो॑ देवीः । दे॒वीः॒ स्वद॑त । स्वद॑तैनम् । ए॒नꣳ॒॒ सम् । सम् ते᳚ । ते॒ प्रा॒णः । प्रा॒णो वा॒युना᳚ । प्रा॒ण इति॑ प्र - अ॒नः । वा॒युना॑ गच्छताम् । ग॒च्छ॒ताꣳ॒॒ सम् । सं ॅयज॑त्रैः । यज॑त्रै॒रङ्गा॑नि । अङ्गा॑नि॒ सम् । सं ॅय॒ज्ञ्प॑तिः । य॒ज्ञ्॑पतिरा॒शिषा᳚ । य॒ज्ञ्प॑ति॒रिति॑ य॒ज्ञ् - प॒तिः॒ । आ॒शिषा॑ घृ॒तेन॑ । आ॒शिषेत्या᳚ - शिषा᳚ । घृ॒तेना॒क्तौ । अ॒क्तौ प॒शुम् । प॒शुम् त्रा॑येथाम् । त्रा॒ये॒थाꣳ॒॒ रेव॑तीः । रेव॑तीर् य॒ज्ञ्प॑तिम् । य॒ज्ञ्प॑तिम् प्रिय॒धा । य॒ज्ञ्प॑ति॒मिति॑ य॒ज्ञ् - प॒ति॒म् । प्रि॒य॒धा ऽऽ वि॑शत । प्रि॒य॒धेति॑ प्रिय - धा । आ वि॑शत । वि॒श॒तोरो᳚ । उरो॑ अन्तरिक्ष । उरो॒ इत्युरो᳚ । अ॒न्त॒रि॒क्ष॒ स॒जूः । स॒जूर् दे॒वेन॑ ( ) । स॒जूरिति॑ स - जूः । दे॒वेन॒ वाते॑न \newline

\textbf{Jatai Paata} \newline

1. आ द॑दे दद॒ आ द॑दे । \newline
2. द॒द॒ ऋ॒तस्य॒ र्तस्य॑ ददे दद ऋ॒तस्य॑ । \newline
3. ऋ॒तस्य॑ त्वा त्व॒र्तस्य॒ र्तस्य॑ त्वा । \newline
4. त्वा॒ दे॒व॒ह॒वि॒र् दे॒व॒ह॒वि॒ स्त्वा॒ त्वा॒ दे॒व॒ह॒विः॒ । \newline
5. दे॒व॒ह॒विः॒ पाशे॑न॒ पाशे॑न देवहविर् देवहविः॒ पाशे॑न । \newline
6. दे॒व॒ह॒वि॒रिति॑ देव - ह॒विः॒ । \newline
7. पाशे॒ना पाशे॑न॒ पाशे॒ना । \newline
8. आ र॑भे रभ॒ आ र॑भे । \newline
9. र॒भे॒ धर्.ष॒ धर्.ष॑ रभे रभे॒ धर्.ष॑ । \newline
10. धर्.षा॒ मानु॑षा॒न् मानु॑षा॒न् धर्.ष॒ धर्.षा॒ मानु॑षान् । \newline
11. मानु॑षा न॒द्भ्यो᳚ ऽद्भ्यो मानु॑षा॒न् मानु॑षा न॒द्भ्यः । \newline
12. अ॒द्भ्य स्त्वा᳚ त्वा॒ ऽद्भ्यो᳚ ऽद्भ्य स्त्वा᳚ । \newline
13. अ॒द्भ्य इत्य॑त् - भ्यः । \newline
14. त्वौष॑धीभ्य॒ ओष॑धीभ्य स्त्वा॒ त्वौष॑धीभ्यः । \newline
15. ओष॑धीभ्यः॒ प्र प्रौष॑धीभ्य॒ ओष॑धीभ्यः॒ प्र । \newline
16. ओष॑धीभ्य॒ इत्योष॑धि - भ्यः॒ । \newline
17. प्रोक्षा᳚ म्युक्षामि॒ प्र प्रोक्षा॑मि । \newline
18. उ॒क्षा॒ म्य॒पा म॒पा मु॑क्षा म्युक्षा म्य॒पाम् । \newline
19. अ॒पाम् पे॒रुः पे॒रु र॒पा म॒पाम् पे॒रुः । \newline
20. पे॒रु र॑स्यसि पे॒रुः पे॒रु र॑सि । \newline
21. अ॒सि॒ स्वा॒त्तꣳ स्वा॒त्त म॑स्यसि स्वा॒त्तम् । \newline
22. स्वा॒त्तम् चि॑च् चिथ् स्वा॒त्तꣳ स्वा॒त्तम् चि॑त् । \newline
23. चि॒थ् सदे॑व॒(ग्म्॒) सदे॑वम् चिच् चि॒थ् सदे॑वम् । \newline
24. सदे॑वꣳ ह॒व्यꣳ ह॒व्यꣳ सदे॑व॒(ग्म्॒) सदे॑वꣳ ह॒व्यम् । \newline
25. सदे॑व॒मिति॒ स - दे॒व॒म् । \newline
26. ह॒व्य माप॒ आपो॑ ह॒व्यꣳ ह॒व्य मापः॑ । \newline
27. आपो॑ देवीर् देवी॒ राप॒ आपो॑ देवीः । \newline
28. दे॒वीः॒ स्वद॑त॒ स्वद॑त देवीर् देवीः॒ स्वद॑त । \newline
29. स्वद॑तैन मेन॒(ग्ग्॒) स्वद॑त॒ स्वद॑ तैनम् । \newline
30. ए॒न॒(ग्म्॒) सꣳ स मे॑न मेन॒(ग्म्॒) सम् । \newline
31. सम् ते॑ ते॒ सꣳ सम् ते᳚ । \newline
32. ते॒ प्रा॒णः प्रा॒ण स्ते॑ ते प्रा॒णः । \newline
33. प्रा॒णो वा॒युना॑ वा॒युना᳚ प्रा॒णः प्रा॒णो वा॒युना᳚ । \newline
34. प्रा॒ण इति॑ प्र - अ॒नः । \newline
35. वा॒युना॑ गच्छताम् गच्छतां ॅवा॒युना॑ वा॒युना॑ गच्छताम् । \newline
36. ग॒च्छ॒ता॒(ग्म्॒) सꣳ सम् ग॑च्छताम् गच्छता॒(ग्म्॒) सम् । \newline
37. सं ॅयज॑त्रै॒र् यज॑त्रैः॒ सꣳ सं ॅयज॑त्रैः । \newline
38. यज॑त्रै॒ रङ्गा॒ न्यङ्गा॑नि॒ यज॑त्रै॒र् यज॑त्रै॒ रङ्गा॑नि । \newline
39. अङ्गा॑नि॒ सꣳ स मङ्गा॒ न्यङ्गा॑नि॒ सम् । \newline
40. सं ॅय॒ज्ञ्प॑तिर् य॒ज्ञ्प॑तिः॒ सꣳ सं ॅय॒ज्ञ्प॑तिः । \newline
41. य॒ज्ञ्प॑ति रा॒शिषा॒ ऽऽशिषा॑ य॒ज्ञ्प॑तिर् य॒ज्ञ्प॑ति रा॒शिषा᳚ । \newline
42. य॒ज्ञ्प॑ति॒रिति॑ य॒ज्ञ् - प॒तिः॒ । \newline
43. आ॒शिषा॑ घृ॒तेन॑ घृ॒तेना॒शिषा॒ ऽऽशिषा॑ घृ॒तेन॑ । \newline
44. आ॒शिषेत्या᳚ - शिषा᳚ । \newline
45. घृ॒ते ना॒क्ता व॒क्तौ घृ॒तेन॑ घृ॒ते ना॒क्तौ । \newline
46. अ॒क्तौ प॒शुम् प॒शु म॒क्ता व॒क्तौ प॒शुम् । \newline
47. प॒शुम् त्रा॑येथाम् त्रायेथाम् प॒शुम् प॒शुम् त्रा॑येथाम् । \newline
48. त्रा॒ये॒था॒(ग्म्॒) रेव॑ती॒ रेव॑ती स्त्रायेथाम् त्रायेथा॒(ग्म्॒) रेव॑तीः । \newline
49. रेव॑तीर् य॒ज्ञ्प॑तिं ॅय॒ज्ञ्प॑ति॒(ग्म्॒) रेव॑ती॒ रेव॑तीर् य॒ज्ञ्प॑तिम् । \newline
50. य॒ज्ञ्प॑तिम् प्रिय॒धा प्रि॑य॒धा य॒ज्ञ्प॑तिं ॅय॒ज्ञ्प॑तिम् प्रिय॒धा । \newline
51. य॒ज्ञ्प॑ति॒मिति॑ य॒ज्ञ् - प॒ति॒म् । \newline
52. प्रि॒य॒धा ऽऽवि॑शत विश॒ता प्रि॑य॒धा प्रि॑य॒धा ऽऽवि॑शत । \newline
53. प्रि॒य॒धेति॑ प्रिय - धा । \newline
54. आ वि॑शत विश॒ता वि॑शत । \newline
55. वि॒श॒तोरो॒ उरो॑ विशत विश॒तोरो᳚ । \newline
56. उरो॑ अन्तरिक् षान्तरि॒क्षोरो॒ उरो॑ अन्तरिक्ष । \newline
57. उरो॒ इत्युरो᳚ । \newline
58. अ॒न्त॒रि॒क्ष॒ स॒जूः स॒जू र॑न्तरिक् षान्तरिक्ष स॒जूः । \newline
59. स॒जूर् दे॒वेन॑ दे॒वेन॑ स॒जूः स॒जूर् दे॒वेन॑ । \newline
60. स॒जूरिति॑ स - जूः । \newline
61. दे॒वेन॒ वाते॑न॒ वाते॑न दे॒वेन॑ दे॒वेन॒ वाते॑न । \newline

\textbf{Ghana Paata } \newline

1. आ द॑दे दद॒ आ द॑द ऋ॒तस्य॒ र्तस्य॑ दद॒ आ द॑द ऋ॒तस्य॑ । \newline
2. द॒द॒ ऋ॒तस्य॒ र्तस्य॑ ददे दद ऋ॒तस्य॑ त्वा त्व॒र्तस्य॑ ददे दद ऋ॒तस्य॑ त्वा । \newline
3. ऋ॒तस्य॑ त्वा त्व॒र्तस्य॒ र्तस्य॑ त्वा देवहविर् देवहवि स्त्व॒र्तस्य॒ र्तस्य॑ त्वा देवहविः । \newline
4. त्वा॒ दे॒व॒ह॒वि॒र् दे॒व॒ह॒वि॒ स्त्वा॒ त्वा॒ दे॒व॒ह॒विः॒ पाशे॑न॒ पाशे॑न देवहवि स्त्वा त्वा देवहविः॒ पाशे॑न । \newline
5. दे॒व॒ह॒विः॒ पाशे॑न॒ पाशे॑न देवहविर् देवहविः॒ पाशे॒ना पाशे॑न देवहविर् देवहविः॒ पाशे॒ना । \newline
6. दे॒व॒ह॒वि॒रिति॑ देव - ह॒विः॒ । \newline
7. पाशे॒ना पाशे॑न॒ पाशे॒ना र॑भे रभ॒ आ पाशे॑न॒ पाशे॒ना र॑भे । \newline
8. आ र॑भे रभ॒ आ र॑भे॒ धर्.ष॒ धर्.ष॑ रभ॒ आ र॑भे॒ धर्.ष॑ । \newline
9. र॒भे॒ धर्.ष॒ धर्.ष॑ रभे रभे॒ धर्.षा॒ मानु॑षा॒न् मानु॑षा॒न् धर्.ष॑ रभे रभे॒ धर्.षा॒ मानु॑षान् । \newline
10. धर्.षा॒ मानु॑षा॒न् मानु॑षा॒न् धर्.ष॒ धर्.षा॒ मानु॑षा न॒द्भ्यो᳚ ऽद्भ्यो मानु॑षा॒न् धर्.ष॒ धर्.षा॒ मानु॑षा न॒द्भ्यः । \newline
11. मानु॑षा न॒द्भ्यो᳚ ऽद्भ्यो मानु॑षा॒न् मानु॑षा न॒द्भ्य स्त्वा᳚ त्वा॒ ऽद्भ्यो मानु॑षा॒न् मानु॑षा न॒द्भ्य स्त्वा᳚ । \newline
12. अ॒द्भ्य स्त्वा᳚ त्वा॒ ऽद्भ्यो᳚ ऽद्भ्य स्त्वौष॑धीभ्य॒ ओष॑धीभ्य स्त्वा॒ ऽद्भ्यो᳚ ऽद्भ्य स्त्वौष॑धीभ्यः । \newline
13. अ॒द्भ्य इत्य॑त् - भ्यः । \newline
14. त्वौष॑धीभ्य॒ ओष॑धीभ्य स्त्वा॒ त्वौष॑धीभ्यः॒ प्र प्रौष॑धीभ्य स्त्वा॒ त्वौष॑धीभ्यः॒ प्र । \newline
15. ओष॑धीभ्यः॒ प्र प्रौष॑धीभ्य॒ ओष॑धीभ्यः॒ प्रोक्षा᳚ म्युक्षामि॒ प्रौष॑धीभ्य॒ ओष॑धीभ्यः॒ प्रोक्षा॑मि । \newline
16. ओष॑धीभ्य॒ इत्योष॑धि - भ्यः॒ । \newline
17. प्रोक्षा᳚ म्युक्षामि॒ प्र प्रोक्षा᳚ म्य॒पा म॒पा मु॑क्षामि॒ प्र प्रोक्षा᳚ म्य॒पाम् । \newline
18. उ॒क्षा॒ म्य॒पा म॒पा मु॑क्षा म्युक्षा म्य॒पाम् पे॒रुः पे॒रुर॒पा मु॑क्षा म्युक्षा म्य॒पाम् पे॒रुः । \newline
19. अ॒पाम् पे॒रुः पे॒रु र॒पा म॒पाम् पे॒रु र॑स्यसि पे॒रुर॒पा म॒पाम् पे॒रु र॑सि । \newline
20. पे॒रु र॑स्यसि पे॒रुः पे॒रु र॑सि स्वा॒त्तꣳ स्वा॒त्त म॑सि पे॒रुः पे॒रु र॑सि स्वा॒त्तम् । \newline
21. अ॒सि॒ स्वा॒त्तꣳ स्वा॒त्त म॑स्यसि स्वा॒त्तम् चि॑च् चिथ् स्वा॒त्त म॑स्यसि स्वा॒त्तम् चि॑त् । \newline
22. स्वा॒त्तम् चि॑च् चिथ् स्वा॒त्तꣳ स्वा॒त्तम् चि॒थ् सदे॑व॒(ग्म्॒) सदे॑वम् चिथ् स्वा॒त्तꣳ स्वा॒त्तम् चि॒थ् सदे॑वम् । \newline
23. चि॒थ् सदे॑व॒(ग्म्॒) सदे॑वम् चिच् चि॒थ् सदे॑वꣳ ह॒व्यꣳ ह॒व्यꣳ सदे॑वम् चिच् चि॒थ् सदे॑वꣳ ह॒व्यम् । \newline
24. सदे॑वꣳ ह॒व्यꣳ ह॒व्यꣳ सदे॑व॒(ग्म्॒) सदे॑वꣳ ह॒व्य माप॒ आपो॑ ह॒व्यꣳ सदे॑व॒(ग्म्॒) सदे॑वꣳ ह॒व्य मापः॑ । \newline
25. सदे॑व॒मिति॒ स - दे॒व॒म् । \newline
26. ह॒व्य माप॒ आपो॑ ह॒व्यꣳ ह॒व्य मापो॑ देवीर् देवी॒रापो॑ ह॒व्यꣳ ह॒व्य मापो॑ देवीः । \newline
27. आपो॑ देवीर् देवी॒ राप॒ आपो॑ देवीः॒ स्वद॑त॒ स्वद॑त देवी॒ राप॒ आपो॑ देवीः॒ स्वद॑त । \newline
28. दे॒वीः॒ स्वद॑त॒ स्वद॑त देवीर् देवीः॒ स्वद॑तैन मेन॒(ग्ग्॒) स्वद॑त देवीर् देवीः॒ स्वद॑तैनम् । \newline
29. स्वद॑तैन मेन॒(ग्ग्॒) स्वद॑त॒ स्वद॑तैन॒(ग्म्॒) सꣳ स मे॑न॒(ग्ग्॒) स्वद॑त॒ स्वद॑तैन॒(ग्म्॒) सम् । \newline
30. ए॒न॒(ग्म्॒) सꣳ स मे॑न मेन॒(ग्म्॒) सम् ते॑ ते॒ स मे॑न मेन॒(ग्म्॒) सम् ते᳚ । \newline
31. सम् ते॑ ते॒ सꣳ सम् ते᳚ प्रा॒णः प्रा॒ण स्ते॒ सꣳ सम् ते᳚ प्रा॒णः । \newline
32. ते॒ प्रा॒णः प्रा॒ण स्ते॑ ते प्रा॒णो वा॒युना॑ वा॒युना᳚ प्रा॒ण स्ते॑ ते प्रा॒णो वा॒युना᳚ । \newline
33. प्रा॒णो वा॒युना॑ वा॒युना᳚ प्रा॒णः प्रा॒णो वा॒युना॑ गच्छताम् गच्छतां ॅवा॒युना᳚ प्रा॒णः प्रा॒णो वा॒युना॑ गच्छताम् । \newline
34. प्रा॒ण इति॑ प्र - अ॒नः । \newline
35. वा॒युना॑ गच्छताम् गच्छतां ॅवा॒युना॑ वा॒युना॑ गच्छता॒(ग्म्॒) सꣳ सम् ग॑च्छतां ॅवा॒युना॑ वा॒युना॑ गच्छता॒(ग्म्॒) सम् । \newline
36. ग॒च्छ॒ता॒(ग्म्॒) सꣳ सम् ग॑च्छताम् गच्छता॒(ग्म्॒) सं ॅयज॑त्रै॒र् यज॑त्रैः॒ सम् ग॑च्छताम् गच्छता॒(ग्म्॒) सं ॅयज॑त्रैः । \newline
37. सं ॅयज॑त्रै॒र् यज॑त्रैः॒ सꣳ सं ॅयज॑त्रै॒ रङ्गा॒ न्यङ्गा॑नि॒ यज॑त्रैः॒ सꣳ सं ॅयज॑त्रै॒ रङ्गा॑नि । \newline
38. यज॑त्रै॒ रङ्गा॒ न्यङ्गा॑नि॒ यज॑त्रै॒र् यज॑त्रै॒ रङ्गा॑नि॒ सꣳ स मङ्गा॑नि॒ यज॑त्रै॒र् यज॑त्रै॒ रङ्गा॑नि॒ सम् । \newline
39. अङ्गा॑नि॒ सꣳ स मङ्गा॒न्यङ्गा॑नि॒ सं ॅय॒ज्ञ्प॑तिर् य॒ज्ञ्प॑तिः॒ स मङ्गा॒न्यङ्गा॑नि॒ सं ॅय॒ज्ञ्प॑तिः । \newline
40. सं ॅय॒ज्ञ्प॑तिर् य॒ज्ञ्प॑तिः॒ सꣳ सं ॅय॒ज्ञ्प॑ति रा॒शिषा॒ ऽऽशिषा॑ य॒ज्ञ्प॑तिः॒ सꣳ सं ॅय॒ज्ञ्प॑ति रा॒शिषा᳚ । \newline
41. य॒ज्ञ्प॑ति रा॒शिषा॒ ऽऽशिषा॑ य॒ज्ञ्प॑तिर् य॒ज्ञ्प॑ति रा॒शिषा॑ घृ॒तेन॑ घृ॒तेना॒शिषा॑ य॒ज्ञ्प॑तिर् य॒ज्ञ्प॑ति रा॒शिषा॑ घृ॒तेन॑ । \newline
42. य॒ज्ञ्प॑ति॒रिति॑ य॒ज्ञ् - प॒तिः॒ । \newline
43. आ॒शिषा॑ घृ॒तेन॑ घृ॒तेना॒शिषा॒ ऽऽशिषा॑ घृ॒तेना॒क्ता व॒क्तौ घृ॒तेना॒शिषा॒ ऽऽशिषा॑ घृ॒तेना॒क्तौ । \newline
44. आ॒शिषेत्या᳚ - शिषा᳚ । \newline
45. घृ॒तेना॒क्ता व॒क्तौ घृ॒तेन॑ घृ॒तेना॒क्तौ प॒शुम् प॒शु म॒क्तौ घृ॒तेन॑ घृ॒तेना॒क्तौ प॒शुम् । \newline
46. अ॒क्तौ प॒शुम् प॒शु म॒क्ता व॒क्तौ प॒शुम् त्रा॑येथाम् त्रायेथाम् प॒शु म॒क्ता व॒क्तौ प॒शुम् त्रा॑येथाम् । \newline
47. प॒शुम् त्रा॑येथाम् त्रायेथाम् प॒शुम् प॒शुम् त्रा॑येथा॒(ग्म्॒) रेव॑ती॒ रेव॑ती स्त्रायेथाम् प॒शुम् प॒शुम् त्रा॑येथा॒(ग्म्॒) रेव॑तीः । \newline
48. त्रा॒ये॒था॒(ग्म्॒) रेव॑ती॒ रेव॑ती स्त्रायेथाम् त्रायेथा॒(ग्म्॒) रेव॑तीर् य॒ज्ञ्प॑तिं ॅय॒ज्ञ्प॑ति॒(ग्म्॒) रेव॑ती स्त्रायेथाम् त्रायेथा॒(ग्म्॒) रेव॑तीर् य॒ज्ञ्प॑तिम् । \newline
49. रेव॑तीर् य॒ज्ञ्प॑तिं ॅय॒ज्ञ्प॑ति॒(ग्म्॒) रेव॑ती॒ रेव॑तीर् य॒ज्ञ्प॑तिम् प्रिय॒धा प्रि॑य॒धा य॒ज्ञ्प॑ति॒(ग्म्॒) रेव॑ती॒ रेव॑तीर् य॒ज्ञ्प॑तिम् प्रिय॒धा । \newline
50. य॒ज्ञ्प॑तिम् प्रिय॒धा प्रि॑य॒धा य॒ज्ञ्प॑तिं ॅय॒ज्ञ्प॑तिम् प्रिय॒धा ऽऽवि॑शत वि॑श॒ता प्रिय॒धा य॒ज्ञ्प॑तिं ॅय॒ज्ञ्प॑तिम् प्रिय॒धा वि॑शता । \newline
51. य॒ज्ञ्प॑ति॒मिति॑ य॒ज्ञ् - प॒ति॒म् । \newline
52. प्रि॒य॒धा ऽऽवि॑शत विश॒ता प्रि॑य॒धा प्रि॑य॒धा ऽऽवि॑शतो रो॒ उरो ऽऽविश॒ता वि॑श॒तो रो᳚ । \newline
53. प्रि॒य॒धेति॑ प्रिय - धा । \newline
54. आ वि॑शत विश॒ता वि॑श॒तो रो॒ उरो॑ विश॒ता वि॑श॒तो रो᳚ । \newline
55. वि॒श॒तोरो॒ उरो॑ विशत विश॒तो रो॑ अन्तरिक्षा न्तरि॒क्षोरो॑ विशत विश॒तोरो॑ अन्तरिक्ष । \newline
56. उरो॑ अन्तरिक्षा न्तरि॒क्षोरो॒ उरो॑ अन्तरिक्ष स॒जूः स॒जू र॑न्तरि॒क्षोरो॒ उरो॑ अन्तरिक्ष स॒जूः । \newline
57. उरो॒ इत्युरो᳚ । \newline
58. अ॒न्त॒रि॒क्ष॒ स॒जूः स॒जू र॑न्तरिक्षान्तरिक्ष स॒जूर् दे॒वेन॑ दे॒वेन॑ स॒जू र॑न्तरिक्षान्तरिक्ष स॒जूर् दे॒वेन॑ । \newline
59. स॒जूर् दे॒वेन॑ दे॒वेन॑ स॒जूः स॒जूर् दे॒वेन॒ वाते॑न॒ वाते॑न दे॒वेन॑ स॒जूः स॒जूर् दे॒वेन॒ वाते॑न । \newline
60. स॒जूरिति॑ स - जूः । \newline
61. दे॒वेन॒ वाते॑न॒ वाते॑न दे॒वेन॑ दे॒वेन॒ वाते॑ना॒स्यास्य वाते॑न दे॒वेन॑ दे॒वेन॒ वाते॑ना॒स्य । \newline
\pagebreak
\markright{ TS 1.3.8.2  \hfill https://www.vedavms.in \hfill}

\section{ TS 1.3.8.2 }

\textbf{TS 1.3.8.2 } \newline
\textbf{Samhita Paata} \newline

वाते॑ना॒स्य ह॒विष॒स्त्मना॑ यज॒ सम॑स्य त॒नुवा॑ भव॒ वर्.षी॑यो॒ वर्.षी॑यसि य॒ज्ञे य॒ज्ञ्पतिं॑ धाः पृ॑थि॒व्याः सं॒पृचः॑ पाहि॒ नम॑स्त आतानाऽन॒र्वा प्रेहि॑ घृ॒तस्य॑ कु॒॒ल्यामनु॑ स॒ह प्र॒जया॑ स॒ह रा॒यस्पोषे॒णा ऽऽपो॑ देवीः शुद्धायुवः शु॒द्धा यू॒यं दे॒वाꣳ ऊ᳚ढ्वꣳ शु॒द्धा व॒यं परि॑विष्टाः परिवे॒ष्टारो॑ वो भूयास्म ॥ \newline

\textbf{Pada Paata} \newline

वाते॑न । अ॒स्य । ह॒विषः॑ । त्मना᳚ । य॒ज॒ । समिति॑ । अ॒स्य॒ । त॒नुवा᳚ । भ॒व॒ । वर्.षी॑यः । वर्.षी॑यसि । य॒ज्ञे । य॒ज्ञ्प॑ति॒मिति॑ य॒ज्ञ् - प॒ति॒म् । धाः॒ । पृ॒थि॒व्याः । स॒म्पृच॒ इति॑ सम् - पृचः॑ । पा॒हि॒ । नमः॑ । ते॒ । आ॒ता॒नेत्या᳚ - ता॒न॒ । अ॒न॒र्वा । प्रेति॑ । इ॒हि॒ । घृ॒तस्य॑ । कु॒॒ल्याम् । अन्विति॑ । स॒ह । प्र॒जयेति॑ प्र - जया᳚ । स॒ह । रा॒यः । पोषे॑ण । आपः॑ । दे॒वीः॒ । शु॒द्धा॒यु॒व॒ इति॑ शुद्ध - यु॒वः॒ । शु॒द्धाः । यू॒यम् । दे॒वान् । ऊ॒ढ्व॒म् । शु॒द्धाः । व॒यम् । परि॑विष्टा॒ इति॒ परि॑ - वि॒ष्टाः॒ । प॒रि॒वे॒ष्टार॒ इति॑ परि - वे॒ष्टारः॑ । वः॒ । भू॒या॒स्म॒ ॥  \newline


\textbf{Krama Paata} \newline

वाते॑ना॒स्य । अ॒स्य ह॒विषः॑ । ह॒विष॒स्त्मना᳚ । त्मना॑ यज । य॒ज॒ सम् । सम॑स्य । अ॒स्य॒ त॒नुवा᳚ । त॒नुवा॑ भव । भ॒व॒ वर्.षी॑यः । वर्.षी॑यो॒ वर्.षी॑यसि । वर्.षी॑यसि य॒ज्ञे । य॒ज्ञे य॒ज्ञ्प॑तिम् । य॒ज्ञ्प॑तिम् धाः । य॒ज्ञ्प॑ति॒मिति॑ य॒ज्ञ् - प॒ति॒म् । धाः॒ पृ॒थि॒व्याः । पृ॒थि॒व्याः सं॒पृचः॑ । सं॒पृचः॑ पाहि । सं॒पृच॒ इति॑ सं - पृचः॑ । पा॒हि॒ नमः॑ । नम॑स्ते । त॒ आ॒ता॒न॒ । आ॒ता॒ना॒न॒र्वा । आ॒ता॒नेत्या᳚ - ता॒न॒ । अ॒न॒र्वा प्र । प्रेहि॑ । इ॒हि॒ घृ॒तस्य॑ । घृ॒तस्य॑ कु॒ल्याम् । कु॒ल्यामनु॑ । अनु॑ स॒ह । स॒ह प्र॒जया᳚ । प्र॒जया॑ स॒ह । प्र॒जयेति॑ प्र - जया᳚ । स॒ह रा॒यः । रा॒यस्पोषे॑ण । पोषे॒णापः॑ । आपो॑ देवीः । दे॒वीः॒ शु॒द्धा॒यु॒वः॒ । शु॒द्धा॒यु॒वः॒ शु॒द्धाः । शु॒द्धा॒यु॒व॒ इति॑ शुद्ध - यु॒वः॒ । शु॒द्धा यू॒यम् । यू॒यम् दे॒वान् । दे॒वाꣳ ऊ᳚ढ्वम् । ऊ॒ढ्वꣳ॒॒ शु॒द्धाः । शु॒द्धा व॒यम् । व॒यम् परि॑विष्टाः । परि॑विष्टाः परिवे॒ष्टारः॑ । परि॑विष्टा॒ इति॒ परि॑ - वि॒ष्टाः॒ । प॒रि॒वे॒ष्टारो॑ वः । प॒रि॒वे॒ष्टार॒ इति॑ परि - वे॒ष्टारः॑ । वो॒ भू॒या॒स्म॒ । भू॒या॒स्मेति॑ भूयास्म । \newline

\textbf{Jatai Paata} \newline

1. वाते॑ ना॒स्यास्य वाते॑न॒ वाते॑ ना॒स्य । \newline
2. अ॒स्य ह॒विषो॑ ह॒विषो॒ ऽस्यास्य ह॒विषः॑ । \newline
3. ह॒विष॒ स्त्मना॒ त्मना॑ ह॒विषो॑ ह॒विष॒ स्त्मना᳚ । \newline
4. त्मना॑ यज यज॒ त्मना॒ त्मना॑ यज । \newline
5. य॒ज॒ सꣳ सं ॅय॑ज यज॒ सम् । \newline
6. स म॑स्यास्य॒ सꣳ स म॑स्य । \newline
7. अ॒स्य॒ त॒नुवा॑ त॒नुवा᳚ ऽस्यास्य त॒नुवा᳚ । \newline
8. त॒नुवा॑ भव भव त॒नुवा॑ त॒नुवा॑ भव । \newline
9. भ॒व॒ वर्.षी॑यो॒ वर्.षी॑यो भव भव॒ वर्.षी॑यः । \newline
10. वर्.षी॑यो॒ वर्.षी॑यसि॒ वर्.षी॑यसि॒ वर्.षी॑यो॒ वर्.षी॑यो॒ वर्.षी॑यसि । \newline
11. वर्.षी॑यसि य॒ज्ञे य॒ज्ञे वर्.षी॑यसि॒ वर्.षी॑यसि य॒ज्ञे । \newline
12. य॒ज्ञे य॒ज्ञ्प॑तिं ॅय॒ज्ञ्प॑तिं ॅय॒ज्ञे य॒ज्ञे य॒ज्ञ्प॑तिम् । \newline
13. य॒ज्ञ्प॑तिम् धा धा य॒ज्ञ्प॑तिं ॅय॒ज्ञ्प॑तिम् धाः । \newline
14. य॒ज्ञ्प॑ति॒मिति॑ य॒ज्ञ् - प॒ति॒म् । \newline
15. धाः॒ पृ॒थि॒व्याः पृ॑थि॒व्या धा॑ धाः पृथि॒व्याः । \newline
16. पृ॒थि॒व्याः स॒म्पृचः॑ स॒म्पृचः॑ पृथि॒व्याः पृ॑थि॒व्याः स॒म्पृचः॑ । \newline
17. स॒म्पृचः॑ पाहि पाहि स॒म्पृचः॑ स॒म्पृचः॑ पाहि । \newline
18. स॒म्पृच॒ इति॑ सम् - पृचः॑ । \newline
19. पा॒हि॒ नमो॒ नमः॑ पाहि पाहि॒ नमः॑ । \newline
20. नम॑ स्ते ते॒ नमो॒ नम॑ स्ते । \newline
21. त॒ आ॒ता॒ ना॒ता॒न॒ ते॒ त॒ आ॒ता॒न॒ । \newline
22. आ॒ता॒ ना॒न॒र्वा ऽन॒र्वा ऽऽता॑ नाता नान॒र्वा । \newline
23. आ॒ता॒नेत्या᳚ - ता॒न॒ । \newline
24. अ॒न॒र्वा प्र प्राण॒र्वा ऽन॒र्वा प्र । \newline
25. प्रे ही॑हि॒ प्र प्रे हि॑ । \newline
26. इ॒हि॒ घृ॒तस्य॑ घृ॒तस्ये॑ हीहि घृ॒तस्य॑ । \newline
27. घृ॒तस्य॑ कु॒ल्याम् कु॒ल्याम् घृ॒तस्य॑ घृ॒तस्य॑ कु॒ल्याम् । \newline
28. कु॒ल्या मन्वनु॑ कु॒ल्याम् कु॒ल्या मनु॑ । \newline
29. अनु॑ स॒ह स॒हा न्वनु॑ स॒ह । \newline
30. स॒ह प्र॒जया᳚ प्र॒जया॑ स॒ह स॒ह प्र॒जया᳚ । \newline
31. प्र॒जया॑ स॒ह स॒ह प्र॒जया᳚ प्र॒जया॑ स॒ह । \newline
32. प्र॒जयेति॑ प्र - जया᳚ । \newline
33. स॒ह रा॒यो रा॒यः स॒ह स॒ह रा॒यः । \newline
34. रा॒य स्पोषे॑ण॒ पोषे॑ण रा॒यो रा॒य स्पोषे॑ण । \newline
35. पोषे॒णाप॒ आपः॒ पोषे॑ण॒ पोषे॒णापः॑ । \newline
36. आपो॑ देवीर् देवी॒ राप॒ आपो॑ देवीः । \newline
37. दे॒वीः॒ शु॒द्धा॒यु॒वः॒ शु॒द्धा॒यु॒वो॒ दे॒वी॒र् दे॒वीः॒ शु॒द्धा॒यु॒वः॒ । \newline
38. शु॒द्धा॒यु॒वः॒ शु॒द्धाः शु॒द्धाः शु॑द्धायुवः शुद्धायुवः शु॒द्धाः । \newline
39. शु॒द्धा॒यु॒व॒ इति॑ शुद्ध - यु॒वः॒ । \newline
40. शु॒द्धा यू॒यं ॅयू॒यꣳ शु॒द्धाः शु॒द्धा यू॒यम् । \newline
41. यू॒यम् दे॒वान् दे॒वान्. यू॒यं ॅयू॒यम् दे॒वान् । \newline
42. दे॒वाꣳ ऊ᳚ढ्व मूढ्वम् दे॒वान् दे॒वाꣳ ऊ᳚ढ्वम् । \newline
43. ऊ॒ढ्व॒(ग्म्॒) शु॒द्धाः शु॒द्धा ऊ᳚ढ्व मूढ्वꣳ शु॒द्धाः । \newline
44. शु॒द्धा व॒यं ॅव॒यꣳ शु॒द्धाः शु॒द्धा व॒यम् । \newline
45. व॒यम् परि॑विष्टाः॒ परि॑विष्टा व॒यं ॅव॒यम् परि॑विष्टाः । \newline
46. परि॑विष्टाः परिवे॒ष्टारः॑ परिवे॒ष्टारः॒ परि॑विष्टाः॒ परि॑विष्टाः परिवे॒ष्टारः॑ । \newline
47. परि॑विष्टा॒ इति॒ परि॑ - वि॒ष्टाः॒ । \newline
48. प॒रि॒वे॒ष्टारो॑ वो वः परिवे॒ष्टारः॑ परिवे॒ष्टारो॑ वः । \newline
49. प॒रि॒वे॒ष्टार॒ इति॑ परि - वे॒ष्टारः॑ । \newline
50. वो॒ भू॒या॒स्म॒ भू॒या॒स्म॒ वो॒ वो॒ भू॒या॒स्म॒ । \newline
51. भू॒या॒स्मेति॑ भूयास्म । \newline

\textbf{Ghana Paata } \newline

1. वाते॑ना॒स्यास्य वाते॑न॒ वाते॑ना॒स्य ह॒विषो॑ ह॒विषो॒ ऽस्य वाते॑न॒ वाते॑ना॒स्य ह॒विषः॑ । \newline
2. अ॒स्य ह॒विषो॑ ह॒विषो॒ ऽस्यास्य ह॒विष॒ स्त्मना॒ त्मना॑ ह॒विषो॒ ऽस्यास्य ह॒विष॒ स्त्मना᳚ । \newline
3. ह॒विष॒ स्त्मना॒ त्मना॑ ह॒विषो॑ ह॒विष॒ स्त्मना॑ यज यज॒ त्मना॑ ह॒विषो॑ ह॒विष॒ स्त्मना॑ यज । \newline
4. त्मना॑ यज यज॒ त्मना॒ त्मना॑ यज॒ सꣳ सं ॅय॑ज॒ त्मना॒ त्मना॑ यज॒ सम् । \newline
5. य॒ज॒ सꣳ सं ॅय॑ज यज॒ स म॑स्यास्य॒ सं ॅय॑ज यज॒ स म॑स्य । \newline
6. स म॑स्यास्य॒ सꣳ स म॑स्य त॒नुवा॑ त॒नुवा᳚ ऽस्य॒ सꣳ स म॑स्य त॒नुवा᳚ । \newline
7. अ॒स्य॒ त॒नुवा॑ त॒नुवा᳚ ऽस्यास्य त॒नुवा॑ भव भव त॒नुवा᳚ ऽस्यास्य त॒नुवा॑ भव । \newline
8. त॒नुवा॑ भव भव त॒नुवा॑ त॒नुवा॑ भव॒ वर्.षी॑यो॒ वर्.षी॑यो भव त॒नुवा॑ त॒नुवा॑ भव॒ वर्.षी॑यः । \newline
9. भ॒व॒ वर्.षी॑यो॒ वर्.षी॑यो भव भव॒ वर्.षी॑यो॒ वर्.षी॑यसि॒ वर्.षी॑यसि॒ वर्.षी॑यो भव भव॒ वर्.षी॑यो॒ वर्.षी॑यसि । \newline
10. वर्.षी॑यो॒ वर्.षी॑यसि॒ वर्.षी॑यसि॒ वर्.षी॑यो॒ वर्.षी॑यो॒ वर्.षी॑यसि य॒ज्ञे य॒ज्ञे वर्.षी॑यसि॒ वर्.षी॑यो॒ वर्.षी॑यो॒ वर्.षी॑यसि य॒ज्ञे । \newline
11. वर्.षी॑यसि य॒ज्ञे य॒ज्ञे वर्.षी॑यसि॒ वर्.षी॑यसि य॒ज्ञे य॒ज्ञ्प॑तिं ॅय॒ज्ञ्प॑तिं ॅय॒ज्ञे वर्.षी॑यसि॒ वर्.षी॑यसि य॒ज्ञे य॒ज्ञ्प॑तिम् । \newline
12. य॒ज्ञे य॒ज्ञ्प॑तिं ॅय॒ज्ञ्प॑तिं ॅय॒ज्ञे य॒ज्ञे य॒ज्ञ्प॑तिम् धा धा य॒ज्ञ्प॑तिं ॅय॒ज्ञे य॒ज्ञे य॒ज्ञ्प॑तिम् धाः । \newline
13. य॒ज्ञ्प॑तिम् धा धा य॒ज्ञ्प॑तिं ॅय॒ज्ञ्प॑तिम् धाः पृथि॒व्याः पृ॑थि॒व्या धा॑ य॒ज्ञ्प॑तिं ॅय॒ज्ञ्प॑तिम् धाः पृथि॒व्याः । \newline
14. य॒ज्ञ्प॑ति॒मिति॑ य॒ज्ञ् - प॒ति॒म् । \newline
15. धाः॒ पृ॒थि॒व्याः पृ॑थि॒व्या धा॑ धाः पृथि॒व्याः स॒म्पृचः॑ स॒म्पृचः॑ पृथि॒व्या धा॑ धाः पृथि॒व्याः स॒म्पृचः॑ । \newline
16. पृ॒थि॒व्याः स॒म्पृचः॑ स॒म्पृचः॑ पृथि॒व्याः पृ॑थि॒व्याः स॒म्पृचः॑ पाहि पाहि स॒म्पृचः॑ पृथि॒व्याः पृ॑थि॒व्याः स॒म्पृचः॑ पाहि । \newline
17. स॒म्पृचः॑ पाहि पाहि स॒म्पृचः॑ स॒म्पृचः॑ पाहि॒ नमो॒ नमः॑ पाहि स॒म्पृचः॑ स॒म्पृचः॑ पाहि॒ नमः॑ । \newline
18. स॒म्पृच॒ इति॑ सम् - पृचः॑ । \newline
19. पा॒हि॒ नमो॒ नमः॑ पाहि पाहि॒ नम॑स्ते ते॒ नमः॑ पाहि पाहि॒ नम॑स्ते । \newline
20. नम॑स्ते ते॒ नमो॒ नम॑स्त आतानातान ते॒ नमो॒ नम॑स्त आतान । \newline
21. त॒ आ॒ता॒ना॒ता॒न॒ ते॒ त॒ आ॒ता॒ना॒न॒र्वा ऽन॒र्वा ऽऽता॑न ते त आतानान॒र्वा । \newline
22. आ॒ता॒ना॒न॒र्वा ऽन॒र्वा ऽऽता॑नातानान॒र्वा प्र प्राण॒र्वा ऽऽता॑नातानान॒र्वा प्र । \newline
23. आ॒ता॒नेत्या᳚ - ता॒न॒ । \newline
24. अ॒न॒र्वा प्र प्राण॒र्वा ऽन॒र्वा प्रे ही॑हि॒ प्राण॒र्वा ऽन॒र्वा प्रे हि॑ । \newline
25. प्रे ही॑हि॒ प्र प्रे हि॑ घृ॒तस्य॑ घृ॒तस्ये॑ हि॒ प्र प्रे हि॑ घृ॒तस्य॑ । \newline
26. इ॒हि॒ घृ॒तस्य॑ घृ॒तस्ये॑ हीहि घृ॒तस्य॑ कु॒ल्याम् कु॒ल्याम् घृ॒तस्ये॑ हीहि घृ॒तस्य॑ कु॒ल्याम् । \newline
27. घृ॒तस्य॑ कु॒ल्याम् कु॒ल्याम् घृ॒तस्य॑ घृ॒तस्य॑ कु॒ल्या मन्वनु॑ कु॒ल्याम् घृ॒तस्य॑ घृ॒तस्य॑ कु॒ल्या मनु॑ । \newline
28. कु॒ल्या मन्वनु॑ कु॒ल्याम् कु॒ल्या मनु॑ स॒ह स॒हानु॑ कु॒ल्याम् कु॒ल्या मनु॑ स॒ह । \newline
29. अनु॑ स॒ह स॒हान्वनु॑ स॒ह प्र॒जया᳚ प्र॒जया॑ स॒हान्वनु॑ स॒ह प्र॒जया᳚ । \newline
30. स॒ह प्र॒जया᳚ प्र॒जया॑ स॒ह स॒ह प्र॒जया॑ स॒ह स॒ह प्र॒जया॑ स॒ह स॒ह प्र॒जया॑ स॒ह । \newline
31. प्र॒जया॑ स॒ह स॒ह प्र॒जया᳚ प्र॒जया॑ स॒ह रा॒यो रा॒यः स॒ह प्र॒जया᳚ प्र॒जया॑ स॒ह रा॒यः । \newline
32. प्र॒जयेति॑ प्र - जया᳚ । \newline
33. स॒ह रा॒यो रा॒यः स॒ह स॒ह रा॒य स्पोषे॑ण॒ पोषे॑ण रा॒यः स॒ह स॒ह रा॒य स्पोषे॑ण । \newline
34. रा॒य स्पोषे॑ण॒ पोषे॑ण रा॒यो रा॒य स्पोषे॒णाप॒ आपः॒ पोषे॑ण रा॒यो रा॒य स्पोषे॒णापः॑ । \newline
35. पोषे॒णाप॒ आपः॒ पोषे॑ण॒ पोषे॒णापो॑ देवीर् देवी॒ रापः॒ पोषे॑ण॒ पोषे॒णापो॑ देवीः । \newline
36. आपो॑ देवीर् देवी॒राप॒ आपो॑ देवीः शुद्धायुवः शुद्धायुवो देवी॒राप॒ आपो॑ देवीः शुद्धायुवः । \newline
37. दे॒वीः॒ शु॒द्धा॒यु॒वः॒ शु॒द्धा॒यु॒वो॒ दे॒वी॒र् दे॒वीः॒ शु॒द्धा॒यु॒वः॒ शु॒द्धाः शु॒द्धाः शु॑द्धायुवो देवीर् देवीः शुद्धायुवः शु॒द्धाः । \newline
38. शु॒द्धा॒यु॒वः॒ शु॒द्धाः शु॒द्धाः शु॑द्धायुवः शुद्धायुवः शु॒द्धा यू॒यं ॅयू॒यꣳ शु॒द्धाः शु॑द्धायुवः शुद्धायुवः शु॒द्धा यू॒यम् । \newline
39. शु॒द्धा॒यु॒व॒ इति॑ शुद्ध - यु॒वः॒ । \newline
40. शु॒द्धा यू॒यं ॅयू॒यꣳ शु॒द्धाः शु॒द्धा यू॒यम् दे॒वान् दे॒वान्. यू॒यꣳ शु॒द्धाः शु॒द्धा यू॒यम् दे॒वान् । \newline
41. यू॒यम् दे॒वान् दे॒वान्. यू॒यं ॅयू॒यम् दे॒वाꣳ ऊ᳚ढ्व मूढ्वम् दे॒वान्. यू॒यं ॅयू॒यम् दे॒वाꣳ ऊ᳚ढ्वम् । \newline
42. दे॒वाꣳ ऊ᳚ढ्व मूढ्वम् दे॒वान् दे॒वाꣳ ऊ᳚ढ्वꣳ शु॒द्धाः शु॒द्धा ऊ᳚ढ्वम् दे॒वान् दे॒वाꣳ ऊ᳚ढ्वꣳ शु॒द्धाः । \newline
43. ऊ॒ढ्व॒(ग्म्॒) शु॒द्धाः शु॒द्धा ऊ᳚ढ्व मूढ्वꣳ शु॒द्धा व॒यं ॅव॒यꣳ शु॒द्धा ऊ᳚ढ्व मूढ्वꣳ शु॒द्धा व॒यम् । \newline
44. शु॒द्धा व॒यं ॅव॒यꣳ शु॒द्धाः शु॒द्धा व॒यम् परि॑विष्टाः॒ परि॑विष्टा व॒यꣳ शु॒द्धाः शु॒द्धा व॒यम् परि॑विष्टाः । \newline
45. व॒यम् परि॑विष्टाः॒ परि॑विष्टा व॒यं ॅव॒यम् परि॑विष्टाः परिवे॒ष्टारः॑ परिवे॒ष्टारः॒ परि॑विष्टा व॒यं ॅव॒यम् परि॑विष्टाः परिवे॒ष्टारः॑ । \newline
46. परि॑विष्टाः परिवे॒ष्टारः॑ परिवे॒ष्टारः॒ परि॑विष्टाः॒ परि॑विष्टाः परिवे॒ष्टारो॑ वो वः परिवे॒ष्टारः॒ परि॑विष्टाः॒ परि॑विष्टाः परिवे॒ष्टारो॑ वः । \newline
47. परि॑विष्टा॒ इति॒ परि॑ - वि॒ष्टाः॒ । \newline
48. प॒रि॒वे॒ष्टारो॑ वो वः परिवे॒ष्टारः॑ परिवे॒ष्टारो॑ वो भूयास्म भूयास्म वः परिवे॒ष्टारः॑ परिवे॒ष्टारो॑ वो भूयास्म । \newline
49. प॒रि॒वे॒ष्टार॒ इति॑ परि - वे॒ष्टारः॑ । \newline
50. वो॒ भू॒या॒स्म॒ भू॒या॒स्म॒ वो॒ वो॒ भू॒या॒स्म॒ । \newline
51. भू॒या॒स्मेति॑ भूयास्म । \newline
\pagebreak
\markright{ TS 1.3.9.1  \hfill https://www.vedavms.in \hfill}

\section{ TS 1.3.9.1 }

\textbf{TS 1.3.9.1 } \newline
\textbf{Samhita Paata} \newline

वाक्त॒ आ प्या॑यतां प्रा॒णस्त॒ आ प्या॑यतां॒ चक्षु॑स्त॒ आ प्या॑यताꣳ॒॒ श्रोत्रं॑ त॒ आ प्या॑यतां॒ ॅया ते᳚ प्रा॒णाञ्छुग्ज॒गाम॒ या चक्षु॒र्या श्रोत्रं॒ ॅयत्ते᳚ क्रू॒रं ॅयदास्थि॑तं॒ तत्त॒ आ प्या॑यतां॒ तत्त॑ ए॒तेन॑ शुन्धतां॒ नाभि॑स्त॒ आ प्या॑यतां पा॒युस्त॒ आ प्या॑यताꣳ शु॒द्धाश्च॒रित्राः॒ शम॒द्भ्यः - [ ] \newline

\textbf{Pada Paata} \newline

वाक् । ते॒ । एति॑ । प्या॒य॒ता॒म् । प्रा॒ण इति॑ प्र - अ॒नः । ते॒ । एति॑ । प्या॒य॒ता॒म् । चक्षुः॑ । ते॒ । एति॑ । प्या॒य॒ता॒म् । श्रोत्र᳚म् । ते॒ । एति॑ । प्या॒य॒ता॒म् । या । ते॒ । प्रा॒णानिति॑ प्र - अ॒नान् । शुक् । ज॒गाम॑ । या । चक्षुः॑ । या । श्रोत्र᳚म् । यत् । ते॒ । क्रू॒रम् । यत् । आस्थि॑त॒मित्या - स्थि॒त॒म् । तत् । ते॒ । एति॑ । प्या॒य॒ता॒म् । तत् । ते॒ । ए॒तेन॑ । शु॒न्ध॒ता॒म् । नाभिः॑ । ते॒ । एति॑ । प्या॒य॒ता॒म् । पा॒युः । ते॒ । एति॑ । प्या॒य॒ता॒म् । शु॒द्धाः । च॒रित्राः᳚ । शम् । अ॒द्भ्य इत्य॑त् - भ्यः ।  \newline


\textbf{Krama Paata} \newline

वाक्ते᳚ । त॒ आ । आ प्या॑यताम् । प्या॒य॒ता॒म् प्रा॒णः । प्रा॒णस्ते᳚ । प्रा॒ण इति॑ प्र - अ॒नः । त॒ आ । आ प्या॑यताम् । प्या॒य॒ता॒म् चक्षुः॑ । चक्षु॑स्ते । त॒ आ । आ प्या॑यताम् । प्या॒य॒ताꣳ॒॒ श्रोत्र᳚म् । श्रोत्र॑म् ते । त॒ आ । आ प्या॑यताम् । प्या॒य॒तां॒ ॅया । या ते᳚ । ते॒ प्रा॒णान् । प्रा॒णाञ्छुक् । प्रा॒णानिति॑ प्र - अ॒नान् । शुग् ज॒गाम॑ । ज॒गाम॒ या । या चक्षुः॑ । चक्षु॒र् या । या श्रोत्र᳚म् । श्रोत्रं॒ ॅयत् । यत्ते᳚ । ते॒ क्रू॒रम् । क्रू॒रं ॅयत् । यदास्थि॑तम् । आस्थि॑त॒म् तत् । आस्थि॑त॒मित्या - स्थि॒त॒म् । तत्ते᳚ । त॒ आ । आ प्या॑यताम् । प्या॒य॒ता॒म् तत् । तत्ते᳚ । त॒ ए॒तेन॑ । ए॒तेन॑ शुन्धताम् । शु॒न्ध॒ता॒म् नाभिः॑ । नाभि॑स्ते । त॒ आ । आ प्या॑यताम् । प्या॒य॒ता॒म् पा॒युः । पा॒युस्ते᳚ । त॒ आ । आ प्या॑यताम् । प्या॒य॒ताꣳ॒॒ शु॒द्धाः । शु॒द्धाश्च॒रित्राः᳚ । च॒रित्राः॒ शम् । शम॒द्भ्यः । अ॒द्भ्यः शम् । अ॒द्भ्य इत्य॑त् - भ्यः \newline

\textbf{Jatai Paata} \newline

1. वाक् ते॑ ते॒ वाग् वाक् ते᳚ । \newline
2. त॒ आ ते॑ त॒ आ । \newline
3. आ प्या॑यताम् प्यायता॒ मा प्या॑यताम् । \newline
4. प्या॒य॒ता॒म् प्रा॒णः प्रा॒णः प्या॑यताम् प्यायताम् प्रा॒णः । \newline
5. प्रा॒ण स्ते॑ ते प्रा॒णः प्रा॒ण स्ते᳚ । \newline
6. प्रा॒ण इति॑ प्र - अ॒नः । \newline
7. त॒ आ ते॑ त॒ आ । \newline
8. आ प्या॑यताम् प्यायता॒ मा प्या॑यताम् । \newline
9. प्या॒य॒ता॒म् चक्षु॒ श्चक्षुः॑ प्यायताम् प्यायता॒म् चक्षुः॑ । \newline
10. चक्षु॑ स्ते ते॒ चक्षु॒ श्चक्षु॑ स्ते । \newline
11. त॒ आ ते॑ त॒ आ । \newline
12. आ प्या॑यताम् प्यायता॒ मा प्या॑यताम् । \newline
13. प्या॒य॒ता॒(ग्ग्॒) श्रोत्र॒(ग्ग्॒) श्रोत्र॑म् प्यायताम् प्यायता॒(ग्ग्॒) श्रोत्र᳚म् । \newline
14. श्रोत्र॑म् ते ते॒ श्रोत्र॒(ग्ग्॒) श्रोत्र॑म् ते । \newline
15. त॒ आ ते॑ त॒ आ । \newline
16. आ प्या॑यताम् प्यायता॒ मा प्या॑यताम् । \newline
17. प्या॒य॒तां॒ ॅया या प्या॑यताम् प्यायतां॒ ॅया । \newline
18. या ते॑ ते॒ या या ते᳚ । \newline
19. ते॒ प्रा॒णान् प्रा॒णाꣳ स्ते॑ ते प्रा॒णान् । \newline
20. प्रा॒णाञ् छुक् छुक् प्रा॒णान् प्रा॒णाञ् छुक् । \newline
21. प्रा॒णानिति॑ प्र - अ॒नान् । \newline
22. शुग् ज॒गाम॑ ज॒गाम॒ शुक् छुग् ज॒गाम॑ । \newline
23. ज॒गाम॒ या या ज॒गाम॑ ज॒गाम॒ या । \newline
24. या चक्षु॒ श्चक्षु॒र् या या चक्षुः॑ । \newline
25. चक्षु॒र् या या चक्षु॒ श्चक्षु॒र् या । \newline
26. या श्रोत्र॒(ग्ग्॒) श्रोत्रं॒ ॅया या श्रोत्र᳚म् । \newline
27. श्रोत्रं॒ ॅयद् यच्छ्रोत्र॒(ग्ग्॒) श्रोत्रं॒ ॅयत् । \newline
28. यत् ते॑ ते॒ यद् यत् ते᳚ । \newline
29. ते॒ क्रू॒रम् क्रू॒रम् ते॑ ते क्रू॒रम् । \newline
30. क्रू॒रं ॅयद् यत् क्रू॒रम् क्रू॒रं ॅयत् । \newline
31. यदास्थि॑त॒ मास्थि॑तं॒ ॅयद् यदास्थि॑तम् । \newline
32. आस्थि॑त॒म् तत् तदास्थि॑त॒ मास्थि॑त॒म् तत् । \newline
33. आस्थि॑त॒मित्या - स्थि॒त॒म् । \newline
34. तत् ते॑ ते॒ तत् तत् ते᳚ । \newline
35. त॒ आ ते॑ त॒ आ । \newline
36. आ प्या॑यताम् प्यायता॒ मा प्या॑यताम् । \newline
37. प्या॒य॒ता॒म् तत् तत् प्या॑यताम् प्यायता॒म् तत् । \newline
38. तत् ते॑ ते॒ तत् तत् ते᳚ । \newline
39. त॒ ए॒ते नै॒तेन॑ ते त ए॒तेन॑ । \newline
40. ए॒तेन॑ शुन्धताꣳ शुन्धता मे॒तेनै॒तेन॑ शुन्धताम् । \newline
41. शु॒न्ध॒ता॒म् नाभि॒र् नाभिः॑ शुन्धताꣳ शुन्धता॒म् नाभिः॑ । \newline
42. नाभि॑ स्ते ते॒ नाभि॒र् नाभि॑ स्ते । \newline
43. त॒ आ ते॑ त॒ आ । \newline
44. आ प्या॑यताम् प्यायता॒ मा प्या॑यताम् । \newline
45. प्या॒य॒ता॒म् पा॒युः पा॒युः प्या॑यताम् प्यायताम् पा॒युः । \newline
46. पा॒यु स्ते॑ ते पा॒युः पा॒यु स्ते᳚ । \newline
47. त॒ आ ते॑ त॒ आ । \newline
48. आ प्या॑यताम् प्यायता॒ मा प्या॑यताम् । \newline
49. प्या॒य॒ता॒(ग्म्॒) शु॒द्धाः शु॒द्धाः प्या॑यताम् प्यायताꣳ शु॒द्धाः । \newline
50. शु॒द्धा श्च॒रित्रा᳚ श्च॒रित्राः᳚ शु॒द्धाः शु॒द्धा श्च॒रित्राः᳚ । \newline
51. च॒रित्राः॒ शꣳ शम् च॒रित्रा᳚ श्च॒रित्राः॒ शम् । \newline
52. श म॒द्भ्यो᳚ ऽद्भ्यः शꣳ श म॒द्भ्यः । \newline
53. अ॒द्भ्यः शꣳ श म॒द्भ्यो᳚ ऽद्भ्यः शम् । \newline
54. अ॒द्भ्य इत्य॑त् - भ्यः । \newline

\textbf{Ghana Paata } \newline

1. वाक् ते॑ ते॒ वाग् वाक् त॒ आ ते॒ वाग् वाक् त॒ आ । \newline
2. त॒ आ ते॑ त॒ आ प्या॑यताम् प्यायता॒ मा ते॑ त॒ आ प्या॑यताम् । \newline
3. आ प्या॑यताम् प्यायता॒ मा प्या॑यताम् प्रा॒णः प्रा॒णः प्या॑यता॒ मा प्या॑यताम् प्रा॒णः । \newline
4. प्या॒य॒ता॒म् प्रा॒णः प्रा॒णः प्या॑यताम् प्यायताम् प्रा॒ण स्ते॑ ते प्रा॒णः प्या॑यताम् प्यायताम् प्रा॒ण स्ते᳚ । \newline
5. प्रा॒ण स्ते॑ ते प्रा॒णः प्रा॒ण स्त॒ आ ते᳚ प्रा॒णः प्रा॒ण स्त॒ आ । \newline
6. प्रा॒ण इति॑ प्र - अ॒नः । \newline
7. त॒ आ ते॑ त॒ आ प्या॑यताम् प्यायता॒ मा ते॑ त॒ आ प्या॑यताम् । \newline
8. आ प्या॑यताम् प्यायता॒ मा प्या॑यता॒म् चक्षु॒ श्चक्षुः॑ प्यायता॒ मा प्या॑यता॒म् चक्षुः॑ । \newline
9. प्या॒य॒ता॒म् चक्षु॒ श्चक्षुः॑ प्यायताम् प्यायता॒म् चक्षु॑ स्ते ते॒ चक्षुः॑ प्यायताम् प्यायता॒म् चक्षु॑ स्ते । \newline
10. चक्षु॑ स्ते ते॒ चक्षु॒ श्चक्षु॑ स्त॒ आ ते॒ चक्षु॒ श्चक्षु॑ स्त॒ आ । \newline
11. त॒ आ ते॑ त॒ आ प्या॑यताम् प्यायता॒ मा ते॑ त॒ आ प्या॑यताम् । \newline
12. आ प्या॑यताम् प्यायता॒ मा प्या॑यता॒(ग्ग्॒) श्रोत्र॒(ग्ग्॒) श्रोत्र॑म् प्यायता॒ मा प्या॑यता॒(ग्ग्॒) श्रोत्र᳚म् । \newline
13. प्या॒य॒ता॒(ग्ग्॒) श्रोत्र॒(ग्ग्॒) श्रोत्र॑म् प्यायताम् प्यायता॒(ग्ग्॒) श्रोत्र॑म् ते ते॒ श्रोत्र॑म् प्यायताम् प्यायता॒(ग्ग्॒) श्रोत्र॑म् ते । \newline
14. श्रोत्र॑म् ते ते॒ श्रोत्र॒(ग्ग्॒) श्रोत्र॑म् त॒ आ ते॒ श्रोत्र॒(ग्ग्॒) श्रोत्र॑म् त॒ आ । \newline
15. त॒ आ ते॑ त॒ आ प्या॑यताम् प्यायता॒ मा ते॑ त॒ आ प्या॑यताम् । \newline
16. आ प्या॑यताम् प्यायता॒ मा प्या॑यतां॒ ॅया या प्या॑यता॒ मा प्या॑यतां॒ ॅया । \newline
17. प्या॒य॒तां॒ ॅया या प्या॑यताम् प्यायतां॒ ॅया ते॑ ते॒ या प्या॑यताम् प्यायतां॒ ॅया ते᳚ । \newline
18. या ते॑ ते॒ या या ते᳚ प्रा॒णान् प्रा॒णाꣳ स्ते॒ या या ते᳚ प्रा॒णान् । \newline
19. ते॒ प्रा॒णान् प्रा॒णाꣳ स्ते॑ ते प्रा॒णाञ् छुक् छुक् प्रा॒णाꣳ स्ते॑ ते प्रा॒णाञ् छुक् । \newline
20. प्रा॒णाञ् छुक् छुक् प्रा॒णान् प्रा॒णाञ् छुग् ज॒गाम॑ ज॒गाम॒ शुक् प्रा॒णान् प्रा॒णाञ् छुग् ज॒गाम॑ । \newline
21. प्रा॒णानिति॑ प्र - अ॒नान् । \newline
22. शुग् ज॒गाम॑ ज॒गाम॒ शुक् छुग् ज॒गाम॒ या या ज॒गाम॒ शुक् छुग् ज॒गाम॒ या । \newline
23. ज॒गाम॒ या या ज॒गाम॑ ज॒गाम॒ या चक्षु॒ श्चक्षु॒र् या ज॒गाम॑ ज॒गाम॒ या चक्षुः॑ । \newline
24. या चक्षु॒ श्चक्षु॒र् या या चक्षु॒र् या या चक्षु॒र् या या चक्षु॒र् या । \newline
25. चक्षु॒र् या या चक्षु॒ श्चक्षु॒र् या श्रोत्र॒(ग्ग्॒) श्रोत्रं॒ ॅया चक्षु॒ श्चक्षु॒र् या श्रोत्र᳚म् । \newline
26. या श्रोत्र॒(ग्ग्॒) श्रोत्रं॒ ॅया या श्रोत्रं॒ ॅयद् यच्छ्रोत्रं॒ ॅया या श्रोत्रं॒ ॅयत् । \newline
27. श्रोत्रं॒ ॅयद् यच्छ्रोत्र॒(ग्ग्॒) श्रोत्रं॒ ॅयत् ते॑ ते॒ यच्छ्रोत्र॒(ग्ग्॒) श्रोत्रं॒ ॅयत् ते᳚ । \newline
28. यत् ते॑ ते॒ यद् यत् ते᳚ क्रू॒रम् क्रू॒रम् ते॒ यद् यत् ते᳚ क्रू॒रम् । \newline
29. ते॒ क्रू॒रम् क्रू॒रम् ते॑ ते क्रू॒रं ॅयद् यत् क्रू॒रम् ते॑ ते क्रू॒रं ॅयत् । \newline
30. क्रू॒रं ॅयद् यत् क्रू॒रम् क्रू॒रं ॅयदास्थि॑त॒ मास्थि॑तं॒ ॅयत् क्रू॒रम् क्रू॒रं ॅयदास्थि॑तम् । \newline
31. यदास्थि॑त॒ मास्थि॑तं॒ ॅयद् यदास्थि॑त॒म् तत् तदास्थि॑तं॒ ॅयद् यदास्थि॑त॒म् तत् । \newline
32. आस्थि॑त॒म् तत् तदास्थि॑त॒ मास्थि॑त॒म् तत् ते॑ ते॒ तदास्थि॑त॒ मास्थि॑त॒म् तत् ते᳚ । \newline
33. आस्थि॑त॒मित्या - स्थि॒त॒म् । \newline
34. तत् ते॑ ते॒ तत् तत् त॒ आ ते॒ तत् तत् त॒ आ । \newline
35. त॒ आ ते॑ त॒ आ प्या॑यताम् प्यायता॒ मा ते॑ त॒ आ प्या॑यताम् । \newline
36. आ प्या॑यताम् प्यायता॒ मा प्या॑यता॒म् तत् तत् प्या॑यता॒ मा प्या॑यता॒म् तत् । \newline
37. प्या॒य॒ता॒म् तत् तत् प्या॑यताम् प्यायता॒म् तत् ते॑ ते॒ तत् प्या॑यताम् प्यायता॒म् तत् ते᳚ । \newline
38. तत् ते॑ ते॒ तत् तत् त॑ ए॒ते नै॒तेन॑ ते॒ तत् तत् त॑ ए॒तेन॑ । \newline
39. त॒ ए॒तेनै॒तेन॑ ते त ए॒तेन॑ शुन्धताꣳ शुन्धता मे॒तेन॑ ते त ए॒तेन॑ शुन्धताम् । \newline
40. ए॒तेन॑ शुन्धताꣳ शुन्धता मे॒तेनै॒तेन॑ शुन्धता॒म् नाभि॒र् नाभिः॑ शुन्धता मे॒तेनै॒तेन॑ शुन्धता॒म् नाभिः॑ । \newline
41. शु॒न्ध॒ता॒म् नाभि॒र् नाभिः॑ शुन्धताꣳ शुन्धता॒म् नाभि॑ स्ते ते॒ नाभिः॑ शुन्धताꣳ शुन्धता॒म् नाभि॑ स्ते । \newline
42. नाभि॑ स्ते ते॒ नाभि॒र् नाभि॑ स्त॒ आ ते॒ नाभि॒र् नाभि॑ स्त॒ आ । \newline
43. त॒ आ ते॑ त॒ आ प्या॑यताम् प्यायता॒ मा ते॑ त॒ आ प्या॑यताम् । \newline
44. आ प्या॑यताम् प्यायता॒ मा प्या॑यताम् पा॒युः पा॒युः प्या॑यता॒ मा प्या॑यताम् पा॒युः । \newline
45. प्या॒य॒ता॒म् पा॒युः पा॒युः प्या॑यताम् प्यायताम् पा॒यु स्ते॑ ते पा॒युः प्या॑यताम् प्यायताम् पा॒यु स्ते᳚ । \newline
46. पा॒यु स्ते॑ ते पा॒युः पा॒यु स्त॒ आ ते॑ पा॒युः पा॒यु स्त॒ आ । \newline
47. त॒ आ ते॑ त॒ आ प्या॑यताम् प्यायता॒ मा ते॑ त॒ आ प्या॑यताम् । \newline
48. आ प्या॑यताम् प्यायता॒ मा प्या॑यताꣳ शु॒द्धाः शु॒द्धाः प्या॑यता॒ मा प्या॑यताꣳ शु॒द्धाः । \newline
49. प्या॒य॒ता॒(ग्म्॒) शु॒द्धाः शु॒द्धाः प्या॑यताम् प्यायताꣳ शु॒द्धा श्च॒रित्रा᳚ श्च॒रित्राः᳚ शु॒द्धाः प्या॑यताम् प्यायताꣳ शु॒द्धा श्च॒रित्राः᳚ । \newline
50. शु॒द्धा श्च॒रित्रा᳚ श्च॒रित्राः᳚ शु॒द्धाः शु॒द्धा श्च॒रित्राः॒ शꣳ शम् च॒रित्राः᳚ शु॒द्धाः शु॒द्धा श्च॒रित्राः॒ शम् । \newline
51. च॒रित्राः॒ शꣳ शम् च॒रित्रा᳚ श्च॒रित्राः॒ श म॒द्भ्यो᳚ ऽद्भ्यः शम् च॒रित्रा᳚ श्च॒रित्राः॒ श म॒द्भ्यः । \newline
52. श म॒द्भ्यो᳚ ऽद्भ्यः शꣳ श म॒द्भ्यः शꣳ श म॒द्भ्यः शꣳ श म॒द्भ्यः शम् । \newline
53. अ॒द्भ्यः शꣳ श म॒द्भ्यो᳚ ऽद्भ्यः श मोष॑धीभ्य॒ ओष॑धीभ्यः॒ श म॒द्भ्यो᳚ ऽद्भ्यः श मोष॑धीभ्यः । \newline
54. अ॒द्भ्य इत्य॑त् - भ्यः । \newline
\pagebreak
\markright{ TS 1.3.9.2  \hfill https://www.vedavms.in \hfill}

\section{ TS 1.3.9.2 }

\textbf{TS 1.3.9.2 } \newline
\textbf{Samhita Paata} \newline

शमोष॑धीभ्यः॒ शं पृ॑थि॒व्यै शमहो᳚भ्या॒-मोष॑धे॒ त्राय॑स्वैनꣳ॒॒ स्वधि॑ते॒ मैनꣳ॑ हिꣳसी॒ रक्ष॑सां भा॒गो॑ऽसी॒दम॒हꣳ रक्षो॑ऽध॒मं तमो॑ नयामि॒ यो᳚ऽस्मान् द्वेष्टि॒ यं च॑ व॒यं द्वि॒ष्म इ॒दमे॑नमध॒मं तमो॑ नयामी॒षे त्वा॑ घृ॒तेन॑ द्यावापृथिवी॒ प्रोर्ण्वा॑था॒-मच्छि॑न्नो॒ रायः॑ सु॒वीर॑ उ॒र्व॑न्तरि॑क्ष॒मन्वि॑हि॒ वायो॒ वीहि॑ ( ) स्तो॒कानाꣳ॒॒ स्वाहो॒र्द्ध्वन॑भसं मारु॒तं ग॑च्छतं ॥ \newline

\textbf{Pada Paata} \newline

शम् । ओष॑धीभ्य॒ इत्योषा॑ध - भ्यः॒ । शम् । पृ॒थि॒व्यै । शम् । अहो᳚भ्या॒मित्यहः॑ - भ्या॒म् । ओष॑धे । त्राय॑स्व । ए॒न॒म् । स्वधि॑त॒ इति॒ स्व - धि॒ते॒ । मा । ए॒न॒म् । हिꣳ॒॒सीः॒ । रक्ष॑साम् । भा॒गः । अ॒सि॒ । इ॒दम् । अ॒हम् । रक्षः॑ । अ॒ध॒मम् । तमः॑ । न॒या॒मि॒ । यः । अ॒स्मान् । द्वेष्टि॑ । यम् । च॒ । व॒यम् । द्वि॒ष्मः । इ॒दम् । ए॒न॒म् । अ॒ध॒मम् । तमः॑ । न॒या॒मि॒ । इ॒षे । त्वा॒ । घृ॒तेन॑ । द्या॒वा॒पृ॒थि॒वी॒ इति॑ द्यावा - पृ॒थि॒वी॒ । प्रेति॑ । ऊ॒र्ण्वा॒था॒म् । अच्छि॑न्नः । रायः॑ । सु॒वीर॒ इति॑ सु - वीरः॑ । उ॒रु । अ॒न्तरि॑क्षम् । अन्विति॑ । इ॒हि॒ । वायो॒ इति॑ । वीति॑ । इ॒हि॒ ( ) । स्तो॒काना᳚म् । स्वाहा᳚ । ऊ॒र्द्ध्वन॑भस॒मित्यू॒र्द्ध्व - न॒भ॒स॒म् । मा॒रु॒तम् । ग॒च्छ॒त॒म् ॥  \newline


\textbf{Krama Paata} \newline

शमोष॑धीभ्यः । ओष॑धीभ्यः॒ शम् । ओष॑धीभ्य॒ इत्योष॑धि - भ्यः॒ । शम् पृ॑थि॒व्यै । पृ॒थि॒व्यै शम् । शमहो᳚भ्याम् । अहो᳚भ्या॒मोष॑धे । अहो᳚भ्या॒मित्य॑हः - भ्या॒म् । ओष॑धे॒ त्राय॑स्व । त्राय॑स्वैनम् । ए॒नꣳ॒॒ स्वधि॑ते । स्वधि॑ते॒ मा । स्वधि॑त॒ इति॒ स्व - धि॒ते॒ । मैन᳚म् । ए॒नꣳ॒॒ हिꣳ॒॒सीः॒ । हिꣳ॒॒सी॒ रक्ष॑साम् । रक्ष॑साम् भा॒गः । भा॒गो॑ऽसि । अ॒सी॒दम् । इ॒दम॒हम् । अ॒हꣳ रक्षः॑ । रक्षो॑ऽध॒मम् । अ॒ध॒मम् तमः॑ । तमो॑ नयामि । न॒या॒मि॒ यः । यो᳚ऽस्मान् । अ॒स्मान् द्वेष्टि॑ । द्वेष्टि॒ यम् । यम् च॑ । च॒ व॒यम् । व॒यम् द्वि॒ष्मः । द्वि॒ष्म इ॒दम् । इ॒दमे॑नम् । ए॒न॒म॒ध॒मम् । अ॒ध॒मम् तमः॑ । तमो॑ नयामि । न॒या॒मी॒षे । इ॒षे त्वा᳚ । त्वा॒ घृ॒तेन॑ । घृ॒तेन॑ द्यावापृथिवी । द्या॒वा॒पृ॒थि॒वी॒ प्र । द्या॒वा॒पृ॒थि॒वी॒ इति॑ द्यावा - पृ॒थि॒वी॒ । प्रोर्ण्वा॑थाम् । ऊ॒र्ण्वा॒था॒मच्छि॑न्नः । अच्छि॑न्नो॒ रायः॑ । रायः॑ सु॒वीरः॑ । सु॒वीर॑ उ॒रु । सु॒वीर॒ इति॑ सु - वीरः॑ । उ॒र्व॑न्तरि॑क्षम् । अ॒न्तरि॑क्ष॒मनु॑ । अन्वि॑हि । इ॒हि॒ वायो᳚ । वायो॒ वि । वायो॒ इति॒ वायो᳚ । वीहि॑ ( )  । इ॒हि॒ स्तो॒काना᳚म् । स्तो॒कानाꣳ॒॒ स्वाहा᳚ । स्वाहो॒र्द्ध्वन॑भसम् । ऊ॒र्द्ध्वन॑भसम् मारु॒तम् । ऊ॒र्द्ध्वन॑भस॒मित्यू॒र्द्ध्व - न॒भ॒स॒म् । मा॒रु॒तम् ग॑च्छतम् । ग॒च्छ॒त॒मिति॑ गच्छतम् । \newline

\textbf{Jatai Paata} \newline

1. श मोष॑धीभ्य॒ ओष॑धीभ्यः॒ शꣳ श मोष॑धीभ्यः । \newline
2. ओष॑धीभ्यः॒ शꣳ श मोष॑धीभ्य॒ ओष॑धीभ्यः॒ शम् । \newline
3. ओष॑धीभ्य॒ इत्योष॑धि - भ्यः॒ । \newline
4. शम् पृ॑थि॒व्यै पृ॑थि॒व्यै शꣳ शम् पृ॑थि॒व्यै । \newline
5. पृ॒थि॒व्यै शꣳ शम् पृ॑थि॒व्यै पृ॑थि॒व्यै शम् । \newline
6. श महो᳚भ्या॒ महो᳚भ्या॒(ग्म्॒) शꣳ श महो᳚भ्याम् । \newline
7. अहो᳚भ्या॒ मोष॑ध॒ ओष॒धे ऽहो᳚भ्या॒ महो᳚भ्या॒ मोष॑धे । \newline
8. अहो᳚भ्या॒मित्यहः॑ - भ्या॒म् । \newline
9. ओष॑धे॒ त्राय॑स्व॒ त्राय॒ स्वौष॑ध॒ ओष॑धे॒ त्राय॑स्व । \newline
10. त्राय॑ स्वैन मेन॒म् त्राय॑स्व॒ त्राय॑ स्वैनम् । \newline
11. ए॒न॒(ग्ग्॒) स्वधि॑ते॒ स्वधि॑त एन मेन॒(ग्ग्॒) स्वधि॑ते । \newline
12. स्वधि॑ते॒ मा मा स्वधि॑ते॒ स्वधि॑ते॒ मा । \newline
13. स्वधि॑त॒ इति॒ स्व - धि॒ते॒ । \newline
14. मैन॑ मेन॒म् मा मैन᳚म् । \newline
15. ए॒न॒(ग्म्॒) हि॒(ग्म्॒)सी॒र्॒. हि॒(ग्म्॒)सी॒ रे॒न॒ मे॒न॒(ग्म्॒) हि॒(ग्म्॒)सीः॒ । \newline
16. हि॒(ग्म्॒)सी॒ रक्ष॑सा॒(ग्म्॒) रक्ष॑साꣳ हिꣳसीर्. हिꣳसी॒ रक्ष॑साम् । \newline
17. रक्ष॑साम् भा॒गो भा॒गो रक्ष॑सा॒(ग्म्॒) रक्ष॑साम् भा॒गः । \newline
18. भा॒गो᳚ ऽस्यसि भा॒गो भा॒गो॑ ऽसि । \newline
19. अ॒सी॒द मि॒द म॑स्यसी॒दम् । \newline
20. इ॒द म॒ह म॒ह मि॒द मि॒द म॒हम् । \newline
21. अ॒हꣳ रक्षो॒ रक्षो॒ ऽह म॒हꣳ रक्षः॑ । \newline
22. रक्षो॑ ऽध॒म म॑ध॒मꣳ रक्षो॒ रक्षो॑ ऽध॒मम् । \newline
23. अ॒ध॒मम् तम॒ स्तमो॑ ऽध॒म म॑ध॒मम् तमः॑ । \newline
24. तमो॑ नयामि नयामि॒ तम॒ स्तमो॑ नयामि । \newline
25. न॒या॒मि॒ यो यो न॑यामि नयामि॒ यः । \newline
26. यो᳚ ऽस्मा न॒स्मान्. यो यो᳚ ऽस्मान् । \newline
27. अ॒स्मान् द्वेष्टि॒ द्वे ष्ट्य॒स्मा न॒स्मान् द्वेष्टि॑ । \newline
28. द्वेष्टि॒ यं ॅयम् द्वेष्टि॒ द्वेष्टि॒ यम् । \newline
29. यम् च॑ च॒ यं ॅयम् च॑ । \newline
30. च॒ व॒यं ॅव॒यम् च॑ च व॒यम् । \newline
31. व॒यम् द्वि॒ष्मो द्वि॒ष्मो व॒यं ॅव॒यम् द्वि॒ष्मः । \newline
32. द्वि॒ष्म इ॒द मि॒दम् द्वि॒ष्मो द्वि॒ष्म इ॒दम् । \newline
33. इ॒द मे॑न मेन मि॒द मि॒द मे॑नम् । \newline
34. ए॒न॒ म॒ध॒म म॑ध॒म मे॑न मेन मध॒मम् । \newline
35. अ॒ध॒मम् तम॒ स्तमो॑ ऽध॒म म॑ध॒मम् तमः॑ । \newline
36. तमो॑ नयामि नयामि॒ तम॒ स्तमो॑ नयामि । \newline
37. न॒या॒ मी॒ष इ॒षे न॑यामि नया मी॒षे । \newline
38. इ॒षे त्वा᳚ त्वे॒ष इ॒षे त्वा᳚ । \newline
39. त्वा॒ घृ॒तेन॑ घृ॒तेन॑ त्वा त्वा घृ॒तेन॑ । \newline
40. घृ॒तेन॑ द्यावापृथिवी द्यावापृथिवी घृ॒तेन॑ घृ॒तेन॑ द्यावापृथिवी । \newline
41. द्या॒वा॒पृ॒थि॒वी॒ प्र प्र द्या॑वापृथिवी द्यावापृथिवी॒ प्र । \newline
42. द्या॒वा॒पृ॒थि॒वी॒ इति॑ द्यावा - पृ॒थि॒वी॒ । \newline
43. प्रोर्ण्वा॑था मूर्ण्वाथा॒म् प्र प्रोर्ण्वा॑थाम् । \newline
44. ऊ॒र्ण्वा॒था॒ मच्छि॒न्नो ऽच्छि॑न्न ऊर्ण्वाथा मूर्ण्वाथा॒ मच्छि॑न्नः । \newline
45. अच्छि॑न्नो॒ रायो॒ रायो ऽच्छि॒न्नो ऽच्छि॑न्नो॒ रायः॑ । \newline
46. रायः॑ सु॒वीरः॑ सु॒वीरो॒ रायो॒ रायः॑ सु॒वीरः॑ । \newline
47. सु॒वीर॑ उ॒रू॑रु सु॒वीरः॑ सु॒वीर॑ उ॒रु । \newline
48. सु॒वीर॒ इति॑ सु - वीरः॑ । \newline
49. उ॒र्व॑न्तरि॑क्ष म॒न्तरि॑क्ष मु॒रू᳚(1॒)र्व॑न्तरि॑क्षम् । \newline
50. अ॒न्तरि॑क्ष॒ मन्व न्व॒न्तरि॑क्ष म॒न्तरि॑क्ष॒ मनु॑ । \newline
51. अन्वि॑ ही॒ह्य न्वन्वि॑ हि । \newline
52. इ॒हि॒ वायो॒ वायो॑ इहीहि॒ वायो᳚ । \newline
53. वायो॒ वि वि वायो॒ वायो॒ वि । \newline
54. वायो॒ इति॒ वायो᳚ । \newline
55. वीही॑हि॒ वि वीहि॑ । \newline
56. इ॒हि॒ स्तो॒काना(ग्ग्॑) स्तो॒काना॑ मिहीहि स्तो॒काना᳚म् । \newline
57. स्तो॒काना॒(ग्ग्॒) स्वाहा॒ स्वाहा᳚ स्तो॒काना(ग्ग्॑) स्तो॒काना॒(ग्ग्॒) स्वाहा᳚ । \newline
58. स्वा हो॒र्द्ध्वन॑भस मू॒र्द्ध्वन॑भस॒(ग्ग्॒) स्वाहा॒ स्वा हो॒र्द्ध्वन॑भसम् । \newline
59. ऊ॒र्द्ध्वन॑भसम् मारु॒तम् मा॑रु॒त मू॒र्द्ध्वन॑भस मू॒र्द्ध्वन॑भसम् मारु॒तम् । \newline
60. ऊ॒र्द्ध्वन॑भस॒मित्यु॒र्द्ध्व - न॒भ॒स॒म् । \newline
61. मा॒रु॒तम् ग॑च्छतम् गच्छतम् मारु॒तम् मा॑रु॒तम् ग॑च्छतम् । \newline
62. ग॒च्छ॒त॒मिति॑ गच्छतम् । \newline

\textbf{Ghana Paata } \newline

1. श मोष॑धीभ्य॒ ओष॑धीभ्यः॒ शꣳ श मोष॑धीभ्यः॒ शꣳ श मोष॑धीभ्यः॒ शꣳ श मोष॑धीभ्यः॒ शम् । \newline
2. ओष॑धीभ्यः॒ शꣳ श मोष॑धीभ्य॒ ओष॑धीभ्यः॒ शम् पृ॑थि॒व्यै पृ॑थि॒व्यै श मोष॑धीभ्य॒ ओष॑धीभ्यः॒ शम् पृ॑थि॒व्यै । \newline
3. ओष॑धीभ्य॒ इत्योष॑धि - भ्यः॒ । \newline
4. शम् पृ॑थि॒व्यै पृ॑थि॒व्यै शꣳ शम् पृ॑थि॒व्यै शꣳ शम् पृ॑थि॒व्यै शꣳ शम् पृ॑थि॒व्यै शम् । \newline
5. पृ॒थि॒व्यै शꣳ शम् पृ॑थि॒व्यै पृ॑थि॒व्यै श महो᳚भ्या॒ महो᳚भ्या॒(ग्म्॒) शम् पृ॑थि॒व्यै पृ॑थि॒व्यै श महो᳚भ्याम् । \newline
6. श महो᳚भ्या॒ महो᳚भ्या॒(ग्म्॒) शꣳ श महो᳚भ्या॒ मोष॑ध॒ ओष॒धे ऽहो᳚भ्या॒(ग्म्॒) शꣳ श महो᳚भ्या॒ मोष॑धे । \newline
7. अहो᳚भ्या॒ मोष॑ध॒ ओष॒धे ऽहो᳚भ्या॒ महो᳚भ्या॒ मोष॑धे॒ त्राय॑स्व॒ त्राय॒ स्वौष॒धे ऽहो᳚भ्या॒ महो᳚भ्या॒ मोष॑धे॒ त्राय॑स्व । \newline
8. अहो᳚भ्या॒मित्यहः॑ - भ्या॒म् । \newline
9. ओष॑धे॒ त्राय॑स्व॒ त्राय॒ स्वौष॑ध॒ ओष॑धे॒ त्राय॑स्वैन मेन॒म् त्राय॒ स्वौष॑ध॒ ओष॑धे॒ त्राय॑स्वैनम् । \newline
10. त्राय॑स्वैन मेन॒म् त्राय॑स्व॒ त्राय॑स्वैन॒(ग्ग्॒) स्वधि॑ते॒ स्वधि॑त एन॒म् त्राय॑स्व॒ त्राय॑स्वैन॒(ग्ग्॒) स्वधि॑ते । \newline
11. ए॒न॒(ग्ग्॒) स्वधि॑ते॒ स्वधि॑त एन मेन॒(ग्ग्॒) स्वधि॑ते॒ मा मा स्वधि॑त एन मेन॒(ग्ग्॒) स्वधि॑ते॒ मा । \newline
12. स्वधि॑ते॒ मा मा स्वधि॑ते॒ स्वधि॑ते॒ मैन॑ मेन॒म् मा स्वधि॑ते॒ स्वधि॑ते॒ मैन᳚म् । \newline
13. स्वधि॑त॒ इति॒ स्व - धि॒ते॒ । \newline
14. मैन॑ मेन॒म् मा मैन(ग्म्॑) हिꣳसीर्. हिꣳसी रेन॒म् मा मैन(ग्म्॑) हिꣳसीः । \newline
15. ए॒न॒(ग्म्॒) हि॒(ग्म्॒)सी॒र्॒. हि॒(ग्म्॒)सी॒ रे॒न॒ मे॒न॒(ग्म्॒) हि॒(ग्म्॒)सी॒ रक्ष॑सा॒(ग्म्॒) रक्ष॑साꣳ हिꣳसी रेन मेनꣳ हिꣳसी॒ रक्ष॑साम् । \newline
16. हि॒(ग्म्॒)सी॒ रक्ष॑सा॒(ग्म्॒) रक्ष॑साꣳ हिꣳसीर्. हिꣳसी॒ रक्ष॑साम् भा॒गो भा॒गो रक्ष॑साꣳ हिꣳसीर्. हिꣳसी॒ रक्ष॑साम् भा॒गः । \newline
17. रक्ष॑साम् भा॒गो भा॒गो रक्ष॑सा॒(ग्म्॒) रक्ष॑साम् भा॒गो᳚ ऽस्यसि भा॒गो रक्ष॑सा॒(ग्म्॒) रक्ष॑साम् भा॒गो॑ ऽसि । \newline
18. भा॒गो᳚ ऽस्यसि भा॒गो भा॒गो॑ ऽसी॒द मि॒द म॑सि भा॒गो भा॒गो॑ ऽसी॒दम् । \newline
19. अ॒सी॒द मि॒द म॑स्यसी॒द म॒ह म॒ह मि॒द म॑स्यसी॒द म॒हम् । \newline
20. इ॒द म॒ह म॒ह मि॒द मि॒द म॒हꣳ रक्षो॒ रक्षो॒ ऽह मि॒द मि॒द म॒हꣳ रक्षः॑ । \newline
21. अ॒हꣳ रक्षो॒ रक्षो॒ ऽह म॒हꣳ रक्षो॑ ऽध॒म म॑ध॒मꣳ रक्षो॒ ऽह म॒हꣳ रक्षो॑ ऽध॒मम् । \newline
22. रक्षो॑ ऽध॒म म॑ध॒मꣳ रक्षो॒ रक्षो॑ ऽध॒मम् तम॒स्तमो॑ ऽध॒मꣳ रक्षो॒ रक्षो॑ ऽध॒मम् तमः॑ । \newline
23. अ॒ध॒मम् तम॒स्तमो॑ ऽध॒म म॑ध॒मम् तमो॑ नयामि नयामि॒ तमो॑ ऽध॒म म॑ध॒मम् तमो॑ नयामि । \newline
24. तमो॑ नयामि नयामि॒ तम॒ स्तमो॑ नयामि॒ यो यो न॑यामि॒ तम॒ स्तमो॑ नयामि॒ यः । \newline
25. न॒या॒मि॒ यो यो न॑यामि नयामि॒ यो᳚ ऽस्मा न॒स्मान्. यो न॑यामि नयामि॒ यो᳚ ऽस्मान् । \newline
26. यो᳚ ऽस्मा न॒स्मान्. यो यो᳚ ऽस्मान् द्वेष्टि॒ द्वे ष्ट्य॒स्मान्. यो यो᳚ ऽस्मान् द्वेष्टि॑ । \newline
27. अ॒स्मान् द्वेष्टि॒ द्वे ष्ट्य॒स्मा न॒स्मान् द्वेष्टि॒ यं ॅयम् द्वे ष्ट्य॒स्मा न॒स्मान् द्वेष्टि॒ यम् । \newline
28. द्वेष्टि॒ यं ॅयम् द्वेष्टि॒ द्वेष्टि॒ यम् च॑ च॒ यम् द्वेष्टि॒ द्वेष्टि॒ यम् च॑ । \newline
29. यम् च॑ च॒ यं ॅयम् च॑ व॒यं ॅव॒यम् च॒ यं ॅयम् च॑ व॒यम् । \newline
30. च॒ व॒यं ॅव॒यम् च॑ च व॒यम् द्वि॒ष्मो द्वि॒ष्मो व॒यम् च॑ च व॒यम् द्वि॒ष्मः । \newline
31. व॒यम् द्वि॒ष्मो द्वि॒ष्मो व॒यं ॅव॒यम् द्वि॒ष्म इ॒द मि॒दम् द्वि॒ष्मो व॒यं ॅव॒यम् द्वि॒ष्म इ॒दम् । \newline
32. द्वि॒ष्म इ॒द मि॒दम् द्वि॒ष्मो द्वि॒ष्म इ॒द मे॑न मेन मि॒दम् द्वि॒ष्मो द्वि॒ष्म इ॒द मे॑नम् । \newline
33. इ॒द मे॑न मेन मि॒द मि॒द मे॑न मध॒म म॑ध॒म मे॑न मि॒द मि॒द मे॑न मध॒मम् । \newline
34. ए॒न॒ म॒ध॒म म॑ध॒म मे॑न मेन मध॒मम् तम॒ स्तमो॑ ऽध॒म मे॑न मेन मध॒मम् तमः॑ । \newline
35. अ॒ध॒मम् तम॒ स्तमो॑ ऽध॒म म॑ध॒मम् तमो॑ नयामि नयामि॒ तमो॑ ऽध॒म म॑ध॒मम् तमो॑ नयामि । \newline
36. तमो॑ नयामि नयामि॒ तम॒ स्तमो॑ नयामी॒ष इ॒षे न॑यामि॒ तम॒ स्तमो॑ नयामी॒षे । \newline
37. न॒या॒मी॒ष इ॒षे न॑यामि नयामी॒षे त्वा᳚ त्वे॒षे न॑यामि नयामी॒षे त्वा᳚ । \newline
38. इ॒षे त्वा᳚ त्वे॒ष इ॒षे त्वा॑ घृ॒तेन॑ घृ॒तेन॑ त्वे॒ष इ॒षे त्वा॑ घृ॒तेन॑ । \newline
39. त्वा॒ घृ॒तेन॑ घृ॒तेन॑ त्वा त्वा घृ॒तेन॑ द्यावापृथिवी द्यावापृथिवी घृ॒तेन॑ त्वा त्वा घृ॒तेन॑ द्यावापृथिवी । \newline
40. घृ॒तेन॑ द्यावापृथिवी द्यावापृथिवी घृ॒तेन॑ घृ॒तेन॑ द्यावापृथिवी॒ प्र प्र द्या॑वापृथिवी घृ॒तेन॑ घृ॒तेन॑ द्यावापृथिवी॒ प्र । \newline
41. द्या॒वा॒पृ॒थि॒वी॒ प्र प्र द्या॑वापृथिवी द्यावापृथिवी॒ प्रोर्ण्वा॑था मूर्ण्वाथा॒म् प्र द्या॑वापृथिवी द्यावापृथिवी॒ प्रोर्ण्वा॑थाम् । \newline
42. द्या॒वा॒पृ॒थि॒वी॒ इति॑ द्यावा - पृ॒थि॒वी॒ । \newline
43. प्रोर्ण्वा॑था मूर्ण्वाथा॒म् प्र प्रोर्ण्वा॑था॒ मच्छि॒म् नो ऽच्छि॑म् न ऊर्ण्वाथा॒म् प्र प्रोर्ण्वा॑था॒ मच्छि॑म् नः । \newline
44. ऊ॒र्ण्वा॒था॒ मच्छि॒न्नो ऽच्छि॑न्न ऊर्ण्वाथा मूर्ण्वाथा॒ मच्छि॑न्नो॒ रायो॒ रायो ऽच्छि॑न्न ऊर्ण्वाथा मूर्ण्वाथा॒ मच्छि॑न्नो॒ रायः॑ । \newline
45. अच्छि॑न्नो॒ रायो॒ रायो ऽच्छि॒न्नो ऽच्छि॑न्नो॒ रायः॑ सु॒वीरः॑ सु॒वीरो॒ रायो ऽच्छि॒न्नो ऽच्छि॑न्नो॒ रायः॑ सु॒वीरः॑ । \newline
46. रायः॑ सु॒वीरः॑ सु॒वीरो॒ रायो॒ रायः॑ सु॒वीर॑ उ॒रू॑ रु सु॒वीरो॒ रायो॒ रायः॑ सु॒वीर॑ उ॒रु । \newline
47. सु॒वीर॑ उ॒रू॑रु सु॒वीरः॑ सु॒वीर॑ उ॒र्व॑न्तरि॑क्ष म॒न्तरि॑क्ष मु॒रु सु॒वीरः॑ सु॒वीर॑ उ॒र्व॑न्तरि॑क्षम् । \newline
48. सु॒वीर॒ इति॑ सु - वीरः॑ । \newline
49. उ॒र्व॑न्तरि॑क्ष म॒न्तरि॑क्ष मु॒रू᳚(1॒)र्व॑न्तरि॑क्ष॒ मन्वन्व॒न्तरि॑क्ष मु॒रू᳚(1॒)र्व॑न्तरि॑क्ष॒ मनु॑ । \newline
50. अ॒न्तरि॑क्ष॒ मन्वन् व॒न्तरि॑क्ष म॒न्तरि॑क्ष॒ मन्वि॑ही॒ह्यन्व॒न्तरि॑क्ष म॒न्तरि॑क्ष॒ मन्वि॑हि । \newline
51. अन्वि॑ही॒ ह्यन्वन्वि॑हि॒ वायो॒ वायो॑ इ॒ह्यन्वन्वि॑हि॒ वायो᳚ । \newline
52. इ॒हि॒ वायो॒ वायो॑ इहीहि॒ वायो॒ वि वि वायो॑ इहीहि॒ वायो॒ वि । \newline
53. वायो॒ वि वि वायो॒ वायो॒ वीही॑हि॒ वि वायो॒ वायो॒ वीहि॑ । \newline
54. वायो॒ इति॒ वायो᳚ । \newline
55. वीही॑हि॒ वि वीहि॑ स्तो॒काना(ग्ग्॑) स्तो॒काना॑ मिहि॒ वि वीहि॑ स्तो॒काना᳚म् । \newline
56. इ॒हि॒ स्तो॒काना(ग्ग्॑) स्तो॒काना॑ मिहीहि स्तो॒काना॒(ग्ग्॒) स्वाहा॒ स्वाहा᳚ स्तो॒काना॑ मिहीहि स्तो॒काना॒(ग्ग्॒) स्वाहा᳚ । \newline
57. स्तो॒काना॒(ग्ग्॒) स्वाहा॒ स्वाहा᳚ स्तो॒काना(ग्ग्॑) स्तो॒काना॒(ग्ग्॒) स्वाहो॒र्द्ध्वन॑भस मू॒र्द्ध्वन॑भस॒(ग्ग्॒) स्वाहा᳚ स्तो॒काना(ग्ग्॑) स्तो॒काना॒(ग्ग्॒) स्वाहो॒र्द्ध्वन॑भसम् । \newline
58. स्वाहो॒र्द्ध्वन॑भस मू॒र्द्ध्वन॑भस॒(ग्ग्॒) स्वाहा॒ स्वाहो॒र्द्ध्वन॑भसम् मारु॒तम् मा॑रु॒त मू॒र्द्ध्वन॑भस॒(ग्ग्॒) स्वाहा॒ स्वाहो॒र्द्ध्वन॑भसम् मारु॒तम् । \newline
59. ऊ॒र्द्ध्वन॑भसम् मारु॒तम् मा॑रु॒त मू॒र्द्ध्वन॑भस मू॒र्द्ध्वन॑भसम् मारु॒तम् ग॑च्छतम् गच्छतम् मारु॒त मू॒र्द्ध्वन॑भस मू॒र्द्ध्वन॑भसम् मारु॒तम् ग॑च्छतम् । \newline
60. ऊ॒र्द्ध्वन॑भस॒मित्यु॒र्द्ध्व - न॒भ॒स॒म् । \newline
61. मा॒रु॒तम् ग॑च्छतम् गच्छतम् मारु॒तम् मा॑रु॒तम् ग॑च्छतम् । \newline
62. ग॒च्छ॒त॒मिति॑ गच्छतम् । \newline
\pagebreak
\markright{ TS 1.3.10.1  \hfill https://www.vedavms.in \hfill}

\section{ TS 1.3.10.1 }

\textbf{TS 1.3.10.1 } \newline
\textbf{Samhita Paata} \newline

सं ते॒ मन॑सा॒ मनः॒ सं प्रा॒णेन॑ प्रा॒णो जुष्टं॑ दे॒वेभ्यो॑ ह॒व्यं घृ॒तव॒थ् स्वाहै॒न्द्रः प्रा॒णो अङ्गे॑अङ्गे॒ नि दे᳚द्ध्यदै॒न्द्रो॑ ऽपा॒नो अङ्गे॑अङ्गे॒ वि बो॑भुव॒द्देव॑ त्वष्ट॒र्भूरि॑ ते॒ सꣳस॑मेतु॒ विषु॑रूपा॒ यथ् सल॑क्ष्माणो॒ भव॑थ देव॒त्रा यन्त॒मव॑से॒ सखा॒योऽनु॑ त्वा मा॒ता पि॒तरो॑ मदन्तु॒ श्रीर॑स्य॒ग्निस्त्वा᳚ श्रीणा॒त्वापः॒ सम॑रिण॒न् वात॑स्य-[ ] \newline

\textbf{Pada Paata} \newline

समिति॑ । ते॒ । मन॑सा । मनः॑ । समिति॑ । प्रा॒णेनेति॑ प्र - अ॒नेन॑ । प्रा॒ण इति॑ प्र - अ॒नः । जुष्ट᳚म् । दे॒वेभ्यः॑ । ह॒व्यम् । घृ॒तव॒दिति॑ घृ॒त - व॒त् । स्वाहा᳚ । ऐ॒न्द्रः । प्रा॒ण इति॑ प्र - अ॒नः । अङ्गे॑अङ्ग॒ इत्यङ्गे᳚ - अ॒ङ्गे॒ । नीति॑ । दे॒द्ध्य॒त् । ऐ॒न्द्रः । अ॒पा॒न इत्य॑प - अ॒नः । अङ्गे॑अङ्ग॒ इत्यङ्गे᳚ - अ॒ङ्गे॒ । वीति॑ । बो॒भु॒व॒त् । देव॑ । त्व॒ष्टः॒ । भूरि॑ । ते॒ । सꣳस॒मिति॒ सं - स॒म् । ए॒तु॒ । विषु॑रूपा॒ इति॒ विषु॑ - रू॒पाः॒ । यत् । सल॑क्ष्माण॒ इति॒ स - ल॒क्ष्मा॒णः॒ । भव॑थ । दे॒व॒त्रेति॑ देव - त्रा । यन्त᳚म् । अव॑से । सखा॑यः । अन्विति॑ । त्वा॒ । मा॒ता । पि॒तरः॑ । म॒द॒न्तु॒ । श्रीः । अ॒सी॒ । अ॒ग्निः । त्वा॒ । श्री॒णा॒तु॒ । आपः॑ । समिति॑ । अ॒रि॒ण॒न्न् । वात॑स्य ।  \newline


\textbf{Krama Paata} \newline

सन्ते᳚ । ते॒ मन॑सा । मन॑सा॒ मनः॑ । मनः॒ सम् । सं प्रा॒णेन॑ । प्रा॒णेन॑ प्रा॒णः । प्रा॒णेनेति॑ प्र - अ॒नेन॑ । प्रा॒णो जुष्ट᳚म् । प्रा॒ण इति॑ प्र - अ॒नः । जुष्टं॑ दे॒वेभ्यः॑ । दे॒वेभ्यो॑ ह॒व्यम् । ह॒व्यम् घृ॒तव॑त् । घृ॒तव॒थ् स्वाहा᳚ । घृ॒तव॒दिति॑ घृ॒त - व॒त् । स्वाहै॒न्द्रः । ऐ॒न्द्रः प्रा॒णः । प्रा॒णो अङ्गे॑अङ्गे । प्रा॒ण इति॑ प्र - अ॒नः । अङ्गे॑अङ्गे॒ नि । अङ्गे॑अङ्ग॒ इत्यङ्गे᳚ - अ॒ङ्गे॒ । नि दे᳚द्ध्यत् । दे॒द्ध्य॒दै॒न्द्रः । ऐ॒न्द्रो॑ऽपा॒नः । अ॒पा॒नो अङ्गे॑अङ्गे । अ॒पा॒न इत्य॑प - अ॒नः । अङ्गे॑अङ्गे॒ वि । अङ्गे॑अङ्ग॒ इत्यङ्गे᳚ - अ॒ङ्गे॒ । वि बो॑भुवत् । बो॒भु॒व॒द् देव॑ । देव॑ त्वष्टः । त्व॒ष्ट॒र् भूरि॑ । भूरि॑ ते । ते॒ सꣳस᳚म् । सꣳस॑मेतु । सꣳस॒मिति॒ सं - स॒म् । ए॒तु॒ विषु॑रूपाः । विषु॑रूपा॒ यत् । विषु॑रूपा॒ इति॒ विषु॑ - रू॒पाः॒ । यथ् सल॑क्ष्माणः । सल॑क्ष्माणो॒ भव॑थ । सल॑क्ष्माण॒ इति॒ स - ल॒क्ष्मा॒णः॒ । भव॑थ देव॒त्रा । दे॒व॒त्रा यन्त᳚म् । दे॒व॒त्रेति॑ देव - त्रा । यन्त॒मव॑से । अव॑से॒ सखा॑यः । सखा॒योऽनु॑ । अनु॑ त्वा । त्वा॒ मा॒ता । मा॒ता पि॒तरः॑ । पि॒तरो॑ मदन्तु । म॒द॒न्तु॒ श्रीः । श्रीर॑सि । अ॒स्य॒ग्निः । अ॒ग्निस्त्वा᳚ । त्वा॒ श्री॒णा॒तु॒ । श्री॒णा॒त्वापः॑ । आपः॒ सम् । सम॑रिणन्न् । अ॒रि॒ण॒न् वात॑स्य ( ) । वात॑स्य त्वा \newline

\textbf{Jatai Paata} \newline

1. सम् ते॑ ते॒ सꣳ सम् ते᳚ । \newline
2. ते॒ मन॑सा॒ मन॑सा ते ते॒ मन॑सा । \newline
3. मन॑सा॒ मनो॒ मनो॒ मन॑सा॒ मन॑सा॒ मनः॑ । \newline
4. मनः॒ सꣳ सम् मनो॒ मनः॒ सम् । \newline
5. सम् प्रा॒णेन॑ प्रा॒णेन॒ सꣳ सम् प्रा॒णेन॑ । \newline
6. प्रा॒णेन॑ प्रा॒णः प्रा॒णः प्रा॒णेन॑ प्रा॒णेन॑ प्रा॒णः । \newline
7. प्रा॒णेनेति॑ प्र - अ॒नेन॑ । \newline
8. प्रा॒णो जुष्ट॒म् जुष्ट॑म् प्रा॒णः प्रा॒णो जुष्ट᳚म् । \newline
9. प्रा॒ण इति॑ प्र - अ॒नः । \newline
10. जुष्ट॑म् दे॒वेभ्यो॑ दे॒वेभ्यो॒ जुष्ट॒म् जुष्ट॑म् दे॒वेभ्यः॑ । \newline
11. दे॒वेभ्यो॑ ह॒व्यꣳ ह॒व्यम् दे॒वेभ्यो॑ दे॒वेभ्यो॑ ह॒व्यम् । \newline
12. ह॒व्यम् घृ॒तव॑द् घृ॒तव॑ द्ध॒व्यꣳ ह॒व्यम् घृ॒तव॑त् । \newline
13. घृ॒तव॒थ् स्वाहा॒ स्वाहा॑ घृ॒तव॑द् घृ॒तव॒थ् स्वाहा᳚ । \newline
14. घृ॒तव॒दिति॑ घृ॒त - व॒त् । \newline
15. स्वाहै॒न्द्र ऐ॒न्द्रः स्वाहा॒ स्वाहै॒न्द्रः । \newline
16. ऐ॒न्द्रः प्रा॒णः प्रा॒ण ऐ॒न्द्र ऐ॒न्द्रः प्रा॒णः । \newline
17. प्रा॒णो अङ्गे॑अङ्गे॒ अङ्गे॑अङ्गे प्रा॒णः प्रा॒णो अङ्गे॑अङ्गे । \newline
18. प्रा॒ण इति॑ प्र - अ॒नः । \newline
19. अङ्गे॑अङ्गे॒ नि न्यङ्गे॑अङ्गे॒ अङ्गे॑अङ्गे॒ नि । \newline
20. अङ्गे॑अङ्ग॒ इत्यङ्गे᳚ - अ॒ङ्गे॒ । \newline
21. नि दे᳚द्ध्यद् देद्ध्य॒न् नि नि दे᳚द्ध्यत् । \newline
22. दे॒द्ध्य॒दै॒न्द्र ऐ॒न्द्रो दे᳚द्ध्यद् देद्ध्यदै॒न्द्रः । \newline
23. ऐ॒न्द्रो॑ ऽपा॒नो॑ ऽपा॒न ऐ॒न्द्र ऐ॒न्द्रो॑ ऽपा॒नः । \newline
24. अ॒पा॒नो अङ्गे॑अङ्गे॒ अङ्गे॑अङ्गे ऽपा॒नो॑ ऽपा॒नो अङ्गे॑अङ्गे । \newline
25. अ॒पा॒न इत्य॑प - अ॒नः । \newline
26. अङ्गे॑अङ्गे॒ वि व्यङ्गे॑अङ्गे॒ अङ्गे॑अङ्गे॒ वि । \newline
27. अङ्गे॑अङ्ग॒ इत्यङ्गे᳚ - अ॒ङ्गे॒ । \newline
28. वि बो॑भुवद् बोभुव॒द् वि वि बो॑भुवत् । \newline
29. बो॒भु॒व॒द् देव॒ देव॑ बोभुवद् बोभुव॒द् देव॑ । \newline
30. देव॑ त्वष्टस् त्वष्ट॒र् देव॒ देव॑ त्वष्टः । \newline
31. त्व॒ष्ट॒र् भूरि॒ भूरि॑ त्वष्ट स्त्वष्ट॒र् भूरि॑ । \newline
32. भूरि॑ ते ते॒ भूरि॒ भूरि॑ ते । \newline
33. ते॒ सꣳस॒(ग्म्॒) सꣳस॑म् ते ते॒ सꣳस᳚म् । \newline
34. सꣳस॑ मेत्वेतु॒ सꣳस॒(ग्म्॒) सꣳस॑ मेतु । \newline
35. सꣳस॒मिति॒ सं - स॒म् । \newline
36. ए॒तु॒ विषु॑रूपा॒ विषु॑रूपा एत्वेतु॒ विषु॑रूपाः । \newline
37. विषु॑रूपा॒ यद् यद् विषु॑रूपा॒ विषु॑रूपा॒ यत् । \newline
38. विषु॑रूपा॒ इति॒ विषु॑ - रू॒पाः॒ । \newline
39. यथ् सल॑क्ष्माणः॒ सल॑क्ष्माणो॒ यद् यथ् सल॑क्ष्माणः । \newline
40. सल॑क्ष्माणो॒ भव॑थ॒ भव॑थ॒ सल॑क्ष्माणः॒ सल॑क्ष्माणो॒ भव॑थ । \newline
41. सल॑क्ष्माण॒ इति॒ स - ल॒क्ष्मा॒णः॒ । \newline
42. भव॑थ देव॒त्रा दे॑व॒त्रा भव॑थ॒ भव॑थ देव॒त्रा । \newline
43. दे॒व॒त्रा यन्तं॒ ॅयन्त॑म् देव॒त्रा दे॑व॒त्रा यन्त᳚म् । \newline
44. दे॒व॒त्रेति॑ देव - त्रा । \newline
45. यन्त॒ मव॒से ऽव॑से॒ यन्तं॒ ॅयन्त॒ मव॑से । \newline
46. अव॑से॒ सखा॑यः॒ सखा॒यो ऽव॒से ऽव॑से॒ सखा॑यः । \newline
47. सखा॒यो ऽन्वनु॒ सखा॑यः॒ सखा॒यो ऽनु॑ । \newline
48. अनु॑ त्वा॒ त्वा ऽन्वनु॑ त्वा । \newline
49. त्वा॒ मा॒ता मा॒ता त्वा᳚ त्वा मा॒ता । \newline
50. मा॒ता पि॒तरः॑ पि॒तरो॑ मा॒ता मा॒ता पि॒तरः॑ । \newline
51. पि॒तरो॑ मदन्तु मदन्तु पि॒तरः॑ पि॒तरो॑ मदन्तु । \newline
52. म॒द॒न्तु॒ श्रीः श्रीर् म॑दन्तु मदन्तु॒ श्रीः । \newline
53. श्रीर॑स्यसि॒ श्रीः श्रीर॑सि । \newline
54. अ॒स्य॒ग्नि र॒ग्नि र॑स्य स्य॒ग्निः । \newline
55. अ॒ग्नि स्त्वा᳚ त्वा॒ ऽग्नि र॒ग्नि स्त्वा᳚ । \newline
56. त्वा॒ श्री॒णा॒तु॒ श्री॒णा॒तु॒ त्वा॒ त्वा॒ श्री॒णा॒तु॒ । \newline
57. श्री॒णा॒त्वाप॒ आपः॑ श्रीणातु श्रीणा॒त्वापः॑ । \newline
58. आपः॒ सꣳ स माप॒ आपः॒ सम् । \newline
59. स म॑रिणन् नरिण॒न् थ्सꣳ स म॑रिणन्न् । \newline
60. अ॒रि॒ण॒न्॒. वात॑स्य॒ वात॑स्यारिणन् नरिण॒न्॒. वात॑स्य । \newline
61. वात॑स्य त्वा त्वा॒ वात॑स्य॒ वात॑स्य त्वा । \newline

\textbf{Ghana Paata } \newline

1. सम् ते॑ ते॒ सꣳ सम् ते॒ मन॑सा॒ मन॑सा ते॒ सꣳ सम् ते॒ मन॑सा । \newline
2. ते॒ मन॑सा॒ मन॑सा ते ते॒ मन॑सा॒ मनो॒ मनो॒ मन॑सा ते ते॒ मन॑सा॒ मनः॑ । \newline
3. मन॑सा॒ मनो॒ मनो॒ मन॑सा॒ मन॑सा॒ मनः॒ सꣳ सम् मनो॒ मन॑सा॒ मन॑सा॒ मनः॒ सम् । \newline
4. मनः॒ सꣳ सम् मनो॒ मनः॒ सम् प्रा॒णेन॑ प्रा॒णेन॒ सम् मनो॒ मनः॒ सम् प्रा॒णेन॑ । \newline
5. सम् प्रा॒णेन॑ प्रा॒णेन॒ सꣳ सम् प्रा॒णेन॑ प्रा॒णः प्रा॒णः प्रा॒णेन॒ सꣳ सम् प्रा॒णेन॑ प्रा॒णः । \newline
6. प्रा॒णेन॑ प्रा॒णः प्रा॒णः प्रा॒णेन॑ प्रा॒णेन॑ प्रा॒णो जुष्ट॒म् जुष्ट॑म् प्रा॒णः प्रा॒णेन॑ प्रा॒णेन॑ प्रा॒णो जुष्ट᳚म् । \newline
7. प्रा॒णेनेति॑ प्र - अ॒नेन॑ । \newline
8. प्रा॒णो जुष्ट॒म् जुष्ट॑म् प्रा॒णः प्रा॒णो जुष्ट॑म् दे॒वेभ्यो॑ दे॒वेभ्यो॒ जुष्ट॑म् प्रा॒णः प्रा॒णो जुष्ट॑म् दे॒वेभ्यः॑ । \newline
9. प्रा॒ण इति॑ प्र - अ॒नः । \newline
10. जुष्ट॑म् दे॒वेभ्यो॑ दे॒वेभ्यो॒ जुष्ट॒म् जुष्ट॑म् दे॒वेभ्यो॑ ह॒व्यꣳ ह॒व्यम् दे॒वेभ्यो॒ जुष्ट॒म् जुष्ट॑म् दे॒वेभ्यो॑ ह॒व्यम् । \newline
11. दे॒वेभ्यो॑ ह॒व्यꣳ ह॒व्यम् दे॒वेभ्यो॑ दे॒वेभ्यो॑ ह॒व्यम् घृ॒तव॑द् घृ॒तव॑द्ध॒व्यम् दे॒वेभ्यो॑ दे॒वेभ्यो॑ ह॒व्यम् घृ॒तव॑त् । \newline
12. ह॒व्यम् घृ॒तव॑द् घृ॒तव॑द्ध॒व्यꣳ ह॒व्यम् घृ॒तव॒थ् स्वाहा॒ स्वाहा॑ घृ॒तव॑द्ध॒व्यꣳ ह॒व्यम् घृ॒तव॒थ् स्वाहा᳚ । \newline
13. घृ॒तव॒थ् स्वाहा॒ स्वाहा॑ घृ॒तव॑द् घृ॒तव॒थ् स्वाहै॒न्द्र ऐ॒न्द्रः स्वाहा॑ घृ॒तव॑द् घृ॒तव॒थ् स्वाहै॒न्द्रः । \newline
14. घृ॒तव॒दिति॑ घृ॒त - व॒त् । \newline
15. स्वाहै॒न्द्र ऐ॒न्द्रः स्वाहा॒ स्वाहै॒न्द्रः प्रा॒णः प्रा॒ण ऐ॒न्द्रः स्वाहा॒ स्वाहै॒न्द्रः प्रा॒णः । \newline
16. ऐ॒न्द्रः प्रा॒णः प्रा॒ण ऐ॒न्द्र ऐ॒न्द्रः प्रा॒णो अङ्गे॑अङ्गे॒ अङ्गे॑अङ्गे प्रा॒ण ऐ॒न्द्र ऐ॒न्द्रः प्रा॒णो अङ्गे॑अङ्गे । \newline
17. प्रा॒णो अङ्गे॑अङ्गे॒ अङ्गे॑अङ्गे प्रा॒णः प्रा॒णो अङ्गे॑अङ्गे॒ नि न्यङ्गे॑अङ्गे प्रा॒णः प्रा॒णो अङ्गे॑अङ्गे॒ नि । \newline
18. प्रा॒ण इति॑ प्र - अ॒नः । \newline
19. अङ्गे॑अङ्गे॒ नि न्यङ्गे॑अङ्गे॒ अङ्गे॑अङ्गे॒ नि दे᳚द्ध्यद् देद्ध्य॒न् न्यङ्गे॑अङ्गे॒ अङ्गे॑अङ्गे॒ नि दे᳚द्ध्यत् । \newline
20. अङ्गे॑अङ्ग॒ इत्यङ्गे᳚ - अ॒ङ्गे॒ । \newline
21. नि दे᳚द्ध्यद् देद्ध्य॒न् नि नि दे᳚द्ध्यदै॒न्द्र ऐ॒न्द्रो दे᳚द्ध्य॒न् नि नि दे᳚द्ध्यदै॒न्द्रः । \newline
22. दे॒द्ध्य॒दै॒न्द्र ऐ॒न्द्रो दे᳚द्ध्यद् देद्ध्यदै॒न्द्रो॑ ऽपा॒नो॑ ऽपा॒न ऐ॒न्द्रो दे᳚द्ध्यद् देद्ध्यदै॒न्द्रो॑ ऽपा॒नः । \newline
23. ऐ॒न्द्रो॑ ऽपा॒नो॑ ऽपा॒न ऐ॒न्द्र ऐ॒न्द्रो॑ ऽपा॒नो अङ्गे॑अङ्गे॒ अङ्गे॑अङ्गे ऽपा॒न ऐ॒न्द्र ऐ॒न्द्रो॑ ऽपा॒नो अङ्गे॑अङ्गे । \newline
24. अ॒पा॒नो अङ्गे॑अङ्गे॒ अङ्गे॑अङ्गे ऽपा॒नो॑ ऽपा॒नो अङ्गे॑अङ्गे॒ वि व्यङ्गे॑अङ्गे ऽपा॒नो॑ ऽपा॒नो अङ्गे॑अङ्गे॒ वि । \newline
25. अ॒पा॒न इत्य॑प - अ॒नः । \newline
26. अङ्गे॑अङ्गे॒ वि व्यङ्गे॑अङ्गे॒ अङ्गे॑अङ्गे॒ वि बो॑भुवद् बोभुव॒द् व्यङ्गे॑अङ्गे॒ अङ्गे॑अङ्गे॒ वि बो॑भुवत् । \newline
27. अङ्गे॑अङ्ग॒ इत्यङ्गे᳚ - अ॒ङ्गे॒ । \newline
28. वि बो॑भुवद् बोभुव॒द् वि वि बो॑भुव॒द् देव॒ देव॑ बोभुव॒द् वि वि बो॑भुव॒द् देव॑ । \newline
29. बो॒भु॒व॒द् देव॒ देव॑ बोभुवद् बोभुव॒द् देव॑ त्वष्ट स्त्वष्ट॒र् देव॑ बोभुवद् बोभुव॒द् देव॑ त्वष्टः । \newline
30. देव॑ त्वष्ट स्त्वष्ट॒र् देव॒ देव॑ त्वष्ट॒र् भूरि॒ भूरि॑ त्वष्ट॒र् देव॒ देव॑ त्वष्ट॒र् भूरि॑ । \newline
31. त्व॒ष्ट॒र् भूरि॒ भूरि॑ त्वष्ट स्त्वष्ट॒र् भूरि॑ ते ते॒ भूरि॑ त्वष्ट स्त्वष्ट॒र् भूरि॑ ते । \newline
32. भूरि॑ ते ते॒ भूरि॒ भूरि॑ ते॒ सꣳस॒(ग्म्॒) सꣳस॑म् ते॒ भूरि॒ भूरि॑ ते॒ सꣳस᳚म् । \newline
33. ते॒ सꣳस॒(ग्म्॒) सꣳस॑म् ते ते॒ सꣳस॑ मेत्वेतु॒ सꣳस॑म् ते ते॒ सꣳस॑ मेतु । \newline
34. सꣳस॑ मेत्वेतु॒ सꣳस॒(ग्म्॒) सꣳस॑ मेतु॒ विषु॑रूपा॒ विषु॑रूपा एतु॒ सꣳस॒(ग्म्॒) सꣳस॑ मेतु॒ विषु॑रूपाः । \newline
35. सꣳस॒मिति॒ सं - स॒म् । \newline
36. ए॒तु॒ विषु॑रूपा॒ विषु॑रूपा एत्वेतु॒ विषु॑रूपा॒ यद् यद् विषु॑रूपा एत्वेतु॒ विषु॑रूपा॒ यत् । \newline
37. विषु॑रूपा॒ यद् यद् विषु॑रूपा॒ विषु॑रूपा॒ यथ् सल॑क्ष्माणः॒ सल॑क्ष्माणो॒ यद् विषु॑रूपा॒ विषु॑रूपा॒ यथ् सल॑क्ष्माणः । \newline
38. विषु॑रूपा॒ इति॒ विषु॑ - रू॒पाः॒ । \newline
39. यथ् सल॑क्ष्माणः॒ सल॑क्ष्माणो॒ यद् यथ् सल॑क्ष्माणो॒ भव॑थ॒ भव॑थ॒ सल॑क्ष्माणो॒ यद् यथ् सल॑क्ष्माणो॒ भव॑थ । \newline
40. सल॑क्ष्माणो॒ भव॑थ॒ भव॑थ॒ सल॑क्ष्माणः॒ सल॑क्ष्माणो॒ भव॑थ देव॒त्रा दे॑व॒त्रा भव॑थ॒ सल॑क्ष्माणः॒ सल॑क्ष्माणो॒ भव॑थ देव॒त्रा । \newline
41. सल॑क्ष्माण॒ इति॒ स - ल॒क्ष्मा॒णः॒ । \newline
42. भव॑थ देव॒त्रा दे॑व॒त्रा भव॑थ॒ भव॑थ देव॒त्रा यन्तं॒ ॅयन्त॑म् देव॒त्रा भव॑थ॒ भव॑थ देव॒त्रा यन्त᳚म् । \newline
43. दे॒व॒त्रा यन्तं॒ ॅयन्त॑म् देव॒त्रा दे॑व॒त्रा यन्त॒ मव॒से ऽव॑से॒ यन्त॑म् देव॒त्रा दे॑व॒त्रा यन्त॒ मव॑से । \newline
44. दे॒व॒त्रेति॑ देव - त्रा । \newline
45. यन्त॒ मव॒से ऽव॑से॒ यन्तं॒ ॅयन्त॒ मव॑से॒ सखा॑यः॒ सखा॒यो ऽव॑से॒ यन्तं॒ ॅयन्त॒ मव॑से॒ सखा॑यः । \newline
46. अव॑से॒ सखा॑यः॒ सखा॒यो ऽव॒से ऽव॑से॒ सखा॒यो ऽन्वनु॒ सखा॒यो ऽव॒से ऽव॑से॒ सखा॒यो ऽनु॑ । \newline
47. सखा॒यो ऽन्वनु॒ सखा॑यः॒ सखा॒यो ऽनु॑ त्वा॒ त्वा ऽनु॒ सखा॑यः॒ सखा॒यो ऽनु॑ त्वा । \newline
48. अनु॑ त्वा॒ त्वा ऽन्वनु॑ त्वा मा॒ता मा॒ता त्वा ऽन्वनु॑ त्वा मा॒ता । \newline
49. त्वा॒ मा॒ता मा॒ता त्वा᳚ त्वा मा॒ता पि॒तरः॑ पि॒तरो॑ मा॒ता त्वा᳚ त्वा मा॒ता पि॒तरः॑ । \newline
50. मा॒ता पि॒तरः॑ पि॒तरो॑ मा॒ता मा॒ता पि॒तरो॑ मदन्तु मदन्तु पि॒तरो॑ मा॒ता मा॒ता पि॒तरो॑ मदन्तु । \newline
51. पि॒तरो॑ मदन्तु मदन्तु पि॒तरः॑ पि॒तरो॑ मदन्तु॒ श्रीः श्रीर् म॑दन्तु पि॒तरः॑ पि॒तरो॑ मदन्तु॒ श्रीः । \newline
52. म॒द॒न्तु॒ श्रीः श्रीर् म॑दन्तु मदन्तु॒ श्री र॑स्यसि॒ श्रीर् म॑दन्तु मदन्तु॒ श्री र॑सि । \newline
53. श्री र॑स्यसि॒ श्रीः श्री र॑स्य॒ग्नि र॒ग्नि र॑सि॒ श्रीः श्री र॑स्य॒ग्निः । \newline
54. अ॒स्य॒ग्नि र॒ग्नि र॑स्यस्य॒ग्नि स्त्वा᳚ त्वा॒ ऽग्नि र॑स्यस्य॒ग्नि स्त्वा᳚ । \newline
55. अ॒ग्नि स्त्वा᳚ त्वा॒ ऽग्नि र॒ग्नि स्त्वा᳚ श्रीणातु श्रीणातु त्वा॒ ऽग्नि र॒ग्नि स्त्वा᳚ श्रीणातु । \newline
56. त्वा॒ श्री॒णा॒तु॒ श्री॒णा॒तु॒ त्वा॒ त्वा॒ श्री॒णा॒त्वाप॒ आपः॑ श्रीणातु त्वा त्वा श्रीणा॒त्वापः॑ । \newline
57. श्री॒णा॒त्वाप॒ आपः॑ श्रीणातु श्रीणा॒त्वापः॒ सꣳ स मापः॑ श्रीणातु श्रीणा॒त्वापः॒ सम् । \newline
58. आपः॒ सꣳ स माप॒ आपः॒ स म॑रिणन् नरिण॒न् थ्स माप॒ आपः॒ स म॑रिणन्न् । \newline
59. स म॑रिणन् नरिण॒न् थ्सꣳ स म॑रिण॒न्॒. वात॑स्य॒ वात॑स्यारिण॒न् थ्सꣳ स म॑रिण॒न्॒. वात॑स्य । \newline
60. अ॒रि॒ण॒न्॒. वात॑स्य॒ वात॑स्यारिणन् नरिण॒न्॒. वात॑स्य त्वा त्वा॒ वात॑स्यारिणन् नरिण॒न्॒. वात॑स्य त्वा । \newline
61. वात॑स्य त्वा त्वा॒ वात॑स्य॒ वात॑स्य त्वा॒ ध्रज्यै॒ ध्रज्यै᳚ त्वा॒ वात॑स्य॒ वात॑स्य त्वा॒ ध्रज्यै᳚ । \newline
\pagebreak
\markright{ TS 1.3.10.2  \hfill https://www.vedavms.in \hfill}

\section{ TS 1.3.10.2 }

\textbf{TS 1.3.10.2 } \newline
\textbf{Samhita Paata} \newline

त्वा॒ ध्रज्यै॑ पू॒ष्णो रꣳह्या॑ अ॒पामोष॑धीनाꣳ॒॒ रोहि॑ष्यै घृ॒तं घृ॑तपावानः पिबत॒ वसां᳚ ॅवसापावानः पिबता॒ऽन्तरि॑क्षस्य ह॒विर॑सि॒ स्वाहा᳚ त्वा॒ऽन्तरि॑क्षाय॒ दिशः॑ प्र॒दिश॑ आ॒दिशो॑ वि॒दिश॑ उ॒द्दिशः॒ स्वाहा॑ दि॒ग्भ्यो नमो॑ दि॒ग्भ्यः ॥ \newline

\textbf{Pada Paata} \newline

त्वा॒ । ध्रज्यै᳚ । पू॒ष्णः । रꣳह्यै᳚ । अ॒पाम् । ओष॑धीनाम् । रोहि॑ष्यै । घृ॒तम् । घृ॒त॒पा॒वा॒न॒ इति॑ घृत - पा॒वा॒नः॒ । पि॒ब॒त॒ । वसा᳚म् । व॒सा॒पा॒वा॒न॒ इति॑ वसा - पा॒वा॒नः॒ । पि॒ब॒त॒ । अ॒न्तरि॑क्षस्य । ह॒विः । अ॒सि॒ । स्वाहा᳚ । त्वा॒ । अ॒न्तरि॑क्षाय । दिशः॑ । प्र॒दिश॒ इति॑ प्र - दिशः॑ । आ॒दिश॒ इत्या᳚ - दिशः॑ । वि॒दिश॒ इति॑ वि - दिशः॑ । उ॒द्दिश॒ इत्यु॑त् - दिशः॑ । स्वाहा᳚ । दि॒ग्भ्य इति॑ दिक् - भ्यः । नमः॑ । दि॒ग्भ्य इति॑ दिक् - भ्यः ॥  \newline


\textbf{Krama Paata} \newline

त्वा॒ ध्रज्यै᳚ । ध्रज्यै॑ पू॒ष्णः । पू॒ष्णो रꣳह्ये᳚ । रꣳह्या॑ अ॒पाम् । अ॒पामोष॑धीनाम् । ओष॑धीनाꣳ॒॒ रोहि॑ष्यै । रोहि॑ष्यै घृ॒तम् । घृ॒तम् घृ॑तपावानः । घृ॒त॒पा॒वा॒नः॒ पि॒ब॒त॒ । घृ॒त॒पा॒वा॒न॒ इति॑ घृत - पा॒वा॒नः॒ । पि॒ब॒त॒ वसा᳚म् । वसां᳚ ॅवसापावानः । व॒सा॒पा॒वा॒नः॒ पि॒ब॒त॒ । व॒सा॒पा॒वा॒न॒ इति॑ वसा - पा॒वा॒नः॒ । पि॒ब॒ता॒न्तरि॑क्षस्य । अ॒न्तरि॑क्षस्य ह॒विः । ह॒विर॑सि । अ॒सि॒ स्वाहा᳚ । स्वाहा᳚ त्वा । त्वा॒ऽन्तरि॑क्षाय । अ॒न्तरि॑क्षाय॒ दिशः॑ । दिशः॑ प्र॒दिशः॑ । प्र॒दिश॑ आ॒दिशः॑ । प्र॒दिश॒ इति॑ प्र - दिशः॑ । आ॒दिशो॑ वि॒दिशः॑ । आ॒दिश॒ इत्या᳚ - दिशः॑ । वि॒दिश॑ उ॒द्दिशः॑ । वि॒दिश॒ इति॑ वि - दिशः॑ । उ॒द्दिशः॒ स्वाहा᳚ । उ॒द्दिश॒ इत्यु॑त् - दिशः॑ । स्वाहा॑ दि॒ग्भ्यः । दि॒ग्भ्यो नमः॑ । दि॒ग्भ्य इति॑ दिक् - भ्यः । नमो॑ दि॒ग्भ्यः । दि॒ग्भ्य इति॑ दिक् - भ्यः । \newline

\textbf{Jatai Paata} \newline

1. त्वा॒ ध्रज्यै॒ ध्रज्यै᳚ त्वा त्वा॒ ध्रज्यै᳚ । \newline
2. ध्रज्यै॑ पू॒ष्णः पू॒ष्णो ध्रज्यै॒ ध्रज्यै॑ पू॒ष्णः । \newline
3. पू॒ष्णो रꣳह्यै॒ रꣳह्यै॑ पू॒ष्णः पू॒ष्णो रꣳह्यै᳚ । \newline
4. रꣳह्या॑ अ॒पा म॒पाꣳ रꣳह्यै॒ रꣳह्या॑ अ॒पाम् । \newline
5. अ॒पा मोष॑धीना॒ मोष॑धीना म॒पा म॒पा मोष॑धीनाम् । \newline
6. ओष॑धीना॒(ग्म्॒) रोहि॑ष्यै॒ रोहि॑ष्या॒ ओष॑धीना॒ मोष॑धीना॒(ग्म्॒) रोहि॑ष्यै । \newline
7. रोहि॑ष्यै घृ॒तम् घृ॒तꣳ रोहि॑ष्यै॒ रोहि॑ष्यै घृ॒तम् । \newline
8. घृ॒तम् घृ॑तपावानो घृतपावानो घृ॒तम् घृ॒तम् घृ॑तपावानः । \newline
9. घृ॒त॒पा॒वा॒नः॒ पि॒ब॒त॒ पि॒ब॒त॒ घृ॒त॒पा॒वा॒नो॒ घृ॒त॒पा॒वा॒नः॒ पि॒ब॒त॒ । \newline
10. घृ॒त॒पा॒वा॒न॒ इति॑ घृत - पा॒वा॒नः॒ । \newline
11. पि॒ब॒त॒ वसां॒ ॅवसा᳚म् पिबत पिबत॒ वसा᳚म् । \newline
12. वसां᳚ ॅवसापावानो वसापावानो॒ वसां॒ ॅवसां᳚ ॅवसापावानः । \newline
13. व॒सा॒पा॒वा॒नः॒ पि॒ब॒त॒ पि॒ब॒त॒ व॒सा॒पा॒वा॒नो॒ व॒सा॒पा॒वा॒नः॒ पि॒ब॒त॒ । \newline
14. व॒सा॒पा॒वा॒न॒ इति॑ वसा - पा॒वा॒नः॒ । \newline
15. पि॒ब॒ ता॒न्तरि॑क्ष स्या॒न्तरि॑क्षस्य पिबत पिब ता॒न्तरि॑क्षस्य । \newline
16. अ॒न्तरि॑क्षस्य ह॒विर्. ह॒वि र॒न्तरि॑क्ष स्या॒न्तरि॑क्षस्य ह॒विः । \newline
17. ह॒वि र॑स्यसि ह॒विर्. ह॒वि र॑सि । \newline
18. अ॒सि॒ स्वाहा॒ स्वाहा᳚ ऽस्यसि॒ स्वाहा᳚ । \newline
19. स्वाहा᳚ त्वा त्वा॒ स्वाहा॒ स्वाहा᳚ त्वा । \newline
20. त्वा॒ ऽन्तरि॑क्षा या॒न्तरि॑क्षाय त्वा त्वा॒ ऽन्तरि॑क्षाय । \newline
21. अ॒न्तरि॑क्षाय॒ दिशो॒ दिशो॒ ऽन्तरि॑क्षा या॒न्तरि॑क्षाय॒ दिशः॑ । \newline
22. दिशः॑ प्र॒दिशः॑ प्र॒दिशो॒ दिशो॒ दिशः॑ प्र॒दिशः॑ । \newline
23. प्र॒दिश॑ आ॒दिश॑ आ॒दिशः॑ प्र॒दिशः॑ प्र॒दिश॑ आ॒दिशः॑ । \newline
24. प्र॒दिश॒ इति॑ प्र - दिशः॑ । \newline
25. आ॒दिशो॑ वि॒दिशो॑ वि॒दिश॑ आ॒दिश॑ आ॒दिशो॑ वि॒दिशः॑ । \newline
26. आ॒दिश॒ इत्या᳚ - दिशः॑ । \newline
27. वि॒दिश॑ उ॒द्दिश॑ उ॒द्दिशो॑ वि॒दिशो॑ वि॒दिश॑ उ॒द्दिशः॑ । \newline
28. वि॒दिश॒ इति॑ वि - दिशः॑ । \newline
29. उ॒द्दिशः॒ स्वाहा॒ स्वा हो॒द्दिश॑ उ॒द्दिशः॒ स्वाहा᳚ । \newline
30. उ॒द्दिश॒ इत्यु॑त् - दिशः॑ । \newline
31. स्वाहा॑ दि॒ग्भ्यो दि॒ग्भ्यः स्वाहा॒ स्वाहा॑ दि॒ग्भ्यः । \newline
32. दि॒ग्भ्यो नमो॒ नमो॑ दि॒ग्भ्यो दि॒ग्भ्यो नमः॑ । \newline
33. दि॒ग्भ्य इति॑ दिक् - भ्यः । \newline
34. नमो॑ दि॒ग्भ्यो दि॒ग्भ्यो नमो॒ नमो॑ दि॒ग्भ्यः । \newline
35. दि॒ग्भ्य इति॑ दिक् - भ्यः । \newline

\textbf{Ghana Paata } \newline

1. त्वा॒ ध्रज्यै॒ ध्रज्यै᳚ त्वा त्वा॒ ध्रज्यै॑ पू॒ष्णः पू॒ष्णो ध्रज्यै᳚ त्वा त्वा॒ ध्रज्यै॑ पू॒ष्णः । \newline
2. ध्रज्यै॑ पू॒ष्णः पू॒ष्णो ध्रज्यै॒ ध्रज्यै॑ पू॒ष्णो रꣳह्यै॒ रꣳह्यै॑ पू॒ष्णो ध्रज्यै॒ ध्रज्यै॑ पू॒ष्णो रꣳह्यै᳚ । \newline
3. पू॒ष्णो रꣳह्यै॒ रꣳह्यै॑ पू॒ष्णः पू॒ष्णो रꣳह्या॑ अ॒पा म॒पाꣳ रꣳह्यै॑ पू॒ष्णः पू॒ष्णो रꣳह्या॑ अ॒पाम् । \newline
4. रꣳह्या॑ अ॒पा म॒पाꣳ रꣳह्यै॒ रꣳह्या॑ अ॒पा मोष॑धीना॒ मोष॑धीना म॒पाꣳ रꣳह्यै॒ रꣳह्या॑ अ॒पा मोष॑धीनाम् । \newline
5. अ॒पा मोष॑धीना॒ मोष॑धीना म॒पा म॒पा मोष॑धीना॒(ग्म्॒) रोहि॑ष्यै॒ रोहि॑ष्या॒ ओष॑धीना म॒पा म॒पा मोष॑धीना॒(ग्म्॒) रोहि॑ष्यै । \newline
6. ओष॑धीना॒(ग्म्॒) रोहि॑ष्यै॒ रोहि॑ष्या॒ ओष॑धीना॒ मोष॑धीना॒(ग्म्॒) रोहि॑ष्यै घृ॒तम् घृ॒तꣳ रोहि॑ष्या॒ ओष॑धीना॒ मोष॑धीना॒(ग्म्॒) रोहि॑ष्यै घृ॒तम् । \newline
7. रोहि॑ष्यै घृ॒तम् घृ॒तꣳ रोहि॑ष्यै॒ रोहि॑ष्यै घृ॒तम् घृ॑तपावानो घृतपावानो घृ॒तꣳ रोहि॑ष्यै॒ रोहि॑ष्यै घृ॒तम् घृ॑तपावानः । \newline
8. घृ॒तम् घृ॑तपावानो घृतपावानो घृ॒तम् घृ॒तम् घृ॑तपावानः पिबत पिबत घृतपावानो घृ॒तम् घृ॒तम् घृ॑तपावानः पिबत । \newline
9. घृ॒त॒पा॒वा॒नः॒ पि॒ब॒त॒ पि॒ब॒त॒ घृ॒त॒पा॒वा॒नो॒ घृ॒त॒पा॒वा॒नः॒ पि॒ब॒त॒ वसां॒ ॅवसा᳚म् पिबत घृतपावानो घृतपावानः पिबत॒ वसा᳚म् । \newline
10. घृ॒त॒पा॒वा॒न॒ इति॑ घृत - पा॒वा॒नः॒ । \newline
11. पि॒ब॒त॒ वसां॒ ॅवसा᳚म् पिबत पिबत॒ वसां᳚ ॅवसापावानो वसापावानो॒ वसा᳚म् पिबत पिबत॒ वसां᳚ ॅवसापावानः । \newline
12. वसां᳚ ॅवसापावानो वसापावानो॒ वसां॒ ॅवसां᳚ ॅवसापावानः पिबत पिबत वसापावानो॒ वसां॒ ॅवसां᳚ ॅवसापावानः पिबत । \newline
13. व॒सा॒पा॒वा॒नः॒ पि॒ब॒त॒ पि॒ब॒त॒ व॒सा॒पा॒वा॒नो॒ व॒सा॒पा॒वा॒नः॒ पि॒ब॒ता॒ न्तरि॑क्षस्या॒ न्तरि॑क्षस्य पिबत वसापावानो वसापावानः पिबता॒ न्तरि॑क्षस्य । \newline
14. व॒सा॒पा॒वा॒न॒ इति॑ वसा - पा॒वा॒नः॒ । \newline
15. पि॒ब॒ता॒ न्तरि॑क्षस्या॒ न्तरि॑क्षस्य पिबत पिबता॒ न्तरि॑क्षस्य ह॒विर्. ह॒वि र॒न्तरि॑क्षस्य पिबत पिबता॒ न्तरि॑क्षस्य ह॒विः । \newline
16. अ॒न्तरि॑क्षस्य ह॒विर्. ह॒वि र॒न्तरि॑क्षस्या॒ न्तरि॑क्षस्य ह॒विर॑स्यसि ह॒वि र॒न्तरि॑क्षस्या॒ न्तरि॑क्षस्य ह॒विर॑सि । \newline
17. ह॒विर॑स्यसि ह॒विर्. ह॒विर॑सि॒ स्वाहा॒ स्वाहा॑ ऽसि ह॒विर्. ह॒विर॑सि॒ स्वाहा᳚ । \newline
18. अ॒सि॒ स्वाहा॒ स्वाहा᳚ ऽस्यसि॒ स्वाहा᳚ त्वा त्वा॒ स्वाहा᳚ ऽस्यसि॒ स्वाहा᳚ त्वा । \newline
19. स्वाहा᳚ त्वा त्वा॒ स्वाहा॒ स्वाहा᳚ त्वा॒ ऽन्तरि॑क्षाया॒ न्तरि॑क्षाय त्वा॒ स्वाहा॒ स्वाहा᳚ त्वा॒ ऽन्तरि॑क्षाय । \newline
20. त्वा॒ ऽन्तरि॑क्षाया॒ न्तरि॑क्षाय त्वा त्वा॒ ऽन्तरि॑क्षाय॒ दिशो॒ दिशो॒ ऽन्तरि॑क्षाय त्वा त्वा॒ ऽन्तरि॑क्षाय॒ दिशः॑ । \newline
21. अ॒न्तरि॑क्षाय॒ दिशो॒ दिशो॒ ऽन्तरि॑क्षाया॒ न्तरि॑क्षाय॒ दिशः॑ प्र॒दिशः॑ प्र॒दिशो॒ दिशो॒ ऽन्तरि॑क्षाया॒ न्तरि॑क्षाय॒ दिशः॑ प्र॒दिशः॑ । \newline
22. दिशः॑ प्र॒दिशः॑ प्र॒दिशो॒ दिशो॒ दिशः॑ प्र॒दिश॑ आ॒दिश॑ आ॒दिशः॑ प्र॒दिशो॒ दिशो॒ दिशः॑ प्र॒दिश॑ आ॒दिशः॑ । \newline
23. प्र॒दिश॑ आ॒दिश॑ आ॒दिशः॑ प्र॒दिशः॑ प्र॒दिश॑ आ॒दिशो॑ वि॒दिशो॑ वि॒दिश॑ आ॒दिशः॑ प्र॒दिशः॑ प्र॒दिश॑ आ॒दिशो॑ वि॒दिशः॑ । \newline
24. प्र॒दिश॒ इति॑ प्र - दिशः॑ । \newline
25. आ॒दिशो॑ वि॒दिशो॑ वि॒दिश॑ आ॒दिश॑ आ॒दिशो॑ वि॒दिश॑ उ॒द्दिश॑ उ॒द्दिशो॑ वि॒दिश॑ आ॒दिश॑ आ॒दिशो॑ वि॒दिश॑ उ॒द्दिशः॑ । \newline
26. आ॒दिश॒ इत्या᳚ - दिशः॑ । \newline
27. वि॒दिश॑ उ॒द्दिश॑ उ॒द्दिशो॑ वि॒दिशो॑ वि॒दिश॑ उ॒द्दिशः॒ स्वाहा॒ स्वाहो॒द्दिशो॑ वि॒दिशो॑ वि॒दिश॑ उ॒द्दिशः॒ स्वाहा᳚ । \newline
28. वि॒दिश॒ इति॑ वि - दिशः॑ । \newline
29. उ॒द्दिशः॒ स्वाहा॒ स्वाहो॒द्दिश॑ उ॒द्दिशः॒ स्वाहा॑ दि॒ग्भ्यो दि॒ग्भ्यः स्वाहो॒द्दिश॑ उ॒द्दिशः॒ स्वाहा॑ दि॒ग्भ्यः । \newline
30. उ॒द्दिश॒ इत्यु॑त् - दिशः॑ । \newline
31. स्वाहा॑ दि॒ग्भ्यो दि॒ग्भ्यः स्वाहा॒ स्वाहा॑ दि॒ग्भ्यो नमो॒ नमो॑ दि॒ग्भ्यः स्वाहा॒ स्वाहा॑ दि॒ग्भ्यो नमः॑ । \newline
32. दि॒ग्भ्यो नमो॒ नमो॑ दि॒ग्भ्यो दि॒ग्भ्यो नमो॑ दि॒ग्भ्यो दि॒ग्भ्यो नमो॑ दि॒ग्भ्यो दि॒ग्भ्यो नमो॑ दि॒ग्भ्यः । \newline
33. दि॒ग्भ्य इति॑ दिक् - भ्यः । \newline
34. नमो॑ दि॒ग्भ्यो दि॒ग्भ्यो नमो॒ नमो॑ दि॒ग्भ्यः । \newline
35. दि॒ग्भ्य इति॑ दिक् - भ्यः । \newline
\pagebreak
\markright{ TS 1.3.11.1  \hfill https://www.vedavms.in \hfill}

\section{ TS 1.3.11.1 }

\textbf{TS 1.3.11.1 } \newline
\textbf{Samhita Paata} \newline

स॒मु॒द्रं ग॑च्छ॒ स्वाहा॒ऽन्तरि॑क्षं गच्छ॒ स्वाहा॑ दे॒वꣳ स॑वि॒तारं॑ गच्छ॒ स्वाहा॑ऽहोरा॒त्रे ग॑च्छ॒ स्वाहा॑ मि॒त्रावरु॑णौ गच्छ॒ स्वाहा॒ सोमं॑ गच्छ॒ स्वाहा॑ य॒ज्ञ्ं ग॑च्छ॒ स्वाहा॒ छन्दाꣳ॑सि गच्छ॒ स्वाहा॒ द्यावा॑पृथि॒वी ग॑च्छ॒ स्वाहा॒ नभो॑ दि॒व्यं ग॑च्छ॒ स्वाहा॒ऽग्निं ॅवै᳚श्वान॒रं ग॑च्छ॒ स्वाहा॒ऽद्भ्यस्त्वौष॑धीभ्यो॒ मनो॑ मे॒ हार्दि॑ यच्छ त॒नूं त्वचं॑ पु॒त्रं नप्ता॑रमशीय॒ शुग॑सि॒ ( ) तम॒भि शो॑च॒ यो᳚ऽस्मान् द्वेष्टि॒ यं च॑ व॒यं द्वि॒ष्मो धाम्नो॑धाम्नो राजन्नि॒तो व॑रुण नो मुञ्च॒ यदापो॒ अघ्नि॑या॒ वरु॒णेति॒ शपा॑महे॒ ततो॑ वरुण नो मुञ्च ॥ \newline

\textbf{Pada Paata} \newline

स॒मु॒द्रम् । ग॒च्छ॒ । स्वाहा᳚ । अ॒न्तरि॑क्षम् । ग॒च्छ॒ । स्वाहा᳚ । दे॒वम् । स॒वि॒तार᳚म् । ग॒च्छ॒ । स्वाहा᳚ । अ॒हो॒रा॒त्रे इत्य॑हः - रा॒त्रे । ग॒च्छ॒ । स्वाहा᳚ । मि॒त्रावरु॑णा॒विति॑ मि॒त्रा - वरु॑णौ । ग॒च्छ॒ । स्वाहा᳚ । सोम᳚म् । ग॒च्छ॒ । स्वाहा᳚ । य॒ज्ञ्म् । ग॒च्छ॒ । स्वाहा᳚ । छन्दाꣳ॑सि । ग॒च्छ॒ । स्वाहा᳚ । द्यावा॑पृथि॒वी इति॒ द्यावा᳚ - पृ॒थि॒वी । ग॒च्छ॒ । स्वाहा᳚ । नभः॑ । दि॒व्यम् । ग॒च्छ॒ । स्वाहा᳚ । अ॒ग्निम् । वै॒श्वा॒न॒रम् । ग॒च्छ॒ । स्वाहा᳚ । अ॒द्भ्य इत्य॑त् - भ्यः । त्वा॒ । ओष॑धीभ्य॒ इत्योष॑धि - भ्यः॒ । मनः॑ । मे॒ । हार्दि॑ । य॒च्छ॒ । त॒नूम् । त्वच᳚म् । पु॒त्रम् । नप्ता॑रम् । अ॒शी॒य॒ । शुक् । अ॒सि॒ ( ) । तम् । अ॒भीति॑ । शो॒च॒ । यः । अ॒स्मान् । द्वेष्टि॑ । यम् । च॒ । व॒यम् । द्वि॒ष्मः । धाम्नो॑ धाम्न॒ इति॒ धाम्नः॑ - धा॒म्नः॒ । रा॒ज॒न्न् । इ॒तः । व॒रु॒ण॒ । नः॒ । मु॒ञ्च॒ । यत् । आपः॑ । अघ्नि॑याः । वरु॑ण । इति॑ । शपा॑महे । ततः॑ । व॒रु॒ण॒ । नः॒ । मु॒ञ्च॒ ॥  \newline


\textbf{Krama Paata} \newline

स॒मु॒द्रम् ग॑च्छ । ग॒च्छ॒ स्वाहा᳚ । स्वाहा॒ऽन्तरि॑क्षम् । अ॒न्तरि॑क्षम् गच्छ । ग॒च्छ॒ स्वाहा᳚ । स्वाहा॑ दे॒वम् । दे॒वꣳ स॑वि॒तार᳚म् । स॒वि॒तारं॑ गच्छ । ग॒च्छ॒ स्वाहा᳚ । स्वाहा॑ऽहोरा॒त्रे । अ॒हो॒रा॒त्रे ग॑च्छ । अ॒हो॒रा॒त्रे इत्य॑हः - रा॒त्रे । ग॒च्छ॒ स्वाहा᳚ । स्वाहा॑ मि॒त्रावरु॑णौ । मि॒त्रावरु॑णौ गच्छ । मि॒त्रावरु॑णा॒विति॑ मि॒त्रा - वरु॑णौ । ग॒च्छ॒ स्वाहा᳚ । स्वाहा॒ सोम᳚म् । सोम॑म् गच्छ । ग॒च्छ॒ स्वाहा᳚ । स्वाहा॑ य॒ज्ञ्म् । य॒ज्ञ्म् ग॑च्छ । ग॒च्छ॒ स्वाहा᳚ । स्वाहा॒ छन्दाꣳ॑सि । छन्दाꣳ॑सि गच्छ । ग॒च्छ॒ स्वाहा᳚ । स्वाहा॒ द्यावा॑पृथि॒वी । द्यावा॑पृथि॒वी ग॑च्छ । द्यावा॑पृथि॒वी इति॒ द्यावा᳚ - पृ॒थि॒वी । ग॒च्छ॒ स्वाहा᳚ । स्वाहा॒ नभः॑ । नभो॑ दि॒व्यम् । दि॒व्यम् ग॑च्छ । ग॒च्छ॒ स्वाहा᳚ । स्वाहा॒ऽग्निम् । अ॒ग्निं ॅवै᳚श्वान॒रम् । वै॒श्वा॒न॒रम् ग॑च्छ । ग॒च्छ॒ स्वाहा᳚ । स्वाहा॒ऽद्भ्यः । अ॒द्भ्यस्त्वा᳚ । अ॒द्भ्य इत्य॑त् - भ्यः । त्वौष॑धीभ्यः । ओष॑धीभ्यो॒ मनः॑ । ओष॑धीभ्य॒ इत्योष॑धि - भ्यः॒ । मनो॑ मे । मे॒ हार्दि॑ । हार्दि॑ यच्छ । य॒च्छ॒ त॒नूम् । त॒नूम् त्वच᳚म् । त्वच॑म् पु॒त्रम् । पु॒त्रम् नप्ता॑रम् । नप्ता॑रमशीय । अ॒शी॒य॒ शुक् । शुग॑सि ( ) । अ॒सि॒ तम् । तम॒भि । अ॒भि शो॑च । शो॒च॒ यः । यो᳚ऽस्मान् । अ॒स्मान् द्वेष्टि॑ । द्वेष्टि॒ यम् । यम् च॑ । च॒ व॒यम् । व॒यम् द्वि॒ष्मः । द्वि॒ष्मो धाम्नो॑धाम्नः । धाम्नो॑धाम्नो राजन्न् । धाम्नो॑धाम्न॒ इति॒ धाम्नः॑ - धा॒म्नः॒ । रा॒ज॒न्नि॒तः । इ॒तो व॑रुण । व॒रु॒ण॒ नः॒ । नो॒ मु॒ञ्च॒ । मु॒ञ्च॒ यत् । यदापः॑ । आपो॒ अघ्नि॑याः । अघ्नि॑या॒ वरु॑ण । वरु॒णेति॑ । इति॒ शपा॑महे । शपा॑महे॒ ततः॑ । ततो॑ वरुण । व॒रु॒ण॒ नः॒ । नो॒ मु॒ञ्च॒ । मु॒ञ्चेति॑ मुञ्च । \newline

\textbf{Jatai Paata} \newline

1. स॒मु॒द्रम् ग॑च्छ गच्छ समु॒द्रꣳ स॑मु॒द्रम् ग॑च्छ । \newline
2. ग॒च्छ॒ स्वाहा॒ स्वाहा॑ गच्छ गच्छ॒ स्वाहा᳚ । \newline
3. स्वाहा॒ ऽन्तरि॑क्ष म॒न्तरि॑क्ष॒(ग्ग्॒) स्वाहा॒ स्वाहा॒ ऽन्तरि॑क्षम् । \newline
4. अ॒न्तरि॑क्षम् गच्छ गच्छा॒न्तरि॑क्ष म॒न्तरि॑क्षम् गच्छ । \newline
5. ग॒च्छ॒ स्वाहा॒ स्वाहा॑ गच्छ गच्छ॒ स्वाहा᳚ । \newline
6. स्वाहा॑ दे॒वम् दे॒वꣳ स्वाहा॒ स्वाहा॑ दे॒वम् । \newline
7. दे॒वꣳ स॑वि॒तार(ग्म्॑) सवि॒तार॑म् दे॒वम् दे॒वꣳ स॑वि॒तार᳚म् । \newline
8. स॒वि॒तार॑म् गच्छ गच्छ सवि॒तार(ग्म्॑) सवि॒तार॑म् गच्छ । \newline
9. ग॒च्छ॒ स्वाहा॒ स्वाहा॑ गच्छ गच्छ॒ स्वाहा᳚ । \newline
10. स्वाहा॑ ऽहोरा॒त्रे अ॑होरा॒त्रे स्वाहा॒ स्वाहा॑ ऽहोरा॒त्रे । \newline
11. अ॒हो॒रा॒त्रे ग॑च्छ गच्छाहोरा॒त्रे अ॑होरा॒त्रे ग॑च्छ । \newline
12. अ॒हो॒रा॒त्रे इत्य॑हः - रा॒त्रे । \newline
13. ग॒च्छ॒ स्वाहा॒ स्वाहा॑ गच्छ गच्छ॒ स्वाहा᳚ । \newline
14. स्वाहा॑ मि॒त्रावरु॑णौ मि॒त्रावरु॑णौ॒ स्वाहा॒ स्वाहा॑ मि॒त्रावरु॑णौ । \newline
15. मि॒त्रावरु॑णौ गच्छ गच्छ मि॒त्रावरु॑णौ मि॒त्रावरु॑णौ गच्छ । \newline
16. मि॒त्रावरु॑णा॒विति॑ मि॒त्रा - वरु॑णौ । \newline
17. ग॒च्छ॒ स्वाहा॒ स्वाहा॑ गच्छ गच्छ॒ स्वाहा᳚ । \newline
18. स्वाहा॒ सोम॒(ग्म्॒) सोम॒(ग्ग्॒) स्वाहा॒ स्वाहा॒ सोम᳚म् । \newline
19. सोम॑म् गच्छ गच्छ॒ सोम॒(ग्म्॒) सोम॑म् गच्छ । \newline
20. ग॒च्छ॒ स्वाहा॒ स्वाहा॑ गच्छ गच्छ॒ स्वाहा᳚ । \newline
21. स्वाहा॑ य॒ज्ञ्ं ॅय॒ज्ञ्ꣳ स्वाहा॒ स्वाहा॑ य॒ज्ञ्म् । \newline
22. य॒ज्ञ्म् ग॑च्छ गच्छ य॒ज्ञ्ं ॅय॒ज्ञ्म् ग॑च्छ । \newline
23. ग॒च्छ॒ स्वाहा॒ स्वाहा॑ गच्छ गच्छ॒ स्वाहा᳚ । \newline
24. स्वाहा॒ छन्दा(ग्म्॑)सि॒ छन्दा(ग्म्॑)सि॒ स्वाहा॒ स्वाहा॒ छन्दा(ग्म्॑)सि । \newline
25. छन्दा(ग्म्॑)सि गच्छ गच्छ॒ छन्दा(ग्म्॑)सि॒ छन्दा(ग्म्॑)सि गच्छ । \newline
26. ग॒च्छ॒ स्वाहा॒ स्वाहा॑ गच्छ गच्छ॒ स्वाहा᳚ । \newline
27. स्वाहा॒ द्यावा॑पृथि॒वी द्यावा॑पृथि॒वी स्वाहा॒ स्वाहा॒ द्यावा॑पृथि॒वी । \newline
28. द्यावा॑पृथि॒वी ग॑च्छ गच्छ॒ द्यावा॑पृथि॒वी द्यावा॑पृथि॒वी ग॑च्छ । \newline
29. द्यावा॑पृथि॒वी इति॒ द्यावा᳚ - पृ॒थि॒वी । \newline
30. ग॒च्छ॒ स्वाहा॒ स्वाहा॑ गच्छ गच्छ॒ स्वाहा᳚ । \newline
31. स्वाहा॒ नभो॒ नभः॒ स्वाहा॒ स्वाहा॒ नभः॑ । \newline
32. नभो॑ दि॒व्यम् दि॒व्यम् नभो॒ नभो॑ दि॒व्यम् । \newline
33. दि॒व्यम् ग॑च्छ गच्छ दि॒व्यम् दि॒व्यम् ग॑च्छ । \newline
34. ग॒च्छ॒ स्वाहा॒ स्वाहा॑ गच्छ गच्छ॒ स्वाहा᳚ । \newline
35. स्वाहा॒ ऽग्नि म॒ग्निꣳ स्वाहा॒ स्वाहा॒ ऽग्निम् । \newline
36. अ॒ग्निं ॅवै᳚श्वान॒रं ॅवै᳚श्वान॒र म॒ग्नि म॒ग्निं ॅवै᳚श्वान॒रम् । \newline
37. वै॒श्वा॒न॒रम् ग॑च्छ गच्छ वैश्वान॒रं ॅवै᳚श्वान॒रम् ग॑च्छ । \newline
38. ग॒च्छ॒ स्वाहा॒ स्वाहा॑ गच्छ गच्छ॒ स्वाहा᳚ । \newline
39. स्वाहा॒ ऽद्भ्यो᳚ ऽद्भ्यः स्वाहा॒ स्वाहा॒ ऽद्भ्यः । \newline
40. अ॒द्भ्य स्त्वा᳚ त्वा॒ ऽद्भ्यो᳚ ऽद्भ्य स्त्वा᳚ । \newline
41. अ॒द्भ्य इत्य॑त् - भ्यः । \newline
42. त्वौष॑धीभ्य॒ ओष॑धीभ्यस्त्वा॒ त्वौष॑धीभ्यः । \newline
43. ओष॑धीभ्यो॒ मनो॒ मन॒ ओष॑धीभ्य॒ ओष॑धीभ्यो॒ मनः॑ । \newline
44. ओष॑धीभ्य॒ इत्योष॑धि - भ्यः॒ । \newline
45. मनो॑ मे मे॒ मनो॒ मनो॑ मे । \newline
46. मे॒ हार्दि॒ हार्दि॑ मे मे॒ हार्दि॑ । \newline
47. हार्दि॑ यच्छ यच्छ॒ हार्दि॒ हार्दि॑ यच्छ । \newline
48. य॒च्छ॒ त॒नूम् त॒नूं ॅय॑च्छ यच्छ त॒नूम् । \newline
49. त॒नूम् त्वच॒म् त्वच॑म् त॒नूम् त॒नूम् त्वच᳚म् । \newline
50. त्वच॑म् पु॒त्रम् पु॒त्रम् त्वच॒म् त्वच॑म् पु॒त्रम् । \newline
51. पु॒त्रम् नप्ता॑र॒म् नप्ता॑रम् पु॒त्रम् पु॒त्रम् नप्ता॑रम् । \newline
52. नप्ता॑र मशी याशीय॒ नप्ता॑र॒म् नप्ता॑र मशीय । \newline
53. अ॒शी॒य॒ शुक् छुग॑शी याशीय॒ शुक् । \newline
54. शुग॑ स्यसि॒ शुक् छुग॑सि । \newline
55. अ॒सि॒ तम् त म॑स्यसि॒ तम् । \newline
56. त म॒भ्य॑भि तम् त म॒भि । \newline
57. अ॒भि शो॑च शोचा॒ भ्य॑भि शो॑च । \newline
58. शो॒च॒ यो यः शो॑च शोच॒ यः । \newline
59. यो᳚ ऽस्मा न॒स्मान्. यो यो᳚ ऽस्मान् । \newline
60. अ॒स्मान् द्वेष्टि॒ द्वेष्ट्य॒ स्मा न॒स्मान् द्वेष्टि॑ । \newline
61. द्वेष्टि॒ यं ॅयम् द्वेष्टि॒ द्वेष्टि॒ यम् । \newline
62. यम् च॑ च॒ यं ॅयम् च॑ । \newline
63. च॒ व॒यं ॅव॒यम् च॑ च व॒यम् । \newline
64. व॒यम् द्वि॒ष्मो द्वि॒ष्मो व॒यं ॅव॒यम् द्वि॒ष्मः । \newline
65. द्वि॒ष्मो धाम्नो॑धाम्नो॒ धाम्नो॑धाम्नो द्वि॒ष्मो द्वि॒ष्मो धाम्नो॑धाम्नः । \newline
66. धाम्नो॑धाम्नो राजन् राज॒न् धाम्नो॑धाम्नो॒ धाम्नो॑धाम्नो राजन्न् । \newline
67. धाम्नो॑धाम्न॒ इति॒ धाम्नः॑ - धा॒म्नः॒ । \newline
68. रा॒ज॒न् नि॒त इ॒तो रा॑जन् राजन् नि॒तः । \newline
69. इ॒तो व॑रुण वरुणे॒ त इ॒तो व॑रुण । \newline
70. व॒रु॒ण॒ नो॒ नो॒ व॒रु॒ण॒ व॒रु॒ण॒ नः॒ । \newline
71. नो॒ मु॒ञ्च॒ मु॒ञ्च॒ नो॒ नो॒ मु॒ञ्च॒ । \newline
72. मु॒ञ्च॒ यद् यन् मु॑ञ्च मुञ्च॒ यत् । \newline
73. यदाप॒ आपो॒ यद् यदापः॑ । \newline
74. आपो॒ अघ्नि॑या॒ अघ्नि॑या॒ आप॒ आपो॒ अघ्नि॑याः । \newline
75. अघ्नि॑या॒ वरु॑ण॒ वरु॒णाघ्नि॑या॒ अघ्नि॑या॒ वरु॑ण । \newline
76. वरु॒णे तीति॒ वरु॑ण॒ वरु॒णे ति॑ । \newline
77. इति॒ शपा॑महे॒ शपा॑मह॒ इतीति॒ शपा॑महे । \newline
78. शपा॑महे॒ तत॒ स्ततः॒ शपा॑महे॒ शपा॑महे॒ ततः॑ । \newline
79. ततो॑ वरुण वरुण॒ तत॒ स्ततो॑ वरुण । \newline
80. व॒रु॒ण॒ नो॒ नो॒ व॒रु॒ण॒ व॒रु॒ण॒ नः॒ । \newline
81. नो॒ मु॒ञ्च॒ मु॒ञ्च॒ नो॒ नो॒ मु॒ञ्च॒ । \newline
82. मु॒ञ्चेति॑ मुञ्च । \newline

\textbf{Ghana Paata } \newline

1. स॒मु॒द्रम् ग॑च्छ गच्छ समु॒द्रꣳ स॑मु॒द्रम् ग॑च्छ॒ स्वाहा॒ स्वाहा॑ गच्छ समु॒द्रꣳ स॑मु॒द्रम् ग॑च्छ॒ स्वाहा᳚ । \newline
2. ग॒च्छ॒ स्वाहा॒ स्वाहा॑ गच्छ गच्छ॒ स्वाहा॒ ऽन्तरि॑क्ष म॒न्तरि॑क्ष॒(ग्ग्॒) स्वाहा॑ गच्छ गच्छ॒ स्वाहा॒ ऽन्तरि॑क्षम् । \newline
3. स्वाहा॒ ऽन्तरि॑क्ष म॒न्तरि॑क्ष॒(ग्ग्॒) स्वाहा॒ स्वाहा॒ ऽन्तरि॑क्षम् गच्छ गच्छा॒न्तरि॑क्ष॒(ग्ग्॒) स्वाहा॒ स्वाहा॒ ऽन्तरि॑क्षम् गच्छ । \newline
4. अ॒न्तरि॑क्षम् गच्छ गच्छा॒न्तरि॑क्ष म॒न्तरि॑क्षम् गच्छ॒ स्वाहा॒ स्वाहा॑ गच्छा॒न्तरि॑क्ष म॒न्तरि॑क्षम् गच्छ॒ स्वाहा᳚ । \newline
5. ग॒च्छ॒ स्वाहा॒ स्वाहा॑ गच्छ गच्छ॒ स्वाहा॑ दे॒वम् दे॒वꣳ स्वाहा॑ गच्छ गच्छ॒ स्वाहा॑ दे॒वम् । \newline
6. स्वाहा॑ दे॒वम् दे॒वꣳ स्वाहा॒ स्वाहा॑ दे॒वꣳ स॑वि॒तार(ग्म्॑) सवि॒तार॑म् दे॒वꣳ स्वाहा॒ स्वाहा॑ दे॒वꣳ स॑वि॒तार᳚म् । \newline
7. दे॒वꣳ स॑वि॒तार(ग्म्॑) सवि॒तार॑म् दे॒वम् दे॒वꣳ स॑वि॒तार॑म् गच्छ गच्छ सवि॒तार॑म् दे॒वम् दे॒वꣳ स॑वि॒तार॑म् गच्छ । \newline
8. स॒वि॒तार॑म् गच्छ गच्छ सवि॒तार(ग्म्॑) सवि॒तार॑म् गच्छ॒ स्वाहा॒ स्वाहा॑ गच्छ सवि॒तार(ग्म्॑) सवि॒तार॑म् गच्छ॒ स्वाहा᳚ । \newline
9. ग॒च्छ॒ स्वाहा॒ स्वाहा॑ गच्छ गच्छ॒ स्वाहा॑ ऽहोरा॒त्रे अ॑होरा॒त्रे स्वाहा॑ गच्छ गच्छ॒ स्वाहा॑ ऽहोरा॒त्रे । \newline
10. स्वाहा॑ ऽहोरा॒त्रे अ॑होरा॒त्रे स्वाहा॒ स्वाहा॑ ऽहोरा॒त्रे ग॑च्छ गच्छाहोरा॒त्रे स्वाहा॒ स्वाहा॑ ऽहोरा॒त्रे ग॑च्छ । \newline
11. अ॒हो॒रा॒त्रे ग॑च्छ गच्छाहोरा॒त्रे अ॑होरा॒त्रे ग॑च्छ॒ स्वाहा॒ स्वाहा॑ गच्छाहोरा॒त्रे अ॑होरा॒त्रे ग॑च्छ॒ स्वाहा᳚ । \newline
12. अ॒हो॒रा॒त्रे इत्य॑हः - रा॒त्रे । \newline
13. ग॒च्छ॒ स्वाहा॒ स्वाहा॑ गच्छ गच्छ॒ स्वाहा॑ मि॒त्रावरु॑णौ मि॒त्रावरु॑णौ॒ स्वाहा॑ गच्छ गच्छ॒ स्वाहा॑ मि॒त्रावरु॑णौ । \newline
14. स्वाहा॑ मि॒त्रावरु॑णौ मि॒त्रावरु॑णौ॒ स्वाहा॒ स्वाहा॑ मि॒त्रावरु॑णौ गच्छ गच्छ मि॒त्रावरु॑णौ॒ स्वाहा॒ स्वाहा॑ मि॒त्रावरु॑णौ गच्छ । \newline
15. मि॒त्रावरु॑णौ गच्छ गच्छ मि॒त्रावरु॑णौ मि॒त्रावरु॑णौ गच्छ॒ स्वाहा॒ स्वाहा॑ गच्छ मि॒त्रावरु॑णौ मि॒त्रावरु॑णौ गच्छ॒ स्वाहा᳚ । \newline
16. मि॒त्रावरु॑णा॒विति॑ मि॒त्रा - वरु॑णौ । \newline
17. ग॒च्छ॒ स्वाहा॒ स्वाहा॑ गच्छ गच्छ॒ स्वाहा॒ सोम॒(ग्म्॒) सोम॒(ग्ग्॒) स्वाहा॑ गच्छ गच्छ॒ स्वाहा॒ सोम᳚म् । \newline
18. स्वाहा॒ सोम॒(ग्म्॒) सोम॒(ग्ग्॒) स्वाहा॒ स्वाहा॒ सोम॑म् गच्छ गच्छ॒ सोम॒(ग्ग्॒) स्वाहा॒ स्वाहा॒ सोम॑म् गच्छ । \newline
19. सोम॑म् गच्छ गच्छ॒ सोम॒(ग्म्॒) सोम॑म् गच्छ॒ स्वाहा॒ स्वाहा॑ गच्छ॒ सोम॒(ग्म्॒) सोम॑म् गच्छ॒ स्वाहा᳚ । \newline
20. ग॒च्छ॒ स्वाहा॒ स्वाहा॑ गच्छ गच्छ॒ स्वाहा॑ य॒ज्ञ्ं ॅय॒ज्ञ्ꣳ स्वाहा॑ गच्छ गच्छ॒ स्वाहा॑ य॒ज्ञ्म् । \newline
21. स्वाहा॑ य॒ज्ञ्ं ॅय॒ज्ञ्ꣳ स्वाहा॒ स्वाहा॑ य॒ज्ञ्म् ग॑च्छ गच्छ य॒ज्ञ्ꣳ स्वाहा॒ स्वाहा॑ य॒ज्ञ्म् ग॑च्छ । \newline
22. य॒ज्ञ्म् ग॑च्छ गच्छ य॒ज्ञ्ं ॅय॒ज्ञ्म् ग॑च्छ॒ स्वाहा॒ स्वाहा॑ गच्छ य॒ज्ञ्ं ॅय॒ज्ञ्म् ग॑च्छ॒ स्वाहा᳚ । \newline
23. ग॒च्छ॒ स्वाहा॒ स्वाहा॑ गच्छ गच्छ॒ स्वाहा॒ छन्दा(ग्म्॑)सि॒ छन्दा(ग्म्॑)सि॒ स्वाहा॑ गच्छ गच्छ॒ स्वाहा॒ छन्दा(ग्म्॑)सि । \newline
24. स्वाहा॒ छन्दा(ग्म्॑)सि॒ छन्दा(ग्म्॑)सि॒ स्वाहा॒ स्वाहा॒ छन्दा(ग्म्॑)सि गच्छ गच्छ॒ छन्दा(ग्म्॑)सि॒ स्वाहा॒ स्वाहा॒ छन्दा(ग्म्॑)सि गच्छ । \newline
25. छन्दा(ग्म्॑)सि गच्छ गच्छ॒ छन्दा(ग्म्॑)सि॒ छन्दा(ग्म्॑)सि गच्छ॒ स्वाहा॒ स्वाहा॑ गच्छ॒ छन्दा(ग्म्॑)सि॒ छन्दा(ग्म्॑)सि गच्छ॒ स्वाहा᳚ । \newline
26. ग॒च्छ॒ स्वाहा॒ स्वाहा॑ गच्छ गच्छ॒ स्वाहा॒ द्यावा॑पृथि॒वी द्यावा॑पृथि॒वी स्वाहा॑ गच्छ गच्छ॒ स्वाहा॒ द्यावा॑पृथि॒वी । \newline
27. स्वाहा॒ द्यावा॑पृथि॒वी द्यावा॑पृथि॒वी स्वाहा॒ स्वाहा॒ द्यावा॑पृथि॒वी ग॑च्छ गच्छ॒ द्यावा॑पृथि॒वी स्वाहा॒ स्वाहा॒ द्यावा॑पृथि॒वी ग॑च्छ । \newline
28. द्यावा॑पृथि॒वी ग॑च्छ गच्छ॒ द्यावा॑पृथि॒वी द्यावा॑पृथि॒वी ग॑च्छ॒ स्वाहा॒ स्वाहा॑ गच्छ॒ द्यावा॑पृथि॒वी द्यावा॑पृथि॒वी ग॑च्छ॒ स्वाहा᳚ । \newline
29. द्यावा॑पृथि॒वी इति॒ द्यावा᳚ - पृ॒थि॒वी । \newline
30. ग॒च्छ॒ स्वाहा॒ स्वाहा॑ गच्छ गच्छ॒ स्वाहा॒ नभो॒ नभः॒ स्वाहा॑ गच्छ गच्छ॒ स्वाहा॒ नभः॑ । \newline
31. स्वाहा॒ नभो॒ नभः॒ स्वाहा॒ स्वाहा॒ नभो॑ दि॒व्यम् दि॒व्यम् नभः॒ स्वाहा॒ स्वाहा॒ नभो॑ दि॒व्यम् । \newline
32. नभो॑ दि॒व्यम् दि॒व्यम् नभो॒ नभो॑ दि॒व्यम् ग॑च्छ गच्छ दि॒व्यम् नभो॒ नभो॑ दि॒व्यम् ग॑च्छ । \newline
33. दि॒व्यम् ग॑च्छ गच्छ दि॒व्यम् दि॒व्यम् ग॑च्छ॒ स्वाहा॒ स्वाहा॑ गच्छ दि॒व्यम् दि॒व्यम् ग॑च्छ॒ स्वाहा᳚ । \newline
34. ग॒च्छ॒ स्वाहा॒ स्वाहा॑ गच्छ गच्छ॒ स्वाहा॒ ऽग्नि म॒ग्निꣳ स्वाहा॑ गच्छ गच्छ॒ स्वाहा॒ ऽग्निम् । \newline
35. स्वाहा॒ ऽग्नि म॒ग्निꣳ स्वाहा॒ स्वाहा॒ ऽग्निं ॅवै᳚श्वान॒रं ॅवै᳚श्वान॒र म॒ग्निꣳ स्वाहा॒ स्वाहा॒ ऽग्निं ॅवै᳚श्वान॒रम् । \newline
36. अ॒ग्निं ॅवै᳚श्वान॒रं ॅवै᳚श्वान॒र म॒ग्नि म॒ग्निं ॅवै᳚श्वान॒रम् ग॑च्छ गच्छ वैश्वान॒र म॒ग्नि म॒ग्निं ॅवै᳚श्वान॒रम् ग॑च्छ । \newline
37. वै॒श्वा॒न॒रम् ग॑च्छ गच्छ वैश्वान॒रं ॅवै᳚श्वान॒रम् ग॑च्छ॒ स्वाहा॒ स्वाहा॑ गच्छ वैश्वान॒रं ॅवै᳚श्वान॒रम् ग॑च्छ॒ स्वाहा᳚ । \newline
38. ग॒च्छ॒ स्वाहा॒ स्वाहा॑ गच्छ गच्छ॒ स्वाहा॒ ऽद्भ्यो᳚ ऽद्भ्यः स्वाहा॑ गच्छ गच्छ॒ स्वाहा॒ ऽद्भ्यः । \newline
39. स्वाहा॒ ऽद्भ्यो᳚ ऽद्भ्यः स्वाहा॒ स्वाहा॒ ऽद्भ्यस्त्वा᳚ त्वा॒ ऽद्भ्यः स्वाहा॒ स्वाहा॒ ऽद्भ्यस्त्वा᳚ । \newline
40. अ॒द्भ्यस्त्वा᳚ त्वा॒ ऽद्भ्यो᳚ ऽद्भ्य स्त्वौष॑धीभ्य॒ ओष॑धीभ्य स्त्वा॒ ऽद्भ्यो᳚ ऽद्भ्य स्त्वौष॑धीभ्यः । \newline
41. अ॒द्भ्य इत्य॑त् - भ्यः । \newline
42. त्वौष॑धीभ्य॒ ओष॑धीभ्य स्त्वा॒ त्वौष॑धीभ्यो॒ मनो॒ मन॒ ओष॑धीभ्य स्त्वा॒ त्वौष॑धीभ्यो॒ मनः॑ । \newline
43. ओष॑धीभ्यो॒ मनो॒ मन॒ ओष॑धीभ्य॒ ओष॑धीभ्यो॒ मनो॑ मे मे॒ मन॒ ओष॑धीभ्य॒ ओष॑धीभ्यो॒ मनो॑ मे । \newline
44. ओष॑धीभ्य॒ इत्योष॑धि - भ्यः॒ । \newline
45. मनो॑ मे मे॒ मनो॒ मनो॑ मे॒ हार्दि॒ हार्दि॑ मे॒ मनो॒ मनो॑ मे॒ हार्दि॑ । \newline
46. मे॒ हार्दि॒ हार्दि॑ मे मे॒ हार्दि॑ यच्छ यच्छ॒ हार्दि॑ मे मे॒ हार्दि॑ यच्छ । \newline
47. हार्दि॑ यच्छ यच्छ॒ हार्दि॒ हार्दि॑ यच्छ त॒नूम् त॒नूं ॅय॑च्छ॒ हार्दि॒ हार्दि॑ यच्छ त॒नूम् । \newline
48. य॒च्छ॒ त॒नूम् त॒नूं ॅय॑च्छ यच्छ त॒नूम् त्वच॒म् त्वच॑म् त॒नूं ॅय॑च्छ यच्छ त॒नूम् त्वच᳚म् । \newline
49. त॒नूम् त्वच॒म् त्वच॑म् त॒नूम् त॒नूम् त्वच॑म् पु॒त्रम् पु॒त्रम् त्वच॑म् त॒नूम् त॒नूम् त्वच॑म् पु॒त्रम् । \newline
50. त्वच॑म् पु॒त्रम् पु॒त्रम् त्वच॒म् त्वच॑म् पु॒त्रम् नप्ता॑र॒म् नप्ता॑रम् पु॒त्रम् त्वच॒म् त्वच॑म् पु॒त्रम् नप्ता॑रम् । \newline
51. पु॒त्रम् नप्ता॑र॒म् नप्ता॑रम् पु॒त्रम् पु॒त्रम् नप्ता॑र मशीयाशीय॒ नप्ता॑रम् पु॒त्रम् पु॒त्रम् नप्ता॑र मशीय । \newline
52. नप्ता॑र मशीयाशीय॒ नप्ता॑र॒म् नप्ता॑र मशीय॒ शुक् छुग॑शीय॒ नप्ता॑र॒म् नप्ता॑र मशीय॒ शुक् । \newline
53. अ॒शी॒य॒ शुक् छुग॑शीयाशीय॒ शुग॑स्यसि॒ शुग॑शीयाशीय॒ शुग॑सि । \newline
54. शुग॑स्यसि॒ शुक् छुग॑सि॒ तम् त म॑सि॒ शुक् छुग॑सि॒ तम् । \newline
55. अ॒सि॒ तम् त म॑स्यसि॒ त म॒भ्य॑भि त म॑स्यसि॒ त म॒भि । \newline
56. त म॒भ्य॑भि तम् त म॒भि शो॑च शोचा॒भि तम् त म॒भि शो॑च । \newline
57. अ॒भि शो॑च शोचा॒ भ्य॑भि शो॑च॒ यो यः शो॑चा॒ भ्य॑भि शो॑च॒ यः । \newline
58. शो॒च॒ यो यः शो॑च शोच॒ यो᳚ ऽस्मा न॒स्मान्. यः शो॑च शोच॒ यो᳚ ऽस्मान् । \newline
59. यो᳚ ऽस्मा न॒स्मान्. यो यो᳚ ऽस्मान् द्वेष्टि॒ द्वेष्ट्य॒स्मान्. यो यो᳚ ऽस्मान् द्वेष्टि॑ । \newline
60. अ॒स्मान् द्वेष्टि॒ द्वेष्ट्य॒स्मा न॒स्मान् द्वेष्टि॒ यं ॅयम् द्वेष्ट्य॒स्मा न॒स्मान् द्वेष्टि॒ यम् । \newline
61. द्वेष्टि॒ यं ॅयम् द्वेष्टि॒ द्वेष्टि॒ यम् च॑ च॒ यम् द्वेष्टि॒ द्वेष्टि॒ यम् च॑ । \newline
62. यम् च॑ च॒ यं ॅयम् च॑ व॒यं ॅव॒यम् च॒ यं ॅयम् च॑ व॒यम् । \newline
63. च॒ व॒यं ॅव॒यम् च॑ च व॒यम् द्वि॒ष्मो द्वि॒ष्मो व॒यम् च॑ च व॒यम् द्वि॒ष्मः । \newline
64. व॒यम् द्वि॒ष्मो द्वि॒ष्मो व॒यं ॅव॒यम् द्वि॒ष्मो धाम्नो॑धाम्नो॒ धाम्नो॑धाम्नो द्वि॒ष्मो व॒यं ॅव॒यम् द्वि॒ष्मो धाम्नो॑धाम्नः । \newline
65. द्वि॒ष्मो धाम्नो॑धाम्नो॒ धाम्नो॑धाम्नो द्वि॒ष्मो द्वि॒ष्मो धाम्नो॑धाम्नो राजन् राज॒न् धाम्नो॑धाम्नो द्वि॒ष्मो द्वि॒ष्मो धाम्नो॑धाम्नो राजन्न् । \newline
66. धाम्नो॑धाम्नो राजन् राज॒न् धाम्नो॑धाम्नो॒ धाम्नो॑धाम्नो राजन् नि॒त इ॒तो रा॑ज॒न् धाम्नो॑धाम्नो॒ धाम्नो॑धाम्नो राजन् नि॒तः । \newline
67. धाम्नो॑धाम्न॒ इति॒ धाम्नः॑ - धा॒म्नः॒ । \newline
68. रा॒ज॒न् नि॒त इ॒तो रा॑जन् राजन् नि॒तो व॑रुण वरुणे॒ तो रा॑जन् राजन् नि॒तो व॑रुण । \newline
69. इ॒तो व॑रुण वरुणे॒ त इ॒तो व॑रुण नो नो वरुणे॒ त इ॒तो व॑रुण नः । \newline
70. व॒रु॒ण॒ नो॒ नो॒ व॒रु॒ण॒ व॒रु॒ण॒ नो॒ मु॒ञ्च॒ मु॒ञ्च॒ नो॒ व॒रु॒ण॒ व॒रु॒ण॒ नो॒ मु॒ञ्च॒ । \newline
71. नो॒ मु॒ञ्च॒ मु॒ञ्च॒ नो॒ नो॒ मु॒ञ्च॒ यद् यन् मु॑ञ्च नो नो मुञ्च॒ यत् । \newline
72. मु॒ञ्च॒ यद् यन् मु॑ञ्च मुञ्च॒ यदाप॒ आपो॒ यन् मु॑ञ्च मुञ्च॒ यदापः॑ । \newline
73. यदाप॒ आपो॒ यद् यदापो॒ अघ्नि॑या॒ अघ्नि॑या॒ आपो॒ यद् यदापो॒ अघ्नि॑याः । \newline
74. आपो॒ अघ्नि॑या॒ अघ्नि॑या॒ आप॒ आपो॒ अघ्नि॑या॒ वरु॑ण॒ वरु॒णाघ्नि॑या॒ आप॒ आपो॒ अघ्नि॑या॒ वरु॑ण । \newline
75. अघ्नि॑या॒ वरु॑ण॒ वरु॒णाघ्नि॑या॒ अघ्नि॑या॒ वरु॒णे तीति॒ वरु॒णाघ्नि॑या॒ अघ्नि॑या॒ वरु॒णे ति॑ । \newline
76. वरु॒णे तीति॒ वरु॑ण॒ वरु॒णे ति॒ शपा॑महे॒ शपा॑मह॒ इति॒ वरु॑ण॒ वरु॒णे ति॒ शपा॑महे । \newline
77. इति॒ शपा॑महे॒ शपा॑मह॒ इतीति॒ शपा॑महे॒ तत॒ स्ततः॒ शपा॑मह॒ इतीति॒ शपा॑महे॒ ततः॑ । \newline
78. शपा॑महे॒ तत॒ स्ततः॒ शपा॑महे॒ शपा॑महे॒ ततो॑ वरुण वरुण॒ ततः॒ शपा॑महे॒ शपा॑महे॒ ततो॑ वरुण । \newline
79. ततो॑ वरुण वरुण॒ तत॒ स्ततो॑ वरुण नो नो वरुण॒ तत॒ स्ततो॑ वरुण नः । \newline
80. व॒रु॒ण॒ नो॒ नो॒ व॒रु॒ण॒ व॒रु॒ण॒ नो॒ मु॒ञ्च॒ मु॒ञ्च॒ नो॒ व॒रु॒ण॒ व॒रु॒ण॒ नो॒ मु॒ञ्च॒ । \newline
81. नो॒ मु॒ञ्च॒ मु॒ञ्च॒ नो॒ नो॒ मु॒ञ्च॒ । \newline
82. मु॒ञ्चेति॑ मुञ्च । \newline
\pagebreak
\markright{ TS 1.3.12.1  \hfill https://www.vedavms.in \hfill}

\section{ TS 1.3.12.1 }

\textbf{TS 1.3.12.1 } \newline
\textbf{Samhita Paata} \newline

ह॒विष्म॑तीरि॒मा आपो॑ ह॒विष्मा᳚न् दे॒वो अ॑द्ध्व॒रो ह॒विष्माꣳ॒॒ आ वि॑वासति ह॒विष्माꣳ॑ अस्तु॒ सूर्यः॑ ॥ अ॒ग्नेर्वो ऽप॑न्नगृहस्य॒ सद॑सि सादयामि सु॒म्नाय॑ सुम्निनीः सु॒म्ने मा॑ धत्तेन्द्राग्नि॒योर् भा॑ग॒धेयीः᳚ स्थ मि॒त्रावरु॑णयोर् भाग॒धेयीः᳚ स्थ॒ विश्वे॑षां दे॒वानां᳚ भाग॒धेयीः᳚ स्थ य॒ज्ञे जा॑गृत ॥ \newline

\textbf{Pada Paata} \newline

ह॒विष्म॑तीः । इ॒माः । आपः॑ । ह॒विष्मान्॑ । दे॒वः । अ॒द्ध्व॒रः । ह॒विष्मान्॑ । एति॑ । वि॒वा॒स॒ति॒ । ह॒विष्मान्॑ । अ॒स्तु॒ । सूर्यः॑ ॥ अ॒ग्नेः । वः॒ । अप॑न्नगृह॒स्येत्यप॑न्न - गृ॒ह॒स्य॒ । सद॑सि । सा॒द॒या॒मि॒ । सु॒म्नाय॑ । सु॒म्नि॒नीः॒ । सु॒म्ने । मा॒ । ध॒त्त॒ । इ॒न्द्रा॒ग्नि॒योरिती᳚न्द्र - अ॒ग्नि॒योः । भा॒ग॒धेयी॒रिति॑ भाग - धेयीः᳚ । स्थ॒ । मि॒त्रावरु॑णयो॒रिति॑ मि॒त्रा - वरु॑णयोः । भा॒ग॒धेयी॒रिति॑ भाग - धेयीः᳚ । स्थ॒ । विश्वे॑षाम् । दे॒वाना᳚म् । भा॒ग॒धेयी॒रिति॑ भाग - धेयीः᳚ । स्थ॒ । य॒ज्ञे । जा॒गृ॒त॒ ॥  \newline


\textbf{Krama Paata} \newline

ह॒विष्म॑तीरि॒माः ( ) । इ॒मा आपः॑ । आपो॑ ह॒विष्मान्॑ । ह॒विष्मा᳚न् दे॒वः । दे॒वो अ॑द्ध्व॒रः । अ॒द्ध्व॒रो ह॒विष्मान्॑ । ह॒विष्माꣳ॒॒ आ । आ वि॑वासति । वि॒वा॒स॒ति॒ ह॒विष्मान्॑ । ह॒विष्माꣳ॑ अस्तु । अ॒स्तु॒ सूर्यः॑ । सूर्य॒ इति॒ सूर्यः॑ ॥ अ॒ग्नेर् वः॑ । वोऽप॑न्नगृहस्य । अप॑न्नगृहस्य॒ सद॑सि । अप॑न्नगृह॒स्येत्यप॑न्न - गृ॒ह॒स्य॒ । सद॑सि सादयामि । सा॒द॒या॒मि॒ सु॒म्नाय॑ । सु॒म्नाय॑ सुम्निनीः । सु॒म्नि॒नीः॒ सु॒म्ने । सु॒म्ने मा᳚ । मा॒ ध॒त्त॒ । ध॒त्ते॒न्द्रा॒ग्नि॒योः । इ॒न्द्रा॒ग्नि॒योर् भा॑ग॒धेयीः᳚ । इ॒न्द्रा॒ग्नि॒योरिती᳚न्द्र - अ॒ग्नि॒योः । भा॒ग॒धेयीः᳚ स्थ । भा॒ग॒धेयी॒रिति॑ भाग - धेयीः᳚ । स्थ॒ मि॒त्रावरु॑णयोः । मि॒त्रावरु॑णयोर् भाग॒धेयीः᳚ । मि॒त्रावरु॑णयो॒रिति॑ मि॒त्रा - वरु॑णयोः । भा॒ग॒धेयीः᳚ स्थ । भा॒ग॒धेयी॒रिति॑ भाग - धेयीः᳚ । स्थ॒ विश्वे॑षाम् । विश्वे॑षाम् दे॒वाना᳚म् । दे॒वाना᳚म् भाग॒धेयीः᳚ । भा॒ग॒धेयीः᳚ स्थ । भा॒ग॒धेयी॒रिति॑ भाग - धेयीः᳚ । स्थ॒ य॒ज्ञे । य॒ज्ञे जा॑गृत । जा॒गृ॒तेति॑ जागृत । \newline

\textbf{Jatai Paata} \newline

1. ह॒विष्म॑ती रि॒मा इ॒मा ह॒विष्म॑तीर्. ह॒विष्म॑ती रि॒माः । \newline
2. इ॒मा आप॒ आप॑ इ॒मा इ॒मा आपः॑ । \newline
3. आपो॑ ह॒विष्मा॑न्. ह॒विष्मा॒ नाप॒ आपो॑ ह॒विष्मान्॑ । \newline
4. ह॒विष्मा᳚न् दे॒वो दे॒वो ह॒विष्मा॑न्. ह॒विष्मा᳚न् दे॒वः । \newline
5. दे॒वो अ॑द्ध्व॒रो अ॑द्ध्व॒रो दे॒वो दे॒वो अ॑द्ध्व॒रः । \newline
6. अ॒द्ध्व॒रो ह॒विष्मा॑न्. ह॒विष्मा(ग्म्॑) अद्ध्व॒रो अ॑द्ध्व॒रो ह॒विष्मान्॑ । \newline
7. ह॒विष्मा॒(ग्म्॒) आ ह॒विष्मा॑न्. ह॒विष्मा॒(ग्म्॒) आ । \newline
8. आ वि॑वासति विवास॒त्या वि॑वासति । \newline
9. वि॒वा॒स॒ति॒ ह॒विष्मा॑न्. ह॒विष्मा॑न्. विवासति विवासति ह॒विष्मान्॑ । \newline
10. ह॒विष्मा(ग्म्॑) अस्त्वस्तु ह॒विष्मा॑न्. ह॒विष्मा(ग्म्॑) अस्तु । \newline
11. अ॒स्तु॒ सूर्यः॒ सूर्यो॑ अस्त्वस्तु॒ सूर्यः॑ । \newline
12. सुर्य॒ इति॒ सूर्यः॑ । \newline
13. अ॒ग्नेर् वो॑ वो॒ ऽग्ने र॒ग्नेर् वः॑ । \newline
14. वो ऽप॑न्नगृह॒ स्याप॑न्नगृहस्य वो॒ वो ऽप॑न्नगृहस्य । \newline
15. अप॑न्नगृहस्य॒ सद॑सि॒ सद॒स्यप॑न्नगृह॒ स्याप॑न्नगृहस्य॒ सद॑सि । \newline
16. अप॑न्नगृह॒स्येत्यप॑न्न - गृ॒ह॒स्य॒ । \newline
17. सद॑सि सादयामि सादयामि॒ सद॑सि॒ सद॑सि सादयामि । \newline
18. सा॒द॒या॒मि॒ सु॒म्नाय॑ सु॒म्नाय॑ सादयामि सादयामि सु॒म्नाय॑ । \newline
19. सु॒म्नाय॑ सुम्निनीः सुम्निनीः सु॒म्नाय॑ सु॒म्नाय॑ सुम्निनीः । \newline
20. सु॒म्नि॒नीः॒ सु॒म्ने सु॒म्ने सु॑म्निनीः सुम्निनीः सु॒म्ने । \newline
21. सु॒म्ने मा॑ मा सु॒म्ने सु॒म्ने मा᳚ । \newline
22. मा॒ ध॒त्त॒ ध॒त्त॒ मा॒ मा॒ ध॒त्त॒ । \newline
23. ध॒त्ते॒ न्द्रा॒ग्नि॒यो रि॑न्द्राग्नि॒योर् ध॑त्त धत्ते न्द्राग्नि॒योः । \newline
24. इ॒न्द्रा॒ग्नि॒योर् भा॑ग॒धेयी᳚र् भाग॒धेयी॑ रिन्द्राग्नि॒यो रि॑न्द्राग्नि॒योर् भा॑ग॒धेयीः᳚ । \newline
25. इ॒न्द्रा॒ग्नि॒योरिती᳚न्द्र - अ॒ग्नि॒योः । \newline
26. भा॒ग॒धेयीः᳚ स्थ स्थ भाग॒धेयी᳚र् भाग॒धेयीः᳚ स्थ । \newline
27. भा॒ग॒धेयी॒रिति॑ भाग - धेयीः᳚ । \newline
28. स्थ॒ मि॒त्रावरु॑णयोर् मि॒त्रावरु॑णयोः स्थ स्थ मि॒त्रावरु॑णयोः । \newline
29. मि॒त्रावरु॑णयोर् भाग॒धेयी᳚र् भाग॒धेयी᳚र् मि॒त्रावरु॑णयोर् मि॒त्रावरु॑णयोर् भाग॒धेयीः᳚ । \newline
30. मि॒त्रावरु॑णयो॒रिति॑ मि॒त्रा - वरु॑णयोः । \newline
31. भा॒ग॒धेयीः᳚ स्थ स्थ भाग॒धेयी᳚र् भाग॒धेयीः᳚ स्थ । \newline
32. भा॒ग॒धेयी॒रिति॑ भाग - धेयीः᳚ । \newline
33. स्थ॒ विश्वे॑षां॒ ॅविश्वे॑षाꣳ स्थ स्थ॒ विश्वे॑षाम् । \newline
34. विश्वे॑षाम् दे॒वाना᳚म् दे॒वानां॒ ॅविश्वे॑षां॒ ॅविश्वे॑षाम् दे॒वाना᳚म् । \newline
35. दे॒वाना᳚म् भाग॒धेयी᳚र् भाग॒धेयी᳚र् दे॒वाना᳚म् दे॒वाना᳚म् भाग॒धेयीः᳚ । \newline
36. भा॒ग॒धेयीः᳚ स्थ स्थ भाग॒धेयी᳚र् भाग॒धेयीः᳚ स्थ । \newline
37. भा॒ग॒धेयी॒रिति॑ भाग - धेयीः᳚ । \newline
38. स्थ॒ य॒ज्ञे य॒ज्ञे स्थ॑ स्थ य॒ज्ञे । \newline
39. य॒ज्ञे जा॑गृत जागृत य॒ज्ञे य॒ज्ञे जा॑गृत । \newline
40. जा॒गृ॒तेति॑ जागृत । \newline

\textbf{Ghana Paata } \newline

1. ह॒विष्म॑तीरि॒मा इ॒मा ह॒विष्म॑तीर्. ह॒विष्म॑तीरि॒मा आप॒ आप॑ इ॒मा ह॒विष्म॑तीर्. ह॒विष्म॑तीरि॒मा आपः॑ । \newline
2. इ॒मा आप॒ आप॑ इ॒मा इ॒मा आपो॑ ह॒विष्मान्॑. ह॒विष्मा॒ नाप॑ इ॒मा इ॒मा आपो॑ ह॒विष्मान्॑ । \newline
3. आपो॑ ह॒विष्मान्॑. ह॒विष्मा॒ नाप॒ आपो॑ ह॒विष्मा᳚न् दे॒वो दे॒वो ह॒विष्मा॒ नाप॒ आपो॑ ह॒विष्मा᳚न् दे॒वः । \newline
4. ह॒विष्मा᳚न् दे॒वो दे॒वो ह॒विष्मान्॑. ह॒विष्मा᳚न् दे॒वो अ॑द्ध्व॒रो अ॑द्ध्व॒रो दे॒वो ह॒विष्मान्॑. ह॒विष्मा᳚न् दे॒वो अ॑द्ध्व॒रः । \newline
5. दे॒वो अ॑द्ध्व॒रो अ॑द्ध्व॒रो दे॒वो दे॒वो अ॑द्ध्व॒रो ह॒विष्मान्॑. ह॒विष्मा(ग्म्॑) अद्ध्व॒रो दे॒वो दे॒वो अ॑द्ध्व॒रो ह॒विष्मान्॑ । \newline
6. अ॒द्ध्व॒रो ह॒विष्मान्॑. ह॒विष्मा(ग्म्॑) अद्ध्व॒रो अ॑द्ध्व॒रो ह॒विष्मा॒(ग्म्॒) आ ह॒विष्मा(ग्म्॑) अद्ध्व॒रो अ॑द्ध्व॒रो ह॒विष्मा॒(ग्म्॒) आ । \newline
7. ह॒विष्मा॒(ग्म्॒) आ ह॒विष्मान्॑. ह॒विष्मा॒(ग्म्॒) आ वि॑वासति विवास॒त्या ह॒विष्मान्॑. ह॒विष्मा॒(ग्म्॒) आ वि॑वासति । \newline
8. आ वि॑वासति विवास॒त्या वि॑वासति ह॒विष्मान्॑. ह॒विष्मान्॑. विवास॒त्या वि॑वासति ह॒विष्मान्॑ । \newline
9. वि॒वा॒स॒ति॒ ह॒विष्मान्॑. ह॒विष्मान्॑. विवासति विवासति ह॒विष्मा(ग्म्॑) अस्त्वस्तु ह॒विष्मान्॑. विवासति विवासति ह॒विष्मा(ग्म्॑) अस्तु । \newline
10. ह॒विष्मा(ग्म्॑) अस्त्वस्तु ह॒विष्मान्॑. ह॒विष्मा(ग्म्॑) अस्तु॒ सूर्यः॒ सूर्यो॑ अस्तु ह॒विष्मान्॑. ह॒विष्मा(ग्म्॑) अस्तु॒ सूर्यः॑ । \newline
11. अ॒स्तु॒ सूर्यः॒ सूर्यो॑ अस्त्वस्तु॒ सूर्यः॑ । \newline
12. सुर्य॒ इति॒ सूर्यः॑ । \newline
13. अ॒ग्नेर् वो॑ वो॒ ऽग्नेर॒ग्नेर् वो ऽप॑न्नगृह॒ स्याप॑न्नगृहस्य वो॒ ऽग्नेर॒ग्नेर् वो ऽप॑न्नगृहस्य । \newline
14. वो ऽप॑न्नगृह॒ स्याप॑न्नगृहस्य वो॒ वो ऽप॑न्नगृहस्य॒ सद॑सि॒ सद॒ स्यप॑न्नगृहस्य वो॒ वो ऽप॑न्नगृहस्य॒ सद॑सि । \newline
15. अप॑न्नगृहस्य॒ सद॑सि॒ सद॒ स्यप॑न्नगृह॒ स्याप॑न्नगृहस्य॒ सद॑सि सादयामि सादयामि॒ सद॒स्यप॑न्नगृह॒ स्याप॑न्नगृहस्य॒ सद॑सि सादयामि । \newline
16. अप॑न्नगृह॒स्येत्यप॑न्न - गृ॒ह॒स्य॒ । \newline
17. सद॑सि सादयामि सादयामि॒ सद॑सि॒ सद॑सि सादयामि सु॒म्नाय॑ सु॒म्नाय॑ सादयामि॒ सद॑सि॒ सद॑सि सादयामि सु॒म्नाय॑ । \newline
18. सा॒द॒या॒मि॒ सु॒म्नाय॑ सु॒म्नाय॑ सादयामि सादयामि सु॒म्नाय॑ सुम्निनीः सुम्निनीः सु॒म्नाय॑ सादयामि सादयामि सु॒म्नाय॑ सुम्निनीः । \newline
19. सु॒म्नाय॑ सुम्निनीः सुम्निनीः सु॒म्नाय॑ सु॒म्नाय॑ सुम्निनीः सु॒म्ने सु॒म्ने सु॑म्निनीः सु॒म्नाय॑ सु॒म्नाय॑ सुम्निनीः सु॒म्ने । \newline
20. सु॒म्नि॒नीः॒ सु॒म्ने सु॒म्ने सु॑म्निनीः सुम्निनीः सु॒म्ने मा॑ मा सु॒म्ने सु॑म्निनीः सुम्निनीः सु॒म्ने मा᳚ । \newline
21. सु॒म्ने मा॑ मा सु॒म्ने सु॒म्ने मा॑ धत्त धत्त मा सु॒म्ने सु॒म्ने मा॑ धत्त । \newline
22. मा॒ ध॒त्त॒ ध॒त्त॒ मा॒ मा॒ ध॒त्ते॒ न्द्रा॒ग्नि॒यो रि॑न्द्राग्नि॒योर् ध॑त्त मा मा धत्ते न्द्राग्नि॒योः । \newline
23. ध॒त्ते॒ न्द्रा॒ग्नि॒यो रि॑न्द्राग्नि॒योर् ध॑त्त धत्ते न्द्राग्नि॒योर् भा॑ग॒धेयी᳚र् भाग॒धेयी॑ रिन्द्राग्नि॒योर् ध॑त्त धत्ते न्द्राग्नि॒योर् भा॑ग॒धेयीः᳚ । \newline
24. इ॒न्द्रा॒ग्नि॒योर् भा॑ग॒धेयी᳚र् भाग॒धेयी॑ रिन्द्राग्नि॒यो रि॑न्द्राग्नि॒योर् भा॑ग॒धेयीः᳚ स्थ स्थ भाग॒धेयी॑ रिन्द्राग्नि॒यो रि॑न्द्राग्नि॒योर् भा॑ग॒धेयीः᳚ स्थ । \newline
25. इ॒न्द्रा॒ग्नि॒योरिती᳚न्द्र - अ॒ग्नि॒योः । \newline
26. भा॒ग॒धेयीः᳚ स्थ स्थ भाग॒धेयी᳚र् भाग॒धेयीः᳚ स्थ मि॒त्रावरु॑णयोर् मि॒त्रावरु॑णयोः स्थ भाग॒धेयी᳚र् भाग॒धेयीः᳚ स्थ मि॒त्रावरु॑णयोः । \newline
27. भा॒ग॒धेयी॒रिति॑ भाग - धेयीः᳚ । \newline
28. स्थ॒ मि॒त्रावरु॑णयोर् मि॒त्रावरु॑णयोः स्थ स्थ मि॒त्रावरु॑णयोर् भाग॒धेयी᳚र् भाग॒धेयी᳚र् मि॒त्रावरु॑णयोः स्थ स्थ मि॒त्रावरु॑णयोर् भाग॒धेयीः᳚ । \newline
29. मि॒त्रावरु॑णयोर् भाग॒धेयी᳚र् भाग॒धेयी᳚र् मि॒त्रावरु॑णयोर् मि॒त्रावरु॑णयोर् भाग॒धेयीः᳚ स्थ स्थ भाग॒धेयी᳚र् मि॒त्रावरु॑णयोर् मि॒त्रावरु॑णयोर् भाग॒धेयीः᳚ स्थ । \newline
30. मि॒त्रावरु॑णयो॒रिति॑ मि॒त्रा - वरु॑णयोः । \newline
31. भा॒ग॒धेयीः᳚ स्थ स्थ भाग॒धेयी᳚र् भाग॒धेयीः᳚ स्थ॒ विश्वे॑षां॒ ॅविश्वे॑षाꣳ स्थ भाग॒धेयी᳚र् भाग॒धेयीः᳚ स्थ॒ विश्वे॑षाम् । \newline
32. भा॒ग॒धेयी॒रिति॑ भाग - धेयीः᳚ । \newline
33. स्थ॒ विश्वे॑षां॒ ॅविश्वे॑षाꣳ स्थ स्थ॒ विश्वे॑षाम् दे॒वाना᳚म् दे॒वानां॒ ॅविश्वे॑षाꣳ स्थ स्थ॒ विश्वे॑षाम् दे॒वाना᳚म् । \newline
34. विश्वे॑षाम् दे॒वाना᳚म् दे॒वानां॒ ॅविश्वे॑षां॒ ॅविश्वे॑षाम् दे॒वाना᳚म् भाग॒धेयी᳚र् भाग॒धेयी᳚र् दे॒वानां॒ ॅविश्वे॑षां॒ ॅविश्वे॑षाम् दे॒वाना᳚म् भाग॒धेयीः᳚ । \newline
35. दे॒वाना᳚म् भाग॒धेयी᳚र् भाग॒धेयी᳚र् दे॒वाना᳚म् दे॒वाना᳚म् भाग॒धेयीः᳚ स्थ स्थ भाग॒धेयी᳚र् दे॒वाना᳚म् दे॒वाना᳚म् भाग॒धेयीः᳚ स्थ । \newline
36. भा॒ग॒धेयीः᳚ स्थ स्थ भाग॒धेयी᳚र् भाग॒धेयीः᳚ स्थ य॒ज्ञे य॒ज्ञे स्थ॑ भाग॒धेयी᳚र् भाग॒धेयीः᳚ स्थ य॒ज्ञे । \newline
37. भा॒ग॒धेयी॒रिति॑ भाग - धेयीः᳚ । \newline
38. स्थ॒ य॒ज्ञे य॒ज्ञे स्थ॑ स्थ य॒ज्ञे जा॑गृत जागृत य॒ज्ञे स्थ॑ स्थ य॒ज्ञे जा॑गृत । \newline
39. य॒ज्ञे जा॑गृत जागृत य॒ज्ञे य॒ज्ञे जा॑गृत । \newline
40. जा॒गृ॒तेति॑ जागृत । \newline
\pagebreak
\markright{ TS 1.3.13.1  \hfill https://www.vedavms.in \hfill}

\section{ TS 1.3.13.1 }

\textbf{TS 1.3.13.1 } \newline
\textbf{Samhita Paata} \newline

हृ॒दे त्वा॒ मन॑से त्वा दि॒वे त्वा॒ सूर्या॑य त्वो॒र्द्ध्वमि॒मम॑द्ध्व॒रं कृ॑धि दि॒वि दे॒वेषु॒ होत्रा॑ यच्छ॒ सोम॑ राज॒न्नेह्यव॑ रोह॒ मा भेर्मा सं ॅवि॑क्था॒ मा त्वा॑ हिꣳसिषं प्र॒जास्त्वमु॒पाव॑रोह प्र॒जास्त्वामु॒पाव॑रोहन्तु शृ॒णोत्व॒ग्निः स॒मिधा॒ हवं॑ मे शृ॒ण्वन्त्वापो॑ धि॒षणा᳚श्च दे॒वीः । शृ॒णोत॑ ग्रावाणो वि॒दुषो॒ नु - [ ] \newline

\textbf{Pada Paata} \newline

हृ॒दे । त्वा॒ । मन॑से । त्वा॒ । दि॒वे । त्वा॒ । सूर्या॑य । त्वा॒ । ऊ॒र्द्ध्वम् । इ॒मम् । अ॒द्ध्व॒रम् । कृ॒धि॒ । दि॒वि । दे॒वेषु॑ । होत्राः᳚ । य॒च्छ॒ । सोम॑ । रा॒ज॒न्न् । एति॑ । इ॒हि॒ । अवेति॑ । रो॒ह॒ । मा । भेः । मा । समिति॑ । वि॒क्थाः॒ । मा । त्वा॒ । हिꣳ॒॒सि॒ष॒म् । प्र॒जा इति॑ प्र - जाः । त्वम् । उ॒पाव॑रो॒हेत्यु॑प - अव॑रोह । प्र॒जा इति॑ प्र - जाः । त्वाम् । उ॒पाव॑रोह॒न्त्वित्यु॑प - अव॑रोहन्तु । शृ॒णोतु॑ । अ॒ग्निः । स॒मिधेति॑ सम् - इधा᳚ । हव᳚म् । मे॒ । शृ॒ण्वन्तु॑ । आपः॑ । धि॒षणाः᳚ । च॒ । दे॒वीः ॥ शृ॒णोत॑ । ग्रा॒वा॒णः॒ । वि॒दुषः॑ । नु ।  \newline


\textbf{Krama Paata} \newline

हृ॒दे त्वा᳚ । त्वा॒ मन॑से । मन॑से त्वा । त्वा॒ दि॒वे । दि॒वे त्वा᳚ । त्वा॒ सूर्या॑य । सूर्या॑य त्वा । त्वो॒र्द्ध्वम् । ऊ॒र्द्ध्वमि॒मम् । इ॒मम॑द्ध्व॒रम् । अ॒द्ध्व॒रम् कृ॑धि । कृ॒धि॒ दि॒वि । दि॒वि दे॒वेषु॑ । दे॒वेषु॒ होत्राः᳚ । होत्रा॑ यच्छ । य॒च्छ॒ सोम॑ । सोम॑ राजन्न् । रा॒ज॒न्ना । एहि॑ । इ॒ह्यव॑ । अव॑ रोह । रो॒ह॒ मा । मा भेः । भेर् मा । मा सम् । सं ॅवि॑क्थाः । वि॒क्था॒ मा । मा त्वा᳚ । त्वा॒ हिꣳ॒॒सि॒ष॒म् । हिꣳ॒॒सि॒ष॒म् प्र॒जाः । प्र॒जास्त्वम् । प्र॒जा इति॑ प्र - जाः । त्वमु॒पाव॑रोह । उ॒पाव॑रोह प्र॒जाः । उ॒पाव॑रो॒हेत्यु॑प - अव॑रोह । प्र॒जास्त्वाम् । प्र॒जा इति॑ प्र - जाः । त्वा मु॒पाव॑रोहन्तु । उ॒पाव॑रोहन्तु शृ॒णोतु॑ । उ॒पाव॑रोह॒न्त्वित्यु॑प - अव॑रोहन्तु । शृ॒णोत्व॒ग्निः । अ॒ग्निः स॒मिधा᳚ । स॒मिधा॒ हव᳚म् । स॒मिधेति॑ सम् - इधा᳚ । हव॑म् मे । मे॒ शृ॒ण्वन्तु॑ । शृ॒ण्वन्त्वापः॑ । आपो॑ धि॒षणाः᳚ । धि॒षणा᳚श्च । च॒ दे॒वीः । दे॒वीरिति॑ दे॒वीः ॥ शृ॒णोत॑ ग्रावाणः । ग्रा॒वा॒णो॒ वि॒दुषः॑ । वि॒दुषो॒ नु ( ) । नु य॒ज्ञ्म् \newline

\textbf{Jatai Paata} \newline

1. हृ॒दे त्वा᳚ त्वा हृ॒दे हृ॒दे त्वा᳚ । \newline
2. त्वा॒ मन॑से॒ मन॑से त्वा त्वा॒ मन॑से । \newline
3. मन॑से त्वा त्वा॒ मन॑से॒ मन॑से त्वा । \newline
4. त्वा॒ दि॒वे दि॒वे त्वा᳚ त्वा दि॒वे । \newline
5. दि॒वे त्वा᳚ त्वा दि॒वे दि॒वे त्वा᳚ । \newline
6. त्वा॒ सूर्या॑य॒ सूर्या॑य त्वा त्वा॒ सूर्या॑य । \newline
7. सूर्या॑य त्वा त्वा॒ सूर्या॑य॒ सूर्या॑य त्वा । \newline
8. त्वो॒र्द्ध्व मू॒र्द्ध्वम् त्वा᳚ त्वो॒र्द्ध्वम् । \newline
9. ऊ॒र्द्ध्व मि॒म मि॒म मू॒र्द्ध्व मू॒र्द्ध्व मि॒मम् । \newline
10. इ॒म म॑द्ध्व॒र म॑द्ध्व॒र मि॒म मि॒म म॑द्ध्व॒रम् । \newline
11. अ॒द्ध्व॒रम् कृ॑धि कृध्यद्ध्व॒र म॑द्ध्व॒रम् कृ॑धि । \newline
12. कृ॒धि॒ दि॒वि दि॒वि कृ॑धि कृधि दि॒वि । \newline
13. दि॒वि दे॒वेषु॑ दे॒वेषु॑ दि॒वि दि॒वि दे॒वेषु॑ । \newline
14. दे॒वेषु॒ होत्रा॒ होत्रा॑ दे॒वेषु॑ दे॒वेषु॒ होत्राः᳚ । \newline
15. होत्रा॑ यच्छ यच्छ॒ होत्रा॒ होत्रा॑ यच्छ । \newline
16. य॒च्छ॒ सोम॒ सोम॑ यच्छ यच्छ॒ सोम॑ । \newline
17. सोम॑ राजन् राज॒न् थ्सोम॒ सोम॑ राजन्न् । \newline
18. रा॒ज॒न् ना रा॑जन् राज॒न् ना । \newline
19. एही॒ह्येहि॑ । \newline
20. इ॒ह्यवावे॑ ही॒ह्यव॑ । \newline
21. अव॑ रोह रो॒हावाव॑ रोह । \newline
22. रो॒ह॒ मा मा रो॑ह रोह॒ मा । \newline
23. मा भेर् भेर् मा मा भेः । \newline
24. भेर् मा मा भेर् भेर् मा । \newline
25. मा सꣳ सम् मा मा सम् । \newline
26. सं ॅवि॑क्था विक्थाः॒ सꣳ सं ॅवि॑क्थाः । \newline
27. वि॒क्था॒ मा मा वि॑क्था विक्था॒ मा । \newline
28. मा त्वा᳚ त्वा॒ मा मा त्वा᳚ । \newline
29. त्वा॒ हि॒(ग्म्॒)सि॒ष॒(ग्म्॒) हि॒(ग्म्॒)सि॒ष॒म् त्वा॒ त्वा॒ हि॒(ग्म्॒)सि॒ष॒म् । \newline
30. हि॒(ग्म्॒)सि॒ष॒म् प्र॒जाः प्र॒जा हि(ग्म्॑)सिषꣳ हिꣳसिषम् प्र॒जाः । \newline
31. प्र॒जास्त्वम् त्वम् प्र॒जाः प्र॒जास्त्वम् । \newline
32. प्र॒जा इति॑ प्र - जाः । \newline
33. त्व मु॒पाव॑रो हो॒पाव॑रोह॒ त्वम् त्व मु॒पाव॑रोह । \newline
34. उ॒पाव॑रोह प्र॒जाः प्र॒जा उ॒पाव॑रो हो॒पाव॑रोह प्र॒जाः । \newline
35. उ॒पाव॑रो॒हेत्यु॑प - अव॑रोह । \newline
36. प्र॒जा स्त्वाम् त्वाम् प्र॒जाः प्र॒जा स्त्वाम् । \newline
37. प्र॒जा इति॑ प्र - जाः । \newline
38. त्वा मु॒पाव॑रोहन्तू॒ पाव॑रोहन्तु॒ त्वाम् त्वा मु॒पाव॑रोहन्तु । \newline
39. उ॒पाव॑रोहन्तु शृ॒णोतु॑ शृ॒णोतू॒ पाव॑रोहन्तू॒ पाव॑रोहन्तु शृ॒णोतु॑ । \newline
40. उ॒पाव॑रोह॒न्त्वित्यु॑प - अव॑रोहन्तु । \newline
41. शृ॒णो त्व॒ग्नि र॒ग्निः शृ॒णोतु॑ शृ॒णो त्व॒ग्निः । \newline
42. अ॒ग्निः स॒मिधा॑ स॒मिधा॒ ऽग्नि र॒ग्निः स॒मिधा᳚ । \newline
43. स॒मिधा॒ हव॒(ग्म्॒) हव(ग्म्॑) स॒मिधा॑ स॒मिधा॒ हव᳚म् । \newline
44. स॒मिधेति॑ सम् - इधा᳚ । \newline
45. हव॑म् मे मे॒ हव॒(ग्म्॒) हव॑म् मे । \newline
46. मे॒ शृ॒ण्वन्तु॑ शृ॒ण्वन्तु॑ मे मे शृ॒ण्वन्तु॑ । \newline
47. शृ॒ण्वन्त्वाप॒ आपः॑ शृ॒ण्वन्तु॑ शृ॒ण्वन्त्वापः॑ । \newline
48. आपो॑ धि॒षणा॑ धि॒षणा॒ आप॒ आपो॑ धि॒षणाः᳚ । \newline
49. धि॒षणा᳚ श्च च धि॒षणा॑ धि॒षणा᳚ श्च । \newline
50. च॒ दे॒वीर् दे॒वी श्च॑ च दे॒वीः । \newline
51. दे॒वीरिति॑ दे॒वीः । \newline
52. शृ॒णोत॑ ग्रावाणो ग्रावाणः शृ॒णोत॑ शृ॒णोत॑ ग्रावाणः । \newline
53. ग्रा॒वा॒णो॒ वि॒दुषो॑ वि॒दुषो᳚ ग्रावाणो ग्रावाणो वि॒दुषः॑ । \newline
54. वि॒दुषो॒ नु नु वि॒दुषो॑ वि॒दुषो॒ नु । \newline
55. नु य॒ज्ञ्ं ॅय॒ज्ञ्म् नु नु य॒ज्ञ्म् । \newline

\textbf{Ghana Paata } \newline

1. हृ॒दे त्वा᳚ त्वा हृ॒दे हृ॒दे त्वा॒ मन॑से॒ मन॑से त्वा हृ॒दे हृ॒दे त्वा॒ मन॑से । \newline
2. त्वा॒ मन॑से॒ मन॑से त्वा त्वा॒ मन॑से त्वा त्वा॒ मन॑से त्वा त्वा॒ मन॑से त्वा । \newline
3. मन॑से त्वा त्वा॒ मन॑से॒ मन॑से त्वा दि॒वे दि॒वे त्वा॒ मन॑से॒ मन॑से त्वा दि॒वे । \newline
4. त्वा॒ दि॒वे दि॒वे त्वा᳚ त्वा दि॒वे त्वा᳚ त्वा दि॒वे त्वा᳚ त्वा दि॒वे त्वा᳚ । \newline
5. दि॒वे त्वा᳚ त्वा दि॒वे दि॒वे त्वा॒ सूर्या॑य॒ सूर्या॑य त्वा दि॒वे दि॒वे त्वा॒ सूर्या॑य । \newline
6. त्वा॒ सूर्या॑य॒ सूर्या॑य त्वा त्वा॒ सूर्या॑य त्वा त्वा॒ सूर्या॑य त्वा त्वा॒ सूर्या॑य त्वा । \newline
7. सूर्या॑य त्वा त्वा॒ सूर्या॑य॒ सूर्या॑य त्वो॒र्द्ध्व मू॒र्द्ध्वम् त्वा॒ सूर्या॑य॒ सूर्या॑य त्वो॒र्द्ध्वम् । \newline
8. त्वो॒र्द्ध्व मू॒र्द्ध्वम् त्वा᳚ त्वो॒र्द्ध्व मि॒म मि॒म मू॒र्द्ध्वम् त्वा᳚ त्वो॒र्द्ध्व मि॒मम् । \newline
9. ऊ॒र्द्ध्व मि॒म मि॒म मू॒र्द्ध्व मू॒र्द्ध्व मि॒म म॑द्ध्व॒र म॑द्ध्व॒र मि॒म मू॒र्द्ध्व मू॒र्द्ध्व मि॒म म॑द्ध्व॒रम् । \newline
10. इ॒म म॑द्ध्व॒र म॑द्ध्व॒र मि॒म मि॒म म॑द्ध्व॒रम् कृ॑धि कृध्यद्ध्व॒र मि॒म मि॒म म॑द्ध्व॒रम् कृ॑धि । \newline
11. अ॒द्ध्व॒रम् कृ॑धि कृध्यद्ध्व॒र म॑द्ध्व॒रम् कृ॑धि दि॒वि दि॒वि कृ॑ध्यद्ध्व॒र म॑द्ध्व॒रम् कृ॑धि दि॒वि । \newline
12. कृ॒धि॒ दि॒वि दि॒वि कृ॑धि कृधि दि॒वि दे॒वेषु॑ दे॒वेषु॑ दि॒वि कृ॑धि कृधि दि॒वि दे॒वेषु॑ । \newline
13. दि॒वि दे॒वेषु॑ दे॒वेषु॑ दि॒वि दि॒वि दे॒वेषु॒ होत्रा॒ होत्रा॑ दे॒वेषु॑ दि॒वि दि॒वि दे॒वेषु॒ होत्राः᳚ । \newline
14. दे॒वेषु॒ होत्रा॒ होत्रा॑ दे॒वेषु॑ दे॒वेषु॒ होत्रा॑ यच्छ यच्छ॒ होत्रा॑ दे॒वेषु॑ दे॒वेषु॒ होत्रा॑ यच्छ । \newline
15. होत्रा॑ यच्छ यच्छ॒ होत्रा॒ होत्रा॑ यच्छ॒ सोम॒ सोम॑ यच्छ॒ होत्रा॒ होत्रा॑ यच्छ॒ सोम॑ । \newline
16. य॒च्छ॒ सोम॒ सोम॑ यच्छ यच्छ॒ सोम॑ राजन् राज॒न् थ्सोम॑ यच्छ यच्छ॒ सोम॑ राजन्न् । \newline
17. सोम॑ राजन् राज॒न् थ्सोम॒ सोम॑ राज॒न् ना रा॑ज॒न् थ्सोम॒ सोम॑ राज॒न् ना । \newline
18. रा॒ज॒न् ना रा॑जन् राज॒न् नेही॒ह्या रा॑जन् राज॒न् नेहि॑ । \newline
19. एही॒ह्येह्यवावे॒ ह्येह्यव॑ । \newline
20. इ॒ह्यवावे॑ ही॒ह्यव॑ रोह रो॒हावे॑ ही॒ह्यव॑ रोह । \newline
21. अव॑ रोह रो॒हावाव॑ रोह॒ मा मा रो॒हावाव॑ रोह॒ मा । \newline
22. रो॒ह॒ मा मा रो॑ह रोह॒ मा भेर् भेर् मा रो॑ह रोह॒ मा भेः । \newline
23. मा भेर् भेर् मा मा भेर् मा मा भेर् मा मा भेर् मा । \newline
24. भेर् मा मा भेर् भेर् मा सꣳ सम् मा भेर् भेर् मा सम् । \newline
25. मा सꣳ सम् मा मा सं ॅवि॑क्था विक्थाः॒ सम् मा मा सं ॅवि॑क्थाः । \newline
26. सं ॅवि॑क्था विक्थाः॒ सꣳ सं ॅवि॑क्था॒ मा मा वि॑क्थाः॒ सꣳ सं ॅवि॑क्था॒ मा । \newline
27. वि॒क्था॒ मा मा वि॑क्था विक्था॒ मा त्वा᳚ त्वा॒ मा वि॑क्था विक्था॒ मा त्वा᳚ । \newline
28. मा त्वा᳚ त्वा॒ मा मा त्वा॑ हिꣳसिषꣳ हिꣳसिषम् त्वा॒ मा मा त्वा॑ हिꣳसिषम् । \newline
29. त्वा॒ हि॒(ग्म्॒)सि॒ष॒(ग्म्॒) हि॒(ग्म्॒)सि॒ष॒म् त्वा॒ त्वा॒ हि॒(ग्म्॒)सि॒ष॒म् प्र॒जाः प्र॒जा हि(ग्म्॑)सिषम् त्वा त्वा हिꣳसिषम् प्र॒जाः । \newline
30. हि॒(ग्म्॒)सि॒ष॒म् प्र॒जाः प्र॒जा हि(ग्म्॑)सिषꣳ हिꣳसिषम् प्र॒जास्त्वम् त्वम् प्र॒जा हि(ग्म्॑)सिषꣳ हिꣳसिषम् प्र॒जास्त्वम् । \newline
31. प्र॒जास्त्वम् त्वम् प्र॒जाः प्र॒जास्त्व मु॒पा व॑रोहो॒पा व॑रोह॒ त्वम् प्र॒जाः प्र॒जास्त्व मु॒पाव॑रोह । \newline
32. प्र॒जा इति॑ प्र - जाः । \newline
33. त्व मु॒पाव॑ रोहो॒पाव॑रोह॒ त्वम् त्व मु॒पाव॑रोह प्र॒जाः प्र॒जा उ॒पाव॑रोह॒ त्वम् त्व मु॒पाव॑रोह प्र॒जाः । \newline
34. उ॒पाव॑रोह प्र॒जाः प्र॒जा उ॒पाव॑ रोहो॒पा व॑रोह प्र॒जास्त्वाम् त्वाम् प्र॒जा उ॒पा व॑रोहो॒पा व॑रोह प्र॒जास्त्वाम् । \newline
35. उ॒पाव॑रो॒हेत्यु॑प - अव॑रोह । \newline
36. प्र॒जा स्त्वाम् त्वाम् प्र॒जाः प्र॒जा स्त्वा मु॒पाव॑रोहन्तू॑ पाव॑रोहन्तु॒ त्वाम् प्र॒जाः प्र॒जास्त्वा मु॒पाव॑रोहन्तु । \newline
37. प्र॒जा इति॑ प्र - जाः । \newline
38. त्वा मु॒पाव॑रोहन्तू॒ पाव॑रोहन्तु॒ त्वाम् त्वा मु॒पाव॑रोहन्तु शृ॒णोतु॑ शृ॒णोतू॒ पाव॑रोहन्तु॒ त्वाम् त्वा मु॒पाव॑रोहन्तु शृ॒णोतु॑ । \newline
39. उ॒पाव॑रोहन्तु शृ॒णोतु॑ शृ॒णोतू॒ पाव॑रोहन्तू॒ पाव॑रोहन्तु शृ॒णोत्व॒ग्निर॒ग्निः शृ॒णोतू॒ पाव॑रोहन्तू॒ पाव॑रोहन्तु शृ॒णोत्व॒ग्निः । \newline
40. उ॒पाव॑रोह॒न्त्वित्यु॑प - अव॑रोहन्तु । \newline
41. शृ॒णोत्व॒ग्नि र॒ग्निः शृ॒णोतु॑ शृ॒णोत्व॒ग्निः स॒मिधा॑ स॒मिधा॒ ऽग्निः शृ॒णोतु॑ शृ॒णोत्व॒ग्निः स॒मिधा᳚ । \newline
42. अ॒ग्निः स॒मिधा॑ स॒मिधा॒ ऽग्निर॒ग्निः स॒मिधा॒ हव॒(ग्म्॒) हव(ग्म्॑) स॒मिधा॒ ऽग्निर॒ग्निः स॒मिधा॒ हव᳚म् । \newline
43. स॒मिधा॒ हव॒(ग्म्॒) हव(ग्म्॑) स॒मिधा॑ स॒मिधा॒ हव॑म् मे मे॒ हव(ग्म्॑) स॒मिधा॑ स॒मिधा॒ हव॑म् मे । \newline
44. स॒मिधेति॑ सम् - इधा᳚ । \newline
45. हव॑म् मे मे॒ हव॒(ग्म्॒) हव॑म् मे शृ॒ण्वन्तु॑ शृ॒ण्वन्तु॑ मे॒ हव॒(ग्म्॒) हव॑म् मे शृ॒ण्वन्तु॑ । \newline
46. मे॒ शृ॒ण्वन्तु॑ शृ॒ण्वन्तु॑ मे मे शृ॒ण्वन्त्वाप॒ आपः॑ शृ॒ण्वन्तु॑ मे मे शृ॒ण्वन्त्वापः॑ । \newline
47. शृ॒ण्वन्त्वाप॒ आपः॑ शृ॒ण्वन्तु॑ शृ॒ण्वन्त्वापो॑ धि॒षणा॑ धि॒षणा॒ आपः॑ शृ॒ण्वन्तु॑ शृ॒ण्वन्त्वापो॑ धि॒षणाः᳚ । \newline
48. आपो॑ धि॒षणा॑ धि॒षणा॒ आप॒ आपो॑ धि॒षणा᳚श्च च धि॒षणा॒ आप॒ आपो॑ धि॒षणा᳚श्च । \newline
49. धि॒षणा᳚श्च च धि॒षणा॑ धि॒षणा᳚श्च दे॒वीर् दे॒वीश्च॑ धि॒षणा॑ धि॒षणा᳚श्च दे॒वीः । \newline
50. च॒ दे॒वीर् दे॒वीश्च॑ च दे॒वीः । \newline
51. दे॒वीरिति॑ दे॒वीः । \newline
52. शृ॒णोत॑ ग्रावाणो ग्रावाणः शृ॒णोत॑ शृ॒णोत॑ ग्रावाणो वि॒दुषो॑ वि॒दुषो᳚ ग्रावाणः शृ॒णोत॑ शृ॒णोत॑ ग्रावाणो वि॒दुषः॑ । \newline
53. ग्रा॒वा॒णो॒ वि॒दुषो॑ वि॒दुषो᳚ ग्रावाणो ग्रावाणो वि॒दुषो॒ नु नु वि॒दुषो᳚ ग्रावाणो ग्रावाणो वि॒दुषो॒ नु । \newline
54. वि॒दुषो॒ नु नु वि॒दुषो॑ वि॒दुषो॒ नु य॒ज्ञ्ं ॅय॒ज्ञ्म् नु वि॒दुषो॑ वि॒दुषो॒ नु य॒ज्ञ्म् । \newline
55. नु य॒ज्ञ्ं ॅय॒ज्ञ्म् नु नु य॒ज्ञ्ꣳ शृ॒णोतु॑ शृ॒णोतु॑ य॒ज्ञ्म् नु नु य॒ज्ञ्ꣳ शृ॒णोतु॑ । \newline
\pagebreak
\markright{ TS 1.3.13.2  \hfill https://www.vedavms.in \hfill}

\section{ TS 1.3.13.2 }

\textbf{TS 1.3.13.2 } \newline
\textbf{Samhita Paata} \newline

य॒ज्ञ्ꣳ शृ॒णोतु॑ दे॒वः स॑वि॒ता हवं॑ मे । देवी॑रापो अपां नपा॒द्य ऊ॒र्मिर्ह॑वि॒ष्य॑ इन्द्रि॒यावा᳚न् म॒दिन्त॑म॒स्तं दे॒वेभ्यो॑ देव॒त्रा ध॑त्त शु॒क्रꣳ शु॑क्र॒पेभ्यो॒ येषां᳚ भा॒गः स्थ स्वाहा॒ कार्.षि॑र॒स्यपा॒ऽपां मृ॒द्ध्रꣳ स॑मु॒द्रस्य॒ वोऽक्षि॑त्या॒ उन्न॑ये । यम॑ग्ने पृ॒थ्सु मर्त्य॒मावो॒ वाजे॑षु॒ यं जु॒नाः । स यन्ता॒ शश्व॑ती॒रिषः॑ ॥ \newline

\textbf{Pada Paata} \newline

य॒ज्ञ्म् । शृ॒णोतु॑ । दे॒वः । स॒वि॒ता । हव᳚म् । मे॒ ॥ देवीः᳚ । आ॒पः॒ । अ॒पा॒म् । न॒पा॒त् । यः । ऊ॒र्मिः । ह॒वि॒ष्यः॑ । इ॒न्द्रि॒यावा॒निती᳚न्द्रि॒य - वा॒न् । म॒दिन्त॑मः । तम् । दे॒वेभ्यः॑ । दे॒व॒त्रेति॑ देव - त्रा । ध॒त्त॒ । शु॒क्रम् । शु॒क्र॒पेभ्य॒ इति॑ शुक्र - पेभ्यः॑ । येषा᳚म् । भा॒गः । स्थ । स्वाहा᳚ । कार्.षिः॑ । अ॒सि॒ । अपेति॑ । अ॒पाम् । मृ॒द्ध्रम् । स॒मु॒द्रस्य॑ । वः॒ । अक्षि॑त्यै । उदिति॑ । न॒ये॒ ॥ यम् । अ॒ग्ने॒ । पृ॒थ्स्विति॑ पृ॒त् - सु । मर्त्य᳚म् । आवः॑ । वाजे॑षु । यम् । जु॒नाः ॥ सः । यन्ता᳚ । शश्व॑तीः । इषः॑ ॥  \newline


\textbf{Krama Paata} \newline

य॒ज्ञ्ꣳ शृ॒णोतु॑ । शृ॒णोतु॑ दे॒वः । दे॒वः स॑वि॒ता । स॒वि॒ता हव᳚म् । हव॑म् मे । म॒ इति॑ मे ॥ देवी॑रापः । आ॒पो॒ अ॒पा॒म् । अ॒पा॒म् न॒पा॒त्॒ । न॒पा॒द् यः । य ऊ॒र्मिः । ऊ॒र्मिर्. ह॑वि॒ष्यः॑ । ह॒वि॒ष्य॑ इन्द्रि॒यावान्॑ । इ॒न्द्रि॒यावा᳚न् म॒दिन्त॑मः । इ॒न्द्रि॒यावा॒निती᳚न्द्रि॒य - वा॒न्॒ । म॒दिन्त॑म॒स्तम् । तम् दे॒वेभ्यः॑ । दे॒वेभ्यो॑ देव॒त्रा । दे॒व॒त्रा ध॑त्त । दे॒व॒त्रेति॑ देव - त्रा । ध॒त्त॒ शु॒क्रम् । शु॒क्रꣳ शु॑क्र॒पेभ्यः॑ । शु॒क्र॒पेभ्यो॒ येषा᳚म् । शु॒क्र॒पेभ्य॒ इति॑ शुक्र - पेभ्यः॑ । येषा᳚म् भा॒गः । भा॒गः स्थ । स्थ स्वाहा᳚ । स्वाहा॒ कार्.षिः॑ । कार्.षि॑रसि । अ॒स्यप॑ । अपा॒पाम् । अ॒पाम् मृ॒द्ध्रम् । मृ॒द्ध्रꣳ स॑मु॒द्रस्य॑ । स॒मु॒द्रस्य॑ वः । वोऽक्षि॑त्यै । अक्षि॑त्या॒ उत् । उन्न॑ये । न॒य॒ इति॑ नये ॥ यम॑ग्ने । अ॒ग्ने॒ पृ॒थ्सु । पृ॒थ्सु मर्त्य᳚म् । पृ॒थ्स्विति॑ पृत् - सु । मर्त्य॒मावः॑ । आवो॒ वाजे॑षु । वाजे॑षु॒ यम् । यम् जु॒नाः । जु॒ना इति॑ जु॒नाः ॥ स यन्ता᳚ । यन्ता॒ शश्व॑तीः । शश्व॑ती॒रिषः॑ । इष॒ इतीषः॑ । \newline

\textbf{Jatai Paata} \newline

1. य॒ज्ञ्ꣳ शृ॒णोतु॑ शृ॒णोतु॑ य॒ज्ञ्ं ॅय॒ज्ञ्ꣳ शृ॒णोतु॑ । \newline
2. शृ॒णोतु॑ दे॒वो दे॒वः शृ॒णोतु॑ शृ॒णोतु॑ दे॒वः । \newline
3. दे॒वः स॑वि॒ता स॑वि॒ता दे॒वो दे॒वः स॑वि॒ता । \newline
4. स॒वि॒ता हव॒(ग्म्॒) हव(ग्म्॑) सवि॒ता स॑वि॒ता हव᳚म् । \newline
5. हव॑म् मे मे॒ हव॒(ग्म्॒) हव॑म् मे । \newline
6. म॒ इति॑ मे । \newline
7. देवी॑राप आपो॒ देवी॒र् देवी॑रापः । \newline
8. आ॒पो॒ अ॒पा॒ म॒पा॒ मा॒प॒ आ॒पो॒ अ॒पा॒म् । \newline
9. अ॒पा॒म् न॒पा॒न् न॒पा॒द॒पा॒ म॒पा॒म् न॒पा॒त् । \newline
10. न॒पा॒द् यो यो न॑पान् नपा॒द् यः । \newline
11. य ऊ॒र्मि रू॒र्मिर् यो य ऊ॒र्मिः । \newline
12. ऊ॒र्मिर्. ह॑वि॒ष्यो॑ हवि॒ष्य॑ ऊ॒र्मि रू॒र्मिर्. ह॑वि॒ष्यः॑ । \newline
13. ह॒वि॒ष्य॑ इन्द्रि॒यावा॑ निन्द्रि॒यावा॑न्. हवि॒ष्यो॑ हवि॒ष्य॑ इन्द्रि॒यावान्॑ । \newline
14. इ॒न्द्रि॒यावा᳚न् म॒दिन्त॑मो म॒दिन्त॑म इन्द्रि॒यावा॑ निन्द्रि॒यावा᳚न् म॒दिन्त॑मः । \newline
15. इ॒न्द्रि॒यावा॒निती᳚न्द्रि॒य - वा॒न् । \newline
16. म॒दिन्त॑म॒ स्तम् तम् म॒दिन्त॑मो म॒दिन्त॑म॒ स्तम् । \newline
17. तम् दे॒वेभ्यो॑ दे॒वेभ्य॒ स्तम् तम् दे॒वेभ्यः॑ । \newline
18. दे॒वेभ्यो॑ देव॒त्रा दे॑व॒त्रा दे॒वेभ्यो॑ दे॒वेभ्यो॑ देव॒त्रा । \newline
19. दे॒व॒त्रा ध॑त्त धत्त देव॒त्रा दे॑व॒त्रा ध॑त्त । \newline
20. दे॒व॒त्रेति॑ देव - त्रा । \newline
21. ध॒त्त॒ शु॒क्रꣳ शु॒क्रम् ध॑त्त धत्त शु॒क्रम् । \newline
22. शु॒क्रꣳ शु॑क्र॒पेभ्यः॑ शुक्र॒पेभ्यः॑ शु॒क्रꣳ शु॒क्रꣳ शु॑क्र॒पेभ्यः॑ । \newline
23. शु॒क्र॒पेभ्यो॒ येषां॒ ॅयेषा(ग्म्॑) शुक्र॒पेभ्यः॑ शुक्र॒पेभ्यो॒ येषा᳚म् । \newline
24. शु॒क्र॒पेभ्य॒ इति॑ शुक्र - पेभ्यः॑ । \newline
25. येषा᳚म् भा॒गो भा॒गो येषां॒ ॅयेषा᳚म् भा॒गः । \newline
26. भा॒गः स्थ स्थ भा॒गो भा॒गः स्थ । \newline
27. स्थ स्वाहा॒ स्वाहा॒ स्थ स्थ स्वाहा᳚ । \newline
28. स्वाहा॒ कार्.षिः॒ कार्.षिः॒ स्वाहा॒ स्वाहा॒ कार्.षिः॑ । \newline
29. कार्.षि॑ रस्यसि॒ कार्.षिः॒ कार्.षि॑ रसि । \newline
30. अ॒स्य पापा᳚ स्य॒स्यप॑ । \newline
31. अपा॒पा म॒पा मपापा॒ पाम् । \newline
32. अ॒पाम् मृ॒द्ध्रम् मृ॒द्ध्र म॒पा म॒पाम् मृ॒द्ध्रम् । \newline
33. मृ॒द्ध्रꣳ स॑मु॒द्रस्य॑ समु॒द्रस्य॑ मृ॒द्ध्रम् मृ॒द्ध्रꣳ स॑मु॒द्रस्य॑ । \newline
34. स॒मु॒द्रस्य॑ वो वः समु॒द्रस्य॑ समु॒द्रस्य॑ वः । \newline
35. वो ऽक्षि॑त्या॒ अक्षि॑त्यै वो॒ वो ऽक्षि॑त्यै । \newline
36. अक्षि॑त्या॒ उदु दक्षि॑त्या॒ अक्षि॑त्या॒ उत् । \newline
37. उन् न॑ये नय॒ उदुन् न॑ये । \newline
38. न॒य॒ इति॑ नये । \newline
39. य म॑ग्ने ऽग्ने॒ यं ॅय म॑ग्ने । \newline
40. अ॒ग्ने॒ पृ॒थ्सु पृ॒थ्स्व॑ग्ने ऽग्ने पृ॒थ्सु । \newline
41. पृ॒थ्सु मर्त्य॒म् मर्त्य॑म् पृ॒थ्सु पृ॒थ्सु मर्त्य᳚म् । \newline
42. पृ॒थ्स्विति॑ पृ॒त् - सु । \newline
43. मर्त्य॒ माव॒ आवो॒ मर्त्य॒म् मर्त्य॒ मावः॑ । \newline
44. आवो॒ वाजे॑षु॒ वाजे॒ष्वाव॒ आवो॒ वाजे॑षु । \newline
45. वाजे॑षु॒ यं ॅयं ॅवाजे॑षु॒ वाजे॑षु॒ यम् । \newline
46. यम् जु॒ना जु॒ना यं ॅयम् जु॒नाः । \newline
47. जु॒ना इति॑ जु॒नाः । \newline
48. स यन्ता॒ यन्ता॒ स स यन्ता᳚ । \newline
49. यन्ता॒ शश्व॑तीः॒ शश्व॑ती॒र् यन्ता॒ यन्ता॒ शश्व॑तीः । \newline
50. शश्व॑ती॒ रिष॒ इषः॒ शश्व॑तीः॒ शश्व॑ती॒ रिषः॑ । \newline
51. इष॒ इतीषः॑ । \newline

\textbf{Ghana Paata } \newline

1. य॒ज्ञ्ꣳ शृ॒णोतु॑ शृ॒णोतु॑ य॒ज्ञ्ं ॅय॒ज्ञ्ꣳ शृ॒णोतु॑ दे॒वो दे॒वः शृ॒णोतु॑ य॒ज्ञ्ं ॅय॒ज्ञ्ꣳ शृ॒णोतु॑ दे॒वः । \newline
2. शृ॒णोतु॑ दे॒वो दे॒वः शृ॒णोतु॑ शृ॒णोतु॑ दे॒वः स॑वि॒ता स॑वि॒ता दे॒वः शृ॒णोतु॑ शृ॒णोतु॑ दे॒वः स॑वि॒ता । \newline
3. दे॒वः स॑वि॒ता स॑वि॒ता दे॒वो दे॒वः स॑वि॒ता हव॒(ग्म्॒) हव(ग्म्॑) सवि॒ता दे॒वो दे॒वः स॑वि॒ता हव᳚म् । \newline
4. स॒वि॒ता हव॒(ग्म्॒) हव(ग्म्॑) सवि॒ता स॑वि॒ता हव॑म् मे मे॒ हव(ग्म्॑) सवि॒ता स॑वि॒ता हव॑म् मे । \newline
5. हव॑म् मे मे॒ हव॒(ग्म्॒) हव॑म् मे । \newline
6. म॒ इति॑ मे । \newline
7. देवी॑राप आपो॒ देवी॒र् देवी॑ रापो अपा मपा मापो॒ देवी॒र् देवी॑ रापो अपाम् । \newline
8. आ॒पो॒ अ॒पा॒ म॒पा॒ मा॒प॒ आ॒पो॒ अ॒पा॒म् न॒पा॒न् न॒पा॒द॒पा॒ मा॒प॒ आ॒पो॒ अ॒पा॒म् न॒पा॒त् । \newline
9. अ॒पा॒म् न॒पा॒न् न॒पा॒द॒पा॒ म॒पा॒म् न॒पा॒द् यो यो न॑पादपा मपाम् नपा॒द् यः । \newline
10. न॒पा॒द् यो यो न॑पान् नपा॒द् य ऊ॒र्मि रू॒र्मिर् यो न॑पान् नपा॒द् य ऊ॒र्मिः । \newline
11. य ऊ॒र्मि रू॒र्मिर् यो य ऊ॒र्मिर्. ह॑वि॒ष्यो॑ हवि॒ष्य॑ ऊ॒र्मिर् यो य ऊ॒र्मिर्. ह॑वि॒ष्यः॑ । \newline
12. ऊ॒र्मिर्. ह॑वि॒ष्यो॑ हवि॒ष्य॑ ऊ॒र्मि रू॒र्मिर्. ह॑वि॒ष्य॑ इन्द्रि॒यावा॑ निन्द्रि॒यावान्॑. हवि॒ष्य॑ ऊ॒र्मि रू॒र्मिर्. ह॑वि॒ष्य॑ इन्द्रि॒यावान्॑ । \newline
13. ह॒वि॒ष्य॑ इन्द्रि॒यावा॑ निन्द्रि॒यावान्॑. हवि॒ष्यो॑ हवि॒ष्य॑ इन्द्रि॒यावा᳚न् म॒दिन्त॑मो म॒दिन्त॑म इन्द्रि॒यावान्॑. हवि॒ष्यो॑ हवि॒ष्य॑ इन्द्रि॒यावा᳚न् म॒दिन्त॑मः । \newline
14. इ॒न्द्रि॒यावा᳚न् म॒दिन्त॑मो म॒दिन्त॑म इन्द्रि॒यावा॑ निन्द्रि॒यावा᳚न् म॒दिन्त॑म॒ स्तम् तम् म॒दिन्त॑म इन्द्रि॒यावा॑ निन्द्रि॒यावा᳚न् म॒दिन्त॑म॒ स्तम् । \newline
15. इ॒न्द्रि॒यावा॒निती᳚न्द्रि॒य - वा॒न् । \newline
16. म॒दिन्त॑म॒ स्तम् तम् म॒दिन्त॑मो म॒दिन्त॑म॒ स्तम् दे॒वेभ्यो॑ दे॒वेभ्य॒ स्तम् म॒दिन्त॑मो म॒दिन्त॑म॒ स्तम् दे॒वेभ्यः॑ । \newline
17. तम् दे॒वेभ्यो॑ दे॒वेभ्य॒ स्तम् तम् दे॒वेभ्यो॑ देव॒त्रा दे॑व॒त्रा दे॒वेभ्य॒ स्तम् तम् दे॒वेभ्यो॑ देव॒त्रा । \newline
18. दे॒वेभ्यो॑ देव॒त्रा दे॑व॒त्रा दे॒वेभ्यो॑ दे॒वेभ्यो॑ देव॒त्रा ध॑त्त धत्त देव॒त्रा दे॒वेभ्यो॑ दे॒वेभ्यो॑ देव॒त्रा ध॑त्त । \newline
19. दे॒व॒त्रा ध॑त्त धत्त देव॒त्रा दे॑व॒त्रा ध॑त्त शु॒क्रꣳ शु॒क्रम् ध॑त्त देव॒त्रा दे॑व॒त्रा ध॑त्त शु॒क्रम् । \newline
20. दे॒व॒त्रेति॑ देव - त्रा । \newline
21. ध॒त्त॒ शु॒क्रꣳ शु॒क्रम् ध॑त्त धत्त शु॒क्रꣳ शु॑क्र॒पेभ्यः॑ शुक्र॒पेभ्यः॑ शु॒क्रम् ध॑त्त धत्त शु॒क्रꣳ शु॑क्र॒पेभ्यः॑ । \newline
22. शु॒क्रꣳ शु॑क्र॒पेभ्यः॑ शुक्र॒पेभ्यः॑ शु॒क्रꣳ शु॒क्रꣳ शु॑क्र॒पेभ्यो॒ येषां॒ ॅयेषा(ग्म्॑) शुक्र॒पेभ्यः॑ शु॒क्रꣳ शु॒क्रꣳ शु॑क्र॒पेभ्यो॒ येषा᳚म् । \newline
23. शु॒क्र॒पेभ्यो॒ येषां॒ ॅयेषा(ग्म्॑) शुक्र॒पेभ्यः॑ शुक्र॒पेभ्यो॒ येषा᳚म् भा॒गो भा॒गो येषा(ग्म्॑) शुक्र॒पेभ्यः॑ शुक्र॒पेभ्यो॒ येषा᳚म् भा॒गः । \newline
24. शु॒क्र॒पेभ्य॒ इति॑ शुक्र - पेभ्यः॑ । \newline
25. येषा᳚म् भा॒गो भा॒गो येषां॒ ॅयेषा᳚म् भा॒गः स्थ स्थ भा॒गो येषां॒ ॅयेषा᳚म् भा॒गः स्थ । \newline
26. भा॒गः स्थ स्थ भा॒गो भा॒गः स्थ स्वाहा॒ स्वाहा॒ स्थ भा॒गो भा॒गः स्थ स्वाहा᳚ । \newline
27. स्थ स्वाहा॒ स्वाहा॒ स्थ स्थ स्वाहा॒ कार्.षिः॒ कार्.षिः॒ स्वाहा॒ स्थ स्थ स्वाहा॒ कार्.षिः॑ । \newline
28. स्वाहा॒ कार्.षिः॒ कार्.षिः॒ स्वाहा॒ स्वाहा॒ कार्.षि॑ रस्यसि॒ कार्.षिः॒ स्वाहा॒ स्वाहा॒ कार्.षि॑रसि । \newline
29. कार्.षि॑ रस्यसि॒ कार्.षिः॒ कार्.षि॑ र॒स्यपापा॑सि॒ कार्.षिः॒ कार्.षि॑ र॒स्यप॑ । \newline
30. अ॒स्यपापा᳚ स्य॒स्यपा॒पा म॒पा मपा᳚स्य॒ स्यपा॒पाम् । \newline
31. अपा॒पा म॒पा मपापा॒पाम् मृ॒द्ध्रम् मृ॒द्ध्र म॒पा मपापा॒पाम् मृ॒द्ध्रम् । \newline
32. अ॒पाम् मृ॒द्ध्रम् मृ॒द्ध्र म॒पा म॒पाम् मृ॒द्ध्रꣳ स॑मु॒द्रस्य॑ समु॒द्रस्य॑ मृ॒द्ध्र म॒पा म॒पाम् मृ॒द्ध्रꣳ स॑मु॒द्रस्य॑ । \newline
33. मृ॒द्ध्रꣳ स॑मु॒द्रस्य॑ समु॒द्रस्य॑ मृ॒द्ध्रम् मृ॒द्ध्रꣳ स॑मु॒द्रस्य॑ वो वः समु॒द्रस्य॑ मृ॒द्ध्रम् मृ॒द्ध्रꣳ स॑मु॒द्रस्य॑ वः । \newline
34. स॒मु॒द्रस्य॑ वो वः समु॒द्रस्य॑ समु॒द्रस्य॒ वो ऽक्षि॑त्या॒ अक्षि॑त्यै वः समु॒द्रस्य॑ समु॒द्रस्य॒ वो ऽक्षि॑त्यै । \newline
35. वो ऽक्षि॑त्या॒ अक्षि॑त्यै वो॒ वो ऽक्षि॑त्या॒ उदु दक्षि॑त्यै वो॒ वो ऽक्षि॑त्या॒ उत् । \newline
36. अक्षि॑त्या॒ उदु दक्षि॑त्या॒ अक्षि॑त्या॒ उन् न॑ये नय॒ उदक्षि॑त्या॒ अक्षि॑त्या॒ उन् न॑ये । \newline
37. उन् न॑ये नय॒ उदुन् न॑ये । \newline
38. न॒य॒ इति॑ नये । \newline
39. य म॑ग्ने ऽग्ने॒ यं ॅय म॑ग्ने पृ॒थ्सु पृ॒थ्स्व॑ग्ने॒ यं ॅय म॑ग्ने पृ॒थ्सु । \newline
40. अ॒ग्ने॒ पृ॒थ्सु पृ॒थ्स्व॑ग्ने ऽग्ने पृ॒थ्सु मर्त्य॒म् मर्त्य॑म् पृ॒थ्स्व॑ग्ने ऽग्ने पृ॒थ्सु मर्त्य᳚म् । \newline
41. पृ॒थ्सु मर्त्य॒म् मर्त्य॑म् पृ॒थ्सु पृ॒थ्सु मर्त्य॒ माव॒ आवो॒ मर्त्य॑म् पृ॒थ्सु पृ॒थ्सु मर्त्य॒ मावः॑ । \newline
42. पृ॒थ्स्विति॑ पृ॒त् - सु । \newline
43. मर्त्य॒ माव॒ आवो॒ मर्त्य॒म् मर्त्य॒ मावो॒ वाजे॑षु॒ वाजे॒ष्वावो॒ मर्त्य॒म् मर्त्य॒ मावो॒ वाजे॑षु । \newline
44. आवो॒ वाजे॑षु॒ वाजे॒ष्वाव॒ आवो॒ वाजे॑षु॒ यं ॅयं ॅवाजे॒ष्वाव॒ आवो॒ वाजे॑षु॒ यम् । \newline
45. वाजे॑षु॒ यं ॅयं ॅवाजे॑षु॒ वाजे॑षु॒ यम् जु॒ना जु॒ना यं ॅवाजे॑षु॒ वाजे॑षु॒ यम् जु॒नाः । \newline
46. यम् जु॒ना जु॒ना यं ॅयम् जु॒नाः । \newline
47. जु॒ना इति॑ जु॒नाः । \newline
48. स यन्ता॒ यन्ता॒ स स यन्ता॒ शश्व॑तीः॒ शश्व॑ती॒र् यन्ता॒ स स यन्ता॒ शश्व॑तीः । \newline
49. यन्ता॒ शश्व॑तीः॒ शश्व॑ती॒र् यन्ता॒ यन्ता॒ शश्व॑ती॒रिष॒ इषः॒ शश्व॑ती॒र् यन्ता॒ यन्ता॒ शश्व॑ती॒रिषः॑ । \newline
50. शश्व॑ती॒ रिष॒ इषः॒ शश्व॑तीः॒ शश्व॑ती॒ रिषः॑ । \newline
51. इष॒ इतीषः॑ । \newline
\pagebreak
\markright{ TS 1.3.14.1  \hfill https://www.vedavms.in \hfill}

\section{ TS 1.3.14.1 }

\textbf{TS 1.3.14.1 } \newline
\textbf{Samhita Paata} \newline

त्वम॑ग्ने रु॒द्रो असु॑रो म॒हो दि॒वस्त्वꣳ शर्द्धो॒ मारु॑तं पृ॒क्ष ई॑शिषे । त्वं ॅवातै॑ररु॒णैर्या॑सि शंग॒यस्त्वं पू॒षा वि॑ध॒तः पा॑सि॒ नुत्मना᳚ ॥ आ वो॒ राजा॑नमद्ध्व॒रस्य॑ रु॒द्रꣳ होता॑रꣳ सत्य॒यजꣳ॒॒ रोद॑स्योः । अ॒ग्निं पु॒रा त॑नयि॒त्नो र॒चित्ता॒द्धिर॑ण्यरूप॒मव॑से कृणुद्ध्वं ॥ अ॒ग्निर्.होता॒ नि ष॑सादा॒ यजी॑यानु॒पस्थे॑ मा॒तुः सु॑र॒भावु॑ लो॒के । युवा॑ क॒विः पु॑रुनि॒ष्ठ - [ ] \newline

\textbf{Pada Paata} \newline

त्वम् । अ॒ग्ने॒ । रु॒द्रः । असु॑रः । म॒हः । दि॒वः । त्वम् । शर्धः॑ । मारु॑तम् । पृ॒क्षः । ई॒शि॒षे॒ ॥ त्वम् । वातैः᳚ । अ॒रु॒णैः । या॒सि॒ । श॒ङ्ग॒य इति॑ शं - ग॒यः । त्वम् । पू॒षा । वि॒ध॒त इति॑ वि - ध॒तः । पा॒सि॒ । नु । त्मना᳚ ॥ एति॑ । वः॒ । राजा॑नम् । अ॒द्ध्व॒रस्य॑ । रु॒द्रम् । होता॑रम् । स॒त्य॒यज॒मिति॑ सत्य - यज᳚म् । रोद॑स्योः ॥ अ॒ग्निम् । पु॒रा । त॒न॒यि॒त्नोः । अ॒चित्ता᳚त् । हिर॑ण्यरूप॒मिति॒ हिर॑ण्य - रू॒प॒म् । अव॑से । कृ॒णु॒द्ध्व॒म् ॥ अ॒ग्निः । होता᳚ । नीति॑ । स॒सा॒द॒ । यजी॑यान् । उ॒पस्थ॒ इत्यु॒प - स्थे॒ । मा॒तुः । सु॒र॒भौ । उ॒ । लो॒के ॥ युवा᳚ । क॒विः । पु॒रु॒नि॒ष्ठ इति॑ पुरु - नि॒ष्ठः ।  \newline


\textbf{Krama Paata} \newline

त्वम॑ग्ने । अ॒ग्ने॒ रु॒द्रः । रु॒द्रो असु॑रः । असु॑रो म॒हः । म॒हो दि॒वः । दि॒वस्त्वम् । त्वꣳ शर्द्धः॑ । शर्द्धो॒ मारु॑तम् । मारु॑तम् पृ॒क्षः । पृ॒क्ष ई॑शिषे । ई॒शि॒ष॒ इती॑शिषे ॥ त्वं ॅवातैः᳚ । वातै॑ररु॒णैः । अ॒रु॒णैर् या॑सि । या॒सि॒ श॒ङ्ग॒यः । श॒ङ्ग॒यस्त्वम् । श॒ङ्ग॒य इति॑ शम् - ग॒यः । त्वम् पू॒षा । पू॒षा वि॑ध॒तः । वि॒ध॒तः पा॑सि । वि॒ध॒त इति॑ वि - ध॒तः । पा॒सि॒ नु । नु त्मना᳚ । त्मनेति॒ त्मना᳚ ॥ आ वः॑ । वो॒ राजा॑नम् । राजा॑नमद्ध्व॒रस्य॑ । अ॒द्ध्व॒रस्य॑ रु॒द्रम् । रु॒द्रꣳ होता॑रम् । होता॑रꣳ सत्य॒यज᳚म् । स॒त्य॒यजꣳ॒॒ रोद॑स्योः । स॒त्य॒यज॒मिति॑ सत्य - यज᳚म् । रोद॑स्यो॒रिति॒ रोद॑स्योः ॥ अ॒ग्निम् पु॒रा । पु॒रा त॑नयि॒त्नोः । त॒न॒यि॒त्नोर॒चित्ता᳚त् । अ॒चित्ता॒द्धिर॑ण्यरूपम् । हिर॑ण्यरूप॒मव॑से । हिर॑ण्यरूप॒मिति॒ हिर॑ण्य - रू॒प॒म् । अव॑से कृणुद्ध्वम् । कृ॒णु॒द्ध्व॒मिति॑ कृणुद्ध्वम् ॥ अ॒ग्निर्. होता᳚ । होता॒ नि । 
नि ष॑साद । स॒सा॒दा॒ यजी॑यान् । यजी॑यानु॒पस्थे᳚ । उ॒पस्थे॑ मा॒तुः । उ॒पस्थ॒ इत्यु॒प - स्थे॒ । मा॒तुः सु॑र॒भौ । सु॒र॒भावु॑ । उ॒ लो॒के । लो॒क इति॑ लो॒के ॥ युवा॑ क॒विः । क॒विः पु॑रुनि॒ष्ठः । पु॒रु॒नि॒ष्ठ ऋ॒तावा᳚ । पु॒रु॒नि॒ष्ठ इति॑ पुरु - नि॒ष्ठः \newline

\textbf{Jatai Paata} \newline

1. त्व म॑ग्ने अग्ने॒ त्वम् त्व म॑ग्ने । \newline
2. अ॒ग्ने॒ रु॒द्रो रु॒द्रो अ॑ग्ने अग्ने रु॒द्रः । \newline
3. रु॒द्रो असु॑रो॒ असु॑रो रु॒द्रो रु॒द्रो असु॑रः । \newline
4. असु॑रो म॒हो म॒हो असु॑रो॒ असु॑रो म॒हः । \newline
5. म॒हो दि॒वो दि॒वो म॒हो म॒हो दि॒वः । \newline
6. दि॒व स्त्वम् त्वम् दि॒वो दि॒व स्त्वम् । \newline
7. त्वꣳ शर्द्धः॒ शर्द्ध॒ स्त्वम् त्वꣳ शर्द्धः॑ । \newline
8. शर्द्धो॒ मारु॑त॒म् मारु॑त॒(ग्म्॒) शर्द्धः॒ शर्द्धो॒ मारु॑तम् । \newline
9. मारु॑तम् पृ॒क्षः पृ॒क्षो मारु॑त॒म् मारु॑तम् पृ॒क्षः । \newline
10. पृ॒क्ष ई॑शिष ईशिषे पृ॒क्षः पृ॒क्ष ई॑शिषे । \newline
11. ई॒शि॒ष॒ इती॑शिषे । \newline
12. त्वं ॅवातै॒र् वातै॒ स्त्वम् त्वं ॅवातैः᳚ । \newline
13. वातै॑ ररु॒णै र॑रु॒णैर् वातै॒र् वातै॑ ररु॒णैः । \newline
14. अ॒रु॒णैर् या॑सि यास्य रु॒णै र॑रु॒णैर् या॑सि । \newline
15. या॒सि॒ श॒ङ्ग॒यः श॑ङ्ग॒यो या॑सि यासि शङ्ग॒यः । \newline
16. श॒ङ्ग॒य स्त्वम् त्वꣳ श॑ङ्ग॒यः श॑ङ्ग॒य स्त्वम् । \newline
17. श॒ङ्ग॒य इति॑ शं - ग॒यः । \newline
18. त्वम् पू॒षा पू॒षा त्वम् त्वम् पू॒षा । \newline
19. पू॒षा वि॑ध॒तो वि॑ध॒तः पू॒षा पू॒षा वि॑ध॒तः । \newline
20. वि॒ध॒तः पा॑सि पासि विध॒तो वि॑ध॒तः पा॑सि । \newline
21. वि॒ध॒त इति॑ वि - ध॒तः । \newline
22. पा॒सि॒ नु नु पा॑सि पासि॒ नु । \newline
23. नु त्मना॒ त्मना॒ नु नु त्मना᳚ । \newline
24. त्मनेति॒ त्मना᳚ । \newline
25. आ वो॑ व॒ आ वः॑ । \newline
26. वो॒ राजा॑न॒(ग्म्॒) राजा॑नं ॅवो वो॒ राजा॑नम् । \newline
27. राजा॑न मद्ध्व॒ रस्या᳚द्ध्व॒रस्य॒ राजा॑न॒(ग्म्॒) राजा॑न मद्ध्व॒रस्य॑ । \newline
28. अ॒द्ध्व॒रस्य॑ रु॒द्रꣳ रु॒द्र म॑द्ध्व॒ रस्या᳚द्ध्व॒रस्य॑ रु॒द्रम् । \newline
29. रु॒द्रꣳ होता॑र॒(ग्म्॒) होता॑रꣳ रु॒द्रꣳ रु॒द्रꣳ होता॑रम् । \newline
30. होता॑रꣳ सत्य॒यज(ग्म्॑) सत्य॒यज॒(ग्म्॒) होता॑र॒(ग्म्॒) होता॑रꣳ सत्य॒यज᳚म् । \newline
31. स॒त्य॒यज॒(ग्म्॒) रोद॑स्यो॒ रोद॑स्योः सत्य॒यज(ग्म्॑) सत्य॒यज॒(ग्म्॒) रोद॑स्योः । \newline
32. स॒त्य॒यज॒मिति॑ सत्य - यज᳚म् । \newline
33. रोद॑स्यो॒रिति॒ रोद॑स्योः । \newline
34. अ॒ग्निम् पु॒रा पु॒रा ऽग्नि म॒ग्निम् पु॒रा । \newline
35. पु॒रा त॑नयि॒त्नो स्त॑नयि॒त्नोः पु॒रा पु॒रा त॑नयि॒त्नोः । \newline
36. त॒न॒यि॒त्नो र॒चित्ता॑ द॒चित्ता᳚त् तनयि॒त्नो स्त॑नयि॒त्नो र॒चित्ता᳚त् । \newline
37. अ॒चित्ता॒ द्धिर॑ण्यरूप॒(ग्म्॒) हिर॑ण्यरूप म॒चित्ता॑ द॒चित्ता॒ द्धिर॑ण्यरूपम् । \newline
38. हिर॑ण्यरूप॒ मव॒से ऽव॑से॒ हिर॑ण्यरूप॒(ग्म्॒) हिर॑ण्यरूप॒ मव॑से । \newline
39. हिर॑ण्यरूप॒मिति॒ हिर॑ण्य - रू॒प॒म् । \newline
40. अव॑से कृणुद्ध्वम् कृणुद्ध्व॒ मव॒से ऽव॑से कृणुद्ध्वम् । \newline
41. कृ॒णु॒द्ध्व॒मिति॑ कृणुद्ध्वम् । \newline
42. अ॒ग्निर्. होता॒ होता॒ ऽग्नि र॒ग्निर्. होता᳚ । \newline
43. होता॒ नि नि होता॒ होता॒ नि । \newline
44. नि ष॑साद ससाद॒ नि नि ष॑साद । \newline
45. स॒सा॒दा॒ यजी॑या॒न्॒. यजी॑यान् थ्ससाद ससादा॒ यजी॑यान् । \newline
46. यजी॑या नु॒पस्थ॑ उ॒पस्थे॒ यजी॑या॒न्॒. यजी॑या नु॒पस्थे᳚ । \newline
47. उ॒पस्थे॑ मा॒तुर् मा॒तु रु॒पस्थ॑ उ॒पस्थे॑ मा॒तुः । \newline
48. उ॒पस्थ॒ इत्यु॒प - स्थे॒ । \newline
49. मा॒तुः सु॑र॒भौ सु॑र॒भौ मा॒तुर् मा॒तुः सु॑र॒भौ । \newline
50. सु॒र॒भा वु॑ वु सुर॒भौ सु॑र॒भा वु॑ । \newline
51. उ॒ लो॒के लो॒क उ॑ वु लो॒के । \newline
52. लो॒क इति॑ लो॒के । \newline
53. युवा॑ क॒विः क॒विर् युवा॒ युवा॑ क॒विः । \newline
54. क॒विः पु॑रुनि॒ष्ठः पु॑रुनि॒ष्ठः क॒विः क॒विः पु॑रुनि॒ष्ठः । \newline
55. पु॒रु॒नि॒ष्ठ ऋ॒ताव॒र्तावा॑ पुरुनि॒ष्ठः पु॑रुनि॒ष्ठ ऋ॒तावा᳚ । \newline
56. पु॒रु॒नि॒ष्ठ इति॑ पुरु - नि॒ष्ठः । \newline

\textbf{Ghana Paata } \newline

1. त्व म॑ग्ने अग्ने॒ त्वम् त्व म॑ग्ने रु॒द्रो रु॒द्रो अ॑ग्ने॒ त्वम् त्व म॑ग्ने रु॒द्रः । \newline
2. अ॒ग्ने॒ रु॒द्रो रु॒द्रो अ॑ग्ने अग्ने रु॒द्रो असु॑रो॒ असु॑रो रु॒द्रो अ॑ग्ने अग्ने रु॒द्रो असु॑रः । \newline
3. रु॒द्रो असु॑रो॒ असु॑रो रु॒द्रो रु॒द्रो असु॑रो म॒हो म॒हो असु॑रो रु॒द्रो रु॒द्रो असु॑रो म॒हः । \newline
4. असु॑रो म॒हो म॒हो असु॑रो॒ असु॑रो म॒हो दि॒वो दि॒वो म॒हो असु॑रो॒ असु॑रो म॒हो दि॒वः । \newline
5. म॒हो दि॒वो दि॒वो म॒हो म॒हो दि॒व स्त्वम् त्वम् दि॒वो म॒हो म॒हो दि॒व स्त्वम् । \newline
6. दि॒व स्त्वम् त्वम् दि॒वो दि॒व स्त्वꣳ शर्द्धः॒ शर्द्ध॒ स्त्वम् दि॒वो दि॒व स्त्वꣳ शर्द्धः॑ । \newline
7. त्वꣳ शर्द्धः॒ शर्द्ध॒ स्त्वम् त्वꣳ शर्द्धो॒ मारु॑त॒म् मारु॑त॒(ग्म्॒) शर्द्ध॒ स्त्वम् त्वꣳ शर्द्धो॒ मारु॑तम् । \newline
8. शर्द्धो॒ मारु॑त॒म् मारु॑त॒(ग्म्॒) शर्द्धः॒ शर्द्धो॒ मारु॑तम् पृ॒क्षः पृ॒क्षो मारु॑त॒(ग्म्॒) शर्द्धः॒ शर्द्धो॒ मारु॑तम् पृ॒क्षः । \newline
9. मारु॑तम् पृ॒क्षः पृ॒क्षो मारु॑त॒म् मारु॑तम् पृ॒क्ष ई॑शिष ईशिषे पृ॒क्षो मारु॑त॒म् मारु॑तम् पृ॒क्ष ई॑शिषे । \newline
10. पृ॒क्ष ई॑शिष ईशिषे पृ॒क्षः पृ॒क्ष ई॑शिषे । \newline
11. ई॒शि॒ष॒ इती॑शिषे । \newline
12. त्वं ॅवातै॒र् वातै॒ स्त्वम् त्वं ॅवातै॑ ररु॒णै र॑रु॒णैर् वातै॒ स्त्वम् त्वं ॅवातै॑ ररु॒णैः । \newline
13. वातै॑ ररु॒णै र॑रु॒णैर् वातै॒र् वातै॑ ररु॒णैर् या॑सि यास्यरु॒णैर् वातै॒र् वातै॑ ररु॒णैर् या॑सि । \newline
14. अ॒रु॒णैर् या॑सि यास्य रु॒णै र॑रु॒णैर् या॑सि शङ्ग॒यः श॑ङ्ग॒यो या᳚स्य रु॒णैर॑ रु॒णैर् या॑सि शङ्ग॒यः । \newline
15. या॒सि॒ श॒ङ्ग॒यः श॑ङ्ग॒यो या॑सि यासि शङ्ग॒य स्त्वम् त्वꣳ श॑ङ्ग॒यो या॑सि यासि शङ्ग॒य स्त्वम् । \newline
16. श॒ङ्ग॒य स्त्वम् त्वꣳ श॑ङ्ग॒यः श॑ङ्ग॒य स्त्वम् पू॒षा पू॒षा त्वꣳ श॑ङ्ग॒यः श॑ङ्ग॒य स्त्वम् पू॒षा । \newline
17. श॒ङ्ग॒य इति॑ शं - ग॒यः । \newline
18. त्वम् पू॒षा पू॒षा त्वम् त्वम् पू॒षा वि॑ध॒तो वि॑ध॒तः पू॒षा त्वम् त्वम् पू॒षा वि॑ध॒तः । \newline
19. पू॒षा वि॑ध॒तो वि॑ध॒तः पू॒षा पू॒षा वि॑ध॒तः पा॑सि पासि विध॒तः पू॒षा पू॒षा वि॑ध॒तः पा॑सि । \newline
20. वि॒ध॒तः पा॑सि पासि विध॒तो वि॑ध॒तः पा॑सि॒ नु नु पा॑सि विध॒तो वि॑ध॒तः पा॑सि॒ नु । \newline
21. वि॒ध॒त इति॑ वि - ध॒तः । \newline
22. पा॒सि॒ नु नु पा॑सि पासि॒ नु त्मना॒ त्मना॒ नु पा॑सि पासि॒ नु त्मना᳚ । \newline
23. नु त्मना॒ त्मना॒ नु नु त्मना᳚ । \newline
24. त्मनेति॒ त्मना᳚ । \newline
25. आ वो॑ व॒ आ वो॒ राजा॑न॒(ग्म्॒) राजा॑नं ॅव॒ आ वो॒ राजा॑नम् । \newline
26. वो॒ राजा॑न॒(ग्म्॒) राजा॑नं ॅवो वो॒ राजा॑न मद्ध्व॒र स्या᳚द्ध्व॒रस्य॒ राजा॑नं ॅवो वो॒ राजा॑न मद्ध्व॒रस्य॑ । \newline
27. राजा॑न मद्ध्व॒र स्या᳚द्ध्व॒रस्य॒ राजा॑न॒(ग्म्॒) राजा॑न मद्ध्व॒रस्य॑ रु॒द्रꣳ रु॒द्र म॑द्ध्व॒रस्य॒ राजा॑न॒(ग्म्॒) राजा॑न मद्ध्व॒रस्य॑ रु॒द्रम् । \newline
28. अ॒द्ध्व॒रस्य॑ रु॒द्रꣳ रु॒द्र म॑द्ध्व॒र स्या᳚द्ध्व॒रस्य॑ रु॒द्रꣳ होता॑र॒(ग्म्॒) होता॑रꣳ रु॒द्र 
म॑द्ध्व॒रस्या᳚द्ध्व॒रस्य॑ रु॒द्रꣳ होता॑रम् । \newline
29. रु॒द्रꣳ होता॑र॒(ग्म्॒) होता॑रꣳ रु॒द्रꣳ रु॒द्रꣳ होता॑रꣳ सत्य॒यज(ग्म्॑) सत्य॒यज॒(ग्म्॒) होता॑रꣳ रु॒द्रꣳ रु॒द्रꣳ होता॑रꣳ सत्य॒यज᳚म् । \newline
30. होता॑रꣳ सत्य॒यज(ग्म्॑) सत्य॒यज॒(ग्म्॒) होता॑र॒(ग्म्॒) होता॑रꣳ सत्य॒यज॒(ग्म्॒) रोद॑स्यो॒ रोद॑स्योः सत्य॒यज॒(ग्म्॒) होता॑र॒(ग्म्॒) होता॑रꣳ सत्य॒यज॒(ग्म्॒) रोद॑स्योः । \newline
31. स॒त्य॒यज॒(ग्म्॒) रोद॑स्यो॒ रोद॑स्योः सत्य॒यज(ग्म्॑) सत्य॒यज॒(ग्म्॒) रोद॑स्योः । \newline
32. स॒त्य॒यज॒मिति॑ सत्य - यज᳚म् । \newline
33. रोद॑स्यो॒रिति॒ रोद॑स्योः । \newline
34. अ॒ग्निम् पु॒रा पु॒रा ऽग्नि म॒ग्निम् पु॒रा त॑नयि॒त्नो स्त॑नयि॒त्नोः पु॒रा ऽग्नि म॒ग्निम् पु॒रा त॑नयि॒त्नोः । \newline
35. पु॒रा त॑नयि॒त्नो स्त॑नयि॒त्नोः पु॒रा पु॒रा त॑नयि॒त्नो र॒चित्ता॑ द॒चित्ता᳚त् तनयि॒त्नोः पु॒रा पु॒रा त॑नयि॒त्नो र॒चित्ता᳚त् । \newline
36. त॒न॒यि॒त्नो र॒चित्ता॑ द॒चित्ता᳚त् तनयि॒त्नो स्त॑नयि॒त्नो र॒चित्ता॒ द्धिर॑ण्यरूप॒(ग्म्॒) हिर॑ण्यरूप म॒चित्ता᳚त् तनयि॒त्नो स्त॑नयि॒त्नो र॒चित्ता॒ द्धिर॑ण्यरूपम् । \newline
37. अ॒चित्ता॒ द्धिर॑ण्यरूप॒(ग्म्॒) हिर॑ण्यरूप म॒चित्ता॑ द॒चित्ता॒ द्धिर॑ण्यरूप॒ मव॒से ऽव॑से॒ हिर॑ण्यरूप म॒चित्ता॑ द॒चित्ता॒ द्धिर॑ण्यरूप॒ मव॑से । \newline
38. हिर॑ण्यरूप॒ मव॒से ऽव॑से॒ हिर॑ण्यरूप॒(ग्म्॒) हिर॑ण्यरूप॒ मव॑से कृणुद्ध्वम् कृणुद्ध्व॒ मव॑से॒ हिर॑ण्यरूप॒(ग्म्॒) हिर॑ण्यरूप॒ मव॑से कृणुद्ध्वम् । \newline
39. हिर॑ण्यरूप॒मिति॒ हिर॑ण्य - रू॒प॒म् । \newline
40. अव॑से कृणुद्ध्वम् कृणुद्ध्व॒ मव॒से ऽव॑से कृणुद्ध्वम् । \newline
41. कृ॒णु॒द्ध्व॒मिति॑ कृणुद्ध्वम् । \newline
42. अ॒ग्निर्. होता॒ होता॒ ऽग्नि र॒ग्निर्. होता॒ नि नि होता॒ ऽग्नि र॒ग्निर्. होता॒ नि । \newline
43. होता॒ नि नि होता॒ होता॒ नि ष॑साद ससाद॒ नि होता॒ होता॒ नि ष॑साद । \newline
44. नि ष॑साद ससाद॒ नि नि ष॑सादा॒ यजी॑या॒न्. यजी॑यान् थ्ससाद॒ नि नि ष॑सादा॒ यजी॑यान् । \newline
45. स॒सा॒दा॒ यजी॑या॒न्॒. यजी॑यान् थ्ससाद ससादा॒ यजी॑या नु॒पस्थ॑ उ॒पस्थे॒ यजी॑यान् थ्ससाद ससादा॒ यजी॑या नु॒पस्थे᳚ । \newline
46. यजी॑या नु॒पस्थ॑ उ॒पस्थे॒ यजी॑या॒न्.॒ यजी॑या नु॒पस्थे॑ मा॒तुर् मा॒तु रु॒पस्थे॒ यजी॑या॒न्.॒ यजी॑या नु॒पस्थे॑ मा॒तुः । \newline
47. उ॒पस्थे॑ मा॒तुर् मा॒तु रु॒पस्थ॑ उ॒पस्थे॑ मा॒तुः सु॑र॒भौ सु॑र॒भौ मा॒तु रु॒पस्थ॑ उ॒पस्थे॑ मा॒तुः सु॑र॒भौ । \newline
48. उ॒पस्थ॒ इत्यु॒प - स्थे॒ । \newline
49. मा॒तुः सु॑र॒भौ सु॑र॒भौ मा॒तुर् मा॒तुः सु॑र॒भा वु॑ वु सुर॒भौ मा॒तुर् मा॒तुः सु॑र॒भा वु॑ । \newline
50. सु॒र॒भा वु॑ वु सुर॒भौ सु॑र॒भा वु॑ लो॒के लो॒क उ॑ सुर॒भौ सु॑र॒भा वु॑ लो॒के । \newline
51. उ॒ लो॒के लो॒क उ॑ वु लो॒के । \newline
52. लो॒क इति॑ लो॒के । \newline
53. युवा॑ क॒विः क॒विर् युवा॒ युवा॑ क॒विः पु॑रुनि॒ष्ठः पु॑रुनि॒ष्ठः क॒विर् युवा॒ युवा॑ क॒विः पु॑रुनि॒ष्ठः । \newline
54. क॒विः पु॑रुनि॒ष्ठः पु॑रुनि॒ष्ठः क॒विः क॒विः पु॑रुनि॒ष्ठ ऋ॒ताव॒र्तावा॑ पुरुनि॒ष्ठः क॒विः क॒विः पु॑रुनि॒ष्ठ ऋ॒तावा᳚ । \newline
55. पु॒रु॒नि॒ष्ठ ऋ॒ताव॒र्तावा॑ पुरुनि॒ष्ठः पु॑रुनि॒ष्ठ ऋ॒तावा॑ ध॒र्ता ध॒र्तर्तावा॑ पुरुनि॒ष्ठः पु॑रुनि॒ष्ठ ऋ॒तावा॑ ध॒र्ता । \newline
56. पु॒रु॒नि॒ष्ठ इति॑ पुरु - नि॒ष्ठः । \newline
\pagebreak
\markright{ TS 1.3.14.2  \hfill https://www.vedavms.in \hfill}

\section{ TS 1.3.14.2 }

\textbf{TS 1.3.14.2 } \newline
\textbf{Samhita Paata} \newline

ऋ॒तावा॑ ध॒र्ता कृ॑ष्टी॒नामु॒त मद्ध्य॑ इ॒द्धः ॥सा॒द्ध्वीम॑कर् दे॒ववी॑तिं नो अ॒द्य य॒ज्ञ्स्य॑ जि॒ह्वाम॑विदाम॒ गुह्यां᳚ । स आयु॒राऽगा᳚थ् सुर॒भिर्वसा॑नो भ॒द्राम॑कर् दे॒वहू॑तिं नो अ॒द्य ॥ अक्र॑न्दद॒ग्निः स्त॒नय॑न्निव॒ द्यौः क्षामा॒ रेरि॑हद्वी॒रुधः॑ सम॒ञ्जन्न् । स॒द्यो ज॑ज्ञा॒नो वि हीमि॒द्धो अख्य॒दा रोद॑सी भा॒नुना॑ भात्य॒न्तः ॥ त्वे वसू॑नि पुर्वणीक-[ ] \newline

\textbf{Pada Paata} \newline

ऋ॒तावेत्यृ॒ता - वा॒ । ध॒र्ता । कृ॒ष्टी॒नाम् । उ॒त । मद्ध्ये᳚ । इ॒द्धः ॥ सा॒द्ध्वीम् । अ॒कः॒ । दे॒ववी॑ति॒मिति॑ दे॒व - वी॒ति॒म् । नः॒ । अ॒द्य । य॒ज्ञ्स्य॑ । जि॒ह्वाम् । अ॒वि॒दा॒म॒ । गुह्या᳚म् ॥ सः । आयुः॑ । एति॑ । अ॒गा॒त् । सु॒र॒भिः । वसा॑नः । भ॒द्राम् । अ॒कः॒ । दे॒वहू॑ति॒मिति॑ दे॒व - हू॒ति॒म् । नः॒ । अ॒द्य ॥ अक्र॑न्दत् । अ॒ग्निः । स्त॒नयन्न्॑ । इ॒व॒ । द्यौः । क्षाम॑ । रेरि॑हत् । वी॒रुधः॑ । स॒म॒ञ्जन्निति॑ सं - अ॒ञ्जन्न् ॥ स॒द्यः । ज॒ज्ञा॒नः । वीति॑ । हि । ई॒म् । इ॒द्धः । अख्य॑त् । एति॑ । रोद॑सी॒ इति॑ । भा॒नुना᳚ । भा॒ति॒ । अ॒न्तः ॥ त्वे इति॑ । वसू॑नि । पु॒र्व॒णी॒केति॑ पुरु - अ॒नी॒क॒ ।  \newline


\textbf{Krama Paata} \newline

ऋ॒तावा॑ ध॒र्ता । ऋ॒तावेत्यृ॒त - वा॒ । ध॒र्ता कृ॑ष्टी॒नाम् । कृ॒ष्टी॒नामु॒त । उ॒त मद्ध्ये᳚ । मद्ध्य॑ इ॒द्धः । इ॒द्ध इती॒द्धः ॥ सा॒द्ध्वीम॑कः । अ॒क॒र् दे॒ववी॑तिम् । दे॒ववी॑तिम् नः । दे॒ववी॑ति॒मिति॑ दे॒व - वी॒ति॒म् । नो॒ अ॒द्य । अ॒द्य य॒ज्ञ्स्य॑ । य॒ज्ञ्स्य॑ जि॒ह्वाम् । जि॒ह्वाम॑विदाम । अ॒वि॒दा॒म॒ गुह्या᳚म् । गुह्या॒मिति॒ गुह्या᳚म् ॥ स आयुः॑ । आयु॒रा । आऽगा᳚त् । अ॒गा॒थ् सु॒र॒भिः । सु॒र॒भिर् वसा॑नः । वसा॑नो भ॒द्राम् । भ॒द्राम॑कः । अ॒क॒र् दे॒वहू॑तिम् । दे॒वहू॑तिम् नः । दे॒वहू॑ति॒मिति॑ दे॒व - हू॒ति॒म् । नो॒ अ॒द्य । अ॒द्येत्य॒द्य ॥ अक्र॑न्दद॒ग्निः । अ॒ग्निः स्त॒नयन्न्॑ । स्त॒नय॑न्निव । इ॒व॒ द्यौः । द्यौः क्षाम॑ । क्षामा॒ रेरि॑हत् । रेरि॑हद् वी॒रुधः॑ । वी॒रुधः॑ सम॒ञ्जन्न् । स॒म॒ञ्जन्निति॑ सं - अ॒ञ्जन्न् ॥ स॒द्यो ज॑ज्ञा॒नः । ज॒ज्ञा॒नो वि । वि हि । हीम् । ई॒मि॒द्धः । इ॒द्धो अख्य॑त् । अख्य॒दा । आ रोद॑सी । रोद॑सी भा॒नुना᳚ । रोद॑सी॒ इति॒ रोद॑सी । भा॒नुना॑ भाति । भा॒त्य॒न्तः । अ॒न्तरित्य॒न्तः ॥ त्वे वसू॑नि । त्वे इति॒ त्वे । वसू॑नि पुर्वणीक । पु॒र्व॒णी॒क॒ हो॒तः॒ । पु॒र्व॒णी॒केति॑ पुरु - अ॒नी॒क॒ \newline

\textbf{Jatai Paata} \newline

1. ऋ॒तावा॑ ध॒र्ता ध॒र्तर्ताव॒र्तावा॑ ध॒र्ता । \newline
2. ऋ॒तावेत्यृ॒ता - वा॒ । \newline
3. ध॒र्ता कृ॑ष्टी॒नाम् कृ॑ष्टी॒नाम् ध॒र्ता ध॒र्ता कृ॑ष्टी॒नाम् । \newline
4. कृ॒ष्टी॒ना मु॒तोत कृ॑ष्टी॒नाम् कृ॑ष्टी॒ना मु॒त । \newline
5. उ॒त मद्ध्ये॒ मद्ध्य॑ उ॒तोत मद्ध्ये᳚ । \newline
6. मद्ध्य॑ इ॒द्ध इ॒द्धो मद्ध्ये॒ मद्ध्य॑ इ॒द्धः । \newline
7. इ॒द्ध इती॒द्धः । \newline
8. सा॒द्ध्वी म॑क रकः सा॒द्ध्वीꣳ सा॒द्ध्वी म॑कः । \newline
9. अ॒क॒र् दे॒ववी॑तिम् दे॒ववी॑ति मक रकर् दे॒ववी॑तिम् । \newline
10. दे॒ववी॑तिम् नो नो दे॒ववी॑तिम् दे॒ववी॑तिम् नः । \newline
11. दे॒ववी॑ति॒मिति॑ दे॒व - वी॒ति॒म् । \newline
12. नो॒ अ॒द्याद्य नो॑ नो अ॒द्य । \newline
13. अ॒द्य य॒ज्ञ्स्य॑ य॒ज्ञ् स्या॒द्याद्य य॒ज्ञ्स्य॑ । \newline
14. य॒ज्ञ्स्य॑ जि॒ह्वाम् जि॒ह्वां ॅय॒ज्ञ्स्य॑ य॒ज्ञ्स्य॑ जि॒ह्वाम् । \newline
15. जि॒ह्वा म॑विदा माविदाम जि॒ह्वाम् जि॒ह्वा म॑विदाम । \newline
16. अ॒वि॒दा॒म॒ गुह्या॒म् गुह्या॑ मविदा माविदाम॒ गुह्या᳚म् । \newline
17. गुह्या॒मिति॒ गुह्या᳚म् । \newline
18. स आयु॒ रायुः॒ स स आयुः॑ । \newline
19. आयु॒रा आयु॒ रायु॒रा । \newline
20. आ ऽगा॑दगा॒दा ऽगा᳚त् । \newline
21. अ॒गा॒थ् सु॒र॒भिः सु॑र॒भि र॑गा दगाथ् सुर॒भिः । \newline
22. सु॒र॒भिर् वसा॑नो॒ वसा॑नः सुर॒भिः सु॑र॒भिर् वसा॑नः । \newline
23. वसा॑नो भ॒द्राम् भ॒द्रां ॅवसा॑नो॒ वसा॑नो भ॒द्राम् । \newline
24. भ॒द्रा म॑क रकर् भ॒द्राम् भ॒द्रा म॑कः । \newline
25. अ॒क॒र् दे॒वहू॑तिम् दे॒वहू॑ति मक रकर् दे॒वहू॑तिम् । \newline
26. दे॒वहू॑तिम् नो नो दे॒वहू॑तिम् दे॒वहू॑तिम् नः । \newline
27. दे॒वहू॑ति॒मिति॑ दे॒व - हू॒ति॒म् । \newline
28. नो॒ अ॒द्याद्य नो॑ नो अ॒द्य । \newline
29. अ॒द्येत्य॒द्य । \newline
30. अक्र॑न्द द॒ग्नि र॒ग्नि रक्र॑न्द॒ दक्र॑न्द द॒ग्निः । \newline
31. अ॒ग्निः स्त॒नयन्᳚ थ्स्त॒नय॑न् न॒ग्नि र॒ग्निः स्त॒नयन्न्॑ । \newline
32. स्त॒नय॑न् निवे व स्त॒नयन्᳚ थ्स्त॒नय॑न् निव । \newline
33. इ॒व॒ द्यौर् द्यौरि॑वे व॒ द्यौः । \newline
34. द्यौः क्षाम॒ क्षाम॒ द्यौर् द्यौः क्षाम॑ । \newline
35. क्षामा॒ रेरि॑ह॒द् रेरि॑ह॒त् क्षाम॒ क्षामा॒ रेरि॑हत् । \newline
36. रेरि॑हद् वी॒रुधो॑ वी॒रुधो॒ रेरि॑ह॒द् रेरि॑हद् वी॒रुधः॑ । \newline
37. वी॒रुधः॑ सम॒ञ्जन् थ्स॑म॒ञ्जन्. वी॒रुधो॑ वी॒रुधः॑ सम॒ञ्जन्न् । \newline
38. स॒म॒ञ्जन्निति॑ सं - अ॒ञ्जन्न् । \newline
39. स॒द्यो ज॑ज्ञा॒नो ज॑ज्ञा॒नः स॒द्यः स॒द्यो ज॑ज्ञा॒नः । \newline
40. ज॒ज्ञा॒नो वि वि ज॑ज्ञा॒नो ज॑ज्ञा॒नो वि । \newline
41. वि हि हि वि वि हि । \newline
42. ही मी॒(ग्म्॒) हि हीम् । \newline
43. ई॒ मि॒द्ध इ॒द्ध ई॑ मी मि॒द्धः । \newline
44. इ॒द्धो अख्य॒ दख्य॑ दि॒द्ध इ॒द्धो अख्य॑त् । \newline
45. अख्य॒दा ऽख्य॒ दख्य॒दा । \newline
46. आ रोद॑सी॒ रोद॑सी॒ आ रोद॑सी । \newline
47. रोद॑सी भा॒नुना॑ भा॒नुना॒ रोद॑सी॒ रोद॑सी भा॒नुना᳚ । \newline
48. रोद॑सी॒ इति॒ रोद॑सी । \newline
49. भा॒नुना॑ भाति भाति भा॒नुना॑ भा॒नुना॑ भाति । \newline
50. भा॒त्य॒न्त र॒न्तर् भा॑ति भात्य॒न्तः । \newline
51. अ॒न्तरित्य॒न्तः । \newline
52. त्वे वसू॑नि॒ वसू॑नि॒ त्वे त्वे वसू॑नि । \newline
53. त्वे इति॒ त्वे । \newline
54. वसू॑नि पुर्वणीक पुर्वणीक॒ वसू॑नि॒ वसू॑नि पुर्वणीक । \newline
55. पु॒र्व॒णी॒क॒ हो॒त॒र्॒. हो॒तः॒ पु॒र्व॒णी॒क॒ पु॒र्व॒णी॒क॒ हो॒तः॒ । \newline
56. पु॒र्व॒णी॒केति॑ पुरु - अ॒नी॒क॒ । \newline

\textbf{Ghana Paata } \newline

1. ऋ॒तावा॑ ध॒र्ता ध॒र्तर्ताव॒र्तावा॑ ध॒र्ता कृ॑ष्टी॒नाम् कृ॑ष्टी॒नाम् ध॒र्तर्ताव॒र्तावा॑ ध॒र्ता कृ॑ष्टी॒नाम् । \newline
2. ऋ॒तावेत्यृ॒ता - वा॒ । \newline
3. ध॒र्ता कृ॑ष्टी॒नाम् कृ॑ष्टी॒नाम् ध॒र्ता ध॒र्ता कृ॑ष्टी॒ना मु॒तोत कृ॑ष्टी॒नाम् ध॒र्ता ध॒र्ता कृ॑ष्टी॒ना मु॒त । \newline
4. कृ॒ष्टी॒ना मु॒तोत कृ॑ष्टी॒नाम् कृ॑ष्टी॒ना मु॒त मद्ध्ये॒ मद्ध्य॑ उ॒त कृ॑ष्टी॒नाम् कृ॑ष्टी॒ना मु॒त मद्ध्ये᳚ । \newline
5. उ॒त मद्ध्ये॒ मद्ध्य॑ उ॒तोत मद्ध्य॑ इ॒द्ध इ॒द्धो मद्ध्य॑ उ॒तोत मद्ध्य॑ इ॒द्धः । \newline
6. मद्ध्य॑ इ॒द्ध इ॒द्धो मद्ध्ये॒ मद्ध्य॑ इ॒द्धः । \newline
7. इ॒द्ध इती॒द्धः । \newline
8. सा॒द्ध्वी म॑क रकः सा॒द्ध्वीꣳ सा॒द्ध्वी म॑कर् दे॒ववी॑तिम् दे॒ववी॑ति मकः सा॒द्ध्वीꣳ सा॒द्ध्वी म॑कर् दे॒ववी॑तिम् । \newline
9. अ॒क॒र् दे॒ववी॑तिम् दे॒ववी॑ति मक रकर् दे॒ववी॑तिम् नो नो दे॒ववी॑ति मक रकर् दे॒ववी॑तिम् नः । \newline
10. दे॒ववी॑तिम् नो नो दे॒ववी॑तिम् दे॒ववी॑तिम् नो अ॒द्याद्य नो॑ दे॒ववी॑तिम् दे॒ववी॑तिम् नो अ॒द्य । \newline
11. दे॒ववी॑ति॒मिति॑ दे॒व - वी॒ति॒म् । \newline
12. नो॒ अ॒द्याद्य नो॑ नो अ॒द्य य॒ज्ञ्स्य॑ य॒ज्ञ्स्या॒द्य नो॑ नो अ॒द्य य॒ज्ञ्स्य॑ । \newline
13. अ॒द्य य॒ज्ञ्स्य॑ य॒ज्ञ्स्या॒द्याद्य य॒ज्ञ्स्य॑ जि॒ह्वाम् जि॒ह्वां ॅय॒ज्ञ्स्या॒द्याद्य य॒ज्ञ्स्य॑ जि॒ह्वाम् । \newline
14. य॒ज्ञ्स्य॑ जि॒ह्वाम् जि॒ह्वां ॅय॒ज्ञ्स्य॑ य॒ज्ञ्स्य॑ जि॒ह्वा म॑विदा माविदाम जि॒ह्वां ॅय॒ज्ञ्स्य॑ य॒ज्ञ्स्य॑ जि॒ह्वा म॑विदाम । \newline
15. जि॒ह्वा म॑विदा माविदाम जि॒ह्वाम् जि॒ह्वा म॑विदाम॒ गुह्या॒म् गुह्या॑ मविदाम जि॒ह्वाम् जि॒ह्वा म॑विदाम॒ गुह्या᳚म् । \newline
16. अ॒वि॒दा॒म॒ गुह्या॒म् गुह्या॑ मविदा माविदाम॒ गुह्या᳚म् । \newline
17. गुह्या॒मिति॒ गुह्या᳚म् । \newline
18. स आयु॒रायुः॒ स स आयु॒रा आयुः॒ स स आयु॒रा । \newline
19. आयु॒रा आयु॒रायु॒रा ऽगा॑दगा॒दा आयु॒रायु॒रा ऽगा᳚त् । \newline
20. आ ऽगा॑दगा॒दा ऽगा᳚थ् सुर॒भिः सु॑र॒भि र॑गा॒दा ऽगा᳚थ् सुर॒भिः । \newline
21. अ॒गा॒थ् सु॒र॒भिः सु॑र॒भि र॑गादगाथ् सुर॒भिर् वसा॑नो॒ वसा॑नः सुर॒भि र॑गादगाथ् सुर॒भिर् वसा॑नः । \newline
22. सु॒र॒भिर् वसा॑नो॒ वसा॑नः सुर॒भिः सु॑र॒भिर् वसा॑नो भ॒द्राम् भ॒द्रां ॅवसा॑नः सुर॒भिः सु॑र॒भिर् वसा॑नो भ॒द्राम् । \newline
23. वसा॑नो भ॒द्राम् भ॒द्रां ॅवसा॑नो॒ वसा॑नो भ॒द्रा म॑क रकर् भ॒द्रां ॅवसा॑नो॒ वसा॑नो भ॒द्रा म॑कः । \newline
24. भ॒द्रा म॑क रकर् भ॒द्राम् भ॒द्रा म॑कर् दे॒वहू॑तिम् दे॒वहू॑ति मकर् भ॒द्राम् भ॒द्रा म॑कर् दे॒वहू॑तिम् । \newline
25. अ॒क॒र् दे॒वहू॑तिम् दे॒वहू॑ति मक रकर् दे॒वहू॑तिम् नो नो दे॒वहू॑ति मक रकर् दे॒वहू॑तिम् नः । \newline
26. दे॒वहू॑तिम् नो नो दे॒वहू॑तिम् दे॒वहू॑तिम् नो अ॒द्याद्य नो॑ दे॒वहू॑तिम् दे॒वहू॑तिम् नो अ॒द्य । \newline
27. दे॒वहू॑ति॒मिति॑ दे॒व - हू॒ति॒म् । \newline
28. नो॒ अ॒द्याद्य नो॑ नो अ॒द्य । \newline
29. अ॒द्येत्य॒द्य । \newline
30. अक्र॑न्द द॒ग्नि र॒ग्नि रक्र॑न्द॒ दक्र॑न्द द॒ग्निः स्त॒नय᳚न् थ्स्त॒नय॑न् न॒ग्नि रक्र॑न्द॒ दक्र॑न्द द॒ग्निः स्त॒नयन्न्॑ । \newline
31. अ॒ग्निः स्त॒नय᳚न् थ्स्त॒नय॑न् न॒ग्नि र॒ग्निः स्त॒नय॑न् निवे व स्त॒नय॑न् न॒ग्नि र॒ग्निः स्त॒नय॑न् निव । \newline
32. स्त॒नय॑न् निवे व स्त॒नय᳚न् थ्स्त॒नय॑न् निव॒ द्यौर् द्यौरि॑व स्त॒नय᳚न् थ्स्त॒नय॑न् निव॒ द्यौः । \newline
33. इ॒व॒ द्यौर् द्यौ रि॑वे व॒ द्यौः क्षाम॒ क्षाम॒ द्यौरि॑वे व॒ द्यौः क्षाम॑ । \newline
34. द्यौः क्षाम॒ क्षाम॒ द्यौर् द्यौः क्षामा॒ रेरि॑ह॒द् रेरि॑ह॒त् क्षाम॒ द्यौर् द्यौः क्षामा॒ रेरि॑हत् । \newline
35. क्षामा॒ रेरि॑ह॒द् रेरि॑ह॒त् क्षाम॒ क्षामा॒ रेरि॑हद् वी॒रुधो॑ वी॒रुधो॒ रेरि॑ह॒त् क्षाम॒ क्षामा॒ रेरि॑हद् वी॒रुधः॑ । \newline
36. रेरि॑हद् वी॒रुधो॑ वी॒रुधो॒ रेरि॑ह॒द् रेरि॑हद् वी॒रुधः॑ सम॒ञ्जन् थ्स॑म॒ञ्जन्. वी॒रुधो॒ रेरि॑ह॒द् रेरि॑हद् वी॒रुधः॑ सम॒ञ्जन्न् । \newline
37. वी॒रुधः॑ सम॒ञ्जन् थ्स॑म॒ञ्जन्. वी॒रुधो॑ वी॒रुधः॑ सम॒ञ्जन्न् । \newline
38. स॒म॒ञ्जन्निति॑ सं - अ॒ञ्जन्न् । \newline
39. स॒द्यो ज॑ज्ञा॒नो ज॑ज्ञा॒नः स॒द्यः स॒द्यो ज॑ज्ञा॒नो वि वि ज॑ज्ञा॒नः स॒द्यः स॒द्यो ज॑ज्ञा॒नो वि । \newline
40. ज॒ज्ञा॒नो वि वि ज॑ज्ञा॒नो ज॑ज्ञा॒नो वि हि हि वि ज॑ज्ञा॒नो ज॑ज्ञा॒नो वि हि । \newline
41. वि हि हि वि वि ही मी॒(ग्म्॒) हि वि वि हीम् । \newline
42. ही मी॒(ग्म्॒) हि ही मि॒द्ध इ॒द्ध ई॒(ग्म्॒) हि ही मि॒द्धः । \newline
43. ई॒ मि॒द्ध इ॒द्ध ई॑ मी मि॒द्धो अख्य॒ दख्य॑दि॒द्ध ई॑ मी मि॒द्धो अख्य॑त् । \newline
44. इ॒द्धो अख्य॒दख्य॑ दि॒द्ध इ॒द्धो अख्य॒दा ऽख्य॑ दि॒द्ध इ॒द्धो अख्य॒दा । \newline
45. अख्य॒दा ऽख्य॒दख्य॒दा रोद॑सी॒ रोद॑सी॒ आ ऽख्य॒दख्य॒दा रोद॑सी । \newline
46. आ रोद॑सी॒ रोद॑सी॒ आ रोद॑सी भा॒नुना॑ भा॒नुना॒ रोद॑सी॒ आ रोद॑सी भा॒नुना᳚ । \newline
47. रोद॑सी भा॒नुना॑ भा॒नुना॒ रोद॑सी॒ रोद॑सी भा॒नुना॑ भाति भाति भा॒नुना॒ रोद॑सी॒ रोद॑सी भा॒नुना॑ भाति । \newline
48. रोद॑सी॒ इति॒ रोद॑सी । \newline
49. भा॒नुना॑ भाति भाति भा॒नुना॑ भा॒नुना॑ भात्य॒न्त र॒न्तर् भा॑ति भा॒नुना॑ भा॒नुना॑ भात्य॒न्तः । \newline
50. भा॒त्य॒न्त र॒न्तर् भा॑ति भात्य॒न्तः । \newline
51. अ॒न्तरित्य॒न्तः । \newline
52. त्वे वसू॑नि॒ वसू॑नि॒ त्वे त्वे वसू॑नि पुर्वणीक पुर्वणीक॒ वसू॑नि॒ त्वे त्वे वसू॑नि पुर्वणीक । \newline
53. त्वे इति॒ त्वे । \newline
54. वसू॑नि पुर्वणीक पुर्वणीक॒ वसू॑नि॒ वसू॑नि पुर्वणीक होतर्. होतः पुर्वणीक॒ वसू॑नि॒ वसू॑नि पुर्वणीक होतः । \newline
55. पु॒र्व॒णी॒क॒ हो॒त॒र्॒. हो॒तः॒ पु॒र्व॒णी॒क॒ पु॒र्व॒णी॒क॒ हो॒त॒र् दो॒षा दो॒षा हो॑तः पुर्वणीक पुर्वणीक होतर् दो॒षा । \newline
56. पु॒र्व॒णी॒केति॑ पुरु - अ॒नी॒क॒ । \newline
\pagebreak
\markright{ TS 1.3.14.3  \hfill https://www.vedavms.in \hfill}

\section{ TS 1.3.14.3 }

\textbf{TS 1.3.14.3 } \newline
\textbf{Samhita Paata} \newline

होतर्दो॒षा वस्तो॒रेरि॑रे य॒ज्ञिया॑सः । क्षामे॑व॒ विश्वा॒ भुव॑नानि॒ यस्मि॒न्थ् सꣳ सौभ॑गानि दधि॒रे पा॑व॒के ॥ तुभ्यं॒ ता अ॑ङ्गिरस्तम॒ विश्वाः᳚ सुक्षि॒तयः॒ पृथ॑क् । अग्ने॒ कामा॑य येमिरे ॥ अ॒श्याम॒ तं काम॑मग्ने॒ तवो॒त्य॑श्याम॑ र॒यिꣳ र॑यिवः सु॒वीरं᳚ । अ॒श्याम॒ वाज॑म॒भि वा॒जय॑न्तो॒ ऽश्याम॑ द्यु॒म्नम॑जरा॒जरं॑ ते ॥श्रेष्ठं॑ ॅयविष्ठ भार॒ताग्ने᳚ द्यु॒मन्त॒मा भ॑र । \newline

\textbf{Pada Paata} \newline

हो॒तः॒ । दो॒षा । वस्तोः᳚ । एति॑ । ई॒रि॒रे॒ । य॒ज्ञिया॑सः ॥ क्षाम॑ । इ॒व॒ । विश्वा᳚ । भुव॑नानि । यस्मिन्न्॑ । समिति॑ । सौभ॑गानि । द॒धि॒रे । पा॒व॒के ॥ तुभ्य᳚म् । ताः । अ॒ङ्गि॒र॒स्त॒मेत्य॑ङ्गिरः - त॒म॒ । विश्वाः᳚ । सु॒क्षि॒तय॒ इति॑ सु - क्षि॒तयः॑ । पृथ॑क् ॥ अग्ने᳚ । कामा॑य । ये॒मि॒रे॒ ॥ अ॒श्याम॑ । तम् । काम᳚म् । अ॒ग्ने॒ । तव॑ । ऊ॒ती । अ॒श्याम॑ । र॒यिम् । र॒यि॒व॒ इति॑ रयि - वः॒ । सु॒वीर॒मिति॑ सु - वीर᳚म् ॥ अ॒श्याम॑ । वाज᳚म् । अ॒भीति॑ । वा॒जय॑न्तः । अ॒श्याम॑ । द्यु॒म्नम् । अ॒ज॒र॒ । अ॒जर᳚म् । ते॒ ॥ श्रेष्ठ᳚म् । य॒वि॒ष्ठ॒ । भा॒र॒त॒ । अग्ने᳚ । द्यु॒मन्त॒मिति॑ द्यु - मन्त᳚म् । एति॑ । भ॒र॒ ॥  \newline


\textbf{Krama Paata} \newline

हो॒त॒र् दो॒षा । दो॒षा वस्तोः᳚ । वस्तो॒रा । एरि॑रे । ई॒रि॒रे॒ य॒ज्ञिया॑सः । य॒ज्ञिया॑स॒ इति॑ य॒ज्ञिया॑सः ॥ क्षामे॑व । इ॒व॒ विश्वा᳚ । विश्वा॒ भुव॑नानि । भुव॑नानि॒ यस्मिन्न्॑ । यस्मि॒न्थ् सम् । सꣳ सौभ॑गानि । सौभ॑गानि दधि॒रे । द॒धि॒रे पा॑व॒के । पा॒व॒क इति॑ पाव॒के ॥ तुभ्य॒म् ताः । ता अ॑ङ्गिरस्तम । अ॒ङ्गि॒र॒स्त॒म॒ विश्वाः᳚ । अ॒ङ्गि॒र॒स्त॒मेत्य॑ङ्गिरः - त॒म॒ । विश्वाः᳚ सुक्षि॒तयः॑ । सु॒क्षि॒तयः॒ पृथ॑क् । सु॒क्षि॒तय॒ इति॑ सु - क्षि॒तयः॑ । पृथ॒गिति॒ पृथ॑क् ॥ अग्ने॒ कामा॑य । कामा॑य येमिरे । ये॒मि॒र॒ इति॑ येमिरे ॥ अ॒श्याम॒ तम् । तम् काम᳚म् । काम॑मग्ने । अ॒ग्ने॒ तव॑ । तवो॒ती । ऊ॒त्य॑श्याम॑ । अ॒श्याम॑ र॒यिम् । र॒यिꣳ र॑यिवः । र॒यि॒वः॒ सु॒वीर᳚म् । र॒यि॒व॒ इति॑ रयि - वः॒ । सु॒वीर॒मिति॑ सु - वीर᳚म् ॥ अ॒श्याम॒ वाज᳚म् । वाज॑म॒भि । अ॒भि वा॒जय॑न्तः । वा॒जय॑न्तो॒ऽश्याम॑ । अ॒श्याम॑ द्यु॒म्नम् । द्यु॒म्नम॑जर । अ॒ज॒रा॒जर᳚म् । अ॒जर॑म् ते । त॒ इति॑ ते ॥ श्रेष्ठं॑ ॅयविष्ठ । य॒वि॒ष्ठ॒ भा॒र॒त॒ । भा॒र॒ताग्ने᳚ । अग्ने᳚ द्यु॒मन्त᳚म् । द्यु॒मन्त॒मा । द्यु॒मन्त॒मिति॑ द्यु - मन्त᳚म् । आ भ॑र । भ॒रेति॑ भर । \newline

\textbf{Jatai Paata} \newline

1. हो॒त॒र् दो॒षा दो॒षा हो॑तर्. होतर् दो॒षा । \newline
2. दो॒षा वस्तो॒र् वस्तो᳚र् दो॒षा दो॒षा वस्तोः᳚ । \newline
3. वस्तो॒रा वस्तो॒र् वस्तो॒रा । \newline
4. एरि॑र ईरिर॒ एरि॑रे । \newline
5. ई॒रि॒रे॒ य॒ज्ञिया॑सो य॒ज्ञिया॑स ईरिर ईरिरे य॒ज्ञिया॑सः । \newline
6. य॒ज्ञिया॑स॒ इति॑ य॒ज्ञिया॑सः । \newline
7. क्षामे॑ वे व॒ क्षाम॒ क्षामे॑ व । \newline
8. इ॒व॒ विश्वा॒ विश्वे॑वे व॒ विश्वा᳚ । \newline
9. विश्वा॒ भुव॑नानि॒ भुव॑नानि॒ विश्वा॒ विश्वा॒ भुव॑नानि । \newline
10. भुव॑नानि॒ यस्मि॒न्॒. यस्मि॒न् भुव॑नानि॒ भुव॑नानि॒ यस्मिन्न्॑ । \newline
11. यस्मि॒न् थ्सꣳ सं ॅयस्मि॒न्॒. यस्मि॒न् थ्सम् । \newline
12. सꣳ सौभ॑गानि॒ सौभ॑गानि॒ सꣳ सꣳ सौभ॑गानि । \newline
13. सौभ॑गानि दधि॒रे द॑धि॒रे सौभ॑गानि॒ सौभ॑गानि दधि॒रे । \newline
14. द॒धि॒रे पा॑व॒के पा॑व॒के द॑धि॒रे द॑धि॒रे पा॑व॒के । \newline
15. पा॒व॒क इति॑ पाव॒के । \newline
16. तुभ्य॒म् तास्ता स्तुभ्य॒म् तुभ्य॒म् ताः । \newline
17. ता अ॑ङ्गिरस्त माङ्गिरस्तम॒ ता स्ता अ॑ङ्गिरस्तम । \newline
18. अ॒ङ्गि॒र॒स्त॒म॒ विश्वा॒ विश्वा॑ अङ्गिरस्त माङ्गिरस्तम॒ विश्वाः᳚ । \newline
19. अ॒ङ्गि॒र॒स्त॒मेत्य॑ङ्गिरः - त॒म॒ । \newline
20. विश्वाः᳚ सुक्षि॒तयः॑ सुक्षि॒तयो॒ विश्वा॒ विश्वाः᳚ सुक्षि॒तयः॑ । \newline
21. सु॒क्षि॒तयः॒ पृथ॒क् पृथ॑ख् सुक्षि॒तयः॑ सुक्षि॒तयः॒ पृथ॑क् । \newline
22. सु॒क्षि॒तय॒ इति॑ सु - क्षि॒तयः॑ । \newline
23. पृथ॒गिति॒ पृथ॑क् । \newline
24. अग्ने॒ कामा॑य॒ कामा॒याग्ने ऽग्ने॒ कामा॑य । \newline
25. कामा॑य येमिरे येमिरे॒ कामा॑य॒ कामा॑य येमिरे । \newline
26. ये॒मि॒र॒ इति॑ येमिरे । \newline
27. अ॒श्याम॒ तम् त म॒श्या मा॒श्याम॒ तम् । \newline
28. तम् काम॒म् काम॒म् तम् तम् काम᳚म् । \newline
29. काम॑ मग्ने अग्ने॒ काम॒म् काम॑ मग्ने । \newline
30. अ॒ग्ने॒ तव॒ तवा᳚ग्ने अग्ने॒ तव॑ । \newline
31. तवो॒त्यू॑ती तव॒ तवो॒ती । \newline
32. ऊ॒त्य॑श्या मा॒श्यामो॒ त्यू᳚(1॒)त्य॑श्याम॑ । \newline
33. अ॒श्याम॑ र॒यिꣳ र॒यि म॒श्या मा॒श्याम॑ र॒यिम् । \newline
34. र॒यिꣳ र॑यिवो रयिवो र॒यिꣳ र॒यिꣳ र॑यिवः । \newline
35. र॒यि॒वः॒ सु॒वीर(ग्म्॑) सु॒वीर(ग्म्॑) रयिवो रयिवः सु॒वीर᳚म् । \newline
36. र॒यि॒व॒ इति॑ रयि - वः॒ । \newline
37. सु॒वीर॒मिति॑ सु - वीर᳚म् । \newline
38. अ॒श्याम॒ वाजं॒ ॅवाज॑ म॒श्या मा॒श्याम॒ वाज᳚म् । \newline
39. वाज॑ म॒भ्य॑भि वाजं॒ ॅवाज॑ म॒भि । \newline
40. अ॒भि वा॒जय॑न्तो वा॒जय॑न्तो अ॒भ्य॑भि वा॒जय॑न्तः । \newline
41. वा॒जय॑न्तो॒ ऽश्यामा॒श्याम॑ वा॒जय॑न्तो वा॒जय॑न्तो॒ ऽश्याम॑ । \newline
42. अ॒श्याम॑ द्यु॒म्नम् द्यु॒म्न म॒श्या मा॒श्याम॑ द्यु॒म्नम् । \newline
43. द्यु॒म्न म॑जराजर द्यु॒म्नम् द्यु॒म्न म॑जर । \newline
44. अ॒ज॒रा॒जर॑ म॒जर॑ मज राजरा॒जर᳚म् । \newline
45. अ॒जर॑म् ते ते अ॒जर॑ म॒जर॑म् ते । \newline
46. त॒ इति॑ ते । \newline
47. श्रेष्ठं॑ ॅयविष्ठ यविष्ठ॒ श्रेष्ठ॒(ग्ग्॒) श्रेष्ठं॑ ॅयविष्ठ । \newline
48. य॒वि॒ष्ठ॒ भा॒र॒त॒ भा॒र॒त॒ य॒वि॒ष्ठ॒ य॒वि॒ष्ठ॒ भा॒र॒त॒ । \newline
49. भा॒र॒ताग्ने ऽग्ने॑ भारत भार॒ताग्ने᳚ । \newline
50. अग्ने᳚ द्यु॒मन्त॑म् द्यु॒मन्त॒ मग्ने ऽग्ने᳚ द्यु॒मन्त᳚म् । \newline
51. द्यु॒मन्त॒ मा द्यु॒मन्त॑म् द्यु॒मन्त॒ मा । \newline
52. द्यु॒मन्त॒मिति॑ द्यु - मन्त᳚म् । \newline
53. आ भ॑र भ॒रा भ॑र । \newline
54. भ॒रेति॑ भर । \newline

\textbf{Ghana Paata } \newline

1. हो॒त॒र् दो॒षा दो॒षा हो॑तर्. होतर् दो॒षा वस्तो॒र् वस्तो᳚र् दो॒षा हो॑तर्. होतर् दो॒षा वस्तोः᳚ । \newline
2. दो॒षा वस्तो॒र् वस्तो᳚र् दो॒षा दो॒षा वस्तो॒रा वस्तो᳚र् दो॒षा दो॒षा वस्तो॒रा । \newline
3. वस्तो॒रा वस्तो॒र् वस्तो॒ रेरि॑र ईरिर॒ आ वस्तो॒र् वस्तो॒ रेरि॑रे । \newline
4. एरि॑र ईरिर॒ एरि॑रे य॒ज्ञिया॑सो य॒ज्ञिया॑स ईरिर॒ एरि॑रे य॒ज्ञिया॑सः । \newline
5. ई॒रि॒रे॒ य॒ज्ञिया॑सो य॒ज्ञिया॑स ईरिर ईरिरे य॒ज्ञिया॑सः । \newline
6. य॒ज्ञिया॑स॒ इति॑ य॒ज्ञिया॑सः । \newline
7. क्षामे॑ वे व॒ क्षाम॒ क्षामे॑ व॒ विश्वा॒ विश्वे॑व॒ क्षाम॒ क्षामे॑ व॒ विश्वा᳚ । \newline
8. इ॒व॒ विश्वा॒ विश्वे॑वे व॒ विश्वा॒ भुव॑नानि॒ भुव॑नानि॒ विश्वे॑वे व॒ विश्वा॒ भुव॑नानि । \newline
9. विश्वा॒ भुव॑नानि॒ भुव॑नानि॒ विश्वा॒ विश्वा॒ भुव॑नानि॒ यस्मि॒न्॒. यस्मि॒न् भुव॑नानि॒ विश्वा॒ विश्वा॒ भुव॑नानि॒ यस्मिन्न्॑ । \newline
10. भुव॑नानि॒ यस्मि॒न्.॒ यस्मि॒न् भुव॑नानि॒ भुव॑नानि॒ यस्मि॒न् थ्सꣳ सं ॅयस्मि॒न् भुव॑नानि॒ भुव॑नानि॒ यस्मि॒न् थ्सम् । \newline
11. यस्मि॒न् थ्सꣳ सं ॅयस्मि॒न्.॒ यस्मि॒न् थ्सꣳ सौभ॑गानि॒ सौभ॑गानि॒ सं ॅयस्मि॒न्॒. यस्मि॒न् थ्सꣳ सौभ॑गानि । \newline
12. सꣳ सौभ॑गानि॒ सौभ॑गानि॒ सꣳ सꣳ सौभ॑गानि दधि॒रे द॑धि॒रे सौभ॑गानि॒ सꣳ सꣳ सौभ॑गानि दधि॒रे । \newline
13. सौभ॑गानि दधि॒रे द॑धि॒रे सौभ॑गानि॒ सौभ॑गानि दधि॒रे पा॑व॒के पा॑व॒के द॑धि॒रे सौभ॑गानि॒ सौभ॑गानि दधि॒रे पा॑व॒के । \newline
14. द॒धि॒रे पा॑व॒के पा॑व॒के द॑धि॒रे द॑धि॒रे पा॑व॒के । \newline
15. पा॒व॒क इति॑ पाव॒के । \newline
16. तुभ्य॒म् तास्ता स्तुभ्य॒म् तुभ्य॒म् ता अ॑ङ्गिरस्त माङ्गिरस्तम॒ तास्तुभ्य॒म् तुभ्य॒म् ता अ॑ङ्गिरस्तम । \newline
17. ता अ॑ङ्गिरस्त माङ्गिरस्तम॒ तास्ता अ॑ङ्गिरस्तम॒ विश्वा॒ विश्वा॑ अङ्गिरस्तम॒ तास्ता अ॑ङ्गिरस्तम॒ विश्वाः᳚ । \newline
18. अ॒ङ्गि॒र॒स्त॒म॒ विश्वा॒ विश्वा॑ अङ्गिरस्त माङ्गिरस्तम॒ विश्वाः᳚ सुक्षि॒तयः॑ सुक्षि॒तयो॒ विश्वा॑ अङ्गिरस्त माङ्गिरस्तम॒ विश्वाः᳚ सुक्षि॒तयः॑ । \newline
19. अ॒ङ्गि॒र॒स्त॒मेत्य॑ङ्गिरः - त॒म॒ । \newline
20. विश्वाः᳚ सुक्षि॒तयः॑ सुक्षि॒तयो॒ विश्वा॒ विश्वाः᳚ सुक्षि॒तयः॒ पृथ॒क् पृथ॑ख् सुक्षि॒तयो॒ विश्वा॒ विश्वाः᳚ सुक्षि॒तयः॒ पृथ॑क् । \newline
21. सु॒क्षि॒तयः॒ पृथ॒क् पृथ॑ख् सुक्षि॒तयः॑ सुक्षि॒तयः॒ पृथ॑क् । \newline
22. सु॒क्षि॒तय॒ इति॑ सु - क्षि॒तयः॑ । \newline
23. पृथ॒गिति॒ पृथ॑क् । \newline
24. अग्ने॒ कामा॑य॒ कामा॒याग्ने ऽग्ने॒ कामा॑य येमिरे येमिरे॒ कामा॒याग्ने ऽग्ने॒ कामा॑य येमिरे । \newline
25. कामा॑य येमिरे येमिरे॒ कामा॑य॒ कामा॑य येमिरे । \newline
26. ये॒मि॒र॒ इति॑ येमिरे । \newline
27. अ॒श्याम॒ तम् त म॒श्या मा॒श्याम॒ तम् काम॒म् काम॒म् त म॒श्या मा॒श्याम॒ तम् काम᳚म् । \newline
28. तम् काम॒म् काम॒म् तम् तम् काम॑ मग्ने अग्ने॒ काम॒म् तम् तम् काम॑ मग्ने । \newline
29. काम॑ मग्ने अग्ने॒ काम॒म् काम॑ मग्ने॒ तव॒ तवा᳚ग्ने॒ काम॒म् काम॑ मग्ने॒ तव॑ । \newline
30. अ॒ग्ने॒ तव॒ तवा᳚ग्ने अग्ने॒ तवो॒त्यू॑ती तवा᳚ग्ने अग्ने॒ तवो॒ती । \newline
31. तवो॒त्यू॑ती तव॒ तवो॒ त्य॑श्या मा॒श्यामो॒ती तव॒ तवो॒त्य॑श्याम॑ । \newline
32. ऊ॒त्य॑श्या मा॒श्या मो॒त्यू᳚(1॒)त्य॑श्याम॑ र॒यिꣳ र॒यि म॒श्या मो॒त्यू᳚(1॒)त्य॑श्याम॑ र॒यिम् । \newline
33. अ॒श्याम॑ र॒यिꣳ र॒यि म॒श्या मा॒श्याम॑ र॒यिꣳ र॑यिवो रयिवो र॒यि म॒श्या मा॒श्याम॑ र॒यिꣳ र॑यिवः । \newline
34. र॒यिꣳ र॑यिवो रयिवो र॒यिꣳ र॒यिꣳ र॑यिवः सु॒वीर(ग्म्॑) सु॒वीर(ग्म्॑) रयिवो र॒यिꣳ र॒यिꣳ र॑यिवः सु॒वीर᳚म् । \newline
35. र॒यि॒वः॒ सु॒वीर(ग्म्॑) सु॒वीर(ग्म्॑) रयिवो रयिवः सु॒वीर᳚म् । \newline
36. र॒यि॒व॒ इति॑ रयि - वः॒ । \newline
37. सु॒वीर॒मिति॑ सु - वीर᳚म् । \newline
38. अ॒श्याम॒ वाजं॒ ॅवाज॑ म॒श्या मा॒श्याम॒ वाज॑ म॒भ्य॑भि वाज॑ म॒श्या मा॒श्याम॒ वाज॑ म॒भि । \newline
39. वाज॑ म॒भ्य॑भि वाजं॒ ॅवाज॑ म॒भि वा॒जय॑न्तो वा॒जय॑न्तो अ॒भि वाजं॒ ॅवाज॑ म॒भि वा॒जय॑न्तः । \newline
40. अ॒भि वा॒जय॑न्तो वा॒जय॑न्तो अ॒भ्य॑भि वा॒जय॑न्तो॒ ऽश्यामा॒श्याम॑ वा॒जय॑न्तो अ॒भ्य॑भि वा॒जय॑न्तो॒ ऽश्याम॑ । \newline
41. वा॒जय॑न्तो॒ ऽश्यामा॒श्याम॑ वा॒जय॑न्तो वा॒जय॑न्तो॒ ऽश्याम॑ द्यु॒म्नम् द्यु॒म्न म॒श्याम॑ वा॒जय॑न्तो वा॒जय॑न्तो॒ ऽश्याम॑ द्यु॒म्नम् । \newline
42. अ॒श्याम॑ द्यु॒म्नम् द्यु॒म्न म॒श्या मा॒श्याम॑ द्यु॒म्न म॑जराजर द्यु॒म्न म॒श्या मा॒श्याम॑ द्यु॒म्न म॑जर । \newline
43. द्यु॒म्न म॑जराजर द्यु॒म्नम् द्यु॒म्न म॑जरा॒जर॑ म॒जर॑ मजर द्यु॒म्नम् द्यु॒म्न म॑जरा॒जर᳚म् । \newline
44. अ॒ज॒रा॒जर॑ म॒जर॑ मज राजरा॒जर॑म् ते ते अ॒जर॑ मज राजरा॒जर॑म् ते । \newline
45. अ॒जर॑म् ते ते अ॒जर॑ म॒जर॑म् ते । \newline
46. त॒ इति॑ ते । \newline
47. श्रेष्ठं॑ ॅयविष्ठ यविष्ठ॒ श्रेष्ठ॒(ग्ग्॒) श्रेष्ठं॑ ॅयविष्ठ भारत भारत यविष्ठ॒ श्रेष्ठ॒(ग्ग्॒) श्रेष्ठं॑ ॅयविष्ठ भारत । \newline
48. य॒वि॒ष्ठ॒ भा॒र॒त॒ भा॒र॒त॒ य॒वि॒ष्ठ॒ य॒वि॒ष्ठ॒ भा॒र॒ताग्ने ऽग्ने॑ भारत यविष्ठ यविष्ठ भार॒ताग्ने᳚ । \newline
49. भा॒र॒ताग्ने ऽग्ने॑ भारत भार॒ताग्ने᳚ द्यु॒मन्त॑म् द्यु॒मन्त॒ मग्ने॑ भारत भार॒ताग्ने᳚ द्यु॒मन्त᳚म् । \newline
50. अग्ने᳚ द्यु॒मन्त॑म् द्यु॒मन्त॒ मग्ने ऽग्ने᳚ द्यु॒मन्त॒ मा द्यु॒मन्त॒ मग्ने ऽग्ने᳚ द्यु॒मन्त॒ मा । \newline
51. द्यु॒मन्त॒ मा द्यु॒मन्त॑म् द्यु॒मन्त॒ मा भ॑र भ॒रा द्यु॒मन्त॑म् द्यु॒मन्त॒ मा भ॑र । \newline
52. द्यु॒मन्त॒मिति॑ द्यु - मन्त᳚म् । \newline
53. आ भ॑र भ॒रा भ॑र । \newline
54. भ॒रेति॑ भर । \newline
\pagebreak
\markright{ TS 1.3.14.4  \hfill https://www.vedavms.in \hfill}

\section{ TS 1.3.14.4 }

\textbf{TS 1.3.14.4 } \newline
\textbf{Samhita Paata} \newline

वसो॑ पुरु॒स्पृहꣳ॑ र॒यिं ॥ स श्वि॑ता॒नस्त॑न्य॒तू रो॑चन॒स्था अ॒जरे॑भि॒र् नान॑दद्भि॒र्यवि॑ष्ठः । यः पा॑व॒कः पु॑रु॒तमः॑ पु॒रूणि॑ पृ॒थून्य॒ग्निर॑नु॒याति॒ भर्वन्न्॑ ॥ आयु॑ष्टे वि॒श्वतो॑ दधद॒यम॒ग्निर् वरे᳚ण्यः । पुन॑स्ते प्रा॒ण आऽय॑ति॒ परा॒ यक्ष्मꣳ॑ सुवामि ते ॥ आ॒यु॒र्दा अ॑ग्ने ह॒विषो॑ जुषा॒णो घृ॒तप्र॑तीको घृ॒तयो॑निरेधि । घृ॒तं पी॒त्वा मधु॒ चारु॒ गव्यं॑ पि॒तेव॑ पु॒त्रम॒भि - [ ] \newline

\textbf{Pada Paata} \newline

वसो॒ इति॑ । पु॒रु॒स्पृह॒मिति॑ पुरु - स्पृह᳚म् । र॒यिम् ॥ सः । श्वि॒ता॒नः । त॒न्य॒तुः । रो॒च॒न॒स्था इति॑ रोचन - स्थाः । अ॒जरे॑भिः । नान॑दद्भि॒रिति॒ नान॑दत् - भिः॒ । यवि॑ष्ठः ॥ यः । पा॒व॒कः । पु॒रु॒तम॒ इति॑ पुरु - तमः॑ । पु॒रूणि॑ । पृ॒थूनि॑ । अ॒ग्निः । अ॒नु॒यातीत्य॑नु - याति॑ । भर्वन्न्॑ ॥ आयुः॑ । ते॒ । वि॒श्वतः॑ । द॒ध॒त् । अ॒यम् । अ॒ग्निः । वर᳚ण्यः ॥ पुनः॑ । ते॒ । प्रा॒ण इति॑ प्र - अ॒नः । एति॑ । अ॒य॒ति॒ । परेति॑ । यक्ष्म᳚म् । सु॒वा॒मि॒ । ते॒ ॥ आ॒यु॒र्दा इत्या॑युः - दाः । अ॒ग्ने॒ । ह॒विषः॑ । जु॒षा॒णः । घृ॒तप्र॑तीक॒ इति॑ घृ॒त - प्र॒ती॒कः॒ । घृ॒तयो॑नि॒रिति॑ घृ॒त - यो॒निः॒ । ए॒धि॒ ॥ घृ॒तम् । पी॒त्वा । मधु॑ । चारु॑ । गव्य᳚म् । पि॒ता । इ॒व॒ । पु॒त्रम् । अ॒भीति॑ ।  \newline


\textbf{Krama Paata} \newline

वसो॑ पुरु॒स्पृह᳚म् । वसो॒ इति॒ वसो᳚ । पु॒रु॒स्पृहꣳ॑ र॒यिम् । पु॒रु॒स्पृह॒मिति॑ पुरु - स्पृह᳚म् । र॒यिमिति॑ र॒यिम् ॥ स श्वि॑ता॒नः । श्वि॒ता॒नस्त॑न्य॒तुः । त॒न्य॒तू रो॑चन॒स्थाः । रो॒च॒न॒स्था अ॒जरे॑भिः । रो॒च॒न॒स्था इति॑ रोचन - स्थाः । अ॒जरे॑भि॒र् नान॑दद्भिः । नान॑दद्भि॒र् यवि॑ष्ठः । नान॑दद्भि॒रिति॒ नान॑दत् - भिः॒ । यवि॑ष्ठ॒ इति॒ यवि॑ष्ठः ॥ यः पा॑व॒कः । पा॒व॒कः पु॑रु॒तमः॑ । पु॒रु॒तमः॑ पु॒रूणि॑ । पु॒रु॒तम॒ इति॑ पुरु - तमः॑ । पु॒रूणि॑ पृ॒थूनि॑ । पृ॒थून्य॒ग्निः । अ॒ग्निर॑नु॒याति॑ । अ॒नु॒याति॒ भर्वन्न्॑ । अ॒नु॒यातीत्य॑नु - याति॑ । भर्व॒न्निति॒ भर्वन्न्॑ ॥ आयु॑ष्टे । ते॒ वि॒श्वतः॑ । वि॒श्वतो॑ दधत् । द॒ध॒द॒यम् । अ॒यम॒ग्निः । अ॒ग्निर् वरे᳚ण्यः । वरे᳚ण्य॒ इति॒ वरे᳚ण्यः ॥ पुन॑स्ते । ते॒ प्रा॒णः । प्रा॒ण आ । प्रा॒ण इति॑ प्र - अ॒नः । आऽय॑ति । अ॒य॒ति॒ परा᳚ । परा॒ यक्ष्म᳚म् । यक्ष्मꣳ॑ सुवामि । सु॒वा॒मि॒ ते॒ । त॒ इति॑ ते ॥ आ॒यु॒र्दा अ॑ग्ने । आ॒यु॒र्दा इत्या॑युः - दाः । अ॒ग्ने॒ ह॒विषः॑ । ह॒विषो॑ जुषा॒णः । जु॒षा॒णो घृ॒तप्र॑तीकः । घृ॒तप्र॑तीको घृ॒तयो॑निः । घृ॒तप्र॑तीक॒ इति॑ घृ॒त - प्र॒ती॒कः॒ । घृ॒तयो॑निरेधि । घृ॒तयो॑नि॒रिति॑ घृ॒त - यो॒निः॒ । ए॒धीत्ये॑धि ॥ घृ॒तम् पी॒त्वा । पी॒त्वा मधु॑ । मधु॒ चारु॑ । चारु॒ गव्य᳚म् । गव्य॑म् पि॒ता । पि॒तेव॑ । इ॒व॒ पु॒त्रम् । पु॒त्रम॒भि । अ॒भि र॑क्षतात् \newline

\textbf{Jatai Paata} \newline

1. वसो॑ पुरु॒स्पृह॑म् पुरु॒स्पृहं॒ ॅवसो॒ वसो॑ पुरु॒स्पृह᳚म् । \newline
2. वसो॒ इति॒ वसो᳚ । \newline
3. पु॒रु॒स्पृह(ग्म्॑) र॒यिꣳ र॒यिम् पु॑रु॒स्पृह॑म् पुरु॒स्पृह(ग्म्॑) र॒यिम् । \newline
4. पु॒रु॒स्पृह॒मिति॑ पुरु - स्पृह᳚म् । \newline
5. र॒यिमिति॑ रयिम् । \newline
6. स श्वि॑ता॒नः श्वि॑ता॒नः स स श्वि॑ता॒नः । \newline
7. श्वि॒ता॒न स्त॑न्य॒तु स्त॑न्य॒तुः श्वि॑ता॒नः श्वि॑ता॒न स्त॑न्य॒तुः । \newline
8. त॒न्य॒तू रो॑चन॒स्था रो॑चन॒स्था स्त॑न्य॒तु स्त॑न्य॒तू रो॑चन॒स्थाः । \newline
9. रो॒च॒न॒स्था अ॒जरे॑भि र॒जरे॑भी रोचन॒स्था रो॑चन॒स्था अ॒जरे॑भिः । \newline
10. रो॒च॒न॒स्था इति॑ रोचन - स्थाः । \newline
11. अ॒जरे॑भि॒र् नान॑दद्भि॒र् नान॑दद्भि र॒जरे॑भि र॒जरे॑भि॒र् नान॑दद्भिः । \newline
12. नान॑दद्भि॒र् यवि॑ष्ठो॒ यवि॑ष्ठो॒ नान॑दद्भि॒र् नान॑दद्भि॒र् यवि॑ष्ठः । \newline
13. नान॑दद्भि॒रिति॒ नान॑दत् - भिः॒ । \newline
14. यवि॑ष्ठ॒ इति॒ यवि॑ष्ठः । \newline
15. यः पा॑व॒कः पा॑व॒को यो यः पा॑व॒कः । \newline
16. पा॒व॒कः पु॑रु॒तमः॑ पुरु॒तमः॑ पाव॒कः पा॑व॒कः पु॑रु॒तमः॑ । \newline
17. पु॒रु॒तमः॑ पु॒रूणि॑ पु॒रूणि॑ पुरु॒तमः॑ पुरु॒तमः॑ पु॒रूणि॑ । \newline
18. पु॒रु॒तम॒ इति॑ पुरु - तमः॑ । \newline
19. पु॒रूणि॑ पृ॒थूनि॑ पृ॒थूनि॑ पु॒रूणि॑ पु॒रूणि॑ पृ॒थूनि॑ । \newline
20. पृ॒थून्य॒ग्नि र॒ग्निः पृ॒थूनि॑ पृ॒थून्य॒ग्निः । \newline
21. अ॒ग्नि र॑नु॒या त्य॑नु॒या त्य॒ग्नि र॒ग्नि र॑नु॒याति॑ । \newline
22. अ॒नु॒याति॒ भर्व॒न् भर्व॑न् ननु॒यात्य॑नु॒याति॒ भर्वन्न्॑ । \newline
23. अ॒नु॒यातीत्य॑नु - याति॑ । \newline
24. भर्व॒न्निति॒ भर्वन्न्॑ । \newline
25. आयु॑ष्टे त॒ आयु॒ रायु॑ष्टे । \newline
26. ते॒ वि॒श्वतो॑ वि॒श्वत॑ स्ते ते वि॒श्वतः॑ । \newline
27. वि॒श्वतो॑ दधद् दधद् वि॒श्वतो॑ वि॒श्वतो॑ दधत् । \newline
28. द॒ध॒ द॒य म॒यम् द॑धद् दध द॒यम् । \newline
29. अ॒य म॒ग्नि र॒ग्नि र॒य म॒य म॒ग्निः । \newline
30. अ॒ग्निर् वरे᳚ण्यो॒ वरे᳚ण्यो अ॒ग्नि र॒ग्निर् वरे᳚ण्यः । \newline
31. वरे᳚ण्य॒ इति॒ वरे᳚ण्यः । \newline
32. पुन॑ स्ते ते॒ पुनः॒ पुन॑ स्ते । \newline
33. ते॒ प्रा॒णः प्रा॒ण स्ते॑ ते प्रा॒णः । \newline
34. प्रा॒ण आ प्रा॒णः प्रा॒ण आ । \newline
35. प्रा॒ण इति॑ प्र - अ॒नः । \newline
36. आ ऽय॑त्यय॒त्या ऽय॑ति । \newline
37. अ॒य॒ति॒ परा॒ परा॑ ऽयत्ययति॒ परा᳚ । \newline
38. परा॒ यक्ष्मं॒ ॅयक्ष्म॒म् परा॒ परा॒ यक्ष्म᳚म् । \newline
39. यक्ष्म(ग्म्॑) सुवामि सुवामि॒ यक्ष्मं॒ ॅयक्ष्म(ग्म्॑) सुवामि । \newline
40. सु॒वा॒मि॒ ते॒ ते॒ सु॒वा॒मि॒ सु॒वा॒मि॒ ते॒ । \newline
41. त॒ इति॑ ते । \newline
42. आ॒यु॒र्दा अ॑ग्ने अग्न आयु॒र्दा आ॑यु॒र्दा अ॑ग्ने । \newline
43. आ॒यु॒र्दा इत्या॑युः - दाः । \newline
44. अ॒ग्ने॒ ह॒विषो॑ ह॒विषो॑ अग्ने अग्ने ह॒विषः॑ । \newline
45. ह॒विषो॑ जुषा॒णो जु॑षा॒णो ह॒विषो॑ ह॒विषो॑ जुषा॒णः । \newline
46. जु॒षा॒णो घृ॒तप्र॑तीको घृ॒तप्र॑तीको जुषा॒णो जु॑षा॒णो घृ॒तप्र॑तीकः । \newline
47. घृ॒तप्र॑तीको घृ॒तयो॑निर् घृ॒तयो॑निर् घृ॒तप्र॑तीको घृ॒तप्र॑तीको घृ॒तयो॑निः । \newline
48. घृ॒तप्र॑तीक॒ इति॑ घृ॒त - प्र॒ती॒कः॒ । \newline
49. घृ॒तयो॑नि रेध्येधि घृ॒तयो॑निर् घृ॒तयो॑नि रेधि । \newline
50. घृ॒तयो॑नि॒रिति॑ घृ॒त - यो॒निः॒ । \newline
51. ए॒धीत्ये॑धि । \newline
52. घृ॒तम् पी॒त्वा पी॒त्वा घृ॒तम् घृ॒तम् पी॒त्वा । \newline
53. पी॒त्वा मधु॒ मधु॑ पी॒त्वा पी॒त्वा मधु॑ । \newline
54. मधु॒ चारु॒ चारु॒ मधु॒ मधु॒ चारु॑ । \newline
55. चारु॒ गव्य॒म् गव्य॒म् चारु॒ चारु॒ गव्य᳚म् । \newline
56. गव्य॑म् पि॒ता पि॒ता गव्य॒म् गव्य॑म् पि॒ता । \newline
57. पि॒तेवे॑ व पि॒ता पि॒तेव॑ । \newline
58. इ॒व॒ पु॒त्रम् पु॒त्र मि॑वे व पु॒त्रम् । \newline
59. पु॒त्र म॒भ्य॑भि पु॒त्रम् पु॒त्र म॒भि । \newline
60. अ॒भि र॑क्षताद् रक्षता द॒भ्य॑भि र॑क्षतात् । \newline

\textbf{Ghana Paata } \newline

1. वसो॑ पुरु॒स्पृह॑म् पुरु॒स्पृहं॒ ॅवसो॒ वसो॑ पुरु॒स्पृह(ग्म्॑) र॒यिꣳ र॒यिम् पु॑रु॒स्पृहं॒ ॅवसो॒ वसो॑ पुरु॒स्पृह(ग्म्॑) र॒यिम् । \newline
2. वसो॒ इति॒ वसो᳚ । \newline
3. पु॒रु॒स्पृह(ग्म्॑) र॒यिꣳ र॒यिम् पु॑रु॒स्पृह॑म् पुरु॒स्पृह(ग्म्॑) र॒यिम् । \newline
4. पु॒रु॒स्पृह॒मिति॑ पुरु - स्पृह᳚म् । \newline
5. र॒यिमिति॑ रयिम् । \newline
6. स श्वि॑ता॒नः श्वि॑ता॒नः स स श्वि॑ता॒न स्त॑न्य॒तु स्त॑न्य॒तुः श्वि॑ता॒नः स स श्वि॑ता॒न स्त॑न्य॒तुः । \newline
7. श्वि॒ता॒न स्त॑न्य॒तु स्त॑न्य॒तुः श्वि॑ता॒नः श्वि॑ता॒न स्त॑न्य॒तू रो॑चन॒स्था रो॑चन॒स्था स्त॑न्य॒तुः श्वि॑ता॒नः श्वि॑ता॒न स्त॑न्य॒तू रो॑चन॒स्थाः । \newline
8. त॒न्य॒तू रो॑चन॒स्था रो॑चन॒स्था स्त॑न्य॒तु स्त॑न्य॒तू रो॑चन॒स्था अ॒जरे॑भि र॒जरे॑भी रोचन॒स्था स्त॑न्य॒तु स्त॑न्य॒तू रो॑चन॒स्था अ॒जरे॑भिः । \newline
9. रो॒च॒न॒स्था अ॒जरे॑भि र॒जरे॑भी रोचन॒स्था रो॑चन॒स्था अ॒जरे॑भि॒र् नान॑दद्भि॒र् नान॑दद्भि र॒जरे॑भी रोचन॒स्था रो॑चन॒स्था अ॒जरे॑भि॒र् नान॑दद्भिः । \newline
10. रो॒च॒न॒स्था इति॑ रोचन - स्थाः । \newline
11. अ॒जरे॑भि॒र् नान॑दद्भि॒र् नान॑दद्भि र॒जरे॑भि र॒जरे॑भि॒र् नान॑दद्भि॒र् यवि॑ष्ठो॒ यवि॑ष्ठो॒ नान॑दद्भि र॒जरे॑भि र॒जरे॑भि॒र् नान॑दद्भि॒र् यवि॑ष्ठः । \newline
12. नान॑दद्भि॒र् यवि॑ष्ठो॒ यवि॑ष्ठो॒ नान॑दद्भि॒र् नान॑दद्भि॒र् यवि॑ष्ठः । \newline
13. नान॑दद्भि॒रिति॒ नान॑दत् - भिः॒ । \newline
14. यवि॑ष्ठ॒ इति॒ यवि॑ष्ठः । \newline
15. यः पा॑व॒कः पा॑व॒को यो यः पा॑व॒कः पु॑रु॒तमः॑ पुरु॒तमः॑ पाव॒को यो यः पा॑व॒कः पु॑रु॒तमः॑ । \newline
16. पा॒व॒कः पु॑रु॒तमः॑ पुरु॒तमः॑ पाव॒कः पा॑व॒कः पु॑रु॒तमः॑ पु॒रूणि॑ पु॒रूणि॑ पुरु॒तमः॑ पाव॒कः पा॑व॒कः पु॑रु॒तमः॑ पु॒रूणि॑ । \newline
17. पु॒रु॒तमः॑ पु॒रूणि॑ पु॒रूणि॑ पुरु॒तमः॑ पुरु॒तमः॑ पु॒रूणि॑ पृ॒थूनि॑ पृ॒थूनि॑ पु॒रूणि॑ पुरु॒तमः॑ पुरु॒तमः॑ पु॒रूणि॑ पृ॒थूनि॑ । \newline
18. पु॒रु॒तम॒ इति॑ पुरु - तमः॑ । \newline
19. पु॒रूणि॑ पृ॒थूनि॑ पृ॒थूनि॑ पु॒रूणि॑ पु॒रूणि॑ पृ॒थून्य॒ग्नि र॒ग्निः पृ॒थूनि॑ पु॒रूणि॑ पु॒रूणि॑ पृ॒थून्य॒ग्निः । \newline
20. पृ॒थून्य॒ग्नि र॒ग्निः पृ॒थूनि॑ पृ॒थून्य॒ग्नि र॑नु॒या त्य॑नु॒यात्य॒ग्निः पृ॒थूनि॑ पृ॒थून्य॒ग्नि र॑नु॒याति॑ । \newline
21. अ॒ग्नि र॑नु॒या त्य॑नु॒यात्य॒ग्नि र॒ग्नि र॑नु॒याति॒ भर्व॒न् भर्व॑न् ननु॒या त्य॒ग्नि र॒ग्नि र॑नु॒याति॒ भर्वन्न्॑ । \newline
22. अ॒नु॒याति॒ भर्व॒न् भर्व॑न् ननु॒या त्य॑नु॒याति॒ भर्वन्न्॑ । \newline
23. अ॒नु॒यातीत्य॑नु - याति॑ । \newline
24. भर्व॒न्निति॒ भर्वन्न्॑ । \newline
25. आयु॑ष्टे त॒ आयु॒ रायु॑ष्टे वि॒श्वतो॑ वि॒श्वत॑स्त॒ आयु॒ रायु॑ष्टे वि॒श्वतः॑ । \newline
26. ते॒ वि॒श्वतो॑ वि॒श्वत॑स्ते ते वि॒श्वतो॑ दधद् दधद् वि॒श्वत॑स्ते ते वि॒श्वतो॑ दधत् । \newline
27. वि॒श्वतो॑ दधद् दधद् वि॒श्वतो॑ वि॒श्वतो॑ दधद॒य म॒यम् द॑धद् वि॒श्वतो॑ वि॒श्वतो॑ दधद॒यम् । \newline
28. द॒ध॒द॒य म॒यम् द॑धद् दधद॒य म॒ग्नि र॒ग्नि र॒यम् द॑धद् दधद॒य म॒ग्निः । \newline
29. अ॒य म॒ग्नि र॒ग्नि र॒य म॒य म॒ग्निर् वरे᳚ण्यो॒ वरे᳚ण्यो अ॒ग्नि र॒य म॒य म॒ग्निर् वरे᳚ण्यः । \newline
30. अ॒ग्निर् वरे᳚ण्यो॒ वरे᳚ण्यो अ॒ग्नि र॒ग्निर् वरे᳚ण्यः । \newline
31. वरे᳚ण्य॒ इति॒ वरे᳚ण्यः । \newline
32. पुन॑स्ते ते॒ पुनः॒ पुन॑ स्ते प्रा॒णः प्रा॒ण स्ते॒ पुनः॒ पुन॑ स्ते प्रा॒णः । \newline
33. ते॒ प्रा॒णः प्रा॒ण स्ते॑ ते प्रा॒ण आ प्रा॒ण स्ते॑ ते प्रा॒ण आ । \newline
34. प्रा॒ण आ प्रा॒णः प्रा॒ण आ ऽय॑त्यय॒त्या प्रा॒णः प्रा॒ण आ ऽय॑ति । \newline
35. प्रा॒ण इति॑ प्र - अ॒नः । \newline
36. आ ऽय॑त्य य॒त्या ऽय॑ति॒ परा॒ परा॑ ऽय॒त्या ऽय॑ति॒ परा᳚ । \newline
37. अ॒य॒ति॒ परा॒ परा॑ ऽय त्ययति॒ परा॒ यक्ष्मं॒ ॅयक्ष्म॒म् परा॑ ऽय त्ययति॒ परा॒ यक्ष्म᳚म् । \newline
38. परा॒ यक्ष्मं॒ ॅयक्ष्म॒म् परा॒ परा॒ यक्ष्म(ग्म्॑) सुवामि सुवामि॒ यक्ष्म॒म् परा॒ परा॒ यक्ष्म(ग्म्॑) सुवामि । \newline
39. यक्ष्म(ग्म्॑) सुवामि सुवामि॒ यक्ष्मं॒ ॅयक्ष्म(ग्म्॑) सुवामि ते ते सुवामि॒ यक्ष्मं॒ ॅयक्ष्म(ग्म्॑) सुवामि ते । \newline
40. सु॒वा॒मि॒ ते॒ ते॒ सु॒वा॒मि॒ सु॒वा॒मि॒ ते॒ । \newline
41. त॒ इति॑ ते । \newline
42. आ॒यु॒र्दा अ॑ग्ने अग्न आयु॒र्दा आ॑यु॒र्दा अ॑ग्ने ह॒विषो॑ ह॒विषो॑ अग्न आयु॒र्दा आ॑यु॒र्दा अ॑ग्ने ह॒विषः॑ । \newline
43. आ॒यु॒र्दा इत्या॑युः - दाः । \newline
44. अ॒ग्ने॒ ह॒विषो॑ ह॒विषो॑ अग्ने अग्ने ह॒विषो॑ जुषा॒णो जु॑षा॒णो ह॒विषो॑ अग्ने अग्ने ह॒विषो॑ जुषा॒णः । \newline
45. ह॒विषो॑ जुषा॒णो जु॑षा॒णो ह॒विषो॑ ह॒विषो॑ जुषा॒णो घृ॒तप्र॑तीको घृ॒तप्र॑तीको जुषा॒णो ह॒विषो॑ ह॒विषो॑ जुषा॒णो घृ॒तप्र॑तीकः । \newline
46. जु॒षा॒णो घृ॒तप्र॑तीको घृ॒तप्र॑तीको जुषा॒णो जु॑षा॒णो घृ॒तप्र॑तीको घृ॒तयो॑निर् घृ॒तयो॑निर् घृ॒तप्र॑तीको जुषा॒णो जु॑षा॒णो घृ॒तप्र॑तीको घृ॒तयो॑निः । \newline
47. घृ॒तप्र॑तीको घृ॒तयो॑निर् घृ॒तयो॑निर् घृ॒तप्र॑तीको घृ॒तप्र॑तीको घृ॒तयो॑नि रेध्येधि घृ॒तयो॑निर् घृ॒तप्र॑तीको घृ॒तप्र॑तीको घृ॒तयो॑नि रेधि । \newline
48. घृ॒तप्र॑तीक॒ इति॑ घृ॒त - प्र॒ती॒कः॒ । \newline
49. घृ॒तयो॑नि रेध्येधि घृ॒तयो॑निर् घृ॒तयो॑नि रेधि । \newline
50. घृ॒तयो॑नि॒रिति॑ घृ॒त - यो॒निः॒ । \newline
51. ए॒धीत्ये॑धि । \newline
52. घृ॒तम् पी॒त्वा पी॒त्वा घृ॒तम् घृ॒तम् पी॒त्वा मधु॒ मधु॑ पी॒त्वा घृ॒तम् घृ॒तम् पी॒त्वा मधु॑ । \newline
53. पी॒त्वा मधु॒ मधु॑ पी॒त्वा पी॒त्वा मधु॒ चारु॒ चारु॒ मधु॑ पी॒त्वा पी॒त्वा मधु॒ चारु॑ । \newline
54. मधु॒ चारु॒ चारु॒ मधु॒ मधु॒ चारु॒ गव्य॒म् गव्य॒म् चारु॒ मधु॒ मधु॒ चारु॒ गव्य᳚म् । \newline
55. चारु॒ गव्य॒म् गव्य॒म् चारु॒ चारु॒ गव्य॑म् पि॒ता पि॒ता गव्य॒म् चारु॒ चारु॒ गव्य॑म् पि॒ता । \newline
56. गव्य॑म् पि॒ता पि॒ता गव्य॒म् गव्य॑म् पि॒तेवे॑ व पि॒ता गव्य॒म् गव्य॑म् पि॒तेव॑ । \newline
57. पि॒तेवे॑ व पि॒ता पि॒तेव॑ पु॒त्रम् पु॒त्र मि॑व पि॒ता पि॒तेव॑ पु॒त्रम् । \newline
58. इ॒व॒ पु॒त्रम् पु॒त्र मि॑वे व पु॒त्र म॒भ्य॑भि पु॒त्र मि॑वे व पु॒त्र म॒भि । \newline
59. पु॒त्र म॒भ्य॑भि पु॒त्रम् पु॒त्र म॒भि र॑क्षताद् रक्षताद॒भि पु॒त्रम् पु॒त्र म॒भि र॑क्षतात् । \newline
60. अ॒भि र॑क्षताद् रक्षता द॒भ्य॑भि र॑क्षतादि॒म मि॒मꣳ र॑क्षता द॒भ्य॑भि र॑क्षता दि॒मम् । \newline
\pagebreak
\markright{ TS 1.3.14.5  \hfill https://www.vedavms.in \hfill}

\section{ TS 1.3.14.5 }

\textbf{TS 1.3.14.5 } \newline
\textbf{Samhita Paata} \newline

र॑क्षतादि॒मं । तस्मै॑ ते प्रति॒हर्य॑ते॒ जात॑वेदो॒ विच॑र्.षणे । अग्ने॒ जना॑मि सुष्टु॒तिं ॥ दि॒वस्परि॑ प्रथ॒मं ज॑ज्ञे अ॒ग्निर॒स्मद् द्वि॒तीयं॒ परि॑ जा॒तवे॑दाः । तृ॒तीय॑म॒फ्सु नृ॒मणा॒ अज॑स्र॒मिन्धा॑न एनं जरते स्वा॒धीः ॥ शुचिः॑ पावक॒ वन्द्योऽग्ने॑ बृ॒हद्वि रो॑चसे । त्वं घृ॒तेभि॒राहु॑तः ॥ दृ॒शा॒नो रु॒क्म उ॒र्व्या व्य॑द्यौद्-दु॒र्मर्.ष॒मायुः॑ श्रि॒ये रु॑चा॒नः । अ॒ग्निर॒मृतो॑ अभव॒द्वयो॑भि॒र् - [ ] \newline

\textbf{Pada Paata} \newline

र॒क्ष॒ता॒त् । इ॒मम् ॥ तस्मै᳚ । ते॒ । प्र॒ति॒हर्य॑त॒ इति॑ प्रति - हर्य॑ते । जात॑वेद॒ इति॒ जात॑ - वे॒दः॒ । विच॑र्.षण॒ इति॒ वि - च॒र्॒.ष॒णे॒ ॥ अग्ने᳚ । जना॑मि । सु॒ष्टु॒तिमिति॑ सु - स्तु॒तिम् ॥ दि॒वः । परीति॑ । प्र॒थ॒मम् । ज॒ज्ञे॒ । अ॒ग्निः । अ॒स्मत् । द्वि॒तीय᳚म् । परीति॑॑ । जा॒तवे॑दा॒ इति॑ जा॒त - वे॒दाः॒ ॥ तृ॒तीय᳚म् । अ॒फ्स्वित्य॑प् - सु । नृ॒मणा॒ इति॑ नृ - मनाः᳚ । अज॑स्रम् । इन्धा॑नः । ए॒न॒म् । ज॒र॒ते॒ । स्वा॒धीरिति॑ स्व - धीः ॥ शुचिः॑ । पा॒व॒क॒ । वन्द्यः॑ । अग्ने᳚ । बृ॒हत् । वीति॑ । रो॒च॒से॒ ॥ त्वम् । घृ॒तेभिः॑ । आहु॑त॒ इत्या - हु॒तः॒ ॥ दृ॒शा॒नः । रु॒क्मः । उ॒र्व्या । वीति॑ । अ॒द्यौ॒त् । दु॒र्मर्.ष॒मिति॑ दुः - मर्.ष᳚म् । आयुः॑ । श्रि॒ये । रु॒चा॒नः ॥ अ॒ग्निः । अ॒मृतः॑ । अ॒भ॒व॒त् । वयो॑भि॒रिति॒ वयः॑ - भिः॒ ।  \newline


\textbf{Krama Paata} \newline

र॒क्ष॒ता॒दि॒मम् । इ॒ममिती॒मम् ॥ तस्मै॑ ते । ते॒ प्र॒ति॒हर्य॑ते । प्र॒ति॒हर्य॑ते॒ जात॑वेदः । प्र॒ति॒हर्य॑त॒ इति॑ प्रति - हर्य॑ते । जात॑वेदो॒ विच॑र्.षणे । जात॑वेद॒ इति॒ जात॑ - वे॒दः॒ । विच॑र्.षण॒ इति॒ वि - च॒र्॒.ष॒णे॒ ॥ अग्ने॒ जना॑मि । जना॑मि सुष्टु॒तिम् । सु॒ष्टु॒तिमिति॑ सु - स्तु॒तिम् ॥ दि॒वस्परि॑ । परि॑ प्रथ॒मम् । प्र॒थ॒मम् ज॑ज्ञे । ज॒ज्ञे॒ अ॒ग्निः । अ॒ग्निर॒स्मत् । अ॒स्मद् द्वि॒तीय᳚म् । द्वि॒तीय॒म् परि॑ । परि॑ जा॒तवे॑दाः । जा॒तवे॑दा॒ इति॑ जा॒त - वे॒दाः॒ ॥ तृ॒तीय॑म॒फ्सु । अ॒फ्सु नृ॒मणाः᳚ । अ॒फ्स्वित्य॑प् - सु । नृ॒मणा॒ अज॑स्रम् । नृ॒मणा॒ इति॑ नृ - मनाः᳚ । अज॑स्र॒मिन्धा॑नः । इन्धा॑न एनम् । ए॒न॒म् ज॒र॒ते॒ । ज॒र॒ते॒ स्वा॒धीः । स्वा॒धीरिति॑ स्व - धीः ॥ शुचिः॑ पावक । पा॒व॒क॒ वन्द्यः॑ । वन्द्योऽग्ने᳚ । अग्ने॑ बृ॒हत् । बृ॒हद् वि । वि रो॑चसे । रो॒च॒स॒ इति॑ रोचसे ॥ त्वम् घृ॒तेभिः॑ । घृ॒तेभि॒राहु॑तः । आहु॑त॒ इत्या - हु॒तः॒ ॥ दृ॒शा॒नो रु॒क्मः । रु॒क्म उ॒र्व्या । उ॒र्व्या वि । व्य॑द्यौत् । अ॒द्यौ॒द् दु॒र्मर्.ष᳚म् । दु॒र्मर्.ष॒मायुः॑ । दु॒र्मर्.ष॒मिति॑ दुः - मर्.ष᳚म् । आयुः॑ श्रि॒ये । श्रि॒ये रु॑चा॒नः । रु॒चा॒न इति॑ रुचा॒नः ॥ 
अ॒ग्निर॒मृतः॑ । अ॒मृतो॑ अभवत् । अ॒भ॒व॒द् वयो॑भिः । वयो॑भि॒र् यत् । वयो॑भि॒रिति॒ वयः॑ - भिः॒ \newline

\textbf{Jatai Paata} \newline

1. र॒क्ष॒ता॒ दि॒म मि॒मꣳ र॑क्षताद् रक्षता दि॒मम् । \newline
2. इ॒ममिती॒मम् । \newline
3. तस्मै॑ ते ते॒ तस्मै॒ तस्मै॑ ते । \newline
4. ते॒ प्र॒ति॒हर्य॑ते प्रति॒हर्य॑ते ते ते प्रति॒हर्य॑ते । \newline
5. प्र॒ति॒हर्य॑ते॒ जात॑वेदो॒ जात॑वेदः प्रति॒हर्य॑ते प्रति॒हर्य॑ते॒ जात॑वेदः । \newline
6. प्र॒ति॒हर्य॑त॒ इति॑ प्रति - हर्य॑ते । \newline
7. जात॑वेदो॒ विच॑र्.षणे॒ विच॑र्.षणे॒ जात॑वेदो॒ जात॑वेदो॒ विच॑र्.षणे । \newline
8. जात॑वेद॒ इति॒ जात॑ - वे॒दः॒ । \newline
9. विच॑र्.षण॒ इति॒ वि - च॒र्॒.ष॒णे॒ । \newline
10. अग्ने॒ जना॑मि॒ जना॒ म्यग्ने ऽग्ने॒ जना॑मि । \newline
11. जना॑मि सुष्टु॒तिꣳ सु॑ष्टु॒तिम् जना॑मि॒ जना॑मि सुष्टु॒तिम् । \newline
12. सु॒ष्टु॒तिमिति॑ सु - स्तु॒तिम् । \newline
13. दि॒व स्परि॒ परि॑ दि॒वो दि॒व स्परि॑ । \newline
14. परि॑ प्रथ॒मम् प्र॑थ॒मम् परि॒ परि॑ प्रथ॒मम् । \newline
15. प्र॒थ॒मम् ज॑ज्ञे जज्ञे प्रथ॒मम् प्र॑थ॒मम् ज॑ज्ञे । \newline
16. ज॒ज्ञे॒ अ॒ग्निर॒ग्निर् ज॑ज्ञे जज्ञे अ॒ग्निः । \newline
17. अ॒ग्नि र॒स्म द॒स्म द॒ग्नि र॒ग्नि र॒स्मत् । \newline
18. अ॒स्मद् द्वि॒तीय॑म् द्वि॒तीय॑ म॒स्म द॒स्मद् द्वि॒तीय᳚म् । \newline
19. द्वि॒तीय॒म् परि॒ परि॑ द्वि॒तीय॑म् द्वि॒तीय॒म् परि॑ । \newline
20. परि॑ जा॒तवे॑दा जा॒तवे॑दाः॒ परि॒ परि॑ जा॒तवे॑दाः । \newline
21. जा॒तवे॑दा॒ इति॑ जा॒त - वे॒दाः॒ । \newline
22. तृ॒तीय॑ म॒फ्स्व॑फ्सु तृ॒तीय॑म् तृ॒तीय॑ म॒फ्सु । \newline
23. अ॒फ्सु नृ॒मणा॑ नृ॒मणा॑ अ॒फ्स्व॑फ्सु नृ॒मणाः᳚ । \newline
24. अ॒फ्स्वित्य॑प् - सु । \newline
25. नृ॒मणा॒ अज॑स्र॒ मज॑स्रम् नृ॒मणा॑ नृ॒मणा॒ अज॑स्रम् । \newline
26. नृ॒मणा॒ इति॑ नृ - मनाः᳚ । \newline
27. अज॑स्र॒ मिन्धा॑न॒ इन्धा॒नो ऽज॑स्र॒ मज॑स्र॒ मिन्धा॑नः । \newline
28. इन्धा॑न एन मेन॒ मिन्धा॑न॒ इन्धा॑न एनम् । \newline
29. ए॒न॒म् ज॒र॒ते॒ ज॒र॒त॒ ए॒न॒ मे॒न॒म् ज॒र॒ते॒ । \newline
30. ज॒र॒ते॒ स्वा॒धीः स्वा॒धीर् ज॑रते जरते स्वा॒धीः । \newline
31. स्वा॒धीरिति॑ स्व - धीः । \newline
32. शुचिः॑ पावक पावक॒ शुचिः॒ शुचिः॑ पावक । \newline
33. पा॒व॒क॒ वन्द्यो॒ वन्द्यः॑ पावक पावक॒ वन्द्यः॑ । \newline
34. वन्द्यो ऽग्ने ऽग्ने॒ वन्द्यो॒ वन्द्यो ऽग्ने᳚ । \newline
35. अग्ने॑ बृ॒हद् बृ॒हदग्ने ऽग्ने॑ बृ॒हत् । \newline
36. बृ॒हद् वि वि बृ॒हद् बृ॒हद् वि । \newline
37. वि रो॑चसे रोचसे॒ वि वि रो॑चसे । \newline
38. रो॒च॒स॒ इति॑ रोचसे । \newline
39. त्वम् घृ॒तेभि॑र् घृ॒तेभि॒ स्त्वम् त्वम् घृ॒तेभिः॑ । \newline
40. घृ॒तेभि॒ राहु॑त॒ आहु॑तो घृ॒तेभि॑र् घृ॒तेभि॒ राहु॑तः । \newline
41. आहु॑त॒ इत्या - हु॒तः॒ । \newline
42. दृ॒शा॒नो रु॒क्मो रु॒क्मो दृ॑शा॒नो दृ॑शा॒नो रु॒क्मः । \newline
43. रु॒क्म उ॒र्व्योर् व्या रु॒क्मो रु॒क्म उ॒र्व्या । \newline
44. उ॒र्व्या वि व्यु॑र्व्योर् व्या वि । \newline
45. व्य॑द्यौ दद्यौ॒द् वि व्य॑द्यौत् । \newline
46. अ॒द्यौ॒द् दु॒र्मर्.ष॑म् दु॒र्मर्.ष॑ मद्यौ दद्यौद् दु॒र्मर्.ष᳚म् । \newline
47. दु॒र्मर्.ष॒ मायु॒ रायु॑र् दु॒र्मर्.ष॑म् दु॒र्मर्.ष॒ मायुः॑ । \newline
48. दु॒र्मर्.ष॒मिति॑ दुः - मर्.ष᳚म् । \newline
49. आयुः॑ श्रि॒ये श्रि॒य आयु॒ रायुः॑ श्रि॒ये । \newline
50. श्रि॒ये रु॑चा॒नो रु॑चा॒नः श्रि॒ये श्रि॒ये रु॑चा॒नः । \newline
51. रु॒चा॒न इति॑ रुचा॒नः । \newline
52. अ॒ग्नि र॒मृतो॑ अ॒मृतो॑ अ॒ग्नि र॒ग्नि र॒मृतः॑ । \newline
53. अ॒मृतो॑ अभव दभव द॒मृतो॑ अ॒मृतो॑ अभवत् । \newline
54. अ॒भ॒व॒द् वयो॑भि॒र् वयो॑भि रभव दभव॒द् वयो॑भिः । \newline
55. वयो॑भि॒र् यद् यद् वयो॑भि॒र् वयो॑भि॒र् यत् । \newline
56. वयो॑भि॒रिति॒ वयः॑ - भिः॒ । \newline

\textbf{Ghana Paata } \newline

1. र॒क्ष॒ता॒ दि॒म मि॒मꣳ र॑क्षताद् रक्षता दि॒मम् । \newline
2. इ॒ममिती॒मम् । \newline
3. तस्मै॑ ते ते॒ तस्मै॒ तस्मै॑ ते प्रति॒हर्य॑ते प्रति॒हर्य॑ते ते॒ तस्मै॒ तस्मै॑ ते प्रति॒हर्य॑ते । \newline
4. ते॒ प्र॒ति॒हर्य॑ते प्रति॒हर्य॑ते ते ते प्रति॒हर्य॑ते॒ जात॑वेदो॒ जात॑वेदः प्रति॒हर्य॑ते ते ते प्रति॒हर्य॑ते॒ जात॑वेदः । \newline
5. प्र॒ति॒हर्य॑ते॒ जात॑वेदो॒ जात॑वेदः प्रति॒हर्य॑ते प्रति॒हर्य॑ते॒ जात॑वेदो॒ विच॑र्.षणे॒ विच॑र्.षणे॒ जात॑वेदः प्रति॒हर्य॑ते प्रति॒हर्य॑ते॒ जात॑वेदो॒ विच॑र्.षणे । \newline
6. प्र॒ति॒हर्य॑त॒ इति॑ प्रति - हर्य॑ते । \newline
7. जात॑वेदो॒ विच॑र्.षणे॒ विच॑र्.षणे॒ जात॑वेदो॒ जात॑वेदो॒ विच॑र्.षणे । \newline
8. जात॑वेद॒ इति॒ जात॑ - वे॒दः॒ । \newline
9. विच॑र्.षण॒ इति॒ वि - च॒र्॒.ष॒णे॒ । \newline
10. अग्ने॒ जना॑मि॒ जना॒म्यग्ने ऽग्ने॒ जना॑मि सुष्टु॒तिꣳ सु॑ष्टु॒तिम् जना॒म्यग्ने ऽग्ने॒ जना॑मि सुष्टु॒तिम् । \newline
11. जना॑मि सुष्टु॒तिꣳ सु॑ष्टु॒तिम् जना॑मि॒ जना॑मि सुष्टु॒तिम् । \newline
12. सु॒ष्टु॒तिमिति॑ सु - स्तु॒तिम् । \newline
13. दि॒व स्परि॒ परि॑ दि॒वो दि॒व स्परि॑ प्रथ॒मम् प्र॑थ॒मम् परि॑ दि॒वो दि॒व स्परि॑ प्रथ॒मम् । \newline
14. परि॑ प्रथ॒मम् प्र॑थ॒मम् परि॒ परि॑ प्रथ॒मम् ज॑ज्ञे जज्ञे प्रथ॒मम् परि॒ परि॑ प्रथ॒मम् ज॑ज्ञे । \newline
15. प्र॒थ॒मम् ज॑ज्ञे जज्ञे प्रथ॒मम् प्र॑थ॒मम् ज॑ज्ञे अ॒ग्नि र॒ग्निर् ज॑ज्ञे प्रथ॒मम् प्र॑थ॒मम् ज॑ज्ञे अ॒ग्निः । \newline
16. ज॒ज्ञे॒ अ॒ग्नि र॒ग्निर् ज॑ज्ञे जज्ञे अ॒ग्नि र॒स्म द॒स्म द॒ग्निर् ज॑ज्ञे जज्ञे अ॒ग्नि र॒स्मत् । \newline
17. अ॒ग्नि र॒स्म द॒स्म द॒ग्नि र॒ग्नि र॒स्मद् द्वि॒तीय॑म् द्वि॒तीय॑ म॒स्म द॒ग्नि र॒ग्नि र॒स्मद् द्वि॒तीय᳚म् । \newline
18. अ॒स्मद् द्वि॒तीय॑म् द्वि॒तीय॑ म॒स्म द॒स्मद् द्वि॒तीय॒म् परि॒ परि॑ द्वि॒तीय॑ म॒स्म द॒स्मद् द्वि॒तीय॒म् परि॑ । \newline
19. द्वि॒तीय॒म् परि॒ परि॑ द्वि॒तीय॑म् द्वि॒तीय॒म् परि॑ जा॒तवे॑दा जा॒तवे॑दाः॒ परि॑ द्वि॒तीय॑म् द्वि॒तीय॒म् परि॑ जा॒तवे॑दाः । \newline
20. परि॑ जा॒तवे॑दा जा॒तवे॑दाः॒ परि॒ परि॑ जा॒तवे॑दाः । \newline
21. जा॒तवे॑दा॒ इति॑ जा॒त - वे॒दाः॒ । \newline
22. तृ॒तीय॑ म॒फ्स्व॑फ्सु तृ॒तीय॑म् तृ॒तीय॑ म॒फ्सु नृ॒मणा॑ नृ॒मणा॑ अ॒फ्सु तृ॒तीय॑म् तृ॒तीय॑ म॒फ्सु नृ॒मणाः᳚ । \newline
23. अ॒फ्सु नृ॒मणा॑ नृ॒मणा॑ अ॒फ्स्व॑फ्सु नृ॒मणा॒ अज॑स्र॒ मज॑स्रम् नृ॒मणा॑ अ॒फ्स्व॑फ्सु नृ॒मणा॒ अज॑स्रम् । \newline
24. अ॒फ्स्वित्य॑प् - सु । \newline
25. नृ॒मणा॒ अज॑स्र॒ मज॑स्रम् नृ॒मणा॑ नृ॒मणा॒ अज॑स्र॒ मिन्धा॑न॒ इन्धा॒नो ऽज॑स्रम् नृ॒मणा॑ नृ॒मणा॒ अज॑स्र॒ मिन्धा॑नः । \newline
26. नृ॒मणा॒ इति॑ नृ - मनाः᳚ । \newline
27. अज॑स्र॒ मिन्धा॑न॒ इन्धा॒नो ऽज॑स्र॒ मज॑स्र॒ मिन्धा॑न एन मेन॒ मिन्धा॒नो ऽज॑स्र॒ मज॑स्र॒ मिन्धा॑न एनम् । \newline
28. इन्धा॑न एन मेन॒ मिन्धा॑न॒ इन्धा॑न एनम् जरते जरत एन॒ मिन्धा॑न॒ इन्धा॑न एनम् जरते । \newline
29. ए॒न॒म् ज॒र॒ते॒ ज॒र॒त॒ ए॒न॒ मे॒न॒म् ज॒र॒ते॒ स्वा॒धीः स्वा॒धीर् ज॑रत एन मेनम् जरते स्वा॒धीः । \newline
30. ज॒र॒ते॒ स्वा॒धीः स्वा॒धीर् ज॑रते जरते स्वा॒धीः । \newline
31. स्वा॒धीरिति॑ स्व - धीः । \newline
32. शुचिः॑ पावक पावक॒ शुचिः॒ शुचिः॑ पावक॒ वन्द्यो॒ वन्द्यः॑ पावक॒ शुचिः॒ शुचिः॑ पावक॒ वन्द्यः॑ । \newline
33. पा॒व॒क॒ वन्द्यो॒ वन्द्यः॑ पावक पावक॒ वन्द्यो ऽग्ने ऽग्ने॒ वन्द्यः॑ पावक पावक॒ वन्द्यो ऽग्ने᳚ । \newline
34. वन्द्यो ऽग्ने ऽग्ने॒ वन्द्यो॒ वन्द्यो ऽग्ने॑ बृ॒हद् बृ॒हदग्ने॒ वन्द्यो॒ वन्द्यो ऽग्ने॑ बृ॒हत् । \newline
35. अग्ने॑ बृ॒हद् बृ॒हदग्ने ऽग्ने॑ बृ॒हद् वि वि बृ॒हदग्ने ऽग्ने॑ बृ॒हद् वि । \newline
36. बृ॒हद् वि वि बृ॒हद् बृ॒हद् वि रो॑चसे रोचसे॒ वि बृ॒हद् बृ॒हद् वि रो॑चसे । \newline
37. वि रो॑चसे रोचसे॒ वि वि रो॑चसे । \newline
38. रो॒च॒स॒ इति॑ रोचसे । \newline
39. त्वम् घृ॒तेभि॑र् घृ॒तेभि॒ स्त्वम् त्वम् घृ॒तेभि॒ राहु॑त॒ आहु॑तो घृ॒तेभि॒ स्त्वम् त्वम् घृ॒तेभि॒ राहु॑तः । \newline
40. घृ॒तेभि॒ राहु॑त॒ आहु॑तो घृ॒तेभि॑र् घृ॒तेभि॒ राहु॑तः । \newline
41. आहु॑त॒ इत्या - हु॒तः॒ । \newline
42. दृ॒शा॒नो रु॒क्मो रु॒क्मो दृ॑शा॒नो दृ॑शा॒नो रु॒क्म उ॒र्व्योर्व्या रु॒क्मो दृ॑शा॒नो दृ॑शा॒नो रु॒क्म उ॒र्व्या । \newline
43. रु॒क्म उ॒र्व्योर्व्या रु॒क्मो रु॒क्म उ॒र्व्या वि व्यु॑र्व्या रु॒क्मो रु॒क्म उ॒र्व्या वि । \newline
44. उ॒र्व्या वि व्यु॑र्व्योर्व्या व्य॑द्यौ दद्यौ॒द् व्यु॑र्व्योर्व्या व्य॑द्यौत् । \newline
45. व्य॑द्यौ दद्यौ॒द् वि व्य॑द्यौद् दु॒र्मर्.ष॑म् दु॒र्मर्.ष॑ मद्यौ॒द् वि व्य॑द्यौद् दु॒र्मर्.ष᳚म् । \newline
46. अ॒द्यौ॒द् दु॒र्मर्.ष॑म् दु॒र्मर्.ष॑ मद्यौ दद्यौद् दु॒र्मर्.ष॒ मायु॒रायु॑र् दु॒र्मर्.ष॑ मद्यौ दद्यौद् दु॒र्मर्.ष॒ मायुः॑ । \newline
47. दु॒र्मर्.ष॒ मायु॒ रायु॑र् दु॒र्मर्.ष॑म् दु॒र्मर्.ष॒ मायुः॑ श्रि॒ये श्रि॒य आयु॑र् दु॒र्मर्.ष॑म् दु॒र्मर्.ष॒ मायुः॑ श्रि॒ये । \newline
48. दु॒र्मर्.ष॒मिति॑ दुः - मर्.ष᳚म् । \newline
49. आयुः॑ श्रि॒ये श्रि॒य आयु॒ रायुः॑ श्रि॒ये रु॑चा॒नो रु॑चा॒नः श्रि॒य आयु॒ रायुः॑ श्रि॒ये रु॑चा॒नः । \newline
50. श्रि॒ये रु॑चा॒नो रु॑चा॒नः श्रि॒ये श्रि॒ये रु॑चा॒नः । \newline
51. रु॒चा॒न इति॑ रुचा॒नः । \newline
52. अ॒ग्नि र॒मृतो॑ अ॒मृतो॑ अ॒ग्नि र॒ग्नि र॒मृतो॑ अभव दभव द॒मृतो॑ अ॒ग्नि र॒ग्नि र॒मृतो॑ अभवत् । \newline
53. अ॒मृतो॑ अभव दभव द॒मृतो॑ अ॒मृतो॑ अभव॒द् वयो॑भि॒र् वयो॑भि रभव द॒मृतो॑ अ॒मृतो॑ अभव॒द् वयो॑भिः । \newline
54. अ॒भ॒व॒द् वयो॑भि॒र् वयो॑भि रभव दभव॒द् वयो॑भि॒र् यद् यद् वयो॑भि रभव दभव॒द् वयो॑भि॒र् यत् । \newline
55. वयो॑भि॒र् यद् यद् वयो॑भि॒र् वयो॑भि॒र् यदे॑न मेनं॒ ॅयद् वयो॑भि॒र् वयो॑भि॒र् यदे॑नम् । \newline
56. वयो॑भि॒रिति॒ वयः॑ - भिः॒ । \newline
\pagebreak
\markright{ TS 1.3.14.6  \hfill https://www.vedavms.in \hfill}

\section{ TS 1.3.14.6 }

\textbf{TS 1.3.14.6 } \newline
\textbf{Samhita Paata} \newline

यदे॑नं॒ द्यौरज॑नयथ् सु॒रेताः᳚ ॥ आ यदि॒षे नृ॒पतिं॒ तेज॒ आन॒ट्छुचि॒ रेतो॒ निषि॑क्तं॒ द्यौर॒भीके᳚ । अ॒ग्निः शर्द्ध॑मनव॒द्यं ॅयुवा॑नꣳ स्वा॒धियं॑ जनयथ् सू॒दय॑च्च ॥ स तेजी॑यसा॒ मन॑सा॒ त्वोत॑ उ॒त शि॑क्ष स्वप॒त्यस्य॑ शि॒क्षोः । अग्ने॑ रा॒यो नृत॑मस्य॒ प्रभू॑तौ भू॒याम॑ ते सुष्टु॒तय॑श्च॒ वस्वः॑ ॥ अग्ने॒ सह॑न्त॒मा भ॑र द्यु॒म्नस्य॑ प्रा॒सहा॑ र॒यिं ।विश्वा॒ य - [ ] \newline

\textbf{Pada Paata} \newline

यत् । ए॒न॒म् । द्यौः । अज॑नयत् । सु॒रेता॒ इति॑ सु - रेताः᳚ ॥ एति॑ । यत् । इ॒षे । नृ॒पति॒मिति॑ नृ - पति᳚म् । तेजः॑ । आन॑ट् । शुचि॑ । रेतः॑ । निषि॑क्त॒मिति॒ नि - सि॒क्त॒म् । द्यौः । अ॒भीके᳚ ॥ अ॒ग्निः । शर्ध᳚म् । अ॒न॒व॒द्यम् । युवा॑नम् । स्वा॒धिय॒मिति॑ स्व - धिय᳚म् । ज॒न॒य॒त् । सू॒दय॑त् । च॒ ॥ सः । तेजी॑यसा । मन॑सा । त्वोतः॑ । उ॒त । शि॒क्ष॒ । स्व॒प॒त्यस्येति॑ सु - अ॒प॒त्यस्य॑ । शि॒क्षोः ॥ अग्ने᳚ । रा॒यः । नृत॑म॒स्येति॒ नृ - त॒म॒स्य॒ । प्रभू॑ता॒विति॒ प्र - भू॒तौ॒ । भू॒याम॑ । ते॒ । सु॒ष्टु॒तय॒ इति॑ सु - स्तु॒तयः॑ । च॒ । वस्वः॑ ॥ अग्ने᳚ । सह॑न्तम् । एति॑ । भ॒र॒ । द्यु॒म्नस्य॑ । प्रा॒सहेति॑ प्र - सहा᳚ । र॒यिम् ॥ विश्वाः᳚ । यः ।  \newline


\textbf{Krama Paata} \newline

यदे॑नम् । ए॒न॒म् द्यौः । द्यौरज॑नयत् । अज॑नयथ् सु॒रेताः᳚ । सु॒रेता॒ इति॑ सु - रेताः᳚ ॥ आ यत् । यदि॒षे । इ॒षे नृ॒पति᳚म् । नृ॒पति॒म् तेजः॑ । नृ॒पति॒मिति॑ नृ - पति᳚म् । तेज॒ आन॑ट् । आन॒ट्छुचि॑ । शुचि॒ रेतः॑ । रेतो॒ निषि॑क्तम् । निषि॑क्त॒म् द्यौः । निषि॑क्त॒मिति॒ नि - सि॒क्त॒म् । द्यौर॒भीके᳚ । अ॒भीक॒ इत्य॒भीके᳚ ॥ अ॒ग्निः शर्द्ध᳚म् । शर्द्ध॑मनव॒द्यम् । अ॒न॒व॒द्यं ॅयुवा॑नम् । युवा॑नꣳ स्वा॒धिय᳚म् । स्वा॒धिय॑म् जनयत् । स्वा॒धिय॒मिति॑ स्व - धिय᳚म् । ज॒न॒य॒थ् सू॒दय॑त् । सू॒दय॑च्च । चेति॑ च ॥ स तेजी॑यसा । तेजी॑यसा॒ मन॑सा । मन॑सा॒ त्वोतः॑ । त्वोत॑ उ॒त । उ॒त शि॑क्ष । शि॒क्ष॒ स्व॒प॒त्यस्य॑ । स्व॒प॒त्यस्य॑ शि॒क्षोः । स्व॒प॒त्यस्येति॑ सु - अ॒प॒त्यस्य॑ । शि॒क्षोरिति॑ शि॒क्षोः ॥ अग्ने॑ रा॒यः । रा॒यो नृत॑मस्य । नृत॑मस्य॒ प्रभू॑तौ । नृत॑म॒स्येति॒ नृ - त॒म॒स्य॒ । प्रभू॑तौ भू॒याम॑ । प्रभू॑ता॒विति॒ प्र - भू॒तौ॒ । भू॒याम॑ ते । ते॒ सु॒ष्टु॒तयः॑ । सु॒ष्टु॒तय॑श्च । सु॒ष्टु॒तय॒ इति॑ सु - स्तु॒तयः॑ । च॒ वस्वः॑ । वस्व॒ इति॒ वस्वः॑ ॥ अग्ने॒ सह॑न्तम् । सह॑न्त॒मा । आ भ॑र । भ॒र॒ द्यु॒म्नस्य॑ । द्यु॒म्नस्य॑ प्रा॒सहा᳚ । प्रा॒सहा॑ र॒यिम् । प्रा॒सहेति॑ प्र - सहा᳚ । र॒यिमिति॑ र॒यिम् ॥ विश्वा॒ यः । यश्च॑र्.ष॒णीः \newline

\textbf{Jatai Paata} \newline

1. यदे॑न मेनं॒ ॅयद् यदे॑नम् । \newline
2. ए॒न॒म् द्यौर् द्यौरे॑न मेन॒म् द्यौः । \newline
3. द्यौ रज॑नय॒ दज॑नय॒द् द्यौर् द्यौ रज॑नयत् । \newline
4. अज॑नयथ् सु॒रेताः᳚ सु॒रेता॒ अज॑नय॒ दज॑नयथ् सु॒रेताः᳚ । \newline
5. सु॒रेता॒ इति॑ सु - रेताः᳚ । \newline
6. आ यद् यदा यत् । \newline
7. यदि॒ष इ॒षे यद् यदि॒षे । \newline
8. इ॒षे नृ॒पति॑म् नृ॒पति॑ मि॒ष इ॒षे नृ॒पति᳚म् । \newline
9. नृ॒पति॒म् तेज॒ स्तेजो॑ नृ॒पति॑म् नृ॒पति॒म् तेजः॑ । \newline
10. नृ॒पति॒मिति॑ नृ - पति᳚म् । \newline
11. तेज॒ आन॒डान॒ट् तेज॒ स्तेज॒ आन॑ट् । \newline
12. आन॒ट् छुचि॒ शुच्या न॒डान॒ट् छुचि॑ । \newline
13. शुचि॒ रेतो॒ रेतः॒ शुचि॒ शुचि॒ रेतः॑ । \newline
14. रेतो॒ निषि॑क्त॒म् निषि॑क्त॒(ग्म्॒) रेतो॒ रेतो॒ निषि॑क्तम् । \newline
15. निषि॑क्त॒म् द्यौर् द्यौर् निषि॑क्त॒म् निषि॑क्त॒म् द्यौः । \newline
16. निषि॑क्त॒मिति॒ नि - सि॒क्त॒म् । \newline
17. द्यौर॒भीके॑ अ॒भीके॒ द्यौर् द्यौ र॒भीके᳚ । \newline
18. अ॒भीक॒ इत्य॒भीके᳚ । \newline
19. अ॒ग्निः शर्द्ध॒(ग्म्॒) शर्द्ध॑ म॒ग्नि र॒ग्निः शर्द्ध᳚म् । \newline
20. शर्द्ध॑ मनव॒द्य म॑नव॒द्यꣳ शर्द्ध॒(ग्म्॒) शर्द्ध॑ मनव॒द्यम् । \newline
21. अ॒न॒व॒द्यं ॅयुवा॑नं॒ ॅयुवा॑न मनव॒द्य म॑नव॒द्यं ॅयुवा॑नम् । \newline
22. युवा॑नꣳ स्वा॒धिय(ग्ग्॑) स्वा॒धियं॒ ॅयुवा॑नं॒ ॅयुवा॑नꣳ स्वा॒धिय᳚म् । \newline
23. स्वा॒धिय॑म् जनयज् जनयथ् स्वा॒धिय(ग्ग्॑) स्वा॒धिय॑म् जनयत् । \newline
24. स्वा॒धिय॒मिति॑ स्व - धिय᳚म् । \newline
25. ज॒न॒य॒थ् सू॒दय॑थ् सू॒दय॑ज् जनयज् जनयथ् सू॒दय॑त् । \newline
26. सू॒दय॑च् च च सू॒दय॑थ् सू॒दय॑च् च । \newline
27. चेति॑ च । \newline
28. स तेजी॑यसा॒ तेजी॑यसा॒ स स तेजी॑यसा । \newline
29. तेजी॑यसा॒ मन॑सा॒ मन॑सा॒ तेजी॑यसा॒ तेजी॑यसा॒ मन॑सा । \newline
30. मन॑सा॒ त्वोत॒ स्त्वोतो॒ मन॑सा॒ मन॑सा॒ त्वोतः॑ । \newline
31. त्वोत॑ उ॒तोत त्वोत॒ स्त्वोत॑ उ॒त । \newline
32. उ॒त शि॑क्ष शिक्षो॒तोत शि॑क्ष । \newline
33. शि॒क्ष॒ स्व॒प॒त्यस्य॑ स्वप॒त्यस्य॑ शिक्ष शिक्ष स्वप॒त्यस्य॑ । \newline
34. स्व॒प॒त्यस्य॑ शि॒क्षोः शि॒क्षोः स्व॑प॒त्यस्य॑ स्वप॒त्यस्य॑ शि॒क्षोः । \newline
35. स्व॒प॒त्यस्येति॑ सु - अ॒प॒त्यस्य॑ । \newline
36. शि॒क्षोरिति॑ शि॒क्षोः । \newline
37. अग्ने॑ रा॒यो रा॒यो ऽग्ने ऽग्ने॑ रा॒यः । \newline
38. रा॒यो नृत॑मस्य॒ नृत॑मस्य रा॒यो रा॒यो नृत॑मस्य । \newline
39. नृत॑मस्य॒ प्रभू॑तौ॒ प्रभू॑तौ॒ नृत॑मस्य॒ नृत॑मस्य॒ प्रभू॑तौ । \newline
40. नृत॑म॒स्येति॒ नृ - त॒म॒स्य॒ । \newline
41. प्रभू॑तौ भू॒याम॑ भू॒याम॒ प्रभू॑तौ॒ प्रभू॑तौ भू॒याम॑ । \newline
42. प्रभू॑ता॒विति॒ प्र - भू॒तौ॒ । \newline
43. भू॒याम॑ ते ते भू॒याम॑ भू॒याम॑ ते । \newline
44. ते॒ सु॒ष्टु॒तयः॑ सुष्टु॒तय॑ स्ते ते सुष्टु॒तयः॑ । \newline
45. सु॒ष्टु॒तय॑ श्च च सुष्टु॒तयः॑ सुष्टु॒तय॑ श्च । \newline
46. सु॒ष्टु॒तय॒ इति॑ सु - स्तु॒तयः॑ । \newline
47. च॒ वस्वो॒ वस्व॑ श्च च॒ वस्वः॑ । \newline
48. वस्व॒ इति॒ वस्वः॑ । \newline
49. अग्ने॒ सह॑न्त॒(ग्म्॒) सह॑न्त॒ मग्ने ऽग्ने॒ सह॑न्तम् । \newline
50. सह॑न्त॒ मा सह॑न्त॒(ग्म्॒) सह॑न्त॒ मा । \newline
51. आ भ॑र भ॒रा भ॑र । \newline
52. भ॒र॒ द्यु॒म्नस्य॑ द्यु॒म्नस्य॑ भर भर द्यु॒म्नस्य॑ । \newline
53. द्यु॒म्नस्य॑ प्रा॒सहा᳚ प्रा॒सहा᳚ द्यु॒म्नस्य॑ द्यु॒म्नस्य॑ प्रा॒सहा᳚ । \newline
54. प्रा॒सहा॑ र॒यिꣳ र॒यिम् प्रा॒सहा᳚ प्रा॒सहा॑ र॒यिम् । \newline
55. प्रा॒सहेति॑ प्र - सहा᳚ । \newline
56. र॒यिमिति॑ र॒यिम् । \newline
57. विश्वा॒ यो यो विश्वा॒ विश्वा॒ यः । \newline
58. यश्च॑र्.ष॒णी श्च॑र्.ष॒णीर् यो यश्च॑र्.ष॒णीः । \newline

\textbf{Ghana Paata } \newline

1. यदे॑न मेनं॒ ॅयद् यदे॑न॒म् द्यौर् द्यौरे॑नं॒ ॅयद् यदे॑न॒म् द्यौः । \newline
2. ए॒न॒म् द्यौर् द्यौरे॑न मेन॒म् द्यौ रज॑नय॒ दज॑नय॒द् द्यौरे॑न मेन॒म् द्यौ रज॑नयत् । \newline
3. द्यौ रज॑नय॒ दज॑नय॒द् द्यौर् द्यौ रज॑नयथ् सु॒रेताः᳚ सु॒रेता॒ अज॑नय॒द् द्यौर् द्यौ रज॑नयथ् सु॒रेताः᳚ । \newline
4. अज॑नयथ् सु॒रेताः᳚ सु॒रेता॒ अज॑नय॒ दज॑नयथ् सु॒रेताः᳚ । \newline
5. सु॒रेता॒ इति॑ सु - रेताः᳚ । \newline
6. आ यद् यदा यदि॒ष इ॒षे यदा यदि॒षे । \newline
7. यदि॒ष इ॒षे यद् यदि॒षे नृ॒पति॑म् नृ॒पति॑ मि॒षे यद् यदि॒षे नृ॒पति᳚म् । \newline
8. इ॒षे नृ॒पति॑म् नृ॒पति॑ मि॒ष इ॒षे नृ॒पति॒म् तेज॒ स्तेजो॑ नृ॒पति॑ मि॒ष इ॒षे नृ॒पति॒म् तेजः॑ । \newline
9. नृ॒पति॒म् तेज॒ स्तेजो॑ नृ॒पति॑म् नृ॒पति॒म् तेज॒ आन॒ डान॒ट् तेजो॑ नृ॒पति॑म् नृ॒पति॒म् तेज॒ आन॑ट् । \newline
10. नृ॒पति॒मिति॑ नृ - पति᳚म् । \newline
11. तेज॒ आन॒ डान॒ट् तेज॒ स्तेज॒ आन॒ट् छुचि॒ शुच्यान॒ट् तेज॒ स्तेज॒ आन॒ट् छुचि॑ । \newline
12. आन॒ट् छुचि॒ शुच्या न॒डान॒ट् छुचि॒ रेतो॒ रेतः॒ शुच्या न॒डान॒ट् छुचि॒ रेतः॑ । \newline
13. शुचि॒ रेतो॒ रेतः॒ शुचि॒ शुचि॒ रेतो॒ निषि॑क्त॒म् निषि॑क्त॒(ग्म्॒) रेतः॒ शुचि॒ शुचि॒ रेतो॒ निषि॑क्तम् । \newline
14. रेतो॒ निषि॑क्त॒म् निषि॑क्त॒(ग्म्॒) रेतो॒ रेतो॒ निषि॑क्त॒म् द्यौर् द्यौर् निषि॑क्त॒(ग्म्॒) रेतो॒ रेतो॒ निषि॑क्त॒म् द्यौः । \newline
15. निषि॑क्त॒म् द्यौर् द्यौर् निषि॑क्त॒म् निषि॑क्त॒म् द्यौर॒भीके॑ अ॒भीके॒ द्यौर् निषि॑क्त॒म् निषि॑क्त॒म् द्यौर॒भीके᳚ । \newline
16. निषि॑क्त॒मिति॒ नि - सि॒क्त॒म् । \newline
17. द्यौ र॒भीके॑ अ॒भीके॒ द्यौर् द्यौ र॒भीके᳚ । \newline
18. अ॒भीक॒ इत्य॒भीके᳚ । \newline
19. अ॒ग्निः शर्द्ध॒(ग्म्॒) शर्द्ध॑ म॒ग्नि र॒ग्निः शर्द्ध॑ मनव॒द्य म॑नव॒द्यꣳ शर्द्ध॑ म॒ग्नि र॒ग्निः शर्द्ध॑ मनव॒द्यम् । \newline
20. शर्द्ध॑ मनव॒द्य म॑नव॒द्यꣳ शर्द्ध॒(ग्म्॒) शर्द्ध॑ मनव॒द्यं ॅयुवा॑नं॒ ॅयुवा॑न मनव॒द्यꣳ शर्द्ध॒(ग्म्॒) शर्द्ध॑ मनव॒द्यं ॅयुवा॑नम् । \newline
21. अ॒न॒व॒द्यं ॅयुवा॑नं॒ ॅयुवा॑न मनव॒द्य म॑नव॒द्यं ॅयुवा॑नꣳ स्वा॒धिय(ग्ग्॑) स्वा॒धियं॒ ॅयुवा॑न मनव॒द्य म॑नव॒द्यं ॅयुवा॑नꣳ स्वा॒धिय᳚म् । \newline
22. युवा॑नꣳ स्वा॒धिय(ग्ग्॑) स्वा॒धियं॒ ॅयुवा॑नं॒ ॅयुवा॑नꣳ स्वा॒धिय॑म् जनयज् जनयथ् स्वा॒धियं॒ ॅयुवा॑नं॒ ॅयुवा॑नꣳ स्वा॒धिय॑म् जनयत् । \newline
23. स्वा॒धिय॑म् जनयज् जनयथ् स्वा॒धिय(ग्ग्॑) स्वा॒धिय॑म् जनयथ् सू॒दय॑थ् सू॒दय॑ज् जनयथ् स्वा॒धिय(ग्ग्॑) स्वा॒धिय॑म् जनयथ् सू॒दय॑त् । \newline
24. स्वा॒धिय॒मिति॑ स्व - धिय᳚म् । \newline
25. ज॒न॒य॒थ् सू॒दय॑थ् सू॒दय॑ज् जनयज् जनयथ् सू॒दय॑च् च च सू॒दय॑ज् जनयज् जनयथ् सू॒दय॑च् च । \newline
26. सू॒दय॑च् च च सू॒दय॑थ् सू॒दय॑च् च । \newline
27. चेति॑ च । \newline
28. स तेजी॑यसा॒ तेजी॑यसा॒ स स तेजी॑यसा॒ मन॑सा॒ मन॑सा॒ तेजी॑यसा॒ स स तेजी॑यसा॒ मन॑सा । \newline
29. तेजी॑यसा॒ मन॑सा॒ मन॑सा॒ तेजी॑यसा॒ तेजी॑यसा॒ मन॑सा॒ त्वोत॒ स्त्वोतो॒ मन॑सा॒ तेजी॑यसा॒ तेजी॑यसा॒ मन॑सा॒ त्वोतः॑ । \newline
30. मन॑सा॒ त्वोत॒ स्त्वोतो॒ मन॑सा॒ मन॑सा॒ त्वोत॑ उ॒तोत त्वोतो॒ मन॑सा॒ मन॑सा॒ त्वोत॑ उ॒त । \newline
31. त्वोत॑ उ॒तोत त्वोत॒ स्त्वोत॑ उ॒त शि॑क्ष शिक्षो॒त त्वोत॒ स्त्वोत॑ उ॒त शि॑क्ष । \newline
32. उ॒त शि॑क्ष शिक्षो॒तोत शि॑क्ष स्वप॒त्यस्य॑ स्वप॒त्यस्य॑ शिक्षो॒तोत शि॑क्ष स्वप॒त्यस्य॑ । \newline
33. शि॒क्ष॒ स्व॒प॒त्यस्य॑ स्वप॒त्यस्य॑ शिक्ष शिक्ष स्वप॒त्यस्य॑ शि॒क्षोः शि॒क्षोः स्व॑प॒त्यस्य॑ शिक्ष शिक्ष स्वप॒त्यस्य॑ शि॒क्षोः । \newline
34. स्व॒प॒त्यस्य॑ शि॒क्षोः शि॒क्षोः स्व॑प॒त्यस्य॑ स्वप॒त्यस्य॑ शि॒क्षोः । \newline
35. स्व॒प॒त्यस्येति॑ सु - अ॒प॒त्यस्य॑ । \newline
36. शि॒क्षोरिति॑ शि॒क्षोः । \newline
37. अग्ने॑ रा॒यो रा॒यो ऽग्ने ऽग्ने॑ रा॒यो नृत॑मस्य॒ नृत॑मस्य रा॒यो ऽग्ने ऽग्ने॑ रा॒यो नृत॑मस्य । \newline
38. रा॒यो नृत॑मस्य॒ नृत॑मस्य रा॒यो रा॒यो नृत॑मस्य॒ प्रभू॑तौ॒ प्रभू॑तौ॒ नृत॑मस्य रा॒यो रा॒यो नृत॑मस्य॒ प्रभू॑तौ । \newline
39. नृत॑मस्य॒ प्रभू॑तौ॒ प्रभू॑तौ॒ नृत॑मस्य॒ नृत॑मस्य॒ प्रभू॑तौ भू॒याम॑ भू॒याम॒ प्रभू॑तौ॒ नृत॑मस्य॒ नृत॑मस्य॒ प्रभू॑तौ भू॒याम॑ । \newline
40. नृत॑म॒स्येति॒ नृ - त॒म॒स्य॒ । \newline
41. प्रभू॑तौ भू॒याम॑ भू॒याम॒ प्रभू॑तौ॒ प्रभू॑तौ भू॒याम॑ ते ते भू॒याम॒ प्रभू॑तौ॒ प्रभू॑तौ भू॒याम॑ ते । \newline
42. प्रभू॑ता॒विति॒ प्र - भू॒तौ॒ । \newline
43. भू॒याम॑ ते ते भू॒याम॑ भू॒याम॑ ते सुष्टु॒तयः॑ सुष्टु॒तय॑ स्ते भू॒याम॑ भू॒याम॑ ते सुष्टु॒तयः॑ । \newline
44. ते॒ सु॒ष्टु॒तयः॑ सुष्टु॒तय॑ स्ते ते सुष्टु॒तय॑श्च च सुष्टु॒तय॑ स्ते ते सुष्टु॒तय॑श्च । \newline
45. सु॒ष्टु॒तय॑श्च च सुष्टु॒तयः॑ सुष्टु॒तय॑श्च॒ वस्वो॒ वस्व॑श्च सुष्टु॒तयः॑ सुष्टु॒तय॑श्च॒ वस्वः॑ । \newline
46. सु॒ष्टु॒तय॒ इति॑ सु - स्तु॒तयः॑ । \newline
47. च॒ वस्वो॒ वस्व॑श्च च॒ वस्वः॑ । \newline
48. वस्व॒ इति॒ वस्वः॑ । \newline
49. अग्ने॒ सह॑न्त॒(ग्म्॒) सह॑न्त॒ मग्ने ऽग्ने॒ सह॑न्त॒ मा सह॑न्त॒ मग्ने ऽग्ने॒ सह॑न्त॒ मा । \newline
50. सह॑न्त॒ मा सह॑न्त॒(ग्म्॒) सह॑न्त॒ मा भ॑र भ॒रा सह॑न्त॒(ग्म्॒) सह॑न्त॒ मा भ॑र । \newline
51. आ भ॑र भ॒रा भ॑र द्यु॒म्नस्य॑ द्यु॒म्नस्य॑ भ॒रा भ॑र द्यु॒म्नस्य॑ । \newline
52. भ॒र॒ द्यु॒म्नस्य॑ द्यु॒म्नस्य॑ भर भर द्यु॒म्नस्य॑ प्रा॒सहा᳚ प्रा॒सहा᳚ द्यु॒म्नस्य॑ भर भर द्यु॒म्नस्य॑ प्रा॒सहा᳚ । \newline
53. द्यु॒म्नस्य॑ प्रा॒सहा᳚ प्रा॒सहा᳚ द्यु॒म्नस्य॑ द्यु॒म्नस्य॑ प्रा॒सहा॑ र॒यिꣳ र॒यिम् प्रा॒सहा᳚ द्यु॒म्नस्य॑ द्यु॒म्नस्य॑ प्रा॒सहा॑ र॒यिम् । \newline
54. प्रा॒सहा॑ र॒यिꣳ र॒यिम् प्रा॒सहा᳚ प्रा॒सहा॑ र॒यिम् । \newline
55. प्रा॒सहेति॑ प्र - सहा᳚ । \newline
56. र॒यिमिति॑ र॒यिम् । \newline
57. विश्वा॒ यो यो विश्वा॒ विश्वा॒ यश्च॑र्.ष॒णी श्च॑र्.ष॒णीर् यो विश्वा॒ विश्वा॒ यश्च॑र्.ष॒णीः । \newline
58. यश्च॑र्.ष॒णी श्च॑र्.ष॒णीर् यो यश्च॑र्.ष॒णी र॒भ्य॑भि च॑र्.ष॒णीर् यो यश्च॑र्.ष॒णी र॒भि । \newline
\pagebreak
\markright{ TS 1.3.14.7  \hfill https://www.vedavms.in \hfill}

\section{ TS 1.3.14.7 }

\textbf{TS 1.3.14.7 } \newline
\textbf{Samhita Paata} \newline

श्च॑र्॒.ष॒णीर॒भ्या॑सा वाजे॑षु सा॒सह॑त् ॥ तम॑ग्ने पृतना॒सहꣳ॑ र॒यिꣳ स॑हस्व॒ आ भ॑र । त्वꣳ हि स॒त्यो अद्भु॑तो दा॒ता वाज॑स्य॒ गोम॑तः ॥ उ॒क्षान्ना॑य व॒शान्ना॑य॒ सोम॑पृष्ठाय वे॒धसे᳚ । स्तोमै᳚र् विधेमा॒ऽग्नये᳚ ॥ व॒द्मा हि सू॑नो॒ अस्य॑द्म॒सद्वा॑ च॒क्रे अ॒ग्निर् ज॒नुषा ऽज्माऽन्नं᳚ । स त्वं न॑ ऊर्जसन॒ ऊर्जं॑ धा॒ राजे॑व जेरवृ॒के क्षे᳚ष्य॒न्तः ॥ अग्न॒ आयूꣳ॑षि - [ ] \newline

\textbf{Pada Paata} \newline

च॒र्॒.ष॒णीः । अ॒भीति॑ । आ॒सा । वाजे॑षु । सा॒सह॑त् ॥ तम् । अ॒ग्ने॒ । पृ॒त॒ना॒सह॒मिति॑ पृतना - सह᳚म् । र॒यिम् । स॒ह॒स्वः॒ । एति॑ । भ॒र॒ ॥ त्वम् । हि । स॒त्यः । अद्भु॑तः । दा॒ता । वाज॑स्य । गोम॑त॒ इति॒ गो - म॒तः॒ ॥ उ॒क्षान्ना॒येत्यु॒क्ष - अ॒न्ना॒य॒ । व॒शान्ना॒येति॑ व॒शा - अ॒न्ना॒य॒ । सोम॑पृष्ठा॒येति॒ सोम॑ - पृ॒ष्ठा॒य॒ । वे॒धसे᳚ ॥ स्तोमैः᳚ । वि॒धे॒म॒ । अ॒ग्नये᳚ ॥ व॒द्मा । हि । सू॒नो॒ इति॑ । असि॑ । अ॒द्म॒सद्वेत्य॑द्म - सद्वा᳚ । च॒क्रे । अ॒ग्निः । ज॒नुषा᳚ । अज्म॑ । अन्न᳚म् ॥ सः । त्वम् । नः॒ । ऊ॒र्ज॒स॒न॒ इत्यु᳚र्ज - स॒ने॒ । ऊर्ज᳚म् । धाः॒ । राजा᳚ । इ॒व॒ । जेः॒ । अ॒वृ॒के । क्षे॒षि॒ । अ॒न्तः ॥ अग्ने᳚ । आयूꣳ॑षि ।  \newline


\textbf{Krama Paata} \newline

च॒र्॒.ष॒णीर॒भि । अ॒भ्या॑सा । आ॒सा वाजे॑षु । वाजे॑षु सा॒सह॑त् । सा॒सह॒दिति॑ सा॒सह॑त् ॥ तम॑ग्ने । अ॒ग्ने॒ पृ॒त॒ना॒सह᳚म् । पृ॒त॒ना॒सहꣳ॑ र॒यिम् । पृ॒त॒ना॒सह॒मिति॑ पृतना - सह᳚म् । र॒यिꣳ स॑हस्वः । स॒ह॒स्व॒ आ । आ भ॑र । भ॒रेति॑ भर ॥ त्वꣳ हि । हि स॒त्यः । स॒त्यो अद्भु॑तः । अद्भु॑तो दा॒ता । दा॒ता वाज॑स्य । वाज॑स्य॒ गोम॑तः । गोम॑त॒ इति॒ गो - म॒तः॒ । उ॒क्षान्ना॑य व॒शान्ना॑य । उ॒क्षान्ना॒येत्यु॒क्ष - अ॒न्ना॒य॒ । व॒शान्ना॑य॒ सोम॑पृष्ठाय । व॒शान्ना॒येति॑ व॒शा - अ॒न्ना॒य॒ । सोम॑पृष्ठाय वे॒धसे᳚ । सोम॑पृष्ठा॒येति॒ सोम॑ - पृ॒ष्ठा॒य॒ । वे॒धस॒ इति॑ वे॒धसे᳚ ॥ स्तोमै᳚र् विधेम । वि॒धे॒मा॒ग्नये᳚ । अ॒ग्नय॒ इत्य॒ग्नये᳚ ॥ व॒द्मा हि । हि सू॑नो । सू॒नो॒ असि॑ । सू॒नो॒ इति॑ सूनो । अस्य॑द्म॒सद्वा᳚ । अ॒द्म॒सद्वा॑ च॒क्रे । अ॒द्म॒सद्वेत्य॑द्म - सद्वा᳚ । च॒क्रे अ॒ग्निः । अ॒ग्निर् ज॒नुषा᳚ । ज॒नुषाऽज्म॑ । अज्मान्न᳚म् । अन्न॒मित्यन्न᳚म् ॥ स त्वम् । त्वम् नः॑ । न॒ ऊ॒र्ज॒स॒ने॒ । ऊ॒र्ज॒स॒न॒ ऊर्ज᳚म् । ऊ॒र्ज॒स॒न॒ इत्यू᳚र्ज - स॒ने॒ । ऊर्ज॑म् धाः । धा॒ राजा᳚ । राजे॑व । इ॒व॒ जेः॒ । जे॒र॒वृ॒के । अ॒वृ॒के क्षे॑षि । क्षे॒ष्य॒न्तः । अ॒न्तरित्य॒न्तः ॥ अग्न॒ आयूꣳ॑षि । 
आयूꣳ॑षि पवसे \newline

\textbf{Jatai Paata} \newline

1. च॒र्॒.ष॒णी र॒भ्य॑भि च॑र्.ष॒णी श्च॑र्.ष॒णी र॒भि । \newline
2. अ॒भ्या॑सा ऽऽसा ऽभ्या᳚(1॒)भ्या॑सा । \newline
3. आ॒सा वाजे॑षु॒ वाजे᳚ष्वा॒सा ऽऽसा वाजे॑षु । \newline
4. वाजे॑षु सा॒सह॑थ् सा॒सह॒द् वाजे॑षु॒ वाजे॑षु सा॒सह॑त् । \newline
5. सा॒सह॒दिति॑ सा॒सह॑त् । \newline
6. त म॑ग्ने अग्ने॒ तम् त म॑ग्ने । \newline
7. अ॒ग्ने॒ पृ॒त॒ना॒सह॑म् पृतना॒सह॑ मग्ने अग्ने पृतना॒सह᳚म् । \newline
8. पृ॒त॒ना॒सह(ग्म्॑) र॒यिꣳ र॒यिम् पृ॑तना॒सह॑म् पृतना॒सह(ग्म्॑) र॒यिम् । \newline
9. पृ॒त॒ना॒सह॒मिति॑ पृतना - सह᳚म् । \newline
10. र॒यिꣳ स॑हस्वः सहस्वो र॒यिꣳ र॒यिꣳ स॑हस्वः । \newline
11. स॒ह॒स्व॒ आ स॑हस्वः सहस्व॒ आ । \newline
12. आ भ॑र भ॒रा भ॑र । \newline
13. भ॒रेति॑ भर । \newline
14. त्वꣳ हि हि त्वम् त्वꣳ हि । \newline
15. हि स॒त्यः स॒त्यो हि हि स॒त्यः । \newline
16. स॒त्यो अद्भु॑तो॒ अद्भु॑तः स॒त्यः स॒त्यो अद्भु॑तः । \newline
17. अद्भु॑तो दा॒ता दा॒ता ऽद्भु॑तो॒ अद्भु॑तो दा॒ता । \newline
18. दा॒ता वाज॑स्य॒ वाज॑स्य दा॒ता दा॒ता वाज॑स्य । \newline
19. वाज॑स्य॒ गोम॑तो॒ गोम॑तो॒ वाज॑स्य॒ वाज॑स्य॒ गोम॑तः । \newline
20. गोम॑त॒ इति॒ गो - म॒तः॒ । \newline
21. उ॒क्षान्ना॑य व॒शान्ना॑य व॒शान्ना॑ यो॒क्षान्ना॑ यो॒क्षान्ना॑य व॒शान्ना॑य । \newline
22. उ॒क्षान्ना॒येत्यु॒क्ष - अ॒न्ना॒य॒ । \newline
23. व॒शान्ना॑य॒ सोम॑पृष्ठाय॒ सोम॑पृष्ठाय व॒शान्ना॑य व॒शान्ना॑य॒ सोम॑पृष्ठाय । \newline
24. व॒शान्ना॒येति॑ व॒शा - अ॒न्ना॒य॒ । \newline
25. सोम॑पृष्ठाय वे॒धसे॑ वे॒धसे॒ सोम॑पृष्ठाय॒ सोम॑पृष्ठाय वे॒धसे᳚ । \newline
26. सोम॑पृष्ठा॒येति॒ सोम॑ - पृ॒ष्ठा॒य॒ । \newline
27. वे॒धस॒ इति॑ वे॒धसे᳚ । \newline
28. स्तोमै᳚र् विधेम विधेम॒ स्तोमैः॒ स्तोमै᳚र् विधेम । \newline
29. वि॒धे॒मा॒ग्नये॑ अ॒ग्नये॑ विधेम विधेमा॒ग्नये᳚ । \newline
30. अ॒ग्नय॒ इत्य॒ग्नये᳚ । \newline
31. व॒द्मा हि हि व॒द्मा व॒द्मा हि । \newline
32. हि सू॑नो सूनो॒ हि हि सू॑नो । \newline
33. सू॒नो॒ अस्यसि॑ सूनो सूनो॒ असि॑ । \newline
34. सू॒नो॒ इति॑ सूनो । \newline
35. अस्य॑द्म॒सद्वा᳚ ऽद्म॒सद्वा ऽस्य स्य॑द्म॒सद्वा᳚ । \newline
36. अ॒द्म॒सद्वा॑ च॒क्रे च॒क्रे अ॑द्म॒सद्वा᳚ ऽद्म॒सद्वा॑ च॒क्रे । \newline
37. अ॒द्म॒सद्वेत्य॑द्म - सद्वा᳚ । \newline
38. च॒क्रे अ॒ग्नि र॒ग्नि श्च॒क्रे च॒क्रे अ॒ग्निः । \newline
39. अ॒ग्निर् ज॒नुषा॑ ज॒नुषा॒ ऽग्नि र॒ग्निर् ज॒नुषा᳚ । \newline
40. ज॒नुषा ऽज्माज्म॑ ज॒नुषा॑ ज॒नुषा ऽज्म॑ । \newline
41. अज्मान्न॒ मन्न॒ मज्मा ज्मान्न᳚म् । \newline
42. अन्न॒मित्यन्न᳚म् । \newline
43. स त्वम् त्वꣳ स स त्वम् । \newline
44. त्वम् नो॑ न॒स्त्वम् त्वम् नः॑ । \newline
45. न॒ ऊ॒र्ज॒स॒न॒ ऊ॒र्ज॒स॒ने॒ नो॒ न॒ ऊ॒र्ज॒स॒ने॒ । \newline
46. ऊ॒र्ज॒स॒न॒ ऊर्ज॒ मूर्ज॑ मूर्जसन ऊर्जसन॒ ऊर्ज᳚म् । \newline
47. ऊ॒र्ज॒स॒न॒ इत्यू᳚र्ज - स॒ने॒ । \newline
48. ऊर्ज॑म् धा धा॒ ऊर्ज॒ मूर्ज॑म् धाः । \newline
49. धा॒ राजा॒ राजा॑ धा धा॒ राजा᳚ । \newline
50. राजे॑वे व॒ राजा॒ राजे॑व । \newline
51. इ॒व॒ जे॒र् जे॒रि॒वे॒ व॒ जेः॒ । \newline
52. जे॒र॒वृ॒के॑ ऽवृ॒के जे᳚र् जेरवृ॒के । \newline
53. अ॒वृ॒के क्षे॑षि क्षेष्यवृ॒के॑ ऽवृ॒के क्षे॑षि । \newline
54. क्षे॒ष्य॒न्त र॒न्तः क्षे॑षि क्षेष्य॒न्तः । \newline
55. अ॒न्तरित्य॒न्तः । \newline
56. अग्न॒ आयू॒(ग्ग्॒) ष्यायू॒(ग्ग्॒)ष्यग्ने ऽग्न॒ आयू(ग्म्॑)षि । \newline
57. आयू(ग्म्॑)षि पवसे पवस॒ आयू॒(ग्ग्॒) ष्यायू(ग्म्॑)षि पवसे । \newline

\textbf{Ghana Paata } \newline

1. च॒र्॒.ष॒णी र॒भ्य॑भि च॑र्.ष॒णी श्च॑र्.ष॒णी र॒भ्या॑सा ऽऽसा ऽभि च॑र्.ष॒णी श्च॑र्.ष॒णी र॒भ्या॑सा । \newline
2. अ॒भ्या॑सा ऽऽसा ऽभ्या᳚(1॒)भ्या॑सा वाजे॑षु॒ वाजे᳚ष्वा॒सा ऽभ्या᳚(1॒)भ्या॑सा वाजे॑षु । \newline
3. आ॒सा वाजे॑षु॒ वाजे᳚ष्वा॒सा ऽऽसा वाजे॑षु सा॒सह॑थ् सा॒सह॒द् वाजे᳚ष्वा॒सा ऽऽसा वाजे॑षु सा॒सह॑त् । \newline
4. वाजे॑षु सा॒सह॑थ् सा॒सह॒द् वाजे॑षु॒ वाजे॑षु सा॒सह॑त् । \newline
5. सा॒सह॒दिति॑ सा॒सह॑त् । \newline
6. त म॑ग्ने अग्ने॒ तम् त म॑ग्ने पृतना॒सह॑म् पृतना॒सह॑ मग्ने॒ तम् त म॑ग्ने पृतना॒सह᳚म् । \newline
7. अ॒ग्ने॒ पृ॒त॒ना॒सह॑म् पृतना॒सह॑ मग्ने अग्ने पृतना॒सह(ग्म्॑) र॒यिꣳ र॒यिम् पृ॑तना॒सह॑ मग्ने अग्ने पृतना॒सह(ग्म्॑) र॒यिम् । \newline
8. पृ॒त॒ना॒सह(ग्म्॑) र॒यिꣳ र॒यिम् पृ॑तना॒सह॑म् पृतना॒सह(ग्म्॑) र॒यिꣳ स॑हस्वः सहस्वो र॒यिम् पृ॑तना॒सह॑म् पृतना॒सह(ग्म्॑) र॒यिꣳ स॑हस्वः । \newline
9. पृ॒त॒ना॒सह॒मिति॑ पृतना - सह᳚म् । \newline
10. र॒यिꣳ स॑हस्वः सहस्वो र॒यिꣳ र॒यिꣳ स॑हस्व॒ आ स॑हस्वो र॒यिꣳ र॒यिꣳ स॑हस्व॒ आ । \newline
11. स॒ह॒स्व॒ आ स॑हस्वः सहस्व॒ आ भ॑र भ॒रा स॑हस्वः सहस्व॒ आ भ॑र । \newline
12. आ भ॑र भ॒रा भ॑र । \newline
13. भ॒रेति॑ भर । \newline
14. त्वꣳहि हि त्वम् त्वꣳहि स॒त्यः स॒त्यो हि त्वम् त्वꣳहि स॒त्यः । \newline
15. हि स॒त्यः स॒त्यो हि हि स॒त्यो अद्भु॑तो॒ अद्भु॑तः स॒त्यो हि हि स॒त्यो अद्भु॑तः । \newline
16. स॒त्यो अद्भु॑तो॒ अद्भु॑तः स॒त्यः स॒त्यो अद्भु॑तो दा॒ता दा॒ता ऽद्भु॑तः स॒त्यः स॒त्यो अद्भु॑तो दा॒ता । \newline
17. अद्भु॑तो दा॒ता दा॒ता ऽद्भु॑तो॒ अद्भु॑तो दा॒ता वाज॑स्य॒ वाज॑स्य दा॒ता ऽद्भु॑तो॒ अद्भु॑तो दा॒ता वाज॑स्य । \newline
18. दा॒ता वाज॑स्य॒ वाज॑स्य दा॒ता दा॒ता वाज॑स्य॒ गोम॑तो॒ गोम॑तो॒ वाज॑स्य दा॒ता दा॒ता वाज॑स्य॒ गोम॑तः । \newline
19. वाज॑स्य॒ गोम॑तो॒ गोम॑तो॒ वाज॑स्य॒ वाज॑स्य॒ गोम॑तः । \newline
20. गोम॑त॒ इति॒ गो - म॒तः॒ । \newline
21. उ॒क्षान्ना॑य व॒शान्ना॑य व॒शान्ना॑ यो॒क्षान्ना॑ यो॒क्षान्ना॑य व॒शान्ना॑य॒ सोम॑पृष्ठाय॒ सोम॑पृष्ठाय व॒शान्ना॑ यो॒क्षान्ना॑ यो॒क्षान्ना॑य व॒शान्ना॑य॒ सोम॑पृष्ठाय । \newline
22. उ॒क्षान्ना॒येत्यु॒क्ष - अ॒न्ना॒य॒ । \newline
23. व॒शान्ना॑य॒ सोम॑पृष्ठाय॒ सोम॑पृष्ठाय व॒शान्ना॑य व॒शान्ना॑य॒ सोम॑पृष्ठाय वे॒धसे॑ वे॒धसे॒ सोम॑पृष्ठाय व॒शान्ना॑य व॒शान्ना॑य॒ सोम॑पृष्ठाय वे॒धसे᳚ । \newline
24. व॒शान्ना॒येति॑ व॒शा - अ॒न्ना॒य॒ । \newline
25. सोम॑पृष्ठाय वे॒धसे॑ वे॒धसे॒ सोम॑पृष्ठाय॒ सोम॑पृष्ठाय वे॒धसे᳚ । \newline
26. सोम॑पृष्ठा॒येति॒ सोम॑ - पृ॒ष्ठा॒य॒ । \newline
27. वे॒धस॒ इति॑ वे॒धसे᳚ । \newline
28. स्तोमै᳚र् विधेम विधेम॒ स्तोमैः॒ स्तोमै᳚र् विधेमा॒ग्नये॑ अ॒ग्नये॑ विधेम॒ स्तोमैः॒ स्तोमै᳚र् विधेमा॒ग्नये᳚ । \newline
29. वि॒धे॒मा॒ग्नये॑ अ॒ग्नये॑ विधेम विधेमा॒ग्नये᳚ । \newline
30. अ॒ग्नय॒ इत्य॒ग्नये᳚ । \newline
31. व॒द्मा हि हि व॒द्मा व॒द्मा हि सू॑नो सूनो॒ हि व॒द्मा व॒द्मा हि सू॑नो । \newline
32. हि सू॑नो सूनो॒ हि हि सू॑नो॒ अस्यसि॑ सूनो॒ हि हि सू॑नो॒ असि॑ । \newline
33. सू॒नो॒ अस्यसि॑ सूनो सूनो॒ अस्य॑द्म॒सद्वा᳚ ऽद्म॒सद्वा ऽसि॑ सूनो सूनो॒ अस्य॑द्म॒सद्वा᳚ । \newline
34. सू॒नो॒ इति॑ सूनो । \newline
35. अस्य॑द्म॒सद्वा᳚ ऽद्म॒सद्वा ऽस्य स्य॑द्म॒सद्वा॑ च॒क्रे च॒क्रे अ॑द्म॒सद्वा ऽस्य स्य॑द्म॒सद्वा॑ च॒क्रे । \newline
36. अ॒द्म॒सद्वा॑ च॒क्रे च॒क्रे अ॑द्म॒सद्वा᳚ ऽद्म॒सद्वा॑ च॒क्रे अ॒ग्नि र॒ग्नि श्च॒क्रे अ॑द्म॒सद्वा᳚ ऽद्म॒सद्वा॑ च॒क्रे अ॒ग्निः । \newline
37. अ॒द्म॒सद्वेत्य॑द्म - सद्वा᳚ । \newline
38. च॒क्रे अ॒ग्नि र॒ग्नि श्च॒क्रे च॒क्रे अ॒ग्निर् ज॒नुषा॑ ज॒नुषा॒ ऽग्निश्च॒क्रे च॒क्रे अ॒ग्निर् ज॒नुषा᳚ । \newline
39. अ॒ग्निर् ज॒नुषा॑ ज॒नुषा॒ ऽग्नि र॒ग्निर् ज॒नुषा ऽज्माज्म॑ ज॒नुषा॒ ऽग्नि र॒ग्निर् ज॒नुषा ऽज्म॑ । \newline
40. ज॒नुषा ऽज्माज्म॑ ज॒नुषा॑ ज॒नुषा ऽज्मान्न॒ मन्न॒ मज्म॑ ज॒नुषा॑ ज॒नुषा ऽज्मान्न᳚म् । \newline
41. अज्मान्न॒ मन्न॒ मज्माज्मान्न᳚म् । \newline
42. अन्न॒मित्यन्न᳚म् । \newline
43. स त्वम् त्वꣳ स स त्वम् नो॑ न॒ स्त्वꣳ स स त्वम् नः॑ । \newline
44. त्वन्नो॑ न॒ स्त्वम् त्वम् न॑ ऊर्जसन ऊर्जसने न॒ स्त्वम् त्वम् न॑ ऊर्जसने । \newline
45. न॒ ऊ॒र्ज॒स॒न॒ ऊ॒र्ज॒स॒ने॒ नो॒ न॒ ऊ॒र्ज॒स॒न॒ ऊर्ज॒ मूर्ज॑ मूर्जसने नो न ऊर्जसन॒ ऊर्ज᳚म् । \newline
46. ऊ॒र्ज॒स॒न॒ ऊर्ज॒ मूर्ज॑ मूर्जसन ऊर्जसन॒ ऊर्ज॑म् धा धा॒ ऊर्ज॑ मूर्जसन ऊर्जसन॒ ऊर्ज॑म् धाः । \newline
47. ऊ॒र्ज॒स॒न॒ इत्यू᳚र्ज - स॒ने॒ । \newline
48. ऊर्ज॑म् धा धा॒ ऊर्ज॒ मूर्ज॑म् धा॒ राजा॒ राजा॑ धा॒ ऊर्ज॒ मूर्ज॑म् धा॒ राजा᳚ । \newline
49. धा॒ राजा॒ राजा॑ धा धा॒ राजे॑वे व॒ राजा॑ धा धा॒ राजे॑व । \newline
50. राजे॑वे व॒ राजा॒ राजे॑व जेर् जेरिव॒ राजा॒ राजे॑व जेः । \newline
51. इ॒व॒ जे॒र् जे॒ रि॒वे॒ व॒ जे॒ र॒वृ॒के॑ ऽवृ॒के जे॑ रिवे व जे रवृ॒के । \newline
52. जे॒ र॒वृ॒के॑ ऽवृ॒के जे᳚र् जे रवृ॒के क्षे॑षि क्षेष्यवृ॒के जे᳚र् जे रवृ॒के क्षे॑षि । \newline
53. अ॒वृ॒के क्षे॑षि क्षेष्यवृ॒के॑ ऽवृ॒के क्षे᳚ष्य॒न्त र॒न्तः क्षे᳚ष्यवृ॒के॑ ऽवृ॒के क्षे᳚ष्य॒न्तः । \newline
54. क्षे॒ष्य॒न्त र॒न्तः क्षे॑षि क्षेष्य॒न्तः । \newline
55. अ॒न्तरित्य॒न्तः । \newline
56. अग्न॒ आयू॒(ग्ग्॒) ष्यायू॒(ग्ग्॒) ष्यग्ने ऽग्न॒ आयू(ग्म्॑)षि पवसे पवस॒ आयू॒(ग्ग्॒) ष्यग्ने ऽग्न॒ आयू(ग्म्॑)षि पवसे । \newline
57. आयू(ग्म्॑)षि पवसे पवस॒ आयू॒(ग्ग्॒) ष्यायू(ग्म्॑)षि पवस॒ आ प॑वस॒ आयू॒(ग्ग्॒) ष्यायू(ग्म्॑)षि पवस॒ आ । \newline
\pagebreak
\markright{ TS 1.3.14.8  \hfill https://www.vedavms.in \hfill}

\section{ TS 1.3.14.8 }

\textbf{TS 1.3.14.8 } \newline
\textbf{Samhita Paata} \newline

पवस॒ आ सु॒वोर्ज॒मिषं॑ च नः । आ॒रे बा॑धस्व दु॒च्छुनां᳚ ॥ अग्ने॒ पव॑स्व॒ स्वपा॑ अ॒स्मे वर्चः॑ सु॒वीर्यं᳚ । दध॒त्पोषꣳ॑ र॒यिं मयि॑ ॥ अग्ने॑ पावक रो॒चिषा॑ म॒न्द्रया॑ देव जि॒ह्वया᳚ । आ दे॒वान्. व॑क्षि॒ यक्षि॑ च ॥ स नः॑ पावक दीदि॒वोऽग्ने॑ दे॒वाꣳ इ॒हा व॑ह । उप॑ य॒ज्ञ्ꣳ ह॒विश्च॑ नः ॥ अ॒ग्निः शुचि॑व्रततमः॒ शुचि॒र् विप्रः॒ शुचिः॑ ( ) क॒विः । शुची॑ रोचत॒ आहु॑तः ॥ उद॑ग्ने॒ शुच॑य॒स्तव॑ शु॒क्रा भ्राज॑न्त ईरते । तव॒ ज्योतीꣳ॑ष्य॒र्चयः॑ ॥(पु॒रु॒नि॒ष्ठः-पु॑र्वणीक-भरा॒-ऽभि-वयो॑भि॒र्-य-आयूꣳ॑षि॒ - \newline

\textbf{Pada Paata} \newline

प॒व॒से॒ । एति॑ । सु॒व॒ । ऊर्ज᳚म् । इष᳚म् । च॒ । नः॒ ॥ आ॒रे । बा॒ध॒स्व॒ । दु॒च्छुना᳚म् ॥ अग्ने᳚ । पव॑स्व । स्वपा॒ इति॑ सु - अपाः᳚ । अ॒स्मे इति॑ । वर्चः॑ । सु॒वीर्य॒मिति॑ सु - वीर्य᳚म् ॥ दध॑त् । पोष᳚म् । र॒यिम् । मयि॑ ॥ अग्ने᳚ । पा॒व॒क॒ । रो॒चिषा᳚ । म॒न्द्रया᳚ । दे॒व॒ । जि॒ह्वया᳚ ॥ एति॑ । दे॒वान् । व॒क्षि॒ । यक्षि॑ । च॒ ॥ सः । नः॒ । पा॒व॒क॒ । दी॒दि॒वः॒ । अग्ने᳚ । दे॒वान् । इ॒ह । एति॑ । व॒ह॒ ॥ उपेति॑ । य॒ज्ञ्म् । ह॒विः । च॒ । नः॒ ॥ अ॒ग्निः । शुचि॑व्रततम॒ इति॒ शुचि॑व्रत - त॒मः॒ । शुचिः॑ । विप्रः॑ । शुचिः॑ ( ) । क॒विः ॥ शुचिः॑ । रो॒च॒ते॒ । आहु॑त॒ इत्या - हु॒तः॒ ॥ उदिति॑ । अ॒ग्ने॒ । शुच॑यः । तव॑ । शु॒क्राः । भ्राज॑न्तः । ई॒र॒ते॒ ॥ तव॑ । ज्योतीꣳ॑षि । अ॒र्चयः॑ ॥  \newline


\textbf{Krama Paata} \newline

प॒व॒स॒ आ । आ सु॑व । सु॒वोर्ज᳚म् । ऊर्ज॒मिष᳚म् । इष॑म् च । च॒ नः॒ । न॒ इति॑ नः ॥ आ॒रे बा॑धस्व । बा॒ध॒स्व॒ दु॒च्छुना᳚म् । दु॒च्छुना॒मिति॑ दु॒च्छुना᳚म् ॥ अग्ने॒ पव॑स्व । पव॑स्व॒ स्वपाः᳚ । स्वपा॑ अ॒स्मे । स्वपा॒ इति॑ सु - अपाः᳚ । अ॒स्मे वर्चः॑ । अ॒स्मे इत्य॒स्मे । वर्चः॑ सु॒वीर्य᳚म् । सु॒वीर्य॒मिति॑ सु - वीर्य᳚म् ॥ दध॒त् पोष᳚म् । पोषꣳ॑ र॒यिम् । र॒यिम् मयि॑ । मयीति॒ मयि॑ ॥ अग्ने॑ पावक । पा॒व॒क॒ रो॒चिषा᳚ । रो॒चिषा॑ म॒न्द्रया᳚ । म॒न्द्रया॑ देव । दे॒व॒ जि॒ह्वया᳚ । जि॒ह्वयेति॑ जि॒ह्वया᳚ ॥ आ दे॒वान् । दे॒वान्. व॑क्षि । व॒क्षि॒ यक्षि॑ । यक्षि॑ च । चेति॑ च ॥ स नः॑ । नः॒ पा॒व॒क॒ । पा॒व॒क॒ दी॒दि॒वः॒ । दी॒दि॒वोऽग्ने᳚ । अग्ने॑ दे॒वान् । दे॒वाꣳ इ॒ह । इ॒हा । आ व॑ह । व॒हेति॑ वह ॥ उप॑ य॒ज्ञ्म् । य॒ज्ञ्ꣳ ह॒विः । ह॒विश्च॑ । च॒ नः॒ । न॒ इति॑ नः ॥ अ॒ग्निः शुचि॑व्रततमः । शुचि॑व्रततमः॒ शुचिः॑ । शुचि॑व्रततम॒ इति॒ शुचि॑व्रत - त॒मः॒ । शुचि॒र् विप्रः॑ । विप्रः॒ शुचिः॑ ( ) । शुचिः॑ क॒विः ॥ क॒विरिति॑ क॒विः ॥ शुची॑ रोचते । रो॒च॒त॒ आहु॑तः । आहु॑त॒ इत्या - हु॒तः॒ ॥ उद॑ग्ने । अ॒ग्ने॒ शुच॑यः । शुच॑य॒स्तव॑ । तव॑ शु॒क्राः । शु॒क्रा भ्राज॑न्तः । भ्राज॑न्त ईरते । ई॒र॒त॒ इती॑रते ॥ तव॒ ज्योतीꣳ॑षि । ज्योतीꣳ॑ष्य॒र्चयः॑ । अ॒र्चय॒ इत्य॒र्चयः॑ । \newline

\textbf{Jatai Paata} \newline

1. प॒व॒स॒ आ प॑वसे पवस॒ आ । \newline
2. आ सु॑व सु॒वा सु॑व । \newline
3. सु॒वोर्ज॒ मूर्ज(ग्म्॑) सुव सु॒वोर्ज᳚म् । \newline
4. ऊर्ज॒ मिष॒ मिष॒ मूर्ज॒ मूर्ज॒ मिष᳚म् । \newline
5. इष॑म् च॒ चे ष॒ मिष॑म् च । \newline
6. च॒ नो॒ न॒श्च॒ च॒ नः॒ । \newline
7. न॒ इति॑ नः । \newline
8. आ॒रे बा॑धस्व बाधस्वा॒र आ॒रे बा॑धस्व । \newline
9. बा॒ध॒स्व॒ दु॒च्छुना᳚म् दु॒च्छुना᳚म् बाधस्व बाधस्व दु॒च्छुना᳚म् । \newline
10. दु॒च्छुना॒मिति॑ दु॒च्छुना᳚म् । \newline
11. अग्ने॒ पव॑स्व॒ पव॒स्वाग्ने ऽग्ने॒ पव॑स्व । \newline
12. पव॑स्व॒ स्वपाः॒ स्वपाः॒ पव॑स्व॒ पव॑स्व॒ स्वपाः᳚ । \newline
13. स्वपा॑ अ॒स्मे अ॒स्मे स्वपाः॒ स्वपा॑ अ॒स्मे । \newline
14. स्वपा॒ इति॑ सु - अपाः᳚ । \newline
15. अ॒स्मे वर्चो॒ वर्चो॑ अ॒स्मे अ॒स्मे वर्चः॑ । \newline
16. अ॒स्मे इत्य॒स्मे । \newline
17. वर्चः॑ सु॒वीर्य(ग्म्॑) सु॒वीर्यं॒ ॅवर्चो॒ वर्चः॑ सु॒वीर्य᳚म् । \newline
18. सु॒वीर्य॒मिति॑ सु - वीर्य᳚म् । \newline
19. दध॒त् पोष॒म् पोष॒म् दध॒द् दध॒त् पोष᳚म् । \newline
20. पोष(ग्म्॑) र॒यिꣳ र॒यिम् पोष॒म् पोष(ग्म्॑) र॒यिम् । \newline
21. र॒यिम् मयि॒ मयि॑ र॒यिꣳ र॒यिम् मयि॑ । \newline
22. मयीति॒ मयि॑ । \newline
23. अग्ने॑ पावक पाव॒काग्ने ऽग्ने॑ पावक । \newline
24. पा॒व॒क॒ रो॒चिषा॑ रो॒चिषा॑ पावक पावक रो॒चिषा᳚ । \newline
25. रो॒चिषा॑ म॒न्द्रया॑ म॒न्द्रया॑ रो॒चिषा॑ रो॒चिषा॑ म॒न्द्रया᳚ । \newline
26. म॒न्द्रया॑ देव देव म॒न्द्रया॑ म॒न्द्रया॑ देव । \newline
27. दे॒व॒ जि॒ह्वया॑ जि॒ह्वया॑ देव देव जि॒ह्वया᳚ । \newline
28. जि॒ह्वयेति॑ जि॒ह्वया᳚ । \newline
29. आ दे॒वान् दे॒वा ना दे॒वान् । \newline
30. दे॒वान्. व॑क्षि वक्षि दे॒वान् दे॒वान्. व॑क्षि । \newline
31. व॒क्षि॒ यक्षि॒ यक्षि॑ वक्षि वक्षि॒ यक्षि॑ । \newline
32. यक्षि॑ च च॒ यक्षि॒ यक्षि॑ च । \newline
33. चेति॑ च । \newline
34. स नो॑ नः॒ स स नः॑ । \newline
35. नः॒ पा॒व॒क॒ पा॒व॒क॒ नो॒ नः॒ पा॒व॒क॒ । \newline
36. पा॒व॒क॒ दी॒दि॒वो॒ दी॒दि॒वः॒ पा॒व॒क॒ पा॒व॒क॒ दी॒दि॒वः॒ । \newline
37. दी॒दि॒वो ऽग्ने ऽग्ने॑ दीदिवो दीदि॒वो ऽग्ने᳚ । \newline
38. अग्ने॑ दे॒वान् दे॒वाꣳ अग्ने ऽग्ने॑ दे॒वान् । \newline
39. दे॒वाꣳ इ॒हे ह दे॒वान् दे॒वाꣳ इ॒ह । \newline
40. इ॒हेहे हा । \newline
41. आ व॑ह व॒हा व॑ह । \newline
42. व॒हेति॑ वह । \newline
43. उप॑ य॒ज्ञ्ं ॅय॒ज्ञ् मुपोप॑ य॒ज्ञ्म् । \newline
44. य॒ज्ञ्ꣳ ह॒विर्. ह॒विर् य॒ज्ञ्ं ॅय॒ज्ञ्ꣳ ह॒विः । \newline
45. ह॒विश्च॑ च ह॒विर्. ह॒विश्च॑ । \newline
46. च॒ नो॒ न॒श्च॒ च॒ नः॒ । \newline
47. न॒ इति॑ नः । \newline
48. अ॒ग्निः शुचि॑व्रततमः॒ शुचि॑व्रततमो॒ ऽग्नि र॒ग्निः शुचि॑व्रततमः । \newline
49. शुचि॑व्रततमः॒ शुचिः॒ शुचिः॒ शुचि॑व्रततमः॒ शुचि॑व्रततमः॒ शुचिः॑ । \newline
50. शुचि॑व्रततम॒ इति॒ शुचि॑व्रत - त॒मः॒ । \newline
51. शुचि॒र् विप्रो॒ विप्रः॒ शुचिः॒ शुचि॒र् विप्रः॑ । \newline
52. विप्रः॒ शुचिः॒ शुचि॒र् विप्रो॒ विप्रः॒ शुचिः॑ । \newline
53. शुचिः॑ क॒विः क॒विः शुचिः॒ शुचिः॑ क॒विः । \newline
54. क॒विरिति॑ क॒विः । \newline
55. शुची॑ रोचते रोचते॒ शुचिः॒ शुची॑ रोचते । \newline
56. रो॒च॒त॒ आहु॑त॒ आहु॑तो रोचते रोचत॒ आहु॑तः । \newline
57. आहु॑त॒ इत्या - हु॒तः॒ । \newline
58. उद॑ग्ने अग्न॒ उदुद॑ग्ने । \newline
59. अ॒ग्ने॒ शुच॑यः॒ शुच॑यो अग्ने अग्ने॒ शुच॑यः । \newline
60. शुच॑य॒ स्तव॒ तव॒ शुच॑यः॒ शुच॑य॒ स्तव॑ । \newline
61. तव॑ शु॒क्राः शु॒क्रा स्तव॒ तव॑ शु॒क्राः । \newline
62. शु॒क्रा भ्राज॑न्तो॒ भ्राज॑न्तः शु॒क्राः शु॒क्रा भ्राज॑न्तः । \newline
63. भ्राज॑न्त ईरत ईरते॒ भ्राज॑न्तो॒ भ्राज॑न्त ईरते । \newline
64. ई॒र॒त॒ इती॑रते । \newline
65. तव॒ ज्योती(ग्म्॑)षि॒ ज्योती(ग्म्॑)षि॒ तव॒ तव॒ ज्योती(ग्म्॑)षि । \newline
66. ज्योती(ग्ग्॑)ष्य॒र्चयो॑ अ॒र्चयो॒ ज्योती(ग्म्॑)षि॒ ज्योती(ग्ग्॑)ष्य॒र्चयः॑ । \newline
67. अ॒र्चय॒ इत्य॒र्चयः॑ । \newline

\textbf{Ghana Paata } \newline

1. प॒व॒स॒ आ प॑वसे पवस॒ आ सु॑व सु॒वा प॑वसे पवस॒ आ सु॑व । \newline
2. आ सु॑व सु॒वा सु॒वोर्ज॒ मूर्ज(ग्म्॑) सु॒वा सु॒वोर्ज᳚म् । \newline
3. सु॒वोर्ज॒ मूर्ज(ग्म्॑) सुव सु॒वोर्ज॒ मिष॒ मिष॒ मूर्ज(ग्म्॑) सुव सु॒वोर्ज॒ मिष᳚म् । \newline
4. ऊर्ज॒ मिष॒ मिष॒ मूर्ज॒ मूर्ज॒ मिष॑म् च॒ चे ष॒ मूर्ज॒ मूर्ज॒ मिष॑म् च । \newline
5. इष॑म् च॒ चे ष॒ मिष॑म् च नो न॒ श्चे ष॒ मिष॑म् च नः । \newline
6. च॒ नो॒ न॒श्च॒ च॒ नः॒ । \newline
7. न॒ इति॑ नः । \newline
8. आ॒रे बा॑धस्व बाधस्वा॒र आ॒रे बा॑धस्व दु॒च्छुना᳚म् दु॒च्छुना᳚म् बाधस्वा॒र आ॒रे बा॑धस्व दु॒च्छुना᳚म् । \newline
9. बा॒ध॒स्व॒ दु॒च्छुना᳚म् दु॒च्छुना᳚म् बाधस्व बाधस्व दु॒च्छुना᳚म् । \newline
10. दु॒च्छुना॒मिति॑ दु॒च्छुना᳚म् । \newline
11. अग्ने॒ पव॑स्व॒ पव॒स्वाग्ने ऽग्ने॒ पव॑स्व॒ स्वपाः॒ स्वपाः॒ पव॒स्वाग्ने ऽग्ने॒ पव॑स्व॒ स्वपाः᳚ । \newline
12. पव॑स्व॒ स्वपाः॒ स्वपाः॒ पव॑स्व॒ पव॑स्व॒ स्वपा॑ अ॒स्मे अ॒स्मे स्वपाः॒ पव॑स्व॒ पव॑स्व॒ स्वपा॑ अ॒स्मे । \newline
13. स्वपा॑ अ॒स्मे अ॒स्मे स्वपाः॒ स्वपा॑ अ॒स्मे वर्चो॒ वर्चो॑ अ॒स्मे स्वपाः॒ स्वपा॑ अ॒स्मे वर्चः॑ । \newline
14. स्वपा॒ इति॑ सु - अपाः᳚ । \newline
15. अ॒स्मे वर्चो॒ वर्चो॑ अ॒स्मे अ॒स्मे वर्चः॑ सु॒वीर्य(ग्म्॑) सु॒वीर्यं॒ ॅवर्चो॑ अ॒स्मे अ॒स्मे वर्चः॑ सु॒वीर्य᳚म् । \newline
16. अ॒स्मे इत्य॒स्मे । \newline
17. वर्चः॑ सु॒वीर्य(ग्म्॑) सु॒वीर्यं॒ ॅवर्चो॒ वर्चः॑ सु॒वीर्य᳚म् । \newline
18. सु॒वीर्य॒मिति॑ सु - वीर्य᳚म् । \newline
19. दध॒त् पोष॒म् पोष॒म् दध॒द् दध॒त् पोष(ग्म्॑) र॒यिꣳ र॒यिम् पोष॒म् दध॒द् दध॒त् पोष(ग्म्॑) र॒यिम् । \newline
20. पोष(ग्म्॑) र॒यिꣳ र॒यिम् पोष॒म् पोष(ग्म्॑) र॒यिम् मयि॒ मयि॑ र॒यिम् पोष॒म् पोष(ग्म्॑) र॒यिम् मयि॑ । \newline
21. र॒यिम् मयि॒ मयि॑ र॒यिꣳ र॒यिम् मयि॑ । \newline
22. मयीति॒ मयि॑ । \newline
23. अग्ने॑ पावक पाव॒काग्ने ऽग्ने॑ पावक रो॒चिषा॑ रो॒चिषा॑ पाव॒काग्ने ऽग्ने॑ पावक रो॒चिषा᳚ । \newline
24. पा॒व॒क॒ रो॒चिषा॑ रो॒चिषा॑ पावक पावक रो॒चिषा॑ म॒न्द्रया॑ म॒न्द्रया॑ रो॒चिषा॑ पावक पावक रो॒चिषा॑ म॒न्द्रया᳚ । \newline
25. रो॒चिषा॑ म॒न्द्रया॑ म॒न्द्रया॑ रो॒चिषा॑ रो॒चिषा॑ म॒न्द्रया॑ देव देव म॒न्द्रया॑ रो॒चिषा॑ रो॒चिषा॑ म॒न्द्रया॑ देव । \newline
26. म॒न्द्रया॑ देव देव म॒न्द्रया॑ म॒न्द्रया॑ देव जि॒ह्वया॑ जि॒ह्वया॑ देव म॒न्द्रया॑ म॒न्द्रया॑ देव जि॒ह्वया᳚ । \newline
27. दे॒व॒ जि॒ह्वया॑ जि॒ह्वया॑ देव देव जि॒ह्वया᳚ । \newline
28. जि॒ह्वयेति॑ जि॒ह्वया᳚ । \newline
29. आ दे॒वान् दे॒वा ना दे॒वान्. व॑क्षि वक्षि दे॒वा ना दे॒वान्. व॑क्षि । \newline
30. दे॒वान्. व॑क्षि वक्षि दे॒वान् दे॒वान्. व॑क्षि॒ यक्षि॒ यक्षि॑ वक्षि दे॒वान् दे॒वान्. व॑क्षि॒ यक्षि॑ । \newline
31. व॒क्षि॒ यक्षि॒ यक्षि॑ वक्षि वक्षि॒ यक्षि॑ च च॒ यक्षि॑ वक्षि वक्षि॒ यक्षि॑ च । \newline
32. यक्षि॑ च च॒ यक्षि॒ यक्षि॑ च । \newline
33. चेति॑ च । \newline
34. स नो॑ नः॒ स स नः॑ पावक पावक नः॒ स स नः॑ पावक । \newline
35. नः॒ पा॒व॒क॒ पा॒व॒क॒ नो॒ नः॒ पा॒व॒क॒ दी॒दि॒वो॒ दी॒दि॒वः॒ पा॒व॒क॒ नो॒ नः॒ पा॒व॒क॒ दी॒दि॒वः॒ । \newline
36. पा॒व॒क॒ दी॒दि॒वो॒ दी॒दि॒वः॒ पा॒व॒क॒ पा॒व॒क॒ दी॒दि॒वो ऽग्ने ऽग्ने॑ दीदिवः पावक पावक दीदि॒वो ऽग्ने᳚ । \newline
37. दी॒दि॒वो ऽग्ने ऽग्ने॑ दीदिवो दीदि॒वो ऽग्ने॑ दे॒वान् दे॒वाꣳ अग्ने॑ दीदिवो दीदि॒वो ऽग्ने॑ दे॒वान् । \newline
38. अग्ने॑ दे॒वान् दे॒वाꣳ अग्ने ऽग्ने॑ दे॒वाꣳ इ॒हे ह दे॒वाꣳ अग्ने ऽग्ने॑ दे॒वाꣳ इ॒ह । \newline
39. दे॒वाꣳ इ॒हे ह दे॒वान् दे॒वाꣳ इ॒हेह दे॒वान् दे॒वाꣳ इ॒हा । \newline
40. इ॒हे हे हा व॑ह व॒हे हे हा व॑ह । \newline
41. आ व॑ह व॒हा व॑ह । \newline
42. व॒हेति॑ वह । \newline
43. उप॑ य॒ज्ञ्ं ॅय॒ज्ञ् मुपोप॑ य॒ज्ञ्ꣳ ह॒विर्. ह॒विर् य॒ज्ञ् मुपोप॑ य॒ज्ञ्ꣳ ह॒विः । \newline
44. य॒ज्ञ्ꣳ ह॒विर्. ह॒विर् य॒ज्ञ्ं ॅय॒ज्ञ्ꣳ ह॒विश्च॑ च ह॒विर् य॒ज्ञ्ं ॅय॒ज्ञ्ꣳ ह॒विश्च॑ । \newline
45. ह॒विश्च॑ च ह॒विर्. ह॒विश्च॑ नो नश्च ह॒विर्. ह॒विश्च॑ नः । \newline
46. च॒ नो॒ न॒श्च॒ च॒ नः॒ । \newline
47. न॒ इति॑ नः । \newline
48. अ॒ग्निः शुचि॑व्रततमः॒ शुचि॑व्रततमो॒ ऽग्निर॒ग्निः शुचि॑व्रततमः॒ शुचिः॒ शुचिः॒ शुचि॑व्रततमो॒ ऽग्निर॒ग्निः शुचि॑व्रततमः॒ शुचिः॑ । \newline
49. शुचि॑व्रततमः॒ शुचिः॒ शुचिः॒ शुचि॑व्रततमः॒ शुचि॑व्रततमः॒ शुचि॒र् विप्रो॒ विप्रः॒ शुचिः॒ शुचि॑व्रततमः॒ शुचि॑व्रततमः॒ शुचि॒र् विप्रः॑ । \newline
50. शुचि॑व्रततम॒ इति॒ शुचि॑व्रत - त॒मः॒ । \newline
51. शुचि॒र् विप्रो॒ विप्रः॒ शुचिः॒ शुचि॒र् विप्रः॒ शुचिः॒ शुचि॒र् विप्रः॒ शुचिः॒ शुचि॒र् विप्रः॒ शुचिः॑ । \newline
52. विप्रः॒ शुचिः॒ शुचि॒र् विप्रो॒ विप्रः॒ शुचिः॑ क॒विः क॒विः शुचि॒र् विप्रो॒ विप्रः॒ शुचिः॑ क॒विः । \newline
53. शुचिः॑ क॒विः क॒विः शुचिः॒ शुचिः॑ क॒विः । \newline
54. क॒विरिति॑ क॒विः । \newline
55. शुची॑ रोचते रोचते॒ शुचिः॒ शुची॑ रोचत॒ आहु॑त॒ आहु॑तो रोचते॒ शुचिः॒ शुची॑ रोचत॒ आहु॑तः । \newline
56. रो॒च॒त॒ आहु॑त॒ आहु॑तो रोचते रोचत॒ आहु॑तः । \newline
57. आहु॑त॒ इत्या - हु॒तः॒ । \newline
58. उद॑ग्ने अग्न॒ उदुद॑ग्ने॒ शुच॑यः॒ शुच॑यो अग्न॒ उदुद॑ग्ने॒ शुच॑यः । \newline
59. अ॒ग्ने॒ शुच॑यः॒ शुच॑यो अग्ने अग्ने॒ शुच॑य॒ स्तव॒ तव॒ शुच॑यो अग्ने अग्ने॒ शुच॑य॒ स्तव॑ । \newline
60. शुच॑य॒ स्तव॒ तव॒ शुच॑यः॒ शुच॑य॒ स्तव॑ शु॒क्राः शु॒क्रा स्तव॒ शुच॑यः॒ शुच॑य॒ स्तव॑ शु॒क्राः । \newline
61. तव॑ शु॒क्राः शु॒क्रा स्तव॒ तव॑ शु॒क्रा भ्राज॑न्तो॒ भ्राज॑न्तः शु॒क्रा स्तव॒ तव॑ शु॒क्रा भ्राज॑न्तः । \newline
62. शु॒क्रा भ्राज॑न्तो॒ भ्राज॑न्तः शु॒क्राः शु॒क्रा भ्राज॑न्त ईरत ईरते॒ भ्राज॑न्तः शु॒क्राः शु॒क्रा भ्राज॑न्त ईरते । \newline
63. भ्राज॑न्त ईरत ईरते॒ भ्राज॑न्तो॒ भ्राज॑न्त ईरते । \newline
64. ई॒र॒त॒ इती॑रते । \newline
65. तव॒ ज्योती(ग्म्॑)षि॒ ज्योती(ग्म्॑)षि॒ तव॒ तव॒ ज्योती(ग्ग्॑) ष्य॒र्चयो॑ अ॒र्चयो॒ ज्योती(ग्म्॑)षि॒ तव॒ तव॒ ज्योती(ग्ग्॑) ष्य॒र्चयः॑ । \newline
66. ज्योती(ग्ग्॑) ष्य॒र्चयो॑ अ॒र्चयो॒ ज्योती(ग्म्॑)षि॒ ज्योती(ग्ग्॑) ष्य॒र्चयः॑ । \newline
67. अ॒र्चय॒ इत्य॒र्चयः॑ । \newline
\pagebreak


\end{document}