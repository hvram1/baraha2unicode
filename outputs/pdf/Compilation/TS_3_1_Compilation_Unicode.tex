\documentclass[17pt]{extarticle}
\usepackage{babel}
\usepackage{fontspec}
\usepackage{polyglossia}
\usepackage{extsizes}



\setmainlanguage{sanskrit}
\setotherlanguages{english} %% or other languages
\setlength{\parindent}{0pt}
\pagestyle{myheadings}
\newfontfamily\devanagarifont[Script=Devanagari]{AdishilaVedic}


\newcommand{\VAR}[1]{}
\newcommand{\BLOCK}[1]{}




\begin{document}
\begin{titlepage}
    \begin{center}
 
\begin{sanskrit}
    { \Huge
    कृष्ण यजुर्वेदीय तैत्तिरीय संहिता,पद,जटा,घन पाठः 
    }
    \\
    \vspace{2.5cm}
    \mbox{ \Huge
    3.1     तृतीयकाण्डे प्रथमः प्रश्नः - न्यूनकर्माभिधानं   }
\end{sanskrit}
\end{center}

\end{titlepage}
\tableofcontents
\pagebreak

\markright{ TS 3.1.1.1  \hfill https://www.vedavms.in \hfill}
\addcontentsline{toc}{section}{ TS 3.1.1.1 }
\section*{ TS 3.1.1.1 }

\textbf{TS 3.1.1.1 } \newline
\textbf{Samhita Paata} \newline

प्र॒जाप॑तिरकामयत प्र॒जाः सृ॑जे॒येति॒ स तपो॑ऽतप्यत॒ स स॒र्पान॑सृजत॒ सो॑ऽकामयत प्र॒जाः सृ॑जे॒येति॒ सद्वि॒तीय॑मतप्यत॒ स वयाꣳ॑स्य सृजत॒ सो॑ऽकामयत प्र॒जाः सृ॑जे॒येति॒ स तृ॒तीय॑मतप्यत॒ स ए॒तं दी᳚क्षितवा॒द-म॑पश्य॒त् तम॑वद॒त् ततो॒ वै स प्र॒जा अ॑सृजत॒ यत् तप॑स्त॒प्त्वा दी᳚क्षितवा॒दं ॅवद॑ति प्र॒जा ए॒व तद्यज॑मानः - [  ] \newline

\textbf{Pada Paata} \newline

प्र॒जाप॑ति॒रिति॑ प्र॒जा - प॒तिः॒ । अ॒का॒म॒य॒त॒ । प्र॒जा इति॑ प्र - जाः । सृ॒जे॒य॒ । इति॑ । सः । तपः॑ । अ॒त॒प्य॒त॒ । सः । स॒र्पान् । अ॒सृ॒ज॒त॒ । सः । अ॒का॒म॒य॒त॒ । प्र॒जा इति॑ प्र - जाः । सृ॒जे॒य॒ । इति॑ । सः । द्वि॒तीय᳚म् । अ॒त॒प्य॒त॒ । सः । वयाꣳ॑सि । अ॒सृ॒ज॒त॒ । सः । अ॒का॒म॒य॒त॒ । प्र॒जा इति॑ प्र-जाः । सृ॒जे॒य॒ । इति॑ । सः । तृ॒तीय᳚म् । अ॒त॒प्य॒त॒ । सः । ए॒तम् । दी॒क्षि॒त॒वा॒दमिति॑ दीक्षित - वा॒दम् । अ॒प॒श्य॒त् । तम् । अ॒व॒द॒त् । ततः॑ । वै । सः । प्र॒जा इति॑ प्र-जाः । अ॒सृ॒ज॒त॒ । यत् । तपः॑ । त॒प्त्वा । दी॒क्षि॒त॒वा॒दमिति॑ दीक्षित - वा॒दम् । वद॑ति । प्र॒जा इति॑ प्र - जाः । ए॒व । तत् । यज॑मानः ।  \newline




\markright{ TS 3.1.1.2  \hfill https://www.vedavms.in \hfill}
\addcontentsline{toc}{section}{ TS 3.1.1.2 }
\section*{ TS 3.1.1.2 }

\textbf{TS 3.1.1.2 } \newline
\textbf{Samhita Paata} \newline

सृजते॒ यद्वै दी᳚क्षि॒तो॑ऽमे॒द्ध्यं पश्य॒त्यपा᳚स्माद्दी॒क्षाक्रा॑मति॒ नील॑मस्य॒ हरो॒ व्ये᳚त्यब॑द्धं॒ मनो॑ द॒रिद्रं॒ चक्षुः॒ सूर्यो॒ ज्योति॑षाꣳ॒॒श्रेष्ठो॒ दीक्षे॒ मा मा॑हासी॒रित्या॑ह॒ नास्मा᳚द्दी॒क्षाऽप॑क्रामति॒ नास्य॒ नीलं॒ न हरो॒ व्ये॑ति॒ यद्वै दी᳚क्षि॒तम॑भि॒वर्.ष॑तिदि॒व्या आपोऽशा᳚न्ता॒ ओजो॒ बलं॑ दी॒क्षां - [  ] \newline

\textbf{Pada Paata} \newline

सृ॒ज॒ते॒ । यत् । वै । दी॒क्षि॒तः । अ॒मे॒द्ध्यम् । पश्य॑ति । अपेति॑ । अ॒स्मा॒त् । दी॒क्षा । क्रा॒म॒ति॒ । नील᳚म् । अ॒स्य॒ । हरः॑ । वीति॑ । ए॒ति॒ । अब॑द्धम् । मनः॑ । द॒रिद्र᳚म् । चक्षुः॑ । सूर्यः॑ । ज्योति॑षाम् । श्रेष्ठः॑ । दीक्षे᳚ । मा । मा॒ । हा॒सीः॒ । इति॑ । आ॒ह॒ । न । अ॒स्मा॒त् । दी॒क्षा । अपेति॑ । क्रा॒म॒ति॒ । न । अ॒स्य॒ । नील᳚म् । न । हरः॑ । वीति॑ । ए॒ति॒ । यत् । वै । दी॒क्षि॒तम् । अ॒भि॒वर्.ष॒तीत्य॑भि-वर्.ष॑ति । दि॒व्याः । आपः॑ । अशा᳚न्ताः । ओजः॑ । बल᳚म् । दी॒क्षाम् ।  \newline




\markright{ TS 3.1.1.3  \hfill https://www.vedavms.in \hfill}
\addcontentsline{toc}{section}{ TS 3.1.1.3 }
\section*{ TS 3.1.1.3 }

\textbf{TS 3.1.1.3 } \newline
\textbf{Samhita Paata} \newline

तपो᳚ऽस्य॒निर्घ्न॑न्त्युन्द॒तीर् बलं॑ ध॒त्तौजो॑ धत्त॒ बलं॑ धत्त॒ मा मे॑ दी॒क्षां मा तपो॒निर्व॑धि॒ष्टेत्या॑है॒ तदे॒व सर्व॑मा॒त्मन् ध॑त्ते॒ नास्यौजो॒ बलं॒ न दी॒क्षां न तपो॒निर्घ्न॑न्त्य॒ग्निर्वै दी᳚क्षि॒तस्य॑ दे॒वता॒ सो᳚ऽस्मादे॒तर्.हि॑ति॒र इ॑व॒ यर्.हि॒ याति॒ तमी᳚श्व॒रꣳ रक्षाꣳ॑सि॒ हन्तो᳚ -  [  ] \newline

\textbf{Pada Paata} \newline

तपः॑ । अ॒स्य॒ । निरिति॑ । घ्न॒न्ति॒ । उ॒न्द॒तीः । बल᳚म् । ध॒त्त॒ । ओजः॑ । ध॒त्त॒ । बल᳚म् । ध॒त्त॒ । मा । मे॒ । दी॒क्षाम् । मा । तपः॑ । निरिति॑ । व॒धि॒ष्ट॒ । इति॑ । आ॒ह॒ । ए॒तत् । ए॒व । सर्व᳚म् । आ॒त्मन्न् । ध॒त्ते॒ । न । अ॒स्य॒ । ओजः॑ । बल᳚म् । न । दी॒क्षाम् । न । तपः॑ । निरिति॑ । घ्न॒न्ति॒ । अ॒ग्निः । वै । दी॒क्षि॒तस्य॑ । दे॒वता᳚ । सः । अ॒स्मा॒त् । ए॒तर्.हि॑ । ति॒रः । इ॒व॒ । यर्.हि॑ । याति॑ । तम् । ई॒श्व॒रम् । रक्षाꣳ॑सि । हन्तोः᳚ ।  \newline




\markright{ TS 3.1.1.4  \hfill https://www.vedavms.in \hfill}
\addcontentsline{toc}{section}{ TS 3.1.1.4 }
\section*{ TS 3.1.1.4 }

\textbf{TS 3.1.1.4 } \newline
\textbf{Samhita Paata} \newline

र्भ॒द्राद॒भिश्रेयः॒ प्रेहि॒बृह॒स्पतिः॑ पुर ए॒ता ते॑ अ॒स्त्वित्या॑ह॒ब्रह्म॒ वै दे॒वानां॒ बृह॒स्पति॒स्तमे॒वान्वा र॑भते॒ स ए॑नꣳ॒॒ सं पा॑रय॒त्ये दम॑गन्म देव॒यज॑नं पृथि॒व्या इत्या॑ह देव॒यज॑नꣳ॒॒ ह्ये॑ष पृ॑थि॒व्या आ॒गच्छ॑ति॒ यो यज॑ते॒ विश्वे॑ दे॒वा यदजु॑षन्त॒ पूर्व॒ इत्या॑ह॒ विश्वे॒ ह्ये॑तद्दे॒वा जो॒षय॑न्ते॒ यद्ब्रा᳚ह्म॒णा ( ) ऋ॑ख्सा॒माभ्यां॒ ॅयजु॑षा स॒न्तर॑न्त॒ इत्या॑हर्ख्सा॒माभ्याꣳ॒॒ ह्ये॑ष यजु॑षा स॒न्तर॑ति॒ यो यज॑ते रा॒यस्पोषे॑ण॒ समि॒षा-म॑दे॒मेत्या॑-हा॒ऽशिष॑मे॒वै तामा शा᳚स्ते ॥ \newline

\textbf{Pada Paata} \newline

भ॒द्रात् । अ॒भीति॑ । श्रेयः॑ । प्रेति॑ । इ॒हि॒ । बृह॒स्पतिः॑ । पु॒र॒ ए॒तेति॑ पुरः - ए॒ता । ते॒ । अ॒स्तु॒ । इति॑ । आ॒ह॒ । ब्रह्म॑ । वै । दे॒वाना᳚म् । बृह॒स्पतिः॑ । तम् । ए॒व । अ॒न्वार॑भत॒ इत्य॑नु - आर॑भते । सः । ए॒न॒म् । समिति॑ । पा॒र॒य॒ति॒ । एति॑ । इ॒दम् । अ॒ग॒न्म॒ । दे॒व॒यज॑न॒मिति॑ देव - यज॑नम् । पृ॒थि॒व्याः । इति॑ । आ॒ह॒ । दे॒व॒यज॑न॒मिति॑ देव - यज॑नम् । हि । ए॒षः । पृ॒थि॒व्याः । आ॒गच्छ॒तीत्या᳚ - गच्छ॑ति । यः । यज॑ते । विश्वे᳚ । दे॒वाः । यत् । अजु॑षन्त । पूर्वे᳚ । इति॑ । आ॒ह॒ । विश्वे᳚ । हि । ए॒तत् । दे॒वाः । जो॒षय॑न्ते । यत् । ब्रा॒ह्म॒णाः ( ) । ऋ॒ख्सा॒माभ्या॒मित्यृ॑ख्सा॒मा -भ्या॒म् । यजु॑षा । स॒न्तर॑न्त॒ इति॑ सं - तर॑न्तः । इति॑ । आ॒ह॒ । ऋ॒ख्सा॒माभ्या॒मित्यृ॑ख्सा॒मा - भ्या॒म् । हि । ए॒षः । यजु॑षा । स॒न्तर॒तीति॑ सं - तर॑ति । यः । यज॑ते । रा॒यः । पोषे॑ण । समिति॑ । इ॒षा । म॒दे॒म॒ । इति॑ । आ॒ह॒ । आ॒शिष॒मित्या᳚ - शिष᳚म् । ए॒व । ए॒ताम् । एति॑ । शा॒स्ते॒ ॥  \newline




\markright{ TS 3.1.2.1  \hfill https://www.vedavms.in \hfill}
\addcontentsline{toc}{section}{ TS 3.1.2.1 }
\section*{ TS 3.1.2.1 }

\textbf{TS 3.1.2.1 } \newline
\textbf{Samhita Paata} \newline

ए॒ष ते॑ गाय॒त्रो भा॒ग इति॑ मे॒ सोमा॑य ब्रूतादे॒ष ते॒ त्रैष्टु॑भो॒ जाग॑तो भा॒ग इति॑ मे॒ सोमा॑य ब्रूताच्छन्दो॒मानाꣳ॒॒ साम्रा᳚ज्यं ग॒च्छेति॑ मे॒ सोमा॑य ब्रूता॒द् यो वै सोमꣳ॒॒ राजा॑नꣳ॒॒ साम्रा᳚ज्यं ॅलो॒कं ग॑मयि॒त्वा क्री॒णाति॒ गच्छ॑ति॒ स्वानाꣳ॒॒ साम्रा᳚ज्यं॒ छन्दाꣳ॑सि॒ खलु॒ वै सोम॑स्य॒ राज्ञ्ः॒ साम्रा᳚ज्यो लो॒कः पु॒रस्ता॒थ् सोम॑स्य क्र॒यादे॒वम॒भि म॑न्त्रयेत॒ साम्रा᳚ज्यमे॒वै - [  ] \newline

\textbf{Pada Paata} \newline

ए॒षः । ते॒ । गा॒य॒त्रः । भा॒गः । इति॑ । मे॒ । सोमा॑य । ब्रू॒ता॒त् । ए॒षः । ते॒ । त्रैष्टु॑भः । जाग॑तः । भा॒गः । इति॑ । मे॒ । सोमा॑य । ब्रू॒ता॒त् । छ॒न्दो॒माना॒मिति॑ छन्दः - माना᳚म् । साम्रा᳚ज्य॒मिति॒ साम् - रा॒ज्य॒म् । ग॒च्छ॒ । इति॑ । मे॒ । सोमा॑य । ब्रू॒ता॒त् । यः । वै । सोम᳚म् । राजा॑नम् । साम्रा᳚ज्य॒मिति॒ साम् - रा॒ज्य॒म् । लो॒कम् । ग॒म॒यि॒त्वा । क्री॒णाति॑ । गच्छ॑ति । स्वाना᳚म् । साम्रा᳚ज्य॒मिति॒ साम् - रा॒ज्य॒म् । छन्दाꣳ॑सि । खलु॑ । वै । सोम॑स्य । राज्ञ्ः॑ । साम्रा᳚ज्य॒ इति॒ साम् - रा॒ज्यः॒ । लो॒कः । पु॒रस्ता᳚त् । सोम॑स्य । क्र॒यात् । ए॒वम् । अ॒भीति॑ । म॒न्त्र॒ये॒त॒ । साम्रा᳚ज्य॒मिति॒ साम् - रा॒ज्य॒म् । ए॒व ।  \newline




\markright{ TS 3.1.2.2  \hfill https://www.vedavms.in \hfill}
\addcontentsline{toc}{section}{ TS 3.1.2.2 }
\section*{ TS 3.1.2.2 }

\textbf{TS 3.1.2.2 } \newline
\textbf{Samhita Paata} \newline

नं॑ ॅलो॒कं ग॑मयि॒त्वा क्री॑णाति॒ गच्छ॑ति॒ स्वानाꣳ॒॒ साम्रा᳚ज्यं॒ ॅयो वै ता॑नून॒प्त्रस्य॑ प्रति॒ष्ठां ॅवेद॒ प्रत्ये॒व ति॑ष्ठति ब्रह्मवा॒दिनो॑ वदन्ति॒ न प्रा॒श्नन्ति॒ न जु॑ह्व॒त्यथ॒ क्व॑ तानून॒प्त्रं प्रति॑ तिष्ठ॒तीति॑ प्र॒जाप॑तौ॒ मन॒सीति॑ ब्रूया॒त् त्रिरव॑ जिघ्रेत् प्र॒जाप॑तौ त्वा॒ मन॑सि जुहो॒मीत्ये॒षा वै ता॑नून॒प्त्रस्य॑ प्रति॒ष्ठा य ए॒वं ॅवेद॒ प्रत्ये॒व ति॑ष्ठति॒ यो - [  ] \newline

\textbf{Pada Paata} \newline

ए॒न॒म् । लो॒कम् । ग॒म॒यि॒त्वा । क्री॒णा॒ति॒ । गच्छ॑ति । स्वाना᳚म् । साम्रा᳚ज्य॒मिति॒ साम् - रा॒ज्य॒म् । यः । वै । ता॒नू॒न॒प्त्रस्येति॑ तानू - न॒प्त्रस्य॑ । प्र॒ति॒ष्ठामिति॑ प्रति - स्थाम् । वेद॑ । प्रतीति॑ । ए॒व । ति॒ष्ठ॒ति॒ । ब्र॒ह्म॒वा॒दिन॒ इति॑ ब्रह्म - वा॒दिनः॑ । व॒द॒न्ति॒ । न । प्रा॒श्नन्तीति॑ प्र - अ॒श्नन्ति॑ । न । जु॒ह्व॒ति॒ । अथ॑ । क्व॑ । ता॒नू॒न॒प्त्रमिति॑ तानू - न॒प्त्रम् । प्रतीति॑ । ति॒ष्ठ॒ति॒ । इति॑ । प्र॒जाप॑ता॒विति॑ प्र॒जा-प॒तौ॒ । मन॑सि । इति॑ । ब्रू॒या॒त् । त्रिः । अवेति॑ । जि॒घ्रे॒त् । प्र॒जाप॑ता॒विति॑ प्र॒जा - प॒तौ॒ । त्वा॒ । मन॑सि । जु॒हो॒मि॒ । इति॑ । ए॒षा । वै । ता॒नू॒न॒प्त्रस्येति॑ तानू - न॒प्त्रस्य॑ । प्र॒ति॒ष्ठेति॑ प्रति - स्था । यः । ए॒वम् । वेद॑ । प्रतीति॑ । ए॒व । ति॒ष्ठ॒ति॒ । यः ।  \newline




\markright{ TS 3.1.2.3  \hfill https://www.vedavms.in \hfill}
\addcontentsline{toc}{section}{ TS 3.1.2.3 }
\section*{ TS 3.1.2.3 }

\textbf{TS 3.1.2.3 } \newline
\textbf{Samhita Paata} \newline

वा अ॑द्ध्व॒र्योः प्र॑ति॒ष्ठां ॅवेद॒ प्रत्ये॒व ति॑ष्ठति॒ यतो॒ मन्ये॒तान॑भिक्रम्य होष्या॒मीति॒ तत् तिष्ठ॒न्ना श्रा॑वयेदे॒षा वा अ॑द्ध्व॒र्योः प्र॑ति॒ष्ठा य ए॒वं ॅवेद॒ प्रत्ये॒व ति॑ष्ठति॒ यद॑भि॒क्रम्य॑ जुहु॒यात् प्र॑ति॒ष्ठाया॑ इया॒त् तस्मा᳚थ् समा॒नत्र॒ तिष्ठ॑ता होत॒व्यं॑ प्रति॑ष्ठित्यै॒ यो वा अ॑द्ध्व॒र्योः स्वं ॅवेद॒ स्ववा॑ने॒व भ॑वति॒ स्रुग्वा अ॑स्य॒ स्वं ॅवा॑य॒व्य॑मस्य॒ - [  ] \newline

\textbf{Pada Paata} \newline

वै । अ॒द्ध्व॒र्योः । प्र॒ति॒ष्ठामिति॑ प्रति - स्थाम् । वेद॑ । प्रतीति॑ । ए॒व । ति॒ष्ठ॒ति॒ । यतः॑ । मन्ये॑त । अन॑भिक्र॒म्येत्यन॑भि - क्र॒म्य॒ । हो॒ष्या॒मि॒ । इति॑ । तत् । तिष्ठन्न्॑ । एति॑ । श्रा॒व॒ये॒त् । ए॒षा । वै । अ॒द्ध्व॒र्योः । प्र॒ति॒ष्ठेति॑ प्रति - स्था । यः । ए॒वम् । वेद॑ । प्रतीति॑ । ए॒व । ति॒ष्ठ॒ति॒ । यत् । अ॒भि॒क्रम्येत्य॑भि - क्रम्य॑ । जु॒हु॒यात् । प्र॒ति॒ष्ठाया॒ इति॑ प्रति-स्थायाः᳚ । इ॒या॒त् । तस्मा᳚त् । स॒मा॒नत्र॑ । तिष्ठ॑ता । हो॒त॒व्य᳚म् । प्रति॑ष्ठित्या॒ इति॒ प्रति॑ - स्थि॒त्यै॒ । यः । वै । अ॒द्ध्व॒र्योः । स्वम् । वेद॑ । स्ववा॒निति॒ स्व - वा॒न् । ए॒व । भ॒व॒ति॒ । स्रुक् । वै । अ॒स्य॒ । स्वम् । वा॒य॒व्य᳚म् । अ॒स्य॒ ।  \newline




\markright{ TS 3.1.2.4  \hfill https://www.vedavms.in \hfill}
\addcontentsline{toc}{section}{ TS 3.1.2.4 }
\section*{ TS 3.1.2.4 }

\textbf{TS 3.1.2.4 } \newline
\textbf{Samhita Paata} \newline

स्वं च॑म॒सो᳚ऽस्य॒ स्वं ॅयद्वा॑य॒व्यं॑ ॅवा चम॒सं ॅवाऽन॑न्वारभ्याऽऽश्रा॒वये॒थ् स्वादि॑या॒त् तस्मा॑ दन्वा॒रभ्या॒ ऽऽश्राव्यꣳ॒॒ स्वादे॒व नैति॒ यो वै सोम॒म- प्र॑तिष्ठाप्य स्तो॒त्र-मु॑पाक॒रोत्य प्र॑तिष्ठितः॒ सोमो॒ भव॒त्यप्र॑तिष्ठितः॒ स्तोमो- ऽप्र॑तिष्ठिता-न्यु॒क्थान्यप्र॑तिष्ठितो॒ यज॑मा॒नो ऽप्र॑तिष्ठितो ऽध्व॒र्युर्वा॑ य॒व्यं॑ ॅवै सोम॑स्य प्रति॒ष्ठा च॑म॒सो᳚ऽस्य प्रति॒ष्ठा सोमः॒ स्तोम॑स्य॒ स्तोम॑ उ॒क्थानां॒ ग्रहं॑ ॅवा ( ) गृही॒त्वा च॑म॒सं ॅवो॒न्नीय॑ स्तो॒त्रमु॒पा कु॑र्या॒त् प्रत्ये॒व सोमꣳ॑ स्था॒पय॑ति॒ प्रति॒स्तोमं॒ प्रत्यु॒क्थानि॒ प्रति॒ यज॑मान॒स्तिष्ठ॑ति॒ प्रत्य॑द्ध्व॒र्युः ॥ \newline

\textbf{Pada Paata} \newline

स्वम् । च॒म॒सः । अ॒स्य॒ । स्वम् । यत् । वा॒य॒व्य᳚म् । वा॒ । च॒म॒सम् । वा॒ । अन॑न्वार॒भ्येत्यन॑नु - आ॒र॒भ्य॒ । आ॒श्रा॒वये॒दित्या᳚ - श्रा॒वये᳚त् । स्वात् । इ॒या॒त् । तस्मा᳚त् । अ॒न्वा॒रभ्येत्य॑नु - आ॒रभ्य॑ । आ॒श्राव्य॒मित्या᳚ - श्राव्य᳚म् । स्वात् । ए॒व । न । ए॒ति॒ । यः । वै । सोम᳚म् । अप्र॑तिष्ठा॒प्येत्यप्र॑ति - स्था॒प्य॒ । स्तो॒त्रम् । उ॒पा॒क॒रोतीत्यु॑प - आ॒क॒रोति॑ । अप्र॑तिष्ठित॒ इत्यप्र॑ति - स्थि॒तः॒ । सोमः॑ । भव॑ति । अप्र॑तिष्ठित॒ इत्यप्र॑ति - स्थि॒तः॒ । स्तोमः॑ । अप्र॑तिष्ठिता॒नीत्यप्र॑ति - स्थि॒ता॒नि॒ । उ॒क्थानि॑ । अप्र॑तिष्ठित॒ इत्यप्र॑ति - स्थि॒तः॒ । यज॑मानः । अप्र॑तिष्ठित॒ इत्यप्र॑ति - स्थि॒तः॒ । अ॒ध्व॒र्युः । वा॒य॒व्य᳚म् । वै । सोम॑स्य । प्र॒ति॒ष्ठेति॑ प्रति - स्था । च॒म॒सः । अ॒स्य॒ । प्र॒ति॒ष्ठेति॑ प्रति - स्था । सोमः॑ । स्तोम॑स्य । स्तोमः॑ । उ॒क्थाना᳚म् । ग्रह᳚म् । वा॒ ( ) । गृ॒ही॒त्वा । च॒म॒सम् । वा॒ । उ॒न्नीयेत्यु॑त् - नीय॑ । स्तो॒त्रम् । उ॒पाकु॑र्या॒दित्यु॑प - आकु॑र्यात् । प्रतीति॑ । ए॒व । सोम᳚म् । स्था॒पय॑ति । प्रतीति॑ । स्तोम᳚म् । प्रतीति॑ । उ॒क्थानि॑ । प्रतीति॑ । यज॑मानः । तिष्ठ॑ति । प्रतीति॑ । अ॒द्ध्व॒र्युः ॥  \newline




\markright{ TS 3.1.3.1  \hfill https://www.vedavms.in \hfill}
\addcontentsline{toc}{section}{ TS 3.1.3.1 }
\section*{ TS 3.1.3.1 }

\textbf{TS 3.1.3.1 } \newline
\textbf{Samhita Paata} \newline

य॒ज्ञ्ं ॅवा ए॒तथ् सं भ॑रन्ति॒ यथ् सो॑म॒क्रय॑ण्यै प॒दं ॅय॑ज्ञ्मु॒खꣳ ह॑वि॒र्द्धाने॒ यर्.हि॑ हवि॒र्द्धाने॒ प्राची᳚ प्रव॒र्तये॑यु॒स्तर्.हि॒ तेनाक्ष॒मुपा᳚-ञ्ज्याद्-यज्ञ्मु॒ख ए॒व य॒ज्ञ्मनु॒ सन्त॑नोति॒ प्राञ्च॑म॒ग्निं प्र ह॑र॒न्त्युत् पत्नी॒मा न॑य॒न्त्यन्वनाꣳ॑सि॒ प्र व॑र्तय॒न्त्यथ॒ वा अ॑स्यै॒ष धिष्णि॑यो हीयते॒ सोऽनु॑ ध्यायति॒ स ई᳚श्व॒रो रु॒द्रो भू॒त्वा - [  ] \newline

\textbf{Pada Paata} \newline

य॒ज्ञ्म् । वै । ए॒तत् । समिति॑ । भ॒र॒न्ति॒ । यत् । सो॒म॒क्रय॑ण्या॒ इति॑ सोम - क्रय॑ण्यै । प॒दम् । य॒ज्ञ्॒मु॒खमिति॑ यज्ञ् - मु॒खम् । ह॒वि॒द्‌र्धाने॒ इति॑ हविः - धाने᳚ । यर्.हि॑ । ह॒वि॒द्‌र्धाने॒ इति॑ हविः - धाने᳚ । प्राची॒ इति॑ । प्र॒व॒र्तये॑यु॒रिति॑ प्र - व॒र्तये॑युः । तर्.हि॑ । तेन॑ । अक्ष᳚म् । उपेति॑ । अ॒ञ्ज्या॒त् । य॒ज्ञ्॒मु॒ख इति॑ यज्ञ्-मु॒खे । ए॒व । य॒ज्ञ्म् । अनु॑ । समिति॑ । त॒नो॒ति॒ । प्राञ्च᳚म् । अ॒ग्निम् । प्रेति॑ । ह॒र॒न्ति॒ । उदिति॑ । पत्नी᳚म् । एति॑ । न॒य॒न्ति॒ । अन्विति॑ । अनाꣳ॑सि । प्रेति॑ । व॒र्त॒य॒न्ति॒ । अथ॑ । वै । अ॒स्य॒ । ए॒षः । धिष्णि॑यः । ही॒य॒ते॒ । सः । अन्विति॑ । ध्या॒य॒ति॒ । सः । ई॒श्व॒रः । रु॒द्रः । भू॒त्वा ।  \newline




\markright{ TS 3.1.3.2  \hfill https://www.vedavms.in \hfill}
\addcontentsline{toc}{section}{ TS 3.1.3.2 }
\section*{ TS 3.1.3.2 }

\textbf{TS 3.1.3.2 } \newline
\textbf{Samhita Paata} \newline

प्र॒जां प॒शून्. यज॑मानस्य॒ शम॑यितो॒र्यर्.हि॑ प॒शुमा प्री॑त॒मुद॑ञ्चं॒ नय॑न्ति॒ तर्.हि॒ तस्य॑ पशु॒श्रप॑णꣳ हरे॒त् तेनै॒वैनं॑ भा॒गिनं॑ करोति॒ यज॑मानो॒ वा आ॑हव॒नीयो॒ यज॑मानं॒ ॅवा ए॒तद्वि क॑र्.षन्ते॒ यदा॑हव॒नीया᳚त् पशु॒श्रप॑णꣳ॒॒ हर॑न्ति॒ स वै॒व स्यान्नि॑र्म॒न्थ्यं॑ ॅवा कुर्या॒द्-यज॑मानस्य सात्म॒त्वाय॒ यदि॑ प॒शोर॑व॒दानं॒ नश्ये॒दाज्य॑स्य प्रत्या॒ख्याय॒मव॑ द्ये॒थ् सैव ततः॒ ( ) प्राय॑श्चित्ति॒र्ये प॒शुं ॅवि॑मथ्नी॒रन्. यस्तान् का॒मये॒ता ऽऽ*र्ति॒मार्च्छे॑यु॒रिति॑ कु॒विद॒ङ्गेति॒ नमो॑ वृक्तिवत्य॒र्चाऽऽग्नी᳚द्ध्रे जुहुया॒न्नमो॑ वृक्तिमे॒वैषां᳚ ॅवृङ्क्ते ता॒जगार्ति॒मार्च्छ॑न्ति ॥ \newline

\textbf{Pada Paata} \newline

प्र॒जामिति॑ प्र - जाम् । प॒शून् । यज॑मानस्य । शम॑यितोः । यर्.हि॑ । प॒शुम् । आप्री॑त॒मित्या-प्री॒त॒म् । उद॑ञ्चम् । नय॑न्ति । तर्.हि॑ । तस्य॑ । प॒शु॒श्रप॑ण॒मिति॑ पशु - श्रप॑णम् । ह॒रे॒त् । तेन॑ । ए॒व । ए॒न॒म् । भा॒गिन᳚म् । क॒रो॒ति॒ । यज॑मानः । वै । आ॒ह॒व॒नीय॒ इत्या᳚ - ह॒व॒नीयः॑ । यज॑मानम् । वै । ए॒तत् । वीति॑ । क॒र्॒.ष॒न्ते॒ । यत् । आ॒ह॒व॒नीया॒दित्या᳚ - ह॒व॒नीया᳚त् । प॒शु॒श्रप॑ण॒मिति॑ पशु - श्रप॑णम् । हर॑न्ति । सः । वा॒ । ए॒व । स्यात् । नि॒र्म॒न्थ्य॑मिति॑ निः - म॒न्थ्य᳚म् । वा॒ । कु॒र्या॒त् । यज॑मानस्य । सा॒त्म॒त्वायेति॑ सात्म - त्वाय॑ । यदि॑ । प॒शोः । अ॒व॒दान॒मित्य॑व - दान᳚म् । नश्ये᳚त् । आज्य॑स्य । प्र॒त्या॒ख्याय॒मिति॑ प्रति - आ॒ख्याय᳚म् । अवेति॑ । द्ये॒त् । सा । ए॒व । ततः॑ ( ) । प्राय॑श्चित्तिः । ये । प॒शुम् । वि॒म॒थ्नी॒रन्निति॑ वि - म॒थ्नी॒रन्न् । यः । तान् । का॒मये॑त । आर्ति᳚म् । एति॑ । ऋ॒च्छे॒युः॒ । इति॑ । कु॒वित् । अ॒ङ्ग । इति॑ । नमो॑वृक्तिव॒त्येति॒ नमो॑वृक्ति - व॒त्या॒ । ऋ॒चा । आग्नी᳚द्ध्र॒ इत्याग्नि॑ - इ॒ध्रे॒ । जु॒हु॒या॒त् । नमो॑वृक्ति॒मिति॒ नमः॑ - वृ॒क्ति॒म् । ए॒व । ए॒षा॒म् । वृ॒ङ्क्ते॒ । ता॒जक् । आर्ति᳚म् । एति॑ । ऋ॒च्छ॒न्ति॒ ॥  \newline




\markright{ TS 3.1.4.1  \hfill https://www.vedavms.in \hfill}
\addcontentsline{toc}{section}{ TS 3.1.4.1 }
\section*{ TS 3.1.4.1 }

\textbf{TS 3.1.4.1 } \newline
\textbf{Samhita Paata} \newline

प्र॒जाप॑ते॒र्जाय॑मानाः प्र॒जा जा॒ताश्च॒ या इ॒माः । तस्मै॒ प्रति॒ प्र वे॑दयचिकि॒त्वाꣳ अनु॑ मन्यतां ॥इ॒मं प॒शुं प॑शुपते ते अ॒द्य ब॒द्ध्नाम्य॑ग्ने सुकृ॒तस्य॒ मद्ध्ये᳚ । अनु॑ मन्यस्व सु॒यजा॑ यजाम॒ जुष्टं॑ दे॒वाना॑मि॒दम॑स्तु ह॒व्यं ॥ प्र॒जा॒नन्तः॒ प्रति॑गृह्णन्ति॒ पूर्वे᳚ प्रा॒णमङ्गे᳚भ्यः॒ पर्या॒चर॑न्तं ।सुव॒र्गं ॅया॑हि प॒थिभि॑ र्देव॒यानै॒-रोष॑धीषु॒ प्रति॑तिष्ठा॒ शरी॑रैः ॥ येषा॒मीशे॑ - [  ] \newline

\textbf{Pada Paata} \newline

प्र॒जाप॑ते॒रिति॑ प्र॒जा - प॒तेः॒ । जाय॑मानाः । प्र॒जा इति॑ प्र - जाः । जा॒ताः । च॒ । याः । इ॒माः ॥ तस्मै᳚ । प्रति॑ । प्रेति॑ । वे॒द॒य॒ । चि॒कि॒त्वान् । अन्विति॑ । म॒न्य॒ता॒म् ॥ इ॒मम् । प॒शुम् । प॒शु॒प॒त॒ इति॑ पशु - प॒ते॒ । ते॒ । अ॒द्य । ब॒द्ध्नामि॑ । अ॒ग्ने॒ । सु॒कृ॒तस्येति॑ सु - कृ॒तस्य॑ । मद्ध्ये᳚ ॥ अन्विति॑ । म॒न्य॒स्व॒ । सु॒यजेति॑ सु - यजा᳚ । य॒जा॒म॒ । जुष्ट᳚म् । दे॒वाना᳚म् । इ॒दम् । अ॒स्तु॒ । ह॒व्यम् ॥ प्र॒जा॒नन्त॒ इति॑ प्र - जा॒नन्तः॑ । प्रतीति॑ । गृ॒ह्ण॒न्ति॒ । पूर्वे᳚ । प्रा॒णमिति॑ प्र - अ॒नम् । अङ्गे᳚भ्यः । परीति॑ । आ॒चर॑न्त॒मित्या᳚ - चर॑न्तम् ॥ सु॒व॒र्गमिति॑ सुवः - गम् । या॒हि॒ । प॒थिभि॒रिति॑ प॒थि - भिः॒ । दे॒व॒यानै॒रिति॑ देव - यानैः᳚ । ओष॑धीषु । प्रतीति॑ । ति॒ष्ठ॒ । शरी॑रैः ॥ येषा᳚म् । ईशे᳚ ।  \newline




\markright{ TS 3.1.4.2  \hfill https://www.vedavms.in \hfill}
\addcontentsline{toc}{section}{ TS 3.1.4.2 }
\section*{ TS 3.1.4.2 }

\textbf{TS 3.1.4.2 } \newline
\textbf{Samhita Paata} \newline

पशु॒पतिः॑ पशू॒नां चतु॑ष्पदामु॒त च॑ द्वि॒पदां᳚ । निष्क्री॑तो॒ऽयं ॅय॒ज्ञियं॑ भा॒गमे॑तु रा॒यस्पोषा॒ यज॑मानस्य सन्तु ॥ ये ब॒द्ध्यमा॑न॒मनु॑ ब॒द्ध्यमा॑ना अ॒भ्यैक्ष॑न्त॒ मन॑सा॒ चक्षु॑षा च । अ॒ग्निस्ताꣳ अग्रे॒ प्रमु॑मोक्तु दे॒वः प्र॒जाप॑तिः प्र॒जया॑ संॅविदा॒नः ॥ य आ॑र॒ण्याः प॒शवो॑ वि॒श्वरू॑पा॒ विरू॑पाः॒ सन्तो॑ बहु॒धैक॑रूपाः । वा॒युस्ताꣳ अग्रे॒ प्रमु॑मोक्तु दे॒वः प्र॒जाप॑तिः प्र॒जया॑ संॅविदा॒नः ॥ प्र॒मु॒ञ्चमा॑ना॒ - [  ] \newline

\textbf{Pada Paata} \newline

प॒श॒पति॒रिति॑ पशु-पतिः॑ । प॒शू॒नाम् । चतु॑ष्पदा॒मिति॒ चतुः॑ - प॒दा॒म् । उ॒त । च॒ । द्वि॒पदा॒मिति॑ द्वि - पदा᳚म् ॥ निष्क्री॑त॒ इति॒ निः - क्री॒तः॒ । अ॒यम् । य॒ज्ञिय᳚म् । भा॒गम् । ए॒तु॒ । रा॒यः । पोषाः᳚ । यज॑मानस्य । स॒न्तु॒ ॥ ये । ब॒द्ध्यमा॑नम् । अन्विति॑ । ब॒द्ध्यमा॑नाः । अ॒भ्यैक्ष॒न्तेत्य॑भि - ऐक्ष॑न्त । मन॑सा । चक्षु॑षा । च॒ ॥ अ॒ग्निः । तान् । अग्रे᳚ । प्रेति॑ । मु॒मो॒क्तु॒ । दे॒वः । प्र॒जाप॑ति॒रिति॑ प्र॒जा - प॒तिः॒ । प्र॒जयेति॑ प्र - जया᳚ । सं॒ॅवि॒दा॒न इति॑ सं-वि॒दा॒नः ॥ ये । आ॒र॒ण्याः । प॒शवः॑ । वि॒श्वरू॑पा॒ इति॑ वि॒श्व - रू॒पाः॒ । विरू॑पा॒ इति॒ वि-रू॒पाः॒ । सन्तः॑ । ब॒हु॒धेति॑ बहु - धा । एक॑रूपा॒ इत्येक॑ - रू॒पाः॒ ॥ वा॒युः । तान् । अग्रे᳚ । प्रेति॑ । मु॒मो॒क्तु॒ । दे॒वः । प्र॒जाप॑ति॒रिति॑ प्र॒जा - प॒तिः॒ । प्र॒जयेति॑ प्र - जया᳚ । सं॒ॅवि॒दा॒न इति॑ सं - वि॒दा॒नः ॥ प्र॒मु॒ञ्चमा॑ना॒ इति॑ प्र - मु॒ञ्चमा॑नाः ।  \newline




\markright{ TS 3.1.4.3  \hfill https://www.vedavms.in \hfill}
\addcontentsline{toc}{section}{ TS 3.1.4.3 }
\section*{ TS 3.1.4.3 }

\textbf{TS 3.1.4.3 } \newline
\textbf{Samhita Paata} \newline

भुव॑नस्य॒ रेतो॑ गा॒तुं ध॑त्त॒ यज॑मानाय देवाः । उ॒पाकृ॑तꣳ शशमा॒नं ॅयदस्था᳚ज्जी॒वं दे॒वाना॒मप्ये॑तु॒ पाथः॑ ॥ नाना᳚ प्रा॒णो यज॑मानस्य प॒शुना॑ य॒ज्ञो दे॒वेभिः॑ स॒ह दे॑व॒यानः॑ । जी॒वं दे॒वाना॒मप्ये॑तु॒ पाथः॑ स॒त्याः स॑न्तु॒ यज॑मानस्य॒ कामाः᳚ ॥ यत् प॒शुर्मा॒युमकृ॒तोरो॑ वा प॒द्भिरा॑ह॒ते । अ॒ग्निर्मा॒ तस्मा॒देन॑सो॒ विश्वा᳚न् मुञ्च॒त्वꣳह॑सः ॥ शमि॑तार उ॒पेत॑न य॒ज्ञ्ं - [  ] \newline

\textbf{Pada Paata} \newline

भुव॑नस्य । रेतः॑ । गा॒तुम् । ध॒त्त॒ । यज॑मानाय । दे॒वाः॒ ॥ उ॒पाकृ॑त॒मित्यु॑प-आकृ॑तम् । श॒श॒मा॒नम् । यत् । अस्था᳚त् । जी॒वम् । दे॒वाना᳚म् । अपीति॑ । ए॒तु॒ । पाथः॑ ॥ नाना᳚ । प्रा॒ण इति॑ प्र - अ॒नः । यज॑मानस्य । प॒शुना᳚ । य॒ज्ञ्ः । दे॒वेभिः॑ । स॒ह । दे॒व॒यान॒ इति॑ देव - यानः॑ ॥ जी॒वम् । दे॒वाना᳚म् । अपीति॑ । ए॒तु॒ । पाथः॑ । स॒त्याः । स॒न्तु॒ । यज॑मानस्य । कामाः᳚ ॥ यत् । प॒शुः । मा॒युम् । अकृ॑त । उरः॑ । वा॒ । प॒द्भिरिति॑ पत् - भिः । आ॒ह॒त इत्या᳚ - ह॒ते ॥ अ॒ग्निः । मा॒ । तस्मा᳚त् । एन॑सः । विश्वा᳚त् । मु॒ञ्च॒तु॒ । अꣳह॑सः ॥ शमि॑तारः । उ॒पेत॒नेत्यु॑प - एत॑न । य॒ज्ञ्म् ।  \newline




\markright{ TS 3.1.4.4  \hfill https://www.vedavms.in \hfill}
\addcontentsline{toc}{section}{ TS 3.1.4.4 }
\section*{ TS 3.1.4.4 }

\textbf{TS 3.1.4.4 } \newline
\textbf{Samhita Paata} \newline

दे॒वेभि॑रिन्वि॒तं । पाशा᳚त् प॒शुं प्रमु॑ञ्चत ब॒न्धाद्य॒ज्ञ्प॑तिं॒ परि॑ ॥ अदि॑तिः॒ पाशं॒ प्रमु॑मोक्त्वे॒तं नमः॑ प॒शुभ्यः॑ पशु॒पत॑ये करोमि ॥ अ॒रा॒ती॒यन्त॒-मध॑रं कृणोमि॒ यं द्वि॒ष्मस्तस्मि॒न् प्रति॑ मुञ्चामि॒ पाशं᳚ ॥ त्वामु॒ ते द॑धिरे हव्य॒वाहꣳ॑ शृतंक॒र्तार॑मु॒त य॒ज्ञियं॑ च । अग्ने॒ सद॑क्षः॒ सत॑नु॒र्॒.हि भू॒त्वाऽथ॑ ह॒व्या जा॑तवेदो जुषस्व ॥ जात॑वेदो व॒पया॑ गच्छ दे॒वान्त्वꣳ ( ) हि होता᳚ प्रथ॒मो ब॒भूथ॑ । घृ॒तेन॒ त्वं त॒नुवो॑ वर्द्धयस्व॒ स्वाहा॑कृतꣳ ह॒विर॑दन्तु दे॒वाः ॥ स्वाहा॑ दे॒वेभ्यो॑ दे॒वेभ्यः॒ स्वाहा᳚ ॥ \newline

\textbf{Pada Paata} \newline

दे॒वेभिः॑ । इ॒न्वि॒तम् ॥ पाशा᳚त् । प॒शुम् । प्रेति॑ । मु॒ञ्च॒त॒ । ब॒न्धात् । य॒ज्ञ्प॑ति॒मिति॑ य॒ज्ञ् - प॒ति॒म् । परि॑ ॥ अदि॑तिः । पाश᳚म् । प्रेति॑ । मु॒मो॒क्तु॒ । ए॒तम् । नमः॑ । प॒शुभ्य॒ इति॑ प॒शु - भ्यः॒ । प॒शु॒पत॑य॒ इति॑ पशु - पत॑ये । क॒रो॒मि॒ ॥ अ॒रा॒ती॒यन्त᳚म् । अध॑रम् । कृ॒णो॒मि॒ । यम् । द्वि॒ष्मः । तस्मिन्न्॑ । प्रतीति॑ । मु॒ञ्चा॒मि॒ । पाश᳚म् ॥ त्वाम् । उ॒ । ते । द॒धि॒रे॒ । ह॒व्य॒वाह॒मिति॑ हव्य - वाह᳚म् । शृ॒त॒कं॒र्तार॒मिति॑ शृतं - क॒र्तार᳚म् । उ॒त । य॒ज्ञिय᳚म् । च॒ ॥ अग्ने᳚ । सद॑क्ष॒ इति॒ स - द॒क्षः॒ । सत॑नु॒रिति॒ स-त॒नुः॒ । हि । भू॒त्वा । अथ॑ । ह॒व्या । जा॒त॒वे॒द॒ इति॑ जात - वे॒दः॒ । जु॒ष॒स्व॒ ॥ जात॑वेद॒ इति॒ जात॑ - वे॒दः॒ । व॒पया᳚ । ग॒च्छ॒ । दे॒वान् । त्वम् ( ) । हि । होता᳚ । प्र॒थ॒मः । ब॒भूथ॑ ॥ घृ॒तेन॑ । त्वम् । त॒नुवः॑ । व॒द्‌र्ध॒य॒स्व॒ । स्वाहा॑कृत॒मिति॒ स्वाहा᳚ - कृ॒त॒म् । ह॒विः । अ॒द॒न्तु॒ । दे॒वाः ॥ स्वाहा᳚ । दे॒वेभ्यः॑ । दे॒वेभ्यः॑ । स्वाहा᳚ ॥  \newline




\markright{ TS 3.1.5.1  \hfill https://www.vedavms.in \hfill}
\addcontentsline{toc}{section}{ TS 3.1.5.1 }
\section*{ TS 3.1.5.1 }

\textbf{TS 3.1.5.1 } \newline
\textbf{Samhita Paata} \newline

प्रा॒जा॒प॒त्या वै प॒शव॒स्तेषाꣳ॑ रु॒द्रोऽधि॑पति॒र्य-दे॒ताभ्या॑-मुपा क॒रोति॒ ताभ्या॑मे॒वैनं॑ प्रति॒प्रोच्याऽऽल॑भत आ॒त्मनोऽना᳚व्रस्काय॒ द्वाभ्या॑मु॒पाक॑रोति द्वि॒पाद्यज॑मानः॒ प्रति॑ष्ठित्या उपा॒कृत्य॒ पञ्च॑ जुहोति॒ पाङ्क्ताः᳚ प॒शवः॑ प॒शूने॒वा व॑रुन्धेमृ॒त्यवे॒ वा ए॒ष नी॑यते॒ यत् प॒शुस्तं ॅयद॑न्वा॒रभे॑त प्र॒मायु॑को॒ यज॑मानः स्या॒न्नाना᳚ प्रा॒णो यज॑मानस्य प॒शुनेत्या॑ह॒ व्यावृ॑त्त्यै॒ - [  ] \newline

\textbf{Pada Paata} \newline

प्रा॒जा॒प॒त्या इति॑ प्राजा - प॒त्याः । वै । प॒शवः॑ । तेषा᳚म् । रु॒द्रः । अधि॑पति॒रित्यधि॑ - प॒तिः॒ । यत् । ए॒ताभ्या᳚म् । उ॒पा॒क॒रोतीत्यु॑प - आ॒क॒रोति॑ । ताभ्या᳚म् । ए॒व । ए॒न॒म् । प्र॒ति॒प्रोच्येति॑ प्रति - प्रोच्य॑ । एति॑ । ल॒भ॒ते॒ । आ॒त्मनः॑ । अना᳚व्रस्का॒येत्यना᳚ - व्र॒स्का॒य॒ । द्वाभ्या᳚म् । उ॒पाक॑रा॒तीत्यु॑प - आक॑रोति । द्वि॒पादिति॑ द्वि - पात् । यज॑मानः । प्रति॑ष्ठित्या॒ इति॒ प्रति॑ - स्थि॒त्यै॒ । उ॒पा॒कृत्येत्यु॑प - आ॒कृत्य॑ । पञ्च॑ । जु॒हो॒ति॒ । पाङ्क्ताः᳚ । प॒शवः॑ । प॒शून् । ए॒व । अवेति॑ । रु॒न्धे॒ । मृ॒त्यवे᳚ । वै । ए॒षः । नी॒य॒ते॒ । यत् । प॒शुः । तम् । यत् । अ॒न्वा॒रभे॒तेत्य॑नु - आ॒रभे॑त । प्र॒मायु॑क॒ इति॑ प्र - मायु॑कः । यज॑मानः । स्या॒त् । नाना᳚ । प्रा॒ण इति॑ प्र - अ॒नः । यज॑मानस्य । प॒शुना᳚ । इति॑ । आ॒ह॒ । व्यावृ॑त्त्या॒ इति॑ वि - आवृ॑त्त्यै ।  \newline




\markright{ TS 3.1.5.2  \hfill https://www.vedavms.in \hfill}
\addcontentsline{toc}{section}{ TS 3.1.5.2 }
\section*{ TS 3.1.5.2 }

\textbf{TS 3.1.5.2 } \newline
\textbf{Samhita Paata} \newline

यत् प॒शुर्मा॒युमकृ॒तेति॑ जुहोति॒ शान्त्यै॒ शमि॑तार उ॒पेत॒नेत्या॑ह यथाय॒जुरे॒वैतद्व॒पायां॒ ॅवा आ᳚ह्रि॒यमा॑णाया-म॒ग्नेर्मेधोऽप॑ क्रामति॒ त्वामु॒ ते द॑धिरे हव्य॒वाह॒मिति॑ व॒पाम॒भि जु॑होत्य॒ग्नेरे॒व मेध॒मव॑ रु॒न्धेऽथो॑ शृत॒त्वाय॑ पु॒रस्ता᳚थ् स्वाहा कृतयो॒ वा अ॒न्ये दे॒वा उ॒परि॑ष्टाथ् स्वाहाकृतयो॒ऽन्ये स्वाहा॑ दे॒वेभ्यो॑ दे॒वेभ्यः॒ स्वाहेत्य॒भितो॑ व॒पां ( ) जु॑होति॒ ताने॒वोभया᳚न् प्रीणाति ॥ \newline

\textbf{Pada Paata} \newline

यत् । प॒शुः । मा॒युम् । अकृ॑त । इति॑ । जु॒हो॒ति॒ । शान्त्यै᳚ । शमि॑तारः । उ॒पेत॒नेत्यु॑प - एत॑न । इति॑ । आ॒ह॒ । य॒था॒य॒जुरिति॑ यथा - य॒जुः । ए॒व । ए॒तत् । व॒पाया᳚म् । वै । आ॒ह्रि॒यमा॑णाया॒मित्या᳚ - ह्रि॒यमा॑णायाम् । अ॒ग्नेः । मेधः॑ । अपेति॑ । क्रा॒म॒ति॒ । त्वाम् । उ॒ । ते । द॒धि॒रे॒ । ह॒व्य॒वाह॒मिति॑ हव्य - वाह᳚म् । इति॑ । व॒पाम् । अ॒भीति॑ । जु॒हो॒ति॒ । अ॒ग्नेः । ए॒व । मेध᳚म् । अवेति॑ । रु॒न्धे॒ । अथो॒ इति॑ । शृ॒त॒त्वायेति॑ शृत - त्वाय॑ । पु॒रस्ता᳚थ्स्वाहाकृतय॒ इति॑ पु॒रस्ता᳚त् - स्वा॒हा॒कृ॒त॒यः॒ । वै । अ॒न्ये । दे॒वाः । उ॒परि॑ष्टाथ्स्वाहाकृतय॒ इत्यु॒परि॑ष्टात् - स्वा॒हा॒कृ॒त॒यः॒ । अ॒न्ये । स्वाहा᳚ । दे॒वेभ्यः॑ । दे॒वेभ्यः॑ । स्वाहा᳚ । इति॑ । अ॒भितः॑ । व॒पाम् ( ) । जु॒हो॒ति॒ । तान् । ए॒व । उ॒भयान्॑ । प्री॒णा॒ति॒ ॥  \newline




\markright{ TS 3.1.6.1  \hfill https://www.vedavms.in \hfill}
\addcontentsline{toc}{section}{ TS 3.1.6.1 }
\section*{ TS 3.1.6.1 }

\textbf{TS 3.1.6.1 } \newline
\textbf{Samhita Paata} \newline

यो वा अय॑थादेवतं ॅय॒ज्ञ्मु॑प॒चर॒त्या दे॒वता᳚भ्यो वृश्च्यते॒ पापी॑यान् भवति॒ यो य॑थादेव॒तन्न दे॒वता᳚भ्य॒ आ वृ॑श्च्यते॒ वसी॑यान् भवत्याग्ने॒य्यर्चा ऽऽग्नी᳚द्ध्रम॒भि मृ॑शेद् वैष्ण॒व्या ह॑वि॒र्द्धान॑माग्ने॒य्या स्रुचो॑ वाय॒व्य॑या वाय॒व्या᳚न्यैन्द्रि॒या सदो॑ यथादेव॒तमे॒व य॒ज्ञ्मुप॑ चरति॒ न दे॒वता᳚भ्य॒ आ वृ॑श्च्यते॒ वसी॑यान् भवति यु॒नज्मि॑ ते पृथि॒वीं ज्योति॑षा स॒ह यु॒नज्मि॑ वा॒युम॒न्तरि॑क्षेण - [  ] \newline

\textbf{Pada Paata} \newline

यः । वै । अय॑थादेवत॒मित्यय॑था - दे॒व॒त॒म् । य॒ज्ञ्म् । उ॒प॒चर॒तीत्यु॑प - चर॑ति । एति॑ । दे॒वता᳚भ्यः । वृ॒श्च्य॒ते॒ । पापी॑यान् । भ॒व॒ति॒ । यः । य॒था॒दे॒व॒तमिति॑ यथा-दे॒व॒तम् । न । दे॒वता᳚भ्यः । एति॑ । वृ॒श्च्य॒ते॒ । वसी॑यान् । भ॒व॒ति॒ । आ॒ग्ने॒य्या । ऋ॒चा । आग्नी᳚द्ध्र॒मित्याग्नि॑ - इ॒द्ध्र॒म् । अ॒भीति॑ । मृ॒शे॒त् । वै॒ष्ण॒व्या । ह॒वि॒द्‌र्धान॒मिति॑ हविः - धान᳚म् । आ॒ग्ने॒य्या । स्रुचः॑ । वा॒य॒व्य॑या । वा॒य॒व्या॑नि । ऐ॒न्द्रि॒या । सदः॑ । य॒था॒दे॒व॒तमिति॑ यथा-दे॒व॒तम् । ए॒व । य॒ज्ञ्म् । उपेति॑ । च॒र॒ति॒ । न । दे॒वता᳚भ्यः । एति॑ । वृ॒श्च्य॒ते॒ । वसी॑यान् । भ॒व॒ति॒ । यु॒नज्मि॑ । ते॒ । पृ॒थि॒वीम् । ज्योति॑षा । स॒ह । यु॒नज्मि॑ । वा॒युम् । अ॒न्तरि॑क्षेण ।  \newline




\markright{ TS 3.1.6.2  \hfill https://www.vedavms.in \hfill}
\addcontentsline{toc}{section}{ TS 3.1.6.2 }
\section*{ TS 3.1.6.2 }

\textbf{TS 3.1.6.2 } \newline
\textbf{Samhita Paata} \newline

ते स॒ह यु॒नज्मि॒ वाचꣳ॑ स॒ह सूर्ये॑ण ते यु॒नज्मि॑ ति॒स्रो वि॒पृचः॒ सूर्य॑स्य ते । अ॒ग्निर्दे॒वता॑ गाय॒त्री छन्द॑ उपाꣳ॒॒शोः पात्र॑मसि॒ सोमो॑ दे॒वता᳚ त्रि॒ष्टुप् छन्दो᳚ऽन्तर्या॒मस्य॒ पात्र॑म॒सीन्द्रो॑ दे॒वता॒ जग॑ती॒ छन्द॑ इन्द्रवायु॒वोः पात्र॑मसि॒ बृह॒स्पति॑र् दे॒वता॑ऽनु॒ष्टुप् छन्दो॑ मि॒त्रावरु॑णयोः॒ पात्र॑मस्य॒श्विनौ॑ दे॒वता॑ प॒ङ्क्तिश्छन्दो॒ऽश्विनोः॒ पात्र॑मसि॒ सूर्यो॑ दे॒वता॑ बृह॒ती - [  ] \newline

\textbf{Pada Paata} \newline

ते॒ । स॒ह । यु॒नज्मि॑ । वाच᳚म् । स॒ह । सूर्ये॑ण । ते॒ । यु॒नज्मि॑ । ति॒स्रः । वि॒पृच॒ इति॑ वि - पृचः॑ । सूर्य॑स्य । ते॒ ॥ अ॒ग्निः । दे॒वता᳚ । गा॒य॒त्री । छन्दः॑ । उ॒पाꣳ॒॒शोरित्यु॑प - अꣳ॒॒शोः । पात्र᳚म् । अ॒सि॒ । सोमः॑ । दे॒वता᳚ । त्रि॒ष्टुप् । छन्दः॑ । अ॒न्त॒र्या॒मस्येत्य॑न्तः - या॒मस्य॑ । पात्र᳚म् । अ॒सि॒ । इन्द्रः॑ । दे॒वता᳚ । जग॑ती । छन्दः॑ । इ॒न्द्र॒वा॒यु॒वोरिती᳚न्द्र-वा॒यु॒वोः । पात्र᳚म् । अ॒सि॒ । बृह॒स्पतिः॑ । दे॒वता᳚ । अ॒नु॒ष्टुबित्य॑नु - स्तुप् । छन्दः॑ । मि॒त्रावरु॑णयो॒रिति॑ मि॒त्रा-वरु॑णयोः । पात्र᳚म् । अ॒सि॒ । अ॒श्विनौ᳚ । दे॒वता᳚ । प॒ङ्क्तिः । छन्दः॑ । अ॒श्विनोः᳚ । पात्र᳚म् । अ॒सि॒ । सूर्यः॑ । दे॒वता᳚ । बृ॒ह॒ती ।  \newline




\markright{ TS 3.1.6.3  \hfill https://www.vedavms.in \hfill}
\addcontentsline{toc}{section}{ TS 3.1.6.3 }
\section*{ TS 3.1.6.3 }

\textbf{TS 3.1.6.3 } \newline
\textbf{Samhita Paata} \newline

छन्दः॑ शु॒क्रस्य॒ पात्र॑मसि च॒न्द्रमा॑ दे॒वता॑ स॒तो बृ॑हती॒ छन्दो॑ म॒न्थिनः॒ पात्र॑मसि॒ विश्वे॑दे॒वा दे॒वतो॒ष्णिहा॒ छन्द॑ आग्रय॒णस्य॒ पात्र॑म॒सीन्द्रो॑ दे॒वता॑ क॒कुच्छन्द॑ उ॒क्थानां॒ पात्र॑मसि पृथि॒वी दे॒वता॑ वि॒राट् छन्दो᳚ ध्रु॒वस्य॒ पात्र॑मसि ॥ \newline

\textbf{Pada Paata} \newline

छन्दः॑ । शु॒क्रस्य॑ । पात्र᳚म् । अ॒सि॒ । च॒न्द्रमाः᳚ । दे॒वता᳚ । स॒तोबृ॑ह॒तीति॑ स॒तः - बृ॒ह॒ती॒ । छन्दः॑ । म॒न्थिनः॑ । पात्र᳚म् । अ॒सि॒ । विश्वे᳚ । दे॒वाः । दे॒वता᳚ । उ॒ष्णिहा᳚ । छन्दः॑ । आ॒ग्र॒य॒णस्य॑ । पात्र᳚म् । अ॒सि॒ । इन्द्रः॑ । दे॒वता᳚ । क॒कुत् । छन्दः॑ । उ॒क्थाना᳚म् । पात्र᳚म् । अ॒सि॒ । पृ॒थि॒वी । दे॒वता᳚ । वि॒राडिति॑ वि - राट् । छन्दः॑ । ध्रु॒वस्य॑ । पात्र᳚म् । अ॒सि॒ ॥  \newline




\markright{ TS 3.1.7.1  \hfill https://www.vedavms.in \hfill}
\addcontentsline{toc}{section}{ TS 3.1.7.1 }
\section*{ TS 3.1.7.1 }

\textbf{TS 3.1.7.1 } \newline
\textbf{Samhita Paata} \newline

इ॒ष्टर्गो॒ वा अ॑द्ध्व॒र्युर्यज॑मानस्ये॒ष्टर्गः॒ खलु॒ वै पूर्वो॒ऽर्ष्टुः क्षी॑यत आस॒न्या᳚न्मा॒ मन्त्रा᳚त् पाहि॒ कस्या᳚श्चिद॒भिश॑स्त्या॒ इति॑ पु॒रा प्रा॑तरनुवा॒काज्जु॑हुयादा॒त्मन॑ ए॒व तद॑द्ध्व॒र्युः पु॒रस्ता॒च्छर्म॑ नह्य॒तेऽना᳚र्त्यै संॅवे॒शाय॑ त्वोपवे॒शाय॑ त्वा गायत्रि॒या स्त्रि॒ष्टुभो॒ जग॑त्या अ॒भिभू᳚त्यै॒ स्वाहा॒ प्राणा॑पानौ मृ॒त्योर्मा॑ पातं॒ प्राणा॑पानौ॒ मा मा॑ हासिष्टं दे॒वता॑सु॒ वा ए॒ते प्रा॑णापा॒नयो॒र् - [  ] \newline

\textbf{Pada Paata} \newline

इ॒ष्टर्गः॑ । वै । अ॒द्ध्व॒र्युः । यज॑मानस्य । इ॒ष्टर्गः॑ । खलु॑ । वै । पूर्वः॑ । अ॒र्ष्टुः । क्षी॒य॒ते॒ । आ॒स॒न्या᳚त् । मा॒ । मन्त्रा᳚त् । पा॒हि॒ । कस्याः᳚ । चि॒त् । अ॒भिश॑स्त्या॒ इत्य॒भि - श॒स्त्याः॒ । इति॑ । पु॒रा । प्रा॒त॒र॒नु॒वा॒कादिति॑ प्रातः - अ॒नु॒वा॒कात् । जु॒हु॒या॒त् । आ॒त्मने᳚ । ए॒व । तत् । अ॒द्ध्व॒र्युः । पु॒रस्ता᳚त् । शर्म॑ । न॒ह्य॒ते॒ । अना᳚र्त्यै । सं॒ॅवे॒शायेति॑ सं - वे॒शाय॑ । त्वा॒ । उ॒प॒वे॒शायेतु॑प - वे॒शाय॑ । त्वा॒ । गा॒य॒त्रि॒याः । त्रि॒ष्टुभः॑ । जग॑त्याः । अ॒भिभू᳚त्या॒ इत्य॒भि - भू॒त्यै॒ । स्वाहा᳚ । प्राणा॑पाना॒विति॒ प्राण॑ - अ॒पा॒नौ॒ । मृ॒त्योः । मा॒ । पा॒त॒म् । प्राणा॑पाना॒विति॒ प्राण॑ - अ॒पा॒नौ॒ । मा । मा॒ । हा॒सि॒ष्ट॒म् । दे॒वता॑सु । वै । ए॒ते । प्रा॒णा॒पा॒नयो॒रिति॑ प्राण - अ॒पा॒नयोः᳚ ।  \newline




\markright{ TS 3.1.7.2  \hfill https://www.vedavms.in \hfill}
\addcontentsline{toc}{section}{ TS 3.1.7.2 }
\section*{ TS 3.1.7.2 }

\textbf{TS 3.1.7.2 } \newline
\textbf{Samhita Paata} \newline

व्याय॑च्छन्ते॒ येषाꣳ॒॒ सोमः॑ समृ॒च्छते॑ संॅवे॒शाय॑ त्वोपवे॒शाय॒ त्वेत्या॑ह॒ छन्दाꣳ॑सि॒ वै सं॑ॅवे॒श उ॑पवे॒शश्छन्दो॑भिरे॒वास्य॒ छन्दाꣳ॑सि वृङ्क्ते॒ प्रेति॑व॒न्त्याज्या॑नि भवन्त्य॒भिजि॑त्यै म॒रुत्व॑तीः प्रति॒पदो॒ विजि॑त्या उ॒भे बृ॑हद्रथन्त॒रे भ॑वत इ॒यं ॅवाव र॑थन्त॒रम॒सौ बृ॒हदा॒भ्यामे॒वैन॑म॒न्तरे᳚त्य॒द्य वाव र॑थन्त॒रꣳ श्वो बृ॒हद॑द्या॒श्वा दे॒वैन॑म॒न्तरे॑ति भू॒तं - [  ] \newline

\textbf{Pada Paata} \newline

व्याय॑च्छन्त॒ इति॑ वि - आय॑च्छन्ते । येषा᳚म् । सोमः॑ । स॒मृ॒च्छत॒ इति॑ सं - ऋ॒च्छते᳚ । सं॒ॅवे॒शायेति॑ सं - वे॒शाय॑ । त्वा॒ । उ॒प॒वे॒शायेत्यु॑प - वे॒शाय॑ । त्वा॒ । इति॑ । आ॒ह॒ । छन्दाꣳ॑सि । वै । सं॒ॅवे॒श इति॑ सं - वे॒शः । उ॒प॒वे॒श इत्यु॑प - वे॒शः । छन्दो॑भि॒रिति॒ छन्दः॑ - भिः॒ । ए॒व । अ॒स्य॒ । छन्दाꣳ॑सि । वृ॒ङ्क्ते॒ । प्रेति॑व॒न्तीति॒ प्रेति॑ - व॒न्ति॒ । आज्या॑नि । भ॒व॒न्ति॒ । अ॒भिजि॑त्या॒ इत्य॒भि - जि॒त्यै॒ । म॒रुत्व॑तीः । प्र॒ति॒पद॒ इति॑ प्रति - पदः॑ । विजि॑त्या॒ इति॒ वि - जि॒त्यै॒ । उ॒भे इति॑ । बृ॒ह॒द्र॒थ॒न्त॒रे इति॑ बृहत् - र॒थ॒न्त॒रे । भ॒व॒तः॒ । इ॒यम् । वाव । र॒थ॒न्त॒रमिति॑ रथम्-त॒रम् । अ॒सौ । बृ॒हत् । आ॒भ्याम् । ए॒व । ए॒न॒म् । अ॒न्तः । ए॒ति॒ । अ॒द्य । वाव । र॒थ॒न्त॒रमिति॑ रथम् - त॒रम् । श्वः । बृ॒हत् । अ॒द्या॒श्वादित्य॑द्य - श्वात् । ए॒व । ए॒न॒म् । अ॒न्तः । ए॒ति॒ । भू॒तम् ।  \newline




\markright{ TS 3.1.7.3  \hfill https://www.vedavms.in \hfill}
\addcontentsline{toc}{section}{ TS 3.1.7.3 }
\section*{ TS 3.1.7.3 }

\textbf{TS 3.1.7.3 } \newline
\textbf{Samhita Paata} \newline

ॅवाव र॑थन्त॒रं भ॑वि॒ष्यद्-बृ॒॒हद् भू॒ताच्चै॒वैनं॑ भविष्य॒तश्चा॒न्तरे॑ति॒, परि॑मितं॒ ॅवाव र॑थन्त॒रमप॑रिमितं बृ॒हत् परि॑मिताच्चै॒वैन॒-मप॑रिमिताच्चा॒ऽन्तरे॑ति विश्वामित्रजमद॒ग्नी वसि॑ष्ठेनास्पर्द्धेताꣳ॒॒स ए॒तज्ज॒मद॑ग्नि र्विह॒व्य॑म पश्य॒त् तेन॒ वै स वसि॑ष्ठस्येन्द्रि॒यं ॅवी॒र्य॑मवृङ्क्त॒ यद्वि॑ह॒व्यꣳ॑ श॒स्यत॑ इन्द्रि॒यमे॒व तद्वी॒र्यं॑ ॅयज॑मानो॒ भ्रातृ॑व्यस्य वृङ्क्ते॒ ( ) यस्य॒ भूयाꣳ॑सो यज्ञ्क्र॒तव॒ इत्या॑हुः॒ स दे॒वता॑ वृङ्क्त॒ इति॒ यद्य॑ग्निष्टो॒मः सोमः॑ प॒रस्ता॒थ् स्या-दु॒क्थ्यं॑ कुर्वीत॒ यद्यु॒क्थ्यः॑ स्याद॑तिरा॒त्रं कु॑र्वीत यज्ञ्क्र॒तुभि॑रे॒वास्य॑ दे॒वता॑ वृङ्क्ते॒ वसी॑यान् भवति ॥ \newline

\textbf{Pada Paata} \newline

वाव । र॒थ॒न्त॒रमिति॑ रथम् - त॒रम् । भ॒वि॒ष्यत् । बृ॒हत् । भू॒तात् । च॒ । ए॒व । ए॒न॒म् । भ॒वि॒ष्य॒तः । च॒ । अ॒न्तः । ए॒ति॒ । परि॑मित॒मिति॒ परि॑ - मि॒त॒म् । वाव । र॒थ॒न्त॒रमिति॑ रथम् - त॒रम् । अप॑रिमित॒मित्यप॑रि - मि॒त॒म् । बृ॒हत् । परि॑मिता॒दिति॒ परि॑ - मि॒ता॒त् । च॒ । ए॒व । ए॒न॒म् । अप॑रिमिता॒दित्यप॑रि - मि॒ता॒त् । च॒ । अ॒न्तः । ए॒ति॒ । वि॒श्वा॒मि॒त्र॒ज॒म॒द॒ग्नी इति॑ विश्वामित्र - ज॒म॒द॒ग्नी । वसि॑ष्ठेन । अ॒स्प॒द्‌र्धे॒ता॒म् । सः । ए॒तत् । ज॒मद॑ग्निः । वि॒ह॒व्य॑मिति॑ वि - ह॒व्य᳚म् । अ॒प॒श्य॒त् । तेन॑ । वै । सः । वसि॑ष्ठस्य । इ॒न्द्रि॒यम् । वी॒र्य᳚म् । अ॒वृ॒ङ्क्त॒ । यत् । वि॒ह॒व्य॑मिति॑ वि - ह॒व्य᳚म् । श॒स्यते᳚ । इ॒न्द्रि॒यम् । ए॒व । तत् । वी॒र्य᳚म् । यज॑मानः । भ्रातृ॑व्यस्य । वृ॒ङ्क्ते॒ ( ) । यस्य॑ । भूयाꣳ॑सः । य॒ज्ञ्॒क्र॒तव॒ इति॑ यज्ञ् - क्र॒तवः॑ । इति॑ । आ॒हुः॒ । सः । दे॒वताः᳚ । वृ॒ङ्क्ते॒ । इति॑ । यदि॑ । अ॒ग्नि॒ष्टो॒म इत्य॑ग्नि-स्तो॒मः । सोमः॑ । प॒रस्ता᳚त् । स्यात् । उ॒क्थ्य᳚म् । कु॒र्वी॒त॒ । यदि॑ । उ॒क्थ्यः॑ । स्यात् । अ॒ति॒रा॒त्रमित्य॑ति - रा॒त्रम् । कु॒र्वी॒त॒ । य॒ज्ञ्॒क्र॒तुभि॒रिति॑ यज्ञ्क्र॒तु - भिः॒ । ए॒व । अ॒स्य॒ । दे॒वताः᳚ । वृ॒ङ्क्ते॒ । वसी॑यान् । भ॒व॒ति॒ ॥  \newline




\markright{ TS 3.1.8.1  \hfill https://www.vedavms.in \hfill}
\addcontentsline{toc}{section}{ TS 3.1.8.1 }
\section*{ TS 3.1.8.1 }

\textbf{TS 3.1.8.1 } \newline
\textbf{Samhita Paata} \newline

नि॒ग्रा॒भ्याः᳚ स्थ देव॒श्रुत॒ आयु॑र्मे तर्पयत प्रा॒णं मे॑ तर्पयतापा॒नं मे॑ तर्पयत व्या॒नं मे॑ तर्पयत॒ चक्षु॑र्मे तर्पयत॒ श्रोत्रं॑ मे तर्पयत॒ मनो॑मे तर्पयत॒ वाचं॑ मे तर्पयता॒ऽऽ*त्मानं॑ मे तर्पय॒ताङ्गा॑नि मे तर्पयत प्र॒जां मे॑ तर्पयत प॒शून् मे॑ तर्पयत गृ॒हान् मे॑ तर्पयत ग॒णान् मे॑ तर्पयत स॒र्वग॑णं मा तर्पयत त॒र्पय॑त मा - [  ] \newline

\textbf{Pada Paata} \newline

नि॒ग्रा॒भ्या॑ इति॑ नि - ग्रा॒भ्याः᳚ । स्थ॒ । दे॒व॒श्रुत॒ इति॑ देव - श्रुतः॑ । आयुः॑ । मे॒ । त॒र्प॒य॒त॒ । प्रा॒णमिति॑ प्र- अ॒नम् । म॒ । त॒र्प॒य॒त॒ । अ॒पा॒नमित्य॑प - अ॒नम् । मे॒ । त॒र्प॒य॒त॒ । व्या॒नमिति॑ वि - अ॒नम् । मे॒ । त॒र्प॒य॒त॒ । चक्षुः॑ । मे॒ । त॒र्प॒य॒त॒ । श्रोत्र᳚म् । मे॒ । त॒र्प॒य॒त॒ । मनः॑ । मे॒ । त॒र्प॒य॒त॒। वाच᳚म् । मे॒ । त॒र्प॒य॒त॒ । आ॒त्मान᳚म् । मे॒ । त॒र्प॒य॒त॒ । अङ्गा॑नि । म॒ । त॒र्प॒य॒त॒ । प्र॒जामिति॑ प्र - जाम् । मे॒ । त॒र्प॒य॒त॒ । प॒शून् । मे॒ । त॒र्प॒य॒त॒ । गृ॒हान् । मे॒ । त॒र्प॒य॒त॒ । ग॒णान् । मे॒ । त॒र्प॒य॒त॒ । स॒र्वग॑ण॒मिति॑ स॒र्व - ग॒ण॒म् । मा॒ । त॒र्प॒य॒त॒ । त॒र्पय॑त । मा॒ ।  \newline




\markright{ TS 3.1.8.2  \hfill https://www.vedavms.in \hfill}
\addcontentsline{toc}{section}{ TS 3.1.8.2 }
\section*{ TS 3.1.8.2 }

\textbf{TS 3.1.8.2 } \newline
\textbf{Samhita Paata} \newline

ग॒णा मे॒ मा वि तृ॑ष॒न्नोष॑धयो॒ वै सोम॑स्य॒ विशो॒ विशः॒ खलु॒ वै राज्ञ्ः॒ प्रदा॑तोरीश्व॒रा ऐ॒न्द्रः सोमोऽवी॑वृधं ॅवो॒ मन॑सा सुजाता॒ ऋत॑प्रजाता॒ भग॒ इद्वः॑ स्याम । इन्द्रे॑ण दे॒वीर्वी॒रुधः॑ संॅविदा॒ना अनु॑ मन्यन्ताꣳ॒॒ सव॑नाय॒ सोम॒मित्या॒हौष॑धीभ्य ए॒वैनꣳ॒॒ स्वायै॑ वि॒शः स्वायै॑ दे॒वता॑यै नि॒र्याच्या॒भि षु॑णोति॒ यो वै सोम॑स्याभिषू॒यमा॑णस्य - [  ] \newline

\textbf{Pada Paata} \newline

ग॒णाः । मे॒ । मा । वीति॑ । तृ॒ष॒न्न् । ओष॑धयः । वै । सोम॑स्य । विशः॑ । विशः॑ । खलु॑ । वै । राज्ञ्ः॑ । प्रदा॑तो॒रिति॒ प्र - दा॒तोः॒ । ई॒श्व॒राः । ऐ॒न्द्रः । सोमः॑ । अवी॑वृधम् । वः॒ । मन॑सा । सु॒जा॒ता॒ इति॑ सु - जा॒ताः॒ । ऋत॑प्रजाता॒ इत्यृत॑ - प्र॒जा॒ताः॒ । भगे᳚ । इत् । वः॒ । स्या॒म॒ ॥ इन्द्रे॑ण । दे॒वीः । वी॒रुधः॑ । सं॒ॅवि॒दा॒ना इति॑ सं - वि॒दा॒नाः । अन्विति॑ । म॒न्य॒न्ता॒म् । सव॑नाय । सोम᳚म् । इति॑ । आ॒ह॒ । ओष॑धीभ्य॒ इत्योष॑धि - भ्यः॒ । ए॒व । ए॒न॒म् । स्वायै᳚ । वि॒शः । स्वायै᳚ । दे॒वता॑यै । नि॒र्याच्येति॑ निः - याच्य॑ । अ॒भीति॑ । सु॒नो॒ति॒ । यः । वै । सोम॑स्य । अ॒भि॒षू॒यमा॑ण॒स्येत्य॑भि - सू॒यमा॑णस्य ।  \newline




\markright{ TS 3.1.8.3  \hfill https://www.vedavms.in \hfill}
\addcontentsline{toc}{section}{ TS 3.1.8.3 }
\section*{ TS 3.1.8.3 }

\textbf{TS 3.1.8.3 } \newline
\textbf{Samhita Paata} \newline

प्रथ॒मोऽꣳ॑शुः स्कन्द॑ति॒ स ई᳚श्व॒र इ॑न्द्रि॒यं ॅवी॒र्यं॑ प्र॒जां प॒शून्. यज॑मानस्य॒ निर्.ह॑न्तो॒स्तम॒भि म॑न्त्रये॒ताऽऽ मा᳚ऽस्कान्थ्स॒ह प्र॒जया॑ स॒ह रा॒यस्पोषे॑णेन्द्रि॒यं मे॑ वी॒र्यं॑ मा निव॑र्द्धी॒रित्या॒शिष॑मे॒वैतामा शा᳚स्त इन्द्रि॒यस्य॑ वी॒य॑र्.स्य प्र॒जायै॑ पशू॒नामनि॑र्घाताय द्र॒फ्सश्च॑स्कन्द पृथि॒वीमनु॒ द्यामि॒मञ्च॒ योनि॒मनु॒ यश्च॒ ( ) पूर्वः॑ । तृ॒तीयं॒ ॅयोनि॒मनु॑ स॒ञ्चर॑न्तं द्र॒फ्सं जु॑हो॒म्यनु॑ स॒प्त होत्राः᳚ ॥ \newline

\textbf{Pada Paata} \newline

प्र॒थ॒मः । अꣳ॒॒शुः । स्कन्द॑ति । सः । ई॒श्व॒रः । इ॒न्द्रि॒यम् । वी॒र्य᳚म् । प्र॒जामिति॑ प्र - जाम् । प॒शून् । यज॑मानस्य । निर्.ह॑न्तो॒रिति॒ निः - ह॒न्तोः॒ । तम् । अ॒भीति॑ । म॒न्त्र॒ये॒त॒ । एति॑ । मा॒ । अ॒स्का॒न् । स॒ह । प्र॒जयेति॑ प्र - जया᳚ । स॒ह । रा॒यः । पोषे॑ण । इ॒न्द्रि॒यम् । मे॒ । वी॒र्य᳚म् । मा । निरिति॑ । व॒धीः॒ । इति॑ । आ॒शिष॒मित्या᳚-शिष᳚म् । ए॒व । ए॒ताम् । एति॑ । शा॒स्ते॒ । इ॒न्द्रि॒यस्य॑ । वी॒र्य॑स्य । प्र॒जाया॒ इति॑ प्र - जायै᳚ । प॒शू॒नाम् । अनि॑र्घाता॒येत्यनिः॑ - घा॒ता॒य॒ । द्र॒फ्सः । च॒स्क॒न्द॒ । पृ॒थि॒वीम् । अन्विति॑ । द्याम् । इ॒मम् । च॒ । योनि᳚म् । अन्विति॑ । यः । च॒ ( ) । पूर्वः॑ । तृ॒तीय᳚म् । योनि᳚म् । अन्विति॑ । स॒ञ्चर॑न्त॒मिति॑ सं - चर॑न्तम् । द्र॒फ्सम् । जु॒हो॒मि॒ । अन्विति॑ । स॒प्त । होत्राः᳚ ॥  \newline




\markright{ TS 3.1.9.1  \hfill https://www.vedavms.in \hfill}
\addcontentsline{toc}{section}{ TS 3.1.9.1 }
\section*{ TS 3.1.9.1 }

\textbf{TS 3.1.9.1 } \newline
\textbf{Samhita Paata} \newline

यो वै दे॒वान् दे॑वयश॒सेना॒र्पय॑ति मनु॒ष्या᳚न् मनुष्ययश॒सेन॑ देवयश॒स्ये॑व दे॒वेषु॒ भव॑ति मनुष्ययश॒सी म॑नु॒ष्ये॑षु॒ यान् प्रा॒चीन॑-माग्रय॒णाद् ग्रहा᳚न् गृह्णी॒यात् तानु॑पाꣳ॒॒शु गृ॑ह्णीया॒द्यानू॒र्द्ध्वाꣳस्तानु॑पब्दि॒मतो॑ दे॒वाने॒व तद्दे॑वयश॒सेना᳚र्पयति मनु॒ष्या᳚न् मनुष्ययश॒सेन॑ देवयश॒स्ये॑व दे॒वेषु॑ भवति मनुष्ययश॒सी म॑नु॒ष्ये᳚ष्व॒ग्निः प्रा॑तस्सव॒ने पा᳚त्व॒स्मान्. वै᳚श्वान॒रो म॑हि॒ना वि॒श्वश॑म्भूः । स नः॑ पाव॒को द्रवि॑णं दधा॒त्वा - [  ] \newline

\textbf{Pada Paata} \newline

यः । वै । दे॒वान् । दे॒व॒य॒श॒सेनेति॑ देव - य॒श॒सेन॑ । अ॒र्पय॑ति । म॒नु॒ष्यान्॑ । म॒नु॒ष्य॒य॒श॒सेनेति॑ मनुष्य - य॒श॒सेन॑ । दे॒व॒य॒श॒सीति॑ देव - य॒श॒सी । ए॒व । दे॒वेषु॑ । भव॑ति । म॒नु॒ष्य॒य॒श॒सीति॑ मनुष्य-य॒श॒सी । म॒नु॒ष्ये॑षु । यान् । प्रा॒चीन᳚म् । आ॒ग्र॒य॒णात् । ग्रहान्॑ । गृ॒ह्णी॒यात् । तान् । उ॒पाꣳ॒॒श्वित्यु॑प - अꣳ॒॒शु । गृ॒ह्णी॒या॒त् । यान् । ऊ॒द्‌र्ध्वान् । तान् । उ॒प॒ब्दि॒मत॒ इत्यु॑पब्दि - मतः॑ । दे॒वान् । ए॒व । तत् । दे॒व॒य॒श॒सेनेति॑ देव - य॒श॒सेन॑ । अ॒र्प॒य॒ति॒ । म॒नु॒ष्यान्॑ । म॒नु॒ष्य॒य॒श॒सेनेति॑ मनुष्य - य॒श॒सेन॑ । दे॒व॒य॒श॒सीति॑ देव - य॒श॒सी । ए॒व । दे॒वेषु॑ । भ॒व॒ति॒ । म॒नु॒ष्य॒य॒श॒सीति॑ मनुष्य - य॒श॒सी । म॒नु॒ष्ये॑षु । अ॒ग्निः । प्रा॒त॒स्स॒व॒न इति॑ प्रातः - स॒व॒ने । पा॒तु॒ ।अ॒स्मान् । वै॒श्वा॒न॒रः । म॒हि॒ना । वि॒श्वश॑म्भू॒रिति॑ वि॒श्व - श॒म्भूः॒ ॥ सः । नः॒ । पा॒व॒कः । द्रवि॑णम् । द॒धा॒तु॒ ।  \newline




\markright{ TS 3.1.9.2  \hfill https://www.vedavms.in \hfill}
\addcontentsline{toc}{section}{ TS 3.1.9.2 }
\section*{ TS 3.1.9.2 }

\textbf{TS 3.1.9.2 } \newline
\textbf{Samhita Paata} \newline

यु॑ष्मन्तः स॒हभ॑क्षाः स्याम ॥ विश्वे॑ दे॒वा म॒रुत॒ इन्द्रो॑ अ॒स्मान॒स्मिन् द्वि॒तीये॒ सव॑ने॒ न ज॑ह्युः । आयु॑ष्मन्तः प्रि॒यमे॑षां॒ ॅवद॑न्तो व॒यं दे॒वानाꣳ॑ सुम॒तौ स्या॑म ॥ इ॒दं तृ॒तीयꣳ॒॒ सव॑नं कवी॒नामृ॒तेन॒ ये च॑म॒समैर॑यन्त । ते सौ॑धन्व॒नाः सुव॑रानशा॒नाः स्वि॑ष्टिं नो अ॒भि वसी॑यो नयन्तु ॥ आ॒यत॑नवती॒र्वा अ॒न्या आहु॑तयो हू॒यन्ते॑ऽनायत॒ना अ॒न्या या आ॑घा॒रव॑ती॒स्ता आ॒यतन॑वती॒र्याः - [  ] \newline

\textbf{Pada Paata} \newline

आयु॑ष्मन्तः । स॒हभ॑क्षा॒ इति॑ स॒ह-भ॒क्षाः॒ । स्या॒म॒ ॥ विश्वे᳚ । दे॒वाः । म॒रुतः॑ । इन्द्रः॑ । अ॒स्मान् । अ॒स्मिन्न् । द्वि॒तीये᳚ । सव॑ने । न । ज॒ह्युः॒ ॥ आयु॑ष्मन्तः । प्रि॒यम् । ए॒षा॒म् । वद॑न्तः । व॒यम् । दे॒वाना᳚म् । सु॒म॒ताविति॑ सु - म॒तौ । स्या॒म॒ ॥ इ॒दम् । तृ॒तीय᳚म् । सव॑नम् । क॒वी॒नाम् । ऋ॒तेन॑ । ये । च॒म॒सम् । ऐर॑यन्त ॥ ते । सौ॒ध॒न्व॒नाः । सुवः॑ । आ॒न॒शा॒नाः । स्वि॑ष्टि॒मिति॒ सु - इ॒ष्टि॒म् । नः॒ । अ॒भीति॑ । वसी॑यः । न॒य॒न्तु॒ ॥ आ॒यत॑नवती॒रित्या॒यत॑न - व॒तीः॒ । वै । अ॒न्याः । आहु॑तय॒ इत्या - हु॒त॒यः॒ । हू॒यन्ते᳚ । अ॒ना॒य॒त॒ना इत्य॑ना - य॒त॒नाः । अ॒न्याः । याः । आ॒घा॒रव॑ती॒रित्या॑घा॒र - व॒तीः॒ । ताः । आ॒यत॑नवती॒रित्या॒यत॑न - व॒तीः॒ । याः ।  \newline




\markright{ TS 3.1.9.3  \hfill https://www.vedavms.in \hfill}
\addcontentsline{toc}{section}{ TS 3.1.9.3 }
\section*{ TS 3.1.9.3 }

\textbf{TS 3.1.9.3 } \newline
\textbf{Samhita Paata} \newline

सौ॒म्यास्ता अ॑नायत॒ना ऐ᳚न्द्रवाय॒व-मा॒दाया॑ऽऽ*घा॒रमा घा॑रयेदद्ध्व॒रो य॒ज्ञो॑ऽयम॑स्तु देवा॒ ओष॑धीभ्यः प॒शवे॑ नो॒ जना॑य॒ विश्व॑स्मै भू॒ताया᳚ऽद्ध्व॒रो॑ऽसि॒ स पि॑न्वस्व घृ॒तव॑द्देव सो॒मेति॑ सौ॒म्या ए॒व तदाहु॑तीरा॒यत॑नवतीः करोत्या॒यत॑नवान् भवति॒ य ए॒वं ॅवेदाथो॒ द्यावा॑पृथि॒वी ए॒व घृ॒तेन॒ व्यु॑नत्ति॒ ते व्यु॑त्ते उपजीव॒नीये॑ भवत उपजीव॒नीयो॑ भवति॒ - [  ] \newline

\textbf{Pada Paata} \newline

सौ॒म्याः । ताः । अ॒ना॒य॒त॒ना इत्य॑ना - य॒त॒नाः । ऐ॒न्द्र॒वा॒य॒वमित्यै᳚न्द्र - वा॒य॒वम् । आ॒दायेत्या᳚ - दाय॑ । आ॒घा॒रमित्या᳚ - घा॒रम् । एति॑ । घा॒र॒ये॒त् । अ॒द्ध्व॒रः । य॒ज्ञ्ः । अ॒यम् । अ॒स्तु॒ । दे॒वाः॒ । ओष॑धीभ्य॒ इत्योष॑धि-भ्यः॒ । प॒शवे᳚ । नः॒ । जना॑य । विश्व॑स्मै । भू॒ताय॑ । अ॒ध्व॒रः । अ॒सि॒ । सः । पि॒न्व॒स्व॒ । घृ॒तव॒दिति॑ घृ॒त - व॒त् । दे॒व॒ । सो॒म॒ । इति॑ । सौ॒म्याः । ए॒व । तत् । आहु॑ती॒रित्या - हु॒तीः॒ । आ॒यत॑नवती॒रित्या॒यत॑न - व॒तीः॒ । क॒रो॒ति॒ । आ॒यत॑नवा॒नित्या॒यत॑न - वा॒न् । भ॒व॒ति॒ । यः । ए॒वम् । वेद॑ । अथो॒ इति॑ । द्यावा॑पृथि॒वी इति॒ द्यावा᳚-पृ॒थि॒वी । ए॒व । घृ॒तेन॑ । वीति॑ । उ॒न॒त्ति॒ । ते इति॑ । व्यु॑त्ते॒ इति॒ वि - उ॒त्ते॒ । उ॒प॒जी॒व॒नीये॒ इत्यु॑प - जी॒व॒नीये᳚ । भ॒व॒तः॒ । उ॒प॒जी॒व॒नीय॒ इत्यु॑प - जी॒व॒नीयः॑ । भ॒व॒ति॒ ।  \newline




\markright{ TS 3.1.9.4  \hfill https://www.vedavms.in \hfill}
\addcontentsline{toc}{section}{ TS 3.1.9.4 }
\section*{ TS 3.1.9.4 }

\textbf{TS 3.1.9.4 } \newline
\textbf{Samhita Paata} \newline

य ए॒वं ॅवेदै॒ष ते॑ रुद्रभा॒गो यं नि॒रया॑चथा॒स्तं जु॑षस्व वि॒देर्गौ॑प॒त्यꣳ रा॒यस्पोषꣳ॑ सु॒वीर्यꣳ॑ संॅवथ्स॒रीणाꣳ॑ स्व॒स्तिं ॥ मनुः॑ पु॒त्रेभ्यो॑ दा॒यं ॅव्य॑भज॒थ् स नाभा॒नेदि॑ष्ठं ब्रह्म॒चर्यं॒ ॅवस॑न्तं॒ निर॑भज॒थ् स आऽग॑च्छ॒थ् सो᳚ऽब्रवीत् क॒था मा॒ निर॑भा॒गिति॒ न त्वा॒ निर॑भाक्ष॒मित्य॑-ब्रवी॒दङ्गि॑रस इ॒मे स॒त्रमा॑सते॒ ते - [  ] \newline

\textbf{Pada Paata} \newline

यः । ए॒वम् । वेद॑ । ए॒षः । ते॒ । रु॒द्र॒ । भा॒गः । यम् । नि॒रया॑चथा॒ इति॑ निः-अया॑चथाः । तम् । जु॒ष॒स्व॒ । वि॒देः । गौ॒प॒त्यम् । रा॒यः । पोष᳚म् । सु॒वीर्य॒मिति॑ सु - वीर्य᳚म् । सं॒ॅव॒थ्स॒रीणा॒मिति॑ सं - व॒थ्स॒रीणा᳚म् । स्व॒स्तिम् ॥ मनुः॑ । पु॒त्रेभ्यः॑ । दा॒यम् । वीति॑ । अ॒भ॒ज॒त् । सः । नाभा॒नेदि॑ष्ठम् । ब्र॒ह्म॒चर्य॒मिति॑ ब्रह्म-चर्य᳚म् । वस॑न्तम् । निरिति॑ । अ॒भ॒ज॒त् । सः । एति॑ । अ॒ग॒च्छ॒त् । सः । अ॒ब्र॒वी॒त् । क॒था । मा॒ । निरिति॑ । अ॒भा॒क् । इति॑ । न । त्वा॒ । निरिति॑ । अ॒भा॒क्ष॒म् । इति॑ । अ॒ब्र॒वी॒त् । अङ्गि॑रसः । इ॒मे । स॒त्रम् । आ॒स॒ते॒ । ते ।  \newline




\markright{ TS 3.1.9.5  \hfill https://www.vedavms.in \hfill}
\addcontentsline{toc}{section}{ TS 3.1.9.5 }
\section*{ TS 3.1.9.5 }

\textbf{TS 3.1.9.5 } \newline
\textbf{Samhita Paata} \newline

सु॑व॒र्गं ॅलो॒कं न प्रजा॑नन्ति॒ तेभ्य॑ इ॒दं ब्राह्म॑णं ब्रूहि॒ ते सु॑व॒र्गं ॅलो॒कं ॅयन्तो॒ य ए॑षां प॒शव॒स्ताꣳस्ते॑ दास्य॒न्तीति॒ तदे᳚भ्योऽब्रवी॒त् ते सु॑व॒र्गं ॅलो॒कं ॅयन्तो॒ य ए॑षां प॒शव॒ आस॒न् तान॑स्मा अददु॒स्तं प॒शुभि॒श्चर॑न्तं ॅयज्ञ्वा॒स्तौ रु॒द्र आऽग॑च्छ॒थ् सो᳚ऽब्रवी॒न्मम॒ वा इ॒मे प॒शव॒ इत्यदु॒र्वै - [  ] \newline

\textbf{Pada Paata} \newline

सु॒व॒र्गमिति॑ सुवः - गम् । लो॒कम् । न । प्रेति॑ । जा॒न॒न्ति॒ । तेभ्यः॑ । इ॒दम् । ब्राह्म॑णम् । ब्रू॒हि॒ । ते । सु॒व॒र्गमिति॑ सुवः - गम् । लो॒कम् । यन्तः॑ । ये । ए॒षा॒म् । प॒शवः॑ । तान् । ते॒ । दा॒स्य॒न्ति॒ । इति॑ । तत् । ए॒भ्यः॒ । अ॒ब्र॒वी॒त् । ते । सु॒व॒र्गमिति॑ सुवः - गम् । लो॒कम् । यन्तः॑ । ये । ए॒षा॒म् । प॒शवः॑ । आसन्न्॑ । तान् । अ॒स्मै॒ । अ॒द॒दुः॒ । तम् । प॒शुभि॒रिति॑ प॒शु - भिः॒ । चर॑न्तम् । य॒ज्ञ्॒वा॒स्ताविति॑ यज्ञ्-वा॒स्तौ । रु॒द्रः । एति॑ । अ॒ग॒च्छ॒त् । सः । अ॒ब्र॒वी॒त् । मम॑ । वै । इ॒मे । प॒शवः॑ । इति॑ । अदुः॑ । वै ।  \newline




\markright{ TS 3.1.9.6  \hfill https://www.vedavms.in \hfill}
\addcontentsline{toc}{section}{ TS 3.1.9.6 }
\section*{ TS 3.1.9.6 }

\textbf{TS 3.1.9.6 } \newline
\textbf{Samhita Paata} \newline

मह्य॑मि॒मानित्य॑ब्रवी॒न्न वै तस्य॒ त ई॑शत॒ इत्य॑ब्रवी॒द्-यद्-य॑ज्ञ्वा॒स्तौ हीय॑ते॒ मम॒ वै तदिति॒ तस्मा᳚द्-यज्ञ्वा॒स्तु नाभ्य॒वेत्यꣳ॒॒ सो᳚ऽब्रवीद्-य॒ज्ञे मा ऽऽभ॒जाथ॑ ते प॒शून् नाभि मꣳ॑स्य॒ इति॒ तस्मा॑ ए॒तं म॒न्थिनः॑ सꣳ स्रा॒वम॑जुहो॒त् ततो॒ वै तस्य॑ रु॒द्रः प॒शून् नाभ्य॑मन्यत॒ यत्रै॒त ( ) मे॒वं ॅवि॒द्वान् म॒न्थिनः॑ सꣳ स्रा॒वं जु॒होति॒ न तत्र॑ रु॒द्रः प॒शून॒भि म॑न्यते ॥ \newline

\textbf{Pada Paata} \newline

मह्य᳚म् । इ॒मान् । इति॑ । अ॒ब्र॒वी॒त् । न । वै । तस्य॑ । ते । ई॒श॒ते॒ । इति॑ । अ॒ब्र॒वी॒त् । यत् । य॒ज्ञ्॒वा॒स्ताविति॑ यज्ञ्-वा॒स्तौ । हीय॑ते । मम॑ । वै । तत् । इति॑ । तस्मा᳚त् । य॒ज्ञ्॒वा॒स्त्विति॑ यज्ञ् - वा॒स्तु । न । अ॒भ्य॒वेत्य॒मित्य॑भि - अ॒वेत्य᳚म् । सः । अ॒ब्र॒वी॒त् । य॒ज्ञे । मा॒ । एति॑ । भ॒ज॒ । अथ॑ । ते॒ । प॒शून् । न । अ॒भीति॑ । मꣳ॒॒स्ये॒ । इति॑ । तस्मै᳚ । ए॒तम् । म॒न्थिनः॑ । सꣳ॒॒स्रा॒वमिति॑ सं - स्रा॒वम् । अ॒जु॒हो॒त् । ततः॑ । वै । तस्य॑ । रु॒द्रः । प॒शून् । न । अ॒भीति॑ । अ॒म॒न्य॒त॒ । यत्र॑ । ए॒तम् ( ) । ए॒वम् । वि॒द्वान् । म॒न्थिनः॑ । सꣳ॒॒स्रा॒वमिति॑ सं - स्रा॒वम् । जु॒होति॑ । न । तत्र॑ । रु॒द्रः । प॒शून् । अ॒भीति॑ । म॒न्य॒त॒ ॥  \newline




\markright{ TS 3.1.10.1  \hfill https://www.vedavms.in \hfill}
\addcontentsline{toc}{section}{ TS 3.1.10.1 }
\section*{ TS 3.1.10.1 }

\textbf{TS 3.1.10.1 } \newline
\textbf{Samhita Paata} \newline

जुष्टो॑ वा॒चो भू॑यासं॒ जुष्टो॑ वा॒चस्पत॑ये॒ देवि॑ वाक् । यद्वा॒चो मधु॑म॒त् तस्मि॑न् मा धाः॒ स्वाहा॒ सर॑स्वत्यै ॥ ऋ॒चा स्तोमꣳ॒॒ सम॑र्द्धय गाय॒त्रेण॑ रथंत॒रं । बृ॒हद्-गा॑य॒त्रव॑र्तनि ॥यस्ते᳚ द्र॒फ्सः स्कन्द॑ति॒ यस्ते॑ अꣳ॒॒शुर्बा॒हुच्यु॑तो धि॒षण॑योरु॒पस्था᳚त् । अ॒द्ध्व॒र्योर्वा॒ परि॒ यस्ते॑ प॒वित्रा॒थ् स्वाहा॑कृत॒मिन्द्रा॑य॒ तं जु॑होमि ॥ यो द्र॒फ्सो अꣳ॒॒शुः प॑ति॒तः पृ॑थि॒व्यां प॑रिवा॒पात् - [  ] \newline

\textbf{Pada Paata} \newline

जुष्टः॑ । वा॒चः । भू॒या॒स॒म् । जुष्टः॑ । वा॒चः । पत॑ये । देवि॑ । वा॒क् ॥ यत् । वा॒चः । मधु॑म॒दिति॒ मधु॑-म॒त् । तस्मिन्न्॑ । मा॒ । धाः॒ । स्वाहा᳚ । सर॑स्वत्यै ॥ ऋ॒चा । स्तोम᳚म् । समिति॑ । अ॒द्‌र्ध॒य॒ । गा॒य॒त्रेण॑ । र॒थ॒न्त॒रमिति॑ रथं-त॒रम् ॥ बृ॒हत् । गा॒य॒त्रव॑र्त॒नीति॑ गाय॒त्र-व॒र्त॒नि॒ ॥ यः । ते॒ । द्र॒फ्सः । स्कन्द॑ति । यः । ते॒ । अꣳ॒॒शुः । बा॒हुच्यु॑त॒ इति॑ बा॒हु - च्यु॒तः॒ । धि॒षण॑योः । उ॒पस्था॒दित्यु॒प - स्था॒त् ॥ अ॒द्ध्व॒र्योः । वा॒ । परीति॑ । यः । ते॒ । प॒वित्रा᳚त् । स्वाहा॑कृत॒मिति॒ स्वाहा᳚ -कृ॒त॒म् । इन्द्रा॑य । तम् । जु॒हो॒मि॒ ॥ यः । द्र॒फ्सः । अꣳ॒॒शुः । प॒ति॒तः । पृ॒थि॒व्याम् । प॒रि॒वा॒पादिति॑ परि - वा॒पात् ।  \newline




\markright{ TS 3.1.10.2  \hfill https://www.vedavms.in \hfill}
\addcontentsline{toc}{section}{ TS 3.1.10.2 }
\section*{ TS 3.1.10.2 }

\textbf{TS 3.1.10.2 } \newline
\textbf{Samhita Paata} \newline

पु॑रो॒डाशा᳚त् कर॒म्भात् । धा॒ना॒सो॒मान्म॒न्थिन॑ इन्द्र शु॒क्राथ् स्वाहा॑कृत॒मिन्द्रा॑य॒ तं जु॑होमि ॥यस्ते᳚ द्र॒फ्सो मधु॑माꣳ इन्द्रि॒यावा॒न्थ् स्वाहा॑कृतः॒ पुन॑र॒प्येति॑ दे॒वान् । दि॒वः पृ॑थि॒व्याः पर्य॒न्तरि॑क्षा॒थ् स्वाहा॑ कृत॒मिन्द्रा॑य॒ तं जु॑होमि ॥ अ॒द्ध्व॒र्युर्वा ऋ॒त्विजां᳚ प्रथ॒मो यु॑ज्यते॒ तेन॒ स्तोमो॑ योक्त॒व्य॑ इत्या॑हु॒र्वाग॑ग्रे॒गा अग्र॑ एत्वृजु॒गा दे॒वेभ्यो॒ यशो॒ मयि॒ दध॑ती प्रा॒णान् प॒शुषु॑ प्र॒जां मयि॑- [  ] \newline

\textbf{Pada Paata} \newline

पु॒रो॒डाशा᳚त् । क॒र॒म्भात् ॥ धा॒ना॒सो॒मादिति॑ धाना-सो॒मात् । म॒न्थिनः॑ । इ॒न्द्र॒ । शु॒क्रात् । स्वाहा॑कृत॒मिति॒ स्वाहा᳚ - कृ॒त॒म् । इन्द्रा॑य । तम् । जु॒हो॒मि॒ ॥ यः । ते॒ । द्र॒फ्सः । मधु॑मा॒निति॒ मधु॑ - मा॒न् । इ॒न्द्रि॒यावा॒निती᳚न्द्रि॒य - वा॒न् । स्वाहा॑कृत॒ इति स्वाहा᳚-कृ॒तः॒ । पुनः॑ । अ॒प्येतीत्य॑पि - एति॑ । दे॒वान् ॥ दि॒वः । पृ॒थि॒व्याः । परीति॑ । अ॒न्तरि॑क्षात् । स्वाहा॑कृत॒मिति॒ स्वाहा᳚ - कृ॒त॒म् । इन्द्रा॑य । तम् । जु॒हो॒मि॒ ॥ अ॒द्ध्व॒र्युः । वै । ऋ॒त्विजा᳚म् । प्र॒थ॒मः । यु॒ज्य॒ते॒ । तेन॑ । स्तोमः॑ । यो॒क्त॒व्यः॑ । इति॑ । आ॒हुः॒ । वाक् । अ॒ग्रे॒गा इत्य॑ग्रे - गाः । अग्र᳚ । ए॒तु॒ । ऋ॒जु॒गा इत्यृ॑जु - गाः । दे॒वेभ्यः॑ । यशः॑ । मयि॑ । दध॑ती । प्रा॒णानिति॑ प्र - अ॒नान् । प॒शुषु॑ । प्र॒जामिति॑ प्र - जाम् । मयि॑ ।  \newline




\markright{ TS 3.1.10.3  \hfill https://www.vedavms.in \hfill}
\addcontentsline{toc}{section}{ TS 3.1.10.3 }
\section*{ TS 3.1.10.3 }

\textbf{TS 3.1.10.3 } \newline
\textbf{Samhita Paata} \newline

च॒ यज॑माने॒ चेत्या॑ह॒ वाच॑मे॒व तद्य॑ज्ञ्मु॒खे यु॑नक्ति॒ वास्तु॒ वा ए॒तद्य॒ज्ञ्स्य॑ क्रियते॒ यद्ग्रहा᳚न् गृही॒त्वा ब॑हिष्पवमा॒नꣳ सर्प॑न्ति॒परा᳚ञ्चो॒ हि यन्ति॒ परा॑चीभिः स्तु॒वते॑ वैष्ण॒व्यर्चा पुन॒रेत्योप॑ तिष्ठते य॒ज्ञो वै विष्णु॑ र्य॒ज्ञ्मे॒वाक॒र्विष्णो॒ त्वन्नो॒ अन्त॑मः॒ शर्म॑ यच्छ सहन्त्य । प्र ते॒ धारा॑ मधु॒श्चुत॒ उथ्सं॑ दुह्रते॒ ( ) अक्षि॑त॒मित्या॑ह॒ यदे॒वास्य॒ शया॑नस्योप॒शुष्य॑ति॒ तदे॒वास्यै॒तेना ऽऽ*प्या॑ययति ॥ \newline

\textbf{Pada Paata} \newline

च॒ । यज॑माने । च॒ । इति॑ । आ॒ह॒ । वाच᳚म् । ए॒व । तत् । य॒ज्ञ्॒मु॒ख इति॑ यज्ञ् - मु॒खे । यु॒न॒क्ति॒ । वास्तु॑ । वै । ए॒तत् । य॒ज्ञ्स्य॑ । क्रि॒य॒ते॒ । यत् । ग्रहान्॑ । गृ॒ही॒त्वा । ब॒हि॒ष्प॒व॒मा॒नमिति॑ बहिः - प॒व॒मा॒नम् । सर्प॑न्ति । परा᳚ञ्चः । हि । यन्ति॑ । परा॑चीभिः । स्तु॒वते᳚ । वै॒ष्ण॒व्या । ऋ॒चा । पुनः॑ । एत्येत्या᳚ - इत्य॑ । उपेति॑ । ति॒ष्ठ॒ते॒ । य॒ज्ञ्ः । वै । विष्णुः॑ । य॒ज्ञ्म् । ए॒व । अ॒कः॒ । विष्णो॒ इति॑ । त्वम् । नः॒ । अन्त॑मः । शर्म॑ । य॒च्छ॒ । स॒ह॒न्त्य॒ ॥ प्रेति॑ । ते॒ । धाराः᳚ । म॒धु॒श्चुत॒ इति॑ मधु-श्चुतः॑ । उथ्स᳚म् । दु॒ह्र॒ते॒ ( ) । अक्षि॑तम् । इति॑ । आ॒ह॒ । यत् । ए॒व । अ॒स्य॒ । शया॑नस्य । उ॒प॒शुष्य॒तीत्यु॑प - शुष्य॑ति । तत् । ए॒व । अ॒स्य॒ । ए॒तेन॑ । एति॑ । प्या॒य॒य॒ति॒ ॥  \newline




\markright{ TS 3.1.11.1  \hfill https://www.vedavms.in \hfill}
\addcontentsline{toc}{section}{ TS 3.1.11.1 }
\section*{ TS 3.1.11.1 }

\textbf{TS 3.1.11.1 } \newline
\textbf{Samhita Paata} \newline

अ॒ग्निना॑ र॒यिम॑श्नव॒त् पोष॑मे॒व दि॒वेदि॑वे । य॒शसं॑ ॅवी॒रव॑त्तमं ॥गोमाꣳ॑ अ॒ग्नेऽवि॑माꣳ अ॒श्वी य॒ज्ञो नृ॒वथ्स॑खा॒ सद॒मिद॑प्रमृ॒ष्यः । इडा॑वाꣳ ए॒षो अ॑सुर प्र॒जावा᳚न् दी॒र्घो र॒यिः पृ॑थुबु॒ध्नः स॒भावान्॑ ॥आप्या॑यस्व॒>1, सन्ते᳚>2 ॥ इ॒ह त्वष्टा॑रमग्रि॒यं ॅवि॒श्वरू॑प॒मुप॑ ह्वये । अ॒स्माक॑मस्तु॒ केव॑लः ॥ तन्न॑स्तु॒रीप॒मध॑ पोषयि॒त्नु देव॑ त्वष्ट॒र्वि र॑रा॒णः स्य॑स्व । यतो॑ वी॒रः - [  ] \newline

\textbf{Pada Paata} \newline

अ॒ग्निना᳚ । र॒यिम् । अ॒श्न॒व॒त् । पोष᳚म् । ए॒व । दि॒वेदि॑व॒ इति॑ दि॒वे - दि॒व॒ ॥ य॒शस᳚म् । वी॒रव॑त्तम॒मिति॑ वी॒रव॑त् - त॒म॒म् ॥ गोमा॒निति॒ गो - मा॒न् । अ॒ग्ने॒ । अवि॑मा॒नित्यवि॑ - मा॒न् । अ॒श्वी । य॒ज्ञ्ः । नृ॒वथ्स॒खेति॑ नृ॒वत् - स॒खा॒ । सद᳚म् । इत् । अ॒प्र॒मृ॒ष्य इत्य॑प्र - मृ॒ष्यः ॥ इडा॑वा॒नितीडा᳚ - वा॒न् । ए॒षः । अ॒सु॒र॒ । प्र॒जावा॒निति॑ प्र॒जा - वा॒न् । दी॒र्घः । र॒यिः । पृ॒थु॒बु॒द्ध्न इति॑ पृथु - बु॒द्ध्नः । स॒भावा॒निति॑ स॒भा - वा॒न् ॥ एति॑ । प्या॒य॒स्व॒ । समिति॑ । ते॒ ॥ इ॒ह । त्वष्टा॑रम् । अ॒ग्रि॒यम् । वि॒श्वरू॑प॒मिति॑ वि॒श्व - रू॒प॒म् । उपेति॑ । ह्व॒ये॒ ॥ अ॒स्माक᳚म् । अ॒स्तु॒ । केव॑लः ॥ तत् । नः॒ । तु॒रीप᳚म् । अध॑ । पो॒ष॒यि॒त्नु । देव॑ । त्व॒ष्टः॒ । वीति॑ । र॒रा॒णः । स्य॒स्व॒ ॥ यतः॑ । वी॒रः ।  \newline




\markright{ TS 3.1.11.2  \hfill https://www.vedavms.in \hfill}
\addcontentsline{toc}{section}{ TS 3.1.11.2 }
\section*{ TS 3.1.11.2 }

\textbf{TS 3.1.11.2 } \newline
\textbf{Samhita Paata} \newline

क॑र्म॒ण्यः॑ सु॒दक्षो॑ यु॒क्तग्रा॑वा॒ जाय॑ते दे॒वका॑मः ॥शि॒वस्त्व॑ष्टरि॒हाऽऽ* ग॑हि वि॒भुः पोष॑ उ॒तत्मना᳚ । य॒ज्ञेय॑ज्ञे न॒ उद॑व ॥ पि॒शङ्ग॑रूपः सु॒भरो॑ वयो॒धाः श्रु॒ष्टी वी॒रो जा॑यते दे॒वका॑मः । प्र॒जां त्वष्टा॒ विष्य॑तु॒ नाभि॑म॒स्मे अथा॑ दे॒वाना॒मप्ये॑तु॒ पाथः॑ ॥ प्रणो॑दे॒>3 व्या, नो॑ दि॒वः >4 ॥ पी॒पि॒वाꣳ सꣳ॒॒ सर॑स्वतः॒ स्तनं॒ ॅयो वि॒श्वद॑र्.शतः । धुक्षी॒महि॑ प्र॒जामिषं᳚ ॥ \newline

\textbf{Pada Paata} \newline

क॒र्म॒ण्यः॑ । सु॒दक्ष॒ इति॑ सु - दक्षः॑ । यु॒क्तग्रा॒वेति॑ यु॒क्त - ग्रा॒वा॒ । जाय॑ते । दे॒वका॑म॒ इति॑ दे॒व - का॒मः॒ ॥ शि॒वः । त्व॒ष्टः॒ । इ॒ह । एति॑ । ग॒हि॒ । वि॒भुरिति॑ वि - भुः । पोषे᳚ । उ॒त । त्मना᳚ ॥ य॒ज्ञे य॑ज्ञ्॒ इति॑ य॒ज्ञे - य॒ज्ञे॒ । नः॒ । उदिति॑ । अ॒व॒ ॥ पि॒शङ्ग॑रूप॒ इति॑ पि॒शङ्ग॑-रू॒पः॒ । सु॒भर॒ इति॑ सु - भरः॑ । व॒यो॒धा इति॑ वयः - धाः । श्रु॒ष्टी । वी॒रः । जा॒य॒ते॒ । दे॒वका॑म॒ इति॑ दे॒व - का॒मः॒ ॥ प्र॒जामिति॑ प्र - जाम् । त्वष्टा᳚ । वीति॑ । स्य॒तु॒ । नाभि᳚म् । अ॒स्मे इति॑ । अथ॑ । दे॒वाना᳚म् । अपीति॑ । ए॒तु॒ । पाथः॑ ॥ प्रेति॑ । नः॒ । दे॒वी । एति॑ । नः॒ । दि॒वः ॥ पी॒पि॒वाꣳस᳚म् । सर॑स्वतः । स्तन᳚म् । यः । वि॒श्वद॑र्.शत॒ इति॑ वि॒श्व - द॒र्॒.श॒तः॒ ॥ धु॒क्षी॒महि॑ । प्र॒जामिति॑ प्र - जाम् । इष᳚म् ॥  \newline




\markright{ TS 3.1.11.3  \hfill https://www.vedavms.in \hfill}
\addcontentsline{toc}{section}{ TS 3.1.11.3 }
\section*{ TS 3.1.11.3 }

\textbf{TS 3.1.11.3 } \newline
\textbf{Samhita Paata} \newline

ये ते॑ सरस्व ऊ॒र्मयो॒ मधु॑मन्तो घृत॒श्चुतः॑ । तेषां᳚ ते सु॒म्नमी॑महे ॥ यस्य॑ व्र॒तं प॒शवो॒ यन्ति॒ सर्वे॒ यस्य॑ व्र॒तमु॑प॒तिष्ठ॑न्त॒ आपः॑ । यस्य॑ व्र॒ते पु॑ष्टि॒पति॒र्निवि॑ष्ट॒स्तꣳ सर॑स्वन्त॒मव॑से हुवेम ॥ दि॒व्यꣳ सु॑प॒र्णं ॅव॑य॒सं बृ॒हन्त॑म॒पां गर्भं॑ ॅवृष॒भमोष॑धीनां । अ॒भी॒प॒तो वृ॒ष्ट्या त॒र्पय॑न्तं॒ तꣳ सर॑स्वन्त॒मव॑से हुवेम ॥ सिनी॑वालि॒ पृथु॑ष्टुके॒ या दे॒वाना॒मसि॒ स्वसा᳚ । जु॒षस्व॑ ह॒व्य - [  ] \newline

\textbf{Pada Paata} \newline

ये । ते॒ । स॒र॒स्वः॒ । ऊ॒र्मयः॑ । मधु॑मन्त॒ इति॒ मधु॑ - म॒न्तः॒ । घृ॒त॒श्चुत॒ इति॑ घृत - श्चुतः॑ ॥ तेषा᳚म् । ते॒ । सु॒म्नम् । ई॒म॒हे॒ ॥ यस्य॑ । व्र॒तम् । प॒शवः॑ । यन्ति॑ । सर्वे᳚ । यस्य॑ । व्र॒तम् । उ॒प॒तिष्ठ॑न्त॒ इत्यु॑प -तिष्ठ॑न्ते । आपः॑ ॥ यस्य॑ । व्र॒ते । पु॒ष्टि॒पति॒रिति॑ पुष्टि-पतिः॑ । निवि॑ष्ट॒ इति॒ नि - वि॒ष्टः॒ । तम् । सर॑स्वन्तम् । अव॑से । ह॒वे॒म॒ ॥ दि॒व्यम् । सु॒प॒र्णमिति॑ सु - प॒र्णम् । व॒य॒सम् । बृ॒हन्त᳚म् । अ॒पाम् । गर्भ᳚म् । वृ॒ष॒भम् । ओष॑धीनाम् ॥ अ॒भी॒प॒तः । वृ॒ष्ट्या । त॒र्पय॑न्तम् । तम् । सर॑स्वन्तम् । अव॑से । हु॒वे॒म॒ ॥ सिनी॑वालि । पृथु॑ष्टुक॒ इति॒ पृथु॑ - स्तु॒के॒ । या । दे॒वाना᳚म् । असि॑ । स्वसा᳚ ॥ जु॒षस्व॑ । ह॒व्यम् ।  \newline




\markright{ TS 3.1.11.4  \hfill https://www.vedavms.in \hfill}
\addcontentsline{toc}{section}{ TS 3.1.11.4 }
\section*{ TS 3.1.11.4 }

\textbf{TS 3.1.11.4 } \newline
\textbf{Samhita Paata} \newline

माहु॑तं प्र॒जां दे॑वि दिदिड्ढि नः ॥ या सु॑पा॒णिः स्व॑ङ्गु॒रिः सु॒षूमा॑ बहु॒सूव॑री । तस्यै॑ वि॒श्पत्नि॑यै ह॒विः सि॑नीवा॒ल्यै जु॑होतन ॥ इन्द्रं॑ ॅवो वि॒श्वत॒स्परी>5, न्द्रं॒ नरः॑>6 ॥ असि॑तवर्णा॒ हर॑यः सुप॒र्णा मिहो॒ वसा॑ना॒ दिव॒मुत् प॑तन्ति ॥ त आऽव॑वृत्र॒न्थ् सद॑नानि कृ॒त्वाऽऽदित् पृ॑थि॒वी घृ॒तैर्व्यु॑द्यते ॥ हिर॑ण्यकेशो॒ रज॑सो विसा॒रेऽहि॒र्धुनि॒र्वात॑ इव॒ ध्रजी॑मान् । शुचि॑भ्राजा उ॒षसो॒ - [  ] \newline

\textbf{Pada Paata} \newline

आहु॑त॒मित्या - हु॒त॒म् । प्र॒जामिति॑ प्र-जाम् । दे॒वि॒ । दि॒दि॒ड्ढि॒ । नः॒ ॥ या । सु॒पा॒णिरिति॑ सु - पा॒णिः । स्व॒ङ्गु॒रिरिति॑ सु - अ॒ङ्गु॒रिः । सु॒षूमेति॑ सु - सूमा᳚ । ब॒हु॒सूव॒रीति॑ बहु - सूव॑री ॥ तस्यै᳚ । वि॒श्पत्नि॑यै । ह॒विः । सि॒नी॒वा॒ल्यै । जु॒हो॒त॒न॒ ॥ इन्द्र᳚म् । वः॒ । वि॒श्वतः॑ । परीति॑ । इन्द्र᳚म् । नरः॑ ॥ असि॑तवर्णा॒ इत्यसि॑त - व॒र्णाः॒ । हर॑यः । सु॒प॒र्णा इति॑ सु - प॒र्णाः । मिहः॑ । वसा॑नाः । दिव᳚म् । उदिति॑ । प॒त॒न्ति॒ ॥ ते । एति॑ । अ॒व॒वृ॒त्र॒न्न् । सद॑नानि । कृ॒त्वा । आत् । इत् । पृ॒थि॒वी । घृ॒तैः । वीति॑ । उ॒द्य॒ते॒ ॥ हिर॑ण्यकेश॒ इति॒ हिर॑ण्य - के॒शः॒ । रज॑सः । वि॒सा॒र इति॑ वि - सा॒रे । अहिः॑ । धुनिः॑ । वातः॑ । इ॒व॒ । ध्रजी॑मान् ॥ शुचि॑भ्राजा॒ इति॒ शुचि॑ - भ्रा॒जाः॒ । उ॒षसः॑ ।  \newline




\markright{ TS 3.1.11.5  \hfill https://www.vedavms.in \hfill}
\addcontentsline{toc}{section}{ TS 3.1.11.5 }
\section*{ TS 3.1.11.5 }

\textbf{TS 3.1.11.5 } \newline
\textbf{Samhita Paata} \newline

नवे॑दा॒ यश॑स्वतीरप॒स्युवो॒ न स॒त्याः ॥आ ते॑ सुप॒र्णा अ॑मिनन्त॒ एवैः᳚ कृ॒ष्णो नो॑नाव वृष॒भो यदी॒दं । शि॒वाभि॒र्न स्मय॑मानाभि॒राऽगा॒त् पत॑न्ति॒ मिहः॑ स्त॒नय॑न्त्य॒भ्रा ॥ वा॒श्रेव॑ वि॒द्युन्मि॑माति व॒थ्सं न मा॒ता सि॑षक्ति । यदे॑षां ॅवृ॒ष्टिरस॑र्जि ॥ पर्व॑तश्चि॒न्महि॑ वृ॒द्धो बि॑भाय दि॒वश्चि॒थ् सानु॑ रेजत स्व॒ने वः॑ । यत् क्रीड॑थ मरुत - [  ] \newline

\textbf{Pada Paata} \newline

नवे॑दाः । यश॑स्वतीः । अ॒प॒स्युवः॑ । न । स॒त्याः ॥ एति॑ । ते॒ । सु॒प॒र्णा इति॑ सु - प॒र्णाः । अ॒मि॒न॒न्त॒ । एवैः᳚ । कृ॒ष्णः । नो॒ना॒व॒ । वृ॒ष॒भः । यदि॑ । इ॒दम् ॥ शि॒वाभिः॑ । न । स्मय॑मानाभिः । एति॑ । अ॒गा॒त् । पत॑न्ति । मिहः॑ । स्त॒नय॑न्ति । अ॒भ्रा ॥ वा॒श्रा । इ॒व॒ । वि॒द्युदिति॑ वि-द्युत् । मि॒मा॒ति॒ । व॒थ्सम् । न । मा॒ता । सि॒ष॒क्ति॒ ॥ यत् । ए॒षा॒म् । वृ॒ष्टिः । अस॑र्जि ॥ पर्व॑तः । चि॒त् । महि॑ । वृ॒द्धः । बि॒भा॒य॒ । दि॒वः । चि॒त् । सानु॑ । रे॒ज॒त॒ । स्व॒ने । वः॒ ॥ यत् । क्रीड॑थ । म॒रु॒तः॒ ।  \newline




\markright{ TS 3.1.11.6  \hfill https://www.vedavms.in \hfill}
\addcontentsline{toc}{section}{ TS 3.1.11.6 }
\section*{ TS 3.1.11.6 }

\textbf{TS 3.1.11.6 } \newline
\textbf{Samhita Paata} \newline

ऋष्टि॒मन्त॒ आप॑ इव स॒द्ध्रिय॑ञ्चो धवद्ध्वे ॥अ॒भि क्र॑न्द स्त॒नय॒ गर्भ॒मा धा॑ उद॒न्वता॒ परि॑ दीया॒ रथे॑न । दृतिꣳ॒॒ सु क॑र्.ष॒ विषि॑तं॒ न्य॑ञ्चꣳ स॒मा भ॑वन्तू॒द्वता॑ निपा॒दाः ॥त्वं त्या चि॒दच्यु॒ताऽग्ने॑ प॒शुर्न यव॑से । धामा॑ ह॒ यत् ते॑ अजर॒ वना॑ वृ॒श्चन्ति॒ शिक्व॑सः ॥ अग्ने॒ भूरी॑णि॒ तव॑ जातवेदो॒ देव॑ स्वधावो॒ऽमृत॑स्य॒ धाम॑ । याश्च॑ - [  ] \newline

\textbf{Pada Paata} \newline

ऋ॒ष्टि॒मन्त॒ इत्यृ॑ष्टि-मन्तः॑ । आपः॑ । इ॒व॒ । स॒द्ध्रिय॑ञ्चः । ध॒व॒द्ध्वे॒ ॥ अ॒भीति॑ । क्र॒न्द॒ । स्त॒नय॑ । गर्भ᳚म् । एति॑ । धाः॒ । उ॒द॒न्वतेत्यु॑दन्न् - वता᳚ । परीति॑ । दी॒य॒ । रथे॑न ॥ दृति᳚म् । स्विति॑ । क॒र्.॒ष॒ । विषि॑त॒मिति॒ वि - सि॒त॒म् । न्य॑ञ्चम् । स॒माः । भ॒व॒न्तु॒ । उ॒द्वतेत्य॑त् - वता᳚ । नि॒पा॒दा इति॑ नि - पा॒दाः ॥ त्वम् । त्या । चि॒त् । अच्यु॑ता । अग्ने᳚ । प॒शुः । न । यव॑से ॥ धाम॑ । ह॒ । यत् । ते॒ । अ॒ज॒र॒ । वना᳚ । वृ॒श्चन्ति॑ । शिक्व॑सः ॥ अग्ने᳚ । भूरी॑णि । तव॑ । जा॒त॒वे॒द॒ इति॑ जात - वे॒दः॒ । देव॑ । स्व॒धा॒व॒ इति॑ स्वधा - वः॒ । अ॒मृत॑स्य । धाम॑ ॥ याः । च॒ ।  \newline




\markright{ TS 3.1.11.7  \hfill https://www.vedavms.in \hfill}
\addcontentsline{toc}{section}{ TS 3.1.11.7 }
\section*{ TS 3.1.11.7 }

\textbf{TS 3.1.11.7 } \newline
\textbf{Samhita Paata} \newline

मा॒या मा॒यिनां᳚ ॅविश्वमिन्व॒ त्वे पू॒र्वीः स॑न्द॒धुः पृ॑ष्टबन्धो ॥ दि॒वो नो॑ वृ॒ष्टिं म॑रुतो ररीद्ध्वं॒ प्रपि॑न्वत॒ वृष्णो॒ अश्व॑स्य॒ धाराः᳚ । अ॒र्वाङे॒तेन॑ स्तनयि॒त्नुनेह्य॒पो नि॑षि॒ञ्चन्नसु॑रः पि॒ता नः॑ ॥ पिन्व॑न्त्य॒पो म॒रुतः॑ सु॒दान॑वः॒ पयो॑ घृ॒तव॑द्वि॒दथे᳚ष्वा॒ भुवः॑ । अत्यं॒ न मि॒हे वि न॑यन्ति वा॒जिन॒मुथ्सं॑ दुहन्ति स्त॒नय॑न्त॒मक्षि॑तं ॥ उ॒द॒प्रुतो॑ मरुत॒स्ताꣳ इ॑यर्त॒ वृष्टिं॒ - [  ] \newline

\textbf{Pada Paata} \newline

मा॒याः । मा॒यिना᳚म् । वि॒श्व॒मि॒न्वेति॑ विश्वम्-इ॒न्व॒ । त्वे इति॑ । पू॒र्वीः । स॒न्द॒धुरिति॑ सं - द॒धुः । पृ॒ष्ट॒ब॒न्धो॒ इति॑ पृष्ट - ब॒न्धो॒ ॥ दि॒वः । नः॒ । वृ॒ष्टिम् । म॒रु॒तः॒ । र॒री॒द्ध्व॒म् । प्रेति॑ । पि॒न्व॒त॒ । वृष्णः॑ । अश्व॑स्य । धाराः᳚ ॥ अ॒र्वाङ् । ए॒तेन॑ । स्त॒न॒यि॒त्नुना᳚ । एति॑ । इ॒हि॒ । अ॒पः । नि॒षि॒ञ्चन्निति॑ नि -सि॒ञ्चन्न् । असु॑रः । पि॒ता । नः॒ ॥ पिन्व॑न्ति । अ॒पः । म॒रुतः॑ । सु॒दान॑व॒ इति॑ सु - दान॑वः । पयः॑ । घृ॒तव॒दिति॑ घृ॒त - व॒त् । वि॒दथे॑षु । आ॒भुव॒ इत्या᳚ - भुवः॑ ॥ अत्य᳚म् । न । मि॒हे । वीति॑ । न॒य॒न्ति॒ । वा॒जिन᳚म् । उथ्स᳚म् । दु॒ह॒न्ति॒ । स्त॒नय॑न्तम् । अक्षि॑तम् ॥ उ॒द॒प्रुत॒ इत्यु॑द - प्रुतः॑ । म॒रु॒तः॒ । तान् । इ॒य॒र्त॒ । वृष्टि᳚म् ।  \newline




\markright{ TS 3.1.11.8  \hfill https://www.vedavms.in \hfill}
\addcontentsline{toc}{section}{ TS 3.1.11.8 }
\section*{ TS 3.1.11.8 }

\textbf{TS 3.1.11.8 } \newline
\textbf{Samhita Paata} \newline

ॅये विश्वे॑ म॒रुतो॑ जु॒नन्ति॑ । क्रोशा॑ति॒ गर्दा॑ क॒न्ये॑व तु॒न्ना पेरुं॑ तुञ्जा॒ना पत्ये॑व जा॒या ॥ घृ॒तेन॒ द्यावा॑पृथि॒वी मधु॑ना॒ समु॑क्षत॒ पय॑स्वतीः कृण॒ताऽऽ*प॒ ओष॑धीः । ऊर्जं॑ च॒ तत्र॑ सुम॒तिं च॑ पिन्वथ॒ यत्रा॑ नरो मरुतः सि॒ञ्चथा॒ मधु॑ ॥ उदु॒त्यं >7, चि॒त्रं >8 ॥ औ॒र्व॒-भृ॒गु॒वच्छुचि॑मप्नवान॒वदा हु॑वे । अ॒ग्निꣳ स॑मु॒द्रवा॑ससं ॥ आ स॒वꣳ स॑वि॒तुर्य॑था॒ भग॑स्ये ( ) व भु॒जिꣳ हु॑वे । अ॒ग्निꣳ स॑मु॒द्रवा॑ससं ॥ हु॒वे वात॑स्वनं क॒विं प॒र्जन्य॑क्रन्द्यꣳ॒॒ सहः॑ । अ॒ग्निꣳ स॑मु॒द्रवा॑ससं ॥ \newline

\textbf{Pada Paata} \newline

ये । विश्वे᳚ । म॒रुतः॑ । जु॒नन्ति॑ ॥ क्रोशा॑ति । गर्दा᳚ । क॒न्या᳚ । इ॒व॒ । तु॒न्ना । पेरु᳚म् । तु॒ञ्जा॒ना । पत्या᳚ । इ॒व॒ । जा॒या ॥ घृ॒तेन॑ । द्यावा॑पृथि॒वी इति॒ द्यावा᳚ - पृ॒थि॒वी । मधु॑ना । समिति॑ । उ॒क्ष॒त॒ । पय॑स्वतीः । कृ॒णु॒त॒ । आपः॑ । ओष॑धीः ॥ ऊर्ज᳚म् । च॒ । तत्र॑ । सु॒म॒तिमिति॑ सु - म॒तिम् । च॒ । पि॒न्व॒थ॒ । यत्र॑ । न॒रः॒ । म॒रु॒तः॒ । सि॒ञ्चथ॑ । मधु॑ ॥ उदिति॑ । उ॒ । त्यम् । चि॒त्रम् ॥ औ॒र्व॒भृ॒गु॒वदित्यौ᳚वभृगु - वत् । शुचि᳚म् । अ॒प्न॒वा॒न॒वदित्य॑प्नवान - वत् । एति॑ । हु॒वे॒ ॥ अ॒ग्निम् । स॒मु॒द्रवा॑सस॒मिति॑ समु॒द्र - वा॒स॒स॒म् ॥ एति॑ । स॒वम् । स॒वि॒तुः । य॒था॒ । भग॑स्य ( ) । इ॒व॒ । भु॒जिम् । हु॒वे॒ ॥ अ॒ग्निम् । स॒मु॒द्रवा॑सस॒मिति॑ समु॒द्र - वा॒स॒स॒म् ॥ हु॒वे । वात॑स्वन॒मिति॒ वात॑ - स्व॒न॒म् । क॒विम् । प॒र्जन्य॑क्रन्द्य॒मिति॑ प॒र्जन्य॑ - क्र॒न्द्य॒म् । सहः॑ ॥ अ॒ग्निम् । स॒मु॒द्रवा॑सस॒मिति॑ समु॒द्र - वा॒स॒स॒म् ॥  \newline






\end{document}