\documentclass[17pt]{extarticle}
\usepackage{babel}
\usepackage{fontspec}
\usepackage{polyglossia}
\usepackage{extsizes}

\usepackage{color}   %May be necessary if you want to color links
\usepackage{hyperref}
\hypersetup{
    colorlinks=true, %set true if you want colored links
    linktoc=all,     %set to all if you want both sections and subsections linked
    linkcolor=black,  %choose some color if you want links to stand out
}

\setmainlanguage{sanskrit}
\setotherlanguages{english} %% or other languages
\setlength{\parindent}{0pt}
\pagestyle{myheadings}
\newfontfamily\devanagarifont[Script=Devanagari]{AdishilaVedic}
\renewcommand{\theHsection}{\thepart.section.\thesection}

\newcommand{\VAR}[1]{}
\newcommand{\BLOCK}[1]{}




\begin{document}
\begin{titlepage}
    \begin{center}
 
\begin{sanskrit}
    { \Large
    कृष्ण यजुर्वेदीय तैत्तिरीय संहिता,पद,जटा,घन पाठः 
    }
    \\
    \vspace{2.5cm}
    \mbox{ \Large
    3.1     तृतीयकाण्डे प्रथमः प्रश्नः - न्यूनकर्माभिधानं   }
\end{sanskrit}
\end{center}

\end{titlepage}
\tableofcontents
\phantomsection
\pagebreak

\markright{ TS 3.1.1.1  \hfill https://www.vedavms.in \hfill}

\section{ TS 3.1.1.1 }

\textbf{TS 3.1.1.1 } \newline
\textbf{Samhita Paata} \newline

प्र॒जाप॑तिरकामयत प्र॒जाः सृ॑जे॒येति॒ स तपो॑ऽतप्यत॒ स स॒र्पान॑सृजत॒ सो॑ऽकामयत प्र॒जाः सृ॑जे॒येति॒ सद्वि॒तीय॑मतप्यत॒ स वयाꣳ॑स्य सृजत॒ सो॑ऽकामयत प्र॒जाः सृ॑जे॒येति॒ स तृ॒तीय॑मतप्यत॒ स ए॒तं दी᳚क्षितवा॒द-म॑पश्य॒त् तम॑वद॒त् ततो॒ वै स प्र॒जा अ॑सृजत॒ यत् तप॑स्त॒प्त्वा दी᳚क्षितवा॒दं ॅवद॑ति प्र॒जा ए॒व तद्यज॑मानः - [  ] \newline

\textbf{Pada Paata} \newline

प्र॒जाप॑ति॒रिति॑ प्र॒जा - प॒तिः॒ । अ॒का॒म॒य॒त॒ । प्र॒जा इति॑ प्र - जाः । सृ॒जे॒य॒ । इति॑ । सः । तपः॑ । अ॒त॒प्य॒त॒ । सः । स॒र्पान् । अ॒सृ॒ज॒त॒ । सः । अ॒का॒म॒य॒त॒ । प्र॒जा इति॑ प्र - जाः । सृ॒जे॒य॒ । इति॑ । सः । द्वि॒तीय᳚म् । अ॒त॒प्य॒त॒ । सः । वयाꣳ॑सि । अ॒सृ॒ज॒त॒ । सः । अ॒का॒म॒य॒त॒ । प्र॒जा इति॑ प्र-जाः । सृ॒जे॒य॒ । इति॑ । सः । तृ॒तीय᳚म् । अ॒त॒प्य॒त॒ । सः । ए॒तम् । दी॒क्षि॒त॒वा॒दमिति॑ दीक्षित - वा॒दम् । अ॒प॒श्य॒त् । तम् । अ॒व॒द॒त् । ततः॑ । वै । सः । प्र॒जा इति॑ प्र-जाः । अ॒सृ॒ज॒त॒ । यत् । तपः॑ । त॒प्त्वा । दी॒क्षि॒त॒वा॒दमिति॑ दीक्षित - वा॒दम् । वद॑ति । प्र॒जा इति॑ प्र - जाः । ए॒व । तत् । यज॑मानः ।  \newline


\textbf{Krama Paata} \newline

प्र॒जाप॑तिर,कामयत । प्र॒जाप॑ति॒रिति॑ प्र॒जा - प॒तिः॒ । अ॒का॒म॒य॒त॒ प्र॒जाः । प्र॒जाः सृ॑जेय । प्र॒जा इति॑ प्र - जाः । सृ॒जे॒येति॑ । इति॒ सः । स तपः॑ । तपो॑ ऽतप्यत । अ॒त॒प्य॒त॒ सः । स स॒र्पान् । स॒र्पान॑सृजत । अ॒सृ॒ज॒त॒ सः । सो॑ऽकामयत । अ॒का॒म॒य॒त॒ प्र॒जाः । प्र॒जाः सृ॑जेय । प्र॒जा इति॑ प्र - जाः । सृ॒जे॒येति॑ । इति॒ सः । स द्वि॒तीय᳚म् । द्वि॒तीय॑मतप्यत । अ॒त॒प्य॒त॒ सः । स वयाꣳ॑सि । वयाꣳ॑स्यसृजत । अ॒सृ॒ज॒त॒ सः । सो॑ऽकामयत । अ॒का॒म॒य॒त॒ प्र॒जाः । प्र॒जाः सृ॑जेय । प्र॒जा इति॑ प्र - जाः । सृ॒जे॒येति॑ । इति॒ सः । स तृ॒तीय᳚म् । तृ॒तीय॑मतप्यत । अ॒त॒प्य॒त॒ सः । स ए॒तम् । ए॒तम् दी᳚क्षितवा॒दम् । दी॒क्षि॒त॒वा॒दम॑पश्यत् । दी॒क्षि॒त॒वा॒दमिति॑ दीक्षित - वा॒दम् । अ॒प॒श्य॒त् तम् । तम॑वदत् । अ॒व॒द॒त् ततः॑ । ततो॒ वै । वै सः । स प्र॒जाः । प्र॒जा अ॑सृजत । प्र॒जा इति॑ प्र - जाः । अ॒सृ॒ज॒त॒ यत् । यत् तपः॑ । तप॑स्त॒प्त्वा । त॒प्त्वा दी᳚क्षितवा॒दम् । दी॒क्षि॒त॒वा॒दं ॅवद॑ति । दी॒क्षि॒त॒वा॒दमिति॑ दीक्षित - वा॒दम् । वद॑ति प्र॒जाः । प्र॒जा ए॒व । प्र॒जा इति॑ प्र - जाः । ए॒व तत् । तद् यज॑मानः । यज॑मानः सृजते \newline

\textbf{Jatai Paata} \newline

1. प्र॒जाप॑ति रकामयता कामयत प्र॒जाप॑तिः प्र॒जाप॑ति रकामयत । \newline
2. प्र॒जाप॑ति॒रिति॑ प्र॒जा - प॒तिः॒ । \newline
3. अ॒का॒म॒य॒त॒ प्र॒जाः प्र॒जा अ॑कामयता कामयत प्र॒जाः । \newline
4. प्र॒जाः सृ॑जेय सृजेय प्र॒जाः प्र॒जाः सृ॑जेय । \newline
5. प्र॒जा इति॑ प्र - जाः । \newline
6. सृ॒जे॒ये तीति॑ सृजेय सृजे॒ये ति॑ । \newline
7. इति॒ स स इतीति॒ सः । \newline
8. स तप॒ स्तपः॒ स स तपः॑ । \newline
9. तपो॑ ऽतप्यता तप्यत॒ तप॒ स्तपो॑ ऽतप्यत । \newline
10. अ॒त॒प्य॒त॒ स सो॑ ऽतप्यता तप्यत॒ सः । \newline
11. स स॒र्पान् थ्स॒र्पान् थ्स स स॒र्पान् । \newline
12. स॒र्पा न॑सृजता सृजत स॒र्पान् थ्स॒र्पा न॑सृजत । \newline
13. अ॒सृ॒ज॒त॒ स सो॑ ऽसृजता सृजत॒ सः । \newline
14. सो॑ ऽकामयता कामयत॒ स सो॑ ऽकामयत । \newline
15. अ॒का॒म॒य॒त॒ प्र॒जाः प्र॒जा अ॑कामयता कामयत प्र॒जाः । \newline
16. प्र॒जाः सृ॑जेय सृजेय प्र॒जाः प्र॒जाः सृ॑जेय । \newline
17. प्र॒जा इति॑ प्र - जाः । \newline
18. सृ॒जे॒ये तीति॑ सृजेय सृजे॒ये ति॑ । \newline
19. इति॒ स स इतीति॒ सः । \newline
20. स द्वि॒तीय॑म् द्वि॒तीयꣳ॒॒ स स द्वि॒तीय᳚म् । \newline
21. द्वि॒तीय॑ मतप्यता तप्यत द्वि॒तीय॑म् द्वि॒तीय॑ मतप्यत । \newline
22. अ॒त॒प्य॒त॒ स सो॑ ऽतप्यता तप्यत॒ सः । \newline
23. स वयाꣳ॑सि॒ वयाꣳ॑सि॒ स स वयाꣳ॑सि । \newline
24. वयाꣳ॑ स्यसृजता सृजत॒ वयाꣳ॑सि॒ वयाꣳ॑ स्यसृजत । \newline
25. अ॒सृ॒ज॒त॒ स सो॑ ऽसृजता सृजत॒ सः । \newline
26. सो॑ ऽकामयता कामयत॒ स सो॑ ऽकामयत । \newline
27. अ॒का॒म॒य॒त॒ प्र॒जाः प्र॒जा अ॑कामयता कामयत प्र॒जाः । \newline
28. प्र॒जाः सृ॑जेय सृजेय प्र॒जाः प्र॒जाः सृ॑जेय । \newline
29. प्र॒जा इति॑ प्र - जाः । \newline
30. सृ॒जे॒ये तीति॑ सृजेय सृजे॒ये ति॑ । \newline
31. इति॒ स स इतीति॒ सः । \newline
32. स तृ॒तीय॑म् तृ॒तीयꣳ॒॒ स स तृ॒तीय᳚म् । \newline
33. तृ॒तीय॑ मतप्यता तप्यत तृ॒तीय॑म् तृ॒तीय॑ मतप्यत । \newline
34. अ॒त॒प्य॒त॒ स सो॑ ऽतप्यता तप्यत॒ सः । \newline
35. स ए॒त मे॒तꣳ स स ए॒तम् । \newline
36. ए॒तम् दी᳚क्षितवा॒दम् दी᳚क्षितवा॒द मे॒त मे॒तम् दी᳚क्षितवा॒दम् । \newline
37. दी॒क्षि॒त॒वा॒द म॑पश्य दपश्यद् दीक्षितवा॒दम् दी᳚क्षितवा॒द म॑पश्यत् । \newline
38. दी॒क्षि॒त॒वा॒दमिति॑ दीक्षित - वा॒दम् । \newline
39. अ॒प॒श्य॒त् तम् त म॑पश्य दपश्य॒त् तम् । \newline
40. त म॑वद दवद॒त् तम् त म॑वदत् । \newline
41. अ॒व॒द॒त् तत॒ स्ततो॑ ऽवद दवद॒त् ततः॑ । \newline
42. ततो॒ वै वै तत॒ स्ततो॒ वै । \newline
43. वै स स वै वै सः । \newline
44. स प्र॒जाः प्र॒जाः स स प्र॒जाः । \newline
45. प्र॒जा अ॑सृजता सृजत प्र॒जाः प्र॒जा अ॑सृजत । \newline
46. प्र॒जा इति॑ प्र - जाः । \newline
47. अ॒सृ॒ज॒त॒ यद् यद॑सृजता सृजत॒ यत् । \newline
48. यत् तप॒ स्तपो॒ यद् यत् तपः॑ । \newline
49. तप॑ स्त॒प्त्वा त॒प्त्वा तप॒ स्तप॑ स्त॒प्त्वा । \newline
50. त॒प्त्वा दी᳚क्षितवा॒दम् दी᳚क्षितवा॒दम् त॒प्त्वा त॒प्त्वा दी᳚क्षितवा॒दम् । \newline
51. दी॒क्षि॒त॒वा॒दं ॅवद॑ति॒ वद॑ति दीक्षितवा॒दम् दी᳚क्षितवा॒दं ॅवद॑ति । \newline
52. दी॒क्षि॒त॒वा॒दमिति॑ दीक्षित - वा॒दम् । \newline
53. वद॑ति प्र॒जाः प्र॒जा वद॑ति॒ वद॑ति प्र॒जाः । \newline
54. प्र॒जा ए॒वैव प्र॒जाः प्र॒जा ए॒व । \newline
55. प्र॒जा इति॑ प्र - जाः । \newline
56. ए॒व तत् तदे॒वैव तत् । \newline
57. तद् यज॑मानो॒ यज॑मान॒ स्तत् तद् यज॑मानः । \newline
58. यज॑मानः सृजते सृजते॒ यज॑मानो॒ यज॑मानः सृजते । \newline

\textbf{Ghana Paata } \newline

1. प्र॒जाप॑ति रकामयता कामयत प्र॒जाप॑तिः प्र॒जाप॑ति रकामयत प्र॒जाः प्र॒जा अ॑कामयत प्र॒जाप॑तिः प्र॒जाप॑ति रकामयत प्र॒जाः । \newline
2. प्र॒जाप॑ति॒रिति॑ प्र॒जा - प॒तिः॒ । \newline
3. अ॒का॒म॒य॒त॒ प्र॒जाः प्र॒जा अ॑कामयता कामयत प्र॒जाः सृ॑जेय सृजेय प्र॒जा अ॑कामयता कामयत प्र॒जाः सृ॑जेय । \newline
4. प्र॒जाः सृ॑जेय सृजेय प्र॒जाः प्र॒जाः सृ॑जे॒ये तीति॑ सृजेय प्र॒जाः प्र॒जाः सृ॑जे॒ये ति॑ । \newline
5. प्र॒जा इति॑ प्र - जाः । \newline
6. सृ॒जे॒ये तीति॑ सृजेय सृजे॒ये ति॒ स स इति॑ सृजेय सृजे॒ये ति॒ सः । \newline
7. इति॒ स स इतीति॒ स तप॒ स्तपः॒ स इतीति॒ स तपः॑ । \newline
8. स तप॒ स्तपः॒ स स तपो॑ ऽतप्यता तप्यत॒ तपः॒ स स तपो॑ ऽतप्यत । \newline
9. तपो॑ ऽतप्यता तप्यत॒ तप॒ स्तपो॑ ऽतप्यत॒ स सो॑ ऽतप्यत॒ तप॒ स्तपो॑ ऽतप्यत॒ सः । \newline
10. अ॒त॒प्य॒त॒ स सो॑ ऽतप्यता तप्यत॒ स स॒र्पान् थ्स॒र्पान् थ्सो॑ ऽतप्यता तप्यत॒ स स॒र्पान् । \newline
11. स स॒र्पान् थ्स॒र्पान् थ्स स स॒र्पा न॑सृजता सृजत स॒र्पान् थ्स स स॒र्पा न॑सृजत । \newline
12. स॒र्पा न॑सृजता सृजत स॒र्पान् थ्स॒र्पा न॑सृजत॒ स सो॑ ऽसृजत स॒र्पान् थ्स॒र्पा न॑सृजत॒ सः । \newline
13. अ॒सृ॒ज॒त॒ स सो॑ ऽसृजता सृजत॒ सो॑ ऽकामयता कामयत॒ सो॑ ऽसृजता सृजत॒ सो॑ ऽकामयत । \newline
14. सो॑ ऽकामयता कामयत॒ स सो॑ ऽकामयत प्र॒जाः प्र॒जा अ॑कामयत॒ स सो॑ ऽकामयत प्र॒जाः । \newline
15. अ॒का॒म॒य॒त॒ प्र॒जाः प्र॒जा अ॑कामयता कामयत प्र॒जाः सृ॑जेय सृजेय प्र॒जा अ॑कामयता कामयत प्र॒जाः सृ॑जेय । \newline
16. प्र॒जाः सृ॑जेय सृजेय प्र॒जाः प्र॒जाः सृ॑जे॒ये तीति॑ सृजेय प्र॒जाः प्र॒जाः सृ॑जे॒ये ति॑ । \newline
17. प्र॒जा इति॑ प्र - जाः । \newline
18. सृ॒जे॒ये तीति॑ सृजेय सृजे॒ये ति॒ स स इति॑ सृजेय सृजे॒ये ति॒ सः । \newline
19. इति॒ स स इतीति॒ स द्वि॒तीय॑म् द्वि॒तीयꣳ॒॒ स इतीति॒ स द्वि॒तीय᳚म् । \newline
20. स द्वि॒तीय॑म् द्वि॒तीयꣳ॒॒ स स द्वि॒तीय॑ मतप्यता तप्यत द्वि॒तीयꣳ॒॒ स स द्वि॒तीय॑ मतप्यत । \newline
21. द्वि॒तीय॑ मतप्यता तप्यत द्वि॒तीय॑म् द्वि॒तीय॑ मतप्यत॒ स सो॑ ऽतप्यत द्वि॒तीय॑म् द्वि॒तीय॑ मतप्यत॒ सः । \newline
22. अ॒त॒प्य॒त॒ स सो॑ ऽतप्यता तप्यत॒ स वयाꣳ॑सि॒ वयाꣳ॑सि॒ सो॑ ऽतप्यता तप्यत॒ स वयाꣳ॑सि । \newline
23. स वयाꣳ॑सि॒ वयाꣳ॑सि॒ स स वयाꣳ॑ स्यसृजता सृजत॒ वयाꣳ॑सि॒ स स वयाꣳ॑ स्यसृजत । \newline
24. वयाꣳ॑ स्यसृजता सृजत॒ वयाꣳ॑सि॒ वयाꣳ॑ स्यसृजत॒ स सो॑ ऽसृजत॒ वयाꣳ॑सि॒ वयाꣳ॑ स्यसृजत॒ सः । \newline
25. अ॒सृ॒ज॒त॒ स सो॑ ऽसृजता सृजत॒ सो॑ ऽकामयता कामयत॒ सो॑ ऽसृजता सृजत॒ सो॑ ऽकामयत । \newline
26. सो॑ ऽकामयता कामयत॒ स सो॑ ऽकामयत प्र॒जाः प्र॒जा अ॑कामयत॒ स सो॑ ऽकामयत प्र॒जाः । \newline
27. अ॒का॒म॒य॒त॒ प्र॒जाः प्र॒जा अ॑कामयता कामयत प्र॒जाः सृ॑जेय सृजेय प्र॒जा अ॑कामयता कामयत प्र॒जाः सृ॑जेय । \newline
28. प्र॒जाः सृ॑जेय सृजेय प्र॒जाः प्र॒जाः सृ॑जे॒ये तीति॑ सृजेय प्र॒जाः प्र॒जाः सृ॑जे॒ये ति॑ । \newline
29. प्र॒जा इति॑ प्र - जाः । \newline
30. सृ॒जे॒ये तीति॑ सृजेय सृजे॒ये ति॒ स स इति॑ सृजेय सृजे॒ये ति॒ सः । \newline
31. इति॒ स स इतीति॒ स तृ॒तीय॑म् तृ॒तीयꣳ॒॒ स इतीति॒ स तृ॒तीय᳚म् । \newline
32. स तृ॒तीय॑म् तृ॒तीयꣳ॒॒ स स तृ॒तीय॑ मतप्यता तप्यत तृ॒तीयꣳ॒॒ स स तृ॒तीय॑ मतप्यत । \newline
33. तृ॒तीय॑ मतप्यता तप्यत तृ॒तीय॑म् तृ॒तीय॑ मतप्यत॒ स सो॑ ऽतप्यत तृ॒तीय॑म् तृ॒तीय॑ मतप्यत॒ सः । \newline
34. अ॒त॒प्य॒त॒ स सो॑ ऽतप्यता तप्यत॒ स ए॒त मे॒तꣳ सो॑ ऽतप्यता तप्यत॒ स ए॒तम् । \newline
35. स ए॒त मे॒तꣳ स स ए॒तम् दी᳚क्षितवा॒दम् दी᳚क्षितवा॒द मे॒तꣳ स स ए॒तम् दी᳚क्षितवा॒दम् । \newline
36. ए॒तम् दी᳚क्षितवा॒दम् दी᳚क्षितवा॒द मे॒त मे॒तम् दी᳚क्षितवा॒द म॑पश्य दपश्यद् दीक्षितवा॒द मे॒त मे॒तम् दी᳚क्षितवा॒द म॑पश्यत् । \newline
37. दी॒क्षि॒त॒वा॒द म॑पश्य दपश्यद् दीक्षितवा॒दम् दी᳚क्षितवा॒द म॑पश्य॒त् तम् त म॑पश्यद् दीक्षितवा॒दम् दी᳚क्षितवा॒द म॑पश्य॒त् तम् । \newline
38. दी॒क्षि॒त॒वा॒दमिति॑ दीक्षित - वा॒दम् । \newline
39. अ॒प॒श्य॒त् तम् त म॑पश्य दपश्य॒त् त म॑वद दवद॒त् त म॑पश्य दपश्य॒त् त म॑वदत् । \newline
40. त म॑वद दवद॒त् तम् त म॑वद॒त् तत॒ स्ततो॑ ऽवद॒त् तम् त म॑वद॒त् ततः॑ । \newline
41. अ॒व॒द॒त् तत॒ स्ततो॑ ऽवद दवद॒त् ततो॒ वै वै ततो॑ ऽवद दवद॒त् ततो॒ वै । \newline
42. ततो॒ वै वै तत॒ स्ततो॒ वै स स वै तत॒ स्ततो॒ वै सः । \newline
43. वै स स वै वै स प्र॒जाः प्र॒जाः स वै वै स प्र॒जाः । \newline
44. स प्र॒जाः प्र॒जाः स स प्र॒जा अ॑सृजता सृजत प्र॒जाः स स प्र॒जा अ॑सृजत । \newline
45. प्र॒जा अ॑सृजता सृजत प्र॒जाः प्र॒जा अ॑सृजत॒ यद् यद॑सृजत प्र॒जाः प्र॒जा अ॑सृजत॒ यत् । \newline
46. प्र॒जा इति॑ प्र - जाः । \newline
47. अ॒सृ॒ज॒त॒ यद् यद॑सृजता सृजत॒ यत् तप॒ स्तपो॒ यद॑सृजता सृजत॒ यत् तपः॑ । \newline
48. यत् तप॒ स्तपो॒ यद् यत् तप॑ स्त॒प्त्वा त॒प्त्वा तपो॒ यद् यत् तप॑ स्त॒प्त्वा । \newline
49. तप॑ स्त॒प्त्वा त॒प्त्वा तप॒ स्तप॑ स्त॒प्त्वा दी᳚क्षितवा॒दम् दी᳚क्षितवा॒दम् त॒प्त्वा तप॒ स्तप॑ स्त॒प्त्वा दी᳚क्षितवा॒दम् । \newline
50. त॒प्त्वा दी᳚क्षितवा॒दम् दी᳚क्षितवा॒दम् त॒प्त्वा त॒प्त्वा दी᳚क्षितवा॒दं ॅवद॑ति॒ वद॑ति दीक्षितवा॒दम् त॒प्त्वा त॒प्त्वा दी᳚क्षितवा॒दं ॅवद॑ति । \newline
51. दी॒क्षि॒त॒वा॒दं ॅवद॑ति॒ वद॑ति दीक्षितवा॒दम् दी᳚क्षितवा॒दं ॅवद॑ति प्र॒जाः प्र॒जा वद॑ति दीक्षितवा॒दम् दी᳚क्षितवा॒दं ॅवद॑ति प्र॒जाः । \newline
52. दी॒क्षि॒त॒वा॒दमिति॑ दीक्षित - वा॒दम् । \newline
53. वद॑ति प्र॒जाः प्र॒जा वद॑ति॒ वद॑ति प्र॒जा ए॒वैव प्र॒जा वद॑ति॒ वद॑ति प्र॒जा ए॒व । \newline
54. प्र॒जा ए॒वैव प्र॒जाः प्र॒जा ए॒व तत् तदे॒व प्र॒जाः प्र॒जा ए॒व तत् । \newline
55. प्र॒जा इति॑ प्र - जाः । \newline
56. ए॒व तत् तदे॒वैव तद् यज॑मानो॒ यज॑मान॒ स्तदे॒वैव तद् यज॑मानः । \newline
57. तद् यज॑मानो॒ यज॑मान॒ स्तत् तद् यज॑मानः सृजते सृजते॒ यज॑मान॒ स्तत् तद् यज॑मानः सृजते । \newline
58. यज॑मानः सृजते सृजते॒ यज॑मानो॒ यज॑मानः सृजते॒ यद् यथ् सृ॑जते॒ यज॑मानो॒ यज॑मानः सृजते॒ यत् । \newline
\pagebreak
\markright{ TS 3.1.1.2  \hfill https://www.vedavms.in \hfill}

\section{ TS 3.1.1.2 }

\textbf{TS 3.1.1.2 } \newline
\textbf{Samhita Paata} \newline

सृजते॒ यद्वै दी᳚क्षि॒तो॑ऽमे॒द्ध्यं पश्य॒त्यपा᳚स्माद्दी॒क्षाक्रा॑मति॒ नील॑मस्य॒ हरो॒ व्ये᳚त्यब॑द्धं॒ मनो॑ द॒रिद्रं॒ चक्षुः॒ सूर्यो॒ ज्योति॑षाꣳ॒॒श्रेष्ठो॒ दीक्षे॒ मा मा॑हासी॒रित्या॑ह॒ नास्मा᳚द्दी॒क्षाऽप॑क्रामति॒ नास्य॒ नीलं॒ न हरो॒ व्ये॑ति॒ यद्वै दी᳚क्षि॒तम॑भि॒वर्.ष॑तिदि॒व्या आपोऽशा᳚न्ता॒ ओजो॒ बलं॑ दी॒क्षां - [  ] \newline

\textbf{Pada Paata} \newline

सृ॒ज॒ते॒ । यत् । वै । दी॒क्षि॒तः । अ॒मे॒द्ध्यम् । पश्य॑ति । अपेति॑ । अ॒स्मा॒त् । दी॒क्षा । क्रा॒म॒ति॒ । नील᳚म् । अ॒स्य॒ । हरः॑ । वीति॑ । ए॒ति॒ । अब॑द्धम् । मनः॑ । द॒रिद्र᳚म् । चक्षुः॑ । सूर्यः॑ । ज्योति॑षाम् । श्रेष्ठः॑ । दीक्षे᳚ । मा । मा॒ । हा॒सीः॒ । इति॑ । आ॒ह॒ । न । अ॒स्मा॒त् । दी॒क्षा । अपेति॑ । क्रा॒म॒ति॒ । न । अ॒स्य॒ । नील᳚म् । न । हरः॑ । वीति॑ । ए॒ति॒ । यत् । वै । दी॒क्षि॒तम् । अ॒भि॒वर्.ष॒तीत्य॑भि-वर्.ष॑ति । दि॒व्याः । आपः॑ । अशा᳚न्ताः । ओजः॑ । बल᳚म् । दी॒क्षाम् ।  \newline


\textbf{Krama Paata} \newline

सृ॒ज॒ते॒ यत् । यद् वै । वै दी᳚क्षि॒तः । दी॒क्षि॒तो॑ऽमे॒द्ध्यम् । अ॒मे॒द्ध्यम् पश्य॑ति । पश्य॒त्यप॑ । अपा᳚स्मात् । अ॒स्मा॒द् दी॒क्षा । दी॒क्षा क्रा॑मति । क्रा॒म॒ति॒ नील᳚म् । नील॑मस्य । अ॒स्य॒ हरः॑ । हरो॒ वि । व्ये॑ति । ए॒त्यब॑द्धम् । अब॑द्ध॒म् मनः॑ । मनो॑ द॒रिद्र᳚म् । द॒रिद्र॒म् चक्षुः॑ । चक्षुः॒ सूर्यः॑ । सूर्यो॒ ज्योति॑षाम् । ज्योति॑षाꣳ॒॒ श्रेष्ठः॑ । श्रेष्ठो॒ दीक्षे᳚ । दीक्षे॒ मा । मा मा᳚ । मा॒ हा॒सीः॒ । हा॒सी॒रिति॑ । इत्या॑ह । आ॒ह॒ न । नास्मा᳚त् । अ॒स्मा॒द् दी॒क्षा । दी॒क्षाऽप॑ । अप॑ क्रामति । क्रा॒म॒ति॒ न । नास्य॑ । अ॒स्य॒ नील᳚म् । नील॒म् न । न हरः॑ । हरो॒ वि । व्ये॑ति । ए॒ति॒ यत् । यद् वै । वै दी᳚क्षि॒तम् । दी॒क्षि॒तम॑भि॒वर्.ष॑ति । अ॒भि॒वर्.ष॑ति दि॒व्याः । अ॒भि॒वर्.ष॒तीत्य॑भि - वर्.ष॑ति । दि॒व्या आपः॑ । आपोऽशा᳚न्ताः । अशा᳚न्ता॒ ओजः॑ । ओजो॒ बल᳚म् । बल॑म् दी॒क्षाम् । दी॒क्षाम् तपः॑ \newline

\textbf{Jatai Paata} \newline

1. सृ॒ज॒ते॒ यद् यथ् सृ॑जते सृजते॒ यत् । \newline
2. यद् वै वै यद् यद् वै । \newline
3. वै दी᳚क्षि॒तो दी᳚क्षि॒तो वै वै दी᳚क्षि॒तः । \newline
4. दी॒क्षि॒तो॑ ऽमे॒द्ध्य म॑मे॒द्ध्यम् दी᳚क्षि॒तो दी᳚क्षि॒तो॑ ऽमे॒द्ध्यम् । \newline
5. अ॒मे॒द्ध्यम् पश्य॑ति॒ पश्य॑ त्यमे॒द्ध्य म॑मे॒द्ध्यम् पश्य॑ति । \newline
6. पश्य॒ त्यपाप॒ पश्य॑ति॒ पश्य॒ त्यप॑ । \newline
7. अपा᳚स्मा दस्मा॒ दपापा᳚स्मात् । \newline
8. अ॒स्मा॒द् दी॒क्षा दी॒क्षा ऽस्मा॑ दस्माद् दी॒क्षा । \newline
9. दी॒क्षा क्रा॑मति क्रामति दी॒क्षा दी॒क्षा क्रा॑मति । \newline
10. क्रा॒म॒ति॒ नील॒म् नील॑म् क्रामति क्रामति॒ नील᳚म् । \newline
11. नील॑ मस्यास्य॒ नील॒म् नील॑ मस्य । \newline
12. अ॒स्य॒ हरो॒ हरो᳚ ऽस्यास्य॒ हरः॑ । \newline
13. हरो॒ वि वि हरो॒ हरो॒ वि । \newline
14. व्ये᳚त्येति॒ वि व्ये॑ति । \newline
15. ए॒त्यब॑द्ध॒ मब॑द्ध मेत्ये॒ त्यब॑द्धम् । \newline
16. अब॑द्ध॒म् मनो॒ मनो ऽब॑द्ध॒ मब॑द्ध॒म् मनः॑ । \newline
17. मनो॑ द॒रिद्र॑म् द॒रिद्र॒म् मनो॒ मनो॑ द॒रिद्र᳚म् । \newline
18. द॒रिद्र॒म् चक्षु॒ श्चक्षु॑र् द॒रिद्र॑म् द॒रिद्र॒म् चक्षुः॑ । \newline
19. चक्षुः॒ सूर्यः॒ सूर्य॒ श्चक्षु॒ श्चक्षुः॒ सूर्यः॑ । \newline
20. सूर्यो॒ ज्योति॑षा॒म् ज्योति॑षाꣳ॒॒ सूर्यः॒ सूर्यो॒ ज्योति॑षाम् । \newline
21. ज्योति॑षाꣳ॒॒ श्रेष्ठः॒ श्रेष्ठो॒ ज्योति॑षा॒म् ज्योति॑षाꣳ॒॒ श्रेष्ठः॑ । \newline
22. श्रेष्ठो॒ दीक्षे॒ दीक्षे॒ श्रेष्ठः॒ श्रेष्ठो॒ दीक्षे᳚ । \newline
23. दीक्षे॒ मा मा दीक्षे॒ दीक्षे॒ मा । \newline
24. मा मा॑ मा॒ मा मा मा᳚ । \newline
25. मा॒ हा॒सी॒र्॒. हा॒सी॒र् मा॒ मा॒ हा॒सीः॒ । \newline
26. हा॒सी॒ रितीति॑ हासीर्. हासी॒ रिति॑ । \newline
27. इत्या॑हा॒हे तीत्या॑ह । \newline
28. आ॒ह॒ न नाहा॑ह॒ न । \newline
29. नास्मा॑ दस्मा॒न् न नास्मा᳚त् । \newline
30. अ॒स्मा॒द् दी॒क्षा दी॒क्षा ऽस्मा॑ दस्माद् दी॒क्षा । \newline
31. दी॒क्षा ऽपाप॑ दी॒क्षा दी॒क्षा ऽप॑ । \newline
32. अप॑ क्रामति क्राम॒ त्यपाप॑ क्रामति । \newline
33. क्रा॒म॒ति॒ न न क्रा॑मति क्रामति॒ न । \newline
34. नास्या᳚स्य॒ न नास्य॑ । \newline
35. अ॒स्य॒ नील॒म् नील॑ मस्यास्य॒ नील᳚म् । \newline
36. नील॒म् न न नील॒म् नील॒म् न । \newline
37. न हरो॒ हरो॒ न न हरः॑ । \newline
38. हरो॒ वि वि हरो॒ हरो॒ वि । \newline
39. व्ये᳚त्येति॒ वि व्ये॑ति । \newline
40. ए॒ति॒ यद् यदे᳚त्येति॒ यत् । \newline
41. यद् वै वै यद् यद् वै । \newline
42. वै दी᳚क्षि॒तम् दी᳚क्षि॒तं ॅवै वै दी᳚क्षि॒तम् । \newline
43. दी॒क्षि॒त म॑भि॒वर्.ष॑ त्यभि॒वर्.ष॑ति दीक्षि॒तम् दी᳚क्षि॒त म॑भि॒वर्.ष॑ति । \newline
44. अ॒भि॒वर्.ष॑ति दि॒व्या दि॒व्या अ॑भि॒वर्.ष॑ त्यभि॒वर्.ष॑ति दि॒व्याः । \newline
45. अ॒भि॒वर्.ष॒तीत्य॑भि - वर्.ष॑ति । \newline
46. दि॒व्या आप॒ आपो॑ दि॒व्या दि॒व्या आपः॑ । \newline
47. आपो ऽशा᳚न्ता॒ अशा᳚न्ता॒ आप॒ आपो ऽशा᳚न्ताः । \newline
48. अशा᳚न्ता॒ ओज॒ ओजो ऽशा᳚न्ता॒ अशा᳚न्ता॒ ओजः॑ । \newline
49. ओजो॒ बल॒म् बल॒ मोज॒ ओजो॒ बल᳚म् । \newline
50. बल॑म् दी॒क्षाम् दी॒क्षाम् बल॒म् बल॑म् दी॒क्षाम् । \newline
51. दी॒क्षाम् तप॒ स्तपो॑ दी॒क्षाम् दी॒क्षाम् तपः॑ । \newline

\textbf{Ghana Paata } \newline

1. सृ॒ज॒ते॒ यद् यथ् सृ॑जते सृजते॒ यद् वै वै यथ् सृ॑जते सृजते॒ यद् वै । \newline
2. यद् वै वै यद् यद् वै दी᳚क्षि॒तो दी᳚क्षि॒तो वै यद् यद् वै दी᳚क्षि॒तः । \newline
3. वै दी᳚क्षि॒तो दी᳚क्षि॒तो वै वै दी᳚क्षि॒तो॑ ऽमे॒द्ध्य म॑मे॒द्ध्यम् दी᳚क्षि॒तो वै वै दी᳚क्षि॒तो॑ ऽमे॒द्ध्यम् । \newline
4. दी॒क्षि॒तो॑ ऽमे॒द्ध्य म॑मे॒द्ध्यम् दी᳚क्षि॒तो दी᳚क्षि॒तो॑ ऽमे॒द्ध्यम् पश्य॑ति॒ पश्य॑ त्यमे॒द्ध्यम् दी᳚क्षि॒तो दी᳚क्षि॒तो॑ ऽमे॒द्ध्यम् पश्य॑ति । \newline
5. अ॒मे॒द्ध्यम् पश्य॑ति॒ पश्य॑ त्यमे॒द्ध्य म॑मे॒द्ध्यम् पश्य॒ त्यपाप॒ पश्य॑ त्यमे॒द्ध्य म॑मे॒द्ध्यम् पश्य॒ त्यप॑ । \newline
6. पश्य॒ त्यपाप॒ पश्य॑ति॒ पश्य॒ त्यपा᳚स्मा दस्मा॒ दप॒ पश्य॑ति॒ पश्य॒ त्यपा᳚स्मात् । \newline
7. अपा᳚स्मा दस्मा॒ दपापा᳚स्माद् दी॒क्षा दी॒क्षा ऽस्मा॒ दपापा᳚स्माद् दी॒क्षा । \newline
8. अ॒स्मा॒द् दी॒क्षा दी॒क्षा ऽस्मा॑ दस्माद् दी॒क्षा क्रा॑मति क्रामति दी॒क्षा ऽस्मा॑ दस्माद् दी॒क्षा क्रा॑मति । \newline
9. दी॒क्षा क्रा॑मति क्रामति दी॒क्षा दी॒क्षा क्रा॑मति॒ नील॒म् नील॑म् क्रामति दी॒क्षा दी॒क्षा क्रा॑मति॒ नील᳚म् । \newline
10. क्रा॒म॒ति॒ नील॒म् नील॑म् क्रामति क्रामति॒ नील॑ मस्यास्य॒ नील॑म् क्रामति क्रामति॒ नील॑ मस्य । \newline
11. नील॑ मस्यास्य॒ नील॒म् नील॑ मस्य॒ हरो॒ हरो᳚ ऽस्य॒ नील॒म् नील॑ मस्य॒ हरः॑ । \newline
12. अ॒स्य॒ हरो॒ हरो᳚ ऽस्यास्य॒ हरो॒ वि वि हरो᳚ ऽस्यास्य॒ हरो॒ वि । \newline
13. हरो॒ वि वि हरो॒ हरो॒ व्ये᳚त्येति॒ वि हरो॒ हरो॒ व्ये॑ति । \newline
14. व्ये᳚त्येति॒ वि व्ये᳚त्यब॑द्ध॒ मब॑द्ध मेति॒ वि व्ये᳚त्यब॑द्धम् । \newline
15. ए॒त्यब॑द्ध॒ मब॑द्ध मेत्ये॒ त्यब॑द्ध॒म् मनो॒ मनो ऽब॑द्ध मेत्ये॒ त्यब॑द्ध॒म् मनः॑ । \newline
16. अब॑द्ध॒म् मनो॒ मनो ऽब॑द्ध॒ मब॑द्ध॒म् मनो॑ द॒रिद्र॑म् द॒रिद्र॒म् मनो ऽब॑द्ध॒ मब॑द्ध॒म् मनो॑ द॒रिद्र᳚म् । \newline
17. मनो॑ द॒रिद्र॑म् द॒रिद्र॒म् मनो॒ मनो॑ द॒रिद्र॒म् चक्षु॒ श्चक्षु॑र् द॒रिद्र॒म् मनो॒ मनो॑ द॒रिद्र॒म् चक्षुः॑ । \newline
18. द॒रिद्र॒म् चक्षु॒ श्चक्षु॑र् द॒रिद्र॑म् द॒रिद्र॒म् चक्षुः॒ सूर्यः॒ सूर्य॒ श्चक्षु॑र् द॒रिद्र॑म् द॒रिद्र॒म् चक्षुः॒ सूर्यः॑ । \newline
19. चक्षुः॒ सूर्यः॒ सूर्य॒ श्चक्षु॒ श्चक्षुः॒ सूर्यो॒ ज्योति॑षा॒म् ज्योति॑षाꣳ॒॒ सूर्य॒ श्चक्षु॒ श्चक्षुः॒ सूर्यो॒ ज्योति॑षाम् । \newline
20. सूर्यो॒ ज्योति॑षा॒म् ज्योति॑षाꣳ॒॒ सूर्यः॒ सूर्यो॒ ज्योति॑षाꣳ॒॒ श्रेष्ठः॒ श्रेष्ठो॒ ज्योति॑षाꣳ॒॒ सूर्यः॒ सूर्यो॒ ज्योति॑षाꣳ॒॒ श्रेष्ठः॑ । \newline
21. ज्योति॑षाꣳ॒॒ श्रेष्ठः॒ श्रेष्ठो॒ ज्योति॑षा॒म् ज्योति॑षाꣳ॒॒ श्रेष्ठो॒ दीक्षे॒ दीक्षे॒ श्रेष्ठो॒ ज्योति॑षा॒म् ज्योति॑षाꣳ॒॒ श्रेष्ठो॒ दीक्षे᳚ । \newline
22. श्रेष्ठो॒ दीक्षे॒ दीक्षे॒ श्रेष्ठः॒ श्रेष्ठो॒ दीक्षे॒ मा मा दीक्षे॒ श्रेष्ठः॒ श्रेष्ठो॒ दीक्षे॒ मा । \newline
23. दीक्षे॒ मा मा दीक्षे॒ दीक्षे॒ मा मा॑ मा॒ मा दीक्षे॒ दीक्षे॒ मा मा᳚ । \newline
24. मा मा॑ मा॒ मा मा मा॑ हासीर्. हासीर् मा॒ मा मा मा॑ हासीः । \newline
25. मा॒ हा॒सी॒र्॒. हा॒सी॒र् मा॒ मा॒ हा॒सी॒ रितीति॑ हासीर् मा मा हासी॒ रिति॑ । \newline
26. हा॒सी॒ रितीति॑ हासीर्. हासी॒ रित्या॑हा॒हे ति॑ हासीर्. हासी॒ रित्या॑ह । \newline
27. इत्या॑हा॒हे तीत्या॑ह॒ न नाहे तीत्या॑ह॒ न । \newline
28. आ॒ह॒ न नाहा॑ह॒ नास्मा॑ दस्मा॒न् नाहा॑ह॒ नास्मा᳚त् । \newline
29. नास्मा॑ दस्मा॒न् न नास्मा᳚द् दी॒क्षा दी॒क्षा ऽस्मा॒न् न नास्मा᳚द् दी॒क्षा । \newline
30. अ॒स्मा॒द् दी॒क्षा दी॒क्षा ऽस्मा॑ दस्माद् दी॒क्षा ऽपाप॑ दी॒क्षा ऽस्मा॑ दस्माद् दी॒क्षा ऽप॑ । \newline
31. दी॒क्षा ऽपाप॑ दी॒क्षा दी॒क्षा ऽप॑ क्रामति क्राम॒ त्यप॑ दी॒क्षा दी॒क्षा ऽप॑ क्रामति । \newline
32. अप॑ क्रामति क्राम॒ त्यपाप॑ क्रामति॒ न न क्रा॑म॒ त्यपाप॑ क्रामति॒ न । \newline
33. क्रा॒म॒ति॒ न न क्रा॑मति क्रामति॒ नास्या᳚स्य॒ न क्रा॑मति क्रामति॒ नास्य॑ । \newline
34. नास्या᳚स्य॒ न नास्य॒ नील॒म् नील॑ मस्य॒ न नास्य॒ नील᳚म् । \newline
35. अ॒स्य॒ नील॒म् नील॑ मस्यास्य॒ नील॒म् न न नील॑ मस्यास्य॒ नील॒म् न । \newline
36. नील॒म् न न नील॒म् नील॒म् न हरो॒ हरो॒ न नील॒म् नील॒म् न हरः॑ । \newline
37. न हरो॒ हरो॒ न न हरो॒ वि वि हरो॒ न न हरो॒ वि । \newline
38. हरो॒ वि वि हरो॒ हरो॒ व्ये᳚त्येति॒ वि हरो॒ हरो॒ व्ये॑ति । \newline
39. व्ये᳚त्येति॒ वि व्ये॑ति॒ यद् यदे॑ति॒ वि व्ये॑ति॒ यत् । \newline
40. ए॒ति॒ यद् यदे᳚ त्येति॒ यद् वै वै यदे᳚ त्येति॒ यद् वै । \newline
41. यद् वै वै यद् यद् वै दी᳚क्षि॒तम् दी᳚क्षि॒तं ॅवै यद् यद् वै दी᳚क्षि॒तम् । \newline
42. वै दी᳚क्षि॒तम् दी᳚क्षि॒तं ॅवै वै दी᳚क्षि॒त म॑भि॒वर्.ष॑ त्यभि॒वर्.ष॑ति दीक्षि॒तं ॅवै वै दी᳚क्षि॒त म॑भि॒वर्.ष॑ति । \newline
43. दी॒क्षि॒त म॑भि॒वर्.ष॑ त्यभि॒वर्.ष॑ति दीक्षि॒तम् दी᳚क्षि॒त म॑भि॒वर्.ष॑ति दि॒व्या दि॒व्या अ॑भि॒वर्.ष॑ति दीक्षि॒तम् दी᳚क्षि॒त म॑भि॒वर्.ष॑ति दि॒व्याः । \newline
44. अ॒भि॒वर्.ष॑ति दि॒व्या दि॒व्या अ॑भि॒वर्.ष॑ त्यभि॒वर्.ष॑ति दि॒व्या आप॒ आपो॑ दि॒व्या अ॑भि॒वर्.ष॑ त्यभि॒वर्.ष॑ति दि॒व्या आपः॑ । \newline
45. अ॒भि॒वर्.ष॒तीत्य॑भि - वर्.ष॑ति । \newline
46. दि॒व्या आप॒ आपो॑ दि॒व्या दि॒व्या आपो ऽशा᳚न्ता॒ अशा᳚न्ता॒ आपो॑ दि॒व्या दि॒व्या आपो ऽशा᳚न्ताः । \newline
47. आपो ऽशा᳚न्ता॒ अशा᳚न्ता॒ आप॒ आपो ऽशा᳚न्ता॒ ओज॒ ओजो ऽशा᳚न्ता॒ आप॒ आपो ऽशा᳚न्ता॒ ओजः॑ । \newline
48. अशा᳚न्ता॒ ओज॒ ओजो ऽशा᳚न्ता॒ अशा᳚न्ता॒ ओजो॒ बल॒म् बल॒ मोजो ऽशा᳚न्ता॒ अशा᳚न्ता॒ ओजो॒ बल᳚म् । \newline
49. ओजो॒ बल॒म् बल॒ मोज॒ ओजो॒ बल॑म् दी॒क्षाम् दी॒क्षाम् बल॒ मोज॒ ओजो॒ बल॑म् दी॒क्षाम् । \newline
50. बल॑म् दी॒क्षाम् दी॒क्षाम् बल॒म् बल॑म् दी॒क्षाम् तप॒ स्तपो॑ दी॒क्षाम् बल॒म् बल॑म् दी॒क्षाम् तपः॑ । \newline
51. दी॒क्षाम् तप॒ स्तपो॑ दी॒क्षाम् दी॒क्षाम् तपो᳚ ऽस्यास्य॒ तपो॑ दी॒क्षाम् दी॒क्षाम् तपो᳚ ऽस्य । \newline
\pagebreak
\markright{ TS 3.1.1.3  \hfill https://www.vedavms.in \hfill}

\section{ TS 3.1.1.3 }

\textbf{TS 3.1.1.3 } \newline
\textbf{Samhita Paata} \newline

तपो᳚ऽस्य॒निर्घ्न॑न्त्युन्द॒तीर् बलं॑ ध॒त्तौजो॑ धत्त॒ बलं॑ धत्त॒ मा मे॑ दी॒क्षां मा तपो॒निर्व॑धि॒ष्टेत्या॑है॒ तदे॒व सर्व॑मा॒त्मन् ध॑त्ते॒ नास्यौजो॒ बलं॒ न दी॒क्षां न तपो॒निर्घ्न॑न्त्य॒ग्निर्वै दी᳚क्षि॒तस्य॑ दे॒वता॒ सो᳚ऽस्मादे॒तर्.हि॑ति॒र इ॑व॒ यर्.हि॒ याति॒ तमी᳚श्व॒रꣳ रक्षाꣳ॑सि॒ हन्तो᳚ -  [  ] \newline

\textbf{Pada Paata} \newline

तपः॑ । अ॒स्य॒ । निरिति॑ । घ्न॒न्ति॒ । उ॒न्द॒तीः । बल᳚म् । ध॒त्त॒ । ओजः॑ । ध॒त्त॒ । बल᳚म् । ध॒त्त॒ । मा । मे॒ । दी॒क्षाम् । मा । तपः॑ । निरिति॑ । व॒धि॒ष्ट॒ । इति॑ । आ॒ह॒ । ए॒तत् । ए॒व । सर्व᳚म् । आ॒त्मन्न् । ध॒त्ते॒ । न । अ॒स्य॒ । ओजः॑ । बल᳚म् । न । दी॒क्षाम् । न । तपः॑ । निरिति॑ । घ्न॒न्ति॒ । अ॒ग्निः । वै । दी॒क्षि॒तस्य॑ । दे॒वता᳚ । सः । अ॒स्मा॒त् । ए॒तर्.हि॑ । ति॒रः । इ॒व॒ । यर्.हि॑ । याति॑ । तम् । ई॒श्व॒रम् । रक्षाꣳ॑सि । हन्तोः᳚ ।  \newline


\textbf{Krama Paata} \newline

तपो᳚ ऽस्य । अ॒स्य॒ निः । निर् घ्न॑न्ति । घ्न॒न्त्यु॒न्द॒तीः । उ॒न्द॒तीर् बल᳚म् । बल॑म् धत्त । ध॒त्तौजः॑ । ओजो॑ धत्त । ध॒त्त॒ बल᳚म् । बल॑म् धत्त । ध॒त्त॒ मा । मा मे᳚ । मे॒ दी॒क्षाम् । दी॒क्षाम् मा । मा तपः॑ । तपो॒ निः । निर् व॑धिष्ट । व॒धि॒ष्टेति॑ । इत्या॑ह । आ॒है॒तत् । ए॒तदे॒व । ए॒व सर्व᳚म् । सर्व॑मा॒त्मन्न् । आ॒त्मन् ध॑त्ते । ध॒त्ते॒ न । नास्य॑ । अ॒स्यौजः॑ । ओजो॒ बल᳚म् । बल॒म् न । न दी॒क्षाम् । दी॒क्षाम् न । न तपः॑ । तपो॒ निः । निर् घ्न॑न्ति । घ्न॒न्त्य॒ग्निः । अ॒ग्निर् वै । वै दी᳚क्षि॒तस्य॑ । दी॒क्षि॒तस्य॑ दे॒वता᳚ । दे॒वता॒ सः । सो᳚ऽस्मात् । अ॒स्मा॒दे॒तर्.हि॑ । ए॒तर्.हि॑ ति॒रः । ति॒र इ॑व । इ॒व॒ यर्.हि॑ । यर्.हि॒ याति॑ । याति॒ तम् । तमी᳚श्व॒रम् । ई॒श्व॒रꣳ रक्षाꣳ॑सि । रक्षाꣳ॑सि॒ हन्तोः᳚ । हन्तो᳚र् भ॒द्रात् \newline

\textbf{Jatai Paata} \newline

1. तपो᳚ ऽस्यास्य॒ तप॒ स्तपो᳚ ऽस्य । \newline
2. अ॒स्य॒ निर् णि र॑स्यास्य॒ निः । \newline
3. निर् घ्न॑न्ति घ्नन्ति॒ निर् णिर् घ्न॑न्ति । \newline
4. घ्न॒ न्त्यु॒न्द॒ती रु॑न्द॒तीर् घ्न॑न्ति घ्न न्त्युन्द॒तीः । \newline
5. उ॒न्द॒तीर् बल॒म् बल॑ मुन्द॒ती रु॑न्द॒तीर् बल᳚म् । \newline
6. बल॑म् धत्त धत्त॒ बल॒म् बल॑म् धत्त । \newline
7. ध॒त्तौज॒ ओजो॑ धत्त ध॒त्तौजः॑ । \newline
8. ओजो॑ धत्त ध॒त्तौज॒ ओजो॑ धत्त । \newline
9. ध॒त्त॒ बल॒म् बल॑म् धत्त धत्त॒ बल᳚म् । \newline
10. बल॑म् धत्त धत्त॒ बल॒म् बल॑म् धत्त । \newline
11. ध॒त्त॒ मा मा ध॑त्त धत्त॒ मा । \newline
12. मा मे॑ मे॒ मा मा मे᳚ । \newline
13. मे॒ दी॒क्षाम् दी॒क्षाम् मे॑ मे दी॒क्षाम् । \newline
14. दी॒क्षाम् मा मा दी॒क्षाम् दी॒क्षाम् मा । \newline
15. मा तप॒ स्तपो॒ मा मा तपः॑ । \newline
16. तपो॒ निर् णिष् टप॒ स्तपो॒ निः । \newline
17. निर् व॑धिष्ट वधिष्ट॒ निर् णिर् व॑धिष्ट । \newline
18. व॒धि॒ष्टे तीति॑ वधिष्ट वधि॒ष्टे ति॑ । \newline
19. इत्या॑हा॒हे तीत्या॑ह । \newline
20. आ॒है॒त दे॒त दा॑हा है॒तत् । \newline
21. ए॒त दे॒वैवैत दे॒त दे॒व । \newline
22. ए॒व सर्वꣳ॒॒ सर्व॑ मे॒वैव सर्व᳚म् । \newline
23. सर्व॑ मा॒त्मन् ना॒त्मन् थ्सर्वꣳ॒॒ सर्व॑ मा॒त्मन्न् । \newline
24. आ॒त्मन् ध॑त्ते धत्त आ॒त्मन् ना॒त्मन् ध॑त्ते । \newline
25. ध॒त्ते॒ न न ध॑त्ते धत्ते॒ न । \newline
26. नास्या᳚स्य॒ न नास्य॑ । \newline
27. अ॒स्यौज॒ ओजो᳚ ऽस्या॒ स्यौजः॑ । \newline
28. ओजो॒ बल॒म् बल॒ मोज॒ ओजो॒ बल᳚म् । \newline
29. बल॒म् न न बल॒म् बल॒म् न । \newline
30. न दी॒क्षाम् दी॒क्षाम् न न दी॒क्षाम् । \newline
31. दी॒क्षाम् न न दी॒क्षाम् दी॒क्षाम् न । \newline
32. न तप॒ स्तपो॒ न न तपः॑ । \newline
33. तपो॒ निर् णिष् टप॒ स्तपो॒ निः । \newline
34. निर् घ्न॑न्ति घ्नन्ति॒ निर् णिर् घ्न॑न्ति । \newline
35. घ्न॒ न्त्य॒ग्नि र॒ग्निर् घ्न॑न्ति घ्नन् त्य॒ग्निः । \newline
36. अ॒ग्निर् वै वा अ॒ग्नि र॒ग्निर् वै । \newline
37. वै दी᳚क्षि॒तस्य॑ दीक्षि॒तस्य॒ वै वै दी᳚क्षि॒तस्य॑ । \newline
38. दी॒क्षि॒तस्य॑ दे॒वता॑ दे॒वता॑ दीक्षि॒तस्य॑ दीक्षि॒तस्य॑ दे॒वता᳚ । \newline
39. दे॒वता॒ स स दे॒वता॑ दे॒वता॒ सः । \newline
40. सो᳚ ऽस्मा दस्मा॒थ् स सो᳚ ऽस्मात् । \newline
41. अ॒स्मा॒ दे॒तर् ह्ये॒तर् ह्य॑स्मा दस्मा दे॒तर्.हि॑ । \newline
42. ए॒तर्.हि॑ ति॒र स्ति॒र ए॒तर् ह्ये॒तर्.हि॑ ति॒रः । \newline
43. ति॒र इ॑वे व ति॒र स्ति॒र इ॑व । \newline
44. इ॒व॒ यर्.हि॒ यर्.ही॑वे व॒ यर्.हि॑ । \newline
45. यर्.हि॒ याति॒ याति॒ यर्.हि॒ यर्.हि॒ याति॑ । \newline
46. याति॒ तम् तं ॅयाति॒ याति॒ तम् । \newline
47. त मी᳚श्व॒र मी᳚श्व॒रम् तम् त मी᳚श्व॒रम् । \newline
48. ई॒श्व॒रꣳ रक्षाꣳ॑सि॒ रक्षाꣳ॑सीश्व॒र मी᳚श्व॒रꣳ रक्षाꣳ॑सि । \newline
49. रक्षाꣳ॑सि॒ हन्तो॒र्॒. हन्तो॒ रक्षाꣳ॑सि॒ रक्षाꣳ॑सि॒ हन्तोः᳚ । \newline
50. हन्तो᳚र् भ॒द्राद् भ॒द्रा द्धन्तो॒र्॒. हन्तो᳚र् भ॒द्रात् । \newline

\textbf{Ghana Paata } \newline

1. तपो᳚ ऽस्यास्य॒ तप॒ स्तपो᳚ ऽस्य॒ निर् णिर॑स्य॒ तप॒ स्तपो᳚ ऽस्य॒ निः । \newline
2. अ॒स्य॒ निर् णिर॑स्यास्य॒ निर् घ्न॑न्ति घ्नन्ति॒ निर॑स्यास्य॒ निर् घ्न॑न्ति । \newline
3. निर् घ्न॑न्ति घ्नन्ति॒ निर् णिर् घ्न॑ न्त्युन्द॒ती रु॑न्द॒तीर् घ्न॑न्ति॒ निर् णिर् घ्न॑ न्त्युन्द॒तीः । \newline
4. घ्न॒ न्त्यु॒न्द॒ती रु॑न्द॒तीर् घ्न॑न्ति घ्न न्त्युन्द॒तीर् बल॒म् बल॑ मुन्द॒तीर् घ्न॑न्ति घ्न न्त्युन्द॒तीर् बल᳚म् । \newline
5. उ॒न्द॒तीर् बल॒म् बल॑ मुन्द॒ती रु॑न्द॒तीर् बल॑म् धत्त धत्त॒ बल॑ मुन्द॒ती रु॑न्द॒तीर् बल॑म् धत्त । \newline
6. बल॑म् धत्त धत्त॒ बल॒म् बल॑म् ध॒त्तौज॒ ओजो॑ धत्त॒ बल॒म् बल॑म् ध॒त्तौजः॑ । \newline
7. ध॒त्तौज॒ ओजो॑ धत्त ध॒त्तौजो॑ धत्त ध॒त्तौजो॑ धत्त ध॒त्तौजो॑ धत्त । \newline
8. ओजो॑ धत्त ध॒त्तौज॒ ओजो॑ धत्त॒ बल॒म् बल॑म् ध॒त्तौज॒ ओजो॑ धत्त॒ बल᳚म् । \newline
9. ध॒त्त॒ बल॒म् बल॑म् धत्त धत्त॒ बल॑म् धत्त धत्त॒ बल॑म् धत्त धत्त॒ बल॑म् धत्त । \newline
10. बल॑म् धत्त धत्त॒ बल॒म् बल॑म् धत्त॒ मा मा ध॑त्त॒ बल॒म् बल॑म् धत्त॒ मा । \newline
11. ध॒त्त॒ मा मा ध॑त्त धत्त॒ मा मे॑ मे॒ मा ध॑त्त धत्त॒ मा मे᳚ । \newline
12. मा मे॑ मे॒ मा मा मे॑ दी॒क्षाम् दी॒क्षाम् मे॒ मा मा मे॑ दी॒क्षाम् । \newline
13. मे॒ दी॒क्षाम् दी॒क्षाम् मे॑ मे दी॒क्षाम् मा मा दी॒क्षाम् मे॑ मे दी॒क्षाम् मा । \newline
14. दी॒क्षाम् मा मा दी॒क्षाम् दी॒क्षाम् मा तप॒ स्तपो॒ मा दी॒क्षाम् दी॒क्षाम् मा तपः॑ । \newline
15. मा तप॒ स्तपो॒ मा मा तपो॒ निर् णिष् टपो॒ मा मा तपो॒ निः । \newline
16. तपो॒ निर् णिष् टप॒ स्तपो॒ निर् व॑धिष्ट वधिष्ट॒ निष् टप॒ स्तपो॒ निर् व॑धिष्ट । \newline
17. निर् व॑धिष्ट वधिष्ट॒ निर् णिर् व॑धि॒ष्टे तीति॑ वधिष्ट॒ निर् णिर् व॑धि॒ष्टे ति॑ । \newline
18. व॒धि॒ष्टे तीति॑ वधिष्ट वधि॒ष्टे त्या॑हा॒हे ति॑ वधिष्ट वधि॒ष्टे त्या॑ह । \newline
19. इत्या॑हा॒हे तीत्या॑है॒त दे॒तदा॒हे तीत्या॑है॒तत् । \newline
20. आ॒है॒त दे॒तदा॑ हाहै॒त दे॒वैवैत दा॑हाहै॒त दे॒व । \newline
21. ए॒त दे॒वैवैत दे॒त दे॒व सर्वꣳ॒॒ सर्व॑ मे॒वैत दे॒त दे॒व सर्व᳚म् । \newline
22. ए॒व सर्वꣳ॒॒ सर्व॑ मे॒वैव सर्व॑ मा॒त्मन् ना॒त्मन् थ्सर्व॑ मे॒वैव सर्व॑ मा॒त्मन्न् । \newline
23. सर्व॑ मा॒त्मन् ना॒त्मन् थ्सर्वꣳ॒॒ सर्व॑ मा॒त्मन् ध॑त्ते धत्त आ॒त्मन् थ्सर्वꣳ॒॒ सर्व॑ मा॒त्मन् ध॑त्ते । \newline
24. आ॒त्मन् ध॑त्ते धत्त आ॒त्मन् ना॒त्मन् ध॑त्ते॒ न न ध॑त्त आ॒त्मन् ना॒त्मन् ध॑त्ते॒ न । \newline
25. ध॒त्ते॒ न न ध॑त्ते धत्ते॒ नास्या᳚स्य॒ न ध॑त्ते धत्ते॒ नास्य॑ । \newline
26. नास्या᳚स्य॒ न नास्यौज॒ ओजो᳚ ऽस्य॒ न नास्यौजः॑ । \newline
27. अ॒स्यौज॒ ओजो᳚ ऽस्या॒ स्यौजो॒ बल॒म् बल॒ मोजो᳚ ऽस्या॒ स्यौजो॒ बल᳚म् । \newline
28. ओजो॒ बल॒म् बल॒ मोज॒ ओजो॒ बल॒न्न न बल॒ मोज॒ ओजो॒ बल॒न्न । \newline
29. बल॒न्न न बल॒म् बल॒म् न दी॒क्षाम् दी॒क्षाम् न बल॒म् बल॒न्न दी॒क्षाम् । \newline
30. न दी॒क्षाम् दी॒क्षाम् न न दी॒क्षाम् न न दी॒क्षाम् न न दी॒क्षाम् न । \newline
31. दी॒क्षाम् न न दी॒क्षाम् दी॒क्षाम् न तप॒ स्तपो॒ न दी॒क्षाम् दी॒क्षाम् न तपः॑ । \newline
32. न तप॒ स्तपो॒ न न तपो॒ निर् णिष् टपो॒ न न तपो॒ निः । \newline
33. तपो॒ निर् णिष् टप॒ स्तपो॒ निर् घ्न॑न्ति घ्नन्ति॒ निष् टप॒ स्तपो॒ निर् घ्न॑न्ति । \newline
34. निर् घ्न॑न्ति घ्नन्ति॒ निर् णिर् घ्न॑ न्त्य॒ग्नि र॒ग्निर् घ्न॑न्ति॒ निर् णिर् घ्न॑ न्त्य॒ग्निः । \newline
35. घ्न॒ न्त्य॒ग्नि र॒ग्निर् घ्न॑न्ति घ्न न्त्य॒ग्निर् वै वा अ॒ग्निर् घ्न॑न्ति घ्न न्त्य॒ग्निर् वै । \newline
36. अ॒ग्निर् वै वा अ॒ग्नि र॒ग्निर् वै दी᳚क्षि॒तस्य॑ दीक्षि॒तस्य॒ वा अ॒ग्नि र॒ग्निर् वै दी᳚क्षि॒तस्य॑ । \newline
37. वै दी᳚क्षि॒तस्य॑ दीक्षि॒तस्य॒ वै वै दी᳚क्षि॒तस्य॑ दे॒वता॑ दे॒वता॑ दीक्षि॒तस्य॒ वै वै दी᳚क्षि॒तस्य॑ दे॒वता᳚ । \newline
38. दी॒क्षि॒तस्य॑ दे॒वता॑ दे॒वता॑ दीक्षि॒तस्य॑ दीक्षि॒तस्य॑ दे॒वता॒ स स दे॒वता॑ दीक्षि॒तस्य॑ दीक्षि॒तस्य॑ दे॒वता॒ सः । \newline
39. दे॒वता॒ स स दे॒वता॑ दे॒वता॒ सो᳚ ऽस्मा दस्मा॒थ् स दे॒वता॑ दे॒वता॒ सो᳚ ऽस्मात् । \newline
40. सो᳚ ऽस्मा दस्मा॒थ् स सो᳚ ऽस्मा दे॒तर् ह्ये॒तर् ह्य॑स्मा॒थ् स सो᳚ ऽस्मा दे॒तर्.हि॑ । \newline
41. अ॒स्मा॒ दे॒तर् ह्ये॒तर् ह्य॑स्मा दस्मा दे॒तर्.हि॑ ति॒र स्ति॒र ए॒तर् ह्य॑स्मा दस्मा दे॒तर्.हि॑ ति॒रः । \newline
42. ए॒तर्.हि॑ ति॒र स्ति॒र ए॒तर् ह्ये॒तर्.हि॑ ति॒र इ॑वे व ति॒र ए॒तर् ह्ये॒तर्.हि॑ ति॒र इ॑व । \newline
43. ति॒र इ॑वे व ति॒र स्ति॒र इ॑व॒ यर्.हि॒ यर्.ही॑व ति॒र स्ति॒र इ॑व॒ यर्.हि॑ । \newline
44. इ॒व॒ यर्.हि॒ यर्.ही॑वे व॒ यर्.हि॒ याति॒ याति॒ यर्.ही॑वे व॒ यर्.हि॒ याति॑ । \newline
45. यर्.हि॒ याति॒ याति॒ यर्.हि॒ यर्.हि॒ याति॒ तम् तं ॅयाति॒ यर्.हि॒ यर्.हि॒ याति॒ तम् । \newline
46. याति॒ तम् तं ॅयाति॒ याति॒ त मी᳚श्व॒र मी᳚श्व॒रम् तं ॅयाति॒ याति॒ त मी᳚श्व॒रम् । \newline
47. त मी᳚श्व॒र मी᳚श्व॒रम् तम् त मी᳚श्व॒रꣳ रक्षाꣳ॑सि॒ रक्षाꣳ॑ सीश्व॒रम् तम् त मी᳚श्व॒रꣳ रक्षाꣳ॑सि । \newline
48. ई॒श्व॒रꣳ रक्षाꣳ॑सि॒ रक्षाꣳ॑ सीश्व॒र मी᳚श्व॒रꣳ रक्षाꣳ॑सि॒ हन्तो॒र्॒. हन्तो॒ रक्षाꣳ॑ सीश्व॒र मी᳚श्व॒रꣳ रक्षाꣳ॑सि॒ हन्तोः᳚ । \newline
49. रक्षाꣳ॑सि॒ हन्तो॒र्॒. हन्तो॒ रक्षाꣳ॑सि॒ रक्षाꣳ॑सि॒ हन्तो᳚र् भ॒द्राद् भ॒द्रा द्धन्तो॒ रक्षाꣳ॑सि॒ रक्षाꣳ॑सि॒ हन्तो᳚र् भ॒द्रात् । \newline
50. हन्तो᳚र् भ॒द्राद् भ॒द्रा द्धन्तो॒र्॒. हन्तो᳚र् भ॒द्रा द॒भ्य॑भि भ॒द्रा द्धन्तो॒र्॒. हन्तो᳚र् भ॒द्रा द॒भि । \newline
\pagebreak
\markright{ TS 3.1.1.4  \hfill https://www.vedavms.in \hfill}

\section{ TS 3.1.1.4 }

\textbf{TS 3.1.1.4 } \newline
\textbf{Samhita Paata} \newline

र्भ॒द्राद॒भिश्रेयः॒ प्रेहि॒बृह॒स्पतिः॑ पुर ए॒ता ते॑ अ॒स्त्वित्या॑ह॒ब्रह्म॒ वै दे॒वानां॒ बृह॒स्पति॒स्तमे॒वान्वा र॑भते॒ स ए॑नꣳ॒॒ सं पा॑रय॒त्ये दम॑गन्म देव॒यज॑नं पृथि॒व्या इत्या॑ह देव॒यज॑नꣳ॒॒ ह्ये॑ष पृ॑थि॒व्या आ॒गच्छ॑ति॒ यो यज॑ते॒ विश्वे॑ दे॒वा यदजु॑षन्त॒ पूर्व॒ इत्या॑ह॒ विश्वे॒ ह्ये॑तद्दे॒वा जो॒षय॑न्ते॒ यद्ब्रा᳚ह्म॒णा ( ) ऋ॑ख्सा॒माभ्यां॒ ॅयजु॑षा स॒न्तर॑न्त॒ इत्या॑हर्ख्सा॒माभ्याꣳ॒॒ ह्ये॑ष यजु॑षा स॒न्तर॑ति॒ यो यज॑ते रा॒यस्पोषे॑ण॒ समि॒षा-म॑दे॒मेत्या॑-हा॒ऽशिष॑मे॒वै तामा शा᳚स्ते ॥ \newline

\textbf{Pada Paata} \newline

भ॒द्रात् । अ॒भीति॑ । श्रेयः॑ । प्रेति॑ । इ॒हि॒ । बृह॒स्पतिः॑ । पु॒र॒ ए॒तेति॑ पुरः - ए॒ता । ते॒ । अ॒स्तु॒ । इति॑ । आ॒ह॒ । ब्रह्म॑ । वै । दे॒वाना᳚म् । बृह॒स्पतिः॑ । तम् । ए॒व । अ॒न्वार॑भत॒ इत्य॑नु - आर॑भते । सः । ए॒न॒म् । समिति॑ । पा॒र॒य॒ति॒ । एति॑ । इ॒दम् । अ॒ग॒न्म॒ । दे॒व॒यज॑न॒मिति॑ देव - यज॑नम् । पृ॒थि॒व्याः । इति॑ । आ॒ह॒ । दे॒व॒यज॑न॒मिति॑ देव - यज॑नम् । हि । ए॒षः । पृ॒थि॒व्याः । आ॒गच्छ॒तीत्या᳚ - गच्छ॑ति । यः । यज॑ते । विश्वे᳚ । दे॒वाः । यत् । अजु॑षन्त । पूर्वे᳚ । इति॑ । आ॒ह॒ । विश्वे᳚ । हि । ए॒तत् । दे॒वाः । जो॒षय॑न्ते । यत् । ब्रा॒ह्म॒णाः ( ) । ऋ॒ख्सा॒माभ्या॒मित्यृ॑ख्सा॒मा -भ्या॒म् । यजु॑षा । स॒न्तर॑न्त॒ इति॑ सं - तर॑न्तः । इति॑ । आ॒ह॒ । ऋ॒ख्सा॒माभ्या॒मित्यृ॑ख्सा॒मा - भ्या॒म् । हि । ए॒षः । यजु॑षा । स॒न्तर॒तीति॑ सं - तर॑ति । यः । यज॑ते । रा॒यः । पोषे॑ण । समिति॑ । इ॒षा । म॒दे॒म॒ । इति॑ । आ॒ह॒ । आ॒शिष॒मित्या᳚ - शिष᳚म् । ए॒व । ए॒ताम् । एति॑ । शा॒स्ते॒ ॥  \newline


\textbf{Krama Paata} \newline

भ॒द्राद॒भि । अ॒भि श्रेयः॑ । श्रेयः॒ प्र । प्रेहि॑ । इ॒हि॒ बृह॒स्पतिः॑ । बृह॒स्पतिः॑ पुरए॒ता । पु॒र॒ए॒ता ते᳚ । पु॒र॒ए॒तेति॑ पुरः - ए॒ता । ते॒ अ॒स्तु॒ । अ॒स्त्विति॑ । इत्या॑ह । आ॒ह॒ ब्रह्म॑ । ब्रह्म॒ वै । वै दे॒वाना᳚म् । दे॒वाना॒म् बृह॒स्पतिः॑ । बृह॒स्पति॒स्तम् । तमे॒व । ए॒वान्वार॑भते । अ॒न्वार॑भते॒ सः । अ॒न्वार॑भत॒ इत्य॑नु - आर॑भते । स ए॑नम् । ए॒नꣳ॒॒ सम् । सम् पा॑रयति । पा॒र॒य॒त्या । एदम् । इ॒दम॑गन्म । अ॒ग॒न्म॒ दे॒व॒यज॑नम् । दे॒व॒यज॑नम् पृथि॒व्याः । दे॒व॒यज॑न॒मिति॑ देव - यज॑नम् । पृ॒थि॒व्या इति॑ । इत्या॑ह । आ॒ह॒ दे॒व॒यज॑नम् । दे॒व॒यज॑नꣳ॒॒ हि । दे॒व॒यज॑न॒मिति॑ देव - यज॑नम् । ह्ये॑षः । ए॒ष पृ॑थि॒व्याः । पृ॒थि॒व्या आ॒गच्छ॑ति । आ॒गच्छ॑ति॒ यः । आ॒गच्छ॒तीत्या᳚ - गच्छ॑ति । यो यज॑ते । यज॑ते॒ विश्वे᳚ । विश्वे॑ दे॒वाः । दे॒वा यत् । यदजु॑षन्त । अजु॑षन्त॒ पूर्वे᳚ । पूर्व॒ इति॑ । इत्या॑ह । आ॒ह॒ विश्वे᳚ । विश्वे॒ हि । ह्ये॑तत् । ए॒त॒द् दे॒वाः । दे॒वा जो॒षय॑न्ते । जो॒षय॑न्ते॒ यत् । यद् ब्रा᳚ह्म॒णाः ( ) । ब्रा॒ह्म॒णा ऋ॑ख्सा॒माभ्या᳚म् । ऋ॒ख्सा॒माभ्यां॒ ॅयजु॑षा । ऋ॒ख्सा॒माभ्या॒मित्यृ॑ख्सा॒म - भ्या॒म् । यजु॑षा स॒न्तर॑न्तः । स॒न्तर॑न्त॒ इति॑ । स॒न्तर॑न्त॒ इति॑ सम् - तर॑न्तः । इत्या॑ह । आ॒ह॒र्ख्सा॒माभ्या᳚म् । ऋ॒ख्सा॒माभ्याꣳ॒॒ हि । ऋ॒ख्सा॒माभ्या॒मित्यृ॑ख्सा॒म - भ्या॒म् । ह्ये॑षः । ए॒ष यजु॑षा । यजु॑षा स॒न्तर॑ति । स॒न्तर॑ति॒ यः । स॒न्तर॒तीति॑ सम् - तर॑ति । यो यज॑ते । यज॑ते रा॒यः । रा॒यस्पोषे॑ण । पोषे॑ण॒ सम् । समि॒षा । इ॒षा म॑देम । म॒दे॒मेति॑ । इत्या॑ह । आ॒हा॒शिष᳚म् । आ॒शिष॑मे॒व । आ॒शिष॒मित्या᳚ - शिष᳚म् । ए॒वैताम् । ए॒तामा । आ शा᳚स्ते । शा॒स्त॒ इति॑ शास्ते । \newline

\textbf{Jatai Paata} \newline

1. भ॒द्रा द॒भ्य॑भि भ॒द्राद् भ॒द्रा द॒भि । \newline
2. अ॒भि श्रेयः॒ श्रेयो॒ ऽभ्य॑भि श्रेयः॑ । \newline
3. श्रेयः॒ प्र प्र श्रेयः॒ श्रेयः॒ प्र । \newline
4. प्रे ही॑हि॒ प्र प्रे हि॑ । \newline
5. इ॒हि॒ बृह॒स्पति॒र् बृह॒स्पति॑ रिहीहि॒ बृह॒स्पतिः॑ । \newline
6. बृह॒स्पतिः॑ पुरए॒ता पु॑रए॒ता बृह॒स्पति॒र् बृह॒स्पतिः॑ पुरए॒ता । \newline
7. पु॒र॒ए॒ता ते॑ ते पुरए॒ता पु॑रए॒ता ते᳚ । \newline
8. पु॒र॒ए॒तेति॑ पुरः - ए॒ता । \newline
9. ते॒ अ॒स्त्व॒स्तु॒ ते॒ ते॒ अ॒स्तु॒ । \newline
10. अ॒स्त्वि तीत्य॑ स्त्व॒ स्त्विति॑ । \newline
11. इत्या॑हा॒हे तीत्या॑ह । \newline
12. आ॒ह॒ ब्रह्म॒ ब्रह्मा॑हाह॒ ब्रह्म॑ । \newline
13. ब्रह्म॒ वै वै ब्रह्म॒ ब्रह्म॒ वै । \newline
14. वै दे॒वाना᳚म् दे॒वानां॒ ॅवै वै दे॒वाना᳚म् । \newline
15. दे॒वाना॒म् बृह॒स्पति॒र् बृह॒स्पति॑र् दे॒वाना᳚म् दे॒वाना॒म् बृह॒स्पतिः॑ । \newline
16. बृह॒स्पति॒ स्तम् तम् बृह॒स्पति॒र् बृह॒स्पति॒ स्तम् । \newline
17. त मे॒वैव तम् त मे॒व । \newline
18. ए॒वा न्वार॑भते॒ ऽन्वार॑भत ए॒वैवा न्वार॑भते । \newline
19. अ॒न्वार॑भते॒ स सो᳚ ऽन्वार॑भते॒ ऽन्वार॑भते॒ सः । \newline
20. अ॒न्वार॑भत॒ इत्य॑नु - आर॑भते । \newline
21. स ए॑न मेनꣳ॒॒ स स ए॑नम् । \newline
22. ए॒नꣳ॒॒ सꣳ स मे॑न मेनꣳ॒॒ सम् । \newline
23. सम् पा॑रयति पारयति॒ सꣳ सम् पा॑रयति । \newline
24. पा॒र॒य॒त्या पा॑रयति पारय॒त्या । \newline
25. एद मि॒द मेदम् । \newline
26. इ॒द म॑गन्मा गन्मे॒ द मि॒द म॑गन्म । \newline
27. अ॒ग॒न्म॒ दे॒व॒यज॑नम् देव॒यज॑न मगन्मा गन्म देव॒यज॑नम् । \newline
28. दे॒व॒यज॑नम् पृथि॒व्याः पृ॑थि॒व्या दे॑व॒यज॑नम् देव॒यज॑नम् पृथि॒व्याः । \newline
29. दे॒व॒यज॑न॒मिति॑ देव - यज॑नम् । \newline
30. पृ॒थि॒व्या इतीति॑ पृथि॒व्याः पृ॑थि॒व्या इति॑ । \newline
31. इत्या॑हा॒हे तीत्या॑ह । \newline
32. आ॒ह॒ दे॒व॒यज॑नम् देव॒यज॑न माहाह देव॒यज॑नम् । \newline
33. दे॒व॒यज॑नꣳ॒॒ हि हि दे॑व॒यज॑नम् देव॒यज॑नꣳ॒॒ हि । \newline
34. दे॒व॒यज॑न॒मिति॑ देव - यज॑नम् । \newline
35. ह्ये॑ष ए॒ष हि ह्ये॑षः । \newline
36. ए॒ष पृ॑थि॒व्याः पृ॑थि॒व्या ए॒ष ए॒ष पृ॑थि॒व्याः । \newline
37. पृ॒थि॒व्या आ॒गच्छ॑ त्या॒गच्छ॑ति पृथि॒व्याः पृ॑थि॒व्या आ॒गच्छ॑ति । \newline
38. आ॒गच्छ॑ति॒ यो य आ॒गच्छ॑ त्या॒गच्छ॑ति॒ यः । \newline
39. आ॒गच्छ॒तीत्या᳚ - गच्छ॑ति । \newline
40. यो यज॑ते॒ यज॑ते॒ यो यो यज॑ते । \newline
41. यज॑ते॒ विश्वे॒ विश्वे॒ यज॑ते॒ यज॑ते॒ विश्वे᳚ । \newline
42. विश्वे॑ दे॒वा दे॒वा विश्वे॒ विश्वे॑ दे॒वाः । \newline
43. दे॒वा यद् यद् दे॒वा दे॒वा यत् । \newline
44. यदजु॑ष॒न्ता जु॑षन्त॒ यद् यदजु॑षन्त । \newline
45. अजु॑षन्त॒ पूर्वे॒ पूर्वे ऽजु॑ष॒न्ता जु॑षन्त॒ पूर्वे᳚ । \newline
46. पूर्व॒ इतीति॒ पूर्वे॒ पूर्व॒ इति॑ । \newline
47. इत्या॑हा॒हे तीत्या॑ह । \newline
48. आ॒ह॒ विश्वे॒ विश्व॑ आहाह॒ विश्वे᳚ । \newline
49. विश्वे॒ हि हि विश्वे॒ विश्वे॒ हि । \newline
50. ह्ये॑त दे॒त द्धि ह्ये॑तत् । \newline
51. ए॒तद् दे॒वा दे॒वा ए॒त दे॒तद् दे॒वाः । \newline
52. दे॒वा जो॒षय॑न्ते जो॒षय॑न्ते दे॒वा दे॒वा जो॒षय॑न्ते । \newline
53. जो॒षय॑न्ते॒ यद् यज् जो॒षय॑न्ते जो॒षय॑न्ते॒ यत् । \newline
54. यद् ब्रा᳚ह्म॒णा ब्रा᳚ह्म॒णा यद् यद् ब्रा᳚ह्म॒णाः । \newline
55. ब्रा॒ह्म॒णा ऋ॑ख्सा॒माभ्या॑ मृख्सा॒माभ्या᳚म् ब्राह्म॒णा ब्रा᳚ह्म॒णा ऋ॑ख्सा॒माभ्या᳚म् । \newline
56. ऋ॒ख्सा॒माभ्यां॒ ॅयजु॑षा॒ यजु॑ष र्‌ख्सा॒माभ्या॑ मृख्सा॒माभ्यां॒ ॅयजु॑षा । \newline
57. ऋ॒ख्सा॒माभ्या॒मित्यृ॑ख्सा॒म - भ्या॒म् । \newline
58. यजु॑षा स॒न्तर॑न्तः स॒न्तर॑न्तो॒ यजु॑षा॒ यजु॑षा स॒न्तर॑न्तः । \newline
59. स॒न्तर॑न्त॒ इतीति॑ स॒न्तर॑न्तः स॒न्तर॑न्त॒ इति॑ । \newline
60. स॒न्तर॑न्त॒ इति॑ सं - तर॑न्तः । \newline
61. इत्या॑हा॒हे तीत्या॑ह । \newline
62. आ॒ह॒ र्‌ख्सा॒माभ्या॑ मृख्सा॒माभ्या॑ माहाह र्‌ख्सा॒माभ्या᳚म् । \newline
63. ऋ॒ख्सा॒माभ्याꣳ॒॒ हि ह्यृ॑ख्सा॒माभ्या॑ मृख्सा॒माभ्याꣳ॒॒ हि । \newline
64. ऋ॒ख्सा॒माभ्या॒मित्यृ॑ख्सा॒म - भ्या॒म् । \newline
65. ह्ये॑ष ए॒ष हि ह्ये॑षः । \newline
66. ए॒ष यजु॑षा॒ यजु॑षै॒ष ए॒ष यजु॑षा । \newline
67. यजु॑षा स॒न्तर॑ति स॒न्तर॑ति॒ यजु॑षा॒ यजु॑षा स॒न्तर॑ति । \newline
68. स॒न्तर॑ति॒ यो यः स॒न्तर॑ति स॒न्तर॑ति॒ यः । \newline
69. स॒न्तर॒तीति॑ सं - तर॑ति । \newline
70. यो यज॑ते॒ यज॑ते॒ यो यो यज॑ते । \newline
71. यज॑ते रा॒यो रा॒यो यज॑ते॒ यज॑ते रा॒यः । \newline
72. रा॒य स्पोषे॑ण॒ पोषे॑ण रा॒यो रा॒य स्पोषे॑ण । \newline
73. पोषे॑ण॒ सꣳ सम् पोषे॑ण॒ पोषे॑ण॒ सम् । \newline
74. स मि॒षेषा सꣳ स मि॒षा । \newline
75. इ॒षा म॑देम मदेमे॒ षेषा म॑देम । \newline
76. म॒दे॒मे तीति॑ मदेम मदे॒मे ति॑ । \newline
77. इत्या॑हा॒हे तीत्या॑ह । \newline
78. आ॒हा॒ शिष॑ मा॒शिष॑ माहा हा॒शिष᳚म् । \newline
79. आ॒शिष॑ मे॒वै वाशिष॑ मा॒शिष॑ मे॒व । \newline
80. आ॒शिष॒मित्या᳚ - शिष᳚म् । \newline
81. ए॒वैता मे॒ता मे॒वै वैताम् । \newline
82. ए॒ता मैता मे॒ता मा । \newline
83. आ शा᳚स्ते शास्त॒ आ शा᳚स्ते । \newline
84. शा॒स्त॒ इति॑ शास्ते । \newline

\textbf{Ghana Paata } \newline

1. भ॒द्रा द॒भ्य॑भि भ॒द्राद् भ॒द्रा द॒भि श्रेयः॒ श्रेयो॒ ऽभि भ॒द्राद् भ॒द्रा द॒भि श्रेयः॑ । \newline
2. अ॒भि श्रेयः॒ श्रेयो॒ ऽभ्य॑भि श्रेयः॒ प्र प्र श्रेयो॒ ऽभ्य॑भि श्रेयः॒ प्र । \newline
3. श्रेयः॒ प्र प्र श्रेयः॒ श्रेयः॒ प्रे ही॑हि॒ प्र श्रेयः॒ श्रेयः॒ प्रे हि॑ । \newline
4. प्रे ही॑हि॒ प्र प्रे हि॒ बृह॒स्पति॒र् बृह॒स्पति॑ रिहि॒ प्र प्रे हि॒ बृह॒स्पतिः॑ । \newline
5. इ॒हि॒ बृह॒स्पति॒र् बृह॒स्पति॑ रिहीहि॒ बृह॒स्पतिः॑ पुरए॒ता पु॑रए॒ता बृह॒स्पति॑ रिहीहि॒ बृह॒स्पतिः॑ पुरए॒ता । \newline
6. बृह॒स्पतिः॑ पुरए॒ता पु॑रए॒ता बृह॒स्पति॒र् बृह॒स्पतिः॑ पुरए॒ता ते॑ ते पुरए॒ता बृह॒स्पति॒र् बृह॒स्पतिः॑ पुरए॒ता ते᳚ । \newline
7. पु॒र॒ए॒ता ते॑ ते पुरए॒ता पु॑रए॒ता ते॑ अस्त्वस्तु ते पुरए॒ता पु॑रए॒ता ते॑ अस्तु । \newline
8. पु॒र॒ए॒तेति॑ पुरः - ए॒ता । \newline
9. ते॒ अ॒स्त्व॒स्तु॒ ते॒ ते॒ अ॒स्त्विती त्य॑स्तु ते ते अ॒स्त्विति॑ । \newline
10. अ॒स्त्विती त्य॑स्त्व॒स्त्वि त्या॑हा॒हे त्य॑स्त्व॒स्त्वि त्या॑ह । \newline
11. इत्या॑हा॒हे तीत्या॑ह॒ ब्रह्म॒ ब्रह्मा॒हे तीत्या॑ह॒ ब्रह्म॑ । \newline
12. आ॒ह॒ ब्रह्म॒ ब्रह्मा॑हाह॒ ब्रह्म॒ वै वै ब्रह्मा॑हाह॒ ब्रह्म॒ वै । \newline
13. ब्रह्म॒ वै वै ब्रह्म॒ ब्रह्म॒ वै दे॒वाना᳚म् दे॒वानां॒ ॅवै ब्रह्म॒ ब्रह्म॒ वै दे॒वाना᳚म् । \newline
14. वै दे॒वाना᳚म् दे॒वानां॒ ॅवै वै दे॒वाना॒म् बृह॒स्पति॒र् बृह॒स्पति॑र् दे॒वानां॒ ॅवै वै दे॒वाना॒म् बृह॒स्पतिः॑ । \newline
15. दे॒वाना॒म् बृह॒स्पति॒र् बृह॒स्पति॑र् दे॒वाना᳚म् दे॒वाना॒म् बृह॒स्पति॒ स्तम् तम् बृह॒स्पति॑र् दे॒वाना᳚म् दे॒वाना॒म् बृह॒स्पति॒ स्तम् । \newline
16. बृह॒स्पति॒ स्तम् तम् बृह॒स्पति॒र् बृह॒स्पति॒ स्त मे॒वैव तम् बृह॒स्पति॒र् बृह॒स्पति॒ स्त मे॒व । \newline
17. त मे॒वैव तम् त मे॒वा न्वार॑भते॒ ऽन्वार॑भत ए॒व तम् त मे॒वा न्वार॑भते । \newline
18. ए॒वा न्वार॑भते॒ ऽन्वार॑भत ए॒वैवान्वा र॑भते॒ स सो᳚ ऽन्वार॑भत ए॒वैवा न्वार॑भते॒ सः । \newline
19. अ॒न्वार॑भते॒ स सो᳚ ऽन्वार॑भते॒ ऽन्वार॑भते॒ स ए॑न मेनꣳ॒॒ सो᳚ ऽन्वार॑भते॒ ऽन्वार॑भते॒ स ए॑नम् । \newline
20. अ॒न्वार॑भत॒ इत्य॑नु - आर॑भते । \newline
21. स ए॑न मेनꣳ॒॒ स स ए॑नꣳ॒॒ सꣳ स मे॑नꣳ॒॒ स स ए॑नꣳ॒॒ सम् । \newline
22. ए॒नꣳ॒॒ सꣳ स मे॑न मेनꣳ॒॒ सम् पा॑रयति पारयति॒ स मे॑न मेनꣳ॒॒ सम् पा॑रयति । \newline
23. सम् पा॑रयति पारयति॒ सꣳ सम् पा॑रय॒त्या पा॑रयति॒ सꣳ सम् पा॑रय॒त्या । \newline
24. पा॒र॒य॒त्या पा॑रयति पारय॒ त्येद मि॒द मा पा॑रयति पारय॒ त्येदम् । \newline
25. एद मि॒द मेद म॑गन्मा गन्मे॒ द मेद म॑गन्म । \newline
26. इ॒द म॑गन्मा गन्मे॒ द मि॒द म॑गन्म देव॒यज॑नम् देव॒यज॑न मगन्मे॒ द मि॒द म॑गन्म देव॒यज॑नम् । \newline
27. अ॒ग॒न्म॒ दे॒व॒यज॑नम् देव॒यज॑न मगन्मा गन्म देव॒यज॑नम् पृथि॒व्याः पृ॑थि॒व्या दे॑व॒यज॑न मगन्मा गन्म देव॒यज॑नम् पृथि॒व्याः । \newline
28. दे॒व॒यज॑नम् पृथि॒व्याः पृ॑थि॒व्या दे॑व॒यज॑नम् देव॒यज॑नम् पृथि॒व्या इतीति॑ पृथि॒व्या दे॑व॒यज॑नम् देव॒यज॑नम् पृथि॒व्या इति॑ । \newline
29. दे॒व॒यज॑न॒मिति॑ देव - यज॑नम् । \newline
30. पृ॒थि॒व्या इतीति॑ पृथि॒व्याः पृ॑थि॒व्या इत्या॑हा॒हे ति॑ पृथि॒व्याः पृ॑थि॒व्या इत्या॑ह । \newline
31. इत्या॑हा॒हे तीत्या॑ह देव॒यज॑नम् देव॒यज॑न मा॒हे तीत्या॑ह देव॒यज॑नम् । \newline
32. आ॒ह॒ दे॒व॒यज॑नम् देव॒यज॑न माहाह देव॒यज॑नꣳ॒॒ हि हि दे॑व॒यज॑न माहाह देव॒यज॑नꣳ॒॒ हि । \newline
33. दे॒व॒यज॑नꣳ॒॒ हि हि दे॑व॒यज॑नम् देव॒यज॑नꣳ॒॒ ह्ये॑ष ए॒ष हि दे॑व॒यज॑नम् देव॒यज॑नꣳ॒॒ ह्ये॑षः । \newline
34. दे॒व॒यज॑न॒मिति॑ देव - यज॑नम् । \newline
35. ह्ये॑ष ए॒ष हि ह्ये॑ष पृ॑थि॒व्याः पृ॑थि॒व्या ए॒ष हि ह्ये॑ष पृ॑थि॒व्याः । \newline
36. ए॒ष पृ॑थि॒व्याः पृ॑थि॒व्या ए॒ष ए॒ष पृ॑थि॒व्या आ॒गच्छ॑ त्या॒गच्छ॑ति पृथि॒व्या ए॒ष ए॒ष पृ॑थि॒व्या आ॒गच्छ॑ति । \newline
37. पृ॒थि॒व्या आ॒गच्छ॑ त्या॒गच्छ॑ति पृथि॒व्याः पृ॑थि॒व्या आ॒गच्छ॑ति॒ यो य आ॒गच्छ॑ति पृथि॒व्याः पृ॑थि॒व्या आ॒गच्छ॑ति॒ यः । \newline
38. आ॒गच्छ॑ति॒ यो य आ॒गच्छ॑ त्या॒गच्छ॑ति॒ यो यज॑ते॒ यज॑ते॒ य आ॒गच्छ॑ त्या॒गच्छ॑ति॒ यो यज॑ते । \newline
39. आ॒गच्छ॒तीत्या᳚ - गच्छ॑ति । \newline
40. यो यज॑ते॒ यज॑ते॒ यो यो यज॑ते॒ विश्वे॒ विश्वे॒ यज॑ते॒ यो यो यज॑ते॒ विश्वे᳚ । \newline
41. यज॑ते॒ विश्वे॒ विश्वे॒ यज॑ते॒ यज॑ते॒ विश्वे॑ दे॒वा दे॒वा विश्वे॒ यज॑ते॒ यज॑ते॒ विश्वे॑ दे॒वाः । \newline
42. विश्वे॑ दे॒वा दे॒वा विश्वे॒ विश्वे॑ दे॒वा यद् यद् दे॒वा विश्वे॒ विश्वे॑ दे॒वा यत् । \newline
43. दे॒वा यद् यद् दे॒वा दे॒वा यदजु॑ष॒न्ता जु॑षन्त॒ यद् दे॒वा दे॒वा यदजु॑षन्त । \newline
44. यदजु॑ष॒न्ता जु॑षन्त॒ यद् यदजु॑षन्त॒ पूर्वे॒ पूर्वे ऽजु॑षन्त॒ यद् यदजु॑षन्त॒ पूर्वे᳚ । \newline
45. अजु॑षन्त॒ पूर्वे॒ पूर्वे ऽजु॑ष॒न्ता जु॑षन्त॒ पूर्व॒ इतीति॒ पूर्वे ऽजु॑ष॒न्ता जु॑षन्त॒ पूर्व॒ इति॑ । \newline
46. पूर्व॒ इतीति॒ पूर्वे॒ पूर्व॒ इत्या॑हा॒हे ति॒ पूर्वे॒ पूर्व॒ इत्या॑ह । \newline
47. इत्या॑हा॒हे तीत्या॑ह॒ विश्वे॒ विश्व॑ आ॒हे तीत्या॑ह॒ विश्वे᳚ । \newline
48. आ॒ह॒ विश्वे॒ विश्व॑ आहाह॒ विश्वे॒ हि हि विश्व॑ आहाह॒ विश्वे॒ हि । \newline
49. विश्वे॒ हि हि विश्वे॒ विश्वे॒ ह्ये॑त दे॒तद्धि विश्वे॒ विश्वे॒ ह्ये॑तत् । \newline
50. ह्ये॑त दे॒तद्धि ह्ये॑तद् दे॒वा दे॒वा ए॒तद्धि ह्ये॑तद् दे॒वाः । \newline
51. ए॒तद् दे॒वा दे॒वा ए॒त दे॒तद् दे॒वा जो॒षय॑न्ते जो॒षय॑न्ते दे॒वा ए॒त दे॒तद् दे॒वा जो॒षय॑न्ते । \newline
52. दे॒वा जो॒षय॑न्ते जो॒षय॑न्ते दे॒वा दे॒वा जो॒षय॑न्ते॒ यद् यज् जो॒षय॑न्ते दे॒वा दे॒वा जो॒षय॑न्ते॒ यत् । \newline
53. जो॒षय॑न्ते॒ यद् यज् जो॒षय॑न्ते जो॒षय॑न्ते॒ यद् ब्रा᳚ह्म॒णा ब्रा᳚ह्म॒णा यज् जो॒षय॑न्ते जो॒षय॑न्ते॒ यद् ब्रा᳚ह्म॒णाः । \newline
54. यद् ब्रा᳚ह्म॒णा ब्रा᳚ह्म॒णा यद् यद् ब्रा᳚ह्म॒णा ऋ॑ख्सा॒माभ्या॑ मृख्सा॒माभ्या᳚म् ब्राह्म॒णा यद् यद् ब्रा᳚ह्म॒णा ऋ॑ख्सा॒माभ्या᳚म् । \newline
55. ब्रा॒ह्म॒णा ऋ॑ख्सा॒माभ्या॑ मृख्सा॒माभ्या᳚म् ब्राह्म॒णा ब्रा᳚ह्म॒णा ऋ॑ख्सा॒माभ्यां॒ ॅयजु॑षा॒ 
यजु॑ष र्‌ख्सा॒माभ्या᳚म् ब्राह्म॒णा ब्रा᳚ह्म॒णा ऋ॑ख्सा॒माभ्यां॒ ॅयजु॑षा । \newline
56. ऋ॒ख्सा॒माभ्यां॒ ॅयजु॑षा॒ यजु॑ष र्‌ख्सा॒माभ्या॑ मृख्सा॒माभ्यां॒ ॅयजु॑षा स॒न्तर॑न्तः स॒न्तर॑न्तो॒ यजु॑ष र्‌ख्सा॒माभ्या॑ मृख्सा॒माभ्यां॒ ॅयजु॑षा स॒न्तर॑न्तः । \newline
57. ऋ॒ख्सा॒माभ्या॒मित्यृ॑ख्सा॒म - भ्या॒म् । \newline
58. यजु॑षा स॒न्तर॑न्तः स॒न्तर॑न्तो॒ यजु॑षा॒ यजु॑षा स॒न्तर॑न्त॒ इतीति॑ स॒न्तर॑न्तो॒ यजु॑षा॒ यजु॑षा स॒न्तर॑न्त॒ इति॑ । \newline
59. स॒न्तर॑न्त॒ इतीति॑ स॒न्तर॑न्तः स॒न्तर॑न्त॒ इत्या॑हा॒हे ति॑ स॒न्तर॑न्तः स॒न्तर॑न्त॒ इत्या॑ह । \newline
60. स॒न्तर॑न्त॒ इति॑ सं - तर॑न्तः । \newline
61. इत्या॑हा॒हे तीत्या॑ह र्‌ख्सा॒माभ्या॑ मृख्सा॒माभ्या॑ मा॒हे तीत्या॑ह र्‌ख्सा॒माभ्या᳚म् । \newline
62. आ॒ह॒ र्‌ख्सा॒माभ्या॑ मृख्सा॒माभ्या॑ माहाह र्‌ख्सा॒माभ्याꣳ॒॒ हि ह्यृ॑ख्सा॒माभ्या॑ माहाह र्‌ख्सा॒माभ्याꣳ॒॒ हि । \newline
63. ऋ॒ख्सा॒माभ्याꣳ॒॒ हि ह्यृ॑ख्सा॒माभ्या॑ मृख्सा॒माभ्याꣳ॒॒ ह्ये॑ष ए॒ष ह्यृ॑ख्सा॒माभ्या॑ मृख्सा॒माभ्याꣳ॒॒ ह्ये॑षः । \newline
64. ऋ॒ख्सा॒माभ्या॒मित्यृ॑ख्सा॒म - भ्या॒म् । \newline
65. ह्ये॑ष ए॒ष हि ह्ये॑ष यजु॑षा॒ यजु॑षै॒ष हि ह्ये॑ष यजु॑षा । \newline
66. ए॒ष यजु॑षा॒ यजु॑षै॒ष ए॒ष यजु॑षा स॒न्तर॑ति स॒न्तर॑ति॒ यजु॑षै॒ष ए॒ष यजु॑षा स॒न्तर॑ति । \newline
67. यजु॑षा स॒न्तर॑ति स॒न्तर॑ति॒ यजु॑षा॒ यजु॑षा स॒न्तर॑ति॒ यो यः स॒न्तर॑ति॒ यजु॑षा॒ यजु॑षा स॒न्तर॑ति॒ यः । \newline
68. स॒न्तर॑ति॒ यो यः स॒न्तर॑ति स॒न्तर॑ति॒ यो यज॑ते॒ यज॑ते॒ यः स॒न्तर॑ति स॒न्तर॑ति॒ यो यज॑ते । \newline
69. स॒न्तर॒तीति॑ सं - तर॑ति । \newline
70. यो यज॑ते॒ यज॑ते॒ यो यो यज॑ते रा॒यो रा॒यो यज॑ते॒ यो यो यज॑ते रा॒यः । \newline
71. यज॑ते रा॒यो रा॒यो यज॑ते॒ यज॑ते रा॒य स्पोषे॑ण॒ पोषे॑ण रा॒यो यज॑ते॒ यज॑ते रा॒य स्पोषे॑ण । \newline
72. रा॒य स्पोषे॑ण॒ पोषे॑ण रा॒यो रा॒य स्पोषे॑ण॒ सꣳ सम् पोषे॑ण रा॒यो रा॒य स्पोषे॑ण॒ सम् । \newline
73. पोषे॑ण॒ सꣳ सम् पोषे॑ण॒ पोषे॑ण॒ स मि॒षेषा सम् पोषे॑ण॒ पोषे॑ण॒ स मि॒षा । \newline
74. स मि॒षेषा सꣳ स मि॒षा म॑देम मदेमे॒ षा सꣳ स मि॒षा म॑देम । \newline
75. इ॒षा म॑देम मदेमे॒ षेषा म॑दे॒मे तीति॑ मदेमे॒ षेषा म॑दे॒मे ति॑ । \newline
76. म॒दे॒मे तीति॑ मदेम मदे॒मे त्या॑हा॒हे ति॑ मदेम मदे॒मे त्या॑ह । \newline
77. इत्या॑हा॒हे तीत्या॑ हा॒शिष॑ मा॒शिष॑ मा॒हे तीत्या॑हा॒शिष᳚म् । \newline
78. आ॒हा॒शिष॑ मा॒शिष॑ माहा हा॒शिष॑ मे॒वैवाशिष॑ माहा हा॒शिष॑ मे॒व । \newline
79. आ॒शिष॑ मे॒वैवाशिष॑ मा॒शिष॑ मे॒वैता मे॒ता मे॒वाशिष॑ मा॒शिष॑ मे॒वैताम् । \newline
80. आ॒शिष॒मित्या᳚ - शिष᳚म् । \newline
81. ए॒वैता मे॒ता मे॒वैवैता मैता मे॒वैवैता मा । \newline
82. ए॒ता मैता मे॒ता मा शा᳚स्ते शास्त॒ ऐता मे॒ता मा शा᳚स्ते । \newline
83. आ शा᳚स्ते शास्त॒ आ शा᳚स्ते । \newline
84. शा॒स्त॒ इति॑ शास्ते । \newline
\pagebreak
\markright{ TS 3.1.2.1  \hfill https://www.vedavms.in \hfill}

\section{ TS 3.1.2.1 }

\textbf{TS 3.1.2.1 } \newline
\textbf{Samhita Paata} \newline

ए॒ष ते॑ गाय॒त्रो भा॒ग इति॑ मे॒ सोमा॑य ब्रूतादे॒ष ते॒ त्रैष्टु॑भो॒ जाग॑तो भा॒ग इति॑ मे॒ सोमा॑य ब्रूताच्छन्दो॒मानाꣳ॒॒ साम्रा᳚ज्यं ग॒च्छेति॑ मे॒ सोमा॑य ब्रूता॒द् यो वै सोमꣳ॒॒ राजा॑नꣳ॒॒ साम्रा᳚ज्यं ॅलो॒कं ग॑मयि॒त्वा क्री॒णाति॒ गच्छ॑ति॒ स्वानाꣳ॒॒ साम्रा᳚ज्यं॒ छन्दाꣳ॑सि॒ खलु॒ वै सोम॑स्य॒ राज्ञ्ः॒ साम्रा᳚ज्यो लो॒कः पु॒रस्ता॒थ् सोम॑स्य क्र॒यादे॒वम॒भि म॑न्त्रयेत॒ साम्रा᳚ज्यमे॒वै - [  ] \newline

\textbf{Pada Paata} \newline

ए॒षः । ते॒ । गा॒य॒त्रः । भा॒गः । इति॑ । मे॒ । सोमा॑य । ब्रू॒ता॒त् । ए॒षः । ते॒ । त्रैष्टु॑भः । जाग॑तः । भा॒गः । इति॑ । मे॒ । सोमा॑य । ब्रू॒ता॒त् । छ॒न्दो॒माना॒मिति॑ छन्दः - माना᳚म् । साम्रा᳚ज्य॒मिति॒ साम् - रा॒ज्य॒म् । ग॒च्छ॒ । इति॑ । मे॒ । सोमा॑य । ब्रू॒ता॒त् । यः । वै । सोम᳚म् । राजा॑नम् । साम्रा᳚ज्य॒मिति॒ साम् - रा॒ज्य॒म् । लो॒कम् । ग॒म॒यि॒त्वा । क्री॒णाति॑ । गच्छ॑ति । स्वाना᳚म् । साम्रा᳚ज्य॒मिति॒ साम् - रा॒ज्य॒म् । छन्दाꣳ॑सि । खलु॑ । वै । सोम॑स्य । राज्ञ्ः॑ । साम्रा᳚ज्य॒ इति॒ साम् - रा॒ज्यः॒ । लो॒कः । पु॒रस्ता᳚त् । सोम॑स्य । क्र॒यात् । ए॒वम् । अ॒भीति॑ । म॒न्त्र॒ये॒त॒ । साम्रा᳚ज्य॒मिति॒ साम् - रा॒ज्य॒म् । ए॒व ।  \newline


\textbf{Krama Paata} \newline

ए॒ष ते᳚ । ते॒ गा॒य॒त्रः । गा॒य॒त्रो भा॒गः । भा॒ग इति॑ । इति॑ मे । मे॒ 
सोमा॑य । सोमा॑य ब्रूतात् । ब्रू॒ता॒दे॒षः । ए॒ष ते᳚ । 
ते॒ त्रैष्टु॑भः । त्रैष्टु॑भो॒ जाग॑तः । जाग॑तो भा॒गः । 
भा॒ग इति॑ । इति॑ मे । मे॒ सोमा॑य । सोमा॑य ब्रूतात् । ब्रू॒ता॒च्छ॒न्दो॒माना᳚म् । छ॒न्दो॒मानाꣳ॒॒ साम्रा᳚ज्यम् । छ॒न्दो॒माना॒मिति॑ छन्दः - माना᳚म् । साम्रा᳚ज्यम् गच्छ । साम्रा᳚ज्य॒मिति॒ साम् - रा॒ज्य॒म् । ग॒च्छेति॑ । इति॑ मे । मे॒ सोमा॑य । सोमा॑य ब्रूतात् । ब्रू॒ता॒द् यः । यो वै । वै सोम᳚म् । 
सोमꣳ॒॒ राजा॑नम् । राजा॑नꣳ॒॒ साम्रा᳚ज्यम् । साम्रा᳚ज्यं ॅलो॒कम् । साम्रा᳚ज्य॒मिति॒ साम् - रा॒ज्य॒म् । लो॒कम् ग॑मयि॒त्वा । ग॒म॒यि॒त्वा क्री॒णाति॑ । क्री॒णाति॒ गच्छ॑ति । गच्छ॑ति॒ स्वाना᳚म् । स्वानाꣳ॒॒ साम्रा᳚ज्यम् । साम्रा᳚ज्य॒म् छन्दाꣳ॑सि । साम्रा᳚ज्य॒मिति॒ साम् - रा॒ज्य॒म् । छन्दाꣳ॑सि॒ खलु॑ । खलु॒ वै । वै सोम॑स्य । सोम॑स्य॒ राज्ञ्ः॑ । राज्ञ्ः॒ साम्रा᳚ज्यः । साम्रा᳚ज्यो लो॒कः । साम्रा᳚ज्य॒ इति॒ साम् - रा॒ज्यः॒ । लो॒कः पु॒रस्ता᳚त् । पु॒रस्ता॒थ् सोम॑स्य । सोम॑स्य क्र॒यात् । क्र॒यादे॒वम् । ए॒वम॒भि । 
अ॒भि म॑न्त्रयेत । म॒न्त्र॒ये॒त॒ साम्रा᳚ज्यम् । साम्रा᳚ज्यमे॒व । साम्रा᳚ज्य॒मिति॒ साम् - रा॒ज्य॒म् । ए॒वैन᳚म् \newline

\textbf{Jatai Paata} \newline

1. ए॒ष ते॑ त ए॒ष ए॒ष ते᳚ । \newline
2. ते॒ गा॒य॒त्रो गा॑य॒त्र स्ते॑ ते गाय॒त्रः । \newline
3. गा॒य॒त्रो भा॒गो भा॒गो गा॑य॒त्रो गा॑य॒त्रो भा॒गः । \newline
4. भा॒ग इतीति॑ भा॒गो भा॒ग इति॑ । \newline
5. इति॑ मे म॒ इतीति॑ मे । \newline
6. मे॒ सोमा॑य॒ सोमा॑य मे मे॒ सोमा॑य । \newline
7. सोमा॑य ब्रूताद् ब्रूता॒थ् सोमा॑य॒ सोमा॑य ब्रूतात् । \newline
8. ब्रू॒ता॒ दे॒ष ए॒ष ब्रू॑ताद् ब्रूता दे॒षः । \newline
9. ए॒ष ते॑ त ए॒ष ए॒ष ते᳚ । \newline
10. ते॒ त्रैष्टु॑भ॒ स्त्रैष्टु॑भ स्ते ते॒ त्रैष्टु॑भः । \newline
11. त्रैष्टु॑भो॒ जाग॑तो॒ जाग॑त॒ स्त्रैष्टु॑भ॒ स्त्रैष्टु॑भो॒ जाग॑तः । \newline
12. जाग॑तो भा॒गो भा॒गो जाग॑तो॒ जाग॑तो भा॒गः । \newline
13. भा॒ग इतीति॑ भा॒गो भा॒ग इति॑ । \newline
14. इति॑ मे म॒ इतीति॑ मे । \newline
15. मे॒ सोमा॑य॒ सोमा॑य मे मे॒ सोमा॑य । \newline
16. सोमा॑य ब्रूताद् ब्रूता॒थ् सोमा॑य॒ सोमा॑य ब्रूतात् । \newline
17. ब्रू॒ता॒च् छ॒न्दो॒माना᳚म् छन्दो॒माना᳚म् ब्रूताद् ब्रूताच् छन्दो॒माना᳚म् । \newline
18. छ॒न्दो॒मानाꣳ॒॒ साम्रा᳚ज्यꣳ॒॒ साम्रा᳚ज्यम् छन्दो॒माना᳚म् छन्दो॒मानाꣳ॒॒ साम्रा᳚ज्यम् । \newline
19. छ॒न्दो॒माना॒मिति॑ छन्दः - माना᳚म् । \newline
20. साम्रा᳚ज्यम् गच्छ गच्छ॒ साम्रा᳚ज्यꣳ॒॒ साम्रा᳚ज्यम् गच्छ । \newline
21. साम्रा᳚ज्य॒मिति॒ साम् - रा॒ज्य॒म् । \newline
22. ग॒च्छे तीति॑ गच्छ ग॒च्छे ति॑ । \newline
23. इति॑ मे म॒ इतीति॑ मे । \newline
24. मे॒ सोमा॑य॒ सोमा॑य मे मे॒ सोमा॑य । \newline
25. सोमा॑य ब्रूताद् ब्रूता॒थ् सोमा॑य॒ सोमा॑य ब्रूतात् । \newline
26. ब्रू॒ता॒द् यो यो ब्रू॑ताद् ब्रूता॒द् यः । \newline
27. यो वै वै यो यो वै । \newline
28. वै सोमꣳ॒॒ सोमं॒ ॅवै वै सोम᳚म् । \newline
29. सोमꣳ॒॒ राजा॑नꣳ॒॒ राजा॑नꣳ॒॒ सोमꣳ॒॒ सोमꣳ॒॒ राजा॑नम् । \newline
30. राजा॑नꣳ॒॒ साम्रा᳚ज्यꣳ॒॒ साम्रा᳚ज्यꣳ॒॒ राजा॑नꣳ॒॒ राजा॑नꣳ॒॒ साम्रा᳚ज्यम् । \newline
31. साम्रा᳚ज्यम् ॅलो॒कम् ॅलो॒कꣳ साम्रा᳚ज्यꣳ॒॒ साम्रा᳚ज्यम् ॅलो॒कम् । \newline
32. साम्रा᳚ज्य॒मिति॒ साम् - रा॒ज्य॒म् । \newline
33. लो॒कम् ग॑मयि॒त्वा ग॑मयि॒त्वा लो॒कम् ॅलो॒कम् ग॑मयि॒त्वा । \newline
34. ग॒म॒यि॒त्वा क्री॒णाति॑ क्री॒णाति॑ गमयि॒त्वा ग॑मयि॒त्वा क्री॒णाति॑ । \newline
35. क्री॒णाति॒ गच्छ॑ति॒ गच्छ॑ति क्री॒णाति॑ क्री॒णाति॒ गच्छ॑ति । \newline
36. गच्छ॑ति॒ स्वानाꣳ॒॒ स्वाना॒म् गच्छ॑ति॒ गच्छ॑ति॒ स्वाना᳚म् । \newline
37. स्वानाꣳ॒॒ साम्रा᳚ज्यꣳ॒॒ साम्रा᳚ज्यꣳ॒॒ स्वानाꣳ॒॒ स्वानाꣳ॒॒ साम्रा᳚ज्यम् । \newline
38. साम्रा᳚ज्य॒म् छन्दाꣳ॑सि॒ छन्दाꣳ॑सि॒ साम्रा᳚ज्यꣳ॒॒ साम्रा᳚ज्य॒म् छन्दाꣳ॑सि । \newline
39. साम्रा᳚ज्य॒मिति॒ साम् - रा॒ज्य॒म् । \newline
40. छन्दाꣳ॑सि॒ खलु॒ खलु॒ छन्दाꣳ॑सि॒ छन्दाꣳ॑सि॒ खलु॑ । \newline
41. खलु॒ वै वै खलु॒ खलु॒ वै । \newline
42. वै सोम॑स्य॒ सोम॑स्य॒ वै वै सोम॑स्य । \newline
43. सोम॑स्य॒ राज्ञो॒ राज्ञ्ः॒ सोम॑स्य॒ सोम॑स्य॒ राज्ञ्ः॑ । \newline
44. राज्ञ्ः॒ साम्रा᳚ज्यः॒ साम्रा᳚ज्यो॒ राज्ञो॒ राज्ञ्ः॒ साम्रा᳚ज्यः । \newline
45. साम्रा᳚ज्यो लो॒को लो॒कः साम्रा᳚ज्यः॒ साम्रा᳚ज्यो लो॒कः । \newline
46. साम्रा᳚ज्य॒ इति॒ साम् - रा॒ज्यः॒ । \newline
47. लो॒कः पु॒रस्ता᳚त् पु॒रस्ता᳚ ल्लो॒को लो॒कः पु॒रस्ता᳚त् । \newline
48. पु॒रस्ता॒थ् सोम॑स्य॒ सोम॑स्य पु॒रस्ता᳚त् पु॒रस्ता॒थ् सोम॑स्य । \newline
49. सोम॑स्य क्र॒यात् क्र॒याथ् सोम॑स्य॒ सोम॑स्य क्र॒यात् । \newline
50. क्र॒या दे॒व मे॒वम् क्र॒यात् क्र॒या दे॒वम् । \newline
51. ए॒व म॒भ्या᳚(1॒)भ्ये॑व मे॒व म॒भि । \newline
52. अ॒भि म॑न्त्रयेत मन्त्रयेता॒ भ्य॑भि म॑न्त्रयेत । \newline
53. म॒न्त्र॒ये॒त॒ साम्रा᳚ज्यꣳ॒॒ साम्रा᳚ज्यम् मन्त्रयेत मन्त्रयेत॒ साम्रा᳚ज्यम् । \newline
54. साम्रा᳚ज्य मे॒वैव साम्रा᳚ज्यꣳ॒॒ साम्रा᳚ज्य मे॒व । \newline
55. साम्रा᳚ज्य॒मिति॒ साम् - रा॒ज्य॒म् । \newline
56. ए॒वैन॑ मेन मे॒वैवैन᳚म् । \newline

\textbf{Ghana Paata } \newline

1. ए॒ष ते॑ त ए॒ष ए॒ष ते॑ गाय॒त्रो गा॑य॒त्र स्त॑ ए॒ष ए॒ष ते॑ गाय॒त्रः । \newline
2. ते॒ गा॒य॒त्रो गा॑य॒त्र स्ते॑ ते गाय॒त्रो भा॒गो भा॒गो गा॑य॒त्र स्ते॑ ते गाय॒त्रो भा॒गः । \newline
3. गा॒य॒त्रो भा॒गो भा॒गो गा॑य॒त्रो गा॑य॒त्रो भा॒ग इतीति॑ भा॒गो गा॑य॒त्रो गा॑य॒त्रो भा॒ग इति॑ । \newline
4. भा॒ग इतीति॑ भा॒गो भा॒ग इति॑ मे म॒ इति॑ भा॒गो भा॒ग इति॑ मे । \newline
5. इति॑ मे म॒ इतीति॑ मे॒ सोमा॑य॒ सोमा॑य म॒ इतीति॑ मे॒ सोमा॑य । \newline
6. मे॒ सोमा॑य॒ सोमा॑य मे मे॒ सोमा॑य ब्रूताद् ब्रूता॒थ् सोमा॑य मे मे॒ सोमा॑य ब्रूतात् । \newline
7. सोमा॑य ब्रूताद् ब्रूता॒थ् सोमा॑य॒ सोमा॑य ब्रूतादे॒ष ए॒ष ब्रू॑ता॒थ् सोमा॑य॒ सोमा॑य ब्रूतादे॒षः । \newline
8. ब्रू॒ता॒ दे॒ष ए॒ष ब्रू॑ताद् ब्रूता दे॒ष ते॑ त ए॒ष ब्रू॑ताद् ब्रूता दे॒ष ते᳚ । \newline
9. ए॒ष ते॑ त ए॒ष ए॒ष ते॒ त्रैष्टु॑भ॒ स्त्रैष्टु॑भ स्त ए॒ष ए॒ष ते॒ त्रैष्टु॑भः । \newline
10. ते॒ त्रैष्टु॑भ॒ स्त्रैष्टु॑भ स्ते ते॒ त्रैष्टु॑भो॒ जाग॑तो॒ जाग॑त॒ स्त्रैष्टु॑भ स्ते ते॒ त्रैष्टु॑भो॒ जाग॑तः । \newline
11. त्रैष्टु॑भो॒ जाग॑तो॒ जाग॑त॒ स्त्रैष्टु॑भ॒ स्त्रैष्टु॑भो॒ जाग॑तो भा॒गो भा॒गो जाग॑त॒ स्त्रैष्टु॑भ॒ स्त्रैष्टु॑भो॒ जाग॑तो भा॒गः । \newline
12. जाग॑तो भा॒गो भा॒गो जाग॑तो॒ जाग॑तो भा॒ग इतीति॑ भा॒गो जाग॑तो॒ जाग॑तो भा॒ग इति॑ । \newline
13. भा॒ग इतीति॑ भा॒गो भा॒ग इति॑ मे म॒ इति॑ भा॒गो भा॒ग इति॑ मे । \newline
14. इति॑ मे म॒ इतीति॑ मे॒ सोमा॑य॒ सोमा॑य म॒ इतीति॑ मे॒ सोमा॑य । \newline
15. मे॒ सोमा॑य॒ सोमा॑य मे मे॒ सोमा॑य ब्रूताद् ब्रूता॒थ् सोमा॑य मे मे॒ सोमा॑य ब्रूतात् । \newline
16. सोमा॑य ब्रूताद् ब्रूता॒थ् सोमा॑य॒ सोमा॑य ब्रूताच् छन्दो॒माना᳚म् छन्दो॒माना᳚म् ब्रूता॒थ् सोमा॑य॒ सोमा॑य ब्रूताच् छन्दो॒माना᳚म् । \newline
17. ब्रू॒ता॒च् छ॒न्दो॒माना᳚म् छन्दो॒माना᳚म् ब्रूताद् ब्रूताच् छन्दो॒मानाꣳ॒॒ साम्रा᳚ज्यꣳ॒॒ साम्रा᳚ज्यम् छन्दो॒माना᳚म् ब्रूताद् ब्रूताच् छन्दो॒मानाꣳ॒॒ साम्रा᳚ज्यम् । \newline
18. छ॒न्दो॒मानाꣳ॒॒ साम्रा᳚ज्यꣳ॒॒ साम्रा᳚ज्यम् छन्दो॒माना᳚म् छन्दो॒मानाꣳ॒॒ साम्रा᳚ज्यम् गच्छ गच्छ॒ साम्रा᳚ज्यम् छन्दो॒माना᳚म् छन्दो॒मानाꣳ॒॒ साम्रा᳚ज्यम् गच्छ । \newline
19. छ॒न्दो॒माना॒मिति॑ छन्दः - माना᳚म् । \newline
20. साम्रा᳚ज्यम् गच्छ गच्छ॒ साम्रा᳚ज्यꣳ॒॒ साम्रा᳚ज्यम् ग॒च्छे तीति॑ गच्छ॒ साम्रा᳚ज्यꣳ॒॒ साम्रा᳚ज्यम् ग॒च्छे ति॑ । \newline
21. साम्रा᳚ज्य॒मिति॒ साम् - रा॒ज्य॒म् । \newline
22. ग॒च्छे तीति॑ गच्छ ग॒च्छे ति॑ मे म॒ इति॑ गच्छ ग॒च्छे ति॑ मे । \newline
23. इति॑ मे म॒ इतीति॑ मे॒ सोमा॑य॒ सोमा॑य म॒ इतीति॑ मे॒ सोमा॑य । \newline
24. मे॒ सोमा॑य॒ सोमा॑य मे मे॒ सोमा॑य ब्रूताद् ब्रूता॒थ् सोमा॑य मे मे॒ सोमा॑य ब्रूतात् । \newline
25. सोमा॑य ब्रूताद् ब्रूता॒थ् सोमा॑य॒ सोमा॑य ब्रूता॒द् यो यो ब्रू॑ता॒थ् सोमा॑य॒ सोमा॑य ब्रूता॒द् यः । \newline
26. ब्रू॒ता॒द् यो यो ब्रू॑ताद् ब्रूता॒द् यो वै वै यो ब्रू॑ताद् ब्रूता॒द् यो वै । \newline
27. यो वै वै यो यो वै सोमꣳ॒॒ सोमं॒ ॅवै यो यो वै सोम᳚म् । \newline
28. वै सोमꣳ॒॒ सोमं॒ ॅवै वै सोमꣳ॒॒ राजा॑नꣳ॒॒ राजा॑नꣳ॒॒ सोमं॒ ॅवै वै सोमꣳ॒॒ राजा॑नम् । \newline
29. सोमꣳ॒॒ राजा॑नꣳ॒॒ राजा॑नꣳ॒॒ सोमꣳ॒॒ सोमꣳ॒॒ राजा॑नꣳ॒॒ साम्रा᳚ज्यꣳ॒॒ साम्रा᳚ज्यꣳ॒॒ राजा॑नꣳ॒॒ सोमꣳ॒॒ सोमꣳ॒॒ राजा॑नꣳ॒॒ साम्रा᳚ज्यम् । \newline
30. राजा॑नꣳ॒॒ साम्रा᳚ज्यꣳ॒॒ साम्रा᳚ज्यꣳ॒॒ राजा॑नꣳ॒॒ राजा॑नꣳ॒॒ साम्रा᳚ज्यम् ॅलो॒कम् ॅलो॒कꣳ साम्रा᳚ज्यꣳ॒॒ राजा॑नꣳ॒॒ राजा॑नꣳ॒॒ साम्रा᳚ज्यम् ॅलो॒कम् । \newline
31. साम्रा᳚ज्यम् ॅलो॒कम् ॅलो॒कꣳ साम्रा᳚ज्यꣳ॒॒ साम्रा᳚ज्यम् ॅलो॒कम् ग॑मयि॒त्वा ग॑मयि॒त्वा लो॒कꣳ साम्रा᳚ज्यꣳ॒॒ साम्रा᳚ज्यम् ॅलो॒कम् ग॑मयि॒त्वा । \newline
32. साम्रा᳚ज्य॒मिति॒ साम् - रा॒ज्य॒म् । \newline
33. लो॒कम् ग॑मयि॒त्वा ग॑मयि॒त्वा लो॒कम् ॅलो॒कम् ग॑मयि॒त्वा क्री॒णाति॑ क्री॒णाति॑ गमयि॒त्वा लो॒कम् ॅलो॒कम् ग॑मयि॒त्वा क्री॒णाति॑ । \newline
34. ग॒म॒यि॒त्वा क्री॒णाति॑ क्री॒णाति॑ गमयि॒त्वा ग॑मयि॒त्वा क्री॒णाति॒ गच्छ॑ति॒ गच्छ॑ति क्री॒णाति॑ गमयि॒त्वा ग॑मयि॒त्वा क्री॒णाति॒ गच्छ॑ति । \newline
35. क्री॒णाति॒ गच्छ॑ति॒ गच्छ॑ति क्री॒णाति॑ क्री॒णाति॒ गच्छ॑ति॒ स्वानाꣳ॒॒ स्वाना॒म् गच्छ॑ति क्री॒णाति॑ क्री॒णाति॒ गच्छ॑ति॒ स्वाना᳚म् । \newline
36. गच्छ॑ति॒ स्वानाꣳ॒॒ स्वाना॒म् गच्छ॑ति॒ गच्छ॑ति॒ स्वानाꣳ॒॒ साम्रा᳚ज्यꣳ॒॒ साम्रा᳚ज्यꣳ॒॒ स्वाना॒म् गच्छ॑ति॒ गच्छ॑ति॒ स्वानाꣳ॒॒ साम्रा᳚ज्यम् । \newline
37. स्वानाꣳ॒॒ साम्रा᳚ज्यꣳ॒॒ साम्रा᳚ज्यꣳ॒॒ स्वानाꣳ॒॒ स्वानाꣳ॒॒ साम्रा᳚ज्य॒म् छन्दाꣳ॑सि॒ छन्दाꣳ॑सि॒ साम्रा᳚ज्यꣳ॒॒ स्वानाꣳ॒॒ स्वानाꣳ॒॒ साम्रा᳚ज्य॒म् छन्दाꣳ॑सि । \newline
38. साम्रा᳚ज्य॒म् छन्दाꣳ॑सि॒ छन्दाꣳ॑सि॒ साम्रा᳚ज्यꣳ॒॒ साम्रा᳚ज्य॒म् छन्दाꣳ॑सि॒ खलु॒ खलु॒ छन्दाꣳ॑सि॒ साम्रा᳚ज्यꣳ॒॒ साम्रा᳚ज्य॒म् छन्दाꣳ॑सि॒ खलु॑ । \newline
39. साम्रा᳚ज्य॒मिति॒ साम् - रा॒ज्य॒म् । \newline
40. छन्दाꣳ॑सि॒ खलु॒ खलु॒ छन्दाꣳ॑सि॒ छन्दाꣳ॑सि॒ खलु॒ वै वै खलु॒ छन्दाꣳ॑सि॒ छन्दाꣳ॑सि॒ खलु॒ वै । \newline
41. खलु॒ वै वै खलु॒ खलु॒ वै सोम॑स्य॒ सोम॑स्य॒ वै खलु॒ खलु॒ वै सोम॑स्य । \newline
42. वै सोम॑स्य॒ सोम॑स्य॒ वै वै सोम॑स्य॒ राज्ञो॒ राज्ञ्ः॒ सोम॑स्य॒ वै वै सोम॑स्य॒ राज्ञ्ः॑ । \newline
43. सोम॑स्य॒ राज्ञो॒ राज्ञ्ः॒ सोम॑स्य॒ सोम॑स्य॒ राज्ञ्ः॒ साम्रा᳚ज्यः॒ साम्रा᳚ज्यो॒ राज्ञ्ः॒ सोम॑स्य॒ सोम॑स्य॒ राज्ञ्ः॒ साम्रा᳚ज्यः । \newline
44. राज्ञ्ः॒ साम्रा᳚ज्यः॒ साम्रा᳚ज्यो॒ राज्ञो॒ राज्ञ्ः॒ साम्रा᳚ज्यो लो॒को लो॒कः साम्रा᳚ज्यो॒ राज्ञो॒ राज्ञ्ः॒ साम्रा᳚ज्यो लो॒कः । \newline
45. साम्रा᳚ज्यो लो॒को लो॒कः साम्रा᳚ज्यः॒ साम्रा᳚ज्यो लो॒कः पु॒रस्ता᳚त् पु॒रस्ता᳚ ल्लो॒कः साम्रा᳚ज्यः॒ साम्रा᳚ज्यो लो॒कः पु॒रस्ता᳚त् । \newline
46. साम्रा᳚ज्य॒ इति॒ साम् - रा॒ज्यः॒ । \newline
47. लो॒कः पु॒रस्ता᳚त् पु॒रस्ता᳚ ल्लो॒को लो॒कः पु॒रस्ता॒थ् सोम॑स्य॒ सोम॑स्य पु॒रस्ता᳚ ल्लो॒को लो॒कः पु॒रस्ता॒थ् सोम॑स्य । \newline
48. पु॒रस्ता॒थ् सोम॑स्य॒ सोम॑स्य पु॒रस्ता᳚त् पु॒रस्ता॒थ् सोम॑स्य क्र॒यात् क्र॒याथ् सोम॑स्य पु॒रस्ता᳚त् पु॒रस्ता॒थ् सोम॑स्य क्र॒यात् । \newline
49. सोम॑स्य क्र॒यात् क्र॒याथ् सोम॑स्य॒ सोम॑स्य क्र॒या दे॒व मे॒वम् क्र॒याथ् सोम॑स्य॒ सोम॑स्य क्र॒या दे॒वम् । \newline
50. क्र॒या दे॒व मे॒वम् क्र॒यात् क्र॒या दे॒व म॒भ्या᳚(1॒)भ्ये॑वम् क्र॒यात् क्र॒या दे॒व म॒भि । \newline
51. ए॒व म॒भ्या᳚(1॒)भ्ये॑व मे॒व म॒भि म॑न्त्रयेत मन्त्रयेता॒ भ्ये॑व मे॒व म॒भि म॑न्त्रयेत । \newline
52. अ॒भि म॑न्त्रयेत मन्त्रयेता॒ भ्य॑भि म॑न्त्रयेत॒ साम्रा᳚ज्यꣳ॒॒ साम्रा᳚ज्यम् मन्त्रयेता॒ भ्य॑भि म॑न्त्रयेत॒ साम्रा᳚ज्यम् । \newline
53. म॒न्त्र॒ये॒त॒ साम्रा᳚ज्यꣳ॒॒ साम्रा᳚ज्यम् मन्त्रयेत मन्त्रयेत॒ साम्रा᳚ज्य मे॒वैव साम्रा᳚ज्यम् मन्त्रयेत मन्त्रयेत॒ साम्रा᳚ज्य मे॒व । \newline
54. साम्रा᳚ज्य मे॒वैव साम्रा᳚ज्यꣳ॒॒ साम्रा᳚ज्य मे॒वैन॑ मेन मे॒व साम्रा᳚ज्यꣳ॒॒ साम्रा᳚ज्य मे॒वैन᳚म् । \newline
55. साम्रा᳚ज्य॒मिति॒ साम् - रा॒ज्य॒म् । \newline
56. ए॒वैन॑ मेन मे॒वैवैन॑म् ॅलो॒कम् ॅलो॒क मे॑न मे॒वैवैन॑म् ॅलो॒कम् । \newline
\pagebreak
\markright{ TS 3.1.2.2  \hfill https://www.vedavms.in \hfill}

\section{ TS 3.1.2.2 }

\textbf{TS 3.1.2.2 } \newline
\textbf{Samhita Paata} \newline

नं॑ ॅलो॒कं ग॑मयि॒त्वा क्री॑णाति॒ गच्छ॑ति॒ स्वानाꣳ॒॒ साम्रा᳚ज्यं॒ ॅयो वै ता॑नून॒प्त्रस्य॑ प्रति॒ष्ठां ॅवेद॒ प्रत्ये॒व ति॑ष्ठति ब्रह्मवा॒दिनो॑ वदन्ति॒ न प्रा॒श्नन्ति॒ न जु॑ह्व॒त्यथ॒ क्व॑ तानून॒प्त्रं प्रति॑ तिष्ठ॒तीति॑ प्र॒जाप॑तौ॒ मन॒सीति॑ ब्रूया॒त् त्रिरव॑ जिघ्रेत् प्र॒जाप॑तौ त्वा॒ मन॑सि जुहो॒मीत्ये॒षा वै ता॑नून॒प्त्रस्य॑ प्रति॒ष्ठा य ए॒वं ॅवेद॒ प्रत्ये॒व ति॑ष्ठति॒ यो - [  ] \newline

\textbf{Pada Paata} \newline

ए॒न॒म् । लो॒कम् । ग॒म॒यि॒त्वा । क्री॒णा॒ति॒ । गच्छ॑ति । स्वाना᳚म् । साम्रा᳚ज्य॒मिति॒ साम् - रा॒ज्य॒म् । यः । वै । ता॒नू॒न॒प्त्रस्येति॑ तानू - न॒प्त्रस्य॑ । प्र॒ति॒ष्ठामिति॑ प्रति - स्थाम् । वेद॑ । प्रतीति॑ । ए॒व । ति॒ष्ठ॒ति॒ । ब्र॒ह्म॒वा॒दिन॒ इति॑ ब्रह्म - वा॒दिनः॑ । व॒द॒न्ति॒ । न । प्रा॒श्नन्तीति॑ प्र - अ॒श्नन्ति॑ । न । जु॒ह्व॒ति॒ । अथ॑ । क्व॑ । ता॒नू॒न॒प्त्रमिति॑ तानू - न॒प्त्रम् । प्रतीति॑ । ति॒ष्ठ॒ति॒ । इति॑ । प्र॒जाप॑ता॒विति॑ प्र॒जा-प॒तौ॒ । मन॑सि । इति॑ । ब्रू॒या॒त् । त्रिः । अवेति॑ । जि॒घ्रे॒त् । प्र॒जाप॑ता॒विति॑ प्र॒जा - प॒तौ॒ । त्वा॒ । मन॑सि । जु॒हो॒मि॒ । इति॑ । ए॒षा । वै । ता॒नू॒न॒प्त्रस्येति॑ तानू - न॒प्त्रस्य॑ । प्र॒ति॒ष्ठेति॑ प्रति - स्था । यः । ए॒वम् । वेद॑ । प्रतीति॑ । ए॒व । ति॒ष्ठ॒ति॒ । यः ।  \newline


\textbf{Krama Paata} \newline

ए॒नं॒ ॅलो॒कम् । लो॒कम् ग॑मयि॒त्वा । ग॒म॒यि॒त्वा क्री॑णाति । क्री॒णा॒ति॒ गच्छ॑ति । गच्छ॑ति॒ स्वाना᳚म् । स्वानाꣳ॒॒ साम्रा᳚ज्यम् । साम्रा᳚ज्यं॒ ॅयः । साम्रा᳚ज्य॒मिति॒ साम् - रा॒ज्य॒म् । यो वै । वै ता॑नून॒प्त्रस्य॑ । ता॒नू॒न॒प्त्रस्य॑ प्रति॒ष्ठाम् । ता॒नू॒न॒प्त्रस्येति॑ तानू - न॒प्त्रस्य॑ । प्र॒ति॒ष्ठां ॅवेद॑ । प्र॒ति॒ष्ठामिति॑ प्रति - स्थाम् । वेद॒ प्रति॑ । प्रत्ये॒व । ए॒व ति॑ष्ठति । ति॒ष्ठ॒ति॒ ब्र॒ह्म॒वा॒दिनः॑ । ब्र॒ह्म॒वा॒दिनो॑ वदन्ति । ब्र॒ह्म॒वा॒दिन॒ इति॑ ब्रह्म - वा॒दिनः॑ । व॒द॒न्ति॒ न । न प्रा॒श्ञन्ति॑ । प्रा॒श्ञन्ति॒ न । प्रा॒श्ञन्तीति॑ प्र - अ॒श्ञन्ति॑ । न जु॑ह्वति । जु॒ह्व॒त्यथ॑ । अथ॒ क्व॑ । क्व॑ तानून॒प्त्रम् । ता॒नू॒न॒प्त्रम् प्रति॑ । ता॒नू॒न॒प्त्रमिति॑ तानू - न॒प्त्रम् । प्रति॑ तिष्ठति । ति॒ष्ठ॒तीति॑ । इति॑ प्र॒जाप॑तौ । प्र॒जाप॑तौ॒ मन॑सि । प्र॒जाप॑ता॒विति॑ प्र॒जा - प॒तौ॒ । मन॒सीति॑ । इति॑ ब्रूयात् । ब्रू॒या॒त् त्रिः । त्रिरव॑ । अव॑ जिघ्रेत् । जि॒घ्रे॒त् प्र॒जाप॑तौ । प्र॒जाप॑तौ त्वा । प्र॒जाप॑ता॒विति॑ प्र॒जा - प॒तौ॒ । त्वा॒ मन॑सि । मन॑सि जुहोमि । जु॒हो॒मीति॑ । इत्ये॒षा । ए॒षा वै । वै ता॑नून॒प्त्रस्य॑ । ता॒नू॒न॒प्त्रस्य॑ प्रति॒ष्ठा । ता॒नू॒न॒प्त्रस्येति॑ तानू - न॒प्त्रस्य॑ । प्र॒ति॒ष्ठा यः । प्र॒ति॒ष्ठेति॑ प्रति - स्था । य ए॒वम् । ए॒वं ॅवेद॑ । वेद॒ प्रति॑ । प्रत्ये॒व । ए॒व ति॑ष्ठति । ति॒ष्ठ॒ति॒ यः । यो वै \newline

\textbf{Jatai Paata} \newline

1. ए॒न॒म् ॅलो॒कम् ॅलो॒क मे॑न मेनम् ॅलो॒कम् । \newline
2. लो॒कम् ग॑मयि॒त्वा ग॑मयि॒त्वा लो॒कम् ॅलो॒कम् ग॑मयि॒त्वा । \newline
3. ग॒म॒यि॒त्वा क्री॑णाति क्रीणाति गमयि॒त्वा ग॑मयि॒त्वा क्री॑णाति । \newline
4. क्री॒णा॒ति॒ गच्छ॑ति॒ गच्छ॑ति क्रीणाति क्रीणाति॒ गच्छ॑ति । \newline
5. गच्छ॑ति॒ स्वानाꣳ॒॒ स्वाना॒म् गच्छ॑ति॒ गच्छ॑ति॒ स्वाना᳚म् । \newline
6. स्वानाꣳ॒॒ साम्रा᳚ज्यꣳ॒॒ साम्रा᳚ज्यꣳ॒॒ स्वानाꣳ॒॒ स्वानाꣳ॒॒ साम्रा᳚ज्यम् । \newline
7. साम्रा᳚ज्यं॒ ॅयो यः साम्रा᳚ज्यꣳ॒॒ साम्रा᳚ज्यं॒ ॅयः । \newline
8. साम्रा᳚ज्य॒मिति॒ साम् - रा॒ज्य॒म् । \newline
9. यो वै वै यो यो वै । \newline
10. वै ता॑नून॒प्त्रस्य॑ तानून॒प्त्रस्य॒ वै वै ता॑नून॒प्त्रस्य॑ । \newline
11. ता॒नू॒न॒प्त्रस्य॑ प्रति॒ष्ठाम् प्र॑ति॒ष्ठाम् ता॑नून॒प्त्रस्य॑ तानून॒प्त्रस्य॑ प्रति॒ष्ठाम् । \newline
12. ता॒नू॒न॒प्त्रस्येति॑ तानू - न॒प्त्रस्य॑ । \newline
13. प्र॒ति॒ष्ठां ॅवेद॒ वेद॑ प्रति॒ष्ठाम् प्र॑ति॒ष्ठां ॅवेद॑ । \newline
14. प्र॒ति॒ष्ठामिति॑ प्रति - स्थाम् । \newline
15. वेद॒ प्रति॒ प्रति॒ वेद॒ वेद॒ प्रति॑ । \newline
16. प्रत्ये॒वैव प्रति॒ प्रत्ये॒व । \newline
17. ए॒व ति॑ष्ठति तिष्ठ त्ये॒वैव ति॑ष्ठति । \newline
18. ति॒ष्ठ॒ति॒ ब्र॒ह्म॒वा॒दिनो᳚ ब्रह्मवा॒दिन॑ स्तिष्ठति तिष्ठति ब्रह्मवा॒दिनः॑ । \newline
19. ब्र॒ह्म॒वा॒दिनो॑ वदन्ति वदन्ति ब्रह्मवा॒दिनो᳚ ब्रह्मवा॒दिनो॑ वदन्ति । \newline
20. ब्र॒ह्म॒वा॒दिन॒ इति॑ ब्रह्म - वा॒दिनः॑ । \newline
21. व॒द॒न्ति॒ न न व॑दन्ति वदन्ति॒ न । \newline
22. न प्रा॒श्ञन्ति॑ प्रा॒श्ञन्ति॒ न न प्रा॒श्ञन्ति॑ । \newline
23. प्रा॒श्ञन्ति॒ न न प्रा॒श्ञन्ति॑ प्रा॒श्ञन्ति॒ न । \newline
24. प्रा॒श्ञन्तीति॑ प्र - अ॒श्ञन्ति॑ । \newline
25. न जु॑ह्वति जुह्वति॒ न न जु॑ह्वति । \newline
26. जु॒ह्व॒ त्यथाथ॑ जुह्वति जुह्व॒ त्यथ॑ । \newline
27. अथ॒ क्व॑ क्वा॑थाथ॒ क्व॑ । \newline
28. क्व॑ तानून॒प्त्रम् ता॑नून॒प्त्रम् क्वा᳚(1॒) क्व॑ तानून॒प्त्रम् । \newline
29. ता॒नू॒न॒प्त्रम् प्रति॒ प्रति॑ तानून॒प्त्रम् ता॑नून॒प्त्रम् प्रति॑ । \newline
30. ता॒नू॒न॒प्त्रमिति॑ तानू - न॒प्त्रम् । \newline
31. प्रति॑ तिष्ठति तिष्ठति॒ प्रति॒ प्रति॑ तिष्ठति । \newline
32. ति॒ष्ठ॒तीतीति॑ तिष्ठति तिष्ठ॒तीति॑ । \newline
33. इति॑ प्र॒जाप॑तौ प्र॒जाप॑ता॒ वितीति॑ प्र॒जाप॑तौ । \newline
34. प्र॒जाप॑तौ॒ मन॑सि॒ मन॑सि प्र॒जाप॑तौ प्र॒जाप॑तौ॒ मन॑सि । \newline
35. प्र॒जाप॑ता॒विति॑ प्र॒जा - प॒तौ॒ । \newline
36. मन॒सीतीति॒ मन॑सि॒ मन॒सीति॑ । \newline
37. इति॑ ब्रूयाद् ब्रूया॒ दितीति॑ ब्रूयात् । \newline
38. ब्रू॒या॒त् त्रि स्त्रिर् ब्रू॑याद् ब्रूया॒त् त्रिः । \newline
39. त्रि रवाव॒ त्रि स्त्रि रव॑ । \newline
40. अव॑ जिघ्रेज् जिघ्रे॒ दवाव॑ जिघ्रेत् । \newline
41. जि॒घ्रे॒त् प्र॒जाप॑तौ प्र॒जाप॑तौ जिघ्रेज् जिघ्रेत् प्र॒जाप॑तौ । \newline
42. प्र॒जाप॑तौ त्वा त्वा प्र॒जाप॑तौ प्र॒जाप॑तौ त्वा । \newline
43. प्र॒जाप॑ता॒विति॑ प्र॒जा - प॒तौ॒ । \newline
44. त्वा॒ मन॑सि॒ मन॑सि त्वा त्वा॒ मन॑सि । \newline
45. मन॑सि जुहोमि जुहोमि॒ मन॑सि॒ मन॑सि जुहोमि । \newline
46. जु॒हो॒मीतीति॑ जुहोमि जुहो॒मीति॑ । \newline
47. इत्ये॒ षैषेती त्ये॒षा । \newline
48. ए॒षा वै वा ए॒षैषा वै । \newline
49. वै ता॑नून॒प्त्रस्य॑ तानून॒प्त्रस्य॒ वै वै ता॑नून॒प्त्रस्य॑ । \newline
50. ता॒नू॒न॒प्त्रस्य॑ प्रति॒ष्ठा प्र॑ति॒ष्ठा ता॑नून॒प्त्रस्य॑ तानून॒प्त्रस्य॑ प्रति॒ष्ठा । \newline
51. ता॒नू॒न॒प्त्रस्येति॑ तानू - न॒प्त्रस्य॑ । \newline
52. प्र॒ति॒ष्ठा यो यः प्र॑ति॒ष्ठा प्र॑ति॒ष्ठा यः । \newline
53. प्र॒ति॒ष्ठेति॑ प्रति - स्था । \newline
54. य ए॒व मे॒वं ॅयो य ए॒वम् । \newline
55. ए॒वं ॅवेद॒ वेदै॒व मे॒वं ॅवेद॑ । \newline
56. वेद॒ प्रति॒ प्रति॒ वेद॒ वेद॒ प्रति॑ । \newline
57. प्रत्ये॒वैव प्रति॒ प्रत्ये॒व । \newline
58. ए॒व ति॑ष्ठति तिष्ठ त्ये॒वैव ति॑ष्ठति । \newline
59. ति॒ष्ठ॒ति॒ यो य स्ति॑ष्ठति तिष्ठति॒ यः । \newline
60. यो वै वै यो यो वै । \newline

\textbf{Ghana Paata } \newline

1. ए॒न॒म् ॅलो॒कम् ॅलो॒क मे॑न मेनम् ॅलो॒कम् ग॑मयि॒त्वा ग॑मयि॒त्वा लो॒क मे॑न मेनम् ॅलो॒कम् ग॑मयि॒त्वा । \newline
2. लो॒कम् ग॑मयि॒त्वा ग॑मयि॒त्वा लो॒कम् ॅलो॒कम् ग॑मयि॒त्वा क्री॑णाति क्रीणाति गमयि॒त्वा लो॒कम् ॅलो॒कम् ग॑मयि॒त्वा क्री॑णाति । \newline
3. ग॒म॒यि॒त्वा क्री॑णाति क्रीणाति गमयि॒त्वा ग॑मयि॒त्वा क्री॑णाति॒ गच्छ॑ति॒ गच्छ॑ति क्रीणाति गमयि॒त्वा ग॑मयि॒त्वा क्री॑णाति॒ गच्छ॑ति । \newline
4. क्री॒णा॒ति॒ गच्छ॑ति॒ गच्छ॑ति क्रीणाति क्रीणाति॒ गच्छ॑ति॒ स्वानाꣳ॒॒ स्वाना॒म् गच्छ॑ति क्रीणाति क्रीणाति॒ गच्छ॑ति॒ स्वाना᳚म् । \newline
5. गच्छ॑ति॒ स्वानाꣳ॒॒ स्वाना॒म् गच्छ॑ति॒ गच्छ॑ति॒ स्वानाꣳ॒॒ साम्रा᳚ज्यꣳ॒॒ साम्रा᳚ज्यꣳ॒॒ स्वाना॒म् गच्छ॑ति॒ गच्छ॑ति॒ स्वानाꣳ॒॒ साम्रा᳚ज्यम् । \newline
6. स्वानाꣳ॒॒ साम्रा᳚ज्यꣳ॒॒ साम्रा᳚ज्यꣳ॒॒ स्वानाꣳ॒॒ स्वानाꣳ॒॒ साम्रा᳚ज्यं॒ ॅयो यः साम्रा᳚ज्यꣳ॒॒ स्वानाꣳ॒॒ स्वानाꣳ॒॒ साम्रा᳚ज्यं॒ ॅयः । \newline
7. साम्रा᳚ज्यं॒ ॅयो यः साम्रा᳚ज्यꣳ॒॒ साम्रा᳚ज्यं॒ ॅयो वै वै यः साम्रा᳚ज्यꣳ॒॒ साम्रा᳚ज्यं॒ ॅयो वै । \newline
8. साम्रा᳚ज्य॒मिति॒ साम् - रा॒ज्य॒म् । \newline
9. यो वै वै यो यो वै ता॑नून॒प्त्रस्य॑ तानून॒प्त्रस्य॒ वै यो यो वै ता॑नून॒प्त्रस्य॑ । \newline
10. वै ता॑नून॒प्त्रस्य॑ तानून॒प्त्रस्य॒ वै वै ता॑नून॒प्त्रस्य॑ प्रति॒ष्ठाम् प्र॑ति॒ष्ठाम् ता॑नून॒प्त्रस्य॒ वै वै ता॑नून॒प्त्रस्य॑ प्रति॒ष्ठाम् । \newline
11. ता॒नू॒न॒प्त्रस्य॑ प्रति॒ष्ठाम् प्र॑ति॒ष्ठाम् ता॑नून॒प्त्रस्य॑ तानून॒प्त्रस्य॑ प्रति॒ष्ठां ॅवेद॒ वेद॑ प्रति॒ष्ठाम् ता॑नून॒प्त्रस्य॑ तानून॒प्त्रस्य॑ प्रति॒ष्ठां ॅवेद॑ । \newline
12. ता॒नू॒न॒प्त्रस्येति॑ तानू - न॒प्त्रस्य॑ । \newline
13. प्र॒ति॒ष्ठां ॅवेद॒ वेद॑ प्रति॒ष्ठाम् प्र॑ति॒ष्ठां ॅवेद॒ प्रति॒ प्रति॒ वेद॑ प्रति॒ष्ठाम् प्र॑ति॒ष्ठां ॅवेद॒ प्रति॑ । \newline
14. प्र॒ति॒ष्ठामिति॑ प्रति - स्थाम् । \newline
15. वेद॒ प्रति॒ प्रति॒ वेद॒ वेद॒ प्रत्ये॒वैव प्रति॒ वेद॒ वेद॒ प्रत्ये॒व । \newline
16. प्रत्ये॒वैव प्रति॒ प्रत्ये॒व ति॑ष्ठति तिष्ठत्ये॒व प्रति॒ प्रत्ये॒व ति॑ष्ठति । \newline
17. ए॒व ति॑ष्ठति तिष्ठत्ये॒वैव ति॑ष्ठति ब्रह्मवा॒दिनो᳚ ब्रह्मवा॒दिन॑ स्तिष्ठत्ये॒वैव ति॑ष्ठति ब्रह्मवा॒दिनः॑ । \newline
18. ति॒ष्ठ॒ति॒ ब्र॒ह्म॒वा॒दिनो᳚ ब्रह्मवा॒दिन॑ स्तिष्ठति तिष्ठति ब्रह्मवा॒दिनो॑ वदन्ति वदन्ति ब्रह्मवा॒दिन॑ स्तिष्ठति तिष्ठति ब्रह्मवा॒दिनो॑ वदन्ति । \newline
19. ब्र॒ह्म॒वा॒दिनो॑ वदन्ति वदन्ति ब्रह्मवा॒दिनो᳚ ब्रह्मवा॒दिनो॑ वदन्ति॒ न न व॑दन्ति ब्रह्मवा॒दिनो᳚ ब्रह्मवा॒दिनो॑ वदन्ति॒ न । \newline
20. ब्र॒ह्म॒वा॒दिन॒ इति॑ ब्रह्म - वा॒दिनः॑ । \newline
21. व॒द॒न्ति॒ न न व॑दन्ति वदन्ति॒ न प्रा॒श्ञन्ति॑ प्रा॒श्ञन्ति॒ न व॑दन्ति वदन्ति॒ न प्रा॒श्ञन्ति॑ । \newline
22. न प्रा॒श्ञन्ति॑ प्रा॒श्ञन्ति॒ न न प्रा॒श्ञन्ति॒ न न प्रा॒श्ञन्ति॒ न न प्रा॒श्ञन्ति॒ न । \newline
23. प्रा॒श्ञन्ति॒ न न प्रा॒श्ञन्ति॑ प्रा॒श्ञन्ति॒ न जु॑ह्वति जुह्वति॒ न प्रा॒श्ञन्ति॑ प्रा॒श्ञन्ति॒ न जु॑ह्वति । \newline
24. प्रा॒श्ञन्तीति॑ प्र - अ॒श्ञन्ति॑ । \newline
25. न जु॑ह्वति जुह्वति॒ न न जु॑ह्व॒ त्यथाथ॑ जुह्वति॒ न न जु॑ह्व॒त्यथ॑ । \newline
26. जु॒ह्व॒ त्यथाथ॑ जुह्वति जुह्व॒ त्यथ॒ क्व॑ क्वाथ॑ जुह्वति जुह्व॒ त्यथ॒ क्व॑ । \newline
27. अथ॒ क्व॑ क्वा॑थाथ॒ क्व॑ तानून॒प्त्रम् ता॑नून॒प्त्रम् क्वा॑थाथ॒ क्व॑ तानून॒प्त्रम् । \newline
28. क्व॑ तानून॒प्त्रम् ता॑नून॒प्त्रम् क्वा᳚(1॒) क्व॑ तानून॒प्त्रम् प्रति॒ प्रति॑ तानून॒प्त्रम् क्वा᳚(1॒) क्व॑ तानून॒प्त्रम् प्रति॑ । \newline
29. ता॒नू॒न॒प्त्रम् प्रति॒ प्रति॑ तानून॒प्त्रम् ता॑नून॒प्त्रम् प्रति॑ तिष्ठति तिष्ठति॒ प्रति॑ तानून॒प्त्रम् ता॑नून॒प्त्रम् प्रति॑ तिष्ठति । \newline
30. ता॒नू॒न॒प्त्रमिति॑ तानू - न॒प्त्रम् । \newline
31. प्रति॑ तिष्ठति तिष्ठति॒ प्रति॒ प्रति॑ तिष्ठ॒तीतीति॑ तिष्ठति॒ प्रति॒ प्रति॑ तिष्ठ॒तीति॑ । \newline
32. ति॒ष्ठ॒तीतीति॑ तिष्ठति तिष्ठ॒तीति॑ प्र॒जाप॑तौ प्र॒जाप॑ता॒ विति॑ तिष्ठति तिष्ठ॒तीति॑ प्र॒जाप॑तौ । \newline
33. इति॑ प्र॒जाप॑तौ प्र॒जाप॑ता॒ वितीति॑ प्र॒जाप॑तौ॒ मन॑सि॒ मन॑सि प्र॒जाप॑ता॒ वितीति॑ प्र॒जाप॑तौ॒ मन॑सि । \newline
34. प्र॒जाप॑तौ॒ मन॑सि॒ मन॑सि प्र॒जाप॑तौ प्र॒जाप॑तौ॒ मन॒सीतीति॒ मन॑सि प्र॒जाप॑तौ प्र॒जाप॑तौ॒ मन॒सीति॑ । \newline
35. प्र॒जाप॑ता॒विति॑ प्र॒जा - प॒तौ॒ । \newline
36. मन॒सीतीति॒ मन॑सि॒ मन॒सीति॑ ब्रूयाद् ब्रूया॒ दिति॒ मन॑सि॒ मन॒सीति॑ ब्रूयात् । \newline
37. इति॑ ब्रूयाद् ब्रूया॒ दितीति॑ ब्रूया॒त् त्रि स्त्रिर् ब्रू॑या॒ दितीति॑ ब्रूया॒त् त्रिः । \newline
38. ब्रू॒या॒त् त्रि स्त्रिर् ब्रू॑याद् ब्रूया॒त् त्रिरवाव॒ त्रिर् ब्रू॑याद् ब्रूया॒त् त्रिरव॑ । \newline
39. त्रिरवाव॒ त्रि स्त्रिरव॑ जिघ्रेज् जिघ्रे॒ दव॒ त्रि स्त्रि रव॑ जिघ्रेत् । \newline
40. अव॑ जिघ्रेज् जिघ्रे॒ दवाव॑ जिघ्रेत् प्र॒जाप॑तौ प्र॒जाप॑तौ जिघ्रे॒ दवाव॑ जिघ्रेत् प्र॒जाप॑तौ । \newline
41. जि॒घ्रे॒त् प्र॒जाप॑तौ प्र॒जाप॑तौ जिघ्रेज् जिघ्रेत् प्र॒जाप॑तौ त्वा त्वा प्र॒जाप॑तौ जिघ्रेज् जिघ्रेत् प्र॒जाप॑तौ त्वा । \newline
42. प्र॒जाप॑तौ त्वा त्वा प्र॒जाप॑तौ प्र॒जाप॑तौ त्वा॒ मन॑सि॒ मन॑सि त्वा प्र॒जाप॑तौ प्र॒जाप॑तौ त्वा॒ मन॑सि । \newline
43. प्र॒जाप॑ता॒विति॑ प्र॒जा - प॒तौ॒ । \newline
44. त्वा॒ मन॑सि॒ मन॑सि त्वा त्वा॒ मन॑सि जुहोमि जुहोमि॒ मन॑सि त्वा त्वा॒ मन॑सि जुहोमि । \newline
45. मन॑सि जुहोमि जुहोमि॒ मन॑सि॒ मन॑सि जुहो॒मीतीति॑ जुहोमि॒ मन॑सि॒ मन॑सि जुहो॒मीति॑ । \newline
46. जु॒हो॒मीतीति॑ जुहोमि जुहो॒मी त्ये॒षैषेति॑ जुहोमि जुहो॒मी त्ये॒षा । \newline
47. इत्ये॒षैषेती त्ये॒षा वै वा ए॒षेती त्ये॒षा वै । \newline
48. ए॒षा वै वा ए॒षैषा वै ता॑नून॒प्त्रस्य॑ तानून॒प्त्रस्य॒ वा ए॒षैषा वै ता॑नून॒प्त्रस्य॑ । \newline
49. वै ता॑नून॒प्त्रस्य॑ तानून॒प्त्रस्य॒ वै वै ता॑नून॒प्त्रस्य॑ प्रति॒ष्ठा प्र॑ति॒ष्ठा ता॑नून॒प्त्रस्य॒ वै वै ता॑नून॒प्त्रस्य॑ प्रति॒ष्ठा । \newline
50. ता॒नू॒न॒प्त्रस्य॑ प्रति॒ष्ठा प्र॑ति॒ष्ठा ता॑नून॒प्त्रस्य॑ तानून॒प्त्रस्य॑ प्रति॒ष्ठा यो यः प्र॑ति॒ष्ठा ता॑नून॒प्त्रस्य॑ तानून॒प्त्रस्य॑ प्रति॒ष्ठा यः । \newline
51. ता॒नू॒न॒प्त्रस्येति॑ तानू - न॒प्त्रस्य॑ । \newline
52. प्र॒ति॒ष्ठा यो यः प्र॑ति॒ष्ठा प्र॑ति॒ष्ठा य ए॒व मे॒वं ॅयः प्र॑ति॒ष्ठा प्र॑ति॒ष्ठा य ए॒वम् । \newline
53. प्र॒ति॒ष्ठेति॑ प्रति - स्था । \newline
54. य ए॒व मे॒वं ॅयो य ए॒वं ॅवेद॒ वेदै॒वं ॅयो य ए॒वं ॅवेद॑ । \newline
55. ए॒वं ॅवेद॒ वेदै॒व मे॒वं ॅवेद॒ प्रति॒ प्रति॒ वेदै॒व मे॒वं ॅवेद॒ प्रति॑ । \newline
56. वेद॒ प्रति॒ प्रति॒ वेद॒ वेद॒ प्रत्ये॒वैव प्रति॒ वेद॒ वेद॒ प्रत्ये॒व । \newline
57. प्रत्ये॒वैव प्रति॒ प्रत्ये॒व ति॑ष्ठति तिष्ठत्ये॒व प्रति॒ प्रत्ये॒व ति॑ष्ठति । \newline
58. ए॒व ति॑ष्ठति तिष्ठत्ये॒वैव ति॑ष्ठति॒ यो य स्ति॑ष्ठ त्ये॒वैव ति॑ष्ठति॒ यः । \newline
59. ति॒ष्ठ॒ति॒ यो य स्ति॑ष्ठति तिष्ठति॒ यो वै वै य स्ति॑ष्ठति तिष्ठति॒ यो वै । \newline
60. यो वै वै यो यो वा अ॑द्ध्व॒र्यो र॑द्ध्व॒र्योर् वै यो यो वा अ॑द्ध्व॒र्योः । \newline
\pagebreak
\markright{ TS 3.1.2.3  \hfill https://www.vedavms.in \hfill}

\section{ TS 3.1.2.3 }

\textbf{TS 3.1.2.3 } \newline
\textbf{Samhita Paata} \newline

वा अ॑द्ध्व॒र्योः प्र॑ति॒ष्ठां ॅवेद॒ प्रत्ये॒व ति॑ष्ठति॒ यतो॒ मन्ये॒तान॑भिक्रम्य होष्या॒मीति॒ तत् तिष्ठ॒न्ना श्रा॑वयेदे॒षा वा अ॑द्ध्व॒र्योः प्र॑ति॒ष्ठा य ए॒वं ॅवेद॒ प्रत्ये॒व ति॑ष्ठति॒ यद॑भि॒क्रम्य॑ जुहु॒यात् प्र॑ति॒ष्ठाया॑ इया॒त् तस्मा᳚थ् समा॒नत्र॒ तिष्ठ॑ता होत॒व्यं॑ प्रति॑ष्ठित्यै॒ यो वा अ॑द्ध्व॒र्योः स्वं ॅवेद॒ स्ववा॑ने॒व भ॑वति॒ स्रुग्वा अ॑स्य॒ स्वं ॅवा॑य॒व्य॑मस्य॒ - [  ] \newline

\textbf{Pada Paata} \newline

वै । अ॒द्ध्व॒र्योः । प्र॒ति॒ष्ठामिति॑ प्रति - स्थाम् । वेद॑ । प्रतीति॑ । ए॒व । ति॒ष्ठ॒ति॒ । यतः॑ । मन्ये॑त । अन॑भिक्र॒म्येत्यन॑भि - क्र॒म्य॒ । हो॒ष्या॒मि॒ । इति॑ । तत् । तिष्ठन्न्॑ । एति॑ । श्रा॒व॒ये॒त् । ए॒षा । वै । अ॒द्ध्व॒र्योः । प्र॒ति॒ष्ठेति॑ प्रति - स्था । यः । ए॒वम् । वेद॑ । प्रतीति॑ । ए॒व । ति॒ष्ठ॒ति॒ । यत् । अ॒भि॒क्रम्येत्य॑भि - क्रम्य॑ । जु॒हु॒यात् । प्र॒ति॒ष्ठाया॒ इति॑ प्रति-स्थायाः᳚ । इ॒या॒त् । तस्मा᳚त् । स॒मा॒नत्र॑ । तिष्ठ॑ता । हो॒त॒व्य᳚म् । प्रति॑ष्ठित्या॒ इति॒ प्रति॑ - स्थि॒त्यै॒ । यः । वै । अ॒द्ध्व॒र्योः । स्वम् । वेद॑ । स्ववा॒निति॒ स्व - वा॒न् । ए॒व । भ॒व॒ति॒ । स्रुक् । वै । अ॒स्य॒ । स्वम् । वा॒य॒व्य᳚म् । अ॒स्य॒ ।  \newline


\textbf{Krama Paata} \newline

वा अ॑द्ध्व॒र्योः । अ॒द्ध्व॒र्योः प्र॑ति॒ष्ठाम् । प्र॒ति॒ष्ठां ॅवेद॑ । प्र॒ति॒ष्ठामिति॑ प्रति - स्थाम् । वेद॒ प्रति॑ । प्रत्ये॒व । ए॒व ति॑ष्ठति । ति॒ष्ठ॒ति॒ यतः॑ । यतो॒ मन्ये॑त । मन्ये॒तान॑भिक्रम्य । अन॑भिक्रम्य होष्यामि । अन॑भिक्र॒म्येत्यन॑भि - क्र॒म्य॒ । हो॒ष्या॒मीति॑ । इति॒ तत् ।
 तत् तिष्ठन्न्॑ । तिष्ठ॒न्ना । आ श्रा॑वयेत् । श्रा॒व॒ये॒दे॒षा । ए॒षा वै । वा अ॑द्ध्व॒र्योः । अ॒द्ध्व॒र्योः प्र॑ति॒ष्ठा । प्र॒ति॒ष्ठा यः । प्र॒ति॒ष्ठेति॑ प्रति - स्था । य ए॒वम् । ए॒वं ॅवेद॑ । वेद॒ प्रति॑ । प्रत्ये॒व । ए॒व ति॑ष्ठति । ति॒ष्ठ॒ति॒ यत् । यद॑भि॒क्रम्य॑ । अ॒भि॒क्रम्य॑ जुहु॒यात् । अ॒भि॒क्रम्येत्य॑भि - क्रम्य॑ । जु॒हु॒यात् प्र॑ति॒ष्ठायाः᳚ । प्र॒ति॒ष्ठाया॑ इयात् । प्र॒ति॒ष्ठाया॒ इति॑ प्रति - स्थायाः᳚ । इ॒या॒त् तस्मा᳚त् । तस्मा᳚थ् समा॒नत्र॑ । स॒मा॒नत्र॒ तिष्ठ॑ता । तिष्ठ॑ता होत॒व्य᳚म् । हो॒त॒व्य॑म् प्रति॑ष्ठित्यै । प्रति॑ष्ठित्यै॒ यः । प्रति॑ष्ठित्या॒ इति॒ प्रति॑ - स्थि॒त्यै॒ । यो वै । वा अ॑द्ध्व॒र्योः । अ॒द्ध्व॒र्योः स्वम् । स्वं ॅवेद॑ । वेद॒ स्ववान्॑ । स्ववा॑ने॒व । स्ववा॒निति॒ स्व - वा॒न्॒ । ए॒व भ॑वति । भ॒व॒ति॒ स्रुक् । स्रुग् वै । वा अ॑स्य । अ॒स्य॒ स्वम् । स्वं ॅवा॑य॒व्य᳚म् । वा॒य॒व्य॑मस्य । अ॒स्य॒ स्वम् \newline

\textbf{Jatai Paata} \newline

1. वा अ॑द्ध्व॒र्यो र॑द्ध्व॒र्योर् वै वा अ॑द्ध्व॒र्योः । \newline
2. अ॒द्ध्व॒र्योः प्र॑ति॒ष्ठाम् प्र॑ति॒ष्ठा म॑द्ध्व॒र्यो र॑द्ध्व॒र्योः प्र॑ति॒ष्ठाम् । \newline
3. प्र॒ति॒ष्ठां ॅवेद॒ वेद॑ प्रति॒ष्ठाम् प्र॑ति॒ष्ठां ॅवेद॑ । \newline
4. प्र॒ति॒ष्ठामिति॑ प्रति - स्थाम् । \newline
5. वेद॒ प्रति॒ प्रति॒ वेद॒ वेद॒ प्रति॑ । \newline
6. प्रत्ये॒वैव प्रति॒ प्रत्ये॒व । \newline
7. ए॒व ति॑ष्ठति तिष्ठ त्ये॒वैव ति॑ष्ठति । \newline
8. ति॒ष्ठ॒ति॒ यतो॒ यत॑ स्तिष्ठति तिष्ठति॒ यतः॑ । \newline
9. यतो॒ मन्ये॑त॒ मन्ये॑त॒ यतो॒ यतो॒ मन्ये॑त । \newline
10. मन्ये॒ता न॑भिक्र॒म्या न॑भिक्रम्य॒ मन्ये॑त॒ मन्ये॒ता न॑भिक्रम्य । \newline
11. अन॑भिक्रम्य होष्यामि होष्या॒ म्यन॑भिक्र॒म्या न॑भिक्रम्य होष्यामि । \newline
12. अन॑भिक्र॒म्येत्यन॑भि - क्र॒म्य॒ । \newline
13. हो॒ष्या॒मीतीति॑ होष्यामि होष्या॒मीति॑ । \newline
14. इति॒ तत् तदितीति॒ तत् । \newline
15. तत् तिष्ठꣳ॒॒ स्तिष्ठꣳ॒॒ स्तत् तत् तिष्ठन्न्॑ । \newline
16. तिष्ठ॒न् ना तिष्ठꣳ॒॒ स्तिष्ठ॒न् ना । \newline
17. आ श्रा॑वयेच् छ्रावये॒ दा श्रा॑वयेत् । \newline
18. श्रा॒व॒ये॒ दे॒षैषा श्रा॑वयेच् छ्रावये दे॒षा । \newline
19. ए॒षा वै वा ए॒षैषा वै । \newline
20. वा अ॑द्ध्व॒र्यो र॑द्ध्व॒र्योर् वै वा अ॑द्ध्व॒र्योः । \newline
21. अ॒द्ध्व॒र्योः प्र॑ति॒ष्ठा प्र॑ति॒ष्ठा ऽद्ध्व॒र्यो र॑द्ध्व॒र्योः प्र॑ति॒ष्ठा । \newline
22. प्र॒ति॒ष्ठा यो यः प्र॑ति॒ष्ठा प्र॑ति॒ष्ठा यः । \newline
23. प्र॒ति॒ष्ठेति॑ प्रति - स्था । \newline
24. य ए॒व मे॒वं ॅयो य ए॒वम् । \newline
25. ए॒वं ॅवेद॒ वेदै॒व मे॒वं ॅवेद॑ । \newline
26. वेद॒ प्रति॒ प्रति॒ वेद॒ वेद॒ प्रति॑ । \newline
27. प्रत्ये॒वैव प्रति॒ प्रत्ये॒व । \newline
28. ए॒व ति॑ष्ठति तिष्ठ त्ये॒वैव ति॑ष्ठति । \newline
29. ति॒ष्ठ॒ति॒ यद् यत् ति॑ष्ठति तिष्ठति॒ यत् । \newline
30. यद॑भि॒क्रम्या॑ भि॒क्रम्य॒ यद् यद॑भि॒क्रम्य॑ । \newline
31. अ॒भि॒क्रम्य॑ जुहु॒याज् जु॑हु॒या द॑भि॒क्रम्या॑ भि॒क्रम्य॑ जुहु॒यात् । \newline
32. अ॒भि॒क्रम्येत्य॑भि - क्रम्य॑ । \newline
33. जु॒हु॒यात् प्र॑ति॒ष्ठायाः᳚ प्रति॒ष्ठाया॑ जुहु॒याज् जु॑हु॒यात् प्र॑ति॒ष्ठायाः᳚ । \newline
34. प्र॒ति॒ष्ठाया॑ इया दियात् प्रति॒ष्ठायाः᳚ प्रति॒ष्ठाया॑ इयात् । \newline
35. प्र॒ति॒ष्ठाया॒ इति॑ प्रति - स्थायाः᳚ । \newline
36. इ॒या॒त् तस्मा॒त् तस्मा॑ दिया दिया॒त् तस्मा᳚त् । \newline
37. तस्मा᳚थ् समा॒नत्र॑ समा॒नत्र॒ तस्मा॒त् तस्मा᳚थ् समा॒नत्र॑ । \newline
38. स॒मा॒नत्र॒ तिष्ठ॑ता॒ तिष्ठ॑ता समा॒नत्र॑ समा॒नत्र॒ तिष्ठ॑ता । \newline
39. तिष्ठ॑ता होत॒व्यꣳ॑ होत॒व्य॑म् तिष्ठ॑ता॒ तिष्ठ॑ता होत॒व्य᳚म् । \newline
40. हो॒त॒व्य॑म् प्रति॑ष्ठित्यै॒ प्रति॑ष्ठित्यै होत॒व्यꣳ॑ होत॒व्य॑म् प्रति॑ष्ठित्यै । \newline
41. प्रति॑ष्ठित्यै॒ यो यः प्रति॑ष्ठित्यै॒ प्रति॑ष्ठित्यै॒ यः । \newline
42. प्रति॑ष्ठित्या॒ इति॒ प्रति॑ - स्थि॒त्यै॒ । \newline
43. यो वै वै यो यो वै । \newline
44. वा अ॑द्ध्व॒र्यो र॑द्ध्व॒र्योर् वै वा अ॑द्ध्व॒र्योः । \newline
45. अ॒द्ध्व॒र्योः स्वꣳ स्व म॑द्ध्व॒र्यो र॑द्ध्व॒र्योः स्वम् । \newline
46. स्वं ॅवेद॒ वेद॒ स्वꣳ स्वं ॅवेद॑ । \newline
47. वेद॒ स्ववा॒न् थ्स्ववा॒न्॒. वेद॒ वेद॒ स्ववान्॑ । \newline
48. स्ववा॑ ने॒वैव स्ववा॒न् थ्स्ववा॑ ने॒व । \newline
49. स्ववा॒निति॒ स्व - वा॒न् । \newline
50. ए॒व भ॑वति भव त्ये॒वैव भ॑वति । \newline
51. भ॒व॒ति॒ स्रुख् स्रुग् भ॑वति भवति॒ स्रुक् । \newline
52. स्रुग् वै वै स्रुख् स्रुग् वै । \newline
53. वा अ॑स्यास्य॒ वै वा अ॑स्य । \newline
54. अ॒स्य॒ स्वꣳ स्व म॑स्यास्य॒ स्वम् । \newline
55. स्वं ॅवा॑य॒व्यं॑ ॅवाय॒व्यꣳ॑ स्वꣳ स्वं ॅवा॑य॒व्य᳚म् । \newline
56. वा॒य॒व्य॑ मस्यास्य वाय॒व्यं॑ ॅवाय॒व्य॑ मस्य । \newline
57. अ॒स्य॒ स्वꣳ स्व म॑स्यास्य॒ स्वम् । \newline

\textbf{Ghana Paata } \newline

1. वा अ॑द्ध्व॒र्यो र॑द्ध्व॒र्योर् वै वा अ॑द्ध्व॒र्योः प्र॑ति॒ष्ठाम् प्र॑ति॒ष्ठा म॑द्ध्व॒र्योर् वै वा अ॑द्ध्व॒र्योः प्र॑ति॒ष्ठाम् । \newline
2. अ॒द्ध्व॒र्योः प्र॑ति॒ष्ठाम् प्र॑ति॒ष्ठा म॑द्ध्व॒र्यो र॑द्ध्व॒र्योः प्र॑ति॒ष्ठां ॅवेद॒ वेद॑ प्रति॒ष्ठा म॑द्ध्व॒ र्योर॑द्ध्व॒र्योः प्र॑ति॒ष्ठां ॅवेद॑ । \newline
3. प्र॒ति॒ष्ठां ॅवेद॒ वेद॑ प्रति॒ष्ठाम् प्र॑ति॒ष्ठां ॅवेद॒ प्रति॒ प्रति॒ वेद॑ प्रति॒ष्ठाम् प्र॑ति॒ष्ठां ॅवेद॒ प्रति॑ । \newline
4. प्र॒ति॒ष्ठामिति॑ प्रति - स्थाम् । \newline
5. वेद॒ प्रति॒ प्रति॒ वेद॒ वेद॒ प्रत्ये॒वैव प्रति॒ वेद॒ वेद॒ प्रत्ये॒व । \newline
6. प्रत्ये॒वैव प्रति॒ प्रत्ये॒व ति॑ष्ठति तिष्ठत्ये॒व प्रति॒ प्रत्ये॒व ति॑ष्ठति । \newline
7. ए॒व ति॑ष्ठति तिष्ठ त्ये॒वैव ति॑ष्ठति॒ यतो॒ यत॑ स्तिष्ठ त्ये॒वैव ति॑ष्ठति॒ यतः॑ । \newline
8. ति॒ष्ठ॒ति॒ यतो॒ यत॑ स्तिष्ठति तिष्ठति॒ यतो॒ मन्ये॑त॒ मन्ये॑त॒ यत॑ स्तिष्ठति तिष्ठति॒ यतो॒ मन्ये॑त । \newline
9. यतो॒ मन्ये॑त॒ मन्ये॑त॒ यतो॒ यतो॒ मन्ये॒ता न॑भिक्र॒म्या न॑भिक्रम्य॒ मन्ये॑त॒ यतो॒ यतो॒ मन्ये॒ता न॑भिक्रम्य । \newline
10. मन्ये॒ता न॑भिक्र॒म्या न॑भिक्रम्य॒ मन्ये॑त॒ मन्ये॒ता न॑भिक्रम्य होष्यामि होष्या॒ म्यन॑भिक्रम्य॒ मन्ये॑त॒ मन्ये॒ता न॑भिक्रम्य होष्यामि । \newline
11. अन॑भिक्रम्य होष्यामि होष्या॒ म्यन॑भिक्र॒म्या न॑भिक्रम्य होष्या॒मीतीति॑ होष्या॒ म्यन॑भिक्र॒ म्यान॑भिक्रम्य होष्या॒मीति॑ । \newline
12. अन॑भिक्र॒म्येत्यन॑भि - क्र॒म्य॒ । \newline
13. हो॒ष्या॒मीतीति॑ होष्यामि होष्या॒मीति॒ तत् तदिति॑ होष्यामि होष्या॒मीति॒ तत् । \newline
14. इति॒ तत् तदितीति॒ तत् तिष्ठꣳ॒॒ स्तिष्ठꣳ॒॒ स्तदितीति॒ तत् तिष्ठन्न्॑ । \newline
15. तत् तिष्ठꣳ॒॒ स्तिष्ठꣳ॒॒ स्तत् तत् तिष्ठ॒न् ना तिष्ठꣳ॒॒ स्तत् तत् तिष्ठ॒न् ना । \newline
16. तिष्ठ॒न् ना तिष्ठꣳ॒॒ स्तिष्ठ॒न् ना श्रा॑वयेच् छ्रावये॒दा तिष्ठꣳ॒॒ स्तिष्ठ॒न् ना श्रा॑वयेत् । \newline
17. आ श्रा॑वयेच् छ्रावये॒दा श्रा॑वये दे॒षैषा श्रा॑वये॒दा श्रा॑वये दे॒षा । \newline
18. श्रा॒व॒ये॒ दे॒षैषा श्रा॑वयेच् छ्रावये दे॒षा वै वा ए॒षा श्रा॑वयेच् छ्रावये दे॒षा वै । \newline
19. ए॒षा वै वा ए॒षैषा वा अ॑द्ध्व॒र्यो र॑द्ध्व॒र्योर् वा ए॒षैषा वा अ॑द्ध्व॒र्योः । \newline
20. वा अ॑द्ध्व॒र्यो र॑द्ध्व॒र्योर् वै वा अ॑द्ध्व॒र्योः प्र॑ति॒ष्ठा प्र॑ति॒ष्ठा ऽद्ध्व॒र्योर् वै वा अ॑द्ध्व॒र्योः प्र॑ति॒ष्ठा । \newline
21. अ॒द्ध्व॒र्योः प्र॑ति॒ष्ठा प्र॑ति॒ष्ठा ऽद्ध्व॒र्यो र॑द्ध्व॒र्योः प्र॑ति॒ष्ठा यो यः प्र॑ति॒ष्ठा ऽद्ध्व॒र्यो र॑द्ध्व॒र्योः प्र॑ति॒ष्ठा यः । \newline
22. प्र॒ति॒ष्ठा यो यः प्र॑ति॒ष्ठा प्र॑ति॒ष्ठा य ए॒व मे॒वं ॅयः प्र॑ति॒ष्ठा प्र॑ति॒ष्ठा य ए॒वम् । \newline
23. प्र॒ति॒ष्ठेति॑ प्रति - स्था । \newline
24. य ए॒व मे॒वं ॅयो य ए॒वं ॅवेद॒ वेदै॒वं ॅयो य ए॒वं ॅवेद॑ । \newline
25. ए॒वं ॅवेद॒ वेदै॒व मे॒वं ॅवेद॒ प्रति॒ प्रति॒ वेदै॒व मे॒वं ॅवेद॒ प्रति॑ । \newline
26. वेद॒ प्रति॒ प्रति॒ वेद॒ वेद॒ प्रत्ये॒वैव प्रति॒ वेद॒ वेद॒ प्रत्ये॒व । \newline
27. प्रत्ये॒वैव प्रति॒ प्रत्ये॒व ति॑ष्ठति तिष्ठत्ये॒व प्रति॒ प्रत्ये॒व ति॑ष्ठति । \newline
28. ए॒व ति॑ष्ठति तिष्ठ त्ये॒वैव ति॑ष्ठति॒ यद् यत् ति॑ष्ठ त्ये॒वैव ति॑ष्ठति॒ यत् । \newline
29. ति॒ष्ठ॒ति॒ यद् यत् ति॑ष्ठति तिष्ठति॒ यद॑ भि॒क्रम्या॑ भि॒क्रम्य॒ यत् ति॑ष्ठति तिष्ठति॒ यद॑भि॒क्रम्य॑ । \newline
30. यद॑भि॒क्रम्या॑ भि॒क्रम्य॒ यद् यद॑भि॒क्रम्य॑ जुहु॒याज् जु॑हु॒या द॑भि॒क्रम्य॒ यद् यद॑भि॒क्रम्य॑ जुहु॒यात् । \newline
31. अ॒भि॒क्रम्य॑ जुहु॒याज् जु॑हु॒या द॑भि॒क्रम्या॑ भि॒क्रम्य॑ जुहु॒यात् प्र॑ति॒ष्ठायाः᳚ प्रति॒ष्ठाया॑ जुहु॒या द॑भि॒क्रम्या॑ भि॒क्रम्य॑ जुहु॒यात् प्र॑ति॒ष्ठायाः᳚ । \newline
32. अ॒भि॒क्रम्येत्य॑भि - क्रम्य॑ । \newline
33. जु॒हु॒यात् प्र॑ति॒ष्ठायाः᳚ प्रति॒ष्ठाया॑ जुहु॒याज् जु॑हु॒यात् प्र॑ति॒ष्ठाया॑ इया दियात् प्रति॒ष्ठाया॑ जुहु॒याज् जु॑हु॒यात् प्र॑ति॒ष्ठाया॑ इयात् । \newline
34. प्र॒ति॒ष्ठाया॑ इया दियात् प्रति॒ष्ठायाः᳚ प्रति॒ष्ठाया॑ इया॒त् तस्मा॒त् तस्मा॑ दियात् प्रति॒ष्ठायाः᳚ प्रति॒ष्ठाया॑ इया॒त् तस्मा᳚त् । \newline
35. प्र॒ति॒ष्ठाया॒ इति॑ प्रति - स्थायाः᳚ । \newline
36. इ॒या॒त् तस्मा॒त् तस्मा॑ दिया दिया॒त् तस्मा᳚थ् समा॒नत्र॑ समा॒नत्र॒ तस्मा॑ दिया दिया॒त् तस्मा᳚थ् समा॒नत्र॑ । \newline
37. तस्मा᳚थ् समा॒नत्र॑ समा॒नत्र॒ तस्मा॒त् तस्मा᳚थ् समा॒नत्र॒ तिष्ठ॑ता॒ तिष्ठ॑ता समा॒नत्र॒ तस्मा॒त् तस्मा᳚थ् समा॒नत्र॒ तिष्ठ॑ता । \newline
38. स॒मा॒नत्र॒ तिष्ठ॑ता॒ तिष्ठ॑ता समा॒नत्र॑ समा॒नत्र॒ तिष्ठ॑ता होत॒व्यꣳ॑ होत॒व्य॑म् तिष्ठ॑ता समा॒नत्र॑ समा॒नत्र॒ तिष्ठ॑ता होत॒व्य᳚म् । \newline
39. तिष्ठ॑ता होत॒व्यꣳ॑ होत॒व्य॑म् तिष्ठ॑ता॒ तिष्ठ॑ता होत॒व्य॑म् प्रति॑ष्ठित्यै॒ प्रति॑ष्ठित्यै होत॒व्य॑म् तिष्ठ॑ता॒ तिष्ठ॑ता होत॒व्य॑म् प्रति॑ष्ठित्यै । \newline
40. हो॒त॒व्य॑म् प्रति॑ष्ठित्यै॒ प्रति॑ष्ठित्यै होत॒व्यꣳ॑ होत॒व्य॑म् प्रति॑ष्ठित्यै॒ यो यः प्रति॑ष्ठित्यै होत॒व्यꣳ॑ होत॒व्य॑म् प्रति॑ष्ठित्यै॒ यः । \newline
41. प्रति॑ष्ठित्यै॒ यो यः प्रति॑ष्ठित्यै॒ प्रति॑ष्ठित्यै॒ यो वै वै यः प्रति॑ष्ठित्यै॒ प्रति॑ष्ठित्यै॒ यो वै । \newline
42. प्रति॑ष्ठित्या॒ इति॒ प्रति॑ - स्थि॒त्यै॒ । \newline
43. यो वै वै यो यो वा अ॑द्ध्व॒र्यो र॑द्ध्व॒र्योर् वै यो यो वा अ॑द्ध्व॒र्योः । \newline
44. वा अ॑द्ध्व॒र्यो र॑द्ध्व॒र्योर् वै वा अ॑द्ध्व॒र्योः स्वꣳ स्व म॑द्ध्व॒र्योर् वै वा अ॑द्ध्व॒र्योः स्वम् । \newline
45. अ॒द्ध्व॒र्योः स्वꣳ स्व म॑द्ध्व॒र्यो र॑द्ध्व॒र्योः स्वं ॅवेद॒ वेद॒ स्व म॑द्ध्व॒र्यो र॑द्ध्व॒र्योः स्वं ॅवेद॑ । \newline
46. स्वं ॅवेद॒ वेद॒ स्वꣳ स्वं ॅवेद॒ स्ववा॒न् थ्स्ववा॒न्॒. वेद॒ स्वꣳ स्वं ॅवेद॒ स्ववान्॑ । \newline
47. वेद॒ स्ववा॒न् थ्स्ववा॒न्॒. वेद॒ वेद॒ स्ववा॑ ने॒वैव स्ववा॒न्॒. वेद॒ वेद॒ स्ववा॑ ने॒व । \newline
48. स्ववा॑ ने॒वैव स्ववा॒न् थ्स्ववा॑ ने॒व भ॑वति भवत्ये॒व स्ववा॒न् थ्स्ववा॑ ने॒व भ॑वति । \newline
49. स्ववा॒निति॒ स्व - वा॒न् । \newline
50. ए॒व भ॑वति भव त्ये॒वैव भ॑वति॒ स्रुख् स्रुग् भ॑व त्ये॒वैव भ॑वति॒ स्रुक् । \newline
51. भ॒व॒ति॒ स्रुख् स्रुग् भ॑वति भवति॒ स्रुग् वै वै स्रुग् भ॑वति भवति॒ स्रुग् वै । \newline
52. स्रुग् वै वै स्रुख् स्रुग् वा अ॑स्यास्य॒ वै स्रुख् स्रुग् वा अ॑स्य । \newline
53. वा अ॑स्यास्य॒ वै वा अ॑स्य॒ स्वꣳ स्व म॑स्य॒ वै वा अ॑स्य॒ स्वम् । \newline
54. अ॒स्य॒ स्वꣳ स्व म॑स्यास्य॒ स्वं ॅवा॑य॒व्यं॑ ॅवाय॒व्यꣳ॑ स्व म॑स्यास्य॒ स्वं ॅवा॑य॒व्य᳚म् । \newline
55. स्वं ॅवा॑य॒व्यं॑ ॅवाय॒व्यꣳ॑ स्वꣳ स्वं ॅवा॑य॒व्य॑ मस्यास्य वाय॒व्यꣳ॑ स्वꣳ स्वं ॅवा॑य॒व्य॑ मस्य । \newline
56. वा॒य॒व्य॑ मस्यास्य वाय॒व्यं॑ ॅवाय॒व्य॑ मस्य॒ स्वꣳ स्व म॑स्य वाय॒व्यं॑ ॅवाय॒व्य॑ मस्य॒ स्वम् । \newline
57. अ॒स्य॒ स्वꣳ स्व म॑स्यास्य॒ स्वम् च॑म॒स श्च॑म॒सः स्व म॑स्यास्य॒ स्वम् च॑म॒सः । \newline
\pagebreak
\markright{ TS 3.1.2.4  \hfill https://www.vedavms.in \hfill}

\section{ TS 3.1.2.4 }

\textbf{TS 3.1.2.4 } \newline
\textbf{Samhita Paata} \newline

स्वं च॑म॒सो᳚ऽस्य॒ स्वं ॅयद्वा॑य॒व्यं॑ ॅवा चम॒सं ॅवाऽन॑न्वारभ्याऽऽश्रा॒वये॒थ् स्वादि॑या॒त् तस्मा॑ दन्वा॒रभ्या॒ ऽऽश्राव्यꣳ॒॒ स्वादे॒व नैति॒ यो वै सोम॒म- प्र॑तिष्ठाप्य स्तो॒त्र-मु॑पाक॒रोत्य प्र॑तिष्ठितः॒ सोमो॒ भव॒त्यप्र॑तिष्ठितः॒ स्तोमो- ऽप्र॑तिष्ठिता-न्यु॒क्थान्यप्र॑तिष्ठितो॒ यज॑मा॒नो ऽप्र॑तिष्ठितो ऽध्व॒र्युर्वा॑ य॒व्यं॑ ॅवै सोम॑स्य प्रति॒ष्ठा च॑म॒सो᳚ऽस्य प्रति॒ष्ठा सोमः॒ स्तोम॑स्य॒ स्तोम॑ उ॒क्थानां॒ ग्रहं॑ ॅवा ( ) गृही॒त्वा च॑म॒सं ॅवो॒न्नीय॑ स्तो॒त्रमु॒पा कु॑र्या॒त् प्रत्ये॒व सोमꣳ॑ स्था॒पय॑ति॒ प्रति॒स्तोमं॒ प्रत्यु॒क्थानि॒ प्रति॒ यज॑मान॒स्तिष्ठ॑ति॒ प्रत्य॑द्ध्व॒र्युः ॥ \newline

\textbf{Pada Paata} \newline

स्वम् । च॒म॒सः । अ॒स्य॒ । स्वम् । यत् । वा॒य॒व्य᳚म् । वा॒ । च॒म॒सम् । वा॒ । अन॑न्वार॒भ्येत्यन॑नु - आ॒र॒भ्य॒ । आ॒श्रा॒वये॒दित्या᳚ - श्रा॒वये᳚त् । स्वात् । इ॒या॒त् । तस्मा᳚त् । अ॒न्वा॒रभ्येत्य॑नु - आ॒रभ्य॑ । आ॒श्राव्य॒मित्या᳚ - श्राव्य᳚म् । स्वात् । ए॒व । न । ए॒ति॒ । यः । वै । सोम᳚म् । अप्र॑तिष्ठा॒प्येत्यप्र॑ति - स्था॒प्य॒ । स्तो॒त्रम् । उ॒पा॒क॒रोतीत्यु॑प - आ॒क॒रोति॑ । अप्र॑तिष्ठित॒ इत्यप्र॑ति - स्थि॒तः॒ । सोमः॑ । भव॑ति । अप्र॑तिष्ठित॒ इत्यप्र॑ति - स्थि॒तः॒ । स्तोमः॑ । अप्र॑तिष्ठिता॒नीत्यप्र॑ति - स्थि॒ता॒नि॒ । उ॒क्थानि॑ । अप्र॑तिष्ठित॒ इत्यप्र॑ति - स्थि॒तः॒ । यज॑मानः । अप्र॑तिष्ठित॒ इत्यप्र॑ति - स्थि॒तः॒ । अ॒ध्व॒र्युः । वा॒य॒व्य᳚म् । वै । सोम॑स्य । प्र॒ति॒ष्ठेति॑ प्रति - स्था । च॒म॒सः । अ॒स्य॒ । प्र॒ति॒ष्ठेति॑ प्रति - स्था । सोमः॑ । स्तोम॑स्य । स्तोमः॑ । उ॒क्थाना᳚म् । ग्रह᳚म् । वा॒ ( ) । गृ॒ही॒त्वा । च॒म॒सम् । वा॒ । उ॒न्नीयेत्यु॑त् - नीय॑ । स्तो॒त्रम् । उ॒पाकु॑र्या॒दित्यु॑प - आकु॑र्यात् । प्रतीति॑ । ए॒व । सोम᳚म् । स्था॒पय॑ति । प्रतीति॑ । स्तोम᳚म् । प्रतीति॑ । उ॒क्थानि॑ । प्रतीति॑ । यज॑मानः । तिष्ठ॑ति । प्रतीति॑ । अ॒द्ध्व॒र्युः ॥  \newline


\textbf{Krama Paata} \newline

स्वम् च॑म॒सः । च॒म॒सो᳚ ऽस्य । अ॒स्य॒ स्वम् । स्वं ॅयत् । यद् वा॑य॒व्य᳚म् । वा॒य॒व्यं॑ ॅवा । वा॒ च॒म॒सम् । च॒म॒सं ॅवा᳚ । वाऽन॑न्वारभ्य । अन॑न्वारभ्याश्रा॒वये᳚त् । अन॑न्वार॒भ्येत्यन॑नु - आ॒र॒भ्य॒ । आ॒श्रा॒वये॒थ् स्वात् । आ॒श्रा॒वये॒दित्या᳚ - श्रा॒वये᳚त् । स्वादि॑यात् । इ॒या॒त् तस्मा᳚त् । तस्मा॑दन्वा॒रभ्य॑ । अ॒न्वा॒रभ्या॒श्राव्य᳚म् । अ॒न्वा॒रभ्येत्य॑नु - आ॒रभ्य॑ । आ॒श्राव्यꣳ॒॒ स्वात् । आ॒श्राव्य॒मित्या᳚ - श्राव्य᳚म् । स्वादे॒व । ए॒व न । नैति॑ । ए॒ति॒ यः । यो वै । वै सोम᳚म् । सोम॒मप्र॑तिष्ठाप्य । अप्र॑तिष्ठाप्य स्तो॒त्रम् । अप्र॑तिष्ठा॒प्येत्यप्र॑ति - स्था॒प्य॒ । स्तो॒त्रमु॑पाक॒रोति॑ । उ॒पा॒क॒रोत्यप्र॑तिष्ठितः । उ॒पा॒क॒रोतीत्यु॑प - आ॒क॒रोति॑ । अप्र॑तिष्ठितः॒ सोमः॑ । अप्र॑तिष्ठित॒ इत्यप्र॑ति - स्थि॒तः॒ । सोमो॒ भव॑ति । भव॒त्यप्र॑तिष्ठितः । अप्र॑तिष्ठितः॒ स्तोमः॑ । अप्र॑तिष्ठित॒ इत्यप्र॑ति - स्थि॒तः॒ । स्तोमोऽप्र॑तिष्ठितानि । अप्र॑तिष्ठितान्यु॒क्थानि॑ । अप्र॑तिष्ठिता॒नीत्यप्र॑ति - स्थि॒ता॒नि॒ । उ॒क्थान्यप्र॑तिष्ठितः । अप्र॑तिष्ठतो॒ यज॑मानः । अप्र॑ष्ठित॒ इत्यप्र॑ति - स्थि॒तः॒ । यज॑मा॒नो ऽप्र॑तिष्ठितः । अप्र॑तिष्ठितोऽद्ध्व॒र्युः । अप्र॑तिष्ठित॒ इत्यप्र॑ति - स्थि॒तः॒ । अ॒द्ध्व॒र्युर् वा॑य॒व्य᳚म् । वा॒य॒व्यं॑ ॅवै । वै सोम॑स्य । सोम॑स्य प्रति॒ष्ठा । प्र॒ति॒ष्ठा च॑म॒सः । प्र॒ति॒ष्ठेति॑ प्रति - स्था । च॒म॒सो᳚ऽस्य । अ॒स्य॒ प्र॒ति॒ष्ठा । प्र॒ति॒ष्ठा सोमः॑ । प्र॒ति॒ष्ठेति॑ प्रति - स्था । सोमः॒ स्तोम॑स्य । स्तोम॑स्य॒ स्तोमः॑ । स्तोम॑ उ॒क्थाना᳚म् । उ॒क्थाना॒म् ग्रह᳚म् । ग्रहं॑ ॅवा ( ) । वा॒ गृ॒ही॒त्वा । गृ॒ही॒त्वा च॑म॒सम् । च॒म॒सं ॅवा᳚ । वो॒न्नीय॑ । उ॒न्नीय॑ स्तो॒त्रम् । उ॒न्नीयेत्यु॑त् - नीय॑ । स्तो॒त्रमु॒पाकु॑र्यात् । उ॒पाकु॑र्या॒त् प्रति॑ । उ॒पाकु॑र्या॒दित्यु॑प - आकु॑र्यात् । प्रत्ये॒व । ए॒व सोम᳚म् । सोमꣳ॑ स्था॒पय॑ति । स्था॒पय॑ति॒ प्रति॑ । प्रति॒ स्तोम᳚म् । स्तोम॒म् प्रति॑ । प्रत्यु॒क्थानि॑ । उ॒क्थानि॒ प्रति॑ । प्रति॒ यज॑मानः । यज॑मान॒स्तिष्ठ॑ति । तिष्ठ॑ति॒ प्रति॑ । प्रत्य॑द्ध्व॒र्युः । अ॒द्ध्व॒र्युरित्य॑द्ध्व॒र्युः । \newline

\textbf{Jatai Paata} \newline

1. स्वम् च॑म॒स श्च॑म॒सः स्वꣳ स्वम् च॑म॒सः । \newline
2. च॒म॒सो᳚ ऽस्यास्य चम॒स श्च॑म॒सो᳚ ऽस्य । \newline
3. अ॒स्य॒ स्वꣳ स्व म॑स्यास्य॒ स्वम् । \newline
4. स्वं ॅयद् यथ् स्वꣳ स्वं ॅयत् । \newline
5. यद् वा॑य॒व्यं॑ ॅवाय॒व्यं॑ ॅयद् यद् वा॑य॒व्य᳚म् । \newline
6. वा॒य॒व्यं॑ ॅवा वा वाय॒व्यं॑ ॅवाय॒व्यं॑ ॅवा । \newline
7. वा॒ च॒म॒सम् च॑म॒सं ॅवा॑ वा चम॒सम् । \newline
8. च॒म॒सं ॅवा॑ वा चम॒सम् च॑म॒सं ॅवा᳚ । \newline
9. वा ऽन॑न्वार॒भ्या न॑न्वारभ्य वा॒ वा ऽन॑न्वारभ्य । \newline
10. अन॑न्वारभ्या श्रा॒वये॑ दाश्रा॒वये॒ दन॑न्वार॒भ्या न॑न्वारभ्या श्रा॒वये᳚त् । \newline
11. अन॑न्वार॒भ्येत्यन॑नु - आ॒र॒भ्य॒ । \newline
12. आ॒श्रा॒वये॒थ् स्वाथ् स्वादा᳚श्रा॒वये॑ दाश्रा॒वये॒थ् स्वात् । \newline
13. आ॒श्रा॒वये॒दित्या᳚ - श्रा॒वये᳚त् । \newline
14. स्वा दि॑या दिया॒थ् स्वाथ् स्वा दि॑यात् । \newline
15. इ॒या॒त् तस्मा॒त् तस्मा॑ दिया दिया॒त् तस्मा᳚त् । \newline
16. तस्मा॑ दन्वा॒रभ्या᳚ न्वा॒रभ्य॒ तस्मा॒त् तस्मा॑ दन्वा॒रभ्य॑ । \newline
17. अ॒न्वा॒रभ्या॒ श्राव्य॑ मा॒श्राव्य॑ मन्वा॒रभ्या᳚ न्वा॒रभ्या॒ श्राव्य᳚म् । \newline
18. अ॒न्वा॒रभ्येत्य॑नु - आ॒रभ्य॑ । \newline
19. आ॒श्राव्यꣳ॒॒ स्वाथ् स्वा दा॒श्राव्य॑ मा॒श्राव्यꣳ॒॒ स्वात् । \newline
20. आ॒श्राव्य॒मित्या᳚ - श्राव्य᳚म् । \newline
21. स्वा दे॒वैव स्वाथ् स्वा दे॒व । \newline
22. ए॒व न नैवैव न । \newline
23. नैत्ये॑ति॒ न नैति॑ । \newline
24. ए॒ति॒ यो य ए᳚त्येति॒ यः । \newline
25. यो वै वै यो यो वै । \newline
26. वै सोमꣳ॒॒ सोमं॒ ॅवै वै सोम᳚म् । \newline
27. सोम॒ मप्र॑तिष्ठा॒प्या प्र॑तिष्ठाप्य॒ सोमꣳ॒॒ सोम॒ मप्र॑तिष्ठाप्य । \newline
28. अप्र॑तिष्ठाप्य स्तो॒त्रꣳ स्तो॒त्र मप्र॑तिष्ठा॒प्या प्र॑तिष्ठाप्य स्तो॒त्रम् । \newline
29. अप्र॑तिष्ठा॒प्येत्यप्र॑ति - स्था॒प्य॒ । \newline
30. स्तो॒त्र मु॑पाक॒रो त्यु॑पाक॒रोति॑ स्तो॒त्रꣳ स्तो॒त्र मु॑पाक॒रोति॑ । \newline
31. उ॒पा॒क॒रो त्यप्र॑तिष्ठि॒तो ऽप्र॑तिष्ठित उपाक॒रो त्यु॑पाक॒रो त्यप्र॑तिष्ठितः । \newline
32. उ॒पा॒क॒रोतीत्यु॑प - आ॒क॒रोति॑ । \newline
33. अप्र॑तिष्ठितः॒ सोमः॒ सोमो ऽप्र॑तिष्ठि॒तो ऽप्र॑तिष्ठितः॒ सोमः॑ । \newline
34. अप्र॑तिष्ठित॒ इत्यप्र॑ति - स्थि॒तः॒ । \newline
35. सोमो॒ भव॑ति॒ भव॑ति॒ सोमः॒ सोमो॒ भव॑ति । \newline
36. भव॒ त्यप्र॑तिष्ठि॒तो ऽप्र॑तिष्ठितो॒ भव॑ति॒ भव॒ त्यप्र॑तिष्ठितः । \newline
37. अप्र॑तिष्ठितः॒ स्तोमः॒ स्तोमो ऽप्र॑तिष्ठि॒तो ऽप्र॑तिष्ठितः॒ स्तोमः॑ । \newline
38. अप्र॑तिष्ठित॒ इत्यप्र॑ति - स्थि॒तः॒ । \newline
39. स्तोमो ऽप्र॑तिष्ठिता॒ न्यप्र॑तिष्ठितानि॒ स्तोमः॒ स्तोमो ऽप्र॑तिष्ठितानि । \newline
40. अप्र॑तिष्ठिता न्यु॒क्था न्यु॒क्था न्यप्र॑तिष्ठिता॒ न्यप्र॑तिष्ठिता न्यु॒क्थानि॑ । \newline
41. अप्र॑तिष्ठिता॒नीत्यप्र॑ति - स्थि॒ता॒नि॒ । \newline
42. उ॒क्था न्यप्र॑तिष्ठि॒तो ऽप्र॑तिष्ठित उ॒क्था न्यु॒क्था न्यप्र॑तिष्ठितः । \newline
43. अप्र॑तिष्ठितो॒ यज॑मानो॒ यज॑मा॒नो ऽप्र॑तिष्ठि॒तो ऽप्र॑तिष्ठितो॒ यज॑मानः । \newline
44. अप्र॑तिष्ठित॒ इत्यप्र॑ति - स्थि॒तः॒ । \newline
45. यज॑मा॒नो ऽप्र॑तिष्ठि॒तो ऽप्र॑तिष्ठितो॒ यज॑मानो॒ यज॑मा॒नो ऽप्र॑तिष्ठितः । \newline
46. अप्र॑तिष्ठितो ऽद्ध्व॒र्यु र॑द्ध्व॒र्यु रप्र॑तिष्ठि॒तो ऽप्र॑तिष्ठितो ऽद्ध्व॒र्युः । \newline
47. अप्र॑तिष्ठित॒ इत्यप्र॑ति - स्थि॒तः॒ । \newline
48. अ॒द्ध्व॒र्युर् वा॑य॒व्यं॑ ॅवाय॒व्य॑ मद्ध्व॒र्यु र॑द्ध्व॒र्युर् वा॑य॒व्य᳚म् । \newline
49. वा॒य॒व्यं॑ ॅवै वै वा॑य॒व्यं॑ ॅवाय॒व्यं॑ ॅवै । \newline
50. वै सोम॑स्य॒ सोम॑स्य॒ वै वै सोम॑स्य । \newline
51. सोम॑स्य प्रति॒ष्ठा प्र॑ति॒ष्ठा सोम॑स्य॒ सोम॑स्य प्रति॒ष्ठा । \newline
52. प्र॒ति॒ष्ठा च॑म॒स श्च॑म॒सः प्र॑ति॒ष्ठा प्र॑ति॒ष्ठा च॑म॒सः । \newline
53. प्र॒ति॒ष्ठेति॑ प्रति - स्था । \newline
54. च॒म॒सो᳚ ऽस्यास्य चम॒स श्च॑म॒सो᳚ ऽस्य । \newline
55. अ॒स्य॒ प्र॒ति॒ष्ठा प्र॑ति॒ष्ठा ऽस्या᳚स्य प्रति॒ष्ठा । \newline
56. प्र॒ति॒ष्ठा सोमः॒ सोमः॑ प्रति॒ष्ठा प्र॑ति॒ष्ठा सोमः॑ । \newline
57. प्र॒ति॒ष्ठेति॑ प्रति - स्था । \newline
58. सोमः॒ स्तोम॑स्य॒ स्तोम॑स्य॒ सोमः॒ सोमः॒ स्तोम॑स्य । \newline
59. स्तोम॑स्य॒ स्तोमः॒ स्तोमः॒ स्तोम॑स्य॒ स्तोम॑स्य॒ स्तोमः॑ । \newline
60. स्तोम॑ उ॒क्थाना॑ मु॒क्थानाꣳ॒॒ स्तोमः॒ स्तोम॑ उ॒क्थाना᳚म् । \newline
61. उ॒क्थाना॒म् ग्रह॒म् ग्रह॑ मु॒क्थाना॑ मु॒क्थाना॒म् ग्रह᳚म् । \newline
62. ग्रहं॑ ॅवा वा॒ ग्रह॒म् ग्रहं॑ ॅवा । \newline
63. वा॒ गृ॒ही॒त्वा गृ॑ही॒त्वा वा॑ वा गृही॒त्वा । \newline
64. गृ॒ही॒त्वा च॑म॒सम् च॑म॒सम् गृ॑ही॒त्वा गृ॑ही॒त्वा च॑म॒सम् । \newline
65. च॒म॒सं ॅवा॑ वा चम॒सम् च॑म॒सं ॅवा᳚ । \newline
66. वो॒न्नी यो॒न्नीय॑ वा वो॒न्नीय॑ । \newline
67. उ॒न्नीय॑ स्तो॒त्रꣳ स्तो॒त्र मु॒न्नी यो॒न्नीय॑ स्तो॒त्रम् । \newline
68. उ॒न्नीयेत्यु॑त् - नीय॑ । \newline
69. स्तो॒त्र मु॒पाकु॑र्या दु॒पाकु॑र्याथ् स्तो॒त्रꣳ स्तो॒त्र मु॒पाकु॑र्यात् । \newline
70. उ॒पाकु॑र्या॒त् प्रति॒ प्रत्यु॒पाकु॑र्या दु॒पाकु॑र्या॒त् प्रति॑ । \newline
71. उ॒पाकु॑र्या॒दित्यु॑प - आकु॑र्यात् । \newline
72. प्रत्ये॒वैव प्रति॒ प्रत्ये॒व । \newline
73. ए॒व सोमꣳ॒॒ सोम॑ मे॒वैव सोम᳚म् । \newline
74. सोमꣳ॑ स्था॒पय॑ति स्था॒पय॑ति॒ सोमꣳ॒॒ सोमꣳ॑ स्था॒पय॑ति । \newline
75. स्था॒पय॑ति॒ प्रति॒ प्रति॑ स्था॒पय॑ति स्था॒पय॑ति॒ प्रति॑ । \newline
76. प्रति॒ स्तोमꣳ॒॒ स्तोम॒म् प्रति॒ प्रति॒ स्तोम᳚म् । \newline
77. स्तोम॒म् प्रति॒ प्रति॒ स्तोमꣳ॒॒ स्तोम॒म् प्रति॑ । \newline
78. प्रत्यु॒क्था न्यु॒क्थानि॒ प्रति॒ प्रत्यु॒क्थानि॑ । \newline
79. उ॒क्थानि॒ प्रति॒ प्रत्यु॒क्था न्यु॒क्थानि॒ प्रति॑ । \newline
80. प्रति॒ यज॑मानो॒ यज॑मानः॒ प्रति॒ प्रति॒ यज॑मानः । \newline
81. यज॑मान॒ स्तिष्ठ॑ति॒ तिष्ठ॑ति॒ यज॑मानो॒ यज॑मान॒ स्तिष्ठ॑ति । \newline
82. तिष्ठ॑ति॒ प्रति॒ प्रति॒ तिष्ठ॑ति॒ तिष्ठ॑ति॒ प्रति॑ । \newline
83. प्रत्य॑द्ध्व॒र्यु र॑द्ध्व॒र्युः प्रति॒ प्रत्य॑द्ध्व॒र्युः । \newline
84. अ॒द्ध्व॒र्युरित्य॑द्ध्व॒र्युः । \newline

\textbf{Ghana Paata } \newline

1. स्वम् च॑म॒स श्च॑म॒सः स्वꣳ स्वम् च॑म॒सो᳚ ऽस्यास्य चम॒सः स्वꣳ स्वम् च॑म॒सो᳚ ऽस्य । \newline
2. च॒म॒सो᳚ ऽस्यास्य चम॒स श्च॑म॒सो᳚ ऽस्य॒ स्वꣳ स्व म॑स्य चम॒स श्च॑म॒सो᳚ ऽस्य॒ स्वम् । \newline
3. अ॒स्य॒ स्वꣳ स्व म॑स्यास्य॒ स्वं ॅयद् यथ् स्व म॑स्यास्य॒ स्वं ॅयत् । \newline
4. स्वं ॅयद् यथ् स्वꣳ स्वं ॅयद् वा॑य॒व्यं॑ ॅवाय॒व्यं॑ ॅयथ् स्वꣳ स्वं ॅयद् वा॑य॒व्य᳚म् । \newline
5. यद् वा॑य॒व्यं॑ ॅवाय॒व्यं॑ ॅयद् यद् वा॑य॒व्यं॑ ॅवा वा वाय॒व्यं॑ ॅयद् यद् वा॑य॒व्यं॑ ॅवा । \newline
6. वा॒य॒व्यं॑ ॅवा वा वाय॒व्यं॑ ॅवाय॒व्यं॑ ॅवा चम॒सम् च॑म॒सं ॅवा॑ वाय॒व्यं॑ ॅवाय॒व्यं॑ ॅवा चम॒सम् । \newline
7. वा॒ च॒म॒सम् च॑म॒सं ॅवा॑ वा चम॒सं ॅवा॑ वा चम॒सं ॅवा॑ वा चम॒सं ॅवा᳚ । \newline
8. च॒म॒सं ॅवा॑ वा चम॒सम् च॑म॒सं ॅवा ऽन॑न्वार॒भ्या न॑न्वारभ्य वा चम॒सम् च॑म॒सं ॅवा ऽन॑न्वारभ्य । \newline
9. वा ऽन॑न्वार॒भ्या न॑न्वारभ्य वा॒ वा ऽन॑न्वारभ्या श्रा॒वये॑ दाश्रा॒वये॒ दन॑न्वारभ्य वा॒ वा ऽन॑न्वारभ्या श्रा॒वये᳚त् । \newline
10. अन॑न्वारभ्या श्रा॒वये॑ दाश्रा॒वये॒ दन॑न्वार॒भ्या न॑न्वारभ्या श्रा॒वये॒थ् स्वाथ् स्वादा᳚श्रा॒वये॒ दन॑न्वार॒भ्या
न॑न्वारभ्या श्रा॒वये॒थ् स्वात् । \newline
11. अन॑न्वार॒भ्येत्यन॑नु - आ॒र॒भ्य॒ । \newline
12. आ॒श्रा॒वये॒थ् स्वाथ् स्वा दा᳚श्रा॒वये॑ दाश्रा॒वये॒थ् स्वा दि॑या दिया॒थ् स्वा दा᳚श्रा॒वये॑ दाश्रा॒वये॒थ् स्वा दि॑यात् । \newline
13. आ॒श्रा॒वये॒दित्या᳚ - श्रा॒वये᳚त् । \newline
14. स्वा दि॑या दिया॒थ् स्वाथ् स्वा दि॑या॒त् तस्मा॒त् तस्मा॑ दिया॒थ् स्वाथ् स्वा दि॑या॒त् तस्मा᳚त् । \newline
15. इ॒या॒त् तस्मा॒त् तस्मा॑ दिया दिया॒त् तस्मा॑ दन्वा॒रभ्या᳚ न्वा॒रभ्य॒ तस्मा॑ दिया दिया॒त् तस्मा॑ दन्वा॒रभ्य॑ । \newline
16. तस्मा॑ दन्वा॒रभ्या᳚ न्वा॒रभ्य॒ तस्मा॒त् तस्मा॑ दन्वा॒रभ्या॒ श्राव्य॑ मा॒श्राव्य॑ मन्वा॒रभ्य॒ तस्मा॒त् तस्मा॑ दन्वा॒रभ्या॒ श्राव्य᳚म् । \newline
17. अ॒न्वा॒रभ्या॒ श्राव्य॑ मा॒श्राव्य॑ मन्वा॒रभ्या᳚ न्वा॒रभ्या॒ श्राव्यꣳ॒॒ स्वाथ् स्वा दा॒श्राव्य॑ 
मन्वा॒रभ्या᳚ न्वा॒रभ्या॒ श्राव्यꣳ॒॒ स्वात् । \newline
18. अ॒न्वा॒रभ्येत्य॑नु - आ॒रभ्य॑ । \newline
19. आ॒श्राव्यꣳ॒॒ स्वाथ् स्वा दा॒श्राव्य॑ मा॒श्राव्यꣳ॒॒ स्वा दे॒वैव स्वा दा॒श्राव्य॑ मा॒श्राव्यꣳ॒॒ स्वा दे॒व । \newline
20. आ॒श्राव्य॒मित्या᳚ - श्राव्य᳚म् । \newline
21. स्वा दे॒वैव स्वाथ् स्वा दे॒व न नैव स्वाथ् स्वा दे॒व न । \newline
22. ए॒व न नैवैव नैत्ये॑ति॒ नैवैव नैति॑ । \newline
23. नैत्ये॑ति॒ न नैति॒ यो य ए॑ति॒ न नैति॒ यः । \newline
24. ए॒ति॒ यो य ए᳚त्येति॒ यो वै वै य ए᳚त्येति॒ यो वै । \newline
25. यो वै वै यो यो वै सोमꣳ॒॒ सोमं॒ ॅवै यो यो वै सोम᳚म् । \newline
26. वै सोमꣳ॒॒ सोमं॒ ॅवै वै सोम॒ मप्र॑तिष्ठा॒प्या प्र॑तिष्ठाप्य॒ सोमं॒ ॅवै वै सोम॒ मप्र॑तिष्ठाप्य । \newline
27. सोम॒ मप्र॑तिष्ठा॒प्या प्र॑तिष्ठाप्य॒ सोमꣳ॒॒ सोम॒ मप्र॑तिष्ठाप्य स्तो॒त्रꣳ स्तो॒त्र मप्र॑तिष्ठाप्य॒ सोमꣳ॒॒ सोम॒ मप्र॑तिष्ठाप्य स्तो॒त्रम् । \newline
28. अप्र॑तिष्ठाप्य स्तो॒त्रꣳ स्तो॒त्र मप्र॑तिष्ठा॒प्या प्र॑तिष्ठाप्य स्तो॒त्र मु॑पाक॒रो त्यु॑पाक॒रोति॑ 
स्तो॒त्र मप्र॑तिष्ठा॒प्या प्र॑तिष्ठाप्य स्तो॒त्र मु॑पाक॒रोति॑ । \newline
29. अप्र॑तिष्ठा॒प्येत्यप्र॑ति - स्था॒प्य॒ । \newline
30. स्तो॒त्र मु॑पाक॒रो त्यु॑पाक॒रोति॑ स्तो॒त्रꣳ स्तो॒त्र मु॑पाक॒रो त्यप्र॑तिष्ठि॒तो ऽप्र॑तिष्ठित उपाक॒रोति॑ स्तो॒त्रꣳ 
स्तो॒त्र मु॑पाक॒रो त्यप्र॑तिष्ठितः । \newline
31. उ॒पा॒क॒रो त्यप्र॑तिष्ठि॒तो ऽप्र॑तिष्ठित उपाक॒रो त्यु॑पाक॒रो त्यप्र॑तिष्ठितः॒ सोमः॒ सोमो ऽप्र॑तिष्ठित उपाक॒रो त्यु॑पाक॒रो त्यप्र॑तिष्ठितः॒ सोमः॑ । \newline
32. उ॒पा॒क॒रोतीत्यु॑प - आ॒क॒रोति॑ । \newline
33. अप्र॑तिष्ठितः॒ सोमः॒ सोमो ऽप्र॑तिष्ठि॒तो ऽप्र॑तिष्ठितः॒ सोमो॒ भव॑ति॒ भव॑ति॒ सोमो ऽप्र॑तिष्ठि॒तो ऽप्र॑तिष्ठितः॒ सोमो॒ भव॑ति । \newline
34. अप्र॑तिष्ठित॒ इत्यप्र॑ति - स्थि॒तः॒ । \newline
35. सोमो॒ भव॑ति॒ भव॑ति॒ सोमः॒ सोमो॒ भव॒ त्यप्र॑तिष्ठि॒तो ऽप्र॑तिष्ठितो॒ भव॑ति॒ सोमः॒ सोमो॒ भव॒ त्यप्र॑तिष्ठितः । \newline
36. भव॒ त्यप्र॑तिष्ठि॒तो ऽप्र॑तिष्ठितो॒ भव॑ति॒ भव॒ त्यप्र॑तिष्ठितः॒ स्तोमः॒ स्तोमो ऽप्र॑तिष्ठितो॒ भव॑ति॒ भव॒ त्यप्र॑तिष्ठितः॒ स्तोमः॑ । \newline
37. अप्र॑तिष्ठितः॒ स्तोमः॒ स्तोमो ऽप्र॑तिष्ठि॒तो ऽप्र॑तिष्ठितः॒ स्तोमो ऽप्र॑तिष्ठिता॒ न्यप्र॑तिष्ठितानि॒ स्तोमो ऽप्र॑तिष्ठि॒तो ऽप्र॑तिष्ठितः॒ स्तोमो ऽप्र॑तिष्ठितानि । \newline
38. अप्र॑तिष्ठित॒ इत्यप्र॑ति - स्थि॒तः॒ । \newline
39. स्तोमो ऽप्र॑तिष्ठिता॒ न्यप्र॑तिष्ठितानि॒ स्तोमः॒ स्तोमो ऽप्र॑तिष्ठिता न्यु॒क्था न्यु॒क्था न्यप्र॑तिष्ठितानि॒ स्तोमः॒ स्तोमो ऽप्र॑तिष्ठिता न्यु॒क्थानि॑ । \newline
40. अप्र॑तिष्ठिता न्यु॒क्था न्यु॒क्था न्यप्र॑तिष्ठिता॒ न्यप्र॑तिष्ठिता न्यु॒क्था न्यप्र॑तिष्ठि॒तो ऽप्र॑तिष्ठित उ॒क्था न्यप्र॑तिष्ठिता॒ न्यप्र॑तिष्ठिता न्यु॒क्था न्यप्र॑तिष्ठितः । \newline
41. अप्र॑तिष्ठिता॒नीत्यप्र॑ति - स्थि॒ता॒नि॒ । \newline
42. उ॒क्था न्यप्र॑तिष्ठि॒तो ऽप्र॑तिष्ठित उ॒क्था न्यु॒क्था न्यप्र॑तिष्ठितो॒ यज॑मानो॒ यज॑मा॒नो ऽप्र॑तिष्ठित उ॒क्था न्यु॒क्था न्यप्र॑तिष्ठितो॒ यज॑मानः । \newline
43. अप्र॑तिष्ठितो॒ यज॑मानो॒ यज॑मा॒नो ऽप्र॑तिष्ठि॒तो ऽप्र॑तिष्ठितो॒ यज॑मा॒नो ऽप्र॑तिष्ठि॒तो ऽप्र॑तिष्ठितो॒ यज॑मा॒नो ऽप्र॑तिष्ठि॒तो ऽप्र॑तिष्ठितो॒ यज॑मा॒नो ऽप्र॑तिष्ठितः । \newline
44. अप्र॑तिष्ठित॒ इत्यप्र॑ति - स्थि॒तः॒ । \newline
45. यज॑मा॒नो ऽप्र॑तिष्ठि॒तो ऽप्र॑तिष्ठितो॒ यज॑मानो॒ यज॑मा॒नो ऽप्र॑तिष्ठितो ऽद्ध्व॒र्यु र॑द्ध्व॒र्यु रप्र॑तिष्ठितो॒ यज॑मानो॒ यज॑मा॒नो ऽप्र॑तिष्ठितो ऽद्ध्व॒र्युः । \newline
46. अप्र॑तिष्ठितो ऽद्ध्व॒र्यु र॑द्ध्व॒र्यु रप्र॑तिष्ठि॒तो ऽप्र॑तिष्ठितो ऽद्ध्व॒र्युर् वा॑य॒व्यं॑ ॅवाय॒व्य॑ मद्ध्व॒र्यु रप्र॑तिष्ठि॒तो ऽप्र॑तिष्ठितो ऽद्ध्व॒र्युर् वा॑य॒व्य᳚म् । \newline
47. अप्र॑तिष्ठित॒ इत्यप्र॑ति - स्थि॒तः॒ । \newline
48. अ॒द्ध्व॒र्युर् वा॑य॒व्यं॑ ॅवाय॒व्य॑ मद्ध्व॒र्यु र॑द्ध्व॒र्युर् वा॑य॒व्यं॑ ॅवै वै वा॑य॒व्य॑ मद्ध्व॒र्यु र॑द्ध्व॒र्युर् वा॑य॒व्यं॑ ॅवै । \newline
49. वा॒य॒व्यं॑ ॅवै वै वा॑य॒व्यं॑ ॅवाय॒व्यं॑ ॅवै सोम॑स्य॒ सोम॑स्य॒ वै वा॑य॒व्यं॑ ॅवाय॒व्यं॑ ॅवै सोम॑स्य । \newline
50. वै सोम॑स्य॒ सोम॑स्य॒ वै वै सोम॑स्य प्रति॒ष्ठा प्र॑ति॒ष्ठा सोम॑स्य॒ वै वै सोम॑स्य प्रति॒ष्ठा । \newline
51. सोम॑स्य प्रति॒ष्ठा प्र॑ति॒ष्ठा सोम॑स्य॒ सोम॑स्य प्रति॒ष्ठा च॑म॒स श्च॑म॒सः प्र॑ति॒ष्ठा सोम॑स्य॒ सोम॑स्य प्रति॒ष्ठा च॑म॒सः । \newline
52. प्र॒ति॒ष्ठा च॑म॒स श्च॑म॒सः प्र॑ति॒ष्ठा प्र॑ति॒ष्ठा च॑म॒सो᳚ ऽस्यास्य चम॒सः प्र॑ति॒ष्ठा प्र॑ति॒ष्ठा च॑म॒सो᳚ ऽस्य । \newline
53. प्र॒ति॒ष्ठेति॑ प्रति - स्था । \newline
54. च॒म॒सो᳚ ऽस्यास्य चम॒स श्च॑म॒सो᳚ ऽस्य प्रति॒ष्ठा प्र॑ति॒ष्ठा ऽस्य॑ चम॒स श्च॑म॒सो᳚ ऽस्य प्रति॒ष्ठा । \newline
55. अ॒स्य॒ प्र॒ति॒ष्ठा प्र॑ति॒ष्ठा ऽस्या᳚स्य प्रति॒ष्ठा सोमः॒ सोमः॑ प्रति॒ष्ठा ऽस्या᳚स्य प्रति॒ष्ठा सोमः॑ । \newline
56. प्र॒ति॒ष्ठा सोमः॒ सोमः॑ प्रति॒ष्ठा प्र॑ति॒ष्ठा सोमः॒ स्तोम॑स्य॒ स्तोम॑स्य॒ सोमः॑ प्रति॒ष्ठा प्र॑ति॒ष्ठा सोमः॒ स्तोम॑स्य । \newline
57. प्र॒ति॒ष्ठेति॑ प्रति - स्था । \newline
58. सोमः॒ स्तोम॑स्य॒ स्तोम॑स्य॒ सोमः॒ सोमः॒ स्तोम॑स्य॒ स्तोमः॒ स्तोमः॒ स्तोम॑स्य॒ सोमः॒ सोमः॒ स्तोम॑स्य॒ स्तोमः॑ । \newline
59. स्तोम॑स्य॒ स्तोमः॒ स्तोमः॒ स्तोम॑स्य॒ स्तोम॑स्य॒ स्तोम॑ उ॒क्थाना॑ मु॒क्थानाꣳ॒॒ स्तोमः॒ स्तोम॑स्य॒ स्तोम॑स्य॒ स्तोम॑ उ॒क्थाना᳚म् । \newline
60. स्तोम॑ उ॒क्थाना॑ मु॒क्थानाꣳ॒॒ स्तोमः॒ स्तोम॑ उ॒क्थाना॒म् ग्रह॒म् ग्रह॑ मु॒क्थानाꣳ॒॒ स्तोमः॒ स्तोम॑ उ॒क्थाना॒म् ग्रह᳚म् । \newline
61. उ॒क्थाना॒म् ग्रह॒म् ग्रह॑ मु॒क्थाना॑ मु॒क्थाना॒म् ग्रहं॑ ॅवा वा॒ ग्रह॑ मु॒क्थाना॑ मु॒क्थाना॒म् ग्रहं॑ ॅवा । \newline
62. ग्रहं॑ ॅवा वा॒ ग्रह॒म् ग्रहं॑ ॅवा गृही॒त्वा गृ॑ही॒त्वा वा॒ ग्रह॒म् ग्रहं॑ ॅवा गृही॒त्वा । \newline
63. वा॒ गृ॒ही॒त्वा गृ॑ही॒त्वा वा॑ वा गृही॒त्वा च॑म॒सम् च॑म॒सम् गृ॑ही॒त्वा वा॑ वा गृही॒त्वा च॑म॒सम् । \newline
64. गृ॒ही॒त्वा च॑म॒सम् च॑म॒सम् गृ॑ही॒त्वा गृ॑ही॒त्वा च॑म॒सं ॅवा॑ वा चम॒सम् गृ॑ही॒त्वा गृ॑ही॒त्वा च॑म॒सं ॅवा᳚ । \newline
65. च॒म॒सं ॅवा॑ वा चम॒सम् च॑म॒सं ॅवो॒न्नी यो॒न्नीय॑ वा चम॒सम् च॑म॒सं ॅवो॒न्नीय॑ । \newline
66. वो॒न्नी यो॒न्नीय॑ वा वो॒न्नीय॑ स्तो॒त्रꣳ स्तो॒त्र मु॒न्नीय॑ वा वो॒न्नीय॑ स्तो॒त्रम् । \newline
67. उ॒न्नीय॑ स्तो॒त्रꣳ स्तो॒त्र मु॒न्नी यो॒न्नीय॑ स्तो॒त्र मु॒पाकु॑र्या दु॒पाकु॑र्याथ् स्तो॒त्र मु॒न्नी यो॒न्नीय॑ स्तो॒त्र मु॒पाकु॑र्यात् । \newline
68. उ॒न्नीयेत्यु॑त् - नीय॑ । \newline
69. स्तो॒त्र मु॒पाकु॑र्या दु॒पाकु॑र्याथ् स्तो॒त्रꣳ स्तो॒त्र मु॒पाकु॑र्या॒त् प्रति॒ प्रत्यु॒पाकु॑र्याथ् स्तो॒त्रꣳ स्तो॒त्र मु॒पाकु॑र्या॒त् प्रति॑ । \newline
70. उ॒पाकु॑र्या॒त् प्रति॒ प्रत्यु॒पाकु॑र्या दु॒पाकु॑र्या॒त् प्रत्ये॒वैव प्रत्यु॒पाकु॑र्या दु॒पाकु॑र्या॒त् प्रत्ये॒व । \newline
71. उ॒पाकु॑र्या॒दित्यु॑प - आकु॑र्यात् । \newline
72. प्रत्ये॒वैव प्रति॒ प्रत्ये॒व सोमꣳ॒॒ सोम॑ मे॒व प्रति॒ प्रत्ये॒व सोम᳚म् । \newline
73. ए॒व सोमꣳ॒॒ सोम॑ मे॒वैव सोमꣳ॑ स्था॒पय॑ति स्था॒पय॑ति॒ सोम॑ मे॒वैव सोमꣳ॑ स्था॒पय॑ति । \newline
74. सोमꣳ॑ स्था॒पय॑ति स्था॒पय॑ति॒ सोमꣳ॒॒ सोमꣳ॑ स्था॒पय॑ति॒ प्रति॒ प्रति॑ स्था॒पय॑ति॒ सोमꣳ॒॒ सोमꣳ॑ स्था॒पय॑ति॒ प्रति॑ । \newline
75. स्था॒पय॑ति॒ प्रति॒ प्रति॑ स्था॒पय॑ति स्था॒पय॑ति॒ प्रति॒ स्तोमꣳ॒॒ स्तोम॒म् प्रति॑ स्था॒पय॑ति स्था॒पय॑ति॒ प्रति॒ स्तोम᳚म् । \newline
76. प्रति॒ स्तोमꣳ॒॒ स्तोम॒म् प्रति॒ प्रति॒ स्तोम॒म् प्रति॒ प्रति॒ स्तोम॒म् प्रति॒ प्रति॒ स्तोम॒म् प्रति॑ । \newline
77. स्तोम॒म् प्रति॒ प्रति॒ स्तोमꣳ॒॒ स्तोम॒म् प्रत्यु॒क्था न्यु॒क्थानि॒ प्रति॒ स्तोमꣳ॒॒ स्तोम॒म् प्रत्यु॒क्थानि॑ । \newline
78. प्रत्यु॒क्था न्यु॒क्थानि॒ प्रति॒ प्रत्यु॒क्थानि॒ प्रति॒ प्रत्यु॒क्थानि॒ प्रति॒ प्रत्यु॒क्थानि॒ प्रति॑ । \newline
79. उ॒क्थानि॒ प्रति॒ प्रत्यु॒क्था न्यु॒क्थानि॒ प्रति॒ यज॑मानो॒ यज॑मानः॒ प्रत्यु॒क्था न्यु॒क्थानि॒ प्रति॒ यज॑मानः । \newline
80. प्रति॒ यज॑मानो॒ यज॑मानः॒ प्रति॒ प्रति॒ यज॑मान॒ स्तिष्ठ॑ति॒ तिष्ठ॑ति॒ यज॑मानः॒ प्रति॒ प्रति॒ यज॑मान॒ स्तिष्ठ॑ति । \newline
81. यज॑मान॒ स्तिष्ठ॑ति॒ तिष्ठ॑ति॒ यज॑मानो॒ यज॑मान॒ स्तिष्ठ॑ति॒ प्रति॒ प्रति॒ तिष्ठ॑ति॒ यज॑मानो॒ यज॑मान॒ स्तिष्ठ॑ति॒ प्रति॑ । \newline
82. तिष्ठ॑ति॒ प्रति॒ प्रति॒ तिष्ठ॑ति॒ तिष्ठ॑ति॒ प्रत्य॑द्ध्व॒र्यु र॑द्ध्व॒र्युः प्रति॒ तिष्ठ॑ति॒ तिष्ठ॑ति॒ प्रत्य॑द्ध्व॒र्युः । \newline
83. प्रत्य॑द्ध्व॒र्यु र॑द्ध्व॒र्युः प्रति॒ प्रत्य॑द्ध्व॒र्युः । \newline
84. अ॒द्ध्व॒र्युरित्य॑द्ध्व॒र्युः । \newline
\pagebreak
\markright{ TS 3.1.3.1  \hfill https://www.vedavms.in \hfill}

\section{ TS 3.1.3.1 }

\textbf{TS 3.1.3.1 } \newline
\textbf{Samhita Paata} \newline

य॒ज्ञ्ं ॅवा ए॒तथ् सं भ॑रन्ति॒ यथ् सो॑म॒क्रय॑ण्यै प॒दं ॅय॑ज्ञ्मु॒खꣳ ह॑वि॒र्द्धाने॒ यर्.हि॑ हवि॒र्द्धाने॒ प्राची᳚ प्रव॒र्तये॑यु॒स्तर्.हि॒ तेनाक्ष॒मुपा᳚-ञ्ज्याद्-यज्ञ्मु॒ख ए॒व य॒ज्ञ्मनु॒ सन्त॑नोति॒ प्राञ्च॑म॒ग्निं प्र ह॑र॒न्त्युत् पत्नी॒मा न॑य॒न्त्यन्वनाꣳ॑सि॒ प्र व॑र्तय॒न्त्यथ॒ वा अ॑स्यै॒ष धिष्णि॑यो हीयते॒ सोऽनु॑ ध्यायति॒ स ई᳚श्व॒रो रु॒द्रो भू॒त्वा - [  ] \newline

\textbf{Pada Paata} \newline

य॒ज्ञ्म् । वै । ए॒तत् । समिति॑ । भ॒र॒न्ति॒ । यत् । सो॒म॒क्रय॑ण्या॒ इति॑ सोम - क्रय॑ण्यै । प॒दम् । य॒ज्ञ्॒मु॒खमिति॑ यज्ञ् - मु॒खम् । ह॒वि॒द्‌र्धाने॒ इति॑ हविः - धाने᳚ । यर्.हि॑ । ह॒वि॒द्‌र्धाने॒ इति॑ हविः - धाने᳚ । प्राची॒ इति॑ । प्र॒व॒र्तये॑यु॒रिति॑ प्र - व॒र्तये॑युः । तर्.हि॑ । तेन॑ । अक्ष᳚म् । उपेति॑ । अ॒ञ्ज्या॒त् । य॒ज्ञ्॒मु॒ख इति॑ यज्ञ्-मु॒खे । ए॒व । य॒ज्ञ्म् । अनु॑ । समिति॑ । त॒नो॒ति॒ । प्राञ्च᳚म् । अ॒ग्निम् । प्रेति॑ । ह॒र॒न्ति॒ । उदिति॑ । पत्नी᳚म् । एति॑ । न॒य॒न्ति॒ । अन्विति॑ । अनाꣳ॑सि । प्रेति॑ । व॒र्त॒य॒न्ति॒ । अथ॑ । वै । अ॒स्य॒ । ए॒षः । धिष्णि॑यः । ही॒य॒ते॒ । सः । अन्विति॑ । ध्या॒य॒ति॒ । सः । ई॒श्व॒रः । रु॒द्रः । भू॒त्वा ।  \newline


\textbf{Krama Paata} \newline

य॒ज्ञ्ं ॅवै । वा ए॒तत् । ए॒तथ् सम् । सम् भ॑रन्ति । भ॒र॒न्ति॒ यत् । यथ् सो॑म॒क्रय॑ण्यै । सो॒म॒क्रय॑ण्यै प॒दम् । सो॒म॒क्रय॑ण्या॒ इति॑ सोम - क्रय॑ण्यै । प॒दं ॅय॑ज्ञ्मु॒खम् । य॒ज्ञ्॒मु॒खꣳ ह॑वि॒र्द्धाने᳚ । य॒ज्ञ्॒मु॒खमिति॑ यज्ञ् - मु॒खम् । ह॒वि॒र्द्धाने॒ यर्.हि॑ । ह॒वि॒र्द्धाने॒ इति॑ हविः - धाने᳚ । यर्.हि॑ हवि॒र्द्धाने᳚ । ह॒वि॒र्द्धाने॒ प्राची᳚ । ह॒वि॒र्द्धाने॒ इति॑ हविः - धाने᳚ । प्राची᳚ प्रव॒र्तये॑युः । प्राची॒ इति॒ प्राची᳚ । प्र॒व॒र्तये॑यु॒स्तर्.हि॑ । प्र॒व॒र्तये॑यु॒रिति॑ प्र - व॒र्तये॑युः । तर्.हि॒ तेन॑ । तेनाक्ष᳚म् । अक्ष॒मुप॑ । उपा᳚ञ्ज्यात् । आ॒ञ्ज्या॒द् य॒ज्ञ्॒मु॒खे । य॒ज्ञ्॒मु॒ख ए॒व । य॒ज्ञ्॒मु॒ख इति॑ यज्ञ् - मु॒खे । ए॒व य॒ज्ञ्म् । य॒ज्ञ्मनु॑ । अनु॒ सम् । सम् त॑नोति । त॒नो॒ति॒ प्राञ्च᳚म् । प्राञ्च॑म॒ग्निम् । अ॒ग्निम् प्र । प्र ह॑रन्ति । ह॒र॒न्त्युत् । उत् पत्नी᳚म् । पत्नी॒मा । आ न॑यन्ति । न॒य॒न्त्यनु॑ । अन्वनाꣳ॑सि । अनाꣳ॑सि॒ प्र । प्र व॑र्तयन्ति । व॒र्त॒य॒न्त्यथ॑ । अथ॒ वै । वा अ॑स्य । अ॒स्यै॒षः । ए॒ष धिष्णि॑यः । धिष्णि॑यो हीयते । ही॒य॒ते॒ सः । सो ऽनु॑ । अनु॑ ध्यायति । ध्या॒य॒ति॒ सः । स ई᳚श्व॒रः । ई॒श्व॒रो रु॒द्रः । रु॒द्रो भू॒त्वा । भू॒त्वा प्र॒जाम् \newline

\textbf{Jatai Paata} \newline

1. य॒ज्ञ्ं ॅवै वै य॒ज्ञ्ं ॅय॒ज्ञ्ं ॅवै । \newline
2. वा ए॒त दे॒तद् वै वा ए॒तत् । \newline
3. ए॒तथ् सꣳ स मे॒त दे॒तथ् सम् । \newline
4. सम् भ॑रन्ति भरन्ति॒ सꣳ सम् भ॑रन्ति । \newline
5. भ॒र॒न्ति॒ यद् यद् भ॑रन्ति भरन्ति॒ यत् । \newline
6. यथ् सो॑म॒क्रय॑ण्यै सोम॒क्रय॑ण्यै॒ यद् यथ् सो॑म॒क्रय॑ण्यै । \newline
7. सो॒म॒क्रय॑ण्यै प॒दम् प॒दꣳ सो॑म॒क्रय॑ण्यै सोम॒क्रय॑ण्यै प॒दम् । \newline
8. सो॒म॒क्रय॑ण्या॒ इति॑ सोम - क्रय॑ण्यै । \newline
9. प॒दं ॅय॑ज्ञ्मु॒खं ॅय॑ज्ञ्मु॒खम् प॒दम् प॒दं ॅय॑ज्ञ्मु॒खम् । \newline
10. य॒ज्ञ्॒मु॒खꣳ ह॑वि॒र्द्धाने॑ हवि॒र्द्धाने॑ यज्ञ्मु॒खं ॅय॑ज्ञ्मु॒खꣳ ह॑वि॒र्द्धाने᳚ । \newline
11. य॒ज्ञ्॒मु॒खमिति॑ यज्ञ् - मु॒खम् । \newline
12. ह॒वि॒र्द्धाने॒ यर्.हि॒ यर्.हि॑ हवि॒र्द्धाने॑ हवि॒र्द्धाने॒ यर्.हि॑ । \newline
13. ह॒वि॒र्द्धाने॒ इति॑ हविः - धाने᳚ । \newline
14. यर्.हि॑ हवि॒र्द्धाने॑ हवि॒र्द्धाने॒ यर्.हि॒ यर्.हि॑ हवि॒र्द्धाने᳚ । \newline
15. ह॒वि॒र्द्धाने॒ प्राची॒ प्राची॑ हवि॒र्द्धाने॑ हवि॒र्द्धाने॒ प्राची᳚ । \newline
16. ह॒वि॒र्द्धाने॒ इति॑ हविः - धाने᳚ । \newline
17. प्राची᳚ प्रव॒र्तये॑युः प्रव॒र्तये॑युः॒ प्राची॒ प्राची᳚ प्रव॒र्तये॑युः । \newline
18. प्राची॒ इति॒ प्राची᳚ । \newline
19. प्र॒व॒र्तये॑यु॒ स्तर्.हि॒ तर्.हि॑ प्रव॒र्तये॑युः प्रव॒र्तये॑यु॒ स्तर्.हि॑ । \newline
20. प्र॒व॒र्तये॑यु॒रिति॑ प्र - व॒र्तये॑युः । \newline
21. तर्.हि॒ तेन॒ तेन॒ तर्.हि॒ तर्.हि॒ तेन॑ । \newline
22. तेनाक्ष॒ मक्ष॒म् तेन॒ तेनाक्ष᳚म् । \newline
23. अक्ष॒ मुपोपाक्ष॒ मक्ष॒ मुप॑ । \newline
24. उपा᳚ञ्ज्या दञ्ज्या॒ दुपोपा᳚ञ्ज्यात् । \newline
25. अ॒ञ्ज्या॒द् य॒ज्ञ्॒मु॒खे य॑ज्ञ्मु॒खे᳚ ऽञ्ज्या दञ्ज्याद् यज्ञ्मु॒खे । \newline
26. य॒ज्ञ्॒मु॒ख ए॒वैव य॑ज्ञ्मु॒खे य॑ज्ञ्मु॒ख ए॒व । \newline
27. य॒ज्ञ्॒मु॒ख इति॑ यज्ञ् - मु॒खे । \newline
28. ए॒व य॒ज्ञ्ं ॅय॒ज्ञ् मे॒वैव य॒ज्ञ्म् । \newline
29. य॒ज्ञ् मन्वनु॑ य॒ज्ञ्ं ॅय॒ज्ञ् मनु॑ । \newline
30. अनु॒ सꣳ स मन्वनु॒ सम् । \newline
31. सम् त॑नोति तनोति॒ सꣳ सम् त॑नोति । \newline
32. त॒नो॒ति॒ प्राञ्च॒म् प्राञ्च॑म् तनोति तनोति॒ प्राञ्च᳚म् । \newline
33. प्राञ्च॑ म॒ग्नि म॒ग्निम् प्राञ्च॒म् प्राञ्च॑ म॒ग्निम् । \newline
34. अ॒ग्निम् प्र प्राग्नि म॒ग्निम् प्र । \newline
35. प्र ह॑रन्ति हरन्ति॒ प्र प्र ह॑रन्ति । \newline
36. ह॒र॒ न्त्युदु द्ध॑रन्ति हर॒ न्त्युत् । \newline
37. उत् पत्नी॒म् पत्नी॒ मुदुत् पत्नी᳚म् । \newline
38. पत्नी॒ मा पत्नी॒म् पत्नी॒ मा । \newline
39. आ न॑यन्ति नय॒न्त्या न॑यन्ति । \newline
40. न॒य॒ न्त्यन्वनु॑ नयन्ति नय॒ न्त्यनु॑ । \newline
41. अन्वनाꣳ॒॒ स्यनाꣳ॒॒ स्यन्व न्वनाꣳ॑सि । \newline
42. अनाꣳ॑सि॒ प्र प्राणाꣳ॒॒ स्यनाꣳ॑सि॒ प्र । \newline
43. प्र व॑र्तयन्ति वर्तयन्ति॒ प्र प्र व॑र्तयन्ति । \newline
44. व॒र्त॒य॒ न्त्यथाथ॑ वर्तयन्ति वर्तय॒ न्त्यथ॑ । \newline
45. अथ॒ वै वा अथाथ॒ वै । \newline
46. वा अ॑स्यास्य॒ वै वा अ॑स्य । \newline
47. अ॒स्यै॒ष ए॒षो᳚ ऽस्या स्यै॒षः । \newline
48. ए॒ष धिष्णि॑यो॒ धिष्णि॑य ए॒ष ए॒ष धिष्णि॑यः । \newline
49. धिष्णि॑यो हीयते हीयते॒ धिष्णि॑यो॒ धिष्णि॑यो हीयते । \newline
50. ही॒य॒ते॒ स स ही॑यते हीयते॒ सः । \newline
51. सो ऽन्वनु॒ स सो ऽनु॑ । \newline
52. अनु॑ ध्यायति ध्याय॒ त्यन्वनु॑ ध्यायति । \newline
53. ध्या॒य॒ति॒ स स ध्या॑यति ध्यायति॒ सः । \newline
54. स ई᳚श्व॒र ई᳚श्व॒रः स स ई᳚श्व॒रः । \newline
55. ई॒श्व॒रो रु॒द्रो रु॒द्र ई᳚श्व॒र ई᳚श्व॒रो रु॒द्रः । \newline
56. रु॒द्रो भू॒त्वा भू॒त्वा रु॒द्रो रु॒द्रो भू॒त्वा । \newline
57. भू॒त्वा प्र॒जाम् प्र॒जाम् भू॒त्वा भू॒त्वा प्र॒जाम् । \newline

\textbf{Ghana Paata } \newline

1. य॒ज्ञ्ं ॅवै वै य॒ज्ञ्ं ॅय॒ज्ञ्ं ॅवा ए॒त दे॒तद् वै य॒ज्ञ्ं ॅय॒ज्ञ्ं ॅवा ए॒तत् । \newline
2. वा ए॒त दे॒तद् वै वा ए॒तथ् सꣳ स मे॒तद् वै वा ए॒तथ् सम् । \newline
3. ए॒तथ् सꣳ स मे॒त दे॒तथ् सम् भ॑रन्ति भरन्ति॒ स मे॒त दे॒तथ् सम् भ॑रन्ति । \newline
4. सम् भ॑रन्ति भरन्ति॒ सꣳ सम् भ॑रन्ति॒ यद् यद् भ॑रन्ति॒ सꣳ सम् भ॑रन्ति॒ यत् । \newline
5. भ॒र॒न्ति॒ यद् यद् भ॑रन्ति भरन्ति॒ यथ् सो॑म॒क्रय॑ण्यै सोम॒क्रय॑ण्यै॒ यद् भ॑रन्ति भरन्ति॒ यथ् सो॑म॒क्रय॑ण्यै । \newline
6. यथ् सो॑म॒क्रय॑ण्यै सोम॒क्रय॑ण्यै॒ यद् यथ् सो॑म॒क्रय॑ण्यै प॒दम् प॒दꣳ सो॑म॒क्रय॑ण्यै॒ यद् यथ् सो॑म॒क्रय॑ण्यै प॒दम् । \newline
7. सो॒म॒क्रय॑ण्यै प॒दम् प॒दꣳ सो॑म॒क्रय॑ण्यै सोम॒क्रय॑ण्यै प॒दं ॅय॑ज्ञ्मु॒खं ॅय॑ज्ञ्मु॒खम् प॒दꣳ सो॑म॒क्रय॑ण्यै सोम॒क्रय॑ण्यै प॒दं ॅय॑ज्ञ्मु॒खम् । \newline
8. सो॒म॒क्रय॑ण्या॒ इति॑ सोम - क्रय॑ण्यै । \newline
9. प॒दं ॅय॑ज्ञ्मु॒खं ॅय॑ज्ञ्मु॒खम् प॒दम् प॒दं ॅय॑ज्ञ्मु॒खꣳ ह॑वि॒र्द्धाने॑ हवि॒र्द्धाने॑ यज्ञ्मु॒खम् प॒दम् प॒दं ॅय॑ज्ञ्मु॒खꣳ ह॑वि॒र्द्धाने᳚ । \newline
10. य॒ज्ञ्॒मु॒खꣳ ह॑वि॒र्द्धाने॑ हवि॒र्द्धाने॑ यज्ञ्मु॒खं ॅय॑ज्ञ्मु॒खꣳ ह॑वि॒र्द्धाने॒ यर्.हि॒ यर्.हि॑ हवि॒र्द्धाने॑ यज्ञ्मु॒खं ॅय॑ज्ञ्मु॒खꣳ ह॑वि॒र्द्धाने॒ यर्.हि॑ । \newline
11. य॒ज्ञ्॒मु॒खमिति॑ यज्ञ् - मु॒खम् । \newline
12. ह॒वि॒र्द्धाने॒ यर्.हि॒ यर्.हि॑ हवि॒र्द्धाने॑ हवि॒र्द्धाने॒ यर्.हि॑ हवि॒र्द्धाने॑ हवि॒र्द्धाने॒ यर्.हि॑ हवि॒र्द्धाने॑ हवि॒र्द्धाने॒ यर्.हि॑ हवि॒र्द्धाने᳚ । \newline
13. ह॒वि॒र्द्धाने॒ इति॑ हविः - धाने᳚ । \newline
14. यर्.हि॑ हवि॒र्द्धाने॑ हवि॒र्द्धाने॒ यर्.हि॒ यर्.हि॑ हवि॒र्द्धाने॒ प्राची॒ प्राची॑ हवि॒र्द्धाने॒ यर्.हि॒ यर्.हि॑ हवि॒र्द्धाने॒ प्राची᳚ । \newline
15. ह॒वि॒र्द्धाने॒ प्राची॒ प्राची॑ हवि॒र्द्धाने॑ हवि॒र्द्धाने॒ प्राची᳚ प्रव॒र्तये॑युः प्रव॒र्तये॑युः॒ प्राची॑ हवि॒र्द्धाने॑ हवि॒र्द्धाने॒ प्राची᳚ प्रव॒र्तये॑युः । \newline
16. ह॒वि॒र्द्धाने॒ इति॑ हविः - धाने᳚ । \newline
17. प्राची᳚ प्रव॒र्तये॑युः प्रव॒र्तये॑युः॒ प्राची॒ प्राची᳚ प्रव॒र्तये॑यु॒ स्तर्.हि॒ तर्.हि॑ प्रव॒र्तये॑युः॒ प्राची॒ प्राची᳚ प्रव॒र्तये॑यु॒ स्तर्.हि॑ । \newline
18. प्राची॒ इति॒ प्राची᳚ । \newline
19. प्र॒व॒र्तये॑यु॒ स्तर्.हि॒ तर्.हि॑ प्रव॒र्तये॑युः प्रव॒र्तये॑यु॒ स्तर्.हि॒ तेन॒ तेन॒ तर्.हि॑ प्रव॒र्तये॑युः प्रव॒र्तये॑यु॒ स्तर्.हि॒ तेन॑ । \newline
20. प्र॒व॒र्तये॑यु॒रिति॑ प्र - व॒र्तये॑युः । \newline
21. तर्.हि॒ तेन॒ तेन॒ तर्.हि॒ तर्.हि॒ तेनाक्ष॒ मक्ष॒म् तेन॒ तर्.हि॒ तर्.हि॒ तेनाक्ष᳚म् । \newline
22. तेनाक्ष॒ मक्ष॒म् तेन॒ तेनाक्ष॒ मुपोपाक्ष॒म् तेन॒ तेनाक्ष॒ मुप॑ । \newline
23. अक्ष॒ मुपोपाक्ष॒ मक्ष॒ मुपा᳚ञ्ज्या दञ्ज्या॒ दुपाक्ष॒ मक्ष॒ मुपा᳚ञ्ज्यात् । \newline
24. उपा᳚ञ्ज्या दञ्ज्या॒ दुपोपा᳚ञ्ज्याद् यज्ञ्मु॒खे य॑ज्ञ्मु॒खे᳚ ऽञ्ज्या॒ दुपोपा᳚ञ्ज्याद् यज्ञ्मु॒खे । \newline
25. अ॒ञ्ज्या॒द् य॒ज्ञ्॒मु॒खे य॑ज्ञ्मु॒खे᳚ ऽञ्ज्या दञ्ज्याद् यज्ञ्मु॒ख ए॒वैव य॑ज्ञ्मु॒खे᳚ ऽञ्ज्या दञ्ज्याद् यज्ञ्मु॒ख ए॒व । \newline
26. य॒ज्ञ्॒मु॒ख ए॒वैव य॑ज्ञ्मु॒खे य॑ज्ञ्मु॒ख ए॒व य॒ज्ञ्ं ॅय॒ज्ञ् मे॒व य॑ज्ञ्मु॒खे य॑ज्ञ्मु॒ख ए॒व य॒ज्ञ्म् । \newline
27. य॒ज्ञ्॒मु॒ख इति॑ यज्ञ् - मु॒खे । \newline
28. ए॒व य॒ज्ञ्ं ॅय॒ज्ञ् मे॒वैव य॒ज्ञ् मन्वनु॑ य॒ज्ञ् मे॒वैव य॒ज्ञ् मनु॑ । \newline
29. य॒ज्ञ् मन्वनु॑ य॒ज्ञ्ं ॅय॒ज्ञ् मनु॒ सꣳ स मनु॑ य॒ज्ञ्ं ॅय॒ज्ञ् मनु॒ सम् । \newline
30. अनु॒ सꣳ स मन्वनु॒ सम् त॑नोति तनोति॒ स मन्वनु॒ सम् त॑नोति । \newline
31. सम् त॑नोति तनोति॒ सꣳ सम् त॑नोति॒ प्राञ्च॒म् प्राञ्च॑म् तनोति॒ सꣳ सम् त॑नोति॒ प्राञ्च᳚म् । \newline
32. त॒नो॒ति॒ प्राञ्च॒म् प्राञ्च॑म् तनोति तनोति॒ प्राञ्च॑ म॒ग्नि म॒ग्निम् प्राञ्च॑म् तनोति तनोति॒ प्राञ्च॑ म॒ग्निम् । \newline
33. प्राञ्च॑ म॒ग्नि म॒ग्निम् प्राञ्च॒म् प्राञ्च॑ म॒ग्निम् प्र प्राग्निम् प्राञ्च॒म् प्राञ्च॑ म॒ग्निम् प्र । \newline
34. अ॒ग्निम् प्र प्राग्नि म॒ग्निम् प्र ह॑रन्ति हरन्ति॒ प्राग्नि म॒ग्निम् प्र ह॑रन्ति । \newline
35. प्र ह॑रन्ति हरन्ति॒ प्र प्र ह॑र॒ न्त्युदु द्ध॑रन्ति॒ प्र प्र ह॑र॒ न्त्युत् । \newline
36. ह॒र॒ न्त्युदु द्ध॑रन्ति हर॒न्त्युत् पत्नी॒म् पत्नी॒ मुद्ध॑रन्ति हर॒ न्त्युत् पत्नी᳚म् । \newline
37. उत् पत्नी॒म् पत्नी॒ मुदुत् पत्नी॒ मा पत्नी॒ मुदुत् पत्नी॒ मा । \newline
38. पत्नी॒ मा पत्नी॒म् पत्नी॒ मा न॑यन्ति नय॒ न्त्या पत्नी॒म् पत्नी॒ मा न॑यन्ति । \newline
39. आ न॑यन्ति नय॒ न्त्या न॑य॒ न्त्यन्वनु॑ नय॒न्त्या न॑य॒ न्त्यनु॑ । \newline
40. न॒य॒ न्त्यन्वनु॑ नयन्ति नय॒ न्त्यन्वनाꣳ॒॒ स्यनाꣳ॒॒ स्यनु॑ नयन्ति नय॒ न्त्यन्वनाꣳ॑सि । \newline
41. अन्वनाꣳ॒॒ स्यनाꣳ॒॒ स्यन्व न्वनाꣳ॑सि॒ प्र प्राणाꣳ॒॒ स्यन्वन्वनाꣳ॑सि॒ प्र । \newline
42. अनाꣳ॑सि॒ प्र प्राणाꣳ॒॒ स्यनाꣳ॑सि॒ प्र व॑र्तयन्ति वर्तयन्ति॒ प्राणाꣳ॒॒ स्यनाꣳ॑सि॒ प्र व॑र्तयन्ति । \newline
43. प्र व॑र्तयन्ति वर्तयन्ति॒ प्र प्र व॑र्तय॒ न्त्यथाथ॑ वर्तयन्ति॒ प्र प्र व॑र्तय॒ न्त्यथ॑ । \newline
44. व॒र्त॒य॒ न्त्यथाथ॑ वर्तयन्ति वर्तय॒ न्त्यथ॒ वै वा अथ॑ वर्तयन्ति वर्तय॒ न्त्यथ॒ वै । \newline
45. अथ॒ वै वा अथाथ॒ वा अ॑स्यास्य॒ वा अथाथ॒ वा अ॑स्य । \newline
46. वा अ॑स्यास्य॒ वै वा अ॑स्यै॒ष ए॒षो᳚ ऽस्य॒ वै वा अ॑स्यै॒षः । \newline
47. अ॒स्यै॒ष ए॒षो᳚ ऽस्यास्यै॒ष धिष्णि॑यो॒ धिष्णि॑य ए॒षो᳚ ऽस्यास्यै॒ष धिष्णि॑यः । \newline
48. ए॒ष धिष्णि॑यो॒ धिष्णि॑य ए॒ष ए॒ष धिष्णि॑यो हीयते हीयते॒ धिष्णि॑य ए॒ष ए॒ष धिष्णि॑यो हीयते । \newline
49. धिष्णि॑यो हीयते हीयते॒ धिष्णि॑यो॒ धिष्णि॑यो हीयते॒ स स ही॑यते॒ धिष्णि॑यो॒ धिष्णि॑यो हीयते॒ सः । \newline
50. ही॒य॒ते॒ स स ही॑यते हीयते॒ सो ऽन्वनु॒ स ही॑यते हीयते॒ सो ऽनु॑ । \newline
51. सो ऽन्वनु॒ स सो ऽनु॑ ध्यायति ध्याय॒ त्यनु॒ स सो ऽनु॑ ध्यायति । \newline
52. अनु॑ ध्यायति ध्याय॒ त्यन्वनु॑ ध्यायति॒ स स ध्या॑य॒ त्यन्वनु॑ ध्यायति॒ सः । \newline
53. ध्या॒य॒ति॒ स स ध्या॑यति ध्यायति॒ स ई᳚श्व॒र ई᳚श्व॒रः स ध्या॑यति ध्यायति॒ स ई᳚श्व॒रः । \newline
54. स ई᳚श्व॒र ई᳚श्व॒रः स स ई᳚श्व॒रो रु॒द्रो रु॒द्र ई᳚श्व॒रः स स ई᳚श्व॒रो रु॒द्रः । \newline
55. ई॒श्व॒रो रु॒द्रो रु॒द्र ई᳚श्व॒र ई᳚श्व॒रो रु॒द्रो भू॒त्वा भू॒त्वा रु॒द्र ई᳚श्व॒र ई᳚श्व॒रो रु॒द्रो भू॒त्वा । \newline
56. रु॒द्रो भू॒त्वा भू॒त्वा रु॒द्रो रु॒द्रो भू॒त्वा प्र॒जाम् प्र॒जाम् भू॒त्वा रु॒द्रो रु॒द्रो भू॒त्वा प्र॒जाम् । \newline
57. भू॒त्वा प्र॒जाम् प्र॒जाम् भू॒त्वा भू॒त्वा प्र॒जाम् प॒शून् प॒शून् प्र॒जाम् भू॒त्वा भू॒त्वा प्र॒जाम् प॒शून् । \newline
\pagebreak
\markright{ TS 3.1.3.2  \hfill https://www.vedavms.in \hfill}

\section{ TS 3.1.3.2 }

\textbf{TS 3.1.3.2 } \newline
\textbf{Samhita Paata} \newline

प्र॒जां प॒शून्. यज॑मानस्य॒ शम॑यितो॒र्यर्.हि॑ प॒शुमा प्री॑त॒मुद॑ञ्चं॒ नय॑न्ति॒ तर्.हि॒ तस्य॑ पशु॒श्रप॑णꣳ हरे॒त् तेनै॒वैनं॑ भा॒गिनं॑ करोति॒ यज॑मानो॒ वा आ॑हव॒नीयो॒ यज॑मानं॒ ॅवा ए॒तद्वि क॑र्.षन्ते॒ यदा॑हव॒नीया᳚त् पशु॒श्रप॑णꣳ॒॒ हर॑न्ति॒ स वै॒व स्यान्नि॑र्म॒न्थ्यं॑ ॅवा कुर्या॒द्-यज॑मानस्य सात्म॒त्वाय॒ यदि॑ प॒शोर॑व॒दानं॒ नश्ये॒दाज्य॑स्य प्रत्या॒ख्याय॒मव॑ द्ये॒थ् सैव ततः॒ ( ) प्राय॑श्चित्ति॒र्ये प॒शुं ॅवि॑मथ्नी॒रन्. यस्तान् का॒मये॒ता ऽऽ*र्ति॒मार्च्छे॑यु॒रिति॑ कु॒विद॒ङ्गेति॒ नमो॑ वृक्तिवत्य॒र्चाऽऽग्नी᳚द्ध्रे जुहुया॒न्नमो॑ वृक्तिमे॒वैषां᳚ ॅवृङ्क्ते ता॒जगार्ति॒मार्च्छ॑न्ति ॥ \newline

\textbf{Pada Paata} \newline

प्र॒जामिति॑ प्र - जाम् । प॒शून् । यज॑मानस्य । शम॑यितोः । यर्.हि॑ । प॒शुम् । आप्री॑त॒मित्या-प्री॒त॒म् । उद॑ञ्चम् । नय॑न्ति । तर्.हि॑ । तस्य॑ । प॒शु॒श्रप॑ण॒मिति॑ पशु - श्रप॑णम् । ह॒रे॒त् । तेन॑ । ए॒व । ए॒न॒म् । भा॒गिन᳚म् । क॒रो॒ति॒ । यज॑मानः । वै । आ॒ह॒व॒नीय॒ इत्या᳚ - ह॒व॒नीयः॑ । यज॑मानम् । वै । ए॒तत् । वीति॑ । क॒र्॒.ष॒न्ते॒ । यत् । आ॒ह॒व॒नीया॒दित्या᳚ - ह॒व॒नीया᳚त् । प॒शु॒श्रप॑ण॒मिति॑ पशु - श्रप॑णम् । हर॑न्ति । सः । वा॒ । ए॒व । स्यात् । नि॒र्म॒न्थ्य॑मिति॑ निः - म॒न्थ्य᳚म् । वा॒ । कु॒र्या॒त् । यज॑मानस्य । सा॒त्म॒त्वायेति॑ सात्म - त्वाय॑ । यदि॑ । प॒शोः । अ॒व॒दान॒मित्य॑व - दान᳚म् । नश्ये᳚त् । आज्य॑स्य । प्र॒त्या॒ख्याय॒मिति॑ प्रति - आ॒ख्याय᳚म् । अवेति॑ । द्ये॒त् । सा । ए॒व । ततः॑ ( ) । प्राय॑श्चित्तिः । ये । प॒शुम् । वि॒म॒थ्नी॒रन्निति॑ वि - म॒थ्नी॒रन्न् । यः । तान् । का॒मये॑त । आर्ति᳚म् । एति॑ । ऋ॒च्छे॒युः॒ । इति॑ । कु॒वित् । अ॒ङ्ग । इति॑ । नमो॑वृक्तिव॒त्येति॒ नमो॑वृक्ति - व॒त्या॒ । ऋ॒चा । आग्नी᳚द्ध्र॒ इत्याग्नि॑ - इ॒ध्रे॒ । जु॒हु॒या॒त् । नमो॑वृक्ति॒मिति॒ नमः॑ - वृ॒क्ति॒म् । ए॒व । ए॒षा॒म् । वृ॒ङ्क्ते॒ । ता॒जक् । आर्ति᳚म् । एति॑ । ऋ॒च्छ॒न्ति॒ ॥  \newline


\textbf{Krama Paata} \newline

प्र॒जाम् प॒शून् । प्र॒जामिति॑ प्र - जाम् । प॒शून्. यज॑मानस्य । यज॑मानस्य॒ शम॑यितोः । शम॑यितो॒र् यर्.हि॑ । यर्.हि॑ प॒शुम् । प॒शुमाप्री॑तम् । आप्री॑त॒मुद॑ञ्चम् । आप्री॑त॒मित्या - प्री॒त॒म् । उद॑ञ्च॒म् नय॑न्ति । नय॑न्ति॒ तर्.हि॑ । तर्.हि॒ तस्य॑ । तस्य॑ पशु॒श्रप॑णम् । प॒शु॒श्रप॑णꣳ हरेत् । प॒शु॒श्रप॑ण॒मिति॑ पशु - श्रप॑णम् । ह॒रे॒त् तेन॑ । तेनै॒व । ए॒वैन᳚म् । ए॒न॒म् भा॒गिन᳚म् । भा॒गिन॑म् करोति । क॒रो॒ति॒ यज॑मानः । यज॑मानो॒ वै । वा आ॑हव॒नीयः॑ । आ॒ह॒व॒नीयो॒ यज॑मानम् । आ॒ह॒व॒नीय॒ इत्या᳚ - ह॒व॒नीयः॑ । यज॑मानं॒ ॅवै । वा ए॒तत् । ए॒तद् वि । वि क॑र्.षन्ते । क॒र्॒.ष॒न्ते॒ यत् । यदा॑हव॒नीया᳚त् । आ॒ह॒व॒नीया᳚त् पशु॒श्रप॑णम् । आ॒ह॒व॒नीया॒दित्या᳚ - ह॒व॒नीया᳚त् । प॒शु॒श्रप॑णꣳ॒॒ हर॑न्ति । प॒शु॒श्रप॑ण॒मिति॑ पशु - श्रप॑णम् । हर॑न्ति॒ सः । स वा᳚ । वै॒व । 
ए॒व स्यात् । स्यान् नि॑र्म॒न्थ्य᳚म् । नि॒र्म॒न्थ्यं॑ ॅवा । नि॒र्म॒न्थ्य॑मिति॑ निः - म॒न्थ्य᳚म् । वा॒ कु॒र्या॒त्॒ । कु॒र्या॒द् यज॑मानस्य । यज॑मानस्य सात्म॒त्वाय॑ । सा॒त्म॒त्वाय॒ यदि॑ । सा॒त्म॒त्वायेति॑ सात्म - त्वाय॑ । यदि॑ प॒शोः । प॒शोर॑व॒दान᳚म् । अ॒व॒दान॒म् नश्ये᳚त् । अ॒व॒दान॒मित्य॑व - दान᳚म् । नश्ये॒दाज्य॑स्य । आज्य॑स्य प्रत्या॒ख्याय᳚म् । प्र॒त्या॒ख्याय॒मव॑ । प्र॒त्या॒ख्याय॒मिति॑ प्रति - आ॒ख्याय᳚म् । अव॑ द्येत् । द्ये॒थ् सा । सैव । ए॒व ततः॑ ( ) । ततः॒ प्राय॑श्चित्तिः । प्राय॑श्चित्ति॒र् ये । ये प॒शुम् । प॒शुं ॅवि॑मथ्नी॒रन्न् । वि॒म॒थ्नी॒रन्. यः । वि॒म॒थ्नी॒रन्निति॑ वि - म॒थ्नी॒रन्न् । यस्तान् । तान् का॒मये॑त । का॒मये॒तार्ति᳚म् । आर्ति॒मा । आर्च्छे॑युः । ऋ॒च्छे॒यु॒रिति॑ । इति॑ कु॒वित् । कु॒विद॒ङ्ग । अ॒ङ्गेति॑ । इति॒ नमो॑वृक्तिवत्या । नमो॑वृक्तिवत्य॒र्चा । नमो॑वृक्तिव॒त्येति॒ नमो॑वृक्ति - व॒त्या॒ । ऋ॒चाऽऽग्नी᳚ध्रे । आग्नी᳚ध्रे जुहुयात् । आग्नी᳚ध्र॒ इत्याग्नि॑ - इ॒ध्रे॒ । जु॒हु॒या॒न्नमो॑वृक्तिम् । नमो॑वृक्तिमे॒व । नमो॑वृक्ति॒मिति॒ नमः॑ - वृ॒क्ति॒म् । ए॒वैषा᳚म् । ए॒षां॒ ॅवृ॒ङ्क्ते॒ । वृ॒ङ्क्ते॒ ता॒जक् । ता॒जगार्ति᳚म् । आर्ति॒मा । आर्च्छ॑न्ति । ऋ॒च्छ॒न्तीत्यृ॑च्छन्ति । \newline

\textbf{Jatai Paata} \newline

1. प्र॒जाम् प॒शून् प॒शून् प्र॒जाम् प्र॒जाम् प॒शून् । \newline
2. प्र॒जामिति॑ प्र - जाम् । \newline
3. प॒शून्. यज॑मानस्य॒ यज॑मानस्य प॒शून् प॒शून्. यज॑मानस्य । \newline
4. यज॑मानस्य॒ शम॑यितोः॒ शम॑यितो॒र् यज॑मानस्य॒ यज॑मानस्य॒ शम॑यितोः । \newline
5. शम॑यितो॒र् यर्.हि॒ यर्.हि॒ शम॑यितोः॒ शम॑यितो॒र् यर्.हि॑ । \newline
6. यर्.हि॑ प॒शुम् प॒शुं ॅयर्.हि॒ यर्.हि॑ प॒शुम् । \newline
7. प॒शु माप्री॑त॒ माप्री॑तम् प॒शुम् प॒शु माप्री॑तम् । \newline
8. आप्री॑त॒ मुद॑ञ्च॒ मुद॑ञ्च॒ माप्री॑त॒ माप्री॑त॒ मुद॑ञ्चम् । \newline
9. आप्री॑त॒मित्या - प्री॒त॒म् । \newline
10. उद॑ञ्च॒म् नय॑न्ति॒ नय॒ न्त्युद॑ञ्च॒ मुद॑ञ्च॒म् नय॑न्ति । \newline
11. नय॑न्ति॒ तर्.हि॒ तर्.हि॒ नय॑न्ति॒ नय॑न्ति॒ तर्.हि॑ । \newline
12. तर्.हि॒ तस्य॒ तस्य॒ तर्.हि॒ तर्.हि॒ तस्य॑ । \newline
13. तस्य॑ पशु॒श्रप॑णम् पशु॒श्रप॑ण॒म् तस्य॒ तस्य॑ पशु॒श्रप॑णम् । \newline
14. प॒शु॒श्रप॑णꣳ हरे द्धरेत् पशु॒श्रप॑णम् पशु॒श्रप॑णꣳ हरेत् । \newline
15. प॒शु॒श्रप॑ण॒मिति॑ पशु - श्रप॑णम् । \newline
16. ह॒रे॒त् तेन॒ तेन॑ हरे द्धरे॒त् तेन॑ । \newline
17. तेनै॒ वैव तेन॒ तेनै॒व । \newline
18. ए॒वैन॑ मेन मे॒वैवैन᳚म् । \newline
19. ए॒न॒म् भा॒गिन॑म् भा॒गिन॑ मेन मेनम् भा॒गिन᳚म् । \newline
20. भा॒गिन॑म् करोति करोति भा॒गिन॑म् भा॒गिन॑म् करोति । \newline
21. क॒रो॒ति॒ यज॑मानो॒ यज॑मानः करोति करोति॒ यज॑मानः । \newline
22. यज॑मानो॒ वै वै यज॑मानो॒ यज॑मानो॒ वै । \newline
23. वा आ॑हव॒नीय॑ आहव॒नीयो॒ वै वा आ॑हव॒नीयः॑ । \newline
24. आ॒ह॒व॒नीयो॒ यज॑मानं॒ ॅयज॑मान माहव॒नीय॑ आहव॒नीयो॒ यज॑मानम् । \newline
25. आ॒ह॒व॒नीय॒ इत्या᳚ - ह॒व॒नीयः॑ । \newline
26. यज॑मानं॒ ॅवै वै यज॑मानं॒ ॅयज॑मानं॒ ॅवै । \newline
27. वा ए॒त दे॒तद् वै वा ए॒तत् । \newline
28. ए॒तद् वि व्ये॑त दे॒तद् वि । \newline
29. वि क॑र्.षन्ते कर्.षन्ते॒ वि वि क॑र्.षन्ते । \newline
30. क॒र्॒.ष॒न्ते॒ यद् यत् क॑र्.षन्ते कर्.षन्ते॒ यत् । \newline
31. यदा॑हव॒नीया॑ दाहव॒नीया॒द् यद् यदा॑हव॒नीया᳚त् । \newline
32. आ॒ह॒व॒नीया᳚त् पशु॒श्रप॑णम् पशु॒श्रप॑ण माहव॒नीया॑ दाहव॒नीया᳚त् पशु॒श्रप॑णम् । \newline
33. आ॒ह॒व॒नीया॒दित्या᳚ - ह॒व॒नीया᳚त् । \newline
34. प॒शु॒श्रप॑णꣳ॒॒ हर॑न्ति॒ हर॑न्ति पशु॒श्रप॑णम् पशु॒श्रप॑णꣳ॒॒ हर॑न्ति । \newline
35. प॒शु॒श्रप॑ण॒मिति॑ पशु - श्रप॑णम् । \newline
36. हर॑न्ति॒ स स हर॑न्ति॒ हर॑न्ति॒ सः । \newline
37. स वा॑ वा॒ स स वा᳚ । \newline
38. वै॒वैव वा॑ वै॒व । \newline
39. ए॒व स्याथ् स्या दे॒वैव स्यात् । \newline
40. स्यान् नि॑र्म॒न्थ्य॑म् निर्म॒न्थ्यꣳ॑ स्याथ् स्यान् नि॑र्म॒न्थ्य᳚म् । \newline
41. नि॒र्म॒न्थ्यं॑ ॅवा वा निर्म॒न्थ्य॑म् निर्म॒न्थ्यं॑ ॅवा । \newline
42. नि॒र्म॒न्थ्य॑मिति॑ निः - म॒न्थ्य᳚म् । \newline
43. वा॒ कु॒र्या॒त् कु॒र्या॒द् वा॒ वा॒ कु॒र्या॒त् । \newline
44. कु॒र्या॒द् यज॑मानस्य॒ यज॑मानस्य कुर्यात् कुर्या॒द् यज॑मानस्य । \newline
45. यज॑मानस्य सात्म॒त्वाय॑ सात्म॒त्वाय॒ यज॑मानस्य॒ यज॑मानस्य सात्म॒त्वाय॑ । \newline
46. सा॒त्म॒त्वाय॒ यदि॒ यदि॑ सात्म॒त्वाय॑ सात्म॒त्वाय॒ यदि॑ । \newline
47. सा॒त्म॒त्वायेति॑ सात्म - त्वाय॑ । \newline
48. यदि॑ प॒शोः प॒शोर् यदि॒ यदि॑ प॒शोः । \newline
49. प॒शो र॑व॒दान॑ मव॒दान॑म् प॒शोः प॒शो र॑व॒दान᳚म् । \newline
50. अ॒व॒दान॒म् नश्ये॒न् नश्ये॑ दव॒दान॑ मव॒दान॒म् नश्ये᳚त् । \newline
51. अ॒व॒दान॒मित्य॑व - दान᳚म् । \newline
52. नश्ये॒ दाज्य॒स्या ज्य॑स्य॒ नश्ये॒न् नश्ये॒ दाज्य॑स्य । \newline
53. आज्य॑स्य प्रत्या॒ख्याय॑म् प्रत्या॒ख्याय॒ माज्य॒स्या ज्य॑स्य प्रत्या॒ख्याय᳚म् । \newline
54. प्र॒त्या॒ख्याय॒ मवाव॑ प्रत्या॒ख्याय॑म् प्रत्या॒ख्याय॒ मव॑ । \newline
55. प्र॒त्या॒ख्याय॒मिति॑ प्रति - आ॒ख्याय᳚म् । \newline
56. अव॑ द्येद् द्ये॒ दवाव॑ द्येत् । \newline
57. द्ये॒थ् सा सा द्ये᳚द् द्ये॒थ् सा । \newline
58. सैवैव सा सैव । \newline
59. ए॒व तत॒ स्तत॑ ए॒वैव ततः॑ । \newline
60. ततः॒ प्राय॑श्चित्तिः॒ प्राय॑श्चित्ति॒ स्तत॒ स्ततः॒ प्राय॑श्चित्तिः । \newline
61. प्राय॑श्चित्ति॒र् ये ये प्राय॑श्चित्तिः॒ प्राय॑श्चित्ति॒र् ये । \newline
62. ये प॒शुम् प॒शुं ॅये ये प॒शुम् । \newline
63. प॒शुं ॅवि॑मथ्नी॒रन्. वि॑मथ्नी॒रन् प॒शुम् प॒शुं ॅवि॑मथ्नी॒रन्न् । \newline
64. वि॒म॒थ्नी॒रन्. यो यो वि॑मथ्नी॒रन्. वि॑मथ्नी॒रन्. यः । \newline
65. वि॒म॒थ्नी॒रन्निति॑ वि - म॒थ्नी॒रन्न् । \newline
66. य स्ताꣳ स्तान्. यो य स्तान् । \newline
67. तान् का॒मये॑त का॒मये॑त॒ ताꣳ स्तान् का॒मये॑त । \newline
68. का॒मये॒तार्ति॒ मार्ति॑म् का॒मये॑त का॒मये॒तार्ति᳚म् । \newline
69. आर्ति॒ मा ऽऽर्ति॒ मार्ति॒ मा । \newline
70. आर्च्छे॑युर्. ऋच्छेयु॒ रार्च्छे॑युः । \newline
71. ऋ॒च्छे॒यु॒ रिती त्यृ॑च्छेयुर्. ऋच्छेयु॒ रिति॑ । \newline
72. इति॑ कु॒वित् कु॒वि दितीति॑ कु॒वित् । \newline
73. कु॒वि द॒ङ्गाङ्ग कु॒वित् कु॒वि द॒ङ्ग । \newline
74. अ॒ङ्गे ती त्य॒ङ्गाङ्गे ति॑ । \newline
75. इति॒ नमो॑वृक्तिवत्या॒ नमो॑वृक्तिव॒ त्येतीति॒ नमो॑वृक्तिवत्या । \newline
76. नमो॑वृक्तिव त्य॒र्चर्चा नमो॑वृक्तिवत्या॒ नमो॑वृक्तिव त्य॒र्चा । \newline
77. नमो॑वृक्तिव॒त्येति॒ नमो॑वृक्ति - व॒त्या॒ । \newline
78. ऋ॒चा ऽऽग्नी᳚द्ध्र॒ आग्नी᳚द्ध्र ऋ॒चर्चा ऽऽग्नी᳚द्ध्रे । \newline
79. आग्नी᳚द्ध्रे जुहुयाज् जुहुया॒ दाग्नी᳚द्ध्र॒ आग्नी᳚द्ध्रे जुहुयात् । \newline
80. आग्नी᳚द्ध्र॒ इत्याग्नि॑ - इ॒ध्रे॒ । \newline
81. जु॒हु॒या॒न् नमो॑वृक्ति॒म् नमो॑वृक्तिम् जुहुयाज् जुहुया॒न् नमो॑वृक्तिम् । \newline
82. नमो॑वृक्ति मे॒वैव नमो॑वृक्ति॒म् नमो॑वृक्ति मे॒व । \newline
83. नमो॑वृक्ति॒मिति॒ नमः॑ - वृ॒क्ति॒म् । \newline
84. ए॒वैषा॑ मेषा मे॒वैवैषा᳚म् । \newline
85. ए॒षां॒ ॅवृ॒ङ्क्ते॒ वृ॒ङ्क्त॒ ए॒षा॒ मे॒षां॒ ॅवृ॒ङ्क्ते॒ । \newline
86. वृ॒ङ्क्ते॒ ता॒जक् ता॒जग् वृ॑ङ्क्ते वृङ्क्ते ता॒जक् । \newline
87. ता॒जगार्ति॒ मार्ति॑म् ता॒जक् ता॒जगार्ति᳚म् । \newline
88. आर्ति॒ मा ऽऽर्ति॒ मार्ति॒ मा । \newline
89. आर्च्छ॑ न्त्यृच्छ॒ न्त्यार्च्छ॑न्ति । \newline
90. ऋ॒च्छ॒न्तीत्यृ॑च्छन्ति । \newline

\textbf{Ghana Paata } \newline

1. प्र॒जाम् प॒शून् प॒शून् प्र॒जाम् प्र॒जाम् प॒शून्. यज॑मानस्य॒ यज॑मानस्य प॒शून् प्र॒जाम् प्र॒जाम् प॒शून्. यज॑मानस्य । \newline
2. प्र॒जामिति॑ प्र - जाम् । \newline
3. प॒शून्. यज॑मानस्य॒ यज॑मानस्य प॒शून् प॒शून्. यज॑मानस्य॒ शम॑यितोः॒ शम॑यितो॒र् यज॑मानस्य प॒शून् प॒शून्. यज॑मानस्य॒ शम॑यितोः । \newline
4. यज॑मानस्य॒ शम॑यितोः॒ शम॑यितो॒र् यज॑मानस्य॒ यज॑मानस्य॒ शम॑यितो॒र् यर्.हि॒ यर्.हि॒ शम॑यितो॒र् यज॑मानस्य॒ यज॑मानस्य॒ शम॑यितो॒र् यर्.हि॑ । \newline
5. शम॑यितो॒र् यर्.हि॒ यर्.हि॒ शम॑यितोः॒ शम॑यितो॒र् यर्.हि॑ प॒शुम् प॒शुं ॅयर्.हि॒ शम॑यितोः॒ शम॑यितो॒र् यर्.हि॑ प॒शुम् । \newline
6. यर्.हि॑ प॒शुम् प॒शुं ॅयर्.हि॒ यर्.हि॑ प॒शु माप्री॑त॒ माप्री॑तम् प॒शुं ॅयर्.हि॒ यर्.हि॑ प॒शु माप्री॑तम् । \newline
7. प॒शु माप्री॑त॒ माप्री॑तम् प॒शुम् प॒शु माप्री॑त॒ मुद॑ञ्च॒ मुद॑ञ्च॒ माप्री॑तम् प॒शुम् प॒शु माप्री॑त॒ मुद॑ञ्चम् । \newline
8. आप्री॑त॒ मुद॑ञ्च॒ मुद॑ञ्च॒ माप्री॑त॒ माप्री॑त॒ मुद॑ञ्च॒म् नय॑न्ति॒ नय॒ न्त्युद॑ञ्च॒ माप्री॑त॒ माप्री॑त॒ मुद॑ञ्च॒म् नय॑न्ति । \newline
9. आप्री॑त॒मित्या - प्री॒त॒म् । \newline
10. उद॑ञ्च॒म् नय॑न्ति॒ नय॒ न्त्युद॑ञ्च॒ मुद॑ञ्च॒म् नय॑न्ति॒ तर्.हि॒ तर्.हि॒ नय॒ न्त्युद॑ञ्च॒ मुद॑ञ्च॒म् नय॑न्ति॒ तर्.हि॑ । \newline
11. नय॑न्ति॒ तर्.हि॒ तर्.हि॒ नय॑न्ति॒ नय॑न्ति॒ तर्.हि॒ तस्य॒ तस्य॒ तर्.हि॒ नय॑न्ति॒ नय॑न्ति॒ तर्.हि॒ तस्य॑ । \newline
12. तर्.हि॒ तस्य॒ तस्य॒ तर्.हि॒ तर्.हि॒ तस्य॑ पशु॒श्रप॑णम् पशु॒श्रप॑ण॒म् तस्य॒ तर्.हि॒ तर्.हि॒ तस्य॑ पशु॒श्रप॑णम् । \newline
13. तस्य॑ पशु॒श्रप॑णम् पशु॒श्रप॑ण॒म् तस्य॒ तस्य॑ पशु॒श्रप॑णꣳ हरे द्धरेत् पशु॒श्रप॑ण॒म् तस्य॒ तस्य॑ पशु॒श्रप॑णꣳ हरेत् । \newline
14. प॒शु॒श्रप॑णꣳ हरे द्धरेत् पशु॒श्रप॑णम् पशु॒श्रप॑णꣳ हरे॒त् तेन॒ तेन॑ हरेत् पशु॒श्रप॑णम् पशु॒श्रप॑णꣳ हरे॒त् तेन॑ । \newline
15. प॒शु॒श्रप॑ण॒मिति॑ पशु - श्रप॑णम् । \newline
16. ह॒रे॒त् तेन॒ तेन॑ हरे द्धरे॒त् तेनै॒वैव तेन॑ हरे द्धरे॒त् तेनै॒व । \newline
17. तेनै॒वैव तेन॒ तेनै॒वैन॑ मेन मे॒व तेन॒ तेनै॒वैन᳚म् । \newline
18. ए॒वैन॑ मेन मे॒वैवैन॑म् भा॒गिन॑म् भा॒गिन॑ मेन मे॒वैवैन॑म् भा॒गिन᳚म् । \newline
19. ए॒न॒म् भा॒गिन॑म् भा॒गिन॑ मेन मेनम् भा॒गिन॑म् करोति करोति भा॒गिन॑ मेन मेनम् भा॒गिन॑म् करोति । \newline
20. भा॒गिन॑म् करोति करोति भा॒गिन॑म् भा॒गिन॑म् करोति॒ यज॑मानो॒ यज॑मानः करोति भा॒गिन॑म् भा॒गिन॑म् करोति॒ यज॑मानः । \newline
21. क॒रो॒ति॒ यज॑मानो॒ यज॑मानः करोति करोति॒ यज॑मानो॒ वै वै यज॑मानः करोति करोति॒ यज॑मानो॒ वै । \newline
22. यज॑मानो॒ वै वै यज॑मानो॒ यज॑मानो॒ वा आ॑हव॒नीय॑ आहव॒नीयो॒ वै यज॑मानो॒ यज॑मानो॒ वा आ॑हव॒नीयः॑ । \newline
23. वा आ॑हव॒नीय॑ आहव॒नीयो॒ वै वा आ॑हव॒नीयो॒ यज॑मानं॒ ॅयज॑मान माहव॒नीयो॒ वै वा आ॑हव॒नीयो॒ यज॑मानम् । \newline
24. आ॒ह॒व॒नीयो॒ यज॑मानं॒ ॅयज॑मान माहव॒नीय॑ आहव॒नीयो॒ यज॑मानं॒ ॅवै वै यज॑मान माहव॒नीय॑ आहव॒नीयो॒ यज॑मानं॒ ॅवै । \newline
25. आ॒ह॒व॒नीय॒ इत्या᳚ - ह॒व॒नीयः॑ । \newline
26. यज॑मानं॒ ॅवै वै यज॑मानं॒ ॅयज॑मानं॒ ॅवा ए॒त दे॒तद् वै यज॑मानं॒ ॅयज॑मानं॒ ॅवा ए॒तत् । \newline
27. वा ए॒त दे॒तद् वै वा ए॒तद् वि व्ये॑तद् वै वा ए॒तद् वि । \newline
28. ए॒तद् वि व्ये॑त दे॒तद् वि क॑र्.षन्ते कर्.षन्ते॒ व्ये॑त दे॒तद् वि क॑र्.षन्ते । \newline
29. वि क॑र्.षन्ते कर्.षन्ते॒ वि वि क॑र्.षन्ते॒ यद् यत् क॑र्.षन्ते॒ वि वि क॑र्.षन्ते॒ यत् । \newline
30. क॒र्॒.ष॒न्ते॒ यद् यत् क॑र्.षन्ते कर्.षन्ते॒ यदा॑हव॒नीया॑ दाहव॒नीया॒द् यत् क॑र्.षन्ते कर्.षन्ते॒ यदा॑हव॒नीया᳚त् । \newline
31. यदा॑हव॒नीया॑ दाहव॒नीया॒द् यद् यदा॑हव॒नीया᳚त् पशु॒श्रप॑णम् पशु॒श्रप॑ण माहव॒नीया॒द् यद् यदा॑हव॒नीया᳚त् पशु॒श्रप॑णम् । \newline
32. आ॒ह॒व॒नीया᳚त् पशु॒श्रप॑णम् पशु॒श्रप॑ण माहव॒नीया॑ दाहव॒नीया᳚त् पशु॒श्रप॑णꣳ॒॒ हर॑न्ति॒ हर॑न्ति पशु॒श्रप॑ण माहव॒नीया॑ दाहव॒नीया᳚त् पशु॒श्रप॑णꣳ॒॒ हर॑न्ति । \newline
33. आ॒ह॒व॒नीया॒दित्या᳚ - ह॒व॒नीया᳚त् । \newline
34. प॒शु॒श्रप॑णꣳ॒॒ हर॑न्ति॒ हर॑न्ति पशु॒श्रप॑णम् पशु॒श्रप॑णꣳ॒॒ हर॑न्ति॒ स स हर॑न्ति पशु॒श्रप॑णम् पशु॒श्रप॑णꣳ॒॒ हर॑न्ति॒ सः । \newline
35. प॒शु॒श्रप॑ण॒मिति॑ पशु - श्रप॑णम् । \newline
36. हर॑न्ति॒ स स हर॑न्ति॒ हर॑न्ति॒ स वा॑ वा॒ स हर॑न्ति॒ हर॑न्ति॒ स वा᳚ । \newline
37. स वा॑ वा॒ स स वै॒वै ववा॒ स स वै॒व । \newline
38. वै॒वैव वा॑ वै॒व स्याथ् स्या दे॒व वा॑ वै॒व स्यात् । \newline
39. ए॒व स्याथ् स्यादे॒वैव स्यान् नि॑र्म॒न्थ्य॑म् निर्म॒न्थ्यꣳ॑ स्या दे॒वैव स्यान् नि॑र्म॒न्थ्य᳚म् । \newline
40. स्यान् नि॑र्म॒न्थ्य॑म् निर्म॒न्थ्यꣳ॑ स्याथ् स्यान् नि॑र्म॒न्थ्यं॑ ॅवा वा निर्म॒न्थ्यꣳ॑ स्याथ् स्यान् नि॑र्म॒न्थ्यं॑ ॅवा । \newline
41. नि॒र्म॒न्थ्यं॑ ॅवा वा निर्म॒न्थ्य॑म् निर्म॒न्थ्यं॑ ॅवा कुर्यात् कुर्याद् वा निर्म॒न्थ्य॑म् निर्म॒न्थ्यं॑ ॅवा कुर्यात् । \newline
42. नि॒र्म॒न्थ्य॑मिति॑ निः - म॒न्थ्य᳚म् । \newline
43. वा॒ कु॒र्या॒त् कु॒र्या॒द् वा॒ वा॒ कु॒र्या॒द् यज॑मानस्य॒ यज॑मानस्य कुर्याद् वा वा कुर्या॒द् यज॑मानस्य । \newline
44. कु॒र्या॒द् यज॑मानस्य॒ यज॑मानस्य कुर्यात् कुर्या॒द् यज॑मानस्य सात्म॒त्वाय॑ सात्म॒त्वाय॒ यज॑मानस्य कुर्यात् कुर्या॒द् यज॑मानस्य सात्म॒त्वाय॑ । \newline
45. यज॑मानस्य सात्म॒त्वाय॑ सात्म॒त्वाय॒ यज॑मानस्य॒ यज॑मानस्य सात्म॒त्वाय॒ यदि॒ यदि॑ सात्म॒त्वाय॒ यज॑मानस्य॒ यज॑मानस्य सात्म॒त्वाय॒ यदि॑ । \newline
46. सा॒त्म॒त्वाय॒ यदि॒ यदि॑ सात्म॒त्वाय॑ सात्म॒त्वाय॒ यदि॑ प॒शोः प॒शोर् यदि॑ सात्म॒त्वाय॑ सात्म॒त्वाय॒ यदि॑ प॒शोः । \newline
47. सा॒त्म॒त्वायेति॑ सात्म - त्वाय॑ । \newline
48. यदि॑ प॒शोः प॒शोर् यदि॒ यदि॑ प॒शो र॑व॒दान॑ मव॒दान॑म् प॒शोर् यदि॒ यदि॑ प॒शो र॑व॒दान᳚म् । \newline
49. प॒शो र॑व॒दान॑ मव॒दान॑म् प॒शोः प॒शो र॑व॒दान॒म् नश्ये॒न् नश्ये॑ दव॒दान॑म् प॒शोः प॒शो र॑व॒दान॒म् नश्ये᳚त् । \newline
50. अ॒व॒दान॒म् नश्ये॒न् नश्ये॑ दव॒दान॑ मव॒दान॒म् नश्ये॒ दाज्य॒स्या ज्य॑स्य॒ नश्ये॑ दव॒दान॑ मव॒दान॒म् नश्ये॒ दाज्य॑स्य । \newline
51. अ॒व॒दान॒मित्य॑व - दान᳚म् । \newline
52. नश्ये॒ दाज्य॒स्या ज्य॑स्य॒ नश्ये॒न् नश्ये॒ दाज्य॑स्य प्रत्या॒ख्याय॑म् प्रत्या॒ख्याय॒ माज्य॑स्य॒ नश्ये॒न् नश्ये॒ दाज्य॑स्य प्रत्या॒ख्याय᳚म् । \newline
53. आज्य॑स्य प्रत्या॒ख्याय॑म् प्रत्या॒ख्याय॒ माज्य॒स्या ज्य॑स्य प्रत्या॒ख्याय॒ मवाव॑ प्रत्या॒ख्याय॒ माज्य॒स्या ज्य॑स्य प्रत्या॒ख्याय॒ मव॑ । \newline
54. प्र॒त्या॒ख्याय॒ मवाव॑ प्रत्या॒ख्याय॑म् प्रत्या॒ख्याय॒ मव॑ द्येद् द्ये॒दव॑ प्रत्या॒ख्याय॑म् प्रत्या॒ख्याय॒ मव॑ द्येत् । \newline
55. प्र॒त्या॒ख्याय॒मिति॑ प्रति - आ॒ख्याय᳚म् । \newline
56. अव॑ द्येद् द्ये॒ दवाव॑ द्ये॒थ् सा सा द्ये॒ दवाव॑ द्ये॒थ् सा । \newline
57. द्ये॒थ् सा सा द्ये᳚द् द्ये॒थ् सैवैव सा द्ये᳚द् द्ये॒थ् सैव । \newline
58. सैवैव सा सैव तत॒ स्तत॑ ए॒व सा सैव ततः॑ । \newline
59. ए॒व तत॒ स्तत॑ ए॒वैव ततः॒ प्राय॑श्चित्तिः॒ प्राय॑श्चित्ति॒ स्तत॑ ए॒वैव ततः॒ प्राय॑श्चित्तिः । \newline
60. ततः॒ प्राय॑श्चित्तिः॒ प्राय॑श्चित्ति॒ स्तत॒ स्ततः॒ प्राय॑श्चित्ति॒र् ये ये प्राय॑श्चित्ति॒ स्तत॒ स्ततः॒ प्राय॑श्चित्ति॒र् ये । \newline
61. प्राय॑श्चित्ति॒र् ये ये प्राय॑श्चित्तिः॒ प्राय॑श्चित्ति॒र् ये प॒शुम् प॒शुं ॅये प्राय॑श्चित्तिः॒ प्राय॑श्चित्ति॒र् ये प॒शुम् । \newline
62. ये प॒शुम् प॒शुं ॅये ये प॒शुं ॅवि॑मथ्नी॒रन्. वि॑मथ्नी॒रन् प॒शुं ॅये ये प॒शुं ॅवि॑मथ्नी॒रन्न् । \newline
63. प॒शुं ॅवि॑मथ्नी॒रन्. वि॑मथ्नी॒रन् प॒शुम् प॒शुं ॅवि॑मथ्नी॒रन्. यो यो वि॑मथ्नी॒रन् प॒शुम् प॒शुं ॅवि॑मथ्नी॒रन्. यः । \newline
64. वि॒म॒थ्नी॒रन्. यो यो वि॑मथ्नी॒रन्. वि॑मथ्नी॒रन्. यस्ताꣳ स्तान्. यो वि॑मथ्नी॒रन्. वि॑मथ्नी॒रन्. यस्तान् । \newline
65. वि॒म॒थ्नी॒रन्निति॑ वि - म॒थ्नी॒रन्न् । \newline
66. यस्ताꣳ स्तान्. यो यस्तान् का॒मये॑त का॒मये॑त॒ तान्. यो यस्तान् का॒मये॑त । \newline
67. तान् का॒मये॑त का॒मये॑त॒ ताꣳ स्तान् का॒मये॒तार्ति॒ मार्ति॑म् का॒मये॑त॒ ताꣳ स्तान् का॒मये॒तार्ति᳚म् । \newline
68. का॒मये॒तार्ति॒ मार्ति॑म् का॒मये॑त का॒मये॒तार्ति॒ मा ऽऽर्ति॑म् का॒मये॑त का॒मये॒तार्ति॒ मा । \newline
69. आर्ति॒ मा ऽऽर्ति॒ मार्ति॒ मार्च्छे॑युर्. ऋच्छेयु॒रा ऽऽर्ति॒ मार्ति॒ मार्च्छे॑युः । \newline
70. आर्च्छे॑युर्. ऋच्छेयु॒ रार्च्छे॑यु॒ रिती त्यृ॑च्छेयु॒ रार्च्छे॑यु॒ रिति॑ । \newline
71. ऋ॒च्छे॒यु॒ रिती त्यृ॑च्छेयुर्. ऋच्छेयु॒रिति॑ कु॒वित् कु॒विदि त्यृ॑च्छेयुर्. ऋच्छेयु॒रिति॑ कु॒वित् । \newline
72. इति॑ कु॒वित् कु॒वि दितीति॑ कु॒वि द॒ङ्गाङ्ग कु॒वि दितीति॑ कु॒वि द॒ङ्ग । \newline
73. कु॒वि द॒ङ्गाङ्ग कु॒वित् कु॒वि द॒ङ्गे तीत्य॒ङ्ग कु॒वित् कु॒वि द॒ङ्गे ति॑ । \newline
74. अ॒ङ्गे तीत्य॒ङ्गाङ्गे ति॒ नमो॑वृक्तिवत्या॒ नमो॑वृक्तिव॒ त्येत्य॒ङ्गाङ्गे ति॒ नमो॑वृक्तिवत्या । \newline
75. इति॒ नमो॑वृक्तिवत्या॒ नमो॑वृक्तिव॒ त्येतीति॒ नमो॑वृक्तिवत्य॒ र्‌चर्चा नमो॑वृक्तिव॒ त्येतीति॒ नमो॑वृक्तिवत्य॒र्चा । \newline
76. नमो॑वृक्तिव त्य॒र्चर्चा नमो॑वृक्तिवत्या॒ नमो॑वृक्तिव त्य॒र्चा ऽऽग्नी᳚द्ध्र॒ आग्नी᳚द्ध्र ऋ॒चा नमो॑वृक्तिवत्या॒ नमो॑वृक्तिवत्य॒र्चा ऽऽग्नी᳚द्ध्रे । \newline
77. नमो॑वृक्तिव॒त्येति॒ नमो॑वृक्ति - व॒त्या॒ । \newline
78. ऋ॒चा ऽऽग्नी᳚द्ध्र॒ आग्नी᳚द्ध्र ऋ॒चर्चा ऽऽग्नी᳚द्ध्रे जुहुयाज् जुहुया॒ दाग्नी᳚द्ध्र ऋ॒चर्चा ऽऽग्नी᳚द्ध्रे जुहुयात् । \newline
79. आग्नी᳚द्ध्रे जुहुयाज् जुहुया॒ दाग्नी᳚द्ध्र॒ आग्नी᳚द्ध्रे जुहुया॒न् नमो॑वृक्ति॒म् नमो॑वृक्तिम् जुहुया॒ दाग्नी᳚द्ध्र॒ आग्नी᳚द्ध्रे जुहुया॒न् नमो॑वृक्तिम् । \newline
80. आग्नी᳚द्ध्र॒ इत्याग्नि॑ - इ॒ध्रे॒ । \newline
81. जु॒हु॒या॒न् नमो॑वृक्ति॒म् नमो॑वृक्तिम् जुहुयाज् जुहुया॒न् नमो॑वृक्ति मे॒वैव नमो॑वृक्तिम् जुहुयाज् जुहुया॒न् नमो॑वृक्ति मे॒व । \newline
82. नमो॑वृक्ति मे॒वैव नमो॑वृक्ति॒म् नमो॑वृक्ति मे॒वैषा॑ मेषा मे॒व नमो॑वृक्ति॒म् नमो॑वृक्ति मे॒वैषा᳚म् । \newline
83. नमो॑वृक्ति॒मिति॒ नमः॑ - वृ॒क्ति॒म् । \newline
84. ए॒वैषा॑ मेषा मे॒वैवैषां᳚ ॅवृङ्क्ते वृङ्क्त एषा मे॒वैवैषां᳚ ॅवृङ्क्ते । \newline
85. ए॒षां॒ ॅवृ॒ङ्क्ते॒ वृ॒ङ्क्त॒ ए॒षा॒ मे॒षां॒ ॅवृ॒ङ्क्ते॒ ता॒जक् ता॒जग् वृ॑ङ्क्त एषा मेषां ॅवृङ्क्ते ता॒जक् । \newline
86. वृ॒ङ्क्ते॒ ता॒जक् ता॒जग् वृ॑ङ्क्ते वृङ्क्ते ता॒जगार्ति॒ मार्ति॑म् ता॒जग् वृ॑ङ्क्ते वृङ्क्ते ता॒जगार्ति᳚म् । \newline
87. ता॒जगार्ति॒ मार्ति॑म् ता॒जक् ता॒जगार्ति॒ मा ऽऽर्ति॑म् ता॒जक् ता॒जगार्ति॒ मा । \newline
88. आर्ति॒ मा ऽऽर्ति॒ मार्ति॒ मार्च्छ॑ न्त्यृच्छ॒न्त्या ऽऽर्ति॒ मार्ति॒ मार्च्छ॑न्ति । \newline
89. आर्च्छ॑ न्त्यृच्छ॒ न्त्यार्च्छ॑न्ति । \newline
90. ऋ॒च्छ॒न्तीत्यृ॑च्छन्ति । \newline
\pagebreak
\markright{ TS 3.1.4.1  \hfill https://www.vedavms.in \hfill}

\section{ TS 3.1.4.1 }

\textbf{TS 3.1.4.1 } \newline
\textbf{Samhita Paata} \newline

प्र॒जाप॑ते॒र्जाय॑मानाः प्र॒जा जा॒ताश्च॒ या इ॒माः । तस्मै॒ प्रति॒ प्र वे॑दयचिकि॒त्वाꣳ अनु॑ मन्यतां ॥इ॒मं प॒शुं प॑शुपते ते अ॒द्य ब॒द्ध्नाम्य॑ग्ने सुकृ॒तस्य॒ मद्ध्ये᳚ । अनु॑ मन्यस्व सु॒यजा॑ यजाम॒ जुष्टं॑ दे॒वाना॑मि॒दम॑स्तु ह॒व्यं ॥ प्र॒जा॒नन्तः॒ प्रति॑गृह्णन्ति॒ पूर्वे᳚ प्रा॒णमङ्गे᳚भ्यः॒ पर्या॒चर॑न्तं ।सुव॒र्गं ॅया॑हि प॒थिभि॑ र्देव॒यानै॒-रोष॑धीषु॒ प्रति॑तिष्ठा॒ शरी॑रैः ॥ येषा॒मीशे॑ - [  ] \newline

\textbf{Pada Paata} \newline

प्र॒जाप॑ते॒रिति॑ प्र॒जा - प॒तेः॒ । जाय॑मानाः । प्र॒जा इति॑ प्र - जाः । जा॒ताः । च॒ । याः । इ॒माः ॥ तस्मै᳚ । प्रति॑ । प्रेति॑ । वे॒द॒य॒ । चि॒कि॒त्वान् । अन्विति॑ । म॒न्य॒ता॒म् ॥ इ॒मम् । प॒शुम् । प॒शु॒प॒त॒ इति॑ पशु - प॒ते॒ । ते॒ । अ॒द्य । ब॒द्ध्नामि॑ । अ॒ग्ने॒ । सु॒कृ॒तस्येति॑ सु - कृ॒तस्य॑ । मद्ध्ये᳚ ॥ अन्विति॑ । म॒न्य॒स्व॒ । सु॒यजेति॑ सु - यजा᳚ । य॒जा॒म॒ । जुष्ट᳚म् । दे॒वाना᳚म् । इ॒दम् । अ॒स्तु॒ । ह॒व्यम् ॥ प्र॒जा॒नन्त॒ इति॑ प्र - जा॒नन्तः॑ । प्रतीति॑ । गृ॒ह्ण॒न्ति॒ । पूर्वे᳚ । प्रा॒णमिति॑ प्र - अ॒नम् । अङ्गे᳚भ्यः । परीति॑ । आ॒चर॑न्त॒मित्या᳚ - चर॑न्तम् ॥ सु॒व॒र्गमिति॑ सुवः - गम् । या॒हि॒ । प॒थिभि॒रिति॑ प॒थि - भिः॒ । दे॒व॒यानै॒रिति॑ देव - यानैः᳚ । ओष॑धीषु । प्रतीति॑ । ति॒ष्ठ॒ । शरी॑रैः ॥ येषा᳚म् । ईशे᳚ ।  \newline


\textbf{Krama Paata} \newline

प्र॒जाप॑ते॒र् जाय॑मानाः । प्र॒जाप॑ते॒रिति॑ प्र॒जा - प॒तेः॒ । जाय॑मानाः प्र॒जाः । प्र॒जा जा॒ताः । प्र॒जा इति॑ प्र - जाः । जा॒ताश्च॑ । च॒ याः । या इ॒माः । इ॒मा इती॒माः ॥ तस्मै॒ प्रति॑ । प्रति॒ प्र । प्र वे॑दय । वे॒द॒य॒ चि॒कि॒त्वान् । चि॒कि॒त्वाꣳ अनु॑ । अनु॑ मन्यताम् । म॒न्य॒ता॒मिति॑ मन्यताम् ॥ इ॒मम् प॒शुम् । प॒शुम् प॑शुपते । प॒शु॒प॒ते॒ ते॒ । प॒शु॒प॒त॒ इति॑ पशु - प॒ते॒ । ते॒ अ॒द्य । अ॒द्य ब॒द्ध्नामि॑ । ब॒द्ध्नाम्य॑ग्ने । अ॒ग्ने॒ सु॒कृ॒तस्य॑ । सु॒कृ॒तस्य॒ मद्ध्ये᳚ । सु॒कृ॒तस्येति॑ सु - कृ॒तस्य॑ । मद्ध्य॒ इति॒ मद्ध्ये᳚ ॥ अनु॑ मन्यस्व । म॒न्य॒स्व॒ सु॒यजा᳚ । सु॒यजा॑ यजाम । सु॒यजेति॑ सु - यजा᳚ । य॒जा॒म॒ जुष्ट᳚म् । जुष्ट॑म् दे॒वाना᳚म् । दे॒वाना॑मि॒दम् । इ॒दम॑स्तु । अ॒स्तु॒ ह॒व्यम् । ह॒व्यमिति॑ ह॒व्यम् ॥ प्र॒जा॒नन्तः॒ प्रति॑ । प्र॒जा॒नन्त॒ इति॑ प्र - जा॒नन्तः॑ । प्रति॑ गृह्णन्ति । गृ॒ह्ण॒न्ति॒ पूर्वे᳚ । पूर्वे᳚ प्रा॒णम् । प्रा॒णमङ्गे᳚भ्यः । प्रा॒णमिति॑ प्र - अ॒नम् । अङ्गे᳚भ्यः॒ परि॑ । पर्या॒चर॑न्तम् । आ॒चर॑न्त॒मित्या᳚ - चर॑न्तम् ॥ सु॒व॒र्गं ॅया॑हि । सु॒व॒र्गमिति॑ सुवः - गम् । या॒हि॒ प॒थिभिः॑ । प॒थिभि॑र् देव॒यानैः᳚ । प॒थिभि॒रिति॑ प॒थि - भिः॒ । दे॒व॒यानै॒रोष॑धीषु । दे॒व॒यानै॒रिति॑ देव - यानैः᳚ । ओष॑धीषु॒ प्रति॑ । प्रति॑ तिष्ठ । ति॒ष्ठा॒ शरी॑रैः । शरी॑रै॒रिति॒ शरी॑रैः ॥ येषा॒मीशे᳚ । ईशे॑ पशु॒पतिः॑ \newline

\textbf{Jatai Paata} \newline

1. प्र॒जाप॑ते॒र् जाय॑माना॒ जाय॑मानाः प्र॒जाप॑तेः प्र॒जाप॑ते॒र् जाय॑मानाः । \newline
2. प्र॒जाप॑ते॒रिति॑ प्र॒जा - प॒तेः॒ । \newline
3. जाय॑मानाः प्र॒जाः प्र॒जा जाय॑माना॒ जाय॑मानाः प्र॒जाः । \newline
4. प्र॒जा जा॒ता जा॒ताः प्र॒जाः प्र॒जा जा॒ताः । \newline
5. प्र॒जा इति॑ प्र - जाः । \newline
6. जा॒ताश्च॑ च जा॒ता जा॒ताश्च॑ । \newline
7. च॒ या याश्च॑ च॒ याः । \newline
8. या इ॒मा इ॒मा या या इ॒माः । \newline
9. इ॒मा इती॒माः । \newline
10. तस्मै॒ प्रति॒ प्रति॒ तस्मै॒ तस्मै॒ प्रति॑ । \newline
11. प्रति॒ प्र प्र प्रति॒ प्रति॒ प्र । \newline
12. प्र वे॑दय वेदय॒ प्र प्र वे॑दय । \newline
13. वे॒द॒य॒ चि॒कि॒त्वाꣳ श्चि॑कि॒त्वान्. वे॑दय वेदय चिकि॒त्वान् । \newline
14. चि॒कि॒त्वाꣳ अन्वनु॑ चिकि॒त्वाꣳ श्चि॑कि॒त्वाꣳ अनु॑ । \newline
15. अनु॑ मन्यताम् मन्यता॒ मन्वनु॑ मन्यताम् । \newline
16. म॒न्य॒ता॒मिति॑ मन्यताम् । \newline
17. इ॒मम् प॒शुम् प॒शु मि॒म मि॒मम् प॒शुम् । \newline
18. प॒शुम् प॑शुपते पशुपते प॒शुम् प॒शुम् प॑शुपते । \newline
19. प॒शु॒प॒ते॒ ते॒ ते॒ प॒शु॒प॒ते॒ प॒शु॒प॒ते॒ ते॒ । \newline
20. प॒शु॒प॒त॒ इति॑ पशु - प॒ते॒ । \newline
21. ते॒ अ॒द्याद्य ते॑ ते अ॒द्य । \newline
22. अ॒द्य ब॒द्ध्नामि॑ ब॒द्ध्ना म्य॒द्याद्य ब॒द्ध्नामि॑ । \newline
23. ब॒द्ध्ना म्य॑ग्ने ऽग्ने ब॒द्ध्नामि॑ ब॒द्ध्ना म्य॑ग्ने । \newline
24. अ॒ग्ने॒ सु॒कृ॒तस्य॑ सुकृ॒तस्या᳚ग्ने ऽग्ने सुकृ॒तस्य॑ । \newline
25. सु॒कृ॒तस्य॒ मद्ध्ये॒ मद्ध्ये॑ सुकृ॒तस्य॑ सुकृ॒तस्य॒ मद्ध्ये᳚ । \newline
26. सु॒कृ॒तस्येति॑ सु - कृ॒तस्य॑ । \newline
27. मद्ध्य॒ इति॒ मद्ध्ये᳚ । \newline
28. अनु॑ मन्यस्व मन्य॒स्वा न्वनु॑ मन्यस्व । \newline
29. म॒न्य॒स्व॒ सु॒यजा॑ सु॒यजा॑ मन्यस्व मन्यस्व सु॒यजा᳚ । \newline
30. सु॒यजा॑ यजाम यजाम सु॒यजा॑ सु॒यजा॑ यजाम । \newline
31. सु॒यजेति॑ सु - यजा᳚ । \newline
32. य॒जा॒म॒ जुष्ट॒म् जुष्टं॑ ॅयजाम यजाम॒ जुष्ट᳚म् । \newline
33. जुष्ट॑म् दे॒वाना᳚म् दे॒वाना॒म् जुष्ट॒म् जुष्ट॑म् दे॒वाना᳚म् । \newline
34. दे॒वाना॑ मि॒द मि॒दम् दे॒वाना᳚म् दे॒वाना॑ मि॒दम् । \newline
35. इ॒द म॑स्त्व स्त्वि॒द मि॒द म॑स्तु । \newline
36. अ॒स्तु॒ ह॒व्यꣳ ह॒व्य म॑स्त्वस्तु ह॒व्यम् । \newline
37. ह॒व्यमिति॑ ह॒व्यम् । \newline
38. प्र॒जा॒नन्तः॒ प्रति॒ प्रति॑ प्रजा॒नन्तः॑ प्रजा॒नन्तः॒ प्रति॑ । \newline
39. प्र॒जा॒नन्त॒ इति॑ प्र - जा॒नन्तः॑ । \newline
40. प्रति॑ गृह्णन्ति गृह्णन्ति॒ प्रति॒ प्रति॑ गृह्णन्ति । \newline
41. गृ॒ह्ण॒न्ति॒ पूर्वे॒ पूर्वे॑ गृह्णन्ति गृह्णन्ति॒ पूर्वे᳚ । \newline
42. पूर्वे᳚ प्रा॒णम् प्रा॒णम् पूर्वे॒ पूर्वे᳚ प्रा॒णम् । \newline
43. प्रा॒ण मङ्गे॒भ्यो ऽङ्गे᳚भ्यः प्रा॒णम् प्रा॒ण मङ्गे᳚भ्यः । \newline
44. प्रा॒णमिति॑ प्र - अ॒नम् । \newline
45. अङ्गे᳚भ्यः॒ परि॒ पर्यङ्गे॒भ्यो ऽङ्गे᳚भ्यः॒ परि॑ । \newline
46. पर्या॒चर॑न्त मा॒चर॑न्त॒म् परि॒ पर्या॒चर॑न्तम् । \newline
47. आ॒चर॑न्त॒मित्या᳚ - चर॑न्तम् । \newline
48. सु॒व॒र्गं ॅया॑हि याहि सुव॒र्गꣳ सु॑व॒र्गं ॅया॑हि । \newline
49. सु॒व॒र्गमिति॑ सुवः - गम् । \newline
50. या॒हि॒ प॒थिभिः॑ प॒थिभि॑र् याहि याहि प॒थिभिः॑ । \newline
51. प॒थिभि॑र् देव॒यानै᳚र् देव॒यानैः᳚ प॒थिभिः॑ प॒थिभि॑र् देव॒यानैः᳚ । \newline
52. प॒थिभि॒रिति॑ प॒थि - भिः॒ । \newline
53. दे॒व॒यानै॒ रोष॑धी॒ ष्वोष॑धीषु देव॒यानै᳚र् देव॒यानै॒ रोष॑धीषु । \newline
54. दे॒व॒यानै॒रिति॑ देव - यानैः᳚ । \newline
55. ओष॑धीषु॒ प्रति॒ प्रत्योष॑धी॒ ष्वोष॑धीषु॒ प्रति॑ । \newline
56. प्रति॑ तिष्ठ तिष्ठ॒ प्रति॒ प्रति॑ तिष्ठ । \newline
57. ति॒ष्ठा॒ शरी॑रैः॒ शरी॑रै स्तिष्ठ तिष्ठा॒ शरी॑रैः । \newline
58. शरी॑रै॒रिति॒ शरी॑रैः । \newline
59. येषा॒ मीश॒ ईशे॒ येषां॒ ॅयेषा॒ मीशे᳚ । \newline
60. ईशे॑ पशु॒पतिः॑ पशु॒पति॒ रीश॒ ईशे॑ पशु॒पतिः॑ । \newline

\textbf{Ghana Paata } \newline

1. प्र॒जाप॑ते॒र् जाय॑माना॒ जाय॑मानाः प्र॒जाप॑तेः प्र॒जाप॑ते॒र् जाय॑मानाः प्र॒जाः प्र॒जा जाय॑मानाः प्र॒जाप॑तेः प्र॒जाप॑ते॒र् जाय॑मानाः प्र॒जाः । \newline
2. प्र॒जाप॑ते॒रिति॑ प्र॒जा - प॒तेः॒ । \newline
3. जाय॑मानाः प्र॒जाः प्र॒जा जाय॑माना॒ जाय॑मानाः प्र॒जा जा॒ता जा॒ताः प्र॒जा जाय॑माना॒ जाय॑मानाः प्र॒जा जा॒ताः । \newline
4. प्र॒जा जा॒ता जा॒ताः प्र॒जाः प्र॒जा जा॒ताश्च॑ च जा॒ताः प्र॒जाः प्र॒जा जा॒ताश्च॑ । \newline
5. प्र॒जा इति॑ प्र - जाः । \newline
6. जा॒ताश्च॑ च जा॒ता जा॒ताश्च॒ या याश्च॑ जा॒ता जा॒ताश्च॒ याः । \newline
7. च॒ या याश्च॑ च॒ या इ॒मा इ॒मा याश्च॑ च॒ या इ॒माः । \newline
8. या इ॒मा इ॒मा या या इ॒माः । \newline
9. इ॒मा इती॒माः । \newline
10. तस्मै॒ प्रति॒ प्रति॒ तस्मै॒ तस्मै॒ प्रति॒ प्र प्र प्रति॒ तस्मै॒ तस्मै॒ प्रति॒ प्र । \newline
11. प्रति॒ प्र प्र प्रति॒ प्रति॒ प्र वे॑दय वेदय॒ प्र प्रति॒ प्रति॒ प्र वे॑दय । \newline
12. प्र वे॑दय वेदय॒ प्र प्र वे॑दय चिकि॒त्वाꣳ श्चि॑कि॒त्वान्. वे॑दय॒ प्र प्र वे॑दय चिकि॒त्वान् । \newline
13. वे॒द॒य॒ चि॒कि॒त्वाꣳ श्चि॑कि॒त्वान्. वे॑दय वेदय चिकि॒त्वाꣳ अन्वनु॑ चिकि॒त्वान्. वे॑दय वेदय चिकि॒त्वाꣳ अनु॑ । \newline
14. चि॒कि॒त्वाꣳ अन्वनु॑ चिकि॒त्वाꣳ श्चि॑कि॒त्वाꣳ अनु॑ मन्यताम् मन्यता॒ मनु॑ चिकि॒त्वाꣳ श्चि॑कि॒त्वाꣳ अनु॑ मन्यताम् । \newline
15. अनु॑ मन्यताम् मन्यता॒ मन्वनु॑ मन्यताम् । \newline
16. म॒न्य॒ता॒मिति॑ मन्यताम् । \newline
17. इ॒मम् प॒शुम् प॒शु मि॒म मि॒मम् प॒शुम् प॑शुपते पशुपते प॒शु मि॒म मि॒मम् प॒शुम् प॑शुपते । \newline
18. प॒शुम् प॑शुपते पशुपते प॒शुम् प॒शुम् प॑शुपते ते ते पशुपते प॒शुम् प॒शुम् प॑शुपते ते । \newline
19. प॒शु॒प॒ते॒ ते॒ ते॒ प॒शु॒प॒ते॒ प॒शु॒प॒ते॒ ते॒ अ॒द्याद्य ते॑ पशुपते पशुपते ते अ॒द्य । \newline
20. प॒शु॒प॒त॒ इति॑ पशु - प॒ते॒ । \newline
21. ते॒ अ॒द्याद्य ते॑ ते अ॒द्य ब॒द्ध्नामि॑ ब॒द्ध्ना म्य॒द्य ते॑ ते अ॒द्य ब॒द्ध्नामि॑ । \newline
22. अ॒द्य ब॒द्ध्नामि॑ ब॒द्ध्ना म्य॒द्याद्य ब॒द्ध्नाम्य॑ग्ने ऽग्ने ब॒द्ध्ना म्य॒द्याद्य ब॒द्ध्नाम्य॑ग्ने । \newline
23. ब॒द्ध्नाम्य॑ग्ने ऽग्ने ब॒द्ध्नामि॑ ब॒द्ध्ना म्य॑ग्ने सुकृ॒तस्य॑ सुकृ॒तस्या᳚ग्ने ब॒द्ध्नामि॑ ब॒द्ध्ना म्य॑ग्ने सुकृ॒तस्य॑ । \newline
24. अ॒ग्ने॒ सु॒कृ॒तस्य॑ सुकृ॒तस्या᳚ग्ने ऽग्ने सुकृ॒तस्य॒ मद्ध्ये॒ मद्ध्ये॑ सुकृ॒तस्या᳚ग्ने ऽग्ने सुकृ॒तस्य॒ मद्ध्ये᳚ । \newline
25. सु॒कृ॒तस्य॒ मद्ध्ये॒ मद्ध्ये॑ सुकृ॒तस्य॑ सुकृ॒तस्य॒ मद्ध्ये᳚ । \newline
26. सु॒कृ॒तस्येति॑ सु - कृ॒तस्य॑ । \newline
27. मद्ध्य॒ इति॒ मद्ध्ये᳚ । \newline
28. अनु॑ मन्यस्व मन्य॒स्वान्वनु॑ मन्यस्व सु॒यजा॑ सु॒यजा॑ मन्य॒स्वान्वनु॑ मन्यस्व सु॒यजा᳚ । \newline
29. म॒न्य॒स्व॒ सु॒यजा॑ सु॒यजा॑ मन्यस्व मन्यस्व सु॒यजा॑ यजाम यजाम सु॒यजा॑ मन्यस्व मन्यस्व सु॒यजा॑ यजाम । \newline
30. सु॒यजा॑ यजाम यजाम सु॒यजा॑ सु॒यजा॑ यजाम॒ जुष्ट॒म् जुष्टं॑ ॅयजाम सु॒यजा॑ सु॒यजा॑ यजाम॒ जुष्ट᳚म् । \newline
31. सु॒यजेति॑ सु - यजा᳚ । \newline
32. य॒जा॒म॒ जुष्ट॒म् जुष्टं॑ ॅयजाम यजाम॒ जुष्ट॑म् दे॒वाना᳚म् दे॒वाना॒म् जुष्टं॑ ॅयजाम यजाम॒ जुष्ट॑म् दे॒वाना᳚म् । \newline
33. जुष्ट॑म् दे॒वाना᳚म् दे॒वाना॒म् जुष्ट॒म् जुष्ट॑म् दे॒वाना॑ मि॒द मि॒दम् दे॒वाना॒म् जुष्ट॒म् जुष्ट॑म् दे॒वाना॑ मि॒दम् । \newline
34. दे॒वाना॑ मि॒द मि॒दम् दे॒वाना᳚म् दे॒वाना॑ मि॒द म॑स्त्व स्त्वि॒दम् दे॒वाना᳚म् दे॒वाना॑ मि॒द म॑स्तु । \newline
35. इ॒द म॑स्त्व स्त्वि॒द मि॒द म॑स्तु ह॒व्यꣳ ह॒व्य म॑स्त्वि॒द मि॒द म॑स्तु ह॒व्यम् । \newline
36. अ॒स्तु॒ ह॒व्यꣳ ह॒व्य म॑स्त्वस्तु ह॒व्यम् । \newline
37. ह॒व्यमिति॑ ह॒व्यम् । \newline
38. प्र॒जा॒नन्तः॒ प्रति॒ प्रति॑ प्रजा॒नन्तः॑ प्रजा॒नन्तः॒ प्रति॑ गृह्णन्ति गृह्णन्ति॒ प्रति॑ प्रजा॒नन्तः॑ प्रजा॒नन्तः॒ प्रति॑ गृह्णन्ति । \newline
39. प्र॒जा॒नन्त॒ इति॑ प्र - जा॒नन्तः॑ । \newline
40. प्रति॑ गृह्णन्ति गृह्णन्ति॒ प्रति॒ प्रति॑ गृह्णन्ति॒ पूर्वे॒ पूर्वे॑ गृह्णन्ति॒ प्रति॒ प्रति॑ गृह्णन्ति॒ पूर्वे᳚ । \newline
41. गृ॒ह्ण॒न्ति॒ पूर्वे॒ पूर्वे॑ गृह्णन्ति गृह्णन्ति॒ पूर्वे᳚ प्रा॒णम् प्रा॒णम् पूर्वे॑ गृह्णन्ति गृह्णन्ति॒ पूर्वे᳚ प्रा॒णम् । \newline
42. पूर्वे᳚ प्रा॒णम् प्रा॒णम् पूर्वे॒ पूर्वे᳚ प्रा॒ण मङ्गे॒भ्यो ऽङ्गे᳚भ्यः प्रा॒णम् पूर्वे॒ पूर्वे᳚ प्रा॒ण मङ्गे᳚भ्यः । \newline
43. प्रा॒ण मङ्गे॒भ्यो ऽङ्गे᳚भ्यः प्रा॒णम् प्रा॒ण मङ्गे᳚भ्यः॒ परि॒ पर्यङ्गे᳚भ्यः प्रा॒णम् प्रा॒ण मङ्गे᳚भ्यः॒ परि॑ । \newline
44. प्रा॒णमिति॑ प्र - अ॒नम् । \newline
45. अङ्गे᳚भ्यः॒ परि॒ पर्यङ्गे॒भ्यो ऽङ्गे᳚भ्यः॒ पर्या॒चर॑न्त मा॒चर॑न्त॒म् पर्यङ्गे॒भ्यो ऽङ्गे᳚भ्यः॒ पर्या॒चर॑न्तम् । \newline
46. पर्या॒चर॑न्त मा॒चर॑न्त॒म् परि॒ पर्या॒चर॑न्तम् । \newline
47. आ॒चर॑न्त॒मित्या᳚ - चर॑न्तम् । \newline
48. सु॒व॒र्गं ॅया॑हि याहि सुव॒र्गꣳ सु॑व॒र्गं ॅया॑हि प॒थिभिः॑ प॒थिभि॑र् याहि सुव॒र्गꣳ सु॑व॒र्गं ॅया॑हि प॒थिभिः॑ । \newline
49. सु॒व॒र्गमिति॑ सुवः - गम् । \newline
50. या॒हि॒ प॒थिभिः॑ प॒थिभि॑र् याहि याहि प॒थिभि॑र् देव॒यानै᳚र् देव॒यानैः᳚ प॒थिभि॑र् याहि याहि प॒थिभि॑र् देव॒यानैः᳚ । \newline
51. प॒थिभि॑र् देव॒यानै᳚र् देव॒यानैः᳚ प॒थिभिः॑ प॒थिभि॑र् देव॒यानै॒ रोष॑धी॒ ष्वोष॑धीषु देव॒यानैः᳚ प॒थिभिः॑ प॒थिभि॑र् देव॒यानै॒ रोष॑धीषु । \newline
52. प॒थिभि॒रिति॑ प॒थि - भिः॒ । \newline
53. दे॒व॒यानै॒ रोष॑धी॒ ष्वोष॑धीषु देव॒यानै᳚र् देव॒यानै॒ रोष॑धीषु॒ प्रति॒ प्रत्योष॑धीषु देव॒यानै᳚र् देव॒यानै॒ रोष॑धीषु॒ प्रति॑ । \newline
54. दे॒व॒यानै॒रिति॑ देव - यानैः᳚ । \newline
55. ओष॑धीषु॒ प्रति॒ प्रत्योष॑धी॒ ष्वोष॑धीषु॒ प्रति॑ तिष्ठ तिष्ठ॒ प्रत्योष॑धी॒ ष्वोष॑धीषु॒ प्रति॑ तिष्ठ । \newline
56. प्रति॑ तिष्ठ तिष्ठ॒ प्रति॒ प्रति॑ तिष्ठा॒ शरी॑रैः॒ शरी॑रै स्तिष्ठ॒ प्रति॒ प्रति॑ तिष्ठा॒ शरी॑रैः । \newline
57. ति॒ष्ठा॒ शरी॑रैः॒ शरी॑रै स्तिष्ठ तिष्ठा॒ शरी॑रैः । \newline
58. शरी॑रै॒रिति॒ शरी॑रैः । \newline
59. येषा॒ मीश॒ ईशे॒ येषां॒ ॅयेषा॒ मीशे॑ पशु॒पतिः॑ पशु॒पति॒ रीशे॒ येषां॒ ॅयेषा॒ मीशे॑ पशु॒पतिः॑ । \newline
60. ईशे॑ पशु॒पतिः॑ पशु॒पति॒ रीश॒ ईशे॑ पशु॒पतिः॑ पशू॒नाम् प॑शू॒नाम् प॑शु॒पति॒ रीश॒ ईशे॑ पशु॒पतिः॑ पशू॒नाम् । \newline
\pagebreak
\markright{ TS 3.1.4.2  \hfill https://www.vedavms.in \hfill}

\section{ TS 3.1.4.2 }

\textbf{TS 3.1.4.2 } \newline
\textbf{Samhita Paata} \newline

पशु॒पतिः॑ पशू॒नां चतु॑ष्पदामु॒त च॑ द्वि॒पदां᳚ । निष्क्री॑तो॒ऽयं ॅय॒ज्ञियं॑ भा॒गमे॑तु रा॒यस्पोषा॒ यज॑मानस्य सन्तु ॥ ये ब॒द्ध्यमा॑न॒मनु॑ ब॒द्ध्यमा॑ना अ॒भ्यैक्ष॑न्त॒ मन॑सा॒ चक्षु॑षा च । अ॒ग्निस्ताꣳ अग्रे॒ प्रमु॑मोक्तु दे॒वः प्र॒जाप॑तिः प्र॒जया॑ संॅविदा॒नः ॥ य आ॑र॒ण्याः प॒शवो॑ वि॒श्वरू॑पा॒ विरू॑पाः॒ सन्तो॑ बहु॒धैक॑रूपाः । वा॒युस्ताꣳ अग्रे॒ प्रमु॑मोक्तु दे॒वः प्र॒जाप॑तिः प्र॒जया॑ संॅविदा॒नः ॥ प्र॒मु॒ञ्चमा॑ना॒ - [  ] \newline

\textbf{Pada Paata} \newline

प॒श॒पति॒रिति॑ पशु-पतिः॑ । प॒शू॒नाम् । चतु॑ष्पदा॒मिति॒ चतुः॑ - प॒दा॒म् । उ॒त । च॒ । द्वि॒पदा॒मिति॑ द्वि - पदा᳚म् ॥ निष्क्री॑त॒ इति॒ निः - क्री॒तः॒ । अ॒यम् । य॒ज्ञिय᳚म् । भा॒गम् । ए॒तु॒ । रा॒यः । पोषाः᳚ । यज॑मानस्य । स॒न्तु॒ ॥ ये । ब॒द्ध्यमा॑नम् । अन्विति॑ । ब॒द्ध्यमा॑नाः । अ॒भ्यैक्ष॒न्तेत्य॑भि - ऐक्ष॑न्त । मन॑सा । चक्षु॑षा । च॒ ॥ अ॒ग्निः । तान् । अग्रे᳚ । प्रेति॑ । मु॒मो॒क्तु॒ । दे॒वः । प्र॒जाप॑ति॒रिति॑ प्र॒जा - प॒तिः॒ । प्र॒जयेति॑ प्र - जया᳚ । सं॒ॅवि॒दा॒न इति॑ सं-वि॒दा॒नः ॥ ये । आ॒र॒ण्याः । प॒शवः॑ । वि॒श्वरू॑पा॒ इति॑ वि॒श्व - रू॒पाः॒ । विरू॑पा॒ इति॒ वि-रू॒पाः॒ । सन्तः॑ । ब॒हु॒धेति॑ बहु - धा । एक॑रूपा॒ इत्येक॑ - रू॒पाः॒ ॥ वा॒युः । तान् । अग्रे᳚ । प्रेति॑ । मु॒मो॒क्तु॒ । दे॒वः । प्र॒जाप॑ति॒रिति॑ प्र॒जा - प॒तिः॒ । प्र॒जयेति॑ प्र - जया᳚ । सं॒ॅवि॒दा॒न इति॑ सं - वि॒दा॒नः ॥ प्र॒मु॒ञ्चमा॑ना॒ इति॑ प्र - मु॒ञ्चमा॑नाः ।  \newline


\textbf{Krama Paata} \newline

प॒शु॒पतिः॑ पशू॒नाम् । प॒शु॒पति॒रिति॑ पशु - पतिः॑ । प॒शू॒नाम् चतु॑ष्पदाम् । चतु॑ष्पदामु॒त । चतु॑ष्पदा॒मिति॒ चतुः॑ - प॒दा॒म् । उ॒त च॑ । च॒ द्वि॒पदा᳚म् । द्वि॒पदा॒मिति॑ द्वि - पदा᳚म् ॥ निष्क्री॑तो॒ऽयम् । निष्क्री॑त॒ इति॒ निः - क्री॒तः॒ । अ॒यं ॅय॒ज्ञिय᳚म् । य॒ज्ञिय॑म् भा॒गम् । भा॒गमे॑तु । ए॒तु॒ रा॒यः । रा॒यस्पोषाः᳚ । पोषा॒ यज॑मानस्य । यज॑मानस्य सन्तु । स॒न्त्विति॑ सन्तु ॥ ये ब॒द्ध्यमा॑नम् । ब॒द्ध्यमा॑न॒मनु॑ । अनु॑ ब॒द्ध्यमा॑नाः । ब॒द्ध्यमा॑ना अ॒भ्यैक्ष॑न्त । अ॒भ्यैक्ष॑न्त॒ मन॑सा । अ॒भ्यैक्ष॒न्तेत्य॑भि - ऐक्ष॑न्त । मन॑सा॒ चक्षु॑षा । चक्षु॑षा च । चेति॑ च ॥ अ॒ग्निस्तान् । ताꣳ अग्रे᳚ । अग्रे॒ प्र । प्र मु॑मोक्तु । मु॒मो॒क्तु॒ दे॒वः । दे॒वः प्र॒जाप॑तिः । प्र॒जाप॑तिः प्र॒जया᳚ । प्र॒जाप॑ति॒रिति॑ प्र॒जा - प॒तिः॒ । प्र॒जया॑ सम्ॅविदा॒नः । प्र॒जयेति॑ प्र - जया᳚ । स॒म्ॅवि॒दा॒न इति॑ सं - वि॒दा॒नः ॥ य आ॑र॒ण्याः । आ॒र॒ण्याः प॒शवः॑ । प॒शवो॑ वि॒श्वरू॑पाः । वि॒श्वरू॑पा॒ विरू॑पाः । वि॒श्वरू॑पा॒ इति॑ वि॒श्व - रू॒पाः॒ । विरू॑पाः॒ सन्तः॑ । विरू॑पा॒ इति॒ वि - रू॒पाः॒ । सन्तो॑ बहु॒धा । ब॒हु॒धैक॑रूपाः । ब॒हु॒धेति॑ बहु - धा । एक॑रूपा॒ इत्येक॑ - रू॒पाः॒ ॥ वा॒युस्तान् । ताꣳ अग्रे᳚ । अग्रे॒ प्र । प्र मु॑मोक्तु । मु॒मो॒क्तु॒ दे॒वः । दे॒वः प्र॒जाप॑तिः । प्र॒जाप॑तिः प्र॒जया᳚ । प्र॒जाप॑ति॒रिति॑ प्र॒जा - प॒तिः॒ । प्र॒जया॑ सम्ॅविदा॒नः । प्र॒जयेति॑ प्र - जया᳚ । स॒म्ॅवि॒दा॒न इति॑ सं - वि॒दा॒नः ॥ प्र॒मु॒ञ्चमा॑ना॒ भुव॑नस्य । प्र॒मु॒ञ्चमा॑ना॒ इति॑ प्र - मु॒ञ्चमा॑नाः \newline

\textbf{Jatai Paata} \newline

1. प॒शु॒पतिः॑ पशू॒नाम् प॑शू॒नाम् प॑शु॒पतिः॑ पशु॒पतिः॑ पशू॒नाम् । \newline
2. प॒शु॒पति॒रिति॑ पशु - पतिः॑ । \newline
3. प॒शू॒नाम् चतु॑ष्पदा॒म् चतु॑ष्पदाम् पशू॒नाम् प॑शू॒नाम् चतु॑ष्पदाम् । \newline
4. चतु॑ष्पदा मु॒तोत चतु॑ष्पदा॒म् चतु॑ष्पदा मु॒त । \newline
5. चतु॑ष्पदा॒मिति॒ चतुः॑ - प॒दा॒म् । \newline
6. उ॒त च॑ चो॒तोत च॑ । \newline
7. च॒ द्वि॒पदा᳚म् द्वि॒पदा᳚म् च च द्वि॒पदा᳚म् । \newline
8. द्वि॒पदा॒मिति॑ द्वि - पदा᳚म् । \newline
9. निष्क्री॑तो॒ ऽय म॒यम् निष्क्री॑तो॒ निष्क्री॑तो॒ ऽयम् । \newline
10. निष्क्री॑त॒ इति॒ निः - क्री॒तः॒ । \newline
11. अ॒यं ॅय॒ज्ञियं॑ ॅय॒ज्ञिय॑ म॒य म॒यं ॅय॒ज्ञिय᳚म् । \newline
12. य॒ज्ञिय॑म् भा॒गम् भा॒गं ॅय॒ज्ञियं॑ ॅय॒ज्ञिय॑म् भा॒गम् । \newline
13. भा॒ग मे᳚त्वेतु भा॒गम् भा॒ग मे॑तु । \newline
14. ए॒तु॒ रा॒यो रा॒य ए᳚त्वेतु रा॒यः । \newline
15. रा॒य स्पोषाः॒ पोषा॑ रा॒यो रा॒य स्पोषाः᳚ । \newline
16. पोषा॒ यज॑मानस्य॒ यज॑मानस्य॒ पोषाः॒ पोषा॒ यज॑मानस्य । \newline
17. यज॑मानस्य सन्तु सन्तु॒ यज॑मानस्य॒ यज॑मानस्य सन्तु । \newline
18. स॒न्त्विति॑ सन्तु । \newline
19. ये ब॒द्ध्यमा॑नम् ब॒द्ध्यमा॑नं॒ ॅये ये ब॒द्ध्यमा॑नम् । \newline
20. ब॒द्ध्यमा॑न॒ मन्वनु॑ ब॒द्ध्यमा॑नम् ब॒द्ध्यमा॑न॒ मनु॑ । \newline
21. अनु॑ ब॒द्ध्यमा॑ना ब॒द्ध्यमा॑ना॒ अन्वनु॑ ब॒द्ध्यमा॑नाः । \newline
22. ब॒द्ध्यमा॑ना अ॒भ्यैक्ष॑न्ता॒ भ्यैक्ष॑न्त ब॒द्ध्यमा॑ना ब॒द्ध्यमा॑ना अ॒भ्यैक्ष॑न्त । \newline
23. अ॒भ्यैक्ष॑न्त॒ मन॑सा॒ मन॑सा॒ ऽभ्यैक्ष॑न्ता॒ भ्यैक्ष॑न्त॒ मन॑सा । \newline
24. अ॒भ्यैक्ष॒न्तेत्य॑भि - ऐक्ष॑न्त । \newline
25. मन॑सा॒ चक्षु॑षा॒ चक्षु॑षा॒ मन॑सा॒ मन॑सा॒ चक्षु॑षा । \newline
26. चक्षु॑षा च च॒ चक्षु॑षा॒ चक्षु॑षा च । \newline
27. चेति॑ च । \newline
28. अ॒ग्नि स्ताꣳ स्ताꣳ अ॒ग्नि र॒ग्नि स्तान् । \newline
29. ताꣳ अग्रे ऽग्रे॒ ताꣳ स्ताꣳ अग्रे᳚ । \newline
30. अग्रे॒ प्र प्राग्रे ऽग्रे॒ प्र । \newline
31. प्र मु॑मोक्तु मुमोक्तु॒ प्र प्र मु॑मोक्तु । \newline
32. मु॒मो॒क्तु॒ दे॒वो दे॒वो मु॑मोक्तु मुमोक्तु दे॒वः । \newline
33. दे॒वः प्र॒जाप॑तिः प्र॒जाप॑तिर् दे॒वो दे॒वः प्र॒जाप॑तिः । \newline
34. प्र॒जाप॑तिः प्र॒जया᳚ प्र॒जया᳚ प्र॒जाप॑तिः प्र॒जाप॑तिः प्र॒जया᳚ । \newline
35. प्र॒जाप॑ति॒रिति॑ प्र॒जा - प॒तिः॒ । \newline
36. प्र॒जया॑ संॅविदा॒नः सं॑ॅविदा॒नः प्र॒जया᳚ प्र॒जया॑ संॅविदा॒नः । \newline
37. प्र॒जयेति॑ प्र - जया᳚ । \newline
38. सं॒ॅवि॒दा॒न इति॑ सं - वि॒दा॒नः । \newline
39. य आ॑र॒ण्या आ॑र॒ण्या ये य आ॑र॒ण्याः । \newline
40. आ॒र॒ण्याः प॒शवः॑ प॒शव॑ आर॒ण्या आ॑र॒ण्याः प॒शवः॑ । \newline
41. प॒शवो॑ वि॒श्वरू॑पा वि॒श्वरू॑पाः प॒शवः॑ प॒शवो॑ वि॒श्वरू॑पाः । \newline
42. वि॒श्वरू॑पा॒ विरू॑पा॒ विरू॑पा वि॒श्वरू॑पा वि॒श्वरू॑पा॒ विरू॑पाः । \newline
43. वि॒श्वरू॑पा॒ इति॑ वि॒श्व - रू॒पाः॒ । \newline
44. विरू॑पाः॒ सन्तः॒ सन्तो॒ विरू॑पा॒ विरू॑पाः॒ सन्तः॑ । \newline
45. विरू॑पा॒ इति॒ वि - रू॒पाः॒ । \newline
46. सन्तो॑ बहु॒धा ब॑हु॒धा सन्तः॒ सन्तो॑ बहु॒धा । \newline
47. ब॒हु॒ धैक॑रूपा॒ एक॑रूपा बहु॒धा ब॑हु॒ धैक॑रूपाः । \newline
48. ब॒हु॒धेति॑ बहु - धा । \newline
49. एक॑रूपा॒ इत्येक॑ - रू॒पाः॒ । \newline
50. वा॒यु स्ताꣳ स्तान्. वा॒युर् वा॒यु स्तान् । \newline
51. ताꣳ अग्रे ऽग्रे॒ ताꣳ स्ताꣳ अग्रे᳚ । \newline
52. अग्रे॒ प्र प्राग्रे ऽग्रे॒ प्र । \newline
53. प्र मु॑मोक्तु मुमोक्तु॒ प्र प्र मु॑मोक्तु । \newline
54. मु॒मो॒क्तु॒ दे॒वो दे॒वो मु॑मोक्तु मुमोक्तु दे॒वः । \newline
55. दे॒वः प्र॒जाप॑तिः प्र॒जाप॑तिर् दे॒वो दे॒वः प्र॒जाप॑तिः । \newline
56. प्र॒जाप॑तिः प्र॒जया᳚ प्र॒जया᳚ प्र॒जाप॑तिः प्र॒जाप॑तिः प्र॒जया᳚ । \newline
57. प्र॒जाप॑ति॒रिति॑ प्र॒जा - प॒तिः॒ । \newline
58. प्र॒जया॑ संॅविदा॒नः सं॑ॅविदा॒नः प्र॒जया᳚ प्र॒जया॑ संॅविदा॒नः । \newline
59. प्र॒जयेति॑ प्र - जया᳚ । \newline
60. सं॒ॅवि॒दा॒न इति॑ सं - वि॒दा॒नः । \newline
61. प्र॒मु॒ञ्चमा॑ना॒ भुव॑नस्य॒ भुव॑नस्य प्रमु॒ञ्चमा॑नाः प्रमु॒ञ्चमा॑ना॒ भुव॑नस्य । \newline
62. प्र॒मु॒ञ्चमा॑ना॒ इति॑ प्र - मु॒ञ्चमा॑नाः । \newline

\textbf{Ghana Paata } \newline

1. प॒शु॒पतिः॑ पशू॒नाम् प॑शू॒नाम् प॑शु॒पतिः॑ पशु॒पतिः॑ पशू॒नाम् चतु॑ष्पदा॒म् चतु॑ष्पदाम् पशू॒नाम् प॑शु॒पतिः॑ पशु॒पतिः॑ पशू॒नाम् चतु॑ष्पदाम् । \newline
2. प॒शु॒पति॒रिति॑ पशु - पतिः॑ । \newline
3. प॒शू॒नाम् चतु॑ष्पदा॒म् चतु॑ष्पदाम् पशू॒नाम् प॑शू॒नाम् चतु॑ष्पदा मु॒तोत चतु॑ष्पदाम् पशू॒नाम् प॑शू॒नाम् चतु॑ष्पदा मु॒त । \newline
4. चतु॑ष्पदा मु॒तोत चतु॑ष्पदा॒म् चतु॑ष्पदा मु॒त च॑ चो॒त चतु॑ष्पदा॒म् चतु॑ष्पदा मु॒त च॑ । \newline
5. चतु॑ष्पदा॒मिति॒ चतुः॑ - प॒दा॒म् । \newline
6. उ॒त च॑ चो॒तोत च॑ द्वि॒पदा᳚म् द्वि॒पदा᳚म् चो॒तोत च॑ द्वि॒पदा᳚म् । \newline
7. च॒ द्वि॒पदा᳚म् द्वि॒पदा᳚म् च च द्वि॒पदा᳚म् । \newline
8. द्वि॒पदा॒मिति॑ द्वि - पदा᳚म् । \newline
9. निष्क्री॑तो॒ ऽय म॒यम् निष्क्री॑तो॒ निष्क्री॑तो॒ ऽयं ॅय॒ज्ञियं॑ ॅय॒ज्ञिय॑ म॒यम् निष्क्री॑तो॒ निष्क्री॑तो॒ ऽयं ॅय॒ज्ञिय᳚म् । \newline
10. निष्क्री॑त॒ इति॒ निः - क्री॒तः॒ । \newline
11. अ॒यं ॅय॒ज्ञियं॑ ॅय॒ज्ञिय॑ म॒य म॒यं ॅय॒ज्ञिय॑म् भा॒गम् भा॒गं ॅय॒ज्ञिय॑ म॒य म॒यं ॅय॒ज्ञिय॑म् भा॒गम् । \newline
12. य॒ज्ञिय॑म् भा॒गम् भा॒गं ॅय॒ज्ञियं॑ ॅय॒ज्ञिय॑म् भा॒ग मे᳚त्वेतु भा॒गं ॅय॒ज्ञियं॑ ॅय॒ज्ञिय॑म् भा॒ग मे॑तु । \newline
13. भा॒ग मे᳚त्वेतु भा॒गम् भा॒ग मे॑तु रा॒यो रा॒य ए॑तु भा॒गम् भा॒ग मे॑तु रा॒यः । \newline
14. ए॒तु॒ रा॒यो रा॒य ए᳚त्वेतु रा॒य स्पोषाः॒ पोषा॑ रा॒य ए᳚त्वेतु रा॒य स्पोषाः᳚ । \newline
15. रा॒य स्पोषाः॒ पोषा॑ रा॒यो रा॒य स्पोषा॒ यज॑मानस्य॒ यज॑मानस्य॒ पोषा॑ रा॒यो रा॒य स्पोषा॒ यज॑मानस्य । \newline
16. पोषा॒ यज॑मानस्य॒ यज॑मानस्य॒ पोषाः॒ पोषा॒ यज॑मानस्य सन्तु सन्तु॒ यज॑मानस्य॒ पोषाः॒ पोषा॒ यज॑मानस्य सन्तु । \newline
17. यज॑मानस्य सन्तु सन्तु॒ यज॑मानस्य॒ यज॑मानस्य सन्तु । \newline
18. स॒न्त्विति॑ सन्तु । \newline
19. ये ब॒द्ध्यमा॑नम् ब॒द्ध्यमा॑नं॒ ॅये ये ब॒द्ध्यमा॑न॒ मन्वनु॑ ब॒द्ध्यमा॑नं॒ ॅये ये ब॒द्ध्यमा॑न॒ मनु॑ । \newline
20. ब॒द्ध्यमा॑न॒ मन्वनु॑ ब॒द्ध्यमा॑नम् ब॒द्ध्यमा॑न॒ मनु॑ ब॒द्ध्यमा॑ना ब॒द्ध्यमा॑ना॒ अनु॑ ब॒द्ध्यमा॑नम् ब॒द्ध्यमा॑न॒ मनु॑ ब॒द्ध्यमा॑नाः । \newline
21. अनु॑ ब॒द्ध्यमा॑ना ब॒द्ध्यमा॑ना॒ अन्वनु॑ ब॒द्ध्यमा॑ना अ॒भ्यैक्ष॑न्ता॒ भ्यैक्ष॑न्त ब॒द्ध्यमा॑ना॒ अन्वनु॑ ब॒द्ध्यमा॑ना अ॒भ्यैक्ष॑न्त । \newline
22. ब॒द्ध्यमा॑ना अ॒भ्यैक्ष॑न्ता॒ भ्यैक्ष॑न्त ब॒द्ध्यमा॑ना ब॒द्ध्यमा॑ना अ॒भ्यैक्ष॑न्त॒ मन॑सा॒ मन॑सा॒ ऽभ्यैक्ष॑न्त ब॒द्ध्यमा॑ना ब॒द्ध्यमा॑ना अ॒भ्यैक्ष॑न्त॒ मन॑सा । \newline
23. अ॒भ्यैक्ष॑न्त॒ मन॑सा॒ मन॑सा॒ ऽभ्यैक्ष॑न्ता॒ भ्यैक्ष॑न्त॒ मन॑सा॒ चक्षु॑षा॒ चक्षु॑षा॒ मन॑सा॒ ऽभ्यैक्ष॑न्ता॒ भ्यैक्ष॑न्त॒ मन॑सा॒ चक्षु॑षा । \newline
24. अ॒भ्यैक्ष॒न्तेत्य॑भि - ऐक्ष॑न्त । \newline
25. मन॑सा॒ चक्षु॑षा॒ चक्षु॑षा॒ मन॑सा॒ मन॑सा॒ चक्षु॑षा च च॒ चक्षु॑षा॒ मन॑सा॒ मन॑सा॒ चक्षु॑षा च । \newline
26. चक्षु॑षा च च॒ चक्षु॑षा॒ चक्षु॑षा च । \newline
27. चेति॑ च । \newline
28. अ॒ग्नि स्ताꣳ स्ताꣳ अ॒ग्नि र॒ग्नि स्ताꣳ अग्रे ऽग्रे॒ ताꣳ अ॒ग्नि र॒ग्नि स्ताꣳ अग्रे᳚ । \newline
29. ताꣳ अग्रे ऽग्रे॒ ताꣳ स्ताꣳ अग्रे॒ प्र प्राग्रे॒ ताꣳ स्ताꣳ अग्रे॒ प्र । \newline
30. अग्रे॒ प्र प्राग्रे ऽग्रे॒ प्र मु॑मोक्तु मुमोक्तु॒ प्राग्रे ऽग्रे॒ प्र मु॑मोक्तु । \newline
31. प्र मु॑मोक्तु मुमोक्तु॒ प्र प्र मु॑मोक्तु दे॒वो दे॒वो मु॑मोक्तु॒ प्र प्र मु॑मोक्तु दे॒वः । \newline
32. मु॒मो॒क्तु॒ दे॒वो दे॒वो मु॑मोक्तु मुमोक्तु दे॒वः प्र॒जाप॑तिः प्र॒जाप॑तिर् दे॒वो मु॑मोक्तु मुमोक्तु दे॒वः प्र॒जाप॑तिः । \newline
33. दे॒वः प्र॒जाप॑तिः प्र॒जाप॑तिर् दे॒वो दे॒वः प्र॒जाप॑तिः प्र॒जया᳚ प्र॒जया᳚ प्र॒जाप॑तिर् दे॒वो दे॒वः प्र॒जाप॑तिः प्र॒जया᳚ । \newline
34. प्र॒जाप॑तिः प्र॒जया᳚ प्र॒जया᳚ प्र॒जाप॑तिः प्र॒जाप॑तिः प्र॒जया॑ संॅविदा॒नः सं॑ॅविदा॒नः प्र॒जया᳚ प्र॒जाप॑तिः प्र॒जाप॑तिः प्र॒जया॑ संॅविदा॒नः । \newline
35. प्र॒जाप॑ति॒रिति॑ प्र॒जा - प॒तिः॒ । \newline
36. प्र॒जया॑ संॅविदा॒नः सं॑ॅविदा॒नः प्र॒जया᳚ प्र॒जया॑ संॅविदा॒नः । \newline
37. प्र॒जयेति॑ प्र - जया᳚ । \newline
38. सं॒ॅवि॒दा॒न इति॑ सं - वि॒दा॒नः । \newline
39. य आ॑र॒ण्या आ॑र॒ण्या ये य आ॑र॒ण्याः प॒शवः॑ प॒शव॑ आर॒ण्या ये य आ॑र॒ण्याः प॒शवः॑ । \newline
40. आ॒र॒ण्याः प॒शवः॑ प॒शव॑ आर॒ण्या आ॑र॒ण्याः प॒शवो॑ वि॒श्वरू॑पा वि॒श्वरू॑पाः प॒शव॑ आर॒ण्या आ॑र॒ण्याः प॒शवो॑ वि॒श्वरू॑पाः । \newline
41. प॒शवो॑ वि॒श्वरू॑पा वि॒श्वरू॑पाः प॒शवः॑ प॒शवो॑ वि॒श्वरू॑पा॒ विरू॑पा॒ विरू॑पा वि॒श्वरू॑पाः प॒शवः॑ प॒शवो॑ वि॒श्वरू॑पा॒ विरू॑पाः । \newline
42. वि॒श्वरू॑पा॒ विरू॑पा॒ विरू॑पा वि॒श्वरू॑पा वि॒श्वरू॑पा॒ विरू॑पाः॒ सन्तः॒ सन्तो॒ विरू॑पा वि॒श्वरू॑पा वि॒श्वरू॑पा॒ विरू॑पाः॒ सन्तः॑ । \newline
43. वि॒श्वरू॑पा॒ इति॑ वि॒श्व - रू॒पाः॒ । \newline
44. विरू॑पाः॒ सन्तः॒ सन्तो॒ विरू॑पा॒ विरू॑पाः॒ सन्तो॑ बहु॒धा ब॑हु॒धा सन्तो॒ विरू॑पा॒ विरू॑पाः॒ सन्तो॑ बहु॒धा । \newline
45. विरू॑पा॒ इति॒ वि - रू॒पाः॒ । \newline
46. सन्तो॑ बहु॒धा ब॑हु॒धा सन्तः॒ सन्तो॑ बहु॒धैक॑रूपा॒ एक॑रूपा बहु॒धा सन्तः॒ सन्तो॑ बहु॒धैक॑रूपाः । \newline
47. ब॒हु॒धैक॑रूपा॒ एक॑रूपा बहु॒धा ब॑हु॒धैक॑रूपाः । \newline
48. ब॒हु॒धेति॑ बहु - धा । \newline
49. एक॑रूपा॒ इत्येक॑ - रू॒पाः॒ । \newline
50. वा॒यु स्ताꣳ स्तान्. वा॒युर् वा॒यु स्ताꣳ अग्रे ऽग्रे॒ तान्. वा॒युर् वा॒यु स्ताꣳ अग्रे᳚ । \newline
51. ताꣳ अग्रे ऽग्रे॒ ताꣳ स्ताꣳ अग्रे॒ प्र प्राग्रे॒ ताꣳ स्ताꣳ अग्रे॒ प्र । \newline
52. अग्रे॒ प्र प्राग्रे ऽग्रे॒ प्र मु॑मोक्तु मुमोक्तु॒ प्राग्रे ऽग्रे॒ प्र मु॑मोक्तु । \newline
53. प्र मु॑मोक्तु मुमोक्तु॒ प्र प्र मु॑मोक्तु दे॒वो दे॒वो मु॑मोक्तु॒ प्र प्र मु॑मोक्तु दे॒वः । \newline
54. मु॒मो॒क्तु॒ दे॒वो दे॒वो मु॑मोक्तु मुमोक्तु दे॒वः प्र॒जाप॑तिः प्र॒जाप॑तिर् दे॒वो मु॑मोक्तु मुमोक्तु दे॒वः प्र॒जाप॑तिः । \newline
55. दे॒वः प्र॒जाप॑तिः प्र॒जाप॑तिर् दे॒वो दे॒वः प्र॒जाप॑तिः प्र॒जया᳚ प्र॒जया᳚ प्र॒जाप॑तिर् दे॒वो दे॒वः प्र॒जाप॑तिः प्र॒जया᳚ । \newline
56. प्र॒जाप॑तिः प्र॒जया᳚ प्र॒जया᳚ प्र॒जाप॑तिः प्र॒जाप॑तिः प्र॒जया॑ संॅविदा॒नः सं॑ॅविदा॒नः प्र॒जया᳚ प्र॒जाप॑तिः प्र॒जाप॑तिः प्र॒जया॑ संॅविदा॒नः । \newline
57. प्र॒जाप॑ति॒रिति॑ प्र॒जा - प॒तिः॒ । \newline
58. प्र॒जया॑ संॅविदा॒नः सं॑ॅविदा॒नः प्र॒जया᳚ प्र॒जया॑ संॅविदा॒नः । \newline
59. प्र॒जयेति॑ प्र - जया᳚ । \newline
60. सं॒ॅवि॒दा॒न इति॑ सं - वि॒दा॒नः । \newline
61. प्र॒मु॒ञ्चमा॑ना॒ भुव॑नस्य॒ भुव॑नस्य प्रमु॒ञ्चमा॑नाः प्रमु॒ञ्चमा॑ना॒ भुव॑नस्य॒ रेतो॒ रेतो॒ भुव॑नस्य प्रमु॒ञ्चमा॑नाः प्रमु॒ञ्चमा॑ना॒ भुव॑नस्य॒ रेतः॑ । \newline
62. प्र॒मु॒ञ्चमा॑ना॒ इति॑ प्र - मु॒ञ्चमा॑नाः । \newline
\pagebreak
\markright{ TS 3.1.4.3  \hfill https://www.vedavms.in \hfill}

\section{ TS 3.1.4.3 }

\textbf{TS 3.1.4.3 } \newline
\textbf{Samhita Paata} \newline

भुव॑नस्य॒ रेतो॑ गा॒तुं ध॑त्त॒ यज॑मानाय देवाः । उ॒पाकृ॑तꣳ शशमा॒नं ॅयदस्था᳚ज्जी॒वं दे॒वाना॒मप्ये॑तु॒ पाथः॑ ॥ नाना᳚ प्रा॒णो यज॑मानस्य प॒शुना॑ य॒ज्ञो दे॒वेभिः॑ स॒ह दे॑व॒यानः॑ । जी॒वं दे॒वाना॒मप्ये॑तु॒ पाथः॑ स॒त्याः स॑न्तु॒ यज॑मानस्य॒ कामाः᳚ ॥ यत् प॒शुर्मा॒युमकृ॒तोरो॑ वा प॒द्भिरा॑ह॒ते । अ॒ग्निर्मा॒ तस्मा॒देन॑सो॒ विश्वा᳚न् मुञ्च॒त्वꣳह॑सः ॥ शमि॑तार उ॒पेत॑न य॒ज्ञ्ं - [  ] \newline

\textbf{Pada Paata} \newline

भुव॑नस्य । रेतः॑ । गा॒तुम् । ध॒त्त॒ । यज॑मानाय । दे॒वाः॒ ॥ उ॒पाकृ॑त॒मित्यु॑प-आकृ॑तम् । श॒श॒मा॒नम् । यत् । अस्था᳚त् । जी॒वम् । दे॒वाना᳚म् । अपीति॑ । ए॒तु॒ । पाथः॑ ॥ नाना᳚ । प्रा॒ण इति॑ प्र - अ॒नः । यज॑मानस्य । प॒शुना᳚ । य॒ज्ञ्ः । दे॒वेभिः॑ । स॒ह । दे॒व॒यान॒ इति॑ देव - यानः॑ ॥ जी॒वम् । दे॒वाना᳚म् । अपीति॑ । ए॒तु॒ । पाथः॑ । स॒त्याः । स॒न्तु॒ । यज॑मानस्य । कामाः᳚ ॥ यत् । प॒शुः । मा॒युम् । अकृ॑त । उरः॑ । वा॒ । प॒द्भिरिति॑ पत् - भिः । आ॒ह॒त इत्या᳚ - ह॒ते ॥ अ॒ग्निः । मा॒ । तस्मा᳚त् । एन॑सः । विश्वा᳚त् । मु॒ञ्च॒तु॒ । अꣳह॑सः ॥ शमि॑तारः । उ॒पेत॒नेत्यु॑प - एत॑न । य॒ज्ञ्म् ।  \newline


\textbf{Krama Paata} \newline

भुव॑नस्य॒ रेतः॑ । रेतो॑ गा॒तुम् । गा॒तुम् ध॑त्त । ध॒त्त॒ यज॑मानाय । यज॑मानाय देवाः । दे॒वा॒ इति॑ देवाः ॥ उ॒पाकृ॑तꣳ शशमा॒नम् । उ॒पाकृ॑त॒मित्यु॑प - आकृ॑तम् । श॒श॒मा॒नं ॅयत् । यदस्था᳚त् । अस्था᳚ज् जी॒वम् । जी॒वम् दे॒वाना᳚म् । दे॒वाना॒मपि॑ । अप्ये॑तु । ए॒तु॒ पाथः॑ । पाथ॒ इति॒ पाथः॑ ॥ नाना᳚ प्रा॒णः । प्रा॒णो यज॑मानस्य । प्रा॒ण इति॑ प्र - अ॒नः । यज॑मानस्य प॒शुना᳚ । प॒शुना॑ य॒ज्ञ्ः । य॒ज्ञो दे॒वेभिः॑ । दे॒वेभिः॑ स॒ह । स॒ह दे॑व॒यानः॑ । दे॒व॒यान॒ इति॑ देव - यानः॑ ॥ जी॒वम् दे॒वाना᳚म् । दे॒वाना॒मपि॑ । अप्ये॑तु । ए॒तु॒ पाथः॑ । पाथः॑ स॒त्याः । स॒त्याः स॑न्तु । स॒न्तु॒ यज॑मानस्य । यज॑मानस्य॒ कामाः᳚ । कामा॒ इति॒ कामाः᳚ ॥ यत् प॒शुः । प॒शुर् मा॒युम् । मा॒युमकृ॑त । अकृ॒तोरः॑ । उरो॑ वा । वा॒ प॒द्भिः । प॒द्भिरा॑ह॒ते । प॒द्भिरिति॑ पत् - भिः । आ॒ह॒त इत्या᳚ - ह॒ते ॥ अ॒ग्निर् मा᳚ । मा॒ तस्मा᳚त् । तस्मा॒देन॑सः । एन॑सो॒ विश्वा᳚त् । विश्वा᳚न् मुञ्चतु । मु॒ञ्च॒त्वꣳह॑सः । अꣳह॑स॒ इत्यꣳह॑सः ॥ शमि॑तार उ॒पेत॑न । उ॒पेत॑न य॒ज्ञ्म् । उ॒पेत॒नेत्यु॑प - एत॑न । य॒ज्ञ्म् दे॒वेभिः॑ \newline

\textbf{Jatai Paata} \newline

1. भुव॑नस्य॒ रेतो॒ रेतो॒ भुव॑नस्य॒ भुव॑नस्य॒ रेतः॑ । \newline
2. रेतो॑ गा॒तुम् गा॒तुꣳ रेतो॒ रेतो॑ गा॒तुम् । \newline
3. गा॒तुम् ध॑त्त धत्त गा॒तुम् गा॒तुम् ध॑त्त । \newline
4. ध॒त्त॒ यज॑मानाय॒ यज॑मानाय धत्त धत्त॒ यज॑मानाय । \newline
5. यज॑मानाय देवा देवा॒ यज॑मानाय॒ यज॑मानाय देवाः । \newline
6. दे॒वा॒ इति॑ देवाः । \newline
7. उ॒पाकृ॑तꣳ शशमा॒नꣳ श॑शमा॒न मु॒पाकृ॑त मु॒पाकृ॑तꣳ शशमा॒नम् । \newline
8. उ॒पाकृ॑त॒मित्यु॑प - आकृ॑तम् । \newline
9. श॒श॒मा॒नं ॅयद् यच् छ॑शमा॒नꣳ श॑शमा॒नं ॅयत् । \newline
10. यदस्था॒ दस्था॒द् यद् यदस्था᳚त् । \newline
11. अस्था᳚ज् जी॒वम् जी॒व मस्था॒ दस्था᳚ज् जी॒वम् । \newline
12. जी॒वम् दे॒वाना᳚म् दे॒वाना᳚म् जी॒वम् जी॒वम् दे॒वाना᳚म् । \newline
13. दे॒वाना॒ मप्यपि॑ दे॒वाना᳚म् दे॒वाना॒ मपि॑ । \newline
14. अप्ये᳚ त्वे॒ त्वप्य प्ये॑तु । \newline
15. ए॒तु॒ पाथः॒ पाथ॑ एत्वेतु॒ पाथः॑ । \newline
16. पाथ॒ इति॒ पाथः॑ । \newline
17. नाना᳚ प्रा॒णः प्रा॒णो नाना॒ नाना᳚ प्रा॒णः । \newline
18. प्रा॒णो यज॑मानस्य॒ यज॑मानस्य प्रा॒णः प्रा॒णो यज॑मानस्य । \newline
19. प्रा॒ण इति॑ प्र - अ॒नः । \newline
20. यज॑मानस्य प॒शुना॑ प॒शुना॒ यज॑मानस्य॒ यज॑मानस्य प॒शुना᳚ । \newline
21. प॒शुना॑ य॒ज्ञो य॒ज्ञ्ः प॒शुना॑ प॒शुना॑ य॒ज्ञ्ः । \newline
22. य॒ज्ञो दे॒वेभि॑र् दे॒वेभि॑र् य॒ज्ञो य॒ज्ञो दे॒वेभिः॑ । \newline
23. दे॒वेभिः॑ स॒ह स॒ह दे॒वेभि॑र् दे॒वेभिः॑ स॒ह । \newline
24. स॒ह दे॑व॒यानो॑ देव॒यानः॑ स॒ह स॒ह दे॑व॒यानः॑ । \newline
25. दे॒व॒यान॒ इति॑ देव - यानः॑ । \newline
26. जी॒वम् दे॒वाना᳚म् दे॒वाना᳚म् जी॒वम् जी॒वम् दे॒वाना᳚म् । \newline
27. दे॒वाना॒ मप्यपि॑ दे॒वाना᳚म् दे॒वाना॒ मपि॑ । \newline
28. अप्ये᳚ त्वे॒त्वप्य प्ये॑तु । \newline
29. ए॒तु॒ पाथः॒ पाथ॑ एत्वेतु॒ पाथः॑ । \newline
30. पाथः॑ स॒त्याः स॒त्याः पाथः॒ पाथः॑ स॒त्याः । \newline
31. स॒त्याः स॑न्तु सन्तु स॒त्याः स॒त्याः स॑न्तु । \newline
32. स॒न्तु॒ यज॑मानस्य॒ यज॑मानस्य सन्तु सन्तु॒ यज॑मानस्य । \newline
33. यज॑मानस्य॒ कामाः॒ कामा॒ यज॑मानस्य॒ यज॑मानस्य॒ कामाः᳚ । \newline
34. कामा॒ इति॒ कामाः᳚ । \newline
35. यत् प॒शुः प॒शुर् यद् यत् प॒शुः । \newline
36. प॒शुर् मा॒युम् मा॒युम् प॒शुः प॒शुर् मा॒युम् । \newline
37. मा॒यु मकृ॒ता कृ॑त मा॒युम् मा॒यु मकृ॑त । \newline
38. अकृ॒तोर॒ उरो ऽकृ॒ता कृ॒तोरः॑ । \newline
39. उरो॑ वा॒ वोर॒ उरो॑ वा । \newline
40. वा॒ प॒द्भिः प॒द्भिर् वा॑ वा प॒द्भिः । \newline
41. प॒द्भि रा॑ह॒त आ॑ह॒ते प॒द्भिः प॒द्भि रा॑ह॒ते । \newline
42. प॒द्भिरिति॑ पत् - भिः । \newline
43. आ॒ह॒त इत्या᳚ - ह॒ते । \newline
44. अ॒ग्निर् मा॑ मा॒ ऽग्नि र॒ग्निर् मा᳚ । \newline
45. मा॒ तस्मा॒त् तस्मा᳚न् मा मा॒ तस्मा᳚त् । \newline
46. तस्मा॒ देन॑स॒ एन॑स॒ स्तस्मा॒त् तस्मा॒ देन॑सः । \newline
47. एन॑सो॒ विश्वा॒द् विश्वा॒ देन॑स॒ एन॑सो॒ विश्वा᳚त् । \newline
48. विश्वा᳚न् मुञ्चतु मुञ्चतु॒ विश्वा॒द् विश्वा᳚न् मुञ्चतु । \newline
49. मु॒ञ्च॒ त्वꣳह॒सो ऽꣳह॑सो मुञ्चतु मुञ्च॒ त्वꣳह॑सः । \newline
50. अꣳह॑स॒ इत्यꣳह॑सः । \newline
51. शमि॑तार उ॒पेत॑ नो॒पेत॑न॒ शमि॑तारः॒ शमि॑तार उ॒पेत॑न । \newline
52. उ॒पेत॑न य॒ज्ञ्ं ॅय॒ज्ञ् मु॒पेत॑ नो॒पेत॑न य॒ज्ञ्म् । \newline
53. उ॒पेत॒नेत्यु॑प - एत॑न । \newline
54. य॒ज्ञ्म् दे॒वेभि॑र् दे॒वेभि॑र् य॒ज्ञ्ं ॅय॒ज्ञ्म् दे॒वेभिः॑ । \newline

\textbf{Ghana Paata } \newline

1. भुव॑नस्य॒ रेतो॒ रेतो॒ भुव॑नस्य॒ भुव॑नस्य॒ रेतो॑ गा॒तुम् गा॒तुꣳ रेतो॒ भुव॑नस्य॒ भुव॑नस्य॒ रेतो॑ गा॒तुम् । \newline
2. रेतो॑ गा॒तुम् गा॒तुꣳ रेतो॒ रेतो॑ गा॒तुम् ध॑त्त धत्त गा॒तुꣳ रेतो॒ रेतो॑ गा॒तुम् ध॑त्त । \newline
3. गा॒तुम् ध॑त्त धत्त गा॒तुम् गा॒तुम् ध॑त्त॒ यज॑मानाय॒ यज॑मानाय धत्त गा॒तुम् गा॒तुम् ध॑त्त॒ यज॑मानाय । \newline
4. ध॒त्त॒ यज॑मानाय॒ यज॑मानाय धत्त धत्त॒ यज॑मानाय देवा देवा॒ यज॑मानाय धत्त धत्त॒ यज॑मानाय देवाः । \newline
5. यज॑मानाय देवा देवा॒ यज॑मानाय॒ यज॑मानाय देवाः । \newline
6. दे॒वा॒ इति॑ देवाः । \newline
7. उ॒पाकृ॑तꣳ शशमा॒नꣳ श॑शमा॒न मु॒पाकृ॑त मु॒पाकृ॑तꣳ शशमा॒नं ॅयद् यच्छ॑शमा॒न मु॒पाकृ॑त मु॒पाकृ॑तꣳ शशमा॒नं ॅयत् । \newline
8. उ॒पाकृ॑त॒मित्यु॑प - आकृ॑तम् । \newline
9. श॒श॒मा॒नं ॅयद् यच्छ॑शमा॒नꣳ श॑शमा॒नं ॅयदस्था॒ दस्था॒द् यच्छ॑शमा॒नꣳ श॑शमा॒नं ॅयदस्था᳚त् । \newline
10. यदस्था॒ दस्था॒द् यद् यदस्था᳚ज् जी॒वम् जी॒व मस्था॒द् यद् यदस्था᳚ज् जी॒वम् । \newline
11. अस्था᳚ज् जी॒वम् जी॒व मस्था॒ दस्था᳚ज् जी॒वम् दे॒वाना᳚म् दे॒वाना᳚म् जी॒व मस्था॒ दस्था᳚ज् जी॒वम् दे॒वाना᳚म् । \newline
12. जी॒वम् दे॒वाना᳚म् दे॒वाना᳚म् जी॒वम् जी॒वम् दे॒वाना॒ मप्यपि॑ दे॒वाना᳚म् जी॒वम् जी॒वम् दे॒वाना॒ मपि॑ । \newline
13. दे॒वाना॒ मप्यपि॑ दे॒वाना᳚म् दे॒वाना॒ मप्ये᳚त्वे॒त्वपि॑ दे॒वाना᳚म् दे॒वाना॒ मप्ये॑तु । \newline
14. अप्ये᳚ त्वे॒ त्व प्यप्ये॑तु॒ पाथः॒ पाथ॑ ए॒त्व प्यप्ये॑तु॒ पाथः॑ । \newline
15. ए॒तु॒ पाथः॒ पाथ॑ एत्वेतु॒ पाथः॑ । \newline
16. पाथ॒ इति॒ पाथः॑ । \newline
17. नाना᳚ प्रा॒णः प्रा॒णो नाना॒ नाना᳚ प्रा॒णो यज॑मानस्य॒ यज॑मानस्य प्रा॒णो नाना॒ नाना᳚ प्रा॒णो यज॑मानस्य । \newline
18. प्रा॒णो यज॑मानस्य॒ यज॑मानस्य प्रा॒णः प्रा॒णो यज॑मानस्य प॒शुना॑ प॒शुना॒ यज॑मानस्य प्रा॒णः प्रा॒णो यज॑मानस्य प॒शुना᳚ । \newline
19. प्रा॒ण इति॑ प्र - अ॒नः । \newline
20. यज॑मानस्य प॒शुना॑ प॒शुना॒ यज॑मानस्य॒ यज॑मानस्य प॒शुना॑ य॒ज्ञो य॒ज्ञ्ः प॒शुना॒ यज॑मानस्य॒ यज॑मानस्य प॒शुना॑ य॒ज्ञ्ः । \newline
21. प॒शुना॑ य॒ज्ञो य॒ज्ञ्ः प॒शुना॑ प॒शुना॑ य॒ज्ञो दे॒वेभि॑र् दे॒वेभि॑र् य॒ज्ञ्ः प॒शुना॑ प॒शुना॑ य॒ज्ञो दे॒वेभिः॑ । \newline
22. य॒ज्ञो दे॒वेभि॑र् दे॒वेभि॑र् य॒ज्ञो य॒ज्ञो दे॒वेभिः॑ स॒ह स॒ह दे॒वेभि॑र् य॒ज्ञो य॒ज्ञो दे॒वेभिः॑ स॒ह । \newline
23. दे॒वेभिः॑ स॒ह स॒ह दे॒वेभि॑र् दे॒वेभिः॑ स॒ह दे॑व॒यानो॑ देव॒यानः॑ स॒ह दे॒वेभि॑र् दे॒वेभिः॑ स॒ह दे॑व॒यानः॑ । \newline
24. स॒ह दे॑व॒यानो॑ देव॒यानः॑ स॒ह स॒ह दे॑व॒यानः॑ । \newline
25. दे॒व॒यान॒ इति॑ देव - यानः॑ । \newline
26. जी॒वम् दे॒वाना᳚म् दे॒वाना᳚म् जी॒वम् जी॒वम् दे॒वाना॒ मप्यपि॑ दे॒वाना᳚म् जी॒वम् जी॒वम् दे॒वाना॒ मपि॑ । \newline
27. दे॒वाना॒ मप्यपि॑ दे॒वाना᳚म् दे॒वाना॒ मप्ये᳚ त्वे॒ त्वपि॑ दे॒वाना᳚म् दे॒वाना॒ मप्ये॑तु । \newline
28. अप्ये᳚ त्वे॒त्व प्यप्ये॑तु॒ पाथः॒ पाथ॑ ए॒त्व प्यप्ये॑तु॒ पाथः॑ । \newline
29. ए॒तु॒ पाथः॒ पाथ॑ एत्वेतु॒ पाथः॑ स॒त्याः स॒त्याः पाथ॑ एत्वेतु॒ पाथः॑ स॒त्याः । \newline
30. पाथः॑ स॒त्याः स॒त्याः पाथः॒ पाथः॑ स॒त्याः स॑न्तु सन्तु स॒त्याः पाथः॒ पाथः॑ स॒त्याः स॑न्तु । \newline
31. स॒त्याः स॑न्तु सन्तु स॒त्याः स॒त्याः स॑न्तु॒ यज॑मानस्य॒ यज॑मानस्य सन्तु स॒त्याः स॒त्याः स॑न्तु॒ यज॑मानस्य । \newline
32. स॒न्तु॒ यज॑मानस्य॒ यज॑मानस्य सन्तु सन्तु॒ यज॑मानस्य॒ कामाः॒ कामा॒ यज॑मानस्य सन्तु सन्तु॒ यज॑मानस्य॒ कामाः᳚ । \newline
33. यज॑मानस्य॒ कामाः॒ कामा॒ यज॑मानस्य॒ यज॑मानस्य॒ कामाः᳚ । \newline
34. कामा॒ इति॒ कामाः᳚ । \newline
35. यत् प॒शुः प॒शुर् यद् यत् प॒शुर् मा॒युम् मा॒युम् प॒शुर् यद् यत् प॒शुर् मा॒युम् । \newline
36. प॒शुर् मा॒युम् मा॒युम् प॒शुः प॒शुर् मा॒यु मकृ॒ता कृ॑त मा॒युम् प॒शुः प॒शुर् मा॒यु मकृ॑त । \newline
37. मा॒यु मकृ॒ता कृ॑त मा॒युम् मा॒यु मकृ॒तोर॒ उरो ऽकृ॑त मा॒युम् मा॒यु मकृ॒तोरः॑ । \newline
38. अकृ॒तोर॒ उरो ऽकृ॒ता कृ॒तोरो॑ वा॒ वोरो ऽकृ॒ता कृ॒तोरो॑ वा । \newline
39. उरो॑ वा॒ वोर॒ उरो॑ वा प॒द्भिः प॒द्भिर् वोर॒ उरो॑ वा प॒द्भिः । \newline
40. वा॒ प॒द्भिः प॒द्भिर् वा॑ वा प॒द्भि रा॑ह॒त आ॑ह॒ते प॒द्भिर् वा॑ वा प॒द्भि रा॑ह॒ते । \newline
41. प॒द्भि रा॑ह॒त आ॑ह॒ते प॒द्भिः प॒द्भि रा॑ह॒ते । \newline
42. प॒द्भिरिति॑ पत् - भिः । \newline
43. आ॒ह॒त इत्या᳚ - ह॒ते । \newline
44. अ॒ग्निर् मा॑ मा॒ ऽग्नि र॒ग्निर् मा॒ तस्मा॒त् तस्मा᳚न् मा॒ ऽग्नि र॒ग्निर् मा॒ तस्मा᳚त् । \newline
45. मा॒ तस्मा॒त् तस्मा᳚न् मा मा॒ तस्मा॒ देन॑स॒ एन॑स॒ स्तस्मा᳚न् मा मा॒ तस्मा॒ देन॑सः । \newline
46. तस्मा॒ देन॑स॒ एन॑स॒ स्तस्मा॒त् तस्मा॒देन॑सो॒ विश्वा॒द् विश्वा॒ देन॑स॒ स्तस्मा॒त् तस्मा॒ देन॑सो॒ विश्वा᳚त् । \newline
47. एन॑सो॒ विश्वा॒द् विश्वा॒ देन॑स॒ एन॑सो॒ विश्वा᳚न् मुञ्चतु मुञ्चतु॒ विश्वा॒ देन॑स॒ एन॑सो॒ विश्वा᳚न् मुञ्चतु । \newline
48. विश्वा᳚न् मुञ्चतु मुञ्चतु॒ विश्वा॒द् विश्वा᳚न् मुञ्च॒ त्वꣳह॒सो ऽꣳह॑सो मुञ्चतु॒ विश्वा॒द् विश्वा᳚न् मुञ्च॒ त्वꣳह॑सः । \newline
49. मु॒ञ्च॒त्वꣳह॒सो ऽꣳह॑सो मुञ्चतु मुञ्च॒त्वꣳह॑सः । \newline
50. अꣳह॑स॒ इत्यꣳह॑सः । \newline
51. शमि॑तार उ॒पेत॑ नो॒पेत॑न॒ शमि॑तारः॒ शमि॑तार उ॒पेत॑न य॒ज्ञ्ं ॅय॒ज्ञ् मु॒पेत॑न॒ शमि॑तारः॒ शमि॑तार उ॒पेत॑न य॒ज्ञ्म् । \newline
52. उ॒पेत॑न य॒ज्ञ्ं ॅय॒ज्ञ् मु॒पेत॑ नो॒पेत॑न य॒ज्ञ्म् दे॒वेभि॑र् दे॒वेभि॑र् य॒ज्ञ् मु॒पेत॑ नो॒पेत॑न य॒ज्ञ्म् दे॒वेभिः॑ । \newline
53. उ॒पेत॒नेत्यु॑प - एत॑न । \newline
54. य॒ज्ञ्म् दे॒वेभि॑र् दे॒वेभि॑र् य॒ज्ञ्ं ॅय॒ज्ञ्म् दे॒वेभि॑ रिन्वि॒त मि॑न्वि॒तम् दे॒वेभि॑र् य॒ज्ञ्ं ॅय॒ज्ञ्म् दे॒वेभि॑ रिन्वि॒तम् । \newline
\pagebreak
\markright{ TS 3.1.4.4  \hfill https://www.vedavms.in \hfill}

\section{ TS 3.1.4.4 }

\textbf{TS 3.1.4.4 } \newline
\textbf{Samhita Paata} \newline

दे॒वेभि॑रिन्वि॒तं । पाशा᳚त् प॒शुं प्रमु॑ञ्चत ब॒न्धाद्य॒ज्ञ्प॑तिं॒ परि॑ ॥ अदि॑तिः॒ पाशं॒ प्रमु॑मोक्त्वे॒तं नमः॑ प॒शुभ्यः॑ पशु॒पत॑ये करोमि ॥ अ॒रा॒ती॒यन्त॒-मध॑रं कृणोमि॒ यं द्वि॒ष्मस्तस्मि॒न् प्रति॑ मुञ्चामि॒ पाशं᳚ ॥ त्वामु॒ ते द॑धिरे हव्य॒वाहꣳ॑ शृतंक॒र्तार॑मु॒त य॒ज्ञियं॑ च । अग्ने॒ सद॑क्षः॒ सत॑नु॒र्॒.हि भू॒त्वाऽथ॑ ह॒व्या जा॑तवेदो जुषस्व ॥ जात॑वेदो व॒पया॑ गच्छ दे॒वान्त्वꣳ ( ) हि होता᳚ प्रथ॒मो ब॒भूथ॑ । घृ॒तेन॒ त्वं त॒नुवो॑ वर्द्धयस्व॒ स्वाहा॑कृतꣳ ह॒विर॑दन्तु दे॒वाः ॥ स्वाहा॑ दे॒वेभ्यो॑ दे॒वेभ्यः॒ स्वाहा᳚ ॥ \newline

\textbf{Pada Paata} \newline

दे॒वेभिः॑ । इ॒न्वि॒तम् ॥ पाशा᳚त् । प॒शुम् । प्रेति॑ । मु॒ञ्च॒त॒ । ब॒न्धात् । य॒ज्ञ्प॑ति॒मिति॑ य॒ज्ञ् - प॒ति॒म् । परि॑ ॥ अदि॑तिः । पाश᳚म् । प्रेति॑ । मु॒मो॒क्तु॒ । ए॒तम् । नमः॑ । प॒शुभ्य॒ इति॑ प॒शु - भ्यः॒ । प॒शु॒पत॑य॒ इति॑ पशु - पत॑ये । क॒रो॒मि॒ ॥ अ॒रा॒ती॒यन्त᳚म् । अध॑रम् । कृ॒णो॒मि॒ । यम् । द्वि॒ष्मः । तस्मिन्न्॑ । प्रतीति॑ । मु॒ञ्चा॒मि॒ । पाश᳚म् ॥ त्वाम् । उ॒ । ते । द॒धि॒रे॒ । ह॒व्य॒वाह॒मिति॑ हव्य - वाह᳚म् । शृ॒त॒कं॒र्तार॒मिति॑ शृतं - क॒र्तार᳚म् । उ॒त । य॒ज्ञिय᳚म् । च॒ ॥ अग्ने᳚ । सद॑क्ष॒ इति॒ स - द॒क्षः॒ । सत॑नु॒रिति॒ स-त॒नुः॒ । हि । भू॒त्वा । अथ॑ । ह॒व्या । जा॒त॒वे॒द॒ इति॑ जात - वे॒दः॒ । जु॒ष॒स्व॒ ॥ जात॑वेद॒ इति॒ जात॑ - वे॒दः॒ । व॒पया᳚ । ग॒च्छ॒ । दे॒वान् । त्वम् ( ) । हि । होता᳚ । प्र॒थ॒मः । ब॒भूथ॑ ॥ घृ॒तेन॑ । त्वम् । त॒नुवः॑ । व॒द्‌र्ध॒य॒स्व॒ । स्वाहा॑कृत॒मिति॒ स्वाहा᳚ - कृ॒त॒म् । ह॒विः । अ॒द॒न्तु॒ । दे॒वाः ॥ स्वाहा᳚ । दे॒वेभ्यः॑ । दे॒वेभ्यः॑ । स्वाहा᳚ ॥  \newline


\textbf{Krama Paata} \newline

दे॒वेभि॑रिन्वि॒तम् । इ॒न्वि॒तमिती᳚न्वि॒तम् ॥ पाशा᳚त् प॒शुम् । प॒शुम् प्र । प्र मु॑ञ्चत । मु॒ञ्च॒त॒ ब॒न्धात् । ब॒न्धाद् य॒ज्ञ्प॑तिम् । य॒ज्ञ्प॑ति॒म् परि॑ । य॒ज्ञ्प॑ति॒मिति॑ य॒ज्ञ् - प॒ति॒म् । परीति॒ परि॑ ॥ अदि॑तिः॒ पाश᳚म् । पाश॒म् प्र । प्र मु॑मोक्तु । मु॒मो॒क्त्वे॒तम् । ए॒तम् नमः॑ । नमः॑ प॒शुभ्यः॑ । प॒शुभ्यः॑ पशु॒पत॑ये । प॒शुभ्य॒ इति॑ प॒शु - भ्यः॒ । प॒शु॒पत॑ये करोमि । प॒शु॒पत॑य॒ इति॑ पशु - पत॑ये । क॒रो॒मीति॑ करोमि ॥ अ॒रा॒ती॒यन्त॒मध॑रम् । अध॑रम् कृणोमि । कृ॒णो॒मि॒ यम् । यम् द्वि॒ष्मः । द्वि॒ष्मस्तस्मिन्न्॑ । तस्मि॒न् प्रति॑ । प्रति॑ मुञ्चामि । मु॒ञ्चा॒मि॒ पाश᳚म् । पाश॒मिति॒ पाश᳚म् ॥ त्वामु॑ । उ॒ ते । ते द॑धिरे । द॒धि॒रे॒ ह॒व्य॒वाह᳚म् । ह॒व्य॒वाहꣳ॑ शृतङ्क॒र्तार᳚म् । ह॒व्य॒वाह॒मिति॑ हव्य - वाह᳚म् । शृ॒त॒ङ्क॒र्तार॑मु॒त । शृ॒त॒ङ्क॒र्तार॒मिति॑ शृतम् - क॒र्तार᳚म् । उ॒त य॒ज्ञिय᳚म् । य॒ज्ञिय॑म् च । चेति॑ च ॥ अग्ने॒ सद॑क्षः । सद॑क्षः॒ सत॑नुः । सद॑क्ष॒ इति॒ स - द॒क्षः॒ । सत॑नु॒र्॒. हि । सत॑नु॒रिति॒ स - त॒नुः॒ । हि भू॒त्वा । भू॒त्वाऽथ॑ । अथ॑ ह॒व्या । ह॒व्या जा॑तवेदः । जा॒त॒वे॒दो॒ जु॒ष॒स्व॒ । जा॒त॒वे॒द॒ इति॑ जात - वे॒दः॒ । जु॒ष॒स्वेति॑ जुषस्व ॥ जात॑वेदो व॒पया᳚ । जात॑वेद॒ इति॒ जात॑ - वे॒दः॒ । व॒पया॑ गच्छ । ग॒च्छ॒ दे॒वान् । दे॒वान् त्वम् ( ) । त्वꣳ हि । हि होता᳚ । होता᳚ प्रथ॒मः । प्र॒थ॒मो ब॒भूथ॑ । ब॒भूथेति॑ ब॒भूथ॑ ॥ घृ॒तेन॒ त्वम् । त्वम् त॒नुवः॑ । त॒नुवो॑ वर्द्धयस्व । व॒र्द्ध॒य॒स्व॒ स्वाहा॑कृतम् । स्वाहा॑कृतꣳ ह॒विः । स्वाहा॑कृत॒मिति॒ स्वाहा᳚ - कृ॒त॒म् । ह॒विर॑दन्तु । अ॒द॒न्तु॒ दे॒वाः । दे॒वा इति॑ दे॒वाः ॥ स्वाहा॑ दे॒वेभ्यः॑ । दे॒वेभ्यो॑ दे॒वेभ्यः॑ । दे॒वेभ्यः॒ स्वाहा᳚ । स्वाहेति॒ स्वाहा᳚ । \newline

\textbf{Jatai Paata} \newline

1. दे॒वेभि॑ रिन्वि॒त मि॑न्वि॒तम् दे॒वेभि॑र् दे॒वेभि॑ रिन्वि॒तम् । \newline
2. इ॒न्वि॒तमिती᳚न्वि॒तम् । \newline
3. पाशा᳚त् प॒शुम् प॒शुम् पाशा॒त् पाशा᳚त् प॒शुम् । \newline
4. प॒शुम् प्र प्र प॒शुम् प॒शुम् प्र । \newline
5. प्र मु॑ञ्चत मुञ्चत॒ प्र प्र मु॑ञ्चत । \newline
6. मु॒ञ्च॒त॒ ब॒न्धाद् ब॒न्धान् मु॑ञ्चत मुञ्चत ब॒न्धात् । \newline
7. ब॒न्धाद् य॒ज्ञ्प॑तिं ॅय॒ज्ञ्प॑तिम् ब॒न्धाद् ब॒न्धाद् य॒ज्ञ्प॑तिम् । \newline
8. य॒ज्ञ्प॑ति॒म् परि॒ परि॑ य॒ज्ञ्प॑तिं ॅय॒ज्ञ्प॑ति॒म् परि॑ । \newline
9. य॒ज्ञ्प॑ति॒मिति॑ य॒ज्ञ् - प॒ति॒म् । \newline
10. परीति॒ परि॑ । \newline
11. अदि॑तिः॒ पाश॒म् पाश॒ मदि॑ति॒ रदि॑तिः॒ पाश᳚म् । \newline
12. पाश॒म् प्र प्र पाश॒म् पाश॒म् प्र । \newline
13. प्र मु॑मोक्तु मुमोक्तु॒ प्र प्र मु॑मोक्तु । \newline
14. मु॒मो॒क्त्वे॒त मे॒तम् मु॑मोक्तु मुमोक्त्वे॒तम् । \newline
15. ए॒तम् नमो॒ नम॑ ए॒त मे॒तम् नमः॑ । \newline
16. नमः॑ प॒शुभ्यः॑ प॒शुभ्यो॒ नमो॒ नमः॑ प॒शुभ्यः॑ । \newline
17. प॒शुभ्यः॑ पशु॒पत॑ये पशु॒पत॑ये प॒शुभ्यः॑ प॒शुभ्यः॑ पशु॒पत॑ये । \newline
18. प॒शुभ्य॒ इति॑ प॒शु - भ्यः॒ । \newline
19. प॒शु॒पत॑ये करोमि करोमि पशु॒पत॑ये पशु॒पत॑ये करोमि । \newline
20. प॒शु॒पत॑य॒ इति॑ पशु - पत॑ये । \newline
21. क॒रो॒मीति॑ करोमि । \newline
22. अ॒रा॒ती॒यन्त॒ मध॑र॒ मध॑र मराती॒यन्त॑ मराती॒यन्त॒ मध॑रम् । \newline
23. अध॑रम् कृणोमि कृणो॒ म्यध॑र॒ मध॑रम् कृणोमि । \newline
24. कृ॒णो॒मि॒ यं ॅयम् कृ॑णोमि कृणोमि॒ यम् । \newline
25. यम् द्वि॒ष्मो द्वि॒ष्मो यं ॅयम् द्वि॒ष्मः । \newline
26. द्वि॒ष्म स्तस्मिꣳ॒॒ स्तस्मि॑न् द्वि॒ष्मो द्वि॒ष्म स्तस्मिन्न्॑ । \newline
27. तस्मि॒न् प्रति॒ प्रति॒ तस्मिꣳ॒॒ स्तस्मि॒न् प्रति॑ । \newline
28. प्रति॑ मुञ्चामि मुञ्चामि॒ प्रति॒ प्रति॑ मुञ्चामि । \newline
29. मु॒ञ्चा॒मि॒ पाश॒म् पाश॑म् मुञ्चामि मुञ्चामि॒ पाश᳚म् । \newline
30. पाश॒मिति॒ पाश᳚म् । \newline
31. त्वा मु॑ वु॒ त्वाम् त्वा मु॑ । \newline
32. उ॒ ते त उ॑ वु॒ ते । \newline
33. ते द॑धिरे दधिरे॒ ते ते द॑धिरे । \newline
34. द॒धि॒रे॒ ह॒व्य॒वाहꣳ॑ हव्य॒वाह॑म् दधिरे दधिरे हव्य॒वाह᳚म् । \newline
35. ह॒व्य॒वाहꣳ॑ शृतङ्क॒र्तारꣳ॑ शृतङ्क॒र्तारꣳ॑ हव्य॒वाहꣳ॑ हव्य॒वाहꣳ॑ शृतङ्क॒र्तार᳚म् । \newline
36. ह॒व्य॒वाह॒मिति॑ हव्य - वाह᳚म् । \newline
37. शृ॒त॒ङ्क॒र्तार॑ मु॒तोत शृ॑तङ्क॒र्तारꣳ॑ शृतङ्क॒र्तार॑ मु॒त । \newline
38. शृ॒त॒ङ्क॒र्तार॒मिति॑ शृतं - क॒र्तार᳚म् । \newline
39. उ॒त य॒ज्ञियं॑ ॅय॒ज्ञिय॑ मु॒तोत य॒ज्ञिय᳚म् । \newline
40. य॒ज्ञिय॑म् च च य॒ज्ञियं॑ ॅय॒ज्ञिय॑म् च । \newline
41. चेति॑ च । \newline
42. अग्ने॒ सद॑क्षः॒ सद॒क्षो ऽग्ने ऽग्ने॒ सद॑क्षः । \newline
43. सद॑क्षः॒ सत॑नुः॒ सत॑नुः॒ सद॑क्षः॒ सद॑क्षः॒ सत॑नुः । \newline
44. सद॑क्ष॒ इति॒ स - द॒क्षः॒ । \newline
45. सत॑नु॒र्॒. हि हि सत॑नुः॒ सत॑नु॒र्॒. हि । \newline
46. सत॑नु॒रिति॒ स - त॒नुः॒ । \newline
47. हि भू॒त्वा भू॒त्वा हि हि भू॒त्वा । \newline
48. भू॒त्वा ऽथाथ॑ भू॒त्वा भू॒त्वा ऽथ॑ । \newline
49. अथ॑ ह॒व्या ह॒व्या ऽथाथ॑ ह॒व्या । \newline
50. ह॒व्या जा॑तवेदो जातवेदो ह॒व्या ह॒व्या जा॑तवेदः । \newline
51. जा॒त॒वे॒दो॒ जु॒ष॒स्व॒ जु॒ष॒स्व॒ जा॒त॒वे॒दो॒ जा॒त॒वे॒दो॒ जु॒ष॒स्व॒ । \newline
52. जा॒त॒वे॒द॒ इति॑ जात - वे॒दः॒ । \newline
53. जु॒ष॒स्वेति॑ जुषस्व । \newline
54. जात॑वेदो व॒पया॑ व॒पया॒ जात॑वेदो॒ जात॑वेदो व॒पया᳚ । \newline
55. जात॑वेद॒ इति॒ जात॑ - वे॒दः॒ । \newline
56. व॒पया॑ गच्छ गच्छ व॒पया॑ व॒पया॑ गच्छ । \newline
57. ग॒च्छ॒ दे॒वान् दे॒वान् ग॑च्छ गच्छ दे॒वान् । \newline
58. दे॒वान् त्वम् त्वम् दे॒वान् दे॒वान् त्वम् । \newline
59. त्वꣳ हि हि त्वम् त्वꣳ हि । \newline
60. हि होता॒ होता॒ हि हि होता᳚ । \newline
61. होता᳚ प्रथ॒मः प्र॑थ॒मो होता॒ होता᳚ प्रथ॒मः । \newline
62. प्र॒थ॒मो ब॒भूथ॑ ब॒भूथ॑ प्रथ॒मः प्र॑थ॒मो ब॒भूथ॑ । \newline
63. ब॒भूथेति॑ ब॒भूथ॑ । \newline
64. घृ॒तेन॒ त्वम् त्वम् घृ॒तेन॑ घृ॒तेन॒ त्वम् । \newline
65. त्वम् त॒नुव॑ स्त॒नुव॒ स्त्वम् त्वम् त॒नुवः॑ । \newline
66. त॒नुवो॑ वर्द्धयस्व वर्द्धयस्व त॒नुव॑ स्त॒नुवो॑ वर्द्धयस्व । \newline
67. व॒र्द्ध॒य॒स्व॒ स्वाहा॑कृतꣳ॒॒ स्वाहा॑कृतं ॅवर्द्धयस्व वर्द्धयस्व॒ स्वाहा॑कृतम् । \newline
68. स्वाहा॑कृतꣳ ह॒विर्. ह॒विः स्वाहा॑कृतꣳ॒॒ स्वाहा॑कृतꣳ ह॒विः । \newline
69. स्वाहा॑कृत॒मिति॒ स्वाहा᳚ - कृ॒त॒म् । \newline
70. ह॒वि र॑द न्त्वदन्तु ह॒विर्. ह॒वि र॑दन्तु । \newline
71. अ॒द॒न्तु॒ दे॒वा दे॒वा अ॑द न्त्वदन्तु दे॒वाः । \newline
72. दे॒वा इति॑ दे॒वाः । \newline
73. स्वाहा॑ दे॒वेभ्यो॑ दे॒वेभ्यः॒ स्वाहा॒ स्वाहा॑ दे॒वेभ्यः॑ । \newline
74. दे॒वेभ्यो॑ दे॒वेभ्यः॑ । \newline
75. दे॒वेभ्यः॒ स्वाहा॒ स्वाहा॑ दे॒वेभ्यो॑ दे॒वेभ्यः॒ स्वाहा᳚ । \newline
76. स्वाहेति॒ स्वाहा᳚ । \newline

\textbf{Ghana Paata } \newline

1. दे॒वेभि॑ रिन्वि॒त मि॑न्वि॒तम् दे॒वेभि॑र् दे॒वेभि॑ रिन्वि॒तम् । \newline
2. इ॒न्वि॒तमिती᳚न्वि॒तम् । \newline
3. पाशा᳚त् प॒शुम् प॒शुम् पाशा॒त् पाशा᳚त् प॒शुम् प्र प्र प॒शुम् पाशा॒त् पाशा᳚त् प॒शुम् प्र । \newline
4. प॒शुम् प्र प्र प॒शुम् प॒शुम् प्र मु॑ञ्चत मुञ्चत॒ प्र प॒शुम् प॒शुम् प्र मु॑ञ्चत । \newline
5. प्र मु॑ञ्चत मुञ्चत॒ प्र प्र मु॑ञ्चत ब॒न्धाद् ब॒न्धान् मु॑ञ्चत॒ प्र प्र मु॑ञ्चत ब॒न्धात् । \newline
6. मु॒ञ्च॒त॒ ब॒न्धाद् ब॒न्धान् मु॑ञ्चत मुञ्चत ब॒न्धाद् य॒ज्ञ्प॑तिं ॅय॒ज्ञ्प॑तिम् ब॒न्धान् मु॑ञ्चत मुञ्चत ब॒न्धाद् य॒ज्ञ्प॑तिम् । \newline
7. ब॒न्धाद् य॒ज्ञ्प॑तिं ॅय॒ज्ञ्प॑तिम् ब॒न्धाद् ब॒न्धाद् य॒ज्ञ्प॑ति॒म् परि॒ परि॑ य॒ज्ञ्प॑तिम् ब॒न्धाद् ब॒न्धाद् य॒ज्ञ्प॑ति॒म् परि॑ । \newline
8. य॒ज्ञ्प॑ति॒म् परि॒ परि॑ य॒ज्ञ्प॑तिं ॅय॒ज्ञ्प॑ति॒म् परि॑ । \newline
9. य॒ज्ञ्प॑ति॒मिति॑ य॒ज्ञ् - प॒ति॒म् । \newline
10. परीति॒ परि॑ । \newline
11. अदि॑तिः॒ पाश॒म् पाश॒ मदि॑ति॒ रदि॑तिः॒ पाश॒म् प्र प्र पाश॒ मदि॑ति॒ रदि॑तिः॒ पाश॒म् प्र । \newline
12. पाश॒म् प्र प्र पाश॒म् पाश॒म् प्र मु॑मोक्तु मुमोक्तु॒ प्र पाश॒म् पाश॒म् प्र मु॑मोक्तु । \newline
13. प्र मु॑मोक्तु मुमोक्तु॒ प्र प्र मु॑मोक्त्वे॒त मे॒तम् मु॑मोक्तु॒ प्र प्र मु॑मोक्त्वे॒तम् । \newline
14. मु॒मो॒क्त्वे॒त मे॒तम् मु॑मोक्तु मुमोक्त्वे॒तम् नमो॒ नम॑ ए॒तम् मु॑मोक्तु मुमोक्त्वे॒तम् नमः॑ । \newline
15. ए॒तम् नमो॒ नम॑ ए॒त मे॒तम् नमः॑ प॒शुभ्यः॑ प॒शुभ्यो॒ नम॑ ए॒त मे॒तम् नमः॑ प॒शुभ्यः॑ । \newline
16. नमः॑ प॒शुभ्यः॑ प॒शुभ्यो॒ नमो॒ नमः॑ प॒शुभ्यः॑ पशु॒पत॑ये पशु॒पत॑ये प॒शुभ्यो॒ नमो॒ नमः॑ प॒शुभ्यः॑ पशु॒पत॑ये । \newline
17. प॒शुभ्यः॑ पशु॒पत॑ये पशु॒पत॑ये प॒शुभ्यः॑ प॒शुभ्यः॑ पशु॒पत॑ये करोमि करोमि पशु॒पत॑ये प॒शुभ्यः॑ प॒शुभ्यः॑ पशु॒पत॑ये करोमि । \newline
18. प॒शुभ्य॒ इति॑ प॒शु - भ्यः॒ । \newline
19. प॒शु॒पत॑ये करोमि करोमि पशु॒पत॑ये पशु॒पत॑ये करोमि । \newline
20. प॒शु॒पत॑य॒ इति॑ पशु - पत॑ये । \newline
21. क॒रो॒मीति॑ करोमि । \newline
22. अ॒रा॒ती॒यन्त॒ मध॑र॒ मध॑र मराती॒यन्त॑ मराती॒यन्त॒ मध॑रम् कृणोमि कृणो॒म्यध॑र मराती॒यन्त॑ मराती॒यन्त॒ मध॑रम् कृणोमि । \newline
23. अध॑रम् कृणोमि कृणो॒ म्यध॑र॒ मध॑रम् कृणोमि॒ यं ॅयम् कृ॑णो॒ म्यध॑र॒ मध॑रम् कृणोमि॒ यम् । \newline
24. कृ॒णो॒मि॒ यं ॅयम् कृ॑णोमि कृणोमि॒ यम् द्वि॒ष्मो द्वि॒ष्मो यम् कृ॑णोमि कृणोमि॒ यम् द्वि॒ष्मः । \newline
25. यम् द्वि॒ष्मो द्वि॒ष्मो यं ॅयम् द्वि॒ष्म स्तस्मिꣳ॒॒ स्तस्मि॑न् द्वि॒ष्मो यं ॅयम् द्वि॒ष्म स्तस्मिन्न्॑ । \newline
26. द्वि॒ष्म स्तस्मिꣳ॒॒ स्तस्मि॑न् द्वि॒ष्मो द्वि॒ष्म स्तस्मि॒न् प्रति॒ प्रति॒ तस्मि॑न् द्वि॒ष्मो द्वि॒ष्म स्तस्मि॒न् प्रति॑ । \newline
27. तस्मि॒न् प्रति॒ प्रति॒ तस्मिꣳ॒॒ स्तस्मि॒न् प्रति॑ मुञ्चामि मुञ्चामि॒ प्रति॒ तस्मिꣳ॒॒ स्तस्मि॒न् प्रति॑ मुञ्चामि । \newline
28. प्रति॑ मुञ्चामि मुञ्चामि॒ प्रति॒ प्रति॑ मुञ्चामि॒ पाश॒म् पाश॑म् मुञ्चामि॒ प्रति॒ प्रति॑ मुञ्चामि॒ पाश᳚म् । \newline
29. मु॒ञ्चा॒मि॒ पाश॒म् पाश॑म् मुञ्चामि मुञ्चामि॒ पाश᳚म् । \newline
30. पाश॒मिति॒ पाश᳚म् । \newline
31. त्वा मु॑ वु॒ त्वाम् त्वा मु॒ ते त उ॒ त्वाम् त्वा मु॒ ते । \newline
32. उ॒ ते त उ॑ वु॒ ते द॑धिरे दधिरे॒ त उ॑ वु॒ ते द॑धिरे । \newline
33. ते द॑धिरे दधिरे॒ ते ते द॑धिरे हव्य॒वाहꣳ॑ हव्य॒वाह॑म् दधिरे॒ ते ते द॑धिरे हव्य॒वाह᳚म् । \newline
34. द॒धि॒रे॒ ह॒व्य॒वाहꣳ॑ हव्य॒वाह॑म् दधिरे दधिरे हव्य॒वाहꣳ॑ शृतङ्क॒र्तारꣳ॑ शृतङ्क॒र्तारꣳ॑ हव्य॒वाह॑म् दधिरे दधिरे हव्य॒वाहꣳ॑ शृतङ्क॒र्तार᳚म् । \newline
35. ह॒व्य॒वाहꣳ॑ शृतङ्क॒र्तारꣳ॑ शृतङ्क॒र्तारꣳ॑ हव्य॒वाहꣳ॑ हव्य॒वाहꣳ॑ शृतङ्क॒र्तार॑ मु॒तोत शृ॑तङ्क॒र्तारꣳ॑ हव्य॒वाहꣳ॑ हव्य॒वाहꣳ॑ शृतङ्क॒र्तार॑ मु॒त । \newline
36. ह॒व्य॒वाह॒मिति॑ हव्य - वाह᳚म् । \newline
37. शृ॒त॒ङ्क॒र्तार॑ मु॒तोत शृ॑तङ्क॒र्तारꣳ॑ शृतङ्क॒र्तार॑ मु॒त य॒ज्ञियं॑ ॅय॒ज्ञिय॑ मु॒त शृ॑तङ्क॒र्तारꣳ॑ शृतङ्क॒र्तार॑ मु॒त य॒ज्ञिय᳚म् । \newline
38. शृ॒त॒ङ्क॒र्तार॒मिति॑ शृतं - क॒र्तार᳚म् । \newline
39. उ॒त य॒ज्ञियं॑ ॅय॒ज्ञिय॑ मु॒तोत य॒ज्ञिय॑म् च च य॒ज्ञिय॑ मु॒तोत य॒ज्ञिय॑म् च । \newline
40. य॒ज्ञिय॑म् च च य॒ज्ञियं॑ ॅय॒ज्ञिय॑म् च । \newline
41. चेति॑ च । \newline
42. अग्ने॒ सद॑क्षः॒ सद॒क्षो ऽग्ने ऽग्ने॒ सद॑क्षः॒ सत॑नुः॒ सत॑नुः॒ सद॒क्षो ऽग्ने ऽग्ने॒ सद॑क्षः॒ सत॑नुः । \newline
43. सद॑क्षः॒ सत॑नुः॒ सत॑नुः॒ सद॑क्षः॒ सद॑क्षः॒ सत॑नु॒र्॒. हि हि सत॑नुः॒ सद॑क्षः॒ सद॑क्षः॒ सत॑नु॒र्॒. हि । \newline
44. सद॑क्ष॒ इति॒ स - द॒क्षः॒ । \newline
45. सत॑नु॒र्॒. हि हि सत॑नुः॒ सत॑नु॒र्॒. हि भू॒त्वा भू॒त्वा हि सत॑नुः॒ सत॑नु॒र्॒. हि भू॒त्वा । \newline
46. सत॑नु॒रिति॒ स - त॒नुः॒ । \newline
47. हि भू॒त्वा भू॒त्वा हि हि भू॒त्वा ऽथाथ॑ भू॒त्वा हि हि भू॒त्वा ऽथ॑ । \newline
48. भू॒त्वा ऽथाथ॑ भू॒त्वा भू॒त्वा ऽथ॑ ह॒व्या ह॒व्या ऽथ॑ भू॒त्वा भू॒त्वा ऽथ॑ ह॒व्या । \newline
49. अथ॑ ह॒व्या ह॒व्या ऽथाथ॑ ह॒व्या जा॑तवेदो जातवेदो ह॒व्या ऽथाथ॑ ह॒व्या जा॑तवेदः । \newline
50. ह॒व्या जा॑तवेदो जातवेदो ह॒व्या ह॒व्या जा॑तवेदो जुषस्व जुषस्व जातवेदो ह॒व्या ह॒व्या जा॑तवेदो जुषस्व । \newline
51. जा॒त॒वे॒दो॒ जु॒ष॒स्व॒ जु॒ष॒स्व॒ जा॒त॒वे॒दो॒ जा॒त॒वे॒दो॒ जु॒ष॒स्व॒ । \newline
52. जा॒त॒वे॒द॒ इति॑ जात - वे॒दः॒ । \newline
53. जु॒ष॒स्वेति॑ जुषस्व । \newline
54. जात॑वेदो व॒पया॑ व॒पया॒ जात॑वेदो॒ जात॑वेदो व॒पया॑ गच्छ गच्छ व॒पया॒ जात॑वेदो॒ जात॑वेदो व॒पया॑ गच्छ । \newline
55. जात॑वेद॒ इति॒ जात॑ - वे॒दः॒ । \newline
56. व॒पया॑ गच्छ गच्छ व॒पया॑ व॒पया॑ गच्छ दे॒वान् दे॒वान् ग॑च्छ व॒पया॑ व॒पया॑ गच्छ दे॒वान् । \newline
57. ग॒च्छ॒ दे॒वान् दे॒वान् ग॑च्छ गच्छ दे॒वान् त्वम् त्वम् दे॒वान् ग॑च्छ गच्छ दे॒वान् त्वम् । \newline
58. दे॒वान् त्वम् त्वम् दे॒वान् दे॒वान् त्वꣳ हि हि त्वम् दे॒वान् दे॒वान् त्वꣳ हि । \newline
59. त्वꣳ हि हि त्वम् त्वꣳ हि होता॒ होता॒ हि त्वम् त्वꣳ हि होता᳚ । \newline
60. हि होता॒ होता॒ हि हि होता᳚ प्रथ॒मः प्र॑थ॒मो होता॒ हि हि होता᳚ प्रथ॒मः । \newline
61. होता᳚ प्रथ॒मः प्र॑थ॒मो होता॒ होता᳚ प्रथ॒मो ब॒भूथ॑ ब॒भूथ॑ प्रथ॒मो होता॒ होता᳚ प्रथ॒मो ब॒भूथ॑ । \newline
62. प्र॒थ॒मो ब॒भूथ॑ ब॒भूथ॑ प्रथ॒मः प्र॑थ॒मो ब॒भूथ॑ । \newline
63. ब॒भूथेति॑ ब॒भूथ॑ । \newline
64. घृ॒तेन॒ त्वम् त्वम् घृ॒तेन॑ घृ॒तेन॒ त्वम् त॒नुव॑ स्त॒नुव॒ स्त्वम् घृ॒तेन॑ घृ॒तेन॒ त्वम् त॒नुवः॑ । \newline
65. त्वम् त॒नुव॑ स्त॒नुव॒ स्त्वम् त्वम् त॒नुवो॑ वर्द्धयस्व वर्द्धयस्व त॒नुव॒ स्त्वम् त्वम् त॒नुवो॑ वर्द्धयस्व । \newline
66. त॒नुवो॑ वर्द्धयस्व वर्द्धयस्व त॒नुव॑ स्त॒नुवो॑ वर्द्धयस्व॒ स्वाहा॑कृतꣳ॒॒ स्वाहा॑कृतं ॅवर्द्धयस्व त॒नुव॑ स्त॒नुवो॑ वर्द्धयस्व॒ स्वाहा॑कृतम् । \newline
67. व॒र्द्ध॒य॒स्व॒ स्वाहा॑कृतꣳ॒॒ स्वाहा॑कृतं ॅवर्द्धयस्व वर्द्धयस्व॒ स्वाहा॑कृतꣳ ह॒विर्. ह॒विः स्वाहा॑कृतं ॅवर्द्धयस्व वर्द्धयस्व॒ स्वाहा॑कृतꣳ ह॒विः । \newline
68. स्वाहा॑कृतꣳ ह॒विर्. ह॒विः स्वाहा॑कृतꣳ॒॒ स्वाहा॑कृतꣳ ह॒वि र॑द न्त्वदन्तु ह॒विः स्वाहा॑कृतꣳ॒॒ स्वाहा॑कृतꣳ ह॒वि र॑दन्तु । \newline
69. स्वाहा॑कृत॒मिति॒ स्वाहा᳚ - कृ॒त॒म् । \newline
70. ह॒वि र॑द न्त्वदन्तु ह॒विर्. ह॒वि र॑दन्तु दे॒वा दे॒वा अ॑दन्तु ह॒विर्. ह॒वि र॑दन्तु दे॒वाः । \newline
71. अ॒द॒न्तु॒ दे॒वा दे॒वा अ॑द न्त्वदन्तु दे॒वाः । \newline
72. दे॒वा इति॑ दे॒वाः । \newline
73. स्वाहा॑ दे॒वेभ्यो॑ दे॒वेभ्यः॒ स्वाहा॒ स्वाहा॑ दे॒वेभ्यः॑ । \newline
74. दे॒वेभ्यो॑ दे॒वेभ्यः॑ । \newline
75. दे॒वेभ्यः॒ स्वाहा॒ स्वाहा॑ दे॒वेभ्यो॑ दे॒वेभ्यः॒ स्वाहा᳚ । \newline
76. स्वाहेति॒ स्वाहा᳚ । \newline
\pagebreak
\markright{ TS 3.1.5.1  \hfill https://www.vedavms.in \hfill}

\section{ TS 3.1.5.1 }

\textbf{TS 3.1.5.1 } \newline
\textbf{Samhita Paata} \newline

प्रा॒जा॒प॒त्या वै प॒शव॒स्तेषाꣳ॑ रु॒द्रोऽधि॑पति॒र्य-दे॒ताभ्या॑-मुपा क॒रोति॒ ताभ्या॑मे॒वैनं॑ प्रति॒प्रोच्याऽऽल॑भत आ॒त्मनोऽना᳚व्रस्काय॒ द्वाभ्या॑मु॒पाक॑रोति द्वि॒पाद्यज॑मानः॒ प्रति॑ष्ठित्या उपा॒कृत्य॒ पञ्च॑ जुहोति॒ पाङ्क्ताः᳚ प॒शवः॑ प॒शूने॒वा व॑रुन्धेमृ॒त्यवे॒ वा ए॒ष नी॑यते॒ यत् प॒शुस्तं ॅयद॑न्वा॒रभे॑त प्र॒मायु॑को॒ यज॑मानः स्या॒न्नाना᳚ प्रा॒णो यज॑मानस्य प॒शुनेत्या॑ह॒ व्यावृ॑त्त्यै॒ - [  ] \newline

\textbf{Pada Paata} \newline

प्रा॒जा॒प॒त्या इति॑ प्राजा - प॒त्याः । वै । प॒शवः॑ । तेषा᳚म् । रु॒द्रः । अधि॑पति॒रित्यधि॑ - प॒तिः॒ । यत् । ए॒ताभ्या᳚म् । उ॒पा॒क॒रोतीत्यु॑प - आ॒क॒रोति॑ । ताभ्या᳚म् । ए॒व । ए॒न॒म् । प्र॒ति॒प्रोच्येति॑ प्रति - प्रोच्य॑ । एति॑ । ल॒भ॒ते॒ । आ॒त्मनः॑ । अना᳚व्रस्का॒येत्यना᳚ - व्र॒स्का॒य॒ । द्वाभ्या᳚म् । उ॒पाक॑रा॒तीत्यु॑प - आक॑रोति । द्वि॒पादिति॑ द्वि - पात् । यज॑मानः । प्रति॑ष्ठित्या॒ इति॒ प्रति॑ - स्थि॒त्यै॒ । उ॒पा॒कृत्येत्यु॑प - आ॒कृत्य॑ । पञ्च॑ । जु॒हो॒ति॒ । पाङ्क्ताः᳚ । प॒शवः॑ । प॒शून् । ए॒व । अवेति॑ । रु॒न्धे॒ । मृ॒त्यवे᳚ । वै । ए॒षः । नी॒य॒ते॒ । यत् । प॒शुः । तम् । यत् । अ॒न्वा॒रभे॒तेत्य॑नु - आ॒रभे॑त । प्र॒मायु॑क॒ इति॑ प्र - मायु॑कः । यज॑मानः । स्या॒त् । नाना᳚ । प्रा॒ण इति॑ प्र - अ॒नः । यज॑मानस्य । प॒शुना᳚ । इति॑ । आ॒ह॒ । व्यावृ॑त्त्या॒ इति॑ वि - आवृ॑त्त्यै ।  \newline


\textbf{Krama Paata} \newline

प्रा॒जा॒प॒त्या वै । प्रा॒जा॒प॒त्या इति॑ प्राजा - प॒त्याः । वै प॒शवः॑ । प॒शव॒ स्तेषा᳚म् । तेषाꣳ॑ रु॒द्रः । रु॒द्रोऽधि॑पतिः । अधि॑पति॒र् यत् । अधि॑पति॒रित्यधि॑ - प॒तिः॒ । यदे॒ताभ्या᳚म् । ए॒ताभ्या॑मुपाक॒रोति॑ । उ॒पा॒क॒रोति॒ ताभ्या᳚म् । उ॒पा॒क॒रोतीत्यु॑प - आ॒क॒रोति॑ । ताभ्या॑मे॒व । ए॒वैन᳚म् । ए॒न॒म् प्र॒ति॒प्रोच्य॑ । प्र॒ति॒प्रोच्या । प्र॒ति॒प्रोच्येति॑ प्रति - प्रोच्य॑ । आ ल॑भते । ल॒भ॒त॒ आ॒त्मनः॑ । आ॒त्मनोऽना᳚व्रस्काय । अना᳚व्रस्काय॒ द्वाभ्या᳚म् । अना᳚व्रस्का॒येत्यना᳚ - व्र॒स्का॒य॒ । द्वाभ्या॑मु॒पाक॑रोति । उ॒पाक॑रोति द्वि॒पात् । उ॒पाक॑रो॒तीत्यु॑प - आक॑रोति । द्वि॒पाद् यज॑मानः । द्वि॒पादिति॑ द्वि - पात् । यज॑मानः॒ प्रति॑ष्ठित्यै । प्रति॑ष्ठित्या उपा॒कृत्य॑ । प्रति॑ष्ठित्या॒ इति॒ प्रति॑ - स्थि॒त्यै॒ । उ॒पा॒कृत्य॒ पञ्च॑ । उ॒पा॒कृत्येत्यु॑प - आ॒कृत्य॑ । पञ्च॑ जुहोति । जु॒हो॒ति॒ पाङ्क्ताः᳚ । पाङ्क्ताः᳚ प॒शवः॑ । प॒शवः॑ प॒शून् । प॒शूने॒व । ए॒वाव॑ । अव॑ रुन्धे । रु॒न्धे॒ मृ॒त्यवे᳚ । मृ॒त्यवे॒ वै । वा ए॒षः । ए॒ष नी॑यते । नी॒य॒ते॒ यत् । यत् प॒शुः । प॒शुस्तम् । तम् यत् । यद॑न्वा॒रभे॑त । अ॒न्वा॒रभे॑त प्र॒मायु॑कः । अ॒न्वा॒रभे॒तेत्य॑नु - आ॒रभे॑त । प्र॒मायु॑को॒ यज॑मानः । प्र॒मायु॑क॒ इति॑ प्र - मायु॑कः । यज॑मानः स्यात् । स्या॒न् नाना᳚ । नाना᳚ प्रा॒णः । प्रा॒णो यज॑मानस्य । प्रा॒ण इति॑ प्र - अ॒नः । यज॑मानस्य प॒शुना᳚ । प॒शुनेति॑ । इत्या॑ह । आ॒ह॒ व्यावृ॑त्त्यै । व्यावृ॑त्त्यै॒ यत् । व्यावृ॑त्त्या॒ इति॑ वि - आवृ॑त्त्यै \newline

\textbf{Jatai Paata} \newline

1. प्रा॒जा॒प॒त्या वै वै प्रा॑जाप॒त्याः प्रा॑जाप॒त्या वै । \newline
2. प्रा॒जा॒प॒त्या इति॑ प्राजा - प॒त्याः । \newline
3. वै प॒शवः॑ प॒शवो॒ वै वै प॒शवः॑ । \newline
4. प॒शव॒ स्तेषा॒म् तेषा᳚म् प॒शवः॑ प॒शव॒ स्तेषा᳚म् । \newline
5. तेषाꣳ॑ रु॒द्रो रु॒द्र स्तेषा॒म् तेषाꣳ॑ रु॒द्रः । \newline
6. रु॒द्रो ऽधि॑पति॒ रधि॑पती रु॒द्रो रु॒द्रो ऽधि॑पतिः । \newline
7. अधि॑पति॒र् यद् यदधि॑पति॒ रधि॑पति॒र् यत् । \newline
8. अधि॑पति॒रित्यधि॑ - प॒तिः॒ । \newline
9. यदे॒ताभ्या॑ मे॒ताभ्यां॒ ॅयद् यदे॒ताभ्या᳚म् । \newline
10. ए॒ताभ्या॑ मुपाक॒रो त्यु॑पाक॒रो त्ये॒ताभ्या॑ मे॒ताभ्या॑ मुपाक॒रोति॑ । \newline
11. उ॒पा॒क॒रोति॒ ताभ्या॒म् ताभ्या॑ मुपाक॒रो त्यु॑पाक॒रोति॒ ताभ्या᳚म् । \newline
12. उ॒पा॒क॒रोतीत्यु॑प - आ॒क॒रोति॑ । \newline
13. ताभ्या॑ मे॒वैव ताभ्या॒म् ताभ्या॑ मे॒व । \newline
14. ए॒वैन॑ मेन मे॒वैवैन᳚म् । \newline
15. ए॒न॒म् प्र॒ति॒प्रोच्य॑ प्रति॒प्रोच्यै॑न मेनम् प्रति॒प्रोच्य॑ । \newline
16. प्र॒ति॒प्रोच्या प्र॑ति॒प्रोच्य॑ प्रति॒प्रोच्या । \newline
17. प्र॒ति॒प्रोच्येति॑ प्रति - प्रोच्य॑ । \newline
18. आ ल॑भते लभत॒ आ ल॑भते । \newline
19. ल॒भ॒त॒ आ॒त्मन॑ आ॒त्मनो॑ लभते लभत आ॒त्मनः॑ । \newline
20. आ॒त्मनो ऽना᳚व्रस्का॒या ना᳚व्रस्काया॒ त्मन॑ आ॒त्मनो ऽना᳚व्रस्काय । \newline
21. अना᳚व्रस्काय॒ द्वाभ्या॒म् द्वाभ्या॒ मना᳚व्रस्का॒या ना᳚व्रस्काय॒ द्वाभ्या᳚म् । \newline
22. अना᳚व्रस्का॒येत्यना᳚ - व्र॒स्का॒य॒ । \newline
23. द्वाभ्या॑ मु॒पाक॑रो त्यु॒पाक॑रोति॒ द्वाभ्या॒म् द्वाभ्या॑ मु॒पाक॑रोति । \newline
24. उ॒पाक॑रोति द्वि॒पाद् द्वि॒पा दु॒पाक॑रो त्यु॒पाक॑रोति द्वि॒पात् । \newline
25. उ॒पाक॑रा॒तीत्यु॑प - आक॑रोति । \newline
26. द्वि॒पाद् यज॑मानो॒ यज॑मानो द्वि॒पाद् द्वि॒पाद् यज॑मानः । \newline
27. द्वि॒पादिति॑ द्वि - पात् । \newline
28. यज॑मानः॒ प्रति॑ष्ठित्यै॒ प्रति॑ष्ठित्यै॒ यज॑मानो॒ यज॑मानः॒ प्रति॑ष्ठित्यै । \newline
29. प्रति॑ष्ठित्या उपा॒कृ त्यो॑पा॒कृत्य॒ प्रति॑ष्ठित्यै॒ प्रति॑ष्ठित्या उपा॒कृत्य॑ । \newline
30. प्रति॑ष्ठित्या॒ इति॒ प्रति॑ - स्थि॒त्यै॒ । \newline
31. उ॒पा॒कृत्य॒ पञ्च॒ पञ्चो॑ पा॒कृ त्यो॑पा॒कृत्य॒ पञ्च॑ । \newline
32. उ॒पा॒कृत्येत्यु॑प - आ॒कृत्य॑ । \newline
33. पञ्च॑ जुहोति जुहोति॒ पञ्च॒ पञ्च॑ जुहोति । \newline
34. जु॒हो॒ति॒ पाङ्क्ताः॒ पाङ्क्ता॑ जुहोति जुहोति॒ पाङ्क्ताः᳚ । \newline
35. पाङ्क्ताः᳚ प॒शवः॑ प॒शवः॒ पाङ्क्ताः॒ पाङ्क्ताः᳚ प॒शवः॑ । \newline
36. प॒शवः॑ प॒शून् प॒शून् प॒शवः॑ प॒शवः॑ प॒शून् । \newline
37. प॒शू ने॒वैव प॒शून् प॒शू ने॒व । \newline
38. ए॒वा वावै॒ वैवाव॑ । \newline
39. अव॑ रुन्धे रु॒न्धे ऽवाव॑ रुन्धे । \newline
40. रु॒न्धे॒ मृ॒त्यवे॑ मृ॒त्यवे॑ रुन्धे रुन्धे मृ॒त्यवे᳚ । \newline
41. मृ॒त्यवे॒ वै वै मृ॒त्यवे॑ मृ॒त्यवे॒ वै । \newline
42. वा ए॒ष ए॒ष वै वा ए॒षः । \newline
43. ए॒ष नी॑यते नीयत ए॒ष ए॒ष नी॑यते । \newline
44. नी॒य॒ते॒ यद् यन् नी॑यते नीयते॒ यत् । \newline
45. यत् प॒शुः प॒शुर् यद् यत् प॒शुः । \newline
46. प॒शु स्तम् तम् प॒शुः प॒शु स्तम् । \newline
47. तं ॅयद् यत् तम् तं ॅयत् । \newline
48. यद॑न्वा॒रभे॑ता न्वा॒रभे॑त॒ यद् यद॑न्वा॒रभे॑त । \newline
49. अ॒न्वा॒रभे॑त प्र॒मायु॑कः प्र॒मायु॑को ऽन्वा॒रभे॑ता न्वा॒रभे॑त प्र॒मायु॑कः । \newline
50. अ॒न्वा॒रभे॒तेत्य॑नु - आ॒रभे॑त । \newline
51. प्र॒मायु॑को॒ यज॑मानो॒ यज॑मानः प्र॒मायु॑कः प्र॒मायु॑को॒ यज॑मानः । \newline
52. प्र॒मायु॑क॒ इति॑ प्र - मायु॑कः । \newline
53. यज॑मानः स्याथ् स्या॒द् यज॑मानो॒ यज॑मानः स्यात् । \newline
54. स्या॒न् नाना॒ नाना᳚ स्याथ् स्या॒न् नाना᳚ । \newline
55. नाना᳚ प्रा॒णः प्रा॒णो नाना॒ नाना᳚ प्रा॒णः । \newline
56. प्रा॒णो यज॑मानस्य॒ यज॑मानस्य प्रा॒णः प्रा॒णो यज॑मानस्य । \newline
57. प्रा॒ण इति॑ प्र - अ॒नः । \newline
58. यज॑मानस्य प॒शुना॑ प॒शुना॒ यज॑मानस्य॒ यज॑मानस्य प॒शुना᳚ । \newline
59. प॒शुनेतीति॑ प॒शुना॑ प॒शुनेति॑ । \newline
60. इत्या॑हा॒हे तीत्या॑ह । \newline
61. आ॒ह॒ व्यावृ॑त्त्यै॒ व्यावृ॑त्त्या आहाह॒ व्यावृ॑त्त्यै । \newline
62. व्यावृ॑त्त्यै॒ यद् यद् व्यावृ॑त्त्यै॒ व्यावृ॑त्त्यै॒ यत् । \newline
63. व्यावृ॑त्त्या॒ इति॑ वि - आवृ॑त्त्यै । \newline

\textbf{Ghana Paata } \newline

1. प्रा॒जा॒प॒त्या वै वै प्रा॑जाप॒त्याः प्रा॑जाप॒त्या वै प॒शवः॑ प॒शवो॒ वै प्रा॑जाप॒त्याः प्रा॑जाप॒त्या वै प॒शवः॑ । \newline
2. प्रा॒जा॒प॒त्या इति॑ प्राजा - प॒त्याः । \newline
3. वै प॒शवः॑ प॒शवो॒ वै वै प॒शव॒ स्तेषा॒म् तेषा᳚म् प॒शवो॒ वै वै प॒शव॒ स्तेषा᳚म् । \newline
4. प॒शव॒ स्तेषा॒म् तेषा᳚म् प॒शवः॑ प॒शव॒ स्तेषाꣳ॑ रु॒द्रो रु॒द्र स्तेषा᳚म् प॒शवः॑ प॒शव॒ स्तेषाꣳ॑ रु॒द्रः । \newline
5. तेषाꣳ॑ रु॒द्रो रु॒द्र स्तेषा॒म् तेषाꣳ॑ रु॒द्रो ऽधि॑पति॒ रधि॑पती रु॒द्र स्तेषा॒म् तेषाꣳ॑ रु॒द्रो ऽधि॑पतिः । \newline
6. रु॒द्रो ऽधि॑पति॒ रधि॑पती रु॒द्रो रु॒द्रो ऽधि॑पति॒र् यद् यदधि॑पती रु॒द्रो रु॒द्रो ऽधि॑पति॒र् यत् । \newline
7. अधि॑पति॒र् यद् यदधि॑पति॒ रधि॑पति॒र् यदे॒ताभ्या॑ मे॒ताभ्यां॒ ॅयदधि॑पति॒ रधि॑पति॒र् यदे॒ताभ्या᳚म् । \newline
8. अधि॑पति॒रित्यधि॑ - प॒तिः॒ । \newline
9. यदे॒ताभ्या॑ मे॒ताभ्यां॒ ॅयद् यदे॒ताभ्या॑ मुपाक॒रो त्यु॑पाक॒रो त्ये॒ताभ्यां॒ ॅयद् यदे॒ताभ्या॑ मुपाक॒रोति॑ । \newline
10. ए॒ताभ्या॑ मुपाक॒रो त्यु॑पाक॒रो त्ये॒ताभ्या॑ मे॒ताभ्या॑ मुपाक॒रोति॒ ताभ्या॒म् ताभ्या॑ मुपाक॒रो त्ये॒ताभ्या॑ मे॒ताभ्या॑ मुपाक॒रोति॒ ताभ्या᳚म् । \newline
11. उ॒पा॒क॒रोति॒ ताभ्या॒म् ताभ्या॑ मुपाक॒रो त्यु॑पाक॒रोति॒ ताभ्या॑ मे॒वैव ताभ्या॑ मुपाक॒रो त्यु॑पाक॒रोति॒ ताभ्या॑ मे॒व । \newline
12. उ॒पा॒क॒रोतीत्यु॑प - आ॒क॒रोति॑ । \newline
13. ताभ्या॑ मे॒वैव ताभ्या॒म् ताभ्या॑ मे॒वैन॑ मेन मे॒व ताभ्या॒म् ताभ्या॑ मे॒वैन᳚म् । \newline
14. ए॒वैन॑ मेन मे॒वैवैन॑म् प्रति॒प्रोच्य॑ प्रति॒प्रोच्यै॑न मे॒वैवैन॑म् प्रति॒प्रोच्य॑ । \newline
15. ए॒न॒म् प्र॒ति॒प्रोच्य॑ प्रति॒प्रोच्यै॑न मेनम् प्रति॒प्रोच्या प्र॑ति॒प्रोच्यै॑न मेनम् प्रति॒प्रोच्या । \newline
16. प्र॒ति॒प्रोच्या प्र॑ति॒प्रोच्य॑ प्रति॒प्रोच्या ल॑भते लभत॒ आ प्र॑ति॒प्रोच्य॑ प्रति॒प्रोच्या ल॑भते । \newline
17. प्र॒ति॒प्रोच्येति॑ प्रति - प्रोच्य॑ । \newline
18. आ ल॑भते लभत॒ आ ल॑भत आ॒त्मन॑ आ॒त्मनो॑ लभत॒ आ ल॑भत आ॒त्मनः॑ । \newline
19. ल॒भ॒त॒ आ॒त्मन॑ आ॒त्मनो॑ लभते लभत आ॒त्मनो ऽना᳚व्रस्का॒या ना᳚व्रस्काया॒त्मनो॑ लभते लभत आ॒त्मनो ऽना᳚व्रस्काय । \newline
20. आ॒त्मनो ऽना᳚व्रस्का॒या ना᳚व्रस्काया॒त्मन॑ आ॒त्मनो ऽना᳚व्रस्काय॒ द्वाभ्या॒म् द्वाभ्या॒ मना᳚व्रस्काया॒त्मन॑ आ॒त्मनो ऽना᳚व्रस्काय॒ द्वाभ्या᳚म् । \newline
21. अना᳚व्रस्काय॒ द्वाभ्या॒म् द्वाभ्या॒ मना᳚व्रस्का॒या ना᳚व्रस्काय॒ द्वाभ्या॑ मु॒पाक॑रो त्यु॒पाक॑रोति॒ द्वाभ्या॒ मना᳚व्रस्का॒याना᳚व्रस्काय॒ द्वाभ्या॑ मु॒पाक॑रोति । \newline
22. अना᳚व्रस्का॒येत्यना᳚ - व्र॒स्का॒य॒ । \newline
23. द्वाभ्या॑ मु॒पाक॑रो त्यु॒पाक॑रोति॒ द्वाभ्या॒म् द्वाभ्या॑ मु॒पाक॑रोति द्वि॒पाद् द्वि॒पा दु॒पाक॑रोति॒ द्वाभ्या॒म् द्वाभ्या॑ मु॒पाक॑रोति द्वि॒पात् । \newline
24. उ॒पाक॑रोति द्वि॒पाद् द्वि॒पा दु॒पाक॑रो त्यु॒पाक॑रोति द्वि॒पाद् यज॑मानो॒ यज॑मानो द्वि॒पा दु॒पाक॑रो त्यु॒पाक॑रोति द्वि॒पाद् यज॑मानः । \newline
25. उ॒पाक॑रा॒तीत्यु॑प - आक॑रोति । \newline
26. द्वि॒पाद् यज॑मानो॒ यज॑मानो द्वि॒पाद् द्वि॒पाद् यज॑मानः॒ प्रति॑ष्ठित्यै॒ प्रति॑ष्ठित्यै॒ यज॑मानो द्वि॒पाद् द्वि॒पाद् यज॑मानः॒ प्रति॑ष्ठित्यै । \newline
27. द्वि॒पादिति॑ द्वि - पात् । \newline
28. यज॑मानः॒ प्रति॑ष्ठित्यै॒ प्रति॑ष्ठित्यै॒ यज॑मानो॒ यज॑मानः॒ प्रति॑ष्ठित्या उपा॒कृ त्यो॑पा॒कृत्य॒ प्रति॑ष्ठित्यै॒ यज॑मानो॒ यज॑मानः॒ प्रति॑ष्ठित्या उपा॒कृत्य॑ । \newline
29. प्रति॑ष्ठित्या उपा॒कृ त्यो॑पा॒कृत्य॒ प्रति॑ष्ठित्यै॒ प्रति॑ष्ठित्या उपा॒कृत्य॒ पञ्च॒ पञ्चो॑पा॒कृत्य॒ प्रति॑ष्ठित्यै॒ प्रति॑ष्ठित्या उपा॒कृत्य॒ पञ्च॑ । \newline
30. प्रति॑ष्ठित्या॒ इति॒ प्रति॑ - स्थि॒त्यै॒ । \newline
31. उ॒पा॒कृत्य॒ पञ्च॒ पञ्चो॑पा॒कृ त्यो॑पा॒कृत्य॒ पञ्च॑ जुहोति जुहोति॒ पञ्चो॑पा॒कृ त्यो॑पा॒कृत्य॒ पञ्च॑ जुहोति । \newline
32. उ॒पा॒कृत्येत्यु॑प - आ॒कृत्य॑ । \newline
33. पञ्च॑ जुहोति जुहोति॒ पञ्च॒ पञ्च॑ जुहोति॒ पाङ्क्ताः॒ पाङ्क्ता॑ जुहोति॒ पञ्च॒ पञ्च॑ जुहोति॒ पाङ्क्ताः᳚ । \newline
34. जु॒हो॒ति॒ पाङ्क्ताः॒ पाङ्क्ता॑ जुहोति जुहोति॒ पाङ्क्ताः᳚ प॒शवः॑ प॒शवः॒ पाङ्क्ता॑ जुहोति जुहोति॒ पाङ्क्ताः᳚ प॒शवः॑ । \newline
35. पाङ्क्ताः᳚ प॒शवः॑ प॒शवः॒ पाङ्क्ताः॒ पाङ्क्ताः᳚ प॒शवः॑ प॒शून् प॒शून् प॒शवः॒ पाङ्क्ताः॒ पाङ्क्ताः᳚ प॒शवः॑ प॒शून् । \newline
36. प॒शवः॑ प॒शून् प॒शून् प॒शवः॑ प॒शवः॑ प॒शू ने॒वैव प॒शून् प॒शवः॑ प॒शवः॑ प॒शू ने॒व । \newline
37. प॒शू ने॒वैव प॒शून् प॒शू ने॒वावावै॒व प॒शून् प॒शू ने॒वाव॑ । \newline
38. ए॒वावा वै॒वै वाव॑ रुन्धे रु॒न्धे ऽवै॒वैवाव॑ रुन्धे । \newline
39. अव॑ रुन्धे रु॒न्धे ऽवाव॑ रुन्धे मृ॒त्यवे॑ मृ॒त्यवे॑ रु॒न्धे ऽवाव॑ रुन्धे मृ॒त्यवे᳚ । \newline
40. रु॒न्धे॒ मृ॒त्यवे॑ मृ॒त्यवे॑ रुन्धे रुन्धे मृ॒त्यवे॒ वै वै मृ॒त्यवे॑ रुन्धे रुन्धे मृ॒त्यवे॒ वै । \newline
41. मृ॒त्यवे॒ वै वै मृ॒त्यवे॑ मृ॒त्यवे॒ वा ए॒ष ए॒ष वै मृ॒त्यवे॑ मृ॒त्यवे॒ वा ए॒षः । \newline
42. वा ए॒ष ए॒ष वै वा ए॒ष नी॑यते नीयत ए॒ष वै वा ए॒ष नी॑यते । \newline
43. ए॒ष नी॑यते नीयत ए॒ष ए॒ष नी॑यते॒ यद् यन् नी॑यत ए॒ष ए॒ष नी॑यते॒ यत् । \newline
44. नी॒य॒ते॒ यद् यन् नी॑यते नीयते॒ यत् प॒शुः प॒शुर् यन् नी॑यते नीयते॒ यत् प॒शुः । \newline
45. यत् प॒शुः प॒शुर् यद् यत् प॒शु स्तम् तम् प॒शुर् यद् यत् प॒शु स्तम् । \newline
46. प॒शु स्तम् तम् प॒शुः प॒शु स्तं ॅयद् यत् तम् प॒शुः प॒शु स्तं ॅयत् । \newline
47. तं ॅयद् यत् तम् तं ॅयद॑न्वा॒रभे॑ता न्वा॒रभे॑त॒ यत् तम् तं ॅयद॑न्वा॒रभे॑त । \newline
48. यद॑न्वा॒रभे॑ता न्वा॒रभे॑त॒ यद् यद॑न्वा॒रभे॑त प्र॒मायु॑कः प्र॒मायु॑को ऽन्वा॒रभे॑त॒ यद् यद॑न्वा॒रभे॑त प्र॒मायु॑कः । \newline
49. अ॒न्वा॒रभे॑त प्र॒मायु॑कः प्र॒मायु॑को ऽन्वा॒रभे॑ता न्वा॒रभे॑त प्र॒मायु॑को॒ यज॑मानो॒ यज॑मानः प्र॒मायु॑को ऽन्वा॒रभे॑ता न्वा॒रभे॑त प्र॒मायु॑को॒ यज॑मानः । \newline
50. अ॒न्वा॒रभे॒तेत्य॑नु - आ॒रभे॑त । \newline
51. प्र॒मायु॑को॒ यज॑मानो॒ यज॑मानः प्र॒मायु॑कः प्र॒मायु॑को॒ यज॑मानः स्याथ् स्या॒द् यज॑मानः प्र॒मायु॑कः प्र॒मायु॑को॒ यज॑मानः स्यात् । \newline
52. प्र॒मायु॑क॒ इति॑ प्र - मायु॑कः । \newline
53. यज॑मानः स्याथ् स्या॒द् यज॑मानो॒ यज॑मानः स्या॒न् नाना॒ नाना᳚ स्या॒द् यज॑मानो॒ यज॑मानः स्या॒न् नाना᳚ । \newline
54. स्या॒न् नाना॒ नाना᳚ स्याथ् स्या॒न् नाना᳚ प्रा॒णः प्रा॒णो नाना᳚ स्याथ् स्या॒न् नाना᳚ प्रा॒णः । \newline
55. नाना᳚ प्रा॒णः प्रा॒णो नाना॒ नाना᳚ प्रा॒णो यज॑मानस्य॒ यज॑मानस्य प्रा॒णो नाना॒ नाना᳚ प्रा॒णो यज॑मानस्य । \newline
56. प्रा॒णो यज॑मानस्य॒ यज॑मानस्य प्रा॒णः प्रा॒णो यज॑मानस्य प॒शुना॑ प॒शुना॒ यज॑मानस्य प्रा॒णः प्रा॒णो यज॑मानस्य प॒शुना᳚ । \newline
57. प्रा॒ण इति॑ प्र - अ॒नः । \newline
58. यज॑मानस्य प॒शुना॑ प॒शुना॒ यज॑मानस्य॒ यज॑मानस्य प॒शुनेतीति॑ प॒शुना॒ यज॑मानस्य॒ यज॑मानस्य प॒शुनेति॑ । \newline
59. प॒शुनेतीति॑ प॒शुना॑ प॒शुने त्या॑हा॒हे ति॑ प॒शुना॑ प॒शुने त्या॑ह । \newline
60. इत्या॑हा॒हे तीत्या॑ह॒ व्यावृ॑त्त्यै॒ व्यावृ॑त्त्या आ॒हे तीत्या॑ह॒ व्यावृ॑त्त्यै । \newline
61. आ॒ह॒ व्यावृ॑त्त्यै॒ व्यावृ॑त्त्या आहाह॒ व्यावृ॑त्त्यै॒ यद् यद् व्यावृ॑त्त्या आहाह॒ व्यावृ॑त्त्यै॒ यत् । \newline
62. व्यावृ॑त्त्यै॒ यद् यद् व्यावृ॑त्त्यै॒ व्यावृ॑त्त्यै॒ यत् प॒शुः प॒शुर् यद् व्यावृ॑त्त्यै॒ व्यावृ॑त्त्यै॒ यत् प॒शुः । \newline
63. व्यावृ॑त्त्या॒ इति॑ वि - आवृ॑त्त्यै । \newline
\pagebreak
\markright{ TS 3.1.5.2  \hfill https://www.vedavms.in \hfill}

\section{ TS 3.1.5.2 }

\textbf{TS 3.1.5.2 } \newline
\textbf{Samhita Paata} \newline

यत् प॒शुर्मा॒युमकृ॒तेति॑ जुहोति॒ शान्त्यै॒ शमि॑तार उ॒पेत॒नेत्या॑ह यथाय॒जुरे॒वैतद्व॒पायां॒ ॅवा आ᳚ह्रि॒यमा॑णाया-म॒ग्नेर्मेधोऽप॑ क्रामति॒ त्वामु॒ ते द॑धिरे हव्य॒वाह॒मिति॑ व॒पाम॒भि जु॑होत्य॒ग्नेरे॒व मेध॒मव॑ रु॒न्धेऽथो॑ शृत॒त्वाय॑ पु॒रस्ता᳚थ् स्वाहा कृतयो॒ वा अ॒न्ये दे॒वा उ॒परि॑ष्टाथ् स्वाहाकृतयो॒ऽन्ये स्वाहा॑ दे॒वेभ्यो॑ दे॒वेभ्यः॒ स्वाहेत्य॒भितो॑ व॒पां ( ) जु॑होति॒ ताने॒वोभया᳚न् प्रीणाति ॥ \newline

\textbf{Pada Paata} \newline

यत् । प॒शुः । मा॒युम् । अकृ॑त । इति॑ । जु॒हो॒ति॒ । शान्त्यै᳚ । शमि॑तारः । उ॒पेत॒नेत्यु॑प - एत॑न । इति॑ । आ॒ह॒ । य॒था॒य॒जुरिति॑ यथा - य॒जुः । ए॒व । ए॒तत् । व॒पाया᳚म् । वै । आ॒ह्रि॒यमा॑णाया॒मित्या᳚ - ह्रि॒यमा॑णायाम् । अ॒ग्नेः । मेधः॑ । अपेति॑ । क्रा॒म॒ति॒ । त्वाम् । उ॒ । ते । द॒धि॒रे॒ । ह॒व्य॒वाह॒मिति॑ हव्य - वाह᳚म् । इति॑ । व॒पाम् । अ॒भीति॑ । जु॒हो॒ति॒ । अ॒ग्नेः । ए॒व । मेध᳚म् । अवेति॑ । रु॒न्धे॒ । अथो॒ इति॑ । शृ॒त॒त्वायेति॑ शृत - त्वाय॑ । पु॒रस्ता᳚थ्स्वाहाकृतय॒ इति॑ पु॒रस्ता᳚त् - स्वा॒हा॒कृ॒त॒यः॒ । वै । अ॒न्ये । दे॒वाः । उ॒परि॑ष्टाथ्स्वाहाकृतय॒ इत्यु॒परि॑ष्टात् - स्वा॒हा॒कृ॒त॒यः॒ । अ॒न्ये । स्वाहा᳚ । दे॒वेभ्यः॑ । दे॒वेभ्यः॑ । स्वाहा᳚ । इति॑ । अ॒भितः॑ । व॒पाम् ( ) । जु॒हो॒ति॒ । तान् । ए॒व । उ॒भयान्॑ । प्री॒णा॒ति॒ ॥  \newline


\textbf{Krama Paata} \newline

यत् प॒शुः । प॒शुर् मा॒युम् । मा॒युमकृ॑त । अकृ॒तेति॑ । इति॑ जुहोति । जु॒हो॒ति॒ शान्त्यै᳚ । शान्त्यै॒ शमि॑तारः । शमि॑तार उ॒पेत॑न । उ॒पेत॒नेति॑ । उ॒पेत॒नेत्यु॑प - एत॑न । इत्या॑ह । आ॒ह॒ य॒था॒य॒जुः । य॒था॒य॒जुरे॒व । य॒था॒य॒जुरिति॑ यथा - य॒जुः । ए॒वैतत् । ए॒तद् व॒पाया᳚म् । व॒पायां॒ ॅवै । वा आ᳚ह्रि॒यमा॑णायाम् । आ॒ह्रि॒यमा॑णायाम॒ग्नेः । आ॒ह्रि॒यमा॑णाया॒मित्या᳚ - ह्रि॒यमा॑णायाम् । अ॒ग्नेर् मेधः॑ । मेधोऽप॑ । अप॑ क्रामति । क्रा॒म॒ति॒ त्वाम् । त्वामु॑ । उ॒ ते । ते द॑धिरे । द॒धि॒रे॒ ह॒व्य॒वाह᳚म् । ह॒व्य॒वाह॒मिति॑ । ह॒व्य॒वाह॒मिति॑ हव्य - वाह᳚म् । इति॑ व॒पाम् । व॒पाम॒भि । अ॒भि जु॑होति । जु॒हो॒त्य॒ग्नेः । अ॒ग्नेरे॒व । ए॒व मेध᳚म् । मेध॒मव॑ । अव॑ रुन्धे । रु॒न्धेऽथो᳚ । अथो॑ शृत॒त्वाय॑ । अथो॒ इत्यथो᳚ । शृ॒त॒त्वाय॑ पु॒रस्ता᳚थ्,स्वाहाकृतयः । शृ॒त॒त्वायेति॑ शृत - त्वाय॑ । पु॒रस्ता᳚थ् स्वाहाकृतयो॒ वै । पु॒रस्ता᳚थ्,स्वाहाकृतय॒ इति॑ पु॒रस्ता᳚त् - स्वा॒हा॒कृ॒त॒यः॒ । वा अ॒न्ये । अ॒न्ये दे॒वाः । दे॒वा उ॒परि॑ष्टाथ्स्वाहाकृतयः । उ॒परि॑ष्टाथ्स्वाहाकृतयो॒ऽन्ये । उ॒परि॑ष्टाथ्स्वाहाकृतय॒ इत्यु॒परि॑ष्टात् - स्वा॒हा॒कृ॒त॒यः॒ । अ॒न्ये स्वाहा᳚ । स्वाहा॑ दे॒वेभ्यः॑ । दे॒वेभ्यो॑ दे॒वेभ्यः॑ । दे॒वेभ्यः॒ स्वाहा᳚ । स्वाहेति॑ । इत्य॒भितः॑ । अ॒भितो॑ व॒पाम् ( ) । व॒पाम् जु॑होति । जु॒हो॒ति॒ तान् । ताने॒व । ए॒वोभयान्॑ । उ॒भया᳚न् प्रीणाति । प्री॒णा॒तीति॑ प्रीणाति । \newline

\textbf{Jatai Paata} \newline

1. यत् प॒शुः प॒शुर् यद् यत् प॒शुः । \newline
2. प॒शुर् मा॒युम् मा॒युम् प॒शुः प॒शुर् मा॒युम् । \newline
3. मा॒यु मकृ॒ता कृ॑त मा॒युम् मा॒यु मकृ॑त । \newline
4. अकृ॒ते तीत्यकृ॒ता कृ॒ते ति॑ । \newline
5. इति॑ जुहोति जुहो॒तीतीति॑ जुहोति । \newline
6. जु॒हो॒ति॒ शान्त्यै॒ शान्त्यै॑ जुहोति जुहोति॒ शान्त्यै᳚ । \newline
7. शान्त्यै॒ शमि॑तारः॒ शमि॑तारः॒ शान्त्यै॒ शान्त्यै॒ शमि॑तारः । \newline
8. शमि॑तार उ॒पेत॑ नो॒पेत॑न॒ शमि॑तारः॒ शमि॑तार उ॒पेत॑न । \newline
9. उ॒पेत॒ने तीत्यु॒पेत॑ नो॒पेत॒ने ति॑ । \newline
10. उ॒पेत॒नेत्यु॑प - एत॑न । \newline
11. इत्या॑हा॒हे तीत्या॑ह । \newline
12. आ॒ह॒ य॒था॒य॒जुर् य॑थाय॒जु रा॑हाह यथाय॒जुः । \newline
13. य॒था॒य॒जु रे॒वैव य॑थाय॒जुर् य॑थाय॒जु रे॒व । \newline
14. य॒था॒य॒जुरिति॑ यथा - य॒जुः । \newline
15. ए॒वैत दे॒त दे॒वै वैतत् । \newline
16. ए॒तद् व॒पायां᳚ ॅव॒पाया॑ मे॒त दे॒तद् व॒पाया᳚म् । \newline
17. व॒पायां॒ ॅवै वै व॒पायां᳚ ॅव॒पायां॒ ॅवै । \newline
18. वा आ᳚ह्रि॒यमा॑णाया माह्रि॒यमा॑णायां॒ ॅवै वा आ᳚ह्रि॒यमा॑णायाम् । \newline
19. आ॒ह्रि॒यमा॑णाया म॒ग्ने र॒ग्ने रा᳚ह्रि॒यमा॑णाया माह्रि॒यमा॑णाया म॒ग्नेः । \newline
20. आ॒ह्रि॒यमा॑णाया॒मित्या᳚ - ह्रि॒यमा॑णायाम् । \newline
21. अ॒ग्नेर् मेधो॒ मेधो॒ ऽग्ने र॒ग्नेर् मेधः॑ । \newline
22. मेधो ऽपाप॒ मेधो॒ मेधो ऽप॑ । \newline
23. अप॑ क्रामति क्राम॒ त्यपाप॑ क्रामति । \newline
24. क्रा॒म॒ति॒ त्वाम् त्वाम् क्रा॑मति क्रामति॒ त्वाम् । \newline
25. त्वा मु॑ वु॒ त्वाम् त्वा मु॑ । \newline
26. उ॒ ते त उ॑ वु॒ ते । \newline
27. ते द॑धिरे दधिरे॒ ते ते द॑धिरे । \newline
28. द॒धि॒रे॒ ह॒व्य॒वाहꣳ॑ हव्य॒वाह॑म् दधिरे दधिरे हव्य॒वाह᳚म् । \newline
29. ह॒व्य॒वाह॒ मितीति॑ हव्य॒वाहꣳ॑ हव्य॒वाह॒ मिति॑ । \newline
30. ह॒व्य॒वाह॒मिति॑ हव्य - वाह᳚म् । \newline
31. इति॑ व॒पां ॅव॒पा मितीति॑ व॒पाम् । \newline
32. व॒पा म॒भ्य॑भि व॒पां ॅव॒पा म॒भि । \newline
33. अ॒भि जु॑होति जुहो त्य॒भ्य॑भि जु॑होति । \newline
34. जु॒हो॒ त्य॒ग्ने र॒ग्नेर् जु॑होति जुहो त्य॒ग्नेः । \newline
35. अ॒ग्ने रे॒वैवाग्ने र॒ग्ने रे॒व । \newline
36. ए॒व मेध॒म् मेध॑ मे॒वैव मेध᳚म् । \newline
37. मेध॒ मवाव॒ मेध॒म् मेध॒ मव॑ । \newline
38. अव॑ रुन्धे रु॒न्धे ऽवाव॑ रुन्धे । \newline
39. रु॒न्धे ऽथो॒ अथो॑ रुन्धे रु॒न्धे ऽथो᳚ । \newline
40. अथो॑ शृत॒त्वाय॑ शृत॒त्वाया थो॒ अथो॑ शृत॒त्वाय॑ । \newline
41. अथो॒ इत्यथो᳚ । \newline
42. शृ॒त॒त्वाय॑ पु॒रस्ता᳚थ्स्वाहाकृतयः पु॒रस्ता᳚थ्स्वाहाकृतयः शृत॒त्वाय॑ शृत॒त्वाय॑ पु॒रस्ता᳚थ्स्वाहाकृतयः । \newline
43. शृ॒त॒त्वायेति॑ शृत - त्वाय॑ । \newline
44. पु॒रस्ता᳚थ्स्वाहाकृतयो॒ वै वै पु॒रस्ता᳚थ्स्वाहाकृतयः पु॒रस्ता᳚थ्स्वाहाकृतयो॒ वै । \newline
45. पु॒रस्ता᳚थ्स्वाहाकृतय॒ इति॑ पु॒रस्ता᳚त् - स्वा॒हा॒कृ॒त॒यः॒ । \newline
46. वा अ॒न्ये᳚ ऽन्ये वै वा अ॒न्ये । \newline
47. अ॒न्ये दे॒वा दे॒वा अ॒न्ये᳚ ऽन्ये दे॒वाः । \newline
48. दे॒वा उ॒परि॑ष्टाथ्स्वाहाकृतय उ॒परि॑ष्टाथ्स्वाहाकृतयो दे॒वा दे॒वा उ॒परि॑ष्टाथ्स्वाहाकृतयः । \newline
49. उ॒परि॑ष्टाथ्स्वाहाकृतयो॒ ऽन्ये᳚ ऽन्य उ॒परि॑ष्टाथ्स्वाहाकृतय उ॒परि॑ष्टाथ्स्वाहाकृतयो॒ ऽन्ये । \newline
50. उ॒परि॑ष्टाथ्स्वाहाकृतय॒ इत्यु॒परि॑ष्टात् - स्वा॒हा॒कृ॒त॒यः॒ । \newline
51. अ॒न्ये स्वाहा॒ स्वाहा॒ ऽन्ये᳚ ऽन्ये स्वाहा᳚ । \newline
52. स्वाहा॑ दे॒वेभ्यो॑ दे॒वेभ्यः॒ स्वाहा॒ स्वाहा॑ दे॒वेभ्यः॑ । \newline
53. दे॒वेभ्यो॑ दे॒वेभ्यः॑ । \newline
54. दे॒वेभ्यः॒ स्वाहा॒ स्वाहा॑ दे॒वेभ्यो॑ दे॒वेभ्यः॒ स्वाहा᳚ । \newline
55. स्वाहेतीति॒ स्वाहा॒ स्वाहेति॑ । \newline
56. इत्य॒भितो॒ ऽभित॒ इती त्य॒भितः॑ । \newline
57. अ॒भितो॑ व॒पां ॅव॒पा म॒भितो॒ ऽभितो॑ व॒पाम् । \newline
58. व॒पाम् जु॑होति जुहोति व॒पां ॅव॒पाम् जु॑होति । \newline
59. जु॒हो॒ति॒ ताꣳ स्तान् जु॑होति जुहोति॒ तान् । \newline
60. ता ने॒वैव ताꣳ स्ता ने॒व । \newline
61. ए॒वोभया॑ नु॒भया॑ ने॒वै वोभयान्॑ । \newline
62. उ॒भया᳚न् प्रीणाति प्रीणा त्यु॒भया॑ नु॒भया᳚न् प्रीणाति । \newline
63. प्री॒णा॒तीति॑ प्रीणाति । \newline

\textbf{Ghana Paata } \newline

1. यत् प॒शुः प॒शुर् यद् यत् प॒शुर् मा॒युम् मा॒युम् प॒शुर् यद् यत् प॒शुर् मा॒युम् । \newline
2. प॒शुर् मा॒युम् मा॒युम् प॒शुः प॒शुर् मा॒यु मकृ॒ता कृ॑त मा॒युम् प॒शुः प॒शुर् मा॒यु मकृ॑त । \newline
3. मा॒यु मकृ॒ता कृ॑त मा॒युम् मा॒यु मकृ॒ते तीत्यकृ॑त मा॒युम् मा॒यु मकृ॒ते ति॑ । \newline
4. अकृ॒ते तीत्यकृ॒ता कृ॒ते ति॑ जुहोति जुहो॒ती त्यकृ॒ता कृ॒ते ति॑ जुहोति । \newline
5. इति॑ जुहोति जुहो॒तीतीति॑ जुहोति॒ शान्त्यै॒ शान्त्यै॑ जुहो॒तीतीति॑ जुहोति॒ शान्त्यै᳚ । \newline
6. जु॒हो॒ति॒ शान्त्यै॒ शान्त्यै॑ जुहोति जुहोति॒ शान्त्यै॒ शमि॑तारः॒ शमि॑तारः॒ शान्त्यै॑ जुहोति जुहोति॒ शान्त्यै॒ शमि॑तारः । \newline
7. शान्त्यै॒ शमि॑तारः॒ शमि॑तारः॒ शान्त्यै॒ शान्त्यै॒ शमि॑तार उ॒पेत॑ नो॒पेत॑न॒ शमि॑तारः॒ शान्त्यै॒ शान्त्यै॒ शमि॑तार उ॒पेत॑न । \newline
8. शमि॑तार उ॒पेत॑ नो॒पेत॑न॒ शमि॑तारः॒ शमि॑तार उ॒पेत॒ने तीत्यु॒पेत॑न॒ शमि॑तारः॒ शमि॑तार उ॒पेत॒ने ति॑ । \newline
9. उ॒पेत॒ने तीत्यु॒पेत॑ नो॒पेत॒ने त्या॑हा॒हे त्यु॒पेत॑ नो॒पेत॒ने त्या॑ह । \newline
10. उ॒पेत॒नेत्यु॑प - एत॑न । \newline
11. इत्या॑हा॒हे तीत्या॑ह यथाय॒जुर् य॑थाय॒जु रा॒हे तीत्या॑ह यथाय॒जुः । \newline
12. आ॒ह॒ य॒था॒य॒जुर् य॑थाय॒जु रा॑हाह यथाय॒जु रे॒वैव य॑थाय॒जु रा॑हाह यथाय॒जु रे॒व । \newline
13. य॒था॒य॒जु रे॒वैव य॑थाय॒जुर् य॑थाय॒जु रे॒वैत दे॒त दे॒व य॑थाय॒जुर् य॑थाय॒जु रे॒वैतत् । \newline
14. य॒था॒य॒जुरिति॑ यथा - य॒जुः । \newline
15. ए॒वैत दे॒त दे॒वैवैतद् व॒पायां᳚ ॅव॒पाया॑ मे॒त दे॒वैवैतद् व॒पाया᳚म् । \newline
16. ए॒तद् व॒पायां᳚ ॅव॒पाया॑ मे॒त दे॒तद् व॒पायां॒ ॅवै वै व॒पाया॑ मे॒त दे॒तद् व॒पायां॒ ॅवै । \newline
17. व॒पायां॒ ॅवै वै व॒पायां᳚ ॅव॒पायां॒ ॅवा आ᳚ह्रि॒यमा॑णाया माह्रि॒यमा॑णायां॒ ॅवै व॒पायां᳚ ॅव॒पायां॒ ॅवा आ᳚ह्रि॒यमा॑णायाम् । \newline
18. वा आ᳚ह्रि॒यमा॑णाया माह्रि॒यमा॑णायां॒ ॅवै वा आ᳚ह्रि॒यमा॑णाया म॒ग्ने र॒ग्ने रा᳚ह्रि॒यमा॑णायां॒ ॅवै वा आ᳚ह्रि॒यमा॑णाया म॒ग्नेः । \newline
19. आ॒ह्रि॒यमा॑णाया म॒ग्ने र॒ग्ने रा᳚ह्रि॒यमा॑णाया माह्रि॒यमा॑णाया म॒ग्नेर् मेधो॒ मेधो॒ ऽग्ने रा᳚ह्रि॒यमा॑णाया माह्रि॒यमा॑णाया म॒ग्नेर् मेधः॑ । \newline
20. आ॒ह्रि॒यमा॑णाया॒मित्या᳚ - ह्रि॒यमा॑णायाम् । \newline
21. अ॒ग्नेर् मेधो॒ मेधो॒ ऽग्ने र॒ग्नेर् मेधो ऽपाप॒ मेधो॒ ऽग्ने र॒ग्नेर् मेधो ऽप॑ । \newline
22. मेधो ऽपाप॒ मेधो॒ मेधो ऽप॑ क्रामति क्राम॒ त्यप॒ मेधो॒ मेधो ऽप॑ क्रामति । \newline
23. अप॑ क्रामति क्राम॒ त्यपाप॑ क्रामति॒ त्वाम् त्वाम् क्रा॑म॒ त्यपाप॑ क्रामति॒ त्वाम् । \newline
24. क्रा॒म॒ति॒ त्वाम् त्वाम् क्रा॑मति क्रामति॒ त्वा मु॑ वु॒ त्वाम् क्रा॑मति क्रामति॒ त्वा मु॑ । \newline
25. त्वा मु॑ वु॒ त्वाम् त्वा मु॒ ते त उ॒ त्वाम् त्वा मु॒ ते । \newline
26. उ॒ ते त उ॑ वु॒ ते द॑धिरे दधिरे॒ त उ॑ वु॒ ते द॑धिरे । \newline
27. ते द॑धिरे दधिरे॒ ते ते द॑धिरे हव्य॒वाहꣳ॑ हव्य॒वाह॑म् दधिरे॒ ते ते द॑धिरे हव्य॒वाह᳚म् । \newline
28. द॒धि॒रे॒ ह॒व्य॒वाहꣳ॑ हव्य॒वाह॑म् दधिरे दधिरे हव्य॒वाह॒ मितीति॑ हव्य॒वाह॑म् दधिरे दधिरे हव्य॒वाह॒ मिति॑ । \newline
29. ह॒व्य॒वाह॒ मितीति॑ हव्य॒वाहꣳ॑ हव्य॒वाह॒ मिति॑ व॒पां ॅव॒पा मिति॑ हव्य॒वाहꣳ॑ हव्य॒वाह॒ मिति॑ व॒पाम् । \newline
30. ह॒व्य॒वाह॒मिति॑ हव्य - वाह᳚म् । \newline
31. इति॑ व॒पां ॅव॒पा मितीति॑ व॒पा म॒भ्य॑भि व॒पा मितीति॑ व॒पा म॒भि । \newline
32. व॒पा म॒भ्य॑भि व॒पां ॅव॒पा म॒भि जु॑होति जुहो त्य॒भि व॒पां ॅव॒पा म॒भि जु॑होति । \newline
33. अ॒भि जु॑होति जुहो त्य॒भ्य॑भि जु॑हो त्य॒ग्ने र॒ग्नेर् जु॑हो त्य॒भ्य॑भि जु॑हो त्य॒ग्नेः । \newline
34. जु॒हो॒ त्य॒ग्ने र॒ग्नेर् जु॑होति जुहो त्य॒ग्ने रे॒वैवाग्नेर् जु॑होति जुहो त्य॒ग्ने रे॒व । \newline
35. अ॒ग्ने रे॒वैवाग्ने र॒ग्ने रे॒व मेध॒म् मेध॑ मे॒वाग्ने र॒ग्ने रे॒व मेध᳚म् । \newline
36. ए॒व मेध॒म् मेध॑ मे॒वैव मेध॒ मवाव॒ मेध॑ मे॒वैव मेध॒ मव॑ । \newline
37. मेध॒ मवाव॒ मेध॒म् मेध॒ मव॑ रुन्धे रु॒न्धे ऽव॒ मेध॒म् मेध॒ मव॑ रुन्धे । \newline
38. अव॑ रुन्धे रु॒न्धे ऽवाव॑ रु॒न्धे ऽथो॒ अथो॑ रु॒न्धे ऽवाव॑ रु॒न्धे ऽथो᳚ । \newline
39. रु॒न्धे ऽथो॒ अथो॑ रुन्धे रु॒न्धे ऽथो॑ शृत॒त्वाय॑ शृत॒त्वायाथो॑ रुन्धे रु॒न्धे ऽथो॑ शृत॒त्वाय॑ । \newline
40. अथो॑ शृत॒त्वाय॑ शृत॒त्वायाथो॒ अथो॑ शृत॒त्वाय॑ पु॒रस्ता᳚थ्स्वाहाकृतयः पु॒रस्ता᳚थ्स्वाहाकृतयः शृत॒त्वायाथो॒ अथो॑ शृत॒त्वाय॑ पु॒रस्ता᳚थ्स्वाहाकृतयः । \newline
41. अथो॒ इत्यथो᳚ । \newline
42. शृ॒त॒त्वाय॑ पु॒रस्ता᳚थ्स्वाहाकृतयः पु॒रस्ता᳚थ्स्वाहाकृतयः शृत॒त्वाय॑ शृत॒त्वाय॑ पु॒रस्ता᳚थ्स्वाहाकृतयो॒ वै वै पु॒रस्ता᳚थ्स्वाहाकृतयः शृत॒त्वाय॑ शृत॒त्वाय॑ पु॒रस्ता᳚थ्स्वाहाकृतयो॒ वै । \newline
43. शृ॒त॒त्वायेति॑ शृत - त्वाय॑ । \newline
44. पु॒रस्ता᳚थ्स्वाहाकृतयो॒ वै वै पु॒रस्ता᳚थ्स्वाहाकृतयः पु॒रस्ता᳚थ्स्वाहाकृतयो॒ वा अ॒न्ये᳚ ऽन्ये वै पु॒रस्ता᳚थ्स्वाहाकृतयः पु॒रस्ता᳚थ्स्वाहाकृतयो॒ वा अ॒न्ये । \newline
45. पु॒रस्ता᳚थ्स्वाहाकृतय॒ इति॑ पु॒रस्ता᳚त् - स्वा॒हा॒कृ॒त॒यः॒ । \newline
46. वा अ॒न्ये᳚ ऽन्ये वै वा अ॒न्ये दे॒वा दे॒वा अ॒न्ये वै वा अ॒न्ये दे॒वाः । \newline
47. अ॒न्ये दे॒वा दे॒वा अ॒न्ये᳚ ऽन्ये दे॒वा उ॒परि॑ष्टाथ्स्वाहाकृतय उ॒परि॑ष्टाथ्स्वाहाकृतयो दे॒वा अ॒न्ये᳚ ऽन्ये दे॒वा उ॒परि॑ष्टाथ्स्वाहाकृतयः । \newline
48. दे॒वा उ॒परि॑ष्टाथ्स्वाहाकृतय उ॒परि॑ष्टाथ्स्वाहाकृतयो दे॒वा दे॒वा उ॒परि॑ष्टाथ्स्वाहाकृतयो॒ ऽन्ये᳚ ऽन्य उ॒परि॑ष्टाथ्स्वाहाकृतयो दे॒वा दे॒वा उ॒परि॑ष्टाथ्स्वाहाकृतयो॒ ऽन्ये । \newline
49. उ॒परि॑ष्टाथ्स्वाहाकृतयो॒ ऽन्ये᳚ ऽन्य उ॒परि॑ष्टाथ्स्वाहाकृतय उ॒परि॑ष्टाथ्स्वाहाकृतयो॒ ऽन्ये स्वाहा॒ स्वाहा॒ ऽन्य उ॒परि॑ष्टाथ्स्वाहाकृतय उ॒परि॑ष्टाथ्स्वाहाकृतयो॒ ऽन्ये स्वाहा᳚ । \newline
50. उ॒परि॑ष्टाथ्स्वाहाकृतय॒ इत्यु॒परि॑ष्टात् - स्वा॒हा॒कृ॒त॒यः॒ । \newline
51. अ॒न्ये स्वाहा॒ स्वाहा॒ ऽन्ये᳚ ऽन्ये स्वाहा॑ दे॒वेभ्यो॑ दे॒वेभ्यः॒ स्वाहा॒ ऽन्ये᳚ ऽन्ये स्वाहा॑ दे॒वेभ्यः॑ । \newline
52. स्वाहा॑ दे॒वेभ्यो॑ दे॒वेभ्यः॒ स्वाहा॒ स्वाहा॑ दे॒वेभ्यः॑ । \newline
53. दे॒वेभ्यो॑ दे॒वेभ्यः॑ । \newline
54. दे॒वेभ्यः॒ स्वाहा॒ स्वाहा॑ दे॒वेभ्यो॑ दे॒वेभ्यः॒ स्वाहेतीति॒ स्वाहा॑ दे॒वेभ्यो॑ दे॒वेभ्यः॒ स्वाहेति॑ । \newline
55. स्वाहेतीति॒ स्वाहा॒ स्वा हेत्य॒भितो॒ ऽभित॒ इति॒ स्वाहा॒ स्वा हेत्य॒भितः॑ । \newline
56. इत्य॒भितो॒ ऽभित॒ इतीत्य॒भितो॑ व॒पां ॅव॒पा म॒भित॒ इतीत्य॒भितो॑ व॒पाम् । \newline
57. अ॒भितो॑ व॒पां ॅव॒पा म॒भितो॒ ऽभितो॑ व॒पाम् जु॑होति जुहोति व॒पा म॒भितो॒ ऽभितो॑ व॒पाम् जु॑होति । \newline
58. व॒पाम् जु॑होति जुहोति व॒पां ॅव॒पाम् जु॑होति॒ ताꣳ स्तान् जु॑होति व॒पां ॅव॒पाम् जु॑होति॒ तान् । \newline
59. जु॒हो॒ति॒ ताꣳ स्तान् जु॑होति जुहोति॒ ता ने॒वैव तान् जु॑होति जुहोति॒ ता ने॒व । \newline
60. ता ने॒वैव ताꣳ स्ता ने॒वोभया॑ नु॒भया॑ ने॒व ताꣳ स्ता ने॒वोभयान्॑ । \newline
61. ए॒वोभया॑ नु॒भया॑ ने॒वैवोभया᳚न् प्रीणाति प्रीणा त्यु॒भया॑ ने॒वैवोभया᳚न् प्रीणाति । \newline
62. उ॒भया᳚न् प्रीणाति प्रीणा त्यु॒भया॑ नु॒भया᳚न् प्रीणाति । \newline
63. प्री॒णा॒तीति॑ प्रीणाति । \newline
\pagebreak
\markright{ TS 3.1.6.1  \hfill https://www.vedavms.in \hfill}

\section{ TS 3.1.6.1 }

\textbf{TS 3.1.6.1 } \newline
\textbf{Samhita Paata} \newline

यो वा अय॑थादेवतं ॅय॒ज्ञ्मु॑प॒चर॒त्या दे॒वता᳚भ्यो वृश्च्यते॒ पापी॑यान् भवति॒ यो य॑थादेव॒तन्न दे॒वता᳚भ्य॒ आ वृ॑श्च्यते॒ वसी॑यान् भवत्याग्ने॒य्यर्चा ऽऽग्नी᳚द्ध्रम॒भि मृ॑शेद् वैष्ण॒व्या ह॑वि॒र्द्धान॑माग्ने॒य्या स्रुचो॑ वाय॒व्य॑या वाय॒व्या᳚न्यैन्द्रि॒या सदो॑ यथादेव॒तमे॒व य॒ज्ञ्मुप॑ चरति॒ न दे॒वता᳚भ्य॒ आ वृ॑श्च्यते॒ वसी॑यान् भवति यु॒नज्मि॑ ते पृथि॒वीं ज्योति॑षा स॒ह यु॒नज्मि॑ वा॒युम॒न्तरि॑क्षेण - [  ] \newline

\textbf{Pada Paata} \newline

यः । वै । अय॑थादेवत॒मित्यय॑था - दे॒व॒त॒म् । य॒ज्ञ्म् । उ॒प॒चर॒तीत्यु॑प - चर॑ति । एति॑ । दे॒वता᳚भ्यः । वृ॒श्च्य॒ते॒ । पापी॑यान् । भ॒व॒ति॒ । यः । य॒था॒दे॒व॒तमिति॑ यथा-दे॒व॒तम् । न । दे॒वता᳚भ्यः । एति॑ । वृ॒श्च्य॒ते॒ । वसी॑यान् । भ॒व॒ति॒ । आ॒ग्ने॒य्या । ऋ॒चा । आग्नी᳚द्ध्र॒मित्याग्नि॑ - इ॒द्ध्र॒म् । अ॒भीति॑ । मृ॒शे॒त् । वै॒ष्ण॒व्या । ह॒वि॒द्‌र्धान॒मिति॑ हविः - धान᳚म् । आ॒ग्ने॒य्या । स्रुचः॑ । वा॒य॒व्य॑या । वा॒य॒व्या॑नि । ऐ॒न्द्रि॒या । सदः॑ । य॒था॒दे॒व॒तमिति॑ यथा-दे॒व॒तम् । ए॒व । य॒ज्ञ्म् । उपेति॑ । च॒र॒ति॒ । न । दे॒वता᳚भ्यः । एति॑ । वृ॒श्च्य॒ते॒ । वसी॑यान् । भ॒व॒ति॒ । यु॒नज्मि॑ । ते॒ । पृ॒थि॒वीम् । ज्योति॑षा । स॒ह । यु॒नज्मि॑ । वा॒युम् । अ॒न्तरि॑क्षेण ।  \newline


\textbf{Krama Paata} \newline

यो वै । वा अय॑थादेवतम् । अय॑थादेवतं ॅय॒ज्ञ्म् । अय॑थादेवत॒मित्यय॑था - दे॒व॒त॒म् । य॒ज्ञ्मु॑प॒चर॑ति । उ॒प॒चर॒त्या । उ॒प॒चर॒तीत्यु॑प - चर॑ति । आ दे॒वता᳚भ्यः । दे॒वता᳚भ्यो वृश्च्यते । वृ॒श्च्य॒ते॒ पापी॑यान् । पापी॑यान् भवति । भ॒व॒ति॒ यः । यो य॑थादेव॒तम् । य॒था॒दे॒व॒तम् न । य॒था॒दे॒व॒तमिति॑ यथा - दे॒व॒तम् । न दे॒वता᳚भ्यः । दे॒वता᳚भ्य॒ आ । आ वृ॑श्च्यते । वृ॒श्च्य॒ते॒ वसी॑यान् । वसी॑यान् भवति । भ॒व॒त्या॒ग्ने॒य्या । आ॒ग्ने॒य्यर्चा । ऋ॒चाऽऽग्नी᳚ध्रम् । आग्नी᳚ध्रम॒भि । आग्नी᳚ध्र॒मित्याग्नि॑ - इ॒ध्र॒म् । अ॒भि मृ॑शेत् । मृ॒शे॒द् वै॒ष्ण॒व्या । वै॒ष्ण॒व्या ह॑वि॒र्द्धान᳚म् । ह॒वि॒र्द्धान॑माग्ने॒य्या । ह॒वि॒र्द्धान॒मिति॑ हविः - धान᳚म् । आ॒ग्ने॒य्या स्रुचः॑ । स्रुचो॑ वाय॒व्य॑या । वा॒य॒व्य॑या वाय॒व्या॑नि । वा॒य॒व्या᳚न्यैन्द्रि॒या । ऐ॒न्द्रि॒या सदः॑ । सदो॑ यथादेव॒तम् । य॒था॒दे॒व॒तमे॒व । य॒था॒दे॒व॒तमिति॑ यथा - दे॒व॒तम् । ए॒व य॒ज्ञ्म् । य॒ज्ञ्मुप॑ । उप॑ चरति । च॒र॒ति॒ न । न दे॒वता᳚भ्यः । दे॒वता᳚भ्य॒ आ । आ वृ॑श्च्यते । वृ॒श्च्य॒ते॒ वसी॑यान् । वसी॑यान् भवति । भ॒व॒ति॒ यु॒नज्मि॑ । यु॒नज्मि॑ ते । ते॒ पृ॒थि॒वीम् । पृ॒थि॒वीम् ज्योति॑षा । ज्योति॑षा स॒ह । स॒ह यु॒नज्मि॑ । यु॒नज्मि॑ वा॒युम् । वा॒युम॒न्तरि॑क्षेण । अ॒न्तरि॑क्षेण ते \newline

\textbf{Jatai Paata} \newline

1. यो वै वै यो यो वै । \newline
2. वा अय॑थादेवत॒ मय॑थादेवतं॒ ॅवै वा अय॑थादेवतम् । \newline
3. अय॑थादेवतं ॅय॒ज्ञ्ं ॅय॒ज्ञ् मय॑थादेवत॒ मय॑थादेवतं ॅय॒ज्ञ्म् । \newline
4. अय॑थादेवत॒मित्यय॑था - दे॒व॒त॒म् । \newline
5. य॒ज्ञ् मु॑प॒चर॑ त्युप॒चर॑ति य॒ज्ञ्ं ॅय॒ज्ञ् मु॑प॒चर॑ति । \newline
6. उ॒प॒चर॒त् योप॒चर॑ त्युप॒चर॒त्या । \newline
7. उ॒प॒चर॒तीत्यु॑प - चर॑ति । \newline
8. आ दे॒वता᳚भ्यो दे॒वता᳚भ्य॒ आ दे॒वता᳚भ्यः । \newline
9. दे॒वता᳚भ्यो वृश्च्यते वृश्च्यते दे॒वता᳚भ्यो दे॒वता᳚भ्यो वृश्च्यते । \newline
10. वृ॒श्च्य॒ते॒ पापी॑या॒न् पापी॑यान् वृश्च्यते वृश्च्यते॒ पापी॑यान् । \newline
11. पापी॑यान् भवति भवति॒ पापी॑या॒न् पापी॑यान् भवति । \newline
12. भ॒व॒ति॒ यो यो भ॑वति भवति॒ यः । \newline
13. यो य॑थादेव॒तं ॅय॑थादेव॒तं ॅयो यो य॑थादेव॒तम् । \newline
14. य॒था॒दे॒व॒तम् न न य॑थादेव॒तं ॅय॑थादेव॒तम् न । \newline
15. य॒था॒दे॒व॒तमिति॑ यथा - दे॒व॒तम् । \newline
16. न दे॒वता᳚भ्यो दे॒वता᳚भ्यो॒ न न दे॒वता᳚भ्यः । \newline
17. दे॒वता᳚भ्य॒ आ दे॒वता᳚भ्यो दे॒वता᳚भ्य॒ आ । \newline
18. आ वृ॑श्च्यते वृश्च्यत॒ आ वृ॑श्च्यते । \newline
19. वृ॒श्च्य॒ते॒ वसी॑या॒न्॒. वसी॑यान् वृश्च्यते वृश्च्यते॒ वसी॑यान् । \newline
20. वसी॑यान् भवति भवति॒ वसी॑या॒न्॒. वसी॑यान् भवति । \newline
21. भ॒व॒ त्या॒ग्ने॒य्या ऽऽग्ने॒य्या भ॑वति भव त्याग्ने॒य्या । \newline
22. आ॒ग्ने॒य्य र्‌चर्चा ऽऽग्ने॒य्या ऽऽग्ने॒य्य र्‌चा । \newline
23. ऋ॒चा ऽऽग्नी᳚द्ध्र॒ माग्नी᳚द्ध्र मृ॒चर्चा ऽऽग्नी᳚द्ध्रम् । \newline
24. आग्नी᳚द्ध्र म॒भ्य॑ भ्याग्नी᳚द्ध्र॒ माग्नी᳚द्ध्र म॒भि । \newline
25. आग्नी᳚द्ध्र॒मित्याग्नि॑ - इ॒द्ध्र॒म् । \newline
26. अ॒भि मृ॑शेन् मृशे द॒भ्य॑भि मृ॑शेत् । \newline
27. मृ॒शे॒द् वै॒ष्ण॒व्या वै᳚ष्ण॒व्या मृ॑शेन् मृशेद् वैष्ण॒व्या । \newline
28. वै॒ष्ण॒व्या ह॑वि॒र्द्धानꣳ॑ हवि॒र्द्धानं॑ ॅवैष्ण॒व्या वै᳚ष्ण॒व्या ह॑वि॒र्द्धान᳚म् । \newline
29. ह॒वि॒र्द्धान॑ माग्ने॒य्या ऽऽग्ने॒य्या ह॑वि॒र्द्धानꣳ॑ हवि॒र्द्धान॑ माग्ने॒य्या । \newline
30. ह॒वि॒र्द्धान॒मिति॑ हविः - धान᳚म् । \newline
31. आ॒ग्ने॒य्या स्रुचः॒ स्रुच॑ आग्ने॒य्या ऽऽग्ने॒य्या स्रुचः॑ । \newline
32. स्रुचो॑ वाय॒व्य॑या वाय॒व्य॑या॒ स्रुचः॒ स्रुचो॑ वाय॒व्य॑या । \newline
33. वा॒य॒व्य॑या वाय॒व्या॑नि वाय॒व्या॑नि वाय॒व्य॑या वाय॒व्य॑या वाय॒व्या॑नि । \newline
34. वा॒य॒व्या᳚ न्यैन्द्रि॒ यैन्द्रि॒या वा॑य॒व्या॑नि वाय॒व्या᳚ न्यैन्द्रि॒या । \newline
35. ऐ॒न्द्रि॒या सदः॒ सद॑ ऐन्द्रि॒ यैन्द्रि॒या सदः॑ । \newline
36. सदो॑ यथादेव॒तं ॅय॑थादेव॒तꣳ सदः॒ सदो॑ यथादेव॒तम् । \newline
37. य॒था॒दे॒व॒त मे॒वैव य॑थादेव॒तं ॅय॑थादेव॒त मे॒व । \newline
38. य॒था॒दे॒व॒तमिति॑ यथा - दे॒व॒तम् । \newline
39. ए॒व य॒ज्ञ्ं ॅय॒ज्ञ् मे॒वैव य॒ज्ञ्म् । \newline
40. य॒ज्ञ् मुपोप॑ य॒ज्ञ्ं ॅय॒ज्ञ् मुप॑ । \newline
41. उप॑ चरति चर॒ त्युपोप॑ चरति । \newline
42. च॒र॒ति॒ न न च॑रति चरति॒ न । \newline
43. न दे॒वता᳚भ्यो दे॒वता᳚भ्यो॒ न न दे॒वता᳚भ्यः । \newline
44. दे॒वता᳚भ्य॒ आ दे॒वता᳚भ्यो दे॒वता᳚भ्य॒ आ । \newline
45. आ वृ॑श्च्यते वृश्च्यत॒ आ वृ॑श्च्यते । \newline
46. वृ॒श्च्य॒ते॒ वसी॑या॒न्॒. वसी॑यान् वृश्च्यते वृश्च्यते॒ वसी॑यान् । \newline
47. वसी॑यान् भवति भवति॒ वसी॑या॒न्॒. वसी॑यान् भवति । \newline
48. भ॒व॒ति॒ यु॒नज्मि॑ यु॒नज्मि॑ भवति भवति यु॒नज्मि॑ । \newline
49. यु॒नज्मि॑ ते ते यु॒नज्मि॑ यु॒नज्मि॑ ते । \newline
50. ते॒ पृ॒थि॒वीम् पृ॑थि॒वीम् ते॑ ते पृथि॒वीम् । \newline
51. पृ॒थि॒वीम् ज्योति॑षा॒ ज्योति॑षा पृथि॒वीम् पृ॑थि॒वीम् ज्योति॑षा । \newline
52. ज्योति॑षा स॒ह स॒ह ज्योति॑षा॒ ज्योति॑षा स॒ह । \newline
53. स॒ह यु॒नज्मि॑ यु॒नज्मि॑ स॒ह स॒ह यु॒नज्मि॑ । \newline
54. यु॒नज्मि॑ वा॒युं ॅवा॒युं ॅयु॒नज्मि॑ यु॒नज्मि॑ वा॒युम् । \newline
55. वा॒यु म॒न्तरि॑क्षे णा॒न्तरि॑क्षेण वा॒युं ॅवा॒यु म॒न्तरि॑क्षेण । \newline
56. अ॒न्तरि॑क्षेण ते ते॒ ऽन्तरि॑क्षे णा॒न्तरि॑क्षेण ते । \newline

\textbf{Ghana Paata } \newline

1. यो वै वै यो यो वा अय॑थादेवत॒ मय॑थादेवतं॒ ॅवै यो यो वा अय॑थादेवतम् । \newline
2. वा अय॑थादेवत॒ मय॑थादेवतं॒ ॅवै वा अय॑थादेवतं ॅय॒ज्ञ्ं ॅय॒ज्ञ् मय॑थादेवतं॒ ॅवै वा अय॑थादेवतं ॅय॒ज्ञ्म् । \newline
3. अय॑थादेवतं ॅय॒ज्ञ्ं ॅय॒ज्ञ् मय॑थादेवत॒ मय॑थादेवतं ॅय॒ज्ञ् मु॑प॒चर॑ त्युप॒चर॑ति य॒ज्ञ् मय॑थादेवत॒ मय॑थादेवतं ॅय॒ज्ञ् मु॑प॒चर॑ति । \newline
4. अय॑थादेवत॒मित्यय॑था - दे॒व॒त॒म् । \newline
5. य॒ज्ञ् मु॑प॒चर॑ त्युप॒चर॑ति य॒ज्ञ्ं ॅय॒ज्ञ् मु॑प॒चर॒त् योप॒चर॑ति य॒ज्ञ्ं ॅय॒ज्ञ् मु॑प॒चर॒त्या । \newline
6. उ॒प॒चर॒त् योप॒चर॑ त्युप॒चर॒त्या दे॒वता᳚भ्यो दे॒वता᳚भ्य॒ ओप॒चर॑ त्युप॒चर॒त्या दे॒वता᳚भ्यः । \newline
7. उ॒प॒चर॒तीत्यु॑प - चर॑ति । \newline
8. आ दे॒वता᳚भ्यो दे॒वता᳚भ्य॒ आ दे॒वता᳚भ्यो वृश्च्यते वृश्च्यते दे॒वता᳚भ्य॒ आ दे॒वता᳚भ्यो वृश्च्यते । \newline
9. दे॒वता᳚भ्यो वृश्च्यते वृश्च्यते दे॒वता᳚भ्यो दे॒वता᳚भ्यो वृश्च्यते॒ पापी॑या॒न् पापी॑यान् वृश्च्यते दे॒वता᳚भ्यो दे॒वता᳚भ्यो वृश्च्यते॒ पापी॑यान् । \newline
10. वृ॒श्च्य॒ते॒ पापी॑या॒न् पापी॑यान् वृश्च्यते वृश्च्यते॒ पापी॑यान् भवति भवति॒ पापी॑यान् वृश्च्यते वृश्च्यते॒ पापी॑यान् भवति । \newline
11. पापी॑यान् भवति भवति॒ पापी॑या॒न् पापी॑यान् भवति॒ यो यो भ॑वति॒ पापी॑या॒न् पापी॑यान् भवति॒ यः । \newline
12. भ॒व॒ति॒ यो यो भ॑वति भवति॒ यो य॑थादेव॒तं ॅय॑थादेव॒तं ॅयो भ॑वति भवति॒ यो य॑थादेव॒तम् । \newline
13. यो य॑थादेव॒तं ॅय॑थादेव॒तं ॅयो यो य॑थादेव॒तम् न न य॑थादेव॒तं ॅयो यो य॑थादेव॒तम् न । \newline
14. य॒था॒दे॒व॒तम् न न य॑थादेव॒तं ॅय॑थादेव॒तम् न दे॒वता᳚भ्यो दे॒वता᳚भ्यो॒ न य॑थादेव॒तं ॅय॑थादेव॒तम् न दे॒वता᳚भ्यः । \newline
15. य॒था॒दे॒व॒तमिति॑ यथा - दे॒व॒तम् । \newline
16. न दे॒वता᳚भ्यो दे॒वता᳚भ्यो॒ न न दे॒वता᳚भ्य॒ आ दे॒वता᳚भ्यो॒ न न दे॒वता᳚भ्य॒ आ । \newline
17. दे॒वता᳚भ्य॒ आ दे॒वता᳚भ्यो दे॒वता᳚भ्य॒ आ वृ॑श्च्यते वृश्च्यत॒ आ दे॒वता᳚भ्यो दे॒वता᳚भ्य॒ आ वृ॑श्च्यते । \newline
18. आ वृ॑श्च्यते वृश्च्यत॒ आ वृ॑श्च्यते॒ वसी॑या॒न्॒. वसी॑यान् वृश्च्यत॒ आ वृ॑श्च्यते॒ वसी॑यान् । \newline
19. वृ॒श्च्य॒ते॒ वसी॑या॒न्॒. वसी॑यान् वृश्च्यते वृश्च्यते॒ वसी॑यान् भवति भवति॒ वसी॑यान् वृश्च्यते वृश्च्यते॒ वसी॑यान् भवति । \newline
20. वसी॑यान् भवति भवति॒ वसी॑या॒न्॒. वसी॑यान् भव त्याग्ने॒य्या ऽऽग्ने॒य्या भ॑वति॒ वसी॑या॒न्॒. वसी॑यान् भव त्याग्ने॒य्या । \newline
21. भ॒व॒ त्या॒ग्ने॒य्या ऽऽग्ने॒य्या भ॑वति भव त्याग्ने॒य्य र्‌चर्चा ऽऽग्ने॒य्या भ॑वति भवत्याग्ने॒य्य र्‌चा । \newline
22. आ॒ग्ने॒य्य र्‌चर्चा ऽऽग्ने॒य्या ऽऽग्ने॒य्य र्‌चा ऽऽग्नी᳚द्ध्र॒ माग्नी᳚द्ध्र मृ॒चा ऽऽग्ने॒य्या ऽऽग्ने॒य्य र्‌चा ऽऽग्नी᳚द्ध्रम् । \newline
23. ऋ॒चा ऽऽग्नी᳚द्ध्र॒ माग्नी᳚द्ध्र मृ॒च ‌र्चा ऽऽग्नी᳚द्ध्र म॒भ्य॑भ्याग्नी᳚द्ध्र मृ॒च ‌र्चा ऽऽग्नी᳚द्ध्र म॒भि । \newline
24. आग्नी᳚द्ध्र म॒भ्य॑भ्याग्नी᳚द्ध्र॒ माग्नी᳚द्ध्र म॒भि मृ॑शेन् मृशे द॒भ्याग्नी᳚द्ध्र॒ माग्नी᳚द्ध्र म॒भि मृ॑शेत् । \newline
25. आग्नी᳚द्ध्र॒मित्याग्नि॑ - इ॒द्ध्र॒म् । \newline
26. अ॒भि मृ॑शेन् मृशे द॒भ्य॑भि मृ॑शेद् वैष्ण॒व्या वै᳚ष्ण॒व्या मृ॑शे द॒भ्य॑भि मृ॑शेद् वैष्ण॒व्या । \newline
27. मृ॒शे॒द् वै॒ष्ण॒व्या वै᳚ष्ण॒व्या मृ॑शेन् मृशेद् वैष्ण॒व्या ह॑वि॒र्द्धानꣳ॑ हवि॒र्द्धानं॑ ॅवैष्ण॒व्या मृ॑शेन् मृशेद् वैष्ण॒व्या ह॑वि॒र्द्धान᳚म् । \newline
28. वै॒ष्ण॒व्या ह॑वि॒र्द्धानꣳ॑ हवि॒र्द्धानं॑ ॅवैष्ण॒व्या वै᳚ष्ण॒व्या ह॑वि॒र्द्धान॑ माग्ने॒य्या ऽऽग्ने॒य्या ह॑वि॒र्द्धानं॑ ॅवैष्ण॒व्या वै᳚ष्ण॒व्या ह॑वि॒र्द्धान॑ माग्ने॒य्या । \newline
29. ह॒वि॒र्द्धान॑ माग्ने॒य्या ऽऽग्ने॒य्या ह॑वि॒र्द्धानꣳ॑ हवि॒र्द्धान॑ माग्ने॒य्या स्रुचः॒ स्रुच॑ आग्ने॒य्या ह॑वि॒र्द्धानꣳ॑ हवि॒र्द्धान॑ माग्ने॒य्या स्रुचः॑ । \newline
30. ह॒वि॒र्द्धान॒मिति॑ हविः - धान᳚म् । \newline
31. आ॒ग्ने॒य्या स्रुचः॒ स्रुच॑ आग्ने॒य्या ऽऽग्ने॒य्या स्रुचो॑ वाय॒व्य॑या वाय॒व्य॑या॒ स्रुच॑ आग्ने॒य्या ऽऽग्ने॒य्या स्रुचो॑ वाय॒व्य॑या । \newline
32. स्रुचो॑ वाय॒व्य॑या वाय॒व्य॑या॒ स्रुचः॒ स्रुचो॑ वाय॒व्य॑या वाय॒व्या॑नि वाय॒व्या॑नि वाय॒व्य॑या॒ स्रुचः॒ स्रुचो॑ वाय॒व्य॑या वाय॒व्या॑नि । \newline
33. वा॒य॒व्य॑या वाय॒व्या॑नि वाय॒व्या॑नि वाय॒व्य॑या वाय॒व्य॑या वाय॒व्या᳚ न्यैन्द्रि॒ यैन्द्रि॒या वा॑य॒व्या॑नि वाय॒व्य॑या वाय॒व्य॑या वाय॒व्या᳚ न्यैन्द्रि॒या । \newline
34. वा॒य॒व्या᳚ न्यैन्द्रि॒ यैन्द्रि॒या वा॑य॒व्या॑नि वाय॒व्या᳚ न्यैन्द्रि॒या सदः॒ सद॑ ऐन्द्रि॒या वा॑य॒व्या॑नि वाय॒व्या᳚ न्यैन्द्रि॒या सदः॑ । \newline
35. ऐ॒न्द्रि॒या सदः॒ सद॑ ऐन्द्रि॒ यैन्द्रि॒या सदो॑ यथादेव॒तं ॅय॑थादेव॒तꣳ सद॑ ऐन्द्रि॒ यैन्द्रि॒या सदो॑ यथादेव॒तम् । \newline
36. सदो॑ यथादेव॒तं ॅय॑थादेव॒तꣳ सदः॒ सदो॑ यथादेव॒त मे॒वैव य॑थादेव॒तꣳ सदः॒ सदो॑ यथादेव॒त मे॒व । \newline
37. य॒था॒दे॒व॒त मे॒वैव य॑थादेव॒तं ॅय॑थादेव॒त मे॒व य॒ज्ञ्ं ॅय॒ज्ञ् मे॒व य॑थादेव॒तं ॅय॑थादेव॒त मे॒व य॒ज्ञ्म् । \newline
38. य॒था॒दे॒व॒तमिति॑ यथा - दे॒व॒तम् । \newline
39. ए॒व य॒ज्ञ्ं ॅय॒ज्ञ् मे॒वैव य॒ज्ञ् मुपोप॑ य॒ज्ञ् मे॒वैव य॒ज्ञ् मुप॑ । \newline
40. य॒ज्ञ् मुपोप॑ य॒ज्ञ्ं ॅय॒ज्ञ् मुप॑ चरति चर॒ त्युप॑ य॒ज्ञ्ं ॅय॒ज्ञ् मुप॑ चरति । \newline
41. उप॑ चरति चर॒ त्युपोप॑ चरति॒ न न च॑र॒ त्युपोप॑ चरति॒ न । \newline
42. च॒र॒ति॒ न न च॑रति चरति॒ न दे॒वता᳚भ्यो दे॒वता᳚भ्यो॒ न च॑रति चरति॒ न दे॒वता᳚भ्यः । \newline
43. न दे॒वता᳚भ्यो दे॒वता᳚भ्यो॒ न न दे॒वता᳚भ्य॒ आ दे॒वता᳚भ्यो॒ न न दे॒वता᳚भ्य॒ आ । \newline
44. दे॒वता᳚भ्य॒ आ दे॒वता᳚भ्यो दे॒वता᳚भ्य॒ आ वृ॑श्च्यते वृश्च्यत॒ आ दे॒वता᳚भ्यो दे॒वता᳚भ्य॒ आ वृ॑श्च्यते । \newline
45. आ वृ॑श्च्यते वृश्च्यत॒ आ वृ॑श्च्यते॒ वसी॑या॒न्॒. वसी॑यान् वृश्च्यत॒ आ वृ॑श्च्यते॒ वसी॑यान् । \newline
46. वृ॒श्च्य॒ते॒ वसी॑या॒न्॒. वसी॑यान् वृश्च्यते वृश्च्यते॒ वसी॑यान् भवति भवति॒ वसी॑यान् वृश्च्यते वृश्च्यते॒ वसी॑यान् भवति । \newline
47. वसी॑यान् भवति भवति॒ वसी॑या॒न्॒. वसी॑यान् भवति यु॒नज्मि॑ यु॒नज्मि॑ भवति॒ वसी॑या॒न्॒. वसी॑यान् भवति यु॒नज्मि॑ । \newline
48. भ॒व॒ति॒ यु॒नज्मि॑ यु॒नज्मि॑ भवति भवति यु॒नज्मि॑ ते ते यु॒नज्मि॑ भवति भवति यु॒नज्मि॑ ते । \newline
49. यु॒नज्मि॑ ते ते यु॒नज्मि॑ यु॒नज्मि॑ ते पृथि॒वीम् पृ॑थि॒वीम् ते॑ यु॒नज्मि॑ यु॒नज्मि॑ ते पृथि॒वीम् । \newline
50. ते॒ पृ॒थि॒वीम् पृ॑थि॒वीम् ते॑ ते पृथि॒वीम् ज्योति॑षा॒ ज्योति॑षा पृथि॒वीम् ते॑ ते पृथि॒वीम् ज्योति॑षा । \newline
51. पृ॒थि॒वीम् ज्योति॑षा॒ ज्योति॑षा पृथि॒वीम् पृ॑थि॒वीम् ज्योति॑षा स॒ह स॒ह ज्योति॑षा पृथि॒वीम् पृ॑थि॒वीम् ज्योति॑षा स॒ह । \newline
52. ज्योति॑षा स॒ह स॒ह ज्योति॑षा॒ ज्योति॑षा स॒ह यु॒नज्मि॑ यु॒नज्मि॑ स॒ह ज्योति॑षा॒ ज्योति॑षा स॒ह यु॒नज्मि॑ । \newline
53. स॒ह यु॒नज्मि॑ यु॒नज्मि॑ स॒ह स॒ह यु॒नज्मि॑ वा॒युं ॅवा॒युं ॅयु॒नज्मि॑ स॒ह स॒ह यु॒नज्मि॑ वा॒युम् । \newline
54. यु॒नज्मि॑ वा॒युं ॅवा॒युं ॅयु॒नज्मि॑ यु॒नज्मि॑ वा॒यु म॒न्तरि॑क्षेणा॒ न्तरि॑क्षेण वा॒युं ॅयु॒नज्मि॑ यु॒नज्मि॑ वा॒यु म॒न्तरि॑क्षेण । \newline
55. वा॒यु म॒न्तरि॑क्षेणा॒ न्तरि॑क्षेण वा॒युं ॅवा॒यु म॒न्तरि॑क्षेण ते ते॒ ऽन्तरि॑क्षेण वा॒युं ॅवा॒यु म॒न्तरि॑क्षेण ते । \newline
56. अ॒न्तरि॑क्षेण ते ते॒ ऽन्तरि॑क्षेणा॒ न्तरि॑क्षेण ते स॒ह स॒ह ते॒ ऽन्तरि॑क्षेणा॒ न्तरि॑क्षेण ते स॒ह । \newline
\pagebreak
\markright{ TS 3.1.6.2  \hfill https://www.vedavms.in \hfill}

\section{ TS 3.1.6.2 }

\textbf{TS 3.1.6.2 } \newline
\textbf{Samhita Paata} \newline

ते स॒ह यु॒नज्मि॒ वाचꣳ॑ स॒ह सूर्ये॑ण ते यु॒नज्मि॑ ति॒स्रो वि॒पृचः॒ सूर्य॑स्य ते । अ॒ग्निर्दे॒वता॑ गाय॒त्री छन्द॑ उपाꣳ॒॒शोः पात्र॑मसि॒ सोमो॑ दे॒वता᳚ त्रि॒ष्टुप् छन्दो᳚ऽन्तर्या॒मस्य॒ पात्र॑म॒सीन्द्रो॑ दे॒वता॒ जग॑ती॒ छन्द॑ इन्द्रवायु॒वोः पात्र॑मसि॒ बृह॒स्पति॑र् दे॒वता॑ऽनु॒ष्टुप् छन्दो॑ मि॒त्रावरु॑णयोः॒ पात्र॑मस्य॒श्विनौ॑ दे॒वता॑ प॒ङ्क्तिश्छन्दो॒ऽश्विनोः॒ पात्र॑मसि॒ सूर्यो॑ दे॒वता॑ बृह॒ती - [  ] \newline

\textbf{Pada Paata} \newline

ते॒ । स॒ह । यु॒नज्मि॑ । वाच᳚म् । स॒ह । सूर्ये॑ण । ते॒ । यु॒नज्मि॑ । ति॒स्रः । वि॒पृच॒ इति॑ वि - पृचः॑ । सूर्य॑स्य । ते॒ ॥ अ॒ग्निः । दे॒वता᳚ । गा॒य॒त्री । छन्दः॑ । उ॒पाꣳ॒॒शोरित्यु॑प - अꣳ॒॒शोः । पात्र᳚म् । अ॒सि॒ । सोमः॑ । दे॒वता᳚ । त्रि॒ष्टुप् । छन्दः॑ । अ॒न्त॒र्या॒मस्येत्य॑न्तः - या॒मस्य॑ । पात्र᳚म् । अ॒सि॒ । इन्द्रः॑ । दे॒वता᳚ । जग॑ती । छन्दः॑ । इ॒न्द्र॒वा॒यु॒वोरिती᳚न्द्र-वा॒यु॒वोः । पात्र᳚म् । अ॒सि॒ । बृह॒स्पतिः॑ । दे॒वता᳚ । अ॒नु॒ष्टुबित्य॑नु - स्तुप् । छन्दः॑ । मि॒त्रावरु॑णयो॒रिति॑ मि॒त्रा-वरु॑णयोः । पात्र᳚म् । अ॒सि॒ । अ॒श्विनौ᳚ । दे॒वता᳚ । प॒ङ्क्तिः । छन्दः॑ । अ॒श्विनोः᳚ । पात्र᳚म् । अ॒सि॒ । सूर्यः॑ । दे॒वता᳚ । बृ॒ह॒ती ।  \newline


\textbf{Krama Paata} \newline

ते॒ स॒ह । स॒ह यु॒नज्मि॑ । यु॒नज्मि॒ वाच᳚म् । वाचꣳ॑ स॒ह । स॒ह सूर्ये॑ण । सूर्ये॑ण ते । ते॒ यु॒नज्मि॑ । यु॒नज्मि॑ ति॒स्रः । ति॒स्रो वि॒पृचः॑ । वि॒पृचः॒ सूर्य॑स्य । वि॒पृच॒ इति॑ वि - पृचः॑ । सूर्य॑स्य ते । त॒ इति॑ ते ॥ अ॒ग्निर् दे॒वता᳚ । दे॒वता॑ गाय॒त्री । गा॒य॒त्री छन्दः॑ । छन्द॑ उपाꣳ॒॒शोः । उ॒पाꣳ॒॒शोः पात्र᳚म् । उ॒पाꣳ॒॒शोरित्यु॑प - अꣳ॒॒शोः । पात्र॑मसि । अ॒सि॒ सोमः॑ । सोमो॑ दे॒वता᳚ । दे॒वता᳚ त्रि॒ष्टुप् । त्रि॒ष्टुप् छन्दः॑ । छन्दो᳚ऽन्तर्या॒मस्य॑ । अ॒न्त॒र्या॒मस्य॒ पात्र᳚म् । अ॒न्त॒र्या॒मस्येत्य॑न्तः - या॒मस्य॑ । पात्र॑मसि । अ॒सीन्द्रः॑ । इन्द्रो॑ दे॒वता᳚ । दे॒वता॒ जग॑ती । जग॑ती॒ छन्दः॑ । छन्द॑ इन्द्रवायु॒वोः । इ॒न्द्र॒वा॒यु॒वोः पात्र᳚म् । इ॒न्द्र॒वा॒यु॒वोरिती᳚न्द्र - वा॒यु॒वोः । पात्र॑मसि । अ॒सि॒ बृह॒स्पतिः॑ । बृह॒स्पति॑र् दे॒वता᳚ । दे॒वाता॑ऽनु॒ष्टुप् । अ॒नु॒ष्टुप् छन्दः॑ । अ॒नु॒ष्टुबित्य॑नु - स्तुप् । छन्दो॑ मि॒त्रावरु॑णयोः । मि॒त्रावरु॑णयोः॒ पात्र᳚म् । मि॒त्रावरु॑णयो॒रिति॑ मि॒त्रा - वरु॑णयोः । पात्र॑मसि । अ॒स्य॒श्विनौ᳚ । अ॒श्विनौ॑ दे॒वता᳚ । दे॒वता॑ प॒ङ्क्तिः । प॒ङ्क्ति श्छन्दः॑ । छन्दो॒ऽश्विनोः᳚ । अ॒श्विनोः॒ पात्र᳚म् । पात्र॑मसि । अ॒सि॒ सूर्यः॑ । सूर्यो॑ दे॒वता᳚ । दे॒वता॑ बृह॒ती ( ) । बृ॒ह॒ती छन्दः॑ \newline

\textbf{Jatai Paata} \newline

1. ते॒ स॒ह स॒ह ते॑ ते स॒ह । \newline
2. स॒ह यु॒नज्मि॑ यु॒नज्मि॑ स॒ह स॒ह यु॒नज्मि॑ । \newline
3. यु॒नज्मि॒ वाचं॒ ॅवाचं॑ ॅयु॒नज्मि॑ यु॒नज्मि॒ वाच᳚म् । \newline
4. वाचꣳ॑ स॒ह स॒ह वाचं॒ ॅवाचꣳ॑ स॒ह । \newline
5. स॒ह सूर्ये॑ण॒ सूर्ये॑ण स॒ह स॒ह सूर्ये॑ण । \newline
6. सूर्ये॑ण ते ते॒ सूर्ये॑ण॒ सूर्ये॑ण ते । \newline
7. ते॒ यु॒नज्मि॑ यु॒नज्मि॑ ते ते यु॒नज्मि॑ । \newline
8. यु॒नज्मि॑ ति॒स्र स्ति॒स्रो यु॒नज्मि॑ यु॒नज्मि॑ ति॒स्रः । \newline
9. ति॒स्रो वि॒पृचो॑ वि॒पृच॑ स्ति॒स्र स्ति॒स्रो वि॒पृचः॑ । \newline
10. वि॒पृचः॒ सूर्य॑स्य॒ सूर्य॑स्य वि॒पृचो॑ वि॒पृचः॒ सूर्य॑स्य । \newline
11. वि॒पृच॒ इति॑ वि - पृचः॑ । \newline
12. सूर्य॑स्य ते ते॒ सूर्य॑स्य॒ सूर्य॑स्य ते । \newline
13. त॒ इति॑ ते । \newline
14. अ॒ग्निर् दे॒वता॑ दे॒वता॒ ऽग्नि र॒ग्निर् दे॒वता᳚ । \newline
15. दे॒वता॑ गाय॒त्री गा॑य॒त्री दे॒वता॑ दे॒वता॑ गाय॒त्री । \newline
16. गा॒य॒त्री छन्द॒ श्छन्दो॑ गाय॒त्री गा॑य॒त्री छन्दः॑ । \newline
17. छन्द॑ उपाꣳ॒॒शो रु॑पाꣳ॒॒शो श्छन्द॒ श्छन्द॑ उपाꣳ॒॒शोः । \newline
18. उ॒पाꣳ॒॒शोः पात्र॒म् पात्र॑ मुपाꣳ॒॒शो रु॑पाꣳ॒॒शोः पात्र᳚म् । \newline
19. उ॒पाꣳ॒॒शोरित्यु॑प - अꣳ॒॒शोः । \newline
20. पात्र॑ मस्यसि॒ पात्र॒म् पात्र॑ मसि । \newline
21. अ॒सि॒ सोमः॒ सोमो᳚ ऽस्यसि॒ सोमः॑ । \newline
22. सोमो॑ दे॒वता॑ दे॒वता॒ सोमः॒ सोमो॑ दे॒वता᳚ । \newline
23. दे॒वता᳚ त्रि॒ष्टुप् त्रि॒ष्टुब् दे॒वता॑ दे॒वता᳚ त्रि॒ष्टुप् । \newline
24. त्रि॒ष्टुप् छन्द॒ श्छन्द॑ स्त्रि॒ष्टुप् त्रि॒ष्टुप् छन्दः॑ । \newline
25. छन्दो᳚ ऽन्तर्या॒मस्या᳚ न्तर्या॒मस्य॒ छन्द॒ श्छन्दो᳚ ऽन्तर्या॒मस्य॑ । \newline
26. अ॒न्त॒र्या॒मस्य॒ पात्र॒म् पात्र॑ मन्तर्या॒मस्या᳚ न्तर्या॒मस्य॒ पात्र᳚म् । \newline
27. अ॒न्त॒र्या॒मस्येत्य॑न्तः - या॒मस्य॑ । \newline
28. पात्र॑ मस्यसि॒ पात्र॒म् पात्र॑ मसि । \newline
29. अ॒सीन्द्र॒ इन्द्रो᳚ ऽस्य॒सीन्द्रः॑ । \newline
30. इन्द्रो॑ दे॒वता॑ दे॒वतेन्द्र॒ इन्द्रो॑ दे॒वता᳚ । \newline
31. दे॒वता॒ जग॑ती॒ जग॑ती दे॒वता॑ दे॒वता॒ जग॑ती । \newline
32. जग॑ती॒ छन्द॒ श्छन्दो॒ जग॑ती॒ जग॑ती॒ छन्दः॑ । \newline
33. छन्द॑ इन्द्रवायु॒वो रि॑न्द्रवायु॒वो श्छन्द॒ श्छन्द॑ इन्द्रवायु॒वोः । \newline
34. इ॒न्द्र॒वा॒यु॒वोः पात्र॒म् पात्र॑ मिन्द्रवायु॒वो रि॑न्द्रवायु॒वोः पात्र᳚म् । \newline
35. इ॒न्द्र॒वा॒यु॒वोरिती᳚न्द्र - वा॒यु॒वोः । \newline
36. पात्र॑ मस्यसि॒ पात्र॒म् पात्र॑ मसि । \newline
37. अ॒सि॒ बृह॒स्पति॒र् बृह॒स्पति॑ रस्यसि॒ बृह॒स्पतिः॑ । \newline
38. बृह॒स्पति॑र् दे॒वता॑ दे॒वता॒ बृह॒स्पति॒र् बृह॒स्पति॑र् दे॒वता᳚ । \newline
39. दे॒वता॑ ऽनु॒ष्टु ब॑नु॒ष्टुब् दे॒वता॑ दे॒वता॑ ऽनु॒ष्टुप् । \newline
40. अ॒नु॒ष्टुप् छन्द॒ श्छन्दो॑ ऽनु॒ष्टु ब॑नु॒ष्टुप् छन्दः॑ । \newline
41. अ॒नु॒ष्टुबित्य॑नु - स्तुप् । \newline
42. छन्दो॑ मि॒त्रावरु॑णयोर् मि॒त्रावरु॑णयो॒ श्छन्द॒ श्छन्दो॑ मि॒त्रावरु॑णयोः । \newline
43. मि॒त्रावरु॑णयोः॒ पात्र॒म् पात्र॑म् मि॒त्रावरु॑णयोर् मि॒त्रावरु॑णयोः॒ पात्र᳚म् । \newline
44. मि॒त्रावरु॑णयो॒रिति॑ मि॒त्रा - वरु॑णयोः । \newline
45. पात्र॑ मस्यसि॒ पात्र॒म् पात्र॑ मसि । \newline
46. अ॒स्य॒श्विना॑ व॒श्विना॑ वस्य स्य॒श्विनौ᳚ । \newline
47. अ॒श्विनौ॑ दे॒वता॑ दे॒वता॒ ऽश्विना॑ व॒श्विनौ॑ दे॒वता᳚ । \newline
48. दे॒वता॑ प॒ङ्क्तिः प॒ङ्क्तिर् दे॒वता॑ दे॒वता॑ प॒ङ्क्तिः । \newline
49. प॒ङ्क्ति श्छन्द॒ श्छन्दः॑ प॒ङ्क्तिः प॒ङ्क्ति श्छन्दः॑ । \newline
50. छन्दो॒ ऽश्विनो॑ र॒श्विनो॒ श्छन्द॒ श्छन्दो॒ ऽश्विनोः᳚ । \newline
51. अ॒श्विनोः॒ पात्र॒म् पात्र॑ म॒श्विनो॑ र॒श्विनोः॒ पात्र᳚म् । \newline
52. पात्र॑ मस्यसि॒ पात्र॒म् पात्र॑ मसि । \newline
53. अ॒सि॒ सूर्यः॒ सूर्यो᳚ ऽस्यसि॒ सूर्यः॑ । \newline
54. सूर्यो॑ दे॒वता॑ दे॒वता॒ सूर्यः॒ सूर्यो॑ दे॒वता᳚ । \newline
55. दे॒वता॑ बृह॒ती बृ॑ह॒ती दे॒वता॑ दे॒वता॑ बृह॒ती । \newline
56. बृ॒ह॒ती छन्द॒ श्छन्दो॑ बृह॒ती बृ॑ह॒ती छन्दः॑ । \newline

\textbf{Ghana Paata } \newline

1. ते॒ स॒ह स॒ह ते॑ ते स॒ह यु॒नज्मि॑ यु॒नज्मि॑ स॒ह ते॑ ते स॒ह यु॒नज्मि॑ । \newline
2. स॒ह यु॒नज्मि॑ यु॒नज्मि॑ स॒ह स॒ह यु॒नज्मि॒ वाचं॒ ॅवाचं॑ ॅयु॒नज्मि॑ स॒ह स॒ह यु॒नज्मि॒ वाच᳚म् । \newline
3. यु॒नज्मि॒ वाचं॒ ॅवाचं॑ ॅयु॒नज्मि॑ यु॒नज्मि॒ वाचꣳ॑ स॒ह स॒ह वाचं॑ ॅयु॒नज्मि॑ यु॒नज्मि॒ वाचꣳ॑ स॒ह । \newline
4. वाचꣳ॑ स॒ह स॒ह वाचं॒ ॅवाचꣳ॑ स॒ह सूर्ये॑ण॒ सूर्ये॑ण स॒ह वाचं॒ ॅवाचꣳ॑ स॒ह सूर्ये॑ण । \newline
5. स॒ह सूर्ये॑ण॒ सूर्ये॑ण स॒ह स॒ह सूर्ये॑ण ते ते॒ सूर्ये॑ण स॒ह स॒ह सूर्ये॑ण ते । \newline
6. सूर्ये॑ण ते ते॒ सूर्ये॑ण॒ सूर्ये॑ण ते यु॒नज्मि॑ यु॒नज्मि॑ ते॒ सूर्ये॑ण॒ सूर्ये॑ण ते यु॒नज्मि॑ । \newline
7. ते॒ यु॒नज्मि॑ यु॒नज्मि॑ ते ते यु॒नज्मि॑ ति॒स्र स्ति॒स्रो यु॒नज्मि॑ ते ते यु॒नज्मि॑ ति॒स्रः । \newline
8. यु॒नज्मि॑ ति॒स्र स्ति॒स्रो यु॒नज्मि॑ यु॒नज्मि॑ ति॒स्रो वि॒पृचो॑ वि॒पृच॑ स्ति॒स्रो यु॒नज्मि॑ यु॒नज्मि॑ ति॒स्रो वि॒पृचः॑ । \newline
9. ति॒स्रो वि॒पृचो॑ वि॒पृच॑ स्ति॒स्र स्ति॒स्रो वि॒पृचः॒ सूर्य॑स्य॒ सूर्य॑स्य वि॒पृच॑ स्ति॒स्र स्ति॒स्रो वि॒पृचः॒ सूर्य॑स्य । \newline
10. वि॒पृचः॒ सूर्य॑स्य॒ सूर्य॑स्य वि॒पृचो॑ वि॒पृचः॒ सूर्य॑स्य ते ते॒ सूर्य॑स्य वि॒पृचो॑ वि॒पृचः॒ सूर्य॑स्य ते । \newline
11. वि॒पृच॒ इति॑ वि - पृचः॑ । \newline
12. सूर्य॑स्य ते ते॒ सूर्य॑स्य॒ सूर्य॑स्य ते । \newline
13. त॒ इति॑ ते । \newline
14. अ॒ग्निर् दे॒वता॑ दे॒वता॒ ऽग्नि र॒ग्निर् दे॒वता॑ गाय॒त्री गा॑य॒त्री दे॒वता॒ ऽग्नि र॒ग्निर् दे॒वता॑ गाय॒त्री । \newline
15. दे॒वता॑ गाय॒त्री गा॑य॒त्री दे॒वता॑ दे॒वता॑ गाय॒त्री छन्द॒ श्छन्दो॑ गाय॒त्री दे॒वता॑ दे॒वता॑ गाय॒त्री छन्दः॑ । \newline
16. गा॒य॒त्री छन्द॒ श्छन्दो॑ गाय॒त्री गा॑य॒त्री छन्द॑ उपाꣳ॒॒शो रु॑पाꣳ॒॒शो श्छन्दो॑ गाय॒त्री गा॑य॒त्री छन्द॑ उपाꣳ॒॒शोः । \newline
17. छन्द॑ उपाꣳ॒॒शो रु॑पाꣳ॒॒शो श्छन्द॒ श्छन्द॑ उपाꣳ॒॒शोः पात्र॒म् पात्र॑ मुपाꣳ॒॒शो श्छन्द॒ श्छन्द॑ उपाꣳ॒॒शोः पात्र᳚म् । \newline
18. उ॒पाꣳ॒॒शोः पात्र॒म् पात्र॑ मुपाꣳ॒॒शो रु॑पाꣳ॒॒शोः पात्र॑ मस्यसि॒ पात्र॑ मुपाꣳ॒॒शो रु॑पाꣳ॒॒शोः पात्र॑ मसि । \newline
19. उ॒पाꣳ॒॒शोरित्यु॑प - अꣳ॒॒शोः । \newline
20. पात्र॑ मस्यसि॒ पात्र॒म् पात्र॑ मसि॒ सोमः॒ सोमो॑ ऽसि॒ पात्र॒म् पात्र॑ मसि॒ सोमः॑ । \newline
21. अ॒सि॒ सोमः॒ सोमो᳚ ऽस्यसि॒ सोमो॑ दे॒वता॑ दे॒वता॒ सोमो᳚ ऽस्यसि॒ सोमो॑ दे॒वता᳚ । \newline
22. सोमो॑ दे॒वता॑ दे॒वता॒ सोमः॒ सोमो॑ दे॒वता᳚ त्रि॒ष्टुप् त्रि॒ष्टुब् दे॒वता॒ सोमः॒ सोमो॑ दे॒वता᳚ त्रि॒ष्टुप् । \newline
23. दे॒वता᳚ त्रि॒ष्टुप् त्रि॒ष्टुब् दे॒वता॑ दे॒वता᳚ त्रि॒ष्टुप् छन्द॒ श्छन्द॑ स्त्रि॒ष्टुब् दे॒वता॑ दे॒वता᳚ त्रि॒ष्टुप् छन्दः॑ । \newline
24. त्रि॒ष्टुप् छन्द॒ श्छन्द॑ स्त्रि॒ष्टुप् त्रि॒ष्टुप् छन्दो᳚ ऽन्तर्या॒मस्या᳚ न्तर्या॒मस्य॒ छन्द॑ स्त्रि॒ष्टुप् त्रि॒ष्टुप् छन्दो᳚ ऽन्तर्या॒मस्य॑ । \newline
25. छन्दो᳚ ऽन्तर्या॒मस्या᳚ न्तर्या॒मस्य॒ छन्द॒ श्छन्दो᳚ ऽन्तर्या॒मस्य॒ पात्र॒म् पात्र॑ मन्तर्या॒मस्य॒ छन्द॒ श्छन्दो᳚ ऽन्तर्या॒मस्य॒ पात्र᳚म् । \newline
26. अ॒न्त॒र्या॒मस्य॒ पात्र॒म् पात्र॑ मन्तर्या॒मस्या᳚ न्तर्या॒मस्य॒ पात्र॑ मस्यसि॒ पात्र॑ मन्तर्या॒मस्या᳚ न्तर्या॒मस्य॒ पात्र॑ मसि । \newline
27. अ॒न्त॒र्या॒मस्येत्य॑न्तः - या॒मस्य॑ । \newline
28. पात्र॑ मस्यसि॒ पात्र॒म् पात्र॑ म॒सीन्द्र॒ इन्द्रो॑ ऽसि॒ पात्र॒म् पात्र॑ म॒सीन्द्रः॑ । \newline
29. अ॒सीन्द्र॒ इन्द्रो᳚ ऽस्य॒सीन्द्रो॑ दे॒वता॑ दे॒वतेन्द्रो᳚ ऽस्य॒सीन्द्रो॑ दे॒वता᳚ । \newline
30. इन्द्रो॑ दे॒वता॑ दे॒वतेन्द्र॒ इन्द्रो॑ दे॒वता॒ जग॑ती॒ जग॑ती दे॒वतेन्द्र॒ इन्द्रो॑ दे॒वता॒ जग॑ती । \newline
31. दे॒वता॒ जग॑ती॒ जग॑ती दे॒वता॑ दे॒वता॒ जग॑ती॒ छन्द॒ श्छन्दो॒ जग॑ती दे॒वता॑ दे॒वता॒ जग॑ती॒ छन्दः॑ । \newline
32. जग॑ती॒ छन्द॒ श्छन्दो॒ जग॑ती॒ जग॑ती॒ छन्द॑ इन्द्रवायु॒वो रि॑न्द्रवायु॒वो श्छन्दो॒ जग॑ती॒ जग॑ती॒ छन्द॑ इन्द्रवायु॒वोः । \newline
33. छन्द॑ इन्द्रवायु॒वो रि॑न्द्रवायु॒वो श्छन्द॒ श्छन्द॑ इन्द्रवायु॒वोः पात्र॒म् पात्र॑ मिन्द्रवायु॒वो श्छन्द॒ श्छन्द॑ इन्द्रवायु॒वोः पात्र᳚म् । \newline
34. इ॒न्द्र॒वा॒यु॒वोः पात्र॒म् पात्र॑ मिन्द्रवायु॒वो रि॑न्द्रवायु॒वोः पात्र॑ मस्यसि॒ पात्र॑ मिन्द्रवायु॒वो रि॑न्द्रवायु॒वोः पात्र॑ मसि । \newline
35. इ॒न्द्र॒वा॒यु॒वोरिती᳚न्द्र - वा॒यु॒वोः । \newline
36. पात्र॑ मस्यसि॒ पात्र॒म् पात्र॑ मसि॒ बृह॒स्पति॒र् बृह॒स्पति॑ रसि॒ पात्र॒म् पात्र॑ मसि॒ बृह॒स्पतिः॑ । \newline
37. अ॒सि॒ बृह॒स्पति॒र् बृह॒स्पति॑ रस्यसि॒ बृह॒स्पति॑र् दे॒वता॑ दे॒वता॒ बृह॒स्पति॑ रस्यसि॒ बृह॒स्पति॑र् दे॒वता᳚ । \newline
38. बृह॒स्पति॑र् दे॒वता॑ दे॒वता॒ बृह॒स्पति॒र् बृह॒स्पति॑र् दे॒वता॑ ऽनु॒ष्टु ब॑नु॒ष्टुब् दे॒वता॒ बृह॒स्पति॒र् बृह॒स्पति॑र् दे॒वता॑ ऽनु॒ष्टुप् । \newline
39. दे॒वता॑ ऽनु॒ष्टु ब॑नु॒ष्टुब् दे॒वता॑ दे॒वता॑ ऽनु॒ष्टुप् छन्द॒ श्छन्दो॑ ऽनु॒ष्टुब् दे॒वता॑ दे॒वता॑ ऽनु॒ष्टुप् छन्दः॑ । \newline
40. अ॒नु॒ष्टुप् छन्द॒ श्छन्दो॑ ऽनु॒ष्टु ब॑नु॒ष्टुप् छन्दो॑ मि॒त्रावरु॑णयोर् मि॒त्रावरु॑णयो॒ श्छन्दो॑ ऽनु॒ष्टु ब॑नु॒ष्टुप् छन्दो॑ मि॒त्रावरु॑णयोः । \newline
41. अ॒नु॒ष्टुबित्य॑नु - स्तुप् । \newline
42. छन्दो॑ मि॒त्रावरु॑णयोर् मि॒त्रावरु॑णयो॒ श्छन्द॒ श्छन्दो॑ मि॒त्रावरु॑णयोः॒ पात्र॒म् पात्र॑म् मि॒त्रावरु॑णयो॒ श्छन्द॒ श्छन्दो॑ मि॒त्रावरु॑णयोः॒ पात्र᳚म् । \newline
43. मि॒त्रावरु॑णयोः॒ पात्र॒म् पात्र॑म् मि॒त्रावरु॑णयोर् मि॒त्रावरु॑णयोः॒ पात्र॑ मस्यसि॒ पात्र॑म् मि॒त्रावरु॑णयोर् मि॒त्रावरु॑णयोः॒ पात्र॑ मसि । \newline
44. मि॒त्रावरु॑णयो॒रिति॑ मि॒त्रा - वरु॑णयोः । \newline
45. पात्र॑ मस्यसि॒ पात्र॒म् पात्र॑ मस्य॒श्विना॑ व॒श्विना॑ वसि॒ पात्र॒म् पात्र॑ मस्य॒श्विनौ᳚ । \newline
46. अ॒स्य॒श्विना॑ व॒श्विना॑ वस्य स्य॒श्विनौ॑ दे॒वता॑ दे॒वता॒ ऽश्विना॑ वस्य स्य॒श्विनौ॑ दे॒वता᳚ । \newline
47. अ॒श्विनौ॑ दे॒वता॑ दे॒वता॒ ऽश्विना॑ व॒श्विनौ॑ दे॒वता॑ प॒ङ्क्तिः प॒ङ्क्तिर् दे॒वता॒ ऽश्विना॑ व॒श्विनौ॑ दे॒वता॑ प॒ङ्क्तिः । \newline
48. दे॒वता॑ प॒ङ्क्तिः प॒ङ्क्तिर् दे॒वता॑ दे॒वता॑ प॒ङ्क्ति श्छन्द॒ श्छन्दः॑ प॒ङ्क्तिर् दे॒वता॑ दे॒वता॑ प॒ङ्क्ति श्छन्दः॑ । \newline
49. प॒ङ्क्ति श्छन्द॒ श्छन्दः॑ प॒ङ्क्तिः प॒ङ्क्ति श्छन्दो॒ ऽश्विनो॑ र॒श्विनो॒ श्छन्दः॑ प॒ङ्क्तिः प॒ङ्क्ति श्छन्दो॒ ऽश्विनोः᳚ । \newline
50. छन्दो॒ ऽश्विनो॑ र॒श्विनो॒ श्छन्द॒ श्छन्दो॒ ऽश्विनोः॒ पात्र॒म् पात्र॑ म॒श्विनो॒ श्छन्द॒ श्छन्दो॒ ऽश्विनोः॒ पात्र᳚म् । \newline
51. अ॒श्विनोः॒ पात्र॒म् पात्र॑ म॒श्विनो॑ र॒श्विनोः॒ पात्र॑ मस्यसि॒ पात्र॑ म॒श्विनो॑ र॒श्विनोः॒ पात्र॑ मसि । \newline
52. पात्र॑ मस्यसि॒ पात्र॒म् पात्र॑ मसि॒ सूर्यः॒ सूर्यो॑ ऽसि॒ पात्र॒म् पात्र॑ मसि॒ सूर्यः॑ । \newline
53. अ॒सि॒ सूर्यः॒ सूर्यो᳚ ऽस्यसि॒ सूर्यो॑ दे॒वता॑ दे॒वता॒ सूर्यो᳚ ऽस्यसि॒ सूर्यो॑ दे॒वता᳚ । \newline
54. सूर्यो॑ दे॒वता॑ दे॒वता॒ सूर्यः॒ सूर्यो॑ दे॒वता॑ बृह॒ती बृ॑ह॒ती दे॒वता॒ सूर्यः॒ सूर्यो॑ दे॒वता॑ बृह॒ती । \newline
55. दे॒वता॑ बृह॒ती बृ॑ह॒ती दे॒वता॑ दे॒वता॑ बृह॒ती छन्द॒ श्छन्दो॑ बृह॒ती दे॒वता॑ दे॒वता॑ बृह॒ती छन्दः॑ । \newline
56. बृ॒ह॒ती छन्द॒ श्छन्दो॑ बृह॒ती बृ॑ह॒ती छन्दः॑ शु॒क्रस्य॑ शु॒क्रस्य॒ छन्दो॑ बृह॒ती बृ॑ह॒ती छन्दः॑ शु॒क्रस्य॑ । \newline
\pagebreak
\markright{ TS 3.1.6.3  \hfill https://www.vedavms.in \hfill}

\section{ TS 3.1.6.3 }

\textbf{TS 3.1.6.3 } \newline
\textbf{Samhita Paata} \newline

छन्दः॑ शु॒क्रस्य॒ पात्र॑मसि च॒न्द्रमा॑ दे॒वता॑ स॒तो बृ॑हती॒ छन्दो॑ म॒न्थिनः॒ पात्र॑मसि॒ विश्वे॑दे॒वा दे॒वतो॒ष्णिहा॒ छन्द॑ आग्रय॒णस्य॒ पात्र॑म॒सीन्द्रो॑ दे॒वता॑ क॒कुच्छन्द॑ उ॒क्थानां॒ पात्र॑मसि पृथि॒वी दे॒वता॑ वि॒राट् छन्दो᳚ ध्रु॒वस्य॒ पात्र॑मसि ॥ \newline

\textbf{Pada Paata} \newline

छन्दः॑ । शु॒क्रस्य॑ । पात्र᳚म् । अ॒सि॒ । च॒न्द्रमाः᳚ । दे॒वता᳚ । स॒तोबृ॑ह॒तीति॑ स॒तः - बृ॒ह॒ती॒ । छन्दः॑ । म॒न्थिनः॑ । पात्र᳚म् । अ॒सि॒ । विश्वे᳚ । दे॒वाः । दे॒वता᳚ । उ॒ष्णिहा᳚ । छन्दः॑ । आ॒ग्र॒य॒णस्य॑ । पात्र᳚म् । अ॒सि॒ । इन्द्रः॑ । दे॒वता᳚ । क॒कुत् । छन्दः॑ । उ॒क्थाना᳚म् । पात्र᳚म् । अ॒सि॒ । पृ॒थि॒वी । दे॒वता᳚ । वि॒राडिति॑ वि - राट् । छन्दः॑ । ध्रु॒वस्य॑ । पात्र᳚म् । अ॒सि॒ ॥  \newline


\textbf{Krama Paata} \newline

छन्दः॑ शु॒क्रस्य॑ । शु॒क्रस्य॒ पात्र᳚म् । पात्र॑मसि । अ॒सि॒ च॒न्द्रमाः᳚ । च॒न्द्रमा॑ दे॒वता᳚ । दे॒वता॑ स॒तोबृ॑हती । स॒तोबृ॑हती॒ छन्दः॑ । स॒तोबृ॑ह॒तीति॑ स॒तः - बृ॒ह॒ती॒ । छन्दो॑ म॒न्थिनः॑ । म॒न्थिनः॒ पात्र᳚म् । पात्र॑मसि । अ॒सि॒ विश्वे᳚ । विश्वे॑ दे॒वाः । दे॒वा दे॒वता᳚ । दे॒वतो॒ष्णिहा᳚ । उ॒ष्णिहा॒ छन्दः॑ । छन्द॑ आग्रय॒णस्य॑ । आ॒ग्र॒य॒णस्य॒ पात्र᳚म् । पात्र॑मसि । अ॒सीन्द्रः॑ । इन्द्रो॑ दे॒वता᳚ । दे॒वता॑ क॒कुत् । क॒कुच्छन्दः॑ । छन्द॑ उ॒क्थाना᳚म् । उ॒क्थाना॒म् पात्र᳚म् । पात्र॑मसि । अ॒सि॒ पृ॒थि॒वी । पृ॒थि॒वी दे॒वता᳚ । दे॒वता॑ वि॒राट् । वि॒राट् छन्दः॑ । वि॒राडिति॑ वि - राट् । छन्दो᳚ ध्रु॒वस्य॑ । ध्रु॒वस्य॒ पात्र᳚म् । पात्र॑मसि । अ॒सीत्य॑सि । \newline

\textbf{Jatai Paata} \newline

1. छन्दः॑ शु॒क्रस्य॑ शु॒क्रस्य॒ छन्द॒ श्छन्दः॑ शु॒क्रस्य॑ । \newline
2. शु॒क्रस्य॒ पात्र॒म् पात्रꣳ॑ शु॒क्रस्य॑ शु॒क्रस्य॒ पात्र᳚म् । \newline
3. पात्र॑ मस्यसि॒ पात्र॒म् पात्र॑ मसि । \newline
4. अ॒सि॒ च॒न्द्रमा᳚ श्च॒न्द्रमा॑ अस्यसि च॒न्द्रमाः᳚ । \newline
5. च॒न्द्रमा॑ दे॒वता॑ दे॒वता॑ च॒न्द्रमा᳚ श्च॒न्द्रमा॑ दे॒वता᳚ । \newline
6. दे॒वता॑ स॒तोबृ॑हती स॒तोबृ॑हती दे॒वता॑ दे॒वता॑ स॒तोबृ॑हती । \newline
7. स॒तोबृ॑हती॒ छन्द॒ श्छन्दः॑ स॒तोबृ॑हती स॒तोबृ॑हती॒ छन्दः॑ । \newline
8. स॒तोबृ॑ह॒तीति॑ स॒तः - बृ॒ह॒ती॒ । \newline
9. छन्दो॑ म॒न्थिनो॑ म॒न्थिन॒ श्छन्द॒ श्छन्दो॑ म॒न्थिनः॑ । \newline
10. म॒न्थिनः॒ पात्र॒म् पात्र॑म् म॒न्थिनो॑ म॒न्थिनः॒ पात्र᳚म् । \newline
11. पात्र॑ मस्यसि॒ पात्र॒म् पात्र॑ मसि । \newline
12. अ॒सि॒ विश्वे॒ विश्वे᳚ ऽस्यसि॒ विश्वे᳚ । \newline
13. विश्वे॑ दे॒वा दे॒वा विश्वे॒ विश्वे॑ दे॒वाः । \newline
14. दे॒वा दे॒वता॑ दे॒वता॑ दे॒वा दे॒वा दे॒वता᳚ । \newline
15. दे॒व तो॒ष्णिहो॒ ष्णिहा॑ दे॒वता॑ दे॒व तो॒ष्णिहा᳚ । \newline
16. उ॒ष्णिहा॒ छन्द॒ श्छन्द॑ उ॒ष्णि हो॒ष्णिहा॒ छन्दः॑ । \newline
17. छन्द॑ आग्रय॒णस्या᳚ ग्रय॒णस्य॒ छन्द॒ श्छन्द॑ आग्रय॒णस्य॑ । \newline
18. आ॒ग्र॒य॒णस्य॒ पात्र॒म् पात्र॑ माग्रय॒णस्या᳚ ग्रय॒णस्य॒ पात्र᳚म् । \newline
19. पात्र॑ मस्यसि॒ पात्र॒म् पात्र॑ मसि । \newline
20. अ॒सीन्द्र॒ इन्द्रो᳚ ऽस्य॒सीन्द्रः॑ । \newline
21. इन्द्रो॑ दे॒वता॑ दे॒वतेन्द्र॒ इन्द्रो॑ दे॒वता᳚ । \newline
22. दे॒वता॑ क॒कुत् क॒कुद् दे॒वता॑ दे॒वता॑ क॒कुत् । \newline
23. क॒कुच् छन्द॒ श्छन्दः॑ क॒कुत् क॒कुच् छन्दः॑ । \newline
24. छन्द॑ उ॒क्थाना॑ मु॒क्थाना॒म् छन्द॒ श्छन्द॑ उ॒क्थाना᳚म् । \newline
25. उ॒क्थाना॒म् पात्र॒म् पात्र॑ मु॒क्थाना॑ मु॒क्थाना॒म् पात्र᳚म् । \newline
26. पात्र॑ मस्यसि॒ पात्र॒म् पात्र॑ मसि । \newline
27. अ॒सि॒ पृ॒थि॒वी पृ॑थि॒ व्य॑स्यसि पृथि॒वी । \newline
28. पृ॒थि॒वी दे॒वता॑ दे॒वता॑ पृथि॒वी पृ॑थि॒वी दे॒वता᳚ । \newline
29. दे॒वता॑ वि॒राड् वि॒राड् दे॒वता॑ दे॒वता॑ वि॒राट् । \newline
30. वि॒राट् छन्द॒ श्छन्दो॑ वि॒राड् वि॒राट् छन्दः॑ । \newline
31. वि॒राडिति॑ वि - राट् । \newline
32. छन्दो᳚ ध्रु॒वस्य॑ ध्रु॒वस्य॒ छन्द॒ श्छन्दो᳚ ध्रु॒वस्य॑ । \newline
33. ध्रु॒वस्य॒ पात्र॒म् पात्र॑म् ध्रु॒वस्य॑ ध्रु॒वस्य॒ पात्र᳚म् । \newline
34. पात्र॑ मस्यसि॒ पात्र॒म् पात्र॑ मसि । \newline
35. अ॒सीत्य॑सि । \newline

\textbf{Ghana Paata } \newline

1. छन्दः॑ शु॒क्रस्य॑ शु॒क्रस्य॒ छन्द॒ श्छन्दः॑ शु॒क्रस्य॒ पात्र॒म् पात्रꣳ॑ शु॒क्रस्य॒ छन्द॒ श्छन्दः॑ शु॒क्रस्य॒ पात्र᳚म् । \newline
2. शु॒क्रस्य॒ पात्र॒म् पात्रꣳ॑ शु॒क्रस्य॑ शु॒क्रस्य॒ पात्र॑ मस्यसि॒ पात्रꣳ॑ शु॒क्रस्य॑ शु॒क्रस्य॒ पात्र॑ मसि । \newline
3. पात्र॑ मस्यसि॒ पात्र॒म् पात्र॑ मसि च॒न्द्रमा᳚ श्च॒न्द्रमा॑ असि॒ पात्र॒म् पात्र॑ मसि च॒न्द्रमाः᳚ । \newline
4. अ॒सि॒ च॒न्द्रमा᳚ श्च॒न्द्रमा॑ अस्यसि च॒न्द्रमा॑ दे॒वता॑ दे॒वता॑ च॒न्द्रमा॑ अस्यसि च॒न्द्रमा॑ दे॒वता᳚ । \newline
5. च॒न्द्रमा॑ दे॒वता॑ दे॒वता॑ च॒न्द्रमा᳚ श्च॒न्द्रमा॑ दे॒वता॑ स॒तोबृ॑हती स॒तोबृ॑हती दे॒वता॑ च॒न्द्रमा᳚ श्च॒न्द्रमा॑ दे॒वता॑ स॒तोबृ॑हती । \newline
6. दे॒वता॑ स॒तोबृ॑हती स॒तोबृ॑हती दे॒वता॑ दे॒वता॑ स॒तोबृ॑हती॒ छन्द॒ श्छन्दः॑ स॒तोबृ॑हती दे॒वता॑ दे॒वता॑ स॒तोबृ॑हती॒ छन्दः॑ । \newline
7. स॒तोबृ॑हती॒ छन्द॒ श्छन्दः॑ स॒तोबृ॑हती स॒तोबृ॑हती॒ छन्दो॑ म॒न्थिनो॑ म॒न्थिन॒ श्छन्दः॑ स॒तोबृ॑हती स॒तोबृ॑हती॒ छन्दो॑ म॒न्थिनः॑ । \newline
8. स॒तोबृ॑ह॒तीति॑ स॒तः - बृ॒ह॒ती॒ । \newline
9. छन्दो॑ म॒न्थिनो॑ म॒न्थिन॒ श्छन्द॒ श्छन्दो॑ म॒न्थिनः॒ पात्र॒म् पात्र॑म् म॒न्थिन॒ श्छन्द॒ श्छन्दो॑ म॒न्थिनः॒ पात्र᳚म् । \newline
10. म॒न्थिनः॒ पात्र॒म् पात्र॑म् म॒न्थिनो॑ म॒न्थिनः॒ पात्र॑ मस्यसि॒ पात्र॑म् म॒न्थिनो॑ म॒न्थिनः॒ पात्र॑ मसि । \newline
11. पात्र॑ मस्यसि॒ पात्र॒म् पात्र॑ मसि॒ विश्वे॒ विश्वे॑ ऽसि॒ पात्र॒म् पात्र॑ मसि॒ विश्वे᳚ । \newline
12. अ॒सि॒ विश्वे॒ विश्वे᳚ ऽस्यसि॒ विश्वे॑ दे॒वा दे॒वा विश्वे᳚ ऽस्यसि॒ विश्वे॑ दे॒वाः । \newline
13. विश्वे॑ दे॒वा दे॒वा विश्वे॒ विश्वे॑ दे॒वा दे॒वता॑ दे॒वता॑ दे॒वा विश्वे॒ विश्वे॑ दे॒वा दे॒वता᳚ । \newline
14. दे॒वा दे॒वता॑ दे॒वता॑ दे॒वा दे॒वा दे॒वतो॒ष्णि हो॒ष्णिहा॑ दे॒वता॑ दे॒वा दे॒वा दे॒वतो॒ष्णिहा᳚ । \newline
15. दे॒व तो॒ष्णि हो॒ष्णिहा॑ दे॒वता॑ दे॒वतो॒ष्णिहा॒ छन्द॒ श्छन्द॑ उ॒ष्णिहा॑ दे॒वता॑ दे॒वतो॒ष्णिहा॒ छन्दः॑ । \newline
16. उ॒ष्णिहा॒ छन्द॒ श्छन्द॑ उ॒ष्णि हो॒ष्णिहा॒ छन्द॑ आग्रय॒णस्या᳚ ग्रय॒णस्य॒ छन्द॑ उ॒ष्णि हो॒ष्णिहा॒ छन्द॑ आग्रय॒णस्य॑ । \newline
17. छन्द॑ आग्रय॒णस्या᳚ ग्रय॒णस्य॒ छन्द॒ श्छन्द॑ आग्रय॒णस्य॒ पात्र॒म् पात्र॑ माग्रय॒णस्य॒ छन्द॒ श्छन्द॑ आग्रय॒णस्य॒ पात्र᳚म् । \newline
18. आ॒ग्र॒य॒णस्य॒ पात्र॒म् पात्र॑ माग्रय॒णस्या᳚ ग्रय॒णस्य॒ पात्र॑ मस्यसि॒ पात्र॑ माग्रय॒णस्या᳚ ग्रय॒णस्य॒ पात्र॑ मसि । \newline
19. पात्र॑ मस्यसि॒ पात्र॒म् पात्र॑ म॒सीन्द्र॒ इन्द्रो॑ ऽसि॒ पात्र॒म् पात्र॑ म॒सीन्द्रः॑ । \newline
20. अ॒सीन्द्र॒ इन्द्रो᳚ ऽस्य॒सीन्द्रो॑ दे॒वता॑ दे॒वतेन्द्रो᳚ ऽस्य॒सीन्द्रो॑ दे॒वता᳚ । \newline
21. इन्द्रो॑ दे॒वता॑ दे॒वतेन्द्र॒ इन्द्रो॑ दे॒वता॑ क॒कुत् क॒कुद् दे॒वतेन्द्र॒ इन्द्रो॑ दे॒वता॑ क॒कुत् । \newline
22. दे॒वता॑ क॒कुत् क॒कुद् दे॒वता॑ दे॒वता॑ क॒कुच् छन्द॒ श्छन्दः॑ क॒कुद् दे॒वता॑ दे॒वता॑ क॒कुच् छन्दः॑ । \newline
23. क॒कुच् छन्द॒ श्छन्दः॑ क॒कुत् क॒कुच् छन्द॑ उ॒क्थाना॑ मु॒क्थाना॒म् छन्दः॑ क॒कुत् क॒कुच् छन्द॑ उ॒क्थाना᳚म् । \newline
24. छन्द॑ उ॒क्थाना॑ मु॒क्थाना॒म् छन्द॒ श्छन्द॑ उ॒क्थाना॒म् पात्र॒म् पात्र॑ मु॒क्थाना॒म् छन्द॒ श्छन्द॑ उ॒क्थाना॒म् पात्र᳚म् । \newline
25. उ॒क्थाना॒म् पात्र॒म् पात्र॑ मु॒क्थाना॑ मु॒क्थाना॒म् पात्र॑ मस्यसि॒ पात्र॑ मु॒क्थाना॑ मु॒क्थाना॒म् पात्र॑ मसि । \newline
26. पात्र॑ मस्यसि॒ पात्र॒म् पात्र॑ मसि पृथि॒वी पृ॑थि॒व्य॑सि॒ पात्र॒म् पात्र॑ मसि पृथि॒वी । \newline
27. अ॒सि॒ पृ॒थि॒वी पृ॑थि॒व्य॑स्यसि पृथि॒वी दे॒वता॑ दे॒वता॑ पृथि॒व्य॑स्यसि पृथि॒वी दे॒वता᳚ । \newline
28. पृ॒थि॒वी दे॒वता॑ दे॒वता॑ पृथि॒वी पृ॑थि॒वी दे॒वता॑ वि॒राड् वि॒राड् दे॒वता॑ पृथि॒वी पृ॑थि॒वी दे॒वता॑ वि॒राट् । \newline
29. दे॒वता॑ वि॒राड् वि॒राड् दे॒वता॑ दे॒वता॑ वि॒राट् छन्द॒ श्छन्दो॑ वि॒राड् दे॒वता॑ दे॒वता॑ वि॒राट् छन्दः॑ । \newline
30. वि॒राट् छन्द॒ श्छन्दो॑ वि॒राड् वि॒राट् छन्दो᳚ ध्रु॒वस्य॑ ध्रु॒वस्य॒ छन्दो॑ वि॒राड् वि॒राट् छन्दो᳚ ध्रु॒वस्य॑ । \newline
31. वि॒राडिति॑ वि - राट् । \newline
32. छन्दो᳚ ध्रु॒वस्य॑ ध्रु॒वस्य॒ छन्द॒ श्छन्दो᳚ ध्रु॒वस्य॒ पात्र॒म् पात्र॑म् ध्रु॒वस्य॒ छन्द॒ श्छन्दो᳚ ध्रु॒वस्य॒ पात्र᳚म् । \newline
33. ध्रु॒वस्य॒ पात्र॒म् पात्र॑म् ध्रु॒वस्य॑ ध्रु॒वस्य॒ पात्र॑ मस्यसि॒ पात्र॑म् ध्रु॒वस्य॑ ध्रु॒वस्य॒ पात्र॑ मसि । \newline
34. पात्र॑ मस्यसि॒ पात्र॒म् पात्र॑ मसि । \newline
35. अ॒सीत्य॑सि । \newline
\pagebreak
\markright{ TS 3.1.7.1  \hfill https://www.vedavms.in \hfill}

\section{ TS 3.1.7.1 }

\textbf{TS 3.1.7.1 } \newline
\textbf{Samhita Paata} \newline

इ॒ष्टर्गो॒ वा अ॑द्ध्व॒र्युर्यज॑मानस्ये॒ष्टर्गः॒ खलु॒ वै पूर्वो॒ऽर्ष्टुः क्षी॑यत आस॒न्या᳚न्मा॒ मन्त्रा᳚त् पाहि॒ कस्या᳚श्चिद॒भिश॑स्त्या॒ इति॑ पु॒रा प्रा॑तरनुवा॒काज्जु॑हुयादा॒त्मन॑ ए॒व तद॑द्ध्व॒र्युः पु॒रस्ता॒च्छर्म॑ नह्य॒तेऽना᳚र्त्यै संॅवे॒शाय॑ त्वोपवे॒शाय॑ त्वा गायत्रि॒या स्त्रि॒ष्टुभो॒ जग॑त्या अ॒भिभू᳚त्यै॒ स्वाहा॒ प्राणा॑पानौ मृ॒त्योर्मा॑ पातं॒ प्राणा॑पानौ॒ मा मा॑ हासिष्टं दे॒वता॑सु॒ वा ए॒ते प्रा॑णापा॒नयो॒र् - [  ] \newline

\textbf{Pada Paata} \newline

इ॒ष्टर्गः॑ । वै । अ॒द्ध्व॒र्युः । यज॑मानस्य । इ॒ष्टर्गः॑ । खलु॑ । वै । पूर्वः॑ । अ॒र्ष्टुः । क्षी॒य॒ते॒ । आ॒स॒न्या᳚त् । मा॒ । मन्त्रा᳚त् । पा॒हि॒ । कस्याः᳚ । चि॒त् । अ॒भिश॑स्त्या॒ इत्य॒भि - श॒स्त्याः॒ । इति॑ । पु॒रा । प्रा॒त॒र॒नु॒वा॒कादिति॑ प्रातः - अ॒नु॒वा॒कात् । जु॒हु॒या॒त् । आ॒त्मने᳚ । ए॒व । तत् । अ॒द्ध्व॒र्युः । पु॒रस्ता᳚त् । शर्म॑ । न॒ह्य॒ते॒ । अना᳚र्त्यै । सं॒ॅवे॒शायेति॑ सं - वे॒शाय॑ । त्वा॒ । उ॒प॒वे॒शायेतु॑प - वे॒शाय॑ । त्वा॒ । गा॒य॒त्रि॒याः । त्रि॒ष्टुभः॑ । जग॑त्याः । अ॒भिभू᳚त्या॒ इत्य॒भि - भू॒त्यै॒ । स्वाहा᳚ । प्राणा॑पाना॒विति॒ प्राण॑ - अ॒पा॒नौ॒ । मृ॒त्योः । मा॒ । पा॒त॒म् । प्राणा॑पाना॒विति॒ प्राण॑ - अ॒पा॒नौ॒ । मा । मा॒ । हा॒सि॒ष्ट॒म् । दे॒वता॑सु । वै । ए॒ते । प्रा॒णा॒पा॒नयो॒रिति॑ प्राण - अ॒पा॒नयोः᳚ ।  \newline


\textbf{Krama Paata} \newline

इ॒ष्टर्गो॒ वै । वा अ॑द्ध्व॒र्युः । अ॒द्ध्व॒र्युर्. यज॑मानस्य । यज॑मानस्ये॒ष्टर्गः॑ । इ॒ष्टर्गः॒ खलु॑ । खलु॒ वै । वै पूर्वः॑ । पूर्वो॒ऽर्ष्टुः । अ॒र्ष्टुः क्षी॑यते । क्षी॒य॒त॒ आ॒स॒न्या᳚त् । आ॒स॒न्या᳚न् मा । मा॒ मन्त्रा᳚त् । मन्त्रा᳚त् पाहि । पा॒हि॒ कस्याः᳚ । कस्या᳚श्चित् । चि॒द॒भिश॑स्त्याः । अ॒भिश॑स्त्या॒ इति॑ । अ॒भिश॑स्त्या॒ इत्य॒भि - श॒स्त्याः॒ । इति॑ पु॒रा । पु॒रा प्रा॑तरनुवा॒कात् । प्रा॒त॒र॒नु॒वा॒काज् जु॑हुयात् । प्रा॒त॒र॒नु॒वा॒कादिति॑ प्रातः - अ॒नु॒वा॒कात् । जु॒हु॒या॒दा॒त्मने᳚ । आ॒त्मन॑ ए॒व । ए॒व तत् । तद॑द्ध्व॒र्युः । अ॒द्ध्व॒र्युः पु॒रस्ता᳚त् । पु॒रस्ता॒च्छर्म॑ । शर्म॑ नह्यते । न॒ह्य॒ते ऽना᳚र्त्यै । अना᳚र्त्यै सम्ॅवे॒शाय॑ । 
स॒म्ॅवे॒शाय॑ त्वा । स॒म्ॅवे॒शायेति॑ सम् - वे॒शाय॑ । त्वो॒प॒वे॒शाय॑ । उ॒प॒वे॒शाय॑ त्वा । उ॒प॒वे॒शायेत्यु॑प - वे॒शाय॑ । त्वा॒ गा॒य॒त्रि॒याः । गा॒य॒त्रि॒या स्त्रि॒ष्टुभः॑ । त्रि॒ष्टुभो॒ जग॑त्याः । जग॑त्या अ॒भिभू᳚त्यै । अ॒भिभू᳚त्यै॒ स्वाहा᳚ । अ॒भिभू᳚त्या॒ इत्य॒भि - भू॒त्यै॒ । स्वाहा॒ प्राणा॑पानौ । प्राणा॑पानौ मृ॒त्योः । प्राणा॑पाना॒विति॒ प्राण॑ - अ॒पा॒नौ॒ । मृ॒त्योर् मा᳚ । मा॒ पा॒त॒म् । पा॒त॒म् प्राणा॑पानौ । प्राणा॑पानौ॒ मा । प्राणा॑पाना॒विति॒ प्राण॑ - अ॒पा॒नौ॒ । मा मा᳚ । मा॒ हा॒सि॒ष्ट॒म् । हा॒सि॒ष्ट॒म् दे॒वता॑सु । दे॒वता॑सु॒ वै । वा ए॒ते । ए॒ते प्रा॑णापा॒नयोः᳚ । प्रा॒णा॒पा॒नयो॒र् व्याय॑च्छन्ते । प्रा॒णा॒पा॒नयो॒रिति॑ प्राण - अ॒पा॒नयोः᳚ \newline

\textbf{Jatai Paata} \newline

1. इ॒ष्टर्गो॒ वै वा इ॒ष्टर्ग॑ इ॒ष्टर्गो॒ वै । \newline
2. वा अ॑द्ध्व॒र्यु र॑द्ध्व॒र्युर् वै वा अ॑द्ध्व॒र्युः । \newline
3. अ॒द्ध्व॒र्युर् यज॑मानस्य॒ यज॑मानस्या द्ध्व॒र्यु र॑द्ध्व॒र्युर् यज॑मानस्य । \newline
4. यज॑मानस्ये॒ ष्टर्ग॑ इ॒ष्टर्गो॒ यज॑मानस्य॒ यज॑मानस्ये॒ ष्टर्गः॑ । \newline
5. इ॒ष्टर्गः॒ खलु॒ खल्वि॒ष्टर्ग॑ इ॒ष्टर्गः॒ खलु॑ । \newline
6. खलु॒ वै वै खलु॒ खलु॒ वै । \newline
7. वै पूर्वः॒ पूर्वो॒ वै वै पूर्वः॑ । \newline
8. पूर्वो॒ ऽर्ष्टु र॒र्ष्टुः पूर्वः॒ पूर्वो॒ ऽर्ष्टुः । \newline
9. अ॒र्ष्टुः क्षी॑यते क्षीयते॒ ऽर्ष्टु र॒र्ष्टुः क्षी॑यते । \newline
10. क्षी॒य॒त॒ आ॒स॒न्या॑ दास॒न्या᳚त् क्षीयते क्षीयत आस॒न्या᳚त् । \newline
11. आ॒स॒न्या᳚न् मा मा ऽऽस॒न्या॑ दास॒न्या᳚न् मा । \newline
12. मा॒ मन्त्रा॒न् मन्त्रा᳚न् मा मा॒ मन्त्रा᳚त् । \newline
13. मन्त्रा᳚त् पाहि पाहि॒ मन्त्रा॒न् मन्त्रा᳚त् पाहि । \newline
14. पा॒हि॒ कस्याः॒ कस्याः᳚ पाहि पाहि॒ कस्याः᳚ । \newline
15. कस्या᳚ श्चिच् चि॒त् कस्याः॒ कस्या᳚ श्चित् । \newline
16. चि॒द॒भिश॑स्त्या अ॒भिश॑स्त्या श्चिच् चि द॒भिश॑स्त्याः । \newline
17. अ॒भिश॑स्त्या॒ इती त्य॒भिश॑स्त्या अ॒भिश॑स्त्या॒ इति॑ । \newline
18. अ॒भिश॑स्त्या॒ इत्य॒भि - श॒स्त्याः॒ । \newline
19. इति॑ पु॒रा पु॒रेतीति॑ पु॒रा । \newline
20. पु॒रा प्रा॑तरनुवा॒कात् प्रा॑तरनुवा॒कात् पु॒रा पु॒रा प्रा॑तरनुवा॒कात् । \newline
21. प्रा॒त॒र॒नु॒वा॒काज् जु॑हुयाज् जुहुयात् प्रातरनुवा॒कात् प्रा॑तरनुवा॒काज् जु॑हुयात् । \newline
22. प्रा॒त॒र॒नु॒वा॒कादिति॑ प्रातः - अ॒नु॒वा॒कात् । \newline
23. जु॒हु॒या॒ दा॒त्मन॑ आ॒त्मने॑ जुहुयाज् जुहुया दा॒त्मने᳚ । \newline
24. आ॒त्मन॑ ए॒वैवात्मन॑ आ॒त्मन॑ ए॒व । \newline
25. ए॒व तत् त दे॒वैव तत् । \newline
26. तद॑द्ध्व॒र्यु र॑द्ध्व॒र्यु स्तत् तद॑द्ध्व॒र्युः । \newline
27. अ॒द्ध्व॒र्युः पु॒रस्ता᳚त् पु॒रस्ता॑ दद्ध्व॒र्यु र॑द्ध्व॒र्युः पु॒रस्ता᳚त् । \newline
28. पु॒रस्ता॒च् छर्म॒ शर्म॑ पु॒रस्ता᳚त् पु॒रस्ता॒च् छर्म॑ । \newline
29. शर्म॑ नह्यते नह्यते॒ शर्म॒ शर्म॑ नह्यते । \newline
30. न॒ह्य॒ते ऽना᳚र्त्या॒ अना᳚र्त्यै नह्यते नह्य॒ते ऽना᳚र्त्यै । \newline
31. अना᳚र्त्यै संॅवे॒शाय॑ संॅवे॒शाया ना᳚र्त्या॒ अना᳚र्त्यै संॅवे॒शाय॑ । \newline
32. सं॒ॅवे॒शाय॑ त्वा त्वा संॅवे॒शाय॑ संॅवे॒शाय॑ त्वा । \newline
33. सं॒ॅवे॒शायेति॑ सं - वे॒शाय॑ । \newline
34. त्वो॒प॒वे॒शा यो॑पवे॒शाय॑ त्वा त्वोपवे॒शाय॑ । \newline
35. उ॒प॒वे॒शाय॑ त्वा त्वोपवे॒शा यो॑पवे॒शाय॑ त्वा । \newline
36. उ॒प॒वे॒शायेत्यु॑प - वे॒शाय॑ । \newline
37. त्वा॒ गा॒य॒त्रि॒या गा॑यत्रि॒या स्त्वा᳚ त्वा गायत्रि॒याः । \newline
38. गा॒य॒त्रि॒या स्त्रि॒ष्टुभ॑ स्त्रि॒ष्टुभो॑ गायत्रि॒या गा॑यत्रि॒या स्त्रि॒ष्टुभः॑ । \newline
39. त्रि॒ष्टुभो॒ जग॑त्या॒ जग॑त्या स्त्रि॒ष्टुभ॑ स्त्रि॒ष्टुभो॒ जग॑त्याः । \newline
40. जग॑त्या अ॒भिभू᳚त्या अ॒भिभू᳚त्यै॒ जग॑त्या॒ जग॑त्या अ॒भिभू᳚त्यै । \newline
41. अ॒भिभू᳚त्यै॒ स्वाहा॒ स्वाहा॒ ऽभिभू᳚त्या अ॒भिभू᳚त्यै॒ स्वाहा᳚ । \newline
42. अ॒भिभू᳚त्या॒ इत्य॒भि - भू॒त्यै॒ । \newline
43. स्वाहा॒ प्राणा॑पानौ॒ प्राणा॑पानौ॒ स्वाहा॒ स्वाहा॒ प्राणा॑पानौ । \newline
44. प्राणा॑पानौ मृ॒त्योर् मृ॒त्योः प्राणा॑पानौ॒ प्राणा॑पानौ मृ॒त्योः । \newline
45. प्राणा॑पाना॒विति॒ प्राण॑ - अ॒पा॒नौ॒ । \newline
46. मृ॒त्योर् मा॑ मा मृ॒त्योर् मृ॒त्योर् मा᳚ । \newline
47. मा॒ पा॒त॒म् पा॒त॒म् मा॒ मा॒ पा॒त॒म् । \newline
48. पा॒त॒म् प्राणा॑पानौ॒ प्राणा॑पानौ पातम् पात॒म् प्राणा॑पानौ । \newline
49. प्राणा॑पानौ॒ मा मा प्राणा॑पानौ॒ प्राणा॑पानौ॒ मा । \newline
50. प्राणा॑पाना॒विति॒ प्राण॑ - अ॒पा॒नौ॒ । \newline
51. मा मा॑ मा॒ मा मा मा᳚ । \newline
52. मा॒ हा॒सि॒ष्टꣳ॒॒ हा॒सि॒ष्ट॒म् मा॒ मा॒ हा॒सि॒ष्ट॒म् । \newline
53. हा॒सि॒ष्ट॒म् दे॒वता॑सु दे॒वता॑सु हासिष्टꣳ हासिष्टम् दे॒वता॑सु । \newline
54. दे॒वता॑सु॒ वै वै दे॒वता॑सु दे॒वता॑सु॒ वै । \newline
55. वा ए॒त ए॒ते वै वा ए॒ते । \newline
56. ए॒ते प्रा॑णापा॒नयोः᳚ प्राणापा॒नयो॑ रे॒त ए॒ते प्रा॑णापा॒नयोः᳚ । \newline
57. प्रा॒णा॒पा॒नयो॒र् व्याय॑च्छन्ते॒ व्याय॑च्छन्ते प्राणापा॒नयोः᳚ प्राणापा॒नयो॒र् व्याय॑च्छन्ते । \newline
58. प्रा॒णा॒पा॒नयो॒रिति॑ प्राण - अ॒पा॒नयोः᳚ । \newline

\textbf{Ghana Paata } \newline

1. इ॒ष्टर्गो॒ वै वा इ॒ष्टर्ग॑ इ॒ष्टर्गो॒ वा अ॑द्ध्व॒र्यु र॑द्ध्व॒र्युर् वा इ॒ष्टर्ग॑ इ॒ष्टर्गो॒ वा अ॑द्ध्व॒र्युः । \newline
2. वा अ॑द्ध्व॒र्यु र॑द्ध्व॒र्युर् वै वा अ॑द्ध्व॒र्युर् यज॑मानस्य॒ यज॑मानस्या द्ध्व॒र्युर् वै वा अ॑द्ध्व॒र्युर् यज॑मानस्य । \newline
3. अ॒द्ध्व॒र्युर् यज॑मानस्य॒ यज॑मानस्या द्ध्व॒र्यु र॑द्ध्व॒र्युर् यज॑मान स्ये॒ष्टर्ग॑ इ॒ष्टर्गो॒ यज॑मानस्या द्ध्व॒र्यु र॑द्ध्व॒र्युर् यज॑मान स्ये॒ष्टर्गः॑ । \newline
4. यज॑मान स्ये॒ष्टर्ग॑ इ॒ष्टर्गो॒ यज॑मान स्य॒यज॑मानस्ये॒ ष्टर्गः॒ खलु॒ खल्वि॒ष्टर्गो॒ यज॑मानस्य॒ 
यज॑मान स्ये॒ष्टर्गः॒ खलु॑ । \newline
5. इ॒ष्टर्गः॒ खलु॒ खल्वि॒ष्टर्ग॑ इ॒ष्टर्गः॒ खलु॒ वै वै खल्वि॒ष्टर्ग॑ इ॒ष्टर्गः॒ खलु॒ वै । \newline
6. खलु॒ वै वै खलु॒ खलु॒ वै पूर्वः॒ पूर्वो॒ वै खलु॒ खलु॒ वै पूर्वः॑ । \newline
7. वै पूर्वः॒ पूर्वो॒ वै वै पूर्वो॒ ऽर्ष्टुर॒र्ष्टुः पूर्वो॒ वै वै पूर्वो॒ ऽर्ष्टुः । \newline
8. पूर्वो॒ ऽर्ष्टुर॒र्ष्टुः पूर्वः॒ पूर्वो॒ ऽर्ष्टुः क्षी॑यते क्षीयते॒ ऽर्ष्टुः पूर्वः॒ पूर्वो॒ ऽर्ष्टुः क्षी॑यते । \newline
9. अ॒र्ष्टुः क्षी॑यते क्षीयते॒ ऽर्ष्टुर॒र्ष्टुः क्षी॑यत आस॒न्या॑ दास॒न्या᳚त् क्षीयते॒ ऽर्ष्टुर॒र्ष्टुः क्षी॑यत आस॒न्या᳚त् । \newline
10. क्षी॒य॒त॒ आ॒स॒न्या॑ दास॒न्या᳚त् क्षीयते क्षीयत आस॒न्या᳚न् मा मा ऽऽस॒न्या᳚त् क्षीयते क्षीयत आस॒न्या᳚न् मा । \newline
11. आ॒स॒न्या᳚न् मा मा ऽऽस॒न्या॑ दास॒न्या᳚न् मा॒ मन्त्रा॒न् मन्त्रा᳚न् मा ऽऽस॒न्या॑ दास॒न्या᳚न् मा॒ मन्त्रा᳚त् । \newline
12. मा॒ मन्त्रा॒न् मन्त्रा᳚न् मा मा॒ मन्त्रा᳚त् पाहि पाहि॒ मन्त्रा᳚न् मा मा॒ मन्त्रा᳚त् पाहि । \newline
13. मन्त्रा᳚त् पाहि पाहि॒ मन्त्रा॒न् मन्त्रा᳚त् पाहि॒ कस्याः॒ कस्याः᳚ पाहि॒ मन्त्रा॒न् मन्त्रा᳚त् पाहि॒ कस्याः᳚ । \newline
14. पा॒हि॒ कस्याः॒ कस्याः᳚ पाहि पाहि॒ कस्या᳚ श्चिच् चि॒त् कस्याः᳚ पाहि पाहि॒ कस्या᳚ श्चित् । \newline
15. कस्या᳚ श्चिच् चि॒त् कस्याः॒ कस्या᳚ श्चि द॒भिश॑स्त्या अ॒भिश॑स्त्या श्चि॒त् कस्याः॒ कस्या᳚ श्चि द॒भिश॑स्त्याः । \newline
16. चि॒द॒भिश॑स्त्या अ॒भिश॑स्त्या श्चिच् चिद॒भिश॑स्त्या॒ इती त्य॒भिश॑स्त्या श्चिच् चिद॒भिश॑स्त्या॒ इति॑ । \newline
17. अ॒भिश॑स्त्या॒ इती त्य॒भिश॑स्त्या अ॒भिश॑स्त्या॒ इति॑ पु॒रा पु॒रे त्य॒भिश॑स्त्या अ॒भिश॑स्त्या॒ इति॑ पु॒रा । \newline
18. अ॒भिश॑स्त्या॒ इत्य॒भि - श॒स्त्याः॒ । \newline
19. इति॑ पु॒रा पु॒रेतीति॑ पु॒रा प्रा॑तरनुवा॒कात् प्रा॑तरनुवा॒कात् पु॒रेतीति॑ पु॒रा प्रा॑तरनुवा॒कात् । \newline
20. पु॒रा प्रा॑तरनुवा॒कात् प्रा॑तरनुवा॒कात् पु॒रा पु॒रा प्रा॑तरनुवा॒काज् जु॑हुयाज् जुहुयात् प्रातरनुवा॒कात् पु॒रा पु॒रा प्रा॑तरनुवा॒काज् जु॑हुयात् । \newline
21. प्रा॒त॒र॒नु॒वा॒काज् जु॑हुयाज् जुहुयात् प्रातरनुवा॒कात् प्रा॑तरनुवा॒काज् जु॑हुया दा॒त्मन॑ आ॒त्मने॑ जुहुयात् प्रातरनुवा॒कात् प्रा॑तरनुवा॒काज् जु॑हुया दा॒त्मने᳚ । \newline
22. प्रा॒त॒र॒नु॒वा॒कादिति॑ प्रातः - अ॒नु॒वा॒कात् । \newline
23. जु॒हु॒या॒ दा॒त्मन॑ आ॒त्मने॑ जुहुयाज् जुहुया दा॒त्मन॑ ए॒वैवात्मने॑ जुहुयाज् जुहुया दा॒त्मन॑ ए॒व । \newline
24. आ॒त्मन॑ ए॒वैवात्मन॑ आ॒त्मन॑ ए॒व तत् तदे॒वात्मन॑ आ॒त्मन॑ ए॒व तत् । \newline
25. ए॒व तत् तदे॒वैव तद॑द्ध्व॒र्यु र॑द्ध्व॒र्यु स्तदे॒वैव तद॑द्ध्व॒र्युः । \newline
26. तद॑द्ध्व॒र्यु र॑द्ध्व॒र्यु स्तत् तद॑द्ध्व॒र्युः पु॒रस्ता᳚त् पु॒रस्ता॑ दद्ध्व॒र्यु स्तत् तद॑द्ध्व॒र्युः पु॒रस्ता᳚त् । \newline
27. अ॒द्ध्व॒र्युः पु॒रस्ता᳚त् पु॒रस्ता॑ दद्ध्व॒र्यु र॑द्ध्व॒र्युः पु॒रस्ता॒च्छर्म॒ शर्म॑ पु॒रस्ता॑ दद्ध्व॒र्यु र॑द्ध्व॒र्युः पु॒रस्ता॒च् छर्म॑ । \newline
28. पु॒रस्ता॒च्छर्म॒ शर्म॑ पु॒रस्ता᳚त् पु॒रस्ता॒च्छर्म॑ नह्यते नह्यते॒ शर्म॑ पु॒रस्ता᳚त् पु॒रस्ता॒च्छर्म॑ नह्यते । \newline
29. शर्म॑ नह्यते नह्यते॒ शर्म॒ शर्म॑ नह्य॒ते ऽना᳚र्त्या॒ अना᳚र्त्यै नह्यते॒ शर्म॒ शर्म॑ नह्य॒ते ऽना᳚र्त्यै । \newline
30. न॒ह्य॒ते ऽना᳚र्त्या॒ अना᳚र्त्यै नह्यते नह्य॒ते ऽना᳚र्त्यै संॅवे॒शाय॑ संॅवे॒शाया ना᳚र्त्यै नह्यते नह्य॒ते ऽना᳚र्त्यै संॅवे॒शाय॑ । \newline
31. अना᳚र्त्यै संॅवे॒शाय॑ संॅवे॒शाया ना᳚र्त्या॒ अना᳚र्त्यै संॅवे॒शाय॑ त्वा त्वा संॅवे॒शाया ना᳚र्त्या॒ अना᳚र्त्यै संॅवे॒शाय॑ त्वा । \newline
32. सं॒ॅवे॒शाय॑ त्वा त्वा संॅवे॒शाय॑ संॅवे॒शाय॑ त्वोपवे॒शा यो॑पवे॒शाय॑ त्वा संॅवे॒शाय॑ संॅवे॒शाय॑ त्वोपवे॒शाय॑ । \newline
33. सं॒ॅवे॒शायेति॑ सं - वे॒शाय॑ । \newline
34. त्वो॒प॒वे॒शा यो॑पवे॒शाय॑ त्वा त्वोपवे॒शाय॑ त्वा त्वोपवे॒शाय॑ त्वा त्वोपवे॒शाय॑ त्वा । \newline
35. उ॒प॒वे॒शाय॑ त्वा त्वोपवे॒शा यो॑पवे॒शाय॑ त्वा गायत्रि॒या गा॑यत्रि॒या स्त्वो॑पवे॒शा यो॑पवे॒शाय॑ त्वा गायत्रि॒याः । \newline
36. उ॒प॒वे॒शायेत्यु॑प - वे॒शाय॑ । \newline
37. त्वा॒ गा॒य॒त्रि॒या गा॑यत्रि॒या स्त्वा᳚ त्वा गायत्रि॒या स्त्रि॒ष्टुभ॑ स्त्रि॒ष्टुभो॑ गायत्रि॒या स्त्वा᳚ त्वा गायत्रि॒या स्त्रि॒ष्टुभः॑ । \newline
38. गा॒य॒त्रि॒या स्त्रि॒ष्टुभ॑ स्त्रि॒ष्टुभो॑ गायत्रि॒या गा॑यत्रि॒या स्त्रि॒ष्टुभो॒ जग॑त्या॒ जग॑त्या स्त्रि॒ष्टुभो॑ गायत्रि॒या गा॑यत्रि॒या स्त्रि॒ष्टुभो॒ जग॑त्याः । \newline
39. त्रि॒ष्टुभो॒ जग॑त्या॒ जग॑त्या स्त्रि॒ष्टुभ॑ स्त्रि॒ष्टुभो॒ जग॑त्या अ॒भिभू᳚त्या अ॒भिभू᳚त्यै॒ जग॑त्या स्त्रि॒ष्टुभ॑ स्त्रि॒ष्टुभो॒ जग॑त्या अ॒भिभू᳚त्यै । \newline
40. जग॑त्या अ॒भिभू᳚त्या अ॒भिभू᳚त्यै॒ जग॑त्या॒ जग॑त्या अ॒भिभू᳚त्यै॒ स्वाहा॒ स्वाहा॒ ऽभिभू᳚त्यै॒ जग॑त्या॒ जग॑त्या अ॒भिभू᳚त्यै॒ स्वाहा᳚ । \newline
41. अ॒भिभू᳚त्यै॒ स्वाहा॒ स्वाहा॒ ऽभिभू᳚त्या अ॒भिभू᳚त्यै॒ स्वाहा॒ प्राणा॑पानौ॒ प्राणा॑पानौ॒ स्वाहा॒ ऽभिभू᳚त्या अ॒भिभू᳚त्यै॒ स्वाहा॒ प्राणा॑पानौ । \newline
42. अ॒भिभू᳚त्या॒ इत्य॒भि - भू॒त्यै॒ । \newline
43. स्वाहा॒ प्राणा॑पानौ॒ प्राणा॑पानौ॒ स्वाहा॒ स्वाहा॒ प्राणा॑पानौ मृ॒त्योर् मृ॒त्योः प्राणा॑पानौ॒ स्वाहा॒ स्वाहा॒ प्राणा॑पानौ मृ॒त्योः । \newline
44. प्राणा॑पानौ मृ॒त्योर् मृ॒त्योः प्राणा॑पानौ॒ प्राणा॑पानौ मृ॒त्योर् मा॑ मा मृ॒त्योः प्राणा॑पानौ॒ प्राणा॑पानौ मृ॒त्योर् मा᳚ । \newline
45. प्राणा॑पाना॒विति॒ प्राण॑ - अ॒पा॒नौ॒ । \newline
46. मृ॒त्योर् मा॑ मा मृ॒त्योर् मृ॒त्योर् मा॑ पातम् पातम् मा मृ॒त्योर् मृ॒त्योर् मा॑ पातम् । \newline
47. मा॒ पा॒त॒म् पा॒त॒म् मा॒ मा॒ पा॒त॒म् प्राणा॑पानौ॒ प्राणा॑पानौ पातम् मा मा पात॒म् प्राणा॑पानौ । \newline
48. पा॒त॒म् प्राणा॑पानौ॒ प्राणा॑पानौ पातम् पात॒म् प्राणा॑पानौ॒ मा मा प्राणा॑पानौ पातम् पात॒म् प्राणा॑पानौ॒ मा । \newline
49. प्राणा॑पानौ॒ मा मा प्राणा॑पानौ॒ प्राणा॑पानौ॒ मा मा॑ मा॒ मा प्राणा॑पानौ॒ प्राणा॑पानौ॒ मा मा᳚ । \newline
50. प्राणा॑पाना॒विति॒ प्राण॑ - अ॒पा॒नौ॒ । \newline
51. मा मा॑ मा॒ मा मा मा॑ हासिष्टꣳ हासिष्टम् मा॒ मा मा मा॑ हासिष्टम् । \newline
52. मा॒ हा॒सि॒ष्टꣳ॒॒ हा॒सि॒ष्ट॒म् मा॒ मा॒ हा॒सि॒ष्ट॒म् दे॒वता॑सु दे॒वता॑सु हासिष्टम् मा मा हासिष्टम् दे॒वता॑सु । \newline
53. हा॒सि॒ष्ट॒म् दे॒वता॑सु दे॒वता॑सु हासिष्टꣳ हासिष्टम् दे॒वता॑सु॒ वै वै दे॒वता॑सु हासिष्टꣳ हासिष्टम् दे॒वता॑सु॒ वै । \newline
54. दे॒वता॑सु॒ वै वै दे॒वता॑सु दे॒वता॑सु॒ वा ए॒त ए॒ते वै दे॒वता॑सु दे॒वता॑सु॒ वा ए॒ते । \newline
55. वा ए॒त ए॒ते वै वा ए॒ते प्रा॑णापा॒नयोः᳚ प्राणापा॒नयो॑ रे॒ते वै वा ए॒ते प्रा॑णापा॒नयोः᳚ । \newline
56. ए॒ते प्रा॑णापा॒नयोः᳚ प्राणापा॒नयो॑ रे॒त ए॒ते प्रा॑णापा॒नयो॒र् व्याय॑च्छन्ते॒ व्याय॑च्छन्ते प्राणापा॒नयो॑ रे॒त ए॒ते प्रा॑णापा॒नयो॒र् व्याय॑च्छन्ते । \newline
57. प्रा॒णा॒पा॒नयो॒र् व्याय॑च्छन्ते॒ व्याय॑च्छन्ते प्राणापा॒नयोः᳚ प्राणापा॒नयो॒र् व्याय॑च्छन्ते॒ येषां॒ ॅयेषां॒ ॅव्याय॑च्छन्ते प्राणापा॒नयोः᳚ प्राणापा॒नयो॒र् व्याय॑च्छन्ते॒ येषा᳚म् । \newline
58. प्रा॒णा॒पा॒नयो॒रिति॑ प्राण - अ॒पा॒नयोः᳚ । \newline
\pagebreak
\markright{ TS 3.1.7.2  \hfill https://www.vedavms.in \hfill}

\section{ TS 3.1.7.2 }

\textbf{TS 3.1.7.2 } \newline
\textbf{Samhita Paata} \newline

व्याय॑च्छन्ते॒ येषाꣳ॒॒ सोमः॑ समृ॒च्छते॑ संॅवे॒शाय॑ त्वोपवे॒शाय॒ त्वेत्या॑ह॒ छन्दाꣳ॑सि॒ वै सं॑ॅवे॒श उ॑पवे॒शश्छन्दो॑भिरे॒वास्य॒ छन्दाꣳ॑सि वृङ्क्ते॒ प्रेति॑व॒न्त्याज्या॑नि भवन्त्य॒भिजि॑त्यै म॒रुत्व॑तीः प्रति॒पदो॒ विजि॑त्या उ॒भे बृ॑हद्रथन्त॒रे भ॑वत इ॒यं ॅवाव र॑थन्त॒रम॒सौ बृ॒हदा॒भ्यामे॒वैन॑म॒न्तरे᳚त्य॒द्य वाव र॑थन्त॒रꣳ श्वो बृ॒हद॑द्या॒श्वा दे॒वैन॑म॒न्तरे॑ति भू॒तं - [  ] \newline

\textbf{Pada Paata} \newline

व्याय॑च्छन्त॒ इति॑ वि - आय॑च्छन्ते । येषा᳚म् । सोमः॑ । स॒मृ॒च्छत॒ इति॑ सं - ऋ॒च्छते᳚ । सं॒ॅवे॒शायेति॑ सं - वे॒शाय॑ । त्वा॒ । उ॒प॒वे॒शायेत्यु॑प - वे॒शाय॑ । त्वा॒ । इति॑ । आ॒ह॒ । छन्दाꣳ॑सि । वै । सं॒ॅवे॒श इति॑ सं - वे॒शः । उ॒प॒वे॒श इत्यु॑प - वे॒शः । छन्दो॑भि॒रिति॒ छन्दः॑ - भिः॒ । ए॒व । अ॒स्य॒ । छन्दाꣳ॑सि । वृ॒ङ्क्ते॒ । प्रेति॑व॒न्तीति॒ प्रेति॑ - व॒न्ति॒ । आज्या॑नि । भ॒व॒न्ति॒ । अ॒भिजि॑त्या॒ इत्य॒भि - जि॒त्यै॒ । म॒रुत्व॑तीः । प्र॒ति॒पद॒ इति॑ प्रति - पदः॑ । विजि॑त्या॒ इति॒ वि - जि॒त्यै॒ । उ॒भे इति॑ । बृ॒ह॒द्र॒थ॒न्त॒रे इति॑ बृहत् - र॒थ॒न्त॒रे । भ॒व॒तः॒ । इ॒यम् । वाव । र॒थ॒न्त॒रमिति॑ रथम्-त॒रम् । अ॒सौ । बृ॒हत् । आ॒भ्याम् । ए॒व । ए॒न॒म् । अ॒न्तः । ए॒ति॒ । अ॒द्य । वाव । र॒थ॒न्त॒रमिति॑ रथम् - त॒रम् । श्वः । बृ॒हत् । अ॒द्या॒श्वादित्य॑द्य - श्वात् । ए॒व । ए॒न॒म् । अ॒न्तः । ए॒ति॒ । भू॒तम् ।  \newline


\textbf{Krama Paata} \newline

व्याय॑च्छन्ते॒ येषा᳚म् । व्याय॑च्छन्त॒ इति॑ वि - आय॑च्छन्ते । येषाꣳ॒॒ सोमः॑ । सोमः॑ समृ॒च्छते᳚ । स॒मृ॒च्छते॑ सम्ॅवे॒शाय॑ । स॒मृ॒च्छत॒ इति॑ सम् - ऋ॒च्छते᳚ । स॒म्ॅवे॒शाय॑ त्वा । स॒म्ॅवे॒शायेति॑ सं - वे॒शाय॑ । त्वो॒प॒वे॒शाय॑ । उ॒प॒वे॒शाय॑ त्वा । उ॒प॒वे॒शायेत्यु॑प - वे॒शाय॑ । त्वेति॑ । इत्या॑ह । आ॒ह॒ छन्दाꣳ॑सि । छन्दाꣳ॑सि॒ वै । वै स॑म्ॅवे॒शः । स॒म्ॅवे॒श उ॑पवे॒शः । स॒म्ॅवे॒श इति॑ सं - वे॒शः । उ॒प॒वे॒श श्छन्दो॑भिः । उ॒प॒वे॒श इत्यु॑प - वे॒शः । छन्दो॑भिरे॒व । छन्दो॑भि॒रिति॒ छन्दः॑ - भिः॒ । ए॒वास्य॑ । अ॒स्य॒ छन्दाꣳ॑सि । छन्दाꣳ॑सि वृङ्क्ते । वृ॒ङ्क्ते॒ प्रेति॑वन्ति । प्रेति॑व॒न्त्याज्या॑नि । प्रेति॑व॒न्तीति॒ प्रेति॑ - व॒न्ति॒ । आज्या॑नि भवन्ति । भ॒व॒न्त्य॒भिजि॑त्यै । अ॒भिजि॑त्यै म॒रुत्व॑तीः । अ॒भिजि॑त्या॒ इत्य॒भि - जि॒त्यै॒ । म॒रुत्व॑तीः प्रति॒पदः॑ । प्र॒ति॒पदो॒ विजि॑त्यै । प्र॒ति॒पद॒ इति॑ प्रति - पदः॑ । विजि॑त्या उ॒भे । विजि॑त्या॒ इति॒ वि - जि॒त्यै॒ । उ॒भे बृ॑हद्रथन्त॒रे । उ॒भे इत्यु॒भे । बृ॒ह॒द्र॒थ॒न्त॒रे भ॑वतः । बृ॒ह॒द्र॒थ॒न्त॒रे इति॑ बृहत् - र॒थ॒न्त॒रे । भ॒व॒त॒ इ॒यम् । इ॒यं ॅवाव । वाव र॑थन्त॒रम् । र॒थ॒न्त॒रम॒सौ । र॒थ॒न्त॒रमिति॑ रथम् - त॒रम् । अ॒सौ बृ॒हत् । बृ॒हदा॒भ्याम् । आ॒भ्यामे॒व । ए॒वैन᳚म् । ए॒न॒म॒न्तः । अ॒न्तरे॑ति । ए॒त्य॒द्य । अ॒द्य वाव । वाव र॑थन्त॒रम् । र॒थ॒न्त॒रꣳ श्वः । र॒थ॒न्त॒रमिति॑ रथम् - त॒रम् । श्वो बृ॒हत् । बृ॒हद॑द्या॒श्वात् । अ॒द्या॒श्वादे॒व । अ॒द्या॒श्वादित्य॑द्य - श्वात् । ए॒वैन᳚म् । ए॒न॒म॒न्तः । अ॒न्तरे॑ति । ए॒ति॒ भू॒तम् । भू॒तं ॅवाव \newline

\textbf{Jatai Paata} \newline

1. व्याय॑च्छन्ते॒ येषां॒ ॅयेषां॒ ॅव्याय॑च्छन्ते॒ व्याय॑च्छन्ते॒ येषा᳚म् । \newline
2. व्याय॑च्छन्त॒ इति॑ वि - आय॑च्छन्ते । \newline
3. येषाꣳ॒॒ सोमः॒ सोमो॒ येषां॒ ॅयेषाꣳ॒॒ सोमः॑ । \newline
4. सोमः॑ समृ॒च्छते॑ समृ॒च्छते॒ सोमः॒ सोमः॑ समृ॒च्छते᳚ । \newline
5. स॒मृ॒च्छते॑ संॅवे॒शाय॑ संॅवे॒शाय॑ समृ॒च्छते॑ समृ॒च्छते॑ संॅवे॒शाय॑ । \newline
6. स॒मृ॒च्छत॒ इति॑ सं - ऋ॒च्छते᳚ । \newline
7. सं॒ॅवे॒शाय॑ त्वा त्वा संॅवे॒शाय॑ संॅवे॒शाय॑ त्वा । \newline
8. सं॒ॅवे॒शायेति॑ सं - वे॒शाय॑ । \newline
9. त्वो॒प॒वे॒शा यो॑पवे॒शाय॑ त्वा त्वोपवे॒शाय॑ । \newline
10. उ॒प॒वे॒शाय॑ त्वा त्वोपवे॒शा यो॑पवे॒शाय॑ त्वा । \newline
11. उ॒प॒वे॒शायेत्यु॑प - वे॒शाय॑ । \newline
12. त्वेतीति॑ त्वा॒ त्वेति॑ । \newline
13. इत्या॑हा॒हे तीत्या॑ह । \newline
14. आ॒ह॒ छन्दाꣳ॑सि॒ छन्दाꣳ॑ स्याहाह॒ छन्दाꣳ॑सि । \newline
15. छन्दाꣳ॑सि॒ वै वै छन्दाꣳ॑सि॒ छन्दाꣳ॑सि॒ वै । \newline
16. वै सं॑ॅवे॒शः सं॑ॅवे॒शो वै वै सं॑ॅवे॒शः । \newline
17. सं॒ॅवे॒श उ॑पवे॒श उ॑पवे॒शः सं॑ॅवे॒शः सं॑ॅवे॒श उ॑पवे॒शः । \newline
18. सं॒ॅवे॒श इति॑ सं - वे॒शः । \newline
19. उ॒प॒वे॒श श्छन्दो॑भि॒ श्छन्दो॑भि रुपवे॒श उ॑पवे॒श श्छन्दो॑भिः । \newline
20. उ॒प॒वे॒श इत्यु॑प - वे॒शः । \newline
21. छन्दो॑भि रे॒वैव छन्दो॑भि॒ श्छन्दो॑भि रे॒व । \newline
22. छन्दो॑भि॒रिति॒ छन्दः॑ - भिः॒ । \newline
23. ए॒वास्या᳚ स्यै॒वैवास्य॑ । \newline
24. अ॒स्य॒ छन्दाꣳ॑सि॒ छन्दाꣳ॑ स्य स्यास्य॒ छन्दाꣳ॑सि । \newline
25. छन्दाꣳ॑सि वृङ्क्ते वृङ्क्ते॒ छन्दाꣳ॑सि॒ छन्दाꣳ॑सि वृङ्क्ते । \newline
26. वृ॒ङ्क्ते॒ प्रेति॑वन्ति॒ प्रेति॑वन्ति वृङ्क्ते वृङ्क्ते॒ प्रेति॑वन्ति । \newline
27. प्रेति॑व॒ न्त्याज्या॒ न्याज्या॑नि॒ प्रेति॑वन्ति॒ प्रेति॑व॒ न्त्याज्या॑नि । \newline
28. प्रेति॑व॒न्तीति॒ प्रेति॑ - व॒न्ति॒ । \newline
29. आज्या॑नि भवन्ति भव॒ न्त्याज्या॒ न्याज्या॑नि भवन्ति । \newline
30. भ॒व॒ न्त्य॒भिजि॑त्या अ॒भिजि॑त्यै भवन्ति भव न्त्य॒भिजि॑त्यै । \newline
31. अ॒भिजि॑त्यै म॒रुत्व॑तीर् म॒रुत्व॑ती र॒भिजि॑त्या अ॒भिजि॑त्यै म॒रुत्व॑तीः । \newline
32. अ॒भिजि॑त्या॒ इत्य॒भि - जि॒त्यै॒ । \newline
33. म॒रुत्व॑तीः प्रति॒पदः॑ प्रति॒पदो॑ म॒रुत्व॑तीर् म॒रुत्व॑तीः प्रति॒पदः॑ । \newline
34. प्र॒ति॒पदो॒ विजि॑त्यै॒ विजि॑त्यै प्रति॒पदः॑ प्रति॒पदो॒ विजि॑त्यै । \newline
35. प्र॒ति॒पद॒ इति॑ प्रति - पदः॑ । \newline
36. विजि॑त्या उ॒भे उ॒भे विजि॑त्यै॒ विजि॑त्या उ॒भे । \newline
37. विजि॑त्या॒ इति॒ वि - जि॒त्यै॒ । \newline
38. उ॒भे बृ॑हद्रथन्त॒रे बृ॑हद्रथन्त॒रे उ॒भे उ॒भे बृ॑हद्रथन्त॒रे । \newline
39. उ॒भे इत्यु॒भे । \newline
40. बृ॒ह॒द्र॒थ॒न्त॒रे भ॑वतो भवतो बृहद्रथन्त॒रे बृ॑हद्रथन्त॒रे भ॑वतः । \newline
41. बृ॒ह॒द्र॒थ॒न्त॒रे इति॑ बृहत् - र॒थ॒न्त॒रे । \newline
42. भ॒व॒त॒ इ॒य मि॒यम् भ॑वतो भवत इ॒यम् । \newline
43. इ॒यं ॅवाव वावे य मि॒यं ॅवाव । \newline
44. वाव र॑थन्त॒रꣳ र॑थन्त॒रं ॅवाव वाव र॑थन्त॒रम् । \newline
45. र॒थ॒न्त॒र म॒सा व॒सौ र॑थन्त॒रꣳ र॑थन्त॒र म॒सौ । \newline
46. र॒थ॒न्त॒रमिति॑ रथम् - त॒रम् । \newline
47. अ॒सौ बृ॒हद् बृ॒ह द॒सा व॒सौ बृ॒हत् । \newline
48. बृ॒ह दा॒भ्या मा॒भ्याम् बृ॒हद् बृ॒ह दा॒भ्याम् । \newline
49. आ॒भ्या मे॒वै वाभ्या मा॒भ्या मे॒व । \newline
50. ए॒वैन॑ मेन मे॒वैवैन᳚म् । \newline
51. ए॒न॒ म॒न्त र॒न्त रे॑न मेन म॒न्तः । \newline
52. अ॒न्त रे᳚त्ये त्य॒न्त र॒न्त रे॑ति । \newline
53. ए॒त्य॒ द्याद्यै त्ये᳚त्य॒द्य । \newline
54. अ॒द्य वाव वावा द्याद्य वाव । \newline
55. वाव र॑थन्त॒रꣳ र॑थन्त॒रं ॅवाव वाव र॑थन्त॒रम् । \newline
56. र॒थ॒न्त॒रꣳ श्वः श्वो र॑थन्त॒रꣳ र॑थन्त॒रꣳ श्वः । \newline
57. र॒थ॒न्त॒रमिति॑ रथम् - त॒रम् । \newline
58. श्वो बृ॒हद् बृ॒हच् छ्‌वः श्वो बृ॒हत् । \newline
59. बृ॒ह द॑द्या॒श्वा द॑द्या॒ श्वाद् बृ॒हद् बृ॒हद॑द्या॒ श्वात् । \newline
60. अ॒द्या॒श्वा दे॒वै वाद्या॒ श्वा द॑द्या॒ श्वा दे॒व । \newline
61. अ॒द्या॒श्वादित्य॑द्य - श्वात् । \newline
62. ए॒वैन॑ मेन मे॒वैवैन᳚म् । \newline
63. ए॒न॒ म॒न्त र॒न्त रे॑न मेन म॒न्तः । \newline
64. अ॒न्त रे᳚त्ये त्य॒न्त र॒न्त रे॑ति । \newline
65. ए॒ति॒ भू॒तम् भू॒त मे᳚त्येति भू॒तम् । \newline
66. भू॒तं ॅवाव वाव भू॒तम् भू॒तं ॅवाव । \newline

\textbf{Ghana Paata } \newline

1. व्याय॑च्छन्ते॒ येषां॒ ॅयेषां॒ ॅव्याय॑च्छन्ते॒ व्याय॑च्छन्ते॒ येषाꣳ॒॒ सोमः॒ सोमो॒ येषां॒ ॅव्याय॑च्छन्ते॒ व्याय॑च्छन्ते॒ येषाꣳ॒॒ सोमः॑ । \newline
2. व्याय॑च्छन्त॒ इति॑ वि - आय॑च्छन्ते । \newline
3. येषाꣳ॒॒ सोमः॒ सोमो॒ येषां॒ ॅयेषाꣳ॒॒ सोमः॑ समृ॒च्छते॑ समृ॒च्छते॒ सोमो॒ येषां॒ ॅयेषाꣳ॒॒ सोमः॑ समृ॒च्छते᳚ । \newline
4. सोमः॑ समृ॒च्छते॑ समृ॒च्छते॒ सोमः॒ सोमः॑ समृ॒च्छते॑ संॅवे॒शाय॑ संॅवे॒शाय॑ समृ॒च्छते॒ सोमः॒ सोमः॑ समृ॒च्छते॑ संॅवे॒शाय॑ । \newline
5. स॒मृ॒च्छते॑ संॅवे॒शाय॑ संॅवे॒शाय॑ समृ॒च्छते॑ समृ॒च्छते॑ संॅवे॒शाय॑ त्वा त्वा संॅवे॒शाय॑ समृ॒च्छते॑ समृ॒च्छते॑ संॅवे॒शाय॑ त्वा । \newline
6. स॒मृ॒च्छत॒ इति॑ सं - ऋ॒च्छते᳚ । \newline
7. सं॒ॅवे॒शाय॑ त्वा त्वा संॅवे॒शाय॑ संॅवे॒शाय॑ त्वोपवे॒शा यो॑पवे॒शाय॑ त्वा संॅवे॒शाय॑ संॅवे॒शाय॑ त्वोपवे॒शाय॑ । \newline
8. सं॒ॅवे॒शायेति॑ सं - वे॒शाय॑ । \newline
9. त्वो॒प॒वे॒शा यो॑पवे॒शाय॑ त्वा त्वोपवे॒शाय॑ त्वा त्वोपवे॒शाय॑ त्वा त्वोपवे॒शाय॑ त्वा । \newline
10. उ॒प॒वे॒शाय॑ त्वा त्वोपवे॒शा यो॑पवे॒शाय॒ त्वेतीति॑ त्वोपवे॒शा यो॑पवे॒शाय॒ त्वेति॑ । \newline
11. उ॒प॒वे॒शायेत्यु॑प - वे॒शाय॑ । \newline
12. त्वेतीति॑ त्वा॒ त्वेत्या॑ हा॒हे ति॑ त्वा॒ त्वेत्या॑ह । \newline
13. इत्या॑हा॒हे तीत्या॑ह॒ छन्दाꣳ॑सि॒ छन्दाꣳ॑ स्या॒हे तीत्या॑ह॒ छन्दाꣳ॑सि । \newline
14. आ॒ह॒ छन्दाꣳ॑सि॒ छन्दाꣳ॑ स्याहाह॒ छन्दाꣳ॑सि॒ वै वै छन्दाꣳ॑ स्याहाह॒ छन्दाꣳ॑सि॒ वै । \newline
15. छन्दाꣳ॑सि॒ वै वै छन्दाꣳ॑सि॒ छन्दाꣳ॑सि॒ वै सं॑ॅवे॒शः सं॑ॅवे॒शो वै छन्दाꣳ॑सि॒ छन्दाꣳ॑सि॒ वै सं॑ॅवे॒शः । \newline
16. वै सं॑ॅवे॒शः सं॑ॅवे॒शो वै वै सं॑ॅवे॒श उ॑पवे॒श उ॑पवे॒शः सं॑ॅवे॒शो वै वै सं॑ॅवे॒श उ॑पवे॒शः । \newline
17. सं॒ॅवे॒श उ॑पवे॒श उ॑पवे॒शः सं॑ॅवे॒शः सं॑ॅवे॒श उ॑पवे॒श श्छन्दो॑भि॒ श्छन्दो॑भि रुपवे॒शः सं॑ॅवे॒शः सं॑ॅवे॒श उ॑पवे॒श श्छन्दो॑भिः । \newline
18. सं॒ॅवे॒श इति॑ सं - वे॒शः । \newline
19. उ॒प॒वे॒श श्छन्दो॑भि॒ श्छन्दो॑भि रुपवे॒श उ॑पवे॒श श्छन्दो॑भि रे॒वैव छन्दो॑भि रुपवे॒श उ॑पवे॒श श्छन्दो॑भि रे॒व । \newline
20. उ॒प॒वे॒श इत्यु॑प - वे॒शः । \newline
21. छन्दो॑भि रे॒वैव छन्दो॑भि॒ श्छन्दो॑भि रे॒वास्या᳚स्यै॒व छन्दो॑भि॒ श्छन्दो॑भि रे॒वास्य॑ । \newline
22. छन्दो॑भि॒रिति॒ छन्दः॑ - भिः॒ । \newline
23. ए॒वास्या᳚ स्यै॒वैवास्य॒ छन्दाꣳ॑सि॒ छन्दाꣳ॑ स्यस्यै॒ वैवास्य॒ छन्दाꣳ॑सि । \newline
24. अ॒स्य॒ छन्दाꣳ॑सि॒ छन्दाꣳ॑ स्यस्यास्य॒ छन्दाꣳ॑सि वृङ्क्ते वृङ्क्ते॒ छन्दाꣳ॑ स्यस्यास्य॒ छन्दाꣳ॑सि वृङ्क्ते । \newline
25. छन्दाꣳ॑सि वृङ्क्ते वृङ्क्ते॒ छन्दाꣳ॑सि॒ छन्दाꣳ॑सि वृङ्क्ते॒ प्रेति॑वन्ति॒ प्रेति॑वन्ति वृङ्क्ते॒ छन्दाꣳ॑सि॒ छन्दाꣳ॑सि वृङ्क्ते॒ प्रेति॑वन्ति । \newline
26. वृ॒ङ्क्ते॒ प्रेति॑वन्ति॒ प्रेति॑वन्ति वृङ्क्ते वृङ्क्ते॒ प्रेति॑व॒ न्त्याज्या॒ न्याज्या॑नि॒ प्रेति॑वन्ति वृङ्क्ते वृङ्क्ते॒ प्रेति॑व॒ न्त्याज्या॑नि । \newline
27. प्रेति॑व॒ न्त्याज्या॒ न्याज्या॑नि॒ प्रेति॑वन्ति॒ प्रेति॑व॒ न्त्याज्या॑नि भवन्ति भव॒ न्त्याज्या॑नि॒ प्रेति॑वन्ति॒ प्रेति॑व॒ न्त्याज्या॑नि भवन्ति । \newline
28. प्रेति॑व॒न्तीति॒ प्रेति॑ - व॒न्ति॒ । \newline
29. आज्या॑नि भवन्ति भव॒ न्त्याज्या॒ न्याज्या॑नि भव न्त्य॒भिजि॑त्या अ॒भिजि॑त्यै भव॒ न्त्याज्या॒ न्याज्या॑नि भव न्त्य॒भिजि॑त्यै । \newline
30. भ॒व॒ न्त्य॒भिजि॑त्या अ॒भिजि॑त्यै भवन्ति भव न्त्य॒भिजि॑त्यै म॒रुत्व॑तीर् म॒रुत्व॑ती र॒भिजि॑त्यै भवन्ति भव न्त्य॒भिजि॑त्यै म॒रुत्व॑तीः । \newline
31. अ॒भिजि॑त्यै म॒रुत्व॑तीर् म॒रुत्व॑ती र॒भिजि॑त्या अ॒भिजि॑त्यै म॒रुत्व॑तीः प्रति॒पदः॑ प्रति॒पदो॑ म॒रुत्व॑ती र॒भिजि॑त्या अ॒भिजि॑त्यै म॒रुत्व॑तीः प्रति॒पदः॑ । \newline
32. अ॒भिजि॑त्या॒ इत्य॒भि - जि॒त्यै॒ । \newline
33. म॒रुत्व॑तीः प्रति॒पदः॑ प्रति॒पदो॑ म॒रुत्व॑तीर् म॒रुत्व॑तीः प्रति॒पदो॒ विजि॑त्यै॒ विजि॑त्यै प्रति॒पदो॑ म॒रुत्व॑तीर् म॒रुत्व॑तीः प्रति॒पदो॒ विजि॑त्यै । \newline
34. प्र॒ति॒पदो॒ विजि॑त्यै॒ विजि॑त्यै प्रति॒पदः॑ प्रति॒पदो॒ विजि॑त्या उ॒भे उ॒भे विजि॑त्यै प्रति॒पदः॑ प्रति॒पदो॒ विजि॑त्या उ॒भे । \newline
35. प्र॒ति॒पद॒ इति॑ प्रति - पदः॑ । \newline
36. विजि॑त्या उ॒भे उ॒भे विजि॑त्यै॒ विजि॑त्या उ॒भे बृ॑हद्रथन्त॒रे बृ॑हद्रथन्त॒रे उ॒भे विजि॑त्यै॒ विजि॑त्या उ॒भे बृ॑हद्रथन्त॒रे । \newline
37. विजि॑त्या॒ इति॒ वि - जि॒त्यै॒ । \newline
38. उ॒भे बृ॑हद्रथन्त॒रे बृ॑हद्रथन्त॒रे उ॒भे उ॒भे बृ॑हद्रथन्त॒रे भ॑वतो भवतो बृहद्रथन्त॒रे उ॒भे उ॒भे बृ॑हद्रथन्त॒रे भ॑वतः । \newline
39. उ॒भे इत्यु॒भे । \newline
40. बृ॒ह॒द्र॒थ॒न्त॒रे भ॑वतो भवतो बृहद्रथन्त॒रे बृ॑हद्रथन्त॒रे भ॑वत इ॒य मि॒यम् भ॑वतो बृहद्रथन्त॒रे बृ॑हद्रथन्त॒रे भ॑वत इ॒यम् । \newline
41. बृ॒ह॒द्र॒थ॒न्त॒रे इति॑ बृहत् - र॒थ॒न्त॒रे । \newline
42. भ॒व॒त॒ इ॒य मि॒यम् भ॑वतो भवत इ॒यं ॅवाव वावे यम् भ॑वतो भवत इ॒यं ॅवाव । \newline
43. इ॒यं ॅवाव वावे य मि॒यं ॅवाव र॑थन्त॒रꣳ र॑थन्त॒रं ॅवावे य मि॒यं ॅवाव र॑थन्त॒रम् । \newline
44. वाव र॑थन्त॒रꣳ र॑थन्त॒रं ॅवाव वाव र॑थन्त॒र म॒सा व॒सौ र॑थन्त॒रं ॅवाव वाव र॑थन्त॒र म॒सौ । \newline
45. र॒थ॒न्त॒र म॒सा व॒सौ र॑थन्त॒रꣳ र॑थन्त॒र म॒सौ बृ॒हद् बृ॒हद॒सौ र॑थन्त॒रꣳ र॑थन्त॒र म॒सौ बृ॒हत् । \newline
46. र॒थ॒न्त॒रमिति॑ रथम् - त॒रम् । \newline
47. अ॒सौ बृ॒हद् बृ॒ह द॒सा व॒सौ बृ॒ह दा॒भ्या मा॒भ्याम् बृ॒हद॒सा व॒सौ बृ॒ह दा॒भ्याम् । \newline
48. बृ॒ह दा॒भ्या मा॒भ्याम् बृ॒हद् बृ॒ह दा॒भ्या मे॒वैवाभ्याम् बृ॒हद् बृ॒ह दा॒भ्या मे॒व । \newline
49. आ॒भ्या मे॒वैवाभ्या मा॒भ्या मे॒वैन॑ मेन मे॒वाभ्या मा॒भ्या मे॒वैन᳚म् । \newline
50. ए॒वैन॑ मेन मे॒वैवैन॑ म॒न्त र॒न्त रे॑न मे॒वैवैन॑ म॒न्तः । \newline
51. ए॒न॒ म॒न्त र॒न्त रे॑न मेन म॒न्त रे᳚त्ये त्य॒न्त रे॑न मेन म॒न्त रे॑ति । \newline
52. अ॒न्त रे᳚त्ये त्य॒न्त र॒न्त रे᳚त्य॒ द्या द्यैत्य॒न्त र॒न्त रे᳚त्य॒द्य । \newline
53. ए॒त्य॒द्या द्यैत्ये᳚ त्य॒द्य वाव वावाद्यैत्ये᳚ त्य॒द्य वाव । \newline
54. अ॒द्य वाव वावाद्याद्य वाव र॑थन्त॒रꣳ र॑थन्त॒रं ॅवावाद्याद्य वाव र॑थन्त॒रम् । \newline
55. वाव र॑थन्त॒रꣳ र॑थन्त॒रं ॅवाव वाव र॑थन्त॒रꣳ श्वः श्वो र॑थन्त॒रं ॅवाव वाव र॑थन्त॒रꣳ श्वः । \newline
56. र॒थ॒न्त॒रꣳ श्वः श्वो र॑थन्त॒रꣳ र॑थन्त॒रꣳ श्वो बृ॒हद् बृ॒हच्छ्वो र॑थन्त॒रꣳ र॑थन्त॒रꣳ श्वो बृ॒हत् । \newline
57. र॒थ॒न्त॒रमिति॑ रथम् - त॒रम् । \newline
58. श्वो बृ॒हद् बृ॒हच्छ्वः श्वो बृ॒ह द॑द्या॒श्वा द॑द्या॒श्वाद् बृ॒हच्छ्वः श्वो बृ॒ह द॑द्या॒श्वात् । \newline
59. बृ॒ह द॑द्या॒श्वा द॑द्या॒श्वाद् बृ॒हद् बृ॒ह द॑द्या॒श्वा दे॒वैवा द्या॒श्वाद् बृ॒हद् बृ॒ह द॑द्या॒श्वा दे॒व । \newline
60. अ॒द्या॒श्वा दे॒वैवाद्या॒श्वा द॑द्या॒श्वा दे॒वैन॑ मेन मे॒वाद्या॒श्वा द॑द्या॒श्वा दे॒वैन᳚म् । \newline
61. अ॒द्या॒श्वादित्य॑द्य - श्वात् । \newline
62. ए॒वैन॑ मेन मे॒वैवैन॑ म॒न्त र॒न्त रे॑न मे॒वैवैन॑ म॒न्तः । \newline
63. ए॒न॒ म॒न्त र॒न्त रे॑न मेन म॒न्त रे᳚त्ये त्य॒न्त रे॑न मेन म॒न्त रे॑ति । \newline
64. अ॒न्त रे᳚त्ये त्य॒न्त र॒न्त रे॑ति भू॒तम् भू॒त मे᳚त्य॒न्त र॒न्त रे॑ति भू॒तम् । \newline
65. ए॒ति॒ भू॒तम् भू॒त मे᳚त्येति भू॒तं ॅवाव वाव भू॒त मे᳚त्येति भू॒तं ॅवाव । \newline
66. भू॒तं ॅवाव वाव भू॒तम् भू॒तं ॅवाव र॑थन्त॒रꣳ र॑थन्त॒रं ॅवाव भू॒तम् भू॒तं ॅवाव र॑थन्त॒रम् । \newline
\pagebreak
\markright{ TS 3.1.7.3  \hfill https://www.vedavms.in \hfill}

\section{ TS 3.1.7.3 }

\textbf{TS 3.1.7.3 } \newline
\textbf{Samhita Paata} \newline

ॅवाव र॑थन्त॒रं भ॑वि॒ष्यद्-बृ॒॒हद् भू॒ताच्चै॒वैनं॑ भविष्य॒तश्चा॒न्तरे॑ति॒, परि॑मितं॒ ॅवाव र॑थन्त॒रमप॑रिमितं बृ॒हत् परि॑मिताच्चै॒वैन॒-मप॑रिमिताच्चा॒ऽन्तरे॑ति विश्वामित्रजमद॒ग्नी वसि॑ष्ठेनास्पर्द्धेताꣳ॒॒स ए॒तज्ज॒मद॑ग्नि र्विह॒व्य॑म पश्य॒त् तेन॒ वै स वसि॑ष्ठस्येन्द्रि॒यं ॅवी॒र्य॑मवृङ्क्त॒ यद्वि॑ह॒व्यꣳ॑ श॒स्यत॑ इन्द्रि॒यमे॒व तद्वी॒र्यं॑ ॅयज॑मानो॒ भ्रातृ॑व्यस्य वृङ्क्ते॒ ( ) यस्य॒ भूयाꣳ॑सो यज्ञ्क्र॒तव॒ इत्या॑हुः॒ स दे॒वता॑ वृङ्क्त॒ इति॒ यद्य॑ग्निष्टो॒मः सोमः॑ प॒रस्ता॒थ् स्या-दु॒क्थ्यं॑ कुर्वीत॒ यद्यु॒क्थ्यः॑ स्याद॑तिरा॒त्रं कु॑र्वीत यज्ञ्क्र॒तुभि॑रे॒वास्य॑ दे॒वता॑ वृङ्क्ते॒ वसी॑यान् भवति ॥ \newline

\textbf{Pada Paata} \newline

वाव । र॒थ॒न्त॒रमिति॑ रथम् - त॒रम् । भ॒वि॒ष्यत् । बृ॒हत् । भू॒तात् । च॒ । ए॒व । ए॒न॒म् । भ॒वि॒ष्य॒तः । च॒ । अ॒न्तः । ए॒ति॒ । परि॑मित॒मिति॒ परि॑ - मि॒त॒म् । वाव । र॒थ॒न्त॒रमिति॑ रथम् - त॒रम् । अप॑रिमित॒मित्यप॑रि - मि॒त॒म् । बृ॒हत् । परि॑मिता॒दिति॒ परि॑ - मि॒ता॒त् । च॒ । ए॒व । ए॒न॒म् । अप॑रिमिता॒दित्यप॑रि - मि॒ता॒त् । च॒ । अ॒न्तः । ए॒ति॒ । वि॒श्वा॒मि॒त्र॒ज॒म॒द॒ग्नी इति॑ विश्वामित्र - ज॒म॒द॒ग्नी । वसि॑ष्ठेन । अ॒स्प॒द्‌र्धे॒ता॒म् । सः । ए॒तत् । ज॒मद॑ग्निः । वि॒ह॒व्य॑मिति॑ वि - ह॒व्य᳚म् । अ॒प॒श्य॒त् । तेन॑ । वै । सः । वसि॑ष्ठस्य । इ॒न्द्रि॒यम् । वी॒र्य᳚म् । अ॒वृ॒ङ्क्त॒ । यत् । वि॒ह॒व्य॑मिति॑ वि - ह॒व्य᳚म् । श॒स्यते᳚ । इ॒न्द्रि॒यम् । ए॒व । तत् । वी॒र्य᳚म् । यज॑मानः । भ्रातृ॑व्यस्य । वृ॒ङ्क्ते॒ ( ) । यस्य॑ । भूयाꣳ॑सः । य॒ज्ञ्॒क्र॒तव॒ इति॑ यज्ञ् - क्र॒तवः॑ । इति॑ । आ॒हुः॒ । सः । दे॒वताः᳚ । वृ॒ङ्क्ते॒ । इति॑ । यदि॑ । अ॒ग्नि॒ष्टो॒म इत्य॑ग्नि-स्तो॒मः । सोमः॑ । प॒रस्ता᳚त् । स्यात् । उ॒क्थ्य᳚म् । कु॒र्वी॒त॒ । यदि॑ । उ॒क्थ्यः॑ । स्यात् । अ॒ति॒रा॒त्रमित्य॑ति - रा॒त्रम् । कु॒र्वी॒त॒ । य॒ज्ञ्॒क्र॒तुभि॒रिति॑ यज्ञ्क्र॒तु - भिः॒ । ए॒व । अ॒स्य॒ । दे॒वताः᳚ । वृ॒ङ्क्ते॒ । वसी॑यान् । भ॒व॒ति॒ ॥  \newline


\textbf{Krama Paata} \newline

वाव र॑थन्त॒रम् । र॒थ॒न्त॒रम् भ॑वि॒ष्यत् । र॒थ॒न्त॒रमिति॑ रथम् - त॒रम् । भ॒वि॒ष्यद् बृ॒हत् । बृ॒हद् भू॒तात् । भू॒ताच्च॑ । चै॒व । ए॒वैन᳚म् । ए॒न॒म् भ॒वि॒ष्य॒तः । भ॒वि॒ष्य॒तश्च॑ । चा॒न्तः । अ॒न्तरे॑ति । ए॒ति॒ परि॑मितम् । परि॑मितं॒ ॅवाव । परि॑मित॒मिति॒ परि॑ - मि॒त॒म् । वाव र॑थन्त॒रम् । र॒थ॒न्त॒रमप॑रिमितम् । र॒थ॒न्त॒रमिति॑ रथम् - त॒रम् । अप॑रिमितम् बृ॒हत् । अप॑रिमित॒मित्यप॑रि - मि॒त॒म् । बृ॒हत् परि॑मितात् । परि॑मिताच्च । परि॑मिता॒दिति॒ परि॑ - मि॒ता॒त्॒ । चै॒व । ए॒वैन᳚म् । ए॒न॒मप॑रिमितात् । अप॑रिमिताच्च । अप॑रिमिता॒दित्यप॑रि - मि॒ता॒त्॒ । चा॒न्तः । अ॒न्तरे॑ति । ए॒ति॒ वि॒श्वा॒मि॒त्र॒ज॒म॒द॒ग्नी । वि॒श्वा॒मि॒त्र॒ज॒म॒द॒ग्नी वसि॑ष्ठेन । वि॒श्वा॒मि॒त्र॒ज॒म॒द॒ग्नी इति॑ विश्वामित्र - ज॒म॒द॒ग्नी । वसि॑ष्ठेना,स्पर्द्धेताम् । अ॒स्प॒र्द्धे॒ताꣳ॒॒ सः । स ए॒तत् । ए॒तज् ज॒मद॑ग्निः । ज॒मद॑ग्निर् विह॒व्य᳚म् । वि॒ह॒व्य॑मपश्यत् । वि॒ह॒व्य॑मिति॑ वि - ह॒व्य᳚म् । अ॒प॒श्य॒त् तेन॑ । तेन॒ वै । वै सः । स वसि॑ष्ठस्य । वसि॑ष्ठस्येन्द्रि॒यम् । इ॒न्द्रि॒यं ॅवी॒र्य᳚म् । वी॒र्य॑मवृङ्क्त । अ॒वृ॒ङ्क्त॒ यत् । यद् वि॑ह॒व्य᳚म् । वि॒ह॒व्यꣳ॑ श॒स्यते᳚ । वि॒ह॒व्य॑मिति॑ वि - ह॒व्य᳚म् । श॒स्यत॑ इन्द्रि॒यम् । इ॒न्द्रि॒यमे॒व । ए॒व तत् । तद् वी॒र्य᳚म् । वी॒र्यं॑ ॅयज॑मानः । यज॑मानो॒ भ्रातृ॑व्यस्य । भ्रातृ॑व्यस्य वृङ्क्ते ( ) । वृ॒ङ्क्ते॒ यस्य॑ । यस्य॒ भूयाꣳ॑सः । भूयाꣳ॑सो यज्ञ्क्र॒तवः॑ । य॒ज्ञ्॒क्र॒तव॒ इति॑ । य॒ज्ञ्॒क्र॒तव॒ इति॑ यज्ञ् - क्र॒तवः॑ । इत्या॑हुः । आ॒हुः॒ सः । स दे॒वताः᳚ । दे॒वता॑ वृङ्क्ते । वृ॒ङ्क्त॒ इति॑ । इति॒ यदि॑ । यद्य॑ग्निष्टो॒मः । अ॒ग्नि॒ष्टो॒मः सोमः॑ । अ॒ग्नि॒ष्टो॒म इत्य॑ग्नि - स्तो॒मः । सोमः॑ प॒रस्ता᳚त् । प॒रस्ता॒थ् स्यात् । स्यादु॒क्थ्य᳚म् । उ॒क्थ्य॑म् कुर्वीत । कु॒र्वी॒त॒ यदि॑ । यद्यु॒क्थ्यः॑ । उ॒क्थ्यः॑ स्यात् । स्याद॑तिरा॒त्रम् । अ॒ति॒रा॒त्रम् कु॑र्वीत । अ॒ति॒रा॒त्रमित्य॑ति - रा॒त्रम् । कु॒र्वी॒त॒ य॒ज्ञ्॒क्र॒तुभिः॑ । य॒ज्ञ्॒क्र॒तुभि॑रे॒व । य॒ज्ञ्॒क्र॒तुभि॒रिति॑ यज्ञ्क्र॒तु - भिः॒ । ए॒वास्य॑ । अ॒स्य॒ दे॒वताः᳚ । दे॒वता॑ वृङ्क्ते । वृ॒ङ्क्ते॒ वसी॑यान् । वसी॑यान् भवति । भ॒व॒तीति॑ भवति । \newline

\textbf{Jatai Paata} \newline

1. वाव र॑थन्त॒रꣳ र॑थन्त॒रं ॅवाव वाव र॑थन्त॒रम् । \newline
2. र॒थ॒न्त॒रम् भ॑वि॒ष्यद् भ॑वि॒ष्यद् र॑थन्त॒रꣳ र॑थन्त॒रम् भ॑वि॒ष्यत् । \newline
3. र॒थ॒न्त॒रमिति॑ रथम् - त॒रम् । \newline
4. भ॒वि॒ष्यद् बृ॒हद् बृ॒हद् भ॑वि॒ष्यद् भ॑वि॒ष्यद् बृ॒हत् । \newline
5. बृ॒हद् भू॒ताद् भू॒ताद् बृ॒हद् बृ॒हद् भू॒तात् । \newline
6. भू॒ताच् च॑ च भू॒ताद् भू॒ताच् च॑ । \newline
7. चै॒वैव च॑ चै॒व । \newline
8. ए॒वैन॑ मेन मे॒वैवैन᳚म् । \newline
9. ए॒न॒म् भ॒वि॒ष्य॒तो भ॑विष्य॒त ए॑न मेनम् भविष्य॒तः । \newline
10. भ॒वि॒ष्य॒तश्च॑ च भविष्य॒तो भ॑विष्य॒तश्च॑ । \newline
11. चा॒न्त र॒न्तश्च॑ चा॒न्तः । \newline
12. अ॒न्त रे᳚ त्येत्य॒न्त र॒न्त रे॑ति । \newline
13. ए॒ति॒ परि॑मित॒म् परि॑मित मेत्येति॒ परि॑मितम् । \newline
14. परि॑मितं॒ ॅवाव वाव परि॑मित॒म् परि॑मितं॒ ॅवाव । \newline
15. परि॑मित॒मिति॒ परि॑ - मि॒त॒म् । \newline
16. वाव र॑थन्त॒रꣳ र॑थन्त॒रं ॅवाव वाव र॑थन्त॒रम् । \newline
17. र॒थ॒न्त॒र मप॑रिमित॒ मप॑रिमितꣳ रथन्त॒रꣳ र॑थन्त॒र मप॑रिमितम् । \newline
18. र॒थ॒न्त॒रमिति॑ रथम् - त॒रम् । \newline
19. अप॑रिमितम् बृ॒हद् बृ॒ह दप॑रिमित॒ मप॑रिमितम् बृ॒हत् । \newline
20. अप॑रिमित॒मित्यप॑रि - मि॒त॒म् । \newline
21. बृ॒हत् परि॑मिता॒त् परि॑मिताद् बृ॒हद् बृ॒हत् परि॑मितात् । \newline
22. परि॑मिताच् च च॒ परि॑मिता॒त् परि॑मिताच् च । \newline
23. परि॑मिता॒दिति॒ परि॑ - मि॒ता॒त् । \newline
24. चै॒वैव च॑ चै॒व । \newline
25. ए॒वैन॑ मेन मे॒वैवैन᳚म् । \newline
26. ए॒न॒ मप॑रिमिता॒ दप॑रिमिता देन मेन॒ मप॑रिमितात् । \newline
27. अप॑रिमिताच् च॒ चाप॑रिमिता॒ दप॑रिमिताच् च । \newline
28. अप॑रिमिता॒दित्यप॑रि - मि॒ता॒त् । \newline
29. चा॒न्त र॒न्तश्च॑ चा॒न्तः । \newline
30. अ॒न्त रे᳚ त्येत्य॒न्त र॒न्त रे॑ति । \newline
31. ए॒ति॒ वि॒श्वा॒मि॒त्र॒ज॒म॒द॒ग्नी वि॑श्वामित्रजमद॒ग्नी ए᳚त्येति विश्वामित्रजमद॒ग्नी । \newline
32. वि॒श्वा॒मि॒त्र॒ज॒म॒द॒ग्नी वसि॑ष्ठेन॒ वसि॑ष्ठेन विश्वामित्रजमद॒ग्नी वि॑श्वामित्रजमद॒ग्नी वसि॑ष्ठेन । \newline
33. वि॒श्वा॒मि॒त्र॒ज॒म॒द॒ग्नी इति॑ विश्वामित्र - ज॒म॒द॒ग्नी । \newline
34. वसि॑ष्ठेना स्पर्द्धेता मस्पर्द्धेतां॒ ॅवसि॑ष्ठेन॒ वसि॑ष्ठेना स्पर्द्धेताम् । \newline
35. अ॒स्प॒र्द्धे॒ताꣳ॒॒ स सो᳚ ऽस्पर्द्धेता मस्पर्द्धेताꣳ॒॒ सः । \newline
36. स ए॒त दे॒तथ् स स ए॒तत् । \newline
37. ए॒तज् ज॒मद॑ग्निर् ज॒मद॑ग्नि रे॒त दे॒तज् ज॒मद॑ग्निः । \newline
38. ज॒मद॑ग्निर् विह॒व्यं॑ ॅविह॒व्य॑म् ज॒मद॑ग्निर् ज॒मद॑ग्निर् विह॒व्य᳚म् । \newline
39. वि॒ह॒व्य॑ मपश्य दपश्यद् विह॒व्यं॑ ॅविह॒व्य॑ मपश्यत् । \newline
40. वि॒ह॒व्य॑मिति॑ वि - ह॒व्य᳚म् । \newline
41. अ॒प॒श्य॒त् तेन॒ तेना॑पश्य दपश्य॒त् तेन॑ । \newline
42. तेन॒ वै वै तेन॒ तेन॒ वै । \newline
43. वै स स वै वै सः । \newline
44. स वसि॑ष्ठस्य॒ वसि॑ष्ठस्य॒ स स वसि॑ष्ठस्य । \newline
45. वसि॑ष्ठस्ये न्द्रि॒य मि॑न्द्रि॒यं ॅवसि॑ष्ठस्य॒ वसि॑ष्ठस्ये न्द्रि॒यम् । \newline
46. इ॒न्द्रि॒यं ॅवी॒र्यं॑ ॅवी॒र्य॑ मिन्द्रि॒य मि॑न्द्रि॒यं ॅवी॒र्य᳚म् । \newline
47. वी॒र्य॑ मवृङ्क्ता वृङ्क्त वी॒र्यं॑ ॅवी॒र्य॑ मवृङ्क्त । \newline
48. अ॒वृ॒ङ्क्त॒ यद् यद॑वृङ्क्ता वृङ्क्त॒ यत् । \newline
49. यद् वि॑ह॒व्यं॑ ॅविह॒व्यं॑ ॅयद् यद् वि॑ह॒व्य᳚म् । \newline
50. वि॒ह॒व्यꣳ॑ श॒स्यते॑ श॒स्यते॑ विह॒व्यं॑ ॅविह॒व्यꣳ॑ श॒स्यते᳚ । \newline
51. वि॒ह॒व्य॑मिति॑ वि - ह॒व्य᳚म् । \newline
52. श॒स्यत॑ इन्द्रि॒य मि॑न्द्रि॒यꣳ श॒स्यते॑ श॒स्यत॑ इन्द्रि॒यम् । \newline
53. इ॒न्द्रि॒य मे॒वैवे न्द्रि॒य मि॑न्द्रि॒य मे॒व । \newline
54. ए॒व तत् तदे॒वैव तत् । \newline
55. तद् वी॒र्यं॑ ॅवी॒र्य॑म् तत् तद् वी॒र्य᳚म् । \newline
56. वी॒र्यं॑ ॅयज॑मानो॒ यज॑मानो वी॒र्यं॑ ॅवी॒र्यं॑ ॅयज॑मानः । \newline
57. यज॑मानो॒ भ्रातृ॑व्यस्य॒ भ्रातृ॑व्यस्य॒ यज॑मानो॒ यज॑मानो॒ भ्रातृ॑व्यस्य । \newline
58. भ्रातृ॑व्यस्य वृङ्क्ते वृङ्क्ते॒ भ्रातृ॑व्यस्य॒ भ्रातृ॑व्यस्य वृङ्क्ते । \newline
59. वृ॒ङ्क्ते॒ यस्य॒ यस्य॑ वृङ्क्ते वृङ्क्ते॒ यस्य॑ । \newline
60. यस्य॒ भूयाꣳ॑सो॒ भूयाꣳ॑सो॒ यस्य॒ यस्य॒ भूयाꣳ॑सः । \newline
61. भूयाꣳ॑सो यज्ञ्क्र॒तवो॑ यज्ञ्क्र॒तवो॒ भूयाꣳ॑सो॒ भूयाꣳ॑सो यज्ञ्क्र॒तवः॑ । \newline
62. य॒ज्ञ्॒क्र॒तव॒ इतीति॑ यज्ञ्क्र॒तवो॑ यज्ञ्क्र॒तव॒ इति॑ । \newline
63. य॒ज्ञ्॒क्र॒तव॒ इति॑ यज्ञ् - क्र॒तवः॑ । \newline
64. इत्या॑हु राहु॒ रिती त्या॑हुः । \newline
65. आ॒हुः॒ स स आ॑हु राहुः॒ सः । \newline
66. स दे॒वता॑ दे॒वताः॒ स स दे॒वताः᳚ । \newline
67. दे॒वता॑ वृङ्क्ते वृङ्क्ते दे॒वता॑ दे॒वता॑ वृङ्क्ते । \newline
68. वृ॒ङ्क्त॒ इतीति॑ वृङ्क्ते वृङ्क्त॒ इति॑ । \newline
69. इति॒ यदि॒ यदीतीति॒ यदि॑ । \newline
70. यद्य॑ग्निष्टो॒मो᳚ ऽग्निष्टो॒मो यदि॒ यद्य॑ग्निष्टो॒मः । \newline
71. अ॒ग्नि॒ष्टो॒मः सोमः॒ सोमो॑ अग्निष्टो॒मो᳚ ऽग्निष्टो॒मः सोमः॑ । \newline
72. अ॒ग्नि॒ष्टो॒म इत्य॑ग्नि - स्तो॒मः । \newline
73. सोमः॑ प॒रस्ता᳚त् प॒रस्ता॒थ् सोमः॒ सोमः॑ प॒रस्ता᳚त् । \newline
74. प॒रस्ता॒थ् स्याथ् स्यात् प॒रस्ता᳚त् प॒रस्ता॒थ् स्यात् । \newline
75. स्या दु॒क्थ्य॑ मु॒क्थ्यꣳ॑ स्याथ् स्या दु॒क्थ्य᳚म् । \newline
76. उ॒क्थ्य॑म् कुर्वीत कुर्वी तो॒क्थ्य॑ मु॒क्थ्य॑म् कुर्वीत । \newline
77. कु॒र्वी॒त॒ यदि॒ यदि॑ कुर्वीत कुर्वीत॒ यदि॑ । \newline
78. यद्यु॒क्थ्य॑ उ॒क्थ्यो॑ यदि॒ यद्यु॒क्थ्यः॑ । \newline
79. उ॒क्थ्यः॑ स्याथ् स्या दु॒क्थ्य॑ उ॒क्थ्यः॑ स्यात् । \newline
80. स्या द॑तिरा॒त्र म॑तिरा॒त्रꣳ स्याथ् स्या द॑तिरा॒त्रम् । \newline
81. अ॒ति॒रा॒त्रम् कु॑र्वीत कुर्वीतातिरा॒त्र म॑तिरा॒त्रम् कु॑र्वीत । \newline
82. अ॒ति॒रा॒त्रमित्य॑ति - रा॒त्रम् । \newline
83. कु॒र्वी॒त॒ य॒ज्ञ्॒क्र॒तुभि॑र् यज्ञ्क्र॒तुभिः॑ कुर्वीत कुर्वीत यज्ञ्क्र॒तुभिः॑ । \newline
84. य॒ज्ञ्॒क्र॒तुभि॑ रे॒वैव य॑ज्ञ्क्र॒तुभि॑र् यज्ञ्क्र॒तुभि॑ रे॒व । \newline
85. य॒ज्ञ्॒क्र॒तुभि॒रिति॑ यज्ञ्क्र॒तु - भिः॒ । \newline
86. ए॒वा स्या᳚ स्यै॒ वैवास्य॑ । \newline
87. अ॒स्य॒ दे॒वता॑ दे॒वता॑ अस्यास्य दे॒वताः᳚ । \newline
88. दे॒वता॑ वृङ्क्ते वृङ्क्ते दे॒वता॑ दे॒वता॑ वृङ्क्ते । \newline
89. वृ॒ङ्क्ते॒ वसी॑या॒न्॒. वसी॑यान् वृङ्क्ते वृङ्क्ते॒ वसी॑यान् । \newline
90. वसी॑यान् भवति भवति॒ वसी॑या॒न्॒. वसी॑यान् भवति । \newline
91. भ॒व॒तीति॑ भवति । \newline

\textbf{Ghana Paata } \newline

1. वाव र॑थन्त॒रꣳ र॑थन्त॒रं ॅवाव वाव र॑थन्त॒रम् भ॑वि॒ष्यद् भ॑वि॒ष्यद् र॑थन्त॒रं ॅवाव वाव र॑थन्त॒रम् भ॑वि॒ष्यत् । \newline
2. र॒थ॒न्त॒रम् भ॑वि॒ष्यद् भ॑वि॒ष्यद् र॑थन्त॒रꣳ र॑थन्त॒रम् भ॑वि॒ष्यद् बृ॒हद् बृ॒हद् भ॑वि॒ष्यद् र॑थन्त॒रꣳ र॑थन्त॒रम् भ॑वि॒ष्यद् बृ॒हत् । \newline
3. र॒थ॒न्त॒रमिति॑ रथम् - त॒रम् । \newline
4. भ॒वि॒ष्यद् बृ॒हद् बृ॒हद् भ॑वि॒ष्यद् भ॑वि॒ष्यद् बृ॒हद् भू॒ताद् भू॒ताद् बृ॒हद् भ॑वि॒ष्यद् भ॑वि॒ष्यद् बृ॒हद् भू॒तात् । \newline
5. बृ॒हद् भू॒ताद् भू॒ताद् बृ॒हद् बृ॒हद् भू॒ताच् च॑ च भू॒ताद् बृ॒हद् बृ॒हद् भू॒ताच् च॑ । \newline
6. भू॒ताच् च॑ च भू॒ताद् भू॒ताच् चै॒वैव च॑ भू॒ताद् भू॒ताच् चै॒व । \newline
7. चै॒वैव च॑ चै॒वैन॑ मेन मे॒व च॑ चै॒वैन᳚म् । \newline
8. ए॒वैन॑ मेन मे॒वैवैन॑म् भविष्य॒तो भ॑विष्य॒त ए॑न मे॒वैवैन॑म् भविष्य॒तः । \newline
9. ए॒न॒म् भ॒वि॒ष्य॒तो भ॑विष्य॒त ए॑न मेनम् भविष्य॒तश्च॑ च भविष्य॒त ए॑न मेनम् भविष्य॒तश्च॑ । \newline
10. भ॒वि॒ष्य॒तश्च॑ च भविष्य॒तो भ॑विष्य॒त श्चा॒न्त र॒न्तश्च॑ भविष्य॒तो भ॑विष्य॒त श्चा॒न्तः । \newline
11. चा॒न्त र॒न्तश्च॑ चा॒न्त रे᳚त्ये त्य॒न्तश्च॑ चा॒न्त रे॑ति । \newline
12. अ॒न्त रे᳚त्ये त्य॒न्त र॒न्त रे॑ति॒ परि॑मित॒म् परि॑मित मेत्य॒न्त र॒न्त रे॑ति॒ परि॑मितम् । \newline
13. ए॒ति॒ परि॑मित॒म् परि॑मित मेत्येति॒ परि॑मितं॒ ॅवाव वाव परि॑मित मेत्येति॒ परि॑मितं॒ ॅवाव । \newline
14. परि॑मितं॒ ॅवाव वाव परि॑मित॒म् परि॑मितं॒ ॅवाव र॑थन्त॒रꣳ र॑थन्त॒रं ॅवाव परि॑मित॒म् परि॑मितं॒ ॅवाव र॑थन्त॒रम् । \newline
15. परि॑मित॒मिति॒ परि॑ - मि॒त॒म् । \newline
16. वाव र॑थन्त॒रꣳ र॑थन्त॒रं ॅवाव वाव र॑थन्त॒र मप॑रिमित॒ मप॑रिमितꣳ रथन्त॒रं ॅवाव वाव र॑थन्त॒र मप॑रिमितम् । \newline
17. र॒थ॒न्त॒र मप॑रिमित॒ मप॑रिमितꣳ रथन्त॒रꣳ र॑थन्त॒र मप॑रिमितम् बृ॒हद् बृ॒ह दप॑रिमितꣳ रथन्त॒रꣳ र॑थन्त॒र मप॑रिमितम् बृ॒हत् । \newline
18. र॒थ॒न्त॒रमिति॑ रथम् - त॒रम् । \newline
19. अप॑रिमितम् बृ॒हद् बृ॒ह दप॑रिमित॒ मप॑रिमितम् बृ॒हत् परि॑मिता॒त् परि॑मिताद् बृ॒ह दप॑रिमित॒ मप॑रिमितम् बृ॒हत् परि॑मितात् । \newline
20. अप॑रिमित॒मित्यप॑रि - मि॒त॒म् । \newline
21. बृ॒हत् परि॑मिता॒त् परि॑मिताद् बृ॒हद् बृ॒हत् परि॑मिताच् च च॒ परि॑मिताद् बृ॒हद् बृ॒हत् परि॑मिताच् च । \newline
22. परि॑मिताच् च च॒ परि॑मिता॒त् परि॑मिताच् चै॒वैव च॒ परि॑मिता॒त् परि॑मिताच् चै॒व । \newline
23. परि॑मिता॒दिति॒ परि॑ - मि॒ता॒त् । \newline
24. चै॒वैव च॑ चै॒वैन॑ मेन मे॒व च॑ चै॒वैन᳚म् । \newline
25. ए॒वैन॑ मेन मे॒वैवैन॒ मप॑रिमिता॒ दप॑रिमिता देन मे॒वैवैन॒ मप॑रिमितात् । \newline
26. ए॒न॒ मप॑रिमिता॒ दप॑रिमिता देन मेन॒ मप॑रिमिताच् च॒ चाप॑रिमिता देन मेन॒ मप॑रिमिताच् च । \newline
27. अप॑रिमिताच् च॒ चाप॑रिमिता॒ दप॑रिमिताच् चा॒न्त र॒न्त श्चाप॑रिमिता॒ दप॑रिमिताच् चा॒न्तः । \newline
28. अप॑रिमिता॒दित्यप॑रि - मि॒ता॒त् । \newline
29. चा॒न्त र॒न्तश्च॑ चा॒न्त रे᳚त्ये त्य॒न्तश्च॑ चा॒न्त रे॑ति । \newline
30. अ॒न्त रे᳚त्येत्य॒न्त र॒न्त रे॑ति विश्वामित्रजमद॒ग्नी वि॑श्वामित्रजमद॒ग्नी ए᳚त्य॒न्त र॒न्त रे॑ति विश्वामित्रजमद॒ग्नी । \newline
31. ए॒ति॒ वि॒श्वा॒मि॒त्र॒ज॒म॒द॒ग्नी वि॑श्वामित्रजमद॒ग्नी ए᳚त्येति विश्वामित्रजमद॒ग्नी वसि॑ष्ठेन॒ वसि॑ष्ठेन विश्वामित्रजमद॒ग्नी ए᳚त्येति विश्वामित्रजमद॒ग्नी वसि॑ष्ठेन । \newline
32. वि॒श्वा॒मि॒त्र॒ज॒म॒द॒ग्नी वसि॑ष्ठेन॒ वसि॑ष्ठेन विश्वामित्रजमद॒ग्नी वि॑श्वामित्रजमद॒ग्नी वसि॑ष्ठेना स्पर्द्धेता मस्पर्द्धेतां॒ ॅवसि॑ष्ठेन विश्वामित्रजमद॒ग्नी वि॑श्वामित्रजमद॒ग्नी वसि॑ष्ठेना स्पर्द्धेताम् । \newline
33. वि॒श्वा॒मि॒त्र॒ज॒म॒द॒ग्नी इति॑ विश्वामित्र - ज॒म॒द॒ग्नी । \newline
34. वसि॑ष्ठेना स्पर्द्धेता मस्पर्द्धेतां॒ ॅवसि॑ष्ठेन॒ वसि॑ष्ठेना स्पर्द्धेताꣳ॒॒ स सो᳚ ऽस्पर्द्धेतां॒ ॅवसि॑ष्ठेन॒ वसि॑ष्ठेना स्पर्द्धेताꣳ॒॒ सः । \newline
35. अ॒स्प॒र्द्धे॒ताꣳ॒॒ स सो᳚ ऽस्पर्द्धेता मस्पर्द्धेताꣳ॒॒ स ए॒त दे॒तथ् सो᳚ ऽस्पर्द्धेता मस्पर्द्धेताꣳ॒॒ स ए॒तत् । \newline
36. स ए॒तदे॒तथ् स स ए॒तज् ज॒मद॑ग्निर् ज॒मद॑ग्नि रे॒तथ् स स ए॒तज् ज॒मद॑ग्निः । \newline
37. ए॒तज् ज॒मद॑ग्निर् ज॒मद॑ग्नि रे॒त दे॒तज् ज॒मद॑ग्निर् विह॒व्यं॑ ॅविह॒व्य॑म् ज॒मद॑ग्नि रे॒त दे॒तज् ज॒मद॑ग्निर् विह॒व्य᳚म् । \newline
38. ज॒मद॑ग्निर् विह॒व्यं॑ ॅविह॒व्य॑म् ज॒मद॑ग्निर् ज॒मद॑ग्निर् विह॒व्य॑ मपश्य दपश्यद् विह॒व्य॑म् ज॒मद॑ग्निर् ज॒मद॑ग्निर् विह॒व्य॑ मपश्यत् । \newline
39. वि॒ह॒व्य॑ मपश्य दपश्यद् विह॒व्यं॑ ॅविह॒व्य॑ मपश्य॒त् तेन॒ तेना॑ पश्यद् विह॒व्यं॑ ॅविह॒व्य॑ मपश्य॒त् तेन॑ । \newline
40. वि॒ह॒व्य॑मिति॑ वि - ह॒व्य᳚म् । \newline
41. अ॒प॒श्य॒त् तेन॒ तेना॑पश्य दपश्य॒त् तेन॒ वै वै तेना॑पश्य दपश्य॒त् तेन॒ वै । \newline
42. तेन॒ वै वै तेन॒ तेन॒ वै स स वै तेन॒ तेन॒ वै सः । \newline
43. वै स स वै वै स वसि॑ष्ठस्य॒ वसि॑ष्ठस्य॒ स वै वै स वसि॑ष्ठस्य । \newline
44. स वसि॑ष्ठस्य॒ वसि॑ष्ठस्य॒ स स वसि॑ष्ठस्ये न्द्रि॒य मि॑न्द्रि॒यं ॅवसि॑ष्ठस्य॒ स स वसि॑ष्ठस्ये न्द्रि॒यम् । \newline
45. वसि॑ष्ठस्ये न्द्रि॒य मि॑न्द्रि॒यं ॅवसि॑ष्ठस्य॒ वसि॑ष्ठस्ये न्द्रि॒यं ॅवी॒र्यं॑ ॅवी॒र्य॑ मिन्द्रि॒यं ॅवसि॑ष्ठस्य॒ वसि॑ष्ठस्ये न्द्रि॒यं ॅवी॒र्य᳚म् । \newline
46. इ॒न्द्रि॒यं ॅवी॒र्यं॑ ॅवी॒र्य॑ मिन्द्रि॒य मि॑न्द्रि॒यं ॅवी॒र्य॑ मवृङ्क्तावृङ्क्त वी॒र्य॑ मिन्द्रि॒य मि॑न्द्रि॒यं ॅवी॒र्य॑ मवृङ्क्त । \newline
47. वी॒र्य॑ मवृङ्क्तावृङ्क्त वी॒र्यं॑ ॅवी॒र्य॑ मवृङ्क्त॒ यद् यद॑वृङ्क्त वी॒र्यं॑ ॅवी॒र्य॑ मवृङ्क्त॒ यत् । \newline
48. अ॒वृ॒ङ्क्त॒ यद् यद॑वृङ्क्ता वृङ्क्त॒ यद् वि॑ह॒व्यं॑ ॅविह॒व्यं॑ ॅयद॑वृङ्क्ता वृङ्क्त॒ यद् वि॑ह॒व्य᳚म् । \newline
49. यद् वि॑ह॒व्यं॑ ॅविह॒व्यं॑ ॅयद् यद् वि॑ह॒व्यꣳ॑ श॒स्यते॑ श॒स्यते॑ विह॒व्यं॑ ॅयद् यद् वि॑ह॒व्यꣳ॑ श॒स्यते᳚ । \newline
50. वि॒ह॒व्यꣳ॑ श॒स्यते॑ श॒स्यते॑ विह॒व्यं॑ ॅविह॒व्यꣳ॑ श॒स्यत॑ इन्द्रि॒य मि॑न्द्रि॒यꣳ श॒स्यते॑ विह॒व्यं॑ ॅविह॒व्यꣳ॑ श॒स्यत॑ इन्द्रि॒यम् । \newline
51. वि॒ह॒व्य॑मिति॑ वि - ह॒व्य᳚म् । \newline
52. श॒स्यत॑ इन्द्रि॒य मि॑न्द्रि॒यꣳ श॒स्यते॑ श॒स्यत॑ इन्द्रि॒य मे॒वैवे न्द्रि॒यꣳ श॒स्यते॑ श॒स्यत॑ इन्द्रि॒य मे॒व । \newline
53. इ॒न्द्रि॒य मे॒वैवे न्द्रि॒य मि॑न्द्रि॒य मे॒व तत् तदे॒वे न्द्रि॒य मि॑न्द्रि॒य मे॒व तत् । \newline
54. ए॒व तत् तदे॒वैव तद् वी॒र्यं॑ ॅवी॒र्य॑म् तदे॒वैव तद् वी॒र्य᳚म् । \newline
55. तद् वी॒र्यं॑ ॅवी॒र्य॑म् तत् तद् वी॒र्यं॑ ॅयज॑मानो॒ यज॑मानो वी॒र्य॑म् तत् तद् वी॒र्यं॑ ॅयज॑मानः । \newline
56. वी॒र्यं॑ ॅयज॑मानो॒ यज॑मानो वी॒र्यं॑ ॅवी॒र्यं॑ ॅयज॑मानो॒ भ्रातृ॑व्यस्य॒ भ्रातृ॑व्यस्य॒ यज॑मानो वी॒र्यं॑ ॅवी॒र्यं॑ ॅयज॑मानो॒ भ्रातृ॑व्यस्य । \newline
57. यज॑मानो॒ भ्रातृ॑व्यस्य॒ भ्रातृ॑व्यस्य॒ यज॑मानो॒ यज॑मानो॒ भ्रातृ॑व्यस्य वृङ्क्ते वृङ्क्ते॒ भ्रातृ॑व्यस्य॒ यज॑मानो॒ यज॑मानो॒ भ्रातृ॑व्यस्य वृङ्क्ते । \newline
58. भ्रातृ॑व्यस्य वृङ्क्ते वृङ्क्ते॒ भ्रातृ॑व्यस्य॒ भ्रातृ॑व्यस्य वृङ्क्ते॒ यस्य॒ यस्य॑ वृङ्क्ते॒ भ्रातृ॑व्यस्य॒ भ्रातृ॑व्यस्य वृङ्क्ते॒ यस्य॑ । \newline
59. वृ॒ङ्क्ते॒ यस्य॒ यस्य॑ वृङ्क्ते वृङ्क्ते॒ यस्य॒ भूयाꣳ॑सो॒ भूयाꣳ॑सो॒ यस्य॑ वृङ्क्ते वृङ्क्ते॒ यस्य॒ भूयाꣳ॑सः । \newline
60. यस्य॒ भूयाꣳ॑सो॒ भूयाꣳ॑सो॒ यस्य॒ यस्य॒ भूयाꣳ॑सो यज्ञ्क्र॒तवो॑ यज्ञ्क्र॒तवो॒ भूयाꣳ॑सो॒ यस्य॒ यस्य॒ भूयाꣳ॑सो यज्ञ्क्र॒तवः॑ । \newline
61. भूयाꣳ॑सो यज्ञ्क्र॒तवो॑ यज्ञ्क्र॒तवो॒ भूयाꣳ॑सो॒ भूयाꣳ॑सो यज्ञ्क्र॒तव॒ इतीति॑ यज्ञ्क्र॒तवो॒ भूयाꣳ॑सो॒ भूयाꣳ॑सो यज्ञ्क्र॒तव॒ इति॑ । \newline
62. य॒ज्ञ्॒क्र॒तव॒ इतीति॑ यज्ञ्क्र॒तवो॑ यज्ञ्क्र॒तव॒ इत्या॑हु राहु॒रिति॑ यज्ञ्क्र॒तवो॑ यज्ञ्क्र॒तव॒ इत्या॑हुः । \newline
63. य॒ज्ञ्॒क्र॒तव॒ इति॑ यज्ञ् - क्र॒तवः॑ । \newline
64. इत्या॑हु राहु॒ रितीत्या॑हुः॒ स स आ॑हु॒ रितीत्या॑हुः॒ सः । \newline
65. आ॒हुः॒ स स आ॑हु राहुः॒ स दे॒वता॑ दे॒वताः॒ स आ॑हु राहुः॒ स दे॒वताः᳚ । \newline
66. स दे॒वता॑ दे॒वताः॒ स स दे॒वता॑ वृङ्क्ते वृङ्क्ते दे॒वताः॒ स स दे॒वता॑ वृङ्क्ते । \newline
67. दे॒वता॑ वृङ्क्ते वृङ्क्ते दे॒वता॑ दे॒वता॑ वृङ्क्त॒ इतीति॑ वृङ्क्ते दे॒वता॑ दे॒वता॑ वृङ्क्त॒ इति॑ । \newline
68. वृ॒ङ्क्त॒ इतीति॑ वृङ्क्ते वृङ्क्त॒ इति॒ यदि॒ यदीति॑ वृङ्क्ते वृङ्क्त॒ इति॒ यदि॑ । \newline
69. इति॒ यदि॒ यदीतीति॒ यद्य॑ग्निष्टो॒मो᳚ ऽग्निष्टो॒मो यदीतीति॒ यद्य॑ग्निष्टो॒मः । \newline
70. यद्य॑ग्निष्टो॒मो᳚ ऽग्निष्टो॒मो यदि॒ यद्य॑ग्निष्टो॒मः सोमः॒ सोमो॑ अग्निष्टो॒मो यदि॒ यद्य॑ग्निष्टो॒मः सोमः॑ । \newline
71. अ॒ग्नि॒ष्टो॒मः सोमः॒ सोमो॑ अग्निष्टो॒मो᳚ ऽग्निष्टो॒मः सोमः॑ प॒रस्ता᳚त् प॒रस्ता॒थ् सोमो॑ अग्निष्टो॒मो᳚ ऽग्निष्टो॒मः सोमः॑ प॒रस्ता᳚त् । \newline
72. अ॒ग्नि॒ष्टो॒म इत्य॑ग्नि - स्तो॒मः । \newline
73. सोमः॑ प॒रस्ता᳚त् प॒रस्ता॒थ् सोमः॒ सोमः॑ प॒रस्ता॒थ् स्याथ् स्यात् प॒रस्ता॒थ् सोमः॒ सोमः॑ प॒रस्ता॒थ् स्यात् । \newline
74. प॒रस्ता॒थ् स्याथ् स्यात् प॒रस्ता᳚त् प॒रस्ता॒थ् स्या दु॒क्थ्य॑ मु॒क्थ्यꣳ॑ स्यात् प॒रस्ता᳚त् प॒रस्ता॒थ् स्या दु॒क्थ्य᳚म् । \newline
75. स्यादु॒क्थ्य॑ मु॒क्थ्यꣳ॑ स्याथ् स्या दु॒क्थ्य॑म् कुर्वीत कुर्वी तो॒क्थ्यꣳ॑ स्याथ् स्या दु॒क्थ्य॑म् कुर्वीत । \newline
76. उ॒क्थ्य॑म् कुर्वीत कुर्वीतो॒क्थ्य॑ मु॒क्थ्य॑म् कुर्वीत॒ यदि॒ यदि॑ कुर्वीतो॒क्थ्य॑ मु॒क्थ्य॑म् कुर्वीत॒ यदि॑ । \newline
77. कु॒र्वी॒त॒ यदि॒ यदि॑ कुर्वीत कुर्वीत॒ यद्यु॒क्थ्य॑ उ॒क्थ्यो॑ यदि॑ कुर्वीत कुर्वीत॒ यद्यु॒क्थ्यः॑ । \newline
78. यद्यु॒क्थ्य॑ उ॒क्थ्यो॑ यदि॒ यद्यु॒क्थ्यः॑ स्याथ् स्यादु॒क्थ्यो॑ यदि॒ यद्यु॒क्थ्यः॑ स्यात् । \newline
79. उ॒क्थ्यः॑ स्याथ् स्या दु॒क्थ्य॑ उ॒क्थ्यः॑ स्या द॑तिरा॒त्र म॑तिरा॒त्रꣳ स्या दु॒क्थ्य॑ उ॒क्थ्यः॑ स्या द॑तिरा॒त्रम् । \newline
80. स्या द॑तिरा॒त्र म॑तिरा॒त्रꣳ स्याथ् स्या द॑तिरा॒त्रम् कु॑र्वीत कुर्वीता तिरा॒त्रꣳ स्याथ् स्या द॑तिरा॒त्रम् कु॑र्वीत । \newline
81. अ॒ति॒रा॒त्रम् कु॑र्वीत कुर्वीता तिरा॒त्र म॑तिरा॒त्रम् कु॑र्वीत यज्ञ्क्र॒तुभि॑र् यज्ञ्क्र॒तुभिः॑ कुर्वीता तिरा॒त्र म॑तिरा॒त्रम् कु॑र्वीत यज्ञ्क्र॒तुभिः॑ । \newline
82. अ॒ति॒रा॒त्रमित्य॑ति - रा॒त्रम् । \newline
83. कु॒र्वी॒त॒ य॒ज्ञ्॒क्र॒तुभि॑र् यज्ञ्क्र॒तुभिः॑ कुर्वीत कुर्वीत यज्ञ्क्र॒तुभि॑ रे॒वैव य॑ज्ञ्क्र॒तुभिः॑ कुर्वीत कुर्वीत यज्ञ्क्र॒तुभि॑ रे॒व । \newline
84. य॒ज्ञ्॒क्र॒तुभि॑ रे॒वैव य॑ज्ञ्क्र॒तुभि॑र् यज्ञ्क्र॒तुभि॑ रे॒वास्या᳚स्यै॒व य॑ज्ञ्क्र॒तुभि॑र् यज्ञ्क्र॒तुभि॑ रे॒वास्य॑ । \newline
85. य॒ज्ञ्॒क्र॒तुभि॒रिति॑ यज्ञ्क्र॒तु - भिः॒ । \newline
86. ए॒वास्या᳚ स्यै॒वैवास्य॑ दे॒वता॑ दे॒वता॑ अस्यै॒वैवास्य॑ दे॒वताः᳚ । \newline
87. अ॒स्य॒ दे॒वता॑ दे॒वता॑ अस्यास्य दे॒वता॑ वृङ्क्ते वृङ्क्ते दे॒वता॑ अस्यास्य दे॒वता॑ वृङ्क्ते । \newline
88. दे॒वता॑ वृङ्क्ते वृङ्क्ते दे॒वता॑ दे॒वता॑ वृङ्क्ते॒ वसी॑या॒न्॒. वसी॑यान् वृङ्क्ते दे॒वता॑ दे॒वता॑ वृङ्क्ते॒ वसी॑यान् । \newline
89. वृ॒ङ्क्ते॒ वसी॑या॒न्॒. वसी॑यान् वृङ्क्ते वृङ्क्ते॒ वसी॑यान् भवति भवति॒ वसी॑यान् वृङ्क्ते वृङ्क्ते॒ वसी॑यान् भवति । \newline
90. वसी॑यान् भवति भवति॒ वसी॑या॒न्॒. वसी॑यान् भवति । \newline
91. भ॒व॒तीति॑ भवति । \newline
\pagebreak
\markright{ TS 3.1.8.1  \hfill https://www.vedavms.in \hfill}

\section{ TS 3.1.8.1 }

\textbf{TS 3.1.8.1 } \newline
\textbf{Samhita Paata} \newline

नि॒ग्रा॒भ्याः᳚ स्थ देव॒श्रुत॒ आयु॑र्मे तर्पयत प्रा॒णं मे॑ तर्पयतापा॒नं मे॑ तर्पयत व्या॒नं मे॑ तर्पयत॒ चक्षु॑र्मे तर्पयत॒ श्रोत्रं॑ मे तर्पयत॒ मनो॑मे तर्पयत॒ वाचं॑ मे तर्पयता॒ऽऽ*त्मानं॑ मे तर्पय॒ताङ्गा॑नि मे तर्पयत प्र॒जां मे॑ तर्पयत प॒शून् मे॑ तर्पयत गृ॒हान् मे॑ तर्पयत ग॒णान् मे॑ तर्पयत स॒र्वग॑णं मा तर्पयत त॒र्पय॑त मा - [  ] \newline

\textbf{Pada Paata} \newline

नि॒ग्रा॒भ्या॑ इति॑ नि - ग्रा॒भ्याः᳚ । स्थ॒ । दे॒व॒श्रुत॒ इति॑ देव - श्रुतः॑ । आयुः॑ । मे॒ । त॒र्प॒य॒त॒ । प्रा॒णमिति॑ प्र- अ॒नम् । म॒ । त॒र्प॒य॒त॒ । अ॒पा॒नमित्य॑प - अ॒नम् । मे॒ । त॒र्प॒य॒त॒ । व्या॒नमिति॑ वि - अ॒नम् । मे॒ । त॒र्प॒य॒त॒ । चक्षुः॑ । मे॒ । त॒र्प॒य॒त॒ । श्रोत्र᳚म् । मे॒ । त॒र्प॒य॒त॒ । मनः॑ । मे॒ । त॒र्प॒य॒त॒। वाच᳚म् । मे॒ । त॒र्प॒य॒त॒ । आ॒त्मान᳚म् । मे॒ । त॒र्प॒य॒त॒ । अङ्गा॑नि । म॒ । त॒र्प॒य॒त॒ । प्र॒जामिति॑ प्र - जाम् । मे॒ । त॒र्प॒य॒त॒ । प॒शून् । मे॒ । त॒र्प॒य॒त॒ । गृ॒हान् । मे॒ । त॒र्प॒य॒त॒ । ग॒णान् । मे॒ । त॒र्प॒य॒त॒ । स॒र्वग॑ण॒मिति॑ स॒र्व - ग॒ण॒म् । मा॒ । त॒र्प॒य॒त॒ । त॒र्पय॑त । मा॒ ।  \newline


\textbf{Krama Paata} \newline

नि॒ग्रा॒भ्याः᳚ स्थ । नि॒ग्रा॒भ्या॑ इति॑ नि - ग्रा॒भ्याः᳚ । स्थ॒ दे॒व॒श्रुतः॑ । दे॒व॒श्रुत॒ आयुः॑ । दे॒व॒श्रुत॒ इति॑ देव - श्रुतः॑ । आयु॑र् मे । मे॒ त॒र्प॒य॒त॒ । त॒र्प॒य॒त॒ प्रा॒णम् । प्रा॒णम् मे᳚ । प्रा॒णमिति॑ प्र - अ॒नम् । मे॒ त॒र्प॒य॒त॒ । त॒र्प॒य॒ता॒पा॒नम् । अ॒पा॒नम् मे᳚ । अ॒पा॒नमित्य॑प - अ॒नम् । मे॒ त॒र्प॒य॒त॒ । त॒र्प॒य॒त॒ व्या॒नम् । व्या॒नम् मे᳚ । व्या॒नमिति॑ वि - अ॒नम् । मे॒ त॒र्प॒य॒त॒ । त॒र्प॒य॒त॒ चक्षुः॑ । चक्षु॑र् मे । मे॒ त॒र्प॒य॒त॒ । त॒र्प॒य॒त॒ श्रोत्र᳚म् । श्रोत्र॑म् मे । मे॒ त॒र्प॒य॒त॒ । त॒र्प॒य॒त॒ मनः॑ । मनो॑ मे । मे॒ त॒र्प॒य॒त॒ । त॒र्प॒य॒त॒ वाच᳚म् । वाच॑म् मे । मे॒ त॒र्प॒य॒त॒ । त॒र्प॒य॒ता॒त्मान᳚म् । आ॒त्मान॑म् मे । मे॒ त॒र्प॒य॒त॒ । त॒र्प॒य॒ताङ्गा॑नि । अङ्गा॑नि मे । मे॒ त॒र्प॒य॒त॒ । त॒र्प॒य॒त॒ प्र॒जाम् । प्र॒जाम् मे᳚ । प्र॒जामिति॑ प्र - जाम् । मे॒ त॒र्प॒य॒त॒ । त॒र्प॒य॒त॒ प॒शून् । प॒शून् मे᳚ । मे॒ त॒र्प॒य॒त॒ । त॒र्प॒य॒त॒ गृ॒हान् । गृ॒हान् मे᳚ । मे॒ त॒र्प॒य॒त॒ । त॒र्प॒य॒त॒ ग॒णान् । ग॒णान् मे᳚ । मे॒ त॒र्प॒य॒त॒ । त॒र्प॒य॒त॒ स॒र्वग॑णम् । स॒र्वग॑णम् मा । स॒र्वग॑ण॒मिति॑ स॒र्व - ग॒ण॒म् । मा॒ त॒र्प॒य॒त॒ । त॒र्प॒य॒त॒ त॒र्पय॑त । त॒र्पय॑त मा । मा॒ ग॒णाः \newline

\textbf{Jatai Paata} \newline

1. नि॒ग्रा॒भ्याः᳚ स्थ स्थ निग्रा॒भ्या॑ निग्रा॒भ्याः᳚ स्थ । \newline
2. नि॒ग्रा॒भ्या॑ इति॑ नि - ग्रा॒भ्याः᳚ । \newline
3. स्थ॒ दे॒व॒श्रुतो॑ देव॒श्रुतः॑ स्थ स्थ देव॒श्रुतः॑ । \newline
4. दे॒व॒श्रुत॒ आयु॒ रायु॑र् देव॒श्रुतो॑ देव॒श्रुत॒ आयुः॑ । \newline
5. दे॒व॒श्रुत॒ इति॑ देव - श्रुतः॑ । \newline
6. आयु॑र् मे म॒ आयु॒ रायु॑र् मे । \newline
7. मे॒ त॒र्प॒य॒त॒ त॒र्प॒य॒त॒ मे॒ मे॒ त॒र्प॒य॒त॒ । \newline
8. त॒र्प॒य॒त॒ प्रा॒णम् प्रा॒णम् त॑र्पयत तर्पयत प्रा॒णम् । \newline
9. प्रा॒णम् मे॑ मे प्रा॒णम् प्रा॒णम् मे᳚ । \newline
10. प्रा॒णमिति॑ प्र - अ॒नम् । \newline
11. मे॒ त॒र्प॒य॒त॒ त॒र्प॒य॒त॒ मे॒ मे॒ त॒र्प॒य॒त॒ । \newline
12. त॒र्प॒य॒ता ॒पा॒न म॑पा॒नम् त॑र्पयत तर्पयता पा॒नम् । \newline
13. अ॒पा॒नम् मे॑ मे ऽपा॒न म॑पा॒नम् मे᳚ । \newline
14. अ॒पा॒नमित्य॑प - अ॒नम् । \newline
15. मे॒ त॒र्प॒य॒त॒ त॒र्प॒य॒त॒ मे॒ मे॒ त॒र्प॒य॒त॒ । \newline
16. त॒र्प॒य॒त॒ व्या॒नं ॅव्या॒नम् त॑र्पयत तर्पयत व्या॒नम् । \newline
17. व्या॒नम् मे॑ मे व्या॒नं ॅव्या॒नम् मे᳚ । \newline
18. व्या॒नमिति॑ वि - अ॒नम् । \newline
19. मे॒ त॒र्प॒य॒त॒ त॒र्प॒य॒त॒ मे॒ मे॒ त॒र्प॒य॒त॒ । \newline
20. त॒र्प॒य॒त॒ चक्षु॒ श्चक्षु॑ स्तर्पयत तर्पयत॒ चक्षुः॑ । \newline
21. चक्षु॑र् मे मे॒ चक्षु॒ श्चक्षु॑र् मे । \newline
22. मे॒ त॒र्प॒य॒त॒ त॒र्प॒य॒त॒ मे॒ मे॒ त॒र्प॒य॒त॒ । \newline
23. त॒र्प॒य॒त॒ श्रोत्रꣳ॒॒ श्रोत्र॑म् तर्पयत तर्पयत॒ श्रोत्र᳚म् । \newline
24. श्रोत्र॑म् मे मे॒ श्रोत्रꣳ॒॒ श्रोत्र॑म् मे । \newline
25. मे॒ त॒र्प॒य॒त॒ त॒र्प॒य॒त॒ मे॒ मे॒ त॒र्प॒य॒त॒ । \newline
26. त॒र्प॒य॒त॒ मनो॒ मन॑ स्तर्पयत तर्पयत॒ मनः॑ । \newline
27. मनो॑ मे मे॒ मनो॒ मनो॑ मे । \newline
28. मे॒ त॒र्प॒य॒त॒ त॒र्प॒य॒त॒ मे॒ मे॒ त॒र्प॒य॒त॒ । \newline
29. त॒र्प॒य॒त॒ वाचं॒ ॅवाच॑म् तर्पयत तर्पयत॒ वाच᳚म् । \newline
30. वाच॑म् मे मे॒ वाचं॒ ॅवाच॑म् मे । \newline
31. मे॒ त॒र्प॒य॒त॒ त॒र्प॒य॒त॒ मे॒ मे॒ त॒र्प॒य॒त॒ । \newline
32. त॒र्प॒य॒ता॒त्मान॑ मा॒त्मान॑म् तर्पयत तर्पयता॒त्मान᳚म् । \newline
33. आ॒त्मान॑म् मे म आ॒त्मान॑ मा॒त्मान॑म् मे । \newline
34. मे॒ त॒र्प॒य॒त॒ त॒र्प॒य॒त॒ मे॒ मे॒ त॒र्प॒य॒त॒ । \newline
35. त॒र्प॒य॒ताङ्गा॒ न्यङ्गा॑नि तर्पयत तर्पय॒ताङ्गा॑नि । \newline
36. अङ्गा॑नि मे॒ मे ऽङ्गा॒ न्यङ्गा॑नि मे । \newline
37. मे॒ त॒र्प॒य॒त॒ त॒र्प॒य॒त॒ मे॒ मे॒ त॒र्प॒य॒त॒ । \newline
38. त॒र्प॒य॒त॒ प्र॒जाम् प्र॒जाम् त॑र्पयत तर्पयत प्र॒जाम् । \newline
39. प्र॒जाम् मे॑ मे प्र॒जाम् प्र॒जाम् मे᳚ । \newline
40. प्र॒जामिति॑ प्र - जाम् । \newline
41. मे॒ त॒र्प॒य॒त॒ त॒र्प॒य॒त॒ मे॒ मे॒ त॒र्प॒य॒त॒ । \newline
42. त॒र्प॒य॒त॒ प॒शून् प॒शून् त॑र्पयत तर्पयत प॒शून् । \newline
43. प॒शून् मे॑ मे प॒शून् प॒शून् मे᳚ । \newline
44. मे॒ त॒र्प॒य॒त॒ त॒र्प॒य॒त॒ मे॒ मे॒ त॒र्प॒य॒त॒ । \newline
45. त॒र्प॒य॒त॒ गृ॒हान् गृ॒हान् त॑र्पयत तर्पयत गृ॒हान् । \newline
46. गृ॒हान् मे॑ मे गृ॒हान् गृ॒हान् मे᳚ । \newline
47. मे॒ त॒र्प॒य॒त॒ त॒र्प॒य॒त॒ मे॒ मे॒ त॒र्प॒य॒त॒ । \newline
48. त॒र्प॒य॒त॒ ग॒णान् ग॒णान् त॑र्पयत तर्पयत ग॒णान् । \newline
49. ग॒णान् मे॑ मे ग॒णान् ग॒णान् मे᳚ । \newline
50. मे॒ त॒र्प॒य॒त॒ त॒र्प॒य॒त॒ मे॒ मे॒ त॒र्प॒य॒त॒ । \newline
51. त॒र्प॒य॒त॒ स॒र्वग॑णꣳ स॒र्वग॑णम् तर्पयत तर्पयत स॒र्वग॑णम् । \newline
52. स॒र्वग॑णम् मा मा स॒र्वग॑णꣳ स॒र्वग॑णम् मा । \newline
53. स॒र्वग॑ण॒मिति॑ स॒र्व - ग॒ण॒म् । \newline
54. मा॒ त॒र्प॒य॒त॒ त॒र्प॒य॒त॒ मा॒ मा॒ त॒र्प॒य॒त॒ । \newline
55. त॒र्प॒य॒त॒ त॒र्पय॑त त॒र्पय॑त तर्पयत तर्पयत त॒र्पय॑त । \newline
56. त॒र्पय॑त मा मा त॒र्पय॑त त॒र्पय॑त मा । \newline
57. मा॒ ग॒णा ग॒णा मा॑ मा ग॒णाः । \newline

\textbf{Ghana Paata } \newline

1. नि॒ग्रा॒भ्याः᳚ स्थ स्थ निग्रा॒भ्या॑ निग्रा॒भ्याः᳚ स्थ देव॒श्रुतो॑ देव॒श्रुतः॑ स्थ निग्रा॒भ्या॑ निग्रा॒भ्याः᳚ स्थ देव॒श्रुतः॑ । \newline
2. नि॒ग्रा॒भ्या॑ इति॑ नि - ग्रा॒भ्याः᳚ । \newline
3. स्थ॒ दे॒व॒श्रुतो॑ देव॒श्रुतः॑ स्थ स्थ देव॒श्रुत॒ आयु॒ रायु॑र् देव॒श्रुतः॑ स्थ स्थ देव॒श्रुत॒ आयुः॑ । \newline
4. दे॒व॒श्रुत॒ आयु॒ रायु॑र् देव॒श्रुतो॑ देव॒श्रुत॒ आयु॑र् मे म॒ आयु॑र् देव॒श्रुतो॑ देव॒श्रुत॒ आयु॑र् मे । \newline
5. दे॒व॒श्रुत॒ इति॑ देव - श्रुतः॑ । \newline
6. आयु॑र् मे म॒ आयु॒ रायु॑र् मे तर्पयत तर्पयत म॒ आयु॒ रायु॑र् मे तर्पयत । \newline
7. मे॒ त॒र्प॒य॒त॒ त॒र्प॒य॒त॒ मे॒ मे॒ त॒र्प॒य॒त॒ प्रा॒णम् प्रा॒णम् त॑र्पयत मे मे तर्पयत प्रा॒णम् । \newline
8. त॒र्प॒य॒त॒ प्रा॒णम् प्रा॒णम् त॑र्पयत तर्पयत प्रा॒णम् मे॑ मे प्रा॒णम् त॑र्पयत तर्पयत प्रा॒णम् मे᳚ । \newline
9. प्रा॒णम् मे॑ मे प्रा॒णम् प्रा॒णम् मे॑ तर्पयत तर्पयत मे प्रा॒णम् प्रा॒णम् मे॑ तर्पयत । \newline
10. प्रा॒णमिति॑ प्र - अ॒नम् । \newline
11. मे॒ त॒र्प॒य॒त॒ त॒र्प॒य॒त॒ मे॒ मे॒ त॒र्प॒य॒ता॒ पा॒न म॑पा॒नम् त॑र्पयत मे मे तर्पयता पा॒नम् । \newline
12. त॒र्प॒य॒ता॒ पा॒न म॑पा॒नम् त॑र्पयत तर्पयता पा॒नम् मे॑ मे ऽपा॒नम् त॑र्पयत तर्पयता पा॒नम् मे᳚ । \newline
13. अ॒पा॒नम् मे॑ मे ऽपा॒न म॑पा॒नम् मे॑ तर्पयत तर्पयत मे ऽपा॒न म॑पा॒नम् मे॑ तर्पयत । \newline
14. अ॒पा॒नमित्य॑प - अ॒नम् । \newline
15. मे॒ त॒र्प॒य॒त॒ त॒र्प॒य॒त॒ मे॒ मे॒ त॒र्प॒य॒त॒ व्या॒नं ॅव्या॒नम् त॑र्पयत मे मे तर्पयत व्या॒नम् । \newline
16. त॒र्प॒य॒त॒ व्या॒नं ॅव्या॒नम् त॑र्पयत तर्पयत व्या॒नम् मे॑ मे व्या॒नम् त॑र्पयत तर्पयत व्या॒नम् मे᳚ । \newline
17. व्या॒नम् मे॑ मे व्या॒नं ॅव्या॒नम् मे॑ तर्पयत तर्पयत मे व्या॒नं ॅव्या॒नम् मे॑ तर्पयत । \newline
18. व्या॒नमिति॑ वि - अ॒नम् । \newline
19. मे॒ त॒र्प॒य॒त॒ त॒र्प॒य॒त॒ मे॒ मे॒ त॒र्प॒य॒त॒ चक्षु॒ श्चक्षु॑ स्तर्पयत मे मे तर्पयत॒ चक्षुः॑ । \newline
20. त॒र्प॒य॒त॒ चक्षु॒ श्चक्षु॑ स्तर्पयत तर्पयत॒ चक्षु॑र् मे मे॒ चक्षु॑ स्तर्पयत तर्पयत॒ चक्षु॑र् मे । \newline
21. चक्षु॑र् मे मे॒ चक्षु॒ श्चक्षु॑र् मे तर्पयत तर्पयत मे॒ चक्षु॒ श्चक्षु॑र् मे तर्पयत । \newline
22. मे॒ त॒र्प॒य॒त॒ त॒र्प॒य॒त॒ मे॒ मे॒ त॒र्प॒य॒त॒ श्रोत्रꣳ॒॒ श्रोत्र॑म् तर्पयत मे मे तर्पयत॒ श्रोत्र᳚म् । \newline
23. त॒र्प॒य॒त॒ श्रोत्रꣳ॒॒ श्रोत्र॑म् तर्पयत तर्पयत॒ श्रोत्र॑म् मे मे॒ श्रोत्र॑म् तर्पयत तर्पयत॒ श्रोत्र॑म् मे । \newline
24. श्रोत्र॑म् मे मे॒ श्रोत्रꣳ॒॒ श्रोत्र॑म् मे तर्पयत तर्पयत मे॒ श्रोत्रꣳ॒॒ श्रोत्र॑म् मे तर्पयत । \newline
25. मे॒ त॒र्प॒य॒त॒ त॒र्प॒य॒त॒ मे॒ मे॒ त॒र्प॒य॒त॒ मनो॒ मन॑ स्तर्पयत मे मे तर्पयत॒ मनः॑ । \newline
26. त॒र्प॒य॒त॒ मनो॒ मन॑ स्तर्पयत तर्पयत॒ मनो॑ मे मे॒ मन॑ स्तर्पयत तर्पयत॒ मनो॑ मे । \newline
27. मनो॑ मे मे॒ मनो॒ मनो॑ मे तर्पयत तर्पयत मे॒ मनो॒ मनो॑ मे तर्पयत । \newline
28. मे॒ त॒र्प॒य॒त॒ त॒र्प॒य॒त॒ मे॒ मे॒ त॒र्प॒य॒त॒ वाचं॒ ॅवाच॑म् तर्पयत मे मे तर्पयत॒ वाच᳚म् । \newline
29. त॒र्प॒य॒त॒ वाचं॒ ॅवाच॑म् तर्पयत तर्पयत॒ वाच॑म् मे मे॒ वाच॑म् तर्पयत तर्पयत॒ वाच॑म् मे । \newline
30. वाच॑म् मे मे॒ वाचं॒ ॅवाच॑म् मे तर्पयत तर्पयत मे॒ वाचं॒ ॅवाच॑म् मे तर्पयत । \newline
31. मे॒ त॒र्प॒य॒त॒ त॒र्प॒य॒त॒ मे॒ मे॒ त॒र्प॒य॒ता॒ त्मान॑ मा॒त्मान॑म् तर्पयत मे मे तर्पयता॒ त्मान᳚म् । \newline
32. त॒र्प॒य॒ता॒ त्मान॑ मा॒त्मान॑म् तर्पयत तर्पयता॒ त्मान॑म् मे म आ॒त्मान॑म् तर्पयत तर्पयता॒ त्मान॑म् मे । \newline
33. आ॒त्मान॑म् मे म आ॒त्मान॑ मा॒त्मान॑म् मे तर्पयत तर्पयत म आ॒त्मान॑ मा॒त्मान॑म् मे तर्पयत । \newline
34. मे॒ त॒र्प॒य॒त॒ त॒र्प॒य॒त॒ मे॒ मे॒ त॒र्प॒य॒ता ङ्गा॒ न्यङ्गा॑नि तर्पयत मे मे तर्पय॒ता ङ्गा॑नि । \newline
35. त॒र्प॒य॒ता ङ्गा॒ न्यङ्गा॑नि तर्पयत तर्पय॒ता ङ्गा॑नि मे॒ मे ऽङ्गा॑नि तर्पयत तर्पय॒ता ङ्गा॑नि मे । \newline
36. अङ्गा॑नि मे॒ मे ऽङ्गा॒ न्यङ्गा॑नि मे तर्पयत तर्पयत॒ मे ऽङ्गा॒ न्यङ्गा॑नि मे तर्पयत । \newline
37. मे॒ त॒र्प॒य॒त॒ त॒र्प॒य॒त॒ मे॒ मे॒ त॒र्प॒य॒त॒ प्र॒जाम् प्र॒जाम् त॑र्पयत मे मे तर्पयत प्र॒जाम् । \newline
38. त॒र्प॒य॒त॒ प्र॒जाम् प्र॒जाम् त॑र्पयत तर्पयत प्र॒जाम् मे॑ मे प्र॒जाम् त॑र्पयत तर्पयत प्र॒जाम् मे᳚ । \newline
39. प्र॒जाम् मे॑ मे प्र॒जाम् प्र॒जाम् मे॑ तर्पयत तर्पयत मे प्र॒जाम् प्र॒जाम् मे॑ तर्पयत । \newline
40. प्र॒जामिति॑ प्र - जाम् । \newline
41. मे॒ त॒र्प॒य॒त॒ त॒र्प॒य॒त॒ मे॒ मे॒ त॒र्प॒य॒त॒ प॒शून् प॒शून् त॑र्पयत मे मे तर्पयत प॒शून् । \newline
42. त॒र्प॒य॒त॒ प॒शून् प॒शून् त॑र्पयत तर्पयत प॒शून् मे॑ मे प॒शून् त॑र्पयत तर्पयत प॒शून् मे᳚ । \newline
43. प॒शून् मे॑ मे प॒शून् प॒शून् मे॑ तर्पयत तर्पयत मे प॒शून् प॒शून् मे॑ तर्पयत । \newline
44. मे॒ त॒र्प॒य॒त॒ त॒र्प॒य॒त॒ मे॒ मे॒ त॒र्प॒य॒त॒ गृ॒हान् गृ॒हान् त॑र्पयत मे मे तर्पयत गृ॒हान् । \newline
45. त॒र्प॒य॒त॒ गृ॒हान् गृ॒हान् त॑र्पयत तर्पयत गृ॒हान् मे॑ मे गृ॒हान् त॑र्पयत तर्पयत गृ॒हान् मे᳚ । \newline
46. गृ॒हान् मे॑ मे गृ॒हान् गृ॒हान् मे॑ तर्पयत तर्पयत मे गृ॒हान् गृ॒हान् मे॑ तर्पयत । \newline
47. मे॒ त॒र्प॒य॒त॒ त॒र्प॒य॒त॒ मे॒ मे॒ त॒र्प॒य॒त॒ ग॒णान् ग॒णान् त॑र्पयत मे मे तर्पयत ग॒णान् । \newline
48. त॒र्प॒य॒त॒ ग॒णान् ग॒णान् त॑र्पयत तर्पयत ग॒णान् मे॑ मे ग॒णान् त॑र्पयत तर्पयत ग॒णान् मे᳚ । \newline
49. ग॒णान् मे॑ मे ग॒णान् ग॒णान् मे॑ तर्पयत तर्पयत मे ग॒णान् ग॒णान् मे॑ तर्पयत । \newline
50. मे॒ त॒र्प॒य॒त॒ त॒र्प॒य॒त॒ मे॒ मे॒ त॒र्प॒य॒त॒ स॒र्वग॑णꣳ स॒र्वग॑णम् तर्पयत मे मे तर्पयत स॒र्वग॑णम् । \newline
51. त॒र्प॒य॒त॒ स॒र्वग॑णꣳ स॒र्वग॑णम् तर्पयत तर्पयत स॒र्वग॑णम् मा मा स॒र्वग॑णम् तर्पयत तर्पयत स॒र्वग॑णम् मा । \newline
52. स॒र्वग॑णम् मा मा स॒र्वग॑णꣳ स॒र्वग॑णम् मा तर्पयत तर्पयत मा स॒र्वग॑णꣳ स॒र्वग॑णम् मा तर्पयत । \newline
53. स॒र्वग॑ण॒मिति॑ स॒र्व - ग॒ण॒म् । \newline
54. मा॒ त॒र्प॒य॒त॒ त॒र्प॒य॒त॒ मा॒ मा॒ त॒र्प॒य॒त॒ त॒र्पय॑त त॒र्पय॑त तर्पयत मा मा तर्पयत त॒र्पय॑त । \newline
55. त॒र्प॒य॒त॒ त॒र्पय॑त त॒र्पय॑त तर्पयत तर्पयत त॒र्पय॑त मा मा त॒र्पय॑त तर्पयत तर्पयत त॒र्पय॑त मा । \newline
56. त॒र्पय॑त मा मा त॒र्पय॑त त॒र्पय॑त मा ग॒णा ग॒णा मा॑ त॒र्पय॑त त॒र्पय॑त मा ग॒णाः । \newline
57. मा॒ ग॒णा ग॒णा मा॑ मा ग॒णा मे॑ मे ग॒णा मा॑ मा ग॒णा मे᳚ । \newline
\pagebreak
\markright{ TS 3.1.8.2  \hfill https://www.vedavms.in \hfill}

\section{ TS 3.1.8.2 }

\textbf{TS 3.1.8.2 } \newline
\textbf{Samhita Paata} \newline

ग॒णा मे॒ मा वि तृ॑ष॒न्नोष॑धयो॒ वै सोम॑स्य॒ विशो॒ विशः॒ खलु॒ वै राज्ञ्ः॒ प्रदा॑तोरीश्व॒रा ऐ॒न्द्रः सोमोऽवी॑वृधं ॅवो॒ मन॑सा सुजाता॒ ऋत॑प्रजाता॒ भग॒ इद्वः॑ स्याम । इन्द्रे॑ण दे॒वीर्वी॒रुधः॑ संॅविदा॒ना अनु॑ मन्यन्ताꣳ॒॒ सव॑नाय॒ सोम॒मित्या॒हौष॑धीभ्य ए॒वैनꣳ॒॒ स्वायै॑ वि॒शः स्वायै॑ दे॒वता॑यै नि॒र्याच्या॒भि षु॑णोति॒ यो वै सोम॑स्याभिषू॒यमा॑णस्य - [  ] \newline

\textbf{Pada Paata} \newline

ग॒णाः । मे॒ । मा । वीति॑ । तृ॒ष॒न्न् । ओष॑धयः । वै । सोम॑स्य । विशः॑ । विशः॑ । खलु॑ । वै । राज्ञ्ः॑ । प्रदा॑तो॒रिति॒ प्र - दा॒तोः॒ । ई॒श्व॒राः । ऐ॒न्द्रः । सोमः॑ । अवी॑वृधम् । वः॒ । मन॑सा । सु॒जा॒ता॒ इति॑ सु - जा॒ताः॒ । ऋत॑प्रजाता॒ इत्यृत॑ - प्र॒जा॒ताः॒ । भगे᳚ । इत् । वः॒ । स्या॒म॒ ॥ इन्द्रे॑ण । दे॒वीः । वी॒रुधः॑ । सं॒ॅवि॒दा॒ना इति॑ सं - वि॒दा॒नाः । अन्विति॑ । म॒न्य॒न्ता॒म् । सव॑नाय । सोम᳚म् । इति॑ । आ॒ह॒ । ओष॑धीभ्य॒ इत्योष॑धि - भ्यः॒ । ए॒व । ए॒न॒म् । स्वायै᳚ । वि॒शः । स्वायै᳚ । दे॒वता॑यै । नि॒र्याच्येति॑ निः - याच्य॑ । अ॒भीति॑ । सु॒नो॒ति॒ । यः । वै । सोम॑स्य । अ॒भि॒षू॒यमा॑ण॒स्येत्य॑भि - सू॒यमा॑णस्य ।  \newline


\textbf{Krama Paata} \newline

ग॒णा मे᳚ । मे॒ मा । मा वि । वि तृ॑षन्न् । तृ॒ष॒न्नोष॑धयः । ओष॑धयो॒ वै । वै सोम॑स्य । सोम॑स्य॒ विशः॑ । विशो॒ विशः॑ । विशः॒ खलु॑ । खलु॒ वै । वै राज्ञ्ः॑ । राज्ञ्ः॒ प्रदा॑तोः । प्रदा॑तोरीश्व॒राः । प्रदा॑तो॒रिति॒ प्र - दा॒तोः॒ । ई॒श्व॒रा ऐ॒न्द्रः । ऐ॒न्द्रः सोमः॑ । सोमोऽवी॑वृधम् । अवी॑वृधं ॅवः । वो॒ मन॑सा । मन॑सा सुजाताः । सु॒जा॒ता॒ ऋत॑प्रजाताः । सु॒जा॒ता॒ इति॑ सु - जा॒ताः॒ । ऋत॑प्रजाता॒ भगे᳚ । ऋत॑प्रजाता॒ इत्यृत॑ - प्र॒जा॒ताः॒ । भग॒ इत् । इद् वः॑ । वः॒ स्या॒म॒ । स्या॒मेति॑ स्याम ॥ इन्द्रे॑ण दे॒वीः । दे॒वीर् वी॒रुधः॑ । वी॒रुधः॑ सम्ॅविदा॒नाः । स॒म्ॅवि॒दा॒ना अनु॑ । स॒म्ॅवि॒दा॒ना इति॑ सम् - वि॒दा॒नाः । अनु॑ मन्यन्ताम् । म॒न्य॒न्ताꣳ॒॒ सव॑नाय । सव॑नाय॒ सोम᳚म् । सोम॒मिति॑ । इत्या॑ह । आ॒हौष॑धीभ्यः । ओष॑धीभ्य ए॒व । ओष॑धीभ्य॒ इत्योष॑धि - भ्यः॒ । ए॒वैन᳚म् । ए॒नꣳ॒॒ स्वायै᳚ । स्वायै॑ वि॒शः । वि॒शः स्वायै᳚ । स्वायै॑ दे॒वता॑यै । दे॒वता॑यै नि॒र्याच्य॑ । नि॒र्याच्या॒भि । नि॒र्याच्येति॑ निः - याच्य॑ । अ॒भि षु॑णोति । सु॒नो॒ति॒ यः । यो वै । वै सोम॑स्य । सोम॑स्या,भिषू॒यमा॑णस्य । अ॒भि॒षू॒यमा॑णस्य प्रथ॒मः । अ॒भि॒षू॒यमा॑ण॒स्येत्य॑भि - सू॒यमा॑नस्य \newline

\textbf{Jatai Paata} \newline

1. ग॒णा मे॑ मे ग॒णा ग॒णा मे᳚ । \newline
2. मे॒ मा मा मे॑ मे॒ मा । \newline
3. मा वि वि मा मा वि । \newline
4. वि तृ॑षन् तृष॒न्॒. वि वि तृ॑षन्न् । \newline
5. तृ॒ष॒न् नोष॑धय॒ ओष॑धय स्तृषन् तृष॒न् नोष॑धयः । \newline
6. ओष॑धयो॒ वै वा ओष॑धय॒ ओष॑धयो॒ वै । \newline
7. वै सोम॑स्य॒ सोम॑स्य॒ वै वै सोम॑स्य । \newline
8. सोम॑स्य॒ विशो॒ विशः॒ सोम॑स्य॒ सोम॑स्य॒ विशः॑ । \newline
9. विशो॒ विशः॑ । \newline
10. विशः॒ खलु॒ खलु॒ विशो॒ विशः॒ खलु॑ । \newline
11. खलु॒ वै वै खलु॒ खलु॒ वै । \newline
12. वै राज्ञो॒ राज्ञो॒ वै वै राज्ञ्ः॑ । \newline
13. राज्ञ्ः॒ प्रदा॑तोः॒ प्रदा॑तो॒ राज्ञो॒ राज्ञ्ः॒ प्रदा॑तोः । \newline
14. प्रदा॑तो रीश्व॒रा ई᳚श्व॒राः प्रदा॑तोः॒ प्रदा॑तो रीश्व॒राः । \newline
15. प्रदा॑तो॒रिति॒ प्र - दा॒तोः॒ । \newline
16. ई॒श्व॒रा ऐ॒न्द्र ऐ॒न्द्र ई᳚श्व॒रा ई᳚श्व॒रा ऐ॒न्द्रः । \newline
17. ऐ॒न्द्रः सोमः॒ सोम॑ ऐ॒न्द्र ऐ॒न्द्रः सोमः॑ । \newline
18. सोमो ऽवी॑वृध॒ मवी॑वृधꣳ॒॒ सोमः॒ सोमो ऽवी॑वृधम् । \newline
19. अवी॑वृधं ॅवो॒ वो ऽवी॑वृध॒ मवी॑वृधं ॅवः । \newline
20. वो॒ मन॑सा॒ मन॑सा वो वो॒ मन॑सा । \newline
21. मन॑सा सुजाताः सुजाता॒ मन॑सा॒ मन॑सा सुजाताः । \newline
22. सु॒जा॒ता॒ ऋत॑प्रजाता॒ ऋत॑प्रजाताः सुजाताः सुजाता॒ ऋत॑प्रजाताः । \newline
23. सु॒जा॒ता॒ इति॑ सु - जा॒ताः॒ । \newline
24. ऋत॑प्रजाता॒ भगे॒ भग॒ ऋत॑प्रजाता॒ ऋत॑प्रजाता॒ भगे᳚ । \newline
25. ऋत॑प्रजाता॒ इत्यृत॑ - प्र॒जा॒ताः॒ । \newline
26. भग॒ इदिद् भगे॒ भग॒ इत् । \newline
27. इद् वो॑ व॒ इदिद् वः॑ । \newline
28. वः॒ स्या॒म॒ स्या॒म॒ वो॒ वः॒ स्या॒म॒ । \newline
29. स्या॒मेति॑ स्याम । \newline
30. इन्द्रे॑ण दे॒वीर् दे॒वी रिन्द्रे॒णे न्द्रे॑ण दे॒वीः । \newline
31. दे॒वीर् वी॒रुधो॑ वी॒रुधो॑ दे॒वीर् दे॒वीर् वी॒रुधः॑ । \newline
32. वी॒रुधः॑ संॅविदा॒नाः सं॑ॅविदा॒ना वी॒रुधो॑ वी॒रुधः॑ संॅविदा॒नाः । \newline
33. सं॒ॅवि॒दा॒ना अन्वनु॑ संॅविदा॒नाः सं॑ॅविदा॒ना अनु॑ । \newline
34. सं॒ॅवि॒दा॒ना इति॑ सं - वि॒दा॒नाः । \newline
35. अनु॑ मन्यन्ताम् मन्यन्ता॒ मन्वनु॑ मन्यन्ताम् । \newline
36. म॒न्य॒न्ताꣳ॒॒ सव॑नाय॒ सव॑नाय मन्यन्ताम् मन्यन्ताꣳ॒॒ सव॑नाय । \newline
37. सव॑नाय॒ सोमꣳ॒॒ सोमꣳ॒॒ सव॑नाय॒ सव॑नाय॒ सोम᳚म् । \newline
38. सोम॒ मितीति॒ सोमꣳ॒॒ सोम॒ मिति॑ । \newline
39. इत्या॑हा॒हे तीत्या॑ह । \newline
40. आ॒हौष॑धीभ्य॒ ओष॑धीभ्य आहा॒ हौष॑धीभ्यः । \newline
41. ओष॑धीभ्य ए॒वै वौष॑धीभ्य॒ ओष॑धीभ्य ए॒व । \newline
42. ओष॑धीभ्य॒ इत्योष॑धि - भ्यः॒ । \newline
43. ए॒वैन॑ मेन मे॒वैवैन᳚म् । \newline
44. ए॒नꣳ॒॒ स्वायै॒ स्वाया॑ एन मेनꣳ॒॒ स्वायै᳚ । \newline
45. स्वायै॑ वि॒शो वि॒शः स्वायै॒ स्वायै॑ वि॒शः । \newline
46. वि॒शः स्वायै॒ स्वायै॑ वि॒शो वि॒शः स्वायै᳚ । \newline
47. स्वायै॑ दे॒वता॑यै दे॒वता॑यै॒ स्वायै॒ स्वायै॑ दे॒वता॑यै । \newline
48. दे॒वता॑यै नि॒र्याच्य॑ नि॒र्याच्य॑ दे॒वता॑यै दे॒वता॑यै नि॒र्याच्य॑ । \newline
49. नि॒र्याच्या॒ भ्य॑भि नि॒र्याच्य॑ नि॒र्या च्या॒भि । \newline
50. नि॒र्याच्येति॑ निः - याच्य॑ । \newline
51. अ॒भि षु॑णोति सुनो त्य॒भ्य॑भि षु॑णोति । \newline
52. सु॒नो॒ति॒ यो यः सु॑नोति सुनोति॒ यः । \newline
53. यो वै वै यो यो वै । \newline
54. वै सोम॑स्य॒ सोम॑स्य॒ वै वै सोम॑स्य । \newline
55. सोम॑स्या भिषू॒यमा॑णस्या भिषू॒यमा॑णस्य॒ सोम॑स्य॒ सोम॑स्या भिषू॒यमा॑णस्य । \newline
56. अ॒भि॒षू॒यमा॑णस्य प्रथ॒मः प्र॑थ॒मो॑ ऽभिषू॒यमा॑णस्या भिषू॒यमा॑णस्य प्रथ॒मः । \newline
57. अ॒भि॒षू॒यमा॑ण॒स्येत्य॑भि - सू॒यमा॑नस्य । \newline

\textbf{Ghana Paata } \newline

1. ग॒णा मे॑ मे ग॒णा ग॒णा मे॒ मा मा मे॑ ग॒णा ग॒णा मे॒ मा । \newline
2. मे॒ मा मा मे॑ मे॒ मा वि वि मा मे॑ मे॒ मा वि । \newline
3. मा वि वि मा मा वि तृ॑षन् तृष॒न्॒. वि मा मा वि तृ॑षन्न् । \newline
4. वि तृ॑षन् तृष॒न्॒. वि वि तृ॑ष॒न् नोष॑धय॒ ओष॑धय स्तृष॒न्॒. वि वि तृ॑ष॒न् नोष॑धयः । \newline
5. तृ॒ष॒न् नोष॑धय॒ ओष॑धय स्तृषन् तृष॒न् नोष॑धयो॒ वै वा ओष॑धय स्तृषन् तृष॒न् नोष॑धयो॒ वै । \newline
6. ओष॑धयो॒ वै वा ओष॑धय॒ ओष॑धयो॒ वै सोम॑स्य॒ सोम॑स्य॒ वा ओष॑धय॒ ओष॑धयो॒ वै सोम॑स्य । \newline
7. वै सोम॑स्य॒ सोम॑स्य॒ वै वै सोम॑स्य॒ विशो॒ विशः॒ सोम॑स्य॒ वै वै सोम॑स्य॒ विशः॑ । \newline
8. सोम॑स्य॒ विशो॒ विशः॒ सोम॑स्य॒ सोम॑स्य॒ विशः॑ । \newline
9. विशो॒ विशः॑ । \newline
10. विशः॒ खलु॒ खलु॒ विशो॒ विशः॒ खलु॒ वै वै खलु॒ विशो॒ विशः॒ खलु॒ वै । \newline
11. खलु॒ वै वै खलु॒ खलु॒ वै राज्ञो॒ राज्ञो॒ वै खलु॒ खलु॒ वै राज्ञ्ः॑ । \newline
12. वै राज्ञो॒ राज्ञो॒ वै वै राज्ञ्ः॒ प्रदा॑तोः॒ प्रदा॑तो॒ राज्ञो॒ वै वै राज्ञ्ः॒ प्रदा॑तोः । \newline
13. राज्ञ्ः॒ प्रदा॑तोः॒ प्रदा॑तो॒ राज्ञो॒ राज्ञ्ः॒ प्रदा॑तो रीश्व॒रा ई᳚श्व॒राः प्रदा॑तो॒ राज्ञो॒ राज्ञ्ः॒ प्रदा॑तो रीश्व॒राः । \newline
14. प्रदा॑तो रीश्व॒रा ई᳚श्व॒राः प्रदा॑तोः॒ प्रदा॑तो रीश्व॒रा ऐ॒न्द्र ऐ॒न्द्र ई᳚श्व॒राः प्रदा॑तोः॒ प्रदा॑तो रीश्व॒रा ऐ॒न्द्रः । \newline
15. प्रदा॑तो॒रिति॒ प्र - दा॒तोः॒ । \newline
16. ई॒श्व॒रा ऐ॒न्द्र ऐ॒न्द्र ई᳚श्व॒रा ई᳚श्व॒रा ऐ॒न्द्रः सोमः॒ सोम॑ ऐ॒न्द्र ई᳚श्व॒रा ई᳚श्व॒रा ऐ॒न्द्रः सोमः॑ । \newline
17. ऐ॒न्द्रः सोमः॒ सोम॑ ऐ॒न्द्र ऐ॒न्द्रः सोमो ऽवी॑वृध॒ मवी॑वृधꣳ॒॒ सोम॑ ऐ॒न्द्र ऐ॒न्द्रः सोमो ऽवी॑वृधम् । \newline
18. सोमो ऽवी॑वृध॒ मवी॑वृधꣳ॒॒ सोमः॒ सोमो ऽवी॑वृधं ॅवो॒ वो ऽवी॑वृधꣳ॒॒ सोमः॒ सोमो ऽवी॑वृधं ॅवः । \newline
19. अवी॑वृधं ॅवो॒ वो ऽवी॑वृध॒ मवी॑वृधं ॅवो॒ मन॑सा॒ मन॑सा॒ वो ऽवी॑वृध॒ मवी॑वृधं ॅवो॒ मन॑सा । \newline
20. वो॒ मन॑सा॒ मन॑सा वो वो॒ मन॑सा सुजाताः सुजाता॒ मन॑सा वो वो॒ मन॑सा सुजाताः । \newline
21. मन॑सा सुजाताः सुजाता॒ मन॑सा॒ मन॑सा सुजाता॒ ऋत॑प्रजाता॒ ऋत॑प्रजाताः सुजाता॒ मन॑सा॒ मन॑सा सुजाता॒ ऋत॑प्रजाताः । \newline
22. सु॒जा॒ता॒ ऋत॑प्रजाता॒ ऋत॑प्रजाताः सुजाताः सुजाता॒ ऋत॑प्रजाता॒ भगे॒ भग॒ ऋत॑प्रजाताः सुजाताः सुजाता॒ ऋत॑प्रजाता॒ भगे᳚ । \newline
23. सु॒जा॒ता॒ इति॑ सु - जा॒ताः॒ । \newline
24. ऋत॑प्रजाता॒ भगे॒ भग॒ ऋत॑प्रजाता॒ ऋत॑प्रजाता॒ भग॒ इदिद् भग॒ ऋत॑प्रजाता॒ ऋत॑प्रजाता॒ भग॒ इत् । \newline
25. ऋत॑प्रजाता॒ इत्यृत॑ - प्र॒जा॒ताः॒ । \newline
26. भग॒ इदिद् भगे॒ भग॒ इद् वो॑ व॒ इद् भगे॒ भग॒ इद् वः॑ । \newline
27. इद् वो॑ व॒ इदिद् वः॑ स्याम स्याम व॒ इदिद् वः॑ स्याम । \newline
28. वः॒ स्या॒म॒ स्या॒म॒ वो॒ वः॒ स्या॒म॒ । \newline
29. स्या॒मेति॑ स्याम । \newline
30. इन्द्रे॑ण दे॒वीर् दे॒वी रिन्द्रे॒णे न्द्रे॑ण दे॒वीर् वी॒रुधो॑ वी॒रुधो॑ दे॒वी रिन्द्रे॒णे न्द्रे॑ण दे॒वीर् वी॒रुधः॑ । \newline
31. दे॒वीर् वी॒रुधो॑ वी॒रुधो॑ दे॒वीर् दे॒वीर् वी॒रुधः॑ संॅविदा॒नाः सं॑ॅविदा॒ना वी॒रुधो॑ दे॒वीर् दे॒वीर् वी॒रुधः॑ संॅविदा॒नाः । \newline
32. वी॒रुधः॑ संॅविदा॒नाः सं॑ॅविदा॒ना वी॒रुधो॑ वी॒रुधः॑ संॅविदा॒ना अन्वनु॑ संॅविदा॒ना वी॒रुधो॑ वी॒रुधः॑ संॅविदा॒ना अनु॑ । \newline
33. सं॒ॅवि॒दा॒ना अन्वनु॑ संॅविदा॒नाः सं॑ॅविदा॒ना अनु॑ मन्यन्ताम् मन्यन्ता॒ मनु॑ संॅविदा॒नाः सं॑ॅविदा॒ना अनु॑ मन्यन्ताम् । \newline
34. सं॒ॅवि॒दा॒ना इति॑ सं - वि॒दा॒नाः । \newline
35. अनु॑ मन्यन्ताम् मन्यन्ता॒ मन्वनु॑ मन्यन्ताꣳ॒॒ सव॑नाय॒ सव॑नाय मन्यन्ता॒ मन्वनु॑ मन्यन्ताꣳ॒॒ सव॑नाय । \newline
36. म॒न्य॒न्ताꣳ॒॒ सव॑नाय॒ सव॑नाय मन्यन्ताम् मन्यन्ताꣳ॒॒ सव॑नाय॒ सोमꣳ॒॒ सोमꣳ॒॒ सव॑नाय मन्यन्ताम् मन्यन्ताꣳ॒॒ सव॑नाय॒ सोम᳚म् । \newline
37. सव॑नाय॒ सोमꣳ॒॒ सोमꣳ॒॒ सव॑नाय॒ सव॑नाय॒ सोम॒ मितीति॒ सोमꣳ॒॒ सव॑नाय॒ सव॑नाय॒ सोम॒ मिति॑ । \newline
38. सोम॒ मितीति॒ सोमꣳ॒॒ सोम॒ मित्या॑हा॒हे ति॒ सोमꣳ॒॒ सोम॒ मित्या॑ह । \newline
39. इत्या॑हा॒हे तीत्या॒ हौष॑धीभ्य॒ ओष॑धीभ्य आ॒हे तीत्या॒ हौष॑धीभ्यः । \newline
40. आ॒हौष॑धीभ्य॒ ओष॑धीभ्य आहा॒ हौष॑धीभ्य ए॒वैवौष॑धीभ्य आहा॒ हौष॑धीभ्य ए॒व । \newline
41. ओष॑धीभ्य ए॒वैवौष॑धीभ्य॒ ओष॑धीभ्य ए॒वैन॑ मेन मे॒वौष॑धीभ्य॒ ओष॑धीभ्य ए॒वैन᳚म् । \newline
42. ओष॑धीभ्य॒ इत्योष॑धि - भ्यः॒ । \newline
43. ए॒वैन॑ मेन मे॒वैवैनꣳ॒॒ स्वायै॒ स्वाया॑ एन मे॒वैवैनꣳ॒॒ स्वायै᳚ । \newline
44. ए॒नꣳ॒॒ स्वायै॒ स्वाया॑ एन मेनꣳ॒॒ स्वायै॑ वि॒शो वि॒शः स्वाया॑ एन मेनꣳ॒॒ स्वायै॑ वि॒शः । \newline
45. स्वायै॑ वि॒शो वि॒शः स्वायै॒ स्वायै॑ वि॒शः स्वायै॒ स्वायै॑ वि॒शः स्वायै॒ स्वायै॑ वि॒शः स्वायै᳚ । \newline
46. वि॒शः स्वायै॒ स्वायै॑ वि॒शो वि॒शः स्वायै॑ दे॒वता॑यै दे॒वता॑यै॒ स्वायै॑ वि॒शो वि॒शः स्वायै॑ दे॒वता॑यै । \newline
47. स्वायै॑ दे॒वता॑यै दे॒वता॑यै॒ स्वायै॒ स्वायै॑ दे॒वता॑यै नि॒र्याच्य॑ नि॒र्याच्य॑ दे॒वता॑यै॒ स्वायै॒ स्वायै॑ दे॒वता॑यै नि॒र्याच्य॑ । \newline
48. दे॒वता॑यै नि॒र्याच्य॑ नि॒र्याच्य॑ दे॒वता॑यै दे॒वता॑यै नि॒र्याच्या॒भ्य॑भि नि॒र्याच्य॑ दे॒वता॑यै दे॒वता॑यै नि॒र्याच्या॒भि । \newline
49. नि॒र्याच्या॒भ्य॑भि नि॒र्याच्य॑ नि॒र्याच्या॒भि षु॑णोति सुनोत्य॒भि नि॒र्याच्य॑ नि॒र्याच्या॒भि षु॑णोति । \newline
50. नि॒र्याच्येति॑ निः - याच्य॑ । \newline
51. अ॒भि षु॑णोति सुनो त्य॒भ्य॑भि षु॑णोति॒ यो यः सु॑नो त्य॒भ्य॑भि षु॑णोति॒ यः । \newline
52. सु॒नो॒ति॒ यो यः सु॑नोति सुनोति॒ यो वै वै यः सु॑नोति सुनोति॒ यो वै । \newline
53. यो वै वै यो यो वै सोम॑स्य॒ सोम॑स्य॒ वै यो यो वै सोम॑स्य । \newline
54. वै सोम॑स्य॒ सोम॑स्य॒ वै वै सोम॑स्या भिषू॒यमा॑णस्या भिषू॒यमा॑णस्य॒ सोम॑स्य॒ वै वै सोम॑स्या भिषू॒यमा॑णस्य । \newline
55. सोम॑स्या भिषू॒यमा॑णस्या भिषू॒यमा॑णस्य॒ सोम॑स्य॒ सोम॑ स्याभिषू॒यमा॑णस्य प्रथ॒मः 
प्र॑थ॒मो॑ ऽभिषू॒यमा॑णस्य॒ सोम॑स्य॒ सोम॑ स्याभिषू॒यमा॑णस्य प्रथ॒मः । \newline
56. अ॒भि॒षू॒यमा॑णस्य प्रथ॒मः प्र॑थ॒मो॑ ऽभिषू॒यमा॑णस्या भिषू॒यमा॑णस्य प्रथ॒मो ऽꣳ॑शु रꣳ॒॒शुः प्र॑थ॒मो॑ ऽभिषू॒यमा॑णस्या भिषू॒यमा॑णस्य प्रथ॒मो ऽꣳ॑शुः । \newline
57. अ॒भि॒षू॒यमा॑ण॒स्येत्य॑भि - सू॒यमा॑नस्य । \newline
\pagebreak
\markright{ TS 3.1.8.3  \hfill https://www.vedavms.in \hfill}

\section{ TS 3.1.8.3 }

\textbf{TS 3.1.8.3 } \newline
\textbf{Samhita Paata} \newline

प्रथ॒मोऽꣳ॑शुः स्कन्द॑ति॒ स ई᳚श्व॒र इ॑न्द्रि॒यं ॅवी॒र्यं॑ प्र॒जां प॒शून्. यज॑मानस्य॒ निर्.ह॑न्तो॒स्तम॒भि म॑न्त्रये॒ताऽऽ मा᳚ऽस्कान्थ्स॒ह प्र॒जया॑ स॒ह रा॒यस्पोषे॑णेन्द्रि॒यं मे॑ वी॒र्यं॑ मा निव॑र्द्धी॒रित्या॒शिष॑मे॒वैतामा शा᳚स्त इन्द्रि॒यस्य॑ वी॒य॑र्.स्य प्र॒जायै॑ पशू॒नामनि॑र्घाताय द्र॒फ्सश्च॑स्कन्द पृथि॒वीमनु॒ द्यामि॒मञ्च॒ योनि॒मनु॒ यश्च॒ ( ) पूर्वः॑ । तृ॒तीयं॒ ॅयोनि॒मनु॑ स॒ञ्चर॑न्तं द्र॒फ्सं जु॑हो॒म्यनु॑ स॒प्त होत्राः᳚ ॥ \newline

\textbf{Pada Paata} \newline

प्र॒थ॒मः । अꣳ॒॒शुः । स्कन्द॑ति । सः । ई॒श्व॒रः । इ॒न्द्रि॒यम् । वी॒र्य᳚म् । प्र॒जामिति॑ प्र - जाम् । प॒शून् । यज॑मानस्य । निर्.ह॑न्तो॒रिति॒ निः - ह॒न्तोः॒ । तम् । अ॒भीति॑ । म॒न्त्र॒ये॒त॒ । एति॑ । मा॒ । अ॒स्का॒न् । स॒ह । प्र॒जयेति॑ प्र - जया᳚ । स॒ह । रा॒यः । पोषे॑ण । इ॒न्द्रि॒यम् । मे॒ । वी॒र्य᳚म् । मा । निरिति॑ । व॒धीः॒ । इति॑ । आ॒शिष॒मित्या᳚-शिष᳚म् । ए॒व । ए॒ताम् । एति॑ । शा॒स्ते॒ । इ॒न्द्रि॒यस्य॑ । वी॒र्य॑स्य । प्र॒जाया॒ इति॑ प्र - जायै᳚ । प॒शू॒नाम् । अनि॑र्घाता॒येत्यनिः॑ - घा॒ता॒य॒ । द्र॒फ्सः । च॒स्क॒न्द॒ । पृ॒थि॒वीम् । अन्विति॑ । द्याम् । इ॒मम् । च॒ । योनि᳚म् । अन्विति॑ । यः । च॒ ( ) । पूर्वः॑ । तृ॒तीय᳚म् । योनि᳚म् । अन्विति॑ । स॒ञ्चर॑न्त॒मिति॑ सं - चर॑न्तम् । द्र॒फ्सम् । जु॒हो॒मि॒ । अन्विति॑ । स॒प्त । होत्राः᳚ ॥  \newline


\textbf{Krama Paata} \newline

प्र॒थ॒मोऽꣳ॑शुः । अꣳ॒॒शुः स्कन्द॑ति । स्कन्द॑ति॒ सः । स ई᳚श्व॒रः । ई॒श्व॒र इ॑न्द्रि॒यम् । इ॒न्द्रि॒यं ॅवी॒र्य᳚म् । वी॒र्य॑म् प्र॒जाम् । प्र॒जाम् प॒शून् । प्र॒जामिति॑ प्र - जाम् । प॒शून्. यज॑मानस्य । यज॑मानस्य॒ निर्.ह॑न्तोः । निर्.ह॑न्तो॒स्तम् । निर्.ह॑न्तो॒रिति॒ निः - ह॒न्तोः॒ । तम॒भि । अ॒भि म॑न्त्रयेत । म॒न्त्र॒ये॒ता । आ मा᳚ । मा॒ऽस्का॒न्॒ । अ॒स्का॒न्थ् स॒ह । स॒ह प्र॒जया᳚ । प्र॒जया॑ स॒ह । प्र॒जयेति॑ प्र - जया᳚ । स॒ह रा॒यः । रा॒यस्पोषे॑ण । पोषे॑णेन्द्रि॒यम् । इ॒न्द्रि॒यम् मे᳚ । मे॒ वी॒र्य᳚म् । वी॒र्य॑म् मा । मा निः । निर् व॑धीः । व॒धी॒रिति॑ । इत्या॒शिष᳚म् । आ॒शिष॑मे॒व । आ॒शिष॒मित्या᳚ - शिष᳚म् । ए॒वैताम् । ए॒तामा । आ शा᳚स्ते । शा॒स्त॒ इ॒न्द्रि॒यस्य॑ । इ॒न्द्रि॒यस्य॑ वी॒र्य॑स्य । वी॒र्य॑स्य प्र॒जायै᳚ । प्र॒जायै॑ पशू॒नाम् । प्र॒जाया॒ इति॑ प्र - जायै᳚ । प॒शू॒नामनि॑र्घाताय । अनि॑र्घाताय द्र॒फ्सः । अनि॑र्घाता॒येत्यनिः॑ - घा॒ता॒य॒ । द्र॒फ्सश्च॑स्कन्द । च॒स्क॒न्द॒ पृ॒थि॒वीम् । पृ॒थि॒वीमनु॑ । अनु॒ द्याम् । द्यामि॒मम् । इ॒मम् च॑ । च॒ योनि᳚म् । योनि॒मनु॑ । अनु॒ यः । यश्च॑ ( ) । च॒ पूर्वः॑ । पूर्व॒ इति॒ पूर्वः॑ ॥ तृ॒तीयं॒ ॅयोनि᳚म् । योनि॒ मनु॑ । अनु॑ स॒ञ्चर॑न्तम् । स॒ञ्चर॑न्तम् द्र॒फ्सम् । स॒ञ्चर॑न्त॒मिति॑ सम् - चर॑न्तम् । द्र॒फ्सम् जु॑होमि । जु॒हो॒म्यनु॑ । अनु॑ स॒प्त । स॒प्त होत्राः᳚ । होत्रा॒ इति॒ होत्राः᳚ । \newline

\textbf{Jatai Paata} \newline

1. प्र॒थ॒मो ऽꣳ॑शु रꣳ॒॒शुः प्र॑थ॒मः प्र॑थ॒मो ऽꣳ॑शुः । \newline
2. अꣳ॒॒शुः स्कन्द॑ति॒ स्कन्द॑ त्यꣳ॒॒शु रꣳ॒॒शुः स्कन्द॑ति । \newline
3. स्कन्द॑ति॒ स स स्कन्द॑ति॒ स्कन्द॑ति॒ सः । \newline
4. स ई᳚श्व॒र ई᳚श्व॒रः स स ई᳚श्व॒रः । \newline
5. ई॒श्व॒र इ॑न्द्रि॒य मि॑न्द्रि॒य मी᳚श्व॒र ई᳚श्व॒र इ॑न्द्रि॒यम् । \newline
6. इ॒न्द्रि॒यं ॅवी॒र्यं॑ ॅवी॒र्य॑ मिन्द्रि॒य मि॑न्द्रि॒यं ॅवी॒र्य᳚म् । \newline
7. वी॒र्य॑म् प्र॒जाम् प्र॒जां ॅवी॒र्यं॑ ॅवी॒र्य॑म् प्र॒जाम् । \newline
8. प्र॒जाम् प॒शून् प॒शून् प्र॒जाम् प्र॒जाम् प॒शून् । \newline
9. प्र॒जामिति॑ प्र - जाम् । \newline
10. प॒शून्. यज॑मानस्य॒ यज॑मानस्य प॒शून् प॒शून्. यज॑मानस्य । \newline
11. यज॑मानस्य॒ निर्.ह॑न्तो॒र् निर्.ह॑न्तो॒र् यज॑मानस्य॒ यज॑मानस्य॒ निर्.ह॑न्तोः । \newline
12. निर्.ह॑न्तो॒ स्तम् तन्निर्.ह॑न्तो॒र् निर्.ह॑न्तो॒ स्तम् । \newline
13. निर्.ह॑न्तो॒रिति॒ निः - ह॒न्तोः॒ । \newline
14. त म॒भ्य॑भि तम् त म॒भि । \newline
15. अ॒भि म॑न्त्रयेत मन्त्रयेता॒ भ्य॑भि म॑न्त्रयेत । \newline
16. म॒न्त्र॒ये॒ता म॑न्त्रयेत मन्त्रये॒ता । \newline
17. आ मा॒ मा ऽऽमा᳚ । \newline
18. मा॒ ऽस्का॒ न॒स्का॒न् मा॒ मा॒ ऽस्का॒न् । \newline
19. अ॒स्का॒न् थ्स॒ह स॒हास्का॑ नस्कान् थ्स॒ह । \newline
20. स॒ह प्र॒जया᳚ प्र॒जया॑ स॒ह स॒ह प्र॒जया᳚ । \newline
21. प्र॒जया॑ स॒ह स॒ह प्र॒जया᳚ प्र॒जया॑ स॒ह । \newline
22. प्र॒जयेति॑ प्र - जया᳚ । \newline
23. स॒ह रा॒यो रा॒यः स॒ह स॒ह रा॒यः । \newline
24. रा॒य स्पोषे॑ण॒ पोषे॑ण रा॒यो रा॒य स्पोषे॑ण । \newline
25. पोषे॑णे न्द्रि॒य मि॑न्द्रि॒यम् पोषे॑ण॒ पोषे॑णे न्द्रि॒यम् । \newline
26. इ॒न्द्रि॒यम् मे॑ म इन्द्रि॒य मि॑न्द्रि॒यम् मे᳚ । \newline
27. मे॒ वी॒र्यं॑ ॅवी॒र्य॑म् मे मे वी॒र्य᳚म् । \newline
28. वी॒र्य॑म् मा मा वी॒र्यं॑ ॅवी॒र्य॑म् मा । \newline
29. मा निर् णिर् मा मा निः । \newline
30. निर् व॑धीर् वधी॒र् निर् णिर् व॑धीः । \newline
31. व॒धी॒ रितीति॑ वधीर् वधी॒ रिति॑ । \newline
32. इत्या॒शिष॑ मा॒शिष॒ मिती त्या॒शिष᳚म् । \newline
33. आ॒शिष॑ मे॒वैवाशिष॑ मा॒शिष॑ मे॒व । \newline
34. आ॒शिष॒मित्या᳚ - शिष᳚म् । \newline
35. ए॒वैता मे॒ता मे॒वैवैताम् । \newline
36. ए॒ता मैता मे॒ता मा । \newline
37. आ शा᳚स्ते शास्त॒ आ शा᳚स्ते । \newline
38. शा॒स्त॒ इ॒न्द्रि॒यस्ये᳚ न्द्रि॒यस्य॑ शास्ते शास्त इन्द्रि॒यस्य॑ । \newline
39. इ॒न्द्रि॒यस्य॑ वी॒र्य॑स्य वी॒र्य॑स्ये न्द्रि॒यस्ये᳚ न्द्रि॒यस्य॑ वी॒र्य॑स्य । \newline
40. वी॒र्य॑स्य प्र॒जायै᳚ प्र॒जायै॑ वी॒र्य॑स्य वी॒र्य॑स्य प्र॒जायै᳚ । \newline
41. प्र॒जायै॑ पशू॒नाम् प॑शू॒नाम् प्र॒जायै᳚ प्र॒जायै॑ पशू॒नाम् । \newline
42. प्र॒जाया॒ इति॑ प्र - जायै᳚ । \newline
43. प॒शू॒ना मनि॑र्घाता॒या नि॑र्घाताय पशू॒नाम् प॑शू॒ना मनि॑र्घाताय । \newline
44. अनि॑र्घाताय द्र॒फ्सो द्र॒फ्सो ऽनि॑र्घाता॒या नि॑र्घाताय द्र॒फ्सः । \newline
45. अनि॑र्घाता॒येत्यनिः॑ - घा॒ता॒य॒ । \newline
46. द्र॒फ्स श्च॑स्कन्द चस्कन्द द्र॒फ्सो द्र॒फ्स श्च॑स्कन्द । \newline
47. च॒स्क॒न्द॒ पृ॒थि॒वीम् पृ॑थि॒वीम् च॑स्कन्द चस्कन्द पृथि॒वीम् । \newline
48. पृ॒थि॒वी मन्वनु॑ पृथि॒वीम् पृ॑थि॒वी मनु॑ । \newline
49. अनु॒ द्याम् द्या मन्वनु॒ द्याम् । \newline
50. द्या मि॒म मि॒मम् द्याम् द्या मि॒मम् । \newline
51. इ॒मम् च॑ चे॒ म मि॒मम् च॑ । \newline
52. च॒ योनिं॒ ॅयोनि॑म् च च॒ योनि᳚म् । \newline
53. योनि॒ मन्वनु॒ योनिं॒ ॅयोनि॒ मनु॑ । \newline
54. अनु॒ यो यो ऽन्वनु॒ यः । \newline
55. यश्च॑ च॒ यो यश्च॑ । \newline
56. च॒ पूर्वः॒ पूर्व॑श्च च॒ पूर्वः॑ । \newline
57. पूर्व॒ इति॒ पूर्वः॑ । \newline
58. तृ॒तीयं॒ ॅयोनिं॒ ॅयोनि॑म् तृ॒तीय॑म् तृ॒तीयं॒ ॅयोनि᳚म् । \newline
59. योनि॒ मन्वनु॒ योनिं॒ ॅयोनि॒ मनु॑ । \newline
60. अनु॑ स॒ञ्चर॑न्तꣳ स॒ञ्चर॑न्त॒ मन्वनु॑ स॒ञ्चर॑न्तम् । \newline
61. स॒ञ्चर॑न्तम् द्र॒फ्सम् द्र॒फ्सꣳ स॒ञ्चर॑न्तꣳ स॒ञ्चर॑न्तम् द्र॒फ्सम् । \newline
62. स॒ञ्चर॑न्त॒मिति॑ सं - चर॑न्तम् । \newline
63. द्र॒फ्सम् जु॑होमि जुहोमि द्र॒फ्सम् द्र॒फ्सम् जु॑होमि । \newline
64. जु॒हो॒ म्यन्वनु॑ जुहोमि जुहो॒ म्यनु॑ । \newline
65. अनु॑ स॒प्त स॒प्तान्वनु॑ स॒प्त । \newline
66. स॒प्त होत्रा॒ होत्राः᳚ स॒प्त स॒प्त होत्राः᳚ । \newline
67. होत्रा॒ इति॒ होत्राः᳚ । \newline

\textbf{Ghana Paata } \newline

1. प्र॒थ॒मो ऽꣳ॑शु रꣳ॒॒शुः प्र॑थ॒मः प्र॑थ॒मो ऽꣳ॑शुः स्कन्द॑ति॒ स्कन्द॑ त्यꣳ॒॒शुः प्र॑थ॒मः 
प्र॑थ॒मो ऽꣳ॑शुः स्कन्द॑ति । \newline
2. अꣳ॒॒शुः स्कन्द॑ति॒ स्कन्द॑ त्यꣳ॒॒शु रꣳ॒॒शुः स्कन्द॑ति॒ स स स्कन्द॑ त्यꣳ॒॒शु रꣳ॒॒शुः स्कन्द॑ति॒ सः । \newline
3. स्कन्द॑ति॒ स स स्कन्द॑ति॒ स्कन्द॑ति॒ स ई᳚श्व॒र ई᳚श्व॒रः स स्कन्द॑ति॒ स्कन्द॑ति॒ स ई᳚श्व॒रः । \newline
4. स ई᳚श्व॒र ई᳚श्व॒रः स स ई᳚श्व॒र इ॑न्द्रि॒य मि॑न्द्रि॒य मी᳚श्व॒रः स स ई᳚श्व॒र इ॑न्द्रि॒यम् । \newline
5. ई॒श्व॒र इ॑न्द्रि॒य मि॑न्द्रि॒य मी᳚श्व॒र ई᳚श्व॒र इ॑न्द्रि॒यं ॅवी॒र्यं॑ ॅवी॒र्य॑ मिन्द्रि॒य मी᳚श्व॒र ई᳚श्व॒र इ॑न्द्रि॒यं ॅवी॒र्य᳚म् । \newline
6. इ॒न्द्रि॒यं ॅवी॒र्यं॑ ॅवी॒र्य॑ मिन्द्रि॒य मि॑न्द्रि॒यं ॅवी॒र्य॑म् प्र॒जाम् प्र॒जां ॅवी॒र्य॑ मिन्द्रि॒य मि॑न्द्रि॒यं ॅवी॒र्य॑म् प्र॒जाम् । \newline
7. वी॒र्य॑म् प्र॒जाम् प्र॒जां ॅवी॒र्यं॑ ॅवी॒र्य॑म् प्र॒जाम् प॒शून् प॒शून् प्र॒जां ॅवी॒र्यं॑ ॅवी॒र्य॑म् प्र॒जाम् प॒शून् । \newline
8. प्र॒जाम् प॒शून् प॒शून् प्र॒जाम् प्र॒जाम् प॒शून्. यज॑मानस्य॒ यज॑मानस्य प॒शून् प्र॒जाम् प्र॒जाम् प॒शून्. यज॑मानस्य । \newline
9. प्र॒जामिति॑ प्र - जाम् । \newline
10. प॒शून्. यज॑मानस्य॒ यज॑मानस्य प॒शून् प॒शून्. यज॑मानस्य॒ निर्.ह॑न्तो॒र् निर्.ह॑न्तो॒र् यज॑मानस्य प॒शून् प॒शून्. यज॑मानस्य॒ निर्.ह॑न्तोः । \newline
11. यज॑मानस्य॒ निर्.ह॑न्तो॒र् निर्.ह॑न्तो॒र् यज॑मानस्य॒ यज॑मानस्य॒ निर्.ह॑न्तो॒ स्तम् तम् निर्.ह॑न्तो॒र् यज॑मानस्य॒ यज॑मानस्य॒ निर्.ह॑न्तो॒ स्तम् । \newline
12. निर्.ह॑न्तो॒ स्तम् तम् निर्.ह॑न्तो॒र् निर्.ह॑न्तो॒ स्त म॒भ्य॑भि तम् निर्.ह॑न्तो॒र् निर्.ह॑न्तो॒ स्त म॒भि । \newline
13. निर्.ह॑न्तो॒रिति॒ निः - ह॒न्तोः॒ । \newline
14. त म॒भ्य॑भि तम् त म॒भि म॑न्त्रयेत मन्त्रयेता॒भि तम् त म॒भि म॑न्त्रयेत । \newline
15. अ॒भि म॑न्त्रयेत मन्त्रयेता॒ भ्य॑भि म॑न्त्रये॒ता म॑न्त्रयेता॒ भ्य॑भि म॑न्त्रये॒ता । \newline
16. म॒न्त्र॒ये॒ता म॑न्त्रयेत मन्त्रये॒ता मा॒ मा ऽऽम॑न्त्रयेत मन्त्रये॒ता मा᳚ । \newline
17. आ मा॒ मा ऽऽमा᳚ ऽस्का नस्का॒न् मा ऽऽमा᳚ ऽस्कान् । \newline
18. मा॒ ऽस्का॒ न॒स्का॒न् मा॒ मा॒ ऽस्का॒न् थ्स॒ह स॒हास्का᳚न् मा मा ऽस्कान् थ्स॒ह । \newline
19. अ॒स्का॒न् थ्स॒ह स॒हास्का॑ नस्कान् थ्स॒ह प्र॒जया᳚ प्र॒जया॑ स॒हास्का॑ नस्कान् थ्स॒ह प्र॒जया᳚ । \newline
20. स॒ह प्र॒जया᳚ प्र॒जया॑ स॒ह स॒ह प्र॒जया॑ स॒ह स॒ह प्र॒जया॑ स॒ह स॒ह प्र॒जया॑ स॒ह । \newline
21. प्र॒जया॑ स॒ह स॒ह प्र॒जया᳚ प्र॒जया॑ स॒ह रा॒यो रा॒यः स॒ह प्र॒जया᳚ प्र॒जया॑ स॒ह रा॒यः । \newline
22. प्र॒जयेति॑ प्र - जया᳚ । \newline
23. स॒ह रा॒यो रा॒यः स॒ह स॒ह रा॒य स्पोषे॑ण॒ पोषे॑ण रा॒यः स॒ह स॒ह रा॒य स्पोषे॑ण । \newline
24. रा॒य स्पोषे॑ण॒ पोषे॑ण रा॒यो रा॒य स्पोषे॑णे न्द्रि॒य मि॑न्द्रि॒यम् पोषे॑ण रा॒यो रा॒य स्पोषे॑णे न्द्रि॒यम् । \newline
25. पोषे॑णे न्द्रि॒य मि॑न्द्रि॒यम् पोषे॑ण॒ पोषे॑णे न्द्रि॒यम् मे॑ म इन्द्रि॒यम् पोषे॑ण॒ पोषे॑णे न्द्रि॒यम् मे᳚ । \newline
26. इ॒न्द्रि॒यम् मे॑ म इन्द्रि॒य मि॑न्द्रि॒यम् मे॑ वी॒र्यं॑ ॅवी॒र्य॑म् म इन्द्रि॒य मि॑न्द्रि॒यम् मे॑ वी॒र्य᳚म् । \newline
27. मे॒ वी॒र्यं॑ ॅवी॒र्य॑म् मे मे वी॒र्य॑म् मा मा वी॒र्य॑म् मे मे वी॒र्य॑म् मा । \newline
28. वी॒र्य॑म् मा मा वी॒र्यं॑ ॅवी॒र्य॑म् मा निर् णिर् मा वी॒र्यं॑ ॅवी॒र्य॑म् मा निः । \newline
29. मा निर् णिर् मा मा निर् व॑धीर् वधी॒र् निर् मा मा निर् व॑धीः । \newline
30. निर् व॑धीर् वधी॒र् निर् णिर् व॑धी॒ रितीति॑ वधी॒र् निर् णिर् व॑धी॒ रिति॑ । \newline
31. व॒धी॒ रितीति॑ वधीर् वधी॒ रित्या॒शिष॑ मा॒शिष॒ मिति॑ वधीर् वधी॒ रित्या॒शिष᳚म् । \newline
32. इत्या॒शिष॑ मा॒शिष॒ मिती त्या॒शिष॑ मे॒वैवाशिष॒ मिती त्या॒शिष॑ मे॒व । \newline
33. आ॒शिष॑ मे॒वैवाशिष॑ मा॒शिष॑ मे॒वैता मे॒ता मे॒वाशिष॑ मा॒शिष॑ मे॒वैताम् । \newline
34. आ॒शिष॒मित्या᳚ - शिष᳚म् । \newline
35. ए॒वैता मे॒ता मे॒वैवैता मैता मे॒वैवैता मा । \newline
36. ए॒ता मैता मे॒ता मा शा᳚स्ते शास्त॒ ऐता मे॒ता मा शा᳚स्ते । \newline
37. आ शा᳚स्ते शास्त॒ आ शा᳚स्त इन्द्रि॒यस्ये᳚ न्द्रि॒यस्य॑ शास्त॒ आ शा᳚स्त इन्द्रि॒यस्य॑ । \newline
38. शा॒स्त॒ इ॒न्द्रि॒यस्ये᳚ न्द्रि॒यस्य॑ शास्ते शास्त इन्द्रि॒यस्य॑ वी॒र्य॑स्य वी॒र्य॑स्ये न्द्रि॒यस्य॑ शास्ते शास्त इन्द्रि॒यस्य॑ वी॒र्य॑स्य । \newline
39. इ॒न्द्रि॒यस्य॑ वी॒र्य॑स्य वी॒र्य॑स्ये न्द्रि॒यस्ये᳚ न्द्रि॒यस्य॑ वी॒र्य॑स्य प्र॒जायै᳚ प्र॒जायै॑ वी॒र्य॑स्ये न्द्रि॒यस्ये᳚ न्द्रि॒यस्य॑ वी॒र्य॑स्य प्र॒जायै᳚ । \newline
40. वी॒र्य॑स्य प्र॒जायै᳚ प्र॒जायै॑ वी॒र्य॑स्य वी॒र्य॑स्य प्र॒जायै॑ पशू॒नाम् प॑शू॒नाम् प्र॒जायै॑ वी॒र्य॑स्य वी॒र्य॑स्य प्र॒जायै॑ पशू॒नाम् । \newline
41. प्र॒जायै॑ पशू॒नाम् प॑शू॒नाम् प्र॒जायै᳚ प्र॒जायै॑ पशू॒ना मनि॑र्घाता॒या नि॑र्घाताय पशू॒नाम् प्र॒जायै᳚ प्र॒जायै॑ पशू॒ना मनि॑र्घाताय । \newline
42. प्र॒जाया॒ इति॑ प्र - जायै᳚ । \newline
43. प॒शू॒ना मनि॑र्घाता॒या नि॑र्घाताय पशू॒नाम् प॑शू॒ना मनि॑र्घाताय द्र॒फ्सो द्र॒फ्सो ऽनि॑र्घाताय पशू॒नाम् प॑शू॒ना मनि॑र्घाताय द्र॒फ्सः । \newline
44. अनि॑र्घाताय द्र॒फ्सो द्र॒फ्सो ऽनि॑र्घाता॒या नि॑र्घाताय द्र॒फ्स श्च॑स्कन्द चस्कन्द द्र॒फ्सो ऽनि॑र्घाता॒या नि॑र्घाताय द्र॒फ्स श्च॑स्कन्द । \newline
45. अनि॑र्घाता॒येत्यनिः॑ - घा॒ता॒य॒ । \newline
46. द्र॒फ्स श्च॑स्कन्द चस्कन्द द्र॒फ्सो द्र॒फ्स श्च॑स्कन्द पृथि॒वीम् पृ॑थि॒वीम् च॑स्कन्द द्र॒फ्सो द्र॒फ्स श्च॑स्कन्द पृथि॒वीम् । \newline
47. च॒स्क॒न्द॒ पृ॒थि॒वीम् पृ॑थि॒वीम् च॑स्कन्द चस्कन्द पृथि॒वी मन्वनु॑ पृथि॒वीम् च॑स्कन्द चस्कन्द पृथि॒वी मनु॑ । \newline
48. पृ॒थि॒वी मन्वनु॑ पृथि॒वीम् पृ॑थि॒वी मनु॒ द्याम् द्या मनु॑ पृथि॒वीम् पृ॑थि॒वी मनु॒ द्याम् । \newline
49. अनु॒ द्याम् द्या मन्वनु॒ द्या मि॒म मि॒मम् द्या मन्वनु॒ द्या मि॒मम् । \newline
50. द्या मि॒म मि॒मम् द्याम् द्या मि॒मम् च॑ चे॒ मम् द्याम् द्या मि॒मम् च॑ । \newline
51. इ॒मम् च॑ चे॒ म मि॒मम् च॒ योनिं॒ ॅयोनि॑म् चे॒ म मि॒मम् च॒ योनि᳚म् । \newline
52. च॒ योनिं॒ ॅयोनि॑म् च च॒ योनि॒ मन्वनु॒ योनि॑म् च च॒ योनि॒ मनु॑ । \newline
53. योनि॒ मन्वनु॒ योनिं॒ ॅयोनि॒ मनु॒ यो यो ऽनु॒ योनिं॒ ॅयोनि॒ मनु॒ यः । \newline
54. अनु॒ यो यो ऽन्वनु॒ यश्च॑ च॒ यो ऽन्वनु॒ यश्च॑ । \newline
55. यश्च॑ च॒ यो यश्च॒ पूर्वः॒ पूर्व॑श्च॒ यो यश्च॒ पूर्वः॑ । \newline
56. च॒ पूर्वः॒ पूर्व॑श्च च॒ पूर्वः॑ । \newline
57. पूर्व॒ इति॒ पूर्वः॑ । \newline
58. तृ॒तीयं॒ ॅयोनिं॒ ॅयोनि॑म् तृ॒तीय॑म् तृ॒तीयं॒ ॅयोनि॒ मन्वनु॒ योनि॑म् तृ॒तीय॑म् तृ॒तीयं॒ ॅयोनि॒ मनु॑ । \newline
59. योनि॒ मन्वनु॒ योनिं॒ ॅयोनि॒ मनु॑ स॒ञ्चर॑न्तꣳ स॒ञ्चर॑न्त॒ मनु॒ योनिं॒ ॅयोनि॒ मनु॑ स॒ञ्चर॑न्तम् । \newline
60. अनु॑ स॒ञ्चर॑न्तꣳ स॒ञ्चर॑न्त॒ मन्वनु॑ स॒ञ्चर॑न्तम् द्र॒फ्सम् द्र॒फ्सꣳ स॒ञ्चर॑न्त॒ मन्वनु॑ स॒ञ्चर॑न्तम् द्र॒फ्सम् । \newline
61. स॒ञ्चर॑न्तम् द्र॒फ्सम् द्र॒फ्सꣳ स॒ञ्चर॑न्तꣳ स॒ञ्चर॑न्तम् द्र॒फ्सम् जु॑होमि जुहोमि द्र॒फ्सꣳ स॒ञ्चर॑न्तꣳ स॒ञ्चर॑न्तम् द्र॒फ्सम् जु॑होमि । \newline
62. स॒ञ्चर॑न्त॒मिति॑ सं - चर॑न्तम् । \newline
63. द्र॒फ्सम् जु॑होमि जुहोमि द्र॒फ्सम् द्र॒फ्सम् जु॑हो॒ म्यन्वनु॑ जुहोमि द्र॒फ्सम् द्र॒फ्सम् जु॑हो॒ म्यनु॑ । \newline
64. जु॒हो॒ म्यन्वनु॑ जुहोमि जुहो॒ म्यनु॑ स॒प्त स॒प्तानु॑ जुहोमि जुहो॒ म्यनु॑ स॒प्त । \newline
65. अनु॑ स॒प्त स॒प्तान्वनु॑ स॒प्त होत्रा॒ होत्राः᳚ स॒प्तान्वनु॑ स॒प्त होत्राः᳚ । \newline
66. स॒प्त होत्रा॒ होत्राः᳚ स॒प्त स॒प्त होत्राः᳚ । \newline
67. होत्रा॒ इति॒ होत्राः᳚ । \newline
\pagebreak
\markright{ TS 3.1.9.1  \hfill https://www.vedavms.in \hfill}

\section{ TS 3.1.9.1 }

\textbf{TS 3.1.9.1 } \newline
\textbf{Samhita Paata} \newline

यो वै दे॒वान् दे॑वयश॒सेना॒र्पय॑ति मनु॒ष्या᳚न् मनुष्ययश॒सेन॑ देवयश॒स्ये॑व दे॒वेषु॒ भव॑ति मनुष्ययश॒सी म॑नु॒ष्ये॑षु॒ यान् प्रा॒चीन॑-माग्रय॒णाद् ग्रहा᳚न् गृह्णी॒यात् तानु॑पाꣳ॒॒शु गृ॑ह्णीया॒द्यानू॒र्द्ध्वाꣳस्तानु॑पब्दि॒मतो॑ दे॒वाने॒व तद्दे॑वयश॒सेना᳚र्पयति मनु॒ष्या᳚न् मनुष्ययश॒सेन॑ देवयश॒स्ये॑व दे॒वेषु॑ भवति मनुष्ययश॒सी म॑नु॒ष्ये᳚ष्व॒ग्निः प्रा॑तस्सव॒ने पा᳚त्व॒स्मान्. वै᳚श्वान॒रो म॑हि॒ना वि॒श्वश॑म्भूः । स नः॑ पाव॒को द्रवि॑णं दधा॒त्वा - [  ] \newline

\textbf{Pada Paata} \newline

यः । वै । दे॒वान् । दे॒व॒य॒श॒सेनेति॑ देव - य॒श॒सेन॑ । अ॒र्पय॑ति । म॒नु॒ष्यान्॑ । म॒नु॒ष्य॒य॒श॒सेनेति॑ मनुष्य - य॒श॒सेन॑ । दे॒व॒य॒श॒सीति॑ देव - य॒श॒सी । ए॒व । दे॒वेषु॑ । भव॑ति । म॒नु॒ष्य॒य॒श॒सीति॑ मनुष्य-य॒श॒सी । म॒नु॒ष्ये॑षु । यान् । प्रा॒चीन᳚म् । आ॒ग्र॒य॒णात् । ग्रहान्॑ । गृ॒ह्णी॒यात् । तान् । उ॒पाꣳ॒॒श्वित्यु॑प - अꣳ॒॒शु । गृ॒ह्णी॒या॒त् । यान् । ऊ॒द्‌र्ध्वान् । तान् । उ॒प॒ब्दि॒मत॒ इत्यु॑पब्दि - मतः॑ । दे॒वान् । ए॒व । तत् । दे॒व॒य॒श॒सेनेति॑ देव - य॒श॒सेन॑ । अ॒र्प॒य॒ति॒ । म॒नु॒ष्यान्॑ । म॒नु॒ष्य॒य॒श॒सेनेति॑ मनुष्य - य॒श॒सेन॑ । दे॒व॒य॒श॒सीति॑ देव - य॒श॒सी । ए॒व । दे॒वेषु॑ । भ॒व॒ति॒ । म॒नु॒ष्य॒य॒श॒सीति॑ मनुष्य - य॒श॒सी । म॒नु॒ष्ये॑षु । अ॒ग्निः । प्रा॒त॒स्स॒व॒न इति॑ प्रातः - स॒व॒ने । पा॒तु॒ ।अ॒स्मान् । वै॒श्वा॒न॒रः । म॒हि॒ना । वि॒श्वश॑म्भू॒रिति॑ वि॒श्व - श॒म्भूः॒ ॥ सः । नः॒ । पा॒व॒कः । द्रवि॑णम् । द॒धा॒तु॒ ।  \newline


\textbf{Krama Paata} \newline

यो वै । वै दे॒वान् । दे॒वान् दे॑वयश॒सेन॑ । दे॒व॒य॒श॒सेना॒र्पय॑ति । दे॒व॒य॒श॒सेनेति॑ देव - य॒श॒सेन॑ । अ॒र्पय॑ति मनु॒ष्यान्॑ । म॒नु॒ष्या᳚न् मनुष्ययश॒सेन॑ । म॒नु॒ष्य॒य॒श॒सेन॑ देवयश॒सी । म॒नु॒ष्य॒य॒श॒सेनेति॑ मनुष्य - य॒श॒सेन॑ । दे॒व॒य॒श॒स्ये॑व । दे॒व॒य॒श॒सीति॑ देव - य॒श॒सी । ए॒व दे॒वेषु॑ । दे॒वेषु॒ भव॑ति । भव॑ति मनुष्ययश॒सी । म॒नु॒ष्य॒य॒श॒सी म॑नु॒ष्ये॑षु । म॒नु॒ष्य॒य॒श॒सीति॑ मनुष्य - य॒श॒सी । म॒नु॒ष्ये॑षु॒ यान् । यान् प्रा॒चीन᳚म् । प्रा॒चीन॑माग्रय॒णात् । आ॒ग्र॒य॒णाद् ग्रहान्॑ । ग्रहा᳚न् गृह्णी॒यात् । गृ॒ह्णी॒यात् तान् । तानु॑पाꣳ॒॒शु । उ॒पाꣳ॒॒शु गृ॑ह्णीयात् । उ॒पाꣳ॒॒श्वित्यु॑प - अꣳ॒॒शु । गृ॒ह्णी॒या॒द् यान् । यानू॒र्द्ध्वान् । ऊ॒र्द्ध्वाꣳस्तान् । तानु॑पब्दि॒मतः॑ । उ॒प॒ब्दि॒मतो॑ दे॒वान् । उ॒प॒ब्दि॒मत॒ इत्यु॑पब्दि - मतः॑ । दे॒वाने॒व । ए॒व तत् । तद् दे॑वयश॒सेन॑ । दे॒व॒य॒श॒सेना᳚र्पयति । दे॒व॒य॒श॒सेनेति॑ देव - य॒श॒सेन॑ । अ॒र्प॒य॒ति॒ म॒नु॒ष्यान्॑ । म॒नु॒ष्या᳚न् मनुष्ययश॒सेन॑ । म॒नु॒ष्य॒य॒श॒सेन॑ देवयश॒सी । म॒नु॒ष्य॒य॒श॒सेनेति॑ मनुष्य - य॒श॒सेन॑ । दे॒व॒य॒श॒स्ये॑व । दे॒व॒य॒श॒सीति॑ देव - य॒श॒सी । ए॒व दे॒वेषु॑ । दे॒वेषु॑ भवति । भ॒व॒ति॒ म॒नु॒ष्य॒य॒श॒सी । म॒नु॒ष्य॒य॒श॒सी म॑नु॒ष्ये॑षु । म॒नु॒ष्य॒य॒श॒सीति॑ मनुष्य - य॒श॒सी । म॒नु॒ष्ये᳚ष्व॒ग्निः । अ॒ग्निः प्रा॑तस्सव॒ने । प्रा॒त॒स्स॒व॒ने पा॑तु । प्रा॒त॒स्स॒व॒न इति॑ प्रातः - स॒व॒ने । पा॒त्व॒स्मान् । अ॒स्मान्. वै᳚श्वान॒रः । वै॒श्वा॒न॒रो म॑हि॒ना । म॒हि॒ना वि॒श्वश॑म्भूः । वि॒श्वश॑म्भू॒रिति॑ वि॒श्व - श॒म्भूः॒ ॥ स नः॑ । नः॒ पा॒व॒कः । पा॒व॒को द्रवि॑णम् । द्रवि॑णम् दधातु । द॒धा॒त्वायु॑ष्मन्तः \newline

\textbf{Jatai Paata} \newline

1. यो वै वै यो यो वै । \newline
2. वै दे॒वान् दे॒वान्. वै वै दे॒वान् । \newline
3. दे॒वान् दे॑वयश॒सेन॑ देवयश॒सेन॑ दे॒वान् दे॒वान् दे॑वयश॒सेन॑ । \newline
4. दे॒व॒य॒श॒सेना॒ र्पय॑ त्य॒र्पय॑ति देवयश॒सेन॑ देवयश॒सेना॒ र्पय॑ति । \newline
5. दे॒व॒य॒श॒सेनेति॑ देव - य॒श॒सेन॑ । \newline
6. अ॒र्पय॑ति मनु॒ष्या᳚न् मनु॒ष्या॑ न॒र्पय॑ त्य॒र्पय॑ति मनु॒ष्यान्॑ । \newline
7. म॒नु॒ष्या᳚न् मनुष्ययश॒सेन॑ मनुष्ययश॒सेन॑ मनु॒ष्या᳚न् मनु॒ष्या᳚न् मनुष्ययश॒सेन॑ । \newline
8. म॒नु॒ष्य॒य॒श॒सेन॑ देवयश॒सी दे॑वयश॒सी म॑नुष्ययश॒सेन॑ मनुष्ययश॒सेन॑ देवयश॒सी । \newline
9. म॒नु॒ष्य॒य॒श॒सेनेति॑ मनुष्य - य॒श॒सेन॑ । \newline
10. दे॒व॒य॒श॒ स्ये॑वैव दे॑वयश॒सी दे॑वयश॒ स्ये॑व । \newline
11. दे॒व॒य॒श॒सीति॑ देव - य॒श॒सी । \newline
12. ए॒व दे॒वेषु॑ दे॒वे ष्वे॒वैव दे॒वेषु॑ । \newline
13. दे॒वेषु॒ भव॑ति॒ भव॑ति दे॒वेषु॑ दे॒वेषु॒ भव॑ति । \newline
14. भव॑ति मनुष्ययश॒सी म॑नुष्ययश॒सी भव॑ति॒ भव॑ति मनुष्ययश॒सी । \newline
15. म॒नु॒ष्य॒य॒श॒सी म॑नु॒ष्ये॑षु मनु॒ष्ये॑षु मनुष्ययश॒सी म॑नुष्ययश॒सी म॑नु॒ष्ये॑षु । \newline
16. म॒नु॒ष्य॒य॒श॒सीति॑ मनुष्य - य॒श॒सी । \newline
17. म॒नु॒ष्ये॑षु॒ यान्. यान् म॑नु॒ष्ये॑षु मनु॒ष्ये॑षु॒ यान् । \newline
18. यान् प्रा॒चीन॑म् प्रा॒चीनं॒ ॅयान्. यान् प्रा॒चीन᳚म् । \newline
19. प्रा॒चीन॑ माग्रय॒णा दा᳚ग्रय॒णात् प्रा॒चीन॑म् प्रा॒चीन॑ माग्रय॒णात् । \newline
20. आ॒ग्र॒य॒णाद् ग्रहा॒न् ग्रहा॑ नाग्रय॒णा दा᳚ग्रय॒णाद् ग्रहान्॑ । \newline
21. ग्रहा᳚न् गृह्णी॒याद् गृ॑ह्णी॒याद् ग्रहा॒न् ग्रहा᳚न् गृह्णी॒यात् । \newline
22. गृ॒ह्णी॒यात् ताꣳ स्तान् गृ॑ह्णी॒याद् गृ॑ह्णी॒यात् तान् । \newline
23. ता नु॑पाꣳ॒॒शू॑ पाꣳ॒॒शु ताꣳ स्ता नु॑पाꣳ॒॒शु । \newline
24. उ॒पाꣳ॒॒शु गृ॑ह्णीयाद् गृह्णीया दुपाꣳ॒॒शू॑ पाꣳ॒॒शु गृ॑ह्णीयात् । \newline
25. उ॒पाꣳ॒॒श्वित्यु॑प - अꣳ॒॒शु । \newline
26. गृ॒ह्णी॒या॒द् यान्. यान् गृ॑ह्णीयाद् गृह्णीया॒द् यान् । \newline
27. या नू॒र्द्ध्वा नू॒र्द्ध्वान्. यान्. या नू॒र्द्ध्वान् । \newline
28. ऊ॒र्द्ध्वाꣳ ताꣳ स्ता नू॒र्द्ध्वा नू॒र्द्ध्वाꣳ तान् । \newline
29. ता नु॑पब्दि॒मत॑ उपब्दि॒मत॒ स्ताꣳ स्ता नु॑पब्दि॒मतः॑ । \newline
30. उ॒प॒ब्दि॒मतो॑ दे॒वान् दे॒वा नु॑पब्दि॒मत॑ उपब्दि॒मतो॑ दे॒वान् । \newline
31. उ॒प॒ब्दि॒मत॒ इत्यु॑पब्दि - मतः॑ । \newline
32. दे॒वा ने॒वैव दे॒वान् दे॒वा ने॒व । \newline
33. ए॒व तत् तदे॒वैव तत् । \newline
34. तद् दे॑वयश॒सेन॑ देवयश॒सेन॒ तत् तद् दे॑वयश॒सेन॑ । \newline
35. दे॒व॒य॒श॒सेना᳚ र्पयत्यर्पयति देवयश॒सेन॑ देवयश॒सेना᳚ र्पयति । \newline
36. दे॒व॒य॒श॒सेनेति॑ देव - य॒श॒सेन॑ । \newline
37. अ॒र्प॒य॒ति॒ म॒नु॒ष्या᳚न् मनु॒ष्या॑ नर्पय त्यर्पयति मनु॒ष्यान्॑ । \newline
38. म॒नु॒ष्या᳚न् मनुष्ययश॒सेन॑ मनुष्ययश॒सेन॑ मनु॒ष्या᳚न् मनु॒ष्या᳚न् मनुष्ययश॒सेन॑ । \newline
39. म॒नु॒ष्य॒य॒श॒सेन॑ देवयश॒सी दे॑वयश॒सी म॑नुष्ययश॒सेन॑ मनुष्ययश॒सेन॑ देवयश॒सी । \newline
40. म॒नु॒ष्य॒य॒श॒सेनेति॑ मनुष्य - य॒श॒सेन॑ । \newline
41. दे॒व॒य॒श॒ स्ये॑वैव दे॑वयश॒सी दे॑वयश॒ स्ये॑व । \newline
42. दे॒व॒य॒श॒सीति॑ देव - य॒श॒सी । \newline
43. ए॒व दे॒वेषु॑ दे॒वे ष्वे॒वैव दे॒वेषु॑ । \newline
44. दे॒वेषु॑ भवति भवति दे॒वेषु॑ दे॒वेषु॑ भवति । \newline
45. भ॒व॒ति॒ म॒नु॒ष्य॒य॒श॒सी म॑नुष्ययश॒सी भ॑वति भवति मनुष्ययश॒सी । \newline
46. म॒नु॒ष्य॒य॒श॒सी म॑नु॒ष्ये॑षु मनु॒ष्ये॑षु मनुष्ययश॒सी म॑नुष्ययश॒सी म॑नु॒ष्ये॑षु । \newline
47. म॒नु॒ष्य॒य॒श॒सीति॑ मनुष्य - य॒श॒सी । \newline
48. म॒नु॒ष्ये᳚ ष्व॒ग्नि र॒ग्निर् म॑नु॒ष्ये॑षु मनु॒ष्ये᳚ ष्व॒ग्निः । \newline
49. अ॒ग्निः प्रा॑तस्सव॒ने प्रा॑तस्सव॒ने᳚ ऽग्निर॒ग्निः प्रा॑तस्सव॒ने । \newline
50. प्रा॒त॒स्स॒व॒ने पा॑तु पातु प्रातस्सव॒ने प्रा॑तस्सव॒ने पा॑तु । \newline
51. प्रा॒त॒स्स॒व॒न इति॑ प्रातः - स॒व॒ने । \newline
52. पा॒त्व॒स्मा न॒स्मान् पा॑तु पात्व॒स्मान् । \newline
53. अ॒स्मान्. वै᳚श्वान॒रो वै᳚श्वान॒रो᳚ ऽस्मा न॒स्मान्. वै᳚श्वान॒रः । \newline
54. वै॒श्वा॒न॒रो म॑हि॒ना म॑हि॒ना वै᳚श्वान॒रो वै᳚श्वान॒रो म॑हि॒ना । \newline
55. म॒हि॒ना वि॒श्वश॑म्भूर् वि॒श्वश॑म्भूर् महि॒ना म॑हि॒ना वि॒श्वश॑म्भूः । \newline
56. वि॒श्वश॑म्भू॒रिति॑ वि॒श्व - श॒म्भूः॒ । \newline
57. स नो॑ नः॒ स स नः॑ । \newline
58. नः॒ पा॒व॒कः पा॑व॒को नो॑ नः पाव॒कः । \newline
59. पा॒व॒को द्रवि॑ण॒म् द्रवि॑णम् पाव॒कः पा॑व॒को द्रवि॑णम् । \newline
60. द्रवि॑णम् दधातु दधातु॒ द्रवि॑ण॒म् द्रवि॑णम् दधातु । \newline
61. द॒धा॒ त्वायु॑ष्मन्त॒ आयु॑ष्मन्तो दधातु दधा॒ त्वायु॑ष्मन्तः । \newline

\textbf{Ghana Paata } \newline

1. यो वै वै यो यो वै दे॒वान् दे॒वान्. वै यो यो वै दे॒वान् । \newline
2. वै दे॒वान् दे॒वान्. वै वै दे॒वान् दे॑वयश॒सेन॑ देवयश॒सेन॑ दे॒वान्. वै वै दे॒वान् दे॑वयश॒सेन॑ । \newline
3. दे॒वान् दे॑वयश॒सेन॑ देवयश॒सेन॑ दे॒वान् दे॒वान् दे॑वयश॒सेना॒ र्पय॑ त्य॒र्पय॑ति देवयश॒सेन॑ दे॒वान् दे॒वान् दे॑वयश॒सेना॒ र्पय॑ति । \newline
4. दे॒व॒य॒श॒सेना॒ र्पय॑ त्य॒र्पय॑ति देवयश॒सेन॑ देवयश॒सेना॒ र्पय॑ति मनु॒ष्या᳚न् मनु॒ष्या॑ न॒र्पय॑ति देवयश॒सेन॑ देवयश॒सेना॒ र्पय॑ति मनु॒ष्यान्॑ । \newline
5. दे॒व॒य॒श॒सेनेति॑ देव - य॒श॒सेन॑ । \newline
6. अ॒र्पय॑ति मनु॒ष्या᳚न् मनु॒ष्या॑ न॒र्पय॑ त्य॒र्पय॑ति मनु॒ष्या᳚न् मनुष्ययश॒सेन॑ मनुष्ययश॒सेन॑ मनु॒ष्या॑ न॒र्पय॑ त्य॒र्पय॑ति मनु॒ष्या᳚न् मनुष्ययश॒सेन॑ । \newline
7. म॒नु॒ष्या᳚न् मनुष्ययश॒सेन॑ मनुष्ययश॒सेन॑ मनु॒ष्या᳚न् मनु॒ष्या᳚न् मनुष्ययश॒सेन॑ देवयश॒सी दे॑वयश॒सी म॑नुष्ययश॒सेन॑ मनु॒ष्या᳚न् मनु॒ष्या᳚न् मनुष्ययश॒सेन॑ देवयश॒सी । \newline
8. म॒नु॒ष्य॒य॒श॒सेन॑ देवयश॒सी दे॑वयश॒सी म॑नुष्ययश॒सेन॑ मनुष्ययश॒सेन॑ देवयश॒स्ये॑वैव दे॑वयश॒सी म॑नुष्ययश॒सेन॑ मनुष्ययश॒सेन॑ देवयश॒स्ये॑व । \newline
9. म॒नु॒ष्य॒य॒श॒सेनेति॑ मनुष्य - य॒श॒सेन॑ । \newline
10. दे॒व॒य॒श॒स्ये॑वैव दे॑वयश॒सी दे॑वयश॒स्ये॑व दे॒वेषु॑ दे॒वेष्वे॒व दे॑वयश॒सी दे॑वयश॒स्ये॑व दे॒वेषु॑ । \newline
11. दे॒व॒य॒श॒सीति॑ देव - य॒श॒सी । \newline
12. ए॒व दे॒वेषु॑ दे॒वे ष्वे॒वैव दे॒वेषु॒ भव॑ति॒ भव॑ति दे॒वे ष्वे॒वैव दे॒वेषु॒ भव॑ति । \newline
13. दे॒वेषु॒ भव॑ति॒ भव॑ति दे॒वेषु॑ दे॒वेषु॒ भव॑ति मनुष्ययश॒सी म॑नुष्ययश॒सी भव॑ति दे॒वेषु॑ दे॒वेषु॒ भव॑ति मनुष्ययश॒सी । \newline
14. भव॑ति मनुष्ययश॒सी म॑नुष्ययश॒सी भव॑ति॒ भव॑ति मनुष्ययश॒सी म॑नु॒ष्ये॑षु मनु॒ष्ये॑षु मनुष्ययश॒सी भव॑ति॒ भव॑ति मनुष्ययश॒सी म॑नु॒ष्ये॑षु । \newline
15. म॒नु॒ष्य॒य॒श॒सी म॑नु॒ष्ये॑षु मनु॒ष्ये॑षु मनुष्ययश॒सी म॑नुष्ययश॒सी म॑नु॒ष्ये॑षु॒ यान्. यान् म॑नु॒ष्ये॑षु मनुष्ययश॒सी म॑नुष्ययश॒सी म॑नु॒ष्ये॑षु॒ यान् । \newline
16. म॒नु॒ष्य॒य॒श॒सीति॑ मनुष्य - य॒श॒सी । \newline
17. म॒नु॒ष्ये॑षु॒ यान्. यान् म॑नु॒ष्ये॑षु मनु॒ष्ये॑षु॒ यान् प्रा॒चीन॑म् प्रा॒चीनं॒ ॅयान् म॑नु॒ष्ये॑षु मनु॒ष्ये॑षु॒ यान् प्रा॒चीन᳚म् । \newline
18. यान् प्रा॒चीन॑म् प्रा॒चीनं॒ ॅयान्. यान् प्रा॒चीन॑ माग्रय॒णा दा᳚ग्रय॒णात् प्रा॒चीनं॒ ॅयान्. यान् प्रा॒चीन॑ माग्रय॒णात् । \newline
19. प्रा॒चीन॑ माग्रय॒णा दा᳚ग्रय॒णात् प्रा॒चीन॑म् प्रा॒चीन॑ माग्रय॒णाद् ग्रहा॒न् ग्रहा॑ नाग्रय॒णात् प्रा॒चीन॑म् प्रा॒चीन॑ माग्रय॒णाद् ग्रहान्॑ । \newline
20. आ॒ग्र॒य॒णाद् ग्रहा॒न् ग्रहा॑ नाग्रय॒णा दा᳚ग्रय॒णाद् ग्रहा᳚न् गृह्णी॒याद् गृ॑ह्णी॒याद् ग्रहा॑ नाग्रय॒णा दा᳚ग्रय॒णाद् ग्रहा᳚न् गृह्णी॒यात् । \newline
21. ग्रहा᳚न् गृह्णी॒याद् गृ॑ह्णी॒याद् ग्रहा॒न् ग्रहा᳚न् गृह्णी॒यात् ताꣳ स्तान् गृ॑ह्णी॒याद् ग्रहा॒न् ग्रहा᳚न् गृह्णी॒यात् तान् । \newline
22. गृ॒ह्णी॒यात् ताꣳ स्तान् गृ॑ह्णी॒याद् गृ॑ह्णी॒यात् ता नु॑पाꣳ॒॒शू॑ पाꣳ॒॒शु तान् गृ॑ह्णी॒याद् गृ॑ह्णी॒यात् ता नु॑पाꣳ॒॒शु । \newline
23. ता नु॑पाꣳ॒॒शू॑ पाꣳ॒॒शु ताꣳ स्ता नु॑पाꣳ॒॒शु गृ॑ह्णीयाद् गृह्णीया दुपाꣳ॒॒शु ताꣳ स्ता 
नु॑पाꣳ॒॒शु गृ॑ह्णीयात् । \newline
24. उ॒पाꣳ॒॒शु गृ॑ह्णीयाद् गृह्णीया दुपाꣳ॒॒शू॑ पाꣳ॒॒शु गृ॑ह्णीया॒द् यान्. यान् गृ॑ह्णीया दुपाꣳ॒॒शू॑ पाꣳ॒॒शु गृ॑ह्णीया॒द् यान् । \newline
25. उ॒पाꣳ॒॒श्वित्यु॑प - अꣳ॒॒शु । \newline
26. गृ॒ह्णी॒या॒द् यान्. यान् गृ॑ह्णीयाद् गृह्णीया॒द् या नू॒र्द्ध्वा नू॒र्द्ध्वान्. यान् गृ॑ह्णीयाद् गृह्णीया॒द् या नू॒र्द्ध्वान् । \newline
27. या नू॒र्द्ध्वा नू॒र्द्ध्वान्. यान्. या नू॒र्द्ध्वाꣳ ताꣳ स्ता नू॒र्द्ध्वान्. यान्. या नू॒र्द्ध्वाꣳ तान् । \newline
28. ऊ॒र्द्ध्वाꣳ ताꣳ स्ता नू॒र्द्ध्वा नू॒र्द्ध्वाꣳ ता नु॑पब्दि॒मत॑ उपब्दि॒मत॒ स्ता नू॒र्द्ध्वा नू॒र्द्ध्वाꣳ ता नु॑पब्दि॒मतः॑ । \newline
29. ता नु॑पब्दि॒मत॑ उपब्दि॒मत॒ स्ताꣳ स्ता नु॑पब्दि॒मतो॑ दे॒वान् दे॒वा नु॑पब्दि॒मत॒ स्ताꣳ स्ता नु॑पब्दि॒मतो॑ दे॒वान् । \newline
30. उ॒प॒ब्दि॒मतो॑ दे॒वान् दे॒वा नु॑पब्दि॒मत॑ उपब्दि॒मतो॑ दे॒वा ने॒वैव दे॒वा नु॑पब्दि॒मत॑ उपब्दि॒मतो॑ दे॒वा ने॒व । \newline
31. उ॒प॒ब्दि॒मत॒ इत्यु॑पब्दि - मतः॑ । \newline
32. दे॒वा ने॒वैव दे॒वान् दे॒वा ने॒व तत् तदे॒व दे॒वान् दे॒वा ने॒व तत् । \newline
33. ए॒व तत् तदे॒वैव तद् दे॑वयश॒सेन॑ देवयश॒सेन॒ तदे॒वैव तद् दे॑वयश॒सेन॑ । \newline
34. तद् दे॑वयश॒सेन॑ देवयश॒सेन॒ तत् तद् दे॑वयश॒सेना᳚ र्पय त्यर्पयति देवयश॒सेन॒ तत् तद् दे॑वयश॒सेना᳚ र्पयति । \newline
35. दे॒व॒य॒श॒सेना᳚ र्पय त्यर्पयति देवयश॒सेन॑ देवयश॒सेना᳚ र्पयति मनु॒ष्या᳚न् मनु॒ष्या॑ नर्पयति देवयश॒सेन॑ देवयश॒सेना᳚ र्पयति मनु॒ष्यान्॑ । \newline
36. दे॒व॒य॒श॒सेनेति॑ देव - य॒श॒सेन॑ । \newline
37. अ॒र्प॒य॒ति॒ म॒नु॒ष्या᳚न् मनु॒ष्या॑ नर्पय त्यर्पयति मनु॒ष्या᳚न् मनुष्ययश॒सेन॑ मनुष्ययश॒सेन॑ मनु॒ष्या॑ नर्पय त्यर्पयति मनु॒ष्या᳚न् मनुष्ययश॒सेन॑ । \newline
38. म॒नु॒ष्या᳚न् मनुष्ययश॒सेन॑ मनुष्ययश॒सेन॑ मनु॒ष्या᳚न् मनु॒ष्या᳚न् मनुष्ययश॒सेन॑ देवयश॒सी दे॑वयश॒सी म॑नुष्ययश॒सेन॑ मनु॒ष्या᳚न् मनु॒ष्या᳚न् मनुष्ययश॒सेन॑ देवयश॒सी । \newline
39. म॒नु॒ष्य॒य॒श॒सेन॑ देवयश॒सी दे॑वयश॒सी म॑नुष्ययश॒सेन॑ मनुष्ययश॒सेन॑ देवयश॒स्ये॑वैव दे॑वयश॒सी म॑नुष्ययश॒सेन॑ मनुष्ययश॒सेन॑ देवयश॒स्ये॑व । \newline
40. म॒नु॒ष्य॒य॒श॒सेनेति॑ मनुष्य - य॒श॒सेन॑ । \newline
41. दे॒व॒य॒श॒ स्ये॑वैव दे॑वयश॒सी दे॑वयश॒ स्ये॑व दे॒वेषु॑ दे॒वेष्वे॒व दे॑वयश॒सी दे॑वयश॒ स्ये॑व दे॒वेषु॑ । \newline
42. दे॒व॒य॒श॒सीति॑ देव - य॒श॒सी । \newline
43. ए॒व दे॒वेषु॑ दे॒वे ष्वे॒वैव दे॒वेषु॑ भवति भवति दे॒वे ष्वे॒वैव दे॒वेषु॑ भवति । \newline
44. दे॒वेषु॑ भवति भवति दे॒वेषु॑ दे॒वेषु॑ भवति मनुष्ययश॒सी म॑नुष्ययश॒सी भ॑वति दे॒वेषु॑ दे॒वेषु॑ भवति मनुष्ययश॒सी । \newline
45. भ॒व॒ति॒ म॒नु॒ष्य॒य॒श॒सी म॑नुष्ययश॒सी भ॑वति भवति मनुष्ययश॒सी म॑नु॒ष्ये॑षु मनु॒ष्ये॑षु मनुष्ययश॒सी भ॑वति भवति मनुष्ययश॒सी म॑नु॒ष्ये॑षु । \newline
46. म॒नु॒ष्य॒य॒श॒सी म॑नु॒ष्ये॑षु मनु॒ष्ये॑षु मनुष्ययश॒सी म॑नुष्ययश॒सी म॑नु॒ष्ये᳚ ष्व॒ग्नि र॒ग्निर् म॑नु॒ष्ये॑षु मनुष्ययश॒सी म॑नुष्ययश॒सी म॑नु॒ष्ये᳚ ष्व॒ग्निः । \newline
47. म॒नु॒ष्य॒य॒श॒सीति॑ मनुष्य - य॒श॒सी । \newline
48. म॒नु॒ ष्ये᳚ष्व॒ग्नि र॒ग्निर् म॑नु॒ष्ये॑षु मनु॒ष्ये᳚ष्व॒ग्निः प्रा॑तस्सव॒ने प्रा॑तस्सव॒ने᳚ ऽग्निर् म॑नु॒ष्ये॑षु मनु॒ष्ये᳚ ष्व॒ग्निः प्रा॑तस्सव॒ने । \newline
49. अ॒ग्निः प्रा॑तस्सव॒ने प्रा॑तस्सव॒ने᳚ ऽग्नि र॒ग्निः प्रा॑तस्सव॒ने पा॑तु पातु प्रातस्सव॒ने᳚ ऽग्नि र॒ग्निः प्रा॑तस्सव॒ने पा॑तु । \newline
50. प्रा॒त॒स्स॒व॒ने पा॑तु पातु प्रातस्सव॒ने प्रा॑तस्सव॒ने पा᳚त्व॒स्मा न॒स्मान् पा॑तु प्रातस्सव॒ने प्रा॑तस्सव॒ने पा᳚त्व॒स्मान् । \newline
51. प्रा॒त॒स्स॒व॒न इति॑ प्रातः - स॒व॒ने । \newline
52. पा॒त्व॒स्मा न॒स्मान् पा॑तु पात्व॒स्मान्. वै᳚श्वान॒रो वै᳚श्वान॒रो᳚ ऽस्मान् पा॑तु पात्व॒स्मान्. वै᳚श्वान॒रः । \newline
53. अ॒स्मान्. वै᳚श्वान॒रो वै᳚श्वान॒रो᳚ ऽस्मा न॒स्मान्. वै᳚श्वान॒रो म॑हि॒ना म॑हि॒ना वै᳚श्वान॒रो᳚ ऽस्मा न॒स्मान्. वै᳚श्वान॒रो म॑हि॒ना । \newline
54. वै॒श्वा॒न॒रो म॑हि॒ना म॑हि॒ना वै᳚श्वान॒रो वै᳚श्वान॒रो म॑हि॒ना वि॒श्वश॑म्भूर् वि॒श्वश॑म्भूर् महि॒ना वै᳚श्वान॒रो वै᳚श्वान॒रो म॑हि॒ना वि॒श्वश॑म्भूः । \newline
55. म॒हि॒ना वि॒श्वश॑म्भूर् वि॒श्वश॑म्भूर् महि॒ना म॑हि॒ना वि॒श्वश॑म्भूः । \newline
56. वि॒श्वश॑म्भू॒रिति॑ वि॒श्व - श॒म्भूः॒ । \newline
57. स नो॑ नः॒ स स नः॑ पाव॒कः पा॑व॒को नः॒ स स नः॑ पाव॒कः । \newline
58. नः॒ पा॒व॒कः पा॑व॒को नो॑ नः पाव॒को द्रवि॑ण॒म् द्रवि॑णम् पाव॒को नो॑ नः पाव॒को द्रवि॑णम् । \newline
59. पा॒व॒को द्रवि॑ण॒म् द्रवि॑णम् पाव॒कः पा॑व॒को द्रवि॑णम् दधातु दधातु॒ द्रवि॑णम् पाव॒कः पा॑व॒को द्रवि॑णम् दधातु । \newline
60. द्रवि॑णम् दधातु दधातु॒ द्रवि॑ण॒म् द्रवि॑णम् दधा॒ त्वायु॑ष्मन्त॒ आयु॑ष्मन्तो दधातु॒ द्रवि॑ण॒म् द्रवि॑णम् दधा॒ त्वायु॑ष्मन्तः । \newline
61. द॒धा॒ त्वायु॑ष्मन्त॒ आयु॑ष्मन्तो दधातु दधा॒ त्वायु॑ष्मन्तः स॒हभ॑क्षाः स॒हभ॑क्षा॒ आयु॑ष्मन्तो दधातु दधा॒ त्वायु॑ष्मन्तः स॒हभ॑क्षाः । \newline
\pagebreak
\markright{ TS 3.1.9.2  \hfill https://www.vedavms.in \hfill}

\section{ TS 3.1.9.2 }

\textbf{TS 3.1.9.2 } \newline
\textbf{Samhita Paata} \newline

यु॑ष्मन्तः स॒हभ॑क्षाः स्याम ॥ विश्वे॑ दे॒वा म॒रुत॒ इन्द्रो॑ अ॒स्मान॒स्मिन् द्वि॒तीये॒ सव॑ने॒ न ज॑ह्युः । आयु॑ष्मन्तः प्रि॒यमे॑षां॒ ॅवद॑न्तो व॒यं दे॒वानाꣳ॑ सुम॒तौ स्या॑म ॥ इ॒दं तृ॒तीयꣳ॒॒ सव॑नं कवी॒नामृ॒तेन॒ ये च॑म॒समैर॑यन्त । ते सौ॑धन्व॒नाः सुव॑रानशा॒नाः स्वि॑ष्टिं नो अ॒भि वसी॑यो नयन्तु ॥ आ॒यत॑नवती॒र्वा अ॒न्या आहु॑तयो हू॒यन्ते॑ऽनायत॒ना अ॒न्या या आ॑घा॒रव॑ती॒स्ता आ॒यतन॑वती॒र्याः - [  ] \newline

\textbf{Pada Paata} \newline

आयु॑ष्मन्तः । स॒हभ॑क्षा॒ इति॑ स॒ह-भ॒क्षाः॒ । स्या॒म॒ ॥ विश्वे᳚ । दे॒वाः । म॒रुतः॑ । इन्द्रः॑ । अ॒स्मान् । अ॒स्मिन्न् । द्वि॒तीये᳚ । सव॑ने । न । ज॒ह्युः॒ ॥ आयु॑ष्मन्तः । प्रि॒यम् । ए॒षा॒म् । वद॑न्तः । व॒यम् । दे॒वाना᳚म् । सु॒म॒ताविति॑ सु - म॒तौ । स्या॒म॒ ॥ इ॒दम् । तृ॒तीय᳚म् । सव॑नम् । क॒वी॒नाम् । ऋ॒तेन॑ । ये । च॒म॒सम् । ऐर॑यन्त ॥ ते । सौ॒ध॒न्व॒नाः । सुवः॑ । आ॒न॒शा॒नाः । स्वि॑ष्टि॒मिति॒ सु - इ॒ष्टि॒म् । नः॒ । अ॒भीति॑ । वसी॑यः । न॒य॒न्तु॒ ॥ आ॒यत॑नवती॒रित्या॒यत॑न - व॒तीः॒ । वै । अ॒न्याः । आहु॑तय॒ इत्या - हु॒त॒यः॒ । हू॒यन्ते᳚ । अ॒ना॒य॒त॒ना इत्य॑ना - य॒त॒नाः । अ॒न्याः । याः । आ॒घा॒रव॑ती॒रित्या॑घा॒र - व॒तीः॒ । ताः । आ॒यत॑नवती॒रित्या॒यत॑न - व॒तीः॒ । याः ।  \newline


\textbf{Krama Paata} \newline

आयु॑ष्मन्तः स॒हभ॑क्षाः । स॒हभ॑क्षाः स्याम । स॒हभ॑क्षा॒ इति॑ स॒ह - भ॒क्षाः॒ । स्या॒मेति॑ स्याम ॥ विश्वे॑ दे॒वाः । दे॒वा म॒रुतः॑ । म॒रुत॒ इन्द्रः॑ । इन्द्रो॑ अ॒स्मान् । अ॒स्मान॒स्मिन्न् । अ॒स्मिन् द्वि॒तीये᳚ । द्वि॒तीये॒ सव॑ने । सव॑ने॒ न । न ज॑ह्युः । ज॒ह्यु॒रिति॑ जह्युः ॥ आयु॑ष्मन्तः प्रि॒यम् । प्रि॒यमे॑षाम् । ए॒षां॒ ॅवद॑न्तः । वद॑न्तो व॒यम् । व॒यम् दे॒वाना᳚म् । दे॒वानाꣳ॑ सुम॒तौ । सु॒म॒तौ स्या॑म । सु॒म॒ताविति॑ सु - म॒तौ । स्या॒मेति॑ स्याम ॥ इ॒दम् तृ॒तीय᳚म् । तृ॒तीयꣳ॒॒ सव॑नम् । सव॑नम् कवी॒नाम् । क॒वी॒नामृ॒तेन॑ । ऋ॒तेन॒ ये । ये च॑म॒सम् । च॒म॒समैर॑यन्त । ऐर॑य॒न्तेत्यैर॑यन्त ॥ ते सौ॑धन्व॒नाः । सौ॒ध॒न्व॒नाः सुवः॑ । सुव॑रानशा॒नाः । आ॒न॒शा॒नाः स्वि॑ष्टिम् । स्वि॑ष्टिम् नः । स्वि॑ष्टि॒मिति॒ सु - इ॒ष्टि॒म् । नो॒ अ॒भि । अ॒भि वसी॑यः । वसी॑यो नयन्तु । न॒य॒न्त्विति॑ नयन्तु ॥ आ॒यत॑नवती॒र् वै । आ॒यत॑नवती॒रित्या॒यत॑न - व॒तीः॒ । वा अ॒न्याः । अ॒न्या आहु॑तयः । आहु॑तयो हू॒यन्ते᳚ । आहु॑तय॒ इत्या - हु॒त॒यः॒ । हू॒यन्ते॑ ऽनायत॒नाः । अ॒ना॒य॒त॒ना अ॒न्याः । अ॒ना॒य॒त॒ना इत्य॑ना - य॒त॒नाः । अ॒न्या याः । या आ॑घा॒रव॑तीः । आ॒घा॒रव॑ती॒ स्ताः । आ॒घा॒रव॑ती॒रित्या॑घा॒र - व॒तीः॒ । ता आ॒यत॑नवतीः । आ॒यत॑नवती॒र् याः । आ॒यत॑नवती॒रित्या॒यत॑न - व॒तीः॒ । याः सौ॒म्याः \newline

\textbf{Jatai Paata} \newline

1. आयु॑ष्मन्तः स॒हभ॑क्षाः स॒हभ॑क्षा॒ आयु॑ष्मन्त॒ आयु॑ष्मन्तः स॒हभ॑क्षाः । \newline
2. स॒हभ॑क्षाः स्याम स्याम स॒हभ॑क्षाः स॒हभ॑क्षाः स्याम । \newline
3. स॒हभ॑क्षा॒ इति॑ स॒ह - भ॒क्षाः॒ । \newline
4. स्या॒मेति॑ स्याम । \newline
5. विश्वे॑ दे॒वा दे॒वा विश्वे॒ विश्वे॑ दे॒वाः । \newline
6. दे॒वा म॒रुतो॑ म॒रुतो॑ दे॒वा दे॒वा म॒रुतः॑ । \newline
7. म॒रुत॒ इन्द्र॒ इन्द्रो॑ म॒रुतो॑ म॒रुत॒ इन्द्रः॑ । \newline
8. इन्द्रो॑ अ॒स्मा न॒स्मा निन्द्र॒ इन्द्रो॑ अ॒स्मान् । \newline
9. अ॒स्मा न॒स्मिन् न॒स्मिन् न॒स्मा न॒स्मा न॒स्मिन्न् । \newline
10. अ॒स्मिन् द्वि॒तीये᳚ द्वि॒तीये॒ ऽस्मिन् न॒स्मिन् द्वि॒तीये᳚ । \newline
11. द्वि॒तीये॒ सव॑ने॒ सव॑ने द्वि॒तीये᳚ द्वि॒तीये॒ सव॑ने । \newline
12. सव॑ने॒ न न सव॑ने॒ सव॑ने॒ न । \newline
13. न ज॑ह्युर् जह्यु॒र् न न ज॑ह्युः । \newline
14. ज॒ह्यु॒रिति॑ जह्युः । \newline
15. आयु॑ष्मन्तः प्रि॒यम् प्रि॒य मायु॑ष्मन्त॒ आयु॑ष्मन्तः प्रि॒यम् । \newline
16. प्रि॒य मे॑षा मेषाम् प्रि॒यम् प्रि॒य मे॑षाम् । \newline
17. ए॒षां॒ ॅवद॑न्तो॒ वद॑न्त एषा मेषां॒ ॅवद॑न्तः । \newline
18. वद॑न्तो व॒यं ॅव॒यं ॅवद॑न्तो॒ वद॑न्तो व॒यम् । \newline
19. व॒यम् दे॒वाना᳚म् दे॒वानां᳚ ॅव॒यं ॅव॒यम् दे॒वाना᳚म् । \newline
20. दे॒वानाꣳ॑ सुम॒तौ सु॑म॒तौ दे॒वाना᳚म् दे॒वानाꣳ॑ सुम॒तौ । \newline
21. सु॒म॒तौ स्या॑म स्याम सुम॒तौ सु॑म॒तौ स्या॑म । \newline
22. सु॒म॒ताविति॑ सु - म॒तौ । \newline
23. स्या॒मेति॑ स्याम । \newline
24. इ॒दम् तृ॒तीय॑म् तृ॒तीय॑ मि॒द मि॒दम् तृ॒तीय᳚म् । \newline
25. तृ॒तीयꣳ॒॒ सव॑नꣳ॒॒ सव॑नम् तृ॒तीय॑म् तृ॒तीयꣳ॒॒ सव॑नम् । \newline
26. सव॑नम् कवी॒नाम् क॑वी॒नाꣳ सव॑नꣳ॒॒ सव॑नम् कवी॒नाम् । \newline
27. क॒वी॒ना मृ॒तेन॒ र्तेन॑ कवी॒नाम् क॑वी॒ना मृ॒तेन॑ । \newline
28. ऋ॒तेन॒ ये य ऋ॒तेन॒ र्तेन॒ ये । \newline
29. ये च॑म॒सम् च॑म॒सं ॅये ये च॑म॒सम् । \newline
30. च॒म॒स मैर॑य॒ न्तैर॑यन्त चम॒सम् च॑म॒स मैर॑यन्त । \newline
31. ऐर॑य॒न्तेत्यैर॑यन्त । \newline
32. ते सौ॑धन्व॒नाः सौ॑धन्व॒नास्ते ते सौ॑धन्व॒नाः । \newline
33. सौ॒ध॒न्व॒नाः सुवः॒ सुवः॑ सौधन्व॒नाः सौ॑धन्व॒नाः सुवः॑ । \newline
34. सुव॑ रानशा॒ना आ॑नशा॒नाः सुवः॒ सुव॑ रानशा॒नाः । \newline
35. आ॒न॒शा॒नाः स्वि॑ष्टिꣳ॒॒ स्वि॑ष्टि मानशा॒ना आ॑नशा॒नाः स्वि॑ष्टिम् । \newline
36. स्वि॑ष्टिम् नो नः॒ स्वि॑ष्टिꣳ॒॒ स्वि॑ष्टिम् नः । \newline
37. स्वि॑ष्टि॒मिति॒ सु - इ॒ष्टि॒म् । \newline
38. नो॒ अ॒भ्य॑भि नो॑ नो अ॒भि । \newline
39. अ॒भि वसी॑यो॒ वसी॑यो॒ ऽभ्य॑भि वसी॑यः । \newline
40. वसी॑यो नयन्तु नयन्तु॒ वसी॑यो॒ वसी॑यो नयन्तु । \newline
41. न॒य॒न्त्विति॑ नयन्तु । \newline
42. आ॒यत॑नवती॒र् वै वा आ॒यत॑नवती रा॒यत॑नवती॒र् वै । \newline
43. आ॒यत॑नवती॒रित्या॒यत॑न - व॒तीः॒ । \newline
44. वा अ॒न्या अ॒न्या वै वा अ॒न्याः । \newline
45. अ॒न्या आहु॑तय॒ आहु॑तयो॒ ऽन्या अ॒न्या आहु॑तयः । \newline
46. आहु॑तयो हू॒यन्ते॑ हू॒यन्त॒ आहु॑तय॒ आहु॑तयो हू॒यन्ते᳚ । \newline
47. आहु॑तय॒ इत्या - हु॒त॒यः॒ । \newline
48. हू॒यन्ते॑ ऽनायत॒ना अ॑नायत॒ना हू॒यन्ते॑ हू॒यन्ते॑ ऽनायत॒नाः । \newline
49. अ॒ना॒य॒त॒ना अ॒न्या अ॒न्या अ॑नायत॒ना अ॑नायत॒ना अ॒न्याः । \newline
50. अ॒ना॒य॒त॒ना इत्य॑ना - य॒त॒नाः । \newline
51. अ॒न्या या या अ॒न्या अ॒न्या याः । \newline
52. या आ॑घा॒रव॑ती राघा॒रव॑ती॒र् या या आ॑घा॒रव॑तीः । \newline
53. आ॒घा॒रव॑ती॒ स्ता स्ता आ॑घा॒रव॑ती राघा॒रव॑ती॒ स्ताः । \newline
54. आ॒घा॒रव॑ती॒रित्या॑घा॒र - व॒तीः॒ । \newline
55. ता आ॒यत॑नवती रा॒यत॑नवती॒ स्ता स्ता आ॒यत॑नवतीः । \newline
56. आ॒यत॑नवती॒र् या या आ॒यत॑नवती रा॒यत॑नवती॒र् याः । \newline
57. आ॒यत॑नवती॒रित्या॒यत॑न - व॒तीः॒ । \newline
58. याः सौ॒म्याः सौ॒म्या या याः सौ॒म्याः । \newline

\textbf{Ghana Paata } \newline

1. आयु॑ष्मन्तः स॒हभ॑क्षाः स॒हभ॑क्षा॒ आयु॑ष्मन्त॒ आयु॑ष्मन्तः स॒हभ॑क्षाः स्याम स्याम स॒हभ॑क्षा॒ आयु॑ष्मन्त॒ आयु॑ष्मन्तः स॒हभ॑क्षाः स्याम । \newline
2. स॒हभ॑क्षाः स्याम स्याम स॒हभ॑क्षाः स॒हभ॑क्षाः स्याम । \newline
3. स॒हभ॑क्षा॒ इति॑ स॒ह - भ॒क्षाः॒ । \newline
4. स्या॒मेति॑ स्याम । \newline
5. विश्वे॑ दे॒वा दे॒वा विश्वे॒ विश्वे॑ दे॒वा म॒रुतो॑ म॒रुतो॑ दे॒वा विश्वे॒ विश्वे॑ दे॒वा म॒रुतः॑ । \newline
6. दे॒वा म॒रुतो॑ म॒रुतो॑ दे॒वा दे॒वा म॒रुत॒ इन्द्र॒ इन्द्रो॑ म॒रुतो॑ दे॒वा दे॒वा म॒रुत॒ इन्द्रः॑ । \newline
7. म॒रुत॒ इन्द्र॒ इन्द्रो॑ म॒रुतो॑ म॒रुत॒ इन्द्रो॑ अ॒स्मा न॒स्मा निन्द्रो॑ म॒रुतो॑ म॒रुत॒ इन्द्रो॑ अ॒स्मान् । \newline
8. इन्द्रो॑ अ॒स्मा न॒स्मा निन्द्र॒ इन्द्रो॑ अ॒स्मा न॒स्मिन् न॒स्मिन् न॒स्मा निन्द्र॒ इन्द्रो॑ अ॒स्मा न॒स्मिन्न् । \newline
9. अ॒स्मा न॒स्मिन् न॒स्मिन् न॒स्मा न॒स्मा न॒स्मिन् द्वि॒तीये᳚ द्वि॒तीये॒ ऽस्मिन् न॒स्मा न॒स्मा न॒स्मिन् द्वि॒तीये᳚ । \newline
10. अ॒स्मिन् द्वि॒तीये᳚ द्वि॒तीये॒ ऽस्मिन् न॒स्मिन् द्वि॒तीये॒ सव॑ने॒ सव॑ने द्वि॒तीये॒ ऽस्मिन् न॒स्मिन् द्वि॒तीये॒ सव॑ने । \newline
11. द्वि॒तीये॒ सव॑ने॒ सव॑ने द्वि॒तीये᳚ द्वि॒तीये॒ सव॑ने॒ न न सव॑ने द्वि॒तीये᳚ द्वि॒तीये॒ सव॑ने॒ न । \newline
12. सव॑ने॒ न न सव॑ने॒ सव॑ने॒ न ज॑ह्युर् जह्यु॒र् न सव॑ने॒ सव॑ने॒ न ज॑ह्युः । \newline
13. न ज॑ह्युर् जह्यु॒र् न न ज॑ह्युः । \newline
14. ज॒ह्यु॒रिति॑ जह्युः । \newline
15. आयु॑ष्मन्तः प्रि॒यम् प्रि॒य मायु॑ष्मन्त॒ आयु॑ष्मन्तः प्रि॒य मे॑षा मेषाम् प्रि॒य मायु॑ष्मन्त॒ आयु॑ष्मन्तः प्रि॒य मे॑षाम् । \newline
16. प्रि॒य मे॑षा मेषाम् प्रि॒यम् प्रि॒य मे॑षां॒ ॅवद॑न्तो॒ वद॑न्त एषाम् प्रि॒यम् प्रि॒य मे॑षां॒ ॅवद॑न्तः । \newline
17. ए॒षां॒ ॅवद॑न्तो॒ वद॑न्त एषा मेषां॒ ॅवद॑न्तो व॒यं ॅव॒यं ॅवद॑न्त एषा मेषां॒ ॅवद॑न्तो व॒यम् । \newline
18. वद॑न्तो व॒यं ॅव॒यं ॅवद॑न्तो॒ वद॑न्तो व॒यम् दे॒वाना᳚म् दे॒वानां᳚ ॅव॒यं ॅवद॑न्तो॒ वद॑न्तो व॒यम् दे॒वाना᳚म् । \newline
19. व॒यम् दे॒वाना᳚म् दे॒वानां᳚ ॅव॒यं ॅव॒यम् दे॒वानाꣳ॑ सुम॒तौ सु॑म॒तौ दे॒वानां᳚ ॅव॒यं ॅव॒यम् दे॒वानाꣳ॑ सुम॒तौ । \newline
20. दे॒वानाꣳ॑ सुम॒तौ सु॑म॒तौ दे॒वाना᳚म् दे॒वानाꣳ॑ सुम॒तौ स्या॑म स्याम सुम॒तौ दे॒वाना᳚म् दे॒वानाꣳ॑ सुम॒तौ स्या॑म । \newline
21. सु॒म॒तौ स्या॑म स्याम सुम॒तौ सु॑म॒तौ स्या॑म । \newline
22. सु॒म॒ताविति॑ सु - म॒तौ । \newline
23. स्या॒मेति॑ स्याम । \newline
24. इ॒दम् तृ॒तीय॑म् तृ॒तीय॑ मि॒द मि॒दम् तृ॒तीयꣳ॒॒ सव॑नꣳ॒॒ सव॑नम् तृ॒तीय॑ मि॒द मि॒दम् तृ॒तीयꣳ॒॒ सव॑नम् । \newline
25. तृ॒तीयꣳ॒॒ सव॑नꣳ॒॒ सव॑नम् तृ॒तीय॑म् तृ॒तीयꣳ॒॒ सव॑नम् कवी॒नाम् क॑वी॒नाꣳ सव॑नम् तृ॒तीय॑म् तृ॒तीयꣳ॒॒ सव॑नम् कवी॒नाम् । \newline
26. सव॑नम् कवी॒नाम् क॑वी॒नाꣳ सव॑नꣳ॒॒ सव॑नम् कवी॒ना मृ॒तेन॒ र्तेन॑ कवी॒नाꣳ सव॑नꣳ॒॒ सव॑नम् कवी॒ना मृ॒तेन॑ । \newline
27. क॒वी॒ना मृ॒तेन॒ र्तेन॑ कवी॒नाम् क॑वी॒ना मृ॒तेन॒ ये य ऋ॒तेन॑ कवी॒नाम् क॑वी॒ना मृ॒तेन॒ ये । \newline
28. ऋ॒तेन॒ ये य ऋ॒तेन॒ र्तेन॒ ये च॑म॒सम् च॑म॒सं ॅय ऋ॒तेन॒ र्तेन॒ ये च॑म॒सम् । \newline
29. ये च॑म॒सम् च॑म॒सं ॅये ये च॑म॒स मैर॑य॒ न्तैर॑यन्त चम॒सं ॅये ये च॑म॒स मैर॑यन्त । \newline
30. च॒म॒स मैर॑य॒ न्तैर॑यन्त चम॒सम् च॑म॒स मैर॑यन्त । \newline
31. ऐर॑य॒न्तेत्यैर॑यन्त । \newline
32. ते सौ॑धन्व॒नाः सौ॑धन्व॒ना स्ते ते सौ॑धन्व॒नाः सुवः॒ सुवः॑ सौधन्व॒ना स्ते ते सौ॑धन्व॒नाः सुवः॑ । \newline
33. सौ॒ध॒न्व॒नाः सुवः॒ सुवः॑ सौधन्व॒नाः सौ॑धन्व॒नाः सुव॑ रानशा॒ना आ॑नशा॒नाः सुवः॑ सौधन्व॒नाः सौ॑धन्व॒नाः सुव॑ रानशा॒नाः । \newline
34. सुव॑ रानशा॒ना आ॑नशा॒नाः सुवः॒ सुव॑ रानशा॒नाः स्वि॑ष्टिꣳ॒॒ स्वि॑ष्टि मानशा॒नाः सुवः॒ सुव॑ रानशा॒नाः स्वि॑ष्टिम् । \newline
35. आ॒न॒शा॒नाः स्वि॑ष्टिꣳ॒॒ स्वि॑ष्टि मानशा॒ना आ॑नशा॒नाः स्वि॑ष्टिम् नो नः॒ स्वि॑ष्टि मानशा॒ना आ॑नशा॒नाः स्वि॑ष्टिम् नः । \newline
36. स्वि॑ष्टिम् नो नः॒ स्वि॑ष्टिꣳ॒॒ स्वि॑ष्टिम् नो अ॒भ्य॑भि नः॒ स्वि॑ष्टिꣳ॒॒ स्वि॑ष्टिम् नो अ॒भि । \newline
37. स्वि॑ष्टि॒मिति॒ सु - इ॒ष्टि॒म् । \newline
38. नो॒ अ॒भ्य॑भि नो॑ नो अ॒भि वसी॑यो॒ वसी॑यो॒ ऽभि नो॑ नो अ॒भि वसी॑यः । \newline
39. अ॒भि वसी॑यो॒ वसी॑यो॒ ऽभ्य॑भि वसी॑यो नयन्तु नयन्तु॒ वसी॑यो॒ ऽभ्य॑भि वसी॑यो नयन्तु । \newline
40. वसी॑यो नयन्तु नयन्तु॒ वसी॑यो॒ वसी॑यो नयन्तु । \newline
41. न॒य॒न्त्विति॑ नयन्तु । \newline
42. आ॒यत॑नवती॒र् वै वा आ॒यत॑नवती रा॒यत॑नवती॒र् वा अ॒न्या अ॒न्या वा आ॒यत॑नवती रा॒यत॑नवती॒र् वा अ॒न्याः । \newline
43. आ॒यत॑नवती॒रित्या॒यत॑न - व॒तीः॒ । \newline
44. वा अ॒न्या अ॒न्या वै वा अ॒न्या आहु॑तय॒ आहु॑तयो॒ ऽन्या वै वा अ॒न्या आहु॑तयः । \newline
45. अ॒न्या आहु॑तय॒ आहु॑तयो॒ ऽन्या अ॒न्या आहु॑तयो हू॒यन्ते॑ हू॒यन्त॒ आहु॑तयो॒ ऽन्या अ॒न्या आहु॑तयो हू॒यन्ते᳚ । \newline
46. आहु॑तयो हू॒यन्ते॑ हू॒यन्त॒ आहु॑तय॒ आहु॑तयो हू॒यन्ते॑ ऽनायत॒ना अ॑नायत॒ना हू॒यन्त॒ आहु॑तय॒ आहु॑तयो हू॒यन्ते॑ ऽनायत॒नाः । \newline
47. आहु॑तय॒ इत्या - हु॒त॒यः॒ । \newline
48. हू॒यन्ते॑ ऽनायत॒ना अ॑नायत॒ना हू॒यन्ते॑ हू॒यन्ते॑ ऽनायत॒ना अ॒न्या अ॒न्या अ॑नायत॒ना हू॒यन्ते॑ हू॒यन्ते॑ ऽनायत॒ना अ॒न्याः । \newline
49. अ॒ना॒य॒त॒ना अ॒न्या अ॒न्या अ॑नायत॒ना अ॑नायत॒ना अ॒न्या या या अ॒न्या अ॑नायत॒ना अ॑नायत॒ना अ॒न्या याः । \newline
50. अ॒ना॒य॒त॒ना इत्य॑ना - य॒त॒नाः । \newline
51. अ॒न्या या या अ॒न्या अ॒न्या या आ॑घा॒रव॑ती राघा॒रव॑ती॒र् या अ॒न्या अ॒न्या या आ॑घा॒रव॑तीः । \newline
52. या आ॑घा॒रव॑ती राघा॒रव॑ती॒र् या या आ॑घा॒रव॑ती॒ स्ता स्ता आ॑घा॒रव॑ती॒र् या या आ॑घा॒रव॑ती॒ स्ताः । \newline
53. आ॒घा॒रव॑ती॒ स्ता स्ता आ॑घा॒रव॑ती राघा॒रव॑ती॒ स्ता आ॒यत॑नवती रा॒यत॑नवती॒ स्ता आ॑घा॒रव॑ती राघा॒रव॑ती॒ स्ता आ॒यत॑नवतीः । \newline
54. आ॒घा॒रव॑ती॒रित्या॑घा॒र - व॒तीः॒ । \newline
55. ता आ॒यत॑नवती रा॒यत॑नवती॒ स्ता स्ता आ॒यत॑नवती॒र् या या आ॒यत॑नवती॒ स्ता स्ता आ॒यत॑नवती॒र् याः । \newline
56. आ॒यत॑नवती॒र् या या आ॒यत॑नवती रा॒यत॑नवती॒र् याः सौ॒म्याः सौ॒म्या या आ॒यत॑नवती रा॒यत॑नवती॒र् याः सौ॒म्याः । \newline
57. आ॒यत॑नवती॒रित्या॒यत॑न - व॒तीः॒ । \newline
58. याः सौ॒म्याः सौ॒म्या या याः सौ॒म्या स्ता स्ताः सौ॒म्या या याः सौ॒म्या स्ताः । \newline
\pagebreak
\markright{ TS 3.1.9.3  \hfill https://www.vedavms.in \hfill}

\section{ TS 3.1.9.3 }

\textbf{TS 3.1.9.3 } \newline
\textbf{Samhita Paata} \newline

सौ॒म्यास्ता अ॑नायत॒ना ऐ᳚न्द्रवाय॒व-मा॒दाया॑ऽऽ*घा॒रमा घा॑रयेदद्ध्व॒रो य॒ज्ञो॑ऽयम॑स्तु देवा॒ ओष॑धीभ्यः प॒शवे॑ नो॒ जना॑य॒ विश्व॑स्मै भू॒ताया᳚ऽद्ध्व॒रो॑ऽसि॒ स पि॑न्वस्व घृ॒तव॑द्देव सो॒मेति॑ सौ॒म्या ए॒व तदाहु॑तीरा॒यत॑नवतीः करोत्या॒यत॑नवान् भवति॒ य ए॒वं ॅवेदाथो॒ द्यावा॑पृथि॒वी ए॒व घृ॒तेन॒ व्यु॑नत्ति॒ ते व्यु॑त्ते उपजीव॒नीये॑ भवत उपजीव॒नीयो॑ भवति॒ - [  ] \newline

\textbf{Pada Paata} \newline

सौ॒म्याः । ताः । अ॒ना॒य॒त॒ना इत्य॑ना - य॒त॒नाः । ऐ॒न्द्र॒वा॒य॒वमित्यै᳚न्द्र - वा॒य॒वम् । आ॒दायेत्या᳚ - दाय॑ । आ॒घा॒रमित्या᳚ - घा॒रम् । एति॑ । घा॒र॒ये॒त् । अ॒द्ध्व॒रः । य॒ज्ञ्ः । अ॒यम् । अ॒स्तु॒ । दे॒वाः॒ । ओष॑धीभ्य॒ इत्योष॑धि-भ्यः॒ । प॒शवे᳚ । नः॒ । जना॑य । विश्व॑स्मै । भू॒ताय॑ । अ॒ध्व॒रः । अ॒सि॒ । सः । पि॒न्व॒स्व॒ । घृ॒तव॒दिति॑ घृ॒त - व॒त् । दे॒व॒ । सो॒म॒ । इति॑ । सौ॒म्याः । ए॒व । तत् । आहु॑ती॒रित्या - हु॒तीः॒ । आ॒यत॑नवती॒रित्या॒यत॑न - व॒तीः॒ । क॒रो॒ति॒ । आ॒यत॑नवा॒नित्या॒यत॑न - वा॒न् । भ॒व॒ति॒ । यः । ए॒वम् । वेद॑ । अथो॒ इति॑ । द्यावा॑पृथि॒वी इति॒ द्यावा᳚-पृ॒थि॒वी । ए॒व । घृ॒तेन॑ । वीति॑ । उ॒न॒त्ति॒ । ते इति॑ । व्यु॑त्ते॒ इति॒ वि - उ॒त्ते॒ । उ॒प॒जी॒व॒नीये॒ इत्यु॑प - जी॒व॒नीये᳚ । भ॒व॒तः॒ । उ॒प॒जी॒व॒नीय॒ इत्यु॑प - जी॒व॒नीयः॑ । भ॒व॒ति॒ ।  \newline


\textbf{Krama Paata} \newline

सौ॒म्यास्ताः । ता अ॑नायत॒नाः । अ॒ना॒य॒त॒ना ऐ᳚न्द्रवाय॒वम् । अ॒ना॒य॒त॒ना इत्य॑ना - य॒त॒नाः । ऐ॒न्द्र॒वा॒य॒वमा॒दाय॑ । ऐ॒न्द्र॒वा॒य॒वमित्यै᳚न्द्र - वा॒य॒वम् । आ॒दाया॑घा॒रम् । आ॒दायेत्या᳚ - दाय॑ । आ॒घा॒रमा । आ॒घा॒रमित्या᳚ - घा॒रम् । आ घा॑रयेत् । घा॒र॒ये॒द॒द्ध्व॒रः । अ॒द्ध्व॒रो य॒ज्ञ्ः । य॒ज्ञो॑ ऽयम् । अ॒यम॑स्तु । अ॒स्तु॒ दे॒वाः॒ । दे॒वा॒ ओष॑धीभ्यः । ओष॑धीभ्यः प॒शवे᳚ । ओष॑धीभ्य॒ इत्योष॑धि - भ्यः॒ । प॒शवे॑ नः । नो॒ जना॑य । जना॑य॒ विश्व॑स्मै । विश्व॑स्मै भू॒ताय॑ । भू॒ताया᳚द्ध्व॒रः । अ॒द्ध्व॒रो॑ ऽसि । अ॒सि॒ सः । स पि॑न्वस्व । पि॒न्व॒स्व॒ घृ॒तव॑त् । घृ॒तव॑द् देव । घृ॒तव॒दिति॑ घृ॒त - व॒त्॒ । दे॒व॒ सो॒म॒ । सो॒मेति॑ । इति॑ सौ॒म्याः । सौ॒म्या ए॒व । ए॒व तत् । तदाहु॑तीः । आहु॑तीरा॒यत॑नवतीः । आहु॑ती॒रित्या - हु॒तीः॒ । आ॒यत॑नवतीः करोति । आ॒यत॑नवती॒रित्या॒यत॑न - व॒तीः॒ । क॒रो॒त्या॒यत॑नवान् । आ॒यत॑नवान् भवति । आ॒यत॑नवा॒नित्या॒यत॑न - वा॒न्॒ । भ॒व॒ति॒ यः । य ए॒वम् । ए॒वं ॅवेद॑ । वेदाथो᳚ । अथो॒ द्यावा॑पृथि॒वी । अथो॒ इत्यथो᳚ । द्यावा॑पृथि॒वी ए॒व । द्यावा॑पृथि॒वी इति॒ द्यावा᳚ - पृ॒थि॒वी । ए॒व घृ॒तेन॑ । घृ॒तेन॒ वि । व्यु॑नत्ति । उ॒न॒त्ति॒ ते । ते व्यु॑त्ते । ते इति॒ ते । व्यु॑त्ते उपजीव॒नीये᳚ । व्यु॑त्ते॒ इति॒ वि - उ॒त्ते॒ । उ॒प॒जी॒व॒नीये॑ भवतः । उ॒प॒जी॒व॒नीये॒ इत्यु॑प - जी॒व॒नीये᳚ । भ॒व॒त॒ उ॒प॒जी॒व॒नीयः॑ । उ॒प॒जी॒व॒नीयो॑ भवति । उ॒प॒जी॒व॒नीय॒ इत्यु॑प - जी॒व॒नीयः॑ । भ॒व॒ति॒ यः \newline

\textbf{Jatai Paata} \newline

1. सौ॒म्या स्ता स्ताः सौ॒म्याः सौ॒म्या स्ताः । \newline
2. ता अ॑नायत॒ना अ॑नायत॒ना स्ता स्ता अ॑नायत॒नाः । \newline
3. अ॒ना॒य॒त॒ना ऐ᳚न्द्रवाय॒व मै᳚न्द्रवाय॒व म॑नायत॒ना अ॑नायत॒ना ऐ᳚न्द्रवाय॒वम् । \newline
4. अ॒ना॒य॒त॒ना इत्य॑ना - य॒त॒नाः । \newline
5. ऐ॒न्द्र॒वा॒य॒व मा॒दाया॒दा यै᳚न्द्रवाय॒व मै᳚न्द्रवाय॒व मा॒दाय॑ । \newline
6. ऐ॒न्द्र॒वा॒य॒वमित्यै᳚न्द्र - वा॒य॒वम् । \newline
7. आ॒दाया॑घा॒र मा॑घा॒र मा॒दाया॒ दाया॑ घा॒रम् । \newline
8. आ॒दायेत्या᳚ - दाय॑ । \newline
9. आ॒घा॒र मा ऽऽघा॒र मा॑घा॒र मा । \newline
10. आ॒घा॒रमित्या᳚ - घा॒रम् । \newline
11. आ घा॑रयेद् घारये॒दा घा॑रयेत् । \newline
12. घा॒र॒ये॒ द॒द्ध्व॒रो अ॑द्ध्व॒रो घा॑रयेद् घारये दद्ध्व॒रः । \newline
13. अ॒द्ध्व॒रो य॒ज्ञो य॒ज्ञो अ॑द्ध्व॒रो अ॑द्ध्व॒रो य॒ज्ञ्ः । \newline
14. य॒ज्ञो॑ ऽय म॒यं ॅय॒ज्ञो य॒ज्ञो॑ ऽयम् । \newline
15. अ॒य म॑स्त्व स्त्व॒य म॒य म॑स्तु । \newline
16. अ॒स्तु॒ दे॒वा॒ दे॒वा॒ अ॒स्त्व॒ स्तु॒ दे॒वाः॒ । \newline
17. दे॒वा॒ ओष॑धीभ्य॒ ओष॑धीभ्यो देवा देवा॒ ओष॑धीभ्यः । \newline
18. ओष॑धीभ्यः प॒शवे॑ प॒शव॒ ओष॑धीभ्य॒ ओष॑धीभ्यः प॒शवे᳚ । \newline
19. ओष॑धीभ्य॒ इत्योष॑धि - भ्यः॒ । \newline
20. प॒शवे॑ नो नः प॒शवे॑ प॒शवे॑ नः । \newline
21. नो॒ जना॑य॒ जना॑य नो नो॒ जना॑य । \newline
22. जना॑य॒ विश्व॑स्मै॒ विश्व॑स्मै॒ जना॑य॒ जना॑य॒ विश्व॑स्मै । \newline
23. विश्व॑स्मै भू॒ताय॑ भू॒ताय॒ विश्व॑स्मै॒ विश्व॑स्मै भू॒ताय॑ । \newline
24. भू॒ताया᳚ द्ध्व॒रो अ॑द्ध्व॒रो भू॒ताय॑ भू॒ताया᳚ द्ध्व॒रः । \newline
25. अ॒द्ध्व॒रो᳚ ऽस्य स्यद्ध्व॒रो अ॑द्ध्व॒रो॑ ऽसि । \newline
26. अ॒सि॒ स सो᳚ ऽस्यसि॒ सः । \newline
27. स पि॑न्वस्व पिन्वस्व॒ स स पि॑न्वस्व । \newline
28. पि॒न्व॒स्व॒ घृ॒तव॑द् घृ॒तव॑त् पिन्वस्व पिन्वस्व घृ॒तव॑त् । \newline
29. घृ॒तव॑द् देव देव घृ॒तव॑द् घृ॒तव॑द् देव । \newline
30. घृ॒तव॒दिति॑ घृ॒त - व॒त् । \newline
31. दे॒व॒ सो॒म॒ सो॒म॒ दे॒व॒ दे॒व॒ सो॒म॒ । \newline
32. सो॒मे तीति॑ सोम सो॒मे ति॑ । \newline
33. इति॑ सौ॒म्याः सौ॒म्या इतीति॑ सौ॒म्याः । \newline
34. सौ॒म्या ए॒वैव सौ॒म्याः सौ॒म्या ए॒व । \newline
35. ए॒व तत् तदे॒वैव तत् । \newline
36. तदाहु॑ती॒ राहु॑ती॒ स्तत् तदाहु॑तीः । \newline
37. आहु॑ती रा॒यत॑नवती रा॒यत॑नवती॒ राहु॑ती॒ राहु॑ती रा॒यत॑नवतीः । \newline
38. आहु॑ती॒रित्या - हु॒तीः॒ । \newline
39. आ॒यत॑नवतीः करोति करो त्या॒यत॑नवती रा॒यत॑नवतीः करोति । \newline
40. आ॒यत॑नवती॒रित्या॒यत॑न - व॒तीः॒ । \newline
41. क॒रो॒ त्या॒यत॑नवा ना॒यत॑नवान् करोति करो त्या॒यत॑नवान् । \newline
42. आ॒यत॑नवान् भवति भव त्या॒यत॑नवा ना॒यत॑नवान् भवति । \newline
43. आ॒यत॑नवा॒नित्या॒यत॑न - वा॒न् । \newline
44. भ॒व॒ति॒ यो यो भ॑वति भवति॒ यः । \newline
45. य ए॒व मे॒वं ॅयो य ए॒वम् । \newline
46. ए॒वं ॅवेद॒ वेदै॒व मे॒वं ॅवेद॑ । \newline
47. वेदाथो॒ अथो॒ वेद॒ वेदाथो᳚ । \newline
48. अथो॒ द्यावा॑पृथि॒वी द्यावा॑पृथि॒वी अथो॒ अथो॒ द्यावा॑पृथि॒वी । \newline
49. अथो॒ इत्यथो᳚ । \newline
50. द्यावा॑पृथि॒वी ए॒वैव द्यावा॑पृथि॒वी द्यावा॑पृथि॒वी ए॒व । \newline
51. द्यावा॑पृथि॒वी इति॒ द्यावा᳚ - पृ॒थि॒वी । \newline
52. ए॒व घृ॒तेन॑ घृ॒ते नै॒वैव घृ॒तेन॑ । \newline
53. घृ॒तेन॒ वि वि घृ॒तेन॑ घृ॒तेन॒ वि । \newline
54. व्यु॑न त्त्युनत्ति॒ वि व्यु॑नत्ति । \newline
55. उ॒न॒त्ति॒ ते ते उ॑न त्त्युनत्ति॒ ते । \newline
56. ते व्यु॑त्ते॒ व्यु॑त्ते॒ ते ते व्यु॑त्ते । \newline
57. ते इति॒ ते । \newline
58. व्यु॑त्ते उपजीव॒नीये॑ उपजीव॒नीये॒ व्यु॑त्ते॒ व्यु॑त्ते उपजीव॒नीये᳚ । \newline
59. व्यु॑त्ते॒ इति॒ वि - उ॒त्ते॒ । \newline
60. उ॒प॒जी॒व॒नीये॑ भवतो भवत उपजीव॒नीये॑ उपजीव॒नीये॑ भवतः । \newline
61. उ॒प॒जी॒व॒नीये॒ इत्यु॑प - जी॒व॒नीये᳚ । \newline
62. भ॒व॒त॒ उ॒प॒जी॒व॒नीय॑ उपजीव॒नीयो॑ भवतो भवत उपजीव॒नीयः॑ । \newline
63. उ॒प॒जी॒व॒नीयो॑ भवति भव त्युपजीव॒नीय॑ उपजीव॒नीयो॑ भवति । \newline
64. उ॒प॒जी॒व॒नीय॒ इत्यु॑प - जी॒व॒नीयः॑ । \newline
65. भ॒व॒ति॒ यो यो भ॑वति भवति॒ यः । \newline

\textbf{Ghana Paata } \newline

1. सौ॒म्या स्ता स्ताः सौ॒म्याः सौ॒म्या स्ता अ॑नायत॒ना अ॑नायत॒ना स्ताः सौ॒म्याः सौ॒म्या स्ता अ॑नायत॒नाः । \newline
2. ता अ॑नायत॒ना अ॑नायत॒ना स्ता स्ता अ॑नायत॒ना ऐ᳚न्द्रवाय॒व मै᳚न्द्रवाय॒व म॑नायत॒ना स्ता स्ता अ॑नायत॒ना ऐ᳚न्द्रवाय॒वम् । \newline
3. अ॒ना॒य॒त॒ना ऐ᳚न्द्रवाय॒व मै᳚न्द्रवाय॒व म॑नायत॒ना अ॑नायत॒ना ऐ᳚न्द्रवाय॒व मा॒दाया॒दा यै᳚न्द्रवाय॒व म॑नायत॒ना अ॑नायत॒ना ऐ᳚न्द्रवाय॒व मा॒दाय॑ । \newline
4. अ॒ना॒य॒त॒ना इत्य॑ना - य॒त॒नाः । \newline
5. ऐ॒न्द्र॒वा॒य॒व मा॒दाया॒दा यै᳚न्द्रवाय॒व मै᳚न्द्रवाय॒व मा॒दाया॑घा॒र मा॑घा॒र मा॒दा यै᳚न्द्रवाय॒व मै᳚न्द्रवाय॒व मा॒दाया॑घा॒रम् । \newline
6. ऐ॒न्द्र॒वा॒य॒वमित्यै᳚न्द्र - वा॒य॒वम् । \newline
7. आ॒दाया॑ घा॒र मा॑घा॒र मा॒दाया॒ दाया॑ घा॒र मा ऽऽघा॒र मा॒दाया॒ दाया॑ घा॒र मा । \newline
8. आ॒दायेत्या᳚ - दाय॑ । \newline
9. आ॒घा॒र मा ऽऽघा॒र मा॑घा॒र मा घा॑रयेद् घारये॒ दा ऽऽघा॒र मा॑घा॒र मा घा॑रयेत् । \newline
10. आ॒घा॒रमित्या᳚ - घा॒रम् । \newline
11. आ घा॑रयेद् घारये॒ दाघा॑रये दद्ध्व॒रो अ॑द्ध्व॒रो घा॑रये॒ दाघा॑रये दद्ध्व॒रः । \newline
12. घा॒र॒ये॒ द॒द्ध्व॒रो अ॑द्ध्व॒रो घा॑रयेद् घारये दद्ध्व॒रो य॒ज्ञो य॒ज्ञो अ॑द्ध्व॒रो घा॑रयेद् घारये दद्ध्व॒रो य॒ज्ञ्ः । \newline
13. अ॒द्ध्व॒रो य॒ज्ञो य॒ज्ञो अ॑द्ध्व॒रो अ॑द्ध्व॒रो य॒ज्ञो॑ ऽय म॒यं ॅय॒ज्ञो अ॑द्ध्व॒रो अ॑द्ध्व॒रो य॒ज्ञो॑ ऽयम् । \newline
14. य॒ज्ञो॑ ऽय म॒यं ॅय॒ज्ञो य॒ज्ञो॑ ऽय म॑स्त्व स्त्व॒यं ॅय॒ज्ञो य॒ज्ञो॑ ऽय म॑स्तु । \newline
15. अ॒य म॑स्त्व स्त्व॒य म॒य म॑स्तु देवा देवा अस्त्व॒य म॒य म॑स्तु देवाः । \newline
16. अ॒स्तु॒ दे॒वा॒ दे॒वा॒ अ॒स्त्व॒ स्तु॒ दे॒वा॒ ओष॑धीभ्य॒ ओष॑धीभ्यो देवा अस्त्व स्तु देवा॒ ओष॑धीभ्यः । \newline
17. दे॒वा॒ ओष॑धीभ्य॒ ओष॑धीभ्यो देवा देवा॒ ओष॑धीभ्यः प॒शवे॑ प॒शव॒ ओष॑धीभ्यो देवा देवा॒ ओष॑धीभ्यः प॒शवे᳚ । \newline
18. ओष॑धीभ्यः प॒शवे॑ प॒शव॒ ओष॑धीभ्य॒ ओष॑धीभ्यः प॒शवे॑ नो नः प॒शव॒ ओष॑धीभ्य॒ ओष॑धीभ्यः प॒शवे॑ नः । \newline
19. ओष॑धीभ्य॒ इत्योष॑धि - भ्यः॒ । \newline
20. प॒शवे॑ नो नः प॒शवे॑ प॒शवे॑ नो॒ जना॑य॒ जना॑य नः प॒शवे॑ प॒शवे॑ नो॒ जना॑य । \newline
21. नो॒ जना॑य॒ जना॑य नो नो॒ जना॑य॒ विश्व॑स्मै॒ विश्व॑स्मै॒ जना॑य नो नो॒ जना॑य॒ विश्व॑स्मै । \newline
22. जना॑य॒ विश्व॑स्मै॒ विश्व॑स्मै॒ जना॑य॒ जना॑य॒ विश्व॑स्मै भू॒ताय॑ भू॒ताय॒ विश्व॑स्मै॒ जना॑य॒ जना॑य॒ विश्व॑स्मै भू॒ताय॑ । \newline
23. विश्व॑स्मै भू॒ताय॑ भू॒ताय॒ विश्व॑स्मै॒ विश्व॑स्मै भू॒ताया᳚द्ध्व॒रो अ॑द्ध्व॒रो भू॒ताय॒ विश्व॑स्मै॒ विश्व॑स्मै भू॒ताया᳚द्ध्व॒रः । \newline
24. भू॒ताया᳚द्ध्व॒रो अ॑द्ध्व॒रो भू॒ताय॑ भू॒ताया᳚द्ध्व॒रो᳚ ऽस्य स्यद्ध्व॒रो भू॒ताय॑ भू॒ताया᳚द्ध्व॒रो॑ ऽसि । \newline
25. अ॒द्ध्व॒रो᳚ ऽस्यस्यद्ध्व॒रो अ॑द्ध्व॒रो॑ ऽसि॒ स सो᳚ ऽस्यद्ध्व॒रो अ॑द्ध्व॒रो॑ ऽसि॒ सः । \newline
26. अ॒सि॒ स सो᳚ ऽस्यसि॒ स पि॑न्वस्व पिन्वस्व॒ सो᳚ ऽस्यसि॒ स पि॑न्वस्व । \newline
27. स पि॑न्वस्व पिन्वस्व॒ स स पि॑न्वस्व घृ॒तव॑द् घृ॒तव॑त् पिन्वस्व॒ स स पि॑न्वस्व घृ॒तव॑त् । \newline
28. पि॒न्व॒स्व॒ घृ॒तव॑द् घृ॒तव॑त् पिन्वस्व पिन्वस्व घृ॒तव॑द् देव देव घृ॒तव॑त् पिन्वस्व पिन्वस्व घृ॒तव॑द् देव । \newline
29. घृ॒तव॑द् देव देव घृ॒तव॑द् घृ॒तव॑द् देव सोम सोम देव घृ॒तव॑द् घृ॒तव॑द् देव सोम । \newline
30. घृ॒तव॒दिति॑ घृ॒त - व॒त् । \newline
31. दे॒व॒ सो॒म॒ सो॒म॒ दे॒व॒ दे॒व॒ सो॒मे तीति॑ सोम देव देव सो॒मे ति॑ । \newline
32. सो॒मे तीति॑ सोम सो॒मे ति॑ सौ॒म्याः सौ॒म्या इति॑ सोम सो॒मे ति॑ सौ॒म्याः । \newline
33. इति॑ सौ॒म्याः सौ॒म्या इतीति॑ सौ॒म्या ए॒वैव सौ॒म्या इतीति॑ सौ॒म्या ए॒व । \newline
34. सौ॒म्या ए॒वैव सौ॒म्याः सौ॒म्या ए॒व तत् तदे॒व सौ॒म्याः सौ॒म्या ए॒व तत् । \newline
35. ए॒व तत् तदे॒वैव तदाहु॑ती॒ राहु॑ती॒ स्त दे॒वैव तदाहु॑तीः । \newline
36. तदाहु॑ती॒ राहु॑ती॒ स्तत् तदाहु॑ती रा॒यत॑नवती रा॒यत॑नवती॒ राहु॑ती॒ स्तत् तदाहु॑ती रा॒यत॑नवतीः । \newline
37. आहु॑ती रा॒यत॑नवती रा॒यत॑नवती॒ राहु॑ती॒ राहु॑ती रा॒यत॑नवतीः करोति करो त्या॒यत॑नवती॒ राहु॑ती॒ राहु॑ती रा॒यत॑नवतीः करोति । \newline
38. आहु॑ती॒रित्या - हु॒तीः॒ । \newline
39. आ॒यत॑नवतीः करोति करो त्या॒यत॑नवती रा॒यत॑नवतीः करो त्या॒यत॑नवा ना॒यत॑नवान् करो त्या॒यत॑नवती रा॒यत॑नवतीः करो त्या॒यत॑नवान् । \newline
40. आ॒यत॑नवती॒रित्या॒यत॑न - व॒तीः॒ । \newline
41. क॒रो॒ त्या॒यत॑नवा ना॒यत॑नवान् करोति करो त्या॒यत॑नवान् भवति भव त्या॒यत॑नवान् करोति करो त्या॒यत॑नवान् भवति । \newline
42. आ॒यत॑नवान् भवति भव त्या॒यत॑नवा ना॒यत॑नवान् भवति॒ यो यो भ॑व त्या॒यत॑नवा ना॒यत॑नवान् भवति॒ यः । \newline
43. आ॒यत॑नवा॒नित्या॒यत॑न - वा॒न् । \newline
44. भ॒व॒ति॒ यो यो भ॑वति भवति॒ य ए॒व मे॒वं ॅयो भ॑वति भवति॒ य ए॒वम् । \newline
45. य ए॒व मे॒वं ॅयो य ए॒वं ॅवेद॒ वेदै॒वं ॅयो य ए॒वं ॅवेद॑ । \newline
46. ए॒वं ॅवेद॒ वेदै॒व मे॒वं ॅवेदाथो॒ अथो॒ वेदै॒व मे॒वं ॅवेदाथो᳚ । \newline
47. वेदाथो॒ अथो॒ वेद॒ वेदाथो॒ द्यावा॑पृथि॒वी द्यावा॑पृथि॒वी अथो॒ वेद॒ वेदाथो॒ द्यावा॑पृथि॒वी । \newline
48. अथो॒ द्यावा॑पृथि॒वी द्यावा॑पृथि॒वी अथो॒ अथो॒ द्यावा॑पृथि॒वी ए॒वैव द्यावा॑पृथि॒वी अथो॒ अथो॒ द्यावा॑पृथि॒वी ए॒व । \newline
49. अथो॒ इत्यथो᳚ । \newline
50. द्यावा॑पृथि॒वी ए॒वैव द्यावा॑पृथि॒वी द्यावा॑पृथि॒वी ए॒व घृ॒तेन॑ घृ॒तेनै॒व द्यावा॑पृथि॒वी द्यावा॑पृथि॒वी ए॒व घृ॒तेन॑ । \newline
51. द्यावा॑पृथि॒वी इति॒ द्यावा᳚ - पृ॒थि॒वी । \newline
52. ए॒व घृ॒तेन॑ घृ॒तेनै॒वैव घृ॒तेन॒ वि वि घृ॒तेनै॒वैव घृ॒तेन॒ वि । \newline
53. घृ॒तेन॒ वि वि घृ॒तेन॑ घृ॒तेन॒ व्यु॑न त्त्युनत्ति॒ वि घृ॒तेन॑ घृ॒तेन॒ व्यु॑नत्ति । \newline
54. व्यु॑न त्त्युनत्ति॒ वि व्यु॑नत्ति॒ ते ते उ॑नत्ति॒ वि व्यु॑नत्ति॒ ते । \newline
55. उ॒न॒त्ति॒ ते ते उ॑न त्त्युनत्ति॒ ते व्यु॑त्ते॒ व्यु॑त्ते॒ ते उ॑न त्त्युनत्ति॒ ते व्यु॑त्ते । \newline
56. ते व्यु॑त्ते॒ व्यु॑त्ते॒ ते ते व्यु॑त्ते उपजीव॒नीये॑ उपजीव॒नीये॒ व्यु॑त्ते॒ ते ते व्यु॑त्ते उपजीव॒नीये᳚ । \newline
57. ते इति॒ ते । \newline
58. व्यु॑त्ते उपजीव॒नीये॑ उपजीव॒नीये॒ व्यु॑त्ते॒ व्यु॑त्ते उपजीव॒नीये॑ भवतो भवत उपजीव॒नीये॒ व्यु॑त्ते॒ व्यु॑त्ते उपजीव॒नीये॑ भवतः । \newline
59. व्यु॑त्ते॒ इति॒ वि - उ॒त्ते॒ । \newline
60. उ॒प॒जी॒व॒नीये॑ भवतो भवत उपजीव॒नीये॑ उपजीव॒नीये॑ भवत उपजीव॒नीय॑ उपजीव॒नीयो॑ भवत उपजीव॒नीये॑ उपजीव॒नीये॑ भवत उपजीव॒नीयः॑ । \newline
61. उ॒प॒जी॒व॒नीये॒ इत्यु॑प - जी॒व॒नीये᳚ । \newline
62. भ॒व॒त॒ उ॒प॒जी॒व॒नीय॑ उपजीव॒नीयो॑ भवतो भवत उपजीव॒नीयो॑ भवति भव त्युपजीव॒नीयो॑ भवतो भवत उपजीव॒नीयो॑ भवति । \newline
63. उ॒प॒जी॒व॒नीयो॑ भवति भव त्युपजीव॒नीय॑ उपजीव॒नीयो॑ भवति॒ यो यो भ॑व त्युपजीव॒नीय॑ उपजीव॒नीयो॑ भवति॒ यः । \newline
64. उ॒प॒जी॒व॒नीय॒ इत्यु॑प - जी॒व॒नीयः॑ । \newline
65. भ॒व॒ति॒ यो यो भ॑वति भवति॒ य ए॒व मे॒वं ॅयो भ॑वति भवति॒ य ए॒वम् । \newline
\pagebreak
\markright{ TS 3.1.9.4  \hfill https://www.vedavms.in \hfill}

\section{ TS 3.1.9.4 }

\textbf{TS 3.1.9.4 } \newline
\textbf{Samhita Paata} \newline

य ए॒वं ॅवेदै॒ष ते॑ रुद्रभा॒गो यं नि॒रया॑चथा॒स्तं जु॑षस्व वि॒देर्गौ॑प॒त्यꣳ रा॒यस्पोषꣳ॑ सु॒वीर्यꣳ॑ संॅवथ्स॒रीणाꣳ॑ स्व॒स्तिं ॥ मनुः॑ पु॒त्रेभ्यो॑ दा॒यं ॅव्य॑भज॒थ् स नाभा॒नेदि॑ष्ठं ब्रह्म॒चर्यं॒ ॅवस॑न्तं॒ निर॑भज॒थ् स आऽग॑च्छ॒थ् सो᳚ऽब्रवीत् क॒था मा॒ निर॑भा॒गिति॒ न त्वा॒ निर॑भाक्ष॒मित्य॑-ब्रवी॒दङ्गि॑रस इ॒मे स॒त्रमा॑सते॒ ते - [  ] \newline

\textbf{Pada Paata} \newline

यः । ए॒वम् । वेद॑ । ए॒षः । ते॒ । रु॒द्र॒ । भा॒गः । यम् । नि॒रया॑चथा॒ इति॑ निः-अया॑चथाः । तम् । जु॒ष॒स्व॒ । वि॒देः । गौ॒प॒त्यम् । रा॒यः । पोष᳚म् । सु॒वीर्य॒मिति॑ सु - वीर्य᳚म् । सं॒ॅव॒थ्स॒रीणा॒मिति॑ सं - व॒थ्स॒रीणा᳚म् । स्व॒स्तिम् ॥ मनुः॑ । पु॒त्रेभ्यः॑ । दा॒यम् । वीति॑ । अ॒भ॒ज॒त् । सः । नाभा॒नेदि॑ष्ठम् । ब्र॒ह्म॒चर्य॒मिति॑ ब्रह्म-चर्य᳚म् । वस॑न्तम् । निरिति॑ । अ॒भ॒ज॒त् । सः । एति॑ । अ॒ग॒च्छ॒त् । सः । अ॒ब्र॒वी॒त् । क॒था । मा॒ । निरिति॑ । अ॒भा॒क् । इति॑ । न । त्वा॒ । निरिति॑ । अ॒भा॒क्ष॒म् । इति॑ । अ॒ब्र॒वी॒त् । अङ्गि॑रसः । इ॒मे । स॒त्रम् । आ॒स॒ते॒ । ते ।  \newline


\textbf{Krama Paata} \newline

य ए॒वम् । ए॒वं ॅवेद॑ । वेदै॒षः । ए॒ष ते᳚ । ते॒ रु॒द्र॒ । रु॒द्र॒ भा॒गः । भा॒गो यम् । यम् नि॒रया॑चथाः । नि॒रया॑चथा॒स्तम् । नि॒रया॑चथा॒ इति॑ निः - अया॑चथाः । तम् जु॑षस्व । जु॒ष्व॒स्व॒ वि॒देः । वि॒देर् गौ॑प॒त्यम् । गौ॒प॒त्यꣳ रा॒यः । रा॒यस्पोष᳚म् । पोषꣳ॑ सु॒वीर्य᳚म् । सु॒वीर्यꣳ॑ सम्ॅवथ्स॒रीणा᳚म् । सु॒वीर्य॒मिति॑ सु - वीर्य᳚म् । स॒म्ॅव॒थ्स॒रीणाꣳ॑ स्व॒स्तिम् । स॒म्ॅव॒थ्स॒रीणा॒मिति॑ सं - व॒थ्स॒रीणा᳚म् । स्व॒स्तिमिति॑ स्व॒स्तिम् ॥ मनुः॑ पु॒त्रेभ्यः॑ । पु॒त्रेभ्यो॑ दा॒यम् । दा॒यं ॅवि । व्य॑भजत् । अ॒भ॒ज॒थ् सः । स नाभा॒नेदि॑ष्ठम् । नाभा॒नेदि॑ष्ठम् ब्रह्म॒चर्य᳚म् । ब्र॒ह्म॒चर्यं॒ ॅवस॑न्तम् । ब्र॒ह्म॒चर्य॒मिति॑ ब्रह्म - चर्य᳚म् । वस॑न्त॒म् निः । निर॑भजत् । अ॒भ॒ज॒थ् सः । स आ । आऽग॑च्छत् । अ॒ग॒च्छ॒थ् सः । सो᳚ऽब्रवीत् । अ॒ब्र॒वी॒त् क॒था । क॒था मा᳚ । मा॒ निः । निर॑भाक् । अ॒भा॒गिति॑ । इति॒ न । न त्वा᳚ । त्वा॒ निः । निर॑भाक्षम् । अ॒भा॒क्ष॒मिति॑ । इत्य॑ब्रवीत् । अ॒ब्र॒वी॒दङ्गि॑रसः । अङ्गि॑रस इ॒मे । इ॒मे स॒त्रम् । स॒त्रमा॑सते । आ॒स॒ते॒ ते । ते सु॑व॒र्गम् \newline

\textbf{Jatai Paata} \newline

1. य ए॒व मे॒वं ॅयो य ए॒वम् । \newline
2. ए॒वं ॅवेद॒ वेदै॒व मे॒वं ॅवेद॑ । \newline
3. वेदै॒ष ए॒ष वेद॒ वेदै॒षः । \newline
4. ए॒ष ते॑ त ए॒ष ए॒ष ते᳚ । \newline
5. ते॒ रु॒द्र॒ रु॒द्र॒ ते॒ ते॒ रु॒द्र॒ । \newline
6. रु॒द्र॒ भा॒गो भा॒गो रु॑द्र रुद्र भा॒गः । \newline
7. भा॒गो यं ॅयम् भा॒गो भा॒गो यम् । \newline
8. यम् नि॒रया॑चथा नि॒रया॑चथा॒ यं ॅयम् नि॒रया॑चथाः । \newline
9. नि॒रया॑चथा॒ स्तम् तम् नि॒रया॑चथा नि॒रया॑चथा॒ स्तम् । \newline
10. नि॒रया॑चथा॒ इति॑ निः - अया॑चथाः । \newline
11. तम् जु॑षस्व जुषस्व॒ तम् तम् जु॑षस्व । \newline
12. जु॒ष॒स्व॒ वि॒देर् वि॒देर् जु॑षस्व जुषस्व वि॒देः । \newline
13. वि॒देर् गौ॑प॒त्यम् गौ॑प॒त्यं ॅवि॒देर् वि॒देर् गौ॑प॒त्यम् । \newline
14. गौ॒प॒त्यꣳ रा॒यो रा॒यो गौ॑प॒त्यम् गौ॑प॒त्यꣳ रा॒यः । \newline
15. रा॒य स्पोष॒म् पोषꣳ॑ रा॒यो रा॒य स्पोष᳚म् । \newline
16. पोषꣳ॑ सु॒वीर्यꣳ॑ सु॒वीर्य॒म् पोष॒म् पोषꣳ॑ सु॒वीर्य᳚म् । \newline
17. सु॒वीर्यꣳ॑ संॅवथ्स॒रीणाꣳ॑ संॅवथ्स॒रीणाꣳ॑ सु॒वीर्यꣳ॑ सु॒वीर्यꣳ॑ संॅवथ्स॒रीणा᳚म् । \newline
18. सु॒वीर्य॒मिति॑ सु - वीर्य᳚म् । \newline
19. सं॒ॅव॒थ्स॒रीणाꣳ॑ स्व॒स्तिꣳ स्व॒स्तिꣳ सं॑ॅवथ्स॒रीणाꣳ॑ संॅवथ्स॒रीणाꣳ॑ स्व॒स्तिम् । \newline
20. सं॒ॅव॒थ्स॒रीणा॒मिति॑ सं - व॒थ्स॒रीणा᳚म् । \newline
21. स्व॒स्तिमिति॑ स्व॒स्तिम् । \newline
22. मनुः॑ पु॒त्रेभ्यः॑ पु॒त्रेभ्यो॒ मनु॒र् मनुः॑ पु॒त्रेभ्यः॑ । \newline
23. पु॒त्रेभ्यो॑ दा॒यम् दा॒यम् पु॒त्रेभ्यः॑ पु॒त्रेभ्यो॑ दा॒यम् । \newline
24. दा॒यं ॅवि वि दा॒यम् दा॒यं ॅवि । \newline
25. व्य॑भज दभज॒द् वि व्य॑भजत् । \newline
26. अ॒भ॒ज॒थ् स सो॑ ऽभज दभज॒थ् सः । \newline
27. स नाभा॒नेदि॑ष्ठ॒म् नाभा॒नेदि॑ष्ठꣳ॒॒ स स नाभा॒नेदि॑ष्ठम् । \newline
28. नाभा॒नेदि॑ष्ठम् ब्रह्म॒चर्य॑म् ब्रह्म॒चर्य॒म् नाभा॒नेदि॑ष्ठ॒म् नाभा॒नेदि॑ष्ठम् ब्रह्म॒चर्य᳚म् । \newline
29. ब्र॒ह्म॒चर्यं॒ ॅवस॑न्तं॒ ॅवस॑न्तम् ब्रह्म॒चर्य॑म् ब्रह्म॒चर्यं॒ ॅवस॑न्तम् । \newline
30. ब्र॒ह्म॒चर्य॒मिति॑ ब्रह्म - चर्य᳚म् । \newline
31. वस॑न्त॒म् निर् णिर् वस॑न्तं॒ ॅवस॑न्त॒म् निः । \newline
32. निर॑भज दभज॒न् निर् णि र॑भजत् । \newline
33. अ॒भ॒ज॒थ् स सो॑ ऽभज दभज॒थ् सः । \newline
34. स आ स स आ । \newline
35. आ ऽग॑च्छद गच्छ॒दा ऽग॑च्छत् । \newline
36. अ॒ग॒च्छ॒थ् स सो॑ ऽगच्छ दगच्छ॒थ् सः । \newline
37. सो᳚ ऽब्रवी दब्रवी॒थ् स सो᳚ ऽब्रवीत् । \newline
38. अ॒ब्र॒वी॒त् क॒था क॒था ऽब्र॑वी दब्रवीत् क॒था । \newline
39. क॒था मा॑ मा क॒था क॒था मा᳚ । \newline
40. मा॒ निर् णिर् मा॑ मा॒ निः । \newline
41. निर॑भागभा॒ङ् निर् णिर॑भाक् । \newline
42. अ॒भा॒ गिती त्य॑भागभा॒ गिति॑ । \newline
43. इति॒ न ने तीति॒ न । \newline
44. न त्वा᳚ त्वा॒ न न त्वा᳚ । \newline
45. त्वा॒ निर् णिष् ट्वा᳚ त्वा॒ निः । \newline
46. निर॑भाक्ष मभाक्ष॒म् निर् णिर॑भाक्षम् । \newline
47. अ॒भा॒क्ष॒ मितीत्य॑भाक्ष मभाक्ष॒ मिति॑ । \newline
48. इत्य॑ब्रवी दब्रवी॒ दिती त्य॑ब्रवीत् । \newline
49. अ॒ब्र॒वी॒ दङ्गि॑र॒सो ऽङ्गि॑रसो ऽब्रवी दब्रवी॒ दङ्गि॑रसः । \newline
50. अङ्गि॑रस इ॒म इ॒मे ऽङ्गि॑र॒सो ऽङ्गि॑रस इ॒मे । \newline
51. इ॒मे स॒त्रꣳ स॒त्र मि॒म इ॒मे स॒त्रम् । \newline
52. स॒त्र मा॑सत आसते स॒त्रꣳ स॒त्र मा॑सते । \newline
53. आ॒स॒ते॒ ते त आ॑सत आसते॒ ते । \newline
54. ते सु॑व॒र्गꣳ सु॑व॒र्गम् ते ते सु॑व॒र्गम् । \newline

\textbf{Ghana Paata } \newline

1. य ए॒व मे॒वं ॅयो य ए॒वं ॅवेद॒ वेदै॒वं ॅयो य ए॒वं ॅवेद॑ । \newline
2. ए॒वं ॅवेद॒ वेदै॒व मे॒वं ॅवेदै॒ष ए॒ष वेदै॒व मे॒वं ॅवेदै॒षः । \newline
3. वेदै॒ष ए॒ष वेद॒ वेदै॒ष ते॑ त ए॒ष वेद॒ वेदै॒ष ते᳚ । \newline
4. ए॒ष ते॑ त ए॒ष ए॒ष ते॑ रुद्र रुद्र त ए॒ष ए॒ष ते॑ रुद्र । \newline
5. ते॒ रु॒द्र॒ रु॒द्र॒ ते॒ ते॒ रु॒द्र॒ भा॒गो भा॒गो रु॑द्र ते ते रुद्र भा॒गः । \newline
6. रु॒द्र॒ भा॒गो भा॒गो रु॑द्र रुद्र भा॒गो यं ॅयम् भा॒गो रु॑द्र रुद्र भा॒गो यम् । \newline
7. भा॒गो यं ॅयम् भा॒गो भा॒गो यम् नि॒रया॑चथा नि॒रया॑चथा॒ यम् भा॒गो भा॒गो यम् नि॒रया॑चथाः । \newline
8. यम् नि॒रया॑चथा नि॒रया॑चथा॒ यं ॅयम् नि॒रया॑चथा॒ स्तम् तम् नि॒रया॑चथा॒ यं ॅयम् नि॒रया॑चथा॒ स्तम् । \newline
9. नि॒रया॑चथा॒ स्तम् तम् नि॒रया॑चथा नि॒रया॑चथा॒ स्तम् जु॑षस्व जुषस्व॒ तम् नि॒रया॑चथा नि॒रया॑चथा॒ स्तम् जु॑षस्व । \newline
10. नि॒रया॑चथा॒ इति॑ निः - अया॑चथाः । \newline
11. तम् जु॑षस्व जुषस्व॒ तम् तम् जु॑षस्व वि॒देर् वि॒देर् जु॑षस्व॒ तम् तम् जु॑षस्व वि॒देः । \newline
12. जु॒ष॒स्व॒ वि॒देर् वि॒देर् जु॑षस्व जुषस्व वि॒देर् गौ॑प॒त्यम् गौ॑प॒त्यं ॅवि॒देर् जु॑षस्व जुषस्व वि॒देर् गौ॑प॒त्यम् । \newline
13. वि॒देर् गौ॑प॒त्यम् गौ॑प॒त्यं ॅवि॒देर् वि॒देर् गौ॑प॒त्यꣳ रा॒यो रा॒यो गौ॑प॒त्यं ॅवि॒देर् वि॒देर् गौ॑प॒त्यꣳ रा॒यः । \newline
14. गौ॒प॒त्यꣳ रा॒यो रा॒यो गौ॑प॒त्यम् गौ॑प॒त्यꣳ रा॒य स्पोष॒म् पोषꣳ॑ रा॒यो गौ॑प॒त्यम् गौ॑प॒त्यꣳ रा॒य स्पोष᳚म् । \newline
15. रा॒य स्पोष॒म् पोषꣳ॑ रा॒यो रा॒य स्पोषꣳ॑ सु॒वीर्यꣳ॑ सु॒वीर्य॒म् पोषꣳ॑ रा॒यो रा॒य स्पोषꣳ॑ सु॒वीर्य᳚म् । \newline
16. पोषꣳ॑ सु॒वीर्यꣳ॑ सु॒वीर्य॒म् पोष॒म् पोषꣳ॑ सु॒वीर्यꣳ॑ संॅवथ्स॒रीणाꣳ॑ संॅवथ्स॒रीणाꣳ॑ सु॒वीर्य॒म् पोष॒म् पोषꣳ॑ सु॒वीर्यꣳ॑ संॅवथ्स॒रीणा᳚म् । \newline
17. सु॒वीर्यꣳ॑ संॅवथ्स॒रीणाꣳ॑ संॅवथ्स॒रीणाꣳ॑ सु॒वीर्यꣳ॑ सु॒वीर्यꣳ॑ संॅवथ्स॒रीणाꣳ॑ स्व॒स्तिꣳ स्व॒स्तिꣳ सं॑ॅवथ्स॒रीणाꣳ॑ सु॒वीर्यꣳ॑ सु॒वीर्यꣳ॑ संॅवथ्स॒रीणाꣳ॑ स्व॒स्तिम् । \newline
18. सु॒वीर्य॒मिति॑ सु - वीर्य᳚म् । \newline
19. सं॒ॅव॒थ्स॒रीणाꣳ॑ स्व॒स्तिꣳ स्व॒स्तिꣳ सं॑ॅवथ्स॒रीणाꣳ॑ संॅवथ्स॒रीणाꣳ॑ स्व॒स्तिम् । \newline
20. सं॒ॅव॒थ्स॒रीणा॒मिति॑ सं - व॒थ्स॒रीणा᳚म् । \newline
21. स्व॒स्तिमिति॑ स्व॒स्तिम् । \newline
22. मनुः॑ पु॒त्रेभ्यः॑ पु॒त्रेभ्यो॒ मनु॒र् मनुः॑ पु॒त्रेभ्यो॑ दा॒यम् दा॒यम् पु॒त्रेभ्यो॒ मनु॒र् मनुः॑ पु॒त्रेभ्यो॑ दा॒यम् । \newline
23. पु॒त्रेभ्यो॑ दा॒यम् दा॒यम् पु॒त्रेभ्यः॑ पु॒त्रेभ्यो॑ दा॒यं ॅवि वि दा॒यम् पु॒त्रेभ्यः॑ पु॒त्रेभ्यो॑ दा॒यं ॅवि । \newline
24. दा॒यं ॅवि वि दा॒यम् दा॒यं ॅव्य॑भज दभज॒द् वि दा॒यम् दा॒यं ॅव्य॑भजत् । \newline
25. व्य॑भज दभज॒द् वि व्य॑भज॒थ् स सो॑ ऽभज॒द् वि व्य॑भज॒थ् सः । \newline
26. अ॒भ॒ज॒थ् स सो॑ ऽभज दभज॒थ् स नाभा॒नेदि॑ष्ठ॒म् नाभा॒नेदि॑ष्ठꣳ॒॒ सो॑ ऽभज दभज॒थ् स नाभा॒नेदि॑ष्ठम् । \newline
27. स नाभा॒नेदि॑ष्ठ॒म् नाभा॒नेदि॑ष्ठꣳ॒॒ स स नाभा॒नेदि॑ष्ठम् ब्रह्म॒चर्य॑म् ब्रह्म॒चर्य॒म् नाभा॒नेदि॑ष्ठꣳ॒॒ स स नाभा॒नेदि॑ष्ठम् ब्रह्म॒चर्य᳚म् । \newline
28. नाभा॒नेदि॑ष्ठम् ब्रह्म॒चर्य॑म् ब्रह्म॒चर्य॒म् नाभा॒नेदि॑ष्ठ॒म् नाभा॒नेदि॑ष्ठम् ब्रह्म॒चर्यं॒ ॅवस॑न्तं॒ ॅवस॑न्तम् ब्रह्म॒चर्य॒म् नाभा॒नेदि॑ष्ठ॒म् नाभा॒नेदि॑ष्ठम् ब्रह्म॒चर्यं॒ ॅवस॑न्तम् । \newline
29. ब्र॒ह्म॒चर्यं॒ ॅवस॑न्तं॒ ॅवस॑न्तम् ब्रह्म॒चर्य॑म् ब्रह्म॒चर्यं॒ ॅवस॑न्त॒म् निर् णिर् वस॑न्तम् ब्रह्म॒चर्य॑म् ब्रह्म॒चर्यं॒ ॅवस॑न्त॒म् निः । \newline
30. ब्र॒ह्म॒चर्य॒मिति॑ ब्रह्म - चर्य᳚म् । \newline
31. वस॑न्त॒म् निर् णिर् वस॑न्तं॒ ॅवस॑न्त॒म् निर॑भज दभज॒न् निर् वस॑न्तं॒ ॅवस॑न्त॒म् निर॑भजत् । \newline
32. निर॑भज दभज॒न् निर् णिर॑भज॒थ् स सो॑ ऽभज॒न् निर् णिर॑भज॒थ् सः । \newline
33. अ॒भ॒ज॒थ् स सो॑ ऽभज दभज॒थ् स आ सो॑ ऽभज दभज॒थ् स आ । \newline
34. स आ स स आ ऽग॑च्छ दगच्छ॒दा स स आ ऽग॑च्छत् । \newline
35. आ ऽग॑च्छ दगच्छ॒दा ऽग॑च्छ॒थ् स सो॑ ऽगच्छ॒दा ऽग॑च्छ॒थ् सः । \newline
36. अ॒ग॒च्छ॒थ् स सो॑ ऽगच्छ दगच्छ॒थ् सो᳚ ऽब्रवी दब्रवी॒थ् सो॑ ऽगच्छ दगच्छ॒थ् सो᳚ ऽब्रवीत् । \newline
37. सो᳚ ऽब्रवी दब्रवी॒थ् स सो᳚ ऽब्रवीत् क॒था क॒था ऽब्र॑वी॒थ् स सो᳚ ऽब्रवीत् क॒था । \newline
38. अ॒ब्र॒वी॒त् क॒था क॒था ऽब्र॑वी दब्रवीत् क॒था मा॑ मा क॒था ऽब्र॑वी दब्रवीत् क॒था मा᳚ । \newline
39. क॒था मा॑ मा क॒था क॒था मा॒ निर् णिर् मा॑ क॒था क॒था मा॒ निः । \newline
40. मा॒ निर् णिर् मा॑ मा॒ निर॑भा गभा॒ङ् निर् मा॑ मा॒ निर॑भाक् । \newline
41. निर॑भा गभा॒ङ् निर् णिर॑भा॒ गिती त्य॑भा॒ङ् निर् णिर॑भा॒गिति॑ । \newline
42. अ॒भा॒ गिती त्य॑भा गभा॒ गिति॒ न ने त्य॑भा गभा॒ गिति॒ न । \newline
43. इति॒ न ने तीति॒ न त्वा᳚ त्वा॒ ने तीति॒ न त्वा᳚ । \newline
44. न त्वा᳚ त्वा॒ न न त्वा॒ निर् णिष् ट्वा॒ न न त्वा॒ निः । \newline
45. त्वा॒ निर् णिष् ट्वा᳚ त्वा॒ निर॑भाक्ष मभाक्ष॒म् निष् ट्वा᳚ त्वा॒ निर॑भाक्षम् । \newline
46. निर॑भाक्ष मभाक्ष॒म् निर् णिर॑भाक्ष॒ मिती त्य॑भाक्ष॒म् निर् णिर॑भाक्ष॒ मिति॑ । \newline
47. अ॒भा॒क्ष॒ मिती त्य॑भाक्ष मभाक्ष॒ मित्य॑ब्रवी दब्रवी॒ दित्य॑भाक्ष मभाक्ष॒ मित्य॑ब्रवीत् । \newline
48. इत्य॑ब्रवी दब्रवी॒ दिती त्य॑ब्रवी॒ दङ्गि॑र॒सो ऽङ्गि॑रसो ऽब्रवी॒ दिती त्य॑ब्रवी॒ दङ्गि॑रसः । \newline
49. अ॒ब्र॒वी॒ दङ्गि॑र॒सो ऽङ्गि॑रसो ऽब्रवी दब्रवी॒ दङ्गि॑रस इ॒म इ॒मे ऽङ्गि॑रसो ऽब्रवी दब्रवी॒ दङ्गि॑रस इ॒मे । \newline
50. अङ्गि॑रस इ॒म इ॒मे ऽङ्गि॑र॒सो ऽङ्गि॑रस इ॒मे स॒त्रꣳ स॒त्र मि॒मे ऽङ्गि॑र॒सो ऽङ्गि॑रस इ॒मे स॒त्रम् । \newline
51. इ॒मे स॒त्रꣳ स॒त्र मि॒म इ॒मे स॒त्र मा॑सत आसते स॒त्र मि॒म इ॒मे स॒त्र मा॑सते । \newline
52. स॒त्र मा॑सत आसते स॒त्रꣳ स॒त्र मा॑सते॒ ते त आ॑सते स॒त्रꣳ स॒त्र मा॑सते॒ ते । \newline
53. आ॒स॒ते॒ ते त आ॑सत आसते॒ ते सु॑व॒र्गꣳ सु॑व॒र्गम् त आ॑सत आसते॒ ते सु॑व॒र्गम् । \newline
54. ते सु॑व॒र्गꣳ सु॑व॒र्गम् ते ते सु॑व॒र्गम् ॅलो॒कम् ॅलो॒कꣳ सु॑व॒र्गम् ते ते सु॑व॒र्गम् ॅलो॒कम् । \newline
\pagebreak
\markright{ TS 3.1.9.5  \hfill https://www.vedavms.in \hfill}

\section{ TS 3.1.9.5 }

\textbf{TS 3.1.9.5 } \newline
\textbf{Samhita Paata} \newline

सु॑व॒र्गं ॅलो॒कं न प्रजा॑नन्ति॒ तेभ्य॑ इ॒दं ब्राह्म॑णं ब्रूहि॒ ते सु॑व॒र्गं ॅलो॒कं ॅयन्तो॒ य ए॑षां प॒शव॒स्ताꣳस्ते॑ दास्य॒न्तीति॒ तदे᳚भ्योऽब्रवी॒त् ते सु॑व॒र्गं ॅलो॒कं ॅयन्तो॒ य ए॑षां प॒शव॒ आस॒न् तान॑स्मा अददु॒स्तं प॒शुभि॒श्चर॑न्तं ॅयज्ञ्वा॒स्तौ रु॒द्र आऽग॑च्छ॒थ् सो᳚ऽब्रवी॒न्मम॒ वा इ॒मे प॒शव॒ इत्यदु॒र्वै - [  ] \newline

\textbf{Pada Paata} \newline

सु॒व॒र्गमिति॑ सुवः - गम् । लो॒कम् । न । प्रेति॑ । जा॒न॒न्ति॒ । तेभ्यः॑ । इ॒दम् । ब्राह्म॑णम् । ब्रू॒हि॒ । ते । सु॒व॒र्गमिति॑ सुवः - गम् । लो॒कम् । यन्तः॑ । ये । ए॒षा॒म् । प॒शवः॑ । तान् । ते॒ । दा॒स्य॒न्ति॒ । इति॑ । तत् । ए॒भ्यः॒ । अ॒ब्र॒वी॒त् । ते । सु॒व॒र्गमिति॑ सुवः - गम् । लो॒कम् । यन्तः॑ । ये । ए॒षा॒म् । प॒शवः॑ । आसन्न्॑ । तान् । अ॒स्मै॒ । अ॒द॒दुः॒ । तम् । प॒शुभि॒रिति॑ प॒शु - भिः॒ । चर॑न्तम् । य॒ज्ञ्॒वा॒स्ताविति॑ यज्ञ्-वा॒स्तौ । रु॒द्रः । एति॑ । अ॒ग॒च्छ॒त् । सः । अ॒ब्र॒वी॒त् । मम॑ । वै । इ॒मे । प॒शवः॑ । इति॑ । अदुः॑ । वै ।  \newline


\textbf{Krama Paata} \newline

सु॒व॒र्गं ॅलो॒कम् । सु॒व॒र्गमिति॑ सुवः - गम् । लो॒कम् न । न प्र । प्र जा॑नन्ति । जा॒न॒न्ति॒ तेभ्यः॑ । तेभ्य॑ इ॒दम् । इ॒दम् ब्राह्म॑णम् । ब्राह्म॑णम् ब्रूहि । ब्रू॒हि॒ ते । ते सु॑व॒र्गम् । सु॒व॒र्गं ॅलो॒कम् । सु॒व॒र्गमिति॑ सुवः - गम् । लो॒कं ॅयन्तः॑ । यन्तो॒ ये । य ए॑षाम् । ए॒षा॒म् प॒शवः॑ । प॒शव॒ स्तान् । ताꣳस्ते᳚ । ते॒ दा॒स्य॒न्ति॒ । दा॒स्य॒न्तीति॑ । इति॒ तत् । तदे᳚भ्यः । ए॒भ्यो॒ ऽब्र॒वी॒त्॒ । अ॒ब्र॒वी॒त् ते । ते सु॑व॒र्गम् । सु॒व॒र्गं ॅलो॒कम् । सु॒व॒र्गमिति॑ सुवः - गम् । लो॒कं ॅयन्तः॑ । यन्तो॒ ये । य ए॑षाम् । ए॒षा॒म् प॒शवः॑ । प॒शव॒ आसन्न्॑ । आस॒न् तान् । तान॑स्मै । अ॒स्मा॒ अ॒द॒दुः॒ । अ॒द॒दु॒स्तम् । तम् प॒शुभिः॑ । प॒शुभि॒श्चर॑न्तम् । प॒शुभि॒रिति॑ प॒शु - भिः॒ । चर॑न्तं ॅयज्ञ्वा॒स्तौ । य॒ज्ञ्॒वा॒स्तौ रु॒द्रः । य॒ज्ञ्॒वा॒स्ताविति॑ यज्ञ् - वा॒स्तौ । रु॒द्र आ । आऽग॑च्छत् । अ॒ग॒च्छ॒थ् सः । सो᳚ऽब्रवीत् । अ॒ब्र॒वी॒न् मम॑ । मम॒ वै । वा इ॒मे । इ॒मे प॒शवः॑ । प॒शव॒ इति॑ । इत्यदुः॑ । अदु॒र् वै । वै मह्य᳚म् \newline

\textbf{Jatai Paata} \newline

1. सु॒व॒र्गम् ॅलो॒कम् ॅलो॒कꣳ सु॑व॒र्गꣳ सु॑व॒र्गम् ॅलो॒कम् । \newline
2. सु॒व॒र्गमिति॑ सुवः - गम् । \newline
3. लो॒कम् न न लो॒कम् ॅलो॒कम् न । \newline
4. न प्र प्र ण न प्र । \newline
5. प्र जा॑नन्ति जानन्ति॒ प्र प्र जा॑नन्ति । \newline
6. जा॒न॒न्ति॒ तेभ्य॒ स्तेभ्यो॑ जानन्ति जानन्ति॒ तेभ्यः॑ । \newline
7. तेभ्य॑ इ॒द मि॒दम् तेभ्य॒ स्तेभ्य॑ इ॒दम् । \newline
8. इ॒दम् ब्राह्म॑ण॒म् ब्राह्म॑ण मि॒द मि॒दम् ब्राह्म॑णम् । \newline
9. ब्राह्म॑णम् ब्रूहि ब्रूहि॒ ब्राह्म॑ण॒म् ब्राह्म॑णम् ब्रूहि । \newline
10. ब्रू॒हि॒ ते ते ब्रू॑हि ब्रूहि॒ ते । \newline
11. ते सु॑व॒र्गꣳ सु॑व॒र्गम् ते ते सु॑व॒र्गम् । \newline
12. सु॒व॒र्गम् ॅलो॒कम् ॅलो॒कꣳ सु॑व॒र्गꣳ सु॑व॒र्गम् ॅलो॒कम् । \newline
13. सु॒व॒र्गमिति॑ सुवः - गम् । \newline
14. लो॒कं ॅयन्तो॒ यन्तो॑ लो॒कम् ॅलो॒कं ॅयन्तः॑ । \newline
15. यन्तो॒ ये ये यन्तो॒ यन्तो॒ ये । \newline
16. य ए॑षा मेषां॒ ॅये य ए॑षाम् । \newline
17. ए॒षा॒म् प॒शवः॑ प॒शव॑ एषा मेषाम् प॒शवः॑ । \newline
18. प॒शव॒ स्ताꣳ स्तान् प॒शवः॑ प॒शव॒ स्तान् । \newline
19. ताꣳ स्ते॑ ते॒ ताꣳ स्ताꣳ स्ते᳚ । \newline
20. ते॒ दा॒स्य॒न्ति॒ दा॒स्य॒न्ति॒ ते॒ ते॒ दा॒स्य॒न्ति॒ । \newline
21. दा॒स्य॒न्तीतीति॑ दास्यन्ति दास्य॒न्तीति॑ । \newline
22. इति॒ तत् तदितीति॒ तत् । \newline
23. तदे᳚भ्य एभ्य॒ स्तत् तदे᳚भ्यः । \newline
24. ए॒भ्यो॒ ऽब्र॒वी॒ द॒ब्र॒वी॒ दे॒भ्य॒ ए॒भ्यो॒ ऽब्र॒वी॒त् । \newline
25. अ॒ब्र॒वी॒त् ते ते᳚ ऽब्रवी दब्रवी॒त् ते । \newline
26. ते सु॑व॒र्गꣳ सु॑व॒र्गम् ते ते सु॑व॒र्गम् । \newline
27. सु॒व॒र्गम् ॅलो॒कम् ॅलो॒कꣳ सु॑व॒र्गꣳ सु॑व॒र्गम् ॅलो॒कम् । \newline
28. सु॒व॒र्गमिति॑ सुवः - गम् । \newline
29. लो॒कं ॅयन्तो॒ यन्तो॑ लो॒कम् ॅलो॒कं ॅयन्तः॑ । \newline
30. यन्तो॒ ये ये यन्तो॒ यन्तो॒ ये । \newline
31. य ए॑षा मेषां॒ ॅये य ए॑षाम् । \newline
32. ए॒षा॒म् प॒शवः॑ प॒शव॑ एषा मेषाम् प॒शवः॑ । \newline
33. प॒शव॒ आस॒न् नास॑न् प॒शवः॑ प॒शव॒ आसन्न्॑ । \newline
34. आस॒न् ताꣳ स्ता नास॒न् नास॒न् तान् । \newline
35. ता न॑स्मा अस्मै॒ ताꣳ स्ता न॑स्मै । \newline
36. अ॒स्मा॒ अ॒द॒दु॒ र॒द॒दु॒ र॒स्मा॒ अ॒स्मा॒ अ॒द॒दुः॒ । \newline
37. अ॒द॒दु॒ स्तम् त म॑ददु रददु॒ स्तम् । \newline
38. तम् प॒शुभिः॑ प॒शुभि॒ स्तम् तम् प॒शुभिः॑ । \newline
39. प॒शुभि॒ श्चर॑न्त॒म् चर॑न्तम् प॒शुभिः॑ प॒शुभि॒ श्चर॑न्तम् । \newline
40. प॒शुभि॒रिति॑ प॒शु - भिः॒ । \newline
41. चर॑न्तं ॅयज्ञ्वा॒स्तौ य॑ज्ञ्वा॒स्तौ चर॑न्त॒म् चर॑न्तं ॅयज्ञ्वा॒स्तौ । \newline
42. य॒ज्ञ्॒वा॒स्तौ रु॒द्रो रु॒द्रो य॑ज्ञ्वा॒स्तौ य॑ज्ञ्वा॒स्तौ रु॒द्रः । \newline
43. य॒ज्ञ्॒वा॒स्ताविति॑ यज्ञ् - वा॒स्तौ । \newline
44. रु॒द्र आ रु॒द्रो रु॒द्र आ । \newline
45. आ ऽग॑च्छ दगच्छ॒ दा ऽग॑च्छत् । \newline
46. अ॒ग॒च्छ॒थ् स सो॑ ऽगच्छ दगच्छ॒थ् सः । \newline
47. सो᳚ ऽब्रवी दब्रवी॒थ् स सो᳚ ऽब्रवीत् । \newline
48. अ॒ब्र॒वी॒न् मम॒ ममा᳚ब्रवी दब्रवी॒न् मम॑ । \newline
49. मम॒ वै वै मम॒ मम॒ वै । \newline
50. वा इ॒म इ॒मे वै वा इ॒मे । \newline
51. इ॒मे प॒शवः॑ प॒शव॑ इ॒म इ॒मे प॒शवः॑ । \newline
52. प॒शव॒ इतीति॑ प॒शवः॑ प॒शव॒ इति॑ । \newline
53. इत्यदु॒ रदु॒रिती त्यदुः॑ । \newline
54. अदु॒र् वै वा अदु॒ रदु॒र् वै । \newline
55. वै मह्य॒म् मह्यं॒ ॅवै वै मह्य᳚म् । \newline

\textbf{Ghana Paata } \newline

1. सु॒व॒र्गम् ॅलो॒कम् ॅलो॒कꣳ सु॑व॒र्गꣳ सु॑व॒र्गम् ॅलो॒कन्न न लो॒कꣳ सु॑व॒र्गꣳ सु॑व॒र्गम् ॅलो॒कन्न । \newline
2. सु॒व॒र्गमिति॑ सुवः - गम् । \newline
3. लो॒कम् न न लो॒कम् ॅलो॒कम् न प्र प्र ण लो॒कम् ॅलो॒कम् न प्र । \newline
4. न प्र प्र ण न प्र जा॑नन्ति जानन्ति॒ प्र ण न प्र जा॑नन्ति । \newline
5. प्र जा॑नन्ति जानन्ति॒ प्र प्र जा॑नन्ति॒ तेभ्य॒ स्तेभ्यो॑ जानन्ति॒ प्र प्र जा॑नन्ति॒ तेभ्यः॑ । \newline
6. जा॒न॒न्ति॒ तेभ्य॒ स्तेभ्यो॑ जानन्ति जानन्ति॒ तेभ्य॑ इ॒द मि॒दम् तेभ्यो॑ जानन्ति जानन्ति॒ तेभ्य॑ इ॒दम् । \newline
7. तेभ्य॑ इ॒द मि॒दम् तेभ्य॒ स्तेभ्य॑ इ॒दम् ब्राह्म॑ण॒म् ब्राह्म॑ण मि॒दम् तेभ्य॒ स्तेभ्य॑ इ॒दम् ब्राह्म॑णम् । \newline
8. इ॒दम् ब्राह्म॑ण॒म् ब्राह्म॑ण मि॒द मि॒दम् ब्राह्म॑णम् ब्रूहि ब्रूहि॒ ब्राह्म॑ण मि॒द मि॒दम् ब्राह्म॑णम् ब्रूहि । \newline
9. ब्राह्म॑णम् ब्रूहि ब्रूहि॒ ब्राह्म॑ण॒म् ब्राह्म॑णम् ब्रूहि॒ ते ते ब्रू॑हि॒ ब्राह्म॑ण॒म् ब्राह्म॑णम् ब्रूहि॒ ते । \newline
10. ब्रू॒हि॒ ते ते ब्रू॑हि ब्रूहि॒ ते सु॑व॒र्गꣳ सु॑व॒र्गम् ते ब्रू॑हि ब्रूहि॒ ते सु॑व॒र्गम् । \newline
11. ते सु॑व॒र्गꣳ सु॑व॒र्गम् ते ते सु॑व॒र्गम् ॅलो॒कम् ॅलो॒कꣳ सु॑व॒र्गम् ते ते सु॑व॒र्गम् ॅलो॒कम् । \newline
12. सु॒व॒र्गम् ॅलो॒कम् ॅलो॒कꣳ सु॑व॒र्गꣳ सु॑व॒र्गम् ॅलो॒कं ॅयन्तो॒ यन्तो॑ लो॒कꣳ सु॑व॒र्गꣳ सु॑व॒र्गम् ॅलो॒कं ॅयन्तः॑ । \newline
13. सु॒व॒र्गमिति॑ सुवः - गम् । \newline
14. लो॒कं ॅयन्तो॒ यन्तो॑ लो॒कम् ॅलो॒कं ॅयन्तो॒ ये ये यन्तो॑ लो॒कम् ॅलो॒कं ॅयन्तो॒ ये । \newline
15. यन्तो॒ ये ये यन्तो॒ यन्तो॒ य ए॑षा मेषां॒ ॅये यन्तो॒ यन्तो॒ य ए॑षाम् । \newline
16. य ए॑षा मेषां॒ ॅये य ए॑षाम् प॒शवः॑ प॒शव॑ एषां॒ ॅये य ए॑षाम् प॒शवः॑ । \newline
17. ए॒षा॒म् प॒शवः॑ प॒शव॑ एषा मेषाम् प॒शव॒ स्ताꣳ स्तान् प॒शव॑ एषा मेषाम् प॒शव॒ स्तान् । \newline
18. प॒शव॒ स्ताꣳ स्तान् प॒शवः॑ प॒शव॒ स्ताꣳ स्ते॑ ते॒ तान् प॒शवः॑ प॒शव॒ स्ताꣳ स्ते᳚ । \newline
19. ताꣳ स्ते॑ ते॒ ताꣳ स्ताꣳ स्ते॑ दास्यन्ति दास्यन्ति ते॒ ताꣳ स्ताꣳ स्ते॑ दास्यन्ति । \newline
20. ते॒ दा॒स्य॒न्ति॒ दा॒स्य॒न्ति॒ ते॒ ते॒ दा॒स्य॒न्तीतीति॑ दास्यन्ति ते ते दास्य॒न्तीति॑ । \newline
21. दा॒स्य॒न्तीतीति॑ दास्यन्ति दास्य॒न्तीति॒ तत् तदिति॑ दास्यन्ति दास्य॒न्तीति॒ तत् । \newline
22. इति॒ तत् तदितीति॒ तदे᳚भ्य एभ्य॒ स्तदितीति॒ तदे᳚भ्यः । \newline
23. तदे᳚भ्य एभ्य॒ स्तत् तदे᳚भ्यो ऽब्रवी दब्रवी देभ्य॒ स्तत् तदे᳚भ्यो ऽब्रवीत् । \newline
24. ए॒भ्यो॒ ऽब्र॒वी॒ द॒ब्र॒वी॒ दे॒भ्य॒ ए॒भ्यो॒ ऽब्र॒वी॒त् ते ते᳚ ऽब्रवी देभ्य एभ्यो ऽब्रवी॒त् ते । \newline
25. अ॒ब्र॒वी॒त् ते ते᳚ ऽब्रवी दब्रवी॒त् ते सु॑व॒र्गꣳ सु॑व॒र्गम् ते᳚ ऽब्रवी दब्रवी॒त् ते सु॑व॒र्गम् । \newline
26. ते सु॑व॒र्गꣳ सु॑व॒र्गम् ते ते सु॑व॒र्गम् ॅलो॒कम् ॅलो॒कꣳ सु॑व॒र्गम् ते ते सु॑व॒र्गम् ॅलो॒कम् । \newline
27. सु॒व॒र्गम् ॅलो॒कम् ॅलो॒कꣳ सु॑व॒र्गꣳ सु॑व॒र्गम् ॅलो॒कं ॅयन्तो॒ यन्तो॑ लो॒कꣳ सु॑व॒र्गꣳ सु॑व॒र्गम् ॅलो॒कं ॅयन्तः॑ । \newline
28. सु॒व॒र्गमिति॑ सुवः - गम् । \newline
29. लो॒कं ॅयन्तो॒ यन्तो॑ लो॒कम् ॅलो॒कं ॅयन्तो॒ ये ये यन्तो॑ लो॒कम् ॅलो॒कं ॅयन्तो॒ ये । \newline
30. यन्तो॒ ये ये यन्तो॒ यन्तो॒ य ए॑षा मेषां॒ ॅये यन्तो॒ यन्तो॒ य ए॑षाम् । \newline
31. य ए॑षा मेषां॒ ॅये य ए॑षाम् प॒शवः॑ प॒शव॑ एषां॒ ॅये य ए॑षाम् प॒शवः॑ । \newline
32. ए॒षा॒म् प॒शवः॑ प॒शव॑ एषा मेषाम् प॒शव॒ आस॒न् नास॑न् प॒शव॑ एषा मेषाम् प॒शव॒ आसन्न्॑ । \newline
33. प॒शव॒ आस॒न् नास॑न् प॒शवः॑ प॒शव॒ आस॒न् ताꣳ स्ता नास॑न् प॒शवः॑ प॒शव॒ आस॒न् तान् । \newline
34. आस॒न् ताꣳ स्ता नास॒न् नास॒न् ता न॑स्मा अस्मै॒ ता नास॒न् नास॒न् ता न॑स्मै । \newline
35. ता न॑स्मा अस्मै॒ ताꣳ स्ता न॑स्मा अददु रददु रस्मै॒ ताꣳ स्ता न॑स्मा अददुः । \newline
36. अ॒स्मा॒ अ॒द॒दु॒ र॒द॒दु॒ र॒स्मा॒ अ॒स्मा॒ अ॒द॒दु॒ स्तम् त म॑ददु रस्मा अस्मा अद दु॒स्तम् । \newline
37. अ॒द॒दु॒ स्तम् त म॑ददु रददु॒ स्तम् प॒शुभिः॑ प॒शुभि॒ स्त म॑ददु रददु॒ स्तम् प॒शुभिः॑ । \newline
38. तम् प॒शुभिः॑ प॒शुभि॒ स्तम् तम् प॒शुभि॒ श्चर॑न्त॒म् चर॑न्तम् प॒शुभि॒ स्तम् तम् प॒शुभि॒ श्चर॑न्तम् । \newline
39. प॒शुभि॒ श्चर॑न्त॒म् चर॑न्तम् प॒शुभिः॑ प॒शुभि॒ श्चर॑न्तं ॅयज्ञ्वा॒स्तौ य॑ज्ञ्वा॒स्तौ चर॑न्तम् प॒शुभिः॑ प॒शुभि॒ श्चर॑न्तं ॅयज्ञ्वा॒स्तौ । \newline
40. प॒शुभि॒रिति॑ प॒शु - भिः॒ । \newline
41. चर॑न्तं ॅयज्ञ्वा॒स्तौ य॑ज्ञ्वा॒स्तौ चर॑न्त॒म् चर॑न्तं ॅयज्ञ्वा॒स्तौ रु॒द्रो रु॒द्रो य॑ज्ञ्वा॒स्तौ चर॑न्त॒म् चर॑न्तं ॅयज्ञ्वा॒स्तौ रु॒द्रः । \newline
42. य॒ज्ञ्॒वा॒स्तौ रु॒द्रो रु॒द्रो य॑ज्ञ्वा॒स्तौ य॑ज्ञ्वा॒स्तौ रु॒द्र आ रु॒द्रो य॑ज्ञ्वा॒स्तौ य॑ज्ञ्वा॒स्तौ रु॒द्र आ । \newline
43. य॒ज्ञ्॒वा॒स्ताविति॑ यज्ञ् - वा॒स्तौ । \newline
44. रु॒द्र आ रु॒द्रो रु॒द्र आ ऽग॑च्छ दगच्छ॒दा रु॒द्रो रु॒द्र आ ऽग॑च्छत् । \newline
45. आ ऽग॑च्छ दगच्छ॒दा ऽग॑च्छ॒थ् स सो॑ ऽगच्छ॒दा ऽग॑च्छ॒थ् सः । \newline
46. अ॒ग॒च्छ॒थ् स सो॑ ऽगच्छ दगच्छ॒थ् सो᳚ ऽब्रवी दब्रवी॒थ् सो॑ ऽगच्छ दगच्छ॒थ् सो᳚ ऽब्रवीत् । \newline
47. सो᳚ ऽब्रवी दब्रवी॒थ् स सो᳚ ऽब्रवी॒न् मम॒ ममा᳚ब्रवी॒थ् स सो᳚ ऽब्रवी॒न् मम॑ । \newline
48. अ॒ब्र॒वी॒न् मम॒ ममा᳚ब्रवी दब्रवी॒न् मम॒ वै वै ममा᳚ब्रवी दब्रवी॒न् मम॒ वै । \newline
49. मम॒ वै वै मम॒ मम॒ वा इ॒म इ॒मे वै मम॒ मम॒ वा इ॒मे । \newline
50. वा इ॒म इ॒मे वै वा इ॒मे प॒शवः॑ प॒शव॑ इ॒मे वै वा इ॒मे प॒शवः॑ । \newline
51. इ॒मे प॒शवः॑ प॒शव॑ इ॒म इ॒मे प॒शव॒ इतीति॑ प॒शव॑ इ॒म इ॒मे प॒शव॒ इति॑ । \newline
52. प॒शव॒ इतीति॑ प॒शवः॑ प॒शव॒ इत्यदु॒ रदु॒रिति॑ प॒शवः॑ प॒शव॒ इत्यदुः॑ । \newline
53. इत्यदु॒ रदु॒ रितीत्यदु॒र् वै वा अदु॒ रितीत्यदु॒र् वै । \newline
54. अदु॒र् वै वा अदु॒ रदु॒र् वै मह्य॒म् मह्यं॒ ॅवा अदु॒ रदु॒र् वै मह्य᳚म् । \newline
55. वै मह्य॒म् मह्यं॒ ॅवै वै मह्य॑ मि॒मा नि॒मान् मह्यं॒ ॅवै वै मह्य॑ मि॒मान् । \newline
\pagebreak
\markright{ TS 3.1.9.6  \hfill https://www.vedavms.in \hfill}

\section{ TS 3.1.9.6 }

\textbf{TS 3.1.9.6 } \newline
\textbf{Samhita Paata} \newline

मह्य॑मि॒मानित्य॑ब्रवी॒न्न वै तस्य॒ त ई॑शत॒ इत्य॑ब्रवी॒द्-यद्-य॑ज्ञ्वा॒स्तौ हीय॑ते॒ मम॒ वै तदिति॒ तस्मा᳚द्-यज्ञ्वा॒स्तु नाभ्य॒वेत्यꣳ॒॒ सो᳚ऽब्रवीद्-य॒ज्ञे मा ऽऽभ॒जाथ॑ ते प॒शून् नाभि मꣳ॑स्य॒ इति॒ तस्मा॑ ए॒तं म॒न्थिनः॑ सꣳ स्रा॒वम॑जुहो॒त् ततो॒ वै तस्य॑ रु॒द्रः प॒शून् नाभ्य॑मन्यत॒ यत्रै॒त ( ) मे॒वं ॅवि॒द्वान् म॒न्थिनः॑ सꣳ स्रा॒वं जु॒होति॒ न तत्र॑ रु॒द्रः प॒शून॒भि म॑न्यते ॥ \newline

\textbf{Pada Paata} \newline

मह्य᳚म् । इ॒मान् । इति॑ । अ॒ब्र॒वी॒त् । न । वै । तस्य॑ । ते । ई॒श॒ते॒ । इति॑ । अ॒ब्र॒वी॒त् । यत् । य॒ज्ञ्॒वा॒स्ताविति॑ यज्ञ्-वा॒स्तौ । हीय॑ते । मम॑ । वै । तत् । इति॑ । तस्मा᳚त् । य॒ज्ञ्॒वा॒स्त्विति॑ यज्ञ् - वा॒स्तु । न । अ॒भ्य॒वेत्य॒मित्य॑भि - अ॒वेत्य᳚म् । सः । अ॒ब्र॒वी॒त् । य॒ज्ञे । मा॒ । एति॑ । भ॒ज॒ । अथ॑ । ते॒ । प॒शून् । न । अ॒भीति॑ । मꣳ॒॒स्ये॒ । इति॑ । तस्मै᳚ । ए॒तम् । म॒न्थिनः॑ । सꣳ॒॒स्रा॒वमिति॑ सं - स्रा॒वम् । अ॒जु॒हो॒त् । ततः॑ । वै । तस्य॑ । रु॒द्रः । प॒शून् । न । अ॒भीति॑ । अ॒म॒न्य॒त॒ । यत्र॑ । ए॒तम् ( ) । ए॒वम् । वि॒द्वान् । म॒न्थिनः॑ । सꣳ॒॒स्रा॒वमिति॑ सं - स्रा॒वम् । जु॒होति॑ । न । तत्र॑ । रु॒द्रः । प॒शून् । अ॒भीति॑ । म॒न्य॒त॒ ॥  \newline


\textbf{Krama Paata} \newline

मह्य॑मि॒मान् । इ॒मानिति॑ । इत्य॑ब्रवीत् । अ॒ब्र॒वी॒न् न । न वै । वै तस्य॑ । तस्य॒ ते । त ई॑शते । ई॒श॒त॒ इति॑ । इत्य॑ब्रवीत् । अ॒ब्र॒वी॒द् यत् । यद् य॑ज्ञ्वा॒स्तौ । य॒ज्ञ्॒वा॒स्तौ हीय॑ते । य॒ज्ञ्॒वा॒स्ताविति॑ यज्ञ् - वा॒स्तौ । हीय॑ते॒ मम॑ । मम॒ वै । वै तत् । तदिति॑ । इति॒ तस्मा᳚त् । तस्मा᳚द् यज्ञ्वा॒स्तु । य॒ज्ञ्॒वा॒स्तु न । य॒ज्ञ्॒वा॒स्त्विति॑ यज्ञ् - वा॒स्तु । नाभ्य॒वेत्य᳚म् । अ॒भ्य॒वेत्यꣳ॒॒ सः । अ॒भ्य॒वेत्य॒मित्य॑भि - अ॒वेत्य᳚म् । सो᳚ऽब्रवीत् । अ॒ब्र॒वी॒द् य॒ज्ञे । य॒ज्ञे मा᳚ । मा । आ भ॑ज । भ॒जाथ॑ । अथ॑ ते । ते॒ प॒शून् । प॒शून् न । नाभि । अ॒भि मꣳ॑स्ये । मꣳ॒॒स्य॒ इति॑ । इति॒ तस्मै᳚ । तस्मा॑ ए॒तम् । ए॒तम् म॒न्थिनः॑ । म॒न्थिन॑ सꣳस्रा॒वम् । सꣳ॒॒स्रा॒वम॑जुहोत् । सꣳ॒॒स्रा॒वमिति॑ सम् - स्रा॒वम् । अ॒जु॒हो॒त् ततः॑ । ततो॒ वै । वै तस्य॑ । तस्य॑ रु॒द्रः । रु॒द्रः प॒शून् । प॒शून् न । नाभि । अ॒भ्य॑मन्यत । अ॒म॒न्य॒त॒ यत्र॑ । यत्रै॒तम् ( ) । ए॒तमे॒वम् । ए॒वं ॅवि॒द्वान् । वि॒द्वान् म॒न्थिनः॑ । म॒न्थिनः॑ सꣳस्रा॒वम् । सꣳ॒॒स्रा॒वम् जु॒होति॑ । सꣳ॒॒स्रा॒वमिति॑ सम् - स्रा॒वम् । जु॒होति॒ न । न तत्र॑ । तत्र॑ रु॒द्रः । रु॒द्रः प॒शून् । प॒शून॒भि । अ॒भि म॑न्यते । म॒न्य॒त॒ इति॑ मन्यते । \newline

\textbf{Jatai Paata} \newline

1. मह्य॑ मि॒मा नि॒मान् मह्य॒म् मह्य॑ मि॒मान् । \newline
2. इ॒मा नितीती॒मा नि॒मा निति॑ । \newline
3. इत्य॑ब्रवी दब्रवी॒ दिती त्य॑ब्रवीत् । \newline
4. अ॒ब्र॒वी॒न् न नाब्र॑वी दब्रवी॒न् न । \newline
5. न वै वै न न वै । \newline
6. वै तस्य॒ तस्य॒ वै वै तस्य॑ । \newline
7. तस्य॒ ते ते तस्य॒ तस्य॒ ते । \newline
8. त ई॑शत ईशते॒ ते त ई॑शते । \newline
9. ई॒श॒त॒ इतीती॑शत ईशत॒ इति॑ । \newline
10. इत्य॑ब्रवी दब्रवी॒ दिती त्य॑ब्रवीत् । \newline
11. अ॒ब्र॒वी॒द् यद् यद॑ब्रवी दब्रवी॒द् यत् । \newline
12. यद् य॑ज्ञ्वा॒स्तौ य॑ज्ञ्वा॒स्तौ यद् यद् य॑ज्ञ्वा॒स्तौ । \newline
13. य॒ज्ञ्॒वा॒स्तौ हीय॑ते॒ हीय॑ते यज्ञ्वा॒स्तौ य॑ज्ञ्वा॒स्तौ हीय॑ते । \newline
14. य॒ज्ञ्॒वा॒स्ताविति॑ यज्ञ् - वा॒स्तौ । \newline
15. हीय॑ते॒ मम॒ मम॒ हीय॑ते॒ हीय॑ते॒ मम॑ । \newline
16. मम॒ वै वै मम॒ मम॒ वै । \newline
17. वै तत् तद् वै वै तत् । \newline
18. तदितीति॒ तत् तदिति॑ । \newline
19. इति॒ तस्मा॒त् तस्मा॒ दितीति॒ तस्मा᳚त् । \newline
20. तस्मा᳚द् यज्ञ्वा॒स्तु य॑ज्ञ्वा॒स्तु तस्मा॒त् तस्मा᳚द् यज्ञ्वा॒स्तु । \newline
21. य॒ज्ञ्॒वा॒स्तु न न य॑ज्ञ्वा॒स्तु य॑ज्ञ्वा॒स्तु न । \newline
22. य॒ज्ञ्॒वा॒स्त्विति॑ यज्ञ् - वा॒स्तु । \newline
23. नाभ्य॒वेत्य॑ मभ्य॒वेत्य॒म् न नाभ्य॒वेत्य᳚म् । \newline
24. अ॒भ्य॒वेत्यꣳ॒॒ स सो᳚ ऽभ्य॒वेत्य॑ मभ्य॒वेत्यꣳ॒॒ सः । \newline
25. अ॒भ्य॒वेत्य॒मित्य॑भि - अ॒वेत्य᳚म् । \newline
26. सो᳚ ऽब्रवी दब्रवी॒थ् स सो᳚ ऽब्रवीत् । \newline
27. अ॒ब्र॒वी॒द् य॒ज्ञे य॒ज्ञे᳚ ऽब्रवी दब्रवीद् य॒ज्ञे । \newline
28. य॒ज्ञे मा॑ मा य॒ज्ञे य॒ज्ञे मा᳚ । \newline
29. मा ऽऽमा॒ मा । \newline
30. आ भ॑ज भ॒जा भ॑ज । \newline
31. भ॒जाथाथ॑ भज भ॒जाथ॑ । \newline
32. अथ॑ ते॒ ते ऽथाथ॑ ते । \newline
33. ते॒ प॒शून् प॒शून् ते॑ ते प॒शून् । \newline
34. प॒शून् न न प॒शून् प॒शून् न । \newline
35. नाभ्य॑भि न नाभि । \newline
36. अ॒भि मꣳ॑स्ये मꣳस्ये॒ ऽभ्य॑भि मꣳ॑स्ये । \newline
37. मꣳ॒॒स्य॒ इतीति॑ मꣳस्ये मꣳस्य॒ इति॑ । \newline
38. इति॒ तस्मै॒ तस्मा॒ इतीति॒ तस्मै᳚ । \newline
39. तस्मा॑ ए॒त मे॒तम् तस्मै॒ तस्मा॑ ए॒तम् । \newline
40. ए॒तम् म॒न्थिनो॑ म॒न्थिन॑ ए॒त मे॒तम् म॒न्थिनः॑ । \newline
41. म॒न्थिनः॑ सꣳस्रा॒वꣳ सꣳ॑स्रा॒वम् म॒न्थिनो॑ म॒न्थिनः॑ सꣳस्रा॒वम् । \newline
42. सꣳ॒॒स्रा॒व म॑जुहो दजुहोथ् सꣳस्रा॒वꣳ सꣳ॑स्रा॒व म॑जुहोत् । \newline
43. सꣳ॒॒स्रा॒वमिति॑ सं - स्रा॒वम् । \newline
44. अ॒जु॒हो॒त् तत॒ स्ततो॑ ऽजुहो दजुहो॒त् ततः॑ । \newline
45. ततो॒ वै वै तत॒ स्ततो॒ वै । \newline
46. वै तस्य॒ तस्य॒ वै वै तस्य॑ । \newline
47. तस्य॑ रु॒द्रो रु॒द्र स्तस्य॒ तस्य॑ रु॒द्रः । \newline
48. रु॒द्रः प॒शून् प॒शून् रु॒द्रो रु॒द्रः प॒शून् । \newline
49. प॒शून् न न प॒शून् प॒शून् न । \newline
50. नाभ्य॑भि न नाभि । \newline
51. अ॒भ्य॑मन्यता मन्यता॒ भ्या᳚(1॒)भ्य॑मन्यत । \newline
52. अ॒म॒न्य॒त॒ यत्र॒ यत्रा॑मन्यता मन्यत॒ यत्र॑ । \newline
53. यत्रै॒त मे॒तं ॅयत्र॒ यत्रै॒तम् । \newline
54. ए॒त मे॒व मे॒व मे॒त मे॒त मे॒वम् । \newline
55. ए॒वं ॅवि॒द्वान्. वि॒द्वा ने॒व मे॒वं ॅवि॒द्वान् । \newline
56. वि॒द्वान् म॒न्थिनो॑ म॒न्थिनो॑ वि॒द्वान्. वि॒द्वान् म॒न्थिनः॑ । \newline
57. म॒न्थिनः॑ सꣳस्रा॒वꣳ सꣳ॑स्रा॒वम् म॒न्थिनो॑ म॒न्थिनः॑ सꣳस्रा॒वम् । \newline
58. सꣳ॒॒स्रा॒वम् जु॒होति॑ जु॒होति॑ सꣳस्रा॒वꣳ सꣳ॑स्रा॒वम् जु॒होति॑ । \newline
59. सꣳ॒॒स्रा॒वमिति॑ सं - स्रा॒वम् । \newline
60. जु॒होति॒ न न जु॒होति॑ जु॒होति॒ न । \newline
61. न तत्र॒ तत्र॒ न न तत्र॑ । \newline
62. तत्र॑ रु॒द्रो रु॒द्र स्तत्र॒ तत्र॑ रु॒द्रः । \newline
63. रु॒द्रः प॒शून् प॒शून् रु॒द्रो रु॒द्रः प॒शून् । \newline
64. प॒शू न॒भ्य॑भि प॒शून् प॒शू न॒भि । \newline
65. अ॒भि म॑न्यते मन्यते॒ ऽभ्य॑भि म॑न्यते । \newline
66. म॒न्य॒त॒ इति॑ मन्यते । \newline

\textbf{Ghana Paata } \newline

1. मह्य॑ मि॒मा नि॒मान् मह्य॒म् मह्य॑ मि॒मा नितीती॒मान् मह्य॒म् मह्य॑ मि॒मा निति॑ । \newline
2. इ॒मा नितीती॒मा नि॒मा नित्य॑ब्रवी दब्रवी॒ दिती॒मा नि॒मा नित्य॑ब्रवीत् । \newline
3. इत्य॑ब्रवी दब्रवी॒ दिती त्य॑ब्रवी॒न् न नाब्र॑वी॒ दिती त्य॑ब्रवी॒न् न । \newline
4. अ॒ब्र॒वी॒न् न नाब्र॑वी दब्रवी॒न् न वै वै नाब्र॑वी दब्रवी॒न् न वै । \newline
5. न वै वै न न वै तस्य॒ तस्य॒ वै न न वै तस्य॑ । \newline
6. वै तस्य॒ तस्य॒ वै वै तस्य॒ ते ते तस्य॒ वै वै तस्य॒ ते । \newline
7. तस्य॒ ते ते तस्य॒ तस्य॒ त ई॑शत ईशते॒ ते तस्य॒ तस्य॒ त ई॑शते । \newline
8. त ई॑शत ईशते॒ ते त ई॑शत॒ इतीती॑शते॒ ते त ई॑शत॒ इति॑ । \newline
9. ई॒श॒त॒ इतीती॑शत ईशत॒ इत्य॑ब्रवी दब्रवी॒ दिती॑शत ईशत॒ इत्य॑ब्रवीत् । \newline
10. इत्य॑ब्रवी दब्रवी॒ दिती त्य॑ब्रवी॒द् यद् यद॑ब्रवी॒ दिती त्य॑ब्रवी॒द् यत् । \newline
11. अ॒ब्र॒वी॒द् यद् यद॑ब्रवी दब्रवी॒द् यद् य॑ज्ञ्वा॒स्तौ य॑ज्ञ्वा॒स्तौ यद॑ब्रवी दब्रवी॒द् यद् य॑ज्ञ्वा॒स्तौ । \newline
12. यद् य॑ज्ञ्वा॒स्तौ य॑ज्ञ्वा॒स्तौ यद् यद् य॑ज्ञ्वा॒स्तौ हीय॑ते॒ हीय॑ते यज्ञ्वा॒स्तौ यद् यद् य॑ज्ञ्वा॒स्तौ हीय॑ते । \newline
13. य॒ज्ञ्॒वा॒स्तौ हीय॑ते॒ हीय॑ते यज्ञ्वा॒स्तौ य॑ज्ञ्वा॒स्तौ हीय॑ते॒ मम॒ मम॒ हीय॑ते यज्ञ्वा॒स्तौ य॑ज्ञ्वा॒स्तौ हीय॑ते॒ मम॑ । \newline
14. य॒ज्ञ्॒वा॒स्ताविति॑ यज्ञ् - वा॒स्तौ । \newline
15. हीय॑ते॒ मम॒ मम॒ हीय॑ते॒ हीय॑ते॒ मम॒ वै वै मम॒ हीय॑ते॒ हीय॑ते॒ मम॒ वै । \newline
16. मम॒ वै वै मम॒ मम॒ वै तत् तद् वै मम॒ मम॒ वै तत् । \newline
17. वै तत् तद् वै वै तदितीति॒ तद् वै वै तदिति॑ । \newline
18. तदितीति॒ तत् तदिति॒ तस्मा॒त् तस्मा॒ दिति॒ तत् तदिति॒ तस्मा᳚त् । \newline
19. इति॒ तस्मा॒त् तस्मा॒ दितीति॒ तस्मा᳚द् यज्ञ्वा॒स्तु य॑ज्ञ्वा॒स्तु तस्मा॒ दितीति॒ तस्मा᳚द् यज्ञ्वा॒स्तु । \newline
20. तस्मा᳚द् यज्ञ्वा॒स्तु य॑ज्ञ्वा॒स्तु तस्मा॒त् तस्मा᳚द् यज्ञ्वा॒स्तु न न य॑ज्ञ्वा॒स्तु तस्मा॒त् तस्मा᳚द् यज्ञ्वा॒स्तु न । \newline
21. य॒ज्ञ्॒वा॒स्तु न न य॑ज्ञ्वा॒स्तु य॑ज्ञ्वा॒स्तु नाभ्य॒वेत्य॑ मभ्य॒वेत्य॒म् न य॑ज्ञ्वा॒स्तु य॑ज्ञ्वा॒स्तु नाभ्य॒वेत्य᳚म् । \newline
22. य॒ज्ञ्॒वा॒स्त्विति॑ यज्ञ् - वा॒स्तु । \newline
23. नाभ्य॒वेत्य॑ मभ्य॒वेत्य॒म् न नाभ्य॒वेत्यꣳ॒॒ स सो᳚ ऽभ्य॒वेत्य॒म् न नाभ्य॒वेत्यꣳ॒॒ सः । \newline
24. अ॒भ्य॒वेत्यꣳ॒॒ स सो᳚ ऽभ्य॒वेत्य॑ मभ्य॒वेत्यꣳ॒॒ सो᳚ ऽब्रवी दब्रवी॒थ् सो᳚ ऽभ्य॒वेत्य॑ मभ्य॒वेत्यꣳ॒॒ सो᳚ ऽब्रवीत् । \newline
25. अ॒भ्य॒वेत्य॒मित्य॑भि - अ॒वेत्य᳚म् । \newline
26. सो᳚ ऽब्रवी दब्रवी॒थ् स सो᳚ ऽब्रवीद् य॒ज्ञे य॒ज्ञे᳚ ऽब्रवी॒थ् स सो᳚ ऽब्रवीद् य॒ज्ञे । \newline
27. अ॒ब्र॒वी॒द् य॒ज्ञे य॒ज्ञे᳚ ऽब्रवी दब्रवीद् य॒ज्ञे मा॑ मा य॒ज्ञे᳚ ऽब्रवी दब्रवीद् य॒ज्ञे मा᳚ । \newline
28. य॒ज्ञे मा॑ मा य॒ज्ञे य॒ज्ञे मा ऽऽमा॑ य॒ज्ञे य॒ज्ञे मा । \newline
29. मा ऽऽमा॒ मा ऽऽभ॑ज भ॒जा मा॒ मा ऽऽभ॑ज । \newline
30. आ भ॑ज भ॒जा भ॒जाथाथ॑ भ॒जा भ॒जाथ॑ । \newline
31. भ॒जाथाथ॑ भज भ॒जाथ॑ ते॒ ते ऽथ॑ भज भ॒जाथ॑ ते । \newline
32. अथ॑ ते॒ ते ऽथाथ॑ ते प॒शून् प॒शून् ते ऽथाथ॑ ते प॒शून् । \newline
33. ते॒ प॒शून् प॒शून् ते॑ ते प॒शून् न न प॒शून् ते॑ ते प॒शून् न । \newline
34. प॒शून् न न प॒शून् प॒शून् नाभ्य॑भि न प॒शून् प॒शून् नाभि । \newline
35. नाभ्य॑भि न नाभि मꣳ॑स्ये मꣳस्ये॒ ऽभि न नाभि मꣳ॑स्ये । \newline
36. अ॒भि मꣳ॑स्ये मꣳस्ये॒ ऽभ्य॑भि मꣳ॑स्य॒ इतीति॑ मꣳस्ये॒ ऽभ्य॑भि मꣳ॑स्य॒ इति॑ । \newline
37. मꣳ॒॒स्य॒ इतीति॑ मꣳस्ये मꣳस्य॒ इति॒ तस्मै॒ तस्मा॒ इति॑ मꣳस्ये मꣳस्य॒ इति॒ तस्मै᳚ । \newline
38. इति॒ तस्मै॒ तस्मा॒ इतीति॒ तस्मा॑ ए॒त मे॒तम् तस्मा॒ इतीति॒ तस्मा॑ ए॒तम् । \newline
39. तस्मा॑ ए॒त मे॒तम् तस्मै॒ तस्मा॑ ए॒तम् म॒न्थिनो॑ म॒न्थिन॑ ए॒तम् तस्मै॒ तस्मा॑ ए॒तम् म॒न्थिनः॑ । \newline
40. ए॒तम् म॒न्थिनो॑ म॒न्थिन॑ ए॒त मे॒तम् म॒न्थिनः॑ सꣳस्रा॒वꣳ सꣳ॑स्रा॒वम् म॒न्थिन॑ ए॒त मे॒तम् म॒न्थिनः॑ सꣳस्रा॒वम् । \newline
41. म॒न्थिनः॑ सꣳस्रा॒वꣳ सꣳ॑स्रा॒वम् म॒न्थिनो॑ म॒न्थिनः॑ सꣳस्रा॒व म॑जुहो दजुहोथ् सꣳस्रा॒वम् म॒न्थिनो॑ म॒न्थिनः॑ सꣳस्रा॒व म॑जुहोत् । \newline
42. सꣳ॒॒स्रा॒व म॑जुहो दजुहोथ् सꣳस्रा॒वꣳ सꣳ॑स्रा॒व म॑जुहो॒त् तत॒ स्ततो॑ ऽजुहोथ् सꣳस्रा॒वꣳ सꣳ॑स्रा॒व म॑जुहो॒त् ततः॑ । \newline
43. सꣳ॒॒स्रा॒वमिति॑ सं - स्रा॒वम् । \newline
44. अ॒जु॒हो॒त् तत॒ स्ततो॑ ऽजुहो दजुहो॒त् ततो॒ वै वै ततो॑ ऽजुहो दजुहो॒त् ततो॒ वै । \newline
45. ततो॒ वै वै तत॒ स्ततो॒ वै तस्य॒ तस्य॒ वै तत॒ स्ततो॒ वै तस्य॑ । \newline
46. वै तस्य॒ तस्य॒ वै वै तस्य॑ रु॒द्रो रु॒द्र स्तस्य॒ वै वै तस्य॑ रु॒द्रः । \newline
47. तस्य॑ रु॒द्रो रु॒द्र स्तस्य॒ तस्य॑ रु॒द्रः प॒शून् प॒शून् रु॒द्र स्तस्य॒ तस्य॑ रु॒द्रः प॒शून् । \newline
48. रु॒द्रः प॒शून् प॒शून् रु॒द्रो रु॒द्रः प॒शून् न न प॒शून् रु॒द्रो रु॒द्रः प॒शून् न । \newline
49. प॒शून् न न प॒शून् प॒शून् नाभ्य॑भि न प॒शून् प॒शून् नाभि । \newline
50. नाभ्य॑भि न नाभ्य॑मन्यता मन्यता॒भि न नाभ्य॑मन्यत । \newline
51. अ॒भ्य॑मन्यता मन्यता॒ भ्या᳚(1॒)भ्य॑मन्यत॒ यत्र॒ यत्रा॑मन्यता॒ भ्या᳚(1॒)भ्य॑मन्यत॒ यत्र॑ । \newline
52. अ॒म॒न्य॒त॒ यत्र॒ यत्रा॑मन्यता मन्यत॒ यत्रै॒त मे॒तं ॅयत्रा॑मन्यता मन्यत॒ यत्रै॒तम् । \newline
53. यत्रै॒त मे॒तं ॅयत्र॒ यत्रै॒त मे॒व मे॒व मे॒तं ॅयत्र॒ यत्रै॒त मे॒वम् । \newline
54. ए॒त मे॒व मे॒व मे॒त मे॒त मे॒वं ॅवि॒द्वान्. वि॒द्वा ने॒व मे॒त मे॒त मे॒वं ॅवि॒द्वान् । \newline
55. ए॒वं ॅवि॒द्वान्. वि॒द्वा ने॒व मे॒वं ॅवि॒द्वान् म॒न्थिनो॑ म॒न्थिनो॑ वि॒द्वा ने॒व मे॒वं ॅवि॒द्वान् म॒न्थिनः॑ । \newline
56. वि॒द्वान् म॒न्थिनो॑ म॒न्थिनो॑ वि॒द्वान्. वि॒द्वान् म॒न्थिनः॑ सꣳस्रा॒वꣳ सꣳ॑स्रा॒वम् म॒न्थिनो॑ वि॒द्वान्. वि॒द्वान् म॒न्थिनः॑ सꣳस्रा॒वम् । \newline
57. म॒न्थिनः॑ सꣳस्रा॒वꣳ सꣳ॑स्रा॒वम् म॒न्थिनो॑ म॒न्थिनः॑ सꣳस्रा॒वम् जु॒होति॑ जु॒होति॑ सꣳस्रा॒वम् म॒न्थिनो॑ म॒न्थिनः॑ सꣳस्रा॒वम् जु॒होति॑ । \newline
58. सꣳ॒॒स्रा॒वम् जु॒होति॑ जु॒होति॑ सꣳस्रा॒वꣳ सꣳ॑स्रा॒वम् जु॒होति॒ न न जु॒होति॑ सꣳस्रा॒वꣳ सꣳ॑स्रा॒वम् जु॒होति॒ न । \newline
59. सꣳ॒॒स्रा॒वमिति॑ सं - स्रा॒वम् । \newline
60. जु॒होति॒ न न जु॒होति॑ जु॒होति॒ न तत्र॒ तत्र॒ न जु॒होति॑ जु॒होति॒ न तत्र॑ । \newline
61. न तत्र॒ तत्र॒ न न तत्र॑ रु॒द्रो रु॒द्र स्तत्र॒ न न तत्र॑ रु॒द्रः । \newline
62. तत्र॑ रु॒द्रो रु॒द्र स्तत्र॒ तत्र॑ रु॒द्रः प॒शून् प॒शून् रु॒द्र स्तत्र॒ तत्र॑ रु॒द्रः प॒शून् । \newline
63. रु॒द्रः प॒शून् प॒शून् रु॒द्रो रु॒द्रः प॒शू न॒भ्य॑भि प॒शून् रु॒द्रो रु॒द्रः प॒शू न॒भि । \newline
64. प॒शू न॒भ्य॑भि प॒शून् प॒शू न॒भि म॑न्यते मन्यते॒ ऽभि प॒शून् प॒शू न॒भि म॑न्यते । \newline
65. अ॒भि म॑न्यते मन्यते॒ ऽभ्य॑भि म॑न्यते । \newline
66. म॒न्य॒त॒ इति॑ मन्यते । \newline
\pagebreak
\markright{ TS 3.1.10.1  \hfill https://www.vedavms.in \hfill}

\section{ TS 3.1.10.1 }

\textbf{TS 3.1.10.1 } \newline
\textbf{Samhita Paata} \newline

जुष्टो॑ वा॒चो भू॑यासं॒ जुष्टो॑ वा॒चस्पत॑ये॒ देवि॑ वाक् । यद्वा॒चो मधु॑म॒त् तस्मि॑न् मा धाः॒ स्वाहा॒ सर॑स्वत्यै ॥ ऋ॒चा स्तोमꣳ॒॒ सम॑र्द्धय गाय॒त्रेण॑ रथंत॒रं । बृ॒हद्-गा॑य॒त्रव॑र्तनि ॥यस्ते᳚ द्र॒फ्सः स्कन्द॑ति॒ यस्ते॑ अꣳ॒॒शुर्बा॒हुच्यु॑तो धि॒षण॑योरु॒पस्था᳚त् । अ॒द्ध्व॒र्योर्वा॒ परि॒ यस्ते॑ प॒वित्रा॒थ् स्वाहा॑कृत॒मिन्द्रा॑य॒ तं जु॑होमि ॥ यो द्र॒फ्सो अꣳ॒॒शुः प॑ति॒तः पृ॑थि॒व्यां प॑रिवा॒पात् - [  ] \newline

\textbf{Pada Paata} \newline

जुष्टः॑ । वा॒चः । भू॒या॒स॒म् । जुष्टः॑ । वा॒चः । पत॑ये । देवि॑ । वा॒क् ॥ यत् । वा॒चः । मधु॑म॒दिति॒ मधु॑-म॒त् । तस्मिन्न्॑ । मा॒ । धाः॒ । स्वाहा᳚ । सर॑स्वत्यै ॥ ऋ॒चा । स्तोम᳚म् । समिति॑ । अ॒द्‌र्ध॒य॒ । गा॒य॒त्रेण॑ । र॒थ॒न्त॒रमिति॑ रथं-त॒रम् ॥ बृ॒हत् । गा॒य॒त्रव॑र्त॒नीति॑ गाय॒त्र-व॒र्त॒नि॒ ॥ यः । ते॒ । द्र॒फ्सः । स्कन्द॑ति । यः । ते॒ । अꣳ॒॒शुः । बा॒हुच्यु॑त॒ इति॑ बा॒हु - च्यु॒तः॒ । धि॒षण॑योः । उ॒पस्था॒दित्यु॒प - स्था॒त् ॥ अ॒द्ध्व॒र्योः । वा॒ । परीति॑ । यः । ते॒ । प॒वित्रा᳚त् । स्वाहा॑कृत॒मिति॒ स्वाहा᳚ -कृ॒त॒म् । इन्द्रा॑य । तम् । जु॒हो॒मि॒ ॥ यः । द्र॒फ्सः । अꣳ॒॒शुः । प॒ति॒तः । पृ॒थि॒व्याम् । प॒रि॒वा॒पादिति॑ परि - वा॒पात् ।  \newline


\textbf{Krama Paata} \newline

जुष्टो॑ वा॒चः । वा॒चो भू॑यासम् । भू॒या॒स॒म् जुष्टः॑ । जुष्टो॑ वा॒चः । वा॒चस्पत॑ये । पत॑ये॒ देवि॑ । देवि॑ वाक् । वा॒गिति॑ वाक् ॥ यद् वा॒चः । वा॒चो मधु॑मत् । मधु॑म॒त् तस्मिन्न्॑ । मधु॑म॒दिति॒ मधु॑ - म॒त्॒ । तस्मि॑न् मा । मा॒ धाः॒ । धाः॒ स्वाहा᳚ । स्वाहा॒ सर॑स्वत्यै । सर॑स्वत्या॒ इति॒ सर॑स्वत्यै ॥ ऋ॒चा स्तोम᳚म् । स्तोमꣳ॒॒ सम् । सम॑र्द्धय । अ॒र्द्ध॒य॒ गा॒य॒त्रेण॑ । गा॒य॒त्रेण॑ रथन्त॒रम् । र॒थ॒न्त॒रमिति॑ रथम् - त॒रम् ॥ बृहद् गा॑य॒त्रव॑र्तनि । गा॒य॒त्रव॑र्त॒नीति॑ गाय॒त्र - व॒र्त॒नि॒ ॥ यस्ते᳚ । ते॒ द्र॒फ्सः । द्र॒फ्सः स्कन्द॑ति । स्कन्द॑ति॒ यः । यस्ते᳚ । ते॒ अꣳ॒॒शुः । अꣳ॒॒शुर् बा॒हुच्यु॑तः । बा॒हुच्यु॑तो धि॒षण॑योः । बा॒हुच्यु॑त॒ इति॑ बा॒हु - च्यु॒तः॒ । धि॒षण॑योरु॒पस्था᳚त् । उ॒पस्था॒दित्यु॒प - स्था॒त् ॥ अ॒द्ध्व॒र्योर् वा᳚ । वा॒ परि॑ । परि॒ यः । यस्ते᳚ । ते॒ प॒वित्रा᳚त् । प॒वित्रा॒थ् स्वाहा॑कृतम् । स्वाहा॑कृत॒मिन्द्रा॑य । स्वाहा॑कृत॒मिति॒ स्वाहा᳚ - कृ॒त॒म् । इन्द्रा॑य॒ तम् । तम् जु॑होमि । जु॒हो॒मीति॑ जुहोमि ॥ यो द्र॒फ्सः । द्र॒फ्सो अꣳ॒॒शुः । अꣳ॒॒शुः प॑ति॒तः । प॒ति॒तः पृ॑थि॒व्याम् । पृ॒थि॒व्याम् प॑रिवा॒पात् । प॒रि॒वा॒पात् पु॑रो॒डाशा᳚त् । प॒रि॒वा॒पादिति॑ परि - वा॒पात् \newline

\textbf{Jatai Paata} \newline

1. जुष्टो॑ वा॒चो वा॒चो जुष्टो॒ जुष्टो॑ वा॒चः । \newline
2. वा॒चो भू॑यासम् भूयासं ॅवा॒चो वा॒चो भू॑यासम् । \newline
3. भू॒या॒स॒म् जुष्टो॒ जुष्टो॑ भूयासम् भूयास॒म् जुष्टः॑ । \newline
4. जुष्टो॑ वा॒चो वा॒चो जुष्टो॒ जुष्टो॑ वा॒चः । \newline
5. वा॒च स्पत॑ये॒ पत॑ये वा॒चो वा॒च स्पत॑ये । \newline
6. पत॑ये॒ देवि॒ देवि॒ पत॑ये॒ पत॑ये॒ देवि॑ । \newline
7. देवि॑ वाग् वा॒ग् देवि॒ देवि॑ वाक् । \newline
8. वा॒गिति॑ वाक् । \newline
9. यद् वा॒चो वा॒चो यद् यद् वा॒चः । \newline
10. वा॒चो मधु॑म॒न् मधु॑मद् वा॒चो वा॒चो मधु॑मत् । \newline
11. मधु॑म॒त् तस्मिꣳ॒॒ स्तस्मि॒न् मधु॑म॒न् मधु॑म॒त् तस्मिन्न्॑ । \newline
12. मधु॑म॒दिति॒ मधु॑ - म॒त् । \newline
13. तस्मि॑न् मा मा॒ तस्मिꣳ॒॒ स्तस्मि॑न् मा । \newline
14. मा॒ धा॒ धा॒ मा॒ मा॒ धाः॒ । \newline
15. धाः॒ स्वाहा॒ स्वाहा॑ धा धाः॒ स्वाहा᳚ । \newline
16. स्वाहा॒ सर॑स्वत्यै॒ सर॑स्वत्यै॒ स्वाहा॒ स्वाहा॒ सर॑स्वत्यै । \newline
17. सर॑स्वत्या॒ इति॒ सर॑स्वत्यै । \newline
18. ऋ॒चा स्तोमꣳ॒॒ स्तोम॑ मृ॒चर्चा स्तोम᳚म् । \newline
19. स्तोमꣳ॒॒ सꣳ सꣳ स्तोमꣳ॒॒ स्तोमꣳ॒॒ सम् । \newline
20. स म॑र्द्धया र्द्धय॒ सꣳ स म॑र्द्धय । \newline
21. अ॒र्द्ध॒य॒ गा॒य॒त्रेण॑ गाय॒त्रेणा᳚ र्द्धया र्द्धय गाय॒त्रेण॑ । \newline
22. गा॒य॒त्रेण॑ रथन्त॒रꣳ र॑थन्त॒रम् गा॑य॒त्रेण॑ गाय॒त्रेण॑ रथन्त॒रम् । \newline
23. र॒थ॒न्त॒रमिति॑ रथं - त॒रम् । \newline
24. बृ॒हद् गा॑य॒त्रव॑र्तनि गाय॒त्रव॑र्तनि बृ॒हद् बृ॒हद् गा॑य॒त्रव॑र्तनि । \newline
25. गा॒य॒त्रव॑र्त॒नीति॑ गाय॒त्र - व॒र्त॒नि॒ । \newline
26. यस्ते॑ ते॒ यो यस्ते᳚ । \newline
27. ते॒ द्र॒फ्सो द्र॒फ्स स्ते॑ ते द्र॒फ्सः । \newline
28. द्र॒फ्सः स्कन्द॑ति॒ स्कन्द॑ति द्र॒फ्सो द्र॒फ्सः स्कन्द॑ति । \newline
29. स्कन्द॑ति॒ यो यः स्कन्द॑ति॒ स्कन्द॑ति॒ यः । \newline
30. यस्ते॑ ते॒ यो यस्ते᳚ । \newline
31. ते॒ अꣳ॒॒शु रꣳ॒॒शुस्ते॑ ते अꣳ॒॒शुः । \newline
32. अꣳ॒॒शुर् बा॒हुच्यु॑तो बा॒हुच्यु॑तो अꣳ॒॒शु रꣳ॒॒शुर् बा॒हुच्यु॑तः । \newline
33. बा॒हुच्यु॑तो धि॒षण॑योर् धि॒षण॑योर् बा॒हुच्यु॑तो बा॒हुच्यु॑तो धि॒षण॑योः । \newline
34. बा॒हुच्यु॑त॒ इति॑ बा॒हु - च्यु॒तः॒ । \newline
35. धि॒षण॑यो रु॒पस्था॑ दु॒पस्था᳚द् धि॒षण॑योर् धि॒षण॑यो रु॒पस्था᳚त् । \newline
36. उ॒पस्था॒दित्यु॒प - स्था॒त् । \newline
37. अ॒द्ध्व॒र्योर् वा॑ वा ऽद्ध्व॒र्यो र॑द्ध्व॒र्योर् वा᳚ । \newline
38. वा॒ परि॒ परि॑ वा वा॒ परि॑ । \newline
39. परि॒ यो यः परि॒ परि॒ यः । \newline
40. यस्ते॑ ते॒ यो यस्ते᳚ । \newline
41. ते॒ प॒वित्रा᳚त् प॒वित्रा᳚त् ते ते प॒वित्रा᳚त् । \newline
42. प॒वित्रा॒थ् स्वाहा॑कृतꣳ॒॒ स्वाहा॑कृतम् प॒वित्रा᳚त् प॒वित्रा॒थ् स्वाहा॑कृतम् । \newline
43. स्वाहा॑कृत॒ मिन्द्रा॒ये न्द्रा॑य॒ स्वाहा॑कृतꣳ॒॒ स्वाहा॑कृत॒ मिन्द्रा॑य । \newline
44. स्वाहा॑कृत॒मिति॒ स्वाहा᳚ - कृ॒त॒म् । \newline
45. इन्द्रा॑य॒ तम् त मिन्द्रा॒ये न्द्रा॑य॒ तम् । \newline
46. तम् जु॑होमि जुहोमि॒ तम् तम् जु॑होमि । \newline
47. जु॒हो॒मीति॑ जुहोमि । \newline
48. यो द्र॒फ्सो द्र॒फ्सो यो यो द्र॒फ्सः । \newline
49. द्र॒फ्सो अꣳ॒॒शु रꣳ॒॒शुर् द्र॒फ्सो द्र॒फ्सो अꣳ॒॒शुः । \newline
50. अꣳ॒॒शुः प॑ति॒तः प॑ति॒तो अꣳ॒॒शु रꣳ॒॒शुः प॑ति॒तः । \newline
51. प॒ति॒तः पृ॑थि॒व्याम् पृ॑थि॒व्याम् प॑ति॒तः प॑ति॒तः पृ॑थि॒व्याम् । \newline
52. पृ॒थि॒व्याम् प॑रिवा॒पात् प॑रिवा॒पात् पृ॑थि॒व्याम् पृ॑थि॒व्याम् प॑रिवा॒पात् । \newline
53. प॒रि॒वा॒पात् पु॑रो॒डाशा᳚त् पुरो॒डाशा᳚त् परिवा॒पात् प॑रिवा॒पात् पु॑रो॒डाशा᳚त् । \newline
54. प॒रि॒वा॒पादिति॑ परि - वा॒पात् । \newline

\textbf{Ghana Paata } \newline

1. जुष्टो॑ वा॒चो वा॒चो जुष्टो॒ जुष्टो॑ वा॒चो भू॑यासम् भूयासं ॅवा॒चो जुष्टो॒ जुष्टो॑ वा॒चो भू॑यासम् । \newline
2. वा॒चो भू॑यासम् भूयासं ॅवा॒चो वा॒चो भू॑यास॒म् जुष्टो॒ जुष्टो॑ भूयासं ॅवा॒चो वा॒चो भू॑यास॒म् जुष्टः॑ । \newline
3. भू॒या॒स॒म् जुष्टो॒ जुष्टो॑ भूयासम् भूयास॒म् जुष्टो॑ वा॒चो वा॒चो जुष्टो॑ भूयासम् भूयास॒म् जुष्टो॑ वा॒चः । \newline
4. जुष्टो॑ वा॒चो वा॒चो जुष्टो॒ जुष्टो॑ वा॒च स्पत॑ये॒ पत॑ये वा॒चो जुष्टो॒ जुष्टो॑ वा॒च स्पत॑ये । \newline
5. वा॒च स्पत॑ये॒ पत॑ये वा॒चो वा॒च स्पत॑ये॒ देवि॒ देवि॒ पत॑ये वा॒चो वा॒च स्पत॑ये॒ देवि॑ । \newline
6. पत॑ये॒ देवि॒ देवि॒ पत॑ये॒ पत॑ये॒ देवि॑ वाग् वा॒ग् देवि॒ पत॑ये॒ पत॑ये॒ देवि॑ वाक् । \newline
7. देवि॑ वाग् वा॒ग् देवि॒ देवि॑ वाक् । \newline
8. वा॒गिति॑ वाक् । \newline
9. यद् वा॒चो वा॒चो यद् यद् वा॒चो मधु॑म॒न् मधु॑मद् वा॒चो यद् यद् वा॒चो मधु॑मत् । \newline
10. वा॒चो मधु॑म॒न् मधु॑मद् वा॒चो वा॒चो मधु॑म॒त् तस्मिꣳ॒॒ स्तस्मि॒न् मधु॑मद् वा॒चो वा॒चो मधु॑म॒त् तस्मिन्न्॑ । \newline
11. मधु॑म॒त् तस्मिꣳ॒॒ स्तस्मि॒न् मधु॑म॒न् मधु॑म॒त् तस्मि॑न् मा मा॒ तस्मि॒न् मधु॑म॒न् मधु॑म॒त् तस्मि॑न् मा । \newline
12. मधु॑म॒दिति॒ मधु॑ - म॒त् । \newline
13. तस्मि॑न् मा मा॒ तस्मिꣳ॒॒ स्तस्मि॑न् मा धा धा मा॒ तस्मिꣳ॒॒ स्तस्मि॑न् मा धाः । \newline
14. मा॒ धा॒ धा॒ मा॒ मा॒ धाः॒ स्वाहा॒ स्वाहा॑ धा मा मा धाः॒ स्वाहा᳚ । \newline
15. धाः॒ स्वाहा॒ स्वाहा॑ धा धाः॒ स्वाहा॒ सर॑स्वत्यै॒ सर॑स्वत्यै॒ स्वाहा॑ धा धाः॒ स्वाहा॒ सर॑स्वत्यै । \newline
16. स्वाहा॒ सर॑स्वत्यै॒ सर॑स्वत्यै॒ स्वाहा॒ स्वाहा॒ सर॑स्वत्यै । \newline
17. सर॑स्वत्या॒ इति॒ सर॑स्वत्यै । \newline
18. ऋ॒चा स्तोमꣳ॒॒ स्तोम॑ मृ॒चर्चा स्तोमꣳ॒॒ सꣳ सꣳ स्तोम॑ मृ॒चर्चा स्तोमꣳ॒॒ सम् । \newline
19. स्तोमꣳ॒॒ सꣳ सꣳ स्तोमꣳ॒॒ स्तोमꣳ॒॒ स म॑र्द्धया र्द्धय॒ सꣳ स्तोमꣳ॒॒ स्तोमꣳ॒॒ स म॑र्द्धय । \newline
20. स म॑र्द्धया र्द्धय॒ सꣳ स म॑र्द्धय गाय॒त्रेण॑ गाय॒त्रेणा᳚ र्द्धय॒ सꣳ स म॑र्द्धय गाय॒त्रेण॑ । \newline
21. अ॒र्द्ध॒य॒ गा॒य॒त्रेण॑ गाय॒त्रेणा᳚ र्द्धया र्द्धय गाय॒त्रेण॑ रथन्त॒रꣳ र॑थन्त॒रम् गा॑य॒त्रेणा᳚ र्द्धया र्द्धय गाय॒त्रेण॑ रथन्त॒रम् । \newline
22. गा॒य॒त्रेण॑ रथन्त॒रꣳ र॑थन्त॒रम् गा॑य॒त्रेण॑ गाय॒त्रेण॑ रथन्त॒रम् । \newline
23. र॒थ॒न्त॒रमिति॑ रथं - त॒रम् । \newline
24. बृ॒हद् गा॑य॒त्रव॑र्तनि गाय॒त्रव॑र्त॒नि बृ॒हद् बृ॒हद् गा॑य॒त्रव॑र्तनि । \newline
25. गा॒य॒त्रव॑र्त॒नीति॑ गाय॒त्र - व॒र्त॒नि॒ । \newline
26. यस्ते॑ ते॒ यो यस्ते᳚ द्र॒फ्सो द्र॒फ्स स्ते॒ यो यस्ते᳚ द्र॒फ्सः । \newline
27. ते॒ द्र॒फ्सो द्र॒फ्स स्ते॑ ते द्र॒फ्सः स्कन्द॑ति॒ स्कन्द॑ति द्र॒फ्स स्ते॑ ते द्र॒फ्सः स्कन्द॑ति । \newline
28. द्र॒फ्सः स्कन्द॑ति॒ स्कन्द॑ति द्र॒फ्सो द्र॒फ्सः स्कन्द॑ति॒ यो यः स्कन्द॑ति द्र॒फ्सो द्र॒फ्सः स्कन्द॑ति॒ यः । \newline
29. स्कन्द॑ति॒ यो यः स्कन्द॑ति॒ स्कन्द॑ति॒ य स्ते॑ ते॒ यः स्कन्द॑ति॒ स्कन्द॑ति॒ यस्ते᳚ । \newline
30. यस्ते॑ ते॒ यो यस्ते॑ अꣳ॒॒शु रꣳ॒॒शु स्ते॒ यो यस्ते॑ अꣳ॒॒शुः । \newline
31. ते॒ अꣳ॒॒शु रꣳ॒॒शु स्ते॑ ते अꣳ॒॒शुर् बा॒हुच्यु॑तो बा॒हुच्यु॑तो अꣳ॒॒शुस्ते॑ ते अꣳ॒॒शुर् बा॒हुच्यु॑तः । \newline
32. अꣳ॒॒शुर् बा॒हुच्यु॑तो बा॒हुच्यु॑तो अꣳ॒॒शु रꣳ॒॒शुर् बा॒हुच्यु॑तो धि॒षण॑योर् धि॒षण॑योर् बा॒हुच्यु॑तो अꣳ॒॒शुरꣳ॒॒शुर् बा॒हुच्यु॑तो धि॒षण॑योः । \newline
33. बा॒हुच्यु॑तो धि॒षण॑योर् धि॒षण॑योर् बा॒हुच्यु॑तो बा॒हुच्यु॑तो धि॒षण॑यो रु॒पस्था॑ दु॒पस्था᳚द् धि॒षण॑योर् बा॒हुच्यु॑तो बा॒हुच्यु॑तो धि॒षण॑यो रु॒पस्था᳚त् । \newline
34. बा॒हुच्यु॑त॒ इति॑ बा॒हु - च्यु॒तः॒ । \newline
35. धि॒षण॑यो रु॒पस्था॑ दु॒पस्था᳚द् धि॒षण॑योर् धि॒षण॑यो रु॒पस्था᳚त् । \newline
36. उ॒पस्था॒दित्यु॒प - स्था॒त् । \newline
37. अ॒द्ध्व॒र्योर् वा॑ वा ऽद्ध्व॒र्यो र॑द्ध्व॒र्योर् वा॒ परि॒ परि॑ वा ऽद्ध्व॒र्यो र॑द्ध्व॒र्योर् वा॒ परि॑ । \newline
38. वा॒ परि॒ परि॑ वा वा॒ परि॒ यो यः परि॑ वा वा॒ परि॒ यः । \newline
39. परि॒ यो यः परि॒ परि॒ यस्ते॑ ते॒ यः परि॒ परि॒ यस्ते᳚ । \newline
40. यस्ते॑ ते॒ यो यस्ते॑ प॒वित्रा᳚त् प॒वित्रा᳚त् ते॒ यो यस्ते॑ प॒वित्रा᳚त् । \newline
41. ते॒ प॒वित्रा᳚त् प॒वित्रा᳚त् ते ते प॒वित्रा॒थ् स्वाहा॑कृतꣳ॒॒ स्वाहा॑कृतम् प॒वित्रा᳚त् ते ते प॒वित्रा॒थ् स्वाहा॑कृतम् । \newline
42. प॒वित्रा॒थ् स्वाहा॑कृतꣳ॒॒ स्वाहा॑कृतम् प॒वित्रा᳚त् प॒वित्रा॒थ् स्वाहा॑कृत॒ मिन्द्रा॒ये न्द्रा॑य॒ स्वाहा॑कृतम् प॒वित्रा᳚त् प॒वित्रा॒थ् स्वाहा॑कृत॒ मिन्द्रा॑य । \newline
43. स्वाहा॑कृत॒ मिन्द्रा॒ये न्द्रा॑य॒ स्वाहा॑कृतꣳ॒॒ स्वाहा॑कृत॒ मिन्द्रा॑य॒ तम् त मिन्द्रा॑य॒ स्वाहा॑कृतꣳ॒॒ स्वाहा॑कृत॒ मिन्द्रा॑य॒ तम् । \newline
44. स्वाहा॑कृत॒मिति॒ स्वाहा᳚ - कृ॒त॒म् । \newline
45. इन्द्रा॑य॒ तम् त मिन्द्रा॒ये न्द्रा॑य॒ तम् जु॑होमि जुहोमि॒ त मिन्द्रा॒ये न्द्रा॑य॒ तम् जु॑होमि । \newline
46. तम् जु॑होमि जुहोमि॒ तम् तम् जु॑होमि । \newline
47. जु॒हो॒मीति॑ जुहोमि । \newline
48. यो द्र॒फ्सो द्र॒फ्सो यो यो द्र॒फ्सो अꣳ॒॒शु रꣳ॒॒शुर् द्र॒फ्सो यो यो द्र॒फ्सो अꣳ॒॒शुः । \newline
49. द्र॒फ्सो अꣳ॒॒शु रꣳ॒॒शुर् द्र॒फ्सो द्र॒फ्सो अꣳ॒॒शुः प॑ति॒तः प॑ति॒तो अꣳ॒॒शुर् द्र॒फ्सो द्र॒फ्सो अꣳ॒॒शुः प॑ति॒तः । \newline
50. अꣳ॒॒शुः प॑ति॒तः प॑ति॒तो अꣳ॒॒शु रꣳ॒॒शुः प॑ति॒तः पृ॑थि॒व्याम् पृ॑थि॒व्याम् प॑ति॒तो अꣳ॒॒शु रꣳ॒॒शुः प॑ति॒तः पृ॑थि॒व्याम् । \newline
51. प॒ति॒तः पृ॑थि॒व्याम् पृ॑थि॒व्याम् प॑ति॒तः प॑ति॒तः पृ॑थि॒व्याम् प॑रिवा॒पात् प॑रिवा॒पात् पृ॑थि॒व्याम् प॑ति॒तः प॑ति॒तः पृ॑थि॒व्याम् प॑रिवा॒पात् । \newline
52. पृ॒थि॒व्याम् प॑रिवा॒पात् प॑रिवा॒पात् पृ॑थि॒व्याम् पृ॑थि॒व्याम् प॑रिवा॒पात् पु॑रो॒डाशा᳚त् पुरो॒डाशा᳚त् परिवा॒पात् पृ॑थि॒व्याम् पृ॑थि॒व्याम् प॑रिवा॒पात् पु॑रो॒डाशा᳚त् । \newline
53. प॒रि॒वा॒पात् पु॑रो॒डाशा᳚त् पुरो॒डाशा᳚त् परिवा॒पात् प॑रिवा॒पात् पु॑रो॒डाशा᳚त् कर॒म्भात् क॑र॒म्भात् पु॑रो॒डाशा᳚त् परिवा॒पात् प॑रिवा॒पात् पु॑रो॒डाशा᳚त् कर॒म्भात् । \newline
54. प॒रि॒वा॒पादिति॑ परि - वा॒पात् । \newline
\pagebreak
\markright{ TS 3.1.10.2  \hfill https://www.vedavms.in \hfill}

\section{ TS 3.1.10.2 }

\textbf{TS 3.1.10.2 } \newline
\textbf{Samhita Paata} \newline

पु॑रो॒डाशा᳚त् कर॒म्भात् । धा॒ना॒सो॒मान्म॒न्थिन॑ इन्द्र शु॒क्राथ् स्वाहा॑कृत॒मिन्द्रा॑य॒ तं जु॑होमि ॥यस्ते᳚ द्र॒फ्सो मधु॑माꣳ इन्द्रि॒यावा॒न्थ् स्वाहा॑कृतः॒ पुन॑र॒प्येति॑ दे॒वान् । दि॒वः पृ॑थि॒व्याः पर्य॒न्तरि॑क्षा॒थ् स्वाहा॑ कृत॒मिन्द्रा॑य॒ तं जु॑होमि ॥ अ॒द्ध्व॒र्युर्वा ऋ॒त्विजां᳚ प्रथ॒मो यु॑ज्यते॒ तेन॒ स्तोमो॑ योक्त॒व्य॑ इत्या॑हु॒र्वाग॑ग्रे॒गा अग्र॑ एत्वृजु॒गा दे॒वेभ्यो॒ यशो॒ मयि॒ दध॑ती प्रा॒णान् प॒शुषु॑ प्र॒जां मयि॑- [  ] \newline

\textbf{Pada Paata} \newline

पु॒रो॒डाशा᳚त् । क॒र॒म्भात् ॥ धा॒ना॒सो॒मादिति॑ धाना-सो॒मात् । म॒न्थिनः॑ । इ॒न्द्र॒ । शु॒क्रात् । स्वाहा॑कृत॒मिति॒ स्वाहा᳚ - कृ॒त॒म् । इन्द्रा॑य । तम् । जु॒हो॒मि॒ ॥ यः । ते॒ । द्र॒फ्सः । मधु॑मा॒निति॒ मधु॑ - मा॒न् । इ॒न्द्रि॒यावा॒निती᳚न्द्रि॒य - वा॒न् । स्वाहा॑कृत॒ इति स्वाहा᳚-कृ॒तः॒ । पुनः॑ । अ॒प्येतीत्य॑पि - एति॑ । दे॒वान् ॥ दि॒वः । पृ॒थि॒व्याः । परीति॑ । अ॒न्तरि॑क्षात् । स्वाहा॑कृत॒मिति॒ स्वाहा᳚ - कृ॒त॒म् । इन्द्रा॑य । तम् । जु॒हो॒मि॒ ॥ अ॒द्ध्व॒र्युः । वै । ऋ॒त्विजा᳚म् । प्र॒थ॒मः । यु॒ज्य॒ते॒ । तेन॑ । स्तोमः॑ । यो॒क्त॒व्यः॑ । इति॑ । आ॒हुः॒ । वाक् । अ॒ग्रे॒गा इत्य॑ग्रे - गाः । अग्र᳚ । ए॒तु॒ । ऋ॒जु॒गा इत्यृ॑जु - गाः । दे॒वेभ्यः॑ । यशः॑ । मयि॑ । दध॑ती । प्रा॒णानिति॑ प्र - अ॒नान् । प॒शुषु॑ । प्र॒जामिति॑ प्र - जाम् । मयि॑ ।  \newline


\textbf{Krama Paata} \newline

पु॒रो॒डाशा᳚त् कर॒म्भात् । क॒र॒म्भादिति॑ कर॒म्भात् ॥ धा॒ना॒सो॒मान् म॒न्थिनः॑ । धा॒ना॒सो॒मादिति॑ धाना - सो॒मात् । म॒न्थिन॑ इन्द्र । इ॒न्द्र॒ शु॒क्रात् । शु॒क्राथ् स्वाहा॑कृतम् । स्वाहा॑कृत॒मिन्द्रा॑य । स्वाहा॑कृत॒मिति॒ स्वाहा᳚ - कृ॒त॒म् । इन्द्रा॑य॒ तम् । तम् जु॑होमि । जु॒हो॒मीति॑ जुहोमि ॥ यस्ते᳚ । ते॒ द्र॒फ्सः । द्र॒फ्सो मधु॑मान् । मधु॑माꣳ इन्द्रि॒यावान्॑ । मधु॑मा॒निति॒ मधु॑ - मा॒न्॒ । इ॒न्द्रि॒यावा॒न्थ् स्वाहा॑कृतः । इ॒न्द्रि॒यावा॒निती᳚न्द्रि॒य - वा॒न्॒ । स्वाहा॑कृतः॒ पुनः॑ । स्वाहा॑कृत॒ इति॒ स्वाहा᳚ - कृ॒तः॒ । पुन॑र॒प्येति॑ । अ॒प्येति॑ दे॒वान् । अ॒प्येतीत्य॑पि - एति॑ । दे॒वानिति॑ दे॒वान् ॥ दि॒वः पृ॑थि॒व्याः । पृ॒थि॒व्याः परि॑ । पर्य॒न्तरि॑क्षात् । अ॒न्तरि॑क्षा॒थ् स्वाहा॑कृतम् । स्वाहा॑कृत॒मिन्द्रा॑य । स्वाहा॑कृत॒मिति॒ स्वाहा᳚ - कृ॒त॒म् । इन्द्रा॑य॒ तम् । तम् जु॑होमि । जु॒हो॒मीति॑ जुहोमि ॥ अ॒द्ध्व॒र्युर् वै । वा ऋ॒त्विजा᳚म् । ऋ॒त्विजा᳚म् प्रथ॒मः । प्र॒थ॒मो यु॑ज्यते । यु॒ज्य॒ते॒ तेन॑ । तेन॒ स्तोमः॑ । स्तोमो॑ योक्त॒व्यः॑ । यो॒क्त॒व्य॑ इति॑ । इत्या॑हुः । आ॒हु॒र् वाक् । वाग॑ग्रे॒गाः । अ॒ग्रे॒गा अग्रे᳚ । अ॒ग्रे॒गा इत्य॑ग्रे - गाः । अग्र॑ एतु । ए॒त्वृ॒जु॒गाः । ऋ॒जु॒गा दे॒वेभ्यः॑ । ऋ॒जु॒गा इत्यृ॑जु - गाः । दे॒वेभ्यो॒ यशः॑ । यशो॒ मयि॑ । मयि॒ दध॑ती । दध॑ती प्रा॒णान् । प्रा॒णान् प॒शुषु॑ । प्रा॒णानिति॑ प्र - अ॒नान् । प॒शुषु॑ प्र॒जाम् । प्र॒जाम् मयि॑ । प्र॒जामिति॑ प्र - जाम् । मयि॑ च \newline

\textbf{Jatai Paata} \newline

1. पु॒रो॒डाशा᳚त् कर॒म्भात् क॑र॒म्भात् पु॑रो॒डाशा᳚त् पुरो॒डाशा᳚त् कर॒म्भात् । \newline
2. क॒र॒म्भादिति॑ कर॒म्भात् । \newline
3. धा॒ना॒सो॒मान् म॒न्थिनो॑ म॒न्थिनो॑ धानासो॒माद् धा॑नासो॒मान् म॒न्थिनः॑ । \newline
4. धा॒ना॒सो॒मादिति॑ धाना - सो॒मात् । \newline
5. म॒न्थिन॑ इन्द्रे न्द्र म॒न्थिनो॑ म॒न्थिन॑ इन्द्र । \newline
6. इ॒न्द्र॒ शु॒क्राच् छु॒क्रा दि॑न्द्रे न्द्र शु॒क्रात् । \newline
7. शु॒क्राथ् स्वाहा॑कृतꣳ॒॒ स्वाहा॑कृतꣳ शु॒क्राच् छु॒क्राथ् स्वाहा॑कृतम् । \newline
8. स्वाहा॑कृत॒ मिन्द्रा॒ये न्द्रा॑य॒ स्वाहा॑कृतꣳ॒॒ स्वाहा॑कृत॒ मिन्द्रा॑य । \newline
9. स्वाहा॑कृत॒मिति॒ स्वाहा᳚ - कृ॒त॒म् । \newline
10. इन्द्रा॑य॒ तम् त मिन्द्रा॒ये न्द्रा॑य॒ तम् । \newline
11. तम् जु॑होमि जुहोमि॒ तम् तम् जु॑होमि । \newline
12. जु॒हो॒मीति॑ जुहोमि । \newline
13. यस्ते॑ ते॒ यो यस्ते᳚ । \newline
14. ते॒ द्र॒फ्सो द्र॒फ्स स्ते॑ ते द्र॒फ्सः । \newline
15. द्र॒फ्सो मधु॑मा॒न् मधु॑मान् द्र॒फ्सो द्र॒फ्सो मधु॑मान् । \newline
16. मधु॑माꣳ इन्द्रि॒यावा॑ निन्द्रि॒यावा॒न् मधु॑मा॒न् मधु॑माꣳ इन्द्रि॒यावान्॑ । \newline
17. मधु॑मा॒निति॒ मधु॑ - मा॒न् । \newline
18. इ॒न्द्रि॒यावा॒न् थ्स्वाहा॑कृतः॒ स्वाहा॑कृत इन्द्रि॒यावा॑ निन्द्रि॒यावा॒न् थ्स्वाहा॑कृतः । \newline
19. इ॒न्द्रि॒यावा॒निती᳚न्द्रि॒य - वा॒न् । \newline
20. स्वाहा॑कृतः॒ पुनः॒ पुनः॒ स्वाहा॑कृतः॒ स्वाहा॑कृतः॒ पुनः॑ । \newline
21. स्वाहा॑कृत॒ इति॒ स्वाहा᳚ - कृ॒तः॒ । \newline
22. पुन॑ र॒प्ये त्य॒प्येति॒ पुनः॒ पुन॑ र॒प्येति॑ । \newline
23. अ॒प्येति॑ दे॒वान् दे॒वा न॒प्ये त्य॒प्येति॑ दे॒वान् । \newline
24. अ॒प्येतीत्य॑पि - एति॑ । \newline
25. दे॒वानिति॑ दे॒वान् । \newline
26. दि॒वः पृ॑थि॒व्याः पृ॑थि॒व्या दि॒वो दि॒वः पृ॑थि॒व्याः । \newline
27. पृ॒थि॒व्याः परि॒ परि॑ पृथि॒व्याः पृ॑थि॒व्याः परि॑ । \newline
28. पर्य॒न्तरि॑क्षा द॒न्तरि॑क्षा॒त् परि॒ पर्य॒न्तरि॑क्षात् । \newline
29. अ॒न्तरि॑क्षा॒थ् स्वाहा॑कृतꣳ॒॒ स्वाहा॑कृत म॒न्तरि॑क्षा द॒न्तरि॑क्षा॒थ् स्वाहा॑कृतम् । \newline
30. स्वाहा॑कृत॒ मिन्द्रा॒ये न्द्रा॑य॒ स्वाहा॑कृतꣳ॒॒ स्वाहा॑कृत॒ मिन्द्रा॑य । \newline
31. स्वाहा॑कृत॒मिति॒ स्वाहा᳚ - कृ॒त॒म् । \newline
32. इन्द्रा॑य॒ तम् त मिन्द्रा॒ये न्द्रा॑य॒ तम् । \newline
33. तम् जु॑होमि जुहोमि॒ तम् तम् जु॑होमि । \newline
34. जु॒हो॒मीति॑ जुहोमि । \newline
35. अ॒द्ध्व॒र्युर् वै वा अ॑द्ध्व॒र्यु र॑द्ध्व॒र्युर् वै । \newline
36. वा ऋ॒त्विजा॑ मृ॒त्विजां॒ ॅवै वा ऋ॒त्विजा᳚म् । \newline
37. ऋ॒त्विजा᳚म् प्रथ॒मः प्र॑थ॒म ऋ॒त्विजा॑ मृ॒त्विजा᳚म् प्रथ॒मः । \newline
38. प्र॒थ॒मो यु॑ज्यते युज्यते प्रथ॒मः प्र॑थ॒मो यु॑ज्यते । \newline
39. यु॒ज्य॒ते॒ तेन॒ तेन॑ युज्यते युज्यते॒ तेन॑ । \newline
40. तेन॒ स्तोमः॒ स्तोम॒ स्तेन॒ तेन॒ स्तोमः॑ । \newline
41. स्तोमो॑ योक्त॒व्यो॑ योक्त॒व्यः॑ स्तोमः॒ स्तोमो॑ योक्त॒व्यः॑ । \newline
42. यो॒क्त॒व्य॑ इतीति॑ योक्त॒व्यो॑ योक्त॒व्य॑ इति॑ । \newline
43. इत्या॑हु राहु॒ रितीत्या॑हुः । \newline
44. आ॒हु॒र् वाग् वागा॑हु राहु॒र् वाक् । \newline
45. वाग॑ग्रे॒गा अ॑ग्रे॒गा वाग् वाग॑ग्रे॒गाः । \newline
46. अ॒ग्रे॒गा अग्रे॒ अग्रे॑ अग्रे॒गा अ॑ग्रे॒गा अग्रे᳚ । \newline
47. अ॒ग्रे॒गा इत्य॑ग्रे - गाः । \newline
48. अग्र॑ एत्वे॒ त्वग्रे॒ अग्र॑ एतु । \newline
49. ए॒त्वृ॒जु॒गा ऋ॑जु॒गा ए᳚त्वे त्वृजु॒गाः । \newline
50. ऋ॒जु॒गा दे॒वेभ्यो॑ दे॒वेभ्य॑ ऋजु॒गा ऋ॑जु॒गा दे॒वेभ्यः॑ । \newline
51. ऋ॒जु॒गा इत्यृ॑जु - गाः । \newline
52. दे॒वेभ्यो॒ यशो॒ यशो॑ दे॒वेभ्यो॑ दे॒वेभ्यो॒ यशः॑ । \newline
53. यशो॒ मयि॒ मयि॒ यशो॒ यशो॒ मयि॑ । \newline
54. मयि॒ दध॑ती॒ दध॑ती॒ मयि॒ मयि॒ दध॑ती । \newline
55. दध॑ती प्रा॒णान् प्रा॒णान् दध॑ती॒ दध॑ती प्रा॒णान् । \newline
56. प्रा॒णान् प॒शुषु॑ प॒शुषु॑ प्रा॒णान् प्रा॒णान् प॒शुषु॑ । \newline
57. प्रा॒णानिति॑ प्र - अ॒नान् । \newline
58. प॒शुषु॑ प्र॒जाम् प्र॒जाम् प॒शुषु॑ प॒शुषु॑ प्र॒जाम् । \newline
59. प्र॒जाम् मयि॒ मयि॑ प्र॒जाम् प्र॒जाम् मयि॑ । \newline
60. प्र॒जामिति॑ प्र - जाम् । \newline
61. मयि॑ च च॒ मयि॒ मयि॑ च । \newline

\textbf{Ghana Paata } \newline

1. पु॒रो॒डाशा᳚त् कर॒म्भात् क॑र॒म्भात् पु॑रो॒डाशा᳚त् पुरो॒डाशा᳚त् कर॒म्भात् । \newline
2. क॒र॒म्भादिति॑ कर॒म्भात् । \newline
3. धा॒ना॒सो॒मान् म॒न्थिनो॑ म॒न्थिनो॑ धानासो॒माद् धा॑नासो॒मान् म॒न्थिन॑ इन्द्रे न्द्र म॒न्थिनो॑ धानासो॒माद् धा॑नासो॒मान् म॒न्थिन॑ इन्द्र । \newline
4. धा॒ना॒सो॒मादिति॑ धाना - सो॒मात् । \newline
5. म॒न्थिन॑ इन्द्रे न्द्र म॒न्थिनो॑ म॒न्थिन॑ इन्द्र शु॒क्राच् छु॒क्रा दि॑न्द्र म॒न्थिनो॑ म॒न्थिन॑ इन्द्र शु॒क्रात् । \newline
6. इ॒न्द्र॒ शु॒क्राच् छु॒क्रा दि॑न्द्रे न्द्र शु॒क्राथ् स्वाहा॑कृतꣳ॒॒ स्वाहा॑कृतꣳ शु॒क्रा दि॑न्द्रे न्द्र शु॒क्राथ् स्वाहा॑कृतम् । \newline
7. शु॒क्राथ् स्वाहा॑कृतꣳ॒॒ स्वाहा॑कृतꣳ शु॒क्राच् छु॒क्राथ् स्वाहा॑कृत॒ मिन्द्रा॒ये न्द्रा॑य॒ स्वाहा॑कृतꣳ शु॒क्राच् छु॒क्राथ् स्वाहा॑कृत॒ मिन्द्रा॑य । \newline
8. स्वाहा॑कृत॒ मिन्द्रा॒ये न्द्रा॑य॒ स्वाहा॑कृतꣳ॒॒ स्वाहा॑कृत॒ मिन्द्रा॑य॒ तम् त मिन्द्रा॑य॒ स्वाहा॑कृतꣳ॒॒ स्वाहा॑कृत॒ मिन्द्रा॑य॒ तम् । \newline
9. स्वाहा॑कृत॒मिति॒ स्वाहा᳚ - कृ॒त॒म् । \newline
10. इन्द्रा॑य॒ तम् त मिन्द्रा॒ये न्द्रा॑य॒ तम् जु॑होमि जुहोमि॒ त मिन्द्रा॒ये न्द्रा॑य॒ तम् जु॑होमि । \newline
11. तम् जु॑होमि जुहोमि॒ तम् तम् जु॑होमि । \newline
12. जु॒हो॒मीति॑ जुहोमि । \newline
13. यस्ते॑ ते॒ यो यस्ते᳚ द्र॒फ्सो द्र॒फ्स स्ते॒ यो यस्ते᳚ द्र॒फ्सः । \newline
14. ते॒ द्र॒फ्सो द्र॒फ्स स्ते॑ ते द्र॒फ्सो मधु॑मा॒न् मधु॑मान् द्र॒फ्स स्ते॑ ते द्र॒फ्सो मधु॑मान् । \newline
15. द्र॒फ्सो मधु॑मा॒न् मधु॑मान् द्र॒फ्सो द्र॒फ्सो मधु॑माꣳ इन्द्रि॒यावा॑ निन्द्रि॒यावा॒न् मधु॑मान् द्र॒फ्सो द्र॒फ्सो मधु॑माꣳ इन्द्रि॒यावान्॑ । \newline
16. मधु॑माꣳ इन्द्रि॒यावा॑ निन्द्रि॒यावा॒न् मधु॑मा॒न् मधु॑माꣳ इन्द्रि॒यावा॒न् थ्स्वाहा॑कृतः॒ स्वाहा॑कृत इन्द्रि॒यावा॒न् मधु॑मा॒न् मधु॑माꣳ इन्द्रि॒यावा॒न् थ्स्वाहा॑कृतः । \newline
17. मधु॑मा॒निति॒ मधु॑ - मा॒न् । \newline
18. इ॒न्द्रि॒यावा॒न् थ्स्वाहा॑कृतः॒ स्वाहा॑कृत इन्द्रि॒यावा॑ निन्द्रि॒यावा॒न् थ्स्वाहा॑कृतः॒ पुनः॒ पुनः॒ स्वाहा॑कृत इन्द्रि॒यावा॑ निन्द्रि॒यावा॒न् थ्स्वाहा॑कृतः॒ पुनः॑ । \newline
19. इ॒न्द्रि॒यावा॒निती᳚न्द्रि॒य - वा॒न् । \newline
20. स्वाहा॑कृतः॒ पुनः॒ पुनः॒ स्वाहा॑कृतः॒ स्वाहा॑कृतः॒ पुन॑ र॒प्ये त्य॒प्येति॒ पुनः॒ स्वाहा॑कृतः॒ स्वाहा॑कृतः॒ पुन॑ र॒प्येति॑ । \newline
21. स्वाहा॑कृत॒ इति॒ स्वाहा᳚ - कृ॒तः॒ । \newline
22. पुन॑ र॒प्ये त्य॒प्येति॒ पुनः॒ पुन॑ र॒प्येति॑ दे॒वान् दे॒वा न॒प्येति॒ पुनः॒ पुन॑ र॒प्येति॑ दे॒वान् । \newline
23. अ॒प्येति॑ दे॒वान् दे॒वा न॒प्ये त्य॒प्येति॑ दे॒वान् । \newline
24. अ॒प्येतीत्य॑पि - एति॑ । \newline
25. दे॒वानिति॑ दे॒वान् । \newline
26. दि॒वः पृ॑थि॒व्याः पृ॑थि॒व्या दि॒वो दि॒वः पृ॑थि॒व्याः परि॒ परि॑ पृथि॒व्या दि॒वो दि॒वः पृ॑थि॒व्याः परि॑ । \newline
27. पृ॒थि॒व्याः परि॒ परि॑ पृथि॒व्याः पृ॑थि॒व्याः पर्य॒न्तरि॑क्षा द॒न्तरि॑क्षा॒त् परि॑ पृथि॒व्याः पृ॑थि॒व्याः पर्य॒न्तरि॑क्षात् । \newline
28. पर्य॒न्तरि॑क्षा द॒न्तरि॑क्षा॒त् परि॒ पर्य॒न्तरि॑क्षा॒थ् स्वाहा॑कृतꣳ॒॒ स्वाहा॑कृत म॒न्तरि॑क्षा॒त् परि॒ पर्य॒न्तरि॑क्षा॒थ् स्वाहा॑कृतम् । \newline
29. अ॒न्तरि॑क्षा॒थ् स्वाहा॑कृतꣳ॒॒ स्वाहा॑कृत म॒न्तरि॑क्षा द॒न्तरि॑क्षा॒थ् स्वाहा॑कृत॒ मिन्द्रा॒ये न्द्रा॑य॒ स्वाहा॑कृत म॒न्तरि॑क्षा द॒न्तरि॑क्षा॒थ् स्वाहा॑कृत॒ मिन्द्रा॑य । \newline
30. स्वाहा॑कृत॒ मिन्द्रा॒ये न्द्रा॑य॒ स्वाहा॑कृतꣳ॒॒ स्वाहा॑कृत॒ मिन्द्रा॑य॒ तम् त मिन्द्रा॑य॒ स्वाहा॑कृतꣳ॒॒ स्वाहा॑कृत॒ मिन्द्रा॑य॒ तम् । \newline
31. स्वाहा॑कृत॒मिति॒ स्वाहा᳚ - कृ॒त॒म् । \newline
32. इन्द्रा॑य॒ तम् त मिन्द्रा॒ये न्द्रा॑य॒ तम् जु॑होमि जुहोमि॒ त मिन्द्रा॒ये न्द्रा॑य॒ तम् जु॑होमि । \newline
33. तम् जु॑होमि जुहोमि॒ तम् तम् जु॑होमि । \newline
34. जु॒हो॒मीति॑ जुहोमि । \newline
35. अ॒द्ध्व॒र्युर् वै वा अ॑द्ध्व॒र्यु र॑द्ध्व॒र्युर् वा ऋ॒त्विजा॑ मृ॒त्विजां॒ ॅवा अ॑द्ध्व॒र्यु र॑द्ध्व॒र्युर् वा ऋ॒त्विजा᳚म् । \newline
36. वा ऋ॒त्विजा॑ मृ॒त्विजां॒ ॅवै वा ऋ॒त्विजा᳚म् प्रथ॒मः प्र॑थ॒म ऋ॒त्विजां॒ ॅवै वा ऋ॒त्विजा᳚म् प्रथ॒मः । \newline
37. ऋ॒त्विजा᳚म् प्रथ॒मः प्र॑थ॒म ऋ॒त्विजा॑ मृ॒त्विजा᳚म् प्रथ॒मो यु॑ज्यते युज्यते प्रथ॒म ऋ॒त्विजा॑ मृ॒त्विजा᳚म् प्रथ॒मो यु॑ज्यते । \newline
38. प्र॒थ॒मो यु॑ज्यते युज्यते प्रथ॒मः प्र॑थ॒मो यु॑ज्यते॒ तेन॒ तेन॑ युज्यते प्रथ॒मः प्र॑थ॒मो यु॑ज्यते॒ तेन॑ । \newline
39. यु॒ज्य॒ते॒ तेन॒ तेन॑ युज्यते युज्यते॒ तेन॒ स्तोमः॒ स्तोम॒ स्तेन॑ युज्यते युज्यते॒ तेन॒ स्तोमः॑ । \newline
40. तेन॒ स्तोमः॒ स्तोम॒ स्तेन॒ तेन॒ स्तोमो॑ योक्त॒व्यो॑ योक्त॒व्यः॑ स्तोम॒ स्तेन॒ तेन॒ स्तोमो॑ योक्त॒व्यः॑ । \newline
41. स्तोमो॑ योक्त॒व्यो॑ योक्त॒व्यः॑ स्तोमः॒ स्तोमो॑ योक्त॒व्य॑ इतीति॑ योक्त॒व्यः॑ स्तोमः॒ स्तोमो॑ योक्त॒व्य॑ इति॑ । \newline
42. यो॒क्त॒व्य॑ इतीति॑ योक्त॒व्यो॑ योक्त॒व्य॑ इत्या॑हु राहु॒रिति॑ योक्त॒व्यो॑ योक्त॒व्य॑ इत्या॑हुः । \newline
43. इत्या॑हु राहु॒ रितीत्या॑हु॒र् वाग् वागा॑हु॒ रितीत्या॑हु॒र् वाक् । \newline
44. आ॒हु॒र् वाग् वागा॑हु राहु॒र् वाग॑ग्रे॒गा अ॑ग्रे॒गा वागा॑हु राहु॒र् वाग॑ग्रे॒गाः । \newline
45. वाग॑ग्रे॒गा अ॑ग्रे॒गा वाग् वाग॑ग्रे॒गा अग्रे॒ अग्रे॑ अग्रे॒गा वाग् वाग॑ग्रे॒गा अग्रे᳚ । \newline
46. अ॒ग्रे॒गा अग्रे॒ अग्रे॑ अग्रे॒गा अ॑ग्रे॒गा अग्र॑ एत्वे॒ त्वग्रे॑ अग्रे॒गा अ॑ग्रे॒गा अग्र॑ एतु । \newline
47. अ॒ग्रे॒गा इत्य॑ग्रे - गाः । \newline
48. अग्र॑ एत्वे॒ त्वग्रे॒ अग्र॑ एत्वृजु॒गा ऋ॑जु॒गा ए॒त्वग्रे॒ अग्र॑ एत्वृजु॒गाः । \newline
49. ए॒त्वृ॒जु॒गा ऋ॑जु॒गा ए᳚त्वे त्वृजु॒गा दे॒वेभ्यो॑ दे॒वेभ्य॑ ऋजु॒गा ए᳚त्वे त्वृजु॒गा दे॒वेभ्यः॑ । \newline
50. ऋ॒जु॒गा दे॒वेभ्यो॑ दे॒वेभ्य॑ ऋजु॒गा ऋ॑जु॒गा दे॒वेभ्यो॒ यशो॒ यशो॑ दे॒वेभ्य॑ ऋजु॒गा ऋ॑जु॒गा दे॒वेभ्यो॒ यशः॑ । \newline
51. ऋ॒जु॒गा इत्यृ॑जु - गाः । \newline
52. दे॒वेभ्यो॒ यशो॒ यशो॑ दे॒वेभ्यो॑ दे॒वेभ्यो॒ यशो॒ मयि॒ मयि॒ यशो॑ दे॒वेभ्यो॑ दे॒वेभ्यो॒ यशो॒ मयि॑ । \newline
53. यशो॒ मयि॒ मयि॒ यशो॒ यशो॒ मयि॒ दध॑ती॒ दध॑ती॒ मयि॒ यशो॒ यशो॒ मयि॒ दध॑ती । \newline
54. मयि॒ दध॑ती॒ दध॑ती॒ मयि॒ मयि॒ दध॑ती प्रा॒णान् प्रा॒णान् दध॑ती॒ मयि॒ मयि॒ दध॑ती प्रा॒णान् । \newline
55. दध॑ती प्रा॒णान् प्रा॒णान् दध॑ती॒ दध॑ती प्रा॒णान् प॒शुषु॑ प॒शुषु॑ प्रा॒णान् दध॑ती॒ दध॑ती प्रा॒णान् प॒शुषु॑ । \newline
56. प्रा॒णान् प॒शुषु॑ प॒शुषु॑ प्रा॒णान् प्रा॒णान् प॒शुषु॑ प्र॒जाम् प्र॒जाम् प॒शुषु॑ प्रा॒णान् प्रा॒णान् प॒शुषु॑ प्र॒जाम् । \newline
57. प्रा॒णानिति॑ प्र - अ॒नान् । \newline
58. प॒शुषु॑ प्र॒जाम् प्र॒जाम् प॒शुषु॑ प॒शुषु॑ प्र॒जाम् मयि॒ मयि॑ प्र॒जाम् प॒शुषु॑ प॒शुषु॑ प्र॒जाम् मयि॑ । \newline
59. प्र॒जाम् मयि॒ मयि॑ प्र॒जाम् प्र॒जाम् मयि॑ च च॒ मयि॑ प्र॒जाम् प्र॒जाम् मयि॑ च । \newline
60. प्र॒जामिति॑ प्र - जाम् । \newline
61. मयि॑ च च॒ मयि॒ मयि॑ च॒ यज॑माने॒ यज॑माने च॒ मयि॒ मयि॑ च॒ यज॑माने । \newline
\pagebreak
\markright{ TS 3.1.10.3  \hfill https://www.vedavms.in \hfill}

\section{ TS 3.1.10.3 }

\textbf{TS 3.1.10.3 } \newline
\textbf{Samhita Paata} \newline

च॒ यज॑माने॒ चेत्या॑ह॒ वाच॑मे॒व तद्य॑ज्ञ्मु॒खे यु॑नक्ति॒ वास्तु॒ वा ए॒तद्य॒ज्ञ्स्य॑ क्रियते॒ यद्ग्रहा᳚न् गृही॒त्वा ब॑हिष्पवमा॒नꣳ सर्प॑न्ति॒परा᳚ञ्चो॒ हि यन्ति॒ परा॑चीभिः स्तु॒वते॑ वैष्ण॒व्यर्चा पुन॒रेत्योप॑ तिष्ठते य॒ज्ञो वै विष्णु॑ र्य॒ज्ञ्मे॒वाक॒र्विष्णो॒ त्वन्नो॒ अन्त॑मः॒ शर्म॑ यच्छ सहन्त्य । प्र ते॒ धारा॑ मधु॒श्चुत॒ उथ्सं॑ दुह्रते॒ ( ) अक्षि॑त॒मित्या॑ह॒ यदे॒वास्य॒ शया॑नस्योप॒शुष्य॑ति॒ तदे॒वास्यै॒तेना ऽऽ*प्या॑ययति ॥ \newline

\textbf{Pada Paata} \newline

च॒ । यज॑माने । च॒ । इति॑ । आ॒ह॒ । वाच᳚म् । ए॒व । तत् । य॒ज्ञ्॒मु॒ख इति॑ यज्ञ् - मु॒खे । यु॒न॒क्ति॒ । वास्तु॑ । वै । ए॒तत् । य॒ज्ञ्स्य॑ । क्रि॒य॒ते॒ । यत् । ग्रहान्॑ । गृ॒ही॒त्वा । ब॒हि॒ष्प॒व॒मा॒नमिति॑ बहिः - प॒व॒मा॒नम् । सर्प॑न्ति । परा᳚ञ्चः । हि । यन्ति॑ । परा॑चीभिः । स्तु॒वते᳚ । वै॒ष्ण॒व्या । ऋ॒चा । पुनः॑ । एत्येत्या᳚ - इत्य॑ । उपेति॑ । ति॒ष्ठ॒ते॒ । य॒ज्ञ्ः । वै । विष्णुः॑ । य॒ज्ञ्म् । ए॒व । अ॒कः॒ । विष्णो॒ इति॑ । त्वम् । नः॒ । अन्त॑मः । शर्म॑ । य॒च्छ॒ । स॒ह॒न्त्य॒ ॥ प्रेति॑ । ते॒ । धाराः᳚ । म॒धु॒श्चुत॒ इति॑ मधु-श्चुतः॑ । उथ्स᳚म् । दु॒ह्र॒ते॒ ( ) । अक्षि॑तम् । इति॑ । आ॒ह॒ । यत् । ए॒व । अ॒स्य॒ । शया॑नस्य । उ॒प॒शुष्य॒तीत्यु॑प - शुष्य॑ति । तत् । ए॒व । अ॒स्य॒ । ए॒तेन॑ । एति॑ । प्या॒य॒य॒ति॒ ॥  \newline


\textbf{Krama Paata} \newline

च॒ यज॑माने । यज॑माने च । चेति॑ । इत्या॑ह । आ॒ह॒ वाच᳚म् । वाच॑मे॒व । ए॒व तत् । तद् य॑ज्ञ्मु॒खे । य॒ज्ञ्॒मु॒खे यु॑नक्ति । य॒ज्ञ्॒मु॒ख इति॑ यज्ञ् - मु॒खे । यु॒न॒क्ति॒ वास्तु॑ । वास्तु॒ वै । वा ए॒तत् । ए॒तद् य॒ज्ञ्स्य॑ । य॒ज्ञ्स्य॑ क्रियते । क्रि॒य॒ते॒ यत् । यद् ग्रहान्॑ । ग्रहा᳚न् गृही॒त्वा । गृ॒ही॒त्वा ब॑हिष्पवमा॒नम् । ब॒हि॒ष्प॒व॒मा॒नꣳ सर्प॑न्ति । ब॒हि॒ष्प॒व॒मा॒नमिति॑ बहिः - प॒व॒मा॒नम् । सर्प॑न्ति॒ परा᳚ञ्चः । परा᳚ञ्चो॒ हि । हि यन्ति॑ । यन्ति॒ परा॑चीभिः । परा॑चीभिः स्तु॒वते᳚ । स्तु॒वते॑ वैष्ण॒व्या । वै॒ष्ण॒व्यर्चा । ऋ॒चा पुनः॑ । पुन॒रेत्य॑ । एत्योप॑ । एत्येत्या᳚ - इत्य॑ । उप॑ तिष्ठते । ति॒ष्ठ॒ते॒ य॒ज्ञ्ः । य॒ज्ञो वै । वै विष्णुः॑ । विष्णु॑र् य॒ज्ञ्म् । य॒ज्ञ्मे॒व । ए॒वाकः॑ । अ॒क॒र् विष्णो᳚ । विष्णो॒ त्वम् । विष्णो॒ इति॒ विष्णो᳚ । त्वम् नः॑ । नो॒ अन्त॑मः । अन्त॑मः॒ शर्म॑ । शर्म॑ यच्छ । य॒च्छ॒ स॒ह॒न्त्य॒ । स॒ह॒न्त्येति॑ सहन्त्य ॥ प्र ते᳚ । ते॒ धाराः᳚ । धारा॑ मधु॒श्चुतः॑ । म॒धु॒श्चुत॒ उथ्स᳚म् । म॒धु॒श्चुत॒ इति॑ मधु - श्चुतः॑ । उथ्स॑म् दुह्रते ( ) । दु॒ह्र॒ते॒ अक्षि॑तम् । अक्षि॑त॒मिति॑ । इत्या॑ह । आ॒ह॒ यत् । यदे॒व । ए॒वास्य॑ । अ॒स्य॒ शया॑नस्य । शया॑नस्योप॒शुष्य॑ति । उ॒प॒शुष्य॑ति॒ तत् । उ॒प॒शुष्य॒तीत्यु॑प - शुष्य॑ति । तदे॒व । ए॒वास्य॑ । अ॒स्यै॒तेन॑ । ए॒तेना । आ प्या॑ययति । प्या॒य॒य॒तीति॑ प्याययति । \newline

\textbf{Jatai Paata} \newline

1. च॒ यज॑माने॒ यज॑माने च च॒ यज॑माने । \newline
2. यज॑माने च च॒ यज॑माने॒ यज॑माने च । \newline
3. चे तीति॑ च॒ चे ति॑ । \newline
4. इत्या॑हा॒हे तीत्या॑ह । \newline
5. आ॒ह॒ वाचं॒ ॅवाच॑ माहाह॒ वाच᳚म् । \newline
6. वाच॑ मे॒वैव वाचं॒ ॅवाच॑ मे॒व । \newline
7. ए॒व तत् तदे॒वैव तत् । \newline
8. तद् य॑ज्ञ्मु॒खे य॑ज्ञ्मु॒खे तत् तद् य॑ज्ञ्मु॒खे । \newline
9. य॒ज्ञ्॒मु॒खे यु॑नक्ति युनक्ति यज्ञ्मु॒खे य॑ज्ञ्मु॒खे यु॑नक्ति । \newline
10. य॒ज्ञ्॒मु॒ख इति॑ यज्ञ् - मु॒खे । \newline
11. यु॒न॒क्ति॒ वास्तु॒ वास्तु॑ युनक्ति युनक्ति॒ वास्तु॑ । \newline
12. वास्तु॒ वै वै वास्तु॒ वास्तु॒ वै । \newline
13. वा ए॒त दे॒तद् वै वा ए॒तत् । \newline
14. ए॒तद् य॒ज्ञ्स्य॑ य॒ज्ञ् स्यै॒त दे॒तद् य॒ज्ञ्स्य॑ । \newline
15. य॒ज्ञ्स्य॑ क्रियते क्रियते य॒ज्ञ्स्य॑ य॒ज्ञ्स्य॑ क्रियते । \newline
16. क्रि॒य॒ते॒ यद् यत् क्रि॑यते क्रियते॒ यत् । \newline
17. यद् ग्रहा॒न् ग्रहा॒न्॒. यद् यद् ग्रहान्॑ । \newline
18. ग्रहा᳚न् गृही॒त्वा गृ॑ही॒त्वा ग्रहा॒न् ग्रहा᳚न् गृही॒त्वा । \newline
19. गृ॒ही॒त्वा ब॑हिष्पवमा॒नम् ब॑हिष्पवमा॒नम् गृ॑ही॒त्वा गृ॑ही॒त्वा ब॑हिष्पवमा॒नम् । \newline
20. ब॒हि॒ष्प॒व॒मा॒नꣳ सर्प॑न्ति॒ सर्प॑न्ति बहिष्पवमा॒नम् ब॑हिष्पवमा॒नꣳ सर्प॑न्ति । \newline
21. ब॒हि॒ष्प॒व॒मा॒नमिति॑ बहिः - प॒व॒मा॒नम् । \newline
22. सर्प॑न्ति॒ परा᳚ञ्चः॒ परा᳚ञ्चः॒ सर्प॑न्ति॒ सर्प॑न्ति॒ परा᳚ञ्चः । \newline
23. परा᳚ञ्चो॒ हि हि परा᳚ञ्चः॒ परा᳚ञ्चो॒ हि । \newline
24. हि यन्ति॒ यन्ति॒ हि हि यन्ति॑ । \newline
25. यन्ति॒ परा॑चीभिः॒ परा॑चीभि॒र् यन्ति॒ यन्ति॒ परा॑चीभिः । \newline
26. परा॑चीभिः स्तु॒वते᳚ स्तु॒वते॒ परा॑चीभिः॒ परा॑चीभिः स्तु॒वते᳚ । \newline
27. स्तु॒वते॑ वैष्ण॒व्या वै᳚ष्ण॒व्या स्तु॒वते᳚ स्तु॒वते॑ वैष्ण॒व्या । \newline
28. वै॒ष्ण॒व्यर्चर्चा वै᳚ष्ण॒व्या वै᳚ष्ण॒व्यर्चा । \newline
29. ऋ॒चा पुनः॒ पुनर्॑. ऋ॒चर्चा पुनः॑ । \newline
30. पुन॒ रेत्येत्य॒ पुनः॒ पुन॒ रेत्य॑ । \newline
31. एत्यो पोपे त्येत्योप॑ । \newline
32. एत्येत्या᳚ - इत्य॑ । \newline
33. उप॑ तिष्ठते तिष्ठत॒ उपोप॑ तिष्ठते । \newline
34. ति॒ष्ठ॒ते॒ य॒ज्ञो य॒ज्ञ् स्ति॑ष्ठते तिष्ठते य॒ज्ञ्ः । \newline
35. य॒ज्ञो वै वै य॒ज्ञो य॒ज्ञो वै । \newline
36. वै विष्णु॒र् विष्णु॒र् वै वै विष्णुः॑ । \newline
37. विष्णु॑र् य॒ज्ञ्ं ॅय॒ज्ञ्ं ॅविष्णु॒र् विष्णु॑र् य॒ज्ञ्म् । \newline
38. य॒ज्ञ् मे॒वैव य॒ज्ञ्ं ॅय॒ज्ञ् मे॒व । \newline
39. ए॒वाक॑ रक रे॒वैवाकः॑ । \newline
40. अ॒क॒र् विष्णो॒ विष्णो॑ अक रक॒र् विष्णो᳚ । \newline
41. विष्णो॒ त्वम् त्वं ॅविष्णो॒ विष्णो॒ त्वम् । \newline
42. विष्णो॒ इति॒ विष्णो᳚ । \newline
43. त्वम् नो॑ न॒ स्त्वम् त्वम् नः॑ । \newline
44. नो॒ अन्त॑मो॒ अन्त॑मो नो नो॒ अन्त॑मः । \newline
45. अन्त॑मः॒ शर्म॒ शर्मान्त॑मो॒ अन्त॑मः॒ शर्म॑ । \newline
46. शर्म॑ यच्छ यच्छ॒ शर्म॒ शर्म॑ यच्छ । \newline
47. य॒च्छ॒ स॒ह॒न्त्य॒ स॒ह॒न्त्य॒ य॒च्छ॒ य॒च्छ॒ स॒ह॒न्त्य॒ । \newline
48. स॒ह॒ न्त्येति॑ सहन्त्य । \newline
49. प्र ते॑ ते॒ प्र प्र ते᳚ । \newline
50. ते॒ धारा॒ धारा᳚ स्ते ते॒ धाराः᳚ । \newline
51. धारा॑ मधु॒श्चुतो॑ मधु॒श्चुतो॒ धारा॒ धारा॑ मधु॒श्चुतः॑ । \newline
52. म॒धु॒श्चुत॒ उथ्स॒ मुथ्स॑म् मधु॒श्चुतो॑ मधु॒श्चुत॒ उथ्स᳚म् । \newline
53. म॒धु॒श्चुत॒ इति॑ मधु - श्चुतः॑ । \newline
54. उथ्स॑म् दुह्रते दुह्रत॒ उथ्स॒ मुथ्स॑म् दुह्रते । \newline
55. दु॒ह्र॒ते॒ अक्षि॑त॒ मक्षि॑तम् दुह्रते दुह्रते॒ अक्षि॑तम् । \newline
56. अक्षि॑त॒ मिती त्यक्षि॑त॒ मक्षि॑त॒ मिति॑ । \newline
57. इत्या॑हा॒हे तीत्या॑ह । \newline
58. आ॒ह॒ यद् यदा॑हाह॒ यत् । \newline
59. यदे॒वैव यद् यदे॒व । \newline
60. ए॒वास्या᳚ स्यै॒वै वास्य॑ । \newline
61. अ॒स्य॒ शया॑नस्य॒ शया॑नस्या स्यास्य॒ शया॑नस्य । \newline
62. शया॑न स्योप॒शुष्य॑ त्युप॒शुष्य॑ति॒ शया॑नस्य॒ शया॑न स्योप॒शुष्य॑ति । \newline
63. उ॒प॒शुष्य॑ति॒ तत् तदु॑प॒शुष्य॑ त्युप॒शुष्य॑ति॒ तत् । \newline
64. उ॒प॒शुष्य॒तीत्यु॑प - शुष्य॑ति । \newline
65. तदे॒वैव तत् तदे॒व । \newline
66. ए॒वास्या᳚ स्यै॒वैवास्य॑ । \newline
67. अ॒स्यै॒ तेनै॒तेना᳚ स्या स्यै॒तेन॑ । \newline
68. ए॒ते नैते नै॒तेना । \newline
69. आ प्या॑ययति प्यायय॒त्या प्या॑ययति । \newline
70. प्या॒य॒य॒तीति॑ प्याययति । \newline

\textbf{Ghana Paata } \newline

1. च॒ यज॑माने॒ यज॑माने च च॒ यज॑माने च च॒ यज॑माने च च॒ यज॑माने च । \newline
2. यज॑माने च च॒ यज॑माने॒ यज॑माने॒ चे तीति॑ च॒ यज॑माने॒ यज॑माने॒ चे ति॑ । \newline
3. चे तीति॑ च॒ चे त्या॑हा॒हे ति॑ च॒ चे त्या॑ह । \newline
4. इत्या॑हा॒हे तीत्या॑ह॒ वाचं॒ ॅवाच॑ मा॒हे तीत्या॑ह॒ वाच᳚म् । \newline
5. आ॒ह॒ वाचं॒ ॅवाच॑ माहाह॒ वाच॑ मे॒वैव वाच॑ माहाह॒ वाच॑ मे॒व । \newline
6. वाच॑ मे॒वैव वाचं॒ ॅवाच॑ मे॒व तत् तदे॒व वाचं॒ ॅवाच॑ मे॒व तत् । \newline
7. ए॒व तत् तदे॒वैव तद् य॑ज्ञ्मु॒खे य॑ज्ञ्मु॒खे तदे॒वैव तद् य॑ज्ञ्मु॒खे । \newline
8. तद् य॑ज्ञ्मु॒खे य॑ज्ञ्मु॒खे तत् तद् य॑ज्ञ्मु॒खे यु॑नक्ति युनक्ति यज्ञ्मु॒खे तत् तद् य॑ज्ञ्मु॒खे यु॑नक्ति । \newline
9. य॒ज्ञ्॒मु॒खे यु॑नक्ति युनक्ति यज्ञ्मु॒खे य॑ज्ञ्मु॒खे यु॑नक्ति॒ वास्तु॒ वास्तु॑ युनक्ति यज्ञ्मु॒खे य॑ज्ञ्मु॒खे यु॑नक्ति॒ वास्तु॑ । \newline
10. य॒ज्ञ्॒मु॒ख इति॑ यज्ञ् - मु॒खे । \newline
11. यु॒न॒क्ति॒ वास्तु॒ वास्तु॑ युनक्ति युनक्ति॒ वास्तु॒ वै वै वास्तु॑ युनक्ति युनक्ति॒ वास्तु॒ वै । \newline
12. वास्तु॒ वै वै वास्तु॒ वास्तु॒ वा ए॒त दे॒तद् वै वास्तु॒ वास्तु॒ वा ए॒तत् । \newline
13. वा ए॒त दे॒तद् वै वा ए॒तद् य॒ज्ञ्स्य॑ य॒ज्ञ्स्यै॒तद् वै वा ए॒तद् य॒ज्ञ्स्य॑ । \newline
14. ए॒तद् य॒ज्ञ्स्य॑ य॒ज्ञ्स्यै॒त दे॒तद् य॒ज्ञ्स्य॑ क्रियते क्रियते य॒ज्ञ्स्यै॒त दे॒तद् य॒ज्ञ्स्य॑ क्रियते । \newline
15. य॒ज्ञ्स्य॑ क्रियते क्रियते य॒ज्ञ्स्य॑ य॒ज्ञ्स्य॑ क्रियते॒ यद् यत् क्रि॑यते य॒ज्ञ्स्य॑ य॒ज्ञ्स्य॑ क्रियते॒ यत् । \newline
16. क्रि॒य॒ते॒ यद् यत् क्रि॑यते क्रियते॒ यद् ग्रहा॒न् ग्रहा॒न्॒. यत् क्रि॑यते क्रियते॒ यद् ग्रहान्॑ । \newline
17. यद् ग्रहा॒न् ग्रहा॒न्॒. यद् यद् ग्रहा᳚न् गृही॒त्वा गृ॑ही॒त्वा ग्रहा॒न्॒. यद् यद् ग्रहा᳚न् गृही॒त्वा । \newline
18. ग्रहा᳚न् गृही॒त्वा गृ॑ही॒त्वा ग्रहा॒न् ग्रहा᳚न् गृही॒त्वा ब॑हिष्पवमा॒नम् ब॑हिष्पवमा॒नम् गृ॑ही॒त्वा ग्रहा॒न् ग्रहा᳚न् गृही॒त्वा ब॑हिष्पवमा॒नम् । \newline
19. गृ॒ही॒त्वा ब॑हिष्पवमा॒नम् ब॑हिष्पवमा॒नम् गृ॑ही॒त्वा गृ॑ही॒त्वा ब॑हिष्पवमा॒नꣳ सर्प॑न्ति॒ सर्प॑न्ति बहिष्पवमा॒नम् गृ॑ही॒त्वा गृ॑ही॒त्वा ब॑हिष्पवमा॒नꣳ सर्प॑न्ति । \newline
20. ब॒हि॒ष्प॒व॒मा॒नꣳ सर्प॑न्ति॒ सर्प॑न्ति बहिष्पवमा॒नम् ब॑हिष्पवमा॒नꣳ सर्प॑न्ति॒ परा᳚ञ्चः॒ परा᳚ञ्चः॒ सर्प॑न्ति बहिष्पवमा॒नम् ब॑हिष्पवमा॒नꣳ सर्प॑न्ति॒ परा᳚ञ्चः । \newline
21. ब॒हि॒ष्प॒व॒मा॒नमिति॑ बहिः - प॒व॒मा॒नम् । \newline
22. सर्प॑न्ति॒ परा᳚ञ्चः॒ परा᳚ञ्चः॒ सर्प॑न्ति॒ सर्प॑न्ति॒ परा᳚ञ्चो॒ हि हि परा᳚ञ्चः॒ सर्प॑न्ति॒ सर्प॑न्ति॒ परा᳚ञ्चो॒ हि । \newline
23. परा᳚ञ्चो॒ हि हि परा᳚ञ्चः॒ परा᳚ञ्चो॒ हि यन्ति॒ यन्ति॒ हि परा᳚ञ्चः॒ परा᳚ञ्चो॒ हि यन्ति॑ । \newline
24. हि यन्ति॒ यन्ति॒ हि हि यन्ति॒ परा॑चीभिः॒ परा॑चीभि॒र् यन्ति॒ हि हि यन्ति॒ परा॑चीभिः । \newline
25. यन्ति॒ परा॑चीभिः॒ परा॑चीभि॒र् यन्ति॒ यन्ति॒ परा॑चीभिः स्तु॒वते᳚ स्तु॒वते॒ परा॑चीभि॒र् यन्ति॒ यन्ति॒ परा॑चीभिः स्तु॒वते᳚ । \newline
26. परा॑चीभिः स्तु॒वते᳚ स्तु॒वते॒ परा॑चीभिः॒ परा॑चीभिः स्तु॒वते॑ वैष्ण॒व्या वै᳚ष्ण॒व्या स्तु॒वते॒ परा॑चीभिः॒ परा॑चीभिः स्तु॒वते॑ वैष्ण॒व्या । \newline
27. स्तु॒वते॑ वैष्ण॒व्या वै᳚ष्ण॒व्या स्तु॒वते᳚ स्तु॒वते॑ वैष्ण॒व्यर्चर्चा वै᳚ष्ण॒व्या स्तु॒वते᳚ स्तु॒वते॑ वैष्ण॒व्यर्चा । \newline
28. वै॒ष्ण॒व्यर्चर्चा वै᳚ष्ण॒व्या वै᳚ष्ण॒व्यर्चा पुनः॒ पुनर्॑. ऋ॒चा वै᳚ष्ण॒व्या वै᳚ष्ण॒व्यर्चा पुनः॑ । \newline
29. ऋ॒चा पुनः॒ पुनर्॑. ऋ॒चर्चा पुन॒ रेत्येत्य॒ पुनर्॑. ऋ॒चर्चा पुन॒ रेत्य॑ । \newline
30. पुन॒ रेत्येत्य॒ पुनः॒ पुन॒ रेत्योपो पेत्य॒ पुनः॒ पुन॒ रेत्योप॑ । \newline
31. एत्योपो पेत्येत्योप॑ तिष्ठते तिष्ठत॒ उपेतेत्योप॑ तिष्ठते । \newline
32. एत्येत्या᳚ - इत्य॑ । \newline
33. उप॑ तिष्ठते तिष्ठत॒ उपोप॑ तिष्ठते य॒ज्ञो य॒ज्ञ् स्ति॑ष्ठत॒ उपोप॑ तिष्ठते य॒ज्ञ्ः । \newline
34. ति॒ष्ठ॒ते॒ य॒ज्ञो य॒ज्ञ् स्ति॑ष्ठते तिष्ठते य॒ज्ञो वै वै य॒ज्ञ् स्ति॑ष्ठते तिष्ठते य॒ज्ञो वै । \newline
35. य॒ज्ञो वै वै य॒ज्ञो य॒ज्ञो वै विष्णु॒र् विष्णु॒र् वै य॒ज्ञो य॒ज्ञो वै विष्णुः॑ । \newline
36. वै विष्णु॒र् विष्णु॒र् वै वै विष्णु॑र् य॒ज्ञ्ं ॅय॒ज्ञ्ं ॅविष्णु॒र् वै वै विष्णु॑र् य॒ज्ञ्म् । \newline
37. विष्णु॑र् य॒ज्ञ्ं ॅय॒ज्ञ्ं ॅविष्णु॒र् विष्णु॑र् य॒ज्ञ् मे॒वैव य॒ज्ञ्ं ॅविष्णु॒र् विष्णु॑र् य॒ज्ञ् मे॒व । \newline
38. य॒ज्ञ् मे॒वैव य॒ज्ञ्ं ॅय॒ज्ञ् मे॒वाक॑ रक रे॒व य॒ज्ञ्ं ॅय॒ज्ञ् मे॒वाकः॑ । \newline
39. ए॒वाक॑ रक रे॒वैवाक॒र् विष्णो॒ विष्णो॑ अक रे॒वैवाक॒र् विष्णो᳚ । \newline
40. अ॒क॒र् विष्णो॒ विष्णो॑ अक रक॒र् विष्णो॒ त्वम् त्वं ॅविष्णो॑ अक रक॒र् विष्णो॒ त्वम् । \newline
41. विष्णो॒ त्वम् त्वं ॅविष्णो॒ विष्णो॒ त्वम् नो॑ न॒स्त्वं ॅविष्णो॒ विष्णो॒ त्वम् नः॑ । \newline
42. विष्णो॒ इति॒ विष्णो᳚ । \newline
43. त्वम् नो॑ न॒स्त्वम् त्वम् नो॒ अन्त॑मो॒ अन्त॑मो न॒स्त्वम् त्वम् नो॒ अन्त॑मः । \newline
44. नो॒ अन्त॑मो॒ अन्त॑मो नो नो॒ अन्त॑मः॒ शर्म॒ शर्मान्त॑मो नो नो॒ अन्त॑मः॒ शर्म॑ । \newline
45. अन्त॑मः॒ शर्म॒ शर्मान्त॑मो॒ अन्त॑मः॒ शर्म॑ यच्छ यच्छ॒ शर्मान्त॑मो॒ अन्त॑मः॒ शर्म॑ यच्छ । \newline
46. शर्म॑ यच्छ यच्छ॒ शर्म॒ शर्म॑ यच्छ सहन्त्य सहन्त्य यच्छ॒ शर्म॒ शर्म॑ यच्छ सहन्त्य । \newline
47. य॒च्छ॒ स॒ह॒न्त्य॒ स॒ह॒न्त्य॒ य॒च्छ॒ य॒च्छ॒ स॒ह॒न्त्य॒ । \newline
48. स॒ह॒न्त्येति॑ सहन्त्य । \newline
49. प्र ते॑ ते॒ प्र प्र ते॒ धारा॒ धारा᳚ स्ते॒ प्र प्र ते॒ धाराः᳚ । \newline
50. ते॒ धारा॒ धारा᳚ स्ते ते॒ धारा॑ मधु॒श्चुतो॑ मधु॒श्चुतो॒ धारा᳚ स्ते ते॒ धारा॑ मधु॒श्चुतः॑ । \newline
51. धारा॑ मधु॒श्चुतो॑ मधु॒श्चुतो॒ धारा॒ धारा॑ मधु॒श्चुत॒ उथ्स॒ मुथ्स॑म् मधु॒श्चुतो॒ धारा॒ धारा॑ मधु॒श्चुत॒ उथ्स᳚म् । \newline
52. म॒धु॒श्चुत॒ उथ्स॒ मुथ्स॑म् मधु॒श्चुतो॑ मधु॒श्चुत॒ उथ्स॑म् दुह्रते दुह्रत॒ उथ्स॑म् मधु॒श्चुतो॑ मधु॒श्चुत॒ उथ्स॑म् दुह्रते । \newline
53. म॒धु॒श्चुत॒ इति॑ मधु - श्चुतः॑ । \newline
54. उथ्स॑म् दुह्रते दुह्रत॒ उथ्स॒ मुथ्स॑म् दुह्रते॒ अक्षि॑त॒ मक्षि॑तम् दुह्रत॒ उथ्स॒ मुथ्स॑म् दुह्रते॒ अक्षि॑तम् । \newline
55. दु॒ह्र॒ते॒ अक्षि॑त॒ मक्षि॑तम् दुह्रते दुह्रते॒ अक्षि॑त॒ मितीत्यक्षि॑तम् दुह्रते दुह्रते॒ अक्षि॑त॒ मिति॑ । \newline
56. अक्षि॑त॒ मितीत्यक्षि॑त॒ मक्षि॑त॒ मित्या॑हा॒हे त्यक्षि॑त॒ मक्षि॑त॒ मित्या॑ह । \newline
57. इत्या॑हा॒हे तीत्या॑ह॒ यद् यदा॒हे तीत्या॑ह॒ यत् । \newline
58. आ॒ह॒ यद् यदा॑हाह॒ यदे॒वैव यदा॑हाह॒ यदे॒व । \newline
59. यदे॒वैव यद् यदे॒वास्या᳚स्यै॒व यद् यदे॒वास्य॑ । \newline
60. ए॒वास्या᳚ स्यै॒वैवास्य॒ शया॑नस्य॒ शया॑नस्या स्यै॒वैवास्य॒ शया॑नस्य । \newline
61. अ॒स्य॒ शया॑नस्य॒ शया॑नस्या स्यास्य॒ शया॑न स्योप॒शुष्य॑ त्युप॒शुष्य॑ति॒ शया॑नस्या स्यास्य॒ शया॑न स्योप॒शुष्य॑ति । \newline
62. शया॑न स्योप॒शुष्य॑ त्युप॒शुष्य॑ति॒ शया॑नस्य॒ शया॑न स्योप॒शुष्य॑ति॒ तत् तदु॑प॒शुष्य॑ति॒ शया॑नस्य॒ शया॑न स्योप॒शुष्य॑ति॒ तत् । \newline
63. उ॒प॒शुष्य॑ति॒ तत् तदु॑प॒शुष्य॑ त्युप॒शुष्य॑ति॒ तदे॒वैव तदु॑प॒शुष्य॑ त्युप॒शुष्य॑ति॒ तदे॒व । \newline
64. उ॒प॒शुष्य॒तीत्यु॑प - शुष्य॑ति । \newline
65. तदे॒वैव तत् तदे॒वास्या᳚स्यै॒व तत् तदे॒वास्य॑ । \newline
66. ए॒वास्या᳚ स्यै॒वैवा स्यै॒ते नै॒तेना᳚ स्यै॒वैवा स्यै॒तेन॑ । \newline
67. अ॒स्यै॒ते नै॒तेना᳚स्या स्यै॒ते नैतेना᳚स्या स्यै॒तेना । \newline
68. ए॒तेनैते नै॒तेना प्या॑ययति प्यायय॒ त्यैते नै॒तेना प्या॑ययति । \newline
69. आ प्या॑ययति प्यायय॒त्या प्या॑ययति । \newline
70. प्या॒य॒य॒तीति॑ प्याययति । \newline
\pagebreak
\markright{ TS 3.1.11.1  \hfill https://www.vedavms.in \hfill}

\section{ TS 3.1.11.1 }

\textbf{TS 3.1.11.1 } \newline
\textbf{Samhita Paata} \newline

अ॒ग्निना॑ र॒यिम॑श्नव॒त् पोष॑मे॒व दि॒वेदि॑वे । य॒शसं॑ ॅवी॒रव॑त्तमं ॥गोमाꣳ॑ अ॒ग्नेऽवि॑माꣳ अ॒श्वी य॒ज्ञो नृ॒वथ्स॑खा॒ सद॒मिद॑प्रमृ॒ष्यः । इडा॑वाꣳ ए॒षो अ॑सुर प्र॒जावा᳚न् दी॒र्घो र॒यिः पृ॑थुबु॒ध्नः स॒भावान्॑ ॥आप्या॑यस्व॒>1, सन्ते᳚>2 ॥ इ॒ह त्वष्टा॑रमग्रि॒यं ॅवि॒श्वरू॑प॒मुप॑ ह्वये । अ॒स्माक॑मस्तु॒ केव॑लः ॥ तन्न॑स्तु॒रीप॒मध॑ पोषयि॒त्नु देव॑ त्वष्ट॒र्वि र॑रा॒णः स्य॑स्व । यतो॑ वी॒रः - [  ] \newline

\textbf{Pada Paata} \newline

अ॒ग्निना᳚ । र॒यिम् । अ॒श्न॒व॒त् । पोष᳚म् । ए॒व । दि॒वेदि॑व॒ इति॑ दि॒वे - दि॒व॒ ॥ य॒शस᳚म् । वी॒रव॑त्तम॒मिति॑ वी॒रव॑त् - त॒म॒म् ॥ गोमा॒निति॒ गो - मा॒न् । अ॒ग्ने॒ । अवि॑मा॒नित्यवि॑ - मा॒न् । अ॒श्वी । य॒ज्ञ्ः । नृ॒वथ्स॒खेति॑ नृ॒वत् - स॒खा॒ । सद᳚म् । इत् । अ॒प्र॒मृ॒ष्य इत्य॑प्र - मृ॒ष्यः ॥ इडा॑वा॒नितीडा᳚ - वा॒न् । ए॒षः । अ॒सु॒र॒ । प्र॒जावा॒निति॑ प्र॒जा - वा॒न् । दी॒र्घः । र॒यिः । पृ॒थु॒बु॒द्ध्न इति॑ पृथु - बु॒द्ध्नः । स॒भावा॒निति॑ स॒भा - वा॒न् ॥ एति॑ । प्या॒य॒स्व॒ । समिति॑ । ते॒ ॥ इ॒ह । त्वष्टा॑रम् । अ॒ग्रि॒यम् । वि॒श्वरू॑प॒मिति॑ वि॒श्व - रू॒प॒म् । उपेति॑ । ह्व॒ये॒ ॥ अ॒स्माक᳚म् । अ॒स्तु॒ । केव॑लः ॥ तत् । नः॒ । तु॒रीप᳚म् । अध॑ । पो॒ष॒यि॒त्नु । देव॑ । त्व॒ष्टः॒ । वीति॑ । र॒रा॒णः । स्य॒स्व॒ ॥ यतः॑ । वी॒रः ।  \newline


\textbf{Krama Paata} \newline

अ॒ग्निना॑ र॒यिम् । र॒यिम॑श्ञवत् । अ॒श्ञ॒व॒त् पोष᳚म् । पोष॑मे॒व । ए॒व दि॒वेदि॑वे । दि॒वेदि॑व॒ इति॑ दि॒वे - दि॒वे॒ ॥ य॒शसं॑ ॅवी॒रव॑त्तमम् । वी॒रव॑त्तम॒मिति॑ वी॒रव॑त् - त॒म॒म् ॥ गोमाꣳ॑ अग्ने । गोमा॒निति॒ गो - मा॒न्॒ । अ॒ग्नेऽवि॑मान् । अवि॑माꣳ अ॒श्वी । अवि॑मा॒नित्यवि॑ - मा॒न्॒ । अ॒श्वी य॒ज्ञ्ः । य॒ज्ञो नृ॒वथ्स॑खा । नृ॒वथ्स॑खा॒ सद᳚म् । नृ॒वथ्स॒खेति॑ नृ॒वत् - स॒खा॒ । सद॒मित् । इद॑प्रमृ॒ष्यः । अ॒प्र॒मृ॒ष्य इत्य॑प्र - मृ॒ष्यः ॥ इडा॑वाꣳ ए॒षः । इडा॑वा॒नितीडा᳚ - वा॒न्॒ । ए॒षो अ॑सुर । अ॒सु॒र॒ प्र॒जावान्॑ । प्र॒जावा᳚न् दी॒र्घः । प्र॒जावा॒निति॑ प्र॒जा - वा॒न्॒ । दी॒र्घो र॒यिः । र॒यिः पृ॑थुबु॒ध्नः । पृ॒थु॒बु॒ध्नः स॒भावान्॑ । पृ॒थु॒बु॒ध्न इति॑ पृथु - बु॒ध्नः । स॒भावा॒निति॑ स॒भा - वा॒न् ॥ आ प्या॑यस्व । प्या॒य॒स्व॒ सम् । सम् ते᳚ । त॒ इति॑ ते ॥ इ॒ह त्वष्टा॑रम् । त्वष्टा॑रमग्रि॒यम् । अ॒ग्रि॒यं ॅवि॒श्वरू॑पम् । वि॒श्वरू॑प॒मुप॑ । वि॒श्वरू॑प॒मिति॑ वि॒श्व - रू॒प॒म् । उप॑ ह्वये । ह्व॒य॒ इति॑ ह्वये ॥ अ॒स्माक॑मस्तु । अ॒स्तु॒ केव॑लः । केव॑ल॒ इति॒ केव॑लः ॥ तन्नः॑ । न॒स्तु॒रीप᳚म् । तु॒रीप॒मध॑ । अध॑ पोषयि॒त्नु । पो॒ष॒यि॒त्नु देव॑ । देव॑ त्वष्टः । त्व॒ष्ट॒र् वि । वि र॑रा॒णः । र॒रा॒णः स्य॑स्व । स्य॒स्वेति॑ स्यस्व ॥ यतो॑ वी॒रः । वी॒रः क॑र्म॒ण्यः॑ \newline

\textbf{Jatai Paata} \newline

1. अ॒ग्निना॑ र॒यिꣳ र॒यि म॒ग्निना॒ ऽग्निना॑ र॒यिम् । \newline
2. र॒यि म॑श्ञव दश्ञवद् र॒यिꣳ र॒यि म॑श्ञवत् । \newline
3. अ॒श्ञ॒व॒त् पोष॒म् पोष॑ मश्ञव दश्ञव॒त् पोष᳚म् । \newline
4. पोष॑ मे॒वैव पोष॒म् पोष॑ मे॒व । \newline
5. ए॒व दि॒वेदि॑वे दि॒वेदि॑व ए॒वैव दि॒वेदि॑वे । \newline
6. दि॒वेदि॑व॒ इति॑ दि॒वे - दि॒वे॒ । \newline
7. य॒शसं॑ ॅवी॒रव॑त्तमं ॅवी॒रव॑त्तमं ॅय॒शसं॑ ॅय॒शसं॑ ॅवी॒रव॑त्तमम् । \newline
8. वी॒रव॑त्तम॒मिति॑ वी॒रव॑त् - त॒म॒म् । \newline
9. गोमाꣳ॑ अग्ने ऽग्ने॒ गोमा॒न् गोमाꣳ॑ अग्ने । \newline
10. गोमा॒निति॒ गो - मा॒न् । \newline
11. अ॒ग्ने ऽवि॑माꣳ॒॒ अवि॑माꣳ अग्ने॒ ऽग्ने ऽवि॑मान् । \newline
12. अवि॑माꣳ अ॒श्व्य॑ श्व्यवि॑माꣳ॒॒ अवि॑माꣳ अ॒श्वी । \newline
13. अवि॑मा॒नित्यवि॑ - मा॒न् । \newline
14. अ॒श्वी य॒ज्ञो य॒ज्ञो᳚(1॒ओ) ऽश्व्य॑श्वी य॒ज्ञ्ः । \newline
15. य॒ज्ञो नृ॒वथ्स॑खा नृ॒वथ्स॑खा य॒ज्ञो य॒ज्ञो नृ॒वथ्स॑खा । \newline
16. नृ॒वथ्स॑खा॒ सदꣳ॒॒ सद॑म् नृ॒वथ्स॑खा नृ॒वथ्स॑खा॒ सद᳚म् । \newline
17. नृ॒वथ्स॒खेति॑ नृ॒वत् - स॒खा॒ । \newline
18. सद॒ मिदिथ् सदꣳ॒॒ सद॒ मित् । \newline
19. इद॑प्रमृ॒ष्यो᳚ ऽप्रमृ॒ष्य इदिद॑प्रमृ॒ष्यः । \newline
20. अ॒प्र॒मृ॒ष्य इत्य॑प्र - मृ॒ष्यः । \newline
21. इडा॑वाꣳ ए॒ष ए॒ष इडा॑वाꣳ॒॒ इडा॑वाꣳ ए॒षः । \newline
22. इडा॑वा॒नितीडा᳚ - वा॒न् । \newline
23. ए॒षो अ॑सुरा सुरै॒ष ए॒षो अ॑सुर । \newline
24. अ॒सु॒र॒ प्र॒जावा᳚न् प्र॒जावा॑ नसुरासुर प्र॒जावान्॑ । \newline
25. प्र॒जावा᳚न् दी॒र्घो दी॒र्घः प्र॒जावा᳚न् प्र॒जावा᳚न् दी॒र्घः । \newline
26. प्र॒जावा॒निति॑ प्र॒जा - वा॒न् । \newline
27. दी॒र्घो र॒यी र॒यिर् दी॒र्घो दी॒र्घो र॒यिः । \newline
28. र॒यिः पृ॑थुबु॒द्ध्नः पृ॑थुबु॒द्ध्नो र॒यी र॒यिः पृ॑थुबु॒द्ध्नः । \newline
29. पृ॒थु॒बु॒द्ध्नः स॒भावा᳚न् थ्स॒भावा᳚न् पृथुबु॒द्ध्नः पृ॑थुबु॒द्ध्नः स॒भावान्॑ । \newline
30. पृ॒थु॒बु॒द्ध्न इति॑ पृथु - बु॒द्ध्नः । \newline
31. स॒भावा॒निति॑ स॒भा - वा॒न् । \newline
32. आ प्या॑यस्व प्याय॒स्वा प्या॑यस्व । \newline
33. प्या॒य॒स्व॒ सꣳ सम् प्या॑यस्व प्यायस्व॒ सम् । \newline
34. सम् ते॑ ते॒ सꣳ सम् ते᳚ । \newline
35. त॒ इति॑ ते । \newline
36. इ॒ह त्वष्टा॑र॒म् त्वष्टा॑र मि॒हे ह त्वष्टा॑रम् । \newline
37. त्वष्टा॑र मग्रि॒य म॑ग्रि॒यम् त्वष्टा॑र॒म् त्वष्टा॑र मग्रि॒यम् । \newline
38. अ॒ग्रि॒यं ॅवि॒श्वरू॑पं ॅवि॒श्वरू॑प मग्रि॒य म॑ग्रि॒यं ॅवि॒श्वरू॑पम् । \newline
39. वि॒श्वरू॑प॒ मुपोप॑ वि॒श्वरू॑पं ॅवि॒श्वरू॑प॒ मुप॑ । \newline
40. वि॒श्वरू॑प॒मिति॑ वि॒श्व - रू॒प॒म् । \newline
41. उप॑ ह्वये ह्वय॒ उपोप॑ ह्वये । \newline
42. ह्व॒य॒ इति॑ ह्वये । \newline
43. अ॒स्माक॑ मस्त्व स्त्व॒स्माक॑ म॒स्माक॑ मस्तु । \newline
44. अ॒स्तु॒ केव॑लः॒ केव॑लो अस्त्वस्तु॒ केव॑लः । \newline
45. केव॑ल॒ इति॒ केव॑लः । \newline
46. तन् नो॑ न॒ स्तत् तन् नः॑ । \newline
47. न॒ स्तु॒रीप॑म् तु॒रीप॑म् नो न स्तु॒रीप᳚म् । \newline
48. तु॒रीप॒ मधाध॑ तु॒रीप॑म् तु॒रीप॒ मध॑ । \newline
49. अध॑ पोषयि॒त्नु पो॑षयि॒त्न्वधाध॑ पोषयि॒त्नु । \newline
50. पो॒ष॒यि॒त्नु देव॒ देव॑ पोषयि॒त्नु पो॑षयि॒त्नु देव॑ । \newline
51. देव॑ त्वष्ट स्त्वष्ट॒र् देव॒ देव॑ त्वष्टः । \newline
52. त्व॒ष्ट॒र् वि वि त्व॑ष्ट स्त्वष्ट॒र् वि । \newline
53. वि र॑रा॒णो र॑रा॒णो वि वि र॑रा॒णः । \newline
54. र॒रा॒णः स्य॑स्व स्यस्व ररा॒णो र॑रा॒णः स्य॑स्व । \newline
55. स्य॒स्वेति॑ स्यस्व । \newline
56. यतो॑ वी॒रो वी॒रो यतो॒ यतो॑ वी॒रः । \newline
57. वी॒रः क॑र्म॒ण्यः॑ कर्म॒ण्यो॑ वी॒रो वी॒रः क॑र्म॒ण्यः॑ । \newline

\textbf{Ghana Paata } \newline

1. अ॒ग्निना॑ र॒यिꣳ र॒यि म॒ग्निना॒ ऽग्निना॑ र॒यि म॑श्ञव दश्ञवद् र॒यि म॒ग्निना॒ ऽग्निना॑ र॒यि म॑श्ञवत् । \newline
2. र॒यि म॑श्ञव दश्ञवद् र॒यिꣳ र॒यि म॑श्ञव॒त् पोष॒म् पोष॑ मश्ञवद् र॒यिꣳ र॒यि म॑श्ञव॒त् पोष᳚म् । \newline
3. अ॒श्ञ॒व॒त् पोष॒म् पोष॑ मश्ञव दश्ञव॒त् पोष॑ मे॒वैव पोष॑ मश्ञव दश्ञव॒त् पोष॑ मे॒व । \newline
4. पोष॑ मे॒वैव पोष॒म् पोष॑ मे॒व दि॒वेदि॑वे दि॒वेदि॑व ए॒व पोष॒म् पोष॑ मे॒व दि॒वेदि॑वे । \newline
5. ए॒व दि॒वेदि॑वे दि॒वेदि॑व ए॒वैव दि॒वेदि॑वे । \newline
6. दि॒वेदि॑व॒ इति॑ दि॒वे - दि॒वे॒ । \newline
7. य॒शसं॑ ॅवी॒रव॑त्तमं ॅवी॒रव॑त्तमं ॅय॒शसं॑ ॅय॒शसं॑ ॅवी॒रव॑त्तमम् । \newline
8. वी॒रव॑त्तम॒मिति॑ वी॒रव॑त् - त॒म॒म् । \newline
9. गोमाꣳ॑ अग्ने ऽग्ने॒ गोमा॒न् गोमाꣳ॑ अ॒ग्ने ऽवि॑माꣳ॒॒ अवि॑माꣳ अग्ने॒ गोमा॒न् गोमाꣳ॑ अ॒ग्ने ऽवि॑मान् । \newline
10. गोमा॒निति॒ गो - मा॒न् । \newline
11. अ॒ग्ने ऽवि॑माꣳ॒॒ अवि॑माꣳ अग्ने॒ ऽग्ने ऽवि॑माꣳ अ॒श्व्य॑ श्व्यवि॑माꣳ अग्ने॒ ऽग्ने ऽवि॑माꣳ अ॒श्वी । \newline
12. अवि॑माꣳ अ॒श्व्य॑ श्व्यवि॑माꣳ॒॒ अवि॑माꣳ अ॒श्वी य॒ज्ञो य॒ज्ञो᳚ ऽश्व्यवि॑माꣳ॒॒ अवि॑माꣳ अ॒श्वी य॒ज्ञ्ः । \newline
13. अवि॑मा॒नित्यवि॑ - मा॒न् । \newline
14. अ॒श्वी य॒ज्ञो य॒ज्ञो᳚(ओ1॒) ऽश्व्य॑श्वी य॒ज्ञो नृ॒वथ्स॑खा नृ॒वथ्स॑खा य॒ज्ञो᳚(ओ1॒) ऽश्व्य॑श्वी य॒ज्ञो नृ॒वथ्स॑खा । \newline
15. य॒ज्ञो नृ॒वथ्स॑खा नृ॒वथ्स॑खा य॒ज्ञो य॒ज्ञो नृ॒वथ्स॑खा॒ सदꣳ॒॒ सद॑म् नृ॒वथ्स॑खा य॒ज्ञो य॒ज्ञो नृ॒वथ्स॑खा॒ सद᳚म् । \newline
16. नृ॒वथ्स॑खा॒ सदꣳ॒॒ सद॑म् नृ॒वथ्स॑खा नृ॒वथ्स॑खा॒ सद॒ मिदिथ् सद॑म् नृ॒वथ्स॑खा नृ॒वथ्स॑खा॒ सद॒ मित् । \newline
17. नृ॒वथ्स॒खेति॑ नृ॒वत् - स॒खा॒ । \newline
18. सद॒ मिदिथ् सदꣳ॒॒ सद॒ मिद॑प्रमृ॒ष्यो᳚ ऽप्रमृ॒ष्य इथ् सदꣳ॒॒ सद॒ मिद॑प्रमृ॒ष्यः । \newline
19. इद॑प्रमृ॒ष्यो᳚ ऽप्रमृ॒ष्य इदिद॑प्रमृ॒ष्यः । \newline
20. अ॒प्र॒मृ॒ष्य इत्य॑प्र - मृ॒ष्यः । \newline
21. इडा॑वाꣳ ए॒ष ए॒ष इडा॑वाꣳ॒॒ इडा॑वाꣳ ए॒षो अ॑सुरा सुरै॒ष इडा॑वाꣳ॒॒ इडा॑वाꣳ ए॒षो अ॑सुर । \newline
22. इडा॑वा॒नितीडा᳚ - वा॒न् । \newline
23. ए॒षो अ॑सुरा सुरै॒ष ए॒षो अ॑सुर प्र॒जावा᳚न् प्र॒जावा॑ नसुरै॒ष ए॒षो अ॑सुर प्र॒जावान्॑ । \newline
24. अ॒सु॒र॒ प्र॒जावा᳚न् प्र॒जावा॑ नसुरासुर प्र॒जावा᳚न् दी॒र्घो दी॒र्घः प्र॒जावा॑ नसुरा सुर प्र॒जावा᳚न् दी॒र्घः । \newline
25. प्र॒जावा᳚न् दी॒र्घो दी॒र्घः प्र॒जावा᳚न् प्र॒जावा᳚न् दी॒र्घो र॒यी र॒यिर् दी॒र्घः प्र॒जावा᳚न् प्र॒जावा᳚न् दी॒र्घो र॒यिः । \newline
26. प्र॒जावा॒निति॑ प्र॒जा - वा॒न् । \newline
27. दी॒र्घो र॒यी र॒यिर् दी॒र्घो दी॒र्घो र॒यिः पृ॑थुबु॒द्ध्नः पृ॑थुबु॒द्ध्नो र॒यिर् दी॒र्घो दी॒र्घो र॒यिः पृ॑थुबु॒द्ध्नः । \newline
28. र॒यिः पृ॑थुबु॒द्ध्नः पृ॑थुबु॒द्ध्नो र॒यी र॒यिः पृ॑थुबु॒द्ध्नः स॒भावा᳚न् थ्स॒भावा᳚न् पृथुबु॒द्ध्नो र॒यी र॒यिः पृ॑थुबु॒द्ध्नः स॒भावान्॑ । \newline
29. पृ॒थु॒बु॒द्ध्नः स॒भावा᳚न् थ्स॒भावा᳚न् पृथुबु॒द्ध्नः पृ॑थुबु॒द्ध्नः स॒भावान्॑ । \newline
30. पृ॒थु॒बु॒द्ध्न इति॑ पृथु - बु॒द्ध्नः । \newline
31. स॒भावा॒निति॑ स॒भा - वा॒न् । \newline
32. आ प्या॑यस्व प्याय॒स्वा प्या॑यस्व॒ सꣳ सम् प्या॑य॒स्वा प्या॑यस्व॒ सम् । \newline
33. प्या॒य॒स्व॒ सꣳ सम् प्या॑यस्व प्यायस्व॒ सम् ते॑ ते॒ सम् प्या॑यस्व प्यायस्व॒ सम् ते᳚ । \newline
34. सम् ते॑ ते॒ सꣳ सम् ते᳚ । \newline
35. त॒ इति॑ ते । \newline
36. इ॒ह त्वष्टा॑र॒म् त्वष्टा॑र मि॒हे ह त्वष्टा॑र मग्रि॒य म॑ग्रि॒यम् त्वष्टा॑र मि॒हे ह त्वष्टा॑र मग्रि॒यम् । \newline
37. त्वष्टा॑र मग्रि॒य म॑ग्रि॒यम् त्वष्टा॑र॒म् त्वष्टा॑र मग्रि॒यं ॅवि॒श्वरू॑पं ॅवि॒श्वरू॑प मग्रि॒यम् त्वष्टा॑र॒म् त्वष्टा॑र मग्रि॒यं ॅवि॒श्वरू॑पम् । \newline
38. अ॒ग्रि॒यं ॅवि॒श्वरू॑पं ॅवि॒श्वरू॑प मग्रि॒य म॑ग्रि॒यं ॅवि॒श्वरू॑प॒ मुपोप॑ वि॒श्वरू॑प मग्रि॒य म॑ग्रि॒यं ॅवि॒श्वरू॑प॒ मुप॑ । \newline
39. वि॒श्वरू॑प॒ मुपोप॑ वि॒श्वरू॑पं ॅवि॒श्वरू॑प॒ मुप॑ ह्वये ह्वय॒ उप॑ वि॒श्वरू॑पं ॅवि॒श्वरू॑प॒ मुप॑ ह्वये । \newline
40. वि॒श्वरू॑प॒मिति॑ वि॒श्व - रू॒प॒म् । \newline
41. उप॑ ह्वये ह्वय॒ उपोप॑ ह्वये । \newline
42. ह्व॒य॒ इति॑ ह्वये । \newline
43. अ॒स्माक॑ मस्त्व स्त्व॒स्माक॑ म॒स्माक॑ मस्तु॒ केव॑लः॒ केव॑लो अस्त्व॒स्माक॑ म॒स्माक॑ मस्तु॒ केव॑लः । \newline
44. अ॒स्तु॒ केव॑लः॒ केव॑लो अस्त्वस्तु॒ केव॑लः । \newline
45. केव॑ल॒ इति॒ केव॑लः । \newline
46. तन् नो॑ न॒स्तत् तन् न॑ स्तु॒रीप॑म् तु॒रीप॑म् न॒स्तत् तन् न॑स्तु॒रीप᳚म् । \newline
47. न॒स्तु॒रीप॑म् तु॒रीप॑म् नो नस्तु॒रीप॒ मधाध॑ तु॒रीप॑म् नो नस्तु॒रीप॒ मध॑ । \newline
48. तु॒रीप॒ मधाध॑ तु॒रीप॑म् तु॒रीप॒ मध॑ पोषयि॒त्नु पो॑षयि॒त्‌न्वध॑ तु॒रीप॑म् तु॒रीप॒ मध॑ पोषयि॒त्नु । \newline
49. अध॑ पोषयि॒त्नु पो॑षयि॒त्‌न्वधाध॑ पोषयि॒त्नु देव॒ देव॑ पोषयि॒त्‌न्वधाध॑ पोषयि॒त्नु देव॑ । \newline
50. पो॒ष॒यि॒त्नु देव॒ देव॑ पोषयि॒त्नु पो॑षयि॒त्नु देव॑ त्वष्ट स्त्वष्ट॒र् देव॑ पोषयि॒त्नु पो॑षयि॒त्नु देव॑ त्वष्टः । \newline
51. देव॑ त्वष्ट स्त्वष्ट॒र् देव॒ देव॑ त्वष्ट॒र् वि वि त्व॑ष्ट॒र् देव॒ देव॑ त्वष्ट॒र् वि । \newline
52. त्व॒ष्ट॒र् वि वि त्व॑ष्ट स्त्वष्ट॒र् वि र॑रा॒णो र॑रा॒णो वि त्व॑ष्ट स्त्वष्ट॒र् वि र॑रा॒णः । \newline
53. वि र॑रा॒णो र॑रा॒णो वि वि र॑रा॒णः स्य॑स्व स्यस्व ररा॒णो वि वि र॑रा॒णः स्य॑स्व । \newline
54. र॒रा॒णः स्य॑स्व स्यस्व ररा॒णो र॑रा॒णः स्य॑स्व । \newline
55. स्य॒स्वेति॑ स्यस्व । \newline
56. यतो॑ वी॒रो वी॒रो यतो॒ यतो॑ वी॒रः क॑र्म॒ण्यः॑ कर्म॒ण्यो॑ वी॒रो यतो॒ यतो॑ वी॒रः क॑र्म॒ण्यः॑ । \newline
57. वी॒रः क॑र्म॒ण्यः॑ कर्म॒ण्यो॑ वी॒रो वी॒रः क॑र्म॒ण्यः॑ सु॒दक्षः॑ सु॒दक्षः॑ कर्म॒ण्यो॑ वी॒रो वी॒रः क॑र्म॒ण्यः॑ सु॒दक्षः॑ । \newline
\pagebreak
\markright{ TS 3.1.11.2  \hfill https://www.vedavms.in \hfill}

\section{ TS 3.1.11.2 }

\textbf{TS 3.1.11.2 } \newline
\textbf{Samhita Paata} \newline

क॑र्म॒ण्यः॑ सु॒दक्षो॑ यु॒क्तग्रा॑वा॒ जाय॑ते दे॒वका॑मः ॥शि॒वस्त्व॑ष्टरि॒हाऽऽ* ग॑हि वि॒भुः पोष॑ उ॒तत्मना᳚ । य॒ज्ञेय॑ज्ञे न॒ उद॑व ॥ पि॒शङ्ग॑रूपः सु॒भरो॑ वयो॒धाः श्रु॒ष्टी वी॒रो जा॑यते दे॒वका॑मः । प्र॒जां त्वष्टा॒ विष्य॑तु॒ नाभि॑म॒स्मे अथा॑ दे॒वाना॒मप्ये॑तु॒ पाथः॑ ॥ प्रणो॑दे॒>3 व्या, नो॑ दि॒वः >4 ॥ पी॒पि॒वाꣳ सꣳ॒॒ सर॑स्वतः॒ स्तनं॒ ॅयो वि॒श्वद॑र्.शतः । धुक्षी॒महि॑ प्र॒जामिषं᳚ ॥ \newline

\textbf{Pada Paata} \newline

क॒र्म॒ण्यः॑ । सु॒दक्ष॒ इति॑ सु - दक्षः॑ । यु॒क्तग्रा॒वेति॑ यु॒क्त - ग्रा॒वा॒ । जाय॑ते । दे॒वका॑म॒ इति॑ दे॒व - का॒मः॒ ॥ शि॒वः । त्व॒ष्टः॒ । इ॒ह । एति॑ । ग॒हि॒ । वि॒भुरिति॑ वि - भुः । पोषे᳚ । उ॒त । त्मना᳚ ॥ य॒ज्ञे य॑ज्ञ्॒ इति॑ य॒ज्ञे - य॒ज्ञे॒ । नः॒ । उदिति॑ । अ॒व॒ ॥ पि॒शङ्ग॑रूप॒ इति॑ पि॒शङ्ग॑-रू॒पः॒ । सु॒भर॒ इति॑ सु - भरः॑ । व॒यो॒धा इति॑ वयः - धाः । श्रु॒ष्टी । वी॒रः । जा॒य॒ते॒ । दे॒वका॑म॒ इति॑ दे॒व - का॒मः॒ ॥ प्र॒जामिति॑ प्र - जाम् । त्वष्टा᳚ । वीति॑ । स्य॒तु॒ । नाभि᳚म् । अ॒स्मे इति॑ । अथ॑ । दे॒वाना᳚म् । अपीति॑ । ए॒तु॒ । पाथः॑ ॥ प्रेति॑ । नः॒ । दे॒वी । एति॑ । नः॒ । दि॒वः ॥ पी॒पि॒वाꣳस᳚म् । सर॑स्वतः । स्तन᳚म् । यः । वि॒श्वद॑र्.शत॒ इति॑ वि॒श्व - द॒र्॒.श॒तः॒ ॥ धु॒क्षी॒महि॑ । प्र॒जामिति॑ प्र - जाम् । इष᳚म् ॥  \newline


\textbf{Krama Paata} \newline

क॒र्म॒ण्यः॑ सु॒दक्षः॑ । सु॒दक्षो॑ यु॒क्तग्रा॑वा । सु॒दक्ष॒ इति॑ सु - दक्षः॑ । यु॒क्तग्रा॑वा॒ जाय॑ते । यु॒क्तग्रा॒वेति॑ यु॒क्त - ग्रा॒वा॒ । जाय॑ते दे॒वका॑मः । दे॒वका॑म॒ इति॑ दे॒व - का॒मः॒ ॥ शि॒वस्त्व॑ष्टः । त्व॒ष्ट॒रि॒ह । इ॒हा । आ ग॑हि । ग॒हि॒ वि॒भुः । वि॒भुः पोषे᳚ । वि॒भुरिति॑ वि - भुः । पोष॑ उ॒त । उ॒त त्मना᳚ । त्मनेति॒ त्मना᳚ ॥ य॒ज्ञेय॑ज्ञे नः । य॒ज्ञेय॑ज्ञ्॒ इति॑ य॒ज्ञे - य॒ज्ञे॒ । न॒ उत् । उद॑व । अ॒वेत्य॑व ॥ पि॒शङ्ग॑रूपः सु॒भरः॑ । पि॒शङ्ग॑रूप॒ इति॑ पि॒शङ्ग॑ - रू॒पः॒ । सु॒भरो॑ वयो॒धाः । सु॒भर॒ इति॑ सु - भरः॑ । व॒यो॒धाः श्रु॒ष्टी । व॒यो॒धा इति॑ वयः - धाः । श्रु॒ष्टी वी॒रः । वी॒रो जा॑यते । जा॒य॒ते॒ दे॒वका॑मः । दे॒वका॑म॒ इति॑ दे॒व - का॒मः॒ ॥ प्र॒जाम् त्वष्टा᳚ । प्र॒जामिति॑ प्र - जाम् । त्वष्टा॒ वि । वि ष्य॑तु । स्य॒तु॒ नाभि᳚म् । नाभि॑म॒स्मे । अ॒स्मे अथ॑ । अ॒स्मे इत्य॒स्मे । अथा॑ दे॒वाना᳚म् । दे॒वाना॒मपि॑ । अप्ये॑तु । ए॒तु॒ पाथः॑ । पाथ॒ इति॒ पाथः॑ ॥ प्र णः॑ । नो॒ दे॒वी । दे॒व्या । आ नः॑ । नो॒ दि॒वः । दि॒व इति॑ दि॒वः ॥ पी॒पि॒वाꣳसꣳ॒॒ सर॑स्वतः । सर॑स्वतः॒ स्तन᳚म् । स्तनं॒ ॅयः । यो वि॒श्वद॑र्.शतः । वि॒श्वद॑र्.शत॒ इति॑ वि॒श्व - द॒र्॒.श॒तः॒ ॥ धु॒क्षी॒महि॑ प्र॒जाम् । प्र॒जामिष᳚म् । प्र॒जामिति॑ प्र - जाम् । इष॒मितीष᳚म् । \newline

\textbf{Jatai Paata} \newline

1. क॒र्म॒ण्यः॑ सु॒दक्षः॑ सु॒दक्षः॑ कर्म॒ण्यः॑ कर्म॒ण्यः॑ सु॒दक्षः॑ । \newline
2. सु॒दक्षो॑ यु॒क्तग्रा॑वा यु॒क्तग्रा॑वा सु॒दक्षः॑ सु॒दक्षो॑ यु॒क्तग्रा॑वा । \newline
3. सु॒दक्ष॒ इति॑ सु - दक्षः॑ । \newline
4. यु॒क्तग्रा॑वा॒ जाय॑ते॒ जाय॑ते यु॒क्तग्रा॑वा यु॒क्तग्रा॑वा॒ जाय॑ते । \newline
5. यु॒क्तग्रा॒वेति॑ यु॒क्त - ग्रा॒वा॒ । \newline
6. जाय॑ते दे॒वका॑मो दे॒वका॑मो॒ जाय॑ते॒ जाय॑ते दे॒वका॑मः । \newline
7. दे॒वका॑म॒ इति॑ दे॒व - का॒मः॒ । \newline
8. शि॒व स्त्व॑ष्ट स्त्वष्टः शि॒वः शि॒व स्त्व॑ष्टः । \newline
9. त्व॒ष्ट॒ रि॒हे ह त्व॑ष्ट स्त्वष्ट रि॒ह । \newline
10. इ॒हेहे हा । \newline
11. आ ग॑हि ग॒ह्या ग॑हि । \newline
12. ग॒हि॒ वि॒भुर् वि॒भुर् ग॑हि गहि वि॒भुः । \newline
13. वि॒भुः पोषे॒ पोषे॑ वि॒भुर् वि॒भुः पोषे᳚ । \newline
14. वि॒भुरिति॑ वि - भुः । \newline
15. पोष॑ उ॒तोत पोषे॒ पोष॑ उ॒त । \newline
16. उ॒त त्मना॒ त्मनो॒तोत त्मना᳚ । \newline
17. त्मनेति॒ त्मना᳚ । \newline
18. य॒ज्ञेय॑ज्ञे नो नो य॒ज्ञेय॑ज्ञे य॒ज्ञेय॑ज्ञे नः । \newline
19. य॒ज्ञेय॑ज्ञ्॒ इति॑ य॒ज्ञे - य॒ज्ञे॒ । \newline
20. न॒ उदुन् नो॑ न॒ उत् । \newline
21. उ द॑वा॒वो दुद॑व । \newline
22. अ॒वेत्य॑व । \newline
23. पि॒शङ्ग॑रूपः सु॒भरः॑ सु॒भरः॑ पि॒शङ्ग॑रूपः पि॒शङ्ग॑रूपः सु॒भरः॑ । \newline
24. पि॒शङ्ग॑रूप॒ इति॑ पि॒शङ्ग॑ - रू॒पः॒ । \newline
25. सु॒भरो॑ वयो॒धा व॑यो॒धाः सु॒भरः॑ सु॒भरो॑ वयो॒धाः । \newline
26. सु॒भर॒ इति॑ सु - भरः॑ । \newline
27. व॒यो॒धाः श्रु॒ष्टी श्रु॒ष्टी व॑यो॒धा व॑यो॒धाः श्रु॒ष्टी । \newline
28. व॒यो॒धा इति॑ वयः - धाः । \newline
29. श्रु॒ष्टी वी॒रो वी॒रः श्रु॒ष्टी श्रु॒ष्टी वी॒रः । \newline
30. वी॒रो जा॑यते जायते वी॒रो वी॒रो जा॑यते । \newline
31. जा॒य॒ते॒ दे॒वका॑मो दे॒वका॑मो जायते जायते दे॒वका॑मः । \newline
32. दे॒वका॑म॒ इति॑ दे॒व - का॒मः॒ । \newline
33. प्र॒जाम् त्वष्टा॒ त्वष्टा᳚ प्र॒जाम् प्र॒जाम् त्वष्टा᳚ । \newline
34. प्र॒जामिति॑ प्र - जाम् । \newline
35. त्वष्टा॒ वि वि त्वष्टा॒ त्वष्टा॒ वि । \newline
36. वि ष्य॑तु स्यतु॒ वि वि ष्य॑तु । \newline
37. स्य॒तु॒ नाभि॒म् नाभिꣳ॑ स्यतु स्यतु॒ नाभि᳚म् । \newline
38. नाभि॑ म॒स्मे अ॒स्मे नाभि॒म् नाभि॑ म॒स्मे । \newline
39. अ॒स्मे अथाथा॒स्मे अ॒स्मे अथ॑ । \newline
40. अ॒स्मे इत्य॒स्मे । \newline
41. अथा॑ दे॒वाना᳚म् दे॒वाना॒ मथाथा॑ दे॒वाना᳚म् । \newline
42. दे॒वाना॒ मप्यपि॑ दे॒वाना᳚म् दे॒वाना॒ मपि॑ । \newline
43. अप्ये᳚ त्वे॒ त्वप्य प्ये॑तु । \newline
44. ए॒तु॒ पाथः॒ पाथ॑ एत्वेतु॒ पाथः॑ । \newline
45. पाथ॒ इति॒ पाथः॑ । \newline
46. प्र णो॑ नः॒ प्र प्र णः॑ । \newline
47. नो॒ दे॒वी दे॒वी नो॑ नो दे॒वी । \newline
48. दे॒व्या दे॒वी दे॒व्या । \newline
49. आ नो॑ न॒ आ नः॑ । \newline
50. नो॒ दि॒वो दि॒वो नो॑ नो दि॒वः । \newline
51. दि॒व इति॑ दि॒वः । \newline
52. पी॒पि॒वाꣳसꣳ॒॒ सर॑स्वतः॒ सर॑स्वतः पीपि॒वाꣳस॑म् पीपि॒वाꣳसꣳ॒॒ सर॑स्वतः । \newline
53. सर॑स्वतः॒ स्तनꣳ॒॒ स्तनꣳ॒॒ सर॑स्वतः॒ सर॑स्वतः॒ स्तन᳚म् । \newline
54. स्तनं॒ ॅयो यः स्तनꣳ॒॒ स्तनं॒ ॅयः । \newline
55. यो वि॒श्वद॑र्.शतो वि॒श्वद॑र्.शतो॒ यो यो वि॒श्वद॑र्.शतः । \newline
56. वि॒श्वद॑र्.शत॒ इति॑ वि॒श्व - द॒र्॒.श॒तः॒ । \newline
57. धु॒क्षी॒महि॑ प्र॒जाम् प्र॒जाम् धु॑क्षी॒महि॑ धुक्षी॒महि॑ प्र॒जाम् । \newline
58. प्र॒जा मिष॒ मिष॑म् प्र॒जाम् प्र॒जा मिष᳚म् । \newline
59. प्र॒जामिति॑ प्र - जाम् । \newline
60. इष॒मितीष᳚म् । \newline

\textbf{Ghana Paata } \newline

1. क॒र्म॒ण्यः॑ सु॒दक्षः॑ सु॒दक्षः॑ कर्म॒ण्यः॑ कर्म॒ण्यः॑ सु॒दक्षो॑ यु॒क्तग्रा॑वा यु॒क्तग्रा॑वा सु॒दक्षः॑ कर्म॒ण्यः॑ कर्म॒ण्यः॑ सु॒दक्षो॑ यु॒क्तग्रा॑वा । \newline
2. सु॒दक्षो॑ यु॒क्तग्रा॑वा यु॒क्तग्रा॑वा सु॒दक्षः॑ सु॒दक्षो॑ यु॒क्तग्रा॑वा॒ जाय॑ते॒ जाय॑ते यु॒क्तग्रा॑वा सु॒दक्षः॑ सु॒दक्षो॑ यु॒क्तग्रा॑वा॒ जाय॑ते । \newline
3. सु॒दक्ष॒ इति॑ सु - दक्षः॑ । \newline
4. यु॒क्तग्रा॑वा॒ जाय॑ते॒ जाय॑ते यु॒क्तग्रा॑वा यु॒क्तग्रा॑वा॒ जाय॑ते दे॒वका॑मो दे॒वका॑मो॒ जाय॑ते यु॒क्तग्रा॑वा यु॒क्तग्रा॑वा॒ जाय॑ते दे॒वका॑मः । \newline
5. यु॒क्तग्रा॒वेति॑ यु॒क्त - ग्रा॒वा॒ । \newline
6. जाय॑ते दे॒वका॑मो दे॒वका॑मो॒ जाय॑ते॒ जाय॑ते दे॒वका॑मः । \newline
7. दे॒वका॑म॒ इति॑ दे॒व - का॒मः॒ । \newline
8. शि॒व स्त्व॑ष्ट स्त्वष्टः शि॒वः शि॒व स्त्व॑ष्ट रि॒हे ह त्व॑ष्टः शि॒वः शि॒व स्त्व॑ष्ट रि॒ह । \newline
9. त्व॒ष्ट॒ रि॒हे ह त्व॑ ष्टस्त्वष्ट रि॒हेह त्व॑ष्ट स्त्वष्ट रि॒हा । \newline
10. इ॒हेहे हा ग॑हि ग॒ह्येहे हा ग॑हि । \newline
11. आ ग॑हि ग॒ह्या ग॑हि वि॒भुर् वि॒भुर् ग॒ह्या ग॑हि वि॒भुः । \newline
12. ग॒हि॒ वि॒भुर् वि॒भुर् ग॑हि गहि वि॒भुः पोषे॒ पोषे॑ वि॒भुर् ग॑हि गहि वि॒भुः पोषे᳚ । \newline
13. वि॒भुः पोषे॒ पोषे॑ वि॒भुर् वि॒भुः पोष॑ उ॒तोत पोषे॑ वि॒भुर् वि॒भुः पोष॑ उ॒त । \newline
14. वि॒भुरिति॑ वि - भुः । \newline
15. पोष॑ उ॒तोत पोषे॒ पोष॑ उ॒त त्मना॒ त्मनो॒त पोषे॒ पोष॑ उ॒त त्मना᳚ । \newline
16. उ॒त त्मना॒ त्मनो॒तोत त्मना᳚ । \newline
17. त्मनेति॒ त्मना᳚ । \newline
18. य॒ज्ञेय॑ज्ञे नो नो य॒ज्ञेय॑ज्ञे य॒ज्ञेय॑ज्ञे न॒ उदुन् नो॑ य॒ज्ञेय॑ज्ञे य॒ज्ञेय॑ज्ञे न॒ उत् । \newline
19. य॒ज्ञेय॑ज्ञ्॒ इति॑ य॒ज्ञे - य॒ज्ञे॒ । \newline
20. न॒ उदुन् नो॑ न॒ उद॑वा॒वोन् नो॑ न॒ उद॑व । \newline
21. उद॑वा॒ वोदुद॑व । \newline
22. अ॒वेत्य॑व । \newline
23. पि॒शङ्ग॑रूपः सु॒भरः॑ सु॒भरः॑ पि॒शङ्ग॑रूपः पि॒शङ्ग॑रूपः सु॒भरो॑ वयो॒धा व॑यो॒धाः सु॒भरः॑ पि॒शङ्ग॑रूपः पि॒शङ्ग॑रूपः सु॒भरो॑ वयो॒धाः । \newline
24. पि॒शङ्ग॑रूप॒ इति॑ पि॒शङ्ग॑ - रू॒पः॒ । \newline
25. सु॒भरो॑ वयो॒धा व॑यो॒धाः सु॒भरः॑ सु॒भरो॑ वयो॒धाः श्रु॒ष्टी श्रु॒ष्टी व॑यो॒धाः सु॒भरः॑ सु॒भरो॑ वयो॒धाः श्रु॒ष्टी । \newline
26. सु॒भर॒ इति॑ सु - भरः॑ । \newline
27. व॒यो॒धाः श्रु॒ष्टी श्रु॒ष्टी व॑यो॒धा व॑यो॒धाः श्रु॒ष्टी वी॒रो वी॒रः श्रु॒ष्टी व॑यो॒धा व॑यो॒धाः श्रु॒ष्टी वी॒रः । \newline
28. व॒यो॒धा इति॑ वयः - धाः । \newline
29. श्रु॒ष्टी वी॒रो वी॒रः श्रु॒ष्टी श्रु॒ष्टी वी॒रो जा॑यते जायते वी॒रः श्रु॒ष्टी श्रु॒ष्टी वी॒रो जा॑यते । \newline
30. वी॒रो जा॑यते जायते वी॒रो वी॒रो जा॑यते दे॒वका॑मो दे॒वका॑मो जायते वी॒रो वी॒रो जा॑यते दे॒वका॑मः । \newline
31. जा॒य॒ते॒ दे॒वका॑मो दे॒वका॑मो जायते जायते दे॒वका॑मः । \newline
32. दे॒वका॑म॒ इति॑ दे॒व - का॒मः॒ । \newline
33. प्र॒जाम् त्वष्टा॒ त्वष्टा᳚ प्र॒जाम् प्र॒जाम् त्वष्टा॒ वि वि त्वष्टा᳚ प्र॒जाम् प्र॒जाम् त्वष्टा॒ वि । \newline
34. प्र॒जामिति॑ प्र - जाम् । \newline
35. त्वष्टा॒ वि वि त्वष्टा॒ त्वष्टा॒ वि ष्य॑तु स्यतु॒ वि त्वष्टा॒ त्वष्टा॒ वि ष्य॑तु । \newline
36. वि ष्य॑तु स्यतु॒ वि वि ष्य॑तु॒ नाभि॒म् नाभिꣳ॑ स्यतु॒ वि वि ष्य॑तु॒ नाभि᳚म् । \newline
37. स्य॒तु॒ नाभि॒म् नाभिꣳ॑ स्यतु स्यतु॒ नाभि॑ म॒स्मे अ॒स्मे नाभिꣳ॑ स्यतु स्यतु॒ नाभि॑ म॒स्मे । \newline
38. नाभि॑ म॒स्मे अ॒स्मे नाभि॒म् नाभि॑ म॒स्मे अथाथा॒स्मे नाभि॒म् नाभि॑ म॒स्मे अथ॑ । \newline
39. अ॒स्मे अथाथा॒स्मे अ॒स्मे अथा॑ दे॒वाना᳚म् दे॒वाना॒ मथा॒स्मे अ॒स्मे अथा॑ दे॒वाना᳚म् । \newline
40. अ॒स्मे इत्य॒स्मे । \newline
41. अथा॑ दे॒वाना᳚म् दे॒वाना॒ मथाथा॑ दे॒वाना॒ मप्यपि॑ दे॒वाना॒ मथाथा॑ दे॒वाना॒ मपि॑ । \newline
42. दे॒वाना॒ मप्यपि॑ दे॒वाना᳚म् दे॒वाना॒ मप्ये᳚ त्वे॒त्वपि॑ दे॒वाना᳚म् दे॒वाना॒ मप्ये॑तु । \newline
43. अप्ये᳚ त्वे॒त्वप्यप्ये॑तु॒ पाथः॒ पाथ॑ ए॒त्वप्यप्ये॑तु॒ पाथः॑ । \newline
44. ए॒तु॒ पाथः॒ पाथ॑ एत्वेतु॒ पाथः॑ । \newline
45. पाथ॒ इति॒ पाथः॑ । \newline
46. प्र णो॑ नः॒ प्र प्र णो॑ दे॒वी दे॒वी नः॒ प्र प्र णो॑ दे॒वी । \newline
47. नो॒ दे॒वी दे॒वी नो॑ नो दे॒व्या दे॒वी नो॑ नो दे॒व्या । \newline
48. दे॒व्या दे॒वी दे॒व्या नो॑ न॒ आ दे॒वी दे॒व्या नः॑ । \newline
49. आ नो॑ न॒ आ नो॑ दि॒वो दि॒वो न॒ आ नो॑ दि॒वः । \newline
50. नो॒ दि॒वो दि॒वो नो॑ नो दि॒वः । \newline
51. दि॒व इति॑ दि॒वः । \newline
52. पी॒पि॒वाꣳसꣳ॒॒ सर॑स्वतः॒ सर॑स्वतः पीपि॒वाꣳस॑म् पीपि॒वाꣳसꣳ॒॒ सर॑स्वतः॒ स्तनꣳ॒॒ स्तनꣳ॒॒ सर॑स्वतः पीपि॒वाꣳस॑म् पीपि॒वाꣳसꣳ॒॒ सर॑स्वतः॒ स्तन᳚म् । \newline
53. सर॑स्वतः॒ स्तनꣳ॒॒ स्तनꣳ॒॒ सर॑स्वतः॒ सर॑स्वतः॒ स्तनं॒ ॅयो यः स्तनꣳ॒॒ सर॑स्वतः॒ सर॑स्वतः॒ स्तनं॒ ॅयः । \newline
54. स्तनं॒ ॅयो यः स्तनꣳ॒॒ स्तनं॒ ॅयो वि॒श्वद॑र्.शतो वि॒श्वद॑र्.शतो॒ यः स्तनꣳ॒॒ स्तनं॒ ॅयो वि॒श्वद॑र्.शतः । \newline
55. यो वि॒श्वद॑र्.शतो वि॒श्वद॑र्.शतो॒ यो यो वि॒श्वद॑र्.शतः । \newline
56. वि॒श्वद॑र्.शत॒ इति॑ वि॒श्व - द॒र्॒.श॒तः॒ । \newline
57. धु॒क्षी॒महि॑ प्र॒जाम् प्र॒जाम् धु॑क्षी॒महि॑ धुक्षी॒महि॑ प्र॒जा मिष॒ मिष॑म् प्र॒जाम् धु॑क्षी॒महि॑ धुक्षी॒महि॑ प्र॒जा मिष᳚म् । \newline
58. प्र॒जा मिष॒ मिष॑म् प्र॒जाम् प्र॒जा मिष᳚म् । \newline
59. प्र॒जामिति॑ प्र - जाम् । \newline
60. इष॒मितीष᳚म् । \newline
\pagebreak
\markright{ TS 3.1.11.3  \hfill https://www.vedavms.in \hfill}

\section{ TS 3.1.11.3 }

\textbf{TS 3.1.11.3 } \newline
\textbf{Samhita Paata} \newline

ये ते॑ सरस्व ऊ॒र्मयो॒ मधु॑मन्तो घृत॒श्चुतः॑ । तेषां᳚ ते सु॒म्नमी॑महे ॥ यस्य॑ व्र॒तं प॒शवो॒ यन्ति॒ सर्वे॒ यस्य॑ व्र॒तमु॑प॒तिष्ठ॑न्त॒ आपः॑ । यस्य॑ व्र॒ते पु॑ष्टि॒पति॒र्निवि॑ष्ट॒स्तꣳ सर॑स्वन्त॒मव॑से हुवेम ॥ दि॒व्यꣳ सु॑प॒र्णं ॅव॑य॒सं बृ॒हन्त॑म॒पां गर्भं॑ ॅवृष॒भमोष॑धीनां । अ॒भी॒प॒तो वृ॒ष्ट्या त॒र्पय॑न्तं॒ तꣳ सर॑स्वन्त॒मव॑से हुवेम ॥ सिनी॑वालि॒ पृथु॑ष्टुके॒ या दे॒वाना॒मसि॒ स्वसा᳚ । जु॒षस्व॑ ह॒व्य - [  ] \newline

\textbf{Pada Paata} \newline

ये । ते॒ । स॒र॒स्वः॒ । ऊ॒र्मयः॑ । मधु॑मन्त॒ इति॒ मधु॑ - म॒न्तः॒ । घृ॒त॒श्चुत॒ इति॑ घृत - श्चुतः॑ ॥ तेषा᳚म् । ते॒ । सु॒म्नम् । ई॒म॒हे॒ ॥ यस्य॑ । व्र॒तम् । प॒शवः॑ । यन्ति॑ । सर्वे᳚ । यस्य॑ । व्र॒तम् । उ॒प॒तिष्ठ॑न्त॒ इत्यु॑प -तिष्ठ॑न्ते । आपः॑ ॥ यस्य॑ । व्र॒ते । पु॒ष्टि॒पति॒रिति॑ पुष्टि-पतिः॑ । निवि॑ष्ट॒ इति॒ नि - वि॒ष्टः॒ । तम् । सर॑स्वन्तम् । अव॑से । ह॒वे॒म॒ ॥ दि॒व्यम् । सु॒प॒र्णमिति॑ सु - प॒र्णम् । व॒य॒सम् । बृ॒हन्त᳚म् । अ॒पाम् । गर्भ᳚म् । वृ॒ष॒भम् । ओष॑धीनाम् ॥ अ॒भी॒प॒तः । वृ॒ष्ट्या । त॒र्पय॑न्तम् । तम् । सर॑स्वन्तम् । अव॑से । हु॒वे॒म॒ ॥ सिनी॑वालि । पृथु॑ष्टुक॒ इति॒ पृथु॑ - स्तु॒के॒ । या । दे॒वाना᳚म् । असि॑ । स्वसा᳚ ॥ जु॒षस्व॑ । ह॒व्यम् ।  \newline


\textbf{Krama Paata} \newline

ये ते᳚ । ते॒ स॒र॒स्वः॒ । स॒र॒स्व॒ ऊ॒र्मयः॑ । ऊ॒र्मयो॒ मधु॑मन्तः । मधु॑मन्तो घृत॒श्चुतः॑ । मधु॑मन्त॒ इति॒ मधु॑ - म॒न्तः॒ । घृ॒त॒श्चुत॒ इति॑ घृत - श्चुतः॑ ॥ तेषा᳚म् ते । ते॒ सु॒म्नम् । सु॒म्नमी॑महे । ई॒म॒ह॒ इती॑महे ॥ यस्य॑ व्र॒तम् । व्र॒तम् प॒शवः॑ । प॒शवो॒ यन्ति॑ । यन्ति॒ सर्वे᳚ । सर्वे॒ यस्य॑ । यस्य॑ व्र॒तम् । व्र॒तमु॑प॒तिष्ठ॑न्ते । उ॒प॒तिष्ठ॑न्त॒ आपः॑ । उ॒प॒तिष्ठ॑न्त॒ इत्यु॑प - तिष्ठ॑न्ते । आप॒ इत्यापः॑ ॥ यस्य॑ व्र॒ते । व्र॒ते पु॑ष्टि॒पतिः॑ । पु॒ष्टि॒पति॒र् निवि॑ष्टः । पु॒ष्टि॒पति॒रिति॑ पुष्टि - पतिः॑ । निवि॑ष्ट॒ स्तम् । निवि॑ष्ट॒ इति॒ नि - वि॒ष्टः॒ । तꣳ सर॑स्वन्तम् । सर॑स्वन्त॒मव॑से । अव॑से हुवेम । हु॒वे॒मेति॑ हुवेम ॥ दि॒व्यꣳ सु॑प॒र्णम् । सु॒प॒र्णं ॅव॑य॒सम् । सु॒प॒र्णमिति॑ सु - प॒र्णम् । व॒य॒सम् बृ॒हन्त᳚म् । बृ॒हन्त॑म॒पाम् । अ॒पाम् गर्भ᳚म् । गर्भं॑ ॅवृष॒भम् । वृ॒ष॒भमोष॑धीनां । ओष॑धीना॒मित्योष॑धीनाम् ॥ अ॒भी॒प॒तो वृ॒ष्ट्या । वृ॒ष्ट्या त॒र्पय॑न्तम् । त॒र्पय॑न्त॒म् तम् । तꣳ सर॑स्वन्तम् । सर॑स्वन्त॒मव॑से । अव॑से हुवेम । हु॒वे॒मेति॑ हुवेम ॥ सिनी॑वालि॒ पृथु॑ष्टुके । पृथु॑ष्टुके॒ या । पृथु॑ष्टुक॒ इति॒ पृथु॑ - स्तु॒के॒ । या दे॒वाना᳚म् । दे॒वाना॒मसि॑ । असि॒ स्वसा᳚ । स्वसेति॒ स्वसा᳚ ॥ जु॒षस्व॑ ह॒व्यम् । ह॒व्यमाहु॑तम् \newline

\textbf{Jatai Paata} \newline

1. ये ते॑ ते॒ ये ये ते᳚ । \newline
2. ते॒ स॒र॒स्वः॒ स॒र॒स्व॒ स्ते॒ ते॒ स॒र॒स्वः॒ । \newline
3. स॒र॒स्व॒ ऊ॒र्मय॑ ऊ॒र्मयः॑ सरस्वः सरस्व ऊ॒र्मयः॑ । \newline
4. ऊ॒र्मयो॒ मधु॑मन्तो॒ मधु॑मन्त ऊ॒र्मय॑ ऊ॒र्मयो॒ मधु॑मन्तः । \newline
5. मधु॑मन्तो घृत॒श्चुतो॑ घृत॒श्चुतो॒ मधु॑मन्तो॒ मधु॑मन्तो घृत॒श्चुतः॑ । \newline
6. मधु॑मन्त॒ इति॒ मधु॑ - म॒न्तः॒ । \newline
7. घृ॒त॒श्चुत॒ इति॑ घृत - श्चुतः॑ । \newline
8. तेषा᳚म् ते ते॒ तेषा॒म् तेषा᳚म् ते । \newline
9. ते॒ सु॒म्नꣳ सु॒म्नम् ते॑ ते सु॒म्नम् । \newline
10. सु॒म्न मी॑मह ईमहे सु॒म्नꣳ सु॒म्न मी॑महे । \newline
11. ई॒म॒ह॒ इती॑महे । \newline
12. यस्य॑ व्र॒तं ॅव्र॒तं ॅयस्य॒ यस्य॑ व्र॒तम् । \newline
13. व्र॒तम् प॒शवः॑ प॒शवो᳚ व्र॒तं ॅव्र॒तम् प॒शवः॑ । \newline
14. प॒शवो॒ यन्ति॒ यन्ति॑ प॒शवः॑ प॒शवो॒ यन्ति॑ । \newline
15. यन्ति॒ सर्वे॒ सर्वे॒ यन्ति॒ यन्ति॒ सर्वे᳚ । \newline
16. सर्वे॒ यस्य॒ यस्य॒ सर्वे॒ सर्वे॒ यस्य॑ । \newline
17. यस्य॑ व्र॒तं ॅव्र॒तं ॅयस्य॒ यस्य॑ व्र॒तम् । \newline
18. व्र॒त मु॑प॒तिष्ठ॑न्त उप॒तिष्ठ॑न्ते व्र॒तं ॅव्र॒त मु॑प॒तिष्ठ॑न्ते । \newline
19. उ॒प॒तिष्ठ॑न्त॒ आप॒ आप॑ उप॒तिष्ठ॑न्त उप॒तिष्ठ॑न्त॒ आपः॑ । \newline
20. उ॒प॒तिष्ठ॑न्त॒ इत्यु॑प - तिष्ठ॑न्ते । \newline
21. आप॒ इत्यापः॑ । \newline
22. यस्य॑ व्र॒ते व्र॒ते यस्य॒ यस्य॑ व्र॒ते । \newline
23. व्र॒ते पु॑ष्टि॒पतिः॑ पुष्टि॒पति॑र् व्र॒ते व्र॒ते पु॑ष्टि॒पतिः॑ । \newline
24. पु॒ष्टि॒पति॒र् निवि॑ष्टो॒ निवि॑ष्टः पुष्टि॒पतिः॑ पुष्टि॒पति॒र् निवि॑ष्टः । \newline
25. पु॒ष्टि॒पति॒रिति॑ पुष्टि - पतिः॑ । \newline
26. निवि॑ष्ट॒ स्तम् तम् निवि॑ष्टो॒ निवि॑ष्ट॒ स्तम् । \newline
27. निवि॑ष्ट॒ इति॒ नि - वि॒ष्टः॒ । \newline
28. तꣳ सर॑स्वन्तꣳ॒॒ सर॑स्वन्त॒म् तम् तꣳ सर॑स्वन्तम् । \newline
29. सर॑स्वन्त॒ मव॒से ऽव॑से॒ सर॑स्वन्तꣳ॒॒ सर॑स्वन्त॒ मव॑से । \newline
30. अव॑से हुवेम हुवे॒मा व॒से ऽव॑से हुवेम । \newline
31. हु॒वे॒मेति॑ हुवेम । \newline
32. दि॒व्यꣳ सु॑प॒र्णꣳ सु॑प॒र्णम् दि॒व्यम् दि॒व्यꣳ सु॑प॒र्णम् । \newline
33. सु॒प॒र्णं ॅव॑य॒सं ॅव॑य॒सꣳ सु॑प॒र्णꣳ सु॑प॒र्णं ॅव॑य॒सम् । \newline
34. सु॒प॒र्णमिति॑ सु - प॒र्णम् । \newline
35. व॒य॒सम् बृ॒हन्त॑म् बृ॒हन्तं॑ ॅवय॒सं ॅव॑य॒सम् बृ॒हन्त᳚म् । \newline
36. बृ॒हन्त॑ म॒पा म॒पाम् बृ॒हन्त॑म् बृ॒हन्त॑ म॒पाम् । \newline
37. अ॒पाम् गर्भ॒म् गर्भ॑ म॒पा म॒पाम् गर्भ᳚म् । \newline
38. गर्भं॑ ॅवृष॒भं ॅवृ॑ष॒भम् गर्भ॒म् गर्भं॑ ॅवृष॒भम् । \newline
39. वृ॒ष॒भ मोष॑धीना॒ मोष॑धीनां ॅवृष॒भं ॅवृ॑ष॒भ मोष॑धीनाम् । \newline
40. ओष॑धीना॒मित्योष॑धीनाम् । \newline
41. अ॒भी॒प॒तो वृ॒ष्ट्या वृ॒ष्ट्या ऽभी॑प॒तो अ॑भीप॒तो वृ॒ष्ट्या । \newline
42. वृ॒ष्ट्या त॒र्पय॑न्तम् त॒र्पय॑न्तं ॅवृ॒ष्ट्या वृ॒ष्ट्या त॒र्पय॑न्तम् । \newline
43. त॒र्पय॑न्त॒म् तम् तम् त॒र्पय॑न्तम् त॒र्पय॑न्त॒म् तम् । \newline
44. तꣳ सर॑स्वन्तꣳ॒॒ सर॑स्वन्त॒म् तम् तꣳ सर॑स्वन्तम् । \newline
45. सर॑स्वन्त॒ मव॒से ऽव॑से॒ सर॑स्वन्तꣳ॒॒ सर॑स्वन्त॒ मव॑से । \newline
46. अव॑से हुवेम हुवे॒मा व॒से ऽव॑से हुवेम । \newline
47. हु॒वे॒मेति॑ हुवेम । \newline
48. सिनी॑वालि॒ पृथु॑ष्टुके॒ पृथु॑ष्टुके॒ सिनी॑वालि॒ सिनी॑वालि॒ पृथु॑ष्टुके । \newline
49. पृथु॑ष्टुके॒ या या पृथु॑ष्टुके॒ पृथु॑ष्टुके॒ या । \newline
50. पृथु॑ष्टुक॒ इति॒ पृथु॑ - स्तु॒के॒ । \newline
51. या दे॒वाना᳚म् दे॒वानां॒ ॅया या दे॒वाना᳚म् । \newline
52. दे॒वाना॒ मस्यसि॑ दे॒वाना᳚म् दे॒वाना॒ मसि॑ । \newline
53. असि॒ स्वसा॒ स्वसा ऽस्यसि॒ स्वसा᳚ । \newline
54. स्वसेति॒ स्वसा᳚ । \newline
55. जु॒षस्व॑ ह॒व्यꣳ ह॒व्यम् जु॒षस्व॑ जु॒षस्व॑ ह॒व्यम् । \newline
56. ह॒व्य माहु॑त॒ माहु॑तꣳ ह॒व्यꣳ ह॒व्य माहु॑तम् । \newline

\textbf{Ghana Paata } \newline

1. ये ते॑ ते॒ ये ये ते॑ सरस्वः सरस्व स्ते॒ ये ये ते॑ सरस्वः । \newline
2. ते॒ स॒र॒स्वः॒ स॒र॒स्व॒ स्ते॒ ते॒ स॒र॒स्व॒ ऊ॒र्मय॑ ऊ॒र्मयः॑ सरस्व स्ते ते सरस्व ऊ॒र्मयः॑ । \newline
3. स॒र॒स्व॒ ऊ॒र्मय॑ ऊ॒र्मयः॑ सरस्वः सरस्व ऊ॒र्मयो॒ मधु॑मन्तो॒ मधु॑मन्त ऊ॒र्मयः॑ सरस्वः सरस्व ऊ॒र्मयो॒ मधु॑मन्तः । \newline
4. ऊ॒र्मयो॒ मधु॑मन्तो॒ मधु॑मन्त ऊ॒र्मय॑ ऊ॒र्मयो॒ मधु॑मन्तो घृत॒श्चुतो॑ घृत॒श्चुतो॒ मधु॑मन्त ऊ॒र्मय॑ ऊ॒र्मयो॒ मधु॑मन्तो घृत॒श्चुतः॑ । \newline
5. मधु॑मन्तो घृत॒श्चुतो॑ घृत॒श्चुतो॒ मधु॑मन्तो॒ मधु॑मन्तो घृत॒श्चुतः॑ । \newline
6. मधु॑मन्त॒ इति॒ मधु॑ - म॒न्तः॒ । \newline
7. घृ॒त॒श्चुत॒ इति॑ घृत - श्चुतः॑ । \newline
8. तेषा᳚म् ते ते॒ तेषा॒म् तेषा᳚म् ते सु॒म्नꣳ सु॒म्नम् ते॒ तेषा॒म् तेषा᳚म् ते सु॒म्नम् । \newline
9. ते॒ सु॒म्नꣳ सु॒म्नम् ते॑ ते सु॒म्न मी॑मह ईमहे सु॒म्नम् ते॑ ते सु॒म्न मी॑महे । \newline
10. सु॒म्न मी॑मह ईमहे सु॒म्नꣳ सु॒म्न मी॑महे । \newline
11. ई॒म॒ह॒ इती॑महे । \newline
12. यस्य॑ व्र॒तं ॅव्र॒तं ॅयस्य॒ यस्य॑ व्र॒तम् प॒शवः॑ प॒शवो᳚ व्र॒तं ॅयस्य॒ यस्य॑ व्र॒तम् प॒शवः॑ । \newline
13. व्र॒तम् प॒शवः॑ प॒शवो᳚ व्र॒तं ॅव्र॒तम् प॒शवो॒ यन्ति॒ यन्ति॑ प॒शवो᳚ व्र॒तं ॅव्र॒तम् प॒शवो॒ यन्ति॑ । \newline
14. प॒शवो॒ यन्ति॒ यन्ति॑ प॒शवः॑ प॒शवो॒ यन्ति॒ सर्वे॒ सर्वे॒ यन्ति॑ प॒शवः॑ प॒शवो॒ यन्ति॒ सर्वे᳚ । \newline
15. यन्ति॒ सर्वे॒ सर्वे॒ यन्ति॒ यन्ति॒ सर्वे॒ यस्य॒ यस्य॒ सर्वे॒ यन्ति॒ यन्ति॒ सर्वे॒ यस्य॑ । \newline
16. सर्वे॒ यस्य॒ यस्य॒ सर्वे॒ सर्वे॒ यस्य॑ व्र॒तं ॅव्र॒तं ॅयस्य॒ सर्वे॒ सर्वे॒ यस्य॑ व्र॒तम् । \newline
17. यस्य॑ व्र॒तं ॅव्र॒तं ॅयस्य॒ यस्य॑ व्र॒त मु॑प॒तिष्ठ॑न्त उप॒तिष्ठ॑न्ते व्र॒तं ॅयस्य॒ यस्य॑ व्र॒त मु॑प॒तिष्ठ॑न्ते । \newline
18. व्र॒त मु॑प॒तिष्ठ॑न्त उप॒तिष्ठ॑न्ते व्र॒तं ॅव्र॒त मु॑प॒तिष्ठ॑न्त॒ आप॒ आप॑ उप॒तिष्ठ॑न्ते व्र॒तं ॅव्र॒त मु॑प॒तिष्ठ॑न्त॒ आपः॑ । \newline
19. उ॒प॒तिष्ठ॑न्त॒ आप॒ आप॑ उप॒तिष्ठ॑न्त उप॒तिष्ठ॑न्त॒ आपः॑ । \newline
20. उ॒प॒तिष्ठ॑न्त॒ इत्यु॑प - तिष्ठ॑न्ते । \newline
21. आप॒ इत्यापः॑ । \newline
22. यस्य॑ व्र॒ते व्र॒ते यस्य॒ यस्य॑ व्र॒ते पु॑ष्टि॒पतिः॑ पुष्टि॒पति॑र् व्र॒ते यस्य॒ यस्य॑ व्र॒ते पु॑ष्टि॒पतिः॑ । \newline
23. व्र॒ते पु॑ष्टि॒पतिः॑ पुष्टि॒पति॑र् व्र॒ते व्र॒ते पु॑ष्टि॒पति॒र् निवि॑ष्टो॒ निवि॑ष्टः पुष्टि॒पति॑र् व्र॒ते व्र॒ते पु॑ष्टि॒पति॒र् निवि॑ष्टः । \newline
24. पु॒ष्टि॒पति॒र् निवि॑ष्टो॒ निवि॑ष्टः पुष्टि॒पतिः॑ पुष्टि॒पति॒र् निवि॑ष्ट॒ स्तम् तन्निवि॑ष्टः पुष्टि॒पतिः॑ पुष्टि॒पति॒र् निवि॑ष्ट॒ स्तम् । \newline
25. पु॒ष्टि॒पति॒रिति॑ पुष्टि - पतिः॑ । \newline
26. निवि॑ष्ट॒ स्तम् तम् निवि॑ष्टो॒ निवि॑ष्ट॒ स्तꣳ सर॑स्वन्तꣳ॒॒ सर॑स्वन्त॒म् तम् निवि॑ष्टो॒ निवि॑ष्ट॒ स्तꣳ सर॑स्वन्तम् । \newline
27. निवि॑ष्ट॒ इति॒ नि - वि॒ष्टः॒ । \newline
28. तꣳ सर॑स्वन्तꣳ॒॒ सर॑स्वन्त॒म् तम् तꣳ सर॑स्वन्त॒ मव॒से ऽव॑से॒ सर॑स्वन्त॒म् तम् तꣳ सर॑स्वन्त॒ मव॑से । \newline
29. सर॑स्वन्त॒ मव॒से ऽव॑से॒ सर॑स्वन्तꣳ॒॒ सर॑स्वन्त॒ मव॑से हुवेम हुवे॒माव॑से॒ सर॑स्वन्तꣳ॒॒ सर॑स्वन्त॒ मव॑से हुवेम । \newline
30. अव॑से हुवेम हुवे॒माव॒से ऽव॑से हुवेम । \newline
31. हु॒वे॒मेति॑ हुवेम । \newline
32. दि॒व्यꣳ सु॑प॒र्णꣳ सु॑प॒र्णम् दि॒व्यम् दि॒व्यꣳ सु॑प॒र्णं ॅव॑य॒सं ॅव॑य॒सꣳ सु॑प॒र्णम् दि॒व्यम् दि॒व्यꣳ सु॑प॒र्णं ॅव॑य॒सम् । \newline
33. सु॒प॒र्णं ॅव॑य॒सं ॅव॑य॒सꣳ सु॑प॒र्णꣳ सु॑प॒र्णं ॅव॑य॒सम् बृ॒हन्त॑म् बृ॒हन्तं॑ ॅवय॒सꣳ सु॑प॒र्णꣳ सु॑प॒र्णं ॅव॑य॒सम् बृ॒हन्त᳚म् । \newline
34. सु॒प॒र्णमिति॑ सु - प॒र्णम् । \newline
35. व॒य॒सम् बृ॒हन्त॑म् बृ॒हन्तं॑ ॅवय॒सं ॅव॑य॒सम् बृ॒हन्त॑ म॒पा म॒पाम् बृ॒हन्तं॑ ॅवय॒सं ॅव॑य॒सम् बृ॒हन्त॑ म॒पाम् । \newline
36. बृ॒हन्त॑ म॒पा म॒पाम् बृ॒हन्त॑म् बृ॒हन्त॑ म॒पाम् गर्भ॒म् गर्भ॑ म॒पाम् बृ॒हन्त॑म् बृ॒हन्त॑ म॒पाम् गर्भ᳚म् । \newline
37. अ॒पाम् गर्भ॒म् गर्भ॑ म॒पा म॒पाम् गर्भं॑ ॅवृष॒भं ॅवृ॑ष॒भम् गर्भ॑ म॒पा म॒पाम् गर्भं॑ ॅवृष॒भम् । \newline
38. गर्भं॑ ॅवृष॒भं ॅवृ॑ष॒भम् गर्भ॒म् गर्भं॑ ॅवृष॒भ मोष॑धीना॒ मोष॑धीनां ॅवृष॒भम् गर्भ॒म् गर्भं॑ ॅवृष॒भ मोष॑धीनाम् । \newline
39. वृ॒ष॒भ मोष॑धीना॒ मोष॑धीनां ॅवृष॒भं ॅवृ॑ष॒भ मोष॑धीनाम् । \newline
40. ओष॑धीना॒मित्योष॑धीनाम् । \newline
41. अ॒भी॒प॒तो वृ॒ष्ट्या वृ॒ष्ट्या ऽभी॑प॒तो अ॑भीप॒तो वृ॒ष्ट्या त॒र्पय॑न्तम् त॒र्पय॑न्तं 
ॅवृ॒ष्ट्या ऽभी॑प॒तो अ॑भीप॒तो वृ॒ष्ट्या त॒र्पय॑न्तम् । \newline
42. वृ॒ष्ट्या त॒र्पय॑न्तम् त॒र्पय॑न्तं ॅवृ॒ष्ट्या वृ॒ष्ट्या त॒र्पय॑न्त॒म् तम् तम् त॒र्पय॑न्तं ॅवृ॒ष्ट्या वृ॒ष्ट्या त॒र्पय॑न्त॒म् तम् । \newline
43. त॒र्पय॑न्त॒म् तम् तम् त॒र्पय॑न्तम् त॒र्पय॑न्त॒म् तꣳ सर॑स्वन्तꣳ॒॒ सर॑स्वन्त॒म् तम् त॒र्पय॑न्तम् त॒र्पय॑न्त॒म् तꣳ सर॑स्वन्तम् । \newline
44. तꣳ सर॑स्वन्तꣳ॒॒ सर॑स्वन्त॒म् तम् तꣳ सर॑स्वन्त॒ मव॒से ऽव॑से॒ सर॑स्वन्त॒म् तम् तꣳ सर॑स्वन्त॒ मव॑से । \newline
45. सर॑स्वन्त॒ मव॒से ऽव॑से॒ सर॑स्वन्तꣳ॒॒ सर॑स्वन्त॒ मव॑से हुवेम हुवे॒माव॑से॒ सर॑स्वन्तꣳ॒॒ सर॑स्वन्त॒ मव॑से हुवेम । \newline
46. अव॑से हुवेम हुवे॒मा व॒से ऽव॑से हुवेम । \newline
47. हु॒वे॒मेति॑ हुवेम । \newline
48. सिनी॑वालि॒ पृथु॑ष्टुके॒ पृथु॑ष्टुके॒ सिनी॑वालि॒ सिनी॑वालि॒ पृथु॑ष्टुके॒ या या पृथु॑ष्टुके॒ सिनी॑वालि॒ सिनी॑वालि॒ पृथु॑ष्टुके॒ या । \newline
49. पृथु॑ष्टुके॒ या या पृथु॑ष्टुके॒ पृथु॑ष्टुके॒ या दे॒वाना᳚म् दे॒वानां॒ ॅया पृथु॑ष्टुके॒ पृथु॑ष्टुके॒ या दे॒वाना᳚म् । \newline
50. पृथु॑ष्टुक॒ इति॒ पृथु॑ - स्तु॒के॒ । \newline
51. या दे॒वाना᳚म् दे॒वानां॒ ॅया या दे॒वाना॒ मस्यसि॑ दे॒वानां॒ ॅया या दे॒वाना॒ मसि॑ । \newline
52. दे॒वाना॒ मस्यसि॑ दे॒वाना᳚म् दे॒वाना॒ मसि॒ स्वसा॒ स्वसा ऽसि॑ दे॒वाना᳚म् दे॒वाना॒ मसि॒ स्वसा᳚ । \newline
53. असि॒ स्वसा॒ स्वसा ऽस्यसि॒ स्वसा᳚ । \newline
54. स्वसेति॒ स्वसा᳚ । \newline
55. जु॒षस्व॑ ह॒व्यꣳ ह॒व्यम् जु॒षस्व॑ जु॒षस्व॑ ह॒व्य माहु॑त॒ माहु॑तꣳ ह॒व्यम् जु॒षस्व॑ जु॒षस्व॑ ह॒व्य माहु॑तम् । \newline
56. ह॒व्य माहु॑त॒ माहु॑तꣳ ह॒व्यꣳ ह॒व्य माहु॑तम् प्र॒जाम् प्र॒जा माहु॑तꣳ ह॒व्यꣳ ह॒व्य माहु॑तम् प्र॒जाम् । \newline
\pagebreak
\markright{ TS 3.1.11.4  \hfill https://www.vedavms.in \hfill}

\section{ TS 3.1.11.4 }

\textbf{TS 3.1.11.4 } \newline
\textbf{Samhita Paata} \newline

माहु॑तं प्र॒जां दे॑वि दिदिड्ढि नः ॥ या सु॑पा॒णिः स्व॑ङ्गु॒रिः सु॒षूमा॑ बहु॒सूव॑री । तस्यै॑ वि॒श्पत्नि॑यै ह॒विः सि॑नीवा॒ल्यै जु॑होतन ॥ इन्द्रं॑ ॅवो वि॒श्वत॒स्परी>5, न्द्रं॒ नरः॑>6 ॥ असि॑तवर्णा॒ हर॑यः सुप॒र्णा मिहो॒ वसा॑ना॒ दिव॒मुत् प॑तन्ति ॥ त आऽव॑वृत्र॒न्थ् सद॑नानि कृ॒त्वाऽऽदित् पृ॑थि॒वी घृ॒तैर्व्यु॑द्यते ॥ हिर॑ण्यकेशो॒ रज॑सो विसा॒रेऽहि॒र्धुनि॒र्वात॑ इव॒ ध्रजी॑मान् । शुचि॑भ्राजा उ॒षसो॒ - [  ] \newline

\textbf{Pada Paata} \newline

आहु॑त॒मित्या - हु॒त॒म् । प्र॒जामिति॑ प्र-जाम् । दे॒वि॒ । दि॒दि॒ड्ढि॒ । नः॒ ॥ या । सु॒पा॒णिरिति॑ सु - पा॒णिः । स्व॒ङ्गु॒रिरिति॑ सु - अ॒ङ्गु॒रिः । सु॒षूमेति॑ सु - सूमा᳚ । ब॒हु॒सूव॒रीति॑ बहु - सूव॑री ॥ तस्यै᳚ । वि॒श्पत्नि॑यै । ह॒विः । सि॒नी॒वा॒ल्यै । जु॒हो॒त॒न॒ ॥ इन्द्र᳚म् । वः॒ । वि॒श्वतः॑ । परीति॑ । इन्द्र᳚म् । नरः॑ ॥ असि॑तवर्णा॒ इत्यसि॑त - व॒र्णाः॒ । हर॑यः । सु॒प॒र्णा इति॑ सु - प॒र्णाः । मिहः॑ । वसा॑नाः । दिव᳚म् । उदिति॑ । प॒त॒न्ति॒ ॥ ते । एति॑ । अ॒व॒वृ॒त्र॒न्न् । सद॑नानि । कृ॒त्वा । आत् । इत् । पृ॒थि॒वी । घृ॒तैः । वीति॑ । उ॒द्य॒ते॒ ॥ हिर॑ण्यकेश॒ इति॒ हिर॑ण्य - के॒शः॒ । रज॑सः । वि॒सा॒र इति॑ वि - सा॒रे । अहिः॑ । धुनिः॑ । वातः॑ । इ॒व॒ । ध्रजी॑मान् ॥ शुचि॑भ्राजा॒ इति॒ शुचि॑ - भ्रा॒जाः॒ । उ॒षसः॑ ।  \newline


\textbf{Krama Paata} \newline

आहु॑तम् प्र॒जाम् । आहु॑त॒मित्या - हु॒त॒म् । प्र॒जाम् दे॑वि । प्र॒जामिति॑ प्र - जाम् । दे॒वि॒ दि॒दि॒ड्ढि॒ । दि॒दि॒ड्ढि॒ नः॒ । न॒ इति॑ नः ॥ या सु॑पा॒णिः । सु॒पा॒णिः स्व॑ङ्गु॒रिः । सु॒पा॒णिरिति॑ सु - पा॒णिः । स्व॒ङ्गु॒रिः सु॒षूमा᳚ । स्व॒ङ्गु॒रिति॑ सु - अ॒ङ्गु॒रिः । सु॒षूमा॑ बहु॒सूव॑री । सु॒षूमेति॑ सु - सूमा᳚ । ब॒॒हु॒सूव॒रीति॑ बहु - सूव॑री ॥ तस्यै॑ वि॒श्पत्नि॑यै । वि॒श्पत्नि॑यै ह॒विः । ह॒विः सि॑नीवा॒ल्यै । सि॒नी॒वा॒ल्यै जु॑होतन । जु॒हो॒त॒नेति॑ जुहोतन ॥ इन्द्रं॑ ॅवः । वो॒ वि॒श्वतः॑ । वि॒श्वत॒स्परि॑ । परीन्द्र᳚म् । इन्द्र॒म् नरः॑ । नर॒ इति॒ नरः॑ ॥ असि॑तवर्णा॒ हर॑यः । असि॑तवर्णा॒ इत्यसि॑त - व॒र्णाः॒ । हर॑यः सुप॒र्णाः । सु॒प॒र्णा मिहः॑ । सु॒र्प॒णा इति॑ सु - प॒र्णाः । मिहो॒ वसा॑नाः । वसा॑ना॒ दिव᳚म् । दिव॒मुत् । उत् प॑तन्ति । प॒त॒न्तीति॑ पतन्ति ॥ त आ । आऽव॑वृत्रन्न् । अ॒व॒वृ॒त्र॒न्थ् सद॑नानि । सद॑नानि कृ॒त्वा । कृ॒त्वा ऽऽत् । आदित् । इत् पृ॑थि॒वी । पृ॒थि॒वी घृ॒तैः । घृ॒तैर् वि । व्यु॑द्यते । उ॒द्य॒त॒ इत्यु॑द्यते ॥ हिर॑ण्यकेशो॒ रज॑सः । हिर॑ण्यकेश॒ इति॒ हिर॑ण्य - के॒शः॒ । रज॑सो विसा॒रे । वि॒सा॒रेऽहिः॑ । वि॒सा॒र इति॑ वि - सा॒रे । अहि॒र् धुनिः॑ । धुनि॒र् वातः॑ । वात॑ इव । इ॒व॒ ध्रजी॑मान् । ध्रजी॑मा॒निति॒ ध्रजी॑मान् ॥ शुचि॑भ्राजा उ॒षसः॑ । शुचि॑भ्राजा॒ इति॒ शुचि॑ - भ्रा॒जाः॒ । उ॒षसो॒ नवे॑दाः \newline

\textbf{Jatai Paata} \newline

1. आहु॑तम् प्र॒जाम् प्र॒जा माहु॑त॒ माहु॑तम् प्र॒जाम् । \newline
2. आहु॑त॒मित्या - हु॒त॒म् । \newline
3. प्र॒जाम् दे॑वि देवि प्र॒जाम् प्र॒जाम् दे॑वि । \newline
4. प्र॒जामिति॑ प्र - जाम् । \newline
5. दे॒वि॒ दि॒दि॒ड्ढि॒ दि॒दि॒ड्ढि॒ दे॒वि॒ दे॒वि॒ दि॒दि॒ड्ढि॒ । \newline
6. दि॒दि॒ड्ढि॒ नो॒ नो॒ दि॒दि॒ड्ढि॒ दि॒दि॒ड्ढि॒ नः॒ । \newline
7. न॒ इति॑ नः । \newline
8. या सु॑पा॒णिः सु॑पा॒णिर् या या सु॑पा॒णिः । \newline
9. सु॒पा॒णिः स्व॑ङ्गु॒रिः स्व॑ङ्गु॒रिः सु॑पा॒णिः सु॑पा॒णिः स्व॑ङ्गु॒रिः । \newline
10. सु॒पा॒णिरिति॑ सु - पा॒णिः । \newline
11. स्व॒ङ्गु॒रिः सु॒षूमा॑ सु॒षूमा᳚ स्वङ्गु॒रिः स्व॑ङ्गु॒रिः सु॒षूमा᳚ । \newline
12. स्व॒ङ्गु॒रिरिति॑ सु - अ॒ङ्गु॒रिः । \newline
13. सु॒षूमा॑ बहु॒सूव॑री बहु॒सूव॑री सु॒षूमा॑ सु॒षूमा॑ बहु॒सूव॑री । \newline
14. सु॒षूमेति॑ सु - सूमा᳚ । \newline
15. ब॒हु॒सूव॒रीति॑ बहु - सूव॑री । \newline
16. तस्यै॑ वि॒श्पत्नि॑यै वि॒श्पत्नि॑यै॒ तस्यै॒ तस्यै॑ वि॒श्पत्नि॑यै । \newline
17. वि॒श्पत्नि॑यै ह॒विर्. ह॒विर् वि॒श्पत्नि॑यै वि॒श्पत्नि॑यै ह॒विः । \newline
18. ह॒विः सि॑नीवा॒ल्यै सि॑नीवा॒ल्यै ह॒विर्. ह॒विः सि॑नीवा॒ल्यै । \newline
19. सि॒नी॒वा॒ल्यै जु॑होतन जुहोतन सिनीवा॒ल्यै सि॑नीवा॒ल्यै जु॑होतन । \newline
20. जु॒हो॒त॒नेति॑ जुहोतन । \newline
21. इन्द्रं॑ ॅवो व॒ इन्द्र॒ मिन्द्रं॑ ॅवः । \newline
22. वो॒ वि॒श्वतो॑ वि॒श्वतो॑ वो वो वि॒श्वतः॑ । \newline
23. वि॒श्वत॒ स्परि॒ परि॑ वि॒श्वतो॑ वि॒श्वत॒ स्परि॑ । \newline
24. परीन्द्र॒ मिन्द्र॒म् परि॒ परीन्द्र᳚म् । \newline
25. इन्द्र॒म् नरो॒ नर॒ इन्द्र॒ मिन्द्र॒म् नरः॑ । \newline
26. नर॒ इति॒ नरः॑ । \newline
27. असि॑तवर्णा॒ हर॑यो॒ हर॑यो॒ असि॑तवर्णा॒ असि॑तवर्णा॒ हर॑यः । \newline
28. असि॑तवर्णा॒ इत्यसि॑त - व॒र्णाः॒ । \newline
29. हर॑यः सुप॒र्णाः सु॑प॒र्णा हर॑यो॒ हर॑यः सुप॒र्णाः । \newline
30. सु॒प॒र्णा मिहो॒ मिहः॑ सुप॒र्णाः सु॑प॒र्णा मिहः॑ । \newline
31. सु॒प॒र्णा इति॑ सु - प॒र्णाः । \newline
32. मिहो॒ वसा॑ना॒ वसा॑ना॒ मिहो॒ मिहो॒ वसा॑नाः । \newline
33. वसा॑ना॒ दिव॒म् दिवं॒ ॅवसा॑ना॒ वसा॑ना॒ दिव᳚म् । \newline
34. दिव॒ मुदुद् दिव॒म् दिव॒ मुत् । \newline
35. उत् प॑तन्ति पत॒ न्त्युदुत् प॑तन्ति । \newline
36. प॒त॒न्तीति॑ पतन्ति । \newline
37. त आ ते त आ । \newline
38. आ ऽव॑वृत्रन् नववृत्र॒न् ना ऽव॑वृत्रन्न् । \newline
39. अ॒व॒वृ॒त्र॒न् थ्सद॑नानि॒ सद॑ नान्यववृत्रन् नववृत्र॒न् थ्सद॑नानि । \newline
40. सद॑नानि कृ॒त्वा कृ॒त्वा सद॑नानि॒ सद॑नानि कृ॒त्वा । \newline
41. कृ॒त्वा ऽऽदात् कृ॒त्वा कृ॒त्वा ऽऽत् । \newline
42. आदि दिदा दादित् । \newline
43. इत् पृ॑थि॒वी पृ॑थि॒वी दित् पृ॑थि॒वी । \newline
44. पृ॒थि॒वी घृ॒तैर् घृ॒तैः पृ॑थि॒वी पृ॑थि॒वी घृ॒तैः । \newline
45. घृ॒तैर् वि वि घृ॒तैर् घृ॒तैर् वि । \newline
46. व्यु॑द्यत उद्यते॒ वि व्यु॑द्यते । \newline
47. उ॒द्य॒त॒ इत्यु॑द्यते । \newline
48. हिर॑ण्यकेशो॒ रज॑सो॒ रज॑सो॒ हिर॑ण्यकेशो॒ हिर॑ण्यकेशो॒ रज॑सः । \newline
49. हिर॑ण्यकेश॒ इति॒ हिर॑ण्य - के॒शः॒ । \newline
50. रज॑सो विसा॒रे वि॑सा॒रे रज॑सो॒ रज॑सो विसा॒रे । \newline
51. वि॒सा॒रे ऽहि॒ रहि॑र् विसा॒रे वि॑सा॒रे ऽहिः॑ । \newline
52. वि॒सा॒र इति॑ वि - सा॒रे । \newline
53. अहि॒र् धुनि॒र् धुनि॒ रहि॒ रहि॒र् धुनिः॑ । \newline
54. धुनि॒र् वातो॒ वातो॒ धुनि॒र् धुनि॒र् वातः॑ । \newline
55. वात॑ इवे व॒ वातो॒ वात॑ इव । \newline
56. इ॒व॒ ध्रजी॑मा॒न् ध्रजी॑मा निवे व॒ ध्रजी॑मान् । \newline
57. ध्रजी॑मा॒निति॒ ध्रजी॑मान् । \newline
58. शुचि॑भ्राजा उ॒षस॑ उ॒षसः॒ शुचि॑भ्राजाः॒ शुचि॑भ्राजा उ॒षसः॑ । \newline
59. शुचि॑भ्राजा॒ इति॒ शुचि॑ - भ्रा॒जाः॒ । \newline
60. उ॒षसो॒ नवे॑दा॒ नवे॑दा उ॒षस॑ उ॒षसो॒ नवे॑दाः । \newline

\textbf{Ghana Paata } \newline

1. आहु॑तम् प्र॒जाम् प्र॒जा माहु॑त॒ माहु॑तम् प्र॒जाम् दे॑वि देवि प्र॒जा माहु॑त॒ माहु॑तम् प्र॒जाम् दे॑वि । \newline
2. आहु॑त॒मित्या - हु॒त॒म् । \newline
3. प्र॒जाम् दे॑वि देवि प्र॒जाम् प्र॒जाम् दे॑वि दिदिड्ढि दिदिड्ढि देवि प्र॒जाम् प्र॒जाम् दे॑वि दिदिड्ढि । \newline
4. प्र॒जामिति॑ प्र - जाम् । \newline
5. दे॒वि॒ दि॒दि॒ड्ढि॒ दि॒दि॒ड्ढि॒ दे॒वि॒ दे॒वि॒ दि॒दि॒ड्ढि॒ नो॒ नो॒ दि॒दि॒ड्ढि॒ दे॒वि॒ दे॒वि॒ दि॒दि॒ड्ढि॒ नः॒ । \newline
6. दि॒दि॒ड्ढि॒ नो॒ नो॒ दि॒दि॒ड्ढि॒ दि॒दि॒ड्ढि॒ नः॒ । \newline
7. न॒ इति॑ नः । \newline
8. या सु॑पा॒णिः सु॑पा॒णिर् या या सु॑पा॒णिः स्व॑ङ्गु॒रिः स्व॑ङ्गु॒रिः सु॑पा॒णिर् या या सु॑पा॒णिः स्व॑ङ्गु॒रिः । \newline
9. सु॒पा॒णिः स्व॑ङ्गु॒रिः स्व॑ङ्गु॒रिः सु॑पा॒णिः सु॑पा॒णिः स्व॑ङ्गु॒रिः सु॒षूमा॑ सु॒षूमा᳚ स्वङ्गु॒रिः सु॑पा॒णिः सु॑पा॒णिः स्व॑ङ्गु॒रिः सु॒षूमा᳚ । \newline
10. सु॒पा॒णिरिति॑ सु - पा॒णिः । \newline
11. स्व॒ङ्गु॒रिः सु॒षूमा॑ सु॒षूमा᳚ स्वङ्गु॒रिः स्व॑ङ्गु॒रिः सु॒षूमा॑ बहु॒सूव॑री बहु॒सूव॑री सु॒षूमा᳚ स्वङ्गु॒रिः स्व॑ङ्गु॒रिः सु॒षूमा॑ बहु॒सूव॑री । \newline
12. स्व॒ङ्गु॒रिरिति॑ सु - अ॒ङ्गु॒रिः । \newline
13. सु॒षूमा॑ बहु॒सूव॑री बहु॒सूव॑री सु॒षूमा॑ सु॒षूमा॑ बहु॒सूव॑री । \newline
14. सु॒षूमेति॑ सु - सूमा᳚ । \newline
15. ब॒हु॒सूव॒रीति॑ बहु - सूव॑री । \newline
16. तस्यै॑ वि॒श्पत्नि॑यै वि॒श्पत्नि॑यै॒ तस्यै॒ तस्यै॑ वि॒श्पत्नि॑यै ह॒विर्. ह॒विर् वि॒श्पत्नि॑यै॒ तस्यै॒ तस्यै॑ वि॒श्पत्नि॑यै ह॒विः । \newline
17. वि॒श्पत्नि॑यै ह॒विर्. ह॒विर् वि॒श्पत्नि॑यै वि॒श्पत्नि॑यै ह॒विः सि॑नीवा॒ल्यै सि॑नीवा॒ल्यै ह॒विर् वि॒श्पत्नि॑यै वि॒श्पत्नि॑यै ह॒विः सि॑नीवा॒ल्यै । \newline
18. ह॒विः सि॑नीवा॒ल्यै सि॑नीवा॒ल्यै ह॒विर्. ह॒विः सि॑नीवा॒ल्यै जु॑होतन जुहोतन सिनीवा॒ल्यै ह॒विर्. ह॒विः सि॑नीवा॒ल्यै जु॑होतन । \newline
19. सि॒नी॒वा॒ल्यै जु॑होतन जुहोतन सिनीवा॒ल्यै सि॑नीवा॒ल्यै जु॑होतन । \newline
20. जु॒हो॒त॒नेति॑ जुहोतन । \newline
21. इन्द्रं॑ ॅवो व॒ इन्द्र॒ मिन्द्रं॑ ॅवो वि॒श्वतो॑ वि॒श्वतो॑ व॒ इन्द्र॒ मिन्द्रं॑ ॅवो वि॒श्वतः॑ । \newline
22. वो॒ वि॒श्वतो॑ वि॒श्वतो॑ वो वो वि॒श्वत॒ स्परि॒ परि॑ वि॒श्वतो॑ वो वो वि॒श्वत॒ स्परि॑ । \newline
23. वि॒श्वत॒ स्परि॒ परि॑ वि॒श्वतो॑ वि॒श्वत॒ स्परीन्द्र॒ मिन्द्र॒म् परि॑ वि॒श्वतो॑ वि॒श्वत॒ स्परीन्द्र᳚म् । \newline
24. परीन्द्र॒ मिन्द्र॒म् परि॒ परीन्द्र॒म् नरो॒ नर॒ इन्द्र॒म् परि॒ परीन्द्र॒म् नरः॑ । \newline
25. इन्द्र॒म् नरो॒ नर॒ इन्द्र॒ मिन्द्र॒म् नरः॑ । \newline
26. नर॒ इति॒ नरः॑ । \newline
27. असि॑तवर्णा॒ हर॑यो॒ हर॑यो॒ असि॑तवर्णा॒ असि॑तवर्णा॒ हर॑यः सुप॒र्णाः सु॑प॒र्णा हर॑यो॒ असि॑तवर्णा॒ असि॑तवर्णा॒ हर॑यः सुप॒र्णाः । \newline
28. असि॑तवर्णा॒ इत्यसि॑त - व॒र्णाः॒ । \newline
29. हर॑यः सुप॒र्णाः सु॑प॒र्णा हर॑यो॒ हर॑यः सुप॒र्णा मिहो॒ मिहः॑ सुप॒र्णा हर॑यो॒ हर॑यः सुप॒र्णा मिहः॑ । \newline
30. सु॒प॒र्णा मिहो॒ मिहः॑ सुप॒र्णाः सु॑प॒र्णा मिहो॒ वसा॑ना॒ वसा॑ना॒ मिहः॑ सुप॒र्णाः सु॑प॒र्णा मिहो॒ वसा॑नाः । \newline
31. सु॒प॒र्णा इति॑ सु - प॒र्णाः । \newline
32. मिहो॒ वसा॑ना॒ वसा॑ना॒ मिहो॒ मिहो॒ वसा॑ना॒ दिव॒म् दिवं॒ ॅवसा॑ना॒ मिहो॒ मिहो॒ वसा॑ना॒ दिव᳚म् । \newline
33. वसा॑ना॒ दिव॒म् दिवं॒ ॅवसा॑ना॒ वसा॑ना॒ दिव॒ मुदुद् दिवं॒ ॅवसा॑ना॒ वसा॑ना॒ दिव॒ मुत् । \newline
34. दिव॒ मुदुद् दिव॒म् दिव॒ मुत् प॑तन्ति पत॒न्त्युद् दिव॒म् दिव॒ मुत् प॑तन्ति । \newline
35. उत् प॑तन्ति पत॒न्त्युदुत् प॑तन्ति । \newline
36. प॒त॒न्तीति॑ पतन्ति । \newline
37. त आ ते त आ ऽव॑वृत्रन् नववृत्र॒न् ना ते त आ ऽव॑वृत्रन्न् । \newline
38. आ ऽव॑वृत्रन् नववृत्र॒न् ना ऽव॑वृत्र॒न् थ्सद॑नानि॒ सद॑ना न्यववृत्र॒न् ना ऽव॑वृत्र॒न् थ्सद॑नानि । \newline
39. अ॒व॒वृ॒त्र॒न् थ्सद॑नानि॒ सद॑ना न्यववृत्रन् नववृत्र॒न् थ्सद॑नानि कृ॒त्वा कृ॒त्वा सद॑ना न्यववृत्रन् नववृत्र॒न् थ्सद॑नानि कृ॒त्वा । \newline
40. सद॑नानि कृ॒त्वा कृ॒त्वा सद॑नानि॒ सद॑नानि कृ॒त्वा ऽऽदात् कृ॒त्वा सद॑नानि॒ सद॑नानि कृ॒त्वा ऽऽत् । \newline
41. कृ॒त्वा ऽऽदात् कृ॒त्वा कृ॒त्वा ऽऽदिदिदात् कृ॒त्वा कृ॒त्वा ऽऽदित् । \newline
42. आदि दिदा दादित् पृ॑थि॒वी पृ॑थि॒वी दा दादित् पृ॑थि॒वी । \newline
43. इत् पृ॑थि॒वी पृ॑थि॒वीदित् पृ॑थि॒वी घृ॒तैर् घृ॒तैः पृ॑थि॒वीदित् पृ॑थि॒वी घृ॒तैः । \newline
44. पृ॒थि॒वी घृ॒तैर् घृ॒तैः पृ॑थि॒वी पृ॑थि॒वी घृ॒तैर् वि वि घृ॒तैः पृ॑थि॒वी पृ॑थि॒वी घृ॒तैर् वि । \newline
45. घृ॒तैर् वि वि घृ॒तैर् घृ॒तैर् व्यु॑द्यत उद्यते॒ वि घृ॒तैर् घृ॒तैर् व्यु॑द्यते । \newline
46. व्यु॑द्यत उद्यते॒ वि व्यु॑द्यते । \newline
47. उ॒द्य॒त॒ इत्यु॑द्यते । \newline
48. हिर॑ण्यकेशो॒ रज॑सो॒ रज॑सो॒ हिर॑ण्यकेशो॒ हिर॑ण्यकेशो॒ रज॑सो विसा॒रे वि॑सा॒रे रज॑सो॒ हिर॑ण्यकेशो॒ हिर॑ण्यकेशो॒ रज॑सो विसा॒रे । \newline
49. हिर॑ण्यकेश॒ इति॒ हिर॑ण्य - के॒शः॒ । \newline
50. रज॑सो विसा॒रे वि॑सा॒रे रज॑सो॒ रज॑सो विसा॒रे ऽहि॒ रहि॑र् विसा॒रे रज॑सो॒ रज॑सो विसा॒रे ऽहिः॑ । \newline
51. वि॒सा॒रे ऽहि॒ रहि॑र् विसा॒रे वि॑सा॒रे ऽहि॒र् धुनि॒र् धुनि॒ रहि॑र् विसा॒रे वि॑सा॒रे ऽहि॒र् धुनिः॑ । \newline
52. वि॒सा॒र इति॑ वि - सा॒रे । \newline
53. अहि॒र् धुनि॒र् धुनि॒ रहि॒ रहि॒र् धुनि॒र् वातो॒ वातो॒ धुनि॒ रहि॒ रहि॒र् धुनि॒र् वातः॑ । \newline
54. धुनि॒र् वातो॒ वातो॒ धुनि॒र् धुनि॒र् वात॑ इवे व॒ वातो॒ धुनि॒र् धुनि॒र् वात॑ इव । \newline
55. वात॑ इवे व॒ वातो॒ वात॑ इव॒ ध्रजी॑मा॒न् ध्रजी॑मा निव॒ वातो॒ वात॑ इव॒ ध्रजी॑मान् । \newline
56. इ॒व॒ ध्रजी॑मा॒न् ध्रजी॑मा निवे व॒ ध्रजी॑मान् । \newline
57. ध्रजी॑मा॒निति॒ ध्रजी॑मान् । \newline
58. शुचि॑भ्राजा उ॒षस॑ उ॒षसः॒ शुचि॑भ्राजाः॒ शुचि॑भ्राजा उ॒षसो॒ नवे॑दा॒ नवे॑दा उ॒षसः॒ शुचि॑भ्राजाः॒ शुचि॑भ्राजा उ॒षसो॒ नवे॑दाः । \newline
59. शुचि॑भ्राजा॒ इति॒ शुचि॑ - भ्रा॒जाः॒ । \newline
60. उ॒षसो॒ नवे॑दा॒ नवे॑दा उ॒षस॑ उ॒षसो॒ नवे॑दा॒ यश॑स्वती॒र् यश॑स्वती॒र् नवे॑दा उ॒षस॑ उ॒षसो॒ नवे॑दा॒ यश॑स्वतीः । \newline
\pagebreak
\markright{ TS 3.1.11.5  \hfill https://www.vedavms.in \hfill}

\section{ TS 3.1.11.5 }

\textbf{TS 3.1.11.5 } \newline
\textbf{Samhita Paata} \newline

नवे॑दा॒ यश॑स्वतीरप॒स्युवो॒ न स॒त्याः ॥आ ते॑ सुप॒र्णा अ॑मिनन्त॒ एवैः᳚ कृ॒ष्णो नो॑नाव वृष॒भो यदी॒दं । शि॒वाभि॒र्न स्मय॑मानाभि॒राऽगा॒त् पत॑न्ति॒ मिहः॑ स्त॒नय॑न्त्य॒भ्रा ॥ वा॒श्रेव॑ वि॒द्युन्मि॑माति व॒थ्सं न मा॒ता सि॑षक्ति । यदे॑षां ॅवृ॒ष्टिरस॑र्जि ॥ पर्व॑तश्चि॒न्महि॑ वृ॒द्धो बि॑भाय दि॒वश्चि॒थ् सानु॑ रेजत स्व॒ने वः॑ । यत् क्रीड॑थ मरुत - [  ] \newline

\textbf{Pada Paata} \newline

नवे॑दाः । यश॑स्वतीः । अ॒प॒स्युवः॑ । न । स॒त्याः ॥ एति॑ । ते॒ । सु॒प॒र्णा इति॑ सु - प॒र्णाः । अ॒मि॒न॒न्त॒ । एवैः᳚ । कृ॒ष्णः । नो॒ना॒व॒ । वृ॒ष॒भः । यदि॑ । इ॒दम् ॥ शि॒वाभिः॑ । न । स्मय॑मानाभिः । एति॑ । अ॒गा॒त् । पत॑न्ति । मिहः॑ । स्त॒नय॑न्ति । अ॒भ्रा ॥ वा॒श्रा । इ॒व॒ । वि॒द्युदिति॑ वि-द्युत् । मि॒मा॒ति॒ । व॒थ्सम् । न । मा॒ता । सि॒ष॒क्ति॒ ॥ यत् । ए॒षा॒म् । वृ॒ष्टिः । अस॑र्जि ॥ पर्व॑तः । चि॒त् । महि॑ । वृ॒द्धः । बि॒भा॒य॒ । दि॒वः । चि॒त् । सानु॑ । रे॒ज॒त॒ । स्व॒ने । वः॒ ॥ यत् । क्रीड॑थ । म॒रु॒तः॒ ।  \newline


\textbf{Krama Paata} \newline

नवे॑दा॒ यश॑स्वतीः । यश॑स्वती,रप॒स्युवः॑ । अ॒प॒स्युवो॒ न । न स॒त्याः । स॒त्या इति॑ स॒त्याः ॥ आ ते᳚ । ते॒ सु॒प॒र्णाः । सु॒प॒र्णा अ॑मिनन्त । सु॒प॒र्णा इति॑ सु - प॒र्णाः । अ॒मि॒न॒न्त॒ एवैः᳚ । एवैः᳚ कृ॒ष्णः । कृ॒ष्णो नो॑नाव । नो॒ना॒व॒ वृ॒ष॒भः । वृ॒ष॒भो यदि॑ । यदी॒दम् । इ॒दमिती॒दम् ॥ शि॒वाभि॒र् न । न स्मय॑मानाभिः । स्मय॑मानाभि॒रा । आ ऽगा᳚त् । अ॒गा॒त् पत॑न्ति । पत॑न्ति॒मिहः॑ । मिहः॑ स्त॒नय॑न्ति । स्त॒नय॑न्त्य॒भ्रा । अ॒भ्रेत्य॒भ्रा ॥ वा॒श्रेव॑ । इ॒व॒ वि॒द्युत् । वि॒द्युन् मि॑माति । वि॒द्युदिति॑ वि - द्युत् । मि॒मा॒ति॒ व॒थ्सम् । व॒थ्सम् न । न मा॒ता । मा॒ता सि॑षक्ति । सि॒ष॒क्तीति॑ सिषक्ति ॥ यदे॑षाम् । ए॒षां॒ ॅवृ॒ष्टिः । वृ॒ष्टिरस॑र्जि । अस॒र्जीत्यस॑र्जि ॥ पर्व॑तश्चित् । चि॒न् महि॑ । महि॑ वृ॒द्धः । वृ॒द्धो बि॑भाय । बि॒भा॒य॒ दि॒वः । दि॒वश्चि॑त् । चि॒थ् सानु॑ । सानु॑ रेजत । रे॒ज॒त॒ स्व॒ने । स्व॒ने वः॑ । व॒ इति॑ वः ॥ यत् क्रीड॑थ । क्रीड॑थ मरुतः । म॒रु॒त॒ ऋ॒ष्टि॒मन्तः॑ \newline

\textbf{Jatai Paata} \newline

1. नवे॑दा॒ यश॑स्वती॒र् यश॑स्वती॒र् नवे॑दा॒ नवे॑दा॒ यश॑स्वतीः । \newline
2. यश॑स्वती रप॒स्युवो॑ अप॒स्युवो॒ यश॑स्वती॒र् यश॑स्वती रप॒स्युवः॑ । \newline
3. अ॒प॒स्युवो॒ न नाप॒स्युवो॑ अप॒स्युवो॒ न । \newline
4. न स॒त्याः स॒त्या न न स॒त्याः । \newline
5. स॒त्या इति॑ स॒त्याः । \newline
6. आ ते॑ त॒ आ ते᳚ । \newline
7. ते॒ सु॒प॒र्णाः सु॑प॒र्णा स्ते॑ ते सुप॒र्णाः । \newline
8. सु॒प॒र्णा अ॑मिनन्ता मिनन्त सुप॒र्णाः सु॑प॒र्णा अ॑मिनन्त । \newline
9. सु॒प॒र्णा इति॑ सु - प॒र्णाः । \newline
10. अ॒मि॒न॒न्त॒ एवै॒ रेवै॑ रमिनन्ता मिनन्त॒ एवैः᳚ । \newline
11. एवैः᳚ कृ॒ष्णः कृ॒ष्ण एवै॒ रेवैः᳚ कृ॒ष्णः । \newline
12. कृ॒ष्णो नो॑नाव नोनाव कृ॒ष्णः कृ॒ष्णो नो॑नाव । \newline
13. नो॒ना॒व॒ वृ॒ष॒भो वृ॑ष॒भो नो॑नाव नोनाव वृष॒भः । \newline
14. वृ॒ष॒भो यदि॒ यदि॑ वृष॒भो वृ॑ष॒भो यदि॑ । \newline
15. यदी॒द मि॒दं ॅयदि॒ यदी॒दम् । \newline
16. इ॒दमिती॒दम् । \newline
17. शि॒वाभि॒र् न न शि॒वाभिः॑ शि॒वाभि॒र् न । \newline
18. न स्मय॑मानाभिः॒ स्मय॑मानाभि॒र् न न स्मय॑मानाभिः । \newline
19. स्मय॑मानाभि॒रा स्मय॑मानाभिः॒ स्मय॑मानाभि॒रा । \newline
20. आ ऽगा॑ दगा॒ दा ऽगा᳚त् । \newline
21. अ॒गा॒त् पत॑न्ति॒ पत॑ न्त्यगा दगा॒त् पत॑न्ति । \newline
22. पत॑न्ति॒ मिहो॒ मिहः॒ पत॑न्ति॒ पत॑न्ति॒ मिहः॑ । \newline
23. मिहः॑ स्त॒नय॑न्ति स्त॒नय॑न्ति॒ मिहो॒ मिहः॑ स्त॒नय॑न्ति । \newline
24. स्त॒नय॑ न्त्य॒भ्रा ऽभ्रा स्त॒नय॑न्ति स्त॒नय॑ न्त्य॒भ्रा । \newline
25. अ॒भ्रेत्य॒भ्रा । \newline
26. वा॒श्रेवे॑ व वा॒श्रा वा॒श्रेव॑ । \newline
27. इ॒व॒ वि॒द्युद् वि॒द्यु दि॑वे व वि॒द्युत् । \newline
28. वि॒द्युन् मि॑माति मिमाति वि॒द्युद् वि॒द्युन् मि॑माति । \newline
29. वि॒द्युदिति॑ वि - द्युत् । \newline
30. मि॒मा॒ति॒ व॒थ्सं ॅव॒थ्सम् मि॑माति मिमाति व॒थ्सम् । \newline
31. व॒थ्सम् न न व॒थ्सं ॅव॒थ्सम् न । \newline
32. न मा॒ता मा॒ता न न मा॒ता । \newline
33. मा॒ता सि॑षक्ति सिषक्ति मा॒ता मा॒ता सि॑षक्ति । \newline
34. सि॒ष॒क्तीति॑ सिषक्ति । \newline
35. यदे॑षा मेषां॒ ॅयद् यदे॑षाम् । \newline
36. ए॒षां॒ ॅवृ॒ष्टिर् वृ॒ष्टि रे॑षा मेषां ॅवृ॒ष्टिः । \newline
37. वृ॒ष्टि रस॒ र्ज्यस॑र्जि वृ॒ष्टिर् वृ॒ष्टि रस॑र्जि । \newline
38. अस॒र्जीत्यस॑र्जि । \newline
39. पर्व॑त श्चिच् चि॒त् पर्व॑तः॒ पर्व॑त श्चित् । \newline
40. चि॒न् महि॒ महि॑ चिच् चि॒न् महि॑ । \newline
41. महि॑ वृ॒द्धो वृ॒द्धो महि॒ महि॑ वृ॒द्धः । \newline
42. वृ॒द्धो बि॑भाय बिभाय वृ॒द्धो वृ॒द्धो बि॑भाय । \newline
43. बि॒भा॒य॒ दि॒वो दि॒वो बि॑भाय बिभाय दि॒वः । \newline
44. दि॒व श्चि॑च् चिद् दि॒वो दि॒व श्चि॑त् । \newline
45. चि॒थ् सानु॒ सानु॑ चिच् चि॒थ् सानु॑ । \newline
46. सानु॑ रेजत रेजत॒ सानु॒ सानु॑ रेजत । \newline
47. रे॒ज॒त॒ स्व॒ने स्व॒ने रे॑जत रेजत स्व॒ने । \newline
48. स्व॒ने वो॑ वः स्व॒ने स्व॒ने वः॑ । \newline
49. व॒ इति॑ वः । \newline
50. यत् क्रीड॑थ॒ क्रीड॑थ॒ यद् यत् क्रीड॑थ । \newline
51. क्रीड॑थ मरुतो मरुतः॒ क्रीड॑थ॒ क्रीड॑थ मरुतः । \newline
52. म॒रु॒त॒ ऋ॒ष्टि॒मन्त॑ ऋष्टि॒मन्तो॑ मरुतो मरुत ऋष्टि॒मन्तः॑ । \newline

\textbf{Ghana Paata } \newline

1. नवे॑दा॒ यश॑स्वती॒र् यश॑स्वती॒र् नवे॑दा॒ नवे॑दा॒ यश॑स्वती रप॒स्युवो॑ अप॒स्युवो॒ यश॑स्वती॒र् नवे॑दा॒ नवे॑दा॒ यश॑स्वती रप॒स्युवः॑ । \newline
2. यश॑स्वती रप॒स्युवो॑ अप॒स्युवो॒ यश॑स्वती॒र् यश॑स्वती रप॒स्युवो॒ न नाप॒स्युवो॒ यश॑स्वती॒र् यश॑स्वती रप॒स्युवो॒ न । \newline
3. अ॒प॒स्युवो॒ न नाप॒स्युवो॑ अप॒स्युवो॒ न स॒त्याः स॒त्या नाप॒स्युवो॑ अप॒स्युवो॒ न स॒त्याः । \newline
4. न स॒त्याः स॒त्या न न स॒त्याः । \newline
5. स॒त्या इति॑ स॒त्याः । \newline
6. आ ते॑ त॒ आ ते॑ सुप॒र्णाः सु॑प॒र्णा स्त॒ आ ते॑ सुप॒र्णाः । \newline
7. ते॒ सु॒प॒र्णाः सु॑प॒र्णा स्ते॑ ते सुप॒र्णा अ॑मिनन्ता मिनन्त सुप॒र्णा स्ते॑ ते सुप॒र्णा अ॑मिनन्त । \newline
8. सु॒प॒र्णा अ॑मिनन्ता मिनन्त सुप॒र्णाः सु॑प॒र्णा अ॑मिनन्त॒ एवै॒ रेवै॑ रमिनन्त सुप॒र्णाः सु॑प॒र्णा अ॑मिनन्त॒ एवैः᳚ । \newline
9. सु॒प॒र्णा इति॑ सु - प॒र्णाः । \newline
10. अ॒मि॒न॒न्त॒ एवै॒ रेवै॑ रमिनन्ता मिनन्त॒ एवैः᳚ कृ॒ष्णः कृ॒ष्ण एवै॑ रमिनन्ता मिनन्त॒ एवैः᳚ कृ॒ष्णः । \newline
11. एवैः᳚ कृ॒ष्णः कृ॒ष्ण एवै॒रेवैः᳚ कृ॒ष्णो नो॑नाव नोनाव कृ॒ष्ण एवै॒ रेवैः᳚ कृ॒ष्णो नो॑नाव । \newline
12. कृ॒ष्णो नो॑नाव नोनाव कृ॒ष्णः कृ॒ष्णो नो॑नाव वृष॒भो वृ॑ष॒भो नो॑नाव कृ॒ष्णः कृ॒ष्णो नो॑नाव वृष॒भः । \newline
13. नो॒ना॒व॒ वृ॒ष॒भो वृ॑ष॒भो नो॑नाव नोनाव वृष॒भो यदि॒ यदि॑ वृष॒भो नो॑नाव नोनाव वृष॒भो यदि॑ । \newline
14. वृ॒ष॒भो यदि॒ यदि॑ वृष॒भो वृ॑ष॒भो यदी॒द मि॒दं ॅयदि॑ वृष॒भो वृ॑ष॒भो यदी॒दम् । \newline
15. यदी॒द मि॒दं ॅयदि॒ यदी॒दम् । \newline
16. इ॒दमिती॒दम् । \newline
17. शि॒वाभि॒र् न न शि॒वाभिः॑ शि॒वाभि॒र् न स्मय॑मानाभिः॒ स्मय॑मानाभि॒र् न शि॒वाभिः॑ शि॒वाभि॒र् न स्मय॑मानाभिः । \newline
18. न स्मय॑मानाभिः॒ स्मय॑मानाभि॒र् न न स्मय॑मानाभि॒रा स्मय॑मानाभि॒र् न न स्मय॑मानाभि॒रा । \newline
19. स्मय॑मानाभि॒रा स्मय॑मानाभिः॒ स्मय॑मानाभि॒रा ऽगा॑ दगा॒दा स्मय॑मानाभिः॒ स्मय॑मानाभि॒रा ऽगा᳚त् । \newline
20. आ ऽगा॑ दगा॒दा ऽगा॒त् पत॑न्ति॒ पत॑ न्त्यगा॒दा ऽगा॒त् पत॑न्ति । \newline
21. अ॒गा॒त् पत॑न्ति॒ पत॑ न्त्यगादगा॒त् पत॑न्ति॒ मिहो॒ मिहः॒ पत॑ न्त्यगादगा॒त् पत॑न्ति॒ मिहः॑ । \newline
22. पत॑न्ति॒ मिहो॒ मिहः॒ पत॑न्ति॒ पत॑न्ति॒ मिहः॑ स्त॒नय॑न्ति स्त॒नय॑न्ति॒ मिहः॒ पत॑न्ति॒ पत॑न्ति॒ मिहः॑ स्त॒नय॑न्ति । \newline
23. मिहः॑ स्त॒नय॑न्ति स्त॒नय॑न्ति॒ मिहो॒ मिहः॑ स्त॒नय॑ न्त्य॒भ्रा ऽभ्रा स्त॒नय॑न्ति॒ मिहो॒ मिहः॑ स्त॒नय॑ न्त्य॒भ्रा । \newline
24. स्त॒नय॑ न्त्य॒भ्रा ऽभ्रा स्त॒नय॑न्ति स्त॒नय॑ न्त्य॒भ्रा । \newline
25. अ॒भ्रेत्य॒भ्रा । \newline
26. वा॒श्रेवे॑ व वा॒श्रा वा॒श्रेव॑ वि॒द्युद् वि॒द्युदि॑व वा॒श्रा वा॒श्रेव॑ वि॒द्युत् । \newline
27. इ॒व॒ वि॒द्युद् वि॒द्युदि॑वे व वि॒द्युन् मि॑माति मिमाति वि॒द्युदि॑वे व वि॒द्युन् मि॑माति । \newline
28. वि॒द्युन् मि॑माति मिमाति वि॒द्युद् वि॒द्युन् मि॑माति व॒थ्सं ॅव॒थ्सम् मि॑माति वि॒द्युद् वि॒द्युन् मि॑माति व॒थ्सम् । \newline
29. वि॒द्युदिति॑ वि - द्युत् । \newline
30. मि॒मा॒ति॒ व॒थ्सं ॅव॒थ्सम् मि॑माति मिमाति व॒थ्सन्न न व॒थ्सम् मि॑माति मिमाति व॒थ्सन्न । \newline
31. व॒थ्सम् न न व॒थ्सं ॅव॒थ्सम् न मा॒ता मा॒ता न व॒थ्सं ॅव॒थ्सम् न मा॒ता । \newline
32. न मा॒ता मा॒ता न न मा॒ता सि॑षक्ति सिषक्ति मा॒ता न न मा॒ता सि॑षक्ति । \newline
33. मा॒ता सि॑षक्ति सिषक्ति मा॒ता मा॒ता सि॑षक्ति । \newline
34. सि॒ष॒क्तीति॑ सिषक्ति । \newline
35. यदे॑षा मेषां॒ ॅयद् यदे॑षां ॅवृ॒ष्टिर् वृ॒ष्टि रे॑षां॒ ॅयद् यदे॑षां ॅवृ॒ष्टिः । \newline
36. ए॒षां॒ ॅवृ॒ष्टिर् वृ॒ष्टि रे॑षा मेषां ॅवृ॒ष्टि रस॒ र्ज्यस॑र्जि वृ॒ष्टिरे॑षा मेषां ॅवृ॒ष्टि रस॑र्जि । \newline
37. वृ॒ष्टि रस॒र्ज्यस॑र्जि वृ॒ष्टिर् वृ॒ष्टि रस॑र्जि । \newline
38. अस॒र्जीत्यस॑र्जि । \newline
39. पर्व॑त श्चिच् चि॒त् पर्व॑तः॒ पर्व॑त श्चि॒न् महि॒ महि॑ चि॒त् पर्व॑तः॒ पर्व॑त श्चि॒न् महि॑ । \newline
40. चि॒न् महि॒ महि॑ चिच् चि॒न् महि॑ वृ॒द्धो वृ॒द्धो महि॑ चिच् चि॒न् महि॑ वृ॒द्धः । \newline
41. महि॑ वृ॒द्धो वृ॒द्धो महि॒ महि॑ वृ॒द्धो बि॑भाय बिभाय वृ॒द्धो महि॒ महि॑ वृ॒द्धो बि॑भाय । \newline
42. वृ॒द्धो बि॑भाय बिभाय वृ॒द्धो वृ॒द्धो बि॑भाय दि॒वो दि॒वो बि॑भाय वृ॒द्धो वृ॒द्धो बि॑भाय दि॒वः । \newline
43. बि॒भा॒य॒ दि॒वो दि॒वो बि॑भाय बिभाय दि॒व श्चि॑च् चिद् दि॒वो बि॑भाय बिभाय दि॒व श्चि॑त् । \newline
44. दि॒व श्चि॑च् चिद् दि॒वो दि॒व श्चि॒थ् सानु॒ सानु॑ चिद् दि॒वो दि॒व श्चि॒थ् सानु॑ । \newline
45. चि॒थ् सानु॒ सानु॑ चिच् चि॒थ् सानु॑ रेजत रेजत॒ सानु॑ चिच् चि॒थ् सानु॑ रेजत । \newline
46. सानु॑ रेजत रेजत॒ सानु॒ सानु॑ रेजत स्व॒ने स्व॒ने रे॑जत॒ सानु॒ सानु॑ रेजत स्व॒ने । \newline
47. रे॒ज॒त॒ स्व॒ने स्व॒ने रे॑जत रेजत स्व॒ने वो॑ वः स्व॒ने रे॑जत रेजत स्व॒ने वः॑ । \newline
48. स्व॒ने वो॑ वः स्व॒ने स्व॒ने वः॑ । \newline
49. व॒ इति॑ वः । \newline
50. यत् क्रीड॑थ॒ क्रीड॑थ॒ यद् यत् क्रीड॑थ मरुतो मरुतः॒ क्रीड॑थ॒ यद् यत् क्रीड॑थ मरुतः । \newline
51. क्रीड॑थ मरुतो मरुतः॒ क्रीड॑थ॒ क्रीड॑थ मरुत ऋष्टि॒मन्त॑ ऋष्टि॒मन्तो॑ मरुतः॒ क्रीड॑थ॒ क्रीड॑थ मरुत ऋष्टि॒मन्तः॑ । \newline
52. म॒रु॒त॒ ऋ॒ष्टि॒मन्त॑ ऋष्टि॒मन्तो॑ मरुतो मरुत ऋष्टि॒मन्त॒ आप॒ आप॑ ऋष्टि॒मन्तो॑ मरुतो मरुत ऋष्टि॒मन्त॒ आपः॑ । \newline
\pagebreak
\markright{ TS 3.1.11.6  \hfill https://www.vedavms.in \hfill}

\section{ TS 3.1.11.6 }

\textbf{TS 3.1.11.6 } \newline
\textbf{Samhita Paata} \newline

ऋष्टि॒मन्त॒ आप॑ इव स॒द्ध्रिय॑ञ्चो धवद्ध्वे ॥अ॒भि क्र॑न्द स्त॒नय॒ गर्भ॒मा धा॑ उद॒न्वता॒ परि॑ दीया॒ रथे॑न । दृतिꣳ॒॒ सु क॑र्.ष॒ विषि॑तं॒ न्य॑ञ्चꣳ स॒मा भ॑वन्तू॒द्वता॑ निपा॒दाः ॥त्वं त्या चि॒दच्यु॒ताऽग्ने॑ प॒शुर्न यव॑से । धामा॑ ह॒ यत् ते॑ अजर॒ वना॑ वृ॒श्चन्ति॒ शिक्व॑सः ॥ अग्ने॒ भूरी॑णि॒ तव॑ जातवेदो॒ देव॑ स्वधावो॒ऽमृत॑स्य॒ धाम॑ । याश्च॑ - [  ] \newline

\textbf{Pada Paata} \newline

ऋ॒ष्टि॒मन्त॒ इत्यृ॑ष्टि-मन्तः॑ । आपः॑ । इ॒व॒ । स॒द्ध्रिय॑ञ्चः । ध॒व॒द्ध्वे॒ ॥ अ॒भीति॑ । क्र॒न्द॒ । स्त॒नय॑ । गर्भ᳚म् । एति॑ । धाः॒ । उ॒द॒न्वतेत्यु॑दन्न् - वता᳚ । परीति॑ । दी॒य॒ । रथे॑न ॥ दृति᳚म् । स्विति॑ । क॒र्.॒ष॒ । विषि॑त॒मिति॒ वि - सि॒त॒म् । न्य॑ञ्चम् । स॒माः । भ॒व॒न्तु॒ । उ॒द्वतेत्य॑त् - वता᳚ । नि॒पा॒दा इति॑ नि - पा॒दाः ॥ त्वम् । त्या । चि॒त् । अच्यु॑ता । अग्ने᳚ । प॒शुः । न । यव॑से ॥ धाम॑ । ह॒ । यत् । ते॒ । अ॒ज॒र॒ । वना᳚ । वृ॒श्चन्ति॑ । शिक्व॑सः ॥ अग्ने᳚ । भूरी॑णि । तव॑ । जा॒त॒वे॒द॒ इति॑ जात - वे॒दः॒ । देव॑ । स्व॒धा॒व॒ इति॑ स्वधा - वः॒ । अ॒मृत॑स्य । धाम॑ ॥ याः । च॒ ।  \newline


\textbf{Krama Paata} \newline

ऋ॒ष्टि॒मन्त॒ आपः॑ । ऋ॒ष्टि॒मन्त॒ इत्यृ॑ष्टि - मन्तः॑ । आप॑ इव । इ॒व॒ स॒ध्रिय॑ञ्चः । स॒ध्रिय॑ञ्चो धवद्ध्वे । ध॒व॒द्ध्व॒ इति॑ धवद्ध्वे ॥ अ॒भि क्र॑न्द । क्र॒न्द॒ स्त॒नय॑ । स्त॒नय॒ गर्भ᳚म् । गर्भ॒मा । आ धाः᳚ । धा॒ उ॒द॒न्वता᳚ । उ॒द॒न्वता॒ परि॑ । उ॒द॒न्वतेत्यु॑दन्न् - वता᳚ । परि॑ दीय । दी॒य॒ रथे॑न । रथे॒नेति॒ रथे॑न ॥ दृतिꣳ॒॒ सु । सु क॑र्.ष । क॒र्॒.ष॒ विषि॑तम् । विषि॑त॒म् न्य॑ञ्चम् । विषि॑त॒मिति॒ वि - सि॒त॒म् । न्य॑ञ्चꣳ स॒माः । स॒मा भ॑वन्तु । भ॒व॒न्तू॒द्वता᳚ । उ॒द्वता॑ निपा॒दाः । उ॒द्वतेत्यु॑त् - वता᳚ । नि॒पा॒दा इति॑ नि - पा॒दाः ॥ त्वम् त्या । त्या चि॑त् । चि॒दच्यु॑त । अच्यु॒ताग्ने᳚ । अग्ने॑ प॒शुः । प॒शुर् न । न यव॑से । यव॑स॒ इति॒ यव॑से ॥ धामा॑ ह । ह॒ यत् । यत् ते᳚ । ते॒ अ॒ज॒र॒ । अ॒ज॒र॒ वना᳚ । वना॑ वृ॒श्चन्ति॑ । वृ॒श्चन्ति॒ शिक्व॑सः । शिक्व॑स॒ इति॒ शिक्व॑सः ॥ अग्ने॒ भूरी॑णि । भूरी॑णि॒ तव॑ । तव॑ जातवेदः । जा॒त॒वे॒दो॒ देव॑ । जा॒त॒वे॒द॒ इति॑ जात - वे॒दः॒ । देव॑ स्वधावः । स्व॒धा॒वो॒ ऽमृत॑स्य । स्व॒धा॒व॒ इति॑ स्वधा - वः॒ । अ॒मृत॑स्य॒ धाम॑ । धामेति॒ धाम॑ ॥ याश्च॑ । च॒ मा॒याः \newline

\textbf{Jatai Paata} \newline

1. ऋ॒ष्टि॒मन्त॒ आप॒ आप॑ ऋष्टि॒मन्त॑ ऋष्टि॒मन्त॒ आपः॑ । \newline
2. ऋ॒ष्टि॒मन्त॒ इत्यृ॑ष्टि - मन्तः॑ । \newline
3. आप॑ इवे॒ वाप॒ आप॑ इव । \newline
4. इ॒व॒ स॒द्ध्रिय॑ञ्चः स॒द्ध्रिय॑ञ्च इवे व स॒द्ध्रिय॑ञ्चः । \newline
5. स॒द्ध्रिय॑ञ्चो धवद्ध्वे धवद्ध्वे स॒द्ध्रिय॑ञ्चः स॒द्ध्रिय॑ञ्चो धवद्ध्वे । \newline
6. ध॒व॒द्ध्व॒ इति॑ धवद्ध्वे । \newline
7. अ॒भि क्र॑न्द क्रन्दा॒ भ्य॑भि क्र॑न्द । \newline
8. क्र॒न्द॒ स्त॒नय॑ स्त॒नय॑ क्रन्द क्रन्द स्त॒नय॑ । \newline
9. स्त॒नय॒ गर्भ॒म् गर्भꣳ॑ स्त॒नय॑ स्त॒नय॒ गर्भ᳚म् । \newline
10. गर्भ॒ मा गर्भ॒म् गर्भ॒ मा । \newline
11. आ धा॑ धा॒ आ धाः᳚ । \newline
12. धा॒ उ॒द॒न्व तो॑द॒न्वता॑ धा धा उद॒न्वता᳚ । \newline
13. उ॒द॒न्वता॒ परि॒ पर्यु॑द॒न्व तो॑द॒न्वता॒ परि॑ । \newline
14. उ॒द॒न्वतेत्यु॑दन्न् - वता᳚ । \newline
15. परि॑ दीय दीय॒ परि॒ परि॑ दीय । \newline
16. दी॒या॒ रथे॑न॒ रथे॑न दीय दीया॒ रथे॑न । \newline
17. रथे॒नेति॒ रथे॑न । \newline
18. दृतिꣳ॒॒ सु सु दृति॒म् दृतिꣳ॒॒ सु । \newline
19. सु क॑र्.ष कर्.ष॒ सु सु क॑र्.ष । \newline
20. क॒र्॒.ष॒ विषि॑तं॒ ॅविषि॑तम् कर्.ष कर्.ष॒ विषि॑तम् । \newline
21. विषि॑त॒म् न्य॑ञ्च॒म् न्य॑ञ्चं॒ ॅविषि॑तं॒ ॅविषि॑त॒म् न्य॑ञ्चम् । \newline
22. विषि॑त॒मिति॒ वि - सि॒त॒म् । \newline
23. न्य॑ञ्चꣳ स॒माः स॒मा न्य॑ञ्च॒म् न्य॑ञ्चꣳ स॒माः । \newline
24. स॒मा भ॑वन्तु भवन्तु स॒माः स॒मा भ॑वन्तु । \newline
25. भ॒व॒न्तू॒ द्‍वतो॒ द्‍वता॑ भवन्तु भवन्तू॒ द्‍वता᳚ । \newline
26. उ॒द्वता॑ निपा॒दा नि॑पा॒दा उ॒द्वतो॒ द्‍वता॑ निपा॒दाः । \newline
27. उ॒द्वतेत्य॑त् - वता᳚ । \newline
28. नि॒पा॒दा इति॑ नि - पा॒दाः । \newline
29. त्वम् त्या त्या त्वम् त्वम् त्या । \newline
30. त्या चि॑च् चि॒त् त्या त्या चि॑त् । \newline
31. चि॒ दच्यु॒ता ऽच्यु॑ता चिच् चि॒ दच्यु॑ता । \newline
32. अच्यु॒ता ऽग्ने ऽग्ने॒ अच्यु॒ता ऽच्यु॒ता ऽग्ने᳚ । \newline
33. अग्ने॑ प॒शुः प॒शु रग्ने ऽग्ने॑ प॒शुः । \newline
34. प॒शुर् न न प॒शुः प॒शुर् न । \newline
35. न यव॑से॒ यव॑से॒ न न यव॑से । \newline
36. यव॑स॒ इति॒ यव॑से । \newline
37. धामा॑ ह ह॒ धाम॒ धामा॑ ह । \newline
38. ह॒ यद् य द्ध॑ ह॒ यत् । \newline
39. यत् ते॑ ते॒ यद् यत् ते᳚ । \newline
40. ते॒ अ॒ज॒रा॒ ज॒र॒ ते॒ ते॒ अ॒ज॒र॒ । \newline
41. अ॒ज॒र॒ वना॒ वना॑ ऽजरा जर॒ वना᳚ । \newline
42. वना॑ वृ॒श्चन्ति॑ वृ॒श्चन्ति॒ वना॒ वना॑ वृ॒श्चन्ति॑ । \newline
43. वृ॒श्चन्ति॒ शिक्व॑सः॒ शिक्व॑सो वृ॒श्चन्ति॑ वृ॒श्चन्ति॒ शिक्व॑सः । \newline
44. शिक्व॑स॒ इति॒ शिक्व॑सः । \newline
45. अग्ने॒ भूरी॑णि॒ भूरी॒ण्यग्ने ऽग्ने॒ भूरी॑णि । \newline
46. भूरी॑णि॒ तव॒ तव॒ भूरी॑णि॒ भूरी॑णि॒ तव॑ । \newline
47. तव॑ जातवेदो जातवेद॒ स्तव॒ तव॑ जातवेदः । \newline
48. जा॒त॒वे॒दो॒ देव॒ देव॑ जातवेदो जातवेदो॒ देव॑ । \newline
49. जा॒त॒वे॒द॒ इति॑ जात - वे॒दः॒ । \newline
50. देव॑ स्वधावः स्वधावो॒ देव॒ देव॑ स्वधावः । \newline
51. स्व॒धा॒वो॒ ऽमृत॑स्या॒ मृत॑स्य स्वधावः स्वधावो॒ ऽमृत॑स्य । \newline
52. स्व॒धा॒व॒ इति॑ स्वधा - वः॒ । \newline
53. अ॒मृत॑स्य॒ धाम॒ धामा॒ मृत॑स्या॒ मृत॑स्य॒ धाम॑ । \newline
54. धामेति॒ धाम॑ । \newline
55. याश्च॑ च॒ या याश्च॑ । \newline
56. च॒ मा॒या मा॒याश्च॑ च मा॒याः । \newline

\textbf{Ghana Paata } \newline

1. ऋ॒ष्टि॒मन्त॒ आप॒ आप॑ ऋष्टि॒मन्त॑ ऋष्टि॒मन्त॒ आप॑ इवे॒ वाप॑ ऋष्टि॒मन्त॑ ऋष्टि॒मन्त॒ आप॑ इव । \newline
2. ऋ॒ष्टि॒मन्त॒ इत्यृ॑ष्टि - मन्तः॑ । \newline
3. आप॑ इवे॒ वाप॒ आप॑ इव स॒द्ध्रिय॑ञ्चः स॒द्ध्रिय॑ञ्च इ॒वाप॒ आप॑ इव स॒द्ध्रिय॑ञ्चः । \newline
4. इ॒व॒ स॒द्ध्रिय॑ञ्चः स॒द्ध्रिय॑ञ्च इवे व स॒द्ध्रिय॑ञ्चो धवद्ध्वे धवद्ध्वे स॒द्ध्रिय॑ञ्च इवे व स॒द्ध्रिय॑ञ्चो धवद्ध्वे । \newline
5. स॒द्ध्रिय॑ञ्चो धवद्ध्वे धवद्ध्वे स॒द्ध्रिय॑ञ्चः स॒द्ध्रिय॑ञ्चो धवद्ध्वे । \newline
6. ध॒व॒द्ध्व॒ इति॑ धवद्ध्वे । \newline
7. अ॒भि क्र॑न्द क्रन्दा॒भ्य॑भि क्र॑न्द स्त॒नय॑ स्त॒नय॑ क्रन्दा॒भ्य॑भि क्र॑न्द स्त॒नय॑ । \newline
8. क्र॒न्द॒ स्त॒नय॑ स्त॒नय॑ क्रन्द क्रन्द स्त॒नय॒ गर्भ॒म् गर्भꣳ॑ स्त॒नय॑ क्रन्द क्रन्द स्त॒नय॒ गर्भ᳚म् । \newline
9. स्त॒नय॒ गर्भ॒म् गर्भꣳ॑ स्त॒नय॑ स्त॒नय॒ गर्भ॒ मा गर्भꣳ॑ स्त॒नय॑ स्त॒नय॒ गर्भ॒ मा । \newline
10. गर्भ॒ मा गर्भ॒म् गर्भ॒ मा धा॑ धा॒ आ गर्भ॒म् गर्भ॒ मा धाः᳚ । \newline
11. आ धा॑ धा॒ आ धा॑ उद॒न्व तो॑द॒न्वता॑ धा॒ आ धा॑ उद॒न्वता᳚ । \newline
12. धा॒ उ॒द॒न्व तो॑द॒न्वता॑ धा धा उद॒न्वता॒ परि॒ पर्यु॑द॒न्वता॑ धा धा उद॒न्वता॒ परि॑ । \newline
13. उ॒द॒न्वता॒ परि॒ पर्यु॑द॒न्व तो॑द॒न्वता॒ परि॑ दीय दीय॒ पर्यु॑द॒न्व तो॑द॒न्वता॒ परि॑ दीय । \newline
14. उ॒द॒न्वतेत्यु॑दन्न् - वता᳚ । \newline
15. परि॑ दीय दीय॒ परि॒ परि॑ दीया॒ रथे॑न॒ रथे॑न दीय॒ परि॒ परि॑ दीया॒ रथे॑न । \newline
16. दी॒या॒ रथे॑न॒ रथे॑न दीय दीया॒ रथे॑न । \newline
17. रथे॒नेति॒ रथे॑न । \newline
18. दृतिꣳ॒॒ सु सु दृति॒म् दृतिꣳ॒॒ सु क॑र्.ष कर्.ष॒ सु दृति॒म् दृतिꣳ॒॒ सु क॑र्.ष । \newline
19. सु क॑र्.ष कर्.ष॒ सु सु क॑र्.ष॒ विषि॑तं॒ ॅविषि॑तम् कर्.ष॒ सु सु क॑र्.ष॒ विषि॑तम् । \newline
20. क॒र्॒.ष॒ विषि॑तं॒ ॅविषि॑तम् कर्.ष कर्.ष॒ विषि॑त॒म् न्य॑ञ्च॒म् न्य॑ञ्चं॒ ॅविषि॑तम् कर्.ष कर्.ष॒ विषि॑त॒म् न्य॑ञ्चम् । \newline
21. विषि॑त॒म् न्य॑ञ्च॒म् न्य॑ञ्चं॒ ॅविषि॑तं॒ ॅविषि॑त॒म् न्य॑ञ्चꣳ स॒माः स॒मा न्य॑ञ्चं॒ ॅविषि॑तं॒ ॅविषि॑त॒म् न्य॑ञ्चꣳ स॒माः । \newline
22. विषि॑त॒मिति॒ वि - सि॒त॒म् । \newline
23. न्य॑ञ्चꣳ स॒माः स॒मा न्य॑ञ्च॒म् न्य॑ञ्चꣳ स॒मा भ॑वन्तु भवन्तु स॒मा न्य॑ञ्च॒म् न्य॑ञ्चꣳ स॒मा भ॑वन्तु । \newline
24. स॒मा भ॑वन्तु भवन्तु स॒माः स॒मा भ॑व न्तू॒द्व तो॒द्वता॑ भवन्तु स॒माः स॒मा भ॑व न्तू॒द्वता᳚ । \newline
25. भ॒व॒ न्तू॒द्व तो॒द्वता॑ भवन्तु भवन्तू॒द्वता॑ निपा॒दा नि॑पा॒दा उ॒द्वता॑ भवन्तु भवन्तू॒द्वता॑ निपा॒दाः । \newline
26. उ॒द्वता॑ निपा॒दा नि॑पा॒दा उ॒द्वतो॒द्वता॑ निपा॒दाः । \newline
27. उ॒द्वतेत्य॑त् - वता᳚ । \newline
28. नि॒पा॒दा इति॑ नि - पा॒दाः । \newline
29. त्वम् त्या त्या त्वम् त्वम् त्या चि॑च् चि॒त् त्या त्वम् त्वम् त्या चि॑त् । \newline
30. त्या चि॑च् चि॒त् त्या त्या चि॒दच्यु॒ता ऽच्यु॑ता चि॒त् त्या त्या चि॒दच्यु॑ता । \newline
31. चि॒दच्यु॒ता ऽच्यु॑ता चिच् चि॒दच्यु॒ता ऽग्ने ऽग्ने॒ अच्यु॑ता चिच् चि॒दच्यु॒ता ऽग्ने᳚ । \newline
32. अच्यु॒ता ऽग्ने ऽग्ने॒ अच्यु॒ता ऽच्यु॒ता ऽग्ने॑ प॒शुः प॒शुरग्ने॒ अच्यु॒ता ऽच्यु॒ता ऽग्ने॑ प॒शुः । \newline
33. अग्ने॑ प॒शुः प॒शु रग्ने ऽग्ने॑ प॒शुर् न न प॒शु रग्ने ऽग्ने॑ प॒शुर् न । \newline
34. प॒शुर् न न प॒शुः प॒शुर् न यव॑से॒ यव॑से॒ न प॒शुः प॒शुर् न यव॑से । \newline
35. न यव॑से॒ यव॑से॒ न न यव॑से । \newline
36. यव॑स॒ इति॒ यव॑से । \newline
37. धामा॑ ह ह॒ धाम॒ धामा॑ ह॒ यद् यद्ध॒ धाम॒ धामा॑ ह॒ यत् । \newline
38. ह॒ यद् यद्ध॑ ह॒ यत् ते॑ ते॒ यद्ध॑ ह॒ यत् ते᳚ । \newline
39. यत् ते॑ ते॒ यद् यत् ते॑ अजरा जर ते॒ यद् यत् ते॑ अजर । \newline
40. ते॒ अ॒ज॒रा॒ ज॒र॒ ते॒ ते॒ अ॒ज॒र॒ वना॒ वना॑ ऽजर ते ते अजर॒ वना᳚ । \newline
41. अ॒ज॒र॒ वना॒ वना॑ ऽजरा जर॒ वना॑ वृ॒श्चन्ति॑ वृ॒श्चन्ति॒ वना॑ ऽजरा जर॒ वना॑ वृ॒श्चन्ति॑ । \newline
42. वना॑ वृ॒श्चन्ति॑ वृ॒श्चन्ति॒ वना॒ वना॑ वृ॒श्चन्ति॒ शिक्व॑सः॒ शिक्व॑सो वृ॒श्चन्ति॒ वना॒ वना॑ वृ॒श्चन्ति॒ शिक्व॑सः । \newline
43. वृ॒श्चन्ति॒ शिक्व॑सः॒ शिक्व॑सो वृ॒श्चन्ति॑ वृ॒श्चन्ति॒ शिक्व॑सः । \newline
44. शिक्व॑स॒ इति॒ शिक्व॑सः । \newline
45. अग्ने॒ भूरी॑णि॒ भूरी॒ण्यग्ने ऽग्ने॒ भूरी॑णि॒ तव॒ तव॒ भूरी॒ण्यग्ने ऽग्ने॒ भूरी॑णि॒ तव॑ । \newline
46. भूरी॑णि॒ तव॒ तव॒ भूरी॑णि॒ भूरी॑णि॒ तव॑ जातवेदो जातवेद॒ स्तव॒ भूरी॑णि॒ भूरी॑णि॒ तव॑ जातवेदः । \newline
47. तव॑ जातवेदो जातवेद॒ स्तव॒ तव॑ जातवेदो॒ देव॒ देव॑ जातवेद॒ स्तव॒ तव॑ जातवेदो॒ देव॑ । \newline
48. जा॒त॒वे॒दो॒ देव॒ देव॑ जातवेदो जातवेदो॒ देव॑ स्वधावः स्वधावो॒ देव॑ जातवेदो जातवेदो॒ देव॑ स्वधावः । \newline
49. जा॒त॒वे॒द॒ इति॑ जात - वे॒दः॒ । \newline
50. देव॑ स्वधावः स्वधावो॒ देव॒ देव॑ स्वधावो॒ ऽमृत॑स्या॒ मृत॑स्य स्वधावो॒ देव॒ देव॑ स्वधावो॒ ऽमृत॑स्य । \newline
51. स्व॒धा॒वो॒ ऽमृत॑स्या॒ मृत॑स्य स्वधावः स्वधावो॒ ऽमृत॑स्य॒ धाम॒ धामा॒मृत॑स्य स्वधावः स्वधावो॒ ऽमृत॑स्य॒ धाम॑ । \newline
52. स्व॒धा॒व॒ इति॑ स्वधा - वः॒ । \newline
53. अ॒मृत॑स्य॒ धाम॒ धामा॒ मृत॑स्या॒ मृत॑स्य॒ धाम॑ । \newline
54. धामेति॒ धाम॑ । \newline
55. याश्च॑ च॒ या याश्च॑ मा॒या मा॒याश्च॒ या याश्च॑ मा॒याः । \newline
56. च॒ मा॒या मा॒याश्च॑ च मा॒या मा॒यिना᳚म् मा॒यिना᳚म् मा॒याश्च॑ च मा॒या मा॒यिना᳚म् । \newline
\pagebreak
\markright{ TS 3.1.11.7  \hfill https://www.vedavms.in \hfill}

\section{ TS 3.1.11.7 }

\textbf{TS 3.1.11.7 } \newline
\textbf{Samhita Paata} \newline

मा॒या मा॒यिनां᳚ ॅविश्वमिन्व॒ त्वे पू॒र्वीः स॑न्द॒धुः पृ॑ष्टबन्धो ॥ दि॒वो नो॑ वृ॒ष्टिं म॑रुतो ररीद्ध्वं॒ प्रपि॑न्वत॒ वृष्णो॒ अश्व॑स्य॒ धाराः᳚ । अ॒र्वाङे॒तेन॑ स्तनयि॒त्नुनेह्य॒पो नि॑षि॒ञ्चन्नसु॑रः पि॒ता नः॑ ॥ पिन्व॑न्त्य॒पो म॒रुतः॑ सु॒दान॑वः॒ पयो॑ घृ॒तव॑द्वि॒दथे᳚ष्वा॒ भुवः॑ । अत्यं॒ न मि॒हे वि न॑यन्ति वा॒जिन॒मुथ्सं॑ दुहन्ति स्त॒नय॑न्त॒मक्षि॑तं ॥ उ॒द॒प्रुतो॑ मरुत॒स्ताꣳ इ॑यर्त॒ वृष्टिं॒ - [  ] \newline

\textbf{Pada Paata} \newline

मा॒याः । मा॒यिना᳚म् । वि॒श्व॒मि॒न्वेति॑ विश्वम्-इ॒न्व॒ । त्वे इति॑ । पू॒र्वीः । स॒न्द॒धुरिति॑ सं - द॒धुः । पृ॒ष्ट॒ब॒न्धो॒ इति॑ पृष्ट - ब॒न्धो॒ ॥ दि॒वः । नः॒ । वृ॒ष्टिम् । म॒रु॒तः॒ । र॒री॒द्ध्व॒म् । प्रेति॑ । पि॒न्व॒त॒ । वृष्णः॑ । अश्व॑स्य । धाराः᳚ ॥ अ॒र्वाङ् । ए॒तेन॑ । स्त॒न॒यि॒त्नुना᳚ । एति॑ । इ॒हि॒ । अ॒पः । नि॒षि॒ञ्चन्निति॑ नि -सि॒ञ्चन्न् । असु॑रः । पि॒ता । नः॒ ॥ पिन्व॑न्ति । अ॒पः । म॒रुतः॑ । सु॒दान॑व॒ इति॑ सु - दान॑वः । पयः॑ । घृ॒तव॒दिति॑ घृ॒त - व॒त् । वि॒दथे॑षु । आ॒भुव॒ इत्या᳚ - भुवः॑ ॥ अत्य᳚म् । न । मि॒हे । वीति॑ । न॒य॒न्ति॒ । वा॒जिन᳚म् । उथ्स᳚म् । दु॒ह॒न्ति॒ । स्त॒नय॑न्तम् । अक्षि॑तम् ॥ उ॒द॒प्रुत॒ इत्यु॑द - प्रुतः॑ । म॒रु॒तः॒ । तान् । इ॒य॒र्त॒ । वृष्टि᳚म् ।  \newline


\textbf{Krama Paata} \newline

मा॒या मा॒यिना᳚म् । मा॒यिनां᳚ ॅविश्वमिन्व । वि॒श्व॒मि॒न्व॒ त्वे । वि॒श्व॒मि॒न्वेति॑ विश्वम् - इ॒न्व॒ । त्वे पू॒र्वीः । त्वे इति॒ त्वे । पू॒र्वीः स॑न्द॒धुः । स॒न्द॒धुः पृ॑ष्टबन्धो । स॒न्द॒धुरिति॑ सं - द॒धुः । पृ॒ष्ट॒ब॒न्धो॒ इति॑ पृष्ट - ब॒न्धो॒ ॥ दि॒वो नः॑ । नो॒ वृ॒ष्टिम् । वृ॒ष्टिम् म॑रुतः । म॒रु॒तो॒ र॒री॒द्ध्व॒म् । र॒री॒द्ध्व॒म् प्र । प्र पि॑न्वत । पि॒न्व॒त॒ वृष्णः॑ । वृष्णो॒ अश्व॑स्य । अश्व॑स्य॒ धाराः᳚ । धारा॒ इति॒ धाराः᳚ ॥ अ॒र्वाङे॒तेन॑ । ए॒तेन॑ स्तनयि॒त्नुना᳚ । स्त॒न॒यि॒त्नुना । एहि॑ । इ॒ह्य॒पः । अ॒पो नि॑षि॒ञ्चन्न् । नि॒षि॒ञ्चन्नसु॑रः । नि॒षि॒ञ्चन्निति॑ नि - सि॒ञ्चन्न् । असु॑रः पि॒ता । पि॒ता नः॑ । न॒ इति॑ नः ॥ पिन्व॑न्त्य॒पः । अ॒पो म॒रुतः॑ । म॒रुतः॑ सु॒दान॑वः । सु॒दान॑वः॒ पयः॑ । सु॒दान॑व॒ इति॑ सु - दान॑वः । पयो॑ घृ॒तव॑त् । घृ॒तव॑द् वि॒दथे॑षु । घृ॒तव॒दिति॑ घृ॒त - व॒त्॒ । वि॒दथे᳚ष्वा॒भुवः॑ । आ॒भुव॒ इत्या᳚ - भुवः॑ ॥ अत्य॒म् न । न मि॒हे । मि॒हे वि । वि न॑यन्ति । न॒य॒न्ति॒ वा॒जिन᳚म् । वा॒जिन॒मुथ्स᳚म् । उथ्स॑म् दुहन्ति । दु॒ह॒न्ति॒ स्त॒नय॑न्तम् । स्त॒नय॑न्त॒मक्षि॑तम् । अक्षि॑त॒मित्यक्षि॑तम् ॥ उ॒द॒प्रुतो॑ मरुतः । उ॒द॒प्रुत॒ इत्यु॑द - प्रुतः॑ । म॒रु॒त॒स्तान् । ताꣳ इ॑यर्त । इ॒य॒र्त॒ वृष्टि᳚म् । वृष्टिं॒ ॅये \newline

\textbf{Jatai Paata} \newline

1. मा॒या मा॒यिना᳚म् मा॒यिना᳚म् मा॒या मा॒या मा॒यिना᳚म् । \newline
2. मा॒यिनां᳚ ॅविश्वमिन्व विश्वमिन्व मा॒यिना᳚म् मा॒यिनां᳚ ॅविश्वमिन्व । \newline
3. वि॒श्व॒मि॒न्व॒ त्वे त्वे वि॑श्वमिन्व विश्वमिन्व॒ त्वे । \newline
4. वि॒श्व॒मि॒न्वेति॑ विश्वम् - इ॒न्व॒ । \newline
5. त्वे पू॒र्वीः पू॒र्वी स्त्वे त्वे पू॒र्वीः । \newline
6. त्वे इति॒ त्वे । \newline
7. पू॒र्वीः स॑न्द॒धुः स॑न्द॒धुः पू॒र्वीः पू॒र्वीः स॑न्द॒धुः । \newline
8. स॒न्द॒धुः पृ॑ष्टबन्धो पृष्टबन्धो सन्द॒धुः स॑न्द॒धुः पृ॑ष्टबन्धो । \newline
9. स॒न्द॒धुरिति॑ सं - द॒धुः । \newline
10. पृ॒ष्ट॒ब॒न्धो॒ इति॑ पृष्ट - ब॒न्धो॒ । \newline
11. दि॒वो नो॑ नो दि॒वो दि॒वो नः॑ । \newline
12. नो॒ वृ॒ष्टिं ॅवृ॒ष्टिम् नो॑ नो वृ॒ष्टिम् । \newline
13. वृ॒ष्टिम् म॑रुतो मरुतो वृ॒ष्टिं ॅवृ॒ष्टिम् म॑रुतः । \newline
14. म॒रु॒तो॒ र॒री॒द्ध्वꣳ॒॒ र॒री॒द्ध्व॒म् म॒रु॒तो॒ म॒रु॒तो॒ र॒री॒द्ध्व॒म् । \newline
15. र॒री॒द्ध्व॒म् प्र प्र र॑रीद्ध्वꣳ ररीद्ध्व॒म् प्र । \newline
16. प्र पि॑न्वत पिन्वत॒ प्र प्र पि॑न्वत । \newline
17. पि॒न्व॒त॒ वृष्णो॒ वृष्णः॑ पिन्वत पिन्वत॒ वृष्णः॑ । \newline
18. वृष्णो॒ अश्व॒स्या श्व॑स्य॒ वृष्णो॒ वृष्णो॒ अश्व॑स्य । \newline
19. अश्व॑स्य॒ धारा॒ धारा॒ अश्व॒स्या श्व॑स्य॒ धाराः᳚ । \newline
20. धारा॒ इति॒ धाराः᳚ । \newline
21. अ॒र्वा ङे॒ते नै॒तेना॒र्वा ङ॒र्वा ङे॒तेन॑ । \newline
22. ए॒तेन॑ स्तनयि॒त्नुना᳚ स्तनयि॒त्नु नै॒ते नै॒तेन॑ स्तनयि॒त्नुना᳚ । \newline
23. स्त॒न॒यि॒त्नुना ऽऽस्त॑नयि॒त्नुना᳚ स्तनयि॒त्नुना । \newline
24. एही॒ह्येहि॑ । \newline
25. इ॒ह्य॒पो अ॒प इ॑हीह्य॒पः । \newline
26. अ॒पो नि॑षि॒ञ्चन् नि॑षि॒ञ्चन् न॒पो अ॒पो नि॑षि॒ञ्चन्न् । \newline
27. नि॒षि॒ञ्चन् नसु॑रो॒ असु॑रो निषि॒ञ्चन् नि॑षि॒ञ्चन् नसु॑रः । \newline
28. नि॒षि॒ञ्चन्निति॑ नि - सि॒ञ्चन्न् । \newline
29. असु॑रः पि॒ता पि॒ता ऽसु॑रो॒ असु॑रः पि॒ता । \newline
30. पि॒ता नो॑ नः पि॒ता पि॒ता नः॑ । \newline
31. न॒ इति॑ नः । \newline
32. पिन्व॑ न्त्य॒पो अ॒पः पिन्व॑न्ति॒ पिन्व॑ न्त्य॒पः । \newline
33. अ॒पो म॒रुतो॑ म॒रुतो॑ अ॒पो अ॒पो म॒रुतः॑ । \newline
34. म॒रुतः॑ सु॒दान॑वः सु॒दान॑वो म॒रुतो॑ म॒रुतः॑ सु॒दान॑वः । \newline
35. सु॒दान॑वः॒ पयः॒ पयः॑ सु॒दान॑वः सु॒दान॑वः॒ पयः॑ । \newline
36. सु॒दान॑व॒ इति॑ सु - दान॑वः । \newline
37. पयो॑ घृ॒तव॑द् घृ॒तव॒त् पयः॒ पयो॑ घृ॒तव॑त् । \newline
38. घृ॒तव॑द् वि॒दथे॑षु वि॒दथे॑षु घृ॒तव॑द् घृ॒तव॑द् वि॒दथे॑षु । \newline
39. घृ॒तव॒दिति॑ घृ॒त - व॒त् । \newline
40. वि॒दथे᳚ ष्वा॒भुव॑ आ॒भुवो॑ वि॒दथे॑षु वि॒दथे᳚ ष्वा॒भुवः॑ । \newline
41. आ॒भुव॒ इत्या᳚ - भुवः॑ । \newline
42. अत्य॒म् न नात्य॒ मत्य॒म् न । \newline
43. न मि॒हे मि॒हे न न मि॒हे । \newline
44. मि॒हे वि वि मि॒हे मि॒हे वि । \newline
45. वि न॑यन्ति नयन्ति॒ वि वि न॑यन्ति । \newline
46. न॒य॒न्ति॒ वा॒जिनं॑ ॅवा॒जिन॑म् नयन्ति नयन्ति वा॒जिन᳚म् । \newline
47. वा॒जिन॒ मुथ्स॒ मुथ्सं॑ ॅवा॒जिनं॑ ॅवा॒जिन॒ मुथ्स᳚म् । \newline
48. उथ्स॑म् दुहन्ति दुह॒ न्त्युथ्स॒ मुथ्स॑म् दुहन्ति । \newline
49. दु॒ह॒न्ति॒ स्त॒नय॑न्तꣳ स्त॒नय॑न्तम् दुहन्ति दुहन्ति स्त॒नय॑न्तम् । \newline
50. स्त॒नय॑न्त॒ मक्षि॑त॒ मक्षि॑तꣳ स्त॒नय॑न्तꣳ स्त॒नय॑न्त॒ मक्षि॑तम् । \newline
51. अक्षि॑त॒मित्यक्षि॑तम् । \newline
52. उ॒द॒प्रुतो॑ मरुतो मरुत उद॒प्रुत॑ उद॒प्रुतो॑ मरुतः । \newline
53. उ॒द॒प्रुत॒ इत्यु॑द - प्रुतः॑ । \newline
54. म॒रु॒त॒ स्ताꣳ स्तान् म॑रुतो मरुत॒ स्तान् । \newline
55. ताꣳ इ॑यर्ते यर्त॒ ताꣳ स्ताꣳ इ॑यर्त । \newline
56. इ॒य॒र्त॒ वृष्टिं॒ ॅवृष्टि॑ मियर्ते यर्त॒ वृष्टि᳚म् । \newline
57. वृष्टिं॒ ॅये ये वृष्टिं॒ ॅवृष्टिं॒ ॅये । \newline

\textbf{Ghana Paata } \newline

1. मा॒या मा॒यिना᳚म् मा॒यिना᳚म् मा॒या मा॒या मा॒यिनां᳚ ॅविश्वमिन्व विश्वमिन्व मा॒यिना᳚म् मा॒या मा॒या मा॒यिनां᳚ ॅविश्वमिन्व । \newline
2. मा॒यिनां᳚ ॅविश्वमिन्व विश्वमिन्व मा॒यिना᳚म् मा॒यिनां᳚ ॅविश्वमिन्व॒ त्वे त्वे वि॑श्वमिन्व मा॒यिना᳚म् मा॒यिनां᳚ ॅविश्वमिन्व॒ त्वे । \newline
3. वि॒श्व॒मि॒न्व॒ त्वे त्वे वि॑श्वमिन्व विश्वमिन्व॒ त्वे पू॒र्वीः पू॒र्वी स्त्वे वि॑श्वमिन्व विश्वमिन्व॒ त्वे पू॒र्वीः । \newline
4. वि॒श्व॒मि॒न्वेति॑ विश्वम् - इ॒न्व॒ । \newline
5. त्वे पू॒र्वीः पू॒र्वी स्त्वे त्वे पू॒र्वीः स॑न्द॒धुः स॑न्द॒धुः पू॒र्वी स्त्वे त्वे पू॒र्वीः स॑न्द॒धुः । \newline
6. त्वे इति॒ त्वे । \newline
7. पू॒र्वीः स॑न्द॒धुः स॑न्द॒धुः पू॒र्वीः पू॒र्वीः स॑न्द॒धुः पृ॑ष्टबन्धो पृष्टबन्धो सन्द॒धुः पू॒र्वीः पू॒र्वीः स॑न्द॒धुः पृ॑ष्टबन्धो । \newline
8. स॒न्द॒धुः पृ॑ष्टबन्धो पृष्टबन्धो सन्द॒धुः स॑न्द॒धुः पृ॑ष्टबन्धो । \newline
9. स॒न्द॒धुरिति॑ सं - द॒धुः । \newline
10. पृ॒ष्ट॒ब॒न्धो॒ इति॑ पृष्ट - ब॒न्धो॒ । \newline
11. दि॒वो नो॑ नो दि॒वो दि॒वो नो॑ वृ॒ष्टिं ॅवृ॒ष्टिम् नो॑ दि॒वो दि॒वो नो॑ वृ॒ष्टिम् । \newline
12. नो॒ वृ॒ष्टिं ॅवृ॒ष्टिम् नो॑ नो वृ॒ष्टिम् म॑रुतो मरुतो वृ॒ष्टिम् नो॑ नो वृ॒ष्टिम् म॑रुतः । \newline
13. वृ॒ष्टिम् म॑रुतो मरुतो वृ॒ष्टिं ॅवृ॒ष्टिम् म॑रुतो ररीद्ध्वꣳ ररीद्ध्वम् मरुतो वृ॒ष्टिं ॅवृ॒ष्टिम् म॑रुतो ररीद्ध्वम् । \newline
14. म॒रु॒तो॒ र॒री॒द्ध्वꣳ॒॒ र॒री॒द्ध्व॒म् म॒रु॒तो॒ म॒रु॒तो॒ र॒री॒द्ध्व॒म् प्र प्र र॑रीद्ध्वम् मरुतो मरुतो ररीद्ध्व॒म् प्र । \newline
15. र॒री॒द्ध्व॒म् प्र प्र र॑रीद्ध्वꣳ ररीद्ध्व॒म् प्र पि॑न्वत पिन्वत॒ प्र र॑रीद्ध्वꣳ ररीद्ध्व॒म् प्र पि॑न्वत । \newline
16. प्र पि॑न्वत पिन्वत॒ प्र प्र पि॑न्वत॒ वृष्णो॒ वृष्णः॑ पिन्वत॒ प्र प्र पि॑न्वत॒ वृष्णः॑ । \newline
17. पि॒न्व॒त॒ वृष्णो॒ वृष्णः॑ पिन्वत पिन्वत॒ वृष्णो॒ अश्व॒स्या श्व॑स्य॒ वृष्णः॑ पिन्वत पिन्वत॒ वृष्णो॒ अश्व॑स्य । \newline
18. वृष्णो॒ अश्व॒स्या श्व॑स्य॒ वृष्णो॒ वृष्णो॒ अश्व॑स्य॒ धारा॒ धारा॒ अश्व॑स्य॒ वृष्णो॒ वृष्णो॒ अश्व॑स्य॒ धाराः᳚ । \newline
19. अश्व॑स्य॒ धारा॒ धारा॒ अश्व॒स्या श्व॑स्य॒ धाराः᳚ । \newline
20. धारा॒ इति॒ धाराः᳚ । \newline
21. अ॒र्वा ङे॒ते नै॒ते ना॒र्वा ङ॒र्वा ङे॒तेन॑ स्तनयि॒त्नुना᳚ स्तनयि॒त्नु नै॒तेना॒र्वा ङ॒र्वा ङे॒तेन॑ स्तनयि॒त्नुना᳚ । \newline
22. ए॒तेन॑ स्तनयि॒त्नुना᳚ स्तनयि॒त्नु नै॒तेनै॒तेन॑ स्तनयि॒त्नुना ऽऽस्त॑नयि॒त्नु नै॒तेनै॒तेन॑ स्तनयि॒त्नुना । \newline
23. स्त॒न॒यि॒त्नुना ऽऽस्त॑नयि॒त्नुना᳚ स्तनयि॒त्नु नेही॒ह्या स्त॑नयि॒त्नुना᳚ स्तनयि॒त्नु नेहि॑ । \newline
24. एही॒ ह्येह्य॒पो अ॒प इ॒ह्येह्य॒पः । \newline
25. इ॒ह्य॒पो अ॒प इ॑हीह्य॒पो नि॑षि॒ञ्चन् नि॑षि॒ञ्चन् न॒प इ॑हीह्य॒पो नि॑षि॒ञ्चन्न् । \newline
26. अ॒पो नि॑षि॒ञ्चन् नि॑षि॒ञ्चन् न॒पो अ॒पो नि॑षि॒ञ्चन् नसु॑रो॒ असु॑रो निषि॒ञ्चन् न॒पो अ॒पो नि॑षि॒ञ्चन् नसु॑रः । \newline
27. नि॒षि॒ञ्चन् नसु॑रो॒ असु॑रो निषि॒ञ्चन् नि॑षि॒ञ्चन् नसु॑रः पि॒ता पि॒ता ऽसु॑रो निषि॒ञ्चन् नि॑षि॒ञ्चन् नसु॑रः पि॒ता । \newline
28. नि॒षि॒ञ्चन्निति॑ नि - सि॒ञ्चन्न् । \newline
29. असु॑रः पि॒ता पि॒ता ऽसु॑रो॒ ऽसु॑रः पि॒ता नो॑ नः पि॒ता ऽसु॑रो॒ असु॑रः पि॒ता नः॑ । \newline
30. पि॒ता नो॑ नः पि॒ता पि॒ता नः॑ । \newline
31. न॒ इति॑ नः । \newline
32. पिन्व॑ न्त्य॒पो अ॒पः पिन्व॑न्ति॒ पिन्व॑ न्त्य॒पो म॒रुतो॑ म॒रुतो॑ अ॒पः पिन्व॑न्ति॒ पिन्व॑ न्त्य॒पो म॒रुतः॑ । \newline
33. अ॒पो म॒रुतो॑ म॒रुतो॑ अ॒पो अ॒पो म॒रुतः॑ सु॒दान॑वः सु॒दान॑वो म॒रुतो॑ अ॒पो अ॒पो म॒रुतः॑ सु॒दान॑वः । \newline
34. म॒रुतः॑ सु॒दान॑वः सु॒दान॑वो म॒रुतो॑ म॒रुतः॑ सु॒दान॑वः॒ पयः॒ पयः॑ सु॒दान॑वो म॒रुतो॑ म॒रुतः॑ सु॒दान॑वः॒ पयः॑ । \newline
35. सु॒दान॑वः॒ पयः॒ पयः॑ सु॒दान॑वः सु॒दान॑वः॒ पयो॑ घृ॒तव॑द् घृ॒तव॒त् पयः॑ सु॒दान॑वः सु॒दान॑वः॒ पयो॑ घृ॒तव॑त् । \newline
36. सु॒दान॑व॒ इति॑ सु - दान॑वः । \newline
37. पयो॑ घृ॒तव॑द् घृ॒तव॒त् पयः॒ पयो॑ घृ॒तव॑द् वि॒दथे॑षु वि॒दथे॑षु घृ॒तव॒त् पयः॒ पयो॑ घृ॒तव॑द् वि॒दथे॑षु । \newline
38. घृ॒तव॑द् वि॒दथे॑षु वि॒दथे॑षु घृ॒तव॑द् घृ॒तव॑द् वि॒दथे᳚ ष्वा॒भुव॑ आ॒भुवो॑ वि॒दथे॑षु घृ॒तव॑द् घृ॒तव॑द् वि॒दथे᳚ ष्वा॒भुवः॑ । \newline
39. घृ॒तव॒दिति॑ घृ॒त - व॒त् । \newline
40. वि॒दथे᳚ ष्वा॒भुव॑ आ॒भुवो॑ वि॒दथे॑षु वि॒दथे᳚ ष्वा॒भुवः॑ । \newline
41. आ॒भुव॒ इत्या᳚ - भुवः॑ । \newline
42. अत्य॒म् न नात्य॒ मत्य॒म् न मि॒हे मि॒हे नात्य॒ मत्य॒म् न मि॒हे । \newline
43. न मि॒हे मि॒हे न न मि॒हे वि वि मि॒हे न न मि॒हे वि । \newline
44. मि॒हे वि वि मि॒हे मि॒हे वि न॑यन्ति नयन्ति॒ वि मि॒हे मि॒हे वि न॑यन्ति । \newline
45. वि न॑यन्ति नयन्ति॒ वि वि न॑यन्ति वा॒जिनं॑ ॅवा॒जिन॑म् नयन्ति॒ वि वि न॑यन्ति वा॒जिन᳚म् । \newline
46. न॒य॒न्ति॒ वा॒जिनं॑ ॅवा॒जिन॑म् नयन्ति नयन्ति वा॒जिन॒ मुथ्स॒ मुथ्सं॑ ॅवा॒जिन॑म् नयन्ति नयन्ति वा॒जिन॒ मुथ्स᳚म् । \newline
47. वा॒जिन॒ मुथ्स॒ मुथ्सं॑ ॅवा॒जिनं॑ ॅवा॒जिन॒ मुथ्स॑म् दुहन्ति दुह॒ न्त्युथ्सं॑ ॅवा॒जिनं॑ ॅवा॒जिन॒ मुथ्स॑म् दुहन्ति । \newline
48. उथ्स॑म् दुहन्ति दुह॒ न्त्युथ्स॒ मुथ्स॑म् दुहन्ति स्त॒नय॑न्तꣳ स्त॒नय॑न्तम् दुह॒ न्त्युथ्स॒ मुथ्स॑म् दुहन्ति स्त॒नय॑न्तम् । \newline
49. दु॒ह॒न्ति॒ स्त॒नय॑न्तꣳ स्त॒नय॑न्तम् दुहन्ति दुहन्ति स्त॒नय॑न्त॒ मक्षि॑त॒ मक्षि॑तꣳ स्त॒नय॑न्तम् दुहन्ति दुहन्ति स्त॒नय॑न्त॒ मक्षि॑तम् । \newline
50. स्त॒नय॑न्त॒ मक्षि॑त॒ मक्षि॑तꣳ स्त॒नय॑न्तꣳ स्त॒नय॑न्त॒ मक्षि॑तम् । \newline
51. अक्षि॑त॒मित्यक्षि॑तम् । \newline
52. उ॒द॒प्रुतो॑ मरुतो मरुत उद॒प्रुत॑ उद॒प्रुतो॑ मरुत॒ स्ताꣳ स्तान् म॑रुत उद॒प्रुत॑ उद॒प्रुतो॑ मरुत॒ स्तान् । \newline
53. उ॒द॒प्रुत॒ इत्यु॑द - प्रुतः॑ । \newline
54. म॒रु॒त॒ स्ताꣳ स्तान् म॑रुतो मरुत॒ स्ताꣳ इ॑यर्ते यर्त॒ तान् म॑रुतो मरुत॒ स्ताꣳ इ॑यर्त । \newline
55. ताꣳ इ॑यर्ते यर्त॒ ताꣳ स्ताꣳ इ॑यर्त॒ वृष्टिं॒ ॅवृष्टि॑ मियर्त॒ ताꣳ स्ताꣳ इ॑यर्त॒ वृष्टि᳚म् । \newline
56. इ॒य॒र्त॒ वृष्टिं॒ ॅवृष्टि॑ मियर्ते यर्त॒ वृष्टिं॒ ॅये ये वृष्टि॑ मियर्ते यर्त॒ वृष्टिं॒ ॅये । \newline
57. वृष्टिं॒ ॅये ये वृष्टिं॒ ॅवृष्टिं॒ ॅये विश्वे॒ विश्वे॒ ये वृष्टिं॒ ॅवृष्टिं॒ ॅये विश्वे᳚ । \newline
\pagebreak
\markright{ TS 3.1.11.8  \hfill https://www.vedavms.in \hfill}

\section{ TS 3.1.11.8 }

\textbf{TS 3.1.11.8 } \newline
\textbf{Samhita Paata} \newline

ॅये विश्वे॑ म॒रुतो॑ जु॒नन्ति॑ । क्रोशा॑ति॒ गर्दा॑ क॒न्ये॑व तु॒न्ना पेरुं॑ तुञ्जा॒ना पत्ये॑व जा॒या ॥ घृ॒तेन॒ द्यावा॑पृथि॒वी मधु॑ना॒ समु॑क्षत॒ पय॑स्वतीः कृण॒ताऽऽ*प॒ ओष॑धीः । ऊर्जं॑ च॒ तत्र॑ सुम॒तिं च॑ पिन्वथ॒ यत्रा॑ नरो मरुतः सि॒ञ्चथा॒ मधु॑ ॥ उदु॒त्यं >7, चि॒त्रं >8 ॥ औ॒र्व॒-भृ॒गु॒वच्छुचि॑मप्नवान॒वदा हु॑वे । अ॒ग्निꣳ स॑मु॒द्रवा॑ससं ॥ आ स॒वꣳ स॑वि॒तुर्य॑था॒ भग॑स्ये ( ) व भु॒जिꣳ हु॑वे । अ॒ग्निꣳ स॑मु॒द्रवा॑ससं ॥ हु॒वे वात॑स्वनं क॒विं प॒र्जन्य॑क्रन्द्यꣳ॒॒ सहः॑ । अ॒ग्निꣳ स॑मु॒द्रवा॑ससं ॥ \newline

\textbf{Pada Paata} \newline

ये । विश्वे᳚ । म॒रुतः॑ । जु॒नन्ति॑ ॥ क्रोशा॑ति । गर्दा᳚ । क॒न्या᳚ । इ॒व॒ । तु॒न्ना । पेरु᳚म् । तु॒ञ्जा॒ना । पत्या᳚ । इ॒व॒ । जा॒या ॥ घृ॒तेन॑ । द्यावा॑पृथि॒वी इति॒ द्यावा᳚ - पृ॒थि॒वी । मधु॑ना । समिति॑ । उ॒क्ष॒त॒ । पय॑स्वतीः । कृ॒णु॒त॒ । आपः॑ । ओष॑धीः ॥ ऊर्ज᳚म् । च॒ । तत्र॑ । सु॒म॒तिमिति॑ सु - म॒तिम् । च॒ । पि॒न्व॒थ॒ । यत्र॑ । न॒रः॒ । म॒रु॒तः॒ । सि॒ञ्चथ॑ । मधु॑ ॥ उदिति॑ । उ॒ । त्यम् । चि॒त्रम् ॥ औ॒र्व॒भृ॒गु॒वदित्यौ᳚वभृगु - वत् । शुचि᳚म् । अ॒प्न॒वा॒न॒वदित्य॑प्नवान - वत् । एति॑ । हु॒वे॒ ॥ अ॒ग्निम् । स॒मु॒द्रवा॑सस॒मिति॑ समु॒द्र - वा॒स॒स॒म् ॥ एति॑ । स॒वम् । स॒वि॒तुः । य॒था॒ । भग॑स्य ( ) । इ॒व॒ । भु॒जिम् । हु॒वे॒ ॥ अ॒ग्निम् । स॒मु॒द्रवा॑सस॒मिति॑ समु॒द्र - वा॒स॒स॒म् ॥ हु॒वे । वात॑स्वन॒मिति॒ वात॑ - स्व॒न॒म् । क॒विम् । प॒र्जन्य॑क्रन्द्य॒मिति॑ प॒र्जन्य॑ - क्र॒न्द्य॒म् । सहः॑ ॥ अ॒ग्निम् । स॒मु॒द्रवा॑सस॒मिति॑ समु॒द्र - वा॒स॒स॒म् ॥  \newline


\textbf{Krama Paata} \newline

ये विश्वे᳚ । विश्वे॑ म॒रुतः॑ । म॒रुतो॑ जु॒नन्ति॑ । जु॒नन्तीति॑ जु॒नन्ति॑ ॥ क्रोशा॑ति॒ गर्दा᳚ । गर्दा॑ क॒न्या᳚ । क॒न्ये॑व । इ॒व॒ तु॒न्ना । तु॒न्ना पेरु᳚म् । पेरु॑म् तुञ्जा॒ना । तु॒ञ्जा॒ना पत्या᳚ । पत्ये॑व । इ॒व॒ जा॒या । जा॒येति॑ जा॒या ॥ घृ॒तेन॒ द्यावा॑पृथि॒वी । द्यावा॑पृथि॒वी मधु॑ना । द्यावा॑पृथि॒वी इति॒ द्यावा᳚ - पृ॒थि॒वी । मधु॑ना॒ सम् । समु॑क्षत । उ॒क्ष॒त॒ पय॑स्वतीः । पय॑स्वतीः कृणुत । कृ॒णु॒तापः॑ । आप॒ ओष॑धीः । ओष॑धी॒रित्योष॑धीः ॥ ऊर्ज॑म् च । च॒ तत्र॑ । तत्र॑ सुम॒तिम् । सु॒म॒तिम् च॑ । सु॒म॒तिमिति॑ सु - म॒तिम् । च॒ पि॒न्व॒थ॒ । पि॒न्व॒थ॒ यत्र॑ । यत्रा॑ नरः । न॒रो॒ म॒रु॒तः॒ । म॒रु॒तः॒ सि॒ञ्चथ॑ । सि॒ञ्चथा॒ मधु॑ । मध्विति॒ मधु॑ ॥ उदु॑ । उ॒ त्यम् । त्यम् चि॒त्रम् । चि॒त्रमिति॑ चि॒त्रम् ॥ औ॒र्व॒भृ॒गु॒वच्छुचि᳚म् । औ॒र्व॒भृ॒गु॒वदि,त्यौ᳚र्वभृगु - वत् । शुचि॑मप्नवान॒वत् । अ॒प्न॒वा॒न॒वदा । अ॒प्न॒वा॒न॒वदित्य॑प्नवान - वत् । आ हु॑वे । हु॒व॒ इति॑ हुवे ॥ अ॒ग्निꣳ स॑मु॒द्रवा॑ससम् । स॒मु॒द्रवा॑सस॒मिति॑ समु॒द्र - वा॒स॒स॒म् ॥ आ स॒वम् । स॒वꣳ स॑वि॒तुः । स॒वि॒तुर् य॑था । य॒था॒ भग॑स्य ( ) । भग॑स्येव । इ॒व॒ भु॒जिम् । भु॒जिꣳ हु॑वे । हु॒व॒ इति॑ हुवे ॥ अ॒ग्निꣳ स॑मु॒द्रवा॑ससम् । स॒मु॒द्रवा॑सस॒मिति॑ समु॒द्र - वा॒स॒स॒म् ॥ हु॒वे वात॑स्वनम् । वात॑स्वनम् क॒विम् । वात॑स्वन॒मिति॒ वात॑ - स्व॒न॒म् । क॒विम् प॒र्जन्य॑क्रन्द्यम् । प॒र्जन्य॑क्रन्द्यꣳ॒॒ सहः॑ । प॒र्जन्य॑क्रन्द्य॒मिति॑ प॒र्जन्य॑ - क्र॒न्द्य॒म् । सह॒ इति॒ सहः॑ ॥ अ॒ग्निꣳ स॑मु॒द्रवा॑ससम् । स॒मु॒द्रवा॑सस॒मिति॑ समु॒द्र - वा॒स॒स॒म्॒ । \newline

\textbf{Jatai Paata} \newline

1. ये विश्वे॒ विश्वे॒ ये ये विश्वे᳚ । \newline
2. विश्वे॑ म॒रुतो॑ म॒रुतो॒ विश्वे॒ विश्वे॑ म॒रुतः॑ । \newline
3. म॒रुतो॑ जु॒नन्ति॑ जु॒नन्ति॑ म॒रुतो॑ म॒रुतो॑ जु॒नन्ति॑ । \newline
4. जु॒नन्तीति॑ जु॒नन्ति॑ । \newline
5. क्रोशा॑ति॒ गर्दा॒ गर्दा॒ क्रोशा॑ति॒ क्रोशा॑ति॒ गर्दा᳚ । \newline
6. गर्दा॑ क॒न्या॑ क॒न्या॑ गर्दा॒ गर्दा॑ क॒न्या᳚ । \newline
7. क॒न्ये॑वे व क॒न्या॑ क॒न्ये॑व । \newline
8. इ॒व॒ तु॒न्ना तु॒न्नेवे॑ व तु॒न्ना । \newline
9. तु॒न्ना पेरु॒म् पेरु॑म् तु॒न्ना तु॒न्ना पेरु᳚म् । \newline
10. पेरु॑म् तुञ्जा॒ना तु॑ञ्जा॒ना पेरु॒म् पेरु॑म् तुञ्जा॒ना । \newline
11. तु॒ञ्जा॒ना पत्या॒ पत्या॑ तुञ्जा॒ना तु॑ञ्जा॒ना पत्या᳚ । \newline
12. पत्ये॑वे व॒ पत्या॒ पत्ये॑व । \newline
13. इ॒व॒ जा॒या जा॒येवे॑ व जा॒या । \newline
14. जा॒येति॑ जा॒या । \newline
15. घृ॒तेन॒ द्यावा॑पृथि॒वी द्यावा॑पृथि॒वी घृ॒तेन॑ घृ॒तेन॒ द्यावा॑पृथि॒वी । \newline
16. द्यावा॑पृथि॒वी मधु॑ना॒ मधु॑ना॒ द्यावा॑पृथि॒वी द्यावा॑पृथि॒वी मधु॑ना । \newline
17. द्यावा॑पृथि॒वी इति॒ द्यावा᳚ - पृ॒थि॒वी । \newline
18. मधु॑ना॒ सꣳ सम् मधु॑ना॒ मधु॑ना॒ सम् । \newline
19. स मु॑क्ष तोक्षत॒ सꣳ स मु॑क्षत । \newline
20. उ॒क्ष॒त॒ पय॑स्वतीः॒ पय॑स्वती रुक्षतोक्षत॒ पय॑स्वतीः । \newline
21. पय॑स्वतीः कृणुत कृणुत॒ पय॑स्वतीः॒ पय॑स्वतीः कृणुत । \newline
22. कृ॒णु॒ताप॒ आपः॑ कृणुत कृणु॒तापः॑ । \newline
23. आप॒ ओष॑धी॒ रोष॑धी॒राप॒ आप॒ ओष॑धीः । \newline
24. ओष॑धी॒रित्योष॑धीः । \newline
25. ऊर्ज॑म् च॒ चोर्ज॒ मूर्ज॑म् च । \newline
26. च॒ तत्र॒ तत्र॑ च च॒ तत्र॑ । \newline
27. तत्र॑ सुम॒तिꣳ सु॑म॒तिम् तत्र॒ तत्र॑ सुम॒तिम् । \newline
28. सु॒म॒तिम् च॑ च सुम॒तिꣳ सु॑म॒तिम् च॑ । \newline
29. सु॒म॒तिमिति॑ सु - म॒तिम् । \newline
30. च॒ पि॒न्व॒थ॒ पि॒न्व॒थ॒ च॒ च॒ पि॒न्व॒थ॒ । \newline
31. पि॒न्व॒थ॒ यत्र॒ यत्र॑ पिन्वथ पिन्वथ॒ यत्र॑ । \newline
32. यत्रा॑ नरो नरो॒ यत्र॒ यत्रा॑ नरः । \newline
33. न॒रो॒ म॒रु॒तो॒ म॒रु॒तो॒ न॒रो॒ न॒रो॒ म॒रु॒तः॒ । \newline
34. म॒रु॒तः॒ सि॒ञ्चथ॑ सि॒ञ्चथ॑ मरुतो मरुतः सि॒ञ्चथ॑ । \newline
35. सि॒ञ्चथा॒ मधु॒ मधु॑ सि॒ञ्चथ॑ सि॒ञ्चथा॒ मधु॑ । \newline
36. मध्विति॒ मधु॑ । \newline
37. उदु॑ वु॒ वुदुदु॑ । \newline
38. उ॒ त्यम् त्य मु॑ वु॒ त्यम् । \newline
39. त्यम् चि॒त्रम् चि॒त्रम् त्यम् त्यम् चि॒त्रम् । \newline
40. चि॒त्रमिति॑ चि॒त्रम् । \newline
41. औ॒र्व॒भृ॒गु॒वच् छुचिꣳ॒॒ शुचि॑ मौर्वभृगु॒व दौ᳚र्वभृगु॒वच् छुचि᳚म् । \newline
42. औ॒र्व॒भृ॒गु॒वदित्यौ᳚वभृगु - वत् । \newline
43. शुचि॑ मप्नवान॒व द॑प्नवान॒वच् छुचिꣳ॒॒ शुचि॑ मप्नवान॒वत् । \newline
44. अ॒प्न॒वा॒न॒वदा ऽप्न॑वान॒व द॑प्नवान॒वदा । \newline
45. अ॒प्न॒वा॒न॒वदित्य॑प्नवान - वत् । \newline
46. आ हु॑वे हुव॒ आ हु॑वे । \newline
47. हु॒व॒ इति॑ हुवे । \newline
48. अ॒ग्निꣳ स॑मु॒द्रवा॑ससꣳ समु॒द्रवा॑सस म॒ग्नि म॒ग्निꣳ स॑मु॒द्रवा॑ससम् । \newline
49. स॒मु॒द्रवा॑सस॒मिति॑ समु॒द्र - वा॒स॒स॒म् । \newline
50. आ स॒वꣳ स॒व मा स॒वम् । \newline
51. स॒वꣳ स॑वि॒तुः स॑वि॒तुः स॒वꣳ स॒वꣳ स॑वि॒तुः । \newline
52. स॒वि॒तुर् य॑था यथा सवि॒तुः स॑वि॒तुर् य॑था । \newline
53. य॒था॒ भग॑स्य॒ भग॑स्य यथा यथा॒ भग॑स्य । \newline
54. भग॑स्ये वे व॒ भग॑स्य॒ भग॑स्ये व । \newline
55. इ॒व॒ भु॒जिम् भु॒जि मि॑वे व भु॒जिम् । \newline
56. भु॒जिꣳ हु॑वे हुवे भु॒जिम् भु॒जिꣳ हु॑वे । \newline
57. हु॒व॒ इति॑ हुवे । \newline
58. अ॒ग्निꣳ स॑मु॒द्रवा॑ससꣳ समु॒द्रवा॑सस म॒ग्नि म॒ग्निꣳ स॑मु॒द्रवा॑ससम् । \newline
59. स॒मु॒द्रवा॑सस॒मिति॑ समु॒द्र - वा॒स॒स॒म् । \newline
60. हु॒वे वात॑स्वनं॒ ॅवात॑स्वनꣳ हु॒वे हु॒वे वात॑स्वनम् । \newline
61. वात॑स्वनम् क॒विम् क॒विं ॅवात॑स्वनं॒ ॅवात॑स्वनम् क॒विम् । \newline
62. वात॑स्वन॒मिति॒ वात॑ - स्व॒न॒म् । \newline
63. क॒विम् प॒र्जन्य॑क्रन्द्यम् प॒र्जन्य॑क्रन्द्यम् क॒विम् क॒विम् प॒र्जन्य॑क्रन्द्यम् । \newline
64. प॒र्जन्य॑क्रन्द्यꣳ॒॒ सहः॒ सहः॑ प॒र्जन्य॑क्रन्द्यम् प॒र्जन्य॑क्रन्द्यꣳ॒॒ सहः॑ । \newline
65. प॒र्जन्य॑क्रन्द्य॒मिति॑ प॒र्जन्य॑ - क्र॒न्द्य॒म् । \newline
66. सह॒ इति॒ सहः॑ । \newline
67. अ॒ग्निꣳ स॑मु॒द्रवा॑ससꣳ समु॒द्रवा॑सस म॒ग्नि म॒ग्निꣳ स॑मु॒द्रवा॑ससम् । \newline
68. स॒मु॒द्रवा॑सस॒मिति॑ समु॒द्र - वा॒स॒स॒म् । \newline

\textbf{Ghana Paata } \newline

1. ये विश्वे॒ विश्वे॒ ये ये विश्वे॑ म॒रुतो॑ म॒रुतो॒ विश्वे॒ ये ये विश्वे॑ म॒रुतः॑ । \newline
2. विश्वे॑ म॒रुतो॑ म॒रुतो॒ विश्वे॒ विश्वे॑ म॒रुतो॑ जु॒नन्ति॑ जु॒नन्ति॑ म॒रुतो॒ विश्वे॒ विश्वे॑ म॒रुतो॑ जु॒नन्ति॑ । \newline
3. म॒रुतो॑ जु॒नन्ति॑ जु॒नन्ति॑ म॒रुतो॑ म॒रुतो॑ जु॒नन्ति॑ । \newline
4. जु॒नन्तीति॑ जु॒नन्ति॑ । \newline
5. क्रोशा॑ति॒ गर्दा॒ गर्दा॒ क्रोशा॑ति॒ क्रोशा॑ति॒ गर्दा॑ क॒न्या॑ क॒न्या॑ गर्दा॒ क्रोशा॑ति॒ क्रोशा॑ति॒ गर्दा॑ क॒न्या᳚ । \newline
6. गर्दा॑ क॒न्या॑ क॒न्या॑ गर्दा॒ गर्दा॑ क॒न्ये॑वे व क॒न्या॑ गर्दा॒ गर्दा॑ क॒न्ये॑व । \newline
7. क॒न्ये॑वे व क॒न्या॑ क॒न्ये॑व तु॒न्ना तु॒न्नेव॑ क॒न्या॑ क॒न्ये॑व तु॒न्ना । \newline
8. इ॒व॒ तु॒न्ना तु॒न्नेवे॑ व तु॒न्ना पेरु॒म् पेरु॑म् तु॒न्नेवे॑ व तु॒न्ना पेरु᳚म् । \newline
9. तु॒न्ना पेरु॒म् पेरु॑म् तु॒न्ना तु॒न्ना पेरु॑म् तुञ्जा॒ना तु॑ञ्जा॒ना पेरु॑म् तु॒न्ना तु॒न्ना पेरु॑म् तुञ्जा॒ना । \newline
10. पेरु॑म् तुञ्जा॒ना तु॑ञ्जा॒ना पेरु॒म् पेरु॑म् तुञ्जा॒ना पत्या॒ पत्या॑ तुञ्जा॒ना पेरु॒म् पेरु॑म् तुञ्जा॒ना पत्या᳚ । \newline
11. तु॒ञ्जा॒ना पत्या॒ पत्या॑ तुञ्जा॒ना तु॑ञ्जा॒ना पत्ये॑वे व॒ पत्या॑ तुञ्जा॒ना तु॑ञ्जा॒ना पत्ये॑व । \newline
12. पत्ये॑वे व॒ पत्या॒ पत्ये॑व जा॒या जा॒येव॒ पत्या॒ पत्ये॑व जा॒या । \newline
13. इ॒व॒ जा॒या जा॒येवे॑ व जा॒या । \newline
14. जा॒येति॑ जा॒या । \newline
15. घृ॒तेन॒ द्यावा॑पृथि॒वी द्यावा॑पृथि॒वी घृ॒तेन॑ घृ॒तेन॒ द्यावा॑पृथि॒वी मधु॑ना॒ मधु॑ना॒ द्यावा॑पृथि॒वी घृ॒तेन॑ घृ॒तेन॒ द्यावा॑पृथि॒वी मधु॑ना । \newline
16. द्यावा॑पृथि॒वी मधु॑ना॒ मधु॑ना॒ द्यावा॑पृथि॒वी द्यावा॑पृथि॒वी मधु॑ना॒ सꣳ सम् मधु॑ना॒ द्यावा॑पृथि॒वी द्यावा॑पृथि॒वी मधु॑ना॒ सम् । \newline
17. द्यावा॑पृथि॒वी इति॒ द्यावा᳚ - पृ॒थि॒वी । \newline
18. मधु॑ना॒ सꣳ सम् मधु॑ना॒ मधु॑ना॒ स मु॑क्षतोक्षत॒ सम् मधु॑ना॒ मधु॑ना॒ स मु॑क्षत । \newline
19. स मु॑क्षतोक्षत॒ सꣳ स मु॑क्षत॒ पय॑स्वतीः॒ पय॑स्वती रुक्षत॒ सꣳ स मु॑क्षत॒ पय॑स्वतीः । \newline
20. उ॒क्ष॒त॒ पय॑स्वतीः॒ पय॑स्वती रुक्षतोक्षत॒ पय॑स्वतीः कृणुत कृणुत॒ पय॑स्वती रुक्षतोक्षत॒ पय॑स्वतीः कृणुत । \newline
21. पय॑स्वतीः कृणुत कृणुत॒ पय॑स्वतीः॒ पय॑स्वतीः कृणु॒ताप॒ आपः॑ कृणुत॒ पय॑स्वतीः॒ पय॑स्वतीः कृणु॒तापः॑ । \newline
22. कृ॒णु॒ताप॒ आपः॑ कृणुत कृणु॒ताप॒ ओष॑धी॒ रोष॑धी॒ रापः॑ कृणुत कृणु॒ताप॒ ओष॑धीः । \newline
23. आप॒ ओष॑धी॒ रोष॑धी॒ राप॒ आप॒ ओष॑धीः । \newline
24. ओष॑धी॒रित्योष॑धीः । \newline
25. ऊर्ज॑म् च॒ चोर्ज॒ मूर्ज॑म् च॒ तत्र॒ तत्र॒ चोर्ज॒ मूर्ज॑म् च॒ तत्र॑ । \newline
26. च॒ तत्र॒ तत्र॑ च च॒ तत्र॑ सुम॒तिꣳ सु॑म॒तिम् तत्र॑ च च॒ तत्र॑ सुम॒तिम् । \newline
27. तत्र॑ सुम॒तिꣳ सु॑म॒तिम् तत्र॒ तत्र॑ सुम॒तिम् च॑ च सुम॒तिम् तत्र॒ तत्र॑ सुम॒तिम् च॑ । \newline
28. सु॒म॒तिम् च॑ च सुम॒तिꣳ सु॑म॒तिम् च॑ पिन्वथ पिन्वथ च सुम॒तिꣳ सु॑म॒तिम् च॑ पिन्वथ । \newline
29. सु॒म॒तिमिति॑ सु - म॒तिम् । \newline
30. च॒ पि॒न्व॒थ॒ पि॒न्व॒थ॒ च॒ च॒ पि॒न्व॒थ॒ यत्र॒ यत्र॑ पिन्वथ च च पिन्वथ॒ यत्र॑ । \newline
31. पि॒न्व॒थ॒ यत्र॒ यत्र॑ पिन्वथ पिन्वथ॒ यत्रा॑ नरो नरो॒ यत्र॑ पिन्वथ पिन्वथ॒ यत्रा॑ नरः । \newline
32. यत्रा॑ नरो नरो॒ यत्र॒ यत्रा॑ नरो मरुतो मरुतो नरो॒ यत्र॒ यत्रा॑ नरो मरुतः । \newline
33. न॒रो॒ म॒रु॒तो॒ म॒रु॒तो॒ न॒रो॒ न॒रो॒ म॒रु॒तः॒ सि॒ञ्चथ॑ सि॒ञ्चथ॑ मरुतो नरो नरो मरुतः सि॒ञ्चथ॑ । \newline
34. म॒रु॒तः॒ सि॒ञ्चथ॑ सि॒ञ्चथ॑ मरुतो मरुतः सि॒ञ्चथा॒ मधु॒ मधु॑ सि॒ञ्चथ॑ मरुतो मरुतः सि॒ञ्चथा॒ मधु॑ । \newline
35. सि॒ञ्चथा॒ मधु॒ मधु॑ सि॒ञ्चथ॑ सि॒ञ्चथा॒ मधु॑ । \newline
36. मध्विति॒ मधु॑ । \newline
37. उदु॑ वु॒ वुदुदु॒ त्यम् त्य मु॒ वुदुदु॒ त्यम् । \newline
38. उ॒ त्यम् त्य मु॑ वु॒ त्यम् चि॒त्रम् चि॒त्रम् त्य मु॑ वु॒ त्यम् चि॒त्रम् । \newline
39. त्यम् चि॒त्रम् चि॒त्रम् त्यम् त्यम् चि॒त्रम् । \newline
40. चि॒त्रमिति॑ चि॒त्रम् । \newline
41. औ॒र्व॒भृ॒गु॒वच् छुचिꣳ॒॒ शुचि॑ मौर्वभृगु॒व दौ᳚र्वभृगु॒वच् छुचि॑ मप्नवान॒व द॑प्नवान॒वच् छुचि॑ मौर्वभृगु॒व दौ᳚र्वभृगु॒वच् छुचि॑ मप्नवान॒वत् । \newline
42. औ॒र्व॒भृ॒गु॒वदित्यौ᳚वभृगु - वत् । \newline
43. शुचि॑ मप्नवान॒व द॑प्नवान॒वच् छुचिꣳ॒॒ शुचि॑ मप्नवान॒वदा ऽप्न॑वान॒वच् छुचिꣳ॒॒ शुचि॑ मप्नवान॒वदा । \newline
44. अ॒प्न॒वा॒न॒वदा ऽप्न॑वान॒व द॑प्नवान॒वदा हु॑वे हुव॒ आ ऽप्न॑वान॒व द॑प्नवान॒वदा हु॑वे । \newline
45. अ॒प्न॒वा॒न॒वदित्य॑प्नवान - वत् । \newline
46. आ हु॑वे हुव॒ आ हु॑वे । \newline
47. हु॒व॒ इति॑ हुवे । \newline
48. अ॒ग्निꣳ स॑मु॒द्रवा॑ससꣳ समु॒द्रवा॑सस म॒ग्नि म॒ग्निꣳ स॑मु॒द्रवा॑ससम् । \newline
49. स॒मु॒द्रवा॑सस॒मिति॑ समु॒द्र - वा॒स॒स॒म् । \newline
50. आ स॒वꣳ स॒व मा स॒वꣳ स॑वि॒तुः स॑वि॒तुः स॒व मा स॒वꣳ स॑वि॒तुः । \newline
51. स॒वꣳ स॑वि॒तुः स॑वि॒तुः स॒वꣳ स॒वꣳ स॑वि॒तुर् य॑था यथा सवि॒तुः स॒वꣳ स॒वꣳ स॑वि॒तुर् य॑था । \newline
52. स॒वि॒तुर् य॑था यथा सवि॒तुः स॑वि॒तुर् य॑था॒ भग॑स्य॒ भग॑स्य यथा सवि॒तुः स॑वि॒तुर् य॑था॒ भग॑स्य । \newline
53. य॒था॒ भग॑स्य॒ भग॑स्य यथा यथा॒ भग॑स्ये वे व॒ भग॑स्य यथा यथा॒ भग॑स्ये व । \newline
54. भग॑स्ये वे व॒ भग॑स्य॒ भग॑स्ये व भु॒जिम् भु॒जि मि॑व॒ भग॑स्य॒ भग॑स्ये व भु॒जिम् । \newline
55. इ॒व॒ भु॒जिम् भु॒जि मि॑वे व भु॒जिꣳ हु॑वे हुवे भु॒जि मि॑वे व भु॒जिꣳ हु॑वे । \newline
56. भु॒जिꣳ हु॑वे हुवे भु॒जिम् भु॒जिꣳ हु॑वे । \newline
57. हु॒व॒ इति॑ हुवे । \newline
58. अ॒ग्निꣳ स॑मु॒द्रवा॑ससꣳ समु॒द्रवा॑सस म॒ग्नि म॒ग्निꣳ स॑मु॒द्रवा॑ससम् । \newline
59. स॒मु॒द्रवा॑सस॒मिति॑ समु॒द्र - वा॒स॒स॒म् । \newline
60. हु॒वे वात॑स्वनं॒ ॅवात॑स्वनꣳ हु॒वे हु॒वे वात॑स्वनम् क॒विम् क॒विं ॅवात॑स्वनꣳ हु॒वे हु॒वे वात॑स्वनम् क॒विम् । \newline
61. वात॑स्वनम् क॒विम् क॒विं ॅवात॑स्वनं॒ ॅवात॑स्वनम् क॒विम् प॒र्जन्य॑क्रन्द्यम् प॒र्जन्य॑क्रन्द्यम् क॒विं ॅवात॑स्वनं॒ ॅवात॑स्वनम् क॒विम् प॒र्जन्य॑क्रन्द्यम् । \newline
62. वात॑स्वन॒मिति॒ वात॑ - स्व॒न॒म् । \newline
63. क॒विम् प॒र्जन्य॑क्रन्द्यम् प॒र्जन्य॑क्रन्द्यम् क॒विम् क॒विम् प॒र्जन्य॑क्रन्द्यꣳ॒॒ सहः॒ सहः॑ प॒र्जन्य॑क्रन्द्यम् क॒विम् क॒विम् प॒र्जन्य॑क्रन्द्यꣳ॒॒ सहः॑ । \newline
64. प॒र्जन्य॑क्रन्द्यꣳ॒॒ सहः॒ सहः॑ प॒र्जन्य॑क्रन्द्यम् प॒र्जन्य॑क्रन्द्यꣳ॒॒ सहः॑ । \newline
65. प॒र्जन्य॑क्रन्द्य॒मिति॑ प॒र्जन्य॑ - क्र॒न्द्य॒म् । \newline
66. सह॒ इति॒ सहः॑ । \newline
67. अ॒ग्निꣳ स॑मु॒द्रवा॑ससꣳ समु॒द्रवा॑सस म॒ग्नि म॒ग्निꣳ स॑मु॒द्रवा॑ससम् । \newline
68. स॒मु॒द्रवा॑सस॒मिति॑ समु॒द्र - वा॒स॒स॒म् । \newline
\pagebreak


\end{document}