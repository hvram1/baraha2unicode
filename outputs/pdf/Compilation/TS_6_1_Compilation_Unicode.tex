\documentclass[17pt]{extarticle}
\usepackage{babel}
\usepackage{fontspec}
\usepackage{polyglossia}
\usepackage{extsizes}



\setmainlanguage{sanskrit}
\setotherlanguages{english} %% or other languages
\setlength{\parindent}{0pt}
\pagestyle{myheadings}
\newfontfamily\devanagarifont[Script=Devanagari]{AdishilaVedic}


\newcommand{\VAR}[1]{}
\newcommand{\BLOCK}[1]{}




\begin{document}
\begin{titlepage}
    \begin{center}
 
\begin{sanskrit}
    { \Huge
    कृष्ण यजुर्वेदीय तैत्तिरीय संहिता,पद,जटा,घन पाठः 
    }
    \\
    \vspace{2.5cm}
    \mbox{ \Huge
    6.1      षष्ठकाण्डे प्रथमः प्रश्नः - सोममन्त्रब्राह्मणनिरूपणं   }
\end{sanskrit}
\end{center}

\end{titlepage}
\tableofcontents
\pagebreak

\markright{ TS 6.1.1.1  \hfill https://www.vedavms.in \hfill}
\addcontentsline{toc}{section}{ TS 6.1.1.1 }
\section*{ TS 6.1.1.1 }

\textbf{TS 6.1.1.1 } \newline
\textbf{Samhita Paata} \newline

प्रा॒चीन॑वꣳशं करोति देवमनु॒ष्या दिशो॒ व्य॑भजन्त॒ प्राचीं᳚ दे॒वा द॑क्षि॒णा पि॒तरः॑ प्र॒तीचीं᳚ मनु॒ष्या॑ उदी॑चीꣳ रु॒द्रा यत् प्रा॒चीन॑वꣳशं क॒रोति॑ देवलो॒कमे॒व तद्-यज॑मान उ॒पाव॑र्तते॒ परि॑ श्रयत्य॒न्तर्.हि॑तो॒हि दे॑वलो॒को म॑नुष्यलो॒का-न्नास्माल्लो॒काथ् स्वे॑तव्यमि॒वेत्या॑हुः॒ को हि तद्-वेद॒ यद्-य॒मुष्मि॑ॅल्लो॒केऽस्ति॑ वा॒ न वेति॑ दि॒क्ष्व॑ती का॒शान् क॑रो - [  ] \newline

\textbf{Pada Paata} \newline

प्रा॒चीन॑वꣳश॒मिति॑ प्रा॒चीन॑ - वꣳ॒॒श॒म् । क॒रो॒ति॒ । दे॒व॒म॒नु॒ष्या इति॑ देव - म॒नु॒ष्याः । दिशः॑ । वीति॑ । अ॒भ॒ज॒न्त॒ । प्राची᳚म् । दे॒वाः । द॒क्षि॒णा । पि॒तरः॑ । प्र॒तीची᳚म् । म॒नु॒ष्याः᳚ । उदी॑चीम् । रु॒द्राः । यत् । प्रा॒चीन॑वꣳश॒मिति॑ प्रा॒चीन॑ - वꣳ॒॒श॒म् । क॒रोति॑ । दे॒व॒लो॒कमिति॑ देव - लो॒कम् । ए॒व । तत् । यज॑मानः । उ॒पाव॑र्तत॒ इत्यु॑प - आव॑र्तते । परीति॑ । श्र॒य॒ति॒ । अ॒न्तर्.हि॑त॒ इत्य॒न्तः - हि॒तः॒ । हि । दे॒व॒लो॒क इति॑ देव - लो॒कः । म॒नु॒ष्य॒लो॒कादिति॑ मनुष्य - लो॒कात् । न । अ॒स्मात् । लो॒कात् । स्वे॑तव्य॒मिति॒ सु - ए॒त॒व्य॒म् । इ॒व॒ । इति॑ । आ॒हुः॒ । कः । हि । तत् । वेद॑ । यदि॑ । अ॒मुष्मिन्न्॑ । लो॒के । अस्ति॑ । वा॒ । न । वा॒ । इति॑ । दि॒क्षु । अ॒ती॒का॒शान् । क॒रो॒ति॒ ।  \newline




\markright{ TS 6.1.1.2  \hfill https://www.vedavms.in \hfill}
\addcontentsline{toc}{section}{ TS 6.1.1.2 }
\section*{ TS 6.1.1.2 }

\textbf{TS 6.1.1.2 } \newline
\textbf{Samhita Paata} \newline

-त्यु॒भयो᳚र्लो॒कयो॑-र॒भिजि॑त्यै केशश्म॒श्रु व॑पते न॒खानि॒ नि कृ॑न्तते मृ॒ता वा ए॒षा त्वग॑मे॒द्ध्या यत् के॑शश्म॒श्रु मृ॒तामे॒व त्वच॑म-मे॒द्ध्याम॑प॒हत्य॑ य॒ज्ञियो॑ भू॒त्वा मेध॒मुपै॒त्यङ्गि॑रसः सुव॒र्गं ॅलो॒कं ॅयन्तो॒ऽफ्सु दी᳚क्षात॒पसी॒ प्रावे॑शयन्न॒फ्सु स्ना॑ति सा॒क्षादे॒व दी᳚क्षात॒पसी॒ अव॑ रुन्धे ती॒र्थे स्ना॑ति ती॒र्थे हि ते तां प्रावे॑शयन् ती॒र्थे स्ना॑ति - [  ] \newline

\textbf{Pada Paata} \newline

उ॒भयोः᳚ । लो॒कयोः᳚ । अ॒भिजि॑त्या॒ इत्य॒भि - जि॒त्यै॒ । के॒श॒श्म॒श्र्विति॑ केश-श्म॒श्रु । व॒प॒ते॒ । न॒खानि॑ । नीति॑ । कृ॒न्त॒ते॒ । मृ॒ता । वै । ए॒षा । त्वक् । अ॒मे॒द्ध्या । यत् । के॒श॒श्म॒श्र्विति॑ केश - श्म॒श्रु । मृ॒ताम् । ए॒व । त्वच᳚म् । अ॒मे॒द्ध्याम् । अ॒प॒हत्येत्य॑प-हत्य॑ । य॒ज्ञियः॑ । भू॒त्वा । मेध᳚म् । उपेति॑ । ए॒ति॒ । अङ्गि॑रसः । सु॒व॒र्गमिति॑ सुवः - गम् । लो॒कम् । यन्तः॑ । अ॒फ्स्वित्य॑प्-सु । दी॒क्षा॒त॒पसी॒ इति॑ दीक्षा-त॒पसी᳚ । प्रेति॑ । अ॒वे॒श॒य॒न्न् । अ॒फ्स्वित्य॑प्- सु । स्ना॒ति॒ । सा॒क्षादिति॑ स - अ॒क्षात् । ए॒व । दी॒क्षा॒त॒पसी॒ इति॑ दीक्षा - त॒पसी᳚ । अवेति॑ । रु॒न्धे॒ । ती॒र्थे । स्ना॒ति॒ । ती॒र्थे । हि । ते । ताम् । प्रेति॑ । अवे॑शयन्न् । ती॒र्थे । स्ना॒ति॒ ।  \newline




\markright{ TS 6.1.1.3  \hfill https://www.vedavms.in \hfill}
\addcontentsline{toc}{section}{ TS 6.1.1.3 }
\section*{ TS 6.1.1.3 }

\textbf{TS 6.1.1.3 } \newline
\textbf{Samhita Paata} \newline

ती॒र्थमे॒व स॑मा॒नानां᳚ भवत्य॒पो᳚ऽश्नात्यन्तर॒त ए॒व मेद्ध्यो॑ भवति॒ वास॑सा दीक्षयति सौ॒म्यं ॅवै क्षौमं॑ दे॒वत॑या॒ सोम॑मे॒ष दे॒वता॒मुपै॑ति॒ यो दीक्ष॑ते॒ सोम॑स्य त॒नूर॑सि त॒नुवं॑ मे पा॒हीत्या॑ह॒ स्वामे॒व दे॒वता॒मुपै॒त्यथो॑ आ॒शिष॑मे॒वैतामा शा᳚स्ते॒ ऽग्नेस्तू॑षा॒धानं॑ ॅवा॒योर्वा॑त॒पानं॑ पितृ॒णां नी॒विरोष॑धीनां प्रघा॒त - [  ] \newline

\textbf{Pada Paata} \newline

ती॒र्थम् । ए॒व । स॒मा॒नाना᳚म् । भ॒व॒ति॒ । अ॒पः । अ॒श्ना॒ति॒ । अ॒न्त॒र॒तः । ए॒व । मेद्ध्यः॑ । भ॒व॒ति॒ । वास॑सा । दी॒क्ष॒य॒ति॒ । सौ॒म्यम् । वै । क्षौम᳚म् । दे॒वत॑या । सोम᳚म् । ए॒षः । दे॒वता᳚म् । उपेति॑ । ए॒ति॒ । यः । दीक्ष॑ते । सोम॑स्य । त॒नूः । अ॒सि॒ । त॒नुव᳚म् । मे॒ । पा॒हि॒ । इति॑ । आ॒ह॒ । स्वाम् । ए॒व । दे॒वता᳚म् । उपेति॑ । ए॒ति॒ । अथो॒ इति॑ । आ॒शिष॒मित्या᳚ - शिष᳚म् । ए॒व । ए॒ताम् । एति॑ । शा॒स्ते॒ । अ॒ग्नेः । तू॒षा॒धान॒मिति॑ तूष - आ॒धान᳚म् । वा॒योः । वा॒त॒पान॒मिति॑ वात-पान᳚म् । पि॒तृ॒णाम् । नी॒विः । ओष॑धीनाम् । प्र॒घा॒त इति॑ प्र - घा॒तः ।  \newline




\markright{ TS 6.1.1.4  \hfill https://www.vedavms.in \hfill}
\addcontentsline{toc}{section}{ TS 6.1.1.4 }
\section*{ TS 6.1.1.4 }

\textbf{TS 6.1.1.4 } \newline
\textbf{Samhita Paata} \newline

आ॑दि॒त्यानां᳚ प्राचीनता॒नो विश्वे॑षां दे॒वाना॒मोतु॒ र्नक्ष॑त्राणा-मतीका॒शास्तद्वा ए॒तथ् स॑र्व देव॒त्यं॑ ॅयद्-वासो॒ यद्-वास॑सा दी॒क्षय॑ति॒ सर्वा॑भिरे॒वैनं॑ दे॒वता॑भि-र्दीक्षयति ब॒हिःप्रा॑णो॒ वै म॑नु॒ष्य॑स्त-स्याश॑नं प्रा॒णो᳚-ऽश्नाति॒ सप्रा॑ण ए॒व दी᳚क्षत॒ आशि॑तो भवति॒ यावा॑ने॒वास्य॑ प्रा॒णस्तेन॑ स॒ह मेध॒मुपै॑ति घृ॒तं दे॒वानां॒ मस्तु॑ पितृ॒णां निष्प॑क्वं मनु॒ष्या॑णां॒ तद्वा - [  ] \newline

\textbf{Pada Paata} \newline

आ॒दि॒त्याना᳚म् । प्रा॒ची॒न॒ता॒न इति॑ प्राचीन - ता॒नः । विश्वे॑षाम् । दे॒वाना᳚म् । ओतुः॑ । नक्ष॑त्राणाम् । अ॒ती॒का॒शाः । तत् । वै । ए॒तत् । स॒र्व॒दे॒व॒त्य॑मिति॑ सर्व - दे॒व॒त्य᳚म् । यत् । वासः॑ । यत् । वास॑सा । दी॒क्षय॑ति । सर्वा॑भिः । ए॒व । ए॒न॒म् । दे॒वता॑भिः । दी॒क्ष॒य॒ति॒ । ब॒हिःप्रा॑ण॒ इति॑ ब॒हिः - प्रा॒णः॒ । वै । म॒नु॒ष्यः॑ । तस्य॑ । अश॑नम् । प्रा॒ण इति॑ प्र -अ॒नः । अ॒श्नाति॑ । सप्रा॑ण॒ इति॒ स - प्रा॒णः॒ । ए॒व । दी॒क्ष॒ते॒ । आशि॑तः । भ॒व॒ति॒ । यावान्॑ । ए॒व । अ॒स्य॒ । प्रा॒ण इति॑ प्र -अ॒नः । तेन॑ । स॒ह । मेध᳚म् । उपेति॑ । ए॒ति॒ । घृ॒तम् । दे॒वाना᳚म् । मस्तु॑ । पि॒तृ॒णाम् । निष्प॑क्व॒मिति॒ निः - प॒क्व॒म् । म॒नु॒ष्या॑णाम् । तत् । वै ।  \newline




\markright{ TS 6.1.1.5  \hfill https://www.vedavms.in \hfill}
\addcontentsline{toc}{section}{ TS 6.1.1.5 }
\section*{ TS 6.1.1.5 }

\textbf{TS 6.1.1.5 } \newline
\textbf{Samhita Paata} \newline

ए॒तथ् स॑र्वदेव॒त्यं॑ ॅयन्नव॑नीतं॒ ॅयन्नव॑नीतेनाभ्य॒ङ्क्ते सर्वा॑ ए॒व दे॒वताः᳚ प्रीणाति॒ प्रच्यु॑तो॒ वा ए॒षो᳚ऽस्माल्लो॒कादग॑तो देवलो॒कं ॅयो दी᳚क्षि॒तो᳚ ऽन्त॒रेव॒ नव॑नीतं॒ तस्मा॒-न्नव॑नीतेना॒भ्य॑ङ्क्ते ऽनुलो॒मं ॅयजु॑षा॒ व्यावृ॑त्त्या॒ इन्द्रो॑ वृ॒त्रम॑ह॒न् तस्य॑ क॒नीनि॑का॒ परा॑ऽपत॒त् तदाञ्ज॑नम-भव॒द्यदा॒ङ्क्ते चक्षु॑रे॒व भ्रातृ॑व्यस्य वृङ्क्ते॒ दक्षि॑णं॒ पूर्व॒माऽङ्क्ते॑ - [  ] \newline

\textbf{Pada Paata} \newline

ए॒तत् । स॒र्व॒दे॒व॒त्य॑मिति॑ सर्व - दे॒व॒त्य᳚म् । यत् । नव॑नीत॒मिति॒ नव॑ - नी॒त॒म् । यत् । नव॑नीते॒नेति॒ नव॑ - नी॒ते॒न॒ । अ॒भ्य॒ङ्क्त इत्य॑भि - अ॒ङ्क्ते । सर्वाः᳚ । ए॒व । दे॒वताः᳚ । प्री॒णा॒ति॒ । प्रच्यु॑त॒ इति॒ प्र - च्यु॒तः॒ । वै । ए॒षः । अ॒स्मात् । लो॒कात् । अग॑तः । दे॒व॒लो॒कमिति॑ देव - लो॒कम् । यः । दी॒क्षि॒तः । अ॒न्त॒रा । इ॒व॒ । नव॑नीत॒मिति॒ नव॑ - नी॒त॒म् । तस्मा᳚त् । नव॑नीते॒नेति॒ नव॑ - नी॒ते॒न॒ । अ॒भीति॑ । अ॒ङ्क्ते॒ । अ॒नु॒लो॒ममित्य॑नु - लो॒मम् । यजु॑षा । व्यावृ॑त्त्या॒ इति॑ वि - आवृ॑त्त्यै । इन्द्रः॑ । वृ॒त्रम् । अ॒ह॒न्न् । तस्य॑ । क॒नीनि॑का । परेति॑ । अ॒प॒त॒त् । तत् । आञ्ज॑न॒मित्या᳚-अञ्ज॑नम् । अ॒भ॒व॒त् । यत् । आ॒ङ्क्त इत्या᳚ - अ॒ङ्क्ते । चक्षुः॑ । ए॒व । भ्रातृ॑व्यस्य । वृ॒ङ्क्ते॒ । दक्षि॑णम् । पूर्व᳚म् । एति॑ । अ॒ङ्क्ते॒ ।  \newline




\markright{ TS 6.1.1.6  \hfill https://www.vedavms.in \hfill}
\addcontentsline{toc}{section}{ TS 6.1.1.6 }
\section*{ TS 6.1.1.6 }

\textbf{TS 6.1.1.6 } \newline
\textbf{Samhita Paata} \newline

स॒व्यꣳ हि पूर्वं॑ मनु॒ष्या॑ आ॒ञ्जते॒ न नि धा॑वते॒ नीव॒ हि म॑नु॒ष्या॑ धाव॑न्ते॒ पञ्च॒ कृत्व॒ आऽङ्क्ते॒ पञ्चा᳚क्षरा प॒ङ्क्तिः पाङ्क्तो॑ य॒ज्ञो य॒ज्ञ्मे॒वाव॑ रुन्धे॒ परि॑मित॒माङ्क्ते ऽप॑रिमितꣳ॒॒ हि म॑नु॒ष्या॑ आ॒ञ्जते॒ सतू॑ल॒याऽऽङ्क्ते- ऽप॑तूलया॒ हि म॑नु॒ष्या॑ आ॒ञ्जते॒ व्यावृ॑त्त्यै॒ यदप॑तूलयाञ्जी॒त वज्र॑ इव स्या॒थ् सतू॑ल॒याऽऽङ्क्ते॑ मित्र॒त्वाये - [  ] \newline

\textbf{Pada Paata} \newline

स॒व्यम् । हि । पूर्व᳚म् । म॒नु॒ष्याः᳚ । आ॒ञ्जत॒ इत्या᳚ - अ॒ञ्जते᳚ । न । नीति॑ । धा॒व॒ते॒ । नीति॑ । इ॒व॒ । हि । म॒नु॒ष्याः᳚ । धाव॑न्ते । पञ्च॑ । कृत्वः॑ । एति॑ । अ॒ङ्क्ते॒ । पञ्चा᳚क्ष॒रेति॒ पञ्च॑-अ॒क्ष॒रा॒ । प॒ङ्क्तिः । पाङ्क्तः॑ । य॒ज्ञ्ः । य॒ज्ञ्म् । ए॒व । अवेति॑ । रु॒न्धे॒ । परि॑मित॒मिति॒ परि॑ - मि॒त॒म् । एति॑ । अ॒ङ्क्ते॒ । अप॑रिमित॒मित्यप॑रि - मि॒त॒म् । हि । म॒नु॒ष्याः᳚ । आ॒ञ्जत॒ इत्या᳚ - अ॒ञ्जते᳚ । सतू॑ल॒येति॒ स-तू॒ल॒या॒ । एति॑ । अ॒ङ्क्ते॒ । अप॑तूल॒येत्यप॑-तू॒ल॒या॒ । हि । म॒नु॒ष्याः᳚ । आ॒ञ्जत॒ इत्या᳚ - अ॒ञ्जते᳚ । व्यावृ॑त्त्या॒ इति॑ वि - आवृ॑त्त्यै । यत् । अप॑तूल॒येत्यप॑ - तू॒ल॒या॒ । आ॒ञ्जी॒तेत्या᳚ - अ॒ञ्जी॒त । वज्रः॑ । इ॒व॒ । स्या॒त् । सतू॑ल॒येति॒ स - तू॒ल॒या॒ । एति॑ । अ॒ङ्क्ते॒ । मि॒त्र॒त्वायेति॑ मित्र - त्वाय॑ ।  \newline




\markright{ TS 6.1.1.7  \hfill https://www.vedavms.in \hfill}
\addcontentsline{toc}{section}{ TS 6.1.1.7 }
\section*{ TS 6.1.1.7 }

\textbf{TS 6.1.1.7 } \newline
\textbf{Samhita Paata} \newline

-न्द्रो॑ वृ॒त्रम॑ह॒न्थ्सो᳚ऽ(1॒)पो᳚ऽ(1॒)भ्य॑-म्रियत॒ तासां॒ ॅयन्मेद्ध्यं॑ ॅय॒ज्ञियꣳ॒॒ सदे॑व॒मासी॒त् तद॒पोद॑क्राम॒त् ते द॒र्भा अ॑भव॒न॒. यद्द॑र्भपुञ्जी॒लैः प॒वय॑ति॒ या ए॒व मेद्ध्या॑ य॒ज्ञियाः॒ सदे॑वा॒ आप॒स्ताभि॑रे॒वैनं॑ पवयति॒ द्वाभ्यां᳚ पवयत्य-होरा॒त्राभ्या॑मे॒वैनं॑ पवयति त्रि॒भिः प॑वयति॒ त्रय॑ इ॒मे लो॒का ए॒भिरे॒वैनं॑ ॅलो॒कैः प॑वयति प॒ञ्चभिः॑ - [  ] \newline

\textbf{Pada Paata} \newline

इन्द्रः॑ । वृ॒त्रम् । अ॒ह॒न्न् । सः । अ॒पः । अ॒भीति॑ । अ॒म्रि॒य॒त॒ । तासा᳚म् । यत् । मेद्ध्य᳚म् । य॒ज्ञिय᳚म् । सदे॑व॒मिति॒ स - दे॒व॒म् । आसी᳚त् । तत् । अप॑ । उदिति॑ । अ॒क्रा॒म॒त् । ते । द॒र्भाः । अ॒भ॒व॒न्न् । यत् । द॒र्भ॒पु॒ञ्जी॒लैरिति॑ दर्भ - पु॒ञ्जी॒लैः । प॒वय॑ति । याः । ए॒व । मेद्ध्याः᳚ । य॒ज्ञियाः᳚ । सदे॑वा॒ इति॒ स - दे॒वाः॒ । आपः॑ । ताभिः॑ । ए॒व । ए॒न॒म् । प॒व॒य॒ति॒ । द्वाभ्या᳚म् । प॒व॒य॒ति॒ । अ॒हो॒रा॒त्राभ्या॒मित्य॑हः-रा॒त्राभ्या᳚म् । ए॒व । ए॒न॒म् । प॒व॒य॒ति॒ । त्रि॒भिरिति॑ त्रि - भिः । प॒व॒य॒ति॒ । त्रयः॑ । इ॒मे । लो॒काः । ए॒भिः । ए॒व । ए॒न॒म् । लो॒कैः । प॒व॒य॒ति॒ । प॒ञ्चभि॒रिति॑ प॒ञ्च - भिः॒ ।  \newline




\markright{ TS 6.1.1.8  \hfill https://www.vedavms.in \hfill}
\addcontentsline{toc}{section}{ TS 6.1.1.8 }
\section*{ TS 6.1.1.8 }

\textbf{TS 6.1.1.8 } \newline
\textbf{Samhita Paata} \newline

पवयति॒ पञ्चा᳚क्षरा प॒ङ्क्तिः पाङ्क्तो॑ य॒ज्ञो य॒ज्ञायै॒वैनं॑ पवयति ष॒ड्भिः प॑वयति॒ षड्वा ऋ॒तव॑ ऋ॒तुभि॑रे॒वैनं॑ पवयति स॒प्तभिः॑ पवयति स॒प्त छन्दाꣳ॑सि॒ छन्दो॑भिरे॒वैनं॑ पवयति न॒वभिः॑ पवयति॒ नव॒ वै पुरु॑षे प्रा॒णाः सप्रा॑णमे॒वैनं॑ पवय॒त्येक॑विꣳशत्या पवयति॒ दश॒ हस्त्या॑ अ॒ङ्गुल॑यो॒ दश॒ पद्या॑ आ॒त्मैक॑विꣳ॒॒शो यावा॑ने॒व पुरु॑ष॒स्तमप॑रिवर्गं - [  ] \newline

\textbf{Pada Paata} \newline

प॒व॒य॒ति॒ । पञ्चा᳚क्ष॒रेति॒ पञ्च॑ - अ॒क्ष॒रा॒ । प॒ङ्क्तिः । पाङ्क्तः॑ । य॒ज्ञ्ः । य॒ज्ञाय॑ । ए॒व । ए॒न॒म् । प॒व॒य॒ति॒ । ष॒ड्भिरिति॑ षट् - भिः । प॒व॒य॒ति॒ । षट् । वै । ऋ॒तवः॑ । ऋ॒तुभि॒रित्यृ॒तु - भिः॒ । ए॒व । ए॒न॒म् । प॒व॒य॒ति॒ । स॒प्तभि॒रिति॑ स॒प्त - भिः॒ । प॒व॒य॒ति॒ । स॒प्त । छन्दाꣳ॑सि । छन्दो॑भि॒रिति॒ छन्दः॑ - भिः॒ । ए॒व । ए॒न॒म् । प॒व॒य॒ति॒ । न॒वभि॒रिति॑ न॒व - भिः॒ । प॒व॒य॒ति॒ । नव॑ । वै । पुरु॑षे । प्रा॒णा इति॑ प्र - अ॒नाः । सप्रा॑ण॒मिति॒ स-प्रा॒ण॒म् । ए॒व । ए॒न॒म् । प॒व॒य॒ति॒ । एक॑विꣳश॒त्येत्येक॑-विꣳ॒॒श॒त्या॒ । प॒व॒य॒ति॒ । दश॑ । हस्त्याः᳚ । अ॒ङ्गुल॑यः । दश॑ । पद्याः᳚ । आ॒त्मा । ए॒क॒विꣳ॒॒श इत्येक॑ - विꣳ॒॒शः । यावान्॑ । ए॒व । पुरु॑षः । तम् । अप॑रिवर्ग॒मित्यप॑रि - व॒र्ग॒म् ।  \newline




\markright{ TS 6.1.1.9  \hfill https://www.vedavms.in \hfill}
\addcontentsline{toc}{section}{ TS 6.1.1.9 }
\section*{ TS 6.1.1.9 }

\textbf{TS 6.1.1.9 } \newline
\textbf{Samhita Paata} \newline

पवयति चि॒त्पति॑स्त्वा पुना॒त्वित्या॑ह॒ मनो॒ वै चि॒त्पति॒र्मन॑सै॒वैनं॑ पवयति वा॒क्पति॑स्त्वा पुना॒त्वित्या॑ह वा॒चैवैनं॑ पवयति दे॒वस्त्वा॑ सवि॒ता पु॑ना॒त्वित्या॑ह सवि॒तृप्र॑सूत ए॒वैनं॑ पवयति॒ तस्य॑ ते पवित्रपते प॒वित्रे॑ण॒ यस्मै॒ कं पु॒ने तच्छ॑केय॒मित्या॑-हा॒ऽऽ*शिष॑मे॒वैतामा शा᳚स्ते ॥ \newline

\textbf{Pada Paata} \newline

प॒व॒य॒ति॒ । चि॒त्पति॒रिति॑ चित् - पतिः॑ । त्वा॒ । पु॒ना॒तु॒ । इति॑ । आ॒ह॒ । मनः॑ । वै । चि॒त्पति॒रिति॑ चित् - पतिः॑ । मन॑सा । ए॒व । ए॒न॒म् । प॒व॒य॒ति॒ । वा॒क्पति॒रिति॑ वाक् - पतिः॑ । त्वा॒ । पु॒ना॒तु॒ । इति॑ । आ॒ह॒ । वा॒चा । ए॒व । ए॒न॒म् । प॒व॒य॒ति॒ । दे॒वः । त्वा॒ । स॒वि॒ता । पु॒ना॒तु॒ । इति॑ । आ॒ह॒ । स॒वि॒तृप्र॑सूत॒ इति॑ सवि॒तृ - प्र॒सू॒तः॒ । ए॒व । ए॒न॒म् । प॒व॒य॒ति॒ । तस्य॑ । ते॒ । प॒वि॒त्र॒प॒त॒ इति॑ पवित्र - प॒ते॒ । प॒वित्रे॑ण । यस्मै᳚ । कम् । पु॒ने । तत् । श॒के॒य॒म् । इति॑ । आ॒ह॒ । आ॒शिष॒मित्या᳚ - शिष᳚म् । ए॒व । ए॒ताम् । एति॑ । शा॒स्ते॒ ॥  \newline




\markright{ TS 6.1.2.1  \hfill https://www.vedavms.in \hfill}
\addcontentsline{toc}{section}{ TS 6.1.2.1 }
\section*{ TS 6.1.2.1 }

\textbf{TS 6.1.2.1 } \newline
\textbf{Samhita Paata} \newline

याव॑न्तो॒ वै दे॒वा य॒ज्ञायापु॑नत॒ त ए॒वाभ॑व॒न॒. य ए॒वं ॅवि॒द्वान्. य॒ज्ञाय॑ पुनी॒ते भव॑त्ये॒व ब॒हिः प॑वयि॒त्वाऽन्तः प्र पा॑दयति मनुष्यलो॒क ए॒वैनं॑ पवयि॒त्वा पू॒तं दे॑वलो॒कं प्र ण॑य॒त्यदी᳚क्षित॒ एक॒याऽऽहु॒त्येत्या॑हुः स्रु॒वेण॒ चत॑स्रो जुहोति दीक्षित॒त्वाय॑ स्रु॒चा प॑ञ्च॒मीं पञ्चा᳚क्षरा प॒ङ्क्तिः पाङ्क्तो॑ य॒ज्ञो य॒ज्ञ्मे॒वाव॑ रुन्ध॒ आकू᳚त्यै प्र॒युजे॒ऽग्नये॒ - [  ] \newline

\textbf{Pada Paata} \newline

याव॑न्तः । वै । दे॒वाः । य॒ज्ञाय॑ । अपु॑नत । ते । ए॒व । अ॒भ॒व॒न्न् । यः । ए॒वम् । वि॒द्वान् । य॒ज्ञाय॑ । पु॒नी॒ते । भव॑ति । ए॒व । ब॒हिः । प॒व॒यि॒त्वा । अ॒न्तः । प्रेति॑ । पा॒द॒य॒ति॒ । म॒नु॒ष्य॒लो॒क इति॑ मनुष्य - लो॒के । ए॒व । ए॒न॒म् । प॒व॒यि॒त्वा । पू॒तम् । दे॒व॒लो॒कमिति॑ देव - लो॒कम् । प्रेति॑ । न॒य॒ति॒ । अदी᳚क्षितः । एक॑या । आहु॒त्येत्या - हु॒त्या॒ । इति॑ । आ॒हुः॒ । स्रु॒वेण॑ । चत॑स्रः । जु॒हो॒ति॒ । दी॒क्षि॒त॒त्वायेति॑ दीक्षित-त्वाय॑ । स्रु॒चा । प॒ञ्च॒मीम् । पञ्चा᳚क्ष॒रेति॒ पञ्च॑ - अ॒क्ष॒रा॒ । प॒ङ्क्तिः । पाङ्क्तः॑ । य॒ज्ञ्ः । य॒ज्ञ्म् । ए॒व । अवेति॑ । रु॒न्धे॒ । आकू᳚त्या॒ इत्या - कू॒त्यै॒ । प्र॒युज॒ इति॑ प्र - युजे᳚ । अ॒ग्नये᳚ ।  \newline




\markright{ TS 6.1.2.2  \hfill https://www.vedavms.in \hfill}
\addcontentsline{toc}{section}{ TS 6.1.2.2 }
\section*{ TS 6.1.2.2 }

\textbf{TS 6.1.2.2 } \newline
\textbf{Samhita Paata} \newline

स्वाहेत्या॒हाऽऽ*कू᳚त्या॒ हि पुरु॑षो य॒ज्ञ्म॒भि प्र॑यु॒ङ्क्ते यजे॒येति॑ मे॒धायै॒ मन॑से॒ऽग्नये॒ स्वाहेत्या॑ह मे॒धया॒ हि मन॑सा॒ पुरु॑षो य॒ज्ञ्म॑भि॒गच्छ॑ति॒ सर॑स्वत्यै पू॒ष्णे᳚ऽग्नये॒ स्वाहेत्या॑ह॒ वाग्वै सर॑स्वती पृथि॒वी पू॒षा वा॒चैव पृ॑थि॒व्या य॒ज्ञ्ं प्रयु॑ङ्क्त॒ आपो॑ देवी-र्बृहती-र्विश्वशंभुव॒ इत्या॑ह॒ या वै वर्ष्या॒स्ता - [  ] \newline

\textbf{Pada Paata} \newline

स्वाहा᳚ । इति॑ । आ॒ह॒ । आकू॒त्येत्या - कू॒त्या॒ । हि । पुरु॑षः । य॒ज्ञ्म् । अ॒भीति॑ । प्र॒यु॒ङ्क्त इति॑ प्र-यु॒ङ्क्ते । यजे॑य । इति॑ । मे॒धायै᳚ । मन॑से । अ॒ग्नये᳚ । स्वाहा᳚ । इति॑ । आ॒ह॒ । मे॒धया᳚ । हि । मन॑सा । पुरु॑षः । य॒ज्ञ्म् । अ॒भि॒गच्छ॒तीत्य॑भि - गच्छ॑ति । सर॑स्वत्यै । पू॒ष्णे । अ॒ग्नये᳚ । स्वाहा᳚ । इति॑ । आ॒ह॒ । वाक् । वै । सर॑स्वती । पृ॒थि॒वी । पू॒षा । वा॒चा । ए॒व । पृ॒थि॒व्या । य॒ज्ञ्म् । प्रेति॑ । यु॒ङ्क्ते॒ । आपः॑ । दे॒वीः॒ । बृ॒ह॒तीः॒ । वि॒श्व॒श॒भुं॒व॒ इति॑ विश्व-श॒भुं॒वः॒ । इति॑ । आ॒ह॒ । याः । वै । वर्ष्याः᳚ । ताः ।  \newline




\markright{ TS 6.1.2.3  \hfill https://www.vedavms.in \hfill}
\addcontentsline{toc}{section}{ TS 6.1.2.3 }
\section*{ TS 6.1.2.3 }

\textbf{TS 6.1.2.3 } \newline
\textbf{Samhita Paata} \newline

आपो॑ दे॒वी-र्बृ॑ह॒ती-र्वि॒श्वश॑भुंवो॒ यदे॒तद्-यजु॒र्न ब्रू॒याद्-दि॒व्या आपोऽशा᳚न्ता इ॒मं ॅलो॒कमा ग॑च्छेयु॒रापो॑ देवी-र्बृहती-र्विश्वशंभुव॒ इत्या॑हा॒स्मा ए॒वैना॑ लो॒काय॑ शमयति॒ तस्मा᳚च्छा॒न्ता इ॒मं ॅलो॒कमा ग॑च्छन्ति॒ द्यावा॑पृथि॒वी इत्या॑ह॒ द्यावा॑पृथि॒व्योर्.हि य॒ज्ञ् उ॒र्व॑न्तरि॑क्ष॒-मित्या॑हा॒न्तरि॑क्षे॒ हि य॒ज्ञो बृह॒स्पति॑र्नो ह॒विषा॑ वृधा॒त्वि - [  ] \newline

\textbf{Pada Paata} \newline

आपः॑ । दे॒वीः । बृ॒ह॒तीः । वि॒श्वश॑भुंव॒ इति॑ वि॒श्व - श॒भुं॒वः॒ । यत् । ए॒तत् । यजुः॑ । न । ब्रू॒यात् । दि॒व्याः । आपः॑ । अशा᳚न्ताः । इ॒मम् । लो॒कम् । एति॑ । ग॒च्छे॒युः॒ । आपः॑ । दे॒वीः॒ । बृ॒ह॒तीः॒ । वि॒श्व॒श॒भुं॒व॒ इति॑ विश्व-श॒भुं॒वः॒ । इति॑ । आ॒ह॒ । अ॒स्मै । ए॒व । ए॒नाः॒ । लो॒काय॑ । श॒म॒य॒ति॒ । तस्मा᳚त् । शा॒न्ताः । इ॒मम् । लो॒कम् । एति॑ । ग॒च्छ॒न्ति॒ । द्यावा॑पृथि॒वी इति॒ द्यावा᳚ -पृ॒थि॒वी । इति॑ । आ॒ह॒ । द्यावा॑पृथि॒व्योरिति॒ द्यावा᳚-पृ॒थि॒व्योः । हि । य॒ज्ञ्ः । उ॒रु । अ॒न्तरि॑क्षम् । इति॑ । आ॒ह॒ । अ॒न्तरि॑क्षे । हि । य॒ज्ञ्ः । बृह॒स्पतिः॑ । नः॒ । ह॒विषा᳚ । वृ॒धा॒तु॒ ।  \newline




\markright{ TS 6.1.2.4  \hfill https://www.vedavms.in \hfill}
\addcontentsline{toc}{section}{ TS 6.1.2.4 }
\section*{ TS 6.1.2.4 }

\textbf{TS 6.1.2.4 } \newline
\textbf{Samhita Paata} \newline

-त्या॑ह॒ ब्रह्म॒ वै दे॒वानां॒ बृह॒स्पति॒-र्ब्रह्म॑णै॒वास्मै॑ य॒ज्ञ्मव॑ रुन्धे॒ यद्- ब्रू॒याद्-वि॑धे॒रिति॑ यज्ञ्स्था॒णुमृ॑च्छेद्-वृधा॒त्वित्या॑ह यज्ञ्स्था॒णुमे॒व परि॑ वृणक्ति प्र॒जाप॑ति-र्य॒ज्ञ्म॑सृजत॒ सो᳚ऽस्माथ् सृ॒ष्टः परा॑ङै॒थ् स प्र यजु॒रव्ली॑ना॒त् प्र साम॒ तमृगुद॑यच्छ॒द्-यदृगु॒दय॑च्छ॒त् तदौ᳚द्-ग्रह॒णस्यौ᳚द्-ग्रहण॒त्व मृ॒चा - [  ] \newline

\textbf{Pada Paata} \newline

इति॑ । आ॒ह॒ । ब्रह्म॑ । वै । दे॒वाना᳚म् । बृह॒स्पतिः॑ । ब्रह्म॑णा । ए॒व । अ॒स्मै॒ । य॒ज्ञ्म् । अवेति॑ । रु॒न्धे॒ । यत् । ब्रू॒यात् । वि॒धेः॒ । इति॑ । य॒ज्ञ्॒स्था॒णुमिति॑ यज्ञ् - स्था॒णुम् । ऋ॒च्छे॒त् । वृ॒धा॒तु॒ । इति॑ । आ॒ह॒ । य॒ज्ञ्॒स्था॒णुमिति॑ यज्ञ् - स्था॒णुम् । ए॒व । परीति॑ । वृ॒ण॒क्ति॒ । प्र॒जाप॑ति॒रिति॑ प्र॒जा - प॒तिः॒ । य॒ज्ञ्म् । अ॒सृ॒ज॒त॒ । सः । अ॒स्मा॒त् । सृ॒ष्टः । पराङ्॑ । ऐ॒त् । सः । प्रेति॑ । यजुः॑ । अव्ली॑नात् । प्रेति॑ । साम॑ । तम् । ऋक् । उदिति॑ । अ॒य॒च्छ॒त् । यत् । ऋक् । उ॒दय॑च्छ॒दित्यु॑त् - अय॑च्छत् । तत् । औ॒द्ग्र॒ह॒णस्येत्यौ᳚त्-ग्र॒ह॒णस्य॑ । औ॒द्ग्र॒ह॒ण॒त्वमित्यौ᳚द्ग्रहण - त्वम् । ऋ॒चा ।  \newline




\markright{ TS 6.1.2.5  \hfill https://www.vedavms.in \hfill}
\addcontentsline{toc}{section}{ TS 6.1.2.5 }
\section*{ TS 6.1.2.5 }

\textbf{TS 6.1.2.5 } \newline
\textbf{Samhita Paata} \newline

जु॑होति य॒ज्ञ्स्योद्य॑त्या अनु॒ष्टुप्-छन्द॑सा॒-मुद॑यच्छ॒दित्या॑-हु॒स्तस्मा॑दनु॒ष्टुभा॑ जुहोति य॒ज्ञ्स्योद्य॑त्यै॒ द्वाद॑श वाथ्सब॒न्धान्युद॑यच्छ॒-न्नित्या॑हु॒-स्तस्मा᳚द्- द्वाद॒शभि॑-र्वाथ्सबन्ध॒विदो॑ दीक्षयन्ति॒ सा वा ए॒षर्ग॑नु॒ष्टुग्-वाग॑नु॒ष्टुग्-यदे॒तय॒र्चा दी॒क्षय॑ति वा॒चैवैनꣳ॒॒ सर्व॑या दीक्षयति॒ विश्वे॑ दे॒वस्य॑ ने॒तुरित्या॑ह सावि॒त्र्ये॑तेन॒ मर्तो॑ वृणीत स॒ख्य - [  ] \newline

\textbf{Pada Paata} \newline

जु॒हो॒ति॒ । य॒ज्ञ्स्य॑ । उद्य॑त्या॒ इत्युत् - य॒त्यै॒ । अ॒नु॒ष्टुबित्य॑नु - स्तुप् । छन्द॑साम् । उदिति॑ । अ॒य॒च्छ॒त् । इति॑ । आ॒हुः॒ । तस्मा᳚त् । अ॒नु॒ष्टुभेत्य॑नु - स्तुभा᳚ । जु॒हो॒ति॒ । य॒ज्ञ्स्य॑ । उद्य॑त्या॒ इत्युत् - य॒त्यै॒ । द्वाद॑श । वा॒थ्स॒ब॒न्धानीति॑ वाथ्स - ब॒न्धानि॑ । उदिति॑ । अ॒य॒च्छ॒न्न् । इति॑ । आ॒हुः॒ । तस्मा᳚त् । द्वा॒द॒शभि॒रिति॑ द्वाद॒श-भिः॒ । आ॒थ्स॒ब॒न्ध॒विद॒ इति॑ वाथ्सबन्ध - विदः॑ । दी॒क्ष॒य॒न्ति॒ । सा । वै । ए॒षा । ऋक् । अ॒नु॒ष्टुगित्य॑नु - स्तुक् । वाक् । अ॒नु॒ष्टुगित्य॑नु - स्तुक् । यत् । ए॒तया᳚ । ऋ॒चा । दी॒क्षय॑ति । वा॒चा । ए॒व । ए॒न॒म् । सर्व॑या । दी॒क्ष॒य॒ति॒ । विश्वे᳚ । दे॒वस्य॑ । ने॒तुः । इति॑ । आ॒ह॒ । सा॒वि॒त्री । ए॒तेन॑ । मर्तः॑ । वृ॒णी॒त॒ । स॒ख्यम् ।  \newline




\markright{ TS 6.1.2.6  \hfill https://www.vedavms.in \hfill}
\addcontentsline{toc}{section}{ TS 6.1.2.6 }
\section*{ TS 6.1.2.6 }

\textbf{TS 6.1.2.6 } \newline
\textbf{Samhita Paata} \newline

-मित्या॑ह पितृदेव॒त्यै॑तेन॒ विश्वे॑ रा॒य इ॑षुद्ध्य॒सीत्या॑ह वैश्वदे॒व्ये॑तेन॑ द्यु॒म्नं ॅवृ॑णीत पु॒ष्यस॒ इत्या॑ह पौ॒ष्ण्ये॑तेन॒ सा वा ए॒षर्ख्स॑र्वदेव॒त्या॑ यदे॒तय॒र्चा दी॒क्षय॑ति॒ सर्वा॑भिरे॒वैनं॑ दे॒वता॑भिर्दीक्षयति स॒प्ताक्ष॑रं प्रथ॒मं प॒दम॒ष्टाक्ष॑राणि॒ त्रीणि॒ यानि॒ त्रीणि॒ तान्य॒ष्टावुप॑ यन्ति॒ यानि॑ च॒त्वारि॒ तान्य॒ष्टौ यद॒ष्टाक्ष॑रा॒ तेन॑ - [  ] \newline

\textbf{Pada Paata} \newline

इति॑ । आ॒ह॒ । पि॒तृ॒दे॒व॒त्येति॑ पितृ - दे॒व॒त्या᳚ । ए॒तेन॑ । विश्वे᳚ । रा॒यः । इ॒षु॒द्ध्य॒सि॒ । इति॑ । आ॒ह॒ । वै॒श्व॒दे॒वीति॑ वैश्व-दे॒वी । ए॒तेन॑ । द्यु॒म्नम् । वृ॒णी॒त॒ । पु॒ष्यसे᳚ । इति॑ । आ॒ह॒ । पौ॒ष्णी । ए॒तेन॑ । सा । वै । ए॒षा । ऋक् । स॒र्व॒दे॒व॒त्येति॑ सर्व - दे॒व॒त्या᳚ । यत् । ए॒तया᳚ । ऋ॒चा । दी॒क्षय॑ति । सर्वा॑भिः । ए॒व । ए॒न॒म् । दे॒वता॑भिः । दी॒क्ष॒य॒ति॒ । स॒प्ताक्ष॑र॒मिति॑ स॒प्त - अ॒क्ष॒र॒म् । प्र॒थ॒मम् । प॒दम् । अ॒ष्टाक्ष॑रा॒णीत्य॒ष्टा -अ॒क्ष॒रा॒णि॒ । त्रीणि॑ । यानि॑ । त्रीणि॑ । तानि॑ । अ॒ष्टौ । उपेति॑ । य॒न्ति॒ । यानि॑ । च॒त्वारि॑ । तानि॑ । अ॒ष्टौ । यत् । अ॒ष्टाक्ष॒रेत्य॒ष्टा - अ॒क्ष॒रा॒ । तेन॑ ।  \newline




\markright{ TS 6.1.2.7  \hfill https://www.vedavms.in \hfill}
\addcontentsline{toc}{section}{ TS 6.1.2.7 }
\section*{ TS 6.1.2.7 }

\textbf{TS 6.1.2.7 } \newline
\textbf{Samhita Paata} \newline

गाय॒त्री यदेका॑दशाक्षरा॒ तेन॑ त्रि॒ष्टुग्यद् द्वाद॑शाक्षरा॒ तेन॒ जग॑ती॒ सा वा ए॒षर्ख्सर्वा॑णि॒ छन्दाꣳ॑सि॒ यदे॒तय॒र्चा दी॒क्षय॑ति॒ सर्वे॑भिरे॒वैनं॒ छन्दो॑भिर्दीक्षयति स॒प्ताक्ष॑रं प्रथ॒मं प॒दꣳ स॒प्तप॑दा॒ शक्व॑री प॒शवः॒ शक्व॑री प॒शूने॒वाव॑ रुन्ध॒ एक॑स्माद॒क्षरा॒दना᳚प्तं प्रथ॒मं प॒दं तस्मा॒द्-यद्-वा॒चोऽना᳚प्तं॒ तन्म॑नु॒ष्या॑ उप॑ जीवन्ति पू॒र्णया॑ जुहोति ( ) पू॒र्ण इ॑व॒ हि प्र॒जाप॑तिः प्र॒जाप॑ते॒राप्त्यै॒ न्यू॑नया जुहोति॒ न्यू॑ना॒द्धि प्र॒जाप॑तिः प्र॒जा असृ॑जत प्र॒जानाꣳ॒॒ सृष्ट्यै᳚ ॥ \newline

\textbf{Pada Paata} \newline

गा॒य॒त्री । यत् । एका॑दशाक्ष॒रेत्येका॑दश - अ॒क्ष॒रा॒ । तेन॑ । त्रि॒ष्टुक् । यत् । द्वाद॑शाक्ष॒रेति॒ द्वाद॑श - अ॒क्ष॒रा॒ । तेन॑ । जग॑ती । सा । वै । ए॒षा । ऋक् । सर्वा॑णि । छन्दाꣳ॑सि । यत् । ए॒तया᳚ । ऋ॒चा । दी॒क्षय॑ति । सर्वे॑भिः । ए॒व । ए॒न॒म् । छन्दो॑भि॒रिति॒ छन्दः॑ - भिः॒ । दी॒क्ष॒य॒ति॒ । स॒प्ताक्ष॑र॒मिति॑ स॒प्त - अ॒क्ष॒रम् । प्र॒थ॒मम् । प॒दम् । स॒प्तप॒देति॑ स॒प्त - प॒दा॒ । शक्व॑री । प॒शवः॑ । शक्व॑री । प॒शून् । ए॒व । अवेति॑ । रु॒न्धे॒ । एक॑स्मात् । अ॒क्षरा᳚त् । अना᳚प्तम् । प्र॒थ॒मम् । प॒दम् । तस्मा᳚त् । यत् । वा॒चः । अना᳚प्तम् । तत् । म॒नु॒ष्याः᳚ । उपेति॑ । जी॒व॒न्ति॒ । पू॒र्णया᳚ । जु॒हो॒ति॒ ( ) । पू॒र्णः । इ॒व॒ । हि । प्र॒जाप॑ति॒रिति॑ प्र॒जा-प॒तिः॒ । प्र॒जाप॑ते॒रिति॑ प्र॒जा-प॒तेः॒ । आप्त्यै᳚ । न्यू॑न॒येति॒ नि-ऊ॒न॒या॒ । जु॒हो॒ति॒ । न्यू॑ना॒दिति॒ नि - ऊ॒ना॒त् । हि । प्र॒जाप॑ति॒रिति॑ प्र॒जा - प॒तिः॒ । प्र॒जा इति॑ प्र - जाः । असृ॑जत । प्र॒जाना॒मिति॑ प्र-जाना᳚म् । सृष्ट्यै᳚ ॥  \newline




\markright{ TS 6.1.3.1  \hfill https://www.vedavms.in \hfill}
\addcontentsline{toc}{section}{ TS 6.1.3.1 }
\section*{ TS 6.1.3.1 }

\textbf{TS 6.1.3.1 } \newline
\textbf{Samhita Paata} \newline

ऋ॒ख् सा॒मे वै दे॒वेभ्यो॑ य॒ज्ञायाऽति॑ष्ठमाने॒ कृष्णो॑ रू॒पं कृ॒त्वा- ऽप॒क्रम्या॑तिष्ठतां॒ ते॑ऽमन्यन्त॒ यं ॅवा इ॒मे उ॑पाव॒र्थ्स्यतः॒ स इ॒दं भ॑विष्य॒तीति॒ ते उपा॑मन्त्रयन्त॒ ते अ॑होरा॒त्रयो᳚-र्महि॒मान॑-मपनि॒धाय॑ दे॒वानु॒पाव॑र्तेतामे॒ष वा ऋ॒चो वर्णो॒ यच्छु॒क्लं कृ॑ष्णाजि॒नस्यै॒ष साम्नो॒ यत् कृ॒ष्णमृ॑ख्सा॒मयोः॒ शिल्पे᳚ स्थ॒ इत्या॑हर्ख्सा॒मे ए॒वाव॑ रुन्ध ए॒ष - [  ] \newline

\textbf{Pada Paata} \newline

ऋ॒ख्सा॒मे इत्यृ॑क्-सा॒मे । वै । दे॒वेभ्यः॑ । य॒ज्ञाय॑ । अति॑ष्ठमाने॒ इति॑ । कृष्णः॑ । रू॒पम् । कृ॒त्वा । अ॒प॒क्रम्येत्य॑प -क्रम्य॑ । अ॒ति॒ष्ठ॒ता॒म् । ते । अ॒म॒न्य॒न्त॒ । यम् । वै । इ॒मे इति॑ । उ॒पा॒व॒र्थ्स्यत॒ इत्यु॑प-आ॒व॒र्थ्स्यतः॑ । सः । इ॒दम् । भ॒वि॒ष्य॒ति॒ । इति॑ । ते इति॑ । उपेति॑ । अ॒म॒न्त्र॒य॒न्त॒ । ते इति॑ । अ॒हो॒रा॒त्रयो॒रित्य॑हः - रा॒त्रयोः᳚ । म॒हि॒मान᳚म् । अ॒प॒नि॒धायेत्य॑प - नि॒धाय॑ । दे॒वान् । उ॒पाव॑र्तेता॒मित्यु॑प-आव॑र्तेताम् । ए॒षः । वै । ऋ॒चः । वर्णः॑ । यत् । शु॒क्लम् । कृ॒ष्णा॒जि॒नस्येति॑ कृष्ण - अ॒जि॒नस्य॑ । ए॒षः । साम्नः॑ । यत् । कृ॒ष्णम् । ऋ॒ख्सा॒मयो॒रित्यृ॑क् - सा॒मयोः᳚ । शिल्पे॒ इति॑ । स्थः॒ । इति॑ । आ॒ह॒ । ऋ॒ख्सा॒मे इत्यृ॑क् - सा॒मे । ए॒व । अवेति॑ । रु॒न्धे॒ । ए॒षः ।  \newline




\markright{ TS 6.1.3.2  \hfill https://www.vedavms.in \hfill}
\addcontentsline{toc}{section}{ TS 6.1.3.2 }
\section*{ TS 6.1.3.2 }

\textbf{TS 6.1.3.2 } \newline
\textbf{Samhita Paata} \newline

वा अह्नो॒ वर्णो॒ यच्छु॒क्लं कृ॑ष्णाजि॒नस्यै॒ष रात्रि॑या॒ यत् कृ॒ष्णं ॅयदे॒वैन॑यो॒स्तत्र॒ न्य॑क्तं॒ तदे॒वाव॑ रुन्धे कृष्णाजि॒नेन॑ दीक्षयति॒ ब्रह्म॑णो॒ वा ए॒तद्-रू॒पं ॅयत् कृ॑ष्णाजि॒नं ब्रह्म॑णै॒वैनं॑ दीक्षयती॒मां धियꣳ॒॒ शिक्ष॑माणस्य दे॒वेत्या॑ह यथाय॒जुरे॒वैतद्-गर्भो॒ वा ए॒ष यद्-दी᳚क्षि॒त उल्बं॒ ॅवासः॒ प्रोर्णु॑ते॒ तस्मा॒द् - [  ] \newline

\textbf{Pada Paata} \newline

वै । अह्नः॑ । वर्णः॑ । यत् । शु॒क्लम् । कृ॒ष्णा॒जि॒नस्येति॑ कृष्ण - अ॒जि॒नस्य॑ । ए॒षः । रात्रि॑याः । यत् । कृ॒ष्णम् । यत् । ए॒व । ए॒न॒योः॒ । तत्र॑ । न्य॑क्त॒मिति॒ नि-अ॒क्त॒म् । तत् । ए॒व । अवेति॑ । रु॒न्धे॒ । कृ॒ष्णा॒जि॒नेनेति॑ कृष्ण-अ॒जि॒नेन॑ । दी॒क्ष॒य॒ति॒ । ब्रह्म॑णः । वै । ए॒तत् । रू॒पम् । यत् । कृ॒ष्णा॒जि॒नमिति॑ कृष्ण - अ॒जि॒नम् । ब्रह्म॑णा । ए॒व । ए॒न॒म् । दी॒क्ष॒य॒ति॒ । इ॒माम् । धिय᳚म् । शिक्ष॑माणस्य । दे॒व॒ । इति॑ । आ॒ह॒ । य॒था॒य॒जुरिति॑ यथा - य॒जुः । ए॒व । ए॒तत् । गर्भः॑ । वै । ए॒षः । यत् । दी॒क्षि॒तः । उल्ब᳚म् । वासः॑ । प्रेति॑ । ऊ॒र्णु॒ते॒ । तस्मा᳚त् ।  \newline




\markright{ TS 6.1.3.3  \hfill https://www.vedavms.in \hfill}
\addcontentsline{toc}{section}{ TS 6.1.3.3 }
\section*{ TS 6.1.3.3 }

\textbf{TS 6.1.3.3 } \newline
\textbf{Samhita Paata} \newline

गर्भाः॒ प्रावृ॑ता जायन्ते॒ न पु॒रा सोम॑स्य क्र॒यादपो᳚र्ण्वीत॒ यत् पु॒रा सोम॑स्य क्र॒याद॑पोर्ण्वी॒त गर्भाः᳚ प्र॒जानां᳚ परा॒पातु॑काः स्युः क्री॒ते सोमेऽपो᳚र्णुते॒ जाय॑त ए॒व तदथो॒ यथा॒ वसी॑याꣳ सं प्रत्यपोर्णु॒ते ता॒दृगे॒व तदङ्गि॑रसः सुव॒र्गं ॅलो॒कं ॅयन्त॒ ऊर्जं॒ ॅव्य॑भजन्त॒ ततो॒ यद॒त्यशि॑ष्यत॒ ते श॒रा अ॑भव॒न्नूर्ग्वै श॒रा यच्छ॑र॒मयी॒ - [  ] \newline

\textbf{Pada Paata} \newline

गर्भाः᳚ । प्रावृ॑ताः । जा॒य॒न्ते॒ । न । पु॒रा । सोम॑स्य । क्र॒यात् । अपेति॑ । ऊ॒र्ण्वी॒त॒ । यत् । पु॒रा । सोम॑स्य । क्र॒यात् । अ॒पो॒र्ण्वी॒तेत्य॑प-ऊ॒र्ण्वी॒त । गर्भाः᳚ । प्र॒जाना॒मिति॑ प्र - जाना᳚म् । प॒रा॒पातु॑का॒ इति॑ परा-पातु॑काः । स्युः॒ । क्री॒ते । सोमे᳚ । अपेति॑ । ऊ॒र्णु॒ते॒ । जाय॑ते । ए॒व । तत् । अथो॒ इति॑ । यथा᳚ । वसी॑याꣳसम् । प्र॒त्य॒पो॒र्णु॒त इति॑ प्रति -अ॒पो॒र्णु॒ते । ता॒दृक् । ए॒व । तत् । अङ्गि॑रसः । सु॒व॒र्गमिति॑ सुवः - गम् । लो॒कम् । यन्तः॑ । ऊर्ज᳚म् । वीति॑ । अ॒भ॒ज॒न्त॒ । ततः॑ । यत् । अ॒त्यशि॑ष्य॒तेत्य॑ति - अशि॑ष्यत । ते । श॒राः । अ॒भ॒व॒न्न् । ऊर्क् । वै । श॒राः । यत् । श॒र॒मयीति॑ शर - मयी᳚ ।  \newline




\markright{ TS 6.1.3.4  \hfill https://www.vedavms.in \hfill}
\addcontentsline{toc}{section}{ TS 6.1.3.4 }
\section*{ TS 6.1.3.4 }

\textbf{TS 6.1.3.4 } \newline
\textbf{Samhita Paata} \newline

मेख॑ला॒ भव॒त्यूर्ज॑मे॒वाव॑ रुन्धे मद्ध्य॒तः संन॑ह्यति मद्ध्य॒त ए॒वास्मा॒ ऊर्जं॑ दधाति॒ तस्मा᳚न्मद्ध्य॒त ऊ॒र्जा भु॑ञ्जत ऊ॒र्द्ध्वं ॅवै पुरु॑षस्य॒ नाभ्यै॒ मेद्ध्य॑-मवा॒चीन॑-ममे॒द्ध्यं ॅयन्म॑द्ध्य॒तः स॒नंह्य॑ति॒ मेद्ध्यं॑ चै॒वास्या॑मे॒द्ध्यं च॒ व्याव॑र्तय॒तीन्द्रो॑ वृ॒त्राय॒ वज्रं॒ प्राह॑र॒थ् स त्रे॒धा व्य॑भव॒थ् स्फ्यस्तृती॑यꣳ॒॒ रथ॒स्तृती॑यं॒ ॅयूप॒स्तृती॑यं॒ - [  ] \newline

\textbf{Pada Paata} \newline

मेख॑ला । भव॑ति । ऊर्ज᳚म् । ए॒व । अवेति॑ । रु॒न्धे॒ । म॒द्ध्य॒तः । समिति॑ । न॒ह्य॒ति॒ । म॒द्ध्य॒तः । ए॒व । अ॒स्मै॒ । ऊर्ज᳚म् । द॒धा॒ति॒ । तस्मा᳚त् । म॒द्ध्य॒तः । ऊ॒र्जा । भु॒ञ्ज॒ते॒ । ऊ॒द्‌र्ध्वम् । वै । पुरु॑षस्य । नाभ्यै᳚ । मेद्ध्य᳚म् । अ॒वा॒चीन᳚म् । अ॒मे॒द्ध्यम् । यत् । म॒द्ध्य॒तः । स॒न्नह्य॒तीति॑ सं - नह्य॑ति । मेद्ध्य᳚म् । च॒ । ए॒व । अ॒स्य॒ । अ॒मे॒द्ध्यम् । च॒ । व्याव॑र्तय॒तीति॑ वि-आव॑र्तयति । इन्द्रः॑ । वृ॒त्राय॑ । वज्र᳚म् । प्रेति॑ । अ॒ह॒र॒त् । सः । त्रे॒धा । वीति॑ । अ॒भ॒व॒त् । स्फ्यः । तृती॑यम् । रथः॑ । तृती॑यम् । यूपः॑ । तृती॑यम् ।  \newline




\markright{ TS 6.1.3.5  \hfill https://www.vedavms.in \hfill}
\addcontentsline{toc}{section}{ TS 6.1.3.5 }
\section*{ TS 6.1.3.5 }

\textbf{TS 6.1.3.5 } \newline
\textbf{Samhita Paata} \newline

ॅये᳚ऽन्तः श॒रा अशी᳚र्यन्त॒ ते श॒रा अ॑भव॒न् तच्छ॒राणाꣳ॑ शर॒त्वं ॅवज्रो॒ वै श॒राः क्षुत् खलु॒ वै म॑नु॒ष्य॑स्य॒ भ्रातृ॑व्यो॒ यच्छ॑र॒मयी॒ मेख॑ला॒ भव॑ति॒ वज्रे॑णै॒व सा॒क्षात् क्षुधं॒ भ्रातृ॑व्यं मद्ध्य॒तोऽप॑ हते त्रि॒वृद्-भ॑वति त्रि॒वृद्वै प्रा॒णस्त्रि॒वृत॑मे॒व प्रा॒णं म॑द्ध्य॒तो यज॑माने दधाति पृ॒थ्वी भ॑वति॒ रज्जू॑नां॒ ॅव्यावृ॑त्यै॒ मेख॑लया॒ यज॑मानं दीक्षयति॒ योक्त्रे॑ण॒ पत्नीं᳚ मिथुन॒त्वाय॑ - [  ] \newline

\textbf{Pada Paata} \newline

ये । अ॒न्त॒श्श॒रा इत्य॑न्तः-श॒राः । अशी᳚र्यन्त । ते । श॒राः । अ॒भ॒व॒न्न् । तत् । श॒राणा᳚म् । श॒र॒त्वमिति॑ शर-त्वम् । वज्रः॑ । वै । श॒राः । क्षुत् । खलु॑ । वै । म॒नु॒ष्य॑स्य । भ्रातृ॑व्यः । यत् । श॒र॒मयीति॑ शर - मयी᳚ । मेख॑ला । भव॑ति । वज्रे॑ण । ए॒व । सा॒क्षादिति॑ स-अ॒क्षात् । क्षुध᳚म् । भ्रातृ॑व्यम् । म॒द्ध्य॒तः । अपेति॑ । ह॒ते॒ । त्रि॒वृदिति॑ त्रि - वृत् । भ॒व॒ति॒ । त्रि॒वृदिति॑ त्रि - वृत् । वै । प्रा॒ण इति॑ प्र -अ॒नः । त्रि॒वृत॒मिति॑ त्रि - वृत᳚म् । ए॒व । प्रा॒णमिति॑ प्र - अ॒नम् । म॒द्ध्य॒तः । यज॑माने । द॒धा॒ति॒ । पृ॒थ्वी । भ॒व॒ति॒ । रज्जू॑नाम् । व्यावृ॑त्या॒ इति॑ वि - आवृ॑त्यै । मेख॑लया । यज॑मानम् । दी॒क्ष॒य॒ति॒ । योक्त्रे॑ण । पत्नी᳚म् । मि॒थु॒न॒त्वायेति॑ मिथुन - त्वाय॑ ।  \newline




\markright{ TS 6.1.3.6  \hfill https://www.vedavms.in \hfill}
\addcontentsline{toc}{section}{ TS 6.1.3.6 }
\section*{ TS 6.1.3.6 }

\textbf{TS 6.1.3.6 } \newline
\textbf{Samhita Paata} \newline

य॒ज्ञो दक्षि॑णाम॒भ्य॑द्ध्याय॒त् ताꣳ सम॑भव॒त् तदिन्द्रो॑-ऽचाय॒थ् सो॑ऽमन्यत॒ यो वा इ॒तो ज॑नि॒ष्यते॒ स इ॒दं भ॑विष्य॒तीति॒ तां प्रावि॑श॒त् तस्या॒ इन्द्र॑ ए॒वाजा॑यत॒ सो॑ऽमन्यत॒ यो वै मदि॒तो ऽप॑रो जनि॒ष्यते॒ स इ॒दं भ॑विष्य॒तीति॒ तस्या॑ अनु॒मृश्य॒ योनि॒मा-ऽच्छि॑न॒थ् सा सू॒तव॑शाऽभव॒त् तथ् सू॒तव॑शायै॒ जन्म॒ - [  ] \newline

\textbf{Pada Paata} \newline

य॒ज्ञ्ः । दक्षि॑णाम् । अ॒भीति॑ । अ॒द्ध्या॒य॒त् । ताम् । समिति॑ । अ॒भ॒व॒त् । तत् । इन्द्रः॑ । अ॒चा॒य॒त् । सः । अ॒म॒न्य॒त॒ । यः । वै । इ॒तः । ज॒नि॒ष्यते᳚ । सः । इ॒दम् । भ॒वि॒ष्य॒ति॒ । इति॑ । ताम् । प्रेति॑ । अ॒वि॒श॒त् । तस्याः᳚ । इन्द्रः॑ । ए॒व । अ॒जा॒य॒त॒ । सः । अ॒म॒न्य॒त॒ । यः । वै । मत् । इ॒तः । अप॑रः । ज॒नि॒ष्यते᳚ । सः । इ॒दम् । भ॒वि॒ष्य॒ति॒ । इति॑ । तस्याः᳚ । अ॒नु॒मृश्येत्य॑नु - मृश्य॑ । योनि᳚म् । एति॑ । अ॒च्छि॒न॒त् । सा । सू॒तव॒शेति॑ सू॒त - व॒शा॒ । अ॒भ॒व॒त् । तत् । सू॒तव॑शाया॒ इति॑ सू॒त - व॒शा॒यै॒ । जन्म॑ ।  \newline




\markright{ TS 6.1.3.7  \hfill https://www.vedavms.in \hfill}
\addcontentsline{toc}{section}{ TS 6.1.3.7 }
\section*{ TS 6.1.3.7 }

\textbf{TS 6.1.3.7 } \newline
\textbf{Samhita Paata} \newline

ताꣳ हस्ते॒ न्य॑वेष्टयत॒ तां मृ॒गेषु॒ न्य॑दधा॒थ् सा कृ॑ष्णविषा॒णा- ऽभ॑व॒दिन्द्र॑स्य॒ योनि॑रसि॒ मा मा॑ हिꣳसी॒रिति॑ कृष्णविषा॒णां प्र य॑च्छति॒ सयो॑निमे॒व य॒ज्ञ्ं क॑रोति॒ सयो॑निं॒ दक्षि॑णाꣳ॒॒ सयो॑नि॒मिन्द्रꣳ॑ सयोनि॒त्वाय॑ कृ॒ष्यै त्वा॑ सुस॒स्याया॒ इत्या॑ह॒ तस्मा॑दकृष्टप॒च्या ओष॑धयः पच्यन्ते सुपिप्प॒लाभ्य॒-स्त्वौष॑धीभ्य॒ इत्या॑ह॒ तस्मा॒दोष॑धयः॒ फलं॑ गृह्णन्ति॒ यद्धस्ते॑न - [  ] \newline

\textbf{Pada Paata} \newline

ताम् । हस्ते᳚ । नीति॑ । अ॒वे॒ष्ट॒य॒त॒ । ताम् । मृ॒गेषु॑ । नीति॑ । अ॒द॒धा॒त् । सा । कृ॒ष्ण॒वि॒षा॒णेति॑ कृष्ण - वि॒षा॒णा । अ॒भ॒व॒त् । इन्द्र॑स्य । योनिः॑ । अ॒सि॒ । मा । मा॒ । हिꣳ॒॒सीः॒ । इति॑ । कृ॒ष्ण॒वि॒षा॒णामिति॑ कृष्ण - वि॒षा॒णाम् । प्रेति॑ । य॒च्छ॒ति॒ । सयो॑नि॒मिति॒ स - यो॒नि॒म् । ए॒व । य॒ज्ञ्म् । क॒रो॒ति॒ । सयो॑नि॒मिति॒ स - यो॒नि॒म् । दक्षि॑णाम् । सयो॑नि॒मिति॒ स - यो॒नि॒म् । इन्द्र᳚म् । स॒यो॒नि॒त्वायेति॑ सयोनि-त्वाय॑ । कृ॒ष्यै । त्वा॒ । सु॒स॒स्याया॒ इति॑ सु - स॒स्यायै᳚ । इति॑ । आ॒ह॒ । तस्मा᳚त् । अ॒कृ॒ष्ट॒प॒च्या इत्य॑कृष्ट - प॒च्याः । ओष॑धयः । प॒च्य॒न्ते॒ । सु॒पि॒प्प॒लाभ्य॒ इति॑ सु - पि॒प्प॒लाभ्यः॑ । त्वा॒ । ओष॑धीभ्य॒ इत्योष॑धि - भ्यः॒ । इति॑ । आ॒ह॒ । तस्मा᳚त् । ओष॑धयः । फल᳚म् । गृ॒ह्ण॒न्ति॒ । यत् । हस्ते॑न ।  \newline




\markright{ TS 6.1.3.8  \hfill https://www.vedavms.in \hfill}
\addcontentsline{toc}{section}{ TS 6.1.3.8 }
\section*{ TS 6.1.3.8 }

\textbf{TS 6.1.3.8 } \newline
\textbf{Samhita Paata} \newline

कण्डू॒येत॑ पामनं॒ भावु॑काः प्र॒जाः स्यु॒र्यथ् स्मये॑त नग्नं॒ भावु॑काः कृष्णविषा॒णया॑ कण्डूयतेऽपि॒गृह्य॑ स्मयते प्र॒जानां᳚ गोपी॒थाय॒ न पु॒रा दक्षि॑णाभ्यो॒ नेतोः᳚ कृष्णविषा॒णामव॑ चृते॒द्यत् पु॒रा दक्षि॑णाभ्यो॒ नेतोः᳚ कृष्णविषा॒णाम॑व चृ॒तेद्योनिः॑ प्र॒जानां᳚ परा॒पातु॑का स्यान्नी॒तासु॒ दक्षि॑णासु॒ चात्वा॑ले कृष्णविषा॒णां प्रास्य॑ति॒ योनि॒र्वै य॒ज्ञ्स्य॒ चात्वा॑लं॒ ॅयोनिः॑ कृष्णविषा॒णा योना॑वे॒व योनिं॑ दधाति य॒ज्ञ्स्य॑ सयोनि॒त्वाय॑ ॥ \newline

\textbf{Pada Paata} \newline

क॒ण्डू॒येत॑ । पा॒म॒न॒म्भावु॑का॒ इति॑ पामनम् - भावु॑काः । प्र॒जा इति॑ प्र - जाः । स्युः॒ । यत् । स्मये॑त । न॒ग्न॒म्भावु॑का॒ इति॑ नग्नम् - भावु॑काः । कृ॒ष्ण॒वि॒षा॒णयेति॑ कृष्ण-वि॒षा॒णया᳚ । क॒ण्डू॒य॒ते॒ । अ॒पि॒गृह्येत्य॑पि - गृह्य॑ । स्म॒य॒ते॒ । प्र॒जाना॒मिति॑ प्र - जाना᳚म् । गो॒पी॒थाय॑ । न । पु॒रा । दक्षि॑णाभ्यः । नेतोः᳚ । कृ॒ष्ण॒वि॒षा॒णामिति॑ कृष्ण - वि॒षा॒णाम् । अवेति॑ । चृ॒ते॒त् । यत् । पु॒रा । दक्षि॑णाभ्यः । नेतोः᳚ । कृ॒ष्ण॒वि॒षा॒णामिति॑ कृष्ण-वि॒षा॒णाम् । अ॒व॒चृ॒तेदित्य॑व-चृ॒तेत् । योनिः॑ । प्र॒जाना॒मिति॑ प्र - जाना᳚म् । प॒रा॒पातु॒केति॑ परा - पातु॑का । स्या॒त् । नी॒तासु॑ । दक्षि॑णासु । चात्वा॑ले । कृ॒ष्ण॒वि॒षा॒णामिति॑ कृष्ण - वि॒षा॒णाम् । प्रेति॑ । अ॒स्य॒ति॒ । योनिः॑ । वै । य॒ज्ञ्स्य॑ । चात्वा॑लम् । योनिः॑ । कृ॒ष्ण॒वि॒षा॒णेति॑ कृष्ण - वि॒षा॒णा । योनौ᳚ । ए॒व । योनि᳚म् । द॒धा॒ति॒ । य॒ज्ञ्स्य॑ । स॒यो॒नि॒त्वायेति॑ सयोनि - त्वाय॑ ॥  \newline




\markright{ TS 6.1.4.1  \hfill https://www.vedavms.in \hfill}
\addcontentsline{toc}{section}{ TS 6.1.4.1 }
\section*{ TS 6.1.4.1 }

\textbf{TS 6.1.4.1 } \newline
\textbf{Samhita Paata} \newline

वाग्वै दे॒वेभ्यो ऽपा᳚क्रामद्-य॒ज्ञायाति॑ष्ठमाना॒ सा वन॒स्पती॒न् प्रावि॑श॒थ् सैषा वाग्वन॒स्पति॑षु वदति॒ या दु॑न्दु॒भौ या तूण॑वे॒ या वीणा॑यां॒ ॅयद्-दी᳚क्षितद॒ण्डं प्र॒यच्छ॑ति॒ वाच॑मे॒वाव॑ रुन्ध॒ औदु॑बंरो भव॒त्यूर्ग्वा उ॑दु॒बंर॒ ऊर्ज॑मे॒वाव॑ रुन्धे॒ मुखे॑न॒ संमि॑तो भवति मुख॒त ए॒वास्मा॒ ऊर्जं॑ दधाति॒ तस्मा᳚न् मुख॒त ऊ॒र्जा भु॑ञ्जते - [  ] \newline

\textbf{Pada Paata} \newline

वाक् । वै । दे॒वेभ्यः॑ । अपेति॑ । अ॒क्रा॒म॒त् । य॒ज्ञाय॑ । अति॑ष्ठमाना । सा । वन॒स्पतीन्॑ । प्रेति॑ । अ॒वि॒श॒त् । सा । ए॒षा । वाक् । वन॒स्पति॑षु । व॒द॒ति॒ । या । दु॒न्दु॒भौ । या । तूण॑वे । या । वीणा॑याम् । यत् । दी॒क्षि॒त॒द॒ण्डमिति॑ दीक्षित-द॒ण्डम् । प्र॒यच्छ॒तीति॑ प्र-यच्छ॑ति । वाच᳚म् । ए॒व । अवेति॑ । रु॒न्धे॒ । औदु॑म्बरः । भ॒व॒ति॒ । ऊर्क् । वै । उ॒दु॒बंरः॑ । ऊर्ज᳚म् । ए॒व । अवेति॑ । रु॒न्धे॒ । मुखे॑न । सम्मि॑त॒ इति॒ सं - मि॒तः॒ । भ॒व॒ति॒ । मु॒ख॒तः । ए॒व । अ॒स्मै॒ । ऊर्ज᳚म् । द॒धा॒ति॒ । तस्मा᳚त् । मु॒ख॒तः । ऊ॒र्जा । भु॒ञ्ज॒ते॒ ।  \newline




\markright{ TS 6.1.4.2  \hfill https://www.vedavms.in \hfill}
\addcontentsline{toc}{section}{ TS 6.1.4.2 }
\section*{ TS 6.1.4.2 }

\textbf{TS 6.1.4.2 } \newline
\textbf{Samhita Paata} \newline

क्री॒ते सोमे॑ मैत्रावरु॒णाय॑ द॒ण्डं प्र य॑च्छति मैत्रावरु॒णो हि पु॒रस्ता॑-दृ॒त्विग्भ्यो॒ वाचं॑ ॅवि॒भज॑ति॒ तामृ॒त्विजो॒ यज॑माने॒ प्रति॑ ष्ठापयन्ति॒ स्वाहा॑ य॒ज्ञ्ं मन॒सेत्या॑ह॒ मन॑सा॒ हि पुरु॑षो य॒ज्ञ्म॑भि॒गच्छ॑ति॒ स्वाहा॒ द्यावा॑पृथि॒वीभ्या॒ -मित्या॑ह॒ द्यावा॑पृथि॒व्योर्.हि य॒ज्ञ्ः स्वाहो॒रोर॒-न्तरि॑क्षा॒ -दित्या॑हा॒न्तरि॑क्षे॒ हि य॒ज्ञ्ः स्वाहा॑ य॒ज्ञ्ं ॅवाता॒दा र॑भ॒ इत्या॑हा॒यं - [  ] \newline

\textbf{Pada Paata} \newline

क्री॒ते । सोमे᳚ । मै॒त्रा॒व॒रु॒णायेति॑ मैत्रा - व॒रु॒णाय॑ । द॒ण्डम् । प्रेति॑ । य॒च्छ॒ति॒ । मै॒त्रा॒व॒रु॒ण इति॑ मैत्रा - व॒रु॒णः । हि । पु॒रस्ता᳚त् । ऋ॒त्विग्भ्य॒ इत्यृ॒त्विक् - भ्यः॒ । वाच᳚म् । वि॒भज॒तीति॑ वि - भज॑ति । ताम् । ऋ॒त्विजः॑ । यज॑माने । प्रतीति॑ । स्था॒प॒य॒न्ति॒ । स्वाहा᳚ । य॒ज्ञ्म् । मन॑सा । इति॑ । आ॒ह॒ । मन॑सा । हि । पुरु॑षः । य॒ज्ञ्म् । अ॒भि॒गच्छ॒तीत्य॑भि - गच्छ॑ति । स्वाहा᳚ । द्यावा॑पृथि॒वीभ्या॒मिति॒ द्यावा᳚ - पृ॒थि॒वीभ्या᳚म् । इति॑ । आ॒ह॒ । द्यावा॑पृथि॒व्योरिति॒ द्यावा᳚-पृ॒थि॒व्योः । हि । य॒ज्ञ्ः । स्वाहा᳚ । उ॒रोः । अ॒न्तरि॑क्षात् । इति॑ । आ॒ह॒ । अ॒न्तरि॑क्षे । हि । य॒ज्ञ्ः । स्वाहा᳚ । य॒ज्ञ्म् । वाता᳚त् । एति॑ । र॒भे॒ । इति॑ । आ॒ह॒ । अ॒यम् ।  \newline




\markright{ TS 6.1.4.3  \hfill https://www.vedavms.in \hfill}
\addcontentsline{toc}{section}{ TS 6.1.4.3 }
\section*{ TS 6.1.4.3 }

\textbf{TS 6.1.4.3 } \newline
\textbf{Samhita Paata} \newline

ॅवाव यः पव॑ते॒ स य॒ज्ञ्स्तमे॒व सा॒क्षादा र॑भते मु॒ष्टी क॑रोति॒ वाचं॑ ॅयच्छति य॒ज्ञ्स्य॒ धृत्या॒ अदी᳚क्षिष्टा॒यं ब्रा᳚ह्म॒ण इति॒ त्रिरु॑पाꣳ॒॒श्वा॑ह दे॒वेभ्य॑ ए॒वैनं॒ प्राऽऽ*ह॒ त्रिरु॒च्चैरु॒भये᳚भ्य ए॒वैनं॑ देवमनु॒ष्येभ्यः॒ प्राऽऽ*ह॒ न पु॒रा नक्ष॑त्रेभ्यो॒ वाचं॒ ॅवि सृ॑जे॒द्-यत्पु॒रा नक्ष॑त्रेभ्यो॒ वाचं॑ ॅविसृ॒जेद्- य॒ज्ञ्ं ॅविच्छि॑न्द्या॒ - [  ] \newline

\textbf{Pada Paata} \newline

वाव । यः । पव॑ते । सः । य॒ज्ञ्ः । तम् । ए॒व । सा॒क्षादिति॑ स-अ॒क्षात् । एति॑ । र॒भ॒ते॒ । मु॒ष्टी इति॑ । क॒रो॒ति॒ । वाच᳚म् । य॒च्छ॒ति॒ । य॒ज्ञ्स्य॑ । धृत्यै᳚ । अदी᳚क्षिष्ट । अ॒यम् । ब्रा॒ह्म॒णः । इति॑ । त्रिः । उ॒पाꣳ॒॒श्वित्यु॑प - अꣳ॒॒शु । आ॒ह॒ । दे॒वेभ्यः॑ । ए॒व । ए॒न॒म् । प्रेति॑ । आ॒ह॒ । त्रिः । उ॒च्चैः । उ॒भये᳚भ्यः । ए॒व । ए॒न॒म् । दे॒व॒म॒नु॒ष्येभ्य॒ इति॑ देव - म॒नु॒ष्येभ्यः॑ । प्रेति॑ । आ॒ह॒ । न । पु॒रा । नक्ष॑त्रेभ्यः । वाच᳚म् । वीति॑ । सृ॒जे॒त् । यत् । पु॒रा । नक्ष॑त्रेभ्यः । वाच᳚म् । वि॒सृ॒जेदिति॑ वि - सृ॒जेत् । य॒ज्ञ्म् । वीति॑ । छि॒न्द्या॒त् ।  \newline




\markright{ TS 6.1.4.4  \hfill https://www.vedavms.in \hfill}
\addcontentsline{toc}{section}{ TS 6.1.4.4 }
\section*{ TS 6.1.4.4 }

\textbf{TS 6.1.4.4 } \newline
\textbf{Samhita Paata} \newline

-दुदि॑तेषु॒ नक्ष॑त्रेषु व्र॒तं कृ॑णु॒तेति॒ वाचं॒ ॅवि सृ॑जति य॒ज्ञ्व्र॑तो॒ वै दी᳚क्षि॒तो य॒ज्ञ्मे॒वाभि वाचं॒ ॅवि सृ॑जति॒ यदि॑ विसृ॒जेद्-वै᳚ष्ण॒वीमृच॒मनु॑ ब्रूयाद्-य॒ज्ञो वै विष्णु॑र्य॒ज्ञेनै॒व य॒ज्ञ्ꣳ सं त॑नोति॒ दैवीं॒ धियं॑ मनामह॒ इत्या॑ह य॒ज्ञ्मे॒व तन्म्र॑दयति सुपा॒रा नो॑ अस॒द्वश॒ इत्या॑ह॒ व्यु॑ष्टिमे॒वाव॑ रुन्धे - [  ] \newline

\textbf{Pada Paata} \newline

उदि॑ते॒ष्वित्युत् - इ॒ते॒षु॒ । नक्ष॑त्रेषु । व्र॒तम् । कृ॒णु॒त॒ । इति॑ । वाच᳚म् । वीति॑ । सृ॒ज॒ति॒ । य॒ज्ञ्व्र॑त॒ इति॑ य॒ज्ञ्-व्र॒तः॒ । वै । दी॒क्षि॒तः । य॒ज्ञ्म् । ए॒व । अ॒भीति॑ । वाच᳚म् । वीति॑ । सृ॒ज॒ति॒ । यदि॑ । वि॒सृ॒जेदिति॑ वि - सृ॒जेत् । वै॒ष्ण॒वीम् । ऋच᳚म् । अन्विति॑ । ब्रू॒या॒त् । य॒ज्ञ्ः । वै । विष्णुः॑ । य॒ज्ञेन॑ । ए॒व । य॒ज्ञ्म् । समिति॑ । त॒नो॒ति॒ । दैवी᳚म् । धिय᳚म् । म॒ना॒म॒हे॒ । इति॑ । आ॒ह॒ । य॒ज्ञ्म् । ए॒व । तत् । म्र॒द॒य॒ति॒ । सु॒पा॒रेति॑ सु - पा॒रा । नः॒ । अ॒स॒त् । वशे᳚ । इति॑ । आ॒ह॒ । व्यु॑ष्टि॒मिति॒ वि - उ॒ष्टि॒म् । ए॒व । अवेति॑ । रु॒न्धे॒ ।  \newline




\markright{ TS 6.1.4.5  \hfill https://www.vedavms.in \hfill}
\addcontentsline{toc}{section}{ TS 6.1.4.5 }
\section*{ TS 6.1.4.5 }

\textbf{TS 6.1.4.5 } \newline
\textbf{Samhita Paata} \newline

ब्रह्मवा॒दिनो॑ वदन्ति होत॒व्यं॑ दीक्षि॒तस्य॑ गृ॒हा(3) इ न हो॑त॒व्या(3)मिति॑ ह॒विर्वै दी᳚क्षि॒तो यज्जु॑हु॒याद्-यज॑मानस्याव॒दाय॑ जुहुया॒द्-यन्न जु॑हु॒याद्-य॑ज्ञ्प॒रुर॒न्तरि॑या॒द्ये दे॒वा मनो॑जाता मनो॒युज॒ इत्या॑ह प्रा॒णा वै दे॒वा मनो॑जाता मनो॒युज॒स्तेष्वे॒व प॒रोक्षं॑ जुहोति॒ तन्नेव॑ हु॒तं नेवाहु॑तꣳ स्व॒पन्तं॒ ॅवै दी᳚क्षि॒तꣳ रक्षाꣳ॑सि जिघाꣳसन्त्य॒ग्निः- [  ] \newline

\textbf{Pada Paata} \newline

ब्र॒ह्म॒वा॒दिन॒ इति॑ ब्रह्म - वा॒दिनः॑ । व॒द॒न्ति॒ । हो॒त॒व्य᳚म् । दी॒क्षि॒तस्य॑ । गृ॒हा(3) इ । न । हो॒त॒व्या(3)म् । इति॑ । ह॒विः । वै । दी॒क्षि॒तः । यत् । जु॒हु॒यात् । यज॑मानस्य । अ॒व॒दायेत्य॑व - दाय॑ । जु॒हु॒या॒त् । यत् । न । जु॒हु॒यात् । य॒ज्ञ्॒प॒रुरिति॑ यज्ञ् - प॒रुः । अ॒न्तः । इ॒या॒त् । ये । दे॒वाः । मनो॑जाता॒ इति॒ मनः॑ - जा॒ताः॒ । म॒नो॒युज॒ इति॑ मनः-युजः॑ । इति॑ । आ॒ह॒ । प्रा॒णा इति॑ प्र - अ॒नाः । वै । दे॒वाः । मनो॑जाता॒ इति॒ मनः॑ - जा॒ताः॒ । म॒नो॒युज॒ इति॑ मनः - युजः॑ । तेषु॑ । ए॒व । प॒रोक्ष॒मिति॑ परः - अक्ष᳚म् । जु॒हो॒ति॒ । तत् । न । इ॒व॒ । हु॒तम् । न । इ॒व॒ । अहु॑तम् । स्व॒पन्त᳚म् । वै । दी॒क्षि॒तम् । रक्षाꣳ॑सि । जि॒घाꣳ॒॒स॒न्ति॒ । अ॒ग्निः ।  \newline




\markright{ TS 6.1.4.6  \hfill https://www.vedavms.in \hfill}
\addcontentsline{toc}{section}{ TS 6.1.4.6 }
\section*{ TS 6.1.4.6 }

\textbf{TS 6.1.4.6 } \newline
\textbf{Samhita Paata} \newline

खलु॒ वै र॑क्षो॒हाऽग्ने॒ त्वꣳ सु जा॑गृहि व॒यꣳ सु म॑न्दिषीम॒हीत्या॑हा॒ग्नि-मे॒वाधि॒पां कृ॒त्वा स्व॑पिति॒ रक्ष॑सा॒मप॑हत्या अव्र॒त्यमि॑व॒ वा ए॒ष क॑रोति॒ यो दी᳚क्षि॒तः स्वपि॑ति॒ त्वम॑ग्ने व्रत॒पा अ॒सीत्या॑हा॒ग्निर्वै दे॒वानां᳚ ॅव्र॒तप॑तिः॒ स ए॒वैनं॑ ॅव्र॒तमा ल॑भंयति दे॒व आ मर्त्ये॒ष्वेत्या॑ह दे॒वो - [  ] \newline

\textbf{Pada Paata} \newline

खलु॑ । वै । र॒क्षो॒हेति॑ रक्षः - हा । अग्ने᳚ । त्वम् । स्विति॑ । जा॒गृ॒हि॒ । व॒यम् । स्विति॑ । म॒न्दि॒षी॒म॒हि॒ । इति॑ । आ॒ह॒ । अ॒ग्निम् । ए॒व । अ॒धि॒पामित्य॑धि - पाम् । कृ॒त्वा । स्व॒पि॒ति॒ । रक्ष॑साम् । अप॑हत्या॒ इत्यप॑ - ह॒त्यै॒ । अ॒व्र॒त्यम् । इ॒व॒ । वै । ए॒षः । क॒रो॒ति॒ । यः । दी॒क्षि॒तः । स्वपि॑ति । त्वम् । अ॒ग्ने॒ । व्र॒त॒पा इति॑ व्रत-पाः । अ॒सि॒ । इति॑ । आ॒ह॒ । अ॒ग्निः । वै । दे॒वाना᳚म् । व्र॒तप॑ति॒रिति॑ व्र॒त - प॒तिः॒ । सः । ए॒व । ए॒न॒म् । व्र॒तम् । एति॑ । ल॒म्भ॒य॒ति॒ । दे॒वः । एति॑ । मर्त्ये॑षु । एति॑ । इति॑ । आ॒ह॒ । दे॒वः ।  \newline




\markright{ TS 6.1.4.7  \hfill https://www.vedavms.in \hfill}
\addcontentsline{toc}{section}{ TS 6.1.4.7 }
\section*{ TS 6.1.4.7 }

\textbf{TS 6.1.4.7 } \newline
\textbf{Samhita Paata} \newline

ह्ये॑ष सन् मर्त्ये॑षु॒ त्वं ॅय॒ज्ञेष्वीड्य॒ इत्या॑है॒तꣳ हि य॒ज्ञेष्वीड॒तेऽप॒ वै दी᳚क्षि॒ताथ् सु॑षु॒पुष॑ इन्द्रि॒यं दे॒वताः᳚ क्रामन्ति॒ विश्वे॑ दे॒वा अ॒भि मामाऽव॑वृत्र॒-न्नित्या॑-हेन्द्रि॒येणै॒वैनं॑ दे॒वता॑भिः॒ सं न॑यति॒ यदे॒तद्-यजु॒र्न ब्रू॒याद्-याव॑त ए॒व प॒शून॒भि दीक्षे॑त॒ ताव॑न्तोऽस्य प॒शवः॑ स्यू॒ रास्वेय॑थ् - [  ] \newline

\textbf{Pada Paata} \newline

हि । ए॒षः । सन्न् । मर्त्ये॑षु । त्वम् । य॒ज्ञेषु॑ । ईड्यः॑ । इति॑ । आ॒ह॒ । ए॒तम् । हि । य॒ज्ञेषु॑ । ईड॑ते । अपेति॑ । वै । दी॒क्षि॒तात् । सु॒षु॒पुषः॑ । इ॒न्द्रि॒यम् । दे॒वताः᳚ । क्रा॒म॒न्ति॒ । विश्वे᳚ । दे॒वाः । अ॒भीति॑ । माम् । एति॑ । अ॒व॒वृ॒त्र॒न्न् । इति॑ । आ॒ह॒ । इ॒न्द्रि॒येण॑ । ए॒व । ए॒न॒म् । दे॒वता॑भिः । समिति॑ । न॒य॒ति॒ । यत् । ए॒तत् । यजुः॑ । न । ब्रू॒यात् । याव॑तः । ए॒व । प॒शून् । अ॒भीति॑ । दीक्षे॑त । ताव॑न्तः । अ॒स्य॒ । प॒शवः॑ । स्युः॒ । रास्व॑ । इय॑त् ।  \newline




\markright{ TS 6.1.4.8  \hfill https://www.vedavms.in \hfill}
\addcontentsline{toc}{section}{ TS 6.1.4.8 }
\section*{ TS 6.1.4.8 }

\textbf{TS 6.1.4.8 } \newline
\textbf{Samhita Paata} \newline

सो॒माऽऽ* भूयो॑ भ॒रेत्या॒हा-प॑रिमिताने॒व प॒शूनव॑ रुन्धे च॒न्द्रम॑सि॒ मम॒ भोगा॑य भ॒वेत्या॑ह यथादेव॒तमे॒वैनाः॒ प्रति॑ गृह्णाति वा॒यवे᳚ त्वा॒ वरु॑णाय॒ त्वेति॒ यदे॒वमे॒ता नानु॑दि॒शेदय॑थादेवतं॒ दक्षि॑णा गमये॒दा दे॒वता᳚भ्यो वृश्च्येत॒ यदे॒वमे॒ता अ॑नुदि॒शति॑ यथादेव॒तमे॒व दक्षि॑णा गमयति॒ न दे॒वता᳚भ्य॒ आ - [  ] \newline

\textbf{Pada Paata} \newline

सो॒म॒ । एति॑ । भूयः॑ । भ॒र॒ । इति॑ । आ॒ह॒ । अप॑रिमिता॒नित्यप॑रि-मि॒ता॒न् । ए॒व । प॒शून् । अवेति॑ । रु॒न्धे॒ । च॒न्द्रम् । अ॒सि॒ । मम॑ । भोगा॑य । भ॒व॒ । इति॑ । आ॒ह॒ । य॒था॒दे॒व॒तमिति॑ यथा - दे॒व॒तम् । ए॒व । ए॒नाः॒ । प्रतीति॑ । गृ॒ह्णा॒ति॒ । वा॒यवे᳚ । त्वा॒ । वरु॑णाय । त्वा॒ । इति॑ । यत् । ए॒वम् । ए॒ताः । न । अ॒नु॒दि॒शेदित्य॑नु - दि॒शेत् । अय॑थादेवत॒मित्यय॑था - दे॒व॒त॒म् । दक्षि॑णाः । ग॒म॒ये॒त् । एति॑ । दे॒वता᳚भ्यः । वृ॒श्च्ये॒त॒ । यत् । ए॒वम् । ए॒ताः । अ॒नु॒दि॒शतीत्य॑नु - दि॒शति॑ । य॒था॒दे॒व॒तमिति॑ यथा - दे॒व॒तम् । ए॒व । दक्षि॑णाः । ग॒म॒य॒ति॒ । न । दे॒वता᳚भ्यः । एति॑ ।  \newline




\markright{ TS 6.1.4.9  \hfill https://www.vedavms.in \hfill}
\addcontentsline{toc}{section}{ TS 6.1.4.9 }
\section*{ TS 6.1.4.9 }

\textbf{TS 6.1.4.9 } \newline
\textbf{Samhita Paata} \newline

वृ॑श्च्यते॒ देवी॑रापो अपां नपा॒दित्या॑ह॒ यद्वो॒ मेद्ध्यं॑ ॅय॒ज्ञियꣳ॒॒ सदे॑वं॒ तद्वो॒ माऽव॑ क्रमिष॒मिति॒ वावैतदा॒हाच्छि॑न्नं॒ तन्तुं॑ पृथि॒व्या अनु॑ गेष॒मित्या॑ह॒ सेतु॑मे॒व कृ॒त्वाऽत्ये॑ति ॥ \newline

\textbf{Pada Paata} \newline

वृ॒श्च्य॒ते॒ । देवीः᳚ । आ॒पः॒ । अ॒पा॒म् । न॒पा॒त् । इति॑ । आ॒ह॒ । यत् । वः॒ । मेद्ध्य᳚म् । य॒ज्ञिय᳚म् । सदे॑व॒मिति॒ स - दे॒व॒म् । तत् । वः॒ । मा । अवेति॑ । क्र॒मि॒ष॒म् । इति॑ । वाव । ए॒तत् । आ॒ह॒ । अच्छि॑न्नम् । तन्तु᳚म् । पृ॒थि॒व्याः । अन्विति॑ । गे॒ष॒म् । इति॑ । आ॒ह॒ । सेतु᳚म् । ए॒व । कृ॒त्वा । अतीति॑ । ए॒ति॒ ॥  \newline




\markright{ TS 6.1.5.1  \hfill https://www.vedavms.in \hfill}
\addcontentsline{toc}{section}{ TS 6.1.5.1 }
\section*{ TS 6.1.5.1 }

\textbf{TS 6.1.5.1 } \newline
\textbf{Samhita Paata} \newline

दे॒वा वै दे॑व॒यज॑न-मद्ध्यव॒साय॒ दिशो॒ न प्राजा॑न॒न् ते᳚(1॒)ऽन्यो᳚-ऽन्यमुपा॑धाव॒न् त्वया॒ प्र जा॑नाम॒ त्वयेति॒ तेऽदि॑त्याꣳ॒॒ सम॒॑द्ध्रयन्त॒ त्वया॒ प्र जा॑ना॒मेति॒ साऽब्र॑वी॒द्-वरं॑ ॅवृणै॒ मत्प्रा॑यणा ए॒व वो॑ य॒ज्ञा मदु॑दयना अस॒न्निति॒ तस्मा॑दादि॒त्यः प्रा॑य॒णीयो॑ य॒ज्ञाना॑मादि॒त्य उ॑दय॒नीयः॒ पञ्च॑ दे॒वता॑ यजति॒ पञ्च॒ दिशो॑ दि॒शां प्रज्ञा᳚त्या॒ - [  ] \newline

\textbf{Pada Paata} \newline

दे॒वाः । वै । दे॒व॒यज॑न॒मिति॑ देव - यज॑नम् । अ॒द्ध्य॒व॒सायेत्य॑धि - अ॒व॒साय॑ । दिशः॑ । न । प्रेति॑ । अ॒जा॒न॒न्न् । ते । अ॒न्यः । अ॒न्यम् । उपेति॑ । अ॒धा॒व॒न्न् । त्वया᳚ । प्रेति॑ । जा॒ना॒म॒ । त्वया᳚ । इति॑ । ते । अदि॑त्याम् । समिति॑ । अ॒द्ध्र॒य॒न्त॒ । त्वया᳚ । प्रेति॑ । जा॒ना॒म॒ । इति॑ । सा । अ॒ब्र॒वी॒त् । वर᳚म् । वृ॒णै॒ । मत्प्रा॑यणा॒ इति॒ मत् - प्रा॒य॒णाः॒ । ए॒व । वः॒ । य॒ज्ञाः । मदु॑दयना॒ इति॒ मत्-उ॒द॒य॒नाः॒ । अ॒स॒न्न् । इति॑ । तस्मा᳚त् । आ॒दि॒त्यः । प्रा॒य॒णीय॒ इति॑ प्र-अ॒य॒नीयः॑ । य॒ज्ञाना᳚म् । आ॒दि॒त्यः । उ॒द॒य॒नीय॒ इत्यु॑त् - अ॒य॒नीयः॑ । पञ्च॑ । दे॒वताः᳚ । य॒ज॒ति॒ । पञ्च॑ । दिशः॑ । दि॒शाम् । प्रज्ञा᳚त्या॒ इति॒ प्र - ज्ञा॒त्यै॒ ।  \newline




\markright{ TS 6.1.5.2  \hfill https://www.vedavms.in \hfill}
\addcontentsline{toc}{section}{ TS 6.1.5.2 }
\section*{ TS 6.1.5.2 }

\textbf{TS 6.1.5.2 } \newline
\textbf{Samhita Paata} \newline

अथो॒ पञ्चा᳚क्षरा प॒ङ्क्तिः पाङ्क्तो॑ य॒ज्ञो य॒ज्ञ्मे॒वाव॑ रुन्धे॒ पथ्याꣳ॑ स्व॒स्तिम॑यज॒न् प्राची॑मे॒व तया॒ दिशं॒ प्राजा॑नन्न॒ग्निना॑ दक्षि॒णा सोमे॑न प्र॒तीचीꣳ॑ सवि॒त्रोदी॑ची॒-मदि॑त्यो॒र्द्ध्वां पथ्याꣳ॑ स्व॒स्तिं  ॅय॑जति॒ प्राची॑मे॒व तया॒ दिशं॒ प्र जा॑नाति॒ पथ्याꣳ॑ स्व॒स्तिमि॒ष्ट्वाऽग्नीषोमौ॑ यजति॒ चक्षु॑षी॒ वा ए॒ते य॒ज्ञ्स्य॒ यद॒ग्नीषोमौ॒ ताभ्या॑मे॒वानु॑ पश्य - [  ] \newline

\textbf{Pada Paata} \newline

अथो॒ इति॑ । पञ्चा᳚क्ष॒रेति॒ पञ्च॑ - अ॒क्ष॒रा॒ । प॒ङ्क्तिः । पाङ्क्तः॑ । य॒ज्ञ्ः । य॒ज्ञ्म् । ए॒व । अवेति॑ । रु॒न्धे॒ । पथ्या᳚म् । स्व॒स्तिम् । अ॒य॒ज॒न्न् । प्राची᳚म् । ए॒व । तया᳚ । दिश᳚म् । प्रेति॑ । अ॒जा॒न॒न्न् । अ॒ग्निना᳚ । द॒क्षि॒णा । सोमे॑न । प्र॒तीची᳚म् । स॒वि॒त्रा । उदी॑चीम् । अदि॑त्या । ऊ॒द्‌र्ध्वाम् । पथ्या᳚म् । स्व॒स्तिम् । य॒ज॒ति॒ । प्राची᳚म् । ए॒व । तया᳚ । दिश᳚म् । प्रेति॑ । जा॒ना॒ति॒ । पथ्या᳚म् । स्व॒स्तिम् । इ॒ष्ट्वा । अ॒ग्नीषोमा॒वित्य॒ग्नी-सोमौ᳚ । य॒ज॒ति॒ । चक्षु॑षी॒ इति॑ । वै । ए॒ते इति॑ । य॒ज्ञ्स्य॑ । यत् । अ॒ग्नीषोमा॒वित्य॒ग्नी - सोमौ᳚ । ताभ्या᳚म् । ए॒व । अन्विति॑ । प॒श्य॒ति॒ ।  \newline




\markright{ TS 6.1.5.3  \hfill https://www.vedavms.in \hfill}
\addcontentsline{toc}{section}{ TS 6.1.5.3 }
\section*{ TS 6.1.5.3 }

\textbf{TS 6.1.5.3 } \newline
\textbf{Samhita Paata} \newline

-त्य॒ग्नीषोमा॑वि॒ष्ट्वा स॑वि॒तारं॑ ॅयजति सवि॒तृप्र॑सूत ए॒वानु॑ पश्यति सवि॒तार॑मि॒ष्ट्वाऽदि॑तिं ॅयजती॒यं ॅवा अदि॑तिर॒स्यामे॒व प्र॑ति॒ष्ठायानु॑ पश्य॒त्यदि॑तिमि॒ष्ट्वा मा॑रु॒तीमृच॒मन्वा॑ह म॒रुतो॒ वै दे॒वानां॒ ॅविशो॑ देववि॒शां खलु॒ वै कल्प॑मानं मनुष्यवि॒शमनु॑ कल्पते॒ यन् मा॑रु॒तीमृच॑म॒न्वाह॑ वि॒शां क्लृप्त्यै᳚ ब्रह्मवा॒दिनो॑ वदन्ति प्रया॒जव॑दननूया॒जं प्रा॑य॒णीयं॑ का॒र्य॑-मनूया॒जव॑ - [  ] \newline

\textbf{Pada Paata} \newline

अ॒ग्नीषोमा॒वित्य॒ग्नी - सोमौ᳚ । इ॒ष्ट्वा । स॒वि॒तार᳚म् । य॒ज॒ति॒ । स॒वि॒तृप्र॑सूत॒ इति॑ सवि॒तृ - प्र॒सू॒तः॒ । ए॒व । अन्विति॑ । प॒श्य॒ति॒ । स॒वि॒तार᳚म् । इ॒ष्ट्वा । अदि॑तिम् । य॒ज॒ति॒ । इ॒यम् । वै । अदि॑तिः । अ॒स्याम् । ए॒व । प्र॒ति॒ष्ठायेति॑ प्रति - स्थाय॑ । अन्विति॑ । प॒श्य॒ति॒ । अदि॑तिम् । इ॒ष्ट्वा । मा॒रु॒तीम् । ऋच᳚म् । अन्विति॑ । आ॒ह॒ । म॒रुतः॑ । वै । दे॒वाना᳚म् । विशः॑ । दे॒व॒वि॒शमिति॑ देव - वि॒शम् । खलु॑ । वै । कल्प॑मानम् । म॒नु॒ष्य॒वि॒शमिति॑ मनुष्य-वि॒शम् । अन्विति॑ । क॒ल्प॒ते॒ । यत् । मा॒रु॒तीम् । ऋच᳚म् । अ॒न्वाहेत्य॑नु-आह॑ । वि॒शाम् । क्लृप्त्यै᳚ । ब्र॒ह्म॒वा॒दिन॒ इति॑ ब्रह्म - वा॒दिनः॑ । व॒द॒न्ति॒ । प्र॒या॒जव॒दिति॑ प्रया॒ज - व॒त् । अ॒न॒नू॒या॒जमित्य॑ननु - या॒जम् । प्रा॒य॒णीय॒मिति॑ प्र - अ॒य॒णीय᳚म् । का॒र्य᳚म् । अ॒नू॒या॒जव॒दित्य॑नूया॒ज - व॒त् ।  \newline




\markright{ TS 6.1.5.4  \hfill https://www.vedavms.in \hfill}
\addcontentsline{toc}{section}{ TS 6.1.5.4 }
\section*{ TS 6.1.5.4 }

\textbf{TS 6.1.5.4 } \newline
\textbf{Samhita Paata} \newline

दप्रया॒ज-मु॑दय॒नीय॒मिती॒मे वै प्र॑या॒जा अ॒मी अ॑नूया॒जाः सैव सा य॒ज्ञ्स्य॒ सन्त॑ति॒स्तत् तथा॒ न का॒र्य॑मा॒त्मा वै प्र॑या॒जाः प्र॒जाऽनू॑या॒जा यत् प्र॑या॒जा-न॑न्तरि॒यादा॒त्मान॑म॒-न्तरि॑या॒द्-यद॑नूया॒जा-न॑न्तरि॒यात् प्र॒जाम॒न्तरि॑या॒द्यतः॒ खलु॒ वै य॒ज्ञ्स्य॒ वित॑तस्य॒ न क्रि॒यते॒ तदनु॑ य॒ज्ञ्ः परा॑ भवति य॒ज्ञ्ं प॑रा॒भव॑न्तं॒ ॅयज॑मा॒नोऽनु॒ - [  ] \newline

\textbf{Pada Paata} \newline

अ॒प्र॒या॒जमित्य॑प्र - या॒जम् । उ॒द॒य॒नीय॒मित्यु॑त् - अ॒य॒नीय᳚म् । इति॑ । इ॒मे । वै । प्र॒या॒जा इति॑ प्र - या॒जाः । अ॒मी इति॑ । अ॒नू॒या॒जा इत्य॑नु - या॒जाः । सा । ए॒व । सा । य॒ज्ञ्स्य॑ । सन्त॑ति॒रिति॒ सं - त॒तिः॒ । तत् । तथा᳚ । न । का॒र्य᳚म् । आ॒त्मा । वै । प्र॒या॒जा इति॑ प्र - या॒जाः । प्र॒जेति॑ प्र - जा । अ॒नू॒या॒जा इत्य॑नु-या॒जाः । यत् । प्र॒या॒जानिति॑ प्र-या॒जान् । अ॒न्त॒रि॒यादित्य॑न्तः - इ॒यात् । आ॒त्मान᳚म् । अ॒न्तः । इ॒या॒त् । यत् । अ॒नू॒या॒जानित्य॑नु - या॒जान् । अ॒न्त॒रि॒यादित्य॑न्तः - इ॒यात् । प्र॒जामिति॑ प्र-जाम् । अ॒न्तः । इ॒या॒त् । यतः॑ । खलु॑ । वै । य॒ज्ञ्स्य॑ । वित॑त॒स्येति॒ वि-त॒त॒स्य॒ । न । क्रि॒यते᳚ । तत् । अन्विति॑ । य॒ज्ञ्ः । परेति॑ । भ॒व॒ति॒ । य॒ज्ञ्म् । प॒रा॒भव॑न्त॒मिति॑ परा - भव॑न्तम् । यज॑मानः । अनु॑ ।  \newline




\markright{ TS 6.1.5.5  \hfill https://www.vedavms.in \hfill}
\addcontentsline{toc}{section}{ TS 6.1.5.5 }
\section*{ TS 6.1.5.5 }

\textbf{TS 6.1.5.5 } \newline
\textbf{Samhita Paata} \newline

परा॑ भवति प्रया॒जव॑दे॒वा-नू॑या॒जव॑त् प्राय॒णीयं॑ का॒र्यं॑ प्रया॒जव॑दनूया॒जव॑-दुदय॒नीयं॒ नाऽऽ*त्मान॑मन्त॒रेति॒ न प्र॒जां न य॒ज्ञ्ः प॑रा॒भव॑ति॒ न यज॑मानः प्राय॒णीय॑स्य निष्का॒स उ॑दय॒नीय॑म॒भि निर्व॑पति॒ सैव सा य॒ज्ञ्स्य॒ सन्त॑ति॒र्याः प्रा॑य॒णीय॑स्य या॒ज्या॑ यत् ता उ॑दय॒नीय॑स्य या॒ज्याः᳚ कु॒र्यात् परा॑ङ॒मुं ॅलो॒कमा रो॑हेत् प्र॒मायु॑कः स्या॒द्याः प्रा॑य॒णीय॑स्य पुरोऽनुवा॒क्या᳚स्ता ( ) उ॑दय॒नीय॑स्य या॒ज्याः᳚ करोत्य॒स्मिन्ने॒व लो॒के प्रति॑ तिष्ठति ॥ \newline

\textbf{Pada Paata} \newline

परेति॑ । भ॒व॒ति॒ । प्र॒या॒जव॒दिति॑ प्रया॒ज - व॒त् । ए॒व । अ॒नू॒या॒जव॒दित्य॑नूया॒ज - व॒त् । प्रा॒य॒णीय॒मिति॑ प्र - अ॒य॒नीय᳚म् । का॒र्य᳚म् । प्र॒या॒जव॒दिति॑ प्रया॒ज - व॒त् । अ॒नू॒या॒जव॒दित्य॑नूया॒ज-व॒त् । उ॒द॒य॒नीय॒मित्यु॑त्-अ॒य॒नीय᳚म् । न । आ॒त्मान᳚म् । अ॒न्त॒रेतीत्य॑न्तः-एति॑ । न । प्र॒जामिति॑ प्र - जाम् । न । य॒ज्ञ्ः । प॒रा॒भव॒तीति॑ परा - भव॑ति । न । यज॑मानः । प्रा॒य॒णीय॒स्येति॑ प्र - अ॒य॒नीय॑स्य । नि॒ष्का॒से । उ॒द॒य॒नीय॒मित्यु॑त् - अ॒य॒नीय᳚म् । अ॒भि । निरिति॑ । व॒प॒ति॒ । सा । ए॒व । सा । य॒ज्ञ्स्य॑ । सन्त॑ति॒रिति॒ सं - त॒तिः॒ । याः । प्रा॒य॒णीय॒स्येति॑ प्र - अ॒य॒नीय॑स्य । या॒ज्याः᳚ । यत् । ताः । उ॒द॒य॒नीय॒स्येत्यु॑त्-अ॒य॒नीय॑स्य । या॒ज्याः᳚ । कु॒र्यात् । पराङ्॑ । अ॒मुम् । लो॒कम् । एति॑ । रो॒हे॒त् । प्र॒मायु॑क॒ इति॑ प्र - मायु॑कः । स्या॒त् । याः । प्रा॒य॒णीय॒स्येति॑ प्र - अ॒य॒नीय॑स्य । पु॒रो॒नु॒वा॒क्या॑ इति॑ पुरः - अ॒नु॒वा॒क्याः᳚ । ताः ( ) । उ॒द॒य॒नीय॒स्येत्यु॑त् - अ॒य॒नीय॑स्य । या॒ज्याः᳚ । क॒रो॒ति॒ । अ॒स्मिन्न् । ए॒व । लो॒के । प्रतीति॑ । ति॒ष्ठ॒ति॒ ॥  \newline




\markright{ TS 6.1.6.1  \hfill https://www.vedavms.in \hfill}
\addcontentsline{toc}{section}{ TS 6.1.6.1 }
\section*{ TS 6.1.6.1 }

\textbf{TS 6.1.6.1 } \newline
\textbf{Samhita Paata} \newline

क॒द्रूश्च॒ वै सु॑प॒र्णी चा᳚ऽऽ*त्मरू॒पयो॑रस्पर्द्धेताꣳ॒॒ सा क॒द्रूः सु॑प॒र्णीम॑जय॒थ् साऽब्र॑वीत् तृ॒तीय॑स्यामि॒तो दि॒वि सोम॒स्तमा ह॑र॒ तेना॒ऽऽ*त्मानं॒ निष्क्री॑णी॒ष्वेती॒यं ॅवै क॒द्रूर॒सौ सु॑प॒र्णी छन्दाꣳ॑सि सौपर्णे॒याः साब्र॑वीद॒स्मै वै पि॒तरौ॑ पु॒त्रान् बि॑भृत-स्तृ॒तीय॑स्यामि॒तो दि॒वि सोम॒स्तमा ह॑र॒ तेना॒ऽऽ*त्मानं॒ निष्क्री॑णी॒ष्वे - [  ] \newline

\textbf{Pada Paata} \newline

क॒द्रूः । च॒ । वै । सु॒प॒र्णीति॑ सु - प॒र्णी । च॒ । आ॒त्म॒रू॒पयो॒रित्या᳚त्म - रू॒पयोः᳚ । अ॒स्प॒द्‌र्धे॒ता॒म् । सा । क॒द्रूः । सु॒प॒र्णीमिति॑ सु-प॒र्णीम् । अ॒ज॒य॒त् । सा । अ॒ब्र॒वी॒त् । तृ॒तीय॑स्याम् । इ॒तः । दि॒वि । सोमः॑ । तम् । एति॑ । ह॒र॒ । तेन॑ । आ॒त्मान᳚म् । निरिति॑ । क्री॒णी॒ष्व॒ । इति॑ । इ॒यम् । वै । क॒द्रूः । अ॒सौ । सु॒प॒र्णीति॑ सु-प॒र्णी । छन्दाꣳ॑सि । सौ॒प॒र्णे॒याः । सा । अ॒ब्र॒वी॒त् । अ॒स्मै । वै । पि॒तरौ᳚ । पु॒त्रान् । बि॒भृ॒तः॒ । तृ॒तीय॑स्याम् । इ॒तः । दि॒वि । सोमः॑ । तम् । एति॑ । ह॒र॒ । तेन॑ । आ॒त्मान᳚म् । निरिति॑ । क्री॒णी॒ष्व॒ ।  \newline




\markright{ TS 6.1.6.2  \hfill https://www.vedavms.in \hfill}
\addcontentsline{toc}{section}{ TS 6.1.6.2 }
\section*{ TS 6.1.6.2 }

\textbf{TS 6.1.6.2 } \newline
\textbf{Samhita Paata} \newline

ति॑ मा क॒द्रूर॑वोच॒दिति॒ जग॒त्युद॑पत॒-च्चतु॑र्दशाक्षरा स॒ती सा ऽप्रा᳚प्य॒ न्य॑वर्तत॒ तस्यै॒ द्वे अ॒क्षरे॑ अमीयेताꣳ॒॒ सा प॒शुभि॑श्च दी॒क्षया॒ चाऽग॑च्छ॒त् तस्मा॒ज्जग॑ती॒ छन्द॑सां पश॒व्य॑तमा॒ तस्मा᳚त् पशु॒मन्तं॑ दी॒क्षोप॑ नमति त्रि॒ष्टुगुद॑पत॒त् त्रयो॑दशाक्षरा स॒ती सा ऽप्रा᳚प्य॒ न्य॑वर्तत॒ तस्यै॒ द्वे अ॒क्षरे॑ अमीयेताꣳ॒॒ सा दक्षि॑णाभिश्च॒ - [  ] \newline

\textbf{Pada Paata} \newline

इति॑ । मा॒ । क॒द्रूः । अ॒वो॒च॒त् । इति॑ । जग॑ती । उदिति॑ । अ॒प॒त॒त् । चतु॑र्दशाक्ष॒रेति॒ चतु॑र्दश -   अ॒क्ष॒रा॒ । स॒ती । सा । अप्रा॒प्येत्यप्र॑ - आ॒प्य॒ । नीति॑ । अ॒व॒र्त॒त॒ । तस्यै᳚ । द्वे इति॑ । अ॒क्षरे॒ इति॑ । अ॒मी॒ये॒ता॒म् । सा । प॒शुभि॒रिति॑ प॒शु - भिः॒ । च॒ । दी॒क्षया᳚ । च॒ । एति॑ । अ॒ग॒च्छ॒त् । तस्मा᳚त् । जग॑ती । छन्द॑साम् । प॒श॒व्य॑त॒मेति॑ पश॒व्य॑ - त॒मा॒ । तस्मा᳚त् । प॒शु॒मन्त॒मिति॑ पशु - मन्त᳚म् । दी॒क्षा । उपेति॑ । न॒म॒ति॒ । त्रि॒ष्टुक् । उदिति॑ । अ॒प॒त॒त् । त्रयो॑दशाक्ष॒रेति॒ त्रयो॑दश - अ॒क्ष॒रा॒ । स॒ती । सा । अप्रा॒प्येत्यप्र॑ - आ॒प्य॒ । नीति॑ । अ॒व॒र्त॒त॒ । तस्यै᳚ । द्वे इति॑ । अ॒क्षरे॒ इति॑ । अ॒मी॒ये॒ता॒म् । सा । दक्षि॑णाभिः । च॒ ।  \newline




\markright{ TS 6.1.6.3  \hfill https://www.vedavms.in \hfill}
\addcontentsline{toc}{section}{ TS 6.1.6.3 }
\section*{ TS 6.1.6.3 }

\textbf{TS 6.1.6.3 } \newline
\textbf{Samhita Paata} \newline

तप॑सा॒ चाऽग॑च्छ॒त् तस्मा᳚त् त्रि॒ष्टुभो॑ लो॒के माद्ध्य॑न्दिने॒ सव॑ने॒ दक्षि॑णा नीयन्त ए॒तत् खलु॒ वाव तप॒ इत्या॑हु॒र्यः स्वं ददा॒तीति॑ गाय॒त्र्युद॑पत॒च्चतु॑रक्षरा स॒त्य॑जया॒ ज्योति॑षा॒ तम॑स्या अ॒जाऽभ्य॑रुन्ध॒ तद॒जाया॑ अज॒त्वꣳ सा सोमं॒ चाऽऽह॑रच्च॒त्वारि॑ चा॒क्षरा॑णि॒ साऽष्टाक्ष॑रा॒ सम॑पद्यत ब्रह्मवा॒दिनो॑ वदन्ति॒ - [  ] \newline

\textbf{Pada Paata} \newline

तप॑सा । च॒ । एति॑ । अ॒ग॒च्छ॒त् । तस्मा᳚त् । त्रि॒ष्टुभः॑ । लो॒के । माद्ध्य॑न्दिने । सव॑ने । दक्षि॑णाः । नी॒य॒न्ते॒ । ए॒तत् । खलु॑ । वाव । तपः॑ । इति॑ । आ॒हुः॒ । यः । स्वम् । ददा॑ति । इति॑ । गा॒य॒त्री । उदिति॑ । अ॒प॒त॒त् । चतु॑रक्ष॒रेति॒ चतुः॑ - अ॒क्ष॒रा॒ । स॒ती । अ॒जया᳚ । ज्योति॑षा । तम् । अ॒स्यै॒ । अ॒जा । अ॒भीति॑ । अ॒रु॒न्ध॒ । तत् । अ॒जायाः᳚ । अ॒ज॒त्वमित्य॑ज-त्वम् । सा । सोम᳚म् । च॒ । एति॑ । अह॑रत् । च॒त्वारि॑ । च॒ । अ॒क्षरा॑णि । सा । अ॒ष्टाक्ष॒रेत्य॒ष्टा - अ॒क्ष॒रा॒ । समिति॑ । अ॒प॒द्य॒त॒ । ब्र॒ह्म॒वा॒दिन॒ इति॑ ब्रह्म - वा॒दिनः॑ । व॒द॒न्ति॒ ।  \newline




\markright{ TS 6.1.6.4  \hfill https://www.vedavms.in \hfill}
\addcontentsline{toc}{section}{ TS 6.1.6.4 }
\section*{ TS 6.1.6.4 }

\textbf{TS 6.1.6.4 } \newline
\textbf{Samhita Paata} \newline

कस्मा᳚थ् स॒त्याद्-गा॑य॒त्री कनि॑ष्ठा॒ छन्द॑साꣳ स॒ती य॑ज्ञ्मु॒खं परी॑या॒येति॒ यदे॒वादः सोम॒माऽह॑र॒त् तस्मा᳚द्-यज्ञ्मु॒खं पर्यै॒त् तस्मा᳚त् तेज॒स्विनी॑तमा प॒द्भ्यां द्वे सव॑ने स॒मगृ॑ह्णा॒न् मुखे॒नैकं॒ ॅयन्मुखे॑न स॒मगृ॑ह्णा॒त् तद॑धय॒त् तस्मा॒द् द्वे सव॑ने शु॒क्र॑वती प्रातस्सव॒नं च॒ माद्ध्य॑न्दिनं च॒ तस्मा᳚त् तृतीय सव॒न ऋ॑जी॒षम॒भि षु॑ण्वन्ति धी॒तमि॑व॒ हि मन्य॑न्त - [  ] \newline

\textbf{Pada Paata} \newline

कस्मा᳚त् । स॒त्यात् । गा॒य॒त्री । कनि॑ष्ठा । छन्द॑साम् । स॒ती । य॒ज्ञ्॒मु॒खमिति॑ यज्ञ् - मु॒खम् । परीति॑ । इ॒या॒य॒ । इति॑ । यत् । ए॒व । अ॒दः । सोम᳚म् । एति॑ । अह॑रत् । तस्मा᳚त् । य॒ज्ञ्॒मु॒खमिति॑ यज्ञ् - मु॒खम् । परीति॑ । ऐ॒त् । तस्मा᳚त् । ते॒ज॒स्विनी॑त॒मेति॑ तेज॒स्विनी᳚-त॒मा॒ । प॒द्भ्यामिति॑ पत्-भ्याम् । द्वे इति॑ । सव॑ने॒ इति॑ । स॒मगृ॑ह्णा॒दिति॑ सं - अगृ॑ह्णात् । मुखे॑न । एक᳚म् । यत् । मुखे॑न । स॒मगृ॑ह्णा॒दिति॑ सं - अगृ॑ह्णात् । तत् । अ॒ध॒य॒त् । तस्मा᳚त् । द्वे इति॑ । सव॑ने॒ इति॑ । शु॒क्रव॑ती॒ इति॑ शु॒क्र - व॒ती॒ । प्रा॒त॒स्स॒व॒नमिति॑ प्रातः - स॒व॒नम् । च॒ । माद्ध्य॑न्दिनम् । च॒ । तस्मा᳚त् । तृ॒ती॒य॒स॒व॒न इति॑ तृतीय - स॒व॒ने । ऋ॒जी॒षम् । अ॒भीति॑ । सु॒न्व॒न्ति॒ । धी॒तम् । इ॒व॒ । हि । मन्य॑न्ते ।  \newline




\markright{ TS 6.1.6.5  \hfill https://www.vedavms.in \hfill}
\addcontentsline{toc}{section}{ TS 6.1.6.5 }
\section*{ TS 6.1.6.5 }

\textbf{TS 6.1.6.5 } \newline
\textbf{Samhita Paata} \newline

आ॒शिर॒मव॑ नयति सशुक्र॒त्वायाथो॒ सं भ॑रत्ये॒वैन॒त् तꣳ सोम॑-माह्रि॒यमा॑णं गन्ध॒र्वो वि॒श्वाव॑सः॒ पर्य॑मुष्णा॒थ् स ति॒स्रो रात्रीः॒ परि॑मुषितोऽवस॒त् तस्मा᳚त् ति॒स्रो रात्रीः᳚ क्री॒तः सोमो॑ वसति॒ ते दे॒वा अ॑ब्रुव॒न्थ् स्त्रीका॑मा॒ वै ग॑न्ध॒र्वा स्स्त्रि॒या निष्क्री॑णा॒मेति॒ ते वाचꣳ॒॒ स्त्रिय॒मेक॑हायनीं कृ॒त्वा तया॒ निर॑क्रीण॒न्थ् सा रो॒हिद्-रू॒पं कृ॒त्वा ग॑न्ध॒र्वेभ्यो॑- [  ] \newline

\textbf{Pada Paata} \newline

आ॒शिर᳚म् । अवेति॑ । न॒य॒ति॒ । स॒शु॒क्र॒त्वायेति॑ सशुक्र - त्वाय॑ । अथो॒ इति॑ । समिति॑ । भ॒र॒ति॒ । ए॒व । ए॒न॒त् । तम् । सोम᳚म् । आ॒ह्रि॒यमा॑ण॒मित्या᳚ - ह्रि॒यमा॑णम् । ग॒न्ध॒र्वः । वि॒श्वाव॑सु॒रिति॑ वि॒श्व - व॒सुः॒ । परीति॑ । अ॒मु॒ष्णा॒त् । सः । ति॒स्रः । रात्रीः᳚ । परि॑मुषित॒ इति॒ परि॑ - मु॒षि॒तः॒ । अ॒व॒स॒त् । तस्मा᳚त् । ति॒स्रः । रात्रीः᳚ । क्री॒तः । सोमः॑ । व॒स॒ति॒ । ते । दे॒वाः । अ॒ब्रु॒व॒न्न् । स्त्रीका॑मा॒ इति॒ स्त्री - का॒माः॒ । वै । ग॒न्ध॒र्वाः । स्त्रि॒या । निरिति॑ । क्री॒णा॒म॒ । इति॑ । ते । वाच᳚म् । स्त्रिय᳚म् । एक॑हायनी॒मित्येक॑ - हा॒य॒नी॒म् । कृ॒त्वा । तया᳚ । निरिति॑ । अ॒क्री॒ण॒न्न् । सा । रो॒हित् । रू॒पम् । कृ॒त्वा । ग॒न्ध॒र्वेभ्यः॑ ।  \newline




\markright{ TS 6.1.6.6  \hfill https://www.vedavms.in \hfill}
\addcontentsline{toc}{section}{ TS 6.1.6.6 }
\section*{ TS 6.1.6.6 }

\textbf{TS 6.1.6.6 } \newline
\textbf{Samhita Paata} \newline

ऽप॒क्रम्या॑तिष्ठ॒त् तद्-रो॒हितो॒ जन्म॒ ते दे॒वा अ॑ब्रुव॒न्नप॑ यु॒ष्मदक्र॑मी॒-न्नास्मानु॒-पाव॑र्तते॒ वि ह्व॑यामहा॒ इति॒ ब्रह्म॑ गन्ध॒र्वा अव॑द॒न्नगा॑यन् दे॒वाः सा दे॒वान् गाय॑त उ॒पाव॑र्तत॒ तस्मा॒द्-गाय॑न्तꣳ॒॒ स्त्रियः॑ कामयन्ते॒ कामु॑का एनꣳ॒॒ स्त्रियो॑ भवन्ति॒ य ए॒वं ॅवेदाथो॒ य ए॒वं ॅवि॒द्वानपि॒ जन्ये॑षु॒ भव॑ति॒ तेभ्य॑ ए॒व द॑दत्यु॒त यद्-ब॒हुत॑या॒ - [  ] \newline

\textbf{Pada Paata} \newline

अ॒प॒क्रम्येत्य॑प - क्रम्य॑ । अ॒ति॒ष्ठ॒त् । तत् । रो॒हितः॑ । जन्म॑ । ते । दे॒वाः । अ॒ब्रु॒व॒न्न् । अपेति॑ । यु॒ष्मत् । अक्र॑मीत् । न । अ॒स्मान् । उ॒पाव॑र्तत॒ इत्यु॑प - आव॑र्तते । वीति॑ । ह्व॒या॒म॒है॒ । इति॑ । ब्रह्म॑ । ग॒न्ध॒र्वाः । अव॑दन्न् । अगा॑यन्न् । दे॒वाः । सा । दे॒वान् । गाय॑तः । उ॒पाव॑र्त॒तेत्यु॑प-आव॑र्तत । तस्मा᳚त् । गाय॑न्तम् । स्त्रियः॑ । का॒म॒य॒न्ते॒ । कामु॑काः । ए॒न॒म् । स्त्रियः॑ । भ॒व॒न्ति॒ । यः । ए॒वम् । वेद॑ । अथो॒ इति॑ । यः । ए॒वम् । वि॒द्वान् । अपीति॑ । जन्ये॑षु । भव॑ति । तेभ्यः॑ । ए॒व । द॒द॒ति॒ । उ॒त । यत् । ब॒हुत॑या॒ इति॑ ब॒हु - त॒याः॒ ।  \newline




\markright{ TS 6.1.6.7  \hfill https://www.vedavms.in \hfill}
\addcontentsline{toc}{section}{ TS 6.1.6.7 }
\section*{ TS 6.1.6.7 }

\textbf{TS 6.1.6.7 } \newline
\textbf{Samhita Paata} \newline

भव॒न्त्येक॑हायन्या क्रीणाति वा॒चैवैनꣳ॒॒ सर्व॑या क्रीणाति॒ तस्मा॒देक॑हायना मनु॒ष्या॑ वाचं॑ ॅवद॒न्त्यकू॑ट॒या ऽक॑र्ण॒याऽ का॑ण॒याश्लो॑ण॒या ऽस॑प्तशफया क्रीणाति॒ सर्व॑यै॒वैनं॑ क्रीणाति॒ यच्छ्वे॒तया᳚ क्रीणी॒याद्-दु॒श्चर्मा॒ यज॑मानः स्या॒द्यत् कृ॒ष्णया॑-ऽनु॒स्तर॑णी स्यात् प्र॒मायु॑को॒ यज॑मानः स्या॒द्यद् द्वि॑रू॒पया॒ वात्र॑घ्नी स्या॒थ् स वा॒ऽन्यं जि॑नी॒यात् तं ॅवा॒ऽन्यो जि॑नीयादरु॒णया॑ पिङ्गा॒क्ष्या ( ) क्री॑णात्ये॒तद्वै सोम॑स्य रू॒पꣳ स्वयै॒वैनं॑ दे॒वत॑या क्रीणाति ॥ \newline

\textbf{Pada Paata} \newline

भव॑न्ति । एक॑हाय॒न्येत्येक॑ - हा॒य॒न्या॒ । क्री॒णा॒ति॒ । वा॒चा । ए॒व । ए॒न॒म् । सर्व॑या । क्री॒णा॒ति॒ । तस्मा᳚त् । एक॑हायना॒ इत्येक॑-हा॒य॒नाः॒ । म॒नु॒ष्याः᳚ । वाच᳚म् । व॒द॒न्ति॒ । अकू॑टया । अक॑र्णया । अका॑णया । अश्लो॑णया । अस॑प्तशफ॒येत्यस॑प्त - श॒फ॒या॒ । क्री॒णा॒ति॒ । सर्व॑या । ए॒व । ए॒न॒म् । क्री॒णा॒ति॒ । यत् । श्वे॒तया᳚ । क्री॒णी॒यात् । दु॒श्चर्मेति॑ दुः - चर्मा᳚ । यज॑मानः । स्या॒त् । यत् । कृ॒ष्णया᳚ । अ॒नु॒स्तर॒णीत्य॑नु - स्तर॑णी । स्या॒त् । प्र॒मायु॑क॒ इति॑ प्र - मायु॑कः । यज॑मानः । स्या॒त् । यत् । द्वि॒रू॒पयेति॑ द्वि - रू॒पया᳚ । वात्र॒घ्नीति॒ वात्र॑ - घ्नी॒ । स्या॒त् । सः । वा॒ । अ॒न्यम् । जि॒नी॒यात् । तम् । वा॒ । अ॒न्यः । जि॒नी॒या॒त् । अ॒रु॒णया᳚ । पि॒ङ्गा॒क्ष्येति॑ पिङ्ग - अ॒क्ष्या ( ) । क्री॒णा॒ति॒ । ए॒तत् । वै । सोम॑स्य । रू॒पम् । स्वया᳚ । ए॒व । ए॒न॒म् । दे॒वत॑या । क्री॒णा॒ति॒ ॥  \newline




\markright{ TS 6.1.7.1  \hfill https://www.vedavms.in \hfill}
\addcontentsline{toc}{section}{ TS 6.1.7.1 }
\section*{ TS 6.1.7.1 }

\textbf{TS 6.1.7.1 } \newline
\textbf{Samhita Paata} \newline

तद्धिर॑ण्यमभव॒त् तस्मा॑द॒द्भ्यो हिर॑ण्यं पुनन्ति ब्रह्मवा॒दिनो॑ वदन्ति॒ कस्मा᳚थ् स॒त्याद॑न॒स्थिके॑न प्र॒जाः प्र॒वीय॑न्ते ऽस्थ॒न्वती᳚र्जायन्त॒ इति॒ यद्धिर॑ण्यं घृ॒ते॑ऽव॒धाय॑ जु॒होति॒ तस्मा॑दन॒स्थिके॑न प्र॒जाः प्र वी॑यन्ते ऽस्थ॒न्वती᳚र्जायन्त ए॒तद्वा अ॒ग्नेः प्रि॒यं धाम॒ यद्-घृ॒तं तेजो॒ हिर॑ण्यमि॒यन्ते॑ शुक्र त॒नूरि॒दं ॅवर्च॒ इत्या॑ह॒ सते॑जसमे॒वैनꣳ॒॒ सत॑नुं - [  ] \newline

\textbf{Pada Paata} \newline

तत् । हिर॑ण्यम् । अ॒भ॒व॒त् । तस्मा᳚त् । अ॒द्भ्य इत्य॑त् - भ्यः । हिर॑ण्यम् । पु॒न॒न्ति॒ । ब्र॒ह्म॒वा॒दिन॒ इति॑ ब्रह्म - वा॒दिनः॑ । व॒द॒न्ति॒ । कस्मा᳚त् । स॒त्यात् । अ॒न॒स्थिके॑न । प्र॒जा इति॑ प्र - जाः । प्र॒वीय॑न्त॒ इति॑ प्र - वीय॑न्ते । अ॒स्थ॒न्वती॒रित्य॑स्थन्न् - वतीः᳚ । जा॒य॒न्ते॒ । इति॑ । यत् । हिर॑ण्यम् । घृ॒ते । अ॒व॒धायेत्य॑व - धाय॑ । जु॒होति॑ । तस्मा᳚त् । अ॒न॒स्थिके॑न । प्र॒जा इति॑ प्र - जाः । प्रेति॑ । वी॒य॒न्ते॒ । अ॒स्थ॒न्वती॒रित्य॑स्थन्न् - वतीः᳚ । जा॒य॒न्ते॒ । ए॒तत् । वै । अ॒ग्नेः । प्रि॒यम् । धाम॑ । यत् । घृ॒तम् । तेजः॑ । हिर॑ण्यम् । इ॒यम् । ते॒ । शु॒क्र॒ । त॒नूः । इ॒दम् । वर्चः॑ । इति॑ । आ॒ह॒ । सते॑जस॒मिति॒ स - ते॒ज॒स॒म् । ए॒व । ए॒न॒म् । सत॑नु॒मिति॒ स - त॒नु॒म् ।  \newline




\markright{ TS 6.1.7.2  \hfill https://www.vedavms.in \hfill}
\addcontentsline{toc}{section}{ TS 6.1.7.2 }
\section*{ TS 6.1.7.2 }

\textbf{TS 6.1.7.2 } \newline
\textbf{Samhita Paata} \newline

करो॒त्यथो॒ सं भ॑रत्ये॒वैनं॒ ॅयदब॑द्धम-वद॒द्ध्याद्-गर्भाः᳚ प्र॒जानां᳚ परा॒पातु॑काः स्युर्ब॒द्धमव॑ दधाति॒ गर्भा॑णां॒ धृत्यै॑ निष्ट॒र्क्यं॑ बद्ध्नाति प्र॒जानां᳚ प्र॒जन॑नाय॒ वाग्वा ए॒षा यथ् सो॑म॒क्रय॑णी॒ जूर॒सीत्या॑ह॒ यद्धि मन॑सा॒ जव॑ते॒ तद्-वा॒चा वद॑ति धृ॒ता मन॒सेत्या॑ह॒ मन॑सा॒ हि वाग्धृ॒ता जुष्टा॒ विष्ण॑व॒ इत्या॑ह - [  ] \newline

\textbf{Pada Paata} \newline

क॒रो॒ति॒ । अथो॒ इति॑ । समिति॑ । भ॒र॒ति॒ । ए॒व । ए॒न॒म् । यत् । अब॑द्धम् । अ॒व॒द॒द्ध्यादित्य॑व -द॒द्ध्यात् । गर्भाः᳚ । प्र॒जाना॒मिति॑ प्र - जाना᳚म् । प॒रा॒पातु॑का॒ इति॑ परा-पातु॑काः । स्युः॒ । ब॒द्धम् । अवेति॑ । द॒धा॒ति॒ । गर्भा॑णाम् । धृत्यै᳚ । नि॒ष्ट॒र्क्य᳚म् । ब॒द्ध्ना॒ति॒ । प्र॒जाना॒मिति॑ प्र - जाना᳚म् । प्र॒जन॑ना॒येति॑ प्र-जन॑नाय । वाक् । वै । ए॒षा । यत् । सो॒म॒क्रय॒णीति॑ सोम - क्रय॑णी । जूः । अ॒सि॒ । इति॑ । आ॒ह॒ । यत् । हि । मन॑सा । जव॑ते । तत् । वा॒चा । वद॑ति । धृ॒ता । मन॑सा । इति॑ । आ॒ह॒ । मन॑सा । हि । वाक् । धृ॒ता । जुष्टा᳚ । विष्ण॑वे । इति॑ । आ॒ह॒ ।  \newline




\markright{ TS 6.1.7.3  \hfill https://www.vedavms.in \hfill}
\addcontentsline{toc}{section}{ TS 6.1.7.3 }
\section*{ TS 6.1.7.3 }

\textbf{TS 6.1.7.3 } \newline
\textbf{Samhita Paata} \newline

य॒ज्ञो वै विष्णु॑ र्य॒ज्ञायै॒वैनां॒ जुष्टां᳚ करोति॒ तस्या᳚स्ते स॒त्यस॑वसः प्रस॒व इत्या॑ह सवि॒तृ-प्र॑सूतामे॒व वाच॒मव॑ रुन्धे॒ काण्डे॑काण्डे॒ वै क्रि॒यमा॑णे य॒ज्ञ्ꣳ रक्षाꣳ॑सि जिघाꣳसन्त्ये॒ष खलु॒ वा अर॑क्षोहतः॒ पन्था॒ यो᳚ऽग्नेश्च॒ सूर्य॑स्य च॒ सूर्य॑स्य॒ चक्षु॒रा-ऽरु॑हम॒ग्नेर॒क्ष्णः क॒नीनि॑का॒मित्या॑ह॒ य ए॒वार॑क्षोहतः॒ पन्था॒स्तꣳ स॒मारो॑हति॒ - [  ] \newline

\textbf{Pada Paata} \newline

य॒ज्ञ्ः । वै । विष्णुः॑ । य॒ज्ञाय॑ । ए॒व । ए॒ना॒म् । जुष्टा᳚म् । क॒रो॒ति॒ । तस्याः᳚ । ते॒ । स॒त्यस॑वस॒ इति॑ स॒त्य - स॒व॒सः॒ । प्र॒स॒व इति॑ प्र - स॒वे । इति॑ । आ॒ह॒ । स॒वि॒तृप्र॑सूता॒मिति॑ सवि॒तृ - प्र॒सू॒ता॒म् । ए॒व । वाच᳚म् । अवेति॑ । रु॒न्धे॒ । काण्डे॑काण्ड॒ इति॒ काण्डे᳚-का॒ण्डे॒ । वै । क्रि॒यमा॑णे । य॒ज्ञ्म् । रक्षाꣳ॑सि । जि॒घाꣳ॒॒स॒न्ति॒ । ए॒षः । खलु॑ । वै । अर॑क्षोहत॒ इत्यर॑क्षः - ह॒तः॒ । पन्थाः᳚ । यः । अ॒ग्नेः । च॒ । सूर्य॑स्य । च॒ । सूर्य॑स्य । चक्षुः॑ । एति॑ । अ॒रु॒ह॒म् । अ॒ग्नेः । अ॒क्ष्णः । क॒नीनि॑काम् । इति॑ । आ॒ह॒ । यः । ए॒व । अर॑क्षोहत॒ इत्यर॑क्षः-ह॒तः॒ । पन्थाः᳚ । तम् । स॒मारो॑ह॒तीति॑ सं - आरो॑हति ।  \newline




\markright{ TS 6.1.7.4  \hfill https://www.vedavms.in \hfill}
\addcontentsline{toc}{section}{ TS 6.1.7.4 }
\section*{ TS 6.1.7.4 }

\textbf{TS 6.1.7.4 } \newline
\textbf{Samhita Paata} \newline

वाग्वा ए॒षा यथ् सो॑म॒क्रय॑णी॒ चिद॑सि म॒नाऽसीत्या॑ह॒ शास्त्ये॒वैना॑मे॒तत् तस्मा᳚च्छि॒ष्टाः प्र॒जा जा॑यन्ते॒ चिद॒सीत्या॑ह॒ यद्धि मन॑सा चे॒तय॑ते॒ तद्-वा॒चा वद॑ति म॒नाऽसीत्या॑ह॒ यद्धि मन॑साऽभि॒गच्छ॑ति॒ तत् क॒रोति॒ धीर॒सीत्या॑ह॒ यद्धि मन॑सा॒ ध्याय॑ति॒ तद्-वा॒चा - [  ] \newline

\textbf{Pada Paata} \newline

वाक् । वै । ए॒षा । यत् । सो॒म॒क्रय॒णीति॑ सोम - क्रय॑णी । चित् । अ॒सि॒ । म॒ना । अ॒सि॒ । इति॑ । आ॒ह॒ । शास्ति॑ । ए॒व । ए॒ना॒म् । ए॒तत् । तस्मा᳚त् । शि॒ष्टाः । प्र॒जा इति॑ प्र - जाः । जा॒य॒न्ते॒ । चित् । अ॒सि॒ । इति॑ । आ॒ह॒ । यत् । हि । मन॑सा । चे॒तय॑ते । तत् । वा॒चा । वद॑ति । म॒ना । अ॒सि॒ । इति॑ । आ॒ह॒ । यत् । हि । मन॑सा । अ॒भि॒गच्छ॒तीत्य॑भि - गच्छ॑ति । तत् । क॒रोति॑ । धीः । अ॒सि॒ । इति॑ । आ॒ह॒ । यत् । हि । मन॑सा । ध्याय॑ति । तत् । वा॒चा ।  \newline




\markright{ TS 6.1.7.5  \hfill https://www.vedavms.in \hfill}
\addcontentsline{toc}{section}{ TS 6.1.7.5 }
\section*{ TS 6.1.7.5 }

\textbf{TS 6.1.7.5 } \newline
\textbf{Samhita Paata} \newline

वद॑ति॒ दक्षि॑णा॒ऽसीत्या॑ह॒ दक्षि॑णा॒ ह्ये॑षा य॒ज्ञिया॒ऽसीत्या॑ह य॒ज्ञिया॑मे॒वैनां᳚ करोति क्ष॒त्रिया॒सीत्या॑ह क्ष॒त्रिया॒ ह्ये॑षा ऽदि॑तिरस्युभ॒यत॑श्शी॒र्ष्णीत्या॑ह॒ यदे॒वाऽऽ*दि॒त्यः प्रा॑य॒णीयो॑ य॒ज्ञाना॑मादि॒त्य उ॑दय॒नीय॒-स्तस्मा॑दे॒वमा॑ह॒ यदब॑द्धा॒ स्यादय॑ता स्या॒द्यत् प॑दिब॒द्धा-ऽनु॒स्तर॑णी स्यात् प्र॒मायु॑को॒ यज॑मानः स्या॒ - [  ] \newline

\textbf{Pada Paata} \newline

वद॑ति । दक्षि॑णा । अ॒सि॒ । इति॑ । आ॒ह॒ । दक्षि॑णा । हि । ए॒षा । य॒ज्ञिया᳚ । अ॒सि॒ । इति॑ । आ॒ह॒ । य॒ज्ञिया᳚म् । ए॒व । ए॒ना॒म् । क॒रो॒ति॒ । क्ष॒त्रिया᳚ । अ॒सि॒ । इति॑ । आ॒ह॒ । क्ष॒त्रिया᳚ । हि । ए॒षा । अदि॑तिः । अ॒सि॒ । उ॒भ॒यत॑श्शी॒र्ष्णीत्यु॑भ॒यतः॑ - शी॒र्ष्णी॒ । इति॑ । आ॒ह॒ । यत् । ए॒व । आ॒दि॒त्यः । प्रा॒य॒णीय॒ इति॑ प्र - अ॒य॒णीयः॑ । य॒ज्ञाना᳚म् । आ॒दि॒त्यः । उ॒द॒य॒नीय॒ इत्यु॑त् - अ॒य॒नीयः॑ । तस्मा᳚त् । ए॒वम् । आ॒ह॒ । यत् । अब॑द्धा । स्यात् । अय॑ता । स्या॒त् । यत् । प॒दि॒ब॒द्धेति॑ पदि - ब॒द्धा । अ॒नु॒स्तर॒णीत्य॑नु - स्तर॑णी । स्या॒त् । प्र॒मायु॑क॒ इति॑ प्र - मायु॑कः । यज॑मानः । स्या॒त् ।  \newline




\markright{ TS 6.1.7.6  \hfill https://www.vedavms.in \hfill}
\addcontentsline{toc}{section}{ TS 6.1.7.6 }
\section*{ TS 6.1.7.6 }

\textbf{TS 6.1.7.6 } \newline
\textbf{Samhita Paata} \newline

-द्यत् क॑र्णगृही॒ता वार्त्र॑घ्नी स्या॒थ् स वा॒ऽन्यं जि॑नी॒यात् तं ॅवा॒ऽन्यो जि॑नीयान्मि॒त्रस्त्वा॑ प॒दि ब॑द्ध्ना॒त्वित्या॑ह मि॒त्रो वै शि॒वो दे॒वानां॒ तेनै॒वैनां᳚ प॒दि ब॑द्ध्नाति पू॒षाऽद्ध्व॑नः पा॒त्वित्या॑हे॒यं ॅवै पू॒षेमामे॒वास्या॑ अधि॒पाम॑कः॒ सम॑ष्ट्या॒ इन्द्रा॒या-द्ध्य॑क्षा॒येत्या॒हेन्द्र॑मे॒वास्या॒ अद्ध्य॑क्षं करो॒ - [  ] \newline

\textbf{Pada Paata} \newline

यत् । क॒र्ण॒गृ॒ही॒तेति॑ कर्ण-गृ॒ही॒ता । वार्त्र॒घ्नीति॒ वार्त्र॑ - घ्नी॒ । स्या॒त् । सः । वा॒ । अ॒न्यम् । जि॒नी॒यात् । तम् । वा॒ । अ॒न्यः । जि॒नी॒या॒त् । मि॒त्रः । त्वा॒ । प॒दि । ब॒द्ध्ना॒तु॒ । इति॑ । आ॒ह॒ । मि॒त्रः । वै । शि॒वः । दे॒वाना᳚म् । तेन॑ । ए॒व । ए॒ना॒म् । प॒दि । ब॒द्ध्ना॒ति॒ । पू॒षा । अद्ध्व॑नः । पा॒तु॒ । इति॑ । आ॒ह॒ । इ॒यम् । वै । पू॒षा । इ॒माम् । ए॒व । अ॒स्याः॒ । अ॒धि॒पामित्य॑धि - पाम् । अ॒कः॒ । सम॑ष्ट्या॒ इति॒ सं - अ॒ष्ट्यै॒ । इन्द्रा॑य । अद्ध्य॑क्षा॒येत्यधि॑ - अ॒क्षा॒य॒ । इति॑ । आ॒ह॒ । इन्द्र᳚म् । ए॒व । अ॒स्याः॒ । अद्ध्य॑क्ष॒मित्यधि॑ - अ॒क्ष॒म् । क॒रो॒ति॒ ।  \newline




\markright{ TS 6.1.7.7  \hfill https://www.vedavms.in \hfill}
\addcontentsline{toc}{section}{ TS 6.1.7.7 }
\section*{ TS 6.1.7.7 }

\textbf{TS 6.1.7.7 } \newline
\textbf{Samhita Paata} \newline

-त्यनु॑ त्वा मा॒ता म॑न्यता॒मनु॑ पि॒तेत्या॒हा-नु॑मतयै॒वैन॑या क्रीणाति॒ सा दे॑वि दे॒वमच्छे॒हीत्या॑ह दे॒वी ह्ये॑षा दे॒वः सोम॒ इन्द्रा॑य॒ सोम॒मित्या॒हेन्द्रा॑य॒ हि सोम॑ आह्रि॒यते॒ यदे॒तद्-यजु॒र्न ब्रू॒यात् परा᳚च्ये॒व सो॑म॒क्रय॑णीयाद्-रु॒द्रस्त्वाऽऽ व॑र्तय॒त्वित्या॑ह रु॒द्रो वै क्रू॒रो - [  ] \newline

\textbf{Pada Paata} \newline

अन्विति॑ । त्वा॒ । मा॒ता । म॒न्य॒ता॒म् । अन्विति॑ । पि॒ता । इति॑ । आ॒ह॒ । अनु॑मत॒येत्यनु॑ - म॒त॒या॒ । ए॒व । ए॒न॒या॒ । क्री॒णा॒ति॒ । सा । दे॒वि॒ । दे॒वम् । अच्छ॑ । इ॒हि॒ । इति॑ । आ॒ह॒ । दे॒वी । हि । ए॒षा । दे॒वः । सोमः॑ । इन्द्रा॑य । सोम᳚म् । इति॑ । आ॒ह॒ । इन्द्रा॑य । हि । सोमः॑ । आ॒ह्रि॒यत॒ इत्या᳚ - ह्रि॒यते᳚ । यत् । ए॒तत् । यजुः॑ । न । ब्रू॒यात् । परा॑ची । ए॒व । सो॒म॒क्रय॒णीति॑ सोम - क्रय॑णी । इ॒या॒त् । रु॒द्रः । त्वा॒ । एति॑ । व॒र्त॒य॒तु॒ । इति॑ । आ॒ह॒ । रु॒द्रः । वै । क्रू॒रः ।  \newline




\markright{ TS 6.1.7.8  \hfill https://www.vedavms.in \hfill}
\addcontentsline{toc}{section}{ TS 6.1.7.8 }
\section*{ TS 6.1.7.8 }

\textbf{TS 6.1.7.8 } \newline
\textbf{Samhita Paata} \newline

दे॒वानां॒ तमे॒वास्यै॑ प॒रस्ता᳚द्-दधा॒त्यावृ॑त्त्यै क्रू॒रमि॑व॒ वा ए॒तत् क॑रोति॒ यद्-रु॒द्रस्य॑ की॒र्तय॑ति मि॒त्रस्य॑ प॒थेत्या॑ह॒ शान्त्यै॑ वा॒चा वा ए॒ष वि क्री॑णीते॒ यः सो॑म॒क्रय॑ण्या स्व॒स्ति सोम॑सखा॒ पुन॒रेहि॑ स॒ह र॒य्येत्या॑ह वा॒चैव वि॒क्रीय॒ पुन॑रा॒त्मन् वाचं॑ ध॒त्तेऽनु॑पदासुकाऽस्य॒ वाग्भ॑वति॒ य ए॒वं ॅवेद॑ ( ) ॥ \newline

\textbf{Pada Paata} \newline

दे॒वाना᳚म् । तम् । ए॒व । अ॒स्यै॒ । प॒रस्ता᳚त् । द॒धा॒ति॒ । आवृ॑त्त्या॒ इत्या - वृ॒त्त्यै॒ । क्रू॒रम् । इ॒व॒ । वै । ए॒तत् । क॒रो॒ति॒ । यत् । रु॒द्रस्य॑ । की॒र्तय॑ति । मि॒त्रस्य॑ । प॒था । इति॑ । आ॒ह॒ । शान्त्यै᳚ । वा॒चा । वै । ए॒षः । वीति॑ । क्री॒णी॒ते॒ । यः । सो॒म॒क्रय॒ण्येति॑ सोम - क्रय॑ण्या । स्व॒स्ति । सोम॑स॒खेति॒ सोम॑ - स॒खा॒ । पुनः॑ । एति॑ । इ॒हि॒ । स॒ह । र॒य्या । इति॑ । आ॒ह॒ । वा॒चा । ए॒व । वि॒क्रीयेति॑ वि - क्रीय॑ । पुनः॑ । आ॒त्मन् । वाच᳚म् । ध॒त्ते॒ । अनु॑पदासु॒केत्यनु॑प - दा॒सु॒का॒ । अ॒स्य॒ । वाक् । भ॒व॒ति॒ । यः । ए॒वम् । वेद॑ ( ) ॥  \newline




\markright{ TS 6.1.8.1  \hfill https://www.vedavms.in \hfill}
\addcontentsline{toc}{section}{ TS 6.1.8.1 }
\section*{ TS 6.1.8.1 }

\textbf{TS 6.1.8.1 } \newline
\textbf{Samhita Paata} \newline

षट् प॒दान्यनु॒ नि क्रा॑मति षड॒हं ॅवाङ्नाति॑ वदत्यु॒त सं॑ॅवथ्स॒रस्याय॑ने॒ याव॑त्ये॒व वाक्तामव॑ रुन्धे सप्त॒मे प॒दे जु॑होति स॒प्तप॑दा॒ शक्व॑री प॒शवः॒ शक्व॑री प॒शूने॒वाव॑ रुन्धे स॒प्त ग्रा॒म्याः प॒शवः॑ स॒प्ता*ऽऽर॒ण्याः स॒प्त छन्दाꣳ॑-स्यु॒भय॒स्या-व॑रुद्ध्यै॒ वस्व्य॑सि रु॒द्राऽसीत्या॑ह रू॒पमे॒वास्या॑ ए॒तन् म॑हि॒मानं॒ - [  ] \newline

\textbf{Pada Paata} \newline

षट् । प॒दानि॑ । अनु॑ । नीति॑ । क्रा॒म॒ति॒ । ष॒ड॒हमिति॑ षट् - अ॒हम् । वाक् । न । अतीति॑ । व॒द॒ति॒ । उ॒त । सं॒ॅव॒थ्स॒रस्येति॑ सं-व॒थ्स॒रस्य॑ । अय॑ने । याव॑ती । ए॒व । वाक् । ताम् । अवेति॑ । रु॒न्धे॒ । स॒प्त॒मे । प॒दे । जु॒हो॒ति॒ । स॒प्तप॒देति॑ स॒प्त - प॒दा॒ । शक्व॑री । प॒शवः॑ । शक्व॑री । प॒शून् । ए॒व । अवेति॑ । रु॒न्धे॒ । स॒प्त । ग्रा॒म्याः । प॒शवः॑ । स॒प्त । आ॒र॒ण्याः । स॒प्त । छन्दाꣳ॑सि । उ॒भय॑स्य । अव॑रुद्ध्या॒ इत्यव॑ - रु॒द्ध्यै॒ । वस्वी᳚ । अ॒सि॒ । रु॒द्रा । अ॒सि॒ । इति॑ । आ॒ह॒ । रू॒पम् । ए॒व । अ॒स्याः॒ । ए॒तत् । म॒हि॒मान᳚म् ।  \newline




\markright{ TS 6.1.8.2  \hfill https://www.vedavms.in \hfill}
\addcontentsline{toc}{section}{ TS 6.1.8.2 }
\section*{ TS 6.1.8.2 }

\textbf{TS 6.1.8.2 } \newline
\textbf{Samhita Paata} \newline

ॅव्याच॑ष्टे॒ बृह॒स्पति॑स्त्वा सु॒म्ने र॑ण्व॒त्वित्या॑ह॒ ब्रह्म॒ वै दे॒वानां॒ बृह॒स्पति॒-र्ब्रह्म॑णै॒वास्मै॑ प॒शूनव॑ रुन्धे रु॒द्रो वसु॑भि॒रा चि॑के॒त्वित्या॒हाऽऽ*वृ॑त्त्यै पृथि॒व्यास्त्वा॑ मू॒र्द्धन्ना जि॑घर्मि देव॒यज॑न॒ इत्या॑ह पृथि॒व्या ह्ये॑ष मू॒र्द्धा यद्-दे॑व॒यज॑न॒मिडा॑याः प॒द इत्या॒हेडा॑यै॒ ह्ये॑तत् प॒दं ॅयथ् सो॑म॒क्रय॑ण्यै घृ॒तव॑ति॒ स्वाहे - [  ] \newline

\textbf{Pada Paata} \newline

व्याच॑ष्ट॒ इति॑ वि-आच॑ष्टे । बृह॒स्पतिः॑ । त्वा॒ । सु॒म्ने । र॒ण्व॒तु॒ । इति॑ । आ॒ह॒ । ब्रह्म॑ । वै । दे॒वाना᳚म् । बृह॒स्पतिः॑ । ब्रह्म॑णा । ए॒व । अ॒स्मै॒ । प॒शून् । अवेति॑ । रु॒न्धे॒ । रु॒द्रः । वसु॑भि॒रिति॒ वसु॑ - भिः॒ । एति॑ । चि॒के॒तु॒ । इति॑ । आ॒ह॒ । आवृ॑त्त्या॒ इत्या - वृ॒त्त्यै॒ । पृ॒थि॒व्याः । त्वा॒ । मू॒द्‌र्धन्न् । एति॑ । जि॒घ॒र्मि॒ । दे॒व॒यज॑न॒ इति॑ देव-यज॑ने । इति॑ । आ॒ह॒ । पृ॒थि॒व्याः । हि । ए॒षः । मू॒द्‌र्धा । यत् । दे॒व॒यज॑न॒मिति॑ देव - यज॑नम् । इडा॑याः । प॒दे । इति॑ । आ॒ह॒ । इडा॑यै । हि । ए॒तत् । प॒दम् । यत् । सो॒म॒क्रय॑ण्या॒ इति॑ सोम - क्रय॑ण्यै । घृ॒तव॒तीति॑ घृ॒त-व॒ति॒ । स्वाहा᳚ ।  \newline




\markright{ TS 6.1.8.3  \hfill https://www.vedavms.in \hfill}
\addcontentsline{toc}{section}{ TS 6.1.8.3 }
\section*{ TS 6.1.8.3 }

\textbf{TS 6.1.8.3 } \newline
\textbf{Samhita Paata} \newline

-त्या॑ह॒ यदे॒वास्यै॑ प॒दाद्-घृ॒तमपी᳚ड्यत॒ तस्मा॑दे॒वमा॑ह॒ यद॑द्ध्व॒र्युर॑न॒ग्नावाहु॑तिं जुहु॒याद॒न्धो᳚ऽद्ध्व॒र्युः स्या॒द्-रक्षाꣳ॑सि य॒ज्ञ्ꣳ ह॑न्यु॒र्॒.हिर॑ण्यमु॒पास्य॑ जुहोत्यग्नि॒वत्ये॒व जु॑होति॒ नान्धो᳚ऽद्ध्व॒र्यु र्भव॑ति॒ न य॒ज्ञ्ꣳ रक्षाꣳ॑सि घ्नन्ति॒ काण्डे॑काण्डे॒ वै क्रि॒यमा॑णे य॒ज्ञ्ꣳ रक्षाꣳ॑सि जिघाꣳसन्ति॒ परि॑लिखितꣳ॒॒ रक्षः॒ परि॑लिखिता॒ अरा॑तय॒ इत्या॑ह॒ रक्ष॑सा॒मप॑हत्या - [  ] \newline

\textbf{Pada Paata} \newline

इति॑ । आ॒ह॒ । यत् । ए॒व । अ॒स्यै॒ । प॒दात् । घृ॒तम् । अपी᳚ड्यत । तस्मा᳚त् । ए॒वम् । आ॒ह॒ । यत् । अ॒द्ध्व॒र्युः । अ॒न॒ग्नौ । आहु॑ति॒मित्या - हु॒ति॒म् । जु॒हु॒यात् । अ॒न्धः । अ॒द्ध्व॒र्युः । स्या॒त् । रक्षाꣳ॑सि । य॒ज्ञ्म् । ह॒न्युः॒ । हिर॑ण्यम् । उ॒पास्येत्यु॑प - अस्य॑ । जु॒हो॒ति॒ । अ॒ग्नि॒वतीत्य॑ग्नि - वति॑ । ए॒व । जु॒हो॒ति॒ । न । अ॒न्धः । अ॒द्ध्व॒र्युः । भव॑ति । न । य॒ज्ञ्म् । रक्षाꣳ॑सि । घ्न॒न्ति॒ । काण्डे॑काण्ड॒ इति॒ काण्डे᳚ - का॒ण्डे॒ । वै । क्रि॒यमा॑णे । य॒ज्ञ्म् । रक्षाꣳ॑सि । जि॒घाꣳ॒॒स॒न्ति॒ । परि॑लिखित॒मिति॒ परि॑ - लि॒खि॒त॒म् । रक्षः॑ । परि॑लिखिता॒ इति॒ परि॑ - लि॒खि॒ताः॒ । अरा॑तयः । इति॑ । आ॒ह॒ । रक्ष॑साम् । अप॑हत्या॒ इत्यप॑ - ह॒त्यै॒ ।  \newline




\markright{ TS 6.1.8.4  \hfill https://www.vedavms.in \hfill}
\addcontentsline{toc}{section}{ TS 6.1.8.4 }
\section*{ TS 6.1.8.4 }

\textbf{TS 6.1.8.4 } \newline
\textbf{Samhita Paata} \newline

इ॒दम॒हꣳ रक्ष॑सो ग्री॒वा अपि॑ कृन्तामि॒ यो᳚ऽस्मान् द्वेष्टि॒ यं च॑ व॒यं द्वि॒ष्म इत्या॑ह॒ द्वौ वाव पुरु॑षौ॒ यं चै॒व द्वेष्टि॒ यश्चैनं॒ द्वेष्टि॒ तयो॑रे॒वान॑न्तरायं ग्री॒वाः कृ॑न्तति प॒शवो॒ वै सो॑म॒क्रय॑ण्यै प॒दं ॅया॑वत्त्मू॒तꣳ सं ॅव॑पति प॒शूने॒वाव॑ रुन्धे॒ऽस्मे राय॒ इति॒ सं ॅव॑पत्या॒त्मान॑-मे॒वाद्ध्व॒र्युः - [  ] \newline

\textbf{Pada Paata} \newline

इ॒दम् । अ॒हम् । रक्ष॑सः । ग्री॒वाः । अपीति॑ । कृ॒न्ता॒मि॒ । यः । अ॒स्मान् । द्वेष्टि॑ । यम् । च॒ । व॒यम् । द्वि॒ष्मः । इति॑ । आ॒ह॒ । द्वौ । वाव । पुरु॑षौ । यम् । च॒ । ए॒व । द्वेष्टि॑ । यः । च॒ । ए॒न॒म् । द्वेष्टि॑ । तयोः᳚ । ए॒व । अन॑न्तराय॒मित्यन॑न्तः - आ॒य॒म् । ग्री॒वाः । कृ॒न्त॒ति॒ । प॒शवः॑ । वै । सो॒म॒क्रय॑ण्या॒ इति॑ सोम - क्रय॑ण्यै । प॒दम् । या॒वत्त्मू॒तमिति॑ यावत् - त्मू॒तम् । समिति॑ । व॒प॒ति॒ । प॒शून् । ए॒व । अवेति॑ । रु॒न्धे॒ । अ॒स्मे इति॑ । रायः॑ । इति॑ । समिति॑ । व॒प॒ति॒ । आ॒त्मान᳚म् । ए॒व । अ॒द्ध्व॒र्युः ।  \newline




\markright{ TS 6.1.8.5  \hfill https://www.vedavms.in \hfill}
\addcontentsline{toc}{section}{ TS 6.1.8.5 }
\section*{ TS 6.1.8.5 }

\textbf{TS 6.1.8.5 } \newline
\textbf{Samhita Paata} \newline

प॒शुभ्यो॒ नान्तरे॑ति॒ त्वे राय॒ इति॒ यज॑मानाय॒ प्र य॑च्छति॒ यज॑मान ए॒व र॒यिं द॑धाति॒ तोते॒ राय॒ इति॒ पत्नि॑या अ॒र्द्धो वा ए॒ष आ॒त्मनो॒ यत् पत्नी॒ यथा॑ गृ॒हेषु॑ निध॒त्ते ता॒दृगे॒व तत् त्वष्टी॑मती ते सपे॒येत्या॑ह॒ त्वष्टा॒ वै प॑शू॒नां मि॑थु॒नानाꣳ॑ रूप॒कृद्-रू॒पमे॒व प॒शुषु॑ दधात्य॒स्मै वै लो॒काय॒ गार्.ह॑पत्य॒ आ धी॑यते॒ ( ) ऽमुष्मा॑ आहव॒नीयो॒ यद्-गार्.ह॑पत्य उप॒वपे॑द॒स्मिन् ॅलो॒के प॑श॒मान्थ् स्या॒द्-यदा॑हव॒नीये॒ ऽमुष्मि॑न् ॅलो॒के प॑शु॒मान्थ् स्या॑दु॒भयो॒रुप॑ वपत्यु॒भयो॑रे॒वैनं॑ ॅलो॒कयोः᳚ पशु॒मन्तं॑ करोति ॥ \newline

\textbf{Pada Paata} \newline

प॒शुभ्य॒ इति॑ प॒शु - भ्यः॒ । न । अ॒न्तः । ए॒ति॒ । त्वे इति॑ । रायः॑ । इति॑ । यज॑मानाय । प्रेति॑ । य॒च्छ॒ति॒ । यज॑माने । ए॒व । र॒यिम् । द॒धा॒ति॒ । तोते᳚ । रायः॑ । इति॑ । पत्नि॑यै । अ॒द्‌र्धः । वै । ए॒षः । आ॒त्मनः॑ । यत् । पत्नी᳚ । यथा᳚ । गृ॒हेषु॑ । नि॒ध॒त्त इति॑ नि - ध॒त्ते । ता॒दृक् । ए॒व । तत् । त्वष्टी॑मती । ते॒ । स॒पे॒य॒ । इति॑ । आ॒ह॒ । त्वष्टा᳚ । वै । प॒शू॒नाम् । मि॒थु॒नाना᳚म् । रू॒प॒कृदिति॑ रूप - कृत् । रू॒पम् । ए॒व । प॒शुषु॑ । द॒धा॒ति॒ । अ॒स्मै । वै । लो॒काय॑ । गार्.ह॑पत्य॒ इति॒ गार्.ह॑ - प॒त्यः॒ । एति॑ । धी॒य॒ते॒ ( ) । अ॒मुष्मै᳚ । आ॒ह॒व॒नीय॒ इत्या᳚ - ह॒व॒नीयः॑ । यत् । गार्.ह॑पत्य॒ इति॒ गार्.ह॑- प॒त्ये॒ । उ॒प॒वपे॒दित्यु॑प - वपे᳚त् । अ॒स्मिन्न् । लो॒के । प॒शु॒मानिति॑ पशु - मान् । स्या॒त् । यत् । आ॒ह॒व॒नीय॒ इत्या᳚ - ह॒व॒नीये᳚ । अ॒मुष्मिन्न्॑ । लो॒के । प॒शु॒मानिति॑ पशु - मान् । स्या॒त् । उ॒भयोः᳚ । उपेति॑ । व॒प॒ति॒ । उ॒भयोः᳚ । ए॒व । ए॒न॒म् । लो॒कयोः᳚ । प॒शु॒मन्त॒मिति॑ पशु - मन्त᳚म् । क॒रो॒ति॒ ॥  \newline




\markright{ TS 6.1.9.1  \hfill https://www.vedavms.in \hfill}
\addcontentsline{toc}{section}{ TS 6.1.9.1 }
\section*{ TS 6.1.9.1 }

\textbf{TS 6.1.9.1 } \newline
\textbf{Samhita Paata} \newline

ब्र॒ह्म॒वा॒दिनो॑ वदन्ति वि॒चित्यः॒ सोमा(3) न वि॒चित्या(3) इति॒ सोमो॒ वा ओष॑धीनाꣳ॒॒ राजा॒ तस्मि॒न॒. यदाप॑न्नं ग्रसि॒तमे॒वास्य॒ तद्-यद्-वि॑चिनु॒याद्-यथा॒ ऽऽस्या᳚द्ग्रसि॒तं नि॑ष्खि॒दति॑ ता॒दृगे॒व तद्यन्न वि॑चिनु॒याद्-यथा॒ ऽक्षन्नाप॑न्नं ॅवि॒धाव॑ति ता॒दृगे॒व तत् क्षोधु॑को ऽद्ध्व॒र्युः स्यात् क्षोधु॑को॒ यज॑मानः॒ सोम॑विक्रयि॒न्थ् सोमꣳ॑ शोध॒येत्ये॒व ब्रू॑या॒द् यदीत॑रं॒ - [  ] \newline

\textbf{Pada Paata} \newline

ब्र॒ह्म॒वा॒दिन॒ इति॑ ब्रह्म - वा॒दिनः॑ । व॒द॒न्ति॒ । वि॒चित्य॒ इति॑ वि-चित्यः॑ । सोमा(3)ः । न । वि॒चित्या(3) इति वि - चित्या(3)ः । इति॑ । सोमः॑ । वै । ओष॑धीनाम् । राजा᳚ । तस्मिन्न्॑ । यत् । आप॑न्न॒मित्या - प॒न्न॒म् । ग्र॒सि॒तम् । ए॒व । अ॒स्य॒ । तत् । यत् । वि॒चि॒नु॒यादिति॑ वि - चि॒नु॒यात् । यथा᳚ । आ॒स्या᳚त् । ग्र॒सि॒तम् । नि॒ष्खि॒दतीति॑ निः -   खि॒दति॑ । ता॒दृक् । ए॒व । तत् । यत् । न । वि॒चि॒नु॒यादिति॑ वि - चि॒नु॒यात् । यथा᳚ । अ॒क्षन्न् । आप॑न्न॒मित्या - प॒न्न॒म् । वि॒धाव॒तीति॑ वि - धाव॑ति । ता॒दृक् । ए॒व । तत् । क्षोधु॑कः । अ॒द्ध्व॒र्युः । स्यात् । क्षोधु॑कः । यज॑मानः । सोम॑विक्रयि॒न्निति॒ सोम॑ -वि॒क्र॒यि॒न्न् । सोम᳚म् । शो॒ध॒य॒ । इति॑ । ए॒व । ब्रू॒या॒त् । यदि॑ । इत॑रम् ।  \newline




\markright{ TS 6.1.9.2  \hfill https://www.vedavms.in \hfill}
\addcontentsline{toc}{section}{ TS 6.1.9.2 }
\section*{ TS 6.1.9.2 }

\textbf{TS 6.1.9.2 } \newline
\textbf{Samhita Paata} \newline

ॅयदीत॑रमु॒भये॑नै॒व सो॑मविक्र॒यिण॑-मर्पयति॒ तस्मा᳚थ् सोमविक्र॒यी क्षोधु॑को ऽरु॒णो ह॑ स्मा॒ऽऽ*हौप॑वेशिः सोम॒क्रय॑ण ए॒वाहं तृ॑तीय सव॒नमव॑ रुन्ध॒ इति॑ पशू॒नां चर्म॑न् मिमीते प॒शूने॒वाव॑ रुन्धे प॒शवो॒ हि तृ॒तीयꣳ॒॒ सव॑नं॒ ॅयं का॒मये॑ताप॒शुः स्या॒दित्यृ॑क्ष॒तस्तस्य॑ मिमीत॒र्क्षं ॅवा अ॑पश॒व्यम॑प॒शुरे॒व भ॑वति॒ यं का॒मये॑त पशु॒मान्थ् स्या॒ - [  ] \newline

\textbf{Pada Paata} \newline

यदि॑ । इत॑रम् । उ॒भये॑न । ए॒व । सो॒म॒वि॒क्र॒यिण॒मिति॑ सोम - वि॒क्र॒यिण᳚म् । अ॒र्प॒य॒ति॒ । तस्मा᳚त् । सो॒म॒वि॒क्र॒यीति॑ सोम - वि॒क्र॒यी । क्षोधु॑कः । अ॒रु॒णः । ह॒ । स्म॒ । आ॒ह॒ । औप॑वेशि॒रित्यौप॑-वे॒शिः॒ । सो॒म॒क्रय॑ण॒ इति॑ सोम - क्रय॑णे । ए॒व । अ॒हम् । तृ॒ती॒य॒स॒व॒नमिति॑ तृतीय - स॒व॒नम् । अवेति॑ । रु॒न्धे॒ । इति॑ । प॒शू॒नाम् । चर्मन्न्॑ । मि॒मी॒ते॒ । प॒शून् । ए॒व । अवेति॑ । रु॒न्धे॒ । प॒शवः॑ । हि । तृ॒तीय᳚म् । सव॑नम् । यम् । का॒मये॑त । अ॒प॒शुः । स्या॒त् । इति॑ । ऋ॒क्ष॒तः । तस्य॑ । मि॒मी॒त॒ । ऋ॒क्षम् । वै । अ॒प॒श॒व्यम् । अ॒प॒शुः । ए॒व । भ॒व॒ति॒ । यम् । का॒मये॑त । प॒शु॒मानिति॑ पशु - मान् । स्या॒त् ।  \newline




\markright{ TS 6.1.9.3  \hfill https://www.vedavms.in \hfill}
\addcontentsline{toc}{section}{ TS 6.1.9.3 }
\section*{ TS 6.1.9.3 }

\textbf{TS 6.1.9.3 } \newline
\textbf{Samhita Paata} \newline

-दिति॑ लोम॒तस्तस्य॑ मिमीतै॒तद्वै प॑शू॒नाꣳ रू॒पꣳ रू॒पेणै॒वास्मै॑ प॒शूनव॑ रुन्धे पशु॒माने॒व भ॑वत्य॒पामन्ते᳚ क्रीणाति॒ सर॑समे॒वैनं॑ क्रीणात्य॒-मात्यो॒ऽसीत्या॑हा॒मैवैनं॑ कुरुते शु॒क्रस्ते॒ ग्रह॒ इत्या॑ह शु॒क्रो ह्य॑स्य॒ ग्रहो ऽन॒साऽच्छ॑ याति महि॒मान॑-मे॒वास्याच्छ॑ या॒त्यन॒सा - [  ] \newline

\textbf{Pada Paata} \newline

इति॑ । लो॒म॒तः । तस्य॑ । मि॒मी॒त॒ । ए॒तत् । वै । प॒शू॒नाम् । रू॒पम् । रू॒पेण॑ । ए॒व । अ॒स्मै॒ । प॒शून् । अवेति॑ । रु॒न्धे॒ । प॒शु॒मानिति॑ पशु - मान् । ए॒व । भ॒व॒ति॒ । अ॒पाम् । अन्ते᳚ । क्री॒णा॒ति॒ । सर॑स॒मिति॒ स-र॒स॒म् । ए॒व । ए॒न॒म् । क्री॒णा॒ति॒ । अ॒मात्यः॑ । अ॒सि॒ । इति॑ । आ॒ह॒ । अ॒मा । ए॒व । ए॒न॒म् । कु॒रु॒ते॒ । शु॒क्रः । ते॒ । ग्रहः॑ । इति॑ । आ॒ह॒ । शु॒क्रः । हि । अ॒स्य॒ । ग्रहः॑ । अन॑सा । अच्छ॑ । या॒ति॒ । म॒हि॒मान᳚म् । ए॒व । अ॒स्य॒ । अच्छ॑ । या॒ति॒ । अन॑सा ।  \newline




\markright{ TS 6.1.9.4  \hfill https://www.vedavms.in \hfill}
\addcontentsline{toc}{section}{ TS 6.1.9.4 }
\section*{ TS 6.1.9.4 }

\textbf{TS 6.1.9.4 } \newline
\textbf{Samhita Paata} \newline

ऽच्छ॑ याति॒ तस्मा॑दनोवा॒ह्यꣳ॑ स॒मे जीव॑नं॒ ॅयत्र॒ खलु॒ वा ए॒तꣳ शी॒र्ष्णा हर॑न्ति॒ तस्मा᳚च्छीर्.षहा॒र्यं॑ गि॒रौ जीव॑नम॒भि त्यं दे॒वꣳ स॑वि॒तार॒मित्यति॑-च्छन्दस॒र्चा मि॑मी॒ते ऽति॑च्छन्दा॒ वै सर्वा॑णि॒ छन्दाꣳ॑सि॒ सर्वे॑भिरे॒वैनं॒ छन्दो॑भिर्मिमीते॒ वर्ष्म॒ वा ए॒षा छन्द॑सां॒ ॅयदति॑च्छन्दा॒ यदति॑च्छन्दस॒र्चा मिमी॑ते॒ वर्ष्मै॒वैनꣳ॑ समा॒नानां᳚ करो॒त्येक॑यैकयो॒थ् सर्गं॑- [  ] \newline

\textbf{Pada Paata} \newline

अच्छ॑ । या॒ति॒ । तस्मा᳚त् । अ॒नो॒वा॒ह्य॑मित्य॑नः - वा॒ह्य᳚म् । स॒मे । जीव॑नम् । यत्र॑ । खलु॑ । वै । ए॒तम् । शी॒र्ष्णा । हर॑न्ति । तस्मा᳚त् । शी॒र्॒.ष॒हा॒र्य॑मिति॑ शीर्.ष-हा॒र्य᳚म् । गि॒रौ । जीव॑नम् । अ॒भीति॑ । त्यम् । दे॒वम् । स॒वि॒तार᳚म् । इति॑ । अति॑च्छन्द॒सेत्यति॑ - छ॒न्द॒सा॒ । ऋ॒चा । मि॒मी॒त॒ । अति॑च्छन्दा॒ इत्यति॑ - छ॒न्दाः॒ । वै । सर्वा॑णि । छन्दाꣳ॑सि । सर्वे॑भिः । ए॒व । ए॒न॒म् । छन्दो॑भि॒रिति॒ छन्दः॑-भिः॒ । मि॒मी॒ते॒ । वर्ष्म॑ । वै । ए॒षा । छन्द॑साम् । यत् । अति॑च्छन्दा॒ इत्यति॑ - छ॒न्दाः॒ । यत् । अति॑च्छन्द॒सेत्यति॑-छ॒न्द॒सा॒ । ऋ॒चा । मिमी॑ते । वर्ष्म॑ । ए॒व । ए॒न॒म् । स॒मा॒नाना᳚म् । क॒रो॒ति॒ । एक॑यैक॒येत्येक॑या - ए॒क॒या॒ । उ॒थ्सर्ग॒मित्यु॑त् - सर्ग᳚म् ।  \newline




\markright{ TS 6.1.9.5  \hfill https://www.vedavms.in \hfill}
\addcontentsline{toc}{section}{ TS 6.1.9.5 }
\section*{ TS 6.1.9.5 }

\textbf{TS 6.1.9.5 } \newline
\textbf{Samhita Paata} \newline

मिमी॒ते ऽया॑तयाम्नियायातयाम्नियै॒वैनं॑ मिमीते॒ तस्मा॒न्नाना॑वीर्या अ॒ङ्गुल॑यः॒ सर्वा᳚स्वङ्गु॒ष्ठमुप॒ नि गृ॑ह्णाति॒ तस्मा᳚थ् स॒माव॑द्वीर्यो॒ऽन्याभि॑-र॒ङ्गुलि॑भि॒स्तस्मा॒थ् सर्वा॒ अनु॒ सं च॑रति॒ यथ् स॒ह सर्वा॑भि॒र्मिमी॑त॒ सꣳश्लि॑ष्टा अ॒ङ्गुल॑यो जायेर॒-न्नेक॑यैकयो॒थ् सर्गं॑ मिमीते॒ तस्मा॒द् विभ॑क्ता जायन्ते॒ पञ्च॒ कृत्वो॒ यजु॑षा मिमीते॒ पञ्चा᳚क्षरा प॒ङ्क्तिः पाङ्क्तो॑ य॒ज्ञो य॒ज्ञ्मे॒वाव॑ रुन्धे॒ पञ्च॒ कृत्व॑स्तू॒ष्णीं- [  ] \newline

\textbf{Pada Paata} \newline

मि॒मी॒ते॒ । अया॑तयाम्नियायातयाम्नि॒येत्यया॑तयाम्निया-अ॒या॒त॒या॒म्नि॒या॒ । ए॒व । ए॒न॒म् । मि॒मी॒ते॒ । तस्मा᳚त् । नाना॑वीर्या॒ इति॒ नाना᳚ - वी॒र्याः॒ । अ॒ङ्गुल॑यः । सर्वा॑सु । अ॒ङ्गु॒ष्ठम् । उप॑ । नीति॑ । गृ॒ह्णा॒ति॒ । तस्मा᳚त् । स॒माव॑द्वीर्य॒ इति॑ स॒माव॑त् - वी॒र्यः॒ । अ॒न्याभिः॑ । अ॒ङ्गुलि॑भि॒रित्य॒ङ्गुलि॑ - भिः॒ । तस्मा᳚त् । सर्वाः᳚ । अनु॑ । समिति॑ । च॒र॒ति॒ । यत् । स॒ह । सर्वा॑भिः । मिमी॑त । सꣳश्लि॑ष्टा॒ इति॒ सं - श्लि॒ष्टाः॒ । अ॒ङ्गुल॑यः । जा॒ये॒र॒न्न् । एक॑यैक॒येत्येक॑या-ए॒क॒या॒ । उ॒थ्सर्ग॒मित्यु॑त् - सर्ग᳚म् । मि॒मी॒ते॒ । तस्मा᳚त् । विभ॑क्ता॒ इति॒ वि - भ॒क्ताः॒ । जा॒य॒न्ते॒ । पञ्च॑ । कृत्वः॑ । यजु॑षा । मि॒मी॒ते॒ । पञ्चा᳚क्ष॒रेति॒ पञ्च॑ - अ॒क्ष॒रा॒ । प॒ङ्क्तिः । पाङ्क्तः॑ । य॒ज्ञ्ः । य॒ज्ञ्म् । ए॒व । अवेति॑ । रु॒न्धे॒ । पञ्च॑ । कृत्वः॑ । तू॒ष्णीम् ।  \newline




\markright{ TS 6.1.9.6  \hfill https://www.vedavms.in \hfill}
\addcontentsline{toc}{section}{ TS 6.1.9.6 }
\section*{ TS 6.1.9.6 }

\textbf{TS 6.1.9.6 } \newline
\textbf{Samhita Paata} \newline

-दश॒ सं प॑द्यन्ते॒ दशा᳚क्षरा वि॒राडन्नं॑ ॅवि॒राड् वि॒राजै॒वान्नाद्य॒मव॑ रुन्धे॒ यद् यजु॑षा॒ मिमी॑ते भू॒तमे॒वाव॑ रुन्धे॒ यत् तू॒ष्णीं भ॑वि॒ष्यद्-यद्-वै तावा॑ने॒व सोमः॒ स्याद् याव॑न्तं॒ मिमी॑ते॒ यज॑मानस्यै॒व स्या॒न्नापि॑ सद॒स्या॑नां प्र॒जाभ्य॒स्त्वेत्युप॒ समू॑हति सद॒स्या॑ने॒वान्वा भ॑जति॒ वास॒सोप॑ नह्यति सर्वदे॒वत्यं॑ ॅवै - [  ] \newline

\textbf{Pada Paata} \newline

दश॑ । समिति॑ । प॒द्य॒न्ते॒ । दशा᳚क्ष॒रेति॒ दश॑ - अ॒क्ष॒रा॒ । वि॒राडिति॑ वि - राट् । अन्न᳚म् । वि॒राडिति॑ वि - राट् । वि॒राजेति॑ वि - राजा᳚ । ए॒व । अ॒न्नाद्य॒मित्य॑न्न -अद्य᳚म् । अवेति॑ । रु॒न्धे॒ । यत् । यजु॑षा । मिमी॑ते । भू॒तम् । ए॒व । अवेति॑ । रु॒न्धे॒ । यत् । तू॒ष्णीम् । भ॒वि॒ष्यत् । यत् । वै । तावान्॑ । ए॒व । सोमः॑ । स्यात् । याव॑न्तम् । मिमी॑ते । यज॑मानस्य । ए॒व । स्या॒त् । न । अपीति॑ । स॒द॒स्या॑नाम् । प्र॒जाभ्य॒ इति॑ प्र - जाभ्यः॑ । त्वा॒ । इति॑ । उप॑ । समिति॑ । ऊ॒ह॒ति॒ । स॒द॒स्यान्॑ । ए॒व । अ॒न्वाभ॑ज॒तीत्य॑नु-आभ॑जति । वास॑सा । उपेति॑ । न॒ह्य॒ति॒ । स॒र्व॒दे॒व॒त्य॑मिति॑ सर्व - दे॒व॒त्य᳚म् । वै ।  \newline




\markright{ TS 6.1.9.7  \hfill https://www.vedavms.in \hfill}
\addcontentsline{toc}{section}{ TS 6.1.9.7 }
\section*{ TS 6.1.9.7 }

\textbf{TS 6.1.9.7 } \newline
\textbf{Samhita Paata} \newline

वासः॒ सर्वा॑भिरे॒वैनं॑ दे॒वता॑भिः॒ सम॑र्द्धयति प॒शवो॒ वै सोमः॑ प्रा॒णाय॒ त्वेत्युप॑ नह्यति प्रा॒णमे॒व प॒शुषु॑ दधाति व्या॒नाय॒ त्वेत्यनु॑ शृन्थति व्या॒नमे॒व प॒शुषु॑ दधाति॒ तस्मा᳚थ् स्व॒पन्तं॑ प्रा॒णा न ज॑हति ॥ \newline

\textbf{Pada Paata} \newline

वासः॑ । सर्वा॑भिः । ए॒व । ए॒न॒म् । दे॒वता॑भिः । समिति॑ । अ॒द्‌र्ध॒य॒ति॒ । प॒शवः॑ । वै । सोमः॑ । प्रा॒णायेति॑ प्र - अ॒नाय॑ । त्वा॒ । इति॑ । उपेति॑ । न॒ह्य॒ति॒ । प्रा॒णमिति॑ प्र - अ॒नम् । ए॒व । प॒शुषु॑ । द॒धा॒ति॒ । व्या॒नायेति॑ वि - अ॒नाय॑ । त्वा॒ । इति॑ । अन्विति॑ । शृ॒न्थ॒ति॒ । व्या॒नमिति॑ वि - अ॒नम् । ए॒व । प॒शुषु॑ । द॒धा॒ति॒ । तस्मा᳚त् । स्व॒पन्त᳚म् । प्रा॒णा इति॑ प्र - अ॒नाः । न । ज॒ह॒ति॒ ॥  \newline




\markright{ TS 6.1.10.1  \hfill https://www.vedavms.in \hfill}
\addcontentsline{toc}{section}{ TS 6.1.10.1 }
\section*{ TS 6.1.10.1 }

\textbf{TS 6.1.10.1 } \newline
\textbf{Samhita Paata} \newline

यत् क॒लया॑ ते श॒फेन॑ ते क्रीणा॒नीति॒ पणे॒तागो॑अर्घꣳ॒॒ सोमं॑ कु॒र्यादगो॑अर्घं॒ ॅयज॑मान॒-मगो॑अर्घमद्ध्व॒र्युं गोस्तु म॑हि॒मानं॒ नाव॑ तिरे॒द् गवा॑ ते क्रीणा॒नीत्ये॒व ब्रू॑याद् गोअ॒र्घमे॒व सोमं॑ क॒रोति॑ गोअ॒र्घं ॅयज॑मानं गोअ॒र्घम॑द्ध्व॒र्युं न गोर्म॑हि॒मान॒मव॑ तिरत्य॒जया᳚ क्रीणाति॒ सत॑पसमे॒वैनं॑ क्रीणाति॒ हिर॑ण्येन क्रीणाति॒ सशु॑क्रमे॒वै - [  ] \newline

\textbf{Pada Paata} \newline

यत् । क॒लया᳚ । ते॒ । श॒फेन॑ । ते॒ । क्री॒णा॒नि॒ । इति॑ । पणे॑त । अगो॑अर्घ॒मित्यगो᳚ - अ॒र्घ॒म् । सोम᳚म् । कु॒र्यात् । अगो॑अर्घ॒मित्यगो᳚ - अ॒र्घ॒म् । यज॑मानम् । अगो॑अर्घ॒मित्यगो᳚-अ॒र्घ॒म् । अ॒द्ध्व॒र्युम् । गोः । तु । म॒हि॒मान᳚म् । न । अवेति॑ । ति॒रे॒त् । गवा᳚ । ते॒ । क्री॒णा॒नि॒ । इति॑ । ए॒व । ब्रू॒या॒त् । गो॒अ॒र्घमिति॑ गो - अ॒र्घम् । ए॒व । सोम᳚म् । क॒रोति॑ । गो॒अ॒र्घमिति॑ गो - अ॒र्घम् । यज॑मानम् । गो॒अ॒र्घमिति॑ गो - अ॒र्घम् । अ॒द्ध्व॒र्युम् । न । गोः । म॒हि॒मान᳚म् । अवेति॑ । ति॒र॒ति॒ । अ॒जया᳚ । क्री॒णा॒ति॒ । सत॑पस॒मिति॒ स - त॒प॒स॒म् । ए॒व । ए॒न॒म् । क्री॒णा॒ति॒ । हिर॑ण्येन । क्री॒णा॒ति॒ । सशु॑क्र॒मिति॒ स - शु॒क्र॒म् । ए॒व ।  \newline




\markright{ TS 6.1.10.2  \hfill https://www.vedavms.in \hfill}
\addcontentsline{toc}{section}{ TS 6.1.10.2 }
\section*{ TS 6.1.10.2 }

\textbf{TS 6.1.10.2 } \newline
\textbf{Samhita Paata} \newline

-नं॑ क्रीणाति धे॒न्वा क्री॑णाति॒ साशि॑रमे॒वैनं॑ क्रीणात्यृष॒भेण॑ क्रीणाति॒ सेन्द्र॑मे॒वैनं॑ क्रीणात्यन॒डुहा᳚ क्रीणाति॒ वह्नि॒र्वा अ॑न॒ड्वान्. वह्नि॑नै॒व वह्नि॑ य॒ज्ञ्स्य॑ क्रीणाति मिथु॒नाभ्यां᳚ क्रीणाति मिथु॒नस्याव॑-रुद्ध्यै॒ वास॑सा क्रीणाति सर्वदेव॒त्यं॑ ॅवै वा॒सः सर्वा᳚भ्य ए॒वैनं॑ दे॒वता᳚भ्यः क्रीणाति॒ दश॒ सं प॑द्यन्ते॒ दशा᳚क्षरा वि॒राडन्नं॑ ॅवि॒राड् वि॒राजै॒वान्नाद्य॒मव॑ रुन्धे॒ - [  ] \newline

\textbf{Pada Paata} \newline

ए॒न॒म् । क्री॒णा॒ति॒ । धे॒न्वा । क्री॒णा॒ति॒ । साशि॑र॒मिति॒ स - आ॒शि॒र॒म् । ए॒व । ए॒न॒म् । क्री॒णा॒ति॒ । ऋ॒ष॒भेण॑ । क्री॒णा॒ति॒ । सेन्द्र॒मिति॒ स-इ॒न्द्र॒म् । ए॒व । ए॒न॒म् । क्री॒णा॒ति॒ । अ॒न॒डुहा᳚ । क्री॒णा॒ति॒ । वह्निः॑ । वै । अ॒न॒ड्वान् । वह्नि॑ना । ए॒व । वह्नि॑ । य॒ज्ञ्स्य॑ । क्री॒णा॒ति॒ । मि॒थु॒नाभ्या᳚म् । क्री॒णा॒ति॒ । मि॒थु॒नस्य॑ । अव॑रुद्ध्या॒ इत्यव॑ - रु॒द्ध्यै॒ । वास॑सा । क्री॒णा॒ति॒ । स॒र्व॒दे॒व॒त्य॑मिति॑ सर्व - दे॒व॒त्य᳚म् । वै । वासः॑ । सर्वा᳚भ्यः । ए॒व । ए॒न॒म् । दे॒वता᳚भ्यः । क्री॒णा॒ति॒ । दश॑ । समिति॑ । प॒द्य॒न्ते॒ । दशा᳚क्ष॒रेति॒ दश॑ - अ॒क्ष॒रा॒ । वि॒राडिति॑ वि - राट् । अन्न᳚म् । वि॒राडिति॑ वि - राट् । वि॒राजेति॑ वि - राजा᳚ । ए॒व । अ॒न्नाद्य॒मित्य॑न्न -अद्य᳚म् । अवेति॑ । रु॒न्धे॒ ।  \newline




\markright{ TS 6.1.10.3  \hfill https://www.vedavms.in \hfill}
\addcontentsline{toc}{section}{ TS 6.1.10.3 }
\section*{ TS 6.1.10.3 }

\textbf{TS 6.1.10.3 } \newline
\textbf{Samhita Paata} \newline

तप॑सस्त॒नूर॑सि प्र॒जाप॑ते॒र्वर्ण॒ इत्या॑ह प॒शुभ्य॑ ए॒व तद॑द्ध्व॒र्युर्नि॑ ह्नु॑त आ॒त्मनोऽना᳚व्रस्काय॒ गच्छ॑ति॒ श्रियं॒ प्र प॒शूना᳚प्नोति॒ य ए॒वं ॅवेद॑ शु॒क्रं ते॑ शु॒क्रेण॑ क्रीणा॒मीत्या॑ह यथा य॒जुरे॒वैतद् दे॒वा वै येन॒ हिर॑ण्येन॒ सोम॒मक्री॑ण॒न् तद॑भी॒षहा॒ पुन॒राऽद॑दत॒ को हि तेज॑सा विक्रे॒ष्यत॒ इति॒ येन॒ हिर॑ण्येन॒ - [  ] \newline

\textbf{Pada Paata} \newline

तपसः॑ । त॒नूः । अ॒सि॒ । प्र॒जाप॑ते॒रिति॑ प्र॒जा - प॒तेः॒ । वर्णः॑ । इति॑ । आ॒ह॒ । प॒शुभ्य॒ इति॑ प॒शु - भ्यः॒ । ए॒व । तत् । अ॒द्ध्व॒र्युः । नीति॑ । ह्नु॒ते॒ । आ॒त्मनः॑ । अना᳚व्रस्का॒येत्यना᳚ - व्र॒स्का॒य॒ । गच्छ॑ति । श्रिय᳚म् । प्रेति॑ । प॒शून् । आ॒प्नो॒ति॒ । यः । ए॒वम् । वेद॑ । शु॒क्रम् । ते॒ । शु॒क्रेण॑ । क्री॒णा॒मि॒ । इति॑ । आ॒ह॒ । य॒था॒य॒जुरिति॑ यथा-यजुः । ए॒व । ए॒तत् । दे॒वाः । वै । येन॑ । हिर॑ण्येन । सोम᳚म् । अक्री॑णन्न् । तत् । अ॒भी॒षहेत्य॑भि-सहा᳚ । पुनः॑ । एति॑ । अ॒द॒द॒त॒ । कः । हि । तेज॑सा । वि॒क्रे॒ष्यत॒ इति॑ वि - क्रे॒ष्यते᳚ । इति॑ । येन॑ । हिर॑ण्येन ।  \newline




\markright{ TS 6.1.10.4  \hfill https://www.vedavms.in \hfill}
\addcontentsline{toc}{section}{ TS 6.1.10.4 }
\section*{ TS 6.1.10.4 }

\textbf{TS 6.1.10.4 } \newline
\textbf{Samhita Paata} \newline

सोमं॑ क्रीणी॒यात् तद॑भी॒षहा॒ पुन॒रा द॑दीत॒ तेज॑ ए॒वाऽऽत्मन् ध॑त्ते॒ऽस्मे ज्योतिः॑ सोमविक्र॒यिणि॒ तम॒ इत्या॑ह॒ ज्योति॑रे॒व यज॑माने दधाति॒ तम॑सा सोमविक्र॒यिण॑मर्पयति॒ यदनु॑पग्रथ्य ह॒न्याद्-द॑न्द॒शूका॒स्ताꣳ समाꣳ॑ स॒र्पाः स्यु॑रि॒दम॒हꣳ स॒र्पाणां᳚ दन्द॒शूका॑नां ग्री॒वा उप॑ ग्रथ्ना॒मीत्या॒हा-द॑न्दशूका॒स्ताꣳ समाꣳ॑ स॒र्पा भ॑वन्ति॒ तम॑सा सोमविक्र॒यिणं॑ ॅविद्ध्यति॒ स्वान॒ - [  ] \newline

\textbf{Pada Paata} \newline

सोम᳚म् । क्री॒णी॒यात् । तत् । अ॒भी॒षहेत्य॑भ - सहा᳚ । पुनः॑ । एति॑ । द॒दी॒त॒ । तेजः॑ । ए॒व । आ॒त्मन्न् । ध॒त्ते॒ । अ॒स्मे इति॑ । ज्योतिः॑ । सो॒म॒वि॒क्र॒यिणीति॑ सोम-वि॒क्र॒यिणि॑ । तमः॑ । इति॑ । आ॒ह॒ । ज्योतिः॑ । ए॒व । यज॑माने । द॒धा॒ति॒ । तम॑सा । सो॒म॒वि॒क्र॒यिण॒मिति॑ सोम - वि॒क्र॒यिण᳚म् । अ॒र्प॒य॒ति॒ । यत् । अनु॑पग्र॒थ्येत्यनु॑प - ग्र॒थ्य॒ । ह॒न्यात् । द॒न्द॒शूकाः᳚ । ताम् । समा᳚म् । स॒र्पाः । स्युः॒ । इ॒दम् । अ॒हम् । स॒र्पाणा᳚म् । द॒न्द॒शूका॑नाम् । ग्री॒वाः । उपेति॑ । ग्र॒थ्ना॒मि॒ । इति॑ । आ॒ह॒ । अद॑न्दशूकाः । ताम् । समा᳚म् । स॒र्पाः । भ॒व॒न्ति॒ । तम॑सा । सो॒म॒वि॒क्र॒यिण॒मिति॑ सोम - वि॒क्र॒यिण᳚म् । वि॒द्ध्य॒ति॒ । स्वान॑ ।  \newline




\markright{ TS 6.1.10.5  \hfill https://www.vedavms.in \hfill}
\addcontentsline{toc}{section}{ TS 6.1.10.5 }
\section*{ TS 6.1.10.5 }

\textbf{TS 6.1.10.5 } \newline
\textbf{Samhita Paata} \newline

भ्राजेत्या॑है॒ते वा अ॒मुष्मि॑न् ॅलो॒के सोम॑मरक्ष॒न् तेभ्योऽधि॒ सोम॒माऽह॑र॒न्॒. यदे॒तेभ्यः॑ सोम॒क्रय॑णा॒-न्नानु॑दि॒शेदक्री॑तोऽस्य॒ सोमः॑ स्या॒न्नास्यै॒ते॑ ऽमुष्मि॑न् ॅलो॒के सोमꣳ॑ रक्षेयु॒र्यदे॒तेभ्यः॑ सोम॒क्रय॑णाननुदि॒शति॑ क्री॒तो᳚ऽस्य॒ सोमो॑ भवत्ये॒ते᳚ऽस्या॒मुष्मि॑न् ॅलो॒के सोमꣳ॑ रक्षन्ति ॥ \newline

\textbf{Pada Paata} \newline

भ्राज॑ । इति॑ । आ॒ह॒ । ए॒ते । वै । अ॒मुष्मिन्न्॑ । लो॒के । सोम᳚म् । अ॒र॒क्ष॒न्न् । तेभ्यः॑ । अधीति॑ । सोम᳚म् । एति॑ । अ॒ह॒र॒न्न् । यत् । ए॒तेभ्यः॑ । सो॒म॒क्रय॑णा॒निति॑ सोम - क्रय॑णान् । न । अ॒नु॒दि॒शेदित्य॑नु - दि॒शेत् । अक्री॑तः । अ॒स्य॒ । सोमः॑ । स्या॒त् । न । अ॒स्य॒ । ए॒ते । अ॒मुष्मिन्न्॑ । लो॒के । सोम᳚म् । र॒क्षे॒युः॒ । यत् । ए॒तेभ्यः॑ । सो॒म॒क्रय॑णा॒निति॑ सोम - क्रय॑णान् । अ॒नु॒दि॒शतीत्य॑नु - दि॒शति॑ । क्री॒तः । अ॒स्य॒ । सोमः॑ । भ॒व॒ति॒ । ए॒ते । अ॒स्य॒ । अ॒मुष्मिन्न्॑ । लो॒के । सोम᳚म् । र॒क्ष॒न्ति॒ ॥  \newline




\markright{ TS 6.1.11.1  \hfill https://www.vedavms.in \hfill}
\addcontentsline{toc}{section}{ TS 6.1.11.1 }
\section*{ TS 6.1.11.1 }

\textbf{TS 6.1.11.1 } \newline
\textbf{Samhita Paata} \newline

वा॒रु॒णो वै क्री॒तः सोम॒ उप॑नद्धो मि॒त्रो न॒ एहि॒ सुमि॑त्रधा॒ इत्या॑ह॒ शान्त्या॒ इन्द्र॑स्यो॒रुमा वि॑श॒ दक्षि॑ण॒मित्या॑ह दे॒वा वै यꣳ सोम॒मक्री॑णन् तमिन्द्र॑स्यो॒रौ दक्षि॑ण॒ आ ऽसा॑दयन्ने॒ष खलु॒ वा ए॒तर्.हीन्द्रो॒ यो यज॑ते॒ तस्मा॑दे॒वमा॒होदायु॑षा स्वा॒युषेत्या॑ह दे॒वता॑ ए॒वा-न्वा॒रभ्योत् - [  ] \newline

\textbf{Pada Paata} \newline

वा॒रु॒णः । वै । क्री॒तः । सोमः॑ । उप॑नद्ध॒ इत्युप॑-न॒द्धः॒ । मि॒त्रः । नः॒ । एति॑ । इ॒हि॒ । सुमि॑त्रधा॒ इति॒ सुमि॑त्र - धाः॒ । इति॑ । आ॒ह॒ । शान्त्यै᳚ । इन्द्र॑स्य । ऊ॒रुम् । एति॑ । वि॒श॒ । दक्षि॑णम् । इति॑ । आ॒ह॒ । दे॒वाः । वै । यम् । सोम᳚म् । अक्री॑णन्न् । तम् । इन्द्र॑स्य । ऊ॒रौ । दक्षि॑णे । एति॑ । अ॒सा॒द॒य॒न्न् । ए॒षः । खलु॑ । वै । ए॒तर्.हि॑ । इन्द्रः॑ । यः । यज॑ते । तस्मा᳚त् । ए॒वम् । आ॒ह॒ । उदिति॑ । आयु॑षा । स्वा॒युषेति॑ सु - आ॒युषा᳚ । इति॑ । आ॒ह॒ । दे॒वताः᳚ । ए॒व । अ॒न्वा॒रभ्येत्य॑नु - आ॒रभ्य॑ । उदिति॑ ।  \newline




\markright{ TS 6.1.11.2  \hfill https://www.vedavms.in \hfill}
\addcontentsline{toc}{section}{ TS 6.1.11.2 }
\section*{ TS 6.1.11.2 }

\textbf{TS 6.1.11.2 } \newline
\textbf{Samhita Paata} \newline

ति॑ष्ठत्यु॒-र्व॑न्तरि॑क्ष॒-मन्वि॒हीत्या॑हा-न्तरिक्षदेव॒त्यो᳚(1॒) ह्ये॑तर्.हि॒ सोमोऽदि॑त्याः॒ सदो॒ऽस्यदि॑त्याः॒ सद॒ आ सी॒देत्या॑ह यथाय॒जुरे॒वैतद् द्वि वा ए॑नमे॒तद॑र्द्धयति॒ यद् वा॑रु॒णꣳ सन्तं॑ मै॒त्रं क॒रोति॑ वारु॒ण्यर्चा॑ऽऽ सा॑दयति॒ स्वयै॒वैनं॑ दे॒वत॑या॒ सम॑र्द्धयति॒ वास॑सा प॒र्यान॑ह्यति सर्वदेव॒त्यं॑ ॅवै वासः॒ सर्वा॑भिरे॒वै - [  ] \newline

\textbf{Pada Paata} \newline

ति॒ष्ठ॒ति॒ । उ॒रु । अ॒न्तरि॑क्षम् । अन्विति॑ । इ॒हि॒ । इति॑ । आ॒ह॒ । अ॒न्त॒रि॒क्ष॒दे॒व॒त्य॑ इत्य॑न्तरिक्ष - दे॒व॒त्यः॑ । हि । ए॒तर्.हि॑ । सोमः॑ । अदि॑त्याः । सदः॑ । अ॒सि॒ । अदि॑त्याः । सदः॑ । एति॑ । सी॒द॒ । इति॑ । आ॒ह॒ । य॒था॒य॒जुरिति॑ यथा-य॒जुः । ए॒व । ए॒तत् । वीति॑ । वै । ए॒न॒म् । ए॒तत् । अ॒द्‌र्ध॒य॒ति॒ । यत् । वा॒रु॒णम् । सन्त᳚म् । मै॒त्रम् । क॒रोति॑ । वा॒रु॒ण्या । ऋ॒चा । एति॑ । सा॒द॒य॒ति॒ । स्वया᳚ । ए॒व । ए॒न॒म् । दे॒वत॑या । समिति॑ । अ॒द्‌र्ध॒य॒ति॒ । वास॑सा । प॒र्यान॑ह्य॒तीति॑ परि - आन॑ह्यति । स॒र्व॒दे॒व॒त्य॑मिति॑ सर्व -दे॒व॒त्य᳚म् । वै । वासः॑ । सर्वा॑भिः । ए॒व ।  \newline




\markright{ TS 6.1.11.3  \hfill https://www.vedavms.in \hfill}
\addcontentsline{toc}{section}{ TS 6.1.11.3 }
\section*{ TS 6.1.11.3 }

\textbf{TS 6.1.11.3 } \newline
\textbf{Samhita Paata} \newline

-नं॑ दे॒वता॑भिः॒ सम॑र्द्धय॒त्यथो॒ रक्ष॑सा॒मप॑हत्यै॒ वने॑षु॒ व्य॑न्तरि॑क्षं तता॒नेत्या॑ह॒ वने॑षु॒ हि व्य॑न्तरि॑क्षं त॒तान॒ वाज॒मर्व॒थ्स्वित्या॑ह॒ वाजꣳ॒॒ ह्यर्व॑थ्सु॒ पयो॑ अघ्नि॒यास्वित्या॑ह॒ पयो॒ ह्य॑घ्नि॒यासु॑ हृ॒थ्सु क्रतु॒मित्या॑ह हृ॒थ्सु हि क्रतुं॒ ॅवरु॑णो वि॒क्ष्व॑ग्निमित्या॑ह॒ वरु॑णो॒ हि वि॒क्ष्व॑ग्निं दि॒वि सूर्य॒ - [  ] \newline

\textbf{Pada Paata} \newline

ए॒न॒म् । दे॒वता॑भिः । समिति॑ । अ॒द्‌र्ध॒य॒ति॒ । अथो॒ इति॑ । रक्ष॑साम् । अप॑हत्या॒ इत्यप॑-ह॒त्यै॒ । वने॑षु । वीति॑ । अ॒न्तरि॑क्षम् । त॒ता॒न॒ । इति॑ । आ॒ह॒ । वने॑षु । हि । वीति॑ । अ॒न्तरि॑क्षम् । त॒तान॑ । वाज᳚म् । अर्व॒थ्स्वित्यर्व॑त् - सु॒ । इति॑ । आ॒ह॒ । वाज᳚म् । हि । अर्व॒थ्स्वित्यर्व॑त् - सु॒ । पयः॑ । अ॒घ्नि॒यासु॑ । इति॑ । आ॒ह॒ । पयः॑ । हि । अ॒घ्नि॒यासु॑ । हृ॒थ्स्विति॑ हृत् - सु । क्रतु᳚म् । इति॑ । आ॒ह॒ । हृ॒थ्स्विति॑ हृत्-सु । हि । क्रतु᳚म् । वरु॑णः । वि॒क्षु । अ॒ग्निम् । इति॑ । आ॒ह॒ । वरु॑णः । हि । वि॒क्षु । अ॒ग्निम् । दि॒वि । सूर्य᳚म् ।  \newline




\markright{ TS 6.1.11.4  \hfill https://www.vedavms.in \hfill}
\addcontentsline{toc}{section}{ TS 6.1.11.4 }
\section*{ TS 6.1.11.4 }

\textbf{TS 6.1.11.4 } \newline
\textbf{Samhita Paata} \newline

मित्या॑ह दि॒वि हि सूर्यꣳ॒॒ सोम॒मद्रा॒वित्या॑ह॒ ग्रावा॑णो॒ वा अद्र॑य॒स्तेषु॒ वा ए॒ष सोमं॑ दधाति॒ यो यज॑ते॒ तस्मा॑दे॒वमा॒होदु॒ त्यं जा॒तवे॑दस॒मिति॑ सौ॒र्यर्चा कृ॑ष्णाजि॒नं प्र॒त्यान॑ह्यति॒ रक्ष॑सा॒मप॑हत्या॒ उस्रा॒वेतं॑ धूर्.षाहा॒वित्या॑ह यथाय॒जुरे॒वैतत् प्र च्य॑वस्व भुवस्पत॒ इत्या॑ह भू॒तानाꣳ॒॒ ह्ये॑ - [  ] \newline

\textbf{Pada Paata} \newline

इति॑ । आ॒ह॒ । दि॒वि । हि । सूर्य᳚म् । सोम᳚म् । अद्रौ᳚ । इति॑ । आ॒ह॒ । ग्रावा॑णः । वै । अद्र॑यः । तेषु॑ । वै । ए॒षः । सोम᳚म् । द॒धा॒ति॒ । यः । यज॑ते । तस्मा᳚त् । ए॒वम् । आ॒ह॒ । उदिति॑ । उ॒ । त्यम् । जा॒तवे॑दस॒मिति॑ जा॒त - वे॒द॒स॒म् । इति॑ । सौ॒र्या । ऋ॒चा । कृ॒ष्णा॒जि॒नमिति॑ कृष्ण - अ॒जि॒नम् । प्र॒त्यान॑ह्य॒तीति॑ प्रति-आन॑ह्यति । रक्ष॑साम् । अप॑हत्या॒ इत्यप॑ - ह॒त्यै॒ । उस्रौ᳚ । एति॑ । इ॒त॒म् । धू॒र्॒.षा॒हा॒विति॑ धूः - सा॒हौ॒ । इति॑ । आ॒ह॒ । य॒था॒य॒जुरिति॑ यथा - य॒जुः । ए॒व । ए॒तत् । प्रेति॑ । च्य॒व॒स्व॒ । भु॒वः॒ । प॒ते॒ । इति॑ । आ॒ह॒ । भू॒ताना᳚म् । हि ।  \newline




\markright{ TS 6.1.11.5  \hfill https://www.vedavms.in \hfill}
\addcontentsline{toc}{section}{ TS 6.1.11.5 }
\section*{ TS 6.1.11.5 }

\textbf{TS 6.1.11.5 } \newline
\textbf{Samhita Paata} \newline

-ष पति॒र्विश्वा᳚न्य॒भि धामा॒नीत्या॑ह॒ विश्वा॑नि॒ ह्ये᳚(1॒) षो॑ऽभि धामा॑नि प्र॒च्यव॑ते॒ मा त्वा॑ परिप॒री वि॑द॒दित्या॑ह॒ यदे॒वादः सोम॑माह्रि॒यमा॑णं गन्ध॒र्वो वि॒श्वाव॑सुः प॒र्यमु॑ष्णा॒त् तस्मा॑-दे॒वमा॒हाप॑रिमोषाय॒ यज॑मानस्य स्व॒स्त्यय॑न्य॒सीत्या॑ह॒ यज॑मानस्यै॒वैष य॒ज्ञ्स्या᳚न्वार॒भ्ॐ ऽन॑वच्छित्त्यै॒ वरु॑णो॒ वा ए॒ष यज॑मानम॒भ्यैति॒ यत् - [  ] \newline

\textbf{Pada Paata} \newline

ए॒षः । पतिः॑ । विश्वा॑नि । अ॒भीति॑ । धामा॑नि । इति॑ । आ॒ह॒ । विश्वा॑नि । हि । ए॒षः । अ॒भीति॑ । धामा॑नि । प्र॒च्यव॑त॒ इति॑ प्र - च्यव॑ते । मा । त्वा॒ । प॒रि॒प॒रीति॑ परि - प॒री । वि॒द॒त् । इति॑ । आ॒ह॒ । यत् । ए॒व । अ॒दः । सोम᳚म् । आ॒ह्रि॒यमा॑ण॒मित्या᳚ - ह्रि॒यमा॑णम् । ग॒न्ध॒र्वः । वि॒श्वाव॑सु॒रिति॑ वि॒श्व - व॒सुः॒ । प॒र्यमु॑ष्णा॒दिति॑ परि - अमु॑ष्णात् । तस्मा᳚त् । ए॒वम् । आ॒ह॒ । अप॑रिमोषा॒येत्यप॑रि - मो॒षा॒य॒ । यज॑मानस्य । स्व॒स्त्यय॒नीति॑ स्वस्ति - अय॑नी । अ॒सि॒ । इति॑ । आ॒ह॒ । यज॑मानस्य । ए॒व । ए॒षः । य॒ज्ञ्स्य॑ । अ॒न्वा॒र॒म्भ इत्य॑नु - आ॒र॒म्भः । अन॑वच्छित्त्या॒ इत्यन॑व - छि॒त्यै॒ । वरु॑णः । वै । ए॒षः । यज॑मानम् । अ॒भि । एति॑ । ए॒ति॒ । यत् ।  \newline




\markright{ TS 6.1.11.6  \hfill https://www.vedavms.in \hfill}
\addcontentsline{toc}{section}{ TS 6.1.11.6 }
\section*{ TS 6.1.11.6 }

\textbf{TS 6.1.11.6 } \newline
\textbf{Samhita Paata} \newline

क्री॒तः सोम॒ उप॑नद्धो॒ नमो॑ मि॒त्रस्य॒ वरु॑णस्य॒ चक्ष॑स॒ इत्या॑ह॒ शान्त्या॒ आ सोमं॒ ॅवह॑न्त्य॒ग्निना॒ प्रति॑ तिष्ठते॒ तौ स॒भंव॑न्तौ॒ यज॑मानम॒भि सं भ॑वतः पु॒रा खलु॒ वावैष मेधा॑या॒ऽऽ*त्मान॑मा॒रभ्य॑ चरति॒ यो दी᳚क्षि॒तो यद॑ग्नीषो॒मीयं॑ प॒शुमा॒लभ॑त आत्मनि॒ष्क्रय॑ण ए॒वास्य॒ स तस्मा॒त् तस्य॒ नाऽऽ*श्यं॑ पुरुषनि॒ष्क्रय॑ण इव॒ ह्यथो॒ खल्वा॑हु ( ) र॒ग्नीषोमा᳚भ्यां॒ ॅवा इन्द्रो॑ वृ॒त्रम॑ह॒न्निति॒ यद॑ग्नीषो॒मीयं॑ प॒शुमा॒लभ॑त॒ वार्त्र॑घ्न ए॒वास्य॒ स तस्मा᳚द्-वा॒श्यं॑ ॅवारु॒ण्यर्चा परि॑ चरति॒ स्वयै॒वैनं॑ दे॒वत॑या॒ परि॑ चरति ॥ \newline

\textbf{Pada Paata} \newline

क्री॒तः । सोमः॑ । उप॑नद्ध॒ इत्युप॑ - न॒द्धः॒ । नमः॑ । मि॒त्रस्य॑ । वरु॑णस्य । चक्ष॑से । इति॑ । आ॒ह॒ । शान्त्यै᳚ । एति॑ । सोम᳚म् । वह॑न्ति । अ॒ग्निना᳚ । प्रतीति॑ । ति॒ष्ठ॒ते॒ । तौ । स॒भंव॑न्ता॒विति॑ सं - भव॑न्तौ । यज॑मानम् । अ॒भि । समिति॑ । भ॒व॒तः॒ । पु॒रा । खलु॑ । वाव । ए॒षः । मेधा॑य । आ॒त्मान᳚म् । आ॒रभ्येत्या᳚-रभ्य॑ । च॒र॒ति॒ । यः । दी॒क्षि॒तः । यत् । अ॒ग्नी॒षो॒मीय॒मित्य॑ग्नी - सो॒मीय᳚म् । प॒शुम् । आ॒लभ॑त॒ इत्या᳚ - लभ॑ते । आ॒त्म॒नि॒ष्क्रय॑ण॒ इत्या᳚त्म - नि॒ष्क्रय॑णः । ए॒व । अ॒स्य॒ । सः । तस्मा᳚त् । तस्य॑ । न । आ॒श्य᳚म् । पु॒रु॒ष॒नि॒ष्क्रय॑ण॒ इति॑ पुरुष - नि॒ष्क्रय॑णः । इ॒व॒ । हि । अथो॒ इति॑ । खलु॑ । आ॒हुः॒ ( ) । अ॒ग्नीषोमा᳚भ्या॒मित्य॒ग्नी - सोमा᳚भ्याम् । वै । इन्द्रः॑ । वृ॒त्रम् । अ॒ह॒न्न् । इति॑ । यत् । अ॒ग्नी॒षो॒मीय॒मित्य॑ग्नी - सो॒मीय᳚म् । प॒शुम् । आ॒लभ॑त॒ इत्या᳚-लभ॑ते । वार्त्र॑घ्न॒ इति॒ वार्त्र॑ - घ्नः॒ । ए॒व । अ॒स्य॒ । सः । तस्मा᳚त् । उ॒ । आ॒श्य᳚म् । वा॒रु॒ण्या । ऋ॒चा । परीति॑ । च॒र॒ति॒ । स्वया᳚ । ए॒व । ए॒न॒म् । दे॒वत॑या । परीति॑ । च॒र॒ति॒ ॥  \newline






\end{document}