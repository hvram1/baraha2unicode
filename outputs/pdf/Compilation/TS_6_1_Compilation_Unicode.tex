\documentclass[17pt]{extarticle}
\usepackage{babel}
\usepackage{fontspec}
\usepackage{polyglossia}
\usepackage{extsizes}

\usepackage{color}   %May be necessary if you want to color links
\usepackage{hyperref}
\hypersetup{
    colorlinks=true, %set true if you want colored links
    linktoc=all,     %set to all if you want both sections and subsections linked
    linkcolor=black,  %choose some color if you want links to stand out
}

\setmainlanguage{sanskrit}
\setotherlanguages{english} %% or other languages
\setlength{\parindent}{0pt}
\pagestyle{myheadings}
\newfontfamily\devanagarifont[Script=Devanagari]{AdishilaVedic}
\renewcommand{\theHsection}{\thepart.section.\thesection}

\newcommand{\VAR}[1]{}
\newcommand{\BLOCK}[1]{}




\begin{document}
\begin{titlepage}
    \begin{center}
 
\begin{sanskrit}
    { \Large
    कृष्ण यजुर्वेदीय तैत्तिरीय संहिता,पद,जटा,घन पाठः 
    }
    \\
    \vspace{2.5cm}
    \mbox{ \Large
    6.1      षष्ठकाण्डे प्रथमः प्रश्नः - सोममन्त्रब्राह्मणनिरूपणं   }
\end{sanskrit}
\end{center}

\end{titlepage}
\tableofcontents
\phantomsection
\pagebreak

\markright{ TS 6.1.1.1  \hfill https://www.vedavms.in \hfill}

\section{ TS 6.1.1.1 }

\textbf{TS 6.1.1.1 } \newline
\textbf{Samhita Paata} \newline

प्रा॒चीन॑वꣳशं करोति देवमनु॒ष्या दिशो॒ व्य॑भजन्त॒ प्राचीं᳚ दे॒वा द॑क्षि॒णा पि॒तरः॑ प्र॒तीचीं᳚ मनु॒ष्या॑ उदी॑चीꣳ रु॒द्रा यत् प्रा॒चीन॑वꣳशं क॒रोति॑ देवलो॒कमे॒व तद्-यज॑मान उ॒पाव॑र्तते॒ परि॑ श्रयत्य॒न्तर्.हि॑तो॒हि दे॑वलो॒को म॑नुष्यलो॒का-न्नास्माल्लो॒काथ् स्वे॑तव्यमि॒वेत्या॑हुः॒ को हि तद्-वेद॒ यद्-य॒मुष्मि॑ॅल्लो॒केऽस्ति॑ वा॒ न वेति॑ दि॒क्ष्व॑ती का॒शान् क॑रो - [  ] \newline

\textbf{Pada Paata} \newline

प्रा॒चीन॑वꣳश॒मिति॑ प्रा॒चीन॑ - वꣳ॒॒श॒म् । क॒रो॒ति॒ । दे॒व॒म॒नु॒ष्या इति॑ देव - म॒नु॒ष्याः । दिशः॑ । वीति॑ । अ॒भ॒ज॒न्त॒ । प्राची᳚म् । दे॒वाः । द॒क्षि॒णा । पि॒तरः॑ । प्र॒तीची᳚म् । म॒नु॒ष्याः᳚ । उदी॑चीम् । रु॒द्राः । यत् । प्रा॒चीन॑वꣳश॒मिति॑ प्रा॒चीन॑ - वꣳ॒॒श॒म् । क॒रोति॑ । दे॒व॒लो॒कमिति॑ देव - लो॒कम् । ए॒व । तत् । यज॑मानः । उ॒पाव॑र्तत॒ इत्यु॑प - आव॑र्तते । परीति॑ । श्र॒य॒ति॒ । अ॒न्तर्.हि॑त॒ इत्य॒न्तः - हि॒तः॒ । हि । दे॒व॒लो॒क इति॑ देव - लो॒कः । म॒नु॒ष्य॒लो॒कादिति॑ मनुष्य - लो॒कात् । न । अ॒स्मात् । लो॒कात् । स्वे॑तव्य॒मिति॒ सु - ए॒त॒व्य॒म् । इ॒व॒ । इति॑ । आ॒हुः॒ । कः । हि । तत् । वेद॑ । यदि॑ । अ॒मुष्मिन्न्॑ । लो॒के । अस्ति॑ । वा॒ । न । वा॒ । इति॑ । दि॒क्षु । अ॒ती॒का॒शान् । क॒रो॒ति॒ ।  \newline


\textbf{Krama Paata} \newline

प्रा॒चीन॑वꣳशम् करोति । प्रा॒चीन॑वꣳश॒मिति॑ प्रा॒चीन॑ - वꣳ॒॒श॒म् । क॒रो॒ति॒ दे॒व॒म॒नु॒ष्याः । दे॒व॒म॒नु॒ष्या दिशः॑ । दे॒व॒म॒नु॒ष्या इति॑ देव - म॒नु॒ष्याः । दिशो॒ वि । व्य॑भजन्त । अ॒भ॒ज॒न्त॒ प्राची᳚म् । प्राची᳚म् दे॒वाः । दे॒वा द॑क्षि॒णा । द॒क्षि॒णा पि॒तरः॑ । पि॒तरः॑ प्र॒तीची᳚म् । प्र॒तीची᳚म् मनु॒ष्याः᳚ । म॒नु॒ष्या॑ उदी॑चीम् । उदी॑चीꣳ रु॒द्राः । रु॒द्रा यत् । यत् प्रा॒चीन॑वꣳशम् । प्रा॒चीन॑वꣳशम् क॒रोति॑ । प्रा॒चीन॑वꣳश॒मिति॑ प्रा॒चीन॑ - वꣳ॒॒श॒म् । क॒रोति॑ देवलो॒कम् । दे॒व॒लो॒कमे॒व । दे॒व॒लो॒कमिति॑ देव - लो॒कम् । ए॒व तत् । तद् यज॑मानः । यज॑मान उ॒पाव॑र्तते । उ॒पाव॑र्तते॒ परि॑ । उ॒पाव॑र्तत॒ इत्यु॑प - आव॑र्तते । परि॑ श्रयति । श्र॒य॒त्य॒न्तर्.हि॑तः । अ॒न्तर्.हि॑तो॒ हि । अ॒न्तर्.हि॑त॒ इत्य॒न्तः - हि॒तः॒ । हि दे॑वलो॒कः । दे॒व॒लो॒को म॑नुष्यलो॒कात् । दे॒व॒लो॒क इति॑ देव - लो॒कः । म॒नु॒ष्य॒लो॒कान् न । म॒नु॒ष्य॒लो॒कादिति॑ मनुष्य - लो॒कात् । नास्मात् । अ॒स्माल्लो॒कात् । लो॒काथ् स्वे॑तव्यम् । स्वे॑तव्यमिव । स्वे॑तव्य॒मिति॒ सु - ए॒त॒व्य॒म् । इ॒वेति॑ । इत्या॑हुः । आ॒हुः॒ कः । को हि । हि तत् । तद् वेद॑ । वेद॒ यदि॑ । यद्य॒मुष्मिन्न्॑ । अ॒मुष्मि॑न् ॅलो॒के । लो॒केऽस्ति॑ । अस्ति॑ वा । वा॒ न । न वा᳚ । वेति॑ । इति॑ दि॒क्षु । दि॒क्ष्व॑तीका॒शान् । अ॒ती॒का॒शान् क॑रोति । क॒रो॒त्यु॒भयोः᳚ \newline

\textbf{Jatai Paata} \newline

1. प्रा॒चीन॑वꣳशम् करोति करोति प्रा॒चीन॑वꣳशम् प्रा॒चीन॑वꣳशम् करोति । \newline
2. प्रा॒चीन॑वꣳश॒मिति॑ प्रा॒चीन॑ - वꣳ॒॒श॒म् । \newline
3. क॒रो॒ति॒ दे॒व॒म॒नु॒ष्या दे॑वमनु॒ष्याः क॑रोति करोति देवमनु॒ष्याः । \newline
4. दे॒व॒म॒नु॒ष्या दिशो॒ दिशो॑ देवमनु॒ष्या दे॑वमनु॒ष्या दिशः॑ । \newline
5. दे॒व॒म॒नु॒ष्या इति॑ देव - म॒नु॒ष्याः । \newline
6. दिशो॒ वि वि दिशो॒ दिशो॒ वि । \newline
7. व्य॑भजन्ता भजन्त॒ वि व्य॑भजन्त । \newline
8. अ॒भ॒ज॒न्त॒ प्राची॒म् प्राची॑ मभजन्ता भजन्त॒ प्राची᳚म् । \newline
9. प्राची᳚म् दे॒वा दे॒वाः प्राची॒म् प्राची᳚म् दे॒वाः । \newline
10. दे॒वा द॑क्षि॒णा द॑क्षि॒णा दे॒वा दे॒वा द॑क्षि॒णा । \newline
11. द॒क्षि॒णा पि॒तरः॑ पि॒तरो॑ दक्षि॒णा द॑क्षि॒णा पि॒तरः॑ । \newline
12. पि॒तरः॑ प्र॒तीची᳚म् प्र॒तीची᳚म् पि॒तरः॑ पि॒तरः॑ प्र॒तीची᳚म् । \newline
13. प्र॒तीची᳚म् मनु॒ष्या॑ मनु॒ष्याः᳚ प्र॒तीची᳚म् प्र॒तीची᳚म् मनु॒ष्याः᳚ । \newline
14. म॒नु॒ष्या॑ उदी॑ची॒ मुदी॑चीम् मनु॒ष्या॑ मनु॒ष्या॑ उदी॑चीम् । \newline
15. उदी॑चीꣳ रु॒द्रा रु॒द्रा उदी॑ची॒ मुदी॑चीꣳ रु॒द्राः । \newline
16. रु॒द्रा यद् यद् रु॒द्रा रु॒द्रा यत् । \newline
17. यत् प्रा॒चीन॑वꣳशम् प्रा॒चीन॑वꣳशं॒ ॅयद् यत् प्रा॒चीन॑वꣳशम् । \newline
18. प्रा॒चीन॑वꣳशम् क॒रोति॑ क॒रोति॑ प्रा॒चीन॑वꣳशम् प्रा॒चीन॑वꣳशम् क॒रोति॑ । \newline
19. प्रा॒चीन॑वꣳश॒मिति॑ प्रा॒चीन॑ - वꣳ॒॒श॒म् । \newline
20. क॒रोति॑ देवलो॒कम् दे॑वलो॒कम् क॒रोति॑ क॒रोति॑ देवलो॒कम् । \newline
21. दे॒व॒लो॒क मे॒वैव दे॑वलो॒कम् दे॑वलो॒क मे॒व । \newline
22. दे॒व॒लो॒कमिति॑ देव - लो॒कम् । \newline
23. ए॒व तत् तदे॒ वैव तत् । \newline
24. तद् यज॑मानो॒ यज॑मान॒ स्तत् तद् यज॑मानः । \newline
25. यज॑मान उ॒पाव॑र्तत उ॒पाव॑र्तते॒ यज॑मानो॒ यज॑मान उ॒पाव॑र्तते । \newline
26. उ॒पाव॑र्तते॒ परि॒ पर्यु॒ पाव॑र्तत उ॒पाव॑र्तते॒ परि॑ । \newline
27. उ॒पाव॑र्तत॒ इत्यु॑प - आव॑र्तते । \newline
28. परि॑ श्रयति श्रयति॒ परि॒ परि॑ श्रयति । \newline
29. श्र॒य॒ त्य॒न्तर्.हि॑तो॒ ऽन्तर्.हि॑तः श्रयति श्रय त्य॒न्तर्.हि॑तः । \newline
30. अ॒न्तर्.हि॑तो॒ हि ह्य॑न्तर्.हि॑तो॒ ऽन्तर्.हि॑तो॒ हि । \newline
31. अ॒न्तर्.हि॑त॒ इत्य॒न्तः - हि॒तः॒ । \newline
32. हि दे॑वलो॒को दे॑वलो॒को हि हि दे॑वलो॒कः । \newline
33. दे॒व॒लो॒को म॑नुष्यलो॒कान् म॑नुष्यलो॒काद् दे॑वलो॒को दे॑वलो॒को म॑नुष्यलो॒कात् । \newline
34. दे॒व॒लो॒क इति॑ देव - लो॒कः । \newline
35. म॒नु॒ष्य॒लो॒कान् न न म॑नुष्यलो॒कान् म॑नुष्यलो॒कान् न । \newline
36. म॒नु॒ष्य॒लो॒कादिति॑ मनुष्य - लो॒कात् । \newline
37. नास्मा द॒स्मान् न नास्मात् । \newline
38. अ॒स्मा ल्लो॒का ल्लो॒का द॒स्मा द॒स्मा ल्लो॒कात् । \newline
39. लो॒काथ् स्वे॑तव्यꣳ॒॒ स्वे॑तव्यम् ॅलो॒का ल्लो॒काथ् स्वे॑तव्यम् । \newline
40. स्वे॑तव्य मिवेव॒ स्वे॑तव्यꣳ॒॒ स्वे॑तव्य मिव । \newline
41. स्वे॑तव्य॒मिति॒ सु - ए॒त॒व्य॒म् । \newline
42. इ॒वे तीती॑वे॒ वेति॑ । \newline
43. इत्या॑हु राहु॒ रिती त्या॑हुः । \newline
44. आ॒हुः॒ कः क आ॑हु राहुः॒ कः । \newline
45. को हि हि कः को हि । \newline
46. हि तत् तद्धि हि तत् । \newline
47. तद् वेद॒ वेद॒ तत् तद् वेद॑ । \newline
48. वेद॒ यदि॒ यदि॒ वेद॒ वेद॒ यदि॑ । \newline
49. यद्य॒मुष्मि॑न् न॒मुष्मि॒न्॒. यदि॒ यद्य॒मुष्मिन्न्॑ । \newline
50. अ॒मुष्मि॑न् ॅलो॒के लो॒के॑ ऽमुष्मि॑न् न॒मुष्मि॑न् ॅलो॒के । \newline
51. लो॒के ऽस्त्यस्ति॑ लो॒के लो॒के ऽस्ति॑ । \newline
52. अस्ति॑ वा॒ वा ऽस्त्यस्ति॑ वा । \newline
53. वा॒ न न वा॑ वा॒ न । \newline
54. न वा॑ वा॒ न न वा᳚ । \newline
55. वेतीति॑ वा॒ वेति॑ । \newline
56. इति॑ दि॒क्षु दि॒क्ष्वि तीति॑ दि॒क्षु । \newline
57. दि॒क्ष्व॑ तीका॒शा न॑तीका॒शान् दि॒क्षु दि॒क्ष्व॑ तीका॒शान् । \newline
58. अ॒ती॒का॒शान् क॑रोति करोत्य तीका॒शा न॑तीका॒शान् क॑रोति । \newline
59. क॒रो॒ त्यु॒भयो॑ रु॒भयोः᳚ करोति करो त्यु॒भयोः᳚ । \newline

\textbf{Ghana Paata } \newline

1. प्रा॒चीन॑वꣳशम् करोति करोति प्रा॒चीन॑वꣳशम् प्रा॒चीन॑वꣳशम् करोति देवमनु॒ष्या दे॑वमनु॒ष्याः क॑रोति प्रा॒चीन॑वꣳशम् प्रा॒चीन॑वꣳशम् करोति देवमनु॒ष्याः । \newline
2. प्रा॒चीन॑वꣳश॒मिति॑ प्रा॒चीन॑ - वꣳ॒॒श॒म् । \newline
3. क॒रो॒ति॒ दे॒व॒म॒नु॒ष्या दे॑वमनु॒ष्याः क॑रोति करोति देवमनु॒ष्या दिशो॒ दिशो॑ देवमनु॒ष्याः क॑रोति करोति देवमनु॒ष्या दिशः॑ । \newline
4. दे॒व॒म॒नु॒ष्या दिशो॒ दिशो॑ देवमनु॒ष्या दे॑वमनु॒ष्या दिशो॒ वि वि दिशो॑ देवमनु॒ष्या दे॑वमनु॒ष्या दिशो॒ वि । \newline
5. दे॒व॒म॒नु॒ष्या इति॑ देव - म॒नु॒ष्याः । \newline
6. दिशो॒ वि वि दिशो॒ दिशो॒ व्य॑भजन्ता भजन्त॒ वि दिशो॒ दिशो॒ व्य॑भजन्त । \newline
7. व्य॑भजन्ता भजन्त॒ वि व्य॑भजन्त॒ प्राची॒म् प्राची॑ मभजन्त॒ वि व्य॑भजन्त॒ प्राची᳚म् । \newline
8. अ॒भ॒ज॒न्त॒ प्राची॒म् प्राची॑ मभजन्ता भजन्त॒ प्राची᳚म् दे॒वा दे॒वाः प्राची॑ मभजन्ता भजन्त॒ प्राची᳚म् दे॒वाः । \newline
9. प्राची᳚म् दे॒वा दे॒वाः प्राची॒म् प्राची᳚म् दे॒वा द॑क्षि॒णा द॑क्षि॒णा दे॒वाः प्राची॒म् प्राची᳚म् दे॒वा द॑क्षि॒णा । \newline
10. दे॒वा द॑क्षि॒णा द॑क्षि॒णा दे॒वा दे॒वा द॑क्षि॒णा पि॒तरः॑ पि॒तरो॑ दक्षि॒णा दे॒वा दे॒वा द॑क्षि॒णा पि॒तरः॑ । \newline
11. द॒क्षि॒णा पि॒तरः॑ पि॒तरो॑ दक्षि॒णा द॑क्षि॒णा पि॒तरः॑ प्र॒तीची᳚म् प्र॒तीची᳚म् पि॒तरो॑ दक्षि॒णा द॑क्षि॒णा पि॒तरः॑ प्र॒तीची᳚म् । \newline
12. पि॒तरः॑ प्र॒तीची᳚म् प्र॒तीची᳚म् पि॒तरः॑ पि॒तरः॑ प्र॒तीची᳚म् मनु॒ष्या॑ मनु॒ष्याः᳚ प्र॒तीची᳚म् पि॒तरः॑ पि॒तरः॑ प्र॒तीची᳚म् मनु॒ष्याः᳚ । \newline
13. प्र॒तीची᳚म् मनु॒ष्या॑ मनु॒ष्याः᳚ प्र॒तीची᳚म् प्र॒तीची᳚म् मनु॒ष्या॑ उदी॑ची॒ मुदी॑चीम् मनु॒ष्याः᳚ प्र॒तीची᳚म् प्र॒तीची᳚म् मनु॒ष्या॑ उदी॑चीम् । \newline
14. म॒नु॒ष्या॑ उदी॑ची॒ मुदी॑चीम् मनु॒ष्या॑ मनु॒ष्या॑ उदी॑चीꣳ रु॒द्रा रु॒द्रा उदी॑चीम् मनु॒ष्या॑ मनु॒ष्या॑ उदी॑चीꣳ रु॒द्राः । \newline
15. उदी॑चीꣳ रु॒द्रा रु॒द्रा उदी॑ची॒ मुदी॑चीꣳ रु॒द्रा यद् यद् रु॒द्रा उदी॑ची॒ मुदी॑चीꣳ रु॒द्रा यत् । \newline
16. रु॒द्रा यद् यद् रु॒द्रा रु॒द्रा यत् प्रा॒चीन॑वꣳशम् प्रा॒चीन॑वꣳशं॒ ॅयद् रु॒द्रा रु॒द्रा यत् प्रा॒चीन॑वꣳशम् । \newline
17. यत् प्रा॒चीन॑वꣳशम् प्रा॒चीन॑वꣳशं॒ ॅयद् यत् प्रा॒चीन॑वꣳशम् क॒रोति॑ क॒रोति॑ प्रा॒चीन॑वꣳशं॒ ॅयद् यत् प्रा॒चीन॑वꣳशम् क॒रोति॑ । \newline
18. प्रा॒चीन॑वꣳशम् क॒रोति॑ क॒रोति॑ प्रा॒चीन॑वꣳशम् प्रा॒चीन॑वꣳशम् क॒रोति॑ देवलो॒कम् दे॑वलो॒कम् क॒रोति॑ प्रा॒चीन॑वꣳशम् प्रा॒चीन॑वꣳशम् क॒रोति॑ देवलो॒कम् । \newline
19. प्रा॒चीन॑वꣳश॒मिति॑ प्रा॒चीन॑ - वꣳ॒॒श॒म् । \newline
20. क॒रोति॑ देवलो॒कम् दे॑वलो॒कम् क॒रोति॑ क॒रोति॑ देवलो॒क मे॒वैव दे॑वलो॒कम् क॒रोति॑ क॒रोति॑ देवलो॒क मे॒व । \newline
21. दे॒व॒लो॒क मे॒वैव दे॑वलो॒कम् दे॑वलो॒क मे॒व तत् तदे॒व दे॑वलो॒कम् दे॑वलो॒क मे॒व तत् । \newline
22. दे॒व॒लो॒कमिति॑ देव - लो॒कम् । \newline
23. ए॒व तत् तदे॒ वैव तद् यज॑मानो॒ यज॑मान॒ स्त दे॒वैव तद् यज॑मानः । \newline
24. तद् यज॑मानो॒ यज॑मान॒ स्तत् तद् यज॑मान उ॒पाव॑र्तत उ॒पाव॑र्तते॒ यज॑मान॒ स्तत् तद् यज॑मान उ॒पाव॑र्तते । \newline
25. यज॑मान उ॒पाव॑र्तत उ॒पाव॑र्तते॒ यज॑मानो॒ यज॑मान उ॒पाव॑र्तते॒ परि॒ पर्यु॒पा व॑र्तते॒ यज॑मानो॒ यज॑मान उ॒पाव॑र्तते॒ परि॑ । \newline
26. उ॒पाव॑र्तते॒ परि॒ पर्यु॒पाव॑र्तत उ॒पाव॑र्तते॒ परि॑ श्रयति श्रयति॒ पर्यु॒पाव॑र्तत उ॒पाव॑र्तते॒ परि॑ श्रयति । \newline
27. उ॒पाव॑र्तत॒ इत्यु॑प - आव॑र्तते । \newline
28. परि॑ श्रयति श्रयति॒ परि॒ परि॑ श्रय त्य॒न्तर्.हि॑तो॒ ऽन्तर्.हि॑तः श्रयति॒ परि॒ परि॑ श्रय त्य॒न्तर्.हि॑तः । \newline
29. श्र॒य॒ त्य॒न्तर्.हि॑तो॒ ऽन्तर्.हि॑तः श्रयति श्रय त्य॒न्तर्.हि॑तो॒ हि ह्य॑न्तर्.हि॑तः श्रयति श्रय त्य॒न्तर्.हि॑तो॒ हि । \newline
30. अ॒न्तर्.हि॑तो॒ हि ह्य॑न्तर्.हि॑तो॒ ऽन्तर्.हि॑तो॒ हि दे॑वलो॒को दे॑वलो॒को ह्य॑न्तर्.हि॑तो॒ ऽन्तर्.हि॑तो॒ हि दे॑वलो॒कः । \newline
31. अ॒न्तर्.हि॑त॒ इत्य॒न्तः - हि॒तः॒ । \newline
32. हि दे॑वलो॒को दे॑वलो॒को हि हि दे॑वलो॒को म॑नुष्यलो॒कान् म॑नुष्यलो॒काद् दे॑वलो॒को हि हि दे॑वलो॒को म॑नुष्यलो॒कात् । \newline
33. दे॒व॒लो॒को म॑नुष्यलो॒कान् म॑नुष्यलो॒काद् दे॑वलो॒को दे॑वलो॒को म॑नुष्यलो॒कान् न न म॑नुष्यलो॒काद् दे॑वलो॒को दे॑वलो॒को म॑नुष्यलो॒कान् न । \newline
34. दे॒व॒लो॒क इति॑ देव - लो॒कः । \newline
35. म॒नु॒ष्य॒लो॒कान् न न म॑नुष्यलो॒कान् म॑नुष्यलो॒कान् नास्मा द॒स्मान् न म॑नुष्यलो॒कान् म॑नुष्यलो॒कान् नास्मात् । \newline
36. म॒नु॒ष्य॒लो॒कादिति॑ मनुष्य - लो॒कात् । \newline
37. नास्मा द॒स्मान् न नास्माल् लो॒काल् लो॒का द॒स्मान् न नास्माल् लो॒कात् । \newline
38. अ॒स्माल् लो॒काल् लो॒का द॒स्मा द॒स्माल् लो॒काथ् स्वे॑तव्यꣳ॒॒ स्वे॑तव्यम् ॅलो॒का द॒स्मा द॒स्माल् लो॒काथ् स्वे॑तव्यम् । \newline
39. लो॒काथ् स्वे॑तव्यꣳ॒॒ स्वे॑तव्यम् ॅलो॒काल् लो॒काथ् स्वे॑तव्य मिवेव॒ स्वे॑तव्यम् ॅलो॒काल् लो॒काथ् स्वे॑तव्य मिव । \newline
40. स्वे॑तव्य मिवेव॒ स्वे॑तव्यꣳ॒॒ स्वे॑तव्य मि॒वे तीती॑व॒ स्वे॑तव्यꣳ॒॒ स्वे॑तव्य मि॒वेति॑ । \newline
41. स्वे॑तव्य॒मिति॒ सु - ए॒त॒व्य॒म् । \newline
42. इ॒वे तीती॑ वे॒वे त्या॑हु राहु॒ रिती॑वे॒वे त्या॑हुः । \newline
43. इत्या॑हु राहु॒ रिती त्या॑हुः॒ कः क आ॑हु॒ रिती त्या॑हुः॒ कः । \newline
44. आ॒हुः॒ कः क आ॑हु राहुः॒ को हि हि क आ॑हु राहुः॒ को हि । \newline
45. को हि हि कः को हि तत् तद्धि कः को हि तत् । \newline
46. हि तत् तद्धि हि तद् वेद॒ वेद॒ तद्धि हि तद् वेद॑ । \newline
47. तद् वेद॒ वेद॒ तत् तद् वेद॒ यदि॒ यदि॒ वेद॒ तत् तद् वेद॒ यदि॑ । \newline
48. वेद॒ यदि॒ यदि॒ वेद॒ वेद॒ यद्य॒ मुष्मि॑न् न॒मुष्मि॒न्॒. यदि॒ वेद॒ वेद॒ यद्य॒ मुष्मिन्न्॑ । \newline
49. यद्य॒ मुष्मि॑न् न॒मुष्मि॒न्॒. यदि॒ यद्य॒ मुष्मि॑न् ॅलो॒के लो॒के॑ ऽमुष्मि॒न्॒. यदि॒ यद्य॒-मुष्मि॑न् ॅलो॒के । \newline
50. अ॒मुष्मि॑न् ॅलो॒के लो॒के॑ ऽमुष्मि॑न् न॒मुष्मि॑न् ॅलो॒के ऽस्त्यस्ति॑ लो॒के॑ ऽमुष्मि॑न् न॒मुष्मि॑न् ॅलो॒के ऽस्ति॑ । \newline
51. लो॒के ऽस्त्यस्ति॑ लो॒के लो॒के ऽस्ति॑ वा॒ वा ऽस्ति॑ लो॒के लो॒के ऽस्ति॑ वा । \newline
52. अस्ति॑ वा॒ वा ऽस्त्यस्ति॑ वा॒ न न वा ऽस्त्यस्ति॑ वा॒ न । \newline
53. वा॒ न न वा॑ वा॒ न वा॑ वा॒ न वा॑ वा॒ न वा᳚ । \newline
54. न वा॑ वा॒ न न वेतीति॑ वा॒ न न वेति॑ । \newline
55. वेतीति॑ वा॒ वेति॑ दि॒क्षु दि॒क्ष्विति॑ वा॒ वेति॑ दि॒क्षु । \newline
56. इति॑ दि॒क्षु दि॒क्ष्वितीति॑ दि॒क्ष्व॑तीका॒शा न॑तीका॒शान् दि॒क्ष्वि तीति॑ दि॒क्ष्व॑तीका॒शान् । \newline
57. दि॒क्ष्व॑तीका॒शा न॑तीका॒शान् दि॒क्षु दि॒क्ष्व॑तीका॒शान् क॑रोति करो त्यतीका॒शान् दि॒क्षु दि॒क्ष्व॑तीका॒शान् क॑रोति । \newline
58. अ॒ती॒का॒शान् क॑रोति करो त्यतीका॒शा न॑तीका॒शान् क॑रो त्यु॒भयो॑ रु॒भयोः᳚ करो त्यतीका॒शा न॑तीका॒शान् क॑रो त्यु॒भयोः᳚ । \newline
59. क॒रो॒ त्यु॒भयो॑ रु॒भयोः᳚ करोति करो त्यु॒भयो᳚र् लो॒कयो᳚र् लो॒कयो॑ रु॒भयोः᳚ करोति करो त्यु॒भयो᳚र् लो॒कयोः᳚ । \newline
\pagebreak
\markright{ TS 6.1.1.2  \hfill https://www.vedavms.in \hfill}

\section{ TS 6.1.1.2 }

\textbf{TS 6.1.1.2 } \newline
\textbf{Samhita Paata} \newline

-त्यु॒भयो᳚र्लो॒कयो॑-र॒भिजि॑त्यै केशश्म॒श्रु व॑पते न॒खानि॒ नि कृ॑न्तते मृ॒ता वा ए॒षा त्वग॑मे॒द्ध्या यत् के॑शश्म॒श्रु मृ॒तामे॒व त्वच॑म-मे॒द्ध्याम॑प॒हत्य॑ य॒ज्ञियो॑ भू॒त्वा मेध॒मुपै॒त्यङ्गि॑रसः सुव॒र्गं ॅलो॒कं ॅयन्तो॒ऽफ्सु दी᳚क्षात॒पसी॒ प्रावे॑शयन्न॒फ्सु स्ना॑ति सा॒क्षादे॒व दी᳚क्षात॒पसी॒ अव॑ रुन्धे ती॒र्थे स्ना॑ति ती॒र्थे हि ते तां प्रावे॑शयन् ती॒र्थे स्ना॑ति - [  ] \newline

\textbf{Pada Paata} \newline

उ॒भयोः᳚ । लो॒कयोः᳚ । अ॒भिजि॑त्या॒ इत्य॒भि - जि॒त्यै॒ । के॒श॒श्म॒श्र्विति॑ केश-श्म॒श्रु । व॒प॒ते॒ । न॒खानि॑ । नीति॑ । कृ॒न्त॒ते॒ । मृ॒ता । वै । ए॒षा । त्वक् । अ॒मे॒द्ध्या । यत् । के॒श॒श्म॒श्र्विति॑ केश - श्म॒श्रु । मृ॒ताम् । ए॒व । त्वच᳚म् । अ॒मे॒द्ध्याम् । अ॒प॒हत्येत्य॑प-हत्य॑ । य॒ज्ञियः॑ । भू॒त्वा । मेध᳚म् । उपेति॑ । ए॒ति॒ । अङ्गि॑रसः । सु॒व॒र्गमिति॑ सुवः - गम् । लो॒कम् । यन्तः॑ । अ॒फ्स्वित्य॑प्-सु । दी॒क्षा॒त॒पसी॒ इति॑ दीक्षा-त॒पसी᳚ । प्रेति॑ । अ॒वे॒श॒य॒न्न् । अ॒फ्स्वित्य॑प्- सु । स्ना॒ति॒ । सा॒क्षादिति॑ स - अ॒क्षात् । ए॒व । दी॒क्षा॒त॒पसी॒ इति॑ दीक्षा - त॒पसी᳚ । अवेति॑ । रु॒न्धे॒ । ती॒र्थे । स्ना॒ति॒ । ती॒र्थे । हि । ते । ताम् । प्रेति॑ । अवे॑शयन्न् । ती॒र्थे । स्ना॒ति॒ ।  \newline


\textbf{Krama Paata} \newline

उ॒भयो᳚र् लो॒कयोः᳚ । लो॒कयो॑र॒भिजि॑त्यै । अ॒भिजि॑त्यै केशश्म॒श्रु । अ॒भिजि॑त्या॒ इत्य॒भि - जि॒त्यै॒ । के॒श॒श्म॒श्रु व॑पते । के॒श॒श्म॒श्र्विति॑ केश - श्म॒श्रु । व॒प॒ते॒ न॒खानि॑ । न॒खानि॒ नि । नि कृ॑न्तते । कृ॒न्त॒ते॒ मृ॒ता । मृ॒ता वै । वा ए॒षा । ए॒षा त्वक् । त्वग॑मे॒द्ध्या । अ॒मे॒द्ध्या यत् । यत् के॑शश्म॒श्रु । के॒श॒श्म॒श्रु मृ॒ताम् । के॒श॒श्म॒श्र्विति॑ केश - श्म॒श्रु । मृ॒तामे॒व । ए॒व त्वच᳚म् । त्वच॑ममे॒द्ध्याम् । अ॒मे॒द्ध्याम॑प॒हत्य॑ । अ॒प॒हत्य॑ य॒ज्ञियः॑ । अ॒प॒हत्येत्य॑प - हत्य॑ । य॒ज्ञियो॑ भू॒त्वा । भू॒त्वा मेद्ध᳚म् । मेद्ध॒मुप॑ । उपै॑ति । ए॒त्यङ्‍गि॑रसः । अङ्‍गि॑रसः सुव॒र्गम् । सु॒व॒र्गम् ॅलो॒कम् । सु॒व॒र्गमिति॑ सुवः - गम् । लो॒कम् ॅयन्तः॑ । यन्तो॒ऽफ्सु । अ॒फ्सु दी᳚क्षात॒पसी᳚ । अ॒फ्स्वित्य॑प् - सु । दी॒क्षा॒त॒पसी॒ प्र । दी॒क्षा॒त॒पसी॒ इति॑ दीक्षा - त॒पसी᳚ । प्रावे॑शयन्न् । अ॒वे॒श॒य॒न्न॒फ्सु । अ॒फ्सु स्ना॑ति । अ॒फ्स्वित्य॑प् - सु । स्ना॒ति॒ सा॒क्षात् । सा॒क्षादे॒व । सा॒क्षादिति॑ स - अ॒क्षात् । ए॒व दी᳚क्षात॒पसी᳚ । दी॒क्षा॒त॒पसी॒ अव॑ । दी॒क्षा॒त॒पसी॒ इति॑ दीक्षा - त॒पसी᳚ । अव॑ रुन्धे । रु॒न्धे॒ ती॒र्थे । ती॒र्थे स्ना॑ति । स्ना॒ति॒ ती॒र्थे । ती॒र्थे हि । हि ते । ते ताम् । ताम् प्र । प्रावे॑शयन्न् । अवे॑शयन् ती॒र्थे । ती॒र्थे स्ना॑ति । स्ना॒ति॒ ती॒र्थम् \newline

\textbf{Jatai Paata} \newline

1. उ॒भयो᳚र् लो॒कयो᳚र् लो॒कयो॑ रु॒भयो॑ रु॒भयो᳚र् लो॒कयोः᳚ । \newline
2. लो॒कयो॑ र॒भिजि॑त्या अ॒भिजि॑त्यै लो॒कयो᳚र् लो॒कयो॑ र॒भिजि॑त्यै । \newline
3. अ॒भिजि॑त्यै केशश्म॒श्रु के॑शश्म॒ श्र्व॑भिजि॑त्या अ॒भिजि॑त्यै केशश्म॒श्रु । \newline
4. अ॒भिजि॑त्या॒ इत्य॒भि - जि॒त्यै॒ । \newline
5. के॒श॒श्म॒श्रु व॑पते वपते केशश्म॒श्रु के॑शश्म॒श्रु व॑पते । \newline
6. के॒श॒श्म॒श्र्विति॑ केश - श्म॒श्रु । \newline
7. व॒प॒ते॒ न॒खानि॑ न॒खानि॑ वपते वपते न॒खानि॑ । \newline
8. न॒खानि॒ नि नि न॒खानि॑ न॒खानि॒ नि । \newline
9. नि कृ॑न्तते कृन्तते॒ नि नि कृ॑न्तते । \newline
10. कृ॒न्त॒ते॒ मृ॒ता मृ॒ता कृ॑न्तते कृन्तते मृ॒ता । \newline
11. मृ॒ता वै वै मृ॒ता मृ॒ता वै । \newline
12. वा ए॒षैषा वै वा ए॒षा । \newline
13. ए॒षा त्वक् त्वगे॒ षैषा त्वक् । \newline
14. त्व ग॑मे॒द्ध्या ऽमे॒द्ध्या त्वक् त्व ग॑मे॒द्ध्या । \newline
15. अ॒मे॒द्ध्या यद् यद॑मे॒द्ध्या ऽमे॒द्ध्या यत् । \newline
16. यत् के॑शश्म॒श्रु के॑शश्म॒श्रु यद् यत् के॑शश्म॒श्रु । \newline
17. के॒श॒श्म॒श्रु मृ॒ताम् मृ॒ताम् के॑शश्म॒श्रु के॑शश्म॒श्रु मृ॒ताम् । \newline
18. के॒श॒श्म॒श्र्विति॑ केश - श्म॒श्रु । \newline
19. मृ॒ता मे॒वैव मृ॒ताम् मृ॒ता मे॒व । \newline
20. ए॒व त्वच॒म् त्वच॑ मे॒वैव त्वच᳚म् । \newline
21. त्वच॑ ममे॒द्ध्या म॑मे॒द्ध्याम् त्वच॒म् त्वच॑ ममे॒द्ध्याम् । \newline
22. अ॒मे॒द्ध्या म॑प॒हत्या॑ प॒हत्या॑ मे॒द्ध्या म॑मे॒द्ध्या म॑प॒हत्य॑ । \newline
23. अ॒प॒हत्य॑ य॒ज्ञियो॑ य॒ज्ञियो॑ ऽप॒हत्या॑ प॒हत्य॑ य॒ज्ञियः॑ । \newline
24. अ॒प॒हत्येत्य॑प - हत्य॑ । \newline
25. य॒ज्ञियो॑ भू॒त्वा भू॒त्वा य॒ज्ञियो॑ य॒ज्ञियो॑ भू॒त्वा । \newline
26. भू॒त्वा मेध॒म् मेध॑म् भू॒त्वा भू॒त्वा मेध᳚म् । \newline
27. मेध॒ मुपोप॒ मेध॒म् मेध॒ मुप॑ । \newline
28. उपै᳚त्ये॒ त्युपो पै॑ति । \newline
29. ए॒त्यङ्गि॑र॒सो ऽङ्गि॑रस एत्ये॒ त्यङ्गि॑रसः । \newline
30. अङ्गि॑रसः सुव॒र्गꣳ सु॑व॒र्ग मङ्गि॑र॒सो ऽङ्गि॑रसः सुव॒र्गम् । \newline
31. सु॒व॒र्गम् ॅलो॒कम् ॅलो॒कꣳ सु॑व॒र्गꣳ सु॑व॒र्गम् ॅलो॒कम् । \newline
32. सु॒व॒र्गमिति॑ सुवः - गम् । \newline
33. लो॒कं ॅयन्तो॒ यन्तो॑ लो॒कम् ॅलो॒कं ॅयन्तः॑ । \newline
34. यन्तो॒ ऽफ्स्व॑फ्सु यन्तो॒ यन्तो॒ ऽफ्सु । \newline
35. अ॒फ्सु दी᳚क्षात॒पसी॑ दीक्षात॒पसी॑ अ॒फ्स्व॑फ्सु दी᳚क्षात॒पसी᳚ । \newline
36. अ॒फ्स्वित्य॑प् - सु । \newline
37. दी॒क्षा॒त॒पसी॒ प्र प्र दी᳚क्षात॒पसी॑ दीक्षात॒पसी॒ प्र । \newline
38. दी॒क्षा॒त॒पसी॒ इति॑ दीक्षा - त॒पसी᳚ । \newline
39. प्रावे॑शयन् नवेशय॒न् प्र प्रावे॑शयन्न् । \newline
40. अ॒वे॒श॒य॒न् न॒फ्स्वा᳚(1॒)फ्स्व॑वेशयन् नवेशयन् न॒फ्सु । \newline
41. अ॒फ्सु स्ना॑ति स्ना त्य॒फ्स्व॑फ्सु स्ना॑ति । \newline
42. अ॒फ्स्वित्य॑प् - सु । \newline
43. स्ना॒ति॒ सा॒क्षाथ् सा॒क्षाथ् स्ना॑ति स्नाति सा॒क्षात् । \newline
44. सा॒क्षा दे॒वैव सा॒क्षाथ् सा॒क्षा दे॒व । \newline
45. सा॒क्षादिति॑ स - अ॒क्षात् । \newline
46. ए॒व दी᳚क्षात॒पसी॑ दीक्षात॒पसी॑ ए॒वैव दी᳚क्षात॒पसी᳚ । \newline
47. दी॒क्षा॒त॒पसी॒ अवाव॑ दीक्षात॒पसी॑ दीक्षात॒पसी॒ अव॑ । \newline
48. दी॒क्षा॒त॒पसी॒ इति॑ दीक्षा - त॒पसी᳚ । \newline
49. अव॑ रुन्धे रु॒न्धे ऽवाव॑ रुन्धे । \newline
50. रु॒न्धे॒ ती॒र्थे ती॒र्थे रु॑न्धे रुन्धे ती॒र्थे । \newline
51. ती॒र्थे स्ना॑ति स्नाति ती॒र्थे ती॒र्थे स्ना॑ति । \newline
52. स्ना॒ति॒ ती॒र्थे ती॒र्थे स्ना॑ति स्नाति ती॒र्थे । \newline
53. ती॒र्थे हि हि ती॒र्थे ती॒र्थे हि । \newline
54. हि ते ते हि हि ते । \newline
55. ते ताम् ताम् ते ते ताम् । \newline
56. ताम् प्र प्र ताम् ताम् प्र । \newline
57. प्रावे॑शय॒न् नवे॑शय॒न् प्र प्रावे॑शयन्न् । \newline
58. अवे॑शयन् ती॒र्थे ती॒र्थे ऽवे॑शय॒न् नवे॑शयन् ती॒र्थे । \newline
59. ती॒र्थे स्ना॑ति स्नाति ती॒र्थे ती॒र्थे स्ना॑ति । \newline
60. स्ना॒ति॒ ती॒र्थम् ती॒र्थꣳ स्ना॑ति स्नाति ती॒र्थम् । \newline

\textbf{Ghana Paata } \newline

1. उ॒भयो᳚र् लो॒कयो᳚र् लो॒कयो॑ रु॒भयो॑ रु॒भयो᳚र् लो॒कयो॑ र॒भिजि॑त्या अ॒भिजि॑त्यै लो॒कयो॑ रु॒भयो॑ रु॒भयो᳚र् लो॒कयो॑ र॒भिजि॑त्यै । \newline
2. लो॒कयो॑ र॒भिजि॑त्या अ॒भिजि॑त्यै लो॒कयो᳚र् लो॒कयो॑ र॒भिजि॑त्यै केशश्म॒श्रु के॑शश्म॒ श्र्व॑भिजि॑त्यै लो॒कयो᳚र् लो॒कयो॑ र॒भिजि॑त्यै केशश्म॒श्रु । \newline
3. अ॒भिजि॑त्यै केशश्म॒श्रु के॑शश्म॒ श्र्व॑भिजि॑त्या अ॒भिजि॑त्यै केशश्म॒श्रु व॑पते वपते केशश्म॒ श्र्व॑भिजि॑त्या अ॒भिजि॑त्यै केशश्म॒श्रु व॑पते । \newline
4. अ॒भिजि॑त्या॒ इत्य॒भि - जि॒त्यै॒ । \newline
5. के॒श॒श्म॒श्रु व॑पते वपते केशश्म॒श्रु के॑शश्म॒श्रु व॑पते न॒खानि॑ न॒खानि॑ वपते केशश्म॒श्रु के॑शश्म॒श्रु व॑पते न॒खानि॑ । \newline
6. के॒श॒श्म॒श्र्विति॑ केश - श्म॒श्रु । \newline
7. व॒प॒ते॒ न॒खानि॑ न॒खानि॑ वपते वपते न॒खानि॒ नि नि न॒खानि॑ वपते वपते न॒खानि॒ नि । \newline
8. न॒खानि॒ नि नि न॒खानि॑ न॒खानि॒ नि कृ॑न्तते कृन्तते॒ नि न॒खानि॑ न॒खानि॒ नि कृ॑न्तते । \newline
9. नि कृ॑न्तते कृन्तते॒ नि नि कृ॑न्तते मृ॒ता मृ॒ता कृ॑न्तते॒ नि नि कृ॑न्तते मृ॒ता । \newline
10. कृ॒न्त॒ते॒ मृ॒ता मृ॒ता कृ॑न्तते कृन्तते मृ॒ता वै वै मृ॒ता कृ॑न्तते कृन्तते मृ॒ता वै । \newline
11. मृ॒ता वै वै मृ॒ता मृ॒ता वा ए॒षैषा वै मृ॒ता मृ॒ता वा ए॒षा । \newline
12. वा ए॒षैषा वै वा ए॒षा त्वक् त्वगे॒षा वै वा ए॒षा त्वक् । \newline
13. ए॒षा त्वक् त्वगे॒ षैषा त्वग॑ मे॒द्ध्या ऽमे॒द्ध्या त्वगे॒ षैषा त्वग॑ मे॒द्ध्या । \newline
14. त्वग॑ मे॒द्ध्या ऽमे॒द्ध्या त्वक् त्वग॑ मे॒द्ध्या यद् यद॑ मे॒द्ध्या त्वक् त्वग॑ मे॒द्ध्या यत् । \newline
15. अ॒मे॒द्ध्या यद् यद॑ मे॒द्ध्या ऽमे॒द्ध्या यत् के॑शश्म॒श्रु के॑शश्म॒श्रु यद॑ मे॒द्ध्या ऽमे॒द्ध्या यत् के॑शश्म॒श्रु । \newline
16. यत् के॑शश्म॒श्रु के॑शश्म॒श्रु यद् यत् के॑शश्म॒श्रु मृ॒ताम् मृ॒ताम् के॑शश्म॒श्रु यद् यत् के॑शश्म॒श्रु मृ॒ताम् । \newline
17. के॒श॒श्म॒श्रु मृ॒ताम् मृ॒ताम् के॑शश्म॒श्रु के॑शश्म॒श्रु मृ॒ता मे॒वैव मृ॒ताम् के॑शश्म॒श्रु के॑शश्म॒श्रु मृ॒ता मे॒व । \newline
18. के॒श॒श्म॒श्र्विति॑ केश - श्म॒श्रु । \newline
19. मृ॒ता मे॒वैव मृ॒ताम् मृ॒ता मे॒व त्वच॒म् त्वच॑ मे॒व मृ॒ताम् मृ॒ता मे॒व त्वच᳚म् । \newline
20. ए॒व त्वच॒म् त्वच॑ मे॒वैव त्वच॑ ममे॒द्ध्या म॑मे॒द्ध्याम् त्वच॑ मे॒वैव त्वच॑ ममे॒द्ध्याम् । \newline
21. त्वच॑ ममे॒द्ध्या म॑मे॒द्ध्याम् त्वच॒म् त्वच॑ ममे॒द्ध्या म॑प॒हत्या॑ प॒हत्या॑ मे॒द्ध्याम् त्वच॒म् त्वच॑ ममे॒द्ध्या म॑प॒हत्य॑ । \newline
22. अ॒मे॒द्ध्या म॑प॒हत्या॑ प॒हत्या॑ मे॒द्ध्या म॑मे॒द्ध्या म॑प॒हत्य॑ य॒ज्ञियो॑ य॒ज्ञियो॑ ऽप॒हत्या॑ मे॒द्ध्या म॑मे॒द्ध्या म॑प॒हत्य॑ य॒ज्ञियः॑ । \newline
23. अ॒प॒हत्य॑ य॒ज्ञियो॑ य॒ज्ञियो॑ ऽप॒हत्या॑ प॒हत्य॑ य॒ज्ञियो॑ भू॒त्वा भू॒त्वा य॒ज्ञियो॑ ऽप॒हत्या॑ प॒हत्य॑ य॒ज्ञियो॑ भू॒त्वा । \newline
24. अ॒प॒हत्येत्य॑प - हत्य॑ । \newline
25. य॒ज्ञियो॑ भू॒त्वा भू॒त्वा य॒ज्ञियो॑ य॒ज्ञियो॑ भू॒त्वा मेध॒म् मेध॑म् भू॒त्वा य॒ज्ञियो॑ य॒ज्ञियो॑ भू॒त्वा मेध᳚म् । \newline
26. भू॒त्वा मेध॒म् मेध॑म् भू॒त्वा भू॒त्वा मेध॒ मुपोप॒ मेध॑म् भू॒त्वा भू॒त्वा मेध॒ मुप॑ । \newline
27. मेध॒ मुपोप॒ मेध॒म् मेध॒ मुपै᳚ त्ये॒त्युप॒ मेध॒म् मेध॒ मुपै॑ति । \newline
28. उपै᳚त्ये॒ त्युपो पै॒त्यङ्गि॑र॒सो ऽङ्गि॑रस ए॒त्युपो पै॒त्यङ्गि॑रसः । \newline
29. ए॒त्यङ्गि॑र॒सो ऽङ्गि॑रस एत्ये॒त्यङ्गि॑रसः सुव॒र्गꣳ सु॑व॒र्ग मङ्गि॑रस एत्ये॒त्यङ्गि॑रसः सुव॒र्गम् । \newline
30. अङ्गि॑रसः सुव॒र्गꣳ सु॑व॒र्ग मङ्गि॑र॒सो ऽङ्गि॑रसः सुव॒र्गम् ॅलो॒कम् ॅलो॒कꣳ सु॑व॒र्ग मङ्गि॑र॒सो ऽङ्गि॑रसः सुव॒र्गम् ॅलो॒कम् । \newline
31. सु॒व॒र्गम् ॅलो॒कम् ॅलो॒कꣳ सु॑व॒र्गꣳ सु॑व॒र्गम् ॅलो॒कं ॅयन्तो॒ यन्तो॑ लो॒कꣳ सु॑व॒र्गꣳ सु॑व॒र्गम् ॅलो॒कं ॅयन्तः॑ । \newline
32. सु॒व॒र्गमिति॑ सुवः - गम् । \newline
33. लो॒कं ॅयन्तो॒ यन्तो॑ लो॒कम् ॅलो॒कं ॅयन्तो॒ ऽफ्स्व॑फ्सु यन्तो॑ लो॒कम् ॅलो॒कं ॅयन्तो॒ ऽफ्सु । \newline
34. यन्तो॒ ऽफ्स्व॑फ्सु यन्तो॒ यन्तो॒ ऽफ्सु दी᳚क्षात॒पसी॑ दीक्षात॒पसी॑ अ॒फ्सु यन्तो॒ यन्तो॒ ऽफ्सु दी᳚क्षात॒पसी᳚ । \newline
35. अ॒फ्सु दी᳚क्षात॒पसी॑ दीक्षात॒पसी॑ अ॒फ्स्व॑फ्सु दी᳚क्षात॒पसी॒ प्र प्र दी᳚क्षात॒पसी॑ अ॒फ्स्व॑फ्सु दी᳚क्षात॒पसी॒ प्र । \newline
36. अ॒फ्स्वित्य॑प् - सु । \newline
37. दी॒क्षा॒त॒पसी॒ प्र प्र दी᳚क्षात॒पसी॑ दीक्षात॒पसी॒ प्रा वे॑शयन् नवेशय॒न् प्र दी᳚क्षात॒पसी॑ दीक्षात॒पसी॒ प्रावे॑शयन्न् । \newline
38. दी॒क्षा॒त॒पसी॒ इति॑ दीक्षा - त॒पसी᳚ । \newline
39. प्रावे॑शयन् नवेशय॒न् प्र प्रावे॑शयन् न॒फ्स्वा᳚(1॒) फ्स्व॑वेशय॒न् प्र प्रावे॑शयन् न॒फ्सु । \newline
40. अ॒वे॒श॒य॒न् न॒फ्स्वा᳚(1॒) फ्स्व॑वेशयन्  नवेशयन् न॒फ्सु स्ना॑ति स्ना त्य॒फ्स्व॑वेशयन् नवेशयन् न॒फ्सु स्ना॑ति । \newline
41. अ॒फ्सु स्ना॑ति स्नात्य॒ फ्स्व॑फ्सु स्ना॑ति सा॒क्षाथ् सा॒क्षाथ् स्ना᳚ त्य॒फ्स्व॑फ्सु स्ना॑ति सा॒क्षात् । \newline
42. अ॒फ्स्वित्य॑प् - सु । \newline
43. स्ना॒ति॒ सा॒क्षाथ् सा॒क्षाथ् स्ना॑ति स्नाति सा॒क्षा दे॒वैव सा॒क्षाथ् स्ना॑ति स्नाति सा॒क्षा दे॒व । \newline
44. सा॒क्षा दे॒वैव सा॒क्षाथ् सा॒क्षा दे॒व दी᳚क्षात॒पसी॑ दीक्षात॒पसी॑ ए॒व सा॒क्षाथ् सा॒क्षादे॒व दी᳚क्षात॒पसी᳚ । \newline
45. सा॒क्षादिति॑ स - अ॒क्षात् । \newline
46. ए॒व दी᳚क्षात॒पसी॑ दीक्षात॒पसी॑ ए॒वैव दी᳚क्षात॒पसी॒ अवाव॑ दीक्षात॒पसी॑ ए॒वैव दी᳚क्षात॒पसी॒ अव॑ । \newline
47. दी॒क्षा॒त॒पसी॒ अवाव॑ दीक्षात॒पसी॑ दीक्षात॒पसी॒ अव॑ रुन्धे रु॒न्धे ऽव॑ दीक्षात॒पसी॑ दीक्षात॒पसी॒ अव॑ रुन्धे । \newline
48. दी॒क्षा॒त॒पसी॒ इति॑ दीक्षा - त॒पसी᳚ । \newline
49. अव॑ रुन्धे रु॒न्धे ऽवाव॑ रुन्धे ती॒र्थे ती॒र्थे रु॒न्धे ऽवाव॑ रुन्धे ती॒र्थे । \newline
50. रु॒न्धे॒ ती॒र्थे ती॒र्थे रु॑न्धे रुन्धे ती॒र्थे स्ना॑ति स्नाति ती॒र्थे रु॑न्धे रुन्धे ती॒र्थे स्ना॑ति । \newline
51. ती॒र्थे स्ना॑ति स्नाति ती॒र्थे ती॒र्थे स्ना॑ति ती॒र्थे ती॒र्थे स्ना॑ति ती॒र्थे ती॒र्थे स्ना॑ति ती॒र्थे । \newline
52. स्ना॒ति॒ ती॒र्थे ती॒र्थे स्ना॑ति स्नाति ती॒र्थे हि हि ती॒र्थे स्ना॑ति स्नाति ती॒र्थे हि । \newline
53. ती॒र्थे हि हि ती॒र्थे ती॒र्थे हि ते ते हि ती॒र्थे ती॒र्थे हि ते । \newline
54. हि ते ते हि हि ते ताम् ताम् ते हि हि ते ताम् । \newline
55. ते ताम् ताम् ते ते ताम् प्र प्र ताम् ते ते ताम् प्र । \newline
56. ताम् प्र प्र ताम् ताम् प्रावे॑शय॒न् नवे॑शय॒न् प्र ताम् ताम् प्रावे॑शयन्न् । \newline
57. प्रावे॑शय॒न् नवे॑शय॒न् प्र प्रावे॑शयन् ती॒र्थे ती॒र्थे ऽवे॑शय॒न् प्र प्रावे॑शयन् ती॒र्थे । \newline
58. अवे॑शयन् ती॒र्थे ती॒र्थे ऽवे॑शय॒न् नवे॑शयन् ती॒र्थे स्ना॑ति स्नाति ती॒र्थे ऽवे॑शय॒न् नवे॑शयन् ती॒र्थे स्ना॑ति । \newline
59. ती॒र्थे स्ना॑ति स्नाति ती॒र्थे ती॒र्थे स्ना॑ति ती॒र्थम् ती॒र्थꣳ स्ना॑ति ती॒र्थे ती॒र्थे स्ना॑ति ती॒र्थम् । \newline
60. स्ना॒ति॒ ती॒र्थम् ती॒र्थꣳ स्ना॑ति स्नाति ती॒र्थ मे॒वैव ती॒र्थꣳ स्ना॑ति स्नाति ती॒र्थ मे॒व । \newline
\pagebreak
\markright{ TS 6.1.1.3  \hfill https://www.vedavms.in \hfill}

\section{ TS 6.1.1.3 }

\textbf{TS 6.1.1.3 } \newline
\textbf{Samhita Paata} \newline

ती॒र्थमे॒व स॑मा॒नानां᳚ भवत्य॒पो᳚ऽश्नात्यन्तर॒त ए॒व मेद्ध्यो॑ भवति॒ वास॑सा दीक्षयति सौ॒म्यं ॅवै क्षौमं॑ दे॒वत॑या॒ सोम॑मे॒ष दे॒वता॒मुपै॑ति॒ यो दीक्ष॑ते॒ सोम॑स्य त॒नूर॑सि त॒नुवं॑ मे पा॒हीत्या॑ह॒ स्वामे॒व दे॒वता॒मुपै॒त्यथो॑ आ॒शिष॑मे॒वैतामा शा᳚स्ते॒ ऽग्नेस्तू॑षा॒धानं॑ ॅवा॒योर्वा॑त॒पानं॑ पितृ॒णां नी॒विरोष॑धीनां प्रघा॒त - [  ] \newline

\textbf{Pada Paata} \newline

ती॒र्थम् । ए॒व । स॒मा॒नाना᳚म् । भ॒व॒ति॒ । अ॒पः । अ॒श्ना॒ति॒ । अ॒न्त॒र॒तः । ए॒व । मेद्ध्यः॑ । भ॒व॒ति॒ । वास॑सा । दी॒क्ष॒य॒ति॒ । सौ॒म्यम् । वै । क्षौम᳚म् । दे॒वत॑या । सोम᳚म् । ए॒षः । दे॒वता᳚म् । उपेति॑ । ए॒ति॒ । यः । दीक्ष॑ते । सोम॑स्य । त॒नूः । अ॒सि॒ । त॒नुव᳚म् । मे॒ । पा॒हि॒ । इति॑ । आ॒ह॒ । स्वाम् । ए॒व । दे॒वता᳚म् । उपेति॑ । ए॒ति॒ । अथो॒ इति॑ । आ॒शिष॒मित्या᳚ - शिष᳚म् । ए॒व । ए॒ताम् । एति॑ । शा॒स्ते॒ । अ॒ग्नेः । तू॒षा॒धान॒मिति॑ तूष - आ॒धान᳚म् । वा॒योः । वा॒त॒पान॒मिति॑ वात-पान᳚म् । पि॒तृ॒णाम् । नी॒विः । ओष॑धीनाम् । प्र॒घा॒त इति॑ प्र - घा॒तः ।  \newline


\textbf{Krama Paata} \newline

ती॒र्थमे॒व । ए॒व स॑मा॒नाना᳚म् । स॒मा॒नाना᳚म् भवति । भ॒व॒त्य॒पः । अ॒पो᳚ऽश्ञाति । अ॒श्ञा॒त्य॒न्त॒र॒तः । अ॒न्त॒र॒त ए॒व । ए॒व मेद्ध्यः॑ । मेद्ध्यो॑ भवति । भ॒व॒ति॒ वास॑सा । वास॑सा दीक्षयति । दी॒क्ष॒य॒ति॒ सौ॒म्यम् । सौ॒म्यम् ॅवै । वै क्षौम᳚म् । क्षौम॑म् दे॒वत॑या । दे॒वत॑या॒ सोम᳚म् । सोम॑मे॒षः । ए॒ष दे॒वता᳚म् । दे॒वता॒मुप॑ । उपै॑ति । ए॒ति॒ यः । यो दीक्ष॑ते । दीक्ष॑ते॒ सोम॑स्य । सोम॑स्य त॒नूः । त॒नूर॑सि । अ॒सि॒ त॒नुव᳚म् । त॒नुव॑म् मे । मे॒ पा॒हि॒ । पा॒हीति॑ । इत्या॑ह । आ॒ह॒ स्वाम् । स्वामे॒व । ए॒व दे॒वता᳚म् । दे॒वता॒मुप॑ । उपै॑ति । ए॒त्यथो᳚ । अथो॑ आ॒शिष᳚म् । अथो॒ इत्यथो᳚ । आ॒शिष॑मे॒व । आ॒शिष॒मित्या᳚ - शिष᳚म् । ए॒वैताम् । ए॒तामा । आ शा᳚स्ते । शा॒स्ते॒ऽग्नेः । अ॒ग्नेस्तू॑षा॒द्धान᳚म् । तू॒षा॒द्धान॑म् ॅवा॒योः । तू॒षा॒द्धान॒मिति॑ तूष - आ॒द्धान᳚म् । वा॒योर् वा॑त॒पान᳚म् । वा॒त॒पान॑म् पितृ॒णाम् । वा॒त॒पान॒मिति॑ वात - पान᳚म् । पि॒तृ॒णाम् नी॒विः । नी॒विरोष॑धीनाम् । ओष॑धीनाम् प्रघा॒तः । प्र॒घा॒त आ॑दि॒त्याना᳚म् । प्र॒घा॒त इति॑ प्र - घा॒तः \newline

\textbf{Jatai Paata} \newline

1. ती॒र्थ मे॒वैव ती॒र्थम् ती॒र्थ मे॒व । \newline
2. ए॒व स॑मा॒नानाꣳ॑ समा॒नाना॑ मे॒वैव स॑मा॒नाना᳚म् । \newline
3. स॒मा॒नाना᳚म् भवति भवति समा॒नानाꣳ॑ समा॒नाना᳚म् भवति । \newline
4. भ॒व॒ त्य॒पो॑ ऽपो भ॑वति भव त्य॒पः । \newline
5. अ॒पो᳚ ऽश्ञा त्यश्ञा त्य॒पो᳚(1॒) ऽपो᳚ ऽश्ञाति । \newline
6. अ॒श्ञा॒ त्य॒न्त॒र॒तो᳚ ऽन्तर॒तो᳚ ऽश्ञा त्यश्ञा त्यन्तर॒तः । \newline
7. अ॒न्त॒र॒त ए॒वै वान्त॑र॒तो᳚ ऽन्तर॒त ए॒व । \newline
8. ए॒व मेद्ध्यो॒ मेद्ध्य॑ ए॒वैव मेद्ध्यः॑ । \newline
9. मेद्ध्यो॑ भवति भवति॒ मेद्ध्यो॒ मेद्ध्यो॑ भवति । \newline
10. भ॒व॒ति॒ वास॑सा॒ वास॑सा भवति भवति॒ वास॑सा । \newline
11. वास॑सा दीक्षयति दीक्षयति॒ वास॑सा॒ वास॑सा दीक्षयति । \newline
12. दी॒क्ष॒य॒ति॒ सौ॒म्यꣳ सौ॒म्यम् दी᳚क्षयति दीक्षयति सौ॒म्यम् । \newline
13. सौ॒म्यं ॅवै वै सौ॒म्यꣳ सौ॒म्यं ॅवै । \newline
14. वै क्षौम॒म् क्षौमं॒ ॅवै वै क्षौम᳚म् । \newline
15. क्षौम॑म् दे॒वत॑या दे॒वत॑या॒ क्षौम॒म् क्षौम॑म् दे॒वत॑या । \newline
16. दे॒वत॑या॒ सोमꣳ॒॒ सोम॑म् दे॒वत॑या दे॒वत॑या॒ सोम᳚म् । \newline
17. सोम॑ मे॒ष ए॒ष सोमꣳ॒॒ सोम॑ मे॒षः । \newline
18. ए॒ष दे॒वता᳚म् दे॒वता॑ मे॒ष ए॒ष दे॒वता᳚म् । \newline
19. दे॒वता॒ मुपोप॑ दे॒वता᳚म् दे॒वता॒ मुप॑ । \newline
20. उपै᳚त्ये॒ त्युपो पै॑ति । \newline
21. ए॒ति॒ यो य ए᳚त्येति॒ यः । \newline
22. यो दीक्ष॑ते॒ दीक्ष॑ते॒ यो यो दीक्ष॑ते । \newline
23. दीक्ष॑ते॒ सोम॑स्य॒ सोम॑स्य॒ दीक्ष॑ते॒ दीक्ष॑ते॒ सोम॑स्य । \newline
24. सोम॑स्य त॒नू स्त॒नूः सोम॑स्य॒ सोम॑स्य त॒नूः । \newline
25. त॒नू र॑स्यसि त॒नू स्त॒नू र॑सि । \newline
26. अ॒सि॒ त॒नुव॑म् त॒नुव॑ मस्यसि त॒नुव᳚म् । \newline
27. त॒नुव॑म् मे मे त॒नुव॑म् त॒नुव॑म् मे । \newline
28. मे॒ पा॒हि॒ पा॒हि॒ मे॒ मे॒ पा॒हि॒ । \newline
29. पा॒ही तीति॑ पाहि पा॒हीति॑ । \newline
30. इत्या॑हा॒हे तीत्या॑ह । \newline
31. आ॒ह॒ स्वाꣳ स्वा मा॑हाह॒ स्वाम् । \newline
32. स्वा मे॒वैव स्वाꣳ स्वा मे॒व । \newline
33. ए॒व दे॒वता᳚म् दे॒वता॑ मे॒वैव दे॒वता᳚म् । \newline
34. दे॒वता॒ मुपोप॑ दे॒वता᳚म् दे॒वता॒ मुप॑ । \newline
35. उपै᳚त्ये॒ त्युपो पै॑ति । \newline
36. ए॒त्यथो॒ अथो॑ एत्ये॒ त्यथो᳚ । \newline
37. अथो॑ आ॒शिष॑ मा॒शिष॒ मथो॒ अथो॑ आ॒शिष᳚म् । \newline
38. अथो॒ इत्यथो᳚ । \newline
39. आ॒शिष॑ मे॒वै वाशिष॑ मा॒शिष॑ मे॒व । \newline
40. आ॒शिष॒मित्या᳚ - शिष᳚म् । \newline
41. ए॒वैता मे॒ता मे॒वै वैताम् । \newline
42. ए॒ता मैता मे॒ता मा । \newline
43. आ शा᳚स्ते शास्त॒ आ शा᳚स्ते । \newline
44. शा॒स्ते॒ ऽग्ने र॒ग्नेः शा᳚स्ते शास्ते॒ ऽग्नेः । \newline
45. अ॒ग्ने स्तू॑षा॒धान॑म् तूषा॒धान॑ म॒ग्ने र॒ग्ने स्तू॑षा॒धान᳚म् । \newline
46. तू॒षा॒धानं॑ ॅवा॒योर् वा॒यो स्तू॑षा॒धान॑म् तूषा॒धानं॑ ॅवा॒योः । \newline
47. तू॒षा॒धान॒मिति॑ तूष - आ॒धान᳚म् । \newline
48. वा॒योर् वा॑त॒पानं॑ ॅवात॒पानं॑ ॅवा॒योर् वा॒योर् वा॑त॒पान᳚म् । \newline
49. वा॒त॒पान॑म् पितृ॒णाम् पि॑तृ॒णां ॅवा॑त॒पानं॑ ॅवात॒पान॑म् पितृ॒णाम् । \newline
50. वा॒त॒पान॒मिति॑ वात - पान᳚म् । \newline
51. पि॒तृ॒णान् नी॒विर् नी॒विः पि॑तृ॒णाम् पि॑तृ॒णान् नी॒विः । \newline
52. नी॒वि रोष॑धीना॒ मोष॑धीनान् नी॒विर् नी॒वि रोष॑धीनाम् । \newline
53. ओष॑धीनाम् प्रघा॒तः प्र॑घा॒त ओष॑धीना॒ मोष॑धीनाम् प्रघा॒तः । \newline
54. प्र॒घा॒त आ॑दि॒त्याना॑ मादि॒त्याना᳚म् प्रघा॒तः प्र॑घा॒त आ॑दि॒त्याना᳚म् । \newline
55. प्र॒घा॒त इति॑ प्र - घा॒तः । \newline

\textbf{Ghana Paata } \newline

1. ती॒र्थ मे॒वैव ती॒र्थम् ती॒र्थ मे॒व स॑मा॒नानाꣳ॑ समा॒नाना॑ मे॒व ती॒र्थम् ती॒र्थ मे॒व स॑मा॒नाना᳚म् । \newline
2. ए॒व स॑मा॒नानाꣳ॑ समा॒नाना॑ मे॒वैव स॑मा॒नाना᳚म् भवति भवति समा॒नाना॑ मे॒वैव स॑मा॒नाना᳚म् भवति । \newline
3. स॒मा॒नाना᳚म् भवति भवति समा॒नानाꣳ॑ समा॒नाना᳚म् भव त्य॒पो॑ ऽपो भ॑वति समा॒नानाꣳ॑ समा॒नाना᳚म् भव त्य॒पः । \newline
4. भ॒व॒ त्य॒पो॑ ऽपो भ॑वति भव त्य॒पो᳚ ऽश्ञा त्यश्ञा त्य॒पो भ॑वति भव त्य॒पो᳚ ऽश्ञाति । \newline
5. अ॒पो᳚ ऽश्ञा त्यश्ञा त्य॒पो᳚(1॒) ऽपो᳚ ऽश्ञा त्यन्तर॒तो᳚ ऽन्तर॒तो᳚ ऽश्ञा त्य॒पो᳚(1॒) ऽपो᳚ ऽश्ञा त्यन्तर॒तः । \newline
6. अ॒श्ञा॒ त्य॒न्त॒र॒तो᳚ ऽन्तर॒तो᳚ ऽश्ञा त्यश्ञा त्यन्तर॒त ए॒वै वान्त॑र॒तो᳚ ऽश्ञा त्यश्ञा त्यन्तर॒त ए॒व । \newline
7. अ॒न्त॒र॒त ए॒वै वान्त॑र॒तो᳚ ऽन्तर॒त ए॒व मेद्ध्यो॒ मेद्ध्य॑ ए॒वान्त॑र॒तो᳚ ऽन्तर॒त ए॒व मेद्ध्यः॑ । \newline
8. ए॒व मेद्ध्यो॒ मेद्ध्य॑ ए॒वैव मेद्ध्यो॑ भवति भवति॒ मेद्ध्य॑ ए॒वैव मेद्ध्यो॑ भवति । \newline
9. मेद्ध्यो॑ भवति भवति॒ मेद्ध्यो॒ मेद्ध्यो॑ भवति॒ वास॑सा॒ वास॑सा भवति॒ मेद्ध्यो॒ मेद्ध्यो॑ भवति॒ वास॑सा । \newline
10. भ॒व॒ति॒ वास॑सा॒ वास॑सा भवति भवति॒ वास॑सा दीक्षयति दीक्षयति॒ वास॑सा भवति भवति॒ वास॑सा दीक्षयति । \newline
11. वास॑सा दीक्षयति दीक्षयति॒ वास॑सा॒ वास॑सा दीक्षयति सौ॒म्यꣳ सौ॒म्यम् दी᳚क्षयति॒ वास॑सा॒ वास॑सा दीक्षयति सौ॒म्यम् । \newline
12. दी॒क्ष॒य॒ति॒ सौ॒म्यꣳ सौ॒म्यम् दी᳚क्षयति दीक्षयति सौ॒म्यं ॅवै वै सौ॒म्यम् दी᳚क्षयति दीक्षयति सौ॒म्यं ॅवै । \newline
13. सौ॒म्यं ॅवै वै सौ॒म्यꣳ सौ॒म्यं ॅवै क्षौम॒म् क्षौमं॒ ॅवै सौ॒म्यꣳ सौ॒म्यं ॅवै क्षौम᳚म् । \newline
14. वै क्षौम॒म् क्षौमं॒ ॅवै वै क्षौम॑म् दे॒वत॑या दे॒वत॑या॒ क्षौमं॒ ॅवै वै क्षौम॑म् दे॒वत॑या । \newline
15. क्षौम॑म् दे॒वत॑या दे॒वत॑या॒ क्षौम॒म् क्षौम॑म् दे॒वत॑या॒ सोमꣳ॒॒ सोम॑म् दे॒वत॑या॒ क्षौम॒म् क्षौम॑म् दे॒वत॑या॒ सोम᳚म् । \newline
16. दे॒वत॑या॒ सोमꣳ॒॒ सोम॑म् दे॒वत॑या दे॒वत॑या॒ सोम॑ मे॒ष ए॒ष सोम॑म् दे॒वत॑या दे॒वत॑या॒ सोम॑ मे॒षः । \newline
17. सोम॑ मे॒ष ए॒ष सोमꣳ॒॒ सोम॑ मे॒ष दे॒वता᳚म् दे॒वता॑ मे॒ष सोमꣳ॒॒ सोम॑ मे॒ष दे॒वता᳚म् । \newline
18. ए॒ष दे॒वता᳚म् दे॒वता॑ मे॒ष ए॒ष दे॒वता॒ मुपोप॑ दे॒वता॑ मे॒ष ए॒ष दे॒वता॒ मुप॑ । \newline
19. दे॒वता॒ मुपोप॑ दे॒वता᳚म् दे॒वता॒ मुपै᳚ त्ये॒त्युप॑ दे॒वता᳚म् दे॒वता॒ मुपै॑ति । \newline
20. उपै᳚ त्ये॒त्युपो पै॑ति॒ यो य ए॒त्युपो पै॑ति॒ यः । \newline
21. ए॒ति॒ यो य ए᳚त्येति॒ यो दीक्ष॑ते॒ दीक्ष॑ते॒ य ए᳚त्येति॒ यो दीक्ष॑ते । \newline
22. यो दीक्ष॑ते॒ दीक्ष॑ते॒ यो यो दीक्ष॑ते॒ सोम॑स्य॒ सोम॑स्य॒ दीक्ष॑ते॒ यो यो दीक्ष॑ते॒ सोम॑स्य । \newline
23. दीक्ष॑ते॒ सोम॑स्य॒ सोम॑स्य॒ दीक्ष॑ते॒ दीक्ष॑ते॒ सोम॑स्य त॒नू स्त॒नूः सोम॑स्य॒ दीक्ष॑ते॒ दीक्ष॑ते॒ सोम॑स्य त॒नूः । \newline
24. सोम॑स्य त॒नू स्त॒नूः सोम॑स्य॒ सोम॑स्य त॒नू र॑स्यसि त॒नूः सोम॑स्य॒ सोम॑स्य त॒नू र॑सि । \newline
25. त॒नू र॑स्यसि त॒नू स्त॒नू र॑सि त॒नुव॑म् त॒नुव॑ मसि त॒नू स्त॒नू र॑सि त॒नुव᳚म् । \newline
26. अ॒सि॒ त॒नुव॑म् त॒नुव॑ मस्यसि त॒नुव॑म् मे मे त॒नुव॑ मस्यसि त॒नुव॑म् मे । \newline
27. त॒नुव॑म् मे मे त॒नुव॑म् त॒नुव॑म् मे पाहि पाहि मे त॒नुव॑म् त॒नुव॑म् मे पाहि । \newline
28. मे॒ पा॒हि॒ पा॒हि॒ मे॒ मे॒ पा॒ही तीति॑ पाहि मे मे पा॒हीति॑ । \newline
29. पा॒ही तीति॑ पाहि पा॒हीत्या॑ हा॒हेति॑ पाहि पा॒ही त्या॑ह । \newline
30. इत्या॑ हा॒हे तीत्या॑ह॒ स्वाꣳ स्वा मा॒हे तीत्या॑ह॒ स्वाम् । \newline
31. आ॒ह॒ स्वाꣳ स्वा मा॑हाह॒ स्वा मे॒वैव स्वा मा॑हाह॒ स्वा मे॒व । \newline
32. स्वा मे॒वैव स्वाꣳ स्वा मे॒व दे॒वता᳚म् दे॒वता॑ मे॒व स्वाꣳ स्वा मे॒व दे॒वता᳚म् । \newline
33. ए॒व दे॒वता᳚म् दे॒वता॑ मे॒वैव दे॒वता॒ मुपोप॑ दे॒वता॑ मे॒वैव दे॒वता॒ मुप॑ । \newline
34. दे॒वता॒ मुपोप॑ दे॒वता᳚म् दे॒वता॒ मुपै᳚त्ये॒ त्युप॑ दे॒वता᳚म् दे॒वता॒ मुपै॑ति । \newline
35. उपै᳚त्ये॒ त्युपोपै॒ त्यथो॒ अथो॑ ए॒त्युपो पै॒त्यथो᳚ । \newline
36. ए॒त्यथो॒ अथो॑ एत्ये॒ त्यथो॑ आ॒शिष॑ मा॒शिष॒ मथो॑ एत्ये॒ त्यथो॑ आ॒शिष᳚म् । \newline
37. अथो॑ आ॒शिष॑ मा॒शिष॒ मथो॒ अथो॑ आ॒शिष॑ मे॒वै वाशिष॒ मथो॒ अथो॑ आ॒शिष॑ मे॒व । \newline
38. अथो॒ इत्यथो᳚ । \newline
39. आ॒शिष॑ मे॒वै वाशिष॑ मा॒शिष॑ मे॒वैता मे॒ता मे॒वाशिष॑ मा॒शिष॑ मे॒वैताम् । \newline
40. आ॒शिष॒मित्या᳚ - शिष᳚म् । \newline
41. ए॒वैता मे॒ता मे॒वै वैता मैता मे॒वै वैता मा । \newline
42. ए॒ता मैता मे॒ता मा शा᳚स्ते शास्त॒ ऐता मे॒ता मा शा᳚स्ते । \newline
43. आ शा᳚स्ते शास्त॒ आ शा᳚स्ते॒ ऽग्ने र॒ग्नेः शा᳚स्त॒ आ शा᳚स्ते॒ ऽग्नेः । \newline
44. शा॒स्ते॒ ऽग्ने र॒ग्नेः शा᳚स्ते शास्ते॒ ऽग्ने स्तू॑षा॒धान॑म् तूषा॒धान॑ म॒ग्नेः शा᳚स्ते शास्ते॒ ऽग्ने स्तू॑षा॒धान᳚म् । \newline
45. अ॒ग्ने स्तू॑षा॒धान॑म् तूषा॒धान॑ म॒ग्ने र॒ग्ने स्तू॑षा॒धानं॑ ॅवा॒योर् वा॒यो स्तू॑षा॒धान॑ म॒ग्ने र॒ग्ने स्तू॑षा॒धानं॑ ॅवा॒योः । \newline
46. तू॒षा॒धानं॑ ॅवा॒योर् वा॒यो स्तू॑षा॒धान॑म् तूषा॒धानं॑ ॅवा॒योर् वा॑त॒पानं॑ ॅवात॒पानं॑ ॅवा॒यो स्तू॑षा॒धान॑म् तूषा॒धानं॑ ॅवा॒योर् वा॑त॒पान᳚म् । \newline
47. तू॒षा॒धान॒मिति॑ तूष - आ॒धान᳚म् । \newline
48. वा॒योर् वा॑त॒पानं॑ ॅवात॒पानं॑ ॅवा॒योर् वा॒योर् वा॑त॒पान॑म् पितृ॒णाम् पि॑तृ॒णां ॅवा॑त॒पानं॑ ॅवा॒योर् वा॒योर् वा॑त॒पान॑म् पितृ॒णाम् । \newline
49. वा॒त॒पान॑म् पितृ॒णाम् पि॑तृ॒णां ॅवा॑त॒पानं॑ ॅवात॒पान॑म् पितृ॒णाम् नी॒विर् नी॒विः पि॑तृ॒णां ॅवा॑त॒पानं॑ ॅवात॒पान॑म् पितृ॒णान् नी॒विः । \newline
50. वा॒त॒पान॒मिति॑ वात - पान᳚म् । \newline
51. पि॒तृ॒णाम् नी॒विर् नी॒विः पि॑तृ॒णाम् पि॑तृ॒णाम् नी॒वि रोष॑धीना॒ मोष॑धीनाम् नी॒विः पि॑तृ॒णाम् पि॑तृ॒णाम् नी॒वि रोष॑धीनाम् । \newline
52. नी॒वि रोष॑धीना॒ मोष॑धीनाम् नी॒विर् नी॒वि रोष॑धीनाम् प्रघा॒तः प्र॑घा॒त ओष॑धीनाम् नी॒विर् नी॒वि रोष॑धीनाम् प्रघा॒तः । \newline
53. ओष॑धीनाम् प्रघा॒तः प्र॑घा॒त ओष॑धीना॒ मोष॑धीनाम् प्रघा॒त आ॑दि॒त्याना॑ मादि॒त्याना᳚म् प्रघा॒त ओष॑धीना॒ मोष॑धीनाम् प्रघा॒त आ॑दि॒त्याना᳚म् । \newline
54. प्र॒घा॒त आ॑दि॒त्याना॑ मादि॒त्याना᳚म् प्रघा॒तः प्र॑घा॒त आ॑दि॒त्याना᳚म् प्राचीनता॒नः प्रा॑चीनता॒न आ॑दि॒त्याना᳚म् प्रघा॒तः प्र॑घा॒त आ॑दि॒त्याना᳚म् प्राचीनता॒नः । \newline
55. प्र॒घा॒त इति॑ प्र - घा॒तः । \newline
\pagebreak
\markright{ TS 6.1.1.4  \hfill https://www.vedavms.in \hfill}

\section{ TS 6.1.1.4 }

\textbf{TS 6.1.1.4 } \newline
\textbf{Samhita Paata} \newline

आ॑दि॒त्यानां᳚ प्राचीनता॒नो विश्वे॑षां दे॒वाना॒मोतु॒ र्नक्ष॑त्राणा-मतीका॒शास्तद्वा ए॒तथ् स॑र्व देव॒त्यं॑ ॅयद्-वासो॒ यद्-वास॑सा दी॒क्षय॑ति॒ सर्वा॑भिरे॒वैनं॑ दे॒वता॑भि-र्दीक्षयति ब॒हिःप्रा॑णो॒ वै म॑नु॒ष्य॑स्त-स्याश॑नं प्रा॒णो᳚-ऽश्नाति॒ सप्रा॑ण ए॒व दी᳚क्षत॒ आशि॑तो भवति॒ यावा॑ने॒वास्य॑ प्रा॒णस्तेन॑ स॒ह मेध॒मुपै॑ति घृ॒तं दे॒वानां॒ मस्तु॑ पितृ॒णां निष्प॑क्वं मनु॒ष्या॑णां॒ तद्वा - [  ] \newline

\textbf{Pada Paata} \newline

आ॒दि॒त्याना᳚म् । प्रा॒ची॒न॒ता॒न इति॑ प्राचीन - ता॒नः । विश्वे॑षाम् । दे॒वाना᳚म् । ओतुः॑ । नक्ष॑त्राणाम् । अ॒ती॒का॒शाः । तत् । वै । ए॒तत् । स॒र्व॒दे॒व॒त्य॑मिति॑ सर्व - दे॒व॒त्य᳚म् । यत् । वासः॑ । यत् । वास॑सा । दी॒क्षय॑ति । सर्वा॑भिः । ए॒व । ए॒न॒म् । दे॒वता॑भिः । दी॒क्ष॒य॒ति॒ । ब॒हिःप्रा॑ण॒ इति॑ ब॒हिः - प्रा॒णः॒ । वै । म॒नु॒ष्यः॑ । तस्य॑ । अश॑नम् । प्रा॒ण इति॑ प्र -अ॒नः । अ॒श्नाति॑ । सप्रा॑ण॒ इति॒ स - प्रा॒णः॒ । ए॒व । दी॒क्ष॒ते॒ । आशि॑तः । भ॒व॒ति॒ । यावान्॑ । ए॒व । अ॒स्य॒ । प्रा॒ण इति॑ प्र -अ॒नः । तेन॑ । स॒ह । मेध᳚म् । उपेति॑ । ए॒ति॒ । घृ॒तम् । दे॒वाना᳚म् । मस्तु॑ । पि॒तृ॒णाम् । निष्प॑क्व॒मिति॒ निः - प॒क्व॒म् । म॒नु॒ष्या॑णाम् । तत् । वै ।  \newline


\textbf{Krama Paata} \newline

आ॒दि॒त्याना᳚म् प्राचीनता॒नः । प्रा॒ची॒न॒ता॒नो विश्वे॑षाम् । प्रा॒ची॒न॒ता॒न इति॑ प्राचीन - ता॒नः । विश्वे॑षाम् दे॒वाना᳚म् । दे॒वाना॒मोतुः॑ । ओतु॒र् नक्ष॑त्राणाम् । नक्ष॑त्राणामतीका॒शाः । अ॒ती॒का॒शास्तत् । तद् वै । वा ए॒तत् । ए॒तथ् स॑र्वदेव॒त्य᳚म् । स॒र्व॒दे॒व॒त्य॑म् ॅयत् । स॒र्व॒दे॒वत्य॑मिति॑ सर्व - दे॒व॒त्य᳚म् । यद् वासः॑ । वासो॒ यत् । यद् वास॑सा । वास॑सा दी॒क्षय॑ति । दी॒क्षय॑ति॒ सर्वा॑भिः । सर्वा॑भिरे॒व । ए॒वैन᳚म् । ए॒न॒म् दे॒वता॑भिः । दे॒वता॑भिर् दीक्षयति । दी॒क्ष॒य॒ति॒ ब॒हिःप्रा॑णः । ब॒हिःप्रा॑णो॒ वै । ब॒हिःप्रा॑ण॒ इति॑ ब॒हिः - प्रा॒णः॒ । वै म॑नु॒ष्यः॑ । म॒नु॒ष्य॑स्तस्य॑ । तस्याश॑नम् । अश॑नम् प्रा॒णः । प्रा॒णो᳚ऽश्ञाति॑ । प्रा॒ण इति॑ प्र - अ॒नः । अ॒श्ञाति॒ सप्रा॑णः । सप्रा॑ण ए॒व । सप्रा॑ण॒ इति॒ स - प्रा॒णः॒ । ए॒व दी᳚क्षते । दी॒क्ष॒त॒ आशि॑तः । आशि॑तो भवति । भ॒व॒ति॒ यावान्॑ । यावा॑ने॒व । ए॒वास्य॑ । अ॒स्य॒ प्रा॒णः । प्रा॒णस्तेन॑ । प्रा॒ण इति॑ प्र - अ॒नः । तेन॑ स॒ह । स॒ह मेध᳚म् । मेध॒मुप॑ । उपै॑ति । ए॒ति॒ घृ॒तम् । घृ॒तम् दे॒वाना᳚म् । दे॒वाना॒म् मस्तु॑ । मस्तु॑ पितृ॒णाम् । पि॒तृ॒णाम् निष्प॑क्वम् । निष्प॑क्वम् मनु॒ष्या॑णाम् । निष्प॑क्व॒मिति॒ निः - प॒क्व॒म् । म॒नु॒ष्या॑णा॒म् तत् । तद् वै । वा ए॒तत् \newline

\textbf{Jatai Paata} \newline

1. आ॒दि॒त्याना᳚म् प्राचीनता॒नः प्रा॑चीनता॒न आ॑दि॒त्याना॑ मादि॒त्याना᳚म् प्राचीनता॒नः । \newline
2. प्रा॒ची॒न॒ता॒नो विश्वे॑षां॒ ॅविश्वे॑षाम् प्राचीनता॒नः प्रा॑चीनता॒नो विश्वे॑षाम् । \newline
3. प्रा॒ची॒न॒ता॒न इति॑ प्राचीन - ता॒नः । \newline
4. विश्वे॑षाम् दे॒वाना᳚म् दे॒वानां॒ ॅविश्वे॑षां॒ ॅविश्वे॑षाम् दे॒वाना᳚म् । \newline
5. दे॒वाना॒ मोतु॒ रोतु॑र् दे॒वाना᳚म् दे॒वाना॒ मोतुः॑ । \newline
6. ओतु॒र् नक्ष॑त्राणा॒म् नक्ष॑त्राणा॒ मोतु॒ रोतु॒र् नक्ष॑त्राणाम् । \newline
7. नक्ष॑त्राणा मतीका॒शा अ॑तीका॒शा नक्ष॑त्राणा॒म् नक्ष॑त्राणा मतीका॒शाः । \newline
8. अ॒ती॒का॒शा स्तत् तद॑तीका॒शा अ॑तीका॒शा स्तत् । \newline
9. तद् वै वै तत् तद् वै । \newline
10. वा ए॒त दे॒तद् वै वा ए॒तत् । \newline
11. ए॒तथ् स॑र्वदेव॒त्यꣳ॑ सर्वदेव॒त्य॑ मे॒त दे॒तथ् स॑र्वदेव॒त्य᳚म् । \newline
12. स॒र्व॒दे॒व॒त्यं॑ ॅयद् यथ् स॑र्वदेव॒त्यꣳ॑ सर्वदेव॒त्यं॑ ॅयत् । \newline
13. स॒र्व॒दे॒व॒त्य॑मिति॑ सर्व - दे॒व॒त्य᳚म् । \newline
14. यद् वासो॒ वासो॒ यद् यद् वासः॑ । \newline
15. वासो॒ यद् यद् वासो॒ वासो॒ यत् । \newline
16. यद् वास॑सा॒ वास॑सा॒ यद् यद् वास॑सा । \newline
17. वास॑सा दी॒क्षय॑ति दी॒क्षय॑ति॒ वास॑सा॒ वास॑सा दी॒क्षय॑ति । \newline
18. दी॒क्षय॑ति॒ सर्वा॑भिः॒ सर्वा॑भिर् दी॒क्षय॑ति दी॒क्षय॑ति॒ सर्वा॑भिः । \newline
19. सर्वा॑भि रे॒वैव सर्वा॑भिः॒ सर्वा॑भि रे॒व । \newline
20. ए॒वैन॑ मेन मे॒वै वैन᳚म् । \newline
21. ए॒न॒म् दे॒वता॑भिर् दे॒वता॑भि रेन मेनम् दे॒वता॑भिः । \newline
22. दे॒वता॑भिर् दीक्षयति दीक्षयति दे॒वता॑भिर् दे॒वता॑भिर् दीक्षयति । \newline
23. दी॒क्ष॒य॒ति॒ ब॒हिःप्रा॑णो ब॒हिःप्रा॑णो दीक्षयति दीक्षयति ब॒हिःप्रा॑णः । \newline
24. ब॒हिःप्रा॑णो॒ वै वै ब॒हिःप्रा॑णो ब॒हिःप्रा॑णो॒ वै । \newline
25. ब॒हिःप्रा॑ण॒ इति॑ ब॒हिः - प्रा॒णः॒ । \newline
26. वै म॑नु॒ष्यो॑ मनु॒ष्यो॑ वै वै म॑नु॒ष्यः॑ । \newline
27. म॒नु॒ष्य॑ स्तस्य॒ तस्य॑ मनु॒ष्यो॑ मनु॒ष्य॑ स्तस्य॑ । \newline
28. तस्या श॑न॒ मश॑न॒म् तस्य॒ तस्या श॑नम् । \newline
29. अश॑नम् प्रा॒णः प्रा॒णो ऽश॑न॒ मश॑नम् प्रा॒णः । \newline
30. प्रा॒णो᳚ ऽश्ञात्य॒ श्ञाति॑ प्रा॒णः प्रा॒णो᳚ ऽश्ञाति॑ । \newline
31. प्रा॒ण इति॑ प्र - अ॒नः । \newline
32. अ॒श्ञाति॒ सप्रा॑णः॒ सप्रा॑णो॒ ऽश्ञात्य॒ श्ञाति॒ सप्रा॑णः । \newline
33. सप्रा॑ण ए॒वैव सप्रा॑णः॒ सप्रा॑ण ए॒व । \newline
34. सप्रा॑ण॒ इति॒ स - प्रा॒णः॒ । \newline
35. ए॒व दी᳚क्षते दीक्षत ए॒वैव दी᳚क्षते । \newline
36. दी॒क्ष॒त॒ आशि॑त॒ आशि॑तो दीक्षते दीक्षत॒ आशि॑तः । \newline
37. आशि॑तो भवति भव॒त्या शि॑त॒ आशि॑तो भवति । \newline
38. भ॒व॒ति॒ यावा॒न्॒. यावा᳚न् भवति भवति॒ यावान्॑ । \newline
39. यावा॑ने॒ वैव यावा॒न्॒. यावा॑ने॒व । \newline
40. ए॒वास्या᳚ स्यै॒वै वास्य॑ । \newline
41. अ॒स्य॒ प्रा॒णः प्रा॒णो᳚ ऽस्यास्य प्रा॒णः । \newline
42. प्रा॒ण स्तेन॒ तेन॑ प्रा॒णः प्रा॒ण स्तेन॑ । \newline
43. प्रा॒ण इति॑ प्र - अ॒नः । \newline
44. तेन॑ स॒ह स॒ह तेन॒ तेन॑ स॒ह । \newline
45. स॒ह मेध॒म् मेधꣳ॑ स॒ह स॒ह मेध᳚म् । \newline
46. मेध॒ मुपोप॒ मेध॒म् मेध॒ मुप॑ । \newline
47. उपै᳚त्ये॒ त्युपो पै॑ति । \newline
48. ए॒ति॒ घृ॒तम् घृ॒त मे᳚त्येति घृ॒तम् । \newline
49. घृ॒तम् दे॒वाना᳚म् दे॒वाना᳚म् घृ॒तम् घृ॒तम् दे॒वाना᳚म् । \newline
50. दे॒वाना॒म् मस्तु॒ मस्तु॑ दे॒वाना᳚म् दे॒वाना॒म् मस्तु॑ । \newline
51. मस्तु॑ पितृ॒णाम् पि॑तृ॒णाम् मस्तु॒ मस्तु॑ पितृ॒णाम् । \newline
52. पि॒तृ॒णाम् निष्प॑क्व॒म् निष्प॑क्वम् पितृ॒णाम् पि॑तृ॒णाम् निष्प॑क्वम् । \newline
53. निष्प॑क्वम् मनु॒ष्या॑णाम् मनु॒ष्या॑णा॒न् निष्प॑क्व॒म् निष्प॑क्वम् मनु॒ष्या॑णाम् । \newline
54. निष्प॑क्व॒मिति॒ निः - प॒क्व॒म् । \newline
55. म॒नु॒ष्या॑णा॒म् तत् तन् म॑नु॒ष्या॑णाम् मनु॒ष्या॑णा॒म् तत् । \newline
56. तद् वै वै तत् तद् वै । \newline
57. वा ए॒त दे॒तद् वै वा ए॒तत् । \newline

\textbf{Ghana Paata } \newline

1. आ॒दि॒त्याना᳚म् प्राचीनता॒नः प्रा॑चीनता॒न आ॑दि॒त्याना॑ मादि॒त्याना᳚म् प्राचीनता॒नो विश्वे॑षां॒ ॅविश्वे॑षाम् प्राचीनता॒न आ॑दि॒त्याना॑ मादि॒त्याना᳚म् प्राचीनता॒नो विश्वे॑षाम् । \newline
2. प्रा॒ची॒न॒ता॒नो विश्वे॑षां॒ ॅविश्वे॑षाम् प्राचीनता॒नः प्रा॑चीनता॒नो विश्वे॑षाम् दे॒वाना᳚म् दे॒वानां॒ ॅविश्वे॑षाम् प्राचीनता॒नः प्रा॑चीनता॒नो विश्वे॑षाम् दे॒वाना᳚म् । \newline
3. प्रा॒ची॒न॒ता॒न इति॑ प्राचीन - ता॒नः । \newline
4. विश्वे॑षाम् दे॒वाना᳚म् दे॒वानां॒ ॅविश्वे॑षां॒ ॅविश्वे॑षाम् दे॒वाना॒ मोतु॒ रोतु॑र् दे॒वानां॒ ॅविश्वे॑षां॒ ॅविश्वे॑षाम् दे॒वाना॒ मोतुः॑ । \newline
5. दे॒वाना॒ मोतु॒ रोतु॑र् दे॒वाना᳚म् दे॒वाना॒ मोतु॒र् नक्ष॑त्राणा॒म् नक्ष॑त्राणा॒ मोतु॑र् दे॒वाना᳚म् दे॒वाना॒ मोतु॒र् नक्ष॑त्राणाम् । \newline
6. ओतु॒र् नक्ष॑त्राणा॒म् नक्ष॑त्राणा॒ मोतु॒ रोतु॒र् नक्ष॑त्राणा मतीका॒शा अ॑तीका॒शा नक्ष॑त्राणा॒ मोतु॒ रोतु॒र् नक्ष॑त्राणा मतीका॒शाः । \newline
7. नक्ष॑त्राणा मतीका॒शा अ॑तीका॒शा नक्ष॑त्राणा॒म् नक्ष॑त्राणा मतीका॒शा स्तत् तद॑तीका॒शा नक्ष॑त्राणा॒म् नक्ष॑त्राणा मतीका॒शा स्तत् । \newline
8. अ॒ती॒का॒शा स्तत् तद॑तीका॒शा अ॑तीका॒शा स्तद् वै वै तद॑तीका॒शा अ॑तीका॒शा स्तद् वै । \newline
9. तद् वै वै तत् तद् वा ए॒त दे॒तद् वै तत् तद् वा ए॒तत् । \newline
10. वा ए॒त दे॒तद् वै वा ए॒तथ् स॑र्वदेव॒त्यꣳ॑ सर्वदेव॒त्य॑ मे॒तद् वै वा ए॒तथ् स॑र्वदेव॒त्य᳚म् । \newline
11. ए॒तथ् स॑र्वदेव॒त्यꣳ॑ सर्वदेव॒त्य॑ मे॒त दे॒तथ् स॑र्वदेव॒त्यं॑ ॅयद् यथ् स॑र्वदेव॒त्य॑ मे॒त दे॒तथ् स॑र्वदेव॒त्यं॑ ॅयत् । \newline
12. स॒र्व॒दे॒व॒त्यं॑ ॅयद् यथ् स॑र्वदेव॒त्यꣳ॑ सर्वदेव॒त्यं॑ ॅयद् वासो॒ वासो॒ यथ् स॑र्वदेव॒त्यꣳ॑ सर्वदेव॒त्यं॑ ॅयद् वासः॑ । \newline
13. स॒र्व॒दे॒व॒त्य॑मिति॑ सर्व - दे॒व॒त्य᳚म् । \newline
14. यद् वासो॒ वासो॒ यद् यद् वासो॒ यद् यद् वासो॒ यद् यद् वासो॒ यत् । \newline
15. वासो॒ यद् यद् वासो॒ वासो॒ यद् वास॑सा॒ वास॑सा॒ यद् वासो॒ वासो॒ यद् वास॑सा । \newline
16. यद् वास॑सा॒ वास॑सा॒ यद् यद् वास॑सा दी॒क्षय॑ति दी॒क्षय॑ति॒ वास॑सा॒ यद् यद् वास॑सा दी॒क्षय॑ति । \newline
17. वास॑सा दी॒क्षय॑ति दी॒क्षय॑ति॒ वास॑सा॒ वास॑सा दी॒क्षय॑ति॒ सर्वा॑भिः॒ सर्वा॑भिर् दी॒क्षय॑ति॒ वास॑सा॒ वास॑सा दी॒क्षय॑ति॒ सर्वा॑भिः । \newline
18. दी॒क्षय॑ति॒ सर्वा॑भिः॒ सर्वा॑भिर् दी॒क्षय॑ति दी॒क्षय॑ति॒ सर्वा॑भि रे॒वैव सर्वा॑भिर् दी॒क्षय॑ति दी॒क्षय॑ति॒ सर्वा॑भि रे॒व । \newline
19. सर्वा॑भि रे॒वैव सर्वा॑भिः॒ सर्वा॑भि रे॒वैन॑ मेन मे॒व सर्वा॑भिः॒ सर्वा॑भि रे॒वैन᳚म् । \newline
20. ए॒वैन॑ मेन मे॒वै वैन॑म् दे॒वता॑भिर् दे॒वता॑भि रेन मे॒वै वैन॑म् दे॒वता॑भिः । \newline
21. ए॒न॒म् दे॒वता॑भिर् दे॒वता॑भि रेन मेनम् दे॒वता॑भिर् दीक्षयति दीक्षयति दे॒वता॑भि रेन मेनम् दे॒वता॑भिर् दीक्षयति । \newline
22. दे॒वता॑भिर् दीक्षयति दीक्षयति दे॒वता॑भिर् दे॒वता॑भिर् दीक्षयति ब॒हिःप्रा॑णो ब॒हिःप्रा॑णो दीक्षयति दे॒वता॑भिर् दे॒वता॑भिर् दीक्षयति ब॒हिःप्रा॑णः । \newline
23. दी॒क्ष॒य॒ति॒ ब॒हिःप्रा॑णो ब॒हिःप्रा॑णो दीक्षयति दीक्षयति ब॒हिःप्रा॑णो॒ वै वै ब॒हिःप्रा॑णो दीक्षयति दीक्षयति ब॒हिःप्रा॑णो॒ वै । \newline
24. ब॒हिःप्रा॑णो॒ वै वै ब॒हिःप्रा॑णो ब॒हिःप्रा॑णो॒ वै म॑नु॒ष्यो॑ मनु॒ष्यो॑ वै ब॒हिःप्रा॑णो ब॒हिःप्रा॑णो॒ वै म॑नु॒ष्यः॑ । \newline
25. ब॒हिःप्रा॑ण॒ इति॑ ब॒हिः - प्रा॒णः॒ । \newline
26. वै म॑नु॒ष्यो॑ मनु॒ष्यो॑ वै वै म॑नु॒ष्य॑ स्तस्य॒ तस्य॑ मनु॒ष्यो॑ वै वै म॑नु॒ष्य॑ स्तस्य॑ । \newline
27. म॒नु॒ष्य॑ स्तस्य॒ तस्य॑ मनु॒ष्यो॑ मनु॒ष्य॑ स्तस्याश॑न॒ मश॑न॒म् तस्य॑ मनु॒ष्यो॑ मनु॒ष्य॑ स्तस्याश॑नम् । \newline
28. तस्याश॑न॒ मश॑न॒म् तस्य॒ तस्याश॑नम् प्रा॒णः प्रा॒णो ऽश॑न॒म् तस्य॒ तस्याश॑नम् प्रा॒णः । \newline
29. अश॑नम् प्रा॒णः प्रा॒णो ऽश॑न॒ मश॑नम् प्रा॒णो᳚ ऽश्ञा त्य॒श्ञाति॑ प्रा॒णो ऽश॑न॒ मश॑नम् प्रा॒णो᳚ ऽश्ञाति॑ । \newline
30. प्रा॒णो᳚ ऽश्ञा त्य॒श्ञाति॑ प्रा॒णः प्रा॒णो᳚ ऽश्ञाति॒ सप्रा॑णः॒ सप्रा॑णो॒ ऽश्ञाति॑ प्रा॒णः प्रा॒णो᳚ ऽश्ञाति॒ सप्रा॑णः । \newline
31. प्रा॒ण इति॑ प्र - अ॒नः । \newline
32. अ॒श्ञाति॒ सप्रा॑णः॒ सप्रा॑णो॒ ऽश्ञा त्य॒श्ञाति॒ सप्रा॑ण ए॒वैव सप्रा॑णो॒ ऽश्ञा त्य॒श्ञाति॒ सप्रा॑ण ए॒व । \newline
33. सप्रा॑ण ए॒वैव सप्रा॑णः॒ सप्रा॑ण ए॒व दी᳚क्षते दीक्षत ए॒व सप्रा॑णः॒ सप्रा॑ण ए॒व दी᳚क्षते । \newline
34. सप्रा॑ण॒ इति॒ स - प्रा॒णः॒ । \newline
35. ए॒व दी᳚क्षते दीक्षत ए॒वैव दी᳚क्षत॒ आशि॑त॒ आशि॑तो दीक्षत ए॒वैव दी᳚क्षत॒ आशि॑तः । \newline
36. दी॒क्ष॒त॒ आशि॑त॒ आशि॑तो दीक्षते दीक्षत॒ आशि॑तो भवति भव॒ त्याशि॑तो दीक्षते दीक्षत॒ आशि॑तो भवति । \newline
37. आशि॑तो भवति भव॒ त्याशि॑त॒ आशि॑तो भवति॒ यावा॒न्॒. यावा᳚न् भव॒ त्याशि॑त॒ आशि॑तो भवति॒ यावान्॑ । \newline
38. भ॒व॒ति॒ यावा॒न्॒. यावा᳚न् भवति भवति॒ यावा॑ने॒वैव यावा᳚न् भवति भवति॒ यावा॑ने॒व । \newline
39. यावा॑ ने॒वैव यावा॒न्॒. यावा॑ ने॒वास्या᳚ स्यै॒व यावा॒न्॒. यावा॑ ने॒वास्य॑ । \newline
40. ए॒वास्या᳚ स्यै॒वै वास्य॑ प्रा॒णः प्रा॒णो᳚ ऽस्यै॒वै वास्य॑ प्रा॒णः । \newline
41. अ॒स्य॒ प्रा॒णः प्रा॒णो᳚ ऽस्यास्य प्रा॒ण स्तेन॒ तेन॑ प्रा॒णो᳚ ऽस्यास्य प्रा॒ण स्तेन॑ । \newline
42. प्रा॒ण स्तेन॒ तेन॑ प्रा॒णः प्रा॒ण स्तेन॑ स॒ह स॒ह तेन॑ प्रा॒णः प्रा॒ण स्तेन॑ स॒ह । \newline
43. प्रा॒ण इति॑ प्र - अ॒नः । \newline
44. तेन॑ स॒ह स॒ह तेन॒ तेन॑ स॒ह मेध॒म् मेधꣳ॑ स॒ह तेन॒ तेन॑ स॒ह मेध᳚म् । \newline
45. स॒ह मेध॒म् मेधꣳ॑ स॒ह स॒ह मेध॒ मुपोप॒ मेधꣳ॑ स॒ह स॒ह मेध॒ मुप॑ । \newline
46. मेध॒ मुपोप॒ मेध॒म् मेध॒ मुपै᳚त्ये॒ त्युप॒ मेध॒म् मेध॒ मुपै॑ति । \newline
47. उपै᳚त्ये॒ त्युपोपै॑ति घृ॒तम् घृ॒त मे॒त्यु पोपै॑ति घृ॒तम् । \newline
48. ए॒ति॒ घृ॒तम् घृ॒त मे᳚त्येति घृ॒तम् दे॒वाना᳚म् दे॒वाना᳚म् घृ॒त मे᳚त्येति घृ॒तम् दे॒वाना᳚म् । \newline
49. घृ॒तम् दे॒वाना᳚म् दे॒वाना᳚म् घृ॒तम् घृ॒तम् दे॒वाना॒म् मस्तु॒ मस्तु॑ दे॒वाना᳚म् घृ॒तम् घृ॒तम् दे॒वाना॒म् मस्तु॑ । \newline
50. दे॒वाना॒म् मस्तु॒ मस्तु॑ दे॒वाना᳚म् दे॒वाना॒म् मस्तु॑ पितृ॒णाम् पि॑तृ॒णाम् मस्तु॑ दे॒वाना᳚म् दे॒वाना॒म् मस्तु॑ पितृ॒णाम् । \newline
51. मस्तु॑ पितृ॒णाम् पि॑तृ॒णाम् मस्तु॒ मस्तु॑ पितृ॒णाम् निष्प॑क्व॒म् निष्प॑क्वम् पितृ॒णाम् मस्तु॒ मस्तु॑ पितृ॒णाम् निष्प॑क्वम् । \newline
52. पि॒तृ॒णाम् निष्प॑क्व॒म् निष्प॑क्वम् पितृ॒णाम् पि॑तृ॒णाम् निष्प॑क्वम् मनु॒ष्या॑णाम् मनु॒ष्या॑णा॒म् निष्प॑क्वम् पितृ॒णाम् पि॑तृ॒णाम् निष्प॑क्वम् मनु॒ष्या॑णाम् । \newline
53. निष्प॑क्वम् मनु॒ष्या॑णाम् मनु॒ष्या॑णा॒म् निष्प॑क्व॒म् निष्प॑क्वम् मनु॒ष्या॑णा॒म् तत् तन् म॑नु॒ष्या॑णा॒म् निष्प॑क्व॒म् निष्प॑क्वम् मनु॒ष्या॑णा॒म् तत् । \newline
54. निष्प॑क्व॒मिति॒ निः - प॒क्व॒म् । \newline
55. म॒नु॒ष्या॑णा॒म् तत् तन् म॑नु॒ष्या॑णाम् मनु॒ष्या॑णा॒म् तद् वै वै तन् म॑नु॒ष्या॑णाम् मनु॒ष्या॑णा॒म् तद् वै । \newline
56. तद् वै वै तत् तद् वा ए॒त दे॒तद् वै तत् तद् वा ए॒तत् । \newline
57. वा ए॒त दे॒तद् वै वा ए॒तथ् स॑र्वदेव॒त्यꣳ॑ सर्वदेव॒त्य॑ मे॒तद् वै वा ए॒तथ् स॑र्वदेव॒त्य᳚म् । \newline
\pagebreak
\markright{ TS 6.1.1.5  \hfill https://www.vedavms.in \hfill}

\section{ TS 6.1.1.5 }

\textbf{TS 6.1.1.5 } \newline
\textbf{Samhita Paata} \newline

ए॒तथ् स॑र्वदेव॒त्यं॑ ॅयन्नव॑नीतं॒ ॅयन्नव॑नीतेनाभ्य॒ङ्क्ते सर्वा॑ ए॒व दे॒वताः᳚ प्रीणाति॒ प्रच्यु॑तो॒ वा ए॒षो᳚ऽस्माल्लो॒कादग॑तो देवलो॒कं ॅयो दी᳚क्षि॒तो᳚ ऽन्त॒रेव॒ नव॑नीतं॒ तस्मा॒-न्नव॑नीतेना॒भ्य॑ङ्क्ते ऽनुलो॒मं ॅयजु॑षा॒ व्यावृ॑त्त्या॒ इन्द्रो॑ वृ॒त्रम॑ह॒न् तस्य॑ क॒नीनि॑का॒ परा॑ऽपत॒त् तदाञ्ज॑नम-भव॒द्यदा॒ङ्क्ते चक्षु॑रे॒व भ्रातृ॑व्यस्य वृङ्क्ते॒ दक्षि॑णं॒ पूर्व॒माऽङ्क्ते॑ - [  ] \newline

\textbf{Pada Paata} \newline

ए॒तत् । स॒र्व॒दे॒व॒त्य॑मिति॑ सर्व - दे॒व॒त्य᳚म् । यत् । नव॑नीत॒मिति॒ नव॑ - नी॒त॒म् । यत् । नव॑नीते॒नेति॒ नव॑ - नी॒ते॒न॒ । अ॒भ्य॒ङ्क्त इत्य॑भि - अ॒ङ्क्ते । सर्वाः᳚ । ए॒व । दे॒वताः᳚ । प्री॒णा॒ति॒ । प्रच्यु॑त॒ इति॒ प्र - च्यु॒तः॒ । वै । ए॒षः । अ॒स्मात् । लो॒कात् । अग॑तः । दे॒व॒लो॒कमिति॑ देव - लो॒कम् । यः । दी॒क्षि॒तः । अ॒न्त॒रा । इ॒व॒ । नव॑नीत॒मिति॒ नव॑ - नी॒त॒म् । तस्मा᳚त् । नव॑नीते॒नेति॒ नव॑ - नी॒ते॒न॒ । अ॒भीति॑ । अ॒ङ्क्ते॒ । अ॒नु॒लो॒ममित्य॑नु - लो॒मम् । यजु॑षा । व्यावृ॑त्त्या॒ इति॑ वि - आवृ॑त्त्यै । इन्द्रः॑ । वृ॒त्रम् । अ॒ह॒न्न् । तस्य॑ । क॒नीनि॑का । परेति॑ । अ॒प॒त॒त् । तत् । आञ्ज॑न॒मित्या᳚-अञ्ज॑नम् । अ॒भ॒व॒त् । यत् । आ॒ङ्क्त इत्या᳚ - अ॒ङ्क्ते । चक्षुः॑ । ए॒व । भ्रातृ॑व्यस्य । वृ॒ङ्क्ते॒ । दक्षि॑णम् । पूर्व᳚म् । एति॑ । अ॒ङ्क्ते॒ ।  \newline


\textbf{Krama Paata} \newline

ए॒तथ् स॑र्वदेव॒त्य᳚म् । स॒र्व॒दे॒व॒त्य॑म् ॅयत् । स॒र्व॒दे॒व॒त्य॑मिति॑ सर्व - दे॒व॒त्य᳚म् । यन् नव॑नीतम् । नव॑नीत॒म् ॅयत् । नव॑नीत॒मिति॒ नव॑ - नी॒त॒म् । यन् नव॑नीतेन । नव॑नीतेनाभ्य॒ङ्‍क्ते । नव॑नीते॒नेति॒ नव॑ - नी॒ते॒न॒ । अ॒भ्य॒ङ्‍क्ते सर्वाः᳚ । अ॒भ्य॒ङ्‍क्त इत्य॑भि - अ॒ङ्‍क्ते । सर्वा॑ ए॒व । ए॒व दे॒वताः᳚ । दे॒वताः᳚ प्रीणाति । प्री॒णा॒ति॒ प्रच्यु॑तः । प्रच्यु॑तो॒ वै । प्रच्यु॑त॒ इति॒ प्र - च्यु॒तः॒ । वा ए॒षः । ए॒षो᳚ऽस्मात् । अ॒स्माल्लो॒कात् । लो॒कादग॑तः । अग॑तो देवलो॒कम् । दे॒व॒लो॒कम् ॅयः । दे॒व॒लो॒कमिति॑ देव - लो॒कम् । यो दी᳚क्षि॒तः । दी॒क्षि॒तो᳚ऽन्त॒रा । अ॒न्त॒रेव॑ । इ॒व॒ नव॑नीतम् । नव॑नीत॒म् तस्मा᳚त् । नव॑नीत॒मिति॒ नव॑ - नी॒त॒म् । तस्मा॒न् नव॑नीतेन । नव॑नीतेना॒भि । नव॑नीते॒नेति॒ नव॑ - नी॒ते॒न॒ । अ॒भ्य॑ङ्‍क्ते । अ॒ङ्‍क्ते॒ऽनु॒लो॒मम् । अ॒नु॒लो॒मम् ॅयजु॑षा । अ॒नु॒लो॒ममित्य॑नु - लो॒मम् । यजु॑षा॒ व्यावृ॑त्त्यै । व्यावृ॑त्त्या॒ इन्द्रः॑ । व्यावृ॑त्त्या॒ इति॑ वि - आवृ॑त्त्यै । इन्द्रो॑ वृ॒त्रम् । वृ॒त्रम॑हन्न् । अ॒ह॒न् तस्य॑ । तस्य॑ क॒नीनि॑का । क॒नीनि॑का॒ परा᳚ । परा॑ऽपतत् । अ॒प॒त॒त् तत् । तदाञ्ज॑नम् । आञ्ज॑नमभवत् । आञ्ज॑न॒मित्या᳚ - अञ्ज॑नम् । अ॒भ॒व॒द् यत् । यदा॒ङ्‍क्ते । आ॒ङ्‍क्ते चक्षुः॑ । आ॒ङ्‍क्त इत्या᳚ - अ॒ङ्‍क्ते । चक्षु॑रे॒व । ए॒व भ्रातृ॑व्यस्य । भ्रातृ॑व्यस्य वृङ्‍क्ते । वृ॒ङ्‍क्ते॒ दक्षि॑णम् । दक्षि॑ण॒म् पूर्व᳚म् । पूर्व॒मा । आऽङ्‍क्ते᳚ । अ॒ङ्‍क्ते॒ स॒व्यम् \newline

\textbf{Jatai Paata} \newline

1. ए॒तथ् स॑र्वदेव॒त्यꣳ॑ सर्वदेव॒त्य॑ मे॒त दे॒तथ् स॑र्वदेव॒त्य᳚म् । \newline
2. स॒र्व॒दे॒व॒त्यं॑ ॅयद् यथ् स॑र्वदेव॒त्यꣳ॑ सर्वदेव॒त्यं॑ ॅयत् । \newline
3. स॒र्व॒दे॒व॒त्य॑मिति॑ सर्व - दे॒व॒त्य᳚म् । \newline
4. यन् नव॑नीत॒म् नव॑नीतं॒ ॅयद् यन् नव॑नीतम् । \newline
5. नव॑नीतं॒ ॅयद् यन् नव॑नीत॒म् नव॑नीतं॒ ॅयत् । \newline
6. नव॑नीत॒मिति॒ नव॑ - नी॒त॒म् । \newline
7. यन् नव॑नीतेन॒ नव॑नीतेन॒ यद् यन् नव॑नीतेन । \newline
8. नव॑नीते नाभ्य॒ङ्क्ते᳚ ऽभ्य॒ङ्क्ते नव॑नीतेन॒ नव॑नीते नाभ्य॒ङ्क्ते । \newline
9. नव॑नीते॒नेति॒ नव॑ - नी॒ते॒न॒ । \newline
10. अ॒भ्य॒ङ्क्ते सर्वाः॒ सर्वा॑ अभ्य॒ङ्क्ते᳚ ऽभ्य॒ङ्क्ते सर्वाः᳚ । \newline
11. अ॒भ्य॒ङ्क्त इत्य॑भि - अ॒ङ्क्ते । \newline
12. सर्वा॑ ए॒वैव सर्वाः॒ सर्वा॑ ए॒व । \newline
13. ए॒व दे॒वता॑ दे॒वता॑ ए॒वैव दे॒वताः᳚ । \newline
14. दे॒वताः᳚ प्रीणाति प्रीणाति दे॒वता॑ दे॒वताः᳚ प्रीणाति । \newline
15. प्री॒णा॒ति॒ प्रच्यु॑तः॒ प्रच्यु॑तः प्रीणाति प्रीणाति॒ प्रच्यु॑तः । \newline
16. प्रच्यु॑तो॒ वै वै प्रच्यु॑तः॒ प्रच्यु॑तो॒ वै । \newline
17. प्रच्यु॑त॒ इति॒ प्र - च्यु॒तः॒ । \newline
18. वा ए॒ष ए॒ष वै वा ए॒षः । \newline
19. ए॒षो᳚ ऽस्मा द॒स्मा दे॒ष ए॒षो᳚ ऽस्मात् । \newline
20. अ॒स्मा ल्लो॒का ल्लो॒का द॒स्मा द॒स्मा ल्लो॒कात् । \newline
21. लो॒का दग॒तो ऽग॑तो लो॒का ल्लो॒का दग॑तः । \newline
22. अग॑तो देवलो॒कम् दे॑वलो॒क मग॒तो ऽग॑तो देवलो॒कम् । \newline
23. दे॒व॒लो॒कं ॅयो यो दे॑वलो॒कम् दे॑वलो॒कं ॅयः । \newline
24. दे॒व॒लो॒कमिति॑ देव - लो॒कम् । \newline
25. यो दी᳚क्षि॒तो दी᳚क्षि॒तो यो यो दी᳚क्षि॒तः । \newline
26. दी॒क्षि॒तो᳚ ऽन्त॒रा ऽन्त॒रा दी᳚क्षि॒तो दी᳚क्षि॒तो᳚ ऽन्त॒रा । \newline
27. अ॒न्त॒ रेवे॑ वान्त॒रा ऽन्त॒ रेव॑ । \newline
28. इ॒व॒ नव॑नीत॒म् नव॑नीत मिवेव॒ नव॑नीतम् । \newline
29. नव॑नीत॒म् तस्मा॒त् तस्मा॒न् नव॑नीत॒म् नव॑नीत॒म् तस्मा᳚त् । \newline
30. नव॑नीत॒मिति॒ नव॑ - नी॒त॒म् । \newline
31. तस्मा॒न् नव॑नीतेन॒ नव॑नीतेन॒ तस्मा॒त् तस्मा॒न् नव॑नीतेन । \newline
32. नव॑नीते ना॒भ्य॑भि नव॑नीतेन॒ नव॑नीते ना॒भि । \newline
33. नव॑नीते॒नेति॒ नव॑ - नी॒ते॒न॒ । \newline
34. अ॒भ्य॑ङ्क्ते ऽङ्क्ते॒ ऽभ्या᳚(1॒) भ्य॑ङ्क्ते । \newline
35. अ॒ङ्क्ते॒ ऽनु॒लो॒म म॑नुलो॒म म॑ङ्क्ते ऽङ्क्ते ऽनुलो॒मम् । \newline
36. अ॒नु॒लो॒मं ॅयजु॑षा॒ यजु॑षा ऽनुलो॒म म॑नुलो॒मं ॅयजु॑षा । \newline
37. अ॒नु॒लो॒ममित्य॑नु - लो॒मम् । \newline
38. यजु॑षा॒ व्यावृ॑त्त्यै॒ व्यावृ॑त्त्यै॒ यजु॑षा॒ यजु॑षा॒ व्यावृ॑त्त्यै । \newline
39. व्यावृ॑त्त्या॒ इन्द्र॒ इन्द्रो॒ व्यावृ॑त्त्यै॒ व्यावृ॑त्त्या॒ इन्द्रः॑ । \newline
40. व्यावृ॑त्त्या॒ इति॑ वि - आवृ॑त्त्यै । \newline
41. इन्द्रो॑ वृ॒त्रं ॅवृ॒त्र मिन्द्र॒ इन्द्रो॑ वृ॒त्रम् । \newline
42. वृ॒त्र म॑हन् नहन् वृ॒त्रं ॅवृ॒त्र म॑हन्न् । \newline
43. अ॒ह॒न् तस्य॒ तस्या॑हन् नह॒न् तस्य॑ । \newline
44. तस्य॑ क॒नीनि॑का क॒नीनि॑का॒ तस्य॒ तस्य॑ क॒नीनि॑का । \newline
45. क॒नीनि॑का॒ परा॒ परा॑ क॒नीनि॑का क॒नीनि॑का॒ परा᳚ । \newline
46. परा॑ ऽपत दपत॒त् परा॒ परा॑ ऽपतत् । \newline
47. अ॒प॒त॒त् तत् तद॑पत दपत॒त् तत् । \newline
48. तदाञ्ज॑न॒ माञ्ज॑न॒म् तत् तदाञ्ज॑नम् । \newline
49. आञ्ज॑न मभव दभव॒ दाञ्ज॑न॒ माञ्ज॑न मभवत् । \newline
50. आञ्ज॑न॒मित्या᳚ - अञ्ज॑नम् । \newline
51. अ॒भ॒व॒द् यद् यद॑भव दभव॒द् यत् । \newline
52. यदा॒ङ्क्त आ॒ङ्क्ते यद् यदा॒ङ्क्ते । \newline
53. आ॒ङ्क्ते चक्षु॒ श्चक्षु॑ रा॒ङ्क्त आ॒ङ्क्ते चक्षुः॑ । \newline
54. आ॒ङ्क्त इत्या᳚ - अ॒ङ्क्ते । \newline
55. चक्षु॑ रे॒वैव चक्षु॒ श्चक्षु॑ रे॒व । \newline
56. ए॒व भ्रातृ॑व्यस्य॒ भ्रातृ॑व्य स्यै॒वैव भ्रातृ॑व्यस्य । \newline
57. भ्रातृ॑व्यस्य वृङ्क्ते वृङ्क्ते॒ भ्रातृ॑व्यस्य॒ भ्रातृ॑व्यस्य वृङ्क्ते । \newline
58. वृ॒ङ्क्ते॒ दक्षि॑ण॒म् दक्षि॑णं ॅवृङ्क्ते वृङ्क्ते॒ दक्षि॑णम् । \newline
59. दक्षि॑ण॒म् पूर्व॒म् पूर्व॒म् दक्षि॑ण॒म् दक्षि॑ण॒म् पूर्व᳚म् । \newline
60. पूर्व॒ मा पूर्व॒म् पूर्व॒ मा । \newline
61. आ ऽङ्क्ते᳚ ऽङ्क्त॒ आ ऽङ्क्ते᳚ । \newline
62. अ॒ङ्क्ते॒ स॒व्यꣳ स॒व्य म॑ङ्क्ते ऽङ्क्ते स॒व्यम् । \newline

\textbf{Ghana Paata } \newline

1. ए॒तथ् स॑र्वदेव॒त्यꣳ॑ सर्वदेव॒त्य॑ मे॒त दे॒तथ् स॑र्वदेव॒त्यं॑ ॅयद् यथ् स॑र्वदेव॒त्य॑ मे॒त दे॒तथ् स॑र्वदेव॒त्यं॑ ॅयत् । \newline
2. स॒र्व॒दे॒व॒त्यं॑ ॅयद् यथ् स॑र्वदेव॒त्यꣳ॑ सर्वदेव॒त्यं॑ ॅयन् नव॑नीत॒म् नव॑नीतं॒ ॅयथ् स॑र्वदेव॒त्यꣳ॑ सर्वदेव॒त्यं॑ ॅयन् नव॑नीतम् । \newline
3. स॒र्व॒दे॒व॒त्य॑मिति॑ सर्व - दे॒व॒त्य᳚म् । \newline
4. यन् नव॑नीत॒म् नव॑नीतं॒ ॅयद् यन् नव॑नीतं॒ ॅयद् यन् नव॑नीतं॒ ॅयद् यन् नव॑नीतं॒ ॅयत् । \newline
5. नव॑नीतं॒ ॅयद् यन् नव॑नीत॒म् नव॑नीतं॒ ॅयन् नव॑नीतेन॒ नव॑नीतेन॒ यन् नव॑नीत॒म् नव॑नीतं॒ ॅयन् नव॑नीतेन । \newline
6. नव॑नीत॒मिति॒ नव॑ - नी॒त॒म् । \newline
7. यन् नव॑नीतेन॒ नव॑नीतेन॒ यद् यन् नव॑नीतेना भ्य॒ङ्क्ते᳚ ऽभ्य॒ङ्क्ते नव॑नीतेन॒ यद् यन् नव॑नीतेना भ्य॒ङ्क्ते । \newline
8. नव॑नीतेना भ्य॒ङ्क्ते᳚ ऽभ्य॒ङ्क्ते नव॑नीतेन॒ नव॑नीतेना भ्य॒ङ्क्ते सर्वाः॒ सर्वा॑ अभ्य॒ङ्क्ते नव॑नीतेन॒ नव॑नीतेना भ्य॒ङ्क्ते सर्वाः᳚ । \newline
9. नव॑नीते॒नेति॒ नव॑ - नी॒ते॒न॒ । \newline
10. अ॒भ्य॒ङ्क्ते सर्वाः॒ सर्वा॑ अभ्य॒ङ्क्ते᳚ ऽभ्य॒ङ्क्ते सर्वा॑ ए॒वैव सर्वा॑ अभ्य॒ङ्क्ते᳚ ऽभ्य॒ङ्क्ते सर्वा॑ ए॒व । \newline
11. अ॒भ्य॒ङ्क्त इत्य॑भि - अ॒ङ्क्ते । \newline
12. सर्वा॑ ए॒वैव सर्वाः॒ सर्वा॑ ए॒व दे॒वता॑ दे॒वता॑ ए॒व सर्वाः॒ सर्वा॑ ए॒व दे॒वताः᳚ । \newline
13. ए॒व दे॒वता॑ दे॒वता॑ ए॒वैव दे॒वताः᳚ प्रीणाति प्रीणाति दे॒वता॑ ए॒वैव दे॒वताः᳚ प्रीणाति । \newline
14. दे॒वताः᳚ प्रीणाति प्रीणाति दे॒वता॑ दे॒वताः᳚ प्रीणाति॒ प्रच्यु॑तः॒ प्रच्यु॑तः प्रीणाति दे॒वता॑ दे॒वताः᳚ प्रीणाति॒ प्रच्यु॑तः । \newline
15. प्री॒णा॒ति॒ प्रच्यु॑तः॒ प्रच्यु॑तः प्रीणाति प्रीणाति॒ प्रच्यु॑तो॒ वै वै प्रच्यु॑तः प्रीणाति प्रीणाति॒ प्रच्यु॑तो॒ वै । \newline
16. प्रच्यु॑तो॒ वै वै प्रच्यु॑तः॒ प्रच्यु॑तो॒ वा ए॒ष ए॒ष वै प्रच्यु॑तः॒ प्रच्यु॑तो॒ वा ए॒षः । \newline
17. प्रच्यु॑त॒ इति॒ प्र - च्यु॒तः॒ । \newline
18. वा ए॒ष ए॒ष वै वा ए॒षो᳚ ऽस्मा द॒स्मा दे॒ष वै वा ए॒षो᳚ ऽस्मात् । \newline
19. ए॒षो᳚ ऽस्मा द॒स्मा दे॒ष ए॒षो᳚ ऽस्माल् लो॒काल् लो॒का द॒स्मा दे॒ष ए॒षो᳚ ऽस्माल् लो॒कात् । \newline
20. अ॒स्माल् लो॒काल् लो॒का द॒स्मा द॒स्माल् लो॒का दग॒तो ऽग॑तो लो॒का द॒स्मा द॒स्माल् लो॒का दग॑तः । \newline
21. लो॒का दग॒तो ऽग॑तो लो॒काल् लो॒का दग॑तो देवलो॒कम् दे॑वलो॒क मग॑तो लो॒काल् लो॒का दग॑तो देवलो॒कम् । \newline
22. अग॑तो देवलो॒कम् दे॑वलो॒क मग॒तो ऽग॑तो देवलो॒कं ॅयो यो दे॑वलो॒क मग॒तो ऽग॑तो देवलो॒कं ॅयः । \newline
23. दे॒व॒लो॒कं ॅयो यो दे॑वलो॒कम् दे॑वलो॒कं ॅयो दी᳚क्षि॒तो दी᳚क्षि॒तो यो दे॑वलो॒कम् दे॑वलो॒कं ॅयो दी᳚क्षि॒तः । \newline
24. दे॒व॒लो॒कमिति॑ देव - लो॒कम् । \newline
25. यो दी᳚क्षि॒तो दी᳚क्षि॒तो यो यो दी᳚क्षि॒तो᳚ ऽन्त॒रा ऽन्त॒रा दी᳚क्षि॒तो यो यो दी᳚क्षि॒तो᳚ ऽन्त॒रा । \newline
26. दी॒क्षि॒तो᳚ ऽन्त॒रा ऽन्त॒रा दी᳚क्षि॒तो दी᳚क्षि॒तो᳚ ऽन्त॒रेवे॑ वान्त॒रा दी᳚क्षि॒तो दी᳚क्षि॒तो᳚ ऽन्त॒रेव॑ । \newline
27. अ॒न्त॒रेवे॑ वान्त॒रा ऽन्त॒रेव॒ नव॑नीत॒म् नव॑नीत मिवान्त॒रा ऽन्त॒रेव॒ नव॑नीतम् । \newline
28. इ॒व॒ नव॑नीत॒म् नव॑नीत मिवेव॒ नव॑नीत॒म् तस्मा॒त् तस्मा॒म् नव॑नीत मिवेव॒ नव॑नीत॒म् तस्मा᳚त् । \newline
29. नव॑नीत॒म् तस्मा॒त् तस्मा॒न् नव॑नीत॒म् नव॑नीत॒म् तस्मा॒न् नव॑नीतेन॒ नव॑नीतेन॒ तस्मा॒न् नव॑नीत॒म् नव॑नीत॒म् तस्मा॒न् नव॑नीतेन । \newline
30. नव॑नीत॒मिति॒ नव॑ - नी॒त॒म् । \newline
31. तस्मा॒न् नव॑नीतेन॒ नव॑नीतेन॒ तस्मा॒त् तस्मा॒न् नव॑नीतेना॒ भ्य॑भि नव॑नीतेन॒ तस्मा॒त् तस्मा॒न् नव॑नीते ना॒भि । \newline
32. नव॑नीतेना॒ भ्य॑भि नव॑नीतेन॒ नव॑नीते ना॒भ्य॑ङ्क्ते ऽङ्क्ते॒ ऽभि नव॑नीतेन॒ नव॑नीते ना॒भ्य॑ङ्क्ते । \newline
33. नव॑नीते॒नेति॒ नव॑ - नी॒ते॒न॒ । \newline
34. अ॒भ्य॑ङ्क्ते ऽङ्क्ते॒ ऽभ्या᳚(1॒)भ्य॑ङ्क्ते ऽनुलो॒म म॑नुलो॒म म॑ङ्क्ते॒ ऽभ्या᳚(1॒)भ्य॑ङ्क्ते ऽनुलो॒मम् । \newline
35. अ॒ङ्क्ते॒ ऽनु॒लो॒म म॑नुलो॒म म॑ङ्क्ते ऽङ्क्ते ऽनुलो॒मं ॅयजु॑षा॒ यजु॑षा ऽनुलो॒म म॑ङ्क्ते ऽङ्क्ते ऽनुलो॒मं ॅयजु॑षा । \newline
36. अ॒नु॒लो॒मं ॅयजु॑षा॒ यजु॑षा ऽनुलो॒म म॑नुलो॒मं ॅयजु॑षा॒ व्यावृ॑त्त्यै॒ व्यावृ॑त्त्यै॒ यजु॑षा ऽनुलो॒म म॑नुलो॒मं ॅयजु॑षा॒ व्यावृ॑त्त्यै । \newline
37. अ॒नु॒लो॒ममित्य॑नु - लो॒मम् । \newline
38. यजु॑षा॒ व्यावृ॑त्त्यै॒ व्यावृ॑त्त्यै॒ यजु॑षा॒ यजु॑षा॒ व्यावृ॑त्त्या॒ इन्द्र॒ इन्द्रो॒ व्यावृ॑त्त्यै॒ यजु॑षा॒ यजु॑षा॒ व्यावृ॑त्त्या॒ इन्द्रः॑ । \newline
39. व्यावृ॑त्त्या॒ इन्द्र॒ इन्द्रो॒ व्यावृ॑त्त्यै॒ व्यावृ॑त्त्या॒ इन्द्रो॑ वृ॒त्रं ॅवृ॒त्र मिन्द्रो॒ व्यावृ॑त्त्यै॒ व्यावृ॑त्त्या॒ इन्द्रो॑ वृ॒त्रम् । \newline
40. व्यावृ॑त्त्या॒ इति॑ वि - आवृ॑त्त्यै । \newline
41. इन्द्रो॑ वृ॒त्रं ॅवृ॒त्र मिन्द्र॒ इन्द्रो॑ वृ॒त्र म॑हन् नहन् वृ॒त्र मिन्द्र॒ इन्द्रो॑ वृ॒त्र म॑हन्न् । \newline
42. वृ॒त्र म॑हन् नहन् वृ॒त्रं ॅवृ॒त्र म॑ह॒न् तस्य॒ तस्या॑हन् वृ॒त्रं ॅवृ॒त्र म॑ह॒न् तस्य॑ । \newline
43. अ॒ह॒न् तस्य॒ तस्या॑हन् नह॒न् तस्य॑ क॒नीनि॑का क॒नीनि॑का॒ तस्या॑हन् नह॒न् तस्य॑ क॒नीनि॑का । \newline
44. तस्य॑ क॒नीनि॑का क॒नीनि॑का॒ तस्य॒ तस्य॑ क॒नीनि॑का॒ परा॒ परा॑ क॒नीनि॑का॒ तस्य॒ तस्य॑ क॒नीनि॑का॒ परा᳚ । \newline
45. क॒नीनि॑का॒ परा॒ परा॑ क॒नीनि॑का क॒नीनि॑का॒ परा॑ ऽपत दपत॒त् परा॑ क॒नीनि॑का क॒नीनि॑का॒ परा॑ ऽपतत् । \newline
46. परा॑ ऽपत दपत॒त् परा॒ परा॑ ऽपत॒त् तत् तद॑पत॒त् परा॒ परा॑ ऽपत॒त् तत् । \newline
47. अ॒प॒त॒त् तत् तद॑पत दपत॒त् तदाञ्ज॑न॒ माञ्ज॑न॒म् तद॑पत दपत॒त् तदाञ्ज॑नम् । \newline
48. तदाञ्ज॑न॒ माञ्ज॑न॒म् तत् तदाञ्ज॑न मभव दभव॒ दाञ्ज॑न॒म् तत् तदाञ्ज॑न मभवत् । \newline
49. आञ्ज॑न मभव दभव॒ दाञ्ज॑न॒ माञ्ज॑न मभव॒द् यद् यद॑भव॒ दाञ्ज॑न॒ माञ्ज॑न मभव॒द् यत् । \newline
50. आञ्ज॑न॒मित्या᳚ - अञ्ज॑नम् । \newline
51. अ॒भ॒व॒द् यद् यद॑भव दभव॒द् यदा॒ङ्क्त आ॒ङ्क्ते यद॑भव दभव॒द् यदा॒ङ्क्ते । \newline
52. यदा॒ङ्क्त आ॒ङ्क्ते यद् यदा॒ङ्क्ते चक्षु॒ श्चक्षु॑ रा॒ङ्क्ते यद् यदा॒ङ्क्ते चक्षुः॑ । \newline
53. आ॒ङ्क्ते चक्षु॒ श्चक्षु॑ रा॒ङ्क्त आ॒ङ्क्ते चक्षु॑ रे॒वैव चक्षु॑ रा॒ङ्क्त आ॒ङ्क्ते चक्षु॑ रे॒व । \newline
54. आ॒ङ्क्त इत्या᳚ - अ॒ङ्क्ते । \newline
55. चक्षु॑ रे॒वैव चक्षु॒ श्चक्षु॑ रे॒व भ्रातृ॑व्यस्य॒ भ्रातृ॑व्य स्यै॒व चक्षु॒ श्चक्षु॑ रे॒व भ्रातृ॑व्यस्य । \newline
56. ए॒व भ्रातृ॑व्यस्य॒ भ्रातृ॑व्य स्यै॒वैव भ्रातृ॑व्यस्य वृङ्क्ते वृङ्क्ते॒ भ्रातृ॑व्य स्यै॒वैव भ्रातृ॑व्यस्य वृङ्क्ते । \newline
57. भ्रातृ॑व्यस्य वृङ्क्ते वृङ्क्ते॒ भ्रातृ॑व्यस्य॒ भ्रातृ॑व्यस्य वृङ्क्ते॒ दक्षि॑ण॒म् दक्षि॑णं ॅवृङ्क्ते॒ भ्रातृ॑व्यस्य॒ भ्रातृ॑व्यस्य वृङ्क्ते॒ दक्षि॑णम् । \newline
58. वृ॒ङ्क्ते॒ दक्षि॑ण॒म् दक्षि॑णं ॅवृङ्क्ते वृङ्क्ते॒ दक्षि॑ण॒म् पूर्व॒म् पूर्व॒म् दक्षि॑णं ॅवृङ्क्ते वृङ्क्ते॒ दक्षि॑ण॒म् पूर्व᳚म् । \newline
59. दक्षि॑ण॒म् पूर्व॒म् पूर्व॒म् दक्षि॑ण॒म् दक्षि॑ण॒म् पूर्व॒ मा पूर्व॒म् दक्षि॑ण॒म् दक्षि॑ण॒म् पूर्व॒ मा । \newline
60. पूर्व॒ मा पूर्व॒म् पूर्व॒ मा ऽङ्क्ते᳚ ऽङ्क्त॒ आ पूर्व॒म् पूर्व॒ मा ऽङ्क्ते᳚ । \newline
61. आ ऽङ्क्ते᳚ ऽङ्क्त॒ आ ऽङ्क्ते॑ स॒व्यꣳ स॒व्य म॑ङ्क्त॒ आ ऽङ्क्ते॑ स॒व्यम् । \newline
62. अ॒ङ्क्ते॒ स॒व्यꣳ स॒व्य म॑ङ्क्ते ऽङ्क्ते स॒व्यꣳ हि हि स॒व्य म॑ङ्क्ते ऽङ्क्ते स॒व्यꣳ हि । \newline
\pagebreak
\markright{ TS 6.1.1.6  \hfill https://www.vedavms.in \hfill}

\section{ TS 6.1.1.6 }

\textbf{TS 6.1.1.6 } \newline
\textbf{Samhita Paata} \newline

स॒व्यꣳ हि पूर्वं॑ मनु॒ष्या॑ आ॒ञ्जते॒ न नि धा॑वते॒ नीव॒ हि म॑नु॒ष्या॑ धाव॑न्ते॒ पञ्च॒ कृत्व॒ आऽङ्क्ते॒ पञ्चा᳚क्षरा प॒ङ्क्तिः पाङ्क्तो॑ य॒ज्ञो य॒ज्ञ्मे॒वाव॑ रुन्धे॒ परि॑मित॒माङ्क्ते ऽप॑रिमितꣳ॒॒ हि म॑नु॒ष्या॑ आ॒ञ्जते॒ सतू॑ल॒याऽऽङ्क्ते- ऽप॑तूलया॒ हि म॑नु॒ष्या॑ आ॒ञ्जते॒ व्यावृ॑त्त्यै॒ यदप॑तूलयाञ्जी॒त वज्र॑ इव स्या॒थ् सतू॑ल॒याऽऽङ्क्ते॑ मित्र॒त्वाये - [  ] \newline

\textbf{Pada Paata} \newline

स॒व्यम् । हि । पूर्व᳚म् । म॒नु॒ष्याः᳚ । आ॒ञ्जत॒ इत्या᳚ - अ॒ञ्जते᳚ । न । नीति॑ । धा॒व॒ते॒ । नीति॑ । इ॒व॒ । हि । म॒नु॒ष्याः᳚ । धाव॑न्ते । पञ्च॑ । कृत्वः॑ । एति॑ । अ॒ङ्क्ते॒ । पञ्चा᳚क्ष॒रेति॒ पञ्च॑-अ॒क्ष॒रा॒ । प॒ङ्क्तिः । पाङ्क्तः॑ । य॒ज्ञ्ः । य॒ज्ञ्म् । ए॒व । अवेति॑ । रु॒न्धे॒ । परि॑मित॒मिति॒ परि॑ - मि॒त॒म् । एति॑ । अ॒ङ्क्ते॒ । अप॑रिमित॒मित्यप॑रि - मि॒त॒म् । हि । म॒नु॒ष्याः᳚ । आ॒ञ्जत॒ इत्या᳚ - अ॒ञ्जते᳚ । सतू॑ल॒येति॒ स-तू॒ल॒या॒ । एति॑ । अ॒ङ्क्ते॒ । अप॑तूल॒येत्यप॑-तू॒ल॒या॒ । हि । म॒नु॒ष्याः᳚ । आ॒ञ्जत॒ इत्या᳚ - अ॒ञ्जते᳚ । व्यावृ॑त्त्या॒ इति॑ वि - आवृ॑त्त्यै । यत् । अप॑तूल॒येत्यप॑ - तू॒ल॒या॒ । आ॒ञ्जी॒तेत्या᳚ - अ॒ञ्जी॒त । वज्रः॑ । इ॒व॒ । स्या॒त् । सतू॑ल॒येति॒ स - तू॒ल॒या॒ । एति॑ । अ॒ङ्क्ते॒ । मि॒त्र॒त्वायेति॑ मित्र - त्वाय॑ ।  \newline


\textbf{Krama Paata} \newline

स॒व्यꣳ हि । हि पूर्व᳚म् । पूर्व॑म् मनु॒ष्याः᳚ । म॒नु॒ष्या॑ आ॒ञ्जते᳚ । आ॒ञ्जते॒ न । आ॒ञ्जत॒ इत्या᳚ - अ॒ञ्जते᳚ । न नि । नि धा॑वते । धा॒व॒ते॒ नि । नीव॑ । इ॒व॒ हि । हि म॑नु॒ष्याः᳚ । म॒नु॒ष्या॑ धाव॑न्ते । धाव॑न्ते॒ पञ्च॑ । पञ्च॒ कृत्वः॑ । कृत्व॒ आ । आऽङ्‍क्ते᳚ । अ॒ङ्‍क्ते॒ पञ्चा᳚क्षरा । पञ्चा᳚क्षरा प॒ङ्‍क्तिः । पञ्चा᳚क्ष॒रेति॒ पञ्च॑ - अ॒क्ष॒रा॒ । प॒ङ्‍क्तिः पाङ्‍क्तः॑ । पाङ्‍क्तो॑ य॒ज्ञ्ः । य॒ज्ञो य॒ज्ञ्म् । य॒ज्ञ्मे॒व । ए॒वाव॑ । अव॑ रुन्धे । रु॒न्धे॒ परि॑मितम् । परि॑मित॒मा । परि॑मित॒मिति॒ परि॑ - मि॒त॒म् । आऽङ्‍क्ते᳚ । अ॒ङ्‍क्तेऽप॑रिमितम् । अप॑रिमितꣳ॒॒ हि । अप॑रिमित॒मित्यप॑रि - मि॒त॒म् । हि म॑नु॒ष्याः᳚ । म॒नु॒ष्या॑ आ॒ञ्जते᳚ । आ॒ञ्जते॒ सतू॑लया । आ॒ञ्जत॒ इत्या᳚ - अ॒ञ्जते᳚ । सतू॑ल॒या । सतू॑ल॒येति॒ स - तू॒ल॒या॒ । आऽङ्‍क्ते᳚ । अ॒ङ्‍क्तेऽप॑तूलया । अप॑तूलया॒ हि । अप॑तूल॒येत्यप॑ - तू॒ल॒या॒ । हि म॑नु॒ष्याः᳚ । म॒नु॒ष्या॑ आ॒ञ्जते᳚ । आ॒ञ्जते॒ व्यावृ॑त्त्यै । आ॒ञ्जत॒ इत्या᳚ - अ॒ञ्जते᳚ । व्यावृ॑त्त्यै॒ यत् । व्यावृ॑त्त्या॒ इति॑ वि - आवृ॑त्त्यै । यदप॑तूलया । अप॑तूलयाऽऽञ्जी॒त । अप॑तूल॒येत्यप॑ - तू॒ल॒या॒ । आ॒ञ्जी॒त वज्रः॑ । आ॒ञ्जी॒तेत्या᳚ - अ॒ञ्जी॒त । वज्र॑ इव । इ॒व॒ स्या॒त्॒ । स्या॒थ् सतू॑लया । सतू॑ल॒या । सतू॑ल॒येति॒ स - तू॒ल॒या॒ । आऽङ्‍क्ते᳚ । अ॒ङ्‍क्ते॒ मि॒त्र॒त्वाय॑ । मि॒त्र॒त्वायेन्द्रः॑ । मि॒त्र॒त्वायेति॑ मित्र - त्वाय॑ \newline

\textbf{Jatai Paata} \newline

1. स॒व्यꣳ हि हि स॒व्यꣳ स॒व्यꣳ हि । \newline
2. हि पूर्व॒म् पूर्वꣳ॒॒ हि हि पूर्व᳚म् । \newline
3. पूर्व॑म् मनु॒ष्या॑ मनु॒ष्याः᳚ पूर्व॒म् पूर्व॑म् मनु॒ष्याः᳚ । \newline
4. म॒नु॒ष्या॑ आ॒ञ्जत॑ आ॒ञ्जते॑ मनु॒ष्या॑ मनु॒ष्या॑ आ॒ञ्जते᳚ । \newline
5. आ॒ञ्जते॒ न नाञ्जत॑ आ॒ञ्जते॒ न । \newline
6. आ॒ञ्जत॒ इत्या᳚ - अ॒ञ्जते᳚ । \newline
7. न नि नि न न नि । \newline
8. नि धा॑वते धावते॒ नि नि धा॑वते । \newline
9. धा॒व॒ते॒ नि नि धा॑वते धावते॒ नि । \newline
10. नीवे॑व॒ नि नीव॑ । \newline
11. इ॒व॒ हि हीवे॑व॒ हि । \newline
12. हि म॑नु॒ष्या॑ मनु॒ष्या॑ हि हि म॑नु॒ष्याः᳚ । \newline
13. म॒नु॒ष्या॑ धाव॑न्ते॒ धाव॑न्ते मनु॒ष्या॑ मनु॒ष्या॑ धाव॑न्ते । \newline
14. धाव॑न्ते॒ पञ्च॒ पञ्च॒ धाव॑न्ते॒ धाव॑न्ते॒ पञ्च॑ । \newline
15. पञ्च॒ कृत्वः॒ कृत्वः॒ पञ्च॒ पञ्च॒ कृत्वः॑ । \newline
16. कृत्व॒ आ कृत्वः॒ कृत्व॒ आ । \newline
17. आ ऽङ्क्ते᳚ ऽङ्क्त॒ आ ऽङ्क्ते᳚ । \newline
18. अ॒ङ्क्ते॒ पञ्चा᳚क्षरा॒ पञ्चा᳚क्षरा ऽङ्क्ते ऽङ्क्ते॒ पञ्चा᳚क्षरा । \newline
19. पञ्चा᳚क्षरा प॒ङ्क्तिः प॒ङ्क्तिः पञ्चा᳚क्षरा॒ पञ्चा᳚क्षरा प॒ङ्क्तिः । \newline
20. पञ्चा᳚क्ष॒रेति॒ पञ्च॑ - अ॒क्ष॒रा॒ । \newline
21. प॒ङ्क्तिः पाङ्क्तः॒ पाङ्क्तः॑ प॒ङ्क्तिः प॒ङ्क्तिः पाङ्क्तः॑ । \newline
22. पाङ्क्तो॑ य॒ज्ञो य॒ज्ञ्ः पाङ्क्तः॒ पाङ्क्तो॑ य॒ज्ञ्ः । \newline
23. य॒ज्ञो य॒ज्ञ्ं ॅय॒ज्ञ्ं ॅय॒ज्ञो य॒ज्ञो य॒ज्ञ्म् । \newline
24. य॒ज्ञ् मे॒वैव य॒ज्ञ्ं ॅय॒ज्ञ् मे॒व । \newline
25. ए॒वावा वै॒वै वाव॑ । \newline
26. अव॑ रुन्धे रु॒न्धे ऽवाव॑ रुन्धे । \newline
27. रु॒न्धे॒ परि॑मित॒म् परि॑मितꣳ रुन्धे रुन्धे॒ परि॑मितम् । \newline
28. परि॑मित॒ मा परि॑मित॒म् परि॑मित॒ मा । \newline
29. परि॑मित॒मिति॒ परि॑ - मि॒त॒म् । \newline
30. आ ऽङ्क्ते᳚ ऽङ्क्त॒ आ ऽङ्क्ते᳚ । \newline
31. अ॒ङ्क्ते ऽप॑रिमित॒ मप॑रिमित मङ्क्ते॒ ऽङ्क्ते ऽप॑रिमितम् । \newline
32. अप॑रिमितꣳ॒॒ हि ह्यप॑रिमित॒ मप॑रिमितꣳ॒॒ हि । \newline
33. अप॑रिमित॒मित्यप॑रि - मि॒त॒म् । \newline
34. हि म॑नु॒ष्या॑ मनु॒ष्या॑ हि हि म॑नु॒ष्याः᳚ । \newline
35. म॒नु॒ष्या॑ आ॒ञ्जत॑ आ॒ञ्जते॑ मनु॒ष्या॑ मनु॒ष्या॑ आ॒ञ्जते᳚ । \newline
36. आ॒ञ्जते॒ सतू॑लया॒ सतू॑लया॒ ऽऽञ्जत॑ आ॒ञ्जते॒ सतू॑लया । \newline
37. आ॒ञ्जत॒ इत्या᳚ - अ॒ञ्जते᳚ । \newline
38. सतू॑ल॒या ऽऽसतू॑लया॒ सतू॑ल॒या । \newline
39. सतू॑ल॒येति॒ स - तू॒ल॒या॒ । \newline
40. आ ऽङ्क्ते᳚ ऽङ्क्त॒ आ ऽङ्क्ते᳚ । \newline
41. अ॒ङ्क्ते ऽप॑तूल॒या ऽप॑तूलया ऽङ्क्ते॒ ऽङ्क्ते ऽप॑तूलया । \newline
42. अप॑तूलया॒ हि ह्यप॑तूल॒या ऽप॑तूलया॒ हि । \newline
43. अप॑तूल॒येत्यप॑ - तू॒ल॒या॒ । \newline
44. हि म॑नु॒ष्या॑ मनु॒ष्या॑ हि हि म॑नु॒ष्याः᳚ । \newline
45. म॒नु॒ष्या॑ आ॒ञ्जत॑ आ॒ञ्जते॑ मनु॒ष्या॑ मनु॒ष्या॑ आ॒ञ्जते᳚ । \newline
46. आ॒ञ्जते॒ व्यावृ॑त्त्यै॒ व्यावृ॑त्त्या आ॒ञ्जत॑ आ॒ञ्जते॒ व्यावृ॑त्त्यै । \newline
47. आ॒ञ्जत॒ इत्या᳚ - अ॒ञ्जते᳚ । \newline
48. व्यावृ॑त्त्यै॒ यद् यद् व्यावृ॑त्त्यै॒ व्यावृ॑त्त्यै॒ यत् । \newline
49. व्यावृ॑त्त्या॒ इति॑ वि - आवृ॑त्त्यै । \newline
50. यदप॑तूल॒या ऽप॑तूलया॒ यद् यदप॑तूलया । \newline
51. अप॑तूलया ऽऽञ्जी॒ ताञ्जी॒ ताप॑तूल॒या ऽप॑तूलया ऽऽञ्जी॒त । \newline
52. अप॑तूल॒येत्यप॑ - तू॒ल॒या॒ । \newline
53. आ॒ञ्जी॒त वज्रो॒ वज्र॑ आञ्जी॒ताञ्जी॒त वज्रः॑ । \newline
54. आ॒ञ्जी॒तेत्या᳚ - अ॒ञ्जी॒त । \newline
55. वज्र॑ इवेव॒ वज्रो॒ वज्र॑ इव । \newline
56. इ॒व॒ स्या॒थ् स्या॒ दि॒वे॒व॒ स्या॒त् । \newline
57. स्या॒थ् सतू॑लया॒ सतू॑लया स्याथ् स्या॒थ् सतू॑लया । \newline
58. सतू॑ल॒या ऽऽसतू॑लया॒ सतू॑ल॒या । \newline
59. सतू॑ल॒येति॒ स - तू॒ल॒या॒ । \newline
60. आ ऽङ्क्ते᳚ ऽङ्क्त॒ आ ऽङ्क्ते᳚ । \newline
61. अ॒ङ्क्ते॒ मि॒त्र॒त्वाय॑ मित्र॒त्वाया᳚ङ्क्ते ऽङ्क्ते मित्र॒त्वाय॑ । \newline
62. मि॒त्र॒त्वायेन्द्र॒ इन्द्रो॑ मित्र॒त्वाय॑ मित्र॒त्वायेन्द्रः॑ । \newline
63. मि॒त्र॒त्वायेति॑ मित्र - त्वाय॑ । \newline

\textbf{Ghana Paata } \newline

1. स॒व्यꣳ हि हि स॒व्यꣳ स॒व्यꣳ हि पूर्व॒म् पूर्वꣳ॒॒ हि स॒व्यꣳ स॒व्यꣳ हि पूर्व᳚म् । \newline
2. हि पूर्व॒म् पूर्वꣳ॒॒ हि हि पूर्व॑म् मनु॒ष्या॑ मनु॒ष्याः᳚ पूर्वꣳ॒॒ हि हि पूर्व॑म् मनु॒ष्याः᳚ । \newline
3. पूर्व॑म् मनु॒ष्या॑ मनु॒ष्याः᳚ पूर्व॒म् पूर्व॑म् मनु॒ष्या॑ आ॒ञ्जत॑ आ॒ञ्जते॑ मनु॒ष्याः᳚ पूर्व॒म् पूर्व॑म् मनु॒ष्या॑ आ॒ञ्जते᳚ । \newline
4. म॒नु॒ष्या॑ आ॒ञ्जत॑ आ॒ञ्जते॑ मनु॒ष्या॑ मनु॒ष्या॑ आ॒ञ्जते॒ न नाञ्जते॑ मनु॒ष्या॑ मनु॒ष्या॑ आ॒ञ्जते॒ न । \newline
5. आ॒ञ्जते॒ न नाञ्जत॑ आ॒ञ्जते॒ न नि नि नाञ्जत॑ आ॒ञ्जते॒ न नि । \newline
6. आ॒ञ्जत॒ इत्या᳚ - अ॒ञ्जते᳚ । \newline
7. न नि नि न न नि धा॑वते धावते॒ नि न न नि धा॑वते । \newline
8. नि धा॑वते धावते॒ नि नि धा॑वते॒ नि नि धा॑वते॒ नि नि धा॑वते॒ नि । \newline
9. धा॒व॒ते॒ नि नि धा॑वते धावते॒ नीवे॑व॒ नि धा॑वते धावते॒ नीव॑ । \newline
10. नीवे॑व॒ नि नीव॒ हि हीव॒ नि नीव॒ हि । \newline
11. इ॒व॒ हि हीवे॑व॒ हि म॑नु॒ष्या॑ मनु॒ष्या॑ हीवे॑व॒ हि म॑नु॒ष्याः᳚ । \newline
12. हि म॑नु॒ष्या॑ मनु॒ष्या॑ हि हि म॑नु॒ष्या॑ धाव॑न्ते॒ धाव॑न्ते मनु॒ष्या॑ हि हि म॑नु॒ष्या॑ धाव॑न्ते । \newline
13. म॒नु॒ष्या॑ धाव॑न्ते॒ धाव॑न्ते मनु॒ष्या॑ मनु॒ष्या॑ धाव॑न्ते॒ पञ्च॒ पञ्च॒ धाव॑न्ते मनु॒ष्या॑ मनु॒ष्या॑ धाव॑न्ते॒ पञ्च॑ । \newline
14. धाव॑न्ते॒ पञ्च॒ पञ्च॒ धाव॑न्ते॒ धाव॑न्ते॒ पञ्च॒ कृत्वः॒ कृत्वः॒ पञ्च॒ धाव॑न्ते॒ धाव॑न्ते॒ पञ्च॒ कृत्वः॑ । \newline
15. पञ्च॒ कृत्वः॒ कृत्वः॒ पञ्च॒ पञ्च॒ कृत्व॒ आ कृत्वः॒ पञ्च॒ पञ्च॒ कृत्व॒ आ । \newline
16. कृत्व॒ आ कृत्वः॒ कृत्व॒ आ ऽङ्क्ते᳚ ऽङ्क्त॒ आ कृत्वः॒ कृत्व॒ आ ऽङ्क्ते᳚ । \newline
17. आ ऽङ्क्ते᳚ ऽङ्क्त॒ आ ऽङ्क्ते॒ पञ्चा᳚क्षरा॒ पञ्चा᳚क्षरा ऽङ्क्त॒ आ ऽङ्क्ते॒ पञ्चा᳚क्षरा । \newline
18. अ॒ङ्क्ते॒ पञ्चा᳚क्षरा॒ पञ्चा᳚क्षरा ऽङ्क्ते ऽङ्क्ते॒ पञ्चा᳚क्षरा प॒ङ्क्तिः प॒ङ्क्तिः पञ्चा᳚क्षरा ऽङ्क्ते ऽङ्क्ते॒ पञ्चा᳚क्षरा प॒ङ्क्तिः । \newline
19. पञ्चा᳚क्षरा प॒ङ्क्तिः प॒ङ्क्तिः पञ्चा᳚क्षरा॒ पञ्चा᳚क्षरा प॒ङ्क्तिः पाङ्क्तः॒ पाङ्क्तः॑ प॒ङ्क्तिः पञ्चा᳚क्षरा॒ पञ्चा᳚क्षरा प॒ङ्क्तिः पाङ्क्तः॑ । \newline
20. पञ्चा᳚क्ष॒रेति॒ पञ्च॑ - अ॒क्ष॒रा॒ । \newline
21. प॒ङ्क्तिः पाङ्क्तः॒ पाङ्क्तः॑ प॒ङ्क्तिः प॒ङ्क्तिः पाङ्क्तो॑ य॒ज्ञो य॒ज्ञ्ः पाङ्क्तः॑ प॒ङ्क्तिः प॒ङ्क्तिः पाङ्क्तो॑ य॒ज्ञ्ः । \newline
22. पाङ्क्तो॑ य॒ज्ञो य॒ज्ञ्ः पाङ्क्तः॒ पाङ्क्तो॑ य॒ज्ञो य॒ज्ञ्ं ॅय॒ज्ञ्ं ॅय॒ज्ञ्ः पाङ्क्तः॒ पाङ्क्तो॑ य॒ज्ञो य॒ज्ञ्म् । \newline
23. य॒ज्ञो य॒ज्ञ्ं ॅय॒ज्ञ्ं ॅय॒ज्ञो य॒ज्ञो य॒ज्ञ् मे॒वैव य॒ज्ञ्ं ॅय॒ज्ञो य॒ज्ञो य॒ज्ञ् मे॒व । \newline
24. य॒ज्ञ् मे॒वैव य॒ज्ञ्ं ॅय॒ज्ञ् मे॒वावा वै॒व य॒ज्ञ्ं ॅय॒ज्ञ् मे॒वाव॑ । \newline
25. ए॒वावा वै॒वै वाव॑ रुन्धे रु॒न्धे ऽवै॒वै वाव॑ रुन्धे । \newline
26. अव॑ रुन्धे रु॒न्धे ऽवाव॑ रुन्धे॒ परि॑मित॒म् परि॑मितꣳ रु॒न्धे ऽवाव॑ रुन्धे॒ परि॑मितम् । \newline
27. रु॒न्धे॒ परि॑मित॒म् परि॑मितꣳ रुन्धे रुन्धे॒ परि॑मित॒ मा परि॑मितꣳ रुन्धे रुन्धे॒ परि॑मित॒ मा । \newline
28. परि॑मित॒ मा परि॑मित॒म् परि॑मित॒ मा ऽङ्क्ते᳚ ऽङ्क्त॒ आ परि॑मित॒म् परि॑मित॒ मा ऽङ्क्ते᳚ । \newline
29. परि॑मित॒मिति॒ परि॑ - मि॒त॒म् । \newline
30. आ ऽङ्क्ते᳚ ऽङ्क्त॒ आ ऽङ्क्ते ऽप॑रिमित॒ मप॑रिमित मङ्क्त॒ आ ऽङ्क्ते ऽप॑रिमितम् । \newline
31. अ॒ङ्क्ते ऽप॑रिमित॒ मप॑रिमित मङ्क्ते॒ ऽङ्क्ते ऽप॑रिमितꣳ॒॒ हि ह्यप॑रिमित मङ्क्ते॒ ऽङ्क्ते ऽप॑रिमितꣳ॒॒ हि । \newline
32. अप॑रिमितꣳ॒॒ हि ह्यप॑रिमित॒ मप॑रिमितꣳ॒॒ हि म॑नु॒ष्या॑ मनु॒ष्या᳚ ह्यप॑रिमित॒ मप॑रिमितꣳ॒॒ हि म॑नु॒ष्याः᳚ । \newline
33. अप॑रिमित॒मित्यप॑रि - मि॒त॒म् । \newline
34. हि म॑नु॒ष्या॑ मनु॒ष्या॑ हि हि म॑नु॒ष्या॑ आ॒ञ्जत॑ आ॒ञ्जते॑ मनु॒ष्या॑ हि हि म॑नु॒ष्या॑ आ॒ञ्जते᳚ । \newline
35. म॒नु॒ष्या॑ आ॒ञ्जत॑ आ॒ञ्जते॑ मनु॒ष्या॑ मनु॒ष्या॑ आ॒ञ्जते॒ सतू॑लया॒ सतू॑लया॒ ऽऽञ्जते॑ मनु॒ष्या॑ मनु॒ष्या॑ आ॒ञ्जते॒ सतू॑लया । \newline
36. आ॒ञ्जते॒ सतू॑लया॒ सतू॑लया॒ ऽऽञ्जत॑ आ॒ञ्जते॒ सतू॑ल॒या ऽऽसतू॑लया॒ ऽऽञ्जत॑ आ॒ञ्जते॒ सतू॑ल॒या । \newline
37. आ॒ञ्जत॒ इत्या᳚ - अ॒ञ्जते᳚ । \newline
38. सतू॑ल॒या ऽऽसतू॑लया॒ सतू॑ल॒या ऽङ्क्ते᳚ ऽङ्क्त॒ आ सतू॑लया॒ सतू॑ल॒या ऽङ्क्ते᳚ । \newline
39. सतू॑ल॒येति॒ स - तू॒ल॒या॒ । \newline
40. आ ऽङ्क्ते᳚ ऽङ्क्त॒ आ ऽङ्क्ते ऽप॑तूल॒या ऽप॑तूलया ऽङ्क्त॒ आ ऽङ्क्ते ऽप॑तूलया । \newline
41. अ॒ङ्क्ते ऽप॑तूल॒या ऽप॑तूलया ऽङ्क्ते॒ ऽङ्क्ते ऽप॑तूलया॒ हि ह्यप॑तूलया ऽङ्क्ते॒ ऽङ्क्ते ऽप॑तूलया॒ हि । \newline
42. अप॑तूलया॒ हि ह्यप॑तूल॒या ऽप॑तूलया॒ हि म॑नु॒ष्या॑ मनु॒ष्या᳚ ह्यप॑तूल॒या ऽप॑तूलया॒ हि म॑नु॒ष्याः᳚ । \newline
43. अप॑तूल॒येत्यप॑ - तू॒ल॒या॒ । \newline
44. हि म॑नु॒ष्या॑ मनु॒ष्या॑ हि हि म॑नु॒ष्या॑ आ॒ञ्जत॑ आ॒ञ्जते॑ मनु॒ष्या॑ हि हि म॑नु॒ष्या॑ आ॒ञ्जते᳚ । \newline
45. म॒नु॒ष्या॑ आ॒ञ्जत॑ आ॒ञ्जते॑ मनु॒ष्या॑ मनु॒ष्या॑ आ॒ञ्जते॒ व्यावृ॑त्त्यै॒ व्यावृ॑त्त्या आ॒ञ्जते॑ मनु॒ष्या॑ मनु॒ष्या॑ आ॒ञ्जते॒ व्यावृ॑त्त्यै । \newline
46. आ॒ञ्जते॒ व्यावृ॑त्त्यै॒ व्यावृ॑त्त्या आ॒ञ्जत॑ आ॒ञ्जते॒ व्यावृ॑त्त्यै॒ यद् यद् व्यावृ॑त्त्या आ॒ञ्जत॑ आ॒ञ्जते॒ व्यावृ॑त्त्यै॒ यत् । \newline
47. आ॒ञ्जत॒ इत्या᳚ - अ॒ञ्जते᳚ । \newline
48. व्यावृ॑त्त्यै॒ यद् यद् व्यावृ॑त्त्यै॒ व्यावृ॑त्त्यै॒ यदप॑तूल॒या ऽप॑तूलया॒ यद् व्यावृ॑त्त्यै॒ व्यावृ॑त्त्यै॒ यदप॑तूलया । \newline
49. व्यावृ॑त्त्या॒ इति॑ वि - आवृ॑त्त्यै । \newline
50. यदप॑तूल॒या ऽप॑तूलया॒ यद् यदप॑तूलया ऽऽञ्जी॒ता ञ्जी॒ता प॑तूलया॒ यद् यदप॑तूलया ऽऽञ्जी॒त । \newline
51. अप॑तूलया ऽऽञ्जी॒ता ञ्जी॒ता प॑तूल॒या ऽप॑तूलया ऽऽञ्जी॒त वज्रो॒ वज्र॑ आञ्जी॒ता प॑तूल॒या ऽप॑तूलया ऽऽञ्जी॒त वज्रः॑ । \newline
52. अप॑तूल॒येत्यप॑ - तू॒ल॒या॒ । \newline
53. आ॒ञ्जी॒त वज्रो॒ वज्र॑ आञ्जी॒ता ञ्जी॒त वज्र॑ इवेव॒ वज्र॑ आञ्जी॒ता ञ्जी॒त वज्र॑ इव । \newline
54. आ॒ञ्जी॒तेत्या᳚ - अ॒ञ्जी॒त । \newline
55. वज्र॑ इवेव॒ वज्रो॒ वज्र॑ इव स्याथ् स्यादिव॒ वज्रो॒ वज्र॑ इव स्यात् । \newline
56. इ॒व॒ स्या॒थ् स्या॒दि॒वे॒व॒ स्या॒थ् सतू॑लया॒ सतू॑लया स्यादिवेव स्या॒थ् सतू॑लया । \newline
57. स्या॒थ् सतू॑लया॒ सतू॑लया स्याथ् स्या॒थ् सतू॑ल॒या ऽऽसतू॑लया स्याथ् स्या॒थ् सतू॑ल॒या । \newline
58. सतू॑ल॒या ऽऽसतू॑लया॒ सतू॑ल॒या ऽङ्क्ते᳚ ऽङ्क्त॒ आ सतू॑लया॒ सतू॑ल॒या ऽङ्क्ते᳚ । \newline
59. सतू॑ल॒येति॒ स - तू॒ल॒या॒ । \newline
60. आ ऽङ्क्ते᳚ ऽङ्क्त॒ आ ऽङ्क्ते॑ मित्र॒त्वाय॑ मित्र॒त्वाया᳚ङ्क्त॒ आ ऽङ्क्ते॑ मित्र॒त्वाय॑ । \newline
61. अ॒ङ्क्ते॒ मि॒त्र॒त्वाय॑ मित्र॒त्वाया᳚ङ्क्ते ऽङ्क्ते मित्र॒त्वायेन्द्र॒ इन्द्रो॑ मित्र॒त्वाया᳚ङ्क्ते ऽङ्क्ते मित्र॒त्वायेन्द्रः॑ । \newline
62. मि॒त्र॒त्वायेन्द्र॒ इन्द्रो॑ मित्र॒त्वाय॑ मित्र॒त्वायेन्द्रो॑ वृ॒त्रं ॅवृ॒त्र मिन्द्रो॑ मित्र॒त्वाय॑ मित्र॒त्वायेन्द्रो॑ वृ॒त्रम् । \newline
63. मि॒त्र॒त्वायेति॑ मित्र - त्वाय॑ । \newline
\pagebreak
\markright{ TS 6.1.1.7  \hfill https://www.vedavms.in \hfill}

\section{ TS 6.1.1.7 }

\textbf{TS 6.1.1.7 } \newline
\textbf{Samhita Paata} \newline

-न्द्रो॑ वृ॒त्रम॑ह॒न्थ्सो᳚ऽ(1॒)पो᳚ऽ(1॒)भ्य॑-म्रियत॒ तासां॒ ॅयन्मेद्ध्यं॑ ॅय॒ज्ञियꣳ॒॒ सदे॑व॒मासी॒त् तद॒पोद॑क्राम॒त् ते द॒र्भा अ॑भव॒न॒. यद्द॑र्भपुञ्जी॒लैः प॒वय॑ति॒ या ए॒व मेद्ध्या॑ य॒ज्ञियाः॒ सदे॑वा॒ आप॒स्ताभि॑रे॒वैनं॑ पवयति॒ द्वाभ्यां᳚ पवयत्य-होरा॒त्राभ्या॑मे॒वैनं॑ पवयति त्रि॒भिः प॑वयति॒ त्रय॑ इ॒मे लो॒का ए॒भिरे॒वैनं॑ ॅलो॒कैः प॑वयति प॒ञ्चभिः॑ - [  ] \newline

\textbf{Pada Paata} \newline

इन्द्रः॑ । वृ॒त्रम् । अ॒ह॒न्न् । सः । अ॒पः । अ॒भीति॑ । अ॒म्रि॒य॒त॒ । तासा᳚म् । यत् । मेद्ध्य᳚म् । य॒ज्ञिय᳚म् । सदे॑व॒मिति॒ स - दे॒व॒म् । आसी᳚त् । तत् । अप॑ । उदिति॑ । अ॒क्रा॒म॒त् । ते । द॒र्भाः । अ॒भ॒व॒न्न् । यत् । द॒र्भ॒पु॒ञ्जी॒लैरिति॑ दर्भ - पु॒ञ्जी॒लैः । प॒वय॑ति । याः । ए॒व । मेद्ध्याः᳚ । य॒ज्ञियाः᳚ । सदे॑वा॒ इति॒ स - दे॒वाः॒ । आपः॑ । ताभिः॑ । ए॒व । ए॒न॒म् । प॒व॒य॒ति॒ । द्वाभ्या᳚म् । प॒व॒य॒ति॒ । अ॒हो॒रा॒त्राभ्या॒मित्य॑हः-रा॒त्राभ्या᳚म् । ए॒व । ए॒न॒म् । प॒व॒य॒ति॒ । त्रि॒भिरिति॑ त्रि - भिः । प॒व॒य॒ति॒ । त्रयः॑ । इ॒मे । लो॒काः । ए॒भिः । ए॒व । ए॒न॒म् । लो॒कैः । प॒व॒य॒ति॒ । प॒ञ्चभि॒रिति॑ प॒ञ्च - भिः॒ ।  \newline


\textbf{Krama Paata} \newline

इन्द्रो॑ वृ॒त्रम् । वृ॒त्रम॑हन्न् । अ॒ह॒न्थ् सः । सो॑ऽपः । अ॒पो॑ऽभि । अ॒भ्य॑म्रियत । अ॒म्रि॒य॒त॒ तासा᳚म् । तासा॒म् ॅयत् । यन् मेद्ध्य᳚म् । मेद्ध्य॑म् ॅय॒ज्ञिय᳚म् । य॒ज्ञियꣳ॒॒ सदे॑वम् । सदे॑व॒मासी᳚त् । सदे॑व॒मिति॒ स - दे॒व॒म् । आसी॒त् तत् । तदप॑ । अपोत् । उद॑क्रामत् । अ॒क्रा॒म॒त् ते । ते द॒र्भाः । द॒र्भा अ॑भवन्न् । अ॒भ॒व॒न्॒. यत् । यद् द॑र्भपुञ्जी॒लैः । द॒र्भ॒पु॒ञ्जी॒लैः प॒वय॑ति । द॒र्भ॒पु॒ञ्जी॒लैरिति॑ दर्भ - पु॒ञ्जी॒लैः । प॒वय॑ति॒ याः । या ए॒व । ए॒व मेद्ध्याः᳚ । मेद्ध्या॑ य॒ज्ञियाः᳚ । य॒ज्ञियाः॒ सदे॑वाः । सदे॑वा॒ आपः॑ । सदे॑वा॒ इति॒ स - दे॒वाः॒ । आप॒स्ताभिः॑ । ताभि॑रे॒व । ए॒वैन᳚म् । ए॒न॒म् प॒व॒य॒ति॒ । प॒व॒य॒ति॒ द्वाभ्या᳚म् । द्वाभ्या᳚म् पवयति । प॒व॒य॒त्य॒हो॒रा॒त्राभ्या᳚म् । अ॒हो॒रा॒त्राभ्या॑मे॒व । अ॒हो॒रा॒त्राभ्या॒मित्य॑हः - रा॒त्राभ्या᳚म् । ए॒वैन᳚म् । ए॒न॒म् प॒व॒य॒ति॒ । प॒व॒य॒ति॒ त्रि॒भिः । त्रि॒भिः प॑वयति । त्रि॒भिरिति॑ त्रि - भिः । प॒व॒य॒ति॒ त्रयः॑ । त्रय॑ इ॒मे । इ॒मे लो॒काः । लो॒का ए॒भिः । ए॒भिरे॒व । ए॒वैन᳚म् । ए॒न॒म् ॅलो॒कैः । लो॒कैः प॑वयति । प॒व॒य॒ति॒ प॒ञ्चभिः॑ । प॒ञ्चभिः॑ पवयति । प॒ञ्चभि॒रिति॑ प॒ञ्च - भिः॒ \newline

\textbf{Jatai Paata} \newline

1. इन्द्रो॑ वृ॒त्रं ॅवृ॒त्र मिन्द्र॒ इन्द्रो॑ वृ॒त्रम् । \newline
2. वृ॒त्र म॑हन् नहन् वृ॒त्रं ॅवृ॒त्र म॑हन्न् । \newline
3. अ॒ह॒न् थ्स सो॑ ऽहन् नह॒न् थ्सः । \newline
4. सो᳚(1॒) ऽपो॑ ऽपः स सो॑ ऽपः । \newline
5. अ॒पो᳚(1॒) ऽभ्या᳚(1॒)भ्या᳚(1॒)पो᳚(1॒) ऽपो॑ ऽभि । \newline
6. अ॒भ्य॑म्रियता म्रियता॒भ्या᳚(1॒) भ्य॑म्रियत । \newline
7. अ॒म्रि॒य॒त॒ तासा॒म् तासा॑ मम्रिय ताम्रियत॒ तासा᳚म् । \newline
8. तासां॒ ॅयद् यत् तासा॒म् तासां॒ ॅयत् । \newline
9. यन् मेद्ध्य॒म् मेद्ध्यं॒ ॅयद् यन् मेद्ध्य᳚म् । \newline
10. मेद्ध्यं॑ ॅय॒ज्ञियं॑ ॅय॒ज्ञिय॒म् मेद्ध्य॒म् मेद्ध्यं॑ ॅय॒ज्ञिय᳚म् । \newline
11. य॒ज्ञियꣳ॒॒ सदे॑वꣳ॒॒ सदे॑वं ॅय॒ज्ञियं॑ ॅय॒ज्ञियꣳ॒॒ सदे॑वम् । \newline
12. सदे॑व॒ मासी॒ दासी॒थ् सदे॑वꣳ॒॒ सदे॑व॒ मासी᳚त् । \newline
13. सदे॑व॒मिति॒ स - दे॒व॒म् । \newline
14. आसी॒त् तत् तदासी॒ दासी॒त् तत् । \newline
15. तद पाप॒ तत् तदप॑ । \newline
16. अपो दुद पापोत् । \newline
17. उद॑क्राम दक्राम॒ दुदु द॑क्रामत् । \newline
18. अ॒क्रा॒म॒त् ते ते᳚ ऽक्राम दक्राम॒त् ते । \newline
19. ते द॒र्भा द॒र्भा स्ते ते द॒र्भाः । \newline
20. द॒र्भा अ॑भवन् नभवन् द॒र्भा द॒र्भा अ॑भवन्न् । \newline
21. अ॒भ॒व॒न्॒. यद् यद॑भवन् नभव॒न्॒. यत् । \newline
22. यद् द॑र्भपुञ्जी॒लैर् द॑र्भपुञ्जी॒लैर् यद् यद् द॑र्भपुञ्जी॒लैः । \newline
23. द॒र्भ॒पु॒ञ्जी॒लैः प॒वय॑ति प॒वय॑ति दर्भपुञ्जी॒लैर् द॑र्भपुञ्जी॒लैः प॒वय॑ति । \newline
24. द॒र्भ॒पु॒ञ्जी॒लैरिति॑ दर्भ - पु॒ञ्जी॒लैः । \newline
25. प॒वय॑ति॒ या याः प॒वय॑ति प॒वय॑ति॒ याः । \newline
26. या ए॒वैव या या ए॒व । \newline
27. ए॒व मेद्ध्या॒ मेद्ध्या॑ ए॒वैव मेद्ध्याः᳚ । \newline
28. मेद्ध्या॑ य॒ज्ञिया॑ य॒ज्ञिया॒ मेद्ध्या॒ मेद्ध्या॑ य॒ज्ञियाः᳚ । \newline
29. य॒ज्ञियाः॒ सदे॑वाः॒ सदे॑वा य॒ज्ञिया॑ य॒ज्ञियाः॒ सदे॑वाः । \newline
30. सदे॑वा॒ आप॒ आपः॒ सदे॑वाः॒ सदे॑वा॒ आपः॑ । \newline
31. सदे॑वा॒ इति॒ स - दे॒वाः॒ । \newline
32. आप॒ स्ताभि॒ स्ताभि॒ राप॒ आप॒ स्ताभिः॑ । \newline
33. ताभि॑ रे॒वैव ताभि॒ स्ताभि॑ रे॒व । \newline
34. ए॒वैन॑ मेन मे॒वै वैन᳚म् । \newline
35. ए॒न॒म् प॒व॒य॒ति॒ प॒व॒य॒ त्ये॒न॒ मे॒न॒म् प॒व॒य॒ति॒ । \newline
36. प॒व॒य॒ति॒ द्वाभ्या॒म् द्वाभ्या᳚म् पवयति पवयति॒ द्वाभ्या᳚म् । \newline
37. द्वाभ्या᳚म् पवयति पवयति॒ द्वाभ्या॒म् द्वाभ्या᳚म् पवयति । \newline
38. प॒व॒य॒ त्य॒हो॒रा॒त्राभ्या॑ महोरा॒त्राभ्या᳚म् पवयति पवय त्यहोरा॒त्राभ्या᳚म् । \newline
39. अ॒हो॒रा॒त्राभ्या॑ मे॒वै वाहो॑रा॒त्राभ्या॑ महोरा॒त्राभ्या॑ मे॒व । \newline
40. अ॒हो॒रा॒त्राभ्या॒मित्य॑हः - रा॒त्राभ्या᳚म् । \newline
41. ए॒वैन॑ मेन मे॒वै वैन᳚म् । \newline
42. ए॒न॒म् प॒व॒य॒ति॒ प॒व॒य॒ त्ये॒न॒ मे॒न॒म् प॒व॒य॒ति॒ । \newline
43. प॒व॒य॒ति॒ त्रि॒भि स्त्रि॒भिः प॑वयति पवयति त्रि॒भिः । \newline
44. त्रि॒भिः प॑वयति पवयति त्रि॒भि स्त्रि॒भिः प॑वयति । \newline
45. त्रि॒भिरिति॑ त्रि - भिः । \newline
46. प॒व॒य॒ति॒ त्रय॒ स्त्रयः॑ पवयति पवयति॒ त्रयः॑ । \newline
47. त्रय॑ इ॒म इ॒मे त्रय॒ स्त्रय॑ इ॒मे । \newline
48. इ॒मे लो॒का लो॒का इ॒म इ॒मे लो॒काः । \newline
49. लो॒का ए॒भि रे॒भिर् लो॒का लो॒का ए॒भिः । \newline
50. ए॒भि रे॒वै वैभि रे॒भि रे॒व । \newline
51. ए॒वैन॑ मेन मे॒वै वैन᳚म् । \newline
52. ए॒न॒म् ॅलो॒कैर् लो॒कै रे॑न मेनम् ॅलो॒कैः । \newline
53. लो॒कैः प॑वयति पवयति लो॒कैर् लो॒कैः प॑वयति । \newline
54. प॒व॒य॒ति॒ प॒ञ्चभिः॑ प॒ञ्चभिः॑ पवयति पवयति प॒ञ्चभिः॑ । \newline
55. प॒ञ्चभिः॑ पवयति पवयति प॒ञ्चभिः॑ प॒ञ्चभिः॑ पवयति । \newline
56. प॒ञ्चभि॒रिति॑ प॒ञ्च - भिः॒ । \newline

\textbf{Ghana Paata } \newline

1. इन्द्रो॑ वृ॒त्रं ॅवृ॒त्र मिन्द्र॒ इन्द्रो॑ वृ॒त्र म॑हन् नहन् वृ॒त्र मिन्द्र॒ इन्द्रो॑ वृ॒त्र म॑हन्न् । \newline
2. वृ॒त्र म॑हन् नहन् वृ॒त्रं ॅवृ॒त्र म॑ह॒न् थ्स सो॑ ऽहन् वृ॒त्रं ॅवृ॒त्र म॑ह॒न् थ्सः । \newline
3. अ॒ह॒न् थ्स सो॑ ऽहन् नह॒न् थ्सो᳚(1॒) ऽपो॑ ऽपः सो॑ ऽहन् नह॒न् थ्सो॑ ऽपः । \newline
4. सो᳚(1॒) ऽपो॑ ऽपः स सो᳚(1॒) ऽपो᳚(1॒) ऽभ्या᳚(1॒)भ्य॑पः स सो᳚(1॒) ऽपो॑ ऽभि । \newline
5. अ॒पो᳚(1॒) ऽभ्या᳚(1॒)भ्या᳚(1॒)पो᳚(1॒) ऽपो᳚(1॒) ऽभ्य॑म्रियता म्रियता॒भ्या᳚(1॒)पो᳚(1॒) ऽपो᳚(1॒) ऽभ्य॑म्रियत । \newline
6. अ॒भ्य॑म्रियता म्रियता॒भ्या᳚(1॒)भ्य॑म्रियत॒ तासा॒म् तासा॑ मम्रियता॒भ्या᳚(1॒) भ्य॑म्रियत॒ तासा᳚म् । \newline
7. अ॒म्रि॒य॒त॒ तासा॒म् तासा॑ मम्रियता म्रियत॒ तासां॒ ॅयद् यत् तासा॑ मम्रियता म्रियत॒ तासां॒ ॅयत् । \newline
8. तासां॒ ॅयद् यत् तासा॒म् तासां॒ ॅयन् मेद्ध्य॒म् मेद्ध्यं॒ ॅयत् तासा॒म् तासां॒ ॅयन् मेद्ध्य᳚म् । \newline
9. यन् मेद्ध्य॒म् मेद्ध्यं॒ ॅयद् यन् मेद्ध्यं॑ ॅय॒ज्ञियं॑ ॅय॒ज्ञिय॒म् मेद्ध्यं॒ ॅयद् यन् मेद्ध्यं॑ ॅय॒ज्ञिय᳚म् । \newline
10. मेद्ध्यं॑ ॅय॒ज्ञियं॑ ॅय॒ज्ञिय॒म् मेद्ध्य॒म् मेद्ध्यं॑ ॅय॒ज्ञियꣳ॒॒ सदे॑वꣳ॒॒ सदे॑वं ॅय॒ज्ञिय॒म् मेद्ध्य॒म् मेद्ध्यं॑ ॅय॒ज्ञियꣳ॒॒ सदे॑वम् । \newline
11. य॒ज्ञियꣳ॒॒ सदे॑वꣳ॒॒ सदे॑वं ॅय॒ज्ञियं॑ ॅय॒ज्ञियꣳ॒॒ सदे॑व॒ मासी॒ दासी॒थ् सदे॑वं ॅय॒ज्ञियं॑ ॅय॒ज्ञियꣳ॒॒ सदे॑व॒ मासी᳚त् । \newline
12. सदे॑व॒ मासी॒ दासी॒थ् सदे॑वꣳ॒॒ सदे॑व॒ मासी॒त् तत् तदासी॒थ् सदे॑वꣳ॒॒ सदे॑व॒ मासी॒त् तत् । \newline
13. सदे॑व॒मिति॒ स - दे॒व॒म् । \newline
14. आसी॒त् तत् तदासी॒ दासी॒त् तदपाप॒ तदासी॒ दासी॒त् तदप॑ । \newline
15. तदपाप॒ तत् तद पोदु दप॒ तत् तदपोत् । \newline
16. अपोदु दपा पोद॑क्राम दक्राम॒ दुदपा पोद॑क्रामत् । \newline
17. उद॑क्राम दक्राम॒ दुदु द॑क्राम॒त् ते ते᳚ ऽक्राम॒ दुदु द॑क्राम॒त् ते । \newline
18. अ॒क्रा॒म॒त् ते ते᳚ ऽक्रामद क्राम॒त् ते द॒र्भा द॒र्भा स्ते᳚ ऽक्राम दक्राम॒त् ते द॒र्भाः । \newline
19. ते द॒र्भा द॒र्भा स्ते ते द॒र्भा अ॑भवन् नभवन् द॒र्भा स्ते ते द॒र्भा अ॑भवन्न् । \newline
20. द॒र्भा अ॑भवन् नभवन् द॒र्भा द॒र्भा अ॑भव॒न्॒. यद् यद॑भवन् द॒र्भा द॒र्भा अ॑भव॒न्॒. यत् । \newline
21. अ॒भ॒व॒न्॒. यद् यद॑भवन् नभव॒न्॒. यद् द॑र्भपुञ्जी॒लैर् द॑र्भपुञ्जी॒लैर् यद॑भवन् नभव॒न्॒. यद् द॑र्भपुञ्जी॒लैः । \newline
22. यद् द॑र्भपुञ्जी॒लैर् द॑र्भपुञ्जी॒लैर् यद् यद् द॑र्भपुञ्जी॒लैः प॒वय॑ति प॒वय॑ति दर्भपुञ्जी॒लैर् यद् यद् द॑र्भपुञ्जी॒लैः प॒वय॑ति । \newline
23. द॒र्भ॒पु॒ञ्जी॒लैः प॒वय॑ति प॒वय॑ति दर्भपुञ्जी॒लैर् द॑र्भपुञ्जी॒लैः प॒वय॑ति॒ या याः प॒वय॑ति दर्भपुञ्जी॒लैर् द॑र्भपुञ्जी॒लैः प॒वय॑ति॒ याः । \newline
24. द॒र्भ॒पु॒ञ्जी॒लैरिति॑ दर्भ - पु॒ञ्जी॒लैः । \newline
25. प॒वय॑ति॒ या याः प॒वय॑ति प॒वय॑ति॒ या ए॒वैव याः प॒वय॑ति प॒वय॑ति॒ या ए॒व । \newline
26. या ए॒वैव या या ए॒व मेद्ध्या॒ मेद्ध्या॑ ए॒व या या ए॒व मेद्ध्याः᳚ । \newline
27. ए॒व मेद्ध्या॒ मेद्ध्या॑ ए॒वैव मेद्ध्या॑ य॒ज्ञिया॑ य॒ज्ञिया॒ मेद्ध्या॑ ए॒वैव मेद्ध्या॑ य॒ज्ञियाः᳚ । \newline
28. मेद्ध्या॑ य॒ज्ञिया॑ य॒ज्ञिया॒ मेद्ध्या॒ मेद्ध्या॑ य॒ज्ञियाः॒ सदे॑वाः॒ सदे॑वा य॒ज्ञिया॒ मेद्ध्या॒ मेद्ध्या॑ य॒ज्ञियाः॒ सदे॑वाः । \newline
29. य॒ज्ञियाः॒ सदे॑वाः॒ सदे॑वा य॒ज्ञिया॑ य॒ज्ञियाः॒ सदे॑वा॒ आप॒ आपः॒ सदे॑वा य॒ज्ञिया॑ य॒ज्ञियाः॒ सदे॑वा॒ आपः॑ । \newline
30. सदे॑वा॒ आप॒ आपः॒ सदे॑वाः॒ सदे॑वा॒ आप॒ स्ताभि॒ स्ताभि॒ रापः॒ सदे॑वाः॒ सदे॑वा॒ आप॒ स्ताभिः॑ । \newline
31. सदे॑वा॒ इति॒ स - दे॒वाः॒ । \newline
32. आप॒ स्ताभि॒ स्ताभि॒ राप॒ आप॒ स्ताभि॑ रे॒वैव ताभि॒ राप॒ आप॒ स्ताभि॑ रे॒व । \newline
33. ताभि॑ रे॒वैव ताभि॒ स्ताभि॑ रे॒वैन॑ मेन मे॒व ताभि॒ स्ताभि॑ रे॒वैन᳚म् । \newline
34. ए॒वैन॑ मेन मे॒वै वैन॑म् पवयति पवय त्येन मे॒वै वैन॑म् पवयति । \newline
35. ए॒न॒म् प॒व॒य॒ति॒ प॒व॒य॒ त्ये॒न॒ मे॒न॒म् प॒व॒य॒ति॒ द्वाभ्या॒म् द्वाभ्या᳚म् पवय त्येन मेनम् पवयति॒ द्वाभ्या᳚म् । \newline
36. प॒व॒य॒ति॒ द्वाभ्या॒म् द्वाभ्या᳚म् पवयति पवयति॒ द्वाभ्या᳚म् पवयति पवयति॒ द्वाभ्या᳚म् पवयति पवयति॒ द्वाभ्या᳚म् पवयति । \newline
37. द्वाभ्या᳚म् पवयति पवयति॒ द्वाभ्या॒म् द्वाभ्या᳚म् पवय त्यहोरा॒त्राभ्या॑ महोरा॒त्राभ्या᳚म् पवयति॒ द्वाभ्या॒म् द्वाभ्या᳚म् पवय त्यहोरा॒त्राभ्या᳚म् । \newline
38. प॒व॒य॒ त्य॒हो॒रा॒त्राभ्या॑ महोरा॒त्राभ्या᳚म् पवयति पवय त्यहोरा॒त्राभ्या॑ मे॒वै वाहो॑रा॒त्राभ्या᳚म् पवयति पवय त्यहोरा॒त्राभ्या॑ मे॒व । \newline
39. अ॒हो॒रा॒त्राभ्या॑ मे॒वै वाहो॑रा॒त्राभ्या॑ महोरा॒त्राभ्या॑ मे॒वैन॑ मेन मे॒वाहो॑रा॒त्राभ्या॑ महोरा॒त्राभ्या॑ मे॒वैन᳚म् । \newline
40. अ॒हो॒रा॒त्राभ्या॒मित्य॑हः - रा॒त्राभ्या᳚म् । \newline
41. ए॒वैन॑ मेन मे॒वै वैन॑म् पवयति पवय त्येन मे॒वै वैन॑म् पवयति । \newline
42. ए॒न॒म् प॒व॒य॒ति॒ प॒व॒य॒ त्ये॒न॒ मे॒न॒म् प॒व॒य॒ति॒ त्रि॒भि स्त्रि॒भिः प॑वय त्येन मेनम् पवयति त्रि॒भिः । \newline
43. प॒व॒य॒ति॒ त्रि॒भि स्त्रि॒भिः प॑वयति पवयति त्रि॒भिः प॑वयति पवयति त्रि॒भिः प॑वयति पवयति त्रि॒भिः प॑वयति । \newline
44. त्रि॒भिः प॑वयति पवयति त्रि॒भि स्त्रि॒भिः प॑वयति॒ त्रय॒ स्त्रयः॑ पवयति त्रि॒भि स्त्रि॒भिः प॑वयति॒ त्रयः॑ । \newline
45. त्रि॒भिरिति॑ त्रि - भिः । \newline
46. प॒व॒य॒ति॒ त्रय॒ स्त्रयः॑ पवयति पवयति॒ त्रय॑ इ॒म इ॒मे त्रयः॑ पवयति पवयति॒ त्रय॑ इ॒मे । \newline
47. त्रय॑ इ॒म इ॒मे त्रय॒ स्त्रय॑ इ॒मे लो॒का लो॒का इ॒मे त्रय॒ स्त्रय॑ इ॒मे लो॒काः । \newline
48. इ॒मे लो॒का लो॒का इ॒म इ॒मे लो॒का ए॒भि रे॒भिर् लो॒का इ॒म इ॒मे लो॒का ए॒भिः । \newline
49. लो॒का ए॒भि रे॒भिर् लो॒का लो॒का ए॒भि रे॒वै वैभिर् लो॒का लो॒का ए॒भि रे॒व । \newline
50. ए॒भि रे॒वै वैभि रे॒भि रे॒वैन॑ मेन मे॒वैभि रे॒भि रे॒वैन᳚म् । \newline
51. ए॒वैन॑ मेन मे॒वै वैन॑म् ॅलो॒कैर् लो॒कै रे॑न मे॒वै वैन॑म् ॅलो॒कैः । \newline
52. ए॒न॒म् ॅलो॒कैर् लो॒कै रे॑न मेनम् ॅलो॒कैः प॑वयति पवयति लो॒कै रे॑न मेनम् ॅलो॒कैः प॑वयति । \newline
53. लो॒कैः प॑वयति पवयति लो॒कैर् लो॒कैः प॑वयति प॒ञ्चभिः॑ प॒ञ्चभिः॑ पवयति लो॒कैर् लो॒कैः प॑वयति प॒ञ्चभिः॑ । \newline
54. प॒व॒य॒ति॒ प॒ञ्चभिः॑ प॒ञ्चभिः॑ पवयति पवयति प॒ञ्चभिः॑ पवयति पवयति प॒ञ्चभिः॑ पवयति पवयति प॒ञ्चभिः॑ पवयति । \newline
55. प॒ञ्चभिः॑ पवयति पवयति प॒ञ्चभिः॑ प॒ञ्चभिः॑ पवयति॒ पञ्चा᳚क्षरा॒ पञ्चा᳚क्षरा पवयति प॒ञ्चभिः॑ प॒ञ्चभिः॑ पवयति॒ पञ्चा᳚क्षरा । \newline
56. प॒ञ्चभि॒रिति॑ प॒ञ्च - भिः॒ । \newline
\pagebreak
\markright{ TS 6.1.1.8  \hfill https://www.vedavms.in \hfill}

\section{ TS 6.1.1.8 }

\textbf{TS 6.1.1.8 } \newline
\textbf{Samhita Paata} \newline

पवयति॒ पञ्चा᳚क्षरा प॒ङ्क्तिः पाङ्क्तो॑ य॒ज्ञो य॒ज्ञायै॒वैनं॑ पवयति ष॒ड्भिः प॑वयति॒ षड्वा ऋ॒तव॑ ऋ॒तुभि॑रे॒वैनं॑ पवयति स॒प्तभिः॑ पवयति स॒प्त छन्दाꣳ॑सि॒ छन्दो॑भिरे॒वैनं॑ पवयति न॒वभिः॑ पवयति॒ नव॒ वै पुरु॑षे प्रा॒णाः सप्रा॑णमे॒वैनं॑ पवय॒त्येक॑विꣳशत्या पवयति॒ दश॒ हस्त्या॑ अ॒ङ्गुल॑यो॒ दश॒ पद्या॑ आ॒त्मैक॑विꣳ॒॒शो यावा॑ने॒व पुरु॑ष॒स्तमप॑रिवर्गं - [  ] \newline

\textbf{Pada Paata} \newline

प॒व॒य॒ति॒ । पञ्चा᳚क्ष॒रेति॒ पञ्च॑ - अ॒क्ष॒रा॒ । प॒ङ्क्तिः । पाङ्क्तः॑ । य॒ज्ञ्ः । य॒ज्ञाय॑ । ए॒व । ए॒न॒म् । प॒व॒य॒ति॒ । ष॒ड्भिरिति॑ षट् - भिः । प॒व॒य॒ति॒ । षट् । वै । ऋ॒तवः॑ । ऋ॒तुभि॒रित्यृ॒तु - भिः॒ । ए॒व । ए॒न॒म् । प॒व॒य॒ति॒ । स॒प्तभि॒रिति॑ स॒प्त - भिः॒ । प॒व॒य॒ति॒ । स॒प्त । छन्दाꣳ॑सि । छन्दो॑भि॒रिति॒ छन्दः॑ - भिः॒ । ए॒व । ए॒न॒म् । प॒व॒य॒ति॒ । न॒वभि॒रिति॑ न॒व - भिः॒ । प॒व॒य॒ति॒ । नव॑ । वै । पुरु॑षे । प्रा॒णा इति॑ प्र - अ॒नाः । सप्रा॑ण॒मिति॒ स-प्रा॒ण॒म् । ए॒व । ए॒न॒म् । प॒व॒य॒ति॒ । एक॑विꣳश॒त्येत्येक॑-विꣳ॒॒श॒त्या॒ । प॒व॒य॒ति॒ । दश॑ । हस्त्याः᳚ । अ॒ङ्गुल॑यः । दश॑ । पद्याः᳚ । आ॒त्मा । ए॒क॒विꣳ॒॒श इत्येक॑ - विꣳ॒॒शः । यावान्॑ । ए॒व । पुरु॑षः । तम् । अप॑रिवर्ग॒मित्यप॑रि - व॒र्ग॒म् ।  \newline


\textbf{Krama Paata} \newline

प॒व॒य॒ति॒ पञ्चा᳚क्षरा । पञ्चा᳚क्षरा प॒ङ्‍क्तिः । पञ्चा᳚क्ष॒रेति॒ पञ्च॑ - अ॒क्ष॒रा॒ । प॒ङ्‍क्तिः पाङ्‍क्तः॑ । पाङ्‍क्तो॑ य॒ज्ञ्ः । य॒ज्ञो य॒ज्ञाय॑ । य॒ज्ञायै॒व । ए॒वैन᳚म् । ए॒न॒म् प॒व॒य॒ति॒ । प॒व॒य॒ति॒ ष॒ड्भिः । ष॒ड्भिः प॑वयति । ष॒ड्भिरिति॑ षट् - भिः । प॒व॒य॒ति॒ षट् । षड् वै । वा ऋ॒तवः॑ । ऋ॒तव॑ ऋ॒तुभिः॑ । ऋ॒तुभि॑रे॒व । ऋ॒तुभि॒रित्यृ॒तु - भिः॒ । ए॒वैन᳚म् । ए॒न॒म् प॒व॒य॒ति॒ । प॒व॒य॒ति॒ स॒प्तभिः॑ । स॒प्तभिः॑ पवयति । स॒प्तभि॒रिति॑ स॒प्त - भिः॒ । प॒व॒य॒ति॒ स॒प्त । स॒प्त छन्दाꣳ॑सि । छन्दाꣳ॑सि॒ छन्दो॑भिः । छन्दो॑भिरे॒व । छन्दो॑भि॒रिति॒ छन्दः॑ - भिः॒ । ए॒वैन᳚म् । ए॒न॒म् प॒व॒य॒ति॒ । प॒व॒य॒ति॒ न॒वभिः॑ । न॒वभिः॑ पवयति । न॒वभि॒रिति॑ न॒व - भिः॒ । प॒व॒य॒ति॒ नव॑ । नव॒ वै । वै पुरु॑षे । पुरु॑षे प्रा॒णाः । प्रा॒णाः सप्रा॑णम् । प्रा॒णा इति॑ प्र - अ॒नाः । सप्रा॑णमे॒व । सप्रा॑ण॒मिति॒ स - प्रा॒ण॒म् । ए॒वैन᳚म् । ए॒न॒म् प॒व॒य॒ति॒ । प॒व॒य॒त्येक॑विꣳशत्या । एक॑विꣳशत्या पवयति । एक॑विꣳश॒त्येत्येक॑ - विꣳ॒॒श॒त्या॒ । प॒व॒य॒ति॒ दश॑ । दश॒ हस्त्याः᳚ । हस्त्या॑ अ॒ङ्‍गुल॑यः । अ॒ङ्‍गुल॑यो॒ दश॑ । दश॒ पद्याः᳚ । पद्या॑ आ॒त्मा । आ॒त्मैक॑विꣳ॒॒शः । ए॒क॒विꣳ॒॒शो यावान्॑ । ए॒क॒विꣳ॒॒श इत्ये॑क - विꣳ॒॒शः । यावा॑ने॒व । ए॒व पुरु॑षः । पुरु॑ष॒स्तम् । तमप॑रिवर्गम् ( ) । अप॑रिवर्गम् पवयति । अप॑रिवर्ग॒मित्यप॑रि - व॒र्ग॒म् \newline

\textbf{Jatai Paata} \newline

1. प॒व॒य॒ति॒ पञ्चा᳚क्षरा॒ पञ्चा᳚क्षरा पवयति पवयति॒ पञ्चा᳚क्षरा । \newline
2. पञ्चा᳚क्षरा प॒ङ्क्तिः प॒ङ्क्तिः पञ्चा᳚क्षरा॒ पञ्चा᳚क्षरा प॒ङ्क्तिः । \newline
3. पञ्चा᳚क्ष॒रेति॒ पञ्च॑ - अ॒क्ष॒रा॒ । \newline
4. प॒ङ्क्तिः पाङ्क्तः॒ पाङ्क्तः॑ प॒ङ्क्तिः प॒ङ्क्तिः पाङ्क्तः॑ । \newline
5. पाङ्क्तो॑ य॒ज्ञो य॒ज्ञ्ः पाङ्क्तः॒ पाङ्क्तो॑ य॒ज्ञ्ः । \newline
6. य॒ज्ञो य॒ज्ञाय॑ य॒ज्ञाय॑ य॒ज्ञो य॒ज्ञो य॒ज्ञाय॑ । \newline
7. य॒ज्ञायै॒ वैव य॒ज्ञाय॑ य॒ज्ञायै॒व । \newline
8. ए॒वैन॑ मेन मे॒वै वैन᳚म् । \newline
9. ए॒न॒म् प॒व॒य॒ति॒ प॒व॒य॒ त्ये॒न॒ मे॒न॒म् प॒व॒य॒ति॒ । \newline
10. प॒व॒य॒ति॒ ष॒ड्भि ष्ष॒ड्भिः प॑वयति पवयति ष॒ड्भिः । \newline
11. ष॒ड्भिः प॑वयति पवयति ष॒ड्भि ष्ष॒ड्भिः प॑वयति । \newline
12. ष॒ड्भिरिति॑ षट् - भिः । \newline
13. प॒व॒य॒ति॒ षट् थ्षट् प॑वयति पवयति॒ षट् । \newline
14. षड् वै वै षट् थ्षड् वै । \newline
15. वा ऋ॒तव॑ ऋ॒तवो॒ वै वा ऋ॒तवः॑ । \newline
16. ऋ॒तव॑ ऋ॒तुभिर्॑. ऋ॒तुभिर्॑. ऋ॒तव॑ ऋ॒तव॑ ऋ॒तुभिः॑ । \newline
17. ऋ॒तुभि॑ रे॒वैव र्‌तुभिर्॑. ऋ॒तुभि॑ रे॒व । \newline
18. ऋ॒तुभि॒रित्यृ॒तु - भिः॒ । \newline
19. ए॒वैन॑ मेन मे॒वै वैन᳚म् । \newline
20. ए॒न॒म् प॒व॒य॒ति॒ प॒व॒य॒ त्ये॒न॒ मे॒न॒म् प॒व॒य॒ति॒ । \newline
21. प॒व॒य॒ति॒ स॒प्तभिः॑ स॒प्तभिः॑ पवयति पवयति स॒प्तभिः॑ । \newline
22. स॒प्तभिः॑ पवयति पवयति स॒प्तभिः॑ स॒प्तभिः॑ पवयति । \newline
23. स॒प्तभि॒रिति॑ स॒प्त - भिः॒ । \newline
24. प॒व॒य॒ति॒ स॒प्त स॒प्त प॑वयति पवयति स॒प्त । \newline
25. स॒प्त छन्दाꣳ॑सि॒ छन्दाꣳ॑सि स॒प्त स॒प्त छन्दाꣳ॑सि । \newline
26. छन्दाꣳ॑सि॒ छन्दो॑भि॒ श्छन्दो॑भि॒ श्छन्दाꣳ॑सि॒ छन्दाꣳ॑सि॒ छन्दो॑भिः । \newline
27. छन्दो॑भि रे॒वैव छन्दो॑भि॒ श्छन्दो॑भि रे॒व । \newline
28. छन्दो॑भि॒रिति॒ छन्दः॑ - भिः॒ । \newline
29. ए॒वैन॑ मेन मे॒वै वैन᳚म् । \newline
30. ए॒न॒म् प॒व॒य॒ति॒ प॒व॒य॒ त्ये॒न॒ मे॒न॒म् प॒व॒य॒ति॒ । \newline
31. प॒व॒य॒ति॒ न॒वभि॑र् न॒वभिः॑ पवयति पवयति न॒वभिः॑ । \newline
32. न॒वभिः॑ पवयति पवयति न॒वभि॑र् न॒वभिः॑ पवयति । \newline
33. न॒वभि॒रिति॑ न॒व - भिः॒ । \newline
34. प॒व॒य॒ति॒ नव॒ नव॑ पवयति पवयति॒ नव॑ । \newline
35. नव॒ वै वै नव॒ नव॒ वै । \newline
36. वै पुरु॑षे॒ पुरु॑षे॒ वै वै पुरु॑षे । \newline
37. पुरु॑षे प्रा॒णाः प्रा॒णाः पुरु॑षे॒ पुरु॑षे प्रा॒णाः । \newline
38. प्रा॒णाः सप्रा॑णꣳ॒॒ सप्रा॑णम् प्रा॒णाः प्रा॒णाः सप्रा॑णम् । \newline
39. प्रा॒णा इति॑ प्र - अ॒नाः । \newline
40. सप्रा॑ण मे॒वैव सप्रा॑णꣳ॒॒ सप्रा॑ण मे॒व । \newline
41. सप्रा॑ण॒मिति॒ स - प्रा॒ण॒म् । \newline
42. ए॒वैन॑ मेन मे॒वै वैन᳚म् । \newline
43. ए॒न॒म् प॒व॒य॒ति॒ प॒व॒य॒ त्ये॒न॒ मे॒न॒म् प॒व॒य॒ति॒ । \newline
44. प॒व॒य॒ त्येक॑विꣳश॒त्यै क॑विꣳशत्या पवयति पवय॒ त्येक॑विꣳशत्या । \newline
45. एक॑विꣳशत्या पवयति पवय॒त्येक॑विꣳश॒ त्यैक॑विꣳशत्या पवयति । \newline
46. एक॑विꣳश॒त्येत्येक॑ - विꣳ॒॒श॒त्या॒ । \newline
47. प॒व॒य॒ति॒ दश॒ दश॑ पवयति पवयति॒ दश॑ । \newline
48. दश॒ हस्त्या॒ हस्त्या॒ दश॒ दश॒ हस्त्याः᳚ । \newline
49. हस्त्या॑ अ॒ङ्गुल॑यो॒ ऽङ्गुल॑यो॒ हस्त्या॒ हस्त्या॑ अ॒ङ्गुल॑यः । \newline
50. अ॒ङ्गुल॑यो॒ दश॒ दशा॒ ङ्गुल॑यो॒ ऽङ्गुल॑यो॒ दश॑ । \newline
51. दश॒ पद्याः॒ पद्या॒ दश॒ दश॒ पद्याः᳚ । \newline
52. पद्या॑ आ॒त्मा ऽऽत्मा पद्याः॒ पद्या॑ आ॒त्मा । \newline
53. आ॒त्मै क॑विꣳ॒॒श ए॑कविꣳ॒॒श आ॒त्मा ऽऽत्मै क॑विꣳ॒॒शः । \newline
54. ए॒क॒विꣳ॒॒शो यावा॒न्॒. यावा॑ नेकविꣳ॒॒श ए॑कविꣳ॒॒शो यावान्॑ । \newline
55. ए॒क॒विꣳ॒॒श इत्येक॑ - विꣳ॒॒शः । \newline
56. यावा॑ने॒ वैव यावा॒न्॒. यावा॑ने॒व । \newline
57. ए॒व पुरु॑षः॒ पुरु॑ष ए॒वैव पुरु॑षः । \newline
58. पुरु॑ष॒ स्तम् तम् पुरु॑षः॒ पुरु॑ष॒ स्तम् । \newline
59. तमप॑रिवर्ग॒ मप॑रिवर्ग॒म् तम् तमप॑रिवर्गम् । \newline
60. अप॑रिवर्गम् पवयति पवय॒ त्यप॑रिवर्ग॒ मप॑रिवर्गम् पवयति । \newline
61. अप॑रिवर्ग॒मित्यप॑रि - व॒र्ग॒म् । \newline

\textbf{Ghana Paata } \newline

1. प॒व॒य॒ति॒ पञ्चा᳚क्षरा॒ पञ्चा᳚क्षरा पवयति पवयति॒ पञ्चा᳚क्षरा प॒ङ्क्तिः प॒ङ्क्तिः पञ्चा᳚क्षरा पवयति पवयति॒ पञ्चा᳚क्षरा प॒ङ्क्तिः । \newline
2. पञ्चा᳚क्षरा प॒ङ्क्तिः प॒ङ्क्तिः पञ्चा᳚क्षरा॒ पञ्चा᳚क्षरा प॒ङ्क्तिः पाङ्क्तः॒ पाङ्क्तः॑ प॒ङ्क्तिः पञ्चा᳚क्षरा॒ पञ्चा᳚क्षरा प॒ङ्क्तिः पाङ्क्तः॑ । \newline
3. पञ्चा᳚क्ष॒रेति॒ पञ्च॑ - अ॒क्ष॒रा॒ । \newline
4. प॒ङ्क्तिः पाङ्क्तः॒ पाङ्क्तः॑ प॒ङ्क्तिः प॒ङ्क्तिः पाङ्क्तो॑ य॒ज्ञो य॒ज्ञ्ः पाङ्क्तः॑ प॒ङ्क्तिः प॒ङ्क्तिः पाङ्क्तो॑ य॒ज्ञ्ः । \newline
5. पाङ्क्तो॑ य॒ज्ञो य॒ज्ञ्ः पाङ्क्तः॒ पाङ्क्तो॑ य॒ज्ञो य॒ज्ञाय॑ य॒ज्ञाय॑ य॒ज्ञ्ः पाङ्क्तः॒ पाङ्क्तो॑ य॒ज्ञो य॒ज्ञाय॑ । \newline
6. य॒ज्ञो य॒ज्ञाय॑ य॒ज्ञाय॑ य॒ज्ञो य॒ज्ञो य॒ज्ञायै॒वैव य॒ज्ञाय॑ य॒ज्ञो य॒ज्ञो य॒ज्ञा यै॒व । \newline
7. य॒ज्ञायै॒वैव य॒ज्ञाय॑ य॒ज्ञा यै॒वैन॑ मेन मे॒व य॒ज्ञाय॑ य॒ज्ञा यै॒वैन᳚म् । \newline
8. ए॒वैन॑ मेन मे॒वै वैन॑म् पवयति पवय त्येन मे॒वै वैन॑म् पवयति । \newline
9. ए॒न॒म् प॒व॒य॒ति॒ प॒व॒य॒ त्ये॒न॒ मे॒न॒म् प॒व॒य॒ति॒ ष॒ड्भि ष्ष॒ड्भिः प॑वय त्येन मेनम् पवयति ष॒ड्भिः । \newline
10. प॒व॒य॒ति॒ ष॒ड्भि ष्ष॒ड्भिः प॑वयति पवयति ष॒ड्भिः प॑वयति पवयति ष॒ड्भिः प॑वयति पवयति ष॒ड्भिः प॑वयति । \newline
11. ष॒ड्भिः प॑वयति पवयति ष॒ड्भि ष्ष॒ड्भिः प॑वयति॒ षट् थ्षट् प॑वयति ष॒ड्भि ष्ष॒ड्भिः प॑वयति॒ षट् । \newline
12. ष॒ड्भिरिति॑ षट् - भिः । \newline
13. प॒व॒य॒ति॒ षट् थ्षट् प॑वयति पवयति॒ षड् वै वै षट् प॑वयति पवयति॒ षड् वै । \newline
14. षड् वै वै षट् थ्षड् वा ऋ॒तव॑ ऋ॒तवो॒ वै षट् थ्षड् वा ऋ॒तवः॑ । \newline
15. वा ऋ॒तव॑ ऋ॒तवो॒ वै वा ऋ॒तव॑ ऋ॒तुभिर्॑. ऋ॒तुभिर्॑. ऋ॒तवो॒ वै वा ऋ॒तव॑ ऋ॒तुभिः॑ । \newline
16. ऋ॒तव॑ ऋ॒तुभिर्॑. ऋ॒तुभिर्॑. ऋ॒तव॑ ऋ॒तव॑ ऋ॒तुभि॑ रे॒वैव र्‌तुभिर्॑. ऋ॒तव॑ ऋ॒तव॑ ऋ॒तुभि॑ रे॒व । \newline
17. ऋ॒तुभि॑ रे॒वैव र्‌तुभिर्॑. ऋ॒तुभि॑ रे॒वैन॑ मेन मे॒व र्‌तुभिर्॑. ऋ॒तुभि॑ रे॒वैन᳚म् । \newline
18. ऋ॒तुभि॒रित्यृ॒तु - भिः॒ । \newline
19. ए॒वैन॑ मेन मे॒वै वैन॑म् पवयति पवय त्येन मे॒वै वैन॑म् पवयति । \newline
20. ए॒न॒म् प॒व॒य॒ति॒ प॒व॒य॒ त्ये॒न॒ मे॒न॒म् प॒व॒य॒ति॒ स॒प्तभिः॑ स॒प्तभिः॑ पवय त्येन मेनम् पवयति स॒प्तभिः॑ । \newline
21. प॒व॒य॒ति॒ स॒प्तभिः॑ स॒प्तभिः॑ पवयति पवयति स॒प्तभिः॑ पवयति पवयति स॒प्तभिः॑ पवयति पवयति स॒प्तभिः॑ पवयति । \newline
22. स॒प्तभिः॑ पवयति पवयति स॒प्तभिः॑ स॒प्तभिः॑ पवयति स॒प्त स॒प्त प॑वयति स॒प्तभिः॑ स॒प्तभिः॑ पवयति स॒प्त । \newline
23. स॒प्तभि॒रिति॑ स॒प्त - भिः॒ । \newline
24. प॒व॒य॒ति॒ स॒प्त स॒प्त प॑वयति पवयति स॒प्त छन्दाꣳ॑सि॒ छन्दाꣳ॑सि स॒प्त प॑वयति पवयति स॒प्त छन्दाꣳ॑सि । \newline
25. स॒प्त छन्दाꣳ॑सि॒ छन्दाꣳ॑सि स॒प्त स॒प्त छन्दाꣳ॑सि॒ छन्दो॑भि॒ श्छन्दो॑भि॒ श्छन्दाꣳ॑सि स॒प्त स॒प्त छन्दाꣳ॑सि॒ छन्दो॑भिः । \newline
26. छन्दाꣳ॑सि॒ छन्दो॑भि॒ श्छन्दो॑भि॒ श्छन्दाꣳ॑सि॒ छन्दाꣳ॑सि॒ छन्दो॑भि रे॒वैव छन्दो॑भि॒ श्छन्दाꣳ॑सि॒ छन्दाꣳ॑सि॒ छन्दो॑भि रे॒व । \newline
27. छन्दो॑भि रे॒वैव छन्दो॑भि॒ श्छन्दो॑भि रे॒वैन॑ मेन मे॒व छन्दो॑भि॒ श्छन्दो॑भि रे॒वैन᳚म् । \newline
28. छन्दो॑भि॒रिति॒ छन्दः॑ - भिः॒ । \newline
29. ए॒वैन॑ मेन मे॒वै वैन॑म् पवयति पवय त्येन मे॒वै वैन॑म् पवयति । \newline
30. ए॒न॒म् प॒व॒य॒ति॒ प॒व॒य॒ त्ये॒न॒ मे॒न॒म् प॒व॒य॒ति॒ न॒वभि॑र् न॒वभिः॑ पवय त्येन मेनम् पवयति न॒वभिः॑ । \newline
31. प॒व॒य॒ति॒ न॒वभि॑र् न॒वभिः॑ पवयति पवयति न॒वभिः॑ पवयति पवयति न॒वभिः॑ पवयति पवयति न॒वभिः॑ पवयति । \newline
32. न॒वभिः॑ पवयति पवयति न॒वभि॑र् न॒वभिः॑ पवयति॒ नव॒ नव॑ पवयति न॒वभि॑र् न॒वभिः॑ पवयति॒ नव॑ । \newline
33. न॒वभि॒रिति॑ न॒व - भिः॒ । \newline
34. प॒व॒य॒ति॒ नव॒ नव॑ पवयति पवयति॒ नव॒ वै वै नव॑ पवयति पवयति॒ नव॒ वै । \newline
35. नव॒ वै वै नव॒ नव॒ वै पुरु॑षे॒ पुरु॑षे॒ वै नव॒ नव॒ वै पुरु॑षे । \newline
36. वै पुरु॑षे॒ पुरु॑षे॒ वै वै पुरु॑षे प्रा॒णाः प्रा॒णाः पुरु॑षे॒ वै वै पुरु॑षे प्रा॒णाः । \newline
37. पुरु॑षे प्रा॒णाः प्रा॒णाः पुरु॑षे॒ पुरु॑षे प्रा॒णाः सप्रा॑णꣳ॒॒ सप्रा॑णम् प्रा॒णाः पुरु॑षे॒ पुरु॑षे प्रा॒णाः सप्रा॑णम् । \newline
38. प्रा॒णाः सप्रा॑णꣳ॒॒ सप्रा॑णम् प्रा॒णाः प्रा॒णाः सप्रा॑ण मे॒वैव सप्रा॑णम् प्रा॒णाः प्रा॒णाः सप्रा॑ण मे॒व । \newline
39. प्रा॒णा इति॑ प्र - अ॒नाः । \newline
40. सप्रा॑ण मे॒वैव सप्रा॑णꣳ॒॒ सप्रा॑ण मे॒वैन॑ मेन मे॒व सप्रा॑णꣳ॒॒ सप्रा॑ण मे॒वैन᳚म् । \newline
41. सप्रा॑ण॒मिति॒ स - प्रा॒ण॒म् । \newline
42. ए॒वैन॑ मेन मे॒वै वैन॑म् पवयति पवय त्येन मे॒वै वैन॑म् पवयति । \newline
43. ए॒न॒म् प॒व॒य॒ति॒ प॒व॒य॒ त्ये॒न॒ मे॒न॒म् प॒व॒य॒ त्येक॑विꣳश॒ त्यैक॑विꣳशत्या पवय त्येन मेनम् पवय॒ त्येक॑विꣳशत्या । \newline
44. प॒व॒य॒ त्येक॑विꣳश॒ त्यैक॑विꣳशत्या पवयति पवय॒ त्येक॑विꣳशत्या पवयति पवय॒ त्येक॑विꣳशत्या पवयति पवय॒ त्येक॑विꣳशत्या पवयति । \newline
45. एक॑विꣳशत्या पवयति पवय॒ त्येक॑विꣳश॒ त्यैक॑विꣳशत्या पवयति॒ दश॒ दश॑ पवय॒ त्येक॑विꣳश॒ त्यैक॑विꣳशत्या पवयति॒ दश॑ । \newline
46. एक॑विꣳश॒त्येत्येक॑ - विꣳ॒॒श॒त्या॒ । \newline
47. प॒व॒य॒ति॒ दश॒ दश॑ पवयति पवयति॒ दश॒ हस्त्या॒ हस्त्या॒ दश॑ पवयति पवयति॒ दश॒ हस्त्याः᳚ । \newline
48. दश॒ हस्त्या॒ हस्त्या॒ दश॒ दश॒ हस्त्या॑ अ॒ङ्गुल॑यो॒ ऽङ्गुल॑यो॒ हस्त्या॒ दश॒ दश॒ हस्त्या॑ अ॒ङ्गुल॑यः । \newline
49. हस्त्या॑ अ॒ङ्गुल॑यो॒ ऽङ्गुल॑यो॒ हस्त्या॒ हस्त्या॑ अ॒ङ्गुल॑यो॒ दश॒ दशा॒ङ्गुल॑यो॒ हस्त्या॒ हस्त्या॑ अ॒ङ्गुल॑यो॒ दश॑ । \newline
50. अ॒ङ्गुल॑यो॒ दश॒ दशा॒ङ्गुल॑यो॒ ऽङ्गुल॑यो॒ दश॒ पद्याः॒ पद्या॒ दशा॒ङ्गुल॑यो॒ ऽङ्गुल॑यो॒ दश॒ पद्याः᳚ । \newline
51. दश॒ पद्याः॒ पद्या॒ दश॒ दश॒ पद्या॑ आ॒त्मा ऽऽत्मा पद्या॒ दश॒ दश॒ पद्या॑ आ॒त्मा । \newline
52. पद्या॑ आ॒त्मा ऽऽत्मा पद्याः॒ पद्या॑ आ॒त्मै क॑विꣳ॒॒श ए॑कविꣳ॒॒श आ॒त्मा पद्याः॒ पद्या॑ आ॒त्मै क॑विꣳ॒॒शः । \newline
53. आ॒त्मै क॑विꣳ॒॒श ए॑कविꣳ॒॒श आ॒त्मा ऽऽत्मैक॑विꣳ॒॒शो यावा॒न्॒. यावा॑ नेकविꣳ॒॒श आ॒त्मा ऽऽत्मै क॑विꣳ॒॒शो यावान्॑ । \newline
54. ए॒क॒विꣳ॒॒शो यावा॒न्॒. यावा॑ नेकविꣳ॒॒श ए॑कविꣳ॒॒शो यावा॑ ने॒वैव यावा॑ नेकविꣳ॒॒श ए॑कविꣳ॒॒शो यावा॑ ने॒व । \newline
55. ए॒क॒विꣳ॒॒श इत्येक॑ - विꣳ॒॒शः । \newline
56. यावा॑ने॒ वैव यावा॒न्॒. यावा॑ ने॒व पुरु॑षः॒ पुरु॑ष ए॒व यावा॒न्॒. यावा॑ ने॒व पुरु॑षः । \newline
57. ए॒व पुरु॑षः॒ पुरु॑ष ए॒वैव पुरु॑ष॒ स्तम् तम् पुरु॑ष ए॒वैव पुरु॑ष॒ स्तम् । \newline
58. पुरु॑ष॒ स्तम् तम् पुरु॑षः॒ पुरु॑ष॒ स्त मप॑रिवर्ग॒ मप॑रिवर्ग॒म् तम् पुरु॑षः॒ पुरु॑ष॒ स्त मप॑रिवर्गम् । \newline
59. त मप॑रिवर्ग॒ मप॑रिवर्ग॒म् तम् त मप॑रिवर्गम् पवयति पवय॒ त्यप॑रिवर्ग॒म् तम् त मप॑रिवर्गम् पवयति । \newline
60. अप॑रिवर्गम् पवयति पवय॒ त्यप॑रिवर्ग॒ मप॑रिवर्गम् पवयति चि॒त्पति॑ श्चि॒त्पतिः॑ पवय॒ त्यप॑रिवर्ग॒ मप॑रिवर्गम् पवयति चि॒त्पतिः॑ । \newline
61. अप॑रिवर्ग॒मित्यप॑रि - व॒र्ग॒म् । \newline
\pagebreak
\markright{ TS 6.1.1.9  \hfill https://www.vedavms.in \hfill}

\section{ TS 6.1.1.9 }

\textbf{TS 6.1.1.9 } \newline
\textbf{Samhita Paata} \newline

पवयति चि॒त्पति॑स्त्वा पुना॒त्वित्या॑ह॒ मनो॒ वै चि॒त्पति॒र्मन॑सै॒वैनं॑ पवयति वा॒क्पति॑स्त्वा पुना॒त्वित्या॑ह वा॒चैवैनं॑ पवयति दे॒वस्त्वा॑ सवि॒ता पु॑ना॒त्वित्या॑ह सवि॒तृप्र॑सूत ए॒वैनं॑ पवयति॒ तस्य॑ ते पवित्रपते प॒वित्रे॑ण॒ यस्मै॒ कं पु॒ने तच्छ॑केय॒मित्या॑-हा॒ऽऽ*शिष॑मे॒वैतामा शा᳚स्ते ॥ \newline

\textbf{Pada Paata} \newline

प॒व॒य॒ति॒ । चि॒त्पति॒रिति॑ चित् - पतिः॑ । त्वा॒ । पु॒ना॒तु॒ । इति॑ । आ॒ह॒ । मनः॑ । वै । चि॒त्पति॒रिति॑ चित् - पतिः॑ । मन॑सा । ए॒व । ए॒न॒म् । प॒व॒य॒ति॒ । वा॒क्पति॒रिति॑ वाक् - पतिः॑ । त्वा॒ । पु॒ना॒तु॒ । इति॑ । आ॒ह॒ । वा॒चा । ए॒व । ए॒न॒म् । प॒व॒य॒ति॒ । दे॒वः । त्वा॒ । स॒वि॒ता । पु॒ना॒तु॒ । इति॑ । आ॒ह॒ । स॒वि॒तृप्र॑सूत॒ इति॑ सवि॒तृ - प्र॒सू॒तः॒ । ए॒व । ए॒न॒म् । प॒व॒य॒ति॒ । तस्य॑ । ते॒ । प॒वि॒त्र॒प॒त॒ इति॑ पवित्र - प॒ते॒ । प॒वित्रे॑ण । यस्मै᳚ । कम् । पु॒ने । तत् । श॒के॒य॒म् । इति॑ । आ॒ह॒ । आ॒शिष॒मित्या᳚ - शिष᳚म् । ए॒व । ए॒ताम् । एति॑ । शा॒स्ते॒ ॥  \newline


\textbf{Krama Paata} \newline

प॒व॒य॒ति॒ चि॒त्पतिः॑ । चि॒त्पति॑स्त्वा । चि॒त्पति॒रिति॑ चित् - पतिः॑ । त्वा॒ पु॒ना॒तु॒ । पु॒ना॒त्विति॑ । इत्या॑ह । आ॒ह॒ मनः॑ । मनो॒ वै । वै चि॒त्पतिः॑ । चि॒त्पति॒र् मन॑सा । चि॒त्पति॒रिति॑ चित् - पतिः॑ । मन॑सै॒व । ए॒वैन᳚म् । ए॒न॒म् प॒व॒य॒ति॒ । प॒व॒य॒ति॒ वा॒क्पतिः॑ । वा॒क्पति॑स्त्वा । वा॒क्पति॒रिति॑ वाक् - पतिः॑ । त्वा॒ पु॒ना॒तु॒ । पु॒ना॒त्विति॑ । इत्या॑ह । आ॒ह॒ वा॒चा । वा॒चैव । ए॒वैन᳚म् । ए॒न॒म् प॒व॒य॒ति॒ । प॒व॒य॒ति॒ दे॒वः । दे॒वस्त्वा᳚ । त्वा॒ स॒वि॒ता । स॒वि॒ता पु॑नातु । पु॒ना॒त्विति॑ । इत्या॑ह । आ॒ह॒ स॒वि॒तृप्र॑सूतः । स॒वि॒तृप्र॑सूत ए॒व । स॒वि॒तृप्र॑सूत॒ इति॑ सवि॒तृ - प्र॒सू॒तः॒ । ए॒वैन᳚म् । ए॒न॒म् प॒व॒य॒ति॒ । प॒व॒य॒ति॒ तस्य॑ । तस्य॑ ते । ते॒ प॒वि॒त्र॒प॒ते॒ । प॒वि॒त्र॒प॒ते॒ प॒वित्रे॑ण । प॒वि॒त्र॒प॒त॒ इति॑ पवित्र - प॒ते॒ । प॒वित्रे॑ण॒ यस्मै᳚ । यस्मै॒ कम् । कम् पु॒ने । पु॒ने तत् । तच्छ॑केयम् । श॒के॒य॒मिति॑ । इत्या॑ह । आ॒हा॒शिष᳚म् । आ॒शिष॑मे॒व । आ॒शिष॒मित्या᳚ - शिष᳚म् । ए॒वैताम् । ए॒तामा । आ शा᳚स्ते । शा॒स्त॒ इति॑ शास्ते । \newline

\textbf{Jatai Paata} \newline

1. प॒व॒य॒ति॒ चि॒त्पति॑ श्चि॒त्पतिः॑ पवयति पवयति चि॒त्पतिः॑ । \newline
2. चि॒त्पति॑ स्त्वा त्वा चि॒त्पति॑ श्चि॒त्पति॑ स्त्वा । \newline
3. चि॒त्पति॒रिति॑ चित् - पतिः॑ । \newline
4. त्वा॒ पु॒ना॒तु॒ पु॒ना॒तु॒ त्वा॒ त्वा॒ पु॒ना॒तु॒ । \newline
5. पु॒ना॒ त्वितीति॑ पुनातु पुना॒ त्विति॑ । \newline
6. इत्या॑हा॒हे तीत्या॑ह । \newline
7. आ॒ह॒ मनो॒ मन॑ आहाह॒ मनः॑ । \newline
8. मनो॒ वै वै मनो॒ मनो॒ वै । \newline
9. वै चि॒त्पति॑ श्चि॒त्पति॒र् वै वै चि॒त्पतिः॑ । \newline
10. चि॒त्पति॒र् मन॑सा॒ मन॑सा चि॒त्पति॑ श्चि॒त्पति॒र् मन॑सा । \newline
11. चि॒त्पति॒रिति॑ चित् - पतिः॑ । \newline
12. मन॑सै॒ वैव मन॑सा॒ मन॑ सै॒व । \newline
13. ए॒वैन॑ मेन मे॒वै वैन᳚म् । \newline
14. ए॒न॒म् प॒व॒य॒ति॒ प॒व॒य॒ त्ये॒न॒ मे॒न॒म् प॒व॒य॒ति॒ । \newline
15. प॒व॒य॒ति॒ वा॒क्पति॑र् वा॒क्पतिः॑ पवयति पवयति वा॒क्पतिः॑ । \newline
16. वा॒क्पति॑ स्त्वा त्वा वा॒क्पति॑र् वा॒क्पति॑ स्त्वा । \newline
17. वा॒क्पति॒रिति॑ वाक् - पतिः॑ । \newline
18. त्वा॒ पु॒ना॒तु॒ पु॒ना॒तु॒ त्वा॒ त्वा॒ पु॒ना॒तु॒ । \newline
19. पु॒ना॒ त्वितीति॑ पुनातु पुना॒ त्विति॑ । \newline
20. इत्या॑हा॒हे तीत्या॑ह । \newline
21. आ॒ह॒ वा॒चा वा॒चा ऽऽहा॑ह वा॒चा । \newline
22. वा॒चै वैव वा॒चा वा॒चैव । \newline
23. ए॒वैन॑ मेन मे॒वै वैन᳚म् । \newline
24. ए॒न॒म् प॒व॒य॒ति॒ प॒व॒य॒ त्ये॒न॒ मे॒न॒म् प॒व॒य॒ति॒ । \newline
25. प॒व॒य॒ति॒ दे॒वो दे॒वः प॑वयति पवयति दे॒वः । \newline
26. दे॒व स्त्वा᳚ त्वा दे॒वो दे॒व स्त्वा᳚ । \newline
27. त्वा॒ स॒वि॒ता स॑वि॒ता त्वा᳚ त्वा सवि॒ता । \newline
28. स॒वि॒ता पु॑नातु पुनातु सवि॒ता स॑वि॒ता पु॑नातु । \newline
29. पु॒ना॒ त्वितीति॑ पुनातु पुना॒ त्विति॑ । \newline
30. इत्या॑हा॒हे तीत्या॑ह । \newline
31. आ॒ह॒ स॒वि॒तृप्र॑सूतः सवि॒तृप्र॑सूत आहाह सवि॒तृप्र॑सूतः । \newline
32. स॒वि॒तृप्र॑सूत ए॒वैव स॑वि॒तृप्र॑सूतः सवि॒तृप्र॑सूत ए॒व । \newline
33. स॒वि॒तृप्र॑सूत॒ इति॑ सवि॒तृ - प्र॒सू॒तः॒ । \newline
34. ए॒वैन॑ मेन मे॒वै वैन᳚म् । \newline
35. ए॒न॒म् प॒व॒य॒ति॒ प॒व॒य॒ त्ये॒न॒ मे॒न॒म् प॒व॒य॒ति॒ । \newline
36. प॒व॒य॒ति॒ तस्य॒ तस्य॑ पवयति पवयति॒ तस्य॑ । \newline
37. तस्य॑ ते ते॒ तस्य॒ तस्य॑ ते । \newline
38. ते॒ प॒वि॒त्र॒प॒ते॒ प॒वि॒त्र॒प॒ते॒ ते॒ ते॒ प॒वि॒त्र॒प॒ते॒ । \newline
39. प॒वि॒त्र॒प॒ते॒ प॒वित्रे॑ण प॒वित्रे॑ण पवित्रपते पवित्रपते प॒वित्रे॑ण । \newline
40. प॒वि॒त्र॒प॒त॒ इति॑ पवित्र - प॒ते॒ । \newline
41. प॒वित्रे॑ण॒ यस्मै॒ यस्मै॑ प॒वित्रे॑ण प॒वित्रे॑ण॒ यस्मै᳚ । \newline
42. यस्मै॒ कम् कं ॅयस्मै॒ यस्मै॒ कम् । \newline
43. कम् पु॒ने पु॒ने कम् कम् पु॒ने । \newline
44. पु॒ने तत् तत् पु॒ने पु॒ने तत् । \newline
45. तच्छ॑केयꣳ शकेय॒म् तत् तच्छ॑केयम् । \newline
46. श॒के॒य॒ मितीति॑ शकेयꣳ शकेय॒ मिति॑ । \newline
47. इत्या॑हा॒हे तीत्या॑ह । \newline
48. आ॒हा॒शिष॑ मा॒शिष॑ माहाहा॒ शिष᳚म् । \newline
49. आ॒शिष॑ मे॒वै वाशिष॑ मा॒शिष॑ मे॒व । \newline
50. आ॒शिष॒मित्या᳚ - शिष᳚म् । \newline
51. ए॒वैता मे॒ता मे॒वै वैताम् । \newline
52. ए॒ता मैता मे॒ता मा । \newline
53. आ शा᳚स्ते शास्त॒ आ शा᳚स्ते । \newline
54. शा॒स्त॒ इति॑ शास्ते । \newline

\textbf{Ghana Paata } \newline

1. प॒व॒य॒ति॒ चि॒त्पति॑ श्चि॒त्पतिः॑ पवयति पवयति चि॒त्पति॑ स्त्वा त्वा चि॒त्पतिः॑ पवयति पवयति चि॒त्पति॑ स्त्वा । \newline
2. चि॒त्पति॑ स्त्वा त्वा चि॒त्पति॑ श्चि॒त्पति॑ स्त्वा पुनातु पुनातु त्वा चि॒त्पति॑ श्चि॒त्पति॑ स्त्वा पुनातु । \newline
3. चि॒त्पति॒रिति॑ चित् - पतिः॑ । \newline
4. त्वा॒ पु॒ना॒तु॒ पु॒ना॒तु॒ त्वा॒ त्वा॒ पु॒ना॒ त्वितीति॑ पुनातु त्वा त्वा पुना॒ त्विति॑ । \newline
5. पु॒ना॒ त्वितीति॑ पुनातु पुना॒ त्वित्या॑ हा॒हेति॑ पुनातु पुना॒ त्वित्या॑ह । \newline
6. इत्या॑हा॒हे तीत्या॑ह॒ मनो॒ मन॑ आ॒हे तीत्या॑ह॒ मनः॑ । \newline
7. आ॒ह॒ मनो॒ मन॑ आहाह॒ मनो॒ वै वै मन॑ आहाह॒ मनो॒ वै । \newline
8. मनो॒ वै वै मनो॒ मनो॒ वै चि॒त्पति॑ श्चि॒त्पति॒र् वै मनो॒ मनो॒ वै चि॒त्पतिः॑ । \newline
9. वै चि॒त्पति॑ श्चि॒त्पति॒र् वै वै चि॒त्पति॒र् मन॑सा॒ मन॑सा चि॒त्पति॒र् वै वै चि॒त्पति॒र् मन॑सा । \newline
10. चि॒त्पति॒र् मन॑सा॒ मन॑सा चि॒त्पति॑ श्चि॒त्पति॒र् मन॑सै॒ वैव मन॑सा चि॒त्पति॑ श्चि॒त्पति॒र् मन॑सै॒व । \newline
11. चि॒त्पति॒रिति॑ चित् - पतिः॑ । \newline
12. मन॑सै॒ वैव मन॑सा॒ मन॑सै॒ वैन॑ मेन मे॒व मन॑सा॒ मन॑सै॒ वैन᳚म् । \newline
13. ए॒वैन॑ मेन मे॒वै वैन॑म् पवयति पवय त्येन मे॒वै वैन॑म् पवयति । \newline
14. ए॒न॒म् प॒व॒य॒ति॒ प॒व॒य॒ त्ये॒न॒ मे॒न॒म् प॒व॒य॒ति॒ वा॒क्पति॑र् वा॒क्पतिः॑ पवय त्येन मेनम् पवयति वा॒क्पतिः॑ । \newline
15. प॒व॒य॒ति॒ वा॒क्पति॑र् वा॒क्पतिः॑ पवयति पवयति वा॒क्पति॑ स्त्वा त्वा वा॒क्पतिः॑ पवयति पवयति वा॒क्पति॑ स्त्वा । \newline
16. वा॒क्पति॑ स्त्वा त्वा वा॒क्पति॑र् वा॒क्पति॑ स्त्वा पुनातु पुनातु त्वा वा॒क्पति॑र् वा॒क्पति॑ स्त्वा पुनातु । \newline
17. वा॒क्पति॒रिति॑ वाक् - पतिः॑ । \newline
18. त्वा॒ पु॒ना॒तु॒ पु॒ना॒तु॒ त्वा॒ त्वा॒ पु॒ना॒ त्वितीति॑ पुनातु त्वा त्वा पुना॒ त्विति॑ । \newline
19. पु॒ना॒ त्वितीति॑ पुनातु पुना॒ त्वित्या॑ हा॒हेति॑ पुनातु पुना॒ त्वित्या॑ह । \newline
20. इत्या॑हा॒हे तीत्या॑ह वा॒चा वा॒चा ऽऽहे तीत्या॑ह वा॒चा । \newline
21. आ॒ह॒ वा॒चा वा॒चा ऽऽहा॑ह वा॒चै वैव वा॒चा ऽऽहा॑ह वा॒चैव । \newline
22. वा॒चै वैव वा॒चा वा॒चै वैन॑ मेन मे॒व वा॒चा वा॒चै वैन᳚म् । \newline
23. ए॒वैन॑ मेन मे॒वै वैन॑म् पवयति पवय त्येन मे॒वै वैन॑म् पवयति । \newline
24. ए॒न॒म् प॒व॒य॒ति॒ प॒व॒य॒ त्ये॒न॒ मे॒न॒म् प॒व॒य॒ति॒ दे॒वो दे॒वः प॑वय त्येन मेनम् पवयति दे॒वः । \newline
25. प॒व॒य॒ति॒ दे॒वो दे॒वः प॑वयति पवयति दे॒व स्त्वा᳚ त्वा दे॒वः प॑वयति पवयति दे॒व स्त्वा᳚ । \newline
26. दे॒व स्त्वा᳚ त्वा दे॒वो दे॒व स्त्वा॑ सवि॒ता स॑वि॒ता त्वा॑ दे॒वो दे॒व स्त्वा॑ सवि॒ता । \newline
27. त्वा॒ स॒वि॒ता स॑वि॒ता त्वा᳚ त्वा सवि॒ता पु॑नातु पुनातु सवि॒ता त्वा᳚ त्वा सवि॒ता पु॑नातु । \newline
28. स॒वि॒ता पु॑नातु पुनातु सवि॒ता स॑वि॒ता पु॑ना॒ त्वितीति॑ पुनातु सवि॒ता स॑वि॒ता पु॑ना॒ त्विति॑ । \newline
29. पु॒ना॒ त्वितीति॑ पुनातु पुना॒ त्वित्या॑ हा॒हेति॑ पुनातु पुना॒ त्वित्या॑ह । \newline
30. इत्या॑हा॒हे तीत्या॑ह सवि॒तृप्र॑सूतः सवि॒तृप्र॑सूत आ॒हे तीत्या॑ह सवि॒तृप्र॑सूतः । \newline
31. आ॒ह॒ स॒वि॒तृप्र॑सूतः सवि॒तृप्र॑सूत आहाह सवि॒तृप्र॑सूत ए॒वैव स॑वि॒तृप्र॑सूत आहाह सवि॒तृप्र॑सूत ए॒व । \newline
32. स॒वि॒तृप्र॑सूत ए॒वैव स॑वि॒तृप्र॑सूतः सवि॒तृप्र॑सूत ए॒वैन॑ मेन मे॒व स॑वि॒तृप्र॑सूतः सवि॒तृप्र॑सूत ए॒वैन᳚म् । \newline
33. स॒वि॒तृप्र॑सूत॒ इति॑ सवि॒तृ - प्र॒सू॒तः॒ । \newline
34. ए॒वैन॑ मेन मे॒वै वैन॑म् पवयति पवय त्येन मे॒वै वैन॑म् पवयति । \newline
35. ए॒न॒म् प॒व॒य॒ति॒ प॒व॒य॒ त्ये॒न॒ मे॒न॒म् प॒व॒य॒ति॒ तस्य॒ तस्य॑ पवय त्येन मेनम् पवयति॒ तस्य॑ । \newline
36. प॒व॒य॒ति॒ तस्य॒ तस्य॑ पवयति पवयति॒ तस्य॑ ते ते॒ तस्य॑ पवयति पवयति॒ तस्य॑ ते । \newline
37. तस्य॑ ते ते॒ तस्य॒ तस्य॑ ते पवित्रपते पवित्रपते ते॒ तस्य॒ तस्य॑ ते पवित्रपते । \newline
38. ते॒ प॒वि॒त्र॒प॒ते॒ प॒वि॒त्र॒प॒ते॒ ते॒ ते॒ प॒वि॒त्र॒प॒ते॒ प॒वित्रे॑ण प॒वित्रे॑ण पवित्रपते ते ते पवित्रपते प॒वित्रे॑ण । \newline
39. प॒वि॒त्र॒प॒ते॒ प॒वित्रे॑ण प॒वित्रे॑ण पवित्रपते पवित्रपते प॒वित्रे॑ण॒ यस्मै॒ यस्मै॑ प॒वित्रे॑ण पवित्रपते पवित्रपते प॒वित्रे॑ण॒ यस्मै᳚ । \newline
40. प॒वि॒त्र॒प॒त॒ इति॑ पवित्र - प॒ते॒ । \newline
41. प॒वित्रे॑ण॒ यस्मै॒ यस्मै॑ प॒वित्रे॑ण प॒वित्रे॑ण॒ यस्मै॒ कम् कं ॅयस्मै॑ प॒वित्रे॑ण प॒वित्रे॑ण॒ यस्मै॒ कम् । \newline
42. यस्मै॒ कम् कं ॅयस्मै॒ यस्मै॒ कम् पु॒ने पु॒ने कं ॅयस्मै॒ यस्मै॒ कम् पु॒ने । \newline
43. कम् पु॒ने पु॒ने कम् कम् पु॒ने तत् तत् पु॒ने कम् कम् पु॒ने तत् । \newline
44. पु॒ने तत् तत् पु॒ने पु॒ने तच्छ॑केयꣳ शकेय॒म् तत् पु॒ने पु॒ने तच्छ॑केयम् । \newline
45. तच्छ॑केयꣳ शकेय॒म् तत् तच्छ॑केय॒ मितीति॑ शकेय॒म् तत् तच्छ॑केय॒ मिति॑ । \newline
46. श॒के॒य॒ मितीति॑ शकेयꣳ शकेय॒ मित्या॑ हा॒हेति॑ शकेयꣳ शकेय॒ मित्या॑ह । \newline
47. इत्या॑हा॒हे तीत्या॑ हा॒शिष॑ मा॒शिष॑ मा॒हे तीत्या॑ हा॒शिष᳚म् । \newline
48. आ॒हा॒शिष॑ मा॒शिष॑ माहा हा॒शिष॑ मे॒वै वाशिष॑ माहा हा॒शिष॑ मे॒व । \newline
49. आ॒शिष॑ मे॒वै वाशिष॑ मा॒शिष॑ मे॒वैता मे॒ता मे॒वा शिष॑ मा॒शिष॑ मे॒वैताम् । \newline
50. आ॒शिष॒मित्या᳚ - शिष᳚म् । \newline
51. ए॒वैता मे॒ता मे॒वै वैता मैता मे॒वै वैता मा । \newline
52. ए॒ता मैता मे॒ता मा शा᳚स्ते शास्त॒ ऐता मे॒ता मा शा᳚स्ते । \newline
53. आ शा᳚स्ते शास्त॒ आ शा᳚स्ते । \newline
54. शा॒स्त॒ इति॑ शास्ते । \newline
\pagebreak
\markright{ TS 6.1.2.1  \hfill https://www.vedavms.in \hfill}

\section{ TS 6.1.2.1 }

\textbf{TS 6.1.2.1 } \newline
\textbf{Samhita Paata} \newline

याव॑न्तो॒ वै दे॒वा य॒ज्ञायापु॑नत॒ त ए॒वाभ॑व॒न॒. य ए॒वं ॅवि॒द्वान्. य॒ज्ञाय॑ पुनी॒ते भव॑त्ये॒व ब॒हिः प॑वयि॒त्वाऽन्तः प्र पा॑दयति मनुष्यलो॒क ए॒वैनं॑ पवयि॒त्वा पू॒तं दे॑वलो॒कं प्र ण॑य॒त्यदी᳚क्षित॒ एक॒याऽऽहु॒त्येत्या॑हुः स्रु॒वेण॒ चत॑स्रो जुहोति दीक्षित॒त्वाय॑ स्रु॒चा प॑ञ्च॒मीं पञ्चा᳚क्षरा प॒ङ्क्तिः पाङ्क्तो॑ य॒ज्ञो य॒ज्ञ्मे॒वाव॑ रुन्ध॒ आकू᳚त्यै प्र॒युजे॒ऽग्नये॒ - [  ] \newline

\textbf{Pada Paata} \newline

याव॑न्तः । वै । दे॒वाः । य॒ज्ञाय॑ । अपु॑नत । ते । ए॒व । अ॒भ॒व॒न्न् । यः । ए॒वम् । वि॒द्वान् । य॒ज्ञाय॑ । पु॒नी॒ते । भव॑ति । ए॒व । ब॒हिः । प॒व॒यि॒त्वा । अ॒न्तः । प्रेति॑ । पा॒द॒य॒ति॒ । म॒नु॒ष्य॒लो॒क इति॑ मनुष्य - लो॒के । ए॒व । ए॒न॒म् । प॒व॒यि॒त्वा । पू॒तम् । दे॒व॒लो॒कमिति॑ देव - लो॒कम् । प्रेति॑ । न॒य॒ति॒ । अदी᳚क्षितः । एक॑या । आहु॒त्येत्या - हु॒त्या॒ । इति॑ । आ॒हुः॒ । स्रु॒वेण॑ । चत॑स्रः । जु॒हो॒ति॒ । दी॒क्षि॒त॒त्वायेति॑ दीक्षित-त्वाय॑ । स्रु॒चा । प॒ञ्च॒मीम् । पञ्चा᳚क्ष॒रेति॒ पञ्च॑ - अ॒क्ष॒रा॒ । प॒ङ्क्तिः । पाङ्क्तः॑ । य॒ज्ञ्ः । य॒ज्ञ्म् । ए॒व । अवेति॑ । रु॒न्धे॒ । आकू᳚त्या॒ इत्या - कू॒त्यै॒ । प्र॒युज॒ इति॑ प्र - युजे᳚ । अ॒ग्नये᳚ ।  \newline


\textbf{Krama Paata} \newline

याव॑न्तो॒ वै । वै दे॒वाः । दे॒वा य॒ज्ञाय॑ । य॒ज्ञायापु॑नत । अपु॑नत॒ ते । त ए॒व । ए॒वाभ॑वन्न् । अ॒भ॒व॒न्॒. यः । य ए॒वम् । ए॒वम् ॅवि॒द्वान् । वि॒द्वान्. य॒ज्ञाय॑ । य॒ज्ञाय॑ पुनी॒ते । पु॒नी॒ते भव॑ति । भव॑त्ये॒व । ए॒व ब॒हिः । ब॒हिः प॑वयि॒त्वा । प॒व॒यि॒त्वाऽन्तः । अ॒न्तः प्र । प्र पा॑दयति । पा॒द॒य॒ति॒ म॒नु॒ष्य॒लो॒के । म॒नु॒ष्य॒लो॒क ए॒व । म॒नु॒ष्य॒लो॒क इति॑ मनुष्य - लो॒के । ए॒वैन᳚म् । ए॒न॒म् प॒व॒यि॒त्वा । प॒व॒यि॒त्वा पू॒तम् । पू॒तम् दे॑वलो॒कम् । दे॒व॒लो॒कम् प्र । दे॒व॒लो॒कमिति॑ देव - लो॒कम् । प्र ण॑यति । न॒य॒त्यदी᳚क्षितः । अदी᳚क्षित॒ एक॑या । एक॒याऽऽहु॑त्या । आहु॒त्येति॑ । आहु॒त्येत्या - हु॒त्या॒ । इत्या॑हुः । आ॒हुः॒ स्रु॒वेण॑ । स्रु॒वेण॒ चत॑स्रः । चत॑स्रो जुहोति । जु॒हो॒ति॒ दी॒क्षि॒त॒त्वाय॑ । दी॒क्षि॒त॒त्वाय॑ स्रु॒चा । दी॒क्षि॒त॒त्वायेति॑ दीक्षित - त्वाय॑ । स्रु॒चा प॑ञ्च॒मीम् । प॒ञ्च॒मीम् पञ्चा᳚क्षरा । पञ्चा᳚क्षरा प॒ङ्‍क्तिः । पञ्चा᳚क्ष॒रेति॒ पञ्च॑ - अ॒क्ष॒रा॒ । प॒ङ्‍क्तिः पाङ्‍क्तः॑ । पाङ्‍क्तो॑ य॒ज्ञ्ः । य॒ज्ञो य॒ज्ञ्म् । य॒ज्ञ्मे॒व । ए॒वाव॑ । अव॑ रुन्धे । रु॒न्ध॒ आकू᳚त्यै । आकू᳚त्यै प्र॒युजे᳚ । आकू᳚त्या॒ इत्या - कू॒त्यै॒ । प्र॒युजे॒ऽग्नये᳚ । प्र॒युज॒ इति॑ प्र - युजे᳚ । अ॒ग्नये॒ स्वाहा᳚ \newline

\textbf{Jatai Paata} \newline

1. याव॑न्तो॒ वै वै याव॑न्तो॒ याव॑न्तो॒ वै । \newline
2. वै दे॒वा दे॒वा वै वै दे॒वाः । \newline
3. दे॒वा य॒ज्ञाय॑ य॒ज्ञाय॑ दे॒वा दे॒वा य॒ज्ञाय॑ । \newline
4. य॒ज्ञाया पु॑न॒ता पु॑नत य॒ज्ञाय॑ य॒ज्ञाया पु॑नत । \newline
5. अपु॑नत॒ ते ते ऽपु॑न॒ता पु॑नत॒ ते । \newline
6. त ए॒वैव ते त ए॒व । \newline
7. ए॒वाभ॑वन् नभवन्ने॒ वैवा भ॑वन्न् । \newline
8. अ॒भ॒व॒न्॒. यो यो॑ ऽभवन्न भव॒न्॒. यः । \newline
9. य ए॒व मे॒वं ॅयो य ए॒वम् । \newline
10. ए॒वं ॅवि॒द्वान्. वि॒द्वाने॒व मे॒वं ॅवि॒द्वान् । \newline
11. वि॒द्वान्. य॒ज्ञाय॑ य॒ज्ञाय॑ वि॒द्वान्. वि॒द्वान्. य॒ज्ञाय॑ । \newline
12. य॒ज्ञाय॑ पुनी॒ते पु॑नी॒ते य॒ज्ञाय॑ य॒ज्ञाय॑ पुनी॒ते । \newline
13. पु॒नी॒ते भव॑ति॒ भव॑ति पुनी॒ते पु॑नी॒ते भव॑ति । \newline
14. भव॑ त्ये॒वैव भव॑ति॒ भव॑ त्ये॒व । \newline
15. ए॒व ब॒हिर् ब॒हि रे॒वैव ब॒हिः । \newline
16. ब॒हिः प॑वयि॒त्वा प॑वयि॒त्वा ब॒हिर् ब॒हिः प॑वयि॒त्वा । \newline
17. प॒व॒यि॒त्वा ऽन्त र॒न्तः प॑वयि॒त्वा प॑वयि॒त्वा ऽन्तः । \newline
18. अ॒न्तः प्र प्रान्त र॒न्तः प्र । \newline
19. प्र पा॑दयति पादयति॒ प्र प्र पा॑दयति । \newline
20. पा॒द॒य॒ति॒ म॒नु॒ष्य॒लो॒के म॑नुष्यलो॒के पा॑दयति पादयति मनुष्यलो॒के । \newline
21. म॒नु॒ष्य॒लो॒क ए॒वैव म॑नुष्यलो॒के म॑नुष्यलो॒क ए॒व । \newline
22. म॒नु॒ष्य॒लो॒क इति॑ मनुष्य - लो॒के । \newline
23. ए॒वैन॑ मेन मे॒वै वैन᳚म् । \newline
24. ए॒न॒म् प॒व॒यि॒त्वा प॑वयि॒त्वैन॑ मेनम् पवयि॒त्वा । \newline
25. प॒व॒यि॒त्वा पू॒तम् पू॒तम् प॑वयि॒त्वा प॑वयि॒त्वा पू॒तम् । \newline
26. पू॒तम् दे॑वलो॒कम् दे॑वलो॒कम् पू॒तम् पू॒तम् दे॑वलो॒कम् । \newline
27. दे॒व॒लो॒कम् प्र प्र दे॑वलो॒कम् दे॑वलो॒कम् प्र । \newline
28. दे॒व॒लो॒कमिति॑ देव - लो॒कम् । \newline
29. प्र ण॑यति नयति॒ प्र प्र ण॑यति । \newline
30. न॒य॒ त्यदी᳚क्षि॒तो ऽदी᳚क्षितो नयति नय॒ त्यदी᳚क्षितः । \newline
31. अदी᳚क्षित॒ एक॒यैक॒या ऽदी᳚क्षि॒तो ऽदी᳚क्षित॒ एक॑या । \newline
32. एक॒या ऽऽहु॒त्या ऽऽहु॒ त्यैक॒ यैक॒या ऽऽहु॑त्या । \newline
33. आहु॒त्ये तीत्या हु॒त्या ऽऽहु॒ त्येति॑ । \newline
34. आहु॒त्येत्या - हु॒त्या॒ । \newline
35. इत्या॑हु राहु॒ रिती त्या॑हुः । \newline
36. आ॒हुः॒ स्रु॒वेण॑ स्रु॒वेणा॑ हुराहुः स्रु॒वेण॑ । \newline
37. स्रु॒वेण॒ चत॑स्र॒ श्चत॑स्रः स्रु॒वेण॑ स्रु॒वेण॒ चत॑स्रः । \newline
38. चत॑स्रो जुहोति जुहोति॒ चत॑स्र॒ श्चत॑स्रो जुहोति । \newline
39. जु॒हो॒ति॒ दी॒क्षि॒त॒त्वाय॑ दीक्षित॒त्वाय॑ जुहोति जुहोति दीक्षित॒त्वाय॑ । \newline
40. दी॒क्षि॒त॒त्वाय॑ स्रु॒चा स्रु॒चा दी᳚क्षित॒त्वाय॑ दीक्षित॒त्वाय॑ स्रु॒चा । \newline
41. दी॒क्षि॒त॒त्वायेति॑ दीक्षित - त्वाय॑ । \newline
42. स्रु॒चा प॑ञ्च॒मीम् प॑ञ्च॒मीꣳ स्रु॒चा स्रु॒चा प॑ञ्च॒मीम् । \newline
43. प॒ञ्च॒मीम् पञ्चा᳚क्षरा॒ पञ्चा᳚क्षरा पञ्च॒मीम् प॑ञ्च॒मीम् पञ्चा᳚क्षरा । \newline
44. पञ्चा᳚क्षरा प॒ङ्क्तिः प॒ङ्क्तिः पञ्चा᳚क्षरा॒ पञ्चा᳚क्षरा प॒ङ्क्तिः । \newline
45. पञ्चा᳚क्ष॒रेति॒ पञ्च॑ - अ॒क्ष॒रा॒ । \newline
46. प॒ङ्क्तिः पाङ्क्तः॒ पाङ्क्तः॑ प॒ङ्क्तिः प॒ङ्क्तिः पाङ्क्तः॑ । \newline
47. पाङ्क्तो॑ य॒ज्ञो य॒ज्ञ्ः पाङ्क्तः॒ पाङ्क्तो॑ य॒ज्ञ्ः । \newline
48. य॒ज्ञो य॒ज्ञ्ं ॅय॒ज्ञ्ं ॅय॒ज्ञो य॒ज्ञो य॒ज्ञ्म् । \newline
49. य॒ज्ञ् मे॒वैव य॒ज्ञ्ं ॅय॒ज्ञ् मे॒व । \newline
50. ए॒वावा वै॒वै वाव॑ । \newline
51. अव॑ रुन्धे रु॒न्धे ऽवाव॑ रुन्धे । \newline
52. रु॒न्ध॒ आकू᳚त्या॒ आकू᳚त्यै रुन्धे रुन्ध॒ आकू᳚त्यै । \newline
53. आकू᳚त्यै प्र॒युजे᳚ प्र॒युज॒ आकू᳚त्या॒ आकू᳚त्यै प्र॒युजे᳚ । \newline
54. आकू᳚त्या॒ इत्या - कू॒त्यै॒ । \newline
55. प्र॒युजे॒ ऽग्नये॒ ऽग्नये᳚ प्र॒युजे᳚ प्र॒युजे॒ ऽग्नये᳚ । \newline
56. प्र॒युज॒ इति॑ प्र - युजे᳚ । \newline
57. अ॒ग्नये॒ स्वाहा॒ स्वाहा॒ ऽग्नये॒ ऽग्नये॒ स्वाहा᳚ । \newline

\textbf{Ghana Paata } \newline

1. याव॑न्तो॒ वै वै याव॑न्तो॒ याव॑न्तो॒ वै दे॒वा दे॒वा वै याव॑न्तो॒ याव॑न्तो॒ वै दे॒वाः । \newline
2. वै दे॒वा दे॒वा वै वै दे॒वा य॒ज्ञाय॑ य॒ज्ञाय॑ दे॒वा वै वै दे॒वा य॒ज्ञाय॑ । \newline
3. दे॒वा य॒ज्ञाय॑ य॒ज्ञाय॑ दे॒वा दे॒वा य॒ज्ञाया पु॑न॒ता पु॑नत य॒ज्ञाय॑ दे॒वा दे॒वा य॒ज्ञाया पु॑नत । \newline
4. य॒ज्ञाया पु॑न॒ता पु॑नत य॒ज्ञाय॑ य॒ज्ञाया पु॑नत॒ ते ते ऽपु॑नत य॒ज्ञाय॑ य॒ज्ञाया पु॑नत॒ ते । \newline
5. अपु॑नत॒ ते ते ऽपु॑न॒ता पु॑नत॒ त ए॒वैव ते ऽपु॑न॒ता पु॑नत॒ त ए॒व । \newline
6. त ए॒वैव ते त ए॒वाभ॑वन् नभवन् ने॒व ते त ए॒वाभ॑वन्न् । \newline
7. ए॒वाभ॑वन् नभवन् ने॒वै वाभ॑व॒न्॒. यो यो॑ ऽभवन् ने॒वै वाभ॑व॒न्॒. यः । \newline
8. अ॒भ॒व॒न्॒. यो यो॑ ऽभवन् नभव॒न्॒. य ए॒व मे॒वं ॅयो॑ ऽभवन् नभव॒न्॒. य ए॒वम् । \newline
9. य ए॒व मे॒वं ॅयो य ए॒वं ॅवि॒द्वान्. वि॒द्वा ने॒वं ॅयो य ए॒वं ॅवि॒द्वान् । \newline
10. ए॒वं ॅवि॒द्वान्. वि॒द्वा ने॒व मे॒वं ॅवि॒द्वान्. य॒ज्ञाय॑ य॒ज्ञाय॑ वि॒द्वा ने॒व मे॒वं ॅवि॒द्वान्. य॒ज्ञाय॑ । \newline
11. वि॒द्वान्. य॒ज्ञाय॑ य॒ज्ञाय॑ वि॒द्वान्. वि॒द्वान्. य॒ज्ञाय॑ पुनी॒ते पु॑नी॒ते य॒ज्ञाय॑ वि॒द्वान्. वि॒द्वान्. य॒ज्ञाय॑ पुनी॒ते । \newline
12. य॒ज्ञाय॑ पुनी॒ते पु॑नी॒ते य॒ज्ञाय॑ य॒ज्ञाय॑ पुनी॒ते भव॑ति॒ भव॑ति पुनी॒ते य॒ज्ञाय॑ य॒ज्ञाय॑ पुनी॒ते भव॑ति । \newline
13. पु॒नी॒ते भव॑ति॒ भव॑ति पुनी॒ते पु॑नी॒ते भव॑ त्ये॒वैव भव॑ति पुनी॒ते पु॑नी॒ते भव॑ त्ये॒व । \newline
14. भव॑ त्ये॒वैव भव॑ति॒ भव॑ त्ये॒व ब॒हिर् ब॒हि रे॒व भव॑ति॒ भव॑ त्ये॒व ब॒हिः । \newline
15. ए॒व ब॒हिर् ब॒हि रे॒वैव ब॒हिः प॑वयि॒त्वा प॑वयि॒त्वा ब॒हि रे॒वैव ब॒हिः प॑वयि॒त्वा । \newline
16. ब॒हिः प॑वयि॒त्वा प॑वयि॒त्वा ब॒हिर् ब॒हिः प॑वयि॒त्वा ऽन्त र॒न्तः प॑वयि॒त्वा ब॒हिर् ब॒हिः प॑वयि॒त्वा ऽन्तः । \newline
17. प॒व॒यि॒त्वा ऽन्त र॒न्तः प॑वयि॒त्वा प॑वयि॒त्वा ऽन्तः प्र प्रान्तः प॑वयि॒त्वा प॑वयि॒त्वा ऽन्तः प्र । \newline
18. अ॒न्तः प्र प्रान्त र॒न्तः प्र पा॑दयति पादयति॒ प्रान्त र॒न्तः प्र पा॑दयति । \newline
19. प्र पा॑दयति पादयति॒ प्र प्र पा॑दयति मनुष्यलो॒के म॑नुष्यलो॒के पा॑दयति॒ प्र प्र पा॑दयति मनुष्यलो॒के । \newline
20. पा॒द॒य॒ति॒ म॒नु॒ष्य॒लो॒के म॑नुष्यलो॒के पा॑दयति पादयति मनुष्यलो॒क ए॒वैव म॑नुष्यलो॒के पा॑दयति पादयति मनुष्यलो॒क ए॒व । \newline
21. म॒नु॒ष्य॒लो॒क ए॒वैव म॑नुष्यलो॒के म॑नुष्यलो॒क ए॒वैन॑ मेन मे॒व म॑नुष्यलो॒के म॑नुष्यलो॒क ए॒वैन᳚म् । \newline
22. म॒नु॒ष्य॒लो॒क इति॑ मनुष्य - लो॒के । \newline
23. ए॒वैन॑ मेन मे॒वै वैन॑म् पवयि॒त्वा प॑वयि॒ त्वैन॑ मे॒वै वैन॑म् पवयि॒त्वा । \newline
24. ए॒न॒म् प॒व॒यि॒त्वा प॑वयि॒ त्वैन॑ मेनम् पवयि॒त्वा पू॒तम् पू॒तम् प॑वयि॒ त्वैन॑ मेनम् पवयि॒त्वा पू॒तम् । \newline
25. प॒व॒यि॒त्वा पू॒तम् पू॒तम् प॑वयि॒त्वा प॑वयि॒त्वा पू॒तम् दे॑वलो॒कम् दे॑वलो॒कम् पू॒तम् प॑वयि॒त्वा प॑वयि॒त्वा पू॒तम् दे॑वलो॒कम् । \newline
26. पू॒तम् दे॑वलो॒कम् दे॑वलो॒कम् पू॒तम् पू॒तम् दे॑वलो॒कम् प्र प्र दे॑वलो॒कम् पू॒तम् पू॒तम् दे॑वलो॒कम् प्र । \newline
27. दे॒व॒लो॒कम् प्र प्र दे॑वलो॒कम् दे॑वलो॒कम् प्र ण॑यति नयति॒ प्र दे॑वलो॒कम् दे॑वलो॒कम् प्र ण॑यति । \newline
28. दे॒व॒लो॒कमिति॑ देव - लो॒कम् । \newline
29. प्र ण॑यति नयति॒ प्र प्र ण॑य॒ त्यदी᳚क्षि॒तो ऽदी᳚क्षितो नयति॒ प्र प्र ण॑य॒ त्यदी᳚क्षितः । \newline
30. न॒य॒ त्यदी᳚क्षि॒तो ऽदी᳚क्षितो नयति नय॒ त्यदी᳚क्षित॒ एक॒ यैक॒या ऽदी᳚क्षितो नयति नय॒ त्यदी᳚क्षित॒ एक॑या । \newline
31. अदी᳚क्षित॒ एक॒ यैक॒या ऽदी᳚क्षि॒तो ऽदी᳚क्षित॒ एक॒या ऽऽहु॒त्या ऽऽहु॒त्यै क॒या ऽदी᳚क्षि॒तो ऽदी᳚क्षित॒ एक॒या ऽऽहु॑त्या । \newline
32. एक॒या ऽऽहु॒त्या ऽऽहु॒त्यैक॒ यैक॒या ऽऽहु॒त्ये तीत्याहु॒ त्यैक॒ यैक॒या ऽऽहु॒त्येति॑ । \newline
33. आहु॒त्ये तीत्या हु॒त्या ऽऽहु॒त्ये त्या॑हु राहु॒ रित्याहु॒त्या ऽऽहु॒त्ये त्या॑हुः । \newline
34. आहु॒त्येत्या - हु॒त्या॒ । \newline
35. इत्या॑हु राहु॒ रितीत्या॑हुः स्रु॒वेण॑ स्रु॒वेणा॑हु॒ रितीत्या॑हुः स्रु॒वेण॑ । \newline
36. आ॒हुः॒ स्रु॒वेण॑ स्रु॒वेणा॑हु राहुः स्रु॒वेण॒ चत॑स्र॒ श्चत॑स्रः स्रु॒वेणा॑हु राहुः स्रु॒वेण॒ चत॑स्रः । \newline
37. स्रु॒वेण॒ चत॑स्र॒ श्चत॑स्रः स्रु॒वेण॑ स्रु॒वेण॒ चत॑स्रो जुहोति जुहोति॒ चत॑स्रः स्रु॒वेण॑ स्रु॒वेण॒ चत॑स्रो जुहोति । \newline
38. चत॑स्रो जुहोति जुहोति॒ चत॑स्र॒ श्चत॑स्रो जुहोति दीक्षित॒त्वाय॑ दीक्षित॒त्वाय॑ जुहोति॒ चत॑स्र॒ श्चत॑स्रो जुहोति दीक्षित॒त्वाय॑ । \newline
39. जु॒हो॒ति॒ दी॒क्षि॒त॒त्वाय॑ दीक्षित॒त्वाय॑ जुहोति जुहोति दीक्षित॒त्वाय॑ स्रु॒चा स्रु॒चा दी᳚क्षित॒त्वाय॑ जुहोति जुहोति दीक्षित॒त्वाय॑ स्रु॒चा । \newline
40. दी॒क्षि॒त॒त्वाय॑ स्रु॒चा स्रु॒चा दी᳚क्षित॒त्वाय॑ दीक्षित॒त्वाय॑ स्रु॒चा प॑ञ्च॒मीम् प॑ञ्च॒मीꣳ स्रु॒चा दी᳚क्षित॒त्वाय॑ दीक्षित॒त्वाय॑ स्रु॒चा प॑ञ्च॒मीम् । \newline
41. दी॒क्षि॒त॒त्वायेति॑ दीक्षित - त्वाय॑ । \newline
42. स्रु॒चा प॑ञ्च॒मीम् प॑ञ्च॒मीꣳ स्रु॒चा स्रु॒चा प॑ञ्च॒मीम् पञ्चा᳚क्षरा॒ पञ्चा᳚क्षरा पञ्च॒मीꣳ स्रु॒चा स्रु॒चा प॑ञ्च॒मीम् पञ्चा᳚क्षरा । \newline
43. प॒ञ्च॒मीम् पञ्चा᳚क्षरा॒ पञ्चा᳚क्षरा पञ्च॒मीम् प॑ञ्च॒मीम् पञ्चा᳚क्षरा प॒ङ्क्तिः प॒ङ्क्तिः पञ्चा᳚क्षरा पञ्च॒मीम् प॑ञ्च॒मीम् पञ्चा᳚क्षरा प॒ङ्क्तिः । \newline
44. पञ्चा᳚क्षरा प॒ङ्क्तिः प॒ङ्क्तिः पञ्चा᳚क्षरा॒ पञ्चा᳚क्षरा प॒ङ्क्तिः पाङ्क्तः॒ पाङ्क्तः॑ प॒ङ्क्तिः पञ्चा᳚क्षरा॒ पञ्चा᳚क्षरा प॒ङ्क्तिः पाङ्क्तः॑ । \newline
45. पञ्चा᳚क्ष॒रेति॒ पञ्च॑ - अ॒क्ष॒रा॒ । \newline
46. प॒ङ्क्तिः पाङ्क्तः॒ पाङ्क्तः॑ प॒ङ्क्तिः प॒ङ्क्तिः पाङ्क्तो॑ य॒ज्ञो य॒ज्ञ्ः पाङ्क्तः॑ प॒ङ्क्तिः प॒ङ्क्तिः पाङ्क्तो॑ य॒ज्ञ्ः । \newline
47. पाङ्क्तो॑ य॒ज्ञो य॒ज्ञ्ः पाङ्क्तः॒ पाङ्क्तो॑ य॒ज्ञो य॒ज्ञ्ं ॅय॒ज्ञ्ं ॅय॒ज्ञ्ः पाङ्क्तः॒ पाङ्क्तो॑ य॒ज्ञो य॒ज्ञ्म् । \newline
48. य॒ज्ञो य॒ज्ञ्ं ॅय॒ज्ञ्ं ॅय॒ज्ञो य॒ज्ञो य॒ज्ञ् मे॒वैव य॒ज्ञ्ं ॅय॒ज्ञो य॒ज्ञो य॒ज्ञ् मे॒व । \newline
49. य॒ज्ञ् मे॒वैव य॒ज्ञ्ं ॅय॒ज्ञ् मे॒वावा वै॒व य॒ज्ञ्ं ॅय॒ज्ञ् मे॒वाव॑ । \newline
50. ए॒वावा वै॒वै वाव॑ रुन्धे रु॒न्धे ऽवै॒वै वाव॑ रुन्धे । \newline
51. अव॑ रुन्धे रु॒न्धे ऽवाव॑ रुन्ध॒ आकू᳚त्या॒ आकू᳚त्यै रु॒न्धे ऽवाव॑ रुन्ध॒ आकू᳚त्यै । \newline
52. रु॒न्ध॒ आकू᳚त्या॒ आकू᳚त्यै रुन्धे रुन्ध॒ आकू᳚त्यै प्र॒युजे᳚ प्र॒युज॒ आकू᳚त्यै रुन्धे रुन्ध॒ आकू᳚त्यै प्र॒युजे᳚ । \newline
53. आकू᳚त्यै प्र॒युजे᳚ प्र॒युज॒ आकू᳚त्या॒ आकू᳚त्यै प्र॒युजे॒ ऽग्नये॒ ऽग्नये᳚ प्र॒युज॒ आकू᳚त्या॒ आकू᳚त्यै प्र॒युजे॒ ऽग्नये᳚ । \newline
54. आकू᳚त्या॒ इत्या - कू॒त्यै॒ । \newline
55. प्र॒युजे॒ ऽग्नये॒ ऽग्नये᳚ प्र॒युजे᳚ प्र॒युजे॒ ऽग्नये॒ स्वाहा॒ स्वाहा॒ ऽग्नये᳚ प्र॒युजे᳚ प्र॒युजे॒ ऽग्नये॒ स्वाहा᳚ । \newline
56. प्र॒युज॒ इति॑ प्र - युजे᳚ । \newline
57. अ॒ग्नये॒ स्वाहा॒ स्वाहा॒ ऽग्नये॒ ऽग्नये॒ स्वाहेतीति॒ स्वाहा॒ ऽग्नये॒ ऽग्नये॒ स्वाहेति॑ । \newline
\pagebreak
\markright{ TS 6.1.2.2  \hfill https://www.vedavms.in \hfill}

\section{ TS 6.1.2.2 }

\textbf{TS 6.1.2.2 } \newline
\textbf{Samhita Paata} \newline

स्वाहेत्या॒हाऽऽ*कू᳚त्या॒ हि पुरु॑षो य॒ज्ञ्म॒भि प्र॑यु॒ङ्क्ते यजे॒येति॑ मे॒धायै॒ मन॑से॒ऽग्नये॒ स्वाहेत्या॑ह मे॒धया॒ हि मन॑सा॒ पुरु॑षो य॒ज्ञ्म॑भि॒गच्छ॑ति॒ सर॑स्वत्यै पू॒ष्णे᳚ऽग्नये॒ स्वाहेत्या॑ह॒ वाग्वै सर॑स्वती पृथि॒वी पू॒षा वा॒चैव पृ॑थि॒व्या य॒ज्ञ्ं प्रयु॑ङ्क्त॒ आपो॑ देवी-र्बृहती-र्विश्वशंभुव॒ इत्या॑ह॒ या वै वर्ष्या॒स्ता - [  ] \newline

\textbf{Pada Paata} \newline

स्वाहा᳚ । इति॑ । आ॒ह॒ । आकू॒त्येत्या - कू॒त्या॒ । हि । पुरु॑षः । य॒ज्ञ्म् । अ॒भीति॑ । प्र॒यु॒ङ्क्त इति॑ प्र-यु॒ङ्क्ते । यजे॑य । इति॑ । मे॒धायै᳚ । मन॑से । अ॒ग्नये᳚ । स्वाहा᳚ । इति॑ । आ॒ह॒ । मे॒धया᳚ । हि । मन॑सा । पुरु॑षः । य॒ज्ञ्म् । अ॒भि॒गच्छ॒तीत्य॑भि - गच्छ॑ति । सर॑स्वत्यै । पू॒ष्णे । अ॒ग्नये᳚ । स्वाहा᳚ । इति॑ । आ॒ह॒ । वाक् । वै । सर॑स्वती । पृ॒थि॒वी । पू॒षा । वा॒चा । ए॒व । पृ॒थि॒व्या । य॒ज्ञ्म् । प्रेति॑ । यु॒ङ्क्ते॒ । आपः॑ । दे॒वीः॒ । बृ॒ह॒तीः॒ । वि॒श्व॒श॒भुं॒व॒ इति॑ विश्व-श॒भुं॒वः॒ । इति॑ । आ॒ह॒ । याः । वै । वर्ष्याः᳚ । ताः ।  \newline


\textbf{Krama Paata} \newline

स्वाहेति॑ । इत्या॑ह । आ॒हाकू᳚त्या । आकू᳚त्या॒ हि । आकू॒त्येत्या - कू॒त्या॒ । हि पुरु॑षः । पुरु॑षो य॒ज्ञ्म् । य॒ज्ञ्म॒भि । अ॒भि प्र॑यु॒ङ्‍क्ते । प्र॒यु॒ङ्‍क्ते यजे॑य । प्र॒यु॒ङ्‍क्त इति॑ प्र - यु॒ङ्‍क्ते । यजे॒येति॑ । इति॑ मे॒धायै᳚ । मे॒धायै॒ मन॑से । मन॑से॒ऽग्नये᳚ । अ॒ग्नये॒ स्वाहा᳚ । स्वाहेति॑ । इत्या॑ह । आ॒ह॒ मे॒धया᳚ । मे॒धया॒ हि । हि मन॑सा । मन॑सा॒ पुरु॑षः । पुरु॑षो य॒ज्ञ्म् । य॒ज्ञ्म॑भि॒गच्छ॑ति । अ॒भि॒गच्छ॑ति॒ सर॑स्वत्यै । अ॒भि॒गच्छ॒तीत्य॑भि - गच्छ॑ति । सर॑स्वत्यै पू॒ष्णे । पू॒ष्णे᳚ऽग्नये᳚ । अ॒ग्नये॒ स्वाहा᳚ । स्वाहेति॑ । इत्या॑ह । आ॒ह॒ वाक् । वाग् वै । वै सर॑स्वती । सर॑स्वती पृथि॒वी । पृ॒थि॒वी पू॒षा । पू॒षा वा॒चा । वा॒चैव । ए॒व पृ॑थि॒व्या । पृ॒थि॒व्या य॒ज्ञ्म् । य॒ज्ञ्म् प्र । प्र यु॑ङ्‍क्ते । यु॒ङ्‍क्त॒ आपः॑ । आपो॑ देवीः । दे॒वी॒र् बृ॒ह॒तीः॒ । बृ॒ह॒ती॒र् वि॒श्व॒श॒म्भु॒वः॒ । वि॒श्व॒श॒म्भु॒व॒ इति॑ । वि॒श्व॒श॒म्भु॒व॒ इति॑ विश्व - श॒म्भु॒वः॒ । इत्या॑ह । आ॒ह॒ याः । या वै । वै वर्ष्याः᳚ । वर्ष्या॒स्ताः । ता आपः॑ \newline

\textbf{Jatai Paata} \newline

1. स्वाहेतीति॒ स्वाहा॒ स्वाहेति॑ । \newline
2. इत्या॑हा॒हे तीत्या॑ह । \newline
3. आ॒हा कू॒त्या ऽऽकू᳚त्या ऽऽहा॒हा कू᳚त्या । \newline
4. आकू᳚त्या॒ हि ह्याकू॒त्या ऽऽकू᳚त्या॒ हि । \newline
5. आकू॒त्येत्या - कू॒त्या॒ । \newline
6. हि पुरु॑षः॒ पुरु॑षो॒ हि हि पुरु॑षः । \newline
7. पुरु॑षो य॒ज्ञ्ं ॅय॒ज्ञ्म् पुरु॑षः॒ पुरु॑षो य॒ज्ञ्म् । \newline
8. य॒ज्ञ् म॒भ्य॑भि य॒ज्ञ्ं ॅय॒ज्ञ् म॒भि । \newline
9. अ॒भि प्र॑यु॒ङ्क्ते प्र॑यु॒ङ्क्ते᳚(1॒) ऽभ्य॑भि प्र॑यु॒ङ्क्ते । \newline
10. प्र॒यु॒ङ्क्ते यजे॑य॒ यजे॑य प्रयु॒ङ्क्ते प्र॑यु॒ङ्क्ते यजे॑य । \newline
11. प्र॒यु॒ङ्क्त इति॑ प्र - यु॒ङ्क्ते । \newline
12. यजे॒ये तीति॒ यजे॑य॒ यजे॒येति॑ । \newline
13. इति॑ मे॒धायै॑ मे॒धाया॒ इतीति॑ मे॒धायै᳚ । \newline
14. मे॒धायै॒ मन॑से॒ मन॑से मे॒धायै॑ मे॒धायै॒ मन॑से । \newline
15. मन॑से॒ ऽग्नये॒ ऽग्नये॒ मन॑से॒ मन॑से॒ ऽग्नये᳚ । \newline
16. अ॒ग्नये॒ स्वाहा॒ स्वाहा॒ ऽग्नये॒ ऽग्नये॒ स्वाहा᳚ । \newline
17. स्वाहे तीति॒ स्वाहा॒ स्वाहेति॑ । \newline
18. इत्या॑हा॒हे तीत्या॑ह । \newline
19. आ॒ह॒ मे॒धया॑ मे॒धया॑ ऽऽहाह मे॒धया᳚ । \newline
20. मे॒धया॒ हि हि मे॒धया॑ मे॒धया॒ हि । \newline
21. हि मन॑सा॒ मन॑सा॒ हि हि मन॑सा । \newline
22. मन॑सा॒ पुरु॑षः॒ पुरु॑षो॒ मन॑सा॒ मन॑सा॒ पुरु॑षः । \newline
23. पुरु॑षो य॒ज्ञ्ं ॅय॒ज्ञ्म् पुरु॑षः॒ पुरु॑षो य॒ज्ञ्म् । \newline
24. य॒ज्ञ् म॑भि॒गच्छ॑ त्यभि॒गच्छ॑ति य॒ज्ञ्ं ॅय॒ज्ञ् म॑भि॒गच्छ॑ति । \newline
25. अ॒भि॒गच्छ॑ति॒ सर॑स्वत्यै॒ सर॑स्वत्या अभि॒गच्छ॑ त्यभि॒गच्छ॑ति॒ सर॑स्वत्यै । \newline
26. अ॒भि॒गच्छ॒तीत्य॑भि - गच्छ॑ति । \newline
27. सर॑स्वत्यै पू॒ष्णे पू॒ष्णे सर॑स्वत्यै॒ सर॑स्वत्यै पू॒ष्णे । \newline
28. पू॒ष्णे᳚ ऽग्नये॒ ऽग्नये॑ पू॒ष्णे पू॒ष्णे᳚ ऽग्नये᳚ । \newline
29. अ॒ग्नये॒ स्वाहा॒ स्वाहा॒ ऽग्नये॒ ऽग्नये॒ स्वाहा᳚ । \newline
30. स्वाहे तीति॒ स्वाहा॒ स्वाहेति॑ । \newline
31. इत्या॑हा॒हे तीत्या॑ह । \newline
32. आ॒ह॒ वाग् वागा॑ हाह॒ वाक् । \newline
33. वाग् वै वै वाग् वाग् वै । \newline
34. वै सर॑स्वती॒ सर॑स्वती॒ वै वै सर॑स्वती । \newline
35. सर॑स्वती पृथि॒वी पृ॑थि॒वी सर॑स्वती॒ सर॑स्वती पृथि॒वी । \newline
36. पृ॒थि॒वी पू॒षा पू॒षा पृ॑थि॒वी पृ॑थि॒वी पू॒षा । \newline
37. पू॒षा वा॒चा वा॒चा पू॒षा पू॒षा वा॒चा । \newline
38. वा॒चै वैव वा॒चा वा॒चैव । \newline
39. ए॒व पृ॑थि॒व्या पृ॑थि॒व्यै वैव पृ॑थि॒व्या । \newline
40. पृ॒थि॒व्या य॒ज्ञ्ं ॅय॒ज्ञ्म् पृ॑थि॒व्या पृ॑थि॒व्या य॒ज्ञ्म् । \newline
41. य॒ज्ञ्म् प्र प्र य॒ज्ञ्ं ॅय॒ज्ञ्म् प्र । \newline
42. प्र यु॑ङ्क्ते युङ्क्ते॒ प्र प्र यु॑ङ्क्ते । \newline
43. यु॒ङ्क्त॒ आप॒ आपो॑ युङ्क्ते युङ्क्त॒ आपः॑ । \newline
44. आपो॑ देवीर् देवी॒ राप॒ आपो॑ देवीः । \newline
45. दे॒वी॒र् बृ॒ह॒ती॒र् बृ॒ह॒ती॒र् दे॒वी॒र् दे॒वी॒र् बृ॒ह॒तीः॒ । \newline
46. बृ॒ह॒ती॒र् वि॒श्व॒शं॒भु॒वो॒ वि॒श्व॒शं॒भु॒वो॒ बृ॒ह॒ती॒र् बृ॒ह॒ती॒र् वि॒श्व॒शं॒भु॒वः॒ । \newline
47. वि॒श्व॒शं॒भु॒व॒ इतीति॑ विश्वशंभुवो विश्वशंभुव॒ इति॑ । \newline
48. वि॒श्व॒शं॒भु॒व॒ इति॑ विश्व - शं॒भु॒वः॒ । \newline
49. इत्या॑हा॒हे तीत्या॑ह । \newline
50. आ॒ह॒ या या आ॑हाह॒ याः । \newline
51. या वै वै या या वै । \newline
52. वै वर्ष्या॒ वर्ष्या॒ वै वै वर्ष्याः᳚ । \newline
53. वर्ष्या॒ स्ता स्ता वर्ष्या॒ वर्ष्या॒ स्ताः । \newline
54. ता आप॒ आप॒ स्ता स्ता आपः॑ । \newline

\textbf{Ghana Paata } \newline

1. स्वाहेतीति॒ स्वाहा॒ स्वाहेत्या॑ हा॒हेति॒ स्वाहा॒ स्वाहे त्या॑ह । \newline
2. इत्या॑हा॒हे तीत्या॒हा कू॒त्या ऽऽकू᳚त्या॒ ऽऽहे तीत्या॒हा कू᳚त्या । \newline
3. आ॒हा कू॒त्या ऽऽकू᳚त्या ऽऽहा॒हा कू᳚त्या॒ हि ह्याकू᳚त्या ऽऽहा॒हा कू᳚त्या॒ हि । \newline
4. आकू᳚त्या॒ हि ह्याकू॒त्या ऽऽकू᳚त्या॒ हि पुरु॑षः॒ पुरु॑षो॒ ह्याकू॒त्या ऽऽकू᳚त्या॒ हि पुरु॑षः । \newline
5. आकू॒त्येत्या - कू॒त्या॒ । \newline
6. हि पुरु॑षः॒ पुरु॑षो॒ हि हि पुरु॑षो य॒ज्ञ्ं ॅय॒ज्ञ्म् पुरु॑षो॒ हि हि पुरु॑षो य॒ज्ञ्म् । \newline
7. पुरु॑षो य॒ज्ञ्ं ॅय॒ज्ञ्म् पुरु॑षः॒ पुरु॑षो य॒ज्ञ् म॒भ्य॑भि य॒ज्ञ्म् पुरु॑षः॒ पुरु॑षो य॒ज्ञ् म॒भि । \newline
8. य॒ज्ञ् म॒भ्य॑भि य॒ज्ञ्ं ॅय॒ज्ञ् म॒भि प्र॑यु॒ङ्क्ते प्र॑यु॒ङ्क्ते॑ ऽभि य॒ज्ञ्ं ॅय॒ज्ञ् म॒भि प्र॑यु॒ङ्क्ते । \newline
9. अ॒भि प्र॑यु॒ङ्क्ते प्र॑यु॒ङ्क्ते᳚(1॒) ऽभ्य॑भि प्र॑यु॒ङ्क्ते यजे॑य॒ यजे॑य प्रयु॒ङ्क्ते᳚(1॒) ऽभ्य॑भि प्र॑यु॒ङ्क्ते यजे॑य । \newline
10. प्र॒यु॒ङ्क्ते यजे॑य॒ यजे॑य प्रयु॒ङ्क्ते प्र॑यु॒ङ्क्ते यजे॒ये तीति॒ यजे॑य प्रयु॒ङ्क्ते प्र॑यु॒ङ्क्ते यजे॒येति॑ । \newline
11. प्र॒यु॒ङ्क्त इति॑ प्र - यु॒ङ्क्ते । \newline
12. यजे॒ये तीति॒ यजे॑य॒ यजे॒येति॑ मे॒धायै॑ मे॒धाया॒ इति॒ यजे॑य॒ यजे॒येति॑ मे॒धायै᳚ । \newline
13. इति॑ मे॒धायै॑ मे॒धाया॒ इतीति॑ मे॒धायै॒ मन॑से॒ मन॑से मे॒धाया॒ इतीति॑ मे॒धायै॒ मन॑से । \newline
14. मे॒धायै॒ मन॑से॒ मन॑से मे॒धायै॑ मे॒धायै॒ मन॑से॒ ऽग्नये॒ ऽग्नये॒ मन॑से मे॒धायै॑ मे॒धायै॒ मन॑से॒ ऽग्नये᳚ । \newline
15. मन॑से॒ ऽग्नये॒ ऽग्नये॒ मन॑से॒ मन॑से॒ ऽग्नये॒ स्वाहा॒ स्वाहा॒ ऽग्नये॒ मन॑से॒ मन॑से॒ ऽग्नये॒ स्वाहा᳚ । \newline
16. अ॒ग्नये॒ स्वाहा॒ स्वाहा॒ ऽग्नये॒ ऽग्नये॒ स्वाहेतीति॒ स्वाहा॒ ऽग्नये॒ ऽग्नये॒ स्वाहेति॑ । \newline
17. स्वाहेतीति॒ स्वाहा॒ स्वाहेत्या॑ हा॒हेति॒ स्वाहा॒ स्वाहे त्या॑ह । \newline
18. इत्या॑हा॒हे तीत्या॑ह मे॒धया॑ मे॒धया॒ ऽऽहे तीत्या॑ह मे॒धया᳚ । \newline
19. आ॒ह॒ मे॒धया॑ मे॒धया॑ ऽऽहाह मे॒धया॒ हि हि मे॒धया॑ ऽऽहाह मे॒धया॒ हि । \newline
20. मे॒धया॒ हि हि मे॒धया॑ मे॒धया॒ हि मन॑सा॒ मन॑सा॒ हि मे॒धया॑ मे॒धया॒ हि मन॑सा । \newline
21. हि मन॑सा॒ मन॑सा॒ हि हि मन॑सा॒ पुरु॑षः॒ पुरु॑षो॒ मन॑सा॒ हि हि मन॑सा॒ पुरु॑षः । \newline
22. मन॑सा॒ पुरु॑षः॒ पुरु॑षो॒ मन॑सा॒ मन॑सा॒ पुरु॑षो य॒ज्ञ्ं ॅय॒ज्ञ्म् पुरु॑षो॒ मन॑सा॒ मन॑सा॒ पुरु॑षो य॒ज्ञ्म् । \newline
23. पुरु॑षो य॒ज्ञ्ं ॅय॒ज्ञ्म् पुरु॑षः॒ पुरु॑षो य॒ज्ञ् म॑भि॒गच्छ॑ त्यभि॒गच्छ॑ति य॒ज्ञ्म् पुरु॑षः॒ पुरु॑षो य॒ज्ञ् म॑भि॒गच्छ॑ति । \newline
24. य॒ज्ञ् म॑भि॒गच्छ॑ त्यभि॒गच्छ॑ति य॒ज्ञ्ं ॅय॒ज्ञ् म॑भि॒गच्छ॑ति॒ सर॑स्वत्यै॒ सर॑स्वत्या अभि॒गच्छ॑ति य॒ज्ञ्ं ॅय॒ज्ञ् म॑भि॒गच्छ॑ति॒ सर॑स्वत्यै । \newline
25. अ॒भि॒गच्छ॑ति॒ सर॑स्वत्यै॒ सर॑स्वत्या अभि॒गच्छ॑ त्यभि॒गच्छ॑ति॒ सर॑स्वत्यै पू॒ष्णे पू॒ष्णे सर॑स्वत्या अभि॒गच्छ॑ त्यभि॒गच्छ॑ति॒ सर॑स्वत्यै पू॒ष्णे । \newline
26. अ॒भि॒गच्छ॒तीत्य॑भि - गच्छ॑ति । \newline
27. सर॑स्वत्यै पू॒ष्णे पू॒ष्णे सर॑स्वत्यै॒ सर॑स्वत्यै पू॒ष्णे᳚ ऽग्नये॒ ऽग्नये॑ पू॒ष्णे सर॑स्वत्यै॒ सर॑स्वत्यै पू॒ष्णे᳚ ऽग्नये᳚ । \newline
28. पू॒ष्णे᳚ ऽग्नये॒ ऽग्नये॑ पू॒ष्णे पू॒ष्णे᳚ ऽग्नये॒ स्वाहा॒ स्वाहा॒ ऽग्नये॑ पू॒ष्णे पू॒ष्णे᳚ ऽग्नये॒ स्वाहा᳚ । \newline
29. अ॒ग्नये॒ स्वाहा॒ स्वाहा॒ ऽग्नये॒ ऽग्नये॒ स्वाहे तीति॒ स्वाहा॒ ऽग्नये॒ ऽग्नये॒ स्वाहेति॑ । \newline
30. स्वाहे तीति॒ स्वाहा॒ स्वाहे त्या॑हा॒हेति॒ स्वाहा॒ स्वाहे त्या॑ह । \newline
31. इत्या॑हा॒हे तीत्या॑ह॒ वाग् वागा॒हे तीत्या॑ह॒ वाक् । \newline
32. आ॒ह॒ वाग् वागा॑ हाह॒ वाग् वै वै वागा॑ हाह॒ वाग् वै । \newline
33. वाग् वै वै वाग् वाग् वै सर॑स्वती॒ सर॑स्वती॒ वै वाग् वाग् वै सर॑स्वती । \newline
34. वै सर॑स्वती॒ सर॑स्वती॒ वै वै सर॑स्वती पृथि॒वी पृ॑थि॒वी सर॑स्वती॒ वै वै सर॑स्वती पृथि॒वी । \newline
35. सर॑स्वती पृथि॒वी पृ॑थि॒वी सर॑स्वती॒ सर॑स्वती पृथि॒वी पू॒षा पू॒षा पृ॑थि॒वी सर॑स्वती॒ सर॑स्वती पृथि॒वी पू॒षा । \newline
36. पृ॒थि॒वी पू॒षा पू॒षा पृ॑थि॒वी पृ॑थि॒वी पू॒षा वा॒चा वा॒चा पू॒षा पृ॑थि॒वी पृ॑थि॒वी पू॒षा वा॒चा । \newline
37. पू॒षा वा॒चा वा॒चा पू॒षा पू॒षा वा॒चै वैव वा॒चा पू॒षा पू॒षा वा॒चैव । \newline
38. वा॒चै वैव वा॒चा वा॒चैव पृ॑थि॒व्या पृ॑थि॒व्यैव वा॒चा वा॒चैव पृ॑थि॒व्या । \newline
39. ए॒व पृ॑थि॒व्या पृ॑थि॒व्यै वैव पृ॑थि॒व्या य॒ज्ञ्ं ॅय॒ज्ञ्म् पृ॑थि॒व्यै वैव पृ॑थि॒व्या य॒ज्ञ्म् । \newline
40. पृ॒थि॒व्या य॒ज्ञ्ं ॅय॒ज्ञ्म् पृ॑थि॒व्या पृ॑थि॒व्या य॒ज्ञ्म् प्र प्र य॒ज्ञ्म् पृ॑थि॒व्या पृ॑थि॒व्या य॒ज्ञ्म् प्र । \newline
41. य॒ज्ञ्म् प्र प्र य॒ज्ञ्ं ॅय॒ज्ञ्म् प्र यु॑ङ्क्ते युङ्क्ते॒ प्र य॒ज्ञ्ं ॅय॒ज्ञ्म् प्र यु॑ङ्क्ते । \newline
42. प्र यु॑ङ्क्ते युङ्क्ते॒ प्र प्र यु॑ङ्क्त॒ आप॒ आपो॑ युङ्क्ते॒ प्र प्र यु॑ङ्क्त॒ आपः॑ । \newline
43. यु॒ङ्क्त॒ आप॒ आपो॑ युङ्क्ते युङ्क्त॒ आपो॑ देवीर् देवी॒ रापो॑ युङ्क्ते युङ्क्त॒ आपो॑ देवीः । \newline
44. आपो॑ देवीर् देवी॒ राप॒ आपो॑ देवीर् बृहतीर् बृहतीर् देवी॒ राप॒ आपो॑ देवीर् बृहतीः । \newline
45. दे॒वी॒र् बृ॒ह॒ती॒र् बृ॒ह॒ती॒र् दे॒वी॒र् दे॒वी॒र् बृ॒ह॒ती॒र् वि॒श्व॒शं॒भु॒वो॒ वि॒श्व॒शं॒भु॒वो॒ बृ॒ह॒ती॒र् दे॒वी॒र् दे॒वी॒र् बृ॒ह॒ती॒र् वि॒श्व॒शं॒भु॒वः॒ । \newline
46. बृ॒ह॒ती॒र् वि॒श्व॒शं॒भु॒वो॒ वि॒श्व॒शं॒भु॒वो॒ बृ॒ह॒ती॒र् बृ॒ह॒ती॒र् वि॒श्व॒शं॒भु॒व॒ इतीति॑ विश्वशंभुवो बृहतीर् बृहतीर् विश्वशंभुव॒ इति॑ । \newline
47. वि॒श्व॒शं॒भु॒व॒ इतीति॑ विश्वशंभुवो विश्वशंभुव॒ इत्या॑हा॒हेति॑ विश्वशंभुवो विश्वशंभुव॒ इत्या॑ह । \newline
48. वि॒श्व॒शं॒भु॒व॒ इति॑ विश्व - शं॒भु॒वः॒ । \newline
49. इत्या॑हा॒हे तीत्या॑ह॒ या या आ॒हे तीत्या॑ह॒ याः । \newline
50. आ॒ह॒ या या आ॑हाह॒ या वै वै या आ॑हाह॒ या वै । \newline
51. या वै वै या या वै वर्ष्या॒ वर्ष्या॒ वै या या वै वर्ष्याः᳚ । \newline
52. वै वर्ष्या॒ वर्ष्या॒ वै वै वर्ष्या॒ स्ता स्ता वर्ष्या॒ वै वै वर्ष्या॒ स्ताः । \newline
53. वर्ष्या॒ स्ता स्ता वर्ष्या॒ वर्ष्या॒ स्ता आप॒ आप॒ स्ता वर्ष्या॒ वर्ष्या॒ स्ता आपः॑ । \newline
54. ता आप॒ आप॒ स्ता स्ता आपो॑ दे॒वीर् दे॒वी राप॒ स्ता स्ता आपो॑ दे॒वीः । \newline
\pagebreak
\markright{ TS 6.1.2.3  \hfill https://www.vedavms.in \hfill}

\section{ TS 6.1.2.3 }

\textbf{TS 6.1.2.3 } \newline
\textbf{Samhita Paata} \newline

आपो॑ दे॒वी-र्बृ॑ह॒ती-र्वि॒श्वश॑भुंवो॒ यदे॒तद्-यजु॒र्न ब्रू॒याद्-दि॒व्या आपोऽशा᳚न्ता इ॒मं ॅलो॒कमा ग॑च्छेयु॒रापो॑ देवी-र्बृहती-र्विश्वशंभुव॒ इत्या॑हा॒स्मा ए॒वैना॑ लो॒काय॑ शमयति॒ तस्मा᳚च्छा॒न्ता इ॒मं ॅलो॒कमा ग॑च्छन्ति॒ द्यावा॑पृथि॒वी इत्या॑ह॒ द्यावा॑पृथि॒व्योर्.हि य॒ज्ञ् उ॒र्व॑न्तरि॑क्ष॒-मित्या॑हा॒न्तरि॑क्षे॒ हि य॒ज्ञो बृह॒स्पति॑र्नो ह॒विषा॑ वृधा॒त्वि - [  ] \newline

\textbf{Pada Paata} \newline

आपः॑ । दे॒वीः । बृ॒ह॒तीः । वि॒श्वश॑भुंव॒ इति॑ वि॒श्व - श॒भुं॒वः॒ । यत् । ए॒तत् । यजुः॑ । न । ब्रू॒यात् । दि॒व्याः । आपः॑ । अशा᳚न्ताः । इ॒मम् । लो॒कम् । एति॑ । ग॒च्छे॒युः॒ । आपः॑ । दे॒वीः॒ । बृ॒ह॒तीः॒ । वि॒श्व॒श॒भुं॒व॒ इति॑ विश्व-श॒भुं॒वः॒ । इति॑ । आ॒ह॒ । अ॒स्मै । ए॒व । ए॒नाः॒ । लो॒काय॑ । श॒म॒य॒ति॒ । तस्मा᳚त् । शा॒न्ताः । इ॒मम् । लो॒कम् । एति॑ । ग॒च्छ॒न्ति॒ । द्यावा॑पृथि॒वी इति॒ द्यावा᳚ -पृ॒थि॒वी । इति॑ । आ॒ह॒ । द्यावा॑पृथि॒व्योरिति॒ द्यावा᳚-पृ॒थि॒व्योः । हि । य॒ज्ञ्ः । उ॒रु । अ॒न्तरि॑क्षम् । इति॑ । आ॒ह॒ । अ॒न्तरि॑क्षे । हि । य॒ज्ञ्ः । बृह॒स्पतिः॑ । नः॒ । ह॒विषा᳚ । वृ॒धा॒तु॒ ।  \newline


\textbf{Krama Paata} \newline

आपो॑ दे॒वीः । दे॒वीर् बृ॑ह॒तीः । बृ॒ह॒तीर् वि॒श्वश॑म्भुवः । वि॒श्वश॑म्भुवो॒ यत् । वि॒श्वश॑म्भुव॒ इति॑ वि॒श्व - श॒म्भु॒वः॒ । यदे॒तत् । ए॒तद् यजुः॑ । यजु॒र् न । न ब्रू॒यात् । ब्रू॒याद् दि॒व्याः । दि॒व्या आपः॑ । आपोऽशा᳚न्ताः । अशा᳚न्ता इ॒मम् । इ॒मम् ॅलो॒कम् । लो॒कमा । आ ग॑च्छेयुः । ग॒च्छे॒यु॒रापः॑ । आपो॑ देवीः । दे॒वी॒र् बृ॒ह॒तीः॒ । बृ॒ह॒ती॒र् वि॒श्व॒श॒म्भु॒वः॒ । वि॒श्व॒श॒म्भु॒व॒ इति॑ । वि॒श्व॒श॒म्भु॒व॒ इति॑ विश्व - श॒म्भु॒वः॒ । इत्या॑ह । आ॒हा॒स्मै । अ॒स्मा ए॒व । ए॒वैनाः᳚ । ए॒ना॒ लो॒काय॑ । लो॒काय॑ शमयति । श॒म॒य॒ति॒ तस्मा᳚त् । तस्मा᳚च्छा॒न्ताः । शा॒न्ता इ॒मम् । इ॒मम् ॅलो॒कम् । लो॒कमा । आ ग॑च्छन्ति । ग॒च्छ॒न्ति॒ द्यावा॑पृथि॒वी । द्यावा॑पृथि॒वी इति॑ । द्यावा॑पृथि॒वी इति॒ द्यावा᳚ - पृ॒थि॒वी । इत्या॑ह । आ॒ह॒ द्यावा॑पृथि॒व्योः । द्यावा॑पृथि॒व्योर् हि । द्यावा॑पृथि॒व्योरिति॒ द्यावा᳚ - पृ॒थि॒व्योः । हि य॒ज्ञ्ः । य॒ज्ञ् उ॒रु । उ॒र्व॑न्तरि॑क्षम् । अ॒न्तरि॑क्ष॒मिति॑ । इत्या॑ह । आ॒हा॒न्तरि॑क्षे । अ॒न्तरि॑क्षे॒ हि । हि य॒ज्ञ्ः । य॒ज्ञो बृह॒स्पतिः॑ । बृह॒स्पति॑र् नः । नो॒ ह॒विषा᳚ । ह॒विषा॑ वृधातु । वृ॒धा॒त्विति॑ \newline

\textbf{Jatai Paata} \newline

1. आपो॑ दे॒वीर् दे॒वी राप॒ आपो॑ दे॒वीः । \newline
2. दे॒वीर् बृ॑ह॒तीर् बृ॑ह॒तीर् दे॒वीर् दे॒वीर् बृ॑ह॒तीः । \newline
3. बृ॒ह॒तीर् वि॒श्वशं॑भुवो वि॒श्वशं॑भुवो बृह॒तीर् बृ॑ह॒तीर् वि॒श्वशं॑भुवः । \newline
4. वि॒श्वशं॑भुवो॒ यद् यद् वि॒श्वशं॑भुवो वि॒श्वशं॑भुवो॒ यत् । \newline
5. वि॒श्वशं॑भुव॒ इति॑ वि॒श्व - शं॒भु॒वः॒ । \newline
6. यदे॒त दे॒तद् यद् यदे॒तत् । \newline
7. ए॒तद् यजु॒र् यजु॑ रे॒त दे॒तद् यजुः॑ । \newline
8. यजु॒र् न न यजु॒र् यजु॒र् न । \newline
9. न ब्रू॒याद् ब्रू॒यान् न न ब्रू॒यात् । \newline
10. ब्रू॒याद् दि॒व्या दि॒व्या ब्रू॒याद् ब्रू॒याद् दि॒व्याः । \newline
11. दि॒व्या आप॒ आपो॑ दि॒व्या दि॒व्या आपः॑ । \newline
12. आपो ऽशा᳚न्ता॒ अशा᳚न्ता॒ आप॒ आपो ऽशा᳚न्ताः । \newline
13. अशा᳚न्ता इ॒म मि॒म मशा᳚न्ता॒ अशा᳚न्ता इ॒मम् । \newline
14. इ॒मम् ॅलो॒कम् ॅलो॒क मि॒म मि॒मम् ॅलो॒कम् । \newline
15. लो॒क मा लो॒कम् ॅलो॒क मा । \newline
16. आ ग॑च्छेयुर् गच्छेयु॒रा ग॑च्छेयुः । \newline
17. ग॒च्छे॒यु॒ राप॒ आपो॑ गच्छेयुर् गच्छेयु॒ रापः॑ । \newline
18. आपो॑ देवीर् देवी॒ राप॒ आपो॑ देवीः । \newline
19. दे॒वी॒र् बृ॒ह॒ती॒र् बृ॒ह॒ती॒र् दे॒वी॒र् दे॒वी॒र् बृ॒ह॒तीः॒ । \newline
20. बृ॒ह॒ती॒र् वि॒श्व॒शं॒भु॒वो॒ वि॒श्व॒शं॒भु॒वो॒ बृ॒ह॒ती॒र् बृ॒ह॒ती॒र् वि॒श्व॒शं॒भु॒वः॒ । \newline
21. वि॒श्व॒शं॒भु॒व॒ इतीति॑ विश्वशंभुवो विश्वशंभुव॒ इति॑ । \newline
22. वि॒श्व॒शं॒भु॒व॒ इति॑ विश्व - शं॒भु॒वः॒ । \newline
23. इत्या॑हा॒हे तीत्या॑ह । \newline
24. आ॒हा॒स्मा अ॒स्मा आ॑हा हा॒स्मै । \newline
25. अ॒स्मा ए॒वै वास्मा अ॒स्मा ए॒व । \newline
26. ए॒वैना॑ एना ए॒वै वैनाः᳚ । \newline
27. ए॒ना॒ लो॒काय॑ लो॒कायै॑ना एना लो॒काय॑ । \newline
28. लो॒काय॑ शमयति शमयति लो॒काय॑ लो॒काय॑ शमयति । \newline
29. श॒म॒य॒ति॒ तस्मा॒त् तस्मा᳚ च्छमयति शमयति॒ तस्मा᳚त् । \newline
30. तस्मा᳚ च्छा॒न्ताः शा॒न्ता स्तस्मा॒त् तस्मा᳚ च्छा॒न्ताः । \newline
31. शा॒न्ता इ॒म मि॒मꣳ शा॒न्ताः शा॒न्ता इ॒मम् । \newline
32. इ॒मम् ॅलो॒कम् ॅलो॒क मि॒म मि॒मम् ॅलो॒कम् । \newline
33. लो॒क मा लो॒कम् ॅलो॒क मा । \newline
34. आ ग॑च्छन्ति गच्छ॒न्त्या ग॑च्छन्ति । \newline
35. ग॒च्छ॒न्ति॒ द्यावा॑पृथि॒वी द्यावा॑पृथि॒वी ग॑च्छन्ति गच्छन्ति॒ द्यावा॑पृथि॒वी । \newline
36. द्यावा॑पृथि॒वी इतीति॒ द्यावा॑पृथि॒वी द्यावा॑पृथि॒वी इति॑ । \newline
37. द्यावा॑पृथि॒वी इति॒ द्यावा᳚ - पृ॒थि॒वी । \newline
38. इत्या॑हा॒हे तीत्या॑ह । \newline
39. आ॒ह॒ द्यावा॑पृथि॒व्योर् द्यावा॑पृथि॒व्यो रा॑हाह॒ द्यावा॑पृथि॒व्योः । \newline
40. द्यावा॑पृथि॒व्योर्. हि हि द्यावा॑पृथि॒व्योर् द्यावा॑पृथि॒व्योर्. हि । \newline
41. द्यावा॑पृथि॒व्योरिति॒ द्यावा᳚ - पृ॒थि॒व्योः । \newline
42. हि य॒ज्ञो य॒ज्ञो हि हि य॒ज्ञ्ः । \newline
43. य॒ज्ञ् उ॒रू॑रु य॒ज्ञो य॒ज्ञ् उ॒रु । \newline
44. उ॒र्व॑न्तरि॑क्ष म॒न्तरि॑क्ष मु॒रू᳚(1॒)र्व॑न्तरि॑क्षम् । \newline
45. अ॒न्तरि॑क्ष॒ मितीत्य॒ न्तरि॑क्ष म॒न्तरि॑क्ष॒ मिति॑ । \newline
46. इत्या॑हा॒हे तीत्या॑ह । \newline
47. आ॒हा॒न्तरि॑क्षे॒ ऽन्तरि॑क्ष आहाहा॒ न्तरि॑क्षे । \newline
48. अ॒न्तरि॑क्षे॒ हि ह्य॑न्तरि॑क्षे॒ ऽन्तरि॑क्षे॒ हि । \newline
49. हि य॒ज्ञो य॒ज्ञो हि हि य॒ज्ञ्ः । \newline
50. य॒ज्ञो बृह॒स्पति॒र् बृह॒स्पति॑र् य॒ज्ञो य॒ज्ञो बृह॒स्पतिः॑ । \newline
51. बृह॒स्पति॑र् नो नो॒ बृह॒स्पति॒र् बृह॒स्पति॑र् नः । \newline
52. नो॒ ह॒विषा॑ ह॒विषा॑ नो नो ह॒विषा᳚ । \newline
53. ह॒विषा॑ वृधातु वृधातु ह॒विषा॑ ह॒विषा॑ वृधातु । \newline
54. वृ॒धा॒ त्वितीति॑ वृधातु वृधा॒ त्विति॑ । \newline

\textbf{Ghana Paata } \newline

1. आपो॑ दे॒वीर् दे॒वी राप॒ आपो॑ दे॒वीर् बृ॑ह॒तीर् बृ॑ह॒तीर् दे॒वी राप॒ आपो॑ दे॒वीर् बृ॑ह॒तीः । \newline
2. दे॒वीर् बृ॑ह॒तीर् बृ॑ह॒तीर् दे॒वीर् दे॒वीर् बृ॑ह॒तीर् वि॒श्वशं॑भुवो वि॒श्वशं॑भुवो बृह॒तीर् दे॒वीर् दे॒वीर् बृ॑ह॒तीर् वि॒श्वशं॑भुवः । \newline
3. बृ॒ह॒तीर् वि॒श्वशं॑भुवो वि॒श्वशं॑भुवो बृह॒तीर् बृ॑ह॒तीर् वि॒श्वशं॑भुवो॒ यद् यद् वि॒श्वशं॑भुवो बृह॒तीर् बृ॑ह॒तीर् वि॒श्वशं॑भुवो॒ यत् । \newline
4. वि॒श्वशं॑भुवो॒ यद् यद् वि॒श्वशं॑भुवो वि॒श्वशं॑भुवो॒ यदे॒त दे॒तद् यद् वि॒श्वशं॑भुवो वि॒श्वशं॑भुवो॒ यदे॒तत् । \newline
5. वि॒श्वशं॑भुव॒ इति॑ वि॒श्व - शं॒भु॒वः॒ । \newline
6. यदे॒त दे॒तद् यद् यदे॒तद् यजु॒र् यजु॑ रे॒तद् यद् यदे॒तद् यजुः॑ । \newline
7. ए॒तद् यजु॒र् यजु॑ रे॒त दे॒तद् यजु॒र् न न यजु॑ रे॒त दे॒तद् यजु॒र् न । \newline
8. यजु॒र् न न यजु॒र् यजु॒र् न ब्रू॒याद् ब्रू॒यान् न यजु॒र् यजु॒र् न ब्रू॒यात् । \newline
9. न ब्रू॒याद् ब्रू॒यान् न न ब्रू॒याद् दि॒व्या दि॒व्या ब्रू॒यान् न न ब्रू॒याद् दि॒व्याः । \newline
10. ब्रू॒याद् दि॒व्या दि॒व्या ब्रू॒याद् ब्रू॒याद् दि॒व्या आप॒ आपो॑ दि॒व्या ब्रू॒याद् ब्रू॒याद् दि॒व्या आपः॑ । \newline
11. दि॒व्या आप॒ आपो॑ दि॒व्या दि॒व्या आपो ऽशा᳚न्ता॒ अशा᳚न्ता॒ आपो॑ दि॒व्या दि॒व्या आपो ऽशा᳚न्ताः । \newline
12. आपो ऽशा᳚न्ता॒ अशा᳚न्ता॒ आप॒ आपो ऽशा᳚न्ता इ॒म मि॒म मशा᳚न्ता॒ आप॒ आपो ऽशा᳚न्ता इ॒मम् । \newline
13. अशा᳚न्ता इ॒म मि॒म मशा᳚न्ता॒ अशा᳚न्ता इ॒मम् ॅलो॒कम् ॅलो॒क मि॒म मशा᳚न्ता॒ अशा᳚न्ता इ॒मम् ॅलो॒कम् । \newline
14. इ॒मम् ॅलो॒कम् ॅलो॒क मि॒म मि॒मम् ॅलो॒क मा लो॒क मि॒म मि॒मम् ॅलो॒क मा । \newline
15. लो॒क मा लो॒कम् ॅलो॒क मा ग॑च्छेयुर् गच्छेयु॒रा लो॒कम् ॅलो॒क मा ग॑च्छेयुः । \newline
16. आ ग॑च्छेयुर् गच्छेयु॒रा ग॑च्छेयु॒ राप॒ आपो॑ गच्छेयु॒रा ग॑च्छेयु॒ रापः॑ । \newline
17. ग॒च्छे॒यु॒ राप॒ आपो॑ गच्छेयुर् गच्छेयु॒ रापो॑ देवीर् देवी॒ रापो॑ गच्छेयुर् गच्छेयु॒ रापो॑ देवीः । \newline
18. आपो॑ देवीर् देवी॒ राप॒ आपो॑ देवीर् बृहतीर् बृहतीर् देवी॒ राप॒ आपो॑ देवीर् बृहतीः । \newline
19. दे॒वी॒र् बृ॒ह॒ती॒र् बृ॒ह॒ती॒र् दे॒वी॒र् दे॒वी॒र् बृ॒ह॒ती॒र् वि॒श्व॒शं॒भु॒वो॒ वि॒श्व॒शं॒भु॒वो॒ बृ॒ह॒ती॒र् दे॒वी॒र् दे॒वी॒र् बृ॒ह॒ती॒र् वि॒श्व॒शं॒भु॒वः॒ । \newline
20. बृ॒ह॒ती॒र् वि॒श्व॒शं॒भु॒वो॒ वि॒श्व॒शं॒भु॒वो॒ बृ॒ह॒ती॒र् बृ॒ह॒ती॒र् वि॒श्व॒शं॒भु॒व॒ इतीति॑ विश्वशंभुवो बृहतीर् बृहतीर् विश्वशंभुव॒ इति॑ । \newline
21. वि॒श्व॒शं॒भु॒व॒ इतीति॑ विश्वशंभुवो विश्वशंभुव॒ इत्या॑हा॒ हेति॑ विश्वशंभुवो विश्वशंभुव॒ इत्या॑ह । \newline
22. वि॒श्व॒शं॒भु॒व॒ इति॑ विश्व - शं॒भु॒वः॒ । \newline
23. इत्या॑हा॒हे तीत्या॑ हा॒स्मा अ॒स्मा आ॒हे तीत्या॑ हा॒स्मै । \newline
24. आ॒हा॒स्मा अ॒स्मा आ॑हा हा॒स्मा ए॒वै वास्मा आ॑हा हा॒स्मा ए॒व । \newline
25. अ॒स्मा ए॒वै वास्मा अ॒स्मा ए॒वैना॑ एना ए॒वास्मा अ॒स्मा ए॒वैनाः᳚ । \newline
26. ए॒वैना॑ एना ए॒वै वैना॑ लो॒काय॑ लो॒कायै॑ना ए॒वै वैना॑ लो॒काय॑ । \newline
27. ए॒ना॒ लो॒काय॑ लो॒कायै॑ना एना लो॒काय॑ शमयति शमयति लो॒कायै॑ना एना लो॒काय॑ शमयति । \newline
28. लो॒काय॑ शमयति शमयति लो॒काय॑ लो॒काय॑ शमयति॒ तस्मा॒त् तस्मा᳚च् छमयति लो॒काय॑ लो॒काय॑ शमयति॒ तस्मा᳚त् । \newline
29. श॒म॒य॒ति॒ तस्मा॒त् तस्मा᳚च् छमयति शमयति॒ तस्मा᳚ च्छा॒न्ताः शा॒न्ता स्तस्मा᳚च् छमयति शमयति॒ तस्मा᳚च्छा॒न्ताः । \newline
30. तस्मा᳚च् छा॒न्ताः शा॒न्ता स्तस्मा॒त् तस्मा᳚च् छा॒न्ता इ॒म मि॒मꣳ शा॒न्ता स्तस्मा॒त् तस्मा᳚च् छा॒न्ता इ॒मम् । \newline
31. शा॒न्ता इ॒म मि॒मꣳ शा॒न्ताः शा॒न्ता इ॒मम् ॅलो॒कम् ॅलो॒क मि॒मꣳ शा॒न्ताः शा॒न्ता इ॒मम् ॅलो॒कम् । \newline
32. इ॒मम् ॅलो॒कम् ॅलो॒क मि॒म मि॒मम् ॅलो॒क मा लो॒क मि॒म मि॒मम् ॅलो॒क मा । \newline
33. लो॒क मा लो॒कम् ॅलो॒क मा ग॑च्छन्ति गच्छ॒न्त्या लो॒कम् ॅलो॒क मा ग॑च्छन्ति । \newline
34. आ ग॑च्छन्ति गच्छ॒न्त्या ग॑च्छन्ति॒ द्यावा॑पृथि॒वी द्यावा॑पृथि॒वी ग॑च्छ॒न्त्या ग॑च्छन्ति॒ द्यावा॑पृथि॒वी । \newline
35. ग॒च्छ॒न्ति॒ द्यावा॑पृथि॒वी द्यावा॑पृथि॒वी ग॑च्छन्ति गच्छन्ति॒ द्यावा॑पृथि॒वी इतीति॒ द्यावा॑पृथि॒वी ग॑च्छन्ति गच्छन्ति॒ द्यावा॑पृथि॒वी इति॑ । \newline
36. द्यावा॑पृथि॒वी इतीति॒ द्यावा॑पृथि॒वी द्यावा॑पृथि॒वी इत्या॑हा॒हेति॒ द्यावा॑पृथि॒वी द्यावा॑पृथि॒वी इत्या॑ह । \newline
37. द्यावा॑पृथि॒वी इति॒ द्यावा᳚ - पृ॒थि॒वी । \newline
38. इत्या॑हा॒हे तीत्या॑ह॒ द्यावा॑पृथि॒व्योर् द्यावा॑पृथि॒व्यो रा॒हे तीत्या॑ह॒ द्यावा॑पृथि॒व्योः । \newline
39. आ॒ह॒ द्यावा॑पृथि॒व्योर् द्यावा॑पृथि॒व्यो रा॑हाह॒ द्यावा॑पृथि॒व्योर्. हि हि द्यावा॑पृथि॒व्यो रा॑हाह॒ द्यावा॑पृथि॒व्योर्. हि । \newline
40. द्यावा॑पृथि॒व्योर्. हि हि द्यावा॑पृथि॒व्योर् द्यावा॑पृथि॒व्योर्. हि य॒ज्ञो य॒ज्ञो हि द्यावा॑पृथि॒व्योर् द्यावा॑पृथि॒व्योर्. हि य॒ज्ञ्ः । \newline
41. द्यावा॑पृथि॒व्योरिति॒ द्यावा᳚ - पृ॒थि॒व्योः । \newline
42. हि य॒ज्ञो य॒ज्ञो हि हि य॒ज्ञ् उ॒रू॑रु य॒ज्ञो हि हि य॒ज्ञ् उ॒रु । \newline
43. य॒ज्ञ् उ॒रू॑रु य॒ज्ञो य॒ज्ञ् उ॒र्व॑न्तरि॑क्ष म॒न्तरि॑क्ष मु॒रु य॒ज्ञो य॒ज्ञ् उ॒र्व॑न्तरि॑क्षम् । \newline
44. उ॒र्व॑न्तरि॑क्ष म॒न्तरि॑क्ष मु॒रू᳚(1॒)र्व॑न्तरि॑क्ष॒ मिती त्य॒न्तरि॑क्ष मु॒रू᳚(1॒)र्व॑न्तरि॑क्ष॒ मिति॑ । \newline
45. अ॒न्तरि॑क्ष॒ मिती त्य॒न्तरि॑क्ष म॒न्तरि॑क्ष॒ मित्या॑हा॒हे त्य॒न्तरि॑क्ष म॒न्तरि॑क्ष॒ मित्या॑ह । \newline
46. इत्या॑हा॒हे तीत्या॑हा॒ न्तरि॑क्षे॒ ऽन्तरि॑क्ष आ॒हे तीत्या॑हा॒ न्तरि॑क्षे । \newline
47. आ॒हा॒ न्तरि॑क्षे॒ ऽन्तरि॑क्ष आहाहा॒ न्तरि॑क्षे॒ हि ह्य॑न्तरि॑क्ष आहाहा॒ न्तरि॑क्षे॒ हि । \newline
48. अ॒न्तरि॑क्षे॒ हि ह्य॑न्तरि॑क्षे॒ ऽन्तरि॑क्षे॒ हि य॒ज्ञो य॒ज्ञो ह्य॑न्तरि॑क्षे॒ ऽन्तरि॑क्षे॒ हि य॒ज्ञ्ः । \newline
49. हि य॒ज्ञो य॒ज्ञो हि हि य॒ज्ञो बृह॒स्पति॒र् बृह॒स्पति॑र् य॒ज्ञो हि हि य॒ज्ञो बृह॒स्पतिः॑ । \newline
50. य॒ज्ञो बृह॒स्पति॒र् बृह॒स्पति॑र् य॒ज्ञो य॒ज्ञो बृह॒स्पति॑र् नो नो॒ बृह॒स्पति॑र् य॒ज्ञो य॒ज्ञो बृह॒स्पति॑र् नः । \newline
51. बृह॒स्पति॑र् नो नो॒ बृह॒स्पति॒र् बृह॒स्पति॑र् नो ह॒विषा॑ ह॒विषा॑ नो॒ बृह॒स्पति॒र् बृह॒स्पति॑र् नो ह॒विषा᳚ । \newline
52. नो॒ ह॒विषा॑ ह॒विषा॑ नो नो ह॒विषा॑ वृधातु वृधातु ह॒विषा॑ नो नो ह॒विषा॑ वृधातु । \newline
53. ह॒विषा॑ वृधातु वृधातु ह॒विषा॑ ह॒विषा॑ वृधा॒ त्वितीति॑ वृधातु ह॒विषा॑ ह॒विषा॑ वृधा॒ त्विति॑ । \newline
54. वृ॒धा॒ त्वितीति॑ वृधातु वृधा॒ त्वित्या॑ हा॒हेति॑ वृधातु वृधा॒ त्वित्या॑ह । \newline
\pagebreak
\markright{ TS 6.1.2.4  \hfill https://www.vedavms.in \hfill}

\section{ TS 6.1.2.4 }

\textbf{TS 6.1.2.4 } \newline
\textbf{Samhita Paata} \newline

-त्या॑ह॒ ब्रह्म॒ वै दे॒वानां॒ बृह॒स्पति॒-र्ब्रह्म॑णै॒वास्मै॑ य॒ज्ञ्मव॑ रुन्धे॒ यद्- ब्रू॒याद्-वि॑धे॒रिति॑ यज्ञ्स्था॒णुमृ॑च्छेद्-वृधा॒त्वित्या॑ह यज्ञ्स्था॒णुमे॒व परि॑ वृणक्ति प्र॒जाप॑ति-र्य॒ज्ञ्म॑सृजत॒ सो᳚ऽस्माथ् सृ॒ष्टः परा॑ङै॒थ् स प्र यजु॒रव्ली॑ना॒त् प्र साम॒ तमृगुद॑यच्छ॒द्-यदृगु॒दय॑च्छ॒त् तदौ᳚द्-ग्रह॒णस्यौ᳚द्-ग्रहण॒त्व मृ॒चा - [  ] \newline

\textbf{Pada Paata} \newline

इति॑ । आ॒ह॒ । ब्रह्म॑ । वै । दे॒वाना᳚म् । बृह॒स्पतिः॑ । ब्रह्म॑णा । ए॒व । अ॒स्मै॒ । य॒ज्ञ्म् । अवेति॑ । रु॒न्धे॒ । यत् । ब्रू॒यात् । वि॒धेः॒ । इति॑ । य॒ज्ञ्॒स्था॒णुमिति॑ यज्ञ् - स्था॒णुम् । ऋ॒च्छे॒त् । वृ॒धा॒तु॒ । इति॑ । आ॒ह॒ । य॒ज्ञ्॒स्था॒णुमिति॑ यज्ञ् - स्था॒णुम् । ए॒व । परीति॑ । वृ॒ण॒क्ति॒ । प्र॒जाप॑ति॒रिति॑ प्र॒जा - प॒तिः॒ । य॒ज्ञ्म् । अ॒सृ॒ज॒त॒ । सः । अ॒स्मा॒त् । सृ॒ष्टः । पराङ्॑ । ऐ॒त् । सः । प्रेति॑ । यजुः॑ । अव्ली॑नात् । प्रेति॑ । साम॑ । तम् । ऋक् । उदिति॑ । अ॒य॒च्छ॒त् । यत् । ऋक् । उ॒दय॑च्छ॒दित्यु॑त् - अय॑च्छत् । तत् । औ॒द्ग्र॒ह॒णस्येत्यौ᳚त्-ग्र॒ह॒णस्य॑ । औ॒द्ग्र॒ह॒ण॒त्वमित्यौ᳚द्ग्रहण - त्वम् । ऋ॒चा ।  \newline


\textbf{Krama Paata} \newline

इत्या॑ह । आ॒ह॒ ब्रह्म॑ । ब्रह्म॒ वै । वै दे॒वाना᳚म् । दे॒वाना॒म् बृह॒स्पतिः॑ । बृह॒स्पति॒र् ब्रह्म॑णा । ब्रह्म॑णै॒व । ए॒वास्मै᳚ । अ॒स्मै॒ य॒ज्ञ्म् । य॒ज्ञ्मव॑ । अव॑ रुन्धे । रु॒न्धे॒ यत् । यद् ब्रू॒यात् । ब्रू॒याद् वि॑धेः । वि॒धे॒रिति॑ । इति॑ यज्ञ्स्था॒णुम् । य॒ज्ञ्॒स्था॒णुमृ॑च्छेत् । य॒ज्ञ्॒स्था॒णुमिति॑ यज्ञ् - स्था॒णुम् । ऋ॒च्छे॒द् वृ॒धा॒तु॒ । वृ॒धा॒त्विति॑ । इत्या॑ह । आ॒ह॒ य॒ज्ञ्॒स्था॒णुम् । य॒ज्ञ्॒स्था॒णुमे॒व । य॒ज्ञ्॒स्था॒णुमिति॑ यज्ञ् - स्था॒णुम् । ए॒व परि॑ । परि॑ वृणक्ति । वृ॒ण॒क्ति॒ प्र॒जाप॑तिः । प्र॒जाप॑तिर् य॒ज्ञ्म् । प्र॒जाप॑ति॒रिति॑ प्र॒जा - प॒तिः॒ । य॒ज्ञ्म॑सृजत । अ॒सृ॒ज॒त॒ सः । सो᳚ऽस्मात् । अ॒स्मा॒थ् सृ॒ष्टः । सृ॒ष्टः पराङ्॑ । परा॑ङैत् । ऐ॒थ् सः । स प्र । प्र यजुः॑ । यजु॒रव्ली॑नात् । अव्ली॑ना॒त् प्र । प्र साम॑ । साम॒ तम् । तमृक् । ऋगुत् । उद॑यच्छत् । अ॒य॒च्छ॒द् यत् । यदृक् । ऋगु॒दय॑च्छत् । उ॒दय॑च्छ॒त् तत् । उ॒दय॑च्छ॒दित्यु॑त् - अय॑च्छत् । तदौ᳚द्ग्रह॒णस्य॑ । औ॒द्ग्र॒ह॒णस्यौ᳚द्,ग्रहण॒त्वम् । औ॒द्ग्र॒ह॒णस्येत्यौ᳚त् - ग्र॒ह॒णस्य॑ । औ॒द्ग्र॒ह॒ण॒त्वमृ॒चा । औ॒द्ग्र॒ह॒ण॒त्व,मित्यौ᳚द्ग्रहण - त्वम् । ऋ॒चा जु॑होति \newline

\textbf{Jatai Paata} \newline

1. इत्या॑हा॒हे तीत्या॑ह । \newline
2. आ॒ह॒ ब्रह्म॒ ब्रह्मा॑हाह॒ ब्रह्म॑ । \newline
3. ब्रह्म॒ वै वै ब्रह्म॒ ब्रह्म॒ वै । \newline
4. वै दे॒वाना᳚म् दे॒वानां॒ ॅवै वै दे॒वाना᳚म् । \newline
5. दे॒वाना॒म् बृह॒स्पति॒र् बृह॒स्पति॑र् दे॒वाना᳚म् दे॒वाना॒म् बृह॒स्पतिः॑ । \newline
6. बृह॒स्पति॒र् ब्रह्म॑णा॒ ब्रह्म॑णा॒ बृह॒स्पति॒र् बृह॒स्पति॒र् ब्रह्म॑णा । \newline
7. ब्रह्म॑णै॒ वैव ब्रह्म॑णा॒ ब्रह्म॑णै॒व । \newline
8. ए॒वास्मा॑ अस्मा ए॒वै वास्मै᳚ । \newline
9. अ॒स्मै॒ य॒ज्ञ्ं ॅय॒ज्ञ् म॑स्मा अस्मै य॒ज्ञ्म् । \newline
10. य॒ज्ञ् मवाव॑ य॒ज्ञ्ं ॅय॒ज्ञ् मव॑ । \newline
11. अव॑ रुन्धे रु॒न्धे ऽवाव॑ रुन्धे । \newline
12. रु॒न्धे॒ यद् यद् रु॑न्धे रुन्धे॒ यत् । \newline
13. यद् ब्रू॒याद् ब्रू॒याद् यद् यद् ब्रू॒यात् । \newline
14. ब्रू॒याद् वि॑धेर् विधेर् ब्रू॒याद् ब्रू॒याद् वि॑धेः । \newline
15. वि॒धे॒ रितीति॑ विधेर् विधे॒ रिति॑ । \newline
16. इति॑ यज्ञ्स्था॒णुं ॅय॑ज्ञ्स्था॒णु मितीति॑ यज्ञ्स्था॒णुम् । \newline
17. य॒ज्ञ्॒स्था॒णु मृ॑च्छे दृच्छेद् यज्ञ्स्था॒णुं ॅय॑ज्ञ्स्था॒णु मृ॑च्छेत् । \newline
18. य॒ज्ञ्॒स्था॒णुमिति॑ यज्ञ् - स्था॒णुम् । \newline
19. ऋ॒च्छे॒द् वृ॒धा॒तु॒ वृ॒धा॒ त्वृ॒च्छे॒ दृ॒च्छे॒द् वृ॒धा॒तु॒ । \newline
20. वृ॒धा॒ त्वितीति॑ वृधातु वृधा॒ त्विति॑ । \newline
21. इत्या॑हा॒हे तीत्या॑ह । \newline
22. आ॒ह॒ य॒ज्ञ्॒स्था॒णुं ॅय॑ज्ञ्स्था॒णु मा॑हाह यज्ञ्स्था॒णुम् । \newline
23. य॒ज्ञ्॒स्था॒णु मे॒वैव य॑ज्ञ्स्था॒णुं ॅय॑ज्ञ्स्था॒णु मे॒व । \newline
24. य॒ज्ञ्॒स्था॒णुमिति॑ यज्ञ् - स्था॒णुम् । \newline
25. ए॒व परि॒ पर्ये॒ वैव परि॑ । \newline
26. परि॑ वृणक्ति वृणक्ति॒ परि॒ परि॑ वृणक्ति । \newline
27. वृ॒ण॒क्ति॒ प्र॒जाप॑तिः प्र॒जाप॑तिर् वृणक्ति वृणक्ति प्र॒जाप॑तिः । \newline
28. प्र॒जाप॑तिर् य॒ज्ञ्ं ॅय॒ज्ञ्म् प्र॒जाप॑तिः प्र॒जाप॑तिर् य॒ज्ञ्म् । \newline
29. प्र॒जाप॑ति॒रिति॑ प्र॒जा - प॒तिः॒ । \newline
30. य॒ज्ञ् म॑सृजता सृजत य॒ज्ञ्ं ॅय॒ज्ञ् म॑सृजत । \newline
31. अ॒सृ॒ज॒त॒ स सो॑ ऽसृजता सृजत॒ सः । \newline
32. सो᳚ ऽस्मा दस्मा॒थ् स सो᳚ ऽस्मात् । \newline
33. अ॒स्मा॒थ् सृ॒ष्टः सृ॒ष्टो᳚ ऽस्मा दस्माथ् सृ॒ष्टः । \newline
34. सृ॒ष्टः परा॒ङ् परा᳚ङ् ख्सृ॒ष्टः सृ॒ष्टः पराङ्॑ । \newline
35. परा॑ङैदै॒त् परा॒ङ् परा॑ङैत् । \newline
36. ऐ॒थ् स स ऐ॑दै॒थ् सः । \newline
37. स प्र प्र स स प्र । \newline
38. प्र यजु॒र् यजुः॒ प्र प्र यजुः॑ । \newline
39. यजु॒ रव्ली॑ना॒ दव्ली॑ना॒द् यजु॒र् यजु॒ रव्ली॑नात् । \newline
40. अव्ली॑ना॒त् प्र प्राव्ली॑ना॒ दव्ली॑ना॒त् प्र । \newline
41. प्र साम॒ साम॒ प्र प्र साम॑ । \newline
42. साम॒ तम् तꣳ साम॒ साम॒ तम् । \newline
43. त मृगृक् तम् त मृक् । \newline
44. ऋगुदु दृगृगुत् । \newline
45. उद॑यच्छ दयच्छ॒ दुदु द॑यच्छत् । \newline
46. अ॒य॒च्छ॒द् यद् यद॑यच्छ दयच्छ॒द् यत् । \newline
47. यद् ऋगृग् यद् यदृक् । \newline
48. ऋगु॒दय॑च्छ दु॒दय॑च्छ॒ दृगृगु॒ दय॑च्छत् । \newline
49. उ॒दय॑च्छ॒त् तत् तदु॒दय॑च्छ दु॒दय॑च्छ॒त् तत् । \newline
50. उ॒दय॑च्छ॒दित्यु॑त् - अय॑च्छत् । \newline
51. तदौ᳚द्ग्रह॒णस्यौ᳚ द्ग्रह॒णस्य॒ तत् तदौ᳚द्ग्रह॒णस्य॑ । \newline
52. औ॒द्ग्र॒ह॒ण स्यौ᳚द्ग्रहण॒त्व मौ᳚द्ग्रहण॒त्व मौ᳚द्ग्रह॒ण स्यौ᳚द्ग्रह॒ण स्यौ᳚द्ग्रहण॒त्वम् । \newline
53. औ॒द्ग्र॒ह॒णस्येत्यौ᳚त् - ग्र॒ह॒णस्य॑ । \newline
54. औ॒द्ग्र॒ह॒ण॒त्व मृ॒च र्चौद्ग्र॑हण॒त्व मौ᳚द्ग्रहण॒त्व मृ॒चा । \newline
55. औ॒द्ग्र॒ह॒ण॒त्वमित्यौ᳚द्ग्रहण - त्वम् । \newline
56. ऋ॒चा जु॑होति जुहो त्यृ॒च र्‌चा जु॑होति । \newline

\textbf{Ghana Paata } \newline

1. इत्या॑हा॒हे तीत्या॑ह॒ ब्रह्म॒ ब्रह्मा॒हे तीत्या॑ह॒ ब्रह्म॑ । \newline
2. आ॒ह॒ ब्रह्म॒ ब्रह्मा॑हाह॒ ब्रह्म॒ वै वै ब्रह्मा॑हाह॒ ब्रह्म॒ वै । \newline
3. ब्रह्म॒ वै वै ब्रह्म॒ ब्रह्म॒ वै दे॒वाना᳚म् दे॒वानां॒ ॅवै ब्रह्म॒ ब्रह्म॒ वै दे॒वाना᳚म् । \newline
4. वै दे॒वाना᳚म् दे॒वानां॒ ॅवै वै दे॒वाना॒म् बृह॒स्पति॒र् बृह॒स्पति॑र् दे॒वानां॒ ॅवै वै दे॒वाना॒म् बृह॒स्पतिः॑ । \newline
5. दे॒वाना॒म् बृह॒स्पति॒र् बृह॒स्पति॑र् दे॒वाना᳚म् दे॒वाना॒म् बृह॒स्पति॒र् ब्रह्म॑णा॒ ब्रह्म॑णा॒ बृह॒स्पति॑र् दे॒वाना᳚म् दे॒वाना॒म् बृह॒स्पति॒र् ब्रह्म॑णा । \newline
6. बृह॒स्पति॒र् ब्रह्म॑णा॒ ब्रह्म॑णा॒ बृह॒स्पति॒र् बृह॒स्पति॒र् ब्रह्म॑ णै॒वैव ब्रह्म॑णा॒ बृह॒स्पति॒र् बृह॒स्पति॒र् ब्रह्म॑णै॒व । \newline
7. ब्रह्म॑णै॒वैव ब्रह्म॑णा॒ ब्रह्म॑ णै॒वास्मा॑ अस्मा ए॒व ब्रह्म॑णा॒ ब्रह्म॑ णै॒वास्मै᳚ । \newline
8. ए॒वास्मा॑ अस्मा ए॒वै वास्मै॑ य॒ज्ञ्ं ॅय॒ज्ञ् म॑स्मा ए॒वै वास्मै॑ य॒ज्ञ्म् । \newline
9. अ॒स्मै॒ य॒ज्ञ्ं ॅय॒ज्ञ् म॑स्मा अस्मै य॒ज्ञ् मवाव॑ य॒ज्ञ् म॑स्मा अस्मै य॒ज्ञ् मव॑ । \newline
10. य॒ज्ञ् मवाव॑ य॒ज्ञ्ं ॅय॒ज्ञ् मव॑ रुन्धे रु॒न्धे ऽव॑ य॒ज्ञ्ं ॅय॒ज्ञ् मव॑ रुन्धे । \newline
11. अव॑ रुन्धे रु॒न्धे ऽवाव॑ रुन्धे॒ यद् यद् रु॒न्धे ऽवाव॑ रुन्धे॒ यत् । \newline
12. रु॒न्धे॒ यद् यद् रु॑न्धे रुन्धे॒ यद् ब्रू॒याद् ब्रू॒याद् यद् रु॑न्धे रुन्धे॒ यद् ब्रू॒यात् । \newline
13. यद् ब्रू॒याद् ब्रू॒याद् यद् यद् ब्रू॒याद् वि॑धेर् विधेर् ब्रू॒याद् यद् यद् ब्रू॒याद् वि॑धेः । \newline
14. ब्रू॒याद् वि॑धेर् विधेर् ब्रू॒याद् ब्रू॒याद् वि॑धे॒रि तीति॑ विधेर् ब्रू॒याद् ब्रू॒याद् वि॑धे॒ रिति॑ । \newline
15. वि॒धे॒रि तीति॑ विधेर् विधे॒रिति॑ यज्ञ्स्था॒णुं ॅय॑ज्ञ्स्था॒णु मिति॑ विधेर् विधे॒रिति॑ यज्ञ्स्था॒णुम् । \newline
16. इति॑ यज्ञ्स्था॒णुं ॅय॑ज्ञ्स्था॒णु मितीति॑ यज्ञ्स्था॒णु मृ॑च्छे दृच्छेद् यज्ञ्स्था॒णु मितीति॑ यज्ञ्स्था॒णु मृ॑च्छेत् । \newline
17. य॒ज्ञ्॒स्था॒णु मृ॑च्छे दृच्छेद् यज्ञ्स्था॒णुं ॅय॑ज्ञ्स्था॒णु मृ॑च्छेद् वृधातु वृधा त्वृच्छेद् यज्ञ्स्था॒णुं ॅय॑ज्ञ्स्था॒णु मृ॑च्छेद् वृधातु । \newline
18. य॒ज्ञ्॒स्था॒णुमिति॑ यज्ञ् - स्था॒णुम् । \newline
19. ऋ॒च्छे॒द् वृ॒धा॒तु॒ वृ॒धा॒ त्वृ॒च्छे॒ दृ॒च्छे॒द् वृ॒धा॒त्वि तीति॑ वृधा त्वृच्छे दृच्छेद् वृधा॒ त्विति॑ । \newline
20. वृ॒धा॒ त्वितीति॑ वृधातु वृधा॒ त्वित्या॑ हा॒हेति॑ वृधातु वृधा॒ त्वित्या॑ह । \newline
21. इत्या॑हा॒हे तीत्या॑ह यज्ञ्स्था॒णुं ॅय॑ज्ञ्स्था॒णु मा॒हे तीत्या॑ह यज्ञ्स्था॒णुम् । \newline
22. आ॒ह॒ य॒ज्ञ्॒स्था॒णुं ॅय॑ज्ञ्स्था॒णु मा॑हाह यज्ञ्स्था॒णु मे॒वैव य॑ज्ञ्स्था॒णु मा॑हाह यज्ञ्स्था॒णु मे॒व । \newline
23. य॒ज्ञ्॒स्था॒णु मे॒वैव य॑ज्ञ्स्था॒णुं ॅय॑ज्ञ्स्था॒णु मे॒व परि॒ पर्ये॒व य॑ज्ञ्स्था॒णुं ॅय॑ज्ञ्स्था॒णु मे॒व परि॑ । \newline
24. य॒ज्ञ्॒स्था॒णुमिति॑ यज्ञ् - स्था॒णुम् । \newline
25. ए॒व परि॒ पर्ये॒ वैव परि॑ वृणक्ति वृणक्ति॒ पर्ये॒ वैव परि॑ वृणक्ति । \newline
26. परि॑ वृणक्ति वृणक्ति॒ परि॒ परि॑ वृणक्ति प्र॒जाप॑तिः प्र॒जाप॑तिर् वृणक्ति॒ परि॒ परि॑ वृणक्ति प्र॒जाप॑तिः । \newline
27. वृ॒ण॒क्ति॒ प्र॒जाप॑तिः प्र॒जाप॑तिर् वृणक्ति वृणक्ति प्र॒जाप॑तिर् य॒ज्ञ्ं ॅय॒ज्ञ्म् प्र॒जाप॑तिर् वृणक्ति वृणक्ति प्र॒जाप॑तिर् य॒ज्ञ्म् । \newline
28. प्र॒जाप॑तिर् य॒ज्ञ्ं ॅय॒ज्ञ्म् प्र॒जाप॑तिः प्र॒जाप॑तिर् य॒ज्ञ् म॑सृजता सृजत य॒ज्ञ्म् प्र॒जाप॑तिः प्र॒जाप॑तिर् य॒ज्ञ् म॑सृजत । \newline
29. प्र॒जाप॑ति॒रिति॑ प्र॒जा - प॒तिः॒ । \newline
30. य॒ज्ञ् म॑सृजता सृजत य॒ज्ञ्ं ॅय॒ज्ञ् म॑सृजत॒ स सो॑ ऽसृजत य॒ज्ञ्ं ॅय॒ज्ञ् म॑सृजत॒ सः । \newline
31. अ॒सृ॒ज॒त॒ स सो॑ ऽसृजता सृजत॒ सो᳚ ऽस्मा दस्मा॒थ् सो॑ ऽसृजता सृजत॒ सो᳚ ऽस्मात् । \newline
32. सो᳚ ऽस्मा दस्मा॒थ् स सो᳚ ऽस्माथ् सृ॒ष्टः सृ॒ष्टो᳚ ऽस्मा॒थ् स सो᳚ ऽस्माथ् सृ॒ष्टः । \newline
33. अ॒स्मा॒थ् सृ॒ष्टः सृ॒ष्टो᳚ ऽस्मा दस्माथ् सृ॒ष्टः परा॒ङ् परा᳚ङ् ख्सृ॒ष्टो᳚ ऽस्मा दस्माथ् सृ॒ष्टः पराङ्॑ । \newline
34. सृ॒ष्टः परा॒ङ् परा᳚ङ् ख्सृ॒ष्टः सृ॒ष्टः परा॑ङै दै॒त् परा᳚ङ् ख्सृ॒ष्टः सृ॒ष्टः पराङै॑त् । \newline
35. पराङै॑ दै॒त् परा॒ङ् पराङै॒थ् स स ऐ॒त् परा॒ङ् पराङै॒थ् सः । \newline
36. ऐ॒थ् स स ऐ॑दै॒थ् स प्र प्र स ऐ॑दै॒थ् स प्र । \newline
37. स प्र प्र स स प्र यजु॒र् यजुः॒ प्र स स प्र यजुः॑ । \newline
38. प्र यजु॒र् यजुः॒ प्र प्र यजु॒ रव्ली॑ना॒ दव्ली॑ना॒द् यजुः॒ प्र प्र यजु॒ रव्ली॑नात् । \newline
39. यजु॒ रव्ली॑ना॒ दव्ली॑ना॒द् यजु॒र् यजु॒ रव्ली॑ना॒त् प्र प्राव्ली॑ना॒द् यजु॒र् यजु॒ रव्ली॑ना॒त् प्र । \newline
40. अव्ली॑ना॒त् प्र प्राव्ली॑ना॒ दव्ली॑ना॒त् प्र साम॒ साम॒ प्राव्ली॑ना॒ दव्ली॑ना॒त् प्र साम॑ । \newline
41. प्र साम॒ साम॒ प्र प्र साम॒ तम् तꣳ साम॒ प्र प्र साम॒ तम् । \newline
42. साम॒ तम् तꣳ साम॒ साम॒ त मृगृक् तꣳ साम॒ साम॒ त मृक् । \newline
43. त मृगृक् तम् त मृगु दुदृक् तम् त मृगुत् । \newline
44. ऋगुदु दृग् ऋगु द॑यच्छ दयच्छ॒ दुदृग् ऋगु द॑यच्छत् । \newline
45. उद॑यच्छ दयच्छ॒ दुदु द॑यच्छ॒द् यद् यद॑यच् छ॒दुदु द॑यच्छ॒द् यत् । \newline
46. अ॒य॒च्छ॒द् यद् यद॑यच्छ दयच्छ॒द् यदृगृग् यद॑यच्छ दयच्छ॒द् यदृक् । \newline
47. यदृगृग् यद् यदृगु॒ दय॑च्छ दु॒दय॑च्छ॒ दृग् यद् यदृगु॒ दय॑च्छत् । \newline
48. ऋगु॒ दय॑च्छ दु॒दय॑च्छ॒ दृगृ गु॒दय॑च्छ॒त् तत् तदु॒दय॑च्छ॒ दृगृ गु॒दय॑च्छ॒त् तत् । \newline
49. उ॒दय॑च्छ॒त् तत् तदु॒दय॑च्छ दु॒दय॑च्छ॒त् तदौ᳚द्ग्रह॒ण स्यौ᳚द्ग्रह॒णस्य॒ तदु॒दय॑च्छ दु॒दय॑च्छ॒त् तदौ᳚द्ग्रह॒णस्य॑ । \newline
50. उ॒दय॑च्छ॒दित्यु॑त् - अय॑च्छत् । \newline
51. तदौ᳚द्ग्रह॒ण स्यौ᳚द्ग्रह॒णस्य॒ तत् तदौ᳚द्ग्रह॒ण स्यौ᳚द्ग्रहण॒त्व मौ᳚द्ग्रहण॒त्वमौ᳚द् ग्रह॒णस्य॒ तत् तदौ᳚द्ग्रह॒ण स्यौ᳚द्ग्रहण॒त्वम् । \newline
52. औ॒द्ग्र॒ह॒ण स्यौ᳚द्ग्रहण॒त्व मौ᳚द्ग्रहण॒त्व मौ᳚द्ग्रह॒ण स्यौ᳚द्ग्रह॒ण स्यौ᳚द्ग्रहण॒त्व मृ॒च र्‌चौद्ग्र॑हण॒त्व मौ᳚द्ग्रह॒ण स्यौ᳚द्ग्रह॒ण स्यौ᳚द्ग्रहण॒त्व मृ॒चा । \newline
53. औ॒द्ग्र॒ह॒णस्येत्यौ᳚त् - ग्र॒ह॒णस्य॑ । \newline
54. औ॒द्ग्र॒ह॒ण॒त्व मृ॒चर्चौद्ग्र॑हण॒त्व मौ᳚द्ग्रहण॒त्व मृ॒चा जु॑होति जुहो त्यृ॒चौद्ग्र॑हण॒त्व मौ᳚द्ग्रहण॒त्व मृ॒चा जु॑होति । \newline
55. औ॒द्ग्र॒ह॒ण॒त्वमित्यौ᳚द्ग्रहण - त्वम् । \newline
56. ऋ॒चा जु॑होति जुहो त्यृ॒च र्‌चा जु॑होति य॒ज्ञ्स्य॑ य॒ज्ञ्स्य॑ जुहो त्यृ॒च र्‌चा जु॑होति य॒ज्ञ्स्य॑ । \newline
\pagebreak
\markright{ TS 6.1.2.5  \hfill https://www.vedavms.in \hfill}

\section{ TS 6.1.2.5 }

\textbf{TS 6.1.2.5 } \newline
\textbf{Samhita Paata} \newline

जु॑होति य॒ज्ञ्स्योद्य॑त्या अनु॒ष्टुप्-छन्द॑सा॒-मुद॑यच्छ॒दित्या॑-हु॒स्तस्मा॑दनु॒ष्टुभा॑ जुहोति य॒ज्ञ्स्योद्य॑त्यै॒ द्वाद॑श वाथ्सब॒न्धान्युद॑यच्छ॒-न्नित्या॑हु॒-स्तस्मा᳚द्- द्वाद॒शभि॑-र्वाथ्सबन्ध॒विदो॑ दीक्षयन्ति॒ सा वा ए॒षर्ग॑नु॒ष्टुग्-वाग॑नु॒ष्टुग्-यदे॒तय॒र्चा दी॒क्षय॑ति वा॒चैवैनꣳ॒॒ सर्व॑या दीक्षयति॒ विश्वे॑ दे॒वस्य॑ ने॒तुरित्या॑ह सावि॒त्र्ये॑तेन॒ मर्तो॑ वृणीत स॒ख्य - [  ] \newline

\textbf{Pada Paata} \newline

जु॒हो॒ति॒ । य॒ज्ञ्स्य॑ । उद्य॑त्या॒ इत्युत् - य॒त्यै॒ । अ॒नु॒ष्टुबित्य॑नु - स्तुप् । छन्द॑साम् । उदिति॑ । अ॒य॒च्छ॒त् । इति॑ । आ॒हुः॒ । तस्मा᳚त् । अ॒नु॒ष्टुभेत्य॑नु - स्तुभा᳚ । जु॒हो॒ति॒ । य॒ज्ञ्स्य॑ । उद्य॑त्या॒ इत्युत् - य॒त्यै॒ । द्वाद॑श । वा॒थ्स॒ब॒न्धानीति॑ वाथ्स - ब॒न्धानि॑ । उदिति॑ । अ॒य॒च्छ॒न्न् । इति॑ । आ॒हुः॒ । तस्मा᳚त् । द्वा॒द॒शभि॒रिति॑ द्वाद॒श-भिः॒ । आ॒थ्स॒ब॒न्ध॒विद॒ इति॑ वाथ्सबन्ध - विदः॑ । दी॒क्ष॒य॒न्ति॒ । सा । वै । ए॒षा । ऋक् । अ॒नु॒ष्टुगित्य॑नु - स्तुक् । वाक् । अ॒नु॒ष्टुगित्य॑नु - स्तुक् । यत् । ए॒तया᳚ । ऋ॒चा । दी॒क्षय॑ति । वा॒चा । ए॒व । ए॒न॒म् । सर्व॑या । दी॒क्ष॒य॒ति॒ । विश्वे᳚ । दे॒वस्य॑ । ने॒तुः । इति॑ । आ॒ह॒ । सा॒वि॒त्री । ए॒तेन॑ । मर्तः॑ । वृ॒णी॒त॒ । स॒ख्यम् ।  \newline


\textbf{Krama Paata} \newline

जु॒हो॒ति॒ य॒ज्ञ्स्य॑ । य॒ज्ञ्स्योद्य॑त्यै । उद्य॑त्या अनु॒ष्टुप् । उद्य॑त्या॒ इत्युत् - य॒त्यै॒ । अ॒नु॒ष्टुप् छन्द॑साम् । अ॒नु॒ष्टुबित्य॑नु - स्तुप् । छन्द॑सा॒मुत् । उद॑यच्छत् । अ॒य॒च्छ॒दिति॑ । इत्या॑हुः । आ॒हु॒स्तस्मा᳚त् । तस्मा॑दनु॒ष्टुभा᳚ । अ॒नु॒ष्टुभा॑ जुहोति । अ॒नु॒ष्टुभेत्य॑नु - स्तुभा᳚ । जु॒हो॒ति॒ य॒ज्ञ्स्य॑ । य॒ज्ञ्स्योद्य॑त्यै । उद्य॑त्यै॒ द्वाद॑श । उद्य॑त्या॒ इत्युत् - य॒त्यै॒ । द्वाद॑श वाथ्सब॒न्धानि॑ । वा॒थ्स॒ब॒न्धान्युत् । वा॒थ्स॒ब॒न्धानीति॑ वाथ्स - ब॒न्धानि॑ । उद॑यच्छन्॑ । अ॒य॒च्छ॒न्निति॑ । इत्या॑हुः । आ॒हु॒स्तस्मा᳚त् । तस्मा᳚द् द्वाद॒शभिः॑ । द्वा॒द॒शभि॑र् वाथ्सबन्ध॒विदः॑ । द्वा॒द॒शभि॒रिति॑ द्वाद॒श - भिः॒ । वा॒थ्स॒ब॒न्ध॒विदो॑ दीक्षयन्ति । वा॒थ्स॒ब॒न्ध॒विद॒ इति॑ वाथ्सबन्ध - विदः॑ । दी॒क्ष॒य॒न्ति॒ सा । सा वै । वा ए॒षा । ए॒षर्क् । ऋग॑नु॒ष्टुक् । अ॒नु॒ष्टुग् वाक् । अ॒नु॒ष्टुगित्य॑नु - स्तुक् । वाग॑नु॒ष्टुक् । अ॒नु॒ष्टुग् यत् । अ॒नु॒ष्टुगित्य॑नु - स्तुक् । यदे॒तया᳚ । ए॒तय॒र्चा । ऋ॒चा दी॒क्षय॑ति । दी॒क्षय॑ति वा॒चा । वा॒चैव । ए॒वैन᳚म् । ए॒नꣳ॒॒ सर्व॑या । सर्व॑या दीक्षयति । दी॒क्ष॒य॒ति॒ विश्वे᳚ । विश्वे॑ दे॒वस्य॑ । दे॒वस्य॑ ने॒तुः । ने॒तुरिति॑ । इत्या॑ह । आ॒ह॒ सा॒वि॒त्री । सा॒वि॒त्र्ये॑तेन॑ । ए॒तेन॒ मर्तः॑ । मर्तो॑ वृणीत । वृ॒णी॒त॒ स॒ख्यम् । स॒ख्यमिति॑ \newline

\textbf{Jatai Paata} \newline

1. जु॒हो॒ति॒ य॒ज्ञ्स्य॑ य॒ज्ञ्स्य॑ जुहोति जुहोति य॒ज्ञ्स्य॑ । \newline
2. य॒ज्ञ् स्योद्य॑त्या॒ उद्य॑त्यै य॒ज्ञ्स्य॑ य॒ज्ञ् स्योद्य॑त्यै । \newline
3. उद्य॑त्या अनु॒ष्टुब॑ नु॒ष्टु बुद्य॑त्या॒ उद्य॑त्या अनु॒ष्टुप् । \newline
4. उद्य॑त्या॒ इत्युत् - य॒त्यै॒ । \newline
5. अ॒नु॒ष्टुप् छन्द॑सा॒म् छन्द॑सा मनु॒ष्टु ब॑नु॒ष्टुप् छन्द॑साम् । \newline
6. अ॒नु॒ष्टुबित्य॑नु - स्तुप् । \newline
7. छन्द॑सा॒ मुदु च्छन्द॑सा॒म् छन्द॑सा॒ मुत् । \newline
8. उद॑यच्छ दयच्छ॒ दुदु द॑यच्छत् । \newline
9. अ॒य॒च्छ॒ दिती त्य॑यच्छ दयच्छ॒ दिति॑ । \newline
10. इत्या॑हु राहु॒ रिती त्या॑हुः । \newline
11. आ॒हु॒ स्तस्मा॒त् तस्मा॑ दाहु राहु॒ स्तस्मा᳚त् । \newline
12. तस्मा॑ दनु॒ष्टुभा॑ ऽनु॒ष्टुभा॒ तस्मा॒त् तस्मा॑ दनु॒ष्टुभा᳚ । \newline
13. अ॒नु॒ष्टुभा॑ जुहोति जुहो त्यनु॒ष्टुभा॑ ऽनु॒ष्टुभा॑ जुहोति । \newline
14. अ॒नु॒ष्टुभेत्य॑नु - स्तुभा᳚ । \newline
15. जु॒हो॒ति॒ य॒ज्ञ्स्य॑ य॒ज्ञ्स्य॑ जुहोति जुहोति य॒ज्ञ्स्य॑ । \newline
16. य॒ज्ञ् स्योद्य॑त्या॒ उद्य॑त्यै य॒ज्ञ्स्य॑ य॒ज्ञ् स्योद्य॑त्यै । \newline
17. उद्य॑त्यै॒ द्वाद॑श॒ द्वाद॒शो द्य॑त्या॒ उद्य॑त्यै॒ द्वाद॑श । \newline
18. उद्य॑त्या॒ इत्युत् - य॒त्यै॒ । \newline
19. द्वाद॑श वाथ्सब॒न्धानि॑ वाथ्सब॒न्धानि॒ द्वाद॑श॒ द्वाद॑श वाथ्सब॒न्धानि॑ । \newline
20. वा॒थ्स॒ब॒न्धा न्युदुद् वा᳚थ्सब॒न्धानि॑ वाथ्सब॒न्धा न्युत् । \newline
21. वा॒थ्स॒ब॒न्धानीति॑ वाथ्स - ब॒न्धानि॑ । \newline
22. उद॑यच्छन् नयच्छ॒न् नुदु द॑यच्छन्न् । \newline
23. अ॒य॒च्छ॒न् निती त्य॑यच्छन् नयच्छ॒न् निति॑ । \newline
24. इत्या॑हु राहु॒ रिती त्या॑हुः । \newline
25. आ॒हु॒ स्तस्मा॒त् तस्मा॑ दाहु राहु॒ स्तस्मा᳚त् । \newline
26. तस्मा᳚द् द्वाद॒शभि॑र् द्वाद॒शभि॒ स्तस्मा॒त् तस्मा᳚द् द्वाद॒शभिः॑ । \newline
27. द्वा॒द॒शभि॑र् वाथ्सबन्ध॒विदो॑ वाथ्सबन्ध॒विदो᳚ द्वाद॒शभि॑र् द्वाद॒शभि॑र् वाथ्सबन्ध॒विदः॑ । \newline
28. द्वा॒द॒शभि॒रिति॑ द्वाद॒श - भिः॒ । \newline
29. वा॒थ्स॒ब॒न्ध॒विदो॑ दीक्षयन्ति दीक्षयन्ति वाथ्सबन्ध॒विदो॑ वाथ्सबन्ध॒विदो॑ दीक्षयन्ति । \newline
30. वा॒थ्स॒ब॒न्ध॒विद॒ इति॑ वाथ्सबन्ध - विदः॑ । \newline
31. दी॒क्ष॒य॒न्ति॒ सा सा दी᳚क्षयन्ति दीक्षयन्ति॒ सा । \newline
32. सा वै वै सा सा वै । \newline
33. वा ए॒षैषा वै वा ए॒षा । \newline
34. ए॒ष र्गृगे॒षैष र्‌क् । \newline
35. ऋग॑नु॒ष्टु ग॑नु॒ष्टु गृगृ ग॑नु॒ष्टुक् । \newline
36. अ॒नु॒ष्टुग् वाग् वाग॑नु॒ष्टु ग॑नु॒ष्टुग् वाक् । \newline
37. अ॒नु॒ष्टुगित्य॑नु - स्तुक् । \newline
38. वाग॑नु॒ष्टु ग॑नु॒ष्टुग् वाग् वाग॑नु॒ष्टुक् । \newline
39. अ॒नु॒ष्टुग् यद् यद॑नु॒ष्टु ग॑नु॒ष्टुग् यत् । \newline
40. अ॒नु॒ष्टुगित्य॑नु - स्तुक् । \newline
41. यदे॒त यै॒तया॒ यद् यदे॒तया᳚ । \newline
42. ए॒तय॒ र्‌च र्‌चैत यै॒तय॒ र्‌चा । \newline
43. ऋ॒चा दी॒क्षय॑ति दी॒क्षय॑ त्यृ॒च र्‌चा दी॒क्षय॑ति । \newline
44. दी॒क्षय॑ति वा॒चा वा॒चा दी॒क्षय॑ति दी॒क्षय॑ति वा॒चा । \newline
45. वा॒चै वैव वा॒चा वा॒चैव । \newline
46. ए॒वैन॑ मेन मे॒वै वैन᳚म् । \newline
47. ए॒नꣳ॒॒ सर्व॑या॒ सर्व॑यैन मेनꣳ॒॒ सर्व॑या । \newline
48. सर्व॑या दीक्षयति दीक्षयति॒ सर्व॑या॒ सर्व॑या दीक्षयति । \newline
49. दी॒क्ष॒य॒ति॒ विश्वे॒ विश्वे॑ दीक्षयति दीक्षयति॒ विश्वे᳚ । \newline
50. विश्वे॑ दे॒वस्य॑ दे॒वस्य॒ विश्वे॒ विश्वे॑ दे॒वस्य॑ । \newline
51. दे॒वस्य॑ ने॒तुर् ने॒तुर् दे॒वस्य॑ दे॒वस्य॑ ने॒तुः । \newline
52. ने॒तु रितीति॑ ने॒तुर् ने॒तु रिति॑ । \newline
53. इत्या॑हा॒हे तीत्या॑ह । \newline
54. आ॒ह॒ सा॒वि॒त्री सा॑वि॒ त्र्या॑हाह सावि॒त्री । \newline
55. सा॒वि॒ त्र्ये॑ते नै॒तेन॑ सावि॒त्री सा॑वि॒ त्र्ये॑तेन॑ । \newline
56. ए॒तेन॒ मर्तो॒ मर्त॑ ए॒ते नै॒तेन॒ मर्तः॑ । \newline
57. मर्तो॑ वृणीत वृणीत॒ मर्तो॒ मर्तो॑ वृणीत । \newline
58. वृ॒णी॒त॒ स॒ख्यꣳ स॒ख्यं ॅवृ॑णीत वृणीत स॒ख्यम् । \newline
59. स॒ख्य मितीति॑ स॒ख्यꣳ स॒ख्य मिति॑ । \newline

\textbf{Ghana Paata } \newline

1. जु॒हो॒ति॒ य॒ज्ञ्स्य॑ य॒ज्ञ्स्य॑ जुहोति जुहोति य॒ज्ञ् स्योद्य॑त्या॒ उद्य॑त्यै य॒ज्ञ्स्य॑ जुहोति जुहोति य॒ज्ञ् स्योद्य॑त्यै । \newline
2. य॒ज्ञ् स्योद्य॑त्या॒ उद्य॑त्यै य॒ज्ञ्स्य॑ य॒ज्ञ् स्योद्य॑त्या अनु॒ष्टु ब॑नु॒ष्टु बुद्य॑त्यै य॒ज्ञ्स्य॑ 
य॒ज्ञ् स्योद्य॑त्या अनु॒ष्टुप् । \newline
3. उद्य॑त्या अनु॒ष्टु ब॑नु॒ष्टु बुद्य॑त्या॒ उद्य॑त्या अनु॒ष्टुप् छन्द॑सा॒म् छन्द॑सा मनु॒ष्टु बुद्य॑त्या॒ उद्य॑त्या अनु॒ष्टुप् छन्द॑साम् । \newline
4. उद्य॑त्या॒ इत्युत् - य॒त्यै॒ । \newline
5. अ॒नु॒ष्टुप् छन्द॑सा॒म् छन्द॑सा मनु॒ष्टु ब॑नु॒ष्टुप् छन्द॑सा॒ मुदुच् छन्द॑सा मनु॒ष्टु ब॑नु॒ष्टुप् छन्द॑सा॒ मुत् । \newline
6. अ॒नु॒ष्टुबित्य॑नु - स्तुप् । \newline
7. छन्द॑सा॒ मुदुच् छन्द॑सा॒म् छन्द॑सा॒ मुद॑यच्छ दयच्छ॒दुच् छन्द॑सा॒म् छन्द॑सा॒ मुद॑यच्छत् । \newline
8. उद॑यच्छ दयच्छ॒ दुदु द॑यच्छ॒ दिती त्य॑यच्छ॒ दुदु द॑यच्छ॒ दिति॑ । \newline
9. अ॒य॒च्छ॒ दिती त्य॑यच्छ दयच्छ॒ दित्या॑हु राहु॒रि त्य॑यच्छ दयच्छ॒दि त्या॑हुः । \newline
10. इत्या॑हु राहु॒रि तीत्या॑हु॒ स्तस्मा॒त् तस्मा॑ दाहु॒ रितीत्या॑हु॒ स्तस्मा᳚त् । \newline
11. आ॒हु॒ स्तस्मा॒त् तस्मा॑ दाहु राहु॒ स्तस्मा॑ दनु॒ष्टुभा॑ ऽनु॒ष्टुभा॒ तस्मा॑ दाहु राहु॒ स्तस्मा॑ दनु॒ष्टुभा᳚ । \newline
12. तस्मा॑द नु॒ष्टुभा॑ ऽनु॒ष्टुभा॒ तस्मा॒त् तस्मा॑ दनु॒ष्टुभा॑ जुहोति जुहो त्यनु॒ष्टुभा॒ तस्मा॒त् तस्मा॑ दनु॒ष्टुभा॑ जुहोति । \newline
13. अ॒नु॒ष्टुभा॑ जुहोति जुहो त्यनु॒ष्टुभा॑ ऽनु॒ष्टुभा॑ जुहोति य॒ज्ञ्स्य॑ य॒ज्ञ्स्य॑ जुहो त्यनु॒ष्टुभा॑ ऽनु॒ष्टुभा॑ जुहोति य॒ज्ञ्स्य॑ । \newline
14. अ॒नु॒ष्टुभेत्य॑नु - स्तुभा᳚ । \newline
15. जु॒हो॒ति॒ य॒ज्ञ्स्य॑ य॒ज्ञ्स्य॑ जुहोति जुहोति य॒ज्ञ् स्योद्य॑त्या॒ उद्य॑त्यै य॒ज्ञ्स्य॑ जुहोति जुहोति य॒ज्ञ् स्योद्य॑त्यै । \newline
16. य॒ज्ञ् स्योद्य॑त्या॒ उद्य॑त्यै य॒ज्ञ्स्य॑ य॒ज्ञ् स्योद्य॑त्यै॒ द्वाद॑श॒ द्वाद॒शो द्य॑त्यै य॒ज्ञ्स्य॑ 
य॒ज्ञ् स्योद्य॑त्यै॒ द्वाद॑श । \newline
17. उद्य॑त्यै॒ द्वाद॑श॒ द्वाद॒ शोद्य॑त्या॒ उद्य॑त्यै॒ द्वाद॑श वाथ्सब॒न्धानि॑ वाथ्सब॒न्धानि॒ द्वाद॒ शोद्य॑त्या॒ उद्य॑त्यै॒ द्वाद॑श वाथ्सब॒न्धानि॑ । \newline
18. उद्य॑त्या॒ इत्युत् - य॒त्यै॒ । \newline
19. द्वाद॑श वाथ्सब॒न्धानि॑ वाथ्सब॒न्धानि॒ द्वाद॑श॒ द्वाद॑श वाथ्सब॒न्धा न्युदुद् वा᳚थ्सब॒न्धानि॒ द्वाद॑श॒ द्वाद॑श वाथ्सब॒न्धान्युत् । \newline
20. वा॒थ्स॒ब॒न्धा न्युदुद् वा᳚थ्सब॒न्धानि॑ वाथ्सब॒न्धान्यु द॑यच्छन् नयच्छ॒न् नुद् वा᳚थ्सब॒न्धानि॑ वाथ्सब॒न्धा न्युद॑यच्छन्न् । \newline
21. वा॒थ्स॒ब॒न्धानीति॑ वाथ्स - ब॒न्धानि॑ । \newline
22. उद॑यच्छन् नयच्छ॒न् नुदु द॑यच्छ॒न् निती त्य॑यच्छ॒न् नुदु द॑यच्छ॒न् निति॑ । \newline
23. अ॒य॒च्छ॒न् निती त्य॑यच्छन् नयच्छ॒न् नित्या॑हु राहु॒ रित्य॑यच्छन् नयच्छ॒न् नित्या॑हुः । \newline
24. इत्या॑हु राहु॒ रितीत्या॑हु॒ स्तस्मा॒त् तस्मा॑ दाहु॒ रितीत्या॑हु॒ स्तस्मा᳚त् । \newline
25. आ॒हु॒ स्तस्मा॒त् तस्मा॑ दाहु राहु॒ स्तस्मा᳚द् द्वाद॒शभि॑र् द्वाद॒शभि॒ स्तस्मा॑ दाहु राहु॒ स्तस्मा᳚द् द्वाद॒शभिः॑ । \newline
26. तस्मा᳚द् द्वाद॒शभि॑र् द्वाद॒शभि॒ स्तस्मा॒त् तस्मा᳚द् द्वाद॒शभि॑र् वाथ्सबन्ध॒विदो॑ वाथ्सबन्ध॒विदो᳚ द्वाद॒शभि॒ स्तस्मा॒त् तस्मा᳚द् द्वाद॒शभि॑र् वाथ्सबन्ध॒विदः॑ । \newline
27. द्वा॒द॒शभि॑र् वाथ्सबन्ध॒विदो॑ वाथ्सबन्ध॒विदो᳚ द्वाद॒शभि॑र् द्वाद॒शभि॑र् वाथ्सबन्ध॒विदो॑ दीक्षयन्ति दीक्षयन्ति वाथ्सबन्ध॒विदो᳚ द्वाद॒शभि॑र् द्वाद॒शभि॑र् वाथ्सबन्ध॒विदो॑ दीक्षयन्ति । \newline
28. द्वा॒द॒शभि॒रिति॑ द्वाद॒श - भिः॒ । \newline
29. वा॒थ्स॒ब॒न्ध॒विदो॑ दीक्षयन्ति दीक्षयन्ति वाथ्सबन्ध॒विदो॑ वाथ्सबन्ध॒विदो॑ दीक्षयन्ति॒ सा सा 
दी᳚क्षयन्ति वाथ्सबन्ध॒विदो॑ वाथ्सबन्ध॒विदो॑ दीक्षयन्ति॒ सा । \newline
30. वा॒थ्स॒ब॒न्ध॒विद॒ इति॑ वाथ्सबन्ध - विदः॑ । \newline
31. दी॒क्ष॒य॒न्ति॒ सा सा दी᳚क्षयन्ति दीक्षयन्ति॒ सा वै वै सा दी᳚क्षयन्ति दीक्षयन्ति॒ सा वै । \newline
32. सा वै वै सा सा वा ए॒षैषा वै सा सा वा ए॒षा । \newline
33. वा ए॒षैषा वै वा ए॒ष र्‌गृ गे॒षा वै वा ए॒षर्क् । \newline
34. ए॒ष र्‌गृ गे॒षैष र्ग॑नु॒ष्टु ग॑नु॒ष्टु गृगे॒षैष र्‌ग॑नु॒ष्टुक् । \newline
35. ऋग॑नु॒ष्टु ग॑नु॒ष्टु गृगृ ग॑नु॒ष्टुग् वाग् वाग॑नु॒ष्टु गृगृ ग॑नु॒ष्टुग् वाक् । \newline
36. अ॒नु॒ष्टुग् वाग् वाग॑नु॒ष्टु ग॑नु॒ष्टुग् वाग॑नु॒ष्टु ग॑नु॒ष्टुग् वाग॑नु॒ष्टु ग॑नु॒ष्टुग् वाग॑नु॒ष्टुक् । \newline
37. अ॒नु॒ष्टुगित्य॑नु - स्तुक् । \newline
38. वाग॑नु॒ष्टु ग॑नु॒ष्टुग् वाग् वाग॑नु॒ष्टुग् यद् यद॑नु॒ष्टुग् वाग् वाग॑नु॒ष्टुग् यत् । \newline
39. अ॒नु॒ष्टुग् यद् यद॑नु॒ष्टु ग॑नु॒ष्टुग् यदे॒त यै॒तया॒ यद॑नु॒ष्टु ग॑नु॒ष्टुग् यदे॒तया᳚ । \newline
40. अ॒नु॒ष्टुगित्य॑नु - स्तुक् । \newline
41. यदे॒त यै॒तया॒ यद् यदे॒तय॒ र्‌च र्‌चैतया॒ यद् यदे॒तय॒ र्‌चा । \newline
42. ए॒तय॒ र्‌च र्‌चैत यै॒तय॒ र्‌चा दी॒क्षय॑ति दी॒क्षय॑ त्यृ॒चैत यै॒तय॒ र्‌चा दी॒क्षय॑ति । \newline
43. ऋ॒चा दी॒क्षय॑ति दी॒क्षय॑ त्यृ॒च र्‌चा दी॒क्षय॑ति वा॒चा वा॒चा दी॒क्षय॑ त्यृ॒च र्‌चा दी॒क्षय॑ति वा॒चा । \newline
44. दी॒क्षय॑ति वा॒चा वा॒चा दी॒क्षय॑ति दी॒क्षय॑ति वा॒चै वैव वा॒चा दी॒क्षय॑ति दी॒क्षय॑ति वा॒चैव । \newline
45. वा॒चै वैव वा॒चा वा॒चै वैन॑ मेन मे॒व वा॒चा वा॒चै वैन᳚म् । \newline
46. ए॒वैन॑ मेन मे॒वै वैनꣳ॒॒ सर्व॑या॒ सर्व॑यैन मे॒वै वैनꣳ॒॒ सर्व॑या । \newline
47. ए॒नꣳ॒॒ सर्व॑या॒ सर्व॑यैन मेनꣳ॒॒ सर्व॑या दीक्षयति दीक्षयति॒ सर्व॑यैन मेनꣳ॒॒ सर्व॑या दीक्षयति । \newline
48. सर्व॑या दीक्षयति दीक्षयति॒ सर्व॑या॒ सर्व॑या दीक्षयति॒ विश्वे॒ विश्वे॑ दीक्षयति॒ सर्व॑या॒ सर्व॑या दीक्षयति॒ विश्वे᳚ । \newline
49. दी॒क्ष॒य॒ति॒ विश्वे॒ विश्वे॑ दीक्षयति दीक्षयति॒ विश्वे॑ दे॒वस्य॑ दे॒वस्य॒ विश्वे॑ दीक्षयति दीक्षयति॒ विश्वे॑ दे॒वस्य॑ । \newline
50. विश्वे॑ दे॒वस्य॑ दे॒वस्य॒ विश्वे॒ विश्वे॑ दे॒वस्य॑ ने॒तुर् ने॒तुर् दे॒वस्य॒ विश्वे॒ विश्वे॑ दे॒वस्य॑ ने॒तुः । \newline
51. दे॒वस्य॑ ने॒तुर् ने॒तुर् दे॒वस्य॑ दे॒वस्य॑ ने॒तु रितीति॑ ने॒तुर् दे॒वस्य॑ दे॒वस्य॑ ने॒तु रिति॑ । \newline
52. ने॒तु रितीति॑ ने॒तुर् ने॒तुरित्या॑ हा॒हेति॑ ने॒तुर् ने॒तुरि त्या॑ह । \newline
53. इत्या॑हा॒हे तीत्या॑ह सावि॒त्री सा॑वि॒त्र्या॑हे तीत्या॑ह सावि॒त्री । \newline
54. आ॒ह॒ सा॒वि॒त्री सा॑वि॒ त्र्या॑हाह सावि॒त्र्ये॑ते नै॒तेन॑ सावि॒त्र्या॑हाह सावि॒ त्र्ये॑तेन॑ । \newline
55. सा॒वि॒त्र्ये॑ते नै॒तेन॑ सावि॒त्री सा॑वि॒ त्र्ये॑तेन॒ मर्तो॒ मर्त॑ ए॒तेन॑ सावि॒त्री सा॑वि॒ त्र्ये॑तेन॒ मर्तः॑ । \newline
56. ए॒तेन॒ मर्तो॒ मर्त॑ ए॒ते नै॒तेन॒ मर्तो॑ वृणीत वृणीत॒ मर्त॑ ए॒ते नै॒तेन॒ मर्तो॑ वृणीत । \newline
57. मर्तो॑ वृणीत वृणीत॒ मर्तो॒ मर्तो॑ वृणीत स॒ख्यꣳ स॒ख्यं ॅवृ॑णीत॒ मर्तो॒ मर्तो॑ वृणीत स॒ख्यम् । \newline
58. वृ॒णी॒त॒ स॒ख्यꣳ स॒ख्यं ॅवृ॑णीत वृणीत स॒ख्य मितीति॑ स॒ख्यं ॅवृ॑णीत वृणीत स॒ख्य मिति॑ । \newline
59. स॒ख्य मितीति॑ स॒ख्यꣳ स॒ख्य मित्या॑हा॒हेति॑ स॒ख्यꣳ स॒ख्य मित्या॑ह । \newline
\pagebreak
\markright{ TS 6.1.2.6  \hfill https://www.vedavms.in \hfill}

\section{ TS 6.1.2.6 }

\textbf{TS 6.1.2.6 } \newline
\textbf{Samhita Paata} \newline

-मित्या॑ह पितृदेव॒त्यै॑तेन॒ विश्वे॑ रा॒य इ॑षुद्ध्य॒सीत्या॑ह वैश्वदे॒व्ये॑तेन॑ द्यु॒म्नं ॅवृ॑णीत पु॒ष्यस॒ इत्या॑ह पौ॒ष्ण्ये॑तेन॒ सा वा ए॒षर्ख्स॑र्वदेव॒त्या॑ यदे॒तय॒र्चा दी॒क्षय॑ति॒ सर्वा॑भिरे॒वैनं॑ दे॒वता॑भिर्दीक्षयति स॒प्ताक्ष॑रं प्रथ॒मं प॒दम॒ष्टाक्ष॑राणि॒ त्रीणि॒ यानि॒ त्रीणि॒ तान्य॒ष्टावुप॑ यन्ति॒ यानि॑ च॒त्वारि॒ तान्य॒ष्टौ यद॒ष्टाक्ष॑रा॒ तेन॑ - [  ] \newline

\textbf{Pada Paata} \newline

इति॑ । आ॒ह॒ । पि॒तृ॒दे॒व॒त्येति॑ पितृ - दे॒व॒त्या᳚ । ए॒तेन॑ । विश्वे᳚ । रा॒यः । इ॒षु॒द्ध्य॒सि॒ । इति॑ । आ॒ह॒ । वै॒श्व॒दे॒वीति॑ वैश्व-दे॒वी । ए॒तेन॑ । द्यु॒म्नम् । वृ॒णी॒त॒ । पु॒ष्यसे᳚ । इति॑ । आ॒ह॒ । पौ॒ष्णी । ए॒तेन॑ । सा । वै । ए॒षा । ऋक् । स॒र्व॒दे॒व॒त्येति॑ सर्व - दे॒व॒त्या᳚ । यत् । ए॒तया᳚ । ऋ॒चा । दी॒क्षय॑ति । सर्वा॑भिः । ए॒व । ए॒न॒म् । दे॒वता॑भिः । दी॒क्ष॒य॒ति॒ । स॒प्ताक्ष॑र॒मिति॑ स॒प्त - अ॒क्ष॒र॒म् । प्र॒थ॒मम् । प॒दम् । अ॒ष्टाक्ष॑रा॒णीत्य॒ष्टा -अ॒क्ष॒रा॒णि॒ । त्रीणि॑ । यानि॑ । त्रीणि॑ । तानि॑ । अ॒ष्टौ । उपेति॑ । य॒न्ति॒ । यानि॑ । च॒त्वारि॑ । तानि॑ । अ॒ष्टौ । यत् । अ॒ष्टाक्ष॒रेत्य॒ष्टा - अ॒क्ष॒रा॒ । तेन॑ ।  \newline


\textbf{Krama Paata} \newline

इत्या॑ह । आ॒ह॒ पि॒तृ॒दे॒व॒त्या᳚ । पि॒तृ॒दे॒व॒त्यै॑तेन॑ । पि॒तृ॒दे॒व॒त्येति॑ पितृ - दे॒वत्या᳚ । ए॒तेन॒ विश्वे᳚ । विश्वे॑ रा॒यः । रा॒य इ॑षुद्ध्यसि । इ॒षु॒द्ध्य॒सीति॑ । इत्या॑ह । आ॒ह॒ वै॒श्व॒दे॒वी । वै॒श्व॒दे॒व्ये॑तेन॑ । वै॒श्व॒दे॒वीति॑ वैश्व - दे॒वी । ए॒तेन॑ द्यु॒म्नम् । द्यु॒म्नम् ॅवृ॑णीत । वृ॒णी॒त॒ पु॒ष्यसे᳚ । पु॒ष्यस॒ इति॑ । इत्या॑ह । आ॒ह॒ पौ॒ष्णी । पौ॒ष्ण्ये॑तेन॑ । ए॒तेन॒ सा । सा वै । वा ए॒षा । ए॒षर्क् । ऋख् स॑र्वदेव॒त्या᳚ । स॒र्व॒दे॒व॒त्या॑ यत् । स॒र्व॒दे॒व॒त्येति॑ सर्व - दे॒व॒त्या᳚ । यदे॒तया᳚ । ए॒तय॒र्चा । ऋ॒चा दी॒क्षय॑ति । दी॒क्षय॑ति॒ सर्वा॑भिः । सर्वा॑भिरे॒व । ए॒वैन᳚म् । ए॒न॒म् दे॒वता॑भिः । दे॒वता॑भिर् दीक्षयति । दी॒क्ष॒य॒ति॒ स॒प्ताक्ष॑रम् । स॒प्ताक्ष॑रम् प्रथ॒मम् । स॒प्ताक्ष॑र॒मिति॑ स॒प्त - अ॒क्ष॒र॒म् । प्र॒थ॒मम् प॒दम् । प॒दम॒ष्टाक्ष॑राणि । अ॒ष्टाक्ष॑राणि॒ त्रीणि॑ । अ॒ष्टाक्ष॑रा॒णीत्य॒ष्टा - अ॒क्ष॒रा॒णि॒ । त्रीणि॒ यानि॑ । यानि॒ त्रीणि॑ । त्रीणि॒ तानि॑ । तान्य॒ष्टौ । अ॒ष्टावुप॑ । उप॑ यन्ति । य॒न्ति॒ यानि॑ । यानि॑ च॒त्वारि॑ । च॒त्वारि॒ तानि॑ । तान्य॒ष्टौ । अ॒ष्टौ यत् । यद॒ष्टाक्ष॑रा । अ॒ष्टाक्ष॑रा॒ तेन॑ । अ॒ष्टाक्ष॒रेत्य॒ष्टा - अ॒क्ष॒रा॒ । तेन॑ गाय॒त्री \newline

\textbf{Jatai Paata} \newline

1. इत्या॑हा॒हे तीत्या॑ह । \newline
2. आ॒ह॒ पि॒तृ॒दे॒व॒त्या॑ पितृदेव॒त्या॑ ऽऽहाह पितृदेव॒त्या᳚ । \newline
3. पि॒तृ॒दे॒व॒ त्यै॑ते नै॒तेन॑ पितृदेव॒त्या॑ पितृदेव॒ त्यै॑तेन॑ । \newline
4. पि॒तृ॒दे॒व॒त्येति॑ पितृ - दे॒व॒त्या᳚ । \newline
5. ए॒तेन॒ विश्वे॒ विश्व॑ ए॒तेनै॒ तेन॒ विश्वे᳚ । \newline
6. विश्वे॑ रा॒यो रा॒यो विश्वे॒ विश्वे॑ रा॒यः । \newline
7. रा॒य इ॑षुद्ध्यसी षुद्ध्यसि रा॒यो रा॒य इ॑षुद्ध्यसि । \newline
8. इ॒षु॒द्ध्य॒ सीतीती॑ षुद्ध्यसी षुद्ध्य॒ सीति॑ । \newline
9. इत्या॑हा॒हे तीत्या॑ह । \newline
10. आ॒ह॒ वै॒श्व॒दे॒वी वै᳚श्वदे॒ व्या॑हाह वैश्वदे॒वी । \newline
11. वै॒श्व॒दे॒व्ये॑ तेनै॒तेन॑ वैश्वदे॒वी वै᳚श्वदे॒व्ये॑ तेन॑ । \newline
12. वै॒श्व॒दे॒वीति॑ वैश्व - दे॒वी । \newline
13. ए॒तेन॑ द्यु॒म्नम् द्यु॒म्न मे॒तेनै॒ तेन॑ द्यु॒म्नम् । \newline
14. द्यु॒म्नं ॅवृ॑णीत वृणीत द्यु॒म्नम् द्यु॒म्नं ॅवृ॑णीत । \newline
15. वृ॒णी॒त॒ पु॒ष्यसे॑ पु॒ष्यसे॑ वृणीत वृणीत पु॒ष्यसे᳚ । \newline
16. पु॒ष्यस॒ इतीति॑ पु॒ष्यसे॑ पु॒ष्यस॒ इति॑ । \newline
17. इत्या॑हा॒हे तीत्या॑ह । \newline
18. आ॒ह॒ पौ॒ष्णी पौ॒ष्ण्या॑ हाह पौ॒ष्णी । \newline
19. पौ॒ष्ण्ये॑ तेनै॒तेन॑ पौ॒ष्णी पौ॒ष्ण्ये॑तेन॑ । \newline
20. ए॒तेन॒ सा सैते नै॒तेन॒ सा । \newline
21. सा वै वै सा सा वै । \newline
22. वा ए॒षैषा वै वा ए॒षा । \newline
23. ए॒ष र्गृगे॒ षैषर्क् । \newline
24. ऋख् स॑र्वदेव॒त्या॑ सर्वदेव॒त्य॑ र्गृख् स॑र्वदेव॒त्या᳚ । \newline
25. स॒र्व॒दे॒व॒त्या॑ यद् यथ् स॑र्वदेव॒त्या॑ सर्वदेव॒त्या॑ यत् । \newline
26. स॒र्व॒दे॒व॒त्येति॑ सर्व - दे॒व॒त्या᳚ । \newline
27. यदे॒त यै॒तया॒ यद् यदे॒तया᳚ । \newline
28. ए॒तय॒ र्‌च र्‌चैत यै॒तय॒ र्‌चा । \newline
29. ऋ॒चा दी॒क्षय॑ति दी॒क्षय॑ त्यृ॒च र्‌चा दी॒क्षय॑ति । \newline
30. दी॒क्षय॑ति॒ सर्वा॑भिः॒ सर्वा॑भिर् दी॒क्षय॑ति दी॒क्षय॑ति॒ सर्वा॑भिः । \newline
31. सर्वा॑भिरे॒ वैव सर्वा॑भिः॒ सर्वा॑भि रे॒व । \newline
32. ए॒वैन॑ मेन मे॒वै वैन᳚म् । \newline
33. ए॒न॒म् दे॒वता॑भिर् दे॒वता॑भि रेन मेनम् दे॒वता॑भिः । \newline
34. दे॒वता॑भिर् दीक्षयति दीक्षयति दे॒वता॑भिर् दे॒वता॑भिर् दीक्षयति । \newline
35. दी॒क्ष॒य॒ति॒ स॒प्ताक्ष॑रꣳ स॒प्ताक्ष॑रम् दीक्षयति दीक्षयति स॒प्ताक्ष॑रम् । \newline
36. स॒प्ताक्ष॑रम् प्रथ॒मम् प्र॑थ॒मꣳ स॒प्ताक्ष॑रꣳ स॒प्ताक्ष॑रम् प्रथ॒मम् । \newline
37. स॒प्ताक्ष॑र॒मिति॑ स॒प्त - अ॒क्ष॒र॒म् । \newline
38. प्र॒थ॒मम् प॒दम् प॒दम् प्र॑थ॒मम् प्र॑थ॒मम् प॒दम् । \newline
39. प॒द म॒ष्टाक्ष॑राण्य॒ ष्टाक्ष॑राणि प॒दम् प॒द म॒ष्टाक्ष॑राणि । \newline
40. अ॒ष्टाक्ष॑राणि॒ त्रीणि॒ त्रीण्य॒ष्टाक्ष॑रा ण्य॒ष्टाक्ष॑राणि॒ त्रीणि॑ । \newline
41. अ॒ष्टाक्ष॑रा॒णीत्य॒ष्टा - अ॒क्ष॒रा॒णि॒ । \newline
42. त्रीणि॒ यानि॒ यानि॒ त्रीणि॒ त्रीणि॒ यानि॑ । \newline
43. यानि॒ त्रीणि॒ त्रीणि॒ यानि॒ यानि॒ त्रीणि॑ । \newline
44. त्रीणि॒ तानि॒ तानि॒ त्रीणि॒ त्रीणि॒ तानि॑ । \newline
45. तान्य॒ष्टा व॒ष्टौ तानि॒ तान्य॒ष्टौ । \newline
46. अ॒ष्टा वुपो पा॒ष्टा व॒ष्टा वुप॑ । \newline
47. उप॑ यन्ति य॒न्त्यु पोप॑ यन्ति । \newline
48. य॒न्ति॒ यानि॒ यानि॑ यन्ति यन्ति॒ यानि॑ । \newline
49. यानि॑ च॒त्वारि॑ च॒त्वारि॒ यानि॒ यानि॑ च॒त्वारि॑ । \newline
50. च॒त्वारि॒ तानि॒ तानि॑ च॒त्वारि॑ च॒त्वारि॒ तानि॑ । \newline
51. तान्य॒ष्टा व॒ष्टौ तानि॒ तान्य॒ष्टौ । \newline
52. अ॒ष्टौ यद् यद॒ष्टा व॒ष्टौ यत् । \newline
53. यद॒ष्टाक्ष॑रा॒ ऽष्टाक्ष॑रा॒ यद् यद॒ष्टाक्ष॑रा । \newline
54. अ॒ष्टाक्ष॑रा॒ तेन॒ तेना॒ ष्टाक्ष॑रा॒ ऽष्टाक्ष॑रा॒ तेन॑ । \newline
55. अ॒ष्टाक्ष॒रेत्य॒ष्टा - अ॒क्ष॒रा॒ । \newline
56. तेन॑ गाय॒त्री गा॑य॒त्री तेन॒ तेन॑ गाय॒त्री । \newline

\textbf{Ghana Paata } \newline

1. इत्या॑हा॒हे तीत्या॑ह पितृदेव॒त्या॑ पितृदेव॒त्या॑ ऽऽहे तीत्या॑ह पितृदेव॒त्या᳚ । \newline
2. आ॒ह॒ पि॒तृ॒दे॒व॒त्या॑ पितृदेव॒त्या॑ ऽऽहाह पितृदेव॒ त्यै॑ते नै॒तेन॑ पितृदेव॒त्या॑ ऽऽहाह पितृदेव॒
त्यै॑तेन॑ । \newline
3. पि॒तृ॒दे॒व॒ त्यै॑ते नै॒तेन॑ पितृदेव॒त्या॑ पितृदेव॒ त्यै॑तेन॒ विश्वे॒ विश्व॑ ए॒तेन॑ पितृदेव॒त्या॑ पितृदेव॒ त्यै॑तेन॒ विश्वे᳚ । \newline
4. पि॒तृ॒दे॒व॒त्येति॑ पितृ - दे॒व॒त्या᳚ । \newline
5. ए॒तेन॒ विश्वे॒ विश्व॑ ए॒ते नै॒तेन॒ विश्वे॑ रा॒यो रा॒यो विश्व॑ ए॒ते नै॒तेन॒ विश्वे॑ रा॒यः । \newline
6. विश्वे॑ रा॒यो रा॒यो विश्वे॒ विश्वे॑ रा॒य इ॑षुद्ध्यसी षुद्ध्यसि रा॒यो विश्वे॒ विश्वे॑ रा॒य इ॑षुद्ध्यसि । \newline
7. रा॒य इ॑षुद्ध्यसी षुद्ध्यसि रा॒यो रा॒य इ॑षुद्ध्य॒सी तीती॑ षुद्ध्यसि रा॒यो रा॒य इ॑षुद्ध्य॒ सीति॑ । \newline
8. इ॒षु॒द्ध्य॒सी तीती॑ षुद्ध्यसी षुद्ध्य॒सी त्या॑हा॒हे ती॑षुद्ध्यसी षुद्ध्य॒सी त्या॑ह । \newline
9. इत्या॑हा॒हे तीत्या॑ह वैश्वदे॒वी वै᳚श्वदे॒व्या॑हे तीत्या॑ह वैश्वदे॒वी । \newline
10. आ॒ह॒ वै॒श्व॒दे॒वी वै᳚श्वदे॒व्या॑ हाह वैश्वदे॒व्ये॑ तेनै॒तेन॑ वैश्वदे॒व्या॑ हाह वैश्वदे॒व्ये॑ तेन॑ । \newline
11. वै॒श्व॒दे॒व्ये॑ते नै॒तेन॑ वैश्वदे॒वी वै᳚श्वदे॒व्ये॑तेन॑ द्यु॒म्नम् द्यु॒म्न मे॒तेन॑ वैश्वदे॒वी वै᳚श्वदे॒व्ये॑तेन॑ द्यु॒म्नम् । \newline
12. वै॒श्व॒दे॒वीति॑ वैश्व - दे॒वी । \newline
13. ए॒तेन॑ द्यु॒म्नम् द्यु॒म्न मे॒ते नै॒तेन॑ द्यु॒म्नं ॅवृ॑णीत वृणीत द्यु॒म्न मे॒ते नै॒तेन॑ द्यु॒म्नं ॅवृ॑णीत । \newline
14. द्यु॒म्नं ॅवृ॑णीत वृणीत द्यु॒म्नम् द्यु॒म्नं ॅवृ॑णीत पु॒ष्यसे॑ पु॒ष्यसे॑ वृणीत द्यु॒म्नम् द्यु॒म्नं ॅवृ॑णीत पु॒ष्यसे᳚ । \newline
15. वृ॒णी॒त॒ पु॒ष्यसे॑ पु॒ष्यसे॑ वृणीत वृणीत पु॒ष्यस॒ इतीति॑ पु॒ष्यसे॑ वृणीत वृणीत पु॒ष्यस॒ इति॑ । \newline
16. पु॒ष्यस॒ इतीति॑ पु॒ष्यसे॑ पु॒ष्यस॒ इत्या॑हा॒हेति॑ पु॒ष्यसे॑ पु॒ष्यस॒ इत्या॑ह । \newline
17. इत्या॑हा॒हे तीत्या॑ह पौ॒ष्णी पौ॒ष्ण्या॑हे तीत्या॑ह पौ॒ष्णी । \newline
18. आ॒ह॒ पौ॒ष्णी पौ॒ष्ण्या॑हाह पौ॒ष्ण्ये॑ते नै॒तेन॑ पौ॒ष्ण्या॑ हाह पौ॒ष्ण्ये॑तेन॑ । \newline
19. पौ॒ष्ण्ये॑ते नै॒तेन॑ पौ॒ष्णी पौ॒ष्ण्ये॑तेन॒ सा सैतेन॑ पौ॒ष्णी पौ॒ष्ण्ये॑तेन॒ सा । \newline
20. ए॒तेन॒ सा सैते नै॒तेन॒ सा वै वै सैते नै॒तेन॒ सा वै । \newline
21. सा वै वै सा सा वा ए॒षैषा वै सा सा वा ए॒षा । \newline
22. वा ए॒षैषा वै वा ए॒ष र्‌गृ गे॒षा वै वा ए॒षर्क् । \newline
23. ए॒ष र्‌गृ गे॒षैष र्‌ख् स॑र्वदेव॒त्या॑ सर्वदेव॒त्य॑ र्‌गे॒ षैषर्ख् स॑र्वदेव॒त्या᳚ । \newline
24. ऋख् स॑र्वदेव॒त्या॑ सर्वदेव॒त्य॑ र्‌गृख् स॑र्वदेव॒त्या॑ यद् यथ् स॑र्वदेव॒त्य॑ र्‌गृख् स॑र्वदेव॒त्या॑ यत् । \newline
25. स॒र्व॒दे॒व॒त्या॑ यद् यथ् स॑र्वदेव॒त्या॑ सर्वदेव॒त्या॑ यदे॒त यै॒तया॒ यथ् स॑र्वदेव॒त्या॑ सर्वदेव॒त्या॑ यदे॒तया᳚ । \newline
26. स॒र्व॒दे॒व॒त्येति॑ सर्व - दे॒व॒त्या᳚ । \newline
27. यदे॒त यै॒तया॒ यद् यदे॒तय॒ र्‌च र्‌चैतया॒ यद् यदे॒तय॒ र्‌चा । \newline
28. ए॒तय॒ र्‌च र्‌चैत यै॒तय॒ र्‌चा दी॒क्षय॑ति दी॒क्षय॑ त्यृ॒चैत यै॒तय॒ र्‌चा दी॒क्षय॑ति । \newline
29. ऋ॒चा दी॒क्षय॑ति दी॒क्षय॑ त्यृ॒च र्‌चा दी॒क्षय॑ति॒ सर्वा॑भिः॒ सर्वा॑भिर् दी॒क्षय॑ त्यृ॒च र्‌चा दी॒क्षय॑ति॒ सर्वा॑भिः । \newline
30. दी॒क्षय॑ति॒ सर्वा॑भिः॒ सर्वा॑भिर् दी॒क्षय॑ति दी॒क्षय॑ति॒ सर्वा॑भि रे॒वैव सर्वा॑भिर् दी॒क्षय॑ति दी॒क्षय॑ति॒ सर्वा॑भि रे॒व । \newline
31. सर्वा॑भि रे॒वैव सर्वा॑भिः॒ सर्वा॑भि रे॒वैन॑ मेन मे॒व सर्वा॑भिः॒ सर्वा॑भि रे॒वैन᳚म् । \newline
32. ए॒वैन॑ मेन मे॒वै वैन॑म् दे॒वता॑भिर् दे॒वता॑भि रेन मे॒वै वैन॑म् दे॒वता॑भिः । \newline
33. ए॒न॒म् दे॒वता॑भिर् दे॒वता॑भि रेन मेनम् दे॒वता॑भिर् दीक्षयति दीक्षयति दे॒वता॑भि रेन मेनम् दे॒वता॑भिर् दीक्षयति । \newline
34. दे॒वता॑भिर् दीक्षयति दीक्षयति दे॒वता॑भिर् दे॒वता॑भिर् दीक्षयति स॒प्ताक्ष॑रꣳ स॒प्ताक्ष॑रम् दीक्षयति दे॒वता॑भिर् दे॒वता॑भिर् दीक्षयति स॒प्ताक्ष॑रम् । \newline
35. दी॒क्ष॒य॒ति॒ स॒प्ताक्ष॑रꣳ स॒प्ताक्ष॑रम् दीक्षयति दीक्षयति स॒प्ताक्ष॑रम् प्रथ॒मम् प्र॑थ॒मꣳ स॒प्ताक्ष॑रम् दीक्षयति दीक्षयति स॒प्ताक्ष॑रम् प्रथ॒मम् । \newline
36. स॒प्ताक्ष॑रम् प्रथ॒मम् प्र॑थ॒मꣳ स॒प्ताक्ष॑रꣳ स॒प्ताक्ष॑रम् प्रथ॒मम् प॒दम् प॒दम् प्र॑थ॒मꣳ स॒प्ताक्ष॑रꣳ स॒प्ताक्ष॑रम् प्रथ॒मम् प॒दम् । \newline
37. स॒प्ताक्ष॑र॒मिति॑ स॒प्त - अ॒क्ष॒र॒म् । \newline
38. प्र॒थ॒मम् प॒दम् प॒दम् प्र॑थ॒मम् प्र॑थ॒मम् प॒द म॒ष्टाक्ष॑रा ण्य॒ष्टाक्ष॑राणि प॒दम् प्र॑थ॒मम् प्र॑थ॒मम् प॒द म॒ष्टाक्ष॑राणि । \newline
39. प॒द म॒ष्टाक्ष॑रा ण्य॒ष्टाक्ष॑राणि प॒दम् प॒द म॒ष्टाक्ष॑राणि॒ त्रीणि॒ त्रीण्य॒ष्टाक्ष॑राणि प॒दम् प॒द म॒ष्टाक्ष॑राणि॒ त्रीणि॑ । \newline
40. अ॒ष्टाक्ष॑राणि॒ त्रीणि॒ त्रीण्य॒ष्टाक्ष॑रा ण्य॒ष्टाक्ष॑राणि॒ त्रीणि॒ यानि॒ यानि॒ त्रीण्य॒ष्टाक्ष॑रा
ण्य॒ष्टाक्ष॑राणि॒ त्रीणि॒ यानि॑ । \newline
41. अ॒ष्टाक्ष॑रा॒णीत्य॒ष्टा - अ॒क्ष॒रा॒णि॒ । \newline
42. त्रीणि॒ यानि॒ यानि॒ त्रीणि॒ त्रीणि॒ यानि॒ त्रीणि॒ त्रीणि॒ यानि॒ त्रीणि॒ त्रीणि॒ यानि॒ त्रीणि॑ । \newline
43. यानि॒ त्रीणि॒ त्रीणि॒ यानि॒ यानि॒ त्रीणि॒ तानि॒ तानि॒ त्रीणि॒ यानि॒ यानि॒ त्रीणि॒ तानि॑ । \newline
44. त्रीणि॒ तानि॒ तानि॒ त्रीणि॒ त्रीणि॒ तान्य॒ष्टा व॒ष्टौ तानि॒ त्रीणि॒ त्रीणि॒ तान्य॒ष्टौ । \newline
45. तान्य॒ष्टा व॒ष्टौ तानि॒ तान्य॒ष्टा वुपो पा॒ष्टौ तानि॒ तान्य॒ष्टा वुप॑ । \newline
46. अ॒ष्टा वुपो पा॒ष्टा व॒ष्टा वुप॑ यन्ति य॒न् त्युपा॒ष्टा व॒ष्टा वुप॑ यन्ति । \newline
47. उप॑ यन्ति य॒न् त्युपोप॑ यन्ति॒ यानि॒ यानि॑ य॒न् त्युपोप॑ यन्ति॒ यानि॑ । \newline
48. य॒न्ति॒ यानि॒ यानि॑ यन्ति यन्ति॒ यानि॑ च॒त्वारि॑ च॒त्वारि॒ यानि॑ यन्ति यन्ति॒ यानि॑ च॒त्वारि॑ । \newline
49. यानि॑ च॒त्वारि॑ च॒त्वारि॒ यानि॒ यानि॑ च॒त्वारि॒ तानि॒ तानि॑ च॒त्वारि॒ यानि॒ यानि॑ च॒त्वारि॒ तानि॑ । \newline
50. च॒त्वारि॒ तानि॒ तानि॑ च॒त्वारि॑ च॒त्वारि॒ तान्य॒ष्टा व॒ष्टौ तानि॑ च॒त्वारि॑ च॒त्वारि॒ तान्य॒ष्टौ । \newline
51. तान्य॒ष्टा व॒ष्टौ तानि॒ तान्य॒ष्टौ यद् यद॒ष्टौ तानि॒ तान्य॒ष्टौ यत् । \newline
52. अ॒ष्टौ यद् यद॒ष्टा व॒ष्टौ यद॒ष्टाक्ष॑रा॒ ऽष्टाक्ष॑रा॒ यद॒ष्टा व॒ष्टौ यद॒ष्टाक्ष॑रा । \newline
53. यद॒ष्टाक्ष॑रा॒ ऽष्टाक्ष॑रा॒ यद् यद॒ष्टाक्ष॑रा॒ तेन॒ तेना॒ ष्टाक्ष॑रा॒ यद् यद॒ष्टाक्ष॑रा॒ तेन॑ । \newline
54. अ॒ष्टाक्ष॑रा॒ तेन॒ तेना॒ ष्टाक्ष॑रा॒ ऽष्टाक्ष॑रा॒ तेन॑ गाय॒त्री गा॑य॒त्री तेना॒ ष्टाक्ष॑रा॒ ऽष्टाक्ष॑रा॒ तेन॑ गाय॒त्री । \newline
55. अ॒ष्टाक्ष॒रेत्य॒ष्टा - अ॒क्ष॒रा॒ । \newline
56. तेन॑ गाय॒त्री गा॑य॒त्री तेन॒ तेन॑ गाय॒त्री यद् यद् गा॑य॒त्री तेन॒ तेन॑ गाय॒त्री यत् । \newline
\pagebreak
\markright{ TS 6.1.2.7  \hfill https://www.vedavms.in \hfill}

\section{ TS 6.1.2.7 }

\textbf{TS 6.1.2.7 } \newline
\textbf{Samhita Paata} \newline

गाय॒त्री यदेका॑दशाक्षरा॒ तेन॑ त्रि॒ष्टुग्यद् द्वाद॑शाक्षरा॒ तेन॒ जग॑ती॒ सा वा ए॒षर्ख्सर्वा॑णि॒ छन्दाꣳ॑सि॒ यदे॒तय॒र्चा दी॒क्षय॑ति॒ सर्वे॑भिरे॒वैनं॒ छन्दो॑भिर्दीक्षयति स॒प्ताक्ष॑रं प्रथ॒मं प॒दꣳ स॒प्तप॑दा॒ शक्व॑री प॒शवः॒ शक्व॑री प॒शूने॒वाव॑ रुन्ध॒ एक॑स्माद॒क्षरा॒दना᳚प्तं प्रथ॒मं प॒दं तस्मा॒द्-यद्-वा॒चोऽना᳚प्तं॒ तन्म॑नु॒ष्या॑ उप॑ जीवन्ति पू॒र्णया॑ जुहोति ( ) पू॒र्ण इ॑व॒ हि प्र॒जाप॑तिः प्र॒जाप॑ते॒राप्त्यै॒ न्यू॑नया जुहोति॒ न्यू॑ना॒द्धि प्र॒जाप॑तिः प्र॒जा असृ॑जत प्र॒जानाꣳ॒॒ सृष्ट्यै᳚ ॥ \newline

\textbf{Pada Paata} \newline

गा॒य॒त्री । यत् । एका॑दशाक्ष॒रेत्येका॑दश - अ॒क्ष॒रा॒ । तेन॑ । त्रि॒ष्टुक् । यत् । द्वाद॑शाक्ष॒रेति॒ द्वाद॑श - अ॒क्ष॒रा॒ । तेन॑ । जग॑ती । सा । वै । ए॒षा । ऋक् । सर्वा॑णि । छन्दाꣳ॑सि । यत् । ए॒तया᳚ । ऋ॒चा । दी॒क्षय॑ति । सर्वे॑भिः । ए॒व । ए॒न॒म् । छन्दो॑भि॒रिति॒ छन्दः॑ - भिः॒ । दी॒क्ष॒य॒ति॒ । स॒प्ताक्ष॑र॒मिति॑ स॒प्त - अ॒क्ष॒रम् । प्र॒थ॒मम् । प॒दम् । स॒प्तप॒देति॑ स॒प्त - प॒दा॒ । शक्व॑री । प॒शवः॑ । शक्व॑री । प॒शून् । ए॒व । अवेति॑ । रु॒न्धे॒ । एक॑स्मात् । अ॒क्षरा᳚त् । अना᳚प्तम् । प्र॒थ॒मम् । प॒दम् । तस्मा᳚त् । यत् । वा॒चः । अना᳚प्तम् । तत् । म॒नु॒ष्याः᳚ । उपेति॑ । जी॒व॒न्ति॒ । पू॒र्णया᳚ । जु॒हो॒ति॒ ( ) । पू॒र्णः । इ॒व॒ । हि । प्र॒जाप॑ति॒रिति॑ प्र॒जा-प॒तिः॒ । प्र॒जाप॑ते॒रिति॑ प्र॒जा-प॒तेः॒ । आप्त्यै᳚ । न्यू॑न॒येति॒ नि-ऊ॒न॒या॒ । जु॒हो॒ति॒ । न्यू॑ना॒दिति॒ नि - ऊ॒ना॒त् । हि । प्र॒जाप॑ति॒रिति॑ प्र॒जा - प॒तिः॒ । प्र॒जा इति॑ प्र - जाः । असृ॑जत । प्र॒जाना॒मिति॑ प्र-जाना᳚म् । सृष्ट्यै᳚ ॥  \newline


\textbf{Krama Paata} \newline

गा॒य॒त्री यत् । यदेका॑दशाक्षरा । एका॑दशाक्षरा॒ तेन॑ । एका॑दशाक्ष॒रेत्येका॑दश - अ॒क्ष॒रा॒ । तेन॑ त्रि॒ष्टुक् । त्रि॒ष्टुग् यत् । यद् द्वाद॑शाक्षरा । द्वाद॑शाक्षरा॒ तेन॑ । द्वाद॑शाक्ष॒रेति॒ द्वाद॑श - अ॒क्ष॒रा॒ । तेन॒ जग॑ती । जग॑ती॒ सा । सा वै । वा ए॒षा । ए॒षर्क् । ऋख् सर्वा॑णि । सर्वा॑णि॒ छन्दाꣳ॑सि । छन्दाꣳ॑सि॒ यत् । यदे॒तया᳚ । ए॒तय॒र्चा । ऋ॒चा दी॒क्षय॑ति । दी॒क्षय॑ति॒ सर्वे॑भिः । सर्वे॑भिरे॒व । ए॒वैन᳚म् । ए॒न॒म् छन्दो॑भिः । छन्दो॑भिर् दीक्षयति । छन्दो॑भि॒रिति॒ छन्दः॑ - भिः॒ । दी॒क्ष॒य॒ति॒ स॒प्ताक्ष॑रम् । स॒प्ताक्ष॑रम् प्रथ॒मम् । स॒प्ताक्ष॑र॒मिति॑ स॒प्त - अ॒क्ष॒र॒म् । प्र॒थ॒मम् प॒दम् । प॒दꣳ स॒प्तप॑दा । स॒प्तप॑दा॒ शक्व॑री । स॒प्तप॒देति॑ स॒प्त - प॒दा॒ । शक्व॑री प॒शवः॑ । प॒शवः॒ शक्व॑री । शक्व॑री प॒शून् । प॒शूने॒व । ए॒वाव॑ । अव॑ रुन्धे । रु॒न्ध॒ एक॑स्मात् । एक॑स्माद॒क्षरा᳚त् । अ॒क्षरा॒दना᳚प्तम् । अना᳚प्तम् प्रथ॒मम् । प्र॒थ॒मम् प॒दम् । प॒दम् तस्मा᳚त् । तस्मा॒द् यत् । यद् वा॒चः । वा॒चोऽना᳚प्तम् । अना᳚प्त॒म् तत् । तन् म॑नु॒ष्याः᳚ । म॒नु॒ष्या॑ उप॑ । उप॑ जीवन्ति । जी॒व॒न्ति॒ पू॒र्णया᳚ । पू॒र्णया॑ जुहोति ( ) । जु॒हो॒ति॒ पू॒र्णः । पू॒र्ण इ॑व । इ॒व॒ हि । हि प्र॒जाप॑तिः । प्र॒जाप॑तिः प्र॒जाप॑तेः । प्र॒जाप॑ति॒रिति॑ प्र॒जा - प॒तिः॒ । प्र॒जाप॑ते॒राप्त्यै᳚ । प्र॒जाप॑ते॒रिति॑ प्र॒जा - प॒तेः॒ । आप्त्यै॒ न्यू॑नया । न्यू॑नया जुहोति । न्यू॑न॒येति॒ नि - ऊ॒न॒या॒ । जु॒हो॒ति॒ न्यू॑नात् । न्यू॑ना॒द्‌धि । न्यू॑ना॒दिति॒ नि - ऊ॒ना॒त्॒ । हि प्र॒जाप॑तिः । प्र॒जाप॑तिः प्र॒जाः । प्र॒जाप॑ति॒रिति॑ प्र॒जा - प॒तिः॒ । प्र॒जा असृ॑जत । प्र॒जा इति॑ प्र - जाः । असृ॑जत प्र॒जाना᳚म् । प्र॒जानाꣳ॒॒ सृष्ट्‍यै᳚ । प्र॒जाना॒मिति॑ प्र - जाना᳚म् । सृष्ट्‍या॒ इति॒ सृष्ट्‍यै᳚ । \newline

\textbf{Jatai Paata} \newline

1. गा॒य॒त्री यद् यद् गा॑य॒त्री गा॑य॒त्री यत् । \newline
2. यदेका॑दशाक्ष॒ रैका॑दशाक्षरा॒ यद् यदेका॑दशाक्षरा । \newline
3. एका॑दशाक्षरा॒ तेन॒ तेनै का॑दशाक्ष॒ रैका॑दशाक्षरा॒ तेन॑ । \newline
4. एका॑दशाक्ष॒रेत्येका॑दश - अ॒क्ष॒रा॒ । \newline
5. तेन॑ त्रि॒ष्टुक् त्रि॒ष्टुक् तेन॒ तेन॑ त्रि॒ष्टुक् । \newline
6. त्रि॒ष्टुग् यद् यत् त्रि॒ष्टुक् त्रि॒ष्टुग् यत् । \newline
7. यद् द्वाद॑शाक्षरा॒ द्वाद॑शाक्षरा॒ यद् यद् द्वाद॑शाक्षरा । \newline
8. द्वाद॑शाक्षरा॒ तेन॒ तेन॒ द्वाद॑शाक्षरा॒ द्वाद॑शाक्षरा॒ तेन॑ । \newline
9. द्वाद॑शाक्ष॒रेति॒ द्वाद॑श - अ॒क्ष॒रा॒ । \newline
10. तेन॒ जग॑ती॒ जग॑ती॒ तेन॒ तेन॒ जग॑ती । \newline
11. जग॑ती॒ सा सा जग॑ती॒ जग॑ती॒ सा । \newline
12. सा वै वै सा सा वै । \newline
13. वा ए॒षैषा वै वा ए॒षा । \newline
14. ए॒ष र्गृगे॒ षैष र्क् । \newline
15. ऋख् सर्वा॑णि॒ सर्वा॒ण्यृग् ऋख् सर्वा॑णि । \newline
16. सर्वा॑णि॒ छन्दाꣳ॑सि॒ छन्दाꣳ॑सि॒ सर्वा॑णि॒ सर्वा॑णि॒ छन्दाꣳ॑सि । \newline
17. छन्दाꣳ॑सि॒ यद् यच् छन्दाꣳ॑सि॒ छन्दाꣳ॑सि॒ यत् । \newline
18. यदे॒त यै॒तया॒ यद् यदे॒तया᳚ । \newline
19. ए॒तय॒ र्‌च र्‌चैत यै॒तय॒ र्‌चा । \newline
20. ऋ॒चा दी॒क्षय॑ति दी॒क्षय॑ त्यृ॒च र्‌चा दी॒क्षय॑ति । \newline
21. दी॒क्षय॑ति॒ सर्वे॑भिः॒ सर्वे॑भिर् दी॒क्षय॑ति दी॒क्षय॑ति॒ सर्वे॑भिः । \newline
22. सर्वे॑भि रे॒वैव सर्वे॑भिः॒ सर्वे॑भि रे॒व । \newline
23. ए॒वैन॑ मेन मे॒वै वैन᳚म् । \newline
24. ए॒न॒म् छन्दो॑भि॒ श्छन्दो॑भि रेन मेन॒म् छन्दो॑भिः । \newline
25. छन्दो॑भिर् दीक्षयति दीक्षयति॒ छन्दो॑भि॒ श्छन्दो॑भिर् दीक्षयति । \newline
26. छन्दो॑भि॒रिति॒ छन्दः॑ - भिः॒ । \newline
27. दी॒क्ष॒य॒ति॒ स॒प्ताक्ष॑रꣳ स॒प्ताक्ष॑रम् दीक्षयति दीक्षयति स॒प्ताक्ष॑रम् । \newline
28. स॒प्ताक्ष॑रम् प्रथ॒मम् प्र॑थ॒मꣳ स॒प्ताक्ष॑रꣳ स॒प्ताक्ष॑रम् प्रथ॒मम् । \newline
29. स॒प्ताक्ष॑र॒मिति॑ स॒प्त - अ॒क्ष॒र॒म् । \newline
30. प्र॒थ॒मम् प॒दम् प॒दम् प्र॑थ॒मम् प्र॑थ॒मम् प॒दम् । \newline
31. प॒दꣳ स॒प्तप॑दा स॒प्तप॑दा प॒दम् प॒दꣳ स॒प्तप॑दा । \newline
32. स॒प्तप॑दा॒ शक्व॑री॒ शक्व॑री स॒प्तप॑दा स॒प्तप॑दा॒ शक्व॑री । \newline
33. स॒प्तप॒देति॑ स॒प्त - प॒दा॒ । \newline
34. शक्व॑री प॒शवः॑ प॒शवः॒ शक्व॑री॒ शक्व॑री प॒शवः॑ । \newline
35. प॒शवः॒ शक्व॑री॒ शक्व॑री प॒शवः॑ प॒शवः॒ शक्व॑री । \newline
36. शक्व॑री प॒शून् प॒शूञ् छक्व॑री॒ शक्व॑री प॒शून् । \newline
37. प॒शू ने॒वैव प॒शून् प॒शू ने॒व । \newline
38. ए॒वावा वै॒वै वाव॑ । \newline
39. अव॑ रुन्धे रु॒न्धे ऽवाव॑ रुन्धे । \newline
40. रु॒न्ध॒ एक॑स्मा॒ देक॑स्माद् रुन्धे रुन्ध॒ एक॑स्मात् । \newline
41. एक॑स्मा द॒क्षरा॑ द॒क्षरा॒ देक॑स्मा॒ देक॑स्मा द॒क्षरा᳚त् । \newline
42. अ॒क्षरा॒ दना᳚प्त॒ मना᳚प्त म॒क्षरा॑ द॒क्षरा॒ दना᳚प्तम् । \newline
43. अना᳚प्तम् प्रथ॒मम् प्र॑थ॒म मना᳚प्त॒ मना᳚प्तम् प्रथ॒मम् । \newline
44. प्र॒थ॒मम् प॒दम् प॒दम् प्र॑थ॒मम् प्र॑थ॒मम् प॒दम् । \newline
45. प॒दम् तस्मा॒त् तस्मा᳚त् प॒दम् प॒दम् तस्मा᳚त् । \newline
46. तस्मा॒द् यद् यत् तस्मा॒त् तस्मा॒द् यत् । \newline
47. यद् वा॒चो वा॒चो यद् यद् वा॒चः । \newline
48. वा॒चो ऽना᳚प्त॒ मना᳚प्तं ॅवा॒चो वा॒चो ऽना᳚प्तम् । \newline
49. अना᳚प्त॒म् तत् तदना᳚प्त॒ मना᳚प्त॒म् तत् । \newline
50. तन् म॑नु॒ष्या॑ मनु॒ष्या᳚ स्तत् तन् म॑नु॒ष्याः᳚ । \newline
51. म॒नु॒ष्या॑ उपोप॑ मनु॒ष्या॑ मनु॒ष्या॑ उप॑ । \newline
52. उप॑ जीवन्ति जीव॒न् त्युपोप॑ जीवन्ति । \newline
53. जी॒व॒न्ति॒ पू॒र्णया॑ पू॒र्णया॑ जीवन्ति जीवन्ति पू॒र्णया᳚ । \newline
54. पू॒र्णया॑ जुहोति जुहोति पू॒र्णया॑ पू॒र्णया॑ जुहोति । \newline
55. जु॒हो॒ति॒ पू॒र्णः पू॒र्णो जु॑होति जुहोति पू॒र्णः । \newline
56. पू॒र्ण इ॑वेव पू॒र्णः पू॒र्ण इ॑व । \newline
57. इ॒व॒ हि हीवे॑व॒ हि । \newline
58. हि प्र॒जाप॑तिः प्र॒जाप॑ति॒र्॒. हि हि प्र॒जाप॑तिः । \newline
59. प्र॒जाप॑तिः प्र॒जाप॑तेः प्र॒जाप॑तेः प्र॒जाप॑तिः प्र॒जाप॑तिः प्र॒जाप॑तेः । \newline
60. प्र॒जाप॑ति॒रिति॑ प्र॒जा - प॒तिः॒ । \newline
61. प्र॒जाप॑ते॒ राप्त्या॒ आप्त्यै᳚ प्र॒जाप॑तेः प्र॒जाप॑ते॒ राप्त्यै᳚ । \newline
62. प्र॒जाप॑ते॒रिति॑ प्र॒जा - प॒तेः॒ । \newline
63. आप्त्यै॒ न्यू॑नया॒ न्यू॑न॒या ऽऽप्त्या॒ आप्त्यै॒ न्यू॑नया । \newline
64. न्यू॑नया जुहोति जुहोति॒ न्यू॑नया॒ न्यू॑नया जुहोति । \newline
65. न्यू॑न॒येति॒ नि - ऊ॒न॒या॒ । \newline
66. जु॒हो॒ति॒ न्यू॑ना॒न् न्यू॑नाज् जुहोति जुहोति॒ न्यू॑नात् । \newline
67. न्यू॑ना॒द्धि हि न्यू॑ना॒न् न्यू॑ना॒द्धि । \newline
68. न्यू॑ना॒दिति॒ नि - ऊ॒ना॒त् । \newline
69. हि प्र॒जाप॑तिः प्र॒जाप॑ति॒र्॒. हि हि प्र॒जाप॑तिः । \newline
70. प्र॒जाप॑तिः प्र॒जाः प्र॒जाः प्र॒जाप॑तिः प्र॒जाप॑तिः प्र॒जाः । \newline
71. प्र॒जाप॑ति॒रिति॑ प्र॒जा - प॒तिः॒ । \newline
72. प्र॒जा असृ॑ज॒ता सृ॑जत प्र॒जाः प्र॒जा असृ॑जत । \newline
73. प्र॒जा इति॑ प्र - जाः । \newline
74. असृ॑जत प्र॒जाना᳚म् प्र॒जाना॒ मसृ॑ज॒ता सृ॑जत प्र॒जाना᳚म् । \newline
75. प्र॒जानाꣳ॒॒ सृष्ट्यै॒ सृष्ट्यै᳚ प्र॒जाना᳚म् प्र॒जानाꣳ॒॒ सृष्ट्यै᳚ । \newline
76. प्र॒जाना॒मिति॑ प्र - जाना᳚म् । \newline
77. सृष्ट्‍या॒ इति॒ सृष्ट्‍यै᳚ । \newline

\textbf{Ghana Paata } \newline

1. गा॒य॒त्री यद् यद् गा॑य॒त्री गा॑य॒त्री यदेका॑दशाक्ष॒ रैका॑दशाक्षरा॒ यद् गा॑य॒त्री गा॑य॒त्री यदेका॑दशाक्षरा । \newline
2. यदेका॑दशाक्ष॒ रैका॑दशाक्षरा॒ यद् यदेका॑दशाक्षरा॒ तेन॒ तेनैका॑दशाक्षरा॒ यद् यदेका॑दशाक्षरा॒ तेन॑ । \newline
3. एका॑दशाक्षरा॒ तेन॒ तेनैका॑दशाक्ष॒ रैका॑दशाक्षरा॒ तेन॑ त्रि॒ष्टुक् त्रि॒ष्टुक् तेनै का॑दशाक्ष॒
रैका॑दशाक्षरा॒ तेन॑ त्रि॒ष्टुक् । \newline
4. एका॑दशाक्ष॒रेत्येका॑दश - अ॒क्ष॒रा॒ । \newline
5. तेन॑ त्रि॒ष्टुक् त्रि॒ष्टुक् तेन॒ तेन॑ त्रि॒ष्टुग् यद् यत् त्रि॒ष्टुक् तेन॒ तेन॑ त्रि॒ष्टुग् यत् । \newline
6. त्रि॒ष्टुग् यद् यत् त्रि॒ष्टुक् त्रि॒ष्टुग् यद् द्वाद॑शाक्षरा॒ द्वाद॑शाक्षरा॒ यत् त्रि॒ष्टुक् त्रि॒ष्टुग् यद् द्वाद॑शाक्षरा । \newline
7. यद् द्वाद॑शाक्षरा॒ द्वाद॑शाक्षरा॒ यद् यद् द्वाद॑शाक्षरा॒ तेन॒ तेन॒ द्वाद॑शाक्षरा॒ यद् यद् द्वाद॑शाक्षरा॒ तेन॑ । \newline
8. द्वाद॑शाक्षरा॒ तेन॒ तेन॒ द्वाद॑शाक्षरा॒ द्वाद॑शाक्षरा॒ तेन॒ जग॑ती॒ जग॑ती॒ तेन॒ द्वाद॑शाक्षरा॒ द्वाद॑शाक्षरा॒ तेन॒ जग॑ती । \newline
9. द्वाद॑शाक्ष॒रेति॒ द्वाद॑श - अ॒क्ष॒रा॒ । \newline
10. तेन॒ जग॑ती॒ जग॑ती॒ तेन॒ तेन॒ जग॑ती॒ सा सा जग॑ती॒ तेन॒ तेन॒ जग॑ती॒ सा । \newline
11. जग॑ती॒ सा सा जग॑ती॒ जग॑ती॒ सा वै वै सा जग॑ती॒ जग॑ती॒ सा वै । \newline
12. सा वै वै सा सा वा ए॒षैषा वै सा सा वा ए॒षा । \newline
13. वा ए॒षैषा वै वा ए॒ष र्‌गृ गे॒षा वै वा ए॒षर्क् । \newline
14. ए॒ष र्‌गृ गे॒षैष र्‌ख् सर्वा॑णि॒ सर्वा॒ ण्यृगे॒षैष र्‌ख् सर्वा॑णि । \newline
15. ऋख् सर्वा॑णि॒ सर्वा॒ ण्यृगृख् सर्वा॑णि॒ छन्दाꣳ॑सि॒ छन्दाꣳ॑सि॒ सर्वा॒ ण्यृगृख् सर्वा॑णि॒ छन्दाꣳ॑सि । \newline
16. सर्वा॑णि॒ छन्दाꣳ॑सि॒ छन्दाꣳ॑सि॒ सर्वा॑णि॒ सर्वा॑णि॒ छन्दाꣳ॑सि॒ यद् यच् छन्दाꣳ॑सि॒ सर्वा॑णि॒ सर्वा॑णि॒ छन्दाꣳ॑सि॒ यत् । \newline
17. छन्दाꣳ॑सि॒ यद् यच् छन्दाꣳ॑सि॒ छन्दाꣳ॑सि॒ यदे॒त यै॒तया॒ यच् छन्दाꣳ॑सि॒ छन्दाꣳ॑सि॒ यदे॒तया᳚ । \newline
18. यदे॒त यै॒तया॒ यद् यदे॒तय॒ र्‌च र्‌चैतया॒ यद् यदे॒ तय॒ र्‌चा । \newline
19. ए॒तय॒ र्‌च र्‌चैत यै॒तय॒ र्‌चा दी॒क्षय॑ति दी॒क्षय॑ त्यृ॒चैत यै॒तय॒ र्‌चा दी॒क्षय॑ति । \newline
20. ऋ॒चा दी॒क्षय॑ति दी॒क्षय॑ त्यृ॒च र्‌चा दी॒क्षय॑ति॒ सर्वे॑भिः॒ सर्वे॑भिर् दी॒क्षय॑ त्यृ॒च र्‌चा दी॒क्षय॑ति॒ सर्वे॑भिः । \newline
21. दी॒क्षय॑ति॒ सर्वे॑भिः॒ सर्वे॑भिर् दी॒क्षय॑ति दी॒क्षय॑ति॒ सर्वे॑भि रे॒वैव सर्वे॑भिर् दी॒क्षय॑ति दी॒क्षय॑ति॒ सर्वे॑भि रे॒व । \newline
22. सर्वे॑भि रे॒वैव सर्वे॑भिः॒ सर्वे॑भि रे॒वैन॑ मेन मे॒व सर्वे॑भिः॒ सर्वे॑भि रे॒वैन᳚म् । \newline
23. ए॒वैन॑ मेन मे॒वै वैन॒म् छन्दो॑भि॒ श्छन्दो॑भि रेन मे॒वै वैन॒म् छन्दो॑भिः । \newline
24. ए॒न॒म् छन्दो॑भि॒ श्छन्दो॑भि रेन मेन॒म् छन्दो॑भिर् दीक्षयति दीक्षयति॒ छन्दो॑भि रेन मेन॒म् छन्दो॑भिर् दीक्षयति । \newline
25. छन्दो॑भिर् दीक्षयति दीक्षयति॒ छन्दो॑भि॒ श्छन्दो॑भिर् दीक्षयति स॒प्ताक्ष॑रꣳ स॒प्ताक्ष॑रम् दीक्षयति॒ छन्दो॑भि॒ श्छन्दो॑भिर् दीक्षयति स॒प्ताक्ष॑रम् । \newline
26. छन्दो॑भि॒रिति॒ छन्दः॑ - भिः॒ । \newline
27. दी॒क्ष॒य॒ति॒ स॒प्ताक्ष॑रꣳ स॒प्ताक्ष॑रम् दीक्षयति दीक्षयति स॒प्ताक्ष॑रम् प्रथ॒मम् प्र॑थ॒मꣳ स॒प्ताक्ष॑रम् दीक्षयति दीक्षयति स॒प्ताक्ष॑रम् प्रथ॒मम् । \newline
28. स॒प्ताक्ष॑रम् प्रथ॒मम् प्र॑थ॒मꣳ स॒प्ताक्ष॑रꣳ स॒प्ताक्ष॑रम् प्रथ॒मम् प॒दम् प॒दम् प्र॑थ॒मꣳ स॒प्ताक्ष॑रꣳ स॒प्ताक्ष॑रम् प्रथ॒मम् प॒दम् । \newline
29. स॒प्ताक्ष॑र॒मिति॑ स॒प्त - अ॒क्ष॒र॒म् । \newline
30. प्र॒थ॒मम् प॒दम् प॒दम् प्र॑थ॒मम् प्र॑थ॒मम् प॒दꣳ स॒प्तप॑दा स॒प्तप॑दा प॒दम् प्र॑थ॒मम् प्र॑थ॒मम् प॒दꣳ स॒प्तप॑दा । \newline
31. प॒दꣳ स॒प्तप॑दा स॒प्तप॑दा प॒दम् प॒दꣳ स॒प्तप॑दा॒ शक्व॑री॒ शक्व॑री स॒प्तप॑दा प॒दम् प॒दꣳ स॒प्तप॑दा॒ शक्व॑री । \newline
32. स॒प्तप॑दा॒ शक्व॑री॒ शक्व॑री स॒प्तप॑दा स॒प्तप॑दा॒ शक्व॑री प॒शवः॑ प॒शवः॒ शक्व॑री स॒प्तप॑दा स॒प्तप॑दा॒ शक्व॑री प॒शवः॑ । \newline
33. स॒प्तप॒देति॑ स॒प्त - प॒दा॒ । \newline
34. शक्व॑री प॒शवः॑ प॒शवः॒ शक्व॑री॒ शक्व॑री प॒शवः॒ शक्व॑री॒ शक्व॑री प॒शवः॒ शक्व॑री॒ शक्व॑री प॒शवः॒ शक्व॑री । \newline
35. प॒शवः॒ शक्व॑री॒ शक्व॑री प॒शवः॑ प॒शवः॒ शक्व॑री प॒शून् प॒शूञ् छक्व॑री प॒शवः॑ प॒शवः॒ शक्व॑री प॒शून् । \newline
36. शक्व॑री प॒शून् प॒शूञ् छक्व॑री॒ शक्व॑री प॒शू ने॒वैव प॒शूञ् छक्व॑री॒ शक्व॑री प॒शूने॒व । \newline
37. प॒शू ने॒वैव प॒शून् प॒शू ने॒वावा वै॒व प॒शून् प॒शू ने॒वाव॑ । \newline
38. ए॒वावा वै॒वै वाव॑ रुन्धे रु॒न्धे ऽवै॒वै वाव॑ रुन्धे । \newline
39. अव॑ रुन्धे रु॒न्धे ऽवाव॑ रुन्ध॒ एक॑स्मा॒ देक॑स्माद् रु॒न्धे ऽवाव॑ रुन्ध॒ एक॑स्मात् । \newline
40. रु॒न्ध॒ एक॑स्मा॒ देक॑स्माद् रुन्धे रुन्ध॒ एक॑स्मा द॒क्षरा॑ द॒क्षरा॒ देक॑स्माद् रुन्धे रुन्ध॒ एक॑स्मा द॒क्षरा᳚त् । \newline
41. एक॑स्मा द॒क्षरा॑ द॒क्षरा॒ देक॑स्मा॒ देक॑स्मा द॒क्षरा॒ दना᳚प्त॒ मना᳚प्त म॒क्षरा॒ देक॑स्मा॒ देक॑स्मा द॒क्षरा॒ दना᳚प्तम् । \newline
42. अ॒क्षरा॒ दना᳚प्त॒ मना᳚प्त म॒क्षरा॑ द॒क्षरा॒ दना᳚प्तम् प्रथ॒मम् प्र॑थ॒म मना᳚प्त म॒क्षरा॑ द॒क्षरा॒ दना᳚प्तम् प्रथ॒मम् । \newline
43. अना᳚प्तम् प्रथ॒मम् प्र॑थ॒म मना᳚प्त॒ मना᳚प्तम् प्रथ॒मम् प॒दम् प॒दम् प्र॑थ॒म मना᳚प्त॒ मना᳚प्तम् प्रथ॒मम् प॒दम् । \newline
44. प्र॒थ॒मम् प॒दम् प॒दम् प्र॑थ॒मम् प्र॑थ॒मम् प॒दम् तस्मा॒त् तस्मा᳚त् प॒दम् प्र॑थ॒मम् प्र॑थ॒मम् प॒दम् तस्मा᳚त् । \newline
45. प॒दम् तस्मा॒त् तस्मा᳚त् प॒दम् प॒दम् तस्मा॒द् यद् यत् तस्मा᳚त् प॒दम् प॒दम् तस्मा॒द् यत् । \newline
46. तस्मा॒द् यद् यत् तस्मा॒त् तस्मा॒द् यद् वा॒चो वा॒चो यत् तस्मा॒त् तस्मा॒द् यद् वा॒चः । \newline
47. यद् वा॒चो वा॒चो यद् यद् वा॒चो ऽना᳚प्त॒ मना᳚प्तं ॅवा॒चो यद् यद् वा॒चो ऽना᳚प्तम् । \newline
48. वा॒चो ऽना᳚प्त॒ मना᳚प्तं ॅवा॒चो वा॒चो ऽना᳚प्त॒म् तत् तदना᳚प्तं ॅवा॒चो वा॒चो ऽना᳚प्त॒म् तत् । \newline
49. अना᳚प्त॒म् तत् तदना᳚प्त॒ मना᳚प्त॒म् तन् म॑नु॒ष्या॑ मनु॒ष्या᳚ स्तदना᳚प्त॒ मना᳚प्त॒म् तन् म॑नु॒ष्याः᳚ । \newline
50. तन् म॑नु॒ष्या॑ मनु॒ष्या᳚ स्तत् तन् म॑नु॒ष्या॑ उपोप॑ मनु॒ष्या᳚ स्तत् तन् म॑नु॒ष्या॑ उप॑ । \newline
51. म॒नु॒ष्या॑ उपोप॑ मनु॒ष्या॑ मनु॒ष्या॑ उप॑ जीवन्ति जीव॒न्त्युप॑ मनु॒ष्या॑ मनु॒ष्या॑ उप॑ जीवन्ति । \newline
52. उप॑ जीवन्ति जीव॒न् त्युपोप॑ जीवन्ति पू॒र्णया॑ पू॒र्णया॑ जीव॒न् त्युपोप॑ जीवन्ति पू॒र्णया᳚ । \newline
53. जी॒व॒न्ति॒ पू॒र्णया॑ पू॒र्णया॑ जीवन्ति जीवन्ति पू॒र्णया॑ जुहोति जुहोति पू॒र्णया॑ जीवन्ति जीवन्ति पू॒र्णया॑ जुहोति । \newline
54. पू॒र्णया॑ जुहोति जुहोति पू॒र्णया॑ पू॒र्णया॑ जुहोति पू॒र्णः पू॒र्णो जु॑होति पू॒र्णया॑ पू॒र्णया॑ जुहोति पू॒र्णः । \newline
55. जु॒हो॒ति॒ पू॒र्णः पू॒र्णो जु॑होति जुहोति पू॒र्ण इ॑वेव पू॒र्णो जु॑होति जुहोति पू॒र्ण इ॑व । \newline
56. पू॒र्ण इ॑वेव पू॒र्णः पू॒र्ण इ॑व॒ हि हीव॑ पू॒र्णः पू॒र्ण इ॑व॒ हि । \newline
57. इ॒व॒ हि हीवे॑व॒ हि प्र॒जाप॑तिः प्र॒जाप॑ति॒र्॒. हीवे॑व॒ हि प्र॒जाप॑तिः । \newline
58. हि प्र॒जाप॑तिः प्र॒जाप॑ति॒र्॒. हि हि प्र॒जाप॑तिः प्र॒जाप॑तेः प्र॒जाप॑तेः प्र॒जाप॑ति॒र्॒. हि हि प्र॒जाप॑तिः प्र॒जाप॑तेः । \newline
59. प्र॒जाप॑तिः प्र॒जाप॑तेः प्र॒जाप॑तेः प्र॒जाप॑तिः प्र॒जाप॑तिः प्र॒जाप॑ते॒ राप्त्या॒ आप्त्यै᳚ प्र॒जाप॑तेः प्र॒जाप॑तिः प्र॒जाप॑तिः प्र॒जाप॑ते॒ राप्त्यै᳚ । \newline
60. प्र॒जाप॑ति॒रिति॑ प्र॒जा - प॒तिः॒ । \newline
61. प्र॒जाप॑ते॒ राप्त्या॒ आप्त्यै᳚ प्र॒जाप॑तेः प्र॒जाप॑ते॒ राप्त्यै॒ न्यू॑नया॒ न्यू॑न॒या ऽऽप्त्यै᳚ प्र॒जाप॑तेः प्र॒जाप॑ते॒ राप्त्यै॒ न्यू॑नया । \newline
62. प्र॒जाप॑ते॒रिति॑ प्र॒जा - प॒तेः॒ । \newline
63. आप्त्यै॒ न्यू॑नया॒ न्यू॑न॒या ऽऽप्त्या॒ आप्त्यै॒ न्यू॑नया जुहोति जुहोति॒ न्यू॑न॒या ऽऽप्त्या॒ आप्त्यै॒ न्यू॑नया जुहोति । \newline
64. न्यू॑नया जुहोति जुहोति॒ न्यू॑नया॒ न्यू॑नया जुहोति॒ न्यू॑ना॒न् न्यू॑नाज् जुहोति॒ न्यू॑नया॒ न्यू॑नया जुहोति॒ न्यू॑नात् । \newline
65. न्यू॑न॒येति॒ नि - ऊ॒न॒या॒ । \newline
66. जु॒हो॒ति॒ न्यू॑ना॒न् न्यू॑नाज् जुहोति जुहोति॒ न्यू॑ना॒द्धि हि न्यू॑नाज् जुहोति जुहोति॒ न्यू॑ना॒द्धि । \newline
67. न्यू॑ना॒द्धि हि न्यू॑ना॒न् न्यू॑ना॒द्धि प्र॒जाप॑तिः प्र॒जाप॑ति॒र्॒. हि न्यू॑ना॒न् न्यू॑ना॒द्धि प्र॒जाप॑तिः । \newline
68. न्यू॑ना॒दिति॒ नि - ऊ॒ना॒त् । \newline
69. हि प्र॒जाप॑तिः प्र॒जाप॑ति॒र्॒. हि हि प्र॒जाप॑तिः प्र॒जाः प्र॒जाः प्र॒जाप॑ति॒र्॒. हि हि प्र॒जाप॑तिः प्र॒जाः । \newline
70. प्र॒जाप॑तिः प्र॒जाः प्र॒जाः प्र॒जाप॑तिः प्र॒जाप॑तिः प्र॒जा असृ॑ज॒ता सृ॑जत प्र॒जाः प्र॒जाप॑तिः प्र॒जाप॑तिः प्र॒जा असृ॑जत । \newline
71. प्र॒जाप॑ति॒रिति॑ प्र॒जा - प॒तिः॒ । \newline
72. प्र॒जा असृ॑ज॒ता सृ॑जत प्र॒जाः प्र॒जा असृ॑जत प्र॒जाना᳚म् प्र॒जाना॒ मसृ॑जत प्र॒जाः प्र॒जा असृ॑जत प्र॒जाना᳚म् । \newline
73. प्र॒जा इति॑ प्र - जाः । \newline
74. असृ॑जत प्र॒जाना᳚म् प्र॒जाना॒ मसृ॑ज॒ता सृ॑जत प्र॒जानाꣳ॒॒ सृष्ट्यै॒ सृष्ट्यै᳚ प्र॒जाना॒ मसृ॑ज॒ता सृ॑जत प्र॒जानाꣳ॒॒ सृष्ट्यै᳚ । \newline
75. प्र॒जानाꣳ॒॒ सृष्ट्यै॒ सृष्ट्यै᳚ प्र॒जाना᳚म् प्र॒जानाꣳ॒॒ सृष्ट्यै᳚ । \newline
76. प्र॒जाना॒मिति॑ प्र - जाना᳚म् । \newline
77. सृष्ट्या॒ इति॒ सृष्ट्यै᳚ । \newline
\pagebreak
\markright{ TS 6.1.3.1  \hfill https://www.vedavms.in \hfill}

\section{ TS 6.1.3.1 }

\textbf{TS 6.1.3.1 } \newline
\textbf{Samhita Paata} \newline

ऋ॒ख् सा॒मे वै दे॒वेभ्यो॑ य॒ज्ञायाऽति॑ष्ठमाने॒ कृष्णो॑ रू॒पं कृ॒त्वा- ऽप॒क्रम्या॑तिष्ठतां॒ ते॑ऽमन्यन्त॒ यं ॅवा इ॒मे उ॑पाव॒र्थ्स्यतः॒ स इ॒दं भ॑विष्य॒तीति॒ ते उपा॑मन्त्रयन्त॒ ते अ॑होरा॒त्रयो᳚-र्महि॒मान॑-मपनि॒धाय॑ दे॒वानु॒पाव॑र्तेतामे॒ष वा ऋ॒चो वर्णो॒ यच्छु॒क्लं कृ॑ष्णाजि॒नस्यै॒ष साम्नो॒ यत् कृ॒ष्णमृ॑ख्सा॒मयोः॒ शिल्पे᳚ स्थ॒ इत्या॑हर्ख्सा॒मे ए॒वाव॑ रुन्ध ए॒ष - [  ] \newline

\textbf{Pada Paata} \newline

ऋ॒ख्सा॒मे इत्यृ॑क्-सा॒मे । वै । दे॒वेभ्यः॑ । य॒ज्ञाय॑ । अति॑ष्ठमाने॒ इति॑ । कृष्णः॑ । रू॒पम् । कृ॒त्वा । अ॒प॒क्रम्येत्य॑प -क्रम्य॑ । अ॒ति॒ष्ठ॒ता॒म् । ते । अ॒म॒न्य॒न्त॒ । यम् । वै । इ॒मे इति॑ । उ॒पा॒व॒र्थ्स्यत॒ इत्यु॑प-आ॒व॒र्थ्स्यतः॑ । सः । इ॒दम् । भ॒वि॒ष्य॒ति॒ । इति॑ । ते इति॑ । उपेति॑ । अ॒म॒न्त्र॒य॒न्त॒ । ते इति॑ । अ॒हो॒रा॒त्रयो॒रित्य॑हः - रा॒त्रयोः᳚ । म॒हि॒मान᳚म् । अ॒प॒नि॒धायेत्य॑प - नि॒धाय॑ । दे॒वान् । उ॒पाव॑र्तेता॒मित्यु॑प-आव॑र्तेताम् । ए॒षः । वै । ऋ॒चः । वर्णः॑ । यत् । शु॒क्लम् । कृ॒ष्णा॒जि॒नस्येति॑ कृष्ण - अ॒जि॒नस्य॑ । ए॒षः । साम्नः॑ । यत् । कृ॒ष्णम् । ऋ॒ख्सा॒मयो॒रित्यृ॑क् - सा॒मयोः᳚ । शिल्पे॒ इति॑ । स्थः॒ । इति॑ । आ॒ह॒ । ऋ॒ख्सा॒मे इत्यृ॑क् - सा॒मे । ए॒व । अवेति॑ । रु॒न्धे॒ । ए॒षः ।  \newline


\textbf{Krama Paata} \newline

ऋ॒ख्‌सा॒मे वै । ऋ॒ख्‌सा॒मे इत्यृ॑क् - सा॒मे । वै दे॒वेभ्यः॑ । दे॒वेभ्यो॑ य॒ज्ञाय॑ । य॒ज्ञायाति॑ष्ठमाने । अति॑ष्ठमाने॒ कृष्णः॑ । अति॑ष्ठमाने॒ इत्यति॑ष्ठमाने । कृष्णो॑ रू॒पम् । रू॒पम् कृ॒त्वा । कृ॒त्वाऽप॒क्रम्य॑ । अ॒प॒क्रम्या॑तिष्ठताम् । अ॒प॒क्रम्येत्य॑प - क्रम्य॑ । अ॒ति॒ष्ठ॒ता॒म् ते । ते॑ऽमन्यन्त । अ॒म॒न्य॒न्त॒ यम् । यम् ॅवै । वा इ॒मे । इ॒मे उ॑पाव॒र्थ्स्यतः॑ । इ॒मे इती॒मे । उ॒पा॒व॒र्थ्स्यतः॒ सः । उ॒पा॒व॒र्थ्स्यत॒ इत्यु॑प - आ॒व॒र्थ्स्यतः॑ । स इ॒दम् । इ॒दम् भ॑विष्यति । भ॒वि॒ष्य॒तीति॑ । इति॒ ते । ते उप॑ । ते इति॒ ते । उपा॑मन्त्रयन्त । अ॒म॒न्त्र॒य॒न्त॒ ते । ते अ॑होरा॒त्रयोः᳚ । ते इति॒ ते । अ॒हो॒रा॒त्रयो᳚र् महि॒मान᳚म् । अ॒हो॒रा॒त्रयो॒रित्य॑हः - रा॒त्रयोः᳚ । म॒हि॒मान॑मपनि॒धाय॑ । अ॒प॒नि॒धाय॑ दे॒वान् । अ॒प॒नि॒धायेत्य॑प - नि॒धाय॑ । दे॒वानु॒पाव॑र्तेताम् । उ॒पाव॑र्तेतामे॒षः । उ॒पाव॑र्तेता॒मित्यु॑प - आव॑र्तेताम् । ए॒ष वै । वा ऋ॒चः । ऋ॒चो वर्णः॑ । वर्णो॒ यत् । यच्छु॒क्लम् । शु॒क्लम् कृ॑ष्णाजि॒नस्य॑ । कृ॒ष्णा॒जि॒नस्यै॒षः । कृ॒ष्णा॒जि॒नस्येति॑ कृष्ण - अ॒जि॒नस्य॑ । ए॒ष साम्नः॑ । साम्नो॒ यत् । यत् कृ॒ष्णम् । कृ॒ष्णमृ॑ख्‌सा॒मयोः᳚ । ऋ॒ख्‌सा॒मयोः॒ शिल्पे᳚ । ऋ॒ख्‌सा॒मयो॒रित्यृ॑क् - सा॒मयोः᳚ । शिल्पे᳚ स्थः । शिल्पे॒ इति॒ शिल्पे᳚ । स्थ॒ इति॑ । इत्या॑ह । आ॒ह॒र्ख्‌सा॒मे । ऋ॒ख्‌सा॒मे ए॒व । ऋ॒ख्‌सा॒मे इत्यृ॑क् - सा॒मे । ए॒वाव॑ । अव॑ रुन्धे । रु॒न्ध॒ ए॒षः । ए॒ष वै \newline

\textbf{Jatai Paata} \newline

1. ऋ॒ख्सा॒मे वै वा ऋ॑ख्सा॒मे ऋ॑ख्सा॒मे वै । \newline
2. ऋ॒ख्सा॒मे इत्यृ॑क् - सा॒मे । \newline
3. वै दे॒वेभ्यो॑ दे॒वेभ्यो॒ वै वै दे॒वेभ्यः॑ । \newline
4. दे॒वेभ्यो॑ य॒ज्ञाय॑ य॒ज्ञाय॑ दे॒वेभ्यो॑ दे॒वेभ्यो॑ य॒ज्ञाय॑ । \newline
5. य॒ज्ञाया ति॑ष्ठमाने॒ अति॑ष्ठमाने य॒ज्ञाय॑ य॒ज्ञाया ति॑ष्ठमाने । \newline
6. अति॑ष्ठमाने॒ कृष्णः॒ कृष्णो ऽति॑ष्ठमाने॒ अति॑ष्ठमाने॒ कृष्णः॑ । \newline
7. अति॑ष्ठमाने॒ इत्यति॑ष्ठमाने । \newline
8. कृष्णो॑ रू॒पꣳ रू॒पम् कृष्णः॒ कृष्णो॑ रू॒पम् । \newline
9. रू॒पम् कृ॒त्वा कृ॒त्वा रू॒पꣳ रू॒पम् कृ॒त्वा । \newline
10. कृ॒त्वा ऽप॒क्रम्या॑ प॒क्रम्य॑ कृ॒त्वा कृ॒त्वा ऽप॒क्रम्य॑ । \newline
11. अ॒प॒क्रम्या॑ तिष्ठता मतिष्ठता मप॒क्रम्या॑ प॒क्रम्या॑ तिष्ठताम् । \newline
12. अ॒प॒क्रम्येत्य॑प - क्रम्य॑ । \newline
13. अ॒ति॒ष्ठ॒ता॒म् ते ते॑ ऽतिष्ठता मतिष्ठता॒म् ते । \newline
14. ते॑ ऽमन्यन्ता मन्यन्त॒ ते ते॑ ऽमन्यन्त । \newline
15. अ॒म॒न्य॒न्त॒ यं ॅय म॑मन्यन्ता मन्यन्त॒ यम् । \newline
16. यं ॅवै वै यं ॅयं ॅवै । \newline
17. वा इ॒मे इ॒मे वै वा इ॒मे । \newline
18. इ॒मे उ॑पाव॒र्थ्स्यत॑ उपाव॒र्थ्स्यत॑ इ॒मे इ॒मे उ॑पाव॒र्थ्स्यतः॑ । \newline
19. इ॒मे इती॒मे । \newline
20. उ॒पा॒व॒र्थ्स्यतः॒ स स उ॑पाव॒र्थ्स्यत॑ उपाव॒र्थ्स्यतः॒ सः । \newline
21. उ॒पा॒व॒र्थ्स्यत॒ इत्यु॑प - आ॒व॒र्थ्स्यतः॑ । \newline
22. स इ॒द मि॒दꣳ स स इ॒दम् । \newline
23. इ॒दम् भ॑विष्यति भविष्यती॒द मि॒दम् भ॑विष्यति । \newline
24. भ॒वि॒ष्य॒ती तीति॑ भविष्यति भविष्य॒तीति॑ । \newline
25. इति॒ ते ते इतीति॒ ते । \newline
26. ते उपोप॒ ते ते उप॑ । \newline
27. ते इति॒ ते । \newline
28. उपा॑मन्त्रयन्ता मन्त्रय॒न्तो पोपा॑ मन्त्रयन्त । \newline
29. अ॒म॒न्त्र॒य॒न्त॒ ते ते अ॑मन्त्रयन्ता मन्त्रयन्त॒ ते । \newline
30. ते अ॑होरा॒त्रयो॑ रहोरा॒त्रयो॒ स्ते ते अ॑होरा॒त्रयोः᳚ । \newline
31. ते इति॒ ते । \newline
32. अ॒हो॒रा॒त्रयो᳚र् महि॒मान॑म् महि॒मान॑ महोरा॒त्रयो॑ रहोरा॒त्रयो᳚र् महि॒मान᳚म् । \newline
33. अ॒हो॒रा॒त्रयो॒रित्य॑हः - रा॒त्रयोः᳚ । \newline
34. म॒हि॒मान॑ मपनि॒धाया॑ पनि॒धाय॑ महि॒मान॑म् महि॒मान॑ मपनि॒धाय॑ । \newline
35. अ॒प॒नि॒धाय॑ दे॒वान् दे॒वान॑ पनि॒धाया॑ पनि॒धाय॑ दे॒वान् । \newline
36. अ॒प॒नि॒धायेत्य॑प - नि॒धाय॑ । \newline
37. दे॒वानु॒ पाव॑र्तेता मु॒पाव॑र्तेताम् दे॒वान् दे॒वानु॒ पाव॑र्तेताम् । \newline
38. उ॒पाव॑र्तेता मे॒ष ए॒ष उ॒पाव॑र्तेता मु॒पाव॑र्तेता मे॒षः । \newline
39. उ॒पाव॑र्तेता॒मित्यु॑प - आव॑र्तेताम् । \newline
40. ए॒ष वै वा ए॒ष ए॒ष वै । \newline
41. वा ऋ॒च ऋ॒चो वै वा ऋ॒चः । \newline
42. ऋ॒चो वर्णो॒ वर्ण॑ ऋ॒च ऋ॒चो वर्णः॑ । \newline
43. वर्णो॒ यद् यद् वर्णो॒ वर्णो॒ यत् । \newline
44. यच्छु॒क्लꣳ शु॒क्लं ॅयद् यच्छु॒क्लम् । \newline
45. शु॒क्लम् कृ॑ष्णाजि॒नस्य॑ कृष्णाजि॒नस्य॑ शु॒क्लꣳ शु॒क्लम् कृ॑ष्णाजि॒नस्य॑ । \newline
46. कृ॒ष्णा॒जि॒न स्यै॒ष ए॒ष कृ॑ष्णाजि॒नस्य॑ कृष्णाजि॒न स्यै॒षः । \newline
47. कृ॒ष्णा॒जि॒नस्येति॑ कृष्ण - अ॒जि॒नस्य॑ । \newline
48. ए॒ष साम्नः॒ साम्न॑ ए॒ष ए॒ष साम्नः॑ । \newline
49. साम्नो॒ यद् यथ् साम्नः॒ साम्नो॒ यत् । \newline
50. यत् कृ॒ष्णम् कृ॒ष्णं ॅयद् यत् कृ॒ष्णम् । \newline
51. कृ॒ष्ण मृ॑ख्सा॒मयोर्॑. ऋख्सा॒मयोः᳚ कृ॒ष्णम् कृ॒ष्ण मृ॑ख्सा॒मयोः᳚ । \newline
52. ऋ॒ख्सा॒मयोः॒ शिल्पे॒ शिल्पे॑ ऋख्सा॒मयोर्॑. ऋख्सा॒मयोः॒ शिल्पे᳚ । \newline
53. ऋ॒ख्सा॒मयो॒रित्यृ॑क् - सा॒मयोः᳚ । \newline
54. शिल्पे᳚ स्थः स्थः॒ शिल्पे॒ शिल्पे᳚ स्थः । \newline
55. शिल्पे॒ इति॒ शिल्पे᳚ । \newline
56. स्थ॒ इतीति॑ स्थः स्थ॒ इति॑ । \newline
57. इत्या॑हा॒हे तीत्या॑ह । \newline
58. आ॒ह॒ र्‌ख्सा॒मे ऋ॑ख्सा॒मे आ॑हाह र्‌ख्सा॒मे । \newline
59. ऋ॒ख्सा॒मे ए॒वैव र्‌ख्सा॒मे ऋ॑ख्सा॒मे ए॒व । \newline
60. ऋ॒ख्सा॒मे इत्यृ॑क् - सा॒मे । \newline
61. ए॒वावा वै॒वै वाव॑ । \newline
62. अव॑ रुन्धे रु॒न्धे ऽवाव॑ रुन्धे । \newline
63. रु॒न्ध॒ ए॒ष ए॒ष रु॑न्धे रुन्ध ए॒षः । \newline
64. ए॒ष वै वा ए॒ष ए॒ष वै । \newline

\textbf{Ghana Paata } \newline

1. ऋ॒ख्सा॒मे वै वा ऋ॑ख्सा॒मे ऋ॑ख्सा॒मे वै दे॒वेभ्यो॑ दे॒वेभ्यो॒ वा ऋ॑ख्सा॒मे ऋ॑ख्सा॒मे वै दे॒वेभ्यः॑ । \newline
2. ऋ॒ख्सा॒मे इत्यृ॑क् - सा॒मे । \newline
3. वै दे॒वेभ्यो॑ दे॒वेभ्यो॒ वै वै दे॒वेभ्यो॑ य॒ज्ञाय॑ य॒ज्ञाय॑ दे॒वेभ्यो॒ वै वै दे॒वेभ्यो॑ य॒ज्ञाय॑ । \newline
4. दे॒वेभ्यो॑ य॒ज्ञाय॑ य॒ज्ञाय॑ दे॒वेभ्यो॑ दे॒वेभ्यो॑ य॒ज्ञाया ति॑ष्ठमाने॒ अति॑ष्ठमाने य॒ज्ञाय॑ दे॒वेभ्यो॑ दे॒वेभ्यो॑ य॒ज्ञाया ति॑ष्ठमाने । \newline
5. य॒ज्ञाया ति॑ष्ठमाने॒ अति॑ष्ठमाने य॒ज्ञाय॑ य॒ज्ञाया ति॑ष्ठमाने॒ कृष्णः॒ कृष्णो ऽति॑ष्ठमाने य॒ज्ञाय॑ य॒ज्ञाया ति॑ष्ठमाने॒ कृष्णः॑ । \newline
6. अति॑ष्ठमाने॒ कृष्णः॒ कृष्णो ऽति॑ष्ठमाने॒ अति॑ष्ठमाने॒ कृष्णो॑ रू॒पꣳ रू॒पम् कृष्णो ऽति॑ष्ठमाने॒ अति॑ष्ठमाने॒ कृष्णो॑ रू॒पम् । \newline
7. अति॑ष्ठमाने॒ इत्यति॑ष्ठमाने । \newline
8. कृष्णो॑ रू॒पꣳ रू॒पम् कृष्णः॒ कृष्णो॑ रू॒पम् कृ॒त्वा कृ॒त्वा रू॒पम् कृष्णः॒ कृष्णो॑ रू॒पम् कृ॒त्वा । \newline
9. रू॒पम् कृ॒त्वा कृ॒त्वा रू॒पꣳ रू॒पम् कृ॒त्वा ऽप॒क्रम्या॑ प॒क्रम्य॑ कृ॒त्वा रू॒पꣳ रू॒पम् कृ॒त्वा ऽप॒क्रम्य॑ । \newline
10. कृ॒त्वा ऽप॒क्रम्या॑ प॒क्रम्य॑ कृ॒त्वा कृ॒त्वा ऽप॒क्रम्या॑ तिष्ठता मतिष्ठता मप॒क्रम्य॑ कृ॒त्वा कृ॒त्वा ऽप॒क्रम्या॑ तिष्ठताम् । \newline
11. अ॒प॒क्रम्या॑ तिष्ठता मतिष्ठता मप॒क्रम्या॑ प॒क्रम्या॑ तिष्ठता॒म् ते ते॑ ऽतिष्ठता मप॒क्रम्या॑ प॒क्रम्या॑ तिष्ठता॒म् ते । \newline
12. अ॒प॒क्रम्येत्य॑प - क्रम्य॑ । \newline
13. अ॒ति॒ष्ठ॒ता॒म् ते ते॑ ऽतिष्ठता मतिष्ठता॒म् ते॑ ऽमन्यन्ता मन्यन्त॒ ते॑ ऽतिष्ठता मतिष्ठता॒म् ते॑ ऽमन्यन्त । \newline
14. ते॑ ऽमन्यन्ता मन्यन्त॒ ते ते॑ ऽमन्यन्त॒ यं ॅय म॑मन्यन्त॒ ते ते॑ ऽमन्यन्त॒ यम् । \newline
15. अ॒म॒न्य॒न्त॒ यं ॅय म॑मन्यन्ता मन्यन्त॒ यं ॅवै वै य म॑मन्यन्ता मन्यन्त॒ यं ॅवै । \newline
16. यं ॅवै वै यं ॅयं ॅवा इ॒मे इ॒मे वै यं ॅयं ॅवा इ॒मे । \newline
17. वा इ॒मे इ॒मे वै वा इ॒मे उ॑पाव॒र्थ्स्यत॑ उपाव॒र्थ्स्यत॑ इ॒मे वै वा इ॒मे उ॑पाव॒र्थ्स्यतः॑ । \newline
18. इ॒मे उ॑पाव॒र्थ्स्यत॑ उपाव॒र्थ्स्यत॑ इ॒मे इ॒मे उ॑पाव॒र्थ्स्यतः॒ स स उ॑पाव॒र्थ्स्यत॑ इ॒मे इ॒मे उ॑पाव॒र्थ्स्यतः॒ सः । \newline
19. इ॒मे इती॒मे । \newline
20. उ॒पा॒व॒र्थ्स्यतः॒ स स उ॑पाव॒र्थ्स्यत॑ उपाव॒र्थ्स्यतः॒ स इ॒द मि॒दꣳ स उ॑पाव॒र्थ्स्यत॑ उपाव॒र्थ्स्यतः॒ स इ॒दम् । \newline
21. उ॒पा॒व॒र्थ्स्यत॒ इत्यु॑प - आ॒व॒र्थ्स्यतः॑ । \newline
22. स इ॒द मि॒दꣳ स स इ॒दम् भ॑विष्यति भविष्यती॒दꣳ स स इ॒दम् भ॑विष्यति । \newline
23. इ॒दम् भ॑विष्यति भविष्यती॒द मि॒दम् भ॑विष्य॒ती तीति॑ भविष्यती॒द मि॒दम् भ॑विष्य॒तीति॑ । \newline
24. भ॒वि॒ष्य॒ती तीति॑ भविष्यति भविष्य॒तीति॒ ते ते इति॑ भविष्यति भविष्य॒तीति॒ ते । \newline
25. इति॒ ते ते इतीति॒ ते उपोप॒ ते इतीति॒ ते उप॑ । \newline
26. ते उपोप॒ ते ते उपा॑मन्त्रयन्ता मन्त्रय॒न्तोप॒ ते ते उपा॑मन्त्रयन्त । \newline
27. ते इति॒ ते । \newline
28. उपा॑मन्त्रयन्ता मन्त्रय॒न्तोपोपा॑ मन्त्रयन्त॒ ते ते अ॑मन्त्रय॒न्तोपोपा॑ मन्त्रयन्त॒ ते । \newline
29. अ॒म॒न्त्र॒य॒न्त॒ ते ते अ॑मन्त्रयन्ता मन्त्रयन्त॒ ते अ॑होरा॒त्रयो॑ रहोरा॒त्रयो॒ स्ते अ॑मन्त्रयन्ता मन्त्रयन्त॒ ते अ॑होरा॒त्रयोः᳚ । \newline
30. ते अ॑होरा॒त्रयो॑ रहोरा॒त्रयो॒ स्ते ते अ॑होरा॒त्रयो᳚र् महि॒मान॑म् महि॒मान॑ महोरा॒त्रयो॒ स्ते ते अ॑होरा॒त्रयो᳚र् महि॒मान᳚म् । \newline
31. ते इति॒ ते । \newline
32. अ॒हो॒रा॒त्रयो᳚र् महि॒मान॑म् महि॒मान॑ महोरा॒त्रयो॑ रहोरा॒त्रयो᳚र् महि॒मान॑ मपनि॒धाया॑ पनि॒धाय॑ महि॒मान॑ महोरा॒त्रयो॑ रहोरा॒त्रयो᳚र् महि॒मान॑ मपनि॒धाय॑ । \newline
33. अ॒हो॒रा॒त्रयो॒रित्य॑हः - रा॒त्रयोः᳚ । \newline
34. म॒हि॒मान॑ मपनि॒धाया॑ पनि॒धाय॑ महि॒मान॑म् महि॒मान॑ मपनि॒धाय॑ दे॒वान् दे॒वा न॑पनि॒धाय॑ महि॒मान॑म् महि॒मान॑ मपनि॒धाय॑ दे॒वान् । \newline
35. अ॒प॒नि॒धाय॑ दे॒वान् दे॒वा न॑पनि॒धाया॑ पनि॒धाय॑ दे॒वा नु॒पाव॑र्तेता मु॒पाव॑र्तेताम् दे॒वा न॑पनि॒धाया॑ पनि॒धाय॑ दे॒वा नु॒पाव॑र्तेताम् । \newline
36. अ॒प॒नि॒धायेत्य॑प - नि॒धाय॑ । \newline
37. दे॒वा नु॒पाव॑र्तेता मु॒पाव॑र्तेताम् दे॒वान् दे॒वा नु॒पाव॑र्तेता मे॒ष ए॒ष उ॒पाव॑र्तेताम् दे॒वान् दे॒वा नु॒पाव॑र्तेता मे॒षः । \newline
38. उ॒पाव॑र्तेता मे॒ष ए॒ष उ॒पाव॑र्तेता मु॒पाव॑र्तेता मे॒ष वै वा ए॒ष उ॒पाव॑र्तेता मु॒पाव॑र्तेता मे॒ष वै । \newline
39. उ॒पाव॑र्तेता॒मित्यु॑प - आव॑र्तेताम् । \newline
40. ए॒ष वै वा ए॒ष ए॒ष वा ऋ॒च ऋ॒चो वा ए॒ष ए॒ष वा ऋ॒चः । \newline
41. वा ऋ॒च ऋ॒चो वै वा ऋ॒चो वर्णो॒ वर्ण॑ ऋ॒चो वै वा ऋ॒चो वर्णः॑ । \newline
42. ऋ॒चो वर्णो॒ वर्ण॑ ऋ॒च ऋ॒चो वर्णो॒ यद् यद् वर्ण॑ ऋ॒च ऋ॒चो वर्णो॒ यत् । \newline
43. वर्णो॒ यद् यद् वर्णो॒ वर्णो॒ यच् छु॒क्लꣳ शु॒क्लं ॅयद् वर्णो॒ वर्णो॒ यच् छु॒क्लम् । \newline
44. यच् छु॒क्लꣳ शु॒क्लं ॅयद् यच् छु॒क्लम् कृ॑ष्णाजि॒नस्य॑ कृष्णाजि॒नस्य॑ शु॒क्लं ॅयद् यच् छु॒क्लम् कृ॑ष्णाजि॒नस्य॑ । \newline
45. शु॒क्लम् कृ॑ष्णाजि॒नस्य॑ कृष्णाजि॒नस्य॑ शु॒क्लꣳ शु॒क्लम् कृ॑ष्णाजि॒न-स्यै॒ष ए॒ष कृ॑ष्णाजि॒नस्य॑ शु॒क्लꣳ शु॒क्लम् कृ॑ष्णाजि॒न स्यै॒षः । \newline
46. कृ॒ष्णा॒जि॒न स्यै॒ष ए॒ष कृ॑ष्णाजि॒नस्य॑ कृष्णाजि॒न स्यै॒ष साम्नः॒ साम्न॑ ए॒ष कृ॑ष्णाजि॒नस्य॑ कृष्णाजि॒न स्यै॒ष साम्नः॑ । \newline
47. कृ॒ष्णा॒जि॒नस्येति॑ कृष्ण - अ॒जि॒नस्य॑ । \newline
48. ए॒ष साम्नः॒ साम्न॑ ए॒ष ए॒ष साम्नो॒ यद् यथ् साम्न॑ ए॒ष ए॒ष साम्नो॒ यत् । \newline
49. साम्नो॒ यद् यथ् साम्नः॒ साम्नो॒ यत् कृ॒ष्णम् कृ॒ष्णं ॅयथ् साम्नः॒ साम्नो॒ यत् कृ॒ष्णम् । \newline
50. यत् कृ॒ष्णम् कृ॒ष्णं ॅयद् यत् कृ॒ष्ण मृ॑ख्सा॒मयोर्॑. ऋख्सा॒मयोः᳚ कृ॒ष्णं ॅयद् यत् कृ॒ष्ण मृ॑ख्सा॒मयोः᳚ । \newline
51. कृ॒ष्ण मृ॑ख्सा॒मयोर्॑. ऋख्सा॒मयोः᳚ कृ॒ष्णम् कृ॒ष्ण मृ॑ख्सा॒मयोः॒ शिल्पे॒ शिल्पे॑ ऋख्सा॒मयोः᳚ कृ॒ष्णम् कृ॒ष्ण मृ॑ख्सा॒मयोः॒ शिल्पे᳚ । \newline
52. ऋ॒ख्सा॒मयोः॒ शिल्पे॒ शिल्पे॑ ऋख्सा॒मयोर्॑. ऋख्सा॒मयोः॒ शिल्पे᳚ स्थः स्थः॒ शिल्पे॑ ऋख्सा॒मयोर्॑. ऋख्सा॒मयोः॒ शिल्पे᳚ स्थः । \newline
53. ऋ॒ख्सा॒मयो॒रित्यृ॑क् - सा॒मयोः᳚ । \newline
54. शिल्पे᳚ स्थः स्थः॒ शिल्पे॒ शिल्पे᳚ स्थ॒ इतीति॑ स्थः॒ शिल्पे॒ शिल्पे᳚ स्थ॒ इति॑ । \newline
55. शिल्पे॒ इति॒ शिल्पे᳚ । \newline
56. स्थ॒ इतीति॑ स्थः स्थ॒ इत्या॑हा॒हेति॑ स्थः स्थ॒ इत्या॑ह । \newline
57. इत्या॑हा॒हे तीत्या॑ह र्‌ख्सा॒मे ऋ॑ख्सा॒मे आ॒हे तीत्या॑ह र्‌ख्सा॒मे । \newline
58. आ॒ह॒ र्‌ख्सा॒मे ऋ॑ख्सा॒मे आ॑हाह र्‌ख्सा॒मे ए॒वैव र्‌‍ख्सा॒मे आ॑हाह र्‌ख्सा॒मे ए॒व । \newline
59. ऋ॒ख्सा॒मे ए॒वैव र्‌ख्सा॒मे ऋ॑ख्सा॒मे ए॒वावा वै॒व र्‌ख्सा॒मे ऋ॑ख्सा॒मे ए॒वाव॑ । \newline
60. ऋ॒ख्सा॒मे इत्यृ॑क् - सा॒मे । \newline
61. ए॒वावा वै॒वै वाव॑ रुन्धे रु॒न्धे ऽवै॒वै वाव॑ रुन्धे । \newline
62. अव॑ रुन्धे रु॒न्धे ऽवाव॑ रुन्ध ए॒ष ए॒ष रु॒न्धे ऽवाव॑ रुन्ध ए॒षः । \newline
63. रु॒न्ध॒ ए॒ष ए॒ष रु॑न्धे रुन्ध ए॒ष वै वा ए॒ष रु॑न्धे रुन्ध ए॒ष वै । \newline
64. ए॒ष वै वा ए॒ष ए॒ष वा अह्नो ऽह्नो॒ वा ए॒ष ए॒ष वा अह्नः॑ । \newline
\pagebreak
\markright{ TS 6.1.3.2  \hfill https://www.vedavms.in \hfill}

\section{ TS 6.1.3.2 }

\textbf{TS 6.1.3.2 } \newline
\textbf{Samhita Paata} \newline

वा अह्नो॒ वर्णो॒ यच्छु॒क्लं कृ॑ष्णाजि॒नस्यै॒ष रात्रि॑या॒ यत् कृ॒ष्णं ॅयदे॒वैन॑यो॒स्तत्र॒ न्य॑क्तं॒ तदे॒वाव॑ रुन्धे कृष्णाजि॒नेन॑ दीक्षयति॒ ब्रह्म॑णो॒ वा ए॒तद्-रू॒पं ॅयत् कृ॑ष्णाजि॒नं ब्रह्म॑णै॒वैनं॑ दीक्षयती॒मां धियꣳ॒॒ शिक्ष॑माणस्य दे॒वेत्या॑ह यथाय॒जुरे॒वैतद्-गर्भो॒ वा ए॒ष यद्-दी᳚क्षि॒त उल्बं॒ ॅवासः॒ प्रोर्णु॑ते॒ तस्मा॒द् - [  ] \newline

\textbf{Pada Paata} \newline

वै । अह्नः॑ । वर्णः॑ । यत् । शु॒क्लम् । कृ॒ष्णा॒जि॒नस्येति॑ कृष्ण - अ॒जि॒नस्य॑ । ए॒षः । रात्रि॑याः । यत् । कृ॒ष्णम् । यत् । ए॒व । ए॒न॒योः॒ । तत्र॑ । न्य॑क्त॒मिति॒ नि-अ॒क्त॒म् । तत् । ए॒व । अवेति॑ । रु॒न्धे॒ । कृ॒ष्णा॒जि॒नेनेति॑ कृष्ण-अ॒जि॒नेन॑ । दी॒क्ष॒य॒ति॒ । ब्रह्म॑णः । वै । ए॒तत् । रू॒पम् । यत् । कृ॒ष्णा॒जि॒नमिति॑ कृष्ण - अ॒जि॒नम् । ब्रह्म॑णा । ए॒व । ए॒न॒म् । दी॒क्ष॒य॒ति॒ । इ॒माम् । धिय᳚म् । शिक्ष॑माणस्य । दे॒व॒ । इति॑ । आ॒ह॒ । य॒था॒य॒जुरिति॑ यथा - य॒जुः । ए॒व । ए॒तत् । गर्भः॑ । वै । ए॒षः । यत् । दी॒क्षि॒तः । उल्ब᳚म् । वासः॑ । प्रेति॑ । ऊ॒र्णु॒ते॒ । तस्मा᳚त् ।  \newline


\textbf{Krama Paata} \newline

वा अह्नः॑ । अह्नो॒ वर्णः॑ । वर्णो॒ यत् । यच्छु॒क्लम् । शु॒क्लम् कृ॑ष्णाजि॒नस्य॑ । कृ॒ष्णा॒जि॒नस्यै॒षः । कृ॒ष्णा॒जि॒नस्येति॑ कृष्ण - अ॒जि॒नस्य॑ । ए॒ष रात्रि॑याः । रात्रि॑या॒ यत् । यत् कृ॒ष्णम् । कृ॒ष्णम् ॅयत् । यदे॒व । ए॒वैन॑योः । एन॑यो॒स्तत्र॑ । तत्र॒ न्य॑क्तम् । न्य॑क्त॒म् तत् । न्य॑क्त॒मिति॒ नि - अ॒क्त॒म् । तदे॒व । ए॒वाव॑ । अव॑ रुन्धे । रु॒न्धे॒ कृ॒ष्णा॒जि॒नेन॑ । कृ॒ष्णा॒जि॒नेन॑ दीक्षयति । कृ॒ष्णा॒जि॒नेनेति॑ कृष्ण - अ॒जि॒नेन॑ । दी॒क्ष॒य॒ति॒ ब्रह्म॑णः । ब्रह्म॑णो॒ वै । वा ए॒तत् । ए॒तद् रू॒पम् । रू॒पम् ॅयत् । यत् कृ॑ष्णाजि॒नम् । कृ॒ष्णा॒जि॒नम् ब्रह्म॑णा । कृ॒ष्णा॒जि॒नमिति॑ कृष्ण - अ॒जि॒नम् । ब्रह्म॑णै॒व । ए॒वैन᳚म् । ए॒न॒म् दी॒क्ष॒य॒ति॒ । दी॒क्ष॒य॒ती॒माम् । इ॒माम् धिय᳚म् । धियꣳ॒॒ शिक्ष॑माणस्य । शिक्ष॑माणस्य देव । दे॒वेति॑ । इत्या॑ह । आ॒ह॒ य॒था॒य॒जुः । य॒था॒य॒जुरे॒व । य॒था॒य॒जुरिति॑ यथा - य॒जुः । ए॒वैतत् । ए॒तद् गर्भः॑ । गर्भो॒ वै । वा ए॒षः । ए॒ष यत् । यद् दी᳚क्षि॒तः । दी॒क्षि॒त उल्ब᳚म् । उल्ब॒म् ॅवासः॑ । वासः॒ प्र । प्रोर्णु॑ते । ऊ॒र्णु॒ते॒ तस्मा᳚त् । तस्मा॒द् गर्भाः᳚ \newline

\textbf{Jatai Paata} \newline

1. वा अह्नो ऽह्नो॒ वै वा अह्नः॑ । \newline
2. अह्नो॒ वर्णो॒ वर्णो ऽह्नो ऽह्नो॒ वर्णः॑ । \newline
3. वर्णो॒ यद् यद् वर्णो॒ वर्णो॒ यत् । \newline
4. यच्छु॒क्लꣳ शु॒क्लं ॅयद् यच्छु॒क्लम् । \newline
5. शु॒क्लम् कृ॑ष्णाजि॒नस्य॑ कृष्णाजि॒नस्य॑ शु॒क्लꣳ शु॒क्लम् कृ॑ष्णाजि॒नस्य॑ । \newline
6. कृ॒ष्णा॒जि॒नस्यै॒ष ए॒ष कृ॑ष्णाजि॒नस्य॑ कृष्णाजि॒नस्यै॒षः । \newline
7. कृ॒ष्णा॒जि॒नस्येति॑ कृष्ण - अ॒जि॒नस्य॑ । \newline
8. ए॒ष रात्रि॑या॒ रात्रि॑या ए॒ष ए॒ष रात्रि॑याः । \newline
9. रात्रि॑या॒ यद् यद् रात्रि॑या॒ रात्रि॑या॒ यत् । \newline
10. यत् कृ॒ष्णम् कृ॒ष्णं ॅयद् यत् कृ॒ष्णम् । \newline
11. कृ॒ष्णं ॅयद् यत् कृ॒ष्णम् कृ॒ष्णं ॅयत् । \newline
12. यदे॒ वैव यद् यदे॒व । \newline
13. ए॒वैन॑यो रेनयो रे॒वै वैन॑योः । \newline
14. ए॒न॒यो॒ स्तत्र॒ तत्रै॑ नयो रेनयो॒ स्तत्र॑ । \newline
15. तत्र॒ न्य॑क्त॒म् न्य॑क्त॒म् तत्र॒ तत्र॒ न्य॑क्तम् । \newline
16. न्य॑क्त॒म् तत् तन् न्य॑क्त॒म् न्य॑क्त॒म् तत् । \newline
17. न्य॑क्त॒मिति॒ नि - अ॒क्त॒म् । \newline
18. तदे॒वैव तत् तदे॒व । \newline
19. ए॒वावा वै॒वै वाव॑ । \newline
20. अव॑ रुन्धे रु॒न्धे ऽवाव॑ रुन्धे । \newline
21. रु॒न्धे॒ कृ॒ष्णा॒जि॒नेन॑ कृष्णाजि॒नेन॑ रुन्धे रुन्धे कृष्णाजि॒नेन॑ । \newline
22. कृ॒ष्णा॒जि॒नेन॑ दीक्षयति दीक्षयति कृष्णाजि॒नेन॑ कृष्णाजि॒नेन॑ दीक्षयति । \newline
23. कृ॒ष्णा॒जि॒नेनेति॑ कृष्ण - अ॒जि॒नेन॑ । \newline
24. दी॒क्ष॒य॒ति॒ ब्रह्म॑णो॒ ब्रह्म॑णो दीक्षयति दीक्षयति॒ ब्रह्म॑णः । \newline
25. ब्रह्म॑णो॒ वै वै ब्रह्म॑णो॒ ब्रह्म॑णो॒ वै । \newline
26. वा ए॒त दे॒तद् वै वा ए॒तत् । \newline
27. ए॒तद् रू॒पꣳ रू॒प मे॒त दे॒तद् रू॒पम् । \newline
28. रू॒पं ॅयद् यद् रू॒पꣳ रू॒पं ॅयत् । \newline
29. यत् कृ॑ष्णाजि॒नम् कृ॑ष्णाजि॒नं ॅयद् यत् कृ॑ष्णाजि॒नम् । \newline
30. कृ॒ष्णा॒जि॒नम् ब्रह्म॑णा॒ ब्रह्म॑णा कृष्णाजि॒नम् कृ॑ष्णाजि॒नम् ब्रह्म॑णा । \newline
31. कृ॒ष्णा॒जि॒नमिति॑ कृष्ण - अ॒जि॒नम् । \newline
32. ब्रह्म॑ णै॒वैव ब्रह्म॑णा॒ ब्रह्म॑णै॒व । \newline
33. ए॒वैन॑ मेन मे॒वै वैन᳚म् । \newline
34. ए॒न॒म् दी॒क्ष॒य॒ति॒ दी॒क्ष॒य॒ त्ये॒न॒ मे॒न॒म् दी॒क्ष॒य॒ति॒ । \newline
35. दी॒क्ष॒य॒ती॒मा मि॒माम् दी᳚क्षयति दीक्षयती॒माम् । \newline
36. इ॒माम् धिय॒म् धिय॑ मि॒मा मि॒माम् धिय᳚म् । \newline
37. धियꣳ॒॒ शिक्ष॑माणस्य॒ शिक्ष॑माणस्य॒ धिय॒म् धियꣳ॒॒ शिक्ष॑माणस्य । \newline
38. शिक्ष॑माणस्य देव देव॒ शिक्ष॑माणस्य॒ शिक्ष॑माणस्य देव । \newline
39. दे॒वे तीति॑ देव दे॒वेति॑ । \newline
40. इत्या॑हा॒हे तीत्या॑ह । \newline
41. आ॒ह॒ य॒था॒य॒जुर् य॑थाय॒जु रा॑हाह यथाय॒जुः । \newline
42. य॒था॒य॒जु रे॒वैव य॑थाय॒जुर् य॑थाय॒जु रे॒व । \newline
43. य॒था॒य॒जुरिति॑ यथा - य॒जुः । \newline
44. ए॒वैत दे॒त दे॒वै वैतत् । \newline
45. ए॒तद् गर्भो॒ गर्भ॑ ए॒त दे॒तद् गर्भः॑ । \newline
46. गर्भो॒ वै वै गर्भो॒ गर्भो॒ वै । \newline
47. वा ए॒ष ए॒ष वै वा ए॒षः । \newline
48. ए॒ष यद् यदे॒ष ए॒ष यत् । \newline
49. यद् दी᳚क्षि॒तो दी᳚क्षि॒तो यद् यद् दी᳚क्षि॒तः । \newline
50. दी॒क्षि॒त उल्ब॒ मुल्ब॑म् दीक्षि॒तो दी᳚क्षि॒त उल्ब᳚म् । \newline
51. उल्बं॒ ॅवासो॒ वास॒ उल्ब॒ मुल्बं॒ ॅवासः॑ । \newline
52. वासः॒ प्र प्र वासो॒ वासः॒ प्र । \newline
53. प्रोर्णु॑त ऊर्णुते॒ प्र प्रोर्णु॑ते । \newline
54. ऊ॒र्णु॒ते॒ तस्मा॒त् तस्मा॑ दूर्णुत ऊर्णुते॒ तस्मा᳚त् । \newline
55. तस्मा॒द् गर्भा॒ गर्भा॒ स्तस्मा॒त् तस्मा॒द् गर्भाः᳚ । \newline

\textbf{Ghana Paata } \newline

1. वा अह्नो ऽह्नो॒ वै वा अह्नो॒ वर्णो॒ वर्णो ऽह्नो॒ वै वा अह्नो॒ वर्णः॑ । \newline
2. अह्नो॒ वर्णो॒ वर्णो ऽह्नो ऽह्नो॒ वर्णो॒ यद् यद् वर्णो ऽह्नो ऽह्नो॒ वर्णो॒ यत् । \newline
3. वर्णो॒ यद् यद् वर्णो॒ वर्णो॒ यच् छु॒क्लꣳ शु॒क्लं ॅयद् वर्णो॒ वर्णो॒ यच् छु॒क्लम् । \newline
4. यच् छु॒क्लꣳ शु॒क्लं ॅयद् यच् छु॒क्लम् कृ॑ष्णाजि॒नस्य॑ कृष्णाजि॒नस्य॑ शु॒क्लं ॅयद् यच् छु॒क्लम् कृ॑ष्णाजि॒नस्य॑ । \newline
5. शु॒क्लम् कृ॑ष्णाजि॒नस्य॑ कृष्णाजि॒नस्य॑ शु॒क्लꣳ शु॒क्लम् कृ॑ष्णाजि॒न स्यै॒ष ए॒ष कृ॑ष्णाजि॒नस्य॑ शु॒क्लꣳ शु॒क्लम् कृ॑ष्णाजि॒न स्यै॒षः । \newline
6. कृ॒ष्णा॒जि॒न स्यै॒ष ए॒ष कृ॑ष्णाजि॒नस्य॑ कृष्णाजि॒न स्यै॒ष रात्रि॑या॒ रात्रि॑या ए॒ष कृ॑ष्णाजि॒नस्य॑ कृष्णाजि॒न स्यै॒ष रात्रि॑याः । \newline
7. कृ॒ष्णा॒जि॒नस्येति॑ कृष्ण - अ॒जि॒नस्य॑ । \newline
8. ए॒ष रात्रि॑या॒ रात्रि॑या ए॒ष ए॒ष रात्रि॑या॒ यद् यद् रात्रि॑या ए॒ष ए॒ष रात्रि॑या॒ यत् । \newline
9. रात्रि॑या॒ यद् यद् रात्रि॑या॒ रात्रि॑या॒ यत् कृ॒ष्णम् कृ॒ष्णं ॅयद् रात्रि॑या॒ रात्रि॑या॒ यत् कृ॒ष्णम् । \newline
10. यत् कृ॒ष्णम् कृ॒ष्णं ॅयद् यत् कृ॒ष्णं ॅयद् यत् कृ॒ष्णं ॅयद् यत् कृ॒ष्णं ॅयत् । \newline
11. कृ॒ष्णं ॅयद् यत् कृ॒ष्णम् कृ॒ष्णं ॅयदे॒ वैव यत् कृ॒ष्णम् कृ॒ष्णं ॅयदे॒व । \newline
12. यदे॒ वैव यद् यदे॒ वैन॑यो रेनयो रे॒व यद् यदे॒वैन॑योः । \newline
13. ए॒वैन॑यो रेनयो रे॒वै वैन॑यो॒ स्तत्र॒ तत्रै॑ नयो रे॒वै वैन॑यो॒ स्तत्र॑ । \newline
14. ए॒न॒यो॒ स्तत्र॒ तत्रै॑नयो रेनयो॒ स्तत्र॒ न्य॑क्त॒म् न्य॑क्त॒म् तत्रै॑नयो रेनयो॒ स्तत्र॒ न्य॑क्तम् । \newline
15. तत्र॒ न्य॑क्त॒म् न्य॑क्त॒म् तत्र॒ तत्र॒ न्य॑क्त॒म् तत् तन् न्य॑क्त॒म् तत्र॒ तत्र॒ न्य॑क्त॒म् तत् । \newline
16. न्य॑क्त॒म् तत् तन् न्य॑क्त॒म् न्य॑क्त॒म् तदे॒ वैव तन् न्य॑क्त॒म् न्य॑क्त॒म् तदे॒व । \newline
17. न्य॑क्त॒मिति॒ नि - अ॒क्त॒म् । \newline
18. तदे॒ वैव तत् तदे॒ वावा वै॒व तत् तदे॒ वाव॑ । \newline
19. ए॒वावा वै॒वै वाव॑ रुन्धे रु॒न्धे ऽवै॒वै वाव॑ रुन्धे । \newline
20. अव॑ रुन्धे रु॒न्धे ऽवाव॑ रुन्धे कृष्णाजि॒नेन॑ कृष्णाजि॒नेन॑ रु॒न्धे ऽवाव॑ रुन्धे कृष्णाजि॒नेन॑ । \newline
21. रु॒न्धे॒ कृ॒ष्णा॒जि॒नेन॑ कृष्णाजि॒नेन॑ रुन्धे रुन्धे कृष्णाजि॒नेन॑ दीक्षयति दीक्षयति कृष्णाजि॒नेन॑ रुन्धे रुन्धे कृष्णाजि॒नेन॑ दीक्षयति । \newline
22. कृ॒ष्णा॒जि॒नेन॑ दीक्षयति दीक्षयति कृष्णाजि॒नेन॑ कृष्णाजि॒नेन॑ दीक्षयति॒ ब्रह्म॑णो॒ ब्रह्म॑णो दीक्षयति कृष्णाजि॒नेन॑ कृष्णाजि॒नेन॑ दीक्षयति॒ ब्रह्म॑णः । \newline
23. कृ॒ष्णा॒जि॒नेनेति॑ कृष्ण - अ॒जि॒नेन॑ । \newline
24. दी॒क्ष॒य॒ति॒ ब्रह्म॑णो॒ ब्रह्म॑णो दीक्षयति दीक्षयति॒ ब्रह्म॑णो॒ वै वै ब्रह्म॑णो दीक्षयति दीक्षयति॒ ब्रह्म॑णो॒ वै । \newline
25. ब्रह्म॑णो॒ वै वै ब्रह्म॑णो॒ ब्रह्म॑णो॒ वा ए॒त दे॒तद् वै ब्रह्म॑णो॒ ब्रह्म॑णो॒ वा ए॒तत् । \newline
26. वा ए॒त दे॒तद् वै वा ए॒तद् रू॒पꣳ रू॒प मे॒तद् वै वा ए॒तद् रू॒पम् । \newline
27. ए॒तद् रू॒पꣳ रू॒प मे॒त दे॒तद् रू॒पं ॅयद् यद् रू॒प मे॒त दे॒तद् रू॒पं ॅयत् । \newline
28. रू॒पं ॅयद् यद् रू॒पꣳ रू॒पं ॅयत् कृ॑ष्णाजि॒नम् कृ॑ष्णाजि॒नं ॅयद् रू॒पꣳ रू॒पं ॅयत् कृ॑ष्णाजि॒नम् । \newline
29. यत् कृ॑ष्णाजि॒नम् कृ॑ष्णाजि॒नं ॅयद् यत् कृ॑ष्णाजि॒नम् ब्रह्म॑णा॒ ब्रह्म॑णा कृष्णाजि॒नं ॅयद् यत् कृ॑ष्णाजि॒नम् ब्रह्म॑णा । \newline
30. कृ॒ष्णा॒जि॒नम् ब्रह्म॑णा॒ ब्रह्म॑णा कृष्णाजि॒नम् कृ॑ष्णाजि॒नम् ब्रह्म॑ णै॒वैव ब्रह्म॑णा कृष्णाजि॒नम् कृ॑ष्णाजि॒नम् ब्रह्म॑णै॒व । \newline
31. कृ॒ष्णा॒जि॒नमिति॑ कृष्ण - अ॒जि॒नम् । \newline
32. ब्रह्म॑ णै॒वैव ब्रह्म॑णा॒ ब्रह्म॑ णै॒वैन॑ मेन मे॒व ब्रह्म॑णा॒ ब्रह्म॑ णै॒वैन᳚म् । \newline
33. ए॒वैन॑ मेन मे॒वै वैन॑म् दीक्षयति दीक्षय त्येन मे॒वै वैन॑म् दीक्षयति । \newline
34. ए॒न॒म् दी॒क्ष॒य॒ति॒ दी॒क्ष॒य॒ त्ये॒न॒ मे॒न॒म् दी॒क्ष॒य॒ ती॒मा मि॒माम् दी᳚क्षय त्येन मेनम् दीक्षयती॒माम् । \newline
35. दी॒क्ष॒य॒ ती॒मा मि॒माम् दी᳚क्षयति दीक्षय ती॒माम् धिय॒म् धिय॑ मि॒माम् दी᳚क्षयति दीक्षय ती॒माम् धिय᳚म् । \newline
36. इ॒माम् धिय॒म् धिय॑ मि॒मा मि॒माम् धियꣳ॒॒ शिक्ष॑माणस्य॒ शिक्ष॑माणस्य॒ धिय॑ मि॒मा मि॒माम् धियꣳ॒॒ शिक्ष॑माणस्य । \newline
37. धियꣳ॒॒ शिक्ष॑माणस्य॒ शिक्ष॑माणस्य॒ धिय॒म् धियꣳ॒॒ शिक्ष॑माणस्य देव देव॒ शिक्ष॑माणस्य॒ धिय॒म् धियꣳ॒॒ शिक्ष॑माणस्य देव । \newline
38. शिक्ष॑माणस्य देव देव॒ शिक्ष॑माणस्य॒ शिक्ष॑माणस्य दे॒वे तीति॑ देव॒ शिक्ष॑माणस्य॒ शिक्ष॑माणस्य दे॒वेति॑ । \newline
39. दे॒वे तीति॑ देव दे॒वे त्या॑हा॒हेति॑ देव दे॒वे त्या॑ह । \newline
40. इत्या॑हा॒हे तीत्या॑ह यथाय॒जुर् य॑थाय॒जु रा॒हे तीत्या॑ह यथाय॒जुः । \newline
41. आ॒ह॒ य॒था॒य॒जुर् य॑थाय॒जु रा॑हाह यथाय॒जु रे॒वैव य॑थाय॒जु रा॑हाह यथाय॒जु रे॒व । \newline
42. य॒था॒य॒जु रे॒वैव य॑थाय॒जुर् य॑थाय॒जु रे॒वैत दे॒त दे॒व य॑थाय॒जुर् य॑थाय॒जु रे॒वैतत् । \newline
43. य॒था॒य॒जुरिति॑ यथा - य॒जुः । \newline
44. ए॒वैत दे॒त दे॒वैवैतद् गर्भो॒ गर्भ॑ ए॒त दे॒वैवैतद् गर्भः॑ । \newline
45. ए॒तद् गर्भो॒ गर्भ॑ ए॒त दे॒तद् गर्भो॒ वै वै गर्भ॑ ए॒त दे॒तद् गर्भो॒ वै । \newline
46. गर्भो॒ वै वै गर्भो॒ गर्भो॒ वा ए॒ष ए॒ष वै गर्भो॒ गर्भो॒ वा ए॒षः । \newline
47. वा ए॒ष ए॒ष वै वा ए॒ष यद् यदे॒ष वै वा ए॒ष यत् । \newline
48. ए॒ष यद् यदे॒ष ए॒ष यद् दी᳚क्षि॒तो दी᳚क्षि॒तो यदे॒ष ए॒ष यद् दी᳚क्षि॒तः । \newline
49. यद् दी᳚क्षि॒तो दी᳚क्षि॒तो यद् यद् दी᳚क्षि॒त उल्ब॒ मुल्ब॑म् दीक्षि॒तो यद् यद् दी᳚क्षि॒त उल्ब᳚म् । \newline
50. दी॒क्षि॒त उल्ब॒ मुल्ब॑म् दीक्षि॒तो दी᳚क्षि॒त उल्बं॒ ॅवासो॒ वास॒ उल्ब॑म् दीक्षि॒तो दी᳚क्षि॒त उल्बं॒ ॅवासः॑ । \newline
51. उल्बं॒ ॅवासो॒ वास॒ उल्ब॒ मुल्बं॒ ॅवासः॒ प्र प्र वास॒ उल्ब॒ मुल्बं॒ ॅवासः॒ प्र । \newline
52. वासः॒ प्र प्र वासो॒ वासः॒ प्रोर्णु॑त ऊर्णुते॒ प्र वासो॒ वासः॒ प्रोर्णु॑ते । \newline
53. प्रोर्णु॑त ऊर्णुते॒ प्र प्रोर्णु॑ते॒ तस्मा॒त् तस्मा॑ दूर्णुते॒ प्र प्रोर्णु॑ते॒ तस्मा᳚त् । \newline
54. ऊ॒र्णु॒ते॒ तस्मा॒त् तस्मा॑ दूर्णुत ऊर्णुते॒ तस्मा॒द् गर्भा॒ गर्भा॒ स्तस्मा॑ दूर्णुत ऊर्णुते॒ तस्मा॒द् गर्भाः᳚ । \newline
55. तस्मा॒द् गर्भा॒ गर्भा॒ स्तस्मा॒त् तस्मा॒द् गर्भाः॒ प्रावृ॑ताः॒ प्रावृ॑ता॒ गर्भा॒ स्तस्मा॒त् तस्मा॒द् गर्भाः॒ प्रावृ॑ताः । \newline
\pagebreak
\markright{ TS 6.1.3.3  \hfill https://www.vedavms.in \hfill}

\section{ TS 6.1.3.3 }

\textbf{TS 6.1.3.3 } \newline
\textbf{Samhita Paata} \newline

गर्भाः॒ प्रावृ॑ता जायन्ते॒ न पु॒रा सोम॑स्य क्र॒यादपो᳚र्ण्वीत॒ यत् पु॒रा सोम॑स्य क्र॒याद॑पोर्ण्वी॒त गर्भाः᳚ प्र॒जानां᳚ परा॒पातु॑काः स्युः क्री॒ते सोमेऽपो᳚र्णुते॒ जाय॑त ए॒व तदथो॒ यथा॒ वसी॑याꣳ सं प्रत्यपोर्णु॒ते ता॒दृगे॒व तदङ्गि॑रसः सुव॒र्गं ॅलो॒कं ॅयन्त॒ ऊर्जं॒ ॅव्य॑भजन्त॒ ततो॒ यद॒त्यशि॑ष्यत॒ ते श॒रा अ॑भव॒न्नूर्ग्वै श॒रा यच्छ॑र॒मयी॒ - [  ] \newline

\textbf{Pada Paata} \newline

गर्भाः᳚ । प्रावृ॑ताः । जा॒य॒न्ते॒ । न । पु॒रा । सोम॑स्य । क्र॒यात् । अपेति॑ । ऊ॒र्ण्वी॒त॒ । यत् । पु॒रा । सोम॑स्य । क्र॒यात् । अ॒पो॒र्ण्वी॒तेत्य॑प-ऊ॒र्ण्वी॒त । गर्भाः᳚ । प्र॒जाना॒मिति॑ प्र - जाना᳚म् । प॒रा॒पातु॑का॒ इति॑ परा-पातु॑काः । स्युः॒ । क्री॒ते । सोमे᳚ । अपेति॑ । ऊ॒र्णु॒ते॒ । जाय॑ते । ए॒व । तत् । अथो॒ इति॑ । यथा᳚ । वसी॑याꣳसम् । प्र॒त्य॒पो॒र्णु॒त इति॑ प्रति -अ॒पो॒र्णु॒ते । ता॒दृक् । ए॒व । तत् । अङ्गि॑रसः । सु॒व॒र्गमिति॑ सुवः - गम् । लो॒कम् । यन्तः॑ । ऊर्ज᳚म् । वीति॑ । अ॒भ॒ज॒न्त॒ । ततः॑ । यत् । अ॒त्यशि॑ष्य॒तेत्य॑ति - अशि॑ष्यत । ते । श॒राः । अ॒भ॒व॒न्न् । ऊर्क् । वै । श॒राः । यत् । श॒र॒मयीति॑ शर - मयी᳚ ।  \newline


\textbf{Krama Paata} \newline

गर्भाः॒ प्रावृ॑ताः । प्रावृ॑ता जायन्ते । जा॒य॒न्ते॒ न । न पु॒रा । पु॒रा सोम॑स्य । सोम॑स्य क्र॒यात् । क्र॒यादप॑ । अपो᳚र्ण्वीत । ऊ॒र्ण्वी॒त॒ यत् । यत् पु॒रा । पु॒रा सोम॑स्य । सोम॑स्य क्र॒यात् । क्र॒याद॑पोर्ण्वी॒त । अ॒पो॒र्ण्वी॒त गर्भाः᳚ । अ॒पो॒र्ण्वी॒तेत्य॑प - ऊ॒र्ण्वी॒त । गर्भाः᳚ प्र॒जाना᳚म् । प्र॒जाना᳚म् परा॒पातु॑काः । प्र॒जाना॒मिति॑ प्र - जाना᳚म् । प॒रा॒पातु॑काः स्युः । प॒रा॒पातु॑का॒ इति॑ परा - पातु॑काः । स्युः॒ क्री॒ते । क्री॒ते सोमे᳚ । सोमेऽप॑ । अपो᳚र्णुते । ऊ॒र्णु॒ते॒ जाय॑ते । जाय॑त ए॒व । ए॒व तत् । तदथो᳚ । अथो॒ यथा᳚ । अथो॒ इत्यथो᳚ । यथा॒ वसी॑याꣳसम् । वसी॑याꣳसम् प्रत्यपोर्णु॒ते । प्र॒त्य॒पो॒र्णु॒ते ता॒दृक् । प्र॒त्य॒पो॒र्णु॒त इति॑ प्रति - अ॒पो॒र्णु॒ते । ता॒दृगे॒व । ए॒व तत् । तदङ्‍गि॑रसः । अङ्‍गि॑रसः सुव॒र्गम् । सु॒व॒र्गम् ॅलो॒कम् । सु॒व॒र्गमिति॑ सुवः - गम् । लो॒कम् ॅयन्तः॑ । यन्त॒ ऊर्ज᳚म् । ऊर्ज॒म् ॅवि । व्य॑भजन्त । अ॒भ॒ज॒न्त॒ ततः॑ । ततो॒ यत् । यद॒त्यशि॑ष्यत । अ॒त्यशि॑ष्यत॒ ते । अ॒त्यशि॑ष्य॒तेत्य॑ति - अशि॑ष्यत । ते श॒राः । श॒रा अ॑भवन्न् । अ॒भ॒व॒न्नूर्क् । ऊर्ग् वै । वै श॒राः । श॒रा यत् । यच्छ॑र॒मयी᳚ । श॒र॒मयी॒ मेख॑ला । श॒र॒मयीति॑ शर - मयी᳚ \newline

\textbf{Jatai Paata} \newline

1. गर्भाः॒ प्रावृ॑ताः॒ प्रावृ॑ता॒ गर्भा॒ गर्भाः॒ प्रावृ॑ताः । \newline
2. प्रावृ॑ता जायन्ते जायन्ते॒ प्रावृ॑ताः॒ प्रावृ॑ता जायन्ते । \newline
3. जा॒य॒न्ते॒ न न जा॑यन्ते जायन्ते॒ न । \newline
4. न पु॒रा पु॒रा न न पु॒रा । \newline
5. पु॒रा सोम॑स्य॒ सोम॑स्य पु॒रा पु॒रा सोम॑स्य । \newline
6. सोम॑स्य क्र॒यात् क्र॒याथ् सोम॑स्य॒ सोम॑स्य क्र॒यात् । \newline
7. क्र॒या दपाप॑ क्र॒यात् क्र॒या दप॑ । \newline
8. अपो᳚र्ण्वी तोर्ण्वी॒ता पापो᳚र्ण्वीत । \newline
9. ऊ॒र्ण्वी॒त॒ यद् यदू᳚ र्ण्वीतो र्ण्वीत॒ यत् । \newline
10. यत् पु॒रा पु॒रा यद् यत् पु॒रा । \newline
11. पु॒रा सोम॑स्य॒ सोम॑स्य पु॒रा पु॒रा सोम॑स्य । \newline
12. सोम॑स्य क्र॒यात् क्र॒याथ् सोम॑स्य॒ सोम॑स्य क्र॒यात् । \newline
13. क्र॒या द॑पोर्ण्वी॒ता पो᳚र्ण्वी॒त क्र॒यात् क्र॒या द॑पोर्ण्वी॒त । \newline
14. अ॒पो॒र्ण्वी॒त गर्भा॒ गर्भा॑ अपोर्ण्वी॒ता पो᳚र्ण्वी॒त गर्भाः᳚ । \newline
15. अ॒पो॒र्ण्वी॒तेत्य॑प - ऊ॒र्ण्वी॒त । \newline
16. गर्भाः᳚ प्र॒जाना᳚म् प्र॒जाना॒म् गर्भा॒ गर्भाः᳚ प्र॒जाना᳚म् । \newline
17. प्र॒जाना᳚म् परा॒पातु॑काः परा॒पातु॑काः प्र॒जाना᳚म् प्र॒जाना᳚म् परा॒पातु॑काः । \newline
18. प्र॒जाना॒मिति॑ प्र - जाना᳚म् । \newline
19. प॒रा॒पातु॑काः स्युः स्युः परा॒पातु॑काः परा॒पातु॑काः स्युः । \newline
20. प॒रा॒पातु॑का॒ इति॑ परा - पातु॑काः । \newline
21. स्युः॒ क्री॒ते क्री॒ते स्युः॑ स्युः क्री॒ते । \newline
22. क्री॒ते सोमे॒ सोमे᳚ क्री॒ते क्री॒ते सोमे᳚ । \newline
23. सोमे ऽपाप॒ सोमे॒ सोमे ऽप॑ । \newline
24. अपो᳚र्णुत ऊर्णु॒ते ऽपापो᳚ र्णुते । \newline
25. ऊ॒र्णु॒ते॒ जाय॑ते॒ जाय॑त ऊर्णुत ऊर्णुते॒ जाय॑ते । \newline
26. जाय॑त ए॒वैव जाय॑ते॒ जाय॑त ए॒व । \newline
27. ए॒व तत् तदे॒ वैव तत् । \newline
28. तदथो॒ अथो॒ तत् तदथो᳚ । \newline
29. अथो॒ यथा॒ यथा ऽथो॒ अथो॒ यथा᳚ । \newline
30. अथो॒ इत्यथो᳚ । \newline
31. यथा॒ वसी॑याꣳसं॒ ॅवसी॑याꣳसं॒ ॅयथा॒ यथा॒ वसी॑याꣳसम् । \newline
32. वसी॑याꣳसम् प्रत्यपोर्णु॒ते प्र॑त्यपोर्णु॒ते वसी॑याꣳसं॒ ॅवसी॑याꣳसम् प्रत्यपोर्णु॒ते । \newline
33. प्र॒त्य॒पो॒र्णु॒ते ता॒दृक् ता॒दृक् प्र॑त्यपोर्णु॒ते प्र॑त्यपोर्णु॒ते ता॒दृक् । \newline
34. प्र॒त्य॒पो॒र्णु॒त इति॑ प्रति - अ॒पो॒र्णु॒ते । \newline
35. ता॒दृ गे॒वैव ता॒दृक् ता॒दृ गे॒व । \newline
36. ए॒व तत् तदे॒ वैव तत् । \newline
37. तदङ्गि॑र॒सो ऽङ्गि॑रस॒ स्तत् तदङ्गि॑रसः । \newline
38. अङ्गि॑रसः सुव॒र्गꣳ सु॑व॒र्ग मङ्गि॑र॒सो ऽङ्गि॑रसः सुव॒र्गम् । \newline
39. सु॒व॒र्गम् ॅलो॒कम् ॅलो॒कꣳ सु॑व॒र्गꣳ सु॑व॒र्गम् ॅलो॒कम् । \newline
40. सु॒व॒र्गमिति॑ सुवः - गम् । \newline
41. लो॒कं ॅयन्तो॒ यन्तो॑ लो॒कम् ॅलो॒कं ॅयन्तः॑ । \newline
42. यन्त॒ ऊर्ज॒ मूर्जं॒ ॅयन्तो॒ यन्त॒ ऊर्ज᳚म् । \newline
43. ऊर्जं॒ ॅवि व्यूर्ज॒ मूर्जं॒ ॅवि । \newline
44. व्य॑भजन्ता भजन्त॒ वि व्य॑भजन्त । \newline
45. अ॒भ॒ज॒न्त॒ तत॒ स्ततो॑ ऽभजन्ता भजन्त॒ ततः॑ । \newline
46. ततो॒ यद् यत् तत॒ स्ततो॒ यत् । \newline
47. यद॒त्यशि॑ष्यता॒ त्यशि॑ष्यत॒ यद् यद॒त्यशि॑ष्यत । \newline
48. अ॒त्यशि॑ष्यत॒ ते ते᳚ ऽत्यशि॑ष्यता॒ त्यशि॑ष्यत॒ ते । \newline
49. अ॒त्यशि॑ष्य॒तेत्य॑ति - अशि॑ष्यत । \newline
50. ते श॒राः श॒रा स्ते ते श॒राः । \newline
51. श॒रा अ॑भवन् नभवञ् छ॒राः श॒रा अ॑भवन्न् । \newline
52. अ॒भ॒व॒न् नूर्गूर् ग॑भवन् नभव॒न् नूर्क् । \newline
53. ऊर्ग् वै वा ऊर्गूर्ग् वै । \newline
54. वै श॒राः श॒रा वै वै श॒राः । \newline
55. श॒रा यद् यच्छ॒राः श॒रा यत् । \newline
56. यच्छ॑र॒मयी॑ शर॒मयी॒ यद् यच्छ॑र॒मयी᳚ । \newline
57. श॒र॒मयी॒ मेख॑ला॒ मेख॑ला शर॒मयी॑ शर॒मयी॒ मेख॑ला । \newline
58. श॒र॒मयीति॑ शर - मयी᳚ । \newline

\textbf{Ghana Paata } \newline

1. गर्भाः॒ प्रावृ॑ताः॒ प्रावृ॑ता॒ गर्भा॒ गर्भाः॒ प्रावृ॑ता जायन्ते जायन्ते॒ प्रावृ॑ता॒ गर्भा॒ गर्भाः॒ प्रावृ॑ता जायन्ते । \newline
2. प्रावृ॑ता जायन्ते जायन्ते॒ प्रावृ॑ताः॒ प्रावृ॑ता जायन्ते॒ न न जा॑यन्ते॒ प्रावृ॑ताः॒ प्रावृ॑ता जायन्ते॒ न । \newline
3. जा॒य॒न्ते॒ न न जा॑यन्ते जायन्ते॒ न पु॒रा पु॒रा न जा॑यन्ते जायन्ते॒ न पु॒रा । \newline
4. न पु॒रा पु॒रा न न पु॒रा सोम॑स्य॒ सोम॑स्य पु॒रा न न पु॒रा सोम॑स्य । \newline
5. पु॒रा सोम॑स्य॒ सोम॑स्य पु॒रा पु॒रा सोम॑स्य क्र॒यात् क्र॒याथ् सोम॑स्य पु॒रा पु॒रा सोम॑स्य क्र॒यात् । \newline
6. सोम॑स्य क्र॒यात् क्र॒याथ् सोम॑स्य॒ सोम॑स्य क्र॒या दपाप॑ क्र॒याथ् सोम॑स्य॒ सोम॑स्य क्र॒या दप॑ । \newline
7. क्र॒या दपाप॑ क्र॒यात् क्र॒या दपो᳚र्ण्वी तोर्ण्वी॒ताप॑ क्र॒यात् क्र॒या दपो᳚र्ण्वीत । \newline
8. अपो᳚र्ण्वी तोर्ण्वी॒ता पापो᳚ र्ण्वीत॒ यद् यदू᳚र्ण्वी॒ता पापो᳚ र्ण्वीत॒ यत् । \newline
9. ऊ॒र्ण्वी॒त॒ यद् यदू᳚र्ण्वीतो र्ण्वीत॒ यत् पु॒रा पु॒रा यदू᳚र्ण्वीतो र्ण्वीत॒ यत् पु॒रा । \newline
10. यत् पु॒रा पु॒रा यद् यत् पु॒रा सोम॑स्य॒ सोम॑स्य पु॒रा यद् यत् पु॒रा सोम॑स्य । \newline
11. पु॒रा सोम॑स्य॒ सोम॑स्य पु॒रा पु॒रा सोम॑स्य क्र॒यात् क्र॒याथ् सोम॑स्य पु॒रा पु॒रा सोम॑स्य क्र॒यात् । \newline
12. सोम॑स्य क्र॒यात् क्र॒याथ् सोम॑स्य॒ सोम॑स्य क्र॒या द॑पोर्ण्वी॒ता पो᳚र्ण्वी॒त क्र॒याथ् सोम॑स्य॒ सोम॑स्य क्र॒या द॑पोर्ण्वी॒त । \newline
13. क्र॒या द॑पोर्ण्वी॒ता पो᳚र्ण्वी॒त क्र॒यात् क्र॒या द॑पोर्ण्वी॒त गर्भा॒ गर्भा॑ अपोर्ण्वी॒त क्र॒यात् क्र॒या द॑पोर्ण्वी॒त गर्भाः᳚ । \newline
14. अ॒पो॒र्ण्वी॒त गर्भा॒ गर्भा॑ अपोर्ण्वी॒ता पो᳚र्ण्वी॒त गर्भाः᳚ प्र॒जाना᳚म् प्र॒जाना॒म् गर्भा॑ अपोर्ण्वी॒ता पो᳚र्ण्वी॒त गर्भाः᳚ प्र॒जाना᳚म् । \newline
15. अ॒पो॒र्ण्वी॒तेत्य॑प - ऊ॒र्ण्वी॒त । \newline
16. गर्भाः᳚ प्र॒जाना᳚म् प्र॒जाना॒म् गर्भा॒ गर्भाः᳚ प्र॒जाना᳚म् परा॒पातु॑काः परा॒पातु॑काः प्र॒जाना॒म् गर्भा॒ गर्भाः᳚ प्र॒जाना᳚म् परा॒पातु॑काः । \newline
17. प्र॒जाना᳚म् परा॒पातु॑काः परा॒पातु॑काः प्र॒जाना᳚म् प्र॒जाना᳚म् परा॒पातु॑काः स्युः स्युः परा॒पातु॑काः प्र॒जाना᳚म् प्र॒जाना᳚म् परा॒पातु॑काः स्युः । \newline
18. प्र॒जाना॒मिति॑ प्र - जाना᳚म् । \newline
19. प॒रा॒पातु॑काः स्युः स्युः परा॒पातु॑काः परा॒पातु॑काः स्युः क्री॒ते क्री॒ते स्युः॑ परा॒पातु॑काः परा॒पातु॑काः स्युः क्री॒ते । \newline
20. प॒रा॒पातु॑का॒ इति॑ परा - पातु॑काः । \newline
21. स्युः॒ क्री॒ते क्री॒ते स्युः॑ स्युः क्री॒ते सोमे॒ सोमे᳚ क्री॒ते स्युः॑ स्युः क्री॒ते सोमे᳚ । \newline
22. क्री॒ते सोमे॒ सोमे᳚ क्री॒ते क्री॒ते सोमे ऽपाप॒ सोमे᳚ क्री॒ते क्री॒ते सोमे ऽप॑ । \newline
23. सोमे ऽपाप॒ सोमे॒ सोमे ऽपो᳚र्णुत ऊर्णु॒ते ऽप॒ सोमे॒ सोमे ऽपो᳚र्णुते । \newline
24. अपो᳚र्णुत ऊर्णु॒ते ऽपापो᳚र्णुते॒ जाय॑ते॒ जाय॑त ऊर्णु॒ते ऽपापो᳚र्णुते॒ जाय॑ते । \newline
25. ऊ॒र्णु॒ते॒ जाय॑ते॒ जाय॑त ऊर्णुत ऊर्णुते॒ जाय॑त ए॒वैव जाय॑त ऊर्णुत ऊर्णुते॒ जाय॑त ए॒व । \newline
26. जाय॑त ए॒वैव जाय॑ते॒ जाय॑त ए॒व तत् तदे॒व जाय॑ते॒ जाय॑त ए॒व तत् । \newline
27. ए॒व तत् तदे॒ वैव तदथो॒ अथो॒ तदे॒ वैव तदथो᳚ । \newline
28. तदथो॒ अथो॒ तत् तदथो॒ यथा॒ यथा ऽथो॒ तत् तदथो॒ यथा᳚ । \newline
29. अथो॒ यथा॒ यथा ऽथो॒ अथो॒ यथा॒ वसी॑याꣳसं॒ ॅवसी॑याꣳसं॒ ॅयथा ऽथो॒ अथो॒ यथा॒ वसी॑याꣳसम् । \newline
30. अथो॒ इत्यथो᳚ । \newline
31. यथा॒ वसी॑याꣳसं॒ ॅवसी॑याꣳसं॒ ॅयथा॒ यथा॒ वसी॑याꣳसम् प्रत्यपोर्णु॒ते प्र॑त्यपोर्णु॒ते वसी॑याꣳसं॒ ॅयथा॒ यथा॒ वसी॑याꣳसम् प्रत्यपोर्णु॒ते । \newline
32. वसी॑याꣳसम् प्रत्यपोर्णु॒ते प्र॑त्यपोर्णु॒ते वसी॑याꣳसं॒ ॅवसी॑याꣳसम् प्रत्यपोर्णु॒ते ता॒दृक् ता॒दृक् प्र॑त्यपोर्णु॒ते वसी॑याꣳसं॒ ॅवसी॑याꣳसम् प्रत्यपोर्णु॒ते ता॒दृक् । \newline
33. प्र॒त्य॒पो॒र्णु॒ते ता॒दृक् ता॒दृक् प्र॑त्यपोर्णु॒ते प्र॑त्यपोर्णु॒ते ता॒दृगे॒वैव ता॒दृक् प्र॑त्यपोर्णु॒ते प्र॑त्यपोर्णु॒ते ता॒दृगे॒व । \newline
34. प्र॒त्य॒पो॒र्णु॒त इति॑ प्रति - अ॒पो॒र्णु॒ते । \newline
35. ता॒दृ गे॒वैव ता॒दृक् ता॒दृ गे॒व तत् तदे॒व ता॒दृक् ता॒दृ गे॒व तत् । \newline
36. ए॒व तत् तदे॒ वैव तदङ्गि॑र॒सो ऽङ्गि॑रस॒ स्तदे॒ वैव तदङ्गि॑रसः । \newline
37. तदङ्गि॑र॒सो ऽङ्गि॑रस॒ स्तत् तदङ्गि॑रसः सुव॒र्गꣳ सु॑व॒र्ग मङ्गि॑रस॒ स्तत् तदङ्गि॑रसः सुव॒र्गम् । \newline
38. अङ्गि॑रसः सुव॒र्गꣳ सु॑व॒र्ग मङ्गि॑र॒सो ऽङ्गि॑रसः सुव॒र्गम् ॅलो॒कम् ॅलो॒कꣳ सु॑व॒र्ग मङ्गि॑र॒सो ऽङ्गि॑रसः सुव॒र्गम् ॅलो॒कम् । \newline
39. सु॒व॒र्गम् ॅलो॒कम् ॅलो॒कꣳ सु॑व॒र्गꣳ सु॑व॒र्गम् ॅलो॒कं ॅयन्तो॒ यन्तो॑ लो॒कꣳ सु॑व॒र्गꣳ सु॑व॒र्गम् ॅलो॒कं ॅयन्तः॑ । \newline
40. सु॒व॒र्गमिति॑ सुवः - गम् । \newline
41. लो॒कं ॅयन्तो॒ यन्तो॑ लो॒कम् ॅलो॒कं ॅयन्त॒ ऊर्ज॒ मूर्जं॒ ॅयन्तो॑ लो॒कम् ॅलो॒कं ॅयन्त॒ ऊर्ज᳚म् । \newline
42. यन्त॒ ऊर्ज॒ मूर्जं॒ ॅयन्तो॒ यन्त॒ ऊर्जं॒ ॅवि व्यूर्जं॒ ॅयन्तो॒ यन्त॒ ऊर्जं॒ ॅवि । \newline
43. ऊर्जं॒ ॅवि व्यूर्ज॒ मूर्जं॒ ॅव्य॑भजन्ता भजन्त॒ व्यूर्ज॒ मूर्जं॒ ॅव्य॑भजन्त । \newline
44. व्य॑भजन्ता भजन्त॒ वि व्य॑भजन्त॒ तत॒ स्ततो॑ ऽभजन्त॒ वि व्य॑भजन्त॒ ततः॑ । \newline
45. अ॒भ॒ज॒न्त॒ तत॒ स्ततो॑ ऽभजन्ता भजन्त॒ ततो॒ यद् यत् ततो॑ ऽभजन्ता भजन्त॒ ततो॒ यत् । \newline
46. ततो॒ यद् यत् तत॒ स्ततो॒ यद॒त्यशि॑ष्यता॒ त्यशि॑ष्यत॒ यत् तत॒ स्ततो॒ यद॒त्यशि॑ष्यत । \newline
47. यद॒त्यशि॑ष्यता॒ त्यशि॑ष्यत॒ यद् यद॒त्यशि॑ष्यत॒ ते ते᳚ ऽत्यशि॑ष्यत॒ यद् यद॒त्यशि॑ष्यत॒ ते । \newline
48. अ॒त्यशि॑ष्यत॒ ते ते᳚ ऽत्यशि॑ष्यता॒ त्यशि॑ष्यत॒ ते श॒राः श॒रा स्ते᳚ ऽत्यशि॑ष्यता॒ त्यशि॑ष्यत॒ ते श॒राः । \newline
49. अ॒त्यशि॑ष्य॒तेत्य॑ति - अशि॑ष्यत । \newline
50. ते श॒राः श॒रा स्ते ते श॒रा अ॑भवन् नभवञ् छ॒रा स्ते ते श॒रा अ॑भवन्न् । \newline
51. श॒रा अ॑भवन् नभवञ् छ॒राः श॒रा अ॑भव॒न् नूर् गूर् ग॑भवञ् छ॒राः श॒रा अ॑भव॒न् नूर्क् । \newline
52. अ॒भ॒व॒न् नूर् गूर् ग॑भवन् नभव॒न् नूर्ग् वै वा ऊर्ग॑भवन् नभव॒न् नूर्ग् वै । \newline
53. ऊर्ग् वै वा ऊर् गूर्ग् वै श॒राः श॒रा वा ऊर् गूर्ग् वै श॒राः । \newline
54. वै श॒राः श॒रा वै वै श॒रा यद् यच्छ॒रा वै वै श॒रा यत् । \newline
55. श॒रा यद् यच्छ॒राः श॒रा यच् छ॑र॒मयी॑ शर॒मयी॒ यच्छ॒राः श॒रा यच् छ॑र॒मयी᳚ । \newline
56. यच् छ॑र॒मयी॑ शर॒मयी॒ यद् यच् छ॑र॒मयी॒ मेख॑ला॒ मेख॑ला शर॒मयी॒ यद् यच् छ॑र॒मयी॒ मेख॑ला । \newline
57. श॒र॒मयी॒ मेख॑ला॒ मेख॑ला शर॒मयी॑ शर॒मयी॒ मेख॑ला॒ भव॑ति॒ भव॑ति॒ मेख॑ला शर॒मयी॑ शर॒मयी॒ मेख॑ला॒ भव॑ति । \newline
58. श॒र॒मयीति॑ शर - मयी᳚ । \newline
\pagebreak
\markright{ TS 6.1.3.4  \hfill https://www.vedavms.in \hfill}

\section{ TS 6.1.3.4 }

\textbf{TS 6.1.3.4 } \newline
\textbf{Samhita Paata} \newline

मेख॑ला॒ भव॒त्यूर्ज॑मे॒वाव॑ रुन्धे मद्ध्य॒तः संन॑ह्यति मद्ध्य॒त ए॒वास्मा॒ ऊर्जं॑ दधाति॒ तस्मा᳚न्मद्ध्य॒त ऊ॒र्जा भु॑ञ्जत ऊ॒र्द्ध्वं ॅवै पुरु॑षस्य॒ नाभ्यै॒ मेद्ध्य॑-मवा॒चीन॑-ममे॒द्ध्यं ॅयन्म॑द्ध्य॒तः स॒नंह्य॑ति॒ मेद्ध्यं॑ चै॒वास्या॑मे॒द्ध्यं च॒ व्याव॑र्तय॒तीन्द्रो॑ वृ॒त्राय॒ वज्रं॒ प्राह॑र॒थ् स त्रे॒धा व्य॑भव॒थ् स्फ्यस्तृती॑यꣳ॒॒ रथ॒स्तृती॑यं॒ ॅयूप॒स्तृती॑यं॒ - [  ] \newline

\textbf{Pada Paata} \newline

मेख॑ला । भव॑ति । ऊर्ज᳚म् । ए॒व । अवेति॑ । रु॒न्धे॒ । म॒द्ध्य॒तः । समिति॑ । न॒ह्य॒ति॒ । म॒द्ध्य॒तः । ए॒व । अ॒स्मै॒ । ऊर्ज᳚म् । द॒धा॒ति॒ । तस्मा᳚त् । म॒द्ध्य॒तः । ऊ॒र्जा । भु॒ञ्ज॒ते॒ । ऊ॒द्‌र्ध्वम् । वै । पुरु॑षस्य । नाभ्यै᳚ । मेद्ध्य᳚म् । अ॒वा॒चीन᳚म् । अ॒मे॒द्ध्यम् । यत् । म॒द्ध्य॒तः । स॒न्नह्य॒तीति॑ सं - नह्य॑ति । मेद्ध्य᳚म् । च॒ । ए॒व । अ॒स्य॒ । अ॒मे॒द्ध्यम् । च॒ । व्याव॑र्तय॒तीति॑ वि-आव॑र्तयति । इन्द्रः॑ । वृ॒त्राय॑ । वज्र᳚म् । प्रेति॑ । अ॒ह॒र॒त् । सः । त्रे॒धा । वीति॑ । अ॒भ॒व॒त् । स्फ्यः । तृती॑यम् । रथः॑ । तृती॑यम् । यूपः॑ । तृती॑यम् ।  \newline


\textbf{Krama Paata} \newline

मेख॑ला॒ भव॑ति । भव॒त्यूर्ज᳚म् । ऊर्ज॑मे॒व । ए॒वाव॑ । अव॑ रुन्धे । रु॒न्धे॒ म॒द्ध्य॒तः । म॒द्ध्य॒तः सम् । सम् न॑ह्यति । न॒ह्य॒ति॒ म॒द्ध्य॒तः । म॒द्ध्य॒त ए॒व । ए॒वास्मै᳚ । अ॒स्मा॒ ऊर्ज᳚म् । ऊर्ज॑म् दधाति । द॒धा॒ति॒ तस्मा᳚त् । तस्मा᳚न् मद्ध्य॒तः । म॒द्ध्य॒त ऊ॒र्जा । ऊ॒र्जा भु॑ञ्जते । भु॒ञ्ज॒त॒ ऊ॒र्द्ध्वम् । ऊ॒र्द्ध्वम् ॅवै । वै पुरु॑षस्य । पुरु॑षस्य॒ नाभ्यै᳚ । नाभ्यै॒ मेद्ध्य᳚म् । मेद्ध्य॑मवा॒चीन᳚म् । अ॒वा॒चीन॑ममे॒द्ध्यम् । अ॒मे॒द्ध्यम् ॅयत् । यन् म॑द्ध्य॒तः । म॒द्ध्य॒तः स॒न्नह्य॑ति । स॒न्नह्य॑ति॒ मेद्ध्य᳚म् । स॒न्नह्य॒तीति॑ सम् - नह्य॑ति । मेद्ध्य॑म् च । चै॒व । ए॒वास्य॑ । अ॒स्या॒मे॒द्ध्यम् । अ॒मे॒द्ध्यम् च॑ । च॒ व्याव॑र्तयति । व्याव॑र्तय॒तीन्द्रः॑ । व्याव॑र्तय॒तीति॑ वि - आव॑र्तयति । इन्द्रो॑ वृ॒त्राय॑ । वृ॒त्राय॒ वज्र᳚म् । वज्र॒म् प्र । प्राह॑रत् । अ॒ह॒र॒थ् सः । स त्रे॒धा । त्रे॒धा वि । व्य॑भवत् । अ॒भ॒व॒थ् स्फ्यः । स्फ्यस्तृती॑यम् । तृती॑यꣳ॒॒ रथः॑ । रथ॒स्तृती॑यम् । तृती॑य॒म् ॅयूपः॑ । यूप॒स्तृती॑यम् । तृती॑य॒म् ॅये \newline

\textbf{Jatai Paata} \newline

1. मेख॑ला॒ भव॑ति॒ भव॑ति॒ मेख॑ला॒ मेख॑ला॒ भव॑ति । \newline
2. भव॒ त्यूर्ज॒ मूर्ज॒म् भव॑ति॒ भव॒ त्यूर्ज᳚म् । \newline
3. ऊर्ज॑ मे॒वै वोर्ज॒ मूर्ज॑ मे॒व । \newline
4. ए॒वावा वै॒वै वाव॑ । \newline
5. अव॑ रुन्धे रु॒न्धे ऽवाव॑ रुन्धे । \newline
6. रु॒न्धे॒ म॒द्ध्य॒तो म॑द्ध्य॒तो रु॑न्धे रुन्धे मद्ध्य॒तः । \newline
7. म॒द्ध्य॒तः सꣳ सम् म॑द्ध्य॒तो म॑द्ध्य॒तः सम् । \newline
8. सन्न॑ह्यति नह्यति॒ सꣳ सन्न॑ह्यति । \newline
9. न॒ह्य॒ति॒ म॒द्ध्य॒तो म॑द्ध्य॒तो न॑ह्यति नह्यति मद्ध्य॒तः । \newline
10. म॒द्ध्य॒त ए॒वैव म॑द्ध्य॒तो म॑द्ध्य॒त ए॒व । \newline
11. ए॒वास्मा॑ अस्मा ए॒वै वास्मै᳚ । \newline
12. अ॒स्मा॒ ऊर्ज॒ मूर्ज॑ मस्मा अस्मा॒ ऊर्ज᳚म् । \newline
13. ऊर्ज॑म् दधाति दधा॒ त्यूर्ज॒ मूर्ज॑म् दधाति । \newline
14. द॒धा॒ति॒ तस्मा॒त् तस्मा᳚द् दधाति दधाति॒ तस्मा᳚त् । \newline
15. तस्मा᳚न् मद्ध्य॒तो म॑द्ध्य॒त स्तस्मा॒त् तस्मा᳚न् मद्ध्य॒तः । \newline
16. म॒द्ध्य॒त ऊ॒र्जोर्जा म॑द्ध्य॒तो म॑द्ध्य॒त ऊ॒र्जा । \newline
17. ऊ॒र्जा भु॑ञ्जते भुञ्जत ऊ॒र्जोर्जा भु॑ञ्जते । \newline
18. भु॒ञ्ज॒त॒ ऊ॒र्द्ध्व मू॒र्द्ध्वम् भु॑ञ्जते भुञ्जत ऊ॒र्द्ध्वम् । \newline
19. ऊ॒र्द्ध्वं ॅवै वा ऊ॒र्द्ध्व मू॒र्द्ध्वं ॅवै । \newline
20. वै पुरु॑षस्य॒ पुरु॑षस्य॒ वै वै पुरु॑षस्य । \newline
21. पुरु॑षस्य॒ नाभ्यै॒ नाभ्यै॒ पुरु॑षस्य॒ पुरु॑षस्य॒ नाभ्यै᳚ । \newline
22. नाभ्यै॒ मेद्ध्य॒म् मेद्ध्य॒म् नाभ्यै॒ नाभ्यै॒ मेद्ध्य᳚म् । \newline
23. मेद्ध्य॑ मवा॒चीन॑ मवा॒चीन॒म् मेद्ध्य॒म् मेद्ध्य॑ मवा॒चीन᳚म् । \newline
24. अ॒वा॒चीन॑ ममे॒द्ध्य म॑मे॒द्ध्य म॑वा॒चीन॑ मवा॒चीन॑ ममे॒द्ध्यम् । \newline
25. अ॒मे॒द्ध्यं ॅयद् यद॑मे॒द्ध्य म॑मे॒द्ध्यं ॅयत् । \newline
26. यन् म॑द्ध्य॒तो म॑द्ध्य॒तो यद् यन् म॑द्ध्य॒तः । \newline
27. म॒द्ध्य॒तः स॒न्नह्य॑ति स॒न्नह्य॑ति मद्ध्य॒तो म॑द्ध्य॒तः स॒न्नह्य॑ति । \newline
28. स॒न्नह्य॑ति॒ मेद्ध्य॒म् मेद्ध्यꣳ॑ स॒न्नह्य॑ति स॒न्नह्य॑ति॒ मेद्ध्य᳚म् । \newline
29. स॒न्नह्य॒तीति॑ सं - नह्य॑ति । \newline
30. मेद्ध्य॑म् च च॒ मेद्ध्य॒म् मेद्ध्य॑म् च । \newline
31. चै॒वैव च॑ चै॒व । \newline
32. ए॒वास्या᳚ स्यै॒वै वास्य॑ । \newline
33. अ॒स्या॒मे॒द्ध्य म॑मे॒द्ध्य म॑स्यास्या मे॒द्ध्यम् । \newline
34. अ॒मे॒द्ध्यम् च॑ चामे॒द्ध्य म॑मे॒द्ध्यम् च॑ । \newline
35. च॒ व्याव॑र्तयति॒ व्याव॑र्तयति च च॒ व्याव॑र्तयति । \newline
36. व्याव॑र्तय॒ तीन्द्र॒ इन्द्रो॒ व्याव॑र्तयति॒ व्याव॑र्तय॒ तीन्द्रः॑ । \newline
37. व्याव॑र्तय॒तीति॑ वि - आव॑र्तयति । \newline
38. इन्द्रो॑ वृ॒त्राय॑ वृ॒त्रायेन्द्र॒ इन्द्रो॑ वृ॒त्राय॑ । \newline
39. वृ॒त्राय॒ वज्रं॒ ॅवज्रं॑ ॅवृ॒त्राय॑ वृ॒त्राय॒ वज्र᳚म् । \newline
40. वज्र॒म् प्र प्र वज्रं॒ ॅवज्र॒म् प्र । \newline
41. प्राह॑र दहर॒त् प्र प्राह॑रत् । \newline
42. अ॒ह॒र॒थ् स सो॑ ऽहर दहर॒थ् सः । \newline
43. स त्रे॒धा त्रे॒धा स स त्रे॒धा । \newline
44. त्रे॒धा वि वि त्रे॒धा त्रे॒धा वि । \newline
45. व्य॑भव दभव॒द् वि व्य॑भवत् । \newline
46. अ॒भ॒व॒थ् स्फ्यः स्फ्यो॑ ऽभव दभव॒थ् स्फ्यः । \newline
47. स्फ्य स्तृती॑य॒म् तृती॑यꣳ॒॒ स्फ्यः स्फ्य स्तृती॑यम् । \newline
48. तृती॑यꣳ॒॒ रथो॒ रथ॒ स्तृती॑य॒म् तृती॑यꣳ॒॒ रथः॑ । \newline
49. रथ॒ स्तृती॑य॒म् तृती॑यꣳ॒॒ रथो॒ रथ॒ स्तृती॑यम् । \newline
50. तृती॑यं॒ ॅयूपो॒ यूप॒ स्तृती॑य॒म् तृती॑यं॒ ॅयूपः॑ । \newline
51. यूप॒ स्तृती॑य॒म् तृती॑यं॒ ॅयूपो॒ यूप॒ स्तृती॑यम् । \newline
52. तृती॑यं॒ ॅये ये तृती॑य॒म् तृती॑यं॒ ॅये । \newline

\textbf{Ghana Paata } \newline

1. मेख॑ला॒ भव॑ति॒ भव॑ति॒ मेख॑ला॒ मेख॑ला॒ भव॒ त्यूर्ज॒ मूर्ज॒म् भव॑ति॒ मेख॑ला॒ मेख॑ला॒ भव॒ त्यूर्ज᳚म् । \newline
2. भव॒ त्यूर्ज॒ मूर्ज॒म् भव॑ति॒ भव॒ त्यूर्ज॑ मे॒वैवोर्ज॒म् भव॑ति॒ भव॒ त्यूर्ज॑ मे॒व । \newline
3. ऊर्ज॑ मे॒वै वोर्ज॒ मूर्ज॑ मे॒वा वावै॒ वोर्ज॒ मूर्ज॑ मे॒वाव॑ । \newline
4. ए॒वावा वै॒वै वाव॑ रुन्धे रु॒न्धे ऽवै॒वै वाव॑ रुन्धे । \newline
5. अव॑ रुन्धे रु॒न्धे ऽवाव॑ रुन्धे मद्ध्य॒तो म॑द्ध्य॒तो रु॒न्धे ऽवाव॑ रुन्धे मद्ध्य॒तः । \newline
6. रु॒न्धे॒ म॒द्ध्य॒तो म॑द्ध्य॒तो रु॑न्धे रुन्धे मद्ध्य॒तः सꣳ सम् म॑द्ध्य॒तो रु॑न्धे रुन्धे मद्ध्य॒तः सम् । \newline
7. म॒द्ध्य॒तः सꣳ सम् म॑द्ध्य॒तो म॑द्ध्य॒तः सन् न॑ह्यति नह्यति॒ सम् म॑द्ध्य॒तो म॑द्ध्य॒तः सन् न॑ह्यति । \newline
8. सन् न॑ह्यति नह्यति॒ सꣳ सन् न॑ह्यति मद्ध्य॒तो म॑द्ध्य॒तो न॑ह्यति॒ सꣳ सन् न॑ह्यति मद्ध्य॒तः । \newline
9. न॒ह्य॒ति॒ म॒द्ध्य॒तो म॑द्ध्य॒तो न॑ह्यति नह्यति मद्ध्य॒त ए॒वैव म॑द्ध्य॒तो न॑ह्यति नह्यति मद्ध्य॒त ए॒व । \newline
10. म॒द्ध्य॒त ए॒वैव म॑द्ध्य॒तो म॑द्ध्य॒त ए॒वास्मा॑ अस्मा ए॒व म॑द्ध्य॒तो म॑द्ध्य॒त ए॒वास्मै᳚ । \newline
11. ए॒वास्मा॑ अस्मा ए॒वै वास्मा॒ ऊर्ज॒ मूर्ज॑ मस्मा ए॒वै वास्मा॒ ऊर्ज᳚म् । \newline
12. अ॒स्मा॒ ऊर्ज॒ मूर्ज॑ मस्मा अस्मा॒ ऊर्ज॑म् दधाति दधा॒ त्यूर्ज॑ मस्मा अस्मा॒ ऊर्ज॑म् दधाति । \newline
13. ऊर्ज॑म् दधाति दधा॒ त्यूर्ज॒ मूर्ज॑म् दधाति॒ तस्मा॒त् तस्मा᳚द् दधा॒ त्यूर्ज॒ मूर्ज॑म् दधाति॒ तस्मा᳚त् । \newline
14. द॒धा॒ति॒ तस्मा॒त् तस्मा᳚द् दधाति दधाति॒ तस्मा᳚न् मद्ध्य॒तो म॑द्ध्य॒त स्तस्मा᳚द् दधाति दधाति॒ तस्मा᳚न् मद्ध्य॒तः । \newline
15. तस्मा᳚न् मद्ध्य॒तो म॑द्ध्य॒त स्तस्मा॒त् तस्मा᳚न् मद्ध्य॒त ऊ॒र्जोर्जा म॑द्ध्य॒त स्तस्मा॒त् तस्मा᳚न् मद्ध्य॒त ऊ॒र्जा । \newline
16. म॒द्ध्य॒त ऊ॒र्जोर्जा म॑द्ध्य॒तो म॑द्ध्य॒त ऊ॒र्जा भु॑ञ्जते भुञ्जत ऊ॒र्जा म॑द्ध्य॒तो म॑द्ध्य॒त ऊ॒र्जा भु॑ञ्जते । \newline
17. ऊ॒र्जा भु॑ञ्जते भुञ्जत ऊ॒र्जोर्जा भु॑ञ्जत ऊ॒र्द्ध्व मू॒र्द्ध्वम् भु॑ञ्जत ऊ॒र्जोर्जा भु॑ञ्जत ऊ॒र्द्ध्वम् । \newline
18. भु॒ञ्ज॒त॒ ऊ॒र्द्ध्व मू॒र्द्ध्वम् भु॑ञ्जते भुञ्जत ऊ॒र्द्ध्वं ॅवै वा ऊ॒र्द्ध्वम् भु॑ञ्जते भुञ्जत ऊ॒र्द्ध्वं ॅवै । \newline
19. ऊ॒र्द्ध्वं ॅवै वा ऊ॒र्द्ध्व मू॒र्द्ध्वं ॅवै पुरु॑षस्य॒ पुरु॑षस्य॒ वा ऊ॒र्द्ध्व मू॒र्द्ध्वं ॅवै पुरु॑षस्य । \newline
20. वै पुरु॑षस्य॒ पुरु॑षस्य॒ वै वै पुरु॑षस्य॒ नाभ्यै॒ नाभ्यै॒ पुरु॑षस्य॒ वै वै पुरु॑षस्य॒ नाभ्यै᳚ । \newline
21. पुरु॑षस्य॒ नाभ्यै॒ नाभ्यै॒ पुरु॑षस्य॒ पुरु॑षस्य॒ नाभ्यै॒ मेद्ध्य॒म् मेद्ध्य॒म् नाभ्यै॒ पुरु॑षस्य॒ पुरु॑षस्य॒ नाभ्यै॒ मेद्ध्य᳚म् । \newline
22. नाभ्यै॒ मेद्ध्य॒म् मेद्ध्य॒म् नाभ्यै॒ नाभ्यै॒ मेद्ध्य॑ मवा॒चीन॑ मवा॒चीन॒म् मेद्ध्य॒म् नाभ्यै॒ नाभ्यै॒ मेद्ध्य॑ मवा॒चीन᳚म् । \newline
23. मेद्ध्य॑ मवा॒चीन॑ मवा॒चीन॒म् मेद्ध्य॒म् मेद्ध्य॑ मवा॒चीन॑ ममे॒द्ध्य म॑मे॒द्ध्य म॑वा॒चीन॒म् मेद्ध्य॒म् मेद्ध्य॑ मवा॒चीन॑ ममे॒द्ध्यम् । \newline
24. अ॒वा॒चीन॑ ममे॒द्ध्य म॑मे॒द्ध्य म॑वा॒चीन॑ मवा॒चीन॑ ममे॒द्ध्यं ॅयद् यद॑मे॒द्ध्य म॑वा॒चीन॑ मवा॒चीन॑ ममे॒द्ध्यं ॅयत् । \newline
25. अ॒मे॒द्ध्यं ॅयद् यद॑मे॒द्ध्य म॑मे॒द्ध्यं ॅयन् म॑द्ध्य॒तो म॑द्ध्य॒तो यद॑मे॒द्ध्य म॑मे॒द्ध्यं ॅयन् म॑द्ध्य॒तः । \newline
26. यन् म॑द्ध्य॒तो म॑द्ध्य॒तो यद् यन् म॑द्ध्य॒तः स॒न्नह्य॑ति स॒न्नह्य॑ति मद्ध्य॒तो यद् यन् म॑द्ध्य॒तः स॒न्नह्य॑ति । \newline
27. म॒द्ध्य॒तः स॒न्नह्य॑ति स॒न्नह्य॑ति मद्ध्य॒तो म॑द्ध्य॒तः स॒न्नह्य॑ति॒ मेद्ध्य॒म् मेद्ध्यꣳ॑ स॒न्नह्य॑ति मद्ध्य॒तो म॑द्ध्य॒तः स॒न्नह्य॑ति॒ मेद्ध्य᳚म् । \newline
28. स॒न्नह्य॑ति॒ मेद्ध्य॒म् मेद्ध्यꣳ॑ स॒न्नह्य॑ति स॒न्नह्य॑ति॒ मेद्ध्य॑म् च च॒ मेद्ध्यꣳ॑ स॒न्नह्य॑ति स॒न्नह्य॑ति॒ मेद्ध्य॑म् च । \newline
29. स॒न्नह्य॒तीति॑ सं - नह्य॑ति । \newline
30. मेद्ध्य॑म् च च॒ मेद्ध्य॒म् मेद्ध्य॑म् चै॒वैव च॒ मेद्ध्य॒म् मेद्ध्य॑म् चै॒व । \newline
31. चै॒वैव च॑ चै॒वास्या᳚ स्यै॒व च॑ चै॒वास्य॑ । \newline
32. ए॒वास्या᳚ स्यै॒वै वास्या॑ मे॒द्ध्य म॑मे॒द्ध्य म॑स्यै॒वै वास्या॑ मे॒द्ध्यम् । \newline
33. अ॒स्या॒ मे॒द्ध्य म॑मे॒द्ध्य म॑स्यास्या मे॒द्ध्यम् च॑ चामे॒द्ध्य म॑स्यास्या मे॒द्ध्यम् च॑ । \newline
34. अ॒मे॒द्ध्यम् च॑ चामे॒द्ध्य म॑मे॒द्ध्यम् च॒ व्याव॑र्तयति॒ व्याव॑र्तयति चामे॒द्ध्य म॑मे॒द्ध्यम् च॒ व्याव॑र्तयति । \newline
35. च॒ व्याव॑र्तयति॒ व्याव॑र्तयति च च॒ व्याव॑र्तय॒ तीन्द्र॒ इन्द्रो॒ व्याव॑र्तयति च च॒ व्याव॑र्तय॒ तीन्द्रः॑ । \newline
36. व्याव॑र्तय॒ तीन्द्र॒ इन्द्रो॒ व्याव॑र्तयति॒ व्याव॑र्तय॒ तीन्द्रो॑ वृ॒त्राय॑ वृ॒त्राये न्द्रो॒ व्याव॑र्तयति॒ व्याव॑र्तय॒ तीन्द्रो॑ वृ॒त्राय॑ । \newline
37. व्याव॑र्तय॒तीति॑ वि - आव॑र्तयति । \newline
38. इन्द्रो॑ वृ॒त्राय॑ वृ॒त्रायेन्द्र॒ इन्द्रो॑ वृ॒त्राय॒ वज्रं॒ ॅवज्रं॑ ॅवृ॒त्रायेन्द्र॒ इन्द्रो॑ वृ॒त्राय॒ वज्र᳚म् । \newline
39. वृ॒त्राय॒ वज्रं॒ ॅवज्रं॑ ॅवृ॒त्राय॑ वृ॒त्राय॒ वज्र॒म् प्र प्र वज्रं॑ ॅवृ॒त्राय॑ वृ॒त्राय॒ वज्र॒म् प्र । \newline
40. वज्र॒म् प्र प्र वज्रं॒ ॅवज्र॒म् प्राह॑र दहर॒त् प्र वज्रं॒ ॅवज्र॒म् प्राह॑रत् । \newline
41. प्राह॑र दहर॒त् प्र प्राह॑र॒थ् स सो॑ ऽहर॒त् प्र प्राह॑र॒थ् सः । \newline
42. अ॒ह॒र॒थ् स सो॑ ऽहर दहर॒थ् स त्रे॒धा त्रे॒धा सो॑ ऽहर दहर॒थ् स त्रे॒धा । \newline
43. स त्रे॒धा त्रे॒धा स स त्रे॒धा वि वि त्रे॒धा स स त्रे॒धा वि । \newline
44. त्रे॒धा वि वि त्रे॒धा त्रे॒धा व्य॑भव दभव॒द् वि त्रे॒धा त्रे॒धा व्य॑भवत् । \newline
45. व्य॑भव दभव॒द् वि व्य॑भव॒थ् स्फ्यः स्फ्यो॑ ऽभव॒द् वि व्य॑भव॒थ् स्फ्यः । \newline
46. अ॒भ॒व॒थ् स्फ्यः स्फ्यो॑ ऽभव दभव॒थ् स्फ्य स्तृती॑य॒म् तृती॑यꣳ॒॒ स्फ्यो॑ ऽभव दभव॒थ् स्फ्य स्तृती॑यम् । \newline
47. स्फ्य स्तृती॑य॒म् तृती॑यꣳ॒॒ स्फ्यः स्फ्य स्तृती॑यꣳ॒॒ रथो॒ रथ॒ स्तृती॑यꣳ॒॒ स्फ्यः स्फ्य स्तृती॑यꣳ॒॒ रथः॑ । \newline
48. तृती॑यꣳ॒॒ रथो॒ रथ॒ स्तृती॑य॒म् तृती॑यꣳ॒॒ रथ॒ स्तृती॑य॒म् तृती॑यꣳ॒॒ रथ॒ स्तृती॑य॒म् तृती॑यꣳ॒॒ रथ॒ स्तृती॑यम् । \newline
49. रथ॒ स्तृती॑य॒म् तृती॑यꣳ॒॒ रथो॒ रथ॒ स्तृती॑यं॒ ॅयूपो॒ यूप॒ स्तृती॑यꣳ॒॒ रथो॒ रथ॒ स्तृती॑यं॒ ॅयूपः॑ । \newline
50. तृती॑यं॒ ॅयूपो॒ यूप॒ स्तृती॑य॒म् तृती॑यं॒ ॅयूप॒ स्तृती॑य॒म् तृती॑यं॒ ॅयूप॒ स्तृती॑य॒म् तृती॑यं॒ ॅयूप॒ स्तृती॑यम् । \newline
51. यूप॒ स्तृती॑य॒म् तृती॑यं॒ ॅयूपो॒ यूप॒ स्तृती॑यं॒ ॅये ये तृती॑यं॒ ॅयूपो॒ यूप॒ स्तृती॑यं॒ ॅये । \newline
52. तृती॑यं॒ ॅये ये तृती॑य॒म् तृती॑यं॒ ॅये᳚ ऽन्तश्श॒रा अ॑न्तश्श॒रा ये तृती॑य॒म् तृती॑यं॒ ॅये᳚ ऽन्तश्श॒राः । \newline
\pagebreak
\markright{ TS 6.1.3.5  \hfill https://www.vedavms.in \hfill}

\section{ TS 6.1.3.5 }

\textbf{TS 6.1.3.5 } \newline
\textbf{Samhita Paata} \newline

ॅये᳚ऽन्तः श॒रा अशी᳚र्यन्त॒ ते श॒रा अ॑भव॒न् तच्छ॒राणाꣳ॑ शर॒त्वं ॅवज्रो॒ वै श॒राः क्षुत् खलु॒ वै म॑नु॒ष्य॑स्य॒ भ्रातृ॑व्यो॒ यच्छ॑र॒मयी॒ मेख॑ला॒ भव॑ति॒ वज्रे॑णै॒व सा॒क्षात् क्षुधं॒ भ्रातृ॑व्यं मद्ध्य॒तोऽप॑ हते त्रि॒वृद्-भ॑वति त्रि॒वृद्वै प्रा॒णस्त्रि॒वृत॑मे॒व प्रा॒णं म॑द्ध्य॒तो यज॑माने दधाति पृ॒थ्वी भ॑वति॒ रज्जू॑नां॒ ॅव्यावृ॑त्यै॒ मेख॑लया॒ यज॑मानं दीक्षयति॒ योक्त्रे॑ण॒ पत्नीं᳚ मिथुन॒त्वाय॑ - [  ] \newline

\textbf{Pada Paata} \newline

ये । अ॒न्त॒श्श॒रा इत्य॑न्तः-श॒राः । अशी᳚र्यन्त । ते । श॒राः । अ॒भ॒व॒न्न् । तत् । श॒राणा᳚म् । श॒र॒त्वमिति॑ शर-त्वम् । वज्रः॑ । वै । श॒राः । क्षुत् । खलु॑ । वै । म॒नु॒ष्य॑स्य । भ्रातृ॑व्यः । यत् । श॒र॒मयीति॑ शर - मयी᳚ । मेख॑ला । भव॑ति । वज्रे॑ण । ए॒व । सा॒क्षादिति॑ स-अ॒क्षात् । क्षुध᳚म् । भ्रातृ॑व्यम् । म॒द्ध्य॒तः । अपेति॑ । ह॒ते॒ । त्रि॒वृदिति॑ त्रि - वृत् । भ॒व॒ति॒ । त्रि॒वृदिति॑ त्रि - वृत् । वै । प्रा॒ण इति॑ प्र -अ॒नः । त्रि॒वृत॒मिति॑ त्रि - वृत᳚म् । ए॒व । प्रा॒णमिति॑ प्र - अ॒नम् । म॒द्ध्य॒तः । यज॑माने । द॒धा॒ति॒ । पृ॒थ्वी । भ॒व॒ति॒ । रज्जू॑नाम् । व्यावृ॑त्या॒ इति॑ वि - आवृ॑त्यै । मेख॑लया । यज॑मानम् । दी॒क्ष॒य॒ति॒ । योक्त्रे॑ण । पत्नी᳚म् । मि॒थु॒न॒त्वायेति॑ मिथुन - त्वाय॑ ।  \newline


\textbf{Krama Paata} \newline

ये᳚ऽन्तश्श॒राः । अ॒न्त॒श्श॒रा अशी᳚र्यन्त । अ॒न्त॒श्श॒रा इत्य॑न्तः - श॒राः । अशी᳚र्यन्त॒ ते । ते श॒राः । श॒रा अ॑भवन्न् । अ॒भ॒व॒न् तत् । तच्छ॒राणा᳚म् । श॒राणाꣳ॑ शर॒त्वम् । श॒र॒त्वम् ॅवज्रः॑ । श॒र॒त्वमिति॑ शर - त्वम् । वज्रो॒ वै । वै श॒राः । श॒राः क्षुत् । क्षुत् खलु॑ । खलु॒ वै । वै म॑नु॒ष्य॑स्य । म॒नु॒ष्य॑स्य॒ भ्रातृ॑व्यः । भ्रातृ॑व्यो॒ यत् । यच्छ॑र॒मयी᳚ । श॒र॒मयी॒ मेख॑ला । श॒र॒मयीति॑ शर - मयी᳚ । मेख॑ला॒ भव॑ति । भव॑ति॒ वज्रे॑ण । वज्रे॑णै॒व । ए॒व सा॒क्षात् । सा॒क्षात् क्षुध᳚म् । सा॒क्षादिति॑ स - अ॒क्षात् । क्षुध॒म् भ्रातृ॑व्यम् । भ्रातृ॑व्यम् मद्ध्य॒तः । म॒द्ध्य॒तोऽप॑ । अप॑ हते । ह॒ते॒ त्रि॒वृत् । त्रि॒वृद् भ॑वति । त्रि॒वृदिति॑ त्रि - वृत् । भ॒व॒ति॒ त्रि॒वृत् । त्रि॒वृद् वै । त्रि॒वृदिति॑ त्रि - वृत् । वै प्रा॒णः । प्रा॒णस्त्रि॒वृत᳚म् । प्रा॒ण इति॑ प्र - अ॒नः । त्रि॒वृत॑मे॒व । त्रि॒वृत॒मिति॑ त्रि - वृत᳚म् । ए॒व प्रा॒णम् । प्रा॒णम् म॑द्ध्य॒तः । प्रा॒णमिति॑ प्र - अ॒नम् । म॒द्ध्य॒तो यज॑माने । यज॑माने दधाति । द॒धा॒ति॒ पृ॒थ्वी । पृ॒थ्वी भ॑वति । भ॒व॒ति॒ रज्जू॑नाम् । रज्जू॑ना॒म् ॅव्यावृ॑त्त्यै । व्यावृ॑त्त्यै॒ मेख॑लया । व्यावृ॑त्त्या॒ इति॑ वि - आवृ॑त्त्यै । मेख॑लया॒ यज॑मानम् । यज॑मानम् दीक्षयति । दी॒क्ष॒य॒ति॒ योक्त्रे॑ण । योक्त्रे॑ण॒ पत्नी᳚म् । पत्नी᳚म् मिथुन॒त्वाय॑ । मि॒थु॒न॒त्वाय॑ य॒ज्ञ्ः । मि॒थु॒न॒त्वायेति॑ मिथुन - त्वाय॑ \newline

\textbf{Jatai Paata} \newline

1. ये᳚ ऽन्तश्श॒रा अ॑न्तश्श॒रा ये ये᳚ ऽन्तश्श॒राः । \newline
2. अ॒न्त॒श्श॒रा अशी᳚र्य॒न्ता शी᳚र्यन् तान्तश्श॒रा अ॑न्तश्श॒रा अशी᳚र्यन्त । \newline
3. अ॒न्त॒श्श॒रा इत्य॑न्तः - श॒राः । \newline
4. अशी᳚र्यन्त॒ ते ते ऽशी᳚र्य॒न्ता शी᳚र्यन्त॒ ते । \newline
5. ते श॒राः श॒रा स्ते ते श॒राः । \newline
6. श॒रा अ॑भवन् नभवञ् छ॒राः श॒रा अ॑भवन्न् । \newline
7. अ॒भ॒व॒न् तत् तद॑भवन् नभव॒न् तत् । \newline
8. तच्छ॒राणाꣳ॑ श॒राणा॒म् तत् तच्छ॒राणा᳚म् । \newline
9. श॒राणाꣳ॑ शर॒त्वꣳ श॑र॒त्वꣳ श॒राणाꣳ॑ श॒राणाꣳ॑ शर॒त्वम् । \newline
10. श॒र॒त्वं ॅवज्रो॒ वज्रः॑ शर॒त्वꣳ श॑र॒त्वं ॅवज्रः॑ । \newline
11. श॒र॒त्वमिति॑ शर - त्वम् । \newline
12. वज्रो॒ वै वै वज्रो॒ वज्रो॒ वै । \newline
13. वै श॒राः श॒रा वै वै श॒राः । \newline
14. श॒राः क्षुत् क्षु च्छ॒राः श॒राः क्षुत् । \newline
15. क्षुत् खलु॒ खलु॒ क्षुत् क्षुत् खलु॑ । \newline
16. खलु॒ वै वै खलु॒ खलु॒ वै । \newline
17. वै म॑नु॒ष्य॑स्य मनु॒ष्य॑स्य॒ वै वै म॑नु॒ष्य॑स्य । \newline
18. म॒नु॒ष्य॑स्य॒ भ्रातृ॑व्यो॒ भ्रातृ॑व्यो मनु॒ष्य॑स्य मनु॒ष्य॑स्य॒ भ्रातृ॑व्यः । \newline
19. भ्रातृ॑व्यो॒ यद् यद् भ्रातृ॑व्यो॒ भ्रातृ॑व्यो॒ यत् । \newline
20. यच्छ॑र॒मयी॑ शर॒मयी॒ यद् यच्छ॑र॒मयी᳚ । \newline
21. श॒र॒मयी॒ मेख॑ला॒ मेख॑ला शर॒मयी॑ शर॒मयी॒ मेख॑ला । \newline
22. श॒र॒मयीति॑ शर - मयी᳚ । \newline
23. मेख॑ला॒ भव॑ति॒ भव॑ति॒ मेख॑ला॒ मेख॑ला॒ भव॑ति । \newline
24. भव॑ति॒ वज्रे॑ण॒ वज्रे॑ण॒ भव॑ति॒ भव॑ति॒ वज्रे॑ण । \newline
25. वज्रे॑ णै॒वैव वज्रे॑ण॒ वज्रे॑णै॒व । \newline
26. ए॒व सा॒क्षाथ् सा॒क्षा दे॒वैव सा॒क्षात् । \newline
27. सा॒क्षात् क्षुध॒म् क्षुधꣳ॑ सा॒क्षाथ् सा॒क्षात् क्षुध᳚म् । \newline
28. सा॒क्षादिति॑ स - अ॒क्षात् । \newline
29. क्षुध॒म् भ्रातृ॑व्य॒म् भ्रातृ॑व्य॒म् क्षुध॒म् क्षुध॒म् भ्रातृ॑व्यम् । \newline
30. भ्रातृ॑व्यम् मद्ध्य॒तो म॑द्ध्य॒तो भ्रातृ॑व्य॒म् भ्रातृ॑व्यम् मद्ध्य॒तः । \newline
31. म॒द्ध्य॒तो ऽपाप॑ मद्ध्य॒तो म॑द्ध्य॒तो ऽप॑ । \newline
32. अप॑ हते ह॒ते ऽपाप॑ हते । \newline
33. ह॒ते॒ त्रि॒वृत् त्रि॒वृ द्ध॑ते हते त्रि॒वृत् । \newline
34. त्रि॒वृद् भ॑वति भवति त्रि॒वृत् त्रि॒वृद् भ॑वति । \newline
35. त्रि॒वृदिति॑ त्रि - वृत् । \newline
36. भ॒व॒ति॒ त्रि॒वृत् त्रि॒वृद् भ॑वति भवति त्रि॒वृत् । \newline
37. त्रि॒वृद् वै वै त्रि॒वृत् त्रि॒वृद् वै । \newline
38. त्रि॒वृदिति॑ त्रि - वृत् । \newline
39. वै प्रा॒णः प्रा॒णो वै वै प्रा॒णः । \newline
40. प्रा॒ण स्त्रि॒वृत॑म् त्रि॒वृत॑म् प्रा॒णः प्रा॒ण स्त्रि॒वृत᳚म् । \newline
41. प्रा॒ण इति॑ प्र - अ॒नः । \newline
42. त्रि॒वृत॑ मे॒वैव त्रि॒वृत॑म् त्रि॒वृत॑ मे॒व । \newline
43. त्रि॒वृत॒मिति॑ त्रि - वृत᳚म् । \newline
44. ए॒व प्रा॒णम् प्रा॒ण मे॒वैव प्रा॒णम् । \newline
45. प्रा॒णम् म॑द्ध्य॒तो म॑द्ध्य॒तः प्रा॒णम् प्रा॒णम् म॑द्ध्य॒तः । \newline
46. प्रा॒णमिति॑ प्र - अ॒नम् । \newline
47. म॒द्ध्य॒तो यज॑माने॒ यज॑माने मद्ध्य॒तो म॑द्ध्य॒तो यज॑माने । \newline
48. यज॑माने दधाति दधाति॒ यज॑माने॒ यज॑माने दधाति । \newline
49. द॒धा॒ति॒ पृ॒थ्वी पृ॒थ्वी द॑धाति दधाति पृ॒थ्वी । \newline
50. पृ॒थ्वी भ॑वति भवति पृ॒थ्वी पृ॒थ्वी भ॑वति । \newline
51. भ॒व॒ति॒ रज्जू॑नाꣳ॒॒ रज्जू॑नाम् भवति भवति॒ रज्जू॑नाम् । \newline
52. रज्जू॑नां॒ ॅव्यावृ॑त्त्यै॒ व्यावृ॑त्त्यै॒ रज्जू॑नाꣳ॒॒ रज्जू॑नां॒ ॅव्यावृ॑त्त्यै । \newline
53. व्यावृ॑त्त्यै॒ मेख॑लया॒ मेख॑लया॒ व्यावृ॑त्त्यै॒ व्यावृ॑त्त्यै॒ मेख॑लया । \newline
54. व्यावृ॑त्या॒ इति॑ वि - आवृ॑त्यै । \newline
55. मेख॑लया॒ यज॑मानं॒ ॅयज॑मान॒म् मेख॑लया॒ मेख॑लया॒ यज॑मानम् । \newline
56. यज॑मानम् दीक्षयति दीक्षयति॒ यज॑मानं॒ ॅयज॑मानम् दीक्षयति । \newline
57. दी॒क्ष॒य॒ति॒ योक्त्रे॑ण॒ योक्त्रे॑ण दीक्षयति दीक्षयति॒ योक्त्रे॑ण । \newline
58. योक्त्रे॑ण॒ पत्नी॒म् पत्नीं॒ ॅयोक्त्रे॑ण॒ योक्त्रे॑ण॒ पत्नी᳚म् । \newline
59. पत्नी᳚म् मिथुन॒त्वाय॑ मिथुन॒त्वाय॒ पत्नी॒म् पत्नी᳚म् मिथुन॒त्वाय॑ । \newline
60. मि॒थु॒न॒त्वाय॑ य॒ज्ञो य॒ज्ञो मि॑थुन॒त्वाय॑ मिथुन॒त्वाय॑ य॒ज्ञ्ः । \newline
61. मि॒थु॒न॒त्वायेति॑ मिथुन - त्वाय॑ । \newline

\textbf{Ghana Paata } \newline

1. ये᳚ ऽन्तश्श॒रा अ॑न्तश्श॒रा ये ये᳚ ऽन्तश्श॒रा अशी᳚र्य॒न्ता शी᳚र्यन्ता न्तश्श॒रा ये ये᳚ ऽन्तश्श॒रा अशी᳚र्यन्त । \newline
2. अ॒न्त॒श्श॒रा अशी᳚र्य॒न्ता शी᳚र्यन्ता न्तश्श॒रा अ॑न्तश्श॒रा अशी᳚र्यन्त॒ ते ते ऽशी᳚र्यन्ता न्तश्श॒रा अ॑न्तश्श॒रा अशी᳚र्यन्त॒ ते । \newline
3. अ॒न्त॒श्श॒रा इत्य॑न्तः - श॒राः । \newline
4. अशी᳚र्यन्त॒ ते ते ऽशी᳚र्य॒न्ता शी᳚र्यन्त॒ ते श॒राः श॒रा स्ते ऽशी᳚र्य॒न्ता शी᳚र्यन्त॒ ते श॒राः । \newline
5. ते श॒राः श॒रा स्ते ते श॒रा अ॑भवन् नभवञ् छ॒रा स्ते ते श॒रा अ॑भवन्न् । \newline
6. श॒रा अ॑भवन् नभवञ् छ॒राः श॒रा अ॑भव॒न् तत् तद॑भवञ् छ॒राः श॒रा अ॑भव॒न् तत् । \newline
7. अ॒भ॒व॒न् तत् तद॑भवन् नभव॒न् तच् छ॒राणाꣳ॑ श॒राणा॒म् तद॑भवन् नभव॒न् तच् छ॒राणा᳚म् । \newline
8. तच् छ॒राणाꣳ॑ श॒राणा॒म् तत् तच् छ॒राणाꣳ॑ शर॒त्वꣳ श॑र॒त्वꣳ श॒राणा॒म् तत् तच् छ॒राणाꣳ॑ शर॒त्वम् । \newline
9. श॒राणाꣳ॑ शर॒त्वꣳ श॑र॒त्वꣳ श॒राणाꣳ॑ श॒राणाꣳ॑ शर॒त्वं ॅवज्रो॒ वज्रः॑ शर॒त्वꣳ श॒राणाꣳ॑ श॒राणाꣳ॑ शर॒त्वं ॅवज्रः॑ । \newline
10. श॒र॒त्वं ॅवज्रो॒ वज्रः॑ शर॒त्वꣳ श॑र॒त्वं ॅवज्रो॒ वै वै वज्रः॑ शर॒त्वꣳ श॑र॒त्वं ॅवज्रो॒ वै । \newline
11. श॒र॒त्वमिति॑ शर - त्वम् । \newline
12. वज्रो॒ वै वै वज्रो॒ वज्रो॒ वै श॒राः श॒रा वै वज्रो॒ वज्रो॒ वै श॒राः । \newline
13. वै श॒राः श॒रा वै वै श॒राः क्षुत् क्षुच् छ॒रा वै वै श॒राः क्षुत् । \newline
14. श॒राः क्षुत् क्षुच् छ॒राः श॒राः क्षुत् खलु॒ खलु॒ क्षुच् छ॒राः श॒राः क्षुत् खलु॑ । \newline
15. क्षुत् खलु॒ खलु॒ क्षुत् क्षुत् खलु॒ वै वै खलु॒ क्षुत् क्षुत् खलु॒ वै । \newline
16. खलु॒ वै वै खलु॒ खलु॒ वै म॑नु॒ष्य॑स्य मनु॒ष्य॑स्य॒ वै खलु॒ खलु॒ वै म॑नु॒ष्य॑स्य । \newline
17. वै म॑नु॒ष्य॑स्य मनु॒ष्य॑स्य॒ वै वै म॑नु॒ष्य॑स्य॒ भ्रातृ॑व्यो॒ भ्रातृ॑व्यो मनु॒ष्य॑स्य॒ वै वै म॑नु॒ष्य॑स्य॒ भ्रातृ॑व्यः । \newline
18. म॒नु॒ष्य॑स्य॒ भ्रातृ॑व्यो॒ भ्रातृ॑व्यो मनु॒ष्य॑स्य मनु॒ष्य॑स्य॒ भ्रातृ॑व्यो॒ यद् यद् भ्रातृ॑व्यो मनु॒ष्य॑स्य मनु॒ष्य॑स्य॒ भ्रातृ॑व्यो॒ यत् । \newline
19. भ्रातृ॑व्यो॒ यद् यद् भ्रातृ॑व्यो॒ भ्रातृ॑व्यो॒ यच् छ॑र॒मयी॑ शर॒मयी॒ यद् भ्रातृ॑व्यो॒ भ्रातृ॑व्यो॒ यच् छ॑र॒मयी᳚ । \newline
20. यच् छ॑र॒मयी॑ शर॒मयी॒ यद् यच् छ॑र॒मयी॒ मेख॑ला॒ मेख॑ला शर॒मयी॒ यद् यच् छ॑र॒मयी॒ मेख॑ला । \newline
21. श॒र॒मयी॒ मेख॑ला॒ मेख॑ला शर॒मयी॑ शर॒मयी॒ मेख॑ला॒ भव॑ति॒ भव॑ति॒ मेख॑ला शर॒मयी॑ शर॒मयी॒ मेख॑ला॒ भव॑ति । \newline
22. श॒र॒मयीति॑ शर - मयी᳚ । \newline
23. मेख॑ला॒ भव॑ति॒ भव॑ति॒ मेख॑ला॒ मेख॑ला॒ भव॑ति॒ वज्रे॑ण॒ वज्रे॑ण॒ भव॑ति॒ मेख॑ला॒ मेख॑ला॒ भव॑ति॒ वज्रे॑ण । \newline
24. भव॑ति॒ वज्रे॑ण॒ वज्रे॑ण॒ भव॑ति॒ भव॑ति॒ वज्रे॑णै॒वैव वज्रे॑ण॒ भव॑ति॒ भव॑ति॒ वज्रे॑णै॒व । \newline
25. वज्रे॑णै॒वैव वज्रे॑ण॒ वज्रे॑णै॒व सा॒क्षाथ् सा॒क्षा दे॒व वज्रे॑ण॒ वज्रे॑णै॒व सा॒क्षात् । \newline
26. ए॒व सा॒क्षाथ् सा॒क्षा दे॒वैव सा॒क्षात् क्षुध॒म् क्षुधꣳ॑ सा॒क्षा दे॒वैव सा॒क्षात् क्षुध᳚म् । \newline
27. सा॒क्षात् क्षुध॒म् क्षुधꣳ॑ सा॒क्षाथ् सा॒क्षात् क्षुध॒म् भ्रातृ॑व्य॒म् भ्रातृ॑व्य॒म् क्षुधꣳ॑ सा॒क्षाथ् सा॒क्षात् क्षुध॒म् भ्रातृ॑व्यम् । \newline
28. सा॒क्षादिति॑ स - अ॒क्षात् । \newline
29. क्षुध॒म् भ्रातृ॑व्य॒म् भ्रातृ॑व्य॒म् क्षुध॒म् क्षुध॒म् भ्रातृ॑व्यम् मद्ध्य॒तो म॑द्ध्य॒तो भ्रातृ॑व्य॒म् क्षुध॒म् क्षुध॒म् भ्रातृ॑व्यम् मद्ध्य॒तः । \newline
30. भ्रातृ॑व्यम् मद्ध्य॒तो म॑द्ध्य॒तो भ्रातृ॑व्य॒म् भ्रातृ॑व्यम् मद्ध्य॒तो ऽपाप॑ मद्ध्य॒तो भ्रातृ॑व्य॒म् भ्रातृ॑व्यम् मद्ध्य॒तो ऽप॑ । \newline
31. म॒द्ध्य॒तो ऽपाप॑ मद्ध्य॒तो म॑द्ध्य॒तो ऽप॑ हते ह॒ते ऽप॑ मद्ध्य॒तो म॑द्ध्य॒तो ऽप॑ हते । \newline
32. अप॑ हते ह॒ते ऽपाप॑ हते त्रि॒वृत् त्रि॒वृ द्ध॒ते ऽपाप॑ हते त्रि॒वृत् । \newline
33. ह॒ते॒ त्रि॒वृत् त्रि॒वृ द्ध॑ते हते त्रि॒वृद् भ॑वति भवति त्रि॒वृ द्ध॑ते हते त्रि॒वृद् भ॑वति । \newline
34. त्रि॒वृद् भ॑वति भवति त्रि॒वृत् त्रि॒वृद् भ॑वति त्रि॒वृत् त्रि॒वृद् भ॑वति त्रि॒वृत् त्रि॒वृद् भ॑वति त्रि॒वृत् । \newline
35. त्रि॒वृदिति॑ त्रि - वृत् । \newline
36. भ॒व॒ति॒ त्रि॒वृत् त्रि॒वृद् भ॑वति भवति त्रि॒वृद् वै वै त्रि॒वृद् भ॑वति भवति त्रि॒वृद् वै । \newline
37. त्रि॒वृद् वै वै त्रि॒वृत् त्रि॒वृद् वै प्रा॒णः प्रा॒णो वै त्रि॒वृत् त्रि॒वृद् वै प्रा॒णः । \newline
38. त्रि॒वृदिति॑ त्रि - वृत् । \newline
39. वै प्रा॒णः प्रा॒णो वै वै प्रा॒ण स्त्रि॒वृत॑म् त्रि॒वृत॑म् प्रा॒णो वै वै प्रा॒ण स्त्रि॒वृत᳚म् । \newline
40. प्रा॒ण स्त्रि॒वृत॑म् त्रि॒वृत॑म् प्रा॒णः प्रा॒ण स्त्रि॒वृत॑ मे॒वैव त्रि॒वृत॑म् प्रा॒णः प्रा॒ण स्त्रि॒वृत॑ मे॒व । \newline
41. प्रा॒ण इति॑ प्र - अ॒नः । \newline
42. त्रि॒वृत॑ मे॒वैव त्रि॒वृत॑म् त्रि॒वृत॑ मे॒व प्रा॒णम् प्रा॒ण मे॒व त्रि॒वृत॑म् त्रि॒वृत॑ मे॒व प्रा॒णम् । \newline
43. त्रि॒वृत॒मिति॑ त्रि - वृत᳚म् । \newline
44. ए॒व प्रा॒णम् प्रा॒ण मे॒वैव प्रा॒णम् म॑द्ध्य॒तो म॑द्ध्य॒तः प्रा॒ण मे॒वैव प्रा॒णम् म॑द्ध्य॒तः । \newline
45. प्रा॒णम् म॑द्ध्य॒तो म॑द्ध्य॒तः प्रा॒णम् प्रा॒णम् म॑द्ध्य॒तो यज॑माने॒ यज॑माने मद्ध्य॒तः प्रा॒णम् प्रा॒णम् म॑द्ध्य॒तो यज॑माने । \newline
46. प्रा॒णमिति॑ प्र - अ॒नम् । \newline
47. म॒द्ध्य॒तो यज॑माने॒ यज॑माने मद्ध्य॒तो म॑द्ध्य॒तो यज॑माने दधाति दधाति॒ यज॑माने मद्ध्य॒तो म॑द्ध्य॒तो यज॑माने दधाति । \newline
48. यज॑माने दधाति दधाति॒ यज॑माने॒ यज॑माने दधाति पृ॒थ्वी पृ॒थ्वी द॑धाति॒ यज॑माने॒ यज॑माने दधाति पृ॒थ्वी । \newline
49. द॒धा॒ति॒ पृ॒थ्वी पृ॒थ्वी द॑धाति दधाति पृ॒थ्वी भ॑वति भवति पृ॒थ्वी द॑धाति दधाति पृ॒थ्वी भ॑वति । \newline
50. पृ॒थ्वी भ॑वति भवति पृ॒थ्वी पृ॒थ्वी भ॑वति॒ रज्जू॑नाꣳ॒॒ रज्जू॑नाम् भवति पृ॒थ्वी पृ॒थ्वी भ॑वति॒ रज्जू॑नाम् । \newline
51. भ॒व॒ति॒ रज्जू॑नाꣳ॒॒ रज्जू॑नाम् भवति भवति॒ रज्जू॑नां॒ ॅव्यावृ॑त्त्यै॒ व्यावृ॑त्त्यै॒ रज्जू॑नाम् भवति भवति॒ रज्जू॑नां॒ ॅव्यावृ॑त्त्यै । \newline
52. रज्जू॑नां॒ ॅव्यावृ॑त्त्यै॒ व्यावृ॑त्त्यै॒ रज्जू॑नाꣳ॒॒ रज्जू॑नां॒ ॅव्यावृ॑त्त्यै॒ मेख॑लया॒ मेख॑लया॒ व्यावृ॑त्त्यै॒ रज्जू॑नाꣳ॒॒ रज्जू॑नां॒ ॅव्यावृ॑त्त्यै॒ मेख॑लया । \newline
53. व्यावृ॑त्त्यै॒ मेख॑लया॒ मेख॑लया॒ व्यावृ॑त्त्यै॒ व्यावृ॑त्त्यै॒ मेख॑लया॒ यज॑मानं॒ ॅयज॑मान॒म् मेख॑लया॒ व्यावृ॑त्त्यै॒ व्यावृ॑त्त्यै॒ मेख॑लया॒ यज॑मानम् । \newline
54. व्यावृ॑त्या॒ इति॑ वि - आवृ॑त्यै । \newline
55. मेख॑लया॒ यज॑मानं॒ ॅयज॑मान॒म् मेख॑लया॒ मेख॑लया॒ यज॑मानम् दीक्षयति दीक्षयति॒ यज॑मान॒म् मेख॑लया॒ मेख॑लया॒ यज॑मानम् दीक्षयति । \newline
56. यज॑मानम् दीक्षयति दीक्षयति॒ यज॑मानं॒ ॅयज॑मानम् दीक्षयति॒ योक्त्रे॑ण॒ योक्त्रे॑ण दीक्षयति॒ यज॑मानं॒ ॅयज॑मानम् दीक्षयति॒ योक्त्रे॑ण । \newline
57. दी॒क्ष॒य॒ति॒ योक्त्रे॑ण॒ योक्त्रे॑ण दीक्षयति दीक्षयति॒ योक्त्रे॑ण॒ पत्नी॒म् पत्नीं॒ ॅयोक्त्रे॑ण दीक्षयति दीक्षयति॒ योक्त्रे॑ण॒ पत्नी᳚म् । \newline
58. योक्त्रे॑ण॒ पत्नी॒म् पत्नीं॒ ॅयोक्त्रे॑ण॒ योक्त्रे॑ण॒ पत्नी᳚म् मिथुन॒त्वाय॑ मिथुन॒त्वाय॒ पत्नीं॒ ॅयोक्त्रे॑ण॒ योक्त्रे॑ण॒ पत्नी᳚म् मिथुन॒त्वाय॑ । \newline
59. पत्नी᳚म् मिथुन॒त्वाय॑ मिथुन॒त्वाय॒ पत्नी॒म् पत्नी᳚म् मिथुन॒त्वाय॑ य॒ज्ञो य॒ज्ञो मि॑थुन॒त्वाय॒ पत्नी॒म् पत्नी᳚म् मिथुन॒त्वाय॑ य॒ज्ञ्ः । \newline
60. मि॒थु॒न॒त्वाय॑ य॒ज्ञो य॒ज्ञो मि॑थुन॒त्वाय॑ मिथुन॒त्वाय॑ य॒ज्ञो दक्षि॑णा॒म् दक्षि॑णां ॅय॒ज्ञो मि॑थुन॒त्वाय॑ मिथुन॒त्वाय॑ य॒ज्ञो दक्षि॑णाम् । \newline
61. मि॒थु॒न॒त्वायेति॑ मिथुन - त्वाय॑ । \newline
\pagebreak
\markright{ TS 6.1.3.6  \hfill https://www.vedavms.in \hfill}

\section{ TS 6.1.3.6 }

\textbf{TS 6.1.3.6 } \newline
\textbf{Samhita Paata} \newline

य॒ज्ञो दक्षि॑णाम॒भ्य॑द्ध्याय॒त् ताꣳ सम॑भव॒त् तदिन्द्रो॑-ऽचाय॒थ् सो॑ऽमन्यत॒ यो वा इ॒तो ज॑नि॒ष्यते॒ स इ॒दं भ॑विष्य॒तीति॒ तां प्रावि॑श॒त् तस्या॒ इन्द्र॑ ए॒वाजा॑यत॒ सो॑ऽमन्यत॒ यो वै मदि॒तो ऽप॑रो जनि॒ष्यते॒ स इ॒दं भ॑विष्य॒तीति॒ तस्या॑ अनु॒मृश्य॒ योनि॒मा-ऽच्छि॑न॒थ् सा सू॒तव॑शाऽभव॒त् तथ् सू॒तव॑शायै॒ जन्म॒ - [  ] \newline

\textbf{Pada Paata} \newline

य॒ज्ञ्ः । दक्षि॑णाम् । अ॒भीति॑ । अ॒द्ध्या॒य॒त् । ताम् । समिति॑ । अ॒भ॒व॒त् । तत् । इन्द्रः॑ । अ॒चा॒य॒त् । सः । अ॒म॒न्य॒त॒ । यः । वै । इ॒तः । ज॒नि॒ष्यते᳚ । सः । इ॒दम् । भ॒वि॒ष्य॒ति॒ । इति॑ । ताम् । प्रेति॑ । अ॒वि॒श॒त् । तस्याः᳚ । इन्द्रः॑ । ए॒व । अ॒जा॒य॒त॒ । सः । अ॒म॒न्य॒त॒ । यः । वै । मत् । इ॒तः । अप॑रः । ज॒नि॒ष्यते᳚ । सः । इ॒दम् । भ॒वि॒ष्य॒ति॒ । इति॑ । तस्याः᳚ । अ॒नु॒मृश्येत्य॑नु - मृश्य॑ । योनि᳚म् । एति॑ । अ॒च्छि॒न॒त् । सा । सू॒तव॒शेति॑ सू॒त - व॒शा॒ । अ॒भ॒व॒त् । तत् । सू॒तव॑शाया॒ इति॑ सू॒त - व॒शा॒यै॒ । जन्म॑ ।  \newline


\textbf{Krama Paata} \newline

य॒ज्ञो दक्षि॑णाम् । दक्षि॑णाम॒भि । अ॒भ्य॑द्ध्यायत् । अ॒द्ध्या॒य॒त् ताम् । ताꣳ सम् । सम॑भवत् । अ॒भ॒व॒त् तत् । तदिन्द्रः॑ । इन्द्रो॑ऽचायत् । अ॒चा॒य॒थ् सः । सो॑ऽमन्यत । अ॒म॒न्य॒त॒ यः । यो वै । वा इ॒तः । इ॒तो ज॑नि॒ष्यते᳚ । ज॒नि॒ष्यते॒ सः । स इ॒दम् । इ॒दम् भ॑विष्यति । भ॒वि॒ष्य॒तीति॑ । इति॒ ताम् । ताम् प्र । प्रावि॑शत् । अ॒वि॒श॒त् तस्याः᳚ । तस्या॒ इन्द्रः॑ । इन्द्र॑ ए॒व । ए॒वाजा॑यत । अ॒जा॒य॒त॒ सः । सो॑ऽमन्यत । अ॒म॒न्य॒त॒ यः । यो वै । वै मत् । मदि॒तः । इ॒तोऽप॑रः । अप॑रो जनि॒ष्यते᳚ । ज॒नि॒ष्यते॒ सः । स इ॒दम् । इ॒दम् भ॑विष्यति । भ॒वि॒ष्य॒तीति॑ । इति॒ तस्याः᳚ । तस्या॑ अनु॒मृश्य॑ । अ॒नु॒मृश्य॒ योनि᳚म् । अ॒नु॒मृश्येत्य॑नु - मृश्य॑ । योनि॒मा । आऽच्छि॑नत् । अ॒च्छि॒न॒थ् सा । सा सू॒तव॑शा । सू॒तव॑शाऽभवत् । सू॒तव॒शेति॑ सू॒त - व॒शा॒ । अ॒भ॒व॒त् तत् । तथ् सू॒तव॑शायै । सू॒तव॑शायै॒ जन्म॑ । सू॒तव॑शाया॒ इति॑ सू॒त - व॒शा॒यै॒ । जन्म॒ ताम् \newline

\textbf{Jatai Paata} \newline

1. य॒ज्ञो दक्षि॑णा॒म् दक्षि॑णां ॅय॒ज्ञो य॒ज्ञो दक्षि॑णाम् । \newline
2. दक्षि॑णा म॒भ्य॑भि दक्षि॑णा॒म् दक्षि॑णा म॒भि । \newline
3. अ॒भ्य॑द्ध्याय दद्ध्याय द॒भ्या᳚(1॒)भ्य॑द्ध्यायत् । \newline
4. अ॒द्ध्या॒य॒त् ताम् ता म॑द्ध्याय दद्ध्याय॒त् ताम् । \newline
5. ताꣳ सꣳ सम् ताम् ताꣳ सम् । \newline
6. स म॑भव दभव॒थ् सꣳ स म॑भवत् । \newline
7. अ॒भ॒व॒त् तत् तद॑भव दभव॒त् तत् । \newline
8. तदिन्द्र॒ इन्द्र॒ स्तत् तदिन्द्रः॑ । \newline
9. इन्द्रो॑ ऽचाय दचाय॒ दिन्द्र॒ इन्द्रो॑ ऽचायत् । \newline
10. अ॒चा॒य॒थ् स सो॑ ऽचाय दचाय॒थ् सः । \newline
11. सो॑ ऽमन्यता मन्यत॒ स सो॑ ऽमन्यत । \newline
12. अ॒म॒न्य॒त॒ यो यो॑ ऽमन्यता मन्यत॒ यः । \newline
13. यो वै वै यो यो वै । \newline
14. वा इ॒त इ॒तो वै वा इ॒तः । \newline
15. इ॒तो ज॑नि॒ष्यते॑ जनि॒ष्यत॑ इ॒त इ॒तो ज॑नि॒ष्यते᳚ । \newline
16. ज॒नि॒ष्यते॒ स स ज॑नि॒ष्यते॑ जनि॒ष्यते॒ सः । \newline
17. स इ॒द मि॒दꣳ स स इ॒दम् । \newline
18. इ॒दम् भ॑विष्यति भविष्यती॒द मि॒दम् भ॑विष्यति । \newline
19. भ॒वि॒ष्य॒ती तीति॑ भविष्यति भविष्य॒तीति॑ । \newline
20. इति॒ ताम् ता मितीति॒ ताम् । \newline
21. ताम् प्र प्र ताम् ताम् प्र । \newline
22. प्रावि॑श दविश॒त् प्र प्रावि॑शत् । \newline
23. अ॒वि॒श॒त् तस्या॒ स्तस्या॑ अविश दविश॒त् तस्याः᳚ । \newline
24. तस्या॒ इन्द्र॒ इन्द्र॒ स्तस्या॒ स्तस्या॒ इन्द्रः॑ । \newline
25. इन्द्र॑ ए॒वै वेन्द्र॒ इन्द्र॑ ए॒व । \newline
26. ए॒वा जा॑यता जाय तै॒वैवा जा॑यत । \newline
27. अ॒जा॒य॒त॒ स सो॑ ऽजायता जायत॒ सः । \newline
28. सो॑ ऽमन्यता मन्यत॒ स सो॑ ऽमन्यत । \newline
29. अ॒म॒न्य॒त॒ यो यो॑ ऽमन्यता मन्यत॒ यः । \newline
30. यो वै वै यो यो वै । \newline
31. वै मन् मद् वै वै मत् । \newline
32. मदि॒त इ॒तो मन् मदि॒तः । \newline
33. इ॒तो ऽप॒रो ऽप॑र इ॒त इ॒तो ऽप॑रः । \newline
34. अप॑रो जनि॒ष्यते॑ जनि॒ष्यते ऽप॒रो ऽप॑रो जनि॒ष्यते᳚ । \newline
35. ज॒नि॒ष्यते॒ स स ज॑नि॒ष्यते॑ जनि॒ष्यते॒ सः । \newline
36. स इ॒द मि॒दꣳ स स इ॒दम् । \newline
37. इ॒दम् भ॑विष्यति भविष्य ती॒द मि॒दम् भ॑विष्यति । \newline
38. भ॒वि॒ष्य॒ती तीति॑ भविष्यति भविष्य॒ तीति॑ । \newline
39. इति॒ तस्या॒ स्तस्या॒ इतीति॒ तस्याः᳚ । \newline
40. तस्या॑ अनु॒मृश्या॑ नु॒मृश्य॒ तस्या॒ स्तस्या॑ अनु॒मृश्य॑ । \newline
41. अ॒नु॒मृश्य॒ योनिं॒ ॅयोनि॑ मनु॒मृश्या॑ नु॒मृश्य॒ योनि᳚म् । \newline
42. अ॒नु॒मृश्येत्य॑नु - मृश्य॑ । \newline
43. योनि॒ मा योनिं॒ ॅयोनि॒ मा । \newline
44. आ ऽच्छि॑न दच्छिन॒ दा ऽच्छि॑नत् । \newline
45. अ॒च्छि॒न॒थ् सा सा ऽच्छि॑न दच्छिन॒थ् सा । \newline
46. सा सू॒तव॑शा सू॒तव॑शा॒ सा सा सू॒तव॑शा । \newline
47. सू॒तव॑शा ऽभव दभवथ् सू॒तव॑शा सू॒तव॑शा ऽभवत् । \newline
48. सू॒तव॒शेति॑ सू॒त - व॒शा॒ । \newline
49. अ॒भ॒व॒त् तत् तद॑भव दभव॒त् तत् । \newline
50. तथ् सू॒तव॑शायै सू॒तव॑शायै॒ तत् तथ् सू॒तव॑शायै । \newline
51. सू॒तव॑शायै॒ जन्म॒ जन्म॑ सू॒तव॑शायै सू॒तव॑शायै॒ जन्म॑ । \newline
52. सू॒तव॑शाया॒ इति॑ सू॒त - व॒शा॒यै॒ । \newline
53. जन्म॒ ताम् ताम् जन्म॒ जन्म॒ ताम् । \newline

\textbf{Ghana Paata } \newline

1. य॒ज्ञो दक्षि॑णा॒म् दक्षि॑णां ॅय॒ज्ञो य॒ज्ञो दक्षि॑णा म॒भ्य॑भि दक्षि॑णां ॅय॒ज्ञो य॒ज्ञो दक्षि॑णा म॒भि । \newline
2. दक्षि॑णा म॒भ्य॑भि दक्षि॑णा॒म् दक्षि॑णा म॒भ्य॑द्ध्याय दद्ध्याय द॒भि दक्षि॑णा॒म् दक्षि॑णा म॒भ्य॑द्ध्यायत् । \newline
3. अ॒भ्य॑द्ध्याय दद्ध्याय द॒भ्या᳚(1॒) भ्य॑द्ध्याय॒त् ताम् ता म॑द्ध्याय द॒भ्या᳚(1॒)भ्य॑द्ध्याय॒त् ताम् । \newline
4. अ॒द्ध्या॒य॒त् ताम् ता म॑द्ध्याय दद्ध्याय॒त् ताꣳ सꣳ सम् ता म॑द्ध्याय दद्ध्याय॒त् ताꣳ सम् । \newline
5. ताꣳ सꣳ सम् ताम् ताꣳ स म॑भव दभव॒थ् सम् ताम् ताꣳ स म॑भवत् । \newline
6. स म॑भव दभव॒थ् सꣳ स म॑भव॒त् तत् तद॑भव॒थ् सꣳ स म॑भव॒त् तत् । \newline
7. अ॒भ॒व॒त् तत् तद॑भव दभव॒त् तदिन्द्र॒ इन्द्र॒ स्त द॑भव दभव॒त् तदिन्द्रः॑ । \newline
8. तदिन्द्र॒ इन्द्र॒ स्तत् तदिन्द्रो॑ ऽचाय दचाय॒ दिन्द्र॒ स्तत् तदिन्द्रो॑ ऽचायत् । \newline
9. इन्द्रो॑ ऽचाय दचाय॒ दिन्द्र॒ इन्द्रो॑ ऽचाय॒थ् स सो॑ ऽचाय॒ दिन्द्र॒ इन्द्रो॑ ऽचाय॒थ् सः । \newline
10. अ॒चा॒य॒थ् स सो॑ ऽचाय दचाय॒थ् सो॑ ऽमन्यता मन्यत॒ सो॑ ऽचाय दचाय॒थ् सो॑ ऽमन्यत । \newline
11. सो॑ ऽमन्यता मन्यत॒ स सो॑ ऽमन्यत॒ यो यो॑ ऽमन्यत॒ स सो॑ ऽमन्यत॒ यः । \newline
12. अ॒म॒न्य॒त॒ यो यो॑ ऽमन्यता मन्यत॒ यो वै वै यो॑ ऽमन्यता मन्यत॒ यो वै । \newline
13. यो वै वै यो यो वा इ॒त इ॒तो वै यो यो वा इ॒तः । \newline
14. वा इ॒त इ॒तो वै वा इ॒तो ज॑नि॒ष्यते॑ जनि॒ष्यत॑ इ॒तो वै वा इ॒तो ज॑नि॒ष्यते᳚ । \newline
15. इ॒तो ज॑नि॒ष्यते॑ जनि॒ष्यत॑ इ॒त इ॒तो ज॑नि॒ष्यते॒ स स ज॑नि॒ष्यत॑ इ॒त इ॒तो ज॑नि॒ष्यते॒ सः । \newline
16. ज॒नि॒ष्यते॒ स स ज॑नि॒ष्यते॑ जनि॒ष्यते॒ स इ॒द मि॒दꣳ स ज॑नि॒ष्यते॑ जनि॒ष्यते॒ स इ॒दम् । \newline
17. स इ॒द मि॒दꣳ स स इ॒दम् भ॑विष्यति भविष्यती॒दꣳ स स इ॒दम् भ॑विष्यति । \newline
18. इ॒दम् भ॑विष्यति भविष्यती॒द मि॒दम् भ॑विष्य॒ती तीति॑ भविष्यती॒द मि॒दम् भ॑विष्य॒तीति॑ । \newline
19. भ॒वि॒ष्य॒ती तीति॑ भविष्यति भविष्य॒तीति॒ ताम् ता मिति॑ भविष्यति भविष्य॒तीति॒ ताम् । \newline
20. इति॒ ताम् ता मितीति॒ ताम् प्र प्र ता मितीति॒ ताम् प्र । \newline
21. ताम् प्र प्र ताम् ताम् प्रावि॑श दविश॒त् प्र ताम् ताम् प्रावि॑शत् । \newline
22. प्रावि॑श दविश॒त् प्र प्रावि॑श॒त् तस्या॒ स्तस्या॑ अविश॒त् प्र प्रावि॑श॒त् तस्याः᳚ । \newline
23. अ॒वि॒श॒त् तस्या॒ स्तस्या॑ अविश दविश॒त् तस्या॒ इन्द्र॒ इन्द्र॒ स्तस्या॑ अविश दविश॒त् तस्या॒ इन्द्रः॑ । \newline
24. तस्या॒ इन्द्र॒ इन्द्र॒ स्तस्या॒ स्तस्या॒ इन्द्र॑ ए॒वै वेन्द्र॒ स्तस्या॒ स्तस्या॒ इन्द्र॑ ए॒व । \newline
25. इन्द्र॑ ए॒वैवेन्द्र॒ इन्द्र॑ ए॒वाजा॑यता जायतै॒वेन्द्र॒ इन्द्र॑ ए॒वाजा॑यत । \newline
26. ए॒वाजा॑यता जायतै॒वैवा जा॑यत॒ स सो॑ ऽजायतै॒वैवा जा॑यत॒ सः । \newline
27. अ॒जा॒य॒त॒ स सो॑ ऽजायता जायत॒ सो॑ ऽमन्यता मन्यत॒ सो॑ ऽजायता जायत॒ सो॑ ऽमन्यत । \newline
28. सो॑ ऽमन्यता मन्यत॒ स सो॑ ऽमन्यत॒ यो यो॑ ऽमन्यत॒ स सो॑ ऽमन्यत॒ यः । \newline
29. अ॒म॒न्य॒त॒ यो यो॑ ऽमन्यता मन्यत॒ यो वै वै यो॑ ऽमन्यता मन्यत॒ यो वै । \newline
30. यो वै वै यो यो वै मन् मद् वै यो यो वै मत् । \newline
31. वै मन् मद् वै वै मदि॒त इ॒तो मद् वै वै मदि॒तः । \newline
32. मदि॒त इ॒तो मन् मदि॒तो ऽप॒रो ऽप॑र इ॒तो मन् मदि॒तो ऽप॑रः । \newline
33. इ॒तो ऽप॒रो ऽप॑र इ॒त इ॒तो ऽप॑रो जनि॒ष्यते॑ जनि॒ष्यते ऽप॑र इ॒त इ॒तो ऽप॑रो जनि॒ष्यते᳚ । \newline
34. अप॑रो जनि॒ष्यते॑ जनि॒ष्यते ऽप॒रो ऽप॑रो जनि॒ष्यते॒ स स ज॑नि॒ष्यते ऽप॒रो ऽप॑रो जनि॒ष्यते॒ सः । \newline
35. ज॒नि॒ष्यते॒ स स ज॑नि॒ष्यते॑ जनि॒ष्यते॒ स इ॒द मि॒दꣳ स ज॑नि॒ष्यते॑ जनि॒ष्यते॒ स इ॒दम् । \newline
36. स इ॒द मि॒दꣳ स स इ॒दम् भ॑विष्यति भविष्य ती॒दꣳ स स इ॒दम् भ॑विष्यति । \newline
37. इ॒दम् भ॑विष्यति भविष्यती॒द मि॒दम् भ॑विष्य॒ती तीति॑ भविष्यती॒द मि॒दम् भ॑विष्य॒तीति॑ । \newline
38. भ॒वि॒ष्य॒ती तीति॑ भविष्यति भविष्य॒तीति॒ तस्या॒ स्तस्या॒ इति॑ भविष्यति भविष्य॒तीति॒ तस्याः᳚ । \newline
39. इति॒ तस्या॒ स्तस्या॒ इतीति॒ तस्या॑ अनु॒मृश्या॑ नु॒मृश्य॒ तस्या॒ इतीति॒ तस्या॑ अनु॒मृश्य॑ । \newline
40. तस्या॑ अनु॒मृश्या॑ नु॒मृश्य॒ तस्या॒ स्तस्या॑ अनु॒मृश्य॒ योनिं॒ ॅयोनि॑ मनु॒मृश्य॒ तस्या॒ स्तस्या॑ अनु॒मृश्य॒ योनि᳚म् । \newline
41. अ॒नु॒मृश्य॒ योनिं॒ ॅयोनि॑ मनु॒मृश्या॑ नु॒मृश्य॒ योनि॒ मा योनि॑ मनु॒मृश्या॑ नु॒मृश्य॒ योनि॒ मा । \newline
42. अ॒नु॒मृश्येत्य॑नु - मृश्य॑ । \newline
43. योनि॒ मा योनिं॒ ॅयोनि॒ मा ऽच्छि॑न दच्छिन॒दा योनिं॒ ॅयोनि॒ मा ऽच्छि॑नत् । \newline
44. आ ऽच्छि॑न दच्छिन॒दा ऽच्छि॑न॒थ् सा सा ऽच्छि॑न॒दा ऽच्छि॑न॒थ् सा । \newline
45. अ॒च्छि॒न॒थ् सा सा ऽच्छि॑न दच्छिन॒थ् सा सू॒तव॑शा सू॒तव॑शा॒ सा ऽच्छि॑न दच्छिन॒थ् सा सू॒तव॑शा । \newline
46. सा सू॒तव॑शा सू॒तव॑शा॒ सा सा सू॒तव॑शा ऽभव दभवथ् सू॒तव॑शा॒ सा सा सू॒तव॑शा ऽभवत् । \newline
47. सू॒तव॑शा ऽभव दभवथ् सू॒तव॑शा सू॒तव॑शा ऽभव॒त् तत् तद॑भवथ् सू॒तव॑शा सू॒तव॑शा ऽभव॒त् तत् । \newline
48. सू॒तव॒शेति॑ सू॒त - व॒शा॒ । \newline
49. अ॒भ॒व॒त् तत् तद॑भव दभव॒त् तथ् सू॒तव॑शायै सू॒तव॑शायै॒ तद॑भव दभव॒त् तथ् सू॒तव॑शायै । \newline
50. तथ् सू॒तव॑शायै सू॒तव॑शायै॒ तत् तथ् सू॒तव॑शायै॒ जन्म॒ जन्म॑ सू॒तव॑शायै॒ तत् तथ् सू॒तव॑शायै॒ जन्म॑ । \newline
51. सू॒तव॑शायै॒ जन्म॒ जन्म॑ सू॒तव॑शायै सू॒तव॑शायै॒ जन्म॒ ताम् ताम् जन्म॑ सू॒तव॑शायै सू॒तव॑शायै॒ जन्म॒ ताम् । \newline
52. सू॒तव॑शाया॒ इति॑ सू॒त - व॒शा॒यै॒ । \newline
53. जन्म॒ ताम् ताम् जन्म॒ जन्म॒ ताꣳ हस्ते॒ हस्ते॒ ताम् जन्म॒ जन्म॒ ताꣳ हस्ते᳚ । \newline
\pagebreak
\markright{ TS 6.1.3.7  \hfill https://www.vedavms.in \hfill}

\section{ TS 6.1.3.7 }

\textbf{TS 6.1.3.7 } \newline
\textbf{Samhita Paata} \newline

ताꣳ हस्ते॒ न्य॑वेष्टयत॒ तां मृ॒गेषु॒ न्य॑दधा॒थ् सा कृ॑ष्णविषा॒णा- ऽभ॑व॒दिन्द्र॑स्य॒ योनि॑रसि॒ मा मा॑ हिꣳसी॒रिति॑ कृष्णविषा॒णां प्र य॑च्छति॒ सयो॑निमे॒व य॒ज्ञ्ं क॑रोति॒ सयो॑निं॒ दक्षि॑णाꣳ॒॒ सयो॑नि॒मिन्द्रꣳ॑ सयोनि॒त्वाय॑ कृ॒ष्यै त्वा॑ सुस॒स्याया॒ इत्या॑ह॒ तस्मा॑दकृष्टप॒च्या ओष॑धयः पच्यन्ते सुपिप्प॒लाभ्य॒-स्त्वौष॑धीभ्य॒ इत्या॑ह॒ तस्मा॒दोष॑धयः॒ फलं॑ गृह्णन्ति॒ यद्धस्ते॑न - [  ] \newline

\textbf{Pada Paata} \newline

ताम् । हस्ते᳚ । नीति॑ । अ॒वे॒ष्ट॒य॒त॒ । ताम् । मृ॒गेषु॑ । नीति॑ । अ॒द॒धा॒त् । सा । कृ॒ष्ण॒वि॒षा॒णेति॑ कृष्ण - वि॒षा॒णा । अ॒भ॒व॒त् । इन्द्र॑स्य । योनिः॑ । अ॒सि॒ । मा । मा॒ । हिꣳ॒॒सीः॒ । इति॑ । कृ॒ष्ण॒वि॒षा॒णामिति॑ कृष्ण - वि॒षा॒णाम् । प्रेति॑ । य॒च्छ॒ति॒ । सयो॑नि॒मिति॒ स - यो॒नि॒म् । ए॒व । य॒ज्ञ्म् । क॒रो॒ति॒ । सयो॑नि॒मिति॒ स - यो॒नि॒म् । दक्षि॑णाम् । सयो॑नि॒मिति॒ स - यो॒नि॒म् । इन्द्र᳚म् । स॒यो॒नि॒त्वायेति॑ सयोनि-त्वाय॑ । कृ॒ष्यै । त्वा॒ । सु॒स॒स्याया॒ इति॑ सु - स॒स्यायै᳚ । इति॑ । आ॒ह॒ । तस्मा᳚त् । अ॒कृ॒ष्ट॒प॒च्या इत्य॑कृष्ट - प॒च्याः । ओष॑धयः । प॒च्य॒न्ते॒ । सु॒पि॒प्प॒लाभ्य॒ इति॑ सु - पि॒प्प॒लाभ्यः॑ । त्वा॒ । ओष॑धीभ्य॒ इत्योष॑धि - भ्यः॒ । इति॑ । आ॒ह॒ । तस्मा᳚त् । ओष॑धयः । फल᳚म् । गृ॒ह्ण॒न्ति॒ । यत् । हस्ते॑न ।  \newline


\textbf{Krama Paata} \newline

ताꣳ हस्ते᳚ । हस्ते॒ नि । न्य॑वेष्टयत । अ॒वे॒ष्ट॒य॒त॒ ताम् । ताम् मृ॒गेषु॑ । मृ॒गेषु॒ नि । न्य॑दधात् । अ॒द॒धा॒थ् सा । सा कृ॑ष्णविषा॒णा । कृ॒ष्ण॒वि॒षा॒णाऽभ॑वत् । कृ॒ष्ण॒वि॒षा॒णेति॑ कृष्ण - वि॒षा॒णा । अ॒भ॒व॒दिन्द्र॑स्य । इन्द्र॑स्य॒ योनिः॑ । योनि॑रसि । अ॒सि॒ मा । मा मा᳚ । मा॒ हिꣳ॒॒सीः॒ । हिꣳ॒॒सी॒रिति॑ । इति॑ कृष्णविषा॒णाम् । कृ॒ष्ण॒वि॒षा॒णाम् प्र । कृ॒ष्ण॒वि॒षा॒णामिति॑ कृष्ण - वि॒षा॒णाम् । प्र य॑च्छति । य॒च्छ॒ति॒ सयो॑निम् । सयो॑निमे॒व । सयो॑नि॒मिति॒ स - यो॒नि॒म् । ए॒व य॒ज्ञ्म् । य॒ज्ञ्म् क॑रोति । क॒रो॒ति॒ सयो॑निम् । सयो॑नि॒म् दक्षि॑णाम् । सयो॑नि॒मिति॒ स - यो॒नि॒म् । दक्षि॑णाꣳ॒॒ सयो॑निम् । सयो॑नि॒मिन्द्र᳚म् । सयो॑नि॒मिति॒ स - यो॒नि॒म् । इन्द्रꣳ॑ सयोनि॒त्वाय॑ । स॒यो॒नि॒त्वाय॑ कृ॒ष्यै । स॒यो॒नि॒त्वायेति॑ सयोनि - त्वाय॑ । कृ॒ष्यै त्वा᳚ । त्वा॒ सु॒स॒स्यायै᳚ । सु॒स॒स्याया॒ इति॑ । सु॒स॒स्याया॒ इति॑ सु - स॒स्यायै᳚ । इत्या॑ह । आ॒ह॒ तस्मा᳚त् । तस्मा॑दकृष्टप॒च्याः । अ॒कृ॒ष्ट॒प॒च्या ओष॑धयः । अ॒कृ॒ष्ट॒प॒च्या इत्य॑कृष्ट - प॒च्याः । ओष॑धयः पच्यन्ते । प॒च्य॒न्ते॒ सु॒पि॒प्प॒लाभ्यः॑ । सु॒पि॒प्प॒लाभ्य॑स्त्वा । सु॒पि॒प्प॒लाभ्य॒ इति॑ सु - पि॒प्प॒लाभ्यः॑ । त्वौष॑धीभ्यः । ओष॑धीभ्य॒ इति॑ । ओष॑धीभ्य॒ इत्योष॑धि - भ्यः॒ । इत्या॑ह । आ॒ह॒ तस्मा᳚त् । तस्मा॒दोष॑धयः । ओष॑धयः॒ फल᳚म् । फल॑म् गृह्णन्ति । गृ॒ह्ण॒न्ति॒ यत् । यद्‍धस्ते॑न ( ) । हस्ते॑न कण्डू॒येत॑ \newline

\textbf{Jatai Paata} \newline

1. ताꣳ हस्ते॒ हस्ते॒ ताम् ताꣳ हस्ते᳚ । \newline
2. हस्ते॒ नि नि हस्ते॒ हस्ते॒ नि । \newline
3. न्य॑वेष्टयता वेष्टयत॒ नि न्य॑वेष्टयत । \newline
4. अ॒वे॒ष्ट॒य॒त॒ ताम् ता म॑वेष्टयता वेष्टयत॒ ताम् । \newline
5. ताम् मृ॒गेषु॑ मृ॒गेषु॒ ताम् ताम् मृ॒गेषु॑ । \newline
6. मृ॒गेषु॒ नि नि मृ॒गेषु॑ मृ॒गेषु॒ नि । \newline
7. न्य॑दधा ददधा॒न् नि न्य॑दधात् । \newline
8. अ॒द॒धा॒थ् सा सा ऽद॑धा ददधा॒थ् सा । \newline
9. सा कृ॑ष्णविषा॒णा कृ॑ष्णविषा॒णा सा सा कृ॑ष्णविषा॒णा । \newline
10. कृ॒ष्ण॒वि॒षा॒णा ऽभ॑व दभवत् कृष्णविषा॒णा कृ॑ष्णविषा॒णा ऽभ॑वत् । \newline
11. कृ॒ष्ण॒वि॒षा॒णेति॑ कृष्ण - वि॒षा॒णा । \newline
12. अ॒भ॒व॒ दिन्द्र॒ स्येन्द्र॑स्या भव दभव॒ दिन्द्र॑स्य । \newline
13. इन्द्र॑स्य॒ योनि॒र् योनि॒ रिन्द्र॒ स्येन्द्र॑स्य॒ योनिः॑ । \newline
14. योनि॑ रस्यसि॒ योनि॒र् योनि॑ रसि । \newline
15. अ॒सि॒ मा मा ऽस्य॑सि॒ मा । \newline
16. मा मा॑ मा॒ मा मा मा᳚ । \newline
17. मा॒ हिꣳ॒॒सी॒र्॒. हिꣳ॒॒सी॒र् मा॒ मा॒ हिꣳ॒॒सीः॒ । \newline
18. हिꣳ॒॒सी॒ रितीति॑ हिꣳसीर्. हिꣳसी॒ रिति॑ । \newline
19. इति॑ कृष्णविषा॒णाम् कृ॑ष्णविषा॒णा मितीति॑ कृष्णविषा॒णाम् । \newline
20. कृ॒ष्ण॒वि॒षा॒णाम् प्र प्र कृ॑ष्णविषा॒णाम् कृ॑ष्णविषा॒णाम् प्र । \newline
21. कृ॒ष्ण॒वि॒षा॒णामिति॑ कृष्ण - वि॒षा॒णाम् । \newline
22. प्र य॑च्छति यच्छति॒ प्र प्र य॑च्छति । \newline
23. य॒च्छ॒ति॒ सयो॑निꣳ॒॒ सयो॑निं ॅयच्छति यच्छति॒ सयो॑निम् । \newline
24. सयो॑नि मे॒वैव सयो॑निꣳ॒॒ सयो॑नि मे॒व । \newline
25. सयो॑नि॒मिति॒ स - यो॒नि॒म् । \newline
26. ए॒व य॒ज्ञ्ं ॅय॒ज्ञ् मे॒वैव य॒ज्ञ्म् । \newline
27. य॒ज्ञ्म् क॑रोति करोति य॒ज्ञ्ं ॅय॒ज्ञ्म् क॑रोति । \newline
28. क॒रो॒ति॒ सयो॑निꣳ॒॒ सयो॑निम् करोति करोति॒ सयो॑निम् । \newline
29. सयो॑नि॒म् दक्षि॑णा॒म् दक्षि॑णाꣳ॒॒ सयो॑निꣳ॒॒ सयो॑नि॒म् दक्षि॑णाम् । \newline
30. सयो॑नि॒मिति॒ स - यो॒नि॒म् । \newline
31. दक्षि॑णाꣳ॒॒ सयो॑निꣳ॒॒ सयो॑नि॒म् दक्षि॑णा॒म् दक्षि॑णाꣳ॒॒ सयो॑निम् । \newline
32. सयो॑नि॒ मिन्द्र॒ मिन्द्रꣳ॒॒ सयो॑निꣳ॒॒ सयो॑नि॒ मिन्द्र᳚म् । \newline
33. सयो॑नि॒मिति॒ स - यो॒नि॒म् । \newline
34. इन्द्रꣳ॑ सयोनि॒त्वाय॑ सयोनि॒त्वायेन्द्र॒ मिन्द्रꣳ॑ सयोनि॒त्वाय॑ । \newline
35. स॒यो॒नि॒त्वाय॑ कृ॒ष्यै कृ॒ष्यै स॑योनि॒त्वाय॑ सयोनि॒त्वाय॑ कृ॒ष्यै । \newline
36. स॒यो॒नि॒त्वायेति॑ सयोनि - त्वाय॑ । \newline
37. कृ॒ष्यै त्वा᳚ त्वा कृ॒ष्यै कृ॒ष्यै त्वा᳚ । \newline
38. त्वा॒ सु॒स॒स्यायै॑ सुस॒स्यायै᳚ त्वा त्वा सुस॒स्यायै᳚ । \newline
39. सु॒स॒स्याया॒ इतीति॑ सुस॒स्यायै॑ सुस॒स्याया॒ इति॑ । \newline
40. सु॒स॒स्याया॒ इति॑ सु - स॒स्यायै᳚ । \newline
41. इत्या॑हा॒हे तीत्या॑ह । \newline
42. आ॒ह॒ तस्मा॒त् तस्मा॑ दाहाह॒ तस्मा᳚त् । \newline
43. तस्मा॑ दकृष्टप॒च्या अ॑कृष्टप॒च्या स्तस्मा॒त् तस्मा॑ दकृष्टप॒च्याः । \newline
44. अ॒कृ॒ष्ट॒प॒च्या ओष॑धय॒ ओष॑धयो ऽकृष्टप॒च्या अ॑कृष्टप॒च्या ओष॑धयः । \newline
45. अ॒कृ॒ष्ट॒प॒च्या इत्य॑कृष्ट - प॒च्याः । \newline
46. ओष॑धयः पच्यन्ते पच्यन्त॒ ओष॑धय॒ ओष॑धयः पच्यन्ते । \newline
47. प॒च्य॒न्ते॒ सु॒पि॒प्प॒लाभ्यः॑ सुपिप्प॒लाभ्यः॑ पच्यन्ते पच्यन्ते सुपिप्प॒लाभ्यः॑ । \newline
48. सु॒पि॒प्प॒लाभ्य॑ स्त्वा त्वा सुपिप्प॒लाभ्यः॑ सुपिप्प॒लाभ्य॑ स्त्वा । \newline
49. सु॒पि॒प्प॒लाभ्य॒ इति॑ सु - पि॒प्प॒लाभ्यः॑ । \newline
50. त्वौष॑धीभ्य॒ ओष॑धीभ्य स्त्वा॒ त्वौष॑धीभ्यः । \newline
51. ओष॑धीभ्य॒ इतीत्योष॑धीभ्य॒ ओष॑धीभ्य॒ इति॑ । \newline
52. ओष॑धीभ्य॒ इत्योष॑धि - भ्यः॒ । \newline
53. इत्या॑हा॒हे तीत्या॑ह । \newline
54. आ॒ह॒ तस्मा॒त् तस्मा॑ दाहाह॒ तस्मा᳚त् । \newline
55. तस्मा॒ दोष॑धय॒ ओष॑धय॒ स्तस्मा॒त् तस्मा॒ दोष॑धयः । \newline
56. ओष॑धयः॒ फल॒म् फल॒ मोष॑धय॒ ओष॑धयः॒ फल᳚म् । \newline
57. फल॑म् गृह्णन्ति गृह्णन्ति॒ फल॒म् फल॑म् गृह्णन्ति । \newline
58. गृ॒ह्ण॒न्ति॒ यद् यद् गृ॑ह्णन्ति गृह्णन्ति॒ यत् । \newline
59. यद्धस्ते॑न॒ हस्ते॑न॒ यद् यद्धस्ते॑न । \newline
60. हस्ते॑न कण्डू॒येत॑ कण्डू॒येत॒ हस्ते॑न॒ हस्ते॑न कण्डू॒येत॑ । \newline

\textbf{Ghana Paata } \newline

1. ताꣳ हस्ते॒ हस्ते॒ ताम् ताꣳ हस्ते॒ नि नि हस्ते॒ ताम् ताꣳ हस्ते॒ नि । \newline
2. हस्ते॒ नि नि हस्ते॒ हस्ते॒ न्य॑वेष्टयता वेष्टयत॒ नि हस्ते॒ हस्ते॒ न्य॑वेष्टयत । \newline
3. न्य॑वेष्टयता वेष्टयत॒ नि न्य॑वेष्टयत॒ ताम् ता म॑वेष्टयत॒ नि न्य॑वेष्टयत॒ ताम् । \newline
4. अ॒वे॒ष्ट॒य॒त॒ ताम् ता म॑वेष्टयता वेष्टयत॒ ताम् मृ॒गेषु॑ मृ॒गेषु॒ ता म॑वेष्टयता वेष्टयत॒ ताम् मृ॒गेषु॑ । \newline
5. ताम् मृ॒गेषु॑ मृ॒गेषु॒ ताम् ताम् मृ॒गेषु॒ नि नि मृ॒गेषु॒ ताम् ताम् मृ॒गेषु॒ नि । \newline
6. मृ॒गेषु॒ नि नि मृ॒गेषु॑ मृ॒गेषु॒ न्य॑दधा ददधा॒न् नि मृ॒गेषु॑ मृ॒गेषु॒ न्य॑दधात् । \newline
7. न्य॑दधा ददधा॒न् नि न्य॑दधा॒थ् सा सा ऽद॑धा॒न् नि न्य॑दधा॒थ् सा । \newline
8. अ॒द॒धा॒थ् सा सा ऽद॑धा ददधा॒थ् सा कृ॑ष्णविषा॒णा कृ॑ष्णविषा॒णा सा ऽद॑धा ददधा॒थ् सा कृ॑ष्णविषा॒णा । \newline
9. सा कृ॑ष्णविषा॒णा कृ॑ष्णविषा॒णा सा सा कृ॑ष्णविषा॒णा ऽभ॑व दभवत् कृष्णविषा॒णा सा सा कृ॑ष्णविषा॒णा ऽभ॑वत् । \newline
10. कृ॒ष्ण॒वि॒षा॒णा ऽभ॑व दभवत् कृष्णविषा॒णा कृ॑ष्णविषा॒णा ऽभ॑व॒ दिन्द्र॒ स्येन्द्र॑स्याभवत् कृष्णविषा॒णा कृ॑ष्णविषा॒णा ऽभ॑व॒ दिन्द्र॑स्य । \newline
11. कृ॒ष्ण॒वि॒षा॒णेति॑ कृष्ण - वि॒षा॒णा । \newline
12. अ॒भ॒व॒ दिन्द्र॒ स्येन्द्र॑स्या भवद भव॒ दिन्द्र॑स्य॒ योनि॒र् योनि॒ रिन्द्र॑स्या भव दभव॒ दिन्द्र॑स्य॒ योनिः॑ । \newline
13. इन्द्र॑स्य॒ योनि॒र् योनि॒ रिन्द्र॒ स्येन्द्र॑स्य॒ योनि॑ रस्यसि॒ योनि॒ रिन्द्र॒ स्येन्द्र॑स्य॒ योनि॑रसि । \newline
14. योनि॑ रस्यसि॒ योनि॒र् योनि॑ रसि॒ मा मा ऽसि॒ योनि॒र् योनि॑ रसि॒ मा । \newline
15. अ॒सि॒ मा मा ऽस्य॑सि॒ मा मा॑ मा॒ मा ऽस्य॑सि॒ मा मा᳚ । \newline
16. मा मा॑ मा॒ मा मा मा॑ हिꣳसीर्. हिꣳसीर् मा॒ मा मा मा॑ हिꣳसीः । \newline
17. मा॒ हिꣳ॒॒सी॒र्॒. हिꣳ॒॒सी॒र् मा॒ मा॒ हिꣳ॒॒सी॒ रितीति॑ हिꣳसीर् मा मा हिꣳसी॒रिति॑ । \newline
18. हिꣳ॒॒सी॒ रितीति॑ हिꣳसीर्. हिꣳसी॒रिति॑ कृष्णविषा॒णाम् कृ॑ष्णविषा॒णा मिति॑ हिꣳसीर्. हिꣳसी॒रिति॑ कृष्णविषा॒णाम् । \newline
19. इति॑ कृष्णविषा॒णाम् कृ॑ष्णविषा॒णा मितीति॑ कृष्णविषा॒णाम् प्र प्र कृ॑ष्णविषा॒णा मितीति॑ कृष्णविषा॒णाम् प्र । \newline
20. कृ॒ष्ण॒वि॒षा॒णाम् प्र प्र कृ॑ष्णविषा॒णाम् कृ॑ष्णविषा॒णाम् प्र य॑च्छति यच्छति॒ प्र कृ॑ष्णविषा॒णाम् कृ॑ष्णविषा॒णाम् प्र य॑च्छति । \newline
21. कृ॒ष्ण॒वि॒षा॒णामिति॑ कृष्ण - वि॒षा॒णाम् । \newline
22. प्र य॑च्छति यच्छति॒ प्र प्र य॑च्छति॒ सयो॑निꣳ॒॒ सयो॑निं ॅयच्छति॒ प्र प्र य॑च्छति॒ सयो॑निम् । \newline
23. य॒च्छ॒ति॒ सयो॑निꣳ॒॒ सयो॑निं ॅयच्छति यच्छति॒ सयो॑नि मे॒वैव सयो॑निं ॅयच्छति यच्छति॒ सयो॑नि मे॒व । \newline
24. सयो॑नि मे॒वैव सयो॑निꣳ॒॒ सयो॑नि मे॒व य॒ज्ञ्ं ॅय॒ज्ञ् मे॒व सयो॑निꣳ॒॒ सयो॑नि मे॒व य॒ज्ञ्म् । \newline
25. सयो॑नि॒मिति॒ स - यो॒नि॒म् । \newline
26. ए॒व य॒ज्ञ्ं ॅय॒ज्ञ् मे॒वैव य॒ज्ञ्म् क॑रोति करोति य॒ज्ञ् मे॒वैव य॒ज्ञ्म् क॑रोति । \newline
27. य॒ज्ञ्म् क॑रोति करोति य॒ज्ञ्ं ॅय॒ज्ञ्म् क॑रोति॒ सयो॑निꣳ॒॒ सयो॑निम् करोति य॒ज्ञ्ं ॅय॒ज्ञ्म् क॑रोति॒ सयो॑निम् । \newline
28. क॒रो॒ति॒ सयो॑निꣳ॒॒ सयो॑निम् करोति करोति॒ सयो॑नि॒म् दक्षि॑णा॒म् दक्षि॑णाꣳ॒॒ सयो॑निम् करोति करोति॒ सयो॑नि॒म् दक्षि॑णाम् । \newline
29. सयो॑नि॒म् दक्षि॑णा॒म् दक्षि॑णाꣳ॒॒ सयो॑निꣳ॒॒ सयो॑नि॒म् दक्षि॑णाꣳ॒॒ सयो॑निꣳ॒॒ सयो॑नि॒म् दक्षि॑णाꣳ॒॒ सयो॑निꣳ॒॒ सयो॑नि॒म् दक्षि॑णाꣳ॒॒ सयो॑निम् । \newline
30. सयो॑नि॒मिति॒ स - यो॒नि॒म् । \newline
31. दक्षि॑णाꣳ॒॒ सयो॑निꣳ॒॒ सयो॑नि॒म् दक्षि॑णा॒म् दक्षि॑णाꣳ॒॒ सयो॑नि॒ मिन्द्र॒ मिन्द्रꣳ॒॒ सयो॑नि॒म् दक्षि॑णा॒म् दक्षि॑णाꣳ॒॒ सयो॑नि॒ मिन्द्र᳚म् । \newline
32. सयो॑नि॒ मिन्द्र॒ मिन्द्रꣳ॒॒ सयो॑निꣳ॒॒ सयो॑नि॒ मिन्द्रꣳ॑ सयोनि॒त्वाय॑ सयोनि॒त्वा येन्द्रꣳ॒॒ सयो॑निꣳ॒॒ सयो॑नि॒ मिन्द्रꣳ॑ सयोनि॒त्वाय॑ । \newline
33. सयो॑नि॒मिति॒ स - यो॒नि॒म् । \newline
34. इन्द्रꣳ॑ सयोनि॒त्वाय॑ सयोनि॒त्वायेन्द्र॒ मिन्द्रꣳ॑ सयोनि॒त्वाय॑ कृ॒ष्यै कृ॒ष्यै स॑योनि॒त्वायेन्द्र॒ मिन्द्रꣳ॑ सयोनि॒त्वाय॑ कृ॒ष्यै । \newline
35. स॒यो॒नि॒त्वाय॑ कृ॒ष्यै कृ॒ष्यै स॑योनि॒त्वाय॑ सयोनि॒त्वाय॑ कृ॒ष्यै त्वा᳚ त्वा कृ॒ष्यै स॑योनि॒त्वाय॑ सयोनि॒त्वाय॑ कृ॒ष्यै त्वा᳚ । \newline
36. स॒यो॒नि॒त्वायेति॑ सयोनि - त्वाय॑ । \newline
37. कृ॒ष्यै त्वा᳚ त्वा कृ॒ष्यै कृ॒ष्यै त्वा॑ सुस॒स्यायै॑ सुस॒स्यायै᳚ त्वा कृ॒ष्यै कृ॒ष्यै त्वा॑ सुस॒स्यायै᳚ । \newline
38. त्वा॒ सु॒स॒स्यायै॑ सुस॒स्यायै᳚ त्वा त्वा सुस॒स्याया॒ इतीति॑ सुस॒स्यायै᳚ त्वा त्वा सुस॒स्याया॒ इति॑ । \newline
39. सु॒स॒स्याया॒ इतीति॑ सुस॒स्यायै॑ सुस॒स्याया॒ इत्या॑हा॒हेति॑ सुस॒स्यायै॑ सुस॒स्याया॒ इत्या॑ह । \newline
40. सु॒स॒स्याया॒ इति॑ सु - स॒स्यायै᳚ । \newline
41. इत्या॑हा॒हे तीत्या॑ह॒ तस्मा॒त् तस्मा॑ दा॒हे तीत्या॑ह॒ तस्मा᳚त् । \newline
42. आ॒ह॒ तस्मा॒त् तस्मा॑ दाहाह॒ तस्मा॑ दकृष्टप॒च्या अ॑कृष्टप॒च्या स्तस्मा॑ दाहाह॒ तस्मा॑ दकृष्टप॒च्याः । \newline
43. तस्मा॑ दकृष्टप॒च्या अ॑कृष्टप॒च्या स्तस्मा॒त् तस्मा॑ दकृष्टप॒च्या ओष॑धय॒ ओष॑धयो ऽकृष्टप॒च्या स्तस्मा॒त् तस्मा॑ दकृष्टप॒च्या ओष॑धयः । \newline
44. अ॒कृ॒ष्ट॒प॒च्या ओष॑धय॒ ओष॑धयो ऽकृष्टप॒च्या अ॑कृष्टप॒च्या ओष॑धयः पच्यन्ते पच्यन्त॒ ओष॑धयो ऽकृष्टप॒च्या अ॑कृष्टप॒च्या ओष॑धयः पच्यन्ते । \newline
45. अ॒कृ॒ष्ट॒प॒च्या इत्य॑कृष्ट - प॒च्याः । \newline
46. ओष॑धयः पच्यन्ते पच्यन्त॒ ओष॑धय॒ ओष॑धयः पच्यन्ते सुपिप्प॒लाभ्यः॑ सुपिप्प॒लाभ्यः॑ पच्यन्त॒ ओष॑धय॒ ओष॑धयः पच्यन्ते सुपिप्प॒लाभ्यः॑ । \newline
47. प॒च्य॒न्ते॒ सु॒पि॒प्प॒लाभ्यः॑ सुपिप्प॒लाभ्यः॑ पच्यन्ते पच्यन्ते सुपिप्प॒लाभ्य॑ स्त्वा त्वा सुपिप्प॒लाभ्यः॑ पच्यन्ते पच्यन्ते सुपिप्प॒लाभ्य॑ स्त्वा । \newline
48. सु॒पि॒प्प॒लाभ्य॑ स्त्वा त्वा सुपिप्प॒लाभ्यः॑ सुपिप्प॒लाभ्य॒ स्त्वौष॑धीभ्य॒ ओष॑धीभ्य स्त्वा सुपिप्प॒लाभ्यः॑ सुपिप्प॒लाभ्य॒ स्त्वौष॑धीभ्यः । \newline
49. सु॒पि॒प्प॒लाभ्य॒ इति॑ सु - पि॒प्प॒लाभ्यः॑ । \newline
50. त्वौष॑धीभ्य॒ ओष॑धीभ्य स्त्वा॒ त्वौष॑धीभ्य॒ इती त्योष॑धीभ्य स्त्वा॒ त्वौष॑धीभ्य॒ इति॑ । \newline
51. ओष॑धीभ्य॒ इती त्योष॑धीभ्य॒ ओष॑धीभ्य॒ इत्या॑हा॒हे त्योष॑धीभ्य॒ ओष॑धीभ्य॒ इत्या॑ह । \newline
52. ओष॑धीभ्य॒ इत्योष॑धि - भ्यः॒ । \newline
53. इत्या॑हा॒हे तीत्या॑ह॒ तस्मा॒त् तस्मा॑ दा॒हे तीत्या॑ह॒ तस्मा᳚त् । \newline
54. आ॒ह॒ तस्मा॒त् तस्मा॑ दाहाह॒ तस्मा॒ दोष॑धय॒ ओष॑धय॒ स्तस्मा॑ दाहाह॒ तस्मा॒ दोष॑धयः । \newline
55. तस्मा॒ दोष॑धय॒ ओष॑धय॒ स्तस्मा॒त् तस्मा॒ दोष॑धयः॒ फल॒म् फल॒ मोष॑धय॒ स्तस्मा॒त् तस्मा॒ दोष॑धयः॒ फल᳚म् । \newline
56. ओष॑धयः॒ फल॒म् फल॒ मोष॑धय॒ ओष॑धयः॒ फल॑म् गृह्णन्ति गृह्णन्ति॒ फल॒ मोष॑धय॒ ओष॑धयः॒ फल॑म् गृह्णन्ति । \newline
57. फल॑म् गृह्णन्ति गृह्णन्ति॒ फल॒म् फल॑म् गृह्णन्ति॒ यद् यद् गृ॑ह्णन्ति॒ फल॒म् फल॑म् गृह्णन्ति॒ यत् । \newline
58. गृ॒ह्ण॒न्ति॒ यद् यद् गृ॑ह्णन्ति गृह्णन्ति॒ यद्धस्ते॑न॒ हस्ते॑न॒ यद् गृ॑ह्णन्ति गृह्णन्ति॒ यद्धस्ते॑न । \newline
59. यद्धस्ते॑न॒ हस्ते॑न॒ यद् यद्धस्ते॑न कण्डू॒येत॑ कण्डू॒येत॒ हस्ते॑न॒ यद् यद्धस्ते॑न कण्डू॒येत॑ । \newline
60. हस्ते॑न कण्डू॒येत॑ कण्डू॒येत॒ हस्ते॑न॒ हस्ते॑न कण्डू॒येत॑ पामन॒म्भावु॑काः पामन॒म्भावु॑काः कण्डू॒येत॒ हस्ते॑न॒ हस्ते॑न कण्डू॒येत॑ पामन॒म्भावु॑काः । \newline
\pagebreak
\markright{ TS 6.1.3.8  \hfill https://www.vedavms.in \hfill}

\section{ TS 6.1.3.8 }

\textbf{TS 6.1.3.8 } \newline
\textbf{Samhita Paata} \newline

कण्डू॒येत॑ पामनं॒ भावु॑काः प्र॒जाः स्यु॒र्यथ् स्मये॑त नग्नं॒ भावु॑काः कृष्णविषा॒णया॑ कण्डूयतेऽपि॒गृह्य॑ स्मयते प्र॒जानां᳚ गोपी॒थाय॒ न पु॒रा दक्षि॑णाभ्यो॒ नेतोः᳚ कृष्णविषा॒णामव॑ चृते॒द्यत् पु॒रा दक्षि॑णाभ्यो॒ नेतोः᳚ कृष्णविषा॒णाम॑व चृ॒तेद्योनिः॑ प्र॒जानां᳚ परा॒पातु॑का स्यान्नी॒तासु॒ दक्षि॑णासु॒ चात्वा॑ले कृष्णविषा॒णां प्रास्य॑ति॒ योनि॒र्वै य॒ज्ञ्स्य॒ चात्वा॑लं॒ ॅयोनिः॑ कृष्णविषा॒णा योना॑वे॒व योनिं॑ दधाति य॒ज्ञ्स्य॑ सयोनि॒त्वाय॑ ॥ \newline

\textbf{Pada Paata} \newline

क॒ण्डू॒येत॑ । पा॒म॒न॒म्भावु॑का॒ इति॑ पामनम् - भावु॑काः । प्र॒जा इति॑ प्र - जाः । स्युः॒ । यत् । स्मये॑त । न॒ग्न॒म्भावु॑का॒ इति॑ नग्नम् - भावु॑काः । कृ॒ष्ण॒वि॒षा॒णयेति॑ कृष्ण-वि॒षा॒णया᳚ । क॒ण्डू॒य॒ते॒ । अ॒पि॒गृह्येत्य॑पि - गृह्य॑ । स्म॒य॒ते॒ । प्र॒जाना॒मिति॑ प्र - जाना᳚म् । गो॒पी॒थाय॑ । न । पु॒रा । दक्षि॑णाभ्यः । नेतोः᳚ । कृ॒ष्ण॒वि॒षा॒णामिति॑ कृष्ण - वि॒षा॒णाम् । अवेति॑ । चृ॒ते॒त् । यत् । पु॒रा । दक्षि॑णाभ्यः । नेतोः᳚ । कृ॒ष्ण॒वि॒षा॒णामिति॑ कृष्ण-वि॒षा॒णाम् । अ॒व॒चृ॒तेदित्य॑व-चृ॒तेत् । योनिः॑ । प्र॒जाना॒मिति॑ प्र - जाना᳚म् । प॒रा॒पातु॒केति॑ परा - पातु॑का । स्या॒त् । नी॒तासु॑ । दक्षि॑णासु । चात्वा॑ले । कृ॒ष्ण॒वि॒षा॒णामिति॑ कृष्ण - वि॒षा॒णाम् । प्रेति॑ । अ॒स्य॒ति॒ । योनिः॑ । वै । य॒ज्ञ्स्य॑ । चात्वा॑लम् । योनिः॑ । कृ॒ष्ण॒वि॒षा॒णेति॑ कृष्ण - वि॒षा॒णा । योनौ᳚ । ए॒व । योनि᳚म् । द॒धा॒ति॒ । य॒ज्ञ्स्य॑ । स॒यो॒नि॒त्वायेति॑ सयोनि - त्वाय॑ ॥  \newline


\textbf{Krama Paata} \newline

क॒ण्डू॒येत॑ पामन॒म्भावु॑काः । पा॒म॒न॒म्भावु॑काः प्र॒जाः । पा॒म॒न॒म्भावु॑का॒ इति॑ पामनम् - भावु॑काः । प्र॒जाः स्युः॑ । प्र॒जा इति॑ प्र - जाः । स्यु॒र् यत् । यथ् स्मये॑त । स्मये॑त नग्न॒म्भावु॑काः । न॒ग्न॒म्भावु॑काः कृष्णविषा॒णया᳚ । न॒ग्न॒म्भावु॑का॒ इति॑ नग्नम् - भावु॑काः । कृ॒ष्ण॒वि॒षा॒णया॑ कण्डूयते । कृ॒ष्ण॒वि॒षा॒णयेति॑ कृष्ण - वि॒षा॒णया᳚ । क॒ण्डू॒य॒ते॒ऽपि॒गृह्य॑ । अ॒पि॒गृह्य॑ स्मयते । अ॒पि॒गृह्येत्य॑पि - गृह्य॑ । स्म॒य॒ते॒ प्र॒जाना᳚म् । प्र॒जाना᳚म् गोपी॒थाय॑ । प्र॒जाना॒मिति॑ प्र - जाना᳚म् । गो॒पी॒थाय॒ न । न पु॒रा । पु॒रा दक्षि॑णाभ्यः । दक्षि॑णाभ्यो॒ नेतोः᳚ । नेतोः᳚ कृष्णविषा॒णाम् । कृ॒ष्ण॒वि॒षा॒णामव॑ । कृ॒ष्ण॒वि॒षा॒णामिति॑ कृष्ण - वि॒षा॒णाम् । अव॑ चृतेत् । चृ॒ते॒द् यत् । यत् पु॒रा । पु॒रा दक्षि॑णाभ्यः । दक्षि॑णाभ्यो॒ नेतोः᳚ । नेतोः᳚ कृष्णविषा॒णाम् । कृ॒ष्ण॒वि॒षा॒णाम॑वचृ॒तेत् । कृ॒ष्ण॒वि॒षा॒णामिति॑ कृष्ण - वि॒षा॒णाम् । अ॒व॒चृ॒तेद् योनिः॑ । अ॒व॒चृ॒तेदित्य॑व - चृ॒तेत् । योनिः॑ प्र॒जाना᳚म् । प्र॒जाना᳚म् परा॒पातु॑का । प्र॒जाना॒मिति॑ प्र - जाना᳚म् । प॒रा॒पातु॑का स्यात् । प॒रा॒पातु॒केति॑ परा - पातु॑का । स्या॒न् नी॒तासु॑ । नी॒तासु॒ दक्षि॑णासु । दक्षि॑णासु॒ चात्वा॑ले । चात्वा॑ले कृष्णविषा॒णाम् । कृ॒ष्ण॒वि॒षा॒णाम् प्र । कृ॒ष्ण॒वि॒षा॒णामिति॑ कृष्ण - वि॒षा॒णाम् । प्रास्य॑ति । अ॒स्य॒ति॒ योनिः॑ । योनि॒र् वै । वै य॒ज्ञ्स्य॑ । य॒ज्ञ्स्य॒ चात्वा॑लम् । चात्वा॑ल॒म् ॅयोनिः॑ । योनिः॑ कृष्णविषा॒णा । कृ॒ष्ण॒वि॒षा॒णा योनौ᳚ । कृ॒ष्ण॒वि॒षा॒णेति॑ कृष्ण - वि॒षा॒णा । योना॑वे॒व । ए॒व योनि᳚म् । योनि॑म् दधाति । द॒धा॒ति॒ य॒ज्ञ्स्य॑ । य॒ज्ञ्स्य॑ सयोनि॒त्वाय॑ । स॒यो॒नि॒त्वायेति॑ सयोनि - त्वाय॑ । \newline

\textbf{Jatai Paata} \newline

1. क॒ण्डू॒येत॑ पामन॒म्भावु॑काः पामन॒म्भावु॑काः कण्डू॒येत॑ कण्डू॒येत॑ पामन॒म्भावु॑काः । \newline
2. पा॒म॒न॒म्भावु॑काः प्र॒जाः प्र॒जाः पा॑मन॒म्भावु॑काः पामन॒म्भावु॑काः प्र॒जाः । \newline
3. पा॒म॒न॒म्भावु॑का॒ इति॑ पामनम् - भावु॑काः । \newline
4. प्र॒जाः स्युः॑ स्युः प्र॒जाः प्र॒जाः स्युः॑ । \newline
5. प्र॒जा इति॑ प्र - जाः । \newline
6. स्यु॒र् यद् यथ् स्युः॑ स्यु॒र् यत् । \newline
7. यथ् स्मये॑त॒ स्मये॑त॒ यद् यथ् स्मये॑त । \newline
8. स्मये॑त नग्न॒म्भावु॑का नग्न॒म्भावु॑काः॒ स्मये॑त॒ स्मये॑त नग्न॒म्भावु॑काः । \newline
9. न॒ग्न॒म्भावु॑काः कृष्णविषा॒णया॑ कृष्णविषा॒णया॑ नग्न॒म्भावु॑का नग्न॒म्भावु॑काः कृष्णविषा॒णया᳚ । \newline
10. न॒ग्न॒म्भावु॑का॒ इति॑ नग्नम् - भावु॑काः । \newline
11. कृ॒ष्ण॒वि॒षा॒णया॑ कण्डूयते कण्डूयते कृष्णविषा॒णया॑ कृष्णविषा॒णया॑ कण्डूयते । \newline
12. कृ॒ष्ण॒वि॒षा॒णयेति॑ कृष्ण - वि॒षा॒णया᳚ । \newline
13. क॒ण्डू॒य॒ते॒ ऽपि॒गृह्या॑ पि॒गृह्य॑ कण्डूयते कण्डूयते ऽपि॒गृह्य॑ । \newline
14. अ॒पि॒गृह्य॑ स्मयते स्मयते ऽपि॒गृह्या॑ पि॒गृह्य॑ स्मयते । \newline
15. अ॒पि॒गृह्येत्य॑पि - गृह्य॑ । \newline
16. स्म॒य॒ते॒ प्र॒जाना᳚म् प्र॒जानाꣳ॑ स्मयते स्मयते प्र॒जाना᳚म् । \newline
17. प्र॒जाना᳚म् गोपी॒थाय॑ गोपी॒थाय॑ प्र॒जाना᳚म् प्र॒जाना᳚म् गोपी॒थाय॑ । \newline
18. प्र॒जाना॒मिति॑ प्र - जाना᳚म् । \newline
19. गो॒पी॒थाय॒ न न गो॑पी॒थाय॑ गोपी॒थाय॒ न । \newline
20. न पु॒रा पु॒रा न न पु॒रा । \newline
21. पु॒रा दक्षि॑णाभ्यो॒ दक्षि॑णाभ्यः पु॒रा पु॒रा दक्षि॑णाभ्यः । \newline
22. दक्षि॑णाभ्यो॒ नेतो॒र् नेतो॒र् दक्षि॑णाभ्यो॒ दक्षि॑णाभ्यो॒ नेतोः᳚ । \newline
23. नेतोः᳚ कृष्णविषा॒णाम् कृ॑ष्णविषा॒णान् नेतो॒र् नेतोः᳚ कृष्णविषा॒णाम् । \newline
24. कृ॒ष्ण॒वि॒षा॒णा मवाव॑ कृष्णविषा॒णाम् कृ॑ष्णविषा॒णा मव॑ । \newline
25. कृ॒ष्ण॒वि॒षा॒णामिति॑ कृष्ण - वि॒षा॒णाम् । \newline
26. अव॑ चृतेच् चृते॒ दवाव॑ चृतेत् । \newline
27. चृ॒ते॒द् यद् यच् चृ॑तेच् चृते॒द् यत् । \newline
28. यत् पु॒रा पु॒रा यद् यत् पु॒रा । \newline
29. पु॒रा दक्षि॑णाभ्यो॒ दक्षि॑णाभ्यः पु॒रा पु॒रा दक्षि॑णाभ्यः । \newline
30. दक्षि॑णाभ्यो॒ नेतो॒र् नेतो॒र् दक्षि॑णाभ्यो॒ दक्षि॑णाभ्यो॒ नेतोः᳚ । \newline
31. नेतोः᳚ कृष्णविषा॒णाम् कृ॑ष्णविषा॒णान् नेतो॒र् नेतोः᳚ कृष्णविषा॒णाम् । \newline
32. कृ॒ष्ण॒वि॒षा॒णा म॑वचृ॒ते द॑वचृ॒तेत् कृ॑ष्णविषा॒णाम् कृ॑ष्णविषा॒णा म॑वचृ॒तेत् । \newline
33. कृ॒ष्ण॒वि॒षा॒णामिति॑ कृष्ण - वि॒षा॒णाम् । \newline
34. अ॒व॒चृ॒तेद् योनि॒र् योनि॑ रवचृ॒ते द॑वचृ॒तेद् योनिः॑ । \newline
35. अ॒व॒चृ॒तेदित्य॑व - चृ॒तेत् । \newline
36. योनिः॑ प्र॒जाना᳚म् प्र॒जानां॒ ॅयोनि॒र् योनिः॑ प्र॒जाना᳚म् । \newline
37. प्र॒जाना᳚म् परा॒पातु॑का परा॒पातु॑का प्र॒जाना᳚म् प्र॒जाना᳚म् परा॒पातु॑का । \newline
38. प्र॒जाना॒मिति॑ प्र - जाना᳚म् । \newline
39. प॒रा॒पातु॑का स्याथ् स्यात् परा॒पातु॑का परा॒पातु॑का स्यात् । \newline
40. प॒रा॒पातु॒केति॑ परा - पातु॑का । \newline
41. स्या॒न् नी॒तासु॑ नी॒तासु॑ स्याथ् स्यान् नी॒तासु॑ । \newline
42. नी॒तासु॒ दक्षि॑णासु॒ दक्षि॑णासु नी॒तासु॑ नी॒तासु॒ दक्षि॑णासु । \newline
43. दक्षि॑णासु॒ चात्वा॑ले॒ चात्वा॑ले॒ दक्षि॑णासु॒ दक्षि॑णासु॒ चात्वा॑ले । \newline
44. चात्वा॑ले कृष्णविषा॒णाम् कृ॑ष्णविषा॒णाम् चात्वा॑ले॒ चात्वा॑ले कृष्णविषा॒णाम् । \newline
45. कृ॒ष्ण॒वि॒षा॒णाम् प्र प्र कृ॑ष्णविषा॒णाम् कृ॑ष्णविषा॒णाम् प्र । \newline
46. कृ॒ष्ण॒वि॒षा॒णामिति॑ कृष्ण - वि॒षा॒णाम् । \newline
47. प्रास्य॑ त्यस्यति॒ प्र प्रास्य॑ति । \newline
48. अ॒स्य॒ति॒ योनि॒र् योनि॑ रस्य त्यस्यति॒ योनिः॑ । \newline
49. योनि॒र् वै वै योनि॒र् योनि॒र् वै । \newline
50. वै य॒ज्ञ्स्य॑ य॒ज्ञ्स्य॒ वै वै य॒ज्ञ्स्य॑ । \newline
51. य॒ज्ञ्स्य॒ चात्वा॑ल॒म् चात्वा॑लं ॅय॒ज्ञ्स्य॑ य॒ज्ञ्स्य॒ चात्वा॑लम् । \newline
52. चात्वा॑लं॒ ॅयोनि॒र् योनि॒ श्चात्वा॑ल॒म् चात्वा॑लं॒ ॅयोनिः॑ । \newline
53. योनिः॑ कृष्णविषा॒णा कृ॑ष्णविषा॒णा योनि॒र् योनिः॑ कृष्णविषा॒णा । \newline
54. कृ॒ष्ण॒वि॒षा॒णा योनौ॒ योनौ॑ कृष्णविषा॒णा कृ॑ष्णविषा॒णा योनौ᳚ । \newline
55. कृ॒ष्ण॒वि॒षा॒णेति॑ कृष्ण - वि॒षा॒णा । \newline
56. योना॑ वे॒वैव योनौ॒ योना॑ वे॒व । \newline
57. ए॒व योनिं॒ ॅयोनि॑ मे॒वैव योनि᳚म् । \newline
58. योनि॑म् दधाति दधाति॒ योनिं॒ ॅयोनि॑म् दधाति । \newline
59. द॒धा॒ति॒ य॒ज्ञ्स्य॑ य॒ज्ञ्स्य॑ दधाति दधाति य॒ज्ञ्स्य॑ । \newline
60. य॒ज्ञ्स्य॑ सयोनि॒त्वाय॑ सयोनि॒त्वाय॑ य॒ज्ञ्स्य॑ य॒ज्ञ्स्य॑ सयोनि॒त्वाय॑ । \newline
61. स॒यो॒नि॒त्वायेति॑ सयोनि - त्वाय॑ । \newline

\textbf{Ghana Paata } \newline

1. क॒ण्डू॒येत॑ पामन॒म्भावु॑काः पामन॒म्भावु॑काः कण्डू॒येत॑ कण्डू॒येत॑ पामन॒म्भावु॑काः प्र॒जाः प्र॒जाः पा॑मन॒म्भावु॑काः कण्डू॒येत॑ कण्डू॒येत॑ पामन॒म्भावु॑काः प्र॒जाः । \newline
2. पा॒म॒न॒म्भावु॑काः प्र॒जाः प्र॒जाः पा॑मन॒म्भावु॑काः पामन॒म्भावु॑काः प्र॒जाः स्युः॑ स्युः प्र॒जाः पा॑मन॒म्भावु॑काः पामन॒म्भावु॑काः प्र॒जाः स्युः॑ । \newline
3. पा॒म॒न॒म्भावु॑का॒ इति॑ पामनम् - भावु॑काः । \newline
4. प्र॒जाः स्युः॑ स्युः प्र॒जाः प्र॒जाः स्यु॒र् यद् यथ् स्युः॑ प्र॒जाः प्र॒जाः स्यु॒र् यत् । \newline
5. प्र॒जा इति॑ प्र - जाः । \newline
6. स्यु॒र् यद् यथ् स्युः॑ स्यु॒र् यथ् स्मये॑त॒ स्मये॑त॒ यथ् स्युः॑ स्यु॒र् यथ् स्मये॑त । \newline
7. यथ् स्मये॑त॒ स्मये॑त॒ यद् यथ् स्मये॑त नग्न॒म्भावु॑का नग्न॒म्भावु॑काः॒ स्मये॑त॒ यद् यथ् स्मये॑त नग्न॒म्भावु॑काः । \newline
8. स्मये॑त नग्न॒म्भावु॑का नग्न॒म्भावु॑काः॒ स्मये॑त॒ स्मये॑त नग्न॒म्भावु॑काः कृष्णविषा॒णया॑ कृष्णविषा॒णया॑ नग्न॒म्भावु॑काः॒ स्मये॑त॒ स्मये॑त नग्न॒म्भावु॑काः कृष्णविषा॒णया᳚ । \newline
9. न॒ग्न॒म्भावु॑काः कृष्णविषा॒णया॑ कृष्णविषा॒णया॑ नग्न॒म्भावु॑का नग्न॒म्भावु॑काः कृष्णविषा॒णया॑ कण्डूयते कण्डूयते कृष्णविषा॒णया॑ नग्न॒म्भावु॑का नग्न॒म्भावु॑काः कृष्णविषा॒णया॑ कण्डूयते । \newline
10. न॒ग्न॒म्भावु॑का॒ इति॑ नग्नम् - भावु॑काः । \newline
11. कृ॒ष्ण॒वि॒षा॒णया॑ कण्डूयते कण्डूयते कृष्णविषा॒णया॑ कृष्णविषा॒णया॑ कण्डूयते ऽपि॒गृह्या॑ पि॒गृह्य॑ कण्डूयते कृष्णविषा॒णया॑ कृष्णविषा॒णया॑ कण्डूयते ऽपि॒गृह्य॑ । \newline
12. कृ॒ष्ण॒वि॒षा॒णयेति॑ कृष्ण - वि॒षा॒णया᳚ । \newline
13. क॒ण्डू॒य॒ते॒ ऽपि॒गृह्या॑ पि॒गृह्य॑ कण्डूयते कण्डूयते ऽपि॒गृह्य॑ स्मयते स्मयते ऽपि॒गृह्य॑ कण्डूयते कण्डूयते ऽपि॒गृह्य॑ स्मयते । \newline
14. अ॒पि॒गृह्य॑ स्मयते स्मयते ऽपि॒गृह्या॑ पि॒गृह्य॑ स्मयते प्र॒जाना᳚म् प्र॒जानाꣳ॑ स्मयते ऽपि॒गृह्या॑ पि॒गृह्य॑ स्मयते प्र॒जाना᳚म् । \newline
15. अ॒पि॒गृह्येत्य॑पि - गृह्य॑ । \newline
16. स्म॒य॒ते॒ प्र॒जाना᳚म् प्र॒जानाꣳ॑ स्मयते स्मयते प्र॒जाना᳚म् गोपी॒थाय॑ गोपी॒थाय॑ प्र॒जानाꣳ॑ स्मयते स्मयते प्र॒जाना᳚म् गोपी॒थाय॑ । \newline
17. प्र॒जाना᳚म् गोपी॒थाय॑ गोपी॒थाय॑ प्र॒जाना᳚म् प्र॒जाना᳚म् गोपी॒थाय॒ न न गो॑पी॒थाय॑ प्र॒जाना᳚म् प्र॒जाना᳚म् गोपी॒थाय॒ न । \newline
18. प्र॒जाना॒मिति॑ प्र - जाना᳚म् । \newline
19. गो॒पी॒थाय॒ न न गो॑पी॒थाय॑ गोपी॒थाय॒ न पु॒रा पु॒रा न गो॑पी॒थाय॑ गोपी॒थाय॒ न पु॒रा । \newline
20. न पु॒रा पु॒रा न न पु॒रा दक्षि॑णाभ्यो॒ दक्षि॑णाभ्यः पु॒रा न न पु॒रा दक्षि॑णाभ्यः । \newline
21. पु॒रा दक्षि॑णाभ्यो॒ दक्षि॑णाभ्यः पु॒रा पु॒रा दक्षि॑णाभ्यो॒ नेतो॒र् नेतो॒र् दक्षि॑णाभ्यः पु॒रा पु॒रा दक्षि॑णाभ्यो॒ नेतोः᳚ । \newline
22. दक्षि॑णाभ्यो॒ नेतो॒र् नेतो॒र् दक्षि॑णाभ्यो॒ दक्षि॑णाभ्यो॒ नेतोः᳚ कृष्णविषा॒णाम् कृ॑ष्णविषा॒णाम् नेतो॒र् दक्षि॑णाभ्यो॒ दक्षि॑णाभ्यो॒ नेतोः᳚ कृष्णविषा॒णाम् । \newline
23. नेतोः᳚ कृष्णविषा॒णाम् कृ॑ष्णविषा॒णाम् नेतो॒र् नेतोः᳚ कृष्णविषा॒णा मवाव॑ कृष्णविषा॒णाम् नेतो॒र् नेतोः᳚ कृष्णविषा॒णा मव॑ । \newline
24. कृ॒ष्ण॒वि॒षा॒णा मवाव॑ कृष्णविषा॒णाम् कृ॑ष्णविषा॒णा मव॑ चृतेच् चृते॒ दव॑ कृष्णविषा॒णाम् कृ॑ष्णविषा॒णा मव॑ चृतेत् । \newline
25. कृ॒ष्ण॒वि॒षा॒णामिति॑ कृष्ण - वि॒षा॒णाम् । \newline
26. अव॑ चृतेच् चृते॒ दवाव॑ चृते॒द् यद् यच् चृ॑ते॒ दवाव॑ चृते॒द् यत् । \newline
27. चृ॒ते॒द् यद् यच् चृ॑तेच् चृते॒द् यत् पु॒रा पु॒रा यच् चृ॑तेच् चृते॒द् यत् पु॒रा । \newline
28. यत् पु॒रा पु॒रा यद् यत् पु॒रा दक्षि॑णाभ्यो॒ दक्षि॑णाभ्यः पु॒रा यद् यत् पु॒रा दक्षि॑णाभ्यः । \newline
29. पु॒रा दक्षि॑णाभ्यो॒ दक्षि॑णाभ्यः पु॒रा पु॒रा दक्षि॑णाभ्यो॒ नेतो॒र् नेतो॒र् दक्षि॑णाभ्यः पु॒रा पु॒रा दक्षि॑णाभ्यो॒ नेतोः᳚ । \newline
30. दक्षि॑णाभ्यो॒ नेतो॒र् नेतो॒र् दक्षि॑णाभ्यो॒ दक्षि॑णाभ्यो॒ नेतोः᳚ कृष्णविषा॒णाम् कृ॑ष्णविषा॒णाम् नेतो॒र् दक्षि॑णाभ्यो॒ दक्षि॑णाभ्यो॒ नेतोः᳚ कृष्णविषा॒णाम् । \newline
31. नेतोः᳚ कृष्णविषा॒णाम् कृ॑ष्णविषा॒णाम् नेतो॒र् नेतोः᳚ कृष्णविषा॒णा म॑वचृ॒ते द॑वचृ॒तेत् कृ॑ष्णविषा॒णाम् नेतो॒र् नेतोः᳚ कृष्णविषा॒णा म॑वचृ॒तेत् । \newline
32. कृ॒ष्ण॒वि॒षा॒णा म॑वचृ॒ते द॑वचृ॒तेत् कृ॑ष्णविषा॒णाम् कृ॑ष्णविषा॒णा म॑वचृ॒तेद् योनि॒र् योनि॑ रवचृ॒तेत् कृ॑ष्णविषा॒णाम् कृ॑ष्णविषा॒णा म॑वचृ॒तेद् योनिः॑ । \newline
33. कृ॒ष्ण॒वि॒षा॒णामिति॑ कृष्ण - वि॒षा॒णाम् । \newline
34. अ॒व॒चृ॒तेद् योनि॒र् योनि॑ रवचृ॒ते द॑वचृ॒तेद् योनिः॑ प्र॒जाना᳚म् प्र॒जानां॒ ॅयोनि॑ रवचृ॒ते द॑वचृ॒तेद् योनिः॑ प्र॒जाना᳚म् । \newline
35. अ॒व॒चृ॒तेदित्य॑व - चृ॒तेत् । \newline
36. योनिः॑ प्र॒जाना᳚म् प्र॒जानां॒ ॅयोनि॒र् योनिः॑ प्र॒जाना᳚म् परा॒पातु॑का परा॒पातु॑का प्र॒जानां॒ ॅयोनि॒र् योनिः॑ प्र॒जाना᳚म् परा॒पातु॑का । \newline
37. प्र॒जाना᳚म् परा॒पातु॑का परा॒पातु॑का प्र॒जाना᳚म् प्र॒जाना᳚म् परा॒पातु॑का स्याथ् स्यात् परा॒पातु॑का प्र॒जाना᳚म् प्र॒जाना᳚म् परा॒पातु॑का स्यात् । \newline
38. प्र॒जाना॒मिति॑ प्र - जाना᳚म् । \newline
39. प॒रा॒पातु॑का स्याथ् स्यात् परा॒पातु॑का परा॒पातु॑का स्यान् नी॒तासु॑ नी॒तासु॑ स्यात् परा॒पातु॑का परा॒पातु॑का स्यान् नी॒तासु॑ । \newline
40. प॒रा॒पातु॒केति॑ परा - पातु॑का । \newline
41. स्या॒न् नी॒तासु॑ नी॒तासु॑ स्याथ् स्यान् नी॒तासु॒ दक्षि॑णासु॒ दक्षि॑णासु नी॒तासु॑ स्याथ् स्यान् नी॒तासु॒ दक्षि॑णासु । \newline
42. नी॒तासु॒ दक्षि॑णासु॒ दक्षि॑णासु नी॒तासु॑ नी॒तासु॒ दक्षि॑णासु॒ चात्वा॑ले॒ चात्वा॑ले॒ दक्षि॑णासु नी॒तासु॑ नी॒तासु॒ दक्षि॑णासु॒ चात्वा॑ले । \newline
43. दक्षि॑णासु॒ चात्वा॑ले॒ चात्वा॑ले॒ दक्षि॑णासु॒ दक्षि॑णासु॒ चात्वा॑ले कृष्णविषा॒णाम् कृ॑ष्णविषा॒णाम् चात्वा॑ले॒ दक्षि॑णासु॒ दक्षि॑णासु॒ चात्वा॑ले कृष्णविषा॒णाम् । \newline
44. चात्वा॑ले कृष्णविषा॒णाम् कृ॑ष्णविषा॒णाम् चात्वा॑ले॒ चात्वा॑ले कृष्णविषा॒णाम् प्र प्र कृ॑ष्णविषा॒णाम् चात्वा॑ले॒ चात्वा॑ले कृष्णविषा॒णाम् प्र । \newline
45. कृ॒ष्ण॒वि॒षा॒णाम् प्र प्र कृ॑ष्णविषा॒णाम् कृ॑ष्णविषा॒णाम् प्रास्य॑ त्यस्यति॒ प्र कृ॑ष्णविषा॒णाम् कृ॑ष्णविषा॒णाम् प्रास्य॑ति । \newline
46. कृ॒ष्ण॒वि॒षा॒णामिति॑ कृष्ण - वि॒षा॒णाम् । \newline
47. प्रास्य॑ त्यस्यति॒ प्र प्रास्य॑ति॒ योनि॒र् योनि॑ रस्यति॒ प्र प्रास्य॑ति॒ योनिः॑ । \newline
48. अ॒स्य॒ति॒ योनि॒र् योनि॑ रस्य त्यस्यति॒ योनि॒र् वै वै योनि॑ रस्य त्यस्यति॒ योनि॒र् वै । \newline
49. योनि॒र् वै वै योनि॒र् योनि॒र् वै य॒ज्ञ्स्य॑ य॒ज्ञ्स्य॒ वै योनि॒र् योनि॒र् वै य॒ज्ञ्स्य॑ । \newline
50. वै य॒ज्ञ्स्य॑ य॒ज्ञ्स्य॒ वै वै य॒ज्ञ्स्य॒ चात्वा॑ल॒म् चात्वा॑लं ॅय॒ज्ञ्स्य॒ वै वै य॒ज्ञ्स्य॒ चात्वा॑लम् । \newline
51. य॒ज्ञ्स्य॒ चात्वा॑ल॒म् चात्वा॑लं ॅय॒ज्ञ्स्य॑ य॒ज्ञ्स्य॒ चात्वा॑लं॒ ॅयोनि॒र् योनि॒ श्चात्वा॑लं ॅय॒ज्ञ्स्य॑ य॒ज्ञ्स्य॒ चात्वा॑लं॒ ॅयोनिः॑ । \newline
52. चात्वा॑लं॒ ॅयोनि॒र् योनि॒ श्चात्वा॑ल॒म् चात्वा॑लं॒ ॅयोनिः॑ कृष्णविषा॒णा कृ॑ष्णविषा॒णा योनि॒ श्चात्वा॑ल॒म् चात्वा॑लं॒ ॅयोनिः॑ कृष्णविषा॒णा । \newline
53. योनिः॑ कृष्णविषा॒णा कृ॑ष्णविषा॒णा योनि॒र् योनिः॑ कृष्णविषा॒णा योनौ॒ योनौ॑ कृष्णविषा॒णा योनि॒र् योनिः॑ कृष्णविषा॒णा योनौ᳚ । \newline
54. कृ॒ष्ण॒वि॒षा॒णा योनौ॒ योनौ॑ कृष्णविषा॒णा कृ॑ष्णविषा॒णा योना॑ वे॒वैव योनौ॑ कृष्णविषा॒णा कृ॑ष्णविषा॒णा योना॑ वे॒व । \newline
55. कृ॒ष्ण॒वि॒षा॒णेति॑ कृष्ण - वि॒षा॒णा । \newline
56. योना॑ वे॒वैव योनौ॒ योना॑ वे॒व योनिं॒ ॅयोनि॑ मे॒व योनौ॒ योना॑ वे॒व योनि᳚म् । \newline
57. ए॒व योनिं॒ ॅयोनि॑ मे॒वैव योनि॑म् दधाति दधाति॒ योनि॑ मे॒वैव योनि॑म् दधाति । \newline
58. योनि॑म् दधाति दधाति॒ योनिं॒ ॅयोनि॑म् दधाति य॒ज्ञ्स्य॑ य॒ज्ञ्स्य॑ दधाति॒ योनिं॒ ॅयोनि॑म् दधाति य॒ज्ञ्स्य॑ । \newline
59. द॒धा॒ति॒ य॒ज्ञ्स्य॑ य॒ज्ञ्स्य॑ दधाति दधाति य॒ज्ञ्स्य॑ सयोनि॒त्वाय॑ सयोनि॒त्वाय॑ य॒ज्ञ्स्य॑ दधाति दधाति य॒ज्ञ्स्य॑ सयोनि॒त्वाय॑ । \newline
60. य॒ज्ञ्स्य॑ सयोनि॒त्वाय॑ सयोनि॒त्वाय॑ य॒ज्ञ्स्य॑ य॒ज्ञ्स्य॑ सयोनि॒त्वाय॑ । \newline
61. स॒यो॒नि॒त्वायेति॑ सयोनि - त्वाय॑ । \newline
\pagebreak
\markright{ TS 6.1.4.1  \hfill https://www.vedavms.in \hfill}

\section{ TS 6.1.4.1 }

\textbf{TS 6.1.4.1 } \newline
\textbf{Samhita Paata} \newline

वाग्वै दे॒वेभ्यो ऽपा᳚क्रामद्-य॒ज्ञायाति॑ष्ठमाना॒ सा वन॒स्पती॒न् प्रावि॑श॒थ् सैषा वाग्वन॒स्पति॑षु वदति॒ या दु॑न्दु॒भौ या तूण॑वे॒ या वीणा॑यां॒ ॅयद्-दी᳚क्षितद॒ण्डं प्र॒यच्छ॑ति॒ वाच॑मे॒वाव॑ रुन्ध॒ औदु॑बंरो भव॒त्यूर्ग्वा उ॑दु॒बंर॒ ऊर्ज॑मे॒वाव॑ रुन्धे॒ मुखे॑न॒ संमि॑तो भवति मुख॒त ए॒वास्मा॒ ऊर्जं॑ दधाति॒ तस्मा᳚न् मुख॒त ऊ॒र्जा भु॑ञ्जते - [  ] \newline

\textbf{Pada Paata} \newline

वाक् । वै । दे॒वेभ्यः॑ । अपेति॑ । अ॒क्रा॒म॒त् । य॒ज्ञाय॑ । अति॑ष्ठमाना । सा । वन॒स्पतीन्॑ । प्रेति॑ । अ॒वि॒श॒त् । सा । ए॒षा । वाक् । वन॒स्पति॑षु । व॒द॒ति॒ । या । दु॒न्दु॒भौ । या । तूण॑वे । या । वीणा॑याम् । यत् । दी॒क्षि॒त॒द॒ण्डमिति॑ दीक्षित-द॒ण्डम् । प्र॒यच्छ॒तीति॑ प्र-यच्छ॑ति । वाच᳚म् । ए॒व । अवेति॑ । रु॒न्धे॒ । औदु॑म्बरः । भ॒व॒ति॒ । ऊर्क् । वै । उ॒दु॒बंरः॑ । ऊर्ज᳚म् । ए॒व । अवेति॑ । रु॒न्धे॒ । मुखे॑न । सम्मि॑त॒ इति॒ सं - मि॒तः॒ । भ॒व॒ति॒ । मु॒ख॒तः । ए॒व । अ॒स्मै॒ । ऊर्ज᳚म् । द॒धा॒ति॒ । तस्मा᳚त् । मु॒ख॒तः । ऊ॒र्जा । भु॒ञ्ज॒ते॒ ।  \newline


\textbf{Krama Paata} \newline

वाग् वै । वै दे॒वेभ्यः॑ । दे॒वेभ्योऽप॑ । अपा᳚क्रामत् । अ॒क्रा॒म॒द् य॒ज्ञाय॑ । य॒ज्ञायाति॑ष्ठमाना । अति॑ष्ठमाना॒ सा । सा वन॒स्पतीन्॑ । वन॒स्पती॒न् प्र । प्रावि॑शत् । अ॒वि॒श॒थ् सा । सैषा । ए॒षा वाक् । वाग् वन॒स्पति॑षु । वन॒स्पति॑षु वदति । व॒द॒ति॒ या । या दु॑न्दु॒भौ । दु॒न्दु॒भौ या । या तूण॑वे । तूण॑वे॒ या । या वीणा॑याम् । वीणा॑या॒म् ॅयत् । यद् दी᳚क्षितद॒ण्डम् । दी॒क्षि॒त॒द॒ण्डम् प्र॒यच्छ॑ति । दी॒क्षि॒त॒द॒ण्डमिति॑ दीक्षित - द॒ण्डम् । प्र॒यच्छ॑ति॒ वाच᳚म् । प्र॒यच्छ॒तीति॑ प्र - यच्छ॑ति । वाच॑मे॒व । ए॒वाव॑ । अव॑ रुन्धे । रु॒न्ध॒ औदु॑म्बरः । औदु॑म्बरो भवति । भ॒व॒त्यूर्क् । ऊर्ग् वै । वा उ॑दु॒म्बरः॑ । उ॒दु॒म्बर॒ ऊर्ज᳚म् । ऊर्ज॑मे॒व । ए॒वाव॑ । अव॑ रुन्धे । रु॒न्धे॒ मुखे॑न । मुखे॑न॒ सम्मि॑तः । सम्मि॑तो भवति । सम्मि॑त॒ इति॒ सम् - मि॒तः॒ । भ॒व॒ति॒ मु॒ख॒तः । मु॒ख॒त ए॒व । ए॒वास्मै᳚ । अ॒स्मा॒ ऊर्ज᳚म् । ऊर्ज॑म् दधाति । द॒धा॒ति॒ तस्मा᳚त् । तस्मा᳚न् मुख॒तः । मु॒ख॒त ऊ॒र्जा । ऊ॒र्जा भु॑ञ्जते । भु॒ञ्ज॒ते॒ क्री॒ते \newline

\textbf{Jatai Paata} \newline

1. वाग् वै वै वाग् वाग् वै । \newline
2. वै दे॒वेभ्यो॑ दे॒वेभ्यो॒ वै वै दे॒वेभ्यः॑ । \newline
3. दे॒वेभ्यो ऽपाप॑ दे॒वेभ्यो॑ दे॒वेभ्यो ऽप॑ । \newline
4. अपा᳚क्राम दक्राम॒ दपापा᳚ क्रामत् । \newline
5. अ॒क्रा॒म॒द् य॒ज्ञाय॑ य॒ज्ञाया᳚ क्राम दक्रामद् य॒ज्ञाय॑ । \newline
6. य॒ज्ञाया ति॑ष्ठमा॒ना ऽति॑ष्ठमाना य॒ज्ञाय॑ य॒ज्ञाया ति॑ष्ठमाना । \newline
7. अति॑ष्ठमाना॒ सा सा ऽति॑ष्ठमा॒ना ऽति॑ष्ठमाना॒ सा । \newline
8. सा वन॒स्पती॒न्॒. वन॒स्पती॒न् थ्सा सा वन॒स्पतीन्॑ । \newline
9. वन॒स्पती॒न् प्र प्र वण॒स्पती॒न्॒. वन॒स्पती॒न् प्र । \newline
10. प्रावि॑श दविश॒त् प्र प्रावि॑शत् । \newline
11. अ॒वि॒श॒थ् सा सा ऽवि॑श दविश॒थ् सा । \newline
12. सैषैषा सा सैषा । \newline
13. ए॒षा वाग् वागे॒ षैषा वाक् । \newline
14. वाग् वन॒स्पति॑षु॒ वन॒स्पति॑षु॒ वाग् वाग् वन॒स्पति॑षु । \newline
15. वन॒स्पति॑षु वदति वदति॒ वन॒स्पति॑षु॒ वन॒स्पति॑षु वदति । \newline
16. व॒द॒ति॒ या या व॑दति वदति॒ या । \newline
17. या दु॑न्दु॒भौ दु॑न्दु॒भौ या या दु॑न्दु॒भौ । \newline
18. दु॒न्दु॒भौ या या दु॑न्दु॒भौ दु॑न्दु॒भौ या । \newline
19. या तूण॑वे॒ तूण॑वे॒ या या तूण॑वे । \newline
20. तूण॑वे॒ या या तूण॑वे॒ तूण॑वे॒ या । \newline
21. या वीणा॑यां॒ ॅवीणा॑यां॒ ॅया या वीणा॑याम् । \newline
22. वीणा॑यां॒ ॅयद् यद् वीणा॑यां॒ ॅवीणा॑यां॒ ॅयत् । \newline
23. यद् दी᳚क्षितद॒ण्डम् दी᳚क्षितद॒ण्डं ॅयद् यद् दी᳚क्षितद॒ण्डम् । \newline
24. दी॒क्षि॒त॒द॒ण्डम् प्र॒यच्छ॑ति प्र॒यच्छ॑ति दीक्षितद॒ण्डम् दी᳚क्षितद॒ण्डम् प्र॒यच्छ॑ति । \newline
25. दी॒क्षि॒त॒द॒ण्डमिति॑ दीक्षित - द॒ण्डम् । \newline
26. प्र॒यच्छ॑ति॒ वाचं॒ ॅवाच॑म् प्र॒यच्छ॑ति प्र॒यच्छ॑ति॒ वाच᳚म् । \newline
27. प्र॒यच्छ॒तीति॑ प्र - यच्छ॑ति । \newline
28. वाच॑ मे॒वैव वाचं॒ ॅवाच॑ मे॒व । \newline
29. ए॒वावा वै॒वै वाव॑ । \newline
30. अव॑ रुन्धे रु॒न्धे ऽवाव॑ रुन्धे । \newline
31. रु॒न्ध॒ औदु॑म्बर॒ औदु॑म्बरो रुन्धे रुन्ध॒ औदु॑म्बरः । \newline
32. औदु॑म्बरो भवति भव॒ त्यौदु॑म्बर॒ औदु॑म्बरो भवति । \newline
33. भ॒व॒ त्यूर् गूर्ग् भ॑वति भव॒ त्यूर्क् । \newline
34. ऊर्ग् वै वा ऊर् गूर्ग् वै । \newline
35. वा उ॑दुं॒बर॑ उदुं॒बरो॒ वै वा उ॑दुं॒बरः॑ । \newline
36. उ॒दुं॒बर॒ ऊर्ज॒ मूर्ज॑ मुदुं॒बर॑ उदुं॒बर॒ ऊर्ज᳚म् । \newline
37. ऊर्ज॑ मे॒वै वोर्ज॒ मूर्ज॑ मे॒व । \newline
38. ए॒वावा वै॒वै वाव॑ । \newline
39. अव॑ रुन्धे रु॒न्धे ऽवाव॑ रुन्धे । \newline
40. रु॒न्धे॒ मुखे॑न॒ मुखे॑न रुन्धे रुन्धे॒ मुखे॑न । \newline
41. मुखे॑न॒ सम्मि॑तः॒ सम्मि॑तो॒ मुखे॑न॒ मुखे॑न॒ सम्मि॑तः । \newline
42. सम्मि॑तो भवति भवति॒ सम्मि॑तः॒ सम्मि॑तो भवति । \newline
43. सम्मि॑त॒ इति॒ सं - मि॒तः॒ । \newline
44. भ॒व॒ति॒ मु॒ख॒तो मु॑ख॒तो भ॑वति भवति मुख॒तः । \newline
45. मु॒ख॒त ए॒वैव मु॑ख॒तो मु॑ख॒त ए॒व । \newline
46. ए॒वास्मा॑ अस्मा ए॒वै वास्मै᳚ । \newline
47. अ॒स्मा॒ ऊर्ज॒ मूर्ज॑ मस्मा अस्मा॒ ऊर्ज᳚म् । \newline
48. ऊर्ज॑म् दधाति दधा॒ त्यूर्ज॒ मूर्ज॑म् दधाति । \newline
49. द॒धा॒ति॒ तस्मा॒त् तस्मा᳚द् दधाति दधाति॒ तस्मा᳚त् । \newline
50. तस्मा᳚न् मुख॒तो मु॑ख॒त स्तस्मा॒त् तस्मा᳚न् मुख॒तः । \newline
51. मु॒ख॒त ऊ॒र्जोर्जा मु॑ख॒तो मु॑ख॒त ऊ॒र्जा । \newline
52. ऊ॒र्जा भु॑ञ्जते भुञ्जत ऊ॒र्जोर्जा भु॑ञ्जते । \newline
53. भु॒ञ्ज॒ते॒ क्री॒ते क्री॒ते भु॑ञ्जते भुञ्जते क्री॒ते । \newline

\textbf{Ghana Paata } \newline

1. वाग् वै वै वाग् वाग् वै दे॒वेभ्यो॑ दे॒वेभ्यो॒ वै वाग् वाग् वै दे॒वेभ्यः॑ । \newline
2. वै दे॒वेभ्यो॑ दे॒वेभ्यो॒ वै वै दे॒वेभ्यो ऽपाप॑ दे॒वेभ्यो॒ वै वै दे॒वेभ्यो ऽप॑ । \newline
3. दे॒वेभ्यो ऽपाप॑ दे॒वेभ्यो॑ दे॒वेभ्यो ऽपा᳚क्राम दक्राम॒ दप॑ दे॒वेभ्यो॑ दे॒वेभ्यो ऽपा᳚क्रामत् । \newline
4. अपा᳚क्राम दक्राम॒ दपापा᳚ क्रामद् य॒ज्ञाय॑ य॒ज्ञाया᳚ क्राम॒ दपापा᳚ क्रामद् य॒ज्ञाय॑ । \newline
5. अ॒क्रा॒म॒द् य॒ज्ञाय॑ य॒ज्ञाया᳚क्राम दक्रामद् य॒ज्ञाया ति॑ष्ठमा॒ना ऽति॑ष्ठमाना य॒ज्ञाया᳚ क्राम दक्रामद् य॒ज्ञाया ति॑ष्ठमाना । \newline
6. य॒ज्ञाया ति॑ष्ठमा॒ना ऽति॑ष्ठमाना य॒ज्ञाय॑ य॒ज्ञाया ति॑ष्ठमाना॒ सा सा ऽति॑ष्ठमाना य॒ज्ञाय॑ य॒ज्ञाया ति॑ष्ठमाना॒ सा । \newline
7. अति॑ष्ठमाना॒ सा सा ऽति॑ष्ठमा॒ना ऽति॑ष्ठमाना॒ सा वन॒स्पती॒न्॒. वन॒स्पती॒न् थ्सा ऽति॑ष्ठमा॒ना ऽति॑ष्ठमाना॒ सा वन॒स्पतीन्॑ । \newline
8. सा वन॒स्पती॒न्॒. वन॒स्पती॒न् थ्सा सा वन॒स्पती॒न् प्र प्र वण॒स्पती॒न् थ्सा सा वन॒स्पती॒न् प्र । \newline
9. वन॒स्पती॒न् प्र प्र वण॒स्पती॒न्॒. वन॒स्पती॒न् प्रावि॑श दविश॒त् प्र वण॒स्पती॒न्॒. वन॒स्पती॒न् प्रावि॑शत् । \newline
10. प्रावि॑श दविश॒त् प्र प्रावि॑श॒थ् सा सा ऽवि॑श॒त् प्र प्रावि॑श॒थ् सा । \newline
11. अ॒वि॒श॒थ् सा सा ऽवि॑श दविश॒थ् सैषैषा सा ऽवि॑श दविश॒थ् सैषा । \newline
12. सैषैषा सा सैषा वाग् वागे॒षा सा सैषा वाक् । \newline
13. ए॒षा वाग् वागे॒ षैषा वाग् वन॒स्पति॑षु॒ वन॒स्पति॑षु॒ वागे॒ षैषा वाग् वन॒स्पति॑षु । \newline
14. वाग् वन॒स्पति॑षु॒ वन॒स्पति॑षु॒ वाग् वाग् वन॒स्पति॑षु वदति वदति॒ वन॒स्पति॑षु॒ वाग् वाग् वन॒स्पति॑षु वदति । \newline
15. वन॒स्पति॑षु वदति वदति॒ वन॒स्पति॑षु॒ वन॒स्पति॑षु वदति॒ या या व॑दति॒ वन॒स्पति॑षु॒ वन॒स्पति॑षु वदति॒ या । \newline
16. व॒द॒ति॒ या या व॑दति वदति॒ या दु॑न्दु॒भौ दु॑न्दु॒भौ या व॑दति वदति॒ या दु॑न्दु॒भौ । \newline
17. या दु॑न्दु॒भौ दु॑न्दु॒भौ या या दु॑न्दु॒भौ या या दु॑न्दु॒भौ या या दु॑न्दु॒भौ या । \newline
18. दु॒न्दु॒भौ या या दु॑न्दु॒भौ दु॑न्दु॒भौ या तूण॑वे॒ तूण॑वे॒ या दु॑न्दु॒भौ दु॑न्दु॒भौ या तूण॑वे । \newline
19. या तूण॑वे॒ तूण॑वे॒ या या तूण॑वे॒ या या तूण॑वे॒ या या तूण॑वे॒ या । \newline
20. तूण॑वे॒ या या तूण॑वे॒ तूण॑वे॒ या वीणा॑यां॒ ॅवीणा॑यां॒ ॅया तूण॑वे॒ तूण॑वे॒ या वीणा॑याम् । \newline
21. या वीणा॑यां॒ ॅवीणा॑यां॒ ॅया या वीणा॑यां॒ ॅयद् यद् वीणा॑यां॒ ॅया या वीणा॑यां॒ ॅयत् । \newline
22. वीणा॑यां॒ ॅयद् यद् वीणा॑यां॒ ॅवीणा॑यां॒ ॅयद् दी᳚क्षितद॒ण्डम् दी᳚क्षितद॒ण्डं ॅयद् वीणा॑यां॒ ॅवीणा॑यां॒ ॅयद् दी᳚क्षितद॒ण्डम् । \newline
23. यद् दी᳚क्षितद॒ण्डम् दी᳚क्षितद॒ण्डं ॅयद् यद् दी᳚क्षितद॒ण्डम् प्र॒यच्छ॑ति प्र॒यच्छ॑ति दीक्षितद॒ण्डं ॅयद् यद् दी᳚क्षितद॒ण्डम् प्र॒यच्छ॑ति । \newline
24. दी॒क्षि॒त॒द॒ण्डम् प्र॒यच्छ॑ति प्र॒यच्छ॑ति दीक्षितद॒ण्डम् दी᳚क्षितद॒ण्डम् प्र॒यच्छ॑ति॒ वाचं॒ ॅवाच॑म् प्र॒यच्छ॑ति दीक्षितद॒ण्डम् दी᳚क्षितद॒ण्डम् प्र॒यच्छ॑ति॒ वाच᳚म् । \newline
25. दी॒क्षि॒त॒द॒ण्डमिति॑ दीक्षित - द॒ण्डम् । \newline
26. प्र॒यच्छ॑ति॒ वाचं॒ ॅवाच॑म् प्र॒यच्छ॑ति प्र॒यच्छ॑ति॒ वाच॑ मे॒वैव वाच॑म् प्र॒यच्छ॑ति प्र॒यच्छ॑ति॒ वाच॑ मे॒व । \newline
27. प्र॒यच्छ॒तीति॑ प्र - यच्छ॑ति । \newline
28. वाच॑ मे॒वैव वाचं॒ ॅवाच॑ मे॒वावा वै॒व वाचं॒ ॅवाच॑ मे॒वाव॑ । \newline
29. ए॒वावा वै॒वै वाव॑ रुन्धे रु॒न्धे ऽवै॒वै वाव॑ रुन्धे । \newline
30. अव॑ रुन्धे रु॒न्धे ऽवाव॑ रुन्ध॒ औदु॑म्बर॒ औदु॑म्बरो रु॒न्धे ऽवाव॑ रुन्ध॒ औदु॑म्बरः । \newline
31. रु॒न्ध॒ औदु॑म्बर॒ औदु॑म्बरो रुन्धे रुन्ध॒ औदु॑म्बरो भवति भव॒ त्यौदु॑म्बरो रुन्धे रुन्ध॒ औदु॑म्बरो भवति । \newline
32. औदु॑म्बरो भवति भव॒ त्यौदु॑म्बर॒ औदु॑म्बरो भव॒ त्यूर्गूर्ग् भ॑व॒ त्यौदु॑म्बर॒ औदु॑म्बरो भव॒त्यूर्क् । \newline
33. भ॒व॒ त्यूर् गूर्ग् भ॑वति भव॒ त्यूर्ग् वै वा ऊर्ग् भ॑वति भव॒ त्यूर्ग् वै । \newline
34. ऊर्ग् वै वा ऊर्गूर्ग् वा उ॑दुं॒बर॑ उदुं॒बरो॒ वा ऊर् गूर्ग् वा उ॑दुं॒बरः॑ । \newline
35. वा उ॑दुं॒बर॑ उदुं॒बरो॒ वै वा उ॑दुं॒बर॒ ऊर्ज॒ मूर्ज॑ मुदुं॒बरो॒ वै वा उ॑दुं॒बर॒ ऊर्ज᳚म् । \newline
36. उ॒दुं॒बर॒ ऊर्ज॒ मूर्ज॑ मुदुं॒बर॑ उदुं॒बर॒ ऊर्ज॑ मे॒वै वोर्ज॑ मुदुं॒बर॑ उदुं॒बर॒ ऊर्ज॑ मे॒व । \newline
37. ऊर्ज॑ मे॒वै वोर्ज॒ मूर्ज॑ मे॒वा वावै॒ वोर्ज॒ मूर्ज॑ मे॒वाव॑ । \newline
38. ए॒वावा वै॒वै वाव॑ रुन्धे रु॒न्धे ऽवै॒वै वाव॑ रुन्धे । \newline
39. अव॑ रुन्धे रु॒न्धे ऽवाव॑ रुन्धे॒ मुखे॑न॒ मुखे॑न रु॒न्धे ऽवाव॑ रुन्धे॒ मुखे॑न । \newline
40. रु॒न्धे॒ मुखे॑न॒ मुखे॑न रुन्धे रुन्धे॒ मुखे॑न॒ सम्मि॑तः॒ सम्मि॑तो॒ मुखे॑न रुन्धे रुन्धे॒ मुखे॑न॒ सम्मि॑तः । \newline
41. मुखे॑न॒ सम्मि॑तः॒ सम्मि॑तो॒ मुखे॑न॒ मुखे॑न॒ सम्मि॑तो भवति भवति॒ सम्मि॑तो॒ मुखे॑न॒ मुखे॑न॒ सम्मि॑तो भवति । \newline
42. सम्मि॑तो भवति भवति॒ सम्मि॑तः॒ सम्मि॑तो भवति मुख॒तो मु॑ख॒तो भ॑वति॒ सम्मि॑तः॒ सम्मि॑तो भवति मुख॒तः । \newline
43. सम्मि॑त॒ इति॒ सं - मि॒तः॒ । \newline
44. भ॒व॒ति॒ मु॒ख॒तो मु॑ख॒तो भ॑वति भवति मुख॒त ए॒वैव मु॑ख॒तो भ॑वति भवति मुख॒त ए॒व । \newline
45. मु॒ख॒त ए॒वैव मु॑ख॒तो मु॑ख॒त ए॒वास्मा॑ अस्मा ए॒व मु॑ख॒तो मु॑ख॒त ए॒वास्मै᳚ । \newline
46. ए॒वास्मा॑ अस्मा ए॒वै वास्मा॒ ऊर्ज॒ मूर्ज॑ मस्मा ए॒वै वास्मा॒ ऊर्ज᳚म् । \newline
47. अ॒स्मा॒ ऊर्ज॒ मूर्ज॑ मस्मा अस्मा॒ ऊर्ज॑म् दधाति दधा॒ त्यूर्ज॑ मस्मा अस्मा॒ ऊर्ज॑म् दधाति । \newline
48. ऊर्ज॑म् दधाति दधा॒ त्यूर्ज॒ मूर्ज॑म् दधाति॒ तस्मा॒त् तस्मा᳚द् दधा॒ त्यूर्ज॒ मूर्ज॑म् दधाति॒ तस्मा᳚त् । \newline
49. द॒धा॒ति॒ तस्मा॒त् तस्मा᳚द् दधाति दधाति॒ तस्मा᳚न् मुख॒तो मु॑ख॒त स्तस्मा᳚द् दधाति दधाति॒ तस्मा᳚न् मुख॒तः । \newline
50. तस्मा᳚न् मुख॒तो मु॑ख॒त स्तस्मा॒त् तस्मा᳚न् मुख॒त ऊ॒र्जोर्जा मु॑ख॒त स्तस्मा॒त् तस्मा᳚न् मुख॒त ऊ॒र्जा । \newline
51. मु॒ख॒त ऊ॒र्जोर्जा मु॑ख॒तो मु॑ख॒त ऊ॒र्जा भु॑ञ्जते भुञ्जत ऊ॒र्जा मु॑ख॒तो मु॑ख॒त ऊ॒र्जा भु॑ञ्जते । \newline
52. ऊ॒र्जा भु॑ञ्जते भुञ्जत ऊ॒र्जोर्जा भु॑ञ्जते क्री॒ते क्री॒ते भु॑ञ्जत ऊ॒र्जोर्जा भु॑ञ्जते क्री॒ते । \newline
53. भु॒ञ्ज॒ते॒ क्री॒ते क्री॒ते भु॑ञ्जते भुञ्जते क्री॒ते सोमे॒ सोमे᳚ क्री॒ते भु॑ञ्जते भुञ्जते क्री॒ते सोमे᳚ । \newline
\pagebreak
\markright{ TS 6.1.4.2  \hfill https://www.vedavms.in \hfill}

\section{ TS 6.1.4.2 }

\textbf{TS 6.1.4.2 } \newline
\textbf{Samhita Paata} \newline

क्री॒ते सोमे॑ मैत्रावरु॒णाय॑ द॒ण्डं प्र य॑च्छति मैत्रावरु॒णो हि पु॒रस्ता॑-दृ॒त्विग्भ्यो॒ वाचं॑ ॅवि॒भज॑ति॒ तामृ॒त्विजो॒ यज॑माने॒ प्रति॑ ष्ठापयन्ति॒ स्वाहा॑ य॒ज्ञ्ं मन॒सेत्या॑ह॒ मन॑सा॒ हि पुरु॑षो य॒ज्ञ्म॑भि॒गच्छ॑ति॒ स्वाहा॒ द्यावा॑पृथि॒वीभ्या॒ -मित्या॑ह॒ द्यावा॑पृथि॒व्योर्.हि य॒ज्ञ्ः स्वाहो॒रोर॒-न्तरि॑क्षा॒ -दित्या॑हा॒न्तरि॑क्षे॒ हि य॒ज्ञ्ः स्वाहा॑ य॒ज्ञ्ं ॅवाता॒दा र॑भ॒ इत्या॑हा॒यं - [  ] \newline

\textbf{Pada Paata} \newline

क्री॒ते । सोमे᳚ । मै॒त्रा॒व॒रु॒णायेति॑ मैत्रा - व॒रु॒णाय॑ । द॒ण्डम् । प्रेति॑ । य॒च्छ॒ति॒ । मै॒त्रा॒व॒रु॒ण इति॑ मैत्रा - व॒रु॒णः । हि । पु॒रस्ता᳚त् । ऋ॒त्विग्भ्य॒ इत्यृ॒त्विक् - भ्यः॒ । वाच᳚म् । वि॒भज॒तीति॑ वि - भज॑ति । ताम् । ऋ॒त्विजः॑ । यज॑माने । प्रतीति॑ । स्था॒प॒य॒न्ति॒ । स्वाहा᳚ । य॒ज्ञ्म् । मन॑सा । इति॑ । आ॒ह॒ । मन॑सा । हि । पुरु॑षः । य॒ज्ञ्म् । अ॒भि॒गच्छ॒तीत्य॑भि - गच्छ॑ति । स्वाहा᳚ । द्यावा॑पृथि॒वीभ्या॒मिति॒ द्यावा᳚ - पृ॒थि॒वीभ्या᳚म् । इति॑ । आ॒ह॒ । द्यावा॑पृथि॒व्योरिति॒ द्यावा᳚-पृ॒थि॒व्योः । हि । य॒ज्ञ्ः । स्वाहा᳚ । उ॒रोः । अ॒न्तरि॑क्षात् । इति॑ । आ॒ह॒ । अ॒न्तरि॑क्षे । हि । य॒ज्ञ्ः । स्वाहा᳚ । य॒ज्ञ्म् । वाता᳚त् । एति॑ । र॒भे॒ । इति॑ । आ॒ह॒ । अ॒यम् ।  \newline


\textbf{Krama Paata} \newline

क्री॒ते सोमे᳚ । सोमे॑ मैत्रावरु॒णाय॑ । मै॒त्रा॒व॒रु॒णाय॑ द॒ण्डम् । मै॒त्रा॒व॒रु॒णायेति॑ मैत्रा - व॒रु॒णाय॑ । द॒ण्डम् प्र । प्र य॑च्छति । य॒च्छ॒ति॒ मै॒त्रा॒व॒रु॒णः । मै॒त्रा॒व॒रु॒णो हि । मै॒त्रा॒व॒रु॒ण इति॑ मैत्रा - व॒रु॒णः । हि पु॒रस्ता᳚त् । पु॒रस्ता॑दृ॒त्विग्भ्यः॑ । ऋ॒त्विग्भ्यो॒ वाच᳚म् । ऋ॒त्विग्भ्य॒ इत्यृ॒त्विक् - भ्यः॒ । वाच॑म् ॅवि॒भज॑ति । वि॒भज॑ति॒ ताम् । वि॒भज॒तीति॑ वि - भज॑ति । तामृ॒त्विजः॑ । ऋ॒त्विजो॒ यज॑माने । यज॑माने॒ प्रति॑ । प्रति॑ ष्ठापयन्ति । स्था॒प॒य॒न्ति॒ स्वाहा᳚ । स्वाहा॑ य॒ज्ञ्म् । य॒ज्ञ्म् मन॑सा । मन॒सेति॑ । इत्या॑ह । आ॒ह॒ मन॑सा । मन॑सा॒ हि । हि पुरु॑षः । पुरु॑षो य॒ज्ञ्म् । य॒ज्ञ्म॑भि॒गच्छ॑ति । अ॒भि॒गच्छ॑ति॒ स्वाहा᳚ । अ॒भि॒गच्छ॒तीत्य॑भि - गच्छ॑ति । स्वाहा॒ द्यावा॑पृथि॒वीभ्या᳚म् । द्यावा॑पृथि॒वीभ्या॒मिति॑ । द्यावा॑पृथि॒वीभ्या॒मिति॒ द्यावा᳚ - पृ॒थि॒वीभ्या᳚म् । इत्या॑ह । आ॒ह॒ द्यावा॑पृथि॒व्योः । द्यावा॑पृथि॒व्योर्. हि । द्यावा॑पृथि॒व्योरिति॒ द्यावा᳚ - पृ॒थि॒व्योः । हि य॒ज्ञ्ः । य॒ज्ञ्ः स्वाहा᳚ । स्वाहो॒रोः । उ॒रोर॒न्तरि॑क्षात् । अ॒न्तरि॑क्षा॒दिति॑ । इत्या॑ह । आ॒हा॒न्तरि॑क्षे । अ॒न्तरि॑क्षे॒ हि । हि य॒ज्ञ्ः । य॒ज्ञ्ः स्वाहा᳚ । स्वाहा॑ य॒ज्ञ्म् । य॒ज्ञ्म् ॅवाता᳚त् । वाता॒दा । आ र॑भे । र॒भ॒ इति॑ । इत्या॑ह । आ॒हा॒यम् । अ॒यम् ॅवाव \newline

\textbf{Jatai Paata} \newline

1. क्री॒ते सोमे॒ सोमे᳚ क्री॒ते क्री॒ते सोमे᳚ । \newline
2. सोमे॑ मैत्रावरु॒णाय॑ मैत्रावरु॒णाय॒ सोमे॒ सोमे॑ मैत्रावरु॒णाय॑ । \newline
3. मै॒त्रा॒व॒रु॒णाय॑ द॒ण्डम् द॒ण्डम् मै᳚त्रावरु॒णाय॑ मैत्रावरु॒णाय॑ द॒ण्डम् । \newline
4. मै॒त्रा॒व॒रु॒णायेति॑ मैत्रा - व॒रु॒णाय॑ । \newline
5. द॒ण्डम् प्र प्र द॒ण्डम् द॒ण्डम् प्र । \newline
6. प्र य॑च्छति यच्छति॒ प्र प्र य॑च्छति । \newline
7. य॒च्छ॒ति॒ मै॒त्रा॒व॒रु॒णो मै᳚त्रावरु॒णो य॑च्छति यच्छति मैत्रावरु॒णः । \newline
8. मै॒त्रा॒व॒रु॒णो हि हि मै᳚त्रावरु॒णो मै᳚त्रावरु॒णो हि । \newline
9. मै॒त्रा॒व॒रु॒ण इति॑ मैत्रा - व॒रु॒णः । \newline
10. हि पु॒रस्ता᳚त् पु॒रस्ता॒ द्धि हि पु॒रस्ता᳚त् । \newline
11. पु॒रस्ता॑ दृ॒त्विग्भ्य॑ ऋ॒त्विग्भ्यः॑ पु॒रस्ता᳚त् पु॒रस्ता॑ दृ॒त्विग्भ्यः॑ । \newline
12. ऋ॒त्विग्भ्यो॒ वाचं॒ ॅवाच॑ मृ॒त्विग्भ्य॑ ऋ॒त्विग्भ्यो॒ वाच᳚म् । \newline
13. ऋ॒त्विग्भ्य॒ इत्यृ॒त्विक् - भ्यः॒ । \newline
14. वाचं॑ ॅवि॒भज॑ति वि॒भज॑ति॒ वाचं॒ ॅवाचं॑ ॅवि॒भज॑ति । \newline
15. वि॒भज॑ति॒ ताम् तां ॅवि॒भज॑ति वि॒भज॑ति॒ ताम् । \newline
16. वि॒भज॒तीति॑ वि - भज॑ति । \newline
17. ता मृ॒त्विज॑ ऋ॒त्विज॒ स्ताम् ता मृ॒त्विजः॑ । \newline
18. ऋ॒त्विजो॒ यज॑माने॒ यज॑मान ऋ॒त्विज॑ ऋ॒त्विजो॒ यज॑माने । \newline
19. यज॑माने॒ प्रति॒ प्रति॒ यज॑माने॒ यज॑माने॒ प्रति॑ । \newline
20. प्रति॑ ष्ठापयन्ति स्थापयन्ति॒ प्रति॒ प्रति॑ ष्ठापयन्ति । \newline
21. स्था॒प॒य॒न्ति॒ स्वाहा॒ स्वाहा᳚ स्थापयन्ति स्थापयन्ति॒ स्वाहा᳚ । \newline
22. स्वाहा॑ य॒ज्ञ्ं ॅय॒ज्ञ्ꣳ स्वाहा॒ स्वाहा॑ य॒ज्ञ्म् । \newline
23. य॒ज्ञ्म् मन॑सा॒ मन॑सा य॒ज्ञ्ं ॅय॒ज्ञ्म् मन॑सा । \newline
24. मन॒से तीति॒ मन॑सा॒ मन॒ सेति॑ । \newline
25. इत्या॑हा॒हे तीत्या॑ह । \newline
26. आ॒ह॒ मन॑सा॒ मन॑सा ऽऽहाह॒ मन॑सा । \newline
27. मन॑सा॒ हि हि मन॑सा॒ मन॑सा॒ हि । \newline
28. हि पुरु॑षः॒ पुरु॑षो॒ हि हि पुरु॑षः । \newline
29. पुरु॑षो य॒ज्ञ्ं ॅय॒ज्ञ्म् पुरु॑षः॒ पुरु॑षो य॒ज्ञ्म् । \newline
30. य॒ज्ञ् म॑भि॒गच्छ॑ त्यभि॒गच्छ॑ति य॒ज्ञ्ं ॅय॒ज्ञ् म॑भि॒गच्छ॑ति । \newline
31. अ॒भि॒गच्छ॑ति॒ स्वाहा॒ स्वाहा॑ ऽभि॒गच्छ॑ त्यभि॒गच्छ॑ति॒ स्वाहा᳚ । \newline
32. अ॒भि॒गच्छ॒तीत्य॑भि - गच्छ॑ति । \newline
33. स्वाहा॒ द्यावा॑पृथि॒वीभ्या॒म् द्यावा॑पृथि॒वीभ्याꣳ॒॒ स्वाहा॒ स्वाहा॒ द्यावा॑पृथि॒वीभ्या᳚म् । \newline
34. द्यावा॑पृथि॒वीभ्या॒ मितीति॒ द्यावा॑पृथि॒वीभ्या॒म् द्यावा॑पृथि॒वीभ्या॒ मिति॑ । \newline
35. द्यावा॑पृथि॒वीभ्या॒मिति॒ द्यावा᳚ - पृ॒थि॒वीभ्या᳚म् । \newline
36. इत्या॑हा॒हे तीत्या॑ह । \newline
37. आ॒ह॒ द्यावा॑पृथि॒व्योर् द्यावा॑पृथि॒व्यो रा॑हाह॒ द्यावा॑पृथि॒व्योः । \newline
38. द्यावा॑पृथि॒व्योर्. हि हि द्यावा॑पृथि॒व्योर् द्यावा॑पृथि॒व्योर्. हि । \newline
39. द्यावा॑पृथि॒व्योरिति॒ द्यावा᳚ - पृ॒थि॒व्योः । \newline
40. हि य॒ज्ञो य॒ज्ञो हि हि य॒ज्ञ्ः । \newline
41. य॒ज्ञ्ः स्वाहा॒ स्वाहा॑ य॒ज्ञो य॒ज्ञ्ः स्वाहा᳚ । \newline
42. स्वाहो॒ रोरु॒रोः स्वाहा॒ स्वाहो॒रोः । \newline
43. उ॒रो र॒न्तरि॑क्षा द॒न्तरि॑क्षा दु॒रो रु॒रो र॒न्तरि॑क्षात् । \newline
44. अ॒न्तरि॑क्षा॒दि तीत्य॒न्तरि॑क्षा द॒न्तरि॑क्षा॒ दिति॑ । \newline
45. इत्या॑हा॒हे तीत्या॑ह । \newline
46. आ॒हा॒ न्तरि॑क्षे॒ ऽन्तरि॑क्ष आहाहा॒ न्तरि॑क्षे । \newline
47. अ॒न्तरि॑क्षे॒ हि ह्य॑न्तरि॑क्षे॒ ऽन्तरि॑क्षे॒ हि । \newline
48. हि य॒ज्ञो य॒ज्ञो हि हि य॒ज्ञ्ः । \newline
49. य॒ज्ञ्ः स्वाहा॒ स्वाहा॑ य॒ज्ञो य॒ज्ञ्ः स्वाहा᳚ । \newline
50. स्वाहा॑ य॒ज्ञ्ं ॅय॒ज्ञ्ꣳ स्वाहा॒ स्वाहा॑ य॒ज्ञ्म् । \newline
51. य॒ज्ञ्ं ॅवाता॒द् वाता᳚द् य॒ज्ञ्ं ॅय॒ज्ञ्ं ॅवाता᳚त् । \newline
52. वाता॒दा वाता॒द् वाता॒दा । \newline
53. आ र॑भे रभ॒ आ र॑भे । \newline
54. र॒भ॒ इतीति॑ रभे रभ॒ इति॑ । \newline
55. इत्या॑हा॒हे तीत्या॑ह । \newline
56. आ॒हा॒य म॒य मा॑हा हा॒यम् । \newline
57. अ॒यं ॅवाव वावाय म॒यं ॅवाव । \newline

\textbf{Ghana Paata } \newline

1. क्री॒ते सोमे॒ सोमे᳚ क्री॒ते क्री॒ते सोमे॑ मैत्रावरु॒णाय॑ मैत्रावरु॒णाय॒ सोमे᳚ क्री॒ते क्री॒ते सोमे॑ मैत्रावरु॒णाय॑ । \newline
2. सोमे॑ मैत्रावरु॒णाय॑ मैत्रावरु॒णाय॒ सोमे॒ सोमे॑ मैत्रावरु॒णाय॑ द॒ण्डम् द॒ण्डम् मै᳚त्रावरु॒णाय॒ सोमे॒ सोमे॑ मैत्रावरु॒णाय॑ द॒ण्डम् । \newline
3. मै॒त्रा॒व॒रु॒णाय॑ द॒ण्डम् द॒ण्डम् मै᳚त्रावरु॒णाय॑ मैत्रावरु॒णाय॑ द॒ण्डम् प्र प्र द॒ण्डम् मै᳚त्रावरु॒णाय॑ मैत्रावरु॒णाय॑ द॒ण्डम् प्र । \newline
4. मै॒त्रा॒व॒रु॒णायेति॑ मैत्रा - व॒रु॒णाय॑ । \newline
5. द॒ण्डम् प्र प्र द॒ण्डम् द॒ण्डम् प्र य॑च्छति यच्छति॒ प्र द॒ण्डम् द॒ण्डम् प्र य॑च्छति । \newline
6. प्र य॑च्छति यच्छति॒ प्र प्र य॑च्छति मैत्रावरु॒णो मै᳚त्रावरु॒णो य॑च्छति॒ प्र प्र य॑च्छति मैत्रावरु॒णः । \newline
7. य॒च्छ॒ति॒ मै॒त्रा॒व॒रु॒णो मै᳚त्रावरु॒णो य॑च्छति यच्छति मैत्रावरु॒णो हि हि मै᳚त्रावरु॒णो य॑च्छति यच्छति मैत्रावरु॒णो हि । \newline
8. मै॒त्रा॒व॒रु॒णो हि हि मै᳚त्रावरु॒णो मै᳚त्रावरु॒णो हि पु॒रस्ता᳚त् पु॒रस्ता॒द्धि मै᳚त्रावरु॒णो मै᳚त्रावरु॒णो हि पु॒रस्ता᳚त् । \newline
9. मै॒त्रा॒व॒रु॒ण इति॑ मैत्रा - व॒रु॒णः । \newline
10. हि पु॒रस्ता᳚त् पु॒रस्ता॒द्धि हि पु॒रस्ता॑ दृ॒त्विग्भ्य॑ ऋ॒त्विग्भ्यः॑ पु॒रस्ता॒द्धि हि पु॒रस्ता॑ दृ॒त्विग्भ्यः॑ । \newline
11. पु॒रस्ता॑ दृ॒त्विग्भ्य॑ ऋ॒त्विग्भ्यः॑ पु॒रस्ता᳚त् पु॒रस्ता॑ दृ॒त्विग्भ्यो॒ वाचं॒ ॅवाच॑ मृ॒त्विग्भ्यः॑ पु॒रस्ता᳚त् पु॒रस्ता॑ दृ॒त्विग्भ्यो॒ वाच᳚म् । \newline
12. ऋ॒त्विग्भ्यो॒ वाचं॒ ॅवाच॑ मृ॒त्विग्भ्य॑ ऋ॒त्विग्भ्यो॒ वाचं॑ ॅवि॒भज॑ति वि॒भज॑ति॒ वाच॑ मृ॒त्विग्भ्य॑ ऋ॒त्विग्भ्यो॒ वाचं॑ ॅवि॒भज॑ति । \newline
13. ऋ॒त्विग्भ्य॒ इत्यृ॒त्विक् - भ्यः॒ । \newline
14. वाचं॑ ॅवि॒भज॑ति वि॒भज॑ति॒ वाचं॒ ॅवाचं॑ ॅवि॒भज॑ति॒ ताम् तां ॅवि॒भज॑ति॒ वाचं॒ ॅवाचं॑ ॅवि॒भज॑ति॒ ताम् । \newline
15. वि॒भज॑ति॒ ताम् तां ॅवि॒भज॑ति वि॒भज॑ति॒ ता मृ॒त्विज॑ ऋ॒त्विज॒ स्तां ॅवि॒भज॑ति वि॒भज॑ति॒ ता मृ॒त्विजः॑ । \newline
16. वि॒भज॒तीति॑ वि - भज॑ति । \newline
17. ता मृ॒त्विज॑ ऋ॒त्विज॒ स्ताम् ता मृ॒त्विजो॒ यज॑माने॒ यज॑मान ऋ॒त्विज॒ स्ताम् ता मृ॒त्विजो॒ यज॑माने । \newline
18. ऋ॒त्विजो॒ यज॑माने॒ यज॑मान ऋ॒त्विज॑ ऋ॒त्विजो॒ यज॑माने॒ प्रति॒ प्रति॒ यज॑मान ऋ॒त्विज॑ ऋ॒त्विजो॒ यज॑माने॒ प्रति॑ । \newline
19. यज॑माने॒ प्रति॒ प्रति॒ यज॑माने॒ यज॑माने॒ प्रति॑ ष्ठापयन्ति स्थापयन्ति॒ प्रति॒ यज॑माने॒ यज॑माने॒ प्रति॑ ष्ठापयन्ति । \newline
20. प्रति॑ ष्ठापयन्ति स्थापयन्ति॒ प्रति॒ प्रति॑ ष्ठापयन्ति॒ स्वाहा॒ स्वाहा᳚ स्थापयन्ति॒ प्रति॒ प्रति॑ ष्ठापयन्ति॒ स्वाहा᳚ । \newline
21. स्था॒प॒य॒न्ति॒ स्वाहा॒ स्वाहा᳚ स्थापयन्ति स्थापयन्ति॒ स्वाहा॑ य॒ज्ञ्ं ॅय॒ज्ञ्ꣳ स्वाहा᳚ स्थापयन्ति स्थापयन्ति॒ स्वाहा॑ य॒ज्ञ्म् । \newline
22. स्वाहा॑ य॒ज्ञ्ं ॅय॒ज्ञ्ꣳ स्वाहा॒ स्वाहा॑ य॒ज्ञ्म् मन॑सा॒ मन॑सा य॒ज्ञ्ꣳ स्वाहा॒ स्वाहा॑ य॒ज्ञ्म् मन॑सा । \newline
23. य॒ज्ञ्म् मन॑सा॒ मन॑सा य॒ज्ञ्ं ॅय॒ज्ञ्म् मन॒से तीति॒ मन॑सा य॒ज्ञ्ं ॅय॒ज्ञ्म् मन॒ सेति॑ । \newline
24. मन॒से तीति॒ मन॑सा॒ मन॒ सेत्या॑ हा॒हेति॒ मन॑सा॒ मन॒ सेत्या॑ह । \newline
25. इत्या॑हा॒हे तीत्या॑ह॒ मन॑सा॒ मन॑सा॒ ऽऽहे तीत्या॑ह॒ मन॑सा । \newline
26. आ॒ह॒ मन॑सा॒ मन॑सा ऽऽहाह॒ मन॑सा॒ हि हि मन॑सा ऽऽहाह॒ मन॑सा॒ हि । \newline
27. मन॑सा॒ हि हि मन॑सा॒ मन॑सा॒ हि पुरु॑षः॒ पुरु॑षो॒ हि मन॑सा॒ मन॑सा॒ हि पुरु॑षः । \newline
28. हि पुरु॑षः॒ पुरु॑षो॒ हि हि पुरु॑षो य॒ज्ञ्ं ॅय॒ज्ञ्म् पुरु॑षो॒ हि हि पुरु॑षो य॒ज्ञ्म् । \newline
29. पुरु॑षो य॒ज्ञ्ं ॅय॒ज्ञ्म् पुरु॑षः॒ पुरु॑षो य॒ज्ञ् म॑भि॒गच्छ॑ त्यभि॒गच्छ॑ति य॒ज्ञ्म् पुरु॑षः॒ पुरु॑षो य॒ज्ञ् म॑भि॒गच्छ॑ति । \newline
30. य॒ज्ञ् म॑भि॒गच्छ॑ त्यभि॒गच्छ॑ति य॒ज्ञ्ं ॅय॒ज्ञ् म॑भि॒गच्छ॑ति॒ स्वाहा॒ स्वाहा॑ ऽभि॒गच्छ॑ति य॒ज्ञ्ं ॅय॒ज्ञ् म॑भि॒गच्छ॑ति॒ स्वाहा᳚ । \newline
31. अ॒भि॒गच्छ॑ति॒ स्वाहा॒ स्वाहा॑ ऽभि॒गच्छ॑ त्यभि॒गच्छ॑ति॒ स्वाहा॒ द्यावा॑पृथि॒वीभ्या॒म् द्यावा॑पृथि॒वीभ्याꣳ॒॒ स्वाहा॑ ऽभि॒गच्छ॑ त्यभि॒गच्छ॑ति॒ स्वाहा॒ द्यावा॑पृथि॒वीभ्या᳚म् । \newline
32. अ॒भि॒गच्छ॒तीत्य॑भि - गच्छ॑ति । \newline
33. स्वाहा॒ द्यावा॑पृथि॒वीभ्या॒म् द्यावा॑पृथि॒वीभ्याꣳ॒॒ स्वाहा॒ स्वाहा॒ द्यावा॑पृथि॒वीभ्या॒ मितीति॒ द्यावा॑पृथि॒वीभ्याꣳ॒॒ स्वाहा॒ स्वाहा॒ द्यावा॑पृथि॒वीभ्या॒ मिति॑ । \newline
34. द्यावा॑पृथि॒वीभ्या॒ मितीति॒ द्यावा॑पृथि॒वीभ्या॒म् द्यावा॑पृथि॒वीभ्या॒ मित्या॑हा॒हेति॒ द्यावा॑पृथि॒वीभ्या॒म् द्यावा॑पृथि॒वीभ्या॒ मित्या॑ह । \newline
35. द्यावा॑पृथि॒वीभ्या॒मिति॒ द्यावा᳚ - पृ॒थि॒वीभ्या᳚म् । \newline
36. इत्या॑हा॒हे तीत्या॑ह॒ द्यावा॑पृथि॒व्योर् द्यावा॑पृथि॒व्यो रा॒हे तीत्या॑ह॒ द्यावा॑पृथि॒व्योः । \newline
37. आ॒ह॒ द्यावा॑पृथि॒व्योर् द्यावा॑पृथि॒व्यो रा॑हाह॒ द्यावा॑पृथि॒व्योर्. हि हि द्यावा॑पृथि॒व्यो रा॑हाह॒ द्यावा॑पृथि॒व्योर्. हि । \newline
38. द्यावा॑पृथि॒व्योर्. हि हि द्यावा॑पृथि॒व्योर् द्यावा॑पृथि॒व्योर्. हि य॒ज्ञो य॒ज्ञो हि द्यावा॑पृथि॒व्योर् द्यावा॑पृथि॒व्योर्. हि य॒ज्ञ्ः । \newline
39. द्यावा॑पृथि॒व्योरिति॒ द्यावा᳚ - पृ॒थि॒व्योः । \newline
40. हि य॒ज्ञो य॒ज्ञो हि हि य॒ज्ञ्ः स्वाहा॒ स्वाहा॑ य॒ज्ञो हि हि य॒ज्ञ्ः स्वाहा᳚ । \newline
41. य॒ज्ञ्ः स्वाहा॒ स्वाहा॑ य॒ज्ञो य॒ज्ञ्ः स्वाहो॒ रो रु॒रोः स्वाहा॑ य॒ज्ञो य॒ज्ञ्ः स्वाहो॒रोः । \newline
42. स्वाहो॒ रो रु॒रोः स्वाहा॒ स्वाहो॒रो र॒न्तरि॑क्षा द॒न्तरि॑क्षा दु॒रोः स्वाहा॒ स्वाहो॒रो र॒न्तरि॑क्षात् । \newline
43. उ॒रो र॒न्तरि॑क्षा द॒न्तरि॑क्षा दु॒रो रु॒रो र॒न्तरि॑क्षा॒ दिती त्य॒न्तरि॑क्षा दु॒रो रु॒रो र॒न्तरि॑क्षा॒ दिति॑ । \newline
44. अ॒न्तरि॑क्षा॒ दिती त्य॒न्तरि॑क्षा द॒न्तरि॑क्षा॒ दित्या॑हा॒हे त्य॒न्तरि॑क्षा द॒न्तरि॑क्षा॒ दित्या॑ह । \newline
45. इत्या॑हा॒हे तीत्या॑ हा॒न्तरि॑क्षे॒ ऽन्तरि॑क्ष आ॒हे तीत्या॑ हा॒न्तरि॑क्षे । \newline
46. आ॒हा॒ न्तरि॑क्षे॒ ऽन्तरि॑क्ष आहाहा॒ न्तरि॑क्षे॒ हि ह्य॑न्तरि॑क्ष आहाहा॒ न्तरि॑क्षे॒ हि । \newline
47. अ॒न्तरि॑क्षे॒ हि ह्य॑न्तरि॑क्षे॒ ऽन्तरि॑क्षे॒ हि य॒ज्ञो य॒ज्ञो ह्य॑न्तरि॑क्षे॒ ऽन्तरि॑क्षे॒ हि य॒ज्ञ्ः । \newline
48. हि य॒ज्ञो य॒ज्ञो हि हि य॒ज्ञ्ः स्वाहा॒ स्वाहा॑ य॒ज्ञो हि हि य॒ज्ञ्ः स्वाहा᳚ । \newline
49. य॒ज्ञ्ः स्वाहा॒ स्वाहा॑ य॒ज्ञो य॒ज्ञ्ः स्वाहा॑ य॒ज्ञ्ं ॅय॒ज्ञ्ꣳ स्वाहा॑ य॒ज्ञो य॒ज्ञ्ः स्वाहा॑ य॒ज्ञ्म् । \newline
50. स्वाहा॑ य॒ज्ञ्ं ॅय॒ज्ञ्ꣳ स्वाहा॒ स्वाहा॑ य॒ज्ञ्ं ॅवाता॒द् वाता᳚द् य॒ज्ञ्ꣳ स्वाहा॒ स्वाहा॑ य॒ज्ञ्ं ॅवाता᳚त् । \newline
51. य॒ज्ञ्ं ॅवाता॒द् वाता᳚द् य॒ज्ञ्ं ॅय॒ज्ञ्ं ॅवाता॒दा वाता᳚द् य॒ज्ञ्ं ॅय॒ज्ञ्ं ॅवाता॒दा । \newline
52. वाता॒दा वाता॒द् वाता॒दा र॑भे रभ॒ आ वाता॒द् वाता॒दा र॑भे । \newline
53. आ र॑भे रभ॒ आ र॑भ॒ इतीति॑ रभ॒ आ र॑भ॒ इति॑ । \newline
54. र॒भ॒ इतीति॑ रभे रभ॒ इत्या॑हा ॒हेति॑ रभे रभ॒ इत्या॑ह । \newline
55. इत्या॑ हा॒हे तीत्या॑हा॒ य म॒य मा॒हे तीत्या॑ हा॒यम् । \newline
56. आ॒हा॒ य म॒य मा॑हा हा॒यं ॅवाव वावाय मा॑हा हा॒यं ॅवाव । \newline
57. अ॒यं ॅवाव वावाय म॒यं ॅवाव यो यो वावाय म॒यं ॅवाव यः । \newline
\pagebreak
\markright{ TS 6.1.4.3  \hfill https://www.vedavms.in \hfill}

\section{ TS 6.1.4.3 }

\textbf{TS 6.1.4.3 } \newline
\textbf{Samhita Paata} \newline

ॅवाव यः पव॑ते॒ स य॒ज्ञ्स्तमे॒व सा॒क्षादा र॑भते मु॒ष्टी क॑रोति॒ वाचं॑ ॅयच्छति य॒ज्ञ्स्य॒ धृत्या॒ अदी᳚क्षिष्टा॒यं ब्रा᳚ह्म॒ण इति॒ त्रिरु॑पाꣳ॒॒श्वा॑ह दे॒वेभ्य॑ ए॒वैनं॒ प्राऽऽ*ह॒ त्रिरु॒च्चैरु॒भये᳚भ्य ए॒वैनं॑ देवमनु॒ष्येभ्यः॒ प्राऽऽ*ह॒ न पु॒रा नक्ष॑त्रेभ्यो॒ वाचं॒ ॅवि सृ॑जे॒द्-यत्पु॒रा नक्ष॑त्रेभ्यो॒ वाचं॑ ॅविसृ॒जेद्- य॒ज्ञ्ं ॅविच्छि॑न्द्या॒ - [  ] \newline

\textbf{Pada Paata} \newline

वाव । यः । पव॑ते । सः । य॒ज्ञ्ः । तम् । ए॒व । सा॒क्षादिति॑ स-अ॒क्षात् । एति॑ । र॒भ॒ते॒ । मु॒ष्टी इति॑ । क॒रो॒ति॒ । वाच᳚म् । य॒च्छ॒ति॒ । य॒ज्ञ्स्य॑ । धृत्यै᳚ । अदी᳚क्षिष्ट । अ॒यम् । ब्रा॒ह्म॒णः । इति॑ । त्रिः । उ॒पाꣳ॒॒श्वित्यु॑प - अꣳ॒॒शु । आ॒ह॒ । दे॒वेभ्यः॑ । ए॒व । ए॒न॒म् । प्रेति॑ । आ॒ह॒ । त्रिः । उ॒च्चैः । उ॒भये᳚भ्यः । ए॒व । ए॒न॒म् । दे॒व॒म॒नु॒ष्येभ्य॒ इति॑ देव - म॒नु॒ष्येभ्यः॑ । प्रेति॑ । आ॒ह॒ । न । पु॒रा । नक्ष॑त्रेभ्यः । वाच᳚म् । वीति॑ । सृ॒जे॒त् । यत् । पु॒रा । नक्ष॑त्रेभ्यः । वाच᳚म् । वि॒सृ॒जेदिति॑ वि - सृ॒जेत् । य॒ज्ञ्म् । वीति॑ । छि॒न्द्या॒त् ।  \newline


\textbf{Krama Paata} \newline

वाव यः । यः पव॑ते । पव॑ते॒ सः । स य॒ज्ञ्ः । य॒ज्ञ्स्तम् । तमे॒व । ए॒व सा॒क्षात् । सा॒क्षादा । सा॒क्षादिति॑ स - अ॒क्षात् । आ र॑भते । र॒भ॒ते॒ मु॒ष्टी । मु॒ष्टी क॑रोति । मु॒ष्टी इति॑ मु॒ष्टी । क॒रो॒ति॒ वाच᳚म् । वाच॑म् ॅयच्छति । य॒च्छ॒ति॒ य॒ज्ञ्स्य॑ । य॒ज्ञ्स्य॒ धृत्यै᳚ । धृत्या॒ अदी᳚क्षिष्ट । अदी᳚क्षिष्टा॒यम् । अ॒यम् ब्रा᳚ह्म॒णः । ब्रा॒ह्म॒ण इति॑ । इति॒ त्रिः । त्रिरु॑पाꣳ॒॒शु । उ॒पाꣳ॒॒श्वा॑ह । उ॒पाꣳ॒॒श्वित्यु॑प - अꣳ॒॒शु । आ॒ह॒ दे॒वेभ्यः॑ । दे॒वेभ्य॑ ए॒व । ए॒वैन᳚म् । ए॒न॒म् प्र । प्राह॑ । आ॒ह॒ त्रिः । त्रिरु॒च्चैः । उ॒च्चैरु॒भये᳚भ्यः । उ॒भये᳚भ्य ए॒व । ए॒वैन᳚म् । ए॒न॒म् दे॒व॒म॒नु॒ष्येभ्यः॑ । दे॒व॒म॒नु॒ष्येभ्यः॒ प्र । दे॒व॒म॒नु॒ष्येभ्य॒ इति॑ देव - म॒नु॒ष्येभ्यः॑ । प्राह॑ । आ॒ह॒ न । न पु॒रा । पु॒रा नक्ष॑त्रेभ्यः । नक्ष॑त्रेभ्यो॒ वाच᳚म् । वाच॒म् ॅवि । वि सृ॑जेत् । सृ॒जे॒द् यत् । यत् पु॒रा । पु॒रा नक्ष॑त्रेभ्यः । नक्ष॑त्रेभ्यो॒ वाच᳚म् । वाच॑म् ॅविसृ॒जेत् । वि॒सृ॒जेद् य॒ज्ञ्म् । वि॒सृ॒जेदिति॑ वि - सृ॒जेत् । य॒ज्ञ्म् ॅवि । विच्छि॑न्द्यात् । छि॒न्द्या॒दुदि॑तेषु \newline

\textbf{Jatai Paata} \newline

1. वाव यो यो वाव वाव यः । \newline
2. यः पव॑ते॒ पव॑ते॒ यो यः पव॑ते । \newline
3. पव॑ते॒ स स पव॑ते॒ पव॑ते॒ सः । \newline
4. स य॒ज्ञो य॒ज्ञ्ः स स य॒ज्ञ्ः । \newline
5. य॒ज्ञ् स्तम् तं ॅय॒ज्ञो य॒ज्ञ् स्तम् । \newline
6. त मे॒वैव तम् त मे॒व । \newline
7. ए॒व सा॒क्षाथ् सा॒क्षा दे॒वैव सा॒क्षात् । \newline
8. सा॒क्षादा सा॒क्षाथ् सा॒क्षादा । \newline
9. सा॒क्षादिति॑ स - अ॒क्षात् । \newline
10. आ र॑भते रभत॒ आ र॑भते । \newline
11. र॒भ॒ते॒ मु॒ष्टी मु॒ष्टी र॑भते रभते मु॒ष्टी । \newline
12. मु॒ष्टी क॑रोति करोति मु॒ष्टी मु॒ष्टी क॑रोति । \newline
13. मु॒ष्टी इति॑ मु॒ष्टी । \newline
14. क॒रो॒ति॒ वाचं॒ ॅवाच॑म् करोति करोति॒ वाच᳚म् । \newline
15. वाचं॑ ॅयच्छति यच्छति॒ वाचं॒ ॅवाचं॑ ॅयच्छति । \newline
16. य॒च्छ॒ति॒ य॒ज्ञ्स्य॑ य॒ज्ञ्स्य॑ यच्छति यच्छति य॒ज्ञ्स्य॑ । \newline
17. य॒ज्ञ्स्य॒ धृत्यै॒ धृत्यै॑ य॒ज्ञ्स्य॑ य॒ज्ञ्स्य॒ धृत्यै᳚ । \newline
18. धृत्या॒ अदी᳚क्षि॒ष्टा दी᳚क्षिष्ट॒ धृत्यै॒ धृत्या॒ अदी᳚क्षिष्ट । \newline
19. अदी᳚क्षिष्टा॒य म॒य मदी᳚क्षि॒ष्टा दी᳚क्षिष्टा॒यम् । \newline
20. अ॒यम् ब्रा᳚ह्म॒णो ब्रा᳚ह्म॒णो॑ ऽय म॒यम् ब्रा᳚ह्म॒णः । \newline
21. ब्रा॒ह्म॒ण इतीति॑ ब्राह्म॒णो ब्रा᳚ह्म॒ण इति॑ । \newline
22. इति॒ त्रि स्त्रिरि तीति॒ त्रिः । \newline
23. त्रि रु॑पाꣳ॒॒शू॑ पाꣳ॒॒शु त्रि स्त्रि रु॑पाꣳ॒॒शु । \newline
24. उ॒पाꣳ॒॒ श्वा॑हा होपाꣳ॒॒शू॑ पाꣳ॒॒श्वा॑ह । \newline
25. उ॒पाꣳ॒॒श्वित्यु॑प - अꣳ॒॒शु । \newline
26. आ॒ह॒ दे॒वेभ्यो॑ दे॒वेभ्य॑ आहाह दे॒वेभ्यः॑ । \newline
27. दे॒वेभ्य॑ ए॒वैव दे॒वेभ्यो॑ दे॒वेभ्य॑ ए॒व । \newline
28. ए॒वैन॑ मेन मे॒वै वैन᳚म् । \newline
29. ए॒न॒म् प्र प्रैन॑ मेन॒म् प्र । \newline
30. प्राहा॑ह॒ प्र प्राह॑ । \newline
31. आ॒ह॒ त्रि स्त्रि रा॑हाह॒ त्रिः । \newline
32. त्रि रु॒च्चै रु॒च्चै स्त्रि स्त्रि रु॒च्चैः । \newline
33. उ॒च्चै रु॒भये᳚भ्य उ॒भये᳚भ्य उ॒च्चै रु॒च्चै रु॒भये᳚भ्यः । \newline
34. उ॒भये᳚भ्य ए॒वैवो भये᳚भ्य उ॒भये᳚भ्य ए॒व । \newline
35. ए॒वैन॑ मेन मे॒वै वैन᳚म् । \newline
36. ए॒न॒म् दे॒व॒म॒नु॒ष्येभ्यो॑ देवमनु॒ष्येभ्य॑ एन मेनम् देवमनु॒ष्येभ्यः॑ । \newline
37. दे॒व॒म॒नु॒ष्येभ्यः॒ प्र प्र दे॑वमनु॒ष्येभ्यो॑ देवमनु॒ष्येभ्यः॒ प्र । \newline
38. दे॒व॒म॒नु॒ष्येभ्य॒ इति॑ देव - म॒नु॒ष्येभ्यः॑ । \newline
39. प्राहा॑ह॒ प्र प्राह॑ । \newline
40. आ॒ह॒ न नाहा॑ह॒ न । \newline
41. न पु॒रा पु॒रा न न पु॒रा । \newline
42. पु॒रा नक्ष॑त्रेभ्यो॒ नक्ष॑त्रेभ्यः पु॒रा पु॒रा नक्ष॑त्रेभ्यः । \newline
43. नक्ष॑त्रेभ्यो॒ वाचं॒ ॅवाच॒म् नक्ष॑त्रेभ्यो॒ नक्ष॑त्रेभ्यो॒ वाच᳚म् । \newline
44. वाचं॒ ॅवि वि वाचं॒ ॅवाचं॒ ॅवि । \newline
45. वि सृ॑जेथ् सृजे॒द् वि वि सृ॑जेत् । \newline
46. सृ॒जे॒द् यद् यथ् सृ॑जेथ् सृजे॒द् यत् । \newline
47. यत् पु॒रा पु॒रा यद् यत् पु॒रा । \newline
48. पु॒रा नक्ष॑त्रेभ्यो॒ नक्ष॑त्रेभ्यः पु॒रा पु॒रा नक्ष॑त्रेभ्यः । \newline
49. नक्ष॑त्रेभ्यो॒ वाचं॒ ॅवाच॒म् नक्ष॑त्रेभ्यो॒ नक्ष॑त्रेभ्यो॒ वाच᳚म् । \newline
50. वाचं॑ ॅविसृ॒जेद् वि॑सृ॒जेद् वाचं॒ ॅवाचं॑ ॅविसृ॒जेत् । \newline
51. वि॒सृ॒जेद् य॒ज्ञ्ं ॅय॒ज्ञ्ं ॅवि॑सृ॒जेद् वि॑सृ॒जेद् य॒ज्ञ्म् । \newline
52. वि॒सृ॒जेदिति॑ वि - सृ॒जेत् । \newline
53. य॒ज्ञ्ं ॅवि वि य॒ज्ञ्ं ॅय॒ज्ञ्ं ॅवि । \newline
54. वि च्छि॑न्द्याच् छिन्द्याद् वि वि च्छि॑न्द्यात् । \newline
55. छि॒न्द्या॒ दुदि॑ते॒षू दि॑तेषु छिन्द्याच् छिन्द्या॒ दुदि॑तेषु । \newline

\textbf{Ghana Paata } \newline

1. वाव यो यो वाव वाव यः पव॑ते॒ पव॑ते॒ यो वाव वाव यः पव॑ते । \newline
2. यः पव॑ते॒ पव॑ते॒ यो यः पव॑ते॒ स स पव॑ते॒ यो यः पव॑ते॒ सः । \newline
3. पव॑ते॒ स स पव॑ते॒ पव॑ते॒ स य॒ज्ञो य॒ज्ञ्ः स पव॑ते॒ पव॑ते॒ स य॒ज्ञ्ः । \newline
4. स य॒ज्ञो य॒ज्ञ्ः स स य॒ज्ञ् स्तम् तं ॅय॒ज्ञ्ः स स य॒ज्ञ् स्तम् । \newline
5. य॒ज्ञ् स्तम् तं ॅय॒ज्ञो य॒ज्ञ् स्त मे॒वैव तं ॅय॒ज्ञो य॒ज्ञ् स्त मे॒व । \newline
6. त मे॒वैव तम् त मे॒व सा॒क्षाथ् सा॒क्षा दे॒व तम् त मे॒व सा॒क्षात् । \newline
7. ए॒व सा॒क्षाथ् सा॒क्षा दे॒वैव सा॒क्षादा सा॒क्षा दे॒वैव सा॒क्षादा । \newline
8. सा॒क्षादा सा॒क्षाथ् सा॒क्षादा र॑भते रभत॒ आ सा॒क्षाथ् सा॒क्षादा र॑भते । \newline
9. सा॒क्षादिति॑ स - अ॒क्षात् । \newline
10. आ र॑भते रभत॒ आ र॑भते मु॒ष्टी मु॒ष्टी र॑भत॒ आ र॑भते मु॒ष्टी । \newline
11. र॒भ॒ते॒ मु॒ष्टी मु॒ष्टी र॑भते रभते मु॒ष्टी क॑रोति करोति मु॒ष्टी र॑भते रभते मु॒ष्टी क॑रोति । \newline
12. मु॒ष्टी क॑रोति करोति मु॒ष्टी मु॒ष्टी क॑रोति॒ वाचं॒ ॅवाच॑म् करोति मु॒ष्टी मु॒ष्टी क॑रोति॒ वाच᳚म् । \newline
13. मु॒ष्टी इति॑ मु॒ष्टी । \newline
14. क॒रो॒ति॒ वाचं॒ ॅवाच॑म् करोति करोति॒ वाचं॑ ॅयच्छति यच्छति॒ वाच॑म् करोति करोति॒ वाचं॑ ॅयच्छति । \newline
15. वाचं॑ ॅयच्छति यच्छति॒ वाचं॒ ॅवाचं॑ ॅयच्छति य॒ज्ञ्स्य॑ य॒ज्ञ्स्य॑ यच्छति॒ वाचं॒ ॅवाचं॑ ॅयच्छति य॒ज्ञ्स्य॑ । \newline
16. य॒च्छ॒ति॒ य॒ज्ञ्स्य॑ य॒ज्ञ्स्य॑ यच्छति यच्छति य॒ज्ञ्स्य॒ धृत्यै॒ धृत्यै॑ य॒ज्ञ्स्य॑ यच्छति यच्छति य॒ज्ञ्स्य॒ धृत्यै᳚ । \newline
17. य॒ज्ञ्स्य॒ धृत्यै॒ धृत्यै॑ य॒ज्ञ्स्य॑ य॒ज्ञ्स्य॒ धृत्या॒ अदी᳚क्षि॒ष्टा दी᳚क्षिष्ट॒ धृत्यै॑ य॒ज्ञ्स्य॑ य॒ज्ञ्स्य॒ धृत्या॒ अदी᳚क्षिष्ट । \newline
18. धृत्या॒ अदी᳚क्षि॒ष्टा दी᳚क्षिष्ट॒ धृत्यै॒ धृत्या॒ अदी᳚क्षिष्टा॒ य म॒य मदी᳚क्षिष्ट॒ धृत्यै॒ धृत्या॒ अदी᳚क्षिष्टा॒यम् । \newline
19. अदी᳚क्षिष्टा॒ य म॒य मदी᳚क्षि॒ष्टा दी᳚क्षिष्टा॒यम् ब्रा᳚ह्म॒णो ब्रा᳚ह्म॒णो॑ ऽय मदी᳚क्षि॒ष्टा दी᳚क्षिष्टा॒यम् ब्रा᳚ह्म॒णः । \newline
20. अ॒यम् ब्रा᳚ह्म॒णो ब्रा᳚ह्म॒णो॑ ऽय म॒यम् ब्रा᳚ह्म॒ण इतीति॑ ब्राह्म॒णो॑ ऽय म॒यम् ब्रा᳚ह्म॒ण इति॑ । \newline
21. ब्रा॒ह्म॒ण इतीति॑ ब्राह्म॒णो ब्रा᳚ह्म॒ण इति॒ त्रि स्त्रि रिति॑ ब्राह्म॒णो ब्रा᳚ह्म॒ण इति॒ त्रिः । \newline
22. इति॒ त्रि स्त्रिरि तीति॒ त्रिरु॑पाꣳ॒॒शू॑ पाꣳ॒॒शु त्रिरि तीति॒ त्रि रु॑पाꣳ॒॒शु । \newline
23. त्रि रु॑पाꣳ॒॒शू॑ पाꣳ॒॒शु त्रि स्त्रि रु॑पाꣳ॒॒श्वा॑हा होपाꣳ॒॒शु त्रि स्त्रि रु॑पाꣳ॒॒श्वा॑ह । \newline
24. उ॒पाꣳ॒॒श्वा॑ हाहोपाꣳ॒॒शू॑ पाꣳ॒॒श्वा॑ह दे॒वेभ्यो॑ दे॒वेभ्य॑ आहोपाꣳ॒॒शू॑ पाꣳ॒॒श्वा॑ह दे॒वेभ्यः॑ । \newline
25. उ॒पाꣳ॒॒श्वित्यु॑प - अꣳ॒॒शु । \newline
26. आ॒ह॒ दे॒वेभ्यो॑ दे॒वेभ्य॑ आहाह दे॒वेभ्य॑ ए॒वैव दे॒वेभ्य॑ आहाह दे॒वेभ्य॑ ए॒व । \newline
27. दे॒वेभ्य॑ ए॒वैव दे॒वेभ्यो॑ दे॒वेभ्य॑ ए॒वैन॑ मेन मे॒व दे॒वेभ्यो॑ दे॒वेभ्य॑ ए॒वैन᳚म् । \newline
28. ए॒वैन॑ मेन मे॒वै वैन॒म् प्र प्रैन॑ मे॒वै वैन॒म् प्र । \newline
29. ए॒न॒म् प्र प्रैन॑ मेन॒म् प्राहा॑ह॒ प्रैन॑ मेन॒म् प्राह॑ । \newline
30. प्राहा॑ह॒ प्र प्राह॒ त्रि स्त्रिरा॑ह॒ प्र प्राह॒ त्रिः । \newline
31. आ॒ह॒ त्रि स्त्रिरा॑ हाह॒ त्रि रु॒च्चै रु॒च्चै स्त्रि रा॑हाह॒ त्रि रु॒च्चैः । \newline
32. त्रि रु॒च्चै रु॒च्चै स्त्रि स्त्रि रु॒च्चै रु॒भये᳚भ्य उ॒भये᳚भ्य उ॒च्चै स्त्रि स्त्रि रु॒च्चै रु॒भये᳚भ्यः । \newline
33. उ॒च्चै रु॒भये᳚भ्य उ॒भये᳚भ्य उ॒च्चै रु॒च्चै रु॒भये᳚भ्य ए॒वै वोभये᳚भ्य उ॒च्चै रु॒च्चै रु॒भये᳚भ्य ए॒व । \newline
34. उ॒भये᳚भ्य ए॒वै वोभये᳚भ्य उ॒भये᳚भ्य ए॒वैन॑ मेन मे॒वोभये᳚भ्य उ॒भये᳚भ्य ए॒वैन᳚म् । \newline
35. ए॒वैन॑ मेन मे॒वै वैन॑म् देवमनु॒ष्येभ्यो॑ देवमनु॒ष्येभ्य॑ एन मे॒वै वैन॑म् देवमनु॒ष्येभ्यः॑ । \newline
36. ए॒न॒म् दे॒व॒म॒नु॒ष्येभ्यो॑ देवमनु॒ष्येभ्य॑ एन मेनम् देवमनु॒ष्येभ्यः॒ प्र प्र दे॑वमनु॒ष्येभ्य॑ एन मेनम् देवमनु॒ष्येभ्यः॒ प्र । \newline
37. दे॒व॒म॒नु॒ष्येभ्यः॒ प्र प्र दे॑वमनु॒ष्येभ्यो॑ देवमनु॒ष्येभ्यः॒ प्राहा॑ह॒ प्र दे॑वमनु॒ष्येभ्यो॑ देवमनु॒ष्येभ्यः॒ प्राह॑ । \newline
38. दे॒व॒म॒नु॒ष्येभ्य॒ इति॑ देव - म॒नु॒ष्येभ्यः॑ । \newline
39. प्राहा॑ह॒ प्र प्राह॒ न नाह॒ प्र प्राह॒ न । \newline
40. आ॒ह॒ न नाहा॑ह॒ न पु॒रा पु॒रा नाहा॑ह॒ न पु॒रा । \newline
41. न पु॒रा पु॒रा न न पु॒रा नक्ष॑त्रेभ्यो॒ नक्ष॑त्रेभ्यः पु॒रा न न पु॒रा नक्ष॑त्रेभ्यः । \newline
42. पु॒रा नक्ष॑त्रेभ्यो॒ नक्ष॑त्रेभ्यः पु॒रा पु॒रा नक्ष॑त्रेभ्यो॒ वाचं॒ ॅवाच॒म् नक्ष॑त्रेभ्यः पु॒रा पु॒रा नक्ष॑त्रेभ्यो॒ वाच᳚म् । \newline
43. नक्ष॑त्रेभ्यो॒ वाचं॒ ॅवाच॒म् नक्ष॑त्रेभ्यो॒ नक्ष॑त्रेभ्यो॒ वाचं॒ ॅवि वि वाच॒म् नक्ष॑त्रेभ्यो॒ नक्ष॑त्रेभ्यो॒ वाचं॒ ॅवि । \newline
44. वाचं॒ ॅवि वि वाचं॒ ॅवाचं॒ ॅवि सृ॑जेथ् सृजे॒द् वि वाचं॒ ॅवाचं॒ ॅवि सृ॑जेत् । \newline
45. वि सृ॑जेथ् सृजे॒द् वि वि सृ॑जे॒द् यद् यथ् सृ॑जे॒द् वि वि सृ॑जे॒द् यत् । \newline
46. सृ॒जे॒द् यद् यथ् सृ॑जेथ् सृजे॒द् यत् पु॒रा पु॒रा यथ् सृ॑जेथ् सृजे॒द् यत् पु॒रा । \newline
47. यत् पु॒रा पु॒रा यद् यत् पु॒रा नक्ष॑त्रेभ्यो॒ नक्ष॑त्रेभ्यः पु॒रा यद् यत् पु॒रा नक्ष॑त्रेभ्यः । \newline
48. पु॒रा नक्ष॑त्रेभ्यो॒ नक्ष॑त्रेभ्यः पु॒रा पु॒रा नक्ष॑त्रेभ्यो॒ वाचं॒ ॅवाच॒म् नक्ष॑त्रेभ्यः पु॒रा पु॒रा नक्ष॑त्रेभ्यो॒ वाच᳚म् । \newline
49. नक्ष॑त्रेभ्यो॒ वाचं॒ ॅवाच॒म् नक्ष॑त्रेभ्यो॒ नक्ष॑त्रेभ्यो॒ वाचं॑ ॅविसृ॒जेद् वि॑सृ॒जेद् वाच॒म् नक्ष॑त्रेभ्यो॒ नक्ष॑त्रेभ्यो॒ वाचं॑ ॅविसृ॒जेत् । \newline
50. वाचं॑ ॅविसृ॒जेद् वि॑सृ॒जेद् वाचं॒ ॅवाचं॑ ॅविसृ॒जेद् य॒ज्ञ्ं ॅय॒ज्ञ्ं ॅवि॑सृ॒जेद् वाचं॒ ॅवाचं॑ ॅविसृ॒जेद् य॒ज्ञ्म् । \newline
51. वि॒सृ॒जेद् य॒ज्ञ्ं ॅय॒ज्ञ्ं ॅवि॑सृ॒जेद् वि॑सृ॒जेद् य॒ज्ञ्ं ॅवि वि य॒ज्ञ्ं ॅवि॑सृ॒जेद् वि॑सृ॒जेद् य॒ज्ञ्ं ॅवि । \newline
52. वि॒सृ॒जेदिति॑ वि - सृ॒जेत् । \newline
53. य॒ज्ञ्ं ॅवि वि य॒ज्ञ्ं ॅय॒ज्ञ्ं ॅवि च्छि॑न्द्याच् छिन्द्याद् वि य॒ज्ञ्ं ॅय॒ज्ञ्ं ॅवि च्छि॑न्द्यात् । \newline
54. वि च्छि॑न्द्याच् छिन्द्यात् वि वि च्छि॑न्द्या॒ दुदि॑ते॒षू दि॑तेषु छिन्द्याद् वि वि च्छि॑न्द्या॒ दुदि॑तेषु । \newline
55. छि॒न्द्या॒ दुदि॑ते॒षू दि॑तेषु छिन्द्याच् छिन्द्या॒ दुदि॑तेषु॒ नक्ष॑त्रेषु॒ नक्ष॑त्रे॒षू दि॑तेषु छिन्द्याच् छिन्द्या॒ दुदि॑तेषु॒ नक्ष॑त्रेषु । \newline
\pagebreak
\markright{ TS 6.1.4.4  \hfill https://www.vedavms.in \hfill}

\section{ TS 6.1.4.4 }

\textbf{TS 6.1.4.4 } \newline
\textbf{Samhita Paata} \newline

-दुदि॑तेषु॒ नक्ष॑त्रेषु व्र॒तं कृ॑णु॒तेति॒ वाचं॒ ॅवि सृ॑जति य॒ज्ञ्व्र॑तो॒ वै दी᳚क्षि॒तो य॒ज्ञ्मे॒वाभि वाचं॒ ॅवि सृ॑जति॒ यदि॑ विसृ॒जेद्-वै᳚ष्ण॒वीमृच॒मनु॑ ब्रूयाद्-य॒ज्ञो वै विष्णु॑र्य॒ज्ञेनै॒व य॒ज्ञ्ꣳ सं त॑नोति॒ दैवीं॒ धियं॑ मनामह॒ इत्या॑ह य॒ज्ञ्मे॒व तन्म्र॑दयति सुपा॒रा नो॑ अस॒द्वश॒ इत्या॑ह॒ व्यु॑ष्टिमे॒वाव॑ रुन्धे - [  ] \newline

\textbf{Pada Paata} \newline

उदि॑ते॒ष्वित्युत् - इ॒ते॒षु॒ । नक्ष॑त्रेषु । व्र॒तम् । कृ॒णु॒त॒ । इति॑ । वाच᳚म् । वीति॑ । सृ॒ज॒ति॒ । य॒ज्ञ्व्र॑त॒ इति॑ य॒ज्ञ्-व्र॒तः॒ । वै । दी॒क्षि॒तः । य॒ज्ञ्म् । ए॒व । अ॒भीति॑ । वाच᳚म् । वीति॑ । सृ॒ज॒ति॒ । यदि॑ । वि॒सृ॒जेदिति॑ वि - सृ॒जेत् । वै॒ष्ण॒वीम् । ऋच᳚म् । अन्विति॑ । ब्रू॒या॒त् । य॒ज्ञ्ः । वै । विष्णुः॑ । य॒ज्ञेन॑ । ए॒व । य॒ज्ञ्म् । समिति॑ । त॒नो॒ति॒ । दैवी᳚म् । धिय᳚म् । म॒ना॒म॒हे॒ । इति॑ । आ॒ह॒ । य॒ज्ञ्म् । ए॒व । तत् । म्र॒द॒य॒ति॒ । सु॒पा॒रेति॑ सु - पा॒रा । नः॒ । अ॒स॒त् । वशे᳚ । इति॑ । आ॒ह॒ । व्यु॑ष्टि॒मिति॒ वि - उ॒ष्टि॒म् । ए॒व । अवेति॑ । रु॒न्धे॒ ।  \newline


\textbf{Krama Paata} \newline

उदि॑तेषु॒ नक्ष॑त्रेषु । उदि॑ते॒ष्वित्युत् - इ॒ते॒षु॒ । नक्ष॑त्रेषु व्र॒तम् । व्र॒तम् कृ॑णुत । कृ॒णु॒तेति॑ । इति॒ वाच᳚म् । वाच॒म् ॅवि । वि सृ॑जति । सृ॒ज॒ति॒ य॒ज्ञ्व्र॑तः । य॒ज्ञ्व्र॑तो॒ वै । य॒ज्ञ्व्र॑त॒ इति॑ य॒ज्ञ् - व्र॒तः॒ । वै दी᳚क्षि॒तः । दी॒क्षि॒तो य॒ज्ञ्म् । य॒ज्ञ्मे॒व । ए॒वाभि । अ॒भि वाच᳚म् । वाच॒म् ॅवि । वि सृ॑जति । सृ॒ज॒ति॒ यदि॑ । यदि॑ विसृ॒जेत् । वि॒सृ॒जेद् वै᳚ष्ण॒वीम् । वि॒सृ॒जेदिति॑ वि - सृ॒जेत् । वै॒ष्ण॒वीमृच᳚म् । ऋच॒मनु॑ । अनु॑ ब्रूयात् । ब्रू॒या॒द् य॒ज्ञ्ः । य॒ज्ञो वै । वै विष्णुः॑ । विष्णु॑र् य॒ज्ञेन॑ । य॒ज्ञेनै॒व । ए॒व य॒ज्ञ्म् । य॒ज्ञ्ꣳ सम् । सम् त॑नोति । त॒नो॒ति॒ दैवी᳚म् । दैवी॒म् धिय᳚म् । धिय॑म् मनामहे । म॒ना॒म॒ह॒ इति॑ । इत्या॑ह । आ॒ह॒ य॒ज्ञ्म् । य॒ज्ञ्मे॒व । ए॒व तत् । तन् म्र॑दयति । म्र॒द॒य॒ति॒ सु॒पा॒रा । सु॒पा॒रा नः॑ । सु॒पा॒रेति॑ सु - पा॒रा । नो॒ अ॒स॒त्॒ । अ॒स॒द् वशे᳚ । वश॒ इति॑ । इत्या॑ह । आ॒ह॒ व्यु॑ष्टिम् । व्यु॑ष्टिमे॒व । व्यु॑ष्टि॒मिति॒ वि - उ॒ष्टि॒म् । ए॒वाव॑ । अव॑ रुन्धे । रु॒न्धे॒ ब्र॒ह्म॒वा॒दिनः॑ \newline

\textbf{Jatai Paata} \newline

1. उदि॑तेषु॒ नक्ष॑त्रेषु॒ नक्ष॑त्रे॒षू दि॑ते॒षू दि॑तेषु॒ नक्ष॑त्रेषु । \newline
2. उदि॑ते॒ष्वित्युत् - इ॒ते॒षु॒ । \newline
3. नक्ष॑त्रेषु व्र॒तं ॅव्र॒तन् नक्ष॑त्रेषु॒ नक्ष॑त्रेषु व्र॒तम् । \newline
4. व्र॒तम् कृ॑णुत कृणुत व्र॒तं ॅव्र॒तम् कृ॑णुत । \newline
5. कृ॒णु॒ते तीति॑ कृणुत कृणु॒तेति॑ । \newline
6. इति॒ वाचं॒ ॅवाच॒ मितीति॒ वाच᳚म् । \newline
7. वाचं॒ ॅवि वि वाचं॒ ॅवाचं॒ ॅवि । \newline
8. वि सृ॑जति सृजति॒ वि वि सृ॑जति । \newline
9. सृ॒ज॒ति॒ य॒ज्ञ्व्र॑तो य॒ज्ञ्व्र॑तः सृजति सृजति य॒ज्ञ्व्र॑तः । \newline
10. य॒ज्ञ्व्र॑तो॒ वै वै य॒ज्ञ्व्र॑तो य॒ज्ञ्व्र॑तो॒ वै । \newline
11. य॒ज्ञ्व्र॑त॒ इति॑ य॒ज्ञ् - व्र॒तः॒ । \newline
12. वै दी᳚क्षि॒तो दी᳚क्षि॒तो वै वै दी᳚क्षि॒तः । \newline
13. दी॒क्षि॒तो य॒ज्ञ्ं ॅय॒ज्ञ्म् दी᳚क्षि॒तो दी᳚क्षि॒तो य॒ज्ञ्म् । \newline
14. य॒ज्ञ् मे॒वैव य॒ज्ञ्ं ॅय॒ज्ञ् मे॒व । \newline
15. ए॒वाभ्या᳚(1॒)भ्ये॑ वैवाभि । \newline
16. अ॒भि वाचं॒ ॅवाच॑ म॒भ्य॑भि वाच᳚म् । \newline
17. वाचं॒ ॅवि वि वाचं॒ ॅवाचं॒ ॅवि । \newline
18. वि सृ॑जति सृजति॒ वि वि सृ॑जति । \newline
19. सृ॒ज॒ति॒ यदि॒ यदि॑ सृजति सृजति॒ यदि॑ । \newline
20. यदि॑ विसृ॒जेद् वि॑सृ॒जेद् यदि॒ यदि॑ विसृ॒जेत् । \newline
21. वि॒सृ॒जेद् वै᳚ष्ण॒वीं ॅवै᳚ष्ण॒वीं ॅवि॑सृ॒जेद् वि॑सृ॒जेद् वै᳚ष्ण॒वीम् । \newline
22. वि॒सृ॒जेदिति॑ वि - सृ॒जेत् । \newline
23. वै॒ष्ण॒वी मृच॒ मृचं॑ ॅवैष्ण॒वीं ॅवै᳚ष्ण॒वी मृच᳚म् । \newline
24. ऋच॒ मन्वन् वृच॒ मृच॒ मनु॑ । \newline
25. अनु॑ ब्रूयाद् ब्रूया॒ दन्वनु॑ ब्रूयात् । \newline
26. ब्रू॒या॒द् य॒ज्ञो य॒ज्ञो ब्रू॑याद् ब्रूयाद् य॒ज्ञ्ः । \newline
27. य॒ज्ञो वै वै य॒ज्ञो य॒ज्ञो वै । \newline
28. वै विष्णु॒र् विष्णु॒र् वै वै विष्णुः॑ । \newline
29. विष्णु॑र् य॒ज्ञेन॑ य॒ज्ञेन॒ विष्णु॒र् विष्णु॑र् य॒ज्ञेन॑ । \newline
30. य॒ज्ञेनै॒ वैव य॒ज्ञेन॑ य॒ज्ञेनै॒व । \newline
31. ए॒व य॒ज्ञ्ं ॅय॒ज्ञ् मे॒वैव य॒ज्ञ्म् । \newline
32. य॒ज्ञ्ꣳ सꣳ सं ॅय॒ज्ञ्ं ॅय॒ज्ञ्ꣳ सम् । \newline
33. सम् त॑नोति तनोति॒ सꣳ सम् त॑नोति । \newline
34. त॒नो॒ति॒ दैवी॒म् दैवी᳚म् तनोति तनोति॒ दैवी᳚म् । \newline
35. दैवी॒म् धिय॒म् धिय॒म् दैवी॒म् दैवी॒म् धिय᳚म् । \newline
36. धिय॑म् मनामहे मनामहे॒ धिय॒म् धिय॑म् मनामहे । \newline
37. म॒ना॒म॒ह॒ इतीति॑ मनामहे मनामह॒ इति॑ । \newline
38. इत्या॑हा॒हे तीत्या॑ह । \newline
39. आ॒ह॒ य॒ज्ञ्ं ॅय॒ज्ञ् मा॑हाह य॒ज्ञ्म् । \newline
40. य॒ज्ञ् मे॒वैव य॒ज्ञ्ं ॅय॒ज्ञ् मे॒व । \newline
41. ए॒व तत् तदे॒ वैव तत् । \newline
42. तन् म्र॑दयति म्रदयति॒ तत् तन् म्र॑दयति । \newline
43. म्र॒द॒य॒ति॒ सु॒पा॒रा सु॑पा॒रा म्र॑दयति म्रदयति सुपा॒रा । \newline
44. सु॒पा॒रा नो॑ नः सुपा॒रा सु॑पा॒रा नः॑ । \newline
45. सु॒पा॒रेति॑ सु - पा॒रा । \newline
46. नो॒ अ॒स॒ द॒स॒न् नो॒ नो॒ अ॒स॒त् । \newline
47. अ॒स॒द् वशे॒ वशे॑ ऽस दस॒द् वशे᳚ । \newline
48. वश॒ इतीति॒ वशे॒ वश॒ इति॑ । \newline
49. इत्या॑हा॒हे तीत्या॑ह । \newline
50. आ॒ह॒ व्यु॑ष्टिं॒ ॅव्यु॑ष्टि माहाह॒ व्यु॑ष्टिम् । \newline
51. व्यु॑ष्टि मे॒वैव व्यु॑ष्टिं॒ ॅव्यु॑ष्टि मे॒व । \newline
52. व्यु॑ष्टि॒मिति॒ वि - उ॒ष्टि॒म् । \newline
53. ए॒वावा वै॒वै वाव॑ । \newline
54. अव॑ रुन्धे रु॒न्धे ऽवाव॑ रुन्धे । \newline
55. रु॒न्धे॒ ब्र॒ह्म॒वा॒दिनो᳚ ब्रह्मवा॒दिनो॑ रुन्धे रुन्धे ब्रह्मवा॒दिनः॑ । \newline

\textbf{Ghana Paata } \newline

1. उदि॑तेषु॒ नक्ष॑त्रेषु॒ नक्ष॑त्रे॒षू दि॑ते॒ षूदि॑तेषु॒ नक्ष॑त्रेषु व्र॒तं ॅव्र॒तन् नक्ष॑त्रे॒षू दि॑ते॒षू दि॑तेषु॒ नक्ष॑त्रेषु व्र॒तम् । \newline
2. उदि॑ते॒ष्वित्युत् - इ॒ते॒षु॒ । \newline
3. नक्ष॑त्रेषु व्र॒तं ॅव्र॒तम् नक्ष॑त्रेषु॒ नक्ष॑त्रेषु व्र॒तम् कृ॑णुत कृणुत व्र॒तम् नक्ष॑त्रेषु॒ नक्ष॑त्रेषु व्र॒तम् कृ॑णुत । \newline
4. व्र॒तम् कृ॑णुत कृणुत व्र॒तं ॅव्र॒तम् कृ॑णु॒ते तीति॑ कृणुत व्र॒तं ॅव्र॒तम् कृ॑णु॒तेति॑ । \newline
5. कृ॒णु॒ते तीति॑ कृणुत कृणु॒तेति॒ वाचं॒ ॅवाच॒ मिति॑ कृणुत कृणु॒तेति॒ वाच᳚म् । \newline
6. इति॒ वाचं॒ ॅवाच॒ मितीति॒ वाचं॒ ॅवि वि वाच॒ मितीति॒ वाचं॒ ॅवि । \newline
7. वाचं॒ ॅवि वि वाचं॒ ॅवाचं॒ ॅवि सृ॑जति सृजति॒ वि वाचं॒ ॅवाचं॒ ॅवि सृ॑जति । \newline
8. वि सृ॑जति सृजति॒ वि वि सृ॑जति य॒ज्ञ्व्र॑तो य॒ज्ञ्व्र॑तः सृजति॒ वि वि सृ॑जति य॒ज्ञ्व्र॑तः । \newline
9. सृ॒ज॒ति॒ य॒ज्ञ्व्र॑तो य॒ज्ञ्व्र॑तः सृजति सृजति य॒ज्ञ्व्र॑तो॒ वै वै य॒ज्ञ्व्र॑तः सृजति सृजति य॒ज्ञ्व्र॑तो॒ वै । \newline
10. य॒ज्ञ्व्र॑तो॒ वै वै य॒ज्ञ्व्र॑तो य॒ज्ञ्व्र॑तो॒ वै दी᳚क्षि॒तो दी᳚क्षि॒तो वै य॒ज्ञ्व्र॑तो य॒ज्ञ्व्र॑तो॒ वै दी᳚क्षि॒तः । \newline
11. य॒ज्ञ्व्र॑त॒ इति॑ य॒ज्ञ् - व्र॒तः॒ । \newline
12. वै दी᳚क्षि॒तो दी᳚क्षि॒तो वै वै दी᳚क्षि॒तो य॒ज्ञ्ं ॅय॒ज्ञ्म् दी᳚क्षि॒तो वै वै दी᳚क्षि॒तो य॒ज्ञ्म् । \newline
13. दी॒क्षि॒तो य॒ज्ञ्ं ॅय॒ज्ञ्म् दी᳚क्षि॒तो दी᳚क्षि॒तो य॒ज्ञ् मे॒वैव य॒ज्ञ्म् दी᳚क्षि॒तो दी᳚क्षि॒तो य॒ज्ञ् मे॒व । \newline
14. य॒ज्ञ् मे॒वैव य॒ज्ञ्ं ॅय॒ज्ञ् मे॒वाभ्या᳚(1॒) भ्ये॑व य॒ज्ञ्ं ॅय॒ज्ञ् मे॒वाभि । \newline
15. ए॒वाभ्या᳚(1॒) भ्ये॑ वैवाभि वाचं॒ ॅवाच॑ म॒भ्ये॑ वैवाभि वाच᳚म् । \newline
16. अ॒भि वाचं॒ ॅवाच॑ म॒भ्य॑भि वाचं॒ ॅवि वि वाच॑ म॒भ्य॑भि वाचं॒ ॅवि । \newline
17. वाचं॒ ॅवि वि वाचं॒ ॅवाचं॒ ॅवि सृ॑जति सृजति॒ वि वाचं॒ ॅवाचं॒ ॅवि सृ॑जति । \newline
18. वि सृ॑जति सृजति॒ वि वि सृ॑जति॒ यदि॒ यदि॑ सृजति॒ वि वि सृ॑जति॒ यदि॑ । \newline
19. सृ॒ज॒ति॒ यदि॒ यदि॑ सृजति सृजति॒ यदि॑ विसृ॒जेद् वि॑सृ॒जेद् यदि॑ सृजति सृजति॒ यदि॑ विसृ॒जेत् । \newline
20. यदि॑ विसृ॒जेद् वि॑सृ॒जेद् यदि॒ यदि॑ विसृ॒जेद् वै᳚ष्ण॒वीं ॅवै᳚ष्ण॒वीं ॅवि॑सृ॒जेद् यदि॒ यदि॑ विसृ॒जेद् वै᳚ष्ण॒वीम् । \newline
21. वि॒सृ॒जेद् वै᳚ष्ण॒वीं ॅवै᳚ष्ण॒वीं ॅवि॑सृ॒जेद् वि॑सृ॒जेद् वै᳚ष्ण॒वी मृच॒ मृचं॑ ॅवैष्ण॒वीं ॅवि॑सृ॒जेद् वि॑सृ॒जेद् वै᳚ष्ण॒वी मृच᳚म् । \newline
22. वि॒सृ॒जेदिति॑ वि - सृ॒जेत् । \newline
23. वै॒ष्ण॒वी मृच॒ मृचं॑ ॅवैष्ण॒वीं ॅवै᳚ष्ण॒वी मृच॒ मन्वन् वृचं॑ ॅवैष्ण॒वीं ॅवै᳚ष्ण॒वी मृच॒ मनु॑ । \newline
24. ऋच॒ मन्वन् वृच॒ मृच॒ मनु॑ ब्रूयाद् ब्रूया॒ दन् वृच॒ मृच॒ मनु॑ ब्रूयात् । \newline
25. अनु॑ ब्रूयाद् ब्रूया॒ दन्वनु॑ ब्रूयाद् य॒ज्ञो य॒ज्ञो ब्रू॑या॒ दन्वनु॑ ब्रूयाद् य॒ज्ञ्ः । \newline
26. ब्रू॒या॒द् य॒ज्ञो य॒ज्ञो ब्रू॑याद् ब्रूयाद् य॒ज्ञो वै वै य॒ज्ञो ब्रू॑याद् ब्रूयाद् य॒ज्ञो वै । \newline
27. य॒ज्ञो वै वै य॒ज्ञो य॒ज्ञो वै विष्णु॒र् विष्णु॒र् वै य॒ज्ञो य॒ज्ञो वै विष्णुः॑ । \newline
28. वै विष्णु॒र् विष्णु॒र् वै वै विष्णु॑र् य॒ज्ञेन॑ य॒ज्ञेन॒ विष्णु॒र् वै वै विष्णु॑र् य॒ज्ञेन॑ । \newline
29. विष्णु॑र् य॒ज्ञेन॑ य॒ज्ञेन॒ विष्णु॒र् विष्णु॑र् य॒ज्ञे नै॒वैव य॒ज्ञेन॒ विष्णु॒र् विष्णु॑र् य॒ज्ञेनै॒व । \newline
30. य॒ज्ञे नै॒वैव य॒ज्ञेन॑ य॒ज्ञेनै॒व य॒ज्ञ्ं ॅय॒ज्ञ् मे॒व य॒ज्ञेन॑ य॒ज्ञेनै॒व य॒ज्ञ्म् । \newline
31. ए॒व य॒ज्ञ्ं ॅय॒ज्ञ् मे॒वैव य॒ज्ञ्ꣳ सꣳ सं ॅय॒ज्ञ् मे॒वैव य॒ज्ञ्ꣳ सम् । \newline
32. य॒ज्ञ्ꣳ सꣳ सं ॅय॒ज्ञ्ं ॅय॒ज्ञ्ꣳ सम् त॑नोति तनोति॒ सं ॅय॒ज्ञ्ं ॅय॒ज्ञ्ꣳ सम् त॑नोति । \newline
33. सम् त॑नोति तनोति॒ सꣳ सम् त॑नोति॒ दैवी॒म् दैवी᳚म् तनोति॒ सꣳ सम् त॑नोति॒ दैवी᳚म् । \newline
34. त॒नो॒ति॒ दैवी॒म् दैवी᳚म् तनोति तनोति॒ दैवी॒म् धिय॒म् धिय॒म् दैवी᳚म् तनोति तनोति॒ दैवी॒म् धिय᳚म् । \newline
35. दैवी॒म् धिय॒म् धिय॒म् दैवी॒म् दैवी॒म् धिय॑म् मनामहे मनामहे॒ धिय॒म् दैवी॒म् दैवी॒म् धिय॑म् मनामहे । \newline
36. धिय॑म् मनामहे मनामहे॒ धिय॒म् धिय॑म् मनामह॒ इतीति॑ मनामहे॒ धिय॒म् धिय॑म् मनामह॒ इति॑ । \newline
37. म॒ना॒म॒ह॒ इतीति॑ मनामहे मनामह॒ इत्या॑हा॒ हेति॑ मनामहे मनामह॒ इत्या॑ह । \newline
38. इत्या॑हा॒हे तीत्या॑ह य॒ज्ञ्ं ॅय॒ज्ञ् मा॒हे तीत्या॑ह य॒ज्ञ्म् । \newline
39. आ॒ह॒ य॒ज्ञ्ं ॅय॒ज्ञ् मा॑हाह य॒ज्ञ् मे॒वैव य॒ज्ञ् मा॑हाह य॒ज्ञ् मे॒व । \newline
40. य॒ज्ञ् मे॒वैव य॒ज्ञ्ं ॅय॒ज्ञ् मे॒व तत् तदे॒व य॒ज्ञ्ं ॅय॒ज्ञ् मे॒व तत् । \newline
41. ए॒व तत् तदे॒ वैव तन् म्र॑दयति म्रदयति॒ तदे॒वैव तन् म्र॑दयति । \newline
42. तन् म्र॑दयति म्रदयति॒ तत् तन् म्र॑दयति सुपा॒रा सु॑पा॒रा म्र॑दयति॒ तत् तन् म्र॑दयति सुपा॒रा । \newline
43. म्र॒द॒य॒ति॒ सु॒पा॒रा सु॑पा॒रा म्र॑दयति म्रदयति सुपा॒रा नो॑ नः सुपा॒रा म्र॑दयति म्रदयति सुपा॒रा नः॑ । \newline
44. सु॒पा॒रा नो॑ नः सुपा॒रा सु॑पा॒रा नो॑ अस दसन् नः सुपा॒रा सु॑पा॒रा नो॑ असत् । \newline
45. सु॒पा॒रेति॑ सु - पा॒रा । \newline
46. नो॒ अ॒स॒ द॒स॒न् नो॒ नो॒ अ॒स॒द् वशे॒ वशे॑ ऽसन् नो नो अस॒द् वशे᳚ । \newline
47. अ॒स॒द् वशे॒ वशे॑ ऽस दस॒द् वश॒ इतीति॒ वशे॑ ऽस दस॒द् वश॒ इति॑ । \newline
48. वश॒ इतीति॒ वशे॒ वश॒ इत्या॑हा॒ हेति॒ वशे॒ वश॒ इत्या॑ह । \newline
49. इत्या॑हा॒हे तीत्या॑ह॒ व्यु॑ष्टिं॒ ॅव्यु॑ष्टि मा॒हे तीत्या॑ह॒ व्यु॑ष्टिम् । \newline
50. आ॒ह॒ व्यु॑ष्टिं॒ ॅव्यु॑ष्टि माहाह॒ व्यु॑ष्टि मे॒वैव व्यु॑ष्टि माहाह॒ व्यु॑ष्टि मे॒व । \newline
51. व्यु॑ष्टि मे॒वैव व्यु॑ष्टिं॒ ॅव्यु॑ष्टि मे॒वावा वै॒व व्यु॑ष्टिं॒ ॅव्यु॑ष्टि मे॒वाव॑ । \newline
52. व्यु॑ष्टि॒मिति॒ वि - उ॒ष्टि॒म् । \newline
53. ए॒वावा वै॒वै वाव॑ रुन्धे रु॒न्धे ऽवै॒वै वाव॑ रुन्धे । \newline
54. अव॑ रुन्धे रु॒न्धे ऽवाव॑ रुन्धे ब्रह्मवा॒दिनो᳚ ब्रह्मवा॒दिनो॑ रु॒न्धे ऽवाव॑ रुन्धे ब्रह्मवा॒दिनः॑ । \newline
55. रु॒न्धे॒ ब्र॒ह्म॒वा॒दिनो᳚ ब्रह्मवा॒दिनो॑ रुन्धे रुन्धे ब्रह्मवा॒दिनो॑ वदन्ति वदन्ति ब्रह्मवा॒दिनो॑ रुन्धे रुन्धे ब्रह्मवा॒दिनो॑ वदन्ति । \newline
\pagebreak
\markright{ TS 6.1.4.5  \hfill https://www.vedavms.in \hfill}

\section{ TS 6.1.4.5 }

\textbf{TS 6.1.4.5 } \newline
\textbf{Samhita Paata} \newline

ब्रह्मवा॒दिनो॑ वदन्ति होत॒व्यं॑ दीक्षि॒तस्य॑ गृ॒हा(3) इ न हो॑त॒व्या(3)मिति॑ ह॒विर्वै दी᳚क्षि॒तो यज्जु॑हु॒याद्-यज॑मानस्याव॒दाय॑ जुहुया॒द्-यन्न जु॑हु॒याद्-य॑ज्ञ्प॒रुर॒न्तरि॑या॒द्ये दे॒वा मनो॑जाता मनो॒युज॒ इत्या॑ह प्रा॒णा वै दे॒वा मनो॑जाता मनो॒युज॒स्तेष्वे॒व प॒रोक्षं॑ जुहोति॒ तन्नेव॑ हु॒तं नेवाहु॑तꣳ स्व॒पन्तं॒ ॅवै दी᳚क्षि॒तꣳ रक्षाꣳ॑सि जिघाꣳसन्त्य॒ग्निः- [  ] \newline

\textbf{Pada Paata} \newline

ब्र॒ह्म॒वा॒दिन॒ इति॑ ब्रह्म - वा॒दिनः॑ । व॒द॒न्ति॒ । हो॒त॒व्य᳚म् । दी॒क्षि॒तस्य॑ । गृ॒हा(3) इ । न । हो॒त॒व्या(3)म् । इति॑ । ह॒विः । वै । दी॒क्षि॒तः । यत् । जु॒हु॒यात् । यज॑मानस्य । अ॒व॒दायेत्य॑व - दाय॑ । जु॒हु॒या॒त् । यत् । न । जु॒हु॒यात् । य॒ज्ञ्॒प॒रुरिति॑ यज्ञ् - प॒रुः । अ॒न्तः । इ॒या॒त् । ये । दे॒वाः । मनो॑जाता॒ इति॒ मनः॑ - जा॒ताः॒ । म॒नो॒युज॒ इति॑ मनः-युजः॑ । इति॑ । आ॒ह॒ । प्रा॒णा इति॑ प्र - अ॒नाः । वै । दे॒वाः । मनो॑जाता॒ इति॒ मनः॑ - जा॒ताः॒ । म॒नो॒युज॒ इति॑ मनः - युजः॑ । तेषु॑ । ए॒व । प॒रोक्ष॒मिति॑ परः - अक्ष᳚म् । जु॒हो॒ति॒ । तत् । न । इ॒व॒ । हु॒तम् । न । इ॒व॒ । अहु॑तम् । स्व॒पन्त᳚म् । वै । दी॒क्षि॒तम् । रक्षाꣳ॑सि । जि॒घाꣳ॒॒स॒न्ति॒ । अ॒ग्निः ।  \newline


\textbf{Krama Paata} \newline

ब्र॒ह्म॒वा॒दिनो॑ वदन्ति । ब्र॒ह्म॒वा॒दिन॒ इति॑ ब्रह्म - वा॒दिनः॑ । व॒द॒न्ति॒ हो॒त॒व्य᳚म् । हो॒त॒व्य॑म् दीक्षि॒तस्य॑ । दी॒क्षि॒तस्य॑ गृ॒हा(3)इ । गृ॒हा(3)इ न । न हो॑त॒व्या(3)म् । हो॒त॒व्या(3)मिति॑ । इति॑ ह॒विः । ह॒विर् वै । वै दी᳚क्षि॒तः । दी॒क्षि॒तो यत् । यज् जु॑हु॒यात् । जु॒हु॒याद् यज॑मानस्य । यज॑मानस्याव॒दाय॑ । अ॒व॒दाय॑ जुहुयात् । अ॒व॒दायेत्य॑व - दाय॑ । जु॒हु॒या॒द् यत् । यन् न । न जु॑हु॒यात् । जु॒हु॒याद् य॑ज्ञ्प॒रुः । य॒ज्ञ्॒प॒रुर॒न्तः । य॒ज्ञ्॒प॒रुरिति॑ यज्ञ् - प॒रुः । अ॒न्तरि॑यात् । इ॒या॒द् ये । ये दे॒वाः । दे॒वा मनो॑जाताः । मनो॑जाता मनो॒युजः॑ । मनो॑जाता॒ इति॒ मनः॑ - जा॒ताः॒ । म॒नो॒युज॒ इति॑ । म॒नो॒युज॒ इति॑ मनः - युजः॑ । इत्या॑ह । आ॒ह॒ प्रा॒णाः । प्रा॒णा वै । प्रा॒णा इति॑ प्र - अ॒नाः । वै दे॒वाः । दे॒वा मनो॑जाताः । मनो॑जाता मनो॒युजः॑ । मनो॑जाता॒ इति॒ मनः॑ - जा॒ताः॒ । म॒नो॒युज॒स्तेषु॑ । म॒नो॒युज॒ इति॑ मनः - युजः॑ । तेष्वे॒व । ए॒व प॒रोक्ष᳚म् । प॒रोक्ष॑म् जुहोति । प॒रोक्ष॒मिति॑ परः - अक्ष᳚म् । जु॒हो॒ति॒ तत् । तन् न । नेव॑ । इ॒व॒ हु॒तम् । हु॒तम् न । नेव॑ । इ॒वाहु॑तम् । अहु॑तꣳ स्व॒पन्त᳚म् । स्व॒पन्त॒म् ॅवै । वै दी᳚क्षि॒तम् । दी॒क्षि॒तꣳ रक्षाꣳ॑सि । रक्षाꣳ॑सि जिघाꣳसन्ति । जि॒घाꣳ॒॒स॒न्त्य॒ग्निः । अ॒ग्निः खलु॑ \newline

\textbf{Jatai Paata} \newline

1. ब्र॒ह्म॒वा॒दिनो॑ वदन्ति वदन्ति ब्रह्मवा॒दिनो᳚ ब्रह्मवा॒दिनो॑ वदन्ति । \newline
2. ब्र॒ह्म॒वा॒दिन॒ इति॑ ब्रह्म - वा॒दिनः॑ । \newline
3. व॒द॒न्ति॒ हो॒त॒व्यꣳ॑ होत॒व्यं॑ ॅवदन्ति वदन्ति होत॒व्य᳚म् । \newline
4. हो॒त॒व्य॑म् दीक्षि॒तस्य॑ दीक्षि॒तस्य॑ होत॒व्यꣳ॑ होत॒व्य॑म् दीक्षि॒तस्य॑ । \newline
5. दी॒क्षि॒तस्य॑ गृ॒हा(3)इ गृ॒हा(3)इ दी᳚क्षि॒तस्य॑ दीक्षि॒तस्य॑ गृ॒हा(3)इ । \newline
6. गृ॒हा(3)इ न न गृ॒हा(3)इ गृ॒हा(3)इ न । \newline
7. न हो॑त॒व्या(3)म् हो॑त॒व्या(3)म् न न हो॑त॒व्या(3)म् । \newline
8. हो॒त॒व्या(3) मितीति॑ होत॒व्या(3)म् हो॑त॒व्या(3) मिति॑ । \newline
9. इति॑ ह॒विर्. ह॒विरि तीति॑ ह॒विः । \newline
10. ह॒विर् वै वै ह॒विर्. ह॒विर् वै । \newline
11. वै दी᳚क्षि॒तो दी᳚क्षि॒तो वै वै दी᳚क्षि॒तः । \newline
12. दी॒क्षि॒तो यद् यद् दी᳚क्षि॒तो दी᳚क्षि॒तो यत् । \newline
13. यज् जु॑हु॒याज् जु॑हु॒याद् यद् यज् जु॑हु॒यात् । \newline
14. जु॒हु॒याद् यज॑मानस्य॒ यज॑मानस्य जुहु॒याज् जु॑हु॒याद् यज॑मानस्य । \newline
15. यज॑मानस्या व॒दाया॑ व॒दाय॒ यज॑मानस्य॒ यज॑मानस्या व॒दाय॑ । \newline
16. अ॒व॒दाय॑ जुहुयाज् जुहुया दव॒दाया॑ व॒दाय॑ जुहुयात् । \newline
17. अ॒व॒दायेत्य॑व - दाय॑ । \newline
18. जु॒हु॒या॒द् यद् यज् जु॑हुयाज् जुहुया॒द् यत् । \newline
19. यन् न न यद् यन् न । \newline
20. न जु॑हु॒याज् जु॑हु॒यान् न न जु॑हु॒यात् । \newline
21. जु॒हु॒याद् य॑ज्ञ्प॒रुर् य॑ज्ञ्प॒रुर् जु॑हु॒याज् जु॑हु॒याद् य॑ज्ञ्प॒रुः । \newline
22. य॒ज्ञ्॒प॒रु र॒न्त र॒न्तर् य॑ज्ञ्प॒रुर् य॑ज्ञ्प॒रु र॒न्तः । \newline
23. य॒ज्ञ्॒प॒रुरिति॑ यज्ञ् - प॒रुः । \newline
24. अ॒न्त रि॑यादिया द॒न्त र॒न्त रि॑यात् । \newline
25. इ॒या॒द् ये य इ॑यादिया॒द् ये । \newline
26. ये दे॒वा दे॒वा ये ये दे॒वाः । \newline
27. दे॒वा मनो॑जाता॒ मनो॑जाता दे॒वा दे॒वा मनो॑जाताः । \newline
28. मनो॑जाता मनो॒युजो॑ मनो॒युजो॒ मनो॑जाता॒ मनो॑जाता मनो॒युजः॑ । \newline
29. मनो॑जाता॒ इति॒ मनः॑ - जा॒ताः॒ । \newline
30. म॒नो॒युज॒ इतीति॑ मनो॒युजो॑ मनो॒युज॒ इति॑ । \newline
31. म॒नो॒युज॒ इति॑ मनः - युजः॑ । \newline
32. इत्या॑हा॒हे तीत्या॑ह । \newline
33. आ॒ह॒ प्रा॒णाः प्रा॒णा आ॑हाह प्रा॒णाः । \newline
34. प्रा॒णा वै वै प्रा॒णाः प्रा॒णा वै । \newline
35. प्रा॒णा इति॑ प्र - अ॒नाः । \newline
36. वै दे॒वा दे॒वा वै वै दे॒वाः । \newline
37. दे॒वा मनो॑जाता॒ मनो॑जाता दे॒वा दे॒वा मनो॑जाताः । \newline
38. मनो॑जाता मनो॒युजो॑ मनो॒युजो॒ मनो॑जाता॒ मनो॑जाता मनो॒युजः॑ । \newline
39. मनो॑जाता॒ इति॒ मनः॑ - जा॒ताः॒ । \newline
40. म॒नो॒युज॒ स्तेषु॒ तेषु॑ मनो॒युजो॑ मनो॒युज॒ स्तेषु॑ । \newline
41. म॒नो॒युज॒ इति॑ मनः - युजः॑ । \newline
42. तेष्वे॒ वैव तेषु॒ तेष्वे॒व । \newline
43. ए॒व प॒रोक्ष॑म् प॒रोक्ष॑ मे॒वैव प॒रोक्ष᳚म् । \newline
44. प॒रोक्ष॑म् जुहोति जुहोति प॒रोक्ष॑म् प॒रोक्ष॑म् जुहोति । \newline
45. प॒रोक्ष॒मिति॑ परः - अक्ष᳚म् । \newline
46. जु॒हो॒ति॒ तत् तज् जु॑होति जुहोति॒ तत् । \newline
47. तन् न न तत् तन् न । \newline
48. नेवे॑व॒ न नेव॑ । \newline
49. इ॒व॒ हु॒तꣳ हु॒त मि॑वेव हु॒तम् । \newline
50. हु॒तम् न न हु॒तꣳ हु॒तम् न । \newline
51. नेवे॑व॒ न नेव॑ । \newline
52. इ॒वाहु॑त॒ महु॑त मिवे॒ वाहु॑तम् । \newline
53. अहु॑तꣳ स्व॒पन्तꣳ॑ स्व॒पन्त॒ महु॑त॒ महु॑तꣳ स्व॒पन्त᳚म् । \newline
54. स्व॒पन्तं॒ ॅवै वै स्व॒पन्तꣳ॑ स्व॒पन्तं॒ ॅवै । \newline
55. वै दी᳚क्षि॒तम् दी᳚क्षि॒तं ॅवै वै दी᳚क्षि॒तम् । \newline
56. दी॒क्षि॒तꣳ रक्षाꣳ॑सि॒ रक्षाꣳ॑सि दीक्षि॒तम् दी᳚क्षि॒तꣳ रक्षाꣳ॑सि । \newline
57. रक्षाꣳ॑सि जिघाꣳसन्ति जिघाꣳसन्ति॒ रक्षाꣳ॑सि॒ रक्षाꣳ॑सि जिघाꣳसन्ति । \newline
58. जि॒घाꣳ॒॒स॒ न्त्य॒ग्नि र॒ग्निर् जि॑घाꣳसन्ति जिघाꣳस न्त्य॒ग्निः । \newline
59. अ॒ग्निः खलु॒ खल्व॒ग्नि र॒ग्निः खलु॑ । \newline

\textbf{Ghana Paata } \newline

1. ब्र॒ह्म॒वा॒दिनो॑ वदन्ति वदन्ति ब्रह्मवा॒दिनो᳚ ब्रह्मवा॒दिनो॑ वदन्ति होत॒व्यꣳ॑ होत॒व्यं॑ ॅवदन्ति ब्रह्मवा॒दिनो᳚ ब्रह्मवा॒दिनो॑ वदन्ति होत॒व्य᳚म् । \newline
2. ब्र॒ह्म॒वा॒दिन॒ इति॑ ब्रह्म - वा॒दिनः॑ । \newline
3. व॒द॒न्ति॒ हो॒त॒व्यꣳ॑ होत॒व्यं॑ ॅवदन्ति वदन्ति होत॒व्य॑म् दीक्षि॒तस्य॑ दीक्षि॒तस्य॑ होत॒व्यं॑ ॅवदन्ति वदन्ति होत॒व्य॑म् दीक्षि॒तस्य॑ । \newline
4. हो॒त॒व्य॑म् दीक्षि॒तस्य॑ दीक्षि॒तस्य॑ होत॒व्यꣳ॑ होत॒व्य॑म् दीक्षि॒तस्य॑ गृ॒हा(3)इ गृ॒हा(3)इ दी᳚क्षि॒तस्य॑ होत॒व्यꣳ॑ होत॒व्य॑म् दीक्षि॒तस्य॑ गृ॒हा(3)इ । \newline
5. दी॒क्षि॒तस्य॑ गृ॒हा(3)इ गृ॒हा(3)इ दी᳚क्षि॒तस्य॑ दीक्षि॒तस्य॑ गृ॒हा(3)इ न न गृ॒हा(3)इ दी᳚क्षि॒तस्य॑ दीक्षि॒तस्य॑ गृ॒हा(3)इ न । \newline
6. गृ॒हा(3)इ न न गृ॒हा(3)इ गृ॒हा(3)इ न हो॑त॒व्या(3)म् हो॑त॒व्या(3)म् न गृ॒हा(3)इ गृ॒हा(3)इ न हो॑त॒व्या(3)म् । \newline
7. न हो॑त॒व्या(3)म्  हो॑त॒व्या(3)म् न न हो॑त॒व्या(3) मितीति॑ होत॒व्या(3)म् न न हो॑त॒व्या(3) मिति॑ । \newline
8. हो॒त॒व्या(3) मितीति॑ होत॒व्या(3)म् हो॑त॒व्या(3) मिति॑ ह॒विर्. ह॒विरिति॑ होत॒व्या(3)ꣳ हो॑त॒व्या(3) मिति॑ ह॒विः । \newline
9. इति॑ ह॒विर्. ह॒विरि तीति॑ ह॒विर् वै वै ह॒वि रितीति॑ ह॒विर् वै । \newline
10. ह॒विर् वै वै ह॒विर्. ह॒विर् वै दी᳚क्षि॒तो दी᳚क्षि॒तो वै ह॒विर्. ह॒विर् वै दी᳚क्षि॒तः । \newline
11. वै दी᳚क्षि॒तो दी᳚क्षि॒तो वै वै दी᳚क्षि॒तो यद् यद् दी᳚क्षि॒तो वै वै दी᳚क्षि॒तो यत् । \newline
12. दी॒क्षि॒तो यद् यद् दी᳚क्षि॒तो दी᳚क्षि॒तो यज् जु॑हु॒याज् जु॑हु॒याद् यद् दी᳚क्षि॒तो दी᳚क्षि॒तो यज् जु॑हु॒यात् । \newline
13. यज् जु॑हु॒याज् जु॑हु॒याद् यद् यज् जु॑हु॒याद् यज॑मानस्य॒ यज॑मानस्य जुहु॒याद् यद् यज् जु॑हु॒याद् यज॑मानस्य । \newline
14. जु॒हु॒याद् यज॑मानस्य॒ यज॑मानस्य जुहु॒याज् जु॑हु॒याद् यज॑मानस्या व॒दाया॑ व॒दाय॒ यज॑मानस्य जुहु॒याज् जु॑हु॒याद् यज॑मानस्या व॒दाय॑ । \newline
15. यज॑मानस्या व॒दाया॑ व॒दाय॒ यज॑मानस्य॒ यज॑मानस्या व॒दाय॑ जुहुयाज् जुहुया दव॒दाय॒ यज॑मानस्य॒ यज॑मानस्या व॒दाय॑ जुहुयात् । \newline
16. अ॒व॒दाय॑ जुहुयाज् जुहुया दव॒दाया॑ व॒दाय॑ जुहुया॒द् यद् यज् जु॑हुया दव॒दाया॑ व॒दाय॑ जुहुया॒द् यत् । \newline
17. अ॒व॒दायेत्य॑व - दाय॑ । \newline
18. जु॒हु॒या॒द् यद् यज् जु॑हुयाज् जुहुया॒द् यन् न न यज् जु॑हुयाज् जुहुया॒द् यन् न । \newline
19. यन् न न यद् यन् न जु॑हु॒याज् जु॑हु॒यान् न यद् यन् न जु॑हु॒यात् । \newline
20. न जु॑हु॒याज् जु॑हु॒यान् न न जु॑हु॒याद् य॑ज्ञ्प॒रुर् य॑ज्ञ्प॒रुर् जु॑हु॒यान् न न जु॑हु॒याद् य॑ज्ञ्प॒रुः । \newline
21. जु॒हु॒याद् य॑ज्ञ्प॒रुर् य॑ज्ञ्प॒रुर् जु॑हु॒याज् जु॑हु॒याद् य॑ज्ञ्प॒रु र॒न्त र॒न्तर् य॑ज्ञ्प॒रुर् जु॑हु॒याज् जु॑हु॒याद् य॑ज्ञ्प॒रु र॒न्तः । \newline
22. य॒ज्ञ्॒प॒रु र॒न्त र॒न्तर् य॑ज्ञ्प॒रुर् य॑ज्ञ्प॒रु र॒न्त रि॑या दिया द॒न्तर् य॑ज्ञ्प॒रुर् य॑ज्ञ्प॒रु र॒न्त रि॑यात् । \newline
23. य॒ज्ञ्॒प॒रुरिति॑ यज्ञ् - प॒रुः । \newline
24. अ॒न्त रि॑यादिया द॒न्त र॒न्त रि॑या॒द् ये य इ॑या द॒न्त र॒न्त रि॑या॒द् ये । \newline
25. इ॒या॒द् ये य इ॑या दिया॒द् ये दे॒वा दे॒वा य इ॑या दिया॒द् ये दे॒वाः । \newline
26. ये दे॒वा दे॒वा ये ये दे॒वा मनो॑जाता॒ मनो॑जाता दे॒वा ये ये दे॒वा मनो॑जाताः । \newline
27. दे॒वा मनो॑जाता॒ मनो॑जाता दे॒वा दे॒वा मनो॑जाता मनो॒युजो॑ मनो॒युजो॒ मनो॑जाता दे॒वा दे॒वा मनो॑जाता मनो॒युजः॑ । \newline
28. मनो॑जाता मनो॒युजो॑ मनो॒युजो॒ मनो॑जाता॒ मनो॑जाता मनो॒युज॒ इतीति॑ मनो॒युजो॒ मनो॑जाता॒ मनो॑जाता मनो॒युज॒ इति॑ । \newline
29. मनो॑जाता॒ इति॒ मनः॑ - जा॒ताः॒ । \newline
30. म॒नो॒युज॒ इतीति॑ मनो॒युजो॑ मनो॒युज॒ इत्या॑हा॒ हेति॑ मनो॒युजो॑ मनो॒युज॒ इत्या॑ह । \newline
31. म॒नो॒युज॒ इति॑ मनः - युजः॑ । \newline
32. इत्या॑हा॒हे तीत्या॑ह प्रा॒णाः प्रा॒णा आ॒हे तीत्या॑ह प्रा॒णाः । \newline
33. आ॒ह॒ प्रा॒णाः प्रा॒णा आ॑हाह प्रा॒णा वै वै प्रा॒णा आ॑हाह प्रा॒णा वै । \newline
34. प्रा॒णा वै वै प्रा॒णाः प्रा॒णा वै दे॒वा दे॒वा वै प्रा॒णाः प्रा॒णा वै दे॒वाः । \newline
35. प्रा॒णा इति॑ प्र - अ॒नाः । \newline
36. वै दे॒वा दे॒वा वै वै दे॒वा मनो॑जाता॒ मनो॑जाता दे॒वा वै वै दे॒वा मनो॑जाताः । \newline
37. दे॒वा मनो॑जाता॒ मनो॑जाता दे॒वा दे॒वा मनो॑जाता मनो॒युजो॑ मनो॒युजो॒ मनो॑जाता दे॒वा दे॒वा मनो॑जाता मनो॒युजः॑ । \newline
38. मनो॑जाता मनो॒युजो॑ मनो॒युजो॒ मनो॑जाता॒ मनो॑जाता मनो॒युज॒ स्तेषु॒ तेषु॑ मनो॒युजो॒ मनो॑जाता॒ मनो॑जाता मनो॒युज॒ स्तेषु॑ । \newline
39. मनो॑जाता॒ इति॒ मनः॑ - जा॒ताः॒ । \newline
40. म॒नो॒युज॒ स्तेषु॒ तेषु॑ मनो॒युजो॑ मनो॒युज॒ स्तेष्वे॒वैव तेषु॑ मनो॒युजो॑ मनो॒युज॒ स्तेष्वे॒व । \newline
41. म॒नो॒युज॒ इति॑ मनः - युजः॑ । \newline
42. तेष्वे॒वैव तेषु॒ तेष्वे॒व प॒रोक्ष॑म् प॒रोक्ष॑ मे॒व तेषु॒ तेष्वे॒व प॒रोक्ष᳚म् । \newline
43. ए॒व प॒रोक्ष॑म् प॒रोक्ष॑ मे॒वैव प॒रोक्ष॑म् जुहोति जुहोति प॒रोक्ष॑ मे॒वैव प॒रोक्ष॑म् जुहोति । \newline
44. प॒रोक्ष॑म् जुहोति जुहोति प॒रोक्ष॑म् प॒रोक्ष॑म् जुहोति॒ तत् तज् जु॑होति प॒रोक्ष॑म् प॒रोक्ष॑म् जुहोति॒ तत् । \newline
45. प॒रोक्ष॒मिति॑ परः - अक्ष᳚म् । \newline
46. जु॒हो॒ति॒ तत् तज् जु॑होति जुहोति॒ तन् न न तज् जु॑होति जुहोति॒ तन् न । \newline
47. तन् न न तत् तन् नेवे॑व॒ न तत् तन् नेव॑ । \newline
48. नेवे॑व॒ न नेव॑ हु॒तꣳ हु॒त मि॑व॒ न नेव॑ हु॒तम् । \newline
49. इ॒व॒ हु॒तꣳ हु॒त मि॑वे व हु॒तन् न न हु॒त मि॑वे व हु॒तन् न । \newline
50. हु॒तन् न न हु॒तꣳ हु॒तन् नेवे॑व॒ न हु॒तꣳ हु॒तन् नेव॑ । \newline
51. नेवे॑व॒ न नेवाहु॑त॒ महु॑त मिव॒ न नेवाहु॑तम् । \newline
52. इ॒वाहु॑त॒ महु॑त मिवे॒ वाहु॑तꣳ स्व॒पन्तꣳ॑ स्व॒पन्त॒ महु॑त मिवे॒ वाहु॑तꣳ स्व॒पन्त᳚म् । \newline
53. अहु॑तꣳ स्व॒पन्तꣳ॑ स्व॒पन्त॒ महु॑त॒ महु॑तꣳ स्व॒पन्तं॒ ॅवै वै स्व॒पन्त॒ महु॑त॒ महु॑तꣳ स्व॒पन्तं॒ ॅवै । \newline
54. स्व॒पन्तं॒ ॅवै वै स्व॒पन्तꣳ॑ स्व॒पन्तं॒ ॅवै दी᳚क्षि॒तम् दी᳚क्षि॒तं ॅवै स्व॒पन्तꣳ॑ स्व॒पन्तं॒ ॅवै दी᳚क्षि॒तम् । \newline
55. वै दी᳚क्षि॒तम् दी᳚क्षि॒तं ॅवै वै दी᳚क्षि॒तꣳ रक्षाꣳ॑सि॒ रक्षाꣳ॑सि दीक्षि॒तं ॅवै वै दी᳚क्षि॒तꣳ रक्षाꣳ॑सि । \newline
56. दी॒क्षि॒तꣳ रक्षाꣳ॑सि॒ रक्षाꣳ॑सि दीक्षि॒तम् दी᳚क्षि॒तꣳ रक्षाꣳ॑सि जिघाꣳसन्ति जिघाꣳसन्ति॒ रक्षाꣳ॑सि दीक्षि॒तम् दी᳚क्षि॒तꣳ रक्षाꣳ॑सि जिघाꣳसन्ति । \newline
57. रक्षाꣳ॑सि जिघाꣳसन्ति जिघाꣳसन्ति॒ रक्षाꣳ॑सि॒ रक्षाꣳ॑सि जिघाꣳसन् त्य॒ग्नि र॒ग्निर् जि॑घाꣳसन्ति॒ रक्षाꣳ॑सि॒ रक्षाꣳ॑सि जिघाꣳसन् त्य॒ग्निः । \newline
58. जि॒घाꣳ॒॒ स॒न्त्य॒ग्नि र॒ग्निर् जि॑घाꣳसन्ति जिघाꣳसन् त्य॒ग्निः खलु॒ खल्व॒ग्निर् जि॑घाꣳसन्ति जिघाꣳसन् त्य॒ग्निः खलु॑ । \newline
59. अ॒ग्निः खलु॒ खल्व॒ग्नि र॒ग्निः खलु॒ वै वै खल्व॒ग्नि र॒ग्निः खलु॒ वै । \newline
\pagebreak
\markright{ TS 6.1.4.6  \hfill https://www.vedavms.in \hfill}

\section{ TS 6.1.4.6 }

\textbf{TS 6.1.4.6 } \newline
\textbf{Samhita Paata} \newline

खलु॒ वै र॑क्षो॒हाऽग्ने॒ त्वꣳ सु जा॑गृहि व॒यꣳ सु म॑न्दिषीम॒हीत्या॑हा॒ग्नि-मे॒वाधि॒पां कृ॒त्वा स्व॑पिति॒ रक्ष॑सा॒मप॑हत्या अव्र॒त्यमि॑व॒ वा ए॒ष क॑रोति॒ यो दी᳚क्षि॒तः स्वपि॑ति॒ त्वम॑ग्ने व्रत॒पा अ॒सीत्या॑हा॒ग्निर्वै दे॒वानां᳚ ॅव्र॒तप॑तिः॒ स ए॒वैनं॑ ॅव्र॒तमा ल॑भंयति दे॒व आ मर्त्ये॒ष्वेत्या॑ह दे॒वो - [  ] \newline

\textbf{Pada Paata} \newline

खलु॑ । वै । र॒क्षो॒हेति॑ रक्षः - हा । अग्ने᳚ । त्वम् । स्विति॑ । जा॒गृ॒हि॒ । व॒यम् । स्विति॑ । म॒न्दि॒षी॒म॒हि॒ । इति॑ । आ॒ह॒ । अ॒ग्निम् । ए॒व । अ॒धि॒पामित्य॑धि - पाम् । कृ॒त्वा । स्व॒पि॒ति॒ । रक्ष॑साम् । अप॑हत्या॒ इत्यप॑ - ह॒त्यै॒ । अ॒व्र॒त्यम् । इ॒व॒ । वै । ए॒षः । क॒रो॒ति॒ । यः । दी॒क्षि॒तः । स्वपि॑ति । त्वम् । अ॒ग्ने॒ । व्र॒त॒पा इति॑ व्रत-पाः । अ॒सि॒ । इति॑ । आ॒ह॒ । अ॒ग्निः । वै । दे॒वाना᳚म् । व्र॒तप॑ति॒रिति॑ व्र॒त - प॒तिः॒ । सः । ए॒व । ए॒न॒म् । व्र॒तम् । एति॑ । ल॒म्भ॒य॒ति॒ । दे॒वः । एति॑ । मर्त्ये॑षु । एति॑ । इति॑ । आ॒ह॒ । दे॒वः ।  \newline


\textbf{Krama Paata} \newline

खलु॒ वै । वै र॑क्षो॒हा । र॒क्षो॒हाऽग्ने᳚ । र॒क्षो॒हेति॑ रक्षः - हा । अग्ने॒ त्वम् । त्वꣳ सु । सु जा॑गृहि । जा॒गृ॒हि॒ व॒यम् । व॒यꣳ सु । सु म॑न्दिषीमहि । म॒न्दि॒षी॒म॒हीति॑ । इत्या॑ह । आ॒हा॒ग्निम् । अ॒ग्निमे॒व । ए॒वाधि॒पाम् । अ॒धि॒पाम् कृ॒त्वा । अ॒धि॒पामित्य॑धि - पाम् । कृ॒त्वा स्व॑पिति । स्व॒पि॒ति॒ रक्ष॑साम् । रक्ष॑सा॒मप॑हत्यै । अप॑हत्या अव्र॒त्यम् । अप॑हत्या॒ इत्यप॑ - ह॒त्यै॒ । अ॒व्र॒त्यमि॑व । इ॒व॒ वै । वा ए॒षः । ए॒ष क॑रोति । क॒रो॒ति॒ यः । यो दी᳚क्षि॒तः । दी॒क्षि॒तः स्वपि॑ति । स्वपि॑ति॒ त्वम् । त्वम॑ग्ने । अ॒ग्ने॒ व्र॒त॒पाः । व्र॒त॒पा अ॑सि । व्र॒त॒पा इति॑ व्रत - पाः । अ॒सीति॑ । इत्या॑ह । आ॒हा॒ग्निः । अ॒ग्निर् वै । वै दे॒वाना᳚म् । दे॒वाना᳚म् ॅव्र॒तप॑तिः । व्र॒तप॑तिः॒ सः । व्र॒तप॑ति॒रिति॑ व्र॒त - प॒तिः॒ । स ए॒व । ए॒वैन᳚म् । ए॒न॒म् ॅव्र॒तम् । व्र॒तमा । आ ल॑म्भयति । ल॒म्भ॒य॒ति॒ दे॒वः । दे॒व आ । आ मर्त्ये॑षु । मर्त्ये॒ष्वा । एति॑ । इत्या॑ह । आ॒ह॒ दे॒वः । दे॒वो हि \newline

\textbf{Jatai Paata} \newline

1. खलु॒ वै वै खलु॒ खलु॒ वै । \newline
2. वै र॑क्षो॒हा र॑क्षो॒हा वै वै र॑क्षो॒हा । \newline
3. र॒क्षो॒हा ऽग्ने॒ अग्ने॑ रक्षो॒हा र॑क्षो॒हा ऽग्ने᳚ । \newline
4. र॒क्षो॒हेति॑ रक्षः - हा । \newline
5. अग्ने॒ त्वम् त्व मग्ने ऽग्ने॒ त्वम् । \newline
6. त्वꣳ सु सु त्वम् त्वꣳ सु । \newline
7. सु जा॑गृहि जागृहि॒ सु सु जा॑गृहि । \newline
8. जा॒गृ॒हि॒ व॒यं ॅव॒यम् जा॑गृहि जागृहि व॒यम् । \newline
9. व॒यꣳ सु सु व॒यं ॅव॒यꣳ सु । \newline
10. सु म॑न्दिषीमहि मन्दिषीमहि॒ सु सु म॑न्दिषीमहि । \newline
11. म॒न्दि॒षी॒म॒ हीतीति॑ मन्दिषीमहि मन्दिषीम॒ हीति॑ । \newline
12. इत्या॑हा॒हे तीत्या॑ह । \newline
13. आ॒हा॒ग्नि म॒ग्नि मा॑हा हा॒ग्निम् । \newline
14. अ॒ग्नि मे॒वै वाग्नि म॒ग्नि मे॒व । \newline
15. ए॒वाधि॒पा म॑धि॒पा मे॒वै वाधि॒पाम् । \newline
16. अ॒धि॒पाम् कृ॒त्वा कृ॒त्वा ऽधि॒पा म॑धि॒पाम् कृ॒त्वा । \newline
17. अ॒धि॒पामित्य॑धि - पाम् । \newline
18. कृ॒त्वा स्व॑पिति स्वपिति कृ॒त्वा कृ॒त्वा स्व॑पिति । \newline
19. स्व॒पि॒ति॒ रक्ष॑साꣳ॒॒ रक्ष॑साꣳ स्वपिति स्वपिति॒ रक्ष॑साम् । \newline
20. रक्ष॑सा॒ मप॑हत्या॒ अप॑हत्यै॒ रक्ष॑साꣳ॒॒ रक्ष॑सा॒ मप॑हत्यै । \newline
21. अप॑हत्या अव्र॒त्य म॑व्र॒त्य मप॑हत्या॒ अप॑हत्या अव्र॒त्यम् । \newline
22. अप॑हत्या॒ इत्यप॑ - ह॒त्यै॒ । \newline
23. अ॒व्र॒त्य मि॑वे वाव्र॒त्य म॑व्र॒त्य मि॑व । \newline
24. इ॒व॒ वै वा इ॑वेव॒ वै । \newline
25. वा ए॒ष ए॒ष वै वा ए॒षः । \newline
26. ए॒ष क॑रोति करो त्ये॒ष ए॒ष क॑रोति । \newline
27. क॒रो॒ति॒ यो यः क॑रोति करोति॒ यः । \newline
28. यो दी᳚क्षि॒तो दी᳚क्षि॒तो यो यो दी᳚क्षि॒तः । \newline
29. दी॒क्षि॒तः स्वपि॑ति॒ स्वपि॑ति दीक्षि॒तो दी᳚क्षि॒तः स्वपि॑ति । \newline
30. स्वपि॑ति॒ त्वम् त्वꣳ स्वपि॑ति॒ स्वपि॑ति॒ त्वम् । \newline
31. त्व म॑ग्ने अग्ने॒ त्वम् त्व म॑ग्ने । \newline
32. अ॒ग्ने॒ व्र॒त॒पा व्र॑त॒पा अ॑ग्ने अग्ने व्रत॒पाः । \newline
33. व्र॒त॒पा अ॑स्यसि व्रत॒पा व्र॑त॒पा अ॑सि । \newline
34. व्र॒त॒पा इति॑ व्रत - पाः । \newline
35. अ॒सीती त्य॑स्य॒ सीति॑ । \newline
36. इत्या॑हा॒हे तीत्या॑ह । \newline
37. आ॒हा॒ग्नि र॒ग्नि रा॑हा हा॒ग्निः । \newline
38. अ॒ग्निर् वै वा अ॒ग्नि र॒ग्निर् वै । \newline
39. वै दे॒वाना᳚म् दे॒वानां॒ ॅवै वै दे॒वाना᳚म् । \newline
40. दे॒वानां᳚ ॅव्र॒तप॑तिर् व्र॒तप॑तिर् दे॒वाना᳚म् दे॒वानां᳚ ॅव्र॒तप॑तिः । \newline
41. व्र॒तप॑तिः॒ स स व्र॒तप॑तिर् व्र॒तप॑तिः॒ सः । \newline
42. व्र॒तप॑ति॒रिति॑ व्र॒त - प॒तिः॒ । \newline
43. स ए॒वैव स स ए॒व । \newline
44. ए॒वैन॑ मेन मे॒वै वैन᳚म् । \newline
45. ए॒नं॒ ॅव्र॒तं ॅव्र॒त मे॑न मेनं ॅव्र॒तम् । \newline
46. व्र॒त मा व्र॒तं ॅव्र॒त मा । \newline
47. आ ल॑म्भयति लम्भय॒ त्या ल॑म्भयति । \newline
48. ल॒म्भ॒य॒ति॒ दे॒वो दे॒वो ल॑म्भयति लम्भयति दे॒वः । \newline
49. दे॒व आ दे॒वो दे॒व आ । \newline
50. आ मर्त्ये॑षु॒ मर्त्ये॒ष्वा मर्त्ये॑षु । \newline
51. मर्त्ये॒ष्वा मर्त्ये॑षु॒ मर्त्ये॒ष्वा । \newline
52. एती त्येति॑ । \newline
53. इत्या॑हा॒हे तीत्या॑ह । \newline
54. आ॒ह॒ दे॒वो दे॒व आ॑हाह दे॒वः । \newline
55. दे॒वो हि हि दे॒वो दे॒वो हि । \newline

\textbf{Ghana Paata } \newline

1. खलु॒ वै वै खलु॒ खलु॒ वै र॑क्षो॒हा र॑क्षो॒हा वै खलु॒ खलु॒ वै र॑क्षो॒हा । \newline
2. वै र॑क्षो॒हा र॑क्षो॒हा वै वै र॑क्षो॒हा ऽग्ने ग्ने॑ रक्षो॒हा वै वै र॑क्षो॒हा ऽग्ने᳚ । \newline
3. र॒क्षो॒हा ऽग्ने ऽग्ने॑ रक्षो॒हा र॑क्षो॒हा ऽग्ने॒ त्वम् त्व मग्ने॑ रक्षो॒हा र॑क्षो॒हा ऽग्ने॒ त्वम् । \newline
4. र॒क्षो॒हेति॑ रक्षः - हा । \newline
5. अग्ने॒ त्वम् त्व मग्ने ऽग्ने॒ त्वꣳ सु सु त्व मग्ने ऽग्ने॒ त्वꣳ सु । \newline
6. त्वꣳ सु सु त्वम् त्वꣳ सु जा॑गृहि जागृहि॒ सु त्वम् त्वꣳ सु जा॑गृहि । \newline
7. सु जा॑गृहि जागृहि॒ सु सु जा॑गृहि व॒यं ॅव॒यम् जा॑गृहि॒ सु सु जा॑गृहि व॒यम् । \newline
8. जा॒गृ॒हि॒ व॒यं ॅव॒यम् जा॑गृहि जागृहि व॒यꣳ सु सु व॒यम् जा॑गृहि जागृहि व॒यꣳ सु । \newline
9. व॒यꣳ सु सु व॒यं ॅव॒यꣳ सु म॑न्दिषीमहि मन्दिषीमहि॒ सु व॒यं ॅव॒यꣳ सु म॑न्दिषीमहि । \newline
10. सु म॑न्दिषीमहि मन्दिषीमहि॒ सु सु म॑न्दिषीम॒ही तीति॑ मन्दिषीमहि॒ सु सु म॑न्दिषीम॒हीति॑ । \newline
11. म॒न्दि॒षी॒म॒ही तीति॑ मन्दिषीमहि मन्दिषीम॒ही त्या॑हा॒हेति॑ मन्दिषीमहि मन्दिषीम॒ही त्या॑ह । \newline
12. इत्या॑हा॒हे तीत्या॑ हा॒ग्नि म॒ग्नि मा॒हे तीत्या॑ हा॒ग्निम् । \newline
13. आ॒हा॒ग्नि म॒ग्नि मा॑हाहा॒ग्नि मे॒वैवाग्नि मा॑हाहा॒ग्नि मे॒व । \newline
14. अ॒ग्नि मे॒वै वाग्नि म॒ग्नि मे॒वाधि॒पा म॑धि॒पा मे॒वाग्नि म॒ग्नि मे॒वाधि॒पाम् । \newline
15. ए॒वाधि॒पा म॑धि॒पा मे॒वै वाधि॒पाम् कृ॒त्वा कृ॒त्वा ऽधि॒पा मे॒वै वाधि॒पाम् कृ॒त्वा । \newline
16. अ॒धि॒पाम् कृ॒त्वा कृ॒त्वा ऽधि॒पा म॑धि॒पाम् कृ॒त्वा स्व॑पिति स्वपिति कृ॒त्वा ऽधि॒पा म॑धि॒पाम् कृ॒त्वा स्व॑पिति । \newline
17. अ॒धि॒पामित्य॑धि - पाम् । \newline
18. कृ॒त्वा स्व॑पिति स्वपिति कृ॒त्वा कृ॒त्वा स्व॑पिति॒ रक्ष॑साꣳ॒॒ रक्ष॑साꣳ स्वपिति कृ॒त्वा कृ॒त्वा स्व॑पिति॒ रक्ष॑साम् । \newline
19. स्व॒पि॒ति॒ रक्ष॑साꣳ॒॒ रक्ष॑साꣳ स्वपिति स्वपिति॒ रक्ष॑सा॒ मप॑हत्या॒ अप॑हत्यै॒ रक्ष॑साꣳ स्वपिति स्वपिति॒ रक्ष॑सा॒ मप॑हत्यै । \newline
20. रक्ष॑सा॒ मप॑हत्या॒ अप॑हत्यै॒ रक्ष॑साꣳ॒॒ रक्ष॑सा॒ मप॑हत्या अव्र॒त्य म॑व्र॒त्य मप॑हत्यै॒ रक्ष॑साꣳ॒॒ रक्ष॑सा॒ मप॑हत्या अव्र॒त्यम् । \newline
21. अप॑हत्या अव्र॒त्य म॑व्र॒त्य मप॑हत्या॒ अप॑हत्या अव्र॒त्य मि॑वे वाव्र॒त्य मप॑हत्या॒ अप॑हत्या अव्र॒त्य मि॑व । \newline
22. अप॑हत्या॒ इत्यप॑ - ह॒त्यै॒ । \newline
23. अ॒व्र॒त्य मि॑वे वाव्र॒त्य म॑व्र॒त्य मि॑व॒ वै वा इ॑वाव्र॒त्य म॑व्र॒त्य मि॑व॒ वै । \newline
24. इ॒व॒ वै वा इ॑वे व॒ वा ए॒ष ए॒ष वा इ॑वे व॒ वा ए॒षः । \newline
25. वा ए॒ष ए॒ष वै वा ए॒ष क॑रोति करो त्ये॒ष वै वा ए॒ष क॑रोति । \newline
26. ए॒ष क॑रोति करो त्ये॒ष ए॒ष क॑रोति॒ यो यः क॑रो त्ये॒ष ए॒ष क॑रोति॒ यः । \newline
27. क॒रो॒ति॒ यो यः क॑रोति करोति॒ यो दी᳚क्षि॒तो दी᳚क्षि॒तो यः क॑रोति करोति॒ यो दी᳚क्षि॒तः । \newline
28. यो दी᳚क्षि॒तो दी᳚क्षि॒तो यो यो दी᳚क्षि॒तः स्वपि॑ति॒ स्वपि॑ति दीक्षि॒तो यो यो दी᳚क्षि॒तः स्वपि॑ति । \newline
29. दी॒क्षि॒तः स्वपि॑ति॒ स्वपि॑ति दीक्षि॒तो दी᳚क्षि॒तः स्वपि॑ति॒ त्वम् त्वꣳ स्वपि॑ति दीक्षि॒तो दी᳚क्षि॒तः स्वपि॑ति॒ त्वम् । \newline
30. स्वपि॑ति॒ त्वम् त्वꣳ स्वपि॑ति॒ स्वपि॑ति॒ त्व म॑ग्ने अग्ने॒ त्वꣳ स्वपि॑ति॒ स्वपि॑ति॒ त्व म॑ग्ने । \newline
31. त्व म॑ग्ने अग्ने॒ त्वम् त्व म॑ग्ने व्रत॒पा व्र॑त॒पा अ॑ग्ने॒ त्वम् त्व म॑ग्ने व्रत॒पाः । \newline
32. अ॒ग्ने॒ व्र॒त॒पा व्र॑त॒पा अ॑ग्ने अग्ने व्रत॒पा अ॑स्यसि व्रत॒पा अ॑ग्ने अग्ने व्रत॒पा अ॑सि । \newline
33. व्र॒त॒पा अ॑स्यसि व्रत॒पा व्र॑त॒पा अ॒सी तीत्य॑सि व्रत॒पा व्र॑त॒पा अ॒सीति॑ । \newline
34. व्र॒त॒पा इति॑ व्रत - पाः । \newline
35. अ॒सी तीत्य॑स्य॒ सीत्या॑ हा॒हे त्य॑स्य॒सी त्या॑ह । \newline
36. इत्या॑हा॒हे तीत्या॑हा॒ ग्नि र॒ग्नि रा॒हे तीत्या॑ हा॒ग्निः । \newline
37. आ॒हा॒ग्नि र॒ग्नि रा॑हा हा॒ग्निर् वै वा अ॒ग्नि रा॑हा हा॒ग्निर् वै । \newline
38. अ॒ग्निर् वै वा अ॒ग्नि र॒ग्निर् वै दे॒वाना᳚म् दे॒वानां॒ ॅवा अ॒ग्नि र॒ग्निर् वै दे॒वाना᳚म् । \newline
39. वै दे॒वाना᳚म् दे॒वानां॒ ॅवै वै दे॒वानां᳚ ॅव्र॒तप॑तिर् व्र॒तप॑तिर् दे॒वानां॒ ॅवै वै दे॒वानां᳚ ॅव्र॒तप॑तिः । \newline
40. दे॒वानां᳚ ॅव्र॒तप॑तिर् व्र॒तप॑तिर् दे॒वाना᳚म् दे॒वानां᳚ ॅव्र॒तप॑तिः॒ स स व्र॒तप॑तिर् दे॒वाना᳚म् दे॒वानां᳚ ॅव्र॒तप॑तिः॒ सः । \newline
41. व्र॒तप॑तिः॒ स स व्र॒तप॑तिर् व्र॒तप॑तिः॒ स ए॒वैव स व्र॒तप॑तिर् व्र॒तप॑तिः॒ स ए॒व । \newline
42. व्र॒तप॑ति॒रिति॑ व्र॒त - प॒तिः॒ । \newline
43. स ए॒वैव स स ए॒वैन॑ मेन मे॒व स स ए॒वैन᳚म् । \newline
44. ए॒वैन॑ मेन मे॒वै वैनं॑ ॅव्र॒तं ॅव्र॒त मे॑न मे॒वै वैनं॑ ॅव्र॒तम् । \newline
45. ए॒नं॒ ॅव्र॒तं ॅव्र॒त मे॑न मेनं ॅव्र॒त मा व्र॒त मे॑न मेनं ॅव्र॒त मा । \newline
46. व्र॒त मा व्र॒तं ॅव्र॒त मा ल॑म्भयति लम्भय॒त्या व्र॒तं ॅव्र॒त मा ल॑म्भयति । \newline
47. आ ल॑म्भयति लम्भय॒ त्या ल॑म्भयति दे॒वो दे॒वो ल॑म्भय॒ त्या ल॑म्भयति दे॒वः । \newline
48. ल॒म्भ॒य॒ति॒ दे॒वो दे॒वो ल॑म्भयति लम्भयति दे॒व आ दे॒वो ल॑म्भयति लम्भयति दे॒व आ । \newline
49. दे॒व आ दे॒वो दे॒व आ मर्त्ये॑षु॒ मर्त्ये॒ष्वा दे॒वो दे॒व आ मर्त्ये॑षु । \newline
50. आ मर्त्ये॑षु॒ मर्त्ये॒ष्वा मर्त्ये॒ष्वा मर्त्ये॒ष्वा मर्त्ये॒ष्वा । \newline
51. मर्त्ये॒ ष्वा मर्त्ये॑षु॒ मर्त्ये॒ ष्वेतीत्या मर्त्ये॑षु॒ मर्त्ये॒ ष्वेति॑ । \newline
52. एतीत्ये त्या॑हा॒हे त्येत्या॑ह । \newline
53. इत्या॑हा॒हे तीत्या॑ह दे॒वो दे॒व आ॒हे तीत्या॑ह दे॒वः । \newline
54. आ॒ह॒ दे॒वो दे॒व आ॑हाह दे॒वो हि हि दे॒व आ॑हाह दे॒वो हि । \newline
55. दे॒वो हि हि दे॒वो दे॒वो ह्ये॑ष ए॒ष हि दे॒वो दे॒वो ह्ये॑षः । \newline
\pagebreak
\markright{ TS 6.1.4.7  \hfill https://www.vedavms.in \hfill}

\section{ TS 6.1.4.7 }

\textbf{TS 6.1.4.7 } \newline
\textbf{Samhita Paata} \newline

ह्ये॑ष सन् मर्त्ये॑षु॒ त्वं ॅय॒ज्ञेष्वीड्य॒ इत्या॑है॒तꣳ हि य॒ज्ञेष्वीड॒तेऽप॒ वै दी᳚क्षि॒ताथ् सु॑षु॒पुष॑ इन्द्रि॒यं दे॒वताः᳚ क्रामन्ति॒ विश्वे॑ दे॒वा अ॒भि मामाऽव॑वृत्र॒-न्नित्या॑-हेन्द्रि॒येणै॒वैनं॑ दे॒वता॑भिः॒ सं न॑यति॒ यदे॒तद्-यजु॒र्न ब्रू॒याद्-याव॑त ए॒व प॒शून॒भि दीक्षे॑त॒ ताव॑न्तोऽस्य प॒शवः॑ स्यू॒ रास्वेय॑थ् - [  ] \newline

\textbf{Pada Paata} \newline

हि । ए॒षः । सन्न् । मर्त्ये॑षु । त्वम् । य॒ज्ञेषु॑ । ईड्यः॑ । इति॑ । आ॒ह॒ । ए॒तम् । हि । य॒ज्ञेषु॑ । ईड॑ते । अपेति॑ । वै । दी॒क्षि॒तात् । सु॒षु॒पुषः॑ । इ॒न्द्रि॒यम् । दे॒वताः᳚ । क्रा॒म॒न्ति॒ । विश्वे᳚ । दे॒वाः । अ॒भीति॑ । माम् । एति॑ । अ॒व॒वृ॒त्र॒न्न् । इति॑ । आ॒ह॒ । इ॒न्द्रि॒येण॑ । ए॒व । ए॒न॒म् । दे॒वता॑भिः । समिति॑ । न॒य॒ति॒ । यत् । ए॒तत् । यजुः॑ । न । ब्रू॒यात् । याव॑तः । ए॒व । प॒शून् । अ॒भीति॑ । दीक्षे॑त । ताव॑न्तः । अ॒स्य॒ । प॒शवः॑ । स्युः॒ । रास्व॑ । इय॑त् ।  \newline


\textbf{Krama Paata} \newline

ह्ये॑षः । ए॒ष सन्न् । सन् मर्त्ये॑षु । मर्त्ये॑षु॒ त्वम् । त्वम् ॅय॒ज्ञेषु॑ । य॒ज्ञेष्वीड्‍यः॑ । ईड्‍य॒ इति॑ । इत्या॑ह । आ॒है॒तम् । ए॒तꣳ हि । हि य॒ज्ञेषु॑ । य॒ज्ञेष्वीड॑ते । ईड॒तेऽप॑ । अप॒ वै । वै दी᳚क्षि॒तात् । दी॒क्षि॒ताथ् सु॑षु॒पुषः॑ । सु॒षु॒पुष॑ इन्द्रि॒यम् । इ॒न्द्रि॒यम् दे॒वताः᳚ । दे॒वताः᳚ क्रामन्ति । क्रा॒म॒न्ति॒ विश्वे᳚ । विश्वे॑ दे॒वाः । दे॒वा अ॒भि । अ॒भि माम् । मामा । आऽव॑वृत्रन्न् । अ॒व॒वृ॒त्र॒न्निति॑ । इत्या॑ह । आ॒हे॒न्द्रि॒येण॑ । इ॒न्द्रि॒येणै॒व । ए॒वैन᳚म् । ए॒न॒म् दे॒वता॑भिः । दे॒वता॑भिः॒ सम् । सम् न॑यति । न॒य॒ति॒ यत् । यदे॒तत् । ए॒तद् यजुः॑ । यजु॒र् न । न ब्रू॒यात् । ब्रू॒याद् याव॑तः । याव॑त ए॒व । ए॒व प॒शून् । प॒शून॒भि । अ॒भि दीक्षे॑त । दीक्षे॑त॒ ताव॑न्तः । ताव॑न्तोऽस्य । अ॒स्य॒ प॒शवः॑ । प॒शवः॑ स्युः । स्यू॒रास्व॑ । रास्वेय॑त् । इय॑थ् सोम \newline

\textbf{Jatai Paata} \newline

1. ह्ये॑ष ए॒ष हि ह्ये॑षः । \newline
2. ए॒ष सन् थ्सन् ने॒ष ए॒ष सन्न् । \newline
3. सन् मर्त्ये॑षु॒ मर्त्ये॑षु॒ सन् थ्सन् मर्त्ये॑षु । \newline
4. मर्त्ये॑षु॒ त्वम् त्वम् मर्त्ये॑षु॒ मर्त्ये॑षु॒ त्वम् । \newline
5. त्वं ॅय॒ज्ञेषु॑ य॒ज्ञेषु॒ त्वम् त्वं ॅय॒ज्ञेषु॑ । \newline
6. य॒ज्ञे ष्वीड्य॒ ईड्यो॑ य॒ज्ञेषु॑ य॒ज्ञे ष्वीड्यः॑ । \newline
7. ईड्य॒ इतीतीड्य॒ ईड्य॒ इति॑ । \newline
8. इत्या॑हा॒हे तीत्या॑ह । \newline
9. आ॒है॒त मे॒त मा॑हा है॒तम् । \newline
10. ए॒तꣳ हि ह्ये॑त मे॒तꣳ हि । \newline
11. हि य॒ज्ञेषु॑ य॒ज्ञेषु॒ हि हि य॒ज्ञेषु॑ । \newline
12. य॒ज्ञेष्वी ड॑त॒ ईड॑ते य॒ज्ञेषु॑ य॒ज्ञेष्वी ड॑ते । \newline
13. ईड॒ते ऽपापे ड॑त॒ ईड॒ते ऽप॑ । \newline
14. अप॒ वै वा अपाप॒ वै । \newline
15. वै दी᳚क्षि॒ताद् दी᳚क्षि॒ताद् वै वै दी᳚क्षि॒तात् । \newline
16. दी॒क्षि॒ताथ् सु॑षु॒पुषः॑ सुषु॒पुषो॑ दीक्षि॒ताद् दी᳚क्षि॒ताथ् सु॑षु॒पुषः॑ । \newline
17. सु॒षु॒पुष॑ इन्द्रि॒य मि॑न्द्रि॒यꣳ सु॑षु॒पुषः॑ सुषु॒पुष॑ इन्द्रि॒यम् । \newline
18. इ॒न्द्रि॒यम् दे॒वता॑ दे॒वता॑ इन्द्रि॒य मि॑न्द्रि॒यम् दे॒वताः᳚ । \newline
19. दे॒वताः᳚ क्रामन्ति क्रामन्ति दे॒वता॑ दे॒वताः᳚ क्रामन्ति । \newline
20. क्रा॒म॒न्ति॒ विश्वे॒ विश्वे᳚ क्रामन्ति क्रामन्ति॒ विश्वे᳚ । \newline
21. विश्वे॑ दे॒वा दे॒वा विश्वे॒ विश्वे॑ दे॒वाः । \newline
22. दे॒वा अ॒भ्य॑भि दे॒वा दे॒वा अ॒भि । \newline
23. अ॒भि माम् मा म॒भ्य॑भि माम् । \newline
24. मा मा माम् मा मा । \newline
25. आ ऽव॑वृत्रन् नववृत्र॒न्ना ऽव॑वृत्रन्न् । \newline
26. अ॒व॒वृ॒त्र॒न् निती त्य॑ववृत्रन् नववृत्र॒न्निति॑ । \newline
27. इत्या॑हा॒हे तीत्या॑ह । \newline
28. आ॒हे॒न्द्रि॒ये णे᳚न्द्रि॒येणा॑ हाहेन्द्रि॒येण॑ । \newline
29. इ॒न्द्रि॒ये णै॒वै वेन्द्रि॒ये णे᳚न्द्रि॒ये णै॒व । \newline
30. ए॒वैन॑ मेन मे॒वै वैन᳚म् । \newline
31. ए॒न॒म् दे॒वता॑भिर् दे॒वता॑भि रेन मेनम् दे॒वता॑भिः । \newline
32. दे॒वता॑भिः॒ सꣳ सम् दे॒वता॑भिर् दे॒वता॑भिः॒ सम् । \newline
33. सन् न॑यति नयति॒ सꣳ सन् न॑यति । \newline
34. न॒य॒ति॒ यद् यन् न॑यति नयति॒ यत् । \newline
35. यदे॒त दे॒तद् यद् यदे॒तत् । \newline
36. ए॒तद् यजु॒र् यजु॑ रे॒त दे॒तद् यजुः॑ । \newline
37. यजु॒र् न न यजु॒र् यजु॒र् न । \newline
38. न ब्रू॒याद् ब्रू॒यान् न न ब्रू॒यात् । \newline
39. ब्रू॒याद् याव॑तो॒ याव॑तो ब्रू॒याद् ब्रू॒याद् याव॑तः । \newline
40. याव॑त ए॒वैव याव॑तो॒ याव॑त ए॒व । \newline
41. ए॒व प॒शून् प॒शूने॒ वैव प॒शून् । \newline
42. प॒शून॒ भ्य॑भि प॒शून् प॒शून॒भि । \newline
43. अ॒भि दीक्षे॑त॒ दीक्षे॑ता॒ भ्य॑भि दीक्षे॑त । \newline
44. दीक्षे॑त॒ ताव॑न्त॒ स्ताव॑न्तो॒ दीक्षे॑त॒ दीक्षे॑त॒ ताव॑न्तः । \newline
45. ताव॑न्तो ऽस्यास्य॒ ताव॑न्त॒ स्ताव॑न्तो ऽस्य । \newline
46. अ॒स्य॒ प॒शवः॑ प॒शवो᳚ ऽस्यास्य प॒शवः॑ । \newline
47. प॒शवः॑ स्युः स्युः प॒शवः॑ प॒शवः॑ स्युः । \newline
48. स्यू॒ रास्व॒ रास्व॑ स्युः स्यू॒ रास्व॑ । \newline
49. रास्वे य॒दिय॒द् रास्व॒ रास्वे य॑त् । \newline
50. इय॑थ् सोम सो॒मे य॒दिय॑थ् सोम । \newline

\textbf{Ghana Paata } \newline

1. ह्ये॑ष ए॒ष हि ह्ये॑ष सन् थ्सन् ने॒ष हि ह्ये॑ष सन्न् । \newline
2. ए॒ष सन् थ्सन् ने॒ष ए॒ष सन् मर्त्ये॑षु॒ मर्त्ये॑षु॒ सन् ने॒ष ए॒ष सन् मर्त्ये॑षु । \newline
3. सन् मर्त्ये॑षु॒ मर्त्ये॑षु॒ सन् थ्सन् मर्त्ये॑षु॒ त्वम् त्वम् मर्त्ये॑षु॒ सन् थ्सन् मर्त्ये॑षु॒ त्वम् । \newline
4. मर्त्ये॑षु॒ त्वम् त्वम् मर्त्ये॑षु॒ मर्त्ये॑षु॒ त्वं ॅय॒ज्ञेषु॑ य॒ज्ञेषु॒ त्वम् मर्त्ये॑षु॒ मर्त्ये॑षु॒ त्वं ॅय॒ज्ञेषु॑ । \newline
5. त्वं ॅय॒ज्ञेषु॑ य॒ज्ञेषु॒ त्वम् त्वं ॅय॒ज्ञेष्वीड्य॒ ईड्यो॑ य॒ज्ञेषु॒ त्वम् त्वं ॅय॒ज्ञेष्वीड्यः॑ । \newline
6. य॒ज्ञेष्वीड्य॒ ईड्यो॑ य॒ज्ञेषु॑ य॒ज्ञेष्वीड्य॒ इतीतीड्यो॑ य॒ज्ञेषु॑ य॒ज्ञेष्वीड्य॒ इति॑ । \newline
7. ईड्य॒ इतीतीड्य॒ ईड्य॒ इत्या॑हा॒हे तीड्य॒ ईड्य॒ इत्या॑ह । \newline
8. इत्या॑हा॒हे तीत्या॑ है॒त मे॒त मा॒हे तीत्या॑ है॒तम् । \newline
9. आ॒है॒त मे॒त मा॑हा है॒तꣳ हि ह्ये॑त मा॑हा है॒तꣳ हि । \newline
10. ए॒तꣳ हि ह्ये॑त मे॒तꣳ हि य॒ज्ञेषु॑ य॒ज्ञेषु॒ ह्ये॑त मे॒तꣳ हि य॒ज्ञेषु॑ । \newline
11. हि य॒ज्ञेषु॑ य॒ज्ञेषु॒ हि हि य॒ज्ञे ष्वीड॑त॒ ईड॑ते य॒ज्ञेषु॒ हि हि य॒ज्ञे ष्वीड॑ते । \newline
12. य॒ज्ञे ष्वीड॑त॒ ईड॑ते य॒ज्ञेषु॑ य॒ज्ञे ष्वीड॒ते ऽपापेड॑ते य॒ज्ञेषु॑ य॒ज्ञे ष्वीड॒ते ऽप॑ । \newline
13. ईड॒ते ऽपापेड॑त॒ ईड॒ते ऽप॒ वै वा अपेड॑त॒ ईड॒ते ऽप॒ वै । \newline
14. अप॒ वै वा अपाप॒ वै दी᳚क्षि॒ताद् दी᳚क्षि॒ताद् वा अपाप॒ वै दी᳚क्षि॒तात् । \newline
15. वै दी᳚क्षि॒ताद् दी᳚क्षि॒ताद् वै वै दी᳚क्षि॒ताथ् सु॑षु॒पुषः॑ सुषु॒पुषो॑ दीक्षि॒ताद् वै वै दी᳚क्षि॒ताथ् सु॑षु॒पुषः॑ । \newline
16. दी॒क्षि॒ताथ् सु॑षु॒पुषः॑ सुषु॒पुषो॑ दीक्षि॒ताद् दी᳚क्षि॒ताथ् सु॑षु॒पुष॑ इन्द्रि॒य मि॑न्द्रि॒यꣳ सु॑षु॒पुषो॑ दीक्षि॒ताद् दी᳚क्षि॒ताथ् सु॑षु॒पुष॑ इन्द्रि॒यम् । \newline
17. सु॒षु॒पुष॑ इन्द्रि॒य मि॑न्द्रि॒यꣳ सु॑षु॒पुषः॑ सुषु॒पुष॑ इन्द्रि॒यम् दे॒वता॑ दे॒वता॑ इन्द्रि॒यꣳ सु॑षु॒पुषः॑ सुषु॒पुष॑ इन्द्रि॒यम् दे॒वताः᳚ । \newline
18. इ॒न्द्रि॒यम् दे॒वता॑ दे॒वता॑ इन्द्रि॒य मि॑न्द्रि॒यम् दे॒वताः᳚ क्रामन्ति क्रामन्ति दे॒वता॑ इन्द्रि॒य मि॑न्द्रि॒यम् दे॒वताः᳚ क्रामन्ति । \newline
19. दे॒वताः᳚ क्रामन्ति क्रामन्ति दे॒वता॑ दे॒वताः᳚ क्रामन्ति॒ विश्वे॒ विश्वे᳚ क्रामन्ति दे॒वता॑ दे॒वताः᳚ क्रामन्ति॒ विश्वे᳚ । \newline
20. क्रा॒म॒न्ति॒ विश्वे॒ विश्वे᳚ क्रामन्ति क्रामन्ति॒ विश्वे॑ दे॒वा दे॒वा विश्वे᳚ क्रामन्ति क्रामन्ति॒ विश्वे॑ दे॒वाः । \newline
21. विश्वे॑ दे॒वा दे॒वा विश्वे॒ विश्वे॑ दे॒वा अ॒भ्य॑भि दे॒वा विश्वे॒ विश्वे॑ दे॒वा अ॒भि । \newline
22. दे॒वा अ॒भ्य॑भि दे॒वा दे॒वा अ॒भि माम् मा म॒भि दे॒वा दे॒वा अ॒भि माम् । \newline
23. अ॒भि माम् मा म॒भ्य॑भि मा मा मा म॒भ्य॑भि मा मा । \newline
24. मा मा माम् मा मा ऽव॑वृत्रन् नववृत्र॒न्ना माम् मा मा ऽव॑वृत्रन्न् । \newline
25. आ ऽव॑वृत्रन् नववृत्र॒न् ना ऽव॑वृत्र॒न् निती त्य॑ववृत्र॒न् ना ऽव॑वृत्र॒न् निति॑ । \newline
26. अ॒व॒वृ॒त्र॒न् नितीत्य॑ ववृत्रन् नववृत्र॒न् नित्या॑हा॒हे त्य॑ववृत्रन् नववृत्र॒न् नित्या॑ह । \newline
27. इत्या॑हा॒हे तीत्या॑ हेन्द्रि॒ये णे᳚न्द्रि॒येणा॒हे तीत्या॑ हेन्द्रि॒येण॑ । \newline
28. आ॒हे॒न्द्रि॒ये णे᳚न्द्रि॒ये णा॑हा हेन्द्रि॒ये णै॒वैवेन्द्रि॒ये णा॑हा हेन्द्रि॒ येणै॒व । \newline
29. इ॒न्द्रि॒येणै॒ वैवेन्द्रि॒ये णे न्द्रि॒ये णै॒वैन॑ मेन मे॒वेन्द्रि॒ये णे᳚न्द्रि॒ये णै॒वैन᳚म् । \newline
30. ए॒वैन॑ मेन मे॒वै वैन॑म् दे॒वता॑भिर् दे॒वता॑भि रेन मे॒वै वैन॑म् दे॒वता॑भिः । \newline
31. ए॒न॒म् दे॒वता॑भिर् दे॒वता॑भिरेन मेनम् दे॒वता॑भिः॒ सꣳ सम् दे॒वता॑भि रेन मेनम् दे॒वता॑भिः॒ सम् । \newline
32. दे॒वता॑भिः॒ सꣳ सम् दे॒वता॑भिर् दे॒वता॑भिः॒ सन् न॑यति नयति॒ सम् दे॒वता॑भिर् दे॒वता॑भिः॒ सन्न॑यति । \newline
33. सन् न॑यति नयति॒ सꣳ सन् न॑यति॒ यद् यन् न॑यति॒ सꣳ सन् न॑यति॒ यत् । \newline
34. न॒य॒ति॒ यद् यन् न॑यति नयति॒ यदे॒त दे॒तद् यन् न॑यति नयति॒ यदे॒तत् । \newline
35. यदे॒त दे॒तद् यद् यदे॒तद् यजु॒र् यजु॑ रे॒तद् यद् यदे॒तद् यजुः॑ । \newline
36. ए॒तद् यजु॒र् यजु॑ रे॒त दे॒तद् यजु॒र् न न यजु॑ रे॒त दे॒तद् यजु॒र् न । \newline
37. यजु॒र् न न यजु॒र् यजु॒र् न ब्रू॒याद् ब्रू॒यान् न यजु॒र् यजु॒र् न ब्रू॒यात् । \newline
38. न ब्रू॒याद् ब्रू॒यान् न न ब्रू॒याद् याव॑तो॒ याव॑तो ब्रू॒यान् न न ब्रू॒याद् याव॑तः । \newline
39. ब्रू॒याद् याव॑तो॒ याव॑तो ब्रू॒याद् ब्रू॒याद् याव॑त ए॒वैव याव॑तो ब्रू॒याद् ब्रू॒याद् याव॑त ए॒व । \newline
40. याव॑त ए॒वैव याव॑तो॒ याव॑त ए॒व प॒शून् प॒शूने॒व याव॑तो॒ याव॑त ए॒व प॒शून् । \newline
41. ए॒व प॒शून् प॒शू ने॒वैव प॒शू न॒भ्य॑भि प॒शू ने॒वैव प॒शून॒भि । \newline
42. प॒शू न॒भ्य॑भि प॒शून् प॒शू न॒भि दीक्षे॑त॒ दीक्षे॑ता॒भि प॒शून् प॒शू न॒भि दीक्षे॑त । \newline
43. अ॒भि दीक्षे॑त॒ दीक्षे॑ता॒ भ्य॑भि दीक्षे॑त॒ ताव॑न्त॒ स्ताव॑न्तो॒ दीक्षे॑ता॒ भ्य॑भि दीक्षे॑त॒ ताव॑न्तः । \newline
44. दीक्षे॑त॒ ताव॑न्त॒ स्ताव॑न्तो॒ दीक्षे॑त॒ दीक्षे॑त॒ ताव॑न्तो ऽस्यास्य॒ ताव॑न्तो॒ दीक्षे॑त॒ दीक्षे॑त॒ ताव॑न्तो ऽस्य । \newline
45. ताव॑न्तो ऽस्यास्य॒ ताव॑न्त॒ स्ताव॑न्तो ऽस्य प॒शवः॑ प॒शवो᳚ ऽस्य॒ ताव॑न्त॒ स्ताव॑न्तो ऽस्य प॒शवः॑ । \newline
46. अ॒स्य॒ प॒शवः॑ प॒शवो᳚ ऽस्यास्य प॒शवः॑ स्युः स्युः प॒शवो᳚ ऽस्यास्य प॒शवः॑ स्युः । \newline
47. प॒शवः॑ स्युः स्युः प॒शवः॑ प॒शवः॑ स्यू॒ रास्व॒ रास्व॑ स्युः प॒शवः॑ प॒शवः॑ स्यू॒ रास्व॑ । \newline
48. स्यू॒ रास्व॒ रास्व॑ स्युः स्यू॒ रास्वे य॒दिय॒द् रास्व॑ स्युः स्यू॒ रास्वेय॑त् । \newline
49. रास्वे य॒दिय॒द् रास्व॒ रास्वेय॑थ् सोम सो॒मे य॒द् रास्व॒ रास्वेय॑थ् सोम । \newline
50. इय॑थ् सोम सो॒मे य॒दिय॑थ् सो॒मा सो॒मे य॒दिय॑थ् सो॒मा । \newline
\pagebreak
\markright{ TS 6.1.4.8  \hfill https://www.vedavms.in \hfill}

\section{ TS 6.1.4.8 }

\textbf{TS 6.1.4.8 } \newline
\textbf{Samhita Paata} \newline

सो॒माऽऽ* भूयो॑ भ॒रेत्या॒हा-प॑रिमिताने॒व प॒शूनव॑ रुन्धे च॒न्द्रम॑सि॒ मम॒ भोगा॑य भ॒वेत्या॑ह यथादेव॒तमे॒वैनाः॒ प्रति॑ गृह्णाति वा॒यवे᳚ त्वा॒ वरु॑णाय॒ त्वेति॒ यदे॒वमे॒ता नानु॑दि॒शेदय॑थादेवतं॒ दक्षि॑णा गमये॒दा दे॒वता᳚भ्यो वृश्च्येत॒ यदे॒वमे॒ता अ॑नुदि॒शति॑ यथादेव॒तमे॒व दक्षि॑णा गमयति॒ न दे॒वता᳚भ्य॒ आ - [  ] \newline

\textbf{Pada Paata} \newline

सो॒म॒ । एति॑ । भूयः॑ । भ॒र॒ । इति॑ । आ॒ह॒ । अप॑रिमिता॒नित्यप॑रि-मि॒ता॒न् । ए॒व । प॒शून् । अवेति॑ । रु॒न्धे॒ । च॒न्द्रम् । अ॒सि॒ । मम॑ । भोगा॑य । भ॒व॒ । इति॑ । आ॒ह॒ । य॒था॒दे॒व॒तमिति॑ यथा - दे॒व॒तम् । ए॒व । ए॒नाः॒ । प्रतीति॑ । गृ॒ह्णा॒ति॒ । वा॒यवे᳚ । त्वा॒ । वरु॑णाय । त्वा॒ । इति॑ । यत् । ए॒वम् । ए॒ताः । न । अ॒नु॒दि॒शेदित्य॑नु - दि॒शेत् । अय॑थादेवत॒मित्यय॑था - दे॒व॒त॒म् । दक्षि॑णाः । ग॒म॒ये॒त् । एति॑ । दे॒वता᳚भ्यः । वृ॒श्च्ये॒त॒ । यत् । ए॒वम् । ए॒ताः । अ॒नु॒दि॒शतीत्य॑नु - दि॒शति॑ । य॒था॒दे॒व॒तमिति॑ यथा - दे॒व॒तम् । ए॒व । दक्षि॑णाः । ग॒म॒य॒ति॒ । न । दे॒वता᳚भ्यः । एति॑ ।  \newline


\textbf{Krama Paata} \newline

सो॒मा । आ भूयः॑ । भूयो॑ भर । भ॒रेति॑ । इत्या॑ह । आ॒हाप॑रिमितान् । अप॑रिमिताने॒व । अप॑रिमिता॒नित्यप॑रि - मि॒ता॒न्॒ । ए॒व प॒शून् । प॒शूनव॑ । अव॑ रुन्धे । रु॒न्धे॒ च॒न्द्रम् । च॒न्द्रम॑सि । अ॒सि॒ मम॑ । मम॒ भोगा॑य । भोगा॑य भव । भ॒वेति॑ । इत्या॑ह । आ॒ह॒ य॒था॒दे॒व॒तम् । य॒था॒दे॒व॒तमे॒व । य॒था॒दे॒व॒तमिति॑ यथा - दे॒व॒तम् । ए॒वैनाः᳚ । एनाः॒ प्रति॑ । प्रति॑ गृह्णाति । गृ॒ह्णा॒ति॒ वा॒यवे᳚ । वा॒यवे᳚ त्वा । त्वा॒ वरु॑णाय । वरु॑णाय त्वा । त्वेति॑ । इति॒ यत् । यदे॒वम् । ए॒वमे॒ताः । ए॒ता न । नानु॑दि॒शेत् । अ॒नु॒दि॒शेदय॑थादेवतम् । अ॒नु॒दि॒शेदित्य॑नु - दि॒शेत् । अय॑थादेवत॒म् दक्षि॑णाः । अय॑थादेवत॒मित्यय॑था - दे॒व॒त॒म् । दक्षि॑णा गमयेत् । ग॒म॒ये॒दा । आ दे॒वता᳚भ्यः । दे॒वता᳚भ्यो वृश्च्येत । वृ॒श्च्ये॒त॒ यत् । यदे॒वम् । ए॒वमे॒ताः । ए॒ता अ॑नुदि॒शति॑ । अ॒नु॒दि॒शति॑ यथादेव॒तम् । अ॒नु॒दि॒शतीत्य॑नु - दि॒शति॑ । य॒था॒दे॒व॒तमे॒व । य॒था॒दे॒व॒तमिति॑ यथा - दे॒व॒तम् । ए॒व दक्षि॑णाः । दक्षि॑णा गमयति । ग॒म॒य॒ति॒ न । न दे॒वता᳚भ्यः । दे॒वता᳚भ्य॒ आ ( ) । आ वृ॑श्च्यते \newline

\textbf{Jatai Paata} \newline

1. सो॒मा सो॑म सो॒मा । \newline
2. आ भूयो॒ भूय॒ आ भूयः॑ । \newline
3. भूयो॑ भर भर॒ भूयो॒ भूयो॑ भर । \newline
4. भ॒रे तीति॑ भर भ॒रेति॑ । \newline
5. इत्या॑हा॒हे तीत्या॑ह । \newline
6. आ॒हा प॑रिमिता॒न प॑रिमिताना हा॒हा प॑रिमितान् । \newline
7. अप॑रिमिताने॒ वैवा प॑रिमिता॒न प॑रिमिताने॒व । \newline
8. अप॑रिमिता॒नित्यप॑रि - मि॒ता॒न् । \newline
9. ए॒व प॒शून् प॒शूने॒ वैव प॒शून् । \newline
10. प॒शून वाव॑ प॒शून् प॒शूनव॑ । \newline
11. अव॑ रुन्धे रु॒न्धे ऽवाव॑ रुन्धे । \newline
12. रु॒न्धे॒ च॒न्द्रम् च॒न्द्रꣳ रु॑न्धे रुन्धे च॒न्द्रम् । \newline
13. च॒न्द्र म॑स्यसि च॒न्द्रम् च॒न्द्र म॑सि । \newline
14. अ॒सि॒ मम॒ ममा᳚ स्यसि॒ मम॑ । \newline
15. मम॒ भोगा॑य॒ भोगा॑य॒ मम॒ मम॒ भोगा॑य । \newline
16. भोगा॑य भव भव॒ भोगा॑य॒ भोगा॑य भव । \newline
17. भ॒वे तीति॑ भव भ॒वेति॑ । \newline
18. इत्या॑हा॒हे तीत्या॑ह । \newline
19. आ॒ह॒ य॒था॒दे॒व॒तं ॅय॑थादेव॒त मा॑हाह यथादेव॒तम् । \newline
20. य॒था॒दे॒व॒त मे॒वैव य॑थादेव॒तं ॅय॑थादेव॒त मे॒व । \newline
21. य॒था॒दे॒व॒तमिति॑ यथा - दे॒व॒तम् । \newline
22. ए॒वैना॑ एना ए॒वै वैनाः᳚ । \newline
23. ए॒नाः॒ प्रति॒ प्रत्ये॑ना एनाः॒ प्रति॑ । \newline
24. प्रति॑ गृह्णाति गृह्णाति॒ प्रति॒ प्रति॑ गृह्णाति । \newline
25. गृ॒ह्णा॒ति॒ वा॒यवे॑ वा॒यवे॑ गृह्णाति गृह्णाति वा॒यवे᳚ । \newline
26. वा॒यवे᳚ त्वा त्वा वा॒यवे॑ वा॒यवे᳚ त्वा । \newline
27. त्वा॒ वरु॑णाय॒ वरु॑णाय त्वा त्वा॒ वरु॑णाय । \newline
28. वरु॑णाय त्वा त्वा॒ वरु॑णाय॒ वरु॑णाय त्वा । \newline
29. त्वेतीति॑ त्वा॒ त्वेति॑ । \newline
30. इति॒ यद् यदितीति॒ यत् । \newline
31. यदे॒व मे॒वं ॅयद् यदे॒वम् । \newline
32. ए॒व मे॒ता ए॒ता ए॒व मे॒व मे॒ताः । \newline
33. ए॒ता न नैता ए॒ता न । \newline
34. नानु॑दि॒शे द॑नुदि॒शेन् न नानु॑दि॒शेत् । \newline
35. अ॒नु॒दि॒शे दय॑थादेवत॒ मय॑थादेवत मनुदि॒शे द॑नुदि॒शे दय॑थादेवतम् । \newline
36. अ॒नु॒दि॒शेदित्य॑नु - दि॒शेत् । \newline
37. अय॑थादेवत॒म् दक्षि॑णा॒ दक्षि॑णा॒ अय॑थादेवत॒ मय॑थादेवत॒म् दक्षि॑णाः । \newline
38. अय॑थादेवत॒मित्यय॑था - दे॒व॒त॒म् । \newline
39. दक्षि॑णा गमयेद् गमये॒द् दक्षि॑णा॒ दक्षि॑णा गमयेत् । \newline
40. ग॒म॒ये॒दा ग॑मयेद् गमये॒दा । \newline
41. आ दे॒वता᳚भ्यो दे॒वता᳚भ्य॒ आ दे॒वता᳚भ्यः । \newline
42. दे॒वता᳚भ्यो वृश्च्येत वृश्च्येत दे॒वता᳚भ्यो दे॒वता᳚भ्यो वृश्च्येत । \newline
43. वृ॒श्च्ये॒त॒ यद् यद् वृ॑श्च्येत वृश्च्येत॒ यत् । \newline
44. यदे॒व मे॒वं ॅयद् यदे॒वम् । \newline
45. ए॒व मे॒ता ए॒ता ए॒व मे॒व मे॒ताः । \newline
46. ए॒ता अ॑नुदि॒श त्य॑नुदि॒श त्ये॒ता ए॒ता अ॑नुदि॒शति॑ । \newline
47. अ॒नु॒दि॒शति॑ यथादेव॒तं ॅय॑थादेव॒त म॑नुदि॒श त्य॑नुदि॒शति॑ यथादेव॒तम् । \newline
48. अ॒नु॒दि॒शतीत्य॑नु - दि॒शति॑ । \newline
49. य॒था॒दे॒व॒त मे॒वैव य॑थादेव॒तं ॅय॑थादेव॒त मे॒व । \newline
50. य॒था॒दे॒व॒तमिति॑ यथा - दे॒व॒तम् । \newline
51. ए॒व दक्षि॑णा॒ दक्षि॑णा ए॒वैव दक्षि॑णाः । \newline
52. दक्षि॑णा गमयति गमयति॒ दक्षि॑णा॒ दक्षि॑णा गमयति । \newline
53. ग॒म॒य॒ति॒ न न ग॑मयति गमयति॒ न । \newline
54. न दे॒वता᳚भ्यो दे॒वता᳚भ्यो॒ न न दे॒वता᳚भ्यः । \newline
55. दे॒वता᳚भ्य॒ आ दे॒वता᳚भ्यो दे॒वता᳚भ्य॒ आ । \newline
56. आ वृ॑श्च्यते वृश्च्यत॒ आ वृ॑श्च्यते । \newline

\textbf{Ghana Paata } \newline

1. सो॒मा सो॑म सो॒मा भूयो॒ भूय॒ आ सो॑म सो॒मा भूयः॑ । \newline
2. आ भूयो॒ भूय॒ आ भूयो॑ भर भर॒ भूय॒ आ भूयो॑ भर । \newline
3. भूयो॑ भर भर॒ भूयो॒ भूयो॑ भ॒रे तीति॑ भर॒ भूयो॒ भूयो॑ भ॒रेति॑ । \newline
4. भ॒रे तीति॑ भर भ॒रे त्या॑हा॒ हेति॑ भर भ॒रे त्या॑ह । \newline
5. इत्या॑हा॒हे तीत्या॒हा प॑रिमिता॒ नप॑रिमिता ना॒हे तीत्या॒हा प॑रिमितान् । \newline
6. आ॒हा प॑रिमिता॒ नप॑रिमिता नाहा॒हा प॑रिमिता ने॒वैवा प॑रिमिता नाहा॒हा प॑रिमिता ने॒व । \newline
7. अप॑रिमिता ने॒वैवा प॑रिमिता॒ नप॑रिमिता ने॒व प॒शून् प॒शू ने॒वा प॑रिमिता॒ नप॑रिमिता ने॒व प॒शून् । \newline
8. अप॑रिमिता॒नित्यप॑रि - मि॒ता॒न् । \newline
9. ए॒व प॒शून् प॒शू ने॒वैव प॒शू नवाव॑ प॒शू ने॒वैव प॒शू नव॑ । \newline
10. प॒शू नवाव॑ प॒शून् प॒शू नव॑ रुन्धे रु॒न्धे ऽव॑ प॒शून् प॒शू नव॑ रुन्धे । \newline
11. अव॑ रुन्धे रु॒न्धे ऽवाव॑ रुन्धे च॒न्द्रम् च॒न्द्रꣳ रु॒न्धे ऽवाव॑ रुन्धे च॒न्द्रम् । \newline
12. रु॒न्धे॒ च॒न्द्रम् च॒न्द्रꣳ रु॑न्धे रुन्धे च॒न्द्र म॑स्यसि च॒न्द्रꣳ रु॑न्धे रुन्धे च॒न्द्र म॑सि । \newline
13. च॒न्द्र म॑स्यसि च॒न्द्रम् च॒न्द्र म॑सि॒ मम॒ ममा॑सि च॒न्द्रम् च॒न्द्र म॑सि॒ मम॑ । \newline
14. अ॒सि॒ मम॒ ममा᳚स्यसि॒ मम॒ भोगा॑य॒ भोगा॑य॒ ममा᳚ स्यसि॒ मम॒ भोगा॑य । \newline
15. मम॒ भोगा॑य॒ भोगा॑य॒ मम॒ मम॒ भोगा॑य भव भव॒ भोगा॑य॒ मम॒ मम॒ भोगा॑य भव । \newline
16. भोगा॑य भव भव॒ भोगा॑य॒ भोगा॑य भ॒वे तीति॑ भव॒ भोगा॑य॒ भोगा॑य भ॒वेति॑ । \newline
17. भ॒वे तीति॑ भव भ॒वे त्या॑हा॒ हेति॑ भव भ॒वे त्या॑ह । \newline
18. इत्या॑हा॒हे तीत्या॑ह यथादेव॒तं ॅय॑थादेव॒त मा॒हे तीत्या॑ह यथादेव॒तम् । \newline
19. आ॒ह॒ य॒था॒दे॒व॒तं ॅय॑थादेव॒त मा॑हाह यथादेव॒त मे॒वैव य॑थादेव॒त मा॑हाह यथादेव॒त मे॒व । \newline
20. य॒था॒दे॒व॒त मे॒वैव य॑थादेव॒तं ॅय॑थादेव॒त मे॒वैना॑ एना ए॒व य॑थादेव॒तं ॅय॑थादेव॒त मे॒वैनाः᳚ । \newline
21. य॒था॒दे॒व॒तमिति॑ यथा - दे॒व॒तम् । \newline
22. ए॒वैना॑ एना ए॒वै वैनाः॒ प्रति॒ प्रत्ये॑ना ए॒वै वैनाः॒ प्रति॑ । \newline
23. ए॒नाः॒ प्रति॒ प्रत्ये॑ना एनाः॒ प्रति॑ गृह्णाति गृह्णाति॒ प्रत्ये॑ना एनाः॒ प्रति॑ गृह्णाति । \newline
24. प्रति॑ गृह्णाति गृह्णाति॒ प्रति॒ प्रति॑ गृह्णाति वा॒यवे॑ वा॒यवे॑ गृह्णाति॒ प्रति॒ प्रति॑ गृह्णाति वा॒यवे᳚ । \newline
25. गृ॒ह्णा॒ति॒ वा॒यवे॑ वा॒यवे॑ गृह्णाति गृह्णाति वा॒यवे᳚ त्वा त्वा वा॒यवे॑ गृह्णाति गृह्णाति वा॒यवे᳚ त्वा । \newline
26. वा॒यवे᳚ त्वा त्वा वा॒यवे॑ वा॒यवे᳚ त्वा॒ वरु॑णाय॒ वरु॑णाय त्वा वा॒यवे॑ वा॒यवे᳚ त्वा॒ वरु॑णाय । \newline
27. त्वा॒ वरु॑णाय॒ वरु॑णाय त्वा त्वा॒ वरु॑णाय त्वा त्वा॒ वरु॑णाय त्वा त्वा॒ वरु॑णाय त्वा । \newline
28. वरु॑णाय त्वा त्वा॒ वरु॑णाय॒ वरु॑णाय॒ त्वेतीति॑ त्वा॒ वरु॑णाय॒ वरु॑णाय॒ त्वेति॑ । \newline
29. त्वेतीति॑ त्वा॒ त्वेति॒ यद् यदिति॑ त्वा॒ त्वेति॒ यत् । \newline
30. इति॒ यद् यदितीति॒ यदे॒व मे॒वं ॅयदितीति॒ यदे॒वम् । \newline
31. यदे॒व मे॒वं ॅयद् यदे॒व मे॒ता ए॒ता ए॒वं ॅयद् यदे॒व मे॒ताः । \newline
32. ए॒व मे॒ता ए॒ता ए॒व मे॒व मे॒ता न नैता ए॒व मे॒व मे॒ता न । \newline
33. ए॒ता न नैता ए॒ता नानु॑दि॒शे द॑नुदि॒शेन् नैता ए॒ता नानु॑दि॒शेत् । \newline
34. नानु॑दि॒शे द॑नुदि॒शेन् न नानु॑दि॒शे दय॑थादेवत॒ मय॑थादेवत मनुदि॒शेन् न नानु॑दि॒शे दय॑थादेवतम् । \newline
35. अ॒नु॒दि॒शे दय॑थादेवत॒ मय॑थादेवत मनुदि॒शे द॑नुदि॒शे दय॑थादेवत॒म् दक्षि॑णा॒ दक्षि॑णा॒ अय॑थादेवत मनुदि॒शे द॑नुदि॒शे दय॑थादेवत॒म् दक्षि॑णाः । \newline
36. अ॒नु॒दि॒शेदित्य॑नु - दि॒शेत् । \newline
37. अय॑थादेवत॒म् दक्षि॑णा॒ दक्षि॑णा॒ अय॑थादेवत॒ मय॑थादेवत॒म् दक्षि॑णा गमयेद् गमये॒द् दक्षि॑णा॒ अय॑थादेवत॒ मय॑थादेवत॒म् दक्षि॑णा गमयेत् । \newline
38. अय॑थादेवत॒मित्यय॑था - दे॒व॒त॒म् । \newline
39. दक्षि॑णा गमयेद् गमये॒द् दक्षि॑णा॒ दक्षि॑णा गमये॒दा ग॑मये॒द् दक्षि॑णा॒ दक्षि॑णा गमये॒दा । \newline
40. ग॒म॒ये॒दा ग॑मयेद् गमये॒दा दे॒वता᳚भ्यो दे॒वता᳚भ्य॒ आ ग॑मयेद् गमये॒दा दे॒वता᳚भ्यः । \newline
41. आ दे॒वता᳚भ्यो दे॒वता᳚भ्य॒ आ दे॒वता᳚भ्यो वृश्च्येत वृश्च्येत दे॒वता᳚भ्य॒ आ दे॒वता᳚भ्यो वृश्च्येत । \newline
42. दे॒वता᳚भ्यो वृश्च्येत वृश्च्येत दे॒वता᳚भ्यो दे॒वता᳚भ्यो वृश्च्येत॒ यद् यद् वृ॑श्च्येत दे॒वता᳚भ्यो दे॒वता᳚भ्यो वृश्च्येत॒ यत् । \newline
43. वृ॒श्च्ये॒त॒ यद् यद् वृ॑श्च्येत वृश्च्येत॒ यदे॒व मे॒वं ॅयद् वृ॑श्च्येत वृश्च्येत॒ यदे॒वम् । \newline
44. यदे॒व मे॒वं ॅयद् यदे॒व मे॒ता ए॒ता ए॒वं ॅयद् यदे॒व मे॒ताः । \newline
45. ए॒व मे॒ता ए॒ता ए॒व मे॒व मे॒ता अ॑नुदि॒श त्य॑नुदि॒श त्ये॒ता ए॒व मे॒व मे॒ता अ॑नुदि॒शति॑ । \newline
46. ए॒ता अ॑नुदि॒श त्य॑नुदि॒श त्ये॒ता ए॒ता अ॑नुदि॒शति॑ यथादेव॒तं ॅय॑थादेव॒त म॑नुदि॒श त्ये॒ता ए॒ता अ॑नुदि॒शति॑ यथादेव॒तम् । \newline
47. अ॒नु॒दि॒शति॑ यथादेव॒तं ॅय॑थादेव॒त म॑नुदि॒श त्य॑नुदि॒शति॑ यथादेव॒त मे॒वैव य॑थादेव॒त म॑नुदि॒श त्य॑नुदि॒शति॑ यथादेव॒त मे॒व । \newline
48. अ॒नु॒दि॒शतीत्य॑नु - दि॒शति॑ । \newline
49. य॒था॒दे॒व॒त मे॒वैव य॑थादेव॒तं ॅय॑थादेव॒त मे॒व दक्षि॑णा॒ दक्षि॑णा ए॒व य॑थादेव॒तं ॅय॑थादेव॒त मे॒व दक्षि॑णाः । \newline
50. य॒था॒दे॒व॒तमिति॑ यथा - दे॒व॒तम् । \newline
51. ए॒व दक्षि॑णा॒ दक्षि॑णा ए॒वैव दक्षि॑णा गमयति गमयति॒ दक्षि॑णा ए॒वैव दक्षि॑णा गमयति । \newline
52. दक्षि॑णा गमयति गमयति॒ दक्षि॑णा॒ दक्षि॑णा गमयति॒ न न ग॑मयति॒ दक्षि॑णा॒ दक्षि॑णा गमयति॒ न । \newline
53. ग॒म॒य॒ति॒ न न ग॑मयति गमयति॒ न दे॒वता᳚भ्यो दे॒वता᳚भ्यो॒ न ग॑मयति गमयति॒ न दे॒वता᳚भ्यः । \newline
54. न दे॒वता᳚भ्यो दे॒वता᳚भ्यो॒ न न दे॒वता᳚भ्य॒ आ दे॒वता᳚भ्यो॒ न न दे॒वता᳚भ्य॒ आ । \newline
55. दे॒वता᳚भ्य॒ आ दे॒वता᳚भ्यो दे॒वता᳚भ्य॒ आ वृ॑श्च्यते वृश्च्यत॒ आ दे॒वता᳚भ्यो दे॒वता᳚भ्य॒ आ वृ॑श्च्यते । \newline
56. आ वृ॑श्च्यते वृश्च्यत॒ आ वृ॑श्च्यते॒ देवी॒र् देवी᳚र् वृश्च्यत॒ आ वृ॑श्च्यते॒ देवीः᳚ । \newline
\pagebreak
\markright{ TS 6.1.4.9  \hfill https://www.vedavms.in \hfill}

\section{ TS 6.1.4.9 }

\textbf{TS 6.1.4.9 } \newline
\textbf{Samhita Paata} \newline

वृ॑श्च्यते॒ देवी॑रापो अपां नपा॒दित्या॑ह॒ यद्वो॒ मेद्ध्यं॑ ॅय॒ज्ञियꣳ॒॒ सदे॑वं॒ तद्वो॒ माऽव॑ क्रमिष॒मिति॒ वावैतदा॒हाच्छि॑न्नं॒ तन्तुं॑ पृथि॒व्या अनु॑ गेष॒मित्या॑ह॒ सेतु॑मे॒व कृ॒त्वाऽत्ये॑ति ॥ \newline

\textbf{Pada Paata} \newline

वृ॒श्च्य॒ते॒ । देवीः᳚ । आ॒पः॒ । अ॒पा॒म् । न॒पा॒त् । इति॑ । आ॒ह॒ । यत् । वः॒ । मेद्ध्य᳚म् । य॒ज्ञिय᳚म् । सदे॑व॒मिति॒ स - दे॒व॒म् । तत् । वः॒ । मा । अवेति॑ । क्र॒मि॒ष॒म् । इति॑ । वाव । ए॒तत् । आ॒ह॒ । अच्छि॑न्नम् । तन्तु᳚म् । पृ॒थि॒व्याः । अन्विति॑ । गे॒ष॒म् । इति॑ । आ॒ह॒ । सेतु᳚म् । ए॒व । कृ॒त्वा । अतीति॑ । ए॒ति॒ ॥  \newline


\textbf{Krama Paata} \newline

वृ॒श्च्य॒ते॒ देवीः᳚ । देवी॑रापः । आ॒पो॒ अ॒पा॒म् । अ॒पा॒म् न॒पा॒त्॒ । न॒पा॒दिति॑ । इत्या॑ह । आ॒ह॒ यत् । यद् वः॑ । वो॒ मेद्ध्य᳚म् । मेद्ध्य॑म् ॅय॒ज्ञिय᳚म् । य॒ज्ञियꣳ॒॒ सदे॑वम् । सदे॑व॒म् तत् । सदे॑व॒मिति॒ स - दे॒व॒म् । तद् वः॑ । वो॒ मा । माऽव॑ । अव॑ क्रमिषम् । क्र॒मि॒ष॒मिति॑ । इति॒ वाव । वावैतत् । ए॒तदा॑ह । आ॒हाच्छि॑न्नम् । अच्छि॑न्न॒म् तन्तु᳚म् । तन्तु॑म् पृथि॒व्याः । पृ॒थि॒व्या अनु॑ । अनु॑ गेषम् । गे॒ष॒मिति॑ । इत्या॑ह । आ॒ह॒ सेतु᳚म् । सेतु॑मे॒व । ए॒व कृ॒त्वा । कृ॒त्वाऽति॑ । अत्ये॑ति । ए॒तीत्ये॑ति । \newline

\textbf{Jatai Paata} \newline

1. वृ॒श्च्य॒ते॒ देवी॒र् देवी᳚र् वृश्च्यते वृश्च्यते॒ देवीः᳚ । \newline
2. देवी॑ राप आपो॒ देवी॒र् देवी॑ रापः । \newline
3. आ॒पो॒ अ॒पा॒ म॒पा॒ मा॒प॒ आ॒पो॒ अ॒पा॒म् । \newline
4. अ॒पा॒न् न॒पा॒न् न॒पा॒ द॒पा॒ म॒पा॒न् न॒पा॒त् । \newline
5. न॒पा॒ दितीति॑ नपान् नपा॒ दिति॑ । \newline
6. इत्या॑हा॒हे तीत्या॑ह । \newline
7. आ॒ह॒ यद् यदा॑हाह॒ यत् । \newline
8. यद् वो॑ वो॒ यद् यद् वः॑ । \newline
9. वो॒ मेद्ध्य॒म् मेद्ध्यं॑ ॅवो वो॒ मेद्ध्य᳚म् । \newline
10. मेद्ध्यं॑ ॅय॒ज्ञियं॑ ॅय॒ज्ञिय॒म् मेद्ध्य॒म् मेद्ध्यं॑ ॅय॒ज्ञिय᳚म् । \newline
11. य॒ज्ञियꣳ॒॒ सदे॑वꣳ॒॒ सदे॑वं ॅय॒ज्ञियं॑ ॅय॒ज्ञियꣳ॒॒ सदे॑वम् । \newline
12. सदे॑व॒म् तत् तथ् सदे॑वꣳ॒॒ सदे॑व॒म् तत् । \newline
13. सदे॑व॒मिति॒ स - दे॒व॒म् । \newline
14. तद् वो॑ व॒ स्तत् तद् वः॑ । \newline
15. वो॒ मा मा वो॑ वो॒ मा । \newline
16. मा ऽवाव॒ मा मा ऽव॑ । \newline
17. अव॑ क्रमिषम् क्रमिष॒ मवाव॑ क्रमिषम् । \newline
18. क्र॒मि॒ष॒ मितीति॑ क्रमिषम् क्रमिष॒ मिति॑ । \newline
19. इति॒ वाव वावे तीति॒ वाव । \newline
20. वावै तदे॒तद् वाव वावैतत् । \newline
21. ए॒त दा॑हाहै॒ तदे॒त दा॑ह । \newline
22. आ॒हा च्छि॑न्न॒ मच्छि॑न्न माहा॒हा च्छि॑न्नम् । \newline
23. अच्छि॑न्न॒म् तन्तु॒म् तन्तु॒ मच्छि॑न्न॒ मच्छि॑न्न॒म् तन्तु᳚म् । \newline
24. तन्तु॑म् पृथि॒व्याः पृ॑थि॒व्या स्तन्तु॒म् तन्तु॑म् पृथि॒व्याः । \newline
25. पृ॒थि॒व्या अन्वनु॑ पृथि॒व्याः पृ॑थि॒व्या अनु॑ । \newline
26. अनु॑ गेषम् गेष॒ मन्वनु॑ गेषम् । \newline
27. गे॒ष॒ मितीति॑ गेषम् गेष॒ मिति॑ । \newline
28. इत्या॑हा॒हे तीत्या॑ह । \newline
29. आ॒ह॒ सेतुꣳ॒॒ सेतु॑ माहाह॒ सेतु᳚म् । \newline
30. सेतु॑ मे॒वैव सेतुꣳ॒॒ सेतु॑ मे॒व । \newline
31. ए॒व कृ॒त्वा कृ॒त्वै वैव कृ॒त्वा । \newline
32. कृ॒त्वा ऽत्यति॑ कृ॒त्वा कृ॒त्वा ऽति॑ । \newline
33. अत्ये᳚त्ये॒ त्यत्य त्ये॑ति । \newline
34. ए॒तीत्ये॑ति । \newline

\textbf{Ghana Paata } \newline

1. वृ॒श्च्य॒ते॒ देवी॒र् देवी᳚र् वृश्च्यते वृश्च्यते॒ देवी॑राप आपो॒ देवी᳚र् वृश्च्यते वृश्च्यते॒ देवी॑रापः । \newline
2. देवी॑ राप आपो॒ देवी॒र् देवी॑ रापो अपा मपा मापो॒ देवी॒र् देवी॑ रापो अपाम् । \newline
3. आ॒पो॒ अ॒पा॒ म॒पा॒ मा॒प॒ आ॒पो॒ अ॒पा॒म् न॒पा॒न् न॒पा॒ द॒पा॒ मा॒प॒ आ॒पो॒ अ॒पा॒म् न॒पा॒त् । \newline
4. अ॒पा॒म् न॒पा॒न् न॒पा॒ द॒पा॒ म॒पा॒म् न॒पा॒दि तीति॑ नपादपा मपाम् नपा॒ दिति॑ । \newline
5. न॒पा॒दि तीति॑ नपान् नपा॒दित्या॑ हा॒हेति॑ नपान् नपा॒ दित्या॑ह । \newline
6. इत्या॑हा॒हे तीत्या॑ह॒ यद् यदा॒हे तीत्या॑ह॒ यत् । \newline
7. आ॒ह॒ यद् यदा॑हाह॒ यद् वो॑ वो॒ यदा॑हाह॒ यद् वः॑ । \newline
8. यद् वो॑ वो॒ यद् यद् वो॒ मेद्ध्य॒म् मेद्ध्यं॑ ॅवो॒ यद् यद् वो॒ मेद्ध्य᳚म् । \newline
9. वो॒ मेद्ध्य॒म् मेद्ध्यं॑ ॅवो वो॒ मेद्ध्यं॑ ॅय॒ज्ञियं॑ ॅय॒ज्ञिय॒म् मेद्ध्यं॑ ॅवो वो॒ मेद्ध्यं॑ ॅय॒ज्ञिय᳚म् । \newline
10. मेद्ध्यं॑ ॅय॒ज्ञियं॑ ॅय॒ज्ञिय॒म् मेद्ध्य॒म् मेद्ध्यं॑ ॅय॒ज्ञियꣳ॒॒ सदे॑वꣳ॒॒ सदे॑वं ॅय॒ज्ञिय॒म् मेद्ध्य॒म् मेद्ध्यं॑ ॅय॒ज्ञियꣳ॒॒ सदे॑वम् । \newline
11. य॒ज्ञियꣳ॒॒ सदे॑वꣳ॒॒ सदे॑वं ॅय॒ज्ञियं॑ ॅय॒ज्ञियꣳ॒॒ सदे॑व॒म् तत् तथ् सदे॑वं ॅय॒ज्ञियं॑ ॅय॒ज्ञियꣳ॒॒ सदे॑व॒म् तत् । \newline
12. सदे॑व॒म् तत् तथ् सदे॑वꣳ॒॒ सदे॑व॒म् तद् वो॑ व॒स्तथ् सदे॑वꣳ॒॒ सदे॑व॒म् तद् वः॑ । \newline
13. सदे॑व॒मिति॒ स - दे॒व॒म् । \newline
14. तद् वो॑ व॒ स्तत् तद् वो॒ मा मा व॒ स्तत् तद् वो॒ मा । \newline
15. वो॒ मा मा वो॑ वो॒ मा ऽवाव॒ मा वो॑ वो॒ मा ऽव॑ । \newline
16. मा ऽवाव॒ मा मा ऽव॑ क्रमिषम् क्रमिष॒ मव॒ मा मा ऽव॑ क्रमिषम् । \newline
17. अव॑ क्रमिषम् क्रमिष॒ मवाव॑ क्रमिष॒ मितीति॑ क्रमिष॒ मवाव॑ क्रमिष॒ मिति॑ । \newline
18. क्र॒मि॒ष॒ मितीति॑ क्रमिषम् क्रमिष॒ मिति॒ वाव वावेति॑ क्रमिषम् क्रमिष॒ मिति॒ वाव । \newline
19. इति॒ वाव वावे तीति॒ वावैत दे॒तद् वावे तीति॒ वावैतत् । \newline
20. वावैत दे॒तद् वाव वावैत दा॑हा है॒तद् वाव वावै तदा॑ह । \newline
21. ए॒त दा॑हा है॒त दे॒त दा॒हा च्छि॑न्न॒ मच्छि॑न्न माहै॒त दे॒त दा॒हा च्छि॑न्नम् । \newline
22. आ॒हाच् छि॑न्न॒ मच्छि॑न्न माहा॒हाच् छि॑न्न॒म् तन्तु॒म् तन्तु॒ मच्छि॑न्न माहा॒हाच् छि॑न्न॒म् तन्तु᳚म् । \newline
23. अच्छि॑न्न॒म् तन्तु॒म् तन्तु॒ मच्छि॑न्न॒ मच्छि॑न्न॒म् तन्तु॑म् पृथि॒व्याः पृ॑थि॒व्या स्तन्तु॒ मच्छि॑न्न॒ मच्छि॑न्न॒म् तन्तु॑म् पृथि॒व्याः । \newline
24. तन्तु॑म् पृथि॒व्याः पृ॑थि॒व्या स्तन्तु॒म् तन्तु॑म् पृथि॒व्या अन्वनु॑ पृथि॒व्या स्तन्तु॒म् तन्तु॑म् पृथि॒व्या अनु॑ । \newline
25. पृ॒थि॒व्या अन्वनु॑ पृथि॒व्याः पृ॑थि॒व्या अनु॑ गेषम् गेष॒ मनु॑ पृथि॒व्याः पृ॑थि॒व्या अनु॑ गेषम् । \newline
26. अनु॑ गेषम् गेष॒ मन्वनु॑ गेष॒ मितीति॑ गेष॒ मन्वनु॑ गेष॒ मिति॑ । \newline
27. गे॒ष॒ मितीति॑ गेषम् गेष॒ मित्या॑ हा॒हेति॑ गेषम् गेष॒ मित्या॑ह । \newline
28. इत्या॑हा॒हे तीत्या॑ह॒ सेतुꣳ॒॒ सेतु॑ मा॒हे तीत्या॑ह॒ सेतु᳚म् । \newline
29. आ॒ह॒ सेतुꣳ॒॒ सेतु॑ माहाह॒ सेतु॑ मे॒वैव सेतु॑ माहाह॒ सेतु॑ मे॒व । \newline
30. सेतु॑ मे॒वैव सेतुꣳ॒॒ सेतु॑ मे॒व कृ॒त्वा कृ॒त्वैव सेतुꣳ॒॒ सेतु॑ मे॒व कृ॒त्वा । \newline
31. ए॒व कृ॒त्वा कृ॒त्वै वैव कृ॒त्वा ऽत्यति॑ कृ॒त्वै वैव कृ॒त्वा ऽति॑ । \newline
32. कृ॒त्वा ऽत्यति॑ कृ॒त्वा कृ॒त्वा ऽत्ये᳚ त्ये॒त्यति॑ कृ॒त्वा कृ॒त्वा ऽत्ये॑ति । \newline
33. अत्ये᳚ त्ये॒त्य त्यत्ये॑ति । \newline
34. ए॒तीत्ये॑ति । \newline
\pagebreak
\markright{ TS 6.1.5.1  \hfill https://www.vedavms.in \hfill}

\section{ TS 6.1.5.1 }

\textbf{TS 6.1.5.1 } \newline
\textbf{Samhita Paata} \newline

दे॒वा वै दे॑व॒यज॑न-मद्ध्यव॒साय॒ दिशो॒ न प्राजा॑न॒न् ते᳚(1॒)ऽन्यो᳚-ऽन्यमुपा॑धाव॒न् त्वया॒ प्र जा॑नाम॒ त्वयेति॒ तेऽदि॑त्याꣳ॒॒ सम॒॑द्ध्रयन्त॒ त्वया॒ प्र जा॑ना॒मेति॒ साऽब्र॑वी॒द्-वरं॑ ॅवृणै॒ मत्प्रा॑यणा ए॒व वो॑ य॒ज्ञा मदु॑दयना अस॒न्निति॒ तस्मा॑दादि॒त्यः प्रा॑य॒णीयो॑ य॒ज्ञाना॑मादि॒त्य उ॑दय॒नीयः॒ पञ्च॑ दे॒वता॑ यजति॒ पञ्च॒ दिशो॑ दि॒शां प्रज्ञा᳚त्या॒ - [  ] \newline

\textbf{Pada Paata} \newline

दे॒वाः । वै । दे॒व॒यज॑न॒मिति॑ देव - यज॑नम् । अ॒द्ध्य॒व॒सायेत्य॑धि - अ॒व॒साय॑ । दिशः॑ । न । प्रेति॑ । अ॒जा॒न॒न्न् । ते । अ॒न्यः । अ॒न्यम् । उपेति॑ । अ॒धा॒व॒न्न् । त्वया᳚ । प्रेति॑ । जा॒ना॒म॒ । त्वया᳚ । इति॑ । ते । अदि॑त्याम् । समिति॑ । अ॒द्ध्र॒य॒न्त॒ । त्वया᳚ । प्रेति॑ । जा॒ना॒म॒ । इति॑ । सा । अ॒ब्र॒वी॒त् । वर᳚म् । वृ॒णै॒ । मत्प्रा॑यणा॒ इति॒ मत् - प्रा॒य॒णाः॒ । ए॒व । वः॒ । य॒ज्ञाः । मदु॑दयना॒ इति॒ मत्-उ॒द॒य॒नाः॒ । अ॒स॒न्न् । इति॑ । तस्मा᳚त् । आ॒दि॒त्यः । प्रा॒य॒णीय॒ इति॑ प्र-अ॒य॒नीयः॑ । य॒ज्ञाना᳚म् । आ॒दि॒त्यः । उ॒द॒य॒नीय॒ इत्यु॑त् - अ॒य॒नीयः॑ । पञ्च॑ । दे॒वताः᳚ । य॒ज॒ति॒ । पञ्च॑ । दिशः॑ । दि॒शाम् । प्रज्ञा᳚त्या॒ इति॒ प्र - ज्ञा॒त्यै॒ ।  \newline


\textbf{Krama Paata} \newline

दे॒वा वै । वै दे॑व॒यज॑नम् । दे॒व॒यज॑नमद्ध्यव॒साय॑ । दे॒व॒यज॑न॒मिति॑ देव - यज॑नम् । अ॒द्ध्य॒व॒साय॒ दिशः॑ । अ॒द्ध्य॒व॒सायेत्य॑धि - अ॒व॒साय॑ । दिशो॒ न । न प्र । प्राजा॑नन्न् । अ॒जा॒न॒न् ते । ते᳚(1॒)ऽन्यः॑ । अ॒न्यो᳚ऽन्यम् । अ॒न्यमुप॑ । उपा॑धावन्न् । अ॒धा॒व॒न् त्वया᳚ । त्वया॒ प्र । प्र जा॑नाम । जा॒ना॒म॒ त्वया᳚ । त्वयेति॑ । इति॒ ते । तेऽदि॑त्याम् । अदि॑त्याꣳ॒॒ सम् । सम॑द्ध्रियन्त । अ॒द्ध्रि॒य॒न्त॒ त्वया᳚ । त्वया॒ प्र । प्र जा॑नाम । जा॒ना॒मेति॑ । इति॒ सा । साऽब्र॑वीत् । अ॒ब्र॒वी॒द् वर᳚म् । वर॑म् ॅवृणै । वृ॒णै॒ मत्प्रा॑यणाः । मत्प्रा॑यणा ए॒व । मत्प्रा॑यणा॒ इति॒ मत् - प्रा॒य॒णाः॒ । ए॒व वः॑ । वो॒ य॒ज्ञाः । य॒ज्ञा मदु॑दयनाः । मदु॑दयना असन्न् । मदु॑दयना॒ इति॒ मत् - उ॒द॒य॒नाः॒ । अ॒स॒न्निति॑ । इति॒ तस्मा᳚त् । तस्मा॑दादि॒त्यः । आ॒दि॒त्यः प्रा॑य॒णीयः॑ । प्रा॒य॒णीयो॑ य॒ज्ञाना᳚म् । प्रा॒य॒णीय॒ इति॑ प्र - अ॒य॒नीयः॑ । य॒ज्ञाना॑मादि॒त्यः । आ॒दि॒त्य उ॑दय॒नीयः॑ । उ॒द॒य॒नीयः॒ पञ्च॑ । उ॒द॒य॒नीय॒ इत्यु॑त् - अ॒य॒नीयः॑ । पञ्च॑ दे॒वताः᳚ । दे॒वता॑ यजति । य॒ज॒ति॒ पञ्च॑ । पञ्च॒ दिशः॑ । दिशो॑ दि॒शाम् । दि॒शाम् प्रज्ञा᳚त्यै । प्रज्ञा᳚त्या॒ अथो᳚ । प्रज्ञा᳚त्या॒ इति॒ प्र - ज्ञा॒त्यै॒ \newline

\textbf{Jatai Paata} \newline

1. दे॒वा वै वै दे॒वा दे॒वा वै । \newline
2. वै दे॑व॒यज॑नम् देव॒यज॑नं॒ ॅवै वै दे॑व॒यज॑नम् । \newline
3. दे॒व॒यज॑न मद्ध्यव॒साया᳚ द्ध्यव॒साय॑ देव॒यज॑नम् देव॒यज॑न मद्ध्यव॒साय॑ । \newline
4. दे॒व॒यज॑न॒मिति॑ देव - यज॑नम् । \newline
5. अ॒द्ध्य॒व॒साय॒ दिशो॒ दिशो᳚ ऽद्ध्यव॒साया᳚ द्ध्यव॒साय॒ दिशः॑ । \newline
6. अ॒द्ध्य॒व॒सायेत्य॑धि - अ॒व॒साय॑ । \newline
7. दिशो॒ न न दिशो॒ दिशो॒ न । \newline
8. न प्र प्र ण न प्र । \newline
9. प्राजा॑नन् नजान॒न् प्र प्राजा॑नन्न् । \newline
10. अ॒जा॒न॒न् ते ते॑ ऽजानन् नजान॒न् ते । \newline
11. ते᳚(1॒) ऽन्यो᳚ ऽन्यस्ते ते᳚ ऽन्यः । \newline
12. अ॒न्यो᳚ ऽन्य म॒न्य म॒न्यो᳚(1॒) ऽन्यो᳚ ऽन्यम् । \newline
13. अ॒न्य मुपो पा॒न्य म॒न्य मुप॑ । \newline
14. उपा॑धावन् नधाव॒न् नुपोपा॑ धावन्न् । \newline
15. अ॒धा॒व॒न् त्वया॒ त्वया॑ ऽधावन् नधाव॒न् त्वया᳚ । \newline
16. त्वया॒ प्र प्र त्वया॒ त्वया॒ प्र । \newline
17. प्र जा॑नाम जानाम॒ प्र प्र जा॑नाम । \newline
18. जा॒ना॒म॒ त्वया॒ त्वया॑ जानाम जानाम॒ त्वया᳚ । \newline
19. त्वये तीति॒ त्वया॒ त्वयेति॑ । \newline
20. इति॒ ते त इतीति॒ ते । \newline
21. ते ऽदि॑त्या॒ मदि॑त्या॒म् ते ते ऽदि॑त्याम् । \newline
22. अदि॑त्याꣳ॒॒ सꣳ स मदि॑त्या॒ मदि॑त्याꣳ॒॒ सम् । \newline
23. स म॑द्ध्रयन्ता द्ध्रयन्त॒ सꣳ स म॑द्ध्रयन्त । \newline
24. अ॒द्ध्र॒य॒न्त॒ त्वया॒ त्वया᳚ ऽद्ध्रयन्ता द्ध्रयन्त॒ त्वया᳚ । \newline
25. त्वया॒ प्र प्र त्वया॒ त्वया॒ प्र । \newline
26. प्र जा॑नाम जानाम॒ प्र प्र जा॑नाम । \newline
27. जा॒ना॒मे तीति॑ जानाम जाना॒मेति॑ । \newline
28. इति॒ सा सेतीति॒ सा । \newline
29. सा ऽब्र॑वी दब्रवी॒थ् सा सा ऽब्र॑वीत् । \newline
30. अ॒ब्र॒वी॒द् वरं॒ ॅवर॑ मब्रवी दब्रवी॒द् वर᳚म् । \newline
31. वरं॑ ॅवृणै वृणै॒ वरं॒ ॅवरं॑ ॅवृणै । \newline
32. वृ॒णै॒ मत्प्रा॑यणा॒ मत्प्रा॑यणा वृणै वृणै॒ मत्प्रा॑यणाः । \newline
33. मत्प्रा॑यणा ए॒वैव मत्प्रा॑यणा॒ मत्प्रा॑यणा ए॒व । \newline
34. मत्प्रा॑यणा॒ इति॒ मत् - प्रा॒य॒णाः॒ । \newline
35. ए॒व वो॑ व ए॒वैव वः॑ । \newline
36. वो॒ य॒ज्ञा य॒ज्ञा वो॑ वो य॒ज्ञाः । \newline
37. य॒ज्ञा मदु॑दयना॒ मदु॑दयना य॒ज्ञा य॒ज्ञा मदु॑दयनाः । \newline
38. मदु॑दयना असन् नस॒न् मदु॑दयना॒ मदु॑दयना असन्न् । \newline
39. मदु॑दयना॒ इति॒ मत् - उ॒द॒य॒नाः॒ । \newline
40. अ॒स॒न् निती त्य॑सन् नस॒न् निति॑ । \newline
41. इति॒ तस्मा॒त् तस्मा॒दि तीति॒ तस्मा᳚त् । \newline
42. तस्मा॑ दादि॒त्य आ॑दि॒त्य स्तस्मा॒त् तस्मा॑ दादि॒त्यः । \newline
43. आ॒दि॒त्यः प्रा॑य॒णीयः॑ प्राय॒णीय॑ आदि॒त्य आ॑दि॒त्यः प्रा॑य॒णीयः॑ । \newline
44. प्रा॒य॒णीयो॑ य॒ज्ञानां᳚ ॅय॒ज्ञाना᳚म् प्राय॒णीयः॑ प्राय॒णीयो॑ य॒ज्ञाना᳚म् । \newline
45. प्रा॒य॒णीय॒ इति॑ प्र - अ॒य॒नीयः॑ । \newline
46. य॒ज्ञाना॑ मादि॒त्य आ॑दि॒त्यो य॒ज्ञानां᳚ ॅय॒ज्ञाना॑ मादि॒त्यः । \newline
47. आ॒दि॒त्य उ॑दय॒नीय॑ उदय॒नीय॑ आदि॒त्य आ॑दि॒त्य उ॑दय॒नीयः॑ । \newline
48. उ॒द॒य॒नीयः॒ पञ्च॒ पञ्चो॑ दय॒नीय॑ उदय॒नीयः॒ पञ्च॑ । \newline
49. उ॒द॒य॒नीय॒ इत्यु॑त् - अ॒य॒नीयः॑ । \newline
50. पञ्च॑ दे॒वता॑ दे॒वताः॒ पञ्च॒ पञ्च॑ दे॒वताः᳚ । \newline
51. दे॒वता॑ यजति यजति दे॒वता॑ दे॒वता॑ यजति । \newline
52. य॒ज॒ति॒ पञ्च॒ पञ्च॑ यजति यजति॒ पञ्च॑ । \newline
53. पञ्च॒ दिशो॒ दिशः॒ पञ्च॒ पञ्च॒ दिशः॑ । \newline
54. दिशो॑ दि॒शाम् दि॒शाम् दिशो॒ दिशो॑ दि॒शाम् । \newline
55. दि॒शाम् प्रज्ञा᳚त्यै॒ प्रज्ञा᳚त्यै दि॒शाम् दि॒शाम् प्रज्ञा᳚त्यै । \newline
56. प्रज्ञा᳚त्या॒ अथो॒ अथो॒ प्रज्ञा᳚त्यै॒ प्रज्ञा᳚त्या॒ अथो᳚ । \newline
57. प्रज्ञा᳚त्या॒ इति॒ प्र - ज्ञा॒त्यै॒ । \newline

\textbf{Ghana Paata } \newline

1. दे॒वा वै वै दे॒वा दे॒वा वै दे॑व॒यज॑नम् देव॒यज॑नं॒ ॅवै दे॒वा दे॒वा वै दे॑व॒यज॑नम् । \newline
2. वै दे॑व॒यज॑नम् देव॒यज॑नं॒ ॅवै वै दे॑व॒यज॑न मद्ध्यव॒साया᳚ द्ध्यव॒साय॑ देव॒यज॑नं॒ ॅवै वै दे॑व॒यज॑न मद्ध्यव॒साय॑ । \newline
3. दे॒व॒यज॑न मद्ध्यव॒साया᳚ द्ध्यव॒साय॑ देव॒यज॑नम् देव॒यज॑न मद्ध्यव॒साय॒ दिशो॒ दिशो᳚ ऽद्ध्यव॒साय॑ देव॒यज॑नम् देव॒यज॑न मद्ध्यव॒साय॒ दिशः॑ । \newline
4. दे॒व॒यज॑न॒मिति॑ देव - यज॑नम् । \newline
5. अ॒द्ध्य॒व॒साय॒ दिशो॒ दिशो᳚ ऽद्ध्यव॒साया᳚ द्ध्यव॒साय॒ दिशो॒ न न दिशो᳚ ऽद्ध्यव॒साया᳚ द्ध्यव॒साय॒ दिशो॒ न । \newline
6. अ॒द्ध्य॒व॒सायेत्य॑धि - अ॒व॒साय॑ । \newline
7. दिशो॒ न न दिशो॒ दिशो॒ न प्र प्र ण दिशो॒ दिशो॒ न प्र । \newline
8. न प्र प्र ण न प्राजा॑नन् नजान॒न् प्र ण न प्राजा॑नन्न् । \newline
9. प्राजा॑नन् नजान॒न् प्र प्राजा॑न॒न् ते ते॑ ऽजान॒न् प्र प्राजा॑न॒न् ते । \newline
10. अ॒जा॒न॒न् ते ते॑ ऽजानन् नजान॒न् ते᳚(1॒) ऽन्यो᳚ ऽन्य स्ते॑ ऽजानन् नजान॒न् ते᳚ ऽन्यः । \newline
11. ते᳚(1॒) ऽन्यो᳚ ऽन्य स्ते ते᳚(1॒) ऽन्यो᳚ ऽन्य म॒न्य म॒न्य स्ते ते᳚(1॒) ऽन्यो᳚ ऽन्यम् । \newline
12. अ॒न्यो᳚ ऽन्य म॒न्य म॒न्यो᳚(1॒) ऽन्यो᳚ ऽन्य मुपोपा॒न्य म॒न्यो᳚(1॒) ऽन्यो᳚ ऽन्य मुप॑ । \newline
13. अ॒न्य मुपोपा॒न्य म॒न्य मुपा॑धावन् नधाव॒न् नुपा॒न्य म॒न्य मुपा॑धावन्न् । \newline
14. उपा॑धावन् नधाव॒न् नुपोपा॑धाव॒न् त्वया॒ त्वया॑ ऽधाव॒न् नुपोपा॑धाव॒न् त्वया᳚ । \newline
15. अ॒धा॒व॒न् त्वया॒ त्वया॑ ऽधावन् नधाव॒न् त्वया॒ प्र प्र त्वया॑ ऽधावन् नधाव॒न् त्वया॒ प्र । \newline
16. त्वया॒ प्र प्र त्वया॒ त्वया॒ प्र जा॑नाम जानाम॒ प्र त्वया॒ त्वया॒ प्र जा॑नाम । \newline
17. प्र जा॑नाम जानाम॒ प्र प्र जा॑नाम॒ त्वया॒ त्वया॑ जानाम॒ प्र प्र जा॑नाम॒ त्वया᳚ । \newline
18. जा॒ना॒म॒ त्वया॒ त्वया॑ जानाम जानाम॒ त्वये तीति॒ त्वया॑ जानाम जानाम॒ त्वयेति॑ । \newline
19. त्वये तीति॒ त्वया॒ त्वयेति॒ ते त इति॒ त्वया॒ त्वयेति॒ ते । \newline
20. इति॒ ते त इतीति॒ ते ऽदि॑त्या॒ मदि॑त्या॒म् त इतीति॒ ते ऽदि॑त्याम् । \newline
21. ते ऽदि॑त्या॒ मदि॑त्या॒म् ते ते ऽदि॑त्याꣳ॒॒ सꣳ स मदि॑त्या॒म् ते ते ऽदि॑त्याꣳ॒॒ सम् । \newline
22. अदि॑त्याꣳ॒॒ सꣳ स मदि॑त्या॒ मदि॑त्याꣳ॒॒ स म॑द्ध्रयन्ता द्ध्रयन्त॒ स मदि॑त्या॒ मदि॑त्याꣳ॒॒ स म॑द्ध्रयन्त । \newline
23. स म॑द्ध्रयन्ता द्ध्रयन्त॒ सꣳ स म॑द्ध्रयन्त॒ त्वया॒ त्वया᳚ ऽद्ध्रयन्त॒ सꣳ स म॑द्ध्रयन्त॒ त्वया᳚ । \newline
24. अ॒द्ध्र॒य॒न्त॒ त्वया॒ त्वया᳚ ऽद्ध्रयन्ता द्ध्रयन्त॒ त्वया॒ प्र प्र त्वया᳚ ऽद्ध्रयन्ता द्ध्रयन्त॒ त्वया॒ प्र । \newline
25. त्वया॒ प्र प्र त्वया॒ त्वया॒ प्र जा॑नाम जानाम॒ प्र त्वया॒ त्वया॒ प्र जा॑नाम । \newline
26. प्र जा॑नाम जानाम॒ प्र प्र जा॑ना॒मे तीति॑ जानाम॒ प्र प्र जा॑ना॒मेति॑ । \newline
27. जा॒ना॒मे तीति॑ जानाम जाना॒मेति॒ सा सेति॑ जानाम जाना॒मेति॒ सा । \newline
28. इति॒ सा सेतीति॒ सा ऽब्र॑वी दब्रवी॒थ् सेतीति॒ सा ऽब्र॑वीत् । \newline
29. सा ऽब्र॑वी दब्रवी॒थ् सा सा ऽब्र॑वी॒द् वरं॒ ॅवर॑ मब्रवी॒थ् सा सा ऽब्र॑वी॒द् वर᳚म् । \newline
30. अ॒ब्र॒वी॒द् वरं॒ ॅवर॑ मब्रवी दब्रवी॒द् वरं॑ ॅवृणै वृणै॒ वर॑ मब्रवी दब्रवी॒द् वरं॑ ॅवृणै । \newline
31. वरं॑ ॅवृणै वृणै॒ वरं॒ ॅवरं॑ ॅवृणै॒ मत्प्रा॑यणा॒ मत्प्रा॑यणा वृणै॒ वरं॒ ॅवरं॑ ॅवृणै॒ मत्प्रा॑यणाः । \newline
32. वृ॒णै॒ मत्प्रा॑यणा॒ मत्प्रा॑यणा वृणै वृणै॒ मत्प्रा॑यणा ए॒वैव मत्प्रा॑यणा वृणै वृणै॒ मत्प्रा॑यणा ए॒व । \newline
33. मत्प्रा॑यणा ए॒वैव मत्प्रा॑यणा॒ मत्प्रा॑यणा ए॒व वो॑ व ए॒व मत्प्रा॑यणा॒ मत्प्रा॑यणा ए॒व वः॑ । \newline
34. मत्प्रा॑यणा॒ इति॒ मत् - प्रा॒य॒णाः॒ । \newline
35. ए॒व वो॑ व ए॒वैव वो॑ य॒ज्ञा य॒ज्ञा व॑ ए॒वैव वो॑ य॒ज्ञाः । \newline
36. वो॒ य॒ज्ञा य॒ज्ञा वो॑ वो य॒ज्ञा मदु॑दयना॒ मदु॑दयना य॒ज्ञा वो॑ वो य॒ज्ञा मदु॑दयनाः । \newline
37. य॒ज्ञा मदु॑दयना॒ मदु॑दयना य॒ज्ञा य॒ज्ञा मदु॑दयना असन् नस॒न् मदु॑दयना य॒ज्ञा य॒ज्ञा मदु॑दयना असन्न् । \newline
38. मदु॑दयना असन् नस॒न् मदु॑दयना॒ मदु॑दयना अस॒न् नितीत्य॑स॒न् मदु॑दयना॒ मदु॑दयना अस॒न् निति॑ । \newline
39. मदु॑दयना॒ इति॒ मत् - उ॒द॒य॒नाः॒ । \newline
40. अ॒स॒न् नितीत्य॑सन् नस॒न् निति॒ तस्मा॒त् तस्मा॒ दित्य॑सन् नस॒न् निति॒ तस्मा᳚त् । \newline
41. इति॒ तस्मा॒त् तस्मा॒दि तीति॒ तस्मा॑ दादि॒त्य आ॑दि॒त्य स्तस्मा॒दि तीति॒ तस्मा॑ दादि॒त्यः । \newline
42. तस्मा॑ दादि॒त्य आ॑दि॒त्य स्तस्मा॒त् तस्मा॑ दादि॒त्यः प्रा॑य॒णीयः॑ प्राय॒णीय॑ आदि॒त्य स्तस्मा॒त् तस्मा॑ दादि॒त्यः प्रा॑य॒णीयः॑ । \newline
43. आ॒दि॒त्यः प्रा॑य॒णीयः॑ प्राय॒णीय॑ आदि॒त्य आ॑दि॒त्यः प्रा॑य॒णीयो॑ य॒ज्ञानां᳚ ॅय॒ज्ञाना᳚म् प्राय॒णीय॑ आदि॒त्य आ॑दि॒त्यः प्रा॑य॒णीयो॑ य॒ज्ञाना᳚म् । \newline
44. प्रा॒य॒णीयो॑ य॒ज्ञानां᳚ ॅय॒ज्ञाना᳚म् प्राय॒णीयः॑ प्राय॒णीयो॑ य॒ज्ञाना॑ मादि॒त्य आ॑दि॒त्यो य॒ज्ञाना᳚म् प्राय॒णीयः॑ प्राय॒णीयो॑ य॒ज्ञाना॑ मादि॒त्यः । \newline
45. प्रा॒य॒णीय॒ इति॑ प्र - अ॒य॒नीयः॑ । \newline
46. य॒ज्ञाना॑ मादि॒त्य आ॑दि॒त्यो य॒ज्ञानां᳚ ॅय॒ज्ञाना॑ मादि॒त्य उ॑दय॒नीय॑ उदय॒नीय॑ आदि॒त्यो य॒ज्ञानां᳚ ॅय॒ज्ञाना॑ मादि॒त्य उ॑दय॒नीयः॑ । \newline
47. आ॒दि॒त्य उ॑दय॒नीय॑ उदय॒नीय॑ आदि॒त्य आ॑दि॒त्य उ॑दय॒नीयः॒ पञ्च॒ पञ्चो॑दय॒नीय॑ आदि॒त्य आ॑दि॒त्य उ॑दय॒नीयः॒ पञ्च॑ । \newline
48. उ॒द॒य॒नीयः॒ पञ्च॒ पञ्चो॑ दय॒नीय॑ उदय॒नीयः॒ पञ्च॑ दे॒वता॑ दे॒वताः॒ पञ्चो॑ दय॒नीय॑ उदय॒नीयः॒ पञ्च॑ दे॒वताः᳚ । \newline
49. उ॒द॒य॒नीय॒ इत्यु॑त् - अ॒य॒नीयः॑ । \newline
50. पञ्च॑ दे॒वता॑ दे॒वताः॒ पञ्च॒ पञ्च॑ दे॒वता॑ यजति यजति दे॒वताः॒ पञ्च॒ पञ्च॑ दे॒वता॑ यजति । \newline
51. दे॒वता॑ यजति यजति दे॒वता॑ दे॒वता॑ यजति॒ पञ्च॒ पञ्च॑ यजति दे॒वता॑ दे॒वता॑ यजति॒ पञ्च॑ । \newline
52. य॒ज॒ति॒ पञ्च॒ पञ्च॑ यजति यजति॒ पञ्च॒ दिशो॒ दिशः॒ पञ्च॑ यजति यजति॒ पञ्च॒ दिशः॑ । \newline
53. पञ्च॒ दिशो॒ दिशः॒ पञ्च॒ पञ्च॒ दिशो॑ दि॒शाम् दि॒शाम् दिशः॒ पञ्च॒ पञ्च॒ दिशो॑ दि॒शाम् । \newline
54. दिशो॑ दि॒शाम् दि॒शाम् दिशो॒ दिशो॑ दि॒शाम् प्रज्ञा᳚त्यै॒ प्रज्ञा᳚त्यै दि॒शाम् दिशो॒ दिशो॑ दि॒शाम् प्रज्ञा᳚त्यै । \newline
55. दि॒शाम् प्रज्ञा᳚त्यै॒ प्रज्ञा᳚त्यै दि॒शाम् दि॒शाम् प्रज्ञा᳚त्या॒ अथो॒ अथो॒ प्रज्ञा᳚त्यै दि॒शाम् दि॒शाम् प्रज्ञा᳚त्या॒ अथो᳚ । \newline
56. प्रज्ञा᳚त्या॒ अथो॒ अथो॒ प्रज्ञा᳚त्यै॒ प्रज्ञा᳚त्या॒ अथो॒ पञ्चा᳚क्षरा॒ पञ्चा᳚क्ष॒रा ऽथो॒ प्रज्ञा᳚त्यै॒ प्रज्ञा᳚त्या॒ अथो॒ पञ्चा᳚क्षरा । \newline
57. प्रज्ञा᳚त्या॒ इति॒ प्र - ज्ञा॒त्यै॒ । \newline
\pagebreak
\markright{ TS 6.1.5.2  \hfill https://www.vedavms.in \hfill}

\section{ TS 6.1.5.2 }

\textbf{TS 6.1.5.2 } \newline
\textbf{Samhita Paata} \newline

अथो॒ पञ्चा᳚क्षरा प॒ङ्क्तिः पाङ्क्तो॑ य॒ज्ञो य॒ज्ञ्मे॒वाव॑ रुन्धे॒ पथ्याꣳ॑ स्व॒स्तिम॑यज॒न् प्राची॑मे॒व तया॒ दिशं॒ प्राजा॑नन्न॒ग्निना॑ दक्षि॒णा सोमे॑न प्र॒तीचीꣳ॑ सवि॒त्रोदी॑ची॒-मदि॑त्यो॒र्द्ध्वां पथ्याꣳ॑ स्व॒स्तिं  ॅय॑जति॒ प्राची॑मे॒व तया॒ दिशं॒ प्र जा॑नाति॒ पथ्याꣳ॑ स्व॒स्तिमि॒ष्ट्वाऽग्नीषोमौ॑ यजति॒ चक्षु॑षी॒ वा ए॒ते य॒ज्ञ्स्य॒ यद॒ग्नीषोमौ॒ ताभ्या॑मे॒वानु॑ पश्य - [  ] \newline

\textbf{Pada Paata} \newline

अथो॒ इति॑ । पञ्चा᳚क्ष॒रेति॒ पञ्च॑ - अ॒क्ष॒रा॒ । प॒ङ्क्तिः । पाङ्क्तः॑ । य॒ज्ञ्ः । य॒ज्ञ्म् । ए॒व । अवेति॑ । रु॒न्धे॒ । पथ्या᳚म् । स्व॒स्तिम् । अ॒य॒ज॒न्न् । प्राची᳚म् । ए॒व । तया᳚ । दिश᳚म् । प्रेति॑ । अ॒जा॒न॒न्न् । अ॒ग्निना᳚ । द॒क्षि॒णा । सोमे॑न । प्र॒तीची᳚म् । स॒वि॒त्रा । उदी॑चीम् । अदि॑त्या । ऊ॒द्‌र्ध्वाम् । पथ्या᳚म् । स्व॒स्तिम् । य॒ज॒ति॒ । प्राची᳚म् । ए॒व । तया᳚ । दिश᳚म् । प्रेति॑ । जा॒ना॒ति॒ । पथ्या᳚म् । स्व॒स्तिम् । इ॒ष्ट्वा । अ॒ग्नीषोमा॒वित्य॒ग्नी-सोमौ᳚ । य॒ज॒ति॒ । चक्षु॑षी॒ इति॑ । वै । ए॒ते इति॑ । य॒ज्ञ्स्य॑ । यत् । अ॒ग्नीषोमा॒वित्य॒ग्नी - सोमौ᳚ । ताभ्या᳚म् । ए॒व । अन्विति॑ । प॒श्य॒ति॒ ।  \newline


\textbf{Krama Paata} \newline

अथो॒ पञ्चा᳚क्षरा । अथो॒ इत्यथो᳚ । पञ्चा᳚क्षरा प॒ङ्‍क्तिः । पञ्चा᳚क्ष॒रेति॒ पञ्च॑ - अ॒क्ष॒रा॒ । प॒ङ्‍क्तिः पाङ्‍क्तः॑ । पाङ्‍क्तो॑ य॒ज्ञ्ः । य॒ज्ञो य॒ज्ञ्म् । य॒ज्ञ्मे॒व । ए॒वाव॑ । अव॑ रुन्धे । रु॒न्धे॒ पथ्या᳚म् । पथ्याꣳ॑ स्व॒स्तिम् । स्व॒स्तिम॑यजन्न् । अ॒य॒ज॒न् प्राची᳚म् । प्राची॑मे॒व । ए॒व तया᳚ । तया॒ दिश᳚म् । दिश॒म् प्र । प्राजा॑नन्न् । अ॒जा॒न॒न्न॒ग्निना᳚ । अ॒ग्निना॑ दक्षि॒णा । द॒क्षि॒णा सोमे॑न । सोमे॑न प्र॒तीची᳚म् । प्र॒तीचीꣳ॑ सवि॒त्रा । स॒वि॒त्रोदी॑चीम् । उदी॑ची॒मदि॑त्या । अदि॑त्यो॒र्द्ध्वाम् । ऊ॒र्द्ध्वाम् पथ्या᳚म् । पथ्याꣳ॑ स्व॒स्तिम् । स्व॒स्तिम् ॅय॑जति । य॒ज॒ति॒ प्राची᳚म् । प्राची॑मे॒व । ए॒व तया᳚ । तया॒ दिश᳚म् । दिश॒म् प्र । प्र जा॑नाति । जा॒ना॒ति॒ पथ्या᳚म् । पथ्याꣳ॑ स्व॒स्तिम् । स्व॒स्तिमि॒ष्ट्वा । इ॒ष्ट्वाऽग्नीषोमौ᳚ । अ॒ग्नीषोमौ॑ यजति । अ॒ग्नीषोमा॒वित्य॒ग्नी - सोमौ᳚ । य॒ज॒ति॒ चक्षु॑षी । चक्षु॑षी॒ वै । चक्षु॑षी॒ इति॒ चक्षु॑षी । वा ए॒ते । ए॒ते य॒ज्ञ्स्य॑ । ए॒ते इत्ये॒ते । य॒ज्ञ्स्य॒ यत् । यद॒ग्नीषोमौ᳚ । अ॒ग्नीषोमौ॒ ताभ्या᳚म् । अ॒ग्नीषोमा॒वित्य॒ग्नी - सोमौ᳚ । ताभ्या॑मे॒व । ए॒वानु॑ । अनु॑ पश्यति । प॒श्य॒त्य॒ग्नीषोमौ᳚ \newline

\textbf{Jatai Paata} \newline

1. अथो॒ पञ्चा᳚क्षरा॒ पञ्चा᳚क्ष॒रा ऽथो॒ अथो॒ पञ्चा᳚क्षरा । \newline
2. अथो॒ इत्यथो᳚ । \newline
3. पञ्चा᳚क्षरा प॒ङ्क्तिः प॒ङ्क्तिः पञ्चा᳚क्षरा॒ पञ्चा᳚क्षरा प॒ङ्क्तिः । \newline
4. पञ्चा᳚क्ष॒रेति॒ पञ्च॑ - अ॒क्ष॒रा॒ । \newline
5. प॒ङ्क्तिः पाङ्क्तः॒ पाङ्क्तः॑ प॒ङ्क्तिः प॒ङ्क्तिः पाङ्क्तः॑ । \newline
6. पाङ्क्तो॑ य॒ज्ञो य॒ज्ञ्ः पाङ्क्तः॒ पाङ्क्तो॑ य॒ज्ञ्ः । \newline
7. य॒ज्ञो य॒ज्ञ्ं ॅय॒ज्ञ्ं ॅय॒ज्ञो य॒ज्ञो य॒ज्ञ्म् । \newline
8. य॒ज्ञ् मे॒वैव य॒ज्ञ्ं ॅय॒ज्ञ् मे॒व । \newline
9. ए॒वावा वै॒वै वाव॑ । \newline
10. अव॑ रुन्धे रु॒न्धे ऽवाव॑ रुन्धे । \newline
11. रु॒न्धे॒ पथ्या॒म् पथ्याꣳ॑ रुन्धे रुन्धे॒ पथ्या᳚म् । \newline
12. पथ्याꣳ॑ स्व॒स्तिꣳ स्व॒स्तिम् पथ्या॒म् पथ्याꣳ॑ स्व॒स्तिम् । \newline
13. स्व॒स्ति म॑यजन् नयजन् थ्स्व॒स्तिꣳ स्व॒स्ति म॑यजन्न् । \newline
14. अ॒य॒ज॒न् प्राची॒म् प्राची॑ मयजन् नयज॒न् प्राची᳚म् । \newline
15. प्राची॑ मे॒वैव प्राची॒म् प्राची॑ मे॒व । \newline
16. ए॒व तया॒ तयै॒ वैव तया᳚ । \newline
17. तया॒ दिश॒म् दिश॒म् तया॒ तया॒ दिश᳚म् । \newline
18. दिश॒म् प्र प्र दिश॒म् दिश॒म् प्र । \newline
19. प्राजा॑नन् नजान॒न् प्र प्राजा॑नन्न् । \newline
20. अ॒जा॒न॒न् न॒ग्निना॒ ऽग्निना॑ ऽजानन् नजानन् न॒ग्निना᳚ । \newline
21. अ॒ग्निना॑ दक्षि॒णा द॑क्षि॒णा ऽग्निना॒ ऽग्निना॑ दक्षि॒णा । \newline
22. द॒क्षि॒णा सोमे॑न॒ सोमे॑न दक्षि॒णा द॑क्षि॒णा सोमे॑न । \newline
23. सोमे॑न प्र॒तीची᳚म् प्र॒तीचीꣳ॒॒ सोमे॑न॒ सोमे॑न प्र॒तीची᳚म् । \newline
24. प्र॒तीचीꣳ॑ सवि॒त्रा स॑वि॒त्रा प्र॒तीची᳚म् प्र॒तीचीꣳ॑ सवि॒त्रा । \newline
25. स॒वि॒त्रो दी॑ची॒ मुदी॑चीꣳ सवि॒त्रा स॑वि॒त्रो दी॑चीम् । \newline
26. उदी॑ची॒ मदि॒त्या ऽदि॒त्यो दी॑ची॒ मुदी॑ची॒ मदि॑त्या । \newline
27. अदि॑त्यो॒ र्द्ध्वा मू॒र्द्ध्वा मदि॒त्या ऽदि॑त्यो॒ र्द्ध्वाम् । \newline
28. ऊ॒र्द्ध्वाम् पथ्या॒म् पथ्या॑ मू॒र्द्ध्वा मू॒र्द्ध्वाम् पथ्या᳚म् । \newline
29. पथ्याꣳ॑ स्व॒स्तिꣳ स्व॒स्तिम् पथ्या॒म् पथ्याꣳ॑ स्व॒स्तिम् । \newline
30. स्व॒स्तिं ॅय॑जति यजति स्व॒स्तिꣳ स्व॒स्तिं ॅय॑जति । \newline
31. य॒ज॒ति॒ प्राची॒म् प्राचीं᳚ ॅयजति यजति॒ प्राची᳚म् । \newline
32. प्राची॑ मे॒वैव प्राची॒म् प्राची॑ मे॒व । \newline
33. ए॒व तया॒ तयै॒ वैव तया᳚ । \newline
34. तया॒ दिश॒म् दिश॒म् तया॒ तया॒ दिश᳚म् । \newline
35. दिश॒म् प्र प्र दिश॒म् दिश॒म् प्र । \newline
36. प्र जा॑नाति जानाति॒ प्र प्र जा॑नाति । \newline
37. जा॒ना॒ति॒ पथ्या॒म् पथ्या᳚म् जानाति जानाति॒ पथ्या᳚म् । \newline
38. पथ्याꣳ॑ स्व॒स्तिꣳ स्व॒स्तिम् पथ्या॒म् पथ्याꣳ॑ स्व॒स्तिम् । \newline
39. स्व॒स्ति मि॒ष्ट्वे ष्ट्वा स्व॒स्तिꣳ स्व॒स्ति मि॒ष्ट्वा । \newline
40. इ॒ष्ट्वा ऽग्नीषोमा॑ व॒ग्नीषोमा॑ वि॒ष्ट्वे ष्ट्वा ऽग्नीषोमौ᳚ । \newline
41. अ॒ग्नीषोमौ॑ यजति यजत्य॒ ग्नीषोमा॑ व॒ग्नीषोमौ॑ यजति । \newline
42. अ॒ग्नीषोमा॒वित्य॒ग्नी - सोमौ᳚ । \newline
43. य॒ज॒ति॒ चक्षु॑षी॒ चक्षु॑षी यजति यजति॒ चक्षु॑षी । \newline
44. चक्षु॑षी॒ वै वै चक्षु॑षी॒ चक्षु॑षी॒ वै । \newline
45. चक्षु॑षी॒ इति॒ चक्षु॑षी । \newline
46. वा ए॒ते ए॒ते वै वा ए॒ते । \newline
47. ए॒ते य॒ज्ञ्स्य॑ य॒ज्ञ्स् यै॒ते ए॒ते य॒ज्ञ्स्य॑ । \newline
48. ए॒ते इत्ये॒ते । \newline
49. य॒ज्ञ्स्य॒ यद् यद् य॒ज्ञ्स्य॑ य॒ज्ञ्स्य॒ यत् । \newline
50. यद॒ग्नीषोमा॑ व॒ग्नीषोमौ॒ यद् यद॒ग्नीषोमौ᳚ । \newline
51. अ॒ग्नीषोमौ॒ ताभ्या॒म् ताभ्या॑ म॒ग्नीषोमा॑ व॒ग्नीषोमौ॒ ताभ्या᳚म् । \newline
52. अ॒ग्नीषोमा॒वित्य॒ग्नी - सोमौ᳚ । \newline
53. ताभ्या॑ मे॒वैव ताभ्या॒म् ताभ्या॑ मे॒व । \newline
54. ए॒वान् वन् वे॒वै वानु॑ । \newline
55. अनु॑ पश्यति पश्य॒ त्यन् वनु॑ पश्यति । \newline
56. प॒श्य॒ त्य॒ग्नीषोमा॑ व॒ग्नीषोमौ॑ पश्यति पश्य त्य॒ग्नीषोमौ᳚ । \newline

\textbf{Ghana Paata } \newline

1. अथो॒ पञ्चा᳚क्षरा॒ पञ्चा᳚क्ष॒रा ऽथो॒ अथो॒ पञ्चा᳚क्षरा प॒ङ्क्तिः प॒ङ्क्तिः पञ्चा᳚क्ष॒रा ऽथो॒ अथो॒ पञ्चा᳚क्षरा प॒ङ्क्तिः । \newline
2. अथो॒ इत्यथो᳚ । \newline
3. पञ्चा᳚क्षरा प॒ङ्क्तिः प॒ङ्क्तिः पञ्चा᳚क्षरा॒ पञ्चा᳚क्षरा प॒ङ्क्तिः पाङ्क्तः॒ पाङ्क्तः॑ प॒ङ्क्तिः पञ्चा᳚क्षरा॒ पञ्चा᳚क्षरा प॒ङ्क्तिः पाङ्क्तः॑ । \newline
4. पञ्चा᳚क्ष॒रेति॒ पञ्च॑ - अ॒क्ष॒रा॒ । \newline
5. प॒ङ्क्तिः पाङ्क्तः॒ पाङ्क्तः॑ प॒ङ्क्तिः प॒ङ्क्तिः पाङ्क्तो॑ य॒ज्ञो य॒ज्ञ्ः पाङ्क्तः॑ प॒ङ्क्तिः प॒ङ्क्तिः पाङ्क्तो॑ य॒ज्ञ्ः । \newline
6. पाङ्क्तो॑ य॒ज्ञो य॒ज्ञ्ः पाङ्क्तः॒ पाङ्क्तो॑ य॒ज्ञो य॒ज्ञ्ं ॅय॒ज्ञ्ं ॅय॒ज्ञ्ः पाङ्क्तः॒ पाङ्क्तो॑ य॒ज्ञो य॒ज्ञ्म् । \newline
7. य॒ज्ञो य॒ज्ञ्ं ॅय॒ज्ञ्ं ॅय॒ज्ञो य॒ज्ञो य॒ज्ञ् मे॒वैव य॒ज्ञ्ं ॅय॒ज्ञो य॒ज्ञो य॒ज्ञ् मे॒व । \newline
8. य॒ज्ञ् मे॒वैव य॒ज्ञ्ं ॅय॒ज्ञ् मे॒वा वावै॒व य॒ज्ञ्ं ॅय॒ज्ञ् मे॒वाव॑ । \newline
9. ए॒वावा वै॒वै वाव॑ रुन्धे रु॒न्धे ऽवै॒वै वाव॑ रुन्धे । \newline
10. अव॑ रुन्धे रु॒न्धे ऽवाव॑ रुन्धे॒ पथ्या॒म् पथ्याꣳ॑ रु॒न्धे ऽवाव॑ रुन्धे॒ पथ्या᳚म् । \newline
11. रु॒न्धे॒ पथ्या॒म् पथ्याꣳ॑ रुन्धे रुन्धे॒ पथ्याꣳ॑ स्व॒स्तिꣳ स्व॒स्तिम् पथ्याꣳ॑ रुन्धे रुन्धे॒ पथ्याꣳ॑ स्व॒स्तिम् । \newline
12. पथ्याꣳ॑ स्व॒स्तिꣳ स्व॒स्तिम् पथ्या॒म् पथ्याꣳ॑ स्व॒स्ति म॑यजन् नयजन् थ्स्व॒स्तिम् पथ्या॒म् पथ्याꣳ॑ स्व॒स्ति म॑यजन्न् । \newline
13. स्व॒स्ति म॑यजन् नयजन् थ्स्व॒स्तिꣳ स्व॒स्ति म॑यज॒न् प्राची॒म् प्राची॑ मयजन् थ्स्व॒स्तिꣳ स्व॒स्ति म॑यज॒न् प्राची᳚म् । \newline
14. अ॒य॒ज॒न् प्राची॒म् प्राची॑ मयजन् नयज॒न् प्राची॑ मे॒वैव प्राची॑ मयजन् नयज॒न् प्राची॑ मे॒व । \newline
15. प्राची॑ मे॒वैव प्राची॒म् प्राची॑ मे॒व तया॒ तयै॒व प्राची॒म् प्राची॑ मे॒व तया᳚ । \newline
16. ए॒व तया॒ तयै॒ वैव तया॒ दिश॒म् दिश॒म् तयै॒ वैव तया॒ दिश᳚म् । \newline
17. तया॒ दिश॒म् दिश॒म् तया॒ तया॒ दिश॒म् प्र प्र दिश॒म् तया॒ तया॒ दिश॒म् प्र । \newline
18. दिश॒म् प्र प्र दिश॒म् दिश॒म् प्राजा॑नन् नजान॒न् प्र दिश॒म् दिश॒म् प्राजा॑नन्न् । \newline
19. प्राजा॑नन् नजान॒न् प्र प्राजा॑नन् न॒ग्निना॒ ऽग्निना॑ ऽजान॒न् प्र प्राजा॑नन् न॒ग्निना᳚ । \newline
20. अ॒जा॒न॒न् न॒ग्निना॒ ऽग्निना॑ ऽजानन् नजानन् न॒ग्निना॑ दक्षि॒णा द॑क्षि॒णा ऽग्निना॑ ऽजानन् नजानन् न॒ग्निना॑ दक्षि॒णा । \newline
21. अ॒ग्निना॑ दक्षि॒णा द॑क्षि॒णा ऽग्निना॒ ऽग्निना॑ दक्षि॒णा सोमे॑न॒ सोमे॑न दक्षि॒णा ऽग्निना॒ ऽग्निना॑ दक्षि॒णा सोमे॑न । \newline
22. द॒क्षि॒णा सोमे॑न॒ सोमे॑न दक्षि॒णा द॑क्षि॒णा सोमे॑न प्र॒तीची᳚म् प्र॒तीचीꣳ॒॒ सोमे॑न दक्षि॒णा द॑क्षि॒णा सोमे॑न प्र॒तीची᳚म् । \newline
23. सोमे॑न प्र॒तीची᳚म् प्र॒तीचीꣳ॒॒ सोमे॑न॒ सोमे॑न प्र॒तीचीꣳ॑ सवि॒त्रा स॑वि॒त्रा प्र॒तीचीꣳ॒॒ सोमे॑न॒ सोमे॑न प्र॒तीचीꣳ॑ सवि॒त्रा । \newline
24. प्र॒तीचीꣳ॑ सवि॒त्रा स॑वि॒त्रा प्र॒तीची᳚म् प्र॒तीचीꣳ॑ सवि॒ त्रोदी॑ची॒ मुदी॑चीꣳ सवि॒त्रा प्र॒तीची᳚म् प्र॒तीचीꣳ॑ सवि॒ त्रोदी॑चीम् । \newline
25. स॒वि॒त्रोदी॑ची॒ मुदी॑चीꣳ सवि॒त्रा स॑वि॒ त्रोदी॑ची॒ मदि॒त्या ऽदि॒त्योदी॑चीꣳ सवि॒त्रा स॑वि॒ त्रोदी॑ची॒ मदि॑त्या । \newline
26. उदी॑ची॒ मदि॒त्या ऽदि॒त्यो दी॑ची॒ मुदी॑ची॒ मदि॑त्यो॒ र्द्ध्वा मू॒र्द्ध्वा मदि॒त्यो दी॑ची॒ मुदी॑ची॒ मदि॑त्यो॒ र्द्ध्वाम् । \newline
27. अदि॑त्यो॒ र्द्ध्वा मू॒र्द्ध्वा मदि॒त्या ऽदि॑त्यो॒ र्द्ध्वाम् पथ्या॒म् पथ्या॑ मू॒र्द्ध्वा मदि॒त्या ऽदि॑त्यो॒ र्द्ध्वाम् पथ्या᳚म् । \newline
28. ऊ॒र्द्ध्वाम् पथ्या॒म् पथ्या॑ मू॒र्द्ध्वा मू॒र्द्ध्वाम् पथ्याꣳ॑ स्व॒स्तिꣳ स्व॒स्तिम् पथ्या॑ मू॒र्द्ध्वा मू॒र्द्ध्वाम् पथ्याꣳ॑ स्व॒स्तिम् । \newline
29. पथ्याꣳ॑ स्व॒स्तिꣳ स्व॒स्तिम् पथ्या॒म् पथ्याꣳ॑ स्व॒स्तिं ॅय॑जति यजति स्व॒स्तिम् पथ्या॒म् पथ्याꣳ॑ स्व॒स्तिं ॅय॑जति । \newline
30. स्व॒स्तिं ॅय॑जति यजति स्व॒स्तिꣳ स्व॒स्तिं ॅय॑जति॒ प्राची॒म् प्राचीं᳚ ॅयजति स्व॒स्तिꣳ स्व॒स्तिं ॅय॑जति॒ प्राची᳚म् । \newline
31. य॒ज॒ति॒ प्राची॒म् प्राचीं᳚ ॅयजति यजति॒ प्राची॑ मे॒वैव प्राचीं᳚ ॅयजति यजति॒ प्राची॑ मे॒व । \newline
32. प्राची॑ मे॒वैव प्राची॒म् प्राची॑ मे॒व तया॒ तयै॒व प्राची॒म् प्राची॑ मे॒व तया᳚ । \newline
33. ए॒व तया॒ तयै॒ वैव तया॒ दिश॒म् दिश॒म् तयै॒ वैव तया॒ दिश᳚म् । \newline
34. तया॒ दिश॒म् दिश॒म् तया॒ तया॒ दिश॒म् प्र प्र दिश॒म् तया॒ तया॒ दिश॒म् प्र । \newline
35. दिश॒म् प्र प्र दिश॒म् दिश॒म् प्र जा॑नाति जानाति॒ प्र दिश॒म् दिश॒म् प्र जा॑नाति । \newline
36. प्र जा॑नाति जानाति॒ प्र प्र जा॑नाति॒ पथ्या॒म् पथ्या᳚म् जानाति॒ प्र प्र जा॑नाति॒ पथ्या᳚म् । \newline
37. जा॒ना॒ति॒ पथ्या॒म् पथ्या᳚म् जानाति जानाति॒ पथ्याꣳ॑ स्व॒स्तिꣳ स्व॒स्तिम् पथ्या᳚म् जानाति जानाति॒ पथ्याꣳ॑ स्व॒स्तिम् । \newline
38. पथ्याꣳ॑ स्व॒स्तिꣳ स्व॒स्तिम् पथ्या॒म् पथ्याꣳ॑ स्व॒स्ति मि॒ष्ट्वे ष्ट्वा स्व॒स्तिम् पथ्या॒म् पथ्याꣳ॑ स्व॒स्ति मि॒ष्ट्वा । \newline
39. स्व॒स्ति मि॒ष्ट्वे ष्ट्वा स्व॒स्तिꣳ स्व॒स्ति मि॒ष्ट्वा ऽग्नीषोमा॑ व॒ग्नीषोमा॑ वि॒ष्ट्वा स्व॒स्तिꣳ स्व॒स्ति मि॒ष्ट्वा ऽग्नीषोमौ᳚ । \newline
40. इ॒ष्ट्वा ऽग्नीषोमा॑ व॒ग्नीषोमा॑ वि॒ष्ट्वे ष्ट्वा ऽग्नीषोमौ॑ यजति यज त्य॒ग्नीषोमा॑ वि॒ष्ट्वेष्ट्वा ऽग्नीषोमौ॑ यजति । \newline
41. अ॒ग्नीषोमौ॑ यजति यजत्य॒ ग्नीषोमा॑ व॒ग्नीषोमौ॑ यजति॒ चक्षु॑षी॒ चक्षु॑षी यजत्य॒ ग्नीषोमा॑ व॒ग्नीषोमौ॑ यजति॒ चक्षु॑षी । \newline
42. अ॒ग्नीषोमा॒वित्य॒ग्नी - सोमौ᳚ । \newline
43. य॒ज॒ति॒ चक्षु॑षी॒ चक्षु॑षी यजति यजति॒ चक्षु॑षी॒ वै वै चक्षु॑षी यजति यजति॒ चक्षु॑षी॒ वै । \newline
44. चक्षु॑षी॒ वै वै चक्षु॑षी॒ चक्षु॑षी॒ वा ए॒ते ए॒ते वै चक्षु॑षी॒ चक्षु॑षी॒ वा ए॒ते । \newline
45. चक्षु॑षी॒ इति॒ चक्षु॑षी । \newline
46. वा ए॒ते ए॒ते वै वा ए॒ते य॒ज्ञ्स्य॑ य॒ज्ञ्स्यै॒ते वै वा ए॒ते य॒ज्ञ्स्य॑ । \newline
47. ए॒ते य॒ज्ञ्स्य॑ य॒ज्ञ्स्यै॒ते ए॒ते य॒ज्ञ्स्य॒ यद् यद् य॒ज्ञ्स्यै॒ते ए॒ते य॒ज्ञ्स्य॒ यत् । \newline
48. ए॒ते इत्ये॒ते । \newline
49. य॒ज्ञ्स्य॒ यद् यद् य॒ज्ञ्स्य॑ य॒ज्ञ्स्य॒ यद॒ग्नीषोमा॑ व॒ग्नीषोमौ॒ यद् य॒ज्ञ्स्य॑ य॒ज्ञ्स्य॒ यद॒ग्नीषोमौ᳚ । \newline
50. यद॒ग्नीषोमा॑ व॒ग्नीषोमौ॒ यद् यद॒ग्नीषोमौ॒ ताभ्या॒म् ताभ्या॑ म॒ग्नीषोमौ॒ यद् यद॒ग्नीषोमौ॒ ताभ्या᳚म् । \newline
51. अ॒ग्नीषोमौ॒ ताभ्या॒म् ताभ्या॑ म॒ग्नीषोमा॑ व॒ग्नीषोमौ॒ ताभ्या॑ मे॒वैव ताभ्या॑ म॒ग्नीषोमा॑ व॒ग्नीषोमौ॒ ताभ्या॑ मे॒व । \newline
52. अ॒ग्नीषोमा॒वित्य॒ग्नी - सोमौ᳚ । \newline
53. ताभ्या॑ मे॒वैव ताभ्या॒म् ताभ्या॑ मे॒वान्वन् वे॒व ताभ्या॒म् ताभ्या॑ मे॒वानु॑ । \newline
54. ए॒वान्‌वन् वे॒वैवानु॑ पश्यति पश्य॒ त्यन् वे॒वैवानु॑ पश्यति । \newline
55. अनु॑ पश्यति पश्य॒ त्यन् वनु॑ पश्य त्य॒ग्नीषोमा॑ व॒ग्नीषोमौ॑ पश्य॒ त्यन् वनु॑ पश्य त्य॒ग्नीषोमौ᳚ । \newline
56. प॒श्य॒ त्य॒ग्नीषोमा॑ व॒ग्नीषोमौ॑ पश्यति पश्य त्य॒ग्नीषोमा॑ वि॒ष्ट्वेष्ट्वा ऽग्नीषोमौ॑ पश्यति पश्य त्य॒ग्नीषोमा॑ वि॒ष्ट्वा । \newline
\pagebreak
\markright{ TS 6.1.5.3  \hfill https://www.vedavms.in \hfill}

\section{ TS 6.1.5.3 }

\textbf{TS 6.1.5.3 } \newline
\textbf{Samhita Paata} \newline

-त्य॒ग्नीषोमा॑वि॒ष्ट्वा स॑वि॒तारं॑ ॅयजति सवि॒तृप्र॑सूत ए॒वानु॑ पश्यति सवि॒तार॑मि॒ष्ट्वाऽदि॑तिं ॅयजती॒यं ॅवा अदि॑तिर॒स्यामे॒व प्र॑ति॒ष्ठायानु॑ पश्य॒त्यदि॑तिमि॒ष्ट्वा मा॑रु॒तीमृच॒मन्वा॑ह म॒रुतो॒ वै दे॒वानां॒ ॅविशो॑ देववि॒शां खलु॒ वै कल्प॑मानं मनुष्यवि॒शमनु॑ कल्पते॒ यन् मा॑रु॒तीमृच॑म॒न्वाह॑ वि॒शां क्लृप्त्यै᳚ ब्रह्मवा॒दिनो॑ वदन्ति प्रया॒जव॑दननूया॒जं प्रा॑य॒णीयं॑ का॒र्य॑-मनूया॒जव॑ - [  ] \newline

\textbf{Pada Paata} \newline

अ॒ग्नीषोमा॒वित्य॒ग्नी - सोमौ᳚ । इ॒ष्ट्वा । स॒वि॒तार᳚म् । य॒ज॒ति॒ । स॒वि॒तृप्र॑सूत॒ इति॑ सवि॒तृ - प्र॒सू॒तः॒ । ए॒व । अन्विति॑ । प॒श्य॒ति॒ । स॒वि॒तार᳚म् । इ॒ष्ट्वा । अदि॑तिम् । य॒ज॒ति॒ । इ॒यम् । वै । अदि॑तिः । अ॒स्याम् । ए॒व । प्र॒ति॒ष्ठायेति॑ प्रति - स्थाय॑ । अन्विति॑ । प॒श्य॒ति॒ । अदि॑तिम् । इ॒ष्ट्वा । मा॒रु॒तीम् । ऋच᳚म् । अन्विति॑ । आ॒ह॒ । म॒रुतः॑ । वै । दे॒वाना᳚म् । विशः॑ । दे॒व॒वि॒शमिति॑ देव - वि॒शम् । खलु॑ । वै । कल्प॑मानम् । म॒नु॒ष्य॒वि॒शमिति॑ मनुष्य-वि॒शम् । अन्विति॑ । क॒ल्प॒ते॒ । यत् । मा॒रु॒तीम् । ऋच᳚म् । अ॒न्वाहेत्य॑नु-आह॑ । वि॒शाम् । क्लृप्त्यै᳚ । ब्र॒ह्म॒वा॒दिन॒ इति॑ ब्रह्म - वा॒दिनः॑ । व॒द॒न्ति॒ । प्र॒या॒जव॒दिति॑ प्रया॒ज - व॒त् । अ॒न॒नू॒या॒जमित्य॑ननु - या॒जम् । प्रा॒य॒णीय॒मिति॑ प्र - अ॒य॒णीय᳚म् । का॒र्य᳚म् । अ॒नू॒या॒जव॒दित्य॑नूया॒ज - व॒त् ।  \newline


\textbf{Krama Paata} \newline

अ॒ग्नीषोमा॑वि॒ष्ट्वा । अ॒ग्नीषोमा॒वित्य॒ग्नी - सोमौ᳚ । इ॒ष्ट्वा स॑वि॒तार᳚म् । स॒वि॒तार॑म् ॅयजति । य॒ज॒ति॒ स॒वि॒तृप्र॑सूतः । स॒वि॒तृप्र॑सूत ए॒व । स॒वि॒तृप्र॑सूत॒ इति॑ सवि॒तृ - प्र॒सू॒तः॒ । ए॒वानु॑ । अनु॑ पश्यति । प॒श्य॒ति॒ स॒वि॒तार᳚म् । स॒वि॒तार॑मि॒ष्ट्वा । इ॒ष्ट्वाऽदि॑तिम् । अदि॑तिम् ॅयजति । य॒ज॒ती॒यम् । इ॒यम् ॅवै । वा अदि॑तिः । अदि॑तिर॒स्याम् । अ॒स्यामे॒व । ए॒व प्र॑ति॒ष्ठाय॑ । प्र॒ति॒ष्ठायानु॑ । प्र॒ति॒ष्ठायेति॑ प्रति - स्थाय॑ । अनु॑ पश्यति । प॒श्य॒त्यदि॑तिम् । अदि॑तिमि॒ष्ट्वा । इ॒ष्ट्वा मा॑रु॒तीम् । मा॒रु॒तीमृच᳚म् । ऋच॒मनु॑ । अन्वा॑ह । आ॒ह॒ म॒रुतः॑ । म॒रुतो॒ वै । वै दे॒वाना᳚म् । दे॒वाना॒म् ॅविशः॑ । विशो॑ देववि॒शम् । दे॒व॒वि॒शम् खलु॑ । दे॒व॒वि॒शमिति॑ देव - वि॒शम् । खलु॒ वै । वै कल्प॑मानम् । कल्प॑मानम् मनुष्यवि॒शम् । म॒नु॒ष्य॒वि॒शमनु॑ । म॒नु॒ष्य॒वि॒शमिति॑ मनुष्य - वि॒शम् । अनु॑ कल्पते । क॒ल्प॒ते॒ यत् । यन् मा॑रु॒तीम् । मा॒रु॒तीमृच᳚म् । ऋच॑म॒न्वाह॑ । अ॒न्वाह॑ वि॒शाम् । अ॒न्वाहेत्य॑नु - आह॑ । वि॒शाम् क्लृप्त्यै᳚ । क्लृप्त्यै᳚ ब्रह्मवा॒दिनः॑ । ब्र॒ह्म॒वा॒दिनो॑ वदन्ति । ब्र॒ह्म॒वा॒दिन॒ इति॑ ब्रह्म - वा॒दिनः॑ । व॒द॒न्ति॒ प्र॒या॒जव॑त् । प्र॒या॒जव॑दननूया॒जम् । प्र॒या॒जव॒दिति॑ प्रया॒ज - व॒त्॒ । अ॒न॒नू॒या॒जम् प्रा॑य॒णीय᳚म् । अ॒न॒नू॒या॒जमित्य॑ननु - या॒जम् । प्रा॒य॒णीय॑म् का॒र्य᳚म् । प्रा॒य॒णीय॒मिति॑ प्र - अ॒य॒नीय᳚म् । का॒र्य॑मनूया॒जव॑त् । अ॒नू॒या॒जव॑दप्रया॒जम् । अ॒नू॒या॒जव॒दित्य॑नूया॒ज - व॒त्॒ \newline

\textbf{Jatai Paata} \newline

1. अ॒ग्नीषोमा॑ वि॒ष्ट्वे ष्ट्वा ऽग्नीषोमा॑ व॒ग्नीषोमा॑ वि॒ष्ट्वा । \newline
2. अ॒ग्नीषोमा॒वित्य॒ग्नी - सोमौ᳚ । \newline
3. इ॒ष्ट्वा स॑वि॒तारꣳ॑ सवि॒तार॑ मि॒ष्ट्वे ष्ट्वा स॑वि॒तार᳚म् । \newline
4. स॒वि॒तारं॑ ॅयजति यजति सवि॒तारꣳ॑ सवि॒तारं॑ ॅयजति । \newline
5. य॒ज॒ति॒ स॒वि॒तृप्र॑सूतः सवि॒तृप्र॑सूतो यजति यजति सवि॒तृप्र॑सूतः । \newline
6. स॒वि॒तृप्र॑सूत ए॒वैव स॑वि॒तृप्र॑सूतः सवि॒तृप्र॑सूत ए॒व । \newline
7. स॒वि॒तृप्र॑सूत॒ इति॑ सवि॒तृ - प्र॒सू॒तः॒ । \newline
8. ए॒वान् वन् वे॒वै वानु॑ । \newline
9. अनु॑ पश्यति पश्य॒ त्यन् वनु॑ पश्यति । \newline
10. प॒श्य॒ति॒ स॒वि॒तारꣳ॑ सवि॒तार॑म् पश्यति पश्यति सवि॒तार᳚म् । \newline
11. स॒वि॒तार॑ मि॒ष्ट्वेष्ट्वा स॑वि॒तारꣳ॑ सवि॒तार॑ मि॒ष्ट्वा । \newline
12. इ॒ष्ट्वा ऽदि॑ति॒ मदि॑ति मि॒ष्ट्वे ष्ट्वा ऽदि॑तिम् । \newline
13. अदि॑तिं ॅयजति यज॒ त्यदि॑ति॒ मदि॑तिं ॅयजति । \newline
14. य॒ज॒ ती॒य मि॒यं ॅय॑जति यज ती॒यम् । \newline
15. इ॒यं ॅवै वा इ॒य मि॒यं ॅवै । \newline
16. वा अदि॑ति॒ रदि॑ति॒र् वै वा अदि॑तिः । \newline
17. अदि॑ति र॒स्या म॒स्या मदि॑ति॒ रदि॑ति र॒स्याम् । \newline
18. अ॒स्या मे॒वै वास्या म॒स्या मे॒व । \newline
19. ए॒व प्र॑ति॒ष्ठाय॑ प्रति॒ष्ठायै॒ वैव प्र॑ति॒ष्ठाय॑ । \newline
20. प्र॒ति॒ष्ठा यान् वनु॑ प्रति॒ष्ठाय॑ प्रति॒ष्ठा यानु॑ । \newline
21. प्र॒ति॒ष्ठायेति॑ प्रति - स्थाय॑ । \newline
22. अनु॑ पश्यति पश्य॒ त्यन् वनु॑ पश्यति । \newline
23. प॒श्य॒ त्यदि॑ति॒ मदि॑तिम् पश्यति पश्य॒ त्यदि॑तिम् । \newline
24. अदि॑ति मि॒ष्ट्वे ष्ट्वा ऽदि॑ति॒ मदि॑ति मि॒ष्ट्वा । \newline
25. इ॒ष्ट्वा मा॑रु॒तीम् मा॑रु॒ती मि॒ष्ट्वे ष्ट्वा मा॑रु॒तीम् । \newline
26. मा॒रु॒ती मृच॒ मृच॑म् मारु॒तीम् मा॑रु॒ती मृच᳚म् । \newline
27. ऋच॒ मन् वन् वृच॒ मृच॒ मनु॑ । \newline
28. अन्वा॑हा॒ हान् वन् वा॑ह । \newline
29. आ॒ह॒ म॒रुतो॑ म॒रुत॑ आहाह म॒रुतः॑ । \newline
30. म॒रुतो॒ वै वै म॒रुतो॑ म॒रुतो॒ वै । \newline
31. वै दे॒वाना᳚म् दे॒वानां॒ ॅवै वै दे॒वाना᳚म् । \newline
32. दे॒वानां॒ ॅविशो॒ विशो॑ दे॒वाना᳚म् दे॒वानां॒ ॅविशः॑ । \newline
33. विशो॑ देववि॒शम् दे॑ववि॒शं ॅविशो॒ विशो॑ देववि॒शम् । \newline
34. दे॒व॒वि॒शम् खलु॒ खलु॑ देववि॒शम् दे॑ववि॒शम् खलु॑ । \newline
35. दे॒व॒वि॒शमिति॑ देव - वि॒शम् । \newline
36. खलु॒ वै वै खलु॒ खलु॒ वै । \newline
37. वै कल्प॑मान॒म् कल्प॑मानं॒ ॅवै वै कल्प॑मानम् । \newline
38. कल्प॑मानम् मनुष्यवि॒शम् म॑नुष्यवि॒शम् कल्प॑मान॒म् कल्प॑मानम् मनुष्यवि॒शम् । \newline
39. म॒नु॒ष्य॒वि॒श मन् वनु॑ मनुष्यवि॒शम् म॑नुष्यवि॒श मनु॑ । \newline
40. म॒नु॒ष्य॒वि॒शमिति॑ मनुष्य - वि॒शम् । \newline
41. अनु॑ कल्पते कल्प॒ते ऽन्वनु॑ कल्पते । \newline
42. क॒ल्प॒ते॒ यद् यत् क॑ल्पते कल्पते॒ यत् । \newline
43. यन् मा॑रु॒तीम् मा॑रु॒तीं ॅयद् यन् मा॑रु॒तीम् । \newline
44. मा॒रु॒ती मृच॒ मृच॑म् मारु॒तीम् मा॑रु॒ती मृच᳚म् । \newline
45. ऋच॑ म॒न् वा हा॒न् वाह र्‌च॒ मृच॑ म॒न् वाह॑ । \newline
46. अ॒न्वाह॑ वि॒शां ॅवि॒शा म॒न् वा हा॒न् वाह॑ वि॒शाम् । \newline
47. अ॒न्वाहेत्य॑नु - आह॑ । \newline
48. वि॒शाम् क्लृप्त्यै॒ क्लृप्त्यै॑ वि॒शां ॅवि॒शाम् क्लृप्त्यै᳚ । \newline
49. क्लृप्त्यै᳚ ब्रह्मवा॒दिनो᳚ ब्रह्मवा॒दिनः॒ क्लृप्त्यै॒ क्लृप्त्यै᳚ ब्रह्मवा॒दिनः॑ । \newline
50. ब्र॒ह्म॒वा॒दिनो॑ वदन्ति वदन्ति ब्रह्मवा॒दिनो᳚ ब्रह्मवा॒दिनो॑ वदन्ति । \newline
51. ब्र॒ह्म॒वा॒दिन॒ इति॑ ब्रह्म - वा॒दिनः॑ । \newline
52. व॒द॒न्ति॒ प्र॒या॒जव॑त् प्रया॒जव॑द् वदन्ति वदन्ति प्रया॒जव॑त् । \newline
53. प्र॒या॒जव॑ दननूया॒ज म॑ननूया॒जम् प्र॑या॒जव॑त् प्रया॒जव॑ दननूया॒जम् । \newline
54. प्र॒या॒जव॒दिति॑ प्रया॒ज - व॒त् । \newline
55. अ॒न॒नू॒या॒जम् प्रा॑य॒णीय॑म् प्राय॒णीय॑ मननूया॒ज म॑ननूया॒जम् प्रा॑य॒णीय᳚म् । \newline
56. अ॒न॒नू॒या॒जमित्य॑ननु - या॒जम् । \newline
57. प्रा॒य॒णीय॑म् का॒र्य॑म् का॒र्य॑म् प्राय॒णीय॑म् प्राय॒णीय॑म् का॒र्य᳚म् । \newline
58. प्रा॒य॒णीय॒मिति॑ प्र - अ॒य॒नीय᳚म् । \newline
59. का॒र्य॑ मनूया॒जव॑ दनूया॒जव॑त् का॒र्य॑म् का॒र्य॑ मनूया॒जव॑त् । \newline
60. अ॒नू॒या॒जव॑ दप्रया॒ज म॑प्रया॒ज म॑नूया॒जव॑ दनूया॒जव॑ दप्रया॒जम् । \newline
61. अ॒नू॒या॒जव॒दित्य॑नूया॒ज - व॒त् । \newline

\textbf{Ghana Paata } \newline

1. अ॒ग्नीषोमा॑ वि॒ष्ट्वे ष्ट्वा ऽग्नीषोमा॑ व॒ग्नीषोमा॑ वि॒ष्ट्वा स॑वि॒तारꣳ॑ सवि॒तार॑ मि॒ष्ट्वा ऽग्नीषोमा॑ व॒ग्नीषोमा॑ वि॒ष्ट्वा स॑वि॒तार᳚म् । \newline
2. अ॒ग्नीषोमा॒वित्य॒ग्नी - सोमौ᳚ । \newline
3. इ॒ष्ट्वा स॑वि॒तारꣳ॑ सवि॒तार॑ मि॒ष्ट्वे ष्ट्वा स॑वि॒तारं॑ ॅयजति यजति सवि॒तार॑ मि॒ष्ट्वेष्ट्वा स॑वि॒तारं॑ ॅयजति । \newline
4. स॒वि॒तारं॑ ॅयजति यजति सवि॒तारꣳ॑ सवि॒तारं॑ ॅयजति सवि॒तृप्र॑सूतः सवि॒तृप्र॑सूतो यजति सवि॒तारꣳ॑ सवि॒तारं॑ ॅयजति सवि॒तृप्र॑सूतः । \newline
5. य॒ज॒ति॒ स॒वि॒तृप्र॑सूतः सवि॒तृप्र॑सूतो यजति यजति सवि॒तृप्र॑सूत ए॒वैव स॑वि॒तृप्र॑सूतो यजति यजति सवि॒तृप्र॑सूत ए॒व । \newline
6. स॒वि॒तृप्र॑सूत ए॒वैव स॑वि॒तृप्र॑सूतः सवि॒तृप्र॑सूत ए॒वान् वन् वे॒व स॑वि॒तृप्र॑सूतः सवि॒तृप्र॑सूत ए॒वानु॑ । \newline
7. स॒वि॒तृप्र॑सूत॒ इति॑ सवि॒तृ - प्र॒सू॒तः॒ । \newline
8. ए॒वान् वन्वे॒ वैवानु॑ पश्यति पश्य॒ त्यन्वे॒ वैवानु॑ पश्यति । \newline
9. अनु॑ पश्यति पश्य॒ त्यन्वनु॑ पश्यति सवि॒तारꣳ॑ सवि॒तार॑म् पश्य॒ त्यन्वनु॑ पश्यति सवि॒तार᳚म् । \newline
10. प॒श्य॒ति॒ स॒वि॒तारꣳ॑ सवि॒तार॑म् पश्यति पश्यति सवि॒तार॑ मि॒ष्ट्वेष्ट्वा स॑वि॒तार॑म् पश्यति पश्यति सवि॒तार॑ मि॒ष्ट्वा । \newline
11. स॒वि॒तार॑ मि॒ष्ट्वे ष्ट्वा स॑वि॒तारꣳ॑ सवि॒तार॑ मि॒ष्ट्वा ऽदि॑ति॒ मदि॑ति मि॒ष्ट्वा स॑वि॒तारꣳ॑ सवि॒तार॑ मि॒ष्ट्वा ऽदि॑तिम् । \newline
12. इ॒ष्ट्वा ऽदि॑ति॒ मदि॑ति मि॒ष्ट्वे ष्ट्वा ऽदि॑तिं ॅयजति यज॒ त्यदि॑ति मि॒ष्ट्वे ष्ट्वा ऽदि॑तिं ॅयजति । \newline
13. अदि॑तिं ॅयजति यज॒ त्यदि॑ति॒ मदि॑तिं ॅयजती॒य मि॒यं ॅय॑ज॒ त्यदि॑ति॒ मदि॑तिं ॅयजती॒यम् । \newline
14. य॒ज॒ ती॒य मि॒यं ॅय॑जति यज ती॒यं ॅवै वा इ॒यं ॅय॑जति यज ती॒यं ॅवै । \newline
15. इ॒यं ॅवै वा इ॒य मि॒यं ॅवा अदि॑ति॒ रदि॑ति॒र् वा इ॒य मि॒यं ॅवा अदि॑तिः । \newline
16. वा अदि॑ति॒ रदि॑ति॒र् वै वा अदि॑ति र॒स्या म॒स्या मदि॑ति॒र् वै वा अदि॑ति र॒स्याम् । \newline
17. अदि॑ति र॒स्या म॒स्या मदि॑ति॒ रदि॑ति र॒स्या मे॒वै वास्या मदि॑ति॒ रदि॑ति र॒स्या मे॒व । \newline
18. अ॒स्या मे॒वै वास्या म॒स्या मे॒व प्र॑ति॒ष्ठाय॑ प्रति॒ष्ठायै॒ वास्या म॒स्या मे॒व प्र॑ति॒ष्ठाय॑ । \newline
19. ए॒व प्र॑ति॒ष्ठाय॑ प्रति॒ष्ठायै॒ वैव प्र॑ति॒ष्ठायान् वनु॑ प्रति॒ष्ठायै॒ वैव प्र॑ति॒ष्ठा यानु॑ । \newline
20. प्र॒ति॒ष्ठायान् वनु॑ प्रति॒ष्ठाय॑ प्रति॒ष्ठा यानु॑ पश्यति पश्य॒ त्यनु॑ प्रति॒ष्ठाय॑ प्रति॒ष्ठा यानु॑ पश्यति । \newline
21. प्र॒ति॒ष्ठायेति॑ प्रति - स्थाय॑ । \newline
22. अनु॑ पश्यति पश्य॒ त्यन्वनु॑ पश्य॒ त्यदि॑ति॒ मदि॑तिम् पश्य॒ त्यन् वनु॑ पश्य॒ त्यदि॑तिम् । \newline
23. प॒श्य॒ त्यदि॑ति॒ मदि॑तिम् पश्यति पश्य॒ त्यदि॑ति मि॒ष्ट्वे ष्ट्वा ऽदि॑तिम् पश्यति पश्य॒ त्यदि॑ति मि॒ष्ट्वा । \newline
24. अदि॑ति मि॒ष्ट्वे ष्ट्वा ऽदि॑ति॒ मदि॑ति मि॒ष्ट्वा मा॑रु॒तीम् मा॑रु॒ती मि॒ष्ट्वा ऽदि॑ति॒ मदि॑ति मि॒ष्ट्वा मा॑रु॒तीम् । \newline
25. इ॒ष्ट्वा मा॑रु॒तीम् मा॑रु॒ती मि॒ष्ट्वे ष्ट्वा मा॑रु॒ती मृच॒ मृच॑म् मारु॒ती मि॒ष्ट्वे ष्ट्वा मा॑रु॒ती मृच᳚म् । \newline
26. मा॒रु॒ती मृच॒ मृच॑म् मारु॒तीम् मा॑रु॒ती मृच॒ मन्वन् वृच॑म् मारु॒तीम् मा॑रु॒ती मृच॒ मनु॑ । \newline
27. ऋच॒ मन्वन् वृच॒ मृच॒ मन्वा॑हा॒ हान् वृच॒ मृच॒ मन्वा॑ह । \newline
28. अन्वा॑हा॒ हान् वन् वा॑ह म॒रुतो॑ म॒रुत॑ आ॒हान् वन् वा॑ह म॒रुतः॑ । \newline
29. आ॒ह॒ म॒रुतो॑ म॒रुत॑ आहाह म॒रुतो॒ वै वै म॒रुत॑ आहाह म॒रुतो॒ वै । \newline
30. म॒रुतो॒ वै वै म॒रुतो॑ म॒रुतो॒ वै दे॒वाना᳚म् दे॒वानां॒ ॅवै म॒रुतो॑ म॒रुतो॒ वै दे॒वाना᳚म् । \newline
31. वै दे॒वाना᳚म् दे॒वानां॒ ॅवै वै दे॒वानां॒ ॅविशो॒ विशो॑ दे॒वानां॒ ॅवै वै दे॒वानां॒ ॅविशः॑ । \newline
32. दे॒वानां॒ ॅविशो॒ विशो॑ दे॒वाना᳚म् दे॒वानां॒ ॅविशो॑ देववि॒शम् दे॑ववि॒शं ॅविशो॑ दे॒वाना᳚म् दे॒वानां॒ ॅविशो॑ देववि॒शम् । \newline
33. विशो॑ देववि॒शम् दे॑ववि॒शं ॅविशो॒ विशो॑ देववि॒शम् खलु॒ खलु॑ देववि॒शं ॅविशो॒ विशो॑ देववि॒शम् खलु॑ । \newline
34. दे॒व॒वि॒शम् खलु॒ खलु॑ देववि॒शम् दे॑ववि॒शम् खलु॒ वै वै खलु॑ देववि॒शम् दे॑ववि॒शम् खलु॒ वै । \newline
35. दे॒व॒वि॒शमिति॑ देव - वि॒शम् । \newline
36. खलु॒ वै वै खलु॒ खलु॒ वै कल्प॑मान॒म् कल्प॑मानं॒ ॅवै खलु॒ खलु॒ वै कल्प॑मानम् । \newline
37. वै कल्प॑मान॒म् कल्प॑मानं॒ ॅवै वै कल्प॑मानम् मनुष्यवि॒शम् म॑नुष्यवि॒शम् कल्प॑मानं॒ ॅवै वै कल्प॑मानम् मनुष्यवि॒शम् । \newline
38. कल्प॑मानम् मनुष्यवि॒शम् म॑नुष्यवि॒शम् कल्प॑मान॒म् कल्प॑मानम् मनुष्यवि॒श मन्वनु॑ मनुष्यवि॒शम् कल्प॑मान॒म् कल्प॑मानम् मनुष्यवि॒श मनु॑ । \newline
39. म॒नु॒ष्य॒वि॒श मन्वनु॑ मनुष्यवि॒शम् म॑नुष्यवि॒श मनु॑ कल्पते कल्प॒ते ऽनु॑ मनुष्यवि॒शम् म॑नुष्यवि॒श मनु॑ कल्पते । \newline
40. म॒नु॒ष्य॒वि॒शमिति॑ मनुष्य - वि॒शम् । \newline
41. अनु॑ कल्पते कल्प॒ते ऽन्वनु॑ कल्पते॒ यद् यत् क॑ल्प॒ते ऽन्वनु॑ कल्पते॒ यत् । \newline
42. क॒ल्प॒ते॒ यद् यत् क॑ल्पते कल्पते॒ यन् मा॑रु॒तीम् मा॑रु॒तीं ॅयत् क॑ल्पते कल्पते॒ यन् मा॑रु॒तीम् । \newline
43. यन् मा॑रु॒तीम् मा॑रु॒तीं ॅयद् यन् मा॑रु॒ती मृच॒ मृच॑म् मारु॒तीं ॅयद् यन् मा॑रु॒ती मृच᳚म् । \newline
44. मा॒रु॒ती मृच॒ मृच॑म् मारु॒तीम् मा॑रु॒ती मृच॑ म॒न्वा हा॒न्वाह र्‌च॑म् मारु॒तीम् मा॑रु॒ती मृच॑ म॒न्वाह॑ । \newline
45. ऋच॑ म॒न्वा हा॒न्वाह र्‌च॒ मृच॑ म॒न्वाह॑ वि॒शां ॅवि॒शा म॒न्वाह र्‌च॒ मृच॑ म॒न्वाह॑ वि॒शाम् । \newline
46. अ॒न्वाह॑ वि॒शां ॅवि॒शा म॒न्वा हा॒न्वाह॑ वि॒शाम् क्लृप्त्यै॒ क्लृप्त्यै॑ वि॒शा म॒न्वा हा॒न्वाह॑ वि॒शाम् क्लृप्त्यै᳚ । \newline
47. अ॒न्वाहेत्य॑नु - आह॑ । \newline
48. वि॒शाम् क्लृप्त्यै॒ क्लृप्त्यै॑ वि॒शां ॅवि॒शाम् क्लृप्त्यै᳚ ब्रह्मवा॒दिनो᳚ ब्रह्मवा॒दिनः॒ क्लृप्त्यै॑ वि॒शां ॅवि॒शाम् क्लृप्त्यै᳚ ब्रह्मवा॒दिनः॑ । \newline
49. क्लृप्त्यै᳚ ब्रह्मवा॒दिनो᳚ ब्रह्मवा॒दिनः॒ क्लृप्त्यै॒ क्लृप्त्यै᳚ ब्रह्मवा॒दिनो॑ वदन्ति वदन्ति ब्रह्मवा॒दिनः॒ क्लृप्त्यै॒ क्लृप्त्यै᳚ ब्रह्मवा॒दिनो॑ वदन्ति । \newline
50. ब्र॒ह्म॒वा॒दिनो॑ वदन्ति वदन्ति ब्रह्मवा॒दिनो᳚ ब्रह्मवा॒दिनो॑ वदन्ति प्रया॒जव॑त् प्रया॒जव॑द् वदन्ति ब्रह्मवा॒दिनो᳚ ब्रह्मवा॒दिनो॑ वदन्ति प्रया॒जव॑त् । \newline
51. ब्र॒ह्म॒वा॒दिन॒ इति॑ ब्रह्म - वा॒दिनः॑ । \newline
52. व॒द॒न्ति॒ प्र॒या॒जव॑त् प्रया॒जव॑द् वदन्ति वदन्ति प्रया॒जव॑ दननूया॒ज म॑ननूया॒जम् प्र॑या॒जव॑द् वदन्ति वदन्ति प्रया॒जव॑ दननूया॒जम् । \newline
53. प्र॒या॒जव॑ दननूया॒ज म॑ननूया॒जम् प्र॑या॒जव॑त् प्रया॒जव॑ दननूया॒जम् प्रा॑य॒णीय॑म् प्राय॒णीय॑ मननूया॒जम् प्र॑या॒जव॑त् प्रया॒जव॑ दननूया॒जम् प्रा॑य॒णीय᳚म् । \newline
54. प्र॒या॒जव॒दिति॑ प्रया॒ज - व॒त् । \newline
55. अ॒न॒नू॒या॒जम् प्रा॑य॒णीय॑म् प्राय॒णीय॑ मननूया॒ज म॑ननूया॒जम् प्रा॑य॒णीय॑म् का॒र्य॑म् का॒र्य॑म् प्राय॒णीय॑ मननूया॒ज म॑ननूया॒जम् प्रा॑य॒णीय॑म् का॒र्य᳚म् । \newline
56. अ॒न॒नू॒या॒जमित्य॑ननु - या॒जम् । \newline
57. प्रा॒य॒णीय॑म् का॒र्य॑म् का॒र्य॑म् प्राय॒णीय॑म् प्राय॒णीय॑म् का॒र्य॑ मनूया॒जव॑ दनूया॒जव॑त् का॒र्य॑म् प्राय॒णीय॑म् प्राय॒णीय॑म् का॒र्य॑ मनूया॒जव॑त् । \newline
58. प्रा॒य॒णीय॒मिति॑ प्र - अ॒य॒नीय᳚म् । \newline
59. का॒र्य॑ मनूया॒जव॑ दनूया॒जव॑त् का॒र्य॑म् का॒र्य॑ मनूया॒जव॑ दप्रया॒ज म॑प्रया॒ज म॑नूया॒जव॑त् का॒र्य॑म् का॒र्य॑ मनूया॒जव॑ दप्रया॒जम् । \newline
60. अ॒नू॒या॒जव॑ दप्रया॒ज म॑प्रया॒ज म॑नूया॒जव॑ दनूया॒जव॑ दप्रया॒ज मु॑दय॒नीय॑ मुदय॒नीय॑ मप्रया॒ज म॑नूया॒जव॑ दनूया॒जव॑ दप्रया॒ज मु॑दय॒नीय᳚म् । \newline
61. अ॒नू॒या॒जव॒दित्य॑नूया॒ज - व॒त् । \newline
\pagebreak
\markright{ TS 6.1.5.4  \hfill https://www.vedavms.in \hfill}

\section{ TS 6.1.5.4 }

\textbf{TS 6.1.5.4 } \newline
\textbf{Samhita Paata} \newline

दप्रया॒ज-मु॑दय॒नीय॒मिती॒मे वै प्र॑या॒जा अ॒मी अ॑नूया॒जाः सैव सा य॒ज्ञ्स्य॒ सन्त॑ति॒स्तत् तथा॒ न का॒र्य॑मा॒त्मा वै प्र॑या॒जाः प्र॒जाऽनू॑या॒जा यत् प्र॑या॒जा-न॑न्तरि॒यादा॒त्मान॑म॒-न्तरि॑या॒द्-यद॑नूया॒जा-न॑न्तरि॒यात् प्र॒जाम॒न्तरि॑या॒द्यतः॒ खलु॒ वै य॒ज्ञ्स्य॒ वित॑तस्य॒ न क्रि॒यते॒ तदनु॑ य॒ज्ञ्ः परा॑ भवति य॒ज्ञ्ं प॑रा॒भव॑न्तं॒ ॅयज॑मा॒नोऽनु॒ - [  ] \newline

\textbf{Pada Paata} \newline

अ॒प्र॒या॒जमित्य॑प्र - या॒जम् । उ॒द॒य॒नीय॒मित्यु॑त् - अ॒य॒नीय᳚म् । इति॑ । इ॒मे । वै । प्र॒या॒जा इति॑ प्र - या॒जाः । अ॒मी इति॑ । अ॒नू॒या॒जा इत्य॑नु - या॒जाः । सा । ए॒व । सा । य॒ज्ञ्स्य॑ । सन्त॑ति॒रिति॒ सं - त॒तिः॒ । तत् । तथा᳚ । न । का॒र्य᳚म् । आ॒त्मा । वै । प्र॒या॒जा इति॑ प्र - या॒जाः । प्र॒जेति॑ प्र - जा । अ॒नू॒या॒जा इत्य॑नु-या॒जाः । यत् । प्र॒या॒जानिति॑ प्र-या॒जान् । अ॒न्त॒रि॒यादित्य॑न्तः - इ॒यात् । आ॒त्मान᳚म् । अ॒न्तः । इ॒या॒त् । यत् । अ॒नू॒या॒जानित्य॑नु - या॒जान् । अ॒न्त॒रि॒यादित्य॑न्तः - इ॒यात् । प्र॒जामिति॑ प्र-जाम् । अ॒न्तः । इ॒या॒त् । यतः॑ । खलु॑ । वै । य॒ज्ञ्स्य॑ । वित॑त॒स्येति॒ वि-त॒त॒स्य॒ । न । क्रि॒यते᳚ । तत् । अन्विति॑ । य॒ज्ञ्ः । परेति॑ । भ॒व॒ति॒ । य॒ज्ञ्म् । प॒रा॒भव॑न्त॒मिति॑ परा - भव॑न्तम् । यज॑मानः । अनु॑ ।  \newline


\textbf{Krama Paata} \newline

अ॒प्र॒या॒जमु॑दय॒नीय᳚म् । अ॒प्र॒या॒जमित्य॑प्र - या॒जम् । उ॒द॒य॒नीय॒मिति॑ । उ॒द॒य॒नीय॒मित्यु॑त् - अ॒य॒नीय᳚म् । इती॒मे । इ॒मे वै । वै प्र॑या॒जाः । प्र॒या॒जा अ॒मी । प्र॒या॒जा इति॑ प्र - या॒जाः । अ॒मी अ॑नूया॒जाः । अ॒मी इत्य॒मी । अ॒नू॒या॒जाः सा । अ॒नू॒या॒जा इत्य॑नु - या॒जाः । सैव । ए॒व सा । सा य॒ज्ञ्स्य॑ । य॒ज्ञ्स्य॒ सन्त॑तिः । सन्त॑ति॒स्तत् । सन्त॑ति॒रिति॒ सम् - त॒तिः॒ । तत् तथा᳚ । तथा॒ न । न का॒र्य᳚म् । का॒र्य॑मा॒त्मा । आ॒त्मा वै । वै प्र॑या॒जाः । प्र॒या॒जाः प्र॒जा । प्र॒या॒जा इति॑ प्र - या॒जाः । प्र॒जाऽनू॑या॒जाः । प्र॒जेति॑ प्र - जा । अ॒नू॒या॒जा यत् । अ॒नू॒या॒जा इत्य॑नु - या॒जाः । यत् प्र॑या॒जान् । प्र॒या॒जान॑न्तरि॒यात् । प्र॒या॒जानिति॑ प्र - या॒जान् । अ॒न्त॒रि॒यादा॒त्मान᳚म् । अ॒न्त॒रि॒यादित्य॑न्तः - इ॒यात् । आ॒त्मान॑म॒न्तः । अ॒न्तरि॑यात् । इ॒या॒द् यत् । यद॑नूया॒जान् । अ॒नू॒या॒जान॑न्तरि॒यात् । अ॒नू॒या॒जानित्य॑नु - या॒जान् । अ॒न्त॒रि॒यात् प्र॒जाम् । अ॒न्त॒रि॒यादित्य॑न्तः - इ॒यात् । प्र॒जाम॒न्तः । प्र॒जामिति॑ प्र - जाम् । अ॒न्तरि॑यात् । इ॒या॒द् यतः॑ । यतः॒ खलु॑ । खलु॒ वै । वै य॒ज्ञ्स्य॑ । य॒ज्ञ्स्य॒ वित॑तस्य । वित॑तस्य॒ न । वित॑त॒स्येति॒ वि - त॒त॒स्य॒ । न क्रि॒यते᳚ । क्रि॒यते॒ तत् । तदनु॑ । अनु॑ य॒ज्ञ्ः । य॒ज्ञ्ः परा᳚ । परा॑ भवति । भ॒व॒ति॒ य॒ज्ञ्म् । य॒ज्ञ्म् प॑रा॒भव॑न्तम् । प॒रा॒भव॑न्त॒म् ॅयज॑मानः । प॒रा॒भव॑न्त॒मिति॑ परा - भव॑न्तम् । यज॑मा॒नोऽनु॑ । अनु॒ परा᳚ \newline

\textbf{Jatai Paata} \newline

1. अ॒प्र॒या॒ज मु॑दय॒नीय॑ मुदय॒नीय॑ मप्रया॒ज म॑प्रया॒ज मु॑दय॒नीय᳚म् । \newline
2. अ॒प्र॒या॒जमित्य॑प्र - या॒जम् । \newline
3. उ॒द॒य॒नीय॒ मिती त्यु॑दय॒नीय॑ मुदय॒नीय॒ मिति॑ । \newline
4. उ॒द॒य॒नीय॒मित्यु॑त् - अ॒य॒नीय᳚म् । \newline
5. इती॒म इ॒म इतीती॒मे । \newline
6. इ॒मे वै वा इ॒म इ॒मे वै । \newline
7. वै प्र॑या॒जाः प्र॑या॒जा वै वै प्र॑या॒जाः । \newline
8. प्र॒या॒जा अ॒मी अ॒मी प्र॑या॒जाः प्र॑या॒जा अ॒मी । \newline
9. प्र॒या॒जा इति॑ प्र - या॒जाः । \newline
10. अ॒मी अ॑नूया॒जा अ॑नूया॒जा अ॒मी अ॒मी अ॑नूया॒जाः । \newline
11. अ॒मी इत्य॒मी । \newline
12. अ॒नू॒या॒जाः सा सा ऽनू॑या॒जा अ॑नूया॒जाः सा । \newline
13. अ॒नू॒या॒जा इत्य॑नु - या॒जाः । \newline
14. सैवैव सा सैव । \newline
15. ए॒व सा सैवैव सा । \newline
16. सा य॒ज्ञ्स्य॑ य॒ज्ञ्स्य॒ सा सा य॒ज्ञ्स्य॑ । \newline
17. य॒ज्ञ्स्य॒ सन्त॑तिः॒ सन्त॑तिर् य॒ज्ञ्स्य॑ य॒ज्ञ्स्य॒ सन्त॑तिः । \newline
18. सन्त॑ति॒ स्तत् तथ् सन्त॑तिः॒ सन्त॑ति॒ स्तत् । \newline
19. सन्त॑ति॒रिति॒ सं - त॒तिः॒ । \newline
20. तत् तथा॒ तथा॒ तत् तत् तथा᳚ । \newline
21. तथा॒ न न तथा॒ तथा॒ न । \newline
22. न का॒र्य॑म् का॒र्य॑म् न न का॒र्य᳚म् । \newline
23. का॒र्य॑ मा॒त्मा ऽऽत्मा का॒र्य॑म् का॒र्य॑ मा॒त्मा । \newline
24. आ॒त्मा वै वा आ॒त्मा ऽऽत्मा वै । \newline
25. वै प्र॑या॒जाः प्र॑या॒जा वै वै प्र॑या॒जाः । \newline
26. प्र॒या॒जाः प्र॒जा प्र॒जा प्र॑या॒जाः प्र॑या॒जाः प्र॒जा । \newline
27. प्र॒या॒जा इति॑ प्र - या॒जाः । \newline
28. प्र॒जा ऽनू॑या॒जा अ॑नूया॒जाः प्र॒जा प्र॒जा ऽनू॑या॒जाः । \newline
29. प्र॒जेति॑ प्र - जा । \newline
30. अ॒नू॒या॒जा यद् यद॑नूया॒जा अ॑नूया॒जा यत् । \newline
31. अ॒नू॒या॒जा इत्य॑नु - या॒जाः । \newline
32. यत् प्र॑या॒जान् प्र॑या॒जान्. यद् यत् प्र॑या॒जान् । \newline
33. प्र॒या॒जा न॑न्तरि॒या द॑न्तरि॒यात् प्र॑या॒जान् प्र॑या॒जा न॑न्तरि॒यात् । \newline
34. प्र॒या॒जानिति॑ प्र - या॒जान् । \newline
35. अ॒न्त॒रि॒या दा॒त्मान॑ मा॒त्मान॑ मन्तरि॒या द॑न्तरि॒या दा॒त्मान᳚म् । \newline
36. अ॒न्त॒रि॒यादित्य॑न्तः - इ॒यात् । \newline
37. आ॒त्मान॑ म॒न्त र॒न्त रा॒त्मान॑ मा॒त्मान॑ म॒न्तः । \newline
38. अ॒न्त रि॑या दिया द॒न्त र॒न्त रि॑यात् । \newline
39. इ॒या॒द् यद् यदि॑या दिया॒द् यत् । \newline
40. यद॑नूया॒जा न॑नूया॒जान्. यद् यद॑नूया॒जान् । \newline
41. अ॒नू॒या॒जा न॑न्तरि॒या द॑न्तरि॒या द॑नूया॒जा न॑नूया॒जा न॑न्तरि॒यात् । \newline
42. अ॒नू॒या॒जानित्य॑नु - या॒जान् । \newline
43. अ॒न्त॒रि॒यात् प्र॒जाम् प्र॒जा म॑न्तरि॒या द॑न्तरि॒यात् प्र॒जाम् । \newline
44. अ॒न्त॒रि॒यादित्य॑न्तः - इ॒यात् । \newline
45. प्र॒जा म॒न्त र॒न्तः प्र॒जाम् प्र॒जा म॒न्तः । \newline
46. प्र॒जामिति॑ प्र - जाम् । \newline
47. अ॒न्त रि॑या दिया द॒न्त र॒न्त रि॑यात् । \newline
48. इ॒या॒द् यतो॒ यत॑ इया दिया॒द् यतः॑ । \newline
49. यतः॒ खलु॒ खलु॒ यतो॒ यतः॒ खलु॑ । \newline
50. खलु॒ वै वै खलु॒ खलु॒ वै । \newline
51. वै य॒ज्ञ्स्य॑ य॒ज्ञ्स्य॒ वै वै य॒ज्ञ्स्य॑ । \newline
52. य॒ज्ञ्स्य॒ वित॑तस्य॒ वित॑तस्य य॒ज्ञ्स्य॑ य॒ज्ञ्स्य॒ वित॑तस्य । \newline
53. वित॑तस्य॒ न न वित॑तस्य॒ वित॑तस्य॒ न । \newline
54. वित॑त॒स्येति॒ वि - त॒त॒स्य॒ । \newline
55. न क्रि॒यते᳚ क्रि॒यते॒ न न क्रि॒यते᳚ । \newline
56. क्रि॒यते॒ तत् तत् क्रि॒यते᳚ क्रि॒यते॒ तत् । \newline
57. तदन् वनु॒ तत् तदनु॑ । \newline
58. अनु॑ य॒ज्ञो य॒ज्ञो ऽन्वनु॑ य॒ज्ञ्ः । \newline
59. य॒ज्ञ्ः परा॒ परा॑ य॒ज्ञो य॒ज्ञ्ः परा᳚ । \newline
60. परा॑ भवति भवति॒ परा॒ परा॑ भवति । \newline
61. भ॒व॒ति॒ य॒ज्ञ्ं ॅय॒ज्ञ्म् भ॑वति भवति य॒ज्ञ्म् । \newline
62. य॒ज्ञ्म् प॑रा॒भव॑न्तम् परा॒भव॑न्तं ॅय॒ज्ञ्ं ॅय॒ज्ञ्म् प॑रा॒भव॑न्तम् । \newline
63. प॒रा॒भव॑न्तं॒ ॅयज॑मानो॒ यज॑मानः परा॒भव॑न्तम् परा॒भव॑न्तं॒ ॅयज॑मानः । \newline
64. प॒रा॒भव॑न्त॒मिति॑ परा - भव॑न्तम् । \newline
65. यज॑मा॒नो ऽन्वनु॒ यज॑मानो॒ यज॑मा॒नो ऽनु॑ । \newline
66. अनु॒ परा॒ परा ऽन्वनु॒ परा᳚ । \newline

\textbf{Ghana Paata } \newline

1. अ॒प्र॒या॒ज मु॑दय॒नीय॑ मुदय॒नीय॑ मप्रया॒ज म॑प्रया॒ज मु॑दय॒नीय॒ मिती त्यु॑दय॒नीय॑ मप्रया॒ज म॑प्रया॒ज मु॑दय॒नीय॒ मिति॑ । \newline
2. अ॒प्र॒या॒जमित्य॑प्र - या॒जम् । \newline
3. उ॒द॒य॒नीय॒ मिती त्यु॑दय॒नीय॑ मुदय॒नीय॒ मिती॒म इ॒म इत्यु॑दय॒नीय॑ मुदय॒नीय॒ मिती॒मे । \newline
4. उ॒द॒य॒नीय॒मित्यु॑त् - अ॒य॒नीय᳚म् । \newline
5. इती॒म इ॒म इतीती॒मे वै वा इ॒म इतीती॒मे वै । \newline
6. इ॒मे वै वा इ॒म इ॒मे वै प्र॑या॒जाः प्र॑या॒जा वा इ॒म इ॒मे वै प्र॑या॒जाः । \newline
7. वै प्र॑या॒जाः प्र॑या॒जा वै वै प्र॑या॒जा अ॒मी अ॒मी प्र॑या॒जा वै वै प्र॑या॒जा अ॒मी । \newline
8. प्र॒या॒जा अ॒मी अ॒मी प्र॑या॒जाः प्र॑या॒जा अ॒मी अ॑नूया॒जा अ॑नूया॒जा अ॒मी प्र॑या॒जाः प्र॑या॒जा अ॒मी अ॑नूया॒जाः । \newline
9. प्र॒या॒जा इति॑ प्र - या॒जाः । \newline
10. अ॒मी अ॑नूया॒जा अ॑नूया॒जा अ॒मी अ॒मी अ॑नूया॒जाः सा सा ऽनू॑या॒जा अ॒मी अ॒मी अ॑नूया॒जाः सा । \newline
11. अ॒मी इत्य॒मी । \newline
12. अ॒नू॒या॒जाः सा सा ऽनू॑या॒जा अ॑नूया॒जाः सैवैव सा ऽनू॑या॒जा अ॑नूया॒जाः सैव । \newline
13. अ॒नू॒या॒जा इत्य॑नु - या॒जाः । \newline
14. सैवैव सा सैव सा सैव सा सैव सा । \newline
15. ए॒व सा सैवैव सा य॒ज्ञ्स्य॑ य॒ज्ञ्स्य॒ सैवैव सा य॒ज्ञ्स्य॑ । \newline
16. सा य॒ज्ञ्स्य॑ य॒ज्ञ्स्य॒ सा सा य॒ज्ञ्स्य॒ सन्त॑तिः॒ सन्त॑तिर् य॒ज्ञ्स्य॒ सा सा य॒ज्ञ्स्य॒ सन्त॑तिः । \newline
17. य॒ज्ञ्स्य॒ सन्त॑तिः॒ सन्त॑तिर् य॒ज्ञ्स्य॑ य॒ज्ञ्स्य॒ सन्त॑ति॒ स्तत् तथ् सन्त॑तिर् य॒ज्ञ्स्य॑ य॒ज्ञ्स्य॒ सन्त॑ति॒ स्तत् । \newline
18. सन्त॑ति॒ स्तत् तथ् सन्त॑तिः॒ सन्त॑ति॒ स्तत् तथा॒ तथा॒ तथ् सन्त॑तिः॒ सन्त॑ति॒ स्तत् तथा᳚ । \newline
19. सन्त॑ति॒रिति॒ सं - त॒तिः॒ । \newline
20. तत् तथा॒ तथा॒ तत् तत् तथा॒ न न तथा॒ तत् तत् तथा॒ न । \newline
21. तथा॒ न न तथा॒ तथा॒ न का॒र्य॑म् का॒र्य॑म् न तथा॒ तथा॒ न का॒र्य᳚म् । \newline
22. न का॒र्य॑म् का॒र्य॑म् न न का॒र्य॑ मा॒त्मा ऽऽत्मा का॒र्य॑म् न न का॒र्य॑ मा॒त्मा । \newline
23. का॒र्य॑ मा॒त्मा ऽऽत्मा का॒र्य॑म् का॒र्य॑ मा॒त्मा वै वा आ॒त्मा का॒र्य॑म् का॒र्य॑ मा॒त्मा वै । \newline
24. आ॒त्मा वै वा आ॒त्मा ऽऽत्मा वै प्र॑या॒जाः प्र॑या॒जा वा आ॒त्मा ऽऽत्मा वै प्र॑या॒जाः । \newline
25. वै प्र॑या॒जाः प्र॑या॒जा वै वै प्र॑या॒जाः प्र॒जा प्र॒जा प्र॑या॒जा वै वै प्र॑या॒जाः प्र॒जा । \newline
26. प्र॒या॒जाः प्र॒जा प्र॒जा प्र॑या॒जाः प्र॑या॒जाः प्र॒जा ऽनू॑या॒जा अ॑नूया॒जाः प्र॒जा प्र॑या॒जाः प्र॑या॒जाः प्र॒जा ऽनू॑या॒जाः । \newline
27. प्र॒या॒जा इति॑ प्र - या॒जाः । \newline
28. प्र॒जा ऽनू॑या॒जा अ॑नूया॒जाः प्र॒जा प्र॒जा ऽनू॑या॒जा यद् यद॑नूया॒जाः प्र॒जा प्र॒जा ऽनू॑या॒जा यत् । \newline
29. प्र॒जेति॑ प्र - जा । \newline
30. अ॒नू॒या॒जा यद् यद॑नूया॒जा अ॑नूया॒जा यत् प्र॑या॒जान् प्र॑या॒जान्. यद॑नूया॒जा अ॑नूया॒जा यत् प्र॑या॒जान् । \newline
31. अ॒नू॒या॒जा इत्य॑नु - या॒जाः । \newline
32. यत् प्र॑या॒जान् प्र॑या॒जान्. यद् यत् प्र॑या॒जा न॑न्तरि॒या द॑न्तरि॒यात् प्र॑या॒जान्. यद् यत् प्र॑या॒जा न॑न्तरि॒यात् । \newline
33. प्र॒या॒जा न॑न्तरि॒या द॑न्तरि॒यात् प्र॑या॒जान् प्र॑या॒जा न॑न्तरि॒या दा॒त्मान॑ मा॒त्मान॑ मन्तरि॒यात् प्र॑या॒जान् प्र॑या॒जा न॑न्तरि॒या दा॒त्मान᳚म् । \newline
34. प्र॒या॒जानिति॑ प्र - या॒जान् । \newline
35. अ॒न्त॒रि॒या दा॒त्मान॑ मा॒त्मान॑ मन्तरि॒या द॑न्तरि॒या दा॒त्मान॑ म॒न्त र॒न्त रा॒त्मान॑ मन्तरि॒या द॑न्तरि॒या दा॒त्मान॑ म॒न्तः । \newline
36. अ॒न्त॒रि॒यादित्य॑न्तः - इ॒यात् । \newline
37. आ॒त्मान॑ म॒न्त र॒न्त रा॒त्मान॑ मा॒त्मान॑ म॒न्त रि॑या दिया द॒न्त रा॒त्मान॑ मा॒त्मान॑ म॒न्त रि॑यात् । \newline
38. अ॒न्त रि॑या दिया द॒न्त र॒न्त रि॑या॒द् यद् यदि॑या द॒न्त र॒न्त रि॑या॒द् यत् । \newline
39. इ॒या॒द् यद् यदि॑या दिया॒द् यद॑नूया॒जा न॑नूया॒जान्. यदि॑या दिया॒द् यद॑नूया॒जान् । \newline
40. यद॑नूया॒जा न॑नूया॒जान्. यद् यद॑नूया॒जा न॑न्तरि॒या द॑न्तरि॒या द॑नूया॒जान्. यद् यद॑नूया॒जा न॑न्तरि॒यात् । \newline
41. अ॒नू॒या॒जा न॑न्तरि॒या द॑न्तरि॒या द॑नूया॒जा न॑नूया॒जा न॑न्तरि॒यात् प्र॒जाम् प्र॒जा म॑न्तरि॒या द॑नूया॒जा न॑नूया॒जा न॑न्तरि॒यात् प्र॒जाम् । \newline
42. अ॒नू॒या॒जानित्य॑नु - या॒जान् । \newline
43. अ॒न्त॒रि॒यात् प्र॒जाम् प्र॒जा म॑न्तरि॒या द॑न्तरि॒यात् प्र॒जा म॒न्त र॒न्तः प्र॒जा म॑न्तरि॒या द॑न्तरि॒यात् प्र॒जा म॒न्तः । \newline
44. अ॒न्त॒रि॒यादित्य॑न्तः - इ॒यात् । \newline
45. प्र॒जा म॒न्त र॒न्तः प्र॒जाम् प्र॒जा म॒न्त रि॑या दिया द॒न्तः प्र॒जाम् प्र॒जा म॒न्त रि॑यात् । \newline
46. प्र॒जामिति॑ प्र - जाम् । \newline
47. अ॒न्त रि॑या दिया द॒न्त र॒न्त रि॑या॒द् यतो॒ यत॑ इया द॒न्त र॒न्त रि॑या॒द् यतः॑ । \newline
48. इ॒या॒द् यतो॒ यत॑ इया दिया॒द् यतः॒ खलु॒ खलु॒ यत॑ इया दिया॒द् यतः॒ खलु॑ । \newline
49. यतः॒ खलु॒ खलु॒ यतो॒ यतः॒ खलु॒ वै वै खलु॒ यतो॒ यतः॒ खलु॒ वै । \newline
50. खलु॒ वै वै खलु॒ खलु॒ वै य॒ज्ञ्स्य॑ य॒ज्ञ्स्य॒ वै खलु॒ खलु॒ वै य॒ज्ञ्स्य॑ । \newline
51. वै य॒ज्ञ्स्य॑ य॒ज्ञ्स्य॒ वै वै य॒ज्ञ्स्य॒ वित॑तस्य॒ वित॑तस्य य॒ज्ञ्स्य॒ वै वै य॒ज्ञ्स्य॒ वित॑तस्य । \newline
52. य॒ज्ञ्स्य॒ वित॑तस्य॒ वित॑तस्य य॒ज्ञ्स्य॑ य॒ज्ञ्स्य॒ वित॑तस्य॒ न न वित॑तस्य य॒ज्ञ्स्य॑ य॒ज्ञ्स्य॒ वित॑तस्य॒ न । \newline
53. वित॑तस्य॒ न न वित॑तस्य॒ वित॑तस्य॒ न क्रि॒यते᳚ क्रि॒यते॒ न वित॑तस्य॒ वित॑तस्य॒ न क्रि॒यते᳚ । \newline
54. वित॑त॒स्येति॒ वि - त॒त॒स्य॒ । \newline
55. न क्रि॒यते᳚ क्रि॒यते॒ न न क्रि॒यते॒ तत् तत् क्रि॒यते॒ न न क्रि॒यते॒ तत् । \newline
56. क्रि॒यते॒ तत् तत् क्रि॒यते᳚ क्रि॒यते॒ तदन् वनु॒ तत् क्रि॒यते᳚ क्रि॒यते॒ तदनु॑ । \newline
57. तदन् वनु॒ तत् तदनु॑ य॒ज्ञो य॒ज्ञो ऽनु॒ तत् तदनु॑ य॒ज्ञ्ः । \newline
58. अनु॑ य॒ज्ञो य॒ज्ञो ऽन्वनु॑ य॒ज्ञ्ः परा॒ परा॑ य॒ज्ञो ऽन्वनु॑ य॒ज्ञ्ः परा᳚ । \newline
59. य॒ज्ञ्ः परा॒ परा॑ य॒ज्ञो य॒ज्ञ्ः परा॑ भवति भवति॒ परा॑ य॒ज्ञो य॒ज्ञ्ः परा॑ भवति । \newline
60. परा॑ भवति भवति॒ परा॒ परा॑ भवति य॒ज्ञ्ं ॅय॒ज्ञ्म् भ॑वति॒ परा॒ परा॑ भवति य॒ज्ञ्म् । \newline
61. भ॒व॒ति॒ य॒ज्ञ्ं ॅय॒ज्ञ्म् भ॑वति भवति य॒ज्ञ्म् प॑रा॒भव॑न्तम् परा॒भव॑न्तं ॅय॒ज्ञ्म् भ॑वति भवति य॒ज्ञ्म् प॑रा॒भव॑न्तम् । \newline
62. य॒ज्ञ्म् प॑रा॒भव॑न्तम् परा॒भव॑न्तं ॅय॒ज्ञ्ं ॅय॒ज्ञ्म् प॑रा॒भव॑न्तं॒ ॅयज॑मानो॒ यज॑मानः परा॒भव॑न्तं ॅय॒ज्ञ्ं ॅय॒ज्ञ्म् प॑रा॒भव॑न्तं॒ ॅयज॑मानः । \newline
63. प॒रा॒भव॑न्तं॒ ॅयज॑मानो॒ यज॑मानः परा॒भव॑न्तम् परा॒भव॑न्तं॒ ॅयज॑मा॒नो ऽन्वनु॒ यज॑मानः परा॒भव॑न्तम् परा॒भव॑न्तं॒ ॅयज॑मा॒नो ऽनु॑ । \newline
64. प॒रा॒भव॑न्त॒मिति॑ परा - भव॑न्तम् । \newline
65. यज॑मा॒नो ऽन्वनु॒ यज॑मानो॒ यज॑मा॒नो ऽनु॒ परा॒ परा ऽनु॒ यज॑मानो॒ यज॑मा॒नो ऽनु॒ परा᳚ । \newline
66. अनु॒ परा॒ परा ऽन्वनु॒ परा॑ भवति भवति॒ परा ऽन्वनु॒ परा॑ भवति । \newline
\pagebreak
\markright{ TS 6.1.5.5  \hfill https://www.vedavms.in \hfill}

\section{ TS 6.1.5.5 }

\textbf{TS 6.1.5.5 } \newline
\textbf{Samhita Paata} \newline

परा॑ भवति प्रया॒जव॑दे॒वा-नू॑या॒जव॑त् प्राय॒णीयं॑ का॒र्यं॑ प्रया॒जव॑दनूया॒जव॑-दुदय॒नीयं॒ नाऽऽ*त्मान॑मन्त॒रेति॒ न प्र॒जां न य॒ज्ञ्ः प॑रा॒भव॑ति॒ न यज॑मानः प्राय॒णीय॑स्य निष्का॒स उ॑दय॒नीय॑म॒भि निर्व॑पति॒ सैव सा य॒ज्ञ्स्य॒ सन्त॑ति॒र्याः प्रा॑य॒णीय॑स्य या॒ज्या॑ यत् ता उ॑दय॒नीय॑स्य या॒ज्याः᳚ कु॒र्यात् परा॑ङ॒मुं ॅलो॒कमा रो॑हेत् प्र॒मायु॑कः स्या॒द्याः प्रा॑य॒णीय॑स्य पुरोऽनुवा॒क्या᳚स्ता ( ) उ॑दय॒नीय॑स्य या॒ज्याः᳚ करोत्य॒स्मिन्ने॒व लो॒के प्रति॑ तिष्ठति ॥ \newline

\textbf{Pada Paata} \newline

परेति॑ । भ॒व॒ति॒ । प्र॒या॒जव॒दिति॑ प्रया॒ज - व॒त् । ए॒व । अ॒नू॒या॒जव॒दित्य॑नूया॒ज - व॒त् । प्रा॒य॒णीय॒मिति॑ प्र - अ॒य॒नीय᳚म् । का॒र्य᳚म् । प्र॒या॒जव॒दिति॑ प्रया॒ज - व॒त् । अ॒नू॒या॒जव॒दित्य॑नूया॒ज-व॒त् । उ॒द॒य॒नीय॒मित्यु॑त्-अ॒य॒नीय᳚म् । न । आ॒त्मान᳚म् । अ॒न्त॒रेतीत्य॑न्तः-एति॑ । न । प्र॒जामिति॑ प्र - जाम् । न । य॒ज्ञ्ः । प॒रा॒भव॒तीति॑ परा - भव॑ति । न । यज॑मानः । प्रा॒य॒णीय॒स्येति॑ प्र - अ॒य॒नीय॑स्य । नि॒ष्का॒से । उ॒द॒य॒नीय॒मित्यु॑त् - अ॒य॒नीय᳚म् । अ॒भि । निरिति॑ । व॒प॒ति॒ । सा । ए॒व । सा । य॒ज्ञ्स्य॑ । सन्त॑ति॒रिति॒ सं - त॒तिः॒ । याः । प्रा॒य॒णीय॒स्येति॑ प्र - अ॒य॒नीय॑स्य । या॒ज्याः᳚ । यत् । ताः । उ॒द॒य॒नीय॒स्येत्यु॑त्-अ॒य॒नीय॑स्य । या॒ज्याः᳚ । कु॒र्यात् । पराङ्॑ । अ॒मुम् । लो॒कम् । एति॑ । रो॒हे॒त् । प्र॒मायु॑क॒ इति॑ प्र - मायु॑कः । स्या॒त् । याः । प्रा॒य॒णीय॒स्येति॑ प्र - अ॒य॒नीय॑स्य । पु॒रो॒नु॒वा॒क्या॑ इति॑ पुरः - अ॒नु॒वा॒क्याः᳚ । ताः ( ) । उ॒द॒य॒नीय॒स्येत्यु॑त् - अ॒य॒नीय॑स्य । या॒ज्याः᳚ । क॒रो॒ति॒ । अ॒स्मिन्न् । ए॒व । लो॒के । प्रतीति॑ । ति॒ष्ठ॒ति॒ ॥  \newline


\textbf{Krama Paata} \newline

परा॑ भवति । भ॒व॒ति॒ प्र॒या॒जव॑त् । प्र॒या॒जव॑दे॒व । प्र॒या॒जव॒दिति॑ प्रया॒ज - व॒त्॒ । ए॒वानू॑या॒जव॑त् । अ॒नू॒या॒जव॑त् प्राय॒णीय᳚म् । अ॒नू॒या॒जव॒दित्य॑नूया॒ज - व॒त्॒ । प्रा॒य॒णीय॑म् का॒र्य᳚म् । प्रा॒य॒णीय॒मिति॑ प्र - अ॒य॒नीय᳚म् । का॒र्य॑म् प्रया॒जव॑त् । प्र॒या॒जव॑दनूया॒जव॑त् । प्र॒या॒जव॒दिति॑ प्रया॒ज - व॒त्॒ । अ॒नू॒या॒जव॑दुदय॒नीय᳚म् । अ॒नू॒या॒जव॒दित्य॑नूया॒ज - व॒त्॒ । उ॒द॒य॒नीय॒म् न । उ॒द॒य॒नीय॒मित्यु॑त् - अ॒य॒नीय᳚म् । नात्मान᳚म् । आ॒त्मान॑मन्त॒रेति॑ । अ॒न्त॒रेति॒ न । अ॒न्त॒रेतीत्य॑न्तः - एति॑ । न प्र॒जाम् । प्र॒जाम् न । प्र॒जामिति॑ प्र - जाम् । न य॒ज्ञ्ः । य॒ज्ञ्ः प॑रा॒भव॑ति । प॒रा॒भव॑ति॒ न । प॒रा॒भव॒तीति॑ परा - भव॑ति । न यज॑मानः । यज॑मानः प्राय॒णीय॑स्य । प्रा॒य॒णीय॑स्य निष्का॒से । प्रा॒य॒णीय॒स्येति॑ प्र - अ॒य॒नीय॑स्य । नि॒ष्का॒स उ॑दय॒नीय᳚म् । उ॒द॒य॒नीय॑म॒भि । उ॒द॒य॒नीय॒मित्यु॑त् - अ॒य॒नीय᳚म् । अ॒भि निः । निर् व॑पति । व॒प॒ति॒ सा । सैव । ए॒व सा । सा य॒ज्ञ्स्य॑ । य॒ज्ञ्स्य॒ सन्त॑तिः । सन्त॑ति॒र् याः । सन्त॑ति॒रिति॒ सम् - त॒तिः॒ । याः प्रा॑य॒णीय॑स्य । प्रा॒य॒णीय॑स्य या॒ज्या᳚ । प्रा॒य॒णीय॒स्येति॑ प्र - अ॒य॒नीय॑स्य । या॒ज्या॑ यत् । यत् ताः । ता उ॑दय॒नीय॑स्य । उ॒द॒य॒नीय॑स्य या॒ज्याः᳚ । उ॒द॒य॒नीय॒स्येत्यु॑त् - अ॒य॒नीय॑स्य । या॒ज्याः᳚ कु॒र्यात् । कु॒र्यात् पराङ्‍॑ । परा॑ङ॒मुम् । अ॒मुम् ॅलो॒कम् । लो॒कमा । आ रो॑हेत् । रो॒हे॒त् प्र॒मायु॑कः । प्र॒मायु॑कः स्यात् । प्र॒मायु॑क॒ इति॑ प्र - मायु॑कः । स्या॒द् याः । याः प्रा॑य॒णीय॑स्य । प्रा॒य॒णीय॑स्य पुरोनुवा॒क्याः᳚ । प्रा॒य॒णीय॒स्येति॑ प्र - अ॒य॒नीय॑स्य । पु॒रो॒नु॒वा॒क्या᳚स्ताः ( ) । पु॒रो॒नु॒वा॒क्या॑ इति॑ पुरः - अ॒नु॒वा॒क्याः᳚ । ता उ॑दय॒नीय॑स्य । उ॒द॒य॒नीय॑स्य या॒ज्याः᳚ । उ॒द॒य॒नीय॒स्येत्यु॑त् - अ॒य॒नीय॑स्य । या॒ज्याः᳚ करोति । क॒रो॒त्य॒स्मिन्न् । अ॒स्मिन्ने॒व । ए॒व लो॒के । लो॒के प्रति॑ । प्रति॑ तिष्ठति । ति॒ष्ठ॒तीति॑ तिष्ठति । \newline

\textbf{Jatai Paata} \newline

1. परा॑ भवति भवति॒ परा॒ परा॑ भवति । \newline
2. भ॒व॒ति॒ प्र॒या॒जव॑त् प्रया॒जव॑द् भवति भवति प्रया॒जव॑त् । \newline
3. प्र॒या॒जव॑ दे॒वैव प्र॑या॒जव॑त् प्रया॒जव॑ दे॒व । \newline
4. प्र॒या॒जव॒दिति॑ प्रया॒ज - व॒त् । \newline
5. ए॒वानू॑या॒जव॑ दनूया॒जव॑ दे॒वै वानू॑या॒जव॑त् । \newline
6. अ॒नू॒या॒जव॑त् प्राय॒णीय॑म् प्राय॒णीय॑ मनूया॒जव॑ दनूया॒जव॑त् प्राय॒णीय᳚म् । \newline
7. अ॒नू॒या॒जव॒दित्य॑नूया॒ज - व॒त् । \newline
8. प्रा॒य॒णीय॑म् का॒र्य॑म् का॒र्य॑म् प्राय॒णीय॑म् प्राय॒णीय॑म् का॒र्य᳚म् । \newline
9. प्रा॒य॒णीय॒मिति॑ प्र - अ॒य॒नीय᳚म् । \newline
10. का॒र्य॑म् प्रया॒जव॑त् प्रया॒जव॑त् का॒र्य॑म् का॒र्य॑म् प्रया॒जव॑त् । \newline
11. प्र॒या॒जव॑ दनूया॒जव॑ दनूया॒जव॑त् प्रया॒जव॑त् प्रया॒जव॑ दनूया॒जव॑त् । \newline
12. प्र॒या॒जव॒दिति॑ प्रया॒ज - व॒त् । \newline
13. अ॒नू॒या॒जव॑ दुदय॒नीय॑ मुदय॒नीय॑ मनूया॒जव॑ दनूया॒जव॑ दुदय॒नीय᳚म् । \newline
14. अ॒नू॒या॒जव॒दित्य॑नूया॒ज - व॒त् । \newline
15. उ॒द॒य॒नीय॒न् न नोद॑य॒नीय॑ मुदय॒नीय॒न् न । \newline
16. उ॒द॒य॒नीय॒मित्यु॑त् - अ॒य॒नीय᳚म् । \newline
17. नात्मान॑ मा॒त्मान॒न् न नात्मान᳚म् । \newline
18. आ॒त्मान॑ मन्त॒रे त्य॑न्त॒रे त्या॒त्मान॑ मा॒त्मान॑ मन्त॒रेति॑ । \newline
19. अ॒न्त॒रेति॒ न नान्त॒रे त्य॑न्त॒रेति॒ न । \newline
20. अ॒न्त॒रेतीत्य॑न्तः - एति॑ । \newline
21. न प्र॒जाम् प्र॒जान् न न प्र॒जाम् । \newline
22. प्र॒जान् न न प्र॒जाम् प्र॒जान् न । \newline
23. प्र॒जामिति॑ प्र - जाम् । \newline
24. न य॒ज्ञो य॒ज्ञो न न य॒ज्ञ्ः । \newline
25. य॒ज्ञ्ः प॑रा॒भव॑ति परा॒भव॑ति य॒ज्ञो य॒ज्ञ्ः प॑रा॒भव॑ति । \newline
26. प॒रा॒भव॑ति॒ न न प॑रा॒भव॑ति परा॒भव॑ति॒ न । \newline
27. प॒रा॒भव॒तीति॑ परा - भव॑ति । \newline
28. न यज॑मानो॒ यज॑मानो॒ न न यज॑मानः । \newline
29. यज॑मानः प्राय॒णीय॑स्य प्राय॒णीय॑स्य॒ यज॑मानो॒ यज॑मानः प्राय॒णीय॑स्य । \newline
30. प्रा॒य॒णीय॑स्य निष्का॒से नि॑ष्का॒से प्रा॑य॒णीय॑स्य प्राय॒णीय॑स्य निष्का॒से । \newline
31. प्रा॒य॒णीय॒स्येति॑ प्र - अ॒य॒नीय॑स्य । \newline
32. नि॒ष्का॒स उ॑दय॒नीय॑ मुदय॒नीय॑म् निष्का॒से नि॑ष्का॒स उ॑दय॒नीय᳚म् । \newline
33. उ॒द॒य॒नीय॑ म॒भ्या᳚(1॒)भ्यु॑दय॒नीय॑ मुदय॒नीय॑ म॒भि । \newline
34. उ॒द॒य॒नीय॒मित्यु॑त् - अ॒य॒नीय᳚म् । \newline
35. अ॒भि निर् णिर॒ भ्य॑भि निः । \newline
36. निर् व॑पति वपति॒ निर् णिर् व॑पति । \newline
37. व॒प॒ति॒ सा सा व॑पति वपति॒ सा । \newline
38. सैवैव सा सैव । \newline
39. ए॒व सा सैवैव सा । \newline
40. सा य॒ज्ञ्स्य॑ य॒ज्ञ्स्य॒ सा सा य॒ज्ञ्स्य॑ । \newline
41. य॒ज्ञ्स्य॒ सन्त॑तिः॒ सन्त॑तिर् य॒ज्ञ्स्य॑ य॒ज्ञ्स्य॒ सन्त॑तिः । \newline
42. सन्त॑ति॒र् या याः सन्त॑तिः॒ सन्त॑ति॒र् याः । \newline
43. सन्त॑ति॒रिति॒ सं - त॒तिः॒ । \newline
44. याः प्रा॑य॒णीय॑स्य प्राय॒णीय॑स्य॒ या याः प्रा॑य॒णीय॑स्य । \newline
45. प्रा॒य॒णीय॑स्य या॒ज्या॑ या॒ज्याः᳚ प्राय॒णीय॑स्य प्राय॒णीय॑स्य या॒ज्याः᳚ । \newline
46. प्रा॒य॒णीय॒स्येति॑ प्र - अ॒य॒नीय॑स्य । \newline
47. या॒ज्या॑ यद् यद् या॒ज्या॑ या॒ज्या॑ यत् । \newline
48. यत् ता स्ता यद् यत् ताः । \newline
49. ता उ॑दय॒नीय॑ स्योदय॒नीय॑स्य॒ तास्ता उ॑दय॒नीय॑स्य । \newline
50. उ॒द॒य॒नीय॑स्य या॒ज्या॑ या॒ज्या॑ उदय॒नीय॑ स्योदय॒नीय॑स्य या॒ज्याः᳚ । \newline
51. उ॒द॒य॒नीय॒स्येत्यु॑त् - अ॒य॒नीय॑स्य । \newline
52. या॒ज्याः᳚ कु॒र्यात् कु॒र्याद् या॒ज्या॑ या॒ज्याः᳚ कु॒र्यात् । \newline
53. कु॒र्यात् परा॒ङ् परा᳚ङ् कु॒र्यात् कु॒र्यात् पराङ्॑ । \newline
54. परा॑ङ॒मु म॒मुम् परा॒ङ् परा॑ङ॒मुम् । \newline
55. अ॒मुम् ॅलो॒कम् ॅलो॒क म॒मु म॒मुम् ॅलो॒कम् । \newline
56. लो॒क मा लो॒कम् ॅलो॒क मा । \newline
57. आ रो॑हेद् रोहे॒दा रो॑हेत् । \newline
58. रो॒हे॒त् प्र॒मायु॑कः प्र॒मायु॑को रोहेद् रोहेत् प्र॒मायु॑कः । \newline
59. प्र॒मायु॑कः स्याथ् स्यात् प्र॒मायु॑कः प्र॒मायु॑कः स्यात् । \newline
60. प्र॒मायु॑क॒ इति॑ प्र - मायु॑कः । \newline
61. स्या॒द् या याः स्या᳚थ् स्या॒द् याः । \newline
62. याः प्रा॑य॒णीय॑स्य प्राय॒णीय॑स्य॒ या याः प्रा॑य॒णीय॑स्य । \newline
63. प्रा॒य॒णीय॑स्य पुरोनुवा॒क्याः᳚ पुरोनुवा॒क्याः᳚ प्राय॒णीय॑स्य प्राय॒णीय॑स्य पुरोनुवा॒क्याः᳚ । \newline
64. प्रा॒य॒णीय॒स्येति॑ प्र - अ॒य॒नीय॑स्य । \newline
65. पु॒रो॒नु॒वा॒क्या᳚ स्ता स्ताः पु॑रोनुवा॒क्याः᳚ पुरोनुवा॒क्या᳚ स्ताः । \newline
66. पु॒रो॒नु॒वा॒क्या॑ इति॑ पुरः - अ॒नु॒वा॒क्याः᳚ । \newline
67. ता उ॑दय॒नीय॑ स्योदय॒नीय॑स्य॒ ता स्ता उ॑दय॒नीय॑स्य । \newline
68. उ॒द॒य॒नीय॑स्य या॒ज्या॑ या॒ज्या॑ उदय॒नीय॑ स्योदय॒नीय॑स्य या॒ज्याः᳚ । \newline
69. उ॒द॒य॒नीय॒स्येत्यु॑त् - अ॒य॒नीय॑स्य । \newline
70. या॒ज्याः᳚ करोति करोति या॒ज्या॑ या॒ज्याः᳚ करोति । \newline
71. क॒रो॒ त्य॒स्मिन् न॒स्मिन् क॑रोति करो त्य॒स्मिन्न् । \newline
72. अ॒स्मिन् ए॒वै वास्मिन् न॒स्मिन् ने॒व । \newline
73. ए॒व लो॒के लो॒क ए॒वैव लो॒के । \newline
74. लो॒के प्रति॒ प्रति॑ लो॒के लो॒के प्रति॑ । \newline
75. प्रति॑ तिष्ठति तिष्ठति॒ प्रति॒ प्रति॑ तिष्ठति । \newline
76. ति॒ष्ठ॒तीति॑ तिष्ठति । \newline

\textbf{Ghana Paata } \newline

1. परा॑ भवति भवति॒ परा॒ परा॑ भवति प्रया॒जव॑त् प्रया॒जव॑द् भवति॒ परा॒ परा॑ भवति प्रया॒जव॑त् । \newline
2. भ॒व॒ति॒ प्र॒या॒जव॑त् प्रया॒जव॑द् भवति भवति प्रया॒जव॑ दे॒वैव प्र॑या॒जव॑द् भवति भवति प्रया॒जव॑ दे॒व । \newline
3. प्र॒या॒जव॑ दे॒वैव प्र॑या॒जव॑त् प्रया॒जव॑ दे॒वा नू॑या॒जव॑ दनूया॒जव॑ दे॒व प्र॑या॒जव॑त् प्रया॒जव॑ दे॒वा नू॑या॒जव॑त् । \newline
4. प्र॒या॒जव॒दिति॑ प्रया॒ज - व॒त् । \newline
5. ए॒वा नू॑या॒जव॑ दनूया॒जव॑ दे॒वैवा नू॑या॒जव॑त् प्राय॒णीय॑म् प्राय॒णीय॑ मनूया॒जव॑ दे॒वैवा नू॑या॒जव॑त् प्राय॒णीय᳚म् । \newline
6. अ॒नू॒या॒जव॑त् प्राय॒णीय॑म् प्राय॒णीय॑ मनूया॒जव॑ दनूया॒जव॑त् प्राय॒णीय॑म् का॒र्य॑म् का॒र्य॑म् प्राय॒णीय॑ मनूया॒जव॑ दनूया॒जव॑त् प्राय॒णीय॑म् का॒र्य᳚म् । \newline
7. अ॒नू॒या॒जव॒दित्य॑नूया॒ज - व॒त् । \newline
8. प्रा॒य॒णीय॑म् का॒र्य॑म् का॒र्य॑म् प्राय॒णीय॑म् प्राय॒णीय॑म् का॒र्य॑म् प्रया॒जव॑त् प्रया॒जव॑त् का॒र्य॑म् प्राय॒णीय॑म् प्राय॒णीय॑म् का॒र्य॑म् प्रया॒जव॑त् । \newline
9. प्रा॒य॒णीय॒मिति॑ प्र - अ॒य॒नीय᳚म् । \newline
10. का॒र्य॑म् प्रया॒जव॑त् प्रया॒जव॑त् का॒र्य॑म् का॒र्य॑म् प्रया॒जव॑ दनूया॒जव॑ दनूया॒जव॑त् प्रया॒जव॑त् का॒र्य॑म् का॒र्य॑म् प्रया॒जव॑ दनूया॒जव॑त् । \newline
11. प्र॒या॒जव॑ दनूया॒जव॑ दनूया॒जव॑त् प्रया॒जव॑त् प्रया॒जव॑ दनूया॒जव॑ दुदय॒नीय॑ मुदय॒नीय॑ मनूया॒जव॑त् प्रया॒जव॑त् प्रया॒जव॑ दनूया॒जव॑ दुदय॒नीय᳚म् । \newline
12. प्र॒या॒जव॒दिति॑ प्रया॒ज - व॒त् । \newline
13. अ॒नू॒या॒जव॑ दुदय॒नीय॑ मुदय॒नीय॑ मनूया॒जव॑ दनूया॒जव॑ दुदय॒नीय॒न् न नोद॑य॒नीय॑ मनूया॒जव॑ दनूया॒जव॑ दुदय॒नीय॒न् न । \newline
14. अ॒नू॒या॒जव॒दित्य॑नूया॒ज - व॒त् । \newline
15. उ॒द॒य॒नीय॒म् न नोद॑य॒नीय॑ मुदय॒नीय॒म् नात्मान॑ मा॒त्मान॒म् नोद॑य॒नीय॑ मुदय॒नीय॒म् नात्मान᳚म् । \newline
16. उ॒द॒य॒नीय॒मित्यु॑त् - अ॒य॒नीय᳚म् । \newline
17. नात्मान॑ मा॒त्मान॒न् न नात्मान॑ मन्त॒रे त्य॑न्त॒रे त्या॒त्मान॒न् न नात्मान॑ मन्त॒रेति॑ । \newline
18. आ॒त्मान॑ मन्त॒रे त्य॑न्त॒रे त्या॒त्मान॑ मा॒त्मान॑ मन्त॒रेति॒ न नान्त॒रे त्या॒त्मान॑ मा॒त्मान॑ मन्त॒रेति॒ न । \newline
19. अ॒न्त॒रेति॒ न नान्त॒रे त्य॑न्त॒रेति॒ न प्र॒जाम् प्र॒जाम् नान्त॒रे त्य॑न्त॒रेति॒ न प्र॒जाम् । \newline
20. अ॒न्त॒रेतीत्य॑न्तः - एति॑ । \newline
21. न प्र॒जाम् प्र॒जान् न न प्र॒जान् न न प्र॒जान् न न प्र॒जान् न । \newline
22. प्र॒जान् न न प्र॒जाम् प्र॒जान् न य॒ज्ञो य॒ज्ञो न प्र॒जाम् प्र॒जान् न य॒ज्ञ्ः । \newline
23. प्र॒जामिति॑ प्र - जाम् । \newline
24. न य॒ज्ञो य॒ज्ञो न न य॒ज्ञ्ः प॑रा॒भव॑ति परा॒भव॑ति य॒ज्ञो न न य॒ज्ञ्ः प॑रा॒भव॑ति । \newline
25. य॒ज्ञ्ः प॑रा॒भव॑ति परा॒भव॑ति य॒ज्ञो य॒ज्ञ्ः प॑रा॒भव॑ति॒ न न प॑रा॒भव॑ति य॒ज्ञो य॒ज्ञ्ः प॑रा॒भव॑ति॒ न । \newline
26. प॒रा॒भव॑ति॒ न न प॑रा॒भव॑ति परा॒भव॑ति॒ न यज॑मानो॒ यज॑मानो॒ न प॑रा॒भव॑ति परा॒भव॑ति॒ न यज॑मानः । \newline
27. प॒रा॒भव॒तीति॑ परा - भव॑ति । \newline
28. न यज॑मानो॒ यज॑मानो॒ न न यज॑मानः प्राय॒णीय॑स्य प्राय॒णीय॑स्य॒ यज॑मानो॒ न न यज॑मानः प्राय॒णीय॑स्य । \newline
29. यज॑मानः प्राय॒णीय॑स्य प्राय॒णीय॑स्य॒ यज॑मानो॒ यज॑मानः प्राय॒णीय॑स्य निष्का॒से नि॑ष्का॒से प्रा॑य॒णीय॑स्य॒ यज॑मानो॒ यज॑मानः प्राय॒णीय॑स्य निष्का॒से । \newline
30. प्रा॒य॒णीय॑स्य निष्का॒से नि॑ष्का॒से प्रा॑य॒णीय॑स्य प्राय॒णीय॑स्य निष्का॒स उ॑दय॒नीय॑ मुदय॒नीय॑म् निष्का॒से प्रा॑य॒णीय॑स्य प्राय॒णीय॑स्य निष्का॒स उ॑दय॒नीय᳚म् । \newline
31. प्रा॒य॒णीय॒स्येति॑ प्र - अ॒य॒नीय॑स्य । \newline
32. नि॒ष्का॒स उ॑दय॒नीय॑ मुदय॒नीय॑म् निष्का॒से नि॑ष्का॒स उ॑दय॒नीय॑ म॒भ्या᳚(1॒)भ्यु॑दय॒नीय॑म् निष्का॒से नि॑ष्का॒स उ॑दय॒नीय॑ म॒भि । \newline
33. उ॒द॒य॒नीय॑ म॒भ्या᳚(1॒) भ्यु॑दय॒नीय॑ मुदय॒नीय॑ म॒भि निर् णिर॒भ्यु॑दय॒नीय॑ मुदय॒नीय॑ म॒भि निः । \newline
34. उ॒द॒य॒नीय॒मित्यु॑त् - अ॒य॒नीय᳚म् । \newline
35. अ॒भि निर् णिर॒भ्य॑भि निर् व॑पति वपति॒ निर॒भ्य॑भि निर् व॑पति । \newline
36. निर् व॑पति वपति॒ निर् णिर् व॑पति॒ सा सा व॑पति॒ निर् णिर् व॑पति॒ सा । \newline
37. व॒प॒ति॒ सा सा व॑पति वपति॒ सैवैव सा व॑पति वपति॒ सैव । \newline
38. सैवैव सा सैव सा सैव सा सैव सा । \newline
39. ए॒व सा सैवैव सा य॒ज्ञ्स्य॑ य॒ज्ञ्स्य॒ सैवैव सा य॒ज्ञ्स्य॑ । \newline
40. सा य॒ज्ञ्स्य॑ य॒ज्ञ्स्य॒ सा सा य॒ज्ञ्स्य॒ सन्त॑तिः॒ सन्त॑तिर् य॒ज्ञ्स्य॒ सा सा य॒ज्ञ्स्य॒ सन्त॑तिः । \newline
41. य॒ज्ञ्स्य॒ सन्त॑तिः॒ सन्त॑तिर् य॒ज्ञ्स्य॑ य॒ज्ञ्स्य॒ सन्त॑ति॒र् या याः सन्त॑तिर् य॒ज्ञ्स्य॑ य॒ज्ञ्स्य॒ सन्त॑ति॒र् याः । \newline
42. सन्त॑ति॒र् या याः सन्त॑तिः॒ सन्त॑ति॒र् याः प्रा॑य॒णीय॑स्य प्राय॒णीय॑स्य॒ याः सन्त॑तिः॒ सन्त॑ति॒र् याः प्रा॑य॒णीय॑स्य । \newline
43. सन्त॑ति॒रिति॒ सं - त॒तिः॒ । \newline
44. याः प्रा॑य॒णीय॑स्य प्राय॒णीय॑स्य॒ या याः प्रा॑य॒णीय॑स्य या॒ज्या॑ या॒ज्याः᳚ प्राय॒णीय॑स्य॒ या याः प्रा॑य॒णीय॑स्य या॒ज्याः᳚ । \newline
45. प्रा॒य॒णीय॑स्य या॒ज्या॑ या॒ज्याः᳚ प्राय॒णीय॑स्य प्राय॒णीय॑स्य या॒ज्या॑ यद् यद् या॒ज्याः᳚ प्राय॒णीय॑स्य प्राय॒णीय॑स्य या॒ज्या॑ यत् । \newline
46. प्रा॒य॒णीय॒स्येति॑ प्र - अ॒य॒नीय॑स्य । \newline
47. या॒ज्या॑ यद् यद् या॒ज्या॑ या॒ज्या॑ यत् ता स्ता यद् या॒ज्या॑ या॒ज्या॑ यत् ताः । \newline
48. यत् ता स्ता यद् यत् ता उ॑दय॒नीय॑ स्योदय॒नीय॑स्य॒ ता यद् यत् ता उ॑दय॒नीय॑स्य । \newline
49. ता उ॑दय॒नीय॑ स्योदय॒नीय॑स्य॒ ता स्ता उ॑दय॒नीय॑स्य या॒ज्या॑ या॒ज्या॑ उदय॒नीय॑स्य॒ ता स्ता उ॑दय॒नीय॑स्य या॒ज्याः᳚ । \newline
50. उ॒द॒य॒नीय॑स्य या॒ज्या॑ या॒ज्या॑ उदय॒नीय॑ स्योदय॒नीय॑स्य या॒ज्याः᳚ कु॒र्यात् कु॒र्याद् या॒ज्या॑ उदय॒नीय॑
स्योदय॒नीय॑स्य या॒ज्याः᳚ कु॒र्यात् । \newline
51. उ॒द॒य॒नीय॒स्येत्यु॑त् - अ॒य॒नीय॑स्य । \newline
52. या॒ज्याः᳚ कु॒र्यात् कु॒र्याद् या॒ज्या॑ या॒ज्याः᳚ कु॒र्यात् परा॒ङ् परा᳚ङ् कु॒र्याद् या॒ज्या॑ या॒ज्याः᳚ कु॒र्यात् पराङ्॑ । \newline
53. कु॒र्यात् परा॒ङ् परा᳚ङ् कु॒र्यात् कु॒र्यात् परा᳚ङ् ङ॒मु म॒मुम् परा᳚ङ् कु॒र्यात् कु॒र्यात् परा॑ङ॒मुम् । \newline
54. परा॑ङ॒मु म॒मुम् परा॒ङ् परा॑ङ॒मुम् ॅलो॒कम् ॅलो॒क म॒मुम् परा॒ङ् परा॑ङ॒मुम् ॅलो॒कम् । \newline
55. अ॒मुम् ॅलो॒कम् ॅलो॒क म॒मु म॒मुम् ॅलो॒क मा लो॒क म॒मु म॒मुम् ॅलो॒क मा । \newline
56. लो॒क मा लो॒कम् ॅलो॒क मा रो॑हेद् रोहे॒दा लो॒कम् ॅलो॒क मा रो॑हेत् । \newline
57. आ रो॑हेद् रोहे॒दा रो॑हेत् प्र॒मायु॑कः प्र॒मायु॑को रोहे॒दा रो॑हेत् प्र॒मायु॑कः । \newline
58. रो॒हे॒त् प्र॒मायु॑कः प्र॒मायु॑को रोहेद् रोहेत् प्र॒मायु॑कः स्याथ् स्यात् प्र॒मायु॑को रोहेद् रोहेत् प्र॒मायु॑कः स्यात् । \newline
59. प्र॒मायु॑कः स्याथ् स्यात् प्र॒मायु॑कः प्र॒मायु॑कः स्या॒द् या याः स्या᳚त् प्र॒मायु॑कः प्र॒मायु॑कः स्या॒द् याः । \newline
60. प्र॒मायु॑क॒ इति॑ प्र - मायु॑कः । \newline
61. स्या॒द् या याः स्या᳚थ् स्या॒द् याः प्रा॑य॒णीय॑स्य प्राय॒णीय॑स्य॒ याः स्या᳚थ् स्या॒द् याः प्रा॑य॒णीय॑स्य । \newline
62. याः प्रा॑य॒णीय॑स्य प्राय॒णीय॑स्य॒ या याः प्रा॑य॒णीय॑स्य पुरोनुवा॒क्याः᳚ पुरोनुवा॒क्याः᳚ प्राय॒णीय॑स्य॒ या याः प्रा॑य॒णीय॑स्य पुरोनुवा॒क्याः᳚ । \newline
63. प्रा॒य॒णीय॑स्य पुरोनुवा॒क्याः᳚ पुरोनुवा॒क्याः᳚ प्राय॒णीय॑स्य प्राय॒णीय॑स्य पुरोनुवा॒क्या᳚ स्ता स्ताः पु॑रोनुवा॒क्याः᳚ प्राय॒णीय॑स्य प्राय॒णीय॑स्य पुरोनुवा॒क्या᳚ स्ताः । \newline
64. प्रा॒य॒णीय॒स्येति॑ प्र - अ॒य॒नीय॑स्य । \newline
65. पु॒रो॒नु॒वा॒क्या᳚ स्ता स्ताः पु॑रोनुवा॒क्याः᳚ पुरोनुवा॒क्या᳚ स्ता उ॑दय॒नीय॑ स्योदय॒नीय॑स्य॒ ताः पु॑रोनुवा॒क्याः᳚ पुरोनुवा॒क्या᳚ स्ता उ॑दय॒नीय॑स्य । \newline
66. पु॒रो॒नु॒वा॒क्या॑ इति॑ पुरः - अ॒नु॒वा॒क्याः᳚ । \newline
67. ता उ॑दय॒नीय॑ स्योदय॒नीय॑स्य॒ ता स्ता उ॑दय॒नीय॑स्य या॒ज्या॑ या॒ज्या॑ उदय॒नीय॑स्य॒ ता स्ता उ॑दय॒नीय॑स्य या॒ज्याः᳚ । \newline
68. उ॒द॒य॒नीय॑स्य या॒ज्या॑ या॒ज्या॑ उदय॒नीय॑ स्योदय॒नीय॑स्य या॒ज्याः᳚ करोति करोति या॒ज्या॑ उदय॒नीय॑
स्योदय॒नीय॑स्य या॒ज्याः᳚ करोति । \newline
69. उ॒द॒य॒नीय॒स्येत्यु॑त् - अ॒य॒नीय॑स्य । \newline
70. या॒ज्याः᳚ करोति करोति या॒ज्या॑ या॒ज्याः᳚ करो त्य॒स्मिन् न॒स्मिन् क॑रोति या॒ज्या॑ या॒ज्याः᳚ करो त्य॒स्मिन्न् । \newline
71. क॒रो॒ त्य॒स्मिन् न॒स्मिन् क॑रोति करो त्य॒स्मिन् ने॒वैवास्मिन् क॑रोति करो त्य॒स्मिन् ने॒व । \newline
72. अ॒स्मिन् ने॒वै वास्मिन् न॒स्मिन् ने॒व लो॒के लो॒क ए॒वास्मिन् न॒स्मिन् ने॒व लो॒के । \newline
73. ए॒व लो॒के लो॒क ए॒वैव लो॒के प्रति॒ प्रति॑ लो॒क ए॒वैव लो॒के प्रति॑ । \newline
74. लो॒के प्रति॒ प्रति॑ लो॒के लो॒के प्रति॑ तिष्ठति तिष्ठति॒ प्रति॑ लो॒के लो॒के प्रति॑ तिष्ठति । \newline
75. प्रति॑ तिष्ठति तिष्ठति॒ प्रति॒ प्रति॑ तिष्ठति । \newline
76. ति॒ष्ठ॒तीति॑ तिष्ठति । \newline
\pagebreak
\markright{ TS 6.1.6.1  \hfill https://www.vedavms.in \hfill}

\section{ TS 6.1.6.1 }

\textbf{TS 6.1.6.1 } \newline
\textbf{Samhita Paata} \newline

क॒द्रूश्च॒ वै सु॑प॒र्णी चा᳚ऽऽ*त्मरू॒पयो॑रस्पर्द्धेताꣳ॒॒ सा क॒द्रूः सु॑प॒र्णीम॑जय॒थ् साऽब्र॑वीत् तृ॒तीय॑स्यामि॒तो दि॒वि सोम॒स्तमा ह॑र॒ तेना॒ऽऽ*त्मानं॒ निष्क्री॑णी॒ष्वेती॒यं ॅवै क॒द्रूर॒सौ सु॑प॒र्णी छन्दाꣳ॑सि सौपर्णे॒याः साब्र॑वीद॒स्मै वै पि॒तरौ॑ पु॒त्रान् बि॑भृत-स्तृ॒तीय॑स्यामि॒तो दि॒वि सोम॒स्तमा ह॑र॒ तेना॒ऽऽ*त्मानं॒ निष्क्री॑णी॒ष्वे - [  ] \newline

\textbf{Pada Paata} \newline

क॒द्रूः । च॒ । वै । सु॒प॒र्णीति॑ सु - प॒र्णी । च॒ । आ॒त्म॒रू॒पयो॒रित्या᳚त्म - रू॒पयोः᳚ । अ॒स्प॒द्‌र्धे॒ता॒म् । सा । क॒द्रूः । सु॒प॒र्णीमिति॑ सु-प॒र्णीम् । अ॒ज॒य॒त् । सा । अ॒ब्र॒वी॒त् । तृ॒तीय॑स्याम् । इ॒तः । दि॒वि । सोमः॑ । तम् । एति॑ । ह॒र॒ । तेन॑ । आ॒त्मान᳚म् । निरिति॑ । क्री॒णी॒ष्व॒ । इति॑ । इ॒यम् । वै । क॒द्रूः । अ॒सौ । सु॒प॒र्णीति॑ सु-प॒र्णी । छन्दाꣳ॑सि । सौ॒प॒र्णे॒याः । सा । अ॒ब्र॒वी॒त् । अ॒स्मै । वै । पि॒तरौ᳚ । पु॒त्रान् । बि॒भृ॒तः॒ । तृ॒तीय॑स्याम् । इ॒तः । दि॒वि । सोमः॑ । तम् । एति॑ । ह॒र॒ । तेन॑ । आ॒त्मान᳚म् । निरिति॑ । क्री॒णी॒ष्व॒ ।  \newline


\textbf{Krama Paata} \newline

क॒द्रूश्च॑ । च॒ वै । वै सु॑प॒र्णी । सु॒प॒र्णी च॑ । सु॒प॒र्णीति॑ सु - प॒र्णी । चा॒त्म॒रू॒पयोः᳚ । आ॒त्म॒रू॒पयो॑,रस्पर्द्धेताम् । आ॒त्म॒रू॒पयो॒रित्या᳚त्म - रू॒पयोः᳚ । अ॒स्प॒र्द्धे॒ताꣳ॒॒ सा । सा क॒द्रूः । क॒द्रूः सु॑प॒र्णीम् । सु॒प॒र्णीम॑जयत् । सु॒प॒र्णीमिति॑ सु - प॒र्णीम् । अ॒ज॒य॒थ् सा । साऽब्र॑वीत् । अ॒ब्र॒वी॒त् तृ॒तीय॑स्याम् । तृ॒तीय॑स्यामि॒तः । इ॒तो दि॒वि । दि॒वि सोमः॑ । सोम॒स्तम् । तमा । आ ह॑र । ह॒र॒ तेन॑ । तेना॒ऽऽत्मान᳚म् । आ॒त्मान॒म् निः । निष्क्री॑णीष्व । क्री॒णी॒ष्वेति॑ । इती॒यम् । इ॒यम् ॅवै । वै क॒द्रूः । क॒द्रूर॒सौ । अ॒सौ सु॑प॒र्णी । सु॒प॒र्णी छन्दाꣳ॑सि । सु॒प॒र्णीति॑ सु - प॒र्णी । छन्दाꣳ॑सि सौपर्णे॒याः । सौ॒प॒र्णे॒याः सा । साऽब्र॑वीत् । अ॒ब्र॒वी॒द॒स्मै । अ॒स्मै वै । वै पि॒तरौ᳚ । पि॒तरौ॑ पु॒त्रान् । पु॒त्रान् बि॑भृतः । बि॒भृ॒त॒स्तृ॒तीय॑स्याम् । तृ॒तीय॑स्यामि॒तः । इ॒तो दि॒वि । दि॒वि सोमः॑ । सोम॒स्तम् । तमा । आ ह॑र । ह॒र॒ तेन॑ । तेना॒ऽऽत्मान᳚म् । आ॒त्मान॒म् निः । निष्क्री॑णीष्व । क्री॒णी॒ष्वेति॑ \newline

\textbf{Jatai Paata} \newline

1. क॒द्रू श्च॑ च क॒द्रूः क॒द्रू श्च॑ । \newline
2. च॒ वै वै च॑ च॒ वै । \newline
3. वै सु॑प॒र्णी सु॑प॒र्णी वै वै सु॑प॒र्णी । \newline
4. सु॒प॒र्णी च॑ च सुप॒र्णी सु॑प॒र्णी च॑ । \newline
5. सु॒प॒र्णीति॑ सु - प॒र्णी । \newline
6. चा॒त्म॒रू॒पयो॑ रात्मरू॒पयो᳚ श्च चात्मरू॒पयोः᳚ । \newline
7. आ॒त्म॒रू॒पयो॑ रस्पर्द्धेता मस्पर्द्धेता मात्मरू॒पयो॑ रात्मरू॒पयो॑ रस्पर्द्धेताम् । \newline
8. आ॒त्म॒रू॒पयो॒रित्या᳚त्म - रू॒पयोः᳚ । \newline
9. अ॒स्प॒र्द्धे॒ताꣳ॒॒ सा सा ऽस्प॑र्द्धेता मस्पर्द्धेताꣳ॒॒ सा । \newline
10. सा क॒द्रूः क॒द्रूः सा सा क॒द्रूः । \newline
11. क॒द्रूः सु॑प॒र्णीꣳ सु॑प॒र्णीम् क॒द्रूः क॒द्रूः सु॑प॒र्णीम् । \newline
12. सु॒प॒र्णी म॑जय दजयथ् सुप॒र्णीꣳ सु॑प॒र्णी म॑जयत् । \newline
13. सु॒प॒र्णीमिति॑ सु - प॒र्णीम् । \newline
14. अ॒ज॒य॒थ् सा सा ऽज॑य दजय॒थ् सा । \newline
15. सा ऽब्र॑वी दब्रवी॒थ् सा सा ऽब्र॑वीत् । \newline
16. अ॒ब्र॒वी॒त् तृ॒तीय॑स्याम् तृ॒तीय॑स्या मब्रवी दब्रवीत् तृ॒तीय॑स्याम् । \newline
17. तृ॒तीय॑स्या मि॒त इ॒त स्तृ॒तीय॑स्याम् तृ॒तीय॑स्या मि॒तः । \newline
18. इ॒तो दि॒वि दि॒वीत इ॒तो दि॒वि । \newline
19. दि॒वि सोमः॒ सोमो॑ दि॒वि दि॒वि सोमः॑ । \newline
20. सोम॒ स्तम् तꣳ सोमः॒ सोम॒ स्तम् । \newline
21. त मा तम् त मा । \newline
22. आ ह॑र ह॒रा ह॑र । \newline
23. ह॒र॒ तेन॒ तेन॑ हर हर॒ तेन॑ । \newline
24. तेना॒त्मान॑ मा॒त्मान॒म् तेन॒ तेना॒त्मान᳚म् । \newline
25. आ॒त्मान॒म् निर् णिरा॒त्मान॑ मा॒त्मान॒म् निः । \newline
26. निष् क्री॑णीष्व क्रीणीष्व॒ निर् णिष् क्री॑णीष्व । \newline
27. क्री॒णी॒ष्वे तीति॑ क्रीणीष्व क्रीणी॒ष्वेति॑ । \newline
28. इती॒य मि॒य मिती ती॒यम् । \newline
29. इ॒यं ॅवै वा इ॒य मि॒यं ॅवै । \newline
30. वै क॒द्रूः क॒द्रूर् वै वै क॒द्रूः । \newline
31. क॒द्रू र॒सा व॒सौ क॒द्रूः क॒द्रू र॒सौ । \newline
32. अ॒सौ सु॑प॒र्णी सु॑प॒र्ण्य॑सा व॒सौ सु॑प॒र्णी । \newline
33. सु॒प॒र्णी छन्दाꣳ॑सि॒ छन्दाꣳ॑सि सुप॒र्णी सु॑प॒र्णी छन्दाꣳ॑सि । \newline
34. सु॒प॒र्णीति॑ सु - प॒र्णी । \newline
35. छन्दाꣳ॑सि सौपर्णे॒याः सौ॑पर्णे॒या श्छन्दाꣳ॑सि॒ छन्दाꣳ॑सि सौपर्णे॒याः । \newline
36. सौ॒प॒र्णे॒याः सा सा सौ॑पर्णे॒याः सौ॑पर्णे॒याः सा । \newline
37. सा ऽब्र॑वी दब्रवी॒थ् सा सा ऽब्र॑वीत् । \newline
38. अ॒ब्र॒वी॒ द॒स्मा अ॒स्मा अ॑ब्रवी दब्रवी द॒स्मै । \newline
39. अ॒स्मै वै वा अ॒स्मा अ॒स्मै वै । \newline
40. वै पि॒तरौ॑ पि॒तरौ॒ वै वै पि॒तरौ᳚ । \newline
41. पि॒तरौ॑ पु॒त्रान् पु॒त्रान् पि॒तरौ॑ पि॒तरौ॑ पु॒त्रान् । \newline
42. पु॒त्रान् बि॑भृतो बिभृतः पु॒त्रान् पु॒त्रान् बि॑भृतः । \newline
43. बि॒भृ॒त॒ स्तृ॒तीय॑स्याम् तृ॒तीय॑स्याम् बिभृतो बिभृत स्तृ॒तीय॑स्याम् । \newline
44. तृ॒तीय॑स्या मि॒त इ॒त स्तृ॒तीय॑स्याम् तृ॒तीय॑स्या मि॒तः । \newline
45. इ॒तो दि॒वि दि॒वीत इ॒तो दि॒वि । \newline
46. दि॒वि सोमः॒ सोमो॑ दि॒वि दि॒वि सोमः॑ । \newline
47. सोम॒ स्तम् तꣳ सोमः॒ सोम॒ स्तम् । \newline
48. त मा तम् त मा । \newline
49. आ ह॑र ह॒रा ह॑र । \newline
50. ह॒र॒ तेन॒ तेन॑ हर हर॒ तेन॑ । \newline
51. तेना॒त्मान॑ मा॒त्मान॒म् तेन॒ तेना॒त्मान᳚म् । \newline
52. आ॒त्मान॒न् निर् णिरा॒त्मान॑ मा॒त्मान॒न् निः । \newline
53. निष् क्री॑णीष्व क्रीणीष्व॒ निर् णिष् क्री॑णीष्व । \newline
54. क्री॒णी॒ष्वे तीति॑ क्रीणीष्व क्रीणी॒ष्वेति॑ । \newline

\textbf{Ghana Paata } \newline

1. क॒द्रूश्च॑ च क॒द्रूः क॒द्रूश्च॒ वै वै च॑ क॒द्रूः क॒द्रूश्च॒ वै । \newline
2. च॒ वै वै च॑ च॒ वै सु॑प॒र्णी सु॑प॒र्णी वै च॑ च॒ वै सु॑प॒र्णी । \newline
3. वै सु॑प॒र्णी सु॑प॒र्णी वै वै सु॑प॒र्णी च॑ च सुप॒र्णी वै वै सु॑प॒र्णी च॑ । \newline
4. सु॒प॒र्णी च॑ च सुप॒र्णी सु॑प॒र्णी चा᳚त्मरू॒पयो॑ रात्मरू॒पयो᳚श्च सुप॒र्णी सु॑प॒र्णी चा᳚त्मरू॒पयोः᳚ । \newline
5. सु॒प॒र्णीति॑ सु - प॒र्णी । \newline
6. चा॒त्म॒रू॒पयो॑ रात्मरू॒पयो᳚श्च चात्मरू॒पयो॑ रस्पर्द्धेता मस्पर्द्धेता मात्मरू॒पयो᳚श्च चात्मरू॒पयो॑ रस्पर्द्धेताम् । \newline
7. आ॒त्म॒रू॒पयो॑ रस्पर्द्धेता मस्पर्द्धेता मात्मरू॒पयो॑ रात्मरू॒पयो॑ रस्पर्द्धेताꣳ॒॒ सा सा ऽस्प॑र्द्धेता मात्मरू॒पयो॑ रात्मरू॒पयो॑ रस्पर्द्धेताꣳ॒॒ सा । \newline
8. आ॒त्म॒रू॒पयो॒रित्या᳚त्म - रू॒पयोः᳚ । \newline
9. अ॒स्प॒र्द्धे॒ताꣳ॒॒ सा सा ऽस्प॑र्द्धेता मस्पर्द्धेताꣳ॒॒ सा क॒द्रूः क॒द्रूः सा ऽस्प॑र्द्धेता मस्पर्द्धेताꣳ॒॒ सा क॒द्रूः । \newline
10. सा क॒द्रूः क॒द्रूः सा सा क॒द्रूः सु॑प॒र्णीꣳ सु॑प॒र्णीम् क॒द्रूः सा सा क॒द्रूः सु॑प॒र्णीम् । \newline
11. क॒द्रूः सु॑प॒र्णीꣳ सु॑प॒र्णीम् क॒द्रूः क॒द्रूः सु॑प॒र्णी म॑जय दजयथ् सुप॒र्णीम् क॒द्रूः क॒द्रूः सु॑प॒र्णी म॑जयत् । \newline
12. सु॒प॒र्णी म॑जय दजयथ् सुप॒र्णीꣳ सु॑प॒र्णी म॑जय॒थ् सा सा ऽज॑यथ् सुप॒र्णीꣳ सु॑प॒र्णी म॑जय॒थ् सा । \newline
13. सु॒प॒र्णीमिति॑ सु - प॒र्णीम् । \newline
14. अ॒ज॒य॒थ् सा सा ऽज॑य दजय॒थ् सा ऽब्र॑वी दब्रवी॒थ् सा ऽज॑य दजय॒थ् सा ऽब्र॑वीत् । \newline
15. सा ऽब्र॑वी दब्रवी॒थ् सा सा ऽब्र॑वीत् तृ॒तीय॑स्याम् तृ॒तीय॑स्या मब्रवी॒थ् सा सा ऽब्र॑वीत् तृ॒तीय॑स्याम् । \newline
16. अ॒ब्र॒वी॒त् तृ॒तीय॑स्याम् तृ॒तीय॑स्या मब्रवी दब्रवीत् तृ॒तीय॑स्या मि॒त इ॒त स्तृ॒तीय॑स्या मब्रवी दब्रवीत् तृ॒तीय॑स्या मि॒तः । \newline
17. तृ॒तीय॑स्या मि॒त इ॒त स्तृ॒तीय॑स्याम् तृ॒तीय॑स्या मि॒तो दि॒वि दि॒वीत स्तृ॒तीय॑स्याम् तृ॒तीय॑स्या मि॒तो दि॒वि । \newline
18. इ॒तो दि॒वि दि॒वीत इ॒तो दि॒वि सोमः॒ सोमो॑ दि॒वीत इ॒तो दि॒वि सोमः॑ । \newline
19. दि॒वि सोमः॒ सोमो॑ दि॒वि दि॒वि सोम॒ स्तम् तꣳ सोमो॑ दि॒वि दि॒वि सोम॒ स्तम् । \newline
20. सोम॒ स्तम् तꣳ सोमः॒ सोम॒ स्त मा तꣳ सोमः॒ सोम॒ स्त मा । \newline
21. त मा तम् त मा ह॑र ह॒रा तम् त मा ह॑र । \newline
22. आ ह॑र ह॒रा ह॑र॒ तेन॒ तेन॑ ह॒रा ह॑र॒ तेन॑ । \newline
23. ह॒र॒ तेन॒ तेन॑ हर हर॒ तेना॒त्मान॑ मा॒त्मान॒म् तेन॑ हर हर॒ तेना॒त्मान᳚म् । \newline
24. तेना॒त्मान॑ मा॒त्मान॒म् तेन॒ तेना॒त्मान॒न् निर् णिरा॒त्मान॒म् तेन॒ तेना॒त्मान॒न् निः । \newline
25. आ॒त्मान॒न् निर् णिरा॒त्मान॑ मा॒त्मान॒न् निष् क्री॑णीष्व क्रीणीष्व॒ निरा॒त्मान॑ मा॒त्मान॒न् निष् क्री॑णीष्व । \newline
26. निष् क्री॑णीष्व क्रीणीष्व॒ निर् णिष् क्री॑णी॒ष्वे तीति॑ क्रीणीष्व॒ निर् णिष् क्री॑णी॒ष्वेति॑ । \newline
27. क्री॒णी॒ष्वे तीति॑ क्रीणीष्व क्रीणी॒ष्वे ती॒य मि॒य मिति॑ क्रीणीष्व क्रीणी॒ष्वे ती॒यम् । \newline
28. इती॒य मि॒य मितीती॒यं ॅवै वा इ॒य मितीती॒यं ॅवै । \newline
29. इ॒यं ॅवै वा इ॒य मि॒यं ॅवै क॒द्रूः क॒द्रूर् वा इ॒य मि॒यं ॅवै क॒द्रूः । \newline
30. वै क॒द्रूः क॒द्रूर् वै वै क॒द्रू र॒सा व॒सौ क॒द्रूर् वै वै क॒द्रू र॒सौ । \newline
31. क॒द्रू र॒सा व॒सौ क॒द्रूः क॒द्रू र॒सौ सु॑प॒र्णी सु॑प॒र्ण्य॑सौ क॒द्रूः क॒द्रू र॒सौ सु॑प॒र्णी । \newline
32. अ॒सौ सु॑प॒र्णी सु॑प॒र्ण्य॑सा व॒सौ सु॑प॒र्णी छन्दाꣳ॑सि॒ छन्दाꣳ॑सि सुप॒र्ण्य॑सा व॒सौ सु॑प॒र्णी छन्दाꣳ॑सि । \newline
33. सु॒प॒र्णी छन्दाꣳ॑सि॒ छन्दाꣳ॑सि सुप॒र्णी सु॑प॒र्णी छन्दाꣳ॑सि सौपर्णे॒याः सौ॑पर्णे॒या श्छन्दाꣳ॑सि सुप॒र्णी सु॑प॒र्णी छन्दाꣳ॑सि सौपर्णे॒याः । \newline
34. सु॒प॒र्णीति॑ सु - प॒र्णी । \newline
35. छन्दाꣳ॑सि सौपर्णे॒याः सौ॑पर्णे॒या श्छन्दाꣳ॑सि॒ छन्दाꣳ॑सि सौपर्णे॒याः सा सा सौ॑पर्णे॒या श्छन्दाꣳ॑सि॒ छन्दाꣳ॑सि सौपर्णे॒याः सा । \newline
36. सौ॒प॒र्णे॒याः सा सा सौ॑पर्णे॒याः सौ॑पर्णे॒याः सा ऽब्र॑वी दब्रवी॒थ् सा सौ॑पर्णे॒याः सौ॑पर्णे॒याः सा ऽब्र॑वीत् । \newline
37. सा ऽब्र॑वी दब्रवी॒थ् सा सा ऽब्र॑वी द॒स्मा अ॒स्मा अ॑ब्रवी॒थ् सा सा ऽब्र॑वी द॒स्मै । \newline
38. अ॒ब्र॒वी॒ द॒स्मा अ॒स्मा अ॑ब्रवी दब्रवी द॒स्मै वै वा अ॒स्मा अ॑ब्रवी दब्रवी द॒स्मै वै । \newline
39. अ॒स्मै वै वा अ॒स्मा अ॒स्मै वै पि॒तरौ॑ पि॒तरौ॒ वा अ॒स्मा अ॒स्मै वै पि॒तरौ᳚ । \newline
40. वै पि॒तरौ॑ पि॒तरौ॒ वै वै पि॒तरौ॑ पु॒त्रान् पु॒त्रान् पि॒तरौ॒ वै वै पि॒तरौ॑ पु॒त्रान् । \newline
41. पि॒तरौ॑ पु॒त्रान् पु॒त्रान् पि॒तरौ॑ पि॒तरौ॑ पु॒त्रान् बि॑भृतो बिभृतः पु॒त्रान् पि॒तरौ॑ पि॒तरौ॑ पु॒त्रान् बि॑भृतः । \newline
42. पु॒त्रान् बि॑भृतो बिभृतः पु॒त्रान् पु॒त्रान् बि॑भृत स्तृ॒तीय॑स्याम् तृ॒तीय॑स्याम् बिभृतः पु॒त्रान् पु॒त्रान् बि॑भृत स्तृ॒तीय॑स्याम् । \newline
43. बि॒भृ॒त॒ स्तृ॒तीय॑स्याम् तृ॒तीय॑स्याम् बिभृतो बिभृत स्तृ॒तीय॑स्या मि॒त इ॒त स्तृ॒तीय॑स्याम् बिभृतो बिभृत स्तृ॒तीय॑स्या मि॒तः । \newline
44. तृ॒तीय॑स्या मि॒त इ॒त स्तृ॒तीय॑स्याम् तृ॒तीय॑स्या मि॒तो दि॒वि दि॒वीत स्तृ॒तीय॑स्याम् तृ॒तीय॑स्या मि॒तो दि॒वि । \newline
45. इ॒तो दि॒वि दि॒वीत इ॒तो दि॒वि सोमः॒ सोमो॑ दि॒वीत इ॒तो दि॒वि सोमः॑ । \newline
46. दि॒वि सोमः॒ सोमो॑ दि॒वि दि॒वि सोम॒ स्तम् तꣳ सोमो॑ दि॒वि दि॒वि सोम॒ स्तम् । \newline
47. सोम॒ स्तम् तꣳ सोमः॒ सोम॒ स्त मा तꣳ सोमः॒ सोम॒ स्त मा । \newline
48. त मा तम् त मा ह॑र ह॒रा तम् त मा ह॑र । \newline
49. आ ह॑र ह॒रा ह॑र॒ तेन॒ तेन॑ ह॒रा ह॑र॒ तेन॑ । \newline
50. ह॒र॒ तेन॒ तेन॑ हर हर॒ तेना॒त्मान॑ मा॒त्मान॒म् तेन॑ हर हर॒ तेना॒त्मान᳚म् । \newline
51. तेना॒त्मान॑ मा॒त्मान॒म् तेन॒ तेना॒त्मान॒न् निर् णिरा॒त्मान॒म् तेन॒ तेना॒त्मान॒न् निः । \newline
52. आ॒त्मान॒न् निर् णिरा॒त्मान॑ मा॒त्मान॒न् निष् क्री॑णीष्व क्रीणीष्व॒ निरा॒त्मान॑ मा॒त्मान॒न् निष् क्री॑णीष्व । \newline
53. निष् क्री॑णीष्व क्रीणीष्व॒ निर् णिष् क्री॑णी॒ष्वे तीति॑ क्रीणीष्व॒ निर् णिष् क्री॑णी॒ष्वेति॑ । \newline
54. क्री॒णी॒ष्वे तीति॑ क्रीणीष्व क्रीणी॒ष्वेति॑ मा॒ मेति॑ क्रीणीष्व क्रीणी॒ष्वेति॑ मा । \newline
\pagebreak
\markright{ TS 6.1.6.2  \hfill https://www.vedavms.in \hfill}

\section{ TS 6.1.6.2 }

\textbf{TS 6.1.6.2 } \newline
\textbf{Samhita Paata} \newline

ति॑ मा क॒द्रूर॑वोच॒दिति॒ जग॒त्युद॑पत॒-च्चतु॑र्दशाक्षरा स॒ती सा ऽप्रा᳚प्य॒ न्य॑वर्तत॒ तस्यै॒ द्वे अ॒क्षरे॑ अमीयेताꣳ॒॒ सा प॒शुभि॑श्च दी॒क्षया॒ चाऽग॑च्छ॒त् तस्मा॒ज्जग॑ती॒ छन्द॑सां पश॒व्य॑तमा॒ तस्मा᳚त् पशु॒मन्तं॑ दी॒क्षोप॑ नमति त्रि॒ष्टुगुद॑पत॒त् त्रयो॑दशाक्षरा स॒ती सा ऽप्रा᳚प्य॒ न्य॑वर्तत॒ तस्यै॒ द्वे अ॒क्षरे॑ अमीयेताꣳ॒॒ सा दक्षि॑णाभिश्च॒ - [  ] \newline

\textbf{Pada Paata} \newline

इति॑ । मा॒ । क॒द्रूः । अ॒वो॒च॒त् । इति॑ । जग॑ती । उदिति॑ । अ॒प॒त॒त् । चतु॑र्दशाक्ष॒रेति॒ चतु॑र्दश -   अ॒क्ष॒रा॒ । स॒ती । सा । अप्रा॒प्येत्यप्र॑ - आ॒प्य॒ । नीति॑ । अ॒व॒र्त॒त॒ । तस्यै᳚ । द्वे इति॑ । अ॒क्षरे॒ इति॑ । अ॒मी॒ये॒ता॒म् । सा । प॒शुभि॒रिति॑ प॒शु - भिः॒ । च॒ । दी॒क्षया᳚ । च॒ । एति॑ । अ॒ग॒च्छ॒त् । तस्मा᳚त् । जग॑ती । छन्द॑साम् । प॒श॒व्य॑त॒मेति॑ पश॒व्य॑ - त॒मा॒ । तस्मा᳚त् । प॒शु॒मन्त॒मिति॑ पशु - मन्त᳚म् । दी॒क्षा । उपेति॑ । न॒म॒ति॒ । त्रि॒ष्टुक् । उदिति॑ । अ॒प॒त॒त् । त्रयो॑दशाक्ष॒रेति॒ त्रयो॑दश - अ॒क्ष॒रा॒ । स॒ती । सा । अप्रा॒प्येत्यप्र॑ - आ॒प्य॒ । नीति॑ । अ॒व॒र्त॒त॒ । तस्यै᳚ । द्वे इति॑ । अ॒क्षरे॒ इति॑ । अ॒मी॒ये॒ता॒म् । सा । दक्षि॑णाभिः । च॒ ।  \newline


\textbf{Krama Paata} \newline

इति॑ मा । मा॒ क॒द्रूः । क॒द्रूर॑वोचत् । अ॒वो॒च॒दिति॑ । इति॒ जग॑ती । जग॒त्युत् । उद॑पतत् । अ॒प॒त॒च् चतु॑र्दशाक्षरा । चतु॑र्दशाक्षारा स॒ती । चतु॑र्दशाक्ष॒रेति॒ चतु॑र्दश - अ॒क्ष॒रा॒ । स॒ती सा । साऽप्रा᳚प्य । अप्रा᳚प्य॒ नि । अप्रा॒प्येत्यप्र॑ - आ॒प्य॒ । न्य॑वर्तत । अ॒व॒र्त॒त॒ तस्यै᳚ । तस्यै॒ द्वे । द्वे अ॒क्षरे᳚ । द्वे इति॒ द्वे । अ॒क्षरे॑ अमीयेताम् । अ॒क्षरे॒ इत्य॒क्षरे᳚ । अ॒मी॒ये॒ताꣳ॒॒ सा । सा प॒शुभिः॑ । प॒शुभि॑श्च । प॒शुभि॒रिति॑ प॒शु - भिः॒ । 
च॒ दी॒क्षया᳚ । दी॒क्षया॑ च । चा । आऽग॑च्छत् । अ॒ग॒च्छ॒त् तस्मा᳚त् । तस्मा॒ज् जग॑ती । जग॑ती॒ छन्द॑साम् । छन्द॑साम् पश॒व्य॑तमा । प॒श॒व्य॑तमा॒ तस्मा᳚त् । प॒श॒व्य॑त॒मेति॑ पश॒व्य॑ - त॒मा॒ । तस्मा᳚त् पशु॒मन्त᳚म् । प॒शु॒मन्त॑म् दी॒क्षा । प॒शु॒मन्त॒मिति॑ पशु - मन्त᳚म् । दी॒क्षोप॑ । उप॑ नमति । न॒म॒ति॒ त्रि॒ष्टुक् । त्रि॒ष्टुगुत् । उद॑पतत् । अ॒प॒त॒त् त्रयो॑दशाक्षरा । त्रयो॑दशाक्षरा स॒ती । त्रयो॑दशाक्ष॒रेति॒ त्रयो॑दश - अ॒क्ष॒रा॒ । स॒ती सा । साऽप्रा᳚प्य । अप्रा᳚प्य॒ नि । अप्रा॒प्येत्यप्र॑ - आ॒प्य॒ । न्य॑वर्तत । अ॒व॒र्त॒त॒ तस्यै᳚ । तस्यै॒ द्वे । द्वे अ॒क्षरे᳚ । द्वे इति॒ द्वे । अ॒क्षरे॑ अमीयेताम् । अ॒क्षरे॒ इत्य॒क्षरे᳚ । अ॒मी॒ये॒ताꣳ॒॒ सा । सा दक्षि॑णाभिः । दक्षि॑णाभिश्च । च॒ तप॑सा \newline

\textbf{Jatai Paata} \newline

1. इति॑ मा॒ मेतीति॑ मा । \newline
2. मा॒ क॒द्रूः क॒द्रूर् मा॑ मा क॒द्रूः । \newline
3. क॒द्रू र॑वोच दवोचत् क॒द्रूः क॒द्रू र॑वोचत् । \newline
4. अ॒वो॒च॒ दिती त्य॑वोच दवोच॒ दिति॑ । \newline
5. इति॒ जग॑ती॒ जग॒ती तीति॒ जग॑ती । \newline
6. जग॒ त्युदुज् जग॑ती॒ जग॒ त्युत् । \newline
7. उद॑पत दपत॒ दुदु द॑पतत् । \newline
8. अ॒प॒त॒च् चतु॑र्दशाक्षरा॒ चतु॑र्दशाक्षरा ऽपत दपत॒च् चतु॑र्दशाक्षरा । \newline
9. चतु॑र्दशाक्षरा स॒ती स॒ती चतु॑र्दशाक्षरा॒ चतु॑र्दशाक्षरा स॒ती । \newline
10. चतु॑र्दशाक्ष॒रेति॒ चतु॑र्दश - अ॒क्ष॒रा॒ । \newline
11. स॒ती सा सा स॒ती स॒ती सा । \newline
12. सा ऽप्रा॒प्या प्रा᳚प्य॒ सा सा ऽप्रा᳚प्य । \newline
13. अप्रा᳚प्य॒ नि न्यप्रा॒प्या प्रा᳚प्य॒ नि । \newline
14. अप्रा॒प्येत्यप्र॑ - आ॒प्य॒ । \newline
15. न्य॑वर्तता वर्तत॒ नि न्य॑वर्तत । \newline
16. अ॒व॒र्त॒त॒ तस्यै॒ तस्या॑ अवर्तता वर्तत॒ तस्यै᳚ । \newline
17. तस्यै॒ द्वे द्वे तस्यै॒ तस्यै॒ द्वे । \newline
18. द्वे अ॒क्षरे॑ अ॒क्षरे॒ द्वे द्वे अ॒क्षरे᳚ । \newline
19. द्वे इति॒ द्वे । \newline
20. अ॒क्षरे॑ अमीयेता ममीयेता म॒क्षरे॑ अ॒क्षरे॑ अमीयेताम् । \newline
21. अ॒क्षरे॒ इत्य॒क्षरे᳚ । \newline
22. अ॒मी॒ये॒ताꣳ॒॒ सा सा ऽमी॑येता ममीयेताꣳ॒॒ सा । \newline
23. सा प॒शुभिः॑ प॒शुभिः॒ सा सा प॒शुभिः॑ । \newline
24. प॒शुभि॑ श्च च प॒शुभिः॑ प॒शुभि॑ श्च । \newline
25. प॒शुभि॒रिति॑ प॒शु - भिः॒ । \newline
26. च॒ दी॒क्षया॑ दी॒क्षया॑ च च दी॒क्षया᳚ । \newline
27. दी॒क्षया॑ च च दी॒क्षया॑ दी॒क्षया॑ च । \newline
28. चा च॒ चा । \newline
29. आ ऽग॑च्छ दगच्छ॒दा ऽग॑च्छत् । \newline
30. अ॒ग॒च्छ॒त् तस्मा॒त् तस्मा॑ दगच्छ दगच्छ॒त् तस्मा᳚त् । \newline
31. तस्मा॒ज् जग॑ती॒ जग॑ती॒ तस्मा॒त् तस्मा॒ज् जग॑ती । \newline
32. जग॑ती॒ छन्द॑सा॒म् छन्द॑सा॒म् जग॑ती॒ जग॑ती॒ छन्द॑साम् । \newline
33. छन्द॑साम् पश॒व्य॑तमा पश॒व्य॑तमा॒ छन्द॑सा॒म् छन्द॑साम् पश॒व्य॑तमा । \newline
34. प॒श॒व्य॑तमा॒ तस्मा॒त् तस्मा᳚त् पश॒व्य॑तमा पश॒व्य॑तमा॒ तस्मा᳚त् । \newline
35. प॒श॒व्य॑त॒मेति॑ पश॒व्य॑ - त॒मा॒ । \newline
36. तस्मा᳚त् पशु॒मन्त॑म् पशु॒मन्त॒म् तस्मा॒त् तस्मा᳚त् पशु॒मन्त᳚म् । \newline
37. प॒शु॒मन्त॑म् दी॒क्षा दी॒क्षा प॑शु॒मन्त॑म् पशु॒मन्त॑म् दी॒क्षा । \newline
38. प॒शु॒मन्त॒मिति॑ पशु - मन्त᳚म् । \newline
39. दी॒क्षो पोप॑ दी॒क्षा दी॒क्षोप॑ । \newline
40. उप॑ नमति नम॒ त्युपोप॑ नमति । \newline
41. न॒म॒ति॒ त्रि॒ष्टुक् त्रि॒ष्टुङ् न॑मति नमति त्रि॒ष्टुक् । \newline
42. त्रि॒ष्टु गुदुत् त्रि॒ष्टुक् त्रि॒ष्टु गुत् । \newline
43. उद॑पत दपत॒ दुदु द॑पतत् । \newline
44. अ॒प॒त॒त् त्रयो॑दशाक्षरा॒ त्रयो॑दशाक्षरा ऽपत दपत॒त् त्रयो॑दशाक्षरा । \newline
45. त्रयो॑दशाक्षरा स॒ती स॒ती त्रयो॑दशाक्षरा॒ त्रयो॑दशाक्षरा स॒ती । \newline
46. त्रयो॑दशाक्ष॒रेति॒ त्रयो॑दश - अ॒क्ष॒रा॒ । \newline
47. स॒ती सा सा स॒ती स॒ती सा । \newline
48. सा ऽप्रा॒प्या प्रा᳚प्य॒ सा सा ऽप्रा᳚प्य । \newline
49. अप्रा᳚प्य॒ नि न्यप्रा॒प्या प्रा᳚प्य॒ नि । \newline
50. अप्रा॒प्येत्यप्र॑ - आ॒प्य॒ । \newline
51. न्य॑वर्तता वर्तत॒ नि न्य॑वर्तत । \newline
52. अ॒व॒र्त॒त॒ तस्यै॒ तस्या॑ अवर्तता वर्तत॒ तस्यै᳚ । \newline
53. तस्यै॒ द्वे द्वे तस्यै॒ तस्यै॒ द्वे । \newline
54. द्वे अ॒क्षरे॑ अ॒क्षरे॒ द्वे द्वे अ॒क्षरे᳚ । \newline
55. द्वे इति॒ द्वे । \newline
56. अ॒क्षरे॑ अमीयेता ममीयेता म॒क्षरे॑ अ॒क्षरे॑ अमीयेताम् । \newline
57. अ॒क्षरे॒ इत्य॒क्षरे᳚ । \newline
58. अ॒मी॒ये॒ताꣳ॒॒ सा सा ऽमी॑येता ममीयेताꣳ॒॒ सा । \newline
59. सा दक्षि॑णाभि॒र् दक्षि॑णाभिः॒ सा सा दक्षि॑णाभिः । \newline
60. दक्षि॑णाभि श्च च॒ दक्षि॑णाभि॒र् दक्षि॑णाभि श्च । \newline
61. च॒ तप॑सा॒ तप॑सा च च॒ तप॑सा । \newline

\textbf{Ghana Paata } \newline

1. इति॑ मा॒ मेतीति॑ मा क॒द्रूः क॒द्रूर् मेतीति॑ मा क॒द्रूः । \newline
2. मा॒ क॒द्रूः क॒द्रूर् मा॑ मा क॒द्रू र॑वोच दवोचत् क॒द्रूर् मा॑ मा क॒द्रू र॑वोचत् । \newline
3. क॒द्रू र॑वोच दवोचत् क॒द्रूः क॒द्रू र॑वोच॒ दिती त्य॑वोचत् क॒द्रूः क॒द्रू र॑वोच॒ दिति॑ । \newline
4. अ॒वो॒च॒ दिती त्य॑वोच दवोच॒ दिति॒ जग॑ती॒ जग॒ती त्य॑वोच दवोच॒ दिति॒ जग॑ती । \newline
5. इति॒ जग॑ती॒ जग॒ती तीति॒ जग॒ त्युदुज् जग॒ती तीति॒ जग॒ त्युत् । \newline
6. जग॒ त्युदुज् जग॑ती॒ जग॒ त्युद॑पत दपत॒ दुज् जग॑ती॒ जग॒ त्युद॑पतत् । \newline
7. उद॑पत दपत॒ दुदु द॑पत॒च् चतु॑र्दशाक्षरा॒ चतु॑र्दशाक्षरा ऽपत॒ दुदु द॑पत॒च् चतु॑र्दशाक्षरा । \newline
8. अ॒प॒त॒च् चतु॑र्दशाक्षरा॒ चतु॑र्दशाक्षरा ऽपत दपत॒च् चतु॑र्दशाक्षरा स॒ती स॒ती चतु॑र्दशाक्षरा ऽपत दपत॒च् चतु॑र्दशाक्षरा स॒ती । \newline
9. चतु॑र्दशाक्षरा स॒ती स॒ती चतु॑र्दशाक्षरा॒ चतु॑र्दशाक्षरा स॒ती सा सा स॒ती चतु॑र्दशाक्षरा॒ चतु॑र्दशाक्षरा स॒ती सा । \newline
10. चतु॑र्दशाक्ष॒रेति॒ चतु॑र्दश - अ॒क्ष॒रा॒ । \newline
11. स॒ती सा सा स॒ती स॒ती सा ऽप्रा॒प्या प्रा᳚प्य॒ सा स॒ती स॒ती सा ऽप्रा᳚प्य । \newline
12. सा ऽप्रा॒प्या प्रा᳚प्य॒ सा सा ऽप्रा᳚प्य॒ नि न्य प्रा᳚प्य॒ सा सा ऽप्रा᳚प्य॒ नि । \newline
13. अप्रा᳚प्य॒ नि न्य प्रा॒प्या प्रा᳚प्य॒ न्य॑वर्तता वर्तत॒ न्य प्रा॒प्या प्रा᳚प्य॒ न्य॑वर्तत । \newline
14. अप्रा॒प्येत्यप्र॑ - आ॒प्य॒ । \newline
15. न्य॑वर्तता वर्तत॒ नि न्य॑वर्तत॒ तस्यै॒ तस्या॑ अवर्तत॒ नि न्य॑वर्तत॒ तस्यै᳚ । \newline
16. अ॒व॒र्त॒त॒ तस्यै॒ तस्या॑ अवर्तता वर्तत॒ तस्यै॒ द्वे द्वे तस्या॑ अवर्तता वर्तत॒ तस्यै॒ द्वे । \newline
17. तस्यै॒ द्वे द्वे तस्यै॒ तस्यै॒ द्वे अ॒क्षरे॑ अ॒क्षरे॒ द्वे तस्यै॒ तस्यै॒ द्वे अ॒क्षरे᳚ । \newline
18. द्वे अ॒क्षरे॑ अ॒क्षरे॒ द्वे द्वे अ॒क्षरे॑ अमीयेता ममीयेता म॒क्षरे॒ द्वे द्वे अ॒क्षरे॑ अमीयेताम् । \newline
19. द्वे इति॒ द्वे । \newline
20. अ॒क्षरे॑ अमीयेता ममीयेता म॒क्षरे॑ अ॒क्षरे॑ अमीयेताꣳ॒॒ सा सा ऽमी॑येता म॒क्षरे॑ अ॒क्षरे॑ अमीयेताꣳ॒॒ सा । \newline
21. अ॒क्षरे॒ इत्य॒क्षरे᳚ । \newline
22. अ॒मी॒ये॒ताꣳ॒॒ सा सा ऽमी॑येता ममीयेताꣳ॒॒ सा प॒शुभिः॑ प॒शुभिः॒ सा ऽमी॑येता ममीयेताꣳ॒॒ सा प॒शुभिः॑ । \newline
23. सा प॒शुभिः॑ प॒शुभिः॒ सा सा प॒शुभि॑श्च च प॒शुभिः॒ सा सा प॒शुभि॑श्च । \newline
24. प॒शुभि॑श्च च प॒शुभिः॑ प॒शुभि॑श्च दी॒क्षया॑ दी॒क्षया॑ च प॒शुभिः॑ प॒शुभि॑श्च दी॒क्षया᳚ । \newline
25. प॒शुभि॒रिति॑ प॒शु - भिः॒ । \newline
26. च॒ दी॒क्षया॑ दी॒क्षया॑ च च दी॒क्षया॑ च च दी॒क्षया॑ च च दी॒क्षया॑ च । \newline
27. दी॒क्षया॑ च च दी॒क्षया॑ दी॒क्षया॒ चा च॑ दी॒क्षया॑ दी॒क्षया॒ चा । \newline
28. चा च॒ चा ऽग॑च्छ दगच्छ॒दा च॒ चा ऽग॑च्छत् । \newline
29. आ ऽग॑च्छ दगच्छ॒दा ऽग॑च्छ॒त् तस्मा॒त् तस्मा॑ दगच्छ॒दा ऽग॑च्छ॒त् तस्मा᳚त् । \newline
30. अ॒ग॒च्छ॒त् तस्मा॒त् तस्मा॑ दगच्छ दगच्छ॒त् तस्मा॒ज् जग॑ती॒ जग॑ती॒ तस्मा॑ दगच्छ दगच्छ॒त् तस्मा॒ज् जग॑ती । \newline
31. तस्मा॒ज् जग॑ती॒ जग॑ती॒ तस्मा॒त् तस्मा॒ज् जग॑ती॒ छन्द॑सा॒म् छन्द॑सा॒म् जग॑ती॒ तस्मा॒त् तस्मा॒ज् जग॑ती॒ छन्द॑साम् । \newline
32. जग॑ती॒ छन्द॑सा॒म् छन्द॑सा॒म् जग॑ती॒ जग॑ती॒ छन्द॑साम् पश॒व्य॑तमा पश॒व्य॑तमा॒ छन्द॑सा॒म् जग॑ती॒ जग॑ती॒ छन्द॑साम् पश॒व्य॑तमा । \newline
33. छन्द॑साम् पश॒व्य॑तमा पश॒व्य॑तमा॒ छन्द॑सा॒म् छन्द॑साम् पश॒व्य॑तमा॒ तस्मा॒त् तस्मा᳚त् पश॒व्य॑तमा॒ छन्द॑सा॒म् छन्द॑साम् पश॒व्य॑तमा॒ तस्मा᳚त् । \newline
34. प॒श॒व्य॑तमा॒ तस्मा॒त् तस्मा᳚त् पश॒व्य॑तमा पश॒व्य॑तमा॒ तस्मा᳚त् पशु॒मन्त॑म् पशु॒मन्त॒म् तस्मा᳚त् पश॒व्य॑तमा पश॒व्य॑तमा॒ तस्मा᳚त् पशु॒मन्त᳚म् । \newline
35. प॒श॒व्य॑त॒मेति॑ पश॒व्य॑ - त॒मा॒ । \newline
36. तस्मा᳚त् पशु॒मन्त॑म् पशु॒मन्त॒म् तस्मा॒त् तस्मा᳚त् पशु॒मन्त॑म् दी॒क्षा दी॒क्षा प॑शु॒मन्त॒म् तस्मा॒त् तस्मा᳚त् पशु॒मन्त॑म् दी॒क्षा । \newline
37. प॒शु॒मन्त॑म् दी॒क्षा दी॒क्षा प॑शु॒मन्त॑म् पशु॒मन्त॑म् दी॒क्षो पोप॑ दी॒क्षा प॑शु॒मन्त॑म् पशु॒मन्त॑म् दी॒क्षोप॑ । \newline
38. प॒शु॒मन्त॒मिति॑ पशु - मन्त᳚म् । \newline
39. दी॒क्षो पोप॑ दी॒क्षा दी॒क्षोप॑ नमति नम॒ त्युप॑ दी॒क्षा दी॒क्षोप॑ नमति । \newline
40. उप॑ नमति नम॒ त्युपोप॑ नमति त्रि॒ष्टुक् त्रि॒ष्टुङ् न॑म॒ त्युपोप॑ नमति त्रि॒ष्टुक् । \newline
41. न॒म॒ति॒ त्रि॒ष्टुक् त्रि॒ष्टुङ् न॑मति नमति त्रि॒ष्टुगुदुत् त्रि॒ष्टुङ् न॑मति नमति त्रि॒ष्टुगुत् । \newline
42. त्रि॒ष्टु गुदुत् त्रि॒ष्टुक् त्रि॒ष्टु गुद॑पत दपत॒ दुत् त्रि॒ष्टुक् त्रि॒ष्टु गुद॑पतत् । \newline
43. उद॑पत दपत॒ दुदु द॑पत॒त् त्रयो॑दशाक्षरा॒ त्रयो॑दशाक्षरा ऽपत॒ दुदु द॑पत॒त् त्रयो॑दशाक्षरा । \newline
44. अ॒प॒त॒त् त्रयो॑दशाक्षरा॒ त्रयो॑दशाक्षरा ऽपत दपत॒त् त्रयो॑दशाक्षरा स॒ती स॒ती त्रयो॑दशाक्षरा ऽपत दपत॒त् त्रयो॑दशाक्षरा स॒ती । \newline
45. त्रयो॑दशाक्षरा स॒ती स॒ती त्रयो॑दशाक्षरा॒ त्रयो॑दशाक्षरा स॒ती सा सा स॒ती त्रयो॑दशाक्षरा॒ त्रयो॑दशाक्षरा स॒ती सा । \newline
46. त्रयो॑दशाक्ष॒रेति॒ त्रयो॑दश - अ॒क्ष॒रा॒ । \newline
47. स॒ती सा सा स॒ती स॒ती सा ऽप्रा॒प्या प्रा᳚प्य॒ सा स॒ती स॒ती सा ऽप्रा᳚प्य । \newline
48. सा ऽप्रा॒प्या प्रा᳚प्य॒ सा सा ऽप्रा᳚प्य॒ नि न्यप्रा᳚प्य॒ सा सा ऽप्रा᳚प्य॒ नि । \newline
49. अप्रा᳚प्य॒ नि न्यप्रा॒प्या प्रा᳚प्य॒ न्य॑वर्तता वर्तत॒ न्यप्रा॒प्या प्रा᳚प्य॒ न्य॑वर्तत । \newline
50. अप्रा॒प्येत्यप्र॑ - आ॒प्य॒ । \newline
51. न्य॑वर्तता वर्तत॒ नि न्य॑वर्तत॒ तस्यै॒ तस्या॑ अवर्तत॒ नि न्य॑वर्तत॒ तस्यै᳚ । \newline
52. अ॒व॒र्त॒त॒ तस्यै॒ तस्या॑ अवर्तता वर्तत॒ तस्यै॒ द्वे द्वे तस्या॑ अवर्तता वर्तत॒ तस्यै॒ द्वे । \newline
53. तस्यै॒ द्वे द्वे तस्यै॒ तस्यै॒ द्वे अ॒क्षरे॑ अ॒क्षरे॒ द्वे तस्यै॒ तस्यै॒ द्वे अ॒क्षरे᳚ । \newline
54. द्वे अ॒क्षरे॑ अ॒क्षरे॒ द्वे द्वे अ॒क्षरे॑ अमीयेता ममीयेता म॒क्षरे॒ द्वे द्वे अ॒क्षरे॑ अमीयेताम् । \newline
55. द्वे इति॒ द्वे । \newline
56. अ॒क्षरे॑ अमीयेता ममीयेता म॒क्षरे॑ अ॒क्षरे॑ अमीयेताꣳ॒॒ सा सा ऽमी॑येता म॒क्षरे॑ अ॒क्षरे॑ अमीयेताꣳ॒॒ सा । \newline
57. अ॒क्षरे॒ इत्य॒क्षरे᳚ । \newline
58. अ॒मी॒ये॒ताꣳ॒॒ सा सा ऽमी॑येता ममीयेताꣳ॒॒ सा दक्षि॑णाभि॒र् दक्षि॑णाभिः॒ सा ऽमी॑येता ममीयेताꣳ॒॒ सा दक्षि॑णाभिः । \newline
59. सा दक्षि॑णाभि॒र् दक्षि॑णाभिः॒ सा सा दक्षि॑णाभि श्च च॒ दक्षि॑णाभिः॒ सा सा दक्षि॑णाभि श्च । \newline
60. दक्षि॑णाभि श्च च॒ दक्षि॑णाभि॒र् दक्षि॑णाभि श्च॒ तप॑सा॒ तप॑सा च॒ दक्षि॑णाभि॒र् दक्षि॑णाभि श्च॒ तप॑सा । \newline
61. च॒ तप॑सा॒ तप॑सा च च॒ तप॑सा च च॒ तप॑सा च च॒ तप॑सा च । \newline
\pagebreak
\markright{ TS 6.1.6.3  \hfill https://www.vedavms.in \hfill}

\section{ TS 6.1.6.3 }

\textbf{TS 6.1.6.3 } \newline
\textbf{Samhita Paata} \newline

तप॑सा॒ चाऽग॑च्छ॒त् तस्मा᳚त् त्रि॒ष्टुभो॑ लो॒के माद्ध्य॑न्दिने॒ सव॑ने॒ दक्षि॑णा नीयन्त ए॒तत् खलु॒ वाव तप॒ इत्या॑हु॒र्यः स्वं ददा॒तीति॑ गाय॒त्र्युद॑पत॒च्चतु॑रक्षरा स॒त्य॑जया॒ ज्योति॑षा॒ तम॑स्या अ॒जाऽभ्य॑रुन्ध॒ तद॒जाया॑ अज॒त्वꣳ सा सोमं॒ चाऽऽह॑रच्च॒त्वारि॑ चा॒क्षरा॑णि॒ साऽष्टाक्ष॑रा॒ सम॑पद्यत ब्रह्मवा॒दिनो॑ वदन्ति॒ - [  ] \newline

\textbf{Pada Paata} \newline

तप॑सा । च॒ । एति॑ । अ॒ग॒च्छ॒त् । तस्मा᳚त् । त्रि॒ष्टुभः॑ । लो॒के । माद्ध्य॑न्दिने । सव॑ने । दक्षि॑णाः । नी॒य॒न्ते॒ । ए॒तत् । खलु॑ । वाव । तपः॑ । इति॑ । आ॒हुः॒ । यः । स्वम् । ददा॑ति । इति॑ । गा॒य॒त्री । उदिति॑ । अ॒प॒त॒त् । चतु॑रक्ष॒रेति॒ चतुः॑ - अ॒क्ष॒रा॒ । स॒ती । अ॒जया᳚ । ज्योति॑षा । तम् । अ॒स्यै॒ । अ॒जा । अ॒भीति॑ । अ॒रु॒न्ध॒ । तत् । अ॒जायाः᳚ । अ॒ज॒त्वमित्य॑ज-त्वम् । सा । सोम᳚म् । च॒ । एति॑ । अह॑रत् । च॒त्वारि॑ । च॒ । अ॒क्षरा॑णि । सा । अ॒ष्टाक्ष॒रेत्य॒ष्टा - अ॒क्ष॒रा॒ । समिति॑ । अ॒प॒द्य॒त॒ । ब्र॒ह्म॒वा॒दिन॒ इति॑ ब्रह्म - वा॒दिनः॑ । व॒द॒न्ति॒ ।  \newline


\textbf{Krama Paata} \newline

तप॑सा च । चा । आऽग॑च्छत् । अ॒ग॒च्छ॒त् तस्मा᳚त् । तस्मा᳚त् त्रि॒ष्टुभः॑ । त्रि॒ष्टुभो॑ लो॒के । लो॒के माद्ध्य॑न्दिने । माद्ध्य॑न्दिने॒ सव॑ने । सव॑ने॒ दक्षि॑णाः । दक्षि॑णा नीयन्ते । नी॒य॒न्त॒ ए॒तत् । ए॒तत् खलु॑ । खलु॒ वाव । वाव तपः॑ । तप॒ इति॑ । इत्या॑हुः । आ॒हु॒र् यः । यः स्वम् । स्वम् ददा॑ति । ददा॒तीति॑ । इति॑ गाय॒त्री । गा॒य॒त्र्युत् । उद॑पतत् । अ॒प॒त॒च् चतु॑रक्षरा । चतु॑रक्षरा स॒ती । चतु॑रक्ष॒रेति॒ चतुः॑ - अ॒क्ष॒रा॒ । स॒त्य॑जया᳚ । अ॒जया॒ ज्योति॑षा । ज्योति॑षा॒ तम् । तम॑स्यै । अ॒स्या॒ अ॒जा । अ॒जाऽभि । अ॒भ्य॑रुन्ध । अ॒रु॒न्ध॒ तत् । तद॒जायाः᳚ । अ॒जाया॑ अज॒त्वम् । अ॒ज॒त्वꣳ सा । अ॒ज॒त्वमित्य॑ज - त्वम् । सा सोम᳚म् । सोम॑म् च । चा । आऽह॑रत् । अह॑रच् च॒त्वारि॑ । च॒त्वारि॑ च । चा॒क्षरा॑णि । अ॒क्षरा॑णि॒ सा । साऽष्टाक्ष॑रा । अ॒ष्टाक्ष॑रा॒ सम् । अ॒ष्टाक्ष॒रेत्य॒ष्टा - अ॒क्ष॒रा॒ । सम॑पद्यत । अ॒प॒द्य॒त॒ ब्र॒ह्म॒वा॒दिनः॑ । ब्र॒ह्म॒वा॒दिनो॑ वदन्ति । ब्र॒ह्म॒वा॒दिन॒ इति॑ ब्रह्म - वा॒दिनः॑ । व॒द॒न्ति॒ कस्मा᳚त् \newline

\textbf{Jatai Paata} \newline

1. तप॑सा च च॒ तप॑सा॒ तप॑सा च । \newline
2. चा च॒ चा । \newline
3. आ ऽग॑च्छ दगच्छ॒दा ऽग॑च्छत् । \newline
4. अ॒ग॒च्छ॒त् तस्मा॒त् तस्मा॑ दगच्छ दगच्छ॒त् तस्मा᳚त् । \newline
5. तस्मा᳚त् त्रि॒ष्टुभ॑ स्त्रि॒ष्टुभ॒ स्तस्मा॒त् तस्मा᳚त् त्रि॒ष्टुभः॑ । \newline
6. त्रि॒ष्टुभो॑ लो॒के लो॒के त्रि॒ष्टुभ॑ स्त्रि॒ष्टुभो॑ लो॒के । \newline
7. लो॒के माद्ध्य॑न्दिने॒ माद्ध्य॑न्दिने लो॒के लो॒के माद्ध्य॑न्दिने । \newline
8. माद्ध्य॑न्दिने॒ सव॑ने॒ सव॑ने॒ माद्ध्य॑न्दिने॒ माद्ध्य॑न्दिने॒ सव॑ने । \newline
9. सव॑ने॒ दक्षि॑णा॒ दक्षि॑णाः॒ सव॑ने॒ सव॑ने॒ दक्षि॑णाः । \newline
10. दक्षि॑णा नीयन्ते नीयन्ते॒ दक्षि॑णा॒ दक्षि॑णा नीयन्ते । \newline
11. नी॒य॒न्त॒ ए॒त दे॒तन् नी॑यन्ते नीयन्त ए॒तत् । \newline
12. ए॒तत् खलु॒ खल्वे॒त दे॒तत् खलु॑ । \newline
13. खलु॒ वाव वाव खलु॒ खलु॒ वाव । \newline
14. वाव तप॒ स्तपो॒ वाव वाव तपः॑ । \newline
15. तप॒ इतीति॒ तप॒ स्तप॒ इति॑ । \newline
16. इत्या॑हु राहु॒रि तीत्या॑हुः । \newline
17. आ॒हु॒र् यो य आ॑हु राहु॒र् यः । \newline
18. यः स्वꣳ स्वं ॅयो यः स्वम् । \newline
19. स्वम् ददा॑ति॒ ददा॑ति॒ स्वꣳ स्वम् ददा॑ति । \newline
20. ददा॒ती तीति॒ ददा॑ति॒ ददा॒तीति॑ । \newline
21. इति॑ गाय॒त्री गा॑य॒त्री तीति॑ गाय॒त्री । \newline
22. गा॒य॒ त्र्युदुद् गा॑य॒त्री गा॑य॒ त्र्युत् । \newline
23. उद॑पत दपत॒ दुदु द॑पतत् । \newline
24. अ॒प॒त॒च् चतु॑रक्षरा॒ चतु॑रक्षरा ऽपत दपत॒च् चतु॑रक्षरा । \newline
25. चतु॑रक्षरा स॒ती स॒ती चतु॑रक्षरा॒ चतु॑रक्षरा स॒ती । \newline
26. चतु॑रक्ष॒रेति॒ चतुः॑ - अ॒क्ष॒रा॒ । \newline
27. स॒त्य॑ जया॒ ऽजया॑ स॒ती स॒त्य॑ जया᳚ । \newline
28. अ॒जया॒ ज्योति॑षा॒ ज्योति॑षा॒ ऽजया॒ ऽजया॒ ज्योति॑षा । \newline
29. ज्योति॑षा॒ तम् तम् ज्योति॑षा॒ ज्योति॑षा॒ तम् । \newline
30. त म॑स्या अस्यै॒ तम् त म॑स्यै । \newline
31. अ॒स्या॒ अ॒जा ऽजा ऽस्या॑ अस्या अ॒जा । \newline
32. अ॒जा ऽभ्या᳚(1॒)भ्य॑जा ऽजा ऽभि । \newline
33. अ॒भ्य॑ रुन्धा रुन्धा॒ भ्या᳚(1॒)भ्य॑ रुन्ध । \newline
34. अ॒रु॒न्ध॒ तत् तद॑रुन्धा रुन्ध॒ तत् । \newline
35. तद॒जाया॑ अ॒जाया॒ स्तत् तद॒जायाः᳚ । \newline
36. अ॒जाया॑ अज॒त्व म॑ज॒त्व म॒जाया॑ अ॒जाया॑ अज॒त्वम् । \newline
37. अ॒ज॒त्वꣳ सा सा ऽज॒त्व म॑ज॒त्वꣳ सा । \newline
38. अ॒ज॒त्वमित्य॑ज - त्वम् । \newline
39. सा सोमꣳ॒॒ सोमꣳ॒॒ सा सा सोम᳚म् । \newline
40. सोम॑म् च च॒ सोमꣳ॒॒ सोम॑म् च । \newline
41. चा च॒ चा । \newline
42. आ ऽह॑र॒ दह॑र॒दा ऽह॑रत् । \newline
43. अह॑रच् च॒त्वारि॑ च॒त्वा र्यह॑र॒ दह॑रच् च॒त्वारि॑ । \newline
44. च॒त्वारि॑ च च च॒त्वारि॑ च॒त्वारि॑ च । \newline
45. चा॒क्षरा᳚ ण्य॒क्षरा॑णि च चा॒क्षरा॑णि । \newline
46. अ॒क्षरा॑णि॒ सा सा ऽक्षरा᳚ ण्य॒क्षरा॑णि॒ सा । \newline
47. सा ऽष्टाक्ष॑रा॒ ऽष्टाक्ष॑रा॒ सा सा ऽष्टाक्ष॑रा । \newline
48. अ॒ष्टाक्ष॑रा॒ सꣳ स म॒ष्टाक्ष॑रा॒ ऽष्टाक्ष॑रा॒ सम् । \newline
49. अ॒ष्टाक्ष॒रेत्य॒ष्टा - अ॒क्ष॒रा॒ । \newline
50. स म॑पद्यता पद्यत॒ सꣳ स म॑पद्यत । \newline
51. अ॒प॒द्य॒त॒ ब्र॒ह्म॒वा॒दिनो᳚ ब्रह्मवा॒दिनो॑ ऽपद्यता पद्यत ब्रह्मवा॒दिनः॑ । \newline
52. ब्र॒ह्म॒वा॒दिनो॑ वदन्ति वदन्ति ब्रह्मवा॒दिनो᳚ ब्रह्मवा॒दिनो॑ वदन्ति । \newline
53. ब्र॒ह्म॒वा॒दिन॒ इति॑ ब्रह्म - वा॒दिनः॑ । \newline
54. व॒द॒न्ति॒ कस्मा॒त् कस्मा᳚द् वदन्ति वदन्ति॒ कस्मा᳚त् । \newline

\textbf{Ghana Paata } \newline

1. तप॑सा च च॒ तप॑सा॒ तप॑सा॒ चा च॒ तप॑सा॒ तप॑सा॒ चा । \newline
2. चा च॒ चा ऽग॑च्छ दगच्छ॒ दा च॒ चा ऽग॑च्छत् । \newline
3. आ ऽग॑च्छ दगच्छ॒ दा ऽग॑च्छ॒त् तस्मा॒त् तस्मा॑ दगच्छ॒ दा ऽग॑च्छ॒त् तस्मा᳚त् । \newline
4. अ॒ग॒च्छ॒त् तस्मा॒त् तस्मा॑ दगच्छ दगच्छ॒त् तस्मा᳚त् त्रि॒ष्टुभ॑ स्त्रि॒ष्टुभ॒ स्तस्मा॑ दगच्छ दगच्छ॒त् तस्मा᳚त् त्रि॒ष्टुभः॑ । \newline
5. तस्मा᳚त् त्रि॒ष्टुभ॑ स्त्रि॒ष्टुभ॒ स्तस्मा॒त् तस्मा᳚त् त्रि॒ष्टुभो॑ लो॒के लो॒के त्रि॒ष्टुभ॒ स्तस्मा॒त् तस्मा᳚त् त्रि॒ष्टुभो॑ लो॒के । \newline
6. त्रि॒ष्टुभो॑ लो॒के लो॒के त्रि॒ष्टुभ॑ स्त्रि॒ष्टुभो॑ लो॒के माद्ध्य॑न्दिने॒ माद्ध्य॑न्दिने लो॒के त्रि॒ष्टुभ॑ स्त्रि॒ष्टुभो॑ लो॒के माद्ध्य॑न्दिने । \newline
7. लो॒के माद्ध्य॑न्दिने॒ माद्ध्य॑न्दिने लो॒के लो॒के माद्ध्य॑न्दिने॒ सव॑ने॒ सव॑ने॒ माद्ध्य॑न्दिने लो॒के लो॒के माद्ध्य॑न्दिने॒ सव॑ने । \newline
8. माद्ध्य॑न्दिने॒ सव॑ने॒ सव॑ने॒ माद्ध्य॑न्दिने॒ माद्ध्य॑न्दिने॒ सव॑ने॒ दक्षि॑णा॒ दक्षि॑णाः॒ सव॑ने॒ माद्ध्य॑न्दिने॒ माद्ध्य॑न्दिने॒ सव॑ने॒ दक्षि॑णाः । \newline
9. सव॑ने॒ दक्षि॑णा॒ दक्षि॑णाः॒ सव॑ने॒ सव॑ने॒ दक्षि॑णा नीयन्ते नीयन्ते॒ दक्षि॑णाः॒ सव॑ने॒ सव॑ने॒ दक्षि॑णा नीयन्ते । \newline
10. दक्षि॑णा नीयन्ते नीयन्ते॒ दक्षि॑णा॒ दक्षि॑णा नीयन्त ए॒त दे॒तन् नी॑यन्ते॒ दक्षि॑णा॒ दक्षि॑णा नीयन्त ए॒तत् । \newline
11. नी॒य॒न्त॒ ए॒त दे॒तन् नी॑यन्ते नीयन्त ए॒तत् खलु॒ खल्वे॒तन् नी॑यन्ते नीयन्त ए॒तत् खलु॑ । \newline
12. ए॒तत् खलु॒ खल्वे॒त दे॒तत् खलु॒ वाव वाव खल्वे॒त दे॒तत् खलु॒ वाव । \newline
13. खलु॒ वाव वाव खलु॒ खलु॒ वाव तप॒ स्तपो॒ वाव खलु॒ खलु॒ वाव तपः॑ । \newline
14. वाव तप॒ स्तपो॒ वाव वाव तप॒ इतीति॒ तपो॒ वाव वाव तप॒ इति॑ । \newline
15. तप॒ इतीति॒ तप॒ स्तप॒ इत्या॑हु राहु॒ रिति॒ तप॒ स्तप॒ इत्या॑हुः । \newline
16. इत्या॑हु राहु॒ रिती त्या॑हु॒र् यो य आ॑हु॒ रिती त्या॑हु॒र् यः । \newline
17. आ॒हु॒र् यो य आ॑हु राहु॒र् यः स्वꣳ स्वं ॅय आ॑हु राहु॒र् यः स्वम् । \newline
18. यः स्वꣳ स्वं ॅयो यः स्वम् ददा॑ति॒ ददा॑ति॒ स्वं ॅयो यः स्वम् ददा॑ति । \newline
19. स्वम् ददा॑ति॒ ददा॑ति॒ स्वꣳ स्वम् ददा॒ती तीति॒ ददा॑ति॒ स्वꣳ स्वम् ददा॒तीति॑ । \newline
20. ददा॒ती तीति॒ ददा॑ति॒ ददा॒तीति॑ गाय॒त्री गा॑य॒त्रीति॒ ददा॑ति॒ ददा॒तीति॑ गाय॒त्री । \newline
21. इति॑ गाय॒त्री गा॑य॒त्री तीति॑ गाय॒ त्र्युदुद् गा॑य॒त्री तीति॑ गाय॒ त्र्युत् । \newline
22. गा॒य॒त्र्युदुद् गा॑य॒त्री गा॑य॒ त्र्युद॑पत दपत॒ दुद् गा॑य॒त्री गा॑य॒ त्र्युद॑पतत् । \newline
23. उद॑पत दपत॒ दुदु द॑पत॒च् चतु॑रक्षरा॒ चतु॑रक्षरा ऽपत॒ दुदु द॑पत॒च् चतु॑रक्षरा । \newline
24. अ॒प॒त॒च् चतु॑रक्षरा॒ चतु॑रक्षरा ऽपत दपत॒च् चतु॑रक्षरा स॒ती स॒ती चतु॑रक्षरा ऽपत दपत॒च् चतु॑रक्षरा स॒ती । \newline
25. चतु॑रक्षरा स॒ती स॒ती चतु॑रक्षरा॒ चतु॑रक्षरा स॒त्य॑जया॒ ऽजया॑ स॒ती चतु॑रक्षरा॒ चतु॑रक्षरा स॒त्य॑जया᳚ । \newline
26. चतु॑रक्ष॒रेति॒ चतुः॑ - अ॒क्ष॒रा॒ । \newline
27. स॒त्य॑जया॒ ऽजया॑ स॒ती स॒त्य॑जया॒ ज्योति॑षा॒ ज्योति॑षा॒ ऽजया॑ स॒ती स॒त्य॑जया॒ ज्योति॑षा । \newline
28. अ॒जया॒ ज्योति॑षा॒ ज्योति॑षा॒ ऽजया॒ ऽजया॒ ज्योति॑षा॒ तम् तम् ज्योति॑षा॒ ऽजया॒ ऽजया॒ ज्योति॑षा॒ तम् । \newline
29. ज्योति॑षा॒ तम् तम् ज्योति॑षा॒ ज्योति॑षा॒ त म॑स्या अस्यै॒ तम् ज्योति॑षा॒ ज्योति॑षा॒ त म॑स्यै । \newline
30. त म॑स्या अस्यै॒ तम् त म॑स्या अ॒जा ऽजा ऽस्यै॒ तम् त म॑स्या अ॒जा । \newline
31. अ॒स्या॒ अ॒जा ऽजा ऽस्या॑ अस्या अ॒जा ऽभ्या᳚(1॒)भ्य॑जा ऽस्या॑ अस्या अ॒जा ऽभि । \newline
32. अ॒जा ऽभ्या᳚(1॒)भ्य॑जा ऽजा ऽभ्य॑ रुन्धा रुन्धा॒ भ्य॑जा ऽजा ऽभ्य॑ रुन्ध । \newline
33. अ॒भ्य॑ रुन्धा रुन्धा॒भ्या᳚(1॒) भ्य॑ रुन्ध॒ तत् तद॑रुन्धा॒ भ्या᳚(1॒)भ्य॑ रुन्ध॒ तत् । \newline
34. अ॒रु॒न्ध॒ तत् तद॑रुन्धा रुन्ध॒ तद॒जाया॑ अ॒जाया॒ स्त द॑रुन्धा रुन्ध॒ तद॒जायाः᳚ । \newline
35. तद॒जाया॑ अ॒जाया॒ स्तत् तद॒जाया॑ अज॒त्व म॑ज॒त्व म॒जाया॒ स्तत् तद॒जाया॑ अज॒त्वम् । \newline
36. अ॒जाया॑ अज॒त्व म॑ज॒त्व म॒जाया॑ अ॒जाया॑ अज॒त्वꣳ सा सा ऽज॒त्व म॒जाया॑ अ॒जाया॑ अज॒त्वꣳ सा । \newline
37. अ॒ज॒त्वꣳ सा सा ऽज॒त्व म॑ज॒त्वꣳ सा सोमꣳ॒॒ सोमꣳ॒॒ सा ऽज॒त्व म॑ज॒त्वꣳ सा सोम᳚म् । \newline
38. अ॒ज॒त्वमित्य॑ज - त्वम् । \newline
39. सा सोमꣳ॒॒ सोमꣳ॒॒ सा सा सोम॑म् च च॒ सोमꣳ॒॒ सा सा सोम॑म् च । \newline
40. सोम॑म् च च॒ सोमꣳ॒॒ सोम॒म् चा च॒ सोमꣳ॒॒ सोम॒म् चा । \newline
41. चा च॒ चा ऽह॑र॒ दह॑र॒ दा च॒ चा ऽह॑रत् । \newline
42. आ ऽह॑र॒ दह॑र॒ दा ऽह॑रच् च॒त्वारि॑ च॒त्वा र्यह॑र॒दा ऽह॑रच् च॒त्वारि॑ । \newline
43. अह॑रच् च॒त्वारि॑ च॒त्वा र्यह॑र॒ दह॑रच् च॒त्वारि॑ च च च॒त्वा र्यह॑र॒ दह॑रच् च॒त्वारि॑ च । \newline
44. च॒त्वारि॑ च च च॒त्वारि॑ च॒त्वारि॑ चा॒क्षरा᳚ ण्य॒क्षरा॑णि च च॒त्वारि॑ च॒त्वारि॑ चा॒क्षरा॑णि । \newline
45. चा॒क्षरा᳚ ण्य॒क्षरा॑णि च चा॒क्षरा॑णि॒ सा सा ऽक्षरा॑णि च चा॒क्षरा॑णि॒ सा । \newline
46. अ॒क्षरा॑णि॒ सा सा ऽक्षरा᳚ ण्य॒क्षरा॑णि॒ सा ऽष्टाक्ष॑रा॒ ऽष्टाक्ष॑रा॒ सा ऽक्षरा᳚ ण्य॒क्षरा॑णि॒ सा ऽष्टाक्ष॑रा । \newline
47. सा ऽष्टाक्ष॑रा॒ ऽष्टाक्ष॑रा॒ सा सा ऽष्टाक्ष॑रा॒ सꣳ स म॒ष्टाक्ष॑रा॒ सा सा ऽष्टाक्ष॑रा॒ सम् । \newline
48. अ॒ष्टाक्ष॑रा॒ सꣳ स म॒ष्टाक्ष॑रा॒ ऽष्टाक्ष॑रा॒ स म॑पद्यता पद्यत॒ स म॒ष्टाक्ष॑रा॒ ऽष्टाक्ष॑रा॒ स म॑पद्यत । \newline
49. अ॒ष्टाक्ष॒रेत्य॒ष्टा - अ॒क्ष॒रा॒ । \newline
50. स म॑पद्यता पद्यत॒ सꣳ स म॑पद्यत ब्रह्मवा॒दिनो᳚ ब्रह्मवा॒दिनो॑ ऽपद्यत॒ सꣳ स म॑पद्यत ब्रह्मवा॒दिनः॑ । \newline
51. अ॒प॒द्य॒त॒ ब्र॒ह्म॒वा॒दिनो᳚ ब्रह्मवा॒दिनो॑ ऽपद्यता पद्यत ब्रह्मवा॒दिनो॑ वदन्ति वदन्ति ब्रह्मवा॒दिनो॑ ऽपद्यता पद्यत ब्रह्मवा॒दिनो॑ वदन्ति । \newline
52. ब्र॒ह्म॒वा॒दिनो॑ वदन्ति वदन्ति ब्रह्मवा॒दिनो᳚ ब्रह्मवा॒दिनो॑ वदन्ति॒ कस्मा॒त् कस्मा᳚द् वदन्ति ब्रह्मवा॒दिनो᳚ ब्रह्मवा॒दिनो॑ वदन्ति॒ कस्मा᳚त् । \newline
53. ब्र॒ह्म॒वा॒दिन॒ इति॑ ब्रह्म - वा॒दिनः॑ । \newline
54. व॒द॒न्ति॒ कस्मा॒त् कस्मा᳚द् वदन्ति वदन्ति॒ कस्मा᳚थ् स॒त्याथ् स॒त्यात् कस्मा᳚द् वदन्ति वदन्ति॒ कस्मा᳚थ् स॒त्यात् । \newline
\pagebreak
\markright{ TS 6.1.6.4  \hfill https://www.vedavms.in \hfill}

\section{ TS 6.1.6.4 }

\textbf{TS 6.1.6.4 } \newline
\textbf{Samhita Paata} \newline

कस्मा᳚थ् स॒त्याद्-गा॑य॒त्री कनि॑ष्ठा॒ छन्द॑साꣳ स॒ती य॑ज्ञ्मु॒खं परी॑या॒येति॒ यदे॒वादः सोम॒माऽह॑र॒त् तस्मा᳚द्-यज्ञ्मु॒खं पर्यै॒त् तस्मा᳚त् तेज॒स्विनी॑तमा प॒द्भ्यां द्वे सव॑ने स॒मगृ॑ह्णा॒न् मुखे॒नैकं॒ ॅयन्मुखे॑न स॒मगृ॑ह्णा॒त् तद॑धय॒त् तस्मा॒द् द्वे सव॑ने शु॒क्र॑वती प्रातस्सव॒नं च॒ माद्ध्य॑न्दिनं च॒ तस्मा᳚त् तृतीय सव॒न ऋ॑जी॒षम॒भि षु॑ण्वन्ति धी॒तमि॑व॒ हि मन्य॑न्त - [  ] \newline

\textbf{Pada Paata} \newline

कस्मा᳚त् । स॒त्यात् । गा॒य॒त्री । कनि॑ष्ठा । छन्द॑साम् । स॒ती । य॒ज्ञ्॒मु॒खमिति॑ यज्ञ् - मु॒खम् । परीति॑ । इ॒या॒य॒ । इति॑ । यत् । ए॒व । अ॒दः । सोम᳚म् । एति॑ । अह॑रत् । तस्मा᳚त् । य॒ज्ञ्॒मु॒खमिति॑ यज्ञ् - मु॒खम् । परीति॑ । ऐ॒त् । तस्मा᳚त् । ते॒ज॒स्विनी॑त॒मेति॑ तेज॒स्विनी᳚-त॒मा॒ । प॒द्भ्यामिति॑ पत्-भ्याम् । द्वे इति॑ । सव॑ने॒ इति॑ । स॒मगृ॑ह्णा॒दिति॑ सं - अगृ॑ह्णात् । मुखे॑न । एक᳚म् । यत् । मुखे॑न । स॒मगृ॑ह्णा॒दिति॑ सं - अगृ॑ह्णात् । तत् । अ॒ध॒य॒त् । तस्मा᳚त् । द्वे इति॑ । सव॑ने॒ इति॑ । शु॒क्रव॑ती॒ इति॑ शु॒क्र - व॒ती॒ । प्रा॒त॒स्स॒व॒नमिति॑ प्रातः - स॒व॒नम् । च॒ । माद्ध्य॑न्दिनम् । च॒ । तस्मा᳚त् । तृ॒ती॒य॒स॒व॒न इति॑ तृतीय - स॒व॒ने । ऋ॒जी॒षम् । अ॒भीति॑ । सु॒न्व॒न्ति॒ । धी॒तम् । इ॒व॒ । हि । मन्य॑न्ते ।  \newline


\textbf{Krama Paata} \newline

कस्मा᳚थ् स॒त्यात् । स॒त्याद् गा॑य॒त्री । गा॒य॒त्री कनि॑ष्ठा । कनि॑ष्ठा॒ छन्द॑साम् । छन्द॑साꣳ स॒ती । स॒ती य॑ज्ञ्मु॒खम् । य॒ज्ञ्॒मु॒खम् परि॑ । य॒ज्ञ्॒मु॒खमिति॑ यज्ञ् - मु॒खम् । परी॑याय । इ॒या॒येति॑ । इति॒ यत् । यदे॒व । ए॒वादः । अ॒दः सोम᳚म् । सोम॒मा । आऽह॑रत् । अह॑र॒त् तस्मा᳚त् । तस्मा᳚द् यज्ञ्मु॒खम् । य॒ज्ञ्॒मु॒खम् परि॑ । य॒ज्ञ्॒मु॒खमिति॑ यज्ञ् - मु॒खम् । पर्यै᳚त् । ऐ॒त् तस्मा᳚त् । तस्मा᳚त् तेज॒स्विनी॑तमा । ते॒ज॒स्विनी॑तमा प॒द्भ्याम् । ते॒ज॒स्विनी॑त॒मेति॑ तेज॒स्विनी᳚ - त॒मा॒ । प॒द्भ्याम् द्वे । प॒द्भ्यामिति॑ पत् - भ्याम् । द्वे सव॑ने । द्वे इति॒ द्वे । सव॑ने स॒मगृ॑ह्णात् । सव॑ने॒ इति॒ सव॑ने । स॒मगृ॑ह्णा॒न् मुखे॑न । स॒मगृ॑ह्णा॒दिति॑ सम् - अगृ॑ह्णात् । मुखे॒नैक᳚म् । एक॒म् ॅयत् । यन् मुखे॑न । मुखे॑न स॒मगृ॑ह्णात् । स॒मगृ॑ह्णा॒त् तत् । स॒मगृ॑ह्णा॒दिति॑ सम् - अगृ॑ह्णात् । तद॑धयत् । अ॒ध॒य॒त् तस्मा᳚त् । तस्मा॒द् द्वे । द्वे सव॑ने । द्वे इति॒ द्वे । सव॑ने शु॒क्रव॑ती । सव॑ने॒ इति॒ सव॑ने । शु॒क्रव॑ती प्रातस्सव॒नम् । शु॒क्रव॑ती॒ इति॑ शु॒क्र - व॒ती॒ । प्रा॒त॒स्स॒व॒नम् च॑ । प्रा॒त॒स्स॒व॒नमिति॑ प्रातः - स॒व॒नम् । च॒ माद्ध्य॑न्दिनम् । माद्ध्य॑न्दिनम् च । च॒ तस्मा᳚त् । तस्मा᳚त् तृतीयसव॒ने । तृ॒ती॒य॒स॒व॒न ऋ॑जी॒षम् । तृ॒ती॒य॒स॒व॒न इति॑ तृतीय - स॒व॒ने । ऋ॒जी॒षम॒भि । अ॒भि षु॑ण्वन्ति । सु॒न्व॒न्ति॒ धी॒तम् । धी॒तमि॑व । इ॒व॒ हि । हि मन्य॑न्ते । मन्य॑न्त आ॒शिर᳚म् \newline

\textbf{Jatai Paata} \newline

1. कस्मा᳚थ् स॒त्याथ् स॒त्यात् कस्मा॒त् कस्मा᳚थ् स॒त्यात् । \newline
2. स॒त्याद् गा॑य॒त्री गा॑य॒त्री स॒त्याथ् स॒त्याद् गा॑य॒त्री । \newline
3. गा॒य॒त्री कनि॑ष्ठा॒ कनि॑ष्ठा गाय॒त्री गा॑य॒त्री कनि॑ष्ठा । \newline
4. कनि॑ष्ठा॒ छन्द॑सा॒म् छन्द॑सा॒म् कनि॑ष्ठा॒ कनि॑ष्ठा॒ छन्द॑साम् । \newline
5. छन्द॑साꣳ स॒ती स॒ती छन्द॑सा॒म् छन्द॑साꣳ स॒ती । \newline
6. स॒ती य॑ज्ञ्मु॒खं ॅय॑ज्ञ्मु॒खꣳ स॒ती स॒ती य॑ज्ञ्मु॒खम् । \newline
7. य॒ज्ञ्॒मु॒खम् परि॒ परि॑ यज्ञ्मु॒खं ॅय॑ज्ञ्मु॒खम् परि॑ । \newline
8. य॒ज्ञ्॒मु॒खमिति॑ यज्ञ् - मु॒खम् । \newline
9. परी॑याये याय॒ परि॒ परी॑याय । \newline
10. इ॒या॒ये तीती॑ याये या॒येति॑ । \newline
11. इति॒ यद् यदितीति॒ यत् । \newline
12. यदे॒ वैव यद् यदे॒व । \newline
13. ए॒वादो॑ ऽद ए॒वै वादः । \newline
14. अ॒दः सोमꣳ॒॒ सोम॑ म॒दो॑ ऽदः सोम᳚म् । \newline
15. सोम॒ मा सोमꣳ॒॒ सोम॒ मा । \newline
16. आ ऽह॑र॒ दह॑र॒दा ऽह॑रत् । \newline
17. अह॑र॒त् तस्मा॒त् तस्मा॒ दह॑र॒ दह॑र॒त् तस्मा᳚त् । \newline
18. तस्मा᳚द् यज्ञ्मु॒खं ॅय॑ज्ञ्मु॒खम् तस्मा॒त् तस्मा᳚द् यज्ञ्मु॒खम् । \newline
19. य॒ज्ञ्॒मु॒खम् परि॒ परि॑ यज्ञ्मु॒खं ॅय॑ज्ञ्मु॒खम् परि॑ । \newline
20. य॒ज्ञ्॒मु॒खमिति॑ यज्ञ् - मु॒खम् । \newline
21. पर्यै॑ दै॒त् परि॒ पर्यै᳚त् । \newline
22. ऐ॒त् तस्मा॒त् तस्मा॑ दैदै॒त् तस्मा᳚त् । \newline
23. तस्मा᳚त् तेज॒स्विनी॑तमा तेज॒स्विनी॑तमा॒ तस्मा॒त् तस्मा᳚त् तेज॒स्विनी॑तमा । \newline
24. ते॒ज॒स्विनी॑तमा प॒द्भ्याम् प॒द्भ्याम् ते॑ज॒स्विनी॑तमा तेज॒स्विनी॑तमा प॒द्भ्याम् । \newline
25. ते॒ज॒स्विनी॑त॒मेति॑ तेज॒स्विनी᳚ - त॒मा॒ । \newline
26. प॒द्भ्याम् द्वे द्वे प॒द्भ्याम् प॒द्भ्याम् द्वे । \newline
27. प॒द्भ्यामिति॑ पत् - भ्याम् । \newline
28. द्वे सव॑ने॒ सव॑ने॒ द्वे द्वे सव॑ने । \newline
29. द्वे इति॒ द्वे । \newline
30. सव॑ने स॒मगृ॑ह्णाथ् स॒मगृ॑ह्णा॒थ् सव॑ने॒ सव॑ने स॒मगृ॑ह्णात् । \newline
31. सव॑ने॒ इति॒ सव॑ने । \newline
32. स॒मगृ॑ह्णा॒न् मुखे॑न॒ मुखे॑न स॒मगृ॑ह्णाथ् स॒मगृ॑ह्णा॒न् मुखे॑न । \newline
33. स॒मगृ॑ह्णा॒दिति॑ सं - अगृ॑ह्णात् । \newline
34. मुखे॒ नैक॒ मेक॒म् मुखे॑न॒ मुखे॒ नैक᳚म् । \newline
35. एकं॒ ॅयद् यदेक॒ मेकं॒ ॅयत् । \newline
36. यन् मुखे॑न॒ मुखे॑न॒ यद् यन् मुखे॑न । \newline
37. मुखे॑न स॒मगृ॑ह्णाथ् स॒मगृ॑ह्णा॒न् मुखे॑न॒ मुखे॑न स॒मगृ॑ह्णात् । \newline
38. स॒मगृ॑ह्णा॒त् तत् तथ् स॒मगृ॑ह्णाथ् स॒मगृ॑ह्णा॒त् तत् । \newline
39. स॒मगृ॑ह्णा॒दिति॑ सं - अगृ॑ह्णात् । \newline
40. तद॑धय दधय॒त् तत् तद॑धयत् । \newline
41. अ॒ध॒य॒त् तस्मा॒त् तस्मा॑ दधय दधय॒त् तस्मा᳚त् । \newline
42. तस्मा॒द् द्वे द्वे तस्मा॒त् तस्मा॒द् द्वे । \newline
43. द्वे सव॑ने॒ सव॑ने॒ द्वे द्वे सव॑ने । \newline
44. द्वे इति॒ द्वे । \newline
45. सव॑ने शु॒क्रव॑ती शु॒क्रव॑ती॒ सव॑ने॒ सव॑ने शु॒क्रव॑ती । \newline
46. सव॑ने॒ इति॒ सव॑ने । \newline
47. शु॒क्रव॑ती प्रातस्सव॒नम् प्रा॑तस्सव॒नꣳ शु॒क्रव॑ती शु॒क्रव॑ती प्रातस्सव॒नम् । \newline
48. शु॒क्रव॑ती॒ इति॑ शु॒क्र - व॒ती॒ । \newline
49. प्रा॒त॒स्स॒व॒नम् च॑ च प्रातस्सव॒नम् प्रा॑तस्सव॒नम् च॑ । \newline
50. प्रा॒त॒स्स॒व॒नमिति॑ प्रातः - स॒व॒नम् । \newline
51. च॒ माद्ध्य॑न्दिन॒म् माद्ध्य॑न्दिनम् च च॒ माद्ध्य॑न्दिनम् । \newline
52. माद्ध्य॑न्दिनम् च च॒ माद्ध्य॑न्दिन॒म् माद्ध्य॑न्दिनम् च । \newline
53. च॒ तस्मा॒त् तस्मा᳚च् च च॒ तस्मा᳚त् । \newline
54. तस्मा᳚त् तृतीयसव॒ने तृ॑तीयसव॒ने तस्मा॒त् तस्मा᳚त् तृतीयसव॒ने । \newline
55. तृ॒ती॒य॒स॒व॒न ऋ॑जी॒ष मृ॑जी॒षम् तृ॑तीयसव॒ने तृ॑तीयसव॒न ऋ॑जी॒षम् । \newline
56. तृ॒ती॒य॒स॒व॒न इति॑ तृतीय - स॒व॒ने । \newline
57. ऋ॒जी॒ष म॒भ्या᳚(1॒) भ्यृ॑जी॒ष मृ॑जी॒ष म॒भि । \newline
58. अ॒भि षु॑न्वन्ति सुन्वन् त्य॒भ्य॑भि षु॑न्वन्ति । \newline
59. सु॒न्व॒न्ति॒ धी॒तम् धी॒तꣳ सु॑न्वन्ति सुन्वन्ति धी॒तम् । \newline
60. धी॒त मि॑वेव धी॒तम् धी॒त मि॑व । \newline
61. इ॒व॒ हि हीवे॑व॒ हि । \newline
62. हि मन्य॑न्ते॒ मन्य॑न्ते॒ हि हि मन्य॑न्ते । \newline
63. मन्य॑न्त आ॒शिर॑ मा॒शिर॒म् मन्य॑न्ते॒ मन्य॑न्त आ॒शिर᳚म् । \newline

\textbf{Ghana Paata } \newline

1. कस्मा᳚थ् स॒त्याथ् स॒त्यात् कस्मा॒त् कस्मा᳚थ् स॒त्याद् गा॑य॒त्री गा॑य॒त्री स॒त्यात् कस्मा॒त् कस्मा᳚थ् स॒त्याद् गा॑य॒त्री । \newline
2. स॒त्याद् गा॑य॒त्री गा॑य॒त्री स॒त्याथ् स॒त्याद् गा॑य॒त्री कनि॑ष्ठा॒ कनि॑ष्ठा गाय॒त्री स॒त्याथ् स॒त्याद् गा॑य॒त्री कनि॑ष्ठा । \newline
3. गा॒य॒त्री कनि॑ष्ठा॒ कनि॑ष्ठा गाय॒त्री गा॑य॒त्री कनि॑ष्ठा॒ छन्द॑सा॒म् छन्द॑सा॒म् कनि॑ष्ठा गाय॒त्री गा॑य॒त्री कनि॑ष्ठा॒ छन्द॑साम् । \newline
4. कनि॑ष्ठा॒ छन्द॑सा॒म् छन्द॑सा॒म् कनि॑ष्ठा॒ कनि॑ष्ठा॒ छन्द॑साꣳ स॒ती स॒ती छन्द॑सा॒म् कनि॑ष्ठा॒ कनि॑ष्ठा॒ छन्द॑साꣳ स॒ती । \newline
5. छन्द॑साꣳ स॒ती स॒ती छन्द॑सा॒म् छन्द॑साꣳ स॒ती य॑ज्ञ्मु॒खं ॅय॑ज्ञ्मु॒खꣳ स॒ती छन्द॑सा॒म् छन्द॑साꣳ स॒ती य॑ज्ञ्मु॒खम् । \newline
6. स॒ती य॑ज्ञ्मु॒खं ॅय॑ज्ञ्मु॒खꣳ स॒ती स॒ती य॑ज्ञ्मु॒खम् परि॒ परि॑ यज्ञ्मु॒खꣳ स॒ती स॒ती य॑ज्ञ्मु॒खम् परि॑ । \newline
7. य॒ज्ञ्॒मु॒खम् परि॒ परि॑ यज्ञ्मु॒खं ॅय॑ज्ञ्मु॒खम् परी॑याये याय॒ परि॑ यज्ञ्मु॒खं ॅय॑ज्ञ्मु॒खम् परी॑याय । \newline
8. य॒ज्ञ्॒मु॒खमिति॑ यज्ञ् - मु॒खम् । \newline
9. परी॑याये याय॒ परि॒ परी॑या॒ये तीती॑याय॒ परि॒ परी॑या॒येति॑ । \newline
10. इ॒या॒येतीती॑ याये या॒येति॒ यद् यदिती॑या येया॒येति॒ यत् । \newline
11. इति॒ यद् यदि तीति॒ यदे॒ वैव यदि तीति॒ यदे॒व । \newline
12. यदे॒ वैव यद् य दे॒वादो॑ ऽद ए॒व यद् य दे॒वादः । \newline
13. ए॒वादो॑ ऽद ए॒वै वादः सोमꣳ॒॒ सोम॑ म॒द ए॒वै वादः सोम᳚म् । \newline
14. अ॒दः सोमꣳ॒॒ सोम॑ म॒दो॑ ऽदः सोम॒ मा सोम॑ म॒दो॑ ऽदः सोम॒ मा । \newline
15. सोम॒ मा सोमꣳ॒॒ सोम॒ मा ऽह॑र॒ दह॑र॒ दा सोमꣳ॒॒ सोम॒ मा ऽह॑रत् । \newline
16. आ ऽह॑र॒ दह॑र॒ दा ऽह॑र॒त् तस्मा॒त् तस्मा॒ दह॑र॒ दा ऽह॑र॒त् तस्मा᳚त् । \newline
17. अह॑र॒त् तस्मा॒त् तस्मा॒ दह॑र॒ दह॑र॒त् तस्मा᳚द् यज्ञ्मु॒खं ॅय॑ज्ञ्मु॒खम् तस्मा॒ दह॑र॒ दह॑र॒त् तस्मा᳚द् यज्ञ्मु॒खम् । \newline
18. तस्मा᳚द् यज्ञ्मु॒खं ॅय॑ज्ञ्मु॒खम् तस्मा॒त् तस्मा᳚द् यज्ञ्मु॒खम् परि॒ परि॑ यज्ञ्मु॒खम् तस्मा॒त् तस्मा᳚द् यज्ञ्मु॒खम् परि॑ । \newline
19. य॒ज्ञ्॒मु॒खम् परि॒ परि॑ यज्ञ्मु॒खं ॅय॑ज्ञ्मु॒खम् पर्यै॑दै॒त् परि॑ यज्ञ्मु॒खं ॅय॑ज्ञ्मु॒खम् पर्यै᳚त् । \newline
20. य॒ज्ञ्॒मु॒खमिति॑ यज्ञ् - मु॒खम् । \newline
21. पर्यै॑ दै॒त् परि॒ पर्यै॒त् तस्मा॒त् तस्मा॑ दै॒त् परि॒ पर्यै॒त् तस्मा᳚त् । \newline
22. ऐ॒त् तस्मा॒त् तस्मा॑ दैदै॒त् तस्मा᳚त् तेज॒स्विनी॑तमा तेज॒स्विनी॑तमा॒ तस्मा॑ दैदै॒त् तस्मा᳚त् तेज॒स्विनी॑तमा । \newline
23. तस्मा᳚त् तेज॒स्विनी॑तमा तेज॒स्विनी॑तमा॒ तस्मा॒त् तस्मा᳚त् तेज॒स्विनी॑तमा प॒द्भ्याम् प॒द्भ्याम् ते॑ज॒स्विनी॑तमा॒ तस्मा॒त् तस्मा᳚त् तेज॒स्विनी॑तमा प॒द्भ्याम् । \newline
24. ते॒ज॒स्विनी॑तमा प॒द्भ्याम् प॒द्भ्याम् ते॑ज॒स्विनी॑तमा तेज॒स्विनी॑तमा प॒द्भ्याम् द्वे द्वे प॒द्भ्याम् ते॑ज॒स्विनी॑तमा तेज॒स्विनी॑तमा प॒द्भ्याम् द्वे । \newline
25. ते॒ज॒स्विनी॑त॒मेति॑ तेज॒स्विनी᳚ - त॒मा॒ । \newline
26. प॒द्भ्याम् द्वे द्वे प॒द्भ्याम् प॒द्भ्याम् द्वे सव॑ने॒ सव॑ने॒ द्वे प॒द्भ्याम् प॒द्भ्याम् द्वे सव॑ने । \newline
27. प॒द्भ्यामिति॑ पत् - भ्याम् । \newline
28. द्वे सव॑ने॒ सव॑ने॒ द्वे द्वे सव॑ने स॒मगृ॑ह्णाथ् स॒मगृ॑ह्णा॒थ् सव॑ने॒ द्वे द्वे सव॑ने स॒मगृ॑ह्णात् । \newline
29. द्वे इति॒ द्वे । \newline
30. सव॑ने स॒मगृ॑ह्णाथ् स॒मगृ॑ह्णा॒थ् सव॑ने॒ सव॑ने स॒मगृ॑ह्णा॒न् मुखे॑न॒ मुखे॑न स॒मगृ॑ह्णा॒थ् सव॑ने॒ सव॑ने स॒मगृ॑ह्णा॒न् मुखे॑न । \newline
31. सव॑ने॒ इति॒ सव॑ने । \newline
32. स॒मगृ॑ह्णा॒न् मुखे॑न॒ मुखे॑न स॒मगृ॑ह्णाथ् स॒मगृ॑ह्णा॒न् मुखे॒ नैक॒ मेक॒म् मुखे॑न स॒मगृ॑ह्णाथ् स॒मगृ॑ह्णा॒न् मुखे॒ नैक᳚म् । \newline
33. स॒मगृ॑ह्णा॒दिति॑ सं - अगृ॑ह्णात् । \newline
34. मुखे॒ नैक॒ मेक॒म् मुखे॑न॒ मुखे॒ नैकं॒ ॅयद् यदेक॒म् मुखे॑न॒ मुखे॒ नैकं॒ ॅयत् । \newline
35. एकं॒ ॅयद् यदेक॒ मेकं॒ ॅयन् मुखे॑न॒ मुखे॑न॒ यदेक॒ मेकं॒ ॅयन् मुखे॑न । \newline
36. यन् मुखे॑न॒ मुखे॑न॒ यद् यन् मुखे॑न स॒मगृ॑ह्णाथ् स॒मगृ॑ह्णा॒न् मुखे॑न॒ यद् यन् मुखे॑न स॒मगृ॑ह्णात् । \newline
37. मुखे॑न स॒मगृ॑ह्णाथ् स॒मगृ॑ह्णा॒न् मुखे॑न॒ मुखे॑न स॒मगृ॑ह्णा॒त् तत् तथ् स॒मगृ॑ह्णा॒न् मुखे॑न॒ मुखे॑न स॒मगृ॑ह्णा॒त् तत् । \newline
38. स॒मगृ॑ह्णा॒त् तत् तथ् स॒मगृ॑ह्णाथ् स॒मगृ॑ह्णा॒त् तद॑धय दधय॒त् तथ् स॒मगृ॑ह्णाथ् स॒मगृ॑ह्णा॒त् तद॑धयत् । \newline
39. स॒मगृ॑ह्णा॒दिति॑ सं - अगृ॑ह्णात् । \newline
40. तद॑धय दधय॒त् तत् तद॑धय॒त् तस्मा॒त् तस्मा॑ दधय॒त् तत् तद॑धय॒त् तस्मा᳚त् । \newline
41. अ॒ध॒य॒त् तस्मा॒त् तस्मा॑ दधय दधय॒त् तस्मा॒द् द्वे द्वे तस्मा॑ दधय दधय॒त् तस्मा॒द् द्वे । \newline
42. तस्मा॒द् द्वे द्वे तस्मा॒त् तस्मा॒द् द्वे सव॑ने॒ सव॑ने॒ द्वे तस्मा॒त् तस्मा॒द् द्वे सव॑ने । \newline
43. द्वे सव॑ने॒ सव॑ने॒ द्वे द्वे सव॑ने शु॒क्रव॑ती शु॒क्रव॑ती॒ सव॑ने॒ द्वे द्वे सव॑ने शु॒क्रव॑ती । \newline
44. द्वे इति॒ द्वे । \newline
45. सव॑ने शु॒क्रव॑ती शु॒क्रव॑ती॒ सव॑ने॒ सव॑ने शु॒क्रव॑ती प्रातस्सव॒नम् प्रा॑तस्सव॒नꣳ शु॒क्रव॑ती॒ सव॑ने॒ सव॑ने शु॒क्रव॑ती प्रातस्सव॒नम् । \newline
46. सव॑ने॒ इति॒ सव॑ने । \newline
47. शु॒क्रव॑ती प्रातस्सव॒नम् प्रा॑तस्सव॒नꣳ शु॒क्रव॑ती शु॒क्रव॑ती प्रातस्सव॒नम् च॑ च प्रातस्सव॒नꣳ शु॒क्रव॑ती शु॒क्रव॑ती प्रातस्सव॒नम् च॑ । \newline
48. शु॒क्रव॑ती॒ इति॑ शु॒क्र - व॒ती॒ । \newline
49. प्रा॒त॒स्स॒व॒नम् च॑ च प्रातस्सव॒नम् प्रा॑तस्सव॒नम् च॒ माद्ध्य॑न्दिन॒म् माद्ध्य॑न्दिनम् च प्रातस्सव॒नम् प्रा॑तस्सव॒नम् च॒ माद्ध्य॑न्दिनम् । \newline
50. प्रा॒त॒स्स॒व॒नमिति॑ प्रातः - स॒व॒नम् । \newline
51. च॒ माद्ध्य॑न्दिन॒म् माद्ध्य॑न्दिनम् च च॒ माद्ध्य॑न्दिनम् च च॒ माद्ध्य॑न्दिनम् च च॒ माद्ध्य॑न्दिनम् च । \newline
52. माद्ध्य॑न्दिनम् च च॒ माद्ध्य॑न्दिन॒म् माद्ध्य॑न्दिनम् च॒ तस्मा॒त् तस्मा᳚च् च॒ माद्ध्य॑न्दिन॒म् माद्ध्य॑न्दिनम् च॒ तस्मा᳚त् । \newline
53. च॒ तस्मा॒त् तस्मा᳚च् च च॒ तस्मा᳚त् तृतीयसव॒ने तृ॑तीयसव॒ने तस्मा᳚च् च च॒ तस्मा᳚त् तृतीयसव॒ने । \newline
54. तस्मा᳚त् तृतीयसव॒ने तृ॑तीयसव॒ने तस्मा॒त् तस्मा᳚त् तृतीयसव॒न ऋ॑जी॒ष मृ॑जी॒षम् तृ॑तीयसव॒ने तस्मा॒त् तस्मा᳚त् तृतीयसव॒न ऋ॑जी॒षम् । \newline
55. तृ॒ती॒य॒स॒व॒न ऋ॑जी॒ष मृ॑जी॒षम् तृ॑तीयसव॒ने तृ॑तीयसव॒न ऋ॑जी॒ष म॒भ्या᳚(1॒)भ्यृ॑जी॒षम् तृ॑तीयसव॒ने तृ॑तीयसव॒न ऋ॑जी॒ष म॒भि । \newline
56. तृ॒ती॒य॒स॒व॒न इति॑ तृतीय - स॒व॒ने । \newline
57. ऋ॒जी॒ष म॒भ्या᳚(1॒)भ्यृ॑जी॒ष मृ॑जी॒ष म॒भि षु॑न्वन्ति सुन्वन् त्य॒भ्यृ॑जी॒ष मृ॑जी॒ष म॒भि षु॑न्वन्ति । \newline
58. अ॒भि षु॑न्वन्ति सुन्वन् त्य॒भ्य॑भि षु॑न्वन्ति धी॒तम् धी॒तꣳ सु॑न्वन् त्य॒भ्य॑भि षु॑न्वन्ति धी॒तम् । \newline
59. सु॒न्व॒न्ति॒ धी॒तम् धी॒तꣳ सु॑न्वन्ति सुन्वन्ति धी॒त मि॑वेव धी॒तꣳ सु॑न्वन्ति सुन्वन्ति धी॒त मि॑व । \newline
60. धी॒त मि॑वेव धी॒तम् धी॒त मि॑व॒ हि हीव॑ धी॒तम् धी॒त मि॑व॒ हि । \newline
61. इ॒व॒ हि हीवे॑व॒ हि मन्य॑न्ते॒ मन्य॑न्ते॒ हीवे॑व॒ हि मन्य॑न्ते । \newline
62. हि मन्य॑न्ते॒ मन्य॑न्ते॒ हि हि मन्य॑न्त आ॒शिर॑ मा॒शिर॒म् मन्य॑न्ते॒ हि हि मन्य॑न्त आ॒शिर᳚म् । \newline
63. मन्य॑न्त आ॒शिर॑ मा॒शिर॒म् मन्य॑न्ते॒ मन्य॑न्त आ॒शिर॒ मवावा॒ शिर॒म् मन्य॑न्ते॒ मन्य॑न्त आ॒शिर॒ मव॑ । \newline
\pagebreak
\markright{ TS 6.1.6.5  \hfill https://www.vedavms.in \hfill}

\section{ TS 6.1.6.5 }

\textbf{TS 6.1.6.5 } \newline
\textbf{Samhita Paata} \newline

आ॒शिर॒मव॑ नयति सशुक्र॒त्वायाथो॒ सं भ॑रत्ये॒वैन॒त् तꣳ सोम॑-माह्रि॒यमा॑णं गन्ध॒र्वो वि॒श्वाव॑सः॒ पर्य॑मुष्णा॒थ् स ति॒स्रो रात्रीः॒ परि॑मुषितोऽवस॒त् तस्मा᳚त् ति॒स्रो रात्रीः᳚ क्री॒तः सोमो॑ वसति॒ ते दे॒वा अ॑ब्रुव॒न्थ् स्त्रीका॑मा॒ वै ग॑न्ध॒र्वा स्स्त्रि॒या निष्क्री॑णा॒मेति॒ ते वाचꣳ॒॒ स्त्रिय॒मेक॑हायनीं कृ॒त्वा तया॒ निर॑क्रीण॒न्थ् सा रो॒हिद्-रू॒पं कृ॒त्वा ग॑न्ध॒र्वेभ्यो॑- [  ] \newline

\textbf{Pada Paata} \newline

आ॒शिर᳚म् । अवेति॑ । न॒य॒ति॒ । स॒शु॒क्र॒त्वायेति॑ सशुक्र - त्वाय॑ । अथो॒ इति॑ । समिति॑ । भ॒र॒ति॒ । ए॒व । ए॒न॒त् । तम् । सोम᳚म् । आ॒ह्रि॒यमा॑ण॒मित्या᳚ - ह्रि॒यमा॑णम् । ग॒न्ध॒र्वः । वि॒श्वाव॑सु॒रिति॑ वि॒श्व - व॒सुः॒ । परीति॑ । अ॒मु॒ष्णा॒त् । सः । ति॒स्रः । रात्रीः᳚ । परि॑मुषित॒ इति॒ परि॑ - मु॒षि॒तः॒ । अ॒व॒स॒त् । तस्मा᳚त् । ति॒स्रः । रात्रीः᳚ । क्री॒तः । सोमः॑ । व॒स॒ति॒ । ते । दे॒वाः । अ॒ब्रु॒व॒न्न् । स्त्रीका॑मा॒ इति॒ स्त्री - का॒माः॒ । वै । ग॒न्ध॒र्वाः । स्त्रि॒या । निरिति॑ । क्री॒णा॒म॒ । इति॑ । ते । वाच᳚म् । स्त्रिय᳚म् । एक॑हायनी॒मित्येक॑ - हा॒य॒नी॒म् । कृ॒त्वा । तया᳚ । निरिति॑ । अ॒क्री॒ण॒न्न् । सा । रो॒हित् । रू॒पम् । कृ॒त्वा । ग॒न्ध॒र्वेभ्यः॑ ।  \newline


\textbf{Krama Paata} \newline

आ॒शिर॒मव॑ । अव॑ नयति । न॒य॒ति॒ स॒शु॒क्र॒त्वाय॑ । स॒शु॒क्र॒त्वायाथो᳚ । स॒शु॒क्र॒त्वायेति॑ सशुक्र - त्वाय॑ । अथो॒ सम् । अथो॒ इत्यथो᳚ । सम्भ॑रति । भ॒र॒त्ये॒व । ए॒वैन॑त् । ए॒न॒त् तम् । तꣳ सोम᳚म् । सोम॑माह्रि॒यमा॑णम् । आ॒ह्रि॒यमा॑णम् गन्ध॒र्वः । आ॒ह्रि॒यमा॑ण॒मित्या᳚ - ह्रि॒यमा॑णम् । ग॒न्ध॒र्वो वि॒श्वावसुः॑ । वि॒श्वाव॑सुः॒ परि॑ । वि॒श्वाव॑सु॒रिति॑ वि॒श्व - व॒सुः॒ । पर्य॑मुष्णात् । अ॒मु॒ष्णा॒थ् सः । स ति॒स्रः । ति॒स्रो रात्रीः᳚ । रात्रीः॒ परि॑मुषितः । परि॑मुषितोऽवसत् । परि॑मुषित॒ इति॒ परि॑ - मु॒षि॒तः॒ । अ॒व॒स॒त् तस्मा᳚त् । तस्मा᳚त् ति॒स्रः । ति॒स्रो रात्रीः᳚ । रात्रीः᳚ क्री॒तः । क्री॒तः सोमः॑ । सोमो॑ वसति । व॒स॒ति॒ ते । ते दे॒वाः । दे॒वा अ॑ब्रुवन्न् । अ॒ब्रु॒व॒न्थ् स्त्रीका॑माः । स्त्रीका॑मा॒ वै । स्त्रीका॑मा॒ इति॒ स्त्री - का॒माः॒ । वै ग॑न्ध॒र्वाः । ग॒न्ध॒र्वाः स्त्रि॒या । स्त्रि॒या निः । निष्क्री॑णाम । क्री॒णा॒मेति॑ । इति॒ ते । ते वाच᳚म् । वाचꣳ॒॒ स्त्रिय᳚म् । स्त्रिय॒मेक॑हायनीम् । एक॑हायनीम् कृ॒त्वा । एक॑हायनी॒मित्येक॑ - हा॒य॒नी॒म् । कृ॒त्वा तया᳚ । तया॒ निः । निर॑क्रीणन्न् । अ॒क्री॒ण॒न्थ् सा । सा रो॒हित् । रो॒हिद् रू॒पम् । रू॒पम् कृ॒त्वा । कृ॒त्वा ग॑न्ध॒र्वेभ्यः॑ । ग॒न्ध॒र्वेभ्यो॑ऽप॒क्रम्य॑ \newline

\textbf{Jatai Paata} \newline

1. आ॒शिर॒ मवावा॒ शिर॑ मा॒शिर॒ मव॑ । \newline
2. अव॑ नयति नय॒ त्यवाव॑ नयति । \newline
3. न॒य॒ति॒ स॒शु॒क्र॒त्वाय॑ सशुक्र॒त्वाय॑ नयति नयति सशुक्र॒त्वाय॑ । \newline
4. स॒शु॒क्र॒त्वा याथो॒ अथो॑ सशुक्र॒त्वाय॑ सशुक्र॒त्वा याथो᳚ । \newline
5. स॒शु॒क्र॒त्वायेति॑ सशुक्र - त्वाय॑ । \newline
6. अथो॒ सꣳ स मथो॒ अथो॒ सम् । \newline
7. अथो॒ इत्यथो᳚ । \newline
8. सम् भ॑रति भरति॒ सꣳ सम् भ॑रति । \newline
9. भ॒र॒ त्ये॒वैव भ॑रति भर त्ये॒व । \newline
10. ए॒वैन॑ देन दे॒वै वैन॑त् । \newline
11. ए॒न॒त् तम् त मे॑न देन॒त् तम् । \newline
12. तꣳ सोमꣳ॒॒ सोम॒म् तम् तꣳ सोम᳚म् । \newline
13. सोम॑ माह्रि॒यमा॑ण माह्रि॒यमा॑णꣳ॒॒ सोमꣳ॒॒ सोम॑ माह्रि॒यमा॑णम् । \newline
14. आ॒ह्रि॒यमा॑णम् गन्ध॒र्वो ग॑न्ध॒र्व आ᳚ह्रि॒यमा॑ण माह्रि॒यमा॑णम् गन्ध॒र्वः । \newline
15. आ॒ह्रि॒यमा॑ण॒मित्या᳚ - ह्रि॒यमा॑णम् । \newline
16. ग॒न्ध॒र्वो वि॒श्वाव॑सुर् वि॒श्वाव॑सुर् गन्ध॒र्वो ग॑न्ध॒र्वो वि॒श्वाव॑सुः । \newline
17. वि॒श्वाव॑सुः॒ परि॒ परि॑ वि॒श्वाव॑सुर् वि॒श्वाव॑सुः॒ परि॑ । \newline
18. वि॒श्वाव॑सु॒रिति॑ वि॒श्व - व॒सुः॒ । \newline
19. पर्य॑मुष्णा दमुष्णा॒त् परि॒ पर्य॑मुष्णात् । \newline
20. अ॒मु॒ष्णा॒थ् स सो॑ ऽमुष्णा दमुष्णा॒थ् सः । \newline
21. स ति॒स्र स्ति॒स्रः स स ति॒स्रः । \newline
22. ति॒स्रो रात्री॒ रात्री᳚ स्ति॒स्र स्ति॒स्रो रात्रीः᳚ । \newline
23. रात्रीः॒ परि॑मुषितः॒ परि॑मुषितो॒ रात्री॒ रात्रीः॒ परि॑मुषितः । \newline
24. परि॑मुषितो ऽवस दवस॒त् परि॑मुषितः॒ परि॑मुषितो ऽवसत् । \newline
25. परि॑मुषित॒ इति॒ परि॑ - मु॒षि॒तः॒ । \newline
26. अ॒व॒स॒त् तस्मा॒त् तस्मा॑ दवस दवस॒त् तस्मा᳚त् । \newline
27. तस्मा᳚त् ति॒स्र स्ति॒स्र स्तस्मा॒त् तस्मा᳚त् ति॒स्रः । \newline
28. ति॒स्रो रात्री॒ रात्री᳚ स्ति॒स्र स्ति॒स्रो रात्रीः᳚ । \newline
29. रात्रीः᳚ क्री॒तः क्री॒तो रात्री॒ रात्रीः᳚ क्री॒तः । \newline
30. क्री॒तः सोमः॒ सोमः॑ क्री॒तः क्री॒तः सोमः॑ । \newline
31. सोमो॑ वसति वसति॒ सोमः॒ सोमो॑ वसति । \newline
32. व॒स॒ति॒ ते ते व॑सति वसति॒ ते । \newline
33. ते दे॒वा दे॒वा स्ते ते दे॒वाः । \newline
34. दे॒वा अ॑ब्रुवन् नब्रुवन् दे॒वा दे॒वा अ॑ब्रुवन्न् । \newline
35. अ॒ब्रु॒व॒न् थ्स्त्रीका॑माः॒ स्त्रीका॑मा अब्रुवन् नब्रुव॒न् थ्स्त्रीका॑माः । \newline
36. स्त्रीका॑मा॒ वै वै स्त्रीका॑माः॒ स्त्रीका॑मा॒ वै । \newline
37. स्त्रीका॑मा॒ इति॒ स्त्री - का॒माः॒ । \newline
38. वै ग॑न्ध॒र्वा ग॑न्ध॒र्वा वै वै ग॑न्ध॒र्वाः । \newline
39. ग॒न्ध॒र्वाः स्त्रि॒या स्त्रि॒या ग॑न्ध॒र्वा ग॑न्ध॒र्वाः स्त्रि॒या । \newline
40. स्त्रि॒या निर् णिः स्त्रि॒या स्त्रि॒या निः । \newline
41. निष् क्री॑णाम क्रीणाम॒ निर् णिष् क्री॑णाम । \newline
42. क्री॒णा॒मे तीति॑ क्रीणाम क्रीणा॒मेति॑ । \newline
43. इति॒ ते त इतीति॒ ते । \newline
44. ते वाचं॒ ॅवाच॒म् ते ते वाच᳚म् । \newline
45. वाचꣳ॒॒ स्त्रियꣳ॒॒ स्त्रियं॒ ॅवाचं॒ ॅवाचꣳ॒॒ स्त्रिय᳚म् । \newline
46. स्त्रिय॒ मेक॑हायनी॒ मेक॑हायनीꣳ॒॒ स्त्रियꣳ॒॒ स्त्रिय॒ मेक॑हायनीम् । \newline
47. एक॑हायनीम् कृ॒त्वा कृ॒त्वैक॑हायनी॒ मेक॑हायनीम् कृ॒त्वा । \newline
48. एक॑हायनी॒मित्येक॑ - हा॒य॒नी॒म् । \newline
49. कृ॒त्वा तया॒ तया॑ कृ॒त्वा कृ॒त्वा तया᳚ । \newline
50. तया॒ निर् णिष् टया॒ तया॒ निः । \newline
51. निर॑क्रीणन् नक्रीण॒न् निर् णिर॑क्रीणन्न् । \newline
52. अ॒क्री॒ण॒न् थ्सा सा ऽक्री॑णन् नक्रीण॒न् थ्सा । \newline
53. सा रो॒हिद् रो॒हिथ् सा सा रो॒हित् । \newline
54. रो॒हिद् रू॒पꣳ रू॒पꣳ रो॒हिद् रो॒हिद् रू॒पम् । \newline
55. रू॒पम् कृ॒त्वा कृ॒त्वा रू॒पꣳ रू॒पम् कृ॒त्वा । \newline
56. कृ॒त्वा ग॑न्ध॒र्वेभ्यो॑ गन्ध॒र्वेभ्यः॑ कृ॒त्वा कृ॒त्वा ग॑न्ध॒र्वेभ्यः॑ । \newline
57. ग॒न्ध॒र्वेभ्यो॑ ऽप॒क्रम्या॑ प॒क्रम्य॑ गन्ध॒र्वेभ्यो॑ गन्ध॒र्वेभ्यो॑ ऽप॒क्रम्य॑ । \newline

\textbf{Ghana Paata } \newline

1. आ॒शिर॒ मवावा॒ शिर॑ मा॒शिर॒ मव॑ नयति नय॒ त्यवा॒ शिर॑ मा॒शिर॒ मव॑ नयति । \newline
2. अव॑ नयति नय॒ त्यवाव॑ नयति सशुक्र॒त्वाय॑ सशुक्र॒त्वाय॑ नय॒त्यवाव॑ नयति सशुक्र॒त्वाय॑ । \newline
3. न॒य॒ति॒ स॒शु॒क्र॒त्वाय॑ सशुक्र॒त्वाय॑ नयति नयति सशुक्र॒त्वा याथो॒ अथो॑ सशुक्र॒त्वाय॑ नयति नयति सशुक्र॒त्वा याथो᳚ । \newline
4. स॒शु॒क्र॒त्वा याथो॒ अथो॑ सशुक्र॒त्वाय॑ सशुक्र॒त्वा याथो॒ सꣳ स मथो॑ सशुक्र॒त्वाय॑ सशुक्र॒त्वा याथो॒ सम् । \newline
5. स॒शु॒क्र॒त्वायेति॑ सशुक्र - त्वाय॑ । \newline
6. अथो॒ सꣳ स मथो॒ अथो॒ सम् भ॑रति भरति॒ स मथो॒ अथो॒ सम् भ॑रति । \newline
7. अथो॒ इत्यथो᳚ । \newline
8. सम् भ॑रति भरति॒ सꣳ सम् भ॑र त्ये॒वैव भ॑रति॒ सꣳ सम् भ॑रत्ये॒व । \newline
9. भ॒र॒ त्ये॒वैव भ॑रति भर त्ये॒वैन॑ देन दे॒व भ॑रति भर त्ये॒वैन॑त् । \newline
10. ए॒वैन॑ देनदे॒वै वैन॒त् तम् त मे॑न दे॒वै वैन॒त् तम् । \newline
11. ए॒न॒त् तम् त मे॑न देन॒त् तꣳ सोमꣳ॒॒ सोम॒म् त मे॑न देन॒त् तꣳ सोम᳚म् । \newline
12. तꣳ सोमꣳ॒॒ सोम॒म् तम् तꣳ सोम॑ माह्रि॒यमा॑ण माह्रि॒यमा॑णꣳ॒॒ सोम॒म् तम् तꣳ सोम॑ माह्रि॒यमा॑णम् । \newline
13. सोम॑ माह्रि॒यमा॑ण माह्रि॒यमा॑णꣳ॒॒ सोमꣳ॒॒ सोम॑ माह्रि॒यमा॑णम् गन्ध॒र्वो ग॑न्ध॒र्व आ᳚ह्रि॒यमा॑णꣳ॒॒ सोमꣳ॒॒ सोम॑ माह्रि॒यमा॑णम् गन्ध॒र्वः । \newline
14. आ॒ह्रि॒यमा॑णम् गन्ध॒र्वो ग॑न्ध॒र्व आ᳚ह्रि॒यमा॑ण माह्रि॒यमा॑णम् गन्ध॒र्वो वि॒श्वाव॑सुर् वि॒श्वाव॑सुर् गन्ध॒र्व आ᳚ह्रि॒यमा॑ण माह्रि॒यमा॑णम् गन्ध॒र्वो वि॒श्वाव॑सुः । \newline
15. आ॒ह्रि॒यमा॑ण॒मित्या᳚ - ह्रि॒यमा॑णम् । \newline
16. ग॒न्ध॒र्वो वि॒श्वाव॑सुर् वि॒श्वाव॑सुर् गन्ध॒र्वो ग॑न्ध॒र्वो वि॒श्वाव॑सुः॒ परि॒ परि॑ वि॒श्वाव॑सुर् गन्ध॒र्वो ग॑न्ध॒र्वो वि॒श्वाव॑सुः॒ परि॑ । \newline
17. वि॒श्वाव॑सुः॒ परि॒ परि॑ वि॒श्वाव॑सुर् वि॒श्वाव॑सुः॒ पर्य॑मुष्णा दमुष्णा॒त् परि॑ वि॒श्वाव॑सुर् वि॒श्वाव॑सुः॒ पर्य॑मुष्णात् । \newline
18. वि॒श्वाव॑सु॒रिति॑ वि॒श्व - व॒सुः॒ । \newline
19. पर्य॑मुष्णा दमुष्णा॒त् परि॒ पर्य॑मुष्णा॒थ् स सो॑ ऽमुष्णा॒त् परि॒ पर्य॑ मुष्णा॒थ् सः । \newline
20. अ॒मु॒ष्णा॒थ् स सो॑ ऽमुष्णा दमुष्णा॒थ् स ति॒स्र स्ति॒स्रः सो॑ ऽमुष्णा दमुष्णा॒थ् स ति॒स्रः । \newline
21. स ति॒स्र स्ति॒स्रः स स ति॒स्रो रात्री॒ रात्री᳚ स्ति॒स्रः स स ति॒स्रो रात्रीः᳚ । \newline
22. ति॒स्रो रात्री॒ रात्री᳚ स्ति॒स्र स्ति॒स्रो रात्रीः॒ परि॑मुषितः॒ परि॑मुषितो॒ रात्री᳚ स्ति॒स्र स्ति॒स्रो रात्रीः॒ परि॑मुषितः । \newline
23. रात्रीः॒ परि॑मुषितः॒ परि॑मुषितो॒ रात्री॒ रात्रीः॒ परि॑मुषितो ऽवस दवस॒त् परि॑मुषितो॒ रात्री॒ रात्रीः॒ परि॑मुषितो ऽवसत् । \newline
24. परि॑मुषितो ऽवस दवस॒त् परि॑मुषितः॒ परि॑मुषितो ऽवस॒त् तस्मा॒त् तस्मा॑ दवस॒त् परि॑मुषितः॒ परि॑मुषितो ऽवस॒त् तस्मा᳚त् । \newline
25. परि॑मुषित॒ इति॒ परि॑ - मु॒षि॒तः॒ । \newline
26. अ॒व॒स॒त् तस्मा॒त् तस्मा॑ दवस दवस॒त् तस्मा᳚त् ति॒स्र स्ति॒स्र स्तस्मा॑ दवस दवस॒त् तस्मा᳚त् ति॒स्रः । \newline
27. तस्मा᳚त् ति॒स्र स्ति॒स्र स्तस्मा॒त् तस्मा᳚त् ति॒स्रो रात्री॒ रात्री᳚ स्ति॒स्र स्तस्मा॒त् तस्मा᳚त् ति॒स्रो रात्रीः᳚ । \newline
28. ति॒स्रो रात्री॒ रात्री᳚ स्ति॒स्र स्ति॒स्रो रात्रीः᳚ क्री॒तः क्री॒तो रात्री᳚ स्ति॒स्र स्ति॒स्रो रात्रीः᳚ क्री॒तः । \newline
29. रात्रीः᳚ क्री॒तः क्री॒तो रात्री॒ रात्रीः᳚ क्री॒तः सोमः॒ सोमः॑ क्री॒तो रात्री॒ रात्रीः᳚ क्री॒तः सोमः॑ । \newline
30. क्री॒तः सोमः॒ सोमः॑ क्री॒तः क्री॒तः सोमो॑ वसति वसति॒ सोमः॑ क्री॒तः क्री॒तः सोमो॑ वसति । \newline
31. सोमो॑ वसति वसति॒ सोमः॒ सोमो॑ वसति॒ ते ते व॑सति॒ सोमः॒ सोमो॑ वसति॒ ते । \newline
32. व॒स॒ति॒ ते ते व॑सति वसति॒ ते दे॒वा दे॒वा स्ते व॑सति वसति॒ ते दे॒वाः । \newline
33. ते दे॒वा दे॒वा स्ते ते दे॒वा अ॑ब्रुवन् नब्रुवन् दे॒वा स्ते ते दे॒वा अ॑ब्रुवन्न् । \newline
34. दे॒वा अ॑ब्रुवन् नब्रुवन् दे॒वा दे॒वा अ॑ब्रुव॒न् थ्स्त्रीका॑माः॒ स्त्रीका॑मा अब्रुवन् दे॒वा दे॒वा अ॑ब्रुव॒न् थ्स्त्रीका॑माः । \newline
35. अ॒ब्रु॒व॒न् थ्स्त्रीका॑माः॒ स्त्रीका॑मा अब्रुवन् नब्रुव॒न् थ्स्त्रीका॑मा॒ वै वै स्त्रीका॑मा अब्रुवन् नब्रुव॒न् थ्स्त्रीका॑मा॒ वै । \newline
36. स्त्रीका॑मा॒ वै वै स्त्रीका॑माः॒ स्त्रीका॑मा॒ वै ग॑न्ध॒र्वा ग॑न्ध॒र्वा वै स्त्रीका॑माः॒ स्त्रीका॑मा॒ वै ग॑न्ध॒र्वाः । \newline
37. स्त्रीका॑मा॒ इति॒ स्त्री - का॒माः॒ । \newline
38. वै ग॑न्ध॒र्वा ग॑न्ध॒र्वा वै वै ग॑न्ध॒र्वाः स्त्रि॒या स्त्रि॒या ग॑न्ध॒र्वा वै वै ग॑न्ध॒र्वाः स्त्रि॒या । \newline
39. ग॒न्ध॒र्वाः स्त्रि॒या स्त्रि॒या ग॑न्ध॒र्वा ग॑न्ध॒र्वाः स्त्रि॒या निर् णिः स्त्रि॒या ग॑न्ध॒र्वा ग॑न्ध॒र्वाः स्त्रि॒या निः । \newline
40. स्त्रि॒या निर् णिः स्त्रि॒या स्त्रि॒या निष् क्री॑णाम क्रीणाम॒ निः स्त्रि॒या स्त्रि॒या निष् क्री॑णाम । \newline
41. निष् क्री॑णाम क्रीणाम॒ निर् णिष् क्री॑णा॒मे तीति॑ क्रीणाम॒ निर् णिष् क्री॑णा॒मेति॑ । \newline
42. क्री॒णा॒मे तीति॑ क्रीणाम क्रीणा॒मेति॒ ते त इति॑ क्रीणाम क्रीणा॒मेति॒ ते । \newline
43. इति॒ ते त इतीति॒ ते वाचं॒ ॅवाच॒म् त इतीति॒ ते वाच᳚म् । \newline
44. ते वाचं॒ ॅवाच॒म् ते ते वाचꣳ॒॒ स्त्रियꣳ॒॒ स्त्रियं॒ ॅवाच॒म् ते ते वाचꣳ॒॒ स्त्रिय᳚म् । \newline
45. वाचꣳ॒॒ स्त्रियꣳ॒॒ स्त्रियं॒ ॅवाचं॒ ॅवाचꣳ॒॒ स्त्रिय॒ मेक॑हायनी॒ मेक॑हायनीꣳ॒॒ स्त्रियं॒ ॅवाचं॒ ॅवाचꣳ॒॒ स्त्रिय॒ मेक॑हायनीम् । \newline
46. स्त्रिय॒ मेक॑हायनी॒ मेक॑हायनीꣳ॒॒ स्त्रियꣳ॒॒ स्त्रिय॒ मेक॑हायनीम् कृ॒त्वा कृ॒त्वैक॑हायनीꣳ॒॒ स्त्रियꣳ॒॒ स्त्रिय॒ मेक॑हायनीम् कृ॒त्वा । \newline
47. एक॑हायनीम् कृ॒त्वा कृ॒त्वै क॑हायनी॒ मेक॑हायनीम् कृ॒त्वा तया॒ तया॑ कृ॒त्वै क॑हायनी॒ मेक॑हायनीम् कृ॒त्वा तया᳚ । \newline
48. एक॑हायनी॒मित्येक॑ - हा॒य॒नी॒म् । \newline
49. कृ॒त्वा तया॒ तया॑ कृ॒त्वा कृ॒त्वा तया॒ निर् णिष्टया॑ कृ॒त्वा कृ॒त्वा तया॒ निः । \newline
50. तया॒ निर् णिष् टया॒ तया॒ निर॑क्रीणन् नक्रीण॒न् निष् टया॒ तया॒ निर॑क्रीणन्न् । \newline
51. निर॑क्रीणन् नक्रीण॒न् निर् णिर॑क्रीण॒न् थ्सा सा ऽक्री॑ण॒न् निर् णिर॑क्रीण॒न् थ्सा । \newline
52. अ॒क्री॒ण॒न् थ्सा सा ऽक्री॑णन् नक्रीण॒न् थ्सा रो॒हिद् रो॒हिथ् सा ऽक्री॑णन् नक्रीण॒न् थ्सा रो॒हित् । \newline
53. सा रो॒हिद् रो॒हिथ् सा सा रो॒हिद् रू॒पꣳ रू॒पꣳ रो॒हिथ् सा सा रो॒हिद् रू॒पम् । \newline
54. रो॒हिद् रू॒पꣳ रू॒पꣳ रो॒हिद् रो॒हिद् रू॒पम् कृ॒त्वा कृ॒त्वा रू॒पꣳ रो॒हिद् रो॒हिद् रू॒पम् कृ॒त्वा । \newline
55. रू॒पम् कृ॒त्वा कृ॒त्वा रू॒पꣳ रू॒पम् कृ॒त्वा ग॑न्ध॒र्वेभ्यो॑ गन्ध॒र्वेभ्यः॑ कृ॒त्वा रू॒पꣳ रू॒पम् कृ॒त्वा ग॑न्ध॒र्वेभ्यः॑ । \newline
56. कृ॒त्वा ग॑न्ध॒र्वेभ्यो॑ गन्ध॒र्वेभ्यः॑ कृ॒त्वा कृ॒त्वा ग॑न्ध॒र्वेभ्यो॑ ऽप॒क्रम्या॑ प॒क्रम्य॑ गन्ध॒र्वेभ्यः॑ कृ॒त्वा कृ॒त्वा ग॑न्ध॒र्वेभ्यो॑ ऽप॒क्रम्य॑ । \newline
57. ग॒न्ध॒र्वेभ्यो॑ ऽप॒क्रम्या॑ प॒क्रम्य॑ गन्ध॒र्वेभ्यो॑ गन्ध॒र्वेभ्यो॑ ऽप॒क्रम्या॑ तिष्ठ दतिष्ठ दप॒क्रम्य॑ गन्ध॒र्वेभ्यो॑ गन्ध॒र्वेभ्यो॑ ऽप॒क्रम्या॑ तिष्ठत् । \newline
\pagebreak
\markright{ TS 6.1.6.6  \hfill https://www.vedavms.in \hfill}

\section{ TS 6.1.6.6 }

\textbf{TS 6.1.6.6 } \newline
\textbf{Samhita Paata} \newline

ऽप॒क्रम्या॑तिष्ठ॒त् तद्-रो॒हितो॒ जन्म॒ ते दे॒वा अ॑ब्रुव॒न्नप॑ यु॒ष्मदक्र॑मी॒-न्नास्मानु॒-पाव॑र्तते॒ वि ह्व॑यामहा॒ इति॒ ब्रह्म॑ गन्ध॒र्वा अव॑द॒न्नगा॑यन् दे॒वाः सा दे॒वान् गाय॑त उ॒पाव॑र्तत॒ तस्मा॒द्-गाय॑न्तꣳ॒॒ स्त्रियः॑ कामयन्ते॒ कामु॑का एनꣳ॒॒ स्त्रियो॑ भवन्ति॒ य ए॒वं ॅवेदाथो॒ य ए॒वं ॅवि॒द्वानपि॒ जन्ये॑षु॒ भव॑ति॒ तेभ्य॑ ए॒व द॑दत्यु॒त यद्-ब॒हुत॑या॒ - [  ] \newline

\textbf{Pada Paata} \newline

अ॒प॒क्रम्येत्य॑प - क्रम्य॑ । अ॒ति॒ष्ठ॒त् । तत् । रो॒हितः॑ । जन्म॑ । ते । दे॒वाः । अ॒ब्रु॒व॒न्न् । अपेति॑ । यु॒ष्मत् । अक्र॑मीत् । न । अ॒स्मान् । उ॒पाव॑र्तत॒ इत्यु॑प - आव॑र्तते । वीति॑ । ह्व॒या॒म॒है॒ । इति॑ । ब्रह्म॑ । ग॒न्ध॒र्वाः । अव॑दन्न् । अगा॑यन्न् । दे॒वाः । सा । दे॒वान् । गाय॑तः । उ॒पाव॑र्त॒तेत्यु॑प-आव॑र्तत । तस्मा᳚त् । गाय॑न्तम् । स्त्रियः॑ । का॒म॒य॒न्ते॒ । कामु॑काः । ए॒न॒म् । स्त्रियः॑ । भ॒व॒न्ति॒ । यः । ए॒वम् । वेद॑ । अथो॒ इति॑ । यः । ए॒वम् । वि॒द्वान् । अपीति॑ । जन्ये॑षु । भव॑ति । तेभ्यः॑ । ए॒व । द॒द॒ति॒ । उ॒त । यत् । ब॒हुत॑या॒ इति॑ ब॒हु - त॒याः॒ ।  \newline


\textbf{Krama Paata} \newline

अ॒प॒क्रम्या॑तिष्ठत् । अ॒प॒क्रम्येत्य॑प - क्रम्य॑ । अ॒ति॒ष्ठ॒त् तत् । तद् रो॒हितः॑ । रो॒हितो॒ जन्म॑ । जन्म॒ ते । ते दे॒वाः । दे॒वा अ॑ब्रुवन्न् । अ॒ब्रु॒व॒न्नप॑ । अप॑ यु॒ष्मत् । यु॒ष्मदक्र॑मीत् । अक्र॑मी॒न् न । नास्मान् । अ॒स्मानु॒पाव॑र्तते । उ॒पाव॑र्तते॒ वि । उ॒पाव॑र्तत॒ इत्यु॑प - आव॑र्तते । वि ह्व॑यामहै । ह्व॒या॒म॒हा॒ इति॑ । इति॒ ब्रह्म॑ । ब्रह्म॑ गन्ध॒र्वाः । ग॒न्ध॒र्वा अव॑दन्न् । अव॑द॒न्नगा॑यन्न् । अगा॑यन् दे॒वाः । दे॒वाः सा । सा दे॒वान् । दे॒वान् गाय॑तः । गाय॑त उ॒पाव॑र्तत । उ॒पाव॑र्तत॒ तस्मा᳚त् । उ॒पाव॑र्त॒तेत्यु॑प - आव॑र्तत । तस्मा॒द् गाय॑न्तम् । गाय॑न्तꣳ॒॒ स्त्रियः॑ । स्त्रियः॑ कामयन्ते । का॒म॒य॒न्ते॒ कामु॑काः । कामु॑का एनम् । ए॒नꣳ॒॒ स्त्रियः॑ । स्त्रियो॑ भवन्ति । भ॒व॒न्ति॒ यः । य ए॒वम् । ए॒वम् ॅवेद॑ । वेदाथो᳚ । अथो॒ यः । अथो॒ इत्यथो᳚ । य ए॒वम् । ए॒वम् ॅवि॒द्वान् । वि॒द्वानपि॑ । अपि॒ जन्ये॑षु । जन्ये॑षु॒ भव॑ति । भव॑ति॒ तेभ्यः॑ । तेभ्य॑ ए॒व । ए॒व द॑दति । द॒द॒त्यु॒त । उ॒त यत् । यद् ब॒हुत॑याः । ब॒हुत॑या॒ भव॑न्ति । ब॒हुत॑या॒ इति॑ ब॒हु - त॒याः॒ \newline

\textbf{Jatai Paata} \newline

1. अ॒प॒क्रम्या॑ तिष्ठदतिष्ठ दप॒क्रम्या॑ प॒क्रम्या॑ तिष्ठत् । \newline
2. अ॒प॒क्रम्येत्य॑प - क्रम्य॑ । \newline
3. अ॒ति॒ष्ठ॒त् तत् तद॑तिष्ठ दतिष्ठ॒त् तत् । \newline
4. तद् रो॒हितो॑ रो॒हित॒ स्तत् तद् रो॒हितः॑ । \newline
5. रो॒हितो॒ जन्म॒ जन्म॑ रो॒हितो॑ रो॒हितो॒ जन्म॑ । \newline
6. जन्म॒ ते ते जन्म॒ जन्म॒ ते । \newline
7. ते दे॒वा दे॒वा स्ते ते दे॒वाः । \newline
8. दे॒वा अ॑ब्रुवन् नब्रुवन् दे॒वा दे॒वा अ॑ब्रुवन्न् । \newline
9. अ॒ब्रु॒व॒न् नपापा᳚ ब्रुवन् नब्रुव॒न् नप॑ । \newline
10. अप॑ यु॒ष्मद् यु॒ष्म दपाप॑ यु॒ष्मत् । \newline
11. यु॒ष्म दक्र॑मी॒ दक्र॑मीद् यु॒ष्मद् यु॒ष्म दक्र॑मीत् । \newline
12. अक्र॑मी॒न् न नाक्र॑मी॒ दक्र॑मी॒न् न । \newline
13. नास्मा न॒स्मान् न नास्मान् । \newline
14. अ॒स्मा नु॒पाव॑र्तत उ॒पाव॑र्तते॒ ऽस्मा न॒स्मा नु॒पाव॑र्तते । \newline
15. उ॒पाव॑र्तते॒ वि व्यु॑पाव॑र्तत उ॒पाव॑र्तते॒ वि । \newline
16. उ॒पाव॑र्तत॒ इत्यु॑प - आव॑र्तते । \newline
17. वि ह्व॑यामहै ह्वयामहै॒ वि वि ह्व॑यामहै । \newline
18. ह्व॒या॒म॒हा॒ इतीति॑ ह्वयामहै ह्वयामहा॒ इति॑ । \newline
19. इति॒ ब्रह्म॒ ब्रह्मे तीति॒ ब्रह्म॑ । \newline
20. ब्रह्म॑ गन्ध॒र्वा ग॑न्ध॒र्वा ब्रह्म॒ ब्रह्म॑ गन्ध॒र्वाः । \newline
21. ग॒न्ध॒र्वा अव॑द॒न् नव॑दन् गन्ध॒र्वा ग॑न्ध॒र्वा अव॑दन्न् । \newline
22. अव॑द॒न् नगा॑य॒न् नगा॑य॒न् नव॑द॒न् नव॑द॒न् नगा॑यन्न् । \newline
23. अगा॑यन् दे॒वा दे॒वा अगा॑य॒न् नगा॑यन् दे॒वाः । \newline
24. दे॒वाः सा सा दे॒वा दे॒वाः सा । \newline
25. सा दे॒वान् दे॒वान् थ्सा सा दे॒वान् । \newline
26. दे॒वान् गाय॑तो॒ गाय॑तो दे॒वान् दे॒वान् गाय॑तः । \newline
27. गाय॑त उ॒पाव॑र्ततो॒ पाव॑र्तत॒ गाय॑तो॒ गाय॑त उ॒पाव॑र्तत । \newline
28. उ॒पाव॑र्तत॒ तस्मा॒त् तस्मा॑ दु॒पाव॑र्ततो॒ पाव॑र्तत॒ तस्मा᳚त् । \newline
29. उ॒पाव॑र्त॒तेत्यु॑प - आव॑र्तत । \newline
30. तस्मा॒द् गाय॑न्त॒म् गाय॑न्त॒म् तस्मा॒त् तस्मा॒द् गाय॑न्तम् । \newline
31. गाय॑न्तꣳ॒॒ स्त्रियः॒ स्त्रियो॒ गाय॑न्त॒म् गाय॑न्तꣳ॒॒ स्त्रियः॑ । \newline
32. स्त्रियः॑ कामयन्ते कामयन्ते॒ स्त्रियः॒ स्त्रियः॑ कामयन्ते । \newline
33. का॒म॒य॒न्ते॒ कामु॑काः॒ कामु॑काः कामयन्ते कामयन्ते॒ कामु॑काः । \newline
34. कामु॑का एन मेन॒म् कामु॑काः॒ कामु॑का एनम् । \newline
35. ए॒नꣳ॒॒ स्त्रियः॒ स्त्रिय॑ एन मेनꣳ॒॒ स्त्रियः॑ । \newline
36. स्त्रियो॑ भवन्ति भवन्ति॒ स्त्रियः॒ स्त्रियो॑ भवन्ति । \newline
37. भ॒व॒न्ति॒ यो यो भ॑वन्ति भवन्ति॒ यः । \newline
38. य ए॒व मे॒वं ॅयो य ए॒वम् । \newline
39. ए॒वं ॅवेद॒ वेदै॒व मे॒वं ॅवेद॑ । \newline
40. वेदाथो॒ अथो॒ वेद॒ वेदाथो᳚ । \newline
41. अथो॒ यो यो ऽथो॒ अथो॒ यः । \newline
42. अथो॒ इत्यथो᳚ । \newline
43. य ए॒व मे॒वं ॅयो य ए॒वम् । \newline
44. ए॒वं ॅवि॒द्वान्. वि॒द्वा ने॒व मे॒वं ॅवि॒द्वान् । \newline
45. वि॒द्वा नप्यपि॑ वि॒द्वान्. वि॒द्वा नपि॑ । \newline
46. अपि॒ जन्ये॑षु॒ जन्ये॒ ष्वप्यपि॒ जन्ये॑षु । \newline
47. जन्ये॑षु॒ भव॑ति॒ भव॑ति॒ जन्ये॑षु॒ जन्ये॑षु॒ भव॑ति । \newline
48. भव॑ति॒ तेभ्य॒ स्तेभ्यो॒ भव॑ति॒ भव॑ति॒ तेभ्यः॑ । \newline
49. तेभ्य॑ ए॒वैव तेभ्य॒ स्तेभ्य॑ ए॒व । \newline
50. ए॒व द॑दति दद त्ये॒वैव द॑दति । \newline
51. द॒द॒ त्यु॒तोत द॑दति दद त्यु॒त । \newline
52. उ॒त यद् यदु॒तोत यत् । \newline
53. यद् ब॒हुत॑या ब॒हुत॑या॒ यद् यद् ब॒हुत॑याः । \newline
54. ब॒हुत॑या॒ भव॑न्ति॒ भव॑न्ति ब॒हुत॑या ब॒हुत॑या॒ भव॑न्ति । \newline
55. ब॒हुत॑या॒ इति॑ ब॒हु - त॒याः॒ । \newline

\textbf{Ghana Paata } \newline

1. अ॒प॒क्रम्या॑ तिष्ठ दतिष्ठ दप॒क्रम्या॑ प॒क्रम्या॑ तिष्ठ॒त् तत् तद॑तिष्ठ दप॒क्रम्या॑ प॒क्रम्या॑ तिष्ठ॒त् तत् । \newline
2. अ॒प॒क्रम्येत्य॑प - क्रम्य॑ । \newline
3. अ॒ति॒ष्ठ॒त् तत् तद॑तिष्ठ दतिष्ठ॒त् तद् रो॒हितो॑ रो॒हित॒ स्त द॑तिष्ठ दतिष्ठ॒त् तद् रो॒हितः॑ । \newline
4. तद् रो॒हितो॑ रो॒हित॒ स्तत् तद् रो॒हितो॒ जन्म॒ जन्म॑ रो॒हित॒ स्तत् तद् रो॒हितो॒ जन्म॑ । \newline
5. रो॒हितो॒ जन्म॒ जन्म॑ रो॒हितो॑ रो॒हितो॒ जन्म॒ ते ते जन्म॑ रो॒हितो॑ रो॒हितो॒ जन्म॒ ते । \newline
6. जन्म॒ ते ते जन्म॒ जन्म॒ ते दे॒वा दे॒वा स्ते जन्म॒ जन्म॒ ते दे॒वाः । \newline
7. ते दे॒वा दे॒वा स्ते ते दे॒वा अ॑ब्रुवन् नब्रुवन् दे॒वा स्ते ते दे॒वा अ॑ब्रुवन्न् । \newline
8. दे॒वा अ॑ब्रुवन् नब्रुवन् दे॒वा दे॒वा अ॑ब्रुव॒न् नपापा᳚ ब्रुवन् दे॒वा दे॒वा अ॑ब्रुव॒न् नप॑ । \newline
9. अ॒ब्रु॒व॒न् नपापा᳚ ब्रुवन् नब्रुव॒न् नप॑ यु॒ष्मद् यु॒ष्म दपा᳚ ब्रुवन् नब्रुव॒न् नप॑ यु॒ष्मत् । \newline
10. अप॑ यु॒ष्मद् यु॒ष्म दपाप॑ यु॒ष्म दक्र॑मी॒ दक्र॑मीद् यु॒ष्म दपाप॑ यु॒ष्म दक्र॑मीत् । \newline
11. यु॒ष्म दक्र॑मी॒ दक्र॑मीद् यु॒ष्मद् यु॒ष्म दक्र॑मी॒न् न नाक्र॑मीद् यु॒ष्मद् यु॒ष्म दक्र॑मी॒न् न । \newline
12. अक्र॑मी॒न् न नाक्र॑मी॒ दक्र॑मी॒न् नास्मा न॒स्मान् नाक्र॑मी॒ दक्र॑मी॒न् नास्मान् । \newline
13. नास्मा न॒स्मान् न नास्मा नु॒पाव॑र्तत उ॒पाव॑र्तते॒ ऽस्मान् न नास्मा नु॒पाव॑र्तते । \newline
14. अ॒स्मा नु॒पाव॑र्तत उ॒पाव॑र्तते॒ ऽस्मा न॒स्मा नु॒पाव॑र्तते॒ वि व्यु॑पाव॑र्तते॒ ऽस्मा न॒स्मा नु॒पाव॑र्तते॒ वि । \newline
15. उ॒पाव॑र्तते॒ वि व्यु॑पाव॑र्तत उ॒पाव॑र्तते॒ वि ह्व॑यामहै ह्वयामहै॒ व्यु॑पाव॑र्तत उ॒पाव॑र्तते॒ वि ह्व॑यामहै । \newline
16. उ॒पाव॑र्तत॒ इत्यु॑प - आव॑र्तते । \newline
17. वि ह्व॑यामहै ह्वयामहै॒ वि वि ह्व॑यामहा॒ इतीति॑ ह्वयामहै॒ वि वि ह्व॑यामहा॒ इति॑ । \newline
18. ह्व॒या॒म॒हा॒ इतीति॑ ह्वयामहै ह्वयामहा॒ इति॒ ब्रह्म॒ ब्रह्मेति॑ ह्वयामहै ह्वयामहा॒ इति॒ ब्रह्म॑ । \newline
19. इति॒ ब्रह्म॒ ब्रह्मे तीति॒ ब्रह्म॑ गन्ध॒र्वा ग॑न्ध॒र्वा ब्रह्मे तीति॒ ब्रह्म॑ गन्ध॒र्वाः । \newline
20. ब्रह्म॑ गन्ध॒र्वा ग॑न्ध॒र्वा ब्रह्म॒ ब्रह्म॑ गन्ध॒र्वा अव॑द॒ न्नव॑दन् गन्ध॒र्वा ब्रह्म॒ ब्रह्म॑ गन्ध॒र्वा अव॑दन्न् । \newline
21. ग॒न्ध॒र्वा अव॑द॒न् नव॑दन् गन्ध॒र्वा ग॑न्ध॒र्वा अव॑द॒न् नगा॑य॒न् नगा॑य॒न् नव॑दन् गन्ध॒र्वा ग॑न्ध॒र्वा अव॑द॒न् नगा॑यन्न् । \newline
22. अव॑द॒न् नगा॑य॒न् नगा॑य॒न् नव॑द॒न् नव॑द॒न् नगा॑यन् दे॒वा दे॒वा अगा॑य॒ नव॑द॒ न्नव॑द॒न् नगा॑यन् दे॒वाः । \newline
23. अगा॑यन् दे॒वा दे॒वा अगा॑य॒ न्नगा॑यन् दे॒वाः सा सा दे॒वा अगा॑य॒न् नगा॑यन् दे॒वाः सा । \newline
24. दे॒वाः सा सा दे॒वा दे॒वाः सा दे॒वान् दे॒वान् थ्सा दे॒वा दे॒वाः सा दे॒वान् । \newline
25. सा दे॒वान् दे॒वान् थ्सा सा दे॒वान् गाय॑तो॒ गाय॑तो दे॒वान् थ्सा सा दे॒वान् गाय॑तः । \newline
26. दे॒वान् गाय॑तो॒ गाय॑तो दे॒वान् दे॒वान् गाय॑त उ॒पाव॑र्ततो॒ पाव॑र्तत॒ गाय॑तो दे॒वान् दे॒वान् गाय॑त उ॒पाव॑र्तत । \newline
27. गाय॑त उ॒पाव॑र्ततो॒ पाव॑र्तत॒ गाय॑तो॒ गाय॑त उ॒पाव॑र्तत॒ तस्मा॒त् तस्मा॑ दु॒पाव॑र्तत॒ गाय॑तो॒ गाय॑त उ॒पाव॑र्तत॒ तस्मा᳚त् । \newline
28. उ॒पाव॑र्तत॒ तस्मा॒त् तस्मा॑ दु॒पाव॑र्ततो॒ पाव॑र्तत॒ तस्मा॒द् गाय॑न्त॒म् गाय॑न्त॒म् तस्मा॑ दु॒पाव॑र्ततो॒ पाव॑र्तत॒ तस्मा॒द् गाय॑न्तम् । \newline
29. उ॒पाव॑र्त॒तेत्यु॑प - आव॑र्तत । \newline
30. तस्मा॒द् गाय॑न्त॒म् गाय॑न्त॒म् तस्मा॒त् तस्मा॒द् गाय॑न्तꣳ॒॒ स्त्रियः॒ स्त्रियो॒ गाय॑न्त॒म् तस्मा॒त् तस्मा॒द् गाय॑न्तꣳ॒॒ स्त्रियः॑ । \newline
31. गाय॑न्तꣳ॒॒ स्त्रियः॒ स्त्रियो॒ गाय॑न्त॒म् गाय॑न्तꣳ॒॒ स्त्रियः॑ कामयन्ते कामयन्ते॒ स्त्रियो॒ गाय॑न्त॒म् गाय॑न्तꣳ॒॒ स्त्रियः॑ कामयन्ते । \newline
32. स्त्रियः॑ कामयन्ते कामयन्ते॒ स्त्रियः॒ स्त्रियः॑ कामयन्ते॒ कामु॑काः॒ कामु॑काः कामयन्ते॒ स्त्रियः॒ स्त्रियः॑ कामयन्ते॒ कामु॑काः । \newline
33. का॒म॒य॒न्ते॒ कामु॑काः॒ कामु॑काः कामयन्ते कामयन्ते॒ कामु॑का एन मेन॒म् कामु॑काः कामयन्ते कामयन्ते॒ कामु॑का एनम् । \newline
34. कामु॑का एन मेन॒म् कामु॑काः॒ कामु॑का एनꣳ॒॒ स्त्रियः॒ स्त्रिय॑ एन॒म् कामु॑काः॒ कामु॑का एनꣳ॒॒ स्त्रियः॑ । \newline
35. ए॒नꣳ॒॒ स्त्रियः॒ स्त्रिय॑ एन मेनꣳ॒॒ स्त्रियो॑ भवन्ति भवन्ति॒ स्त्रिय॑ एन मेनꣳ॒॒ स्त्रियो॑ भवन्ति । \newline
36. स्त्रियो॑ भवन्ति भवन्ति॒ स्त्रियः॒ स्त्रियो॑ भवन्ति॒ यो यो भ॑वन्ति॒ स्त्रियः॒ स्त्रियो॑ भवन्ति॒ यः । \newline
37. भ॒व॒न्ति॒ यो यो भ॑वन्ति भवन्ति॒ य ए॒व मे॒वं ॅयो भ॑वन्ति भवन्ति॒ य ए॒वम् । \newline
38. य ए॒व मे॒वं ॅयो य ए॒वं ॅवेद॒ वेदै॒वं ॅयो य ए॒वं ॅवेद॑ । \newline
39. ए॒वं ॅवेद॒ वेदै॒व मे॒वं ॅवेदाथो॒ अथो॒ वेदै॒व मे॒वं ॅवेदाथो᳚ । \newline
40. वेदाथो॒ अथो॒ वेद॒ वेदाथो॒ यो यो ऽथो॒ वेद॒ वेदाथो॒ यः । \newline
41. अथो॒ यो यो ऽथो॒ अथो॒ य ए॒व मे॒वं ॅयो ऽथो॒ अथो॒ य ए॒वम् । \newline
42. अथो॒ इत्यथो᳚ । \newline
43. य ए॒व मे॒वं ॅयो य ए॒वं ॅवि॒द्वान्. वि॒द्वा‍ने॒वं ॅयो य ए॒वं ॅवि॒द्वान् । \newline
44. ए॒वं ॅवि॒द्वान्. वि॒द्वाने॒व मे॒वं ॅवि॒द्वा नप्यपि॑ वि॒द्वाने॒व मे॒वं ॅवि॒द्वानपि॑ । \newline
45. वि॒द्वा नप्यपि॑ वि॒द्वान्. वि॒द्वानपि॒ जन्ये॑षु॒ जन्ये॒ष्वपि॑ वि॒द्वान्. वि॒द्वानपि॒ जन्ये॑षु । \newline
46. अपि॒ जन्ये॑षु॒ जन्ये॒ ष्वप्यपि॒ जन्ये॑षु॒ भव॑ति॒ भव॑ति॒ जन्ये॒ ष्वप्यपि॒ जन्ये॑षु॒ भव॑ति । \newline
47. जन्ये॑षु॒ भव॑ति॒ भव॑ति॒ जन्ये॑षु॒ जन्ये॑षु॒ भव॑ति॒ तेभ्य॒ स्तेभ्यो॒ भव॑ति॒ जन्ये॑षु॒ जन्ये॑षु॒ भव॑ति॒ तेभ्यः॑ । \newline
48. भव॑ति॒ तेभ्य॒ स्तेभ्यो॒ भव॑ति॒ भव॑ति॒ तेभ्य॑ ए॒वैव तेभ्यो॒ भव॑ति॒ भव॑ति॒ तेभ्य॑ ए॒व । \newline
49. तेभ्य॑ ए॒वैव तेभ्य॒ स्तेभ्य॑ ए॒व द॑दति दद त्ये॒व तेभ्य॒ स्तेभ्य॑ ए॒व द॑दति । \newline
50. ए॒व द॑दति दद त्ये॒वैव द॑द त्यु॒तोत द॑द त्ये॒वैव द॑द त्यु॒त । \newline
51. द॒द॒ त्यु॒तोत द॑दति दद त्यु॒त यद् यदु॒त द॑दति दद त्यु॒त यत् । \newline
52. उ॒त यद् यदु॒तोत यद् ब॒हुत॑या ब॒हुत॑या॒ यदु॒तोत यद् ब॒हुत॑याः । \newline
53. यद् ब॒हुत॑या ब॒हुत॑या॒ यद् यद् ब॒हुत॑या॒ भव॑न्ति॒ भव॑न्ति ब॒हुत॑या॒ यद् यद् ब॒हुत॑या॒ भव॑न्ति । \newline
54. ब॒हुत॑या॒ भव॑न्ति॒ भव॑न्ति ब॒हुत॑या ब॒हुत॑या॒ भव॒न् त्येक॑हाय॒ न्यैक॑हायन्या॒ भव॑न्ति ब॒हुत॑या ब॒हुत॑या॒ भव॒ न्त्येक॑हायन्या । \newline
55. ब॒हुत॑या॒ इति॑ ब॒हु - त॒याः॒ । \newline
\pagebreak
\markright{ TS 6.1.6.7  \hfill https://www.vedavms.in \hfill}

\section{ TS 6.1.6.7 }

\textbf{TS 6.1.6.7 } \newline
\textbf{Samhita Paata} \newline

भव॒न्त्येक॑हायन्या क्रीणाति वा॒चैवैनꣳ॒॒ सर्व॑या क्रीणाति॒ तस्मा॒देक॑हायना मनु॒ष्या॑ वाचं॑ ॅवद॒न्त्यकू॑ट॒या ऽक॑र्ण॒याऽ का॑ण॒याश्लो॑ण॒या ऽस॑प्तशफया क्रीणाति॒ सर्व॑यै॒वैनं॑ क्रीणाति॒ यच्छ्वे॒तया᳚ क्रीणी॒याद्-दु॒श्चर्मा॒ यज॑मानः स्या॒द्यत् कृ॒ष्णया॑-ऽनु॒स्तर॑णी स्यात् प्र॒मायु॑को॒ यज॑मानः स्या॒द्यद् द्वि॑रू॒पया॒ वात्र॑घ्नी स्या॒थ् स वा॒ऽन्यं जि॑नी॒यात् तं ॅवा॒ऽन्यो जि॑नीयादरु॒णया॑ पिङ्गा॒क्ष्या ( ) क्री॑णात्ये॒तद्वै सोम॑स्य रू॒पꣳ स्वयै॒वैनं॑ दे॒वत॑या क्रीणाति ॥ \newline

\textbf{Pada Paata} \newline

भव॑न्ति । एक॑हाय॒न्येत्येक॑ - हा॒य॒न्या॒ । क्री॒णा॒ति॒ । वा॒चा । ए॒व । ए॒न॒म् । सर्व॑या । क्री॒णा॒ति॒ । तस्मा᳚त् । एक॑हायना॒ इत्येक॑-हा॒य॒नाः॒ । म॒नु॒ष्याः᳚ । वाच᳚म् । व॒द॒न्ति॒ । अकू॑टया । अक॑र्णया । अका॑णया । अश्लो॑णया । अस॑प्तशफ॒येत्यस॑प्त - श॒फ॒या॒ । क्री॒णा॒ति॒ । सर्व॑या । ए॒व । ए॒न॒म् । क्री॒णा॒ति॒ । यत् । श्वे॒तया᳚ । क्री॒णी॒यात् । दु॒श्चर्मेति॑ दुः - चर्मा᳚ । यज॑मानः । स्या॒त् । यत् । कृ॒ष्णया᳚ । अ॒नु॒स्तर॒णीत्य॑नु - स्तर॑णी । स्या॒त् । प्र॒मायु॑क॒ इति॑ प्र - मायु॑कः । यज॑मानः । स्या॒त् । यत् । द्वि॒रू॒पयेति॑ द्वि - रू॒पया᳚ । वात्र॒घ्नीति॒ वात्र॑ - घ्नी॒ । स्या॒त् । सः । वा॒ । अ॒न्यम् । जि॒नी॒यात् । तम् । वा॒ । अ॒न्यः । जि॒नी॒या॒त् । अ॒रु॒णया᳚ । पि॒ङ्गा॒क्ष्येति॑ पिङ्ग - अ॒क्ष्या ( ) । क्री॒णा॒ति॒ । ए॒तत् । वै । सोम॑स्य । रू॒पम् । स्वया᳚ । ए॒व । ए॒न॒म् । दे॒वत॑या । क्री॒णा॒ति॒ ॥  \newline


\textbf{Krama Paata} \newline

भव॒न्त्येक॑हायन्या । एक॑हायन्या क्रीणाति । एक॑हाय॒न्येत्येक॑ - हा॒य॒न्या॒ । क्री॒णा॒ति॒ वा॒चा । वा॒चैव । ए॒वैन᳚म् । ए॒नꣳ॒॒ सर्व॑या । सर्व॑या क्रीणाति । क्री॒णा॒ति॒ तस्मा᳚त् । तस्मा॒देक॑हायनाः । एक॑हायना मनु॒ष्याः᳚ । एक॑हायना॒ इत्येक॑ - हा॒य॒नाः॒ । म॒नु॒ष्या॑ वाच᳚म् । वाच॑म् ॅवदन्ति । व॒द॒न्त्यकू॑टया । अकू॑ट॒याऽक॑र्णया । अक॑र्ण॒याऽका॑णया । अका॑ण॒याऽश्लो॑णया । अश्लो॑ण॒याऽस॑प्तशफया । अस॑प्तशफया क्रीणाति । अस॑प्तशफ॒येत्यस॑प्त - श॒फ॒या॒ । क्री॒णा॒ति॒ सर्व॑या । सर्व॑यै॒व । ए॒वैन᳚म् । ए॒न॒म् क्री॒णा॒ति॒ । क्री॒णा॒ति॒ यत् । यच् छ्‌वे॒तया᳚ । श्वे॒तया᳚ क्रीणी॒यात् । क्री॒णी॒याद् दु॒श्चर्मा᳚ । दु॒श्चर्मा॒ यज॑मानः । दु॒श्चर्मेति॑ दुः - चर्मा᳚ । यज॑मानः स्यात् । स्या॒द् यत् । यत् कृ॒ष्णया᳚ । कृ॒ष्णया॑ऽनु॒स्तर॑णी । अ॒नु॒स्तर॑णी स्यात् । अ॒नु॒स्तर॒णीत्य॑नु - स्तर॑णी । स्या॒त् प्र॒मायु॑कः । प्र॒मायु॑को॒ यज॑मानः । प्र॒मायु॑क॒ इति॑ प्र - मायु॑कः । यज॑मानः स्यात् । स्या॒द् यत् । यद् वि॑रू॒पया᳚ । द्वि॒रू॒पया॒ वार्त्र॑घ्नी । द्वि॒रू॒पयेति॑ द्वि - रू॒पया᳚ । वार्त्र॑घ्नी स्यात् । वार्त्र॒घ्नीति॒ वार्त्र॑ - घ्नी॒ । स्या॒थ् सः । स वा᳚ । वा॒ऽन्यम् । अ॒न्यम् जि॑नी॒यात् । जि॒नी॒यात् तम् । तम् ॅवा᳚ । वा॒ऽन्यः । अ॒न्यो जि॑नीयात् । जि॒नी॒या॒द॒रु॒णया᳚ । अ॒रु॒णया॑ पिङ्‍गा॒क्ष्या ( ) । पि॒ङ्‍गा॒क्ष्या क्री॑णाति । पि॒ङ्‍गा॒क्ष्येति॑ पिङ्‍ग - अ॒क्ष्या । क्री॒णा॒त्ये॒तत् । ए॒तद् वै । वै सोम॑स्य । सोम॑स्य रू॒पम् । रू॒पꣳ स्वया᳚ । स्वयै॒व । ए॒वैन᳚म् । ए॒न॒म् दे॒वत॑या । दे॒वत॑या क्रीणाति । क्री॒णा॒तीति॑ क्रीणाति । \newline

\textbf{Jatai Paata} \newline

1. भव॒न् त्येक॑हाय॒ न्यैक॑हायन्या॒ भव॑न्ति॒ भव॒न् त्येक॑हायन्या । \newline
2. एक॑हायन्या क्रीणाति क्रीणा॒त्ये क॑हाय॒ न्यैक॑हायन्या क्रीणाति । \newline
3. एक॑हाय॒न्येत्येक॑ - हा॒य॒न्या॒ । \newline
4. क्री॒णा॒ति॒ वा॒चा वा॒चा क्री॑णाति क्रीणाति वा॒चा । \newline
5. वा॒चै वैव वा॒चा वा॒चैव । \newline
6. ए॒वैन॑ मेन मे॒वै वैन᳚म् । \newline
7. ए॒नꣳ॒॒ सर्व॑या॒ सर्व॑ यैन मेनꣳ॒॒ सर्व॑या । \newline
8. सर्व॑या क्रीणाति क्रीणाति॒ सर्व॑या॒ सर्व॑या क्रीणाति । \newline
9. क्री॒णा॒ति॒ तस्मा॒त् तस्मा᳚त् क्रीणाति क्रीणाति॒ तस्मा᳚त् । \newline
10. तस्मा॒ देक॑हायना॒ एक॑हायना॒ स्तस्मा॒त् तस्मा॒ देक॑हायनाः । \newline
11. एक॑हायना मनु॒ष्या॑ मनु॒ष्या॑ एक॑हायना॒ एक॑हायना मनु॒ष्याः᳚ । \newline
12. एक॑हायना॒ इत्येक॑ - हा॒य॒नाः॒ । \newline
13. म॒नु॒ष्या॑ वाचं॒ ॅवाच॑म् मनु॒ष्या॑ मनु॒ष्या॑ वाच᳚म् । \newline
14. वाचं॑ ॅवदन्ति वदन्ति॒ वाचं॒ ॅवाचं॑ ॅवदन्ति । \newline
15. व॒द॒न् त्यकू॑ट॒या ऽकू॑टया वदन्ति वद॒न् त्यकू॑टया । \newline
16. अकू॑ट॒या ऽक॑र्ण॒या ऽक॑र्ण॒या ऽकू॑ट॒या ऽकू॑ट॒या ऽक॑र्णया । \newline
17. अक॑र्ण॒या ऽका॑ण॒या ऽका॑ण॒या ऽक॑र्ण॒या ऽक॑र्ण॒या ऽका॑णया । \newline
18. अका॑ण॒या ऽश्लो॑ण॒या ऽश्लो॑ण॒या ऽका॑ण॒या ऽका॑ण॒या ऽश्लो॑णया । \newline
19. अश्लो॑ण॒या ऽस॑प्तशफ॒या ऽस॑प्तशफ॒या ऽश्लो॑ण॒या ऽश्लो॑ण॒या ऽस॑प्तशफया । \newline
20. अस॑प्तशफया क्रीणाति क्रीणा॒त्य स॑प्तशफ॒या ऽस॑प्तशफया क्रीणाति । \newline
21. अस॑प्तशफ॒येत्यस॑प्त - श॒फ॒या॒ । \newline
22. क्री॒णा॒ति॒ सर्व॑या॒ सर्व॑या क्रीणाति क्रीणाति॒ सर्व॑या । \newline
23. सर्व॑ यै॒वैव सर्व॑या॒ सर्व॑यै॒व । \newline
24. ए॒वैन॑ मेन मे॒वै वैन᳚म् । \newline
25. ए॒न॒म् क्री॒णा॒ति॒ क्री॒णा॒ त्ये॒न॒ मे॒न॒म् क्री॒णा॒ति॒ । \newline
26. क्री॒णा॒ति॒ यद् यत् क्री॑णाति क्रीणाति॒ यत् । \newline
27. यच्छ्वे॒तया᳚ श्वे॒तया॒ यद् यच्छ्वे॒तया᳚ । \newline
28. श्वे॒तया᳚ क्रीणी॒यात् क्री॑णी॒या च्छ्वे॒तया᳚ श्वे॒तया᳚ क्रीणी॒यात् । \newline
29. क्री॒णी॒याद् दु॒श्चर्मा॑ दु॒श्चर्मा᳚ क्रीणी॒यात् क्री॑णी॒याद् दु॒श्चर्मा᳚ । \newline
30. दु॒श्चर्मा॒ यज॑मानो॒ यज॑मानो दु॒श्चर्मा॑ दु॒श्चर्मा॒ यज॑मानः । \newline
31. दु॒श्चर्मेति॑ दुः - चर्मा᳚ । \newline
32. यज॑मानः स्याथ् स्या॒द् यज॑मानो॒ यज॑मानः स्यात् । \newline
33. स्या॒द् यद् यथ् स्या᳚थ् स्या॒द् यत् । \newline
34. यत् कृ॒ष्णया॑ कृ॒ष्णया॒ यद् यत् कृ॒ष्णया᳚ । \newline
35. कृ॒ष्णया॑ ऽनु॒स्तर॑ण्य नु॒स्तर॑णी कृ॒ष्णया॑ कृ॒ष्णया॑ ऽनु॒स्तर॑णी । \newline
36. अ॒नु॒स्तर॑णी स्याथ् स्या दनु॒स्तर॑ ण्यनु॒स्तर॑णी स्यात् । \newline
37. अ॒नु॒स्तर॒णीत्य॑नु - स्तर॑णी । \newline
38. स्या॒त् प्र॒मायु॑कः प्र॒मायु॑कः स्याथ् स्यात् प्र॒मायु॑कः । \newline
39. प्र॒मायु॑को॒ यज॑मानो॒ यज॑मानः प्र॒मायु॑कः प्र॒मायु॑को॒ यज॑मानः । \newline
40. प्र॒मायु॑क॒ इति॑ प्र - मायु॑कः । \newline
41. यज॑मानः स्याथ् स्या॒द् यज॑मानो॒ यज॑मानः स्यात् । \newline
42. स्या॒द् यद् यथ् स्या᳚थ् स्या॒द् यत् । \newline
43. यद् द्वि॑रू॒पया᳚ द्विरू॒पया॒ यद् यद् द्वि॑रू॒पया᳚ । \newline
44. द्वि॒रू॒पया॒ वार्त्र॑घ्नी॒ वार्त्र॑घ्नी द्विरू॒पया᳚ द्विरू॒पया॒ वार्त्र॑घ्नी । \newline
45. द्वि॒रू॒पयेति॑ द्वि - रू॒पया᳚ । \newline
46. वार्त्र॑घ्नी स्याथ् स्या॒द् वार्त्र॑घ्नी॒ वार्त्र॑घ्नी स्यात् । \newline
47. वार्त्र॒घ्नीति॒ वार्त्र॑ - घ्नी॒ । \newline
48. स्या॒थ् स स स्या᳚थ् स्या॒थ् सः । \newline
49. स वा॑ वा॒ स स वा᳚ । \newline
50. वा॒ ऽन्य म॒न्यं ॅवा॑ वा॒ ऽन्यम् । \newline
51. अ॒न्यम् जि॑नी॒याज् जि॑नी॒या द॒न्य म॒न्यम् जि॑नी॒यात् । \newline
52. जि॒नी॒यात् तम् तम् जि॑नी॒याज् जि॑नी॒यात् तम् । \newline
53. तं ॅवा॑ वा॒ तम् तं ॅवा᳚ । \newline
54. वा॒ ऽन्यो᳚ ऽन्यो वा॑ वा॒ ऽन्यः । \newline
55. अ॒न्यो जि॑नीयाज् जिनीया द॒न्यो᳚ ऽन्यो जि॑नीयात् । \newline
56. जि॒नी॒या॒ द॒रु॒णया॑ ऽरु॒णया॑ जिनीयाज् जिनीया दरु॒णया᳚ । \newline
57. अ॒रु॒णया॑ पिङ्गा॒क्ष्या पि॑ङ्गा॒क्ष्या ऽरु॒णया॑ ऽरु॒णया॑ पिङ्गा॒क्ष्या । \newline
58. पि॒ङ्गा॒क्ष्या क्री॑णाति क्रीणाति पिङ्गा॒क्ष्या पि॑ङ्गा॒क्ष्या क्री॑णाति । \newline
59. पि॒ङ्गा॒क्ष्येति॑ पिङ्ग - अ॒क्ष्या । \newline
60. क्री॒णा॒ त्ये॒त दे॒तत् क्री॑णाति क्रीणा त्ये॒तत् । \newline
61. ए॒तद् वै वा ए॒त दे॒तद् वै । \newline
62. वै सोम॑स्य॒ सोम॑स्य॒ वै वै सोम॑स्य । \newline
63. सोम॑स्य रू॒पꣳ रू॒पꣳ सोम॑स्य॒ सोम॑स्य रू॒पम् । \newline
64. रू॒पꣳ स्वया॒ स्वया॑ रू॒पꣳ रू॒पꣳ स्वया᳚ । \newline
65. स्वयै॒ वैव स्वया॒ स्वयै॒व । \newline
66. ए॒वैन॑ मेन मे॒वै वैन᳚म् । \newline
67. ए॒न॒म् दे॒वत॑या दे॒वत॑ यैन मेनम् दे॒वत॑या । \newline
68. दे॒वत॑या क्रीणाति क्रीणाति दे॒वत॑या दे॒वत॑या क्रीणाति । \newline
69. क्री॒णा॒तीति॑ क्रीणाति । \newline

\textbf{Ghana Paata } \newline

1. भव॒न् त्येक॑हाय॒ न्यैक॑हायन्या॒ भव॑न्ति॒ भव॒न् त्येक॑हायन्या क्रीणाति क्रीणा॒ त्येक॑हायन्या॒ भव॑न्ति॒ भव॒न् त्येक॑हायन्या क्रीणाति । \newline
2. एक॑हायन्या क्रीणाति क्रीणा॒ त्येक॑हाय॒ न्यैक॑हायन्या क्रीणाति वा॒चा वा॒चा क्री॑णा॒ त्येक॑हाय॒ न्यैक॑हायन्या क्रीणाति वा॒चा । \newline
3. एक॑हाय॒न्येत्येक॑ - हा॒य॒न्या॒ । \newline
4. क्री॒णा॒ति॒ वा॒चा वा॒चा क्री॑णाति क्रीणाति वा॒चै वैव वा॒चा क्री॑णाति क्रीणाति वा॒चैव । \newline
5. वा॒चैवैव वा॒चा वा॒चै वैन॑ मेन मे॒व वा॒चा वा॒चै वैन᳚म् । \newline
6. ए॒वैन॑ मेन मे॒वै वैनꣳ॒॒ सर्व॑या॒ सर्व॑यैन मे॒वै वैनꣳ॒॒ सर्व॑या । \newline
7. ए॒नꣳ॒॒ सर्व॑या॒ सर्व॑यैन मेनꣳ॒॒ सर्व॑या क्रीणाति क्रीणाति॒ सर्व॑यैन मेनꣳ॒॒ सर्व॑या क्रीणाति । \newline
8. सर्व॑या क्रीणाति क्रीणाति॒ सर्व॑या॒ सर्व॑या क्रीणाति॒ तस्मा॒त् तस्मा᳚त् क्रीणाति॒ सर्व॑या॒ सर्व॑या क्रीणाति॒ तस्मा᳚त् । \newline
9. क्री॒णा॒ति॒ तस्मा॒त् तस्मा᳚त् क्रीणाति क्रीणाति॒ तस्मा॒ देक॑हायना॒ एक॑हायना॒ स्तस्मा᳚त् क्रीणाति क्रीणाति॒ तस्मा॒ देक॑हायनाः । \newline
10. तस्मा॒ देक॑हायना॒ एक॑हायना॒ स्तस्मा॒त् तस्मा॒ देक॑हायना मनु॒ष्या॑ मनु॒ष्या॑ एक॑हायना॒ स्तस्मा॒त् तस्मा॒ देक॑हायना मनु॒ष्याः᳚ । \newline
11. एक॑हायना मनु॒ष्या॑ मनु॒ष्या॑ एक॑हायना॒ एक॑हायना मनु॒ष्या॑ वाचं॒ ॅवाच॑म् मनु॒ष्या॑ एक॑हायना॒ एक॑हायना मनु॒ष्या॑ वाच᳚म् । \newline
12. एक॑हायना॒ इत्येक॑ - हा॒य॒नाः॒ । \newline
13. म॒नु॒ष्या॑ वाचं॒ ॅवाच॑म् मनु॒ष्या॑ मनु॒ष्या॑ वाचं॑ ॅवदन्ति वदन्ति॒ वाच॑म् मनु॒ष्या॑ मनु॒ष्या॑ वाचं॑ ॅवदन्ति । \newline
14. वाचं॑ ॅवदन्ति वदन्ति॒ वाचं॒ ॅवाचं॑ ॅवद॒ न्त्यकू॑ट॒या ऽकू॑टया वदन्ति॒ वाचं॒ ॅवाचं॑ ॅवद॒न् त्यकू॑टया । \newline
15. व॒द॒न् त्यकू॑ट॒या ऽकू॑टया वदन्ति वद॒न् त्यकू॑ट॒या ऽक॑र्ण॒या ऽक॑र्ण॒या ऽकू॑टया वदन्ति वद॒न् त्यकू॑ट॒या ऽक॑र्णया । \newline
16. अकू॑ट॒या ऽक॑र्ण॒या ऽक॑र्ण॒या ऽकू॑ट॒या ऽकू॑ट॒या ऽक॑र्ण॒या ऽका॑ण॒या ऽका॑ण॒या ऽक॑र्ण॒या ऽकू॑ट॒या ऽकू॑ट॒या ऽक॑र्ण॒या ऽका॑णया । \newline
17. अक॑र्ण॒या ऽका॑ण॒या ऽका॑ण॒या ऽक॑र्ण॒या ऽक॑र्ण॒या ऽका॑ण॒या ऽश्लो॑ण॒या ऽश्लो॑ण॒या ऽका॑ण॒या ऽक॑र्ण॒या ऽक॑र्ण॒या ऽका॑ण॒या ऽश्लो॑णया । \newline
18. अका॑ण॒या ऽश्लो॑ण॒या ऽश्लो॑ण॒या ऽका॑ण॒या ऽका॑ण॒या ऽश्लो॑ण॒या ऽस॑प्तशफ॒या ऽस॑प्तशफ॒या ऽश्लो॑ण॒या ऽका॑ण॒या ऽका॑ण॒या ऽश्लो॑ण॒या ऽस॑प्तशफया । \newline
19. अश्लो॑ण॒या ऽस॑प्तशफ॒या ऽस॑प्तशफ॒या ऽश्लो॑ण॒या ऽश्लो॑ण॒या ऽस॑प्तशफया क्रीणाति क्रीणा॒
त्यस॑प्तशफ॒या ऽश्लो॑ण॒या ऽश्लो॑ण॒या ऽस॑प्तशफया क्रीणाति । \newline
20. अस॑प्तशफया क्रीणाति क्रीणा॒ त्यस॑प्तशफ॒या ऽस॑प्तशफया क्रीणाति॒ सर्व॑या॒ सर्व॑या क्रीणा॒ त्यस॑प्तशफ॒या ऽस॑प्तशफया क्रीणाति॒ सर्व॑या । \newline
21. अस॑प्तशफ॒येत्यस॑प्त - श॒फ॒या॒ । \newline
22. क्री॒णा॒ति॒ सर्व॑या॒ सर्व॑या क्रीणाति क्रीणाति॒ सर्व॑ यै॒वैव सर्व॑या क्रीणाति क्रीणाति॒ सर्व॑यै॒व । \newline
23. सर्व॑ यै॒वैव सर्व॑या॒ सर्व॑ यै॒वैन॑ मेन मे॒व सर्व॑या॒ सर्व॑ यै॒वैन᳚म् । \newline
24. ए॒वैन॑ मेन मे॒वै वैन॑म् क्रीणाति क्रीणा त्येन मे॒वै वैन॑म् क्रीणाति । \newline
25. ए॒न॒म् क्री॒णा॒ति॒ क्री॒णा॒ त्ये॒न॒ मे॒न॒म् क्री॒णा॒ति॒ यद् यत् क्री॑णा त्येन मेनम् क्रीणाति॒ यत् । \newline
26. क्री॒णा॒ति॒ यद् यत् क्री॑णाति क्रीणाति॒ यच् छ्वे॒तया᳚ श्वे॒तया॒ यत् क्री॑णाति क्रीणाति॒ यच् छ्वे॒तया᳚ । \newline
27. यच् छ्वे॒तया᳚ श्वे॒तया॒ यद् यच् छ्वे॒तया᳚ क्रीणी॒यात् क्री॑णी॒याच् छ्वे॒तया॒ यद् यच् छ्वे॒तया᳚ क्रीणी॒यात् । \newline
28. श्वे॒तया᳚ क्रीणी॒यात् क्री॑णी॒याच् छ्वे॒तया᳚ श्वे॒तया᳚ क्रीणी॒याद् दु॒श्चर्मा॑ दु॒श्चर्मा᳚ क्रीणी॒याच् छ्वे॒तया᳚ श्वे॒तया᳚ क्रीणी॒याद् दु॒श्चर्मा᳚ । \newline
29. क्री॒णी॒याद् दु॒श्चर्मा॑ दु॒श्चर्मा᳚ क्रीणी॒यात् क्री॑णी॒याद् दु॒श्चर्मा॒ यज॑मानो॒ यज॑मानो दु॒श्चर्मा᳚ क्रीणी॒यात् क्री॑णी॒याद् दु॒श्चर्मा॒ यज॑मानः । \newline
30. दु॒श्चर्मा॒ यज॑मानो॒ यज॑मानो दु॒श्चर्मा॑ दु॒श्चर्मा॒ यज॑मानः स्याथ् स्या॒द् यज॑मानो दु॒श्चर्मा॑ दु॒श्चर्मा॒ यज॑मानः स्यात् । \newline
31. दु॒श्चर्मेति॑ दुः - चर्मा᳚ । \newline
32. यज॑मानः स्याथ् स्या॒द् यज॑मानो॒ यज॑मानः स्या॒द् यद् यथ् स्या॒द् यज॑मानो॒ यज॑मानः स्या॒द् यत् । \newline
33. स्या॒द् यद् यथ् स्या᳚थ् स्या॒द् यत् कृ॒ष्णया॑ कृ॒ष्णया॒ यथ् स्या᳚थ् स्या॒द् यत् कृ॒ष्णया᳚ । \newline
34. यत् कृ॒ष्णया॑ कृ॒ष्णया॒ यद् यत् कृ॒ष्णया॑ ऽनु॒स्तर॑ ण्यनु॒स्तर॑णी कृ॒ष्णया॒ यद् यत् कृ॒ष्णया॑ ऽनु॒स्तर॑णी । \newline
35. कृ॒ष्णया॑ ऽनु॒स्तर॑ ण्यनु॒स्तर॑णी कृ॒ष्णया॑ कृ॒ष्णया॑ ऽनु॒स्तर॑णी स्याथ् स्या दनु॒स्तर॑णी कृ॒ष्णया॑ कृ॒ष्णया॑ ऽनु॒स्तर॑णी स्यात् । \newline
36. अ॒नु॒स्तर॑णी स्याथ् स्या दनु॒स्तर॑ ण्यनु॒स्तर॑णी स्यात् प्र॒मायु॑कः प्र॒मायु॑कः स्या दनु॒स्तर॑ ण्यनु॒स्तर॑णी स्यात् प्र॒मायु॑कः । \newline
37. अ॒नु॒स्तर॒णीत्य॑नु - स्तर॑णी । \newline
38. स्या॒त् प्र॒मायु॑कः प्र॒मायु॑कः स्याथ् स्यात् प्र॒मायु॑को॒ यज॑मानो॒ यज॑मानः प्र॒मायु॑कः स्याथ् स्यात् प्र॒मायु॑को॒ यज॑मानः । \newline
39. प्र॒मायु॑को॒ यज॑मानो॒ यज॑मानः प्र॒मायु॑कः प्र॒मायु॑को॒ यज॑मानः स्याथ् स्या॒द् यज॑मानः प्र॒मायु॑कः प्र॒मायु॑को॒ यज॑मानः स्यात् । \newline
40. प्र॒मायु॑क॒ इति॑ प्र - मायु॑कः । \newline
41. यज॑मानः स्याथ् स्या॒द् यज॑मानो॒ यज॑मानः स्या॒द् यद् यथ् स्या॒द् यज॑मानो॒ यज॑मानः स्या॒द् यत् । \newline
42. स्या॒द् यद् यथ् स्या᳚थ् स्या॒द् यद् द्वि॑रू॒पया᳚ द्विरू॒पया॒ यथ् स्या᳚थ् स्या॒द् यद् द्वि॑रू॒पया᳚ । \newline
43. यद् द्वि॑रू॒पया᳚ द्विरू॒पया॒ यद् यद् द्वि॑रू॒पया॒ वार्त्र॑घ्नी॒ वार्त्र॑घ्नी द्विरू॒पया॒ यद् यद् द्वि॑रू॒पया॒ वार्त्र॑घ्नी । \newline
44. द्वि॒रू॒पया॒ वार्त्र॑घ्नी॒ वार्त्र॑घ्नी द्विरू॒पया᳚ द्विरू॒पया॒ वार्त्र॑घ्नी स्याथ् स्या॒द् वार्त्र॑घ्नी द्विरू॒पया᳚ द्विरू॒पया॒ वार्त्र॑घ्नी स्यात् । \newline
45. द्वि॒रू॒पयेति॑ द्वि - रू॒पया᳚ । \newline
46. वार्त्र॑घ्नी स्याथ् स्या॒द् वार्त्र॑घ्नी॒ वार्त्र॑घ्नी स्या॒थ् स स स्या॒द् वार्त्र॑घ्नी॒ वार्त्र॑घ्नी स्या॒थ् सः । \newline
47. वार्त्र॒घ्नीति॒ वार्त्र॑ - घ्नी॒ । \newline
48. स्या॒थ् स स स्या᳚थ् स्या॒थ् स वा॑ वा॒ स स्या᳚थ् स्या॒थ् स वा᳚ । \newline
49. स वा॑ वा॒ स स वा॒ ऽन्य म॒न्यं ॅवा॒ स स वा॒ ऽन्यम् । \newline
50. वा॒ ऽन्य म॒न्यं ॅवा॑ वा॒ ऽन्यम् जि॑नी॒याज् जि॑नी॒या द॒न्यं ॅवा॑ वा॒ ऽन्यम् जि॑नी॒यात् । \newline
51. अ॒न्यम् जि॑नी॒याज् जि॑नी॒या द॒न्य म॒न्यम् जि॑नी॒यात् तम् तम् जि॑नी॒या द॒न्य म॒न्यम् जि॑नी॒यात् तम् । \newline
52. जि॒नी॒यात् तम् तम् जि॑नी॒याज् जि॑नी॒यात् तं ॅवा॑ वा॒ तम् जि॑नी॒याज् जि॑नी॒यात् तं ॅवा᳚ । \newline
53. तं ॅवा॑ वा॒ तम् तं ॅवा॒ ऽन्यो᳚ ऽन्यो वा॒ तम् तं ॅवा॒ ऽन्यः । \newline
54. वा॒ ऽन्यो᳚ ऽन्यो वा॑ वा॒ ऽन्यो जि॑नीयाज् जिनीया द॒न्यो वा॑ वा॒ ऽन्यो जि॑नीयात् । \newline
55. अ॒न्यो जि॑नीयाज् जिनीया द॒न्यो᳚ ऽन्यो जि॑नीया दरु॒णया॑ ऽरु॒णया॑ जिनीया द॒न्यो᳚ ऽन्यो जि॑नीया दरु॒णया᳚ । \newline
56. जि॒नी॒या॒ द॒रु॒णया॑ ऽरु॒णया॑ जिनीयाज् जिनीया दरु॒णया॑ पिङ्गा॒क्ष्या पि॑ङ्गा॒क्ष्या ऽरु॒णया॑ जिनीयाज् जिनीया दरु॒णया॑ पिङ्गा॒क्ष्या । \newline
57. अ॒रु॒णया॑ पिङ्गा॒क्ष्या पि॑ङ्गा॒क्ष्या ऽरु॒णया॑ ऽरु॒णया॑ पिङ्गा॒क्ष्या क्री॑णाति क्रीणाति पिङ्गा॒क्ष्या ऽरु॒णया॑ ऽरु॒णया॑ पिङ्गा॒क्ष्या क्री॑णाति । \newline
58. पि॒ङ्गा॒क्ष्या क्री॑णाति क्रीणाति पिङ्गा॒क्ष्या पि॑ङ्गा॒क्ष्या क्री॑णा त्ये॒त दे॒तत् क्री॑णाति पिङ्गा॒क्ष्या पि॑ङ्गा॒क्ष्या क्री॑णा त्ये॒तत् । \newline
59. पि॒ङ्गा॒क्ष्येति॑ पिङ्ग - अ॒क्ष्या । \newline
60. क्री॒णा॒ त्ये॒त दे॒तत् क्री॑णाति क्रीणा त्ये॒तद् वै वा ए॒तत् क्री॑णाति क्रीणा त्ये॒तद् वै । \newline
61. ए॒तद् वै वा ए॒त दे॒तद् वै सोम॑स्य॒ सोम॑स्य॒ वा ए॒त दे॒तद् वै सोम॑स्य । \newline
62. वै सोम॑स्य॒ सोम॑स्य॒ वै वै सोम॑स्य रू॒पꣳ रू॒पꣳ सोम॑स्य॒ वै वै सोम॑स्य रू॒पम् । \newline
63. सोम॑स्य रू॒पꣳ रू॒पꣳ सोम॑स्य॒ सोम॑स्य रू॒पꣳ स्वया॒ स्वया॑ रू॒पꣳ सोम॑स्य॒ सोम॑स्य रू॒पꣳ स्वया᳚ । \newline
64. रू॒पꣳ स्वया॒ स्वया॑ रू॒पꣳ रू॒पꣳ स्वयै॒ वैव स्वया॑ रू॒पꣳ रू॒पꣳ स्वयै॒व । \newline
65. स्वयै॒ वैव स्वया॒ स्वयै॒ वैन॑ मेन मे॒व स्वया॒ स्वयै॒ वैन᳚म् । \newline
66. ए॒वैन॑ मेन मे॒वै वैन॑म् दे॒वत॑या दे॒वत॑यैन मे॒वै वैन॑म् दे॒वत॑या । \newline
67. ए॒न॒म् दे॒वत॑या दे॒वत॑ यैन मेनम् दे॒वत॑या क्रीणाति क्रीणाति दे॒वत॑ यैन मेनम् दे॒वत॑या क्रीणाति । \newline
68. दे॒वत॑या क्रीणाति क्रीणाति दे॒वत॑या दे॒वत॑या क्रीणाति । \newline
69. क्री॒णा॒तीति॑ क्रीणाति । \newline
\pagebreak
\markright{ TS 6.1.7.1  \hfill https://www.vedavms.in \hfill}

\section{ TS 6.1.7.1 }

\textbf{TS 6.1.7.1 } \newline
\textbf{Samhita Paata} \newline

तद्धिर॑ण्यमभव॒त् तस्मा॑द॒द्भ्यो हिर॑ण्यं पुनन्ति ब्रह्मवा॒दिनो॑ वदन्ति॒ कस्मा᳚थ् स॒त्याद॑न॒स्थिके॑न प्र॒जाः प्र॒वीय॑न्ते ऽस्थ॒न्वती᳚र्जायन्त॒ इति॒ यद्धिर॑ण्यं घृ॒ते॑ऽव॒धाय॑ जु॒होति॒ तस्मा॑दन॒स्थिके॑न प्र॒जाः प्र वी॑यन्ते ऽस्थ॒न्वती᳚र्जायन्त ए॒तद्वा अ॒ग्नेः प्रि॒यं धाम॒ यद्-घृ॒तं तेजो॒ हिर॑ण्यमि॒यन्ते॑ शुक्र त॒नूरि॒दं ॅवर्च॒ इत्या॑ह॒ सते॑जसमे॒वैनꣳ॒॒ सत॑नुं - [  ] \newline

\textbf{Pada Paata} \newline

तत् । हिर॑ण्यम् । अ॒भ॒व॒त् । तस्मा᳚त् । अ॒द्भ्य इत्य॑त् - भ्यः । हिर॑ण्यम् । पु॒न॒न्ति॒ । ब्र॒ह्म॒वा॒दिन॒ इति॑ ब्रह्म - वा॒दिनः॑ । व॒द॒न्ति॒ । कस्मा᳚त् । स॒त्यात् । अ॒न॒स्थिके॑न । प्र॒जा इति॑ प्र - जाः । प्र॒वीय॑न्त॒ इति॑ प्र - वीय॑न्ते । अ॒स्थ॒न्वती॒रित्य॑स्थन्न् - वतीः᳚ । जा॒य॒न्ते॒ । इति॑ । यत् । हिर॑ण्यम् । घृ॒ते । अ॒व॒धायेत्य॑व - धाय॑ । जु॒होति॑ । तस्मा᳚त् । अ॒न॒स्थिके॑न । प्र॒जा इति॑ प्र - जाः । प्रेति॑ । वी॒य॒न्ते॒ । अ॒स्थ॒न्वती॒रित्य॑स्थन्न् - वतीः᳚ । जा॒य॒न्ते॒ । ए॒तत् । वै । अ॒ग्नेः । प्रि॒यम् । धाम॑ । यत् । घृ॒तम् । तेजः॑ । हिर॑ण्यम् । इ॒यम् । ते॒ । शु॒क्र॒ । त॒नूः । इ॒दम् । वर्चः॑ । इति॑ । आ॒ह॒ । सते॑जस॒मिति॒ स - ते॒ज॒स॒म् । ए॒व । ए॒न॒म् । सत॑नु॒मिति॒ स - त॒नु॒म् ।  \newline


\textbf{Krama Paata} \newline

तद्‌धिर॑ण्यम् । हिर॑ण्यमभवत् । अ॒भ॒व॒त् तस्मा᳚त् । तस्मा॑द॒द्भ्यः । अ॒द्भ्यो हिर॑ण्यम् । अ॒द्भ्य इत्य॑त् - भ्यः । हिर॑ण्यम् पुनन्ति । पु॒न॒न्ति॒ ब्र॒ह्म॒वा॒दिनः॑ । ब्र॒ह्म॒वा॒दिनो॑ वदन्ति । ब्र॒ह्म॒वा॒दिन॒ इति॑ ब्रह्म - वा॒दिनः॑ । व॒द॒न्ति॒ कस्मा᳚त् । कस्मा᳚थ् स॒त्यात् । स॒त्याद॑न॒स्थिके॑न । अ॒न॒स्थिके॑न प्र॒जाः । प्र॒जाः प्र॒वीय॑न्ते । प्र॒जा इति॑ प्र - जाः । प्र॒वीय॑न्तेऽस्थ॒न्वतीः᳚ । प्र॒वीय॑न्त॒ इति॑ प्र - वीय॑न्ते । अ॒स्थ॒न्वती᳚र् जायन्ते । अ॒स्थ॒न्वती॒रित्य॑स्थन्न् - वतीः᳚ । जा॒य॒न्त॒ इति॑ । इति॒ यत् । यद्‌धिर॑ण्यम् । हिर॑ण्यम् घृ॒ते । घृ॒ते॑ऽव॒धाय॑ । अ॒व॒धाय॑ जु॒होति॑ । अ॒व॒धायेत्य॑व - धाय॑ । जु॒होति॒ तस्मा᳚त् । तस्मा॑दन॒स्थिके॑न । अ॒न॒स्थिके॑न प्र॒जाः । प्र॒जाः प्र । प्र॒जा इति॑ प्र - जाः । प्र वी॑यन्ते । वी॒य॒न्ते॒ऽस्थ॒न्वतीः᳚ । अ॒स्थ॒न्वती᳚र् जायन्ते । अ॒स्थ॒न्वती॒रित्य॑स्थन्न् - वतीः᳚ । जा॒य॒न्त॒ ए॒तत् । ए॒तद् वै । वा अ॒ग्नेः । अ॒ग्नेः प्रि॒यम् । प्रि॒यम् धाम॑ । धाम॒ यत् । यद् घृ॒तम् । घृ॒तम् तेजः॑ । तेजो॒ हिर॑ण्यम् । हिर॑ण्यमि॒यम् । इ॒यम् ते᳚ । ते॒ शु॒क्र॒ । शु॒क्र॒ त॒नूः । त॒नूरि॒दम् । इ॒दम् ॅवर्चः॑ । वर्च॒ इति॑ । इत्या॑ह । आ॒ह॒ सते॑जसम् । सते॑जसमे॒व । सते॑जस॒मिति॒ स - ते॒ज॒स॒म् । ए॒वैन᳚म् । ए॒नꣳ॒॒ सत॑नुम् । सत॑नुम् करोति । सत॑नु॒मिति॒ स - त॒नु॒म् \newline

\textbf{Jatai Paata} \newline

1. तद्धिर॑ण्यꣳ॒॒ हिर॑ण्य॒म् तत् तद्धिर॑ण्यम् । \newline
2. हिर॑ण्य मभव दभव॒ द्धिर॑ण्यꣳ॒॒ हिर॑ण्य मभवत् । \newline
3. अ॒भ॒व॒त् तस्मा॒त् तस्मा॑ दभव दभव॒त् तस्मा᳚त् । \newline
4. तस्मा॑ द॒द्भ्यो᳚ ऽद्भ्य स्तस्मा॒त् तस्मा॑ द॒द्भ्यः । \newline
5. अ॒द्भ्यो हिर॑ण्यꣳ॒॒ हिर॑ण्य म॒द्भ्यो᳚ ऽद्भ्यो हिर॑ण्यम् । \newline
6. अ॒द्भ्य इत्य॑त् - भ्यः । \newline
7. हिर॑ण्यम् पुनन्ति पुनन्ति॒ हिर॑ण्यꣳ॒॒ हिर॑ण्यम् पुनन्ति । \newline
8. पु॒न॒न्ति॒ ब्र॒ह्म॒वा॒दिनो᳚ ब्रह्मवा॒दिनः॑ पुनन्ति पुनन्ति ब्रह्मवा॒दिनः॑ । \newline
9. ब्र॒ह्म॒वा॒दिनो॑ वदन्ति वदन्ति ब्रह्मवा॒दिनो᳚ ब्रह्मवा॒दिनो॑ वदन्ति । \newline
10. ब्र॒ह्म॒वा॒दिन॒ इति॑ ब्रह्म - वा॒दिनः॑ । \newline
11. व॒द॒न्ति॒ कस्मा॒त् कस्मा᳚द् वदन्ति वदन्ति॒ कस्मा᳚त् । \newline
12. कस्मा᳚थ् स॒त्याथ् स॒त्यात् कस्मा॒त् कस्मा᳚थ् स॒त्यात् । \newline
13. स॒त्या द॑न॒स्थिके॑ना न॒स्थिके॑न स॒त्याथ् स॒त्या द॑न॒स्थिके॑न । \newline
14. अ॒न॒स्थिके॑न प्र॒जाः प्र॒जा अ॑न॒स्थिके॑ना न॒स्थिके॑न प्र॒जाः । \newline
15. प्र॒जाः प्र॒वीय॑न्ते प्र॒वीय॑न्ते प्र॒जाः प्र॒जाः प्र॒वीय॑न्ते । \newline
16. प्र॒जा इति॑ प्र - जाः । \newline
17. प्र॒वीय॑न्ते ऽस्थ॒न्वती॑ रस्थ॒न्वतीः᳚ प्र॒वीय॑न्ते प्र॒वीय॑न्ते ऽस्थ॒न्वतीः᳚ । \newline
18. प्र॒वीय॑न्त॒ इति॑ प्र - वीय॑न्ते । \newline
19. अ॒स्थ॒न्वती᳚र् जायन्ते जायन्ते ऽस्थ॒न्वती॑ रस्थ॒न्वती᳚र् जायन्ते । \newline
20. अ॒स्थ॒न्वती॒रित्य॑स्थन्न् - वतीः᳚ । \newline
21. जा॒य॒न्त॒ इतीति॑ जायन्ते जायन्त॒ इति॑ । \newline
22. इति॒ यद् यदितीति॒ यत् । \newline
23. यद्धिर॑ण्यꣳ॒॒ हिर॑ण्यं॒ ॅयद् यद्धिर॑ण्यम् । \newline
24. हिर॑ण्यम् घृ॒ते घृ॒ते हिर॑ण्यꣳ॒॒ हिर॑ण्यम् घृ॒ते । \newline
25. घृ॒ते॑ ऽव॒धाया ॑व॒धाय॑ घृ॒ते घृ॒ते॑ ऽव॒धाय॑ । \newline
26. अ॒व॒धाय॑ जु॒होति॑ जु॒हो त्य॑व॒धाया॑ व॒धाय॑ जु॒होति॑ । \newline
27. अ॒व॒धायेत्य॑व - धाय॑ । \newline
28. जु॒होति॒ तस्मा॒त् तस्मा᳚ज् जु॒होति॑ जु॒होति॒ तस्मा᳚त् । \newline
29. तस्मा॑ दन॒स्थिके॑ना न॒स्थिके॑न॒ तस्मा॒त् तस्मा॑ दन॒स्थिके॑न । \newline
30. अ॒न॒स्थिके॑न प्र॒जाः प्र॒जा अ॑न॒स्थिके॑ना न॒स्थिके॑न प्र॒जाः । \newline
31. प्र॒जाः प्र प्र प्र॒जाः प्र॒जाः प्र । \newline
32. प्र॒जा इति॑ प्र - जाः । \newline
33. प्र वी॑यन्ते वीयन्ते॒ प्र प्र वी॑यन्ते । \newline
34. वी॒य॒न्ते॒ ऽस्थ॒न्वती॑ रस्थ॒न्वती᳚र् वीयन्ते वीयन्ते ऽस्थ॒न्वतीः᳚ । \newline
35. अ॒स्थ॒न्वती᳚र् जायन्ते जायन्ते ऽस्थ॒न्वती॑ रस्थ॒न्वती᳚र् जायन्ते । \newline
36. अ॒स्थ॒न्वती॒रित्य॑स्थन्न् - वतीः᳚ । \newline
37. जा॒य॒न्त॒ ए॒त दे॒तज् जा॑यन्ते जायन्त ए॒तत् । \newline
38. ए॒तद् वै वा ए॒त दे॒तद् वै । \newline
39. वा अ॒ग्ने र॒ग्नेर् वै वा अ॒ग्नेः । \newline
40. अ॒ग्नेः प्रि॒यम् प्रि॒य म॒ग्ने र॒ग्नेः प्रि॒यम् । \newline
41. प्रि॒यम् धाम॒ धाम॑ प्रि॒यम् प्रि॒यम् धाम॑ । \newline
42. धाम॒ यद् यद् धाम॒ धाम॒ यत् । \newline
43. यद् घृ॒तम् घृ॒तं ॅयद् यद् घृ॒तम् । \newline
44. घृ॒तम् तेज॒ स्तेजो॑ घृ॒तम् घृ॒तम् तेजः॑ । \newline
45. तेजो॒ हिर॑ण्यꣳ॒॒ हिर॑ण्य॒म् तेज॒ स्तेजो॒ हिर॑ण्यम् । \newline
46. हिर॑ण्य मि॒य मि॒यꣳ हिर॑ण्यꣳ॒॒ हिर॑ण्य मि॒यम् । \newline
47. इ॒यम् ते॑ त इ॒य मि॒यम् ते᳚ । \newline
48. ते॒ शु॒क्र॒ शु॒क्र॒ ते॒ ते॒ शु॒क्र॒ । \newline
49. शु॒क्र॒ त॒नू स्त॒नूः शु॑क्र शुक्र त॒नूः । \newline
50. त॒नू रि॒द मि॒दम् त॒नू स्त॒नू रि॒दम् । \newline
51. इ॒दं ॅवर्चो॒ वर्च॑ इ॒द मि॒दं ॅवर्चः॑ । \newline
52. वर्च॒ इतीति॒ वर्चो॒ वर्च॒ इति॑ । \newline
53. इत्या॑हा॒हे तीत्या॑ह । \newline
54. आ॒ह॒ सते॑जसꣳ॒॒ सते॑जस माहाह॒ सते॑जसम् । \newline
55. सते॑जस मे॒वैव सते॑जसꣳ॒॒ सते॑जस मे॒व । \newline
56. सते॑जस॒मिति॒ स - ते॒ज॒स॒म् । \newline
57. ए॒वैन॑ मेन मे॒वै वैन᳚म् । \newline
58. ए॒नꣳ॒॒ सत॑नुꣳ॒॒ सत॑नु मेन मेनꣳ॒॒ सत॑नुम् । \newline
59. सत॑नुम् करोति करोति॒ सत॑नुꣳ॒॒ सत॑नुम् करोति । \newline
60. सत॑नु॒मिति॒ स - त॒नु॒म् । \newline

\textbf{Ghana Paata } \newline

1. तद्धिर॑ण्यꣳ॒॒ हिर॑ण्य॒म् तत् तद्धिर॑ण्य मभव दभव॒ द्धिर॑ण्य॒म् तत् तद्धिर॑ण्य मभवत् । \newline
2. हिर॑ण्य मभव दभव॒ द्धिर॑ण्यꣳ॒॒ हिर॑ण्य मभव॒त् तस्मा॒त् तस्मा॑ दभव॒ द्धिर॑ण्यꣳ॒॒ हिर॑ण्य मभव॒त् तस्मा᳚त् । \newline
3. अ॒भ॒व॒त् तस्मा॒त् तस्मा॑ दभव दभव॒त् तस्मा॑ द॒द्भ्यो᳚ ऽद्भ्य स्तस्मा॑ दभव दभव॒त् तस्मा॑ द॒द्भ्यः । \newline
4. तस्मा॑ द॒द्भ्यो᳚ ऽद्भ्य स्तस्मा॒त् तस्मा॑ द॒द्भ्यो हिर॑ण्यꣳ॒॒ हिर॑ण्य म॒द्भ्य स्तस्मा॒त् तस्मा॑ द॒द्भ्यो हिर॑ण्यम् । \newline
5. अ॒द्भ्यो हिर॑ण्यꣳ॒॒ हिर॑ण्य म॒द्भ्यो᳚ ऽद्भ्यो हिर॑ण्यम् पुनन्ति पुनन्ति॒ हिर॑ण्य म॒द्भ्यो᳚ ऽद्भ्यो हिर॑ण्यम् पुनन्ति । \newline
6. अ॒द्भ्य इत्य॑त् - भ्यः । \newline
7. हिर॑ण्यम् पुनन्ति पुनन्ति॒ हिर॑ण्यꣳ॒॒ हिर॑ण्यम् पुनन्ति ब्रह्मवा॒दिनो᳚ ब्रह्मवा॒दिनः॑ पुनन्ति॒ हिर॑ण्यꣳ॒॒ हिर॑ण्यम् पुनन्ति ब्रह्मवा॒दिनः॑ । \newline
8. पु॒न॒न्ति॒ ब्र॒ह्म॒वा॒दिनो᳚ ब्रह्मवा॒दिनः॑ पुनन्ति पुनन्ति ब्रह्मवा॒दिनो॑ वदन्ति वदन्ति ब्रह्मवा॒दिनः॑ पुनन्ति पुनन्ति ब्रह्मवा॒दिनो॑ वदन्ति । \newline
9. ब्र॒ह्म॒वा॒दिनो॑ वदन्ति वदन्ति ब्रह्मवा॒दिनो᳚ ब्रह्मवा॒दिनो॑ वदन्ति॒ कस्मा॒त् कस्मा᳚द् वदन्ति ब्रह्मवा॒दिनो᳚ ब्रह्मवा॒दिनो॑ वदन्ति॒ कस्मा᳚त् । \newline
10. ब्र॒ह्म॒वा॒दिन॒ इति॑ ब्रह्म - वा॒दिनः॑ । \newline
11. व॒द॒न्ति॒ कस्मा॒त् कस्मा᳚द् वदन्ति वदन्ति॒ कस्मा᳚थ् स॒त्याथ् स॒त्यात् कस्मा᳚द् वदन्ति वदन्ति॒ कस्मा᳚थ् स॒त्यात् । \newline
12. कस्मा᳚थ् स॒त्याथ् स॒त्यात् कस्मा॒त् कस्मा᳚थ् स॒त्या द॑न॒स्थिके॑ना न॒स्थिके॑न स॒त्यात् कस्मा॒त् कस्मा᳚थ् स॒त्या द॑न॒स्थिके॑न । \newline
13. स॒त्या द॑न॒स्थिके॑ना न॒स्थिके॑न स॒त्याथ् स॒त्या द॑न॒स्थिके॑न प्र॒जाः प्र॒जा अ॑न॒स्थिके॑न स॒त्याथ् स॒त्या द॑न॒स्थिके॑न प्र॒जाः । \newline
14. अ॒न॒स्थिके॑न प्र॒जाः प्र॒जा अ॑न॒स्थिके॑ना न॒स्थिके॑न प्र॒जाः प्र॒वीय॑न्ते प्र॒वीय॑न्ते प्र॒जा अ॑न॒स्थिके॑ना न॒स्थिके॑न प्र॒जाः प्र॒वीय॑न्ते । \newline
15. प्र॒जाः प्र॒वीय॑न्ते प्र॒वीय॑न्ते प्र॒जाः प्र॒जाः प्र॒वीय॑न्ते ऽस्थ॒न्वती॑ रस्थ॒न्वतीः᳚ प्र॒वीय॑न्ते प्र॒जाः प्र॒जाः प्र॒वीय॑न्ते ऽस्थ॒न्वतीः᳚ । \newline
16. प्र॒जा इति॑ प्र - जाः । \newline
17. प्र॒वीय॑न्ते ऽस्थ॒न्वती॑ रस्थ॒न्वतीः᳚ प्र॒वीय॑न्ते प्र॒वीय॑न्ते ऽस्थ॒न्वती᳚र् जायन्ते जायन्ते ऽस्थ॒न्वतीः᳚ प्र॒वीय॑न्ते प्र॒वीय॑न्ते ऽस्थ॒न्वती᳚र् जायन्ते । \newline
18. प्र॒वीय॑न्त॒ इति॑ प्र - वीय॑न्ते । \newline
19. अ॒स्थ॒न्वती᳚र् जायन्ते जायन्ते ऽस्थ॒न्वती॑ रस्थ॒न्वती᳚र् जायन्त॒ इतीति॑ जायन्ते ऽस्थ॒न्वती॑ रस्थ॒न्वती᳚र् जायन्त॒ इति॑ । \newline
20. अ॒स्थ॒न्वती॒रित्य॑स्थन्न् - वतीः᳚ । \newline
21. जा॒य॒न्त॒ इतीति॑ जायन्ते जायन्त॒ इति॒ यद् यदिति॑ जायन्ते जायन्त॒ इति॒ यत् । \newline
22. इति॒ यद् यदितीति॒ यद्धिर॑ण्यꣳ॒॒ हिर॑ण्यं॒ ॅयदितीति॒ यद्धिर॑ण्यम् । \newline
23. यद्धिर॑ण्यꣳ॒॒ हिर॑ण्यं॒ ॅयद् यद्धिर॑ण्यम् घृ॒ते घृ॒ते हिर॑ण्यं॒ ॅयद् यद्धिर॑ण्यम् घृ॒ते । \newline
24. हिर॑ण्यम् घृ॒ते घृ॒ते हिर॑ण्यꣳ॒॒ हिर॑ण्यम् घृ॒ते॑ ऽव॒धाया॑ व॒धाय॑ घृ॒ते हिर॑ण्यꣳ॒॒ हिर॑ण्यम् घृ॒ते॑ ऽव॒धाय॑ । \newline
25. घृ॒ते॑ ऽव॒धाया॑ व॒धाय॑ घृ॒ते घृ॒ते॑ ऽव॒धाय॑ जु॒होति॑ जु॒हो त्य॑व॒धाय॑ घृ॒ते घृ॒ते॑ ऽव॒धाय॑ जु॒होति॑ । \newline
26. अ॒व॒धाय॑ जु॒होति॑ जु॒हो त्य॑व॒धाया॑ व॒धाय॑ जु॒होति॒ तस्मा॒त् तस्मा᳚ज् जु॒हो त्य॑व॒धाया॑ व॒धाय॑ जु॒होति॒ तस्मा᳚त् । \newline
27. अ॒व॒धायेत्य॑व - धाय॑ । \newline
28. जु॒होति॒ तस्मा॒त् तस्मा᳚ज् जु॒होति॑ जु॒होति॒ तस्मा॑ दन॒स्थिके॑ना न॒स्थिके॑न॒ तस्मा᳚ज् जु॒होति॑ जु॒होति॒ तस्मा॑ दन॒स्थिके॑न । \newline
29. तस्मा॑ दन॒स्थिके॑ना न॒स्थिके॑न॒ तस्मा॒त् तस्मा॑ दन॒स्थिके॑न प्र॒जाः प्र॒जा अ॑न॒स्थिके॑न॒ तस्मा॒त् तस्मा॑ दन॒स्थिके॑न प्र॒जाः । \newline
30. अ॒न॒स्थिके॑न प्र॒जाः प्र॒जा अ॑न॒स्थिके॑ना न॒स्थिके॑न प्र॒जाः प्र प्र प्र॒जा अ॑न॒स्थिके॑ना न॒स्थिके॑न प्र॒जाः प्र । \newline
31. प्र॒जाः प्र प्र प्र॒जाः प्र॒जाः प्र वी॑यन्ते वीयन्ते॒ प्र प्र॒जाः प्र॒जाः प्र वी॑यन्ते । \newline
32. प्र॒जा इति॑ प्र - जाः । \newline
33. प्र वी॑यन्ते वीयन्ते॒ प्र प्र वी॑यन्ते ऽस्थ॒न्वती॑ रस्थ॒न्वती᳚र् वीयन्ते॒ प्र प्र वी॑यन्ते ऽस्थ॒न्वतीः᳚ । \newline
34. वी॒य॒न्ते॒ ऽस्थ॒न्वती॑ रस्थ॒न्वती᳚र् वीयन्ते वीयन्ते ऽस्थ॒न्वती᳚र् जायन्ते जायन्ते ऽस्थ॒न्वती᳚र् वीयन्ते वीयन्ते ऽस्थ॒न्वती᳚र् जायन्ते । \newline
35. अ॒स्थ॒न्वती᳚र् जायन्ते जायन्ते ऽस्थ॒न्वती॑ रस्थ॒न्वती᳚र् जायन्त ए॒त दे॒तज् जा॑यन्ते ऽस्थ॒न्वती॑ रस्थ॒न्वती᳚र् जायन्त ए॒तत् । \newline
36. अ॒स्थ॒न्वती॒रित्य॑स्थन्न् - वतीः᳚ । \newline
37. जा॒य॒न्त॒ ए॒त दे॒तज् जा॑यन्ते जायन्त ए॒तद् वै वा ए॒तज् जा॑यन्ते जायन्त ए॒तद् वै । \newline
38. ए॒तद् वै वा ए॒त दे॒तद् वा अ॒ग्ने र॒ग्नेर् वा ए॒त दे॒तद् वा अ॒ग्नेः । \newline
39. वा अ॒ग्ने र॒ग्नेर् वै वा अ॒ग्नेः प्रि॒यम् प्रि॒य म॒ग्नेर् वै वा अ॒ग्नेः प्रि॒यम् । \newline
40. अ॒ग्नेः प्रि॒यम् प्रि॒य म॒ग्ने र॒ग्नेः प्रि॒यम् धाम॒ धाम॑ प्रि॒य म॒ग्ने र॒ग्नेः प्रि॒यम् धाम॑ । \newline
41. प्रि॒यम् धाम॒ धाम॑ प्रि॒यम् प्रि॒यम् धाम॒ यद् यद् धाम॑ प्रि॒यम् प्रि॒यम् धाम॒ यत् । \newline
42. धाम॒ यद् यद् धाम॒ धाम॒ यद् घृ॒तम् घृ॒तं ॅयद् धाम॒ धाम॒ यद् घृ॒तम् । \newline
43. यद् घृ॒तम् घृ॒तं ॅयद् यद् घृ॒तम् तेज॒ स्तेजो॑ घृ॒तं ॅयद् यद् घृ॒तम् तेजः॑ । \newline
44. घृ॒तम् तेज॒ स्तेजो॑ घृ॒तम् घृ॒तम् तेजो॒ हिर॑ण्यꣳ॒॒ हिर॑ण्य॒म् तेजो॑ घृ॒तम् घृ॒तम् तेजो॒ हिर॑ण्यम् । \newline
45. तेजो॒ हिर॑ण्यꣳ॒॒ हिर॑ण्य॒म् तेज॒ स्तेजो॒ हिर॑ण्य मि॒य मि॒यꣳ हिर॑ण्य॒म् तेज॒ स्तेजो॒ हिर॑ण्य मि॒यम् । \newline
46. हिर॑ण्य मि॒य मि॒यꣳ हिर॑ण्यꣳ॒॒ हिर॑ण्य मि॒यम् ते॑ त इ॒यꣳ हिर॑ण्यꣳ॒॒ हिर॑ण्य मि॒यम् ते᳚ । \newline
47. इ॒यम् ते॑ त इ॒य मि॒यम् ते॑ शुक्र शुक्र त इ॒य मि॒यम् ते॑ शुक्र । \newline
48. ते॒ शु॒क्र॒ शु॒क्र॒ ते॒ ते॒ शु॒क्र॒ त॒नू स्त॒नूः शु॑क्र ते ते शुक्र त॒नूः । \newline
49. शु॒क्र॒ त॒नू स्त॒नूः शु॑क्र शुक्र त॒नू रि॒द मि॒दम् त॒नूः शु॑क्र शुक्र त॒नू रि॒दम् । \newline
50. त॒नू रि॒द मि॒दम् त॒नू स्त॒नू रि॒दं ॅवर्चो॒ वर्च॑ इ॒दम् त॒नू स्त॒नू रि॒दं ॅवर्चः॑ । \newline
51. इ॒दं ॅवर्चो॒ वर्च॑ इ॒द मि॒दं ॅवर्च॒ इतीति॒ वर्च॑ इ॒द मि॒दं ॅवर्च॒ इति॑ । \newline
52. वर्च॒ इतीति॒ वर्चो॒ वर्च॒ इत्या॑हा॒हेति॒ वर्चो॒ वर्च॒ इत्या॑ह । \newline
53. इत्या॑हा॒हे तीत्या॑ह॒ सते॑जसꣳ॒॒ सते॑जस मा॒हे तीत्या॑ह॒ सते॑जसम् । \newline
54. आ॒ह॒ सते॑जसꣳ॒॒ सते॑जस माहाह॒ सते॑जस मे॒वैव सते॑जस माहाह॒ सते॑जस मे॒व । \newline
55. सते॑जस मे॒वैव सते॑जसꣳ॒॒ सते॑जस मे॒वैन॑ मेन मे॒व सते॑जसꣳ॒॒ सते॑जस मे॒वैन᳚म् । \newline
56. सते॑जस॒मिति॒ स - ते॒ज॒स॒म् । \newline
57. ए॒वैन॑ मेन मे॒वै वैनꣳ॒॒ सत॑नुꣳ॒॒ सत॑नु मेन मे॒वै वैनꣳ॒॒ सत॑नुम् । \newline
58. ए॒नꣳ॒॒ सत॑नुꣳ॒॒ सत॑नु मेन मेनꣳ॒॒ सत॑नुम् करोति करोति॒ सत॑नु मेन मेनꣳ॒॒ सत॑नुम् करोति । \newline
59. सत॑नुम् करोति करोति॒ सत॑नुꣳ॒॒ सत॑नुम् करो॒ त्यथो॒ अथो॑ करोति॒ सत॑नुꣳ॒॒ सत॑नुम् करो॒ त्यथो᳚ । \newline
60. सत॑नु॒मिति॒ स - त॒नु॒म् । \newline
\pagebreak
\markright{ TS 6.1.7.2  \hfill https://www.vedavms.in \hfill}

\section{ TS 6.1.7.2 }

\textbf{TS 6.1.7.2 } \newline
\textbf{Samhita Paata} \newline

करो॒त्यथो॒ सं भ॑रत्ये॒वैनं॒ ॅयदब॑द्धम-वद॒द्ध्याद्-गर्भाः᳚ प्र॒जानां᳚ परा॒पातु॑काः स्युर्ब॒द्धमव॑ दधाति॒ गर्भा॑णां॒ धृत्यै॑ निष्ट॒र्क्यं॑ बद्ध्नाति प्र॒जानां᳚ प्र॒जन॑नाय॒ वाग्वा ए॒षा यथ् सो॑म॒क्रय॑णी॒ जूर॒सीत्या॑ह॒ यद्धि मन॑सा॒ जव॑ते॒ तद्-वा॒चा वद॑ति धृ॒ता मन॒सेत्या॑ह॒ मन॑सा॒ हि वाग्धृ॒ता जुष्टा॒ विष्ण॑व॒ इत्या॑ह - [  ] \newline

\textbf{Pada Paata} \newline

क॒रो॒ति॒ । अथो॒ इति॑ । समिति॑ । भ॒र॒ति॒ । ए॒व । ए॒न॒म् । यत् । अब॑द्धम् । अ॒व॒द॒द्ध्यादित्य॑व -द॒द्ध्यात् । गर्भाः᳚ । प्र॒जाना॒मिति॑ प्र - जाना᳚म् । प॒रा॒पातु॑का॒ इति॑ परा-पातु॑काः । स्युः॒ । ब॒द्धम् । अवेति॑ । द॒धा॒ति॒ । गर्भा॑णाम् । धृत्यै᳚ । नि॒ष्ट॒र्क्य᳚म् । ब॒द्ध्ना॒ति॒ । प्र॒जाना॒मिति॑ प्र - जाना᳚म् । प्र॒जन॑ना॒येति॑ प्र-जन॑नाय । वाक् । वै । ए॒षा । यत् । सो॒म॒क्रय॒णीति॑ सोम - क्रय॑णी । जूः । अ॒सि॒ । इति॑ । आ॒ह॒ । यत् । हि । मन॑सा । जव॑ते । तत् । वा॒चा । वद॑ति । धृ॒ता । मन॑सा । इति॑ । आ॒ह॒ । मन॑सा । हि । वाक् । धृ॒ता । जुष्टा᳚ । विष्ण॑वे । इति॑ । आ॒ह॒ ।  \newline


\textbf{Krama Paata} \newline

क॒रो॒त्यथो᳚ । अथो॒ सम् । अथो॒ इत्यथो᳚ । सम्भ॑रति । भ॒र॒त्ये॒व । ए॒वैन᳚म् । ए॒न॒म् ॅयत् । यदब॑द्धम् । अब॑द्धमवद॒द्ध्यात् । अ॒व॒द॒द्ध्याद् गर्भाः᳚ । अ॒व॒द॒द्ध्यादित्य॑व - द॒द्ध्यात् । गर्भाः᳚ प्र॒जाना᳚म् । प्र॒जाना᳚म् परा॒पातु॑काः । प्र॒जाना॒मिति॑ प्र - जाना᳚म् । प॒रा॒पातु॑काः स्युः । प॒रा॒पातु॑का॒ इति॑ परा - पातु॑काः । स्यु॒र् ब॒द्धम् । ब॒द्धमव॑ । अव॑ दधाति । द॒धा॒ति॒ गर्भा॑णाम् । गर्भा॑णा॒म् धृत्यै᳚ । धृत्यै॑ निष्ट॒र्क्य᳚म् । नि॒ष्ट॒र्क्य॑म् बध्नाति । ब॒ध्ना॒ति॒ प्र॒जाना᳚म् । प्र॒जाना᳚म् प्र॒जन॑नाय । प्र॒जाना॒मिति॑ प्र - जाना᳚म् । प्र॒जन॑नाय॒ वाक् । प्र॒जन॑ना॒येति॑ प्र - जन॑नाय । वाग् वै । वा ए॒षा । ए॒षा यत् । यथ् सो॑म॒क्रय॑णी । सो॒म॒क्रय॑णी॒ जूः । सो॒म॒क्रय॒णीति॑ सोम - क्रय॑णी । जूर॑सि । अ॒सीति॑ । इत्या॑ह । आ॒ह॒ यत् । यद्‌धि । हि मन॑सा । मन॑सा॒ जव॑ते । जव॑ते॒ तत् । तद् वा॒चा । वा॒चा वद॑ति । वद॑ति धृ॒ता । धृ॒ता मन॑सा । मन॒सेति॑ । इत्या॑ह । आ॒ह॒ मन॑सा । मन॑सा॒ हि । हि वाक् । वाग् धृ॒ता । धृ॒ता जुष्टा᳚ । जुष्टा॒ विष्ण॑वे । विष्ण॑व॒ इति॑ । इत्या॑ह । आ॒ह॒ य॒ज्ञ्ः \newline

\textbf{Jatai Paata} \newline

1. क॒रो॒ त्यथो॒ अथो॑ करोति करो॒ त्यथो᳚ । \newline
2. अथो॒ सꣳ स मथो॒ अथो॒ सम् । \newline
3. अथो॒ इत्यथो᳚ । \newline
4. सम् भ॑रति भरति॒ सꣳ सम् भ॑रति । \newline
5. भ॒र॒ त्ये॒वैव भ॑रति भर त्ये॒व । \newline
6. ए॒वैन॑ मेन मे॒वै वैन᳚म् । \newline
7. ए॒नं॒ ॅयद् यदे॑न मेनं॒ ॅयत् । \newline
8. यदब॑द्ध॒ मब॑द्धं॒ ॅयद् यदब॑द्धम् । \newline
9. अब॑द्ध मवद॒द्ध्या द॑वद॒द्ध्या दब॑द्ध॒ मब॑द्ध मवद॒द्ध्यात् । \newline
10. अ॒व॒द॒द्ध्याद् गर्भा॒ गर्भा॑ अवद॒द्ध्या द॑वद॒द्ध्याद् गर्भाः᳚ । \newline
11. अ॒व॒द॒द्ध्यादित्य॑व - द॒द्ध्यात् । \newline
12. गर्भाः᳚ प्र॒जाना᳚म् प्र॒जाना॒म् गर्भा॒ गर्भाः᳚ प्र॒जाना᳚म् । \newline
13. प्र॒जाना᳚म् परा॒पातु॑काः परा॒पातु॑काः प्र॒जाना᳚म् प्र॒जाना᳚म् परा॒पातु॑काः । \newline
14. प्र॒जाना॒मिति॑ प्र - जाना᳚म् । \newline
15. प॒रा॒पातु॑काः स्युः स्युः परा॒पातु॑काः परा॒पातु॑काः स्युः । \newline
16. प॒रा॒पातु॑का॒ इति॑ परा - पातु॑काः । \newline
17. स्यु॒र् ब॒द्धम् ब॒द्धꣳ स्युः॑ स्युर् ब॒द्धम् । \newline
18. ब॒द्ध मवाव॑ ब॒द्धम् ब॒द्ध मव॑ । \newline
19. अव॑ दधाति दधा॒ त्यवाव॑ दधाति । \newline
20. द॒धा॒ति॒ गर्भा॑णा॒म् गर्भा॑णाम् दधाति दधाति॒ गर्भा॑णाम् । \newline
21. गर्भा॑णा॒म् धृत्यै॒ धृत्यै॒ गर्भा॑णा॒म् गर्भा॑णा॒म् धृत्यै᳚ । \newline
22. धृत्यै॑ निष्ट॒र्क्य॑न् निष्ट॒र्क्य॑म् धृत्यै॒ धृत्यै॑ निष्ट॒र्क्य᳚म् । \newline
23. नि॒ष्ट॒र्क्य॑म् बद्ध्नाति बद्ध्नाति निष्ट॒र्क्य॑म् निष्ट॒र्क्य॑म् बद्ध्नाति । \newline
24. ब॒द्ध्ना॒ति॒ प्र॒जाना᳚म् प्र॒जाना᳚म् बद्ध्नाति बद्ध्नाति प्र॒जाना᳚म् । \newline
25. प्र॒जाना᳚म् प्र॒जन॑नाय प्र॒जन॑नाय प्र॒जाना᳚म् प्र॒जाना᳚म् प्र॒जन॑नाय । \newline
26. प्र॒जाना॒मिति॑ प्र - जाना᳚म् । \newline
27. प्र॒जन॑नाय॒ वाग् वाक् प्र॒जन॑नाय प्र॒जन॑नाय॒ वाक् । \newline
28. प्र॒जन॑ना॒येति॑ प्र - जन॑नाय । \newline
29. वाग् वै वै वाग् वाग् वै । \newline
30. वा ए॒षैषा वै वा ए॒षा । \newline
31. ए॒षा यद् यदे॒षैषा यत् । \newline
32. यथ् सो॑म॒क्रय॑णी सोम॒क्रय॑णी॒ यद् यथ् सो॑म॒क्रय॑णी । \newline
33. सो॒म॒क्रय॑णी॒ जूर् जूः सो॑म॒क्रय॑णी सोम॒क्रय॑णी॒ जूः । \newline
34. सो॒म॒क्रय॒णीति॑ सोम - क्रय॑णी । \newline
35. जूर॑ स्यसि॒ जूर् जूर॑सि । \newline
36. अ॒सीती त्य॑स्य॒ सीति॑ । \newline
37. इत्या॑हा॒हे तीत्या॑ह । \newline
38. आ॒ह॒ यद् यदा॑ हाह॒ यत् । \newline
39. यद्धि हि यद् यद्धि । \newline
40. हि मन॑सा॒ मन॑सा॒ हि हि मन॑सा । \newline
41. मन॑सा॒ जव॑ते॒ जव॑ते॒ मन॑सा॒ मन॑सा॒ जव॑ते । \newline
42. जव॑ते॒ तत् तज् जव॑ते॒ जव॑ते॒ तत् । \newline
43. तद् वा॒चा वा॒चा तत् तद् वा॒चा । \newline
44. वा॒चा वद॑ति॒ वद॑ति वा॒चा वा॒चा वद॑ति । \newline
45. वद॑ति धृ॒ता धृ॒ता वद॑ति॒ वद॑ति धृ॒ता । \newline
46. धृ॒ता मन॑सा॒ मन॑सा धृ॒ता धृ॒ता मन॑सा । \newline
47. मन॒से तीति॒ मन॑सा॒ मन॒ सेति॑ । \newline
48. इत्या॑हा॒हे तीत्या॑ह । \newline
49. आ॒ह॒ मन॑सा॒ मन॑सा ऽऽहाह॒ मन॑सा । \newline
50. मन॑सा॒ हि हि मन॑सा॒ मन॑सा॒ हि । \newline
51. हि वाग् वाग् घि हि वाक् । \newline
52. वाग् धृ॒ता धृ॒ता वाग् वाग् धृ॒ता । \newline
53. धृ॒ता जुष्टा॒ जुष्टा॑ धृ॒ता धृ॒ता जुष्टा᳚ । \newline
54. जुष्टा॒ विष्ण॑वे॒ विष्ण॑वे॒ जुष्टा॒ जुष्टा॒ विष्ण॑वे । \newline
55. विष्ण॑व॒ इतीति॒ विष्ण॑वे॒ विष्ण॑व॒ इति॑ । \newline
56. इत्या॑हा॒हे तीत्या॑ह । \newline
57. आ॒ह॒ य॒ज्ञो य॒ज्ञ् आ॑हाह य॒ज्ञ्ः । \newline

\textbf{Ghana Paata } \newline

1. क॒रो॒ त्यथो॒ अथो॑ करोति करो॒ त्यथो॒ सꣳ स मथो॑ करोति करो॒ त्यथो॒ सम् । \newline
2. अथो॒ सꣳ स मथो॒ अथो॒ सम् भ॑रति भरति॒ स मथो॒ अथो॒ सम् भ॑रति । \newline
3. अथो॒ इत्यथो᳚ । \newline
4. सम् भ॑रति भरति॒ सꣳ सम् भ॑र त्ये॒वैव भ॑रति॒ सꣳ सम् भ॑र त्ये॒व । \newline
5. भ॒र॒ त्ये॒वैव भ॑रति भर त्ये॒वैन॑ मेन मे॒व भ॑रति भर त्ये॒वैन᳚म् । \newline
6. ए॒वैन॑ मेन मे॒वै वैनं॒ ॅयद् यदे॑न मे॒वै वैनं॒ ॅयत् । \newline
7. ए॒नं॒ ॅयद् यदे॑न मेनं॒ ॅयदब॑द्ध॒ मब॑द्धं॒ ॅयदे॑न मेनं॒ ॅयदब॑द्धम् । \newline
8. यदब॑द्ध॒ मब॑द्धं॒ ॅयद् यदब॑द्ध मवद॒द्ध्या द॑वद॒द्ध्या दब॑द्धं॒ ॅयद् यदब॑द्ध मवद॒द्ध्यात् । \newline
9. अब॑द्ध मवद॒द्ध्या द॑वद॒द्ध्या दब॑द्ध॒ मब॑द्ध मवद॒द्ध्याद् गर्भा॒ गर्भा॑ अवद॒द्ध्या दब॑द्ध॒ मब॑द्ध मवद॒द्ध्याद् गर्भाः᳚ । \newline
10. अ॒व॒द॒द्ध्याद् गर्भा॒ गर्भा॑ अवद॒द्ध्या द॑वद॒द्ध्याद् गर्भाः᳚ प्र॒जाना᳚म् प्र॒जाना॒म् गर्भा॑ अवद॒द्ध्या द॑वद॒द्ध्याद् गर्भाः᳚ प्र॒जाना᳚म् । \newline
11. अ॒व॒द॒द्ध्यादित्य॑व - द॒द्ध्यात् । \newline
12. गर्भाः᳚ प्र॒जाना᳚म् प्र॒जाना॒म् गर्भा॒ गर्भाः᳚ प्र॒जाना᳚म् परा॒पातु॑काः परा॒पातु॑काः प्र॒जाना॒म् गर्भा॒ गर्भाः᳚ प्र॒जाना᳚म् परा॒पातु॑काः । \newline
13. प्र॒जाना᳚म् परा॒पातु॑काः परा॒पातु॑काः प्र॒जाना᳚म् प्र॒जाना᳚म् परा॒पातु॑काः स्युः स्युः परा॒पातु॑काः प्र॒जाना᳚म् प्र॒जाना᳚म् परा॒पातु॑काः स्युः । \newline
14. प्र॒जाना॒मिति॑ प्र - जाना᳚म् । \newline
15. प॒रा॒पातु॑काः स्युः स्युः परा॒पातु॑काः परा॒पातु॑काः स्युर् ब॒द्धम् ब॒द्धꣳ स्युः॑ परा॒पातु॑काः परा॒पातु॑काः स्युर् ब॒द्धम् । \newline
16. प॒रा॒पातु॑का॒ इति॑ परा - पातु॑काः । \newline
17. स्यु॒र् ब॒द्धम् ब॒द्धꣳ स्युः॑ स्युर् ब॒द्ध मवाव॑ ब॒द्धꣳ स्युः॑ स्युर् ब॒द्ध मव॑ । \newline
18. ब॒द्ध मवाव॑ ब॒द्धम् ब॒द्ध मव॑ दधाति दधा॒ त्यव॑ ब॒द्धम् ब॒द्ध मव॑ दधाति । \newline
19. अव॑ दधाति दधा॒ त्यवाव॑ दधाति॒ गर्भा॑णा॒म् गर्भा॑णाम् दधा॒ त्यवाव॑ दधाति॒ गर्भा॑णाम् । \newline
20. द॒धा॒ति॒ गर्भा॑णा॒म् गर्भा॑णाम् दधाति दधाति॒ गर्भा॑णा॒म् धृत्यै॒ धृत्यै॒ गर्भा॑णाम् दधाति दधाति॒ गर्भा॑णा॒म् धृत्यै᳚ । \newline
21. गर्भा॑णा॒म् धृत्यै॒ धृत्यै॒ गर्भा॑णा॒म् गर्भा॑णा॒म् धृत्यै॑ निष्ट॒र्क्य॑न् निष्ट॒र्क्य॑म् धृत्यै॒ गर्भा॑णा॒म् गर्भा॑णा॒म् धृत्यै॑ निष्ट॒र्क्य᳚म् । \newline
22. धृत्यै॑ निष्ट॒र्क्य॑म् निष्ट॒र्क्य॑म् धृत्यै॒ धृत्यै॑ निष्ट॒र्क्य॑म् बद्ध्नाति बद्ध्नाति निष्ट॒र्क्य॑म् धृत्यै॒ धृत्यै॑ निष्ट॒र्क्य॑म् बद्ध्नाति । \newline
23. नि॒ष्ट॒र्क्य॑म् बद्ध्नाति बद्ध्नाति निष्ट॒र्क्य॑न् निष्ट॒र्क्य॑म् बद्ध्नाति प्र॒जाना᳚म् प्र॒जाना᳚म् बद्ध्नाति निष्ट॒र्क्य॑म् निष्ट॒र्क्य॑म् बद्ध्नाति प्र॒जाना᳚म् । \newline
24. ब॒द्ध्ना॒ति॒ प्र॒जाना᳚म् प्र॒जाना᳚म् बद्ध्नाति बद्ध्नाति प्र॒जाना᳚म् प्र॒जन॑नाय प्र॒जन॑नाय प्र॒जाना᳚म् बद्ध्नाति बद्ध्नाति प्र॒जाना᳚म् प्र॒जन॑नाय । \newline
25. प्र॒जाना᳚म् प्र॒जन॑नाय प्र॒जन॑नाय प्र॒जाना᳚म् प्र॒जाना᳚म् प्र॒जन॑नाय॒ वाग् वाक् प्र॒जन॑नाय प्र॒जाना᳚म् प्र॒जाना᳚म् प्र॒जन॑नाय॒ वाक् । \newline
26. प्र॒जाना॒मिति॑ प्र - जाना᳚म् । \newline
27. प्र॒जन॑नाय॒ वाग् वाक् प्र॒जन॑नाय प्र॒जन॑नाय॒ वाग् वै वै वाक् प्र॒जन॑नाय प्र॒जन॑नाय॒ वाग् वै । \newline
28. प्र॒जन॑ना॒येति॑ प्र - जन॑नाय । \newline
29. वाग् वै वै वाग् वाग् वा ए॒षैषा वै वाग् वाग् वा ए॒षा । \newline
30. वा ए॒षैषा वै वा ए॒षा यद् यदे॒षा वै वा ए॒षा यत् । \newline
31. ए॒षा यद् यदे॒ षैषा यथ् सो॑म॒क्रय॑णी सोम॒क्रय॑णी॒ यदे॒ षैषा यथ् सो॑म॒क्रय॑णी । \newline
32. यथ् सो॑म॒क्रय॑णी सोम॒क्रय॑णी॒ यद् यथ् सो॑म॒क्रय॑णी॒ जूर् जूः सो॑म॒क्रय॑णी॒ यद् यथ् सो॑म॒क्रय॑णी॒ जूः । \newline
33. सो॒म॒क्रय॑णी॒ जूर् जूः सो॑म॒क्रय॑णी सोम॒क्रय॑णी॒ जूर॑स्यसि॒ जूः सो॑म॒क्रय॑णी सोम॒क्रय॑णी॒ जूर॑सि । \newline
34. सो॒म॒क्रय॒णीति॑ सोम - क्रय॑णी । \newline
35. जूर॑ स्यसि॒ जूर् जूर॒सी तीत्य॑सि॒ जूर् जूर॒ सीति॑ । \newline
36. अ॒सी तीत्य॑ स्य॒सीत्या॑ हा॒हे त्य॑स्य॒सी त्या॑ह । \newline
37. इत्या॑हा॒हे तीत्या॑ह॒ यद् यदा॒हे तीत्या॑ह॒ यत् । \newline
38. आ॒ह॒ यद् यदा॑हाह॒ यद्धि हि यदा॑हाह॒ यद्धि । \newline
39. यद्धि हि यद् यद्धि मन॑सा॒ मन॑सा॒ हि यद् यद्धि मन॑सा । \newline
40. हि मन॑सा॒ मन॑सा॒ हि हि मन॑सा॒ जव॑ते॒ जव॑ते॒ मन॑सा॒ हि हि मन॑सा॒ जव॑ते । \newline
41. मन॑सा॒ जव॑ते॒ जव॑ते॒ मन॑सा॒ मन॑सा॒ जव॑ते॒ तत् तज् जव॑ते॒ मन॑सा॒ मन॑सा॒ जव॑ते॒ तत् । \newline
42. जव॑ते॒ तत् तज् जव॑ते॒ जव॑ते॒ तद् वा॒चा वा॒चा तज् जव॑ते॒ जव॑ते॒ तद् वा॒चा । \newline
43. तद् वा॒चा वा॒चा तत् तद् वा॒चा वद॑ति॒ वद॑ति वा॒चा तत् तद् वा॒चा वद॑ति । \newline
44. वा॒चा वद॑ति॒ वद॑ति वा॒चा वा॒चा वद॑ति धृ॒ता धृ॒ता वद॑ति वा॒चा वा॒चा वद॑ति धृ॒ता । \newline
45. वद॑ति धृ॒ता धृ॒ता वद॑ति॒ वद॑ति धृ॒ता मन॑सा॒ मन॑सा धृ॒ता वद॑ति॒ वद॑ति धृ॒ता मन॑सा । \newline
46. धृ॒ता मन॑सा॒ मन॑सा धृ॒ता धृ॒ता मन॒से तीति॒ मन॑सा धृ॒ता धृ॒ता मन॒सेति॑ । \newline
47. मन॒से तीति॒ मन॑सा॒ मन॒से त्या॑हा॒हेति॒ मन॑सा॒ मन॒से त्या॑ह । \newline
48. इत्या॑हा॒हे तीत्या॑ह॒ मन॑सा॒ मन॑सा॒ ऽऽहे तीत्या॑ह॒ मन॑सा । \newline
49. आ॒ह॒ मन॑सा॒ मन॑सा ऽऽहाह॒ मन॑सा॒ हि हि मन॑सा ऽऽहाह॒ मन॑सा॒ हि । \newline
50. मन॑सा॒ हि हि मन॑सा॒ मन॑सा॒ हि वाग् वाग्घि मन॑सा॒ मन॑सा॒ हि वाक् । \newline
51. हि वाग् वाग्घि हि वाग् धृ॒ता धृ॒ता वाग्घि हि वाग् धृ॒ता । \newline
52. वाग् धृ॒ता धृ॒ता वाग् वाग् धृ॒ता जुष्टा॒ जुष्टा॑ धृ॒ता वाग् वाग् धृ॒ता जुष्टा᳚ । \newline
53. धृ॒ता जुष्टा॒ जुष्टा॑ धृ॒ता धृ॒ता जुष्टा॒ विष्ण॑वे॒ विष्ण॑वे॒ जुष्टा॑ धृ॒ता धृ॒ता जुष्टा॒ विष्ण॑वे । \newline
54. जुष्टा॒ विष्ण॑वे॒ विष्ण॑वे॒ जुष्टा॒ जुष्टा॒ विष्ण॑व॒ इतीति॒ विष्ण॑वे॒ जुष्टा॒ जुष्टा॒ विष्ण॑व॒ इति॑ । \newline
55. विष्ण॑व॒ इतीति॒ विष्ण॑वे॒ विष्ण॑व॒ इत्या॑हा॒हेति॒ विष्ण॑वे॒ विष्ण॑व॒ इत्या॑ह । \newline
56. इत्या॑हा॒हे तीत्या॑ह य॒ज्ञो य॒ज्ञ् आ॒हे तीत्या॑ह य॒ज्ञ्ः । \newline
57. आ॒ह॒ य॒ज्ञो य॒ज्ञ् आ॑हाह य॒ज्ञो वै वै य॒ज्ञ् आ॑हाह य॒ज्ञो वै । \newline
\pagebreak
\markright{ TS 6.1.7.3  \hfill https://www.vedavms.in \hfill}

\section{ TS 6.1.7.3 }

\textbf{TS 6.1.7.3 } \newline
\textbf{Samhita Paata} \newline

य॒ज्ञो वै विष्णु॑ र्य॒ज्ञायै॒वैनां॒ जुष्टां᳚ करोति॒ तस्या᳚स्ते स॒त्यस॑वसः प्रस॒व इत्या॑ह सवि॒तृ-प्र॑सूतामे॒व वाच॒मव॑ रुन्धे॒ काण्डे॑काण्डे॒ वै क्रि॒यमा॑णे य॒ज्ञ्ꣳ रक्षाꣳ॑सि जिघाꣳसन्त्ये॒ष खलु॒ वा अर॑क्षोहतः॒ पन्था॒ यो᳚ऽग्नेश्च॒ सूर्य॑स्य च॒ सूर्य॑स्य॒ चक्षु॒रा-ऽरु॑हम॒ग्नेर॒क्ष्णः क॒नीनि॑का॒मित्या॑ह॒ य ए॒वार॑क्षोहतः॒ पन्था॒स्तꣳ स॒मारो॑हति॒ - [  ] \newline

\textbf{Pada Paata} \newline

य॒ज्ञ्ः । वै । विष्णुः॑ । य॒ज्ञाय॑ । ए॒व । ए॒ना॒म् । जुष्टा᳚म् । क॒रो॒ति॒ । तस्याः᳚ । ते॒ । स॒त्यस॑वस॒ इति॑ स॒त्य - स॒व॒सः॒ । प्र॒स॒व इति॑ प्र - स॒वे । इति॑ । आ॒ह॒ । स॒वि॒तृप्र॑सूता॒मिति॑ सवि॒तृ - प्र॒सू॒ता॒म् । ए॒व । वाच᳚म् । अवेति॑ । रु॒न्धे॒ । काण्डे॑काण्ड॒ इति॒ काण्डे᳚-का॒ण्डे॒ । वै । क्रि॒यमा॑णे । य॒ज्ञ्म् । रक्षाꣳ॑सि । जि॒घाꣳ॒॒स॒न्ति॒ । ए॒षः । खलु॑ । वै । अर॑क्षोहत॒ इत्यर॑क्षः - ह॒तः॒ । पन्थाः᳚ । यः । अ॒ग्नेः । च॒ । सूर्य॑स्य । च॒ । सूर्य॑स्य । चक्षुः॑ । एति॑ । अ॒रु॒ह॒म् । अ॒ग्नेः । अ॒क्ष्णः । क॒नीनि॑काम् । इति॑ । आ॒ह॒ । यः । ए॒व । अर॑क्षोहत॒ इत्यर॑क्षः-ह॒तः॒ । पन्थाः᳚ । तम् । स॒मारो॑ह॒तीति॑ सं - आरो॑हति ।  \newline


\textbf{Krama Paata} \newline

य॒ज्ञो वै । वै विष्णुः॑ । विष्णु॑र् य॒ज्ञाय॑ । य॒ज्ञायै॒व । ए॒वैना᳚म् । ए॒ना॒म् जुष्टा᳚म् । जुष्टा᳚म् करोति । क॒रो॒ति॒ तस्याः᳚ । तस्या᳚स्ते । ते॒ स॒त्यस॑वसः । स॒त्यस॑वसः प्रस॒वे । स॒त्यसव॑स॒ इति॑ स॒त्य - स॒व॒सः॒ । प्र॒स॒व इति॑ । प्र॒स॒व इति॑ प्र - स॒वे । इत्या॑ह । आ॒ह॒ स॒वि॒तृप्र॑सूताम् । स॒वि॒तृप्र॑सूतामे॒व । स॒वि॒तृप्र॑सूता॒मिति॑ सवि॒तृ - प्र॒सू॒ता॒म् । ए॒व वाच᳚म् । वाच॒मव॑ । अव॑ रुन्धे । रु॒न्धे॒ काण्डे॑काण्डे । काण्डे॑काण्डे॒ वै । काण्डे॑काण्ड॒ इति॒ काण्डे᳚ - का॒ण्डे॒ । वै क्रि॒यमा॑णे । क्रि॒यमा॑णे य॒ज्ञ्म् । य॒ज्ञ्ꣳ रक्षाꣳ॑सि । रक्षाꣳ॑सि जिघाꣳसन्ति । जि॒घाꣳ॒॒स॒न्त्ये॒षः । ए॒ष खलु॑ । खलु॒ वै । वा अर॑क्षोहतः । अर॑क्षोहतः॒ पन्थाः᳚ । अर॑क्षोहत॒ इत्यर॑क्षः - ह॒तः॒ । पन्था॒ यः । यो᳚ऽग्नेः । अ॒ग्नेश्च॑ । च॒ सूर्य॑स्य । सूर्य॑स्य च । च॒ सूर्य॑स्य । सूर्य॑स्य॒ चक्षुः॑ । चक्षु॒रा । आऽरु॑हम् । अ॒रु॒ह॒म॒ग्नेः । अ॒ग्नेर॒क्ष्णः । अ॒क्ष्णः क॒नीनि॑काम् । क॒नीनि॑का॒मिति॑ । इत्या॑ह । आ॒ह॒ यः । य ए॒व । ए॒वार॑क्षोहतः । अर॑क्षोहतः॒ पन्थाः᳚ । अर॑क्षोहत॒ इत्य॑रक्षः - ह॒तः॒ । पन्था॒स्तम् । तꣳ स॒मारो॑हति । स॒मारो॑हति॒ वाक् । स॒मारो॑ह॒तीति॑ सम् - आरो॑हति \newline

\textbf{Jatai Paata} \newline

1. य॒ज्ञो वै वै य॒ज्ञो य॒ज्ञो वै । \newline
2. वै विष्णु॒र् विष्णु॒र् वै वै विष्णुः॑ । \newline
3. विष्णु॑र् य॒ज्ञाय॑ य॒ज्ञाय॒ विष्णु॒र् विष्णु॑र् य॒ज्ञाय॑ । \newline
4. य॒ज्ञायै॒ वैव य॒ज्ञाय॑ य॒ज्ञायै॒व । \newline
5. ए॒वैना॑ मेना मे॒वै वैना᳚म् । \newline
6. ए॒ना॒म् जुष्टा॒म् जुष्टा॑ मेना मेना॒म् जुष्टा᳚म् । \newline
7. जुष्टा᳚म् करोति करोति॒ जुष्टा॒म् जुष्टा᳚म् करोति । \newline
8. क॒रो॒ति॒ तस्या॒ स्तस्याः᳚ करोति करोति॒ तस्याः᳚ । \newline
9. तस्या᳚ स्ते ते॒ तस्या॒ स्तस्या᳚ स्ते । \newline
10. ते॒ स॒त्यस॑वसः स॒त्यस॑वस स्ते ते स॒त्यस॑वसः । \newline
11. स॒त्यस॑वसः प्रस॒वे प्र॑स॒वे स॒त्यस॑वसः स॒त्यस॑वसः प्रस॒वे । \newline
12. स॒त्यस॑वस॒ इति॑ स॒त्य - स॒व॒सः॒ । \newline
13. प्र॒स॒व इतीति॑ प्रस॒वे प्र॑स॒व इति॑ । \newline
14. प्र॒स॒व इति॑ प्र - स॒वे । \newline
15. इत्या॑हा॒हे तीत्या॑ह । \newline
16. आ॒ह॒ स॒वि॒तृप्र॑सूताꣳ सवि॒तृप्र॑सूता माहाह सवि॒तृप्र॑सूताम् । \newline
17. स॒वि॒तृप्र॑सूता मे॒वैव स॑वि॒तृप्र॑सूताꣳ सवि॒तृप्र॑सूता मे॒व । \newline
18. स॒वि॒तृप्र॑सूता॒मिति॑ सवि॒तृ - प्र॒सू॒ता॒म् । \newline
19. ए॒व वाचं॒ ॅवाच॑ मे॒वैव वाच᳚म् । \newline
20. वाच॒ मवाव॒ वाचं॒ ॅवाच॒ मव॑ । \newline
21. अव॑ रुन्धे रु॒न्धे ऽवाव॑ रुन्धे । \newline
22. रु॒न्धे॒ काण्डे॑काण्डे॒ काण्डे॑काण्डे रुन्धे रुन्धे॒ काण्डे॑काण्डे । \newline
23. काण्डे॑काण्डे॒ वै वै काण्डे॑काण्डे॒ काण्डे॑काण्डे॒ वै । \newline
24. काण्डे॑काण्ड॒ इति॒ काण्डे᳚ - का॒ण्डे॒ । \newline
25. वै क्रि॒यमा॑णे क्रि॒यमा॑णे॒ वै वै क्रि॒यमा॑णे । \newline
26. क्रि॒यमा॑णे य॒ज्ञ्ं ॅय॒ज्ञ्म् क्रि॒यमा॑णे क्रि॒यमा॑णे य॒ज्ञ्म् । \newline
27. य॒ज्ञ्ꣳ रक्षाꣳ॑सि॒ रक्षाꣳ॑सि य॒ज्ञ्ं ॅय॒ज्ञ्ꣳ रक्षाꣳ॑सि । \newline
28. रक्षाꣳ॑सि जिघाꣳसन्ति जिघाꣳसन्ति॒ रक्षाꣳ॑सि॒ रक्षाꣳ॑सि जिघाꣳसन्ति । \newline
29. जि॒घाꣳ॒॒स॒न् त्ये॒ष ए॒ष जि॑घाꣳसन्ति जिघाꣳसन् त्ये॒षः । \newline
30. ए॒ष खलु॒ खल्वे॒ष ए॒ष खलु॑ । \newline
31. खलु॒ वै वै खलु॒ खलु॒ वै । \newline
32. वा अर॑क्षोह॒तो ऽर॑क्षोहतो॒ वै वा अर॑क्षोहतः । \newline
33. अर॑क्षोहतः॒ पन्थाः॒ पन्था॒ अर॑क्षोह॒तो ऽर॑क्षोहतः॒ पन्थाः᳚ । \newline
34. अर॑क्षोहत॒ इत्यर॑क्षः - ह॒तः॒ । \newline
35. पन्था॒ यो यः पन्थाः॒ पन्था॒ यः । \newline
36. यो᳚ ऽग्ने र॒ग्नेर् यो यो᳚ ऽग्नेः । \newline
37. अ॒ग्ने श्च॑ चा॒ग्ने र॒ग्ने श्च॑ । \newline
38. च॒ सूर्य॑स्य॒ सूर्य॑स्य च च॒ सूर्य॑स्य । \newline
39. सूर्य॑स्य च च॒ सूर्य॑स्य॒ सूर्य॑स्य च । \newline
40. च॒ सूर्य॑स्य॒ सूर्य॑स्य च च॒ सूर्य॑स्य । \newline
41. सूर्य॑स्य॒ चक्षु॒ श्चक्षुः॒ सूर्य॑स्य॒ सूर्य॑स्य॒ चक्षुः॑ । \newline
42. चक्षु॒रा चक्षु॒ श्चक्षु॒रा । \newline
43. आ ऽरु॑ह मरुह॒ मा ऽरु॑हम् । \newline
44. अ॒रु॒ह॒ म॒ग्ने र॒ग्ने र॑रुह मरुह म॒ग्नेः । \newline
45. अ॒ग्ने र॒क्ष्णो᳚(1॒) ऽक्ष्णो᳚ ऽग्ने र॒ग्ने र॒क्ष्णः । \newline
46. अ॒क्ष्णः क॒नीनि॑काम् क॒नीनि॑का म॒क्ष्णो᳚ ऽक्ष्णः क॒नीनि॑काम् । \newline
47. क॒नीनि॑का॒ मितीति॑ क॒नीनि॑काम् क॒नीनि॑का॒ मिति॑ । \newline
48. इत्या॑हा॒हे तीत्या॑ह । \newline
49. आ॒ह॒ यो य आ॑हाह॒ यः । \newline
50. य ए॒वैव यो य ए॒व । \newline
51. ए॒वा र॑क्षोह॒तो ऽर॑क्षोहत ए॒वैवा र॑क्षोहतः । \newline
52. अर॑क्षोहतः॒ पन्थाः॒ पन्था॒ अर॑क्षोह॒तो ऽर॑क्षोहतः॒ पन्थाः᳚ । \newline
53. अर॑क्षोहत॒ इत्यर॑क्षः - ह॒तः॒ । \newline
54. पन्था॒ स्तम् तम् पन्थाः॒ पन्था॒ स्तम् । \newline
55. तꣳ स॒मारो॑हति स॒मारो॑हति॒ तम् तꣳ स॒मारो॑हति । \newline
56. स॒मारो॑हति॒ वाग् वाख् स॒मारो॑हति स॒मारो॑हति॒ वाक् । \newline
57. स॒मारो॑ह॒तीति॑ सं - आरो॑हति । \newline

\textbf{Ghana Paata } \newline

1. य॒ज्ञो वै वै य॒ज्ञो य॒ज्ञो वै विष्णु॒र् विष्णु॒र् वै य॒ज्ञो य॒ज्ञो वै विष्णुः॑ । \newline
2. वै विष्णु॒र् विष्णु॒र् वै वै विष्णु॑र् य॒ज्ञाय॑ य॒ज्ञाय॒ विष्णु॒र् वै वै विष्णु॑र् य॒ज्ञाय॑ । \newline
3. विष्णु॑र् य॒ज्ञाय॑ य॒ज्ञाय॒ विष्णु॒र् विष्णु॑र् य॒ज्ञायै॒ वैव य॒ज्ञाय॒ विष्णु॒र् विष्णु॑र् य॒ज्ञायै॒व । \newline
4. य॒ज्ञायै॒ वैव य॒ज्ञाय॑ य॒ज्ञायै॒ वैना॑ मेना मे॒व य॒ज्ञाय॑ य॒ज्ञायै॒ वैना᳚म् । \newline
5. ए॒वैना॑ मेना मे॒वै वैना॒म् जुष्टा॒म् जुष्टा॑ मेना मे॒वै वैना॒म् जुष्टा᳚म् । \newline
6. ए॒ना॒म् जुष्टा॒म् जुष्टा॑ मेना मेना॒म् जुष्टा᳚म् करोति करोति॒ जुष्टा॑ मेना मेना॒म् जुष्टा᳚म् करोति । \newline
7. जुष्टा᳚म् करोति करोति॒ जुष्टा॒म् जुष्टा᳚म् करोति॒ तस्या॒ स्तस्याः᳚ करोति॒ जुष्टा॒म् जुष्टा᳚म् करोति॒ तस्याः᳚ । \newline
8. क॒रो॒ति॒ तस्या॒ स्तस्याः᳚ करोति करोति॒ तस्या᳚ स्ते ते॒ तस्याः᳚ करोति करोति॒ तस्या᳚ स्ते । \newline
9. तस्या᳚ स्ते ते॒ तस्या॒ स्तस्या᳚ स्ते स॒त्यस॑वसः स॒त्यस॑वस स्ते॒ तस्या॒ स्तस्या᳚ स्ते स॒त्यस॑वसः । \newline
10. ते॒ स॒त्यस॑वसः स॒त्यस॑वस स्ते ते स॒त्यस॑वसः प्रस॒वे प्र॑स॒वे स॒त्यस॑वस स्ते ते स॒त्यस॑वसः प्रस॒वे । \newline
11. स॒त्यस॑वसः प्रस॒वे प्र॑स॒वे स॒त्यस॑वसः स॒त्यस॑वसः प्रस॒व इतीति॑ प्रस॒वे स॒त्यस॑वसः स॒त्यस॑वसः प्रस॒व इति॑ । \newline
12. स॒त्यस॑वस॒ इति॑ स॒त्य - स॒व॒सः॒ । \newline
13. प्र॒स॒व इतीति॑ प्रस॒वे प्र॑स॒व इत्या॑हा॒ हेति॑ प्रस॒वे प्र॑स॒व इत्या॑ह । \newline
14. प्र॒स॒व इति॑ प्र - स॒वे । \newline
15. इत्या॑हा॒हे तीत्या॑ह सवि॒तृप्र॑सूताꣳ सवि॒तृप्र॑सूता मा॒हे तीत्या॑ह सवि॒तृप्र॑सूताम् । \newline
16. आ॒ह॒ स॒वि॒तृप्र॑सूताꣳ सवि॒तृप्र॑सूता माहाह सवि॒तृप्र॑सूता मे॒वैव स॑वि॒तृप्र॑सूता माहाह सवि॒तृप्र॑सूता मे॒व । \newline
17. स॒वि॒तृप्र॑सूता मे॒वैव स॑वि॒तृप्र॑सूताꣳ सवि॒तृप्र॑सूता मे॒व वाचं॒ ॅवाच॑ मे॒व स॑वि॒तृप्र॑सूताꣳ सवि॒तृप्र॑सूता मे॒व वाच᳚म् । \newline
18. स॒वि॒तृप्र॑सूता॒मिति॑ सवि॒तृ - प्र॒सू॒ता॒म् । \newline
19. ए॒व वाचं॒ ॅवाच॑ मे॒वैव वाच॒ मवाव॒ वाच॑ मे॒वैव वाच॒ मव॑ । \newline
20. वाच॒ मवाव॒ वाचं॒ ॅवाच॒ मव॑ रुन्धे रु॒न्धे ऽव॒ वाचं॒ ॅवाच॒ मव॑ रुन्धे । \newline
21. अव॑ रुन्धे रु॒न्धे ऽवाव॑ रुन्धे॒ काण्डे॑काण्डे॒ काण्डे॑काण्डे रु॒न्धे ऽवाव॑ रुन्धे॒ काण्डे॑काण्डे । \newline
22. रु॒न्धे॒ काण्डे॑काण्डे॒ काण्डे॑काण्डे रुन्धे रुन्धे॒ काण्डे॑काण्डे॒ वै वै काण्डे॑काण्डे रुन्धे रुन्धे॒ काण्डे॑काण्डे॒ वै । \newline
23. काण्डे॑काण्डे॒ वै वै काण्डे॑काण्डे॒ काण्डे॑काण्डे॒ वै क्रि॒यमा॑णे क्रि॒यमा॑णे॒ वै काण्डे॑काण्डे॒ काण्डे॑काण्डे॒ वै क्रि॒यमा॑णे । \newline
24. काण्डे॑काण्ड॒ इति॒ काण्डे᳚ - का॒ण्डे॒ । \newline
25. वै क्रि॒यमा॑णे क्रि॒यमा॑णे॒ वै वै क्रि॒यमा॑णे य॒ज्ञ्ं ॅय॒ज्ञ्म् क्रि॒यमा॑णे॒ वै वै क्रि॒यमा॑णे य॒ज्ञ्म् । \newline
26. क्रि॒यमा॑णे य॒ज्ञ्ं ॅय॒ज्ञ्म् क्रि॒यमा॑णे क्रि॒यमा॑णे य॒ज्ञ्ꣳ रक्षाꣳ॑सि॒ रक्षाꣳ॑सि य॒ज्ञ्म् क्रि॒यमा॑णे क्रि॒यमा॑णे य॒ज्ञ्ꣳ रक्षाꣳ॑सि । \newline
27. य॒ज्ञ्ꣳ रक्षाꣳ॑सि॒ रक्षाꣳ॑सि य॒ज्ञ्ं ॅय॒ज्ञ्ꣳ रक्षाꣳ॑सि जिघाꣳसन्ति जिघाꣳसन्ति॒ रक्षाꣳ॑सि य॒ज्ञ्ं ॅय॒ज्ञ्ꣳ रक्षाꣳ॑सि जिघाꣳसन्ति । \newline
28. रक्षाꣳ॑सि जिघाꣳसन्ति जिघाꣳसन्ति॒ रक्षाꣳ॑सि॒ रक्षाꣳ॑सि जिघाꣳसन् त्ये॒ष ए॒ष जि॑घाꣳसन्ति॒ रक्षाꣳ॑सि॒ रक्षाꣳ॑सि जिघाꣳसन् त्ये॒षः । \newline
29. जि॒घाꣳ॒॒स॒न् त्ये॒ष ए॒ष जि॑घाꣳसन्ति जिघाꣳसन् त्ये॒ष खलु॒ खल्वे॒ष जि॑घाꣳसन्ति जिघाꣳसन् त्ये॒ष खलु॑ । \newline
30. ए॒ष खलु॒ खल्वे॒ष ए॒ष खलु॒ वै वै खल्वे॒ष ए॒ष खलु॒ वै । \newline
31. खलु॒ वै वै खलु॒ खलु॒ वा अर॑क्षोह॒तो ऽर॑क्षोहतो॒ वै खलु॒ खलु॒ वा अर॑क्षोहतः । \newline
32. वा अर॑क्षोह॒तो ऽर॑क्षोहतो॒ वै वा अर॑क्षोहतः॒ पन्थाः॒ पन्था॒ अर॑क्षोहतो॒ वै वा अर॑क्षोहतः॒ पन्थाः᳚ । \newline
33. अर॑क्षोहतः॒ पन्थाः॒ पन्था॒ अर॑क्षोह॒तो ऽर॑क्षोहतः॒ पन्था॒ यो यः पन्था॒ अर॑क्षोह॒तो ऽर॑क्षोहतः॒ पन्था॒ यः । \newline
34. अर॑क्षोहत॒ इत्यर॑क्षः - ह॒तः॒ । \newline
35. पन्था॒ यो यः पन्थाः॒ पन्था॒ यो᳚ ऽग्ने र॒ग्नेर् यः पन्थाः॒ पन्था॒ यो᳚ ऽग्नेः । \newline
36. यो᳚ ऽग्ने र॒ग्नेर् यो यो᳚ ऽग्नेश्च॑ चा॒ग्नेर् यो यो᳚ ऽग्नेश्च॑ । \newline
37. अ॒ग्नेश्च॑ चा॒ग्ने र॒ग्ने श्च॒ सूर्य॑स्य॒ सूर्य॑स्य चा॒ग्ने र॒ग्ने श्च॒ सूर्य॑स्य । \newline
38. च॒ सूर्य॑स्य॒ सूर्य॑स्य च च॒ सूर्य॑स्य च च॒ सूर्य॑स्य च च॒ सूर्य॑स्य च । \newline
39. सूर्य॑स्य च च॒ सूर्य॑स्य॒ सूर्य॑स्य च॒ सूर्य॑स्य॒ सूर्य॑स्य च॒ सूर्य॑स्य॒ सूर्य॑स्य च॒ सूर्य॑स्य । \newline
40. च॒ सूर्य॑स्य॒ सूर्य॑स्य च च॒ सूर्य॑स्य॒ चक्षु॒ श्चक्षुः॒ सूर्य॑स्य च च॒ सूर्य॑स्य॒ चक्षुः॑ । \newline
41. सूर्य॑स्य॒ चक्षु॒ श्चक्षुः॒ सूर्य॑स्य॒ सूर्य॑स्य॒ चक्षु॒रा चक्षुः॒ सूर्य॑स्य॒ सूर्य॑स्य॒ चक्षु॒रा । \newline
42. चक्षु॒रा चक्षु॒ श्चक्षु॒रा ऽरु॑ह मरुह॒ मा चक्षु॒ श्चक्षु॒रा ऽरु॑हम् । \newline
43. आ ऽरु॑ह मरुह॒ मा ऽरु॑ह म॒ग्ने र॒ग्ने र॑रुह॒ मा ऽरु॑ह म॒ग्नेः । \newline
44. अ॒रु॒ह॒ म॒ग्ने र॒ग्ने र॑रुह मरुह म॒ग्ने र॒क्ष्णो᳚(1॒) ऽक्ष्णो᳚ ऽग्ने र॑रुह मरुह म॒ग्ने र॒क्ष्णः । \newline
45. अ॒ग्ने र॒क्ष्णो᳚(1॒) ऽक्ष्णो᳚ ऽग्ने र॒ग्ने र॒क्ष्णः क॒नीनि॑काम् क॒नीनि॑का म॒क्ष्णो᳚ ऽग्ने र॒ग्ने र॒क्ष्णः क॒नीनि॑काम् । \newline
46. अ॒क्ष्णः क॒नीनि॑काम् क॒नीनि॑का म॒क्ष्णो᳚ ऽक्ष्णः क॒नीनि॑का॒ मितीति॑ क॒नीनि॑का म॒क्ष्णो᳚ ऽक्ष्णः क॒नीनि॑का॒ मिति॑ । \newline
47. क॒नीनि॑का॒ मितीति॑ क॒नीनि॑काम् क॒नीनि॑का॒ मित्या॑हा॒हे ति॑ क॒नीनि॑काम् क॒नीनि॑का॒ मित्या॑ह । \newline
48. इत्या॑हा॒हे तीत्या॑ह॒ यो य आ॒हे तीत्या॑ह॒ यः । \newline
49. आ॒ह॒ यो य आ॑हाह॒ य ए॒वैव य आ॑हाह॒ य ए॒व । \newline
50. य ए॒वैव यो य ए॒वा र॑क्षोह॒तो ऽर॑क्षोहत ए॒व यो य ए॒वा र॑क्षोहतः । \newline
51. ए॒वा र॑क्षोह॒तो ऽर॑क्षोहत ए॒वैवा र॑क्षोहतः॒ पन्थाः॒ पन्था॒ अर॑क्षोहत ए॒वैवा र॑क्षोहतः॒ पन्थाः᳚ । \newline
52. अर॑क्षोहतः॒ पन्थाः॒ पन्था॒ अर॑क्षोह॒तो ऽर॑क्षोहतः॒ पन्था॒ स्तम् तम् पन्था॒ अर॑क्षोह॒तो ऽर॑क्षोहतः॒ पन्था॒ स्तम् । \newline
53. अर॑क्षोहत॒ इत्यर॑क्षः - ह॒तः॒ । \newline
54. पन्था॒ स्तम् तम् पन्थाः॒ पन्था॒ स्तꣳ स॒मारो॑हति स॒मारो॑हति॒ तम् पन्थाः॒ पन्था॒ स्तꣳ स॒मारो॑हति । \newline
55. तꣳ स॒मारो॑हति स॒मारो॑हति॒ तम् तꣳ स॒मारो॑हति॒ वाग् वाख् स॒मारो॑हति॒ तम् तꣳ स॒मारो॑हति॒ वाक् । \newline
56. स॒मारो॑हति॒ वाग् वाख् स॒मारो॑हति स॒मारो॑हति॒ वाग् वै वै वाख् स॒मारो॑हति स॒मारो॑हति॒ वाग् वै । \newline
57. स॒मारो॑ह॒तीति॑ सं - आरो॑हति । \newline
\pagebreak
\markright{ TS 6.1.7.4  \hfill https://www.vedavms.in \hfill}

\section{ TS 6.1.7.4 }

\textbf{TS 6.1.7.4 } \newline
\textbf{Samhita Paata} \newline

वाग्वा ए॒षा यथ् सो॑म॒क्रय॑णी॒ चिद॑सि म॒नाऽसीत्या॑ह॒ शास्त्ये॒वैना॑मे॒तत् तस्मा᳚च्छि॒ष्टाः प्र॒जा जा॑यन्ते॒ चिद॒सीत्या॑ह॒ यद्धि मन॑सा चे॒तय॑ते॒ तद्-वा॒चा वद॑ति म॒नाऽसीत्या॑ह॒ यद्धि मन॑साऽभि॒गच्छ॑ति॒ तत् क॒रोति॒ धीर॒सीत्या॑ह॒ यद्धि मन॑सा॒ ध्याय॑ति॒ तद्-वा॒चा - [  ] \newline

\textbf{Pada Paata} \newline

वाक् । वै । ए॒षा । यत् । सो॒म॒क्रय॒णीति॑ सोम - क्रय॑णी । चित् । अ॒सि॒ । म॒ना । अ॒सि॒ । इति॑ । आ॒ह॒ । शास्ति॑ । ए॒व । ए॒ना॒म् । ए॒तत् । तस्मा᳚त् । शि॒ष्टाः । प्र॒जा इति॑ प्र - जाः । जा॒य॒न्ते॒ । चित् । अ॒सि॒ । इति॑ । आ॒ह॒ । यत् । हि । मन॑सा । चे॒तय॑ते । तत् । वा॒चा । वद॑ति । म॒ना । अ॒सि॒ । इति॑ । आ॒ह॒ । यत् । हि । मन॑सा । अ॒भि॒गच्छ॒तीत्य॑भि - गच्छ॑ति । तत् । क॒रोति॑ । धीः । अ॒सि॒ । इति॑ । आ॒ह॒ । यत् । हि । मन॑सा । ध्याय॑ति । तत् । वा॒चा ।  \newline


\textbf{Krama Paata} \newline

वाग् वै । वा ए॒षा । ए॒षा यत् । यथ् सो॑म॒क्रय॑णी । सो॒म॒क्रय॑णी॒ चित् । सो॒म॒क्रय॒णीति॑ सोम - क्रय॑णी । चिद॑सि । अ॒सि॒ म॒ना । म॒नाऽसि॑ । अ॒सीति॑ । इत्या॑ह । आ॒ह॒ शास्ति॑ । शास्त्ये॒व । ए॒वैना᳚म् । ए॒ना॒मे॒तत् । ए॒तत् तस्मा᳚त् । तस्मा᳚च्छि॒ष्टाः । शि॒ष्टाः प्र॒जाः । प्र॒जा जा॑यन्ते । प्र॒जा इति॑ प्र - जाः । जा॒य॒न्ते॒ चित् । चिद॑सि । अ॒सीति॑ । इत्या॑ह । आ॒ह॒ यत् । यद्‌धि । हि मन॑सा । मन॑सा चे॒तय॑ते । चे॒तय॑ते॒ तत् । तद् वा॒चा । वा॒चा वद॑ति । वद॑ति म॒ना । म॒नाऽसि॑ । अ॒सीति॑ । इत्या॑ह । आ॒ह॒ यत् । यद्‌धि । हि मन॑सा । मन॑साऽभि॒गच्छ॑ति । अ॒भि॒गच्छ॑ति॒ तत् । अ॒भि॒गच्छ॒तीत्य॑भि - गच्छ॑ति । तत् क॒रोति॑ । क॒रोति॒ धीः । धीर॑सि । अ॒सीति॑ । इत्या॑ह । आ॒ह॒ यत् । यद्‌धि । हि मन॑सा । मन॑सा॒ ध्याय॑ति । ध्याय॑ति॒ तत् । तद् वा॒चा । वा॒चा वद॑ति \newline

\textbf{Jatai Paata} \newline

1. वाग् वै वै वाग् वाग् वै । \newline
2. वा ए॒षैषा वै वा ए॒षा । \newline
3. ए॒षा यद् यदे॒ षैषा यत् । \newline
4. यथ् सो॑म॒क्रय॑णी सोम॒क्रय॑णी॒ यद् यथ् सो॑म॒क्रय॑णी । \newline
5. सो॒म॒क्रय॑णी॒ चिच् चिथ् सो॑म॒क्रय॑णी सोम॒क्रय॑णी॒ चित् । \newline
6. सो॒म॒क्रय॒णीति॑ सोम - क्रय॑णी । \newline
7. चिद॑ स्यसि॒ चिच् चिद॑सि । \newline
8. अ॒सि॒ म॒ना म॒ना ऽस्य॑सि म॒ना । \newline
9. म॒ना ऽस्य॑सि म॒ना म॒ना ऽसि॑ । \newline
10. अ॒सीती त्य॑स्य॒ सीति॑ । \newline
11. इत्या॑हा॒हे तीत्या॑ह । \newline
12. आ॒ह॒ शास्ति॒ शास्त्या॑ हाह॒ शास्ति॑ । \newline
13. शास्त्ये॒ वैव शास्ति॒ शास्त्ये॒व । \newline
14. ए॒वैना॑ मेना मे॒वै वैना᳚म् । \newline
15. ए॒ना॒ मे॒त दे॒त दे॑ना मेना मे॒तत् । \newline
16. ए॒तत् तस्मा॒त् तस्मा॑ दे॒त दे॒तत् तस्मा᳚त् । \newline
17. तस्मा᳚ च्छि॒ष्टाः शि॒ष्टा स्तस्मा॒त् तस्मा᳚ च्छि॒ष्टाः । \newline
18. शि॒ष्टाः प्र॒जाः प्र॒जाः शि॒ष्टाः शि॒ष्टाः प्र॒जाः । \newline
19. प्र॒जा जा॑यन्ते जायन्ते प्र॒जाः प्र॒जा जा॑यन्ते । \newline
20. प्र॒जा इति॑ प्र - जाः । \newline
21. जा॒य॒न्ते॒ चिच् चिज् जा॑यन्ते जायन्ते॒ चित् । \newline
22. चिद॑ स्यसि॒ चिच् चिद॑सि । \newline
23. अ॒सीती त्य॑स्य॒ सीति॑ । \newline
24. इत्या॑हा॒हे तीत्या॑ह । \newline
25. आ॒ह॒ यद् यदा॑ हाह॒ यत् । \newline
26. यद्धि हि यद् यद्धि । \newline
27. हि मन॑सा॒ मन॑सा॒ हि हि मन॑सा । \newline
28. मन॑सा चे॒तय॑ते चे॒तय॑ते॒ मन॑सा॒ मन॑सा चे॒तय॑ते । \newline
29. चे॒तय॑ते॒ तत् तच् चे॒तय॑ते चे॒तय॑ते॒ तत् । \newline
30. तद् वा॒चा वा॒चा तत् तद् वा॒चा । \newline
31. वा॒चा वद॑ति॒ वद॑ति वा॒चा वा॒चा वद॑ति । \newline
32. वद॑ति म॒ना म॒ना वद॑ति॒ वद॑ति म॒ना । \newline
33. म॒ना ऽस्य॑सि म॒ना म॒ना ऽसि॑ । \newline
34. अ॒सीती त्य॑स्य॒ सीति॑ । \newline
35. इत्या॑हा॒हे तीत्या॑ह । \newline
36. आ॒ह॒ यद् यदा॑ हाह॒ यत् । \newline
37. यद्धि हि यद् यद्धि । \newline
38. हि मन॑सा॒ मन॑सा॒ हि हि मन॑सा । \newline
39. मन॑सा ऽभि॒गच्छ॑ त्यभि॒गच्छ॑ति॒ मन॑सा॒ मन॑सा ऽभि॒गच्छ॑ति । \newline
40. अ॒भि॒गच्छ॑ति॒ तत् तद॑भि॒गच्छ॑ त्यभि॒गच्छ॑ति॒ तत् । \newline
41. अ॒भि॒गच्छ॒तीत्य॑भि - गच्छ॑ति । \newline
42. तत् क॒रोति॑ क॒रोति॒ तत् तत् क॒रोति॑ । \newline
43. क॒रोति॒ धीर् धीः क॒रोति॑ क॒रोति॒ धीः । \newline
44. धीर॑ स्यसि॒ धीर् धीर॑सि । \newline
45. अ॒सीती त्य॑स्य॒ सीति॑ । \newline
46. इत्या॑हा॒हे तीत्या॑ह । \newline
47. आ॒ह॒ यद् यदा॑ हाह॒ यत् । \newline
48. यद्धि हि यद् यद्धि । \newline
49. हि मन॑सा॒ मन॑सा॒ हि हि मन॑सा । \newline
50. मन॑सा॒ ध्याय॑ति॒ ध्याय॑ति॒ मन॑सा॒ मन॑सा॒ ध्याय॑ति । \newline
51. ध्याय॑ति॒ तत् तद् ध्याय॑ति॒ ध्याय॑ति॒ तत् । \newline
52. तद् वा॒चा वा॒चा तत् तद् वा॒चा । \newline
53. वा॒चा वद॑ति॒ वद॑ति वा॒चा वा॒चा वद॑ति । \newline

\textbf{Ghana Paata } \newline

1. वाग् वै वै वाग् वाग् वा ए॒षैषा वै वाग् वाग् वा ए॒षा । \newline
2. वा ए॒षैषा वै वा ए॒षा यद् यदे॒षा वै वा ए॒षा यत् । \newline
3. ए॒षा यद् यदे॒षैषा यथ् सो॑म॒क्रय॑णी सोम॒क्रय॑णी॒ यदे॒षैषा यथ् सो॑म॒क्रय॑णी । \newline
4. यथ् सो॑म॒क्रय॑णी सोम॒क्रय॑णी॒ यद् यथ् सो॑म॒क्रय॑णी॒ चिच् चिथ् सो॑म॒क्रय॑णी॒ यद् यथ् सो॑म॒क्रय॑णी॒ चित् । \newline
5. सो॒म॒क्रय॑णी॒ चिच् चिथ् सो॑म॒क्रय॑णी सोम॒क्रय॑णी॒ चिद॑स्यसि॒ चिथ् सो॑म॒क्रय॑णी सोम॒क्रय॑णी॒ चिद॑सि । \newline
6. सो॒म॒क्रय॒णीति॑ सोम - क्रय॑णी । \newline
7. चिद॑ स्यसि॒ चिच् चिद॑सि म॒ना म॒ना ऽसि॒ चिच् चिद॑सि म॒ना । \newline
8. अ॒सि॒ म॒ना म॒ना ऽस्य॑सि म॒ना ऽस्य॑सि म॒ना ऽस्य॑सि म॒ना ऽसि॑ । \newline
9. म॒ना ऽस्य॑सि म॒ना म॒ना ऽसीती त्य॑सि म॒ना म॒ना ऽसीति॑ । \newline
10. अ॒सीती त्य॑स्य॒ सीत्या॑ हा॒हे त्य॑स्य॒ सीत्या॑ह । \newline
11. इत्या॑हा॒हे तीत्या॑ह॒ शास्ति॒ शास्त्या॒हे तीत्या॑ह॒ शास्ति॑ । \newline
12. आ॒ह॒ शास्ति॒ शास्त्या॑ हाह॒ शास्त्ये॒ वैव शास्त्या॑ हाह॒ शास्त्ये॒व । \newline
13. शास्त्ये॒ वैव शास्ति॒ शास्त्ये॒ वैना॑ मेना मे॒व शास्ति॒ शास्त्ये॒ वैना᳚म् । \newline
14. ए॒वैना॑ मेना मे॒वै वैना॑ मे॒त दे॒त दे॑ना मे॒वै वैना॑ मे॒तत् । \newline
15. ए॒ना॒ मे॒त दे॒त दे॑ना मेना मे॒तत् तस्मा॒त् तस्मा॑ दे॒त दे॑ना मेना मे॒तत् तस्मा᳚त् । \newline
16. ए॒तत् तस्मा॒त् तस्मा॑ दे॒त दे॒तत् तस्मा᳚च् छि॒ष्टाः शि॒ष्टा स्तस्मा॑ दे॒त दे॒तत् तस्मा᳚च् छि॒ष्टाः । \newline
17. तस्मा᳚च् छि॒ष्टाः शि॒ष्टा स्तस्मा॒त् तस्मा᳚च् छि॒ष्टाः प्र॒जाः प्र॒जाः शि॒ष्टा स्तस्मा॒त् तस्मा᳚ च्छि॒ष्टाः प्र॒जाः । \newline
18. शि॒ष्टाः प्र॒जाः प्र॒जाः शि॒ष्टाः शि॒ष्टाः प्र॒जा जा॑यन्ते जायन्ते प्र॒जाः शि॒ष्टाः शि॒ष्टाः प्र॒जा जा॑यन्ते । \newline
19. प्र॒जा जा॑यन्ते जायन्ते प्र॒जाः प्र॒जा जा॑यन्ते॒ चिच् चिज् जा॑यन्ते प्र॒जाः प्र॒जा जा॑यन्ते॒ चित् । \newline
20. प्र॒जा इति॑ प्र - जाः । \newline
21. जा॒य॒न्ते॒ चिच् चिज् जा॑यन्ते जायन्ते॒ चिद॑स्यसि॒ चिज् जा॑यन्ते जायन्ते॒ चिद॑सि । \newline
22. चिद॑ स्यसि॒ चिच् चिद॒सी तीत्य॑सि॒ चिच् चिद॒ सीति॑ । \newline
23. अ॒सी तीत्य॑स्य॒ सीत्या॑ हा॒हे त्य॑स्य॒ सीत्या॑ह । \newline
24. इत्या॑हा॒हे तीत्या॑ह॒ यद् यदा॒हे तीत्या॑ह॒ यत् । \newline
25. आ॒ह॒ यद् यदा॑ हाह॒ यद्धि हि यदा॑ हाह॒ यद्धि । \newline
26. यद्धि हि यद् यद्धि मन॑सा॒ मन॑सा॒ हि यद् यद्धि मन॑सा । \newline
27. हि मन॑सा॒ मन॑सा॒ हि हि मन॑सा चे॒तय॑ते चे॒तय॑ते॒ मन॑सा॒ हि हि मन॑सा चे॒तय॑ते । \newline
28. मन॑सा चे॒तय॑ते चे॒तय॑ते॒ मन॑सा॒ मन॑सा चे॒तय॑ते॒ तत् तच् चे॒तय॑ते॒ मन॑सा॒ मन॑सा चे॒तय॑ते॒ तत् । \newline
29. चे॒तय॑ते॒ तत् तच् चे॒तय॑ते चे॒तय॑ते॒ तद् वा॒चा वा॒चा तच् चे॒तय॑ते चे॒तय॑ते॒ तद् वा॒चा । \newline
30. तद् वा॒चा वा॒चा तत् तद् वा॒चा वद॑ति॒ वद॑ति वा॒चा तत् तद् वा॒चा वद॑ति । \newline
31. वा॒चा वद॑ति॒ वद॑ति वा॒चा वा॒चा वद॑ति म॒ना म॒ना वद॑ति वा॒चा वा॒चा वद॑ति म॒ना । \newline
32. वद॑ति म॒ना म॒ना वद॑ति॒ वद॑ति म॒ना ऽस्य॑सि म॒ना वद॑ति॒ वद॑ति म॒ना ऽसि॑ । \newline
33. म॒ना ऽस्य॑सि म॒ना म॒ना ऽसीती त्य॑सि म॒ना म॒ना ऽसीति॑ । \newline
34. अ॒सीती त्य॑स्य॒ सीत्या॑ हा॒हे त्य॑स्य॒ सीत्या॑ह । \newline
35. इत्या॑हा॒हे तीत्या॑ह॒ यद् यदा॒हे तीत्या॑ह॒ यत् । \newline
36. आ॒ह॒ यद् यदा॑ हाह॒ यद्धि हि यदा॑ हाह॒ यद्धि । \newline
37. यद्धि हि यद् यद्धि मन॑सा॒ मन॑सा॒ हि यद् यद्धि मन॑सा । \newline
38. हि मन॑सा॒ मन॑सा॒ हि हि मन॑सा ऽभि॒गच्छ॑ त्यभि॒गच्छ॑ति॒ मन॑सा॒ हि हि मन॑सा ऽभि॒गच्छ॑ति । \newline
39. मन॑सा ऽभि॒गच्छ॑ त्यभि॒गच्छ॑ति॒ मन॑सा॒ मन॑सा ऽभि॒गच्छ॑ति॒ तत् तद॑भि॒गच्छ॑ति॒ मन॑सा॒ मन॑सा ऽभि॒गच्छ॑ति॒ तत् । \newline
40. अ॒भि॒गच्छ॑ति॒ तत् तद॑भि॒गच्छ॑ त्यभि॒गच्छ॑ति॒ तत् क॒रोति॑ क॒रोति॒ तद॑भि॒गच्छ॑ त्यभि॒गच्छ॑ति॒ तत् क॒रोति॑ । \newline
41. अ॒भि॒गच्छ॒तीत्य॑भि - गच्छ॑ति । \newline
42. तत् क॒रोति॑ क॒रोति॒ तत् तत् क॒रोति॒ धीर् धीः क॒रोति॒ तत् तत् क॒रोति॒ धीः । \newline
43. क॒रोति॒ धीर् धीः क॒रोति॑ क॒रोति॒ धीर॑ स्यसि॒ धीः क॒रोति॑ क॒रोति॒ धीर॑सि । \newline
44. धीर॑ स्यसि॒ धीर् धीर॒सीती त्य॑सि॒ धीर् धीर॒ सीति॑ । \newline
45. अ॒सीतीत्य॑ स्य॒सी त्या॑हा॒हे त्य॑स्य॒ सीत्या॑ह । \newline
46. इत्या॑हा॒हे तीत्या॑ह॒ यद् यदा॒हे तीत्या॑ह॒ यत् । \newline
47. आ॒ह॒ यद् यदा॑ हाह॒ यद्धि हि यदा॑ हाह॒ यद्धि । \newline
48. यद्धि हि यद् यद्धि मन॑सा॒ मन॑सा॒ हि यद् यद्धि मन॑सा । \newline
49. हि मन॑सा॒ मन॑सा॒ हि हि मन॑सा॒ ध्याय॑ति॒ ध्याय॑ति॒ मन॑सा॒ हि हि मन॑सा॒ ध्याय॑ति । \newline
50. मन॑सा॒ ध्याय॑ति॒ ध्याय॑ति॒ मन॑सा॒ मन॑सा॒ ध्याय॑ति॒ तत् तद् ध्याय॑ति॒ मन॑सा॒ मन॑सा॒ ध्याय॑ति॒ तत् । \newline
51. ध्याय॑ति॒ तत् तद् ध्याय॑ति॒ ध्याय॑ति॒ तद् वा॒चा वा॒चा तद् ध्याय॑ति॒ ध्याय॑ति॒ तद् वा॒चा । \newline
52. तद् वा॒चा वा॒चा तत् तद् वा॒चा वद॑ति॒ वद॑ति वा॒चा तत् तद् वा॒चा वद॑ति । \newline
53. वा॒चा वद॑ति॒ वद॑ति वा॒चा वा॒चा वद॑ति॒ दक्षि॑णा॒ दक्षि॑णा॒ वद॑ति वा॒चा वा॒चा वद॑ति॒ दक्षि॑णा । \newline
\pagebreak
\markright{ TS 6.1.7.5  \hfill https://www.vedavms.in \hfill}

\section{ TS 6.1.7.5 }

\textbf{TS 6.1.7.5 } \newline
\textbf{Samhita Paata} \newline

वद॑ति॒ दक्षि॑णा॒ऽसीत्या॑ह॒ दक्षि॑णा॒ ह्ये॑षा य॒ज्ञिया॒ऽसीत्या॑ह य॒ज्ञिया॑मे॒वैनां᳚ करोति क्ष॒त्रिया॒सीत्या॑ह क्ष॒त्रिया॒ ह्ये॑षा ऽदि॑तिरस्युभ॒यत॑श्शी॒र्ष्णीत्या॑ह॒ यदे॒वाऽऽ*दि॒त्यः प्रा॑य॒णीयो॑ य॒ज्ञाना॑मादि॒त्य उ॑दय॒नीय॒-स्तस्मा॑दे॒वमा॑ह॒ यदब॑द्धा॒ स्यादय॑ता स्या॒द्यत् प॑दिब॒द्धा-ऽनु॒स्तर॑णी स्यात् प्र॒मायु॑को॒ यज॑मानः स्या॒ - [  ] \newline

\textbf{Pada Paata} \newline

वद॑ति । दक्षि॑णा । अ॒सि॒ । इति॑ । आ॒ह॒ । दक्षि॑णा । हि । ए॒षा । य॒ज्ञिया᳚ । अ॒सि॒ । इति॑ । आ॒ह॒ । य॒ज्ञिया᳚म् । ए॒व । ए॒ना॒म् । क॒रो॒ति॒ । क्ष॒त्रिया᳚ । अ॒सि॒ । इति॑ । आ॒ह॒ । क्ष॒त्रिया᳚ । हि । ए॒षा । अदि॑तिः । अ॒सि॒ । उ॒भ॒यत॑श्शी॒र्ष्णीत्यु॑भ॒यतः॑ - शी॒र्ष्णी॒ । इति॑ । आ॒ह॒ । यत् । ए॒व । आ॒दि॒त्यः । प्रा॒य॒णीय॒ इति॑ प्र - अ॒य॒णीयः॑ । य॒ज्ञाना᳚म् । आ॒दि॒त्यः । उ॒द॒य॒नीय॒ इत्यु॑त् - अ॒य॒नीयः॑ । तस्मा᳚त् । ए॒वम् । आ॒ह॒ । यत् । अब॑द्धा । स्यात् । अय॑ता । स्या॒त् । यत् । प॒दि॒ब॒द्धेति॑ पदि - ब॒द्धा । अ॒नु॒स्तर॒णीत्य॑नु - स्तर॑णी । स्या॒त् । प्र॒मायु॑क॒ इति॑ प्र - मायु॑कः । यज॑मानः । स्या॒त् ।  \newline


\textbf{Krama Paata} \newline

वद॑ति॒ दक्षि॑णा । दक्षि॑णाऽसि । अ॒सीति॑ । इत्या॑ह । आ॒ह॒ दक्षि॑णा । दक्षि॑णा॒ हि । ह्ये॑षा । ए॒षा य॒ज्ञिया᳚ । य॒ज्ञिया॑ऽसि । अ॒सीति॑ । इत्या॑ह । आ॒ह॒ य॒ज्ञिया᳚म् । य॒ज्ञिया॑मे॒व । ए॒वैना᳚म् । ए॒ना॒म् क॒रो॒ति॒ । क॒रो॒ति॒ क्ष॒त्रिया᳚ । क्ष॒त्रिया॑ऽसि । अ॒सीति॑ । इत्या॑ह । आ॒ह॒ क्ष॒त्रिया᳚ । क्ष॒त्रिया॒ हि । ह्ये॑षा । ए॒षाऽदि॑तिः । अदि॑तिरसि । अ॒स्यु॒भ॒यत॑श्शीर्ष्णी । उ॒भ॒यत॑श्शी॒र्ष्णीति॑ । उ॒भ॒यत॑श्शी॒र्ष्णीत्यु॑भ॒यतः॑ - शी॒र्ष्णी॒ । इत्या॑ह । आ॒ह॒ यत् । यदे॒व । ए॒वादि॒त्यः । आ॒दि॒त्यः प्रा॑य॒णीयः॑ । प्रा॒य॒णीयो॑ य॒ज्ञाना᳚म् । प्रा॒य॒णीय॒ इति॑ प्र - अ॒य॒नीयः॑ । य॒ज्ञाना॑मादि॒त्यः । आ॒दि॒त्य उ॑दय॒नीयः॑ । उ॒द॒य॒नीय॒स्तस्मा᳚त् । उ॒द॒य॒नीय॒ इत्यु॑त् - अ॒य॒नीयः॑ । तस्मा॑दे॒वम् । ए॒वमा॑ह । आ॒ह॒ यत् । यदब॑द्धा । अब॑द्धा॒ स्यात् । स्यादय॑ता । अय॑ता स्यात् । स्या॒द् यत् । यत् प॑दिब॒द्धा । प॒दि॒ब॒द्धाऽनु॒स्तर॑णी । प॒दि॒ब॒द्धेति॑ पदि - ब॒द्धा । अ॒नु॒स्तर॑णी स्यात् । अ॒नु॒स्तर॒णीत्य॑नु - स्तर॑णी । स्या॒त् प्र॒मायु॑कः । प्र॒मायु॑को॒ यज॑मानः । प्र॒यायु॑क॒ इति॑ प्र - मायु॑कः । यज॑मानः स्यात् । स्या॒द् यत् \newline

\textbf{Jatai Paata} \newline

1. वद॑ति॒ दक्षि॑णा॒ दक्षि॑णा॒ वद॑ति॒ वद॑ति॒ दक्षि॑णा । \newline
2. दक्षि॑णा ऽस्यसि॒ दक्षि॑णा॒ दक्षि॑णा ऽसि । \newline
3. अ॒सीती त्य॑स्य॒ सीति॑ । \newline
4. इत्या॑हा॒हे तीत्या॑ह । \newline
5. आ॒ह॒ दक्षि॑णा॒ दक्षि॑णा ऽऽहाह॒ दक्षि॑णा । \newline
6. दक्षि॑णा॒ हि हि दक्षि॑णा॒ दक्षि॑णा॒ हि । \newline
7. ह्ये॑षैषा हि ह्ये॑षा । \newline
8. ए॒षा य॒ज्ञिया॑ य॒ज्ञियै॒ षैषा य॒ज्ञिया᳚ । \newline
9. य॒ज्ञिया᳚ ऽस्यसि य॒ज्ञिया॑ य॒ज्ञिया॑ ऽसि । \newline
10. अ॒सीती त्य॑स्य॒ सीति॑ । \newline
11. इत्या॑हा॒हे तीत्या॑ह । \newline
12. आ॒ह॒ य॒ज्ञियां᳚ ॅय॒ज्ञिया॑ माहाह य॒ज्ञिया᳚म् । \newline
13. य॒ज्ञिया॑ मे॒वैव य॒ज्ञियां᳚ ॅय॒ज्ञिया॑ मे॒व । \newline
14. ए॒वैना॑ मेना मे॒वै वैना᳚म् । \newline
15. ए॒ना॒म् क॒रो॒ति॒ क॒रो॒ त्ये॒ना॒ मे॒ना॒म् क॒रो॒ति॒ । \newline
16. क॒रो॒ति॒ क्ष॒त्रिया᳚ क्ष॒त्रिया॑ करोति करोति क्ष॒त्रिया᳚ । \newline
17. क्ष॒त्रिया᳚ ऽस्यसि क्ष॒त्रिया᳚ क्ष॒त्रिया॑ ऽसि । \newline
18. अ॒सीती त्य॑स्य॒ सीति॑ । \newline
19. इत्या॑हा॒हे तीत्या॑ह । \newline
20. आ॒ह॒ क्ष॒त्रिया᳚ क्ष॒त्रिया॑ ऽऽहाह क्ष॒त्रिया᳚ । \newline
21. क्ष॒त्रिया॒ हि हि क्ष॒त्रिया᳚ क्ष॒त्रिया॒ हि । \newline
22. ह्ये॑षैषा हि ह्ये॑षा । \newline
23. ए॒षा ऽदि॑ति॒ रदि॑ति रे॒षैषा ऽदि॑तिः । \newline
24. अदि॑ति रस्य॒स्य दि॑ति॒ रदि॑ति रसि । \newline
25. अ॒स्यु॒ भ॒यत॑श्शीर् ष्ण्युभ॒यत॑श्शीर् ष्ण्यस्य स्युभ॒यत॑श्शीर्ष्णी । \newline
26. उ॒भ॒यत॑श्शी॒र्ष्णीती त्यु॑भ॒यत॑श्शीर् ष्ण्युभ॒यत॑श्शी॒र्ष्णीति॑ । \newline
27. उ॒भ॒यत॑श्शी॒र्ष्णीत्यु॑भ॒यतः॑ - शी॒र्ष्णी॒ । \newline
28. इत्या॑हा॒हे तीत्या॑ह । \newline
29. आ॒ह॒ यद् यदा॑ हाह॒ यत् । \newline
30. यदे॒ वैव यद् यदे॒व । \newline
31. ए॒वादि॒त्य आ॑दि॒त्य ए॒वै वादि॒त्यः । \newline
32. आ॒दि॒त्यः प्रा॑य॒णीयः॑ प्राय॒णीय॑ आदि॒त्य आ॑दि॒त्यः प्रा॑य॒णीयः॑ । \newline
33. प्रा॒य॒णीयो॑ य॒ज्ञानां᳚ ॅय॒ज्ञाना᳚म् प्राय॒णीयः॑ प्राय॒णीयो॑ य॒ज्ञाना᳚म् । \newline
34. प्रा॒य॒णीय॒ इति॑ प्र - अ॒य॒नीयः॑ । \newline
35. य॒ज्ञाना॑ मादि॒त्य आ॑दि॒त्यो य॒ज्ञानां᳚ ॅय॒ज्ञाना॑ मादि॒त्यः । \newline
36. आ॒दि॒त्य उ॑दय॒नीय॑ उदय॒नीय॑ आदि॒त्य आ॑दि॒त्य उ॑दय॒नीयः॑ । \newline
37. उ॒द॒य॒नीय॒ स्तस्मा॒त् तस्मा॑ दुदय॒नीय॑ उदय॒नीय॒ स्तस्मा᳚त् । \newline
38. उ॒द॒य॒नीय॒ इत्यु॑त् - अ॒य॒नीयः॑ । \newline
39. तस्मा॑ दे॒व मे॒वम् तस्मा॒त् तस्मा॑ दे॒वम् । \newline
40. ए॒व मा॑हा है॒व मे॒व मा॑ह । \newline
41. आ॒ह॒ यद् यदा॑ हाह॒ यत् । \newline
42. यदब॒द्धा ऽब॑द्धा॒ यद् यदब॑द्धा । \newline
43. अब॑द्धा॒ स्याथ् स्या दब॒द्धा ऽब॑द्धा॒ स्यात् । \newline
44. स्या दय॒ता ऽय॑ता॒ स्याथ् स्या दय॑ता । \newline
45. अय॑ता स्याथ् स्या॒ दय॒ता ऽय॑ता स्यात् । \newline
46. स्या॒द् यद् यथ् स्या᳚थ् स्या॒द् यत् । \newline
47. यत् प॑दिब॒द्धा प॑दिब॒द्धा यद् यत् प॑दिब॒द्धा । \newline
48. प॒दि॒ब॒द्धा ऽनु॒स्तर॑ण्य नु॒स्तर॑णी पदिब॒द्धा प॑दिब॒द्धा ऽनु॒स्तर॑णी । \newline
49. प॒दि॒ब॒द्धेति॑ पदि - ब॒द्धा । \newline
50. अ॒नु॒स्तर॑णी स्याथ् स्या दनु॒स्तर॑ ण्यनु॒स्तर॑णी स्यात् । \newline
51. अ॒नु॒स्तर॒णीत्य॑नु - स्तर॑णी । \newline
52. स्या॒त् प्र॒मायु॑कः प्र॒मायु॑कः स्याथ् स्यात् प्र॒मायु॑कः । \newline
53. प्र॒मायु॑को॒ यज॑मानो॒ यज॑मानः प्र॒मायु॑कः प्र॒मायु॑को॒ यज॑मानः । \newline
54. प्र॒मायु॑क॒ इति॑ प्र - मायु॑कः । \newline
55. यज॑मानः स्याथ् स्या॒द् यज॑मानो॒ यज॑मानः स्यात् । \newline
56. स्या॒द् यद् यथ् स्या᳚थ् स्या॒द् यत् । \newline

\textbf{Ghana Paata } \newline

1. वद॑ति॒ दक्षि॑णा॒ दक्षि॑णा॒ वद॑ति॒ वद॑ति॒ दक्षि॑णा ऽस्यसि॒ दक्षि॑णा॒ वद॑ति॒ वद॑ति॒ दक्षि॑णा ऽसि । \newline
2. दक्षि॑णा ऽस्यसि॒ दक्षि॑णा॒ दक्षि॑णा॒ ऽसीती त्य॑सि॒ दक्षि॑णा॒ दक्षि॑णा॒ ऽसीति॑ । \newline
3. अ॒सीती त्य॑स्य॒ सीत्या॑ हा॒हे त्य॑स्य॒ सीत्या॑ह । \newline
4. इत्या॑हा॒हे तीत्या॑ह॒ दक्षि॑णा॒ दक्षि॑णा॒ ऽऽहे तीत्या॑ह॒ दक्षि॑णा । \newline
5. आ॒ह॒ दक्षि॑णा॒ दक्षि॑णा ऽऽहाह॒ दक्षि॑णा॒ हि हि दक्षि॑णा ऽऽहाह॒ दक्षि॑णा॒ हि । \newline
6. दक्षि॑णा॒ हि हि दक्षि॑णा॒ दक्षि॑णा॒ ह्ये॑षैषा हि दक्षि॑णा॒ दक्षि॑णा॒ ह्ये॑षा । \newline
7. ह्ये॑षैषा हि ह्ये॑षा य॒ज्ञिया॑ य॒ज्ञियै॒षा हि ह्ये॑षा य॒ज्ञिया᳚ । \newline
8. ए॒षा य॒ज्ञिया॑ य॒ज्ञियै॒षैषा य॒ज्ञिया᳚ ऽस्यसि य॒ज्ञियै॒षैषा य॒ज्ञिया॑ ऽसि । \newline
9. य॒ज्ञिया᳚ ऽस्यसि य॒ज्ञिया॑ य॒ज्ञिया॒ ऽसीती त्य॑सि य॒ज्ञिया॑ य॒ज्ञिया॒ ऽसीति॑ । \newline
10. अ॒सीती त्य॑स्य॒ सीत्या॑ हा॒हे त्य॑स्य॒ सीत्या॑ह । \newline
11. इत्या॑हा॒हे तीत्या॑ह य॒ज्ञियां᳚ ॅय॒ज्ञिया॑ मा॒हे तीत्या॑ह य॒ज्ञिया᳚म् । \newline
12. आ॒ह॒ य॒ज्ञियां᳚ ॅय॒ज्ञिया॑ माहाह य॒ज्ञिया॑ मे॒वैव य॒ज्ञिया॑ माहाह य॒ज्ञिया॑ मे॒व । \newline
13. य॒ज्ञिया॑ मे॒वैव य॒ज्ञियां᳚ ॅय॒ज्ञिया॑ मे॒वैना॑ मेना मे॒व य॒ज्ञियां᳚ ॅय॒ज्ञिया॑ मे॒वैना᳚म् । \newline
14. ए॒वैना॑ मेना मे॒वै वैना᳚म् करोति करो त्येना मे॒वै वैना᳚म् करोति । \newline
15. ए॒ना॒म् क॒रो॒ति॒ क॒रो॒ त्ये॒ना॒ मे॒ना॒म् क॒रो॒ति॒ क्ष॒त्रिया᳚ क्ष॒त्रिया॑ करो त्येना मेनाम् करोति क्ष॒त्रिया᳚ । \newline
16. क॒रो॒ति॒ क्ष॒त्रिया᳚ क्ष॒त्रिया॑ करोति करोति क्ष॒त्रिया᳚ ऽस्यसि क्ष॒त्रिया॑ करोति करोति क्ष॒त्रिया॑ ऽसि । \newline
17. क्ष॒त्रिया᳚ ऽस्यसि क्ष॒त्रिया᳚ क्ष॒त्रिया॒ ऽसीती त्य॑सि क्ष॒त्रिया᳚ क्ष॒त्रिया॒ ऽसीति॑ । \newline
18. अ॒सीती त्य॑स्य॒ सीत्या॑हा॒हे त्य॑स्य॒ सीत्या॑ह । \newline
19. इत्या॑हा॒हे तीत्या॑ह क्ष॒त्रिया᳚ क्ष॒त्रिया॒ ऽऽहे तीत्या॑ह क्ष॒त्रिया᳚ । \newline
20. आ॒ह॒ क्ष॒त्रिया᳚ क्ष॒त्रिया॑ ऽऽहाह क्ष॒त्रिया॒ हि हि क्ष॒त्रिया॑ ऽऽहाह क्ष॒त्रिया॒ हि । \newline
21. क्ष॒त्रिया॒ हि हि क्ष॒त्रिया᳚ क्ष॒त्रिया॒ ह्ये॑षैषा हि क्ष॒त्रिया᳚ क्ष॒त्रिया॒ ह्ये॑षा । \newline
22. ह्ये॑षैषा हि ह्ये॑षा ऽदि॑ति॒ रदि॑ति रे॒षा हि ह्ये॑षा ऽदि॑तिः । \newline
23. ए॒षा ऽदि॑ति॒ रदि॑ति रे॒षैषा ऽदि॑ति रस्य॒ स्यदि॑ति रे॒षैषा ऽदि॑तिरसि । \newline
24. अदि॑तिरस्य॒ स्यदि॑ति॒ रदि॑ति रस्युभ॒यत॑श्शी र्ष्ण्युभ॒यत॑श्शी र्ष्ण्यस्यदि॑ति॒ रदि॑ति रस्युभ॒यत॑श्शीर्ष्णी । \newline
25. अ॒स्यु॒भ॒यत॑श्शीर् ष्ण्युभ॒यत॑श्शीर् ष्ण्यस्य स्युभ॒यत॑श्शीर्ष्णी॒ती त्यु॑भ॒यत॑श्शीर् 
ष्ण्यस्य स्युभ॒यत॑श्शी॒र्ष्णीति॑ । \newline
26. उ॒भ॒यत॑श्शी॒र्ष्णी तीत्यु॑भ॒यत॑श्शीर् ष्ण्युभ॒यत॑श्शी॒र्ष्णी त्या॑हा॒हे त्यु॑भ॒यत॑श्शीर् 
ष्ण्युभ॒यत॑श्शी॒र्ष्णी त्या॑ह । \newline
27. उ॒भ॒यत॑श्शी॒र्ष्णीत्यु॑भ॒यतः॑ - शी॒र्ष्णी॒ । \newline
28. इत्या॑हा॒हे तीत्या॑ह॒ यद् यदा॒हे तीत्या॑ह॒ यत् । \newline
29. आ॒ह॒ यद् यदा॑ हाह॒ यदे॒ वैव यदा॑ हाह॒ यदे॒व । \newline
30. यदे॒ वैव यद् यदे॒ वादि॒त्य आ॑दि॒त्य ए॒व यद् यदे॒ वादि॒त्यः । \newline
31. ए॒वादि॒त्य आ॑दि॒त्य ए॒वै वादि॒त्यः प्रा॑य॒णीयः॑ प्राय॒णीय॑ आदि॒त्य ए॒वै वादि॒त्यः प्रा॑य॒णीयः॑ । \newline
32. आ॒दि॒त्यः प्रा॑य॒णीयः॑ प्राय॒णीय॑ आदि॒त्य आ॑दि॒त्यः प्रा॑य॒णीयो॑ य॒ज्ञानां᳚ ॅय॒ज्ञाना᳚म् प्राय॒णीय॑ आदि॒त्य आ॑दि॒त्यः प्रा॑य॒णीयो॑ य॒ज्ञाना᳚म् । \newline
33. प्रा॒य॒णीयो॑ य॒ज्ञानां᳚ ॅय॒ज्ञाना᳚म् प्राय॒णीयः॑ प्राय॒णीयो॑ य॒ज्ञाना॑ मादि॒त्य आ॑दि॒त्यो य॒ज्ञाना᳚म् प्राय॒णीयः॑ प्राय॒णीयो॑ य॒ज्ञाना॑ मादि॒त्यः । \newline
34. प्रा॒य॒णीय॒ इति॑ प्र - अ॒य॒नीयः॑ । \newline
35. य॒ज्ञाना॑ मादि॒त्य आ॑दि॒त्यो य॒ज्ञानां᳚ ॅय॒ज्ञाना॑ मादि॒त्य उ॑दय॒नीय॑ उदय॒नीय॑ आदि॒त्यो य॒ज्ञानां᳚ ॅय॒ज्ञाना॑ मादि॒त्य उ॑दय॒नीयः॑ । \newline
36. आ॒दि॒त्य उ॑दय॒नीय॑ उदय॒नीय॑ आदि॒त्य आ॑दि॒त्य उ॑दय॒नीय॒ स्तस्मा॒त् तस्मा॑ दुदय॒नीय॑ आदि॒त्य आ॑दि॒त्य उ॑दय॒नीय॒ स्तस्मा᳚त् । \newline
37. उ॒द॒य॒नीय॒ स्तस्मा॒त् तस्मा॑ दुदय॒नीय॑ उदय॒नीय॒ स्तस्मा॑ दे॒व मे॒वम् तस्मा॑ दुदय॒नीय॑ उदय॒नीय॒ स्तस्मा॑ दे॒वम् । \newline
38. उ॒द॒य॒नीय॒ इत्यु॑त् - अ॒य॒नीयः॑ । \newline
39. तस्मा॑ दे॒व मे॒वम् तस्मा॒त् तस्मा॑ दे॒व मा॑हा है॒वम् तस्मा॒त् तस्मा॑ दे॒व मा॑ह । \newline
40. ए॒व मा॑हा है॒व मे॒व मा॑ह॒ यद् यदा॑ है॒व मे॒व मा॑ह॒ यत् । \newline
41. आ॒ह॒ यद् यदा॑ हाह॒ यदब॒द्धा ऽब॑द्धा॒ यदा॑हाह॒ यदब॑द्धा । \newline
42. यदब॒द्धा ऽब॑द्धा॒ यद् यदब॑द्धा॒ स्याथ् स्या दब॑द्धा॒ यद् यदब॑द्धा॒ स्यात् । \newline
43. अब॑द्धा॒ स्याथ् स्या दब॒द्धा ऽब॑द्धा॒ स्या दय॒ता ऽय॑ता॒ स्या दब॒द्धा ऽब॑द्धा॒ स्या दय॑ता । \newline
44. स्या दय॒ता ऽय॑ता॒ स्याथ् स्या दय॑ता स्याथ् स्या॒ दय॑ता॒ स्याथ् स्या दय॑ता स्यात् । \newline
45. अय॑ता स्याथ् स्या॒ दय॒ता ऽय॑ता स्या॒द् यद् यथ् स्या॒ दय॒ता ऽय॑ता स्या॒द् यत् । \newline
46. स्या॒द् यद् यथ् स्या᳚थ् स्या॒द् यत् प॑दिब॒द्धा प॑दिब॒द्धा यथ् स्या᳚थ् स्या॒द् यत् प॑दिब॒द्धा । \newline
47. यत् प॑दिब॒द्धा प॑दिब॒द्धा यद् यत् प॑दिब॒द्धा ऽनु॒स्तर॑ ण्यनु॒स्तर॑णी पदिब॒द्धा यद् यत् प॑दिब॒द्धा ऽनु॒स्तर॑णी । \newline
48. प॒दि॒ब॒द्धा ऽनु॒स्तर॑ ण्यनु॒स्तर॑णी पदिब॒द्धा प॑दिब॒द्धा ऽनु॒स्तर॑णी स्याथ् स्यादनु॒स्तर॑णी पदिब॒द्धा प॑दिब॒द्धा ऽनु॒स्तर॑णी स्यात् । \newline
49. प॒दि॒ब॒द्धेति॑ पदि - ब॒द्धा । \newline
50. अ॒नु॒स्तर॑णी स्याथ् स्या दनु॒स्तर॑ ण्यनु॒स्तर॑णी स्यात् प्र॒मायु॑कः प्र॒मायु॑कः स्या दनु॒स्तर॑ ण्यनु॒स्तर॑णी स्यात् प्र॒मायु॑कः । \newline
51. अ॒नु॒स्तर॒णीत्य॑नु - स्तर॑णी । \newline
52. स्या॒त् प्र॒मायु॑कः प्र॒मायु॑कः स्याथ् स्यात् प्र॒मायु॑को॒ यज॑मानो॒ यज॑मानः प्र॒मायु॑कः स्याथ् स्यात् प्र॒मायु॑को॒ यज॑मानः । \newline
53. प्र॒मायु॑को॒ यज॑मानो॒ यज॑मानः प्र॒मायु॑कः प्र॒मायु॑को॒ यज॑मानः स्याथ् स्या॒द् यज॑मानः प्र॒मायु॑कः प्र॒मायु॑को॒ यज॑मानः स्यात् । \newline
54. प्र॒मायु॑क॒ इति॑ प्र - मायु॑कः । \newline
55. यज॑मानः स्याथ् स्या॒द् यज॑मानो॒ यज॑मानः स्या॒द् यद् यथ् स्या॒द् यज॑मानो॒ यज॑मानः स्या॒द् यत् । \newline
56. स्या॒द् यद् यथ् स्या᳚थ् स्या॒द् यत् क॑र्णगृही॒ता क॑र्णगृही॒ता यथ् स्या᳚थ् स्या॒द् यत् क॑र्णगृही॒ता । \newline
\pagebreak
\markright{ TS 6.1.7.6  \hfill https://www.vedavms.in \hfill}

\section{ TS 6.1.7.6 }

\textbf{TS 6.1.7.6 } \newline
\textbf{Samhita Paata} \newline

-द्यत् क॑र्णगृही॒ता वार्त्र॑घ्नी स्या॒थ् स वा॒ऽन्यं जि॑नी॒यात् तं ॅवा॒ऽन्यो जि॑नीयान्मि॒त्रस्त्वा॑ प॒दि ब॑द्ध्ना॒त्वित्या॑ह मि॒त्रो वै शि॒वो दे॒वानां॒ तेनै॒वैनां᳚ प॒दि ब॑द्ध्नाति पू॒षाऽद्ध्व॑नः पा॒त्वित्या॑हे॒यं ॅवै पू॒षेमामे॒वास्या॑ अधि॒पाम॑कः॒ सम॑ष्ट्या॒ इन्द्रा॒या-द्ध्य॑क्षा॒येत्या॒हेन्द्र॑मे॒वास्या॒ अद्ध्य॑क्षं करो॒ - [  ] \newline

\textbf{Pada Paata} \newline

यत् । क॒र्ण॒गृ॒ही॒तेति॑ कर्ण-गृ॒ही॒ता । वार्त्र॒घ्नीति॒ वार्त्र॑ - घ्नी॒ । स्या॒त् । सः । वा॒ । अ॒न्यम् । जि॒नी॒यात् । तम् । वा॒ । अ॒न्यः । जि॒नी॒या॒त् । मि॒त्रः । त्वा॒ । प॒दि । ब॒द्ध्ना॒तु॒ । इति॑ । आ॒ह॒ । मि॒त्रः । वै । शि॒वः । दे॒वाना᳚म् । तेन॑ । ए॒व । ए॒ना॒म् । प॒दि । ब॒द्ध्ना॒ति॒ । पू॒षा । अद्ध्व॑नः । पा॒तु॒ । इति॑ । आ॒ह॒ । इ॒यम् । वै । पू॒षा । इ॒माम् । ए॒व । अ॒स्याः॒ । अ॒धि॒पामित्य॑धि - पाम् । अ॒कः॒ । सम॑ष्ट्या॒ इति॒ सं - अ॒ष्ट्यै॒ । इन्द्रा॑य । अद्ध्य॑क्षा॒येत्यधि॑ - अ॒क्षा॒य॒ । इति॑ । आ॒ह॒ । इन्द्र᳚म् । ए॒व । अ॒स्याः॒ । अद्ध्य॑क्ष॒मित्यधि॑ - अ॒क्ष॒म् । क॒रो॒ति॒ ।  \newline


\textbf{Krama Paata} \newline

यत् क॑र्णगृही॒ता । क॒र्ण॒गृ॒ही॒ता वार्त्र॑घ्नी । क॒र्ण॒गृ॒ही॒तेति॑ कर्ण - गृ॒ही॒ता । वार्त्र॑घ्नी स्यात् । वार्त्र॒घ्नीति॒ वार्त्र॑ - घ्नी॒ । स्या॒थ् सः । स वा᳚ । वा॒ऽन्यम् । अ॒न्यम् जि॑नी॒यात् । जि॒नी॒यात् तम् । तम् ॅवा᳚ । वा॒ऽन्यः । अ॒न्यो जि॑नीयात् । जि॒नी॒या॒न् मि॒त्रः । मि॒त्रस्त्वा᳚ । त्वा॒ प॒दि । प॒दि ब॑द्ध्नातु । ब॒द्ध्ना॒त्विति॑ । इत्या॑ह । आ॒ह॒ मि॒त्रः । मि॒त्रो वै । वै शि॒वः । शि॒वो दे॒वाना᳚म् । दे॒वाना॒म् तेन॑ । तेनै॒व । ए॒वैना᳚म् । ए॒ना॒म् प॒दि । प॒दि ब॑द्ध्नाति । ब॒द्ध्ना॒ति॒ पू॒षा । पू॒षाऽद्ध्व॑नः । अद्ध्व॑नः पातु । पा॒त्विति॑ । इत्या॑ह । आ॒हे॒यम् । इ॒यम् ॅवै । वै पू॒षा । पू॒षेमाम् । इ॒मामे॒व । ए॒वास्याः᳚ । अ॒स्या॒ अ॒धि॒पाम् । अ॒धि॒पाम॑कः । अ॒धि॒पामित्य॑धि - पाम् । अ॒कः॒ सम॑ष्ट्‍यै । सम॑ष्ट्‍या॒ इन्द्रा॑य । सम॑ष्ट्‍या॒ इति॒ सम् - अ॒ष्ट्‍यै॒ । इन्द्रा॒याद्ध्य॑क्षाय । अद्ध्य॑क्षा॒येति॑ । अद्ध्य॑क्षा॒येत्यधि॑ - अ॒क्षा॒य॒ । इत्या॑ह । आ॒हेन्द्र᳚म् । इन्द्र॑मे॒व । ए॒वास्याः᳚ । अ॒स्या॒ अद्ध्य॑क्षम् । अद्ध्य॑क्षम् करोति । अद्ध्य॑क्ष॒मित्यधि॑ - अ॒क्ष॒म् । क॒रो॒त्यनु॑ \newline

\textbf{Jatai Paata} \newline

1. यत् क॑र्णगृही॒ता क॑र्णगृही॒ता यद् यत् क॑र्णगृही॒ता । \newline
2. क॒र्ण॒गृ॒ही॒ता वार्त्र॑घ्नी॒ वार्त्र॑घ्नी कर्णगृही॒ता क॑र्णगृही॒ता वार्त्र॑घ्नी । \newline
3. क॒र्ण॒गृ॒ही॒तेति॑ कर्ण - गृ॒ही॒ता । \newline
4. वार्त्र॑घ्नी स्याथ् स्या॒द् वार्त्र॑घ्नी॒ वार्त्र॑घ्नी स्यात् । \newline
5. वार्त्र॒घ्नीति॒ वार्त्र॑ - घ्नी॒ । \newline
6. स्या॒थ् स स स्या᳚थ् स्या॒थ् सः । \newline
7. स वा॑ वा॒ स स वा᳚ । \newline
8. वा॒ ऽन्य म॒न्यं ॅवा॑ वा॒ ऽन्यम् । \newline
9. अ॒न्यम् जि॑नी॒याज् जि॑नी॒या द॒न्य म॒न्यम् जि॑नी॒यात् । \newline
10. जि॒नी॒यात् तम् तम् जि॑नी॒याज् जि॑नी॒यात् तम् । \newline
11. तं ॅवा॑ वा॒ तम् तं ॅवा᳚ । \newline
12. वा॒ ऽन्यो᳚ ऽन्यो वा॑ वा॒ ऽन्यः । \newline
13. अ॒न्यो जि॑नीयाज् जिनीया द॒न्यो᳚ ऽन्यो जि॑नीयात् । \newline
14. जि॒नी॒या॒न् मि॒त्रो मि॒त्रो जि॑नीयाज् जिनीयान् मि॒त्रः । \newline
15. मि॒त्र स्त्वा᳚ त्वा मि॒त्रो मि॒त्र स्त्वा᳚ । \newline
16. त्वा॒ प॒दि प॒दि त्वा᳚ त्वा प॒दि । \newline
17. प॒दि ब॑द्ध्नातु बद्ध्नातु प॒दि प॒दि ब॑द्ध्नातु । \newline
18. ब॒द्ध्ना॒ त्वितीति॑ बद्ध्नातु बद्ध्ना॒ त्विति॑ । \newline
19. इत्या॑हा॒हे तीत्या॑ह । \newline
20. आ॒ह॒ मि॒त्रो मि॒त्र आ॑हाह मि॒त्रः । \newline
21. मि॒त्रो वै वै मि॒त्रो मि॒त्रो वै । \newline
22. वै शि॒वः शि॒वो वै वै शि॒वः । \newline
23. शि॒वो दे॒वाना᳚म् दे॒वानाꣳ॑ शि॒वः शि॒वो दे॒वाना᳚म् । \newline
24. दे॒वाना॒म् तेन॒ तेन॑ दे॒वाना᳚म् दे॒वाना॒म् तेन॑ । \newline
25. तेनै॒ वैव तेन॒ तेनै॒व । \newline
26. ए॒वैना॑ मेना मे॒वै वैना᳚म् । \newline
27. ए॒ना॒म् प॒दि प॒द्ये॑ना मेनाम् प॒दि । \newline
28. प॒दि ब॑द्ध्नाति बद्ध्नाति प॒दि प॒दि ब॑द्ध्नाति । \newline
29. ब॒द्ध्ना॒ति॒ पू॒षा पू॒षा ब॑द्ध्नाति बद्ध्नाति पू॒षा । \newline
30. पू॒षा ऽद्ध्व॒नो ऽद्ध्व॑नः पू॒षा पू॒षा ऽद्ध्व॑नः । \newline
31. अद्ध्व॑नः पातु पा॒त्व द्ध्व॒नो ऽद्ध्व॑नः पातु । \newline
32. पा॒त्वि तीति॑ पातु पा॒त्विति॑ । \newline
33. इत्या॑हा॒हे तीत्या॑ह । \newline
34. आ॒हे॒य मि॒य मा॑हा हे॒यम् । \newline
35. इ॒यं ॅवै वा इ॒य मि॒यं ॅवै । \newline
36. वै पू॒षा पू॒षा वै वै पू॒षा । \newline
37. पू॒षेमा मि॒माम् पू॒षा पू॒षेमाम् । \newline
38. इ॒मा मे॒वैवे मा मि॒मा मे॒व । \newline
39. ए॒वास्या॑ अस्या ए॒वै वास्याः᳚ । \newline
40. अ॒स्या॒ अ॒धि॒पा म॑धि॒पा म॑स्या अस्या अधि॒पाम् । \newline
41. अ॒धि॒पा म॑क रक रधि॒पा म॑धि॒पा म॑कः । \newline
42. अ॒धि॒पामित्य॑धि - पाम् । \newline
43. अ॒कः॒ सम॑ष्ट्यै॒ सम॑ष्ट्या अक रकः॒ सम॑ष्ट्यै । \newline
44. सम॑ष्ट्या॒ इन्द्रा॒ येन्द्रा॑य॒ सम॑ष्ट्यै॒ सम॑ष्ट्या॒ इन्द्रा॑य । \newline
45. सम॑ष्ट्या॒ इति॒ सं - अ॒ष्ट्यै॒ । \newline
46. इन्द्रा॒या द्ध्य॑क्षा॒या द्ध्य॑क्षा॒ येन्द्रा॒ येन्द्रा॒या द्ध्य॑क्षाय । \newline
47. अद्ध्य॑क्षा॒ये तीत्य द्ध्य॑क्षा॒या द्ध्य॑क्षा॒येति॑ । \newline
48. अद्ध्य॑क्षा॒येत्यधि॑ - अ॒क्षा॒य॒ । \newline
49. इत्या॑हा॒हे तीत्या॑ह । \newline
50. आ॒हेन्द्र॒ मिन्द्र॑ माहा॒ हेन्द्र᳚म् । \newline
51. इन्द्र॑ मे॒वैवेन्द्र॒ मिन्द्र॑ मे॒व । \newline
52. ए॒वास्या॑ अस्या ए॒वै वास्याः᳚ । \newline
53. अ॒स्या॒ अद्ध्य॑क्ष॒ मद्ध्य॑क्ष मस्या अस्या॒ अद्ध्य॑क्षम् । \newline
54. अद्ध्य॑क्षम् करोति करो॒ त्यद्ध्य॑क्ष॒ मद्ध्य॑क्षम् करोति । \newline
55. अद्ध्य॑क्ष॒मित्यधि॑ - अ॒क्ष॒म् । \newline
56. क॒रो॒ त्यन्वनु॑ करोति करो॒ त्यनु॑ । \newline

\textbf{Ghana Paata } \newline

1. यत् क॑र्णगृही॒ता क॑र्णगृही॒ता यद् यत् क॑र्णगृही॒ता वार्त्र॑घ्नी॒ वार्त्र॑घ्नी कर्णगृही॒ता यद् यत् क॑र्णगृही॒ता वार्त्र॑घ्नी । \newline
2. क॒र्ण॒गृ॒ही॒ता वार्त्र॑घ्नी॒ वार्त्र॑घ्नी कर्णगृही॒ता क॑र्णगृही॒ता वार्त्र॑घ्नी स्याथ् स्या॒द् वार्त्र॑घ्नी कर्णगृही॒ता क॑र्णगृही॒ता वार्त्र॑घ्नी स्यात् । \newline
3. क॒र्ण॒गृ॒ही॒तेति॑ कर्ण - गृ॒ही॒ता । \newline
4. वार्त्र॑घ्नी स्याथ् स्या॒द् वार्त्र॑घ्नी॒ वार्त्र॑घ्नी स्या॒थ् स स स्या॒द् वार्त्र॑घ्नी॒ वार्त्र॑घ्नी स्या॒थ् सः । \newline
5. वार्त्र॒घ्नीति॒ वार्त्र॑ - घ्नी॒ । \newline
6. स्या॒थ् स स स्या᳚थ् स्या॒थ् स वा॑ वा॒ स स्या᳚थ् स्या॒थ् स वा᳚ । \newline
7. स वा॑ वा॒ स स वा॒ ऽन्य म॒न्यं ॅवा॒ स स वा॒ ऽन्यम् । \newline
8. वा॒ ऽन्य म॒न्यं ॅवा॑ वा॒ ऽन्यम् जि॑नी॒याज् जि॑नी॒या द॒न्यं ॅवा॑ वा॒ ऽन्यम् जि॑नी॒यात् । \newline
9. अ॒न्यम् जि॑नी॒याज् जि॑नी॒या द॒न्य म॒न्यम् जि॑नी॒यात् तम् तम् जि॑नी॒या द॒न्य म॒न्यम् जि॑नी॒यात् तम् । \newline
10. जि॒नी॒यात् तम् तम् जि॑नी॒याज् जि॑नी॒यात् तं ॅवा॑ वा॒ तम् जि॑नी॒याज् जि॑नी॒यात् तं ॅवा᳚ । \newline
11. तं ॅवा॑ वा॒ तम् तं ॅवा॒ ऽन्यो᳚ ऽन्यो वा॒ तम् तं ॅवा॒ ऽन्यः । \newline
12. वा॒ ऽन्यो᳚ ऽन्यो वा॑ वा॒ ऽन्यो जि॑नीयाज् जिनीया द॒न्यो वा॑ वा॒ ऽन्यो जि॑नीयात् । \newline
13. अ॒न्यो जि॑नीयाज् जिनीया द॒न्यो᳚ ऽन्यो जि॑नीयान् मि॒त्रो मि॒त्रो जि॑नीया द॒न्यो᳚ ऽन्यो जि॑नीयान् मि॒त्रः । \newline
14. जि॒नी॒या॒न् मि॒त्रो मि॒त्रो जि॑नीयाज् जिनीयान् मि॒त्र स्त्वा᳚ त्वा मि॒त्रो जि॑नीयाज् जिनीयान् मि॒त्र स्त्वा᳚ । \newline
15. मि॒त्र स्त्वा᳚ त्वा मि॒त्रो मि॒त्र स्त्वा॑ प॒दि प॒दि त्वा॑ मि॒त्रो मि॒त्र स्त्वा॑ प॒दि । \newline
16. त्वा॒ प॒दि प॒दि त्वा᳚ त्वा प॒दि ब॑द्ध्नातु बद्ध्नातु प॒दि त्वा᳚ त्वा प॒दि ब॑द्ध्नातु । \newline
17. प॒दि ब॑द्ध्नातु बद्ध्नातु प॒दि प॒दि ब॑द्ध्ना॒त्वि तीति॑ बद्ध्नातु प॒दि प॒दि ब॑द्ध्ना॒ त्विति॑ । \newline
18. ब॒द्ध्ना॒ त्वितीति॑ बद्ध्नातु बद्ध्ना॒ त्वित्या॑हा॒हेति॑ बद्ध्नातु बद्ध्ना॒ त्वित्या॑ह । \newline
19. इत्या॑हा॒हे तीत्या॑ह मि॒त्रो मि॒त्र आ॒हे तीत्या॑ह मि॒त्रः । \newline
20. आ॒ह॒ मि॒त्रो मि॒त्र आ॑हाह मि॒त्रो वै वै मि॒त्र आ॑हाह मि॒त्रो वै । \newline
21. मि॒त्रो वै वै मि॒त्रो मि॒त्रो वै शि॒वः शि॒वो वै मि॒त्रो मि॒त्रो वै शि॒वः । \newline
22. वै शि॒वः शि॒वो वै वै शि॒वो दे॒वाना᳚म् दे॒वानाꣳ॑ शि॒वो वै वै शि॒वो दे॒वाना᳚म् । \newline
23. शि॒वो दे॒वाना᳚म् दे॒वानाꣳ॑ शि॒वः शि॒वो दे॒वाना॒म् तेन॒ तेन॑ दे॒वानाꣳ॑ शि॒वः शि॒वो दे॒वाना॒म् तेन॑ । \newline
24. दे॒वाना॒म् तेन॒ तेन॑ दे॒वाना᳚म् दे॒वाना॒म् तेनै॒ वैव तेन॑ दे॒वाना᳚म् दे॒वाना॒म् तेनै॒व । \newline
25. तेनै॒ वैव तेन॒ तेनै॒ वैना॑ मेना मे॒व तेन॒ तेनै॒ वैना᳚म् । \newline
26. ए॒वैना॑ मेना मे॒वै वैना᳚म् प॒दि प॒द्ये॑ना मे॒वै वैना᳚म् प॒दि । \newline
27. ए॒ना॒म् प॒दि प॒द्ये॑ना मेनाम् प॒दि ब॑द्ध्नाति बद्ध्नाति प॒द्ये॑ना मेनाम् प॒दि ब॑द्ध्नाति । \newline
28. प॒दि ब॑द्ध्नाति बद्ध्नाति प॒दि प॒दि ब॑द्ध्नाति पू॒षा पू॒षा ब॑द्ध्नाति प॒दि प॒दि ब॑द्ध्नाति पू॒षा । \newline
29. ब॒द्ध्ना॒ति॒ पू॒षा पू॒षा ब॑द्ध्नाति बद्ध्नाति पू॒षा ऽद्ध्व॒नो ऽद्ध्व॑नः पू॒षा ब॑द्ध्नाति बद्ध्नाति पू॒षा ऽद्ध्व॑नः । \newline
30. पू॒षा ऽद्ध्व॒नो ऽद्ध्व॑नः पू॒षा पू॒षा ऽद्ध्व॑नः पातु पा॒त्वद्ध्व॑नः पू॒षा पू॒षा ऽद्ध्व॑नः पातु । \newline
31. अद्ध्व॑नः पातु पा॒त्वद्ध्व॒नो ऽद्ध्व॑नः पा॒त्वितीति॑ पा॒त्वद्ध्व॒नो ऽद्ध्व॑नः पा॒त्विति॑ । \newline
32. पा॒त्वि तीति॑ पातु पा॒त्वित्या॑ हा॒हेति॑ पातु पा॒त्वि त्या॑ह । \newline
33. इत्या॑हा॒हे तीत्या॑हे॒य मि॒य मा॒हे तीत्या॑हे॒यम् । \newline
34. आ॒हे॒य मि॒य मा॑हाहे॒यं ॅवै वा इ॒य मा॑हाहे॒यं ॅवै । \newline
35. इ॒यं ॅवै वा इ॒य मि॒यं ॅवै पू॒षा पू॒षा वा इ॒य मि॒यं ॅवै पू॒षा । \newline
36. वै पू॒षा पू॒षा वै वै पू॒षेमा मि॒माम् पू॒षा वै वै पू॒षेमाम् । \newline
37. पू॒षेमा मि॒माम् पू॒षा पू॒षेमा मे॒वैवे माम् पू॒षा पू॒षेमा मे॒व । \newline
38. इ॒मा मे॒वैवे मा मि॒मा मे॒वास्या॑ अस्या ए॒वे मा मि॒मा मे॒वास्याः᳚ । \newline
39. ए॒वास्या॑ अस्या ए॒वै वास्या॑ अधि॒पा म॑धि॒पा म॑स्या ए॒वै वास्या॑ अधि॒पाम् । \newline
40. अ॒स्या॒ अ॒धि॒पा म॑धि॒पा म॑स्या अस्या अधि॒पा म॑क रक रधि॒पा म॑स्या अस्या अधि॒पा म॑कः । \newline
41. अ॒धि॒पा म॑क रक रधि॒पा म॑धि॒पा म॑कः॒ सम॑ष्ट्यै॒ सम॑ष्ट्या अक रधि॒पा म॑धि॒पा म॑कः॒ सम॑ष्ट्यै । \newline
42. अ॒धि॒पामित्य॑धि - पाम् । \newline
43. अ॒कः॒ सम॑ष्ट्यै॒ सम॑ष्ट्या अक रकः॒ सम॑ष्ट्या॒ इन्द्रा॒ येन्द्रा॑य॒ सम॑ष्ट्या अक रकः॒ सम॑ष्ट्या॒ इन्द्रा॑य । \newline
44. सम॑ष्ट्या॒ इन्द्रा॒ येन्द्रा॑य॒ सम॑ष्ट्यै॒ सम॑ष्ट्या॒ इन्द्रा॒या द्ध्य॑क्षा॒या द्ध्य॑क्षा॒ येन्द्रा॑य॒ सम॑ष्ट्यै॒ सम॑ष्ट्या॒ इन्द्रा॒या द्ध्य॑क्षाय । \newline
45. सम॑ष्ट्या॒ इति॒ सं - अ॒ष्ट्यै॒ । \newline
46. इन्द्रा॒या द्ध्य॑क्षा॒या द्ध्य॑क्षा॒ येन्द्रा॒ येन्द्रा॒या द्ध्य॑क्षा॒ये तीत्यद्ध्य॑क्षा॒ येन्द्रा॒ येन्द्रा॒या द्ध्य॑क्षा॒येति॑ । \newline
47. अद्ध्य॑क्षा॒ये तीत्यद्ध्य॑क्षा॒या द्ध्य॑क्षा॒ येत्या॑हा॒हे त्यद्ध्य॑क्षा॒या द्ध्य॑क्षा॒ये त्या॑ह । \newline
48. अद्ध्य॑क्षा॒येत्यधि॑ - अ॒क्षा॒य॒ । \newline
49. इत्या॑हा॒हे तीत्या॒हेन्द्र॒ मिन्द्र॑ मा॒हे तीत्या॒हेन्द्र᳚म् । \newline
50. आ॒हेन्द्र॒ मिन्द्र॑ माहा॒हेन्द्र॑ मे॒वैवेन्द्र॑ माहा॒हेन्द्र॑ मे॒व । \newline
51. इन्द्र॑ मे॒वैवेन्द्र॒ मिन्द्र॑ मे॒वास्या॑ अस्या ए॒वेन्द्र॒ मिन्द्र॑ मे॒वास्याः᳚ । \newline
52. ए॒वास्या॑ अस्या ए॒वै वास्या॒ अद्ध्य॑क्ष॒ मद्ध्य॑क्ष मस्या ए॒वै वास्या॒ अद्ध्य॑क्षम् । \newline
53. अ॒स्या॒ अद्ध्य॑क्ष॒ मद्ध्य॑क्ष मस्या अस्या॒ अद्ध्य॑क्षम् करोति करो॒ त्यद्ध्य॑क्ष मस्या अस्या॒ अद्ध्य॑क्षम् करोति । \newline
54. अद्ध्य॑क्षम् करोति करो॒ त्यद्ध्य॑क्ष॒ मद्ध्य॑क्षम् करो॒ त्यन्वनु॑ करो॒ त्यद्ध्य॑क्ष॒ मद्ध्य॑क्षम् करो॒ त्यनु॑ । \newline
55. अद्ध्य॑क्ष॒मित्यधि॑ - अ॒क्ष॒म् । \newline
56. क॒रो॒ त्यन्वनु॑ करोति करो॒ त्यनु॑ त्वा॒ त्वा ऽनु॑ करोति करो॒ त्यनु॑ त्वा । \newline
\pagebreak
\markright{ TS 6.1.7.7  \hfill https://www.vedavms.in \hfill}

\section{ TS 6.1.7.7 }

\textbf{TS 6.1.7.7 } \newline
\textbf{Samhita Paata} \newline

-त्यनु॑ त्वा मा॒ता म॑न्यता॒मनु॑ पि॒तेत्या॒हा-नु॑मतयै॒वैन॑या क्रीणाति॒ सा दे॑वि दे॒वमच्छे॒हीत्या॑ह दे॒वी ह्ये॑षा दे॒वः सोम॒ इन्द्रा॑य॒ सोम॒मित्या॒हेन्द्रा॑य॒ हि सोम॑ आह्रि॒यते॒ यदे॒तद्-यजु॒र्न ब्रू॒यात् परा᳚च्ये॒व सो॑म॒क्रय॑णीयाद्-रु॒द्रस्त्वाऽऽ व॑र्तय॒त्वित्या॑ह रु॒द्रो वै क्रू॒रो - [  ] \newline

\textbf{Pada Paata} \newline

अन्विति॑ । त्वा॒ । मा॒ता । म॒न्य॒ता॒म् । अन्विति॑ । पि॒ता । इति॑ । आ॒ह॒ । अनु॑मत॒येत्यनु॑ - म॒त॒या॒ । ए॒व । ए॒न॒या॒ । क्री॒णा॒ति॒ । सा । दे॒वि॒ । दे॒वम् । अच्छ॑ । इ॒हि॒ । इति॑ । आ॒ह॒ । दे॒वी । हि । ए॒षा । दे॒वः । सोमः॑ । इन्द्रा॑य । सोम᳚म् । इति॑ । आ॒ह॒ । इन्द्रा॑य । हि । सोमः॑ । आ॒ह्रि॒यत॒ इत्या᳚ - ह्रि॒यते᳚ । यत् । ए॒तत् । यजुः॑ । न । ब्रू॒यात् । परा॑ची । ए॒व । सो॒म॒क्रय॒णीति॑ सोम - क्रय॑णी । इ॒या॒त् । रु॒द्रः । त्वा॒ । एति॑ । व॒र्त॒य॒तु॒ । इति॑ । आ॒ह॒ । रु॒द्रः । वै । क्रू॒रः ।  \newline


\textbf{Krama Paata} \newline

अनु॑ त्वा । त्वा॒ मा॒ता । मा॒ता म॑न्यताम् । म॒न्य॒ता॒मनु॑ । अनु॑ पि॒ता । पि॒तेति॑ । इत्या॑ह । आ॒हानु॑मतया । अनु॑मतयै॒व । अनु॑मत॒येत्यनु॑ - म॒त॒या॒ । ए॒वैन॑या । ए॒न॒या॒ क्री॒णा॒ति॒ । क्री॒णा॒ति॒ सा । सा दे॑वि । दे॒वि॒ दे॒वम् । दे॒वमच्छ॑ । अच्छे॑हि । इ॒हीति॑ । इत्या॑ह । आ॒ह॒ दे॒वी । दे॒वी हि । ह्ये॑षा । ए॒षा दे॒वः । दे॒वः सोमः॑ । सोम॒ इन्द्रा॑य । इन्द्रा॑य॒ सोम᳚म् । सोम॒मिति॑ । इत्या॑ह । आ॒हेन्द्रा॑य । इन्द्रा॑य॒ हि । हि सोमः॑ । सोम॑ आह्रि॒यते᳚ । आ॒ह्रि॒यते॒ यत् । आ॒ह्रि॒यत॒ इत्या᳚ - ह्रि॒यते᳚ । यदे॒तत् । ए॒तद् यजुः॑ । यजु॒र् न । न ब्रू॒यात् । ब्रू॒यात् परा॑ची । परा᳚च्ये॒व । ए॒व सो॑म॒क्रय॑णी । सो॒म॒क्रय॑णीयात् । सो॒म॒क्रय॒णीति॑ सोम - क्रय॑णी । इ॒या॒द् रु॒द्रः । रु॒द्रस्त्वा᳚ । त्वा । आ व॑र्तयतु । व॒र्त॒य॒त्विति॑ । इत्या॑ह । आ॒ह॒ रु॒द्रः । रु॒द्रो वै । वै क्रू॒रः । क्रू॒रो दे॒वाना᳚म् \newline

\textbf{Jatai Paata} \newline

1. अनु॑ त्वा त्वा॒ ऽन्वनु॑ त्वा । \newline
2. त्वा॒ मा॒ता मा॒ता त्वा᳚ त्वा मा॒ता । \newline
3. मा॒ता म॑न्यताम् मन्यताम् मा॒ता मा॒ता म॑न्यताम् । \newline
4. म॒न्य॒ता॒ मन्वनु॑ मन्यताम् मन्यता॒ मनु॑ । \newline
5. अनु॑ पि॒ता पि॒ता ऽन्वनु॑ पि॒ता । \newline
6. पि॒तेतीति॑ पि॒ता पि॒तेति॑ । \newline
7. इत्या॑हा॒हे तीत्या॑ह । \newline
8. आ॒हा नु॑मत॒या ऽनु॑मतया ऽऽहा॒हा नु॑मतया । \newline
9. अनु॑मत यै॒वै वानु॑मत॒या ऽनु॑मत यै॒व । \newline
10. अनु॑मत॒येत्यनु॑ - म॒त॒या॒ । \newline
11. ए॒वैन॑ यैन यै॒वै वैन॑या । \newline
12. ए॒न॒या॒ क्री॒णा॒ति॒ क्री॒णा॒ त्ये॒न॒ यै॒न॒या॒ क्री॒णा॒ति॒ । \newline
13. क्री॒णा॒ति॒ सा सा क्री॑णाति क्रीणाति॒ सा । \newline
14. सा दे॑वि देवि॒ सा सा दे॑वि । \newline
15. दे॒वि॒ दे॒वम् दे॒वम् दे॑वि देवि दे॒वम् । \newline
16. दे॒व मच्छा च्छ॑ दे॒वम् दे॒व मच्छ॑ । \newline
17. अच्छे॑ ही॒ह्य च्छा च्छे॑हि । \newline
18. इ॒ही तीती॑ही॒ हीति॑ । \newline
19. इत्या॑हा॒हे तीत्या॑ह । \newline
20. आ॒ह॒ दे॒वी दे॒व्या॑ हाह दे॒वी । \newline
21. दे॒वी हि हि दे॒वी दे॒वी हि । \newline
22. ह्ये॑षैषा हि ह्ये॑षा । \newline
23. ए॒षा दे॒वो दे॒व ए॒षैषा दे॒वः । \newline
24. दे॒वः सोमः॒ सोमो॑ दे॒वो दे॒वः सोमः॑ । \newline
25. सोम॒ इन्द्रा॒ येन्द्रा॑य॒ सोमः॒ सोम॒ इन्द्रा॑य । \newline
26. इन्द्रा॑य॒ सोमꣳ॒॒ सोम॒ मिन्द्रा॒ येन्द्रा॑य॒ सोम᳚म् । \newline
27. सोम॒ मितीति॒ सोमꣳ॒॒ सोम॒ मिति॑ । \newline
28. इत्या॑हा॒हे तीत्या॑ह । \newline
29. आ॒हेन्द्रा॒ येन्द्रा॑या हा॒हेन्द्रा॑य । \newline
30. इन्द्रा॑य॒ हि हीन्द्रा॒ येन्द्रा॑य॒ हि । \newline
31. हि सोमः॒ सोमो॒ हि हि सोमः॑ । \newline
32. सोम॑ आह्रि॒यत॑ आह्रि॒यते॒ सोमः॒ सोम॑ आह्रि॒यते᳚ । \newline
33. आ॒ह्रि॒यते॒ यद् यदा᳚ह्रि॒यत॑ आह्रि॒यते॒ यत् । \newline
34. आ॒ह्रि॒यत॒ इत्या᳚ - ह्रि॒यते᳚ । \newline
35. यदे॒त दे॒तद् यद् यदे॒तत् । \newline
36. ए॒तद् यजु॒र् यजु॑ रे॒त दे॒तद् यजुः॑ । \newline
37. यजु॒र् न न यजु॒र् यजु॒र् न । \newline
38. न ब्रू॒याद् ब्रू॒यान् न न ब्रू॒यात् । \newline
39. ब्रू॒यात् परा॑ची॒ परा॑ची ब्रू॒याद् ब्रू॒यात् परा॑ची । \newline
40. परा᳚च्ये॒ वैव परा॑ची॒ परा᳚ च्ये॒व । \newline
41. ए॒व सो॑म॒क्रय॑णी सोम॒क्रय॑ण्ये॒ वैव सो॑म॒क्रय॑णी । \newline
42. सो॒म॒क्रय॑णीया दियाथ् सोम॒क्रय॑णी सोम॒क्रय॑ णीयात् । \newline
43. सो॒म॒क्रय॒णीति॑ सोम - क्रय॑णी । \newline
44. इ॒या॒द् रु॒द्रो रु॒द्र इ॑या दियाद् रु॒द्रः । \newline
45. रु॒द्र स्त्वा᳚ त्वा रु॒द्रो रु॒द्र स्त्वा᳚ । \newline
46. त्वा ऽऽत्वा॒ त्वा । \newline
47. आ व॑र्तयतु वर्तय॒त्वा व॑र्तयतु । \newline
48. व॒र्त॒य॒त्वि तीति॑ वर्तयतु वर्तय॒ त्विति॑ । \newline
49. इत्या॑हा॒हे तीत्या॑ह । \newline
50. आ॒ह॒ रु॒द्रो रु॒द्र आ॑हाह रु॒द्रः । \newline
51. रु॒द्रो वै वै रु॒द्रो रु॒द्रो वै । \newline
52. वै क्रू॒रः क्रू॒रो वै वै क्रू॒रः । \newline
53. क्रू॒रो दे॒वाना᳚म् दे॒वाना᳚म् क्रू॒रः क्रू॒रो दे॒वाना᳚म् । \newline

\textbf{Ghana Paata } \newline

1. अनु॑ त्वा त्वा॒ ऽन्वनु॑ त्वा मा॒ता मा॒ता त्वा॒ ऽन्वनु॑ त्वा मा॒ता । \newline
2. त्वा॒ मा॒ता मा॒ता त्वा᳚ त्वा मा॒ता म॑न्यताम् मन्यताम् मा॒ता त्वा᳚ त्वा मा॒ता म॑न्यताम् । \newline
3. मा॒ता म॑न्यताम् मन्यताम् मा॒ता मा॒ता म॑न्यता॒ मन्वनु॑ मन्यताम् मा॒ता मा॒ता म॑न्यता॒ मनु॑ । \newline
4. म॒न्य॒ता॒ मन्वनु॑ मन्यताम् मन्यता॒ मनु॑ पि॒ता पि॒ता ऽनु॑ मन्यताम् मन्यता॒ मनु॑ पि॒ता । \newline
5. अनु॑ पि॒ता पि॒ता ऽन्वनु॑ पि॒ते तीति॑ पि॒ता ऽन्वनु॑ पि॒तेति॑ । \newline
6. पि॒तेतीति॑ पि॒ता पि॒तेत्या॑ हा॒हेति॑ पि॒ता पि॒तेत्या॑ह । \newline
7. इत्या॑हा॒हे तीत्या॒हा नु॑मत॒या ऽनु॑मतया॒ ऽऽहे तीत्या॒हा नु॑मतया । \newline
8. आ॒हा नु॑मत॒या ऽनु॑मतया ऽऽहा॒हानु॑मत यै॒वै वानु॑मतया ऽऽहा॒हा नु॑मत यै॒व । \newline
9. अनु॑मत यै॒वै वानु॑मत॒या ऽनु॑मत यै॒वैन॑ यैनयै॒ वानु॑मत॒या ऽनु॑मतयै॒ वैन॑या । \newline
10. अनु॑मत॒येत्यनु॑ - म॒त॒या॒ । \newline
11. ए॒वैन॑ यैन यै॒वैवैन॑या क्रीणाति क्रीणा त्येन यै॒वैवैन॑या क्रीणाति । \newline
12. ए॒न॒या॒ क्री॒णा॒ति॒ क्री॒णा॒ त्ये॒न॒ यै॒न॒या॒ क्री॒णा॒ति॒ सा सा क्री॑णा त्येन यैनया क्रीणाति॒ सा । \newline
13. क्री॒णा॒ति॒ सा सा क्री॑णाति क्रीणाति॒ सा दे॑वि देवि॒ सा क्री॑णाति क्रीणाति॒ सा दे॑वि । \newline
14. सा दे॑वि देवि॒ सा सा दे॑वि दे॒वम् दे॒वम् दे॑वि॒ सा सा दे॑वि दे॒वम् । \newline
15. दे॒वि॒ दे॒वम् दे॒वम् दे॑वि देवि दे॒व मच्छाच्छ॑ दे॒वम् दे॑वि देवि दे॒व मच्छ॑ । \newline
16. दे॒व मच्छाच्छ॑ दे॒वम् दे॒व मच्छे॑ ही॒ह्यच्छ॑ दे॒वम् दे॒व मच्छे॑हि । \newline
17. अच्छे॑ही॒ ह्यच्छाच्छे॒ हीतीती॒ ह्यच्छाच्छे॒ हीति॑ । \newline
18. इ॒हीती ती॑ही॒ही त्या॑हा॒हे ती॑ही॒ही त्या॑ह । \newline
19. इत्या॑हा॒हे तीत्या॑ह दे॒वी दे॒व्या॑हे तीत्या॑ह दे॒वी । \newline
20. आ॒ह॒ दे॒वी दे॒व्या॑हाह दे॒वी हि हि दे॒व्या॑हाह दे॒वी हि । \newline
21. दे॒वी हि हि दे॒वी दे॒वी ह्ये॑षैषा हि दे॒वी दे॒वी ह्ये॑षा । \newline
22. ह्ये॑षैषा हि ह्ये॑षा दे॒वो दे॒व ए॒षा हि ह्ये॑षा दे॒वः । \newline
23. ए॒षा दे॒वो दे॒व ए॒षैषा दे॒वः सोमः॒ सोमो॑ दे॒व ए॒षैषा दे॒वः सोमः॑ । \newline
24. दे॒वः सोमः॒ सोमो॑ दे॒वो दे॒वः सोम॒ इन्द्रा॒ येन्द्रा॑य॒ सोमो॑ दे॒वो दे॒वः सोम॒ इन्द्रा॑य । \newline
25. सोम॒ इन्द्रा॒ येन्द्रा॑य॒ सोमः॒ सोम॒ इन्द्रा॑य॒ सोमꣳ॒॒ सोम॒ मिन्द्रा॑य॒ सोमः॒ सोम॒ इन्द्रा॑य॒ सोम᳚म् । \newline
26. इन्द्रा॑य॒ सोमꣳ॒॒ सोम॒ मिन्द्रा॒ येन्द्रा॑य॒ सोम॒ मितीति॒ सोम॒ मिन्द्रा॒ येन्द्रा॑य॒ सोम॒ मिति॑ । \newline
27. सोम॒ मितीति॒ सोमꣳ॒॒ सोम॒ मित्या॑हा॒हेति॒ सोमꣳ॒॒ सोम॒ मित्या॑ह । \newline
28. इत्या॑हा॒हे तीत्या॒ हेन्द्रा॒ येन्द्रा॑या॒हे तीत्या॒ हेन्द्रा॑य । \newline
29. आ॒हेन्द्रा॒ येन्द्रा॑या हा॒हेन्द्रा॑य॒ हि हीन्द्रा॑या हा॒हेन्द्रा॑य॒ हि । \newline
30. इन्द्रा॑य॒ हि हीन्द्रा॒ येन्द्रा॑य॒ हि सोमः॒ सोमो॒ हीन्द्रा॒ येन्द्रा॑य॒ हि सोमः॑ । \newline
31. हि सोमः॒ सोमो॒ हि हि सोम॑ आह्रि॒यत॑ आह्रि॒यते॒ सोमो॒ हि हि सोम॑ आह्रि॒यते᳚ । \newline
32. सोम॑ आह्रि॒यत॑ आह्रि॒यते॒ सोमः॒ सोम॑ आह्रि॒यते॒ यद् यदा᳚ह्रि॒यते॒ सोमः॒ सोम॑ आह्रि॒यते॒ यत् । \newline
33. आ॒ह्रि॒यते॒ यद् यदा᳚ह्रि॒यत॑ आह्रि॒यते॒ यदे॒त दे॒तद् यदा᳚ह्रि॒यत॑ आह्रि॒यते॒ यदे॒तत् । \newline
34. आ॒ह्रि॒यत॒ इत्या᳚ - ह्रि॒यते᳚ । \newline
35. यदे॒त दे॒तद् यद् यदे॒तद् यजु॒र् यजु॑ रे॒तद् यद् यदे॒तद् यजुः॑ । \newline
36. ए॒तद् यजु॒र् यजु॑ रे॒त दे॒तद् यजु॒र् न न यजु॑ रे॒त दे॒तद् यजु॒र् न । \newline
37. यजु॒र् न न यजु॒र् यजु॒र् न ब्रू॒याद् ब्रू॒यान् न यजु॒र् यजु॒र् न ब्रू॒यात् । \newline
38. न ब्रू॒याद् ब्रू॒यान् न न ब्रू॒यात् परा॑ची॒ परा॑ची ब्रू॒यान् न न ब्रू॒यात् परा॑ची । \newline
39. ब्रू॒यात् परा॑ची॒ परा॑ची ब्रू॒याद् ब्रू॒यात् परा᳚च्ये॒ वैव परा॑ची ब्रू॒याद् ब्रू॒यात् परा᳚च्ये॒व । \newline
40. परा᳚च्ये॒ वैव परा॑ची॒ परा᳚च्ये॒व सो॑म॒क्रय॑णी सोम॒क्रय॑ ण्ये॒व परा॑ची॒ परा᳚च्ये॒व सो॑म॒क्रय॑णी । \newline
41. ए॒व सो॑म॒क्रय॑णी सोम॒क्रय॑ ण्ये॒वैव सो॑म॒क्रय॑णीया दियाथ् सोम॒क्रय॑ ण्ये॒वैव सो॑म॒क्रय॑णीयात् । \newline
42. सो॒म॒क्रय॑णीया दियाथ् सोम॒क्रय॑णी सोम॒क्रय॑णीयाद् रु॒द्रो रु॒द्र इ॑याथ् सोम॒क्रय॑णी सोम॒क्रय॑णीयाद् रु॒द्रः । \newline
43. सो॒म॒क्रय॒णीति॑ सोम - क्रय॑णी । \newline
44. इ॒या॒द् रु॒द्रो रु॒द्र इ॑यादियाद् रु॒द्र स्त्वा᳚ त्वा रु॒द्र इ॑या दियाद् रु॒द्र स्त्वा᳚ । \newline
45. रु॒द्र स्त्वा᳚ त्वा रु॒द्रो रु॒द्र स्त्वा ऽऽत्वा॑ रु॒द्रो रु॒द्र स्त्वा । \newline
46. त्वा ऽऽत्वा॒ त्वा ऽऽव॑र्तयतु वर्तय॒त्वा त्वा॒ त्वा ऽऽव॑र्तयतु । \newline
47. आ व॑र्तयतु वर्तय॒त्वा व॑र्तय॒त्वि तीति॑ वर्तय॒त्वा व॑र्तय॒त्विति॑ । \newline
48. व॒र्त॒य॒त्वि तीति॑ वर्तयतु वर्तय॒त्वि त्या॑हा॒हेति॑ वर्तयतु वर्तय॒त्वि त्या॑ह । \newline
49. इत्या॑हा॒हे तीत्या॑ह रु॒द्रो रु॒द्र आ॒हे तीत्या॑ह रु॒द्रः । \newline
50. आ॒ह॒ रु॒द्रो रु॒द्र आ॑हाह रु॒द्रो वै वै रु॒द्र आ॑हाह रु॒द्रो वै । \newline
51. रु॒द्रो वै वै रु॒द्रो रु॒द्रो वै क्रू॒रः क्रू॒रो वै रु॒द्रो रु॒द्रो वै क्रू॒रः । \newline
52. वै क्रू॒रः क्रू॒रो वै वै क्रू॒रो दे॒वाना᳚म् दे॒वाना᳚म् क्रू॒रो वै वै क्रू॒रो दे॒वाना᳚म् । \newline
53. क्रू॒रो दे॒वाना᳚म् दे॒वाना᳚म् क्रू॒रः क्रू॒रो दे॒वाना॒म् तम् तम् दे॒वाना᳚म् क्रू॒रः क्रू॒रो दे॒वाना॒म् तम् । \newline
\pagebreak
\markright{ TS 6.1.7.8  \hfill https://www.vedavms.in \hfill}

\section{ TS 6.1.7.8 }

\textbf{TS 6.1.7.8 } \newline
\textbf{Samhita Paata} \newline

दे॒वानां॒ तमे॒वास्यै॑ प॒रस्ता᳚द्-दधा॒त्यावृ॑त्त्यै क्रू॒रमि॑व॒ वा ए॒तत् क॑रोति॒ यद्-रु॒द्रस्य॑ की॒र्तय॑ति मि॒त्रस्य॑ प॒थेत्या॑ह॒ शान्त्यै॑ वा॒चा वा ए॒ष वि क्री॑णीते॒ यः सो॑म॒क्रय॑ण्या स्व॒स्ति सोम॑सखा॒ पुन॒रेहि॑ स॒ह र॒य्येत्या॑ह वा॒चैव वि॒क्रीय॒ पुन॑रा॒त्मन् वाचं॑ ध॒त्तेऽनु॑पदासुकाऽस्य॒ वाग्भ॑वति॒ य ए॒वं ॅवेद॑ ( ) ॥ \newline

\textbf{Pada Paata} \newline

दे॒वाना᳚म् । तम् । ए॒व । अ॒स्यै॒ । प॒रस्ता᳚त् । द॒धा॒ति॒ । आवृ॑त्त्या॒ इत्या - वृ॒त्त्यै॒ । क्रू॒रम् । इ॒व॒ । वै । ए॒तत् । क॒रो॒ति॒ । यत् । रु॒द्रस्य॑ । की॒र्तय॑ति । मि॒त्रस्य॑ । प॒था । इति॑ । आ॒ह॒ । शान्त्यै᳚ । वा॒चा । वै । ए॒षः । वीति॑ । क्री॒णी॒ते॒ । यः । सो॒म॒क्रय॒ण्येति॑ सोम - क्रय॑ण्या । स्व॒स्ति । सोम॑स॒खेति॒ सोम॑ - स॒खा॒ । पुनः॑ । एति॑ । इ॒हि॒ । स॒ह । र॒य्या । इति॑ । आ॒ह॒ । वा॒चा । ए॒व । वि॒क्रीयेति॑ वि - क्रीय॑ । पुनः॑ । आ॒त्मन् । वाच᳚म् । ध॒त्ते॒ । अनु॑पदासु॒केत्यनु॑प - दा॒सु॒का॒ । अ॒स्य॒ । वाक् । भ॒व॒ति॒ । यः । ए॒वम् । वेद॑ ( ) ॥  \newline


\textbf{Krama Paata} \newline

दे॒वाना॒म् तम् । तमे॒व । ए॒वास्यै᳚ । अ॒स्यै॒ प॒रस्ता᳚त् । प॒रस्ता᳚द् दधाति । द॒धा॒त्यावृ॑त्त्यै । आवृ॑त्त्यै क्रू॒रम् । आवृ॑त्त्या॒ इत्या - वृ॒त्त्यै॒ । क्रू॒रमि॑व । इ॒व॒ वै । वा ए॒तत् । ए॒तत् क॑रोति । क॒रो॒ति॒ यत् । यद् रु॒द्रस्य॑ । 
रु॒द्रस्य॑ की॒र्तय॑ति । की॒र्तय॑ति मि॒त्रस्य॑ । मि॒त्रस्य॑ प॒था । प॒थेति॑ । इत्या॑ह । आ॒ह॒ शान्त्यै᳚ । शान्त्यै॑ वा॒चा । वा॒चा वै । वा ए॒षः । ए॒ष वि । वि क्री॑णीते । क्री॒णी॒ते॒ यः । यः सो॑म॒क्रय॑ण्या । सो॒म॒क्रय॑ण्या स्व॒स्ति । 
सो॒म॒क्रय॒ण्येति॑ सोम - क्रय॑ण्या । स्व॒स्ति सोम॑सखा । सोम॑सखा॒ पुनः॑ । सोम॑स॒खेति॒ सोम॑ - स॒खा॒ । पुन॒रा । एहि॑ । इ॒हि॒ स॒ह । स॒ह र॒य्या । र॒य्येति॑ । इत्या॑ह । आ॒ह॒ वा॒चा । वा॒चैव । ए॒व वि॒क्रीय॑ । वि॒क्रीय॒ पुनः॑ । वि॒क्रीयेति॑ वि - क्रीय॑ । पुन॑रा॒त्मन्न् । आ॒त्मन् वाच᳚म् । वाच॑म् धत्ते । ध॒त्तेऽनु॑पदासुका । अनु॑पदासुकाऽस्य । अनु॑पदासु॒केत्यनु॑प - दा॒सु॒का॒ । अ॒स्य॒ वाक् । वाग् भ॑वति । भ॒व॒ति॒ यः । य ए॒वम् । ए॒वम् ॅवेद॑ ( ) । वेदेति॒ वेद॑ । \newline

\textbf{Jatai Paata} \newline

1. दे॒वाना॒म् तम् तम् दे॒वाना᳚म् दे॒वाना॒म् तम् । \newline
2. त मे॒वैव तम् त मे॒व । \newline
3. ए॒वास्या॑ अस्या ए॒वै वास्यै᳚ । \newline
4. अ॒स्यै॒ प॒रस्ता᳚त् प॒रस्ता॑ दस्या अस्यै प॒रस्ता᳚त् । \newline
5. प॒रस्ता᳚द् दधाति दधाति प॒रस्ता᳚त् प॒रस्ता᳚द् दधाति । \newline
6. द॒धा॒त्या वृ॑त्त्या॒ आवृ॑त्त्यै दधाति दधा॒त्या वृ॑त्त्यै । \newline
7. आवृ॑त्त्यै क्रू॒रम् क्रू॒र मावृ॑त्त्या॒ आवृ॑त्त्यै क्रू॒रम् । \newline
8. आवृ॑त्त्या॒ इत्या - वृ॒त्त्यै॒ । \newline
9. क्रू॒र मि॑वेव क्रू॒रम् क्रू॒र मि॑व । \newline
10. इ॒व॒ वै वा इ॑वे व॒ वै । \newline
11. वा ए॒त दे॒तद् वै वा ए॒तत् । \newline
12. ए॒तत् क॑रोति करो त्ये॒त दे॒तत् क॑रोति । \newline
13. क॒रो॒ति॒ यद् यत् क॑रोति करोति॒ यत् । \newline
14. यद् रु॒द्रस्य॑ रु॒द्रस्य॒ यद् यद् रु॒द्रस्य॑ । \newline
15. रु॒द्रस्य॑ की॒र्तय॑ति की॒र्तय॑ति रु॒द्रस्य॑ रु॒द्रस्य॑ की॒र्तय॑ति । \newline
16. की॒र्तय॑ति मि॒त्रस्य॑ मि॒त्रस्य॑ की॒र्तय॑ति की॒र्तय॑ति मि॒त्रस्य॑ । \newline
17. मि॒त्रस्य॑ प॒था प॒था मि॒त्रस्य॑ मि॒त्रस्य॑ प॒था । \newline
18. प॒थेतीति॑ प॒था प॒थेति॑ । \newline
19. इत्या॑हा॒हे तीत्या॑ह । \newline
20. आ॒ह॒ शान्त्यै॒ शान्त्या॑ आहाह॒ शान्त्यै᳚ । \newline
21. शान्त्यै॑ वा॒चा वा॒चा शान्त्यै॒ शान्त्यै॑ वा॒चा । \newline
22. वा॒चा वै वै वा॒चा वा॒चा वै । \newline
23. वा ए॒ष ए॒ष वै वा ए॒षः । \newline
24. ए॒ष वि व्ये॑ष ए॒ष वि । \newline
25. वि क्री॑णीते क्रीणीते॒ वि वि क्री॑णीते । \newline
26. क्री॒णी॒ते॒ यो यः क्री॑णीते क्रीणीते॒ यः । \newline
27. यः सो॑म॒क्रय॑ण्या सोम॒क्रय॑ण्या॒ यो यः सो॑म॒क्रय॑ण्या । \newline
28. सो॒म॒क्रय॑ण्या स्व॒स्ति स्व॒स्ति सो॑म॒क्रय॑ण्या सोम॒क्रय॑ण्या स्व॒स्ति । \newline
29. सो॒म॒क्रय॒ण्येति॑ सोम - क्रय॑ण्या । \newline
30. स्व॒स्ति सोम॑सखा॒ सोम॑सखा स्व॒स्ति स्व॒स्ति सोम॑सखा । \newline
31. सोम॑सखा॒ पुनः॒ पुनः॒ सोम॑सखा॒ सोम॑सखा॒ पुनः॑ । \newline
32. सोम॑स॒खेति॒ सोम॑ - स॒खा॒ । \newline
33. पुन॒ रा पुनः॒ पुन॒ रा । \newline
34. एही॒ ह्ये हि॑ । \newline
35. इ॒हि॒ स॒ह स॒हे ही॑हि स॒ह । \newline
36. स॒ह र॒य्या र॒य्या स॒ह स॒ह र॒य्या । \newline
37. र॒य्येतीति॑ र॒य्या र॒य्येति॑ । \newline
38. इत्या॑हा॒हे तीत्या॑ह । \newline
39. आ॒ह॒ वा॒चा वा॒चा ऽऽहा॑ह वा॒चा । \newline
40. वा॒चैवैव वा॒चा वा॒चैव । \newline
41. ए॒व वि॒क्रीय॑ वि॒क्री यै॒वैव वि॒क्रीय॑ । \newline
42. वि॒क्रीय॒ पुनः॒ पुन॑र् वि॒क्रीय॑ वि॒क्रीय॒ पुनः॑ । \newline
43. वि॒क्रीयेति॑ वि - क्रीय॑ । \newline
44. पुन॑ रा॒त्मन् ना॒त्मन् पुनः॒ पुन॑ रा॒त्मन्न् । \newline
45. आ॒त्मन्. वाचं॒ ॅवाच॑ मा॒त्मन् ना॒त्मन्. वाच᳚म् । \newline
46. वाच॑म् धत्ते धत्ते॒ वाचं॒ ॅवाच॑म् धत्ते । \newline
47. ध॒त्ते ऽनु॑पदासु॒का ऽनु॑पदासुका धत्ते ध॒त्ते ऽनु॑पदासुका । \newline
48. अनु॑पदासुका ऽस्या॒स्या नु॑पदासु॒का ऽनु॑पदासुका ऽस्य । \newline
49. अनु॑पदासु॒केत्यनु॑प - दा॒सु॒का॒ । \newline
50. अ॒स्य॒ वाग् वाग॑ स्यास्य॒ वाक् । \newline
51. वाग् भ॑वति भवति॒ वाग् वाग् भ॑वति । \newline
52. भ॒व॒ति॒ यो यो भ॑वति भवति॒ यः । \newline
53. य ए॒व मे॒वं ॅयो य ए॒वम् । \newline
54. ए॒वं ॅवेद॒ वेदै॒व मे॒वं ॅवेद॑ । \newline
55. वेदेति॒ वेद॑ । \newline

\textbf{Ghana Paata } \newline

1. दे॒वाना॒म् तम् तम् दे॒वाना᳚म् दे॒वाना॒म् त मे॒वैव तम् दे॒वाना᳚म् दे॒वाना॒म् त मे॒व । \newline
2. त मे॒वैव तम् त मे॒वास्या॑ अस्या ए॒व तम् त मे॒वास्यै᳚ । \newline
3. ए॒वास्या॑ अस्या ए॒वै वास्यै॑ प॒रस्ता᳚त् प॒रस्ता॑ दस्या ए॒वै वास्यै॑ प॒रस्ता᳚त् । \newline
4. अ॒स्यै॒ प॒रस्ता᳚त् प॒रस्ता॑ दस्या अस्यै प॒रस्ता᳚द् दधाति दधाति प॒रस्ता॑ दस्या अस्यै प॒रस्ता᳚द् दधाति । \newline
5. प॒रस्ता᳚द् दधाति दधाति प॒रस्ता᳚त् प॒रस्ता᳚द् दधा॒ त्यावृ॑त्त्या॒ आवृ॑त्त्यै दधाति प॒रस्ता᳚त् प॒रस्ता᳚द् दधा॒ त्यावृ॑त्त्यै । \newline
6. द॒धा॒ त्यावृ॑त्त्या॒ आवृ॑त्त्यै दधाति दधा॒ त्यावृ॑त्त्यै क्रू॒रम् क्रू॒र मावृ॑त्त्यै दधाति दधा॒ त्यावृ॑त्त्यै क्रू॒रम् । \newline
7. आवृ॑त्त्यै क्रू॒रम् क्रू॒र मावृ॑त्त्या॒ आवृ॑त्त्यै क्रू॒र मि॑वेव क्रू॒र मावृ॑त्त्या॒ आवृ॑त्त्यै क्रू॒र मि॑व । \newline
8. आवृ॑त्त्या॒ इत्या - वृ॒त्त्यै॒ । \newline
9. क्रू॒र मि॑वेव क्रू॒रम् क्रू॒र मि॑व॒ वै वा इ॑व क्रू॒रम् क्रू॒र मि॑व॒ वै । \newline
10. इ॒व॒ वै वा इ॑वेव॒ वा ए॒त दे॒तद् वा इ॑वेव॒ वा ए॒तत् । \newline
11. वा ए॒त दे॒तद् वै वा ए॒तत् क॑रोति करो त्ये॒तद् वै वा ए॒तत् क॑रोति । \newline
12. ए॒तत् क॑रोति करो त्ये॒त दे॒तत् क॑रोति॒ यद् यत् क॑रो त्ये॒त दे॒तत् क॑रोति॒ यत् । \newline
13. क॒रो॒ति॒ यद् यत् क॑रोति करोति॒ यद् रु॒द्रस्य॑ रु॒द्रस्य॒ यत् क॑रोति करोति॒ यद् रु॒द्रस्य॑ । \newline
14. यद् रु॒द्रस्य॑ रु॒द्रस्य॒ यद् यद् रु॒द्रस्य॑ की॒र्तय॑ति की॒र्तय॑ति रु॒द्रस्य॒ यद् यद् रु॒द्रस्य॑ की॒र्तय॑ति । \newline
15. रु॒द्रस्य॑ की॒र्तय॑ति की॒र्तय॑ति रु॒द्रस्य॑ रु॒द्रस्य॑ की॒र्तय॑ति मि॒त्रस्य॑ मि॒त्रस्य॑ की॒र्तय॑ति रु॒द्रस्य॑ रु॒द्रस्य॑ की॒र्तय॑ति मि॒त्रस्य॑ । \newline
16. की॒र्तय॑ति मि॒त्रस्य॑ मि॒त्रस्य॑ की॒र्तय॑ति की॒र्तय॑ति मि॒त्रस्य॑ प॒था प॒था मि॒त्रस्य॑ की॒र्तय॑ति की॒र्तय॑ति मि॒त्रस्य॑ प॒था । \newline
17. मि॒त्रस्य॑ प॒था प॒था मि॒त्रस्य॑ मि॒त्रस्य॑ प॒थेतीति॑ प॒था मि॒त्रस्य॑ मि॒त्रस्य॑ प॒थेति॑ । \newline
18. प॒थेतीति॑ प॒था प॒थे त्या॑हा॒हेति॑ प॒था प॒थे त्या॑ह । \newline
19. इत्या॑हा॒हे तीत्या॑ह॒ शान्त्यै॒ शान्त्या॑ आ॒हे तीत्या॑ह॒ शान्त्यै᳚ । \newline
20. आ॒ह॒ शान्त्यै॒ शान्त्या॑ आहाह॒ शान्त्यै॑ वा॒चा वा॒चा शान्त्या॑ आहाह॒ शान्त्यै॑ वा॒चा । \newline
21. शान्त्यै॑ वा॒चा वा॒चा शान्त्यै॒ शान्त्यै॑ वा॒चा वै वै वा॒चा शान्त्यै॒ शान्त्यै॑ वा॒चा वै । \newline
22. वा॒चा वै वै वा॒चा वा॒चा वा ए॒ष ए॒ष वै वा॒चा वा॒चा वा ए॒षः । \newline
23. वा ए॒ष ए॒ष वै वा ए॒ष वि व्ये॑ष वै वा ए॒ष वि । \newline
24. ए॒ष वि व्ये॑ष ए॒ष वि क्री॑णीते क्रीणीते॒ व्ये॑ष ए॒ष वि क्री॑णीते । \newline
25. वि क्री॑णीते क्रीणीते॒ वि वि क्री॑णीते॒ यो यः क्री॑णीते॒ वि वि क्री॑णीते॒ यः । \newline
26. क्री॒णी॒ते॒ यो यः क्री॑णीते क्रीणीते॒ यः सो॑म॒क्रय॑ण्या सोम॒क्रय॑ण्या॒ यः क्री॑णीते क्रीणीते॒ यः सो॑म॒क्रय॑ण्या । \newline
27. यः सो॑म॒क्रय॑ण्या सोम॒क्रय॑ण्या॒ यो यः सो॑म॒क्रय॑ण्या स्व॒स्ति स्व॒स्ति सो॑म॒क्रय॑ण्या॒ यो यः सो॑म॒क्रय॑ण्या स्व॒स्ति । \newline
28. सो॒म॒क्रय॑ण्या स्व॒स्ति स्व॒स्ति सो॑म॒क्रय॑ण्या सोम॒क्रय॑ण्या स्व॒स्ति सोम॑सखा॒ सोम॑सखा स्व॒स्ति सो॑म॒क्रय॑ण्या सोम॒क्रय॑ण्या स्व॒स्ति सोम॑सखा । \newline
29. सो॒म॒क्रय॒ण्येति॑ सोम - क्रय॑ण्या । \newline
30. स्व॒स्ति सोम॑सखा॒ सोम॑सखा स्व॒स्ति स्व॒स्ति सोम॑सखा॒ पुनः॒ पुनः॒ सोम॑सखा स्व॒स्ति स्व॒स्ति सोम॑सखा॒ पुनः॑ । \newline
31. सोम॑सखा॒ पुनः॒ पुनः॒ सोम॑सखा॒ सोम॑सखा॒ पुन॒ रा पुनः॒ सोम॑सखा॒ सोम॑सखा॒ पुन॒ रा । \newline
32. सोम॑स॒खेति॒ सोम॑ - स॒खा॒ । \newline
33. पुन॒ रा पुनः॒ पुन॒ रेही॒ह्या पुनः॒ पुन॒ रेहि॑ । \newline
34. एही॒ह्येहि॑ स॒ह स॒हे ह्येहि॑ स॒ह । \newline
35. इ॒हि॒ स॒ह स॒हे ही॑हि स॒ह र॒य्या र॒य्या स॒हे ही॑हि स॒ह र॒य्या । \newline
36. स॒ह र॒य्या र॒य्या स॒ह स॒ह र॒य्येतीति॑ र॒य्या स॒ह स॒ह र॒य्येति॑ । \newline
37. र॒य्येतीति॑ र॒य्या र॒य्ये त्या॑हा॒हेति॑ र॒य्या र॒य्ये त्या॑ह । \newline
38. इत्या॑हा॒हे तीत्या॑ह वा॒चा वा॒चा ऽऽहे तीत्या॑ह वा॒चा । \newline
39. आ॒ह॒ वा॒चा वा॒चा ऽऽहा॑ह वा॒चै वैव वा॒चा ऽऽहा॑ह वा॒चैव । \newline
40. वा॒चै वैव वा॒चा वा॒चैव वि॒क्रीय॑ वि॒क्रीयै॒व वा॒चा वा॒चैव वि॒क्रीय॑ । \newline
41. ए॒व वि॒क्रीय॑ वि॒क्रीयै॒ वैव वि॒क्रीय॒ पुनः॒ पुन॑र् वि॒क्रीयै॒ वैव वि॒क्रीय॒ पुनः॑ । \newline
42. वि॒क्रीय॒ पुनः॒ पुन॑र् वि॒क्रीय॑ वि॒क्रीय॒ पुन॑ रा॒त्मन् ना॒त्मन् पुन॑र् वि॒क्रीय॑ वि॒क्रीय॒ पुन॑ रा॒त्मन्न् । \newline
43. वि॒क्रीयेति॑ वि - क्रीय॑ । \newline
44. पुन॑ रा॒त्मन् ना॒त्मन् पुनः॒ पुन॑ रा॒त्मन्. वाचं॒ ॅवाच॑ मा॒त्मन् पुनः॒ पुन॑ रा॒त्मन्. वाच᳚म् । \newline
45. आ॒त्मन्. वाचं॒ ॅवाच॑ मा॒त्मन् ना॒त्मन्. वाच॑म् धत्ते धत्ते॒ वाच॑ मा॒त्मन् ना॒त्मन्. वाच॑म् धत्ते । \newline
46. वाच॑म् धत्ते धत्ते॒ वाचं॒ ॅवाच॑म् ध॒त्ते ऽनु॑पदासु॒का ऽनु॑पदासुका धत्ते॒ वाचं॒ ॅवाच॑म् ध॒त्ते ऽनु॑पदासुका । \newline
47. ध॒त्ते ऽनु॑पदासु॒का ऽनु॑पदासुका धत्ते ध॒त्ते ऽनु॑पदासुका ऽस्या॒स्या नु॑पदासुका धत्ते ध॒त्ते ऽनु॑पदासुका ऽस्य । \newline
48. अनु॑पदासुका ऽस्या॒स्या नु॑पदासु॒का ऽनु॑पदासुका ऽस्य॒ वाग् वाग॒स्या नु॑पदासु॒का ऽनु॑पदासुका ऽस्य॒ वाक् । \newline
49. अनु॑पदासु॒केत्यनु॑प - दा॒सु॒का॒ । \newline
50. अ॒स्य॒ वाग् वाग॑ स्यास्य॒ वाग् भ॑वति भवति॒ वाग॑ स्यास्य॒ वाग् भ॑वति । \newline
51. वाग् भ॑वति भवति॒ वाग् वाग् भ॑वति॒ यो यो भ॑वति॒ वाग् वाग् भ॑वति॒ यः । \newline
52. भ॒व॒ति॒ यो यो भ॑वति भवति॒ य ए॒व मे॒वं ॅयो भ॑वति भवति॒ य ए॒वम् । \newline
53. य ए॒व मे॒वं ॅयो य ए॒वं ॅवेद॒ वेदै॒वं ॅयो य ए॒वं ॅवेद॑ । \newline
54. ए॒वं ॅवेद॒ वेदै॒व मे॒वं ॅवेद॑ । \newline
55. वेदेति॒ वेद॑ । \newline
\pagebreak
\markright{ TS 6.1.8.1  \hfill https://www.vedavms.in \hfill}

\section{ TS 6.1.8.1 }

\textbf{TS 6.1.8.1 } \newline
\textbf{Samhita Paata} \newline

षट् प॒दान्यनु॒ नि क्रा॑मति षड॒हं ॅवाङ्नाति॑ वदत्यु॒त सं॑ॅवथ्स॒रस्याय॑ने॒ याव॑त्ये॒व वाक्तामव॑ रुन्धे सप्त॒मे प॒दे जु॑होति स॒प्तप॑दा॒ शक्व॑री प॒शवः॒ शक्व॑री प॒शूने॒वाव॑ रुन्धे स॒प्त ग्रा॒म्याः प॒शवः॑ स॒प्ता*ऽऽर॒ण्याः स॒प्त छन्दाꣳ॑-स्यु॒भय॒स्या-व॑रुद्ध्यै॒ वस्व्य॑सि रु॒द्राऽसीत्या॑ह रू॒पमे॒वास्या॑ ए॒तन् म॑हि॒मानं॒ - [  ] \newline

\textbf{Pada Paata} \newline

षट् । प॒दानि॑ । अनु॑ । नीति॑ । क्रा॒म॒ति॒ । ष॒ड॒हमिति॑ षट् - अ॒हम् । वाक् । न । अतीति॑ । व॒द॒ति॒ । उ॒त । सं॒ॅव॒थ्स॒रस्येति॑ सं-व॒थ्स॒रस्य॑ । अय॑ने । याव॑ती । ए॒व । वाक् । ताम् । अवेति॑ । रु॒न्धे॒ । स॒प्त॒मे । प॒दे । जु॒हो॒ति॒ । स॒प्तप॒देति॑ स॒प्त - प॒दा॒ । शक्व॑री । प॒शवः॑ । शक्व॑री । प॒शून् । ए॒व । अवेति॑ । रु॒न्धे॒ । स॒प्त । ग्रा॒म्याः । प॒शवः॑ । स॒प्त । आ॒र॒ण्याः । स॒प्त । छन्दाꣳ॑सि । उ॒भय॑स्य । अव॑रुद्ध्या॒ इत्यव॑ - रु॒द्ध्यै॒ । वस्वी᳚ । अ॒सि॒ । रु॒द्रा । अ॒सि॒ । इति॑ । आ॒ह॒ । रू॒पम् । ए॒व । अ॒स्याः॒ । ए॒तत् । म॒हि॒मान᳚म् ।  \newline


\textbf{Krama Paata} \newline

षट् प॒दानि॑ । प॒दान्यनु॑ । अनु॒ नि । नि क्रा॑मति । क्रा॒म॒ति॒ ष॒ड॒हम् । ष॒ड॒हम् ॅवाक् । ष॒ड॒हमिति॑ षट् - अ॒हम् । वाङ्‍ न । नाति॑ । अति॑ वदति । व॒द॒त्यु॒त । उ॒त स॑म्ॅवथ्स॒रस्य॑ । स॒म्ॅव॒थ्स॒रस्याय॑ने । स॒म्ॅव॒थ्स॒रस्येति॑ सम् - व॒थ्स॒रस्य॑ । अय॑ने॒ याव॑ती । याव॑त्ये॒व । ए॒व वाक् । वाक् ताम् । तामव॑ । अव॑ रुन्धे । रु॒न्धे॒ स॒प्त॒मे । स॒प्त॒मे प॒दे । प॒दे जु॑होति । जु॒हो॒ति॒ स॒प्तप॑दा । स॒प्तप॑दा॒ शक्व॑री । स॒प्तप॒देति॑ स॒प्त - प॒दा॒ । शक्व॑री प॒शवः॑ । प॒शवः॒ शक्व॑री । शक्व॑री प॒शून् । प॒शूने॒व । ए॒वाव॑ । अव॑ रुन्धे । रु॒न्धे॒ स॒प्त । स॒प्त ग्रा॒म्याः । ग्रा॒म्याः प॒शवः॑ । प॒शवः॑ स॒प्त । स॒प्तार॒ण्याः । आ॒र॒ण्याः स॒प्त । स॒प्त छन्दाꣳ॑सि । छन्दाꣳ॑स्यु॒भय॑स्य । उ॒भय॒स्याव॑रुद्ध्यै । अव॑रुद्ध्यै॒ वस्वी᳚ । अव॑रुद्ध्या॒ इत्यव॑-रु॒द्ध्यै॒ । वस्व्य॑सि । अ॒सि॒ रु॒द्रा । रु॒द्राऽसि॑ । अ॒सीति॑ । इत्या॑ह । आ॒ह॒ रू॒पम् । रू॒पमे॒व । ए॒वास्याः᳚ । अ॒स्या॒ ए॒तत् । ए॒तन् म॑हि॒मान᳚म् । म॒हि॒मान॒म् ॅव्याच॑ष्टे \newline

\textbf{Jatai Paata} \newline

1. षट् प॒दानि॑ प॒दानि॒ षट् थ्षट् प॒दानि॑ । \newline
2. प॒दा न्यन् वनु॑ प॒दानि॑ प॒दा न्यनु॑ । \newline
3. अनु॒ नि न्यन् वनु॒ नि । \newline
4. नि क्रा॑मति क्रामति॒ नि नि क्रा॑मति । \newline
5. क्रा॒म॒ति॒ ष॒ड॒हꣳ ष॑ड॒हम् क्रा॑मति क्रामति षड॒हम् । \newline
6. ष॒ड॒हं ॅवाग् वाख् ष॑ड॒हꣳ ष॑ड॒हं ॅवाक् । \newline
7. ष॒ड॒हमिति॑ षट् - अ॒हम् । \newline
8. वाङ् न न वाग् वाङ् न । \newline
9. नात्यति॒ न नाति॑ । \newline
10. अति॑ वदति वद॒ त्य त्यति॑ वदति । \newline
11. व॒द॒ त्यु॒तोत व॑दति वद त्यु॒त । \newline
12. उ॒त सं॑ॅवथ्स॒रस्य॑ संॅवथ्स॒र स्यो॒तोत सं॑ॅवथ्स॒रस्य॑ । \newline
13. सं॒ॅव॒थ्स॒रस्याय॒ने ऽय॑ने संॅवथ्स॒रस्य॑ संॅवथ्स॒रस्याय॑ने । \newline
14. सं॒ॅव॒थ्स॒रस्येति॑ सं - व॒थ्स॒रस्य॑ । \newline
15. अय॑ने॒ याव॑ती॒ याव॒ त्यय॒ने ऽय॑ने॒ याव॑ती । \newline
16. याव॑ त्ये॒वैव याव॑ती॒ याव॑ त्ये॒व । \newline
17. ए॒व वाग् वागे॒ वैव वाक् । \newline
18. वाक् ताम् तां ॅवाग् वाक् ताम् । \newline
19. ता मवाव॒ ताम् ता मव॑ । \newline
20. अव॑ रुन्धे रु॒न्धे ऽवाव॑ रुन्धे । \newline
21. रु॒न्धे॒ स॒प्त॒मे स॑प्त॒मे रु॑न्धे रुन्धे सप्त॒मे । \newline
22. स॒प्त॒मे प॒दे प॒दे स॑प्त॒मे स॑प्त॒मे प॒दे । \newline
23. प॒दे जु॑होति जुहोति प॒दे प॒दे जु॑होति । \newline
24. जु॒हो॒ति॒ स॒प्तप॑दा स॒प्तप॑दा जुहोति जुहोति स॒प्तप॑दा । \newline
25. स॒प्तप॑दा॒ शक्व॑री॒ शक्व॑री स॒प्तप॑दा स॒प्तप॑दा॒ शक्व॑री । \newline
26. स॒प्तप॒देति॑ स॒प्त - प॒दा॒ । \newline
27. शक्व॑री प॒शवः॑ प॒शवः॒ शक्व॑री॒ शक्व॑री प॒शवः॑ । \newline
28. प॒शवः॒ शक्व॑री॒ शक्व॑री प॒शवः॑ प॒शवः॒ शक्व॑री । \newline
29. शक्व॑री प॒शून् प॒शूञ् छक्व॑री॒ शक्व॑री प॒शून् । \newline
30. प॒शूने॒वैव प॒शून् प॒शूने॒व । \newline
31. ए॒वावा वै॒वै वाव॑ । \newline
32. अव॑ रुन्धे रु॒न्धे ऽवाव॑ रुन्धे । \newline
33. रु॒न्धे॒ स॒प्त स॒प्त रु॑न्धे रुन्धे स॒प्त । \newline
34. स॒प्त ग्रा॒म्या ग्रा॒म्याः स॒प्त स॒प्त ग्रा॒म्याः । \newline
35. ग्रा॒म्याः प॒शवः॑ प॒शवो᳚ ग्रा॒म्या ग्रा॒म्याः प॒शवः॑ । \newline
36. प॒शवः॑ स॒प्त स॒प्त प॒शवः॑ प॒शवः॑ स॒प्त । \newline
37. स॒प्तार॒ण्या आ॑र॒ण्याः स॒प्त स॒प्तार॒ण्याः । \newline
38. आ॒र॒ण्याः स॒प्त स॒प्ता र॒ण्या आ॑र॒ण्याः स॒प्त । \newline
39. स॒प्त छन्दाꣳ॑सि॒ छन्दाꣳ॑सि स॒प्त स॒प्त छन्दाꣳ॑सि । \newline
40. छन्दाꣳ॑ स्यु॒भय॑ स्यो॒भय॑स्य॒ छन्दाꣳ॑सि॒ छन्दाꣳ॑ स्यु॒भय॑स्य । \newline
41. उ॒भय॒स्या व॑रुद्ध्या॒ अव॑रुद्ध्या उ॒भय॑स्यो॒ भय॒स्या व॑रुद्ध्यै । \newline
42. अव॑रुद्ध्यै॒ वस्वी॒ वस्व्यव॑रुद्ध्या॒ अव॑रुद्ध्यै॒ वस्वी᳚ । \newline
43. अव॑रुद्ध्या॒ इत्यव॑ - रु॒द्ध्यै॒ । \newline
44. वस्व्य॑ स्यसि॒ वस्वी॒ वस्व्य॑सि । \newline
45. अ॒सि॒ रु॒द्रा रु॒द्रा ऽस्य॑सि रु॒द्रा । \newline
46. रु॒द्रा ऽस्य॑सि रु॒द्रा रु॒द्रा ऽसि॑ । \newline
47. अ॒सीती त्य॑स्य॒ सीति॑ । \newline
48. इत्या॑हा॒हे तीत्या॑ह । \newline
49. आ॒ह॒ रू॒पꣳ रू॒प मा॑हाह रू॒पम् । \newline
50. रू॒प मे॒वैव रू॒पꣳ रू॒प मे॒व । \newline
51. ए॒वास्या॑ अस्या ए॒वै वास्याः᳚ । \newline
52. अ॒स्या॒ ए॒त दे॒तद॑स्या अस्या ए॒तत् । \newline
53. ए॒तन् म॑हि॒मान॑म् महि॒मान॑ मे॒त दे॒तन् म॑हि॒मान᳚म् । \newline
54. म॒हि॒मानं॒ ॅव्याच॑ष्टे॒ व्याच॑ष्टे महि॒मान॑म् महि॒मानं॒ ॅव्याच॑ष्टे । \newline

\textbf{Ghana Paata } \newline

1. षट् प॒दानि॑ प॒दानि॒ षट् थ्षट् प॒दा न्यन् वनु॑ प॒दानि॒ षट् थ्षट् प॒दा न्यनु॑ । \newline
2. प॒दा न्यन्वनु॑ प॒दानि॑ प॒दा न्यनु॒ नि न्यनु॑ प॒दानि॑ प॒दा न्यनु॒ नि । \newline
3. अनु॒ नि न्यन् वनु॒ नि क्रा॑मति क्रामति॒ न्यन् वनु॒ नि क्रा॑मति । \newline
4. नि क्रा॑मति क्रामति॒ नि नि क्रा॑मति षड॒हꣳ ष॑ड॒हम् क्रा॑मति॒ नि नि क्रा॑मति षड॒हम् । \newline
5. क्रा॒म॒ति॒ ष॒ड॒हꣳ ष॑ड॒हम् क्रा॑मति क्रामति षड॒हं ॅवाग् वाख् ष॑ड॒हम् क्रा॑मति क्रामति षड॒हं ॅवाक् । \newline
6. ष॒ड॒हं ॅवाग् वाख् ष॑ड॒हꣳ ष॑ड॒हं ॅवाङ् न न वाख् ष॑ड॒हꣳ ष॑ड॒हं ॅवाङ् न । \newline
7. ष॒ड॒हमिति॑ षट् - अ॒हम् । \newline
8. वाङ् न न वाग् वाङ् नात्यति॒ न वाग् वाङ् नाति॑ । \newline
9. नात्यति॒ न नाति॑ वदति वद॒त्यति॒ न नाति॑ वदति । \newline
10. अति॑ वदति वद॒ त्यत्यति॑ वद त्यु॒तोत व॑द॒ त्यत्यति॑ वद त्यु॒त । \newline
11. व॒द॒ त्यु॒तोत व॑दति वद त्यु॒त सं॑ॅवथ्स॒रस्य॑ संॅवथ्स॒र स्यो॒त व॑दति वदत्यु॒त सं॑ॅवथ्स॒रस्य॑ । \newline
12. उ॒त सं॑ॅवथ्स॒रस्य॑ संॅवथ्स॒र स्यो॒तोत सं॑ॅवथ्स॒रस्या य॒ने ऽय॑ने संॅवथ्स॒र स्यो॒तोत सं॑ॅवथ्स॒र स्याय॑ने । \newline
13. सं॒ॅव॒थ्स॒र स्याय॒ने ऽय॑ने संॅवथ्स॒रस्य॑ संॅवथ्स॒र स्याय॑ने॒ याव॑ती॒ याव॒ त्यय॑ने संॅवथ्स॒रस्य॑ संॅवथ्स॒र स्याय॑ने॒ याव॑ती । \newline
14. सं॒ॅव॒थ्स॒रस्येति॑ सं - व॒थ्स॒रस्य॑ । \newline
15. अय॑ने॒ याव॑ती॒ याव॒ त्यय॒ने ऽय॑ने॒ याव॑ त्ये॒वैव याव॒ त्यय॒ने ऽय॑ने॒ याव॑ त्ये॒व । \newline
16. याव॑ त्ये॒वैव याव॑ती॒ याव॑ त्ये॒व वाग् वागे॒व याव॑ती॒ याव॑ त्ये॒व वाक् । \newline
17. ए॒व वाग् वागे॒ वैव वाक् ताम् तां ॅवागे॒ वैव वाक् ताम् । \newline
18. वाक् ताम् तां ॅवाग् वाक् ता मवाव॒ तां ॅवाग् वाक् ता मव॑ । \newline
19. ता मवाव॒ ताम् ता मव॑ रुन्धे रु॒न्धे ऽव॒ ताम् ता मव॑ रुन्धे । \newline
20. अव॑ रुन्धे रु॒न्धे ऽवाव॑ रुन्धे सप्त॒मे स॑प्त॒मे रु॒न्धे ऽवाव॑ रुन्धे सप्त॒मे । \newline
21. रु॒न्धे॒ स॒प्त॒मे स॑प्त॒मे रु॑न्धे रुन्धे सप्त॒मे प॒दे प॒दे स॑प्त॒मे रु॑न्धे रुन्धे सप्त॒मे प॒दे । \newline
22. स॒प्त॒मे प॒दे प॒दे स॑प्त॒मे स॑प्त॒मे प॒दे जु॑होति जुहोति प॒दे स॑प्त॒मे स॑प्त॒मे प॒दे जु॑होति । \newline
23. प॒दे जु॑होति जुहोति प॒दे प॒दे जु॑होति स॒प्तप॑दा स॒प्तप॑दा जुहोति प॒दे प॒दे जु॑होति स॒प्तप॑दा । \newline
24. जु॒हो॒ति॒ स॒प्तप॑दा स॒प्तप॑दा जुहोति जुहोति स॒प्तप॑दा॒ शक्व॑री॒ शक्व॑री स॒प्तप॑दा जुहोति जुहोति स॒प्तप॑दा॒ शक्व॑री । \newline
25. स॒प्तप॑दा॒ शक्व॑री॒ शक्व॑री स॒प्तप॑दा स॒प्तप॑दा॒ शक्व॑री प॒शवः॑ प॒शवः॒ शक्व॑री स॒प्तप॑दा स॒प्तप॑दा॒ शक्व॑री प॒शवः॑ । \newline
26. स॒प्तप॒देति॑ स॒प्त - प॒दा॒ । \newline
27. शक्व॑री प॒शवः॑ प॒शवः॒ शक्व॑री॒ शक्व॑री प॒शवः॒ शक्व॑री॒ शक्व॑री प॒शवः॒ शक्व॑री॒ शक्व॑री प॒शवः॒ शक्व॑री । \newline
28. प॒शवः॒ शक्व॑री॒ शक्व॑री प॒शवः॑ प॒शवः॒ शक्व॑री प॒शून् प॒शूञ् छक्व॑री प॒शवः॑ प॒शवः॒ शक्व॑री प॒शून् । \newline
29. शक्व॑री प॒शून् प॒शूञ् छक्व॑री॒ शक्व॑री प॒शू ने॒वैव प॒शूञ् छक्व॑री॒ शक्व॑री प॒शूने॒व । \newline
30. प॒शू ने॒वैव प॒शून् प॒शू ने॒वावा वै॒व प॒शून् प॒शू ने॒वाव॑ । \newline
31. ए॒वावा वै॒वै वाव॑ रुन्धे रु॒न्धे ऽवै॒वै वाव॑ रुन्धे । \newline
32. अव॑ रुन्धे रु॒न्धे ऽवाव॑ रुन्धे स॒प्त स॒प्त रु॒न्धे ऽवाव॑ रुन्धे स॒प्त । \newline
33. रु॒न्धे॒ स॒प्त स॒प्त रु॑न्धे रुन्धे स॒प्त ग्रा॒म्या ग्रा॒म्याः स॒प्त रु॑न्धे रुन्धे स॒प्त ग्रा॒म्याः । \newline
34. स॒प्त ग्रा॒म्या ग्रा॒म्याः स॒प्त स॒प्त ग्रा॒म्याः प॒शवः॑ प॒शवो᳚ ग्रा॒म्याः स॒प्त स॒प्त ग्रा॒म्याः प॒शवः॑ । \newline
35. ग्रा॒म्याः प॒शवः॑ प॒शवो᳚ ग्रा॒म्या ग्रा॒म्याः प॒शवः॑ स॒प्त स॒प्त प॒शवो᳚ ग्रा॒म्या ग्रा॒म्याः प॒शवः॑ स॒प्त । \newline
36. प॒शवः॑ स॒प्त स॒प्त प॒शवः॑ प॒शवः॑ स॒प्ता र॒ण्या आ॑र॒ण्याः स॒प्त प॒शवः॑ प॒शवः॑ स॒प्ता र॒ण्याः । \newline
37. स॒प्ता र॒ण्या आ॑र॒ण्याः स॒प्त स॒प्ता र॒ण्याः स॒प्त स॒प्ता र॒ण्याः स॒प्त स॒प्ता र॒ण्याः स॒प्त । \newline
38. आ॒र॒ण्याः स॒प्त स॒प्ता र॒ण्या आ॑र॒ण्याः स॒प्त छन्दाꣳ॑सि॒ छन्दाꣳ॑सि स॒प्ता र॒ण्या आ॑र॒ण्याः स॒प्त छन्दाꣳ॑सि । \newline
39. स॒प्त छन्दाꣳ॑सि॒ छन्दाꣳ॑सि स॒प्त स॒प्त छन्दाꣳ॑ स्यु॒भय॑ स्यो॒भय॑स्य॒ छन्दाꣳ॑सि स॒प्त स॒प्त छन्दाꣳ॑ स्यु॒भय॑स्य । \newline
40. छन्दाꣳ॑ स्यु॒भय॑ स्यो॒भय॑स्य॒ छन्दाꣳ॑सि॒ छन्दाꣳ॑ स्यु॒भय॒स्या व॑रुद्ध्या॒ अव॑रुद्ध्या उ॒भय॑स्य॒ छन्दाꣳ॑सि॒ छन्दाꣳ॑ स्यु॒भय॒स्या व॑रुद्ध्यै । \newline
41. उ॒भय॒स्या व॑रुद्ध्या॒ अव॑रुद्ध्या उ॒भय॑ स्यो॒भय॒स्या व॑रुद्ध्यै॒ वस्वी॒ वस्व्य व॑रुद्ध्या उ॒भय॑
स्यो॒भय॒स्या व॑रुद्ध्यै॒ वस्वी᳚ । \newline
42. अव॑रुद्ध्यै॒ वस्वी॒ वस्व्य व॑रुद्ध्या॒ अव॑रुद्ध्यै॒ वस्व्य॑स्यसि॒ वस्व्य व॑रुद्ध्या॒ अव॑रुद्ध्यै॒ वस्व्य॑सि । \newline
43. अव॑रुद्ध्या॒ इत्यव॑ - रु॒द्ध्यै॒ । \newline
44. वस्व्य॑स्यसि॒ वस्वी॒ वस्व्य॑सि रु॒द्रा रु॒द्रा ऽसि॒ वस्वी॒ वस्व्य॑सि रु॒द्रा । \newline
45. अ॒सि॒ रु॒द्रा रु॒द्रा ऽस्य॑सि रु॒द्रा ऽस्य॑सि रु॒द्रा ऽस्य॑सि रु॒द्रा ऽसि॑ । \newline
46. रु॒द्रा ऽस्य॑सि रु॒द्रा रु॒द्रा ऽसीती त्य॑सि रु॒द्रा रु॒द्रा ऽसीति॑ । \newline
47. अ॒सीती त्य॑स्य॒ सीत्या॑हा॒हे त्य॑स्य॒ सीत्या॑ह । \newline
48. इत्या॑हा॒हे तीत्या॑ह रू॒पꣳ रू॒प मा॒हे तीत्या॑ह रू॒पम् । \newline
49. आ॒ह॒ रू॒पꣳ रू॒प मा॑हाह रू॒प मे॒वैव रू॒प मा॑हाह रू॒प मे॒व । \newline
50. रू॒प मे॒वैव रू॒पꣳ रू॒प मे॒वास्या॑ अस्या ए॒व रू॒पꣳ रू॒प मे॒वास्याः᳚ । \newline
51. ए॒वास्या॑ अस्या ए॒वै वास्या॑ ए॒त दे॒त द॑स्या ए॒वै वास्या॑ ए॒तत् । \newline
52. अ॒स्या॒ ए॒त दे॒त द॑स्या अस्या ए॒तन् म॑हि॒मान॑म् महि॒मान॑ मे॒त द॑स्या अस्या ए॒तन् म॑हि॒मान᳚म् । \newline
53. ए॒तन् म॑हि॒मान॑म् महि॒मान॑ मे॒त दे॒तन् म॑हि॒मानं॒ ॅव्याच॑ष्टे॒ व्याच॑ष्टे महि॒मान॑ मे॒त दे॒तन् म॑हि॒मानं॒ ॅव्याच॑ष्टे । \newline
54. म॒हि॒मानं॒ ॅव्याच॑ष्टे॒ व्याच॑ष्टे महि॒मान॑म् महि॒मानं॒ ॅव्याच॑ष्टे॒ बृह॒स्पति॒र् बृह॒स्पति॒र् व्याच॑ष्टे महि॒मान॑म् महि॒मानं॒ ॅव्याच॑ष्टे॒ बृह॒स्पतिः॑ । \newline
\pagebreak
\markright{ TS 6.1.8.2  \hfill https://www.vedavms.in \hfill}

\section{ TS 6.1.8.2 }

\textbf{TS 6.1.8.2 } \newline
\textbf{Samhita Paata} \newline

ॅव्याच॑ष्टे॒ बृह॒स्पति॑स्त्वा सु॒म्ने र॑ण्व॒त्वित्या॑ह॒ ब्रह्म॒ वै दे॒वानां॒ बृह॒स्पति॒-र्ब्रह्म॑णै॒वास्मै॑ प॒शूनव॑ रुन्धे रु॒द्रो वसु॑भि॒रा चि॑के॒त्वित्या॒हाऽऽ*वृ॑त्त्यै पृथि॒व्यास्त्वा॑ मू॒र्द्धन्ना जि॑घर्मि देव॒यज॑न॒ इत्या॑ह पृथि॒व्या ह्ये॑ष मू॒र्द्धा यद्-दे॑व॒यज॑न॒मिडा॑याः प॒द इत्या॒हेडा॑यै॒ ह्ये॑तत् प॒दं ॅयथ् सो॑म॒क्रय॑ण्यै घृ॒तव॑ति॒ स्वाहे - [  ] \newline

\textbf{Pada Paata} \newline

व्याच॑ष्ट॒ इति॑ वि-आच॑ष्टे । बृह॒स्पतिः॑ । त्वा॒ । सु॒म्ने । र॒ण्व॒तु॒ । इति॑ । आ॒ह॒ । ब्रह्म॑ । वै । दे॒वाना᳚म् । बृह॒स्पतिः॑ । ब्रह्म॑णा । ए॒व । अ॒स्मै॒ । प॒शून् । अवेति॑ । रु॒न्धे॒ । रु॒द्रः । वसु॑भि॒रिति॒ वसु॑ - भिः॒ । एति॑ । चि॒के॒तु॒ । इति॑ । आ॒ह॒ । आवृ॑त्त्या॒ इत्या - वृ॒त्त्यै॒ । पृ॒थि॒व्याः । त्वा॒ । मू॒द्‌र्धन्न् । एति॑ । जि॒घ॒र्मि॒ । दे॒व॒यज॑न॒ इति॑ देव-यज॑ने । इति॑ । आ॒ह॒ । पृ॒थि॒व्याः । हि । ए॒षः । मू॒द्‌र्धा । यत् । दे॒व॒यज॑न॒मिति॑ देव - यज॑नम् । इडा॑याः । प॒दे । इति॑ । आ॒ह॒ । इडा॑यै । हि । ए॒तत् । प॒दम् । यत् । सो॒म॒क्रय॑ण्या॒ इति॑ सोम - क्रय॑ण्यै । घृ॒तव॒तीति॑ घृ॒त-व॒ति॒ । स्वाहा᳚ ।  \newline


\textbf{Krama Paata} \newline

व्याच॑ष्टे॒ बृह॒स्पतिः॑ । व्याच॑ष्ट॒ इति॒ वि - आच॑ष्टे । बृह॒स्पति॑स्त्वा । त्वा॒ सु॒म्ने । सु॒म्ने र॑ण्वतु । र॒ण्व॒त्विति॑ । इत्या॑ह । आ॒ह॒ ब्रह्म॑ । ब्रह्म॒ वै । वै दे॒वाना᳚म् । दे॒वाना॒म् बृह॒स्पतिः॑ । बृह॒स्पति॒र् ब्रह्म॑णा । ब्रह्म॑णै॒व । ए॒वास्मै᳚ । अ॒स्मै॒ प॒शून् । प॒शूनव॑ । अव॑ रुन्धे । रु॒न्धे॒ रु॒द्रः । रु॒द्रो वसु॑भिः । वसु॑भि॒रा । वसु॑भि॒रिति॒ वसु॑ - भिः॒ । आ चि॑केतु । चि॒के॒त्विति॑ । इत्या॑ह । आ॒हावृ॑त्त्यै । आवृ॑त्त्यै पृथि॒व्याः । आवृ॑त्त्या॒ इत्या - वृ॒त्त्यै॒ । पृ॒थि॒व्यास्त्वा᳚ । त्वा॒ मू॒र्द्धन्न् । मू॒र्द्धन्ना । आ जि॑घर्मि । जि॒घ॒र्मि॒ दे॒व॒यज॑ने । दे॒व॒यज॑न॒ इति॑ । दे॒व॒यज॑न॒ इति॑ देव - यज॑ने । इत्या॑ह । आ॒ह॒ पृ॒थि॒व्याः । पृ॒थि॒व्या हि । ह्ये॑षः । ए॒ष मू॒र्द्धा । मू॒र्द्धा यत् । यद् दे॑व॒यज॑नम् । दे॒व॒यज॑न॒मिडा॑याः । दे॒व॒यज॑न॒मिति॑ देव - यज॑नम् । इडा॑याः प॒दे । प॒द इति॑ । इत्या॑ह । आ॒हेडा॑यै । इडा॑यै॒ हि । ह्ये॑तत् । ए॒तत् प॒दम् । प॒दम् ॅयत् । यथ् सो॑म॒क्रय॑ण्यै । सो॒म॒क्रय॑ण्यै घृ॒तव॑ति । सो॒म॒क्रय॑ण्या॒ इति॑ सोम - क्रय॑ण्यै । घृ॒तव॑ति॒ स्वाहा᳚ । घृ॒तव॒तीति॑ घृ॒त - व॒ति॒ । स्वाहेति॑ \newline

\textbf{Jatai Paata} \newline

1. व्याच॑ष्टे॒ बृह॒स्पति॒र् बृह॒स्पति॒र् व्याच॑ष्टे॒ व्याच॑ष्टे॒ बृह॒स्पतिः॑ । \newline
2. व्याच॑ष्ट॒ इति॑ वि - आच॑ष्टे । \newline
3. बृह॒स्पति॑ स्त्वा त्वा॒ बृह॒स्पति॒र् बृह॒स्पति॑ स्त्वा । \newline
4. त्वा॒ सु॒म्ने सु॒म्ने त्वा᳚ त्वा सु॒म्ने । \newline
5. सु॒म्ने र॑ण्वतु रण्वतु सु॒म्ने सु॒म्ने र॑ण्वतु । \newline
6. र॒ण्व॒त्वि तीति॑ रण्वतु रण्व॒ त्विति॑ । \newline
7. इत्या॑हा॒हे तीत्या॑ह । \newline
8. आ॒ह॒ ब्रह्म॒ ब्रह्मा॑हाह॒ ब्रह्म॑ । \newline
9. ब्रह्म॒ वै वै ब्रह्म॒ ब्रह्म॒ वै । \newline
10. वै दे॒वाना᳚म् दे॒वानां॒ ॅवै वै दे॒वाना᳚म् । \newline
11. दे॒वाना॒म् बृह॒स्पति॒र् बृह॒स्पति॑र् दे॒वाना᳚म् दे॒वाना॒म् बृह॒स्पतिः॑ । \newline
12. बृह॒स्पति॒र् ब्रह्म॑णा॒ ब्रह्म॑णा॒ बृह॒स्पति॒र् बृह॒स्पति॒र् ब्रह्म॑णा । \newline
13. ब्रह्म॑ णै॒वैव ब्रह्म॑णा॒ ब्रह्म॑णै॒व । \newline
14. ए॒वास्मा॑ अस्मा ए॒वै वास्मै᳚ । \newline
15. अ॒स्मै॒ प॒शून् प॒शू न॑स्मा अस्मै प॒शून् । \newline
16. प॒शून वाव॑ प॒शून् प॒शू नव॑ । \newline
17. अव॑ रुन्धे रु॒न्धे ऽवाव॑ रुन्धे । \newline
18. रु॒न्धे॒ रु॒द्रो रु॒द्रो रु॑न्धे रुन्धे रु॒द्रः । \newline
19. रु॒द्रो वसु॑भि॒र् वसु॑भी रु॒द्रो रु॒द्रो वसु॑भिः । \newline
20. वसु॑भि॒रा वसु॑भि॒र् वसु॑भि॒रा । \newline
21. वसु॑भि॒रिति॒ वसु॑ - भिः॒ । \newline
22. आ चि॑केतु चिके॒त्वा चि॑केतु । \newline
23. चि॒के॒ त्वितीति॑ चिकेतु चिके॒ त्विति॑ । \newline
24. इत्या॑हा॒हे तीत्या॑ह । \newline
25. आ॒हावृ॑त्त्या॒ आवृ॑त्त्या आहा॒ हावृ॑त्त्यै । \newline
26. आवृ॑त्त्यै पृथि॒व्याः पृ॑थि॒व्या आवृ॑त्त्या॒ आवृ॑त्त्यै पृथि॒व्याः । \newline
27. आवृ॑त्त्या॒ इत्या - वृ॒त्त्यै॒ । \newline
28. पृ॒थि॒व्या स्त्वा᳚ त्वा पृथि॒व्याः पृ॑थि॒व्या स्त्वा᳚ । \newline
29. त्वा॒ मू॒र्द्धन् मू॒र्द्धन् त्वा᳚ त्वा मू॒र्द्धन्न् । \newline
30. मू॒र्द्धन्ना मू॒र्द्धन् मू॒र्द्धन्ना । \newline
31. आ जि॑घर्मि जिघ॒र्म्या जि॑घर्मि । \newline
32. जि॒घ॒र्मि॒ दे॒व॒यज॑ने देव॒यज॑ने जिघर्मि जिघर्मि देव॒यज॑ने । \newline
33. दे॒व॒यज॑न॒ इतीति॑ देव॒यज॑ने देव॒यज॑न॒ इति॑ । \newline
34. दे॒व॒यज॑न॒ इति॑ देव - यज॑ने । \newline
35. इत्या॑हा॒हे तीत्या॑ह । \newline
36. आ॒ह॒ पृ॒थि॒व्याः पृ॑थि॒व्या आ॑हाह पृथि॒व्याः । \newline
37. पृ॒थि॒व्या हि हि पृ॑थि॒व्याः पृ॑थि॒व्या हि । \newline
38. ह्ये॑ष ए॒ष हि ह्ये॑षः । \newline
39. ए॒ष मू॒र्द्धा मू॒र्द्धैष ए॒ष मू॒र्द्धा । \newline
40. मू॒र्द्धा यद् यन् मू॒र्द्धा मू॒र्द्धा यत् । \newline
41. यद् दे॑व॒यज॑नम् देव॒यज॑नं॒ ॅयद् यद् दे॑व॒यज॑नम् । \newline
42. दे॒व॒यज॑न॒ मिडा॑या॒ इडा॑या देव॒यज॑नम् देव॒यज॑न॒ मिडा॑याः । \newline
43. दे॒व॒यज॑न॒मिति॑ देव - यज॑नम् । \newline
44. इडा॑याः प॒दे प॒द इडा॑या॒ इडा॑याः प॒दे । \newline
45. प॒द इतीति॑ प॒दे प॒द इति॑ । \newline
46. इत्या॑हा॒हे तीत्या॑ह । \newline
47. आ॒हे डा॑या॒ इडा॑या आहा॒हे डा॑यै । \newline
48. इडा॑यै॒ हि हीडा॑या॒ इडा॑यै॒ हि । \newline
49. ह्ये॑त दे॒तद्धि ह्ये॑तत् । \newline
50. ए॒तत् प॒दम् प॒द मे॒त दे॒तत् प॒दम् । \newline
51. प॒दं ॅयद् यत् प॒दम् प॒दं ॅयत् । \newline
52. यथ् सो॑म॒क्रय॑ण्यै सोम॒क्रय॑ण्यै॒ यद् यथ् सो॑म॒क्रय॑ण्यै । \newline
53. सो॒म॒क्रय॑ण्यै घृ॒तव॑ति घृ॒तव॑ति सोम॒क्रय॑ण्यै सोम॒क्रय॑ण्यै घृ॒तव॑ति । \newline
54. सो॒म॒क्रय॑ण्या॒ इति॑ सोम - क्रय॑ण्यै । \newline
55. घृ॒तव॑ति॒ स्वाहा॒ स्वाहा॑ घृ॒तव॑ति घृ॒तव॑ति॒ स्वाहा᳚ । \newline
56. घृ॒तव॒तीति॑ घृ॒त - व॒ति॒ । \newline
57. स्वाहे तीति॒ स्वाहा॒ स्वाहेति॑ । \newline

\textbf{Ghana Paata } \newline

1. व्याच॑ष्टे॒ बृह॒स्पति॒र् बृह॒स्पति॒र् व्याच॑ष्टे॒ व्याच॑ष्टे॒ बृह॒स्पति॑ स्त्वा त्वा॒ बृह॒स्पति॒र् व्याच॑ष्टे॒ व्याच॑ष्टे॒ बृह॒स्पति॑ स्त्वा । \newline
2. व्याच॑ष्ट॒ इति॑ वि - आच॑ष्टे । \newline
3. बृह॒स्पति॑ स्त्वा त्वा॒ बृह॒स्पति॒र् बृह॒स्पति॑ स्त्वा सु॒म्ने सु॒म्ने त्वा॒ बृह॒स्पति॒र् बृह॒स्पति॑ स्त्वा सु॒म्ने । \newline
4. त्वा॒ सु॒म्ने सु॒म्ने त्वा᳚ त्वा सु॒म्ने र॑ण्वतु रण्वतु सु॒म्ने त्वा᳚ त्वा सु॒म्ने र॑ण्वतु । \newline
5. सु॒म्ने र॑ण्वतु रण्वतु सु॒म्ने सु॒म्ने र॑ण्व॒त्वि तीति॑ रण्वतु सु॒म्ने सु॒म्ने र॑ण्व॒त्विति॑ । \newline
6. र॒ण्व॒त्वि तीति॑ रण्वतु रण्व॒त्वि त्या॑हा॒हेति॑ रण्वतु रण्व॒त्वि त्या॑ह । \newline
7. इत्या॑हा॒हे तीत्या॑ह॒ ब्रह्म॒ ब्रह्मा॒हे तीत्या॑ह॒ ब्रह्म॑ । \newline
8. आ॒ह॒ ब्रह्म॒ ब्रह्मा॑हाह॒ ब्रह्म॒ वै वै ब्रह्मा॑हाह॒ ब्रह्म॒ वै । \newline
9. ब्रह्म॒ वै वै ब्रह्म॒ ब्रह्म॒ वै दे॒वाना᳚म् दे॒वानां॒ ॅवै ब्रह्म॒ ब्रह्म॒ वै दे॒वाना᳚म् । \newline
10. वै दे॒वाना᳚म् दे॒वानां॒ ॅवै वै दे॒वाना॒म् बृह॒स्पति॒र् बृह॒स्पति॑र् दे॒वानां॒ ॅवै वै दे॒वाना॒म् बृह॒स्पतिः॑ । \newline
11. दे॒वाना॒म् बृह॒स्पति॒र् बृह॒स्पति॑र् दे॒वाना᳚म् दे॒वाना॒म् बृह॒स्पति॒र् ब्रह्म॑णा॒ ब्रह्म॑णा॒ बृह॒स्पति॑र् दे॒वाना᳚म् दे॒वाना॒म् बृह॒स्पति॒र् ब्रह्म॑णा । \newline
12. बृह॒स्पति॒र् ब्रह्म॑णा॒ ब्रह्म॑णा॒ बृह॒स्पति॒र् बृह॒स्पति॒र् ब्रह्म॑ णै॒वैव ब्रह्म॑णा॒ बृह॒स्पति॒र् बृह॒स्पति॒र् ब्रह्म॑णै॒व । \newline
13. ब्रह्म॑ णै॒वैव ब्रह्म॑णा॒ ब्रह्म॑ णै॒वास्मा॑ अस्मा ए॒व ब्रह्म॑णा॒ ब्रह्म॑ णै॒वास्मै᳚ । \newline
14. ए॒वास्मा॑ अस्मा ए॒वै वास्मै॑ प॒शून् प॒शून॑स्मा ए॒वै वास्मै॑ प॒शून् । \newline
15. अ॒स्मै॒ प॒शून् प॒शू न॑स्मा अस्मै प॒शू नवाव॑ प॒शू न॑स्मा अस्मै प॒शू नव॑ । \newline
16. प॒शू नवाव॑ प॒शून् प॒शू नव॑ रुन्धे रु॒न्धे ऽव॑ प॒शून् प॒शू नव॑ रुन्धे । \newline
17. अव॑ रुन्धे रु॒न्धे ऽवाव॑ रुन्धे रु॒द्रो रु॒द्रो रु॒न्धे ऽवाव॑ रुन्धे रु॒द्रः । \newline
18. रु॒न्धे॒ रु॒द्रो रु॒द्रो रु॑न्धे रुन्धे रु॒द्रो वसु॑भि॒र् वसु॑भी रु॒द्रो रु॑न्धे रुन्धे रु॒द्रो वसु॑भिः । \newline
19. रु॒द्रो वसु॑भि॒र् वसु॑भी रु॒द्रो रु॒द्रो वसु॑भि॒रा वसु॑भी रु॒द्रो रु॒द्रो वसु॑भि॒रा । \newline
20. वसु॑भि॒रा वसु॑भि॒र् वसु॑भि॒रा चि॑केतु चिके॒त्वा वसु॑भि॒र् वसु॑भि॒रा चि॑केतु । \newline
21. वसु॑भि॒रिति॒ वसु॑ - भिः॒ । \newline
22. आ चि॑केतु चिके॒त्वा चि॑के॒त्वि तीति॑ चिके॒त्वा चि॑के॒त्विति॑ । \newline
23. चि॒के॒त्वि तीति॑ चिकेतु चिके॒त्वि त्या॑हा॒हेति॑ चिकेतु चिके॒त्वि त्या॑ह । \newline
24. इत्या॑हा॒हे तीत्या॒हा वृ॑त्त्या॒ आवृ॑त्त्या आ॒हे तीत्या॒हा वृ॑त्त्यै । \newline
25. आ॒हा वृ॑त्त्या॒ आवृ॑त्त्या आहा॒हा वृ॑त्त्यै पृथि॒व्याः पृ॑थि॒व्या आवृ॑त्त्या आहा॒हा वृ॑त्त्यै पृथि॒व्याः । \newline
26. आवृ॑त्त्यै पृथि॒व्याः पृ॑थि॒व्या आवृ॑त्त्या॒ आवृ॑त्त्यै पृथि॒व्या स्त्वा᳚ त्वा पृथि॒व्या आवृ॑त्त्या॒ आवृ॑त्त्यै पृथि॒व्या स्त्वा᳚ । \newline
27. आवृ॑त्त्या॒ इत्या - वृ॒त्त्यै॒ । \newline
28. पृ॒थि॒व्या स्त्वा᳚ त्वा पृथि॒व्याः पृ॑थि॒व्या स्त्वा॑ मू॒र्द्धन् मू॒र्द्धन् त्वा॑ पृथि॒व्याः पृ॑थि॒व्या स्त्वा॑ मू॒र्द्धन्न् । \newline
29. त्वा॒ मू॒र्द्धन् मू॒र्द्धन् त्वा᳚ त्वा मू॒र्द्धन्ना मू॒र्द्धन् त्वा᳚ त्वा मू॒र्द्धन्ना । \newline
30. मू॒र्द्धन्ना मू॒र्द्धन् मू॒र्द्धन्ना जि॑घर्मि जिघ॒र्म्या मू॒र्द्धन् मू॒र्द्धन्ना जि॑घर्मि । \newline
31. आ जि॑घर्मि जिघ॒र्म्या जि॑घर्मि देव॒यज॑ने देव॒यज॑ने जिघ॒र्म्या जि॑घर्मि देव॒यज॑ने । \newline
32. जि॒घ॒र्मि॒ दे॒व॒यज॑ने देव॒यज॑ने जिघर्मि जिघर्मि देव॒यज॑न॒ इतीति॑ देव॒यज॑ने जिघर्मि जिघर्मि देव॒यज॑न॒ इति॑ । \newline
33. दे॒व॒यज॑न॒ इतीति॑ देव॒यज॑ने देव॒यज॑न॒ इत्या॑हा॒हेति॑ देव॒यज॑ने देव॒यज॑न॒ इत्या॑ह । \newline
34. दे॒व॒यज॑न॒ इति॑ देव - यज॑ने । \newline
35. इत्या॑हा॒हे तीत्या॑ह पृथि॒व्याः पृ॑थि॒व्या आ॒हे तीत्या॑ह पृथि॒व्याः । \newline
36. आ॒ह॒ पृ॒थि॒व्याः पृ॑थि॒व्या आ॑हाह पृथि॒व्या हि हि पृ॑थि॒व्या आ॑हाह पृथि॒व्या हि । \newline
37. पृ॒थि॒व्या हि हि पृ॑थि॒व्याः पृ॑थि॒व्या ह्ये॑ष ए॒ष हि पृ॑थि॒व्याः पृ॑थि॒व्या ह्ये॑षः । \newline
38. ह्ये॑ष ए॒ष हि ह्ये॑ष मू॒र्द्धा मू॒र्द्धैष हि ह्ये॑ष मू॒र्द्धा । \newline
39. ए॒ष मू॒र्द्धा मू॒र्द्धैष ए॒ष मू॒र्द्धा यद् यन् मू॒र्द्धैष ए॒ष मू॒र्द्धा यत् । \newline
40. मू॒र्द्धा यद् यन् मू॒र्द्धा मू॒र्द्धा यद् दे॑व॒यज॑नम् देव॒यज॑नं॒ ॅयन् मू॒र्द्धा मू॒र्द्धा यद् दे॑व॒यज॑नम् । \newline
41. यद् दे॑व॒यज॑नम् देव॒यज॑नं॒ ॅयद् यद् दे॑व॒यज॑न॒ मिडा॑या॒ इडा॑या देव॒यज॑नं॒ ॅयद् यद् दे॑व॒यज॑न॒ मिडा॑याः । \newline
42. दे॒व॒यज॑न॒ मिडा॑या॒ इडा॑या देव॒यज॑नम् देव॒यज॑न॒ मिडा॑याः प॒दे प॒द इडा॑या देव॒यज॑नम् देव॒यज॑न॒ मिडा॑याः प॒दे । \newline
43. दे॒व॒यज॑न॒मिति॑ देव - यज॑नम् । \newline
44. इडा॑याः प॒दे प॒द इडा॑या॒ इडा॑याः प॒द इतीति॑ प॒द इडा॑या॒ इडा॑याः प॒द इति॑ । \newline
45. प॒द इतीति॑ प॒दे प॒द इत्या॑हा॒हेति॑ प॒दे प॒द इत्या॑ह । \newline
46. इत्या॑हा॒हे तीत्या॒ हेडा॑या॒ इडा॑या आ॒हे तीत्या॒ हेडा॑यै । \newline
47. आ॒हेडा॑या॒ इडा॑या आहा॒ हेडा॑यै॒ हि हीडा॑या आहा॒ हेडा॑यै॒ हि । \newline
48. इडा॑यै॒ हि हीडा॑या॒ इडा॑यै॒ ह्ये॑त दे॒त द्धीडा॑या॒ इडा॑यै॒ ह्ये॑तत् । \newline
49. ह्ये॑त दे॒त द्धि ह्ये॑तत् प॒दम् प॒द मे॒तद्धि ह्ये॑तत् प॒दम् । \newline
50. ए॒तत् प॒दम् प॒द मे॒त दे॒तत् प॒दं ॅयद् यत् प॒द मे॒त दे॒तत् प॒दं ॅयत् । \newline
51. प॒दं ॅयद् यत् प॒दम् प॒दं ॅयथ् सो॑म॒क्रय॑ण्यै सोम॒क्रय॑ण्यै॒ यत् प॒दम् प॒दं ॅयथ् सो॑म॒क्रय॑ण्यै । \newline
52. यथ् सो॑म॒क्रय॑ण्यै सोम॒क्रय॑ण्यै॒ यद् यथ् सो॑म॒क्रय॑ण्यै घृ॒तव॑ति घृ॒तव॑ति सोम॒क्रय॑ण्यै॒ यद् यथ् सो॑म॒क्रय॑ण्यै घृ॒तव॑ति । \newline
53. सो॒म॒क्रय॑ण्यै घृ॒तव॑ति घृ॒तव॑ति सोम॒क्रय॑ण्यै सोम॒क्रय॑ण्यै घृ॒तव॑ति॒ स्वाहा॒ स्वाहा॑ घृ॒तव॑ति सोम॒क्रय॑ण्यै सोम॒क्रय॑ण्यै घृ॒तव॑ति॒ स्वाहा᳚ । \newline
54. सो॒म॒क्रय॑ण्या॒ इति॑ सोम - क्रय॑ण्यै । \newline
55. घृ॒तव॑ति॒ स्वाहा॒ स्वाहा॑ घृ॒तव॑ति घृ॒तव॑ति॒ स्वाहे तीति॒ स्वाहा॑ घृ॒तव॑ति घृ॒तव॑ति॒ स्वाहेति॑ । \newline
56. घृ॒तव॒तीति॑ घृ॒त - व॒ति॒ । \newline
57. स्वाहे तीति॒ स्वाहा॒ स्वाहेत्या॑ हा॒हेति॒ स्वाहा॒ स्वाहे त्या॑ह । \newline
\pagebreak
\markright{ TS 6.1.8.3  \hfill https://www.vedavms.in \hfill}

\section{ TS 6.1.8.3 }

\textbf{TS 6.1.8.3 } \newline
\textbf{Samhita Paata} \newline

-त्या॑ह॒ यदे॒वास्यै॑ प॒दाद्-घृ॒तमपी᳚ड्यत॒ तस्मा॑दे॒वमा॑ह॒ यद॑द्ध्व॒र्युर॑न॒ग्नावाहु॑तिं जुहु॒याद॒न्धो᳚ऽद्ध्व॒र्युः स्या॒द्-रक्षाꣳ॑सि य॒ज्ञ्ꣳ ह॑न्यु॒र्॒.हिर॑ण्यमु॒पास्य॑ जुहोत्यग्नि॒वत्ये॒व जु॑होति॒ नान्धो᳚ऽद्ध्व॒र्यु र्भव॑ति॒ न य॒ज्ञ्ꣳ रक्षाꣳ॑सि घ्नन्ति॒ काण्डे॑काण्डे॒ वै क्रि॒यमा॑णे य॒ज्ञ्ꣳ रक्षाꣳ॑सि जिघाꣳसन्ति॒ परि॑लिखितꣳ॒॒ रक्षः॒ परि॑लिखिता॒ अरा॑तय॒ इत्या॑ह॒ रक्ष॑सा॒मप॑हत्या - [  ] \newline

\textbf{Pada Paata} \newline

इति॑ । आ॒ह॒ । यत् । ए॒व । अ॒स्यै॒ । प॒दात् । घृ॒तम् । अपी᳚ड्यत । तस्मा᳚त् । ए॒वम् । आ॒ह॒ । यत् । अ॒द्ध्व॒र्युः । अ॒न॒ग्नौ । आहु॑ति॒मित्या - हु॒ति॒म् । जु॒हु॒यात् । अ॒न्धः । अ॒द्ध्व॒र्युः । स्या॒त् । रक्षाꣳ॑सि । य॒ज्ञ्म् । ह॒न्युः॒ । हिर॑ण्यम् । उ॒पास्येत्यु॑प - अस्य॑ । जु॒हो॒ति॒ । अ॒ग्नि॒वतीत्य॑ग्नि - वति॑ । ए॒व । जु॒हो॒ति॒ । न । अ॒न्धः । अ॒द्ध्व॒र्युः । भव॑ति । न । य॒ज्ञ्म् । रक्षाꣳ॑सि । घ्न॒न्ति॒ । काण्डे॑काण्ड॒ इति॒ काण्डे᳚ - का॒ण्डे॒ । वै । क्रि॒यमा॑णे । य॒ज्ञ्म् । रक्षाꣳ॑सि । जि॒घाꣳ॒॒स॒न्ति॒ । परि॑लिखित॒मिति॒ परि॑ - लि॒खि॒त॒म् । रक्षः॑ । परि॑लिखिता॒ इति॒ परि॑ - लि॒खि॒ताः॒ । अरा॑तयः । इति॑ । आ॒ह॒ । रक्ष॑साम् । अप॑हत्या॒ इत्यप॑ - ह॒त्यै॒ ।  \newline


\textbf{Krama Paata} \newline

इत्या॑ह । आ॒ह॒ यत् । यदे॒व । ए॒वास्मै᳚ । अ॒स्मै॒ प॒दात् । प॒दाद् घृ॒तम् । घृ॒तमपी᳚ड्‍यत । अपी᳚ड्‍यत॒ तस्मा᳚त् । तस्मा॑दे॒वम् । ए॒वमा॑ह । आ॒ह॒ यत् । यद॑द्ध्व॒र्युः । अ॒द्ध्व॒र्यु,र॑न॒ग्नौ । अ॒न॒ग्नावाहु॑तिम् । आहु॑तिम् जुहु॒यात् । आहु॑ति॒मित्या - हु॒ति॒म् । जु॒हु॒याद॒न्धः । अ॒न्धो᳚ऽद्ध्व॒र्युः । अ॒द्ध्व॒र्युः स्या᳚त् । स्या॒द् रक्षाꣳ॑सि । रक्षाꣳ॑सि य॒ज्ञ्म् । य॒ज्ञ्ꣳ ह॑न्युः । ह॒न्यु॒र् हिर॑ण्यम् । हिर॑ण्यमु॒पास्य॑ । उ॒पास्य॑ जुहोति । उ॒पास्येत्यु॑प - अस्य॑ । जु॒हो॒त्य॒ग्नि॒वति॑ । अ॒ग्नि॒वत्ये॒व । अ॒ग्नि॒वतीत्य॑ग्नि - वति॑ । ए॒व जु॑होति । जु॒हो॒ति॒ न । नान्धः । अ॒न्धो᳚ऽद्ध्व॒र्युः । अ॒द्ध्व॒र्युर् भव॑ति । भव॑ति॒ न । न य॒ज्ञ्म् । य॒ज्ञ्ꣳ रक्षाꣳ॑सि । रक्षाꣳ॑सि घ्नन्ति । घ्न॒न्ति॒ काण्डे॑काण्डे । काण्डे॑काण्डे॒ वै । काण्डे॑काण्ड॒ इति॒ काण्डे᳚ - का॒ण्डे॒ । वै क्रि॒यमा॑णे । क्रि॒यमा॑णे य॒ज्ञ्म् । य॒ज्ञ्ꣳ रक्षाꣳ॑सि । रक्षाꣳ॑सि जिघाꣳसन्ति । जि॒घाꣳ॒॒स॒न्ति॒ परि॑लिखितम् । परि॑लिखितꣳ॒॒ रक्षः॑ । परि॑लिखित॒मिति॒ परि॑ - लि॒खि॒त॒म् । रक्षः॒ परि॑लिखिताः । परि॑लिखिता॒ अरा॑तयः । परि॑लिखिता॒ इति॒ परि॑ - लि॒खि॒ताः॒ । अरा॑तय॒ इति॑ । इत्या॑ह । आ॒ह॒ रक्ष॑साम् । रक्ष॑सा॒मप॑हत्यै । अप॑हत्या इ॒दम् । अप॑हत्या॒ इत्यप॑ - ह॒त्यै॒ \newline

\textbf{Jatai Paata} \newline

1. इत्या॑हा॒हे तीत्या॑ह । \newline
2. आ॒ह॒ यद् यदा॑ हाह॒ यत् । \newline
3. यदे॒वैव यद् यदे॒व । \newline
4. ए॒वास्या॑ अस्या ए॒वै वास्यै᳚ । \newline
5. अ॒स्यै॒ प॒दात् प॒दा द॑स्या अस्यै प॒दात् । \newline
6. प॒दाद् घृ॒तम् घृ॒तम् प॒दात् प॒दाद् घृ॒तम् । \newline
7. घृ॒त मपी᳚ड्य॒ तापी᳚ड्यत घृ॒तम् घृ॒त मपी᳚ड्यत । \newline
8. अपी᳚ड्यत॒ तस्मा॒त् तस्मा॒ दपी᳚ड्य॒ तापी᳚ड्यत॒ तस्मा᳚त् । \newline
9. तस्मा॑ दे॒व मे॒वम् तस्मा॒त् तस्मा॑ दे॒वम् । \newline
10. ए॒व मा॑हा है॒व मे॒व मा॑ह । \newline
11. आ॒ह॒ यद् यदा॑हाह॒ यत् । \newline
12. यद॑द्ध्व॒र्यु र॑द्ध्व॒र्युर् यद् यद॑द्ध्व॒र्युः । \newline
13. अ॒द्ध्व॒र्यु र॑न॒ग्ना व॑न॒ग्ना व॑द्ध्व॒र्यु र॑द्ध्व॒र्यु र॑न॒ग्नौ । \newline
14. अ॒न॒ग्ना वाहु॑ति॒ माहु॑ति मन॒ग्ना व॑न॒ग्ना वाहु॑तिम् । \newline
15. आहु॑तिम् जुहु॒याज् जु॑हु॒या दाहु॑ति॒ माहु॑तिम् जुहु॒यात् । \newline
16. आहु॑ति॒मित्या - हु॒ति॒म् । \newline
17. जु॒हु॒या द॒न्धो᳚ ऽन्धो जु॑हु॒याज् जु॑हु॒या द॒न्धः । \newline
18. अ॒न्धो᳚ ऽद्ध्व॒र्यु र॑द्ध्व॒र्यु र॒न्धो᳚(1॒) ऽन्धो᳚ ऽद्ध्व॒र्युः । \newline
19. अ॒द्ध्व॒र्युः स्या᳚थ् स्यादद्ध्व॒र्यु र॑द्ध्व॒र्युः स्या᳚त् । \newline
20. स्या॒द् रक्षाꣳ॑सि॒ रक्षाꣳ॑सि स्याथ् स्या॒द् रक्षाꣳ॑सि । \newline
21. रक्षाꣳ॑सि य॒ज्ञ्ं ॅय॒ज्ञ्ꣳ रक्षाꣳ॑सि॒ रक्षाꣳ॑सि य॒ज्ञ्म् । \newline
22. य॒ज्ञ्ꣳ ह॑न्युर्. हन्युर् य॒ज्ञ्ं ॅय॒ज्ञ्ꣳ ह॑न्युः । \newline
23. ह॒न्यु॒र्॒. हिर॑ण्यꣳ॒॒ हिर॑ण्यꣳ हन्युर्. हन्यु॒र्॒. हिर॑ण्यम् । \newline
24. हिर॑ण्य मु॒पा स्यो॒पास्य॒ हिर॑ण्यꣳ॒॒ हिर॑ण्य मु॒पास्य॑ । \newline
25. उ॒पास्य॑ जुहोति जुहो त्यु॒पा स्यो॒पास्य॑ जुहोति । \newline
26. उ॒पास्येत्यु॑प - अस्य॑ । \newline
27. जु॒हो॒ त्य॒ग्नि॒व त्य॑ग्नि॒वति॑ जुहोति जुहो त्यग्नि॒वति॑ । \newline
28. अ॒ग्नि॒व त्ये॒वै वाग्नि॒व त्य॑ग्नि॒व त्ये॒व । \newline
29. अ॒ग्नि॒वतीत्य॑ग्नि - वति॑ । \newline
30. ए॒व जु॑होति जुहो त्ये॒वैव जु॑होति । \newline
31. जु॒हो॒ति॒ न न जु॑होति जुहोति॒ न । \newline
32. नान्धो᳚ ऽन्धो न नान्धः । \newline
33. अ॒न्धो᳚ ऽद्ध्व॒र्यु र॑द्ध्व॒र्यु र॒न्धो᳚(1॒) ऽन्धो᳚ ऽद्ध्व॒र्युः । \newline
34. अ॒द्ध्व॒र्युर् भव॑ति॒ भव॑ त्यद्ध्व॒र्यु र॑द्ध्व॒र्युर् भव॑ति । \newline
35. भव॑ति॒ न न भव॑ति॒ भव॑ति॒ न । \newline
36. न य॒ज्ञ्ं ॅय॒ज्ञ्न्न न य॒ज्ञ्म् । \newline
37. य॒ज्ञ्ꣳ रक्षाꣳ॑सि॒ रक्षाꣳ॑सि य॒ज्ञ्ं ॅय॒ज्ञ्ꣳ रक्षाꣳ॑सि । \newline
38. रक्षाꣳ॑सि घ्नन्ति घ्नन्ति॒ रक्षाꣳ॑सि॒ रक्षाꣳ॑सि घ्नन्ति । \newline
39. घ्न॒न्ति॒ काण्डे॑काण्डे॒ काण्डे॑काण्डे घ्नन्ति घ्नन्ति॒ काण्डे॑काण्डे । \newline
40. काण्डे॑काण्डे॒ वै वै काण्डे॑काण्डे॒ काण्डे॑काण्डे॒ वै । \newline
41. काण्डे॑काण्ड॒ इति॒ काण्डे᳚ - का॒ण्डे॒ । \newline
42. वै क्रि॒यमा॑णे क्रि॒यमा॑णे॒ वै वै क्रि॒यमा॑णे । \newline
43. क्रि॒यमा॑णे य॒ज्ञ्ं ॅय॒ज्ञ्म् क्रि॒यमा॑णे क्रि॒यमा॑णे य॒ज्ञ्म् । \newline
44. य॒ज्ञ्ꣳ रक्षाꣳ॑सि॒ रक्षाꣳ॑सि य॒ज्ञ्ं ॅय॒ज्ञ्ꣳ रक्षाꣳ॑सि । \newline
45. रक्षाꣳ॑सि जिघाꣳसन्ति जिघाꣳसन्ति॒ रक्षाꣳ॑सि॒ रक्षाꣳ॑सि जिघाꣳसन्ति । \newline
46. जि॒घाꣳ॒॒स॒न्ति॒ परि॑लिखित॒म् परि॑लिखितम् जिघाꣳसन्ति जिघाꣳसन्ति॒ परि॑लिखितम् । \newline
47. परि॑लिखितꣳ॒॒ रक्षो॒ रक्षः॒ परि॑लिखित॒म् परि॑लिखितꣳ॒॒ रक्षः॑ । \newline
48. परि॑लिखित॒मिति॒ परि॑ - लि॒खि॒त॒म् । \newline
49. रक्षः॒ परि॑लिखिताः॒ परि॑लिखिता॒ रक्षो॒ रक्षः॒ परि॑लिखिताः । \newline
50. परि॑लिखिता॒ अरा॑त॒यो ऽरा॑तयः॒ परि॑लिखिताः॒ परि॑लिखिता॒ अरा॑तयः । \newline
51. परि॑लिखिता॒ इति॒ परि॑ - लि॒खि॒ताः॒ । \newline
52. अरा॑तय॒ इतीत्य रा॑त॒यो ऽरा॑तय॒ इति॑ । \newline
53. इत्या॑हा॒हे तीत्या॑ह । \newline
54. आ॒ह॒ रक्ष॑साꣳ॒॒ रक्ष॑सा माहाह॒ रक्ष॑साम् । \newline
55. रक्ष॑सा॒ मप॑हत्या॒ अप॑हत्यै॒ रक्ष॑साꣳ॒॒ रक्ष॑सा॒ मप॑हत्यै । \newline
56. अप॑हत्या इ॒द मि॒द मप॑हत्या॒ अप॑हत्या इ॒दम् । \newline
57. अप॑हत्या॒ इत्यप॑ - ह॒त्यै॒ । \newline

\textbf{Ghana Paata } \newline

1. इत्या॑हा॒हे तीत्या॑ह॒ यद् यदा॒हे तीत्या॑ह॒ यत् । \newline
2. आ॒ह॒ यद् यदा॑हाह॒ यदे॒ वैव यदा॑ हाह॒ यदे॒व । \newline
3. यदे॒ वैव यद् यदे॒ वास्या॑ अस्या ए॒व यद् यदे॒ वास्यै᳚ । \newline
4. ए॒वास्या॑ अस्या ए॒वै वास्यै॑ प॒दात् प॒दा द॑स्या ए॒वै वास्यै॑ प॒दात् । \newline
5. अ॒स्यै॒ प॒दात् प॒दा द॑स्या अस्यै प॒दाद् घृ॒तम् घृ॒तम् प॒दा द॑स्या अस्यै प॒दाद् घृ॒तम् । \newline
6. प॒दाद् घृ॒तम् घृ॒तम् प॒दात् प॒दाद् घृ॒त मपी᳚ड्य॒ता पी᳚ड्यत घृ॒तम् प॒दात् प॒दाद् घृ॒त मपी᳚ड्यत । \newline
7. घृ॒त मपी᳚ड्य॒ता पी᳚ड्यत घृ॒तम् घृ॒त मपी᳚ड्यत॒ तस्मा॒त् तस्मा॒ दपी᳚ड्यत घृ॒तम् घृ॒त मपी᳚ड्यत॒ तस्मा᳚त् । \newline
8. अपी᳚ड्यत॒ तस्मा॒त् तस्मा॒ दपी᳚ड्य॒ता पी᳚ड्यत॒ तस्मा॑ दे॒व मे॒वम् तस्मा॒ दपी᳚ड्य॒ता पी᳚ड्यत॒ तस्मा॑ दे॒वम् । \newline
9. तस्मा॑ दे॒व मे॒वम् तस्मा॒त् तस्मा॑ दे॒व मा॑हा है॒वम् तस्मा॒त् तस्मा॑ दे॒व मा॑ह । \newline
10. ए॒व मा॑हा है॒व मे॒व मा॑ह॒ यद् यदा॑ है॒व मे॒व मा॑ह॒ यत् । \newline
11. आ॒ह॒ यद् यदा॑हाह॒ यद॑द्ध्व॒र्यु र॑द्ध्व॒र्युर् यदा॑हाह॒ यद॑द्ध्व॒र्युः । \newline
12. यद॑द्ध्व॒र्यु र॑द्ध्व॒र्युर् यद् यद॑द्ध्व॒र्यु र॑न॒ग्ना व॑न॒ग्ना व॑द्ध्व॒र्युर् यद् यद॑द्ध्व॒र्यु र॑न॒ग्नौ । \newline
13. अ॒द्ध्व॒र्यु र॑न॒ग्ना व॑न॒ग्ना व॑द्ध्व॒र्यु र॑द्ध्व॒र्यु र॑न॒ग्ना वाहु॑ति॒ माहु॑ति मन॒ग्ना व॑द्ध्व॒र्यु र॑द्ध्व॒र्यु र॑न॒ग्ना वाहु॑तिम् । \newline
14. अ॒न॒ग्ना वाहु॑ति॒ माहु॑ति मन॒ग्ना व॑न॒ग्ना वाहु॑तिम् जुहु॒याज् जु॑हु॒या दाहु॑ति मन॒ग्ना व॑न॒ग्ना वाहु॑तिम् जुहु॒यात् । \newline
15. आहु॑तिम् जुहु॒याज् जु॑हु॒या दाहु॑ति॒ माहु॑तिम् जुहु॒या द॒न्धो᳚ ऽन्धो जु॑हु॒या दाहु॑ति॒ माहु॑तिम् जुहु॒या द॒न्धः । \newline
16. आहु॑ति॒मित्या - हु॒ति॒म् । \newline
17. जु॒हु॒या द॒न्धो᳚ ऽन्धो जु॑हु॒याज् जु॑हु॒या द॒न्धो᳚ ऽद्ध्व॒र्यु र॑द्ध्व॒र्यु र॒न्धो जु॑हु॒याज् जु॑हु॒या द॒न्धो᳚ ऽद्ध्व॒र्युः । \newline
18. अ॒न्धो᳚ ऽद्ध्व॒र्यु र॑द्ध्व॒र्यु र॒न्धो᳚(1॒) ऽन्धो᳚ ऽद्ध्व॒र्युः स्या᳚थ् स्या दद्ध्व॒र्यु र॒न्धो᳚(1॒) ऽन्धो᳚ ऽद्ध्व॒र्युः स्या᳚त् । \newline
19. अ॒द्ध्व॒र्युः स्या᳚थ् स्या दद्ध्व॒र्यु र॑द्ध्व॒र्युः स्या॒द् रक्षाꣳ॑सि॒ रक्षाꣳ॑सि स्या दद्ध्व॒र्यु र॑द्ध्व॒र्युः स्या॒द् रक्षाꣳ॑सि । \newline
20. स्या॒द् रक्षाꣳ॑सि॒ रक्षाꣳ॑सि स्याथ् स्या॒द् रक्षाꣳ॑सि य॒ज्ञ्ं ॅय॒ज्ञ्ꣳ रक्षाꣳ॑सि स्याथ् स्या॒द् रक्षाꣳ॑सि य॒ज्ञ्म् । \newline
21. रक्षाꣳ॑सि य॒ज्ञ्ं ॅय॒ज्ञ्ꣳ रक्षाꣳ॑सि॒ रक्षाꣳ॑सि य॒ज्ञ्ꣳ ह॑न्युर्. हन्युर् य॒ज्ञ्ꣳ रक्षाꣳ॑सि॒ रक्षाꣳ॑सि य॒ज्ञ्ꣳ ह॑न्युः । \newline
22. य॒ज्ञ्ꣳ ह॑न्युर्. हन्युर् य॒ज्ञ्ं ॅय॒ज्ञ्ꣳ ह॑न्यु॒र्॒. हिर॑ण्यꣳ॒॒ हिर॑ण्यꣳ हन्युर् य॒ज्ञ्ं ॅय॒ज्ञ्ꣳ ह॑न्यु॒र्॒. हिर॑ण्यम् । \newline
23. ह॒न्यु॒र्॒. हिर॑ण्यꣳ॒॒ हिर॑ण्यꣳ हन्युर्. हन्यु॒र्॒. हिर॑ण्य मु॒पा स्यो॒पास्य॒ हिर॑ण्यꣳ हन्युर्. हन्यु॒र्॒. हिर॑ण्य मु॒पास्य॑ । \newline
24. हिर॑ण्य मु॒पा स्यो॒पास्य॒ हिर॑ण्यꣳ॒॒ हिर॑ण्य मु॒पास्य॑ जुहोति जुहो त्यु॒पास्य॒ हिर॑ण्यꣳ॒॒ हिर॑ण्य मु॒पास्य॑ जुहोति । \newline
25. उ॒पास्य॑ जुहोति जुहो त्यु॒पा स्यो॒पास्य॑ जुहो त्यग्नि॒व त्य॑ग्नि॒वति॑ जुहो त्यु॒पा स्यो॒पास्य॑ जुहो त्यग्नि॒वति॑ । \newline
26. उ॒पास्येत्यु॑प - अस्य॑ । \newline
27. जु॒हो॒ त्य॒ग्नि॒व त्य॑ग्नि॒वति॑ जुहोति जुहो त्यग्नि॒व त्ये॒वैवा ग्नि॒वति॑ जुहोति जुहो त्यग्नि॒व त्ये॒व । \newline
28. अ॒ग्नि॒व त्ये॒वैवा ग्नि॒व त्य॑ग्नि॒व त्ये॒व जु॑होति जुहो त्ये॒वा ग्नि॒व त्य॑ग्नि॒व त्ये॒व जु॑होति । \newline
29. अ॒ग्नि॒वतीत्य॑ग्नि - वति॑ । \newline
30. ए॒व जु॑होति जुहो त्ये॒वैव जु॑होति॒ न न जु॑हो त्ये॒वैव जु॑होति॒ न । \newline
31. जु॒हो॒ति॒ न न जु॑होति जुहोति॒ नान्धो᳚ ऽन्धो न जु॑होति जुहोति॒ नान्धः । \newline
32. नान्धो᳚ ऽन्धो न नान्धो᳚ ऽद्ध्व॒र्यु र॑द्ध्व॒र्यु र॒न्धो न नान्धो᳚ ऽद्ध्व॒र्युः । \newline
33. अ॒न्धो᳚ ऽद्ध्व॒र्यु र॑द्ध्व॒र्यु र॒न्धो᳚(1॒) ऽन्धो᳚ ऽद्ध्व॒र्युर् भव॑ति॒ भव॑ त्यद्ध्व॒र्यु र॒न्धो᳚(1॒) ऽन्धो᳚ ऽद्ध्व॒र्युर् भव॑ति । \newline
34. अ॒द्ध्व॒र्युर् भव॑ति॒ भव॑ त्यद्ध्व॒र्यु र॑द्ध्व॒र्युर् भव॑ति॒ न न भव॑ त्यद्ध्व॒र्यु र॑द्ध्व॒र्युर् भव॑ति॒ न । \newline
35. भव॑ति॒ न न भव॑ति॒ भव॑ति॒ न य॒ज्ञ्ं ॅय॒ज्ञ्न्न भव॑ति॒ भव॑ति॒ न य॒ज्ञ्म् । \newline
36. न य॒ज्ञ्ं ॅय॒ज्ञ्न्न न य॒ज्ञ्ꣳ रक्षाꣳ॑सि॒ रक्षाꣳ॑सि य॒ज्ञ्न्न न य॒ज्ञ्ꣳ रक्षाꣳ॑सि । \newline
37. य॒ज्ञ्ꣳ रक्षाꣳ॑सि॒ रक्षाꣳ॑सि य॒ज्ञ्ं ॅय॒ज्ञ्ꣳ रक्षाꣳ॑सि घ्नन्ति घ्नन्ति॒ रक्षाꣳ॑सि य॒ज्ञ्ं ॅय॒ज्ञ्ꣳ रक्षाꣳ॑सि घ्नन्ति । \newline
38. रक्षाꣳ॑सि घ्नन्ति घ्नन्ति॒ रक्षाꣳ॑सि॒ रक्षाꣳ॑सि घ्नन्ति॒ काण्डे॑काण्डे॒ काण्डे॑काण्डे घ्नन्ति॒ रक्षाꣳ॑सि॒ रक्षाꣳ॑सि घ्नन्ति॒ काण्डे॑काण्डे । \newline
39. घ्न॒न्ति॒ काण्डे॑काण्डे॒ काण्डे॑काण्डे घ्नन्ति घ्नन्ति॒ काण्डे॑काण्डे॒ वै वै काण्डे॑काण्डे घ्नन्ति घ्नन्ति॒ काण्डे॑काण्डे॒ वै । \newline
40. काण्डे॑काण्डे॒ वै वै काण्डे॑काण्डे॒ काण्डे॑काण्डे॒ वै क्रि॒यमा॑णे क्रि॒यमा॑णे॒ वै काण्डे॑काण्डे॒ काण्डे॑काण्डे॒ वै क्रि॒यमा॑णे । \newline
41. काण्डे॑काण्ड॒ इति॒ काण्डे᳚ - का॒ण्डे॒ । \newline
42. वै क्रि॒यमा॑णे क्रि॒यमा॑णे॒ वै वै क्रि॒यमा॑णे य॒ज्ञ्ं ॅय॒ज्ञ्म् क्रि॒यमा॑णे॒ वै वै क्रि॒यमा॑णे य॒ज्ञ्म् । \newline
43. क्रि॒यमा॑णे य॒ज्ञ्ं ॅय॒ज्ञ्म् क्रि॒यमा॑णे क्रि॒यमा॑णे य॒ज्ञ्ꣳ रक्षाꣳ॑सि॒ रक्षाꣳ॑सि य॒ज्ञ्म् क्रि॒यमा॑णे क्रि॒यमा॑णे य॒ज्ञ्ꣳ रक्षाꣳ॑सि । \newline
44. य॒ज्ञ्ꣳ रक्षाꣳ॑सि॒ रक्षाꣳ॑सि य॒ज्ञ्ं ॅय॒ज्ञ्ꣳ रक्षाꣳ॑सि जिघाꣳसन्ति जिघाꣳसन्ति॒ रक्षाꣳ॑सि य॒ज्ञ्ं ॅय॒ज्ञ्ꣳ रक्षाꣳ॑सि जिघाꣳसन्ति । \newline
45. रक्षाꣳ॑सि जिघाꣳसन्ति जिघाꣳसन्ति॒ रक्षाꣳ॑सि॒ रक्षाꣳ॑सि जिघाꣳसन्ति॒ परि॑लिखित॒म् परि॑लिखितम् जिघाꣳसन्ति॒ रक्षाꣳ॑सि॒ रक्षाꣳ॑सि जिघाꣳसन्ति॒ परि॑लिखितम् । \newline
46. जि॒घाꣳ॒॒स॒न्ति॒ परि॑लिखित॒म् परि॑लिखितम् जिघाꣳसन्ति जिघाꣳसन्ति॒ परि॑लिखितꣳ॒॒ रक्षो॒ रक्षः॒ परि॑लिखितम् जिघाꣳसन्ति जिघाꣳसन्ति॒ परि॑लिखितꣳ॒॒ रक्षः॑ । \newline
47. परि॑लिखितꣳ॒॒ रक्षो॒ रक्षः॒ परि॑लिखित॒म् परि॑लिखितꣳ॒॒ रक्षः॒ परि॑लिखिताः॒ परि॑लिखिता॒ रक्षः॒ परि॑लिखित॒म् परि॑लिखितꣳ॒॒ रक्षः॒ परि॑लिखिताः । \newline
48. परि॑लिखित॒मिति॒ परि॑ - लि॒खि॒त॒म् । \newline
49. रक्षः॒ परि॑लिखिताः॒ परि॑लिखिता॒ रक्षो॒ रक्षः॒ परि॑लिखिता॒ अरा॑त॒यो ऽरा॑तयः॒ परि॑लिखिता॒ रक्षो॒ रक्षः॒ परि॑लिखिता॒ अरा॑तयः । \newline
50. परि॑लिखिता॒ अरा॑त॒यो ऽरा॑तयः॒ परि॑लिखिताः॒ परि॑लिखिता॒ अरा॑तय॒ इतीत्य रा॑तयः॒ परि॑लिखिताः॒ परि॑लिखिता॒ अरा॑तय॒ इति॑ । \newline
51. परि॑लिखिता॒ इति॒ परि॑ - लि॒खि॒ताः॒ । \newline
52. अरा॑तय॒ इतीत्य रा॑त॒यो ऽरा॑तय॒ इत्या॑हा॒हे त्यरा॑त॒यो ऽरा॑तय॒ इत्या॑ह । \newline
53. इत्या॑हा॒हे तीत्या॑ह॒ रक्ष॑साꣳ॒॒ रक्ष॑सा मा॒हे तीत्या॑ह॒ रक्ष॑साम् । \newline
54. आ॒ह॒ रक्ष॑साꣳ॒॒ रक्ष॑सा माहाह॒ रक्ष॑सा॒ मप॑हत्या॒ अप॑हत्यै॒ रक्ष॑सा माहाह॒ रक्ष॑सा॒ मप॑हत्यै । \newline
55. रक्ष॑सा॒ मप॑हत्या॒ अप॑हत्यै॒ रक्ष॑साꣳ॒॒ रक्ष॑सा॒ मप॑हत्या इ॒द मि॒द मप॑हत्यै॒ रक्ष॑साꣳ॒॒ रक्ष॑सा॒ मप॑हत्या इ॒दम् । \newline
56. अप॑हत्या इ॒द मि॒द मप॑हत्या॒ अप॑हत्या इ॒द म॒ह म॒ह मि॒द मप॑हत्या॒ अप॑हत्या इ॒द म॒हम् । \newline
57. अप॑हत्या॒ इत्यप॑ - ह॒त्यै॒ । \newline
\pagebreak
\markright{ TS 6.1.8.4  \hfill https://www.vedavms.in \hfill}

\section{ TS 6.1.8.4 }

\textbf{TS 6.1.8.4 } \newline
\textbf{Samhita Paata} \newline

इ॒दम॒हꣳ रक्ष॑सो ग्री॒वा अपि॑ कृन्तामि॒ यो᳚ऽस्मान् द्वेष्टि॒ यं च॑ व॒यं द्वि॒ष्म इत्या॑ह॒ द्वौ वाव पुरु॑षौ॒ यं चै॒व द्वेष्टि॒ यश्चैनं॒ द्वेष्टि॒ तयो॑रे॒वान॑न्तरायं ग्री॒वाः कृ॑न्तति प॒शवो॒ वै सो॑म॒क्रय॑ण्यै प॒दं ॅया॑वत्त्मू॒तꣳ सं ॅव॑पति प॒शूने॒वाव॑ रुन्धे॒ऽस्मे राय॒ इति॒ सं ॅव॑पत्या॒त्मान॑-मे॒वाद्ध्व॒र्युः - [  ] \newline

\textbf{Pada Paata} \newline

इ॒दम् । अ॒हम् । रक्ष॑सः । ग्री॒वाः । अपीति॑ । कृ॒न्ता॒मि॒ । यः । अ॒स्मान् । द्वेष्टि॑ । यम् । च॒ । व॒यम् । द्वि॒ष्मः । इति॑ । आ॒ह॒ । द्वौ । वाव । पुरु॑षौ । यम् । च॒ । ए॒व । द्वेष्टि॑ । यः । च॒ । ए॒न॒म् । द्वेष्टि॑ । तयोः᳚ । ए॒व । अन॑न्तराय॒मित्यन॑न्तः - आ॒य॒म् । ग्री॒वाः । कृ॒न्त॒ति॒ । प॒शवः॑ । वै । सो॒म॒क्रय॑ण्या॒ इति॑ सोम - क्रय॑ण्यै । प॒दम् । या॒वत्त्मू॒तमिति॑ यावत् - त्मू॒तम् । समिति॑ । व॒प॒ति॒ । प॒शून् । ए॒व । अवेति॑ । रु॒न्धे॒ । अ॒स्मे इति॑ । रायः॑ । इति॑ । समिति॑ । व॒प॒ति॒ । आ॒त्मान᳚म् । ए॒व । अ॒द्ध्व॒र्युः ।  \newline


\textbf{Krama Paata} \newline

इ॒दम॒हम् । अ॒हꣳ रक्ष॑सः । रक्ष॑सो ग्री॒वाः । ग्री॒वा अपि॑ । अपि॑ कृन्तामि । कृ॒न्ता॒मि॒ यः । यो᳚ऽस्मान् । अ॒स्मान् द्वेष्टि॑ । द्वेष्टि॒ यम् । यम् च॑ । च॒ व॒यम् । व॒यम् द्वि॒ष्मः । द्वि॒ष्म इति॑ । इत्या॑ह । आ॒ह॒ द्वौ । द्वौ वाव । वाव पुरु॑षौ । पुरु॑षौ॒ यम् । यम् च॑ । चै॒व । ए॒व द्वेष्टि॑ । द्वेष्टि॒ यः । यश्च॑ । चै॒न॒म् । ए॒न॒म् द्वेष्टि॑ । द्वेष्टि॒ तयोः᳚ । तयो॑रे॒व । ए॒वान॑न्तरायम् । अन॑न्तरायम् ग्री॒वाः । अन॑न्तराय॒मित्यन॑न्तः - आ॒य॒म् । ग्री॒वाः कृ॑न्तति । कृ॒न्त॒ति॒ प॒शवः॑ । प॒शवो॒ वै । वै सो॑म॒क्रय॑ण्यै । सो॒म॒क्रय॑ण्यै प॒दम् । सो॒म॒क्रय॑ण्या॒ इति॑ सोम - क्रय॑ण्यै । प॒दम् ॅया॑वत्‌त्मू॒तम् । या॒व॒त्‌त्मू॒तꣳ सम् । या॒व॒त्‌त्मू॒तमिति॑ यावत् - त्मू॒तम् । सम् ॅव॑पति । व॒प॒ति॒ प॒शून् । प॒शूने॒व । ए॒वाव॑ । अव॑ रुन्धे । रु॒न्धे॒ऽस्मे । अ॒स्मे रायः॑ । अ॒स्मे इत्य॒स्मे । राय॒ इति॑ । इति॒ सम् । सम् ॅव॑पति । व॒प॒त्या॒त्मान᳚म् । आ॒त्मान॑मे॒व । ए॒वाद्ध्व॒र्युः । अ॒द्ध्व॒र्युः प॒शुभ्यः॑ \newline

\textbf{Jatai Paata} \newline

1. इ॒द म॒ह म॒ह मि॒द मि॒द म॒हम् । \newline
2. अ॒हꣳ रक्ष॑सो॒ रक्ष॑सो॒ ऽह म॒हꣳ रक्ष॑सः । \newline
3. रक्ष॑सो ग्री॒वा ग्री॒वा रक्ष॑सो॒ रक्ष॑सो ग्री॒वाः । \newline
4. ग्री॒वा अप्यपि॑ ग्री॒वा ग्री॒वा अपि॑ । \newline
5. अपि॑ कृन्तामि कृन्ता॒ म्यप्यपि॑ कृन्तामि । \newline
6. कृ॒न्ता॒मि॒ यो यः कृ॑न्तामि कृन्तामि॒ यः । \newline
7. यो᳚ ऽस्मान॒ स्मान्. यो यो᳚ ऽस्मान् । \newline
8. अ॒स्मान् द्वेष्टि॒ द्वेष्ट्य॒स्मा न॒स्मान् द्वेष्टि॑ । \newline
9. द्वेष्टि॒ यं ॅयम् द्वेष्टि॒ द्वेष्टि॒ यम् । \newline
10. यम् च॑ च॒ यं ॅयम् च॑ । \newline
11. च॒ व॒यं ॅव॒यम् च॑ च व॒यम् । \newline
12. व॒यम् द्वि॒ष्मो द्वि॒ष्मो व॒यं ॅव॒यम् द्वि॒ष्मः । \newline
13. द्वि॒ष्म इतीति॑ द्वि॒ष्मो द्वि॒ष्म इति॑ । \newline
14. इत्या॑हा॒हे तीत्या॑ह । \newline
15. आ॒ह॒ द्वौ द्वा वा॑हाह॒ द्वौ । \newline
16. द्वौ वाव वाव द्वौ द्वौ वाव । \newline
17. वाव पुरु॑षौ॒ पुरु॑षौ॒ वाव वाव पुरु॑षौ । \newline
18. पुरु॑षौ॒ यं ॅयम् पुरु॑षौ॒ पुरु॑षौ॒ यम् । \newline
19. यम् च॑ च॒ यं ॅयम् च॑ । \newline
20. चै॒वैव च॑ चै॒व । \newline
21. ए॒व द्वेष्टि॒ द्वेष्ट्ये॒ वैव द्वेष्टि॑ । \newline
22. द्वेष्टि॒ यो यो द्वेष्टि॒ द्वेष्टि॒ यः । \newline
23. यश्च॑ च॒ यो यश्च॑ । \newline
24. चै॒न॒ मे॒न॒म् च॒ चै॒न॒म् । \newline
25. ए॒न॒म् द्वेष्टि॒ द्वेष्ट्ये॑न मेन॒म् द्वेष्टि॑ । \newline
26. द्वेष्टि॒ तयो॒ स्तयो॒र् द्वेष्टि॒ द्वेष्टि॒ तयोः᳚ । \newline
27. तयो॑ रे॒वैव तयो॒ स्तयो॑ रे॒व । \newline
28. ए॒वा न॑न्तराय॒ मन॑न्तराय मे॒वै वान॑न्तरायम् । \newline
29. अन॑न्तरायम् ग्री॒वा ग्री॒वा अन॑न्तराय॒ मन॑न्तरायम् ग्री॒वाः । \newline
30. अन॑न्तराय॒मित्यन॑न्तः - आ॒य॒म् । \newline
31. ग्री॒वाः कृ॑न्तति कृन्तति ग्री॒वा ग्री॒वाः कृ॑न्तति । \newline
32. कृ॒न्त॒ति॒ प॒शवः॑ प॒शवः॑ कृन्तति कृन्तति प॒शवः॑ । \newline
33. प॒शवो॒ वै वै प॒शवः॑ प॒शवो॒ वै । \newline
34. वै सो॑म॒क्रय॑ण्यै सोम॒क्रय॑ण्यै॒ वै वै सो॑म॒क्रय॑ण्यै । \newline
35. सो॒म॒क्रय॑ण्यै प॒दम् प॒दꣳ सो॑म॒क्रय॑ण्यै सोम॒क्रय॑ण्यै प॒दम् । \newline
36. सो॒म॒क्रय॑ण्या॒ इति॑ सोम - क्रय॑ण्यै । \newline
37. प॒दं ॅया॑वत्त्मू॒तं ॅया॑वत्त्मू॒तम् प॒दम् प॒दं ॅया॑वत्त्मू॒तम् । \newline
38. या॒व॒त्त्मू॒तꣳ सꣳ सं ॅया॑वत्त्मू॒तं ॅया॑वत्त्मू॒तꣳ सम् । \newline
39. या॒व॒त्त्मू॒तमिति॑ यावत् - त्मू॒तम् । \newline
40. सं ॅव॑पति वपति॒ सꣳ सं ॅव॑पति । \newline
41. व॒प॒ति॒ प॒शून् प॒शून्. व॑पति वपति प॒शून् । \newline
42. प॒शूने॒ वैव प॒शून् प॒शूने॒व । \newline
43. ए॒वावा वै॒वै वाव॑ । \newline
44. अव॑ रुन्धे रु॒न्धे ऽवाव॑ रुन्धे । \newline
45. रु॒न्धे॒ ऽस्मे अ॒स्मे रु॑न्धे रुन्धे॒ ऽस्मे । \newline
46. अ॒स्मे रायो॒ रायो॒ ऽस्मे अ॒स्मे रायः॑ । \newline
47. अ॒स्मे इत्य॒स्मे । \newline
48. राय॒ इतीति॒ रायो॒ राय॒ इति॑ । \newline
49. इति॒ सꣳ स मितीति॒ सम् । \newline
50. सं ॅव॑पति वपति॒ सꣳ सं ॅव॑पति । \newline
51. व॒प॒ त्या॒त्मान॑ मा॒त्मानं॑ ॅवपति वप त्या॒त्मान᳚म् । \newline
52. आ॒त्मान॑ मे॒वै वात्मान॑ मा॒त्मान॑ मे॒व । \newline
53. ए॒वाद्ध्व॒र्यु र॑द्ध्व॒र्यु रे॒वै वाद्ध्व॒र्युः । \newline
54. अ॒द्ध्व॒र्युः प॒शुभ्यः॑ प॒शुभ्यो᳚ ऽद्ध्व॒र्यु र॑द्ध्व॒र्युः प॒शुभ्यः॑ । \newline

\textbf{Ghana Paata } \newline

1. इ॒द म॒ह म॒ह मि॒द मि॒द म॒हꣳ रक्ष॑सो॒ रक्ष॑सो॒ ऽह मि॒द मि॒द म॒हꣳ रक्ष॑सः । \newline
2. अ॒हꣳ रक्ष॑सो॒ रक्ष॑सो॒ ऽह म॒हꣳ रक्ष॑सो ग्री॒वा ग्री॒वा रक्ष॑सो॒ ऽह म॒हꣳ रक्ष॑सो ग्री॒वाः । \newline
3. रक्ष॑सो ग्री॒वा ग्री॒वा रक्ष॑सो॒ रक्ष॑सो ग्री॒वा अप्यपि॑ ग्री॒वा रक्ष॑सो॒ रक्ष॑सो ग्री॒वा अपि॑ । \newline
4. ग्री॒वा अप्यपि॑ ग्री॒वा ग्री॒वा अपि॑ कृन्तामि कृन्ता॒म्यपि॑ ग्री॒वा ग्री॒वा अपि॑ कृन्तामि । \newline
5. अपि॑ कृन्तामि कृन्ता॒ म्यप्यपि॑ कृन्तामि॒ यो यः कृ॑न्ता॒ म्यप्यपि॑ कृन्तामि॒ यः । \newline
6. कृ॒न्ता॒मि॒ यो यः कृ॑न्तामि कृन्तामि॒ यो᳚ ऽस्मा न॒स्मान्. यः कृ॑न्तामि कृन्तामि॒ यो᳚ ऽस्मान् । \newline
7. यो᳚ ऽस्मा न॒स्मान्. यो यो᳚ ऽस्मान् द्वेष्टि॒ द्वेष्ट्य॒स्मान्. यो यो᳚ ऽस्मान् द्वेष्टि॑ । \newline
8. अ॒स्मान् द्वेष्टि॒ द्वेष्ट्य॒स्मा न॒स्मान् द्वेष्टि॒ यं ॅयम् द्वेष्ट्य॒स्मा न॒स्मान् द्वेष्टि॒ यम् । \newline
9. द्वेष्टि॒ यं ॅयम् द्वेष्टि॒ द्वेष्टि॒ यम् च॑ च॒ यम् द्वेष्टि॒ द्वेष्टि॒ यम् च॑ । \newline
10. यम् च॑ च॒ यं ॅयम् च॑ व॒यं ॅव॒यम् च॒ यं ॅयम् च॑ व॒यम् । \newline
11. च॒ व॒यं ॅव॒यम् च॑ च व॒यम् द्वि॒ष्मो द्वि॒ष्मो व॒यम् च॑ च व॒यम् द्वि॒ष्मः । \newline
12. व॒यम् द्वि॒ष्मो द्वि॒ष्मो व॒यं ॅव॒यम् द्वि॒ष्म इतीति॑ द्वि॒ष्मो व॒यं ॅव॒यम् द्वि॒ष्म इति॑ । \newline
13. द्वि॒ष्म इतीति॑ द्वि॒ष्मो द्वि॒ष्म इत्या॑हा॒ हेति॑ द्वि॒ष्मो द्वि॒ष्म इत्या॑ह । \newline
14. इत्या॑हा॒हे तीत्या॑ह॒ द्वौ द्वा वा॒हे तीत्या॑ह॒ द्वौ । \newline
15. आ॒ह॒ द्वौ द्वा वा॑हाह॒ द्वौ वाव वाव द्वा वा॑हाह॒ द्वौ वाव । \newline
16. द्वौ वाव वाव द्वौ द्वौ वाव पुरु॑षौ॒ पुरु॑षौ॒ वाव द्वौ द्वौ वाव पुरु॑षौ । \newline
17. वाव पुरु॑षौ॒ पुरु॑षौ॒ वाव वाव पुरु॑षौ॒ यं ॅयम् पुरु॑षौ॒ वाव वाव पुरु॑षौ॒ यम् । \newline
18. पुरु॑षौ॒ यं ॅयम् पुरु॑षौ॒ पुरु॑षौ॒ यम् च॑ च॒ यम् पुरु॑षौ॒ पुरु॑षौ॒ यम् च॑ । \newline
19. यम् च॑ च॒ यं ॅयम् चै॒वैव च॒ यं ॅयम् चै॒व । \newline
20. चै॒वैव च॑ चै॒व द्वेष्टि॒ द्वेष्ट्ये॒व च॑ चै॒व द्वेष्टि॑ । \newline
21. ए॒व द्वेष्टि॒ द्वेष्ट्ये॒ वैव द्वेष्टि॒ यो यो द्वेष्ट्ये॒ वैव द्वेष्टि॒ यः । \newline
22. द्वेष्टि॒ यो यो द्वेष्टि॒ द्वेष्टि॒ यश्च॑ च॒ यो द्वेष्टि॒ द्वेष्टि॒ यश्च॑ । \newline
23. यश्च॑ च॒ यो यश्चै॑न मेनम् च॒ यो यश्चै॑नम् । \newline
24. चै॒न॒ मे॒न॒म् च॒ चै॒न॒म् द्वेष्टि॒ द्वेष्ट्ये॑नम् च चैन॒म् द्वेष्टि॑ । \newline
25. ए॒न॒म् द्वेष्टि॒ द्वेष्ट्ये॑न मेन॒म् द्वेष्टि॒ तयो॒ स्तयो॒र् द्वेष्ट्ये॑न मेन॒म् द्वेष्टि॒ तयोः᳚ । \newline
26. द्वेष्टि॒ तयो॒ स्तयो॒र् द्वेष्टि॒ द्वेष्टि॒ तयो॑ रे॒वैव तयो॒र् द्वेष्टि॒ द्वेष्टि॒ तयो॑ रे॒व । \newline
27. तयो॑ रे॒वैव तयो॒ स्तयो॑ रे॒वान॑न्तराय॒ मन॑न्तराय मे॒व तयो॒ स्तयो॑ रे॒वान॑न्तरायम् । \newline
28. ए॒वान॑न्तराय॒ मन॑न्तराय मे॒वैवा न॑न्तरायम् ग्री॒वा ग्री॒वा अन॑न्तराय मे॒वैवा न॑न्तरायम् ग्री॒वाः । \newline
29. अन॑न्तरायम् ग्री॒वा ग्री॒वा अन॑न्तराय॒ मन॑न्तरायम् ग्री॒वाः कृ॑न्तति कृन्तति ग्री॒वा अन॑न्तराय॒ मन॑न्तरायम् ग्री॒वाः कृ॑न्तति । \newline
30. अन॑न्तराय॒मित्यन॑न्तः - आ॒य॒म् । \newline
31. ग्री॒वाः कृ॑न्तति कृन्तति ग्री॒वा ग्री॒वाः कृ॑न्तति प॒शवः॑ प॒शवः॑ कृन्तति ग्री॒वा ग्री॒वाः कृ॑न्तति प॒शवः॑ । \newline
32. कृ॒न्त॒ति॒ प॒शवः॑ प॒शवः॑ कृन्तति कृन्तति प॒शवो॒ वै वै प॒शवः॑ कृन्तति कृन्तति प॒शवो॒ वै । \newline
33. प॒शवो॒ वै वै प॒शवः॑ प॒शवो॒ वै सो॑म॒क्रय॑ण्यै सोम॒क्रय॑ण्यै॒ वै प॒शवः॑ प॒शवो॒ वै सो॑म॒क्रय॑ण्यै । \newline
34. वै सो॑म॒क्रय॑ण्यै सोम॒क्रय॑ण्यै॒ वै वै सो॑म॒क्रय॑ण्यै प॒दम् प॒दꣳ सो॑म॒क्रय॑ण्यै॒ वै वै सो॑म॒क्रय॑ण्यै प॒दम् । \newline
35. सो॒म॒क्रय॑ण्यै प॒दम् प॒दꣳ सो॑म॒क्रय॑ण्यै सोम॒क्रय॑ण्यै प॒दं ॅया॑वत्त्मू॒तं ॅया॑वत्त्मू॒तम् प॒दꣳ सो॑म॒क्रय॑ण्यै सोम॒क्रय॑ण्यै प॒दं ॅया॑वत्त्मू॒तम् । \newline
36. सो॒म॒क्रय॑ण्या॒ इति॑ सोम - क्रय॑ण्यै । \newline
37. प॒दं ॅया॑वत्त्मू॒तं ॅया॑वत्त्मू॒तम् प॒दम् प॒दं ॅया॑वत्त्मू॒तꣳ सꣳ सं ॅया॑वत्त्मू॒तम् प॒दम् प॒दं ॅया॑वत्त्मू॒तꣳ सम् । \newline
38. या॒व॒त्त्मू॒तꣳ सꣳ सं ॅया॑वत्त्मू॒तं ॅया॑वत्त्मू॒तꣳ सं ॅव॑पति वपति॒ सं ॅया॑वत्त्मू॒तं ॅया॑वत्त्मू॒तꣳ सं ॅव॑पति । \newline
39. या॒व॒त्त्मू॒तमिति॑ यावत् - त्मू॒तम् । \newline
40. सं ॅव॑पति वपति॒ सꣳ सं ॅव॑पति प॒शून् प॒शून्. व॑पति॒ सꣳ सं ॅव॑पति प॒शून् । \newline
41. व॒प॒ति॒ प॒शून् प॒शून्. व॑पति वपति प॒शू ने॒वैव प॒शून्. व॑पति वपति प॒शूने॒व । \newline
42. प॒शू ने॒वैव प॒शून् प॒शू ने॒वावा वै॒व प॒शून् प॒शूने॒ वाव॑ । \newline
43. ए॒वावा वै॒वै वाव॑ रुन्धे रु॒न्धे ऽवै॒वै वाव॑ रुन्धे । \newline
44. अव॑ रुन्धे रु॒न्धे ऽवाव॑ रुन्धे॒ ऽस्मे अ॒स्मे रु॒न्धे ऽवाव॑ रुन्धे॒ ऽस्मे । \newline
45. रु॒न्धे॒ ऽस्मे अ॒स्मे रु॑न्धे रुन्धे॒ ऽस्मे रायो॒ रायो॒ ऽस्मे रु॑न्धे रुन्धे॒ ऽस्मे रायः॑ । \newline
46. अ॒स्मे रायो॒ रायो॒ ऽस्मे अ॒स्मे राय॒ इतीति॒ रायो॒ ऽस्मे अ॒स्मे राय॒ इति॑ । \newline
47. अ॒स्मे इत्य॒स्मे । \newline
48. राय॒ इतीति॒ रायो॒ राय॒ इति॒ सꣳ स मिति॒ रायो॒ राय॒ इति॒ सम् । \newline
49. इति॒ सꣳ स मितीति॒ सं ॅव॑पति वपति॒ स मितीति॒ सं ॅव॑पति । \newline
50. सं ॅव॑पति वपति॒ सꣳ सं ॅव॑प त्या॒त्मान॑ मा॒त्मानं॑ ॅवपति॒ सꣳ सं ॅव॑प त्या॒त्मान᳚म् । \newline
51. व॒प॒ त्या॒त्मान॑ मा॒त्मानं॑ ॅवपति वप त्या॒त्मान॑ मे॒वैवा त्मानं॑ ॅवपति वप त्या॒त्मान॑ मे॒व । \newline
52. आ॒त्मान॑ मे॒वैवा त्मान॑ मा॒त्मान॑ मे॒वाद्ध्व॒र्यु र॑द्ध्व॒र्यु रे॒वात्मान॑ मा॒त्मान॑ मे॒वाद्ध्व॒र्युः । \newline
53. ए॒वाद्ध्व॒र्यु र॑द्ध्व॒र्यु रे॒वै वाद्ध्व॒र्युः प॒शुभ्यः॑ प॒शुभ्यो᳚ ऽद्ध्व॒र्यु रे॒वै वाद्ध्व॒र्युः प॒शुभ्यः॑ । \newline
54. अ॒द्ध्व॒र्युः प॒शुभ्यः॑ प॒शुभ्यो᳚ ऽद्ध्व॒र्यु र॑द्ध्व॒र्युः प॒शुभ्यो॒ न न प॒शुभ्यो᳚ ऽद्ध्व॒र्यु र॑द्ध्व॒र्युः प॒शुभ्यो॒ न । \newline
\pagebreak
\markright{ TS 6.1.8.5  \hfill https://www.vedavms.in \hfill}

\section{ TS 6.1.8.5 }

\textbf{TS 6.1.8.5 } \newline
\textbf{Samhita Paata} \newline

प॒शुभ्यो॒ नान्तरे॑ति॒ त्वे राय॒ इति॒ यज॑मानाय॒ प्र य॑च्छति॒ यज॑मान ए॒व र॒यिं द॑धाति॒ तोते॒ राय॒ इति॒ पत्नि॑या अ॒र्द्धो वा ए॒ष आ॒त्मनो॒ यत् पत्नी॒ यथा॑ गृ॒हेषु॑ निध॒त्ते ता॒दृगे॒व तत् त्वष्टी॑मती ते सपे॒येत्या॑ह॒ त्वष्टा॒ वै प॑शू॒नां मि॑थु॒नानाꣳ॑ रूप॒कृद्-रू॒पमे॒व प॒शुषु॑ दधात्य॒स्मै वै लो॒काय॒ गार्.ह॑पत्य॒ आ धी॑यते॒ ( ) ऽमुष्मा॑ आहव॒नीयो॒ यद्-गार्.ह॑पत्य उप॒वपे॑द॒स्मिन् ॅलो॒के प॑श॒मान्थ् स्या॒द्-यदा॑हव॒नीये॒ ऽमुष्मि॑न् ॅलो॒के प॑शु॒मान्थ् स्या॑दु॒भयो॒रुप॑ वपत्यु॒भयो॑रे॒वैनं॑ ॅलो॒कयोः᳚ पशु॒मन्तं॑ करोति ॥ \newline

\textbf{Pada Paata} \newline

प॒शुभ्य॒ इति॑ प॒शु - भ्यः॒ । न । अ॒न्तः । ए॒ति॒ । त्वे इति॑ । रायः॑ । इति॑ । यज॑मानाय । प्रेति॑ । य॒च्छ॒ति॒ । यज॑माने । ए॒व । र॒यिम् । द॒धा॒ति॒ । तोते᳚ । रायः॑ । इति॑ । पत्नि॑यै । अ॒द्‌र्धः । वै । ए॒षः । आ॒त्मनः॑ । यत् । पत्नी᳚ । यथा᳚ । गृ॒हेषु॑ । नि॒ध॒त्त इति॑ नि - ध॒त्ते । ता॒दृक् । ए॒व । तत् । त्वष्टी॑मती । ते॒ । स॒पे॒य॒ । इति॑ । आ॒ह॒ । त्वष्टा᳚ । वै । प॒शू॒नाम् । मि॒थु॒नाना᳚म् । रू॒प॒कृदिति॑ रूप - कृत् । रू॒पम् । ए॒व । प॒शुषु॑ । द॒धा॒ति॒ । अ॒स्मै । वै । लो॒काय॑ । गार्.ह॑पत्य॒ इति॒ गार्.ह॑ - प॒त्यः॒ । एति॑ । धी॒य॒ते॒ ( ) । अ॒मुष्मै᳚ । आ॒ह॒व॒नीय॒ इत्या᳚ - ह॒व॒नीयः॑ । यत् । गार्.ह॑पत्य॒ इति॒ गार्.ह॑- प॒त्ये॒ । उ॒प॒वपे॒दित्यु॑प - वपे᳚त् । अ॒स्मिन्न् । लो॒के । प॒शु॒मानिति॑ पशु - मान् । स्या॒त् । यत् । आ॒ह॒व॒नीय॒ इत्या᳚ - ह॒व॒नीये᳚ । अ॒मुष्मिन्न्॑ । लो॒के । प॒शु॒मानिति॑ पशु - मान् । स्या॒त् । उ॒भयोः᳚ । उपेति॑ । व॒प॒ति॒ । उ॒भयोः᳚ । ए॒व । ए॒न॒म् । लो॒कयोः᳚ । प॒शु॒मन्त॒मिति॑ पशु - मन्त᳚म् । क॒रो॒ति॒ ॥  \newline


\textbf{Krama Paata} \newline

प॒शुभ्यो॒ न । प॒शुभ्य॒ इति॑ प॒शु - भ्यः॒ । नान्तः । अ॒न्तरे॑ति । ए॒ति॒ त्वे । त्वे रायः॑ । त्वे इति॒ त्वे । राय॒ इति॑ । इति॒ यज॑मनाय । यज॑मानाय॒ प्र । प्र य॑च्छति । य॒च्छ॒ति॒ यज॑माने । यज॑मान ए॒व । ए॒व र॒यिम् । र॒यिम् द॑धाति । द॒धा॒ति॒ तोते᳚ । तोते॒ रायः॑ । राय॒ इति॑ । इति॒ पत्नि॑यै । पत्नि॑या अ॒र्द्धः । अ॒र्द्धो वै । वा ए॒षः । ए॒ष आ॒त्मनः॑ । आ॒त्मनो॒ यत् । यत् पत्नी᳚ । पत्नी॒ यथा᳚ । यथा॑ गृ॒हेषु॑ । गृ॒हेषु॑ निध॒त्ते । नि॒ध॒त्ते ता॒दृक् । नि॒ध॒त्त इति॑ नि - ध॒त्ते । ता॒दृगे॒व । ए॒व तत् । तत् त्वष्टी॑मती । त्वष्टी॑मती ते । ते॒ स॒पे॒य॒ । स॒पे॒येति॑ । इत्या॑ह । आ॒ह॒ त्वष्टा᳚ । त्वष्टा॒ वै । वै प॑शू॒नाम् । प॒शू॒नाम् मि॑थु॒नाना᳚म् । मि॒थु॒नानाꣳ॑ रूप॒कृत् । रू॒प॒कृद् रू॒पम् । रू॒प॒कृदिति॑ रूप - कृत् । रू॒पमे॒व । ए॒व प॒शुषु॑ । प॒शषु॑ दधाति । द॒धा॒त्य॒स्मै । अ॒स्मै वै । वै लो॒काय॑ । लो॒काय॒ गार्.ह॑पत्यः । गार्.ह॑पत्य॒ आ । गार्.ह॑पत्य॒ इति॒ गार्.ह॑ - प॒त्यः॒ । आ धी॑यते ( ) । धी॒य॒ते॒ऽमुष्मै᳚ । अ॒मुष्मा॑ आहव॒नीयः॑ । आ॒ह॒व॒नीयो॒ यत् । आ॒ह॒व॒नीय॒ इत्या᳚ - ह॒व॒नीयः॑ । यद् गार्.ह॑पत्ये । गार्.ह॑पत्य उप॒वपे᳚त् । गार्.ह॑पत्य॒ इति॒ गार्.ह॑ - प॒त्ये॒ । उ॒प॒वपे॑द॒स्मिन्न् । उ॒प॒वपे॒दित्यु॑प - वपे᳚त् । अ॒स्मिन् ॅलो॒के । लो॒के प॑शु॒मान् । प॒शु॒मान्थ् स्या᳚त् । प॒शु॒मानिति॑ पशु - मान् । स्या॒द् यत् । यदा॑हव॒नीये᳚ । आ॒ह॒व॒नीये॒ऽमुष्मिन्न्॑ । आ॒ह॒व॒नीय॒ इत्या᳚ - ह॒व॒नीये᳚ । अ॒मुष्मि॑न् ॅलो॒के । लो॒के प॑शु॒मान् । प॒शु॒मान्थ् स्या᳚त् । प॒शु॒मानिति॑ पशु - मान् । स्या॒दु॒भयोः᳚ । उ॒भयो॒रुप॑ । उप॑ वपति । व॒प॒त्यु॒भयोः᳚ । उ॒भयो॑रे॒व । ए॒वैन᳚म् । ए॒न॒म् ॅलो॒कयोः᳚ । लो॒कयोः᳚ पशु॒मन्त᳚म् । प॒शु॒मन्त॑म् करोति । प॒शु॒मन्त॒मिति॑ पशु - मन्त᳚म् । क॒रो॒तीति॑ करोति । \newline

\textbf{Jatai Paata} \newline

1. प॒शुभ्यो॒ न न प॒शुभ्यः॑ प॒शुभ्यो॒ न । \newline
2. प॒शुभ्य॒ इति॑ प॒शु - भ्यः॒ । \newline
3. नान्त र॒न्तर् न नान्तः । \newline
4. अ॒न्त रे᳚त्ये त्य॒न्त र॒न्त रे॑ति । \newline
5. ए॒ति॒ त्वे त्वे ए᳚त्येति॒ त्वे । \newline
6. त्वे रायो॒ राय॒ स्त्वे त्वे रायः॑ । \newline
7. त्वे इति॒ त्वे । \newline
8. राय॒ इतीति॒ रायो॒ राय॒ इति॑ । \newline
9. इति॒ यज॑मानाय॒ यज॑माना॒ये तीति॒ यज॑मानाय । \newline
10. यज॑मानाय॒ प्र प्र यज॑मानाय॒ यज॑मानाय॒ प्र । \newline
11. प्र य॑च्छति यच्छति॒ प्र प्र य॑च्छति । \newline
12. य॒च्छ॒ति॒ यज॑माने॒ यज॑माने यच्छति यच्छति॒ यज॑माने । \newline
13. यज॑मान ए॒वैव यज॑माने॒ यज॑मान ए॒व । \newline
14. ए॒व र॒यिꣳ र॒यि मे॒वैव र॒यिम् । \newline
15. र॒यिम् द॑धाति दधाति र॒यिꣳ र॒यिम् द॑धाति । \newline
16. द॒धा॒ति॒ तोते॒ तोते॑ दधाति दधाति॒ तोते᳚ । \newline
17. तोते॒ रायो॒ राय॒ स्तोते॒ तोते॒ रायः॑ । \newline
18. राय॒ इतीति॒ रायो॒ राय॒ इति॑ । \newline
19. इति॒ पत्नि॑यै॒ पत्नि॑या॒ इतीति॒ पत्नि॑यै । \newline
20. पत्नि॑या अ॒र्द्धो᳚ ऽर्द्धः पत्नि॑यै॒ पत्नि॑या अ॒र्द्धः । \newline
21. अ॒र्द्धो वै वा अ॒र्द्धो᳚ ऽर्द्धो वै । \newline
22. वा ए॒ष ए॒ष वै वा ए॒षः । \newline
23. ए॒ष आ॒त्मन॑ आ॒त्मन॑ ए॒ष ए॒ष आ॒त्मनः॑ । \newline
24. आ॒त्मनो॒ यद् यदा॒त्मन॑ आ॒त्मनो॒ यत् । \newline
25. यत् पत्नी॒ पत्नी॒ यद् यत् पत्नी᳚ । \newline
26. पत्नी॒ यथा॒ यथा॒ पत्नी॒ पत्नी॒ यथा᳚ । \newline
27. यथा॑ गृ॒हेषु॑ गृ॒हेषु॒ यथा॒ यथा॑ गृ॒हेषु॑ । \newline
28. गृ॒हेषु॑ निध॒त्ते नि॑ध॒त्ते गृ॒हेषु॑ गृ॒हेषु॑ निध॒त्ते । \newline
29. नि॒ध॒त्ते ता॒दृक् ता॒दृङ् नि॑ध॒त्ते नि॑ध॒त्ते ता॒दृक् । \newline
30. नि॒ध॒त्त इति॑ नि - ध॒त्ते । \newline
31. ता॒दृ गे॒वैव ता॒दृक् ता॒दृ गे॒व । \newline
32. ए॒व तत् तदे॒ वैव तत् । \newline
33. तत् त्वष्टी॑मती॒ त्वष्टी॑मती॒ तत् तत् त्वष्टी॑मती । \newline
34. त्वष्टी॑मती ते ते॒ त्वष्टी॑मती॒ त्वष्टी॑मती ते । \newline
35. ते॒ स॒पे॒य॒ स॒पे॒य॒ ते॒ ते॒ स॒पे॒य॒ । \newline
36. स॒पे॒ये तीति॑ सपेय सपे॒ये ति॑ । \newline
37. इत्या॑हा॒हे तीत्या॑ह । \newline
38. आ॒ह॒ त्वष्टा॒ त्वष्टा॑ ऽऽहाह॒ त्वष्टा᳚ । \newline
39. त्वष्टा॒ वै वै त्वष्टा॒ त्वष्टा॒ वै । \newline
40. वै प॑शू॒नाम् प॑शू॒नां ॅवै वै प॑शू॒नाम् । \newline
41. प॒शू॒नाम् मि॑थु॒नाना᳚म् मिथु॒नाना᳚म् पशू॒नाम् प॑शू॒नाम् मि॑थु॒नाना᳚म् । \newline
42. मि॒थु॒नानाꣳ॑ रूप॒कृद् रू॑प॒कृन् मि॑थु॒नाना᳚म् मिथु॒नानाꣳ॑ रूप॒कृत् । \newline
43. रू॒प॒कृद् रू॒पꣳ रू॒पꣳ रू॑प॒कृद् रू॑प॒कृद् रू॒पम् । \newline
44. रू॒प॒कृदिति॑ रूप - कृत् । \newline
45. रू॒प मे॒वैव रू॒पꣳ रू॒प मे॒व । \newline
46. ए॒व प॒शुषु॑ प॒शुष्वे॒ वैव प॒शुषु॑ । \newline
47. प॒शुषु॑ दधाति दधाति प॒शुषु॑ प॒शुषु॑ दधाति । \newline
48. द॒धा॒ त्य॒स्मा अ॒स्मै द॑धाति दधा त्य॒स्मै । \newline
49. अ॒स्मै वै वा अ॒स्मा अ॒स्मै वै । \newline
50. वै लो॒काय॑ लो॒काय॒ वै वै लो॒काय॑ । \newline
51. लो॒काय॒ गार्.ह॑पत्यो॒ गार्.ह॑पत्यो लो॒काय॑ लो॒काय॒ गार्.ह॑पत्यः । \newline
52. गार्.ह॑पत्य॒ आ गार्.ह॑पत्यो॒ गार्.ह॑पत्य॒ आ । \newline
53. गार्.ह॑पत्य॒ इति॒ गार्.ह॑ - प॒त्यः॒ । \newline
54. आ धी॑यते धीयत॒ आ धी॑यते । \newline
55. धी॒य॒ते॒ ऽमुष्मा॑ अ॒मुष्मै॑ धीयते धीयते॒ ऽमुष्मै᳚ । \newline
56. अ॒मुष्मा॑ आहव॒नीय॑ आहव॒नीयो॒ ऽमुष्मा॑ अ॒मुष्मा॑ आहव॒नीयः॑ । \newline
57. आ॒ह॒व॒नीयो॒ यद् यदा॑हव॒नीय॑ आहव॒नीयो॒ यत् । \newline
58. आ॒ह॒व॒नीय॒ इत्या᳚ - ह॒व॒नीयः॑ । \newline
59. यद् गार्.ह॑पत्ये॒ गार्.ह॑पत्ये॒ यद् यद् गार्.ह॑पत्ये । \newline
60. गार्.ह॑पत्य उप॒वपे॑ दुप॒वपे॒द् गार्.ह॑पत्ये॒ गार्.ह॑पत्य उप॒वपे᳚त् । \newline
61. गार्.ह॑पत्य॒ इति॒ गार्.ह॑ - प॒त्ये॒ । \newline
62. उ॒प॒वपे॑ द॒स्मिन् न॒स्मिन् नु॑प॒वपे॑ दुप॒वपे॑ द॒स्मिन्न् । \newline
63. उ॒प॒वपे॒दित्यु॑प - वपे᳚त् । \newline
64. अ॒स्मिन् ॅलो॒के लो॒के᳚ ऽस्मिन् न॒स्मिन् ॅलो॒के । \newline
65. लो॒के प॑शु॒मान् प॑शु॒मान् ॅलो॒के लो॒के प॑शु॒मान् । \newline
66. प॒शु॒मान् थ्स्या᳚थ् स्यात् पशु॒मान् प॑शु॒मान् थ्स्या᳚त् । \newline
67. प॒शु॒मानिति॑ पशु - मान् । \newline
68. स्या॒द् यद् यथ् स्या᳚थ् स्या॒द् यत् । \newline
69. यदा॑हव॒नीय॑ आहव॒नीये॒ यद् यदा॑हव॒नीये᳚ । \newline
70. आ॒ह॒व॒नीये॒ ऽमुष्मि॑न् न॒मुष्मि॑न् नाहव॒नीय॑ आहव॒नीये॒ ऽमुष्मिन्न्॑ । \newline
71. आ॒ह॒व॒नीय॒ इत्या᳚ - ह॒व॒नीये᳚ । \newline
72. अ॒मुष्मि॑न् ॅलो॒के लो॒के॑ ऽमुष्मि॑न् न॒मुष्मि॑न् ॅलो॒के । \newline
73. लो॒के प॑शु॒मान् प॑शु॒मान् ॅलो॒के लो॒के प॑शु॒मान् । \newline
74. प॒शु॒मान् थ्स्या᳚थ् स्यात् पशु॒मान् प॑शु॒मान् थ्स्या᳚त् । \newline
75. प॒शु॒मानिति॑ पशु - मान् । \newline
76. स्या॒ दु॒भयो॑ रु॒भयोः᳚ स्याथ् स्या दु॒भयोः᳚ । \newline
77. उ॒भयो॒ रुपोपो॒ भयो॑ रु॒भयो॒ रुप॑ । \newline
78. उप॑ वपति वप॒ त्युपोप॑ वपति । \newline
79. व॒प॒ त्यु॒भयो॑ रु॒भयो᳚र् वपति वप त्यु॒भयोः᳚ । \newline
80. उ॒भयो॑ रे॒वै वोभयो॑ रु॒भयो॑ रे॒व । \newline
81. ए॒वैन॑ मेन मे॒वै वैन᳚म् । \newline
82. ए॒न॒म् ॅलो॒कयो᳚र् लो॒कयो॑ रेन मेनम् ॅलो॒कयोः᳚ । \newline
83. लो॒कयोः᳚ पशु॒मन्त॑म् पशु॒मन्त॑म् ॅलो॒कयो᳚र् लो॒कयोः᳚ पशु॒मन्त᳚म् । \newline
84. प॒शु॒मन्त॑म् करोति करोति पशु॒मन्त॑म् पशु॒मन्त॑म् करोति । \newline
85. प॒शु॒मन्त॒मिति॑ पशु - मन्त᳚म् । \newline
86. क॒रो॒तीति॑ करोति । \newline

\textbf{Ghana Paata } \newline

1. प॒शुभ्यो॒ न न प॒शुभ्यः॑ प॒शुभ्यो॒ नान्त र॒न्तर् न प॒शुभ्यः॑ प॒शुभ्यो॒ नान्तः । \newline
2. प॒शुभ्य॒ इति॑ प॒शु - भ्यः॒ । \newline
3. नान्त र॒न्तर् न नान्त रे᳚त्ये त्य॒न्तर् न नान्त रे॑ति । \newline
4. अ॒न्त रे᳚त्ये त्य॒न्त र॒न्त रे॑ति॒ त्वे त्वे ए᳚त्य॒न्त र॒न्त रे॑ति॒ त्वे । \newline
5. ए॒ति॒ त्वे त्वे ए᳚त्येति॒ त्वे रायो॒ राय॒ स्त्वे ए᳚त्येति॒ त्वे रायः॑ । \newline
6. त्वे रायो॒ राय॒ स्त्वे त्वे राय॒ इतीति॒ राय॒ स्त्वे त्वे राय॒ इति॑ । \newline
7. त्वे इति॒ त्वे । \newline
8. राय॒ इतीति॒ रायो॒ राय॒ इति॒ यज॑मानाय॒ यज॑माना॒येति॒ रायो॒ राय॒ इति॒ यज॑मानाय । \newline
9. इति॒ यज॑मानाय॒ यज॑माना॒ये तीति॒ यज॑मानाय॒ प्र प्र यज॑माना॒ये तीति॒ यज॑मानाय॒ प्र । \newline
10. यज॑मानाय॒ प्र प्र यज॑मानाय॒ यज॑मानाय॒ प्र य॑च्छति यच्छति॒ प्र यज॑मानाय॒ यज॑मानाय॒ प्र य॑च्छति । \newline
11. प्र य॑च्छति यच्छति॒ प्र प्र य॑च्छति॒ यज॑माने॒ यज॑माने यच्छति॒ प्र प्र य॑च्छति॒ यज॑माने । \newline
12. य॒च्छ॒ति॒ यज॑माने॒ यज॑माने यच्छति यच्छति॒ यज॑मान ए॒वैव यज॑माने यच्छति यच्छति॒ यज॑मान ए॒व । \newline
13. यज॑मान ए॒वैव यज॑माने॒ यज॑मान ए॒व र॒यिꣳ र॒यि मे॒व यज॑माने॒ यज॑मान ए॒व र॒यिम् । \newline
14. ए॒व र॒यिꣳ र॒यि मे॒वैव र॒यिम् द॑धाति दधाति र॒यि मे॒वैव र॒यिम् द॑धाति । \newline
15. र॒यिम् द॑धाति दधाति र॒यिꣳ र॒यिम् द॑धाति॒ तोते॒ तोते॑ दधाति र॒यिꣳ र॒यिम् द॑धाति॒ तोते᳚ । \newline
16. द॒धा॒ति॒ तोते॒ तोते॑ दधाति दधाति॒ तोते॒ रायो॒ राय॒ स्तोते॑ दधाति दधाति॒ तोते॒ रायः॑ । \newline
17. तोते॒ रायो॒ राय॒ स्तोते॒ तोते॒ राय॒ इतीति॒ राय॒ स्तोते॒ तोते॒ राय॒ इति॑ । \newline
18. राय॒ इतीति॒ रायो॒ राय॒ इति॒ पत्नि॑यै॒ पत्नि॑या॒ इति॒ रायो॒ राय॒ इति॒ पत्नि॑यै । \newline
19. इति॒ पत्नि॑यै॒ पत्नि॑या॒ इतीति॒ पत्नि॑या अ॒र्द्धो᳚ ऽर्द्धः पत्नि॑या॒ इतीति॒ पत्नि॑या अ॒र्द्धः । \newline
20. पत्नि॑या अ॒र्द्धो᳚ ऽर्द्धः पत्नि॑यै॒ पत्नि॑या अ॒र्द्धो वै वा अ॒र्द्धः पत्नि॑यै॒ पत्नि॑या अ॒र्द्धो वै । \newline
21. अ॒र्द्धो वै वा अ॒र्द्धो᳚ ऽर्द्धो वा ए॒ष ए॒ष वा अ॒र्द्धो᳚ ऽर्द्धो वा ए॒षः । \newline
22. वा ए॒ष ए॒ष वै वा ए॒ष आ॒त्मन॑ आ॒त्मन॑ ए॒ष वै वा ए॒ष आ॒त्मनः॑ । \newline
23. ए॒ष आ॒त्मन॑ आ॒त्मन॑ ए॒ष ए॒ष आ॒त्मनो॒ यद् यदा॒त्मन॑ ए॒ष ए॒ष आ॒त्मनो॒ यत् । \newline
24. आ॒त्मनो॒ यद् यदा॒त्मन॑ आ॒त्मनो॒ यत् पत्नी॒ पत्नी॒ यदा॒त्मन॑ आ॒त्मनो॒ यत् पत्नी᳚ । \newline
25. यत् पत्नी॒ पत्नी॒ यद् यत् पत्नी॒ यथा॒ यथा॒ पत्नी॒ यद् यत् पत्नी॒ यथा᳚ । \newline
26. पत्नी॒ यथा॒ यथा॒ पत्नी॒ पत्नी॒ यथा॑ गृ॒हेषु॑ गृ॒हेषु॒ यथा॒ पत्नी॒ पत्नी॒ यथा॑ गृ॒हेषु॑ । \newline
27. यथा॑ गृ॒हेषु॑ गृ॒हेषु॒ यथा॒ यथा॑ गृ॒हेषु॑ निध॒त्ते नि॑ध॒त्ते गृ॒हेषु॒ यथा॒ यथा॑ गृ॒हेषु॑ निध॒त्ते । \newline
28. गृ॒हेषु॑ निध॒त्ते नि॑ध॒त्ते गृ॒हेषु॑ गृ॒हेषु॑ निध॒त्ते ता॒दृक् ता॒दृङ् नि॑ध॒त्ते गृ॒हेषु॑ गृ॒हेषु॑ निध॒त्ते ता॒दृक् । \newline
29. नि॒ध॒त्ते ता॒दृक् ता॒दृङ् नि॑ध॒त्ते नि॑ध॒त्ते ता॒दृ गे॒वैव ता॒दृङ् नि॑ध॒त्ते नि॑ध॒त्ते ता॒दृ गे॒व । \newline
30. नि॒ध॒त्त इति॑ नि - ध॒त्ते । \newline
31. ता॒दृ गे॒वैव ता॒दृक् ता॒दृ गे॒व तत् तदे॒व ता॒दृक् ता॒दृ गे॒व तत् । \newline
32. ए॒व तत् तदे॒ वैव तत् त्वष्टी॑मती॒ त्वष्टी॑मती॒ तदे॒ वैव तत् त्वष्टी॑मती । \newline
33. तत् त्वष्टी॑मती॒ त्वष्टी॑मती॒ तत् तत् त्वष्टी॑मती ते ते॒ त्वष्टी॑मती॒ तत् तत् त्वष्टी॑मती ते । \newline
34. त्वष्टी॑मती ते ते॒ त्वष्टी॑मती॒ त्वष्टी॑मती ते सपेय सपेय ते॒ त्वष्टी॑मती॒ त्वष्टी॑मती ते सपेय । \newline
35. ते॒ स॒पे॒य॒ स॒पे॒य॒ ते॒ ते॒ स॒पे॒ये तीति॑ सपेय ते ते सपे॒येति॑ । \newline
36. स॒पे॒ये तीति॑ सपेय सपे॒ये त्या॑हा॒हेति॑ सपेय सपे॒ये त्या॑ह । \newline
37. इत्या॑हा॒हे तीत्या॑ह॒ त्वष्टा॒ त्वष्टा॒ ऽऽहे तीत्या॑ह॒ त्वष्टा᳚ । \newline
38. आ॒ह॒ त्वष्टा॒ त्वष्टा॑ ऽऽहाह॒ त्वष्टा॒ वै वै त्वष्टा॑ ऽऽहाह॒ त्वष्टा॒ वै । \newline
39. त्वष्टा॒ वै वै त्वष्टा॒ त्वष्टा॒ वै प॑शू॒नाम् प॑शू॒नां ॅवै त्वष्टा॒ त्वष्टा॒ वै प॑शू॒नाम् । \newline
40. वै प॑शू॒नाम् प॑शू॒नां ॅवै वै प॑शू॒नाम् मि॑थु॒नाना᳚म् मिथु॒नाना᳚म् पशू॒नां ॅवै वै प॑शू॒नाम् मि॑थु॒नाना᳚म् । \newline
41. प॒शू॒नाम् मि॑थु॒नाना᳚म् मिथु॒नाना᳚म् पशू॒नाम् प॑शू॒नाम् मि॑थु॒नानाꣳ॑ रूप॒कृद् रू॑प॒कृन् मि॑थु॒नाना᳚म् पशू॒नाम् प॑शू॒नाम् मि॑थु॒नानाꣳ॑ रूप॒कृत् । \newline
42. मि॒थु॒नानाꣳ॑ रूप॒कृद् रू॑प॒कृन् मि॑थु॒नाना᳚म् मिथु॒नानाꣳ॑ रूप॒कृद् रू॒पꣳ रू॒पꣳ रू॑प॒कृन् मि॑थु॒नाना᳚म् मिथु॒नानाꣳ॑ रूप॒कृद् रू॒पम् । \newline
43. रू॒प॒कृद् रू॒पꣳ रू॒पꣳ रू॑प॒कृद् रू॑प॒कृद् रू॒प मे॒वैव रू॒पꣳ रू॑प॒कृद् रू॑प॒कृद् रू॒प मे॒व । \newline
44. रू॒प॒कृदिति॑ रूप - कृत् । \newline
45. रू॒प मे॒वैव रू॒पꣳ रू॒प मे॒व प॒शुषु॑ प॒शुष्वे॒व रू॒पꣳ रू॒प मे॒व प॒शुषु॑ । \newline
46. ए॒व प॒शुषु॑ प॒शु ष्वे॒वैव प॒शुषु॑ दधाति दधाति प॒शु ष्वे॒वैव प॒शुषु॑ दधाति । \newline
47. प॒शुषु॑ दधाति दधाति प॒शुषु॑ प॒शुषु॑ दधा त्य॒स्मा अ॒स्मै द॑धाति प॒शुषु॑ प॒शुषु॑ दधा त्य॒स्मै । \newline
48. द॒धा॒ त्य॒स्मा अ॒स्मै द॑धाति दधा त्य॒स्मै वै वा अ॒स्मै द॑धाति दधा त्य॒स्मै वै । \newline
49. अ॒स्मै वै वा अ॒स्मा अ॒स्मै वै लो॒काय॑ लो॒काय॒ वा अ॒स्मा अ॒स्मै वै लो॒काय॑ । \newline
50. वै लो॒काय॑ लो॒काय॒ वै वै लो॒काय॒ गार्.ह॑पत्यो॒ गार्.ह॑पत्यो लो॒काय॒ वै वै लो॒काय॒ गार्.ह॑पत्यः । \newline
51. लो॒काय॒ गार्.ह॑पत्यो॒ गार्.ह॑पत्यो लो॒काय॑ लो॒काय॒ गार्.ह॑पत्य॒ आ गार्.ह॑पत्यो लो॒काय॑ लो॒काय॒ गार्.ह॑पत्य॒ आ । \newline
52. गार्.ह॑पत्य॒ आ गार्.ह॑पत्यो॒ गार्.ह॑पत्य॒ आ धी॑यते धीयत॒ आ गार्.ह॑पत्यो॒ गार्.ह॑पत्य॒ आ धी॑यते । \newline
53. गार्.ह॑पत्य॒ इति॒ गार्.ह॑ - प॒त्यः॒ । \newline
54. आ धी॑यते धीयत॒ आ धी॑यते॒ ऽमुष्मा॑ अ॒मुष्मै॑ धीयत॒ आ धी॑यते॒ ऽमुष्मै᳚ । \newline
55. धी॒य॒ते॒ ऽमुष्मा॑ अ॒मुष्मै॑ धीयते धीयते॒ ऽमुष्मा॑ आहव॒नीय॑ आहव॒नीयो॒ ऽमुष्मै॑ धीयते धीयते॒ ऽमुष्मा॑ आहव॒नीयः॑ । \newline
56. अ॒मुष्मा॑ आहव॒नीय॑ आहव॒नीयो॒ ऽमुष्मा॑ अ॒मुष्मा॑ आहव॒नीयो॒ यद् यदा॑हव॒नीयो॒ ऽमुष्मा॑ अ॒मुष्मा॑ आहव॒नीयो॒ यत् । \newline
57. आ॒ह॒व॒नीयो॒ यद् यदा॑हव॒नीय॑ आहव॒नीयो॒ यद् गार्.ह॑पत्ये॒ गार्.ह॑पत्ये॒ यदा॑हव॒नीय॑ आहव॒नीयो॒ यद् गार्.ह॑पत्ये । \newline
58. आ॒ह॒व॒नीय॒ इत्या᳚ - ह॒व॒नीयः॑ । \newline
59. यद् गार्.ह॑पत्ये॒ गार्.ह॑पत्ये॒ यद् यद् गार्.ह॑पत्य उप॒वपे॑ दुप॒वपे॒द् गार्.ह॑पत्ये॒ यद् यद् गार्.ह॑पत्य उप॒वपे᳚त् । \newline
60. गार्.ह॑पत्य उप॒वपे॑ दुप॒वपे॒द् गार्.ह॑पत्ये॒ गार्.ह॑पत्य उप॒वपे॑ द॒स्मिन् न॒स्मिन् नु॑प॒वपे॒द् गार्.ह॑पत्ये॒ गार्.ह॑पत्य उप॒वपे॑ द॒स्मिन्न् । \newline
61. गार्.ह॑पत्य॒ इति॒ गार्.ह॑ - प॒त्ये॒ । \newline
62. उ॒प॒वपे॑ द॒स्मिन् न॒स्मिन् नु॑प॒वपे॑ दुप॒वपे॑ द॒स्मिन् ॅलो॒के लो॒के᳚ ऽस्मिन् नु॑प॒वपे॑ दुप॒वपे॑ द॒स्मिन् ॅलो॒के । \newline
63. उ॒प॒वपे॒दित्यु॑प - वपे᳚त् । \newline
64. अ॒स्मिन् ॅलो॒के लो॒के᳚ ऽस्मिन् न॒स्मिन् ॅलो॒के प॑शु॒मान् प॑शु॒मान् ॅलो॒के᳚ ऽस्मिन् न॒स्मिन् ॅलो॒के प॑शु॒मान् । \newline
65. लो॒के प॑शु॒मान् प॑शु॒मान् ॅलो॒के लो॒के प॑शु॒मान् थ्स्या᳚थ् स्यात् पशु॒मान् ॅलो॒के लो॒के प॑शु॒मान् थ्स्या᳚त् । \newline
66. प॒शु॒मान् थ्स्या᳚थ् स्यात् पशु॒मान् प॑शु॒मान् थ्स्या॒द् यद् यथ् स्या᳚त् पशु॒मान् प॑शु॒मान् थ्स्या॒द् यत् । \newline
67. प॒शु॒मानिति॑ पशु - मान् । \newline
68. स्या॒द् यद् यथ् स्या᳚थ् स्या॒द् यदा॑हव॒नीय॑ आहव॒नीये॒ यथ् स्या᳚थ् स्या॒द् यदा॑हव॒नीये᳚ । \newline
69. यदा॑हव॒नीय॑ आहव॒नीये॒ यद् यदा॑हव॒नीये॒ ऽमुष्मि॑न् न॒मुष्मि॑न् नाहव॒नीये॒ यद् यदा॑हव॒नीये॒ ऽमुष्मिन्न्॑ । \newline
70. आ॒ह॒व॒नीये॒ ऽमुष्मि॑न् न॒मुष्मि॑न् नाहव॒नीय॑ आहव॒नीये॒ ऽमुष्मि॑न् ॅलो॒के लो॒के॑ ऽमुष्मि॑न् नाहव॒नीय॑ आहव॒नीये॒ ऽमुष्मि॑न् ॅलो॒के । \newline
71. आ॒ह॒व॒नीय॒ इत्या᳚ - ह॒व॒नीये᳚ । \newline
72. अ॒मुष्मि॑न् ॅलो॒के लो॒के॑ ऽमुष्मि॑न् न॒मुष्मि॑न् ॅलो॒के प॑शु॒मान् प॑शु॒मान् ॅलो॒के॑ ऽमुष्मि॑न् न॒मुष्मि॑न् ॅलो॒के प॑शु॒मान् । \newline
73. लो॒के प॑शु॒मान् प॑शु॒मान् ॅलो॒के लो॒के प॑शु॒मान् थ्स्या᳚थ् स्यात् पशु॒मान् ॅलो॒के लो॒के प॑शु॒मान् थ्स्या᳚त् । \newline
74. प॒शु॒मान् थ्स्या᳚थ् स्यात् पशु॒मान् प॑शु॒मान् थ्स्या॑ दु॒भयो॑ रु॒भयोः᳚ स्यात् पशु॒मान् प॑शु॒मान् थ्स्या॑ दु॒भयोः᳚ । \newline
75. प॒शु॒मानिति॑ पशु - मान् । \newline
76. स्या॒ दु॒भयो॑ रु॒भयोः᳚ स्याथ् स्या दु॒भयो॒ रुपोपो॒ भयोः᳚ स्याथ् स्या दु॒भयो॒ रुप॑ । \newline
77. उ॒भयो॒ रुपोपो॒ भयो॑ रु॒भयो॒ रुप॑ वपति वप॒ त्युपो॒ भयो॑ रु॒भयो॒ रुप॑ वपति । \newline
78. उप॑ वपति वप॒ त्युपोप॑ वप त्यु॒भयो॑ रु॒भयो᳚र् वप॒ त्युपोप॑ वप त्यु॒भयोः᳚ । \newline
79. व॒प॒ त्यु॒भयो॑ रु॒भयो᳚र् वपति वप त्यु॒भयो॑ रे॒वैवो भयो᳚र् वपति वप त्यु॒भयो॑ रे॒व । \newline
80. उ॒भयो॑ रे॒वैवो भयो॑ रु॒भयो॑ रे॒वैन॑ मेन मे॒वो भयो॑ रु॒भयो॑ रे॒वैन᳚म् । \newline
81. ए॒वैन॑ मेन मे॒वै वैन॑म् ॅलो॒कयो᳚र् लो॒कयो॑ रेन मे॒वै वैन॑म् ॅलो॒कयोः᳚ । \newline
82. ए॒न॒म् ॅलो॒कयो᳚र् लो॒कयो॑ रेन मेनम् ॅलो॒कयोः᳚ पशु॒मन्त॑म् पशु॒मन्त॑म् ॅलो॒कयो॑रेन मेनम् ॅलो॒कयोः᳚ पशु॒मन्त᳚म् । \newline
83. लो॒कयोः᳚ पशु॒मन्त॑म् पशु॒मन्त॑म् ॅलो॒कयो᳚र् लो॒कयोः᳚ पशु॒मन्त॑म् करोति करोति पशु॒मन्त॑म् ॅलो॒कयो᳚र् लो॒कयोः᳚ पशु॒मन्त॑म् करोति । \newline
84. प॒शु॒मन्त॑म् करोति करोति पशु॒मन्त॑म् पशु॒मन्त॑म् करोति । \newline
85. प॒शु॒मन्त॒मिति॑ पशु - मन्त᳚म् । \newline
86. क॒रो॒तीति॑ करोति । \newline
\pagebreak
\markright{ TS 6.1.9.1  \hfill https://www.vedavms.in \hfill}

\section{ TS 6.1.9.1 }

\textbf{TS 6.1.9.1 } \newline
\textbf{Samhita Paata} \newline

ब्र॒ह्म॒वा॒दिनो॑ वदन्ति वि॒चित्यः॒ सोमा(3) न वि॒चित्या(3) इति॒ सोमो॒ वा ओष॑धीनाꣳ॒॒ राजा॒ तस्मि॒न॒. यदाप॑न्नं ग्रसि॒तमे॒वास्य॒ तद्-यद्-वि॑चिनु॒याद्-यथा॒ ऽऽस्या᳚द्ग्रसि॒तं नि॑ष्खि॒दति॑ ता॒दृगे॒व तद्यन्न वि॑चिनु॒याद्-यथा॒ ऽक्षन्नाप॑न्नं ॅवि॒धाव॑ति ता॒दृगे॒व तत् क्षोधु॑को ऽद्ध्व॒र्युः स्यात् क्षोधु॑को॒ यज॑मानः॒ सोम॑विक्रयि॒न्थ् सोमꣳ॑ शोध॒येत्ये॒व ब्रू॑या॒द् यदीत॑रं॒ - [  ] \newline

\textbf{Pada Paata} \newline

ब्र॒ह्म॒वा॒दिन॒ इति॑ ब्रह्म - वा॒दिनः॑ । व॒द॒न्ति॒ । वि॒चित्य॒ इति॑ वि-चित्यः॑ । सोमा(3)ः । न । वि॒चित्या(3) इति वि - चित्या(3)ः । इति॑ । सोमः॑ । वै । ओष॑धीनाम् । राजा᳚ । तस्मिन्न्॑ । यत् । आप॑न्न॒मित्या - प॒न्न॒म् । ग्र॒सि॒तम् । ए॒व । अ॒स्य॒ । तत् । यत् । वि॒चि॒नु॒यादिति॑ वि - चि॒नु॒यात् । यथा᳚ । आ॒स्या᳚त् । ग्र॒सि॒तम् । नि॒ष्खि॒दतीति॑ निः -   खि॒दति॑ । ता॒दृक् । ए॒व । तत् । यत् । न । वि॒चि॒नु॒यादिति॑ वि - चि॒नु॒यात् । यथा᳚ । अ॒क्षन्न् । आप॑न्न॒मित्या - प॒न्न॒म् । वि॒धाव॒तीति॑ वि - धाव॑ति । ता॒दृक् । ए॒व । तत् । क्षोधु॑कः । अ॒द्ध्व॒र्युः । स्यात् । क्षोधु॑कः । यज॑मानः । सोम॑विक्रयि॒न्निति॒ सोम॑ -वि॒क्र॒यि॒न्न् । सोम᳚म् । शो॒ध॒य॒ । इति॑ । ए॒व । ब्रू॒या॒त् । यदि॑ । इत॑रम् ।  \newline


\textbf{Krama Paata} \newline

ब्र॒ह्म॒वा॒दिनो॑ वदन्ति । ब्र॒ह्म॒वा॒दिन॒ इति॑ ब्रह्म - वा॒दिनः॑ । व॒द॒न्ति॒ वि॒चित्यः॑ । वि॒चित्यः॒ सोमा(3)ः । वि॒चित्य॒ इति॑ वि - चित्यः॑ । सोमा(3) न । न वि॒चित्या(3)ः । वि॒चित्या(3) इति॑ । वि॒चित्या(3) इति॑ वि - चित्या (3)ः । इति॒ सोमः॑ । सोमो॒ वै । वा ओष॑धीनाम् । ओष॑धीनाꣳ॒॒ राजा᳚ । राजा॒ तस्मिन्न्॑ । तस्मि॒न्॒. यत् । यदाप॑न्नम् । आप॑न्नम् ग्रसि॒तम् । आप॑न्न॒मित्या - प॒न्न॒म् । ग्र॒सि॒तमे॒व । ए॒वास्य॑ । अ॒स्य॒ तत् । तद् यत् । यद् वि॑चिनु॒यात् । वि॒चि॒नु॒याद् यथा᳚ । वि॒चि॒नु॒यादिति॑ वि - चि॒नु॒यात् । यथा॒ऽऽस्या᳚त् । आ॒स्या᳚द् ग्रसि॒तम् । ग्र॒सि॒तम् नि॑ष्खि॒दति॑ । नि॒ष्खि॒दति॑ ता॒दृक् । नि॒ष्खि॒दतीति॑ निः - खि॒दति॑ । ता॒दृगे॒व । ए॒व तत् । तद् यत् । यन् न । न वि॑चिनु॒यात् । वि॒चि॒नु॒याद् यथा᳚ । वि॒चि॒नु॒यादिति॑ वि - चि॒नु॒यात् । यथा॒ऽक्षन्न् । अ॒क्षन्नाप॑न्नम् । आप॑न्नम् ॅवि॒धाव॑ति । आप॑न्न॒मित्या - प॒न्न॒म् । वि॒धाव॑ति ता॒दृक् । वि॒धाव॒तीति॑ वि - धाव॑ति । ता॒दृगे॒व । ए॒व तत् । तत् क्षोधु॑कः । क्षोधु॑कोऽद्ध्व॒र्युः । अ॒द्ध्व॒र्युः स्यात् । स्यात् क्षोधु॑कः । क्षोधु॑को॒ यज॑मानः । यज॑मानः॒ सोम॑विक्रयिन्न् । सोम॑विक्रयि॒न्थ् सोम᳚म् । सोम॑विक्रयि॒न्निति॒ सोम॑ - वि॒क्र॒यि॒न्न्॒ । सोमꣳ॑ शोधय । शो॒ध॒येति॑ । इत्ये॒व । ए॒व ब्रू॑यात् । ब्रू॒या॒द् यदि॑ । यदीत॑रम् । इत॑र॒म् ॅयदि॑ \newline

\textbf{Jatai Paata} \newline

1. ब्र॒ह्म॒वा॒दिनो॑ वदन्ति वदन्ति ब्रह्मवा॒दिनो᳚ ब्रह्मवा॒दिनो॑ वदन्ति । \newline
2. ब्र॒ह्म॒वा॒दिन॒ इति॑ ब्रह्म - वा॒दिनः॑ । \newline
3. व॒द॒न्ति॒ वि॒चित्यो॑ वि॒चित्यो॑ वदन्ति वदन्ति वि॒चित्यः॑ । \newline
4. वि॒चित्यः॒ सोमा(3)ः सोमा(3) वि॒चित्यो॑ वि॒चित्यः॒ सोमा(3)ः । \newline
5. वि॒चित्य॒ इति॑ वि - चित्यः॑ । \newline
6. सोमा(3) न न सोमा(3)ः सोमा(3) न । \newline
7. न वि॒चित्या(3) वि॒चित्या(3) न न वि॒चित्या(3)ः । \newline
8. वि॒चित्या(3) इतीति॑ वि॒चित्या(3) वि॒चित्या(3) इति॑ । \newline
9. वि॒चित्या(3) इति॑ वि - चित्या(3)ः । \newline
10. इति॒ सोमः॒ सोम॒ इतीति॒ सोमः॑ । \newline
11. सोमो॒ वै वै सोमः॒ सोमो॒ वै । \newline
12. वा ओष॑धीना॒ मोष॑धीनां॒ ॅवै वा ओष॑धीनाम् । \newline
13. ओष॑धीनाꣳ॒॒ राजा॒ राजौ ष॑धीना॒ मोष॑धीनाꣳ॒॒ राजा᳚ । \newline
14. राजा॒ तस्मिꣳ॒॒स् तस्मि॒न् राजा॒ राजा॒ तस्मिन्न्॑ । \newline
15. तस्मि॒न्॒. यद् यत् तस्मिꣳ॒॒स् तस्मि॒न्॒. यत् । \newline
16. यदाप॑न्न॒ माप॑न्नं॒ ॅयद् यदाप॑न्नम् । \newline
17. आप॑न्नम् ग्रसि॒तम् ग्र॑सि॒त माप॑न्न॒ माप॑न्नम् ग्रसि॒तम् । \newline
18. आप॑न्न॒मित्या - प॒न्न॒म् । \newline
19. ग्र॒सि॒त मे॒वैव ग्र॑सि॒तम् ग्र॑सि॒त मे॒व । \newline
20. ए॒वास्या᳚ स्यै॒वै वास्य॑ । \newline
21. अ॒स्य॒ तत् तद॑ स्यास्य॒ तत् । \newline
22. तद् यद् यत् तत् तद् यत् । \newline
23. यद् वि॑चिनु॒याद् वि॑चिनु॒याद् यद् यद् वि॑चिनु॒यात् । \newline
24. वि॒चि॒नु॒याद् यथा॒ यथा॑ विचिनु॒याद् वि॑चिनु॒याद् यथा᳚ । \newline
25. वि॒चि॒नु॒यादिति॑ वि - चि॒नु॒यात् । \newline
26. यथा॒ ऽऽस्या॑ दा॒स्या᳚द् यथा॒ यथा॒ ऽऽस्या᳚त् । \newline
27. आ॒स्या᳚द् ग्रसि॒तम् ग्र॑सि॒त मा॒स्या॑ दा॒स्या᳚द् ग्रसि॒तम् । \newline
28. ग्र॒सि॒तन् नि॑ष्खि॒दति॑ निष्खि॒दति॑ ग्रसि॒तम् ग्र॑सि॒तन् नि॑ष्खि॒दति॑ । \newline
29. नि॒ष्खि॒दति॑ ता॒दृक् ता॒दृङ् नि॑ष्खि॒दति॑ निष्खि॒दति॑ ता॒दृक् । \newline
30. नि॒ष्खि॒दतीति॑ निः - खि॒दति॑ । \newline
31. ता॒दृ गे॒वैव ता॒दृक् ता॒दृ गे॒व । \newline
32. ए॒व तत् तदे॒ वैव तत् । \newline
33. तद् यद् यत् तत् तद् यत् । \newline
34. यन् न न यद् यन् न । \newline
35. न वि॑चिनु॒याद् वि॑चिनु॒यान् न न वि॑चिनु॒यात् । \newline
36. वि॒चि॒नु॒याद् यथा॒ यथा॑ विचिनु॒याद् वि॑चिनु॒याद् यथा᳚ । \newline
37. वि॒चि॒नु॒यादिति॑ वि - चि॒नु॒यात् । \newline
38. यथा॒ ऽक्षन् न॒क्षन्. यथा॒ यथा॒ ऽक्षन्न् । \newline
39. अ॒क्षन् नाप॑न्न॒ माप॑न्न म॒क्षन् न॒क्षन् नाप॑न्नम् । \newline
40. आप॑न्नं ॅवि॒धाव॑ति वि॒धाव॒ त्याप॑न्न॒ माप॑न्नं ॅवि॒धाव॑ति । \newline
41. आप॑न्न॒मित्या - प॒न्न॒म् । \newline
42. वि॒धाव॑ति ता॒दृक् ता॒दृग् वि॒धाव॑ति वि॒धाव॑ति ता॒दृक् । \newline
43. वि॒धाव॒तीति॑ वि - धाव॑ति । \newline
44. ता॒दृ गे॒वैव ता॒दृक् ता॒दृ गे॒व । \newline
45. ए॒व तत् तदे॒ वैव तत् । \newline
46. तत् क्षोधु॑कः॒ क्षोधु॑क॒ स्तत् तत् क्षोधु॑कः । \newline
47. क्षोधु॑को ऽद्ध्व॒र्यु र॑द्ध्व॒र्युः क्षोधु॑कः॒ क्षोधु॑को ऽद्ध्व॒र्युः । \newline
48. अ॒द्ध्व॒र्युः स्याथ् स्या द॑द्ध्व॒र्यु र॑द्ध्व॒र्युः स्यात् । \newline
49. स्यात् क्षोधु॑कः॒ क्षोधु॑कः॒ स्याथ् स्यात् क्षोधु॑कः । \newline
50. क्षोधु॑को॒ यज॑मानो॒ यज॑मानः॒ क्षोधु॑कः॒ क्षोधु॑को॒ यज॑मानः । \newline
51. यज॑मानः॒ सोम॑विक्रयि॒न् थ्सोम॑विक्रयि॒न्॒. यज॑मानो॒ यज॑मानः॒ सोम॑विक्रयिन्न् । \newline
52. सोम॑विक्रयि॒न् थ्सोमꣳ॒॒ सोमꣳ॒॒ सोम॑विक्रयि॒न् थ्सोम॑विक्रयि॒न् थ्सोम᳚म् । \newline
53. सोम॑विक्रयि॒न्निति॒ सोम॑ - वि॒क्र॒यि॒न्न् । \newline
54. सोमꣳ॑ शोधय शोधय॒ सोमꣳ॒॒ सोमꣳ॑ शोधय । \newline
55. शो॒ध॒ये तीति॑ शोधय शोध॒येति॑ । \newline
56. इत्ये॒वैवे तीत्ये॒व । \newline
57. ए॒व ब्रू॑याद् ब्रूया दे॒वैव ब्रू॑यात् । \newline
58. ब्रू॒या॒द् यदि॒ यदि॑ ब्रूयाद् ब्रूया॒द् यदि॑ । \newline
59. यदीत॑र॒ मित॑रं॒ ॅयदि॒ यदीत॑रम् । \newline
60. इत॑रं॒ ॅयदि॒ यदीत॑र॒ मित॑रं॒ ॅयदि॑ । \newline

\textbf{Ghana Paata } \newline

1. ब्र॒ह्म॒वा॒दिनो॑ वदन्ति वदन्ति ब्रह्मवा॒दिनो᳚ ब्रह्मवा॒दिनो॑ वदन्ति वि॒चित्यो॑ वि॒चित्यो॑ वदन्ति ब्रह्मवा॒दिनो᳚ ब्रह्मवा॒दिनो॑ वदन्ति वि॒चित्यः॑ । \newline
2. ब्र॒ह्म॒वा॒दिन॒ इति॑ ब्रह्म - वा॒दिनः॑ । \newline
3. व॒द॒न्ति॒ वि॒चित्यो॑ वि॒चित्यो॑ वदन्ति वदन्ति वि॒चित्यः॒ सोमा(3)ः सोमा(3) वि॒चित्यो॑ वदन्ति वदन्ति वि॒चित्यः॒ सोमा(3)ः । \newline
4. वि॒चित्यः॒ सोमा(3)ः सोमा(3) वि॒चित्यो॑ वि॒चित्यः॒ सोमा(3) न न सोमा(3) वि॒चित्यो॑ वि॒चित्यः॒ सोमा(3) न । \newline
5. वि॒चित्य॒ इति॑ वि - चित्यः॑ । \newline
6. सोमा(3) न न सोमा(3)ः सोमा(3) न वि॒चित्या(3) वि॒चित्या(3) न सोमा(3)ः सोमा(3) न वि॒चित्या(3)ः । \newline
7. न वि॒चित्या(3) वि॒चित्या(3) न न वि॒चित्या(3) इतीति॑ वि॒चित्या(3) न न वि॒चित्या(3) इति॑ । \newline
8. वि॒चित्या(3) इतीति॑ वि॒चित्या(3) वि॒चित्या(3) इति॒ सोमः॒ सोम॒ इति॑ वि॒चित्या(3) वि॒चित्या(3) इति॒ सोमः॑ । \newline
9. वि॒चित्या(3) इति॑ वि - चित्या(3)ः । \newline
10. इति॒ सोमः॒ सोम॒ इतीति॒ सोमो॒ वै वै सोम॒ इतीति॒ सोमो॒ वै । \newline
11. सोमो॒ वै वै सोमः॒ सोमो॒ वा ओष॑धीना॒ मोष॑धीनां॒ ॅवै सोमः॒ सोमो॒ वा ओष॑धीनाम् । \newline
12. वा ओष॑धीना॒ मोष॑धीनां॒ ॅवै वा ओष॑धीनाꣳ॒॒ राजा॒ राजौष॑धीनां॒ ॅवै वा ओष॑धीनाꣳ॒॒ राजा᳚ । \newline
13. ओष॑धीनाꣳ॒॒ राजा॒ राजौष॑धीना॒ मोष॑धीनाꣳ॒॒ राजा॒ तस्मिꣳ॒॒ स्तस्मि॒न् राजौष॑धीना॒ मोष॑धीनाꣳ॒॒ राजा॒ तस्मिन्न्॑ । \newline
14. राजा॒ तस्मिꣳ॒॒ स्तस्मि॒न् राजा॒ राजा॒ तस्मि॒न्॒. यद् यत् तस्मि॒न् राजा॒ राजा॒ तस्मि॒न्॒. यत् । \newline
15. तस्मि॒न्॒. यद् यत् तस्मिꣳ॒॒ स्तस्मि॒न्॒. यदाप॑न्न॒ माप॑न्नं॒ ॅयत् तस्मिꣳ॒॒ स्तस्मि॒न्॒. यदाप॑न्नम् । \newline
16. यदाप॑न्न॒ माप॑न्नं॒ ॅयद् यदाप॑न्नम् ग्रसि॒तम् ग्र॑सि॒त माप॑न्नं॒ ॅयद् यदाप॑न्नम् ग्रसि॒तम् । \newline
17. आप॑न्नम् ग्रसि॒तम् ग्र॑सि॒त माप॑न्न॒ माप॑न्नम् ग्रसि॒त मे॒वैव ग्र॑सि॒त माप॑न्न॒ माप॑न्नम् ग्रसि॒त मे॒व । \newline
18. आप॑न्न॒मित्या - प॒न्न॒म् । \newline
19. ग्र॒सि॒त मे॒वैव ग्र॑सि॒तम् ग्र॑सि॒त मे॒वास्या᳚ स्यै॒व ग्र॑सि॒तम् ग्र॑सि॒त मे॒वास्य॑ । \newline
20. ए॒वास्या᳚ स्यै॒वै वास्य॒ तत् तद॑ स्यै॒वै वास्य॒ तत् । \newline
21. अ॒स्य॒ तत् तद॑ स्यास्य॒ तद् यद् यत् तद॑ स्यास्य॒ तद् यत् । \newline
22. तद् यद् यत् तत् तद् यद् वि॑चिनु॒याद् वि॑चिनु॒याद् यत् तत् तद् यद् वि॑चिनु॒यात् । \newline
23. यद् वि॑चिनु॒याद् वि॑चिनु॒याद् यद् यद् वि॑चिनु॒याद् यथा॒ यथा॑ विचिनु॒याद् यद् यद् वि॑चिनु॒याद् यथा᳚ । \newline
24. वि॒चि॒नु॒याद् यथा॒ यथा॑ विचिनु॒याद् वि॑चिनु॒याद् यथा॒ ऽऽस्या॑ दा॒स्या᳚द् यथा॑ विचिनु॒याद् वि॑चिनु॒याद् यथा॒ ऽऽस्या᳚त् । \newline
25. वि॒चि॒नु॒यादिति॑ वि - चि॒नु॒यात् । \newline
26. यथा॒ ऽऽस्या॑ दा॒स्या᳚द् यथा॒ यथा॒ ऽऽस्या᳚द् ग्रसि॒तम् ग्र॑सि॒त मा॒स्या᳚द् यथा॒ यथा॒ ऽऽस्या᳚द् ग्रसि॒तम् । \newline
27. आ॒स्या᳚द् ग्रसि॒तम् ग्र॑सि॒त मा॒स्या॑ दा॒स्या᳚द् ग्रसि॒तम् नि॑ष्खि॒दति॑ निष्खि॒दति॑ ग्रसि॒त मा॒स्या॑ दा॒स्या᳚द् ग्रसि॒तम् नि॑ष्खि॒दति॑ । \newline
28. ग्र॒सि॒तम् नि॑ष्खि॒दति॑ निष्खि॒दति॑ ग्रसि॒तम् ग्र॑सि॒तम् नि॑ष्खि॒दति॑ ता॒दृक् ता॒दृङ् नि॑ष्खि॒दति॑ ग्रसि॒तम् ग्र॑सि॒तम् नि॑ष्खि॒दति॑ ता॒दृक् । \newline
29. नि॒ष्खि॒दति॑ ता॒दृक् ता॒दृङ् नि॑ष्खि॒दति॑ निष्खि॒दति॑ ता॒दृ गे॒वैव ता॒दृङ् नि॑ष्खि॒दति॑ निष्खि॒दति॑ ता॒दृगे॒व । \newline
30. नि॒ष्खि॒दतीति॑ निः - खि॒दति॑ । \newline
31. ता॒दृ गे॒वैव ता॒दृक् ता॒दृ गे॒व तत् तदे॒व ता॒दृक् ता॒दृ गे॒व तत् । \newline
32. ए॒व तत् तदे॒ वैव तद् यद् यत् तदे॒ वैव तद् यत् । \newline
33. तद् यद् यत् तत् तद् यन् न न यत् तत् तद् यन् न । \newline
34. यन् न न यद् यन् न वि॑चिनु॒याद् वि॑चिनु॒यान् न यद् यन् न वि॑चिनु॒यात् । \newline
35. न वि॑चिनु॒याद् वि॑चिनु॒यान् न न वि॑चिनु॒याद् यथा॒ यथा॑ विचिनु॒यान् न न वि॑चिनु॒याद् यथा᳚ । \newline
36. वि॒चि॒नु॒याद् यथा॒ यथा॑ विचिनु॒याद् वि॑चिनु॒याद् यथा॒ ऽक्षन् न॒क्षन्. यथा॑ विचिनु॒याद् वि॑चिनु॒याद् यथा॒ ऽक्षन्न् । \newline
37. वि॒चि॒नु॒यादिति॑ वि - चि॒नु॒यात् । \newline
38. यथा॒ ऽक्षन् न॒क्षन्. यथा॒ यथा॒ ऽक्षन् नाप॑न्न॒ माप॑न्न म॒क्षन्. यथा॒ यथा॒ ऽक्षन् नाप॑न्नम् । \newline
39. अ॒क्षन् नाप॑न्न॒ माप॑न्न म॒क्षन् न॒क्षन् नाप॑न्नं ॅवि॒धाव॑ति वि॒धाव॒ त्याप॑न्न म॒क्षन् न॒क्षन् नाप॑न्नं ॅवि॒धाव॑ति । \newline
40. आप॑न्नं ॅवि॒धाव॑ति वि॒धाव॒ त्याप॑न्न॒ माप॑न्नं ॅवि॒धाव॑ति ता॒दृक् ता॒दृग् वि॒धाव॒ त्याप॑न्न॒ माप॑न्नं ॅवि॒धाव॑ति ता॒दृक् । \newline
41. आप॑न्न॒मित्या - प॒न्न॒म् । \newline
42. वि॒धाव॑ति ता॒दृक् ता॒दृग् वि॒धाव॑ति वि॒धाव॑ति ता॒दृ गे॒वैव ता॒दृग् वि॒धाव॑ति वि॒धाव॑ति ता॒दृ गे॒व । \newline
43. वि॒धाव॒तीति॑ वि - धाव॑ति । \newline
44. ता॒दृ गे॒वैव ता॒दृक् ता॒दृ गे॒व तत् तदे॒व ता॒दृक् ता॒दृ गे॒व तत् । \newline
45. ए॒व तत् तदे॒ वैव तत् क्षोधु॑कः॒ क्षोधु॑क॒ स्तदे॒ वैव तत् क्षोधु॑कः । \newline
46. तत् क्षोधु॑कः॒ क्षोधु॑क॒ स्तत् तत् क्षोधु॑को ऽद्ध्व॒र्यु र॑द्ध्व॒र्युः क्षोधु॑क॒ स्तत् तत् क्षोधु॑को ऽद्ध्व॒र्युः । \newline
47. क्षोधु॑को ऽद्ध्व॒र्यु र॑द्ध्व॒र्युः क्षोधु॑कः॒ क्षोधु॑को ऽद्ध्व॒र्युः स्याथ् स्या द॑द्ध्व॒र्युः क्षोधु॑कः॒ क्षोधु॑को ऽद्ध्व॒र्युः स्यात् । \newline
48. अ॒द्ध्व॒र्युः स्याथ् स्या द॑द्ध्व॒र्यु र॑द्ध्व॒र्युः स्यात् क्षोधु॑कः॒ क्षोधु॑कः॒ स्या द॑द्ध्व॒र्यु र॑द्ध्व॒र्युः स्यात् क्षोधु॑कः । \newline
49. स्यात् क्षोधु॑कः॒ क्षोधु॑कः॒ स्याथ् स्यात् क्षोधु॑को॒ यज॑मानो॒ यज॑मानः॒ क्षोधु॑कः॒ स्याथ् स्यात् क्षोधु॑को॒ यज॑मानः । \newline
50. क्षोधु॑को॒ यज॑मानो॒ यज॑मानः॒ क्षोधु॑कः॒ क्षोधु॑को॒ यज॑मानः॒ सोम॑विक्रयि॒न् थ्सोम॑विक्रयि॒न्॒. यज॑मानः॒ क्षोधु॑कः॒ क्षोधु॑को॒ यज॑मानः॒ सोम॑विक्रयिन्न् । \newline
51. यज॑मानः॒ सोम॑विक्रयि॒न् थ्सोम॑विक्रयि॒न्॒. यज॑मानो॒ यज॑मानः॒ सोम॑विक्रयि॒न् थ्सोमꣳ॒॒ सोमꣳ॒॒ सोम॑विक्रयि॒न्॒. यज॑मानो॒ यज॑मानः॒ सोम॑विक्रयि॒न् थ्सोम᳚म् । \newline
52. सोम॑विक्रयि॒न् थ्सोमꣳ॒॒ सोमꣳ॒॒ सोम॑विक्रयि॒न् थ्सोम॑विक्रयि॒न् थ्सोमꣳ॑ शोधय शोधय॒ सोमꣳ॒॒ सोम॑विक्रयि॒न् थ्सोम॑विक्रयि॒न् थ्सोमꣳ॑ शोधय । \newline
53. सोम॑विक्रयि॒न्निति॒ सोम॑ - वि॒क्र॒यि॒न्न् । \newline
54. सोमꣳ॑ शोधय शोधय॒ सोमꣳ॒॒ सोमꣳ॑ शोध॒ये तीति॑ शोधय॒ सोमꣳ॒॒ सोमꣳ॑ शोध॒येति॑ । \newline
55. शो॒ध॒ये तीति॑ शोधय शोध॒ये त्ये॒वै वेति॑ शोधय शोध॒ये त्ये॒व । \newline
56. इत्ये॒वैवे तीत्ये॒व ब्रू॑याद् ब्रूया दे॒वे तीत्ये॒व ब्रू॑यात् । \newline
57. ए॒व ब्रू॑याद् ब्रूया दे॒वैव ब्रू॑या॒द् यदि॒ यदि॑ ब्रूया दे॒वैव ब्रू॑या॒द् यदि॑ । \newline
58. ब्रू॒या॒द् यदि॒ यदि॑ ब्रूयाद् ब्रूया॒द् यदीत॑र॒ मित॑रं॒ ॅयदि॑ ब्रूयाद् ब्रूया॒द् यदीत॑रम् । \newline
59. यदीत॑र॒ मित॑रं॒ ॅयदि॒ यदीत॑रं॒ ॅयदि॒ यदीत॑रं॒ ॅयदि॒ यदीत॑रं॒ ॅयदि॑ । \newline
60. इत॑रं॒ ॅयदि॒ यदीत॑र॒ मित॑रं॒ ॅयदीत॑र॒ मित॑रं॒ ॅयदीत॑र॒ मित॑रं॒ ॅयदीत॑रम् । \newline
\pagebreak
\markright{ TS 6.1.9.2  \hfill https://www.vedavms.in \hfill}

\section{ TS 6.1.9.2 }

\textbf{TS 6.1.9.2 } \newline
\textbf{Samhita Paata} \newline

ॅयदीत॑रमु॒भये॑नै॒व सो॑मविक्र॒यिण॑-मर्पयति॒ तस्मा᳚थ् सोमविक्र॒यी क्षोधु॑को ऽरु॒णो ह॑ स्मा॒ऽऽ*हौप॑वेशिः सोम॒क्रय॑ण ए॒वाहं तृ॑तीय सव॒नमव॑ रुन्ध॒ इति॑ पशू॒नां चर्म॑न् मिमीते प॒शूने॒वाव॑ रुन्धे प॒शवो॒ हि तृ॒तीयꣳ॒॒ सव॑नं॒ ॅयं का॒मये॑ताप॒शुः स्या॒दित्यृ॑क्ष॒तस्तस्य॑ मिमीत॒र्क्षं ॅवा अ॑पश॒व्यम॑प॒शुरे॒व भ॑वति॒ यं का॒मये॑त पशु॒मान्थ् स्या॒ - [  ] \newline

\textbf{Pada Paata} \newline

यदि॑ । इत॑रम् । उ॒भये॑न । ए॒व । सो॒म॒वि॒क्र॒यिण॒मिति॑ सोम - वि॒क्र॒यिण᳚म् । अ॒र्प॒य॒ति॒ । तस्मा᳚त् । सो॒म॒वि॒क्र॒यीति॑ सोम - वि॒क्र॒यी । क्षोधु॑कः । अ॒रु॒णः । ह॒ । स्म॒ । आ॒ह॒ । औप॑वेशि॒रित्यौप॑-वे॒शिः॒ । सो॒म॒क्रय॑ण॒ इति॑ सोम - क्रय॑णे । ए॒व । अ॒हम् । तृ॒ती॒य॒स॒व॒नमिति॑ तृतीय - स॒व॒नम् । अवेति॑ । रु॒न्धे॒ । इति॑ । प॒शू॒नाम् । चर्मन्न्॑ । मि॒मी॒ते॒ । प॒शून् । ए॒व । अवेति॑ । रु॒न्धे॒ । प॒शवः॑ । हि । तृ॒तीय᳚म् । सव॑नम् । यम् । का॒मये॑त । अ॒प॒शुः । स्या॒त् । इति॑ । ऋ॒क्ष॒तः । तस्य॑ । मि॒मी॒त॒ । ऋ॒क्षम् । वै । अ॒प॒श॒व्यम् । अ॒प॒शुः । ए॒व । भ॒व॒ति॒ । यम् । का॒मये॑त । प॒शु॒मानिति॑ पशु - मान् । स्या॒त् ।  \newline


\textbf{Krama Paata} \newline

यदीत॑रम् । इत॑रमु॒भये॑न । उ॒भये॑नै॒व । ए॒व सो॑मविक्र॒यिण᳚म् । सो॒म॒वि॒क्र॒यिण॑मर्पयति । सो॒म॒वि॒क्र॒यिण॒मिति॑ सोम - वि॒क्र॒यिण᳚म् । अ॒र्प॒य॒ति॒ तस्मा᳚त् । तस्मा᳚थ् सोमविक्र॒यी । सो॒म॒वि॒क्र॒यी क्षोधु॑कः । सो॒म॒वि॒क्र॒यीति॑ सोम - वि॒क्र॒यी । क्षोधु॑कोऽरु॒णः । अ॒रु॒णो ह॑ । ह॒ स्म॒ । स्मा॒ह॒ । आ॒हौप॑वेशिः । औप॑वेशिः सोम॒क्रय॑णे । औप॑वेशि॒रित्यौप॑ - वे॒शिः॒ । सो॒म॒क्रय॑ण ए॒व । सो॒म॒क्रय॑ण॒ इति॑ सोम - क्रय॑णे । ए॒वाहम् । अ॒हम् तृ॑तीयसव॒नम् । तृ॒ती॒य॒स॒व॒नमव॑ । तृ॒ती॒य॒स॒व॒नमिति॑ तृतीय - स॒व॒नम् । अव॑ रुन्धे । रु॒न्ध॒ इति॑ । इति॑ पशू॒नाम् । प॒शू॒नाम् चर्मन्न्॑ । चर्म॑न् मिमीते । मि॒मी॒ते॒ प॒शून् । प॒शूने॒व । ए॒वाव॑ । अव॑ रुन्धे । रु॒न्धे॒ प॒शवः॑ । प॒शवो॒ हि । हि तृ॒तीय᳚म् । तृ॒तीयꣳ॒॒ सव॑नम् । सव॑न॒म् ॅयम् । यम् का॒मये॑त । का॒मये॑ताप॒शुः । अ॒प॒शुः स्या᳚त् । स्या॒दिति॑ । इत्यृ॑क्ष॒तः । ऋ॒क्ष॒तस्तस्य॑ । तस्य॑ मिमीत । मि॒मी॒त॒र्क्षम् । ऋ॒क्षम् ॅवै । वा अ॑पश॒व्यम् । अ॒प॒श॒व्यम॑प॒शुः । अ॒प॒शुरे॒व । ए॒व भ॑वति । भ॒व॒ति॒ यम् । यम् का॒मये॑त । का॒मये॑त पशु॒मान् । प॒शु॒मान्थ् स्या᳚त् । प॒शु॒मानिति॑ पशु - मान् । स्या॒दिति॑ \newline

\textbf{Jatai Paata} \newline

1. यदीत॑र॒ मित॑रं॒ ॅयदि॒ यदीत॑रम् । \newline
2. इत॑र मु॒भये॑ नो॒भये॒ने त॑र॒ मित॑र मु॒भये॑न । \newline
3. उ॒भये॑ नै॒वै वोभये॑ नो॒भये॑ नै॒व । \newline
4. ए॒व सो॑मविक्र॒यिणꣳ॑ सोमविक्र॒यिण॑ मे॒वैव सो॑मविक्र॒यिण᳚म् । \newline
5. सो॒म॒वि॒क्र॒यिण॑ मर्पय त्यर्पयति सोमविक्र॒यिणꣳ॑ सोमविक्र॒यिण॑ मर्पयति । \newline
6. सो॒म॒वि॒क्र॒यिण॒मिति॑ सोम - वि॒क्र॒यिण᳚म् । \newline
7. अ॒र्प॒य॒ति॒ तस्मा॒त् तस्मा॑ दर्पय त्यर्पयति॒ तस्मा᳚त् । \newline
8. तस्मा᳚थ् सोमविक्र॒यी सो॑मविक्र॒यी तस्मा॒त् तस्मा᳚थ् सोमविक्र॒यी । \newline
9. सो॒म॒वि॒क्र॒यी क्षोधु॑कः॒ क्षोधु॑कः सोमविक्र॒यी सो॑मविक्र॒यी क्षोधु॑कः । \newline
10. सो॒म॒वि॒क्र॒यीति॑ सोम - वि॒क्र॒यी । \newline
11. क्षोधु॑को ऽरु॒णो॑ ऽरु॒णः क्षोधु॑कः॒ क्षोधु॑को ऽरु॒णः । \newline
12. अ॒रु॒णो ह॑ हारु॒णो॑ ऽरु॒णो ह॑ । \newline
13. ह॒ स्म॒ स्म॒ ह॒ ह॒ स्म॒ । \newline
14. स्मा॒हा॒ह॒ स्म॒ स्मा॒ह॒ । \newline
15. आ॒हौप॑वेशि॒ रौप॑वेशि राहा॒ हौप॑वेशिः । \newline
16. औप॑वेशिः सोम॒क्रय॑णे सोम॒क्रय॑ण॒ औप॑वेशि॒ रौप॑वेशिः सोम॒क्रय॑णे । \newline
17. औप॑वेशि॒रित्यौप॑ - वे॒शिः॒ । \newline
18. सो॒म॒क्रय॑ण ए॒वैव सो॑म॒क्रय॑णे सोम॒क्रय॑ण ए॒व । \newline
19. सो॒म॒क्रय॑ण॒ इति॑ सोम - क्रय॑णे । \newline
20. ए॒वाह म॒ह मे॒वै वाहम् । \newline
21. अ॒हम् तृ॑तीयसव॒नम् तृ॑तीयसव॒न म॒ह म॒हम् तृ॑तीयसव॒नम् । \newline
22. तृ॒ती॒य॒स॒व॒न मवाव॑ तृतीयसव॒नम् तृ॑तीयसव॒न मव॑ । \newline
23. तृ॒ती॒य॒स॒व॒नमिति॑ तृतीय - स॒व॒नम् । \newline
24. अव॑ रुन्धे रु॒न्धे ऽवाव॑ रुन्धे । \newline
25. रु॒न्ध॒ इतीति॑ रुन्धे रुन्ध॒ इति॑ । \newline
26. इति॑ पशू॒नाम् प॑शू॒ना मितीति॑ पशू॒नाम् । \newline
27. प॒शू॒नाम् चर्मꣳ॒॒ श्चर्म॑न् पशू॒नाम् प॑शू॒नाम् चर्मन्न्॑ । \newline
28. चर्म॑न् मिमीते मिमीते॒ चर्मꣳ॒॒ श्चर्म॑न् मिमीते । \newline
29. मि॒मी॒ते॒ प॒शून् प॒शून् मि॑मीते मिमीते प॒शून् । \newline
30. प॒शूने॒ वैव प॒शून् प॒शूने॒व । \newline
31. ए॒वावा वै॒वै वाव॑ । \newline
32. अव॑ रुन्धे रु॒न्धे ऽवाव॑ रुन्धे । \newline
33. रु॒न्धे॒ प॒शवः॑ प॒शवो॑ रुन्धे रुन्धे प॒शवः॑ । \newline
34. प॒शवो॒ हि हि प॒शवः॑ प॒शवो॒ हि । \newline
35. हि तृ॒तीय॑म् तृ॒तीयꣳ॒॒ हि हि तृ॒तीय᳚म् । \newline
36. तृ॒तीयꣳ॒॒ सव॑नꣳ॒॒ सव॑नम् तृ॒तीय॑म् तृ॒तीयꣳ॒॒ सव॑नम् । \newline
37. सव॑नं॒ ॅयं ॅयꣳ सव॑नꣳ॒॒ सव॑नं॒ ॅयम् । \newline
38. यम् का॒मये॑त का॒मये॑त॒ यं ॅयम् का॒मये॑त । \newline
39. का॒मये॑ता प॒शुर॑ प॒शुः का॒मये॑त का॒मये॑ता प॒शुः । \newline
40. अ॒प॒शुः स्या᳚थ् स्या दप॒शु र॑प॒शुः स्या᳚त् । \newline
41. स्या॒ दितीति॑ स्याथ् स्या॒ दिति॑ । \newline
42. इत्यृ॑क्ष॒त ऋ॑क्ष॒त इतीत्यृ॑क्ष॒तः । \newline
43. ऋ॒क्ष॒त स्तस्य॒ तस्य॑ र्‌क्ष॒त ऋ॑क्ष॒त स्तस्य॑ । \newline
44. तस्य॑ मिमीत मिमीत॒ तस्य॒ तस्य॑ मिमीत । \newline
45. मि॒मी॒त॒ र्‌क्ष मृ॒क्षम् मि॑मीत मिमीत॒ र्‌क्षम् । \newline
46. ऋ॒क्षं ॅवै वा ऋ॒क्ष मृ॒क्षं ॅवै । \newline
47. वा अ॑पश॒व्य म॑पश॒व्यं ॅवै वा अ॑पश॒व्यम् । \newline
48. अ॒प॒श॒व्य म॑प॒शु र॑प॒शु र॑पश॒व्य म॑पश॒व्य म॑प॒शुः । \newline
49. अ॒प॒शु रे॒वैवा प॒शु र॑प॒शु रे॒व । \newline
50. ए॒व भ॑वति भव त्ये॒वैव भ॑वति । \newline
51. भ॒व॒ति॒ यं ॅयम् भ॑वति भवति॒ यम् । \newline
52. यम् का॒मये॑त का॒मये॑त॒ यं ॅयम् का॒मये॑त । \newline
53. का॒मये॑त पशु॒मान् प॑शु॒मान् का॒मये॑त का॒मये॑त पशु॒मान् । \newline
54. प॒शु॒मान् थ्स्या᳚थ् स्यात् पशु॒मान् प॑शु॒मान् थ्स्या᳚त् । \newline
55. प॒शु॒मानिति॑ पशु - मान् । \newline
56. स्या॒दि तीति॑ स्याथ् स्या॒ दिति॑ । \newline

\textbf{Ghana Paata } \newline

1. यदीत॑र॒ मित॑रं॒ ॅयदि॒ यदीत॑र मु॒भये॑ नो॒भये॒ने त॑रं॒ ॅयदि॒ यदीत॑र मु॒भये॑न । \newline
2. इत॑र मु॒भये॑ नो॒भये॒ने त॑र॒ मित॑र मु॒भये॑ नै॒वैवोभये॒ने त॑र॒ मित॑र मु॒भये॑नै॒व । \newline
3. उ॒भये॑ नै॒वै वोभये॑ नो॒भये॑नै॒व सो॑मविक्र॒यिणꣳ॑ सोमविक्र॒यिण॑ मे॒वोभये॑ नो॒भये॑नै॒व सो॑मविक्र॒यिण᳚म् । \newline
4. ए॒व सो॑मविक्र॒यिणꣳ॑ सोमविक्र॒यिण॑ मे॒वैव सो॑मविक्र॒यिण॑ मर्पय त्यर्पयति सोमविक्र॒यिण॑ मे॒वैव सो॑मविक्र॒यिण॑ मर्पयति । \newline
5. सो॒म॒वि॒क्र॒यिण॑ मर्पय त्यर्पयति सोमविक्र॒यिणꣳ॑ सोमविक्र॒यिण॑ मर्पयति॒ तस्मा॒त् तस्मा॑ दर्पयति सोमविक्र॒यिणꣳ॑ सोमविक्र॒यिण॑ मर्पयति॒ तस्मा᳚त् । \newline
6. सो॒म॒वि॒क्र॒यिण॒मिति॑ सोम - वि॒क्र॒यिण᳚म् । \newline
7. अ॒र्प॒य॒ति॒ तस्मा॒त् तस्मा॑ दर्पय त्यर्पयति॒ तस्मा᳚थ् सोमविक्र॒यी सो॑मविक्र॒यी तस्मा॑ दर्पय त्यर्पयति॒ तस्मा᳚थ् सोमविक्र॒यी । \newline
8. तस्मा᳚थ् सोमविक्र॒यी सो॑मविक्र॒यी तस्मा॒त् तस्मा᳚थ् सोमविक्र॒यी क्षोधु॑कः॒ क्षोधु॑कः सोमविक्र॒यी तस्मा॒त् तस्मा᳚थ् सोमविक्र॒यी क्षोधु॑कः । \newline
9. सो॒म॒वि॒क्र॒यी क्षोधु॑कः॒ क्षोधु॑कः सोमविक्र॒यी सो॑मविक्र॒यी क्षोधु॑को ऽरु॒णो॑ ऽरु॒णः क्षोधु॑कः सोमविक्र॒यी सो॑मविक्र॒यी क्षोधु॑को ऽरु॒णः । \newline
10. सो॒म॒वि॒क्र॒यीति॑ सोम - वि॒क्र॒यी । \newline
11. क्षोधु॑को ऽरु॒णो॑ ऽरु॒णः क्षोधु॑कः॒ क्षोधु॑को ऽरु॒णो ह॑ हारु॒णः क्षोधु॑कः॒ क्षोधु॑को ऽरु॒णो ह॑ । \newline
12. अ॒रु॒णो ह॑ हारु॒णो॑ ऽरु॒णो ह॑ स्म स्म हारु॒णो॑ ऽरु॒णो ह॑ स्म । \newline
13. ह॒ स्म॒ स्म॒ ह॒ ह॒ स्मा॒हा॒ह॒ स्म॒ ह॒ ह॒ स्मा॒ह॒ । \newline
14. स्मा॒हा॒ह॒ स्म॒ स्मा॒ हौप॑वेशि॒ रौप॑वेशि राह स्म स्मा॒ हौप॑वेशिः । \newline
15. आ॒हौप॑वेशि॒ रौप॑वेशि राहा॒ हौप॑वेशिः सोम॒क्रय॑णे सोम॒क्रय॑ण॒ औप॑वेशि राहा॒ हौप॑वेशिः सोम॒क्रय॑णे । \newline
16. औप॑वेशिः सोम॒क्रय॑णे सोम॒क्रय॑ण॒ औप॑वेशि॒ रौप॑वेशिः सोम॒क्रय॑ण ए॒वैव सो॑म॒क्रय॑ण॒ औप॑वेशि॒ रौप॑वेशिः सोम॒क्रय॑ण ए॒व । \newline
17. औप॑वेशि॒रित्यौप॑ - वे॒शिः॒ । \newline
18. सो॒म॒क्रय॑ण ए॒वैव सो॑म॒क्रय॑णे सोम॒क्रय॑ण ए॒वाह म॒ह मे॒व सो॑म॒क्रय॑णे सोम॒क्रय॑ण ए॒वाहम् । \newline
19. सो॒म॒क्रय॑ण॒ इति॑ सोम - क्रय॑णे । \newline
20. ए॒वाह म॒ह मे॒वैवाहम् तृ॑तीयसव॒नम् तृ॑तीयसव॒न म॒ह मे॒वै वाहम् तृ॑तीयसव॒नम् । \newline
21. अ॒हम् तृ॑तीयसव॒नम् तृ॑तीयसव॒न म॒ह म॒हम् तृ॑तीयसव॒न मवाव॑ तृतीयसव॒न म॒ह म॒हम् तृ॑तीयसव॒न मव॑ । \newline
22. तृ॒ती॒य॒स॒व॒न मवाव॑ तृतीयसव॒नम् तृ॑तीयसव॒न मव॑ रुन्धे रु॒न्धे ऽव॑ तृतीयसव॒नम् तृ॑तीयसव॒न मव॑ रुन्धे । \newline
23. तृ॒ती॒य॒स॒व॒नमिति॑ तृतीय - स॒व॒नम् । \newline
24. अव॑ रुन्धे रु॒न्धे ऽवाव॑ रुन्ध॒ इतीति॑ रु॒न्धे ऽवाव॑ रुन्ध॒ इति॑ । \newline
25. रु॒न्ध॒ इतीति॑ रुन्धे रुन्ध॒ इति॑ पशू॒नाम् प॑शू॒ना मिति॑ रुन्धे रुन्ध॒ इति॑ पशू॒नाम् । \newline
26. इति॑ पशू॒नाम् प॑शू॒ना मितीति॑ पशू॒नाम् चर्मꣳ॒॒ श्चर्म॑न् पशू॒ना मितीति॑ पशू॒नाम् चर्मन्न्॑ । \newline
27. प॒शू॒नाम् चर्मꣳ॒॒ श्चर्म॑न् पशू॒नाम् प॑शू॒नाम् चर्म॑न् मिमीते मिमीते॒ चर्म॑न् पशू॒नाम् प॑शू॒नाम् चर्म॑न् मिमीते । \newline
28. चर्म॑न् मिमीते मिमीते॒ चर्मꣳ॒॒ श्चर्म॑न् मिमीते प॒शून् प॒शून् मि॑मीते॒ चर्मꣳ॒॒ श्चर्म॑न् मिमीते प॒शून् । \newline
29. मि॒मी॒ते॒ प॒शून् प॒शून् मि॑मीते मिमीते प॒शूने॒ वैव प॒शून् मि॑मीते मिमीते प॒शूने॒व । \newline
30. प॒शूने॒ वैव प॒शून् प॒शूने॒ वावा वै॒व प॒शून् प॒शूने॒ वाव॑ । \newline
31. ए॒वावा वै॒वै वाव॑ रुन्धे रु॒न्धे ऽवै॒वै वाव॑ रुन्धे । \newline
32. अव॑ रुन्धे रु॒न्धे ऽवाव॑ रुन्धे प॒शवः॑ प॒शवो॑ रु॒न्धे ऽवाव॑ रुन्धे प॒शवः॑ । \newline
33. रु॒न्धे॒ प॒शवः॑ प॒शवो॑ रुन्धे रुन्धे प॒शवो॒ हि हि प॒शवो॑ रुन्धे रुन्धे प॒शवो॒ हि । \newline
34. प॒शवो॒ हि हि प॒शवः॑ प॒शवो॒ हि तृ॒तीय॑म् तृ॒तीयꣳ॒॒ हि प॒शवः॑ प॒शवो॒ हि तृ॒तीय᳚म् । \newline
35. हि तृ॒तीय॑म् तृ॒तीयꣳ॒॒ हि हि तृ॒तीयꣳ॒॒ सव॑नꣳ॒॒ सव॑नम् तृ॒तीयꣳ॒॒ हि हि तृ॒तीयꣳ॒॒ सव॑नम् । \newline
36. तृ॒तीयꣳ॒॒ सव॑नꣳ॒॒ सव॑नम् तृ॒तीय॑म् तृ॒तीयꣳ॒॒ सव॑नं॒ ॅयं ॅयꣳ सव॑नम् तृ॒तीय॑म् तृ॒तीयꣳ॒॒ सव॑नं॒ ॅयम् । \newline
37. सव॑नं॒ ॅयं ॅयꣳ सव॑नꣳ॒॒ सव॑नं॒ ॅयम् का॒मये॑त का॒मये॑त॒ यꣳ सव॑नꣳ॒॒ सव॑नं॒ ॅयम् का॒मये॑त । \newline
38. यम् का॒मये॑त का॒मये॑त॒ यं ॅयम् का॒मये॑ता प॒शु र॑प॒शुः का॒मये॑त॒ यं ॅयम् का॒मये॑ता प॒शुः । \newline
39. का॒मये॑ता प॒शु र॑प॒शुः का॒मये॑त का॒मये॑ता प॒शुः स्या᳚थ् स्या दप॒शुः का॒मये॑त का॒मये॑ता प॒शुः स्या᳚त् । \newline
40. अ॒प॒शुः स्या᳚थ् स्या दप॒शु र॑प॒शुः स्या॒दि तीति॑ स्या दप॒शु र॑प॒शुः स्या॒ दिति॑ । \newline
41. स्या॒दि तीति॑ स्याथ् स्या॒दि त्यृ॑क्ष॒त ऋ॑क्ष॒त इति॑ स्याथ् स्या॒दि त्यृ॑क्ष॒तः । \newline
42. इत्यृ॑क्ष॒त ऋ॑क्ष॒त इतीत्यृ॑क्ष॒त स्तस्य॒ तस्य॑ र्‌क्ष॒त इतीत्यृ॑क्ष॒त स्तस्य॑ । \newline
43. ऋ॒क्ष॒त स्तस्य॒ तस्य॑ र्‌क्ष॒त ऋ॑क्ष॒त स्तस्य॑ मिमीत मिमीत॒ तस्य॑ र्‌क्ष॒त ऋ॑क्ष॒त स्तस्य॑ मिमीत । \newline
44. तस्य॑ मिमीत मिमीत॒ तस्य॒ तस्य॑ मिमीत॒ र्‌क्ष मृ॒क्षम् मि॑मीत॒ तस्य॒ तस्य॑ मिमीत॒ र्‌क्षम् । \newline
45. मि॒मी॒त॒ र्‌क्ष मृ॒क्षम् मि॑मीत मिमीत॒ र्‌क्षं ॅवै वा ऋ॒क्षम् मि॑मीत मिमीत॒ र्‌क्षं ॅवै । \newline
46. ऋ॒क्षं ॅवै वा ऋ॒क्ष मृ॒क्षं ॅवा अ॑पश॒व्य म॑पश॒व्यं ॅवा ऋ॒क्ष मृ॒क्षं ॅवा अ॑पश॒व्यम् । \newline
47. वा अ॑पश॒व्य म॑पश॒व्यं ॅवै वा अ॑पश॒व्य म॑प॒शु र॑प॒शु र॑पश॒व्यं ॅवै वा अ॑पश॒व्य म॑प॒शुः । \newline
48. अ॒प॒श॒व्य म॑प॒शु र॑प॒शु र॑पश॒व्य म॑पश॒व्य म॑प॒शु रे॒वैवा प॒शु र॑पश॒व्य म॑पश॒व्य म॑प॒शु रे॒व । \newline
49. अ॒प॒शु रे॒वै वाप॒शु र॑प॒शु रे॒व भ॑वति भव त्ये॒वाप॒शु र॑प॒शु रे॒व भ॑वति । \newline
50. ए॒व भ॑वति भव त्ये॒वैव भ॑वति॒ यं ॅयम् भ॑व त्ये॒वैव भ॑वति॒ यम् । \newline
51. भ॒व॒ति॒ यं ॅयम् भ॑वति भवति॒ यम् का॒मये॑त का॒मये॑त॒ यम् भ॑वति भवति॒ यम् का॒मये॑त । \newline
52. यम् का॒मये॑त का॒मये॑त॒ यं ॅयम् का॒मये॑त पशु॒मान् प॑शु॒मान् का॒मये॑त॒ यं ॅयम् का॒मये॑त पशु॒मान् । \newline
53. का॒मये॑त पशु॒मान् प॑शु॒मान् का॒मये॑त का॒मये॑त पशु॒मान् थ्स्या᳚थ् स्यात् पशु॒मान् का॒मये॑त का॒मये॑त पशु॒मान् थ्स्या᳚त् । \newline
54. प॒शु॒मान् थ्स्या᳚थ् स्यात् पशु॒मान् प॑शु॒मान् थ्स्या॒दि तीति॑ स्यात् पशु॒मान् प॑शु॒मान् थ्स्या॒दिति॑ । \newline
55. प॒शु॒मानिति॑ पशु - मान् । \newline
56. स्या॒दि तीति॑ स्याथ् स्या॒दिति॑ लोम॒तो लो॑म॒त इति॑ स्याथ् स्या॒दिति॑ लोम॒तः । \newline
\pagebreak
\markright{ TS 6.1.9.3  \hfill https://www.vedavms.in \hfill}

\section{ TS 6.1.9.3 }

\textbf{TS 6.1.9.3 } \newline
\textbf{Samhita Paata} \newline

-दिति॑ लोम॒तस्तस्य॑ मिमीतै॒तद्वै प॑शू॒नाꣳ रू॒पꣳ रू॒पेणै॒वास्मै॑ प॒शूनव॑ रुन्धे पशु॒माने॒व भ॑वत्य॒पामन्ते᳚ क्रीणाति॒ सर॑समे॒वैनं॑ क्रीणात्य॒-मात्यो॒ऽसीत्या॑हा॒मैवैनं॑ कुरुते शु॒क्रस्ते॒ ग्रह॒ इत्या॑ह शु॒क्रो ह्य॑स्य॒ ग्रहो ऽन॒साऽच्छ॑ याति महि॒मान॑-मे॒वास्याच्छ॑ या॒त्यन॒सा - [  ] \newline

\textbf{Pada Paata} \newline

इति॑ । लो॒म॒तः । तस्य॑ । मि॒मी॒त॒ । ए॒तत् । वै । प॒शू॒नाम् । रू॒पम् । रू॒पेण॑ । ए॒व । अ॒स्मै॒ । प॒शून् । अवेति॑ । रु॒न्धे॒ । प॒शु॒मानिति॑ पशु - मान् । ए॒व । भ॒व॒ति॒ । अ॒पाम् । अन्ते᳚ । क्री॒णा॒ति॒ । सर॑स॒मिति॒ स-र॒स॒म् । ए॒व । ए॒न॒म् । क्री॒णा॒ति॒ । अ॒मात्यः॑ । अ॒सि॒ । इति॑ । आ॒ह॒ । अ॒मा । ए॒व । ए॒न॒म् । कु॒रु॒ते॒ । शु॒क्रः । ते॒ । ग्रहः॑ । इति॑ । आ॒ह॒ । शु॒क्रः । हि । अ॒स्य॒ । ग्रहः॑ । अन॑सा । अच्छ॑ । या॒ति॒ । म॒हि॒मान᳚म् । ए॒व । अ॒स्य॒ । अच्छ॑ । या॒ति॒ । अन॑सा ।  \newline


\textbf{Krama Paata} \newline

इति॑ लोम॒तः । लो॒म॒तस्तस्य॑ । तस्य॑ मिमीत । मि॒मी॒तै॒तत् । ए॒तद् वै । वै प॑शू॒नाम् । प॒शू॒नाꣳ रू॒पम् । रू॒पꣳ रू॒पेण॑ । रू॒पेणै॒व । ए॒वास्मै᳚ । अ॒स्मै॒ प॒शून् । प॒शूनव॑ । अव॑ रुन्धे । रु॒न्धे॒ प॒शु॒मान् । प॒शु॒माने॒व । प॒शु॒मानिति॑ पशु - मान् । ए॒व भ॑वति । भ॒व॒त्य॒पाम् । अ॒पामन्ते᳚ । अन्ते᳚ क्रीणाति । क्री॒णा॒ति॒ सर॑सम् । सर॑समे॒व । सर॑स॒मिति॒ स - र॒स॒म् । ए॒वैन᳚म् । ए॒न॒म् क्री॒णा॒ति॒ । क्री॒णा॒त्य॒मात्यः॑ । अ॒मात्यो॑ऽसि । अ॒सीति॑ । इत्या॑ह । आ॒हा॒मा । अ॒मैव । ए॒वैन᳚म् । ए॒न॒म् कु॒रु॒ते॒ । कु॒रु॒ते॒ शु॒क्रः । शु॒क्रस्ते᳚ । ते॒ ग्रहः॑ । ग्रह॒ इति॑ । इत्या॑ह । आ॒ह॒ शु॒क्रः । शु॒क्रो हि । ह्य॑स्य । अ॒स्य॒ ग्रहः॑ । ग्रहोऽन॑सा । अन॒साऽच्छ॑ । अच्छ॑ याति । या॒ति॒ म॒हि॒मान᳚म् । म॒हि॒मान॑मे॒व । ए॒वास्य॑ । अ॒स्याच्छ॑ । अच्छ॑ याति । या॒त्यन॑सा । अन॒साऽच्छ॑ \newline

\textbf{Jatai Paata} \newline

1. इति॑ लोम॒तो लो॑म॒त इतीति॑ लोम॒तः । \newline
2. लो॒म॒त स्तस्य॒ तस्य॑ लोम॒तो लो॑म॒त स्तस्य॑ । \newline
3. तस्य॑ मिमीत मिमीत॒ तस्य॒ तस्य॑ मिमीत । \newline
4. मि॒मी॒तै॒त दे॒तन् मि॑मीत मिमी तै॒तत् । \newline
5. ए॒तद् वै वा ए॒त दे॒तद् वै । \newline
6. वै प॑शू॒नाम् प॑शू॒नां ॅवै वै प॑शू॒नाम् । \newline
7. प॒शू॒नाꣳ रू॒पꣳ रू॒पम् प॑शू॒नाम् प॑शू॒नाꣳ रू॒पम् । \newline
8. रू॒पꣳ रू॒पेण॑ रू॒पेण॑ रू॒पꣳ रू॒पꣳ रू॒पेण॑ । \newline
9. रू॒पे णै॒वैव रू॒पेण॑ रू॒पे णै॒व । \newline
10. ए॒वास्मा॑ अस्मा ए॒वै वास्मै᳚ । \newline
11. अ॒स्मै॒ प॒शून् प॒शून॑स्मा अस्मै प॒शून् । \newline
12. प॒शू नवाव॑ प॒शून् प॒शू नव॑ । \newline
13. अव॑ रुन्धे रु॒न्धे ऽवाव॑ रुन्धे । \newline
14. रु॒न्धे॒ प॒शु॒मान् प॑शु॒मान् रु॑न्धे रुन्धे पशु॒मान् । \newline
15. प॒शु॒मा ने॒वैव प॑शु॒मान् प॑शु॒माने॒व । \newline
16. प॒शु॒मानिति॑ पशु - मान् । \newline
17. ए॒व भ॑वति भव त्ये॒वैव भ॑वति । \newline
18. भ॒व॒ त्य॒पा म॒पाम् भ॑वति भव त्य॒पाम् । \newline
19. अ॒पा मन्ते ऽन्ते॒ ऽपा म॒पा मन्ते᳚ । \newline
20. अन्ते᳚ क्रीणाति क्रीणा॒ त्यन्ते ऽन्ते᳚ क्रीणाति । \newline
21. क्री॒णा॒ति॒ सर॑सꣳ॒॒ सर॑सम् क्रीणाति क्रीणाति॒ सर॑सम् । \newline
22. सर॑स मे॒वैव सर॑सꣳ॒॒ सर॑स मे॒व । \newline
23. सर॑स॒मिति॒ स - र॒स॒म् । \newline
24. ए॒वैन॑ मेन मे॒वै वैन᳚म् । \newline
25. ए॒न॒म् क्री॒णा॒ति॒ क्री॒णा॒ त्ये॒न॒ मे॒न॒म् क्री॒णा॒ति॒ । \newline
26. क्री॒णा॒ त्य॒मात्यो॒ ऽमात्यः॑ क्रीणाति क्रीणा त्य॒मात्यः॑ । \newline
27. अ॒मात्यो᳚ ऽस्यस्य॒ मात्यो॒ ऽमात्यो॑ ऽसि । \newline
28. अ॒सीती त्य॑स्य॒ सीति॑ । \newline
29. इत्या॑हा॒हे तीत्या॑ह । \newline
30. आ॒हा॒मा ऽमा ऽऽहा॑ हा॒मा । \newline
31. अ॒मैवै वामा ऽमैव । \newline
32. ए॒वैन॑ मेन मे॒वै वैन᳚म् । \newline
33. ए॒न॒म् कु॒रु॒ते॒ कु॒रु॒त॒ ए॒न॒ मे॒न॒म् कु॒रु॒ते॒ । \newline
34. कु॒रु॒ते॒ शु॒क्रः शु॒क्रः कु॑रुते कुरुते शु॒क्रः । \newline
35. शु॒क्र स्ते॑ ते शु॒क्रः शु॒क्र स्ते᳚ । \newline
36. ते॒ ग्रहो॒ ग्रह॑ स्ते ते॒ ग्रहः॑ । \newline
37. ग्रह॒ इतीति॒ ग्रहो॒ ग्रह॒ इति॑ । \newline
38. इत्या॑हा॒हे तीत्या॑ह । \newline
39. आ॒ह॒ शु॒क्रः शु॒क्र आ॑हाह शु॒क्रः । \newline
40. शु॒क्रो हि हि शु॒क्रः शु॒क्रो हि । \newline
41. ह्य॑ स्यास्य॒ हि ह्य॑स्य । \newline
42. अ॒स्य॒ ग्रहो॒ ग्रहो᳚ ऽस्यास्य॒ ग्रहः॑ । \newline
43. ग्रहो ऽन॒सा ऽन॑सा॒ ग्रहो॒ ग्रहो ऽन॑सा । \newline
44. अन॒सा ऽच्छाच्छा न॒सा ऽन॒सा ऽच्छ॑ । \newline
45. अच्छ॑ याति या॒त्य च्छाच्छ॑ याति । \newline
46. या॒ति॒ म॒हि॒मान॑म् महि॒मानं॑ ॅयाति याति महि॒मान᳚म् । \newline
47. म॒हि॒मान॑ मे॒वैव म॑हि॒मान॑म् महि॒मान॑ मे॒व । \newline
48. ए॒वास्या᳚ स्यै॒वै वास्य॑ । \newline
49. अ॒स्या च्छा च्छा᳚ स्या॒स्या च्छ॑ । \newline
50. अच्छ॑ याति या॒त्य च्छाच्छ॑ याति । \newline
51. या॒त्यन॒सा ऽन॑सा याति या॒त्यन॑सा । \newline
52. अन॒सा ऽच्छाच्छा न॒सा ऽन॒सा ऽच्छ॑ । \newline

\textbf{Ghana Paata } \newline

1. इति॑ लोम॒तो लो॑म॒त इतीति॑ लोम॒त स्तस्य॒ तस्य॑ लोम॒त इतीति॑ लोम॒त स्तस्य॑ । \newline
2. लो॒म॒त स्तस्य॒ तस्य॑ लोम॒तो लो॑म॒त स्तस्य॑ मिमीत मिमीत॒ तस्य॑ लोम॒तो लो॑म॒त स्तस्य॑ मिमीत । \newline
3. तस्य॑ मिमीत मिमीत॒ तस्य॒ तस्य॑ मिमीतै॒ तदे॒तन् मि॑मीत॒ तस्य॒ तस्य॑ मिमी तै॒तत् । \newline
4. मि॒मी॒ तै॒त दे॒तन् मि॑मीत मिमी तै॒तद् वै वा ए॒तन् मि॑मीत मिमी तै॒तद् वै । \newline
5. ए॒तद् वै वा ए॒त दे॒तद् वै प॑शू॒नाम् प॑शू॒नां ॅवा ए॒त दे॒तद् वै प॑शू॒नाम् । \newline
6. वै प॑शू॒नाम् प॑शू॒नां ॅवै वै प॑शू॒नाꣳ रू॒पꣳ रू॒पम् प॑शू॒नां ॅवै वै प॑शू॒नाꣳ रू॒पम् । \newline
7. प॒शू॒नाꣳ रू॒पꣳ रू॒पम् प॑शू॒नाम् प॑शू॒नाꣳ रू॒पꣳ रू॒पेण॑ रू॒पेण॑ रू॒पम् प॑शू॒नाम् प॑शू॒नाꣳ रू॒पꣳ रू॒पेण॑ । \newline
8. रू॒पꣳ रू॒पेण॑ रू॒पेण॑ रू॒पꣳ रू॒पꣳ रू॒पे णै॒वैव रू॒पेण॑ रू॒पꣳ रू॒पꣳ रू॒पेणै॒व । \newline
9. रू॒पे णै॒वैव रू॒पेण॑ रू॒पे णै॒वास्मा॑ अस्मा ए॒व रू॒पेण॑ रू॒पे णै॒वास्मै᳚ । \newline
10. ए॒वास्मा॑ अस्मा ए॒वै वास्मै॑ प॒शून् प॒शून॑स्मा ए॒वै वास्मै॑ प॒शून् । \newline
11. अ॒स्मै॒ प॒शून् प॒शू न॑स्मा अस्मै प॒शून वाव॑ प॒शू न॑स्मा अस्मै प॒शूनव॑ । \newline
12. प॒शून-वाव॑ प॒शून् प॒शूनव॑ रुन्धे रु॒न्धे ऽव॑ प॒शून् प॒शूनव॑ रुन्धे । \newline
13. अव॑ रुन्धे रु॒न्धे ऽवाव॑ रुन्धे पशु॒मान् प॑शु॒मान् रु॒न्धे ऽवाव॑ रुन्धे पशु॒मान् । \newline
14. रु॒न्धे॒ प॒शु॒मान् प॑शु॒मान् रु॑न्धे रुन्धे पशु॒माने॒ वैव प॑शु॒मान् रु॑न्धे रुन्धे पशु॒माने॒व । \newline
15. प॒शु॒माने॒ वैव प॑शु॒मान् प॑शु॒माने॒व भ॑वति भव त्ये॒व प॑शु॒मान् प॑शु॒माने॒व भ॑वति । \newline
16. प॒शु॒मानिति॑ पशु - मान् । \newline
17. ए॒व भ॑वति भव त्ये॒वैव भ॑व त्य॒पा म॒पाम् भ॑व त्ये॒वैव भ॑व त्य॒पाम् । \newline
18. भ॒व॒ त्य॒पा म॒पाम् भ॑वति भव त्य॒पा मन्ते ऽन्ते॒ ऽपाम् भ॑वति भव त्य॒पा मन्ते᳚ । \newline
19. अ॒पा मन्ते ऽन्ते॒ ऽपा म॒पा मन्ते᳚ क्रीणाति क्रीणा॒ त्यन्ते॒ ऽपा म॒पा मन्ते᳚ क्रीणाति । \newline
20. अन्ते᳚ क्रीणाति क्रीणा॒ त्यन्ते ऽन्ते᳚ क्रीणाति॒ सर॑सꣳ॒॒ सर॑सम् क्रीणा॒ त्यन्ते ऽन्ते᳚ क्रीणाति॒ सर॑सम् । \newline
21. क्री॒णा॒ति॒ सर॑सꣳ॒॒ सर॑सम् क्रीणाति क्रीणाति॒ सर॑स मे॒वैव सर॑सम् क्रीणाति क्रीणाति॒ सर॑स मे॒व । \newline
22. सर॑स मे॒वैव सर॑सꣳ॒॒ सर॑स मे॒वैन॑ मेन मे॒व सर॑सꣳ॒॒ सर॑स मे॒वैन᳚म् । \newline
23. सर॑स॒मिति॒ स - र॒स॒म् । \newline
24. ए॒वैन॑ मेन मे॒वै वैन॑म् क्रीणाति क्रीणा त्येन मे॒वै वैन॑म् क्रीणाति । \newline
25. ए॒न॒म् क्री॒णा॒ति॒ क्री॒णा॒ त्ये॒न॒ मे॒न॒म् क्री॒णा॒ त्य॒मात्यो॒ ऽमात्यः॑ क्रीणा त्येन मेनम् क्रीणा त्य॒मात्यः॑ । \newline
26. क्री॒णा॒ त्य॒मात्यो॒ ऽमात्यः॑ क्रीणाति क्रीणा त्य॒मात्यो᳚ ऽस्यस्य॒ मात्यः॑ क्रीणाति क्रीणा त्य॒मात्यो॑ ऽसि । \newline
27. अ॒मात्यो᳚ ऽस्यस्य॒ मात्यो॒ ऽमात्यो॒ ऽसीती त्य॑स्य॒ मात्यो॒ ऽमात्यो॒ ऽसीति॑ । \newline
28. अ॒सी तीत्य॑ स्य॒सी त्या॑हा॒हे त्य॑स्य॒ सीत्या॑ह । \newline
29. इत्या॑हा॒हे तीत्या॑हा॒मा ऽमा ऽऽहे तीत्या॑ हा॒मा । \newline
30. आ॒हा॒मा ऽमा ऽऽहा॑हा॒मै वैवामा ऽऽहा॑हा॒ मैव । \newline
31. अ॒मैवै वामा ऽमै वैन॑ मेन मे॒वामा ऽमै वैन᳚म् । \newline
32. ए॒वैन॑ मेन मे॒वै वैन॑म् कुरुते कुरुत एन मे॒वै वैन॑म् कुरुते । \newline
33. ए॒न॒म् कु॒रु॒ते॒ कु॒रु॒त॒ ए॒न॒ मे॒न॒म् कु॒रु॒ते॒ शु॒क्रः शु॒क्रः कु॑रुत एन मेनम् कुरुते शु॒क्रः । \newline
34. कु॒रु॒ते॒ शु॒क्रः शु॒क्रः कु॑रुते कुरुते शु॒क्र स्ते॑ ते शु॒क्रः कु॑रुते कुरुते शु॒क्र स्ते᳚ । \newline
35. शु॒क्र स्ते॑ ते शु॒क्रः शु॒क्र स्ते॒ ग्रहो॒ ग्रह॑ स्ते शु॒क्रः शु॒क्र स्ते॒ ग्रहः॑ । \newline
36. ते॒ ग्रहो॒ ग्रह॑ स्ते ते॒ ग्रह॒ इतीति॒ ग्रह॑ स्ते ते॒ ग्रह॒ इति॑ । \newline
37. ग्रह॒ इतीति॒ ग्रहो॒ ग्रह॒ इत्या॑हा॒ हेति॒ ग्रहो॒ ग्रह॒ इत्या॑ह । \newline
38. इत्या॑हा॒हे तीत्या॑ह शु॒क्रः शु॒क्र आ॒हे तीत्या॑ह शु॒क्रः । \newline
39. आ॒ह॒ शु॒क्रः शु॒क्र आ॑हाह शु॒क्रो हि हि शु॒क्र आ॑हाह शु॒क्रो हि । \newline
40. शु॒क्रो हि हि शु॒क्रः शु॒क्रो ह्य॑स्यास्य॒ हि शु॒क्रः शु॒क्रो ह्य॑स्य । \newline
41. ह्य॑स्यास्य॒ हि ह्य॑स्य॒ ग्रहो॒ ग्रहो᳚ ऽस्य॒ हि ह्य॑स्य॒ ग्रहः॑ । \newline
42. अ॒स्य॒ ग्रहो॒ ग्रहो᳚ ऽस्यास्य॒ ग्रहो ऽन॒सा ऽन॑सा॒ ग्रहो᳚ ऽस्यास्य॒ ग्रहो ऽन॑सा । \newline
43. ग्रहो ऽन॒सा ऽन॑सा॒ ग्रहो॒ ग्रहो ऽन॒सा ऽच्छा च्छान॑सा॒ ग्रहो॒ ग्रहो ऽन॒सा ऽच्छ॑ । \newline
44. अन॒सा ऽच्छा च्छान॒सा ऽन॒सा ऽच्छ॑ याति या॒त्य च्छान॒सा ऽन॒सा ऽच्छ॑ याति । \newline
45. अच्छ॑ याति या॒त्यच्छा च्छ॑ याति महि॒मान॑म् महि॒मानं॑ ॅया॒त्यच्छा च्छ॑ याति महि॒मान᳚म् । \newline
46. या॒ति॒ म॒हि॒मान॑म् महि॒मानं॑ ॅयाति याति महि॒मान॑ मे॒वैव म॑हि॒मानं॑ ॅयाति याति महि॒मान॑ मे॒व । \newline
47. म॒हि॒मान॑ मे॒वैव म॑हि॒मान॑म् महि॒मान॑ मे॒वास्या᳚ स्यै॒व म॑हि॒मान॑म् महि॒मान॑ मे॒वास्य॑ । \newline
48. ए॒वास्या᳚ स्यै॒वै वास्या च्छा च्छा᳚ स्यै॒वै वास्याच्छ॑ । \newline
49. अ॒स्याच्छा च्छा᳚स्या॒ स्याच्छ॑ याति या॒त्यच्छा᳚ स्या॒ स्याच्छ॑ याति । \newline
50. अच्छ॑ याति या॒त्यच्छा च्छ॑ या॒त्यन॒सा ऽन॑सा या॒त्यच्छा च्छ॑ या॒त्यन॑सा । \newline
51. या॒त्यन॒सा ऽन॑सा याति या॒त्यन॒सा ऽच्छा च्छा न॑सा याति या॒त्यन॒सा ऽच्छ॑ । \newline
52. अन॒सा ऽच्छाच्छा न॒सा ऽन॒सा ऽच्छ॑ याति या॒त्यच्छा न॒सा ऽन॒सा ऽच्छ॑ याति । \newline
\pagebreak
\markright{ TS 6.1.9.4  \hfill https://www.vedavms.in \hfill}

\section{ TS 6.1.9.4 }

\textbf{TS 6.1.9.4 } \newline
\textbf{Samhita Paata} \newline

ऽच्छ॑ याति॒ तस्मा॑दनोवा॒ह्यꣳ॑ स॒मे जीव॑नं॒ ॅयत्र॒ खलु॒ वा ए॒तꣳ शी॒र्ष्णा हर॑न्ति॒ तस्मा᳚च्छीर्.षहा॒र्यं॑ गि॒रौ जीव॑नम॒भि त्यं दे॒वꣳ स॑वि॒तार॒मित्यति॑-च्छन्दस॒र्चा मि॑मी॒ते ऽति॑च्छन्दा॒ वै सर्वा॑णि॒ छन्दाꣳ॑सि॒ सर्वे॑भिरे॒वैनं॒ छन्दो॑भिर्मिमीते॒ वर्ष्म॒ वा ए॒षा छन्द॑सां॒ ॅयदति॑च्छन्दा॒ यदति॑च्छन्दस॒र्चा मिमी॑ते॒ वर्ष्मै॒वैनꣳ॑ समा॒नानां᳚ करो॒त्येक॑यैकयो॒थ् सर्गं॑- [  ] \newline

\textbf{Pada Paata} \newline

अच्छ॑ । या॒ति॒ । तस्मा᳚त् । अ॒नो॒वा॒ह्य॑मित्य॑नः - वा॒ह्य᳚म् । स॒मे । जीव॑नम् । यत्र॑ । खलु॑ । वै । ए॒तम् । शी॒र्ष्णा । हर॑न्ति । तस्मा᳚त् । शी॒र्॒.ष॒हा॒र्य॑मिति॑ शीर्.ष-हा॒र्य᳚म् । गि॒रौ । जीव॑नम् । अ॒भीति॑ । त्यम् । दे॒वम् । स॒वि॒तार᳚म् । इति॑ । अति॑च्छन्द॒सेत्यति॑ - छ॒न्द॒सा॒ । ऋ॒चा । मि॒मी॒त॒ । अति॑च्छन्दा॒ इत्यति॑ - छ॒न्दाः॒ । वै । सर्वा॑णि । छन्दाꣳ॑सि । सर्वे॑भिः । ए॒व । ए॒न॒म् । छन्दो॑भि॒रिति॒ छन्दः॑-भिः॒ । मि॒मी॒ते॒ । वर्ष्म॑ । वै । ए॒षा । छन्द॑साम् । यत् । अति॑च्छन्दा॒ इत्यति॑ - छ॒न्दाः॒ । यत् । अति॑च्छन्द॒सेत्यति॑-छ॒न्द॒सा॒ । ऋ॒चा । मिमी॑ते । वर्ष्म॑ । ए॒व । ए॒न॒म् । स॒मा॒नाना᳚म् । क॒रो॒ति॒ । एक॑यैक॒येत्येक॑या - ए॒क॒या॒ । उ॒थ्सर्ग॒मित्यु॑त् - सर्ग᳚म् ।  \newline


\textbf{Krama Paata} \newline

अच्छ॑ याति । या॒ति॒ तस्मा᳚त् । तस्मा॑दनोवा॒ह्य᳚म् । अ॒नो॒वा॒ह्यꣳ॑ स॒मे । अ॒नो॒वा॒ह्य॑मित्य॑नः - वा॒ह्य᳚म् । स॒मे जीव॑नम् । जीव॑न॒म् ॅयत्र॑ । यत्र॒ खलु॑ । खलु॒ वै । वा ए॒तम् । ए॒तꣳ शी॒र्ष्णा । शी॒र्ष्णा हर॑न्ति । हर॑न्ति॒ तस्मा᳚त् । तस्मा᳚च्छीर्.षहा॒र्य᳚म् । शी॒र्॒.ष॒हा॒र्य॑म् गि॒रौ । शी॒र्॒.ष॒हा॒र्य॑मिति॑ शीर्.ष - हा॒र्य᳚म् । गि॒रौ जीव॑नम् । जीव॑नम॒भि । अ॒भि त्यम् । त्यम् दे॒वम् । दे॒वꣳ स॑वि॒तार᳚म् । स॒वि॒तार॒मिति॑ । इत्यति॑च्छन्दसा । अति॑च्छन्दस॒र्चा । अति॑च्छन्द॒सेत्यति॑ - छ॒न्द॒सा॒ । ऋ॒चा मि॑मीते । मि॒मी॒तेऽति॑च्छन्दाः । अति॑च्छन्दा॒ वै । अति॑च्छन्दा॒ इत्यति॑ - छ॒न्दाः॒ । वै सर्वा॑णि । सर्वा॑णि॒ छन्दाꣳ॑सि । छन्दाꣳ॑सि॒ सर्वे॑भिः । सर्वे॑भिरे॒व । ए॒वैन᳚म् । ए॒न॒म् छन्दो॑भिः । छन्दो॑भिर् मिमीते । छन्दो॑भि॒रिति॒ छन्दः॑ - भिः॒ । मि॒मी॒ते॒ वर्ष्म॑ । वर्ष्म॒ वै । वा ए॒षा । ए॒षा छन्द॑साम् । छन्द॑सा॒म् ॅयत् । यदति॑च्छन्दाः । अति॑च्छन्दा॒ यत् । अति॑च्छन्दा॒ इत्यति॑ - छ॒न्दाः॒ । यदति॑च्छन्दसा । अति॑च्छन्दस॒र्चा । अति॑च्छन्द॒सेत्यति॑ - छ॒न्द॒सा॒ । ऋ॒चा मिमी॑ते । मिमी॑ते॒ वर्ष्म॑ । वर्ष्मै॒व । ए॒वैन᳚म् । ए॒नꣳ॒॒ स॒मा॒नाना᳚म् । स॒मा॒नाना᳚म् करोति । क॒रो॒त्येक॑यैकया । एक॑यैकयो॒थ्सर्ग᳚म् । एक॑यैक॒येत्येक॑या - ए॒क॒या॒ । उ॒थ्सर्ग॑म् मिमीते । उ॒थ्सर्ग॒मित्यु॑त् - सर्ग᳚म् \newline

\textbf{Jatai Paata} \newline

1. अच्छ॑ याति या॒त्य च्छाच्छ॑ याति । \newline
2. या॒ति॒ तस्मा॒त् तस्मा᳚द् याति याति॒ तस्मा᳚त् । \newline
3. तस्मा॑ दनोवा॒ह्य॑ मनोवा॒ह्य॑म् तस्मा॒त् तस्मा॑ दनोवा॒ह्य᳚म् । \newline
4. अ॒नो॒वा॒ह्यꣳ॑ स॒मे स॒मे॑ ऽनोवा॒ह्य॑ मनोवा॒ह्यꣳ॑ स॒मे । \newline
5. अ॒नो॒वा॒ह्य॑मित्य॑नः - वा॒ह्य᳚म् । \newline
6. स॒मे जीव॑न॒म् जीव॑नꣳ स॒मे स॒मे जीव॑नम् । \newline
7. जीव॑नं॒ ॅयत्र॒ यत्र॒ जीव॑न॒म् जीव॑नं॒ ॅयत्र॑ । \newline
8. यत्र॒ खलु॒ खलु॒ यत्र॒ यत्र॒ खलु॑ । \newline
9. खलु॒ वै वै खलु॒ खलु॒ वै । \newline
10. वा ए॒त मे॒तं ॅवै वा ए॒तम् । \newline
11. ए॒तꣳ शी॒र्ष्णा शी॒र्ष्णैत मे॒तꣳ शी॒र्ष्णा । \newline
12. शी॒र्ष्णा हर॑न्ति॒ हर॑न्ति शी॒र्ष्णा शी॒र्ष्णा हर॑न्ति । \newline
13. हर॑न्ति॒ तस्मा॒त् तस्मा॒ द्धर॑न्ति॒ हर॑न्ति॒ तस्मा᳚त् । \newline
14. तस्मा᳚ च्छीर्.षहा॒र्यꣳ॑ शीर्.षहा॒र्य॑म् तस्मा॒त् तस्मा᳚ च्छीर्.षहा॒र्य᳚म् । \newline
15. शी॒र्॒.ष॒हा॒र्य॑म् गि॒रौ गि॒रौ शी॑र्.षहा॒र्यꣳ॑ शीर्.षहा॒र्य॑म् गि॒रौ । \newline
16. शी॒र्॒.ष॒हा॒र्य॑मिति॑ शीर्.ष - हा॒र्य᳚म् । \newline
17. गि॒रौ जीव॑न॒म् जीव॑नम् गि॒रौ गि॒रौ जीव॑नम् । \newline
18. जीव॑न म॒भ्य॑भि जीव॑न॒म् जीव॑न म॒भि । \newline
19. अ॒भि त्यम् त्य म॒भ्य॑भि त्यम् । \newline
20. त्यम् दे॒वम् दे॒वम् त्यम् त्यम् दे॒वम् । \newline
21. दे॒वꣳ स॑वि॒तारꣳ॑ सवि॒तार॑म् दे॒वम् दे॒वꣳ स॑वि॒तार᳚म् । \newline
22. स॒वि॒तार॒ मितीति॑ सवि॒तारꣳ॑ सवि॒तार॒ मिति॑ । \newline
23. इत्यति॑च्छन्द॒सा ऽति॑च्छन्द॒सेती त्यति॑च्छन्दसा । \newline
24. अति॑च्छन्दस॒ र्‌च र्‌चा ऽति॑च्छन्द॒सा ऽति॑च्छन्दस॒ र्‌चा । \newline
25. अति॑च्छन्द॒सेत्यति॑ - छ॒न्द॒सा॒ । \newline
26. ऋ॒चा मि॑मीत मिमीत॒ र्‌च र्‌चा मि॑मीत । \newline
27. मि॒मी॒ता ति॑च्छन्दा॒ अति॑च्छन्दा मिमीत मिमी॒ता ति॑च्छन्दाः । \newline
28. अति॑च्छन्दा॒ वै वा अति॑च्छन्दा॒ अति॑च्छन्दा॒ वै । \newline
29. अति॑च्छन्दा॒ इत्यति॑ - छ॒न्दाः॒ । \newline
30. वै सर्वा॑णि॒ सर्वा॑णि॒ वै वै सर्वा॑णि । \newline
31. सर्वा॑णि॒ छन्दाꣳ॑सि॒ छन्दाꣳ॑सि॒ सर्वा॑णि॒ सर्वा॑णि॒ छन्दाꣳ॑सि । \newline
32. छन्दाꣳ॑सि॒ सर्वे॑भिः॒ सर्वे॑भि॒ श्छन्दाꣳ॑सि॒ छन्दाꣳ॑सि॒ सर्वे॑भिः । \newline
33. सर्वे॑भि रे॒वैव सर्वे॑भिः॒ सर्वे॑भि रे॒व । \newline
34. ए॒वैन॑ मेन मे॒वै वैन᳚म् । \newline
35. ए॒न॒म् छन्दो॑भि॒ श्छन्दो॑भि रेन मेन॒म् छन्दो॑भिः । \newline
36. छन्दो॑भिर् मिमीते मिमीते॒ छन्दो॑भि॒ श्छन्दो॑भिर् मिमीते । \newline
37. छन्दो॑भि॒रिति॒ छन्दः॑ - भिः॒ । \newline
38. मि॒मी॒ते॒ वर्ष्म॒ वर्ष्म॑ मिमीते मिमीते॒ वर्ष्म॑ । \newline
39. वर्ष्म॒ वै वै वर्ष्म॒ वर्ष्म॒ वै । \newline
40. वा ए॒षैषा वै वा ए॒षा । \newline
41. ए॒षा छन्द॑सा॒म् छन्द॑सा मे॒षैषा छन्द॑साम् । \newline
42. छन्द॑सां॒ ॅयद् यच् छन्द॑सा॒म् छन्द॑सां॒ ॅयत् । \newline
43. यदति॑च्छन्दा॒ अति॑च्छन्दा॒ यद् यदति॑च्छन्दाः । \newline
44. अति॑च्छन्दा॒ यद् यदति॑च्छन्दा॒ अति॑च्छन्दा॒ यत् । \newline
45. अति॑च्छन्दा॒ इत्यति॑ - छ॒न्दाः॒ । \newline
46. यदति॑च्छन्द॒सा ऽति॑च्छन्दसा॒ यद् यदति॑च्छन्दसा । \newline
47. अति॑च्छन्दस॒ र्‌च र्‌चा ऽति॑च्छन्द॒सा ऽति॑च्छन्दस॒ र्‌चा । \newline
48. अति॑च्छन्द॒सेत्यति॑ - छ॒न्द॒सा॒ । \newline
49. ऋ॒चा मिमी॑ते॒ मिमी॑त ऋ॒च र्‌चा मिमी॑ते । \newline
50. मिमी॑ते॒ वर्ष्म॒ वर्ष्म॒ मिमी॑ते॒ मिमी॑ते॒ वर्ष्म॑ । \newline
51. वर्ष्मै॒ वैव वर्ष्म॒ वर्ष्मै॒व । \newline
52. ए॒वैन॑ मेन मे॒वै वैन᳚म् । \newline
53. ए॒नꣳ॒॒ स॒मा॒नानाꣳ॑ समा॒नाना॑ मेन मेनꣳ समा॒नाना᳚म् । \newline
54. स॒मा॒नाना᳚म् करोति करोति समा॒नानाꣳ॑ समा॒नाना᳚म् करोति । \newline
55. क॒रो॒ त्येक॑यैक॒ यैक॑यैकया करोति करो॒ त्येक॑यैकया । \newline
56. एक॑यैक यो॒थ्सर्ग॑ मु॒थ्सर्ग॒ मेक॑यैक॒ यैक॑यैक यो॒थ्सर्ग᳚म् । \newline
57. एक॑यैक॒येत्येक॑या - ए॒क॒या॒ । \newline
58. उ॒थ्सर्ग॑म् मिमीते मिमीत उ॒थ्सर्ग॑ मु॒थ्सर्ग॑म् मिमीते । \newline
59. उ॒थ्सर्ग॒मित्यु॑त् - सर्ग᳚म् । \newline

\textbf{Ghana Paata } \newline

1. अच्छ॑ याति या॒त्यच्छा च्छ॑ याति॒ तस्मा॒त् तस्मा᳚द् या॒त्यच्छा च्छ॑ याति॒ तस्मा᳚त् । \newline
2. या॒ति॒ तस्मा॒त् तस्मा᳚द् याति याति॒ तस्मा॑ दनोवा॒ह्य॑ मनोवा॒ह्य॑म् तस्मा᳚द् याति याति॒ तस्मा॑ दनोवा॒ह्य᳚म् । \newline
3. तस्मा॑ दनोवा॒ह्य॑ मनोवा॒ह्य॑म् तस्मा॒त् तस्मा॑ दनोवा॒ह्यꣳ॑ स॒मे स॒मे॑ ऽनोवा॒ह्य॑म् तस्मा॒त् तस्मा॑ दनोवा॒ह्यꣳ॑ स॒मे । \newline
4. अ॒नो॒वा॒ह्यꣳ॑ स॒मे स॒मे॑ ऽनोवा॒ह्य॑ मनोवा॒ह्यꣳ॑ स॒मे जीव॑न॒म् जीव॑नꣳ स॒मे॑ ऽनोवा॒ह्य॑ मनोवा॒ह्यꣳ॑ स॒मे जीव॑नम् । \newline
5. अ॒नो॒वा॒ह्य॑मित्य॑नः - वा॒ह्य᳚म् । \newline
6. स॒मे जीव॑न॒म् जीव॑नꣳ स॒मे स॒मे जीव॑नं॒ ॅयत्र॒ यत्र॒ जीव॑नꣳ स॒मे स॒मे जीव॑नं॒ ॅयत्र॑ । \newline
7. जीव॑नं॒ ॅयत्र॒ यत्र॒ जीव॑न॒म् जीव॑नं॒ ॅयत्र॒ खलु॒ खलु॒ यत्र॒ जीव॑न॒म् जीव॑नं॒ ॅयत्र॒ खलु॑ । \newline
8. यत्र॒ खलु॒ खलु॒ यत्र॒ यत्र॒ खलु॒ वै वै खलु॒ यत्र॒ यत्र॒ खलु॒ वै । \newline
9. खलु॒ वै वै खलु॒ खलु॒ वा ए॒त मे॒तं ॅवै खलु॒ खलु॒ वा ए॒तम् । \newline
10. वा ए॒त मे॒तं ॅवै वा ए॒तꣳ शी॒र्ष्णा शी॒र्ष्णैतं ॅवै वा ए॒तꣳ शी॒र्ष्णा । \newline
11. ए॒तꣳ शी॒र्ष्णा शी॒र्ष्णैत मे॒तꣳ शी॒र्ष्णा हर॑न्ति॒ हर॑न्ति शी॒र्ष्णैत मे॒तꣳ शी॒र्ष्णा हर॑न्ति । \newline
12. शी॒र्ष्णा हर॑न्ति॒ हर॑न्ति शी॒र्ष्णा शी॒र्ष्णा हर॑न्ति॒ तस्मा॒त् तस्मा॒ द्धर॑न्ति शी॒र्ष्णा शी॒र्ष्णा हर॑न्ति॒ तस्मा᳚त् । \newline
13. हर॑न्ति॒ तस्मा॒त् तस्मा॒ द्धर॑न्ति॒ हर॑न्ति॒ तस्मा᳚च् छीर्.षहा॒र्यꣳ॑ शीर्.षहा॒र्य॑म् तस्मा॒ द्धर॑न्ति॒ हर॑न्ति॒ तस्मा᳚ च्छीर्.षहा॒र्य᳚म् । \newline
14. तस्मा᳚च् छीर्.षहा॒र्यꣳ॑ शीर्.षहा॒र्य॑म् तस्मा॒त् तस्मा᳚च् छीर्.षहा॒र्य॑म् गि॒रौ गि॒रौ शी॑र्.षहा॒र्य॑म् तस्मा॒त् तस्मा᳚च् छीर्.षहा॒र्य॑म् गि॒रौ । \newline
15. शी॒र्॒.ष॒हा॒र्य॑म् गि॒रौ गि॒रौ शी॑र्.षहा॒र्यꣳ॑ शीर्.षहा॒र्य॑म् गि॒रौ जीव॑न॒म् जीव॑नम् गि॒रौ शी॑र्.षहा॒र्यꣳ॑ शीर्.षहा॒र्य॑म् गि॒रौ जीव॑नम् । \newline
16. शी॒र्॒.ष॒हा॒र्य॑मिति॑ शीर्.ष - हा॒र्य᳚म् । \newline
17. गि॒रौ जीव॑न॒म् जीव॑नम् गि॒रौ गि॒रौ जीव॑न म॒भ्य॑भि जीव॑नम् गि॒रौ गि॒रौ जीव॑न म॒भि । \newline
18. जीव॑न म॒भ्य॑भि जीव॑न॒म् जीव॑न म॒भि त्यम् त्य म॒भि जीव॑न॒म् जीव॑न म॒भि त्यम् । \newline
19. अ॒भि त्यम् त्य म॒भ्य॑भि त्यम् दे॒वम् दे॒वम् त्य म॒भ्य॑भि त्यम् दे॒वम् । \newline
20. त्यम् दे॒वम् दे॒वम् त्यम् त्यम् दे॒वꣳ स॑वि॒तारꣳ॑ सवि॒तार॑म् दे॒वम् त्यम् त्यम् दे॒वꣳ स॑वि॒तार᳚म् । \newline
21. दे॒वꣳ स॑वि॒तारꣳ॑ सवि॒तार॑म् दे॒वम् दे॒वꣳ स॑वि॒तार॒ मितीति॑ सवि॒तार॑म् दे॒वम् दे॒वꣳ स॑वि॒तार॒ मिति॑ । \newline
22. स॒वि॒तार॒ मितीति॑ सवि॒तारꣳ॑ सवि॒तार॒ मित्यति॑च्छन्द॒सा ऽति॑च्छन्द॒सेति॑ सवि॒तारꣳ॑ सवि॒तार॒ मित्यति॑च्छन्दसा । \newline
23. इत्यति॑च्छन्द॒सा ऽति॑च्छन्द॒से तीत्यति॑च्छन्दस॒ र्‌च र्‌चा ऽति॑च्छन्द॒से तीत्यति॑च्छन्दस॒ र्‌चा । \newline
24. अति॑च्छन्दस॒ र्‌च र्‌चा ऽति॑च्छन्द॒सा ऽति॑च्छन्दस॒ र्‌चा मि॑मीत मिमीत॒ र्‌चा ऽति॑च्छन्द॒सा ऽति॑च्छन्दस॒ र्‌चा मि॑मीत । \newline
25. अति॑च्छन्द॒सेत्यति॑ - छ॒न्द॒सा॒ । \newline
26. ऋ॒चा मि॑मीत मिमीत॒ र्‌च र्‌चा मि॑मी॒ता ति॑च्छन्दा॒ अति॑च्छन्दा मिमीत॒ र्‌च र्‌चा मि॑मी॒ता ति॑च्छन्दाः । \newline
27. मि॒मी॒ता ति॑च्छन्दा॒ अति॑च्छन्दा मिमीत मिमी॒ता ति॑च्छन्दा॒ वै वा अति॑च्छन्दा मिमीत मिमी॒ता ति॑च्छन्दा॒ वै । \newline
28. अति॑च्छन्दा॒ वै वा अति॑च्छन्दा॒ अति॑च्छन्दा॒ वै सर्वा॑णि॒ सर्वा॑णि॒ वा अति॑च्छन्दा॒ अति॑च्छन्दा॒ वै सर्वा॑णि । \newline
29. अति॑च्छन्दा॒ इत्यति॑ - छ॒न्दाः॒ । \newline
30. वै सर्वा॑णि॒ सर्वा॑णि॒ वै वै सर्वा॑णि॒ छन्दाꣳ॑सि॒ छन्दाꣳ॑सि॒ सर्वा॑णि॒ वै वै सर्वा॑णि॒ छन्दाꣳ॑सि । \newline
31. सर्वा॑णि॒ छन्दाꣳ॑सि॒ छन्दाꣳ॑सि॒ सर्वा॑णि॒ सर्वा॑णि॒ छन्दाꣳ॑सि॒ सर्वे॑भिः॒ सर्वे॑भि॒ श्छन्दाꣳ॑सि॒ सर्वा॑णि॒ सर्वा॑णि॒ छन्दाꣳ॑सि॒ सर्वे॑भिः । \newline
32. छन्दाꣳ॑सि॒ सर्वे॑भिः॒ सर्वे॑भि॒ श्छन्दाꣳ॑सि॒ छन्दाꣳ॑सि॒ सर्वे॑भि रे॒वैव सर्वे॑भि॒ श्छन्दाꣳ॑सि॒ छन्दाꣳ॑सि॒ सर्वे॑भि रे॒व । \newline
33. सर्वे॑भि रे॒वैव सर्वे॑भिः॒ सर्वे॑भि रे॒वैन॑ मेन मे॒व सर्वे॑भिः॒ सर्वे॑भि रे॒वैन᳚म् । \newline
34. ए॒वैन॑ मेन मे॒वै वैन॒म् छन्दो॑भि॒ श्छन्दो॑भि रेन मे॒वै वैन॒म् छन्दो॑भिः । \newline
35. ए॒न॒म् छन्दो॑भि॒ श्छन्दो॑भि रेन मेन॒म् छन्दो॑भिर् मिमीते मिमीते॒ छन्दो॑भि रेन मेन॒म् छन्दो॑भिर् मिमीते । \newline
36. छन्दो॑भिर् मिमीते मिमीते॒ छन्दो॑भि॒ श्छन्दो॑भिर् मिमीते॒ वर्ष्म॒ वर्ष्म॑ मिमीते॒ छन्दो॑भि॒ श्छन्दो॑भिर् मिमीते॒ वर्ष्म॑ । \newline
37. छन्दो॑भि॒रिति॒ छन्दः॑ - भिः॒ । \newline
38. मि॒मी॒ते॒ वर्ष्म॒ वर्ष्म॑ मिमीते मिमीते॒ वर्ष्म॒ वै वै वर्ष्म॑ मिमीते मिमीते॒ वर्ष्म॒ वै । \newline
39. वर्ष्म॒ वै वै वर्ष्म॒ वर्ष्म॒ वा ए॒षैषा वै वर्ष्म॒ वर्ष्म॒ वा ए॒षा । \newline
40. वा ए॒षैषा वै वा ए॒षा छन्द॑सा॒म् छन्द॑सा मे॒षा वै वा ए॒षा छन्द॑साम् । \newline
41. ए॒षा छन्द॑सा॒म् छन्द॑सा मे॒षैषा छन्द॑सां॒ ॅयद् यच् छन्द॑सा मे॒षैषा छन्द॑सां॒ ॅयत् । \newline
42. छन्द॑सां॒ ॅयद् यच् छन्द॑सा॒म् छन्द॑सां॒ ॅयदति॑च्छन्दा॒ अति॑च्छन्दा॒ यच् छन्द॑सा॒म् छन्द॑सां॒ ॅयदति॑च्छन्दाः । \newline
43. यदति॑च्छन्दा॒ अति॑च्छन्दा॒ यद् यदति॑च्छन्दा॒ यद् यदति॑च्छन्दा॒ यद् यदति॑च्छन्दा॒ यत् । \newline
44. अति॑च्छन्दा॒ यद् यदति॑च्छन्दा॒ अति॑च्छन्दा॒ यदति॑च्छन्द॒सा ऽति॑च्छन्दसा॒ यदति॑च्छन्दा॒  अति॑च्छन्दा॒ यदति॑च्छन्दसा । \newline
45. अति॑च्छन्दा॒ इत्यति॑ - छ॒न्दाः॒ । \newline
46. यदति॑च्छन्द॒सा ऽति॑च्छन्दसा॒ यद् यदति॑च्छन्दस॒ र्‌च र्‌चा ऽति॑च्छन्दसा॒ यद् यदति॑च्छन्दस॒ र्‌चा । \newline
47. अति॑च्छन्दस॒ र्‌च र्‌चा ऽति॑च्छन्द॒सा ऽति॑च्छन्दस॒ र्‌चा मिमी॑ते॒ मिमी॑त ऋ॒चा ऽति॑च्छन्द॒सा ऽति॑च्छन्दस॒ र्‌चा मिमी॑ते । \newline
48. अति॑च्छन्द॒सेत्यति॑ - छ॒न्द॒सा॒ । \newline
49. ऋ॒चा मिमी॑ते॒ मिमी॑त ऋ॒च र्‌चा मिमी॑ते॒ वर्ष्म॒ वर्ष्म॒ मिमी॑त ऋ॒च र्‌चा मिमी॑ते॒ वर्ष्म॑ । \newline
50. मिमी॑ते॒ वर्ष्म॒ वर्ष्म॒ मिमी॑ते॒ मिमी॑ते॒ वर्ष्मै॒ वैव वर्ष्म॒ मिमी॑ते॒ मिमी॑ते॒ वर्ष्मै॒व । \newline
51. वर्ष्मै॒ वैव वर्ष्म॒ वर्ष्मै॒ वैन॑ मेन मे॒व वर्ष्म॒ वर्ष्मै॒ वैन᳚म् । \newline
52. ए॒वैन॑ मेन मे॒वै वैनꣳ॑ समा॒नानाꣳ॑ समा॒नाना॑ मेन मे॒वै वैनꣳ॑ समा॒नाना᳚म् । \newline
53. ए॒नꣳ॒॒ स॒मा॒नानाꣳ॑ समा॒नाना॑ मेन मेनꣳ समा॒नाना᳚म् करोति करोति समा॒नाना॑ मेन मेनꣳ समा॒नाना᳚म् करोति । \newline
54. स॒मा॒नाना᳚म् करोति करोति समा॒नानाꣳ॑ समा॒नाना᳚म् करो॒ त्येक॑यैक॒ यैक॑यैकया करोति समा॒नानाꣳ॑ समा॒नाना᳚म् करो॒ त्येक॑यैकया । \newline
55. क॒रो॒ त्येक॑यैक॒ यैक॑यैकया करोति करो॒ त्येक॑यैक यो॒थ्सर्ग॑ मु॒थ्सर्ग॒ मेक॑यैकया करोति करो॒ त्येक॑यैक यो॒थ्सर्ग᳚म् । \newline
56. एक॑यैक यो॒थ्सर्ग॑ मु॒थ्सर्ग॒ मेक॑यैक॒ यैक॑यैक यो॒थ्सर्ग॑म् मिमीते मिमीत उ॒थ्सर्ग॒ मेक॑यैक॒ यैक॑यैक यो॒थ्सर्ग॑म् मिमीते । \newline
57. एक॑यैक॒येत्येक॑या - ए॒क॒या॒ । \newline
58. उ॒थ्सर्ग॑म् मिमीते मिमीत उ॒थ्सर्ग॑ मु॒थ्सर्ग॑म् मिमी॒ते ऽया॑तयाम्नियायातयाम्नि॒या ऽया॑तयाम्नियायातयाम्निया मिमीत उ॒थ्सर्ग॑ मु॒थ्सर्ग॑म् मिमी॒ते ऽया॑तयाम्नियायातयाम्निया । \newline
59. उ॒थ्सर्ग॒मित्यु॑त् - सर्ग᳚म् । \newline
\pagebreak
\markright{ TS 6.1.9.5  \hfill https://www.vedavms.in \hfill}

\section{ TS 6.1.9.5 }

\textbf{TS 6.1.9.5 } \newline
\textbf{Samhita Paata} \newline

मिमी॒ते ऽया॑तयाम्नियायातयाम्नियै॒वैनं॑ मिमीते॒ तस्मा॒न्नाना॑वीर्या अ॒ङ्गुल॑यः॒ सर्वा᳚स्वङ्गु॒ष्ठमुप॒ नि गृ॑ह्णाति॒ तस्मा᳚थ् स॒माव॑द्वीर्यो॒ऽन्याभि॑-र॒ङ्गुलि॑भि॒स्तस्मा॒थ् सर्वा॒ अनु॒ सं च॑रति॒ यथ् स॒ह सर्वा॑भि॒र्मिमी॑त॒ सꣳश्लि॑ष्टा अ॒ङ्गुल॑यो जायेर॒-न्नेक॑यैकयो॒थ् सर्गं॑ मिमीते॒ तस्मा॒द् विभ॑क्ता जायन्ते॒ पञ्च॒ कृत्वो॒ यजु॑षा मिमीते॒ पञ्चा᳚क्षरा प॒ङ्क्तिः पाङ्क्तो॑ य॒ज्ञो य॒ज्ञ्मे॒वाव॑ रुन्धे॒ पञ्च॒ कृत्व॑स्तू॒ष्णीं- [  ] \newline

\textbf{Pada Paata} \newline

मि॒मी॒ते॒ । अया॑तयाम्नियायातयाम्नि॒येत्यया॑तयाम्निया-अ॒या॒त॒या॒म्नि॒या॒ । ए॒व । ए॒न॒म् । मि॒मी॒ते॒ । तस्मा᳚त् । नाना॑वीर्या॒ इति॒ नाना᳚ - वी॒र्याः॒ । अ॒ङ्गुल॑यः । सर्वा॑सु । अ॒ङ्गु॒ष्ठम् । उप॑ । नीति॑ । गृ॒ह्णा॒ति॒ । तस्मा᳚त् । स॒माव॑द्वीर्य॒ इति॑ स॒माव॑त् - वी॒र्यः॒ । अ॒न्याभिः॑ । अ॒ङ्गुलि॑भि॒रित्य॒ङ्गुलि॑ - भिः॒ । तस्मा᳚त् । सर्वाः᳚ । अनु॑ । समिति॑ । च॒र॒ति॒ । यत् । स॒ह । सर्वा॑भिः । मिमी॑त । सꣳश्लि॑ष्टा॒ इति॒ सं - श्लि॒ष्टाः॒ । अ॒ङ्गुल॑यः । जा॒ये॒र॒न्न् । एक॑यैक॒येत्येक॑या-ए॒क॒या॒ । उ॒थ्सर्ग॒मित्यु॑त् - सर्ग᳚म् । मि॒मी॒ते॒ । तस्मा᳚त् । विभ॑क्ता॒ इति॒ वि - भ॒क्ताः॒ । जा॒य॒न्ते॒ । पञ्च॑ । कृत्वः॑ । यजु॑षा । मि॒मी॒ते॒ । पञ्चा᳚क्ष॒रेति॒ पञ्च॑ - अ॒क्ष॒रा॒ । प॒ङ्क्तिः । पाङ्क्तः॑ । य॒ज्ञ्ः । य॒ज्ञ्म् । ए॒व । अवेति॑ । रु॒न्धे॒ । पञ्च॑ । कृत्वः॑ । तू॒ष्णीम् ।  \newline


\textbf{Krama Paata} \newline

मि॒मी॒तेऽया॑तयाम्नियायातयाम्निया । अया॑तयाम्नियायातयाम्नियै॒व । अया॑तयाम्नियायातयाम्नि॒येत्यया॑तयाम्निया - अ॒या॒त॒या॒म्नि॒या॒ । ए॒वैन᳚म् । ए॒न॒म् मि॒मी॒ते॒ । मि॒मी॒ते॒ तस्मा᳚त् । तस्मा॒न् नाना॑वीर्याः । नाना॑वीर्या अ॒ङ्‍गुल॑यः । नाना॑वीर्या॒ इति॒ नाना᳚ - वी॒र्याः॒ । अ॒ङ्‍गुल॑यः॒ सर्वा॑सु । सर्वा᳚स्वङ्‍गु॒ष्ठम् । अ॒ङ्‍गु॒ष्ठमुप॑ । उप॒ नि । नि गृ॑ह्णाति । गृ॒ह्णा॒ति॒ तस्मा᳚त् । तस्मा᳚थ् स॒माव॑द्वीर्यः । स॒माव॑द्वीर्यो॒ऽन्याभिः॑ । स॒माव॑द्वीर्य॒ इति॑ स॒माव॑त् - वी॒र्यः॒ । अ॒न्याभि॑र॒ङ्‍गुलि॑भिः । अ॒ङ्‍गुलि॑भि॒,स्तस्मा᳚त् । अ॒ङ्‍गुलि॑भि॒रित्य॒ङ्‍गुलि॑ - भिः॒ । तस्मा॒थ् सर्वाः᳚ । सर्वा॒ अनु॑ । अनु॒ सम् । सम् च॑रति । च॒र॒ति॒ यत् । यथ् स॒ह । स॒ह सर्वा॑भिः । सर्वा॑भि॒र् मिमी॑त । मिमी॑त॒ सꣳश्लि॑ष्टाः । सꣳश्लि॑ष्टा अ॒ङ्‍गुल॑यः । सꣳश्लि॑ष्टा॒ इति॒ सम् - श्लि॒ष्टाः॒ । अ॒ङ्‍गुल॑यो जायेरन्न् । जा॒ये॒र॒न्नैक॑यैकया । एक॑यैकयो॒थ्सर्ग᳚म् । एक॑यैक॒येत्येक॑या - ए॒क॒या॒ । उ॒थ्सर्ग॑म् मिमीते । उ॒थ्सर्ग॒मित्यु॑त् - सर्ग᳚म् । मि॒मी॒ते॒ तस्मा᳚त् । तस्मा॒द् विभ॑क्ताः । विभ॑क्ता जायन्ते । विभ॑क्ता॒ इति॒ वि - भ॒क्ताः॒ । जा॒य॒न्ते॒ पञ्च॑ । पञ्च॒ कृत्वः॑ । कृत्वो॒ यजु॑षा । यजु॑षा मिमीते । मि॒मी॒ते॒ पञ्चा᳚क्षरा । पञ्चा᳚क्षरा प॒ङ्‍क्तिः । पञ्चा᳚क्ष॒रेति॒ पञ्च॑ - अ॒क्ष॒रा॒ । प॒ङ्‍क्तिः पाङ्‍क्तः॑ । पाङ्‍क्तो॑ य॒ज्ञ्ः । य॒ज्ञो य॒ज्ञ्म् । य॒ज्ञ्मे॒व । ए॒वाव॑ । अव॑ रुन्धे । रु॒न्धे॒ पञ्च॑ । पञ्च॒ कृत्वः॑ । कृत्व॑स्तू॒ष्णीम् । तू॒ष्णीम् दश॑ \newline

\textbf{Jatai Paata} \newline

1. मि॒मी॒ते ऽया॑तयाम्नियायातयाम्नि॒या ऽया॑तयाम्नियायातयाम्निया मिमीते मिमी॒ते ऽया॑तयाम्नियायातयाम्निया । \newline
2. अया॑तयाम्नियायातयाम्नि यै॒वैवा या॑तयाम्नियायातयाम्नि॒या ऽया॑तयाम्नियायातयाम्नि यै॒व । \newline
3. अया॑तयाम्नियायातयाम्नि॒येत्यया॑तयाम्निया - अ॒या॒त॒या॒म्नि॒या॒ । \newline
4. ए॒वैन॑ मेन मे॒वै वैन᳚म् । \newline
5. ए॒न॒म् मि॒मी॒ते॒ मि॒मी॒त॒ ए॒न॒ मे॒न॒म् मि॒मी॒ते॒ । \newline
6. मि॒मी॒ते॒ तस्मा॒त् तस्मा᳚न् मिमीते मिमीते॒ तस्मा᳚त् । \newline
7. तस्मा॒न् नाना॑वीर्या॒ नाना॑वीर्या॒ स्तस्मा॒त् तस्मा॒न् नाना॑वीर्याः । \newline
8. नाना॑वीर्या अ॒ङ्गुल॑यो॒ ऽङ्गुल॑यो॒ नाना॑वीर्या॒ नाना॑वीर्या अ॒ङ्गुल॑यः । \newline
9. नाना॑वीर्या॒ इति॒ नाना᳚ - वी॒र्याः॒ । \newline
10. अ॒ङ्गुल॑यः॒ सर्वा॑सु॒ सर्वा᳚ स्व॒ङ्गुल॑यो॒ ऽङ्गुल॑यः॒ सर्वा॑सु । \newline
11. सर्वा᳚ स्वङ्गु॒ष्ठ म॑ङ्गु॒ष्ठꣳ सर्वा॑सु॒ सर्वा᳚ स्वङ्गु॒ष्ठम् । \newline
12. अ॒ङ्गु॒ष्ठ मुपोपा᳚ङ्गु॒ष्ठ म॑ङ्गु॒ष्ठ मुप॑ । \newline
13. उप॒ नि न्युपोप॒ नि । \newline
14. नि गृ॑ह्णाति गृह्णाति॒ नि नि गृ॑ह्णाति । \newline
15. गृ॒ह्णा॒ति॒ तस्मा॒त् तस्मा᳚द् गृह्णाति गृह्णाति॒ तस्मा᳚त् । \newline
16. तस्मा᳚थ् स॒माव॑द्वीर्यः स॒माव॑द्वीर्य॒ स्तस्मा॒त् तस्मा᳚थ् स॒माव॑द्वीर्यः । \newline
17. स॒माव॑द्वीर्यो॒ ऽन्याभि॑ र॒न्याभिः॑ स॒माव॑द्वीर्यः स॒माव॑द्वीर्यो॒ ऽन्याभिः॑ । \newline
18. स॒माव॑द्वीर्य॒ इति॑ स॒माव॑त् - वी॒र्यः॒ । \newline
19. अ॒न्याभि॑ र॒ङ्गुलि॑भि र॒ङ्गुलि॑भि र॒न्याभि॑ र॒न्याभि॑ र॒ङ्गुलि॑भिः । \newline
20. अ॒ङ्गुलि॑भि॒ स्तस्मा॒त् तस्मा॑ द॒ङ्गुलि॑भि र॒ङ्गुलि॑भि॒ स्तस्मा᳚त् । \newline
21. अ॒ङ्गुलि॑भि॒रित्य॒ङ्गुलि॑ - भिः॒ । \newline
22. तस्मा॒थ् सर्वाः॒ सर्वा॒ स्तस्मा॒त् तस्मा॒थ् सर्वाः᳚ । \newline
23. सर्वा॒ अन्वनु॒ सर्वाः॒ सर्वा॒ अनु॑ । \newline
24. अनु॒ सꣳ स मन्वनु॒ सम् । \newline
25. सम् च॑रति चरति॒ सꣳ सम् च॑रति । \newline
26. च॒र॒ति॒ यद् यच् च॑रति चरति॒ यत् । \newline
27. यथ् स॒ह स॒ह यद् यथ् स॒ह । \newline
28. स॒ह सर्वा॑भिः॒ सर्वा॑भिः स॒ह स॒ह सर्वा॑भिः । \newline
29. सर्वा॑भि॒र् मिमी॑त॒ मिमी॑त॒ सर्वा॑भिः॒ सर्वा॑भि॒र् मिमी॑त । \newline
30. मिमी॑त॒ सꣳश्लि॑ष्टाः॒ सꣳश्लि॑ष्टा॒ मिमी॑त॒ मिमी॑त॒ सꣳश्लि॑ष्टाः । \newline
31. सꣳश्लि॑ष्टा अ॒ङ्गुल॑यो॒ ऽङ्गुल॑यः॒ सꣳश्लि॑ष्टाः॒ सꣳश्लि॑ष्टा अ॒ङ्गुल॑यः । \newline
32. सꣳश्लि॑ष्टा॒ इति॒ सं - श्लि॒ष्टाः॒ । \newline
33. अ॒ङ्गुल॑यो जायेरन् जायेरन् न॒ङ्गुल॑यो॒ ऽङ्गुल॑यो जायेरन्न् । \newline
34. जा॒ये॒र॒न् नेक॑यैक॒ यैक॑यैकया जायेरन् जायेर॒न् नेक॑यैकया । \newline
35. एक॑यैक यो॒थ्सर्ग॑ मु॒थ्सर्ग॒ मेक॑यैक॒ यैक॑यैक यो॒थ्सर्ग᳚म् । \newline
36. एक॑यैक॒येत्येक॑या - ए॒क॒या॒ । \newline
37. उ॒थ्सर्ग॑म् मिमीते मिमीत उ॒थ्सर्ग॑ मु॒थ्सर्ग॑म् मिमीते । \newline
38. उ॒थ्सर्ग॒मित्यु॑त् - सर्ग᳚म् । \newline
39. मि॒मी॒ते॒ तस्मा॒त् तस्मा᳚न् मिमीते मिमीते॒ तस्मा᳚त् । \newline
40. तस्मा॒द् विभ॑क्ता॒ विभ॑क्ता॒ स्तस्मा॒त् तस्मा॒द् विभ॑क्ताः । \newline
41. विभ॑क्ता जायन्ते जायन्ते॒ विभ॑क्ता॒ विभ॑क्ता जायन्ते । \newline
42. विभ॑क्ता॒ इति॒ वि - भ॒क्ताः॒ । \newline
43. जा॒य॒न्ते॒ पञ्च॒ पञ्च॑ जायन्ते जायन्ते॒ पञ्च॑ । \newline
44. पञ्च॒ कृत्वः॒ कृत्वः॒ पञ्च॒ पञ्च॒ कृत्वः॑ । \newline
45. कृत्वो॒ यजु॑षा॒ यजु॑षा॒ कृत्वः॒ कृत्वो॒ यजु॑षा । \newline
46. यजु॑षा मिमीते मिमीते॒ यजु॑षा॒ यजु॑षा मिमीते । \newline
47. मि॒मी॒ते॒ पञ्चा᳚क्षरा॒ पञ्चा᳚क्षरा मिमीते मिमीते॒ पञ्चा᳚क्षरा । \newline
48. पञ्चा᳚क्षरा प॒ङ्क्तिः प॒ङ्क्तिः पञ्चा᳚क्षरा॒ पञ्चा᳚क्षरा प॒ङ्क्तिः । \newline
49. पञ्चा᳚क्ष॒रेति॒ पञ्च॑ - अ॒क्ष॒रा॒ । \newline
50. प॒ङ्क्तिः पाङ्क्तः॒ पाङ्क्तः॑ प॒ङ्क्तिः प॒ङ्क्तिः पाङ्क्तः॑ । \newline
51. पाङ्क्तो॑ य॒ज्ञो य॒ज्ञ्ः पाङ्क्तः॒ पाङ्क्तो॑ य॒ज्ञ्ः । \newline
52. य॒ज्ञो य॒ज्ञ्ं ॅय॒ज्ञ्ं ॅय॒ज्ञो य॒ज्ञो य॒ज्ञ्म् । \newline
53. य॒ज्ञ् मे॒वैव य॒ज्ञ्ं ॅय॒ज्ञ् मे॒व । \newline
54. ए॒वावा वै॒वै वाव॑ । \newline
55. अव॑ रुन्धे रु॒न्धे ऽवाव॑ रुन्धे । \newline
56. रु॒न्धे॒ पञ्च॒ पञ्च॑ रुन्धे रुन्धे॒ पञ्च॑ । \newline
57. पञ्च॒ कृत्वः॒ कृत्वः॒ पञ्च॒ पञ्च॒ कृत्वः॑ । \newline
58. कृत्व॑ स्तू॒ष्णीम् तू॒ष्णीम् कृत्वः॒ कृत्व॑ स्तू॒ष्णीम् । \newline
59. तू॒ष्णीम् दश॒ दश॑ तू॒ष्णीम् तू॒ष्णीम् दश॑ । \newline

\textbf{Ghana Paata } \newline

1. मि॒मी॒ते ऽया॑तयाम्नियायातयाम्नि॒या ऽया॑तयाम्नियायातयाम्निया मिमीते मिमी॒ते ऽया॑तयाम्नियायातयाम्नि यै॒वैवा या॑तयाम्नियायातयाम्निया मिमीते मिमी॒ते ऽया॑तयाम्नियायातयाम्नि यै॒व । \newline
2. अया॑तयाम्नियायातयाम्नि यै॒वै वाया॑तयाम्नियायातयाम्नि॒या ऽया॑तयाम्नियायातयाम्नि यै॒वैन॑ मेन मे॒वाया॑तयाम्नियायातयाम्नि॒या ऽया॑तयाम्नियायातयाम्नि यै॒वैन᳚म् । \newline
3. अया॑तयाम्नियायातयाम्नि॒येत्यया॑तयाम्निया - अ॒या॒त॒या॒म्नि॒या॒ । \newline
4. ए॒वैन॑ मेन मे॒वै वैन॑म् मिमीते मिमीत एन मे॒वै वैन॑म् मिमीते । \newline
5. ए॒न॒म् मि॒मी॒ते॒ मि॒मी॒त॒ ए॒न॒ मे॒न॒म् मि॒मी॒ते॒ तस्मा॒त् तस्मा᳚न् मिमीत एन मेनम् मिमीते॒ तस्मा᳚त् । \newline
6. मि॒मी॒ते॒ तस्मा॒त् तस्मा᳚न् मिमीते मिमीते॒ तस्मा॒न् नाना॑वीर्या॒ नाना॑वीर्या॒ स्तस्मा᳚न् मिमीते मिमीते॒ तस्मा॒न् नाना॑वीर्याः । \newline
7. तस्मा॒न् नाना॑वीर्या॒ नाना॑वीर्या॒ स्तस्मा॒त् तस्मा॒न् नाना॑वीर्या अ॒ङ्गुल॑यो॒ ऽङ्गुल॑यो॒ नाना॑वीर्या॒ स्तस्मा॒त् तस्मा॒न् नाना॑वीर्या अ॒ङ्गुल॑यः । \newline
8. नाना॑वीर्या अ॒ङ्गुल॑यो॒ ऽङ्गुल॑यो॒ नाना॑वीर्या॒ नाना॑वीर्या अ॒ङ्गुल॑यः॒ सर्वा॑सु॒ सर्वा᳚ स्व॒ङ्गुल॑यो॒ नाना॑वीर्या॒ नाना॑वीर्या अ॒ङ्गुल॑यः॒ सर्वा॑सु । \newline
9. नाना॑वीर्या॒ इति॒ नाना᳚ - वी॒र्याः॒ । \newline
10. अ॒ङ्गुल॑यः॒ सर्वा॑सु॒ सर्वा᳚ स्व॒ङ्गुल॑यो॒ ऽङ्गुल॑यः॒ सर्वा᳚ स्वङ्गु॒ष्ठ म॑ङ्गु॒ष्ठꣳ सर्वा᳚स्व॒ङ्गुल॑यो॒ ऽङ्गुल॑यः॒ सर्वा᳚ स्वङ्गु॒ष्ठम् । \newline
11. सर्वा᳚ स्वङ्गु॒ष्ठ म॑ङ्गु॒ष्ठꣳ सर्वा॑सु॒ सर्वा᳚ स्वङ्गु॒ष्ठ मुपोपा᳚ङ्गु॒ष्ठꣳ सर्वा॑सु॒ सर्वा᳚ स्वङ्गु॒ष्ठ मुप॑ । \newline
12. अ॒ङ्गु॒ष्ठ मुपोपा᳚ ङ्गु॒ष्ठ म॑ङ्गु॒ष्ठ मुप॒ नि न्युपा᳚ ङ्गु॒ष्ठ म॑ङ्गु॒ष्ठ मुप॒ नि । \newline
13. उप॒ नि न्युपोप॒ नि गृ॑ह्णाति गृह्णाति॒ न्युपोप॒ नि गृ॑ह्णाति । \newline
14. नि गृ॑ह्णाति गृह्णाति॒ नि नि गृ॑ह्णाति॒ तस्मा॒त् तस्मा᳚द् गृह्णाति॒ नि नि गृ॑ह्णाति॒ तस्मा᳚त् । \newline
15. गृ॒ह्णा॒ति॒ तस्मा॒त् तस्मा᳚द् गृह्णाति गृह्णाति॒ तस्मा᳚थ् स॒माव॑द्वीर्यः स॒माव॑द्वीर्य॒ स्तस्मा᳚द् गृह्णाति गृह्णाति॒ तस्मा᳚थ् स॒माव॑द्वीर्यः । \newline
16. तस्मा᳚थ् स॒माव॑द्वीर्यः स॒माव॑द्वीर्य॒ स्तस्मा॒त् तस्मा᳚थ् स॒माव॑द्वीर्यो॒ ऽन्याभि॑ र॒न्याभिः॑ स॒माव॑द्वीर्य॒ स्तस्मा॒त् तस्मा᳚थ् स॒माव॑द्वीर्यो॒ ऽन्याभिः॑ । \newline
17. स॒माव॑द्वीर्यो॒ ऽन्याभि॑ र॒न्याभिः॑ स॒माव॑द्वीर्यः स॒माव॑द्वीर्यो॒ ऽन्याभि॑ र॒ङ्गुलि॑भि र॒ङ्गुलि॑भि र॒न्याभिः॑ स॒माव॑द्वीर्यः स॒माव॑द्वीर्यो॒ ऽन्याभि॑ र॒ङ्गुलि॑भिः । \newline
18. स॒माव॑द्वीर्य॒ इति॑ स॒माव॑त् - वी॒र्यः॒ । \newline
19. अ॒न्याभि॑ र॒ङ्गुलि॑भि र॒ङ्गुलि॑भि र॒न्याभि॑ र॒न्याभि॑ र॒ङ्गुलि॑भि॒ स्तस्मा॒त् तस्मा॑ द॒ङ्गुलि॑भि र॒न्याभि॑ र॒न्याभि॑ र॒ङ्गुलि॑भि॒ स्तस्मा᳚त् । \newline
20. अ॒ङ्गुलि॑भि॒ स्तस्मा॒त् तस्मा॑ द॒ङ्गुलि॑भि र॒ङ्गुलि॑भि॒ स्तस्मा॒थ् सर्वाः॒ सर्वा॒ स्तस्मा॑ द॒ङ्गुलि॑भि र॒ङ्गुलि॑भि॒ स्तस्मा॒थ् सर्वाः᳚ । \newline
21. अ॒ङ्गुलि॑भि॒रित्य॒ङ्गुलि॑ - भिः॒ । \newline
22. तस्मा॒थ् सर्वाः॒ सर्वा॒ स्तस्मा॒त् तस्मा॒थ् सर्वा॒ अन्वनु॒ सर्वा॒ स्तस्मा॒त् तस्मा॒थ् सर्वा॒ अनु॑ । \newline
23. सर्वा॒ अन्वनु॒ सर्वाः॒ सर्वा॒ अनु॒ सꣳ स मनु॒ सर्वाः॒ सर्वा॒ अनु॒ सम् । \newline
24. अनु॒ सꣳ स मन्वनु॒ सम् च॑रति चरति॒ स मन्वनु॒ सम् च॑रति । \newline
25. सम् च॑रति चरति॒ सꣳ सम् च॑रति॒ यद् यच् च॑रति॒ सꣳ सम् च॑रति॒ यत् । \newline
26. च॒र॒ति॒ यद् यच् च॑रति चरति॒ यथ् स॒ह स॒ह यच् च॑रति चरति॒ यथ् स॒ह । \newline
27. यथ् स॒ह स॒ह यद् यथ् स॒ह सर्वा॑भिः॒ सर्वा॑भिः स॒ह यद् यथ् स॒ह सर्वा॑भिः । \newline
28. स॒ह सर्वा॑भिः॒ सर्वा॑भिः स॒ह स॒ह सर्वा॑भि॒र् मिमी॑त॒ मिमी॑त॒ सर्वा॑भिः स॒ह स॒ह सर्वा॑भि॒र् मिमी॑त । \newline
29. सर्वा॑भि॒र् मिमी॑त॒ मिमी॑त॒ सर्वा॑भिः॒ सर्वा॑भि॒र् मिमी॑त॒ सꣳश्लि॑ष्टाः॒ सꣳश्लि॑ष्टा॒ मिमी॑त॒ सर्वा॑भिः॒ सर्वा॑भि॒र् मिमी॑त॒ सꣳश्लि॑ष्टाः । \newline
30. मिमी॑त॒ सꣳश्लि॑ष्टाः॒ सꣳश्लि॑ष्टा॒ मिमी॑त॒ मिमी॑त॒ सꣳश्लि॑ष्टा अ॒ङ्गुल॑यो॒ ऽङ्गुल॑यः॒ सꣳश्लि॑ष्टा॒ मिमी॑त॒ मिमी॑त॒ सꣳश्लि॑ष्टा अ॒ङ्गुल॑यः । \newline
31. सꣳश्लि॑ष्टा अ॒ङ्गुल॑यो॒ ऽङ्गुल॑यः॒ सꣳश्लि॑ष्टाः॒ सꣳश्लि॑ष्टा अ॒ङ्गुल॑यो जायेरन् जायेरन् न॒ङ्गुल॑यः॒ सꣳश्लि॑ष्टाः॒ सꣳश्लि॑ष्टा अ॒ङ्गुल॑यो जायेरन्न् । \newline
32. सꣳश्लि॑ष्टा॒ इति॒ सं - श्लि॒ष्टाः॒ । \newline
33. अ॒ङ्गुल॑यो जायेरन् जायेरन् न॒ङ्गुल॑यो॒ ऽङ्गुल॑यो जायेर॒न् नेक॑यैक॒ यैक॑यैकया जायेरन् न॒ङ्गुल॑यो॒ ऽङ्गुल॑यो जायेर॒न् नेक॑यैकया । \newline
34. जा॒ये॒र॒न् नेक॑यैक॒ यैक॑यैकया जायेरन् जायेर॒न् नेक॑यैक यो॒थ्सर्ग॑ मु॒थ्सर्ग॒ मेक॑यैकया जायेरन् जायेर॒न् नेक॑यैक यो॒थ्सर्ग᳚म् । \newline
35. एक॑यैक यो॒थ्सर्ग॑ मु॒थ्सर्ग॒ मेक॑यैक॒ यैक॑यैक यो॒थ्सर्ग॑म् मिमीते मिमीत उ॒थ्सर्ग॒ मेक॑यैक॒
यैक॑यैक यो॒थ्सर्ग॑म् मिमीते । \newline
36. एक॑यैक॒येत्येक॑या - ए॒क॒या॒ । \newline
37. उ॒थ्सर्ग॑म् मिमीते मिमीत उ॒थ्सर्ग॑ मु॒थ्सर्ग॑म् मिमीते॒ तस्मा॒त् तस्मा᳚न् मिमीत उ॒थ्सर्ग॑ मु॒थ्सर्ग॑म् मिमीते॒ तस्मा᳚त् । \newline
38. उ॒थ्सर्ग॒मित्यु॑त् - सर्ग᳚म् । \newline
39. मि॒मी॒ते॒ तस्मा॒त् तस्मा᳚न् मिमीते मिमीते॒ तस्मा॒द् विभ॑क्ता॒ विभ॑क्ता॒ स्तस्मा᳚न् मिमीते मिमीते॒ तस्मा॒द् विभ॑क्ताः । \newline
40. तस्मा॒द् विभ॑क्ता॒ विभ॑क्ता॒ स्तस्मा॒त् तस्मा॒द् विभ॑क्ता जायन्ते जायन्ते॒ विभ॑क्ता॒ स्तस्मा॒त् तस्मा॒द् विभ॑क्ता जायन्ते । \newline
41. विभ॑क्ता जायन्ते जायन्ते॒ विभ॑क्ता॒ विभ॑क्ता जायन्ते॒ पञ्च॒ पञ्च॑ जायन्ते॒ विभ॑क्ता॒ विभ॑क्ता जायन्ते॒ पञ्च॑ । \newline
42. विभ॑क्ता॒ इति॒ वि - भ॒क्ताः॒ । \newline
43. जा॒य॒न्ते॒ पञ्च॒ पञ्च॑ जायन्ते जायन्ते॒ पञ्च॒ कृत्वः॒ कृत्वः॒ पञ्च॑ जायन्ते जायन्ते॒ पञ्च॒ कृत्वः॑ । \newline
44. पञ्च॒ कृत्वः॒ कृत्वः॒ पञ्च॒ पञ्च॒ कृत्वो॒ यजु॑षा॒ यजु॑षा॒ कृत्वः॒ पञ्च॒ पञ्च॒ कृत्वो॒ यजु॑षा । \newline
45. कृत्वो॒ यजु॑षा॒ यजु॑षा॒ कृत्वः॒ कृत्वो॒ यजु॑षा मिमीते मिमीते॒ यजु॑षा॒ कृत्वः॒ कृत्वो॒ यजु॑षा मिमीते । \newline
46. यजु॑षा मिमीते मिमीते॒ यजु॑षा॒ यजु॑षा मिमीते॒ पञ्चा᳚क्षरा॒ पञ्चा᳚क्षरा मिमीते॒ यजु॑षा॒ यजु॑षा मिमीते॒ पञ्चा᳚क्षरा । \newline
47. मि॒मी॒ते॒ पञ्चा᳚क्षरा॒ पञ्चा᳚क्षरा मिमीते मिमीते॒ पञ्चा᳚क्षरा प॒ङ्क्तिः प॒ङ्क्तिः पञ्चा᳚क्षरा मिमीते मिमीते॒ पञ्चा᳚क्षरा प॒ङ्क्तिः । \newline
48. पञ्चा᳚क्षरा प॒ङ्क्तिः प॒ङ्क्तिः पञ्चा᳚क्षरा॒ पञ्चा᳚क्षरा प॒ङ्क्तिः पाङ्क्तः॒ पाङ्क्तः॑ प॒ङ्क्तिः पञ्चा᳚क्षरा॒ पञ्चा᳚क्षरा प॒ङ्क्तिः पाङ्क्तः॑ । \newline
49. पञ्चा᳚क्ष॒रेति॒ पञ्च॑ - अ॒क्ष॒रा॒ । \newline
50. प॒ङ्क्तिः पाङ्क्तः॒ पाङ्क्तः॑ प॒ङ्क्तिः प॒ङ्क्तिः पाङ्क्तो॑ य॒ज्ञो य॒ज्ञ्ः पाङ्क्तः॑ प॒ङ्क्तिः प॒ङ्क्तिः पाङ्क्तो॑ य॒ज्ञ्ः । \newline
51. पाङ्क्तो॑ य॒ज्ञो य॒ज्ञ्ः पाङ्क्तः॒ पाङ्क्तो॑ य॒ज्ञो य॒ज्ञ्ं ॅय॒ज्ञ्ं ॅय॒ज्ञ्ः पाङ्क्तः॒ पाङ्क्तो॑ य॒ज्ञो य॒ज्ञ्म् । \newline
52. य॒ज्ञो य॒ज्ञ्ं ॅय॒ज्ञ्ं ॅय॒ज्ञो य॒ज्ञो य॒ज्ञ् मे॒वैव य॒ज्ञ्ं ॅय॒ज्ञो य॒ज्ञो य॒ज्ञ् मे॒व । \newline
53. य॒ज्ञ् मे॒वैव य॒ज्ञ्ं ॅय॒ज्ञ् मे॒वावा वै॒व य॒ज्ञ्ं ॅय॒ज्ञ् मे॒वाव॑ । \newline
54. ए॒वावा वै॒वै वाव॑ रुन्धे रु॒न्धे ऽवै॒वै वाव॑ रुन्धे । \newline
55. अव॑ रुन्धे रु॒न्धे ऽवाव॑ रुन्धे॒ पञ्च॒ पञ्च॑ रु॒न्धे ऽवाव॑ रुन्धे॒ पञ्च॑ । \newline
56. रु॒न्धे॒ पञ्च॒ पञ्च॑ रुन्धे रुन्धे॒ पञ्च॒ कृत्वः॒ कृत्वः॒ पञ्च॑ रुन्धे रुन्धे॒ पञ्च॒ कृत्वः॑ । \newline
57. पञ्च॒ कृत्वः॒ कृत्वः॒ पञ्च॒ पञ्च॒ कृत्व॑ स्तू॒ष्णीम् तू॒ष्णीम् कृत्वः॒ पञ्च॒ पञ्च॒ कृत्व॑ स्तू॒ष्णीम् । \newline
58. कृत्व॑ स्तू॒ष्णीम् तू॒ष्णीम् कृत्वः॒ कृत्व॑ स्तू॒ष्णीम् दश॒ दश॑ तू॒ष्णीम् कृत्वः॒ कृत्व॑ स्तू॒ष्णीम् दश॑ । \newline
59. तू॒ष्णीम् दश॒ दश॑ तू॒ष्णीम् तू॒ष्णीम् दश॒ सꣳ सम् दश॑ तू॒ष्णीम् तू॒ष्णीम् दश॒ सम् । \newline
\pagebreak
\markright{ TS 6.1.9.6  \hfill https://www.vedavms.in \hfill}

\section{ TS 6.1.9.6 }

\textbf{TS 6.1.9.6 } \newline
\textbf{Samhita Paata} \newline

-दश॒ सं प॑द्यन्ते॒ दशा᳚क्षरा वि॒राडन्नं॑ ॅवि॒राड् वि॒राजै॒वान्नाद्य॒मव॑ रुन्धे॒ यद् यजु॑षा॒ मिमी॑ते भू॒तमे॒वाव॑ रुन्धे॒ यत् तू॒ष्णीं भ॑वि॒ष्यद्-यद्-वै तावा॑ने॒व सोमः॒ स्याद् याव॑न्तं॒ मिमी॑ते॒ यज॑मानस्यै॒व स्या॒न्नापि॑ सद॒स्या॑नां प्र॒जाभ्य॒स्त्वेत्युप॒ समू॑हति सद॒स्या॑ने॒वान्वा भ॑जति॒ वास॒सोप॑ नह्यति सर्वदे॒वत्यं॑ ॅवै - [  ] \newline

\textbf{Pada Paata} \newline

दश॑ । समिति॑ । प॒द्य॒न्ते॒ । दशा᳚क्ष॒रेति॒ दश॑ - अ॒क्ष॒रा॒ । वि॒राडिति॑ वि - राट् । अन्न᳚म् । वि॒राडिति॑ वि - राट् । वि॒राजेति॑ वि - राजा᳚ । ए॒व । अ॒न्नाद्य॒मित्य॑न्न -अद्य᳚म् । अवेति॑ । रु॒न्धे॒ । यत् । यजु॑षा । मिमी॑ते । भू॒तम् । ए॒व । अवेति॑ । रु॒न्धे॒ । यत् । तू॒ष्णीम् । भ॒वि॒ष्यत् । यत् । वै । तावान्॑ । ए॒व । सोमः॑ । स्यात् । याव॑न्तम् । मिमी॑ते । यज॑मानस्य । ए॒व । स्या॒त् । न । अपीति॑ । स॒द॒स्या॑नाम् । प्र॒जाभ्य॒ इति॑ प्र - जाभ्यः॑ । त्वा॒ । इति॑ । उप॑ । समिति॑ । ऊ॒ह॒ति॒ । स॒द॒स्यान्॑ । ए॒व । अ॒न्वाभ॑ज॒तीत्य॑नु-आभ॑जति । वास॑सा । उपेति॑ । न॒ह्य॒ति॒ । स॒र्व॒दे॒व॒त्य॑मिति॑ सर्व - दे॒व॒त्य᳚म् । वै ।  \newline


\textbf{Krama Paata} \newline

दश॒ सम् । सम् प॑द्यन्ते । प॒द्य॒न्ते॒ दशा᳚क्षरा । दशा᳚क्षरा वि॒राट् । दशा᳚क्ष॒रेति॒ दश॑ - अ॒क्ष॒रा॒ । वि॒राडन्न᳚म् । वि॒राडिति॑ वि - राट् । अन्न॑म् ॅवि॒राट् । वि॒राड् वि॒राजा᳚ । वि॒राडिति॑ वि - राट् । वि॒राजै॒व । वि॒राजेति॑ वि - राजा᳚ । ए॒वान्नाद्य᳚म् । अ॒न्नाद्य॒मव॑ । अ॒न्नाद्य॒मित्य॑न्न - अद्य᳚म् । अव॑ रुन्धे । रु॒न्धे॒ यत् । यद् यजु॑षा । यजु॑षा॒ मिमी॑ते । मिमी॑ते भू॒तम् । भू॒तमे॒व । ए॒वाव॑ । अव॑ रुन्धे । रु॒न्धे॒ यत् । यत् तू॒ष्णीम् । तू॒ष्णीम् भ॑वि॒ष्यत् । भ॒वि॒ष्यद् यत् । यद् वै । वै तावान्॑ । तावा॑ने॒व । ए॒व सोमः॑ । सोमः॒ स्यात् । स्याद् याव॑न्तम् । याव॑न्त॒म् मिमी॑ते । मिमी॑ते॒ यज॑मानस्य । यज॑मानस्यै॒व । ए॒व स्या᳚त् । स्या॒न् न । नापि॑ । अपि॑ सद॒स्या॑नाम् । स॒द॒स्या॑नाम् प्र॒जाभ्यः॑ । प्र॒जाभ्य॑स्त्वा । प्र॒जाभ्य॒ इति॑ प्र - जाभ्यः॑ । त्वेति॑ । इत्युप॑ । उप॒ सम् । समू॑हति । ऊ॒ह॒ति॒ स॒द॒स्यान्॑ । स॒द॒स्या॑ने॒व । ए॒वान्वाभ॑जति । अ॒न्वाभ॑जति॒ वास॑सा । अ॒न्वाभ॑ज॒तीत्य॑नु - आभ॑जति । वास॒सोप॑ । उप॑ नह्यति । न॒ह्य॒ति॒ स॒र्व॒दे॒व॒त्य᳚म् । स॒र्व॒दे॒व॒त्य॑म् ॅवै ( ) । स॒र्व॒दे॒व॒त्य॑मिति॑ सर्व - दे॒व॒त्य᳚म् । वै वासः॑ \newline

\textbf{Jatai Paata} \newline

1. दश॒ सꣳ सम् दश॒ दश॒ सम् । \newline
2. सम् प॑द्यन्ते पद्यन्ते॒ सꣳ सम् प॑द्यन्ते । \newline
3. प॒द्य॒न्ते॒ दशा᳚क्षरा॒ दशा᳚क्षरा पद्यन्ते पद्यन्ते॒ दशा᳚क्षरा । \newline
4. दशा᳚क्षरा वि॒राड् वि॒राड् दशा᳚क्षरा॒ दशा᳚क्षरा वि॒राट् । \newline
5. दशा᳚क्ष॒रेति॒ दश॑ - अ॒क्ष॒रा॒ । \newline
6. वि॒रा डन्न॒ मन्नं॑ ॅवि॒राड् वि॒रा डन्न᳚म् । \newline
7. वि॒राडिति॑ वि - राट् । \newline
8. अन्नं॑ ॅवि॒राड् वि॒रा डन्न॒ मन्नं॑ ॅवि॒राट् । \newline
9. वि॒राड् वि॒राजा॑ वि॒राजा॑ वि॒राड् वि॒राड् वि॒राजा᳚ । \newline
10. वि॒राडिति॑ वि - राट् । \newline
11. वि॒राजै॒वैव वि॒राजा॑ वि॒राजै॒व । \newline
12. वि॒राजेति॑ वि - राजा᳚ । \newline
13. ए॒वान्नाद्य॑ म॒न्नाद्य॑ मे॒वै वान्नाद्य᳚म् । \newline
14. अ॒न्नाद्य॒ मवावा॒ न्नाद्य॑ म॒न्नाद्य॒ मव॑ । \newline
15. अ॒न्नाद्य॒मित्य॑न्न - अद्य᳚म् । \newline
16. अव॑ रुन्धे रु॒न्धे ऽवाव॑ रुन्धे । \newline
17. रु॒न्धे॒ यद् यद् रु॑न्धे रुन्धे॒ यत् । \newline
18. यद् यजु॑षा॒ यजु॑षा॒ यद् यद् यजु॑षा । \newline
19. यजु॑षा॒ मिमी॑ते॒ मिमी॑ते॒ यजु॑षा॒ यजु॑षा॒ मिमी॑ते । \newline
20. मिमी॑ते भू॒तम् भू॒तम् मिमी॑ते॒ मिमी॑ते भू॒तम् । \newline
21. भू॒त मे॒वैव भू॒तम् भू॒त मे॒व । \newline
22. ए॒वावा वै॒वै वाव॑ । \newline
23. अव॑ रुन्धे रु॒न्धे ऽवाव॑ रुन्धे । \newline
24. रु॒न्धे॒ यद् यद् रु॑न्धे रुन्धे॒ यत् । \newline
25. यत् तू॒ष्णीम् तू॒ष्णीं ॅयद् यत् तू॒ष्णीम् । \newline
26. तू॒ष्णीम् भ॑वि॒ष्यद् भ॑वि॒ष्यत् तू॒ष्णीम् तू॒ष्णीम् भ॑वि॒ष्यत् । \newline
27. भ॒वि॒ष्यद् यद् यद् भ॑वि॒ष्यद् भ॑वि॒ष्यद् यत् । \newline
28. यद् वै वै यद् यद् वै । \newline
29. वै तावा॒न् तावा॒न्॒. वै वै तावान्॑ । \newline
30. तावा॑ ने॒वैव तावा॒न् तावा॑ ने॒व । \newline
31. ए॒व सोमः॒ सोम॑ ए॒वैव सोमः॑ । \newline
32. सोमः॒ स्याथ् स्याथ् सोमः॒ सोमः॒ स्यात् । \newline
33. स्याद् याव॑न्तं॒ ॅयाव॑न्तꣳ॒॒ स्याथ् स्याद् याव॑न्तम् । \newline
34. याव॑न्त॒म् मिमी॑ते॒ मिमी॑ते॒ याव॑न्तं॒ ॅयाव॑न्त॒म् मिमी॑ते । \newline
35. मिमी॑ते॒ यज॑मानस्य॒ यज॑मानस्य॒ मिमी॑ते॒ मिमी॑ते॒ यज॑मानस्य । \newline
36. यज॑मान स्यै॒वैव यज॑मानस्य॒ यज॑मान स्यै॒व । \newline
37. ए॒व स्या᳚थ् स्या दे॒वैव स्या᳚त् । \newline
38. स्या॒न् न न स्या᳚थ् स्या॒न् न । \newline
39. नाप्यपि॒ न नापि॑ । \newline
40. अपि॑ सद॒स्या॑नाꣳ सद॒स्या॑ना॒ मप्यपि॑ सद॒स्या॑नाम् । \newline
41. स॒द॒स्या॑नाम् प्र॒जाभ्यः॑ प्र॒जाभ्यः॑ सद॒स्या॑नाꣳ सद॒स्या॑नाम् प्र॒जाभ्यः॑ । \newline
42. प्र॒जाभ्य॑ स्त्वा त्वा प्र॒जाभ्यः॑ प्र॒जाभ्य॑ स्त्वा । \newline
43. प्र॒जाभ्य॒ इति॑ प्र - जाभ्यः॑ । \newline
44. त्वेतीति॑ त्वा॒ त्वेति॑ । \newline
45. इत्युपोपे तीत्युप॑ । \newline
46. उप॒ सꣳ स मुपोप॒ सम् । \newline
47. स मू॑ह त्यूहति॒ सꣳ स मू॑हति । \newline
48. ऊ॒ह॒ति॒ स॒द॒स्या᳚न् थ्सद॒स्या॑ नूह त्यूहति सद॒स्यान्॑ । \newline
49. स॒द॒स्या॑ ने॒वैव स॑द॒स्या᳚न् थ्सद॒स्या॑ ने॒व । \newline
50. ए॒वा न्वाभ॑ज त्य॒न्वाभ॑ज त्ये॒वैवा न्वाभ॑जति । \newline
51. अ॒न्वाभ॑जति॒ वास॑सा॒ वास॑सा॒ ऽन्वाभ॑ज त्य॒न्वाभ॑जति॒ वास॑सा । \newline
52. अ॒न्वाभ॑ज॒तीत्य॑नु - आभ॑जति । \newline
53. वास॒सो पोप॒ वास॑सा॒ वास॒ सोप॑ । \newline
54. उप॑ नह्यति नह्य॒ त्युपोप॑ नह्यति । \newline
55. न॒ह्य॒ति॒ स॒र्व॒दे॒व॒त्यꣳ॑ सर्वदेव॒त्य॑म् नह्यति नह्यति सर्वदेव॒त्य᳚म् । \newline
56. स॒र्व॒दे॒व॒त्यं॑ ॅवै वै स॑र्वदेव॒त्यꣳ॑ सर्वदेव॒त्यं॑ ॅवै । \newline
57. स॒र्व॒दे॒व॒त्य॑मिति॑ सर्व - दे॒व॒त्य᳚म् । \newline
58. वै वासो॒ वासो॒ वै वै वासः॑ । \newline

\textbf{Ghana Paata } \newline

1. दश॒ सꣳ सम् दश॒ दश॒ सम् प॑द्यन्ते पद्यन्ते॒ सम् दश॒ दश॒ सम् प॑द्यन्ते । \newline
2. सम् प॑द्यन्ते पद्यन्ते॒ सꣳ सम् प॑द्यन्ते॒ दशा᳚क्षरा॒ दशा᳚क्षरा पद्यन्ते॒ सꣳ सम् प॑द्यन्ते॒ दशा᳚क्षरा । \newline
3. प॒द्य॒न्ते॒ दशा᳚क्षरा॒ दशा᳚क्षरा पद्यन्ते पद्यन्ते॒ दशा᳚क्षरा वि॒राड् वि॒राड् दशा᳚क्षरा पद्यन्ते पद्यन्ते॒ दशा᳚क्षरा वि॒राट् । \newline
4. दशा᳚क्षरा वि॒राड् वि॒राड् दशा᳚क्षरा॒ दशा᳚क्षरा वि॒राडन्न॒ मन्नं॑ ॅवि॒राड् दशा᳚क्षरा॒ दशा᳚क्षरा वि॒राडन्न᳚म् । \newline
5. दशा᳚क्ष॒रेति॒ दश॑ - अ॒क्ष॒रा॒ । \newline
6. वि॒राडन्न॒ मन्नं॑ ॅवि॒राड् वि॒राडन्नं॑ ॅवि॒राड् वि॒राडन्नं॑ ॅवि॒राड् वि॒राडन्नं॑ ॅवि॒राट् । \newline
7. वि॒राडिति॑ वि - राट् । \newline
8. अन्नं॑ ॅवि॒राड् वि॒राडन्न॒ मन्नं॑ ॅवि॒राड् वि॒राजा॑ वि॒राजा॑ वि॒राडन्न॒ मन्नं॑ ॅवि॒राड् वि॒राजा᳚ । \newline
9. वि॒राड् वि॒राजा॑ वि॒राजा॑ वि॒राड् वि॒राड् वि॒राजै॒वैव वि॒राज॑ वि॒राड् वि॒राड् वि॒राजै॒व । \newline
10. वि॒राडिति॑ वि - राट् । \newline
11. वि॒राजै॒वैव वि॒राजा॑ वि॒रा जै॒वान्नाद्य॑ म॒न्नाद्य॑ मे॒व वि॒राजा॑ वि॒रा जै॒वान्नाद्य᳚म् । \newline
12. वि॒राजेति॑ वि - राजा᳚ । \newline
13. ए॒वान्नाद्य॑ म॒न्नाद्य॑ मे॒वै वान्नाद्य॒ मवावा॒न्नाद्य॑ मे॒वै वान्नाद्य॒ मव॑ । \newline
14. अ॒न्नाद्य॒ मवा वा॒न्नाद्य॑ म॒न्नाद्य॒ मव॑ रुन्धे रु॒न्धे ऽवा॒न्नाद्य॑ म॒न्नाद्य॒ मव॑ रुन्धे । \newline
15. अ॒न्नाद्य॒मित्य॑न्न - अद्य᳚म् । \newline
16. अव॑ रुन्धे रु॒न्धे ऽवाव॑ रुन्धे॒ यद् यद् रु॒न्धे ऽवाव॑ रुन्धे॒ यत् । \newline
17. रु॒न्धे॒ यद् यद् रु॑न्धे रुन्धे॒ यद् यजु॑षा॒ यजु॑षा॒ यद् रु॑न्धे रुन्धे॒ यद् यजु॑षा । \newline
18. यद् यजु॑षा॒ यजु॑षा॒ यद् यद् यजु॑षा॒ मिमी॑ते॒ मिमी॑ते॒ यजु॑षा॒ यद् यद् यजु॑षा॒ मिमी॑ते । \newline
19. यजु॑षा॒ मिमी॑ते॒ मिमी॑ते॒ यजु॑षा॒ यजु॑षा॒ मिमी॑ते भू॒तम् भू॒तम् मिमी॑ते॒ यजु॑षा॒ यजु॑षा॒ मिमी॑ते भू॒तम् । \newline
20. मिमी॑ते भू॒तम् भू॒तम् मिमी॑ते॒ मिमी॑ते भू॒त मे॒वैव भू॒तम् मिमी॑ते॒ मिमी॑ते भू॒त मे॒व । \newline
21. भू॒त मे॒वैव भू॒तम् भू॒त मे॒वावा वै॒व भू॒तम् भू॒त मे॒वाव॑ । \newline
22. ए॒वावा वै॒वै वाव॑ रुन्धे रु॒न्धे ऽवै॒वै वाव॑ रुन्धे । \newline
23. अव॑ रुन्धे रु॒न्धे ऽवाव॑ रुन्धे॒ यद् यद् रु॒न्धे ऽवाव॑ रुन्धे॒ यत् । \newline
24. रु॒न्धे॒ यद् यद् रु॑न्धे रुन्धे॒ यत् तू॒ष्णीम् तू॒ष्णीं ॅयद् रु॑न्धे रुन्धे॒ यत् तू॒ष्णीम् । \newline
25. यत् तू॒ष्णीम् तू॒ष्णीं ॅयद् यत् तू॒ष्णीम् भ॑वि॒ष्यद् भ॑वि॒ष्यत् तू॒ष्णीं ॅयद् यत् तू॒ष्णीम् भ॑वि॒ष्यत् । \newline
26. तू॒ष्णीम् भ॑वि॒ष्यद् भ॑वि॒ष्यत् तू॒ष्णीम् तू॒ष्णीम् भ॑वि॒ष्यद् यद् यद् भ॑वि॒ष्यत् तू॒ष्णीम् तू॒ष्णीम् भ॑वि॒ष्यद् यत् । \newline
27. भ॒वि॒ष्यद् यद् यद् भ॑वि॒ष्यद् भ॑वि॒ष्यद् यद् वै वै यद् भ॑वि॒ष्यद् भ॑वि॒ष्यद् यद् वै । \newline
28. यद् वै वै यद् यद् वै तावा॒न् तावा॒न्॒. वै यद् यद् वै तावान्॑ । \newline
29. वै तावा॒न् तावा॒न्॒. वै वै तावा॑ ने॒वैव तावा॒न्॒. वै वै तावा॑ने॒व । \newline
30. तावा॑ने॒ वैव तावा॒न् तावा॑ने॒व सोमः॒ सोम॑ ए॒व तावा॒न् तावा॑ने॒व सोमः॑ । \newline
31. ए॒व सोमः॒ सोम॑ ए॒वैव सोमः॒ स्याथ् स्याथ् सोम॑ ए॒वैव सोमः॒ स्यात् । \newline
32. सोमः॒ स्याथ् स्याथ् सोमः॒ सोमः॒ स्याद् याव॑न्तं॒ ॅयाव॑न्तꣳ॒॒ स्याथ् सोमः॒ सोमः॒ स्याद् याव॑न्तम् । \newline
33. स्याद् याव॑न्तं॒ ॅयाव॑न्तꣳ॒॒ स्याथ् स्याद् याव॑न्त॒म् मिमी॑ते॒ मिमी॑ते॒ याव॑न्तꣳ॒॒ स्याथ् स्याद् याव॑न्त॒म् मिमी॑ते । \newline
34. याव॑न्त॒म् मिमी॑ते॒ मिमी॑ते॒ याव॑न्तं॒ ॅयाव॑न्त॒म् मिमी॑ते॒ यज॑मानस्य॒ यज॑मानस्य॒ मिमी॑ते॒ याव॑न्तं॒ ॅयाव॑न्त॒म् मिमी॑ते॒ यज॑मानस्य । \newline
35. मिमी॑ते॒ यज॑मानस्य॒ यज॑मानस्य॒ मिमी॑ते॒ मिमी॑ते॒ यज॑मान स्यै॒वैव यज॑मानस्य॒ मिमी॑ते॒ मिमी॑ते॒ यज॑मान स्यै॒व । \newline
36. यज॑मान स्यै॒वैव यज॑मानस्य॒ यज॑मान स्यै॒व स्या᳚थ् स्यादे॒व यज॑मानस्य॒ यज॑मान स्यै॒व स्या᳚त् । \newline
37. ए॒व स्या᳚थ् स्या दे॒वैव स्या॒न् न न स्या॑ दे॒वैव स्या॒न् न । \newline
38. स्या॒न् न न स्या᳚थ् स्या॒न् नाप्यपि॒ न स्या᳚थ् स्या॒न् नापि॑ । \newline
39. नाप्यपि॒ न नापि॑ सद॒स्या॑नाꣳ सद॒स्या॑ना॒ मपि॒ न नापि॑ सद॒स्या॑नाम् । \newline
40. अपि॑ सद॒स्या॑नाꣳ सद॒स्या॑ना॒ मप्यपि॑ सद॒स्या॑नाम् प्र॒जाभ्यः॑ प्र॒जाभ्यः॑ सद॒स्या॑ना॒ मप्यपि॑ सद॒स्या॑नाम् प्र॒जाभ्यः॑ । \newline
41. स॒द॒स्या॑नाम् प्र॒जाभ्यः॑ प्र॒जाभ्यः॑ सद॒स्या॑नाꣳ सद॒स्या॑नाम् प्र॒जाभ्य॑ स्त्वा त्वा प्र॒जाभ्यः॑ सद॒स्या॑नाꣳ सद॒स्या॑नाम् प्र॒जाभ्य॑ स्त्वा । \newline
42. प्र॒जाभ्य॑स्त्वा त्वा प्र॒जाभ्यः॑ प्र॒जाभ्य॒स्त्वे तीति॑ त्वा प्र॒जाभ्यः॑ प्र॒जाभ्य॒ स्त्वेति॑ । \newline
43. प्र॒जाभ्य॒ इति॑ प्र - जाभ्यः॑ । \newline
44. त्वेतीति॑ त्वा॒ त्वेत्युपोपेति॑ त्वा॒ त्वेत्युप॑ । \newline
45. इत्युपोपे तीत्युप॒ सꣳ स मुपे तीत्युप॒ सम् । \newline
46. उप॒ सꣳ स मुपोप॒ स मू॑ह त्यूहति॒ स मुपोप॒ स मू॑हति । \newline
47. स मू॑ह त्यूहति॒ सꣳ स मू॑हति सद॒स्या᳚न् थ्सद॒स्या॑ नूहति॒ सꣳ स मू॑हति सद॒स्यान्॑ । \newline
48. ऊ॒ह॒ति॒ स॒द॒स्या᳚न् थ्सद॒स्या॑ नूह त्यूहति सद॒स्या॑ने॒ वैव स॑द॒स्या॑नूह त्यूहति सद॒स्या॑ने॒व । \newline
49. स॒द॒स्या॑ने ॒वैव स॑द॒स्या᳚न् थ्सद॒स्या॑ने॒वा न्वाभ॑ज त्य॒न्वाभ॑ज त्ये॒व स॑द॒स्या᳚न् थ्सद॒स्या॑ने॒वा न्वाभ॑जति । \newline
50. ए॒वा न्वाभ॑ज त्य॒न्वाभ॑ज त्ये॒वैवा न्वाभ॑जति॒ वास॑सा॒ वास॑सा॒ ऽन्वाभ॑ज त्ये॒वैवा न्वाभ॑जति॒ वास॑सा । \newline
51. अ॒न्वाभ॑जति॒ वास॑सा॒ वास॑सा॒ ऽन्वाभ॑ज त्य॒न्वाभ॑जति॒ वास॒सो पोप॒ वास॑सा॒ ऽन्वाभ॑ज त्य॒न्वाभ॑जति॒ वास॒सोप॑ । \newline
52. अ॒न्वाभ॑ज॒तीत्य॑नु - आभ॑जति । \newline
53. वास॒सोपोप॒ वास॑सा॒ वास॒सोप॑ नह्यति नह्य॒ त्युप॒ वास॑सा॒ वास॒सोप॑ नह्यति । \newline
54. उप॑ नह्यति नह्य॒ त्युपोप॑ नह्यति सर्वदेव॒त्यꣳ॑ सर्वदेव॒त्य॑म् नह्य॒ त्युपोप॑ नह्यति सर्वदेव॒त्य᳚म् । \newline
55. न॒ह्य॒ति॒ स॒र्व॒दे॒व॒त्यꣳ॑ सर्वदेव॒त्य॑म् नह्यति नह्यति सर्वदेव॒त्यं॑ ॅवै वै स॑र्वदेव॒त्य॑म् नह्यति नह्यति सर्वदेव॒त्यं॑ ॅवै । \newline
56. स॒र्व॒दे॒व॒त्यं॑ ॅवै वै स॑र्वदेव॒त्यꣳ॑ सर्वदेव॒त्यं॑ ॅवै वासो॒ वासो॒ वै स॑र्वदेव॒त्यꣳ॑ सर्वदेव॒त्यं॑ ॅवै वासः॑ । \newline
57. स॒र्व॒दे॒व॒त्य॑मिति॑ सर्व - दे॒व॒त्य᳚म् । \newline
58. वै वासो॒ वासो॒ वै वै वासः॒ सर्वा॑भिः॒ सर्वा॑भि॒र् वासो॒ वै वै वासः॒ सर्वा॑भिः । \newline
\pagebreak
\markright{ TS 6.1.9.7  \hfill https://www.vedavms.in \hfill}

\section{ TS 6.1.9.7 }

\textbf{TS 6.1.9.7 } \newline
\textbf{Samhita Paata} \newline

वासः॒ सर्वा॑भिरे॒वैनं॑ दे॒वता॑भिः॒ सम॑र्द्धयति प॒शवो॒ वै सोमः॑ प्रा॒णाय॒ त्वेत्युप॑ नह्यति प्रा॒णमे॒व प॒शुषु॑ दधाति व्या॒नाय॒ त्वेत्यनु॑ शृन्थति व्या॒नमे॒व प॒शुषु॑ दधाति॒ तस्मा᳚थ् स्व॒पन्तं॑ प्रा॒णा न ज॑हति ॥ \newline

\textbf{Pada Paata} \newline

वासः॑ । सर्वा॑भिः । ए॒व । ए॒न॒म् । दे॒वता॑भिः । समिति॑ । अ॒द्‌र्ध॒य॒ति॒ । प॒शवः॑ । वै । सोमः॑ । प्रा॒णायेति॑ प्र - अ॒नाय॑ । त्वा॒ । इति॑ । उपेति॑ । न॒ह्य॒ति॒ । प्रा॒णमिति॑ प्र - अ॒नम् । ए॒व । प॒शुषु॑ । द॒धा॒ति॒ । व्या॒नायेति॑ वि - अ॒नाय॑ । त्वा॒ । इति॑ । अन्विति॑ । शृ॒न्थ॒ति॒ । व्या॒नमिति॑ वि - अ॒नम् । ए॒व । प॒शुषु॑ । द॒धा॒ति॒ । तस्मा᳚त् । स्व॒पन्त᳚म् । प्रा॒णा इति॑ प्र - अ॒नाः । न । ज॒ह॒ति॒ ॥  \newline


\textbf{Krama Paata} \newline

वासः॒ सर्वा॑भिः । सर्वा॑भिरे॒व । ए॒वैन᳚म् । ए॒न॒म् दे॒वता॑भिः । दे॒वता॑भिः॒ सम् । सम॑र्द्धयति । अ॒र्द्ध॒य॒ति॒ प॒शवः॑ । प॒शवो॒ वै । वै सोमः॑ । सोमः॑ प्रा॒णाय॑ । प्रा॒णाय॑ त्वा । प्रा॒णायेति॑ प्र - अ॒नाय॑ । त्वेति॑ । इत्युप॑ । उप॑ नह्यति । न॒ह्य॒ति॒ प्रा॒णम् । प्रा॒णमे॒व । प्रा॒णमिति॑ प्र - अ॒नम् । ए॒व प॒शुषु॑ । प॒शुषु॑ दधाति । द॒धा॒ति॒ व्या॒नाय॑ । व्या॒नाय॑ त्वा । व्या॒नायेति॑ वि - अ॒नाय॑ । त्वेति॑ । इत्यनु॑ । अनु॑ शृन्थति । शृ॒न्थ॒ति॒ व्या॒नम् । व्या॒नमे॒व । व्या॒नमिति॑ वि - अ॒नम् । ए॒व प॒शुषु॑ । प॒शुषु॑ दधाति । द॒धा॒ति॒ तस्मा᳚त् । तस्मा᳚थ् स्व॒पन्त᳚म् । स्व॒पन्त॑म् प्रा॒णाः । प्रा॒णा न । प्रा॒णा इति॑ प्र - अ॒नाः । न ज॑हति । ज॒ह॒तीति॑ जहति । \newline

\textbf{Jatai Paata} \newline

1. वासः॒ सर्वा॑भिः॒ सर्वा॑भि॒र् वासो॒ वासः॒ सर्वा॑भिः । \newline
2. सर्वा॑भिरे॒ वैव सर्वा॑भिः॒ सर्वा॑भि रे॒व । \newline
3. ए॒वैन॑ मेन मे॒वै वैन᳚म् । \newline
4. ए॒न॒म् दे॒वता॑भिर् दे॒वता॑भि रेन मेनम् दे॒वता॑भिः । \newline
5. दे॒वता॑भिः॒ सꣳ सम् दे॒वता॑भिर् दे॒वता॑भिः॒ सम् । \newline
6. स म॑र्द्धय त्यर्द्धयति॒ सꣳ स म॑र्द्धयति । \newline
7. अ॒र्द्ध॒य॒ति॒ प॒शवः॑ प॒शवो᳚ ऽर्द्धय त्यर्द्धयति प॒शवः॑ । \newline
8. प॒शवो॒ वै वै प॒शवः॑ प॒शवो॒ वै । \newline
9. वै सोमः॒ सोमो॒ वै वै सोमः॑ । \newline
10. सोमः॑ प्रा॒णाय॑ प्रा॒णाय॒ सोमः॒ सोमः॑ प्रा॒णाय॑ । \newline
11. प्रा॒णाय॑ त्वा त्वा प्रा॒णाय॑ प्रा॒णाय॑ त्वा । \newline
12. प्रा॒णायेति॑ प्र - अ॒नाय॑ । \newline
13. त्वेतीति॑ त्वा॒ त्वेति॑ । \newline
14. इत्युपोपे तीत्युप॑ । \newline
15. उप॑ नह्यति नह्य॒ त्युपोप॑ नह्यति । \newline
16. न॒ह्य॒ति॒ प्रा॒णम् प्रा॒णम् न॑ह्यति नह्यति प्रा॒णम् । \newline
17. प्रा॒ण मे॒वैव प्रा॒णम् प्रा॒ण मे॒व । \newline
18. प्रा॒णमिति॑ प्र - अ॒नम् । \newline
19. ए॒व प॒शुषु॑ प॒शु ष्वे॒वैव प॒शुषु॑ । \newline
20. प॒शुषु॑ दधाति दधाति प॒शुषु॑ प॒शुषु॑ दधाति । \newline
21. द॒धा॒ति॒ व्या॒नाय॑ व्या॒नाय॑ दधाति दधाति व्या॒नाय॑ । \newline
22. व्या॒नाय॑ त्वा त्वा व्या॒नाय॑ व्या॒नाय॑ त्वा । \newline
23. व्या॒नायेति॑ वि - अ॒नाय॑ । \newline
24. त्वेतीति॑ त्वा॒ त्वेति॑ । \newline
25. इत्यन् वन् विती त्यनु॑ । \newline
26. अनु॑ शृन्थति शृन्थ॒ त्यन् वनु॑ शृन्थति । \newline
27. शृ॒न्थ॒ति॒ व्या॒नं ॅव्या॒नꣳ शृ॑न्थति शृन्थति व्या॒नम् । \newline
28. व्या॒न मे॒वैव व्या॒नं ॅव्या॒न मे॒व । \newline
29. व्या॒नमिति॑ वि - अ॒नम् । \newline
30. ए॒व प॒शुषु॑ प॒शु ष्वे॒वैव प॒शुषु॑ । \newline
31. प॒शुषु॑ दधाति दधाति प॒शुषु॑ प॒शुषु॑ दधाति । \newline
32. द॒धा॒ति॒ तस्मा॒त् तस्मा᳚द् दधाति दधाति॒ तस्मा᳚त् । \newline
33. तस्मा᳚थ् स्व॒पन्तꣳ॑ स्व॒पन्त॒म् तस्मा॒त् तस्मा᳚थ् स्व॒पन्त᳚म् । \newline
34. स्व॒पन्त॑म् प्रा॒णाः प्रा॒णाः स्व॒पन्तꣳ॑ स्व॒पन्त॑म् प्रा॒णाः । \newline
35. प्रा॒णा न न प्रा॒णाः प्रा॒णा न । \newline
36. प्रा॒णा इति॑ प्र - अ॒नाः । \newline
37. न ज॑हति जहति॒ न न ज॑हति । \newline
38. ज॒ह॒तीति॑ जहति । \newline

\textbf{Ghana Paata } \newline

1. वासः॒ सर्वा॑भिः॒ सर्वा॑भि॒र् वासो॒ वासः॒ सर्वा॑भि रे॒वैव सर्वा॑भि॒र् वासो॒ वासः॒ सर्वा॑भि रे॒व । \newline
2. सर्वा॑भि रे॒वैव सर्वा॑भिः॒ सर्वा॑भि रे॒वैन॑ मेन मे॒व सर्वा॑भिः॒ सर्वा॑भि रे॒वैन᳚म् । \newline
3. ए॒वैन॑ मेन मे॒वै वैन॑म् दे॒वता॑भिर् दे॒वता॑भि रेन मे॒वै वैन॑म् दे॒वता॑भिः । \newline
4. ए॒न॒म् दे॒वता॑भिर् दे॒वता॑भि रेन मेनम् दे॒वता॑भिः॒ सꣳ सम् दे॒वता॑भि रेन मेनम् दे॒वता॑भिः॒ सम् । \newline
5. दे॒वता॑भिः॒ सꣳ सम् दे॒वता॑भिर् दे॒वता॑भिः॒ स म॑र्द्धय त्यर्द्धयति॒ सम् दे॒वता॑भिर् दे॒वता॑भिः॒ स म॑र्द्धयति । \newline
6. सम॑र्द्धय त्यर्द्धयति॒ सꣳ स म॑र्द्धयति प॒शवः॑ प॒शवो᳚ ऽर्द्धयति॒ सꣳ स म॑र्द्धयति प॒शवः॑ । \newline
7. अ॒र्द्ध॒य॒ति॒ प॒शवः॑ प॒शवो᳚ ऽर्द्धय त्यर्द्धयति प॒शवो॒ वै वै प॒शवो᳚ ऽर्द्धय त्यर्द्धयति प॒शवो॒ वै । \newline
8. प॒शवो॒ वै वै प॒शवः॑ प॒शवो॒ वै सोमः॒ सोमो॒ वै प॒शवः॑ प॒शवो॒ वै सोमः॑ । \newline
9. वै सोमः॒ सोमो॒ वै वै सोमः॑ प्रा॒णाय॑ प्रा॒णाय॒ सोमो॒ वै वै सोमः॑ प्रा॒णाय॑ । \newline
10. सोमः॑ प्रा॒णाय॑ प्रा॒णाय॒ सोमः॒ सोमः॑ प्रा॒णाय॑ त्वा त्वा प्रा॒णाय॒ सोमः॒ सोमः॑ प्रा॒णाय॑ त्वा । \newline
11. प्रा॒णाय॑ त्वा त्वा प्रा॒णाय॑ प्रा॒णाय॒ त्वेतीति॑ त्वा प्रा॒णाय॑ प्रा॒णाय॒ त्वेति॑ । \newline
12. प्रा॒णायेति॑ प्र - अ॒नाय॑ । \newline
13. त्वेतीति॑ त्वा॒ त्वेत्युपोपेति॑ त्वा॒ त्वेत्युप॑ । \newline
14. इत्युपोपे तीत्युप॑ नह्यति नह्य॒ त्युपे तीत्युप॑ नह्यति । \newline
15. उप॑ नह्यति नह्य॒ त्युपोप॑ नह्यति प्रा॒णम् प्रा॒णम् न॑ह्य॒ त्युपोप॑ नह्यति प्रा॒णम् । \newline
16. न॒ह्य॒ति॒ प्रा॒णम् प्रा॒णम् न॑ह्यति नह्यति प्रा॒ण मे॒वैव प्रा॒णम् न॑ह्यति नह्यति प्रा॒ण मे॒व । \newline
17. प्रा॒ण मे॒वैव प्रा॒णम् प्रा॒ण मे॒व प॒शुषु॑ प॒शुष्वे॒व प्रा॒णम् प्रा॒ण मे॒व प॒शुषु॑ । \newline
18. प्रा॒णमिति॑ प्र - अ॒नम् । \newline
19. ए॒व प॒शुषु॑ प॒शुष्वे॒ वैव प॒शुषु॑ दधाति दधाति प॒शुष्वे॒ वैव प॒शुषु॑ दधाति । \newline
20. प॒शुषु॑ दधाति दधाति प॒शुषु॑ प॒शुषु॑ दधाति व्या॒नाय॑ व्या॒नाय॑ दधाति प॒शुषु॑ प॒शुषु॑ दधाति व्या॒नाय॑ । \newline
21. द॒धा॒ति॒ व्या॒नाय॑ व्या॒नाय॑ दधाति दधाति व्या॒नाय॑ त्वा त्वा व्या॒नाय॑ दधाति दधाति व्या॒नाय॑ त्वा । \newline
22. व्या॒नाय॑ त्वा त्वा व्या॒नाय॑ व्या॒नाय॒ त्वेतीति॑ त्वा व्या॒नाय॑ व्या॒नाय॒ त्वेति॑ । \newline
23. व्या॒नायेति॑ वि - अ॒नाय॑ । \newline
24. त्वेतीति॑ त्वा॒ त्वेत्यन् वन् विति॑ त्वा॒ त्वेत्यनु॑ । \newline
25. इत्यन्वन् विती त्यनु॑ शृन्थति शृन्थ॒ त्यन्विती त्यनु॑ शृन्थति । \newline
26. अनु॑ शृन्थति शृन्थ॒ त्यन्वनु॑ शृन्थति व्या॒नं ॅव्या॒नꣳ शृ॑न्थ॒ त्यन्वनु॑ शृन्थति व्या॒नम् । \newline
27. शृ॒न्थ॒ति॒ व्या॒नं ॅव्या॒नꣳ शृ॑न्थति शृन्थति व्या॒न मे॒वैव व्या॒नꣳ शृ॑न्थति शृन्थति व्या॒न मे॒व । \newline
28. व्या॒न मे॒वैव व्या॒नं ॅव्या॒न मे॒व प॒शुषु॑ प॒शुष्वे॒व व्या॒नं ॅव्या॒न मे॒व प॒शुषु॑ । \newline
29. व्या॒नमिति॑ वि - अ॒नम् । \newline
30. ए॒व प॒शुषु॑ प॒शुष्वे॒ वैव प॒शुषु॑ दधाति दधाति प॒शुष्वे॒ वैव प॒शुषु॑ दधाति । \newline
31. प॒शुषु॑ दधाति दधाति प॒शुषु॑ प॒शुषु॑ दधाति॒ तस्मा॒त् तस्मा᳚द् दधाति प॒शुषु॑ प॒शुषु॑ दधाति॒ तस्मा᳚त् । \newline
32. द॒धा॒ति॒ तस्मा॒त् तस्मा᳚द् दधाति दधाति॒ तस्मा᳚थ् स्व॒पन्तꣳ॑ स्व॒पन्त॒म् तस्मा᳚द् दधाति दधाति॒ तस्मा᳚थ् स्व॒पन्त᳚म् । \newline
33. तस्मा᳚थ् स्व॒पन्तꣳ॑ स्व॒पन्त॒म् तस्मा॒त् तस्मा᳚थ् स्व॒पन्त॑म् प्रा॒णाः प्रा॒णाः स्व॒पन्त॒म् तस्मा॒त् तस्मा᳚थ् स्व॒पन्त॑म् प्रा॒णाः । \newline
34. स्व॒पन्त॑म् प्रा॒णाः प्रा॒णाः स्व॒पन्तꣳ॑ स्व॒पन्त॑म् प्रा॒णा न न प्रा॒णाः स्व॒पन्तꣳ॑ स्व॒पन्त॑म् प्रा॒णा न । \newline
35. प्रा॒णा न न प्रा॒णाः प्रा॒णा न ज॑हति जहति॒ न प्रा॒णाः प्रा॒णा न ज॑हति । \newline
36. प्रा॒णा इति॑ प्र - अ॒नाः । \newline
37. न ज॑हति जहति॒ न न ज॑हति । \newline
38. ज॒ह॒तीति॑ जहति । \newline
\pagebreak
\markright{ TS 6.1.10.1  \hfill https://www.vedavms.in \hfill}

\section{ TS 6.1.10.1 }

\textbf{TS 6.1.10.1 } \newline
\textbf{Samhita Paata} \newline

यत् क॒लया॑ ते श॒फेन॑ ते क्रीणा॒नीति॒ पणे॒तागो॑अर्घꣳ॒॒ सोमं॑ कु॒र्यादगो॑अर्घं॒ ॅयज॑मान॒-मगो॑अर्घमद्ध्व॒र्युं गोस्तु म॑हि॒मानं॒ नाव॑ तिरे॒द् गवा॑ ते क्रीणा॒नीत्ये॒व ब्रू॑याद् गोअ॒र्घमे॒व सोमं॑ क॒रोति॑ गोअ॒र्घं ॅयज॑मानं गोअ॒र्घम॑द्ध्व॒र्युं न गोर्म॑हि॒मान॒मव॑ तिरत्य॒जया᳚ क्रीणाति॒ सत॑पसमे॒वैनं॑ क्रीणाति॒ हिर॑ण्येन क्रीणाति॒ सशु॑क्रमे॒वै - [  ] \newline

\textbf{Pada Paata} \newline

यत् । क॒लया᳚ । ते॒ । श॒फेन॑ । ते॒ । क्री॒णा॒नि॒ । इति॑ । पणे॑त । अगो॑अर्घ॒मित्यगो᳚ - अ॒र्घ॒म् । सोम᳚म् । कु॒र्यात् । अगो॑अर्घ॒मित्यगो᳚ - अ॒र्घ॒म् । यज॑मानम् । अगो॑अर्घ॒मित्यगो᳚-अ॒र्घ॒म् । अ॒द्ध्व॒र्युम् । गोः । तु । म॒हि॒मान᳚म् । न । अवेति॑ । ति॒रे॒त् । गवा᳚ । ते॒ । क्री॒णा॒नि॒ । इति॑ । ए॒व । ब्रू॒या॒त् । गो॒अ॒र्घमिति॑ गो - अ॒र्घम् । ए॒व । सोम᳚म् । क॒रोति॑ । गो॒अ॒र्घमिति॑ गो - अ॒र्घम् । यज॑मानम् । गो॒अ॒र्घमिति॑ गो - अ॒र्घम् । अ॒द्ध्व॒र्युम् । न । गोः । म॒हि॒मान᳚म् । अवेति॑ । ति॒र॒ति॒ । अ॒जया᳚ । क्री॒णा॒ति॒ । सत॑पस॒मिति॒ स - त॒प॒स॒म् । ए॒व । ए॒न॒म् । क्री॒णा॒ति॒ । हिर॑ण्येन । क्री॒णा॒ति॒ । सशु॑क्र॒मिति॒ स - शु॒क्र॒म् । ए॒व ।  \newline


\textbf{Krama Paata} \newline

यत् क॒लया᳚ । क॒लया॑ ते । ते॒ श॒फेन॑ । श॒फेन॑ ते । ते॒ क्री॒णा॒नि॒ । क्री॒णा॒नीति॑ । इति॒ पणे॑त । पणे॒तागो॑अर्घम् । अगो॑अर्घꣳ॒॒ सोम᳚म् । अगो॑अर्घ॒मित्यगो᳚ - अ॒र्घ॒म् । सोम॑म् कु॒र्यात् । कु॒र्यादगो॑अर्घम् । अगो॑अर्घ॒म् ॅयज॑मानम् । अगो॑अर्घ॒मित्यगो᳚ - अ॒र्घ॒म् । यज॑मान॒मगो॑अर्घम् । अगो॑अर्घमद्ध्व॒र्युम् । अगो॑अर्घ॒मित्यगो᳚ - अ॒र्घ॒म् । अ॒द्ध्व॒र्युम् गोः । गोस्तु । तु म॑हि॒मान᳚म् । म॒हि॒मान॒म् न । नाव॑ । अव॑ तिरेत् । ति॒रे॒द् गवा᳚ । गवा॑ ते । ते॒ क्री॒णा॒नि॒ । क्री॒णा॒नीति॑ । इत्ये॒व । ए॒व ब्रू॑यात् । ब्रू॒या॒द् गो॒अ॒र्घम् । गो॒अ॒र्घमे॒व । गो॒अ॒र्घमिति॑ गो - अ॒र्घम् । ए॒व सोम᳚म् । सोम॑म् क॒रोति॑ । क॒रोति॑ गोअ॒र्घम् । गो॒अ॒र्घम् ॅयज॑मानम् । गो॒अ॒र्घमिति॑ गो - अ॒र्घम् । यज॑मानम् गोअ॒र्घम् । गो॒अ॒र्घम॑द्ध्व॒र्युम् । गो॒अ॒र्घमिति॑ गो - अ॒र्घम् । अ॒द्ध्व॒र्युम् न । न गोः । गोर् म॑हि॒मान᳚म् । म॒हि॒मान॒मव॑ । अव॑ तिरति । ति॒र॒त्य॒जया᳚ । अ॒जया᳚ क्रीणाति । क्री॒णा॒ति॒ सत॑पसम् । सत॑पसमे॒व । सत॑पस॒मिति॒ स - त॒प॒स॒म् । ए॒वैन᳚म् । ए॒न॒म् क्री॒णा॒ति॒ । क्री॒णा॒ति॒ हिर॑ण्येन । हिर॑ण्येन क्रीणाति । क्री॒णा॒ति॒ सशु॑क्रम् । सशु॑क्रमे॒व । सशु॑क्र॒मिति॒ स - शु॒क्र॒म् । ए॒वैन᳚म् \newline

\textbf{Jatai Paata} \newline

1. यत् क॒लया॑ क॒लया॒ यद् यत् क॒लया᳚ । \newline
2. क॒लया॑ ते ते क॒लया॑ क॒लया॑ ते । \newline
3. ते॒ श॒फेन॑ श॒फेन॑ ते ते श॒फेन॑ । \newline
4. श॒फेन॑ ते ते श॒फेन॑ श॒फेन॑ ते । \newline
5. ते॒ क्री॒णा॒नि॒ क्री॒णा॒नि॒ ते॒ ते॒ क्री॒णा॒नि॒ । \newline
6. क्री॒णा॒नीतीति॑ क्रीणानि क्रीणा॒नीति॑ । \newline
7. इति॒ पणे॑त॒ पणे॒ते तीति॒ पणे॑त । \newline
8. पणे॒ता गो॑अर्घ॒ मगो॑अर्घ॒म् पणे॑त॒ पणे॒ता गो॑अर्घम् । \newline
9. अगो॑अर्घꣳ॒॒ सोमꣳ॒॒ सोम॒ मगो॑अर्घ॒ मगो॑अर्घꣳ॒॒ सोम᳚म् । \newline
10. अगो॑अर्घ॒मित्यगो᳚ - अ॒र्घ॒म् । \newline
11. सोम॑म् कु॒र्यात् कु॒र्याथ् सोमꣳ॒॒ सोम॑म् कु॒र्यात् । \newline
12. कु॒र्या दगो॑अर्घ॒ मगो॑अर्घम् कु॒र्यात् कु॒र्या दगो॑अर्घम् । \newline
13. अगो॑अर्घं॒ ॅयज॑मानं॒ ॅयज॑मान॒ मगो॑अर्घ॒ मगो॑अर्घं॒ ॅयज॑मानम् । \newline
14. अगो॑अर्घ॒मित्यगो᳚ - अ॒र्घ॒म् । \newline
15. यज॑मान॒ मगो॑अर्घ॒ मगो॑अर्घं॒ ॅयज॑मानं॒ ॅयज॑मान॒ मगो॑अर्घम् । \newline
16. अगो॑अर्घ मद्ध्व॒र्यु म॑द्ध्व॒र्यु मगो॑अर्घ॒ मगो॑अर्घ मद्ध्व॒र्युम् । \newline
17. अगो॑अर्घ॒मित्यगो᳚ - अ॒र्घ॒म् । \newline
18. अ॒द्ध्व॒र्युम् गोर् गोर॑द्ध्व॒र्यु म॑द्ध्व॒र्युम् गोः । \newline
19. गो स्तु तु गोर् गो स्तु । \newline
20. तु म॑हि॒मान॑म् महि॒मान॒म् तु तु म॑हि॒मान᳚म् । \newline
21. म॒हि॒मान॒न् न न म॑हि॒मान॑म् महि॒मान॒न् न । \newline
22. नावाव॒ न नाव॑ । \newline
23. अव॑ तिरेत् तिरे॒ दवाव॑ तिरेत् । \newline
24. ति॒रे॒द् गवा॒ गवा॑ तिरेत् तिरे॒द् गवा᳚ । \newline
25. गवा॑ ते ते॒ गवा॒ गवा॑ ते । \newline
26. ते॒ क्री॒णा॒नि॒ क्री॒णा॒नि॒ ते॒ ते॒ क्री॒णा॒नि॒ । \newline
27. क्री॒णा॒नी तीति॑ क्रीणानि क्रीणा॒नीति॑ । \newline
28. इत्ये॒वैवे तीत्ये॒व । \newline
29. ए॒व ब्रू॑याद् ब्रूया दे॒वैव ब्रू॑यात् । \newline
30. ब्रू॒या॒द् गो॒अ॒र्घम् गो॑अ॒र्घम् ब्रू॑याद् ब्रूयाद् गोअ॒र्घम् । \newline
31. गो॒अ॒र्घ मे॒वैव गो॑अ॒र्घम् गो॑अ॒र्घ मे॒व । \newline
32. गो॒अ॒र्घमिति॑ गो - अ॒र्घम् । \newline
33. ए॒व सोमꣳ॒॒ सोम॑ मे॒वैव सोम᳚म् । \newline
34. सोम॑म् क॒रोति॑ क॒रोति॒ सोमꣳ॒॒ सोम॑म् क॒रोति॑ । \newline
35. क॒रोति॑ गोअ॒र्घम् गो॑अ॒र्घम् क॒रोति॑ क॒रोति॑ गोअ॒र्घम् । \newline
36. गो॒अ॒र्घं ॅयज॑मानं॒ ॅयज॑मानम् गोअ॒र्घम् गो॑अ॒र्घं ॅयज॑मानम् । \newline
37. गो॒अ॒र्घमिति॑ गो - अ॒र्घम् । \newline
38. यज॑मानम् गोअ॒र्घम् गो॑अ॒र्घं ॅयज॑मानं॒ ॅयज॑मानम् गोअ॒र्घम् । \newline
39. गो॒अ॒र्घ म॑द्ध्व॒र्यु म॑द्ध्व॒र्युम् गो॑अ॒र्घम् गो॑अ॒र्घ म॑द्ध्व॒र्युम् । \newline
40. गो॒अ॒र्घमिति॑ गो - अ॒र्घम् । \newline
41. अ॒द्ध्व॒र्युन् न नाद्ध्व॒र्यु म॑द्ध्व॒र्युन् न । \newline
42. न गोर् गोर् न न गोः । \newline
43. गोर् म॑हि॒मान॑म् महि॒मान॒म् गोर् गोर् म॑हि॒मान᳚म् । \newline
44. म॒हि॒मान॒ मवाव॑ महि॒मान॑म् महि॒मान॒ मव॑ । \newline
45. अव॑ तिरति तिर॒ त्यवाव॑ तिरति । \newline
46. ति॒र॒ त्य॒जया॒ ऽजया॑ तिरति तिर त्य॒जया᳚ । \newline
47. अ॒जया᳚ क्रीणाति क्रीणा त्य॒जया॒ ऽजया᳚ क्रीणाति । \newline
48. क्री॒णा॒ति॒ सत॑पसꣳ॒॒ सत॑पसम् क्रीणाति क्रीणाति॒ सत॑पसम् । \newline
49. सत॑पस मे॒वैव सत॑पसꣳ॒॒ सत॑पस मे॒व । \newline
50. सत॑पस॒मिति॒ स - त॒प॒स॒म् । \newline
51. ए॒वैन॑ मेन मे॒वै वैन᳚म् । \newline
52. ए॒न॒म् क्री॒णा॒ति॒ क्री॒णा॒ त्ये॒न॒ मे॒न॒म् क्री॒णा॒ति॒ । \newline
53. क्री॒णा॒ति॒ हिर॑ण्येन॒ हिर॑ण्येन क्रीणाति क्रीणाति॒ हिर॑ण्येन । \newline
54. हिर॑ण्येन क्रीणाति क्रीणाति॒ हिर॑ण्येन॒ हिर॑ण्येन क्रीणाति । \newline
55. क्री॒णा॒ति॒ सशु॑क्रꣳ॒॒ सशु॑क्रम् क्रीणाति क्रीणाति॒ सशु॑क्रम् । \newline
56. सशु॑क्र मे॒वैव सशु॑क्रꣳ॒॒ सशु॑क्र मे॒व । \newline
57. सशु॑क्र॒मिति॒ स - शु॒क्र॒म् । \newline
58. ए॒वैन॑ मेन मे॒वै वैन᳚म् । \newline

\textbf{Ghana Paata } \newline

1. यत् क॒लया॑ क॒लया॒ यद् यत् क॒लया॑ ते ते क॒लया॒ यद् यत् क॒लया॑ ते । \newline
2. क॒लया॑ ते ते क॒लया॑ क॒लया॑ ते श॒फेन॑ श॒फेन॑ ते क॒लया॑ क॒लया॑ ते श॒फेन॑ । \newline
3. ते॒ श॒फेन॑ श॒फेन॑ ते ते श॒फेन॑ ते ते श॒फेन॑ ते ते श॒फेन॑ ते । \newline
4. श॒फेन॑ ते ते श॒फेन॑ श॒फेन॑ ते क्रीणानि क्रीणानि ते श॒फेन॑ श॒फेन॑ ते क्रीणानि । \newline
5. ते॒ क्री॒णा॒नि॒ क्री॒णा॒नि॒ ते॒ ते॒ क्री॒णा॒नी तीति॑ क्रीणानि ते ते क्रीणा॒नीति॑ । \newline
6. क्री॒णा॒नी तीति॑ क्रीणानि क्रीणा॒नीति॒ पणे॑त॒ पणे॒तेति॑ क्रीणानि क्रीणा॒नीति॒ पणे॑त । \newline
7. इति॒ पणे॑त॒ पणे॒ते तीति॒ पणे॒ता गो॑अर्घ॒ मगो॑अर्घ॒म् पणे॒ते तीति॒ पणे॒ता गो॑अर्घम् । \newline
8. पणे॒ता गो॑अर्घ॒ मगो॑अर्घ॒म् पणे॑त॒ पणे॒ता गो॑अर्घꣳ॒॒ सोमꣳ॒॒ सोम॒ मगो॑अर्घ॒म् पणे॑त॒ पणे॒ता गो॑अर्घꣳ॒॒ सोम᳚म् । \newline
9. अगो॑अर्घꣳ॒॒ सोमꣳ॒॒ सोम॒ मगो॑अर्घ॒ मगो॑अर्घꣳ॒॒ सोम॑म् कु॒र्यात् कु॒र्याथ् सोम॒ मगो॑अर्घ॒ मगो॑अर्घꣳ॒॒ सोम॑म् कु॒र्यात् । \newline
10. अगो॑अर्घ॒मित्यगो᳚ - अ॒र्घ॒म् । \newline
11. सोम॑म् कु॒र्यात् कु॒र्याथ् सोमꣳ॒॒ सोम॑म् कु॒र्या दगो॑अर्घ॒ मगो॑अर्घम् कु॒र्याथ् सोमꣳ॒॒ सोम॑म् कु॒र्या दगो॑अर्घम् । \newline
12. कु॒र्या दगो॑अर्घ॒ मगो॑अर्घम् कु॒र्यात् कु॒र्या दगो॑अर्घं॒ ॅयज॑मानं॒ ॅयज॑मान॒ मगो॑अर्घम् कु॒र्यात् कु॒र्या दगो॑अर्घं॒ ॅयज॑मानम् । \newline
13. अगो॑अर्घं॒ ॅयज॑मानं॒ ॅयज॑मान॒ मगो॑अर्घ॒ मगो॑अर्घं॒ ॅयज॑मान॒ मगो॑अर्घ॒ मगो॑अर्घं॒ ॅयज॑मान॒ मगो॑अर्घ॒ मगो॑अर्घं॒ ॅयज॑मान॒ मगो॑अर्घम् । \newline
14. अगो॑अर्घ॒मित्यगो᳚ - अ॒र्घ॒म् । \newline
15. यज॑मान॒ मगो॑अर्घ॒ मगो॑अर्घं॒ ॅयज॑मानं॒ ॅयज॑मान॒ मगो॑अर्घ मद्ध्व॒र्यु म॑द्ध्व॒र्यु मगो॑अर्घं॒ ॅयज॑मानं॒ ॅयज॑मान॒ मगो॑अर्घ मद्ध्व॒र्युम् । \newline
16. अगो॑अर्घ मद्ध्व॒र्यु म॑द्ध्व॒र्यु मगो॑अर्घ॒ मगो॑अर्घ मद्ध्व॒र्युम् गोर् गोर॑द्ध्व॒र्यु मगो॑अर्घ॒ मगो॑अर्घ मद्ध्व॒र्युम् गोः । \newline
17. अगो॑अर्घ॒मित्यगो᳚ - अ॒र्घ॒म् । \newline
18. अ॒द्ध्व॒र्युम् गोर् गोर॑द्ध्व॒र्यु म॑द्ध्व॒र्युम् गो स्तु तु गोर॑द्ध्व॒र्यु म॑द्ध्व॒र्युम् गो स्तु । \newline
19. गो स्तु तु गोर् गो स्तु म॑हि॒मान॑म् महि॒मान॒म् तु गोर् गो स्तु म॑हि॒मान᳚म् । \newline
20. तु म॑हि॒मान॑म् महि॒मान॒म् तु तु म॑हि॒मान॒म् न न म॑हि॒मान॒म् तु तु म॑हि॒मान॒म् न । \newline
21. म॒हि॒मान॒म् न न म॑हि॒मान॑म् महि॒मान॒म् नावाव॒ न म॑हि॒मान॑म् महि॒मान॒म् नाव॑ । \newline
22. नावाव॒ न नाव॑ तिरेत् तिरे॒ दव॒ न नाव॑ तिरेत् । \newline
23. अव॑ तिरेत् तिरे॒ दवाव॑ तिरे॒द् गवा॒ गवा॑ तिरे॒ दवाव॑ तिरे॒द् गवा᳚ । \newline
24. ति॒रे॒द् गवा॒ गवा॑ तिरेत् तिरे॒द् गवा॑ ते ते॒ गवा॑ तिरेत् तिरे॒द् गवा॑ ते । \newline
25. गवा॑ ते ते॒ गवा॒ गवा॑ ते क्रीणानि क्रीणानि ते॒ गवा॒ गवा॑ ते क्रीणानि । \newline
26. ते॒ क्री॒णा॒नि॒ क्री॒णा॒नि॒ ते॒ ते॒ क्री॒णा॒नी तीति॑ क्रीणानि ते ते क्रीणा॒नीति॑ । \newline
27. क्री॒णा॒नी तीति॑ क्रीणानि क्रीणा॒नी त्ये॒वैवेति॑ क्रीणानि क्रीणा॒नी त्ये॒व । \newline
28. इत्ये॒ वैवे तीत्ये॒व ब्रू॑याद् ब्रूया दे॒वे तीत्ये॒व ब्रू॑यात् । \newline
29. ए॒व ब्रू॑याद् ब्रूया दे॒वैव ब्रू॑याद् गोअ॒र्घम् गो॑अ॒र्घम् ब्रू॑या दे॒वैव ब्रू॑याद् गोअ॒र्घम् । \newline
30. ब्रू॒या॒द् गो॒अ॒र्घम् गो॑अ॒र्घम् ब्रू॑याद् ब्रूयाद् गोअ॒र्घ मे॒वैव गो॑अ॒र्घम् ब्रू॑याद् ब्रूयाद् गोअ॒र्घ मे॒व । \newline
31. गो॒अ॒र्घ मे॒वैव गो॑अ॒र्घम् गो॑अ॒र्घ मे॒व सोमꣳ॒॒ सोम॑ मे॒व गो॑अ॒र्घम् गो॑अ॒र्घ मे॒व सोम᳚म् । \newline
32. गो॒अ॒र्घमिति॑ गो - अ॒र्घम् । \newline
33. ए॒व सोमꣳ॒॒ सोम॑ मे॒वैव सोम॑म् क॒रोति॑ क॒रोति॒ सोम॑ मे॒वैव सोम॑म् क॒रोति॑ । \newline
34. सोम॑म् क॒रोति॑ क॒रोति॒ सोमꣳ॒॒ सोम॑म् क॒रोति॑ गोअ॒र्घम् गो॑अ॒र्घम् क॒रोति॒ सोमꣳ॒॒ सोम॑म् क॒रोति॑ गोअ॒र्घम् । \newline
35. क॒रोति॑ गोअ॒र्घम् गो॑अ॒र्घम् क॒रोति॑ क॒रोति॑ गोअ॒र्घं ॅयज॑मानं॒ ॅयज॑मानम् गोअ॒र्घम् क॒रोति॑ क॒रोति॑ गोअ॒र्घं ॅयज॑मानम् । \newline
36. गो॒अ॒र्घं ॅयज॑मानं॒ ॅयज॑मानम् गोअ॒र्घम् गो॑अ॒र्घं ॅयज॑मानम् गोअ॒र्घम् गो॑अ॒र्घं ॅयज॑मानम् गोअ॒र्घम् गो॑अ॒र्घं ॅयज॑मानम् गोअ॒र्घम् । \newline
37. गो॒अ॒र्घमिति॑ गो - अ॒र्घम् । \newline
38. यज॑मानम् गोअ॒र्घम् गो॑अ॒र्घं ॅयज॑मानं॒ ॅयज॑मानम् गोअ॒र्घ म॑द्ध्व॒र्यु म॑द्ध्व॒र्युम् गो॑अ॒र्घं ॅयज॑मानं॒ ॅयज॑मानम् गोअ॒र्घ म॑द्ध्व॒र्युम् । \newline
39. गो॒अ॒र्घ म॑द्ध्व॒र्यु म॑द्ध्व॒र्युम् गो॑अ॒र्घम् गो॑अ॒र्घ म॑द्ध्व॒र्युन् न नाद्ध्व॒र्युम् गो॑अ॒र्घम् गो॑अ॒र्घ म॑द्ध्व॒र्युन् न । \newline
40. गो॒अ॒र्घमिति॑ गो - अ॒र्घम् । \newline
41. अ॒द्ध्व॒र्युम् न नाद्ध्व॒र्यु म॑द्ध्व॒र्युम् न गोर् गोर् नाद्ध्व॒र्यु म॑द्ध्व॒र्युम् न गोः । \newline
42. न गोर् गोर् न न गोर् म॑हि॒मान॑म् महि॒मान॒म् गोर् न न गोर् म॑हि॒मान᳚म् । \newline
43. गोर् म॑हि॒मान॑म् महि॒मान॒म् गोर् गोर् म॑हि॒मान॒ मवाव॑ महि॒मान॒म् गोर् गोर् म॑हि॒मान॒ मव॑ । \newline
44. म॒हि॒मान॒ मवाव॑ महि॒मान॑म् महि॒मान॒ मव॑ तिरति तिर॒ त्यव॑ महि॒मान॑म् महि॒मान॒ मव॑ तिरति । \newline
45. अव॑ तिरति तिर॒ त्यवाव॑ तिर त्य॒जया॒ ऽजया॑ तिर॒ त्यवाव॑ तिर त्य॒जया᳚ । \newline
46. ति॒र॒ त्य॒जया॒ ऽजया॑ तिरति तिर त्य॒जया᳚ क्रीणाति क्रीणा त्य॒जया॑ तिरति तिर त्य॒जया᳚ क्रीणाति । \newline
47. अ॒जया᳚ क्रीणाति क्रीणा त्य॒जया॒ ऽजया᳚ क्रीणाति॒ सत॑पसꣳ॒॒ सत॑पसम् क्रीणा त्य॒जया॒ ऽजया᳚ क्रीणाति॒ सत॑पसम् । \newline
48. क्री॒णा॒ति॒ सत॑पसꣳ॒॒ सत॑पसम् क्रीणाति क्रीणाति॒ सत॑पस मे॒वैव सत॑पसम् क्रीणाति क्रीणाति॒ सत॑पस मे॒व । \newline
49. सत॑पस मे॒वैव सत॑पसꣳ॒॒ सत॑पस मे॒वैन॑ मेन मे॒व सत॑पसꣳ॒॒ सत॑पस मे॒वैन᳚म् । \newline
50. सत॑पस॒मिति॒ स - त॒प॒स॒म् । \newline
51. ए॒वैन॑ मेन मे॒वै वैन॑म् क्रीणाति क्रीणा त्येन मे॒वै वैन॑म् क्रीणाति । \newline
52. ए॒न॒म् क्री॒णा॒ति॒ क्री॒णा॒ त्ये॒न॒ मे॒न॒म् क्री॒णा॒ति॒ हिर॑ण्येन॒ हिर॑ण्येन क्रीणा त्येन मेनम् क्रीणाति॒ हिर॑ण्येन । \newline
53. क्री॒णा॒ति॒ हिर॑ण्येन॒ हिर॑ण्येन क्रीणाति क्रीणाति॒ हिर॑ण्येन क्रीणाति क्रीणाति॒ हिर॑ण्येन क्रीणाति क्रीणाति॒ हिर॑ण्येन क्रीणाति । \newline
54. हिर॑ण्येन क्रीणाति क्रीणाति॒ हिर॑ण्येन॒ हिर॑ण्येन क्रीणाति॒ सशु॑क्रꣳ॒॒ सशु॑क्रम् क्रीणाति॒ हिर॑ण्येन॒ हिर॑ण्येन क्रीणाति॒ सशु॑क्रम् । \newline
55. क्री॒णा॒ति॒ सशु॑क्रꣳ॒॒ सशु॑क्रम् क्रीणाति क्रीणाति॒ सशु॑क्र मे॒वैव सशु॑क्रम् क्रीणाति क्रीणाति॒ सशु॑क्र मे॒व । \newline
56. सशु॑क्र मे॒वैव सशु॑क्रꣳ॒॒ सशु॑क्र मे॒वैन॑ मेन मे॒व सशु॑क्रꣳ॒॒ सशु॑क्र मे॒वैन᳚म् । \newline
57. सशु॑क्र॒मिति॒ स - शु॒क्र॒म् । \newline
58. ए॒वैन॑ मेन मे॒वै वैन॑म् क्रीणाति क्रीणा त्येन मे॒वै वैन॑म् क्रीणाति । \newline
\pagebreak
\markright{ TS 6.1.10.2  \hfill https://www.vedavms.in \hfill}

\section{ TS 6.1.10.2 }

\textbf{TS 6.1.10.2 } \newline
\textbf{Samhita Paata} \newline

-नं॑ क्रीणाति धे॒न्वा क्री॑णाति॒ साशि॑रमे॒वैनं॑ क्रीणात्यृष॒भेण॑ क्रीणाति॒ सेन्द्र॑मे॒वैनं॑ क्रीणात्यन॒डुहा᳚ क्रीणाति॒ वह्नि॒र्वा अ॑न॒ड्वान्. वह्नि॑नै॒व वह्नि॑ य॒ज्ञ्स्य॑ क्रीणाति मिथु॒नाभ्यां᳚ क्रीणाति मिथु॒नस्याव॑-रुद्ध्यै॒ वास॑सा क्रीणाति सर्वदेव॒त्यं॑ ॅवै वा॒सः सर्वा᳚भ्य ए॒वैनं॑ दे॒वता᳚भ्यः क्रीणाति॒ दश॒ सं प॑द्यन्ते॒ दशा᳚क्षरा वि॒राडन्नं॑ ॅवि॒राड् वि॒राजै॒वान्नाद्य॒मव॑ रुन्धे॒ - [  ] \newline

\textbf{Pada Paata} \newline

ए॒न॒म् । क्री॒णा॒ति॒ । धे॒न्वा । क्री॒णा॒ति॒ । साशि॑र॒मिति॒ स - आ॒शि॒र॒म् । ए॒व । ए॒न॒म् । क्री॒णा॒ति॒ । ऋ॒ष॒भेण॑ । क्री॒णा॒ति॒ । सेन्द्र॒मिति॒ स-इ॒न्द्र॒म् । ए॒व । ए॒न॒म् । क्री॒णा॒ति॒ । अ॒न॒डुहा᳚ । क्री॒णा॒ति॒ । वह्निः॑ । वै । अ॒न॒ड्वान् । वह्नि॑ना । ए॒व । वह्नि॑ । य॒ज्ञ्स्य॑ । क्री॒णा॒ति॒ । मि॒थु॒नाभ्या᳚म् । क्री॒णा॒ति॒ । मि॒थु॒नस्य॑ । अव॑रुद्ध्या॒ इत्यव॑ - रु॒द्ध्यै॒ । वास॑सा । क्री॒णा॒ति॒ । स॒र्व॒दे॒व॒त्य॑मिति॑ सर्व - दे॒व॒त्य᳚म् । वै । वासः॑ । सर्वा᳚भ्यः । ए॒व । ए॒न॒म् । दे॒वता᳚भ्यः । क्री॒णा॒ति॒ । दश॑ । समिति॑ । प॒द्य॒न्ते॒ । दशा᳚क्ष॒रेति॒ दश॑ - अ॒क्ष॒रा॒ । वि॒राडिति॑ वि - राट् । अन्न᳚म् । वि॒राडिति॑ वि - राट् । वि॒राजेति॑ वि - राजा᳚ । ए॒व । अ॒न्नाद्य॒मित्य॑न्न -अद्य᳚म् । अवेति॑ । रु॒न्धे॒ ।  \newline


\textbf{Krama Paata} \newline

ए॒न॒म् क्री॒णा॒ति॒ । क्री॒णा॒ति॒ धे॒न्वा । धे॒न्वा क्री॑णाति । क्री॒णा॒ति॒ साशि॑रम् । साशि॑रमे॒व । साशि॑र॒मिति॒ स - आ॒शि॒र॒म् । ए॒वैन᳚म् । ए॒न॒म् क्री॒णा॒ति॒ । क्री॒णा॒त्यृ॒ष॒भेण॑ । ऋ॒ष॒भेण॑ क्रीणाति । क्री॒णा॒ति॒ सेन्द्र᳚म् । सेन्द्र॑मे॒व । सेन्द्र॒मिति॒ स - इ॒न्द्र॒म् । ए॒वैन᳚म् । ए॒न॒म् क्री॒णा॒ति॒ । क्री॒णा॒त्य॒न॒डुहा᳚ । अ॒न॒डुहा᳚ क्रीणाति । क्री॒णा॒ति॒ वह्निः॑ । वह्नि॒र् वै । वा अ॑न॒ड्वान् । अ॒न॒ड्वान् वह्नि॑ना । वह्नि॑नै॒व । ए॒व वह्नि॑ । वह्नि॑ य॒ज्ञ्स्य॑ । य॒ज्ञ्स्य॑ क्रीणाति । क्री॒णा॒ति॒ मि॒थु॒नाभ्या᳚म् । मि॒थु॒नाभ्या᳚म् क्रीणाति । क्री॒णा॒ति॒ मि॒थु॒नस्य॑ । मि॒थु॒नस्याव॑रुद्ध्यै । अव॑रुद्ध्यै॒ वास॑सा । अव॑रुद्ध्या॒ इत्यव॑ - रु॒द्ध्यै॒ । वास॑सा क्रीणाति । क्री॒णा॒ति॒ स॒र्व॒दे॒व॒त्य᳚म् । स॒र्व॒दे॒व॒त्य॑म् ॅवै । स॒र्व॒दे॒व॒त्य॑मिति॑ सर्व - दे॒व॒त्य᳚म् । वै वासः॑ । वासः॒ सर्वा᳚भ्यः । सर्वा᳚भ्य ए॒व । ए॒वैन᳚म् । ए॒न॒म् दे॒वता᳚भ्यः । दे॒वता᳚भ्यः क्रीणाति । क्री॒णा॒ति॒ दश॑ । दश॒ सम् । सम् प॑द्यन्ते । प॒द्य॒न्ते॒ दशा᳚क्षरा । दशा᳚क्षरा वि॒राट् । दशा᳚क्ष॒रेति॒ दश॑ - अ॒क्ष॒रा॒ । वि॒राडन्न᳚म् । वि॒राडिति॑ वि - राट् । अन्न॑म् ॅवि॒राट् । वि॒राड् वि॒राजा᳚ । वि॒राडिति॑ वि - राट् । वि॒राजै॒व । वि॒राजेति॑ वि - राजा᳚ । ए॒वान्नाद्य᳚म् । अ॒न्नाद्य॒मव॑ । अ॒न्नाद्य॒मित्य॑न्न - अद्य᳚म् । अव॑ रुन्धे । रु॒न्धे॒ तप॑सः \newline

\textbf{Jatai Paata} \newline

1. ए॒न॒म् क्री॒णा॒ति॒ क्री॒णा॒ त्ये॒न॒ मे॒न॒म् क्री॒णा॒ति॒ । \newline
2. क्री॒णा॒ति॒ धे॒न्वा धे॒न्वा क्री॑णाति क्रीणाति धे॒न्वा । \newline
3. धे॒न्वा क्री॑णाति क्रीणाति धे॒न्वा धे॒न्वा क्री॑णाति । \newline
4. क्री॒णा॒ति॒ साशि॑रꣳ॒॒ साशि॑रम् क्रीणाति क्रीणाति॒ साशि॑रम् । \newline
5. साशि॑र मे॒वैव साशि॑रꣳ॒॒ साशि॑र मे॒व । \newline
6. साशि॑र॒मिति॒ स - आ॒शि॒र॒म् । \newline
7. ए॒वैन॑ मेन मे॒वै वैन᳚म् । \newline
8. ए॒न॒म् क्री॒णा॒ति॒ क्री॒णा॒ त्ये॒न॒ मे॒न॒म् क्री॒णा॒ति॒ । \newline
9. क्री॒णा॒ त्यृ॒ष॒भेण॑ र्.ष॒भेण॑ क्रीणाति क्रीणा त्यृष॒भेण॑ । \newline
10. ऋ॒ष॒भेण॑ क्रीणाति क्रीणा त्यृष॒भेण॑ र्.ष॒भेण॑ क्रीणाति । \newline
11. क्री॒णा॒ति॒ सेन्द्रꣳ॒॒ सेन्द्र॑म् क्रीणाति क्रीणाति॒ सेन्द्र᳚म् । \newline
12. सेन्द्र॑ मे॒वैव सेन्द्रꣳ॒॒ सेन्द्र॑ मे॒व । \newline
13. सेन्द्र॒मिति॒ स - इ॒न्द्र॒म् । \newline
14. ए॒वैन॑ मेन मे॒वै वैन᳚म् । \newline
15. ए॒न॒म् क्री॒णा॒ति॒ क्री॒णा॒ त्ये॒न॒ मे॒न॒म् क्री॒णा॒ति॒ । \newline
16. क्री॒णा॒ त्य॒न॒डुहा॑ ऽन॒डुहा᳚ क्रीणाति क्रीणा त्यन॒डुहा᳚ । \newline
17. अ॒न॒डुहा᳚ क्रीणाति क्रीणा त्यन॒डुहा॑ ऽन॒डुहा᳚ क्रीणाति । \newline
18. क्री॒णा॒ति॒ वह्नि॒र् वह्निः॑ क्रीणाति क्रीणाति॒ वह्निः॑ । \newline
19. वह्नि॒र् वै वै वह्नि॒र् वह्नि॒र् वै । \newline
20. वा अ॑न॒ड्वा न॑न॒ड्वान्. वै वा अ॑न॒ड्वान् । \newline
21. अ॒न॒ड्वान्. वह्नि॑ना॒ वह्नि॑ना ऽन॒ड्वा न॑न॒ड्वान्. वह्नि॑ना । \newline
22. वह्नि॑ नै॒वैव वह्नि॑ना॒ वह्नि॑ नै॒व । \newline
23. ए॒व वह्नि॒ वह्न्ये॒ वैव वह्नि॑ । \newline
24. वह्नि॑ य॒ज्ञ्स्य॑ य॒ज्ञ्स्य॒ वह्नि॒ वह्नि॑ य॒ज्ञ्स्य॑ । \newline
25. य॒ज्ञ्स्य॑ क्रीणाति क्रीणाति य॒ज्ञ्स्य॑ य॒ज्ञ्स्य॑ क्रीणाति । \newline
26. क्री॒णा॒ति॒ मि॒थु॒नाभ्या᳚म् मिथु॒नाभ्या᳚म् क्रीणाति क्रीणाति मिथु॒नाभ्या᳚म् । \newline
27. मि॒थु॒नाभ्या᳚म् क्रीणाति क्रीणाति मिथु॒नाभ्या᳚म् मिथु॒नाभ्या᳚म् क्रीणाति । \newline
28. क्री॒णा॒ति॒ मि॒थु॒नस्य॑ मिथु॒नस्य॑ क्रीणाति क्रीणाति मिथु॒नस्य॑ । \newline
29. मि॒थु॒नस्या व॑रुद्ध्या॒ अव॑रुद्ध्यै मिथु॒नस्य॑ मिथु॒नस्या व॑रुद्ध्यै । \newline
30. अव॑रुद्ध्यै॒ वास॑सा॒ वास॒सा ऽव॑रुद्ध्या॒ अव॑रुद्ध्यै॒ वास॑सा । \newline
31. अव॑रुद्ध्या॒ इत्यव॑ - रु॒द्ध्यै॒ । \newline
32. वास॑सा क्रीणाति क्रीणाति॒ वास॑सा॒ वास॑सा क्रीणाति । \newline
33. क्री॒णा॒ति॒ स॒र्व॒दे॒व॒त्यꣳ॑ सर्वदेव॒त्य॑म् क्रीणाति क्रीणाति सर्वदेव॒त्य᳚म् । \newline
34. स॒र्व॒दे॒व॒त्यं॑ ॅवै वै स॑र्वदेव॒त्यꣳ॑ सर्वदेव॒त्यं॑ ॅवै । \newline
35. स॒र्व॒दे॒व॒त्य॑मिति॑ सर्व - दे॒व॒त्य᳚म् । \newline
36. वै वासो॒ वासो॒ वै वै वासः॑ । \newline
37. वासः॒ सर्वा᳚भ्यः॒ सर्वा᳚भ्यो॒ वासो॒ वासः॒ सर्वा᳚भ्यः । \newline
38. सर्वा᳚भ्य ए॒वैव सर्वा᳚भ्यः॒ सर्वा᳚भ्य ए॒व । \newline
39. ए॒वैन॑ मेन मे॒वै वैन᳚म् । \newline
40. ए॒न॒म् दे॒वता᳚भ्यो दे॒वता᳚भ्य एन मेनम् दे॒वता᳚भ्यः । \newline
41. दे॒वता᳚भ्यः क्रीणाति क्रीणाति दे॒वता᳚भ्यो दे॒वता᳚भ्यः क्रीणाति । \newline
42. क्री॒णा॒ति॒ दश॒ दश॑ क्रीणाति क्रीणाति॒ दश॑ । \newline
43. दश॒ सꣳ सम् दश॒ दश॒ सम् । \newline
44. सम् प॑द्यन्ते पद्यन्ते॒ सꣳ सम् प॑द्यन्ते । \newline
45. प॒द्य॒न्ते॒ दशा᳚क्षरा॒ दशा᳚क्षरा पद्यन्ते पद्यन्ते॒ दशा᳚क्षरा । \newline
46. दशा᳚क्षरा वि॒राड् वि॒राड् दशा᳚क्षरा॒ दशा᳚क्षरा वि॒राट् । \newline
47. दशा᳚क्ष॒रेति॒ दश॑ - अ॒क्ष॒रा॒ । \newline
48. वि॒रा डन्न॒ मन्नं॑ ॅवि॒राड् वि॒रा डन्न᳚म् । \newline
49. वि॒राडिति॑ वि - राट् । \newline
50. अन्नं॑ ॅवि॒राड् वि॒रा डन्न॒ मन्नं॑ ॅवि॒राट् । \newline
51. वि॒राड् वि॒राजा॑ वि॒राजा॑ वि॒राड् वि॒राड् वि॒राजा᳚ । \newline
52. वि॒राडिति॑ वि - राट् । \newline
53. वि॒रा जै॒वैव वि॒राजा॑ वि॒राजै॒व । \newline
54. वि॒राजेति॑ वि - राजा᳚ । \newline
55. ए॒वान्नाद्य॑ म॒न्नाद्य॑ मे॒वै वान्नाद्य᳚म् । \newline
56. अ॒न्नाद्य॒ मवावा॒ न्नाद्य॑ म॒न्नाद्य॒ मव॑ । \newline
57. अ॒न्नाद्य॒मित्य॑न्न - अद्य᳚म् । \newline
58. अव॑ रुन्धे रु॒न्धे ऽवाव॑ रुन्धे । \newline
59. रु॒न्धे॒ तप॑स॒ स्तप॑सो रुन्धे रुन्धे॒ तप॑सः । \newline

\textbf{Ghana Paata } \newline

1. ए॒न॒म् क्री॒णा॒ति॒ क्री॒णा॒ त्ये॒न॒ मे॒न॒म् क्री॒णा॒ति॒ धे॒न्वा धे॒न्वा क्री॑णा त्येन मेनम् क्रीणाति धे॒न्वा । \newline
2. क्री॒णा॒ति॒ धे॒न्वा धे॒न्वा क्री॑णाति क्रीणाति धे॒न्वा क्री॑णाति क्रीणाति धे॒न्वा क्री॑णाति क्रीणाति धे॒न्वा क्री॑णाति । \newline
3. धे॒न्वा क्री॑णाति क्रीणाति धे॒न्वा धे॒न्वा क्री॑णाति॒ साशि॑रꣳ॒॒ साशि॑रम् क्रीणाति धे॒न्वा धे॒न्वा क्री॑णाति॒ साशि॑रम् । \newline
4. क्री॒णा॒ति॒ साशि॑रꣳ॒॒ साशि॑रम् क्रीणाति क्रीणाति॒ साशि॑र मे॒वैव साशि॑रम् क्रीणाति क्रीणाति॒ साशि॑र मे॒व । \newline
5. साशि॑र मे॒वैव साशि॑रꣳ॒॒ साशि॑र मे॒वैन॑ मेन मे॒व साशि॑रꣳ॒॒ साशि॑र मे॒वैन᳚म् । \newline
6. साशि॑र॒मिति॒ स - आ॒शि॒र॒म् । \newline
7. ए॒वैन॑ मेन मे॒वै वैन॑म् क्रीणाति क्रीणा त्येन मे॒वै वैन॑म् क्रीणाति । \newline
8. ए॒न॒म् क्री॒णा॒ति॒ क्री॒णा॒ त्ये॒न॒ मे॒न॒म् क्री॒णा॒ त्यृ॒ष॒भेण॑ र्.ष॒भेण॑ क्रीणा त्येन मेनम् क्रीणा त्यृष॒भेण॑ । \newline
9. क्री॒णा॒ त्यृ॒ष॒भेण॑ र्.ष॒भेण॑ क्रीणाति क्रीणा त्यृष॒भेण॑ क्रीणाति क्रीणा त्यृष॒भेण॑ क्रीणाति क्रीणा त्यृष॒भेण॑ क्रीणाति । \newline
10. ऋ॒ष॒भेण॑ क्रीणाति क्रीणा त्यृष॒भेण॑ र्.ष॒भेण॑ क्रीणाति॒ सेन्द्रꣳ॒॒ सेन्द्र॑म् क्रीणा त्यृष॒भेण॑ र्.ष॒भेण॑ क्रीणाति॒ सेन्द्र᳚म् । \newline
11. क्री॒णा॒ति॒ सेन्द्रꣳ॒॒ सेन्द्र॑म् क्रीणाति क्रीणाति॒ सेन्द्र॑ मे॒वैव सेन्द्र॑म् क्रीणाति क्रीणाति॒ सेन्द्र॑ मे॒व । \newline
12. सेन्द्र॑ मे॒वैव सेन्द्रꣳ॒॒ सेन्द्र॑ मे॒वैन॑ मेन मे॒व सेन्द्रꣳ॒॒ सेन्द्र॑ मे॒वैन᳚म् । \newline
13. सेन्द्र॒मिति॒ स - इ॒न्द्र॒म् । \newline
14. ए॒वैन॑ मेन मे॒वै वैन॑म् क्रीणाति क्रीणा त्येन मे॒वै वैन॑म् क्रीणाति । \newline
15. ए॒न॒म् क्री॒णा॒ति॒ क्री॒णा॒ त्ये॒न॒ मे॒न॒म् क्री॒णा॒ त्य॒न॒ डुहा॑ ऽन॒डुहा᳚ क्रीणा त्येन मेनम् क्रीणा त्यन॒ डुहा᳚ । \newline
16. क्री॒णा॒ त्य॒न॒ डुहा॑ ऽन॒डुहा᳚ क्रीणाति क्रीणा त्यन॒ डुहा᳚ क्रीणाति क्रीणा त्यन॒ डुहा᳚ क्रीणाति क्रीणा त्यन॒ डुहा᳚ क्रीणाति । \newline
17. अ॒न॒डुहा᳚ क्रीणाति क्रीणा त्यन॒ डुहा॑ ऽन॒डुहा᳚ क्रीणाति॒ वह्नि॒र् वह्निः॑ क्रीणा त्यन॒ डुहा॑ ऽन॒डुहा᳚ क्रीणाति॒ वह्निः॑ । \newline
18. क्री॒णा॒ति॒ वह्नि॒र् वह्निः॑ क्रीणाति क्रीणाति॒ वह्नि॒र् वै वै वह्निः॑ क्रीणाति क्रीणाति॒ वह्नि॒र् वै । \newline
19. वह्नि॒र् वै वै वह्नि॒र् वह्नि॒र् वा अ॑न॒ड्वा न॑न॒ड्वान्. वै वह्नि॒र् वह्नि॒र् वा अ॑न॒ड्वान् । \newline
20. वा अ॑न॒ड्वा न॑न॒ड्वान्. वै वा अ॑न॒ड्वान्. वह्नि॑ना॒ वह्नि॑ना ऽन॒ड्वान्. वै वा अ॑न॒ड्वान्. वह्नि॑ना । \newline
21. अ॒न॒ड्वान्. वह्नि॑ना॒ वह्नि॑ना ऽन॒ड्वा न॑न॒ड्वान्. वह्नि॑नै॒ वैव वह्नि॑ना ऽन॒ड्वा न॑न॒ड्वान्. वह्नि॑नै॒व । \newline
22. वह्नि॑नै॒ वैव वह्नि॑ना॒ वह्नि॑ नै॒व वह्नि॒ वह्न्ये॒व वह्नि॑ना॒ वह्नि॑ नै॒व वह्नि॑ । \newline
23. ए॒व वह्नि॒ वह्न्ये॒ वैव वह्नि॑ य॒ज्ञ्स्य॑ य॒ज्ञ्स्य॒ वह्न्ये॒ वैव वह्नि॑ य॒ज्ञ्स्य॑ । \newline
24. वह्नि॑ य॒ज्ञ्स्य॑ य॒ज्ञ्स्य॒ वह्नि॒ वह्नि॑ य॒ज्ञ्स्य॑ क्रीणाति क्रीणाति य॒ज्ञ्स्य॒ वह्नि॒ वह्नि॑ य॒ज्ञ्स्य॑ क्रीणाति । \newline
25. य॒ज्ञ्स्य॑ क्रीणाति क्रीणाति य॒ज्ञ्स्य॑ य॒ज्ञ्स्य॑ क्रीणाति मिथु॒नाभ्या᳚म् मिथु॒नाभ्या᳚म् क्रीणाति य॒ज्ञ्स्य॑ य॒ज्ञ्स्य॑ क्रीणाति मिथु॒नाभ्या᳚म् । \newline
26. क्री॒णा॒ति॒ मि॒थु॒नाभ्या᳚म् मिथु॒नाभ्या᳚म् क्रीणाति क्रीणाति मिथु॒नाभ्या᳚म् क्रीणाति क्रीणाति मिथु॒नाभ्या᳚म् क्रीणाति क्रीणाति मिथु॒नाभ्या᳚म् क्रीणाति । \newline
27. मि॒थु॒नाभ्या᳚म् क्रीणाति क्रीणाति मिथु॒नाभ्या᳚म् मिथु॒नाभ्या᳚म् क्रीणाति मिथु॒नस्य॑ मिथु॒नस्य॑ क्रीणाति मिथु॒नाभ्या᳚म् मिथु॒नाभ्या᳚म् क्रीणाति मिथु॒नस्य॑ । \newline
28. क्री॒णा॒ति॒ मि॒थु॒नस्य॑ मिथु॒नस्य॑ क्रीणाति क्रीणाति मिथु॒नस्या व॑रुद्ध्या॒ अव॑रुद्ध्यै मिथु॒नस्य॑ क्रीणाति क्रीणाति मिथु॒नस्या व॑रुद्ध्यै । \newline
29. मि॒थु॒नस्या व॑रुद्ध्या॒ अव॑रुद्ध्यै मिथु॒नस्य॑ मिथु॒नस्या व॑रुद्ध्यै॒ वास॑सा॒ वास॒सा ऽव॑रुद्ध्यै मिथु॒नस्य॑ मिथु॒नस्या व॑रुद्ध्यै॒ वास॑सा । \newline
30. अव॑रुद्ध्यै॒ वास॑सा॒ वास॒सा ऽव॑रुद्ध्या॒ अव॑रुद्ध्यै॒ वास॑सा क्रीणाति क्रीणाति॒ वास॒सा ऽव॑रुद्ध्या॒ अव॑रुद्ध्यै॒ वास॑सा क्रीणाति । \newline
31. अव॑रुद्ध्या॒ इत्यव॑ - रु॒द्ध्यै॒ । \newline
32. वास॑सा क्रीणाति क्रीणाति॒ वास॑सा॒ वास॑सा क्रीणाति सर्वदेव॒त्यꣳ॑ सर्वदेव॒त्य॑म् क्रीणाति॒ वास॑सा॒ वास॑सा क्रीणाति सर्वदेव॒त्य᳚म् । \newline
33. क्री॒णा॒ति॒ स॒र्व॒दे॒व॒त्यꣳ॑ सर्वदेव॒त्य॑म् क्रीणाति क्रीणाति सर्वदेव॒त्यं॑ ॅवै वै स॑र्वदेव॒त्य॑म् क्रीणाति क्रीणाति सर्वदेव॒त्यं॑ ॅवै । \newline
34. स॒र्व॒दे॒व॒त्यं॑ ॅवै वै स॑र्वदेव॒त्यꣳ॑ सर्वदेव॒त्यं॑ ॅवै वासो॒ वासो॒ वै स॑र्वदेव॒त्यꣳ॑ सर्वदेव॒त्यं॑ ॅवै वासः॑ । \newline
35. स॒र्व॒दे॒व॒त्य॑मिति॑ सर्व - दे॒व॒त्य᳚म् । \newline
36. वै वासो॒ वासो॒ वै वै वासः॒ सर्वा᳚भ्यः॒ सर्वा᳚भ्यो॒ वासो॒ वै वै वासः॒ सर्वा᳚भ्यः । \newline
37. वासः॒ सर्वा᳚भ्यः॒ सर्वा᳚भ्यो॒ वासो॒ वासः॒ सर्वा᳚भ्य ए॒वैव सर्वा᳚भ्यो॒ वासो॒ वासः॒ सर्वा᳚भ्य ए॒व । \newline
38. सर्वा᳚भ्य ए॒वैव सर्वा᳚भ्यः॒ सर्वा᳚भ्य ए॒वैन॑ मेन मे॒व सर्वा᳚भ्यः॒ सर्वा᳚भ्य ए॒वैन᳚म् । \newline
39. ए॒वैन॑ मेन मे॒वै वैन॑म् दे॒वता᳚भ्यो दे॒वता᳚भ्य एन मे॒वै वैन॑म् दे॒वता᳚भ्यः । \newline
40. ए॒न॒म् दे॒वता᳚भ्यो दे॒वता᳚भ्य एन मेनम् दे॒वता᳚भ्यः क्रीणाति क्रीणाति दे॒वता᳚भ्य एन मेनम् दे॒वता᳚भ्यः क्रीणाति । \newline
41. दे॒वता᳚भ्यः क्रीणाति क्रीणाति दे॒वता᳚भ्यो दे॒वता᳚भ्यः क्रीणाति॒ दश॒ दश॑ क्रीणाति दे॒वता᳚भ्यो दे॒वता᳚भ्यः क्रीणाति॒ दश॑ । \newline
42. क्री॒णा॒ति॒ दश॒ दश॑ क्रीणाति क्रीणाति॒ दश॒ सꣳ सम् दश॑ क्रीणाति क्रीणाति॒ दश॒ सम् । \newline
43. दश॒ सꣳ सम् दश॒ दश॒ सम् प॑द्यन्ते पद्यन्ते॒ सम् दश॒ दश॒ सम् प॑द्यन्ते । \newline
44. सम् प॑द्यन्ते पद्यन्ते॒ सꣳ सम् प॑द्यन्ते॒ दशा᳚क्षरा॒ दशा᳚क्षरा पद्यन्ते॒ सꣳ सम् प॑द्यन्ते॒ दशा᳚क्षरा । \newline
45. प॒द्य॒न्ते॒ दशा᳚क्षरा॒ दशा᳚क्षरा पद्यन्ते पद्यन्ते॒ दशा᳚क्षरा वि॒राड् वि॒राड् दशा᳚क्षरा पद्यन्ते पद्यन्ते॒ दशा᳚क्षरा वि॒राट् । \newline
46. दशा᳚क्षरा वि॒राड् वि॒राड् दशा᳚क्षरा॒ दशा᳚क्षरा वि॒राडन्न॒ मन्नं॑ ॅवि॒राड् दशा᳚क्षरा॒ दशा᳚क्षरा वि॒राडन्न᳚म् । \newline
47. दशा᳚क्ष॒रेति॒ दश॑ - अ॒क्ष॒रा॒ । \newline
48. वि॒राडन्न॒ मन्नं॑ ॅवि॒राड् वि॒राडन्नं॑ ॅवि॒राड् वि॒राडन्नं॑ ॅवि॒राड् वि॒राडन्नं॑ ॅवि॒राट् । \newline
49. वि॒राडिति॑ वि - राट् । \newline
50. अन्नं॑ ॅवि॒राड् वि॒राडन्न॒ मन्नं॑ ॅवि॒राड् वि॒राजा॑ वि॒राजा॑ वि॒राडन्न॒ मन्नं॑ ॅवि॒राड् वि॒राजा᳚ । \newline
51. वि॒राड् वि॒राजा॑ वि॒राजा॑ वि॒राड् वि॒राड् वि॒राजै॒वैव वि॒राजा॑ वि॒राड् वि॒राड् वि॒राजै॒व । \newline
52. वि॒राडिति॑ वि - राट् । \newline
53. वि॒रा जै॒वैव वि॒राजा॑ वि॒रा जै॒वान्नाद्य॑ म॒न्नाद्य॑ मे॒व वि॒राजा॑ वि॒रा जै॒वान्नाद्य᳚म् । \newline
54. वि॒राजेति॑ वि - राजा᳚ । \newline
55. ए॒वान्नाद्य॑ म॒न्नाद्य॑ मे॒वै वान्नाद्य॒ मवावा॒न्नाद्य॑ मे॒वै वान्नाद्य॒ मव॑ । \newline
56. अ॒न्नाद्य॒ मवा वा॒न्नाद्य॑ म॒न्नाद्य॒ मव॑ रुन्धे रु॒न्धे ऽवा॒न्नाद्य॑ म॒न्नाद्य॒ मव॑ रुन्धे । \newline
57. अ॒न्नाद्य॒मित्य॑न्न - अद्य᳚म् । \newline
58. अव॑ रुन्धे रु॒न्धे ऽवाव॑ रुन्धे॒ तप॑स॒ स्तप॑सो रु॒न्धे ऽवाव॑ रुन्धे॒ तप॑सः । \newline
59. रु॒न्धे॒ तप॑स॒ स्तप॑सो रुन्धे रुन्धे॒ तप॑स स्त॒नू स्त॒नू स्तप॑सो रुन्धे रुन्धे॒ तप॑स स्त॒नूः । \newline
\pagebreak
\markright{ TS 6.1.10.3  \hfill https://www.vedavms.in \hfill}

\section{ TS 6.1.10.3 }

\textbf{TS 6.1.10.3 } \newline
\textbf{Samhita Paata} \newline

तप॑सस्त॒नूर॑सि प्र॒जाप॑ते॒र्वर्ण॒ इत्या॑ह प॒शुभ्य॑ ए॒व तद॑द्ध्व॒र्युर्नि॑ ह्नु॑त आ॒त्मनोऽना᳚व्रस्काय॒ गच्छ॑ति॒ श्रियं॒ प्र प॒शूना᳚प्नोति॒ य ए॒वं ॅवेद॑ शु॒क्रं ते॑ शु॒क्रेण॑ क्रीणा॒मीत्या॑ह यथा य॒जुरे॒वैतद् दे॒वा वै येन॒ हिर॑ण्येन॒ सोम॒मक्री॑ण॒न् तद॑भी॒षहा॒ पुन॒राऽद॑दत॒ को हि तेज॑सा विक्रे॒ष्यत॒ इति॒ येन॒ हिर॑ण्येन॒ - [  ] \newline

\textbf{Pada Paata} \newline

तपसः॑ । त॒नूः । अ॒सि॒ । प्र॒जाप॑ते॒रिति॑ प्र॒जा - प॒तेः॒ । वर्णः॑ । इति॑ । आ॒ह॒ । प॒शुभ्य॒ इति॑ प॒शु - भ्यः॒ । ए॒व । तत् । अ॒द्ध्व॒र्युः । नीति॑ । ह्नु॒ते॒ । आ॒त्मनः॑ । अना᳚व्रस्का॒येत्यना᳚ - व्र॒स्का॒य॒ । गच्छ॑ति । श्रिय᳚म् । प्रेति॑ । प॒शून् । आ॒प्नो॒ति॒ । यः । ए॒वम् । वेद॑ । शु॒क्रम् । ते॒ । शु॒क्रेण॑ । क्री॒णा॒मि॒ । इति॑ । आ॒ह॒ । य॒था॒य॒जुरिति॑ यथा-यजुः । ए॒व । ए॒तत् । दे॒वाः । वै । येन॑ । हिर॑ण्येन । सोम᳚म् । अक्री॑णन्न् । तत् । अ॒भी॒षहेत्य॑भि-सहा᳚ । पुनः॑ । एति॑ । अ॒द॒द॒त॒ । कः । हि । तेज॑सा । वि॒क्रे॒ष्यत॒ इति॑ वि - क्रे॒ष्यते᳚ । इति॑ । येन॑ । हिर॑ण्येन ।  \newline


\textbf{Krama Paata} \newline

तप॑सस्त॒नूः । त॒नूर॑सि । अ॒सि॒ प्र॒जाप॑तेः । प्र॒जाप॑ते॒र् वर्णः॑ । प्र॒जाप॑ते॒रिति॑ प्र॒जा - प॒तेः॒ । वर्ण॒ इति॑ । इत्या॑ह । आ॒ह॒ प॒शुभ्यः॑ । प॒शुभ्य॑ ए॒व । प॒शुभ्य॒ इति॑ प॒शु - भ्यः॒ । ए॒व तत् । तद॑द्ध्व॒र्युः । अ॒द्ध्व॒र्युर् नि । नि ह्नु॑ते । ह्नु॒त॒ आ॒त्मनः॑ । आ॒त्मनोऽना᳚व्रस्काय । अना᳚व्रस्काय॒ गच्छ॑ति । अना᳚व्रस्का॒येत्यना᳚ - व्र॒स्का॒य॒ । गच्छ॑ति॒ श्रिय᳚म् । श्रिय॒म् प्र । प्र प॒शून् । प॒शूना᳚प्नोति । आ॒प्नो॒ति॒ यः । य ए॒वम् । ए॒वम् ॅवेद॑ । वेद॑ शु॒क्रम् । शु॒क्रम् ते᳚ । ते॒ शु॒क्रेण॑ । शु॒क्रेण॑ क्रीणामि । क्री॒णा॒मीति॑ । इत्या॑ह । आ॒ह॒ य॒था॒य॒जुः । य॒था॒य॒जुरे॒व । य॒था॒य॒जुरिति॑ यथा - य॒जुः । ए॒वैतत् । ए॒तद् दे॒वाः । दे॒वा वै । वै येन॑ । येन॒ हिर॑ण्येन । हिर॑ण्येन॒ सोम᳚म् । सोम॒मक्री॑णन्न् । अक्री॑ण॒न् तत् । तद॑भी॒षहा᳚ । अ॒भी॒षहा॒ पुनः॑ । अ॒भी॒षहेत्य॑भि - सहा᳚ । पुन॒रा । आऽद॑दत । अ॒द॒द॒त॒ कः । को हि । हि तेज॑सा । तेज॑सा विक्रे॒ष्यते᳚ । वि॒क्रे॒ष्यत॒ इति॑ । वि॒क्रे॒ष्यत॒ इति॑ वि - क्रे॒ष्यते᳚ । इति॒ येन॑ । येन॒ हिर॑ण्येन । हिर॑ण्येन॒ सोम᳚म् \newline

\textbf{Jatai Paata} \newline

1. तप॑स स्त॒नू स्त॒नू स्तप॑स॒ स्तप॑स स्त॒नूः । \newline
2. त॒नू र॑स्यसि त॒नू स्त॒नू र॑सि । \newline
3. अ॒सि॒ प्र॒जाप॑तेः प्र॒जाप॑ते रस्यसि प्र॒जाप॑तेः । \newline
4. प्र॒जाप॑ते॒र् वर्णो॒ वर्णः॑ प्र॒जाप॑तेः प्र॒जाप॑ते॒र् वर्णः॑ । \newline
5. प्र॒जाप॑ते॒रिति॑ प्र॒जा - प॒तेः॒ । \newline
6. वर्ण॒ इतीति॒ वर्णो॒ वर्ण॒ इति॑ । \newline
7. इत्या॑हा॒हे तीत्या॑ह । \newline
8. आ॒ह॒ प॒शुभ्यः॑ प॒शुभ्य॑ आहाह प॒शुभ्यः॑ । \newline
9. प॒शुभ्य॑ ए॒वैव प॒शुभ्यः॑ प॒शुभ्य॑ ए॒व । \newline
10. प॒शुभ्य॒ इति॑ प॒शु - भ्यः॒ । \newline
11. ए॒व तत् तदे॒ वैव तत् । \newline
12. तद॑द्ध्व॒र्यु र॑द्ध्व॒र्यु स्तत् तद॑द्ध्व॒र्युः । \newline
13. अ॒द्ध्व॒र्युर् नि न्य॑द्ध्व॒र्यु र॑द्ध्व॒र्युर् नि । \newline
14. नि ह्नु॑ते ह्नुते॒ नि नि ह्नु॑ते । \newline
15. ह्नु॒त॒ आ॒त्मन॑ आ॒त्मनो᳚ ह्नुते ह्नुत आ॒त्मनः॑ । \newline
16. आ॒त्मनो ऽना᳚व्रस्का॒या ना᳚व्रस्काया॒ त्मन॑ आ॒त्मनो ऽना᳚व्रस्काय । \newline
17. अना᳚व्रस्काय॒ गच्छ॑ति॒ गच्छ॒ त्यना᳚व्रस्का॒या ना᳚व्रस्काय॒ गच्छ॑ति । \newline
18. अना᳚व्रस्का॒येत्यना᳚ - व्र॒स्का॒य॒ । \newline
19. गच्छ॑ति॒ श्रियꣳ॒॒ श्रिय॒म् गच्छ॑ति॒ गच्छ॑ति॒ श्रिय᳚म् । \newline
20. श्रिय॒म् प्र प्र श्रियꣳ॒॒ श्रिय॒म् प्र । \newline
21. प्र प॒शून् प॒शून् प्र प्र प॒शून् । \newline
22. प॒शूना᳚प्नो त्याप्नोति प॒शून् प॒शूना᳚प्नोति । \newline
23. आ॒प्नो॒ति॒ यो य आ᳚प्नो त्याप्नोति॒ यः । \newline
24. य ए॒व मे॒वं ॅयो य ए॒वम् । \newline
25. ए॒वं ॅवेद॒ वेदै॒व मे॒वं ॅवेद॑ । \newline
26. वेद॑ शु॒क्रꣳ शु॒क्रं ॅवेद॒ वेद॑ शु॒क्रम् । \newline
27. शु॒क्रम् ते॑ ते शु॒क्रꣳ शु॒क्रम् ते᳚ । \newline
28. ते॒ शु॒क्रेण॑ शु॒क्रेण॑ ते ते शु॒क्रेण॑ । \newline
29. शु॒क्रेण॑ क्रीणामि क्रीणामि शु॒क्रेण॑ शु॒क्रेण॑ क्रीणामि । \newline
30. क्री॒णा॒मी तीति॑ क्रीणामि क्रीणा॒मीति॑ । \newline
31. इत्या॑हा॒हे तीत्या॑ह । \newline
32. आ॒ह॒ य॒था॒य॒जुर् य॑थाय॒जु रा॑हाह यथाय॒जुः । \newline
33. य॒था॒य॒जु रे॒वैव य॑थाय॒जुर् य॑थाय॒जु रे॒व । \newline
34. य॒था॒य॒जुरिति॑ यथा - य॒जुः । \newline
35. ए॒वैत दे॒त दे॒वै वैतत् । \newline
36. ए॒तद् दे॒वा दे॒वा ए॒त दे॒तद् दे॒वाः । \newline
37. दे॒वा वै वै दे॒वा दे॒वा वै । \newline
38. वै येन॒ येन॒ वै वै येन॑ । \newline
39. येन॒ हिर॑ण्येन॒ हिर॑ण्येन॒ येन॒ येन॒ हिर॑ण्येन । \newline
40. हिर॑ण्येन॒ सोमꣳ॒॒ सोमꣳ॒॒ हिर॑ण्येन॒ हिर॑ण्येन॒ सोम᳚म् । \newline
41. सोम॒ मक्री॑ण॒न् नक्री॑ण॒न् थ्सोमꣳ॒॒ सोम॒ मक्री॑णन्न् । \newline
42. अक्री॑ण॒न् तत् तदक्री॑ण॒न् नक्री॑ण॒न् तत् । \newline
43. तद॑भी॒षहा॑ ऽभी॒षहा॒ तत् तद॑भी॒षहा᳚ । \newline
44. अ॒भी॒षहा॒ पुनः॒ पुन॑ रभी॒षहा॑ ऽभी॒षहा॒ पुनः॑ । \newline
45. अ॒भी॒षहेत्य॑भि - सहा᳚ । \newline
46. पुन॒ रा पुनः॒ पुन॒ रा । \newline
47. आ ऽद॑दता दद॒ता ऽद॑दत । \newline
48. अ॒द॒द॒त॒ कः को॑ ऽददता ददत॒ कः । \newline
49. को हि हि कः को हि । \newline
50. हि तेज॑सा॒ तेज॑सा॒ हि हि तेज॑सा । \newline
51. तेज॑सा विक्रे॒ष्यते॑ विक्रे॒ष्यते॒ तेज॑सा॒ तेज॑सा विक्रे॒ष्यते᳚ । \newline
52. वि॒क्रे॒ष्यत॒ इतीति॑ विक्रे॒ष्यते॑ विक्रे॒ष्यत॒ इति॑ । \newline
53. वि॒क्रे॒ष्यत॒ इति॑ वि - क्रे॒ष्यते᳚ । \newline
54. इति॒ येन॒ येने तीति॒ येन॑ । \newline
55. येन॒ हिर॑ण्येन॒ हिर॑ण्येन॒ येन॒ येन॒ हिर॑ण्येन । \newline
56. हिर॑ण्येन॒ सोमꣳ॒॒ सोमꣳ॒॒ हिर॑ण्येन॒ हिर॑ण्येन॒ सोम᳚म् । \newline

\textbf{Ghana Paata } \newline

1. तप॑स स्त॒नू स्त॒नू स्तप॑स॒ स्तप॑स स्त॒नू र॑स्यसि त॒नू स्तप॑स॒ स्तप॑स स्त॒नू र॑सि । \newline
2. त॒नू र॑स्यसि त॒नू स्त॒नू र॑सि प्र॒जाप॑तेः प्र॒जाप॑ते रसि त॒नू स्त॒नू र॑सि प्र॒जाप॑तेः । \newline
3. अ॒सि॒ प्र॒जाप॑तेः प्र॒जाप॑ते रस्यसि प्र॒जाप॑ते॒र् वर्णो॒ वर्णः॑ प्र॒जाप॑ते रस्यसि प्र॒जाप॑ते॒र् वर्णः॑ । \newline
4. प्र॒जाप॑ते॒र् वर्णो॒ वर्णः॑ प्र॒जाप॑तेः प्र॒जाप॑ते॒र् वर्ण॒ इतीति॒ वर्णः॑ प्र॒जाप॑तेः प्र॒जाप॑ते॒र् वर्ण॒ इति॑ । \newline
5. प्र॒जाप॑ते॒रिति॑ प्र॒जा - प॒तेः॒ । \newline
6. वर्ण॒ इतीति॒ वर्णो॒ वर्ण॒ इत्या॑हा॒हेति॒ वर्णो॒ वर्ण॒ इत्या॑ह । \newline
7. इत्या॑हा॒हे तीत्या॑ह प॒शुभ्यः॑ प॒शुभ्य॑ आ॒हे तीत्या॑ह प॒शुभ्यः॑ । \newline
8. आ॒ह॒ प॒शुभ्यः॑ प॒शुभ्य॑ आहाह प॒शुभ्य॑ ए॒वैव प॒शुभ्य॑ आहाह प॒शुभ्य॑ ए॒व । \newline
9. प॒शुभ्य॑ ए॒वैव प॒शुभ्यः॑ प॒शुभ्य॑ ए॒व तत् तदे॒व प॒शुभ्यः॑ प॒शुभ्य॑ ए॒व तत् । \newline
10. प॒शुभ्य॒ इति॑ प॒शु - भ्यः॒ । \newline
11. ए॒व तत् तदे॒ वैव तद॑द्ध्व॒र्यु र॑द्ध्व॒र्यु स्तदे॒वैव तद॑द्ध्व॒र्युः । \newline
12. तद॑द्ध्व॒र्यु र॑द्ध्व॒र्यु स्तत् तद॑द्ध्व॒र्युर् नि न्य॑द्ध्व॒र्यु स्तत् तद॑द्ध्व॒र्युर् नि । \newline
13. अ॒द्ध्व॒र्युर् नि न्य॑द्ध्व॒र्यु र॑द्ध्व॒र्युर् नि ह्नु॑ते ह्नुते॒ न्य॑द्ध्व॒र्यु र॑द्ध्व॒र्युर् नि ह्नु॑ते । \newline
14. नि ह्नु॑ते ह्नुते॒ नि नि ह्नु॑त आ॒त्मन॑ आ॒त्मनो᳚ ह्नुते॒ नि नि ह्नु॑त आ॒त्मनः॑ । \newline
15. ह्नु॒त॒ आ॒त्मन॑ आ॒त्मनो᳚ ह्नुते ह्नुत आ॒त्मनो ऽना᳚व्रस्का॒या ना᳚व्रस्काया॒ त्मनो᳚ ह्नुते ह्नुत आ॒त्मनो ऽना᳚व्रस्काय । \newline
16. आ॒त्मनो ऽना᳚व्रस्का॒या ना᳚व्रस्का या॒त्मन॑ आ॒त्मनो ऽना᳚व्रस्काय॒ गच्छ॑ति॒ गच्छ॒ त्यना᳚व्रस्का या॒त्मन॑ आ॒त्मनो ऽना᳚व्रस्काय॒ गच्छ॑ति । \newline
17. अना᳚व्रस्काय॒ गच्छ॑ति॒ गच्छ॒ त्यना᳚व्रस्का॒या ना᳚व्रस्काय॒ गच्छ॑ति॒ श्रियꣳ॒॒ श्रिय॒म् गच्छ॒ त्यना᳚व्रस्का॒या ना᳚व्रस्काय॒ गच्छ॑ति॒ श्रिय᳚म् । \newline
18. अना᳚व्रस्का॒येत्यना᳚ - व्र॒स्का॒य॒ । \newline
19. गच्छ॑ति॒ श्रियꣳ॒॒ श्रिय॒म् गच्छ॑ति॒ गच्छ॑ति॒ श्रिय॒म् प्र प्र श्रिय॒म् गच्छ॑ति॒ गच्छ॑ति॒ श्रिय॒म् प्र । \newline
20. श्रिय॒म् प्र प्र श्रियꣳ॒॒ श्रिय॒म् प्र प॒शून् प॒शून् प्र श्रियꣳ॒॒ श्रिय॒म् प्र प॒शून् । \newline
21. प्र प॒शून् प॒शून् प्र प्र प॒शू ना᳚प्नो त्याप्नोति प॒शून् प्र प्र प॒शू ना᳚प्नोति । \newline
22. प॒शू ना᳚प्नो त्याप्नोति प॒शून् प॒शू ना᳚प्नोति॒ यो य आ᳚प्नोति प॒शून् प॒शू ना᳚प्नोति॒ यः । \newline
23. आ॒प्नो॒ति॒ यो य आ᳚प्नो त्याप्नोति॒ य ए॒व मे॒वं ॅय आ᳚प्नो त्याप्नोति॒ य ए॒वम् । \newline
24. य ए॒व मे॒वं ॅयो य ए॒वं ॅवेद॒ वेदै॒वं ॅयो य ए॒वं ॅवेद॑ । \newline
25. ए॒वं ॅवेद॒ वेदै॒व मे॒वं ॅवेद॑ शु॒क्रꣳ शु॒क्रं ॅवेदै॒व मे॒वं ॅवेद॑ शु॒क्रम् । \newline
26. वेद॑ शु॒क्रꣳ शु॒क्रं ॅवेद॒ वेद॑ शु॒क्रम् ते॑ ते शु॒क्रं ॅवेद॒ वेद॑ शु॒क्रम् ते᳚ । \newline
27. शु॒क्रम् ते॑ ते शु॒क्रꣳ शु॒क्रम् ते॑ शु॒क्रेण॑ शु॒क्रेण॑ ते शु॒क्रꣳ शु॒क्रम् ते॑ शु॒क्रेण॑ । \newline
28. ते॒ शु॒क्रेण॑ शु॒क्रेण॑ ते ते शु॒क्रेण॑ क्रीणामि क्रीणामि शु॒क्रेण॑ ते ते शु॒क्रेण॑ क्रीणामि । \newline
29. शु॒क्रेण॑ क्रीणामि क्रीणामि शु॒क्रेण॑ शु॒क्रेण॑ क्रीणा॒मी तीति॑ क्रीणामि शु॒क्रेण॑ शु॒क्रेण॑ क्रीणा॒मीति॑ । \newline
30. क्री॒णा॒मी तीति॑ क्रीणामि क्रीणा॒मी त्या॑हा॒हेति॑ क्रीणामि क्रीणा॒मी त्या॑ह । \newline
31. इत्या॑हा॒हे तीत्या॑ह यथाय॒जुर् य॑थाय॒जु रा॒हे तीत्या॑ह यथाय॒जुः । \newline
32. आ॒ह॒ य॒था॒य॒जुर् य॑थाय॒जु रा॑हाह यथाय॒जु रे॒वैव य॑थाय॒जु रा॑हाह यथाय॒जु रे॒व । \newline
33. य॒था॒य॒जु रे॒वैव य॑थाय॒जुर् य॑थाय॒जु रे॒वैत दे॒त दे॒व य॑थाय॒जुर् य॑थाय॒जु रे॒वैतत् । \newline
34. य॒था॒य॒जुरिति॑ यथा - य॒जुः । \newline
35. ए॒वैत दे॒त दे॒वै वैतद् दे॒वा दे॒वा ए॒त दे॒वै वैतद् दे॒वाः । \newline
36. ए॒तद् दे॒वा दे॒वा ए॒त दे॒तद् दे॒वा वै वै दे॒वा ए॒त दे॒तद् दे॒वा वै । \newline
37. दे॒वा वै वै दे॒वा दे॒वा वै येन॒ येन॒ वै दे॒वा दे॒वा वै येन॑ । \newline
38. वै येन॒ येन॒ वै वै येन॒ हिर॑ण्येन॒ हिर॑ण्येन॒ येन॒ वै वै येन॒ हिर॑ण्येन । \newline
39. येन॒ हिर॑ण्येन॒ हिर॑ण्येन॒ येन॒ येन॒ हिर॑ण्येन॒ सोमꣳ॒॒ सोमꣳ॒॒ हिर॑ण्येन॒ येन॒ येन॒ हिर॑ण्येन॒ सोम᳚म् । \newline
40. हिर॑ण्येन॒ सोमꣳ॒॒ सोमꣳ॒॒ हिर॑ण्येन॒ हिर॑ण्येन॒ सोम॒ मक्री॑ण॒न् नक्री॑ण॒न् थ्सोमꣳ॒॒ हिर॑ण्येन॒ हिर॑ण्येन॒ सोम॒ मक्री॑णन्न् । \newline
41. सोम॒ मक्री॑ण॒न् नक्री॑ण॒न् थ्सोमꣳ॒॒ सोम॒ मक्री॑ण॒न् तत् तदक्री॑ण॒न् थ्सोमꣳ॒॒ सोम॒ मक्री॑ण॒न् तत् । \newline
42. अक्री॑ण॒न् तत् तदक्री॑ण॒न् नक्री॑ण॒न् तद॑भी॒षहा॑ ऽभी॒षहा॒ तदक्री॑ण॒न् नक्री॑ण॒न् तद॑भी॒षहा᳚ । \newline
43. तद॑भी॒षहा॑ ऽभी॒षहा॒ तत् तद॑भी॒षहा॒ पुनः॒ पुन॑ रभी॒षहा॒ तत् तद॑भी॒षहा॒ पुनः॑ । \newline
44. अ॒भी॒षहा॒ पुनः॒ पुन॑ रभी॒षहा॑ ऽभी॒षहा॒ पुन॒रा पुन॑ रभी॒षहा॑ ऽभी॒षहा॒ पुन॒रा । \newline
45. अ॒भी॒षहेत्य॑भि - सहा᳚ । \newline
46. पुन॒रा पुनः॒ पुन॒रा ऽद॑दता दद॒ता पुनः॒ पुन॒रा ऽद॑दत । \newline
47. आ ऽद॑दता दद॒ता ऽद॑दत॒ कः को॑ ऽदद॒ता ऽद॑दत॒ कः । \newline
48. अ॒द॒द॒त॒ कः को॑ ऽददता ददत॒ को हि हि को॑ ऽददता ददत॒ को हि । \newline
49. को हि हि कः को हि तेज॑सा॒ तेज॑सा॒ हि कः को हि तेज॑सा । \newline
50. हि तेज॑सा॒ तेज॑सा॒ हि हि तेज॑सा विक्रे॒ष्यते॑ विक्रे॒ष्यते॒ तेज॑सा॒ हि हि तेज॑सा विक्रे॒ष्यते᳚ । \newline
51. तेज॑सा विक्रे॒ष्यते॑ विक्रे॒ष्यते॒ तेज॑सा॒ तेज॑सा विक्रे॒ष्यत॒ इतीति॑ विक्रे॒ष्यते॒ तेज॑सा॒ तेज॑सा विक्रे॒ष्यत॒ इति॑ । \newline
52. वि॒क्रे॒ष्यत॒ इतीति॑ विक्रे॒ष्यते॑ विक्रे॒ष्यत॒ इति॒ येन॒ येनेति॑ विक्रे॒ष्यते॑ विक्रे॒ष्यत॒ इति॒ येन॑ । \newline
53. वि॒क्रे॒ष्यत॒ इति॑ वि - क्रे॒ष्यते᳚ । \newline
54. इति॒ येन॒ येने तीति॒ येन॒ हिर॑ण्येन॒ हिर॑ण्येन॒ येने तीति॒ येन॒ हिर॑ण्येन । \newline
55. येन॒ हिर॑ण्येन॒ हिर॑ण्येन॒ येन॒ येन॒ हिर॑ण्येन॒ सोमꣳ॒॒ सोमꣳ॒॒ हिर॑ण्येन॒ येन॒ येन॒ हिर॑ण्येन॒ सोम᳚म् । \newline
56. हिर॑ण्येन॒ सोमꣳ॒॒ सोमꣳ॒॒ हिर॑ण्येन॒ हिर॑ण्येन॒ सोम॑म् क्रीणी॒यात् क्री॑णी॒याथ् सोमꣳ॒॒ हिर॑ण्येन॒ हिर॑ण्येन॒ सोम॑म् क्रीणी॒यात् । \newline
\pagebreak
\markright{ TS 6.1.10.4  \hfill https://www.vedavms.in \hfill}

\section{ TS 6.1.10.4 }

\textbf{TS 6.1.10.4 } \newline
\textbf{Samhita Paata} \newline

सोमं॑ क्रीणी॒यात् तद॑भी॒षहा॒ पुन॒रा द॑दीत॒ तेज॑ ए॒वाऽऽत्मन् ध॑त्ते॒ऽस्मे ज्योतिः॑ सोमविक्र॒यिणि॒ तम॒ इत्या॑ह॒ ज्योति॑रे॒व यज॑माने दधाति॒ तम॑सा सोमविक्र॒यिण॑मर्पयति॒ यदनु॑पग्रथ्य ह॒न्याद्-द॑न्द॒शूका॒स्ताꣳ समाꣳ॑ स॒र्पाः स्यु॑रि॒दम॒हꣳ स॒र्पाणां᳚ दन्द॒शूका॑नां ग्री॒वा उप॑ ग्रथ्ना॒मीत्या॒हा-द॑न्दशूका॒स्ताꣳ समाꣳ॑ स॒र्पा भ॑वन्ति॒ तम॑सा सोमविक्र॒यिणं॑ ॅविद्ध्यति॒ स्वान॒ - [  ] \newline

\textbf{Pada Paata} \newline

सोम᳚म् । क्री॒णी॒यात् । तत् । अ॒भी॒षहेत्य॑भ - सहा᳚ । पुनः॑ । एति॑ । द॒दी॒त॒ । तेजः॑ । ए॒व । आ॒त्मन्न् । ध॒त्ते॒ । अ॒स्मे इति॑ । ज्योतिः॑ । सो॒म॒वि॒क्र॒यिणीति॑ सोम-वि॒क्र॒यिणि॑ । तमः॑ । इति॑ । आ॒ह॒ । ज्योतिः॑ । ए॒व । यज॑माने । द॒धा॒ति॒ । तम॑सा । सो॒म॒वि॒क्र॒यिण॒मिति॑ सोम - वि॒क्र॒यिण᳚म् । अ॒र्प॒य॒ति॒ । यत् । अनु॑पग्र॒थ्येत्यनु॑प - ग्र॒थ्य॒ । ह॒न्यात् । द॒न्द॒शूकाः᳚ । ताम् । समा᳚म् । स॒र्पाः । स्युः॒ । इ॒दम् । अ॒हम् । स॒र्पाणा᳚म् । द॒न्द॒शूका॑नाम् । ग्री॒वाः । उपेति॑ । ग्र॒थ्ना॒मि॒ । इति॑ । आ॒ह॒ । अद॑न्दशूकाः । ताम् । समा᳚म् । स॒र्पाः । भ॒व॒न्ति॒ । तम॑सा । सो॒म॒वि॒क्र॒यिण॒मिति॑ सोम - वि॒क्र॒यिण᳚म् । वि॒द्ध्य॒ति॒ । स्वान॑ ।  \newline


\textbf{Krama Paata} \newline

सोम॑म् क्रीणी॒यात् । क्री॒णी॒यात् तत् । तद॑भी॒षहा᳚ । अ॒भी॒षहा॒ पुनः॑ । अ॒भी॒षहेत्य॑भि - सहा᳚ । पुन॒रा । आ द॑दीत । द॒दी॒त॒ तेजः॑ । तेज॑ ए॒व । ए॒वात्मन्न् । आ॒त्मन् ध॑त्ते । ध॒त्ते॒ऽस्मे । अ॒स्मे ज्योतिः॑ । अ॒स्मे इत्य॒स्मे । ज्योतिः॑ सोमविक्र॒यिणि॑ । सो॒म॒वि॒क्र॒यिणि॒ तमः॑ । सो॒म॒वि॒क्र॒यिणीति॑ सोम - वि॒क्र॒यिणि॑ । तम॒ इति॑ । इत्या॑ह । आ॒ह॒ ज्योतिः॑ । ज्योति॑रे॒व । ए॒व यज॑माने । यज॑माने दधाति । द॒धा॒ति॒ तम॑सा । तम॑सा सोमविक्र॒यिण᳚म् । सो॒म॒वि॒क्र॒यिण॑मर्पयति । सो॒म॒वि॒क्र॒यिण॒मिति॑ सोम - वि॒क्र॒यिण᳚म् । अ॒र्प॒य॒ति॒ यत् । यदनु॑पग्रथ्य । अनु॑पग्रथ्य ह॒न्यात् । अनु॑पग्र॒थ्येत्यनु॑प - ग्र॒थ्य॒ । ह॒न्याद् द॑न्द॒शूकाः᳚ । द॒न्द॒शूका॒स्ताम् । ताꣳ समा᳚म् । समाꣳ॑ स॒र्पाः । स॒र्पाः स्युः॑ । स्यु॒रि॒दम् । इ॒दम॒हम् । अ॒हꣳ स॒र्पाणा᳚म् । स॒र्पाणा᳚म् दन्द॒शूका॑नाम् । द॒न्द॒शूका॑नाम् ग्री॒वाः । ग्री॒वा उप॑ । उप॑ ग्रथ्नामि । ग्र॒थ्ना॒मीति॑ । इत्या॑ह । आ॒हाद॑न्दशूकाः । अद॑न्दशूका॒स्ताम् । ताꣳ समा᳚म् । समाꣳ॑ स॒र्पाः । स॒र्पा भ॑वन्ति । भ॒व॒न्ति॒ तम॑सा । तम॑सा सोमविक्र॒यिण᳚म् । सो॒म॒वि॒क्र॒यिण॑म् ॅविद्ध्यति । सो॒म॒वि॒क्र॒यिण॒मिति॑ सोम - वि॒क्र॒यिण᳚म् । वि॒द्ध्य॒ति॒ स्वान॑ ( ) । स्वान॒ भ्राज॑ \newline

\textbf{Jatai Paata} \newline

1. सोम॑म् क्रीणी॒यात् क्री॑णी॒याथ् सोमꣳ॒॒ सोम॑म् क्रीणी॒यात् । \newline
2. क्री॒णी॒यात् तत् तत् क्री॑णी॒यात् क्री॑णी॒यात् तत् । \newline
3. तद॑भी॒षहा॑ ऽभी॒षहा॒ तत् तद॑भी॒षहा᳚ । \newline
4. अ॒भी॒षहा॒ पुनः॒ पुन॑ रभी॒षहा॑ ऽभी॒षहा॒ पुनः॑ । \newline
5. अ॒भी॒षहेत्य॑भि - सहा᳚ । \newline
6. पुन॒ रा पुनः॒ पुन॒ रा । \newline
7. आ द॑दीत ददी॒ता द॑दीत । \newline
8. द॒दी॒त॒ तेज॒ स्तेजो॑ ददीत ददीत॒ तेजः॑ । \newline
9. तेज॑ ए॒वैव तेज॒ स्तेज॑ ए॒व । \newline
10. ए॒वात्मन् ना॒त्मन् ने॒वै वात्मन्न् । \newline
11. आ॒त्मन् ध॑त्ते धत्त आ॒त्मन् ना॒त्मन् ध॑त्ते । \newline
12. ध॒त्ते॒ ऽस्मे अ॒स्मे ध॑त्ते धत्ते॒ ऽस्मे । \newline
13. अ॒स्मे ज्योति॒र् ज्योति॑ र॒स्मे अ॒स्मे ज्योतिः॑ । \newline
14. अ॒स्मे इत्य॒स्मे । \newline
15. ज्योतिः॑ सोमविक्र॒यिणि॑ सोमविक्र॒यिणि॒ ज्योति॒र् ज्योतिः॑ सोमविक्र॒यिणि॑ । \newline
16. सो॒म॒वि॒क्र॒यिणि॒ तम॒ स्तमः॑ सोमविक्र॒यिणि॑ सोमविक्र॒यिणि॒ तमः॑ । \newline
17. सो॒म॒वि॒क्र॒यिणीति॑ सोम - वि॒क्र॒यिणि॑ । \newline
18. तम॒ इतीति॒ तम॒ स्तम॒ इति॑ । \newline
19. इत्या॑हा॒हे तीत्या॑ह । \newline
20. आ॒ह॒ ज्योति॒र् ज्योति॑ राहाह॒ ज्योतिः॑ । \newline
21. ज्योति॑ रे॒वैव ज्योति॒र् ज्योति॑ रे॒व । \newline
22. ए॒व यज॑माने॒ यज॑मान ए॒वैव यज॑माने । \newline
23. यज॑माने दधाति दधाति॒ यज॑माने॒ यज॑माने दधाति । \newline
24. द॒धा॒ति॒ तम॑सा॒ तम॑सा दधाति दधाति॒ तम॑सा । \newline
25. तम॑सा सोमविक्र॒यिणꣳ॑ सोमविक्र॒यिण॒म् तम॑सा॒ तम॑सा सोमविक्र॒यिण᳚म् । \newline
26. सो॒म॒वि॒क्र॒यिण॑ मर्पय त्यर्पयति सोमविक्र॒यिणꣳ॑ सोमविक्र॒यिण॑ मर्पयति । \newline
27. सो॒म॒वि॒क्र॒यिण॒मिति॑ सोम - वि॒क्र॒यिण᳚म् । \newline
28. अ॒र्प॒य॒ति॒ यद् यद॑र्पय त्यर्पयति॒ यत् । \newline
29. यदनु॑पग्र॒थ्या नु॑पग्रथ्य॒ यद् यदनु॑पग्रथ्य । \newline
30. अनु॑पग्रथ्य ह॒न्या द्ध॒न्या दनु॑पग्र॒थ्या नु॑पग्रथ्य ह॒न्यात् । \newline
31. अनु॑पग्र॒थ्येत्यनु॑प - ग्र॒थ्य॒ । \newline
32. ह॒न्याद् द॑न्द॒शूका॑ दन्द॒शूका॑ ह॒न्या द्ध॒न्याद् द॑न्द॒शूकाः᳚ । \newline
33. द॒न्द॒शूका॒ स्ताम् ताम् द॑न्द॒शूका॑ दन्द॒शूका॒ स्ताम् । \newline
34. ताꣳ समाꣳ॒॒ समा॒म् ताम् ताꣳ समा᳚म् । \newline
35. समाꣳ॑ स॒र्पाः स॒र्पाः समाꣳ॒॒ समाꣳ॑ स॒र्पाः । \newline
36. स॒र्पाः स्युः॑ स्युः स॒र्पाः स॒र्पाः स्युः॑ । \newline
37. स्यु॒ रि॒द मि॒दꣳ स्युः॑ स्यु रि॒दम् । \newline
38. इ॒द म॒ह म॒ह मि॒द मि॒द म॒हम् । \newline
39. अ॒हꣳ स॒र्पाणाꣳ॑ स॒र्पाणा॑ म॒ह म॒हꣳ स॒र्पाणा᳚म् । \newline
40. स॒र्पाणा᳚म् दन्द॒शूका॑नाम् दन्द॒शूका॑नाꣳ स॒र्पाणाꣳ॑ स॒र्पाणा᳚म् दन्द॒शूका॑नाम् । \newline
41. द॒न्द॒शूका॑नाम् ग्री॒वा ग्री॒वा द॑न्द॒शूका॑नाम् दन्द॒शूका॑नाम् ग्री॒वाः । \newline
42. ग्री॒वा उपोप॑ ग्री॒वा ग्री॒वा उप॑ । \newline
43. उप॑ ग्रथ्नामि ग्रथ्ना॒ म्युपोप॑ ग्रथ्नामि । \newline
44. ग्र॒थ्ना॒मी तीति॑ ग्रथ्नामि ग्रथ्ना॒ मीति॑ । \newline
45. इत्या॑हा॒हे तीत्या॑ह । \newline
46. आ॒हा द॑न्दशूका॒ अद॑न्दशूका आहा॒हा द॑न्दशूकाः । \newline
47. अद॑न्दशूका॒ स्ताम् ता मद॑न्दशूका॒ अद॑न्दशूका॒ स्ताम् । \newline
48. ताꣳ समाꣳ॒॒ समा॒म् ताम् ताꣳ समा᳚म् । \newline
49. समाꣳ॑ स॒र्पाः स॒र्पाः समाꣳ॒॒ समाꣳ॑ स॒र्पाः । \newline
50. स॒र्पा भ॑वन्ति भवन्ति स॒र्पाः स॒र्पा भ॑वन्ति । \newline
51. भ॒व॒न्ति॒ तम॑सा॒ तम॑सा भवन्ति भवन्ति॒ तम॑सा । \newline
52. तम॑सा सोमविक्र॒यिणꣳ॑ सोमविक्र॒यिण॒म् तम॑सा॒ तम॑सा सोमविक्र॒यिण᳚म् । \newline
53. सो॒म॒वि॒क्र॒यिणं॑ ॅविद्ध्यति विद्ध्यति सोमविक्र॒यिणꣳ॑ सोमविक्र॒यिणं॑ ॅविद्ध्यति । \newline
54. सो॒म॒वि॒क्र॒यिण॒मिति॑ सोम - वि॒क्र॒यिण᳚म् । \newline
55. वि॒द्ध्य॒ति॒ स्वान॒ स्वान॑ विद्ध्यति विद्ध्यति॒ स्वान॑ । \newline
56. स्वान॒ भ्राज॒ भ्राज॒ स्वान॒ स्वान॒ भ्राज॑ । \newline

\textbf{Ghana Paata } \newline

1. सोम॑म् क्रीणी॒यात् क्री॑णी॒याथ् सोमꣳ॒॒ सोम॑म् क्रीणी॒यात् तत् तत् क्री॑णी॒याथ् सोमꣳ॒॒ सोम॑म् क्रीणी॒यात् तत् । \newline
2. क्री॒णी॒यात् तत् तत् क्री॑णी॒यात् क्री॑णी॒यात् तद॑भी॒षहा॑ ऽभी॒षहा॒ तत् क्री॑णी॒यात् क्री॑णी॒यात् तद॑भी॒षहा᳚ । \newline
3. तद॑भी॒षहा॑ ऽभी॒षहा॒ तत् तद॑भी॒षहा॒ पुनः॒ पुन॑ रभी॒षहा॒ तत् तद॑भी॒षहा॒ पुनः॑ । \newline
4. अ॒भी॒षहा॒ पुनः॒ पुन॑ रभी॒षहा॑ ऽभी॒षहा॒ पुन॒रा पुन॑ रभी॒षहा॑ ऽभी॒षहा॒ पुन॒रा । \newline
5. अ॒भी॒षहेत्य॑भि - सहा᳚ । \newline
6. पुन॒रा पुनः॒ पुन॒रा द॑दीत ददी॒ता पुनः॒ पुन॒रा द॑दीत । \newline
7. आ द॑दीत ददी॒ता द॑दीत॒ तेज॒ स्तेजो॑ ददी॒ता द॑दीत॒ तेजः॑ । \newline
8. द॒दी॒त॒ तेज॒ स्तेजो॑ ददीत ददीत॒ तेज॑ ए॒वैव तेजो॑ ददीत ददीत॒ तेज॑ ए॒व । \newline
9. तेज॑ ए॒वैव तेज॒ स्तेज॑ ए॒वात्मन् ना॒त्मन् ने॒व तेज॒ स्तेज॑ ए॒वात्मन्न् । \newline
10. ए॒वात्मन् ना॒त्म न्ने॒वैवात्मन् ध॑त्ते धत्त आ॒त्म न्ने॒वैवात्मन् ध॑त्ते । \newline
11. आ॒त्मन् ध॑त्ते धत्त आ॒त्मन् ना॒त्मन् ध॑त्ते॒ ऽस्मे अ॒स्मे ध॑त्त आ॒त्मन् ना॒त्मन् ध॑त्ते॒ ऽस्मे । \newline
12. ध॒त्ते॒ ऽस्मे अ॒स्मे ध॑त्ते धत्ते॒ ऽस्मे ज्योति॒र् ज्योति॑ र॒स्मे ध॑त्ते धत्ते॒ ऽस्मे ज्योतिः॑ । \newline
13. अ॒स्मे ज्योति॒र् ज्योति॑ र॒स्मे अ॒स्मे ज्योतिः॑ सोमविक्र॒यिणि॑ सोमविक्र॒यिणि॒ ज्योति॑ र॒स्मे अ॒स्मे ज्योतिः॑ सोमविक्र॒यिणि॑ । \newline
14. अ॒स्मे इत्य॒स्मे । \newline
15. ज्योतिः॑ सोमविक्र॒यिणि॑ सोमविक्र॒यिणि॒ ज्योति॒र् ज्योतिः॑ सोमविक्र॒यिणि॒ तम॒ स्तमः॑ सोमविक्र॒यिणि॒ ज्योति॒र् ज्योतिः॑ सोमविक्र॒यिणि॒ तमः॑ । \newline
16. सो॒म॒वि॒क्र॒यिणि॒ तम॒ स्तमः॑ सोमविक्र॒यिणि॑ सोमविक्र॒यिणि॒ तम॒ इतीति॒ तमः॑ सोमविक्र॒यिणि॑ सोमविक्र॒यिणि॒ तम॒ इति॑ । \newline
17. सो॒म॒वि॒क्र॒यिणीति॑ सोम - वि॒क्र॒यिणि॑ । \newline
18. तम॒ इतीति॒ तम॒ स्तम॒ इत्या॑हा॒हेति॒ तम॒ स्तम॒ इत्या॑ह । \newline
19. इत्या॑हा॒हे तीत्या॑ह॒ ज्योति॒र् ज्योति॑ रा॒हे तीत्या॑ह॒ ज्योतिः॑ । \newline
20. आ॒ह॒ ज्योति॒र् ज्योति॑ राहाह॒ ज्योति॑ रे॒वैव ज्योति॑ राहाह॒ ज्योति॑ रे॒व । \newline
21. ज्योति॑ रे॒वैव ज्योति॒र् ज्योति॑ रे॒व यज॑माने॒ यज॑मान ए॒व ज्योति॒र् ज्योति॑ रे॒व यज॑माने । \newline
22. ए॒व यज॑माने॒ यज॑मान ए॒वैव यज॑माने दधाति दधाति॒ यज॑मान ए॒वैव यज॑माने दधाति । \newline
23. यज॑माने दधाति दधाति॒ यज॑माने॒ यज॑माने दधाति॒ तम॑सा॒ तम॑सा दधाति॒ यज॑माने॒ यज॑माने दधाति॒ तम॑सा । \newline
24. द॒धा॒ति॒ तम॑सा॒ तम॑सा दधाति दधाति॒ तम॑सा सोमविक्र॒यिणꣳ॑ सोमविक्र॒यिण॒म् तम॑सा दधाति दधाति॒ तम॑सा सोमविक्र॒यिण᳚म् । \newline
25. तम॑सा सोमविक्र॒यिणꣳ॑ सोमविक्र॒यिण॒म् तम॑सा॒ तम॑सा सोमविक्र॒यिण॑ मर्पय त्यर्पयति सोमविक्र॒यिण॒म् तम॑सा॒ तम॑सा सोमविक्र॒यिण॑ मर्पयति । \newline
26. सो॒म॒वि॒क्र॒यिण॑ मर्पय त्यर्पयति सोमविक्र॒यिणꣳ॑ सोमविक्र॒यिण॑ मर्पयति॒ यद् यद॑र्पयति सोमविक्र॒यिणꣳ॑ सोमविक्र॒यिण॑ मर्पयति॒ यत् । \newline
27. सो॒म॒वि॒क्र॒यिण॒मिति॑ सोम - वि॒क्र॒यिण᳚म् । \newline
28. अ॒र्प॒य॒ति॒ यद् यद॑र्पय त्यर्पयति॒ यदनु॑पग्र॒थ्या नु॑पग्रथ्य॒ यद॑र्पय त्यर्पयति॒ यदनु॑पग्रथ्य । \newline
29. यदनु॑पग्र॒थ्या नु॑पग्रथ्य॒ यद् यदनु॑पग्रथ्य ह॒न्या द्ध॒न्या दनु॑पग्रथ्य॒ यद् यदनु॑पग्रथ्य ह॒न्यात् । \newline
30. अनु॑पग्रथ्य ह॒न्या द्ध॒न्या दनु॑पग्र॒थ्या नु॑पग्रथ्य ह॒न्याद् द॑न्द॒शूका॑ दन्द॒शूका॑ ह॒न्या दनु॑पग्र॒थ्या नु॑पग्रथ्य ह॒न्याद् द॑न्द॒शूकाः᳚ । \newline
31. अनु॑पग्र॒थ्येत्यनु॑प - ग्र॒थ्य॒ । \newline
32. ह॒न्याद् द॑न्द॒शूका॑ दन्द॒शूका॑ ह॒न्या द्ध॒न्याद् द॑न्द॒शूका॒ स्ताम् ताम् द॑न्द॒शूका॑ ह॒न्या द्ध॒न्याद् द॑न्द॒शूका॒ स्ताम् । \newline
33. द॒न्द॒शूका॒ स्ताम् ताम् द॑न्द॒शूका॑ दन्द॒शूका॒ स्ताꣳ समाꣳ॒॒ समा॒म् ताम् द॑न्द॒शूका॑ दन्द॒शूका॒ स्ताꣳ समा᳚म् । \newline
34. ताꣳ समाꣳ॒॒ समा॒म् ताम् ताꣳ समाꣳ॑ स॒र्पाः स॒र्पाः समा॒म् ताम् ताꣳ समाꣳ॑ स॒र्पाः । \newline
35. समाꣳ॑ स॒र्पाः स॒र्पाः समाꣳ॒॒ समाꣳ॑ स॒र्पाः स्युः॑ स्युः स॒र्पाः समाꣳ॒॒ समाꣳ॑ स॒र्पाः स्युः॑ । \newline
36. स॒र्पाः स्युः॑ स्युः स॒र्पाः स॒र्पाः स्यु॑ रि॒द मि॒दꣳ स्युः॑ स॒र्पाः स॒र्पाः स्यु॑ रि॒दम् । \newline
37. स्यु॒ रि॒द मि॒दꣳ स्युः॑ स्यु रि॒द म॒ह म॒ह मि॒दꣳ स्युः॑ स्यु रि॒द म॒हम् । \newline
38. इ॒द म॒ह म॒ह मि॒द मि॒द म॒हꣳ स॒र्पाणाꣳ॑ स॒र्पाणा॑ म॒ह मि॒द मि॒द म॒हꣳ स॒र्पाणा᳚म् । \newline
39. अ॒हꣳ स॒र्पाणाꣳ॑ स॒र्पाणा॑ म॒ह म॒हꣳ स॒र्पाणा᳚म् दन्द॒शूका॑नाम् दन्द॒शूका॑नाꣳ स॒र्पाणा॑ म॒ह म॒हꣳ स॒र्पाणा᳚म् दन्द॒शूका॑नाम् । \newline
40. स॒र्पाणा᳚म् दन्द॒शूका॑नाम् दन्द॒शूका॑नाꣳ स॒र्पाणाꣳ॑ स॒र्पाणा᳚म् दन्द॒शूका॑नाम् ग्री॒वा ग्री॒वा द॑न्द॒शूका॑नाꣳ स॒र्पाणाꣳ॑ स॒र्पाणा᳚म् दन्द॒शूका॑नाम् ग्री॒वाः । \newline
41. द॒न्द॒शूका॑नाम् ग्री॒वा ग्री॒वा द॑न्द॒शूका॑नाम् दन्द॒शूका॑नाम् ग्री॒वा उपोप॑ ग्री॒वा द॑न्द॒शूका॑नाम् दन्द॒शूका॑नाम् ग्री॒वा उप॑ । \newline
42. ग्री॒वा उपोप॑ ग्री॒वा ग्री॒वा उप॑ ग्रथ्नामि ग्रथ्ना॒ म्युप॑ ग्री॒वा ग्री॒वा उप॑ ग्रथ्नामि । \newline
43. उप॑ ग्रथ्नामि ग्रथ्ना॒ म्युपोप॑ ग्रथ्ना॒मी तीति॑ ग्रथ्ना॒ म्युपोप॑ ग्रथ्ना॒ मीति॑ । \newline
44. ग्र॒थ्ना॒मी तीति॑ ग्रथ्नामि ग्रथ्ना॒मी त्या॑हा॒ हेति॑ ग्रथ्नामि ग्रथ्ना॒मी त्या॑ह । \newline
45. इत्या॑हा॒हे तीत्या॒हा द॑न्दशूका॒ अद॑न्दशूका आ॒हे तीत्या॒हा द॑न्दशूकाः । \newline
46. आ॒हा द॑न्दशूका॒ अद॑न्दशूका आहा॒हा द॑न्दशूका॒ स्ताम् ता मद॑न्दशूका आहा॒हा द॑न्दशूका॒ स्ताम् । \newline
47. अद॑न्दशूका॒ स्ताम् ता मद॑न्दशूका॒ अद॑न्दशूका॒ स्ताꣳ समाꣳ॒॒ समा॒म् ता मद॑न्दशूका॒ अद॑न्दशूका॒ स्ताꣳ समा᳚म् । \newline
48. ताꣳ समाꣳ॒॒ समा॒म् ताम् ताꣳ समाꣳ॑ स॒र्पाः स॒र्पाः समा॒म् ताम् ताꣳ समाꣳ॑ स॒र्पाः । \newline
49. समाꣳ॑ स॒र्पाः स॒र्पाः समाꣳ॒॒ समाꣳ॑ स॒र्पा भ॑वन्ति भवन्ति स॒र्पाः समाꣳ॒॒ समाꣳ॑ स॒र्पा भ॑वन्ति । \newline
50. स॒र्पा भ॑वन्ति भवन्ति स॒र्पाः स॒र्पा भ॑वन्ति॒ तम॑सा॒ तम॑सा भवन्ति स॒र्पाः स॒र्पा भ॑वन्ति॒ तम॑सा । \newline
51. भ॒व॒न्ति॒ तम॑सा॒ तम॑सा भवन्ति भवन्ति॒ तम॑सा सोमविक्र॒यिणꣳ॑ सोमविक्र॒यिण॒म् तम॑सा भवन्ति भवन्ति॒ तम॑सा सोमविक्र॒यिण᳚म् । \newline
52. तम॑सा सोमविक्र॒यिणꣳ॑ सोमविक्र॒यिण॒म् तम॑सा॒ तम॑सा सोमविक्र॒यिणं॑ ॅविद्ध्यति विद्ध्यति सोमविक्र॒यिण॒म् तम॑सा॒ तम॑सा सोमविक्र॒यिणं॑ ॅविद्ध्यति । \newline
53. सो॒म॒वि॒क्र॒यिणं॑ ॅविद्ध्यति विद्ध्यति सोमविक्र॒यिणꣳ॑ सोमविक्र॒यिणं॑ ॅविद्ध्यति॒ स्वान॒ स्वान॑ विद्ध्यति सोमविक्र॒यिणꣳ॑ सोमविक्र॒यिणं॑ ॅविद्ध्यति॒ स्वान॑ । \newline
54. सो॒म॒वि॒क्र॒यिण॒मिति॑ सोम - वि॒क्र॒यिण᳚म् । \newline
55. वि॒द्ध्य॒ति॒ स्वान॒ स्वान॑ विद्ध्यति विद्ध्यति॒ स्वान॒ भ्राज॒ भ्राज॒ स्वान॑ विद्ध्यति विद्ध्यति॒ स्वान॒ भ्राज॑ । \newline
56. स्वान॒ भ्राज॒ भ्राज॒ स्वान॒ स्वान॒ भ्राजे तीति॒ भ्राज॒ स्वान॒ स्वान॒ भ्राजेति॑ । \newline
\pagebreak
\markright{ TS 6.1.10.5  \hfill https://www.vedavms.in \hfill}

\section{ TS 6.1.10.5 }

\textbf{TS 6.1.10.5 } \newline
\textbf{Samhita Paata} \newline

भ्राजेत्या॑है॒ते वा अ॒मुष्मि॑न् ॅलो॒के सोम॑मरक्ष॒न् तेभ्योऽधि॒ सोम॒माऽह॑र॒न्॒. यदे॒तेभ्यः॑ सोम॒क्रय॑णा॒-न्नानु॑दि॒शेदक्री॑तोऽस्य॒ सोमः॑ स्या॒न्नास्यै॒ते॑ ऽमुष्मि॑न् ॅलो॒के सोमꣳ॑ रक्षेयु॒र्यदे॒तेभ्यः॑ सोम॒क्रय॑णाननुदि॒शति॑ क्री॒तो᳚ऽस्य॒ सोमो॑ भवत्ये॒ते᳚ऽस्या॒मुष्मि॑न् ॅलो॒के सोमꣳ॑ रक्षन्ति ॥ \newline

\textbf{Pada Paata} \newline

भ्राज॑ । इति॑ । आ॒ह॒ । ए॒ते । वै । अ॒मुष्मिन्न्॑ । लो॒के । सोम᳚म् । अ॒र॒क्ष॒न्न् । तेभ्यः॑ । अधीति॑ । सोम᳚म् । एति॑ । अ॒ह॒र॒न्न् । यत् । ए॒तेभ्यः॑ । सो॒म॒क्रय॑णा॒निति॑ सोम - क्रय॑णान् । न । अ॒नु॒दि॒शेदित्य॑नु - दि॒शेत् । अक्री॑तः । अ॒स्य॒ । सोमः॑ । स्या॒त् । न । अ॒स्य॒ । ए॒ते । अ॒मुष्मिन्न्॑ । लो॒के । सोम᳚म् । र॒क्षे॒युः॒ । यत् । ए॒तेभ्यः॑ । सो॒म॒क्रय॑णा॒निति॑ सोम - क्रय॑णान् । अ॒नु॒दि॒शतीत्य॑नु - दि॒शति॑ । क्री॒तः । अ॒स्य॒ । सोमः॑ । भ॒व॒ति॒ । ए॒ते । अ॒स्य॒ । अ॒मुष्मिन्न्॑ । लो॒के । सोम᳚म् । र॒क्ष॒न्ति॒ ॥  \newline


\textbf{Krama Paata} \newline

भ्राजेति॑ । इत्या॑ह । आ॒है॒ते । ए॒ते वै । वा अ॒मुष्मिन्न्॑ । अ॒मुष्मि॑न् ॅलो॒के । लो॒के सोम᳚म् । सोम॑मरक्षन्न् । अ॒र॒क्ष॒न् तेभ्यः॑ । तेभ्योऽधि॑ । अधि॒ सोम᳚म् । सोम॒मा । आऽह॑रन्न् । अ॒ह॒र॒न्॒. यत् । यदे॒तेभ्यः॑ । ए॒तेभ्यः॑ सोम॒क्रय॑णान् । सो॒म॒क्रय॑णा॒न् न । सो॒म॒क्रय॑णा॒निति॑ सोम - क्रय॑णान् । नानु॑दि॒शेत् । अ॒नु॒दि॒शेदक्री॑तः । अ॒नु॒दि॒शेदित्य॑नु - दि॒शेत् । अक्री॑तोऽस्य । अ॒स्य॒ सोमः॑ । सोमः॑ स्यात् । स्या॒न् न । नास्य॑ । अ॒स्यै॒ते । ए॒ते॑ऽमुष्मिन्न्॑ । अ॒मुष्मि॑न् ॅलो॒के । लो॒के सोम᳚म् । सोमꣳ॑ रक्षेयुः । र॒क्षे॒यु॒र् यत् । यदे॒तेभ्यः॑ । ए॒तेभ्यः॑ सोम॒क्रय॑णान् । सो॒म॒क्रय॑णाननुदि॒शति॑ । सो॒म॒क्रय॑णा॒निति॑ सोम - क्रय॑णान् । अ॒नु॒दि॒शति॑ क्री॒तः । अ॒नु॒दि॒शतीत्य॑नु - दि॒शति॑ । क्री॒तो᳚ऽस्य । अ॒स्य॒ सोमः॑ । सोमो॑ भवति । भ॒व॒त्ये॒ते । ए॒ते᳚ऽस्य । अ॒स्या॒मुष्मिन्न्॑ । अ॒मुष्मि॑न् ॅलो॒के । लो॒के सोम᳚म् । सोमꣳ॑ रक्षन्ति । र॒क्ष॒न्तीति॑ रक्षन्ति । \newline

\textbf{Jatai Paata} \newline

1. भ्राजे तीति॒ भ्राज॒ भ्राजेति॑ । \newline
2. इत्या॑हा॒हे तीत्या॑ह । \newline
3. आ॒है॒त ए॒त आ॑हा है॒ते । \newline
4. ए॒ते वै वा ए॒त ए॒ते वै । \newline
5. वा अ॒मुष्मि॑न् न॒मुष्मि॒न्॒. वै वा अ॒मुष्मिन्न्॑ । \newline
6. अ॒मुष्मि॑न् ॅलो॒के लो॒के॑ ऽमुष्मि॑न् न॒मुष्मि॑न् ॅलो॒के । \newline
7. लो॒के सोमꣳ॒॒ सोम॑म् ॅलो॒के लो॒के सोम᳚म् । \newline
8. सोम॑ मरक्षन् नरक्ष॒न् थ्सोमꣳ॒॒ सोम॑ मरक्षन्न् । \newline
9. अ॒र॒क्ष॒न् तेभ्य॒ स्तेभ्यो॑ ऽरक्षन् नरक्ष॒न् तेभ्यः॑ । \newline
10. तेभ्यो ऽध्यधि॒ तेभ्य॒ स्तेभ्यो ऽधि॑ । \newline
11. अधि॒ सोमꣳ॒॒ सोम॒ मध्यधि॒ सोम᳚म् । \newline
12. सोम॒ मा सोमꣳ॒॒ सोम॒ मा । \newline
13. आ ऽह॑रन् नहर॒न्ना ऽह॑रन्न् । \newline
14. अ॒ह॒र॒न्॒. यद् यद॑हरन् नहर॒न्॒. यत् । \newline
15. यदे॒तेभ्य॑ ए॒तेभ्यो॒ यद् यदे॒तेभ्यः॑ । \newline
16. ए॒तेभ्यः॑ सोम॒क्रय॑णान् थ्सोम॒क्रय॑णा ने॒तेभ्य॑ ए॒तेभ्यः॑ सोम॒क्रय॑णान् । \newline
17. सो॒म॒क्रय॑णा॒न् न न सो॑म॒क्रय॑णान् थ्सोम॒क्रय॑णा॒न् न । \newline
18. सो॒म॒क्रय॑णा॒निति॑ सोम - क्रय॑णान् । \newline
19. नानु॑दि॒शे द॑नुदि॒शेन् न नानु॑दि॒शेत् । \newline
20. अ॒नु॒दि॒शे दक्री॒तो ऽक्री॑तो ऽनुदि॒शे द॑नुदि॒शे दक्री॑तः । \newline
21. अ॒नु॒दि॒शेदित्य॑नु - दि॒शेत् । \newline
22. अक्री॑तो ऽस्या॒स्या क्री॒तो ऽक्री॑तो ऽस्य । \newline
23. अ॒स्य॒ सोमः॒ सोमो᳚ ऽस्यास्य॒ सोमः॑ । \newline
24. सोमः॑ स्याथ् स्या॒थ् सोमः॒ सोमः॑ स्यात् । \newline
25. स्या॒न् न न स्या᳚थ् स्या॒न् न । \newline
26. नास्या᳚स्य॒ न नास्य॑ । \newline
27. अ॒स्यै॒त ए॒ते᳚ ऽस्या स्यै॒ते । \newline
28. ए॒ते॑ ऽमुष्मि॑न् न॒मुष्मि॑न् ने॒त ए॒ते॑ ऽमुष्मिन्न्॑ । \newline
29. अ॒मुष्मि॑न् ॅलो॒के लो॒के॑ ऽमुष्मि॑न् न॒मुष्मि॑न् ॅलो॒के । \newline
30. लो॒के सोमꣳ॒॒ सोम॑म् ॅलो॒के लो॒के सोम᳚म् । \newline
31. सोमꣳ॑ रक्षेयू रक्षेयुः॒ सोमꣳ॒॒ सोमꣳ॑ रक्षेयुः । \newline
32. र॒क्षे॒यु॒र् यद् यद् र॑क्षेयू रक्षेयु॒र् यत् । \newline
33. यदे॒तेभ्य॑ ए॒तेभ्यो॒ यद् यदे॒तेभ्यः॑ । \newline
34. ए॒तेभ्यः॑ सोम॒क्रय॑णान् थ्सोम॒क्रय॑णा ने॒तेभ्य॑ ए॒तेभ्यः॑ सोम॒क्रय॑णान् । \newline
35. सो॒म॒क्रय॑णा ननुदि॒श त्य॑नुदि॒शति॑ सोम॒क्रय॑णान् थ्सोम॒क्रय॑णा ननुदि॒शति॑ । \newline
36. सो॒म॒क्रय॑णा॒निति॑ सोम - क्रय॑णान् । \newline
37. अ॒नु॒दि॒शति॑ क्री॒तः क्री॒तो॑ ऽनुदि॒श त्य॑नुदि॒शति॑ क्री॒तः । \newline
38. अ॒नु॒दि॒शतीत्य॑नु - दि॒शति॑ । \newline
39. क्री॒तो᳚ ऽस्यास्य क्री॒तः क्री॒तो᳚ ऽस्य । \newline
40. अ॒स्य॒ सोमः॒ सोमो᳚ ऽस्यास्य॒ सोमः॑ । \newline
41. सोमो॑ भवति भवति॒ सोमः॒ सोमो॑ भवति । \newline
42. भ॒व॒ त्ये॒त ए॒ते भ॑वति भव त्ये॒ते । \newline
43. ए॒ते᳚ ऽस्या स्यै॒त ए॒ते᳚ ऽस्य । \newline
44. अ॒स्या॒ मुष्मि॑न् न॒मुष्मि॑न् नस्या स्या॒ मुष्मिन्न्॑ । \newline
45. अ॒मुष्मि॑न् ॅलो॒के लो॒के॑ ऽमुष्मि॑न् न॒मुष्मि॑न् ॅलो॒के । \newline
46. लो॒के सोमꣳ॒॒ सोम॑म् ॅलो॒के लो॒के सोम᳚म् । \newline
47. सोमꣳ॑ रक्षन्ति रक्षन्ति॒ सोमꣳ॒॒ सोमꣳ॑ रक्षन्ति । \newline
48. र॒क्ष॒न्तीति॑ रक्षन्ति । \newline

\textbf{Ghana Paata } \newline

1. भ्राजे तीति॒ भ्राज॒ भ्राजे त्या॑हा॒हेति॒ भ्राज॒ भ्राजे त्या॑ह । \newline
2. इत्या॑हा॒हे तीत्या॑है॒त ए॒त आ॒हे तीत्या॑है॒ते । \newline
3. आ॒है॒त ए॒त आ॑हा है॒ते वै वा ए॒त आ॑हा है॒ते वै । \newline
4. ए॒ते वै वा ए॒त ए॒ते वा अ॒मुष्मि॑न् न॒मुष्मि॒न्॒. वा ए॒त ए॒ते वा अ॒मुष्मिन्न्॑ । \newline
5. वा अ॒मुष्मि॑न् न॒मुष्मि॒न्॒. वै वा अ॒मुष्मि॑न् ॅलो॒के लो॒के॑ ऽमुष्मि॒न्॒. वै वा अ॒मुष्मि॑न् ॅलो॒के । \newline
6. अ॒मुष्मि॑न् ॅलो॒के लो॒के॑ ऽमुष्मि॑न् न॒मुष्मि॑न् ॅलो॒के सोमꣳ॒॒ सोम॑म् ॅलो॒के॑ ऽमुष्मि॑न् न॒मुष्मि॑न् ॅलो॒के सोम᳚म् । \newline
7. लो॒के सोमꣳ॒॒ सोम॑म् ॅलो॒के लो॒के सोम॑ मरक्षन् नरक्ष॒न् थ्सोम॑म् ॅलो॒के लो॒के सोम॑ मरक्षन्न् । \newline
8. सोम॑ मरक्षन् नरक्ष॒न् थ्सोमꣳ॒॒ सोम॑ मरक्ष॒न् तेभ्य॒ स्तेभ्यो॑ ऽरक्ष॒न् थ्सोमꣳ॒॒ सोम॑ मरक्ष॒न् तेभ्यः॑ । \newline
9. अ॒र॒क्ष॒न् तेभ्य॒ स्तेभ्यो॑ ऽरक्षन् नरक्ष॒न् तेभ्यो ऽध्यधि॒ तेभ्यो॑ ऽरक्षन् नरक्ष॒न् तेभ्यो ऽधि॑ । \newline
10. तेभ्यो ऽध्यधि॒ तेभ्य॒ स्तेभ्यो ऽधि॒ सोमꣳ॒॒ सोम॒ मधि॒ तेभ्य॒ स्तेभ्यो ऽधि॒ सोम᳚म् । \newline
11. अधि॒ सोमꣳ॒॒ सोम॒ मध्यधि॒ सोम॒ मा सोम॒ मध्यधि॒ सोम॒ मा । \newline
12. सोम॒ मा सोमꣳ॒॒ सोम॒ मा ऽह॑रन् नहर॒न्ना सोमꣳ॒॒ सोम॒ मा ऽह॑रन्न् । \newline
13. आ ऽह॑रन् नहर॒न्ना ऽह॑र॒न्॒. यद् यद॑हर॒न्ना ऽह॑र॒न्॒. यत् । \newline
14. अ॒ह॒र॒न्॒. यद् यद॑हरन् नहर॒न्॒. यदे॒तेभ्य॑ ए॒तेभ्यो॒ यद॑हरन् नहर॒न्॒. यदे॒तेभ्यः॑ । \newline
15. यदे॒तेभ्य॑ ए॒तेभ्यो॒ यद् यदे॒तेभ्यः॑ सोम॒क्रय॑णान् थ्सोम॒क्रय॑णा ने॒तेभ्यो॒ यद् यदे॒तेभ्यः॑ सोम॒क्रय॑णान् । \newline
16. ए॒तेभ्यः॑ सोम॒क्रय॑णान् थ्सोम॒क्रय॑णा ने॒तेभ्य॑ ए॒तेभ्यः॑ सोम॒क्रय॑णा॒न् न न सो॑म॒क्रय॑णा ने॒तेभ्य॑ ए॒तेभ्यः॑ सोम॒क्रय॑णा॒न् न । \newline
17. सो॒म॒क्रय॑णा॒न् न न सो॑म॒क्रय॑णान् थ्सोम॒क्रय॑णा॒न् नानु॑दि॒शे द॑नुदि॒शेन् न सो॑म॒क्रय॑णान् थ्सोम॒क्रय॑णा॒न् नानु॑दि॒शेत् । \newline
18. सो॒म॒क्रय॑णा॒निति॑ सोम - क्रय॑णान् । \newline
19. नानु॑दि॒शे द॑नुदि॒शेन् न नानु॑दि॒शे दक्री॒तो ऽक्री॑तो ऽनुदि॒शेन् न नानु॑दि॒शे दक्री॑तः । \newline
20. अ॒नु॒दि॒शे दक्री॒तो ऽक्री॑तो ऽनुदि॒शे द॑नुदि॒शे दक्री॑तो ऽस्या॒ स्याक्री॑तो ऽनुदि॒शे द॑नुदि॒शे दक्री॑तो ऽस्य । \newline
21. अ॒नु॒दि॒शेदित्य॑नु - दि॒शेत् । \newline
22. अक्री॑तो ऽस्या॒स्या क्री॒तो ऽक्री॑तो ऽस्य॒ सोमः॒ सोमो॒ ऽस्याक्री॒तो ऽक्री॑तो ऽस्य॒ सोमः॑ । \newline
23. अ॒स्य॒ सोमः॒ सोमो᳚ ऽस्यास्य॒ सोमः॑ स्याथ् स्या॒थ् सोमो᳚ ऽस्यास्य॒ सोमः॑ स्यात् । \newline
24. सोमः॑ स्याथ् स्या॒थ् सोमः॒ सोमः॑ स्या॒न् न न स्या॒थ् सोमः॒ सोमः॑ स्या॒न् न । \newline
25. स्या॒न् न न स्या᳚थ् स्या॒न् नास्या᳚स्य॒ न स्या᳚थ् स्या॒न् नास्य॑ । \newline
26. नास्या᳚स्य॒ न नास्यै॒त ए॒ते᳚ ऽस्य॒ न नास्यै॒ते । \newline
27. अ॒स्यै॒त ए॒ते᳚ ऽस्यास्यै॒ते॑ ऽमुष्मि॑न् न॒मुष्मि॑न् ने॒ते᳚ ऽस्यास्यै॒ते॑ ऽमुष्मिन्न्॑ । \newline
28. ए॒ते॑ ऽमुष्मि॑न् न॒मुष्मि॑न् ने॒त ए॒ते॑ ऽमुष्मि॑न् ॅलो॒के लो॒के॑ ऽमुष्मि॑न् ने॒त ए॒ते॑ ऽमुष्मि॑न् ॅलो॒के । \newline
29. अ॒मुष्मि॑न् ॅलो॒के लो॒के॑ ऽमुष्मि॑न् न॒मुष्मि॑न् ॅलो॒के सोमꣳ॒॒ सोम॑म् ॅलो॒के॑ ऽमुष्मि॑न् न॒मुष्मि॑न् ॅलो॒के सोम᳚म् । \newline
30. लो॒के सोमꣳ॒॒ सोम॑म् ॅलो॒के लो॒के सोमꣳ॑ रक्षेयू रक्षेयुः॒ सोम॑म् ॅलो॒के लो॒के सोमꣳ॑ रक्षेयुः । \newline
31. सोमꣳ॑ रक्षेयू रक्षेयुः॒ सोमꣳ॒॒ सोमꣳ॑ रक्षेयु॒र् यद् यद् र॑क्षेयुः॒ सोमꣳ॒॒ सोमꣳ॑ रक्षेयु॒र् यत् । \newline
32. र॒क्षे॒यु॒र् यद् यद् र॑क्षेयू रक्षेयु॒र् यदे॒तेभ्य॑ ए॒तेभ्यो॒ यद् र॑क्षेयू रक्षेयु॒र् यदे॒तेभ्यः॑ । \newline
33. यदे॒तेभ्य॑ ए॒तेभ्यो॒ यद् यदे॒तेभ्यः॑ सोम॒क्रय॑णान् थ्सोम॒क्रय॑णा ने॒तेभ्यो॒ यद् यदे॒तेभ्यः॑ सोम॒क्रय॑णान् । \newline
34. ए॒तेभ्यः॑ सोम॒क्रय॑णान् थ्सोम॒क्रय॑णा ने॒तेभ्य॑ ए॒तेभ्यः॑ सोम॒क्रय॑णा ननुदि॒श त्य॑नुदि॒शति॑ सोम॒क्रय॑णा ने॒तेभ्य॑ ए॒तेभ्यः॑ सोम॒क्रय॑णा ननुदि॒शति॑ । \newline
35. सो॒म॒क्रय॑णा ननुदि॒श त्य॑नुदि॒शति॑ सोम॒क्रय॑णान् थ्सोम॒क्रय॑णा ननुदि॒शति॑ क्री॒तः क्री॒तो॑ ऽनुदि॒शति॑ सोम॒क्रय॑णान् थ्सोम॒क्रय॑णा ननुदि॒शति॑ क्री॒तः । \newline
36. सो॒म॒क्रय॑णा॒निति॑ सोम - क्रय॑णान् । \newline
37. अ॒नु॒दि॒शति॑ क्री॒तः क्री॒तो॑ ऽनुदि॒श त्य॑नुदि॒शति॑ क्री॒तो᳚ ऽस्यास्य क्री॒तो॑ ऽनुदि॒श त्य॑नुदि॒शति॑ क्री॒तो᳚ ऽस्य । \newline
38. अ॒नु॒दि॒शतीत्य॑नु - दि॒शति॑ । \newline
39. क्री॒तो᳚ ऽस्यास्य क्री॒तः क्री॒तो᳚ ऽस्य॒ सोमः॒ सोमो᳚ ऽस्य क्री॒तः क्री॒तो᳚ ऽस्य॒ सोमः॑ । \newline
40. अ॒स्य॒ सोमः॒ सोमो᳚ ऽस्यास्य॒ सोमो॑ भवति भवति॒ सोमो᳚ ऽस्यास्य॒ सोमो॑ भवति । \newline
41. सोमो॑ भवति भवति॒ सोमः॒ सोमो॑ भव त्ये॒त ए॒ते भ॑वति॒ सोमः॒ सोमो॑ भव त्ये॒ते । \newline
42. भ॒व॒ त्ये॒त ए॒ते भ॑वति भव त्ये॒ते᳚ ऽस्या स्यै॒ते भ॑वति भव त्ये॒ते᳚ ऽस्य । \newline
43. ए॒ते᳚ ऽस्या स्यै॒त ए॒ते᳚ ऽस्या॒मुष्मि॑न् न॒मुष्मि॑न् नस्यै॒त ए॒ते᳚ ऽस्या॒मुष्मिन्न्॑ । \newline
44. अ॒स्या॒मुष्मि॑न् न॒मुष्मि॑न् नस्यास्या॒ मुष्मि॑न् ॅलो॒के लो॒के॑ ऽमुष्मि॑न् नस्यास्या॒ मुष्मि॑न् ॅलो॒के । \newline
45. अ॒मुष्मि॑न् ॅलो॒के लो॒के॑ ऽमुष्मि॑न् न॒मुष्मि॑न् ॅलो॒के सोमꣳ॒॒ सोम॑म् ॅलो॒के॑ ऽमुष्मि॑न् न॒मुष्मि॑न् ॅलो॒के सोम᳚म् । \newline
46. लो॒के सोमꣳ॒॒ सोम॑म् ॅलो॒के लो॒के सोमꣳ॑ रक्षन्ति रक्षन्ति॒ सोम॑म् ॅलो॒के लो॒के सोमꣳ॑ रक्षन्ति । \newline
47. सोमꣳ॑ रक्षन्ति रक्षन्ति॒ सोमꣳ॒॒ सोमꣳ॑ रक्षन्ति । \newline
48. र॒क्ष॒न्तीति॑ रक्षन्ति । \newline
\pagebreak
\markright{ TS 6.1.11.1  \hfill https://www.vedavms.in \hfill}

\section{ TS 6.1.11.1 }

\textbf{TS 6.1.11.1 } \newline
\textbf{Samhita Paata} \newline

वा॒रु॒णो वै क्री॒तः सोम॒ उप॑नद्धो मि॒त्रो न॒ एहि॒ सुमि॑त्रधा॒ इत्या॑ह॒ शान्त्या॒ इन्द्र॑स्यो॒रुमा वि॑श॒ दक्षि॑ण॒मित्या॑ह दे॒वा वै यꣳ सोम॒मक्री॑णन् तमिन्द्र॑स्यो॒रौ दक्षि॑ण॒ आ ऽसा॑दयन्ने॒ष खलु॒ वा ए॒तर्.हीन्द्रो॒ यो यज॑ते॒ तस्मा॑दे॒वमा॒होदायु॑षा स्वा॒युषेत्या॑ह दे॒वता॑ ए॒वा-न्वा॒रभ्योत् - [  ] \newline

\textbf{Pada Paata} \newline

वा॒रु॒णः । वै । क्री॒तः । सोमः॑ । उप॑नद्ध॒ इत्युप॑-न॒द्धः॒ । मि॒त्रः । नः॒ । एति॑ । इ॒हि॒ । सुमि॑त्रधा॒ इति॒ सुमि॑त्र - धाः॒ । इति॑ । आ॒ह॒ । शान्त्यै᳚ । इन्द्र॑स्य । ऊ॒रुम् । एति॑ । वि॒श॒ । दक्षि॑णम् । इति॑ । आ॒ह॒ । दे॒वाः । वै । यम् । सोम᳚म् । अक्री॑णन्न् । तम् । इन्द्र॑स्य । ऊ॒रौ । दक्षि॑णे । एति॑ । अ॒सा॒द॒य॒न्न् । ए॒षः । खलु॑ । वै । ए॒तर्.हि॑ । इन्द्रः॑ । यः । यज॑ते । तस्मा᳚त् । ए॒वम् । आ॒ह॒ । उदिति॑ । आयु॑षा । स्वा॒युषेति॑ सु - आ॒युषा᳚ । इति॑ । आ॒ह॒ । दे॒वताः᳚ । ए॒व । अ॒न्वा॒रभ्येत्य॑नु - आ॒रभ्य॑ । उदिति॑ ।  \newline


\textbf{Krama Paata} \newline

वा॒रु॒णो वै । वै क्री॒तः । क्री॒तः सोमः॑ । सोम॒ उप॑नद्धः । उप॑नद्धो मि॒त्रः । उप॑नद्ध॒ इत्युप॑ - न॒द्धः॒ । मि॒त्रो नः॑ । न॒ आ । एहि॑ । इ॒हि॒ सुमि॑त्रधाः । सुमि॑त्रधा॒ इति॑ । सुमि॑त्रधा॒ इति॒ सुमि॑त्र - धाः॒ । इत्या॑ह । आ॒ह॒ शान्त्यै᳚ । शान्त्या॒ इन्द्र॑स्य । इन्द्र॑स्यो॒रुम् । ऊ॒रुमा । आ वि॑श । वि॒श॒ दक्षि॑णम् । दक्षि॑ण॒मिति॑ । इत्या॑ह । आ॒ह॒ दे॒वाः । दे॒वा वै । वै यम् । यꣳ सोम᳚म् । सोम॒मक्री॑णन्न् । अक्री॑ण॒न् तम् । तमिन्द्र॑स्य । इन्द्र॑स्यो॒रौ । उ॒रौ दक्षि॑णे । दक्षि॑ण॒ आ । आऽसा॑दयन्न् । अ॒सा॒द॒य॒न्ने॒षः । ए॒ष खलु॑ । खलु॒ वै । वा ए॒तर्.हि॑ । ए॒तर्.हीन्द्रः॑ । इन्द्रो॒ यः । यो यज॑ते । यज॑ते॒ तस्मा᳚त् । तस्मा॑दे॒वम् । ए॒वमा॑ह । आ॒होत् । उदायु॑षा । आयु॑षा स्वा॒युषा᳚ । स्वा॒युषेति॑ । स्वा॒युषेति॑ सु - आ॒युषा᳚ । इत्या॑ह । आ॒ह॒ दे॒वताः᳚ । दे॒वता॑ ए॒व । ए॒वान्वा॒रभ्य॑ । अ॒न्वा॒रभ्योत् । अ॒न्वा॒रभ्येत्य॑नु - आ॒रभ्य॑ । उत् ति॑ष्ठति \newline

\textbf{Jatai Paata} \newline

1. वा॒रु॒णो वै वै वा॑रु॒णो वा॑रु॒णो वै । \newline
2. वै क्री॒तः क्री॒तो वै वै क्री॒तः । \newline
3. क्री॒तः सोमः॒ सोमः॑ क्री॒तः क्री॒तः सोमः॑ । \newline
4. सोम॒ उप॑नद्ध॒ उप॑नद्धः॒ सोमः॒ सोम॒ उप॑नद्धः । \newline
5. उप॑नद्धो मि॒त्रो मि॒त्र उप॑नद्ध॒ उप॑नद्धो मि॒त्रः । \newline
6. उप॑नद्ध॒ इत्युप॑ - न॒द्धः॒ । \newline
7. मि॒त्रो नो॑ नो मि॒त्रो मि॒त्रो नः॑ । \newline
8. न॒ आ नो॑ न॒ आ । \newline
9. एही॒ ह्येहि॑ । \newline
10. इ॒हि॒ सुमि॑त्रधाः॒ सुमि॑त्रधा इहीहि॒ सुमि॑त्रधाः । \newline
11. सुमि॑त्रधा॒ इतीति॒ सुमि॑त्रधाः॒ सुमि॑त्रधा॒ इति॑ । \newline
12. सुमि॑त्रधा॒ इति॒ सुमि॑त्र - धाः॒ । \newline
13. इत्या॑हा॒हे तीत्या॑ह । \newline
14. आ॒ह॒ शान्त्यै॒ शान्त्या॑ आहाह॒ शान्त्यै᳚ । \newline
15. शान्त्या॒ इन्द्र॒ स्येन्द्र॑स्य॒ शान्त्यै॒ शान्त्या॒ इन्द्र॑स्य । \newline
16. इन्द्र॑ स्यो॒रु मू॒रु मिन्द्र॒ स्येन्द्र॑ स्यो॒रुम् । \newline
17. ऊ॒रु मोरु मू॒रु मा । \newline
18. आ वि॑श वि॒शा वि॑श । \newline
19. वि॒श॒ दक्षि॑ण॒म् दक्षि॑णं ॅविश विश॒ दक्षि॑णम् । \newline
20. दक्षि॑ण॒ मितीति॒ दक्षि॑ण॒म् दक्षि॑ण॒ मिति॑ । \newline
21. इत्या॑हा॒हे तीत्या॑ह । \newline
22. आ॒ह॒ दे॒वा दे॒वा आ॑हाह दे॒वाः । \newline
23. दे॒वा वै वै दे॒वा दे॒वा वै । \newline
24. वै यं ॅयं ॅवै वै यम् । \newline
25. यꣳ सोमꣳ॒॒ सोमं॒ ॅयं ॅयꣳ सोम᳚म् । \newline
26. सोम॒ मक्री॑ण॒न् नक्री॑ण॒न् थ्सोमꣳ॒॒ सोम॒ मक्री॑णन्न् । \newline
27. अक्री॑ण॒न् तम् त मक्री॑ण॒न् नक्री॑ण॒न् तम् । \newline
28. त मिन्द्र॒ स्येन्द्र॑स्य॒ तम् त मिन्द्र॑स्य । \newline
29. इन्द्र॑ स्यो॒रा वू॒रा विन्द्र॒ स्येन्द्र॑ स्यो॒रौ । \newline
30. ऊ॒रौ दक्षि॑णे॒ दक्षि॑ण ऊ॒रा वू॒रौ दक्षि॑णे । \newline
31. दक्षि॑ण॒ आ दक्षि॑णे॒ दक्षि॑ण॒ आ । \newline
32. आ ऽसा॑दयन् नसादय॒न्ना ऽसा॑दयन्न् । \newline
33. अ॒सा॒द॒य॒न् ने॒ष ए॒षो॑ ऽसादयन् नसादयन् ने॒षः । \newline
34. ए॒ष खलु॒ खल्वे॒ष ए॒ष खलु॑ । \newline
35. खलु॒ वै वै खलु॒ खलु॒ वै । \newline
36. वा ए॒तर् ह्ये॒तर्.हि॒ वै वा ए॒तर्.हि॑ । \newline
37. ए॒तर्.हीन्द्र॒ इन्द्र॑ ए॒तर् ह्ये॒तर्.हीन्द्रः॑ । \newline
38. इन्द्रो॒ यो य इन्द्र॒ इन्द्रो॒ यः । \newline
39. यो यज॑ते॒ यज॑ते॒ यो यो यज॑ते । \newline
40. यज॑ते॒ तस्मा॒त् तस्मा॒द् यज॑ते॒ यज॑ते॒ तस्मा᳚त् । \newline
41. तस्मा॑ दे॒व मे॒वम् तस्मा॒त् तस्मा॑ दे॒वम् । \newline
42. ए॒व मा॑हा है॒व मे॒व मा॑ह । \newline
43. आ॒हो दुदा॑ हा॒होत् । \newline
44. उदायु॒षा ऽऽयु॒षोदु दायु॑षा । \newline
45. आयु॑षा स्वा॒युषा᳚ स्वा॒युषा ऽऽयु॒षा ऽऽयु॑षा स्वा॒युषा᳚ । \newline
46. स्वा॒यु षेतीति॑ स्वा॒युषा᳚ स्वा॒यु षेति॑ । \newline
47. स्वा॒युषेति॑ सु - आ॒युषा᳚ । \newline
48. इत्या॑हा॒हे तीत्या॑ह । \newline
49. आ॒ह॒ दे॒वता॑ दे॒वता॑ आहाह दे॒वताः᳚ । \newline
50. दे॒वता॑ ए॒वैव दे॒वता॑ दे॒वता॑ ए॒व । \newline
51. ए॒वा न्वा॒रभ्या᳚ न्वा॒रभ्यै॒ वैवा न्वा॒रभ्य॑ । \newline
52. अ॒न्वा॒रभ्यो दुद॑ न्वा॒रभ्या᳚ न्वा॒रभ्योत् । \newline
53. अ॒न्वा॒रभ्येत्य॑नु - आ॒रभ्य॑ । \newline
54. उत् ति॑ष्ठति तिष्ठ॒ त्युदुत् ति॑ष्ठति । \newline

\textbf{Ghana Paata } \newline

1. वा॒रु॒णो वै वै वा॑रु॒णो वा॑रु॒णो वै क्री॒तः क्री॒तो वै वा॑रु॒णो वा॑रु॒णो वै क्री॒तः । \newline
2. वै क्री॒तः क्री॒तो वै वै क्री॒तः सोमः॒ सोमः॑ क्री॒तो वै वै क्री॒तः सोमः॑ । \newline
3. क्री॒तः सोमः॒ सोमः॑ क्री॒तः क्री॒तः सोम॒ उप॑नद्ध॒ उप॑नद्धः॒ सोमः॑ क्री॒तः क्री॒तः सोम॒ उप॑नद्धः । \newline
4. सोम॒ उप॑नद्ध॒ उप॑नद्धः॒ सोमः॒ सोम॒ उप॑नद्धो मि॒त्रो मि॒त्र उप॑नद्धः॒ सोमः॒ सोम॒ उप॑नद्धो मि॒त्रः । \newline
5. उप॑नद्धो मि॒त्रो मि॒त्र उप॑नद्ध॒ उप॑नद्धो मि॒त्रो नो॑ नो मि॒त्र उप॑नद्ध॒ उप॑नद्धो मि॒त्रो नः॑ । \newline
6. उप॑नद्ध॒ इत्युप॑ - न॒द्धः॒ । \newline
7. मि॒त्रो नो॑ नो मि॒त्रो मि॒त्रो न॒ आ नो॑ मि॒त्रो मि॒त्रो न॒ आ । \newline
8. न॒ आ नो॑ न॒ एही॒ह्या नो॑ न॒ एहि॑ । \newline
9. एही॒ ह्येहि॒ सुमि॑त्रधाः॒ सुमि॑त्रधा इ॒ह्येहि॒ सुमि॑त्रधाः । \newline
10. इ॒हि॒ सुमि॑त्रधाः॒ सुमि॑त्रधा इहीहि॒ सुमि॑त्रधा॒ इतीति॒ सुमि॑त्रधा इहीहि॒ सुमि॑त्रधा॒ इति॑ । \newline
11. सुमि॑त्रधा॒ इतीति॒ सुमि॑त्रधाः॒ सुमि॑त्रधा॒ इत्या॑हा॒हेति॒ सुमि॑त्रधाः॒ सुमि॑त्रधा॒ इत्या॑ह । \newline
12. सुमि॑त्रधा॒ इति॒ सुमि॑त्र - धाः॒ । \newline
13. इत्या॑हा॒हे तीत्या॑ह॒ शान्त्यै॒ शान्त्या॑ आ॒हे तीत्या॑ह॒ शान्त्यै᳚ । \newline
14. आ॒ह॒ शान्त्यै॒ शान्त्या॑ आहाह॒ शान्त्या॒ इन्द्र॒ स्येन्द्र॑स्य॒ शान्त्या॑ आहाह॒ शान्त्या॒ इन्द्र॑स्य । \newline
15. शान्त्या॒ इन्द्र॒ स्येन्द्र॑स्य॒ शान्त्यै॒ शान्त्या॒ इन्द्र॑स्यो॒रु मू॒रु मिन्द्र॑स्य॒ शान्त्यै॒ शान्त्या॒ इन्द्र॑ स्यो॒रुम् । \newline
16. इन्द्र॑ स्यो॒रु मू॒रु मिन्द्र॒ स्येन्द्र॑ स्यो॒रु मोरु मिन्द्र॒ स्येन्द्र॑ स्यो॒रु मा । \newline
17. ऊ॒रु मोरु मू॒रु मा वि॑श वि॒शोरु मू॒रु मा वि॑श । \newline
18. आ वि॑श वि॒शा वि॑श॒ दक्षि॑ण॒म् दक्षि॑णं ॅवि॒शा वि॑श॒ दक्षि॑णम् । \newline
19. वि॒श॒ दक्षि॑ण॒म् दक्षि॑णं ॅविश विश॒ दक्षि॑ण॒ मितीति॒ दक्षि॑णं ॅविश विश॒ दक्षि॑ण॒ मिति॑ । \newline
20. दक्षि॑ण॒ मितीति॒ दक्षि॑ण॒म् दक्षि॑ण॒ मित्या॑हा॒ हेति॒ दक्षि॑ण॒म् दक्षि॑ण॒ मित्या॑ह । \newline
21. इत्या॑हा॒हे तीत्या॑ह दे॒वा दे॒वा आ॒हे तीत्या॑ह दे॒वाः । \newline
22. आ॒ह॒ दे॒वा दे॒वा आ॑हाह दे॒वा वै वै दे॒वा आ॑हाह दे॒वा वै । \newline
23. दे॒वा वै वै दे॒वा दे॒वा वै यं ॅयं ॅवै दे॒वा दे॒वा वै यम् । \newline
24. वै यं ॅयं ॅवै वै यꣳ सोमꣳ॒॒ सोमं॒ ॅयं ॅवै वै यꣳ सोम᳚म् । \newline
25. यꣳ सोमꣳ॒॒ सोमं॒ ॅयं ॅयꣳ सोम॒ मक्री॑ण॒न् नक्री॑ण॒न् थ्सोमं॒ ॅयं ॅयꣳ सोम॒ मक्री॑णन्न् । \newline
26. सोम॒ मक्री॑ण॒न् नक्री॑ण॒न् थ्सोमꣳ॒॒ सोम॒ मक्री॑ण॒न् तम् त मक्री॑ण॒न् थ्सोमꣳ॒॒ सोम॒ मक्री॑ण॒न् तम् । \newline
27. अक्री॑ण॒न् तम् त मक्री॑ण॒न् नक्री॑ण॒न् त मिन्द्र॒ स्येन्द्र॑स्य॒ त मक्री॑ण॒न् नक्री॑ण॒न् त मिन्द्र॑स्य । \newline
28. त मिन्द्र॒ स्येन्द्र॑स्य॒ तम् त मिन्द्र॑ स्यो॒रा वू॒रा विन्द्र॑स्य॒ तम् त मिन्द्र॑ स्यो॒रौ । \newline
29. इन्द्र॑ स्यो॒रा वू॒रा विन्द्र॒ स्येन्द्र॑ स्यो॒रौ दक्षि॑णे॒ दक्षि॑ण ऊ॒रा विन्द्र॒ स्येन्द्र॑ स्यो॒रौ दक्षि॑णे । \newline
30. ऊ॒रौ दक्षि॑णे॒ दक्षि॑ण ऊ॒रा वू॒रौ दक्षि॑ण॒ आ दक्षि॑ण ऊ॒रा वू॒रौ दक्षि॑ण॒ आ । \newline
31. दक्षि॑ण॒ आ दक्षि॑णे॒ दक्षि॑ण॒ आ ऽसा॑दयन् नसादय॒न्ना दक्षि॑णे॒ दक्षि॑ण॒ आ ऽसा॑दयन्न् । \newline
32. आ ऽसा॑दयन् नसादय॒न्ना ऽसा॑दयन् ने॒ष ए॒षो॑ ऽसादय॒न्ना ऽसा॑दयन् ने॒षः । \newline
33. अ॒सा॒द॒य॒न् ने॒ष ए॒षो॑ ऽसादयन् नसादयन् ने॒ष खलु॒ खल्वे॒षो॑ ऽसादयन् नसादयन् ने॒ष खलु॑ । \newline
34. ए॒ष खलु॒ खल्वे॒ष ए॒ष खलु॒ वै वै खल्वे॒ष ए॒ष खलु॒ वै । \newline
35. खलु॒ वै वै खलु॒ खलु॒ वा ए॒तर् ह्ये॒तर्.हि॒ वै खलु॒ खलु॒ वा ए॒तर्.हि॑ । \newline
36. वा ए॒तर् ह्ये॒तर्.हि॒ वै वा ए॒तर्.हीन्द्र॒ इन्द्र॑ ए॒तर्.हि॒ वै वा ए॒तर्.हीन्द्रः॑ । \newline
37. ए॒तर्.हीन्द्र॒ इन्द्र॑ ए॒तर् ह्ये॒तर्.हीन्द्रो॒ यो य इन्द्र॑ ए॒तर् ह्ये॒तर्.हीन्द्रो॒ यः । \newline
38. इन्द्रो॒ यो य इन्द्र॒ इन्द्रो॒ यो यज॑ते॒ यज॑ते॒ य इन्द्र॒ इन्द्रो॒ यो यज॑ते । \newline
39. यो यज॑ते॒ यज॑ते॒ यो यो यज॑ते॒ तस्मा॒त् तस्मा॒द् यज॑ते॒ यो यो यज॑ते॒ तस्मा᳚त् । \newline
40. यज॑ते॒ तस्मा॒त् तस्मा॒द् यज॑ते॒ यज॑ते॒ तस्मा॑ दे॒व मे॒वम् तस्मा॒द् यज॑ते॒ यज॑ते॒ तस्मा॑ दे॒वम् । \newline
41. तस्मा॑ दे॒व मे॒वम् तस्मा॒त् तस्मा॑ दे॒व मा॑हा है॒वम् तस्मा॒त् तस्मा॑ दे॒व मा॑ह । \newline
42. ए॒व मा॑हा है॒व मे॒व मा॒होदु दा॑है॒व मे॒व मा॒होत् । \newline
43. आ॒होदु दा॑हा॒हो दायु॒षा ऽऽयु॒षो दा॑हा॒हो दायु॑षा । \newline
44. उदायु॒षा ऽऽयु॒षोदु दायु॑षा स्वा॒युषा᳚ स्वा॒युषा ऽऽयु॒षोदु दायु॑षा स्वा॒युषा᳚ । \newline
45. आयु॑षा स्वा॒युषा᳚ स्वा॒युषा ऽऽयु॒षा ऽऽयु॑षा स्वा॒युषे तीति॑ स्वा॒युषा ऽऽयु॒षा ऽऽयु॑षा स्वा॒युषेति॑ । \newline
46. स्वा॒युषे तीति॑ स्वा॒युषा᳚ स्वा॒युषे त्या॑हा॒हेति॑ स्वा॒युषा᳚ स्वा॒युषे त्या॑ह । \newline
47. स्वा॒युषेति॑ सु - आ॒युषा᳚ । \newline
48. इत्या॑हा॒हे तीत्या॑ह दे॒वता॑ दे॒वता॑ आ॒हे तीत्या॑ह दे॒वताः᳚ । \newline
49. आ॒ह॒ दे॒वता॑ दे॒वता॑ आहाह दे॒वता॑ ए॒वैव दे॒वता॑ आहाह दे॒वता॑ ए॒व । \newline
50. दे॒वता॑ ए॒वैव दे॒वता॑ दे॒वता॑ ए॒वा न्वा॒रभ्या᳚ न्वा॒रभ्यै॒व दे॒वता॑ दे॒वता॑ ए॒वा न्वा॒रभ्य॑ । \newline
51. ए॒वा न्वा॒रभ्या᳚ न्वा॒रभ्यै॒ वैवा न्वा॒रभ्योदु द॑न्वा॒रभ्यै॒ वैवा न्वा॒रभ्योत् । \newline
52. अ॒न्वा॒रभ्यो दुद॑ न्वा॒रभ्या᳚ न्वा॒रभ्योत् ति॑ष्ठति तिष्ठ॒ त्युद॑न्वा॒रभ्या᳚ न्वा॒रभ्योत् ति॑ष्ठति । \newline
53. अ॒न्वा॒रभ्येत्य॑नु - आ॒रभ्य॑ । \newline
54. उत् ति॑ष्ठति तिष्ठ॒ त्युदुत् ति॑ष्ठ त्यु॒रू॑रु ति॑ष्ठ॒ त्युदुत् ति॑ष्ठ त्यु॒रु । \newline
\pagebreak
\markright{ TS 6.1.11.2  \hfill https://www.vedavms.in \hfill}

\section{ TS 6.1.11.2 }

\textbf{TS 6.1.11.2 } \newline
\textbf{Samhita Paata} \newline

ति॑ष्ठत्यु॒-र्व॑न्तरि॑क्ष॒-मन्वि॒हीत्या॑हा-न्तरिक्षदेव॒त्यो᳚(1॒) ह्ये॑तर्.हि॒ सोमोऽदि॑त्याः॒ सदो॒ऽस्यदि॑त्याः॒ सद॒ आ सी॒देत्या॑ह यथाय॒जुरे॒वैतद् द्वि वा ए॑नमे॒तद॑र्द्धयति॒ यद् वा॑रु॒णꣳ सन्तं॑ मै॒त्रं क॒रोति॑ वारु॒ण्यर्चा॑ऽऽ सा॑दयति॒ स्वयै॒वैनं॑ दे॒वत॑या॒ सम॑र्द्धयति॒ वास॑सा प॒र्यान॑ह्यति सर्वदेव॒त्यं॑ ॅवै वासः॒ सर्वा॑भिरे॒वै - [  ] \newline

\textbf{Pada Paata} \newline

ति॒ष्ठ॒ति॒ । उ॒रु । अ॒न्तरि॑क्षम् । अन्विति॑ । इ॒हि॒ । इति॑ । आ॒ह॒ । अ॒न्त॒रि॒क्ष॒दे॒व॒त्य॑ इत्य॑न्तरिक्ष - दे॒व॒त्यः॑ । हि । ए॒तर्.हि॑ । सोमः॑ । अदि॑त्याः । सदः॑ । अ॒सि॒ । अदि॑त्याः । सदः॑ । एति॑ । सी॒द॒ । इति॑ । आ॒ह॒ । य॒था॒य॒जुरिति॑ यथा-य॒जुः । ए॒व । ए॒तत् । वीति॑ । वै । ए॒न॒म् । ए॒तत् । अ॒द्‌र्ध॒य॒ति॒ । यत् । वा॒रु॒णम् । सन्त᳚म् । मै॒त्रम् । क॒रोति॑ । वा॒रु॒ण्या । ऋ॒चा । एति॑ । सा॒द॒य॒ति॒ । स्वया᳚ । ए॒व । ए॒न॒म् । दे॒वत॑या । समिति॑ । अ॒द्‌र्ध॒य॒ति॒ । वास॑सा । प॒र्यान॑ह्य॒तीति॑ परि - आन॑ह्यति । स॒र्व॒दे॒व॒त्य॑मिति॑ सर्व -दे॒व॒त्य᳚म् । वै । वासः॑ । सर्वा॑भिः । ए॒व ।  \newline


\textbf{Krama Paata} \newline

ति॒ष्ठ॒त्यु॒रु । उ॒र्व॑न्तरि॑क्षम् । अ॒न्तरि॑क्ष॒मनु॑ । अन्वि॑हि । इ॒हीति॑ । इत्या॑ह । आ॒हा॒न्त॒रि॒क्ष॒दे॒व॒त्यः॑ । अ॒न्त॒रि॒क्ष॒दे॒व॒त्यो॑ हि । अ॒न्त॒रि॒क्ष॒दे॒व॒त्य॑ इत्य॑न्तरिक्ष - दे॒व॒त्यः॑ । ह्ये॑तर्.हि॑ । ए॒तर्.हि॒ सोमः॑ । सोमोऽदि॑त्याः । अदि॑त्याः॒ सदः॑ । सदो॑ऽसि । अ॒स्यदि॑त्याः । अदि॑त्याः॒ सदः॑ । सद॒ आ । आ सी॑द । सी॒देति॑ । इत्या॑ह । आ॒ह॒ य॒था॒य॒जुः । य॒था॒य॒जुरे॒व । य॒था॒य॒जुरिति॑ यथा - य॒जुः । ए॒वैतत् । ए॒तद् वि । वि वै । वा ए॑नम् । ए॒न॒मे॒तत् । ए॒तद॑र्द्धयति । अ॒र्द्ध॒य॒ति॒ यत् । यद् वा॑रु॒णम् । वा॒रु॒णꣳ सन्त᳚म् । सन्त॑म् मै॒त्रम् । मै॒त्रम् क॒रोति॑ । क॒रोति॑ वारु॒ण्या । वा॒रु॒ण्यर्चा । ऋ॒चा ऽऽ सा॑दयति । आ सा॑दयति । सा॒द॒य॒ति॒ स्वया᳚ । स्वयै॒व । ए॒वैन᳚म् । ए॒न॒म् दे॒वत॑या । दे॒वत॑या॒ सम् । सम॑र्द्धयति । अ॒र्द्ध॒य॒ति॒ वास॑सा । वास॑सा प॒र्यान॑ह्यति । प॒र्यान॑ह्यति सर्वदेव॒त्य᳚म् । प॒र्यान॑ह्य॒तीति॑ परि - आन॑ह्यति । स॒र्व॒दे॒व॒त्य॑म् ॅवै । स॒र्व॒दे॒व॒त्य॑मिति॑ सर्व - दे॒व॒त्य᳚म् । वै वासः॑ । वासः॒ सर्वा॑भिः । सर्वा॑भिरे॒व । ए॒वैन᳚म् \newline

\textbf{Jatai Paata} \newline

1. ति॒ष्ठ॒ त्यु॒रू॑रु ति॑ष्ठति तिष्ठ त्यु॒रु । \newline
2. उ॒र्व॑न्तरि॑क्ष म॒न्तरि॑क्ष मु॒रू᳚(1॒)र्व॑न्तरि॑क्षम् । \newline
3. अ॒न्तरि॑क्ष॒ मन् वन् व॒न्तरि॑क्ष म॒न्तरि॑क्ष॒ मनु॑ । \newline
4. अन् वि॑ ही॒ह्यन् वन् वि॑हि । \newline
5. इ॒ही तीती॑ ही॒हीति॑ । \newline
6. इत्या॑हा॒हे तीत्या॑ह । \newline
7. आ॒हा॒ न्त॒रि॒क्ष॒दे॒व॒त्यो᳚ ऽन्तरिक्षदेव॒त्य॑ आहाहा न्तरिक्षदेव॒त्यः॑ । \newline
8. अ॒न्त॒रि॒क्ष॒दे॒व॒त्यो॑ हि ह्य॑न्तरिक्षदेव॒त्यो᳚ ऽन्तरिक्षदेव॒त्यो॑ हि । \newline
9. अ॒न्त॒रि॒क्ष॒दे॒व॒त्य॑ इत्य॑न्तरिक्ष - दे॒व॒त्यः॑ । \newline
10. ह्ये॑तर् ह्ये॒तर्.हि॒ हि ह्ये॑तर्.हि॑ । \newline
11. ए॒तर्.हि॒ सोमः॒ सोम॑ ए॒तर् ह्ये॒तर्.हि॒ सोमः॑ । \newline
12. सोमो ऽदि॑त्या॒ अदि॑त्याः॒ सोमः॒ सोमो ऽदि॑त्याः । \newline
13. अदि॑त्याः॒ सदः॒ सदो ऽदि॑त्या॒ अदि॑त्याः॒ सदः॑ । \newline
14. सदो᳚ ऽस्यसि॒ सदः॒ सदो॑ ऽसि । \newline
15. अ॒स्यदि॑त्या॒ अदि॑त्या अस्य॒स्य दि॑त्याः । \newline
16. अदि॑त्याः॒ सदः॒ सदो ऽदि॑त्या॒ अदि॑त्याः॒ सदः॑ । \newline
17. सद॒ आ सदः॒ सद॒ आ । \newline
18. आ सी॑द सी॒दा सी॑द । \newline
19. सी॒दे तीति॑ सीद सी॒देति॑ । \newline
20. इत्या॑हा॒हे तीत्या॑ह । \newline
21. आ॒ह॒ य॒था॒य॒जुर् य॑थाय॒जु रा॑हाह यथाय॒जुः । \newline
22. य॒था॒य॒जु रे॒वैव य॑थाय॒जुर् य॑थाय॒जु रे॒व । \newline
23. य॒था॒य॒जुरिति॑ यथा - य॒जुः । \newline
24. ए॒वैत दे॒त दे॒वै वैतत् । \newline
25. ए॒तद् वि व्ये॑त दे॒तद् वि । \newline
26. वि वै वै वि वि वै । \newline
27. वा ए॑न मेनं॒ ॅवै वा ए॑नम् । \newline
28. ए॒न॒ मे॒त दे॒त दे॑न मेन मे॒तत् । \newline
29. ए॒त द॑र्द्धय त्यर्द्धय त्ये॒त दे॒त द॑र्द्धयति । \newline
30. अ॒र्द्ध॒य॒ति॒ यद् यद॑र्द्धय त्यर्द्धयति॒ यत् । \newline
31. यद् वा॑रु॒णं ॅवा॑रु॒णं ॅयद् यद् वा॑रु॒णम् । \newline
32. वा॒रु॒णꣳ सन्तꣳ॒॒ सन्तं॑ ॅवारु॒णं ॅवा॑रु॒णꣳ सन्त᳚म् । \newline
33. सन्त॑म् मै॒त्रम् मै॒त्रꣳ सन्तꣳ॒॒ सन्त॑म् मै॒त्रम् । \newline
34. मै॒त्रम् क॒रोति॑ क॒रोति॑ मै॒त्रम् मै॒त्रम् क॒रोति॑ । \newline
35. क॒रोति॑ वारु॒ण्या वा॑रु॒ण्या क॒रोति॑ क॒रोति॑ वारु॒ण्या । \newline
36. वा॒रु॒ण्य र्‌च र्‌चा वा॑रु॒ण्या वा॑रु॒ण्य र्‌चा । \newline
37. ऋ॒चा ऽऽसा॑दयति सादयत्या॒ ‌र्च र्‌चा ऽऽसा॑दयति । \newline
38. आ सा॑दयति सादय॒त्या सा॑दयति । \newline
39. सा॒द॒य॒ति॒ स्वया॒ स्वया॑ सादयति सादयति॒ स्वया᳚ । \newline
40. स्वयै॒ वैव स्वया॒ स्वयै॒व । \newline
41. ए॒वैन॑ मेन मे॒वै वैन᳚म् । \newline
42. ए॒न॒म् दे॒वत॑या दे॒वत॑ यैन मेनम् दे॒वत॑या । \newline
43. दे॒वत॑या॒ सꣳ सम् दे॒वत॑या दे॒वत॑या॒ सम् । \newline
44. स म॑र्द्धय त्यर्द्धयति॒ सꣳ स म॑र्द्धयति । \newline
45. अ॒र्द्ध॒य॒ति॒ वास॑सा॒ वास॑सा ऽर्द्धय त्यर्द्धयति॒ वास॑सा । \newline
46. वास॑सा प॒र्यान॑ह्यति प॒र्यान॑ह्यति॒ वास॑सा॒ वास॑सा प॒र्यान॑ह्यति । \newline
47. प॒र्यान॑ह्यति सर्वदेव॒त्यꣳ॑ सर्वदेव॒त्य॑म् प॒र्यान॑ह्यति प॒र्यान॑ह्यति सर्वदेव॒त्य᳚म् । \newline
48. प॒र्यान॑ह्य॒तीति॑ परि - आन॑ह्यति । \newline
49. स॒र्व॒दे॒व॒त्यं॑ ॅवै वै स॑र्वदेव॒त्यꣳ॑ सर्वदेव॒त्यं॑ ॅवै । \newline
50. स॒र्व॒दे॒व॒त्य॑मिति॑ सर्व - दे॒व॒त्य᳚म् । \newline
51. वै वासो॒ वासो॒ वै वै वासः॑ । \newline
52. वासः॒ सर्वा॑भिः॒ सर्वा॑भि॒र् वासो॒ वासः॒ सर्वा॑भिः । \newline
53. सर्वा॑भि रे॒वैव सर्वा॑भिः॒ सर्वा॑भि रे॒व । \newline
54. ए॒वैन॑ मेन मे॒वै वैन᳚म् । \newline

\textbf{Ghana Paata } \newline

1. ति॒ष्ठ॒ त्यु॒रू॑रु ति॑ष्ठति तिष्ठ त्यु॒र्व॑न्तरि॑क्ष म॒न्तरि॑क्ष मु॒रु ति॑ष्ठति तिष्ठ त्यु॒र्व॑न्तरि॑क्षम् । \newline
2. उ॒र्व॑न्तरि॑क्ष म॒न्तरि॑क्ष मु॒रू᳚(1॒)र्व॑ न्तरि॑क्ष॒ मन् वन् व॒न्तरि॑क्ष मु॒रू᳚(1॒)र्व॑ न्तरि॑क्ष॒ मनु॑ । \newline
3. अ॒न्तरि॑क्ष॒ मन् वन् व॒न्तरि॑क्ष म॒न्तरि॑क्ष॒ मन्वि॑ ही॒ह्यन् व॒न् तरि॑क्ष म॒न्तरि॑क्ष॒ मन्वि॑हि । \newline
4. अन्वि॑ही॒ ह्यन् वन् वि॒ हीतीती॒ ह्यन् वन् वि॒ हीति॑ । \newline
5. इ॒ही तीती॑ ही॒ही त्या॑हा॒हे ती॑ही॒ही त्या॑ह । \newline
6. इत्या॑हा॒हे तीत्या॑हा न्तरिक्षदेव॒त्यो᳚ ऽन्तरिक्षदेव॒त्य॑ आ॒हे तीत्या॑हा न्तरिक्षदेव॒त्यः॑ । \newline
7. आ॒हा॒ न्त॒रि॒क्ष॒दे॒व॒त्यो᳚ ऽन्तरिक्षदेव॒त्य॑ आहाहा न्तरिक्षदेव॒त्यो॑ हि ह्य॑न्तरिक्षदेव॒त्य॑ आहाहा न्तरिक्षदेव॒त्यो॑ हि । \newline
8. अ॒न्त॒रि॒क्ष॒दे॒व॒त्यो॑ हि ह्य॑न्तरिक्षदेव॒त्यो᳚ ऽन्तरिक्षदेव॒त्यो᳚(1॒) ह्ये॑तर् ह्ये॒तर्.हि॒ ह्य॑न्तरिक्षदेव॒त्यो᳚ ऽन्तरिक्षदेव॒त्यो᳚(1॒) ह्ये॑तर्.हि॑ । \newline
9. अ॒न्त॒रि॒क्ष॒दे॒व॒त्य॑ इत्य॑न्तरिक्ष - दे॒व॒त्यः॑ । \newline
10. ह्ये॑तर् ह्ये॒तर्.हि॒ हि ह्ये॑तर्.हि॒ सोमः॒ सोम॑ ए॒तर्.हि॒ हि ह्ये॑तर्.हि॒ सोमः॑ । \newline
11. ए॒तर्.हि॒ सोमः॒ सोम॑ ए॒तर् ह्ये॒तर्.हि॒ सोमो ऽदि॑त्या॒ अदि॑त्याः॒ सोम॑ ए॒तर् ह्ये॒तर्.हि॒ सोमो ऽदि॑त्याः । \newline
12. सोमो ऽदि॑त्या॒ अदि॑त्याः॒ सोमः॒ सोमो ऽदि॑त्याः॒ सदः॒ सदो ऽदि॑त्याः॒ सोमः॒ सोमो ऽदि॑त्याः॒ सदः॑ । \newline
13. अदि॑त्याः॒ सदः॒ सदो ऽदि॑त्या॒ अदि॑त्याः॒ सदो᳚ ऽस्यसि॒ सदो ऽदि॑त्या॒ अदि॑त्याः॒ सदो॑ ऽसि । \newline
14. सदो᳚ ऽस्यसि॒ सदः॒ सदो॒ ऽस्यदि॑त्या॒ अदि॑त्या असि॒ सदः॒ सदो॒ ऽस्यदि॑त्याः । \newline
15. अ॒स्यदि॑त्या॒ अदि॑त्या अस्य॒स्य दि॑त्याः॒ सदः॒ सदो ऽदि॑त्या अस्य॒स्य दि॑त्याः॒ सदः॑ । \newline
16. अदि॑त्याः॒ सदः॒ सदो ऽदि॑त्या॒ अदि॑त्याः॒ सद॒ आ सदो ऽदि॑त्या॒ अदि॑त्याः॒ सद॒ आ । \newline
17. सद॒ आ सदः॒ सद॒ आ सी॑द सी॒दा सदः॒ सद॒ आ सी॑द । \newline
18. आ सी॑द सी॒दा सी॒दे तीति॑ सी॒दा सी॒देति॑ । \newline
19. सी॒दे तीति॑ सीद सी॒दे त्या॑हा॒ हेति॑ सीद सी॒दे त्या॑ह । \newline
20. इत्या॑हा॒हे तीत्या॑ह यथाय॒जुर् य॑थाय॒जु रा॒हे तीत्या॑ह यथाय॒जुः । \newline
21. आ॒ह॒ य॒था॒य॒जुर् य॑थाय॒जु रा॑हाह यथाय॒जु रे॒वैव य॑थाय॒जु रा॑हाह यथाय॒जु रे॒व । \newline
22. य॒था॒य॒जु रे॒वैव य॑थाय॒जुर् य॑थाय॒जु रे॒वैत दे॒त दे॒व य॑थाय॒जुर् य॑थाय॒जु रे॒वैतत् । \newline
23. य॒था॒य॒जुरिति॑ यथा - य॒जुः । \newline
24. ए॒वैत दे॒त दे॒वै वैतद् वि व्ये॑त दे॒वै वैतद् वि । \newline
25. ए॒तद् वि व्ये॑त दे॒तद् वि वै वै व्ये॑त दे॒तद् वि वै । \newline
26. वि वै वै वि वि वा ए॑न मेनं॒ ॅवै वि वि वा ए॑नम् । \newline
27. वा ए॑न मेनं॒ ॅवै वा ए॑न मे॒त दे॒त दे॑नं॒ ॅवै वा ए॑न मे॒तत् । \newline
28. ए॒न॒ मे॒त दे॒त दे॑न मेन मे॒त द॑र्द्धय त्यर्द्धय त्ये॒त दे॑न मेन मे॒त द॑र्द्धयति । \newline
29. ए॒त द॑र्द्धय त्यर्द्धय त्ये॒त दे॒त द॑र्द्धयति॒ यद् यद॑र्द्धय त्ये॒त दे॒त द॑र्द्धयति॒ यत् । \newline
30. अ॒र्द्ध॒य॒ति॒ यद् यद॑र्द्धय त्यर्द्धयति॒ यद् वा॑रु॒णं ॅवा॑रु॒णं ॅयद॑र्द्धय त्यर्द्धयति॒ यद् वा॑रु॒णम् । \newline
31. यद् वा॑रु॒णं ॅवा॑रु॒णं ॅयद् यद् वा॑रु॒णꣳ सन्तꣳ॒॒ सन्तं॑ ॅवारु॒णं ॅयद् यद् वा॑रु॒णꣳ सन्त᳚म् । \newline
32. वा॒रु॒णꣳ सन्तꣳ॒॒ सन्तं॑ ॅवारु॒णं ॅवा॑रु॒णꣳ सन्त॑म् मै॒त्रम् मै॒त्रꣳ सन्तं॑ ॅवारु॒णं ॅवा॑रु॒णꣳ सन्त॑म् मै॒त्रम् । \newline
33. सन्त॑म् मै॒त्रम् मै॒त्रꣳ सन्तꣳ॒॒ सन्त॑म् मै॒त्रम् क॒रोति॑ क॒रोति॑ मै॒त्रꣳ सन्तꣳ॒॒ सन्त॑म् मै॒त्रम् क॒रोति॑ । \newline
34. मै॒त्रम् क॒रोति॑ क॒रोति॑ मै॒त्रम् मै॒त्रम् क॒रोति॑ वारु॒ण्या वा॑रु॒ण्या क॒रोति॑ मै॒त्रम् मै॒त्रम् क॒रोति॑ वारु॒ण्या । \newline
35. क॒रोति॑ वारु॒ण्या वा॑रु॒ण्या क॒रोति॑ क॒रोति॑ वारु॒ण्य र्‌च र्‌चा वा॑रु॒ण्या क॒रोति॑ क॒रोति॑ वारु॒ण्य र्‌चा । \newline
36. वा॒रु॒ण्य र्‌च र्‌चा वा॑रु॒ण्या वा॑रु॒ण्य र्‌चा ऽऽसा॑दयति सादयत्या॒ र्‌चा वा॑रु॒ण्या वा॑रु॒ण्य र्‌चा ऽऽसा॑दयति । \newline
37. ऋ॒चा ऽऽसा॑दयति सादयत्या॒ र्‌च र्‌चा ऽऽसा॑दयति॒ स्वया॒ स्वया॑ सादयत्या॒ र्‌च र्‌चा ऽऽसा॑दयति॒ स्वया᳚ । \newline
38. आ सा॑दयति सादय॒त्या सा॑दयति॒ स्वया॒ स्वया॑ सादय॒त्या सा॑दयति॒ स्वया᳚ । \newline
39. सा॒द॒य॒ति॒ स्वया॒ स्वया॑ सादयति सादयति॒ स्वयै॒ वैव स्वया॑ सादयति सादयति॒ स्वयै॒व । \newline
40. स्वयै॒ वैव स्वया॒ स्वयै॒ वैन॑ मेन मे॒व स्वया॒ स्वयै॒ वैन᳚म् । \newline
41. ए॒वैन॑ मेन मे॒वै वैन॑म् दे॒वत॑या दे॒वत॑यैन मे॒वै वैन॑म् दे॒वत॑या । \newline
42. ए॒न॒म् दे॒वत॑या दे॒वत॑यैन मेनम् दे॒वत॑या॒ सꣳ सम् दे॒वत॑यैन मेनम् दे॒वत॑या॒ सम् । \newline
43. दे॒वत॑या॒ सꣳ सम् दे॒वत॑या दे॒वत॑या॒ स म॑र्द्धय त्यर्द्धयति॒ सम् दे॒वत॑या दे॒वत॑या॒ स म॑र्द्धयति । \newline
44. स म॑र्द्धय त्यर्द्धयति॒ सꣳ स म॑र्द्धयति॒ वास॑सा॒ वास॑सा ऽर्द्धयति॒ सꣳ स म॑र्द्धयति॒ वास॑सा । \newline
45. अ॒र्द्ध॒य॒ति॒ वास॑सा॒ वास॑सा ऽर्द्धय त्यर्द्धयति॒ वास॑सा प॒र्यान॑ह्यति प॒र्यान॑ह्यति॒ वास॑सा ऽर्द्धय त्यर्द्धयति॒ वास॑सा प॒र्यान॑ह्यति । \newline
46. वास॑सा प॒र्यान॑ह्यति प॒र्यान॑ह्यति॒ वास॑सा॒ वास॑सा प॒र्यान॑ह्यति सर्वदेव॒त्यꣳ॑ सर्वदेव॒त्य॑म् प॒र्यान॑ह्यति॒ वास॑सा॒ वास॑सा प॒र्यान॑ह्यति सर्वदेव॒त्य᳚म् । \newline
47. प॒र्यान॑ह्यति सर्वदेव॒त्यꣳ॑ सर्वदेव॒त्य॑म् प॒र्यान॑ह्यति प॒र्यान॑ह्यति सर्वदेव॒त्यं॑ ॅवै वै स॑र्वदेव॒त्य॑म् प॒र्यान॑ह्यति प॒र्यान॑ह्यति सर्वदेव॒त्यं॑ ॅवै । \newline
48. प॒र्यान॑ह्य॒तीति॑ परि - आन॑ह्यति । \newline
49. स॒र्व॒दे॒व॒त्यं॑ ॅवै वै स॑र्वदेव॒त्यꣳ॑ सर्वदेव॒त्यं॑ ॅवै वासो॒ वासो॒ वै स॑र्वदेव॒त्यꣳ॑ सर्वदेव॒त्यं॑ ॅवै वासः॑ । \newline
50. स॒र्व॒दे॒व॒त्य॑मिति॑ सर्व - दे॒व॒त्य᳚म् । \newline
51. वै वासो॒ वासो॒ वै वै वासः॒ सर्वा॑भिः॒ सर्वा॑भि॒र् वासो॒ वै वै वासः॒ सर्वा॑भिः । \newline
52. वासः॒ सर्वा॑भिः॒ सर्वा॑भि॒र् वासो॒ वासः॒ सर्वा॑भि रे॒वैव सर्वा॑भि॒र् वासो॒ वासः॒ सर्वा॑भि रे॒व । \newline
53. सर्वा॑भि रे॒वैव सर्वा॑भिः॒ सर्वा॑भि रे॒वैन॑ मेन मे॒व सर्वा॑भिः॒ सर्वा॑भि रे॒वैन᳚म् । \newline
54. ए॒वैन॑ मेन मे॒वै वैन॑म् दे॒वता॑भिर् दे॒वता॑भि रेन मे॒वै वैन॑म् दे॒वता॑भिः । \newline
\pagebreak
\markright{ TS 6.1.11.3  \hfill https://www.vedavms.in \hfill}

\section{ TS 6.1.11.3 }

\textbf{TS 6.1.11.3 } \newline
\textbf{Samhita Paata} \newline

-नं॑ दे॒वता॑भिः॒ सम॑र्द्धय॒त्यथो॒ रक्ष॑सा॒मप॑हत्यै॒ वने॑षु॒ व्य॑न्तरि॑क्षं तता॒नेत्या॑ह॒ वने॑षु॒ हि व्य॑न्तरि॑क्षं त॒तान॒ वाज॒मर्व॒थ्स्वित्या॑ह॒ वाजꣳ॒॒ ह्यर्व॑थ्सु॒ पयो॑ अघ्नि॒यास्वित्या॑ह॒ पयो॒ ह्य॑घ्नि॒यासु॑ हृ॒थ्सु क्रतु॒मित्या॑ह हृ॒थ्सु हि क्रतुं॒ ॅवरु॑णो वि॒क्ष्व॑ग्निमित्या॑ह॒ वरु॑णो॒ हि वि॒क्ष्व॑ग्निं दि॒वि सूर्य॒ - [  ] \newline

\textbf{Pada Paata} \newline

ए॒न॒म् । दे॒वता॑भिः । समिति॑ । अ॒द्‌र्ध॒य॒ति॒ । अथो॒ इति॑ । रक्ष॑साम् । अप॑हत्या॒ इत्यप॑-ह॒त्यै॒ । वने॑षु । वीति॑ । अ॒न्तरि॑क्षम् । त॒ता॒न॒ । इति॑ । आ॒ह॒ । वने॑षु । हि । वीति॑ । अ॒न्तरि॑क्षम् । त॒तान॑ । वाज᳚म् । अर्व॒थ्स्वित्यर्व॑त् - सु॒ । इति॑ । आ॒ह॒ । वाज᳚म् । हि । अर्व॒थ्स्वित्यर्व॑त् - सु॒ । पयः॑ । अ॒घ्नि॒यासु॑ । इति॑ । आ॒ह॒ । पयः॑ । हि । अ॒घ्नि॒यासु॑ । हृ॒थ्स्विति॑ हृत् - सु । क्रतु᳚म् । इति॑ । आ॒ह॒ । हृ॒थ्स्विति॑ हृत्-सु । हि । क्रतु᳚म् । वरु॑णः । वि॒क्षु । अ॒ग्निम् । इति॑ । आ॒ह॒ । वरु॑णः । हि । वि॒क्षु । अ॒ग्निम् । दि॒वि । सूर्य᳚म् ।  \newline


\textbf{Krama Paata} \newline

ए॒न॒म् दे॒वता॑भिः । दे॒वता॑भिः॒ सम् । सम॑र्द्धयति । अ॒र्द्ध॒य॒त्यथो᳚ । अथो॒ रक्ष॑साम् । अथो॒ इत्यथो᳚ । रक्ष॑सा॒मप॑हत्यै । अप॑हत्यै॒ वने॑षु । अप॑हत्या॒ इत्यप॑ - ह॒त्यै॒ । वने॑षु॒ वि । व्य॑न्तरि॑क्षम् । अ॒न्तरि॑क्षम् ततान । त॒ता॒नेति॑ । इत्या॑ह । आ॒ह॒ वने॑षु । वने॑षु॒ हि । हि वि । व्य॑न्तरि॑क्षम् । अ॒न्तरि॑क्षम् त॒तान॑ । त॒तान॒ वाज᳚म् । वाज॒मर्व॑थ्सु । अर्व॒थ्स्विति॑ । अर्व॒थ्स्वित्यर्व॑त् - सु॒ । इत्या॑ह । आ॒ह॒ वाज᳚म् । वाजꣳ॒॒ हि । ह्यर्व॑थ्सु । अर्व॑थ्सु॒ पयः॑ । अर्व॒थ्स्वित्यर्व॑त् - सु॒ । पयो॑ अघ्नि॒यासु॑ । अ॒घ्नि॒यास्विति॑ । इत्या॑ह । आ॒ह॒ पयः॑ । पयो॒ हि । ह्य॑घ्नि॒यासु॑ । अ॒घ्नि॒यासु॑ हृ॒थ्सु । हृ॒थ्सु क्रतु᳚म् । हृ॒थ्स्विति॑ हृत् - सु । क्रतु॒मिति॑ । इत्या॑ह । आ॒ह॒ हृ॒थ्सु । हृ॒थ्सु हि । हृ॒थ्स्विति॑ हृत् - सु । हि क्रतु᳚म् । क्रतु॒म् ॅवरु॑णः । वरु॑णो वि॒क्षु । वि॒क्ष्व॑ग्निम् । अ॒ग्निमिति॑ । इत्या॑ह । आ॒ह॒ वरु॑णः । वरु॑णो॒ हि । हि वि॒क्षु । वि॒क्ष्व॑ग्निम् । अ॒ग्निम् दि॒वि । दि॒वि सूर्य᳚म् । सूर्य॒मिति॑ \newline

\textbf{Jatai Paata} \newline

1. ए॒न॒म् दे॒वता॑भिर् दे॒वता॑भि रेन मेनम् दे॒वता॑भिः । \newline
2. दे॒वता॑भिः॒ सꣳ सम् दे॒वता॑भिर् दे॒वता॑भिः॒ सम् । \newline
3. स म॑र्द्धय त्यर्द्धयति॒ सꣳ स म॑र्द्धयति । \newline
4. अ॒र्द्ध॒य॒ त्यथो॒ अथो॑ अर्द्धय त्यर्द्धय॒ त्यथो᳚ । \newline
5. अथो॒ रक्ष॑साꣳ॒॒ रक्ष॑सा॒ मथो॒ अथो॒ रक्ष॑साम् । \newline
6. अथो॒ इत्यथो᳚ । \newline
7. रक्ष॑सा॒ मप॑हत्या॒ अप॑हत्यै॒ रक्ष॑साꣳ॒॒ रक्ष॑सा॒ मप॑हत्यै । \newline
8. अप॑हत्यै॒ वने॑षु॒ वने॒ ष्वप॑हत्या॒ अप॑हत्यै॒ वने॑षु । \newline
9. अप॑हत्या॒ इत्यप॑ - ह॒त्यै॒ । \newline
10. वने॑षु॒ वि वि वने॑षु॒ वने॑षु॒ वि । \newline
11. व्य॑न्तरि॑क्ष म॒न्तरि॑क्षं॒ ॅवि व्य॑न्तरि॑क्षम् । \newline
12. अ॒न्तरि॑क्षम् ततान तताना॒ न्तरि॑क्ष म॒न्तरि॑क्षम् ततान । \newline
13. त॒ता॒ने तीति॑ ततान तता॒नेति॑ । \newline
14. इत्या॑हा॒हे तीत्या॑ह । \newline
15. आ॒ह॒ वने॑षु॒ वने᳚ ष्वा हाह॒ वने॑षु । \newline
16. वने॑षु॒ हि हि वने॑षु॒ वने॑षु॒ हि । \newline
17. हि वि वि हि हि वि । \newline
18. व्य॑न्तरि॑क्ष म॒न्तरि॑क्षं॒ ॅवि व्य॑न्तरि॑क्षम् । \newline
19. अ॒न्तरि॑क्षम् त॒तान॑ त॒ताना॒ न्तरि॑क्ष म॒न्तरि॑क्षम् त॒तान॑ । \newline
20. त॒तान॒ वाजं॒ ॅवाज॑म् त॒तान॑ त॒तान॒ वाज᳚म् । \newline
21. वाज॒ मर्व॒थ् स्वर्व॑थ्सु॒ वाजं॒ ॅवाज॒ मर्व॑थ्सु । \newline
22. अर्व॒ थ्स्विती त्यर्व॒थ् स्वर्व॒ थ्स्विति॑ । \newline
23. अर्व॒थ्स्वित्यर्व॑त् - सु॒ । \newline
24. इत्या॑हा॒हे तीत्या॑ह । \newline
25. आ॒ह॒ वाजं॒ ॅवाज॑ माहाह॒ वाज᳚म् । \newline
26. वाजꣳ॒॒ हि हि वाजं॒ ॅवाजꣳ॒॒ हि । \newline
27. ह्यर्व॒थ् स्वर्व॑थ्सु॒ हि ह्यर्व॑थ्सु । \newline
28. अर्व॑थ्सु॒ पयः॒ पयो ऽर्व॒थ् स्वर्व॑थ्सु॒ पयः॑ । \newline
29. अर्व॒थ्स्वित्यर्व॑त् - सु॒ । \newline
30. पयो॑ अघ्नि॒या स्व॑घ्नि॒यासु॒ पयः॒ पयो॑ अघ्नि॒यासु॑ । \newline
31. अ॒घ्नि॒या स्विती त्य॑घ्नि॒या स्व॑घ्नि॒या स्विति॑ । \newline
32. इत्या॑हा॒हे तीत्या॑ह । \newline
33. आ॒ह॒ पयः॒ पय॑ आहाह॒ पयः॑ । \newline
34. पयो॒ हि हि पयः॒ पयो॒ हि । \newline
35. ह्य॑ घ्नि॒या स्व॑घ्नि॒यासु॒ हि ह्य॑ घ्नि॒यासु॑ । \newline
36. अ॒घ्नि॒यासु॑ हृ॒थ्सु हृ॒थ् स्व॑घ्नि॒या स्व॑घ्नि॒यासु॑ हृ॒थ्सु । \newline
37. हृ॒थ्सु क्रतु॒म् क्रतुꣳ॑ हृ॒थ्सु हृ॒थ्सु क्रतु᳚म् । \newline
38. हृ॒थ्स्विति॑ हृत् - सु । \newline
39. क्रतु॒ मितीति॒ क्रतु॒म् क्रतु॒ मिति॑ । \newline
40. इत्या॑हा॒हे तीत्या॑ह । \newline
41. आ॒ह॒ हृ॒थ्सु हृ॒थ्स्वा॑ हाह हृ॒थ्सु । \newline
42. हृ॒थ्सु हि हि हृ॒थ्सु हृ॒थ्सु हि । \newline
43. हृ॒थ्स्विति॑ हृत् - सु । \newline
44. हि क्रतु॒म् क्रतुꣳ॒॒ हि हि क्रतु᳚म् । \newline
45. क्रतुं॒ ॅवरु॑णो॒ वरु॑णः॒ क्रतु॒म् क्रतुं॒ ॅवरु॑णः । \newline
46. वरु॑णो वि॒क्षु वि॒क्षु वरु॑णो॒ वरु॑णो वि॒क्षु । \newline
47. वि॒क्ष्व॑ग्नि म॒ग्निं ॅवि॒क्षु वि॒क्ष्व॑ग्निम् । \newline
48. अ॒ग्नि मिती त्य॒ग्नि म॒ग्नि मिति॑ । \newline
49. इत्या॑हा॒हे तीत्या॑ह । \newline
50. आ॒ह॒ वरु॑णो॒ वरु॑ण आहाह॒ वरु॑णः । \newline
51. वरु॑णो॒ हि हि वरु॑णो॒ वरु॑णो॒ हि । \newline
52. हि वि॒क्षु वि॒क्षु हि हि वि॒क्षु । \newline
53. वि॒क्ष्व॑ग्नि म॒ग्निं ॅवि॒क्षु वि॒क्ष्व॑ग्निम् । \newline
54. अ॒ग्निम् दि॒वि दि॒व्य॑ग्नि म॒ग्निम् दि॒वि । \newline
55. दि॒वि सूर्यꣳ॒॒ सूर्य॑म् दि॒वि दि॒वि सूर्य᳚म् । \newline
56. सूर्य॒ मितीति॒ सूर्यꣳ॒॒ सूर्य॒ मिति॑ । \newline

\textbf{Ghana Paata } \newline

1. ए॒न॒म् दे॒वता॑भिर् दे॒वता॑भि रेन मेनम् दे॒वता॑भिः॒ सꣳ सम् दे॒वता॑भि रेन मेनम् दे॒वता॑भिः॒ सम् । \newline
2. दे॒वता॑भिः॒ सꣳ सम् दे॒वता॑भिर् दे॒वता॑भिः॒ स म॑र्द्धय त्यर्द्धयति॒ सम् दे॒वता॑भिर् दे॒वता॑भिः॒ सम॑र्द्धयति । \newline
3. सम॑र्द्धय त्यर्द्धयति॒ सꣳ स म॑र्द्धय॒ त्यथो॒ अथो॑ अर्द्धयति॒ सꣳ स म॑र्द्धय॒ त्यथो᳚ । \newline
4. अ॒र्द्ध॒य॒ त्यथो॒ अथो॑ अर्द्धय त्यर्द्धय॒ त्यथो॒ रक्ष॑साꣳ॒॒ रक्ष॑सा॒ मथो॑ अर्द्धय त्यर्द्धय॒ त्यथो॒ रक्ष॑साम् । \newline
5. अथो॒ रक्ष॑साꣳ॒॒ रक्ष॑सा॒ मथो॒ अथो॒ रक्ष॑सा॒ मप॑हत्या॒ अप॑हत्यै॒ रक्ष॑सा॒ मथो॒ अथो॒ रक्ष॑सा॒ मप॑हत्यै । \newline
6. अथो॒ इत्यथो᳚ । \newline
7. रक्ष॑सा॒ मप॑हत्या॒ अप॑हत्यै॒ रक्ष॑साꣳ॒॒ रक्ष॑सा॒ मप॑हत्यै॒ वने॑षु॒ वने॒ष्व प॑हत्यै॒ रक्ष॑साꣳ॒॒ रक्ष॑सा॒ मप॑हत्यै॒ वने॑षु । \newline
8. अप॑हत्यै॒ वने॑षु॒ वने॒ष्व प॑हत्या॒ अप॑हत्यै॒ वने॑षु॒ वि वि वने॒ष्व प॑हत्या॒ अप॑हत्यै॒ वने॑षु॒ वि । \newline
9. अप॑हत्या॒ इत्यप॑ - ह॒त्यै॒ । \newline
10. वने॑षु॒ वि वि वने॑षु॒ वने॑षु॒ व्य॑न्तरि॑क्ष म॒न्तरि॑क्षं॒ ॅवि वने॑षु॒ वने॑षु॒ व्य॑न्तरि॑क्षम् । \newline
11. व्य॑न्तरि॑क्ष म॒न्तरि॑क्षं॒ ॅवि व्य॑न्तरि॑क्षम् ततान तताना॒ न्तरि॑क्षं॒ ॅवि व्य॑न्तरि॑क्षम् ततान । \newline
12. अ॒न्तरि॑क्षम् ततान तताना॒ न्तरि॑क्ष म॒न्तरि॑क्षम् तता॒ने तीति॑ तताना॒ न्तरि॑क्ष म॒न्तरि॑क्षम् तता॒नेति॑ । \newline
13. त॒ता॒ने तीति॑ ततान तता॒ने त्या॑हा॒ हेति॑ ततान तता॒ने त्या॑ह । \newline
14. इत्या॑हा॒हे तीत्या॑ह॒ वने॑षु॒ वने᳚ष्वा॒हे तीत्या॑ह॒ वने॑षु । \newline
15. आ॒ह॒ वने॑षु॒ वने᳚ष्वा हाह॒ वने॑षु॒ हि हि वने᳚ष्वा हाह॒ वने॑षु॒ हि । \newline
16. वने॑षु॒ हि हि वने॑षु॒ वने॑षु॒ हि वि वि हि वने॑षु॒ वने॑षु॒ हि वि । \newline
17. हि वि वि हि हि व्य॑न्तरि॑क्ष म॒न्तरि॑क्षं॒ ॅवि हि हि व्य॑न्तरि॑क्षम् । \newline
18. व्य॑न्तरि॑क्ष म॒न्तरि॑क्षं॒ ॅवि व्य॑न्तरि॑क्षम् त॒तान॑ त॒ताना॒ न्तरि॑क्षं॒ ॅवि व्य॑न्तरि॑क्षम् त॒तान॑ । \newline
19. अ॒न्तरि॑क्षम् त॒तान॑ त॒ताना॒ न्तरि॑क्ष म॒न्तरि॑क्षम् त॒तान॒ वाजं॒ ॅवाज॑म् त॒ताना॒ न्तरि॑क्ष म॒न्तरि॑क्षम् त॒तान॒ वाज᳚म् । \newline
20. त॒तान॒ वाजं॒ ॅवाज॑म् त॒तान॑ त॒तान॒ वाज॒ मर्व॒थ्-स्वर्व॑थ्सु॒ वाज॑म् त॒तान॑ त॒तान॒ वाज॒ मर्व॑थ्सु । \newline
21. वाज॒ मर्व॒थ् स्वर्व॑थ्सु॒ वाजं॒ ॅवाज॒ मर्व॒थ् स्विती त्यर्व॑थ्सु॒ वाजं॒ ॅवाज॒ मर्व॒थ्स्विति॑ । \newline
22. अर्व॒थ्स्विती त्यर्व॒ थ्स्वर्व॒ थ्स्वि त्या॑हा॒हे त्यर्व॒ थ्स्वर् व॒थ्स्वि त्या॑ह । \newline
23. अर्व॒थ्स्वित्यर्व॑त् - सु॒ । \newline
24. इत्या॑हा॒हे तीत्या॑ह॒ वाजं॒ ॅवाज॑ मा॒हे तीत्या॑ह॒ वाज᳚म् । \newline
25. आ॒ह॒ वाजं॒ ॅवाज॑ माहाह॒ वाजꣳ॒॒ हि हि वाज॑ माहाह॒ वाजꣳ॒॒ हि । \newline
26. वाजꣳ॒॒ हि हि वाजं॒ ॅवाजꣳ॒॒ ह्यर्व॒थ् स्वर्व॑थ्सु॒ हि वाजं॒ ॅवाजꣳ॒॒ ह्यर्व॑थ्सु । \newline
27. ह्यर्व॒थ् स्वर्व॑थ्सु॒ हि ह्यर्व॑थ्सु॒ पयः॒ पयो ऽर्व॑थ्सु॒ हि ह्यर्व॑थ्सु॒ पयः॑ । \newline
28. अर्व॑थ्सु॒ पयः॒ पयो ऽर्व॒थ् स्वर्व॑थ्सु॒ पयो॑ अघ्नि॒या स्व॑घ्नि॒यासु॒ पयो ऽर्व॒थ् स्वर्व॑थ्सु॒ पयो॑ अघ्नि॒यासु॑ । \newline
29. अर्व॒थ्स्वित्यर्व॑त् - सु॒ । \newline
30. पयो॑ अघ्नि॒या स्व॑घ्नि॒यासु॒ पयः॒ पयो॑ अघ्नि॒या स्विती त्य॑घ्नि॒यासु॒ पयः॒ पयो॑ अघ्नि॒या स्विति॑ । \newline
31. अ॒घ्नि॒या स्विती त्य॑घ्नि॒या स्व॑घ्नि॒यास्वि त्या॑हा॒हे त्य॑घ्नि॒या स्व॑घ्नि॒या स्वित्या॑ह । \newline
32. इत्या॑हा॒हे तीत्या॑ह॒ पयः॒ पय॑ आ॒हे तीत्या॑ह॒ पयः॑ । \newline
33. आ॒ह॒ पयः॒ पय॑ आहाह॒ पयो॒ हि हि पय॑ आहाह॒ पयो॒ हि । \newline
34. पयो॒ हि हि पयः॒ पयो॒ ह्य॑घ्नि॒या स्व॑घ्नि॒यासु॒ हि पयः॒ पयो॒ ह्य॑घ्नि॒यासु॑ । \newline
35. ह्य॑घ्नि॒यास्व॑ घ्नि॒यासु॒ हि ह्य॑घ्नि॒यासु॑ हृ॒थ्सु हृ॒थ्स्व॑ घ्नि॒यासु॒ हि ह्य॑घ्नि॒यासु॑ हृ॒थ्सु । \newline
36. अ॒घ्नि॒यासु॑ हृ॒थ्सु हृ॒थ् स्व॑घ्नि॒या स्व॑घ्नि॒यासु॑ हृ॒थ्सु क्रतु॒म् क्रतुꣳ॑ हृ॒थ् स्व॑घ्नि॒या स्व॑घ्नि॒यासु॑ हृ॒थ्सु क्रतु᳚म् । \newline
37. हृ॒थ्सु क्रतु॒म् क्रतुꣳ॑ हृ॒थ्सु हृ॒थ्सु क्रतु॒ मितीति॒ क्रतुꣳ॑ हृ॒थ्सु हृ॒थ्सु क्रतु॒ मिति॑ । \newline
38. हृ॒थ्स्विति॑ हृत् - सु । \newline
39. क्रतु॒ मितीति॒ क्रतु॒म् क्रतु॒ मित्या॑हा॒ हेति॒ क्रतु॒म् क्रतु॒ मित्या॑ह । \newline
40. इत्या॑हा॒हे तीत्या॑ह हृ॒थ्सु हृ॒थ् स्वा॑हे तीत्या॑ह हृ॒थ्सु । \newline
41. आ॒ह॒ हृ॒थ्सु हृ॒थ् स्वा॑हाह हृ॒थ्सु हि हि हृ॒थ् स्वा॑हाह हृ॒थ्सु हि । \newline
42. हृ॒थ्सु हि हि हृ॒थ्सु हृ॒थ्सु हि क्रतु॒म् क्रतुꣳ॒॒ हि हृ॒थ्सु हृ॒थ्सु हि क्रतु᳚म् । \newline
43. हृ॒थ्स्विति॑ हृत् - सु । \newline
44. हि क्रतु॒म् क्रतुꣳ॒॒ हि हि क्रतुं॒ ॅवरु॑णो॒ वरु॑णः॒ क्रतुꣳ॒॒ हि हि क्रतुं॒ ॅवरु॑णः । \newline
45. क्रतुं॒ ॅवरु॑णो॒ वरु॑णः॒ क्रतु॒म् क्रतुं॒ ॅवरु॑णो वि॒क्षु वि॒क्षु वरु॑णः॒ क्रतु॒म् क्रतुं॒ ॅवरु॑णो वि॒क्षु । \newline
46. वरु॑णो वि॒क्षु वि॒क्षु वरु॑णो॒ वरु॑णो वि॒क्ष्व॑ग्नि म॒ग्निं ॅवि॒क्षु वरु॑णो॒ वरु॑णो वि॒क्ष्व॑ग्निम् । \newline
47. वि॒क्ष्व॑ग्नि म॒ग्निं ॅवि॒क्षु वि॒क्ष्व॑ग्नि मिती त्य॒ग्निं ॅवि॒क्षु वि॒क्ष्व॑ग्नि मिति॑ । \newline
48. अ॒ग्नि मिती त्य॒ग्नि म॒ग्नि मित्या॑हा॒हे त्य॒ग्नि म॒ग्नि मित्या॑ह । \newline
49. इत्या॑हा॒हे तीत्या॑ह॒ वरु॑णो॒ वरु॑ण आ॒हे तीत्या॑ह॒ वरु॑णः । \newline
50. आ॒ह॒ वरु॑णो॒ वरु॑ण आहाह॒ वरु॑णो॒ हि हि वरु॑ण आहाह॒ वरु॑णो॒ हि । \newline
51. वरु॑णो॒ हि हि वरु॑णो॒ वरु॑णो॒ हि वि॒क्षु वि॒क्षु हि वरु॑णो॒ वरु॑णो॒ हि वि॒क्षु । \newline
52. हि वि॒क्षु वि॒क्षु हि हि वि॒क्ष्व॑ग्नि म॒ग्निं ॅवि॒क्षु हि हि वि॒क्ष्व॑ग्निम् । \newline
53. वि॒क्ष्व॑ग्नि म॒ग्निं ॅवि॒क्षु वि॒क्ष्व॑ग्निम् दि॒वि दि॒व्य॑ग्निं ॅवि॒क्षु वि॒क्ष्व॑ग्निम् दि॒वि । \newline
54. अ॒ग्निम् दि॒वि दि॒व्य॑ग्नि म॒ग्निम् दि॒वि सूर्यꣳ॒॒ सूर्य॑म् दि॒व्य॑ग्नि म॒ग्निम् दि॒वि सूर्य᳚म् । \newline
55. दि॒वि सूर्यꣳ॒॒ सूर्य॑म् दि॒वि दि॒वि सूर्य॒ मितीति॒ सूर्य॑म् दि॒वि दि॒वि सूर्य॒ मिति॑ । \newline
56. सूर्य॒ मितीति॒ सूर्यꣳ॒॒ सूर्य॒ मित्या॑हा॒ हेति॒ सूर्यꣳ॒॒ सूर्य॒ मित्या॑ह । \newline
\pagebreak
\markright{ TS 6.1.11.4  \hfill https://www.vedavms.in \hfill}

\section{ TS 6.1.11.4 }

\textbf{TS 6.1.11.4 } \newline
\textbf{Samhita Paata} \newline

मित्या॑ह दि॒वि हि सूर्यꣳ॒॒ सोम॒मद्रा॒वित्या॑ह॒ ग्रावा॑णो॒ वा अद्र॑य॒स्तेषु॒ वा ए॒ष सोमं॑ दधाति॒ यो यज॑ते॒ तस्मा॑दे॒वमा॒होदु॒ त्यं जा॒तवे॑दस॒मिति॑ सौ॒र्यर्चा कृ॑ष्णाजि॒नं प्र॒त्यान॑ह्यति॒ रक्ष॑सा॒मप॑हत्या॒ उस्रा॒वेतं॑ धूर्.षाहा॒वित्या॑ह यथाय॒जुरे॒वैतत् प्र च्य॑वस्व भुवस्पत॒ इत्या॑ह भू॒तानाꣳ॒॒ ह्ये॑ - [  ] \newline

\textbf{Pada Paata} \newline

इति॑ । आ॒ह॒ । दि॒वि । हि । सूर्य᳚म् । सोम᳚म् । अद्रौ᳚ । इति॑ । आ॒ह॒ । ग्रावा॑णः । वै । अद्र॑यः । तेषु॑ । वै । ए॒षः । सोम᳚म् । द॒धा॒ति॒ । यः । यज॑ते । तस्मा᳚त् । ए॒वम् । आ॒ह॒ । उदिति॑ । उ॒ । त्यम् । जा॒तवे॑दस॒मिति॑ जा॒त - वे॒द॒स॒म् । इति॑ । सौ॒र्या । ऋ॒चा । कृ॒ष्णा॒जि॒नमिति॑ कृष्ण - अ॒जि॒नम् । प्र॒त्यान॑ह्य॒तीति॑ प्रति-आन॑ह्यति । रक्ष॑साम् । अप॑हत्या॒ इत्यप॑ - ह॒त्यै॒ । उस्रौ᳚ । एति॑ । इ॒त॒म् । धू॒र्॒.षा॒हा॒विति॑ धूः - सा॒हौ॒ । इति॑ । आ॒ह॒ । य॒था॒य॒जुरिति॑ यथा - य॒जुः । ए॒व । ए॒तत् । प्रेति॑ । च्य॒व॒स्व॒ । भु॒वः॒ । प॒ते॒ । इति॑ । आ॒ह॒ । भू॒ताना᳚म् । हि ।  \newline


\textbf{Krama Paata} \newline

इत्या॑ह । आ॒ह॒ दि॒वि । दि॒वि हि । हि सूर्य᳚म् । सूर्यꣳ॒॒ सोम᳚म् । सोम॒मद्रौ᳚ । अद्रा॒विति॑ । इत्या॑ह । आ॒ह॒ ग्रावा॑णः । ग्रावा॑णो॒ वै । वा अद्र॑यः । अद्र॑य॒स्तेषु॑ । तेषु॒ वै । वा ए॒षः । ए॒ष सोम᳚म् । सोम॑म् दधाति । द॒धा॒ति॒ यः । यो यज॑ते । यज॑ते॒ तस्मा᳚त् । तस्मा॑दे॒वम् । ए॒वमा॑ह । आ॒होत् । उदु॑ । उ॒ त्यम् । त्यम् जा॒तवे॑दसम् । जा॒तवे॑दस॒मिति॑ । जा॒तवे॑दस॒मिति॑ जा॒त - वे॒द॒स॒म् । इति॑ सौ॒र्या । सौ॒र्यर्चा । ऋ॒चा कृ॑ष्णाजि॒नम् । कृ॒ष्णा॒जि॒नम् प्र॒त्यान॑ह्यति । कृ॒ष्णा॒जि॒नमिति॑ कृष्ण - अ॒जि॒नम् । प्र॒त्यान॑ह्यति॒ रक्ष॑साम् । प्र॒त्यान॑ह्य॒तीति॑ प्रति - आन॑ह्यति । रक्ष॑सा॒मप॑हत्यै । अप॑हत्या॒ उस्रौ᳚ । अप॑हत्या॒ इत्यप॑ - ह॒त्यै॒ । उस्रा॒वा । एत᳚म् । इ॒त॒म् धू॒र्.॒षा॒हौ॒ । धू॒र्॒.षा॒हा॒विति॑ । धू॒र्॒.षा॒हा॒विति॑ धूः - सा॒हौ॒ । इत्या॑ह । आ॒ह॒ य॒था॒य॒जुः । य॒था॒य॒जुरे॒व । य॒था॒य॒जुरिति॑ यथा - य॒जुः । ए॒वैतत् । ए॒तत् प्र । प्र च्य॑वस्व । च्य॒व॒स्व॒ भु॒वः॒ । भु॒व॒स्प॒ते॒ । प॒त॒ इति॑ । इत्या॑ह । आ॒ह॒ भू॒ताना᳚म् । भू॒तानाꣳ॒॒ हि । ह्ये॑षः \newline

\textbf{Jatai Paata} \newline

1. इत्या॑हा॒हे तीत्या॑ह । \newline
2. आ॒ह॒ दि॒वि दि॒व्या॑ हाह दि॒वि । \newline
3. दि॒वि हि हि दि॒वि दि॒वि हि । \newline
4. हि सूर्यꣳ॒॒ सूर्यꣳ॒॒ हि हि सूर्य᳚म् । \newline
5. सूर्यꣳ॒॒ सोमꣳ॒॒ सोमꣳ॒॒ सूर्यꣳ॒॒ सूर्यꣳ॒॒ सोम᳚म् । \newline
6. सोम॒ मद्रा॒ वद्रौ॒ सोमꣳ॒॒ सोम॒ मद्रौ᳚ । \newline
7. अद्रा॒ विती त्यद्रा॒ वद्रा॒ विति॑ । \newline
8. इत्या॑हा॒हे तीत्या॑ह । \newline
9. आ॒ह॒ ग्रावा॑णो॒ ग्रावा॑ण आहाह॒ ग्रावा॑णः । \newline
10. ग्रावा॑णो॒ वै वै ग्रावा॑णो॒ ग्रावा॑णो॒ वै । \newline
11. वा अद्र॒यो ऽद्र॑यो॒ वै वा अद्र॑यः । \newline
12. अद्र॑य॒ स्तेषु॒ तेष्वद्र॒यो ऽद्र॑य॒ स्तेषु॑ । \newline
13. तेषु॒ वै वै तेषु॒ तेषु॒ वै । \newline
14. वा ए॒ष ए॒ष वै वा ए॒षः । \newline
15. ए॒ष सोमꣳ॒॒ सोम॑ मे॒ष ए॒ष सोम᳚म् । \newline
16. सोम॑म् दधाति दधाति॒ सोमꣳ॒॒ सोम॑म् दधाति । \newline
17. द॒धा॒ति॒ यो यो द॑धाति दधाति॒ यः । \newline
18. यो यज॑ते॒ यज॑ते॒ यो यो यज॑ते । \newline
19. यज॑ते॒ तस्मा॒त् तस्मा॒द् यज॑ते॒ यज॑ते॒ तस्मा᳚त् । \newline
20. तस्मा॑ दे॒व मे॒वम् तस्मा॒त् तस्मा॑ दे॒वम् । \newline
21. ए॒व मा॑हा है॒व मे॒व मा॑ह । \newline
22. आ॒होदु दा॑हा॒ होत् । \newline
23. उदु॑ वु॒वु दुदु॑ । \newline
24. उ॒ त्यम् त्य मु॑ वु॒ त्यम् । \newline
25. त्यम् जा॒तवे॑दसम् जा॒तवे॑दस॒म् त्यम् त्यम् जा॒तवे॑दसम् । \newline
26. जा॒तवे॑दस॒ मितीति॑ जा॒तवे॑दसम् जा॒तवे॑दस॒ मिति॑ । \newline
27. जा॒तवे॑दस॒मिति॑ जा॒त - वे॒द॒स॒म् । \newline
28. इति॑ सौ॒र्या सौ॒र्ये तीति॑ सौ॒र्या । \newline
29. सौ॒र्य र्‌च र्‌चा सौ॒र्या सौ॒र्य र्‌चा । \newline
30. ऋ॒चा कृ॑ष्णाजि॒नम् कृ॑ष्णाजि॒न मृ॒च र्‌चा कृ॑ष्णाजि॒नम् । \newline
31. कृ॒ष्णा॒जि॒नम् प्र॒त्यान॑ह्यति प्र॒त्यान॑ह्यति कृष्णाजि॒नम् कृ॑ष्णाजि॒नम् प्र॒त्यान॑ह्यति । \newline
32. कृ॒ष्णा॒जि॒नमिति॑ कृष्ण - अ॒जि॒नम् । \newline
33. प्र॒त्यान॑ह्यति॒ रक्ष॑साꣳ॒॒ रक्ष॑साम् प्र॒त्यान॑ह्यति प्र॒त्यान॑ह्यति॒ रक्ष॑साम् । \newline
34. प्र॒त्यान॑ह्य॒तीति॑ प्रति - आन॑ह्यति । \newline
35. रक्ष॑सा॒ मप॑हत्या॒ अप॑हत्यै॒ रक्ष॑साꣳ॒॒ रक्ष॑सा॒ मप॑हत्यै । \newline
36. अप॑हत्या॒ उस्रा॒ वुस्रा॒ वप॑हत्या॒ अप॑हत्या॒ उस्रौ᳚ । \newline
37. अप॑हत्या॒ इत्यप॑ - ह॒त्यै॒ । \newline
38. उस्रा॒ वोस्रा॒ वुस्रा॒ वा । \newline
39. एत॑ मित॒ मेत᳚म् । \newline
40. इ॒त॒म् धू॒र्॒.षा॒हौ॒ धू॒र्॒.षा॒हा॒ वि॒त॒ मि॒त॒म् धू॒र्॒.षा॒हौ॒ । \newline
41. धू॒र्॒.षा॒हा॒ वितीति॑ धूर्.षाहौ धूर्.षाहा॒ विति॑ । \newline
42. धू॒र्॒.षा॒हा॒विति॑ धूः - सा॒हौ॒ । \newline
43. इत्या॑हा॒हे तीत्या॑ह । \newline
44. आ॒ह॒ य॒था॒य॒जुर् य॑थाय॒जु रा॑हाह यथाय॒जुः । \newline
45. य॒था॒य॒जु रे॒वैव य॑थाय॒जुर् य॑थाय॒जु रे॒व । \newline
46. य॒था॒य॒जुरिति॑ यथा - य॒जुः । \newline
47. ए॒वैत दे॒त दे॒वै वैतत् । \newline
48. ए॒तत् प्र प्रैत दे॒तत् प्र । \newline
49. प्र च्य॑वस्व च्यवस्व॒ प्र प्र च्य॑वस्व । \newline
50. च्य॒व॒स्व॒ भु॒वो॒ भु॒व॒ श्च्य॒व॒स्व॒ च्य॒व॒स्व॒ भु॒वः॒ । \newline
51. भु॒व॒ स्प॒ते॒ प॒ते॒ भु॒वो॒ भु॒व॒ स्प॒ते॒ । \newline
52. प॒त॒ इतीति॑ पते पत॒ इति॑ । \newline
53. इत्या॑हा॒हे तीत्या॑ह । \newline
54. आ॒ह॒ भू॒ताना᳚म् भू॒ताना॑ माहाह भू॒ताना᳚म् । \newline
55. भू॒तानाꣳ॒॒ हि हि भू॒ताना᳚म् भू॒तानाꣳ॒॒ हि । \newline
56. ह्ये॑ष ए॒ष हि ह्ये॑षः । \newline

\textbf{Ghana Paata } \newline

1. इत्या॑हा॒हे तीत्या॑ह दि॒वि दि॒व्या॑हे तीत्या॑ह दि॒वि । \newline
2. आ॒ह॒ दि॒वि दि॒व्या॑ हाह दि॒वि हि हि दि॒व्या॑ हाह दि॒वि हि । \newline
3. दि॒वि हि हि दि॒वि दि॒वि हि सूर्यꣳ॒॒ सूर्यꣳ॒॒ हि दि॒वि दि॒वि हि सूर्य᳚म् । \newline
4. हि सूर्यꣳ॒॒ सूर्यꣳ॒॒ हि हि सूर्यꣳ॒॒ सोमꣳ॒॒ सोमꣳ॒॒ सूर्यꣳ॒॒ हि हि सूर्यꣳ॒॒ सोम᳚म् । \newline
5. सूर्यꣳ॒॒ सोमꣳ॒॒ सोमꣳ॒॒ सूर्यꣳ॒॒ सूर्यꣳ॒॒ सोम॒ मद्रा॒ वद्रौ॒ सोमꣳ॒॒ सूर्यꣳ॒॒ सूर्यꣳ॒॒ सोम॒ मद्रौ᳚ । \newline
6. सोम॒ मद्रा॒ वद्रौ॒ सोमꣳ॒॒ सोम॒ मद्रा॒ विती त्यद्रौ॒ सोमꣳ॒॒ सोम॒ मद्रा॒ विति॑ । \newline
7. अद्रा॒ विती त्यद्रा॒ वद्रा॒ वित्या॑ हा॒हे त्यद्रा॒ वद्रा॒ वित्या॑ह । \newline
8. इत्या॑हा॒हे तीत्या॑ह॒ ग्रावा॑णो॒ ग्रावा॑ण आ॒हे तीत्या॑ह॒ ग्रावा॑णः । \newline
9. आ॒ह॒ ग्रावा॑णो॒ ग्रावा॑ण आहाह॒ ग्रावा॑णो॒ वै वै ग्रावा॑ण आहाह॒ ग्रावा॑णो॒ वै । \newline
10. ग्रावा॑णो॒ वै वै ग्रावा॑णो॒ ग्रावा॑णो॒ वा अद्र॒यो ऽद्र॑यो॒ वै ग्रावा॑णो॒ ग्रावा॑णो॒ वा अद्र॑यः । \newline
11. वा अद्र॒यो ऽद्र॑यो॒ वै वा अद्र॑य॒ स्तेषु॒ तेष्व द्र॑यो॒ वै वा अद्र॑य॒ स्तेषु॑ । \newline
12. अद्र॑य॒ स्तेषु॒ तेष्व द्र॒यो ऽद्र॑य॒ स्तेषु॒ वै वै तेष्व द्र॒यो ऽद्र॑य॒ स्तेषु॒ वै । \newline
13. तेषु॒ वै वै तेषु॒ तेषु॒ वा ए॒ष ए॒ष वै तेषु॒ तेषु॒ वा ए॒षः । \newline
14. वा ए॒ष ए॒ष वै वा ए॒ष सोमꣳ॒॒ सोम॑ मे॒ष वै वा ए॒ष सोम᳚म् । \newline
15. ए॒ष सोमꣳ॒॒ सोम॑ मे॒ष ए॒ष सोम॑म् दधाति दधाति॒ सोम॑ मे॒ष ए॒ष सोम॑म् दधाति । \newline
16. सोम॑म् दधाति दधाति॒ सोमꣳ॒॒ सोम॑म् दधाति॒ यो यो द॑धाति॒ सोमꣳ॒॒ सोम॑म् दधाति॒ यः । \newline
17. द॒धा॒ति॒ यो यो द॑धाति दधाति॒ यो यज॑ते॒ यज॑ते॒ यो द॑धाति दधाति॒ यो यज॑ते । \newline
18. यो यज॑ते॒ यज॑ते॒ यो यो यज॑ते॒ तस्मा॒त् तस्मा॒द् यज॑ते॒ यो यो यज॑ते॒ तस्मा᳚त् । \newline
19. यज॑ते॒ तस्मा॒त् तस्मा॒द् यज॑ते॒ यज॑ते॒ तस्मा॑ दे॒व मे॒वम् तस्मा॒द् यज॑ते॒ यज॑ते॒ तस्मा॑ दे॒वम् । \newline
20. तस्मा॑ दे॒व मे॒वम् तस्मा॒त् तस्मा॑ दे॒व मा॑हा है॒वम् तस्मा॒त् तस्मा॑ दे॒व मा॑ह । \newline
21. ए॒व मा॑हा है॒व मे॒व मा॒होदु दा॑है॒व मे॒व मा॒होत् । \newline
22. आ॒होदु दा॑हा॒ होदु॑ वु॒ वुदा॑हा॒ होदु॑ । \newline
23. उदु॑ वु॒ वुदुदु॒ त्यम् त्य मु॒ वुदुदु॒ त्यम् । \newline
24. उ॒ त्यम् त्य मु॑ वु॒ त्यम् जा॒तवे॑दसम् जा॒तवे॑दस॒म् त्य मु॑ वु॒ त्यम् जा॒तवे॑दसम् । \newline
25. त्यम् जा॒तवे॑दसम् जा॒तवे॑दस॒म् त्यम् त्यम् जा॒तवे॑दस॒ मितीति॑ जा॒तवे॑दस॒म् त्यम् त्यम् जा॒तवे॑दस॒ मिति॑ । \newline
26. जा॒तवे॑दस॒ मितीति॑ जा॒तवे॑दसम् जा॒तवे॑दस॒ मिति॑ सौ॒र्या सौ॒र्येति॑ जा॒तवे॑दसम् जा॒तवे॑दस॒ मिति॑ सौ॒र्या । \newline
27. जा॒तवे॑दस॒मिति॑ जा॒त - वे॒द॒स॒म् । \newline
28. इति॑ सौ॒र्या सौ॒र् येतीति॑ सौ॒र्य र्‌च र्‌चा सौ॒ र्येतीति॑ सौ॒र्य र्‌चा । \newline
29. सौ॒र्य र्‌च र्‌चा सौ॒र्या सौ॒र्य र्‌चा कृ॑ष्णाजि॒नम् कृ॑ष्णाजि॒न मृ॒चा सौ॒र्या सौ॒र्य र्‌चा कृ॑ष्णाजि॒नम् । \newline
30. ऋ॒चा कृ॑ष्णाजि॒नम् कृ॑ष्णाजि॒न मृ॒च र्‌चा कृ॑ष्णाजि॒नम् प्र॒त्यान॑ह्यति प्र॒त्यान॑ह्यति कृष्णाजि॒न मृ॒च र्‌चा कृ॑ष्णाजि॒नम् प्र॒त्यान॑ह्यति । \newline
31. कृ॒ष्णा॒जि॒नम् प्र॒त्यान॑ह्यति प्र॒त्यान॑ह्यति कृष्णाजि॒नम् कृ॑ष्णाजि॒नम् प्र॒त्यान॑ह्यति॒ रक्ष॑साꣳ॒॒ रक्ष॑साम् प्र॒त्यान॑ह्यति कृष्णाजि॒नम् कृ॑ष्णाजि॒नम् प्र॒त्यान॑ह्यति॒ रक्ष॑साम् । \newline
32. कृ॒ष्णा॒जि॒नमिति॑ कृष्ण - अ॒जि॒नम् । \newline
33. प्र॒त्यान॑ह्यति॒ रक्ष॑साꣳ॒॒ रक्ष॑साम् प्र॒त्यान॑ह्यति प्र॒त्यान॑ह्यति॒ रक्ष॑सा॒ मप॑हत्या॒ अप॑हत्यै॒ रक्ष॑साम् प्र॒त्यान॑ह्यति प्र॒त्यान॑ह्यति॒ रक्ष॑सा॒ मप॑हत्यै । \newline
34. प्र॒त्यान॑ह्य॒तीति॑ प्रति - आन॑ह्यति । \newline
35. रक्ष॑सा॒ मप॑हत्या॒ अप॑हत्यै॒ रक्ष॑साꣳ॒॒ रक्ष॑सा॒ मप॑हत्या॒ उस्रा॒ वुस्रा॒ वप॑हत्यै॒ रक्ष॑साꣳ॒॒ रक्ष॑सा॒ मप॑हत्या॒ उस्रौ᳚ । \newline
36. अप॑हत्या॒ उस्रा॒ वुस्रा॒ वप॑हत्या॒ अप॑हत्या॒ उस्रा॒ वोस्रा॒ वप॑हत्या॒ अप॑हत्या॒ उस्रा॒ वा । \newline
37. अप॑हत्या॒ इत्यप॑ - ह॒त्यै॒ । \newline
38. उस्रा॒ वोस्रा॒ वुस्रा॒ वेत॑ मित॒ मोस्रा॒ वुस्रा॒ वेत᳚म् । \newline
39. एत॑ मित॒ मेत॑म् धूर्.षाहौ धूर्.षाहा वित॒ मेत॑म् धूर्.षाहौ । \newline
40. इ॒त॒म् धू॒र्॒.षा॒हौ॒ धू॒र्॒.षा॒हा॒ वि॒त॒ मि॒त॒म् धू॒र्॒.षा॒हा॒ वितीति॑ धूर्.षाहा वित मितम् धूर्.षाहा॒ विति॑ । \newline
41. धू॒र्॒.षा॒हा॒ वितीति॑ धूर्.षाहौ धूर्.षाहा॒ वित्या॑हा॒ हेति॑ धूर्.षाहौ धूर्.षाहा॒ वित्या॑ह । \newline
42. धू॒र्॒.षा॒हा॒विति॑ धूः - सा॒हौ॒ । \newline
43. इत्या॑हा॒हे तीत्या॑ह यथाय॒जुर् य॑थाय॒जु रा॒हे तीत्या॑ह यथाय॒जुः । \newline
44. आ॒ह॒ य॒था॒य॒जुर् य॑थाय॒जु रा॑हाह यथाय॒जु रे॒वैव य॑थाय॒जु रा॑हाह यथाय॒जु रे॒व । \newline
45. य॒था॒य॒जु रे॒वैव य॑थाय॒जुर् य॑थाय॒जु रे॒वैत दे॒त दे॒व य॑थाय॒जुर् य॑थाय॒जु रे॒वैतत् । \newline
46. य॒था॒य॒जुरिति॑ यथा - य॒जुः । \newline
47. ए॒वैत दे॒त दे॒वै वैतत् प्र प्रैत दे॒वै वैतत् प्र । \newline
48. ए॒तत् प्र प्रैत दे॒तत् प्र च्य॑वस्व च्यवस्व॒ प्रैत दे॒तत् प्र च्य॑वस्व । \newline
49. प्र च्य॑वस्व च्यवस्व॒ प्र प्र च्य॑वस्व भुवो भुव श्च्यवस्व॒ प्र प्र च्य॑वस्व भुवः । \newline
50. च्य॒व॒स्व॒ भु॒वो॒ भु॒व॒ श्च्य॒व॒स्व॒ च्य॒व॒स्व॒ भु॒व॒ स्प॒ते॒ प॒ते॒ भु॒व॒ श्च्य॒व॒स्व॒ च्य॒व॒स्व॒ भु॒व॒ स्प॒ते॒ । \newline
51. भु॒व॒ स्प॒ते॒ प॒ते॒ भु॒वो॒ भु॒व॒ स्प॒त॒ इतीति॑ पते भुवो भुव स्पत॒ इति॑ । \newline
52. प॒त॒ इतीति॑ पते पत॒ इत्या॑हा॒ हेति॑ पते पत॒ इत्या॑ह । \newline
53. इत्या॑हा॒हे तीत्या॑ह भू॒ताना᳚म् भू॒ताना॑ मा॒हे तीत्या॑ह भू॒ताना᳚म् । \newline
54. आ॒ह॒ भू॒ताना᳚म् भू॒ताना॑ माहाह भू॒तानाꣳ॒॒ हि हि भू॒ताना॑ माहाह भू॒तानाꣳ॒॒ हि । \newline
55. भू॒तानाꣳ॒॒ हि हि भू॒ताना᳚म् भू॒तानाꣳ॒॒ ह्ये॑ष ए॒ष हि भू॒ताना᳚म् भू॒तानाꣳ॒॒ ह्ये॑षः । \newline
56. ह्ये॑ष ए॒ष हि ह्ये॑ष पति॒ष् पति॑ रे॒ष हि ह्ये॑ष पतिः॑ । \newline
\pagebreak
\markright{ TS 6.1.11.5  \hfill https://www.vedavms.in \hfill}

\section{ TS 6.1.11.5 }

\textbf{TS 6.1.11.5 } \newline
\textbf{Samhita Paata} \newline

-ष पति॒र्विश्वा᳚न्य॒भि धामा॒नीत्या॑ह॒ विश्वा॑नि॒ ह्ये᳚(1॒) षो॑ऽभि धामा॑नि प्र॒च्यव॑ते॒ मा त्वा॑ परिप॒री वि॑द॒दित्या॑ह॒ यदे॒वादः सोम॑माह्रि॒यमा॑णं गन्ध॒र्वो वि॒श्वाव॑सुः प॒र्यमु॑ष्णा॒त् तस्मा॑-दे॒वमा॒हाप॑रिमोषाय॒ यज॑मानस्य स्व॒स्त्यय॑न्य॒सीत्या॑ह॒ यज॑मानस्यै॒वैष य॒ज्ञ्स्या᳚न्वार॒भ्ॐ ऽन॑वच्छित्त्यै॒ वरु॑णो॒ वा ए॒ष यज॑मानम॒भ्यैति॒ यत् - [  ] \newline

\textbf{Pada Paata} \newline

ए॒षः । पतिः॑ । विश्वा॑नि । अ॒भीति॑ । धामा॑नि । इति॑ । आ॒ह॒ । विश्वा॑नि । हि । ए॒षः । अ॒भीति॑ । धामा॑नि । प्र॒च्यव॑त॒ इति॑ प्र - च्यव॑ते । मा । त्वा॒ । प॒रि॒प॒रीति॑ परि - प॒री । वि॒द॒त् । इति॑ । आ॒ह॒ । यत् । ए॒व । अ॒दः । सोम᳚म् । आ॒ह्रि॒यमा॑ण॒मित्या᳚ - ह्रि॒यमा॑णम् । ग॒न्ध॒र्वः । वि॒श्वाव॑सु॒रिति॑ वि॒श्व - व॒सुः॒ । प॒र्यमु॑ष्णा॒दिति॑ परि - अमु॑ष्णात् । तस्मा᳚त् । ए॒वम् । आ॒ह॒ । अप॑रिमोषा॒येत्यप॑रि - मो॒षा॒य॒ । यज॑मानस्य । स्व॒स्त्यय॒नीति॑ स्वस्ति - अय॑नी । अ॒सि॒ । इति॑ । आ॒ह॒ । यज॑मानस्य । ए॒व । ए॒षः । य॒ज्ञ्स्य॑ । अ॒न्वा॒र॒म्भ इत्य॑नु - आ॒र॒म्भः । अन॑वच्छित्त्या॒ इत्यन॑व - छि॒त्यै॒ । वरु॑णः । वै । ए॒षः । यज॑मानम् । अ॒भि । एति॑ । ए॒ति॒ । यत् ।  \newline


\textbf{Krama Paata} \newline

ए॒ष पतिः॑ । पति॒र् विश्वा॑नि । विश्वा᳚न्य॒भि । अ॒भि धामा॑नि । धामा॒नीति॑ । इत्या॑ह । आ॒ह॒ विश्वा॑नि । विश्वा॑नि॒ हि । ह्ये॑षः । ए॒षो॑ऽभि । अ॒भि धामा॑नि । धामा॑नि प्र॒च्यव॑ते । प्र॒च्यव॑ते॒ मा । प्र॒च्य॑वत॒ इति॑ प्र - च्यव॑ते । मा त्वा᳚ । त्वा॒ प॒रि॒प॒री । प॒रि॒प॒री वि॑दत् । प॒रि॒प॒रीति॑ परि - प॒री । वि॒द॒दिति॑ । इत्या॑ह । आ॒ह॒ यत् । यदे॒व । ए॒वादः । अ॒दः सोम᳚म् । सोम॑माह्रि॒यमा॑णम् । आ॒ह्रि॒यमा॑णम् गन्ध॒र्वः । आ॒ह्रि॒यमा॑ण॒मित्या᳚ - ह्रि॒यमा॑णम् । ग॒न्ध॒र्वो वि॒श्वाव॑सुः । वि॒श्वाव॑सु प॒र्यमु॑ष्णात् । वि॒श्वाव॑सु॒रिति॑ वि॒श्व - व॒सुः॒ । प॒र्यमु॑ष्णा॒त् तस्मा᳚त् । प॒र्यमु॑ष्णा॒दिति॑ परि - अमु॑ष्णात् । तस्मा॑दे॒वम् । ए॒वमा॑ह । आ॒हाप॑रिमोषाय । अप॑रिमोषाय॒ यज॑मानस्य । अप॑रिमोषा॒येत्यप॑रि - मो॒षा॒य॒ । यज॑मानस्य स्व॒स्त्यय॑नी । स्व॒स्त्यय॑न्यसि । स्व॒स्त्यय॒नीति॑ स्वस्ति - अय॑नी । अ॒सीति॑ । इत्या॑ह । आ॒ह॒ यज॑मानस्य । यज॑मानस्यै॒व । ए॒वैषः । ए॒ष य॒ज्ञ्स्य॑ । य॒ज्ञ्स्या᳚न्वार॒म्भः । अ॒न्वा॒र॒म्भोऽन॑वच्छित्त्यै । अ॒न्वा॒र॒म्भ इत्य॑नु - आ॒र॒म्भः । अन॑वच्छित्त्यै॒ वरु॑णः । अन॑वच्छित्त्या॒ इत्यन॑व - छि॒त्त्यै॒ । वरु॑णो॒ वै । वा ए॒षः । ए॒ष यज॑मानम् । यज॑मानम॒भि । अ॒भ्या । ऐति॑ । ए॒ति॒ यत् । यत् क्री॒तः \newline

\textbf{Jatai Paata} \newline

1. ए॒ष पति॒ष् पति॑ रे॒ष ए॒ष पतिः॑ । \newline
2. पति॒र् विश्वा॑नि॒ विश्वा॑नि॒ पति॒ष् पति॒र् विश्वा॑नि । \newline
3. विश्वा᳚न्य॒ भ्य॑भि विश्वा॑नि॒ विश्वा᳚ न्य॒भि । \newline
4. अ॒भि धामा॑नि॒ धामा᳚ न्य॒भ्य॑भि धामा॑नि । \newline
5. धामा॒नी तीति॒ धामा॑नि॒ धामा॒ नीति॑ । \newline
6. इत्या॑हा॒हे तीत्या॑ह । \newline
7. आ॒ह॒ विश्वा॑नि॒ विश्वा᳚ न्याहाह॒ विश्वा॑नि । \newline
8. विश्वा॑नि॒ हि हि विश्वा॑नि॒ विश्वा॑नि॒ हि । \newline
9. ह्ये॑ष ए॒ष हि ह्ये॑षः । \newline
10. ए॒षो᳚(1॒) ऽभ्या᳚(1॒)भ्ये॑ष ए॒षो॑ ऽभि । \newline
11. अ॒भि धामा॑नि॒ धामा᳚ न्य॒भ्य॑भि धामा॑नि । \newline
12. धामा॑नि प्र॒च्यव॑ते प्र॒च्यव॑ते॒ धामा॑नि॒ धामा॑नि प्र॒च्यव॑ते । \newline
13. प्र॒च्यव॑ते॒ मा मा प्र॒च्यव॑ते प्र॒च्यव॑ते॒ मा । \newline
14. प्र॒च्यव॑त॒ इति॑ प्र - च्यव॑ते । \newline
15. मा त्वा᳚ त्वा॒ मा मा त्वा᳚ । \newline
16. त्वा॒ प॒रि॒प॒री प॑रिप॒री त्वा᳚ त्वा परिप॒री । \newline
17. प॒रि॒प॒री वि॑दद् विदत् परिप॒री प॑रिप॒री वि॑दत् । \newline
18. प॒रि॒प॒रीति॑ परि - प॒री । \newline
19. वि॒द॒ दितीति॑ विदद् विद॒ दिति॑ । \newline
20. इत्या॑हा॒हे तीत्या॑ह । \newline
21. आ॒ह॒ यद् यदा॑हाह॒ यत् । \newline
22. यदे॒वैव यद् यदे॒व । \newline
23. ए॒वादो॑ ऽद ए॒वैवादः । \newline
24. अ॒दः सोमꣳ॒॒ सोम॑ म॒दो॑ ऽदः सोम᳚म् । \newline
25. सोम॑ माह्रि॒यमा॑ण माह्रि॒यमा॑णꣳ॒॒ सोमꣳ॒॒ सोम॑ माह्रि॒यमा॑णम् । \newline
26. आ॒ह्रि॒यमा॑णम् गन्ध॒र्वो ग॑न्ध॒र्व आ᳚ह्रि॒यमा॑ण माह्रि॒यमा॑णम् गन्ध॒र्वः । \newline
27. आ॒ह्रि॒यमा॑ण॒मित्या᳚ - ह्रि॒यमा॑णम् । \newline
28. ग॒न्ध॒र्वो वि॒श्वाव॑सुर् वि॒श्वाव॑सुर् गन्ध॒र्वो ग॑न्ध॒र्वो वि॒श्वाव॑सुः । \newline
29. वि॒श्वाव॑सुः प॒र्यमु॑ष्णात् प॒र्यमु॑ष्णाद् वि॒श्वाव॑सुर् वि॒श्वाव॑सुः प॒र्यमु॑ष्णात् । \newline
30. वि॒श्वाव॑सु॒रिति॑ वि॒श्व - व॒सुः॒ । \newline
31. प॒र्यमु॑ष्णा॒त् तस्मा॒त् तस्मा᳚त् प॒र्यमु॑ष्णात् प॒र्यमु॑ष्णा॒त् तस्मा᳚त् । \newline
32. प॒र्यमु॑ष्णा॒दिति॑ परि - अमु॑ष्णात् । \newline
33. तस्मा॑ दे॒व मे॒वम् तस्मा॒त् तस्मा॑ दे॒वम् । \newline
34. ए॒व मा॑हा है॒व मे॒व मा॑ह । \newline
35. आ॒हा प॑रिमोषा॒या प॑रिमोषाया हा॒हा प॑रिमोषाय । \newline
36. अप॑रिमोषाय॒ यज॑मानस्य॒ यज॑मान॒स्या प॑रिमोषा॒या प॑रिमोषाय॒ यज॑मानस्य । \newline
37. अप॑रिमोषा॒येत्यप॑रि - मो॒षा॒य॒ । \newline
38. यज॑मानस्य स्व॒स्त्यय॑नी स्व॒स्त्यय॑नी॒ यज॑मानस्य॒ यज॑मानस्य स्व॒स्त्यय॑नी । \newline
39. स्व॒स्त्यय॑ न्यस्यसि स्व॒स्त्यय॑नी स्व॒स्त्यय॑ न्यसि । \newline
40. स्व॒स्त्यय॒नीति॑ स्वस्ति - अय॑नी । \newline
41. अ॒सीती त्य॑स्य॒ सीति॑ । \newline
42. इत्या॑हा॒हे तीत्या॑ह । \newline
43. आ॒ह॒ यज॑मानस्य॒ यज॑मानस्या हाह॒ यज॑मानस्य । \newline
44. यज॑मान स्यै॒वैव यज॑मानस्य॒ यज॑मान स्यै॒व । \newline
45. ए॒वैष ए॒ष ए॒वै वैषः । \newline
46. ए॒ष य॒ज्ञ्स्य॑ य॒ज्ञ् स्यै॒ष ए॒ष य॒ज्ञ्स्य॑ । \newline
47. य॒ज्ञ्स्या᳚ न्वार॒म्भो᳚ ऽन्वार॒म्भो य॒ज्ञ्स्य॑ य॒ज्ञ्स्या᳚ न्वार॒म्भः । \newline
48. अ॒न्वा॒र॒म्भो ऽन॑वच्छित्त्या॒ अन॑वच्छित्त्या अन्वार॒म्भो᳚ ऽन्वार॒म्भो ऽन॑वच्छित्त्यै । \newline
49. अ॒न्वा॒र॒म्भ इत्य॑नु - आ॒र॒म्भः । \newline
50. अन॑वच्छित्त्यै॒ वरु॑णो॒ वरु॒णो ऽन॑वच्छित्त्या॒ अन॑वच्छित्त्यै॒ वरु॑णः । \newline
51. अन॑वच्छित्त्या॒ इत्यन॑व - छि॒त्त्यै॒ । \newline
52. वरु॑णो॒ वै वै वरु॑णो॒ वरु॑णो॒ वै । \newline
53. वा ए॒ष ए॒ष वै वा ए॒षः । \newline
54. ए॒ष यज॑मानं॒ ॅयज॑मान मे॒ष ए॒ष यज॑मानम् । \newline
55. यज॑मान म॒भ्य॑भि यज॑मानं॒ ॅयज॑मान म॒भि । \newline
56. अ॒भ्या ऽभ्य॑भ्या । \newline
57. ऐत्ये॒ त्यैति॑ । \newline
58. ए॒ति॒ यद् यदे᳚ त्येति॒ यत् । \newline
59. यत् क्री॒तः क्री॒तो यद् यत् क्री॒तः । \newline

\textbf{Ghana Paata } \newline

1. ए॒ष पति॒ष् पति॑ रे॒ष ए॒ष पति॒र् विश्वा॑नि॒ विश्वा॑नि॒ पति॑ रे॒ष ए॒ष पति॒र् विश्वा॑नि । \newline
2. पति॒र् विश्वा॑नि॒ विश्वा॑नि॒ पति॒ष् पति॒र् विश्वा᳚ न्य॒भ्य॑भि विश्वा॑नि॒ पति॒ष् पति॒र् विश्वा᳚ न्य॒भि । \newline
3. विश्वा᳚ न्य॒भ्य॑भि विश्वा॑नि॒ विश्वा᳚ न्य॒भि धामा॑नि॒ धामा᳚ न्य॒भि विश्वा॑नि॒ विश्वा᳚ न्य॒भि धामा॑नि । \newline
4. अ॒भि धामा॑नि॒ धामा᳚ न्य॒भ्य॑भि धामा॒नी तीति॒ धामा᳚ न्य॒भ्य॑भि धामा॒ नीति॑ । \newline
5. धामा॒नी तीति॒ धामा॑नि॒ धामा॒नीत्या॑ हा॒हेति॒ धामा॑नि॒ धामा॒नी त्या॑ह । \newline
6. इत्या॑हा॒हे तीत्या॑ह॒ विश्वा॑नि॒ विश्वा᳚न्या॒हे तीत्या॑ह॒ विश्वा॑नि । \newline
7. आ॒ह॒ विश्वा॑नि॒ विश्वा᳚ न्याहाह॒ विश्वा॑नि॒ हि हि विश्वा᳚ न्याहाह॒ विश्वा॑नि॒ हि । \newline
8. विश्वा॑नि॒ हि हि विश्वा॑नि॒ विश्वा॑नि॒ ह्ये॑ष ए॒ष हि विश्वा॑नि॒ विश्वा॑नि॒ ह्ये॑षः । \newline
9. ह्ये॑ष ए॒ष हि ह्ये᳚(1॒)षो᳚(1॒) ऽभ्या᳚(1॒)भ्ये॑ष हि ह्ये᳚(1॒)षो॑ ऽभि । \newline
10. ए॒षो᳚(1॒) ऽभ्या᳚(1॒)भ्ये॑ष ए॒षो॑ ऽभि धामा॑नि॒ धामा᳚ न्य॒भ्ये॑ष ए॒षो॑ ऽभि धामा॑नि । \newline
11. अ॒भि धामा॑नि॒ धामा᳚ न्य॒भ्य॑भि धामा॑नि प्र॒च्यव॑ते प्र॒च्यव॑ते॒ धामा᳚ न्य॒भ्य॑भि धामा॑नि प्र॒च्यव॑ते । \newline
12. धामा॑नि प्र॒च्यव॑ते प्र॒च्यव॑ते॒ धामा॑नि॒ धामा॑नि प्र॒च्यव॑ते॒ मा मा प्र॒च्यव॑ते॒ धामा॑नि॒ धामा॑नि प्र॒च्यव॑ते॒ मा । \newline
13. प्र॒च्यव॑ते॒ मा मा प्र॒च्यव॑ते प्र॒च्यव॑ते॒ मा त्वा᳚ त्वा॒ मा प्र॒च्यव॑ते प्र॒च्यव॑ते॒ मा त्वा᳚ । \newline
14. प्र॒च्यव॑त॒ इति॑ प्र - च्यव॑ते । \newline
15. मा त्वा᳚ त्वा॒ मा मा त्वा॑ परिप॒री प॑रिप॒री त्वा॒ मा मा त्वा॑ परिप॒री । \newline
16. त्वा॒ प॒रि॒प॒री प॑रिप॒री त्वा᳚ त्वा परिप॒री वि॑दद् विदत् परिप॒री त्वा᳚ त्वा परिप॒री वि॑दत् । \newline
17. प॒रि॒प॒री वि॑दद् विदत् परिप॒री प॑रिप॒री वि॑द॒दितीति॑ विदत् परिप॒री प॑रिप॒री वि॑द॒दिति॑ । \newline
18. प॒रि॒प॒रीति॑ परि - प॒री । \newline
19. वि॒द॒ दितीति॑ विदद् विद॒ दित्या॑हा॒ हेति॑ विदद् विद॒ दित्या॑ह । \newline
20. इत्या॑हा॒हे तीत्या॑ह॒ यद् यदा॒हे तीत्या॑ह॒ यत् । \newline
21. आ॒ह॒ यद् यदा॑हाह॒ यदे॒ वैव यदा॑हाह॒ यदे॒व । \newline
22. यदे॒वैव यद् यदे॒वादो॑ ऽद ए॒व यद् यदे॒वादः । \newline
23. ए॒वादो॑ ऽद ए॒वै वादः सोमꣳ॒॒ सोम॑ म॒द ए॒वै वादः सोम᳚म् । \newline
24. अ॒दः सोमꣳ॒॒ सोम॑ म॒दो॑ ऽदः सोम॑ माह्रि॒यमा॑ण माह्रि॒यमा॑णꣳ॒॒ सोम॑ म॒दो॑ ऽदः सोम॑ माह्रि॒यमा॑णम् । \newline
25. सोम॑ माह्रि॒यमा॑ण माह्रि॒यमा॑णꣳ॒॒ सोमꣳ॒॒ सोम॑ माह्रि॒यमा॑णम् गन्ध॒र्वो ग॑न्ध॒र्व आ᳚ह्रि॒यमा॑णꣳ॒॒ सोमꣳ॒॒ सोम॑ माह्रि॒यमा॑णम् गन्ध॒र्वः । \newline
26. आ॒ह्रि॒यमा॑णम् गन्ध॒र्वो ग॑न्ध॒र्व आ᳚ह्रि॒यमा॑ण माह्रि॒यमा॑णम् गन्ध॒र्वो वि॒श्वाव॑सुर् वि॒श्वाव॑सुर् गन्ध॒र्व आ᳚ह्रि॒यमा॑ण माह्रि॒यमा॑णम् गन्ध॒र्वो वि॒श्वाव॑सुः । \newline
27. आ॒ह्रि॒यमा॑ण॒मित्या᳚ - ह्रि॒यमा॑णम् । \newline
28. ग॒न्ध॒र्वो वि॒श्वाव॑सुर् वि॒श्वाव॑सुर् गन्ध॒र्वो ग॑न्ध॒र्वो वि॒श्वाव॑सुः प॒र्यमु॑ष्णात् प॒र्यमु॑ष्णाद् वि॒श्वाव॑सुर् गन्ध॒र्वो ग॑न्ध॒र्वो वि॒श्वाव॑सुः प॒र्यमु॑ष्णात् । \newline
29. वि॒श्वाव॑सुः प॒र्यमु॑ष्णात् प॒र्यमु॑ष्णाद् वि॒श्वाव॑सुर् वि॒श्वाव॑सुः प॒र्यमु॑ष्णा॒त् तस्मा॒त् तस्मा᳚त् प॒र्यमु॑ष्णाद् वि॒श्वाव॑सुर् वि॒श्वाव॑सुः प॒र्यमु॑ष्णा॒त् तस्मा᳚त् । \newline
30. वि॒श्वाव॑सु॒रिति॑ वि॒श्व - व॒सुः॒ । \newline
31. प॒र्यमु॑ष्णा॒त् तस्मा॒त् तस्मा᳚त् प॒र्यमु॑ष्णात् प॒र्यमु॑ष्णा॒त् तस्मा॑ दे॒व मे॒वम् तस्मा᳚त् प॒र्यमु॑ष्णात् प॒र्यमु॑ष्णा॒त् तस्मा॑ दे॒वम् । \newline
32. प॒र्यमु॑ष्णा॒दिति॑ परि - अमु॑ष्णात् । \newline
33. तस्मा॑ दे॒व मे॒वम् तस्मा॒त् तस्मा॑ दे॒व मा॑हा है॒वम् तस्मा॒त् तस्मा॑ दे॒व मा॑ह । \newline
34. ए॒व मा॑हा है॒व मे॒व मा॒हा प॑रिमोषा॒या प॑रिमोषाया है॒व मे॒व मा॒हा प॑रिमोषाय । \newline
35. आ॒हा प॑रिमोषा॒या प॑रिमोषाया हा॒हा प॑रिमोषाय॒ यज॑मानस्य॒ यज॑मान॒स्या प॑रिमोषाया हा॒हा प॑रिमोषाय॒ यज॑मानस्य । \newline
36. अप॑रिमोषाय॒ यज॑मानस्य॒ यज॑मान॒स्या प॑रिमोषा॒या प॑रिमोषाय॒ यज॑मानस्य स्व॒स्त्यय॑नी स्व॒स्त्यय॑नी॒ यज॑मान॒स्या प॑रिमोषा॒या प॑रिमोषाय॒ यज॑मानस्य स्व॒स्त्यय॑नी । \newline
37. अप॑रिमोषा॒येत्यप॑रि - मो॒षा॒य॒ । \newline
38. यज॑मानस्य स्व॒स्त्यय॑नी स्व॒स्त्यय॑नी॒ यज॑मानस्य॒ यज॑मानस्य स्व॒स्त्यय॑ न्यस्यसि स्व॒स्त्यय॑नी॒ यज॑मानस्य॒ यज॑मानस्य स्व॒स्त्यय॑ न्यसि । \newline
39. स्व॒स्त्यय॑ न्यस्यसि स्व॒स्त्यय॑नी स्व॒स्त्यय॑ न्य॒सी तीत्य॑सि स्व॒स्त्यय॑नी स्व॒स्त्यय॑ न्य॒सीति॑ । \newline
40. स्व॒स्त्यय॒नीति॑ स्वस्ति - अय॑नी । \newline
41. अ॒सीती त्य॑स्य॒ सीत्या॑ हा॒हे त्य॑स्य॒ सीत्या॑ह । \newline
42. इत्या॑हा॒हे तीत्या॑ह॒ यज॑मानस्य॒ यज॑मानस्या॒हे तीत्या॑ह॒ यज॑मानस्य । \newline
43. आ॒ह॒ यज॑मानस्य॒ यज॑मानस्या हाह॒ यज॑मान स्यै॒वैव यज॑मानस्याहाह॒ यज॑मानस्यै॒व । \newline
44. यज॑मान स्यै॒वैव यज॑मानस्य॒ यज॑मान स्यै॒वैष ए॒ष ए॒व यज॑मानस्य॒ यज॑मान स्यै॒वैषः । \newline
45. ए॒वैष ए॒ष ए॒वैवैष य॒ज्ञ्स्य॑ य॒ज्ञ्स्यै॒ष ए॒वैवैष य॒ज्ञ्स्य॑ । \newline
46. ए॒ष य॒ज्ञ्स्य॑ य॒ज्ञ्स्यै॒ष ए॒ष य॒ज्ञ्स्या᳚ न्वार॒म्भो᳚ ऽन्वार॒म्भो य॒ज्ञ्स्यै॒ष ए॒ष य॒ज्ञ्स्या᳚ न्वार॒म्भः । \newline
47. य॒ज्ञ्स्या᳚ न्वार॒म्भो᳚ ऽन्वार॒म्भो य॒ज्ञ्स्य॑ य॒ज्ञ्स्या᳚ न्वार॒म्भो ऽन॑वच्छित्त्या॒ अन॑वच्छित्त्या अन्वार॒म्भो य॒ज्ञ्स्य॑ य॒ज्ञ्स्या᳚ न्वार॒म्भो ऽन॑वच्छित्त्यै । \newline
48. अ॒न्वा॒र॒म्भो ऽन॑वच्छित्त्या॒ अन॑वच्छित्त्या अन्वार॒म्भो᳚ ऽन्वार॒म्भो ऽन॑वच्छित्त्यै॒ वरु॑णो॒ वरु॒णो ऽन॑वच्छित्त्या अन्वार॒म्भो᳚ ऽन्वार॒म्भो ऽन॑वच्छित्त्यै॒ वरु॑णः । \newline
49. अ॒न्वा॒र॒म्भ इत्य॑नु - आ॒र॒म्भः । \newline
50. अन॑वच्छित्त्यै॒ वरु॑णो॒ वरु॒णो ऽन॑वच्छित्त्या॒ अन॑वच्छित्त्यै॒ वरु॑णो॒ वै वै वरु॒णो ऽन॑वच्छित्त्या॒ अन॑वच्छित्त्यै॒ वरु॑णो॒ वै । \newline
51. अन॑वच्छित्त्या॒ इत्यन॑व - छि॒त्त्यै॒ । \newline
52. वरु॑णो॒ वै वै वरु॑णो॒ वरु॑णो॒ वा ए॒ष ए॒ष वै वरु॑णो॒ वरु॑णो॒ वा ए॒षः । \newline
53. वा ए॒ष ए॒ष वै वा ए॒ष यज॑मानं॒ ॅयज॑मान मे॒ष वै वा ए॒ष यज॑मानम् । \newline
54. ए॒ष यज॑मानं॒ ॅयज॑मान मे॒ष ए॒ष यज॑मान म॒भ्य॑भि यज॑मान मे॒ष ए॒ष यज॑मान म॒भि । \newline
55. यज॑मान म॒भ्य॑भि यज॑मानं॒ ॅयज॑मान म॒भ्या ऽभि यज॑मानं॒ ॅयज॑मान म॒भ्या । \newline
56. अ॒भ्या ऽभ्य॑भ्यै त्ये॒त्या ऽभ्य॑भ्यैति॑ । \newline
57. ऐत्ये॒ त्यैति॒ यद् यदे॒ त्यैति॒ यत् । \newline
58. ए॒ति॒ यद् यदे᳚ त्येति॒ यत् क्री॒तः क्री॒तो यदे᳚ त्येति॒ यत् क्री॒तः । \newline
59. यत् क्री॒तः क्री॒तो यद् यत् क्री॒तः सोमः॒ सोमः॑ क्री॒तो यद् यत् क्री॒तः सोमः॑ । \newline
\pagebreak
\markright{ TS 6.1.11.6  \hfill https://www.vedavms.in \hfill}

\section{ TS 6.1.11.6 }

\textbf{TS 6.1.11.6 } \newline
\textbf{Samhita Paata} \newline

क्री॒तः सोम॒ उप॑नद्धो॒ नमो॑ मि॒त्रस्य॒ वरु॑णस्य॒ चक्ष॑स॒ इत्या॑ह॒ शान्त्या॒ आ सोमं॒ ॅवह॑न्त्य॒ग्निना॒ प्रति॑ तिष्ठते॒ तौ स॒भंव॑न्तौ॒ यज॑मानम॒भि सं भ॑वतः पु॒रा खलु॒ वावैष मेधा॑या॒ऽऽ*त्मान॑मा॒रभ्य॑ चरति॒ यो दी᳚क्षि॒तो यद॑ग्नीषो॒मीयं॑ प॒शुमा॒लभ॑त आत्मनि॒ष्क्रय॑ण ए॒वास्य॒ स तस्मा॒त् तस्य॒ नाऽऽ*श्यं॑ पुरुषनि॒ष्क्रय॑ण इव॒ ह्यथो॒ खल्वा॑हु ( ) र॒ग्नीषोमा᳚भ्यां॒ ॅवा इन्द्रो॑ वृ॒त्रम॑ह॒न्निति॒ यद॑ग्नीषो॒मीयं॑ प॒शुमा॒लभ॑त॒ वार्त्र॑घ्न ए॒वास्य॒ स तस्मा᳚द्-वा॒श्यं॑ ॅवारु॒ण्यर्चा परि॑ चरति॒ स्वयै॒वैनं॑ दे॒वत॑या॒ परि॑ चरति ॥ \newline

\textbf{Pada Paata} \newline

क्री॒तः । सोमः॑ । उप॑नद्ध॒ इत्युप॑ - न॒द्धः॒ । नमः॑ । मि॒त्रस्य॑ । वरु॑णस्य । चक्ष॑से । इति॑ । आ॒ह॒ । शान्त्यै᳚ । एति॑ । सोम᳚म् । वह॑न्ति । अ॒ग्निना᳚ । प्रतीति॑ । ति॒ष्ठ॒ते॒ । तौ । स॒भंव॑न्ता॒विति॑ सं - भव॑न्तौ । यज॑मानम् । अ॒भि । समिति॑ । भ॒व॒तः॒ । पु॒रा । खलु॑ । वाव । ए॒षः । मेधा॑य । आ॒त्मान᳚म् । आ॒रभ्येत्या᳚-रभ्य॑ । च॒र॒ति॒ । यः । दी॒क्षि॒तः । यत् । अ॒ग्नी॒षो॒मीय॒मित्य॑ग्नी - सो॒मीय᳚म् । प॒शुम् । आ॒लभ॑त॒ इत्या᳚ - लभ॑ते । आ॒त्म॒नि॒ष्क्रय॑ण॒ इत्या᳚त्म - नि॒ष्क्रय॑णः । ए॒व । अ॒स्य॒ । सः । तस्मा᳚त् । तस्य॑ । न । आ॒श्य᳚म् । पु॒रु॒ष॒नि॒ष्क्रय॑ण॒ इति॑ पुरुष - नि॒ष्क्रय॑णः । इ॒व॒ । हि । अथो॒ इति॑ । खलु॑ । आ॒हुः॒ ( ) । अ॒ग्नीषोमा᳚भ्या॒मित्य॒ग्नी - सोमा᳚भ्याम् । वै । इन्द्रः॑ । वृ॒त्रम् । अ॒ह॒न्न् । इति॑ । यत् । अ॒ग्नी॒षो॒मीय॒मित्य॑ग्नी - सो॒मीय᳚म् । प॒शुम् । आ॒लभ॑त॒ इत्या᳚-लभ॑ते । वार्त्र॑घ्न॒ इति॒ वार्त्र॑ - घ्नः॒ । ए॒व । अ॒स्य॒ । सः । तस्मा᳚त् । उ॒ । आ॒श्य᳚म् । वा॒रु॒ण्या । ऋ॒चा । परीति॑ । च॒र॒ति॒ । स्वया᳚ । ए॒व । ए॒न॒म् । दे॒वत॑या । परीति॑ । च॒र॒ति॒ ॥  \newline


\textbf{Krama Paata} \newline

क्री॒तः सोमः॑ । सोम॒ उप॑नद्धः । उप॑नद्धो॒ नमः॑ । उप॒नद्ध॒ इत्युप॑ - न॒द्धः॒ । नमो॑ मि॒त्रस्य॑ । मि॒त्रस्य॒ वरु॑णस्य । वरु॑णस्य॒ चक्ष॑से । चक्ष॑स॒ इति॑ । इत्या॑ह । आ॒ह॒ शान्त्यै᳚ । शान्त्या॒ आ । आ सोम᳚म् । सोम॒म् ॅवह॑न्ति । वह॑न्त्य॒ग्निना᳚ । अ॒ग्निना॒ प्रति॑ । प्रति॑ तिष्ठते । ति॒ष्ठ॒ते॒ तौ । तौ स॒म्भव॑न्तौ । स॒म्भव॑न्तौ॒ यज॑मानम् । स॒म्भव॑न्ता॒विति॑ सम् - भव॑न्तौ । यज॑मानम॒भि । अ॒भि सम् । सम्भ॑वतः । भ॒व॒तः॒ पु॒रा । पु॒रा खलु॑ । खलु॒ वाव । वावैषः । ए॒ष मेधा॑य । मेधा॑या॒त्मान᳚म् । आ॒त्मान॑मा॒रभ्य॑ । आ॒रभ्य॑ चरति । आ॒रभ्येत्या᳚ - रभ्य॑ । च॒र॒ति॒ यः । यो दी᳚क्षि॒तः । दी॒क्षि॒तो यत् । यद॑ग्नीषो॒मीय᳚म् । अ॒ग्नी॒षो॒मीय॑म् प॒शुम् । अ॒ग्नी॒षो॒मीय॒मित्य॑ग्नी - सो॒मीय᳚म् । प॒शुमा॒लभ॑ते । आ॒लभ॑त आत्मनि॒ष्क्रय॑णः । आ॒लभ॑त॒ इत्या᳚ - लभ॑ते । आ॒त्म॒नि॒ष्क्रय॑ण ए॒व । आ॒त्म॒नि॒ष्क्रय॑ण॒ इत्या᳚त्म - नि॒ष्क्रय॑णः । ए॒वास्य॑ । अ॒स्य॒ सः । स तस्मा᳚त् । तस्मा॒त् तस्य॑ । तस्य॒ न । नाश्य᳚म् । आ॒श्य॑म् पुरुषनि॒ष्क्रय॑णः । पु॒रु॒ष॒नि॒ष्क्रय॑ण इव । पु॒रु॒ष॒नि॒ष्क्रय॑ण॒ इति॑ पुरुष - नि॒ष्क्रय॑णः । इ॒व॒ हि । ह्यथो᳚ । अथो॒ खलु॑ । अथो॒ इत्यथो᳚ । खल्वा॑हुः ( ) । आ॒हु॒र॒ग्नीषोमा᳚भ्याम् । अ॒ग्नीषोमा᳚भ्या॒म् ॅवै । अ॒ग्नीषोमा᳚भ्या॒मित्य॒ग्नी - सोमा᳚भ्याम् । वा इन्द्रः॑ । इन्द्रो॑ वृ॒त्रम् । वृ॒त्रम॑हन्॑ । अ॒ह॒न्निति॑ । इति॒ यत् । यद॑ग्नीषो॒मीय᳚म् । अ॒ग्नी॒षो॒मीय॑म् प॒शुम् । अ॒ग्नी॒षो॒मीय॒मित्य॑ग्नी - सो॒मीय᳚म् । प॒शुमा॒लभ॑ते । आ॒लभ॑ते॒ वार्त्र॑घ्नः । आ॒लभ॑त॒ इत्या᳚ - लभ॑ते । वार्त्र॑घ्न ए॒व । वार्त्र॑घ्न॒ इति॒ वार्त्र॑ - घ्नः॒ । ए॒वास्य॑ । अ॒स्य॒ सः । स तस्मा᳚त् । तस्मा॑दु । उ॒वा॒श्य᳚म् । आ॒श्य॑म् ॅवारु॒ण्या । वा॒रु॒ण्यर्चा । ऋ॒चा परि॑ । परि॑ चरति । च॒र॒ति॒ स्वया᳚ । स्वयै॒व । ए॒वैन᳚म् । ए॒न॒म् दे॒वत॑या । दे॒वत॑या॒ परि॑ । परि॑ चरति । च॒र॒तीति॑ चरति । \newline

\textbf{Jatai Paata} \newline

1. क्री॒तः सोमः॒ सोमः॑ क्री॒तः क्री॒तः सोमः॑ । \newline
2. सोम॒ उप॑नद्ध॒ उप॑नद्धः॒ सोमः॒ सोम॒ उप॑नद्धः । \newline
3. उप॑नद्धो॒ नमो॒ नम॒ उप॑नद्ध॒ उप॑नद्धो॒ नमः॑ । \newline
4. उप॑नद्ध॒ इत्युप॑ - न॒द्धः॒ । \newline
5. नमो॑ मि॒त्रस्य॑ मि॒त्रस्य॒ नमो॒ नमो॑ मि॒त्रस्य॑ । \newline
6. मि॒त्रस्य॒ वरु॑णस्य॒ वरु॑णस्य मि॒त्रस्य॑ मि॒त्रस्य॒ वरु॑णस्य । \newline
7. वरु॑णस्य॒ चक्ष॑से॒ चक्ष॑से॒ वरु॑णस्य॒ वरु॑णस्य॒ चक्ष॑से । \newline
8. चक्ष॑स॒ इतीति॒ चक्ष॑से॒ चक्ष॑स॒ इति॑ । \newline
9. इत्या॑हा॒हे तीत्या॑ह । \newline
10. आ॒ह॒ शान्त्यै॒ शान्त्या॑ आहाह॒ शान्त्यै᳚ । \newline
11. शान्त्या॒ आ शान्त्यै॒ शान्त्या॒ आ । \newline
12. आ सोमꣳ॒॒ सोम॒ मा सोम᳚म् । \newline
13. सोमं॒ ॅवह॑न्ति॒ वह॑न्ति॒ सोमꣳ॒॒ सोमं॒ ॅवह॑न्ति । \newline
14. वह॑न् त्य॒ग्निना॒ ऽग्निना॒ वह॑न्ति॒ वह॑न् त्य॒ग्निना᳚ । \newline
15. अ॒ग्निना॒ प्रति॒ प्रत्य॒ग्निना॒ ऽग्निना॒ प्रति॑ । \newline
16. प्रति॑ तिष्ठते तिष्ठते॒ प्रति॒ प्रति॑ तिष्ठते । \newline
17. ति॒ष्ठ॒ते॒ तौ तौ ति॑ष्ठते तिष्ठते॒ तौ । \newline
18. तौ सं॒भव॑न्तौ सं॒भव॑न्तौ॒ तौ तौ सं॒भव॑न्तौ । \newline
19. सं॒भव॑न्तौ॒ यज॑मानं॒ ॅयज॑मानꣳ सं॒भव॑न्तौ सं॒भव॑न्तौ॒ यज॑मानम् । \newline
20. सं॒भव॑न्ता॒विति॑ सं - भव॑न्तौ । \newline
21. यज॑मान म॒भ्य॑भि यज॑मानं॒ ॅयज॑मान म॒भि । \newline
22. अ॒भि सꣳ स म॒भ्य॑भि सम् । \newline
23. सम् भ॑वतो भवतः॒ सꣳ सम् भ॑वतः । \newline
24. भ॒व॒तः॒ पु॒रा पु॒रा भ॑वतो भवतः पु॒रा । \newline
25. पु॒रा खलु॒ खलु॑ पु॒रा पु॒रा खलु॑ । \newline
26. खलु॒ वाव वाव खलु॒ खलु॒ वाव । \newline
27. वावैष ए॒ष वाव वावैषः । \newline
28. ए॒ष मेधा॑य॒ मेधा॑ यै॒ष ए॒ष मेधा॑य । \newline
29. मेधा॑ या॒त्मान॑ मा॒त्मान॒म् मेधा॑य॒ मेधा॑ या॒त्मान᳚म् । \newline
30. आ॒त्मान॑ मा॒रभ्या॒ रभ्या॒त्मान॑ मा॒त्मान॑ मा॒रभ्य॑ । \newline
31. आ॒रभ्य॑ चरति चर त्या॒रभ्या॒ रभ्य॑ चरति । \newline
32. आ॒रभ्येत्या᳚ - रभ्य॑ । \newline
33. च॒र॒ति॒ यो यश्च॑रति चरति॒ यः । \newline
34. यो दी᳚क्षि॒तो दी᳚क्षि॒तो यो यो दी᳚क्षि॒तः । \newline
35. दी॒क्षि॒तो यद् यद् दी᳚क्षि॒तो दी᳚क्षि॒तो यत् । \newline
36. यद॑ग्नीषो॒मीय॑ मग्नीषो॒मीयं॒ ॅयद् यद॑ग्नीषो॒मीय᳚म् । \newline
37. अ॒ग्नी॒षो॒मीय॑म् प॒शुम् प॒शु म॑ग्नीषो॒मीय॑ मग्नीषो॒मीय॑म् प॒शुम् । \newline
38. अ॒ग्नी॒षो॒मीय॒मित्य॑ग्नी - सो॒मीय᳚म् । \newline
39. प॒शु मा॒लभ॑त आ॒लभ॑ते प॒शुम् प॒शु मा॒लभ॑ते । \newline
40. आ॒लभ॑त आत्मनि॒ष्क्रय॑ण आत्मनि॒ष्क्रय॑ण आ॒लभ॑त आ॒लभ॑त आत्मनि॒ष्क्रय॑णः । \newline
41. आ॒लभ॑त॒ इत्या᳚ - लभ॑ते । \newline
42. आ॒त्म॒नि॒ष्क्रय॑ण ए॒वैवा त्म॑नि॒ष्क्रय॑ण आत्मनि॒ष्क्रय॑ण ए॒व । \newline
43. आ॒त्म॒नि॒ष्क्रय॑ण॒ इत्या᳚त्म - नि॒ष्क्रय॑णः । \newline
44. ए॒वास्या᳚ स्यै॒वै वास्य॑ । \newline
45. अ॒स्य॒ स सो᳚ ऽस्यास्य॒ सः । \newline
46. स तस्मा॒त् तस्मा॒थ् स स तस्मा᳚त् । \newline
47. तस्मा॒त् तस्य॒ तस्य॒ तस्मा॒त् तस्मा॒त् तस्य॑ । \newline
48. तस्य॒ न न तस्य॒ तस्य॒ न । \newline
49. नाश्य॑ मा॒श्य॑न्न नाश्य᳚म् । \newline
50. आ॒श्य॑म् पुरुषनि॒ष्क्रय॑णः पुरुषनि॒ष्क्रय॑ण आ॒श्य॑ मा॒श्य॑म् पुरुषनि॒ष्क्रय॑णः । \newline
51. पु॒रु॒ष॒नि॒ष्क्रय॑ण इवेव पुरुषनि॒ष्क्रय॑णः पुरुषनि॒ष्क्रय॑ण इव । \newline
52. पु॒रु॒ष॒नि॒ष्क्रय॑ण॒ इति॑ पुरुष - नि॒ष्क्रय॑णः । \newline
53. इ॒व॒ हि हीवे॑व॒ हि । \newline
54. ह्यथो॒ अथो॒ हि ह्यथो᳚ । \newline
55. अथो॒ खलु॒ खल्वथो॒ अथो॒ खलु॑ । \newline
56. अथो॒ इत्यथो᳚ । \newline
57. खल्वा॑हु राहुः॒ खलु॒ खल्वा॑हुः । \newline
58. आ॒हु॒ र॒ग्नीषोमा᳚भ्या म॒ग्नीषोमा᳚भ्या माहु राहु र॒ग्नीषोमा᳚भ्याम् । \newline
59. अ॒ग्नीषोमा᳚भ्यां॒ ॅवै वा अ॒ग्नीषोमा᳚भ्या म॒ग्नीषोमा᳚भ्यां॒ ॅवै । \newline
60. अ॒ग्नीषोमा᳚भ्या॒मित्य॒ग्नी - सोमा᳚भ्याम् । \newline
61. वा इन्द्र॒ इन्द्रो॒ वै वा इन्द्रः॑ । \newline
62. इन्द्रो॑ वृ॒त्रं ॅवृ॒त्र मिन्द्र॒ इन्द्रो॑ वृ॒त्रम् । \newline
63. वृ॒त्र म॑हन् नहन् वृ॒त्रं ॅवृ॒त्र म॑हन्न् । \newline
64. अ॒ह॒न् निती त्य॑हन् नह॒न् निति॑ । \newline
65. इति॒ यद् यदितीति॒ यत् । \newline
66. यद॑ग्नीषो॒मीय॑ मग्नीषो॒मीयं॒ ॅयद् यद॑ग्नीषो॒मीय᳚म् । \newline
67. अ॒ग्नी॒षो॒मीय॑म् प॒शुम् प॒शु म॑ग्नीषो॒मीय॑ मग्नीषो॒मीय॑म् प॒शुम् । \newline
68. अ॒ग्नी॒षो॒मीय॒मित्य॑ग्नी - सो॒मीय᳚म् । \newline
69. प॒शु मा॒लभ॑त आ॒लभ॑ते प॒शुम् प॒शु मा॒लभ॑ते । \newline
70. आ॒लभ॑ते॒ वार्त्र॑घ्नो॒ वार्त्र॑घ्न आ॒लभ॑त आ॒लभ॑ते॒ वार्त्र॑घ्नः । \newline
71. आ॒लभ॑त॒ इत्या᳚ - लभ॑ते । \newline
72. वार्त्र॑घ्न ए॒वैव वार्त्र॑घ्नो॒ वार्त्र॑घ्न ए॒व । \newline
73. वार्त्र॑घ्न॒ इति॒ वार्त्र॑ - घ्नः॒ । \newline
74. ए॒वास्या᳚ स्यै॒वै वास्य॑ । \newline
75. अ॒स्य॒ स सो᳚ ऽस्यास्य॒ सः । \newline
76. स तस्मा॒त् तस्मा॒थ् स स तस्मा᳚त् । \newline
77. तस्मा॑दू॒ तस्मा॒त् तस्मा॑दु । \newline
78. उ॒ वा॒श्य॑ मा॒श्य॑ मु वु वा॒श्य᳚म् । \newline
79. आ॒श्यं॑ ॅवारु॒ण्या वा॑रु॒ण्या ऽऽश्य॑ मा॒श्यं॑ ॅवारु॒ण्या । \newline
80. वा॒रु॒ण्य र्‌च र्‌चा वा॑रु॒ण्या वा॑रु॒ण्य र्‌चा । \newline
81. ऋ॒चा परि॒ पर्यृ॒च र्‌चा परि॑ । \newline
82. परि॑ चरति चरति॒ परि॒ परि॑ चरति । \newline
83. च॒र॒ति॒ स्वया॒ स्वया॑ चरति चरति॒ स्वया᳚ । \newline
84. स्वयै॒ वैव स्वया॒ स्वयै॒व । \newline
85. ए॒वैन॑ मेन मे॒वै वैन᳚म् । \newline
86. ए॒न॒म् दे॒वत॑या दे॒वत॑यैन मेनम् दे॒वत॑या । \newline
87. दे॒वत॑या॒ परि॒ परि॑ दे॒वत॑या दे॒वत॑या॒ परि॑ । \newline
88. परि॑ चरति चरति॒ परि॒ परि॑ चरति । \newline
89. च॒र॒तीति॑ चरति । \newline

\textbf{Ghana Paata } \newline

1. क्री॒तः सोमः॒ सोमः॑ क्री॒तः क्री॒तः सोम॒ उप॑नद्ध॒ उप॑नद्धः॒ सोमः॑ क्री॒तः क्री॒तः सोम॒ उप॑नद्धः । \newline
2. सोम॒ उप॑नद्ध॒ उप॑नद्धः॒ सोमः॒ सोम॒ उप॑नद्धो॒ नमो॒ नम॒ उप॑नद्धः॒ सोमः॒ सोम॒ उप॑नद्धो॒ नमः॑ । \newline
3. उप॑नद्धो॒ नमो॒ नम॒ उप॑नद्ध॒ उप॑नद्धो॒ नमो॑ मि॒त्रस्य॑ मि॒त्रस्य॒ नम॒ उप॑नद्ध॒ उप॑नद्धो॒ नमो॑ मि॒त्रस्य॑ । \newline
4. उप॑नद्ध॒ इत्युप॑ - न॒द्धः॒ । \newline
5. नमो॑ मि॒त्रस्य॑ मि॒त्रस्य॒ नमो॒ नमो॑ मि॒त्रस्य॒ वरु॑णस्य॒ वरु॑णस्य मि॒त्रस्य॒ नमो॒ नमो॑ मि॒त्रस्य॒ वरु॑णस्य । \newline
6. मि॒त्रस्य॒ वरु॑णस्य॒ वरु॑णस्य मि॒त्रस्य॑ मि॒त्रस्य॒ वरु॑णस्य॒ चक्ष॑से॒ चक्ष॑से॒ वरु॑णस्य मि॒त्रस्य॑ मि॒त्रस्य॒ वरु॑णस्य॒ चक्ष॑से । \newline
7. वरु॑णस्य॒ चक्ष॑से॒ चक्ष॑से॒ वरु॑णस्य॒ वरु॑णस्य॒ चक्ष॑स॒ इतीति॒ चक्ष॑से॒ वरु॑णस्य॒ वरु॑णस्य॒ चक्ष॑स॒ इति॑ । \newline
8. चक्ष॑स॒ इतीति॒ चक्ष॑से॒ चक्ष॑स॒ इत्या॑हा॒ हेति॒ चक्ष॑से॒ चक्ष॑स॒ इत्या॑ह । \newline
9. इत्या॑हा॒हे तीत्या॑ह॒ शान्त्यै॒ शान्त्या॑ आ॒हे तीत्या॑ह॒ शान्त्यै᳚ । \newline
10. आ॒ह॒ शान्त्यै॒ शान्त्या॑ आहाह॒ शान्त्या॒ आ शान्त्या॑ आहाह॒ शान्त्या॒ आ । \newline
11. शान्त्या॒ आ शान्त्यै॒ शान्त्या॒ आ सोमꣳ॒॒ सोम॒ मा शान्त्यै॒ शान्त्या॒ आ सोम᳚म् । \newline
12. आ सोमꣳ॒॒ सोम॒ मा सोमं॒ ॅवह॑न्ति॒ वह॑न्ति॒ सोम॒ मा सोमं॒ ॅवह॑न्ति । \newline
13. सोमं॒ ॅवह॑न्ति॒ वह॑न्ति॒ सोमꣳ॒॒ सोमं॒ ॅवह॑न् त्य॒ग्निना॒ ऽग्निना॒ वह॑न्ति॒ सोमꣳ॒॒ सोमं॒ ॅवह॑न् त्य॒ग्निना᳚ । \newline
14. वह॑न् त्य॒ग्निना॒ ऽग्निना॒ वह॑न्ति॒ वह॑न् त्य॒ग्निना॒ प्रति॒ प्रत्य॒ ग्निना॒ वह॑न्ति॒ वह॑न् त्य॒ग्निना॒ प्रति॑ । \newline
15. अ॒ग्निना॒ प्रति॒ प्रत्य॒ ग्निना॒ ऽग्निना॒ प्रति॑ तिष्ठते तिष्ठते॒ प्रत्य॒ ग्निना॒ ऽग्निना॒ प्रति॑ तिष्ठते । \newline
16. प्रति॑ तिष्ठते तिष्ठते॒ प्रति॒ प्रति॑ तिष्ठते॒ तौ तौ ति॑ष्ठते॒ प्रति॒ प्रति॑ तिष्ठते॒ तौ । \newline
17. ति॒ष्ठ॒ते॒ तौ तौ ति॑ष्ठते तिष्ठते॒ तौ सं॒भव॑न्तौ सं॒भव॑न्तौ॒ तौ ति॑ष्ठते तिष्ठते॒ तौ सं॒भव॑न्तौ । \newline
18. तौ सं॒भव॑न्तौ सं॒भव॑न्तौ॒ तौ तौ सं॒भव॑न्तौ॒ यज॑मानं॒ ॅयज॑मानꣳ सं॒भव॑न्तौ॒ तौ तौ सं॒भव॑न्तौ॒ यज॑मानम् । \newline
19. सं॒भव॑न्तौ॒ यज॑मानं॒ ॅयज॑मानꣳ सं॒भव॑न्तौ सं॒भव॑न्तौ॒ यज॑मान म॒भ्य॑भि यज॑मानꣳ सं॒भव॑न्तौ सं॒भव॑न्तौ॒ यज॑मान म॒भि । \newline
20. सं॒भव॑न्ता॒विति॑ सं - भव॑न्तौ । \newline
21. यज॑मान म॒भ्य॑भि यज॑मानं॒ ॅयज॑मान म॒भि सꣳ स म॒भि यज॑मानं॒ ॅयज॑मान म॒भि सम् । \newline
22. अ॒भि सꣳ स म॒भ्य॑भि सम् भ॑वतो भवतः॒ स म॒भ्य॑भि सम् भ॑वतः । \newline
23. सम् भ॑वतो भवतः॒ सꣳ सम् भ॑वतः पु॒रा पु॒रा भ॑वतः॒ सꣳ सम् भ॑वतः पु॒रा । \newline
24. भ॒व॒तः॒ पु॒रा पु॒रा भ॑वतो भवतः पु॒रा खलु॒ खलु॑ पु॒रा भ॑वतो भवतः पु॒रा खलु॑ । \newline
25. पु॒रा खलु॒ खलु॑ पु॒रा पु॒रा खलु॒ वाव वाव खलु॑ पु॒रा पु॒रा खलु॒ वाव । \newline
26. खलु॒ वाव वाव खलु॒ खलु॒ वावैष ए॒ष वाव खलु॒ खलु॒ वावैषः । \newline
27. वावैष ए॒ष वाव वावैष मेधा॑य॒ मेधा॑ यै॒ष वाव वावैष मेधा॑य । \newline
28. ए॒ष मेधा॑य॒ मेधा॑यै॒ष ए॒ष मेधा॑ या॒त्मान॑ मा॒त्मान॒म् मेधा॑ यै॒ष ए॒ष मेधा॑ या॒त्मान᳚म् । \newline
29. मेधा॑ या॒त्मान॑ मा॒त्मान॒म् मेधा॑य॒ मेधा॑ या॒त्मान॑ मा॒रभ्या॒ रभ्या॒ त्मान॒म् मेधा॑य॒ मेधा॑ या॒त्मान॑ मा॒रभ्य॑ । \newline
30. आ॒त्मान॑ मा॒रभ्या॒ रभ्या॒त्मान॑ मा॒त्मान॑ मा॒रभ्य॑ चरति चर त्या॒र भ्या॒त्मान॑ मा॒त्मान॑ मा॒रभ्य॑ चरति । \newline
31. आ॒रभ्य॑ चरति चर त्या॒रभ्या॒ रभ्य॑ चरति॒ यो यश्च॑र त्या॒रभ्या॒ रभ्य॑ चरति॒ यः । \newline
32. आ॒रभ्येत्या᳚ - रभ्य॑ । \newline
33. च॒र॒ति॒ यो यश्च॑रति चरति॒ यो दी᳚क्षि॒तो दी᳚क्षि॒तो यश्च॑रति चरति॒ यो दी᳚क्षि॒तः । \newline
34. यो दी᳚क्षि॒तो दी᳚क्षि॒तो यो यो दी᳚क्षि॒तो यद् यद् दी᳚क्षि॒तो यो यो दी᳚क्षि॒तो यत् । \newline
35. दी॒क्षि॒तो यद् यद् दी᳚क्षि॒तो दी᳚क्षि॒तो यद॑ग्नीषो॒मीय॑ मग्नीषो॒मीयं॒ ॅयद् दी᳚क्षि॒तो दी᳚क्षि॒तो यद॑ग्नीषो॒मीय᳚म् । \newline
36. यद॑ग्नीषो॒मीय॑ मग्नीषो॒मीयं॒ ॅयद् यद॑ग्नीषो॒मीय॑म् प॒शुम् प॒शु म॑ग्नीषो॒मीयं॒ ॅयद् यद॑ग्नीषो॒मीय॑म् प॒शुम् । \newline
37. अ॒ग्नी॒षो॒मीय॑म् प॒शुम् प॒शु म॑ग्नीषो॒मीय॑ मग्नीषो॒मीय॑म् प॒शु मा॒लभ॑त आ॒लभ॑ते प॒शु म॑ग्नीषो॒मीय॑ मग्नीषो॒मीय॑म् प॒शु मा॒लभ॑ते । \newline
38. अ॒ग्नी॒षो॒मीय॒मित्य॑ग्नी - सो॒मीय᳚म् । \newline
39. प॒शु मा॒लभ॑त आ॒लभ॑ते प॒शुम् प॒शु मा॒लभ॑त आत्मनि॒ष्क्रय॑ण आत्मनि॒ष्क्रय॑ण आ॒लभ॑ते प॒शुम् प॒शु मा॒लभ॑त आत्मनि॒ष्क्रय॑णः । \newline
40. आ॒लभ॑त आत्मनि॒ष्क्रय॑ण आत्मनि॒ष्क्रय॑ण आ॒लभ॑त आ॒लभ॑त आत्मनि॒ष्क्रय॑ण ए॒वै वात्म॑नि॒ष्क्रय॑ण आ॒लभ॑त आ॒लभ॑त आत्मनि॒ष्क्रय॑ण ए॒व । \newline
41. आ॒लभ॑त॒ इत्या᳚ - लभ॑ते । \newline
42. आ॒त्म॒नि॒ष्क्रय॑ण ए॒वै वात्म॑नि॒ष्क्रय॑ण आत्मनि॒ष्क्रय॑ण ए॒वास्या᳚ स्यै॒वात्म॑नि॒ष्क्रय॑ण आत्मनि॒ष्क्रय॑ण ए॒वास्य॑ । \newline
43. आ॒त्म॒नि॒ष्क्रय॑ण॒ इत्या᳚त्म - नि॒ष्क्रय॑णः । \newline
44. ए॒वास्या᳚ स्यै॒वै वास्य॒ स सो᳚ ऽस्यै॒वै वास्य॒ सः । \newline
45. अ॒स्य॒ स सो᳚ ऽस्यास्य॒ स तस्मा॒त् तस्मा॒थ् सो᳚ ऽस्यास्य॒ स तस्मा᳚त् । \newline
46. स तस्मा॒त् तस्मा॒थ् स स तस्मा॒त् तस्य॒ तस्य॒ तस्मा॒थ् स स तस्मा॒त् तस्य॑ । \newline
47. तस्मा॒त् तस्य॒ तस्य॒ तस्मा॒त् तस्मा॒त् तस्य॒ न न तस्य॒ तस्मा॒त् तस्मा॒त् तस्य॒ न । \newline
48. तस्य॒ न न तस्य॒ तस्य॒ नाश्य॑ मा॒श्य॑न् न तस्य॒ तस्य॒ नाश्य᳚म् । \newline
49. नाश्य॑ मा॒श्य॑न् न नाश्य॑म् पुरुषनि॒ष्क्रय॑णः पुरुषनि॒ष्क्रय॑ण आ॒श्य॑न् न नाश्य॑म् पुरुषनि॒ष्क्रय॑णः । \newline
50. आ॒श्य॑म् पुरुषनि॒ष्क्रय॑णः पुरुषनि॒ष्क्रय॑ण आ॒श्य॑ मा॒श्य॑म् पुरुषनि॒ष्क्रय॑ण इवेव पुरुषनि॒ष्क्रय॑ण आ॒श्य॑ मा॒श्य॑म् पुरुषनि॒ष्क्रय॑ण इव । \newline
51. पु॒रु॒ष॒नि॒ष्क्रय॑ण इवेव पुरुषनि॒ष्क्रय॑णः पुरुषनि॒ष्क्रय॑ण इव॒ हि हीव॑ पुरुषनि॒ष्क्रय॑णः पुरुषनि॒ष्क्रय॑ण इव॒ हि । \newline
52. पु॒रु॒ष॒नि॒ष्क्रय॑ण॒ इति॑ पुरुष - नि॒ष्क्रय॑णः । \newline
53. इ॒व॒ हि हीवे॑व॒ ह्यथो॒ अथो॒ हीवे॑व॒ ह्यथो᳚ । \newline
54. ह्यथो॒ अथो॒ हि ह्यथो॒ खलु॒ खल्वथो॒ हि ह्यथो॒ खलु॑ । \newline
55. अथो॒ खलु॒ खल्वथो॒ अथो॒ खल्वा॑हु राहुः॒ खल्वथो॒ अथो॒ खल्वा॑हुः । \newline
56. अथो॒ इत्यथो᳚ । \newline
57. खल्वा॑हु राहुः॒ खलु॒ खल्वा॑हु र॒ग्नीषोमा᳚भ्या म॒ग्नीषोमा᳚भ्या माहुः॒ खलु॒ खल्वा॑हु र॒ग्नीषोमा᳚भ्याम् । \newline
58. आ॒हु॒ र॒ग्नीषोमा᳚भ्या म॒ग्नीषोमा᳚भ्या माहु राहु र॒ग्नीषोमा᳚भ्यां॒ ॅवै वा अ॒ग्नीषोमा᳚भ्या माहु राहु र॒ग्नीषोमा᳚भ्यां॒ ॅवै । \newline
59. अ॒ग्नीषोमा᳚भ्यां॒ ॅवै वा अ॒ग्नीषोमा᳚भ्या म॒ग्नीषोमा᳚भ्यां॒ ॅवा इन्द्र॒ इन्द्रो॒ वा अ॒ग्नीषोमा᳚भ्या म॒ग्नीषोमा᳚भ्यां॒ ॅवा इन्द्रः॑ । \newline
60. अ॒ग्नीषोमा᳚भ्या॒मित्य॒ग्नी - सोमा᳚भ्याम् । \newline
61. वा इन्द्र॒ इन्द्रो॒ वै वा इन्द्रो॑ वृ॒त्रं ॅवृ॒त्र मिन्द्रो॒ वै वा इन्द्रो॑ वृ॒त्रम् । \newline
62. इन्द्रो॑ वृ॒त्रं ॅवृ॒त्र मिन्द्र॒ इन्द्रो॑ वृ॒त्र म॑हन् नहन् वृ॒त्र मिन्द्र॒ इन्द्रो॑ वृ॒त्र म॑हन्न् । \newline
63. वृ॒त्र म॑हन् नहन् वृ॒त्रं ॅवृ॒त्र म॑ह॒न्निती त्य॑हन् वृ॒त्रं ॅवृ॒त्र म॑ह॒न्निति॑ । \newline
64. अ॒ह॒न्नि तीत्य॑हन् नह॒न् निति॒ यद् यदित्य॑हन् नह॒न् निति॒ यत् । \newline
65. इति॒ यद् यदितीति॒ यद॑ग्नीषो॒मीय॑ मग्नीषो॒मीयं॒ ॅयदितीति॒ यद॑ग्नीषो॒मीय᳚म् । \newline
66. यद॑ग्नीषो॒मीय॑ मग्नीषो॒मीयं॒ ॅयद् यद॑ग्नीषो॒मीय॑म् प॒शुम् प॒शु म॑ग्नीषो॒मीयं॒ ॅयद् यद॑ग्नीषो॒मीय॑म् प॒शुम् । \newline
67. अ॒ग्नी॒षो॒मीय॑म् प॒शुम् प॒शु म॑ग्नीषो॒मीय॑ मग्नीषो॒मीय॑म् प॒शु मा॒लभ॑त आ॒लभ॑ते प॒शु म॑ग्नीषो॒मीय॑ मग्नीषो॒मीय॑म् प॒शु मा॒लभ॑ते । \newline
68. अ॒ग्नी॒षो॒मीय॒मित्य॑ग्नी - सो॒मीय᳚म् । \newline
69. प॒शु मा॒लभ॑त आ॒लभ॑ते प॒शुम् प॒शु मा॒लभ॑ते॒ वार्त्र॑घ्नो॒ वार्त्र॑घ्न आ॒लभ॑ते प॒शुम् प॒शु मा॒लभ॑ते॒ वार्त्र॑घ्नः । \newline
70. आ॒लभ॑ते॒ वार्त्र॑घ्नो॒ वार्त्र॑घ्न आ॒लभ॑त आ॒लभ॑ते॒ वार्त्र॑घ्न ए॒वैव वार्त्र॑घ्न आ॒लभ॑त आ॒लभ॑ते॒ वार्त्र॑घ्न ए॒व । \newline
71. आ॒लभ॑त॒ इत्या᳚ - लभ॑ते । \newline
72. वार्त्र॑घ्न ए॒वैव वार्त्र॑घ्नो॒ वार्त्र॑घ्न ए॒वास्या᳚ स्यै॒व वार्त्र॑घ्नो॒ वार्त्र॑घ्न ए॒वास्य॑ । \newline
73. वार्त्र॑घ्न॒ इति॒ वार्त्र॑ - घ्नः॒ । \newline
74. ए॒वास्या᳚ स्यै॒वै वास्य॒ स सो᳚ ऽस्यै॒वै वास्य॒ सः । \newline
75. अ॒स्य॒ स सो᳚ ऽस्यास्य॒ स तस्मा॒त् तस्मा॒थ् सो᳚ ऽस्यास्य॒ स तस्मा᳚त् । \newline
76. स तस्मा॒त् तस्मा॒थ् स स तस्मा॑दू॒ तस्मा॒थ् स स तस्मा॑दु । \newline
77. तस्मा॑दू॒ तस्मा॒त् तस्मा᳚द् वा॒श्य॑ मा॒श्य॑ मु॒ तस्मा॒त् तस्मा᳚द् वा॒श्य᳚म् । \newline
78. उ॒ वा॒श्य॑ मा॒श्य॑ मु वु वा॒श्यं॑ ॅवारु॒ण्या वा॑रु॒ण्या ऽऽश्य॑ मु वु वा॒श्यं॑ ॅवारु॒ण्या । \newline
79. आ॒श्यं॑ ॅवारु॒ण्या वा॑रु॒ण्या ऽऽश्य॑ मा॒श्यं॑ ॅवारु॒ण्य र्‌च र्‌चा वा॑रु॒ण्या ऽऽश्य॑ मा॒श्यं॑ ॅवारु॒ण्य र्‌चा । \newline
80. वा॒रु॒ण्य र्‌च र्‌चा वा॑रु॒ण्या वा॑रु॒ण्य र्‌चा परि॒ पर्यृ॒चा वा॑रु॒ण्या वा॑रु॒ण्य र्‌चा परि॑ । \newline
81. ऋ॒चा परि॒ पर्यृ॒च र्‌चा परि॑ चरति चरति॒ पर्यृ॒च र्‌चा परि॑ चरति । \newline
82. परि॑ चरति चरति॒ परि॒ परि॑ चरति॒ स्वया॒ स्वया॑ चरति॒ परि॒ परि॑ चरति॒ स्वया᳚ । \newline
83. च॒र॒ति॒ स्वया॒ स्वया॑ चरति चरति॒ स्वयै॒ वैव स्वया॑ चरति चरति॒ स्वयै॒व । \newline
84. स्वयै॒ वैव स्वया॒ स्वयै॒ वैन॑ मेन मे॒व स्वया॒ स्वयै॒ वैन᳚म् । \newline
85. ए॒वैन॑ मेन मे॒वै वैन॑म् दे॒वत॑या दे॒वत॑ यैन मे॒वै वैन॑म् दे॒वत॑या । \newline
86. ए॒न॒म् दे॒वत॑या दे॒वत॑ यैन मेनम् दे॒वत॑या॒ परि॒ परि॑ दे॒वत॑ यैन मेनम् दे॒वत॑या॒ परि॑ । \newline
87. दे॒वत॑या॒ परि॒ परि॑ दे॒वत॑या दे॒वत॑या॒ परि॑ चरति चरति॒ परि॑ दे॒वत॑या दे॒वत॑या॒ परि॑ चरति । \newline
88. परि॑ चरति चरति॒ परि॒ परि॑ चरति । \newline
89. च॒र॒तीति॑ चरति । \newline
\pagebreak


\end{document}