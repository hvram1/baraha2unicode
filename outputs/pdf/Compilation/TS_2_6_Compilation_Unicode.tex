\documentclass[17pt]{extarticle}
\usepackage{babel}
\usepackage{fontspec}
\usepackage{polyglossia}
\usepackage{extsizes}



\setmainlanguage{sanskrit}
\setotherlanguages{english} %% or other languages
\setlength{\parindent}{0pt}
\pagestyle{myheadings}
\newfontfamily\devanagarifont[Script=Devanagari]{AdishilaVedic}


\newcommand{\VAR}[1]{}
\newcommand{\BLOCK}[1]{}




\begin{document}
\begin{titlepage}
    \begin{center}
 
\begin{sanskrit}
    { \Huge
    कृष्ण यजुर्वेदीय तैत्तिरीय संहिता,पद,जटा,घन पाठः 
    }
    \\
    \vspace{2.5cm}
    \mbox{ \Huge
    2.6      द्वितीयकाण्डे षष्ठः प्रश्नः - अवशिष्टकर्माभिधानं   }
\end{sanskrit}
\end{center}

\end{titlepage}
\tableofcontents
\pagebreak

\markright{ TS 2.6.1.1  \hfill https://www.vedavms.in \hfill}
\addcontentsline{toc}{section}{ TS 2.6.1.1 }
\section*{ TS 2.6.1.1 }

\textbf{TS 2.6.1.1 } \newline
\textbf{Samhita Paata} \newline

स॒मिधो॑ यजति वस॒न्तमे॒वर्तू॒नामव॑ रुन्धे॒ तनू॒नपा॑तं ॅयजति ग्री॒ष्ममे॒वाव॑ रुन्ध इ॒डो य॑जति व॒र्॒.षा ए॒वाव॑ रुन्धे ब॒र्॒.हिर्य॑जति श॒रद॑मे॒वाव॑ रुन्धे स्वाहाका॒रं ॅय॑जति हेम॒न्तमे॒वाव॑ रुन्धे॒ तस्मा॒थ् स्वाहा॑कृता॒ हेम॑न् प॒शवोऽव॑ सीदन्ति स॒मिधो॑ यजत्यु॒षस॑ ए॒व दे॒वता॑ना॒मव॑ रुन्धे॒ तनू॒नपा॑तं ॅयजति य॒ज्ञ्मे॒वाव॑ रुन्ध - [  ] \newline

\textbf{Pada Paata} \newline

स॒मिध॒ इति॑ सं - इधः॑ । य॒ज॒ति॒ । व॒स॒न्तम् । ए॒व । ऋ॒तू॒नाम् । अवेति॑ । रु॒न्धे॒ । तनू॒नपा॑त॒मिति॒ तनू᳚ - नपा॑तम् । य॒ज॒ति॒ । ग्री॒ष्मम् । ए॒व । अवेति॑ । रु॒न्धे॒ । इ॒डः । य॒ज॒ति॒ । व॒र्॒.षाः । ए॒व । अवेति॑ । रु॒न्धे॒ । ब॒र्॒.हिः । य॒ज॒ति॒ । श॒रद᳚म् । ए॒व । अवेति॑ । रु॒न्धे॒ । स्वा॒हा॒का॒रमिति॑ स्वाहा - का॒रम् । य॒ज॒ति॒ । हे॒म॒न्तम् । ए॒व । अवेति॑ । रु॒न्धे॒ । तस्मा᳚त् । स्वाहा॑कृता॒ इति॒ स्वाहा᳚-कृ॒ताः॒ । हेमन्न्॑ । प॒शवः॑ । अवेति॑ । सी॒द॒न्ति॒ । स॒मिध॒ इति॑ सं - इधः॑ । य॒ज॒ति॒ । उ॒षसः॑ । ए॒व । दे॒वता॑नाम् । अवेति॑ । रु॒न्धे॒ । तनू॒नपा॑त॒मिति॒ तनू᳚ - नपा॑तम् । य॒ज॒ति॒ । य॒ज्ञ्म् । ए॒व । अवेति॑ । रु॒न्धे॒ ।  \newline


\textbf{Krama Paata} \newline

स॒मिधो॑ यजति । स॒मिध॒ इति॑ सम् - इधः॑ । य॒ज॒ति॒ व॒स॒न्तम् । व॒स॒न्तमे॒व । ए॒वर्तू॒नाम् । ऋ॒तू॒नामव॑ । अव॑ रुन्धे । रु॒न्धे॒ तनू॒नपा॑तम् । तनू॒नपा॑तं ॅयजति । तनू॒नपा॑त॒मिति॒ तनू᳚ - नपा॑तम् । य॒ज॒ति॒ ग्री॒ष्मम् । ग्री॒ष्ममे॒व । ए॒वाव॑ । अव॑ रुन्धे । रु॒न्ध॒ इ॒डः । इ॒डो य॑जति । य॒ज॒ति॒ व॒र्.॒षाः । व॒र्.॒षा ए॒व । ए॒वाव॑ । अव॑ रुन्धे । रु॒न्धे॒ ब॒र्.॒हिः । ब॒र्.॒हिर् य॑जति । य॒ज॒ति॒ श॒रद᳚म् । श॒रद॑मे॒व । ए॒वाव॑ । अव॑ रुन्धे । रु॒न्धे॒ स्वा॒हा॒का॒रम् । स्वा॒हा॒का॒रं ॅय॑जति । स्वा॒हा॒का॒रमिति॑ स्वाहा - का॒रम् । य॒ज॒ति॒ हे॒म॒न्तम् । हे॒म॒न्तमे॒व । ए॒वाव॑ । अव॑ रुन्धे । रु॒न्धे॒ तस्मा᳚त् । तस्मा॒थ् स्वाहा॑कृताः । स्वाहा॑कृता॒ हेमन्न्॑ । स्वाहा॑कृता॒ इति॒ स्वाहा᳚ - कृ॒ताः॒ । हेम॑न् प॒शवः॑ । प॒शवोऽव॑ । अव॑ सीदन्ति । सी॒द॒न्ति॒ स॒मिधः॑ । स॒मिधो॑ यजति । स॒मिध॒ इति॑ सं - इधः॑ । य॒ज॒त्यु॒षसः॑ । उ॒षस॑ ए॒व । ए॒व दे॒वता॑नाम् । दे॒वता॑ना॒मव॑ । अव॑ रुन्धे । रु॒न्धे॒ तनू॒नपा॑तम् । तनू॒नपा॑तं ॅयजति । तनू॒नपा॑त॒मिति॒ तनू᳚ - नपा॑तम् । य॒ज॒ति॒ य॒ज्ञ्म् । य॒ज्ञ्मे॒व । ए॒वाव॑ । अव॑ रुन्धे । रु॒न्ध॒ इ॒डः \newline

\textbf{Jatai Paata} \newline

1. स॒मिधो॑ यजति यजति स॒मिधः॑ स॒मिधो॑ यजति । \newline
2. स॒मिध॒ इति॑ सं - इधः॑ । \newline
3. य॒ज॒ति॒ व॒स॒न्तं ॅव॑स॒न्तं ॅय॑जति यजति वस॒न्तम् । \newline
4. व॒स॒न्त मे॒वैव व॑स॒न्तं ॅव॑स॒न्त मे॒व । \newline
5. ए॒व र्तू॒ना मृ॑तू॒ना मे॒वैव र्तू॒नाम् । \newline
6. ऋ॒तू॒ना मवावा᳚ र्तू॒ना मृ॑तू॒ना मव॑ । \newline
7. अव॑ रुन्धे रु॒न्धे ऽवाव॑ रुन्धे । \newline
8. रु॒न्धे॒ तनू॒नपा॑त॒म् तनू॒नपा॑तꣳ रुन्धे रुन्धे॒ तनू॒नपा॑तम् । \newline
9. तनू॒नपा॑तं ॅयजति यजति॒ तनू॒नपा॑त॒म् तनू॒नपा॑तं ॅयजति । \newline
10. तनू॒नपा॑त॒मिति॒ तनू᳚ - नपा॑तम् । \newline
11. य॒ज॒ति॒ ग्री॒ष्मम् ग्री॒ष्मं ॅय॑जति यजति ग्री॒ष्मम् । \newline
12. ग्री॒ष्म मे॒वैव ग्री॒ष्मम् ग्री॒ष्म मे॒व । \newline
13. ए॒वावा वै॒वै वाव॑ । \newline
14. अव॑ रुन्धे रु॒न्धे ऽवाव॑ रुन्धे । \newline
15. रु॒न्ध॒ इ॒ड इ॒डो रु॑न्धे रुन्ध इ॒डः । \newline
16. इ॒डो य॑जति यजती॒ड इ॒डो य॑जति । \newline
17. य॒ज॒ति॒ व॒र्॒.षा व॒र्॒.षा य॑जति यजति व॒र्॒.षाः । \newline
18. व॒र्॒.षा ए॒वैव व॒र्॒.षा व॒र्॒.षा ए॒व । \newline
19. ए॒वावा वै॒वै वाव॑ । \newline
20. अव॑ रुन्धे रु॒न्धे ऽवाव॑ रुन्धे । \newline
21. रु॒न्धे॒ ब॒र्॒.हिर् ब॒र्॒.ही रु॑न्धे रुन्धे ब॒र्॒.हिः । \newline
22. ब॒र्॒.हिर् य॑जति यजति ब॒र्॒.हिर् ब॒र्॒.हिर् य॑जति । \newline
23. य॒ज॒ति॒ श॒रदꣳ॑ श॒रदं॑ ॅयजति यजति श॒रद᳚म् । \newline
24. श॒रद॑ मे॒वैव श॒रदꣳ॑ श॒रद॑ मे॒व । \newline
25. ए॒वावा वै॒वै वाव॑ । \newline
26. अव॑ रुन्धे रु॒न्धे ऽवाव॑ रुन्धे । \newline
27. रु॒न्धे॒ स्वा॒हा॒का॒रꣳ स्वा॑हाका॒रꣳ रु॑न्धे रुन्धे स्वाहाका॒रम् । \newline
28. स्वा॒हा॒का॒रं ॅय॑जति यजति स्वाहाका॒रꣳ स्वा॑हाका॒रं ॅय॑जति । \newline
29. स्वा॒हा॒का॒रमिति॑ स्वाहा - का॒रम् । \newline
30. य॒ज॒ति॒ हे॒म॒न्तꣳ हे॑म॒न्तं ॅय॑जति यजति हेम॒न्तम् । \newline
31. हे॒म॒न्त मे॒वैव हे॑म॒न्तꣳ हे॑म॒न्त मे॒व । \newline
32. ए॒वावा वै॒वै वाव॑ । \newline
33. अव॑ रुन्धे रु॒न्धे ऽवाव॑ रुन्धे । \newline
34. रु॒न्धे॒ तस्मा॒त् तस्मा᳚द् रुन्धे रुन्धे॒ तस्मा᳚त् । \newline
35. तस्मा॒थ् स्वाहा॑कृताः॒ स्वाहा॑कृता॒ स्तस्मा॒त् तस्मा॒थ् स्वाहा॑कृताः । \newline
36. स्वाहा॑कृता॒ हेम॒न्॒. हेम॒न् थ्स्वाहा॑कृताः॒ स्वाहा॑कृता॒ हेमन्न्॑ । \newline
37. स्वाहा॑कृता॒ इति॒ स्वाहा᳚ - कृ॒ताः॒ । \newline
38. हेम॑न् प॒शवः॑ प॒शवो॒ हेम॒न्॒. हेम॑न् प॒शवः॑ । \newline
39. प॒शवो ऽवाव॑ प॒शवः॑ प॒शवो ऽव॑ । \newline
40. अव॑ सीदन्ति सीद॒ न्त्यवाव॑ सीदन्ति । \newline
41. सी॒द॒न्ति॒ स॒मिधः॑ स॒मिधः॑ सीदन्ति सीदन्ति स॒मिधः॑ । \newline
42. स॒मिधो॑ यजति यजति स॒मिधः॑ स॒मिधो॑ यजति । \newline
43. स॒मिध॒ इति॑ सं - इधः॑ । \newline
44. य॒ज॒ त्यु॒षस॑ उ॒षसो॑ यजति यज त्यु॒षसः॑ । \newline
45. उ॒षस॑ ए॒वैवोषस॑ उ॒षस॑ ए॒व । \newline
46. ए॒व दे॒वता॑नाम् दे॒वता॑ना मे॒वैव दे॒वता॑नाम् । \newline
47. दे॒वता॑ना॒ मवाव॑ दे॒वता॑नाम् दे॒वता॑ना॒ मव॑ । \newline
48. अव॑ रुन्धे रु॒न्धे ऽवाव॑ रुन्धे । \newline
49. रु॒न्धे॒ तनू॒नपा॑त॒म् तनू॒नपा॑तꣳ रुन्धे रुन्धे॒ तनू॒नपा॑तम् । \newline
50. तनू॒नपा॑तं ॅयजति यजति॒ तनू॒नपा॑त॒म् तनू॒नपा॑तं ॅयजति । \newline
51. तनू॒नपा॑त॒मिति॒ तनू᳚ - नपा॑तम् । \newline
52. य॒ज॒ति॒ य॒ज्ञ्ं ॅय॒ज्ञ्ं ॅय॑जति यजति य॒ज्ञ्म् । \newline
53. य॒ज्ञ् मे॒वैव य॒ज्ञ्ं ॅय॒ज्ञ् मे॒व । \newline
54. ए॒वावा वै॒वै वाव॑ । \newline
55. अव॑ रुन्धे रु॒न्धे ऽवाव॑ रुन्धे । \newline
56. रु॒न्ध॒ इ॒ड इ॒डो रु॑न्धे रुन्ध इ॒डः । \newline

\textbf{Ghana Paata } \newline

1. स॒मिधो॑ यजति यजति स॒मिधः॑ स॒मिधो॑ यजति वस॒न्तं ॅव॑स॒न्तं ॅय॑जति स॒मिधः॑ स॒मिधो॑ यजति वस॒न्तम् । \newline
2. स॒मिध॒ इति॑ सं - इधः॑ । \newline
3. य॒ज॒ति॒ व॒स॒न्तं ॅव॑स॒न्तं ॅय॑जति यजति वस॒न्त मे॒वैव व॑स॒न्तं ॅय॑जति यजति वस॒न्त मे॒व । \newline
4. व॒स॒न्त मे॒वैव व॑स॒न्तं ॅव॑स॒न्त मे॒व र्तू॒ना मृ॑तू॒ना मे॒व व॑स॒न्तं ॅव॑स॒न्त मे॒व र्तू॒नाम् । \newline
5. ए॒व र्तू॒ना मृ॑तू॒ना मे॒वैव र्तू॒ना मवावा᳚ र्‌तू॒ना मे॒वैव र्‌तू॒ना मव॑ । \newline
6. ऋ॒तू॒ना मवावा᳚ र्‌तू॒ना मृ॑तू॒ना मव॑ रुन्धे रु॒न्धे ऽवा᳚ र्‌तू॒ना मृ॑तू॒ना मव॑ रुन्धे । \newline
7. अव॑ रुन्धे रु॒न्धे ऽवाव॑ रुन्धे॒ तनू॒नपा॑त॒म् तनू॒नपा॑तꣳ रु॒न्धे ऽवाव॑ रुन्धे॒ तनू॒नपा॑तम् । \newline
8. रु॒न्धे॒ तनू॒नपा॑त॒म् तनू॒नपा॑तꣳ रुन्धे रुन्धे॒ तनू॒नपा॑तं ॅयजति यजति॒ तनू॒नपा॑तꣳ रुन्धे रुन्धे॒ तनू॒नपा॑तं ॅयजति । \newline
9. तनू॒नपा॑तं ॅयजति यजति॒ तनू॒नपा॑त॒म् तनू॒नपा॑तं ॅयजति ग्री॒ष्मम् ग्री॒ष्मं ॅय॑जति॒ तनू॒नपा॑त॒म् तनू॒नपा॑तं ॅयजति ग्री॒ष्मम् । \newline
10. तनू॒नपा॑त॒मिति॒ तनू᳚ - नपा॑तम् । \newline
11. य॒ज॒ति॒ ग्री॒ष्मम् ग्री॒ष्मं ॅय॑जति यजति ग्री॒ष्म मे॒वैव ग्री॒ष्मं ॅय॑जति यजति ग्री॒ष्म मे॒व । \newline
12. ग्री॒ष्म मे॒वैव ग्री॒ष्मम् ग्री॒ष्म मे॒वावा वै॒व ग्री॒ष्मम् ग्री॒ष्म मे॒वाव॑ । \newline
13. ए॒वावा वै॒वै वाव॑ रुन्धे रु॒न्धे ऽवै॒वै वाव॑ रुन्धे । \newline
14. अव॑ रुन्धे रु॒न्धे ऽवाव॑ रुन्ध इ॒ड इ॒डो रु॒न्धे ऽवाव॑ रुन्ध इ॒डः । \newline
15. रु॒न्ध॒ इ॒ड इ॒डो रु॑न्धे रुन्ध इ॒डो य॑जति यजती॒डो रु॑न्धे रुन्ध इ॒डो य॑जति । \newline
16. इ॒डो य॑जति यजती॒ड इ॒डो य॑जति व॒र्॒.षा व॒र्॒.षा य॑जती॒ड इ॒डो य॑जति व॒र्॒.षाः । \newline
17. य॒ज॒ति॒ व॒र्॒.षा व॒र्॒.षा य॑जति यजति व॒र्॒.षा ए॒वैव व॒र्॒.षा य॑जति यजति व॒र्॒.षा ए॒व । \newline
18. व॒र्॒.षा ए॒वैव व॒र्॒.षा व॒र्॒.षा ए॒वावा वै॒व व॒र्॒.षा व॒र्॒.षा ए॒वाव॑ । \newline
19. ए॒वावा वै॒वै वाव॑ रुन्धे रु॒न्धे ऽवै॒वै वाव॑ रुन्धे । \newline
20. अव॑ रुन्धे रु॒न्धे ऽवाव॑ रुन्धे ब॒र्॒.हिर् ब॒र्॒.ही रु॒न्धे ऽवाव॑ रुन्धे ब॒र्॒.हिः । \newline
21. रु॒न्धे॒ ब॒र्॒.हिर् ब॒र्॒.ही रु॑न्धे रुन्धे ब॒र्॒.हिर् य॑जति यजति ब॒र्॒.ही रु॑न्धे रुन्धे ब॒र्॒.हिर् य॑जति । \newline
22. ब॒र्॒.हिर् य॑जति यजति ब॒र्॒.हिर् ब॒र्॒.हिर् य॑जति श॒रदꣳ॑ श॒रदं॑ ॅयजति ब॒र्॒.हिर् ब॒र्॒.हिर् य॑जति श॒रद᳚म् । \newline
23. य॒ज॒ति॒ श॒रदꣳ॑ श॒रदं॑ ॅयजति यजति श॒रद॑ मे॒वैव श॒रदं॑ ॅयजति यजति श॒रद॑ मे॒व । \newline
24. श॒रद॑ मे॒वैव श॒रदꣳ॑ श॒रद॑ मे॒वावा वै॒व श॒रदꣳ॑ श॒रद॑ मे॒वाव॑ । \newline
25. ए॒वावा वै॒वै वाव॑ रुन्धे रु॒न्धे ऽवै॒वै वाव॑ रुन्धे । \newline
26. अव॑ रुन्धे रु॒न्धे ऽवाव॑ रुन्धे स्वाहाका॒रꣳ स्वा॑हाका॒रꣳ रु॒न्धे ऽवाव॑ रुन्धे स्वाहाका॒रम् । \newline
27. रु॒न्धे॒ स्वा॒हा॒का॒रꣳ स्वा॑हाका॒रꣳ रु॑न्धे रुन्धे स्वाहाका॒रं ॅय॑जति यजति स्वाहाका॒रꣳ रु॑न्धे रुन्धे स्वाहाका॒रं ॅय॑जति । \newline
28. स्वा॒हा॒का॒रं ॅय॑जति यजति स्वाहाका॒रꣳ स्वा॑हाका॒रं ॅय॑जति हेम॒न्तꣳ हे॑म॒न्तं ॅय॑जति स्वाहाका॒रꣳ स्वा॑हाका॒रं ॅय॑जति हेम॒न्तम् । \newline
29. स्वा॒हा॒का॒रमिति॑ स्वाहा - का॒रम् । \newline
30. य॒ज॒ति॒ हे॒म॒न्तꣳ हे॑म॒न्तं ॅय॑जति यजति हेम॒न्त मे॒वैव हे॑म॒न्तं ॅय॑जति यजति हेम॒न्त मे॒व । \newline
31. हे॒म॒न्त मे॒वैव हे॑म॒न्तꣳ हे॑म॒न्त मे॒वावावै॒व हे॑म॒न्तꣳ हे॑म॒न्त मे॒वाव॑ । \newline
32. ए॒वावा वै॒वै वाव॑ रुन्धे रु॒न्धे ऽवै॒वै वाव॑ रुन्धे । \newline
33. अव॑ रुन्धे रु॒न्धे ऽवाव॑ रुन्धे॒ तस्मा॒त् तस्मा᳚द् रु॒न्धे ऽवाव॑ रुन्धे॒ तस्मा᳚त् । \newline
34. रु॒न्धे॒ तस्मा॒त् तस्मा᳚द् रुन्धे रुन्धे॒ तस्मा॒थ् स्वाहा॑कृताः॒ स्वाहा॑कृता॒ स्तस्मा᳚द् रुन्धे रुन्धे॒ तस्मा॒थ् स्वाहा॑कृताः । \newline
35. तस्मा॒थ् स्वाहा॑कृताः॒ स्वाहा॑कृता॒ स्तस्मा॒त् तस्मा॒थ् स्वाहा॑कृता॒ हेम॒न्॒. हेम॒न् थ्स्वाहा॑कृता॒ स्तस्मा॒त् तस्मा॒थ् स्वाहा॑कृता॒ हेमन्न्॑ । \newline
36. स्वाहा॑कृता॒ हेम॒न्॒. हेम॒न् थ्स्वाहा॑कृताः॒ स्वाहा॑कृता॒ हेम॑न् प॒शवः॑ प॒शवो॒ हेम॒न् थ्स्वाहा॑कृताः॒ स्वाहा॑कृता॒ हेम॑न् प॒शवः॑ । \newline
37. स्वाहा॑कृता॒ इति॒ स्वाहा᳚ - कृ॒ताः॒ । \newline
38. हेम॑न् प॒शवः॑ प॒शवो॒ हेम॒न्॒. हेम॑न् प॒शवो ऽवाव॑ प॒शवो॒ हेम॒न्॒. हेम॑न् प॒शवो ऽव॑ । \newline
39. प॒शवो ऽवाव॑ प॒शवः॑ प॒शवो ऽव॑ सीदन्ति सीद॒न्त्यव॑ प॒शवः॑ प॒शवो ऽव॑ सीदन्ति । \newline
40. अव॑ सीदन्ति सीद॒ न्त्यवाव॑ सीदन्ति स॒मिधः॑ स॒मिधः॑ सीद॒ न्त्यवाव॑ सीदन्ति स॒मिधः॑ । \newline
41. सी॒द॒न्ति॒ स॒मिधः॑ स॒मिधः॑ सीदन्ति सीदन्ति स॒मिधो॑ यजति यजति स॒मिधः॑ सीदन्ति सीदन्ति स॒मिधो॑ यजति । \newline
42. स॒मिधो॑ यजति यजति स॒मिधः॑ स॒मिधो॑ यज त्यु॒षस॑ उ॒षसो॑ यजति स॒मिधः॑ स॒मिधो॑ यज त्यु॒षसः॑ । \newline
43. स॒मिध॒ इति॑ सं - इधः॑ । \newline
44. य॒ज॒ त्यु॒षस॑ उ॒षसो॑ यजति यज त्यु॒षस॑ ए॒वैवोषसो॑ यजति यज त्यु॒षस॑ ए॒व । \newline
45. उ॒षस॑ ए॒वैवोषस॑ उ॒षस॑ ए॒व दे॒वता॑नाम् दे॒वता॑ना मे॒वोषस॑ उ॒षस॑ ए॒व दे॒वता॑नाम् । \newline
46. ए॒व दे॒वता॑नाम् दे॒वता॑ना मे॒वैव दे॒वता॑ना॒ मवाव॑ दे॒वता॑ना मे॒वैव दे॒वता॑ना॒ मव॑ । \newline
47. दे॒वता॑ना॒ मवाव॑ दे॒वता॑नाम् दे॒वता॑ना॒ मव॑ रुन्धे रु॒न्धे ऽव॑ दे॒वता॑नाम् दे॒वता॑ना॒ मव॑ रुन्धे । \newline
48. अव॑ रुन्धे रु॒न्धे ऽवाव॑ रुन्धे॒ तनू॒नपा॑त॒म् तनू॒नपा॑तꣳ रु॒न्धे ऽवाव॑ रुन्धे॒ तनू॒नपा॑तम् । \newline
49. रु॒न्धे॒ तनू॒नपा॑त॒म् तनू॒नपा॑तꣳ रुन्धे रुन्धे॒ तनू॒नपा॑तं ॅयजति यजति॒ तनू॒नपा॑तꣳ रुन्धे रुन्धे॒ तनू॒नपा॑तं ॅयजति । \newline
50. तनू॒नपा॑तं ॅयजति यजति॒ तनू॒नपा॑त॒म् तनू॒नपा॑तं ॅयजति य॒ज्ञ्ं ॅय॒ज्ञ्ं ॅय॑जति॒ तनू॒नपा॑त॒म् तनू॒नपा॑तं ॅयजति य॒ज्ञ्म् । \newline
51. तनू॒नपा॑त॒मिति॒ तनू᳚ - नपा॑तम् । \newline
52. य॒ज॒ति॒ य॒ज्ञ्ं ॅय॒ज्ञ्ं ॅय॑जति यजति य॒ज्ञ् मे॒वैव य॒ज्ञ्ं ॅय॑जति यजति य॒ज्ञ् मे॒व । \newline
53. य॒ज्ञ् मे॒वैव य॒ज्ञ्ं ॅय॒ज्ञ् मे॒वावा वै॒व य॒ज्ञ्ं ॅय॒ज्ञ् मे॒वाव॑ । \newline
54. ए॒वावा वै॒वै वाव॑ रुन्धे रु॒न्धे ऽवै॒वै वाव॑ रुन्धे । \newline
55. अव॑ रुन्धे रु॒न्धे ऽवाव॑ रुन्ध इ॒ड इ॒डो रु॒न्धे ऽवाव॑ रुन्ध इ॒डः । \newline
56. रु॒न्ध॒ इ॒ड इ॒डो रु॑न्धे रुन्ध इ॒डो य॑जति यजती॒डो रु॑न्धे रुन्ध इ॒डो य॑जति । \newline
\pagebreak
\markright{ TS 2.6.1.2  \hfill https://www.vedavms.in \hfill}
\addcontentsline{toc}{section}{ TS 2.6.1.2 }
\section*{ TS 2.6.1.2 }

\textbf{TS 2.6.1.2 } \newline
\textbf{Samhita Paata} \newline

इ॒डो य॑जति प॒शूने॒वाव॑ रुन्धे ब॒र्.हिर्य॑जति प्र॒जामे॒वाव॑ रुन्धे स॒मान॑यत उप॒भृत॒स्तेजो॒ वा आज्यं॑ प्र॒जा ब॒र्॒.हिः प्र॒जास्वे॒व तेजो॑ दधाति स्वाहाका॒रं ॅय॑जति॒ वाच॑मे॒वाव॑ रुन्धे॒ दश॒ सं प॑द्यन्ते॒ दशा᳚क्षरा वि॒राडन्नं॑ ॅवि॒राड्वि॒राजै॒ वान्नाद्य॒मव॑ रुन्धे स॒मिधो॑ यजत्य॒स्मिन्ने॒व लो॒के प्रति॑तिष्ठति॒ तनू॒नपा॑तं ॅयजति - [  ] \newline

\textbf{Pada Paata} \newline

इ॒डः । य॒ज॒ति॒ । प॒शून् । ए॒व । अवेति॑ । रु॒न्धे॒ । ब॒र्॒.हिः । य॒ज॒ति॒ । प्र॒जामिति॑ प्र - जाम् । ए॒व । अवेति॑ । रु॒न्धे॒ । स॒मान॑यत॒ इति॑ सं - आन॑यते । उ॒प॒भृत॒ इत्यु॑प - भृतः॑ । तेजः॑ । वै । आज्य᳚म् । प्र॒जा इति॑ प्र - जाः । ब॒र्॒.हिः । प्र॒जास्विति॑ प्र - जासु॑ । ए॒व । तेजः॑ । द॒धा॒ति॒ । स्वा॒हा॒का॒रमिति॑ स्वाहा-का॒रम् । य॒ज॒ति॒ । वाच᳚म् । ए॒व । अवेति॑ । रु॒न्धे॒ । दश॑ । समिति॑ । प॒द्य॒न्ते॒ । दशा᳚क्ष॒रेति॒ दश॑ - अ॒क्ष॒रा॒ । वि॒राडिति॑ वि - राट् । अन्न᳚म् । वि॒राडिति॑ वि - राट् । वि॒राजेति॑ वि - राजा᳚ । ए॒व । अ॒न्नाद्य॒मित्य॑न्न - अद्य᳚म् । अवेति॑ । रु॒न्धे॒ । स॒मिध॒ इति॑ सं - इधः॑ । य॒ज॒ति॒ । अ॒स्मिन्न् । ए॒व । लो॒के । प्रतीति॑ । ति॒ष्ठ॒ति॒ । तनू॒नपा॑त॒मिति॒ तनू᳚ - नपा॑तम् । य॒ज॒ति॒ ।  \newline


\textbf{Krama Paata} \newline

इ॒डो य॑जति । य॒ज॒ति॒ प॒शून् । प॒शूने॒व । ए॒वाव॑ । अव॑ रुन्धे । रु॒न्धे॒ ब॒र्.॒हिः । ब॒र्.॒हिर् य॑जति । य॒ज॒ति॒ प्र॒जाम् । प्र॒जामे॒व । प्र॒जामिति॑ प्र - जाम् । ए॒वाव॑ । अव॑ रुन्धे । रु॒न्धे॒ स॒मान॑यते । स॒मान॑यत उप॒भृतः॑ । स॒मान॑यत॒ इति॑ सं - आन॑यते । उ॒प॒भृत॒,स्तेजः॑ । उ॒प॒भृत॒ इत्यु॑प - भृतः॑ । तेजो॒ वै । वा आज्य᳚म् । आज्य॑म् प्र॒जाः । प्र॒जा ब॒र्.॒हिः । प्र॒जा इति॑ प्र - जाः । ब॒र्.॒हिः प्र॒जासु॑ । प्र॒जास्वे॒व । प्र॒जास्विति॑ प्र - जासु॑ । ए॒व तेजः॑ । तेजो॑ दधाति । द॒धा॒ति॒ स्वा॒हा॒का॒रम् । स्वा॒हा॒का॒रं ॅय॑जति । स्वा॒हा॒का॒रमिति॑ स्वाहा - का॒रम् । य॒ज॒ति॒ वाच᳚म् । वाच॑मे॒व । ए॒वाव॑ । अव॑ रुन्धे । रु॒न्धे॒ दश॑ । दश॒ सम् । सं प॑द्यन्ते । प॒द्य॒न्ते॒ दशा᳚क्षरा । दशा᳚क्षरा वि॒राट् । दशा᳚क्ष॒रेति॒ दश॑ - अ॒क्ष॒रा॒ । वि॒राडन्न᳚म् । वि॒राडिति॑ वि - राट् । अन्नं॑ ॅवि॒राट् । वि॒राड् वि॒राजा᳚ । वि॒राडिति॑ वि - राट् । वि॒राजै॒व । वि॒राजेति॑ वि - राजा᳚ । ए॒वान्नाद्य᳚म् । अ॒न्नाद्य॒मव॑ । अ॒न्नाद्य॒मित्य॑न्न - अद्य᳚म् । अव॑ रुन्धे । रु॒न्धे॒ स॒मिधः॑ । स॒मिधो॑ यजति । स॒मिध॒ इति॑ सं - इधः॑ । य॒ज॒त्य॒स्मिन्न् । अ॒स्मिन्ने॒व । ए॒व लो॒के । लो॒के प्रति॑ । प्रति॑ तिष्ठिति । ति॒ष्ठ॒ति॒ तनू॒नपा॑तम् । तनू॒नपा॑तं ॅयजति । तनू॒नपा॑त॒मिति॒ तनू᳚ - नपा॑तम् । य॒ज॒ति॒ य॒ज्ञे \newline

\textbf{Jatai Paata} \newline

1. इ॒डो य॑जति यजती॒ड इ॒डो य॑जति । \newline
2. य॒ज॒ति॒ प॒शून् प॒शून्. य॑जति यजति प॒शून् । \newline
3. प॒शू ने॒वैव प॒शून् प॒शू ने॒व । \newline
4. ए॒वावा वै॒वै वाव॑ । \newline
5. अव॑ रुन्धे रु॒न्धे ऽवाव॑ रुन्धे । \newline
6. रु॒न्धे॒ ब॒र्॒.हिर् ब॒र्॒.ही रु॑न्धे रुन्धे ब॒र्॒.हिः । \newline
7. ब॒र्॒.हिर् य॑जति यजति ब॒र्॒.हिर् ब॒र्॒.हिर् य॑जति । \newline
8. य॒ज॒ति॒ प्र॒जाम् प्र॒जां ॅय॑जति यजति प्र॒जाम् । \newline
9. प्र॒जा मे॒वैव प्र॒जाम् प्र॒जा मे॒व । \newline
10. प्र॒जामिति॑ प्र - जाम् । \newline
11. ए॒वावा वै॒वै वाव॑ । \newline
12. अव॑ रुन्धे रु॒न्धे ऽवाव॑ रुन्धे । \newline
13. रु॒न्धे॒ स॒मान॑यते स॒मान॑यते रुन्धे रुन्धे स॒मान॑यते । \newline
14. स॒मान॑यत उप॒भृत॑ उप॒भृतः॑ स॒मान॑यते स॒मान॑यत उप॒भृतः॑ । \newline
15. स॒मान॑यत॒ इति॑ सं - आन॑यते । \newline
16. उ॒प॒भृत॒ स्तेज॒ स्तेज॑ उप॒भृत॑ उप॒भृत॒ स्तेजः॑ । \newline
17. उ॒प॒भृत॒ इत्यु॑प - भृतः॑ । \newline
18. तेजो॒ वै वै तेज॒ स्तेजो॒ वै । \newline
19. वा आज्य॒ माज्यं॒ ॅवै वा आज्य᳚म् । \newline
20. आज्य॑म् प्र॒जाः प्र॒जा आज्य॒ माज्य॑म् प्र॒जाः । \newline
21. प्र॒जा ब॒र्॒.हिर् ब॒र्॒.हिः प्र॒जाः प्र॒जा ब॒र्॒.हिः । \newline
22. प्र॒जा इति॑ प्र - जाः । \newline
23. ब॒र्॒.हिः प्र॒जासु॑ प्र॒जासु॑ ब॒र्॒.हिर् ब॒र्॒.हिः प्र॒जासु॑ । \newline
24. प्र॒जा स्वे॒वैव प्र॒जासु॑ प्र॒जा स्वे॒व । \newline
25. प्र॒जास्विति॑ प्र - जासु॑ । \newline
26. ए॒व तेज॒ स्तेज॑ ए॒वैव तेजः॑ । \newline
27. तेजो॑ दधाति दधाति॒ तेज॒ स्तेजो॑ दधाति । \newline
28. द॒धा॒ति॒ स्वा॒हा॒का॒रꣳ स्वा॑हाका॒रम् द॑धाति दधाति स्वाहाका॒रम् । \newline
29. स्वा॒हा॒का॒रं ॅय॑जति यजति स्वाहाका॒रꣳ स्वा॑हाका॒रं ॅय॑जति । \newline
30. स्वा॒हा॒का॒रमिति॑ स्वाहा - का॒रम् । \newline
31. य॒ज॒ति॒ वाचं॒ ॅवाचं॑ ॅयजति यजति॒ वाच᳚म् । \newline
32. वाच॑ मे॒वैव वाचं॒ ॅवाच॑ मे॒व । \newline
33. ए॒वावा वै॒वै वाव॑ । \newline
34. अव॑ रुन्धे रु॒न्धे ऽवाव॑ रुन्धे । \newline
35. रु॒न्धे॒ दश॒ दश॑ रुन्धे रुन्धे॒ दश॑ । \newline
36. दश॒ सꣳ सम् दश॒ दश॒ सम् । \newline
37. सम् प॑द्यन्ते पद्यन्ते॒ सꣳ सम् प॑द्यन्ते । \newline
38. प॒द्य॒न्ते॒ दशा᳚क्षरा॒ दशा᳚क्षरा पद्यन्ते पद्यन्ते॒ दशा᳚क्षरा । \newline
39. दशा᳚क्षरा वि॒राड् वि॒राड् दशा᳚क्षरा॒ दशा᳚क्षरा वि॒राट् । \newline
40. दशा᳚क्ष॒रेति॒ दश॑ - अ॒क्ष॒रा॒ । \newline
41. वि॒राडन्न॒ मन्नं॑ ॅवि॒राड् वि॒रा डन्न᳚म् । \newline
42. वि॒राडिति॑ वि - राट् । \newline
43. अन्नं॑ ॅवि॒राड् वि॒रा डन्न॒ मन्नं॑ ॅवि॒राट् । \newline
44. वि॒राड् वि॒राजा॑ वि॒राजा॑ वि॒राड् वि॒राड् वि॒राजा᳚ । \newline
45. वि॒राडिति॑ वि - राट् । \newline
46. वि॒रा जै॒वैव वि॒राजा॑ वि॒राजै॒व । \newline
47. वि॒राजेति॑ वि - राजा᳚ । \newline
48. ए॒वा न्नाद्य॑ म॒न्नाद्य॑ मे॒वैवा न्नाद्य᳚म् । \newline
49. अ॒न्नाद्य॒ मवावा॒ न्नाद्य॑ म॒न्नाद्य॒ मव॑ । \newline
50. अ॒न्नाद्य॒मित्य॑न्न - अद्य᳚म् । \newline
51. अव॑ रुन्धे रु॒न्धे ऽवाव॑ रुन्धे । \newline
52. रु॒न्धे॒ स॒मिधः॑ स॒मिधो॑ रुन्धे रुन्धे स॒मिधः॑ । \newline
53. स॒मिधो॑ यजति यजति स॒मिधः॑ स॒मिधो॑ यजति । \newline
54. स॒मिध॒ इति॑ सं - इधः॑ । \newline
55. य॒ज॒ त्य॒स्मिन् न॒स्मिन्. य॑जति यज त्य॒स्मिन्न् । \newline
56. अ॒स्मिन् ने॒वैवास्मिन् न॒स्मिन् ने॒व । \newline
57. ए॒व लो॒के लो॒क ए॒वैव लो॒के । \newline
58. लो॒के प्रति॒ प्रति॑ लो॒के लो॒के प्रति॑ । \newline
59. प्रति॑ तिष्ठति तिष्ठति॒ प्रति॒ प्रति॑ तिष्ठति । \newline
60. ति॒ष्ठ॒ति॒ तनू॒नपा॑त॒म् तनू॒नपा॑तम् तिष्ठति तिष्ठति॒ तनू॒नपा॑तम् । \newline
61. तनू॒नपा॑तं ॅयजति यजति॒ तनू॒नपा॑त॒म् तनू॒नपा॑तं ॅयजति । \newline
62. तनू॒नपा॑त॒मिति॒ तनू᳚ - नपा॑तम् । \newline
63. य॒ज॒ति॒ य॒ज्ञे य॒ज्ञे य॑जति यजति य॒ज्ञे । \newline

\textbf{Ghana Paata } \newline

1. इ॒डो य॑जति यजती॒ड इ॒डो य॑जति प॒शून् प॒शून्. य॑जती॒ड इ॒डो य॑जति प॒शून् । \newline
2. य॒ज॒ति॒ प॒शून् प॒शून्. य॑जति यजति प॒शू ने॒वैव प॒शून्. य॑जति यजति प॒शू ने॒व । \newline
3. प॒शू ने॒वैव प॒शून् प॒शू ने॒वावा वै॒व प॒शून् प॒शू ने॒वाव॑ । \newline
4. ए॒वावा वै॒वै वाव॑ रुन्धे रु॒न्धे ऽवै॒वै वाव॑ रुन्धे । \newline
5. अव॑ रुन्धे रु॒न्धे ऽवाव॑ रुन्धे ब॒र्॒.हिर् ब॒र्॒.ही रु॒न्धे ऽवाव॑ रुन्धे ब॒र्॒.हिः । \newline
6. रु॒न्धे॒ ब॒र्॒.हिर् ब॒र्॒.ही रु॑न्धे रुन्धे ब॒र्॒.हिर् य॑जति यजति ब॒र्॒.ही रु॑न्धे रुन्धे ब॒र्॒.हिर् य॑जति । \newline
7. ब॒र्॒.हिर् य॑जति यजति ब॒र्॒.हिर् ब॒र्॒.हिर् य॑जति प्र॒जाम् प्र॒जां ॅय॑जति ब॒र्॒.हिर् ब॒र्॒.हिर् य॑जति प्र॒जाम् । \newline
8. य॒ज॒ति॒ प्र॒जाम् प्र॒जां ॅय॑जति यजति प्र॒जा मे॒वैव प्र॒जां ॅय॑जति यजति प्र॒जा मे॒व । \newline
9. प्र॒जा मे॒वैव प्र॒जाम् प्र॒जा मे॒वावा वै॒व प्र॒जाम् प्र॒जा मे॒वाव॑ । \newline
10. प्र॒जामिति॑ प्र - जाम् । \newline
11. ए॒वावा वै॒वै वाव॑ रुन्धे रु॒न्धे ऽवै॒वै वाव॑ रुन्धे । \newline
12. अव॑ रुन्धे रु॒न्धे ऽवाव॑ रुन्धे स॒मान॑यते स॒मान॑यते रु॒न्धे ऽवाव॑ रुन्धे स॒मान॑यते । \newline
13. रु॒न्धे॒ स॒मान॑यते स॒मान॑यते रुन्धे रुन्धे स॒मान॑यत उप॒भृत॑ उप॒भृतः॑ स॒मान॑यते रुन्धे रुन्धे स॒मान॑यत उप॒भृतः॑ । \newline
14. स॒मान॑यत उप॒भृत॑ उप॒भृतः॑ स॒मान॑यते स॒मान॑यत उप॒भृत॒ स्तेज॒ स्तेज॑ उप॒भृतः॑ स॒मान॑यते स॒मान॑यत उप॒भृत॒ स्तेजः॑ । \newline
15. स॒मान॑यत॒ इति॑ सं - आन॑यते । \newline
16. उ॒प॒भृत॒ स्तेज॒ स्तेज॑ उप॒भृत॑ उप॒भृत॒ स्तेजो॒ वै वै तेज॑ उप॒भृत॑ उप॒भृत॒ स्तेजो॒ वै । \newline
17. उ॒प॒भृत॒ इत्यु॑प - भृतः॑ । \newline
18. तेजो॒ वै वै तेज॒ स्तेजो॒ वा आज्य॒ माज्यं॒ ॅवै तेज॒ स्तेजो॒ वा आज्य᳚म् । \newline
19. वा आज्य॒ माज्यं॒ ॅवै वा आज्य॑म् प्र॒जाः प्र॒जा आज्यं॒ ॅवै वा आज्य॑म् प्र॒जाः । \newline
20. आज्य॑म् प्र॒जाः प्र॒जा आज्य॒ माज्य॑म् प्र॒जा ब॒र्॒.हिर् ब॒र्॒.हिः प्र॒जा आज्य॒ माज्य॑म् प्र॒जा ब॒र्॒.हिः । \newline
21. प्र॒जा ब॒र्॒.हिर् ब॒र्॒.हिः प्र॒जाः प्र॒जा ब॒र्॒.हिः प्र॒जासु॑ प्र॒जासु॑ ब॒र्॒.हिः प्र॒जाः प्र॒जा ब॒र्॒.हिः प्र॒जासु॑ । \newline
22. प्र॒जा इति॑ प्र - जाः । \newline
23. ब॒र्॒.हिः प्र॒जासु॑ प्र॒जासु॑ ब॒र्॒.हिर् ब॒र्॒.हिः प्र॒जा स्वे॒वैव प्र॒जासु॑ ब॒र्॒.हिर् ब॒र्॒.हिः प्र॒जास्वे॒व । \newline
24. प्र॒जा स्वे॒वैव प्र॒जासु॑ प्र॒जा स्वे॒व तेज॒ स्तेज॑ ए॒व प्र॒जासु॑ प्र॒जास्वे॒व तेजः॑ । \newline
25. प्र॒जास्विति॑ प्र - जासु॑ । \newline
26. ए॒व तेज॒ स्तेज॑ ए॒वैव तेजो॑ दधाति दधाति॒ तेज॑ ए॒वैव तेजो॑ दधाति । \newline
27. तेजो॑ दधाति दधाति॒ तेज॒ स्तेजो॑ दधाति स्वाहाका॒रꣳ स्वा॑हाका॒रम् द॑धाति॒ तेज॒ स्तेजो॑ दधाति स्वाहाका॒रम् । \newline
28. द॒धा॒ति॒ स्वा॒हा॒का॒रꣳ स्वा॑हाका॒रम् द॑धाति दधाति स्वाहाका॒रं ॅय॑जति यजति स्वाहाका॒रम् द॑धाति दधाति स्वाहाका॒रं ॅय॑जति । \newline
29. स्वा॒हा॒का॒रं ॅय॑जति यजति स्वाहाका॒रꣳ स्वा॑हाका॒रं ॅय॑जति॒ वाचं॒ ॅवाचं॑ ॅयजति स्वाहाका॒रꣳ स्वा॑हाका॒रं ॅय॑जति॒ वाच᳚म् । \newline
30. स्वा॒हा॒का॒रमिति॑ स्वाहा - का॒रम् । \newline
31. य॒ज॒ति॒ वाचं॒ ॅवाचं॑ ॅयजति यजति॒ वाच॑ मे॒वैव वाचं॑ ॅयजति यजति॒ वाच॑ मे॒व । \newline
32. वाच॑ मे॒वैव वाचं॒ ॅवाच॑ मे॒वावा वै॒व वाचं॒ ॅवाच॑ मे॒वाव॑ । \newline
33. ए॒वावा वै॒वै वाव॑ रुन्धे रु॒न्धे ऽवै॒वै वाव॑ रुन्धे । \newline
34. अव॑ रुन्धे रु॒न्धे ऽवाव॑ रुन्धे॒ दश॒ दश॑ रु॒न्धे ऽवाव॑ रुन्धे॒ दश॑ । \newline
35. रु॒न्धे॒ दश॒ दश॑ रुन्धे रुन्धे॒ दश॒ सꣳ सम् दश॑ रुन्धे रुन्धे॒ दश॒ सम् । \newline
36. दश॒ सꣳ सम् दश॒ दश॒ सम् प॑द्यन्ते पद्यन्ते॒ सम् दश॒ दश॒ सम् प॑द्यन्ते । \newline
37. सम् प॑द्यन्ते पद्यन्ते॒ सꣳ सम् प॑द्यन्ते॒ दशा᳚क्षरा॒ दशा᳚क्षरा पद्यन्ते॒ सꣳ सम् प॑द्यन्ते॒ दशा᳚क्षरा । \newline
38. प॒द्य॒न्ते॒ दशा᳚क्षरा॒ दशा᳚क्षरा पद्यन्ते पद्यन्ते॒ दशा᳚क्षरा वि॒राड् वि॒राड् दशा᳚क्षरा पद्यन्ते पद्यन्ते॒ दशा᳚क्षरा वि॒राट् । \newline
39. दशा᳚क्षरा वि॒राड् वि॒राड् दशा᳚क्षरा॒ दशा᳚क्षरा वि॒राडन्न॒ मन्नं॑ ॅवि॒राड् दशा᳚क्षरा॒ दशा᳚क्षरा वि॒राडन्न᳚म् । \newline
40. दशा᳚क्ष॒रेति॒ दश॑ - अ॒क्ष॒रा॒ । \newline
41. वि॒राडन्न॒ मन्नं॑ ॅवि॒राड् वि॒राडन्नं॑ ॅवि॒राड् वि॒राडन्नं॑ ॅवि॒राड् वि॒राडन्नं॑ ॅवि॒राट् । \newline
42. वि॒राडिति॑ वि - राट् । \newline
43. अन्नं॑ ॅवि॒राड् वि॒राडन्न॒ मन्नं॑ ॅवि॒राड् वि॒राजा॑ वि॒राजा॑ वि॒राडन्न॒ मन्नं॑ ॅवि॒राड् वि॒राजा᳚ । \newline
44. वि॒राड् वि॒राजा॑ वि॒राजा॑ वि॒राड् वि॒राड् वि॒राजै॒वैव वि॒राजा॑ वि॒राड् वि॒राड् वि॒राजै॒व । \newline
45. वि॒राडिति॑ वि - राट् । \newline
46. वि॒राजै॒वैव वि॒राजा॑ वि॒राजै॒वा न्नाद्य॑ म॒न्नाद्य॑ मे॒व वि॒राजा॑ वि॒राजै॒वा न्नाद्य᳚म् । \newline
47. वि॒राजेति॑ वि - राजा᳚ । \newline
48. ए॒वा न्नाद्य॑ म॒न्नाद्य॑ मे॒वैवा न्नाद्य॒ मवावा॒ न्नाद्य॑ मे॒वैवा न्नाद्य॒ मव॑ । \newline
49. अ॒न्नाद्य॒ मवावा॒ न्नाद्य॑ म॒न्नाद्य॒ मव॑ रुन्धे रु॒न्धे ऽवा॒न्नाद्य॑ म॒न्नाद्य॒ मव॑ रुन्धे । \newline
50. अ॒न्नाद्य॒मित्य॑न्न - अद्य᳚म् । \newline
51. अव॑ रुन्धे रु॒न्धे ऽवाव॑ रुन्धे स॒मिधः॑ स॒मिधो॑ रु॒न्धे ऽवाव॑ रुन्धे स॒मिधः॑ । \newline
52. रु॒न्धे॒ स॒मिधः॑ स॒मिधो॑ रुन्धे रुन्धे स॒मिधो॑ यजति यजति स॒मिधो॑ रुन्धे रुन्धे स॒मिधो॑ यजति । \newline
53. स॒मिधो॑ यजति यजति स॒मिधः॑ स॒मिधो॑ यज त्य॒स्मिन् न॒स्मिन्. य॑जति स॒मिधः॑ स॒मिधो॑ यज त्य॒स्मिन्न् । \newline
54. स॒मिध॒ इति॑ सं - इधः॑ । \newline
55. य॒ज॒ त्य॒स्मिन् न॒स्मिन्. य॑जति यज त्य॒स्मिन् ने॒वैवास्मिन्. य॑जति यज त्य॒स्मिन् ने॒व । \newline
56. अ॒स्मिन् ने॒वैवास्मिन् न॒स्मिन् ने॒व लो॒के लो॒क ए॒वास्मिन् न॒स्मिन् ने॒व लो॒के । \newline
57. ए॒व लो॒के लो॒क ए॒वैव लो॒के प्रति॒ प्रति॑ लो॒क ए॒वैव लो॒के प्रति॑ । \newline
58. लो॒के प्रति॒ प्रति॑ लो॒के लो॒के प्रति॑ तिष्ठति तिष्ठति॒ प्रति॑ लो॒के लो॒के प्रति॑ तिष्ठति । \newline
59. प्रति॑ तिष्ठति तिष्ठति॒ प्रति॒ प्रति॑ तिष्ठति॒ तनू॒नपा॑त॒म् तनू॒नपा॑तम् तिष्ठति॒ प्रति॒ प्रति॑ तिष्ठति॒ तनू॒नपा॑तम् । \newline
60. ति॒ष्ठ॒ति॒ तनू॒नपा॑त॒म् तनू॒नपा॑तम् तिष्ठति तिष्ठति॒ तनू॒नपा॑तं ॅयजति यजति॒ तनू॒नपा॑तम् तिष्ठति तिष्ठति॒ तनू॒नपा॑तं ॅयजति । \newline
61. तनू॒नपा॑तं ॅयजति यजति॒ तनू॒नपा॑त॒म् तनू॒नपा॑तं ॅयजति य॒ज्ञे य॒ज्ञे य॑जति॒ तनू॒नपा॑त॒म् तनू॒नपा॑तं ॅयजति य॒ज्ञे । \newline
62. तनू॒नपा॑त॒मिति॒ तनू᳚ - नपा॑तम् । \newline
63. य॒ज॒ति॒ य॒ज्ञे य॒ज्ञे य॑जति यजति य॒ज्ञ् ए॒वैव य॒ज्ञे य॑जति यजति य॒ज्ञ् ए॒व । \newline
\pagebreak
\markright{ TS 2.6.1.3  \hfill https://www.vedavms.in \hfill}
\addcontentsline{toc}{section}{ TS 2.6.1.3 }
\section*{ TS 2.6.1.3 }

\textbf{TS 2.6.1.3 } \newline
\textbf{Samhita Paata} \newline

य॒ज्ञ् ए॒वान्तरि॑क्षे॒ प्रति॑ तिष्ठती॒डो य॑जति प॒शुष्वे॒व प्रति॑तिष्ठति ब॒र॒.हिर्य॑जति॒ य ए॒व दे॑व॒यानाः॒ पन्था॑न॒स्तेष्वे॒व प्रति॑तिष्ठति स्वाहाका॒रं ॅय॑जति सुव॒र्ग ए॒व लो॒के प्रति॑ तिष्ठत्ये॒ताव॑न्तो॒ वै दे॑वलो॒कास्तेष्वे॒व य॑थापू॒र्वं प्रति॑तिष्ठति देवासु॒रा ए॒षु लो॒केष्व॑स्पर्द्धन्त॒ ते दे॒वाः प्र॑या॒जैरे॒भ्यो लो॒केभ्यो ऽसु॑रा॒न् प्राणु॑दन्त॒ तत् प्र॑या॒जानां᳚ - [  ] \newline

\textbf{Pada Paata} \newline

य॒ज्ञे । ए॒व । अ॒न्तरि॑क्षे । प्रतीति॑ । ति॒ष्ठ॒ति॒ । इ॒डः । य॒ज॒ति॒ । प॒शुषु॑ । ए॒व । प्रतीति॑ । ति॒ष्ठ॒ति॒ । ब॒र्॒.हिः । य॒ज॒ति॒ । ये । ए॒व । दे॒व॒याना॒ इति॑ देव - यानाः᳚ । पन्था॑नः । तेषु॑ । ए॒व । प्रतीति॑ । ति॒ष्ठ॒ति॒ । स्वा॒हा॒का॒रमिति॑ स्वाहा - का॒रम् । य॒ज॒ति॒ । सु॒व॒र्ग इति॑ सुवः - गे । ए॒व । लो॒के । प्रतीति॑ । ति॒ष्ठ॒ति॒ । ए॒ताव॑न्तः । वै । दे॒व॒लो॒का इति॑ देव - लो॒काः । तेषु॑ । ए॒व । य॒था॒पू॒र्वमिति॑ यथा - पू॒र्वम् । प्रतीति॑ । ति॒ष्ठ॒ति॒ । दे॒वा॒सु॒रा इति॑ देव-अ॒सु॒राः । ए॒षु । लो॒केषु॑ । अ॒स्प॒र्द्ध॒न्त॒ । ते । दे॒वाः । प्र॒या॒जैरिति॑ प्र-या॒जैः । ए॒भ्यः । लो॒केभ्यः॑ । असु॑रान् । प्रेति॑ । अ॒नु॒द॒न्त॒ । तत् । प्र॒या॒जाना॒मिति॑ प्र - या॒जाना᳚म् ।  \newline


\textbf{Krama Paata} \newline

य॒ज्ञ् ए॒व । ए॒वान्तरि॑क्षे । अ॒न्तरि॑क्षे॒ प्रति॑ । प्रति॑ तिष्ठति । ति॒ष्ठ॒ती॒डः । इ॒डो य॑जति । य॒ज॒ति॒ प॒शुषु॑ । प॒शुष्वे॒व । ए॒व प्रति॑ । प्रति॑ तिष्ठति । ति॒ष्ठ॒ति॒ ब॒र्.॒हिः । ब॒र्.॒हिर् य॑जति । य॒ज॒ति॒ ये । य ए॒व । ए॒व दे॑व॒यानाः᳚ । दे॒व॒यानाः॒ पन्था॑नः । दे॒व॒याना॒ इति॑ देव - यानाः᳚ । पन्था॑न॒स्तेषु॑ । तेष्वे॒व । ए॒व प्रति॑ । प्रति॑ तिष्ठति । ति॒ष्ठ॒ति॒ स्वा॒हा॒का॒रम् । स्वा॒हा॒का॒रं ॅय॑जति । स्वा॒हा॒का॒रमिति॑ स्वाहा - का॒रम् । य॒ज॒ति॒ सु॒व॒र्गे । सु॒व॒र्ग ए॒व । सु॒व॒र्ग इति॑ सुवः - गे । ए॒व लो॒के । लो॒के प्रति॑ । प्रति॑ तिष्ठति । ति॒ष्ठ॒त्ये॒ताव॑न्तः । ए॒ताव॑न्तो॒ वै । वै दे॑वलो॒काः । दे॒व॒लो॒कास्तेषु॑ । दे॒व॒लो॒का इति॑ देव - लो॒काः । तेष्वे॒व । ए॒व य॑थापू॒र्वम् । य॒था॒पू॒र्वम् प्रति॑ । य॒था॒पू॒र्वमिति॑ यथा - पू॒र्वम् । प्रति॑ तिष्ठति । ति॒ष्ठ॒ति॒ दे॒वा॒सु॒राः । दे॒वा॒सु॒रा ए॒षु । दे॒वा॒सु॒रा इति॑ देव - अ॒सु॒राः । ए॒षु लो॒केषु॑ । लो॒केष्व॑स्पर्द्धन्त । अ॒स्प॒र्द्ध॒न्त॒ ते । ते दे॒वाः । दे॒वाः प्र॑या॒जैः । प्र॒या॒जैरे॒भ्यः । प्र॒या॒जैरिति॑ प्र - या॒जैः । ए॒भ्यो लो॒केभ्यः॑ । लो॒केभ्यो ऽसु॑रान् । असु॑रा॒न् प्र । प्राणु॑दन्त । अ॒नु॒द॒न्त॒ तत् । तत् प्र॑या॒जाना᳚म् । प्र॒या॒जाना᳚म् प्रयाज॒त्वम् । प्र॒या॒जाना॒मिति॑ प्र - या॒जाना᳚म् \newline

\textbf{Jatai Paata} \newline

1. य॒ज्ञ् ए॒वैव य॒ज्ञे य॒ज्ञ् ए॒व । \newline
2. ए॒वा न्तरि॑क्षे॒ ऽन्तरि॑क्ष ए॒वैवा न्तरि॑क्षे । \newline
3. अ॒न्तरि॑क्षे॒ प्रति॒ प्रत्य॒न्तरि॑क्षे॒ ऽन्तरि॑क्षे॒ प्रति॑ । \newline
4. प्रति॑ तिष्ठति तिष्ठति॒ प्रति॒ प्रति॑ तिष्ठति । \newline
5. ति॒ष्ठ॒ती॒ड इ॒ड स्ति॑ष्ठति तिष्ठती॒डः । \newline
6. इ॒डो य॑जति यजती॒ड इ॒डो य॑जति । \newline
7. य॒ज॒ति॒ प॒शुषु॑ प॒शुषु॑ यजति यजति प॒शुषु॑ । \newline
8. प॒शु ष्वे॒वैव प॒शुषु॑ प॒शु ष्वे॒व । \newline
9. ए॒व प्रति॒ प्रत्ये॒वैव प्रति॑ । \newline
10. प्रति॑ तिष्ठति तिष्ठति॒ प्रति॒ प्रति॑ तिष्ठति । \newline
11. ति॒ष्ठ॒ति॒ ब॒र्॒.हिर् ब॒र्॒.हि स्ति॑ष्ठति तिष्ठति ब॒र्॒.हिः । \newline
12. ब॒र्॒.हिर् य॑जति यजति ब॒र्॒.हिर् ब॒र्॒.हिर् य॑जति । \newline
13. य॒ज॒ति॒ ये ये य॑जति यजति॒ ये । \newline
14. य ए॒वैव ये य ए॒व । \newline
15. ए॒व दे॑व॒याना॑ देव॒याना॑ ए॒वैव दे॑व॒यानाः᳚ । \newline
16. दे॒व॒यानाः॒ पन्था॑नः॒ पन्था॑नो देव॒याना॑ देव॒यानाः॒ पन्था॑नः । \newline
17. दे॒व॒याना॒ इति॑ देव - यानाः᳚ । \newline
18. पन्था॑न॒ स्तेषु॒ तेषु॒ पन्था॑नः॒ पन्था॑न॒ स्तेषु॑ । \newline
19. तेष्वे॒वैव तेषु॒ तेष्वे॒व । \newline
20. ए॒व प्रति॒ प्रत्ये॒वैव प्रति॑ । \newline
21. प्रति॑ तिष्ठति तिष्ठति॒ प्रति॒ प्रति॑ तिष्ठति । \newline
22. ति॒ष्ठ॒ति॒ स्वा॒हा॒का॒रꣳ स्वा॑हाका॒रम् ति॑ष्ठति तिष्ठति स्वाहाका॒रम् । \newline
23. स्वा॒हा॒का॒रं ॅय॑जति यजति स्वाहाका॒रꣳ स्वा॑हाका॒रं ॅय॑जति । \newline
24. स्वा॒हा॒का॒रमिति॑ स्वाहा - का॒रम् । \newline
25. य॒ज॒ति॒ सु॒व॒र्गे सु॑व॒र्गे य॑जति यजति सुव॒र्गे । \newline
26. सु॒व॒र्ग ए॒वैव सु॑व॒र्गे सु॑व॒र्ग ए॒व । \newline
27. सु॒व॒र्ग इति॑ सुवः - गे । \newline
28. ए॒व लो॒के लो॒क ए॒वैव लो॒के । \newline
29. लो॒के प्रति॒ प्रति॑ लो॒के लो॒के प्रति॑ । \newline
30. प्रति॑ तिष्ठति तिष्ठति॒ प्रति॒ प्रति॑ तिष्ठति । \newline
31. ति॒ष्ठ॒ त्ये॒ताव॑न्त ए॒ताव॑न्त स्तिष्ठति तिष्ठ त्ये॒ताव॑न्तः । \newline
32. ए॒ताव॑न्तो॒ वै वा ए॒ताव॑न्त ए॒ताव॑न्तो॒ वै । \newline
33. वै दे॑वलो॒का दे॑वलो॒का वै वै दे॑वलो॒काः । \newline
34. दे॒व॒लो॒का स्तेषु॒ तेषु॑ देवलो॒का दे॑वलो॒का स्तेषु॑ । \newline
35. दे॒व॒लो॒का इति॑ देव - लो॒काः । \newline
36. तेष्वे॒वैव तेषु॒ तेष्वे॒व । \newline
37. ए॒व य॑थापू॒र्वं ॅय॑थापू॒र्व मे॒वैव य॑थापू॒र्वम् । \newline
38. य॒था॒पू॒र्वम् प्रति॒ प्रति॑ यथापू॒र्वं ॅय॑थापू॒र्वम् प्रति॑ । \newline
39. य॒था॒पू॒र्वमिति॑ यथा - पू॒र्वम् । \newline
40. प्रति॑ तिष्ठति तिष्ठति॒ प्रति॒ प्रति॑ तिष्ठति । \newline
41. ति॒ष्ठ॒ति॒ दे॒वा॒सु॒रा दे॑वासु॒रा स्ति॑ष्ठति तिष्ठति देवासु॒राः । \newline
42. दे॒वा॒सु॒रा ए॒ष्वे॑षु दे॑वासु॒रा दे॑वासु॒रा ए॒षु । \newline
43. दे॒वा॒सु॒रा इति॑ देव - अ॒सु॒राः । \newline
44. ए॒षु लो॒केषु॑ लो॒के ष्वे॒ष्वे॑षु लो॒केषु॑ । \newline
45. लो॒के ष्व॑स्पर्द्धन्ता स्पर्द्धन्त लो॒केषु॑ लो॒के ष्व॑स्पर्द्धन्त । \newline
46. अ॒स्प॒र्द्ध॒न्त॒ ते ते᳚ ऽस्पर्द्धन्ता स्पर्द्धन्त॒ ते । \newline
47. ते दे॒वा दे॒वा स्ते ते दे॒वाः । \newline
48. दे॒वाः प्र॑या॒जैः प्र॑या॒जैर् दे॒वा दे॒वाः प्र॑या॒जैः । \newline
49. प्र॒या॒जै रे॒भ्य ए॒भ्यः प्र॑या॒जैः प्र॑या॒जै रे॒भ्यः । \newline
50. प्र॒या॒जैरिति॑ प्र - या॒जैः । \newline
51. ए॒भ्यो लो॒केभ्यो॑ लो॒केभ्य॑ ए॒भ्य ए॒भ्यो लो॒केभ्यः॑ । \newline
52. लो॒केभ्यो ऽसु॑रा॒ नसु॑रान् ॅलो॒केभ्यो॑ लो॒केभ्यो ऽसु॑रान् । \newline
53. असु॑रा॒न् प्र प्रासु॑रा॒ नसु॑रा॒न् प्र । \newline
54. प्राणु॑दन्ता नुदन्त॒ प्र प्राणु॑दन्त । \newline
55. अ॒नु॒द॒न्त॒ तत् तद॑नुदन्ता नुदन्त॒ तत् । \newline
56. तत् प्र॑या॒जाना᳚म् प्रया॒जाना॒म् तत् तत् प्र॑या॒जाना᳚म् । \newline
57. प्र॒या॒जाना᳚म् प्रयाज॒त्वम् प्र॑याज॒त्वम् प्र॑या॒जाना᳚म् प्रया॒जाना᳚म् प्रयाज॒त्वम् । \newline
58. प्र॒या॒जाना॒मिति॑ प्र - या॒जाना᳚म् । \newline

\textbf{Ghana Paata } \newline

1. य॒ज्ञ् ए॒वैव य॒ज्ञे य॒ज्ञ् ए॒वान्तरि॑क्षे॒ ऽन्तरि॑क्ष ए॒व य॒ज्ञे य॒ज्ञ् ए॒वान्तरि॑क्षे । \newline
2. ए॒वान्तरि॑क्षे॒ ऽन्तरि॑क्ष ए॒वैवान्तरि॑क्षे॒ प्रति॒ प्रत्य॒न्तरि॑क्ष ए॒वैवान्तरि॑क्षे॒ प्रति॑ । \newline
3. अ॒न्तरि॑क्षे॒ प्रति॒ प्रत्य॒न्तरि॑क्षे॒ ऽन्तरि॑क्षे॒ प्रति॑ तिष्ठति तिष्ठति॒ प्रत्य॒न्तरि॑क्षे॒ ऽन्तरि॑क्षे॒ प्रति॑ तिष्ठति । \newline
4. प्रति॑ तिष्ठति तिष्ठति॒ प्रति॒ प्रति॑ तिष्ठती॒ड इ॒ड स्ति॑ष्ठति॒ प्रति॒ प्रति॑ तिष्ठती॒डः । \newline
5. ति॒ष्ठ॒ती॒ड इ॒ड स्ति॑ष्ठति तिष्ठती॒डो य॑जति यजती॒ड स्ति॑ष्ठति तिष्ठती॒डो य॑जति । \newline
6. इ॒डो य॑जति यजती॒ड इ॒डो य॑जति प॒शुषु॑ प॒शुषु॑ यजती॒ड इ॒डो य॑जति प॒शुषु॑ । \newline
7. य॒ज॒ति॒ प॒शुषु॑ प॒शुषु॑ यजति यजति प॒शु ष्वे॒वैव प॒शुषु॑ यजति यजति प॒शुष्वे॒व । \newline
8. प॒शु ष्वे॒वैव प॒शुषु॑ प॒शुष्वे॒व प्रति॒ प्रत्ये॒व प॒शुषु॑ प॒शु ष्वे॒व प्रति॑ । \newline
9. ए॒व प्रति॒ प्रत्ये॒वैव प्रति॑ तिष्ठति तिष्ठति॒ प्रत्ये॒वैव प्रति॑ तिष्ठति । \newline
10. प्रति॑ तिष्ठति तिष्ठति॒ प्रति॒ प्रति॑ तिष्ठति ब॒र्॒.हिर् ब॒र्॒.हि स्ति॑ष्ठति॒ प्रति॒ प्रति॑ तिष्ठति ब॒र्॒.हिः । \newline
11. ति॒ष्ठ॒ति॒ ब॒र्॒.हिर् ब॒र्॒.हि स्ति॑ष्ठति तिष्ठति ब॒र्॒.हिर् य॑जति यजति ब॒र्॒.हि स्ति॑ष्ठति तिष्ठति ब॒र्॒.हिर् य॑जति । \newline
12. ब॒र्॒.हिर् य॑जति यजति ब॒र्॒.हिर् ब॒र्॒.हिर् य॑जति॒ ये ये य॑जति ब॒र्॒.हिर् ब॒र्॒.हिर् य॑जति॒ ये । \newline
13. य॒ज॒ति॒ ये ये य॑जति यजति॒ य ए॒वैव ये य॑जति यजति॒ य ए॒व । \newline
14. य ए॒वैव ये य ए॒व दे॑व॒याना॑ देव॒याना॑ ए॒व ये य ए॒व दे॑व॒यानाः᳚ । \newline
15. ए॒व दे॑व॒याना॑ देव॒याना॑ ए॒वैव दे॑व॒यानाः॒ पन्था॑नः॒ पन्था॑नो देव॒याना॑ ए॒वैव दे॑व॒यानाः॒ पन्था॑नः । \newline
16. दे॒व॒यानाः॒ पन्था॑नः॒ पन्था॑नो देव॒याना॑ देव॒यानाः॒ पन्था॑न॒ स्तेषु॒ तेषु॒ पन्था॑नो देव॒याना॑ देव॒यानाः॒ पन्था॑न॒ स्तेषु॑ । \newline
17. दे॒व॒याना॒ इति॑ देव - यानाः᳚ । \newline
18. पन्था॑न॒ स्तेषु॒ तेषु॒ पन्था॑नः॒ पन्था॑न॒ स्तेष्वे॒वैव तेषु॒ पन्था॑नः॒ पन्था॑न॒ स्तेष्वे॒व । \newline
19. तेष्वे॒वैव तेषु॒ तेष्वे॒व प्रति॒ प्रत्ये॒व तेषु॒ तेष्वे॒व प्रति॑ । \newline
20. ए॒व प्रति॒ प्रत्ये॒वैव प्रति॑ तिष्ठति तिष्ठति॒ प्रत्ये॒वैव प्रति॑ तिष्ठति । \newline
21. प्रति॑ तिष्ठति तिष्ठति॒ प्रति॒ प्रति॑ तिष्ठति स्वाहाका॒रꣳ स्वा॑हाका॒रम् ति॑ष्ठति॒ प्रति॒ प्रति॑ तिष्ठति स्वाहाका॒रम् । \newline
22. ति॒ष्ठ॒ति॒ स्वा॒हा॒का॒रꣳ स्वा॑हाका॒रम् ति॑ष्ठति तिष्ठति स्वाहाका॒रं ॅय॑जति यजति स्वाहाका॒रम् ति॑ष्ठति तिष्ठति स्वाहाका॒रं ॅय॑जति । \newline
23. स्वा॒हा॒का॒रं ॅय॑जति यजति स्वाहाका॒रꣳ स्वा॑हाका॒रं ॅय॑जति सुव॒र्गे सु॑व॒र्गे य॑जति स्वाहाका॒रꣳ स्वा॑हाका॒रं ॅय॑जति सुव॒र्गे । \newline
24. स्वा॒हा॒का॒रमिति॑ स्वाहा - का॒रम् । \newline
25. य॒ज॒ति॒ सु॒व॒र्गे सु॑व॒र्गे य॑जति यजति सुव॒र्ग ए॒वैव सु॑व॒र्गे य॑जति यजति सुव॒र्ग ए॒व । \newline
26. सु॒व॒र्ग ए॒वैव सु॑व॒र्गे सु॑व॒र्ग ए॒व लो॒के लो॒क ए॒व सु॑व॒र्गे सु॑व॒र्ग ए॒व लो॒के । \newline
27. सु॒व॒र्ग इति॑ सुवः - गे । \newline
28. ए॒व लो॒के लो॒क ए॒वैव लो॒के प्रति॒ प्रति॑ लो॒क ए॒वैव लो॒के प्रति॑ । \newline
29. लो॒के प्रति॒ प्रति॑ लो॒के लो॒के प्रति॑ तिष्ठति तिष्ठति॒ प्रति॑ लो॒के लो॒के प्रति॑ तिष्ठति । \newline
30. प्रति॑ तिष्ठति तिष्ठति॒ प्रति॒ प्रति॑ तिष्ठ त्ये॒ताव॑न्त ए॒ताव॑न्त स्तिष्ठति॒ प्रति॒ प्रति॑ तिष्ठ त्ये॒ताव॑न्तः । \newline
31. ति॒ष्ठ॒ त्ये॒ताव॑न्त ए॒ताव॑न्त स्तिष्ठति तिष्ठ त्ये॒ताव॑न्तो॒ वै वा ए॒ताव॑न्त स्तिष्ठति तिष्ठ त्ये॒ताव॑न्तो॒ वै । \newline
32. ए॒ताव॑न्तो॒ वै वा ए॒ताव॑न्त ए॒ताव॑न्तो॒ वै दे॑वलो॒का दे॑वलो॒का वा ए॒ताव॑न्त ए॒ताव॑न्तो॒ वै दे॑वलो॒काः । \newline
33. वै दे॑वलो॒का दे॑वलो॒का वै वै दे॑वलो॒का स्तेषु॒ तेषु॑ देवलो॒का वै वै दे॑वलो॒का स्तेषु॑ । \newline
34. दे॒व॒लो॒का स्तेषु॒ तेषु॑ देवलो॒का दे॑वलो॒का स्तेष्वे॒वैव तेषु॑ देवलो॒का दे॑वलो॒का स्तेष्वे॒व । \newline
35. दे॒व॒लो॒का इति॑ देव - लो॒काः । \newline
36. तेष्वे॒वैव तेषु॒ तेष्वे॒व य॑थापू॒र्वं ॅय॑थापू॒र्व मे॒व तेषु॒ तेष्वे॒व य॑थापू॒र्वम् । \newline
37. ए॒व य॑थापू॒र्वं ॅय॑थापू॒र्व मे॒वैव य॑थापू॒र्वम् प्रति॒ प्रति॑ यथापू॒र्व मे॒वैव य॑थापू॒र्वम् प्रति॑ । \newline
38. य॒था॒पू॒र्वम् प्रति॒ प्रति॑ यथापू॒र्वं ॅय॑थापू॒र्वम् प्रति॑ तिष्ठति तिष्ठति॒ प्रति॑ यथापू॒र्वं ॅय॑थापू॒र्वम् प्रति॑ तिष्ठति । \newline
39. य॒था॒पू॒र्वमिति॑ यथा - पू॒र्वम् । \newline
40. प्रति॑ तिष्ठति तिष्ठति॒ प्रति॒ प्रति॑ तिष्ठति देवासु॒रा दे॑वासु॒रा स्ति॑ष्ठति॒ प्रति॒ प्रति॑ तिष्ठति देवासु॒राः । \newline
41. ति॒ष्ठ॒ति॒ दे॒वा॒सु॒रा दे॑वासु॒रा स्ति॑ष्ठति तिष्ठति देवासु॒रा ए॒ष्वे॑षु दे॑वासु॒रा स्ति॑ष्ठति तिष्ठति देवासु॒रा ए॒षु । \newline
42. दे॒वा॒सु॒रा ए॒ष्वे॑षु दे॑वासु॒रा दे॑वासु॒रा ए॒षु लो॒केषु॑ लो॒केष्वे॒षु दे॑वासु॒रा दे॑वासु॒रा ए॒षु लो॒केषु॑ । \newline
43. दे॒वा॒सु॒रा इति॑ देव - अ॒सु॒राः । \newline
44. ए॒षु लो॒केषु॑ लो॒केष्वे॒ष्वे॑षु लो॒के ष्व॑स्पर्द्धन्ता स्पर्द्धन्त लो॒केष्वे॒ष्वे॑षु लो॒के ष्व॑स्पर्द्धन्त । \newline
45. लो॒के ष्व॑स्पर्द्धन्ता स्पर्द्धन्त लो॒केषु॑ लो॒के ष्व॑स्पर्द्धन्त॒ ते ते᳚ ऽस्पर्द्धन्त लो॒केषु॑ लो॒के ष्व॑स्पर्द्धन्त॒ ते । \newline
46. अ॒स्प॒र्द्ध॒न्त॒ ते ते᳚ ऽस्पर्द्धन्ता स्पर्द्धन्त॒ ते दे॒वा दे॒वा स्ते᳚ ऽस्पर्द्धन्ता स्पर्द्धन्त॒ ते दे॒वाः । \newline
47. ते दे॒वा दे॒वा स्ते ते दे॒वाः प्र॑या॒जैः प्र॑या॒जैर् दे॒वा स्ते ते दे॒वाः प्र॑या॒जैः । \newline
48. दे॒वाः प्र॑या॒जैः प्र॑या॒जैर् दे॒वा दे॒वाः प्र॑या॒जै रे॒भ्य ए॒भ्यः प्र॑या॒जैर् दे॒वा दे॒वाः प्र॑या॒जैरे॒भ्यः । \newline
49. प्र॒या॒जै रे॒भ्य ए॒भ्यः प्र॑या॒जैः प्र॑या॒जै रे॒भ्यो लो॒केभ्यो॑ लो॒केभ्य॑ ए॒भ्यः प्र॑या॒जैः प्र॑या॒जै रे॒भ्यो लो॒केभ्यः॑ । \newline
50. प्र॒या॒जैरिति॑ प्र - या॒जैः । \newline
51. ए॒भ्यो लो॒केभ्यो॑ लो॒केभ्य॑ ए॒भ्य ए॒भ्यो लो॒केभ्यो ऽसु॑रा॒ नसु॑रान् ॅलो॒केभ्य॑ ए॒भ्य ए॒भ्यो लो॒केभ्यो ऽसु॑रान् । \newline
52. लो॒केभ्यो ऽसु॑रा॒ नसु॑रान् ॅलो॒केभ्यो॑ लो॒केभ्यो ऽसु॑रा॒न् प्र प्रासु॑रान् ॅलो॒केभ्यो॑ लो॒केभ्यो ऽसु॑रा॒न् प्र । \newline
53. असु॑रा॒न् प्र प्रासु॑रा॒ नसु॑रा॒न् प्राणु॑दन्ता नुदन्त॒ प्रासु॑रा॒ नसु॑रा॒न् प्राणु॑दन्त । \newline
54. प्राणु॑दन्ता नुदन्त॒ प्र प्राणु॑दन्त॒ तत् तद॑नुदन्त॒ प्र प्राणु॑दन्त॒ तत् । \newline
55. अ॒नु॒द॒न्त॒ तत् तद॑नुदन्ता नुदन्त॒ तत् प्र॑या॒जाना᳚म् प्रया॒जाना॒म् तद॑नुदन्ता नुदन्त॒ तत् प्र॑या॒जाना᳚म् । \newline
56. तत् प्र॑या॒जाना᳚म् प्रया॒जाना॒म् तत् तत् प्र॑या॒जाना᳚म् प्रयाज॒त्वम् प्र॑याज॒त्वम् प्र॑या॒जाना॒म् तत् तत् प्र॑या॒जाना᳚म् प्रयाज॒त्वम् । \newline
57. प्र॒या॒जाना᳚म् प्रयाज॒त्वम् प्र॑याज॒त्वम् प्र॑या॒जाना᳚म् प्रया॒जाना᳚म् प्रयाज॒त्वं ॅयस्य॒ यस्य॑ प्रयाज॒त्वम् प्र॑या॒जाना᳚म् प्रया॒जाना᳚म् प्रयाज॒त्वं ॅयस्य॑ । \newline
58. प्र॒या॒जाना॒मिति॑ प्र - या॒जाना᳚म् । \newline
\pagebreak
\markright{ TS 2.6.1.4  \hfill https://www.vedavms.in \hfill}
\addcontentsline{toc}{section}{ TS 2.6.1.4 }
\section*{ TS 2.6.1.4 }

\textbf{TS 2.6.1.4 } \newline
\textbf{Samhita Paata} \newline

प्रयाज॒त्वं ॅयस्यै॒वं ॅवि॒दुषः॑ प्रया॒जा इ॒ज्यन्ते॒ प्रैभ्यो लो॒केभ्यो॒ भ्रातृ॑व्यान्नुदते ऽभि॒क्रामं॑ जुहोत्य॒भिजि॑त्यै॒ यो वै प्र॑या॒जानां᳚ मिथु॒नं ॅवेद॒ प्र प्र॒जया॑ प॒शुभि॑ र्मिथु॒नै र्जा॑यते स॒मिधो॑ ब॒ह्वीरि॑व यजति॒ तनू॒नपा॑त॒मेक॑मिव मिथु॒नं तदि॒डो ब॒ह्वीरि॑व यजति ब॒र्॒.हिरेक॑मिव मिथु॒नं तदे॒तद्वै प्र॑या॒जानां᳚ मिथु॒नं ॅय ए॒वं ॅवेद॒ प्र - [  ] \newline

\textbf{Pada Paata} \newline

प्र॒या॒ज॒त्वमिति॑ प्रयाज - त्वम् । यस्य॑ । ए॒वम् । वि॒दुषः॑ । प्र॒या॒जा इति॑ प्र-या॒जाः । इ॒ज्यन्ते᳚ । प्रेति॑ । ए॒भ्यः । लो॒केभ्यः॑ । भ्रातृ॑व्यान् । नु॒द॒ते॒ । अ॒भि॒क्राम॒मित्य॑भि - क्राम᳚म् । जु॒हो॒ति॒ । अ॒भिजि॑त्या॒ इत्य॒भि - जि॒त्यै॒ । यः । वै । प्र॒या॒जाना॒मिति॑ प्र - या॒जाना᳚म् । मि॒थु॒नम् । वेद॑ । प्रेति॑ । प्र॒जयेति॑ प्र - जया᳚ । प॒शुभि॒रिति॑ प॒शु - भिः॒ । मि॒थु॒नैः । जा॒य॒ते॒ । स॒मिध॒ इति॑ सं - इधः॑ । ब॒ह्वीः । इ॒व॒ । य॒ज॒ति॒ । तनू॒नपा॑त॒मिति॒ तनू᳚ - नपा॑तम् । एक᳚म् । इ॒व॒ । मि॒थु॒नम् । तत् । इ॒डः । ब॒ह्वीः । इ॒व॒ । य॒ज॒ति॒ । ब॒र्॒.हिः । एक᳚म् । इ॒व॒ । मि॒थु॒नम् । तत् । ए॒तत् । वै । प्र॒या॒जाना॒मिति॑ प्र - या॒जाना᳚म् । मि॒थु॒नम् । यः । ए॒वम् । वेद॑ । प्रेति॑ ।  \newline


\textbf{Krama Paata} \newline

प्र॒या॒ज॒त्वं ॅयस्य॑ । प्र॒या॒ज॒त्वमिति॑ प्रयाज - त्वम् । यस्यै॒वम् । ए॒वं ॅवि॒दुषः॑ । वि॒दुषः॑ प्रया॒जाः । प्र॒या॒जा इ॒ज्यन्ते᳚ । प्र॒या॒जा इति॑ प्र - या॒जाः । इ॒ज्यन्ते॒ प्र । प्रैभ्यः । ए॒भ्यो लो॒केभ्यः॑ । लो॒केभ्यो॒ भ्रातृ॑व्यान् । भ्रातृ॑व्यान् नुदते । नु॒द॒ते॒ ऽभि॒क्राम᳚म् । अ॒भि॒क्राम॑म् जुहोति । अ॒भि॒क्राम॒मित्य॑भि - क्राम᳚म् । जु॒हो॒त्य॒भिजि॑त्यै । अ॒भिजि॑त्यै॒ यः । अ॒भिजि॑त्या॒ इत्य॒भि - जि॒त्यै॒ । यो वै । वै प्र॑या॒जाना᳚म् । प्र॒या॒जाना᳚म् मिथु॒नम् । प्र॒या॒जाना॒मिति॑ प्र - या॒जाना᳚म् । मि॒थु॒नं ॅवेद॑ । वेद॒ प्र । प्र प्र॒जया᳚ । प्र॒जया॑ प॒शुभिः॑ । प्र॒जयेति॑ प्र - जया᳚ । प॒शुभि॑र् मिथु॒नैः । प॒शुभि॒रिति॑ प॒शु - भिः॒ । मि॒थु॒नैर् जा॑यते । जा॒य॒ते॒ स॒मिधः॑ । स॒मिधो॑ ब॒ह्वीः । स॒मिध॒ इति॑ सं - इधः॑ । ब॒ह्वीरि॑व । इ॒व॒ य॒ज॒ति॒ । य॒ज॒ति॒ तनू॒नपा॑तम् । तनू॒नपा॑त॒मेक᳚म् । तनू॒नपा॑त॒मिति॒ तनू᳚ - नपा॑तम् । एक॑मिव । इ॒व॒ मि॒थु॒नम् । मि॒थु॒नम् तत् । तदि॒डः । इ॒डो ब॒ह्वीः । ब॒ह्वीरि॑व । इ॒व॒ य॒ज॒ति॒ । य॒ज॒ति॒ ब॒र्.॒हिः । ब॒र्.॒हिरेक᳚म् । एक॑मिव । इ॒व॒ मि॒थु॒नम् । मि॒थु॒नम् तत् । तदे॒तत् । ए॒तद्वै । वै प्र॑या॒जाना᳚म् । प्र॒या॒जाना᳚म् मिथु॒नम् । प्र॒या॒जाना॒मिति॑ प्र - या॒जाना᳚म् । मि॒थु॒नं ॅयः । य ए॒वम् । ए॒वं ॅवेद॑ । वेद॒ प्र । प्र प्र॒जया᳚ \newline

\textbf{Jatai Paata} \newline

1. प्र॒या॒ज॒त्वं ॅयस्य॒ यस्य॑ प्रयाज॒त्वम् प्र॑याज॒त्वं ॅयस्य॑ । \newline
2. प्र॒या॒ज॒त्वमिति॑ प्रयाज - त्वम् । \newline
3. यस्यै॒व मे॒वं ॅयस्य॒ यस्यै॒वम् । \newline
4. ए॒वं ॅवि॒दुषो॑ वि॒दुष॑ ए॒व मे॒वं ॅवि॒दुषः॑ । \newline
5. वि॒दुषः॑ प्रया॒जाः प्र॑या॒जा वि॒दुषो॑ वि॒दुषः॑ प्रया॒जाः । \newline
6. प्र॒या॒जा इ॒ज्यन्त॑ इ॒ज्यन्ते᳚ प्रया॒जाः प्र॑या॒जा इ॒ज्यन्ते᳚ । \newline
7. प्र॒या॒जा इति॑ प्र - या॒जाः । \newline
8. इ॒ज्यन्ते॒ प्र प्रे ज्यन्त॑ इ॒ज्यन्ते॒ प्र । \newline
9. प्रैभ्य ए॒भ्यः प्र प्रैभ्यः । \newline
10. ए॒भ्यो लो॒केभ्यो॑ लो॒केभ्य॑ ए॒भ्य ए॒भ्यो लो॒केभ्यः॑ । \newline
11. लो॒केभ्यो॒ भ्रातृ॑व्या॒न् भ्रातृ॑व्यान् ॅलो॒केभ्यो॑ लो॒केभ्यो॒ भ्रातृ॑व्यान् । \newline
12. भ्रातृ॑व्यान् नुदते नुदते॒ भ्रातृ॑व्या॒न् भ्रातृ॑व्यान् नुदते । \newline
13. नु॒द॒ते॒ ऽभि॒क्राम॑ मभि॒क्राम॑म् नुदते नुदते ऽभि॒क्राम᳚म् । \newline
14. अ॒भि॒क्राम॑म् जुहोति जुहो त्यभि॒क्राम॑ मभि॒क्राम॑म् जुहोति । \newline
15. अ॒भि॒क्राम॒मित्य॑भि - क्राम᳚म् । \newline
16. जु॒हो॒ त्य॒भिजि॑त्या अ॒भिजि॑त्यै जुहोति जुहो त्य॒भिजि॑त्यै । \newline
17. अ॒भिजि॑त्यै॒ यो यो॑ ऽभिजि॑त्या अ॒भिजि॑त्यै॒ यः । \newline
18. अ॒भिजि॑त्या॒ इत्य॒भि - जि॒त्यै॒ । \newline
19. यो वै वै यो यो वै । \newline
20. वै प्र॑या॒जाना᳚म् प्रया॒जानां॒ ॅवै वै प्र॑या॒जाना᳚म् । \newline
21. प्र॒या॒जाना᳚म् मिथु॒नम् मि॑थु॒नम् प्र॑या॒जाना᳚म् प्रया॒जाना᳚म् मिथु॒नम् । \newline
22. प्र॒या॒जाना॒मिति॑ प्र - या॒जाना᳚म् । \newline
23. मि॒थु॒नं ॅवेद॒ वेद॑ मिथु॒नम् मि॑थु॒नं ॅवेद॑ । \newline
24. वेद॒ प्र प्र वेद॒ वेद॒ प्र । \newline
25. प्र प्र॒जया᳚ प्र॒जया॒ प्र प्र प्र॒जया᳚ । \newline
26. प्र॒जया॑ प॒शुभिः॑ प॒शुभिः॑ प्र॒जया᳚ प्र॒जया॑ प॒शुभिः॑ । \newline
27. प्र॒जयेति॑ प्र - जया᳚ । \newline
28. प॒शुभि॑र् मिथु॒नैर् मि॑थु॒नैः प॒शुभिः॑ प॒शुभि॑र् मिथु॒नैः । \newline
29. प॒शुभि॒रिति॑ प॒शु - भिः॒ । \newline
30. मि॒थु॒नैर् जा॑यते जायते मिथु॒नैर् मि॑थु॒नैर् जा॑यते । \newline
31. जा॒य॒ते॒ स॒मिधः॑ स॒मिधो॑ जायते जायते स॒मिधः॑ । \newline
32. स॒मिधो॑ ब॒ह्वीर् ब॒ह्वीः स॒मिधः॑ स॒मिधो॑ ब॒ह्वीः । \newline
33. स॒मिध॒ इति॑ सं - इधः॑ । \newline
34. ब॒ह्वी रि॑वे व ब॒ह्वीर् ब॒ह्वी रि॑व । \newline
35. इ॒व॒ य॒ज॒ति॒ य॒ज॒ती॒वे॒ व॒ य॒ज॒ति॒ । \newline
36. य॒ज॒ति॒ तनू॒नपा॑त॒म् तनू॒नपा॑तं ॅयजति यजति॒ तनू॒नपा॑तम् । \newline
37. तनू॒नपा॑त॒ मेक॒ मेक॒म् तनू॒नपा॑त॒म् तनू॒नपा॑त॒ मेक᳚म् । \newline
38. तनू॒नपा॑त॒मिति॒ तनू᳚ - नपा॑तम् । \newline
39. एक॑ मिवे॒ वैक॒ मेक॑ मिव । \newline
40. इ॒व॒ मि॒थु॒नम् मि॑थु॒न मि॑वे व मिथु॒नम् । \newline
41. मि॒थु॒नम् तत् तन् मि॑थु॒नम् मि॑थु॒नम् तत् । \newline
42. तदि॒ड इ॒ड स्तत् तदि॒डः । \newline
43. इ॒डो ब॒ह्वीर् ब॒ह्वी रि॒ड इ॒डो ब॒ह्वीः । \newline
44. ब॒ह्वी रि॑वे व ब॒ह्वीर् ब॒ह्वी रि॑व । \newline
45. इ॒व॒ य॒ज॒ति॒ य॒ज॒ती॒वे॒ व॒ य॒ज॒ति॒ । \newline
46. य॒ज॒ति॒ ब॒र्॒.हिर् ब॒र्॒.हिर् य॑जति यजति ब॒र्॒.हिः । \newline
47. ब॒र्॒.हि रेक॒ मेक॑म् ब॒र्॒.हिर् ब॒र्॒.हि रेक᳚म् । \newline
48. एक॑ मिवे॒ वैक॒ मेक॑ मिव । \newline
49. इ॒व॒ मि॒थु॒नम् मि॑थु॒न मि॑वे व मिथु॒नम् । \newline
50. मि॒थु॒नम् तत् तन् मि॑थु॒नम् मि॑थु॒नम् तत् । \newline
51. तदे॒त दे॒तत् तत् तदे॒तत् । \newline
52. ए॒तद् वै वा ए॒त दे॒तद् वै । \newline
53. वै प्र॑या॒जाना᳚म् प्रया॒जानां॒ ॅवै वै प्र॑या॒जाना᳚म् । \newline
54. प्र॒या॒जाना᳚म् मिथु॒नम् मि॑थु॒नम् प्र॑या॒जाना᳚म् प्रया॒जाना᳚म् मिथु॒नम् । \newline
55. प्र॒या॒जाना॒मिति॑ प्र - या॒जाना᳚म् । \newline
56. मि॒थु॒नं ॅयो यो मि॑थु॒नम् मि॑थु॒नं ॅयः । \newline
57. य ए॒व मे॒वं ॅयो य ए॒वम् । \newline
58. ए॒वं ॅवेद॒ वेदै॒व मे॒वं ॅवेद॑ । \newline
59. वेद॒ प्र प्र वेद॒ वेद॒ प्र । \newline
60. प्र प्र॒जया᳚ प्र॒जया॒ प्र प्र प्र॒जया᳚ । \newline

\textbf{Ghana Paata } \newline

1. प्र॒या॒ज॒त्वं ॅयस्य॒ यस्य॑ प्रयाज॒त्वम् प्र॑याज॒त्वं ॅयस्यै॒व मे॒वं ॅयस्य॑ प्रयाज॒त्वम् प्र॑याज॒त्वं ॅयस्यै॒वम् । \newline
2. प्र॒या॒ज॒त्वमिति॑ प्रयाज - त्वम् । \newline
3. यस्यै॒व मे॒वं ॅयस्य॒ यस्यै॒वं ॅवि॒दुषो॑ वि॒दुष॑ ए॒वं ॅयस्य॒ यस्यै॒वं ॅवि॒दुषः॑ । \newline
4. ए॒वं ॅवि॒दुषो॑ वि॒दुष॑ ए॒व मे॒वं ॅवि॒दुषः॑ प्रया॒जाः प्र॑या॒जा वि॒दुष॑ ए॒व मे॒वं ॅवि॒दुषः॑ प्रया॒जाः । \newline
5. वि॒दुषः॑ प्रया॒जाः प्र॑या॒जा वि॒दुषो॑ वि॒दुषः॑ प्रया॒जा इ॒ज्यन्त॑ इ॒ज्यन्ते᳚ प्रया॒जा वि॒दुषो॑ वि॒दुषः॑ प्रया॒जा इ॒ज्यन्ते᳚ । \newline
6. प्र॒या॒जा इ॒ज्यन्त॑ इ॒ज्यन्ते᳚ प्रया॒जाः प्र॑या॒जा इ॒ज्यन्ते॒ प्र प्रे ज्यन्ते᳚ प्रया॒जाः प्र॑या॒जा इ॒ज्यन्ते॒ प्र । \newline
7. प्र॒या॒जा इति॑ प्र - या॒जाः । \newline
8. इ॒ज्यन्ते॒ प्र प्रे ज्यन्त॑ इ॒ज्यन्ते॒ प्रैभ्य ए॒भ्यः प्रे ज्यन्त॑ इ॒ज्यन्ते॒ प्रैभ्यः । \newline
9. प्रैभ्य ए॒भ्यः प्र प्रैभ्यो लो॒केभ्यो॑ लो॒केभ्य॑ ए॒भ्यः प्र प्रैभ्यो लो॒केभ्यः॑ । \newline
10. ए॒भ्यो लो॒केभ्यो॑ लो॒केभ्य॑ ए॒भ्य ए॒भ्यो लो॒केभ्यो॒ भ्रातृ॑व्या॒न् भ्रातृ॑व्यान् ॅलो॒केभ्य॑ ए॒भ्य ए॒भ्यो लो॒केभ्यो॒ भ्रातृ॑व्यान् । \newline
11. लो॒केभ्यो॒ भ्रातृ॑व्या॒न् भ्रातृ॑व्यान् ॅलो॒केभ्यो॑ लो॒केभ्यो॒ भ्रातृ॑व्यान् नुदते नुदते॒ भ्रातृ॑व्यान् ॅलो॒केभ्यो॑ लो॒केभ्यो॒ भ्रातृ॑व्यान् नुदते । \newline
12. भ्रातृ॑व्यान् नुदते नुदते॒ भ्रातृ॑व्या॒न् भ्रातृ॑व्यान् नुदते ऽभि॒क्राम॑ मभि॒क्राम॑म् नुदते॒ भ्रातृ॑व्या॒न् भ्रातृ॑व्यान् नुदते ऽभि॒क्राम᳚म् । \newline
13. नु॒द॒ते॒ ऽभि॒क्राम॑ मभि॒क्राम॑म् नुदते नुदते ऽभि॒क्राम॑म् जुहोति जुहोत्यभि॒क्राम॑ न्नुदते नुदते ऽभि॒क्राम॑म् जुहोति । \newline
14. अ॒भि॒क्राम॑म् जुहोति जुहो त्यभि॒क्राम॑ मभि॒क्राम॑म् जुहो त्य॒भिजि॑त्या अ॒भिजि॑त्यै जुहो त्यभि॒क्राम॑ मभि॒क्राम॑म् जुहो त्य॒भिजि॑त्यै । \newline
15. अ॒भि॒क्राम॒मित्य॑भि - क्राम᳚म् । \newline
16. जु॒हो॒ त्य॒भिजि॑त्या अ॒भिजि॑त्यै जुहोति जुहो त्य॒भिजि॑त्यै॒ यो यो॑ ऽभिजि॑त्यै जुहोति जुहो त्य॒भिजि॑त्यै॒ यः । \newline
17. अ॒भिजि॑त्यै॒ यो यो॑ ऽभिजि॑त्या अ॒भिजि॑त्यै॒ यो वै वै यो॑ ऽभिजि॑त्या अ॒भिजि॑त्यै॒ यो वै । \newline
18. अ॒भिजि॑त्या॒ इत्य॒भि - जि॒त्यै॒ । \newline
19. यो वै वै यो यो वै प्र॑या॒जाना᳚म् प्रया॒जानां॒ ॅवै यो यो वै प्र॑या॒जाना᳚म् । \newline
20. वै प्र॑या॒जाना᳚म् प्रया॒जानां॒ ॅवै वै प्र॑या॒जाना᳚म् मिथु॒नम् मि॑थु॒नम् प्र॑या॒जानां॒ ॅवै वै प्र॑या॒जाना᳚म् मिथु॒नम् । \newline
21. प्र॒या॒जाना᳚म् मिथु॒नम् मि॑थु॒नम् प्र॑या॒जाना᳚म् प्रया॒जाना᳚म् मिथु॒नं ॅवेद॒ वेद॑ मिथु॒नम् प्र॑या॒जाना᳚म् प्रया॒जाना᳚म् मिथु॒नं ॅवेद॑ । \newline
22. प्र॒या॒जाना॒मिति॑ प्र - या॒जाना᳚म् । \newline
23. मि॒थु॒नं ॅवेद॒ वेद॑ मिथु॒नम् मि॑थु॒नं ॅवेद॒ प्र प्र वेद॑ मिथु॒नम् मि॑थु॒नं ॅवेद॒ प्र । \newline
24. वेद॒ प्र प्र वेद॒ वेद॒ प्र प्र॒जया᳚ प्र॒जया॒ प्र वेद॒ वेद॒ प्र प्र॒जया᳚ । \newline
25. प्र प्र॒जया᳚ प्र॒जया॒ प्र प्र प्र॒जया॑ प॒शुभिः॑ प॒शुभिः॑ प्र॒जया॒ प्र प्र प्र॒जया॑ प॒शुभिः॑ । \newline
26. प्र॒जया॑ प॒शुभिः॑ प॒शुभिः॑ प्र॒जया᳚ प्र॒जया॑ प॒शुभि॑र् मिथु॒नैर् मि॑थु॒नैः प॒शुभिः॑ प्र॒जया᳚ प्र॒जया॑ प॒शुभि॑र् मिथु॒नैः । \newline
27. प्र॒जयेति॑ प्र - जया᳚ । \newline
28. प॒शुभि॑र् मिथु॒नैर् मि॑थु॒नैः प॒शुभिः॑ प॒शुभि॑र् मिथु॒नैर् जा॑यते जायते मिथु॒नैः प॒शुभिः॑ प॒शुभि॑र् मिथु॒नैर् जा॑यते । \newline
29. प॒शुभि॒रिति॑ प॒शु - भिः॒ । \newline
30. मि॒थु॒नैर् जा॑यते जायते मिथु॒नैर् मि॑थु॒नैर् जा॑यते स॒मिधः॑ स॒मिधो॑ जायते मिथु॒नैर् मि॑थु॒नैर् जा॑यते स॒मिधः॑ । \newline
31. जा॒य॒ते॒ स॒मिधः॑ स॒मिधो॑ जायते जायते स॒मिधो॑ ब॒ह्वीर् ब॒ह्वीः स॒मिधो॑ जायते जायते स॒मिधो॑ ब॒ह्वीः । \newline
32. स॒मिधो॑ ब॒ह्वीर् ब॒ह्वीः स॒मिधः॑ स॒मिधो॑ ब॒ह्वीरि॑वे व ब॒ह्वीः स॒मिधः॑ स॒मिधो॑ ब॒ह्वीरि॑व । \newline
33. स॒मिध॒ इति॑ सं - इधः॑ । \newline
34. ब॒ह्वीरि॑वे व ब॒ह्वीर् ब॒ह्वीरि॑व यजति यजतीव ब॒ह्वीर् ब॒ह्वीरि॑व यजति । \newline
35. इ॒व॒ य॒ज॒ति॒ य॒ज॒ती॒वे॒ व॒ य॒ज॒ति॒ तनू॒नपा॑त॒म् तनू॒नपा॑तं ॅयजतीवे व यजति॒ तनू॒नपा॑तम् । \newline
36. य॒ज॒ति॒ तनू॒नपा॑त॒म् तनू॒नपा॑तं ॅयजति यजति॒ तनू॒नपा॑त॒ मेक॒ मेक॒म् तनू॒नपा॑तं ॅयजति यजति॒ तनू॒नपा॑त॒ मेक᳚म् । \newline
37. तनू॒नपा॑त॒ मेक॒ मेक॒म् तनू॒नपा॑त॒म् तनू॒नपा॑त॒ मेक॑ मिवे॒ वैक॒म् तनू॒नपा॑त॒म् तनू॒नपा॑त॒ मेक॑ मिव । \newline
38. तनू॒नपा॑त॒मिति॒ तनू᳚ - नपा॑तम् । \newline
39. एक॑ मिवे॒ वैक॒ मेक॑ मिव मिथु॒नम् मि॑थु॒न मि॒वैक॒ मेक॑ मिव मिथु॒नम् । \newline
40. इ॒व॒ मि॒थु॒नम् मि॑थु॒न मि॑वे व मिथु॒नम् तत् तन् मि॑थु॒न मि॑वे व मिथु॒नम् तत् । \newline
41. मि॒थु॒नम् तत् तन् मि॑थु॒नम् मि॑थु॒नम् तदि॒ड इ॒ड स्तन् मि॑थु॒नम् मि॑थु॒नम् तदि॒डः । \newline
42. तदि॒ड इ॒ड स्तत् तदि॒डो ब॒ह्वीर् ब॒ह्वी रि॒ड स्तत् तदि॒डो ब॒ह्वीः । \newline
43. इ॒डो ब॒ह्वीर् ब॒ह्वीरि॒ड इ॒डो ब॒ह्वीरि॑वे व ब॒ह्वीरि॒ड इ॒डो ब॒ह्वीरि॑व । \newline
44. ब॒ह्वीरि॑वे व ब॒ह्वीर् ब॒ह्वीरि॑व यजति यजतीव ब॒ह्वीर् ब॒ह्वीरि॑व यजति । \newline
45. इ॒व॒ य॒ज॒ति॒ य॒ज॒ती॒वे॒ व॒ य॒ज॒ति॒ ब॒र्॒.हिर् ब॒र्॒.हिर् य॑जतीवे व यजति ब॒र्॒.हिः । \newline
46. य॒ज॒ति॒ ब॒र्॒.हिर् ब॒र्॒.हिर् य॑जति यजति ब॒र्॒.हिरेक॒ मेक॑म् ब॒र्॒.हिर् य॑जति यजति ब॒र्॒.हिरेक᳚म् । \newline
47. ब॒र्॒.हिरेक॒ मेक॑म् ब॒र्॒.हिर् ब॒र्॒.हिरेक॑ मिवे॒ वैक॑म् ब॒र्॒.हिर् ब॒र्॒.हिरेक॑ मिव । \newline
48. एक॑ मिवे॒ वैक॒ मेक॑ मिव मिथु॒नम् मि॑थु॒न मि॒वैक॒ मेक॑ मिव मिथु॒नम् । \newline
49. इ॒व॒ मि॒थु॒नम् मि॑थु॒न मि॑वे व मिथु॒नम् तत् तन् मि॑थु॒न मि॑वे व मिथु॒नम् तत् । \newline
50. मि॒थु॒नम् तत् तन् मि॑थु॒नम् मि॑थु॒नम् तदे॒त दे॒तत् तन् मि॑थु॒नम् मि॑थु॒नम् तदे॒तत् । \newline
51. तदे॒त दे॒तत् तत् तदे॒तद् वै वा ए॒तत् तत् तदे॒तद् वै । \newline
52. ए॒तद् वै वा ए॒त दे॒तद् वै प्र॑या॒जाना᳚म् प्रया॒जानां॒ ॅवा ए॒त दे॒तद् वै प्र॑या॒जाना᳚म् । \newline
53. वै प्र॑या॒जाना᳚म् प्रया॒जानां॒ ॅवै वै प्र॑या॒जाना᳚म् मिथु॒नम् मि॑थु॒नम् प्र॑या॒जानां॒ ॅवै वै प्र॑या॒जाना᳚म् मिथु॒नम् । \newline
54. प्र॒या॒जाना᳚म् मिथु॒नम् मि॑थु॒नम् प्र॑या॒जाना᳚म् प्रया॒जाना᳚म् मिथु॒नं ॅयो यो मि॑थु॒नम् प्र॑या॒जाना᳚म् प्रया॒जाना᳚म् मिथु॒नं ॅयः । \newline
55. प्र॒या॒जाना॒मिति॑ प्र - या॒जाना᳚म् । \newline
56. मि॒थु॒नं ॅयो यो मि॑थु॒नम् मि॑थु॒नं ॅय ए॒व मे॒वं ॅयो मि॑थु॒नम् मि॑थु॒नं ॅय ए॒वम् । \newline
57. य ए॒व मे॒वं ॅयो य ए॒वं ॅवेद॒ वेदै॒वं ॅयो य ए॒वं ॅवेद॑ । \newline
58. ए॒वं ॅवेद॒ वेदै॒व मे॒वं ॅवेद॒ प्र प्र वेदै॒व मे॒वं ॅवेद॒ प्र । \newline
59. वेद॒ प्र प्र वेद॒ वेद॒ प्र प्र॒जया᳚ प्र॒जया॒ प्र वेद॒ वेद॒ प्र प्र॒जया᳚ । \newline
60. प्र प्र॒जया᳚ प्र॒जया॒ प्र प्र प्र॒जया॑ प॒शुभिः॑ प॒शुभिः॑ प्र॒जया॒ प्र प्र प्र॒जया॑ प॒शुभिः॑ । \newline
\pagebreak
\markright{ TS 2.6.1.5  \hfill https://www.vedavms.in \hfill}
\addcontentsline{toc}{section}{ TS 2.6.1.5 }
\section*{ TS 2.6.1.5 }

\textbf{TS 2.6.1.5 } \newline
\textbf{Samhita Paata} \newline

प्र॒जया॑ प॒शुभि॑ र्मिथु॒नै र्जा॑यते दे॒वानां॒ ॅवा अनि॑ष्टा दे॒वता॒ आस॒न्नथासु॑रा य॒ज्ञ्म॑जिघाꣳ स॒न् ते दे॒वा गा॑य॒त्रीं ॅव्यौ॑ह॒न् पञ्चा॒क्षरा॑णि प्रा॒चीना॑नि॒ त्रीणि॑ प्रती॒चीना॑नि॒ ततो॒ वर्म॑ य॒ज्ञायाभ॑व॒द्वर्म॒ यज॑मानाय॒ यत् प्र॑याजानूया॒जा इ॒ज्यन्ते॒ वर्मै॒व तद्य॒ज्ञाय॑ क्रियते॒ वर्म॒ यज॑मानाय॒ भ्रातृ॑व्याभिभूत्यै॒ तस्मा॒द्-वरू॑थं पु॒रस्ता॒द्-वर्.षी॑यः प॒श्चाद्ध्रसी॑यो दे॒वा वै पु॒रा रक्षो᳚भ्य॒ - [  ] \newline

\textbf{Pada Paata} \newline

प्र॒जयेति॑ प्र - जया᳚ । प॒शुभि॒रिति॑ प॒शु - भिः॒ । मि॒थु॒नैः । जा॒य॒ते॒ । दे॒वाना᳚म् । वै । अनि॑ष्टाः । दे॒वताः᳚ । आसन्न्॑ । अथ॑ । असु॑राः । य॒ज्ञ्म् । अ॒जि॒घाꣳ॒॒स॒न्न् । ते । दे॒वाः । गा॒य॒त्रीम् । वीति॑ । औ॒ह॒न्न् । पञ्च॑ । अ॒क्षरा॑णि । प्रा॒चीना॑नि । त्रीणि॑ । प्र॒ती॒चीना॑नि । ततः॑ । वर्म॑ । य॒ज्ञाय॑ । अभ॑वत् । वर्म॑ । यज॑मानाय । यत् । प्र॒या॒जा॒नू॒या॒जा इति॑ प्रयाज - अ॒नू॒या॒जाः । इ॒ज्यन्ते᳚ । वर्म॑ । ए॒व । तत् । य॒ज्ञाय॑ । क्रि॒य॒ते॒ । वर्म॑ । यज॑मानाय । भ्रातृ॑व्याभिभूत्या॒ इति॒ भ्रातृ॑व्य - अ॒भि॒भू॒त्यै॒ । तस्मा᳚त् । वरू॑थम् । पु॒रस्ता᳚त् । वर्.षी॑यः । प॒श्चात् । ह्रसी॑यः । दे॒वाः । वै । पु॒रा । रक्षो᳚भ्य॒ इति॒ रक्षः॑-भ्यः॒ ।  \newline


\textbf{Krama Paata} \newline

प्र॒जया॑ प॒शुभिः॑ । प्र॒जयेति॑ प्र - जया᳚ । प॒शुभि॑र् मिथु॒नैः । प॒शुभि॒रिति॑ प॒शु - भिः॒ । मि॒थु॒नैर् जा॑यते । जा॒य॒ते॒ दे॒वाना᳚म् । दे॒वानां॒ ॅवै । वा अनि॑ष्टाः । अनि॑ष्टा दे॒वताः᳚ । दे॒वता॒ आसन्न्॑ । आस॒न्नथ॑ । अथासु॑राः । असु॑रा य॒ज्ञ्म् । य॒ज्ञ्म॑जिघाꣳसन्न् । अ॒जि॒घाꣳ॒॒स॒न् ते । ते दे॒वाः । दे॒वा गा॑य॒त्रीम् । गा॒य॒त्रीं ॅवि । व्यौ॑हन्न् । औ॒ह॒न् पञ्च॑ । पञ्चा॒क्षरा॑णि । अ॒क्षरा॑णि प्रा॒चीना॑नि । प्रा॒चीना॑नि॒ त्रीणि॑ । त्रीणि॑ प्रती॒चीना॑नि । प्र॒ती॒चीना॑नि॒ ततः॑ । ततो॒ वर्म॑ । वर्म॑ य॒ज्ञाय॑ । य॒ज्ञायाभ॑वत् । अभ॑व॒द् वर्म॑ । वर्म॒ यज॑मानाय । यज॑मानाय॒ यत् । यत् प्र॑याजानूया॒जाः । प्र॒या॒जा॒नू॒या॒जा इ॒ज्यन्ते᳚ । प्र॒या॒जा॒नू॒या॒जा इति॑ प्रयाज - अ॒नू॒या॒जाः । इ॒ज्यन्ते॒ वर्म॑ । वर्मै॒व । ए॒व तत् । तद् य॒ज्ञाय॑ । य॒ज्ञाय॑ क्रियते । क्रि॒य॒ते॒ वर्म॑ । वर्म॒ यज॑मानाय । यज॑मानाय॒ भ्रातृ॑व्याभिभूत्यै । भ्रातृ॑व्याभिभूत्यै॒ तस्मा᳚त् । भ्रातृ॑व्याभिभूत्या॒ इति॒ भ्रातृ॑व्य - अ॒भि॒भू॒त्यै॒ । तस्मा॒द् वरू॑थम् । वरू॑थम् पु॒रस्ता᳚त् । पु॒रस्ता॒द् वर्.षी॑यः । वर्.षी॑यः प॒श्चात् । प॒श्चाद्ध्रसी॑यः । ह्रसी॑यो दे॒वाः । दे॒वा वै । वै पु॒रा । पु॒रा रक्षो᳚भ्यः । रक्षो᳚भ्य॒ इति॑ । रक्षो᳚भ्य॒ इति॒ रक्षः॑ - भ्यः॒ \newline

\textbf{Jatai Paata} \newline

1. प्र॒जया॑ प॒शुभिः॑ प॒शुभिः॑ प्र॒जया᳚ प्र॒जया॑ प॒शुभिः॑ । \newline
2. प्र॒जयेति॑ प्र - जया᳚ । \newline
3. प॒शुभि॑र् मिथु॒नैर् मि॑थु॒नैः प॒शुभिः॑ प॒शुभि॑र् मिथु॒नैः । \newline
4. प॒शुभि॒रिति॑ प॒शु - भिः॒ । \newline
5. मि॒थु॒नैर् जा॑यते जायते मिथु॒नैर् मि॑थु॒नैर् जा॑यते । \newline
6. जा॒य॒ते॒ दे॒वाना᳚म् दे॒वाना᳚म् जायते जायते दे॒वाना᳚म् । \newline
7. दे॒वानां॒ ॅवै वै दे॒वाना᳚म् दे॒वानां॒ ॅवै । \newline
8. वा अनि॑ष्टा॒ अनि॑ष्टा॒ वै वा अनि॑ष्टाः । \newline
9. अनि॑ष्टा दे॒वता॑ दे॒वता॒ अनि॑ष्टा॒ अनि॑ष्टा दे॒वताः᳚ । \newline
10. दे॒वता॒ आस॒न् नास॑न् दे॒वता॑ दे॒वता॒ आसन्न्॑ । \newline
11. आस॒न् नथाथास॒न् नास॒न् नथ॑ । \newline
12. अथासु॑रा॒ असु॑रा॒ अथाथासु॑राः । \newline
13. असु॑रा य॒ज्ञ्ं ॅय॒ज्ञ् मसु॑रा॒ असु॑रा य॒ज्ञ्म् । \newline
14. य॒ज्ञ् म॑जिघाꣳसन् नजिघाꣳसन्. य॒ज्ञ्ं ॅय॒ज्ञ् म॑जिघाꣳसन्न् । \newline
15. अ॒जि॒घाꣳ॒॒स॒न् ते ते॑ ऽजिघाꣳसन् नजिघाꣳस॒न् ते । \newline
16. ते दे॒वा दे॒वा स्ते ते दे॒वाः । \newline
17. दे॒वा गा॑य॒त्रीम् गा॑य॒त्रीम् दे॒वा दे॒वा गा॑य॒त्रीम् । \newline
18. गा॒य॒त्रीं ॅवि वि गा॑य॒त्रीम् गा॑य॒त्रीं ॅवि । \newline
19. व्यौ॑हन् नौह॒न्॒. वि व्यौ॑हन्न् । \newline
20. औ॒ह॒न् पञ्च॒ पञ्चौ॑हन् नौह॒न् पञ्च॑ । \newline
21. पञ्चा॒क्षरा᳚ ण्य॒क्षरा॑णि॒ पञ्च॒ पञ्चा॒क्षरा॑णि । \newline
22. अ॒क्षरा॑णि प्रा॒चीना॑नि प्रा॒चीना᳚ न्य॒क्षरा᳚ ण्य॒क्षरा॑णि प्रा॒चीना॑नि । \newline
23. प्रा॒चीना॑नि॒ त्रीणि॒ त्रीणि॑ प्रा॒चीना॑नि प्रा॒चीना॑नि॒ त्रीणि॑ । \newline
24. त्रीणि॑ प्रती॒चीना॑नि प्रती॒चीना॑नि॒ त्रीणि॒ त्रीणि॑ प्रती॒चीना॑नि । \newline
25. प्र॒ती॒चीना॑नि॒ तत॒ स्ततः॑ प्रती॒चीना॑नि प्रती॒चीना॑नि॒ ततः॑ । \newline
26. ततो॒ वर्म॒ वर्म॒ तत॒ स्ततो॒ वर्म॑ । \newline
27. वर्म॑ य॒ज्ञाय॑ य॒ज्ञाय॒ वर्म॒ वर्म॑ य॒ज्ञाय॑ । \newline
28. य॒ज्ञाया भ॑व॒ दभ॑वद् य॒ज्ञाय॑ य॒ज्ञाया भ॑वत् । \newline
29. अभ॑व॒द् वर्म॒ वर्मा भ॑व॒ दभ॑व॒द् वर्म॑ । \newline
30. वर्म॒ यज॑मानाय॒ यज॑मानाय॒ वर्म॒ वर्म॒ यज॑मानाय । \newline
31. यज॑मानाय॒ यद् यद् यज॑मानाय॒ यज॑मानाय॒ यत् । \newline
32. यत् प्र॑याजानूया॒जाः प्र॑याजानूया॒जा यद् यत् प्र॑याजानूया॒जाः । \newline
33. प्र॒या॒जा॒नू॒या॒जा इ॒ज्यन्त॑ इ॒ज्यन्ते᳚ प्रयाजानूया॒जाः प्र॑याजानूया॒जा इ॒ज्यन्ते᳚ । \newline
34. प्र॒या॒जा॒नू॒या॒जा इति॑ प्रयाज - अ॒नू॒या॒जाः । \newline
35. इ॒ज्यन्ते॒ वर्म॒ वर्मे॒ ज्यन्त॑ इ॒ज्यन्ते॒ वर्म॑ । \newline
36. वर्मै॒वैव वर्म॒ वर्मै॒व । \newline
37. ए॒व तत् तदे॒वैव तत् । \newline
38. तद् य॒ज्ञाय॑ य॒ज्ञाय॒ तत् तद् य॒ज्ञाय॑ । \newline
39. य॒ज्ञाय॑ क्रियते क्रियते य॒ज्ञाय॑ य॒ज्ञाय॑ क्रियते । \newline
40. क्रि॒य॒ते॒ वर्म॒ वर्म॑ क्रियते क्रियते॒ वर्म॑ । \newline
41. वर्म॒ यज॑मानाय॒ यज॑मानाय॒ वर्म॒ वर्म॒ यज॑मानाय । \newline
42. यज॑मानाय॒ भ्रातृ॑व्याभिभूत्यै॒ भ्रातृ॑व्याभिभूत्यै॒ यज॑मानाय॒ यज॑मानाय॒ भ्रातृ॑व्याभिभूत्यै । \newline
43. भ्रातृ॑व्याभिभूत्यै॒ तस्मा॒त् तस्मा॒द् भ्रातृ॑व्याभिभूत्यै॒ भ्रातृ॑व्याभिभूत्यै॒ तस्मा᳚त् । \newline
44. भ्रातृ॑व्याभिभूत्या॒ इति॒ भ्रातृ॑व्य - अ॒भि॒भू॒त्यै॒ । \newline
45. तस्मा॒द् वरू॑थं॒ ॅवरू॑थ॒म् तस्मा॒त् तस्मा॒द् वरू॑थम् । \newline
46. वरू॑थम् पु॒रस्ता᳚त् पु॒रस्ता॒द् वरू॑थं॒ ॅवरू॑थम् पु॒रस्ता᳚त् । \newline
47. पु॒रस्ता॒द् वर्.षी॑यो॒ वर्.षी॑यः पु॒रस्ता᳚त् पु॒रस्ता॒द् वर्.षी॑यः । \newline
48. वर्.षी॑यः प॒श्चात् प॒श्चाद् वर्.षी॑यो॒ वर्.षी॑यः प॒श्चात् । \newline
49. प॒श्चा द्ध्रसी॑यो॒ ह्रसी॑यः प॒श्चात् प॒श्चा द्ध्रसी॑यः । \newline
50. ह्रसी॑यो दे॒वा दे॒वा ह्रसी॑यो॒ ह्रसी॑यो दे॒वाः । \newline
51. दे॒वा वै वै दे॒वा दे॒वा वै । \newline
52. वै पु॒रा पु॒रा वै वै पु॒रा । \newline
53. पु॒रा रक्षो᳚भ्यो॒ रक्षो᳚भ्यः पु॒रा पु॒रा रक्षो᳚भ्यः । \newline
54. रक्षो᳚भ्य॒ इतीति॒ रक्षो᳚भ्यो॒ रक्षो᳚भ्य॒ इति॑ । \newline
55. रक्षो᳚भ्य॒ इति॒ रक्षः॑ - भ्यः॒ । \newline

\textbf{Ghana Paata } \newline

1. प्र॒जया॑ प॒शुभिः॑ प॒शुभिः॑ प्र॒जया᳚ प्र॒जया॑ प॒शुभि॑र् मिथु॒नैर् मि॑थु॒नैः प॒शुभिः॑ प्र॒जया᳚ प्र॒जया॑ प॒शुभि॑र् मिथु॒नैः । \newline
2. प्र॒जयेति॑ प्र - जया᳚ । \newline
3. प॒शुभि॑र् मिथु॒नैर् मि॑थु॒नैः प॒शुभिः॑ प॒शुभि॑र् मिथु॒नैर् जा॑यते जायते मिथु॒नैः प॒शुभिः॑ प॒शुभि॑र् मिथु॒नैर् जा॑यते । \newline
4. प॒शुभि॒रिति॑ प॒शु - भिः॒ । \newline
5. मि॒थु॒नैर् जा॑यते जायते मिथु॒नैर् मि॑थु॒नैर् जा॑यते दे॒वाना᳚म् दे॒वाना᳚म् जायते मिथु॒नैर् मि॑थु॒नैर् जा॑यते दे॒वाना᳚म् । \newline
6. जा॒य॒ते॒ दे॒वाना᳚म् दे॒वाना᳚म् जायते जायते दे॒वानां॒ ॅवै वै दे॒वाना᳚म् जायते जायते दे॒वानां॒ ॅवै । \newline
7. दे॒वानां॒ ॅवै वै दे॒वाना᳚म् दे॒वानां॒ ॅवा अनि॑ष्टा॒ अनि॑ष्टा॒ वै दे॒वाना᳚म् दे॒वानां॒ ॅवा अनि॑ष्टाः । \newline
8. वा अनि॑ष्टा॒ अनि॑ष्टा॒ वै वा अनि॑ष्टा दे॒वता॑ दे॒वता॒ अनि॑ष्टा॒ वै वा अनि॑ष्टा दे॒वताः᳚ । \newline
9. अनि॑ष्टा दे॒वता॑ दे॒वता॒ अनि॑ष्टा॒ अनि॑ष्टा दे॒वता॒ आस॒न् नास॑न् दे॒वता॒ अनि॑ष्टा॒ अनि॑ष्टा दे॒वता॒ आसन्न्॑ । \newline
10. दे॒वता॒ आस॒न् नास॑न् दे॒वता॑ दे॒वता॒ आस॒न् नथाथास॑न् दे॒वता॑ दे॒वता॒ आस॒न् नथ॑ । \newline
11. आस॒न् नथाथास॒न् नास॒न् नथासु॑रा॒ असु॑रा॒ अथास॒न् नास॒न् नथासु॑राः । \newline
12. अथासु॑रा॒ असु॑रा॒ अथाथा सु॑रा य॒ज्ञ्ं ॅय॒ज्ञ् मसु॑रा॒ अथाथा सु॑रा य॒ज्ञ्म् । \newline
13. असु॑रा य॒ज्ञ्ं ॅय॒ज्ञ् मसु॑रा॒ असु॑रा य॒ज्ञ् म॑जिघाꣳसन् नजिघाꣳसन्. य॒ज्ञ् मसु॑रा॒ असु॑रा य॒ज्ञ् म॑जिघाꣳसन्न् । \newline
14. य॒ज्ञ् म॑जिघाꣳसन् नजिघाꣳसन्. य॒ज्ञ्ं ॅय॒ज्ञ् म॑जिघाꣳस॒न् ते ते॑ ऽजिघाꣳसन्. य॒ज्ञ्ं ॅय॒ज्ञ् म॑जिघाꣳस॒न् ते । \newline
15. अ॒जि॒घाꣳ॒॒स॒न् ते ते॑ ऽजिघाꣳसन् नजिघाꣳस॒न् ते दे॒वा दे॒वा स्ते॑ ऽजिघाꣳसन् नजिघाꣳस॒न् ते दे॒वाः । \newline
16. ते दे॒वा दे॒वा स्ते ते दे॒वा गा॑य॒त्रीम् गा॑य॒त्रीम् दे॒वा स्ते ते दे॒वा गा॑य॒त्रीम् । \newline
17. दे॒वा गा॑य॒त्रीम् गा॑य॒त्रीम् दे॒वा दे॒वा गा॑य॒त्रीं ॅवि वि गा॑य॒त्रीम् दे॒वा दे॒वा गा॑य॒त्रीं ॅवि । \newline
18. गा॒य॒त्रीं ॅवि वि गा॑य॒त्रीम् गा॑य॒त्रीं ॅव्यौ॑हन् नौह॒न्॒. वि गा॑य॒त्रीम् गा॑य॒त्रीं ॅव्यौ॑हन्न् । \newline
19. व्यौ॑हन् नौह॒न्॒. वि व्यौ॑ह॒न् पञ्च॒ पञ्चौ॑ह॒न्॒. वि व्यौ॑ह॒न् पञ्च॑ । \newline
20. औ॒ह॒न् पञ्च॒ पञ्चौ॑हन् नौह॒न् पञ्चा॒क्षरा᳚ ण्य॒क्षरा॑णि॒ पञ्चौ॑हन् नौह॒न् पञ्चा॒क्षरा॑णि । \newline
21. पञ्चा॒क्षरा᳚ ण्य॒क्षरा॑णि॒ पञ्च॒ पञ्चा॒क्षरा॑णि प्रा॒चीना॑नि प्रा॒चीना᳚ न्य॒क्षरा॑णि॒ पञ्च॒ पञ्चा॒क्षरा॑णि प्रा॒चीना॑नि । \newline
22. अ॒क्षरा॑णि प्रा॒चीना॑नि प्रा॒चीना᳚ न्य॒क्षरा᳚ ण्य॒क्षरा॑णि प्रा॒चीना॑नि॒ त्रीणि॒ त्रीणि॑ प्रा॒चीना᳚ न्य॒क्षरा᳚ ण्य॒क्षरा॑णि प्रा॒चीना॑नि॒ त्रीणि॑ । \newline
23. प्रा॒चीना॑नि॒ त्रीणि॒ त्रीणि॑ प्रा॒चीना॑नि प्रा॒चीना॑नि॒ त्रीणि॑ प्रती॒चीना॑नि प्रती॒चीना॑नि॒ त्रीणि॑ प्रा॒चीना॑नि प्रा॒चीना॑नि॒ त्रीणि॑ प्रती॒चीना॑नि । \newline
24. त्रीणि॑ प्रती॒चीना॑नि प्रती॒चीना॑नि॒ त्रीणि॒ त्रीणि॑ प्रती॒चीना॑नि॒ तत॒ स्ततः॑ प्रती॒चीना॑नि॒ त्रीणि॒ त्रीणि॑ प्रती॒चीना॑नि॒ ततः॑ । \newline
25. प्र॒ती॒चीना॑नि॒ तत॒ स्ततः॑ प्रती॒चीना॑नि प्रती॒चीना॑नि॒ ततो॒ वर्म॒ वर्म॒ ततः॑ प्रती॒चीना॑नि प्रती॒चीना॑नि॒ ततो॒ वर्म॑ । \newline
26. ततो॒ वर्म॒ वर्म॒ तत॒ स्ततो॒ वर्म॑ य॒ज्ञाय॑ य॒ज्ञाय॒ वर्म॒ तत॒ स्ततो॒ वर्म॑ य॒ज्ञाय॑ । \newline
27. वर्म॑ य॒ज्ञाय॑ य॒ज्ञाय॒ वर्म॒ वर्म॑ य॒ज्ञाया भ॑व॒ दभ॑वद् य॒ज्ञाय॒ वर्म॒ वर्म॑ य॒ज्ञाया भ॑वत् । \newline
28. य॒ज्ञाया भ॑व॒ दभ॑वद् य॒ज्ञाय॑ य॒ज्ञाया भ॑व॒द् वर्म॒ वर्मा भ॑वद् य॒ज्ञाय॑ य॒ज्ञाया भ॑व॒द् वर्म॑ । \newline
29. अभ॑व॒द् वर्म॒ वर्मा भ॑व॒ दभ॑व॒द् वर्म॒ यज॑मानाय॒ यज॑मानाय॒ वर्मा भ॑व॒ दभ॑व॒द् वर्म॒ यज॑मानाय । \newline
30. वर्म॒ यज॑मानाय॒ यज॑मानाय॒ वर्म॒ वर्म॒ यज॑मानाय॒ यद् यद् यज॑मानाय॒ वर्म॒ वर्म॒ यज॑मानाय॒ यत् । \newline
31. यज॑मानाय॒ यद् यद् यज॑मानाय॒ यज॑मानाय॒ यत् प्र॑याजानूया॒जाः प्र॑याजानूया॒जा यद् यज॑मानाय॒ यज॑मानाय॒ यत् प्र॑याजानूया॒जाः । \newline
32. यत् प्र॑याजानूया॒जाः प्र॑याजानूया॒जा यद् यत् प्र॑याजानूया॒जा इ॒ज्यन्त॑ इ॒ज्यन्ते᳚ प्रयाजानूया॒जा यद् यत् प्र॑याजानूया॒जा इ॒ज्यन्ते᳚ । \newline
33. प्र॒या॒जा॒नू॒या॒जा इ॒ज्यन्त॑ इ॒ज्यन्ते᳚ प्रयाजानूया॒जाः प्र॑याजानूया॒जा इ॒ज्यन्ते॒ वर्म॒ वर्मे॒ ज्यन्ते᳚ प्रयाजानूया॒जाः प्र॑याजानूया॒जा इ॒ज्यन्ते॒ वर्म॑ । \newline
34. प्र॒या॒जा॒नू॒या॒जा इति॑ प्रयाज - अ॒नू॒या॒जाः । \newline
35. इ॒ज्यन्ते॒ वर्म॒ वर्मे॒ ज्यन्त॑ इ॒ज्यन्ते॒ वर्मै॒वैव वर्मे॒ ज्यन्त॑ इ॒ज्यन्ते॒ वर्मै॒व । \newline
36. वर्मै॒वैव वर्म॒ वर्मै॒व तत् तदे॒व वर्म॒ वर्मै॒व तत् । \newline
37. ए॒व तत् तदे॒वैव तद् य॒ज्ञाय॑ य॒ज्ञाय॒ तदे॒वैव तद् य॒ज्ञाय॑ । \newline
38. तद् य॒ज्ञाय॑ य॒ज्ञाय॒ तत् तद् य॒ज्ञाय॑ क्रियते क्रियते य॒ज्ञाय॒ तत् तद् य॒ज्ञाय॑ क्रियते । \newline
39. य॒ज्ञाय॑ क्रियते क्रियते य॒ज्ञाय॑ य॒ज्ञाय॑ क्रियते॒ वर्म॒ वर्म॑ क्रियते य॒ज्ञाय॑ य॒ज्ञाय॑ क्रियते॒ वर्म॑ । \newline
40. क्रि॒य॒ते॒ वर्म॒ वर्म॑ क्रियते क्रियते॒ वर्म॒ यज॑मानाय॒ यज॑मानाय॒ वर्म॑ क्रियते क्रियते॒ वर्म॒ यज॑मानाय । \newline
41. वर्म॒ यज॑मानाय॒ यज॑मानाय॒ वर्म॒ वर्म॒ यज॑मानाय॒ भ्रातृ॑व्याभिभूत्यै॒ भ्रातृ॑व्याभिभूत्यै॒ यज॑मानाय॒ वर्म॒ वर्म॒ यज॑मानाय॒ भ्रातृ॑व्याभिभूत्यै । \newline
42. यज॑मानाय॒ भ्रातृ॑व्याभिभूत्यै॒ भ्रातृ॑व्याभिभूत्यै॒ यज॑मानाय॒ यज॑मानाय॒ भ्रातृ॑व्याभिभूत्यै॒ तस्मा॒त् तस्मा॒द् भ्रातृ॑व्याभिभूत्यै॒ यज॑मानाय॒ यज॑मानाय॒ भ्रातृ॑व्याभिभूत्यै॒ तस्मा᳚त् । \newline
43. भ्रातृ॑व्याभिभूत्यै॒ तस्मा॒त् तस्मा॒द् भ्रातृ॑व्याभिभूत्यै॒ भ्रातृ॑व्याभिभूत्यै॒ तस्मा॒द् वरू॑थं॒ ॅवरू॑थ॒म् तस्मा॒द् भ्रातृ॑व्याभिभूत्यै॒ भ्रातृ॑व्याभिभूत्यै॒ तस्मा॒द् वरू॑थम् । \newline
44. भ्रातृ॑व्याभिभूत्या॒ इति॒ भ्रातृ॑व्य - अ॒भि॒भू॒त्यै॒ । \newline
45. तस्मा॒द् वरू॑थं॒ ॅवरू॑थ॒म् तस्मा॒त् तस्मा॒द् वरू॑थम् पु॒रस्ता᳚त् पु॒रस्ता॒द् वरू॑थ॒म् तस्मा॒त् तस्मा॒द् वरू॑थम् पु॒रस्ता᳚त् । \newline
46. वरू॑थम् पु॒रस्ता᳚त् पु॒रस्ता॒द् वरू॑थं॒ ॅवरू॑थम् पु॒रस्ता॒द् वर्.षी॑यो॒ वर्.षी॑यः पु॒रस्ता॒द् वरू॑थं॒ ॅवरू॑थम् पु॒रस्ता॒द् वर्.षी॑यः । \newline
47. पु॒रस्ता॒द् वर्.षी॑यो॒ वर्.षी॑यः पु॒रस्ता᳚त् पु॒रस्ता॒द् वर्.षी॑यः प॒श्चात् प॒श्चाद् वर्.षी॑यः पु॒रस्ता᳚त् पु॒रस्ता॒द् वर्.षी॑यः प॒श्चात् । \newline
48. वर्.षी॑यः प॒श्चात् प॒श्चाद् वर्.षी॑यो॒ वर्.षी॑यः प॒श्चा द्ध्रसी॑यो॒ ह्रसी॑यः प॒श्चाद् वर्.षी॑यो॒ वर्.षी॑यः प॒श्चा द्ध्रसी॑यः । \newline
49. प॒श्चा द्ध्रसी॑यो॒ ह्रसी॑यः प॒श्चात् प॒श्चा द्ध्रसी॑यो दे॒वा दे॒वा ह्रसी॑यः प॒श्चात् प॒श्चा द्ध्रसी॑यो दे॒वाः । \newline
50. ह्रसी॑यो दे॒वा दे॒वा ह्रसी॑यो॒ ह्रसी॑यो दे॒वा वै वै दे॒वा ह्रसी॑यो॒ ह्रसी॑यो दे॒वा वै । \newline
51. दे॒वा वै वै दे॒वा दे॒वा वै पु॒रा पु॒रा वै दे॒वा दे॒वा वै पु॒रा । \newline
52. वै पु॒रा पु॒रा वै वै पु॒रा रक्षो᳚भ्यो॒ रक्षो᳚भ्यः पु॒रा वै वै पु॒रा रक्षो᳚भ्यः । \newline
53. पु॒रा रक्षो᳚भ्यो॒ रक्षो᳚भ्यः पु॒रा पु॒रा रक्षो᳚भ्य॒ इतीति॒ रक्षो᳚भ्यः पु॒रा पु॒रा रक्षो᳚भ्य॒ इति॑ । \newline
54. रक्षो᳚भ्य॒ इतीति॒ रक्षो᳚भ्यो॒ रक्षो᳚भ्य॒ इति॑ स्वाहाका॒रेण॑ स्वाहाका॒रेणे ति॒ रक्षो᳚भ्यो॒ रक्षो᳚भ्य॒ इति॑ स्वाहाका॒रेण॑ । \newline
55. रक्षो᳚भ्य॒ इति॒ रक्षः॑ - भ्यः॒ । \newline
\pagebreak
\markright{ TS 2.6.1.6  \hfill https://www.vedavms.in \hfill}
\addcontentsline{toc}{section}{ TS 2.6.1.6 }
\section*{ TS 2.6.1.6 }

\textbf{TS 2.6.1.6 } \newline
\textbf{Samhita Paata} \newline

इति॑ स्वाहाका॒रेण॑ प्रया॒जेषु॑ य॒ज्ञ्ꣳ सꣳ॒॒स्थाप्य॑मपश्य॒न् तꣳ स्वा॑हाका॒रेण॑ प्रया॒जेषु॒ सम॑स्थापय॒न् वि वा ए॒तद्-य॒ज्ञ्ं छि॑न्दन्ति॒ यथ् स्वा॑हाका॒रेण॑ प्रया॒जेषु॑ सꣳस्था॒पय॑न्ति प्रया॒जानि॒ष्ट्वा ह॒वीꣳष्य॒भि घा॑रयति य॒ज्ञ्स्य॒ संत॑त्या॒ अथो॑ ह॒विरे॒वाक॒रथो॑ यथापू॒र्वमुपै॑ति पि॒ता वै प्र॑या॒जाः प्र॒जाऽनू॑या॒जा यत् प्र॑या॒जानि॒ष्ट्वा ह॒वीꣳष्य॑भिघा॒रय॑ति पि॒तैव तत् पु॒त्रेण॒ साधा॑रणं - [  ] \newline

\textbf{Pada Paata} \newline

इति॑ । स्वा॒हा॒का॒रेणेति॑ स्वाहा - का॒रेण॑ । प्र॒या॒जेष्विति॑ प्र - या॒जेषु॑ । य॒ज्ञ्म् । सꣳ॒॒स्थाप्य॒मिति॑ सं - स्थाप्य᳚म् । अ॒प॒श्य॒न्न् । तम् । स्वा॒हा॒का॒रेणेति॑ स्वाहा - का॒रेण॑ । प्र॒या॒जेष्विति॑ प्र - या॒जेषु॑ । समिति॑ । अ॒स्था॒प॒य॒न्न् । वीति॑ । वै । ए॒तत् । य॒ज्ञ्म् । छि॒न्द॒न्ति॒ । यत् । स्वा॒हा॒का॒रेणेति॑ स्वाहा - का॒रेण॑ । प्र॒या॒जेष्विति॑ प्र- या॒जेषु॑ । सꣳ॒॒स्था॒पय॒न्तीति॑ सं - स्था॒पय॑न्ति । प्र॒या॒जानिति॑ प्र - या॒जान् । इ॒ष्ट्वा । ह॒वीꣳषि॑ । अ॒भीति॑ । घा॒र॒य॒ति॒ । य॒ज्ञ्स्य॑ । संत॑त्या॒ इति॒ सं - त॒त्यै॒ । अथो॒ इति॑ । ह॒विः । ए॒व । अ॒कः॒ । अथो॒ इति॑ । य॒था॒पू॒र्वमिति॑ यथा - पू॒र्वम् । उपेति॑ । ए॒ति॒ । पि॒ता । वै । प्र॒या॒जा इति॑ प्र - या॒जाः । प्र॒जेति॑ प्र - जा । अ॒नू॒या॒जा इत्य॑नु - या॒जाः । यत् । प्र॒या॒जानिति॑ प्र - या॒जान् । इ॒ष्ट्वा । ह॒वीꣳषि॑ । अ॒भि॒घा॒रय॒तीत्य॑भि - घा॒रय॑ति । पि॒ता । ए॒व । तत् । पु॒त्रेण॑ । साधा॑रणम् ।  \newline


\textbf{Krama Paata} \newline

इति॑ स्वाहाका॒रेण॑ । स्वा॒हा॒का॒रेण॑ प्रया॒जेषु॑ । स्वा॒हा॒का॒रेणेति॑ स्वाहा - का॒रेण॑ । प्र॒या॒जेषु॑ य॒ज्ञ्म् । प्र॒या॒जेष्विति॑ प्र - या॒जेषु॑ । य॒ज्ञ्ꣳ सꣳ॒॒स्थाप्य᳚म् । सꣳ॒॒स्थाप्य॑मपश्यन्न् । सꣳ॒॒स्थाप्य॒मिति॑ सं - स्थाप्य᳚म् । अ॒प॒श्य॒न् तम् । तꣳ स्वा॑हाका॒रेण॑ । स्वा॒हा॒का॒रेण॑ प्रया॒जेषु॑ । स्वा॒हा॒का॒रेणेति॑ स्वाहा - का॒रेण॑ । प्र॒या॒जेषु॒ सम् । प्र॒या॒जेष्विति॑ प्र - या॒जेषु॑ । सम॑स्थापयन्न् । अ॒स्था॒प॒य॒न् वि । वि वै । वा ए॒तत् । ए॒तद् य॒ज्ञ्म् । य॒ज्ञ्म् छि॑न्दन्ति । छि॒न्द॒न्ति॒ यत् । यथ् स्वा॑हाका॒रेण॑ । स्वा॒हा॒का॒रेण॑ प्रया॒जेषु॑ । स्वा॒हा॒का॒रेणेति॑ स्वाहा - का॒रेण॑ । प्र॒या॒जेषु॑ सꣳस्था॒पय॑न्ति । प्र॒या॒जेष्विति॑ प्र - या॒जेषु॑ । सꣳ॒॒स्था॒पय॑न्ति प्रया॒जान् । सꣳ॒॒स्था॒पय॒न्तीति॑ सं - स्था॒पय॑न्ति । प्र॒या॒जानि॒ष्ट्वा । प्र॒या॒जानिति॑ प्र - या॒जान् । इ॒ष्ट्वा ह॒वीꣳषि॑ । ह॒वीꣳष्य॒भि । अ॒भि घा॑रयति । घा॒र॒य॒ति॒ य॒ज्ञ्स्य॑ । य॒ज्ञ्स्य॒ सन्त॑त्यै । सन्त॑त्या॒ अथो᳚ । सन्त॑त्या॒ इति॒ सं - त॒त्यै॒ । अथो॑ ह॒विः । अथो॒ इत्यथो᳚ । ह॒विरे॒व । ए॒वाकः॑ । अ॒क॒रथो᳚ । अथो॑ यथापू॒र्वम् । अथो॒ इत्यथो᳚ । य॒था॒पू॒र्वमुप॑ । य॒था॒पू॒र्वमिति॑ यथा - पू॒र्वम् । उपै॑ति । ए॒ति॒ पि॒ता । पि॒ता वै । वै प्र॑या॒जाः । प्र॒या॒जाः प्र॒जा । प्र॒या॒जा इति॑ प्र - या॒जाः । प्र॒जा ऽनू॑या॒जाः । प्र॒जेति॑ प्र - जा । अ॒नू॒या॒जा यत् । अ॒नू॒या॒जा इत्य॑नु - या॒जाः । यत् प्र॑या॒जान् । प्र॒या॒जानि॒ष्ट्वा । प्र॒या॒जानिति॑ प्र - या॒जान् । इ॒ष्ट्वा ह॒वीꣳषि॑ । ह॒वीꣳष्य॑भिघा॒रय॑ति । अ॒भि॒घा॒रय॑ति पि॒ता । अ॒भि॒घा॒रय॒तीत्य॑भि - घा॒रय॑ति । पि॒तैव । ए॒व तत् । तत् पु॒त्रेण॑ । पु॒त्रेण॒ साधा॑रणम् ( ) । साधा॑रणम् कुरुते \newline

\textbf{Jatai Paata} \newline

1. इति॑ स्वाहाका॒रेण॑ स्वाहाका॒रेणे तीति॑ स्वाहाका॒रेण॑ । \newline
2. स्वा॒हा॒का॒रेण॑ प्रया॒जेषु॑ प्रया॒जेषु॑ स्वाहाका॒रेण॑ स्वाहाका॒रेण॑ प्रया॒जेषु॑ । \newline
3. स्वा॒हा॒का॒रेणेति॑ स्वाहा - का॒रेण॑ । \newline
4. प्र॒या॒जेषु॑ य॒ज्ञ्ं ॅय॒ज्ञ्म् प्र॑या॒जेषु॑ प्रया॒जेषु॑ य॒ज्ञ्म् । \newline
5. प्र॒या॒जेष्विति॑ प्र - या॒जेषु॑ । \newline
6. य॒ज्ञ्ꣳ सꣳ॒॒स्थाप्यꣳ॑ सꣳ॒॒स्थाप्यं॑ ॅय॒ज्ञ्ं ॅय॒ज्ञ्ꣳ सꣳ॒॒स्थाप्य᳚म् । \newline
7. सꣳ॒॒स्थाप्य॑ मपश्यन् नपश्यन् थ्सꣳ॒॒स्थाप्यꣳ॑ सꣳ॒॒स्थाप्य॑ मपश्यन्न् । \newline
8. सꣳ॒॒स्थाप्य॒मिति॑ सं - स्थाप्य᳚म् । \newline
9. अ॒प॒श्य॒न् तम् त म॑पश्यन् नपश्य॒न् तम् । \newline
10. तꣳ स्वा॑हाका॒रेण॑ स्वाहाका॒रेण॒ तम् तꣳ स्वा॑हाका॒रेण॑ । \newline
11. स्वा॒हा॒का॒रेण॑ प्रया॒जेषु॑ प्रया॒जेषु॑ स्वाहाका॒रेण॑ स्वाहाका॒रेण॑ प्रया॒जेषु॑ । \newline
12. स्वा॒हा॒का॒रेणेति॑ स्वाहा - का॒रेण॑ । \newline
13. प्र॒या॒जेषु॒ सꣳ सम् प्र॑या॒जेषु॑ प्रया॒जेषु॒ सम् । \newline
14. प्र॒या॒जेष्विति॑ प्र - या॒जेषु॑ । \newline
15. स म॑स्थापयन् नस्थापय॒न् थ्सꣳ स म॑स्थापयन्न् । \newline
16. अ॒स्था॒प॒य॒न् वि व्य॑स्थापयन् नस्थापय॒न् वि । \newline
17. वि वै वै वि वि वै । \newline
18. वा ए॒त दे॒तद् वै वा ए॒तत् । \newline
19. ए॒तद् य॒ज्ञ्ं ॅय॒ज्ञ् मे॒त दे॒तद् य॒ज्ञ्म् । \newline
20. य॒ज्ञ्म् छि॑न्दन्ति छिन्दन्ति य॒ज्ञ्ं ॅय॒ज्ञ्म् छि॑न्दन्ति । \newline
21. छि॒न्द॒न्ति॒ यद् यच् छि॑न्दन्ति छिन्दन्ति॒ यत् । \newline
22. यथ् स्वा॑हाका॒रेण॑ स्वाहाका॒रेण॒ यद् यथ् स्वा॑हाका॒रेण॑ । \newline
23. स्वा॒हा॒का॒रेण॑ प्रया॒जेषु॑ प्रया॒जेषु॑ स्वाहाका॒रेण॑ स्वाहाका॒रेण॑ प्रया॒जेषु॑ । \newline
24. स्वा॒हा॒का॒रेणेति॑ स्वाहा - का॒रेण॑ । \newline
25. प्र॒या॒जेषु॑ सꣳस्था॒पय॑न्ति सꣳस्था॒पय॑न्ति प्रया॒जेषु॑ प्रया॒जेषु॑ सꣳस्था॒पय॑न्ति । \newline
26. प्र॒या॒जेष्विति॑ प्र - या॒जेषु॑ । \newline
27. सꣳ॒॒स्था॒पय॑न्ति प्रया॒जान् प्र॑या॒जान् थ्सꣳ॑स्था॒पय॑न्ति सꣳस्था॒पय॑न्ति प्रया॒जान् । \newline
28. सꣳ॒॒स्था॒पय॒न्तीति॑ सं - स्था॒पय॑न्ति । \newline
29. प्र॒या॒जा नि॒ष्ट्वेष्ट्वा प्र॑या॒जान् प्र॑या॒जा नि॒ष्ट्वा । \newline
30. प्र॒या॒जानिति॑ प्र - या॒जान् । \newline
31. इ॒ष्ट्वा ह॒वीꣳषि॑ ह॒वीꣳषी॒ ष्ट्वेष्ट्वा ह॒वीꣳषि॑ । \newline
32. ह॒वीꣳ ष्य॒भ्य॑भि ह॒वीꣳषि॑ ह॒वीꣳ ष्य॒भि । \newline
33. अ॒भि घा॑रयति घारय त्य॒भ्य॑भि घा॑रयति । \newline
34. घा॒र॒य॒ति॒ य॒ज्ञ्स्य॑ य॒ज्ञ्स्य॑ घारयति घारयति य॒ज्ञ्स्य॑ । \newline
35. य॒ज्ञ्स्य॒ सन्त॑त्यै॒ सन्त॑त्यै य॒ज्ञ्स्य॑ य॒ज्ञ्स्य॒ सन्त॑त्यै । \newline
36. सन्त॑त्या॒ अथो॒ अथो॒ सन्त॑त्यै॒ सन्त॑त्या॒ अथो᳚ । \newline
37. सन्त॑त्या॒ इति॒ सं - त॒त्यै॒ । \newline
38. अथो॑ ह॒विर्. ह॒वि रथो॒ अथो॑ ह॒विः । \newline
39. अथो॒ इत्यथो᳚ । \newline
40. ह॒वि रे॒वैव ह॒विर्. ह॒वि रे॒व । \newline
41. ए॒वाक॑ रक रे॒वैवाकः॑ । \newline
42. अ॒क॒ रथो॒ अथो॑ अक रक॒ रथो᳚ । \newline
43. अथो॑ यथापू॒र्वं ॅय॑थापू॒र्व मथो॒ अथो॑ यथापू॒र्वम् । \newline
44. अथो॒ इत्यथो᳚ । \newline
45. य॒था॒पू॒र्व मुपोप॑ यथापू॒र्वं ॅय॑थापू॒र्व मुप॑ । \newline
46. य॒था॒पू॒र्वमिति॑ यथा - पू॒र्वम् । \newline
47. उपै᳚ त्ये॒ त्युपोपै॑ति । \newline
48. ए॒ति॒ पि॒ता पि॒तै त्ये॑ति पि॒ता । \newline
49. पि॒ता वै वै पि॒ता पि॒ता वै । \newline
50. वै प्र॑या॒जाः प्र॑या॒जा वै वै प्र॑या॒जाः । \newline
51. प्र॒या॒जाः प्र॒जा प्र॒जा प्र॑या॒जाः प्र॑या॒जाः प्र॒जा । \newline
52. प्र॒या॒जा इति॑ प्र - या॒जाः । \newline
53. प्र॒जा ऽनू॑या॒जा अ॑नूया॒जाः प्र॒जा प्र॒जा ऽनू॑या॒जाः । \newline
54. प्र॒जेति॑ प्र - जा । \newline
55. अ॒नू॒या॒जा यद् यद॑नूया॒जा अ॑नूया॒जा यत् । \newline
56. अ॒नू॒या॒जा इत्य॑नु - या॒जाः । \newline
57. यत् प्र॑या॒जान् प्र॑या॒जान्. यद् यत् प्र॑या॒जान् । \newline
58. प्र॒या॒जा नि॒ष्ट्वेष्ट्वा प्र॑या॒जान् प्र॑या॒जा नि॒ष्ट्वा । \newline
59. प्र॒या॒जानिति॑ प्र - या॒जान् । \newline
60. इ॒ष्ट्वा ह॒वीꣳषि॑ ह॒वीꣳषी॒ ष्ट्वेष्ट्वा ह॒वीꣳषि॑ । \newline
61. ह॒वीꣳ ष्य॑भिघा॒रय॑ त्यभिघा॒रय॑ति ह॒वीꣳषि॑ ह॒वीꣳ ष्य॑भिघा॒रय॑ति । \newline
62. अ॒भि॒घा॒रय॑ति पि॒ता पि॒ता ऽभि॑घा॒रय॑ त्यभिघा॒रय॑ति पि॒ता । \newline
63. अ॒भि॒घा॒रय॒तीत्य॑भि - घा॒रय॑ति । \newline
64. पि॒तैवैव पि॒ता पि॒तैव । \newline
65. ए॒व तत् तदे॒वैव तत् । \newline
66. तत् पु॒त्रेण॑ पु॒त्रेण॒ तत् तत् पु॒त्रेण॑ । \newline
67. पु॒त्रेण॒ साधा॑रणꣳ॒॒ साधा॑रणम् पु॒त्रेण॑ पु॒त्रेण॒ साधा॑रणम् । \newline
68. साधा॑रणम् कुरुते कुरुते॒ साधा॑रणꣳ॒॒ साधा॑रणम् कुरुते । \newline

\textbf{Ghana Paata } \newline

1. इति॑ स्वाहाका॒रेण॑ स्वाहाका॒रेणे तीति॑ स्वाहाका॒रेण॑ प्रया॒जेषु॑ प्रया॒जेषु॑ स्वाहाका॒रेणे तीति॑ स्वाहाका॒रेण॑ प्रया॒जेषु॑ । \newline
2. स्वा॒हा॒का॒रेण॑ प्रया॒जेषु॑ प्रया॒जेषु॑ स्वाहाका॒रेण॑ स्वाहाका॒रेण॑ प्रया॒जेषु॑ य॒ज्ञ्ं ॅय॒ज्ञ्म् प्र॑या॒जेषु॑ स्वाहाका॒रेण॑ स्वाहाका॒रेण॑ प्रया॒जेषु॑ य॒ज्ञ्म् । \newline
3. स्वा॒हा॒का॒रेणेति॑ स्वाहा - का॒रेण॑ । \newline
4. प्र॒या॒जेषु॑ य॒ज्ञ्ं ॅय॒ज्ञ्म् प्र॑या॒जेषु॑ प्रया॒जेषु॑ य॒ज्ञ्ꣳ सꣳ॒॒स्थाप्यꣳ॑ सꣳ॒॒स्थाप्यं॑ ॅय॒ज्ञ्म् प्र॑या॒जेषु॑ प्रया॒जेषु॑ य॒ज्ञ्ꣳ सꣳ॒॒स्थाप्य᳚म् । \newline
5. प्र॒या॒जेष्विति॑ प्र - या॒जेषु॑ । \newline
6. य॒ज्ञ्ꣳ सꣳ॒॒स्थाप्यꣳ॑ सꣳ॒॒स्थाप्यं॑ ॅय॒ज्ञ्ं ॅय॒ज्ञ्ꣳ सꣳ॒॒स्थाप्य॑ मपश्यन् नपश्यन् थ्सꣳ॒॒स्थाप्यं॑ ॅय॒ज्ञ्ं ॅय॒ज्ञ्ꣳ सꣳ॒॒स्थाप्य॑ मपश्यन्न् । \newline
7. सꣳ॒॒स्थाप्य॑ मपश्यन् नपश्यन् थ्सꣳ॒॒स्थाप्यꣳ॑ सꣳ॒॒स्थाप्य॑ मपश्य॒न् तम् त म॑पश्यन् थ्सꣳ॒॒स्थाप्यꣳ॑ सꣳ॒॒स्थाप्य॑ मपश्य॒न् तम् । \newline
8. सꣳ॒॒स्थाप्य॒मिति॑ सं - स्थाप्य᳚म् । \newline
9. अ॒प॒श्य॒न् तम् त म॑पश्यन् नपश्य॒न् तꣳ स्वा॑हाका॒रेण॑ स्वाहाका॒रेण॒ त म॑पश्यन् नपश्य॒न् तꣳ स्वा॑हाका॒रेण॑ । \newline
10. तꣳ स्वा॑हाका॒रेण॑ स्वाहाका॒रेण॒ तम् तꣳ स्वा॑हाका॒रेण॑ प्रया॒जेषु॑ प्रया॒जेषु॑ स्वाहाका॒रेण॒ तम् तꣳ स्वा॑हाका॒रेण॑ प्रया॒जेषु॑ । \newline
11. स्वा॒हा॒का॒रेण॑ प्रया॒जेषु॑ प्रया॒जेषु॑ स्वाहाका॒रेण॑ स्वाहाका॒रेण॑ प्रया॒जेषु॒ सꣳ सम् प्र॑या॒जेषु॑ स्वाहाका॒रेण॑ स्वाहाका॒रेण॑ प्रया॒जेषु॒ सम् । \newline
12. स्वा॒हा॒का॒रेणेति॑ स्वाहा - का॒रेण॑ । \newline
13. प्र॒या॒जेषु॒ सꣳ सम् प्र॑या॒जेषु॑ प्रया॒जेषु॒ स म॑स्थापयन् नस्थापय॒न् थ्सम् प्र॑या॒जेषु॑ प्रया॒जेषु॒ स म॑स्थापयन्न् । \newline
14. प्र॒या॒जेष्विति॑ प्र - या॒जेषु॑ । \newline
15. स म॑स्थापयन् नस्थापय॒न् थ्सꣳ स म॑स्थापय॒न् वि व्य॑स्थापय॒न् थ्सꣳ स म॑स्थापय॒न् वि । \newline
16. अ॒स्था॒प॒य॒न् वि व्य॑स्थापयन् नस्थापय॒न् वि वै वै व्य॑स्थापयन् नस्थापय॒न् वि वै । \newline
17. वि वै वै वि वि वा ए॒त दे॒तद् वै वि वि वा ए॒तत् । \newline
18. वा ए॒त दे॒तद् वै वा ए॒तद् य॒ज्ञ्ं ॅय॒ज्ञ् मे॒तद् वै वा ए॒तद् य॒ज्ञ्म् । \newline
19. ए॒तद् य॒ज्ञ्ं ॅय॒ज्ञ् मे॒त दे॒तद् य॒ज्ञ्म् छि॑न्दन्ति छिन्दन्ति य॒ज्ञ् मे॒त दे॒तद् य॒ज्ञ्म् छि॑न्दन्ति । \newline
20. य॒ज्ञ्म् छि॑न्दन्ति छिन्दन्ति य॒ज्ञ्ं ॅय॒ज्ञ्म् छि॑न्दन्ति॒ यद् यच् छि॑न्दन्ति य॒ज्ञ्ं ॅय॒ज्ञ्म् छि॑न्दन्ति॒ यत् । \newline
21. छि॒न्द॒न्ति॒ यद् यच् छि॑न्दन्ति छिन्दन्ति॒ यथ् स्वा॑हाका॒रेण॑ स्वाहाका॒रेण॒ यच् छि॑न्दन्ति छिन्दन्ति॒ यथ् स्वा॑हाका॒रेण॑ । \newline
22. यथ् स्वा॑हाका॒रेण॑ स्वाहाका॒रेण॒ यद् यथ् स्वा॑हाका॒रेण॑ प्रया॒जेषु॑ प्रया॒जेषु॑ स्वाहाका॒रेण॒ यद् यथ् स्वा॑हाका॒रेण॑ प्रया॒जेषु॑ । \newline
23. स्वा॒हा॒का॒रेण॑ प्रया॒जेषु॑ प्रया॒जेषु॑ स्वाहाका॒रेण॑ स्वाहाका॒रेण॑ प्रया॒जेषु॑ सꣳस्था॒पय॑न्ति सꣳस्था॒पय॑न्ति प्रया॒जेषु॑ स्वाहाका॒रेण॑ स्वाहाका॒रेण॑ प्रया॒जेषु॑ सꣳस्था॒पय॑न्ति । \newline
24. स्वा॒हा॒का॒रेणेति॑ स्वाहा - का॒रेण॑ । \newline
25. प्र॒या॒जेषु॑ सꣳस्था॒पय॑न्ति सꣳस्था॒पय॑न्ति प्रया॒जेषु॑ प्रया॒जेषु॑ सꣳस्था॒पय॑न्ति प्रया॒जान् प्र॑या॒जान् थ्सꣳ॑स्था॒पय॑न्ति प्रया॒जेषु॑ प्रया॒जेषु॑ सꣳस्था॒पय॑न्ति प्रया॒जान् । \newline
26. प्र॒या॒जेष्विति॑ प्र - या॒जेषु॑ । \newline
27. सꣳ॒॒स्था॒पय॑न्ति प्रया॒जान् प्र॑या॒जान् थ्सꣳ॑स्था॒पय॑न्ति सꣳस्था॒पय॑न्ति प्रया॒जा नि॒ष्ट्वेष्ट्वा प्र॑या॒जान् थ्सꣳ॑स्था॒पय॑न्ति सꣳस्था॒पय॑न्ति प्रया॒जा नि॒ष्ट्वा । \newline
28. सꣳ॒॒स्था॒पय॒न्तीति॑ सं - स्था॒पय॑न्ति । \newline
29. प्र॒या॒जा नि॒ष्ट्वेष्ट्वा प्र॑या॒जान् प्र॑या॒जा नि॒ष्ट्वा ह॒वीꣳषि॑ ह॒वीꣳषी॒ष्ट्वा प्र॑या॒जान् प्र॑या॒जा नि॒ष्ट्वा ह॒वीꣳषि॑ । \newline
30. प्र॒या॒जानिति॑ प्र - या॒जान् । \newline
31. इ॒ष्ट्वा ह॒वीꣳषि॑ ह॒वीꣳषी॒ ष्ट्वेष्ट्वा ह॒वीꣳ ष्य॒भ्य॑भि ह॒वीꣳषी॒ ष्ट्वेष्ट्वा ह॒वीꣳष्य॒भि । \newline
32. ह॒वीꣳ ष्य॒भ्य॑भि ह॒वीꣳषि॑ ह॒वीꣳष्य॒भि घा॑रयति घारय त्य॒भि ह॒वीꣳषि॑ ह॒वीꣳष्य॒भि घा॑रयति । \newline
33. अ॒भि घा॑रयति घारय त्य॒भ्य॑भि घा॑रयति य॒ज्ञ्स्य॑ य॒ज्ञ्स्य॑ घारय त्य॒भ्य॑भि घा॑रयति य॒ज्ञ्स्य॑ । \newline
34. घा॒र॒य॒ति॒ य॒ज्ञ्स्य॑ य॒ज्ञ्स्य॑ घारयति घारयति य॒ज्ञ्स्य॒ सन्त॑त्यै॒ सन्त॑त्यै य॒ज्ञ्स्य॑ घारयति घारयति य॒ज्ञ्स्य॒ सन्त॑त्यै । \newline
35. य॒ज्ञ्स्य॒ सन्त॑त्यै॒ सन्त॑त्यै य॒ज्ञ्स्य॑ य॒ज्ञ्स्य॒ सन्त॑त्या॒ अथो॒ अथो॒ सन्त॑त्यै य॒ज्ञ्स्य॑ य॒ज्ञ्स्य॒ सन्त॑त्या॒ अथो᳚ । \newline
36. सन्त॑त्या॒ अथो॒ अथो॒ सन्त॑त्यै॒ सन्त॑त्या॒ अथो॑ ह॒विर्. ह॒विरथो॒ सन्त॑त्यै॒ सन्त॑त्या॒ अथो॑ ह॒विः । \newline
37. सन्त॑त्या॒ इति॒ सं - त॒त्यै॒ । \newline
38. अथो॑ ह॒विर्. ह॒विरथो॒ अथो॑ ह॒वि रे॒वैव ह॒विरथो॒ अथो॑ ह॒विरे॒व । \newline
39. अथो॒ इत्यथो᳚ । \newline
40. ह॒वि रे॒वैव ह॒विर्. ह॒विरे॒वाक॑ रक रे॒व ह॒विर्. ह॒वि रे॒वाकः॑ । \newline
41. ए॒वाक॑ रक रे॒वैवाक॒ रथो॒ अथो॑ अक रे॒वैवाक॒ रथो᳚ । \newline
42. अ॒क॒ रथो॒ अथो॑ अक रक॒ रथो॑ यथापू॒र्वं ॅय॑थापू॒र्व मथो॑ अक रक॒ रथो॑ यथापू॒र्वम् । \newline
43. अथो॑ यथापू॒र्वं ॅय॑थापू॒र्व मथो॒ अथो॑ यथापू॒र्व मुपोप॑ यथापू॒र्व मथो॒ अथो॑ यथापू॒र्व मुप॑ । \newline
44. अथो॒ इत्यथो᳚ । \newline
45. य॒था॒पू॒र्व मुपोप॑ यथापू॒र्वं ॅय॑थापू॒र्व मुपै᳚त्ये॒ त्युप॑ यथापू॒र्वं ॅय॑थापू॒र्व मुपै॑ति । \newline
46. य॒था॒पू॒र्वमिति॑ यथा - पू॒र्वम् । \newline
47. उपै᳚त्ये॒ त्युपोपै॑ति पि॒ता पि॒तै त्युपोपै॑ति पि॒ता । \newline
48. ए॒ति॒ पि॒ता पि॒तैत्ये॑ति पि॒ता वै वै पि॒तैत्ये॑ति पि॒ता वै । \newline
49. पि॒ता वै वै पि॒ता पि॒ता वै प्र॑या॒जाः प्र॑या॒जा वै पि॒ता पि॒ता वै प्र॑या॒जाः । \newline
50. वै प्र॑या॒जाः प्र॑या॒जा वै वै प्र॑या॒जाः प्र॒जा प्र॒जा प्र॑या॒जा वै वै प्र॑या॒जाः प्र॒जा । \newline
51. प्र॒या॒जाः प्र॒जा प्र॒जा प्र॑या॒जाः प्र॑या॒जाः प्र॒जा ऽनू॑या॒जा अ॑नूया॒जाः प्र॒जा प्र॑या॒जाः प्र॑या॒जाः प्र॒जा ऽनू॑या॒जाः । \newline
52. प्र॒या॒जा इति॑ प्र - या॒जाः । \newline
53. प्र॒जा ऽनू॑या॒जा अ॑नूया॒जाः प्र॒जा प्र॒जा ऽनू॑या॒जा यद् यद॑नूया॒जाः प्र॒जा प्र॒जा ऽनू॑या॒जा यत् । \newline
54. प्र॒जेति॑ प्र - जा । \newline
55. अ॒नू॒या॒जा यद् यद॑नूया॒जा अ॑नूया॒जा यत् प्र॑या॒जान् प्र॑या॒जान्. यद॑नूया॒जा अ॑नूया॒जा यत् प्र॑या॒जान् । \newline
56. अ॒नू॒या॒जा इत्य॑नु - या॒जाः । \newline
57. यत् प्र॑या॒जान् प्र॑या॒जान्. यद् यत् प्र॑या॒जा नि॒ष्ट्वेष्ट्वा प्र॑या॒जान्. यद् यत् प्र॑या॒जा नि॒ष्ट्वा । \newline
58. प्र॒या॒जा नि॒ष्ट्वेष्ट्वा प्र॑या॒जान् प्र॑या॒जा नि॒ष्ट्वा ह॒वीꣳषि॑ ह॒वीꣳषी॒ष्ट्वा प्र॑या॒जान् प्र॑या॒जा नि॒ष्ट्वा ह॒वीꣳषि॑ । \newline
59. प्र॒या॒जानिति॑ प्र - या॒जान् । \newline
60. इ॒ष्ट्वा ह॒वीꣳषि॑ ह॒वीꣳषी॒ ष्ट्वेष्ट्वा ह॒वीꣳ ष्य॑भिघा॒रय॑ त्यभिघा॒रय॑ति ह॒वीꣳषी॒ ष्ट्वेष्ट्वा 
ह॒वीꣳ ष्य॑भिघा॒रय॑ति । \newline
61. ह॒वीꣳ ष्य॑भिघा॒रय॑ त्यभिघा॒रय॑ति ह॒वीꣳषि॑ ह॒वीꣳ ष्य॑भिघा॒रय॑ति पि॒ता पि॒ता ऽभि॑घा॒रय॑ति ह॒वीꣳषि॑ ह॒वीꣳ ष्य॑भिघा॒रय॑ति पि॒ता । \newline
62. अ॒भि॒घा॒रय॑ति पि॒ता पि॒ता ऽभि॑घा॒रय॑ त्यभिघा॒रय॑ति पि॒तैवैव पि॒ता ऽभि॑घा॒रय॑ त्यभिघा॒रय॑ति पि॒तैव । \newline
63. अ॒भि॒घा॒रय॒तीत्य॑भि - घा॒रय॑ति । \newline
64. पि॒तैवैव पि॒ता पि॒तैव तत् तदे॒व पि॒ता पि॒तैव तत् । \newline
65. ए॒व तत् तदे॒वैव तत् पु॒त्रेण॑ पु॒त्रेण॒ तदे॒वैव तत् पु॒त्रेण॑ । \newline
66. तत् पु॒त्रेण॑ पु॒त्रेण॒ तत् तत् पु॒त्रेण॒ साधा॑रणꣳ॒॒ साधा॑रणम् पु॒त्रेण॒ तत् तत् पु॒त्रेण॒ साधा॑रणम् । \newline
67. पु॒त्रेण॒ साधा॑रणꣳ॒॒ साधा॑रणम् पु॒त्रेण॑ पु॒त्रेण॒ साधा॑रणम् कुरुते कुरुते॒ साधा॑रणम् पु॒त्रेण॑ पु॒त्रेण॒ साधा॑रणम् कुरुते । \newline
68. साधा॑रणम् कुरुते कुरुते॒ साधा॑रणꣳ॒॒ साधा॑रणम् कुरुते॒ तस्मा॒त् तस्मा᳚त् कुरुते॒ साधा॑रणꣳ॒॒ साधा॑रणम् कुरुते॒ तस्मा᳚त् । \newline
\pagebreak
\markright{ TS 2.6.1.7  \hfill https://www.vedavms.in \hfill}
\addcontentsline{toc}{section}{ TS 2.6.1.7 }
\section*{ TS 2.6.1.7 }

\textbf{TS 2.6.1.7 } \newline
\textbf{Samhita Paata} \newline

कुरुते॒ तस्मा॑दाहु॒र्यश्चै॒वं ॅवेद॒ यश्च॒ न क॒था पु॒त्रस्य॒ केव॑लं क॒था साधा॑रणं पि॒तुरित्यस्क॑न्नमे॒व तद्यत् प्र॑या॒जेष्वि॒ष्टेषु॒ स्कन्द॑ति गाय॒त्र्ये॑व तेन॒ गर्भं॑ धत्ते॒ सा प्र॒जां प॒शून्. यज॑मानाय॒ प्रज॑नयति ॥ \newline

\textbf{Pada Paata} \newline

कु॒रु॒ते॒ । तस्मा᳚त् । आ॒हुः॒ । यः । च॒ । ए॒वम् । वेद॑ । यः । च॒ । न । क॒था । पु॒त्रस्य॑ । केव॑लम् । क॒था । साधा॑रणम् । पि॒तुः । इति॑ । अस्क॑न्नम् । ए॒व । तत् । यत् । प्र॒या॒जेष्विति॑ प्र - या॒जेषु॑ । इ॒ष्टेषु॑ । स्कन्द॑ति । गा॒य॒त्री । ए॒व । तेन॑ । गर्भ᳚म् । ध॒त्ते॒ । सा । प्र॒जामिति॑ प्र - जाम् । प॒शून् । यज॑मानाय । प्रेति॑ । ज॒न॒य॒ति॒ ॥  \newline


\textbf{Krama Paata} \newline

कु॒रु॒ते॒ तस्मा᳚त् । तस्मा॑दाहुः । आ॒हु॒र् यः । यश्च॑ । चै॒वम् । ए॒वं ॅवेद॑ । वेद॒ यः । यश्च॑ । च॒ न । न क॒था । क॒था पु॒त्रस्य॑ । पु॒त्रस्य॒ केव॑लम् । केव॑लम् क॒था । क॒था साधा॑रणम् । साधा॑रणम् पि॒तुः । पि॒तुरिति॑ । इ॒त्यस्क॑न्नम् । अस्क॑न्नमे॒व । ए॒व तत् । तद् यत् । यत् प्र॑या॒जेषु॑ । प्र॒या॒जेष्वि॒ष्टेषु॑ । प्र॒या॒जेष्विति॑ प्र - या॒जेषु॑ । इ॒ष्टेषु॒ स्कन्द॑ति । स्कन्द॑ति गाय॒त्री । गा॒य॒त्र्ये॑व । ए॒व तेन॑ । तेन॒ गर्भ᳚म् । गर्भ॑म् धत्ते । ध॒त्ते॒ सा । सा प्र॒जाम् । प्र॒जाम् प॒शून् । प्र॒जामिति॑ प्र - जाम् । प॒शून्. यज॑मानाय । यज॑मानाय॒ प्र । प्र ज॑नयति । ज॒न॒य॒तीति॑ जनयति । \newline

\textbf{Jatai Paata} \newline

1. कु॒रु॒ते॒ तस्मा॒त् तस्मा᳚त् कुरुते कुरुते॒ तस्मा᳚त् । \newline
2. तस्मा॑ दाहु राहु॒ स्तस्मा॒त् तस्मा॑ दाहुः । \newline
3. आ॒हु॒र् यो य आ॑हु राहु॒र् यः । \newline
4. यश्च॑ च॒ यो यश्च॑ । \newline
5. चै॒व मे॒वम् च॑ चै॒वम् । \newline
6. ए॒वं ॅवेद॒ वेदै॒व मे॒वं ॅवेद॑ । \newline
7. वेद॒ यो यो वेद॒ वेद॒ यः । \newline
8. यश्च॑ च॒ यो यश्च॑ । \newline
9. च॒ न न च॑ च॒ न । \newline
10. न क॒था क॒था न न क॒था । \newline
11. क॒था पु॒त्रस्य॑ पु॒त्रस्य॑ क॒था क॒था पु॒त्रस्य॑ । \newline
12. पु॒त्रस्य॒ केव॑ल॒म् केव॑लम् पु॒त्रस्य॑ पु॒त्रस्य॒ केव॑लम् । \newline
13. केव॑लम् क॒था क॒था केव॑ल॒म् केव॑लम् क॒था । \newline
14. क॒था साधा॑रणꣳ॒॒ साधा॑रणम् क॒था क॒था साधा॑रणम् । \newline
15. साधा॑रणम् पि॒तुः पि॒तुः साधा॑रणꣳ॒॒ साधा॑रणम् पि॒तुः । \newline
16. पि॒तु रितीति॑ पि॒तुः पि॒तु रिति॑ । \newline
17. इत्यस्क॑न्न॒ मस्क॑न्न॒ मिती त्यस्क॑न्नम् । \newline
18. अस्क॑न्न मे॒वैवा स्क॑न्न॒ मस्क॑न्न मे॒व । \newline
19. ए॒व तत् तदे॒वैव तत् । \newline
20. तद् यद् यत् तत् तद् यत् । \newline
21. यत् प्र॑या॒जेषु॑ प्रया॒जेषु॒ यद् यत् प्र॑या॒जेषु॑ । \newline
22. प्र॒या॒जे ष्वि॒ष्टे ष्वि॒ष्टेषु॑ प्रया॒जेषु॑ प्रया॒जे ष्वि॒ष्टेषु॑ । \newline
23. प्र॒या॒जेष्विति॑ प्र - या॒जेषु॑ । \newline
24. इ॒ष्टेषु॒ स्कन्द॑ति॒ स्कन्द॑ ती॒ष्टेष्वि॒ष्टेषु॒ स्कन्द॑ति । \newline
25. स्कन्द॑ति गाय॒त्री गा॑य॒त्री स्कन्द॑ति॒ स्कन्द॑ति गाय॒त्री । \newline
26. गा॒य॒त्र्ये॑वैव गा॑य॒त्री गा॑य॒त्र्ये॑व । \newline
27. ए॒व तेन॒ तेनै॒वैव तेन॑ । \newline
28. तेन॒ गर्भ॒म् गर्भ॒म् तेन॒ तेन॒ गर्भ᳚म् । \newline
29. गर्भ॑म् धत्ते धत्ते॒ गर्भ॒म् गर्भ॑म् धत्ते । \newline
30. ध॒त्ते॒ सा सा ध॑त्ते धत्ते॒ सा । \newline
31. सा प्र॒जाम् प्र॒जाꣳ सा सा प्र॒जाम् । \newline
32. प्र॒जाम् प॒शून् प॒शून् प्र॒जाम् प्र॒जाम् प॒शून् । \newline
33. प्र॒जामिति॑ प्र - जाम् । \newline
34. प॒शून्. यज॑मानाय॒ यज॑मानाय प॒शून् प॒शून्. यज॑मानाय । \newline
35. यज॑मानाय॒ प्र प्र यज॑मानाय॒ यज॑मानाय॒ प्र । \newline
36. प्र ज॑नयति जनयति॒ प्र प्र ज॑नयति । \newline
37. ज॒न॒य॒तीति॑ जनयति । \newline

\textbf{Ghana Paata } \newline

1. कु॒रु॒ते॒ तस्मा॒त् तस्मा᳚त् कुरुते कुरुते॒ तस्मा॑ दाहु राहु॒ स्तस्मा᳚त् कुरुते कुरुते॒ तस्मा॑दाहुः । \newline
2. तस्मा॑ दाहु राहु॒ स्तस्मा॒त् तस्मा॑ दाहु॒र् यो य आ॑हु॒ स्तस्मा॒त् तस्मा॑ दाहु॒र् यः । \newline
3. आ॒हु॒र् यो य आ॑हु राहु॒र् यश्च॑ च॒ य आ॑हु राहु॒र् यश्च॑ । \newline
4. यश्च॑ च॒ यो यश्चै॒व मे॒वम् च॒ यो यश्चै॒वम् । \newline
5. चै॒व मे॒वम् च॑ चै॒वं ॅवेद॒ वेदै॒वम् च॑ चै॒वं ॅवेद॑ । \newline
6. ए॒वं ॅवेद॒ वेदै॒व मे॒वं ॅवेद॒ यो यो वेदै॒व मे॒वं ॅवेद॒ यः । \newline
7. वेद॒ यो यो वेद॒ वेद॒ यश्च॑ च॒ यो वेद॒ वेद॒ यश्च॑ । \newline
8. यश्च॑ च॒ यो यश्च॒ न न च॒ यो यश्च॒ न । \newline
9. च॒ न न च॑ च॒ न क॒था क॒था न च॑ च॒ न क॒था । \newline
10. न क॒था क॒था न न क॒था पु॒त्रस्य॑ पु॒त्रस्य॑ क॒था न न क॒था पु॒त्रस्य॑ । \newline
11. क॒था पु॒त्रस्य॑ पु॒त्रस्य॑ क॒था क॒था पु॒त्रस्य॒ केव॑ल॒म् केव॑लम् पु॒त्रस्य॑ क॒था क॒था पु॒त्रस्य॒ केव॑लम् । \newline
12. पु॒त्रस्य॒ केव॑ल॒म् केव॑लम् पु॒त्रस्य॑ पु॒त्रस्य॒ केव॑लम् क॒था क॒था केव॑लम् पु॒त्रस्य॑ पु॒त्रस्य॒ केव॑लम् क॒था । \newline
13. केव॑लम् क॒था क॒था केव॑ल॒म् केव॑लम् क॒था साधा॑रणꣳ॒॒ साधा॑रणम् क॒था केव॑ल॒म् केव॑लम् क॒था साधा॑रणम् । \newline
14. क॒था साधा॑रणꣳ॒॒ साधा॑रणम् क॒था क॒था साधा॑रणम् पि॒तुः पि॒तुः साधा॑रणम् क॒था क॒था साधा॑रणम् पि॒तुः । \newline
15. साधा॑रणम् पि॒तुः पि॒तुः साधा॑रणꣳ॒॒ साधा॑रणम् पि॒तुरितीति॑ पि॒तुः साधा॑रणꣳ॒॒ साधा॑रणम् पि॒तुरिति॑ । \newline
16. पि॒तुरितीति॑ पि॒तुः पि॒तु रित्यस्क॑न्न॒ मस्क॑न्न॒ मिति॑ पि॒तुः पि॒तु रित्यस्क॑न्नम् । \newline
17. इत्यस्क॑न्न॒ मस्क॑न्न॒ मिती त्यस्क॑न्न मे॒वैवास्क॑न्न॒ मिती त्यस्क॑न्न मे॒व । \newline
18. अस्क॑न्न मे॒वैवास्क॑न्न॒ मस्क॑न्न मे॒व तत् तदे॒वास्क॑न्न॒ मस्क॑न्न मे॒व तत् । \newline
19. ए॒व तत् तदे॒वैव तद् यद् यत् तदे॒वैव तद् यत् । \newline
20. तद् यद् यत् तत् तद् यत् प्र॑या॒जेषु॑ प्रया॒जेषु॒ यत् तत् तद् यत् प्र॑या॒जेषु॑ । \newline
21. यत् प्र॑या॒जेषु॑ प्रया॒जेषु॒ यद् यत् प्र॑या॒जे ष्वि॒ष्टे ष्वि॒ष्टेषु॑ प्रया॒जेषु॒ यद् यत् प्र॑या॒जे ष्वि॒ष्टेषु॑ । \newline
22. प्र॒या॒जे ष्वि॒ष्टे ष्वि॒ष्टेषु॑ प्रया॒जेषु॑ प्रया॒जे ष्वि॒ष्टेषु॒ स्कन्द॑ति॒ स्कन्द॑ती॒ष्टेषु॑ प्रया॒जेषु॑ प्रया॒जे ष्वि॒ष्टेषु॒ स्कन्द॑ति । \newline
23. प्र॒या॒जेष्विति॑ प्र - या॒जेषु॑ । \newline
24. इ॒ष्टेषु॒ स्कन्द॑ति॒ स्कन्द॑ती॒ ष्टेष्वि॒ष्टेषु॒ स्कन्द॑ति गाय॒त्री गा॑य॒त्री स्कन्द॑ती॒ ष्टेष्वि॒ष्टेषु॒ स्कन्द॑ति गाय॒त्री । \newline
25. स्कन्द॑ति गाय॒त्री गा॑य॒त्री स्कन्द॑ति॒ स्कन्द॑ति गाय॒त्र्ये॑वैव गा॑य॒त्री स्कन्द॑ति॒ स्कन्द॑ति गाय॒त्र्ये॑व । \newline
26. गा॒य॒त्र्ये॑वैव गा॑य॒त्री गा॑य॒त्र्ये॑व तेन॒ तेनै॒व गा॑य॒त्री गा॑य॒त्र्ये॑व तेन॑ । \newline
27. ए॒व तेन॒ तेनै॒वैव तेन॒ गर्भ॒म् गर्भ॒म् तेनै॒वैव तेन॒ गर्भ᳚म् । \newline
28. तेन॒ गर्भ॒म् गर्भ॒म् तेन॒ तेन॒ गर्भ॑म् धत्ते धत्ते॒ गर्भ॒म् तेन॒ तेन॒ गर्भ॑म् धत्ते । \newline
29. गर्भ॑म् धत्ते धत्ते॒ गर्भ॒म् गर्भ॑म् धत्ते॒ सा सा ध॑त्ते॒ गर्भ॒म् गर्भ॑म् धत्ते॒ सा । \newline
30. ध॒त्ते॒ सा सा ध॑त्ते धत्ते॒ सा प्र॒जाम् प्र॒जाꣳ सा ध॑त्ते धत्ते॒ सा प्र॒जाम् । \newline
31. सा प्र॒जाम् प्र॒जाꣳ सा सा प्र॒जाम् प॒शून् प॒शून् प्र॒जाꣳ सा सा प्र॒जाम् प॒शून् । \newline
32. प्र॒जाम् प॒शून् प॒शून् प्र॒जाम् प्र॒जाम् प॒शून्. यज॑मानाय॒ यज॑मानाय प॒शून् प्र॒जाम् प्र॒जाम् प॒शून्. यज॑मानाय । \newline
33. प्र॒जामिति॑ प्र - जाम् । \newline
34. प॒शून्. यज॑मानाय॒ यज॑मानाय प॒शून् प॒शून्. यज॑मानाय॒ प्र प्र यज॑मानाय प॒शून् प॒शून्. यज॑मानाय॒ प्र । \newline
35. यज॑मानाय॒ प्र प्र यज॑मानाय॒ यज॑मानाय॒ प्र ज॑नयति जनयति॒ प्र यज॑मानाय॒ यज॑मानाय॒ प्र ज॑नयति । \newline
36. प्र ज॑नयति जनयति॒ प्र प्र ज॑नयति । \newline
37. ज॒न॒य॒तीति॑ जनयति । \newline
\pagebreak
\markright{ TS 2.6.2.1  \hfill https://www.vedavms.in \hfill}
\addcontentsline{toc}{section}{ TS 2.6.2.1 }
\section*{ TS 2.6.2.1 }

\textbf{TS 2.6.2.1 } \newline
\textbf{Samhita Paata} \newline

चक्षु॑षी॒ वा ए॒ते य॒ज्ञ्स्य॒ यदाज्य॑भागौ॒ यदाज्य॑भागौ॒ यज॑ति॒ चक्षु॑षी ए॒व तद्-य॒ज्ञ्स्य॒ प्रति॑ दधाति पूर्वा॒र्द्धे जु॑होति॒ तस्मा᳚त् पूर्वा॒र्द्धे चक्षु॑षी प्र॒बाहु॑ग्-जुहोति॒ तस्मा᳚त् प्र॒बाहु॒क् चक्षु॑षी देवलो॒कं ॅवा अ॒ग्निना॒ यज॑मा॒नोऽनु॑ पश्यति पितृलो॒कꣳ सोमे॑नोत्तरा॒र्द्धे᳚ ऽग्नये॑ जुहोति दक्षिणा॒र्द्धे सोमा॑यै॒वमि॑व॒ हीमौ लो॒काव॒नयो᳚ र्लो॒कयो॒रनु॑ख्यात्यै॒ राजा॑नौ॒ वा ए॒तौ दे॒वता॑नां॒ - [  ] \newline

\textbf{Pada Paata} \newline

चक्षु॑षी॒ इति॑ । वै । ए॒ते इति॑ । य॒ज्ञ्स्य॑ । यत् । आज्य॑भाग॒वित्याज्य॑ - भा॒गौ॒ । यत् । आज्य॑भागा॒वित्याज्य॑-भा॒गौ॒ । यज॑ति । चक्षु॑षी॒ इति॑ । ए॒व । तत् । य॒ज्ञ्स्य॑ । प्रतीति॑ । द॒धा॒ति॒ । पू॒र्वा॒र्द्ध इति॑ पूर्व - अ॒र्द्धे । जु॒हो॒ति॒ । तस्मा᳚त् । पू॒र्वा॒र्द्ध इति॑ पूर्व - अ॒र्द्धे । चक्षु॑षी॒ इति॑ । प्र॒बाहु॒गिति॑ प्र - बाहु॑क् । जु॒हो॒ति॒ । तस्मा᳚त् । प्र॒बाहु॒गिति॑ प्र - बाहु॑क् । चक्षु॑षी॒ इति॑ । दे॒व॒लो॒कमिति॑ देव-लो॒कम् । वै । अ॒ग्निना᳚ । यज॑मानः । अन्विति॑ । प॒श्य॒ति॒ । पि॒तृ॒लो॒कमिति॑ पितृ-लो॒कम् । सोमे॑न । उ॒त्त॒रा॒र्द्ध इत्यु॑त्तर-अ॒र्द्धे । अ॒ग्नये᳚ । जु॒हो॒ति॒ । द॒क्षि॒णा॒र्द्ध इति॑ दक्षिण-अ॒र्द्धे । सोमा॑य । ए॒वम् । इ॒व॒ । हि । इ॒मौ । लो॒कौ । अ॒नयोः᳚ । लो॒कयोः᳚ । अनु॑ख्यात्या॒ इत्यनु॑ - ख्या॒त्यै॒ । राजा॑नौ । वै । ए॒तौ । दे॒वता॑नाम् ।  \newline


\textbf{Krama Paata} \newline

चक्षु॑षी॒ वै । चक्षु॑षी॒ इति॒ चक्षु॑षी । वा ए॒ते । ए॒ते य॒ज्ञ्स्य॑ । ए॒ते इत्ये॒ते । य॒ज्ञ्स्य॒ यत् । यदाज्य॑भागौ । आज्य॑भागौ॒ यत् । आज्य॑भागा॒वित्याज्य॑ - भा॒गौ॒ । यदाज्य॑भागौ । आज्य॑भागौ॒ यज॑ति । आज्य॑भागा॒वित्याज्य॑ - भा॒गौ॒ । यज॑ति॒ चक्षु॑षी । चक्षु॑षी ए॒व । चक्षु॑षी॒ इति॒ चक्षु॑षी । ए॒व तत् । तद् य॒ज्ञ्स्य॑ । य॒ज्ञ्स्य॒ प्रति॑ । प्रति॑ दधाति । द॒धा॒ति॒ पू॒र्वा॒र्द्धे । पू॒र्वा॒र्द्धे जु॑होति । पू॒र्वा॒र्द्ध इति॑ पूर्व - अ॒र्द्धे । जु॒हो॒ति॒ तस्मा᳚त् । तस्मा᳚त् पूर्वा॒र्द्धे । पू॒र्वा॒र्द्धे चक्षु॑षी । पू॒र्वा॒र्द्ध इति॑ पूर्व - अ॒र्द्धे । चक्षु॑षी प्र॒बाहु॑क् । चक्षु॑षी॒ इति॒ चक्षु॑षी । प्र॒बाहु॑ग् जुहोति । प्र॒बाहु॒गिति॑ प्र - बाहु॑क् । जु॒हो॒ति॒ तस्मा᳚त् । तस्मा᳚त् प्र॒बाहु॑क् । प्र॒बाहु॒क् चक्षु॑षी । प्र॒बाहु॒गिति॑ प्र - बाहु॑क् । चक्षु॑षी देवलो॒कम् । चक्षु॑षी॒ इति॒ चक्षु॑षी । दे॒व॒लो॒कं ॅवै । दे॒व॒लो॒कमिति॑ देव - लो॒कम् । वा अ॒ग्निना᳚ । अ॒ग्निना॒ यज॑मानः । यज॑मा॒नोऽनु॑ । अनु॑ पश्यति । प॒श्य॒ति॒ पि॒तृ॒लो॒कम् । पि॒तृ॒लो॒कꣳ सोमे॑न । पि॒तृ॒लो॒कमिति॑ पितृ - लो॒कम् । सोमे॑नोत्तरा॒र्द्धे । उ॒त्त॒रा॒र्द्धे᳚ऽग्नये᳚ । उ॒त्त॒रा॒र्द्ध इत्यु॑त्तर - अ॒र्द्धे । अ॒ग्नये॑ जुहोति । जु॒हो॒ति॒ द॒क्षि॒णा॒र्द्धे । द॒क्षि॒णा॒र्द्धे सोमा॑य । द॒क्षि॒णा॒र्द्ध इति॑ दक्षिण - अ॒र्द्धे । सोमा॑यै॒वम् । ए॒वमि॑व । इ॒व॒ हि । हीमौ । इ॒मौ लो॒कौ । लो॒काव॒नयोः᳚ । अ॒नयो᳚र् लो॒कयोः᳚ । लो॒कयो॒रनु॑ख्यात्यै । अनु॑ख्यात्यै॒ राजा॑नौ । अनु॑ख्यात्या॒ इत्यनु॑ - ख्या॒त्यै॒ । राजा॑नौ॒ वै । वा ए॒तौ । ए॒तौ दे॒वता॑नाम् । दे॒वता॑नां॒ ॅयत् \newline

\textbf{Jatai Paata} \newline

1. चक्षु॑षी॒ वै वै चक्षु॑षी॒ चक्षु॑षी॒ वै । \newline
2. चक्षु॑षी॒ इति॒ चक्षु॑षी । \newline
3. वा ए॒ते ए॒ते वै वा ए॒ते । \newline
4. ए॒ते य॒ज्ञ्स्य॑ य॒ज्ञ्स्यै॒ते ए॒ते य॒ज्ञ्स्य॑ । \newline
5. ए॒ते इत्ये॒ते । \newline
6. य॒ज्ञ्स्य॒ यद् यद् य॒ज्ञ्स्य॑ य॒ज्ञ्स्य॒ यत् । \newline
7. यदाज्य॑भागा॒ वाज्य॑भागौ॒ यद् यदाज्य॑भागौ । \newline
8. आज्य॑भागौ॒ यद् यदाज्य॑भागा॒ वाज्य॑भागौ॒ यत् । \newline
9. आज्य॑भागा॒वित्याज्य॑ - भा॒गौ॒ । \newline
10. यदाज्य॑भागा॒ वाज्य॑भागौ॒ यद् यदाज्य॑भागौ । \newline
11. आज्य॑भागौ॒ यज॑ति॒ यज॒ त्याज्य॑भागा॒ वाज्य॑भागौ॒ यज॑ति । \newline
12. आज्य॑भागा॒वित्याज्य॑ - भा॒गौ॒ । \newline
13. यज॑ति॒ चक्षु॑षी॒ चक्षु॑षी॒ यज॑ति॒ यज॑ति॒ चक्षु॑षी । \newline
14. चक्षु॑षी ए॒वैव चक्षु॑षी॒ चक्षु॑षी ए॒व । \newline
15. चक्षु॑षी॒ इति॒ चक्षु॑षी । \newline
16. ए॒व तत् तदे॒वैव तत् । \newline
17. तद् य॒ज्ञ्स्य॑ य॒ज्ञ्स्य॒ तत् तद् य॒ज्ञ्स्य॑ । \newline
18. य॒ज्ञ्स्य॒ प्रति॒ प्रति॑ य॒ज्ञ्स्य॑ य॒ज्ञ्स्य॒ प्रति॑ । \newline
19. प्रति॑ दधाति दधाति॒ प्रति॒ प्रति॑ दधाति । \newline
20. द॒धा॒ति॒ पू॒र्वा॒र्द्धे पू᳚र्वा॒र्द्धे द॑धाति दधाति पूर्वा॒र्द्धे । \newline
21. पू॒र्वा॒र्द्धे जु॑होति जुहोति पूर्वा॒र्द्धे पू᳚र्वा॒र्द्धे जु॑होति । \newline
22. पू॒र्वा॒र्द्ध इति॑ पूर्व - अ॒र्द्धे । \newline
23. जु॒हो॒ति॒ तस्मा॒त् तस्मा᳚ज् जुहोति जुहोति॒ तस्मा᳚त् । \newline
24. तस्मा᳚त् पूर्वा॒र्द्धे पू᳚र्वा॒र्द्धे तस्मा॒त् तस्मा᳚त् पूर्वा॒र्द्धे । \newline
25. पू॒र्वा॒र्द्धे चक्षु॑षी॒ चक्षु॑षी पूर्वा॒र्द्धे पू᳚र्वा॒र्द्धे चक्षु॑षी । \newline
26. पू॒र्वा॒र्द्ध इति॑ पूर्व - अ॒र्द्धे । \newline
27. चक्षु॑षी प्र॒बाहु॑क् प्र॒बाहु॒क् चक्षु॑षी॒ चक्षु॑षी प्र॒बाहु॑क् । \newline
28. चक्षु॑षी॒ इति॒ चक्षु॑षी । \newline
29. प्र॒बाहु॑ग् जुहोति जुहोति प्र॒बाहु॑क् प्र॒बाहु॑ग् जुहोति । \newline
30. प्र॒बाहु॒गिति॑ प्र - बाहु॑क् । \newline
31. जु॒हो॒ति॒ तस्मा॒त् तस्मा᳚ज् जुहोति जुहोति॒ तस्मा᳚त् । \newline
32. तस्मा᳚त् प्र॒बाहु॑क् प्र॒बाहु॒क् तस्मा॒त् तस्मा᳚त् प्र॒बाहु॑क् । \newline
33. प्र॒बाहु॒क् चक्षु॑षी॒ चक्षु॑षी प्र॒बाहु॑क् प्र॒बाहु॒क् चक्षु॑षी । \newline
34. प्र॒बाहु॒गिति॑ प्र - बाहु॑क् । \newline
35. चक्षु॑षी देवलो॒कम् दे॑वलो॒कम् चक्षु॑षी॒ चक्षु॑षी देवलो॒कम् । \newline
36. चक्षु॑षी॒ इति॒ चक्षु॑षी । \newline
37. दे॒व॒लो॒कं ॅवै वै दे॑वलो॒कम् दे॑वलो॒कं ॅवै । \newline
38. दे॒व॒लो॒कमिति॑ देव - लो॒कम् । \newline
39. वा अ॒ग्निना॒ ऽग्निना॒ वै वा अ॒ग्निना᳚ । \newline
40. अ॒ग्निना॒ यज॑मानो॒ यज॑मानो॒ ऽग्निना॒ ऽग्निना॒ यज॑मानः । \newline
41. यज॑मा॒नो ऽन्वनु॒ यज॑मानो॒ यज॑मा॒नो ऽनु॑ । \newline
42. अनु॑ पश्यति पश्य॒ त्यन्वनु॑ पश्यति । \newline
43. प॒श्य॒ति॒ पि॒तृ॒लो॒कम् पि॑तृलो॒कम् प॑श्यति पश्यति पितृलो॒कम् । \newline
44. पि॒तृ॒लो॒कꣳ सोमे॑न॒ सोमे॑न पितृलो॒कम् पि॑तृलो॒कꣳ सोमे॑न । \newline
45. पि॒तृ॒लो॒कमिति॑ पितृ - लो॒कम् । \newline
46. सोमे॑नोत्तरा॒र्द्ध उ॑त्तरा॒र्द्धे सोमे॑न॒ सोमे॑नोत्तरा॒र्द्धे । \newline
47. उ॒त्त॒रा॒र्द्धे᳚ ऽग्नये॒ ऽग्नय॑ उत्तरा॒र्द्ध उ॑त्तरा॒र्द्धे᳚ ऽग्नये᳚ । \newline
48. उ॒त्त॒रा॒र्द्ध इत्यु॑त्तर - अ॒र्द्धे । \newline
49. अ॒ग्नये॑ जुहोति जुहो त्य॒ग्नये॒ ऽग्नये॑ जुहोति । \newline
50. जु॒हो॒ति॒ द॒क्षि॒णा॒र्द्धे द॑क्षिणा॒र्द्धे जु॑होति जुहोति दक्षिणा॒र्द्धे । \newline
51. द॒क्षि॒णा॒र्द्धे सोमा॑य॒ सोमा॑य दक्षिणा॒र्द्धे द॑क्षिणा॒र्द्धे सोमा॑य । \newline
52. द॒क्षि॒णा॒र्द्ध इति॑ दक्षिण - अ॒र्द्धे । \newline
53. सोमा॑यै॒व मे॒वꣳ सोमा॑य॒ सोमा॑यै॒वम् । \newline
54. ए॒व मि॑वे वै॒व मे॒व मि॑व । \newline
55. इ॒व॒ हि हीवे॑ व॒ हि । \newline
56. हीमा वि॒मौ हि हीमौ । \newline
57. इ॒मौ लो॒कौ लो॒का वि॒मा वि॒मौ लो॒कौ । \newline
58. लो॒का व॒नयो॑ र॒नयो᳚र् लो॒कौ लो॒का व॒नयोः᳚ । \newline
59. अ॒नयो᳚र् लो॒कयो᳚र् लो॒कयो॑ र॒नयो॑ र॒नयो᳚र् लो॒कयोः᳚ । \newline
60. लो॒कयो॒ रनु॑ख्यात्या॒ अनु॑ख्यात्यै लो॒कयो᳚र् लो॒कयो॒ रनु॑ख्यात्यै । \newline
61. अनु॑ख्यात्यै॒ राजा॑नौ॒ राजा॑ना॒ वनु॑ख्यात्या॒ अनु॑ख्यात्यै॒ राजा॑नौ । \newline
62. अनु॑ख्यात्या॒ इत्यनु॑ - ख्या॒त्यै॒ । \newline
63. राजा॑नौ॒ वै वै राजा॑नौ॒ राजा॑नौ॒ वै । \newline
64. वा ए॒ता वे॒तौ वै वा ए॒तौ । \newline
65. ए॒तौ दे॒वता॑नाम् दे॒वता॑ना मे॒ता वे॒तौ दे॒वता॑नाम् । \newline
66. दे॒वता॑नां॒ ॅयद् यद् दे॒वता॑नाम् दे॒वता॑नां॒ ॅयत् । \newline

\textbf{Ghana Paata } \newline

1. चक्षु॑षी॒ वै वै चक्षु॑षी॒ चक्षु॑षी॒ वा ए॒ते ए॒ते वै चक्षु॑षी॒ चक्षु॑षी॒ वा ए॒ते । \newline
2. चक्षु॑षी॒ इति॒ चक्षु॑षी । \newline
3. वा ए॒ते ए॒ते वै वा ए॒ते य॒ज्ञ्स्य॑ य॒ज्ञ्स्यै॒ते वै वा ए॒ते य॒ज्ञ्स्य॑ । \newline
4. ए॒ते य॒ज्ञ्स्य॑ य॒ज्ञ्स्यै॒ते ए॒ते य॒ज्ञ्स्य॒ यद् यद् य॒ज्ञ्स्यै॒ते ए॒ते य॒ज्ञ्स्य॒ यत् । \newline
5. ए॒ते इत्ये॒ते । \newline
6. य॒ज्ञ्स्य॒ यद् यद् य॒ज्ञ्स्य॑ य॒ज्ञ्स्य॒ यदाज्य॑भागा॒ वाज्य॑भागौ॒ यद् य॒ज्ञ्स्य॑ य॒ज्ञ्स्य॒ यदाज्य॑भागौ । \newline
7. यदाज्य॑भागा॒ वाज्य॑भागौ॒ यद् यदाज्य॑भागौ॒ यद् यदाज्य॑भागौ॒ यद् यदाज्य॑भागौ॒ यत् । \newline
8. आज्य॑भागौ॒ यद् यदाज्य॑भागा॒ वाज्य॑भागौ॒ यदाज्य॑भागा॒ वाज्य॑भागौ॒ यदाज्य॑भागा॒ वाज्य॑भागौ॒ यदाज्य॑भागौ । \newline
9. आज्य॑भागा॒वित्याज्य॑ - भा॒गौ॒ । \newline
10. यदाज्य॑भागा॒ वाज्य॑भागौ॒ यद् यदाज्य॑भागौ॒ यज॑ति॒ यज॒ त्याज्य॑भागौ॒ यद् यदाज्य॑भागौ॒ यज॑ति । \newline
11. आज्य॑भागौ॒ यज॑ति॒ यज॒ त्याज्य॑भागा॒ वाज्य॑भागौ॒ यज॑ति॒ चक्षु॑षी॒ चक्षु॑षी॒ यज॒ त्याज्य॑भागा॒ वाज्य॑भागौ॒ यज॑ति॒ चक्षु॑षी । \newline
12. आज्य॑भागा॒वित्याज्य॑ - भा॒गौ॒ । \newline
13. यज॑ति॒ चक्षु॑षी॒ चक्षु॑षी॒ यज॑ति॒ यज॑ति॒ चक्षु॑षी ए॒वैव चक्षु॑षी॒ यज॑ति॒ यज॑ति॒ चक्षु॑षी ए॒व । \newline
14. चक्षु॑षी ए॒वैव चक्षु॑षी॒ चक्षु॑षी ए॒व तत् तदे॒व चक्षु॑षी॒ चक्षु॑षी ए॒व तत् । \newline
15. चक्षु॑षी॒ इति॒ चक्षु॑षी । \newline
16. ए॒व तत् तदे॒वैव तद् य॒ज्ञ्स्य॑ य॒ज्ञ्स्य॒ तदे॒वैव तद् य॒ज्ञ्स्य॑ । \newline
17. तद् य॒ज्ञ्स्य॑ य॒ज्ञ्स्य॒ तत् तद् य॒ज्ञ्स्य॒ प्रति॒ प्रति॑ य॒ज्ञ्स्य॒ तत् तद् य॒ज्ञ्स्य॒ प्रति॑ । \newline
18. य॒ज्ञ्स्य॒ प्रति॒ प्रति॑ य॒ज्ञ्स्य॑ य॒ज्ञ्स्य॒ प्रति॑ दधाति दधाति॒ प्रति॑ य॒ज्ञ्स्य॑ य॒ज्ञ्स्य॒ प्रति॑ दधाति । \newline
19. प्रति॑ दधाति दधाति॒ प्रति॒ प्रति॑ दधाति पूर्वा॒र्द्धे पू᳚र्वा॒र्द्धे द॑धाति॒ प्रति॒ प्रति॑ दधाति पूर्वा॒र्द्धे । \newline
20. द॒धा॒ति॒ पू॒र्वा॒र्द्धे पू᳚र्वा॒र्द्धे द॑धाति दधाति पूर्वा॒र्द्धे जु॑होति जुहोति पूर्वा॒र्द्धे द॑धाति दधाति पूर्वा॒र्द्धे जु॑होति । \newline
21. पू॒र्वा॒र्द्धे जु॑होति जुहोति पूर्वा॒र्द्धे पू᳚र्वा॒र्द्धे जु॑होति॒ तस्मा॒त् तस्मा᳚ज् जुहोति पूर्वा॒र्द्धे पू᳚र्वा॒र्द्धे जु॑होति॒ तस्मा᳚त् । \newline
22. पू॒र्वा॒र्द्ध इति॑ पूर्व - अ॒र्द्धे । \newline
23. जु॒हो॒ति॒ तस्मा॒त् तस्मा᳚ज् जुहोति जुहोति॒ तस्मा᳚त् पूर्वा॒र्द्धे पू᳚र्वा॒र्द्धे तस्मा᳚ज् जुहोति जुहोति॒ तस्मा᳚त् पूर्वा॒र्द्धे । \newline
24. तस्मा᳚त् पूर्वा॒र्द्धे पू᳚र्वा॒र्द्धे तस्मा॒त् तस्मा᳚त् पूर्वा॒र्द्धे चक्षु॑षी॒ चक्षु॑षी पूर्वा॒र्द्धे तस्मा॒त् तस्मा᳚त् पूर्वा॒र्द्धे चक्षु॑षी । \newline
25. पू॒र्वा॒र्द्धे चक्षु॑षी॒ चक्षु॑षी पूर्वा॒र्द्धे पू᳚र्वा॒र्द्धे चक्षु॑षी प्र॒बाहु॑क् प्र॒बाहु॒क् चक्षु॑षी पूर्वा॒र्द्धे 
पू᳚र्वा॒र्द्धे चक्षु॑षी प्र॒बाहु॑क् । \newline
26. पू॒र्वा॒र्द्ध इति॑ पूर्व - अ॒र्द्धे । \newline
27. चक्षु॑षी प्र॒बाहु॑क् प्र॒बाहु॒क् चक्षु॑षी॒ चक्षु॑षी प्र॒बाहु॑ग् जुहोति जुहोति प्र॒बाहु॒क् चक्षु॑षी॒ चक्षु॑षी प्र॒बाहु॑ग् जुहोति । \newline
28. चक्षु॑षी॒ इति॒ चक्षु॑षी । \newline
29. प्र॒बाहु॑ग् जुहोति जुहोति प्र॒बाहु॑क् प्र॒बाहु॑ग् जुहोति॒ तस्मा॒त् तस्मा᳚ज् जुहोति प्र॒बाहु॑क् प्र॒बाहु॑ग् जुहोति॒ तस्मा᳚त् । \newline
30. प्र॒बाहु॒गिति॑ प्र - बाहु॑क् । \newline
31. जु॒हो॒ति॒ तस्मा॒त् तस्मा᳚ज् जुहोति जुहोति॒ तस्मा᳚त् प्र॒बाहु॑क् प्र॒बाहु॒क् तस्मा᳚ज् जुहोति जुहोति॒ तस्मा᳚त् प्र॒बाहु॑क् । \newline
32. तस्मा᳚त् प्र॒बाहु॑क् प्र॒बाहु॒क् तस्मा॒त् तस्मा᳚त् प्र॒बाहु॒क् चक्षु॑षी॒ चक्षु॑षी प्र॒बाहु॒क् तस्मा॒त् तस्मा᳚त् प्र॒बाहु॒क् चक्षु॑षी । \newline
33. प्र॒बाहु॒क् चक्षु॑षी॒ चक्षु॑षी प्र॒बाहु॑क् प्र॒बाहु॒क् चक्षु॑षी देवलो॒कम् दे॑वलो॒कम् चक्षु॑षी प्र॒बाहु॑क् प्र॒बाहु॒क् चक्षु॑षी देवलो॒कम् । \newline
34. प्र॒बाहु॒गिति॑ प्र - बाहु॑क् । \newline
35. चक्षु॑षी देवलो॒कम् दे॑वलो॒कम् चक्षु॑षी॒ चक्षु॑षी देवलो॒कं ॅवै वै दे॑वलो॒कम् चक्षु॑षी॒ चक्षु॑षी देवलो॒कं ॅवै । \newline
36. चक्षु॑षी॒ इति॒ चक्षु॑षी । \newline
37. दे॒व॒लो॒कं ॅवै वै दे॑वलो॒कम् दे॑वलो॒कं ॅवा अ॒ग्निना॒ ऽग्निना॒ वै दे॑वलो॒कम् दे॑वलो॒कं ॅवा अ॒ग्निना᳚ । \newline
38. दे॒व॒लो॒कमिति॑ देव - लो॒कम् । \newline
39. वा अ॒ग्निना॒ ऽग्निना॒ वै वा अ॒ग्निना॒ यज॑मानो॒ यज॑मानो॒ ऽग्निना॒ वै वा अ॒ग्निना॒ यज॑मानः । \newline
40. अ॒ग्निना॒ यज॑मानो॒ यज॑मानो॒ ऽग्निना॒ ऽग्निना॒ यज॑मा॒नो ऽन्वनु॒ यज॑मानो॒ ऽग्निना॒ ऽग्निना॒ यज॑मा॒नो ऽनु॑ । \newline
41. यज॑मा॒नो ऽन्वनु॒ यज॑मानो॒ यज॑मा॒नो ऽनु॑ पश्यति पश्य॒त्यनु॒ यज॑मानो॒ यज॑मा॒नो ऽनु॑ पश्यति । \newline
42. अनु॑ पश्यति पश्य॒ त्यन्वनु॑ पश्यति पितृलो॒कम् पि॑तृलो॒कम् प॑श्य॒ त्यन्वनु॑ पश्यति पितृलो॒कम् । \newline
43. प॒श्य॒ति॒ पि॒तृ॒लो॒कम् पि॑तृलो॒कम् प॑श्यति पश्यति पितृलो॒कꣳ सोमे॑न॒ सोमे॑न पितृलो॒कम् प॑श्यति पश्यति पितृलो॒कꣳ सोमे॑न । \newline
44. पि॒तृ॒लो॒कꣳ सोमे॑न॒ सोमे॑न पितृलो॒कम् पि॑तृलो॒कꣳ सोमे॑नोत्तरा॒र्द्ध उ॑त्तरा॒र्द्धे सोमे॑न पितृलो॒कम् पि॑तृलो॒कꣳ सोमे॑नोत्तरा॒र्द्धे । \newline
45. पि॒तृ॒लो॒कमिति॑ पितृ - लो॒कम् । \newline
46. सोमे॑नोत्तरा॒र्द्ध उ॑त्तरा॒र्द्धे सोमे॑न॒ सोमे॑नोत्तरा॒र्द्धे᳚ ऽग्नये॒ ऽग्नय॑ उत्तरा॒र्द्धे सोमे॑न॒ सोमे॑नोत्तरा॒र्द्धे᳚ ऽग्नये᳚ । \newline
47. उ॒त्त॒रा॒र्द्धे᳚ ऽग्नये॒ ऽग्नय॑ उत्तरा॒र्द्ध उ॑त्तरा॒र्द्धे᳚ ऽग्नये॑ जुहोति जुहो त्य॒ग्नय॑ उत्तरा॒र्द्ध उ॑त्तरा॒र्द्धे᳚ ऽग्नये॑ जुहोति । \newline
48. उ॒त्त॒रा॒र्द्ध इत्यु॑त्तर - अ॒र्द्धे । \newline
49. अ॒ग्नये॑ जुहोति जुहो त्य॒ग्नये॒ ऽग्नये॑ जुहोति दक्षिणा॒र्द्धे द॑क्षिणा॒र्द्धे जु॑हो त्य॒ग्नये॒ ऽग्नये॑ जुहोति दक्षिणा॒र्द्धे । \newline
50. जु॒हो॒ति॒ द॒क्षि॒णा॒र्द्धे द॑क्षिणा॒र्द्धे जु॑होति जुहोति दक्षिणा॒र्द्धे सोमा॑य॒ सोमा॑य दक्षिणा॒र्द्धे जु॑होति जुहोति दक्षिणा॒र्द्धे सोमा॑य । \newline
51. द॒क्षि॒णा॒र्द्धे सोमा॑य॒ सोमा॑य दक्षिणा॒र्द्धे द॑क्षिणा॒र्द्धे सोमा॑यै॒व मे॒वꣳ सोमा॑य दक्षिणा॒र्द्धे द॑क्षिणा॒र्द्धे सोमा॑यै॒वम् । \newline
52. द॒क्षि॒णा॒र्द्ध इति॑ दक्षिण - अ॒र्द्धे । \newline
53. सोमा॑यै॒व मे॒वꣳ सोमा॑य॒ सोमा॑यै॒व मि॑वे वै॒वꣳ सोमा॑य॒ सोमा॑यै॒व मि॑व । \newline
54. ए॒व मि॑वे वै॒व मे॒व मि॑व॒ हि हीवै॒व मे॒व मि॑व॒ हि । \newline
55. इ॒व॒ हि हीवे॑ व॒ हीमा वि॒मौ हीवे॑ व॒ हीमौ । \newline
56. हीमा वि॒मौ हि हीमौ लो॒कौ लो॒का वि॒मौ हि हीमौ लो॒कौ । \newline
57. इ॒मौ लो॒कौ लो॒का वि॒मा वि॒मौ लो॒का व॒नयो॑ र॒नयो᳚र् लो॒का वि॒मा वि॒मौ लो॒का व॒नयोः᳚ । \newline
58. लो॒का व॒नयो॑ र॒नयो᳚र् लो॒कौ लो॒का व॒नयो᳚र् लो॒कयो᳚र् लो॒कयो॑ र॒नयो᳚र् लो॒कौ लो॒का व॒नयो᳚र् लो॒कयोः᳚ । \newline
59. अ॒नयो᳚र् लो॒कयो᳚र् लो॒कयो॑ र॒नयो॑ र॒नयो᳚र् लो॒कयो॒ रनु॑ख्यात्या॒ अनु॑ख्यात्यै लो॒कयो॑ र॒नयो॑ र॒नयो᳚र् लो॒कयो॒ रनु॑ख्यात्यै । \newline
60. लो॒कयो॒ रनु॑ख्यात्या॒ अनु॑ख्यात्यै लो॒कयो᳚र् लो॒कयो॒ रनु॑ख्यात्यै॒ राजा॑नौ॒ राजा॑ना॒ वनु॑ख्यात्यै लो॒कयो᳚र् लो॒कयो॒ रनु॑ख्यात्यै॒ राजा॑नौ । \newline
61. अनु॑ख्यात्यै॒ राजा॑नौ॒ राजा॑ना॒ वनु॑ख्यात्या॒ अनु॑ख्यात्यै॒ राजा॑नौ॒ वै वै राजा॑ना॒ वनु॑ख्यात्या॒ अनु॑ख्यात्यै॒ राजा॑नौ॒ वै । \newline
62. अनु॑ख्यात्या॒ इत्यनु॑ - ख्या॒त्यै॒ । \newline
63. राजा॑नौ॒ वै वै राजा॑नौ॒ राजा॑नौ॒ वा ए॒ता वे॒तौ वै राजा॑नौ॒ राजा॑नौ॒ वा ए॒तौ । \newline
64. वा ए॒ता वे॒तौ वै वा ए॒तौ दे॒वता॑नाम् दे॒वता॑ना मे॒तौ वै वा ए॒तौ दे॒वता॑नाम् । \newline
65. ए॒तौ दे॒वता॑नाम् दे॒वता॑ना मे॒ता वे॒तौ दे॒वता॑नां॒ ॅयद् यद् दे॒वता॑ना मे॒ता वे॒तौ दे॒वता॑नां॒ ॅयत् । \newline
66. दे॒वता॑नां॒ ॅयद् यद् दे॒वता॑नाम् दे॒वता॑नां॒ ॅयद॒ग्नीषोमा॑ व॒ग्नीषोमौ॒ यद् दे॒वता॑नाम् दे॒वता॑नां॒ ॅयद॒ग्नीषोमौ᳚ । \newline
\pagebreak
\markright{ TS 2.6.2.2  \hfill https://www.vedavms.in \hfill}
\addcontentsline{toc}{section}{ TS 2.6.2.2 }
\section*{ TS 2.6.2.2 }

\textbf{TS 2.6.2.2 } \newline
\textbf{Samhita Paata} \newline

ॅयद॒ग्नीषोमा॑वन्त॒रा दे॒वता॑ इज्येते दे॒वता॑नां॒ ॅविधृ॑त्यै॒ तस्मा॒द्-राज्ञा॑ मनु॒ष्या॑ विधृ॑ता ब्रह्मवा॒दिनो॑ वदन्ति॒ किं तद्-य॒ज्ञे यज॑मानः कुरुते॒ येना॒न्यतो॑दतश्च प॒शून् दा॒धा-रो॑भ॒यतो॑दत॒-श्चेत्यृच॑-म॒नूच्या ऽऽज्य॑भागस्य जुषा॒णेन॑ यजति॒ तेना॒न्यतो॑दतो दाधा॒रर्च॑म॒नूच्य॑ ह॒विष॑ ऋ॒चा य॑जति॒ तेनो॑भ॒यतो॑दतो दाधार मूर्द्ध॒न्वती॑ पुरोऽनुवा॒क्या॑ भवति मू॒र्द्धान॑मे॒वैनꣳ॑ समा॒नानां᳚ करोति - [  ] \newline

\textbf{Pada Paata} \newline

यत् । अ॒ग्नीषोमा॒वित्य॒ग्नी-सोमौ᳚ । अ॒न्त॒रा । दे॒वताः᳚ । इ॒ज्ये॒ते॒ इति॑ । दे॒वता॑नाम् । विधृ॑त्या॒ इति॒ वि - धृ॒त्यै॒ । तस्मा᳚त् । राज्ञा᳚ । म॒नु॒ष्याः᳚ । विधृ॑ता॒ इति॒ वि - धृ॒ताः॒ । ब्र॒ह्म॒वा॒दिन॒ इति॑ ब्रह्म-वा॒दिनः॑ । व॒द॒न्ति॒ । किम् । तत् । य॒ज्ञे । यज॑मानः । कु॒रु॒ते॒ । येन॑ । अ॒न्यतो॑दत॒ इत्य॒न्यतः॑ - द॒तः॒ । च॒ । प॒शून् । दा॒धार॑ । उ॒भ॒यतो॑दत॒ इत्यु॑भ॒यतः॑-द॒तः॒ । च॒ । इति॑ । ऋच᳚म् । अ॒नूच्येत्य॑नु - उच्य॑ । आज्य॑भाग॒स्येत्याज्य॑ - भा॒ग॒स्य॒ । जु॒षा॒णेन॑ । य॒ज॒ति॒ । तेन॑ । अ॒न्यतो॑दत॒ इत्य॒न्यतः॑ - द॒तः॒ । दा॒धा॒र॒ । ऋच᳚म् । अ॒नूच्येत्य॑नु - उच्य॑ । ह॒विषः॑ । ऋ॒चा । य॒ज॒ति॒ । तेन॑ । उ॒भ॒यतो॑दत॒ इत्यु॑भ॒यतः॑ - द॒तः॒ । दा॒धा॒र॒ । मू॒र्द्ध॒न्वतीति॑ मूर्द्धन्न् - वती᳚ । पु॒रो॒नु॒वा॒क्येति॑ पुरः-अ॒नु॒वा॒क्या᳚ । भ॒व॒ति॒ । मू॒र्द्धान᳚म् । ए॒व । ए॒न॒म् । स॒मा॒नाना᳚म् । क॒रो॒ति॒ ।  \newline


\textbf{Krama Paata} \newline

यद॒ग्नीषोमौ᳚ । अ॒ग्नीषोमा॑वन्त॒रा । अ॒ग्नीषोमा॒वित्य॒ग्नी - सोमौ᳚ । अ॒न्त॒रा दे॒वताः᳚ । दे॒वता॑ इज्येते । इ॒ज्ये॒ते॒ दे॒वता॑नाम् । इ॒ज्ये॒ते॒ इती᳚ज्येते । दे॒वता॑नां॒ ॅविधृ॑त्यै । विधृ॑त्यै॒ तस्मा᳚त् । विधृ॑त्या॒ इति॒ वि - धृ॒त्यै॒ । तस्मा॒द् राज्ञा᳚ । राज्ञा॑ मनु॒ष्याः᳚ । म॒नु॒ष्या॑ विधृ॑ताः । विधृ॑ता ब्रह्मवा॒दिनः॑ । विधृ॑ता॒ इति॒ वि - धृ॒ताः॒ । ब्र॒ह्म॒वा॒दिनो॑ वदन्ति । ब॒ह्म॒वा॒दिन॒ इति॑ ब्रह्म - वा॒दिनः॑ । व॒द॒न्ति॒ किम् । किम् तत् । तद् य॒ज्ञे । य॒ज्ञे यज॑मानः । यज॑मानः कुरुते । कु॒रु॒ते॒ येन॑ । येना॒न्यतो॑दतः । अ॒न्यतो॑दतश्च । अ॒न्यतो॑दत॒ इत्य॒न्यतः॑ - द॒तः॒ । च॒ प॒शून् । प॒शून् दा॒धार॑ । दा॒धारो॑भ॒यतो॑दतः । उ॒भ॒यतो॑दतश्च । उ॒भ॒यतो॑दत॒ इत्यु॑भ॒यतः॑ - द॒तः॒ । चेति॑ । इत्यृच᳚म् । ऋच॑म॒नूच्य॑ । अ॒नूच्याज्य॑भागस्य । अ॒नूचेत्य॑नु - उच्य॑ । आज्य॑भागस्य जुषा॒णेन॑ । आज्य॑भाग॒स्येत्याज्य॑ - भा॒ग॒स्य॒ । जु॒षा॒णेन॑ यजति । य॒ज॒ति॒ तेन॑ । तेना॒न्यतो॑दतः । अ॒न्यतो॑दतो दाधार । अ॒न्यतो॑दत॒ इत्य॒न्यतः॑ - द॒तः॒ । दा॒धा॒रर्च᳚म् । ऋच॑म॒नूच्य॑ । अ॒नूच्य॑ ह॒विषः॑ । अ॒नूच्येत्य॑नु - उच्य॑ । ह॒विष॑ ऋ॒चा । ऋ॒चा य॑जति । य॒ज॒ति॒ तेन॑ । तेनो॑भ॒यतो॑दतः । उ॒भ॒यतो॑दतो दाधार । उ॒भयतो॑दत॒ इत्यु॑भ॒यतः॑ - द॒तः॒ । दा॒धा॒र॒ मू॒र्द्ध॒न्वती᳚ । मू॒र्द्ध॒न्वती॑ पुरोनुवा॒क्या᳚ । मू॒र्द्ध॒न्वतीति॑ मूर्द्धन्न् - वती᳚ । पु॒रो॒नु॒वा॒क्या॑ भवति । पु॒रो॒नु॒वा॒क्येति॑ पुरः - अ॒नु॒वा॒क्या᳚ । भ॒व॒ति॒ मू॒र्द्धान᳚म् । मू॒र्द्धान॑मे॒व । ए॒वैन᳚म् । ए॒नꣳ॒॒ स॒मा॒नाना᳚म् । स॒मा॒नाना᳚म् करोति । क॒रो॒ति॒ नि॒युत्व॑त्या \newline

\textbf{Jatai Paata} \newline

1. यद॒ग्नीषोमा॑ व॒ग्नीषोमौ॒ यद् यद॒ग्नीषोमौ᳚ । \newline
2. अ॒ग्नीषोमा॑ वन्त॒रा ऽन्त॒रा ऽग्नीषोमा॑ व॒ग्नीषोमा॑ वन्त॒रा । \newline
3. अ॒ग्नीषोमा॒वित्य॒ग्नी - सोमौ᳚ । \newline
4. अ॒न्त॒रा दे॒वता॑ दे॒वता॑ अन्त॒रा ऽन्त॒रा दे॒वताः᳚ । \newline
5. दे॒वता॑ इज्येते इज्येते दे॒वता॑ दे॒वता॑ इज्येते । \newline
6. इ॒ज्ये॒ते॒ दे॒वता॑नाम् दे॒वता॑ना मिज्येते इज्येते दे॒वता॑नाम् । \newline
7. इ॒ज्ये॒ते॒ इती᳚ज्येते । \newline
8. दे॒वता॑नां॒ ॅविधृ॑त्यै॒ विधृ॑त्यै दे॒वता॑नाम् दे॒वता॑नां॒ ॅविधृ॑त्यै । \newline
9. विधृ॑त्यै॒ तस्मा॒त् तस्मा॒द् विधृ॑त्यै॒ विधृ॑त्यै॒ तस्मा᳚त् । \newline
10. विधृ॑त्या॒ इति॒ वि - धृ॒त्यै॒ । \newline
11. तस्मा॒द् राज्ञा॒ राज्ञा॒ तस्मा॒त् तस्मा॒द् राज्ञा᳚ । \newline
12. राज्ञा॑ मनु॒ष्या॑ मनु॒ष्या॑ राज्ञा॒ राज्ञा॑ मनु॒ष्याः᳚ । \newline
13. म॒नु॒ष्या॑ विधृ॑ता॒ विधृ॑ता मनु॒ष्या॑ मनु॒ष्या॑ विधृ॑ताः । \newline
14. विधृ॑ता ब्रह्मवा॒दिनो᳚ ब्रह्मवा॒दिनो॒ विधृ॑ता॒ विधृ॑ता ब्रह्मवा॒दिनः॑ । \newline
15. विधृ॑ता॒ इति॒ वि - धृ॒ताः॒ । \newline
16. ब्र॒ह्म॒वा॒दिनो॑ वदन्ति वदन्ति ब्रह्मवा॒दिनो᳚ ब्रह्मवा॒दिनो॑ वदन्ति । \newline
17. ब्र॒ह्म॒वा॒दिन॒ इति॑ ब्रह्म - वा॒दिनः॑ । \newline
18. व॒द॒न्ति॒ किम् किं ॅव॑दन्ति वदन्ति॒ किम् । \newline
19. किम् तत् तत् किम् किम् तत् । \newline
20. तद् य॒ज्ञे य॒ज्ञे तत् तद् य॒ज्ञे । \newline
21. य॒ज्ञे यज॑मानो॒ यज॑मानो य॒ज्ञे य॒ज्ञे यज॑मानः । \newline
22. यज॑मानः कुरुते कुरुते॒ यज॑मानो॒ यज॑मानः कुरुते । \newline
23. कु॒रु॒ते॒ येन॒ येन॑ कुरुते कुरुते॒ येन॑ । \newline
24. येना॒ न्यतो॑दतो॒ ऽन्यतो॑दतो॒ येन॒ येना॒ न्यतो॑दतः । \newline
25. अ॒न्यतो॑दतश्च चा॒न्यतो॑दतो॒ ऽन्यतो॑दतश्च । \newline
26. अ॒न्यतो॑दत॒ इत्य॒न्यतः॑ - द॒तः॒ । \newline
27. च॒ प॒शून् प॒शूꣳश्च॑ च प॒शून् । \newline
28. प॒शून् दा॒धार॑ दा॒धार॑ प॒शून् प॒शून् दा॒धार॑ । \newline
29. दा॒धा रो॑भ॒यतो॑दत उभ॒यतो॑दतो दा॒धार॑ दा॒धा रो॑भ॒यतो॑दतः । \newline
30. उ॒भ॒यतो॑दतश्च चोभ॒यतो॑दत उभ॒यतो॑दतश्च । \newline
31. उ॒भ॒यतो॑दत॒ इत्यु॑भ॒यतः॑ - द॒तः॒ । \newline
32. चे तीति॑ च॒ चे ति॑ । \newline
33. इत्यृच॒ मृच॒ मिती त्यृच᳚म् । \newline
34. ऋच॑ म॒नूच्या॒ नूच्य र्च॒ मृच॑ म॒नूच्य॑ । \newline
35. अ॒नूच्या ज्य॑भाग॒स्या ज्य॑भागस्या॒ नूच्या॒ नूच्या ज्य॑भागस्य । \newline
36. अ॒नूच्येत्य॑नु - उच्य॑ । \newline
37. आज्य॑भागस्य जुषा॒णेन॑ जुषा॒णेना ज्य॑भाग॒स्या ज्य॑भागस्य जुषा॒णेन॑ । \newline
38. आज्य॑भाग॒स्येत्याज्य॑ - भा॒ग॒स्य॒ । \newline
39. जु॒षा॒णेन॑ यजति यजति जुषा॒णेन॑ जुषा॒णेन॑ यजति । \newline
40. य॒ज॒ति॒ तेन॒ तेन॑ यजति यजति॒ तेन॑ । \newline
41. तेना॒न्यतो॑दतो॒ ऽन्यतो॑दत॒ स्तेन॒ तेना॒न्यतो॑दतः । \newline
42. अ॒न्यतो॑दतो दाधार दाधारा॒ न्यतो॑दतो॒ ऽन्यतो॑दतो दाधार । \newline
43. अ॒न्यतो॑दत॒ इत्य॒न्यतः॑ - द॒तः॒ । \newline
44. दा॒धा॒र र्च॒ मृच॑म् दाधार दाधा॒र र्च᳚म् । \newline
45. ऋच॑ म॒नूच्या॒ नूच्य र्च॒ मृच॑ म॒नूच्य॑ । \newline
46. अ॒नूच्य॑ ह॒विषो॑ ह॒विषो॒ ऽनूच्या॒ नूच्य॑ ह॒विषः॑ । \newline
47. अ॒नूच्येत्य॑नु - उच्य॑ । \newline
48. ह॒विष॑ ऋ॒चर्चा ह॒विषो॑ ह॒विष॑ ऋ॒चा । \newline
49. ऋ॒चा य॑जति यज त्यृ॒चर्चा य॑जति । \newline
50. य॒ज॒ति॒ तेन॒ तेन॑ यजति यजति॒ तेन॑ । \newline
51. तेनो॑भ॒यतो॑दत उभ॒यतो॑दत॒ स्तेन॒ तेनो॑भ॒यतो॑दतः । \newline
52. उ॒भ॒यतो॑दतो दाधार दाधा रोभ॒यतो॑दत उभ॒यतो॑दतो दाधार । \newline
53. उ॒भ॒यतो॑दत॒ इत्यु॑भ॒यतः॑ - द॒तः॒ । \newline
54. दा॒धा॒र॒ मू॒र्द्ध॒न्वती॑ मूर्द्ध॒न्वती॑ दाधार दाधार मूर्द्ध॒न्वती᳚ । \newline
55. मू॒र्द्ध॒न्वती॑ पुरोनुवा॒क्या॑ पुरोनुवा॒क्या॑ मूर्द्ध॒न्वती॑ मूर्द्ध॒न्वती॑ पुरोनुवा॒क्या᳚ । \newline
56. मू॒र्द्ध॒न्वतीति॑ मूर्द्धन्न् - वती᳚ । \newline
57. पु॒रो॒नु॒वा॒क्या॑ भवति भवति पुरोनुवा॒क्या॑ पुरोनुवा॒क्या॑ भवति । \newline
58. पु॒रो॒नु॒वा॒क्येति॑ पुरः - अ॒नु॒वा॒क्या᳚ । \newline
59. भ॒व॒ति॒ मू॒र्द्धान॑म् मू॒र्द्धान॑म् भवति भवति मू॒र्द्धान᳚म् । \newline
60. मू॒र्द्धान॑ मे॒वैव मू॒र्द्धान॑म् मू॒र्द्धान॑ मे॒व । \newline
61. ए॒वैन॑ मेन मे॒वैवैन᳚म् । \newline
62. ए॒नꣳ॒॒ स॒मा॒नानाꣳ॑ समा॒नाना॑ मेन मेनꣳ समा॒नाना᳚म् । \newline
63. स॒मा॒नाना᳚म् करोति करोति समा॒नानाꣳ॑ समा॒नाना᳚म् करोति । \newline
64. क॒रो॒ति॒ नि॒युत्व॑त्या नि॒युत्व॑त्या करोति करोति नि॒युत्व॑त्या । \newline

\textbf{Ghana Paata } \newline

1. यद॒ग्नीषोमा॑ व॒ग्नीषोमौ॒ यद् यद॒ग्नीषोमा॑ वन्त॒रा ऽन्त॒रा ऽग्नीषोमौ॒ यद् यद॒ग्नीषोमा॑ वन्त॒रा । \newline
2. अ॒ग्नीषोमा॑ वन्त॒रा ऽन्त॒रा ऽग्नीषोमा॑ व॒ग्नीषोमा॑ वन्त॒रा दे॒वता॑ दे॒वता॑ अन्त॒रा ऽग्नीषोमा॑ व॒ग्नीषोमा॑ वन्त॒रा दे॒वताः᳚ । \newline
3. अ॒ग्नीषोमा॒वित्य॒ग्नी - सोमौ᳚ । \newline
4. अ॒न्त॒रा दे॒वता॑ दे॒वता॑ अन्त॒रा ऽन्त॒रा दे॒वता॑ इज्येते इज्येते दे॒वता॑ अन्त॒रा ऽन्त॒रा दे॒वता॑ इज्येते । \newline
5. दे॒वता॑ इज्येते इज्येते दे॒वता॑ दे॒वता॑ इज्येते दे॒वता॑नाम् दे॒वता॑ना मिज्येते दे॒वता॑ दे॒वता॑ इज्येते दे॒वता॑नाम् । \newline
6. इ॒ज्ये॒ते॒ दे॒वता॑नाम् दे॒वता॑ना मिज्येते इज्येते दे॒वता॑नां॒ ॅविधृ॑त्यै॒ विधृ॑त्यै दे॒वता॑ना मिज्येते इज्येते दे॒वता॑नां॒ ॅविधृ॑त्यै । \newline
7. इ॒ज्ये॒ते॒ इती᳚ज्येते । \newline
8. दे॒वता॑नां॒ ॅविधृ॑त्यै॒ विधृ॑त्यै दे॒वता॑नाम् दे॒वता॑नां॒ ॅविधृ॑त्यै॒ तस्मा॒त् तस्मा॒द् विधृ॑त्यै दे॒वता॑नाम् दे॒वता॑नां॒ ॅविधृ॑त्यै॒ तस्मा᳚त् । \newline
9. विधृ॑त्यै॒ तस्मा॒त् तस्मा॒द् विधृ॑त्यै॒ विधृ॑त्यै॒ तस्मा॒द् राज्ञा॒ राज्ञा॒ तस्मा॒द् विधृ॑त्यै॒ विधृ॑त्यै॒ तस्मा॒द् राज्ञा᳚ । \newline
10. विधृ॑त्या॒ इति॒ वि - धृ॒त्यै॒ । \newline
11. तस्मा॒द् राज्ञा॒ राज्ञा॒ तस्मा॒त् तस्मा॒द् राज्ञा॑ मनु॒ष्या॑ मनु॒ष्या॑ राज्ञा॒ तस्मा॒त् तस्मा॒द् राज्ञा॑ मनु॒ष्याः᳚ । \newline
12. राज्ञा॑ मनु॒ष्या॑ मनु॒ष्या॑ राज्ञा॒ राज्ञा॑ मनु॒ष्या॑ विधृ॑ता॒ विधृ॑ता मनु॒ष्या॑ राज्ञा॒ राज्ञा॑ मनु॒ष्या॑ विधृ॑ताः । \newline
13. म॒नु॒ष्या॑ विधृ॑ता॒ विधृ॑ता मनु॒ष्या॑ मनु॒ष्या॑ विधृ॑ता ब्रह्मवा॒दिनो᳚ ब्रह्मवा॒दिनो॒ विधृ॑ता मनु॒ष्या॑ मनु॒ष्या॑ विधृ॑ता ब्रह्मवा॒दिनः॑ । \newline
14. विधृ॑ता ब्रह्मवा॒दिनो᳚ ब्रह्मवा॒दिनो॒ विधृ॑ता॒ विधृ॑ता ब्रह्मवा॒दिनो॑ वदन्ति वदन्ति ब्रह्मवा॒दिनो॒ विधृ॑ता॒ विधृ॑ता ब्रह्मवा॒दिनो॑ वदन्ति । \newline
15. विधृ॑ता॒ इति॒ वि - धृ॒ताः॒ । \newline
16. ब्र॒ह्म॒वा॒दिनो॑ वदन्ति वदन्ति ब्रह्मवा॒दिनो᳚ ब्रह्मवा॒दिनो॑ वदन्ति॒ किम् किं ॅव॑दन्ति ब्रह्मवा॒दिनो᳚ ब्रह्मवा॒दिनो॑ वदन्ति॒ किम् । \newline
17. ब्र॒ह्म॒वा॒दिन॒ इति॑ ब्रह्म - वा॒दिनः॑ । \newline
18. व॒द॒न्ति॒ किम् किं ॅव॑दन्ति वदन्ति॒ किम् तत् तत् किं ॅव॑दन्ति वदन्ति॒ किम् तत् । \newline
19. किम् तत् तत् किम् किम् तद् य॒ज्ञे य॒ज्ञे तत् किम् किम् तद् य॒ज्ञे । \newline
20. तद् य॒ज्ञे य॒ज्ञे तत् तद् य॒ज्ञे यज॑मानो॒ यज॑मानो य॒ज्ञे तत् तद् य॒ज्ञे यज॑मानः । \newline
21. य॒ज्ञे यज॑मानो॒ यज॑मानो य॒ज्ञे य॒ज्ञे यज॑मानः कुरुते कुरुते॒ यज॑मानो य॒ज्ञे य॒ज्ञे यज॑मानः कुरुते । \newline
22. यज॑मानः कुरुते कुरुते॒ यज॑मानो॒ यज॑मानः कुरुते॒ येन॒ येन॑ कुरुते॒ यज॑मानो॒ यज॑मानः कुरुते॒ येन॑ । \newline
23. कु॒रु॒ते॒ येन॒ येन॑ कुरुते कुरुते॒ येना॒ न्यतो॑दतो॒ ऽन्यतो॑दतो॒ येन॑ कुरुते कुरुते॒ येना॒ न्यतो॑दतः । \newline
24. येना॒ न्यतो॑दतो॒ ऽन्यतो॑दतो॒ येन॒ येना॒ न्यतो॑दतश्च चा॒ न्यतो॑दतो॒ येन॒ येना॒ न्यतो॑दतश्च । \newline
25. अ॒न्यतो॑दतश्च चा॒ न्यतो॑दतो॒ ऽन्यतो॑दतश्च प॒शून् प॒शूꣳ श्चा॒न्यतो॑दतो॒ ऽन्यतो॑दतश्च प॒शून् । \newline
26. अ॒न्यतो॑दत॒ इत्य॒न्यतः॑ - द॒तः॒ । \newline
27. च॒ प॒शून् प॒शूꣳ श्च॑ च प॒शून् दा॒धार॑ दा॒धार॑ प॒शूꣳ श्च॑ च प॒शून् दा॒धार॑ । \newline
28. प॒शून् दा॒धार॑ दा॒धार॑ प॒शून् प॒शून् दा॒धारो॑ भ॒यतो॑दत उभ॒यतो॑दतो दा॒धार॑ प॒शून् प॒शून् दा॒धारो॑ भ॒यतो॑दतः । \newline
29. दा॒धारो॑ भ॒यतो॑दत उभ॒यतो॑दतो दा॒धार॑ दा॒धारो॑ भ॒यतो॑दतश्च चोभ॒यतो॑दतो दा॒धार॑ दा॒धारो॑ भ॒यतो॑दतश्च । \newline
30. उ॒भ॒यतो॑दतश्च चोभ॒यतो॑दत उभ॒यतो॑दत॒श्चे तीति॑ चोभ॒यतो॑दत उभ॒यतो॑दत॒श्चे ति॑ । \newline
31. उ॒भ॒यतो॑दत॒ इत्यु॑भ॒यतः॑ - द॒तः॒ । \newline
32. चे तीति॑ च॒ चे त्यृच॒ मृच॒ मिति॑ च॒ चे त्यृच᳚म् । \newline
33. इत्यृच॒ मृच॒ मितीत्यृच॑ म॒नूच्या॒ नूच्य र्च॒ मितीत्यृच॑ म॒नूच्य॑ । \newline
34. ऋच॑ म॒नूच्या॒ नूच्य र्च॒ मृच॑ म॒नूच्या ज्य॑भाग॒स्या ज्य॑भागस्या॒ नूच्य र्च॒ मृच॑ म॒नूच्या ज्य॑भागस्य । \newline
35. अ॒नू च्याज्य॑भाग॒स्या ज्य॑भागस्या॒ नूच्या॒ नूच्याज्य॑भागस्य जुषा॒णेन॑ जुषा॒णेना ज्य॑भागस्या॒ नूच्या॒ नूच्याज्य॑भागस्य जुषा॒णेन॑ । \newline
36. अ॒नूच्येत्य॑नु - उच्य॑ । \newline
37. आज्य॑भागस्य जुषा॒णेन॑ जुषा॒णे नाज्य॑भाग॒स्या ज्य॑भागस्य जुषा॒णेन॑ यजति यजति जुषा॒णे नाज्य॑भाग॒स्या ज्य॑भागस्य जुषा॒णेन॑ यजति । \newline
38. आज्य॑भाग॒स्येत्याज्य॑ - भा॒ग॒स्य॒ । \newline
39. जु॒षा॒णेन॑ यजति यजति जुषा॒णेन॑ जुषा॒णेन॑ यजति॒ तेन॒ तेन॑ यजति जुषा॒णेन॑ जुषा॒णेन॑ यजति॒ तेन॑ । \newline
40. य॒ज॒ति॒ तेन॒ तेन॑ यजति यजति॒ तेना॒ न्यतो॑दतो॒ ऽन्यतो॑दत॒ स्तेन॑ यजति यजति॒ तेना॒ न्यतो॑दतः । \newline
41. तेना॒ न्यतो॑दतो॒ ऽन्यतो॑दत॒ स्तेन॒ तेना॒ न्यतो॑दतो दाधार दाधारा॒ न्यतो॑दत॒ स्तेन॒ तेना॒ न्यतो॑दतो दाधार । \newline
42. अ॒न्यतो॑दतो दाधार दाधारा॒ न्यतो॑दतो॒ ऽन्यतो॑दतो दाधा॒र र्च॒ मृच॑म् दाधारा॒ न्यतो॑दतो॒ ऽन्यतो॑दतो दाधा॒र र्च᳚म् । \newline
43. अ॒न्यतो॑दत॒ इत्य॒न्यतः॑ - द॒तः॒ । \newline
44. दा॒धा॒र र्च॒ मृच॑म् दाधार दाधा॒र र्च॑ म॒नूच्या॒ नूच्य र्च॑म् दाधार दाधा॒र र्च॑ म॒नूच्य॑ । \newline
45. ऋच॑ म॒नूच्या॒ नूच्य र्च॒ मृच॑ म॒नूच्य॑ ह॒विषो॑ ह॒विषो॒ ऽनूच्य र्च॒ मृच॑ म॒नूच्य॑ ह॒विषः॑ । \newline
46. अ॒नूच्य॑ ह॒विषो॑ ह॒विषो॒ ऽनूच्या॒ नूच्य॑ ह॒विष॑ ऋ॒चर्चा ह॒विषो॒ ऽनूच्या॒ नूच्य॑ ह॒विष॑ ऋ॒चा । \newline
47. अ॒नूच्येत्य॑नु - उच्य॑ । \newline
48. ह॒विष॑ ऋ॒चर्चा ह॒विषो॑ ह॒विष॑ ऋ॒चा य॑जति यजत्यृ॒चा ह॒विषो॑ ह॒विष॑ ऋ॒चा य॑जति । \newline
49. ऋ॒चा य॑जति यज त्यृ॒चर्चा य॑जति॒ तेन॒ तेन॑ यज त्यृ॒चर्चा य॑जति॒ तेन॑ । \newline
50. य॒ज॒ति॒ तेन॒ तेन॑ यजति यजति॒ तेनो॑भ॒यतो॑दत उभ॒यतो॑दत॒ स्तेन॑ यजति यजति॒ तेनो॑भ॒यतो॑दतः । \newline
51. तेनो॑भ॒यतो॑दत उभ॒यतो॑दत॒ स्तेन॒ तेनो॑भ॒यतो॑दतो दाधार दाधारो भ॒यतो॑दत॒ स्तेन॒ तेनो॑भ॒यतो॑दतो दाधार । \newline
52. उ॒भ॒यतो॑दतो दाधार दाधारो भ॒यतो॑दत उभ॒यतो॑दतो दाधार मूर्द्ध॒न्वती॑ मूर्द्ध॒न्वती॑ दाधारो भ॒यतो॑दत उभ॒यतो॑दतो दाधार मूर्द्ध॒न्वती᳚ । \newline
53. उ॒भ॒यतो॑दत॒ इत्यु॑भ॒यतः॑ - द॒तः॒ । \newline
54. दा॒धा॒र॒ मू॒र्द्ध॒न्वती॑ मूर्द्ध॒न्वती॑ दाधार दाधार मूर्द्ध॒न्वती॑ पुरोनुवा॒क्या॑ पुरोनुवा॒क्या॑ मूर्द्ध॒न्वती॑ दाधार दाधार मूर्द्ध॒न्वती॑ पुरोनुवा॒क्या᳚ । \newline
55. मू॒र्द्ध॒न्वती॑ पुरोनुवा॒क्या॑ पुरोनुवा॒क्या॑ मूर्द्ध॒न्वती॑ मूर्द्ध॒न्वती॑ पुरोनुवा॒क्या॑ भवति भवति पुरोनुवा॒क्या॑ मूर्द्ध॒न्वती॑ मूर्द्ध॒न्वती॑ पुरोनुवा॒क्या॑ भवति । \newline
56. मू॒र्द्ध॒न्वतीति॑ मूर्द्धन्न् - वती᳚ । \newline
57. पु॒रो॒नु॒वा॒क्या॑ भवति भवति पुरोनुवा॒क्या॑ पुरोनुवा॒क्या॑ भवति मू॒र्द्धान॑म् मू॒र्द्धान॑म् भवति पुरोनुवा॒क्या॑ पुरोनुवा॒क्या॑ भवति मू॒र्द्धान᳚म् । \newline
58. पु॒रो॒नु॒वा॒क्येति॑ पुरः - अ॒नु॒वा॒क्या᳚ । \newline
59. भ॒व॒ति॒ मू॒र्द्धान॑म् मू॒र्द्धान॑म् भवति भवति मू॒र्द्धान॑ मे॒वैव मू॒र्द्धान॑म् भवति भवति मू॒र्द्धान॑ मे॒व । \newline
60. मू॒र्द्धान॑ मे॒वैव मू॒र्द्धान॑म् मू॒र्द्धान॑ मे॒वैन॑ मेन मे॒व मू॒र्द्धान॑म् मू॒र्द्धान॑ मे॒वैन᳚म् । \newline
61. ए॒वैन॑ मेन मे॒वैवैनꣳ॑ समा॒नानाꣳ॑ समा॒नाना॑ मेन मे॒वैवैनꣳ॑ समा॒नाना᳚म् । \newline
62. ए॒नꣳ॒॒ स॒मा॒नानाꣳ॑ समा॒नाना॑ मेन मेनꣳ समा॒नाना᳚म् करोति करोति समा॒नाना॑ मेन मेनꣳ समा॒नाना᳚म् करोति । \newline
63. स॒मा॒नाना᳚म् करोति करोति समा॒नानाꣳ॑ समा॒नाना᳚म् करोति नि॒युत्व॑त्या नि॒युत्व॑त्या करोति समा॒नानाꣳ॑ समा॒नाना᳚म् करोति नि॒युत्व॑त्या । \newline
64. क॒रो॒ति॒ नि॒युत्व॑त्या नि॒युत्व॑त्या करोति करोति नि॒युत्व॑त्या यजति यजति नि॒युत्व॑त्या करोति करोति नि॒युत्व॑त्या यजति । \newline
\pagebreak
\markright{ TS 2.6.2.3  \hfill https://www.vedavms.in \hfill}
\addcontentsline{toc}{section}{ TS 2.6.2.3 }
\section*{ TS 2.6.2.3 }

\textbf{TS 2.6.2.3 } \newline
\textbf{Samhita Paata} \newline

नि॒युत्व॑त्या यजति॒ भ्रातृ॑व्यस्यै॒व प॒शून् नि यु॑वते के॒शिनꣳ॑ह दा॒र्भ्यं के॒शी सात्य॑कामिरुवाच स॒प्तप॑दां ते॒ शक्व॑रीꣳ॒॒ श्वो य॒ज्ञे प्र॑यो॒क्तासे॒ यस्यै॑ वी॒र्ये॑ण॒ प्र जा॒तान् भ्रातृ॑व्यान्नु॒दते॒ प्रति॑ जनि॒ष्यमा॑णा॒न॒. यस्यै॑ वी॒र्ये॑णो॒भयो᳚ र्लो॒कयो॒ र्ज्योति॑ र्द्ध॒त्ते यस्यै॑ वी॒र्ये॑ण पूर्वा॒र्द्धेना॑न॒ड्वान् भु॒नक्ति॑ जघना॒र्द्धेन॑ धे॒नुरिति॑ पु॒रस्ता᳚ल्लक्ष्मा पुरोऽनुवा॒क्या॑ भवति जा॒ताने॒व भ्रातृ॑व्या॒न् प्रणु॑दत उ॒परि॑ष्टाल्लक्ष्मा - [  ] \newline

\textbf{Pada Paata} \newline

नि॒युत्व॒त्येति॑ नि - युत्व॑त्या । य॒ज॒ति॒ । भ्रातृ॑व्यस्य । ए॒व । प॒शून् । नीति॑ । यु॒व॒ते॒ । के॒शिन᳚म् । ह॒ । दा॒र्भ्यम् । के॒शी । सात्य॑कामि॒रिति॒ सात्य॑ - का॒मिः॒ । उ॒वा॒च॒ । स॒प्तप॑दा॒मिति॑ स॒प्त - प॒दा॒म् । ते॒ । शक्व॑रीम् । श्वः । य॒ज्ञे । प्र॒यो॒क्तास॒ इति॑ प्र-यो॒क्तासे᳚ । यस्यै᳚ । वी॒र्ये॑ण । प्रेति॑ । जा॒तान् । भ्रातृ॑व्यान् । नु॒दते᳚ । प्रतीति॑ । ज॒नि॒ष्यमा॑णान् । यस्यै᳚ । वी॒र्ये॑ण । उ॒भयोः᳚ । लो॒कयोः᳚ । ज्योतिः॑ । ध॒त्ते । यस्यै᳚ । वी॒र्ये॑ण । पू॒र्वा॒र्द्धेनेति॑ पूर्व - अ॒र्द्धेन॑ । अ॒न॒ड्वान् । भु॒नक्ति॑ । ज॒घ॒ना॒र्द्धेनेति॑ जघन - अ॒र्द्धेन॑ । धे॒नुः । इति॑ । पु॒रस्ता᳚ल्ल॒क्ष्मेति॑ पु॒रस्ता᳚त् - ल॒क्ष्मा॒ । पु॒रो॒नु॒वा॒क्येति॑ पुरः - अ॒नु॒वा॒क्या᳚ । भ॒व॒ति॒ । जा॒तान् । ए॒व । भ्रातृ॑व्यान् । प्रेति॑ । नु॒द॒ते॒ । उ॒परि॑ष्टाल्ल॒क्ष्मेत्यु॒परि॑ष्टात् - ल॒क्ष्मा॒ ।  \newline


\textbf{Krama Paata} \newline

नि॒युत्व॑त्या यजति । नि॒युत्व॒त्येति॑ नि - युत्व॑त्या । य॒ज॒ति॒ भ्रातृ॑व्यस्य । भ्रातृ॑व्यस्यै॒व । ए॒व प॒शून् । प॒शून् नि । नि यु॑वते । यु॒व॒ते॒ के॒शिन᳚म् । के॒शिनꣳ॑ ह । ह॒ दा॒र्भ्यम् । दा॒र्भ्यम् के॒शी । के॒शी सात्य॑कामिः । सात्य॑कामिरुवाच । सात्य॑कामि॒रिति॒ सात्य॑ - का॒मिः॒ । उ॒वा॒च॒ स॒प्तप॑दाम् । स॒प्तप॑दाम् ते । स॒प्तप॑दा॒मिति॑ स॒प्त - प॒दा॒म् । ते॒ शक्व॑रीम् । शक्व॑रीꣳ॒॒ श्वः । श्वो य॒ज्ञे । य॒ज्ञे प्र॑यो॒क्तासे᳚ । प्र॒यो॒क्तासे॒ यस्यै᳚ । प्र॒यो॒क्तास॒ इति॑ प्र - यो॒क्तासे᳚ । यस्यै॑ वी॒र्ये॑ण । वी॒र्ये॑ण॒ प्र । प्र जा॒तान् । जा॒तान् भ्रातृ॑व्यान् । भ्रातृ॑व्यान् नु॒दते᳚ । नु॒दते॒ प्रति॑ । प्रति॑ जनि॒ष्यमा॑णान् । ज॒नि॒ष्यमा॑णा॒न्.॒ यस्यै᳚ । यस्यै॑ वी॒र्ये॑ण । वी॒र्ये॑णो॒भयोः᳚ । उ॒भयो᳚र् लो॒कयोः᳚ । लो॒कयो॒र् ज्योतिः॑ । ज्योति॑र् ध॒त्ते । ध॒त्ते यस्यै᳚ । यस्यै॑ वी॒र्ये॑ण । वी॒र्ये॑ण पूर्वा॒र्द्धेन॑ । पू॒र्वा॒र्द्धेना॑न॒ड्वान् । पू॒र्वा॒र्द्धेनेति॑ पूर्व - अ॒र्द्धेन॑ । अ॒न॒ड्वान् भु॒नक्ति॑ । भु॒नक्ति॑ जघना॒र्द्धेन॑ । ज॒घ॒ना॒र्द्धेन॑ धे॒नुः । ज॒घ॒ना॒र्द्धेनेति॑ जघन - अ॒र्द्धेन॑ । धे॒नुरिति॑ । इति॑ पु॒रस्ता᳚ल्लक्ष्मा । पु॒रस्ता᳚ल्लक्ष्मा पुरोनुवा॒क्या᳚ । पु॒रस्ता᳚ल्ल॒क्ष्मेति॑ पु॒रस्ता᳚त् - ल॒क्ष्मा॒ । पु॒रो॒नु॒वा॒क्या॑ भवति । पु॒रो॒नु॒वा॒क्येति॑ पुरः - अ॒नु॒वा॒क्या᳚ । भ॒व॒ति॒ जा॒तान् । जा॒ताने॒व । ए॒व भ्रातृ॑व्यान् । भ्रातृ॑व्या॒न् प्र । प्र णु॑दते । नु॒द॒त॒ उ॒परि॑ष्टाल्लक्ष्मा । उ॒परि॑ष्टाल्लक्ष्मा या॒ज्या᳚ । उ॒परि॑ष्टाल्ल॒क्ष्मेत्यु॒परि॑ष्टात् - ल॒क्ष्मा॒ \newline

\textbf{Jatai Paata} \newline

1. नि॒युत्व॑त्या यजति यजति नि॒युत्व॑त्या नि॒युत्व॑त्या यजति । \newline
2. नि॒युत्व॒त्येति॑ नि - युत्व॑त्या । \newline
3. य॒ज॒ति॒ भ्रातृ॑व्यस्य॒ भ्रातृ॑व्यस्य यजति यजति॒ भ्रातृ॑व्यस्य । \newline
4. भ्रातृ॑व्य स्यै॒वैव भ्रातृ॑व्यस्य॒ भ्रातृ॑व्य स्यै॒व । \newline
5. ए॒व प॒शून् प॒शू ने॒वैव प॒शून् । \newline
6. प॒शून् नि नि प॒शून् प॒शून् नि । \newline
7. नि यु॑वते युवते॒ नि नि यु॑वते । \newline
8. यु॒व॒ते॒ के॒शिन॑म् के॒शिनं॑ ॅयुवते युवते के॒शिन᳚म् । \newline
9. के॒शिनꣳ॑ ह ह के॒शिन॑म् के॒शिनꣳ॑ ह । \newline
10. ह॒ दा॒र्भ्यम् दा॒र्भ्यꣳ ह॑ ह दा॒र्भ्यम् । \newline
11. दा॒र्भ्यम् के॒शी के॒शी दा॒र्भ्यम् दा॒र्भ्यम् के॒शी । \newline
12. के॒शी सात्य॑कामिः॒ सात्य॑कामिः के॒शी के॒शी सात्य॑कामिः । \newline
13. सात्य॑कामि रुवाचोवाच॒ सात्य॑कामिः॒ सात्य॑कामि रुवाच । \newline
14. सात्य॑कामि॒रिति॒ सात्य॑ - का॒मिः॒ । \newline
15. उ॒वा॒च॒ स॒प्तप॑दाꣳ स॒प्तप॑दा मुवाचोवाच स॒प्तप॑दाम् । \newline
16. स॒प्तप॑दाम् ते ते स॒प्तप॑दाꣳ स॒प्तप॑दाम् ते । \newline
17. स॒प्तप॑दा॒मिति॑ स॒प्त - प॒दा॒म् । \newline
18. ते॒ शक्व॑रीꣳ॒॒ शक्व॑रीम् ते ते॒ शक्व॑रीम् । \newline
19. शक्व॑रीꣳ॒॒ श्वः श्वः शक्व॑रीꣳ॒॒ शक्व॑रीꣳ॒॒ श्वः । \newline
20. श्वो य॒ज्ञे य॒ज्ञे श्वः श्वो य॒ज्ञे । \newline
21. य॒ज्ञे प्र॑यो॒क्तासे᳚ प्रयो॒क्तासे॑ य॒ज्ञे य॒ज्ञे प्र॑यो॒क्तासे᳚ । \newline
22. प्र॒यो॒क्तासे॒ यस्यै॒ यस्यै᳚ प्रयो॒क्तासे᳚ प्रयो॒क्तासे॒ यस्यै᳚ । \newline
23. प्र॒यो॒क्तास॒ इति॑ प्र - यो॒क्तासे᳚ । \newline
24. यस्यै॑ वी॒र्ये॑ण वी॒र्ये॑ण॒ यस्यै॒ यस्यै॑ वी॒र्ये॑ण । \newline
25. वी॒र्ये॑ण॒ प्र प्र वी॒र्ये॑ण वी॒र्ये॑ण॒ प्र । \newline
26. प्र जा॒तान् जा॒तान् प्र प्र जा॒तान् । \newline
27. जा॒तान् भ्रातृ॑व्या॒न् भ्रातृ॑व्यान् जा॒तान् जा॒तान् भ्रातृ॑व्यान् । \newline
28. भ्रातृ॑व्यान् नु॒दते॑ नु॒दते॒ भ्रातृ॑व्या॒न् भ्रातृ॑व्यान् नु॒दते᳚ । \newline
29. नु॒दते॒ प्रति॒ प्रति॑ नु॒दते॑ नु॒दते॒ प्रति॑ । \newline
30. प्रति॑ जनि॒ष्यमा॑णान् जनि॒ष्यमा॑णा॒न् प्रति॒ प्रति॑ जनि॒ष्यमा॑णान् । \newline
31. ज॒नि॒ष्यमा॑णा॒न्॒. यस्यै॒ यस्यै॑ जनि॒ष्यमा॑णान् जनि॒ष्यमा॑णा॒न्॒. यस्यै᳚ । \newline
32. यस्यै॑ वी॒र्ये॑ण वी॒र्ये॑ण॒ यस्यै॒ यस्यै॑ वी॒र्ये॑ण । \newline
33. वी॒र्ये॑णो॒भयो॑ रु॒भयो᳚र् वी॒र्ये॑ण वी॒र्ये॑णो॒भयोः᳚ । \newline
34. उ॒भयो᳚र् लो॒कयो᳚र् लो॒कयो॑ रु॒भयो॑ रु॒भयो᳚र् लो॒कयोः᳚ । \newline
35. लो॒कयो॒र् ज्योति॒र् ज्योति॑र् लो॒कयो᳚र् लो॒कयो॒र् ज्योतिः॑ । \newline
36. ज्योति॑र् ध॒त्ते ध॒त्ते ज्योति॒र् ज्योति॑र् ध॒त्ते । \newline
37. ध॒त्ते यस्यै॒ यस्यै॑ ध॒त्ते ध॒त्ते यस्यै᳚ । \newline
38. यस्यै॑ वी॒र्ये॑ण वी॒र्ये॑ण॒ यस्यै॒ यस्यै॑ वी॒र्ये॑ण । \newline
39. वी॒र्ये॑ण पूर्वा॒र्द्धेन॑ पूर्वा॒र्द्धेन॑ वी॒र्ये॑ण वी॒र्ये॑ण पूर्वा॒र्द्धेन॑ । \newline
40. पू॒र्वा॒र्द्धेना॑ न॒ड्वा न॑न॒ड्वान् पू᳚र्वा॒र्द्धेन॑ पूर्वा॒र्द्धेना॑ न॒ड्वान् । \newline
41. पू॒र्वा॒र्द्धेनेति॑ पूर्व - अ॒र्द्धेन॑ । \newline
42. अ॒न॒ड्वान् भु॒नक्ति॑ भु॒नक्त्य॑न॒ड्वा न॑न॒ड्वान् भु॒नक्ति॑ । \newline
43. भु॒नक्ति॑ जघना॒र्द्धेन॑ जघना॒र्द्धेन॑ भु॒नक्ति॑ भु॒नक्ति॑ जघना॒र्द्धेन॑ । \newline
44. ज॒घ॒ना॒र्द्धेन॑ धे॒नुर् धे॒नुर् ज॑घना॒र्द्धेन॑ जघना॒र्द्धेन॑ धे॒नुः । \newline
45. ज॒घ॒ना॒र्द्धेनेति॑ जघन - अ॒र्द्धेन॑ । \newline
46. धे॒नु रितीति॑ धे॒नुर् धे॒नु रिति॑ । \newline
47. इति॑ पु॒रस्ता᳚ल्लक्ष्मा पु॒रस्ता᳚ल्ल॒क्ष्मेतीति॑ पु॒रस्ता᳚ल्लक्ष्मा । \newline
48. पु॒रस्ता᳚ल्लक्ष्मा पुरोनुवा॒क्या॑ पुरोनुवा॒क्या॑ पु॒रस्ता᳚ल्लक्ष्मा पु॒रस्ता᳚ल्लक्ष्मा पुरोनुवा॒क्या᳚ । \newline
49. पु॒रस्ता᳚ल्ल॒क्ष्मेति॑ पु॒रस्ता᳚त् - ल॒क्ष्मा॒ । \newline
50. पु॒रो॒नु॒वा॒क्या॑ भवति भवति पुरोनुवा॒क्या॑ पुरोनुवा॒क्या॑ भवति । \newline
51. पु॒रो॒नु॒वा॒क्येति॑ पुरः - अ॒नु॒वा॒क्या᳚ । \newline
52. भ॒व॒ति॒ जा॒तान् जा॒तान् भ॑वति भवति जा॒तान् । \newline
53. जा॒ता ने॒वैव जा॒तान् जा॒ता ने॒व । \newline
54. ए॒व भ्रातृ॑व्या॒न् भ्रातृ॑व्या ने॒वैव भ्रातृ॑व्यान् । \newline
55. भ्रातृ॑व्या॒न् प्र प्र भ्रातृ॑व्या॒न् भ्रातृ॑व्या॒न् प्र । \newline
56. प्र णु॑दते नुदते॒ प्र प्र णु॑दते । \newline
57. नु॒द॒त॒ उ॒परि॑ष्टाल्ल क्ष्मो॒परि॑ष्टाल्लक्ष्मा नुदते नुदत उ॒परि॑ष्टाल्लक्ष्मा । \newline
58. उ॒परि॑ष्टाल्लक्ष्मा या॒ज्या॑ या॒ज्यो॑परि॑ष्टाल्ल क्ष्मो॒परि॑ष्टाल्लक्ष्मा या॒ज्या᳚ । \newline
59. उ॒परि॑ष्टाल्ल॒क्ष्मेत्यु॒परि॑ष्टात् - ल॒क्ष्मा॒ । \newline

\textbf{Ghana Paata } \newline

1. नि॒युत्व॑त्या यजति यजति नि॒युत्व॑त्या नि॒युत्व॑त्या यजति॒ भ्रातृ॑व्यस्य॒ भ्रातृ॑व्यस्य यजति नि॒युत्व॑त्या नि॒युत्व॑त्या यजति॒ भ्रातृ॑व्यस्य । \newline
2. नि॒युत्व॒त्येति॑ नि - युत्व॑त्या । \newline
3. य॒ज॒ति॒ भ्रातृ॑व्यस्य॒ भ्रातृ॑व्यस्य यजति यजति॒ भ्रातृ॑व्य स्यै॒वैव भ्रातृ॑व्यस्य यजति यजति॒ भ्रातृ॑व्यस्यै॒व । \newline
4. भ्रातृ॑व्य स्यै॒वैव भ्रातृ॑व्यस्य॒ भ्रातृ॑व्यस्यै॒व प॒शून् प॒शू ने॒व भ्रातृ॑व्यस्य॒ भ्रातृ॑व्यस्यै॒व प॒शून् । \newline
5. ए॒व प॒शून् प॒शू ने॒वैव प॒शून् नि नि प॒शू ने॒वैव प॒शून् नि । \newline
6. प॒शून् नि नि प॒शून् प॒शून् नि यु॑वते युवते॒ नि प॒शून् प॒शून् नि यु॑वते । \newline
7. नि यु॑वते युवते॒ नि नि यु॑वते के॒शिन॑म् के॒शिनं॑ ॅयुवते॒ नि नि यु॑वते के॒शिन᳚म् । \newline
8. यु॒व॒ते॒ के॒शिन॑म् के॒शिनं॑ ॅयुवते युवते के॒शिनꣳ॑ ह ह के॒शिनं॑ ॅयुवते युवते के॒शिनꣳ॑ ह । \newline
9. के॒शिनꣳ॑ ह ह के॒शिन॑म् के॒शिनꣳ॑ ह दा॒र्भ्यम् दा॒र्भ्यꣳ ह॑ के॒शिन॑म् के॒शिनꣳ॑ ह दा॒र्भ्यम् । \newline
10. ह॒ दा॒र्भ्यम् दा॒र्भ्यꣳ ह॑ ह दा॒र्भ्यम् के॒शी के॒शी दा॒र्भ्यꣳ ह॑ ह दा॒र्भ्यम् के॒शी । \newline
11. दा॒र्भ्यम् के॒शी के॒शी दा॒र्भ्यम् दा॒र्भ्यम् के॒शी सात्य॑कामिः॒ सात्य॑कामिः के॒शी दा॒र्भ्यम् दा॒र्भ्यम् के॒शी सात्य॑कामिः । \newline
12. के॒शी सात्य॑कामिः॒ सात्य॑कामिः के॒शी के॒शी सात्य॑कामि रुवाचोवाच॒ सात्य॑कामिः के॒शी के॒शी सात्य॑कामि रुवाच । \newline
13. सात्य॑कामि रुवाचोवाच॒ सात्य॑कामिः॒ सात्य॑कामि रुवाच स॒प्तप॑दाꣳ स॒प्तप॑दा मुवाच॒ सात्य॑कामिः॒ सात्य॑कामि रुवाच स॒प्तप॑दाम् । \newline
14. सात्य॑कामि॒रिति॒ सात्य॑ - का॒मिः॒ । \newline
15. उ॒वा॒च॒ स॒प्तप॑दाꣳ स॒प्तप॑दा मुवाचोवाच स॒प्तप॑दाम् ते ते स॒प्तप॑दा मुवाचोवाच स॒प्तप॑दाम् ते । \newline
16. स॒प्तप॑दाम् ते ते स॒प्तप॑दाꣳ स॒प्तप॑दाम् ते॒ शक्व॑रीꣳ॒॒ शक्व॑रीम् ते स॒प्तप॑दाꣳ स॒प्तप॑दाम् ते॒ शक्व॑रीम् । \newline
17. स॒प्तप॑दा॒मिति॑ स॒प्त - प॒दा॒म् । \newline
18. ते॒ शक्व॑रीꣳ॒॒ शक्व॑रीम् ते ते॒ शक्व॑रीꣳ॒॒ श्वः श्वः शक्व॑रीम् ते ते॒ शक्व॑रीꣳ॒॒ श्वः । \newline
19. शक्व॑रीꣳ॒॒ श्वः श्वः शक्व॑रीꣳ॒॒ शक्व॑रीꣳ॒॒ श्वो य॒ज्ञे य॒ज्ञे श्वः शक्व॑रीꣳ॒॒ शक्व॑रीꣳ॒॒ श्वो य॒ज्ञे । \newline
20. श्वो य॒ज्ञे य॒ज्ञे श्वः श्वो य॒ज्ञे प्र॑यो॒क्तासे᳚ प्रयो॒क्तासे॑ य॒ज्ञे श्वः श्वो य॒ज्ञे प्र॑यो॒क्तासे᳚ । \newline
21. य॒ज्ञे प्र॑यो॒क्तासे᳚ प्रयो॒क्तासे॑ य॒ज्ञे य॒ज्ञे प्र॑यो॒क्तासे॒ यस्यै॒ यस्यै᳚ प्रयो॒क्तासे॑ य॒ज्ञे य॒ज्ञे प्र॑यो॒क्तासे॒ यस्यै᳚ । \newline
22. प्र॒यो॒क्तासे॒ यस्यै॒ यस्यै᳚ प्रयो॒क्तासे᳚ प्रयो॒क्तासे॒ यस्यै॑ वी॒र्ये॑ण वी॒र्ये॑ण॒ यस्यै᳚ प्रयो॒क्तासे᳚ प्रयो॒क्तासे॒ यस्यै॑ वी॒र्ये॑ण । \newline
23. प्र॒यो॒क्तास॒ इति॑ प्र - यो॒क्तासे᳚ । \newline
24. यस्यै॑ वी॒र्ये॑ण वी॒र्ये॑ण॒ यस्यै॒ यस्यै॑ वी॒र्ये॑ण॒ प्र प्र वी॒र्ये॑ण॒ यस्यै॒ यस्यै॑ वी॒र्ये॑ण॒ प्र । \newline
25. वी॒र्ये॑ण॒ प्र प्र वी॒र्ये॑ण वी॒र्ये॑ण॒ प्र जा॒तान् जा॒तान् प्र वी॒र्ये॑ण वी॒र्ये॑ण॒ प्र जा॒तान् । \newline
26. प्र जा॒तान् जा॒तान् प्र प्र जा॒तान् भ्रातृ॑व्या॒न् भ्रातृ॑व्यान् जा॒तान् प्र प्र जा॒तान् भ्रातृ॑व्यान् । \newline
27. जा॒तान् भ्रातृ॑व्या॒न् भ्रातृ॑व्यान् जा॒तान् जा॒तान् भ्रातृ॑व्यान् नु॒दते॑ नु॒दते॒ भ्रातृ॑व्यान् जा॒तान् जा॒तान् भ्रातृ॑व्यान् नु॒दते᳚ । \newline
28. भ्रातृ॑व्यान् नु॒दते॑ नु॒दते॒ भ्रातृ॑व्या॒न् भ्रातृ॑व्यान् नु॒दते॒ प्रति॒ प्रति॑ नु॒दते॒ भ्रातृ॑व्या॒न् भ्रातृ॑व्यान् नु॒दते॒ प्रति॑ । \newline
29. नु॒दते॒ प्रति॒ प्रति॑ नु॒दते॑ नु॒दते॒ प्रति॑ जनि॒ष्यमा॑णान् जनि॒ष्यमा॑णा॒न् प्रति॑ नु॒दते॑ नु॒दते॒ प्रति॑ जनि॒ष्यमा॑णान् । \newline
30. प्रति॑ जनि॒ष्यमा॑णान् जनि॒ष्यमा॑णा॒न् प्रति॒ प्रति॑ जनि॒ष्यमा॑णा॒न्॒. यस्यै॒ यस्यै॑ जनि॒ष्यमा॑णा॒न् प्रति॒ प्रति॑ जनि॒ष्यमा॑णा॒न्॒. यस्यै᳚ । \newline
31. ज॒नि॒ष्यमा॑णा॒न्॒. यस्यै॒ यस्यै॑ जनि॒ष्यमा॑णान् जनि॒ष्यमा॑णा॒न्॒. यस्यै॑ वी॒र्ये॑ण वी॒र्ये॑ण॒ यस्यै॑ जनि॒ष्यमा॑णान् जनि॒ष्यमा॑णा॒न्॒. यस्यै॑ वी॒र्ये॑ण । \newline
32. यस्यै॑ वी॒र्ये॑ण वी॒र्ये॑ण॒ यस्यै॒ यस्यै॑ वी॒र्ये॑णो॒भयो॑ रु॒भयो᳚र् वी॒र्ये॑ण॒ यस्यै॒ यस्यै॑ वी॒र्ये॑णो॒भयोः᳚ । \newline
33. वी॒र्ये॑णो॒भयो॑ रु॒भयो᳚र् वी॒र्ये॑ण वी॒र्ये॑णो॒भयो᳚र् लो॒कयो᳚र् लो॒कयो॑ रु॒भयो᳚र् वी॒र्ये॑ण वी॒र्ये॑णो॒भयो᳚र् लो॒कयोः᳚ । \newline
34. उ॒भयो᳚र् लो॒कयो᳚र् लो॒कयो॑ रु॒भयो॑ रु॒भयो᳚र् लो॒कयो॒र् ज्योति॒र् ज्योति॑र् लो॒कयो॑ रु॒भयो॑ रु॒भयो᳚र् लो॒कयो॒र् ज्योतिः॑ । \newline
35. लो॒कयो॒र् ज्योति॒र् ज्योति॑र् लो॒कयो᳚र् लो॒कयो॒र् ज्योति॑र् ध॒त्ते ध॒त्ते ज्योति॑र् लो॒कयो᳚र् लो॒कयो॒र् ज्योति॑र् ध॒त्ते । \newline
36. ज्योति॑र् ध॒त्ते ध॒त्ते ज्योति॒र् ज्योति॑र् ध॒त्ते यस्यै॒ यस्यै॑ ध॒त्ते ज्योति॒र् ज्योति॑र् ध॒त्ते यस्यै᳚ । \newline
37. ध॒त्ते यस्यै॒ यस्यै॑ ध॒त्ते ध॒त्ते यस्यै॑ वी॒र्ये॑ण वी॒र्ये॑ण॒ यस्यै॑ ध॒त्ते ध॒त्ते यस्यै॑ वी॒र्ये॑ण । \newline
38. यस्यै॑ वी॒र्ये॑ण वी॒र्ये॑ण॒ यस्यै॒ यस्यै॑ वी॒र्ये॑ण पूर्वा॒र्द्धेन॑ पूर्वा॒र्द्धेन॑ वी॒र्ये॑ण॒ यस्यै॒ यस्यै॑ वी॒र्ये॑ण पूर्वा॒र्द्धेन॑ । \newline
39. वी॒र्ये॑ण पूर्वा॒र्द्धेन॑ पूर्वा॒र्द्धेन॑ वी॒र्ये॑ण वी॒र्ये॑ण पूर्वा॒र्द्धेना॑ न॒ड्वा न॑न॒ड्वान् पू᳚र्वा॒र्द्धेन॑ वी॒र्ये॑ण वी॒र्ये॑ण पूर्वा॒र्द्धेना॑ न॒ड्वान् । \newline
40. पू॒र्वा॒र्द्धेना॑ न॒ड्वा न॑न॒ड्वान् पू᳚र्वा॒र्द्धेन॑ पूर्वा॒र्द्धेना॑ न॒ड्वान् भु॒नक्ति॑ भु॒न क्त्य॑न॒ड्वान् पू᳚र्वा॒र्द्धेन॑ पूर्वा॒र्द्धेना॑ न॒ड्वान् भु॒नक्ति॑ । \newline
41. पू॒र्वा॒र्द्धेनेति॑ पूर्व - अ॒र्द्धेन॑ । \newline
42. अ॒न॒ड्वान् भु॒नक्ति॑ भु॒न क्त्य॑न॒ड्वा न॑न॒ड्वान् भु॒नक्ति॑ जघना॒र्द्धेन॑ जघना॒र्द्धेन॑ भु॒नक्त्य॑न॒ड्वा न॑न॒ड्वान् भु॒नक्ति॑ जघना॒र्द्धेन॑ । \newline
43. भु॒नक्ति॑ जघना॒र्द्धेन॑ जघना॒र्द्धेन॑ भु॒नक्ति॑ भु॒नक्ति॑ जघना॒र्द्धेन॑ धे॒नुर् धे॒नुर् ज॑घना॒र्द्धेन॑ भु॒नक्ति॑ भु॒नक्ति॑ जघना॒र्द्धेन॑ धे॒नुः । \newline
44. ज॒घ॒ना॒र्द्धेन॑ धे॒नुर् धे॒नुर् ज॑घना॒र्द्धेन॑ जघना॒र्द्धेन॑ धे॒नुरितीति॑ धे॒नुर् ज॑घना॒र्द्धेन॑ जघना॒र्द्धेन॑ धे॒नुरिति॑ । \newline
45. ज॒घ॒ना॒र्द्धेनेति॑ जघन - अ॒र्द्धेन॑ । \newline
46. धे॒नुरितीति॑ धे॒नुर् धे॒नुरिति॑ पु॒रस्ता᳚ल्लक्ष्मा पु॒रस्ता᳚ल्ल॒क्ष्मेति॑ धे॒नुर् धे॒नुरिति॑ पु॒रस्ता᳚ल्लक्ष्मा । \newline
47. इति॑ पु॒रस्ता᳚ल्लक्ष्मा पु॒रस्ता᳚ल्ल॒क्ष्मेतीति॑ पु॒रस्ता᳚ल्लक्ष्मा पुरोनुवा॒क्या॑ पुरोनुवा॒क्या॑ पु॒रस्ता᳚ल्ल॒क्ष्मेतीति॑ पु॒रस्ता᳚ल्लक्ष्मा पुरोनुवा॒क्या᳚ । \newline
48. पु॒रस्ता᳚ल्लक्ष्मा पुरोनुवा॒क्या॑ पुरोनुवा॒क्या॑ पु॒रस्ता᳚ल्लक्ष्मा पु॒रस्ता᳚ल्लक्ष्मा पुरोनुवा॒क्या॑ भवति भवति पुरोनुवा॒क्या॑ पु॒रस्ता᳚ल्लक्ष्मा पु॒रस्ता᳚ल्लक्ष्मा पुरोनुवा॒क्या॑ भवति । \newline
49. पु॒रस्ता᳚ल्ल॒क्ष्मेति॑ पु॒रस्ता᳚त् - ल॒क्ष्मा॒ । \newline
50. पु॒रो॒नु॒वा॒क्या॑ भवति भवति पुरोनुवा॒क्या॑ पुरोनुवा॒क्या॑ भवति जा॒तान् जा॒तान् भ॑वति पुरोनुवा॒क्या॑ पुरोनुवा॒क्या॑ भवति जा॒तान् । \newline
51. पु॒रो॒नु॒वा॒क्येति॑ पुरः - अ॒नु॒वा॒क्या᳚ । \newline
52. भ॒व॒ति॒ जा॒तान् जा॒तान् भ॑वति भवति जा॒ता ने॒वैव जा॒तान् भ॑वति भवति जा॒ता ने॒व । \newline
53. जा॒ता ने॒वैव जा॒तान् जा॒ता ने॒व भ्रातृ॑व्या॒न् भ्रातृ॑व्या ने॒व जा॒तान् जा॒ता ने॒व भ्रातृ॑व्यान् । \newline
54. ए॒व भ्रातृ॑व्या॒न् भ्रातृ॑व्या ने॒वैव भ्रातृ॑व्या॒न् प्र प्र भ्रातृ॑व्या ने॒वैव भ्रातृ॑व्या॒न् प्र । \newline
55. भ्रातृ॑व्या॒न् प्र प्र भ्रातृ॑व्या॒न् भ्रातृ॑व्या॒न् प्र णु॑दते नुदते॒ प्र भ्रातृ॑व्या॒न् भ्रातृ॑व्या॒न् प्र णु॑दते । \newline
56. प्र णु॑दते नुदते॒ प्र प्र णु॑दत उ॒परि॑ष्टाल्लक्ष्मो॒ परि॑ष्टाल्लक्ष्मा नुदते॒ प्र प्र णु॑दत उ॒परि॑ष्टाल्लक्ष्मा । \newline
57. नु॒द॒त॒ उ॒परि॑ष्टाल्लक्ष्मो॒ परि॑ष्टाल्लक्ष्मा नुदते नुदत उ॒परि॑ष्टाल्लक्ष्मा या॒ज्या॑ या॒ज्यो॑परि॑ष्टाल्लक्ष्मा नुदते नुदत उ॒परि॑ष्टाल्लक्ष्मा या॒ज्या᳚ । \newline
58. उ॒परि॑ष्टाल्ल॒क्ष्मा या॒ज्या॑ या॒ज्यो॑ परि॑ष्टाल्लक्ष्मो॒ परि॑ष्टाल्लक्ष्मा या॒ज्या॑ जनि॒ष्यमा॑णान् जनि॒ष्यमा॑णान्. या॒ज्यो॑ परि॑ष्टाल्लक्ष्मो॒ परि॑ष्टाल्लक्ष्मा या॒ज्या॑ जनि॒ष्यमा॑णान् । \newline
59. उ॒परि॑ष्टाल्ल॒क्ष्मेत्यु॒परि॑ष्टात् - ल॒क्ष्मा॒ । \newline
\pagebreak
\markright{ TS 2.6.2.4  \hfill https://www.vedavms.in \hfill}
\addcontentsline{toc}{section}{ TS 2.6.2.4 }
\section*{ TS 2.6.2.4 }

\textbf{TS 2.6.2.4 } \newline
\textbf{Samhita Paata} \newline

या॒ज्या॑ जनि॒ष्यमा॑णाने॒व प्रति॑नुदते पु॒रस्ता᳚ल्लक्ष्मा पुरोऽनुवा॒क्या॑ भवत्य॒स्मिन्ने॒व लो॒के ज्योति॑र्द्धत्त उ॒परि॑ष्टाल्लक्ष्मा या॒ज्या॑ऽमुष्मि॑न्ने॒व लो॒के ज्योति॑र्द्धत्ते॒ ज्योति॑ष्मन्तावस्मा इ॒मौ लो॒कौ भ॑वतो॒ य ए॒वं ॅवेद॑ पु॒रस्ता᳚ल्लक्ष्मा पुरोऽनुवा॒क्या॑ भवति॒ तस्मा᳚त् पूर्वा॒र्द्धेना॑न॒ड्वान् भु॑नक्त्यु॒परि॑ष्टाल्लक्ष्मा या॒ज्या॑ तस्मा᳚ज्जघना॒र्द्धेन॑ धे॒नुर्य ए॒वं ॅवेद॑ भु॒ङ्क्त ए॑नमे॒तौ वज्र॒ आज्यं॒ ॅवज्र॒ आज्य॑भागौ॒ - [  ] \newline

\textbf{Pada Paata} \newline

या॒ज्या᳚ । ज॒नि॒ष्यमा॑णान् । ए॒व । प्रतीति॑ । नु॒द॒ते॒ । पु॒रस्ता᳚ल्ल॒क्ष्मेति॑ पु॒रस्ता᳚त् - ल॒क्ष्मा॒ । पु॒रो॒नु॒वा॒क्येति॑ पुरः - अ॒नु॒वा॒क्या᳚ । भ॒व॒ति॒ । अ॒स्मिन्न् । ए॒व । लो॒के । ज्योतिः॑ । ध॒त्ते॒ । उ॒परि॑ष्टाल्ल॒क्ष्मेत्यु॒परि॑ष्टात् - ल॒क्ष्मा॒ । या॒ज्या᳚ । अ॒मुष्मिन्न्॑ । ए॒व । लो॒के । ज्योतिः॑ । ध॒त्ते॒ । ज्योति॑ष्मन्तौ । अ॒स्मै॒ । इ॒मौ । लो॒कौ । भ॒व॒तः॒ । यः । ए॒वम् । वेद॑ । पु॒रस्ता᳚ल्ल॒क्ष्मेति॑ पु॒रस्ता᳚त् - ल॒क्ष्मा॒ । पु॒रो॒नु॒वा॒क्येति॑ पुरः - अ॒नु॒वा॒क्या᳚ । भ॒व॒ति॒ । तस्मा᳚त् । पू॒र्वा॒र्द्धेनेति॑ पूर्व - अ॒र्द्धेन॑ । अ॒न॒ड्वान् । भु॒न॒क्ति॒ । उ॒परि॑ष्टाल्ल॒क्ष्मेत्यु॒परि॑ष्टात् - ल॒क्ष्मा॒ । या॒ज्या᳚ । तस्मा᳚त् । ज॒घ॒ना॒र्द्धेनेति॑ जघन - अ॒र्द्धेन॑ । धे॒नुः । यः । ए॒वम् । वेद॑ । भु॒ङ्क्तः । ए॒न॒म् । ए॒तौ । वज्रः॑ । आज्य᳚म् । वज्रः॑ । आज्य॑भागा॒वित्याज्य॑ - भा॒गौ॒ ।  \newline


\textbf{Krama Paata} \newline

या॒ज्या॑ जनि॒ष्यमा॑णान् । ज॒नि॒ष्यमा॑णाने॒व । ए॒व प्रति॑ । प्रति॑ नुदते । नु॒द॒ते॒ पु॒रस्ता᳚ल्लक्ष्मा । पु॒रस्ता᳚ल्लक्ष्मा पुरोनुवा॒क्या᳚ । पु॒रस्ता᳚ल्ल॒क्ष्मेति॑ पु॒रस्ता᳚त् - ल॒क्ष्मा॒ । पु॒रो॒नु॒वा॒क्या॑ भवति । पु॒रो॒नु॒वा॒क्येति॑ पुरः - अ॒नु॒वा॒क्या᳚ । भ॒व॒त्य॒स्मिन्न् । अ॒स्मिन्ने॒व । ए॒व लो॒के । लो॒के ज्योतिः॑ । ज्योति॑र् धत्ते । ध॒त्त॒ उ॒परि॑ष्टाल्लक्ष्मा । उ॒परि॑ष्टाल्लक्ष्मा या॒ज्या᳚ । उ॒प॑रिष्टाल्ल॒क्ष्मेत्यु॒परि॑ष्टात् - ल॒क्ष्मा॒ । या॒ज्या॑ऽमुष्मिन्न्॑ । अ॒मुष्मि॑न्ने॒व । ए॒व लो॒के । लो॒के ज्योतिः॑ । ज्योति॑र् धत्ते । ध॒त्ते॒ ज्योति॑ष्मन्तौ । ज्योति॑ष्मन्तावस्मै । अ॒स्मा॒ इ॒मौ । इ॒मौ लो॒कौ । लो॒कौ भ॑वतः । भ॒व॒तो॒ यः । य ए॒वम् । ए॒वं ॅवेद॑ । वेद॑ पु॒रस्ता᳚ल्लक्ष्मा । पु॒रस्ता᳚ल्लक्ष्मा पुरोनुवा॒क्या᳚ । पु॒रस्ता᳚ल्ल॒क्ष्मेति॑ पु॒रस्ता᳚त् - ल॒क्ष्मा॒ । पु॒रो॒नु॒वा॒क्या॑ भवति । पु॒रो॒नु॒वा॒क्येति॑ पुरः - अ॒नु॒वा॒क्या᳚ । भ॒व॒ति॒ तस्मा᳚त् । तस्मा᳚त् पूर्वा॒र्द्धेन॑ । पू॒र्वा॒र्द्धेना॑न॒ड्वान् । पू॒र्वा॒र्द्धेनेति॑ पूर्व - अ॒र्द्धेन॑ । अ॒न॒ड्वान् भु॑नक्ति । भु॒न॒क्त्यु॒परि॑ष्टाल्लक्ष्मा । उ॒परि॑ष्टाल्लक्ष्मा या॒ज्या᳚ । उ॒परि॑ष्टाल्ल॒क्ष्मेत्यु॒परि॑ष्टात् - ल॒क्ष्मा॒ । या॒ज्या॑ तस्मा᳚त् । तस्मा᳚ज् जघना॒र्द्धेन॑ । ज॒घ॒ना॒र्द्धेन॑ धे॒नुः । ज॒घ॒ना॒र्द्धेनेति॑ जघन - अ॒र्द्धेन॑ । धे॒नुर् यः । य ए॒वम् । ए॒वं ॅवेद॑ । वेद॑ भु॒ङ्क्तः । भु॒ङ्क्त ए॑नम् । ए॒न॒मे॒तौ । ए॒तौ वज्रः॑ । वज्र॒ आज्य᳚म् । आज्यं॒ ॅवज्रः॑ । वज्र॒ आज्य॑भागौ । आज्य॑भागौ॒ वज्रः॑ । आज्य॑भागा॒वित्याज्य॑ - भा॒गौ॒ \newline

\textbf{Jatai Paata} \newline

1. या॒ज्या॑ जनि॒ष्यमा॑णान् जनि॒ष्यमा॑णान्. या॒ज्या॑ या॒ज्या॑ जनि॒ष्यमा॑णान् । \newline
2. ज॒नि॒ष्यमा॑णा ने॒वैव ज॑नि॒ष्यमा॑णान् जनि॒ष्यमा॑णा ने॒व । \newline
3. ए॒व प्रति॒ प्रत्ये॒वैव प्रति॑ । \newline
4. प्रति॑ नुदते नुदते॒ प्रति॒ प्रति॑ नुदते । \newline
5. नु॒द॒ते॒ पु॒रस्ता᳚ल्लक्ष्मा पु॒रस्ता᳚ल्लक्ष्मा नुदते नुदते पु॒रस्ता᳚ल्लक्ष्मा । \newline
6. पु॒रस्ता᳚ल्लक्ष्मा पुरोनुवा॒क्या॑ पुरोनुवा॒क्या॑ पु॒रस्ता᳚ल्लक्ष्मा पु॒रस्ता᳚ल्लक्ष्मा पुरोनुवा॒क्या᳚ । \newline
7. पु॒रस्ता᳚ल्ल॒क्ष्मेति॑ पु॒रस्ता᳚त् - ल॒क्ष्मा॒ । \newline
8. पु॒रो॒नु॒वा॒क्या॑ भवति भवति पुरोनुवा॒क्या॑ पुरोनुवा॒क्या॑ भवति । \newline
9. पु॒रो॒नु॒वा॒क्येति॑ पुरः - अ॒नु॒वा॒क्या᳚ । \newline
10. भ॒व॒ त्य॒स्मिन् न॒स्मिन् भ॑वति भव त्य॒स्मिन्न् । \newline
11. अ॒स्मिन् ने॒वैवास्मिन् न॒स्मिन् ने॒व । \newline
12. ए॒व लो॒के लो॒क ए॒वैव लो॒के । \newline
13. लो॒के ज्योति॒र् ज्योति॑र् लो॒के लो॒के ज्योतिः॑ । \newline
14. ज्योति॑र् धत्ते धत्ते॒ ज्योति॒र् ज्योति॑र् धत्ते । \newline
15. ध॒त्त॒ उ॒परि॑ष्टाल्ल क्ष्मो॒परि॑ष्टाल्लक्ष्मा धत्ते धत्त उ॒परि॑ष्टाल्लक्ष्मा । \newline
16. उ॒परि॑ष्टाल्लक्ष्मा या॒ज्या॑ या॒ज्यो॑परि॑ष्टाल्ल क्ष्मो॒परि॑ष्टाल्लक्ष्मा या॒ज्या᳚ । \newline
17. उ॒परि॑ष्टाल्ल॒क्ष्मेत्यु॒परि॑ष्टात् - ल॒क्ष्मा॒ । \newline
18. या॒ज्या॑ ऽमुष्मि॑न् न॒मुष्मि॑न्. या॒ज्या॑ या॒ज्या॑ ऽमुष्मिन्न्॑ । \newline
19. अ॒मुष्मि॑न् ने॒वैवामुष्मि॑न् न॒मुष्मि॑न् ने॒व । \newline
20. ए॒व लो॒के लो॒क ए॒वैव लो॒के । \newline
21. लो॒के ज्योति॒र् ज्योति॑र् लो॒के लो॒के ज्योतिः॑ । \newline
22. ज्योति॑र् धत्ते धत्ते॒ ज्योति॒र् ज्योति॑र् धत्ते । \newline
23. ध॒त्ते॒ ज्योति॑ष्मन्तौ॒ ज्योति॑ष्मन्तौ धत्ते धत्ते॒ ज्योति॑ष्मन्तौ । \newline
24. ज्योति॑ष्मन्ता वस्मा अस्मै॒ ज्योति॑ष्मन्तौ॒ ज्योति॑ष्मन्ता वस्मै । \newline
25. अ॒स्मा॒ इ॒मा वि॒मा व॑स्मा अस्मा इ॒मौ । \newline
26. इ॒मौ लो॒कौ लो॒का वि॒मा वि॒मौ लो॒कौ । \newline
27. लो॒कौ भ॑वतो भवतो लो॒कौ लो॒कौ भ॑वतः । \newline
28. भ॒व॒तो॒ यो यो भ॑वतो भवतो॒ यः । \newline
29. य ए॒व मे॒वं ॅयो य ए॒वम् । \newline
30. ए॒वं ॅवेद॒ वेदै॒व मे॒वं ॅवेद॑ । \newline
31. वेद॑ पु॒रस्ता᳚ल्लक्ष्मा पु॒रस्ता᳚ल्लक्ष्मा॒ वेद॒ वेद॑ पु॒रस्ता᳚ल्लक्ष्मा । \newline
32. पु॒रस्ता᳚ल्लक्ष्मा पुरोनुवा॒क्या॑ पुरोनुवा॒क्या॑ पु॒रस्ता᳚ल्लक्ष्मा पु॒रस्ता᳚ल्लक्ष्मा पुरोनुवा॒क्या᳚ । \newline
33. पु॒रस्ता᳚ल्ल॒क्ष्मेति॑ पु॒रस्ता᳚त् - ल॒क्ष्मा॒ । \newline
34. पु॒रो॒नु॒वा॒क्या॑ भवति भवति पुरोनुवा॒क्या॑ पुरोनुवा॒क्या॑ भवति । \newline
35. पु॒रो॒नु॒वा॒क्येति॑ पुरः - अ॒नु॒वा॒क्या᳚ । \newline
36. भ॒व॒ति॒ तस्मा॒त् तस्मा᳚द् भवति भवति॒ तस्मा᳚त् । \newline
37. तस्मा᳚त् पूर्वा॒र्द्धेन॑ पूर्वा॒र्द्धेन॒ तस्मा॒त् तस्मा᳚त् पूर्वा॒र्द्धेन॑ । \newline
38. पू॒र्वा॒र्द्धेना॑ न॒ड्वा न॑न॒ड्वान् पू᳚र्वा॒र्द्धेन॑ पूर्वा॒र्द्धेना॑ न॒ड्वान् । \newline
39. पू॒र्वा॒र्द्धेनेति॑ पूर्व - अ॒र्द्धेन॑ । \newline
40. अ॒न॒ड्वान् भु॑नक्ति भुन क्त्यन॒ड्वा न॑न॒ड्वान् भु॑नक्ति । \newline
41. भु॒न॒ क्त्यु॒परि॑ष्टाल्ल क्ष्मो॒परि॑ष्टाल्लक्ष्मा भुनक्ति भुन क्त्यु॒परि॑ष्टाल्लक्ष्मा । \newline
42. उ॒परि॑ष्टाल्लक्ष्मा या॒ज्या॑ या॒ज्यो॑परि॑ष्टाल्ल क्ष्मो॒परि॑ष्टाल्लक्ष्मा या॒ज्या᳚ । \newline
43. उ॒परि॑ष्टाल्ल॒क्ष्मेत्यु॒परि॑ष्टात् - ल॒क्ष्मा॒ । \newline
44. या॒ज्या॑ तस्मा॒त् तस्मा᳚द् या॒ज्या॑ या॒ज्या॑ तस्मा᳚त् । \newline
45. तस्मा᳚ज् जघना॒र्द्धेन॑ जघना॒र्द्धेन॒ तस्मा॒त् तस्मा᳚ज् जघना॒र्द्धेन॑ । \newline
46. ज॒घ॒ना॒र्द्धेन॑ धे॒नुर् धे॒नुर् ज॑घना॒र्द्धेन॑ जघना॒र्द्धेन॑ धे॒नुः । \newline
47. ज॒घ॒ना॒र्द्धेनेति॑ जघन - अ॒र्द्धेन॑ । \newline
48. धे॒नुर् यो यो धे॒नुर् धे॒नुर् यः । \newline
49. य ए॒व मे॒वं ॅयो य ए॒वम् । \newline
50. ए॒वं ॅवेद॒ वेदै॒व मे॒वं ॅवेद॑ । \newline
51. वेद॑ भु॒ङ्क्तो भु॒ङ्क्तो वेद॒ वेद॑ भु॒ङ्क्तः । \newline
52. भु॒ङ्क्त ए॑न मेनम् भु॒ङ्क्तो भु॒ङ्क्त ए॑नम् । \newline
53. ए॒न॒ मे॒ता वे॒ता वे॑न मेन मे॒तौ । \newline
54. ए॒तौ वज्रो॒ वज्र॑ ए॒ता वे॒तौ वज्रः॑ । \newline
55. वज्र॒ आज्य॒ माज्यं॒ ॅवज्रो॒ वज्र॒ आज्य᳚म् । \newline
56. आज्यं॒ ॅवज्रो॒ वज्र॒ आज्य॒ माज्यं॒ ॅवज्रः॑ । \newline
57. वज्र॒ आज्य॑भागा॒ वाज्य॑भागौ॒ वज्रो॒ वज्र॒ आज्य॑भागौ । \newline
58. आज्य॑भागौ॒ वज्रो॒ वज्र॒ आज्य॑भागा॒ वाज्य॑भागौ॒ वज्रः॑ । \newline
59. आज्य॑भागा॒वित्याज्य॑ - भा॒गौ॒ । \newline

\textbf{Ghana Paata } \newline

1. या॒ज्या॑ जनि॒ष्यमा॑णान् जनि॒ष्यमा॑णान्. या॒ज्या॑ या॒ज्या॑ जनि॒ष्यमा॑णा ने॒वैव ज॑नि॒ष्यमा॑णान्. या॒ज्या॑ या॒ज्या॑ जनि॒ष्यमा॑णा ने॒व । \newline
2. ज॒नि॒ष्यमा॑णा ने॒वैव ज॑नि॒ष्यमा॑णान् जनि॒ष्यमा॑णा ने॒व प्रति॒ प्रत्ये॒व ज॑नि॒ष्यमा॑णान् जनि॒ष्यमा॑णा ने॒व प्रति॑ । \newline
3. ए॒व प्रति॒ प्रत्ये॒वैव प्रति॑ नुदते नुदते॒ प्रत्ये॒वैव प्रति॑ नुदते । \newline
4. प्रति॑ नुदते नुदते॒ प्रति॒ प्रति॑ नुदते पु॒रस्ता᳚ल्लक्ष्मा पु॒रस्ता᳚ल्लक्ष्मा नुदते॒ प्रति॒ प्रति॑ नुदते पु॒रस्ता᳚ल्लक्ष्मा । \newline
5. नु॒द॒ते॒ पु॒रस्ता᳚ल्लक्ष्मा पु॒रस्ता᳚ल्लक्ष्मा नुदते नुदते पु॒रस्ता᳚ल्लक्ष्मा पुरोनुवा॒क्या॑ पुरोनुवा॒क्या॑ पु॒रस्ता᳚ल्लक्ष्मा नुदते नुदते पु॒रस्ता᳚ल्लक्ष्मा पुरोनुवा॒क्या᳚ । \newline
6. पु॒रस्ता᳚ल्लक्ष्मा पुरोनुवा॒क्या॑ पुरोनुवा॒क्या॑ पु॒रस्ता᳚ल्लक्ष्मा पु॒रस्ता᳚ल्लक्ष्मा पुरोनुवा॒क्या॑ भवति भवति पुरोनुवा॒क्या॑ पु॒रस्ता᳚ल्लक्ष्मा पु॒रस्ता᳚ल्लक्ष्मा पुरोनुवा॒क्या॑ भवति । \newline
7. पु॒रस्ता᳚ल्ल॒क्ष्मेति॑ पु॒रस्ता᳚त् - ल॒क्ष्मा॒ । \newline
8. पु॒रो॒नु॒वा॒क्या॑ भवति भवति पुरोनुवा॒क्या॑ पुरोनुवा॒क्या॑ भवत्य॒स्मिन् न॒स्मिन् भ॑वति पुरोनुवा॒क्या॑ पुरोनुवा॒क्या॑ भवत्य॒स्मिन्न् । \newline
9. पु॒रो॒नु॒वा॒क्येति॑ पुरः - अ॒नु॒वा॒क्या᳚ । \newline
10. भ॒व॒त्य॒स्मिन् न॒स्मिन् भ॑वति भवत्य॒स्मिन् ने॒वैवास्मिन् भ॑वति भवत्य॒स्मिन् ने॒व । \newline
11. अ॒स्मिन् ने॒वैवास्मिन् न॒स्मिन् ने॒व लो॒के लो॒क ए॒वास्मिन् न॒स्मिन् ने॒व लो॒के । \newline
12. ए॒व लो॒के लो॒क ए॒वैव लो॒के ज्योति॒र् ज्योति॑र् लो॒क ए॒वैव लो॒के ज्योतिः॑ । \newline
13. लो॒के ज्योति॒र् ज्योति॑र् लो॒के लो॒के ज्योति॑र् धत्ते धत्ते॒ ज्योति॑र् लो॒के लो॒के ज्योति॑र् धत्ते । \newline
14. ज्योति॑र् धत्ते धत्ते॒ ज्योति॒र् ज्योति॑र् धत्त उ॒परि॑ष्टाल्ल क्ष्मो॒परि॑ष्टाल्लक्ष्मा धत्ते॒ ज्योति॒र् ज्योति॑र् धत्त उ॒परि॑ष्टाल्लक्ष्मा । \newline
15. ध॒त्त॒ उ॒परि॑ष्टाल्ल क्ष्मो॒परि॑ष्टाल्लक्ष्मा धत्ते धत्त उ॒परि॑ष्टाल्लक्ष्मा या॒ज्या॑ या॒ज्यो॑परि॑ष्टाल्लक्ष्मा धत्ते धत्त उ॒परि॑ष्टाल्लक्ष्मा या॒ज्या᳚ । \newline
16. उ॒परि॑ष्टाल्लक्ष्मा या॒ज्या॑ या॒ज्यो॑परि॑ष्टाल्ल क्ष्मो॒परि॑ष्टाल्लक्ष्मा या॒ज्या॑ ऽमुष्मि॑न् न॒मुष्मिन्॑. या॒ज्यो॑ परि॑ष्टाल्ल क्ष्मो॒परि॑ष्टाल्लक्ष्मा या॒ज्या॑ ऽमुष्मिन्न्॑ । \newline
17. उ॒परि॑ष्टाल्ल॒क्ष्मेत्यु॒परि॑ष्टात् - ल॒क्ष्मा॒ । \newline
18. या॒ज्या॑ ऽमुष्मि॑न् न॒मुष्मिन्॑. या॒ज्या॑ या॒ज्या॑ ऽमुष्मि॑न् ने॒वैवामुष्मिन्॑. या॒ज्या॑ या॒ज्या॑ ऽमुष्मि॑न् ने॒व । \newline
19. अ॒मुष्मि॑न् ने॒वैवामुष्मि॑न् न॒मुष्मि॑न् ने॒व लो॒के लो॒क ए॒वामुष्मि॑न् न॒मुष्मि॑न् ने॒व लो॒के । \newline
20. ए॒व लो॒के लो॒क ए॒वैव लो॒के ज्योति॒र् ज्योति॑र् लो॒क ए॒वैव लो॒के ज्योतिः॑ । \newline
21. लो॒के ज्योति॒र् ज्योति॑र् लो॒के लो॒के ज्योति॑र् धत्ते धत्ते॒ ज्योति॑र् लो॒के लो॒के ज्योति॑र् धत्ते । \newline
22. ज्योति॑र् धत्ते धत्ते॒ ज्योति॒र् ज्योति॑र् धत्ते॒ ज्योति॑ष्मन्तौ॒ ज्योति॑ष्मन्तौ धत्ते॒ ज्योति॒र् ज्योति॑र् धत्ते॒ ज्योति॑ष्मन्तौ । \newline
23. ध॒त्ते॒ ज्योति॑ष्मन्तौ॒ ज्योति॑ष्मन्तौ धत्ते धत्ते॒ ज्योति॑ष्मन्ता वस्मा अस्मै॒ ज्योति॑ष्मन्तौ धत्ते धत्ते॒ ज्योति॑ष्मन्ता वस्मै । \newline
24. ज्योति॑ष्मन्ता वस्मा अस्मै॒ ज्योति॑ष्मन्तौ॒ ज्योति॑ष्मन्ता वस्मा इ॒मा वि॒मा व॑स्मै॒ ज्योति॑ष्मन्तौ॒ ज्योति॑ष्मन्ता वस्मा इ॒मौ । \newline
25. अ॒स्मा॒ इ॒मा वि॒मा व॑स्मा अस्मा इ॒मौ लो॒कौ लो॒का वि॒मा व॑स्मा अस्मा इ॒मौ लो॒कौ । \newline
26. इ॒मौ लो॒कौ लो॒का वि॒मा वि॒मौ लो॒कौ भ॑वतो भवतो लो॒का वि॒मा वि॒मौ लो॒कौ भ॑वतः । \newline
27. लो॒कौ भ॑वतो भवतो लो॒कौ लो॒कौ भ॑वतो॒ यो यो भ॑वतो लो॒कौ लो॒कौ भ॑वतो॒ यः । \newline
28. भ॒व॒तो॒ यो यो भ॑वतो भवतो॒ य ए॒व मे॒वं ॅयो भ॑वतो भवतो॒ य ए॒वम् । \newline
29. य ए॒व मे॒वं ॅयो य ए॒वं ॅवेद॒ वेदै॒वं ॅयो य ए॒वं ॅवेद॑ । \newline
30. ए॒वं ॅवेद॒ वेदै॒व मे॒वं ॅवेद॑ पु॒रस्ता᳚ल्लक्ष्मा पु॒रस्ता᳚ल्लक्ष्मा॒ वेदै॒व मे॒वं ॅवेद॑ पु॒रस्ता᳚ल्लक्ष्मा । \newline
31. वेद॑ पु॒रस्ता᳚ल्लक्ष्मा पु॒रस्ता᳚ल्लक्ष्मा॒ वेद॒ वेद॑ पु॒रस्ता᳚ल्लक्ष्मा पुरोनुवा॒क्या॑ पुरोनुवा॒क्या॑ पु॒रस्ता᳚ल्लक्ष्मा॒ वेद॒ वेद॑ पु॒रस्ता᳚ल्लक्ष्मा पुरोनुवा॒क्या᳚ । \newline
32. पु॒रस्ता᳚ल्लक्ष्मा पुरोनुवा॒क्या॑ पुरोनुवा॒क्या॑ पु॒रस्ता᳚ल्लक्ष्मा पु॒रस्ता᳚ल्लक्ष्मा पुरोनुवा॒क्या॑ भवति भवति पुरोनुवा॒क्या॑ पु॒रस्ता᳚ल्लक्ष्मा पु॒रस्ता᳚ल्लक्ष्मा पुरोनुवा॒क्या॑ भवति । \newline
33. पु॒रस्ता᳚ल्ल॒क्ष्मेति॑ पु॒रस्ता᳚त् - ल॒क्ष्मा॒ । \newline
34. पु॒रो॒नु॒वा॒क्या॑ भवति भवति पुरोनुवा॒क्या॑ पुरोनुवा॒क्या॑ भवति॒ तस्मा॒त् तस्मा᳚द् भवति पुरोनुवा॒क्या॑ पुरोनुवा॒क्या॑ भवति॒ तस्मा᳚त् । \newline
35. पु॒रो॒नु॒वा॒क्येति॑ पुरः - अ॒नु॒वा॒क्या᳚ । \newline
36. भ॒व॒ति॒ तस्मा॒त् तस्मा᳚द् भवति भवति॒ तस्मा᳚त् पूर्वा॒र्द्धेन॑ पूर्वा॒र्द्धेन॒ तस्मा᳚द् भवति भवति॒ तस्मा᳚त् पूर्वा॒र्द्धेन॑ । \newline
37. तस्मा᳚त् पूर्वा॒र्द्धेन॑ पूर्वा॒र्द्धेन॒ तस्मा॒त् तस्मा᳚त् पूर्वा॒र्द्धेना॑ न॒ड्वा न॑न॒ड्वान् पू᳚र्वा॒र्द्धेन॒ तस्मा॒त् तस्मा᳚त् पूर्वा॒र्द्धेना॑ न॒ड्वान् । \newline
38. पू॒र्वा॒र्द्धेना॑ न॒ड्वा न॑न॒ड्वान् पू᳚र्वा॒र्द्धेन॑ पूर्वा॒र्द्धेना॑ न॒ड्वान् भु॑नक्ति भुनक्त्य न॒ड्वान् पू᳚र्वा॒र्द्धेन॑ पूर्वा॒र्द्धेना॑ न॒ड्वान् भु॑नक्ति । \newline
39. पू॒र्वा॒र्द्धेनेति॑ पूर्व - अ॒र्द्धेन॑ । \newline
40. अ॒न॒ड्वान् भु॑नक्ति भुनक्त्य न॒ड्वा न॑न॒ड्वान् भु॑नक्त्यु॒परि॑ष्टाल्लक्ष्मो॒ परि॑ष्टाल्लक्ष्मा भुनक्त्य न॒ड्वा न॑न॒ड्वान् भु॑नक्त्यु॒परि॑ष्टाल्लक्ष्मा । \newline
41. भु॒न॒क्त्यु॒परि॑ष्टाल्ल क्ष्मो॒परि॑ष्टाल्लक्ष्मा भुनक्ति भुनक्त्यु॒परि॑ष्टाल्लक्ष्मा या॒ज्या॑ या॒ज्यो॑परि॑ष्टाल्लक्ष्मा भुनक्ति भुनक्त्यु॒परि॑ष्टाल्लक्ष्मा या॒ज्या᳚ । \newline
42. उ॒परि॑ष्टाल्लक्ष्मा या॒ज्या॑ या॒ज्यो॑परि॑ष्टाल्ल क्ष्मो॒परि॑ष्टाल्लक्ष्मा या॒ज्या॑ तस्मा॒त् तस्मा᳚द् या॒ज्यो॑परि॑ष्टाल्ल क्ष्मो॒परि॑ष्टाल्लक्ष्मा या॒ज्या॑ तस्मा᳚त् । \newline
43. उ॒परि॑ष्टाल्ल॒क्ष्मेत्यु॒परि॑ष्टात् - ल॒क्ष्मा॒ । \newline
44. या॒ज्या॑ तस्मा॒त् तस्मा᳚द् या॒ज्या॑ या॒ज्या॑ तस्मा᳚ज् जघना॒र्द्धेन॑ जघना॒र्द्धेन॒ तस्मा᳚द् या॒ज्या॑ या॒ज्या॑ तस्मा᳚ज् जघना॒र्द्धेन॑ । \newline
45. तस्मा᳚ज् जघना॒र्द्धेन॑ जघना॒र्द्धेन॒ तस्मा॒त् तस्मा᳚ज् जघना॒र्द्धेन॑ धे॒नुर् धे॒नुर् ज॑घना॒र्द्धेन॒ तस्मा॒त् तस्मा᳚ज् जघना॒र्द्धेन॑ धे॒नुः । \newline
46. ज॒घ॒ना॒र्द्धेन॑ धे॒नुर् धे॒नुर् ज॑घना॒र्द्धेन॑ जघना॒र्द्धेन॑ धे॒नुर् यो यो धे॒नुर् ज॑घना॒र्द्धेन॑ जघना॒र्द्धेन॑ धे॒नुर् यः । \newline
47. ज॒घ॒ना॒र्द्धेनेति॑ जघन - अ॒र्द्धेन॑ । \newline
48. धे॒नुर् यो यो धे॒नुर् धे॒नुर् य ए॒व मे॒वं ॅयो धे॒नुर् धे॒नुर् य ए॒वम् । \newline
49. य ए॒व मे॒वं ॅयो य ए॒वं ॅवेद॒ वेदै॒वं ॅयो य ए॒वं ॅवेद॑ । \newline
50. ए॒वं ॅवेद॒ वेदै॒व मे॒वं ॅवेद॑ भु॒ङ्क्तो भु॒ङ्क्तो वेदै॒व मे॒वं ॅवेद॑ भु॒ङ्क्तः । \newline
51. वेद॑ भु॒ङ्क्तो भु॒ङ्क्तो वेद॒ वेद॑ भु॒ङ्क्त ए॑न मेनम् भु॒ङ्क्तो वेद॒ वेद॑ भु॒ङ्क्त ए॑नम् । \newline
52. भु॒ङ्क्त ए॑न मेनम् भु॒ङ्क्तो भु॒ङ्क्त ए॑न मे॒ता वे॒ता वे॑नम् भु॒ङ्क्तो भु॒ङ्क्त ए॑न मे॒तौ । \newline
53. ए॒न॒ मे॒ता वे॒ता वे॑न मेन मे॒तौ वज्रो॒ वज्र॑ ए॒ता वे॑न मेन मे॒तौ वज्रः॑ । \newline
54. ए॒तौ वज्रो॒ वज्र॑ ए॒ता वे॒तौ वज्र॒ आज्य॒ माज्यं॒ ॅवज्र॑ ए॒ता वे॒तौ वज्र॒ आज्य᳚म् । \newline
55. वज्र॒ आज्य॒ माज्यं॒ ॅवज्रो॒ वज्र॒ आज्यं॒ ॅवज्रो॒ वज्र॒ आज्यं॒ ॅवज्रो॒ वज्र॒ आज्यं॒ ॅवज्रः॑ । \newline
56. आज्यं॒ ॅवज्रो॒ वज्र॒ आज्य॒ माज्यं॒ ॅवज्र॒ आज्य॑भागा॒ वाज्य॑भागौ॒ वज्र॒ आज्य॒ माज्यं॒ ॅवज्र॒ आज्य॑भागौ । \newline
57. वज्र॒ आज्य॑भागा॒ वाज्य॑भागौ॒ वज्रो॒ वज्र॒ आज्य॑भागौ॒ वज्रो॒ वज्र॒ आज्य॑भागौ॒ वज्रो॒ वज्र॒ आज्य॑भागौ॒ वज्रः॑ । \newline
58. आज्य॑भागौ॒ वज्रो॒ वज्र॒ आज्य॑भागा॒ वाज्य॑भागौ॒ वज्रो॑ वषट्का॒रो व॑षट्का॒रो वज्र॒ आज्य॑भागा॒ वाज्य॑भागौ॒ वज्रो॑ वषट्का॒रः । \newline
59. आज्य॑भागा॒वित्याज्य॑ - भा॒गौ॒ । \newline
\pagebreak
\markright{ TS 2.6.2.5  \hfill https://www.vedavms.in \hfill}
\addcontentsline{toc}{section}{ TS 2.6.2.5 }
\section*{ TS 2.6.2.5 }

\textbf{TS 2.6.2.5 } \newline
\textbf{Samhita Paata} \newline

वज्रो॑ वषट्का॒रस्त्रि॒वृत॑मे॒व वज्रꣳ॑ स॒म्भृत्य॒ भ्रातृ॑व्याय॒ प्र ह॑र॒त्यच्छ॑म्बट्कार-मप॒गूर्य॒ वष॑ट्करोति॒ स्तृत्यै॑ गाय॒त्री पु॑रोऽनुवा॒क्या॑ भवति त्रि॒ष्टुग् या॒ज्या᳚ ब्रह्म॑न्ने॒व क्ष॒त्रम॒न्वारं॑ भयति॒ तस्मा᳚द्ब्राह्म॒णो मुख्यो॒ मुख्यो॑ भवति॒ य ए॒वं ॅवेद॒ प्रैवैनं॑ पुरोऽनुवा॒क्य॑या ऽऽह॒ प्रण॑यति या॒ज्य॑या ग॒मय॑ति वषट्का॒रेणैवैनं॑ पुरोऽनुवा॒क्य॑या दत्ते॒ प्रय॑च्छति या॒ज्य॑या॒ प्रति॑ - [  ] \newline

\textbf{Pada Paata} \newline

वज्रः॑ । व॒ष॒ट्का॒र इति॑ वषट् - का॒रः । त्रि॒वृत॒मिति॑ त्रि - वृत᳚म् । ए॒व । वज्र᳚म् । स॒भृंत्येति॑ सं - भृत्य॑ । भ्रातृ॑व्याय । प्रेति॑ । ह॒र॒ति॒ । अछ॑बंट्कार॒मित्यछ॑बंट् - का॒र॒म् । अ॒प॒गूर्येत्य॑प - गूर्य॑ । वष॑ट् । क॒रो॒ति॒ । स्तृत्यै᳚ । गा॒य॒त्री । पु॒रो॒नु॒वा॒क्येति॑ पुरः - अ॒नु॒वा॒क्या᳚ । भ॒व॒ति॒ । त्रि॒ष्टुक् । या॒ज्या᳚ । ब्रह्मन्न्॑ । ए॒व । क्ष॒त्रम् । अ॒न्वार॑भंय॒तीत्य॑नु - आर॑भंयति । तस्मा᳚त् । ब्रा॒ह्म॒णः । मुख्यः॑ । मुख्यः॑ । भ॒व॒ति॒ । यः । ए॒वम् । वेद॑ । प्रेति॑ । ए॒व । ए॒न॒म् । पु॒रो॒नु॒वा॒क्य॑येति॑ पुरः - अ॒नु॒वा॒क्य॑या । आ॒ह॒ । प्रेति॑ । न॒य॒ति॒ । या॒ज्य॑या । ग॒मय॑ति । व॒ष॒ट्का॒रेणेति॑ वषट् - का॒रेण॑ । एति॑ । ए॒व । ए॒न॒म् । पु॒रो॒नु॒वा॒क्य॑येति॑ पुरः-अ॒नु॒वा॒क्य॑या । द॒त्ते॒ । प्रेति॑ । य॒च्छ॒ति॒ । या॒ज्य॑या । प्रतीति॑ ।  \newline


\textbf{Krama Paata} \newline

वज्रो॑ वषट्का॒रः । व॒ष॒ट्का॒रस्त्रि॒वृत᳚म् । व॒ष॒ट्का॒र इति॑ वषट् - का॒रः । त्रि॒वृत॑मे॒व । त्रि॒वृत॒मिति॑ त्रि - वृत᳚म् । ए॒व वज्र᳚म् । वज्रꣳ॑ सं॒भृत्य॑ । सं॒भृत्य॒ भ्रातृ॑व्याय । सं॒भृत्येति॑ सं - भृत्य॑ । भ्रातृ॑व्याय॒ प्र । प्र ह॑रति । ह॒र॒त्यछ॑म्बट्कारम् । अछ॑म्बट्कारमप॒गूर्य॑ । अछ॑म्बट्कार॒मित्यछ॑म्बट् - का॒र॒म् । अ॒प॒गूर्य॒ वष॑ट् । अ॒प॒गूर्येत्य॑प - गूर्य॑ । वष॑ट् करोति । क॒रो॒ति॒ स्तृत्यै᳚ । स्तृत्यै॑ गाय॒त्री । गा॒य॒त्री पु॑रोनुवा॒क्या᳚ । पु॒रो॒नु॒वा॒क्या॑ भवति । पु॒रो॒नु॒वा॒क्येति॑ पुरः - अ॒नु॒वा॒क्या᳚ । भ॒व॒ति॒ त्रि॒ष्टुक् । त्रि॒ष्टुग् या॒ज्या᳚ । या॒ज्या᳚ ब्रह्मन्न्॑ । ब्रह्म॑न्ने॒व । ए॒व क्ष॒त्रम् । क्ष॒त्रम॒न्वारं॑भयति । अ॒न्वारं॑भयति॒ तस्मा᳚त् । अ॒न्वारं॑भय॒तीत्य॑नु - आरं॑भयति । तस्मा᳚द् ब्राह्म॒णः । ब्रा॒ह्म॒णो मुख्यः॑ । मुख्यो॒ मुख्यः॑ । मुख्यो॑ भवति । भ॒व॒ति॒ यः । य ए॒वम् । ए॒वं ॅवेद॑ । वेद॒ प्र । प्रैव । ए॒वैन᳚म् । ए॒न॒म् पु॒रो॒नु॒वा॒क्य॑या । पु॒रो॒नु॒वा॒क्य॑याऽऽह । पु॒रो॒नु॒वा॒क्य॑येति॑ पुरः - अ॒नु॒वा॒क्य॑या । आ॒ह॒ प्र । प्र ण॑यति । न॒य॒ति॒ या॒ज्य॑या । या॒ज्य॑या ग॒मय॑ति । ग॒मय॑ति वषट्का॒रेण॑ । व॒ष॒ट्का॒रेणा । व॒ष॒ट्का॒रेणेति॑ वषट् - का॒रेण॑ । ऐव । ए॒वैन᳚म् । ए॒न॒म् पु॒रो॒नु॒वा॒क्य॑या । पु॒रो॒नु॒वा॒क्य॑या दत्ते । पु॒रो॒नु॒वा॒क्य॑येति॑ पुरः - अ॒नु॒वा॒क्य॑या । द॒त्ते॒ प्र । प्र य॑च्छति । य॒च्छ॒ति॒ या॒ज्य॑या । या॒ज्य॑या॒ प्रति॑ । प्रति॑ वषट्का॒रेण॑ \newline

\textbf{Jatai Paata} \newline

1. वज्रो॑ वषट्का॒रो व॑षट्का॒रो वज्रो॒ वज्रो॑ वषट्का॒रः । \newline
2. व॒ष॒ट्का॒र स्त्रि॒वृत॑म् त्रि॒वृतं॑ ॅवषट्का॒रो व॑षट्का॒र स्त्रि॒वृत᳚म् । \newline
3. व॒ष॒ट्का॒र इति॑ वषट् - का॒रः । \newline
4. त्रि॒वृत॑ मे॒वैव त्रि॒वृत॑म् त्रि॒वृत॑ मे॒व । \newline
5. त्रि॒वृत॒मिति॑ त्रि - वृत᳚म् । \newline
6. ए॒व वज्रं॒ ॅवज्र॑ मे॒वैव वज्र᳚म् । \newline
7. वज्रꣳ॑ सं॒भृत्य॑ सं॒भृत्य॒ वज्रं॒ ॅवज्रꣳ॑ सं॒भृत्य॑ । \newline
8. सं॒भृत्य॒ भ्रातृ॑व्याय॒ भ्रातृ॑व्याय सं॒भृत्य॑ सं॒भृत्य॒ भ्रातृ॑व्याय । \newline
9. सं॒भृत्येति॑ सं - भृत्य॑ । \newline
10. भ्रातृ॑व्याय॒ प्र प्र भ्रातृ॑व्याय॒ भ्रातृ॑व्याय॒ प्र । \newline
11. प्र ह॑रति हरति॒ प्र प्र ह॑रति । \newline
12. ह॒र॒ त्यछं॑बट्कार॒ मछं॑बट्कारꣳ हरति हर॒ त्यछं॑बट्कारम् । \newline
13. अछं॑बट्कार मप॒गूर्या॑ प॒गूर्या छं॑बट्कार॒ मछं॑बट्कार मप॒गूर्य॑ । \newline
14. अछं॑बट्कार॒मित्यछं॑बट् - का॒र॒म् । \newline
15. अ॒प॒गूर्य॒ वष॒ड् वष॑ डप॒गूर्या॑ प॒गूर्य॒ वष॑ट् । \newline
16. अ॒प॒गूर्येत्य॑प - गूर्य॑ । \newline
17. वष॑ट् करोति करोति॒ वष॒ड् वष॑ट् करोति । \newline
18. क॒रो॒ति॒ स्तृत्यै॒ स्तृत्यै॑ करोति करोति॒ स्तृत्यै᳚ । \newline
19. स्तृत्यै॑ गाय॒त्री गा॑य॒त्री स्तृत्यै॒ स्तृत्यै॑ गाय॒त्री । \newline
20. गा॒य॒त्री पु॑रोनुवा॒क्या॑ पुरोनुवा॒क्या॑ गाय॒त्री गा॑य॒त्री पु॑रोनुवा॒क्या᳚ । \newline
21. पु॒रो॒नु॒वा॒क्या॑ भवति भवति पुरोनुवा॒क्या॑ पुरोनुवा॒क्या॑ भवति । \newline
22. पु॒रो॒नु॒वा॒क्येति॑ पुरः - अ॒नु॒वा॒क्या᳚ । \newline
23. भ॒व॒ति॒ त्रि॒ष्टुक् त्रि॒ष्टुग् भ॑वति भवति त्रि॒ष्टुक् । \newline
24. त्रि॒ष्टुग् या॒ज्या॑ या॒ज्या᳚ त्रि॒ष्टुक् त्रि॒ष्टुग् या॒ज्या᳚ । \newline
25. या॒ज्या᳚ ब्रह्म॒न् ब्रह्म॑न्. या॒ज्या॑ या॒ज्या᳚ ब्रह्मन्न्॑ । \newline
26. ब्रह्म॑न् ने॒वैव ब्रह्म॒न् ब्रह्म॑न् ने॒व । \newline
27. ए॒व क्ष॒त्रम् क्ष॒त्र मे॒वैव क्ष॒त्रम् । \newline
28. क्ष॒त्र म॒न्वार॑म्भय त्य॒न्वार॑म्भयति क्ष॒त्रम् क्ष॒त्र म॒न्वार॑म्भयति । \newline
29. अ॒न्वार॑म्भयति॒ तस्मा॒त् तस्मा॑ द॒न्वार॑म्भय त्य॒न्वार॑म्भयति॒ तस्मा᳚त् । \newline
30. अ॒न्वार॑म्भय॒तीत्य॑नु - आर॑म्भयति । \newline
31. तस्मा᳚द् ब्राह्म॒णो ब्रा᳚ह्म॒ण स्तस्मा॒त् तस्मा᳚द् ब्राह्म॒णः । \newline
32. ब्रा॒ह्म॒णो मुख्यो॒ मुख्यो᳚ ब्राह्म॒णो ब्रा᳚ह्म॒णो मुख्यः॑ । \newline
33. मुख्यो॒ मुख्यः॑ । \newline
34. मुख्यो॑ भवति भवति॒ मुख्यो॒ मुख्यो॑ भवति । \newline
35. भ॒व॒ति॒ यो यो भ॑वति भवति॒ यः । \newline
36. य ए॒व मे॒वं ॅयो य ए॒वम् । \newline
37. ए॒वं ॅवेद॒ वेदै॒व मे॒वं ॅवेद॑ । \newline
38. वेद॒ प्र प्र वेद॒ वेद॒ प्र । \newline
39. प्रैवैव प्र प्रैव । \newline
40. ए॒वैन॑ मेन मे॒वैवैन᳚म् । \newline
41. ए॒न॒म् पु॒रो॒नु॒वा॒क्य॑या पुरोनुवा॒क्य॑यैन मेनम् पुरोनुवा॒क्य॑या । \newline
42. पु॒रो॒नु॒वा॒क्य॑या ऽऽहाह पुरोनुवा॒क्य॑या पुरोनुवा॒क्य॑या ऽऽह । \newline
43. पु॒रो॒नु॒वा॒क्य॑येति॑ पुरः - अ॒नु॒वा॒क्य॑या । \newline
44. आ॒ह॒ प्र प्राहा॑ह॒ प्र । \newline
45. प्र ण॑यति नयति॒ प्र प्र ण॑यति । \newline
46. न॒य॒ति॒ या॒ज्य॑या या॒ज्य॑या नयति नयति या॒ज्य॑या । \newline
47. या॒ज्य॑या ग॒मय॑ति ग॒मय॑ति या॒ज्य॑या या॒ज्य॑या ग॒मय॑ति । \newline
48. ग॒मय॑ति वषट्का॒रेण॑ वषट्का॒रेण॑ ग॒मय॑ति ग॒मय॑ति वषट्का॒रेण॑ । \newline
49. व॒ष॒ट्का॒रेणा व॑षट्का॒रेण॑ वषट्का॒रेणा । \newline
50. व॒ष॒ट्का॒रेणेति॑ वषट् - का॒रेण॑ । \newline
51. ऐवैवैव । \newline
52. ए॒वैन॑ मेन मे॒वैवैन᳚म् । \newline
53. ए॒न॒म् पु॒रो॒नु॒वा॒क्य॑या पुरोनुवा॒क्य॑यैन मेनम् पुरोनुवा॒क्य॑या । \newline
54. पु॒रो॒नु॒वा॒क्य॑या दत्ते दत्ते पुरोनुवा॒क्य॑या पुरोनुवा॒क्य॑या दत्ते । \newline
55. पु॒रो॒नु॒वा॒क्य॑येति॑ पुरः - अ॒नु॒वा॒क्य॑या । \newline
56. द॒त्ते॒ प्र प्र द॑त्ते दत्ते॒ प्र । \newline
57. प्र य॑च्छति यच्छति॒ प्र प्र य॑च्छति । \newline
58. य॒च्छ॒ति॒ या॒ज्य॑या या॒ज्य॑या यच्छति यच्छति या॒ज्य॑या । \newline
59. या॒ज्य॑या॒ प्रति॒ प्रति॑ या॒ज्य॑या या॒ज्य॑या॒ प्रति॑ । \newline
60. प्रति॑ वषट्का॒रेण॑ वषट्का॒रेण॒ प्रति॒ प्रति॑ वषट्का॒रेण॑ । \newline

\textbf{Ghana Paata } \newline

1. वज्रो॑ वषट्का॒रो व॑षट्का॒रो वज्रो॒ वज्रो॑ वषट्का॒र स्त्रि॒वृत॑म् त्रि॒वृतं॑ ॅवषट्का॒रो वज्रो॒ वज्रो॑ वषट्का॒र स्त्रि॒वृत᳚म् । \newline
2. व॒ष॒ट्का॒र स्त्रि॒वृत॑म् त्रि॒वृतं॑ ॅवषट्का॒रो व॑षट्का॒र स्त्रि॒वृत॑ मे॒वैव त्रि॒वृतं॑ ॅवषट्का॒रो व॑षट्का॒र स्त्रि॒वृत॑ मे॒व । \newline
3. व॒ष॒ट्का॒र इति॑ वषट् - का॒रः । \newline
4. त्रि॒वृत॑ मे॒वैव त्रि॒वृत॑म् त्रि॒वृत॑ मे॒व वज्रं॒ ॅवज्र॑ मे॒व त्रि॒वृत॑म् त्रि॒वृत॑ मे॒व वज्र᳚म् । \newline
5. त्रि॒वृत॒मिति॑ त्रि - वृत᳚म् । \newline
6. ए॒व वज्रं॒ ॅवज्र॑ मे॒वैव वज्रꣳ॑ सं॒भृत्य॑ सं॒भृत्य॒ वज्र॑ मे॒वैव वज्रꣳ॑ सं॒भृत्य॑ । \newline
7. वज्रꣳ॑ सं॒भृत्य॑ सं॒भृत्य॒ वज्रं॒ ॅवज्रꣳ॑ सं॒भृत्य॒ भ्रातृ॑व्याय॒ भ्रातृ॑व्याय सं॒भृत्य॒ वज्रं॒ ॅवज्रꣳ॑ सं॒भृत्य॒ भ्रातृ॑व्याय । \newline
8. सं॒भृत्य॒ भ्रातृ॑व्याय॒ भ्रातृ॑व्याय सं॒भृत्य॑ सं॒भृत्य॒ भ्रातृ॑व्याय॒ प्र प्र भ्रातृ॑व्याय सं॒भृत्य॑ सं॒भृत्य॒ भ्रातृ॑व्याय॒ प्र । \newline
9. सं॒भृत्येति॑ सं - भृत्य॑ । \newline
10. भ्रातृ॑व्याय॒ प्र प्र भ्रातृ॑व्याय॒ भ्रातृ॑व्याय॒ प्र ह॑रति हरति॒ प्र भ्रातृ॑व्याय॒ भ्रातृ॑व्याय॒ प्र ह॑रति । \newline
11. प्र ह॑रति हरति॒ प्र प्र ह॑र॒त्यछं॑बट्कार॒ मछं॑बट्कारꣳ हरति॒ प्र प्र ह॑र॒त्यछं॑बट्कारम् । \newline
12. ह॒र॒त्यछं॑बट्कार॒ मछं॑बट्कारꣳ हरति हर॒त्यछं॑बट्कार मप॒गूर्या॑प॒गूर्या छं॑बट्कारꣳ हरति हर॒त्यछं॑बट्कार मप॒गूर्य॑ । \newline
13. अछं॑बट्कार मप॒गूर्या॑ प॒गूर्या छं॑बट्कार॒ मछं॑बट्कार मप॒गूर्य॒ वष॒ड् वष॑डप॒गूर्या छं॑बट्कार॒ मछं॑बट्कार मप॒गूर्य॒ वष॑ट् । \newline
14. अछं॑बट्कार॒मित्यछं॑बट् - का॒र॒म् । \newline
15. अ॒प॒गूर्य॒ वष॒ड् वष॑डप॒गूर्या॑ प॒गूर्य॒ वष॑ट् करोति करोति॒ वष॑ड प॒गूर्या॑प॒गूर्य॒ वष॑ट् करोति । \newline
16. अ॒प॒गूर्येत्य॑प - गूर्य॑ । \newline
17. वष॑ट् करोति करोति॒ वष॒ड् वष॑ट् करोति॒ स्तृत्यै॒ स्तृत्यै॑ करोति॒ वष॒ड् वष॑ट् करोति॒ स्तृत्यै᳚ । \newline
18. क॒रो॒ति॒ स्तृत्यै॒ स्तृत्यै॑ करोति करोति॒ स्तृत्यै॑ गाय॒त्री गा॑य॒त्री स्तृत्यै॑ करोति करोति॒ स्तृत्यै॑ गाय॒त्री । \newline
19. स्तृत्यै॑ गाय॒त्री गा॑य॒त्री स्तृत्यै॒ स्तृत्यै॑ गाय॒त्री पु॑रोनुवा॒क्या॑ पुरोनुवा॒क्या॑ गाय॒त्री स्तृत्यै॒ स्तृत्यै॑ गाय॒त्री पु॑रोनुवा॒क्या᳚ । \newline
20. गा॒य॒त्री पु॑रोनुवा॒क्या॑ पुरोनुवा॒क्या॑ गाय॒त्री गा॑य॒त्री पु॑रोनुवा॒क्या॑ भवति भवति पुरोनुवा॒क्या॑ गाय॒त्री गा॑य॒त्री पु॑रोनुवा॒क्या॑ भवति । \newline
21. पु॒रो॒नु॒वा॒क्या॑ भवति भवति पुरोनुवा॒क्या॑ पुरोनुवा॒क्या॑ भवति त्रि॒ष्टुक् त्रि॒ष्टुग् भ॑वति पुरोनुवा॒क्या॑ पुरोनुवा॒क्या॑ भवति त्रि॒ष्टुक् । \newline
22. पु॒रो॒नु॒वा॒क्येति॑ पुरः - अ॒नु॒वा॒क्या᳚ । \newline
23. भ॒व॒ति॒ त्रि॒ष्टुक् त्रि॒ष्टुग् भ॑वति भवति त्रि॒ष्टुग् या॒ज्या॑ या॒ज्या᳚ त्रि॒ष्टुग् भ॑वति भवति त्रि॒ष्टुग् या॒ज्या᳚ । \newline
24. त्रि॒ष्टुग् या॒ज्या॑ या॒ज्या᳚ त्रि॒ष्टुक् त्रि॒ष्टुग् या॒ज्या᳚ ब्रह्म॒न् ब्रह्मन्॑. या॒ज्या᳚ त्रि॒ष्टुक् त्रि॒ष्टुग् या॒ज्या᳚ ब्रह्मन्न्॑ । \newline
25. या॒ज्या᳚ ब्रह्म॒न् ब्रह्मन्॑. या॒ज्या॑ या॒ज्या᳚ ब्रह्म॑न् ने॒वैव ब्रह्मन्॑. या॒ज्या॑ या॒ज्या᳚ ब्रह्म॑न् ने॒व । \newline
26. ब्रह्म॑न् ने॒वैव ब्रह्म॒न् ब्रह्म॑न् ने॒व क्ष॒त्रम् क्ष॒त्र मे॒व ब्रह्म॒न् ब्रह्म॑न् ने॒व क्ष॒त्रम् । \newline
27. ए॒व क्ष॒त्रम् क्ष॒त्र मे॒वैव क्ष॒त्र म॒न्वार॑म्भय त्य॒न्वार॑म्भयति क्ष॒त्र मे॒वैव क्ष॒त्र म॒न्वार॑म्भयति । \newline
28. क्ष॒त्र म॒न्वार॑म्भय त्य॒न्वार॑म्भयति क्ष॒त्रम् क्ष॒त्र म॒न्वार॑म्भयति॒ तस्मा॒त् तस्मा॑ द॒न्वार॑म्भयति क्ष॒त्रम् क्ष॒त्र म॒न्वार॑म्भयति॒ तस्मा᳚त् । \newline
29. अ॒न्वार॑म्भयति॒ तस्मा॒त् तस्मा॑ द॒न्वार॑म्भय त्य॒न्वार॑म्भयति॒ तस्मा᳚द् ब्राह्म॒णो 
ब्रा᳚ह्म॒ण स्तस्मा॑ द॒न्वार॑म्भय त्य॒न्वार॑म्भयति॒ तस्मा᳚द् ब्राह्म॒णः । \newline
30. अ॒न्वार॑म्भय॒तीत्य॑नु - आर॑म्भयति । \newline
31. तस्मा᳚द् ब्राह्म॒णो ब्रा᳚ह्म॒ण स्तस्मा॒त् तस्मा᳚द् ब्राह्म॒णो मुख्यो॒ मुख्यो᳚ ब्राह्म॒ण स्तस्मा॒त् तस्मा᳚द् ब्राह्म॒णो मुख्यः॑ । \newline
32. ब्रा॒ह्म॒णो मुख्यो॒ मुख्यो᳚ ब्राह्म॒णो ब्रा᳚ह्म॒णो मुख्यः॑ । \newline
33. मुख्यो॒ मुख्यः॑ । \newline
34. मुख्यो॑ भवति भवति॒ मुख्यो॒ मुख्यो॑ भवति॒ यो यो भ॑वति॒ मुख्यो॒ मुख्यो॑ भवति॒ यः । \newline
35. भ॒व॒ति॒ यो यो भ॑वति भवति॒ य ए॒व मे॒वं ॅयो भ॑वति भवति॒ य ए॒वम् । \newline
36. य ए॒व मे॒वं ॅयो य ए॒वं ॅवेद॒ वेदै॒वं ॅयो य ए॒वं ॅवेद॑ । \newline
37. ए॒वं ॅवेद॒ वेदै॒व मे॒वं ॅवेद॒ प्र प्र वेदै॒व मे॒वं ॅवेद॒ प्र । \newline
38. वेद॒ प्र प्र वेद॒ वेद॒ प्रैवैव प्र वेद॒ वेद॒ प्रैव । \newline
39. प्रैवैव प्र प्रैवैन॑ मेन मे॒व प्र प्रैवैन᳚म् । \newline
40. ए॒वैन॑ मेन मे॒वैवैन॑म् पुरोनुवा॒क्य॑या पुरोनुवा॒क्य॑यैन मे॒वैवैन॑म् पुरोनुवा॒क्य॑या । \newline
41. ए॒न॒म् पु॒रो॒नु॒वा॒क्य॑या पुरोनुवा॒क्य॑यैन मेनम् पुरोनुवा॒क्य॑या ऽऽहाह पुरोनुवा॒क्य॑यैन मेनम् पुरोनुवा॒क्य॑या ऽऽह । \newline
42. पु॒रो॒नु॒वा॒क्य॑या ऽऽहाह पुरोनुवा॒क्य॑या पुरोनुवा॒क्य॑या ऽऽह॒ प्र प्राह॑ पुरोनुवा॒क्य॑या पुरोनुवा॒क्य॑या ऽऽह॒ प्र । \newline
43. पु॒रो॒नु॒वा॒क्य॑येति॑ पुरः - अ॒नु॒वा॒क्य॑या । \newline
44. आ॒ह॒ प्र प्राहा॑ह॒ प्र ण॑यति नयति॒ प्राहा॑ह॒ प्र ण॑यति । \newline
45. प्र ण॑यति नयति॒ प्र प्र ण॑यति या॒ज्य॑या या॒ज्य॑या नयति॒ प्र प्र ण॑यति या॒ज्य॑या । \newline
46. न॒य॒ति॒ या॒ज्य॑या या॒ज्य॑या नयति नयति या॒ज्य॑या ग॒मय॑ति ग॒मय॑ति या॒ज्य॑या नयति नयति या॒ज्य॑या ग॒मय॑ति । \newline
47. या॒ज्य॑या ग॒मय॑ति ग॒मय॑ति या॒ज्य॑या या॒ज्य॑या ग॒मय॑ति वषट्का॒रेण॑ वषट्का॒रेण॑ ग॒मय॑ति या॒ज्य॑या या॒ज्य॑या ग॒मय॑ति वषट्का॒रेण॑ । \newline
48. ग॒मय॑ति वषट्का॒रेण॑ वषट्का॒रेण॑ ग॒मय॑ति ग॒मय॑ति वषट्का॒रेणा व॑षट्का॒रेण॑ ग॒मय॑ति ग॒मय॑ति वषट्का॒रेणा । \newline
49. व॒ष॒ट्का॒रेणा व॑षट्का॒रेण॑ वषट्का॒रेणैवैवा व॑षट्का॒रेण॑ वषट्का॒रेणैव । \newline
50. व॒ष॒ट्का॒रेणेति॑ वषट् - का॒रेण॑ । \newline
51. ऐवै वैवैन॑ मेन मे॒वैवैन᳚म् । \newline
52. ए॒वैन॑ मेन मे॒वैवैन॑म् पुरोनुवा॒क्य॑या पुरोनुवा॒क्य॑यैन मे॒वैवैन॑म् पुरोनुवा॒क्य॑या । \newline
53. ए॒न॒म् पु॒रो॒नु॒वा॒क्य॑या पुरोनुवा॒क्य॑यैन मेनम् पुरोनुवा॒क्य॑या दत्ते दत्ते पुरोनुवा॒क्य॑यैन मेनम् पुरोनुवा॒क्य॑या दत्ते । \newline
54. पु॒रो॒नु॒वा॒क्य॑या दत्ते दत्ते पुरोनुवा॒क्य॑या पुरोनुवा॒क्य॑या दत्ते॒ प्र प्र द॑त्ते पुरोनुवा॒क्य॑या पुरोनुवा॒क्य॑या दत्ते॒ प्र । \newline
55. पु॒रो॒नु॒वा॒क्य॑येति॑ पुरः - अ॒नु॒वा॒क्य॑या । \newline
56. द॒त्ते॒ प्र प्र द॑त्ते दत्ते॒ प्र य॑च्छति यच्छति॒ प्र द॑त्ते दत्ते॒ प्र य॑च्छति । \newline
57. प्र य॑च्छति यच्छति॒ प्र प्र य॑च्छति या॒ज्य॑या या॒ज्य॑या यच्छति॒ प्र प्र य॑च्छति या॒ज्य॑या । \newline
58. य॒च्छ॒ति॒ या॒ज्य॑या या॒ज्य॑या यच्छति यच्छति या॒ज्य॑या॒ प्रति॒ प्रति॑ या॒ज्य॑या यच्छति यच्छति या॒ज्य॑या॒ प्रति॑ । \newline
59. या॒ज्य॑या॒ प्रति॒ प्रति॑ या॒ज्य॑या या॒ज्य॑या॒ प्रति॑ वषट्का॒रेण॑ वषट्का॒रेण॒ प्रति॑ या॒ज्य॑या या॒ज्य॑या॒ प्रति॑ वषट्का॒रेण॑ । \newline
60. प्रति॑ वषट्का॒रेण॑ वषट्का॒रेण॒ प्रति॒ प्रति॑ वषट्का॒रेण॑ स्थापयति स्थापयति वषट्का॒रेण॒ प्रति॒ प्रति॑ वषट्का॒रेण॑ स्थापयति । \newline
\pagebreak
\markright{ TS 2.6.2.6  \hfill https://www.vedavms.in \hfill}
\addcontentsline{toc}{section}{ TS 2.6.2.6 }
\section*{ TS 2.6.2.6 }

\textbf{TS 2.6.2.6 } \newline
\textbf{Samhita Paata} \newline

वषट्का॒रेण॑ स्थापयति त्रि॒पदा॑ पुरोऽनुवा॒क्या॑ भवति॒ त्रय॑ इ॒मे लो॒का ए॒ष्वे॑व लो॒केषु॒ प्रति॑तिष्ठति॒ चतु॑ष्पदा या॒ज्या॑ चतु॑ष्पद ए॒व प॒शूनव॑ रुन्धे द्व्यक्ष॒रो व॑षट्का॒रो द्वि॒पाद्-यज॑मानः प॒शुष्वे॒वोपरि॑ष्टा॒त् प्रति॑तिष्ठति गाय॒त्री पु॑रोऽनुवा॒क्या॑ भवति त्रि॒ष्टुग् या॒ज्यै॑षा वै स॒प्तप॑दा॒ शक्व॑री॒ यद्वा ए॒तया॑ दे॒वा अशि॑क्ष॒न् तद॑शक्नुव॒न्॒. य ए॒वं ॅवेद॑ श॒क्नोत्ये॒व ( ) यच्छिक्ष॑ति ॥ \newline

\textbf{Pada Paata} \newline

व॒ष॒ट्का॒रेणेति॑ वषट् - का॒रेण॑ । स्था॒प॒य॒ति॒ । त्रि॒पदेति॑ त्रि - पदा᳚ । पु॒रो॒नु॒वा॒क्येति॑ पुरः - अ॒नु॒वा॒क्या᳚ । भ॒व॒ति॒ । त्रयः॑ । इ॒मे । लो॒काः । ए॒षु । ए॒व । लो॒केषु॑ । प्रतीति॑ । ति॒ष्ठ॒ति॒ । चतु॑ष्प॒देति॒ चतुः॑ - प॒दा॒ । या॒ज्या᳚ । चतु॑ष्पद॒ इति॒ चतुः॑-प॒दः॒ । ए॒व । प॒शून् । अवेति॑ । रु॒न्धे॒ । द्व्य॒क्ष॒र इति॑ द्वि - अ॒क्ष॒रः॒ । व॒ष॒ट्का॒र इति॑ वषट् - का॒रः । द्वि॒पादिति॑ द्वि-पात् । यज॑मानः । प॒शुषु॑ । ए॒व । उ॒परि॑ष्टात् । प्रतीति॑ । ति॒ष्ठ॒ति॒ । गा॒य॒त्री । पु॒रो॒नु॒वा॒क्येति॑ पुरः - अ॒नु॒वा॒क्या᳚ । भ॒व॒ति॒ । त्रि॒ष्टुक् । या॒ज्या᳚ । ए॒षा । वै । स॒प्तप॒देति॑ स॒प्त - प॒दा॒ । शक्व॑री । यत् । वै । ए॒तया᳚ । दे॒वाः । अशि॑क्षन्न् । तत् । अ॒श॒क्नु॒व॒न्न् । यः । ए॒वम् । वेद॑ । श॒क्नोति॑ । ए॒व ( ) । यत् । शिक्ष॑ति ॥  \newline


\textbf{Krama Paata} \newline

व॒ष॒ट्का॒रेण॑ स्थापयति । व॒ष॒ट्का॒रेणेति॑ वषट् - का॒रेण॑ । स्था॒प॒य॒ति॒ त्रि॒पदा᳚ । त्रि॒पदा॑ पुरोनुवा॒क्या᳚ । त्रि॒पदेति॑ त्रि - पदा᳚ । पु॒रो॒नु॒वा॒क्या॑ भवति । पु॒रो॒नु॒वा॒क्येति॑ पुरः - अ॒नु॒वा॒क्या᳚ । भ॒व॒ति॒ त्रयः॑ । त्रय॑ इ॒मे । इ॒मे लो॒काः । लो॒का ए॒षु । ए॒ष्वे॑व । ए॒व लो॒केषु॑ । लो॒केषु॒ प्रति॑ । प्रति॑ तिष्ठति । ति॒ष्ठ॒ति॒ चतु॑ष्पदा । चतु॑ष्पदा या॒ज्या᳚ । चतु॑ष्प॒देति॒ चतुः॑ - प॒दा॒ । या॒ज्या॑ चतु॑ष्पदः । चतु॑ष्पद ए॒व । चतु॑ष्पद॒ इति॒ चतुः॑ - प॒दः॒ । ए॒व प॒शून् । प॒शूनव॑ । अव॑ रुन्धे । रु॒न्धे॒ द्व्य॒क्ष॒रः । द्व्य॒क्ष॒रो व॑षट्का॒रः । द्व्य॒क्ष॒र इति॑ द्वि - अ॒क्ष॒रः । व॒ष॒ट्का॒रो द्वि॒पात् । व॒ष॒ट्का॒र इति॑ वषट् - का॒रः । द्वि॒पाद् यज॑मानः । द्वि॒पादिति॑ द्वि - पात् । यज॑मानः प॒शुषु॑ । प॒शुष्वे॒व । ए॒वोपरि॑ष्टात् । उ॒परि॑ष्टा॒त् प्रति॑ । प्रति॑ तिष्ठति । ति॒ष्ठ॒ति॒ गा॒य॒त्री । गा॒य॒त्री पु॑रोनुवा॒क्या᳚ । पु॒रो॒नु॒वा॒क्या॑ भवति । पु॒रो॒नु॒वा॒क्येति॑ पुरः - अ॒नु॒वा॒क्या᳚ । भ॒व॒ति॒ त्रि॒ष्टुक् । त्रि॒ष्टुग् या॒ज्या᳚ । या॒ज्यै॑षा । ए॒षा वै । वै स॒प्तप॑दा । स॒प्तप॑दा॒ शक्व॑री । स॒प्तप॒देति॑ स॒प्त - प॒दा॒ । शक्व॑री॒ यत् । यद् वै । वा ए॒तया᳚ । ए॒तया॑ दे॒वाः । दे॒वा अशि॑क्षन्न् । अशि॑क्ष॒न् तत् । तद॑शक्नुवन्न् । अ॒श॒क्नु॒व॒न्.॒ यः । य ए॒वम् । ए॒वं ॅवेद॑ । वेद॑ श॒क्नोति॑ । श॒क्नोत्ये॒व ( ) । ए॒व यत् । यच्छिक्ष॑ति । शिक्ष॒तीति॒ शिक्ष॑ति । \newline

\textbf{Jatai Paata} \newline

1. व॒ष॒ट्का॒रेण॑ स्थापयति स्थापयति वषट्का॒रेण॑ वषट्का॒रेण॑ स्थापयति । \newline
2. व॒ष॒ट्का॒रेणेति॑ वषट् - का॒रेण॑ । \newline
3. स्था॒प॒य॒ति॒ त्रि॒पदा᳚ त्रि॒पदा᳚ स्थापयति स्थापयति त्रि॒पदा᳚ । \newline
4. त्रि॒पदा॑ पुरोनुवा॒क्या॑ पुरोनुवा॒क्या᳚ त्रि॒पदा᳚ त्रि॒पदा॑ पुरोनुवा॒क्या᳚ । \newline
5. त्रि॒पदेति॑ त्रि - पदा᳚ । \newline
6. पु॒रो॒नु॒वा॒क्या॑ भवति भवति पुरोनुवा॒क्या॑ पुरोनुवा॒क्या॑ भवति । \newline
7. पु॒रो॒नु॒वा॒क्येति॑ पुरः - अ॒नु॒वा॒क्या᳚ । \newline
8. भ॒व॒ति॒ त्रय॒स्त्रयो॑ भवति भवति॒ त्रयः॑ । \newline
9. त्रय॑ इ॒म इ॒मे त्रय॒ स्त्रय॑ इ॒मे । \newline
10. इ॒मे लो॒का लो॒का इ॒म इ॒मे लो॒काः । \newline
11. लो॒का ए॒ष्वे॑षु लो॒का लो॒का ए॒षु । \newline
12. ए॒ष्वे॑ वैवै ष्वे᳚(ए1॒)ष्वे॑व । \newline
13. ए॒व लो॒केषु॑ लो॒के ष्वे॒वैव लो॒केषु॑ । \newline
14. लो॒केषु॒ प्रति॒ प्रति॑ लो॒केषु॑ लो॒केषु॒ प्रति॑ । \newline
15. प्रति॑ तिष्ठति तिष्ठति॒ प्रति॒ प्रति॑ तिष्ठति । \newline
16. ति॒ष्ठ॒ति॒ चतु॑ष्पदा॒ चतु॑ष्पदा तिष्ठति तिष्ठति॒ चतु॑ष्पदा । \newline
17. चतु॑ष्पदा या॒ज्या॑ या॒ज्या॑ चतु॑ष्पदा॒ चतु॑ष्पदा या॒ज्या᳚ । \newline
18. चतु॑ष्प॒देति॒ चतुः॑ - प॒दा॒ । \newline
19. या॒ज्या॑ चतु॑ष्पद॒ श्चतु॑ष्पदो या॒ज्या॑ या॒ज्या॑ चतु॑ष्पदः । \newline
20. चतु॑ष्पद ए॒वैव चतु॑ष्पद॒ श्चतु॑ष्पद ए॒व । \newline
21. चतु॑ष्पद॒ इति॒ चतुः॑ - प॒दः॒ । \newline
22. ए॒व प॒शून् प॒शू ने॒वैव प॒शून् । \newline
23. प॒शू नवाव॑ प॒शून् प॒शू नव॑ । \newline
24. अव॑ रुन्धे रु॒न्धे ऽवाव॑ रुन्धे । \newline
25. रु॒न्धे॒ द्व्य॒क्ष॒रो द्व्य॑क्ष॒रो रु॑न्धे रुन्धे द्व्यक्ष॒रः । \newline
26. द्व्य॒क्ष॒रो व॑षट्का॒रो व॑षट्का॒रो द्व्य॑क्ष॒रो द्व्य॑क्ष॒रो व॑षट्का॒रः । \newline
27. द्व्य॒क्ष॒र इति॑ द्वि - अ॒क्ष॒रः । \newline
28. व॒ष॒ट्का॒रो द्वि॒पाद् द्वि॒पाद् व॑षट्का॒रो व॑षट्का॒रो द्वि॒पात् । \newline
29. व॒ष॒ट्का॒र इति॑ वषट् - का॒रः । \newline
30. द्वि॒पाद् यज॑मानो॒ यज॑मानो द्वि॒पाद् द्वि॒पाद् यज॑मानः । \newline
31. द्वि॒पादिति॑ द्वि - पात् । \newline
32. यज॑मानः प॒शुषु॑ प॒शुषु॒ यज॑मानो॒ यज॑मानः प॒शुषु॑ । \newline
33. प॒शु ष्वे॒वैव प॒शुषु॑ प॒शु ष्वे॒व । \newline
34. ए॒वोपरि॑ष्टा दु॒परि॑ष्टा दे॒वै वोपरि॑ष्टात् । \newline
35. उ॒परि॑ष्टा॒त् प्रति॒ प्रत्यु॒परि॑ष्टा दु॒परि॑ष्टा॒त् प्रति॑ । \newline
36. प्रति॑ तिष्ठति तिष्ठति॒ प्रति॒ प्रति॑ तिष्ठति । \newline
37. ति॒ष्ठ॒ति॒ गा॒य॒त्री गा॑य॒त्री ति॑ष्ठति तिष्ठति गाय॒त्री । \newline
38. गा॒य॒त्री पु॑रोनुवा॒क्या॑ पुरोनुवा॒क्या॑ गाय॒त्री गा॑य॒त्री पु॑रोनुवा॒क्या᳚ । \newline
39. पु॒रो॒नु॒वा॒क्या॑ भवति भवति पुरोनुवा॒क्या॑ पुरोनुवा॒क्या॑ भवति । \newline
40. पु॒रो॒नु॒वा॒क्येति॑ पुरः - अ॒नु॒वा॒क्या᳚ । \newline
41. भ॒व॒ति॒ त्रि॒ष्टुक् त्रि॒ष्टुग् भ॑वति भवति त्रि॒ष्टुक् । \newline
42. त्रि॒ष्टुग् या॒ज्या॑ या॒ज्या᳚ त्रि॒ष्टुक् त्रि॒ष्टुग् या॒ज्या᳚ । \newline
43. या॒ज्यै॑षैषा या॒ज्या॑ या॒ज्यै॑षा । \newline
44. ए॒षा वै वा ए॒षैषा वै । \newline
45. वै स॒प्तप॑दा स॒प्तप॑दा॒ वै वै स॒प्तप॑दा । \newline
46. स॒प्तप॑दा॒ शक्व॑री॒ शक्व॑री स॒प्तप॑दा स॒प्तप॑दा॒ शक्व॑री । \newline
47. स॒प्तप॒देति॑ स॒प्त - प॒दा॒ । \newline
48. शक्व॑री॒ यद् यच्छक्व॑री॒ शक्व॑री॒ यत् । \newline
49. यद् वै वै यद् यद् वै । \newline
50. वा ए॒त यै॒तया॒ वै वा ए॒तया᳚ । \newline
51. ए॒तया॑ दे॒वा दे॒वा ए॒त यै॒तया॑ दे॒वाः । \newline
52. दे॒वा अशि॑क्ष॒न् नशि॑क्षन् दे॒वा दे॒वा अशि॑क्षन्न् । \newline
53. अशि॑क्ष॒न् तत् तदशि॑क्ष॒न् नशि॑क्ष॒न् तत् । \newline
54. तद॑शक्नुवन् नशक्नुव॒न् तत् तद॑शक्नुवन्न् । \newline
55. अ॒श॒क्नु॒व॒न्॒. यो यो॑ ऽशक्नुवन् नशक्नुव॒न्॒. यः । \newline
56. य ए॒व मे॒वं ॅयो य ए॒वम् । \newline
57. ए॒वं ॅवेद॒ वेदै॒व मे॒वं ॅवेद॑ । \newline
58. वेद॑ श॒क्नोति॑ श॒क्नोति॒ वेद॒ वेद॑ श॒क्नोति॑ । \newline
59. श॒क्नो त्ये॒वैव श॒क्नोति॑ श॒क्नो त्ये॒व । \newline
60. ए॒व यद् यदे॒वैव यत् । \newline
61. यच्छिक्ष॑ति॒ शिक्ष॑ति॒ यद् यच्छिक्ष॑ति । \newline
62. शिक्ष॒तीति॒ शिक्ष॑ति । \newline

\textbf{Ghana Paata } \newline

1. व॒ष॒ट्का॒रेण॑ स्थापयति स्थापयति वषट्का॒रेण॑ वषट्का॒रेण॑ स्थापयति त्रि॒पदा᳚ त्रि॒पदा᳚ स्थापयति वषट्का॒रेण॑ वषट्का॒रेण॑ स्थापयति त्रि॒पदा᳚ । \newline
2. व॒ष॒ट्का॒रेणेति॑ वषट् - का॒रेण॑ । \newline
3. स्था॒प॒य॒ति॒ त्रि॒पदा᳚ त्रि॒पदा᳚ स्थापयति स्थापयति त्रि॒पदा॑ पुरोनुवा॒क्या॑ पुरोनुवा॒क्या᳚ त्रि॒पदा᳚ स्थापयति स्थापयति त्रि॒पदा॑ पुरोनुवा॒क्या᳚ । \newline
4. त्रि॒पदा॑ पुरोनुवा॒क्या॑ पुरोनुवा॒क्या᳚ त्रि॒पदा᳚ त्रि॒पदा॑ पुरोनुवा॒क्या॑ भवति भवति पुरोनुवा॒क्या᳚ त्रि॒पदा᳚ त्रि॒पदा॑ पुरोनुवा॒क्या॑ भवति । \newline
5. त्रि॒पदेति॑ त्रि - पदा᳚ । \newline
6. पु॒रो॒नु॒वा॒क्या॑ भवति भवति पुरोनुवा॒क्या॑ पुरोनुवा॒क्या॑ भवति॒ त्रय॒ स्त्रयो॑ भवति पुरोनुवा॒क्या॑ पुरोनुवा॒क्या॑ भवति॒ त्रयः॑ । \newline
7. पु॒रो॒नु॒वा॒क्येति॑ पुरः - अ॒नु॒वा॒क्या᳚ । \newline
8. भ॒व॒ति॒ त्रय॒ स्त्रयो॑ भवति भवति॒ त्रय॑ इ॒म इ॒मे त्रयो॑ भवति भवति॒ त्रय॑ इ॒मे । \newline
9. त्रय॑ इ॒म इ॒मे त्रय॒ स्त्रय॑ इ॒मे लो॒का लो॒का इ॒मे त्रय॒ स्त्रय॑ इ॒मे लो॒काः । \newline
10. इ॒मे लो॒का लो॒का इ॒म इ॒मे लो॒का ए॒ष्वे॑षु लो॒का इ॒म इ॒मे लो॒का ए॒षु । \newline
11. लो॒का ए॒ष्वे॑षु लो॒का लो॒का ए॒ष्वे॑वैवैषु लो॒का लो॒का ए॒ष्वे॑व । \newline
12. ए॒ष्वे॑ वैवैष्वे᳚(ए1॒)ष्वे॑व लो॒केषु॑ लो॒के ष्वे॒वैष्वे᳚(ए1॒)ष्वे॑व लो॒केषु॑ । \newline
13. ए॒व लो॒केषु॑ लो॒के ष्वे॒वैव लो॒केषु॒ प्रति॒ प्रति॑ लो॒के ष्वे॒वैव लो॒केषु॒ प्रति॑ । \newline
14. लो॒केषु॒ प्रति॒ प्रति॑ लो॒केषु॑ लो॒केषु॒ प्रति॑ तिष्ठति तिष्ठति॒ प्रति॑ लो॒केषु॑ लो॒केषु॒ प्रति॑ तिष्ठति । \newline
15. प्रति॑ तिष्ठति तिष्ठति॒ प्रति॒ प्रति॑ तिष्ठति॒ चतु॑ष्पदा॒ चतु॑ष्पदा तिष्ठति॒ प्रति॒ प्रति॑ तिष्ठति॒ चतु॑ष्पदा । \newline
16. ति॒ष्ठ॒ति॒ चतु॑ष्पदा॒ चतु॑ष्पदा तिष्ठति तिष्ठति॒ चतु॑ष्पदा या॒ज्या॑ या॒ज्या॑ चतु॑ष्पदा तिष्ठति तिष्ठति॒ चतु॑ष्पदा या॒ज्या᳚ । \newline
17. चतु॑ष्पदा या॒ज्या॑ या॒ज्या॑ चतु॑ष्पदा॒ चतु॑ष्पदा या॒ज्या॑ चतु॑ष्पद॒ श्चतु॑ष्पदो या॒ज्या॑ चतु॑ष्पदा॒ चतु॑ष्पदा या॒ज्या॑ चतु॑ष्पदः । \newline
18. चतु॑ष्प॒देति॒ चतुः॑ - प॒दा॒ । \newline
19. या॒ज्या॑ चतु॑ष्पद॒ श्चतु॑ष्पदो या॒ज्या॑ या॒ज्या॑ चतु॑ष्पद ए॒वैव चतु॑ष्पदो या॒ज्या॑ या॒ज्या॑ चतु॑ष्पद ए॒व । \newline
20. चतु॑ष्पद ए॒वैव चतु॑ष्पद॒ श्चतु॑ष्पद ए॒व प॒शून् प॒शू ने॒व चतु॑ष्पद॒ श्चतु॑ष्पद ए॒व प॒शून् । \newline
21. चतु॑ष्पद॒ इति॒ चतुः॑ - प॒दः॒ । \newline
22. ए॒व प॒शून् प॒शू ने॒वैव प॒शू नवाव॑ प॒शू ने॒वैव प॒शू नव॑ । \newline
23. प॒शू नवाव॑ प॒शून् प॒शू नव॑ रुन्धे रु॒न्धे ऽव॑ प॒शून् प॒शू नव॑ रुन्धे । \newline
24. अव॑ रुन्धे रु॒न्धे ऽवाव॑ रुन्धे द्व्यक्ष॒रो द्व्य॑क्ष॒रो रु॒न्धे ऽवाव॑ रुन्धे द्व्यक्ष॒रः । \newline
25. रु॒न्धे॒ द्व्य॒क्ष॒रो द्व्य॑क्ष॒रो रु॑न्धे रुन्धे द्व्यक्ष॒रो व॑षट्का॒रो व॑षट्का॒रो द्व्य॑क्ष॒रो रु॑न्धे रुन्धे द्व्यक्ष॒रो व॑षट्का॒रः । \newline
26. द्व्य॒क्ष॒रो व॑षट्का॒रो व॑षट्का॒रो द्व्य॑क्ष॒रो द्व्य॑क्ष॒रो व॑षट्का॒रो द्वि॒पाद् द्वि॒पाद् व॑षट्का॒रो द्व्य॑क्ष॒रो द्व्य॑क्ष॒रो व॑षट्का॒रो द्वि॒पात् । \newline
27. द्व्य॒क्ष॒र इति॑ द्वि - अ॒क्ष॒रः । \newline
28. व॒ष॒ट्का॒रो द्वि॒पाद् द्वि॒पाद् व॑षट्का॒रो व॑षट्का॒रो द्वि॒पाद् यज॑मानो॒ यज॑मानो द्वि॒पाद् व॑षट्का॒रो व॑षट्का॒रो द्वि॒पाद् यज॑मानः । \newline
29. व॒ष॒ट्का॒र इति॑ वषट् - का॒रः । \newline
30. द्वि॒पाद् यज॑मानो॒ यज॑मानो द्वि॒पाद् द्वि॒पाद् यज॑मानः प॒शुषु॑ प॒शुषु॒ यज॑मानो द्वि॒पाद् द्वि॒पाद् यज॑मानः प॒शुषु॑ । \newline
31. द्वि॒पादिति॑ द्वि - पात् । \newline
32. यज॑मानः प॒शुषु॑ प॒शुषु॒ यज॑मानो॒ यज॑मानः प॒शुष्वे॒वैव प॒शुषु॒ यज॑मानो॒ यज॑मानः प॒शुष्वे॒व । \newline
33. प॒शु ष्वे॒वैव प॒शुषु॑ प॒शुष्वे॒ वोपरि॑ष्टा दु॒परि॑ष्टा दे॒व प॒शुषु॑ प॒शुष्वे॒ वोपरि॑ष्टात् । \newline
34. ए॒वोपरि॑ष्टा दु॒परि॑ष्टा दे॒वै वोपरि॑ष्टा॒त् प्रति॒ प्रत्यु॒परि॑ष्टा दे॒वै वोपरि॑ष्टा॒त् प्रति॑ । \newline
35. उ॒परि॑ष्टा॒त् प्रति॒ प्रत्यु॒परि॑ष्टा दु॒परि॑ष्टा॒त् प्रति॑ तिष्ठति तिष्ठति॒ प्रत्यु॒परि॑ष्टा दु॒परि॑ष्टा॒त् प्रति॑ तिष्ठति । \newline
36. प्रति॑ तिष्ठति तिष्ठति॒ प्रति॒ प्रति॑ तिष्ठति गाय॒त्री गा॑य॒त्री ति॑ष्ठति॒ प्रति॒ प्रति॑ तिष्ठति गाय॒त्री । \newline
37. ति॒ष्ठ॒ति॒ गा॒य॒त्री गा॑य॒त्री ति॑ष्ठति तिष्ठति गाय॒त्री पु॑रोनुवा॒क्या॑ पुरोनुवा॒क्या॑ गाय॒त्री ति॑ष्ठति तिष्ठति गाय॒त्री पु॑रोनुवा॒क्या᳚ । \newline
38. गा॒य॒त्री पु॑रोनुवा॒क्या॑ पुरोनुवा॒क्या॑ गाय॒त्री गा॑य॒त्री पु॑रोनुवा॒क्या॑ भवति भवति पुरोनुवा॒क्या॑ गाय॒त्री गा॑य॒त्री पु॑रोनुवा॒क्या॑ भवति । \newline
39. पु॒रो॒नु॒वा॒क्या॑ भवति भवति पुरोनुवा॒क्या॑ पुरोनुवा॒क्या॑ भवति त्रि॒ष्टुक् त्रि॒ष्टुग् भ॑वति पुरोनुवा॒क्या॑ पुरोनुवा॒क्या॑ भवति त्रि॒ष्टुक् । \newline
40. पु॒रो॒नु॒वा॒क्येति॑ पुरः - अ॒नु॒वा॒क्या᳚ । \newline
41. भ॒व॒ति॒ त्रि॒ष्टुक् त्रि॒ष्टुग् भ॑वति भवति त्रि॒ष्टुग् या॒ज्या॑ या॒ज्या᳚ त्रि॒ष्टुग् भ॑वति भवति त्रि॒ष्टुग् या॒ज्या᳚ । \newline
42. त्रि॒ष्टुग् या॒ज्या॑ या॒ज्या᳚ त्रि॒ष्टुक् त्रि॒ष्टुग् या॒ज्यै॑षैषा या॒ज्या᳚ त्रि॒ष्टुक् त्रि॒ष्टुग् या॒ज्यै॑षा । \newline
43. या॒ज्यै॑षैषा या॒ज्या॑ या॒ज्यै॑षा वै वा ए॒षा या॒ज्या॑ या॒ज्यै॑षा वै । \newline
44. ए॒षा वै वा ए॒षैषा वै स॒प्तप॑दा स॒प्तप॑दा॒ वा ए॒षैषा वै स॒प्तप॑दा । \newline
45. वै स॒प्तप॑दा स॒प्तप॑दा॒ वै वै स॒प्तप॑दा॒ शक्व॑री॒ शक्व॑री स॒प्तप॑दा॒ वै वै स॒प्तप॑दा॒ शक्व॑री । \newline
46. स॒प्तप॑दा॒ शक्व॑री॒ शक्व॑री स॒प्तप॑दा स॒प्तप॑दा॒ शक्व॑री॒ यद् यच्छक्व॑री स॒प्तप॑दा स॒प्तप॑दा॒ शक्व॑री॒ यत् । \newline
47. स॒प्तप॒देति॑ स॒प्त - प॒दा॒ । \newline
48. शक्व॑री॒ यद् यच्छक्व॑री॒ शक्व॑री॒ यद् वै वै यच्छक्व॑री॒ शक्व॑री॒ यद् वै । \newline
49. यद् वै वै यद् यद् वा ए॒तयै॒तया॒ वै यद् यद् वा ए॒तया᳚ । \newline
50. वा ए॒तयै॒तया॒ वै वा ए॒तया॑ दे॒वा दे॒वा ए॒तया॒ वै वा ए॒तया॑ दे॒वाः । \newline
51. ए॒तया॑ दे॒वा दे॒वा ए॒तयै॒तया॑ दे॒वा अशि॑क्ष॒न् नशि॑क्षन् दे॒वा ए॒तयै॒तया॑ दे॒वा अशि॑क्षन्न् । \newline
52. दे॒वा अशि॑क्ष॒न् नशि॑क्षन् दे॒वा दे॒वा अशि॑क्ष॒न् तत् तदशि॑क्षन् दे॒वा दे॒वा अशि॑क्ष॒न् तत् । \newline
53. अशि॑क्ष॒न् तत् तदशि॑क्ष॒न् नशि॑क्ष॒न् तद॑शक्नुवन् नशक्नुव॒न् तदशि॑क्ष॒न् नशि॑क्ष॒न् तद॑शक्नुवन्न् । \newline
54. तद॑शक्नुवन् नशक्नुव॒न् तत् तद॑शक्नुव॒न्॒. यो यो॑ ऽशक्नुव॒न् तत् तद॑शक्नुव॒न्॒. यः । \newline
55. अ॒श॒क्नु॒व॒न्॒. यो यो॑ ऽशक्नुवन् नशक्नुव॒न्॒. य ए॒व मे॒वं ॅयो॑ ऽशक्नुवन् नशक्नुव॒न्॒. य ए॒वम् । \newline
56. य ए॒व मे॒वं ॅयो य ए॒वं ॅवेद॒ वेदै॒वं ॅयो य ए॒वं ॅवेद॑ । \newline
57. ए॒वं ॅवेद॒ वेदै॒व मे॒वं ॅवेद॑ श॒क्नोति॑ श॒क्नोति॒ वेदै॒व मे॒वं ॅवेद॑ श॒क्नोति॑ । \newline
58. वेद॑ श॒क्नोति॑ श॒क्नोति॒ वेद॒ वेद॑ श॒क्नो त्ये॒वैव श॒क्नोति॒ वेद॒ वेद॑ श॒क्नो त्ये॒व । \newline
59. श॒क्नो त्ये॒वैव श॒क्नोति॑ श॒क्नो त्ये॒व यद् यदे॒व श॒क्नोति॑ श॒क्नो त्ये॒व यत् । \newline
60. ए॒व यद् यदे॒वैव यच्छिक्ष॑ति॒ शिक्ष॑ति॒ यदे॒वैव यच्छिक्ष॑ति । \newline
61. यच्छिक्ष॑ति॒ शिक्ष॑ति॒ यद् यच्छिक्ष॑ति । \newline
62. शिक्ष॒तीति॒ शिक्ष॑ति । \newline
\pagebreak
\markright{ TS 2.6.3.1  \hfill https://www.vedavms.in \hfill}
\addcontentsline{toc}{section}{ TS 2.6.3.1 }
\section*{ TS 2.6.3.1 }

\textbf{TS 2.6.3.1 } \newline
\textbf{Samhita Paata} \newline

प्र॒जाप॑ति र्दे॒वेभ्यो॑ य॒ज्ञान् व्यादि॑श॒थ् स आ॒त्मन्नाज्य॑मधत्त॒ तं दे॒वा अ॑ब्रुवन्ने॒ष वाव य॒ज्ञो यदाज्य॒मप्ये॒व नोऽत्रा॒स्त्विति॒ सो᳚ऽब्रवी॒द्-यजान्॑ व॒ आज्य॑भागा॒वुप॑ स्तृणान॒भि घा॑रया॒निति॒ तस्मा॒द्-यज॒न्त्या-ज्य॑भागा॒वुप॑ स्तृणन्त्य॒भि घा॑रयन्ति ब्रह्मवा॒दिनो॑ वदन्ति॒ कस्मा᳚थ् स॒त्याद्- या॒तया॑मान्य॒न्यानि॑ ह॒वीꣳ-ष्यया॑तयाम॒-माज्य॒मिति॑ प्राजाप॒त्य - [  ] \newline

\textbf{Pada Paata} \newline

प्र॒जाप॑ति॒रिति॑ प्र॒जा - प॒तिः॒ । दे॒वेभ्यः॑ । य॒ज्ञान् । व्यादि॑श॒दिति॑ वि - आदि॑शत् । सः । आ॒त्मन्न् । आज्य᳚म् । अ॒ध॒त्त॒ । तम् । दे॒वाः । अ॒ब्रु॒व॒न्न् । ए॒षः । वाव । य॒ज्ञ्ः । यत् । आज्य᳚म् । अपीति॑ । ए॒व । नः॒ । अत्र॑ । अ॒स्तु॒ । इति॑ । सः । अ॒ब्र॒वी॒त् । यजान्॑ । वः॒ । आज्य॑भागा॒वित्याज्य॑ - भा॒गौ॒ । उपेति॑ । स्तृ॒णा॒न् । अ॒भीति॑ । घा॒र॒या॒न् । इति॑ । तस्मा᳚त् । यज॑न्ति । आज्य॑भागा॒वित्याज्य॑ - भा॒गौ॒ । उपेति॑ । स्तृ॒ण॒न्ति॒ । अ॒भीति॑ । घा॒र॒य॒न्ति॒ । ब्र॒ह्म॒वा॒दिन॒ इति॑ ब्रह्म - वा॒दिनः॑ । व॒द॒न्ति॒ । कस्मा᳚त् । स॒त्यात् । या॒तया॑मा॒नीति॑ या॒त - या॒मा॒नि॒ । अ॒न्यानि॑ । ह॒वीꣳषि॑ । अया॑तयाम॒मित्यया॑त-या॒म॒म् । आज्य᳚म् । इति॑ । प्रा॒जा॒प॒त्यमिति॑ प्राजा - प॒त्यम् ।  \newline


\textbf{Krama Paata} \newline

प्र॒जाप॑तिर् दे॒वेभ्यः॑ । प्र॒जाप॑ति॒रिति॑ प्र॒जा - प॒तिः॒ । दे॒वेभ्यो॑ य॒ज्ञान् । य॒ज्ञान् व्यादि॑शत् । व्यादि॑श॒थ् सः । व्यादि॑श॒दिति॑ वि - आदि॑शत् । स आ॒त्मन्न् । आ॒त्मन्नाज्य᳚म् । आज्य॑मधत्त । अ॒ध॒त्त॒ तम् । तम् दे॒वाः । दे॒वा अ॑ब्रुवन्न् । अ॒ब्रु॒व॒न्ने॒षः । ए॒ष वाव । वाव य॒ज्ञ्ः । य॒ज्ञो यत् । यदाज्य᳚म् । आज्य॒मपि॑ । अप्ये॒व । ए॒व नः॑ । नोऽत्र॑ । अत्रा᳚स्तु । अ॒स्त्विति॑ । इति॒ सः । सो᳚ऽब्रवीत् । अ॒ब्र॒वी॒द् यजान्॑ । यजान्॑. वः । व॒ आज्य॑भागौ । आज्य॑भागा॒वुप॑ । आज्य॑भागा॒वित्याज्य॑ - भा॒गौ॒ । उप॑ स्तृणान् । स्तृ॒णा॒न॒भि । अ॒भि घा॑रयान् । घा॒र॒या॒निति॑ । इति॒ तस्मा᳚त् । तस्मा॒द् यज॑न्ति । यज॒न्त्याज्य॑भागौ । आज्य॑भागा॒वुप॑ । आज्य॑भागा॒वित्याज्य॑ - भा॒गौ॒ । उप॑ स्तृणन्ति । स्तृ॒ण॒न्त्य॒भि । अ॒भि घा॑रयन्ति । घा॒र॒य॒न्ति॒ ब्र॒ह्म॒वा॒दिनः॑ । ब्र॒ह्म॒वा॒दिनो॑ वदन्ति । ब्र॒ह्म॒वा॒दिन॒ इति॑ ब्रह्म - वा॒दिनः॑ । व॒द॒न्ति॒ कस्मा᳚त् । कस्मा᳚थ् स॒त्यात् । स॒त्याद् या॒तया॑मानि । या॒तया॑मान्य॒न्यानि॑ । या॒तया॑मा॒नीति॑ या॒त - या॒मा॒नि॒ । अ॒न्यानि॑ ह॒वीꣳषि॑ । ह॒वीꣳष्यया॑तयामम् । अया॑तयाम॒माज्य᳚म् । अया॑तयाम॒मित्यया॑त - या॒म॒म् । आज्य॒मिति॑ । इति॑ प्राजाप॒त्यम् । प्रा॒जा॒प॒त्यमिति॑ । प्रा॒जा॒प॒त्यमिति॑ प्राजा - प॒त्यम् \newline

\textbf{Jatai Paata} \newline

1. प्र॒जाप॑तिर् दे॒वेभ्यो॑ दे॒वेभ्यः॑ प्र॒जाप॑तिः प्र॒जाप॑तिर् दे॒वेभ्यः॑ । \newline
2. प्र॒जाप॑ति॒रिति॑ प्र॒जा - प॒तिः॒ । \newline
3. दे॒वेभ्यो॑ य॒ज्ञान्. य॒ज्ञान् दे॒वेभ्यो॑ दे॒वेभ्यो॑ य॒ज्ञान् । \newline
4. य॒ज्ञान् व्यादि॑श॒द् व्यादि॑शद् य॒ज्ञान्. य॒ज्ञान् व्यादि॑शत् । \newline
5. व्यादि॑श॒थ् स स व्यादि॑श॒द् व्यादि॑श॒थ् सः । \newline
6. व्यादि॑श॒दिति॑ वि - आदि॑शत् । \newline
7. स आ॒त्मन् ना॒त्मन् थ्स स आ॒त्मन्न् । \newline
8. आ॒त्मन् नाज्य॒ माज्य॑ मा॒त्मन् ना॒त्मन् नाज्य᳚म् । \newline
9. आज्य॑ मधत्ता ध॒त्ताज्य॒ माज्य॑ मधत्त । \newline
10. अ॒ध॒त्त॒ तम् त म॑धत्ताधत्त॒ तम् । \newline
11. तम् दे॒वा दे॒वा स्तम् तम् दे॒वाः । \newline
12. दे॒वा अ॑ब्रुवन् नब्रुवन् दे॒वा दे॒वा अ॑ब्रुवन्न् । \newline
13. अ॒ब्रु॒व॒न् ने॒ष ए॒षो᳚ ऽब्रुवन् नब्रुवन् ने॒षः । \newline
14. ए॒ष वाव वावैष ए॒ष वाव । \newline
15. वाव य॒ज्ञो य॒ज्ञो वाव वाव य॒ज्ञ्ः । \newline
16. य॒ज्ञो यद् यद् य॒ज्ञो य॒ज्ञो यत् । \newline
17. यदाज्य॒ माज्यं॒ ॅयद् यदाज्य᳚म् । \newline
18. आज्य॒ मप्यप्याज्य॒ माज्य॒ मपि॑ । \newline
19. अप्ये॒वैवा प्यप्ये॒व । \newline
20. ए॒व नो॑ न ए॒वैव नः॑ । \newline
21. नो ऽत्रात्र॑ नो॒ नो ऽत्र॑ । \newline
22. अत्रा᳚ स्त्व॒ स्त्वत्रात्रा᳚स्तु । \newline
23. अ॒स्त्विती त्य॑स्त्व॒ स्त्विति॑ । \newline
24. इति॒ स स इतीति॒ सः । \newline
25. सो᳚ ऽब्रवी दब्रवी॒थ् स सो᳚ ऽब्रवीत् । \newline
26. अ॒ब्र॒वी॒द् यजा॒न्॒. यजा॑ नब्रवी दब्रवी॒द् यजान्॑ । \newline
27. यजान्॑. वो वो॒ यजा॒न्॒. यजान्॑. वः । \newline
28. व॒ आज्य॑भागा॒ वाज्य॑भागौ वो व॒ आज्य॑भागौ । \newline
29. आज्य॑भागा॒ वुपोपाज्य॑भागा॒ वाज्य॑भागा॒ वुप॑ । \newline
30. आज्य॑भागा॒वित्याज्य॑ - भा॒गौ॒ । \newline
31. उप॑ स्तृणान् थ्स्तृणा॒ नुपोप॑ स्तृणान् । \newline
32. स्तृ॒णा॒ न॒भ्य॑भि स्तृ॑णान् थ्स्तृणा न॒भि । \newline
33. अ॒भि घा॑रयान् घारया न॒भ्य॑भि घा॑रयान् । \newline
34. घा॒र॒या॒ नितीति॑ घारयान् घारया॒ निति॑ । \newline
35. इति॒ तस्मा॒त् तस्मा॒ दितीति॒ तस्मा᳚त् । \newline
36. तस्मा॒द् यज॑न्ति॒ यज॑न्ति॒ तस्मा॒त् तस्मा॒द् यज॑न्ति । \newline
37. यज॒ न्त्याज्य॑भागा॒ वाज्य॑भागौ॒ यज॑न्ति॒ यज॒ न्त्याज्य॑भागौ । \newline
38. आज्य॑भागा॒ वुपोपाज्य॑भागा॒ वाज्य॑भागा॒ वुप॑ । \newline
39. आज्य॑भागा॒वित्याज्य॑ - भा॒गौ॒ । \newline
40. उप॑ स्तृणन्ति स्तृण॒ न्त्युपोप॑ स्तृणन्ति । \newline
41. स्तृ॒ण॒ न्त्य॒भ्य॑भि स्तृ॑णन्ति स्तृण न्त्य॒भि । \newline
42. अ॒भि घा॑रयन्ति घारय न्त्य॒भ्य॑भि घा॑रयन्ति । \newline
43. घा॒र॒य॒न्ति॒ ब्र॒ह्म॒वा॒दिनो᳚ ब्रह्मवा॒दिनो॑ घारयन्ति घारयन्ति ब्रह्मवा॒दिनः॑ । \newline
44. ब्र॒ह्म॒वा॒दिनो॑ वदन्ति वदन्ति ब्रह्मवा॒दिनो᳚ ब्रह्मवा॒दिनो॑ वदन्ति । \newline
45. ब्र॒ह्म॒वा॒दिन॒ इति॑ ब्रह्म - वा॒दिनः॑ । \newline
46. व॒द॒न्ति॒ कस्मा॒त् कस्मा᳚द् वदन्ति वदन्ति॒ कस्मा᳚त् । \newline
47. कस्मा᳚थ् स॒त्याथ् स॒त्यात् कस्मा॒त् कस्मा᳚थ् स॒त्यात् । \newline
48. स॒त्याद् या॒तया॑मानि या॒तया॑मानि स॒त्याथ् स॒त्याद् या॒तया॑मानि । \newline
49. या॒तया॑मा न्य॒न्या न्य॒न्यानि॑ या॒तया॑मानि या॒तया॑मा न्य॒न्यानि॑ । \newline
50. या॒तया॑मा॒नीति॑ या॒त - या॒मा॒नि॒ । \newline
51. अ॒न्यानि॑ ह॒वीꣳषि॑ ह॒वीꣳ ष्य॒न्या न्य॒न्यानि॑ ह॒वीꣳषि॑ । \newline
52. ह॒वीꣳ ष्यया॑तयाम॒ मया॑तयामꣳ ह॒वीꣳषि॑ ह॒वीꣳ ष्यया॑तयामम् । \newline
53. अया॑तयाम॒ माज्य॒ माज्य॒ मया॑तयाम॒ मया॑तयाम॒ माज्य᳚म् । \newline
54. अया॑तयाम॒मित्यया॑त - या॒म॒म् । \newline
55. आज्य॒ मिती त्याज्य॒ माज्य॒ मिति॑ । \newline
56. इति॑ प्राजाप॒त्यम् प्रा॑जाप॒त्य मितीति॑ प्राजाप॒त्यम् । \newline
57. प्रा॒जा॒प॒त्य मितीति॑ प्राजाप॒त्यम् प्रा॑जाप॒त्य मिति॑ । \newline
58. प्रा॒जा॒प॒त्यमिति॑ प्राजा - प॒त्यम् । \newline

\textbf{Ghana Paata } \newline

1. प्र॒जाप॑तिर् दे॒वेभ्यो॑ दे॒वेभ्यः॑ प्र॒जाप॑तिः प्र॒जाप॑तिर् दे॒वेभ्यो॑ य॒ज्ञान्. य॒ज्ञान् दे॒वेभ्यः॑ प्र॒जाप॑तिः प्र॒जाप॑तिर् दे॒वेभ्यो॑ य॒ज्ञान् । \newline
2. प्र॒जाप॑ति॒रिति॑ प्र॒जा - प॒तिः॒ । \newline
3. दे॒वेभ्यो॑ य॒ज्ञान्. य॒ज्ञान् दे॒वेभ्यो॑ दे॒वेभ्यो॑ य॒ज्ञान् व्यादि॑श॒द् व्यादि॑शद् य॒ज्ञान् दे॒वेभ्यो॑ दे॒वेभ्यो॑ य॒ज्ञान् व्यादि॑शत् । \newline
4. य॒ज्ञान् व्यादि॑श॒द् व्यादि॑शद् य॒ज्ञान्. य॒ज्ञान् व्यादि॑श॒थ् स स व्यादि॑शद् य॒ज्ञान्. य॒ज्ञान् व्यादि॑श॒थ् सः । \newline
5. व्यादि॑श॒थ् स स व्यादि॑श॒द् व्यादि॑श॒थ् स आ॒त्मन् ना॒त्मन् थ्स व्यादि॑श॒द् व्यादि॑श॒थ् स आ॒त्मन्न् । \newline
6. व्यादि॑श॒दिति॑ वि - आदि॑शत् । \newline
7. स आ॒त्मन् ना॒त्मन् थ्स स आ॒त्मन् नाज्य॒ माज्य॑ मा॒त्मन् थ्स स आ॒त्मन् नाज्य᳚म् । \newline
8. आ॒त्मन् नाज्य॒ माज्य॑ मा॒त्मन् ना॒त्मन् नाज्य॑ मधत्ता ध॒त्ताज्य॑ मा॒त्मन् ना॒त्मन् नाज्य॑ मधत्त । \newline
9. आज्य॑ मधत्ता ध॒त्ताज्य॒ माज्य॑ मधत्त॒ तम् त म॑ध॒त्ताज्य॒ माज्य॑ मधत्त॒ तम् । \newline
10. अ॒ध॒त्त॒ तम् त म॑धत्ताधत्त॒ तम् दे॒वा दे॒वास्त म॑धत्ता धत्त॒ तम् दे॒वाः । \newline
11. तम् दे॒वा दे॒वा स्तम् तम् दे॒वा अ॑ब्रुवन् नब्रुवन् दे॒वा स्तम् तम् दे॒वा अ॑ब्रुवन्न् । \newline
12. दे॒वा अ॑ब्रुवन् नब्रुवन् दे॒वा दे॒वा अ॑ब्रुवन् ने॒ष ए॒षो᳚ ऽब्रुवन् दे॒वा दे॒वा अ॑ब्रुवन् ने॒षः । \newline
13. अ॒ब्रु॒व॒न् ने॒ष ए॒षो᳚ ऽब्रुवन् नब्रुवन् ने॒ष वाव वावैषो᳚ ऽब्रुवन् नब्रुवन् ने॒ष वाव । \newline
14. ए॒ष वाव वावैष ए॒ष वाव य॒ज्ञो य॒ज्ञो वावैष ए॒ष वाव य॒ज्ञ्ः । \newline
15. वाव य॒ज्ञो य॒ज्ञो वाव वाव य॒ज्ञो यद् यद् य॒ज्ञो वाव वाव य॒ज्ञो यत् । \newline
16. य॒ज्ञो यद् यद् य॒ज्ञो य॒ज्ञो यदाज्य॒ माज्यं॒ ॅयद् य॒ज्ञो य॒ज्ञो यदाज्य᳚म् । \newline
17. यदाज्य॒ माज्यं॒ ॅयद् यदाज्य॒ मप्यप्याज्यं॒ ॅयद् यदाज्य॒ मपि॑ । \newline
18. आज्य॒ मप्यप्याज्य॒ माज्य॒ मप्ये॒वैवा प्याज्य॒ माज्य॒ मप्ये॒व । \newline
19. अप्ये॒वैवा प्यप्ये॒व नो॑ न ए॒वा प्यप्ये॒व नः॑ । \newline
20. ए॒व नो॑ न ए॒वैव नो ऽत्रात्र॑ न ए॒वैव नो ऽत्र॑ । \newline
21. नो ऽत्रात्र॑ नो॒ नो ऽत्रा᳚ स्त्व॒ स्त्वत्र॑ नो॒ नो ऽत्रा᳚स्तु । \newline
22. अत्रा᳚स्त्व॒ स्त्वत्रात्रा॒ स्त्विती त्य॒स्त्वत्रात्रा॒ स्त्विति॑ । \newline
23. अ॒स्त्विती त्य॑स्त्व॒ स्त्विति॒ स स इत्य॑ स्त्व॒ स्त्विति॒ सः । \newline
24. इति॒ स स इतीति॒ सो᳚ ऽब्रवी दब्रवी॒थ् स इतीति॒ सो᳚ ऽब्रवीत् । \newline
25. सो᳚ ऽब्रवी दब्रवी॒थ् स सो᳚ ऽब्रवी॒द् यजा॒न्॒. यजा॑ नब्रवी॒थ् स सो᳚ ऽब्रवी॒द् यजान्॑ । \newline
26. अ॒ब्र॒वी॒द् यजा॒न्॒. यजा॑ नब्रवी दब्रवी॒द् यजान्॑. वो वो॒ यजा॑ नब्रवी दब्रवी॒द् यजान्॑. वः । \newline
27. यजान्॑. वो वो॒ यजा॒न्॒. यजान्॑. व॒ आज्य॑भागा॒ वाज्य॑भागौ वो॒ यजा॒न्॒. यजान्॑. व॒ आज्य॑भागौ । \newline
28. व॒ आज्य॑भागा॒ वाज्य॑भागौ वो व॒ आज्य॑भागा॒ वुपोपा ज्य॑भागौ वो व॒ आज्य॑भागा॒ वुप॑ । \newline
29. आज्य॑भागा॒ वुपोपाज्य॑भागा॒ वाज्य॑भागा॒ वुप॑ स्तृणान् थ्स्तृणा॒ नुपाज्य॑भागा॒ वाज्य॑भागा॒ वुप॑ स्तृणान् । \newline
30. आज्य॑भागा॒वित्याज्य॑ - भा॒गौ॒ । \newline
31. उप॑ स्तृणान् थ्स्तृणा॒ नुपोप॑ स्तृणा न॒भ्य॑भि स्तृ॑णा॒ नुपोप॑ स्तृणा न॒भि । \newline
32. स्तृ॒णा॒ न॒भ्य॑भि स्तृ॑णान् थ्स्तृणा न॒भि घा॑रयान् घारया न॒भि स्तृ॑णान् थ्स्तृणा न॒भि घा॑रयान् । \newline
33. अ॒भि घा॑रयान् घारया न॒भ्य॑भि घा॑रया॒ नितीति॑ घारया न॒भ्य॑भि घा॑रया॒ निति॑ । \newline
34. घा॒र॒या॒ नितीति॑ घारयान् घारया॒ निति॒ तस्मा॒त् तस्मा॒दिति॑ घारयान् घारया॒ निति॒ तस्मा᳚त् । \newline
35. इति॒ तस्मा॒त् तस्मा॒ दितीति॒ तस्मा॒द् यज॑न्ति॒ यज॑न्ति॒ तस्मा॒ दितीति॒ तस्मा॒द् यज॑न्ति । \newline
36. तस्मा॒द् यज॑न्ति॒ यज॑न्ति॒ तस्मा॒त् तस्मा॒द् यज॒ न्त्याज्य॑भागा॒ वाज्य॑भागौ॒ यज॑न्ति॒ तस्मा॒त् तस्मा॒द् यज॒ न्त्याज्य॑भागौ । \newline
37. यज॒ न्त्याज्य॑भागा॒ वाज्य॑भागौ॒ यज॑न्ति॒ यज॒ न्त्याज्य॑भागा॒ वुपोपा ज्य॑भागौ॒ यज॑न्ति॒ यज॒ न्त्याज्य॑भागा॒ वुप॑ । \newline
38. आज्य॑भागा॒ वुपोपा ज्य॑भागा॒ वाज्य॑भागा॒ वुप॑ स्तृणन्ति स्तृण॒ न्त्युपाज्य॑भागा॒ वाज्य॑भागा॒ वुप॑ स्तृणन्ति । \newline
39. आज्य॑भागा॒वित्याज्य॑ - भा॒गौ॒ । \newline
40. उप॑ स्तृणन्ति स्तृण॒ न्त्युपोप॑ स्तृण न्त्य॒भ्य॑भि स्तृ॑ण॒ न्त्युपोप॑ स्तृण न्त्य॒भि । \newline
41. स्तृ॒ण॒ न्त्य॒भ्य॑भि स्तृ॑णन्ति स्तृण न्त्य॒भि घा॑रयन्ति घारय न्त्य॒भि स्तृ॑णन्ति स्तृण न्त्य॒भि घा॑रयन्ति । \newline
42. अ॒भि घा॑रयन्ति घारय न्त्य॒भ्य॑भि घा॑रयन्ति ब्रह्मवा॒दिनो᳚ ब्रह्मवा॒दिनो॑ घारय न्त्य॒भ्य॑भि घा॑रयन्ति ब्रह्मवा॒दिनः॑ । \newline
43. घा॒र॒य॒न्ति॒ ब्र॒ह्म॒वा॒दिनो᳚ ब्रह्मवा॒दिनो॑ घारयन्ति घारयन्ति ब्रह्मवा॒दिनो॑ वदन्ति वदन्ति ब्रह्मवा॒दिनो॑ घारयन्ति घारयन्ति ब्रह्मवा॒दिनो॑ वदन्ति । \newline
44. ब्र॒ह्म॒वा॒दिनो॑ वदन्ति वदन्ति ब्रह्मवा॒दिनो᳚ ब्रह्मवा॒दिनो॑ वदन्ति॒ कस्मा॒त् कस्मा᳚द् वदन्ति ब्रह्मवा॒दिनो᳚ ब्रह्मवा॒दिनो॑ वदन्ति॒ कस्मा᳚त् । \newline
45. ब्र॒ह्म॒वा॒दिन॒ इति॑ ब्रह्म - वा॒दिनः॑ । \newline
46. व॒द॒न्ति॒ कस्मा॒त् कस्मा᳚द् वदन्ति वदन्ति॒ कस्मा᳚थ् स॒त्याथ् स॒त्यात् कस्मा᳚द् वदन्ति वदन्ति॒ कस्मा᳚थ् स॒त्यात् । \newline
47. कस्मा᳚थ् स॒त्याथ् स॒त्यात् कस्मा॒त् कस्मा᳚थ् स॒त्याद् या॒तया॑मानि या॒तया॑मानि स॒त्यात् कस्मा॒त् कस्मा᳚थ् स॒त्याद् या॒तया॑मानि । \newline
48. स॒त्याद् या॒तया॑मानि या॒तया॑मानि स॒त्याथ् स॒त्याद् या॒तया॑मा न्य॒न्या न्य॒न्यानि॑ या॒तया॑मानि स॒त्याथ् स॒त्याद् या॒तया॑मा न्य॒न्यानि॑ । \newline
49. या॒तया॑मा न्य॒न्या न्य॒न्यानि॑ या॒तया॑मानि या॒तया॑मा न्य॒न्यानि॑ ह॒वीꣳषि॑ ह॒वीꣳ ष्य॒न्यानि॑ या॒तया॑मानि या॒तया॑मा न्य॒न्यानि॑ ह॒वीꣳषि॑ । \newline
50. या॒तया॑मा॒नीति॑ या॒त - या॒मा॒नि॒ । \newline
51. अ॒न्यानि॑ ह॒वीꣳषि॑ ह॒वीꣳ ष्य॒न्या न्य॒न्यानि॑ ह॒वीꣳ ष्यया॑तयाम॒ मया॑तयामꣳ ह॒वीꣳ ष्य॒न्या न्य॒न्यानि॑ ह॒वीꣳ ष्यया॑तयामम् । \newline
52. ह॒वीꣳ ष्यया॑तयाम॒ मया॑तयामꣳ ह॒वीꣳषि॑ ह॒वीꣳ ष्यया॑तयाम॒ माज्य॒ माज्य॒ मया॑तयामꣳ ह॒वीꣳषि॑ ह॒वीꣳ ष्यया॑तयाम॒ माज्य᳚म् । \newline
53. अया॑तयाम॒ माज्य॒ माज्य॒ मया॑तयाम॒ मया॑तयाम॒ माज्य॒ मितीत्याज्य॒ मया॑तयाम॒ मया॑तयाम॒ माज्य॒ मिति॑ । \newline
54. अया॑तयाम॒मित्यया॑त - या॒म॒म् । \newline
55. आज्य॒ मितीत्याज्य॒ माज्य॒ मिति॑ प्राजाप॒त्यम् प्रा॑जाप॒त्य मित्याज्य॒ माज्य॒ मिति॑ प्राजाप॒त्यम् । \newline
56. इति॑ प्राजाप॒त्यम् प्रा॑जाप॒त्य मितीति॑ प्राजाप॒त्य मितीति॑ प्राजाप॒त्य मितीति॑ प्राजाप॒त्य मिति॑ । \newline
57. प्रा॒जा॒प॒त्य मितीति॑ प्राजाप॒त्यम् प्रा॑जाप॒त्य मिति॑ ब्रूयाद् ब्रूया॒दिति॑ प्राजाप॒त्यम् प्रा॑जाप॒त्य मिति॑ ब्रूयात् । \newline
58. प्रा॒जा॒प॒त्यमिति॑ प्राजा - प॒त्यम् । \newline
\pagebreak
\markright{ TS 2.6.3.2  \hfill https://www.vedavms.in \hfill}
\addcontentsline{toc}{section}{ TS 2.6.3.2 }
\section*{ TS 2.6.3.2 }

\textbf{TS 2.6.3.2 } \newline
\textbf{Samhita Paata} \newline

-मिति॑ ब्रूया॒दया॑तयामा॒ हि दे॒वानां᳚ प्र॒जाप॑ति॒रिति॒ छन्दाꣳ॑सि दे॒वेभ्योऽपा᳚क्राम॒न् न वो॑ऽभा॒गानि॑ ह॒व्यं ॅव॑क्ष्याम॒ इति॒ तेभ्य॑ ए॒त-च्च॑तुरव॒त्त-म॑धारयन् पुरोऽनुवा॒क्या॑यै या॒ज्या॑यै दे॒वता॑यै वषट्का॒राय॒ यच्च॑तुरव॒त्तं जु॒होति॒ छन्दाꣳ॑स्ये॒व तत् प्री॑णाति॒ तान्य॑स्य प्री॒तानि॑ दे॒वेभ्यो॑ ह॒व्यं ॅव॑ह॒न्त्यङ्गि॑रसो॒ वा इ॒त उ॑त्त॒माः सु॑व॒र्गं ॅलो॒कमा॑य॒न् तदृष॑यो यज्ञ्वा॒स्त्व॑भ्य॒वाय॒न् ते॑ - [  ] \newline

\textbf{Pada Paata} \newline

इति॑ । ब्रू॒या॒त् । अया॑तया॒मेत्यया॑त - या॒मा॒ । हि । दे॒वाना᳚म् । प्र॒जाप॑ति॒रिति॑ प्र॒जा - प॒तिः॒ । इति॑ । छन्दाꣳ॑सि । दे॒वेभ्यः॑ । अपेति॑ । अ॒क्रा॒म॒न्न् । न । वः॒ । अ॒भा॒गानि॑ । ह॒व्यम् । व॒क्ष्या॒मः॒ । इति॑ । तेभ्यः॑ । ए॒तत् । च॒तु॒र॒व॒त्तमिति॑ चतुः - अ॒व॒त्तम् । अ॒धा॒र॒य॒न्न् । पु॒रो॒नु॒वा॒क्या॑या॒ इति॑ पुरः - अ॒नु॒वा॒क्या॑यै । या॒ज्या॑यै । दे॒वता॑यै । व॒ष॒ट्का॒रायेति॑ वषट् - का॒राय॑ । यत् । च॒तु॒र॒व॒त्तमिति॑ चतुः - अ॒व॒त्तम् । जु॒होति॑ । छन्दाꣳ॑सि । ए॒व । तत् । प्री॒णा॒ति॒ । तानि॑ । अ॒स्य॒ । प्री॒तानि॑ । दे॒वेभ्यः॑ । ह॒व्यम् । व॒ह॒न्ति॒ । अङ्गि॑रसः । वै । इ॒तः । उ॒त्त॒मा इत्यु॑त् - त॒माः । सु॒व॒र्गमिति॑ सुवः - गम् । लो॒कम् । आ॒य॒न्न् । तत् । ऋष॑यः । य॒ज्ञ्॒वा॒स्त्विति॑ यज्ञ् - वा॒स्तु । अ॒भ्य॒वाय॒न्नित्य॑भि - अ॒वायन्न्॑ । ते ।  \newline


\textbf{Krama Paata} \newline

इति॑ ब्रूयात् । ब्रू॒या॒दया॑तयामा । अया॑तयामा॒ हि । अया॑तया॒मेत्यया॑त - या॒मा॒ । हि दे॒वाना᳚म् । दे॒वाना᳚म् प्र॒जाप॑तिः । प्र॒जाप॑ति॒रिति॑ । प्र॒जाप॑ति॒रिति॑ प्र॒जा - प॒तिः॒ । इति॒ छन्दाꣳ॑सि । छन्दाꣳ॑सि दे॒वेभ्यः॑ । दे॒वेभ्योऽप॑ । अपा᳚क्रामन्न् । अ॒क्रा॒म॒न् न । न वः॑ । वो॒ऽभा॒गानि॑ । अ॒भा॒गानि॑ ह॒व्यम् । ह॒व्यं ॅव॑क्ष्यामः । व॒क्ष्या॒म॒ इति॑ । इति॒ तेभ्यः॑ । तेभ्य॑ ए॒तत् । ए॒तच्च॑तुरव॒त्तम् । च॒तु॒र॒व॒त्तम॑धारयन्न् । च॒तु॒र॒व॒त्तमिति॑ चतुः - अ॒व॒त्तम् । अ॒धा॒र॒य॒न् पु॒रो॒नु॒वा॒क्या॑यै । पु॒रो॒नु॒वा॒क्या॑यै या॒ज्या॑यै । पु॒रो॒नु॒वा॒क्या॑या॒ इति॑ पुरः - अ॒नु॒वा॒क्या॑यै । या॒ज्या॑यै दे॒वता॑यै । दे॒वता॑यै वषट्का॒राय॑ । व॒ष॒ट्का॒राय॒ यत् । व॒ष॒ट्का॒रायेति॑ वषट् - का॒राय॑ । यच्च॑तुरव॒त्तम् । च॒तु॒र॒व॒त्तम् जु॒होति॑ । च॒तु॒र॒व॒त्तमिति॑ चतुः - अ॒व॒त्तम् । जु॒होति॒ छन्दाꣳ॑सि । छन्दाꣳ॑स्ये॒व । ए॒व तत् । तत् प्री॑णाति । प्री॒णा॒ति॒ तानि॑ । तान्य॑स्य । अ॒स्य॒ प्री॒तानि॑ । प्री॒तानि॑ दे॒वेभ्यः॑ । दे॒वेभ्यो॑ ह॒व्यम् । ह॒व्यं ॅव॑हन्ति । व॒ह॒न्त्यङ्गि॑रसः । अङ्गि॑रसो॒ वै । वा इ॒तः । इ॒त उ॑त्त॒माः । उ॒त्त॒माः सु॑व॒र्गम् । उ॒त्त॒मा इत्यु॑त् - त॒माः । सु॒व॒र्गं ॅलो॒कम् । सु॒व॒र्गमिति॑ सुवः - गम् । लो॒कमा॑यन्न् । आ॒य॒न् तत् । तदृष॑यः । ऋष॑यो यज्ञ्वा॒स्तु । य॒ज्ञ्॒वा॒स्त्व॑भ्य॒वायन्न्॑ । य॒ज्ञ्॒वा॒स्त्विति॑ यज्ञ् - वा॒स्तु । अ॒भ्य॒वाय॒न् ते । अ॒भ्य॒वाय॒न्नित्य॑भि - अ॒वायन्न्॑ । ते॑ऽपश्यन्न् \newline

\textbf{Jatai Paata} \newline

1. इति॑ ब्रूयाद् ब्रूया॒ दितीति॑ ब्रूयात् । \newline
2. ब्रू॒या॒ दया॑तया॒मा ऽया॑तयामा ब्रूयाद् ब्रूया॒ दया॑तयामा । \newline
3. अया॑तयामा॒ हि ह्यया॑तया॒मा ऽया॑तयामा॒ हि । \newline
4. अया॑तया॒मेत्यया॑त - या॒मा॒ । \newline
5. हि दे॒वाना᳚म् दे॒वानाꣳ॒॒ हि हि दे॒वाना᳚म् । \newline
6. दे॒वाना᳚म् प्र॒जाप॑तिः प्र॒जाप॑तिर् दे॒वाना᳚म् दे॒वाना᳚म् प्र॒जाप॑तिः । \newline
7. प्र॒जाप॑ति॒ रितीति॑ प्र॒जाप॑तिः प्र॒जाप॑ति॒ रिति॑ । \newline
8. प्र॒जाप॑ति॒रिति॑ प्र॒जा - प॒तिः॒ । \newline
9. इति॒ छन्दाꣳ॑सि॒ छन्दाꣳ॒॒सीतीति॒ छन्दाꣳ॑सि । \newline
10. छन्दाꣳ॑सि दे॒वेभ्यो॑ दे॒वेभ्य॒ श्छन्दाꣳ॑सि॒ छन्दाꣳ॑सि दे॒वेभ्यः॑ । \newline
11. दे॒वेभ्यो ऽपाप॑ दे॒वेभ्यो॑ दे॒वेभ्यो ऽप॑ । \newline
12. अपा᳚क्रामन् नक्राम॒न् नपापा᳚ क्रामन्न् । \newline
13. अ॒क्रा॒म॒न् न नाक्रा॑मन् नक्राम॒न् न । \newline
14. न वो॑ वो॒ न न वः॑ । \newline
15. वो॒ ऽभा॒गा न्य॑भा॒गानि॑ वो वो ऽभा॒गानि॑ । \newline
16. अ॒भा॒गानि॑ ह॒व्यꣳ ह॒व्य म॑भा॒गा न्य॑भा॒गानि॑ ह॒व्यम् । \newline
17. ह॒व्यं ॅव॑क्ष्यामो वक्ष्यामो ह॒व्यꣳ ह॒व्यं ॅव॑क्ष्यामः । \newline
18. व॒क्ष्या॒म॒ इतीति॑ वक्ष्यामो वक्ष्याम॒ इति॑ । \newline
19. इति॒ तेभ्य॒ स्तेभ्य॒ इतीति॒ तेभ्यः॑ । \newline
20. तेभ्य॑ ए॒त दे॒तत् तेभ्य॒ स्तेभ्य॑ ए॒तत् । \newline
21. ए॒तच् च॑तुरव॒त्तम् च॑तुरव॒त्त मे॒तदे॒तच् च॑तुरव॒त्तम् । \newline
22. च॒तु॒र॒व॒त्त म॑धारयन् नधारयꣳ श्चतुरव॒त्तम् च॑तुरव॒त्त म॑धारयन्न् । \newline
23. च॒तु॒र॒व॒त्तमिति॑ चतुः - अ॒व॒त्तम् । \newline
24. अ॒धा॒र॒य॒न् पु॒रो॒नु॒वा॒क्या॑यै पुरोनुवा॒क्या॑या अधारयन् नधारयन् पुरोनुवा॒क्या॑यै । \newline
25. पु॒रो॒नु॒वा॒क्या॑यै या॒ज्या॑यै या॒ज्या॑यै पुरोनुवा॒क्या॑यै पुरोनुवा॒क्या॑यै या॒ज्या॑यै । \newline
26. पु॒रो॒नु॒वा॒क्या॑या॒ इति॑ पुरः - अ॒नु॒वा॒क्या॑यै । \newline
27. या॒ज्या॑यै दे॒वता॑यै दे॒वता॑यै या॒ज्या॑यै या॒ज्या॑यै दे॒वता॑यै । \newline
28. दे॒वता॑यै वषट्का॒राय॑ वषट्का॒राय॑ दे॒वता॑यै दे॒वता॑यै वषट्का॒राय॑ । \newline
29. व॒ष॒ट्का॒राय॒ यद् यद् व॑षट्का॒राय॑ वषट्का॒राय॒ यत् । \newline
30. व॒ष॒ट्का॒रायेति॑ वषट् - का॒राय॑ । \newline
31. यच् च॑तुरव॒त्तम् च॑तुरव॒त्तं ॅयद् यच् च॑तुरव॒त्तम् । \newline
32. च॒तु॒र॒व॒त्तम् जु॒होति॑ जु॒होति॑ चतुरव॒त्तम् च॑तुरव॒त्तम् जु॒होति॑ । \newline
33. च॒तु॒र॒व॒त्तमिति॑ चतुः - अ॒व॒त्तम् । \newline
34. जु॒होति॒ छन्दाꣳ॑सि॒ छन्दाꣳ॑सि जु॒होति॑ जु॒होति॒ छन्दाꣳ॑सि । \newline
35. छन्दाꣳ॑ स्ये॒वैव छन्दाꣳ॑सि॒ छन्दाꣳ॑ स्ये॒व । \newline
36. ए॒व तत् तदे॒वैव तत् । \newline
37. तत् प्री॑णाति प्रीणाति॒ तत् तत् प्री॑णाति । \newline
38. प्री॒णा॒ति॒ तानि॒ तानि॑ प्रीणाति प्रीणाति॒ तानि॑ । \newline
39. तान्य॑स्यास्य॒ तानि॒ तान्य॑स्य । \newline
40. अ॒स्य॒ प्री॒तानि॑ प्री॒ता न्य॑स्यास्य प्री॒तानि॑ । \newline
41. प्री॒तानि॑ दे॒वेभ्यो॑ दे॒वेभ्यः॑ प्री॒तानि॑ प्री॒तानि॑ दे॒वेभ्यः॑ । \newline
42. दे॒वेभ्यो॑ ह॒व्यꣳ ह॒व्यम् दे॒वेभ्यो॑ दे॒वेभ्यो॑ ह॒व्यम् । \newline
43. ह॒व्यं ॅव॑हन्ति वहन्ति ह॒व्यꣳ ह॒व्यं ॅव॑हन्ति । \newline
44. व॒ह॒ न्त्यङ्गि॑र॒सो ऽङ्गि॑रसो वहन्ति वह॒ न्त्यङ्गि॑रसः । \newline
45. अङ्गि॑रसो॒ वै वा अङ्गि॑र॒सो ऽङ्गि॑रसो॒ वै । \newline
46. वा इ॒त इ॒तो वै वा इ॒तः । \newline
47. इ॒त उ॑त्त॒मा उ॑त्त॒मा इ॒त इ॒त उ॑त्त॒माः । \newline
48. उ॒त्त॒माः सु॑व॒र्गꣳ सु॑व॒र्ग मु॑त्त॒मा उ॑त्त॒माः सु॑व॒र्गम् । \newline
49. उ॒त्त॒मा इत्यु॑त् - त॒माः । \newline
50. सु॒व॒र्गम् ॅलो॒कम् ॅलो॒कꣳ सु॑व॒र्गꣳ सु॑व॒र्गम् ॅलो॒कम् । \newline
51. सु॒व॒र्गमिति॑ सुवः - गम् । \newline
52. लो॒क मा॑यन् नायन् ॅलो॒कम् ॅलो॒क मा॑यन्न् । \newline
53. आ॒य॒न् तत् तदा॑यन् नाय॒न् तत् । \newline
54. तदृष॑य॒ ऋष॑य॒ स्तत् तदृष॑यः । \newline
55. ऋष॑यो यज्ञ्वा॒स्तु य॑ज्ञ्वा॒ स्त्वृष॑य॒ ऋष॑यो यज्ञ्वा॒स्तु । \newline
56. य॒ज्ञ्॒वा॒ स्त्व॑भ्य॒वाय॑न् नभ्य॒वाय॑न्. यज्ञ्वा॒स्तु य॑ज्ञ्वा॒ स्त्व॑भ्य॒वायन्न्॑ । \newline
57. य॒ज्ञ्॒वा॒स्त्विति॑ यज्ञ् - वा॒स्तु । \newline
58. अ॒भ्य॒वाय॒न् ते ते᳚ ऽभ्य॒वाय॑न् नभ्य॒वाय॒न् ते । \newline
59. अ॒भ्य॒वाय॒न्नित्य॑भि - अ॒वायन्न्॑ । \newline
60. ते॑ ऽपश्यन् नपश्य॒न् ते ते॑ ऽपश्यन्न् । \newline

\textbf{Ghana Paata } \newline

1. इति॑ ब्रूयाद् ब्रूया॒दितीति॑ ब्रूया॒ दया॑तया॒मा ऽया॑तयामा ब्रूया॒दितीति॑ ब्रूया॒ दया॑तयामा । \newline
2. ब्रू॒या॒ दया॑तया॒मा ऽया॑तयामा ब्रूयाद् ब्रूया॒ दया॑तयामा॒ हि ह्यया॑तयामा ब्रूयाद् ब्रूया॒ दया॑तयामा॒ हि । \newline
3. अया॑तयामा॒ हि ह्यया॑तया॒मा ऽया॑तयामा॒ हि दे॒वाना᳚म् दे॒वानाꣳ॒॒ ह्यया॑तया॒मा ऽया॑तयामा॒ हि दे॒वाना᳚म् । \newline
4. अया॑तया॒मेत्यया॑त - या॒मा॒ । \newline
5. हि दे॒वाना᳚म् दे॒वानाꣳ॒॒ हि हि दे॒वाना᳚म् प्र॒जाप॑तिः प्र॒जाप॑तिर् दे॒वानाꣳ॒॒ हि हि दे॒वाना᳚म् प्र॒जाप॑तिः । \newline
6. दे॒वाना᳚म् प्र॒जाप॑तिः प्र॒जाप॑तिर् दे॒वाना᳚म् दे॒वाना᳚म् प्र॒जाप॑ति॒ रितीति॑ प्र॒जाप॑तिर् दे॒वाना᳚म् दे॒वाना᳚म् प्र॒जाप॑ति॒रिति॑ । \newline
7. प्र॒जाप॑ति॒ रितीति॑ प्र॒जाप॑तिः प्र॒जाप॑ति॒रिति॒ छन्दाꣳ॑सि॒ छन्दाꣳ॒॒सीति॑ प्र॒जाप॑तिः प्र॒जाप॑ति॒रिति॒ छन्दाꣳ॑सि । \newline
8. प्र॒जाप॑ति॒रिति॑ प्र॒जा - प॒तिः॒ । \newline
9. इति॒ छन्दाꣳ॑सि॒ छन्दाꣳ॒॒सीतीति॒ छन्दाꣳ॑सि दे॒वेभ्यो॑ दे॒वेभ्य॒ श्छन्दाꣳ॒॒सीतीति॒ छन्दाꣳ॑सि दे॒वेभ्यः॑ । \newline
10. छन्दाꣳ॑सि दे॒वेभ्यो॑ दे॒वेभ्य॒ श्छन्दाꣳ॑सि॒ छन्दाꣳ॑सि दे॒वेभ्यो ऽपाप॑ दे॒वेभ्य॒ श्छन्दाꣳ॑सि॒ छन्दाꣳ॑सि दे॒वेभ्यो ऽप॑ । \newline
11. दे॒वेभ्यो ऽपाप॑ दे॒वेभ्यो॑ दे॒वेभ्यो ऽपा᳚क्रामन् नक्राम॒न् नप॑ दे॒वेभ्यो॑ दे॒वेभ्यो ऽपा᳚क्रामन्न् । \newline
12. अपा᳚क्रामन् नक्राम॒न् नपापा᳚क्राम॒न् न नाक्रा॑म॒न् नपापा᳚क्राम॒न् न । \newline
13. अ॒क्रा॒म॒न् न नाक्रा॑मन् नक्राम॒न् न वो॑ वो॒ नाक्रा॑मन् नक्राम॒न् न वः॑ । \newline
14. न वो॑ वो॒ न न वो॑ ऽभा॒गा न्य॑भा॒गानि॑ वो॒ न न वो॑ ऽभा॒गानि॑ । \newline
15. वो॒ ऽभा॒गा न्य॑भा॒गानि॑ वो वो ऽभा॒गानि॑ ह॒व्यꣳ ह॒व्य म॑भा॒गानि॑ वो वो ऽभा॒गानि॑ ह॒व्यम् । \newline
16. अ॒भा॒गानि॑ ह॒व्यꣳ ह॒व्य म॑भा॒गा न्य॑भा॒गानि॑ ह॒व्यं ॅव॑क्ष्यामो वक्ष्यामो ह॒व्य म॑भा॒गा न्य॑भा॒गानि॑ ह॒व्यं ॅव॑क्ष्यामः । \newline
17. ह॒व्यं ॅव॑क्ष्यामो वक्ष्यामो ह॒व्यꣳ ह॒व्यं ॅव॑क्ष्याम॒ इतीति॑ वक्ष्यामो ह॒व्यꣳ ह॒व्यं ॅव॑क्ष्याम॒ इति॑ । \newline
18. व॒क्ष्या॒म॒ इतीति॑ वक्ष्यामो वक्ष्याम॒ इति॒ तेभ्य॒ स्तेभ्य॒ इति॑ वक्ष्यामो वक्ष्याम॒ इति॒ तेभ्यः॑ । \newline
19. इति॒ तेभ्य॒ स्तेभ्य॒ इतीति॒ तेभ्य॑ ए॒त दे॒तत् तेभ्य॒ इतीति॒ तेभ्य॑ ए॒तत् । \newline
20. तेभ्य॑ ए॒त दे॒तत् तेभ्य॒ स्तेभ्य॑ ए॒तच् च॑तुरव॒त्तम् च॑तुरव॒त्त मे॒तत् तेभ्य॒ स्तेभ्य॑ ए॒तच् च॑तुरव॒त्तम् । \newline
21. ए॒तच् च॑तुरव॒त्तम् च॑तुरव॒त्त मे॒तदे॒तच् च॑तुरव॒त्त म॑धारयन् नधारयꣳ श्चतुरव॒त्त मे॒तदे॒तच् च॑तुरव॒त्त म॑धारयन्न् । \newline
22. च॒तु॒र॒व॒त्त म॑धारयन् नधारयꣳ श्चतुरव॒त्तम् च॑तुरव॒त्त म॑धारयन् पुरोनुवा॒क्या॑यै पुरोनुवा॒क्या॑या अधारयꣳ श्चतुरव॒त्तम् च॑तुरव॒त्त म॑धारयन् पुरोनुवा॒क्या॑यै । \newline
23. च॒तु॒र॒व॒त्तमिति॑ चतुः - अ॒व॒त्तम् । \newline
24. अ॒धा॒र॒य॒न् पु॒रो॒नु॒वा॒क्या॑यै पुरोनुवा॒क्या॑या अधारयन् नधारयन् पुरोनुवा॒क्या॑यै या॒ज्या॑यै या॒ज्या॑यै पुरोनुवा॒क्या॑या अधारयन् नधारयन् पुरोनुवा॒क्या॑यै या॒ज्या॑यै । \newline
25. पु॒रो॒नु॒वा॒क्या॑यै या॒ज्या॑यै या॒ज्या॑यै पुरोनुवा॒क्या॑यै पुरोनुवा॒क्या॑यै या॒ज्या॑यै दे॒वता॑यै दे॒वता॑यै या॒ज्या॑यै पुरोनुवा॒क्या॑यै पुरोनुवा॒क्या॑यै या॒ज्या॑यै दे॒वता॑यै । \newline
26. पु॒रो॒नु॒वा॒क्या॑या॒ इति॑ पुरः - अ॒नु॒वा॒क्या॑यै । \newline
27. या॒ज्या॑यै दे॒वता॑यै दे॒वता॑यै या॒ज्या॑यै या॒ज्या॑यै दे॒वता॑यै वषट्का॒राय॑ वषट्का॒राय॑ दे॒वता॑यै या॒ज्या॑यै या॒ज्या॑यै दे॒वता॑यै वषट्का॒राय॑ । \newline
28. दे॒वता॑यै वषट्का॒राय॑ वषट्का॒राय॑ दे॒वता॑यै दे॒वता॑यै वषट्का॒राय॒ यद् यद् व॑षट्का॒राय॑ दे॒वता॑यै दे॒वता॑यै वषट्का॒राय॒ यत् । \newline
29. व॒ष॒ट्का॒राय॒ यद् यद् व॑षट्का॒राय॑ वषट्का॒राय॒ यच् च॑तुरव॒त्तम् च॑तुरव॒त्तं ॅयद् व॑षट्का॒राय॑ वषट्का॒राय॒ यच् च॑तुरव॒त्तम् । \newline
30. व॒ष॒ट्का॒रायेति॑ वषट् - का॒राय॑ । \newline
31. यच् च॑तुरव॒त्तम् च॑तुरव॒त्तं ॅयद् यच् च॑तुरव॒त्तम् जु॒होति॑ जु॒होति॑ चतुरव॒त्तं ॅयद् यच् च॑तुरव॒त्तम् जु॒होति॑ । \newline
32. च॒तु॒र॒व॒त्तम् जु॒होति॑ जु॒होति॑ चतुरव॒त्तम् च॑तुरव॒त्तम् जु॒होति॒ छन्दाꣳ॑सि॒ छन्दाꣳ॑सि जु॒होति॑ चतुरव॒त्तम् च॑तुरव॒त्तम् जु॒होति॒ छन्दाꣳ॑सि । \newline
33. च॒तु॒र॒व॒त्तमिति॑ चतुः - अ॒व॒त्तम् । \newline
34. जु॒होति॒ छन्दाꣳ॑सि॒ छन्दाꣳ॑सि जु॒होति॑ जु॒होति॒ छन्दाꣳ॑ स्ये॒वैव छन्दाꣳ॑सि जु॒होति॑ जु॒होति॒ छन्दाꣳ॑स्ये॒व । \newline
35. छन्दाꣳ॑ स्ये॒वैव छन्दाꣳ॑सि॒ छन्दाꣳ॑ स्ये॒व तत् तदे॒व छन्दाꣳ॑सि॒ छन्दाꣳ॑ स्ये॒व तत् । \newline
36. ए॒व तत् तदे॒वैव तत् प्री॑णाति प्रीणाति॒ तदे॒वैव तत् प्री॑णाति । \newline
37. तत् प्री॑णाति प्रीणाति॒ तत् तत् प्री॑णाति॒ तानि॒ तानि॑ प्रीणाति॒ तत् तत् प्री॑णाति॒ तानि॑ । \newline
38. प्री॒णा॒ति॒ तानि॒ तानि॑ प्रीणाति प्रीणाति॒ तान्य॑स्यास्य॒ तानि॑ प्रीणाति प्रीणाति॒ तान्य॑स्य । \newline
39. तान्य॑स्यास्य॒ तानि॒ तान्य॑स्य प्री॒तानि॑ प्री॒तान्य॑स्य॒ तानि॒ तान्य॑स्य प्री॒तानि॑ । \newline
40. अ॒स्य॒ प्री॒तानि॑ प्री॒ता न्य॑स्यास्य प्री॒तानि॑ दे॒वेभ्यो॑ दे॒वेभ्यः॑ प्री॒ता न्य॑स्यास्य प्री॒तानि॑ दे॒वेभ्यः॑ । \newline
41. प्री॒तानि॑ दे॒वेभ्यो॑ दे॒वेभ्यः॑ प्री॒तानि॑ प्री॒तानि॑ दे॒वेभ्यो॑ ह॒व्यꣳ ह॒व्यम् दे॒वेभ्यः॑ प्री॒तानि॑ प्री॒तानि॑ दे॒वेभ्यो॑ ह॒व्यम् । \newline
42. दे॒वेभ्यो॑ ह॒व्यꣳ ह॒व्यम् दे॒वेभ्यो॑ दे॒वेभ्यो॑ ह॒व्यं ॅव॑हन्ति वहन्ति ह॒व्यम् दे॒वेभ्यो॑ दे॒वेभ्यो॑ ह॒व्यं ॅव॑हन्ति । \newline
43. ह॒व्यं ॅव॑हन्ति वहन्ति ह॒व्यꣳ ह॒व्यं ॅव॑ह॒ न्त्यङ्गि॑र॒सो ऽङ्गि॑रसो वहन्ति ह॒व्यꣳ ह॒व्यं ॅव॑ह॒ न्त्यङ्गि॑रसः । \newline
44. व॒ह॒न्त्यङ्गि॑र॒सो ऽङ्गि॑रसो वहन्ति वह॒ न्त्यङ्गि॑रसो॒ वै वा अङ्गि॑रसो वहन्ति वह॒ न्त्यङ्गि॑रसो॒ वै । \newline
45. अङ्गि॑रसो॒ वै वा अङ्गि॑र॒सो ऽङ्गि॑रसो॒ वा इ॒त इ॒तो वा अङ्गि॑र॒सो ऽङ्गि॑रसो॒ वा इ॒तः । \newline
46. वा इ॒त इ॒तो वै वा इ॒त उ॑त्त॒मा उ॑त्त॒मा इ॒तो वै वा इ॒त उ॑त्त॒माः । \newline
47. इ॒त उ॑त्त॒मा उ॑त्त॒मा इ॒त इ॒त उ॑त्त॒माः सु॑व॒र्गꣳ सु॑व॒र्ग मु॑त्त॒मा इ॒त इ॒त उ॑त्त॒माः सु॑व॒र्गम् । \newline
48. उ॒त्त॒माः सु॑व॒र्गꣳ सु॑व॒र्ग मु॑त्त॒मा उ॑त्त॒माः सु॑व॒र्गम् ॅलो॒कम् ॅलो॒कꣳ सु॑व॒र्ग मु॑त्त॒मा उ॑त्त॒माः सु॑व॒र्गम् ॅलो॒कम् । \newline
49. उ॒त्त॒मा इत्यु॑त् - त॒माः । \newline
50. सु॒व॒र्गम् ॅलो॒कम् ॅलो॒कꣳ सु॑व॒र्गꣳ सु॑व॒र्गम् ॅलो॒क मा॑यन् नायन् ॅलो॒कꣳ सु॑व॒र्गꣳ सु॑व॒र्गम् ॅलो॒क मा॑यन्न् । \newline
51. सु॒व॒र्गमिति॑ सुवः - गम् । \newline
52. लो॒क मा॑यन् नायन् ॅलो॒कम् ॅलो॒क मा॑य॒न् तत् तदा॑यन् ॅलो॒कम् ॅलो॒क मा॑य॒न् तत् । \newline
53. आ॒य॒न् तत् तदा॑यन् नाय॒न् तदृष॑य॒ ऋष॑य॒ स्तदा॑यन् नाय॒न् तदृष॑यः । \newline
54. तदृष॑य॒ ऋष॑य॒ स्तत् तदृष॑यो यज्ञ्वा॒स्तु य॑ज्ञ्वा॒ स्त्वृष॑य॒ स्तत् तदृष॑यो यज्ञ्वा॒स्तु । \newline
55. ऋष॑यो यज्ञ्वा॒स्तु य॑ज्ञ्वा॒ स्त्वृष॑य॒ ऋष॑यो यज्ञ्वा॒ स्त्व॑भ्य॒वाय॑न् नभ्य॒वाय॑न्. यज्ञ्वा॒ स्त्वृष॑य॒ ऋष॑यो यज्ञ्वा॒ स्त्व॑भ्य॒वायन्न्॑ । \newline
56. य॒ज्ञ्॒वा॒ स्त्व॑भ्य॒वाय॑न् नभ्य॒वाय॑न्. यज्ञ्वा॒स्तु य॑ज्ञ्वा॒ स्त्व॑भ्य॒वाय॒न् ते ते᳚ ऽभ्य॒वाय॑न्. यज्ञ्वा॒स्तु य॑ज्ञ्वा॒ स्त्व॑भ्य॒वाय॒न् ते । \newline
57. य॒ज्ञ्॒वा॒स्त्विति॑ यज्ञ् - वा॒स्तु । \newline
58. अ॒भ्य॒वाय॒न् ते ते᳚ ऽभ्य॒वाय॑न् नभ्य॒वाय॒न् ते॑ ऽपश्यन् नपश्य॒न् ते᳚ ऽभ्य॒वाय॑न् नभ्य॒वाय॒न् ते॑ ऽपश्यन्न् । \newline
59. अ॒भ्य॒वाय॒न्नित्य॑भि - अ॒वायन्न्॑ । \newline
60. ते॑ ऽपश्यन् नपश्य॒न् ते ते॑ ऽपश्यन् पुरो॒डाश॑म् पुरो॒डाश॑ मपश्य॒न् ते ते॑ ऽपश्यन् पुरो॒डाश᳚म् । \newline
\pagebreak
\markright{ TS 2.6.3.3  \hfill https://www.vedavms.in \hfill}
\addcontentsline{toc}{section}{ TS 2.6.3.3 }
\section*{ TS 2.6.3.3 }

\textbf{TS 2.6.3.3 } \newline
\textbf{Samhita Paata} \newline

ऽपश्यन् पुरो॒डाशं॑ कू॒र्मं भू॒तꣳ सर्प॑न्तं॒ तम॑ब्रुव॒न्निन्द्रा॑य ध्रियस्व॒ बृ॒हस्पत॑ये ध्रियस्व॒ विश्वे᳚भ्यो दे॒वेभ्यो᳚ ध्रिय॒स्वेति॒ स नाद्ध्रि॑यत॒ तम॑ब्रुवन्न॒ग्नये᳚ ध्रिय॒स्वेति॒ सो᳚ऽग्नये᳚ऽद्ध्रियत॒ यदा᳚ग्ने॒यो᳚- ऽष्टाक॑पालो- ऽमावा॒स्या॑यां च पौर्णमा॒स्यां चा᳚च्यु॒तो भव॑ति सुव॒र्गस्य॑ लो॒कस्या॒भिजि॑त्यै॒ तम॑ब्रुवन् क॒थाऽहा᳚स्था॒ इत्यनु॑पाक्तो ऽभूव॒मित्य॑ब्रवी॒द्-यथाऽक्षोऽनु॑पाक्तो॒ - [  ] \newline

\textbf{Pada Paata} \newline

अ॒प॒श्य॒न्न् । पु॒रो॒डाश᳚म् । कू॒र्मम् । भू॒तम् । सर्प॑न्तम् । तम् । अ॒ब्रु॒व॒न्न् । इन्द्रा॑य । ध्रि॒य॒स्व॒ । बृह॒स्पत॑ये । ध्रि॒य॒स्व॒ । विश्वे᳚भ्यः । दे॒वेभ्यः॑ । ध्रि॒य॒स्व॒ । इति॑ । सः । न । अ॒द्ध्रि॒य॒त॒ । तम् । अ॒ब्रु॒व॒न्न् । अ॒ग्नये᳚ । ध्रि॒य॒स्व॒ । इति॑ । सः । अ॒ग्नये᳚ । अ॒द्ध्रि॒य॒त॒ । यत् । आ॒ग्ने॒यः । अ॒ष्टाक॑पाल॒ इत्य॒ष्टा - क॒पा॒लः॒ । अ॒मा॒वा॒स्या॑या॒मित्य॑मा - वा॒स्या॑याम् । च॒ । पौ॒र्ण॒मा॒स्यामिति॑ पौर्ण - मा॒स्याम् । च॒ । अ॒च्यु॒तः । भव॑ति । सु॒व॒र्गस्येति॑ सुवः - गस्य॑ । लो॒कस्य॑ । अ॒भिजि॑त्या॒ इत्य॒भि - जि॒त्यै॒ । तम् । अ॒ब्रु॒व॒न्न् । क॒था । अ॒हा॒स्थाः॒ । इति॑ । अनु॑पाक्त॒ इत्यनु॑प - अ॒क्तः॒ । अ॒भू॒व॒म् । इति॑ । अ॒ब्र॒वी॒त् । यथा᳚ । अक्षः॑ । अनु॑पाक्त॒ इत्यनु॑प - अ॒क्तः॒ ।  \newline


\textbf{Krama Paata} \newline

अ॒प॒श्य॒न् पु॒रो॒डाश᳚म् । पु॒रो॒डाश॑म् कू॒र्मम् । कू॒र्मम् भू॒तम् । भू॒तꣳ सर्प॑न्तम् । सर्प॑न्त॒म् तम् । तम॑ब्रुवन्न् । अ॒ब्रु॒व॒न्निन्द्रा॑य । इन्द्रा॑य ध्रियस्व । ध्रि॒य॒स्व॒ बृह॒स्पत॑ये । बृह॒स्पत॑ये ध्रियस्व । ध्रि॒य॒स्व॒ विश्वे᳚भ्यः । विश्वे᳚भ्यो दे॒वेभ्यः॑ । दे॒वेभ्यो᳚ ध्रियस्व । ध्रि॒य॒स्वेति॑ । इति॒ सः । स न । नाध्रि॑यत । अ॒ध्रि॒य॒त॒ तम् । तम॑ब्रुवन्न् । अ॒ब्रु॒व॒न्न॒ग्नये᳚ । अ॒ग्नये᳚ ध्रियस्व । ध्रि॒य॒स्वेति॑ । इति॒ सः । सो᳚ऽग्नये᳚ । अ॒ग्नये᳚ऽध्रियत । अ॒ध्रि॒य॒त॒ यत् । यदा᳚ग्ने॒यः । आ॒ग्ने॒यो᳚ऽष्टाक॑पालः । अ॒ष्टाक॑पालो ऽमावा॒स्या॑याम् । अ॒ष्टाक॑पाल॒ इत्य॒ष्टा - क॒पा॒लः॒ । अ॒मा॒वा॒स्या॑याम् च । अ॒मा॒वा॒स्या॑या॒मित्य॑मा - वा॒स्या॑याम् । च॒ पौ॒र्ण॒मा॒स्याम् । पौ॒ण॒मा॒स्याम् च॑ । पौ॒र्ण॒मा॒स्यामिति॑ पौर्ण - मा॒स्याम् । चा॒च्यु॒तः । अ॒च्यु॒तो भव॑ति । भव॑ति सुव॒र्गस्य॑ । सु॒व॒र्गस्य॑ लो॒कस्य॑ । सु॒व॒र्गस्येति॑ सुवः - गस्य॑ । लो॒कस्या॒भिजि॑त्यै । अ॒भिजि॑त्यै॒ तम् । अ॒भिजि॑त्या॒ इत्य॒भि - जि॒त्यै॒ । तम॑ब्रुवन्न् । अ॒ब्रु॒व॒न् क॒था । क॒था ऽहा᳚स्थाः । अ॒हा॒स्था॒ इति॑ । इत्यनु॑पाक्तः । अनु॑पाक्तो ऽभूवम् । अनु॑पाक्त॒ इत्यनु॑प - अ॒क्तः॒ । अ॒भू॒व॒मिति॑ । इत्य॑ब्रवीत् । अ॒ब्र॒वी॒द् यथा᳚ । यथाऽक्षः॑ । अक्षोऽनु॑पाक्तः । अनु॑पाक्तो॒ ऽवार्च्छ॑ति । अनु॑पाक्त॒ इत्यनु॑प - अ॒क्तः॒ \newline

\textbf{Jatai Paata} \newline

1. अ॒प॒श्य॒न् पु॒रो॒डाश॑म् पुरो॒डाश॑ मपश्यन् नपश्यन् पुरो॒डाश᳚म् । \newline
2. पु॒रो॒डाश॑म् कू॒र्मम् कू॒र्मम् पु॑रो॒डाश॑म् पुरो॒डाश॑म् कू॒र्मम् । \newline
3. कू॒र्मम् भू॒तम् भू॒तम् कू॒र्मम् कू॒र्मम् भू॒तम् । \newline
4. भू॒तꣳ सर्प॑न्तꣳ॒॒ सर्प॑न्तम् भू॒तम् भू॒तꣳ सर्प॑न्तम् । \newline
5. सर्प॑न्त॒म् तम् तꣳ सर्प॑न्तꣳ॒॒ सर्प॑न्त॒म् तम् । \newline
6. त म॑ब्रुवन् नब्रुव॒न् तम् त म॑ब्रुवन्न् । \newline
7. अ॒ब्रु॒व॒न् निन्द्रा॒ये न्द्रा॑याब्रुवन् नब्रुव॒न् निन्द्रा॑य । \newline
8. इन्द्रा॑य ध्रियस्व ध्रिय॒स्वे न्द्रा॒ये न्द्रा॑य ध्रियस्व । \newline
9. ध्रि॒य॒स्व॒ बृह॒स्पत॑ये॒ बृह॒स्पत॑ये ध्रियस्व ध्रियस्व॒ बृह॒स्पत॑ये । \newline
10. बृह॒स्पत॑ये ध्रियस्व ध्रियस्व॒ बृह॒स्पत॑ये॒ बृह॒स्पत॑ये ध्रियस्व । \newline
11. ध्रि॒य॒स्व॒ विश्वे᳚भ्यो॒ विश्वे᳚भ्यो ध्रियस्व ध्रियस्व॒ विश्वे᳚भ्यः । \newline
12. विश्वे᳚भ्यो दे॒वेभ्यो॑ दे॒वेभ्यो॒ विश्वे᳚भ्यो॒ विश्वे᳚भ्यो दे॒वेभ्यः॑ । \newline
13. दे॒वेभ्यो᳚ ध्रियस्व ध्रियस्व दे॒वेभ्यो॑ दे॒वेभ्यो᳚ ध्रियस्व । \newline
14. ध्रि॒य॒स्वे तीति॑ ध्रियस्व ध्रिय॒स्वे ति॑ । \newline
15. इति॒ स स इतीति॒ सः । \newline
16. स न न स स न । \newline
17. नाद्ध्रि॑यता द्ध्रियत॒ न नाद्ध्रि॑यत । \newline
18. अ॒द्ध्रि॒य॒त॒ तम् त म॑द्ध्रियता द्ध्रियत॒ तम् । \newline
19. त म॑ब्रुवन् नब्रुव॒न् तम् त म॑ब्रुवन्न् । \newline
20. अ॒ब्रु॒व॒न् न॒ग्नये॒ ऽग्नये᳚ ऽब्रुवन् नब्रुवन् न॒ग्नये᳚ । \newline
21. अ॒ग्नये᳚ ध्रियस्व ध्रियस्वा॒ग्नये॒ ऽग्नये᳚ ध्रियस्व । \newline
22. ध्रि॒य॒स्वे तीति॑ ध्रियस्व ध्रिय॒स्वे ति॑ । \newline
23. इति॒ स स इतीति॒ सः । \newline
24. सो᳚ ऽग्नये॒ ऽग्नये॒ स सो᳚ ऽग्नये᳚ । \newline
25. अ॒ग्नये᳚ ऽद्ध्रियता द्ध्रियता॒ ग्नये॒ ऽग्नये᳚ ऽद्ध्रियत । \newline
26. अ॒द्ध्रि॒य॒त॒ यद् यद॑द्ध्रियता द्ध्रियत॒ यत् । \newline
27. यदा᳚ग्ने॒य आ᳚ग्ने॒यो यद् यदा᳚ग्ने॒यः । \newline
28. आ॒ग्ने॒यो᳚ ऽष्टाक॑पालो॒ ऽष्टाक॑पाल आग्ने॒य आ᳚ग्ने॒यो᳚ ऽष्टाक॑पालः । \newline
29. अ॒ष्टाक॑पालो ऽमावा॒स्या॑या ममावा॒स्या॑या म॒ष्टाक॑पालो॒ ऽष्टाक॑पालो ऽमावा॒स्या॑याम् । \newline
30. अ॒ष्टाक॑पाल॒ इत्य॒ष्टा - क॒पा॒लः॒ । \newline
31. अ॒मा॒वा॒स्या॑याम् च चामावा॒स्या॑या ममावा॒स्या॑याम् च । \newline
32. अ॒मा॒वा॒स्या॑या॒मित्य॑मा - वा॒स्या॑याम् । \newline
33. च॒ पौ॒र्ण॒मा॒स्याम् पौ᳚र्णमा॒स्याम् च॑ च पौर्णमा॒स्याम् । \newline
34. पौ॒र्ण॒मा॒स्याम् च॑ च पौर्णमा॒स्याम् पौ᳚र्णमा॒स्याम् च॑ । \newline
35. पौ॒र्ण॒मा॒स्यामिति॑ पौर्ण - मा॒स्याम् । \newline
36. चा॒च्यु॒तो अ॑च्यु॒तश्च॑ चाच्यु॒तः । \newline
37. अ॒च्यु॒तो भव॑ति॒ भव॑ त्यच्यु॒तो अ॑च्यु॒तो भव॑ति । \newline
38. भव॑ति सुव॒र्गस्य॑ सुव॒र्गस्य॒ भव॑ति॒ भव॑ति सुव॒र्गस्य॑ । \newline
39. सु॒व॒र्गस्य॑ लो॒कस्य॑ लो॒कस्य॑ सुव॒र्गस्य॑ सुव॒र्गस्य॑ लो॒कस्य॑ । \newline
40. सु॒व॒र्गस्येति॑ सुवः - गस्य॑ । \newline
41. लो॒कस्या॒ भिजि॑त्या अ॒भिजि॑त्यै लो॒कस्य॑ लो॒कस्या॒ भिजि॑त्यै । \newline
42. अ॒भिजि॑त्यै॒ तम् त म॒भिजि॑त्या अ॒भिजि॑त्यै॒ तम् । \newline
43. अ॒भिजि॑त्या॒ इत्य॒भि - जि॒त्यै॒ । \newline
44. त म॑ब्रुवन् नब्रुव॒न् तम् त म॑ब्रुवन्न् । \newline
45. अ॒ब्रु॒व॒न् क॒था क॒था ऽब्रु॑वन् नब्रुवन् क॒था । \newline
46. क॒था ऽहा᳚स्था अहास्थाः क॒था क॒था ऽहा᳚स्थाः । \newline
47. अ॒हा॒स्था॒ इती त्य॑हास्था अहास्था॒ इति॑ । \newline
48. इत्यनु॑पा॒क्तो ऽनु॑पाक्त॒ इतीत्यनु॑पाक्तः । \newline
49. अनु॑पाक्तो ऽभूव मभूव॒ मनु॑पा॒क्तो ऽनु॑पाक्तो ऽभूवम् । \newline
50. अनु॑पाक्त॒ इत्यनु॑प - अ॒क्तः॒ । \newline
51. अ॒भू॒व॒ मिती त्य॑भूव मभूव॒ मिति॑ । \newline
52. इत्य॑ब्रवी दब्रवी॒ दिती त्य॑ब्रवीत् । \newline
53. अ॒ब्र॒वी॒द् यथा॒ यथा᳚ ऽब्रवी दब्रवी॒द् यथा᳚ । \newline
54. यथा ऽक्षो ऽक्षो॒ यथा॒ यथा ऽक्षः॑ । \newline
55. अक्षो ऽनु॑पा॒क्तो ऽनु॑पा॒क्तो ऽक्षो ऽक्षो ऽनु॑पाक्तः । \newline
56. अनु॑पाक्तो॒ ऽवार्च्छ॑ त्य॒वार्च्छ॒ त्यनु॑पा॒क्तो ऽनु॑पाक्तो॒ ऽवार्च्छ॑ति । \newline
57. अनु॑पाक्त॒ इत्यनु॑प - अ॒क्तः॒ । \newline

\textbf{Ghana Paata } \newline

1. अ॒प॒श्य॒न् पु॒रो॒डाश॑म् पुरो॒डाश॑ मपश्यन् नपश्यन् पुरो॒डाश॑म् कू॒र्मम् कू॒र्मम् पु॑रो॒डाश॑ मपश्यन् नपश्यन् पुरो॒डाश॑म् कू॒र्मम् । \newline
2. पु॒रो॒डाश॑म् कू॒र्मम् कू॒र्मम् पु॑रो॒डाश॑म् पुरो॒डाश॑म् कू॒र्मम् भू॒तम् भू॒तम् कू॒र्मम् पु॑रो॒डाश॑म् पुरो॒डाश॑म् कू॒र्मम् भू॒तम् । \newline
3. कू॒र्मम् भू॒तम् भू॒तम् कू॒र्मम् कू॒र्मम् भू॒तꣳ सर्प॑न्तꣳ॒॒ सर्प॑न्तम् भू॒तम् कू॒र्मम् कू॒र्मम् भू॒तꣳ सर्प॑न्तम् । \newline
4. भू॒तꣳ सर्प॑न्तꣳ॒॒ सर्प॑न्तम् भू॒तम् भू॒तꣳ सर्प॑न्त॒म् तम् तꣳ सर्प॑न्तम् भू॒तम् भू॒तꣳ सर्प॑न्त॒म् तम् । \newline
5. सर्प॑न्त॒म् तम् तꣳ सर्प॑न्तꣳ॒॒ सर्प॑न्त॒म् त म॑ब्रुवन् नब्रुव॒न् तꣳ सर्प॑न्तꣳ॒॒ सर्प॑न्त॒म् त म॑ब्रुवन्न् । \newline
6. त म॑ब्रुवन् नब्रुव॒न् तम् त म॑ब्रुव॒न् निन्द्रा॒ये न्द्रा॑याब्रुव॒न् तम् त म॑ब्रुव॒न् निन्द्रा॑य । \newline
7. अ॒ब्रु॒व॒न् निन्द्रा॒ये न्द्रा॑याब्रुवन् नब्रुव॒न् निन्द्रा॑य ध्रियस्व ध्रिय॒स्वे न्द्रा॑याब्रुवन् नब्रुव॒न् निन्द्रा॑य ध्रियस्व । \newline
8. इन्द्रा॑य ध्रियस्व ध्रिय॒स्वे न्द्रा॒ये न्द्रा॑य ध्रियस्व॒ बृह॒स्पत॑ये॒ बृह॒स्पत॑ये ध्रिय॒स्वे न्द्रा॒ये न्द्रा॑य ध्रियस्व॒ बृह॒स्पत॑ये । \newline
9. ध्रि॒य॒स्व॒ बृह॒स्पत॑ये॒ बृह॒स्पत॑ये ध्रियस्व ध्रियस्व॒ बृह॒स्पत॑ये ध्रियस्व ध्रियस्व॒ बृह॒स्पत॑ये ध्रियस्व ध्रियस्व॒ बृह॒स्पत॑ये ध्रियस्व । \newline
10. बृह॒स्पत॑ये ध्रियस्व ध्रियस्व॒ बृह॒स्पत॑ये॒ बृह॒स्पत॑ये ध्रियस्व॒ विश्वे᳚भ्यो॒ विश्वे᳚भ्यो ध्रियस्व॒ बृह॒स्पत॑ये॒ बृह॒स्पत॑ये ध्रियस्व॒ विश्वे᳚भ्यः । \newline
11. ध्रि॒य॒स्व॒ विश्वे᳚भ्यो॒ विश्वे᳚भ्यो ध्रियस्व ध्रियस्व॒ विश्वे᳚भ्यो दे॒वेभ्यो॑ दे॒वेभ्यो॒ विश्वे᳚भ्यो ध्रियस्व ध्रियस्व॒ विश्वे᳚भ्यो दे॒वेभ्यः॑ । \newline
12. विश्वे᳚भ्यो दे॒वेभ्यो॑ दे॒वेभ्यो॒ विश्वे᳚भ्यो॒ विश्वे᳚भ्यो दे॒वेभ्यो᳚ ध्रियस्व ध्रियस्व दे॒वेभ्यो॒ विश्वे᳚भ्यो॒ विश्वे᳚भ्यो दे॒वेभ्यो᳚ ध्रियस्व । \newline
13. दे॒वेभ्यो᳚ ध्रियस्व ध्रियस्व दे॒वेभ्यो॑ दे॒वेभ्यो᳚ ध्रिय॒स्वे तीति॑ ध्रियस्व दे॒वेभ्यो॑ दे॒वेभ्यो᳚ ध्रिय॒स्वे ति॑ । \newline
14. ध्रि॒य॒स्वे तीति॑ ध्रियस्व ध्रिय॒स्वे ति॒ स स इति॑ ध्रियस्व ध्रिय॒स्वे ति॒ सः । \newline
15. इति॒ स स इतीति॒ स न न स इतीति॒ स न । \newline
16. स न न स स नाद्ध्रि॑यता द्ध्रियत॒ न स स नाद्ध्रि॑यत । \newline
17. नाद्ध्रि॑यता द्ध्रियत॒ न नाद्ध्रि॑यत॒ तम् त म॑द्ध्रियत॒ न नाद्ध्रि॑यत॒ तम् । \newline
18. अ॒द्ध्रि॒य॒त॒ तम् त म॑द्ध्रियता द्ध्रियत॒ त म॑ब्रुवन् नब्रुव॒न् त म॑द्ध्रियता द्ध्रियत॒ त म॑ब्रुवन्न् । \newline
19. त म॑ब्रुवन् नब्रुव॒न् तम् त म॑ब्रुवन् न॒ग्नये॒ ऽग्नये᳚ ऽब्रुव॒न् तम् त म॑ब्रुवन् न॒ग्नये᳚ । \newline
20. अ॒ब्रु॒व॒न् न॒ग्नये॒ ऽग्नये᳚ ऽब्रुवन् नब्रुवन् न॒ग्नये᳚ ध्रियस्व ध्रियस्वा॒ग्नये᳚ ऽब्रुवन् नब्रुवन् न॒ग्नये᳚ ध्रियस्व । \newline
21. अ॒ग्नये᳚ ध्रियस्व ध्रियस्वा॒ग्नये॒ ऽग्नये᳚ ध्रिय॒स्वे तीति॑ ध्रियस्वा॒ग्नये॒ ऽग्नये᳚ ध्रिय॒स्वे ति॑ । \newline
22. ध्रि॒य॒स्वे तीति॑ ध्रियस्व ध्रिय॒स्वे ति॒ स स इति॑ ध्रियस्व ध्रिय॒स्वे ति॒ सः । \newline
23. इति॒ स स इतीति॒ सो᳚ ऽग्नये॒ ऽग्नये॒ स इतीति॒ सो᳚ ऽग्नये᳚ । \newline
24. सो᳚ ऽग्नये॒ ऽग्नये॒ स सो᳚ ऽग्नये᳚ ऽद्ध्रियता द्ध्रियता॒ग्नये॒ स सो᳚ ऽग्नये᳚ ऽद्ध्रियत । \newline
25. अ॒ग्नये᳚ ऽद्ध्रियता द्ध्रियता॒ग्नये॒ ऽग्नये᳚ ऽद्ध्रियत॒ यद् यद॑द्ध्रियता॒ग्नये॒ ऽग्नये᳚ ऽद्ध्रियत॒ यत् । \newline
26. अ॒द्ध्रि॒य॒त॒ यद् यद॑द्ध्रियता द्ध्रियत॒ यदा᳚ग्ने॒य आ᳚ग्ने॒यो यद॑द्ध्रियता द्ध्रियत॒ यदा᳚ग्ने॒यः । \newline
27. यदा᳚ग्ने॒य आ᳚ग्ने॒यो यद् यदा᳚ग्ने॒यो᳚ ऽष्टाक॑पालो॒ ऽष्टाक॑पाल आग्ने॒यो यद् यदा᳚ग्ने॒यो᳚ ऽष्टाक॑पालः । \newline
28. आ॒ग्ने॒यो᳚ ऽष्टाक॑पालो॒ ऽष्टाक॑पाल आग्ने॒य आ᳚ग्ने॒यो᳚ ऽष्टाक॑पालो ऽमावा॒स्या॑या ममावा॒स्या॑या म॒ष्टाक॑पाल आग्ने॒य आ᳚ग्ने॒यो᳚ ऽष्टाक॑पालो ऽमावा॒स्या॑याम् । \newline
29. अ॒ष्टाक॑पालो ऽमावा॒स्या॑या ममावा॒स्या॑या म॒ष्टाक॑पालो॒ ऽष्टाक॑पालो ऽमावा॒स्या॑याम् च चामावा॒स्या॑या म॒ष्टाक॑पालो॒ ऽष्टाक॑पालो ऽमावा॒स्या॑याम् च । \newline
30. अ॒ष्टाक॑पाल॒ इत्य॒ष्टा - क॒पा॒लः॒ । \newline
31. अ॒मा॒वा॒स्या॑याम् च चामावा॒स्या॑या ममावा॒स्या॑याम् च पौर्णमा॒स्याम् पौ᳚र्णमा॒स्याम् चा॑मावा॒स्या॑या ममावा॒स्या॑याम् च पौर्णमा॒स्याम् । \newline
32. अ॒मा॒वा॒स्या॑या॒मित्य॑मा - वा॒स्या॑याम् । \newline
33. च॒ पौ॒र्ण॒मा॒स्याम् पौ᳚र्णमा॒स्याम् च॑ च पौर्णमा॒स्याम् च॑ च पौर्णमा॒स्याम् च॑ च पौर्णमा॒स्याम् च॑ । \newline
34. पौ॒र्ण॒मा॒स्याम् च॑ च पौर्णमा॒स्याम् पौ᳚र्णमा॒स्याम् चा᳚च्यु॒तो अ॑च्यु॒तश्च॑ पौर्णमा॒स्याम् पौ᳚र्णमा॒स्याम् चा᳚च्यु॒तः । \newline
35. पौ॒र्ण॒मा॒स्यामिति॑ पौर्ण - मा॒स्याम् । \newline
36. चा॒च्यु॒तो अ॑च्यु॒तश्च॑ चाच्यु॒तो भव॑ति॒ भव॑ त्यच्यु॒तश्च॑ चाच्यु॒तो भव॑ति । \newline
37. अ॒च्यु॒तो भव॑ति॒ भव॑त्यच्यु॒तो अ॑च्यु॒तो भव॑ति सुव॒र्गस्य॑ सुव॒र्गस्य॒ भव॑त्यच्यु॒तो अ॑च्यु॒तो भव॑ति सुव॒र्गस्य॑ । \newline
38. भव॑ति सुव॒र्गस्य॑ सुव॒र्गस्य॒ भव॑ति॒ भव॑ति सुव॒र्गस्य॑ लो॒कस्य॑ लो॒कस्य॑ सुव॒र्गस्य॒ भव॑ति॒ भव॑ति सुव॒र्गस्य॑ लो॒कस्य॑ । \newline
39. सु॒व॒र्गस्य॑ लो॒कस्य॑ लो॒कस्य॑ सुव॒र्गस्य॑ सुव॒र्गस्य॑ लो॒कस्या॒ भिजि॑त्या अ॒भिजि॑त्यै लो॒कस्य॑ सुव॒र्गस्य॑ सुव॒र्गस्य॑ लो॒कस्या॒ भिजि॑त्यै । \newline
40. सु॒व॒र्गस्येति॑ सुवः - गस्य॑ । \newline
41. लो॒कस्या॒ भिजि॑त्या अ॒भिजि॑त्यै लो॒कस्य॑ लो॒कस्या॒ भिजि॑त्यै॒ तम् त म॒भिजि॑त्यै लो॒कस्य॑ लो॒कस्या॒ भिजि॑त्यै॒ तम् । \newline
42. अ॒भिजि॑त्यै॒ तम् त म॒भिजि॑त्या अ॒भिजि॑त्यै॒ त म॑ब्रुवन् नब्रुव॒न् त म॒भिजि॑त्या अ॒भिजि॑त्यै॒ त म॑ब्रुवन्न् । \newline
43. अ॒भिजि॑त्या॒ इत्य॒भि - जि॒त्यै॒ । \newline
44. त म॑ब्रुवन् नब्रुव॒न् तम् त म॑ब्रुवन् क॒था क॒था ऽब्रु॑व॒न् तम् त म॑ब्रुवन् क॒था । \newline
45. अ॒ब्रु॒व॒न् क॒था क॒था ऽब्रु॑वन् नब्रुवन् क॒था ऽहा᳚स्था अहास्थाः क॒था ऽब्रु॑वन् नब्रुवन् क॒था ऽहा᳚स्थाः । \newline
46. क॒था ऽहा᳚स्था अहास्थाः क॒था क॒था ऽहा᳚स्था॒ इतीत्य॑हास्थाः क॒था क॒था ऽहा᳚स्था॒ इति॑ । \newline
47. अ॒हा॒स्था॒ इतीत्य॑हास्था अहास्था॒ इत्यनु॑पा॒क्तो ऽनु॑पाक्त॒ इत्य॑हास्था अहास्था॒ इत्यनु॑पाक्तः । \newline
48. इत्यनु॑पा॒क्तो ऽनु॑पाक्त॒ इतीत्यनु॑पाक्तो ऽभूव मभूव॒ मनु॑पाक्त॒ इतीत्यनु॑पाक्तो ऽभूवम् । \newline
49. अनु॑पाक्तो ऽभूव मभूव॒ मनु॑पा॒क्तो ऽनु॑पाक्तो ऽभूव॒ मितीत्य॑भूव॒ मनु॑पा॒क्तो ऽनु॑पाक्तो ऽभूव॒ मिति॑ । \newline
50. अनु॑पाक्त॒ इत्यनु॑प - अ॒क्तः॒ । \newline
51. अ॒भू॒व॒ मितीत्य॑भूव मभूव॒ मित्य॑ब्रवी दब्रवी॒ दित्य॑भूव मभूव॒ मित्य॑ब्रवीत् । \newline
52. इत्य॑ब्रवी दब्रवी॒ दितीत्य॑ब्रवी॒द् यथा॒ यथा᳚ ऽब्रवी॒ दितीत्य॑ब्रवी॒द् यथा᳚ । \newline
53. अ॒ब्र॒वी॒द् यथा॒ यथा᳚ ऽब्रवी दब्रवी॒द् यथा ऽक्षो ऽक्षो॒ यथा᳚ ऽब्रवी दब्रवी॒द् यथा ऽक्षः॑ । \newline
54. यथा ऽक्षो ऽक्षो॒ यथा॒ यथा ऽक्षो ऽनु॑पा॒क्तो ऽनु॑पा॒क्तो ऽक्षो॒ यथा॒ यथा ऽक्षो ऽनु॑पाक्तः । \newline
55. अक्षो ऽनु॑पा॒क्तो ऽनु॑पा॒क्तो ऽक्षो ऽक्षो ऽनु॑पाक्तो॒ ऽवार्च्छ॑ त्य॒वार्च्छ॒ त्यनु॑पा॒क्तो ऽक्षो ऽक्षो ऽनु॑पाक्तो॒ ऽवार्च्छ॑ति । \newline
56. अनु॑पाक्तो॒ ऽवार्च्छ॑ त्य॒वार्च्छ॒ त्यनु॑पा॒क्तो ऽनु॑पाक्तो॒ ऽवार्च्छ॑त्ये॒व मे॒व म॒वार्च्छ॒ त्यनु॑पा॒क्तो ऽनु॑पाक्तो॒ ऽवार्च्छ॑त्ये॒वम् । \newline
57. अनु॑पाक्त॒ इत्यनु॑प - अ॒क्तः॒ । \newline
\pagebreak
\markright{ TS 2.6.3.4  \hfill https://www.vedavms.in \hfill}
\addcontentsline{toc}{section}{ TS 2.6.3.4 }
\section*{ TS 2.6.3.4 }

\textbf{TS 2.6.3.4 } \newline
\textbf{Samhita Paata} \newline

ऽवार्च्छ॑त्ये॒वमवा॑ ऽऽर॒मित्यु॒परि॑ष्टा-द॒भ्यज्या॒धस्ता॒-दुपा॑नक्ति सुव॒र्गस्य॑ लो॒कस्य॒ सम॑ष्‌ट्यै॒ सर्वा॑णि क॒पाला᳚न्य॒भि प्र॑थयति॒ ताव॑तः पुरो॒डाशा॑न॒मुष्मि॑न् ॅलो॒के॑ऽभि ज॑यति॒ यो विद॑ग्धः॒ स नै॑र्.ऋ॒तो योऽशृ॑तः॒ स रौ॒द्रो यः शृ॒तः स सदे॑व॒स्तस्मा॒दवि॑दहता शृत॒कृंत्यः॑ सदेव॒त्वाय॒ भस्म॑ना॒ऽभि वा॑सयति॒ तस्मा᳚न्माꣳ॒॒ सेनास्थि॑ छ॒न्नं ॅवे॒देना॒भि वा॑सयति॒ तस्मा॒त् - [  ] \newline

\textbf{Pada Paata} \newline

अ॒वार्च्छ॒तीत्य॑व - ऋच्छ॑ति । ए॒वम् । अवेति॑ । आ॒र॒म् । इति॑ । उ॒परि॑ष्टात् । अ॒भ्यज्येत्य॑भि - अज्य॑ । अ॒धस्ता᳚त् । उपेति॑ । अ॒न॒क्ति॒ । सु॒व॒र्गस्येति॑ सुवः - गस्य॑ । लो॒कस्य॑ । सम॑ष्ट्या॒ इति॒ सं-अ॒ष्ट्यै॒ । सर्वा॑णि । क॒पाला॑नि । अ॒भीति॑ । प्र॒थ॒य॒ति॒ । ताव॑तः । पु॒रो॒डाशान्॑ । अ॒मुष्मिन्न्॑ । लो॒के । अ॒भीति॑ । ज॒य॒ति॒ । यः । विद॑ग्ध॒ इति॒ वि - द॒ग्धः॒ । सः । नै॒र्॒.ऋ॒त इति॑ नैः-ऋ॒तः । यः । अशृ॑तः । सः । रौ॒द्रः । यः । शृ॒तः । सः । सदे॑व॒ इति॒ स - दे॒वः॒ । तस्मा᳚त् । अवि॑दह॒तेत्यवि॑ - द॒ह॒ता॒ । शृ॒त॒कृंत्य॒ इति॑ शृतं - कृत्यः॑ । स॒दे॒व॒त्वायेति॑ सदेव - त्वाय॑ । भस्म॑ना । अ॒भीति॑ । वा॒स॒य॒ति॒ । तस्मा᳚त् । माꣳ॒॒सेन॑ । अस्थि॑ । छ॒न्नम् । वे॒देन॑ । अ॒भीति॑ । वा॒स॒य॒ति॒ । तस्मा᳚त् ।  \newline


\textbf{Krama Paata} \newline

अ॒वार्च्छ॑त्ये॒वम् । अ॒वार्च्छ॒तीत्य॑व - ऋच्छ॑ति । ए॒वमव॑ । अवा॑रम् । आ॒रमिति॑ । इत्यु॒परि॑ष्टात् । उ॒परि॑ष्टाद॒भ्यज्य॑ । अ॒भ्यज्या॒धस्ता᳚त् । अ॒भ्यज्येत्य॑भि - अज्य॑ । अ॒धस्ता॒दुप॑ । उपा॑नक्ति । अ॒न॒क्ति॒ सु॒व॒र्गस्य॑ । सु॒व॒र्गस्य॑ लो॒कस्य॑ । सु॒व॒र्गस्येति॑ सुवः - गस्य॑ । लो॒कस्य॒ सम॑ष्ट्यै । सम॑ष्ट्यै॒ सर्वा॑णि । सम॑ष्ट्या॒ इति॒ सं - अ॒ष्ट्यै॒ । सर्वा॑णि क॒पाला॑नि । क॒पाला᳚न्य॒भि । अ॒भि प्र॑थयति । प्र॒थ॒य॒ति॒ ताव॑तः । ताव॑तः पुरो॒डाशान्॑ । पु॒रो॒डाशा॑न॒मुष्मिन्न्॑ । अ॒मुष्मि॑न् ॅलो॒के । लो॒के॑ऽभि । अ॒भि ज॑यति । ज॒य॒ति॒ यः । यो विद॑ग्धः । विद॑ग्धः॒ सः । विद॑ग्ध॒ इति॒ वि - द॒ग्धः॒ । स नै॑र्.ऋ॒तः । नै॒र्॒.ऋ॒तो यः । नै॒र्॒.ऋ॒त इति॑ नैः - ऋ॒तः । योऽशृ॑तः । अशृ॑तः॒ सः । स रौ॒द्रः । रौ॒द्रो यः । यः शृ॒तः । शृ॒तः सः । स सदे॑वः । सदे॑व॒स्तस्मा᳚त् । सदे॑व॒ इति॒ स - दे॒वः॒ । तस्मा॒दवि॑दहता । अवि॑दहता शृत॒ङ्कृत्यः॑ । अवि॑दह॒तेत्यवि॑ - द॒ह॒ता॒ । शृ॒त॒ङ्कृत्यः॑ सदेव॒त्वाय॑ । शृ॒त॒ङ्कृत्य॒ इति॑ शृतम् - कृत्यः॑ । स॒दे॒व॒त्वाय॒ भस्म॑ना । स॒दे॒व॒त्वायेति॑ सदेव - त्वाय॑ । भस्म॑ना॒ऽभि । अ॒भि वा॑सयति । वा॒स॒य॒ति॒ तस्मा᳚त् । तस्मा᳚न् माꣳ॒॒सेन॑ । माꣳ॒॒सेनास्थि॑ । अस्थि॑ छ॒न्नम् । छ॒न्नं ॅवे॒देन॑ । वे॒देना॒भि । अ॒भि वा॑सयति । वा॒स॒य॒ति॒ तस्मा᳚त् । तस्मा॒त् केशैः᳚ \newline

\textbf{Jatai Paata} \newline

1. अ॒वार्च्छ॑ त्ये॒व मे॒व म॒वार्च्छ॑ त्य॒वार्च्छ॑ त्ये॒वम् । \newline
2. अ॒वार्च्छ॒तीत्य॑व - ऋच्छ॑ति । \newline
3. ए॒व मवावै॒व मे॒व मव॑ । \newline
4. अवा॑र मार॒ मवावा॑रम् । \newline
5. आ॒र॒ मिती त्या॑र मार॒ मिति॑ । \newline
6. इत्यु॒परि॑ष्टा दु॒परि॑ष्टा॒ दिती त्यु॒परि॑ष्टात् । \newline
7. उ॒परि॑ष्टा द॒भ्यज्या॒ भ्यज्यो॒परि॑ष्टा दु॒परि॑ष्टा द॒भ्यज्य॑ । \newline
8. अ॒भ्यज्या॒ धस्ता॑ द॒धस्ता॑ द॒भ्यज्या॒ भ्यज्या॒ धस्ता᳚त् । \newline
9. अ॒भ्यज्येत्य॑भि - अज्य॑ । \newline
10. अ॒धस्ता॒ दुपोपा॒धस्ता॑ द॒धस्ता॒दुप॑ । \newline
11. उपा॑नक्त्यन॒ क्त्युपोपा॑नक्ति । \newline
12. अ॒न॒क्ति॒ सु॒व॒र्गस्य॑ सुव॒र्गस्या॑ नक्त्यनक्ति सुव॒र्गस्य॑ । \newline
13. सु॒व॒र्गस्य॑ लो॒कस्य॑ लो॒कस्य॑ सुव॒र्गस्य॑ सुव॒र्गस्य॑ लो॒कस्य॑ । \newline
14. सु॒व॒र्गस्येति॑ सुवः - गस्य॑ । \newline
15. लो॒कस्य॒ सम॑ष्ट्यै॒ सम॑ष्ट्यै लो॒कस्य॑ लो॒कस्य॒ सम॑ष्ट्यै । \newline
16. सम॑ष्ट्यै॒ सर्वा॑णि॒ सर्वा॑णि॒ सम॑ष्ट्यै॒ सम॑ष्ट्यै॒ सर्वा॑णि । \newline
17. सम॑ष्ट्या॒ इति॒ सं - अ॒ष्ट्यै॒ । \newline
18. सर्वा॑णि क॒पाला॑नि क॒पाला॑नि॒ सर्वा॑णि॒ सर्वा॑णि क॒पाला॑नि । \newline
19. क॒पाला᳚ न्य॒भ्य॑भि क॒पाला॑नि क॒पाला᳚ न्य॒भि । \newline
20. अ॒भि प्र॑थयति प्रथय त्य॒भ्य॑भि प्र॑थयति । \newline
21. प्र॒थ॒य॒ति॒ ताव॑त॒ स्ताव॑तः प्रथयति प्रथयति॒ ताव॑तः । \newline
22. ताव॑तः पुरो॒डाशा᳚न् पुरो॒डाशा॒न् ताव॑त॒ स्ताव॑तः पुरो॒डाशान्॑ । \newline
23. पु॒रो॒डाशा॑ न॒मुष्मि॑न् न॒मुष्मि॑न् पुरो॒डाशा᳚न् पुरो॒डाशा॑ न॒मुष्मिन्न्॑ । \newline
24. अ॒मुष्मि॑न् ॅलो॒के लो॒के॑ ऽमुष्मि॑न् न॒मुष्मि॑न् ॅलो॒के । \newline
25. लो॒के᳚(ए1॒) ऽभ्य॑भि लो॒के लो॒के॑ ऽभि । \newline
26. अ॒भि ज॑यति जय त्य॒भ्य॑भि ज॑यति । \newline
27. ज॒य॒ति॒ यो यो ज॑यति जयति॒ यः । \newline
28. यो विद॑ग्धो॒ विद॑ग्धो॒ यो यो विद॑ग्धः । \newline
29. विद॑ग्धः॒ स स विद॑ग्धो॒ विद॑ग्धः॒ सः । \newline
30. विद॑ग्ध॒ इति॒ वि - द॒ग्धः॒ । \newline
31. स नैर्॑.ऋ॒तो नैर्॑.ऋ॒तः स स नैर्॑.ऋ॒तः । \newline
32. नै॒र्॒.ऋ॒तो यो यो नैर्॑.ऋ॒तो नैर्॑.ऋ॒तो यः । \newline
33. नै॒र्॒.ऋ॒त इति॑ नैः - ऋ॒तः । \newline
34. यो ऽशृ॒तो ऽशृ॑तो॒ यो यो ऽशृ॑तः । \newline
35. अशृ॑तः॒ स सो ऽशृ॒तो ऽशृ॑तः॒ सः । \newline
36. स रौ॒द्रो रौ॒द्रः स स रौ॒द्रः । \newline
37. रौ॒द्रो यो यो रौ॒द्रो रौ॒द्रो यः । \newline
38. यः शृ॒तः शृ॒तो यो यः शृ॒तः । \newline
39. शृ॒तः स स शृ॒तः शृ॒तः सः । \newline
40. स सदे॑वः॒ सदे॑वः॒ स स सदे॑वः । \newline
41. सदे॑व॒ स्तस्मा॒त् तस्मा॒थ् सदे॑वः॒ सदे॑व॒ स्तस्मा᳚त् । \newline
42. सदे॑व॒ इति॒ स - दे॒वः॒ । \newline
43. तस्मा॒ दवि॑दह॒ता ऽवि॑दहता॒ तस्मा॒त् तस्मा॒ दवि॑दहता । \newline
44. अवि॑दहता शृत॒ङ्कृत्यः॑ शृत॒ङ्कृत्यो ऽवि॑दह॒ता ऽवि॑दहता शृत॒ङ्कृत्यः॑ । \newline
45. अवि॑दह॒तेत्यवि॑ - द॒ह॒ता॒ । \newline
46. शृ॒त॒ङ्कृत्यः॑ सदेव॒त्वाय॑ सदेव॒त्वाय॑ शृत॒ङ्कृत्यः॑ शृत॒ङ्कृत्यः॑ सदेव॒त्वाय॑ । \newline
47. शृ॒त॒ङ्कृत्य॒ इति॑ शृतं - कृत्यः॑ । \newline
48. स॒दे॒व॒त्वाय॒ भस्म॑ना॒ भस्म॑ना सदेव॒त्वाय॑ सदेव॒त्वाय॒ भस्म॑ना । \newline
49. स॒दे॒व॒त्वायेति॑ सदेव - त्वाय॑ । \newline
50. भस्म॑ना॒ ऽभ्य॑भि भस्म॑ना॒ भस्म॑ना॒ ऽभि । \newline
51. अ॒भि वा॑सयति वासय त्य॒भ्य॑भि वा॑सयति । \newline
52. वा॒स॒य॒ति॒ तस्मा॒त् तस्मा᳚द् वासयति वासयति॒ तस्मा᳚त् । \newline
53. तस्मा᳚न् माꣳ॒॒सेन॑ माꣳ॒॒सेन॒ तस्मा॒त् तस्मा᳚न् माꣳ॒॒सेन॑ । \newline
54. माꣳ॒॒सेना स्थ्यस्थि॑ माꣳ॒॒सेन॑ माꣳ॒॒सेना स्थि॑ । \newline
55. अस्थि॑ छ॒न्नम् छ॒न्न मस्थ्यस्थि॑ छ॒न्नम् । \newline
56. छ॒न्नं ॅवे॒देन॑ वे॒देन॑ छ॒न्नम् छ॒न्नं ॅवे॒देन॑ । \newline
57. वे॒देना॒ भ्य॑भि वे॒देन॑ वे॒देना॒भि । \newline
58. अ॒भि वा॑सयति वासय त्य॒भ्य॑भि वा॑सयति । \newline
59. वा॒स॒य॒ति॒ तस्मा॒त् तस्मा᳚द् वासयति वासयति॒ तस्मा᳚त् । \newline
60. तस्मा॒त् केशैः॒ केशै॒ स्तस्मा॒त् तस्मा॒त् केशैः᳚ । \newline

\textbf{Ghana Paata } \newline

1. अ॒वार्च्छ॑त्ये॒व मे॒व म॒वार्च्छ॑ त्य॒वार्च्छ॑त्ये॒व मवावै॒व म॒वार्च्छ॑ त्य॒वार्च्छ॑त्ये॒व मव॑ । \newline
2. अ॒वार्च्छ॒तीत्य॑व - ऋच्छ॑ति । \newline
3. ए॒व मवावै॒व मे॒व मवा॑र मार॒ मवै॒व मे॒व मवा॑रम् । \newline
4. अवा॑र मार॒ मवावा॑र॒ मितीत्या॑र॒ मवावा॑र॒ मिति॑ । \newline
5. आ॒र॒ मितीत्या॑र मार॒ मित्यु॒परि॑ष्टा दु॒परि॑ष्टा॒ दित्या॑र मार॒ मित्यु॒परि॑ष्टात् । \newline
6. इत्यु॒परि॑ष्टा दु॒परि॑ष्टा॒ दितीत्यु॒परि॑ष्टा द॒भ्यज्या॒भ्यज्यो॒ परि॑ष्टा॒ दितीत्यु॒परि॑ष्टा द॒भ्यज्य॑ । \newline
7. उ॒परि॑ष्टा द॒भ्यज्या॒ भ्यज्यो॒ परि॑ष्टा दु॒परि॑ष्टा द॒भ्यज्या॒ धस्ता॑ द॒धस्ता॑ द॒भ्यज्यो॒ परि॑ष्टा दु॒परि॑ष्टा द॒भ्यज्या॒ धस्ता᳚त् । \newline
8. अ॒भ्यज्या॒ धस्ता॑ द॒धस्ता॑ द॒भ्यज्या॒ भ्यज्या॒ धस्ता॒ दुपोपा॒ धस्ता॑ द॒भ्यज्या॒ भ्यज्या॒ धस्ता॒दुप॑ । \newline
9. अ॒भ्यज्येत्य॑भि - अज्य॑ । \newline
10. अ॒धस्ता॒ दुपोपा॒ धस्ता॑ द॒धस्ता॒ दुपा॑नक् त्यन॒क् त्युपा॒ धस्ता॑ द॒धस्ता॒ दुपा॑नक्ति । \newline
11. उपा॑नक् त्यन॒क् त्युपोपा॑नक्ति सुव॒र्गस्य॑ सुव॒र्गस्या॑ न॒क्त्युपोपा॑नक्ति सुव॒र्गस्य॑ । \newline
12. अ॒न॒क्ति॒ सु॒व॒र्गस्य॑ सुव॒र्गस्या॑ नक्त्यनक्ति सुव॒र्गस्य॑ लो॒कस्य॑ लो॒कस्य॑ सुव॒र्गस्या॑ नक्त्यनक्ति सुव॒र्गस्य॑ लो॒कस्य॑ । \newline
13. सु॒व॒र्गस्य॑ लो॒कस्य॑ लो॒कस्य॑ सुव॒र्गस्य॑ सुव॒र्गस्य॑ लो॒कस्य॒ सम॑ष्ट्यै॒ सम॑ष्ट्यै लो॒कस्य॑ सुव॒र्गस्य॑ सुव॒र्गस्य॑ लो॒कस्य॒ सम॑ष्ट्यै । \newline
14. सु॒व॒र्गस्येति॑ सुवः - गस्य॑ । \newline
15. लो॒कस्य॒ सम॑ष्ट्यै॒ सम॑ष्ट्यै लो॒कस्य॑ लो॒कस्य॒ सम॑ष्ट्यै॒ सर्वा॑णि॒ सर्वा॑णि॒ सम॑ष्ट्यै लो॒कस्य॑ लो॒कस्य॒ सम॑ष्ट्यै॒ सर्वा॑णि । \newline
16. सम॑ष्ट्यै॒ सर्वा॑णि॒ सर्वा॑णि॒ सम॑ष्ट्यै॒ सम॑ष्ट्यै॒ सर्वा॑णि क॒पाला॑नि क॒पाला॑नि॒ सर्वा॑णि॒ सम॑ष्ट्यै॒ सम॑ष्ट्यै॒ सर्वा॑णि क॒पाला॑नि । \newline
17. सम॑ष्ट्या॒ इति॒ सं - अ॒ष्ट्यै॒ । \newline
18. सर्वा॑णि क॒पाला॑नि क॒पाला॑नि॒ सर्वा॑णि॒ सर्वा॑णि क॒पाला᳚ न्य॒भ्य॑भि क॒पाला॑नि॒ सर्वा॑णि॒ सर्वा॑णि क॒पाला᳚न्य॒भि । \newline
19. क॒पाला᳚ न्य॒भ्य॑भि क॒पाला॑नि क॒पाला᳚न्य॒भि प्र॑थयति प्रथयत्य॒भि क॒पाला॑नि क॒पाला᳚न्य॒भि प्र॑थयति । \newline
20. अ॒भि प्र॑थयति प्रथय त्य॒भ्य॑भि प्र॑थयति॒ ताव॑त॒ स्ताव॑तः प्रथय त्य॒भ्य॑भि प्र॑थयति॒ ताव॑तः । \newline
21. प्र॒थ॒य॒ति॒ ताव॑त॒ स्ताव॑तः प्रथयति प्रथयति॒ ताव॑तः पुरो॒डाशा᳚न् पुरो॒डाशा॒न् ताव॑तः प्रथयति प्रथयति॒ ताव॑तः पुरो॒डाशान्॑ । \newline
22. ताव॑तः पुरो॒डाशा᳚न् पुरो॒डाशा॒न् ताव॑त॒ स्ताव॑तः पुरो॒डाशा॑ न॒मुष्मि॑न् न॒मुष्मि॑न् पुरो॒डाशा॒न् ताव॑त॒ स्ताव॑तः पुरो॒डाशा॑ न॒मुष्मिन्न्॑ । \newline
23. पु॒रो॒डाशा॑ न॒मुष्मि॑न् न॒मुष्मि॑न् पुरो॒डाशा᳚न् पुरो॒डाशा॑ न॒मुष्मि॑न् ॅलो॒के लो॒के॑ ऽमुष्मि॑न् पुरो॒डाशा᳚न् पुरो॒डाशा॑ न॒मुष्मि॑न् ॅलो॒के । \newline
24. अ॒मुष्मि॑न् ॅलो॒के लो॒के॑ ऽमुष्मि॑न् न॒मुष्मि॑न् ॅलो॒के᳚(ए1॒) ऽभ्य॑भि लो॒के॑ ऽमुष्मि॑न् न॒मुष्मि॑न् ॅलो॒के॑ ऽभि । \newline
25. लो॒के᳚(ए1॒) ऽभ्य॑भि लो॒के लो॒के॑ ऽभि ज॑यति जयत्य॒भि लो॒के लो॒के॑ ऽभि ज॑यति । \newline
26. अ॒भि ज॑यति जय त्य॒भ्य॑भि ज॑यति॒ यो यो ज॑य त्य॒भ्य॑भि ज॑यति॒ यः । \newline
27. ज॒य॒ति॒ यो यो ज॑यति जयति॒ यो विद॑ग्धो॒ विद॑ग्धो॒ यो ज॑यति जयति॒ यो विद॑ग्धः । \newline
28. यो विद॑ग्धो॒ विद॑ग्धो॒ यो यो विद॑ग्धः॒ स स विद॑ग्धो॒ यो यो विद॑ग्धः॒ सः । \newline
29. विद॑ग्धः॒ स स विद॑ग्धो॒ विद॑ग्धः॒ स नैर्.॑ऋ॒तो नैर्॑.ऋ॒तः स विद॑ग्धो॒ विद॑ग्धः॒ स नैर्॑.ऋ॒तः । \newline
30. विद॑ग्ध॒ इति॒ वि - द॒ग्धः॒ । \newline
31. स नैर्॑.ऋ॒तो नैर्॑.ऋ॒तः स स नैर्॑.ऋ॒तो यो यो नैर्॑.ऋ॒तः स स नैर्॑.ऋ॒तो यः । \newline
32. नै॒र्॒.ऋ॒तो यो यो नैर्॑.ऋ॒तो नैर्॑.ऋ॒तो यो ऽशृ॒तो ऽशृ॑तो॒ यो नैर्॑.ऋ॒तो नैर्॑.ऋ॒तो यो ऽशृ॑तः । \newline
33. नै॒र्॒.ऋ॒त इति॑ नैः - ऋ॒तः । \newline
34. यो ऽशृ॒तो ऽशृ॑तो॒ यो यो ऽशृ॑तः॒ स सो ऽशृ॑तो॒ यो यो ऽशृ॑तः॒ सः । \newline
35. अशृ॑तः॒ स सो ऽशृ॒तो ऽशृ॑तः॒ स रौ॒द्रो रौ॒द्रः सो ऽशृ॒तो ऽशृ॑तः॒ स रौ॒द्रः । \newline
36. स रौ॒द्रो रौ॒द्रः स स रौ॒द्रो यो यो रौ॒द्रः स स रौ॒द्रो यः । \newline
37. रौ॒द्रो यो यो रौ॒द्रो रौ॒द्रो यः शृ॒तः शृ॒तो यो रौ॒द्रो रौ॒द्रो यः शृ॒तः । \newline
38. यः शृ॒तः शृ॒तो यो यः शृ॒तः स स शृ॒तो यो यः शृ॒तः सः । \newline
39. शृ॒तः स स शृ॒तः शृ॒तः स सदे॑वः॒ सदे॑वः॒ स शृ॒तः शृ॒तः स सदे॑वः । \newline
40. स सदे॑वः॒ सदे॑वः॒ स स सदे॑व॒ स्तस्मा॒त् तस्मा॒थ् सदे॑वः॒ स स सदे॑व॒ स्तस्मा᳚त् । \newline
41. सदे॑व॒ स्तस्मा॒त् तस्मा॒थ् सदे॑वः॒ सदे॑व॒ स्तस्मा॒ दवि॑दह॒ता ऽवि॑दहता॒ तस्मा॒थ् सदे॑वः॒ सदे॑व॒ स्तस्मा॒ दवि॑दहता । \newline
42. सदे॑व॒ इति॒ स - दे॒वः॒ । \newline
43. तस्मा॒ दवि॑दह॒ता ऽवि॑दहता॒ तस्मा॒त् तस्मा॒ दवि॑दहता शृत॒ङ्कृत्यः॑ शृत॒ङ्कृत्यो ऽवि॑दहता॒ तस्मा॒त् तस्मा॒ दवि॑दहता शृत॒ङ्कृत्यः॑ । \newline
44. अवि॑दहता शृत॒ङ्कृत्यः॑ शृत॒ङ्कृत्यो ऽवि॑दह॒ता ऽवि॑दहता शृत॒ङ्कृत्यः॑ सदेव॒त्वाय॑ सदेव॒त्वाय॑ शृत॒ङ्कृत्यो ऽवि॑दह॒ता ऽवि॑दहता शृत॒ङ्कृत्यः॑ सदेव॒त्वाय॑ । \newline
45. अवि॑दह॒तेत्यवि॑ - द॒ह॒ता॒ । \newline
46. शृ॒त॒ङ्कृत्यः॑ सदेव॒त्वाय॑ सदेव॒त्वाय॑ शृत॒ङ्कृत्यः॑ शृत॒ङ्कृत्यः॑ सदेव॒त्वाय॒ भस्म॑ना॒ भस्म॑ना सदेव॒त्वाय॑ शृत॒ङ्कृत्यः॑ शृत॒ङ्कृत्यः॑ सदेव॒त्वाय॒ भस्म॑ना । \newline
47. शृ॒त॒ङ्कृत्य॒ इति॑ शृतं - कृत्यः॑ । \newline
48. स॒दे॒व॒त्वाय॒ भस्म॑ना॒ भस्म॑ना सदेव॒त्वाय॑ सदेव॒त्वाय॒ भस्म॑ना॒ ऽभ्य॑भि भस्म॑ना सदेव॒त्वाय॑ सदेव॒त्वाय॒ भस्म॑ना॒ ऽभि । \newline
49. स॒दे॒व॒त्वायेति॑ सदेव - त्वाय॑ । \newline
50. भस्म॑ना॒ ऽभ्य॑भि भस्म॑ना॒ भस्म॑ना॒ ऽभि वा॑सयति वासयत्य॒भि भस्म॑ना॒ भस्म॑ना॒ ऽभि वा॑सयति । \newline
51. अ॒भि वा॑सयति वासय त्य॒भ्य॑भि वा॑सयति॒ तस्मा॒त् तस्मा᳚द् वासय त्य॒भ्य॑भि वा॑सयति॒ तस्मा᳚त् । \newline
52. वा॒स॒य॒ति॒ तस्मा॒त् तस्मा᳚द् वासयति वासयति॒ तस्मा᳚न् माꣳ॒॒सेन॑ माꣳ॒॒सेन॒ तस्मा᳚द् वासयति वासयति॒ तस्मा᳚न् माꣳ॒॒सेन॑ । \newline
53. तस्मा᳚न् माꣳ॒॒सेन॑ माꣳ॒॒सेन॒ तस्मा॒त् तस्मा᳚न् माꣳ॒॒सेना स्थ्य स्थि॑ माꣳ॒॒सेन॒ तस्मा॒त् तस्मा᳚न् माꣳ॒॒सेनास्थि॑ । \newline
54. माꣳ॒॒सेना स्थ्य स्थि॑ माꣳ॒॒सेन॑ माꣳ॒॒सेनास्थि॑ छ॒न्नम् छ॒न्न मस्थि॑ माꣳ॒॒सेन॑ माꣳ॒॒सेनास्थि॑ छ॒न्नम् । \newline
55. अस्थि॑ छ॒न्नम् छ॒न्न मस्थ्य स्थि॑ छ॒न्नं ॅवे॒देन॑ वे॒देन॑ छ॒न्न मस्थ्य स्थि॑ छ॒न्नं ॅवे॒देन॑ । \newline
56. छ॒न्नं ॅवे॒देन॑ वे॒देन॑ छ॒न्नम् छ॒न्नं ॅवे॒देना॒भ्य॑भि वे॒देन॑ छ॒न्नम् छ॒न्नं ॅवे॒देना॒भि । \newline
57. वे॒देना॒भ्य॑भि वे॒देन॑ वे॒देना॒भि वा॑सयति वासयत्य॒भि वे॒देन॑ वे॒देना॒भि वा॑सयति । \newline
58. अ॒भि वा॑सयति वासय त्य॒भ्य॑भि वा॑सयति॒ तस्मा॒त् तस्मा᳚द् वासय त्य॒भ्य॑भि वा॑सयति॒ तस्मा᳚त् । \newline
59. वा॒स॒य॒ति॒ तस्मा॒त् तस्मा᳚द् वासयति वासयति॒ तस्मा॒त् केशैः॒ केशै॒ स्तस्मा᳚द् वासयति वासयति॒ तस्मा॒त् केशैः᳚ । \newline
60. तस्मा॒त् केशैः॒ केशै॒ स्तस्मा॒त् तस्मा॒त् केशैः॒ शिरः॒ शिरः॒ केशै॒ स्तस्मा॒त् तस्मा॒त् केशैः॒ शिरः॑ । \newline
\pagebreak
\markright{ TS 2.6.3.5  \hfill https://www.vedavms.in \hfill}
\addcontentsline{toc}{section}{ TS 2.6.3.5 }
\section*{ TS 2.6.3.5 }

\textbf{TS 2.6.3.5 } \newline
\textbf{Samhita Paata} \newline

केशैः॒ शिरः॑ छ॒न्नं प्रच्यु॑तं॒ ॅवा ए॒तद॒स्माल् लो॒कादग॑तं देवलो॒कं ॅयच्छृ॒तꣳ ह॒विरन॑भिघारित-मभि॒घार्योद्-वा॑सयति देव॒त्रैवैन॑द्-गमयति॒ यद्येकं॑ क॒पालं॒ नश्ये॒देको॒ मासः॑ संॅवथ्स॒रस्यान॑वेतः॒ स्यादथ॒ यज॑मानः॒ प्रमी॑येत॒ यद् द्वे नश्ये॑तां॒ द्वौ मासौ॑ संॅवथ्स॒रस्यान॑वेतौ॒ स्याता॒मथ॒ यज॑मानः॒ प्रमी॑येत स॒ङ्ख्यायोद्-वा॑सयति॒ यज॑मानस्य - [  ] \newline

\textbf{Pada Paata} \newline

केशैः᳚ । शिरः॑ । छ॒न्नम् । प्रच्यु॑त॒मिति॒ प्र - च्यु॒त॒म् । वै । ए॒तत् । अ॒स्मात् । लो॒कात् । अग॑तम् । दे॒व॒लो॒कमिति॑ देव - लो॒कम् । यत् । शृ॒तम् । ह॒विः । अन॑भिघारित॒मित्यन॑भि - घा॒रि॒त॒म् । अ॒भि॒घार्येत्य॑भि - घार्य॑ । उदिति॑ । वा॒स॒य॒ति॒ । दे॒व॒त्रेति॑ देव - त्रा । ए॒व । ए॒न॒त् । ग॒म॒य॒ति॒ । यदि॑ । एक᳚म् । क॒पाल᳚म् । नश्ये᳚त् । एकः॑ । मासः॑ । सं॒ॅव॒थ्स॒रस्येति॑ सं - व॒थ्स॒रस्य॑ । अन॑वेत॒ इत्यन॑व - इ॒तः॒ । स्यात् । अथ॑ । यज॑मानः । प्रेति॑ । मी॒ये॒त॒ । यत् । द्वे इति॑ । नश्ये॑ताम् । द्वौ । मासौ᳚ । सं॒ॅव॒थ्स॒रस्येति॑ सं - व॒थ्स॒रस्य॑ । अन॑वेता॒वित्यन॑व - इ॒तौ॒ । स्याता᳚म् । अथ॑ । यज॑मानः । प्रेति॑ । मी॒ये॒त॒ । स॒ख्यांयेति॑ सं - ख्याय॑ । उदिति॑ । वा॒स॒य॒ति॒ । यज॑मानस्य ।  \newline


\textbf{Krama Paata} \newline

केशैः॒ शिरः॑ । शिर॑ श्छ॒न्नम् । छ॒न्नम् प्रच्यु॑तम् । प्रच्यु॑तं॒ ॅवै । प्रच्यु॑त॒मिति॒ प्र - च्यु॒त॒म् । वा ए॒तत् । ए॒तद॒स्मात् । अ॒स्माल्लो॒कात् । लो॒कादग॑तम् । अग॑तम् देवलो॒कम् । दे॒व॒लो॒कं ॅयत् । दे॒व॒लो॒कमिति॑ देव - लो॒कम् । यच्छृ॒तम् । शृ॒तꣳ ह॒विः । ह॒विरन॑भिघारितम् । अन॑भिघारितमभि॒घार्य॑ । अन॑भिघारित॒मित्यन॑भि - घा॒रि॒त॒म् । अ॒भि॒घार्योत् । अ॒भि॒घार्येत्य॑भि - घार्य॑ । 
उद् वा॑सयति । वा॒स॒य॒ति॒ दे॒व॒त्रा । दे॒व॒त्रैव । दे॒व॒त्रेति॑ देव - त्रा । ए॒वैन॑त् । 
ए॒न॒द् ग॒म॒य॒ति॒ । ग॒म॒य॒ति॒ यदि॑ । यद्येक᳚म् । एक॑म् क॒पाल᳚म् । क॒पाल॒म् नश्ये᳚त् । नश्ये॒देकः॑ । एको॒ मासः॑ । मासः॑ सम्ॅवथ्स॒रस्य॑ । सं॒ॅव॒थ्स॒रस्यान॑वेतः । स॒म्ॅव॒थ्स॒रस्येति॑ सं - व॒थ्स॒रस्य॑ । अन॑वेतः॒ स्यात् । अन॑वेत॒ इत्यन॑व - इ॒तः॒ । स्यादथ॑ । अथ॒ यज॑मानः । यज॑मानः॒ प्र । प्र मी॑येत । मी॒ये॒त॒ यत् । यद् द्वे । द्वे नश्ये॑ताम् । द्वे इति॒ द्वे । नश्ये॑तां॒ द्वौ । द्वौ मासौ᳚ । मासौ॑ सम्ॅवथ्स॒रस्य॑ । स॒म्ॅव॒थ्स॒रस्यान॑वेतौ । स॒म्ॅव॒थ्स॒रस्येति॑ सं - व॒थ्स॒रस्य॑ । अन॑वेतौ॒ स्याता᳚म् । अन॑वेता॒वित्यन॑व - इ॒तौ॒ । स्याता॒मथ॑ । अथ॒ यज॑मानः । यज॑मानः॒ प्र । प्र मी॑येत । मि॒ये॒त॒ स॒ङ्ख्याय॑ । स॒ङ्ख्यायोत् । स॒ङ्ख्यायेति॑ सं - ख्याय॑ । उद् वा॑सयति । वा॒स॒य॒ति॒ यज॑मानस्य ( ) । यज॑मानस्य गोपी॒थाय॑ \newline

\textbf{Jatai Paata} \newline

1. केशैः॒ शिरः॒ शिरः॒ केशैः॒ केशैः॒ शिरः॑ । \newline
2. शिर॑ श्छ॒न्नम् छ॒न्नꣳ शिरः॒ शिर॑ श्छ॒न्नम् । \newline
3. छ॒न्नम् प्रच्यु॑त॒म् प्रच्यु॑तम् छ॒न्नम् छ॒न्नम् प्रच्यु॑तम् । \newline
4. प्रच्यु॑तं॒ ॅवै वै प्रच्यु॑त॒म् प्रच्यु॑तं॒ ॅवै । \newline
5. प्रच्यु॑त॒मिति॒ प्र - च्यु॒त॒म् । \newline
6. वा ए॒त दे॒तद् वै वा ए॒तत् । \newline
7. ए॒तद॒स्मा द॒स्मा दे॒त दे॒त द॒स्मात् । \newline
8. अ॒स्मा ल्लो॒का ल्लो॒का द॒स्मा द॒स्मा ल्लो॒कात् । \newline
9. लो॒का दग॑त॒ मग॑तम् ॅलो॒का ल्लो॒का दग॑तम् । \newline
10. अग॑तम् देवलो॒कम् दे॑वलो॒क मग॑त॒ मग॑तम् देवलो॒कम् । \newline
11. दे॒व॒लो॒कं ॅयद् यद् दे॑वलो॒कम् दे॑वलो॒कं ॅयत् । \newline
12. दे॒व॒लो॒कमिति॑ देव - लो॒कम् । \newline
13. यच् छृ॒तꣳ शृ॒तं ॅयद् यच्छृ॒तम् । \newline
14. शृ॒तꣳ ह॒विर्. ह॒विः शृ॒तꣳ शृ॒तꣳ ह॒विः । \newline
15. ह॒वि रन॑भिघारित॒ मन॑भिघारितꣳ ह॒विर्. ह॒वि रन॑भिघारितम् । \newline
16. अन॑भिघारित मभि॒घार्या॑ भि॒घार्या न॑भिघारित॒ मन॑भिघारित मभि॒घार्य॑ । \newline
17. अन॑भिघारित॒मित्यन॑भि - घा॒रि॒त॒म् । \newline
18. अ॒भि॒घार्यो दुद॑भि॒घार्या॑ भि॒घार्योत् । \newline
19. अ॒भि॒घार्येत्य॑भि - घार्य॑ । \newline
20. उद् वा॑सयति वासय॒ त्युदुद् वा॑सयति । \newline
21. वा॒स॒य॒ति॒ दे॒व॒त्रा दे॑व॒त्रा वा॑सयति वासयति देव॒त्रा । \newline
22. दे॒व॒ त्रैवैव दे॑व॒त्रा दे॑व॒ त्रैव । \newline
23. दे॒व॒त्रेति॑ देव - त्रा । \newline
24. ए॒वैन॑ देन दे॒वैवैन॑त् । \newline
25. ए॒न॒द् ग॒म॒य॒ति॒ ग॒म॒य॒ त्ये॒न॒ दे॒न॒द् ग॒म॒य॒ति॒ । \newline
26. ग॒म॒य॒ति॒ यदि॒ यदि॑ गमयति गमयति॒ यदि॑ । \newline
27. यद्येक॒ मेकं॒ ॅयदि॒ यद्येक᳚म् । \newline
28. एक॑म् क॒पाल॑म् क॒पाल॒ मेक॒ मेक॑म् क॒पाल᳚म् । \newline
29. क॒पाल॒म् नश्ये॒न् नश्ये᳚त् क॒पाल॑म् क॒पाल॒म् नश्ये᳚त् । \newline
30. नश्ये॒ देक॒ एको॒ नश्ये॒न् नश्ये॒ देकः॑ । \newline
31. एको॒ मासो॒ मास॒ एक॒ एको॒ मासः॑ । \newline
32. मासः॑ संॅवथ्स॒रस्य॑ संॅवथ्स॒रस्य॒ मासो॒ मासः॑ संॅवथ्स॒रस्य॑ । \newline
33. सं॒ॅव॒थ्स॒रस्या न॑वे॒तो ऽन॑वेतः संॅवथ्स॒रस्य॑ संॅवथ्स॒रस्या न॑वेतः । \newline
34. सं॒ॅव॒थ्स॒रस्येति॑ सं - व॒थ्स॒रस्य॑ । \newline
35. अन॑वेतः॒ स्याथ् स्या दन॑वे॒तो ऽन॑वेतः॒ स्यात् । \newline
36. अन॑वेत॒ इत्यन॑व - इ॒तः॒ । \newline
37. स्या दथाथ॒ स्याथ् स्या दथ॑ । \newline
38. अथ॒ यज॑मानो॒ यज॑मा॒नो ऽथाथ॒ यज॑मानः । \newline
39. यज॑मानः॒ प्र प्र यज॑मानो॒ यज॑मानः॒ प्र । \newline
40. प्र मी॑येत मीयेत॒ प्र प्र मी॑येत । \newline
41. मी॒ये॒त॒ यद् यन् मी॑येत मीयेत॒ यत् । \newline
42. यद् द्वे द्वे यद् यद् द्वे । \newline
43. द्वे नश्ये॑ता॒म् नश्ये॑ता॒म् द्वे द्वे नश्ये॑ताम् । \newline
44. द्वे इति॒ द्वे । \newline
45. नश्ये॑ता॒म् द्वौ द्वौ नश्ये॑ता॒म् नश्ये॑ता॒म् द्वौ । \newline
46. द्वौ मासौ॒ मासौ॒ द्वौ द्वौ मासौ᳚ । \newline
47. मासौ॑ संॅवथ्स॒रस्य॑ संॅवथ्स॒रस्य॒ मासौ॒ मासौ॑ संॅवथ्स॒रस्य॑ । \newline
48. सं॒ॅव॒थ्स॒रस्या न॑वेता॒ वन॑वेतौ संॅवथ्स॒रस्य॑ संॅवथ्स॒रस्या न॑वेतौ । \newline
49. सं॒ॅव॒थ्स॒रस्येति॑ सं - व॒थ्स॒रस्य॑ । \newline
50. अन॑वेतौ॒ स्याताꣳ॒॒ स्याता॒ मन॑वेता॒ वन॑वेतौ॒ स्याता᳚म् । \newline
51. अन॑वेता॒वित्यन॑व - इ॒तौ॒ । \newline
52. स्याता॒ मथाथ॒ स्याताꣳ॒॒ स्याता॒ मथ॑ । \newline
53. अथ॒ यज॑मानो॒ यज॑मा॒नो ऽथाथ॒ यज॑मानः । \newline
54. यज॑मानः॒ प्र प्र यज॑मानो॒ यज॑मानः॒ प्र । \newline
55. प्र मी॑येत मीयेत॒ प्र प्र मी॑येत । \newline
56. मी॒ये॒त॒ स॒ङ्ख्याय॑ स॒ङ्ख्याय॑ मीयेत मीयेत स॒ङ्ख्याय॑ । \newline
57. स॒ङ्ख्यायो दुथ् स॒ङ्ख्याय॑ स॒ङ्ख्यायोत् । \newline
58. स॒ङ्ख्यायेति॑ सं - ख्याय॑ । \newline
59. उद् वा॑सयति वासय॒ त्युदुद् वा॑सयति । \newline
60. वा॒स॒य॒ति॒ यज॑मानस्य॒ यज॑मानस्य वासयति वासयति॒ यज॑मानस्य । \newline
61. यज॑मानस्य गोपी॒थाय॑ गोपी॒थाय॒ यज॑मानस्य॒ यज॑मानस्य गोपी॒थाय॑ । \newline

\textbf{Ghana Paata } \newline

1. केशैः॒ शिरः॒ शिरः॒ केशैः॒ केशैः॒ शिर॑ श्छ॒न्नम् छ॒न्नꣳ शिरः॒ केशैः॒ केशैः॒ शिर॑ श्छ॒न्नम् । \newline
2. शिर॑ श्छ॒न्नम् छ॒न्नꣳ शिरः॒ शिर॑ श्छ॒न्नम् प्रच्यु॑त॒म् प्रच्यु॑तम् छ॒न्नꣳ शिरः॒ शिर॑ श्छ॒न्नम् प्रच्यु॑तम् । \newline
3. छ॒न्नम् प्रच्यु॑त॒म् प्रच्यु॑तम् छ॒न्नम् छ॒न्नम् प्रच्यु॑तं॒ ॅवै वै प्रच्यु॑तम् छ॒न्नम् छ॒न्नम् प्रच्यु॑तं॒ ॅवै । \newline
4. प्रच्यु॑तं॒ ॅवै वै प्रच्यु॑त॒म् प्रच्यु॑तं॒ ॅवा ए॒त दे॒तद् वै प्रच्यु॑त॒म् प्रच्यु॑तं॒ ॅवा ए॒तत् । \newline
5. प्रच्यु॑त॒मिति॒ प्र - च्यु॒त॒म् । \newline
6. वा ए॒त दे॒तद् वै वा ए॒तद॒स्मा द॒स्मा दे॒तद् वै वा ए॒त द॒स्मात् । \newline
7. ए॒तद॒स्मा द॒स्मा दे॒त दे॒त द॒स्मा ल्लो॒का ल्लो॒का द॒स्मा दे॒त दे॒त द॒स्मा ल्लो॒कात् । \newline
8. अ॒स्मा ल्लो॒का ल्लो॒का द॒स्मा द॒स्मा ल्लो॒का दग॑त॒ मग॑तम् ॅलो॒का द॒स्मा द॒स्मा ल्लो॒का दग॑तम् । \newline
9. लो॒का दग॑त॒ मग॑तम् ॅलो॒का ल्लो॒का दग॑तम् देवलो॒कम् दे॑वलो॒क मग॑तम् ॅलो॒का ल्लो॒का दग॑तम् देवलो॒कम् । \newline
10. अग॑तम् देवलो॒कम् दे॑वलो॒क मग॑त॒ मग॑तम् देवलो॒कं ॅयद् यद् दे॑वलो॒क मग॑त॒ मग॑तम् देवलो॒कं ॅयत् । \newline
11. दे॒व॒लो॒कं ॅयद् यद् दे॑वलो॒कम् दे॑वलो॒कं ॅयच्छृ॒तꣳ शृ॒तं ॅयद् दे॑वलो॒कम् दे॑वलो॒कं ॅयच्छृ॒तम् । \newline
12. दे॒व॒लो॒कमिति॑ देव - लो॒कम् । \newline
13. यच्छृ॒तꣳ शृ॒तं ॅयद् यच्छृ॒तꣳ ह॒विर्. ह॒विः शृ॒तं ॅयद् यच्छृ॒तꣳ ह॒विः । \newline
14. शृ॒तꣳ ह॒विर्. ह॒विः शृ॒तꣳ शृ॒तꣳ ह॒वि रन॑भिघारित॒ मन॑भिघारितꣳ ह॒विः शृ॒तꣳ शृ॒तꣳ ह॒वि रन॑भिघारितम् । \newline
15. ह॒वि रन॑भिघारित॒ मन॑भिघारितꣳ ह॒विर्. ह॒वि रन॑भिघारित मभि॒घार्या॑ भि॒घार्या न॑भिघारितꣳ ह॒विर्. ह॒वि रन॑भिघारित मभि॒घार्य॑ । \newline
16. अन॑भिघारित मभि॒घार्या॑ भि॒घार्या न॑भिघारित॒ मन॑भिघारित मभि॒घार्यो दुद॑भि॒घार्या न॑भिघारित॒ मन॑भिघारित मभि॒घार्योत् । \newline
17. अन॑भिघारित॒मित्यन॑भि - घा॒रि॒त॒म् । \newline
18. अ॒भि॒घार्यो दुद॑भि॒घार्या॑ भि॒घार्योद् वा॑सयति वासय॒ त्युद॑भि॒घार्या॑ भि॒घार्योद् वा॑सयति । \newline
19. अ॒भि॒घार्येत्य॑भि - घार्य॑ । \newline
20. उद् वा॑सयति वासय॒ त्युदुद् वा॑सयति देव॒त्रा दे॑व॒त्रा वा॑सय॒ त्युदुद् वा॑सयति देव॒त्रा । \newline
21. वा॒स॒य॒ति॒ दे॒व॒त्रा दे॑व॒त्रा वा॑सयति वासयति देव॒त्रैवैव दे॑व॒त्रा वा॑सयति वासयति देव॒त्रैव । \newline
22. दे॒व॒त्रैवैव दे॑व॒त्रा दे॑व॒ त्रैवैन॑ देनदे॒व दे॑व॒त्रा दे॑व॒त्रैवैन॑त् । \newline
23. दे॒व॒त्रेति॑ देव - त्रा । \newline
24. ए॒वैन॑ देन दे॒वैवैन॑द् गमयति गमय त्येन दे॒वैवैन॑द् गमयति । \newline
25. ए॒न॒द् ग॒म॒य॒ति॒ ग॒म॒य॒ त्ये॒न॒दे॒न॒द् ग॒म॒य॒ति॒ यदि॒ यदि॑ गमय त्येनदेनद् गमयति॒ यदि॑ । \newline
26. ग॒म॒य॒ति॒ यदि॒ यदि॑ गमयति गमयति॒ यद्येक॒ मेकं॒ ॅयदि॑ गमयति गमयति॒ यद्येक᳚म् । \newline
27. यद्येक॒ मेकं॒ ॅयदि॒ यद्येक॑म् क॒पाल॑म् क॒पाल॒ मेकं॒ ॅयदि॒ यद्येक॑म् क॒पाल᳚म् । \newline
28. एक॑म् क॒पाल॑म् क॒पाल॒ मेक॒ मेक॑म् क॒पाल॒म् नश्ये॒न् नश्ये᳚त् क॒पाल॒ मेक॒ मेक॑म् क॒पाल॒म् नश्ये᳚त् । \newline
29. क॒पाल॒म् नश्ये॒न् नश्ये᳚त् क॒पाल॑म् क॒पाल॒म् नश्ये॒देक॒ एको॒ नश्ये᳚त् क॒पाल॑म् क॒पाल॒म् नश्ये॒देकः॑ । \newline
30. नश्ये॒देक॒ एको॒ नश्ये॒न् नश्ये॒ देको॒ मासो॒ मास॒ एको॒ नश्ये॒न् नश्ये॒ देको॒ मासः॑ । \newline
31. एको॒ मासो॒ मास॒ एक॒ एको॒ मासः॑ संॅवथ्स॒रस्य॑ संॅवथ्स॒रस्य॒ मास॒ एक॒ एको॒ मासः॑ संॅवथ्स॒रस्य॑ । \newline
32. मासः॑ संॅवथ्स॒रस्य॑ संॅवथ्स॒रस्य॒ मासो॒ मासः॑ संॅवथ्स॒रस्या न॑वे॒तो ऽन॑वेतः संॅवथ्स॒रस्य॒ मासो॒ मासः॑ संॅवथ्स॒रस्या न॑वेतः । \newline
33. सं॒ॅव॒थ्स॒रस्या न॑वे॒तो ऽन॑वेतः संॅवथ्स॒रस्य॑ संॅवथ्स॒रस्या न॑वेतः॒ स्याथ् स्या दन॑वेतः संॅवथ्स॒रस्य॑ संॅवथ्स॒रस्या न॑वेतः॒ स्यात् । \newline
34. सं॒ॅव॒थ्स॒रस्येति॑ सं - व॒थ्स॒रस्य॑ । \newline
35. अन॑वेतः॒ स्याथ् स्या दन॑वे॒तो ऽन॑वेतः॒ स्यादथाथ॒ स्या दन॑वे॒तो ऽन॑वेतः॒ स्यादथ॑ । \newline
36. अन॑वेत॒ इत्यन॑व - इ॒तः॒ । \newline
37. स्यादथाथ॒ स्याथ् स्यादथ॒ यज॑मानो॒ यज॑मा॒नो ऽथ॒ स्याथ् स्यादथ॒ यज॑मानः । \newline
38. अथ॒ यज॑मानो॒ यज॑मा॒नो ऽथाथ॒ यज॑मानः॒ प्र प्र यज॑मा॒नो ऽथाथ॒ यज॑मानः॒ प्र । \newline
39. यज॑मानः॒ प्र प्र यज॑मानो॒ यज॑मानः॒ प्र मी॑येत मीयेत॒ प्र यज॑मानो॒ यज॑मानः॒ प्र मी॑येत । \newline
40. प्र मी॑येत मीयेत॒ प्र प्र मी॑येत॒ यद् यन् मी॑येत॒ प्र प्र मी॑येत॒ यत् । \newline
41. मी॒ये॒त॒ यद् यन् मी॑येत मीयेत॒ यद् द्वे द्वे यन् मी॑येत मीयेत॒ यद् द्वे । \newline
42. यद् द्वे द्वे यद् यद् द्वे नश्ये॑ता॒म् नश्ये॑ता॒म् द्वे यद् यद् द्वे नश्ये॑ताम् । \newline
43. द्वे नश्ये॑ता॒म् नश्ये॑ता॒म् द्वे द्वे नश्ये॑ता॒म् द्वौ द्वौ नश्ये॑ता॒म् द्वे द्वे नश्ये॑ता॒म् द्वौ । \newline
44. द्वे इति॒ द्वे । \newline
45. नश्ये॑ता॒म् द्वौ द्वौ नश्ये॑ता॒म् नश्ये॑ता॒म् द्वौ मासौ॒ मासौ॒ द्वौ नश्ये॑ता॒म् नश्ये॑ता॒म् द्वौ मासौ᳚ । \newline
46. द्वौ मासौ॒ मासौ॒ द्वौ द्वौ मासौ॑ संॅवथ्स॒रस्य॑ संॅवथ्स॒रस्य॒ मासौ॒ द्वौ द्वौ मासौ॑ संॅवथ्स॒रस्य॑ । \newline
47. मासौ॑ संॅवथ्स॒रस्य॑ संॅवथ्स॒रस्य॒ मासौ॒ मासौ॑ संॅवथ्स॒रस्या न॑वेता॒ वन॑वेतौ संॅवथ्स॒रस्य॒ मासौ॒ मासौ॑ संॅवथ्स॒रस्या न॑वेतौ । \newline
48. सं॒ॅव॒थ्स॒रस्या न॑वेता॒ वन॑वेतौ संॅवथ्स॒रस्य॑ संॅवथ्स॒रस्या न॑वेतौ॒ स्याताꣳ॒॒ स्याता॒ मन॑वेतौ संॅवथ्स॒रस्य॑ संॅवथ्स॒रस्या न॑वेतौ॒ स्याता᳚म् । \newline
49. सं॒ॅव॒थ्स॒रस्येति॑ सं - व॒थ्स॒रस्य॑ । \newline
50. अन॑वेतौ॒ स्याताꣳ॒॒ स्याता॒ मन॑वेता॒ वन॑वेतौ॒ स्याता॒ मथाथ॒ स्याता॒ मन॑वेता॒ वन॑वेतौ॒ स्याता॒ मथ॑ । \newline
51. अन॑वेता॒वित्यन॑व - इ॒तौ॒ । \newline
52. स्याता॒ मथाथ॒ स्याताꣳ॒॒ स्याता॒ मथ॒ यज॑मानो॒ यज॑मा॒नो ऽथ॒ स्याताꣳ॒॒ स्याता॒ मथ॒ यज॑मानः । \newline
53. अथ॒ यज॑मानो॒ यज॑मा॒नो ऽथाथ॒ यज॑मानः॒ प्र प्र यज॑मा॒नो ऽथाथ॒ यज॑मानः॒ प्र । \newline
54. यज॑मानः॒ प्र प्र यज॑मानो॒ यज॑मानः॒ प्र मी॑येत मीयेत॒ प्र यज॑मानो॒ यज॑मानः॒ प्र मी॑येत । \newline
55. प्र मी॑येत मीयेत॒ प्र प्र मी॑येत स॒ङ्ख्याय॑ स॒ङ्ख्याय॑ मीयेत॒ प्र प्र मी॑येत स॒ङ्ख्याय॑ । \newline
56. मी॒ये॒त॒ स॒ङ्ख्याय॑ स॒ङ्ख्याय॑ मीयेत मीयेत स॒ङ्ख्यायोदुथ् स॒ङ्ख्याय॑ मीयेत मीयेत स॒ङ्ख्यायोत् । \newline
57. स॒ङ्ख्यायोदुथ् स॒ङ्ख्याय॑ स॒ङ्ख्यायोद् वा॑सयति वासय॒त्युथ् स॒ङ्ख्याय॑ स॒ङ्ख्यायोद् वा॑सयति । \newline
58. स॒ङ्ख्यायेति॑ सं - ख्याय॑ । \newline
59. उद् वा॑सयति वासय॒ त्युदुद् वा॑सयति॒ यज॑मानस्य॒ यज॑मानस्य वासय॒ त्युदुद् वा॑सयति॒ यज॑मानस्य । \newline
60. वा॒स॒य॒ति॒ यज॑मानस्य॒ यज॑मानस्य वासयति वासयति॒ यज॑मानस्य गोपी॒थाय॑ गोपी॒थाय॒ यज॑मानस्य वासयति वासयति॒ यज॑मानस्य गोपी॒थाय॑ । \newline
61. यज॑मानस्य गोपी॒थाय॑ गोपी॒थाय॒ यज॑मानस्य॒ यज॑मानस्य गोपी॒थाय॒ यदि॒ यदि॑ गोपी॒थाय॒ यज॑मानस्य॒ यज॑मानस्य गोपी॒थाय॒ यदि॑ । \newline
\pagebreak
\markright{ TS 2.6.3.6  \hfill https://www.vedavms.in \hfill}
\addcontentsline{toc}{section}{ TS 2.6.3.6 }
\section*{ TS 2.6.3.6 }

\textbf{TS 2.6.3.6 } \newline
\textbf{Samhita Paata} \newline

गोपी॒थाय॒ यदि॒ नश्ये॑दाश्वि॒नं द्वि॑कपा॒लं निर्व॑पेद् द्यावापृथि॒व्य॑- मेक॑कपालम॒श्विनौ॒ वै दे॒वानां᳚ भि॒षजौ॒ ताभ्या॑मे॒वास्मै॑ भेष॒जं क॑रोति द्यावापृथि॒व्य॑ एक॑कपालो भवत्य॒नयो॒र्वा ए॒तन्न॑श्यति॒ यन्नश्य॑- त्य॒नयो॑रे॒वैन॑द्-विन्दति॒ प्रति॑ष्ठित्यै ॥ \newline

\textbf{Pada Paata} \newline

गो॒पी॒थाय॑ । यदि॑ । नश्ये᳚त् । आ॒श्वि॒नम् । द्वि॒क॒पा॒लमिति॑ द्वि - क॒पा॒लम् । निरिति॑ । व॒पे॒त् । द्या॒वा॒पृ॒थि॒व्य॑मिति॑ द्यावा - पृ॒थि॒व्य᳚म् । एक॑कपाल॒मित्येक॑-क॒पा॒ल॒म् । अ॒श्विनौ᳚ । वै । दे॒वाना᳚म् । भि॒षजौ᳚ । ताभ्या᳚म् । ए॒व । अ॒स्मै॒ । भे॒ष॒जम् । क॒रो॒ति॒ । द्या॒वा॒पृ॒थि॒व्य॑ इति॑ द्यावा - पृ॒थि॒व्यः॑ । एक॑कपाल॒ इत्येक॑-क॒पा॒लः॒ । भ॒व॒ति॒ । अ॒नयोः᳚ । वै । ए॒तत् । न॒श्य॒ति॒ । यत् । नश्य॑ति । अ॒नयोः᳚ । ए॒व । ए॒न॒त् । वि॒न्द॒ति॒ । प्रति॑ष्ठित्या॒ इति॒ प्रति॑ - स्थि॒त्यै॒ ॥  \newline


\textbf{Krama Paata} \newline

गो॒पी॒थाय॒ यदि॑ । यदि॒ नश्ये᳚त् । नश्ये॑दाश्वि॒नम् । आ॒श्वि॒नम् द्वि॑कपा॒लम् । द्वि॒क॒पा॒लम् निः । द्वि॒क॒पा॒लमिति॑ द्वि - क॒पा॒लम् । निर् व॑पेत् । व॒पे॒द् द्या॒वा॒पृ॒थि॒व्य᳚म् । द्या॒वा॒पृ॒थि॒व्य॑मेक॑पालम् । द्या॒वा॒पृ॒थि॒व्य॑मिति॑ द्यावा - पृ॒थि॒व्य᳚म् । एक॑कपालम॒श्विनौ᳚ । एक॑कपाल॒मित्येक॑ - क॒पा॒ल॒म् । अ॒श्विनौ॒ वै । वै दे॒वाना᳚म् । दे॒वाना᳚म् भि॒षजौ᳚ । भि॒षजौ॒ ताभ्या᳚म् । ताभ्या॑मे॒व । ए॒वास्मै᳚ । अ॒स्मै॒ भे॒ष॒जम् । भे॒ष॒जम् क॑रोति । क॒रो॒ति॒ द्या॒व्या॒पृ॒थि॒व्यः॑ । द्या॒वा॒पृ॒थि॒व्य॑ एक॑कपालः । द्या॒वा॒पृ॒थि॒व्य॑ इति॑ द्यावा - पृ॒थि॒व्यः॑ । एक॑कपालो भवति । एक॑कपाल॒ इत्येक॑ - क॒पा॒लः॒ । भ॒व॒त्य॒नयोः᳚ । अ॒नयो॒र् वै । वा ए॒तत् । ए॒तन् न॑श्यति । न॒श्य॒ति॒ यत् । यन् नश्य॑ति । नश्य॑त्य॒नयोः᳚ । अ॒नयो॑रे॒व । ए॒वैन॑त् । ए॒न॒द् वि॒न्द॒ति॒ । वि॒न्द॒ति॒ प्रति॑ष्ठित्यै । प्रति॑ष्ठित्या॒ इति॒ प्रति॑ - स्थि॒त्यै॒ । \newline

\textbf{Jatai Paata} \newline

1. गो॒पी॒थाय॒ यदि॒ यदि॑ गोपी॒थाय॑ गोपी॒थाय॒ यदि॑ । \newline
2. यदि॒ नश्ये॒न् नश्ये॒द् यदि॒ यदि॒ नश्ये᳚त् । \newline
3. नश्ये॑दाश्वि॒न मा᳚श्वि॒नम् नश्ये॒न् नश्ये॑ दाश्वि॒नम् । \newline
4. आ॒श्वि॒नम् द्वि॑कपा॒लम् द्वि॑कपा॒ल मा᳚श्वि॒न मा᳚श्वि॒नम् द्वि॑कपा॒लम् । \newline
5. द्वि॒क॒पा॒लम् निर् णिर् द्वि॑कपा॒लम् द्वि॑कपा॒लम् निः । \newline
6. द्वि॒क॒पा॒लमिति॑ द्वि - क॒पा॒लम् । \newline
7. निर् व॑पेद् वपे॒न् निर् णिर् व॑पेत् । \newline
8. व॒पे॒द् द्या॒वा॒पृ॒थि॒व्य॑म् द्यावापृथि॒व्यं॑ ॅवपेद् वपेद् द्यावापृथि॒व्य᳚म् । \newline
9. द्या॒वा॒पृ॒थि॒व्य॑ मेक॑कपाल॒ मेक॑कपालम् द्यावापृथि॒व्य॑म् द्यावापृथि॒व्य॑ मेक॑कपालम् । \newline
10. द्या॒वा॒पृ॒थि॒व्य॑मिति॑ द्यावा - पृ॒थि॒व्य᳚म् । \newline
11. एक॑कपाल म॒श्विना॑ व॒श्विना॒ वेक॑कपाल॒ मेक॑कपाल म॒श्विनौ᳚ । \newline
12. एक॑कपाल॒मित्येक॑ - क॒पा॒ल॒म् । \newline
13. अ॒श्विनौ॒ वै वा अ॒श्विना॑ व॒श्विनौ॒ वै । \newline
14. वै दे॒वाना᳚म् दे॒वानां॒ ॅवै वै दे॒वाना᳚म् । \newline
15. दे॒वाना᳚म् भि॒षजौ॑ भि॒षजौ॑ दे॒वाना᳚म् दे॒वाना᳚म् भि॒षजौ᳚ । \newline
16. भि॒षजौ॒ ताभ्या॒म् ताभ्या᳚म् भि॒षजौ॑ भि॒षजौ॒ ताभ्या᳚म् । \newline
17. ताभ्या॑ मे॒वैव ताभ्या॒म् ताभ्या॑ मे॒व । \newline
18. ए॒वास्मा॑ अस्मा ए॒वैवास्मै᳚ । \newline
19. अ॒स्मै॒ भे॒ष॒जम् भे॑ष॒ज म॑स्मा अस्मै भेष॒जम् । \newline
20. भे॒ष॒जम् क॑रोति करोति भेष॒जम् भे॑ष॒जम् क॑रोति । \newline
21. क॒रो॒ति॒ द्या॒वा॒पृ॒थि॒व्यो᳚ द्यावापृथि॒व्यः॑ करोति करोति द्यावापृथि॒व्यः॑ । \newline
22. द्या॒वा॒पृ॒थि॒व्य॑ एक॑कपाल॒ एक॑कपालो द्यावापृथि॒व्यो᳚ द्यावापृथि॒व्य॑ एक॑कपालः । \newline
23. द्या॒वा॒पृ॒थि॒व्य॑ इति॑ द्यावा - पृ॒थि॒व्यः॑ । \newline
24. एक॑कपालो भवति भव॒ त्येक॑कपाल॒ एक॑कपालो भवति । \newline
25. एक॑कपाल॒ इत्येक॑ - क॒पा॒लः॒ । \newline
26. भ॒व॒ त्य॒नयो॑ र॒नयो᳚र् भवति भव त्य॒नयोः᳚ । \newline
27. अ॒नयो॒र् वै वा अ॒नयो॑ र॒नयो॒र् वै । \newline
28. वा ए॒त दे॒तद् वै वा ए॒तत् । \newline
29. ए॒तन् न॑श्यति नश्य त्ये॒त दे॒तन् न॑श्यति । \newline
30. न॒श्य॒ति॒ यद् यन् न॑श्यति नश्यति॒ यत् । \newline
31. यन् नश्य॑ति॒ नश्य॑ति॒ यद् यन् नश्य॑ति । \newline
32. नश्य॑ त्य॒नयो॑ र॒नयो॒र् नश्य॑ति॒ नश्य॑ त्य॒नयोः᳚ । \newline
33. अ॒नयो॑ रे॒वैवा नयो॑ र॒नयो॑ रे॒व । \newline
34. ए॒वैन॑ देन दे॒वैवैन॑त् । \newline
35. ए॒न॒द् वि॒न्द॒ति॒ वि॒न्द॒ त्ये॒न॒ दे॒न॒द् वि॒न्द॒ति॒ । \newline
36. वि॒न्द॒ति॒ प्रति॑ष्ठित्यै॒ प्रति॑ष्ठित्यै विन्दति विन्दति॒ प्रति॑ष्ठित्यै । \newline
37. प्रति॑ष्ठित्या॒ इति॒ प्रति॑ - स्थि॒त्यै॒ । \newline

\textbf{Ghana Paata } \newline

1. गो॒पी॒थाय॒ यदि॒ यदि॑ गोपी॒थाय॑ गोपी॒थाय॒ यदि॒ नश्ये॒न् नश्ये॒द् यदि॑ गोपी॒थाय॑ गोपी॒थाय॒ यदि॒ नश्ये᳚त् । \newline
2. यदि॒ नश्ये॒न् नश्ये॒द् यदि॒ यदि॒ नश्ये॑ दाश्वि॒न मा᳚श्वि॒नम् नश्ये॒द् यदि॒ यदि॒ नश्ये॑ दाश्वि॒नम् । \newline
3. नश्ये॑ दाश्वि॒न मा᳚श्वि॒नम् नश्ये॒न् नश्ये॑ दाश्वि॒नम् द्वि॑कपा॒लम् द्वि॑कपा॒ल मा᳚श्वि॒नम् नश्ये॒न् नश्ये॑ दाश्वि॒नम् द्वि॑कपा॒लम् । \newline
4. आ॒श्वि॒नम् द्वि॑कपा॒लम् द्वि॑कपा॒ल मा᳚श्वि॒न मा᳚श्वि॒नम् द्वि॑कपा॒लम् निर् णिर् द्वि॑कपा॒ल मा᳚श्वि॒न मा᳚श्वि॒नम् द्वि॑कपा॒लम् निः । \newline
5. द्वि॒क॒पा॒लम् निर् णिर् द्वि॑कपा॒लम् द्वि॑कपा॒लम् निर् व॑पेद् वपे॒न् निर् द्वि॑कपा॒लम् द्वि॑कपा॒लम् निर् व॑पेत् । \newline
6. द्वि॒क॒पा॒लमिति॑ द्वि - क॒पा॒लम् । \newline
7. निर् व॑पेद् वपे॒न् निर् णिर् व॑पेद् द्यावापृथि॒व्य॑म् द्यावापृथि॒व्यं॑ ॅवपे॒न् निर् णिर् व॑पेद् द्यावापृथि॒व्य᳚म् । \newline
8. व॒पे॒द् द्या॒वा॒पृ॒थि॒व्य॑म् द्यावापृथि॒व्यं॑ ॅवपेद् वपेद् द्यावापृथि॒व्य॑ 
मेक॑कपाल॒ मेक॑कपालम् द्यावापृथि॒व्यं॑ ॅवपेद् वपेद् द्यावापृथि॒व्य॑ मेक॑कपालम् । \newline
9. द्या॒वा॒पृ॒थि॒व्य॑ मेक॑कपाल॒ मेक॑कपालम् द्यावापृथि॒व्य॑म् द्यावापृथि॒व्य॑ मेक॑कपाल म॒श्विना॑ व॒श्विना॒ वेक॑कपाल॒म् द्यावापृथि॒व्य॑म् द्यावापृथि॒व्य॑ मेक॑कपाल म॒श्विनौ᳚ । \newline
10. द्या॒वा॒पृ॒थि॒व्य॑मिति॑ द्यावा - पृ॒थि॒व्य᳚म् । \newline
11. एक॑कपाल म॒श्विना॑ व॒श्विना॒ वेक॑कपाल॒ मेक॑कपाल म॒श्विनौ॒ वै वा अ॒श्विना॒ वेक॑कपाल॒ मेक॑कपाल म॒श्विनौ॒ वै । \newline
12. एक॑कपाल॒मित्येक॑ - क॒पा॒ल॒म् । \newline
13. अ॒श्विनौ॒ वै वा अ॒श्विना॑ व॒श्विनौ॒ वै दे॒वाना᳚म् दे॒वानां॒ ॅवा अ॒श्विना॑ व॒श्विनौ॒ वै दे॒वाना᳚म् । \newline
14. वै दे॒वाना᳚म् दे॒वानां॒ ॅवै वै दे॒वाना᳚म् भि॒षजौ॑ भि॒षजौ॑ दे॒वानां॒ ॅवै वै दे॒वाना᳚म् भि॒षजौ᳚ । \newline
15. दे॒वाना᳚म् भि॒षजौ॑ भि॒षजौ॑ दे॒वाना᳚म् दे॒वाना᳚म् भि॒षजौ॒ ताभ्या॒म् ताभ्या᳚म् भि॒षजौ॑ दे॒वाना᳚म् दे॒वाना᳚म् भि॒षजौ॒ ताभ्या᳚म् । \newline
16. भि॒षजौ॒ ताभ्या॒म् ताभ्या᳚म् भि॒षजौ॑ भि॒षजौ॒ ताभ्या॑ मे॒वैव ताभ्या᳚म् भि॒षजौ॑ भि॒षजौ॒ ताभ्या॑ मे॒व । \newline
17. ताभ्या॑ मे॒वैव ताभ्या॒म् ताभ्या॑ मे॒वास्मा॑ अस्मा ए॒व ताभ्या॒म् ताभ्या॑ मे॒वास्मै᳚ । \newline
18. ए॒वास्मा॑ अस्मा ए॒वैवास्मै॑ भेष॒जम् भे॑ष॒ज म॑स्मा ए॒वैवास्मै॑ भेष॒जम् । \newline
19. अ॒स्मै॒ भे॒ष॒जम् भे॑ष॒ज म॑स्मा अस्मै भेष॒जम् क॑रोति करोति भेष॒ज म॑स्मा अस्मै भेष॒जम् क॑रोति । \newline
20. भे॒ष॒जम् क॑रोति करोति भेष॒जम् भे॑ष॒जम् क॑रोति द्यावापृथि॒व्यो᳚ द्यावापृथि॒व्यः॑ करोति भेष॒जम् भे॑ष॒जम् क॑रोति द्यावापृथि॒व्यः॑ । \newline
21. क॒रो॒ति॒ द्या॒वा॒पृ॒थि॒व्यो᳚ द्यावापृथि॒व्यः॑ करोति करोति द्यावापृथि॒व्य॑ एक॑कपाल॒ एक॑कपालो द्यावापृथि॒व्यः॑ करोति करोति द्यावापृथि॒व्य॑ एक॑कपालः । \newline
22. द्या॒वा॒पृ॒थि॒व्य॑ एक॑कपाल॒ एक॑कपालो द्यावापृथि॒व्यो᳚ द्यावापृथि॒व्य॑ एक॑कपालो भवति भव॒त्येक॑कपालो द्यावापृथि॒व्यो᳚ द्यावापृथि॒व्य॑ एक॑कपालो भवति । \newline
23. द्या॒वा॒पृ॒थि॒व्य॑ इति॑ द्यावा - पृ॒थि॒व्यः॑ । \newline
24. एक॑कपालो भवति भव॒ त्येक॑कपाल॒ एक॑कपालो भव त्य॒नयो॑ र॒नयो᳚र् भव॒ त्येक॑कपाल॒ एक॑कपालो भव त्य॒नयोः᳚ । \newline
25. एक॑कपाल॒ इत्येक॑ - क॒पा॒लः॒ । \newline
26. भ॒व॒ त्य॒नयो॑ र॒नयो᳚र् भवति भव त्य॒नयो॒र् वै वा अ॒नयो᳚र् भवति भव त्य॒नयो॒र् वै । \newline
27. अ॒नयो॒र् वै वा अ॒नयो॑ र॒नयो॒र् वा ए॒त दे॒तद् वा अ॒नयो॑ र॒नयो॒र् वा ए॒तत् । \newline
28. वा ए॒त दे॒तद् वै वा ए॒तन् न॑श्यति नश्य त्ये॒तद् वै वा ए॒तन् न॑श्यति । \newline
29. ए॒तन् न॑श्यति नश्य त्ये॒त दे॒तन् न॑श्यति॒ यद् यन् न॑श्य त्ये॒त दे॒तन् न॑श्यति॒ यत् । \newline
30. न॒श्य॒ति॒ यद् यन् न॑श्यति नश्यति॒ यन् नश्य॑ति॒ नश्य॑ति॒ यन् न॑श्यति नश्यति॒ यन् नश्य॑ति । \newline
31. यन् नश्य॑ति॒ नश्य॑ति॒ यद् यन् नश्य॑ त्य॒नयो॑ र॒नयो॒र् नश्य॑ति॒ यद् यन् नश्य॑ त्य॒नयोः᳚ । \newline
32. नश्य॑ त्य॒नयो॑ र॒नयो॒र् नश्य॑ति॒ नश्य॑ त्य॒नयो॑ रे॒वैवानयो॒र् नश्य॑ति॒ नश्य॑ त्य॒नयो॑रे॒व । \newline
33. अ॒नयो॑ रे॒वैवानयो॑ र॒नयो॑ रे॒वैन॑ देन दे॒वानयो॑ र॒नयो॑ रे॒वैन॑त् । \newline
34. ए॒वैन॑ देन दे॒वैवैन॑द् विन्दति विन्द त्येन दे॒वैवैन॑द् विन्दति । \newline
35. ए॒न॒द् वि॒न्द॒ति॒ वि॒न्द॒ त्ये॒न॒ दे॒न॒द् वि॒न्द॒ति॒ प्रति॑ष्ठित्यै॒ प्रति॑ष्ठित्यै विन्द त्येन देनद् विन्दति॒ प्रति॑ष्ठित्यै । \newline
36. वि॒न्द॒ति॒ प्रति॑ष्ठित्यै॒ प्रति॑ष्ठित्यै विन्दति विन्दति॒ प्रति॑ष्ठित्यै । \newline
37. प्रति॑ष्ठित्या॒ इति॒ प्रति॑ - स्थि॒त्यै॒ । \newline
\pagebreak
\markright{ TS 2.6.4.1  \hfill https://www.vedavms.in \hfill}
\addcontentsline{toc}{section}{ TS 2.6.4.1 }
\section*{ TS 2.6.4.1 }

\textbf{TS 2.6.4.1 } \newline
\textbf{Samhita Paata} \newline

दे॒वस्य॑ त्वा सवि॒तुः प्र॑स॒व इति॒ स्फ्यमा द॑त्ते॒ प्रसू᳚त्या अ॒श्विनो᳚ र्बा॒हुभ्या॒मित्या॑हा॒-श्विनौ॒ हि दे॒वाना॑मद्ध्व॒र्यू आस्तां᳚ पू॒ष्णो हस्ता᳚भ्या॒मित्या॑ह॒ यत्यै॑ श॒तभृ॑ष्टिरसि वानस्प॒त्यो द्वि॑ष॒तो व॒ध इत्या॑ह॒ वज्र॑मे॒व तथ् सꣳश्य॑ति॒ भ्रातृ॑व्याय प्रहरि॒ष्यन्थ् स्त॑म्ब य॒जुर्. ह॑रत्ये॒ताव॑ती॒ वै पृ॑थि॒वी याव॑ती॒ वेदि॒स्तस्या॑ ए॒ताव॑त ए॒व भ्रातृ॑व्यं॒ निर्भ॑जति॒ - [  ] \newline

\textbf{Pada Paata} \newline

दे॒वस्य॑ । त्वा॒ । स॒वि॒तुः । प्र॒स॒व इति॑ प्र - स॒वे । इति॑ । स्फ्यम् । एति॑ । द॒त्ते॒ । प्रसू᳚त्या॒ इति॒ प्र - सू॒त्यै॒ । अ॒श्विनोः᳚ । बा॒हुभ्या॒मिति॑ बा॒हु-भ्या॒म् । इति॑ । आ॒ह॒ । अ॒श्विनौ᳚ । हि । दे॒वाना᳚म् । अ॒द्ध्व॒र्यू इति॑ । आस्ता᳚म् । पू॒ष्णः । हस्ता᳚भ्याम् । इति॑ । आ॒ह॒ । यत्यै᳚ । श॒तभृ॑ष्टि॒रिति॑ श॒त - भृ॒ष्टिः॒ । अ॒सि॒ । वा॒न॒स्प॒त्यः । द्वि॒ष॒तः । व॒धः । इति॑ । आ॒ह॒ । वज्र᳚म् । ए॒व । तत् । समिति॑ । श्य॒ति॒ । भ्रातृ॑व्याय । प्र॒ह॒रि॒ष्यन्निति॑ प्र - ह॒रि॒ष्यन्न् । स्त॒बं॒य॒जुरिति॑ स्तंब - य॒जुः । ह॒र॒ति॒ । ए॒ताव॑ती । वै । पृ॒थि॒वी । याव॑ती । वेदिः॑ । तस्याः᳚ । ए॒ताव॑तः । ए॒व । भ्रातृ॑व्यम् । निरिति॑ । भ॒ज॒ति॒ ।  \newline


\textbf{Krama Paata} \newline

दे॒वस्य॑ त्वा । त्वा॒ स॒वि॒तुः । स॒वि॒तुः प्र॑स॒वे । प्र॒स॒व इति॑ । प्र॒स॒व इति॑ प्र - स॒वे । इति॒ स्फ्यम् । स्फ्यमा । आ द॑त्ते । द॒त्ते॒ प्रसू᳚त्यै । प्रसू᳚त्या अ॒श्विनोः᳚ । प्रसू᳚त्या॒ इति॒ प्र - सू॒त्यै॒ । अ॒श्विनो᳚र् बा॒हुभ्या᳚म् । बा॒हुभ्या॒मिति॑ । बा॒हुभ्या॒मिति॑ बा॒हु - भ्या॒म् । इत्या॑ह । आ॒हा॒श्विनौ᳚ । अ॒श्विनौ॒ हि । हि दे॒वाना᳚म् । दे॒वाना॑मद्ध्व॒र्यू । अ॒द्ध्व॒र्यू आस्ता᳚म् । अ॒द्ध्व॒र्यू इत्य॑द्ध्व॒र्यू । आस्ता᳚म् पू॒ष्णः । पू॒ष्णो हस्ता᳚भ्याम् । हस्ता᳚भ्या॒मिति॑ । इत्या॑ह । आ॒ह॒ यत्यै᳚ । यत्यै॑ श॒तभृ॑ष्टिः । श॒तभृ॑ष्टिरसि । श॒तभृ॑ष्टि॒रिति॑ श॒त - भृ॒ष्टिः॒ । अ॒सि॒ वा॒न॒स्प॒त्यः । वा॒न॒स्प॒त्यो द्वि॑ष॒तः । द्वि॒ष॒तो व॒धः । व॒ध इति॑ । इत्या॑ह । आ॒ह॒ वज्र᳚म् । वज्र॑मे॒व । ए॒व तत् । तथ् सम् । सꣳ श्य॑ति । श्य॒ति॒ भ्रातृ॑व्याय । भा॑तृव्याय प्रहरि॒ष्यन्न् । प्र॒ह॒रि॒ष्यन्थ् स्त॑म्बय॒जुः । प्र॒ह॒रि॒ष्यन्निति॑ प्र - ह॒रि॒ष्यन्न् । स्त॒म्ब॒य॒जुर्. ह॑रति । स्त॒म्ब॒य॒जुरिति॑ स्तम्ब - य॒जुः । ह॒र॒त्ये॒ताव॑ती । ए॒ताव॑ती॒ वै । वै पृ॑थि॒वी । पृ॒थि॒वी याव॑ती । याव॑ती॒ वेदिः॑ । वेदि॒स्तस्याः᳚ । तस्या॑ ए॒ताव॑तः । ए॒ताव॑त ए॒व । ए॒व भ्रातृ॑व्यम् । भ्रातृ॑व्य॒म् निः । निर् भ॑जति । भ॒ज॒ति॒ तस्मा᳚त् \newline

\textbf{Jatai Paata} \newline

1. दे॒वस्य॑ त्वा त्वा दे॒वस्य॑ दे॒वस्य॑ त्वा । \newline
2. त्वा॒ स॒वि॒तुः स॑वि॒तु स्त्वा᳚ त्वा सवि॒तुः । \newline
3. स॒वि॒तुः प्र॑स॒वे प्र॑स॒वे स॑वि॒तुः स॑वि॒तुः प्र॑स॒वे । \newline
4. प्र॒स॒व इतीति॑ प्रस॒वे प्र॑स॒व इति॑ । \newline
5. प्र॒स॒व इति॑ प्र - स॒वे । \newline
6. इति॒ स्फ्यꣳ स्फ्य मितीति॒ स्फ्यम् । \newline
7. स्फ्य मा स्फ्यꣳ स्फ्य मा । \newline
8. आ द॑त्ते दत्त॒ आ द॑त्ते । \newline
9. द॒त्ते॒ प्रसू᳚त्यै॒ प्रसू᳚त्यै दत्ते दत्ते॒ प्रसू᳚त्यै । \newline
10. प्रसू᳚त्या अ॒श्विनो॑ र॒श्विनोः॒ प्रसू᳚त्यै॒ प्रसू᳚त्या अ॒श्विनोः᳚ । \newline
11. प्रसू᳚त्या॒ इति॒ प्र - सू॒त्यै॒ । \newline
12. अ॒श्विनो᳚र् बा॒हुभ्या᳚म् बा॒हुभ्या॑ म॒श्विनो॑ र॒श्विनो᳚र् बा॒हुभ्या᳚म् । \newline
13. बा॒हुभ्या॒ मितीति॑ बा॒हुभ्या᳚म् बा॒हुभ्या॒ मिति॑ । \newline
14. बा॒हुभ्या॒मिति॑ बा॒हु - भ्या॒म् । \newline
15. इत्या॑हा॒हे तीत्या॑ह । \newline
16. आ॒हा॒श्विना॑ व॒श्विना॑ वाहा हा॒श्विनौ᳚ । \newline
17. अ॒श्विनौ॒ हि ह्य॑श्विना॑ व॒श्विनौ॒ हि । \newline
18. हि दे॒वाना᳚म् दे॒वानाꣳ॒॒ हि हि दे॒वाना᳚म् । \newline
19. दे॒वाना॑ मद्ध्व॒र्यू अ॑द्ध्व॒र्यू दे॒वाना᳚म् दे॒वाना॑ मद्ध्व॒र्यू । \newline
20. अ॒द्ध्व॒र्यू आस्ता॒ मास्ता॑ मद्ध्व॒र्यू अ॑द्ध्व॒र्यू आस्ता᳚म् । \newline
21. अ॒द्ध्व॒र्यू इत्य॑द्ध्व॒र्यू । \newline
22. आस्ता᳚म् पू॒ष्णः पू॒ष्ण आस्ता॒ मास्ता᳚म् पू॒ष्णः । \newline
23. पू॒ष्णो हस्ता᳚भ्याꣳ॒॒ हस्ता᳚भ्याम् पू॒ष्णः पू॒ष्णो हस्ता᳚भ्याम् । \newline
24. हस्ता᳚भ्या॒ मितीति॒ हस्ता᳚भ्याꣳ॒॒ हस्ता᳚भ्या॒ मिति॑ । \newline
25. इत्या॑हा॒हे तीत्या॑ह । \newline
26. आ॒ह॒ यत्यै॒ यत्या॑ आहाह॒ यत्यै᳚ । \newline
27. यत्यै॑ श॒तभृ॑ष्टिः श॒तभृ॑ष्टि॒र् यत्यै॒ यत्यै॑ श॒तभृ॑ष्टिः । \newline
28. श॒तभृ॑ष्टि रस्यसि श॒तभृ॑ष्टिः श॒तभृ॑ष्टि रसि । \newline
29. श॒तभृ॑ष्टि॒रिति॑ श॒त - भृ॒ष्टिः॒ । \newline
30. अ॒सि॒ वा॒न॒स्प॒त्यो वा॑नस्प॒त्यो᳚ ऽस्यसि वानस्प॒त्यः । \newline
31. वा॒न॒स्प॒त्यो द्वि॑ष॒तो द्वि॑ष॒तो वा॑नस्प॒त्यो वा॑नस्प॒त्यो द्वि॑ष॒तः । \newline
32. द्वि॒ष॒तो व॒धो व॒धो द्वि॑ष॒तो द्वि॑ष॒तो व॒धः । \newline
33. व॒ध इतीति॑ व॒धो व॒ध इति॑ । \newline
34. इत्या॑ हा॒हे तीत्या॑ह । \newline
35. आ॒ह॒ वज्रं॒ ॅवज्र॑ माहाह॒ वज्र᳚म् । \newline
36. वज्र॑ मे॒वैव वज्रं॒ ॅवज्र॑ मे॒व । \newline
37. ए॒व तत् तदे॒वैव तत् । \newline
38. तथ् सꣳ सम् तत् तथ् सम् । \newline
39. सꣳ श्य॑ति श्यति॒ सꣳ सꣳ श्य॑ति । \newline
40. श्य॒ति॒ भ्रातृ॑व्याय॒ भ्रातृ॑व्याय श्यति श्यति॒ भ्रातृ॑व्याय । \newline
41. भ्रातृ॑व्याय प्रहरि॒ष्यन् प्र॑हरि॒ष्यन् भ्रातृ॑व्याय॒ भ्रातृ॑व्याय प्रहरि॒ष्यन्न् । \newline
42. प्र॒ह॒रि॒ष्यन् थ्स्त॑म्बय॒जुः स्त॑म्बय॒जुः प्र॑हरि॒ष्यन् प्र॑हरि॒ष्यन् थ्स्त॑म्बय॒जुः । \newline
43. प्र॒ह॒रि॒ष्यन्निति॑ प्र - ह॒रि॒ष्यन्न् । \newline
44. स्त॒म्ब॒य॒जुर्. ह॑रति हरति स्तम्बय॒जुः स्त॑म्बय॒जुर्. ह॑रति । \newline
45. स्त॒म्ब॒य॒जुरिति॑ स्तम्ब - य॒जुः । \newline
46. ह॒र॒ त्ये॒ताव॑ त्ये॒ताव॑ती हरति हर त्ये॒ताव॑ती । \newline
47. ए॒ताव॑ती॒ वै वा ए॒ताव॑ त्ये॒ताव॑ती॒ वै । \newline
48. वै पृ॑थि॒वी पृ॑थि॒वी वै वै पृ॑थि॒वी । \newline
49. पृ॒थि॒वी याव॑ती॒ याव॑ती पृथि॒वी पृ॑थि॒वी याव॑ती । \newline
50. याव॑ती॒ वेदि॒र् वेदि॒र् याव॑ती॒ याव॑ती॒ वेदिः॑ । \newline
51. वेदि॒ स्तस्या॒ स्तस्या॒ वेदि॒र् वेदि॒ स्तस्याः᳚ । \newline
52. तस्या॑ ए॒ताव॑त ए॒ताव॑त॒ स्तस्या॒ स्तस्या॑ ए॒ताव॑तः । \newline
53. ए॒ताव॑त ए॒वैवैताव॑त ए॒ताव॑त ए॒व । \newline
54. ए॒व भ्रातृ॑व्य॒म् भ्रातृ॑व्य मे॒वैव भ्रातृ॑व्यम् । \newline
55. भ्रातृ॑व्य॒म् निर् णिर् भ्रातृ॑व्य॒म् भ्रातृ॑व्य॒म् निः । \newline
56. निर् भ॑जति भजति॒ निर् णिर् भ॑जति । \newline
57. भ॒ज॒ति॒ तस्मा॒त् तस्मा᳚द् भजति भजति॒ तस्मा᳚त् । \newline

\textbf{Ghana Paata } \newline

1. दे॒वस्य॑ त्वा त्वा दे॒वस्य॑ दे॒वस्य॑ त्वा सवि॒तुः स॑वि॒तु स्त्वा॑ दे॒वस्य॑ दे॒वस्य॑ त्वा सवि॒तुः । \newline
2. त्वा॒ स॒वि॒तुः स॑वि॒तु स्त्वा᳚ त्वा सवि॒तुः प्र॑स॒वे प्र॑स॒वे स॑वि॒तु स्त्वा᳚ त्वा सवि॒तुः प्र॑स॒वे । \newline
3. स॒वि॒तुः प्र॑स॒वे प्र॑स॒वे स॑वि॒तुः स॑वि॒तुः प्र॑स॒व इतीति॑ प्रस॒वे स॑वि॒तुः स॑वि॒तुः प्र॑स॒व इति॑ । \newline
4. प्र॒स॒व इतीति॑ प्रस॒वे प्र॑स॒व इति॒ स्फ्यꣳ स्फ्य मिति॑ प्रस॒वे प्र॑स॒व इति॒ स्फ्यम् । \newline
5. प्र॒स॒व इति॑ प्र - स॒वे । \newline
6. इति॒ स्फ्यꣳ स्फ्य मितीति॒ स्फ्य मा स्फ्य मितीति॒ स्फ्य मा । \newline
7. स्फ्य मा स्फ्यꣳ स्फ्य मा द॑त्ते दत्त॒ आ स्फ्यꣳ स्फ्य मा द॑त्ते । \newline
8. आ द॑त्ते दत्त॒ आ द॑त्ते॒ प्रसू᳚त्यै॒ प्रसू᳚त्यै दत्त॒ आ द॑त्ते॒ प्रसू᳚त्यै । \newline
9. द॒त्ते॒ प्रसू᳚त्यै॒ प्रसू᳚त्यै दत्ते दत्ते॒ प्रसू᳚त्या अ॒श्विनो॑ र॒श्विनोः॒ प्रसू᳚त्यै दत्ते दत्ते॒ प्रसू᳚त्या अ॒श्विनोः᳚ । \newline
10. प्रसू᳚त्या अ॒श्विनो॑ र॒श्विनोः॒ प्रसू᳚त्यै॒ प्रसू᳚त्या अ॒श्विनो᳚र् बा॒हुभ्या᳚म् बा॒हुभ्या॑ म॒श्विनोः॒ प्रसू᳚त्यै॒ प्रसू᳚त्या अ॒श्विनो᳚र् बा॒हुभ्या᳚म् । \newline
11. प्रसू᳚त्या॒ इति॒ प्र - सू॒त्यै॒ । \newline
12. अ॒श्विनो᳚र् बा॒हुभ्या᳚म् बा॒हुभ्या॑ म॒श्विनो॑ र॒श्विनो᳚र् बा॒हुभ्या॒ मितीति॑ बा॒हुभ्या॑ म॒श्विनो॑ र॒श्विनो᳚र् बा॒हुभ्या॒ मिति॑ । \newline
13. बा॒हुभ्या॒ मितीति॑ बा॒हुभ्या᳚म् बा॒हुभ्या॒ मित्या॑हा॒हे ति॑ बा॒हुभ्या᳚म् बा॒हुभ्या॒ मित्या॑ह । \newline
14. बा॒हुभ्या॒मिति॑ बा॒हु - भ्या॒म् । \newline
15. इत्या॑हा॒हे तीत्या॑हा॒ श्विना॑ व॒श्विना॑ वा॒हे तीत्या॑हा॒श्विनौ᳚ । \newline
16. आ॒हा॒श्विना॑ व॒श्विना॑ वाहाहा॒ श्विनौ॒ हि ह्य॑श्विना॑ वाहाहा॒ श्विनौ॒ हि । \newline
17. अ॒श्विनौ॒ हि ह्य॑श्विना॑ व॒श्विनौ॒ हि दे॒वाना᳚म् दे॒वानाꣳ॒॒ ह्य॑श्विना॑ व॒श्विनौ॒ हि दे॒वाना᳚म् । \newline
18. हि दे॒वाना᳚म् दे॒वानाꣳ॒॒ हि हि दे॒वाना॑ मद्ध्व॒र्यू अ॑द्ध्व॒र्यू दे॒वानाꣳ॒॒ हि हि दे॒वाना॑ मद्ध्व॒र्यू । \newline
19. दे॒वाना॑ मद्ध्व॒र्यू अ॑द्ध्व॒र्यू दे॒वाना᳚म् दे॒वाना॑ मद्ध्व॒र्यू आस्ता॒ मास्ता॑ मद्ध्व॒र्यू दे॒वाना᳚म् दे॒वाना॑ मद्ध्व॒र्यू आस्ता᳚म् । \newline
20. अ॒द्ध्व॒र्यू आस्ता॒ मास्ता॑ मद्ध्व॒र्यू अ॑द्ध्व॒र्यू आस्ता᳚म् पू॒ष्णः पू॒ष्ण आस्ता॑ मद्ध्व॒र्यू अ॑द्ध्व॒र्यू आस्ता᳚म् पू॒ष्णः । \newline
21. अ॒द्ध्व॒र्यू इत्य॑द्ध्व॒र्यू । \newline
22. आस्ता᳚म् पू॒ष्णः पू॒ष्ण आस्ता॒ मास्ता᳚म् पू॒ष्णो हस्ता᳚भ्याꣳ॒॒ हस्ता᳚भ्याम् पू॒ष्ण आस्ता॒ मास्ता᳚म् पू॒ष्णो हस्ता᳚भ्याम् । \newline
23. पू॒ष्णो हस्ता᳚भ्याꣳ॒॒ हस्ता᳚भ्याम् पू॒ष्णः पू॒ष्णो हस्ता᳚भ्या॒ मितीति॒ हस्ता᳚भ्याम् पू॒ष्णः पू॒ष्णो हस्ता᳚भ्या॒ मिति॑ । \newline
24. हस्ता᳚भ्या॒ मितीति॒ हस्ता᳚भ्याꣳ॒॒ हस्ता᳚भ्या॒ मित्या॑हा॒हे ति॒ हस्ता᳚भ्याꣳ॒॒ हस्ता᳚भ्या॒ मित्या॑ह । \newline
25. इत्या॑हा॒हे तीत्या॑ह॒ यत्यै॒ यत्या॑ आ॒हे तीत्या॑ह॒ यत्यै᳚ । \newline
26. आ॒ह॒ यत्यै॒ यत्या॑ आहाह॒ यत्यै॑ श॒तभृ॑ष्टिः श॒तभृ॑ष्टि॒र् यत्या॑ आहाह॒ यत्यै॑ श॒तभृ॑ष्टिः । \newline
27. यत्यै॑ श॒तभृ॑ष्टिः श॒तभृ॑ष्टि॒र् यत्यै॒ यत्यै॑ श॒तभृ॑ष्टि रस्यसि श॒तभृ॑ष्टि॒र् यत्यै॒ यत्यै॑ श॒तभृ॑ष्टिरसि । \newline
28. श॒तभृ॑ष्टि रस्यसि श॒तभृ॑ष्टिः श॒तभृ॑ष्टिरसि वानस्प॒त्यो वा॑नस्प॒त्यो॑ ऽसि श॒तभृ॑ष्टिः श॒तभृ॑ष्टिरसि वानस्प॒त्यः । \newline
29. श॒तभृ॑ष्टि॒रिति॑ श॒त - भृ॒ष्टिः॒ । \newline
30. अ॒सि॒ वा॒न॒स्प॒त्यो वा॑नस्प॒त्यो᳚ ऽस्यसि वानस्प॒त्यो द्वि॑ष॒तो द्वि॑ष॒तो वा॑नस्प॒त्यो᳚ ऽस्यसि वानस्प॒त्यो द्वि॑ष॒तः । \newline
31. वा॒न॒स्प॒त्यो द्वि॑ष॒तो द्वि॑ष॒तो वा॑नस्प॒त्यो वा॑नस्प॒त्यो द्वि॑ष॒तो व॒धो व॒धो द्वि॑ष॒तो वा॑नस्प॒त्यो वा॑नस्प॒त्यो द्वि॑ष॒तो व॒धः । \newline
32. द्वि॒ष॒तो व॒धो व॒धो द्वि॑ष॒तो द्वि॑ष॒तो व॒ध इतीति॑ व॒धो द्वि॑ष॒तो द्वि॑ष॒तो व॒ध इति॑ । \newline
33. व॒ध इतीति॑ व॒धो व॒ध इत्या॑हा॒हे ति॑ व॒धो व॒ध इत्या॑ह । \newline
34. इत्या॑हा॒हे तीत्या॑ह॒ वज्रं॒ ॅवज्र॑ मा॒हे तीत्या॑ह॒ वज्र᳚म् । \newline
35. आ॒ह॒ वज्रं॒ ॅवज्र॑ माहाह॒ वज्र॑ मे॒वैव वज्र॑ माहाह॒ वज्र॑ मे॒व । \newline
36. वज्र॑ मे॒वैव वज्रं॒ ॅवज्र॑ मे॒व तत् तदे॒व वज्रं॒ ॅवज्र॑ मे॒व तत् । \newline
37. ए॒व तत् तदे॒वैव तथ् सꣳ सम् तदे॒वैव तथ् सम् । \newline
38. तथ् सꣳ सम् तत् तथ् सꣳ श्य॑ति श्यति॒ सम् तत् तथ् सꣳ श्य॑ति । \newline
39. सꣳ श्य॑ति श्यति॒ सꣳ सꣳ श्य॑ति॒ भ्रातृ॑व्याय॒ भ्रातृ॑व्याय श्यति॒ सꣳ सꣳ श्य॑ति॒ भ्रातृ॑व्याय । \newline
40. श्य॒ति॒ भ्रातृ॑व्याय॒ भ्रातृ॑व्याय श्यति श्यति॒ भ्रातृ॑व्याय प्रहरि॒ष्यन् प्र॑हरि॒ष्यन् भ्रातृ॑व्याय श्यति श्यति॒ भ्रातृ॑व्याय प्रहरि॒ष्यन्न् । \newline
41. भ्रातृ॑व्याय प्रहरि॒ष्यन् प्र॑हरि॒ष्यन् भ्रातृ॑व्याय॒ भ्रातृ॑व्याय प्रहरि॒ष्यन् थ्स्त॑म्बय॒जुः स्त॑म्बय॒जुः प्र॑हरि॒ष्यन् भ्रातृ॑व्याय॒ भ्रातृ॑व्याय प्रहरि॒ष्यन् थ्स्त॑म्बय॒जुः । \newline
42. प्र॒ह॒रि॒ष्यन् थ्स्त॑म्बय॒जुः स्त॑म्बय॒जुः प्र॑हरि॒ष्यन् प्र॑हरि॒ष्यन् थ्स्त॑म्बय॒जुर्. ह॑रति हरति स्तम्बय॒जुः प्र॑हरि॒ष्यन् प्र॑हरि॒ष्यन् थ्स्त॑म्बय॒जुर्. ह॑रति । \newline
43. प्र॒ह॒रि॒ष्यन्निति॑ प्र - ह॒रि॒ष्यन्न् । \newline
44. स्त॒म्ब॒य॒जुर्. ह॑रति हरति स्तम्बय॒जुः स्त॑म्बय॒जुर्. ह॑र त्ये॒ताव॑ त्ये॒ताव॑ती हरति स्तम्बय॒जुः स्त॑म्बय॒जुर्. ह॑रत्ये॒ताव॑ती । \newline
45. स्त॒म्ब॒य॒जुरिति॑ स्तम्ब - य॒जुः । \newline
46. ह॒र॒ त्ये॒ताव॑ त्ये॒ताव॑ती हरति हरत्ये॒ताव॑ती॒ वै वा ए॒ताव॑ती हरति हर त्ये॒ताव॑ती॒ वै । \newline
47. ए॒ताव॑ती॒ वै वा ए॒ताव॑ त्ये॒ताव॑ती॒ वै पृ॑थि॒वी पृ॑थि॒वी वा ए॒ताव॑ त्ये॒ताव॑ती॒ वै पृ॑थि॒वी । \newline
48. वै पृ॑थि॒वी पृ॑थि॒वी वै वै पृ॑थि॒वी याव॑ती॒ याव॑ती पृथि॒वी वै वै पृ॑थि॒वी याव॑ती । \newline
49. पृ॒थि॒वी याव॑ती॒ याव॑ती पृथि॒वी पृ॑थि॒वी याव॑ती॒ वेदि॒र् वेदि॒र् याव॑ती पृथि॒वी पृ॑थि॒वी याव॑ती॒ वेदिः॑ । \newline
50. याव॑ती॒ वेदि॒र् वेदि॒र् याव॑ती॒ याव॑ती॒ वेदि॒ स्तस्या॒ स्तस्या॒ वेदि॒र् याव॑ती॒ याव॑ती॒ वेदि॒ स्तस्याः᳚ । \newline
51. वेदि॒ स्तस्या॒ स्तस्या॒ वेदि॒र् वेदि॒ स्तस्या॑ ए॒ताव॑त ए॒ताव॑त॒ स्तस्या॒ वेदि॒र् वेदि॒ स्तस्या॑ ए॒ताव॑तः । \newline
52. तस्या॑ ए॒ताव॑त ए॒ताव॑त॒ स्तस्या॒ स्तस्या॑ ए॒ताव॑त ए॒वैवैताव॑त॒ स्तस्या॒ स्तस्या॑ ए॒ताव॑त ए॒व । \newline
53. ए॒ताव॑त ए॒वैवैताव॑त ए॒ताव॑त ए॒व भ्रातृ॑व्य॒म् भ्रातृ॑व्य मे॒वैताव॑त ए॒ताव॑त ए॒व भ्रातृ॑व्यम् । \newline
54. ए॒व भ्रातृ॑व्य॒म् भ्रातृ॑व्य मे॒वैव भ्रातृ॑व्य॒म् निर् णिर् भ्रातृ॑व्य मे॒वैव भ्रातृ॑व्य॒म् निः । \newline
55. भ्रातृ॑व्य॒म् निर् णिर् भ्रातृ॑व्य॒म् भ्रातृ॑व्य॒म् निर् भ॑जति भजति॒ निर् भ्रातृ॑व्य॒म् भ्रातृ॑व्य॒म् निर् भ॑जति । \newline
56. निर् भ॑जति भजति॒ निर् णिर् भ॑जति॒ तस्मा॒त् तस्मा᳚द् भजति॒ निर् णिर् भ॑जति॒ तस्मा᳚त् । \newline
57. भ॒ज॒ति॒ तस्मा॒त् तस्मा᳚द् भजति भजति॒ तस्मा॒न् न न तस्मा᳚द् भजति भजति॒ तस्मा॒न् न । \newline
\pagebreak
\markright{ TS 2.6.4.2  \hfill https://www.vedavms.in \hfill}
\addcontentsline{toc}{section}{ TS 2.6.4.2 }
\section*{ TS 2.6.4.2 }

\textbf{TS 2.6.4.2 } \newline
\textbf{Samhita Paata} \newline

तस्मा॒न्नाभा॒गं निर्भ॑जन्ति॒ त्रिर्.ह॑रति॒ त्रय॑ इ॒मे लो॒का ए॒भ्य ए॒वैनं॑ ॅलो॒केभ्यो॒ निर्भ॑जति तू॒ष्णीं च॑तु॒र्थꣳ ह॑र॒त्यप॑रिमितादे॒वैनं॒ निर्भ॑ज॒त्युद्ध॑न्ति॒ यदे॒वास्या॑ अमे॒द्ध्यं तदप॑ ह॒न्त्युद्ध॑न्ति॒ तस्मा॒दोष॑धयः॒ परा॑ भवन्ति॒ मूलं॑ छिनत्ति॒ भ्रातृ॑व्यस्यै॒व मूलं॑ छिनत्ति पितृदेव॒त्याऽति॑खा॒तेय॑तीं खनति प्र॒जाप॑तिना - [  ] \newline

\textbf{Pada Paata} \newline

तस्मा᳚त् । न । अ॒भा॒गम् । निरिति॑ । भ॒ज॒न्ति॒ । त्रिः । ह॒र॒ति॒ । त्रयः॑ । इ॒मे । लो॒काः । ए॒भ्यः । ए॒व । ए॒न॒म् । लो॒केभ्यः॑ । निरिति॑ । भ॒ज॒ति॒ । तू॒ष्णीम् । च॒तु॒र्थम् । ह॒र॒ति॒ । अप॑रिमिता॒दित्यप॑रि-मि॒ता॒त् । ए॒व । ए॒न॒म् । निरिति॑ । भ॒ज॒ति॒ । उदिति॑ । ह॒न्ति॒ । यत् । ए॒व । अ॒स्याः॒ । अ॒मे॒द्ध्यम् । तत् । अपेति॑ । ह॒न्ति॒ । उदिति॑ । ह॒न्ति॒ । तस्मा᳚त् । ओष॑धयः । परेति॑ । भ॒व॒न्ति॒ । मूल᳚म् । छि॒न॒त्ति॒ । भ्रातृ॑व्यस्य । ए॒व । मूल᳚म् । छि॒न॒त्ति॒ । पि॒तृ॒दे॒व॒त्येति॑ पितृ - दे॒व॒त्या᳚ । अति॑खा॒तेत्यति॑ - खा॒ता॒ । इय॑तीम् । ख॒न॒ति॒ । प्र॒जाप॑ति॒नेति॑ प्र॒जा - प॒ति॒ना॒ ।  \newline


\textbf{Krama Paata} \newline

तस्मा॒न् न । नाभा॒गम् । अ॒भा॒गम् निः । निर् भ॑जन्ति । भ॒ज॒न्ति॒ त्रिः । त्रिर्. ह॑रति । ह॒र॒ति॒ त्रयः॑ । त्रय॑ इ॒मे । इ॒मे लो॒काः । लो॒का ए॒भ्यः । ए॒भ्य ए॒व । ए॒वैन᳚म् । ए॒नं॒ ॅलो॒केभ्यः॑ । लो॒केभ्यो॒ निः । निर् भ॑जति । भ॒ज॒ति॒ तू॒ष्णीम् । तू॒ष्णीम् च॑तु॒र्त्थम् । च॒तु॒र्त्थꣳ ह॑रति । ह॒र॒त्यप॑रिमितात् । अप॑रिमितादे॒व । अप॑रिमिता॒दित्यप॑रि - मि॒ता॒त्॒ । ए॒वैन᳚म् । ए॒न॒म् निः । निर् भ॑जति । भ॒ज॒त्युत् । उद्ध॑न्ति । ह॒न्ति॒ यत् । यदे॒व । ए॒वास्याः᳚ । अ॒स्या॒ अ॒मे॒द्ध्यम् । अ॒मे॒द्ध्यम् तत् । तदप॑ । अप॑ हन्ति । ह॒न्त्युत् । उद्ध॑न्ति । ह॒न्ति॒ तस्मा᳚त् । तस्मा॒दोष॑धयः । ओष॑धयः॒ परा᳚ । परा॑ भवन्ति । भ॒व॒न्ति॒ मूल᳚म् । मूल॑म् छिनत्ति । छि॒न॒त्ति॒ भातृ॑व्यस्य । भ्रातृ॑व्यस्यै॒व । ए॒व मूल᳚म् । मूल॑म् छिनत्ति । छि॒न॒त्ति॒ पि॒तृ॒दे॒वत्या᳚ । पि॒तृ॒दे॒व॒त्या ऽति॑खाता । पि॒तृ॒दे॒व॒त्येति॑ पितृ - दे॒व॒त्या᳚ । अति॑खा॒तेय॑तीम् । अति॑खा॒तेत्यति॑ - खा॒ता॒ । इय॑तीम् खनति । ख॒न॒ति॒ प्र॒जाप॑तिना । प्र॒जाप॑तिना यज्ञ्मु॒खेन॑ । प्र॒जाप॑ति॒नेति॑ प्र॒जा - प॒ति॒ना॒ \newline

\textbf{Jatai Paata} \newline

1. तस्मा॒न् न न तस्मा॒त् तस्मा॒न् न । \newline
2. नाभा॒ग म॑भा॒गम् न नाभा॒गम् । \newline
3. अ॒भा॒गम् निर् णिर॑भा॒ग म॑भा॒गम् निः । \newline
4. निर् भ॑जन्ति भजन्ति॒ निर् णिर् भ॑जन्ति । \newline
5. भ॒ज॒न्ति॒ त्रि स्त्रिर् भ॑जन्ति भजन्ति॒ त्रिः । \newline
6. त्रिर्. ह॑रति हरति॒ त्रि स्त्रिर्. ह॑रति । \newline
7. ह॒र॒ति॒ त्रय॒ स्त्रयो॑ हरति हरति॒ त्रयः॑ । \newline
8. त्रय॑ इ॒म इ॒मे त्रय॒ स्त्रय॑ इ॒मे । \newline
9. इ॒मे लो॒का लो॒का इ॒म इ॒मे लो॒काः । \newline
10. लो॒का ए॒भ्य ए॒भ्यो लो॒का लो॒का ए॒भ्यः । \newline
11. ए॒भ्य ए॒वैवैभ्य ए॒भ्य ए॒व । \newline
12. ए॒वैन॑ मेन मे॒वैवैन᳚म् । \newline
13. ए॒न॒म् ॅलो॒केभ्यो॑ लो॒केभ्य॑ एन मेनम् ॅलो॒केभ्यः॑ । \newline
14. लो॒केभ्यो॒ निर् णिर् लो॒केभ्यो॑ लो॒केभ्यो॒ निः । \newline
15. निर् भ॑जति भजति॒ निर् णिर् भ॑जति । \newline
16. भ॒ज॒ति॒ तू॒ष्णीम् तू॒ष्णीम् भ॑जति भजति तू॒ष्णीम् । \newline
17. तू॒ष्णीम् च॑तु॒र्थम् च॑तु॒र्थम् तू॒ष्णीम् तू॒ष्णीम् च॑तु॒र्थम् । \newline
18. च॒तु॒र्थꣳ ह॑रति हरति चतु॒र्थम् च॑तु॒र्थꣳ ह॑रति । \newline
19. ह॒र॒ त्यप॑रिमिता॒ दप॑रिमिता द्धरति हर॒ त्यप॑रिमितात् । \newline
20. अप॑रिमिता दे॒वैवा प॑रिमिता॒ दप॑रिमिता दे॒व । \newline
21. अप॑रिमिता॒दित्यप॑रि - मि॒ता॒त् । \newline
22. ए॒वैन॑ मेन मे॒वैवैन᳚म् । \newline
23. ए॒न॒म् निर् णिरे॑न मेन॒म् निः । \newline
24. निर् भ॑जति भजति॒ निर् णिर् भ॑जति । \newline
25. भ॒ज॒ त्युदुद् भ॑जति भज॒ त्युत् । \newline
26. उद्ध॑न्ति ह॒ न्त्युदु द्ध॑न्ति । \newline
27. ह॒न्ति॒ यद् य द्ध॑न्ति हन्ति॒ यत् । \newline
28. यदे॒वैव यद् यदे॒व । \newline
29. ए॒वास्या॑ अस्या ए॒वैवास्याः᳚ । \newline
30. अ॒स्या॒ अ॒मे॒द्ध्य म॑मे॒द्ध्य म॑स्या अस्या अमे॒द्ध्यम् । \newline
31. अ॒मे॒द्ध्यम् तत् तद॑मे॒द्ध्य म॑मे॒द्ध्यम् तत् । \newline
32. तदपाप॒ तत् तदप॑ । \newline
33. अप॑ हन्ति ह॒ न्त्यपाप॑ हन्ति । \newline
34. ह॒ न्त्युदु द्ध॑न्ति ह॒न्त्युत् । \newline
35. उद्ध॑न्ति ह॒न्त्युदु द्ध॑न्ति । \newline
36. ह॒न्ति॒ तस्मा॒त् तस्मा᳚ द्धन्ति हन्ति॒ तस्मा᳚त् । \newline
37. तस्मा॒ दोष॑धय॒ ओष॑धय॒ स्तस्मा॒त् तस्मा॒ दोष॑धयः । \newline
38. ओष॑धयः॒ परा॒ परौष॑धय॒ ओष॑धयः॒ परा᳚ । \newline
39. परा॑ भवन्ति भवन्ति॒ परा॒ परा॑ भवन्ति । \newline
40. भ॒व॒न्ति॒ मूल॒म् मूल॑म् भवन्ति भवन्ति॒ मूल᳚म् । \newline
41. मूल॑म् छिनत्ति छिनत्ति॒ मूल॒म् मूल॑म् छिनत्ति । \newline
42. छि॒न॒त्ति॒ भ्रातृ॑व्यस्य॒ भ्रातृ॑व्यस्य छिनत्ति छिनत्ति॒ भ्रातृ॑व्यस्य । \newline
43. भ्रातृ॑व्य स्यै॒वैव भ्रातृ॑व्यस्य॒ भ्रातृ॑व्य स्यै॒व । \newline
44. ए॒व मूल॒म् मूल॑ मे॒वैव मूल᳚म् । \newline
45. मूल॑म् छिनत्ति छिनत्ति॒ मूल॒म् मूल॑म् छिनत्ति । \newline
46. छि॒न॒त्ति॒ पि॒तृ॒दे॒व॒त्या॑ पितृदेव॒त्या॑ छिनत्ति छिनत्ति पितृदेव॒त्या᳚ । \newline
47. पि॒तृ॒दे॒व॒त्या ऽति॑खा॒ता ऽति॑खाता पितृदेव॒त्या॑ 
पितृदेव॒त्या ऽति॑खाता । \newline
48. पि॒तृ॒दे॒व॒त्येति॑ पितृ - दे॒व॒त्या᳚ । \newline
49. अति॑खा॒तेय॑ती॒ मिय॑ती॒ मति॑खा॒ता ऽति॑खा॒तेय॑तीम् । \newline
50. अति॑खा॒तेत्यति॑ - खा॒ता॒ । \newline
51. इय॑तीम् खनति खन॒तीय॑ती॒ मिय॑तीम् खनति । \newline
52. ख॒न॒ति॒ प्र॒जाप॑तिना प्र॒जाप॑तिना खनति खनति प्र॒जाप॑तिना । \newline
53. प्र॒जाप॑तिना यज्ञ्मु॒खेन॑ यज्ञ्मु॒खेन॑ प्र॒जाप॑तिना प्र॒जाप॑तिना यज्ञ्मु॒खेन॑ । \newline
54. प्र॒जाप॑ति॒नेति॑ प्र॒जा - प॒ति॒ना॒ । \newline

\textbf{Ghana Paata } \newline

1. तस्मा॒न् न न तस्मा॒त् तस्मा॒न् नाभा॒ग म॑भा॒गम् न तस्मा॒त् तस्मा॒न् नाभा॒गम् । \newline
2. नाभा॒ग म॑भा॒गम् न नाभा॒गम् निर् णिर॑भा॒गम् न नाभा॒गम् निः । \newline
3. अ॒भा॒गम् निर् णिर॑भा॒ग म॑भा॒गम् निर् भ॑जन्ति भजन्ति॒ निर॑भा॒ग म॑भा॒गम् निर् भ॑जन्ति । \newline
4. निर् भ॑जन्ति भजन्ति॒ निर् णिर् भ॑जन्ति॒ त्रि स्त्रिर् भ॑जन्ति॒ निर् णिर् भ॑जन्ति॒ त्रिः । \newline
5. भ॒ज॒न्ति॒ त्रि स्त्रिर् भ॑जन्ति भजन्ति॒ त्रिर्. ह॑रति हरति॒ त्रिर् भ॑जन्ति भजन्ति॒ त्रिर्. ह॑रति । \newline
6. त्रिर्. ह॑रति हरति॒ त्रि स्त्रिर्. ह॑रति॒ त्रय॒ स्त्रयो॑ हरति॒ त्रि स्त्रिर्. ह॑रति॒ त्रयः॑ । \newline
7. ह॒र॒ति॒ त्रय॒ स्त्रयो॑ हरति हरति॒ त्रय॑ इ॒म इ॒मे त्रयो॑ हरति हरति॒ त्रय॑ इ॒मे । \newline
8. त्रय॑ इ॒म इ॒मे त्रय॒ स्त्रय॑ इ॒मे लो॒का लो॒का इ॒मे त्रय॒ स्त्रय॑ इ॒मे लो॒काः । \newline
9. इ॒मे लो॒का लो॒का इ॒म इ॒मे लो॒का ए॒भ्य ए॒भ्यो लो॒का इ॒म इ॒मे लो॒का ए॒भ्यः । \newline
10. लो॒का ए॒भ्य ए॒भ्यो लो॒का लो॒का ए॒भ्य ए॒वैवैभ्यो लो॒का लो॒का ए॒भ्य ए॒व । \newline
11. ए॒भ्य ए॒वैवैभ्य ए॒भ्य ए॒वैन॑ मेन मे॒वैभ्य ए॒भ्य ए॒वैन᳚म् । \newline
12. ए॒वैन॑ मेन मे॒वैवैन॑म् ॅलो॒केभ्यो॑ लो॒केभ्य॑ एन मे॒वैवैन॑म् ॅलो॒केभ्यः॑ । \newline
13. ए॒न॒म् ॅलो॒केभ्यो॑ लो॒केभ्य॑ एन मेनम् ॅलो॒केभ्यो॒ निर् णिर् लो॒केभ्य॑ एन मेनम् ॅलो॒केभ्यो॒ निः । \newline
14. लो॒केभ्यो॒ निर् णिर् लो॒केभ्यो॑ लो॒केभ्यो॒ निर् भ॑जति भजति॒ निर् लो॒केभ्यो॑ लो॒केभ्यो॒ निर् भ॑जति । \newline
15. निर् भ॑जति भजति॒ निर् णिर् भ॑जति तू॒ष्णीम् तू॒ष्णीम् भ॑जति॒ निर् णिर् भ॑जति तू॒ष्णीम् । \newline
16. भ॒ज॒ति॒ तू॒ष्णीम् तू॒ष्णीम् भ॑जति भजति तू॒ष्णीम् च॑तु॒र्थम् च॑तु॒र्थम् तू॒ष्णीम् भ॑जति भजति तू॒ष्णीम् च॑तु॒र्थम् । \newline
17. तू॒ष्णीम् च॑तु॒र्थम् च॑तु॒र्थम् तू॒ष्णीम् तू॒ष्णीम् च॑तु॒र्थꣳ ह॑रति हरति चतु॒र्थम् तू॒ष्णीम् तू॒ष्णीम् च॑तु॒र्थꣳ ह॑रति । \newline
18. च॒तु॒र्थꣳ ह॑रति हरति चतु॒र्थम् च॑तु॒र्थꣳ ह॑र॒ त्यप॑रिमिता॒ दप॑रिमिता द्धरति चतु॒र्थम् च॑तु॒र्थꣳ ह॑र॒त्यप॑रिमितात् । \newline
19. ह॒र॒ त्यप॑रिमिता॒ दप॑रिमिता द्धरति हर॒ त्यप॑रिमिता दे॒वैवा प॑रिमिता द्धरति हर॒ त्यप॑रिमितादे॒व । \newline
20. अप॑रिमिता दे॒वैवा प॑रिमिता॒ दप॑रिमिता दे॒वैन॑ मेन मे॒वा प॑रिमिता॒ दप॑रिमिता दे॒वैन᳚म् । \newline
21. अप॑रिमिता॒दित्यप॑रि - मि॒ता॒त् । \newline
22. ए॒वैन॑ मेन मे॒वैवैन॒म् निर् णिरे॑न मे॒वैवैन॒म् निः । \newline
23. ए॒न॒म् निर् णिरे॑न मेन॒म् निर् भ॑जति भजति॒ निरे॑न मेन॒म् निर् भ॑जति । \newline
24. निर् भ॑जति भजति॒ निर् णिर् भ॑ज॒ त्युदुद् भ॑जति॒ निर् णिर् भ॑ज॒त्युत् । \newline
25. भ॒ज॒ त्युदुद् भ॑जति भज॒ त्युद्ध॑न्ति ह॒न्त्युद् भ॑जति भज॒ त्युद्ध॑न्ति । \newline
26. उद्ध॑न्ति ह॒न्त्युदु द्ध॑न्ति॒ यद् यद्ध॒न्त्युदु द्ध॑न्ति॒ यत् । \newline
27. ह॒न्ति॒ यद् यद्ध॑न्ति हन्ति॒ यदे॒वैव यद्ध॑न्ति हन्ति॒ यदे॒व । \newline
28. यदे॒वैव यद् यदे॒वास्या॑ अस्या ए॒व यद् यदे॒वास्याः᳚ । \newline
29. ए॒वास्या॑ अस्या ए॒वैवास्या॑ अमे॒द्ध्य म॑मे॒द्ध्य म॑स्या ए॒वैवास्या॑ अमे॒द्ध्यम् । \newline
30. अ॒स्या॒ अ॒मे॒द्ध्य म॑मे॒द्ध्य म॑स्या अस्या अमे॒द्ध्यम् तत् तद॑मे॒द्ध्य म॑स्या अस्या अमे॒द्ध्यम् तत् । \newline
31. अ॒मे॒द्ध्यम् तत् तद॑मे॒द्ध्य म॑मे॒द्ध्यम् तदपाप॒ तद॑मे॒द्ध्य म॑मे॒द्ध्यम् तदप॑ । \newline
32. तदपाप॒ तत् तदप॑ हन्ति ह॒न्त्यप॒ तत् तदप॑ हन्ति । \newline
33. अप॑ हन्ति ह॒न्त्यपाप॑ ह॒न्त्यु दुद्ध॒ न्त्यपाप॑ ह॒न्त्युत् । \newline
34. ह॒न्त्यु दुद्ध॑न्ति ह॒न्त्यु द्ध॑न्ति ह॒न्त्यु द्ध॑न्ति ह॒न्त्यु द्ध॑न्ति । \newline
35. उद्ध॑न्ति ह॒न्त्यु दुद्ध॑न्ति॒ तस्मा॒त् तस्मा᳚ द्ध॒न्त्यु दुद्ध॑न्ति॒ तस्मा᳚त् । \newline
36. ह॒न्ति॒ तस्मा॒त् तस्मा᳚ द्धन्ति हन्ति॒ तस्मा॒ दोष॑धय॒ ओष॑धय॒ स्तस्मा᳚ द्धन्ति हन्ति॒ तस्मा॒ दोष॑धयः । \newline
37. तस्मा॒ दोष॑धय॒ ओष॑धय॒ स्तस्मा॒त् तस्मा॒ दोष॑धयः॒ परा॒ परौष॑धय॒ स्तस्मा॒त् तस्मा॒ दोष॑धयः॒ परा᳚ । \newline
38. ओष॑धयः॒ परा॒ परौष॑धय॒ ओष॑धयः॒ परा॑ भवन्ति भवन्ति॒ परौष॑धय॒ ओष॑धयः॒ परा॑ भवन्ति । \newline
39. परा॑ भवन्ति भवन्ति॒ परा॒ परा॑ भवन्ति॒ मूल॒म् मूल॑म् भवन्ति॒ परा॒ परा॑ भवन्ति॒ मूल᳚म् । \newline
40. भ॒व॒न्ति॒ मूल॒म् मूल॑म् भवन्ति भवन्ति॒ मूल॑म् छिनत्ति छिनत्ति॒ मूल॑म् भवन्ति भवन्ति॒ मूल॑म् छिनत्ति । \newline
41. मूल॑म् छिनत्ति छिनत्ति॒ मूल॒म् मूल॑म् छिनत्ति॒ भ्रातृ॑व्यस्य॒ भ्रातृ॑व्यस्य छिनत्ति॒ मूल॒म् मूल॑म् छिनत्ति॒ भ्रातृ॑व्यस्य । \newline
42. छि॒न॒त्ति॒ भ्रातृ॑व्यस्य॒ भ्रातृ॑व्यस्य छिनत्ति छिनत्ति॒ भ्रातृ॑व्यस्यै॒वैव भ्रातृ॑व्यस्य छिनत्ति छिनत्ति॒ भ्रातृ॑व्यस्यै॒व । \newline
43. भ्रातृ॑व्यस्यै॒वैव भ्रातृ॑व्यस्य॒ भ्रातृ॑व्यस्यै॒व मूल॒म् मूल॑ मे॒व भ्रातृ॑व्यस्य॒ भ्रातृ॑व्यस्यै॒व मूल᳚म् । \newline
44. ए॒व मूल॒म् मूल॑ मे॒वैव मूल॑म् छिनत्ति छिनत्ति॒ मूल॑ मे॒वैव मूल॑म् छिनत्ति । \newline
45. मूल॑म् छिनत्ति छिनत्ति॒ मूल॒म् मूल॑म् छिनत्ति पितृदेव॒त्या॑ पितृदेव॒त्या॑ छिनत्ति॒ मूल॒म् मूल॑म् छिनत्ति पितृदेव॒त्या᳚ । \newline
46. छि॒न॒त्ति॒ पि॒तृ॒दे॒व॒त्या॑ पितृदेव॒त्या॑ छिनत्ति छिनत्ति पितृदेव॒त्या ऽति॑खा॒ता ऽति॑खाता पितृदेव॒त्या॑ छिनत्ति छिनत्ति 
पितृदेव॒त्या ऽति॑खाता । \newline
47. पि॒तृ॒दे॒व॒त्या ऽति॑खा॒ता ऽति॑खाता पितृदेव॒त्या॑ पितृदेव॒त्या ऽति॑खा॒ तेय॑ती॒ मिय॑ती॒ मति॑खाता पितृदेव॒त्या॑ 
पितृदेव॒त्या ऽति॑खा॒ तेय॑तीम् । \newline
48. पि॒तृ॒दे॒व॒त्येति॑ पितृ - दे॒व॒त्या᳚ । \newline
49. अति॑खा॒ते य॑ती॒ मिय॑ती॒ मति॑खा॒ता ऽति॑खा॒ तेय॑तीम् खनति खन॒तीय॑ती॒ मति॑खा॒ता ऽति॑खा॒ते य॑तीम् खनति । \newline
50. अति॑खा॒तेत्यति॑ - खा॒ता॒ । \newline
51. इय॑तीम् खनति खन॒तीय॑ती॒ मिय॑तीम् खनति प्र॒जाप॑तिना प्र॒जाप॑तिना खन॒तीय॑ती॒ मिय॑तीम् खनति प्र॒जाप॑तिना । \newline
52. ख॒न॒ति॒ प्र॒जाप॑तिना प्र॒जाप॑तिना खनति खनति प्र॒जाप॑तिना यज्ञ्मु॒खेन॑ यज्ञ्मु॒खेन॑ प्र॒जाप॑तिना खनति खनति प्र॒जाप॑तिना यज्ञ्मु॒खेन॑ । \newline
53. प्र॒जाप॑तिना यज्ञ्मु॒खेन॑ यज्ञ्मु॒खेन॑ प्र॒जाप॑तिना प्र॒जाप॑तिना यज्ञ्मु॒खेन॒ सम्मि॑ताꣳ॒॒ सम्मि॑तां ॅयज्ञ्मु॒खेन॑ प्र॒जाप॑तिना प्र॒जाप॑तिना यज्ञ्मु॒खेन॒ सम्मि॑ताम् । \newline
54. प्र॒जाप॑ति॒नेति॑ प्र॒जा - प॒ति॒ना॒ । \newline
\pagebreak
\markright{ TS 2.6.4.3  \hfill https://www.vedavms.in \hfill}
\addcontentsline{toc}{section}{ TS 2.6.4.3 }
\section*{ TS 2.6.4.3 }

\textbf{TS 2.6.4.3 } \newline
\textbf{Samhita Paata} \newline

यज्ञ्मु॒खेन॒ संमि॑ता॒मा प्र॑ति॒ष्ठायै॑ खनति॒ यज॑मानमे॒व प्र॑ति॒ष्ठां ग॑मयति दक्षिण॒तो वर्.षी॑यसीं करोति देव॒यज॑नस्यै॒व रू॒पम॑कः॒ पुरी॑षवतीं करोति प्र॒जावै प॒शवः॒ पुरी॑षं प्र॒जयै॒वैनं॑ प॒शुभिः॒ पुरी॑षवन्तं करो॒त्युत्त॑रं परिग्रा॒हं परि॑ गृह्णात्ये॒ताव॑ती॒ वै पृ॑थि॒वी याव॑ती॒ वेदि॒स्तस्या॑ ए॒ताव॑त ए॒व भ्रातृ॑व्यं नि॒र्भज्या॒ऽऽत्मन॒ उत्त॑रं परिग्रा॒हं परि॑गृह्णाति क्रू॒रमि॑व॒ वा - [  ] \newline

\textbf{Pada Paata} \newline

य॒ज्ञ्॒मु॒खेनेति॑ यज्ञ् - मु॒खेन॑ । संमि॑ता॒मिति॒ सं - मि॒ता॒म् । एति॑ । प्र॒ति॒ष्ठाया॒ इति॑ प्रति - स्थायै᳚ । ख॒न॒ति॒ । यज॑मानम् । ए॒व । प्र॒ति॒ष्ठामिति॑ प्रति - स्थाम् । ग॒म॒य॒ति॒ । द॒क्षि॒ण॒तः । वर्.षी॑यसीम् । क॒रो॒ति॒ । दे॒व॒यज॑न॒स्येति॑ देव - यज॑नस्य । ए॒व । रू॒पम् । अ॒कः॒ । पुरी॑षवती॒मिति॒ पुरी॑ष - व॒ती॒म् । क॒रो॒ति॒ । प्र॒जेति॑ प्र - जा । वै । प॒शवः॑ । पुरी॑षम् । प्र॒जयेति॑ प्र - जया᳚ । ए॒व । ए॒न॒म् । प॒शुभि॒रिति॑ प॒शु - भिः॒ । पुरी॑षवन्त॒मिति॒ पुरी॑ष - व॒न्त॒म् । क॒रो॒ति॒ । उत्त॑र॒मित्युत् - त॒र॒म् । प॒रि॒ग्रा॒हमिति॑ परि-ग्रा॒हम् । परीति॑ । गृ॒ह्णा॒ति॒ । ए॒ताव॑ती । वै । पृ॒थि॒वी । याव॑ती । वेदिः॑ । तस्याः᳚ । ए॒ताव॑तः । ए॒व । भ्रातृ॑व्यम् । नि॒र्भज्येति॑ निः - भज्य॑ । आ॒त्मने᳚ । उत्त॑र॒मित्युत् - त॒र॒म् । प॒रि॒ग्रा॒हमिति॑ परि-ग्रा॒हम् । परीति॑ । गृ॒ह्णा॒ति॒ । क्रू॒रम् । इ॒व॒ । वै ।  \newline


\textbf{Krama Paata} \newline

य॒ज्ञ्॒मु॒खेन॒ संमि॑ताम् । य॒ज्ञ्॒मु॒खेनेति॑ यज्ञ् - मु॒खेन॑ । संमि॑ता॒मा । संमि॑ता॒मिति॒ सं - मि॒ता॒म् । आ प्र॑ति॒ष्ठायै᳚ । प्र॒ति॒ष्ठायै॑ खनति । प्र॒ति॒ष्ठाया॒ इति॑ प्रति - स्थायै᳚ । ख॒न॒ति॒ यज॑मानम् । यज॑मानमे॒व । ए॒व प्र॑ति॒ष्ठाम् । प्र॒ति॒ष्ठाम् ग॑मयति । प्र॒ति॒ष्ठामिति॑ प्रति - स्थाम् । ग॒म॒य॒ति॒ द॒क्षि॒ण॒तः । द॒क्षि॒ण॒तो वर्.षी॑यसीम् । वर्.षी॑यसीम् करोति । क॒रो॒ति॒ दे॒व॒यज॑नस्य । दे॒व॒यज॑नस्यै॒व । दे॒व॒यज॑न॒स्येति॑ देव - यज॑नस्य । ए॒व रू॒पम् । रू॒पम॑कः । अ॒कः॒ पुरी॑षवतीम् । पुरी॑षवतीम् करोति । पुरी॑षवती॒मिति॒ पुरी॑ष - व॒ती॒म् । क॒रो॒ति॒ प्र॒जा । प्र॒जा वै । प्र॒जेति॑ प्र - जा । वै प॒शवः॑ । प॒शवः॒ पुरी॑षम् । पुरी॑षम् प्र॒जया᳚ । प्र॒जयै॒व । प्र॒जयेति॑ प्र - जया᳚ । ए॒वैन᳚म् । ए॒न॒म् प॒शुभिः॑ । प॒शुभिः॒ पुरी॑षवन्तम् । प॒शुभि॒रिति॑ प॒शु - भिः॒ । पुरी॑षवन्तं करोति । पुरी॑षवन्त॒मिति॒ पुरी॑ष - व॒न्त॒म् । क॒रो॒त्युत्त॑रम् । उत्त॑रम् परिग्रा॒हम् । उत्त॑र॒मित्युत् - त॒र॒म् । प॒रि॒ग्रा॒हम् परि॑ । प॒रि॒ग्रा॒हमिति॑ परि - ग्रा॒हम् । परि॑ गृह्णाति । गृ॒ह्णा॒त्ये॒ताव॑ती । ए॒ताव॑ती॒ वै । वै पृ॑थि॒वी । पृ॒थि॒वी याव॑ती । याव॑ती॒ वेदिः॑ । वेदि॒स्तस्याः᳚ । तस्या॑ ए॒ताव॑तः । ए॒ताव॑त ए॒व । ए॒व भ्रातृ॑व्यम् । भ्रातृ॑व्यम् नि॒र्भज्य॑ । नि॒र्भज्या॒त्मने᳚ । नि॒र्भज्येति॑ निः - भज्य॑ । आ॒त्मन॒ उत्त॑रम् । उत्त॑रम् परिग्रा॒हम् । उत्त॑र॒मित्युत् - त॒र॒म् । प॒रि॒ग्रा॒हम् परि॑ । प॒रि॒ग्रा॒हमिति॑ परि - ग्रा॒हम् । परि॑ गृह्णाति । गृ॒ह्णा॒ति॒ क्रू॒रम् । क्रू॒रमि॑व । इ॒व॒ वै ( ) । वा ए॒तत् \newline

\textbf{Jatai Paata} \newline

1. य॒ज्ञ्॒मु॒खेन॒ सम्मि॑ताꣳ॒॒ सम्मि॑तां ॅयज्ञ्मु॒खेन॑ यज्ञ्मु॒खेन॒ सम्मि॑ताम् । \newline
2. य॒ज्ञ्॒मु॒खेनेति॑ यज्ञ् - मु॒खेन॑ । \newline
3. सम्मि॑ता॒ मा सम्मि॑ताꣳ॒॒ सम्मि॑ता॒ मा । \newline
4. सम्मि॑ता॒मिति॒ सं - मि॒ता॒म् । \newline
5. आ प्र॑ति॒ष्ठायै᳚ प्रति॒ष्ठाया॒ आ प्र॑ति॒ष्ठायै᳚ । \newline
6. प्र॒ति॒ष्ठायै॑ खनति खनति प्रति॒ष्ठायै᳚ प्रति॒ष्ठायै॑ खनति । \newline
7. प्र॒ति॒ष्ठाया॒ इति॑ प्रति - स्थायै᳚ । \newline
8. ख॒न॒ति॒ यज॑मानं॒ ॅयज॑मानम् खनति खनति॒ यज॑मानम् । \newline
9. यज॑मान मे॒वैव यज॑मानं॒ ॅयज॑मान मे॒व । \newline
10. ए॒व प्र॑ति॒ष्ठाम् प्र॑ति॒ष्ठा मे॒वैव प्र॑ति॒ष्ठाम् । \newline
11. प्र॒ति॒ष्ठाम् ग॑मयति गमयति प्रति॒ष्ठाम् प्र॑ति॒ष्ठाम् ग॑मयति । \newline
12. प्र॒ति॒ष्ठामिति॑ प्रति - स्थाम् । \newline
13. ग॒म॒य॒ति॒ द॒क्षि॒ण॒तो द॑क्षिण॒तो ग॑मयति गमयति दक्षिण॒तः । \newline
14. द॒क्षि॒ण॒तो वर्.षी॑यसीं॒ ॅवर्.षी॑यसीम् दक्षिण॒तो द॑क्षिण॒तो वर्.षी॑यसीम् । \newline
15. वर्.षी॑यसीम् करोति करोति॒ वर्.षी॑यसीं॒ ॅवर्.षी॑यसीम् करोति । \newline
16. क॒रो॒ति॒ दे॒व॒यज॑नस्य देव॒यज॑नस्य करोति करोति देव॒यज॑नस्य । \newline
17. दे॒व॒यज॑न स्यै॒वैव दे॑व॒यज॑नस्य देव॒यज॑न स्यै॒व । \newline
18. दे॒व॒यज॑न॒स्येति॑ देव - यज॑नस्य । \newline
19. ए॒व रू॒पꣳ रू॒प मे॒वैव रू॒पम् । \newline
20. रू॒प म॑क रका रू॒पꣳ रू॒प म॑कः । \newline
21. अ॒कः॒ पुरी॑षवती॒म् पुरी॑षवती मक रकः॒ पुरी॑षवतीम् । \newline
22. पुरी॑षवतीम् करोति करोति॒ पुरी॑षवती॒म् पुरी॑षवतीम् करोति । \newline
23. पुरी॑षवती॒मिति॒ पुरी॑ष - व॒ती॒म् । \newline
24. क॒रो॒ति॒ प्र॒जा प्र॒जा क॑रोति करोति प्र॒जा । \newline
25. प्र॒जा वै वै प्र॒जा प्र॒जा वै । \newline
26. प्र॒जेति॑ प्र - जा । \newline
27. वै प॒शवः॑ प॒शवो॒ वै वै प॒शवः॑ । \newline
28. प॒शवः॒ पुरी॑ष॒म् पुरी॑षम् प॒शवः॑ प॒शवः॒ पुरी॑षम् । \newline
29. पुरी॑षम् प्र॒जया᳚ प्र॒जया॒ पुरी॑ष॒म् पुरी॑षम् प्र॒जया᳚ । \newline
30. प्र॒जयै॒वैव प्र॒जया᳚ प्र॒जयै॒व । \newline
31. प्र॒जयेति॑ प्र - जया᳚ । \newline
32. ए॒वैन॑ मेन मे॒वैवैन᳚म् । \newline
33. ए॒न॒म् प॒शुभिः॑ प॒शुभि॑ रेन मेनम् प॒शुभिः॑ । \newline
34. प॒शुभिः॒ पुरी॑षवन्त॒म् पुरी॑षवन्तम् प॒शुभिः॑ प॒शुभिः॒ पुरी॑षवन्तम् । \newline
35. प॒शुभि॒रिति॑ प॒शु - भिः॒ । \newline
36. पुरी॑षवन्तम् करोति करोति॒ पुरी॑षवन्त॒म् पुरी॑षवन्तम् करोति । \newline
37. पुरी॑षवन्त॒मिति॒ पुरी॑ष - व॒न्त॒म् । \newline
38. क॒रो॒ त्युत्त॑र॒ मुत्त॑रम् करोति करो॒ त्युत्त॑रम् । \newline
39. उत्त॑रम् परिग्रा॒हम् प॑रिग्रा॒ह मुत्त॑र॒ मुत्त॑रम् परिग्रा॒हम् । \newline
40. उत्त॑र॒मित्युत् - त॒र॒म् । \newline
41. प॒रि॒ग्रा॒हम् परि॒ परि॑ परिग्रा॒हम् प॑रिग्रा॒हम् परि॑ । \newline
42. प॒रि॒ग्रा॒हमिति॑ परि - ग्रा॒हम् । \newline
43. परि॑ गृह्णाति गृह्णाति॒ परि॒ परि॑ गृह्णाति । \newline
44. गृ॒ह्णा॒ त्ये॒ताव॑ त्ये॒ताव॑ती गृह्णाति गृह्णा त्ये॒ताव॑ती । \newline
45. ए॒ताव॑ती॒ वै वा ए॒ताव॑ त्ये॒ताव॑ती॒ वै । \newline
46. वै पृ॑थि॒वी पृ॑थि॒वी वै वै पृ॑थि॒वी । \newline
47. पृ॒थि॒वी याव॑ती॒ याव॑ती पृथि॒वी पृ॑थि॒वी याव॑ती । \newline
48. याव॑ती॒ वेदि॒र् वेदि॒र् याव॑ती॒ याव॑ती॒ वेदिः॑ । \newline
49. वेदि॒ स्तस्या॒ स्तस्या॒ वेदि॒र् वेदि॒ स्तस्याः᳚ । \newline
50. तस्या॑ ए॒ताव॑त ए॒ताव॑त॒ स्तस्या॒ स्तस्या॑ ए॒ताव॑तः । \newline
51. ए॒ताव॑त ए॒वै वैताव॑त ए॒ताव॑त ए॒व । \newline
52. ए॒व भ्रातृ॑व्य॒म् भ्रातृ॑व्य मे॒वैव भ्रातृ॑व्यम् । \newline
53. भ्रातृ॑व्यम् नि॒र्भज्य॑ नि॒र्भज्य॒ भ्रातृ॑व्य॒म् भ्रातृ॑व्यम् नि॒र्भज्य॑ । \newline
54. नि॒र्भज्या॒ त्मन॑ आ॒त्मने॑ नि॒र्भज्य॑ नि॒र्भज्या॒ त्मने᳚ । \newline
55. नि॒र्भज्येति॑ निः - भज्य॑ । \newline
56. आ॒त्मन॒ उत्त॑र॒ मुत्त॑र मा॒त्मन॑ आ॒त्मन॒ उत्त॑रम् । \newline
57. उत्त॑रम् परिग्रा॒हम् प॑रिग्रा॒ह मुत्त॑र॒ मुत्त॑रम् परिग्रा॒हम् । \newline
58. उत्त॑र॒मित्युत् - त॒र॒म् । \newline
59. प॒रि॒ग्रा॒हम् परि॒ परि॑ परिग्रा॒हम् प॑रिग्रा॒हम् परि॑ । \newline
60. प॒रि॒ग्रा॒हमिति॑ परि - ग्रा॒हम् । \newline
61. परि॑ गृह्णाति गृह्णाति॒ परि॒ परि॑ गृह्णाति । \newline
62. गृ॒ह्णा॒ति॒ क्रू॒रम् क्रू॒रम् गृ॑ह्णाति गृह्णाति क्रू॒रम् । \newline
63. क्रू॒र मि॑वे व क्रू॒रम् क्रू॒र मि॑व । \newline
64. इ॒व॒ वै वा इ॑वे व॒ वै । \newline
65. वा ए॒त दे॒तद् वै वा ए॒तत् । \newline

\textbf{Ghana Paata } \newline

1. य॒ज्ञ्॒मु॒खेन॒ सम्मि॑ताꣳ॒॒ सम्मि॑तां ॅयज्ञ्मु॒खेन॑ यज्ञ्मु॒खेन॒ सम्मि॑ता॒ मा सम्मि॑तां ॅयज्ञ्मु॒खेन॑ यज्ञ्मु॒खेन॒ सम्मि॑ता॒ मा । \newline
2. य॒ज्ञ्॒मु॒खेनेति॑ यज्ञ् - मु॒खेन॑ । \newline
3. सम्मि॑ता॒ मा सम्मि॑ताꣳ॒॒ सम्मि॑ता॒ मा प्र॑ति॒ष्ठायै᳚ प्रति॒ष्ठाया॒ आ सम्मि॑ताꣳ॒॒ सम्मि॑ता॒ मा प्र॑ति॒ष्ठायै᳚ । \newline
4. सम्मि॑ता॒मिति॒ सं - मि॒ता॒म् । \newline
5. आ प्र॑ति॒ष्ठायै᳚ प्रति॒ष्ठाया॒ आ प्र॑ति॒ष्ठायै॑ खनति खनति प्रति॒ष्ठाया॒ आ प्र॑ति॒ष्ठायै॑ खनति । \newline
6. प्र॒ति॒ष्ठायै॑ खनति खनति प्रति॒ष्ठायै᳚ प्रति॒ष्ठायै॑ खनति॒ यज॑मानं॒ ॅयज॑मानम् खनति प्रति॒ष्ठायै᳚ प्रति॒ष्ठायै॑ खनति॒ यज॑मानम् । \newline
7. प्र॒ति॒ष्ठाया॒ इति॑ प्रति - स्थायै᳚ । \newline
8. ख॒न॒ति॒ यज॑मानं॒ ॅयज॑मानम् खनति खनति॒ यज॑मान मे॒वैव यज॑मानम् खनति खनति॒ यज॑मान मे॒व । \newline
9. यज॑मान मे॒वैव यज॑मानं॒ ॅयज॑मान मे॒व प्र॑ति॒ष्ठाम् प्र॑ति॒ष्ठा मे॒व यज॑मानं॒ ॅयज॑मान मे॒व प्र॑ति॒ष्ठाम् । \newline
10. ए॒व प्र॑ति॒ष्ठाम् प्र॑ति॒ष्ठा मे॒वैव प्र॑ति॒ष्ठाम् ग॑मयति गमयति प्रति॒ष्ठा मे॒वैव प्र॑ति॒ष्ठाम् ग॑मयति । \newline
11. प्र॒ति॒ष्ठाम् ग॑मयति गमयति प्रति॒ष्ठाम् प्र॑ति॒ष्ठाम् ग॑मयति दक्षिण॒तो द॑क्षिण॒तो ग॑मयति प्रति॒ष्ठाम् प्र॑ति॒ष्ठाम् ग॑मयति दक्षिण॒तः । \newline
12. प्र॒ति॒ष्ठामिति॑ प्रति - स्थाम् । \newline
13. ग॒म॒य॒ति॒ द॒क्षि॒ण॒तो द॑क्षिण॒तो ग॑मयति गमयति दक्षिण॒तो वर्.षी॑यसीं॒ ॅवर्.षी॑यसीम् दक्षिण॒तो ग॑मयति गमयति दक्षिण॒तो वर्.षी॑यसीम् । \newline
14. द॒क्षि॒ण॒तो वर्.षी॑यसीं॒ ॅवर्.षी॑यसीम् दक्षिण॒तो द॑क्षिण॒तो वर्.षी॑यसीम् करोति करोति॒ वर्.षी॑यसीम् दक्षिण॒तो द॑क्षिण॒तो वर्.षी॑यसीम् करोति । \newline
15. वर्.षी॑यसीम् करोति करोति॒ वर्.षी॑यसीं॒ ॅवर्.षी॑यसीम् करोति देव॒यज॑नस्य देव॒यज॑नस्य करोति॒ वर्.षी॑यसीं॒ ॅवर्.षी॑यसीम् करोति देव॒यज॑नस्य । \newline
16. क॒रो॒ति॒ दे॒व॒यज॑नस्य देव॒यज॑नस्य करोति करोति देव॒यज॑न स्यै॒वैव दे॑व॒यज॑नस्य करोति करोति देव॒यज॑न स्यै॒व । \newline
17. दे॒व॒यज॑न स्यै॒वैव दे॑व॒यज॑नस्य देव॒यज॑न स्यै॒व रू॒पꣳ रू॒प मे॒व दे॑व॒यज॑नस्य देव॒यज॑न स्यै॒व रू॒पम् । \newline
18. दे॒व॒यज॑न॒स्येति॑ देव - यज॑नस्य । \newline
19. ए॒व रू॒पꣳ रू॒प मे॒वैव रू॒प म॑क रका रू॒प मे॒वैव रू॒प म॑कः । \newline
20. रू॒प म॑क रका रू॒पꣳ रू॒प म॑कः॒ पुरी॑षवती॒म् पुरी॑षवती मका रू॒पꣳ रू॒प म॑कः॒ पुरी॑षवतीम् । \newline
21. अ॒कः॒ पुरी॑षवती॒म् पुरी॑षवती मक रकः॒ पुरी॑षवतीम् करोति करोति॒ पुरी॑षवती मक रकः॒ पुरी॑षवतीम् करोति । \newline
22. पुरी॑षवतीम् करोति करोति॒ पुरी॑षवती॒म् पुरी॑षवतीम् करोति प्र॒जा प्र॒जा क॑रोति॒ पुरी॑षवती॒म् पुरी॑षवतीम् करोति प्र॒जा । \newline
23. पुरी॑षवती॒मिति॒ पुरी॑ष - व॒ती॒म् । \newline
24. क॒रो॒ति॒ प्र॒जा प्र॒जा क॑रोति करोति प्र॒जा वै वै प्र॒जा क॑रोति करोति प्र॒जा वै । \newline
25. प्र॒जा वै वै प्र॒जा प्र॒जा वै प॒शवः॑ प॒शवो॒ वै प्र॒जा प्र॒जा वै प॒शवः॑ । \newline
26. प्र॒जेति॑ प्र - जा । \newline
27. वै प॒शवः॑ प॒शवो॒ वै वै प॒शवः॒ पुरी॑ष॒म् पुरी॑षम् प॒शवो॒ वै वै प॒शवः॒ पुरी॑षम् । \newline
28. प॒शवः॒ पुरी॑ष॒म् पुरी॑षम् प॒शवः॑ प॒शवः॒ पुरी॑षम् प्र॒जया᳚ प्र॒जया॒ पुरी॑षम् प॒शवः॑ प॒शवः॒ पुरी॑षम् प्र॒जया᳚ । \newline
29. पुरी॑षम् प्र॒जया᳚ प्र॒जया॒ पुरी॑ष॒म् पुरी॑षम् प्र॒जयै॒वैव प्र॒जया॒ पुरी॑ष॒म् पुरी॑षम् प्र॒जयै॒व । \newline
30. प्र॒जयै॒वैव प्र॒जया᳚ प्र॒जयै॒वैन॑ मेन मे॒व प्र॒जया᳚ प्र॒जयै॒वैन᳚म् । \newline
31. प्र॒जयेति॑ प्र - जया᳚ । \newline
32. ए॒वैन॑ मेन मे॒वैवैन॑म् प॒शुभिः॑ प॒शुभि॑रेन मे॒वैवैन॑म् प॒शुभिः॑ । \newline
33. ए॒न॒म् प॒शुभिः॑ प॒शुभि॑रेन मेनम् प॒शुभिः॒ पुरी॑षवन्त॒म् पुरी॑षवन्तम् प॒शुभि॑रेन मेनम् प॒शुभिः॒ पुरी॑षवन्तम् । \newline
34. प॒शुभिः॒ पुरी॑षवन्त॒म् पुरी॑षवन्तम् प॒शुभिः॑ प॒शुभिः॒ पुरी॑षवन्तम् करोति करोति॒ पुरी॑षवन्तम् प॒शुभिः॑ प॒शुभिः॒ पुरी॑षवन्तम् करोति । \newline
35. प॒शुभि॒रिति॑ प॒शु - भिः॒ । \newline
36. पुरी॑षवन्तम् करोति करोति॒ पुरी॑षवन्त॒म् पुरी॑षवन्तम् करो॒त्युत्त॑र॒ मुत्त॑रम् करोति॒ पुरी॑षवन्त॒म् पुरी॑षवन्तम् करो॒त्युत्त॑रम् । \newline
37. पुरी॑षवन्त॒मिति॒ पुरी॑ष - व॒न्त॒म् । \newline
38. क॒रो॒त्युत्त॑र॒ मुत्त॑रम् करोति करो॒त्युत्त॑रम् परिग्रा॒हम् प॑रिग्रा॒ह मुत्त॑रम् करोति करो॒त्युत्त॑रम् परिग्रा॒हम् । \newline
39. उत्त॑रम् परिग्रा॒हम् प॑रिग्रा॒ह मुत्त॑र॒ मुत्त॑रम् परिग्रा॒हम् परि॒ परि॑ परिग्रा॒ह मुत्त॑र॒ मुत्त॑रम् परिग्रा॒हम् परि॑ । \newline
40. उत्त॑र॒मित्युत् - त॒र॒म् । \newline
41. प॒रि॒ग्रा॒हम् परि॒ परि॑ परिग्रा॒हम् प॑रिग्रा॒हम् परि॑ गृह्णाति गृह्णाति॒ परि॑ परिग्रा॒हम् प॑रिग्रा॒हम् परि॑ गृह्णाति । \newline
42. प॒रि॒ग्रा॒हमिति॑ परि - ग्रा॒हम् । \newline
43. परि॑ गृह्णाति गृह्णाति॒ परि॒ परि॑ गृह्णा त्ये॒ताव॑ त्ये॒ताव॑ती गृह्णाति॒ परि॒ परि॑ गृह्णा त्ये॒ताव॑ती । \newline
44. गृ॒ह्णा॒ त्ये॒ताव॑ त्ये॒ताव॑ती गृह्णाति गृह्णा त्ये॒ताव॑ती॒ वै वा ए॒ताव॑ती गृह्णाति गृह्णा त्ये॒ताव॑ती॒ वै । \newline
45. ए॒ताव॑ती॒ वै वा ए॒ताव॑ त्ये॒ताव॑ती॒ वै पृ॑थि॒वी पृ॑थि॒वी वा ए॒ताव॑ त्ये॒ताव॑ती॒ वै पृ॑थि॒वी । \newline
46. वै पृ॑थि॒वी पृ॑थि॒वी वै वै पृ॑थि॒वी याव॑ती॒ याव॑ती पृथि॒वी वै वै पृ॑थि॒वी याव॑ती । \newline
47. पृ॒थि॒वी याव॑ती॒ याव॑ती पृथि॒वी पृ॑थि॒वी याव॑ती॒ वेदि॒र् वेदि॒र् याव॑ती पृथि॒वी पृ॑थि॒वी याव॑ती॒ वेदिः॑ । \newline
48. याव॑ती॒ वेदि॒र् वेदि॒र् याव॑ती॒ याव॑ती॒ वेदि॒ स्तस्या॒ स्तस्या॒ वेदि॒र् याव॑ती॒ याव॑ती॒ वेदि॒ स्तस्याः᳚ । \newline
49. वेदि॒ स्तस्या॒ स्तस्या॒ वेदि॒र् वेदि॒ स्तस्या॑ ए॒ताव॑त ए॒ताव॑त॒ स्तस्या॒ वेदि॒र् वेदि॒ स्तस्या॑ ए॒ताव॑तः । \newline
50. तस्या॑ ए॒ताव॑त ए॒ताव॑त॒ स्तस्या॒ स्तस्या॑ ए॒ताव॑त ए॒वैवैताव॑त॒ स्तस्या॒ स्तस्या॑ ए॒ताव॑त ए॒व । \newline
51. ए॒ताव॑त ए॒वैवैताव॑त ए॒ताव॑त ए॒व भ्रातृ॑व्य॒म् भ्रातृ॑व्य मे॒वैताव॑त ए॒ताव॑त ए॒व भ्रातृ॑व्यम् । \newline
52. ए॒व भ्रातृ॑व्य॒म् भ्रातृ॑व्य मे॒वैव भ्रातृ॑व्यम् नि॒र्भज्य॑ नि॒र्भज्य॒ भ्रातृ॑व्य मे॒वैव भ्रातृ॑व्यम् नि॒र्भज्य॑ । \newline
53. भ्रातृ॑व्यम् नि॒र्भज्य॑ नि॒र्भज्य॒ भ्रातृ॑व्य॒म् भ्रातृ॑व्यम् नि॒र्भज्या॒त्मन॑ आ॒त्मने॑ नि॒र्भज्य॒ भ्रातृ॑व्य॒म् भ्रातृ॑व्यम् नि॒र्भज्या॒त्मने᳚ । \newline
54. नि॒र्भज्या॒त्मन॑ आ॒त्मने॑ नि॒र्भज्य॑ नि॒र्भज्या॒त्मन॒ उत्त॑र॒ मुत्त॑र मा॒त्मने॑ नि॒र्भज्य॑ नि॒र्भज्या॒त्मन॒ उत्त॑रम् । \newline
55. नि॒र्भज्येति॑ निः - भज्य॑ । \newline
56. आ॒त्मन॒ उत्त॑र॒ मुत्त॑र मा॒त्मन॑ आ॒त्मन॒ उत्त॑रम् परिग्रा॒हम् प॑रिग्रा॒ह मुत्त॑र मा॒त्मन॑ आ॒त्मन॒ उत्त॑रम् परिग्रा॒हम् । \newline
57. उत्त॑रम् परिग्रा॒हम् प॑रिग्रा॒ह मुत्त॑र॒ मुत्त॑रम् परिग्रा॒हम् परि॒ परि॑ परिग्रा॒ह मुत्त॑र॒ मुत्त॑रम् परिग्रा॒हम् परि॑ । \newline
58. उत्त॑र॒मित्युत् - त॒र॒म् । \newline
59. प॒रि॒ग्रा॒हम् परि॒ परि॑ परिग्रा॒हम् प॑रिग्रा॒हम् परि॑ गृह्णाति गृह्णाति॒ परि॑ परिग्रा॒हम् प॑रिग्रा॒हम् परि॑ गृह्णाति । \newline
60. प॒रि॒ग्रा॒हमिति॑ परि - ग्रा॒हम् । \newline
61. परि॑ गृह्णाति गृह्णाति॒ परि॒ परि॑ गृह्णाति क्रू॒रम् क्रू॒रम् गृ॑ह्णाति॒ परि॒ परि॑ गृह्णाति क्रू॒रम् । \newline
62. गृ॒ह्णा॒ति॒ क्रू॒रम् क्रू॒रम् गृ॑ह्णाति गृह्णाति क्रू॒र मि॑वे व क्रू॒रम् गृ॑ह्णाति गृह्णाति क्रू॒र मि॑व । \newline
63. क्रू॒र मि॑वे व क्रू॒रम् क्रू॒र मि॑व॒ वै वा इ॑व क्रू॒रम् क्रू॒र मि॑व॒ वै । \newline
64. इ॒व॒ वै वा इ॑वे व॒ वा ए॒त दे॒तद् वा इ॑वे व॒ वा ए॒तत् । \newline
65. वा ए॒तदे॒तद् वै वा ए॒तत् क॑रोति करो त्ये॒तद् वै वा ए॒तत् क॑रोति । \newline
\pagebreak
\markright{ TS 2.6.4.4  \hfill https://www.vedavms.in \hfill}
\addcontentsline{toc}{section}{ TS 2.6.4.4 }
\section*{ TS 2.6.4.4 }

\textbf{TS 2.6.4.4 } \newline
\textbf{Samhita Paata} \newline

ए॒तत् क॑रोति॒ यद्वेदिं॑ क॒रोति॒ धा अ॑सि स्व॒धा अ॒सीति॑ योयुप्यते॒ शान्त्यै॒ प्रोक्ष॑णी॒रा सा॑दय॒त्यापो॒ वै र॑क्षो॒घ्नी रक्ष॑सा॒मप॑हत्यै॒ स्फ्यस्य॒वर्त्मन्᳚थ सादयति य॒ज्ञ्स्य॒ सन्त॑त्यै॒यं द्वि॒ष्यात् तं \ध्या॑येच्छु॒चैवैन॑मर्पयति ॥ \newline

\textbf{Pada Paata} \newline

ए॒तत् । क॒रो॒ति॒ । यत् । वेदि᳚म् । क॒रोति॑ । धाः । अ॒सि॒ । स्व॒धेति॑ स्व - धा । अ॒सि॒ । इति॑ । यो॒यु॒प्य॒ते॒ । शान्त्यै᳚ । प्रोक्ष॑णी॒रिति॑ प्र - उक्ष॑णीः । एति॑ । सा॒द॒य॒ति॒ । आपः॑ । वै । र॒क्षो॒घ्नीरिति॑ रक्षः - घ्नीः । रक्ष॑साम् । अप॑हत्या॒ इत्यप॑ - ह॒त्यै॒ । स्फ्यस्य॑ । वर्त्मन्न्॑ । सा॒द॒य॒ति॒ । य॒ज्ञ्स्य॑ । संत॑त्या॒ इति॒ सं - त॒त्यै॒ । यम् । द्वि॒ष्यात् । तम् । ध्या॒ये॒त् । शु॒चा । ए॒व । ए॒न॒म् । अ॒र्प॒य॒ति॒ ॥  \newline


\textbf{Krama Paata} \newline

ए॒तत् क॑रोति । क॒रो॒ति॒ यत् । यद् वेदि᳚म् । वेदि॑म् क॒रोति॑ । क॒रोति॒ धाः । धा अ॑सि । अ॒सि॒ स्व॒धा । स्व॒धा अ॑सि । स्व॒धेति॑ स्व - धा । अ॒सीति॑ । इति॑ योयुप्यते । यो॒यु॒प्य॒ते॒ शान्त्यै᳚ । शान्त्यै॒ प्रोक्ष॑णीः । प्रोक्ष॑णी॒रा । प्रोक्ष॑णी॒रिति॑ प्र - उक्ष॑णीः । आ सा॑दयति । सा॒द॒य॒त्यापः॑ । आपो॒ वै । वै र॑क्षो॒घ्नीः । र॒क्षो॒घ्नी रक्ष॑साम् । र॒क्षो॒घ्नीरिति॑ रक्षः - घ्नीः । रक्ष॑सा॒मप॑हत्यै । अप॑हत्यै॒ स्फ्यस्य॑ । अप॑हत्या॒ इत्यप॑ - ह॒त्यै॒ । स्फ्यस्य॒ वर्त्मन्न्॑ । वर्त्म᳚न्थ् सादयति । सा॒द॒य॒ति॒ य॒ज्ञ्स्य॑ । य॒ज्ञ्स्य॒ सन्त॑त्यै । सन्त॑त्यै॒ यम् । सन्त॑त्या॒ इति॒ सं - त॒त्यै॒ । यम् द्वि॒ष्यात् । द्वि॒ष्यात् तम् । तम् ध्या॑येत् । ध्या॒ये॒च्छु॒चा । शु॒चैव । ए॒वैन᳚म् । ए॒न॒म॒र्प॒य॒ति॒ । अ॒र्प॒य॒तीत्य॑र्पयति । \newline

\textbf{Jatai Paata} \newline

1. ए॒तत् क॑रोति करो त्ये॒त दे॒तत् क॑रोति । \newline
2. क॒रो॒ति॒ यद् यत् क॑रोति करोति॒ यत् । \newline
3. यद् वेदिं॒ ॅवेदिं॒ ॅयद् यद् वेदि᳚म् । \newline
4. वेदि॑म् क॒रोति॑ क॒रोति॒ वेदिं॒ ॅवेदि॑म् क॒रोति॑ । \newline
5. क॒रोति॒ धा धाः क॒रोति॑ क॒रोति॒ धाः । \newline
6. धा अ॑स्यसि॒ धा धा अ॑सि । \newline
7. अ॒सि॒ स्व॒धा स्व॒धा ऽस्य॑सि स्व॒धा । \newline
8. स्व॒धा अ॑स्यसि स्व॒धा स्व॒धा अ॑सि । \newline
9. स्व॒धेति॑ स्व - धा । \newline
10. अ॒सीतीत्य॑स्य॒सीति॑ । \newline
11. इति॑ योयुप्यते योयुप्यत॒ इतीति॑ योयुप्यते । \newline
12. यो॒यु॒प्य॒ते॒ शान्त्यै॒ शान्त्यै॑ योयुप्यते योयुप्यते॒ शान्त्यै᳚ । \newline
13. शान्त्यै॒ प्रोक्ष॑णीः॒ प्रोक्ष॑णीः॒ शान्त्यै॒ शान्त्यै॒ प्रोक्ष॑णीः । \newline
14. प्रोक्ष॑णी॒रा प्रोक्ष॑णीः॒ प्रोक्ष॑णी॒रा । \newline
15. प्रोक्ष॑णी॒रिति॑ प्र - उक्ष॑णीः । \newline
16. आ सा॑दयति सादय॒त्या सा॑दयति । \newline
17. सा॒द॒य॒ त्याप॒ आपः॑ सादयति सादय॒ त्यापः॑ । \newline
18. आपो॒ वै वा आप॒ आपो॒ वै । \newline
19. वै र॑क्षो॒घ्नी र॑क्षो॒घ्नीर् वै वै र॑क्षो॒घ्नीः । \newline
20. र॒क्षो॒घ्नी रक्ष॑साꣳ॒॒ रक्ष॑साꣳ रक्षो॒घ्नी र॑क्षो॒घ्नी रक्ष॑साम् । \newline
21. र॒क्षो॒घ्नीरिति॑ रक्षः - घ्नीः । \newline
22. रक्ष॑सा॒ मप॑हत्या॒ अप॑हत्यै॒ रक्ष॑साꣳ॒॒ रक्ष॑सा॒ मप॑हत्यै । \newline
23. अप॑हत्यै॒ स्फ्यस्य॒ स्फ्यस्या प॑हत्या॒ अप॑हत्यै॒ स्फ्यस्य॑ । \newline
24. अप॑हत्या॒ इत्यप॑ - ह॒त्यै॒ । \newline
25. स्फ्यस्य॒ वर्त्म॒न्॒. वर्त्म॒न् थ्स्फ्यस्य॒ स्फ्यस्य॒ वर्त्मन्न्॑ । \newline
26. वर्त्मन्᳚ थ्सादयति सादयति॒ वर्त्म॒न्॒. वर्त्मन्᳚ थ्सादयति । \newline
27. सा॒द॒य॒ति॒ य॒ज्ञ्स्य॑ य॒ज्ञ्स्य॑ सादयति सादयति य॒ज्ञ्स्य॑ । \newline
28. य॒ज्ञ्स्य॒ सन्त॑त्यै॒ सन्त॑त्यै य॒ज्ञ्स्य॑ य॒ज्ञ्स्य॒ सन्त॑त्यै । \newline
29. सन्त॑त्यै॒ यं ॅयꣳ सन्त॑त्यै॒ सन्त॑त्यै॒ यम् । \newline
30. सन्त॑त्या॒ इति॒ सं - त॒त्यै॒ । \newline
31. यम् द्वि॒ष्याद् द्वि॒ष्याद् यं ॅयम् द्वि॒ष्यात् । \newline
32. द्वि॒ष्यात् तम् तम् द्वि॒ष्याद् द्वि॒ष्यात् तम् । \newline
33. तम् ध्या॑येद् ध्याये॒त् तम् तम् ध्या॑येत् । \newline
34. ध्या॒ये॒च् छु॒चा शु॒चा ध्या॑येद् ध्यायेच् छु॒चा । \newline
35. शु॒चैवैव शु॒चा शु॒चैव । \newline
36. ए॒वैन॑ मेन मे॒वैवैन᳚म् । \newline
37. ए॒न॒ म॒र्प॒य॒ त्य॒र्प॒य॒ त्ये॒न॒ मे॒न॒ म॒र्प॒य॒ति॒ । \newline
38. अ॒र्प॒य॒तीत्य॑र्पयति । \newline

\textbf{Ghana Paata } \newline

1. ए॒तत् क॑रोति करो त्ये॒त दे॒तत् क॑रोति॒ यद् यत् क॑रो त्ये॒त दे॒तत् क॑रोति॒ यत् । \newline
2. क॒रो॒ति॒ यद् यत् क॑रोति करोति॒ यद् वेदिं॒ ॅवेदिं॒ ॅयत् क॑रोति करोति॒ यद् वेदि᳚म् । \newline
3. यद् वेदिं॒ ॅवेदिं॒ ॅयद् यद् वेदि॑म् क॒रोति॑ क॒रोति॒ वेदिं॒ ॅयद् यद् वेदि॑म् क॒रोति॑ । \newline
4. वेदि॑म् क॒रोति॑ क॒रोति॒ वेदिं॒ ॅवेदि॑म् क॒रोति॒ धा धाः क॒रोति॒ वेदिं॒ ॅवेदि॑म् क॒रोति॒ धाः । \newline
5. क॒रोति॒ धा धाः क॒रोति॑ क॒रोति॒ धा अ॑स्यसि॒ धाः क॒रोति॑ क॒रोति॒ धा अ॑सि । \newline
6. धा अ॑स्यसि॒ धा धा अ॑सि स्व॒धा स्व॒धा ऽसि॒ धा धा अ॑सि स्व॒धा । \newline
7. अ॒सि॒ स्व॒धा स्व॒धा ऽस्य॑सि स्व॒धा अ॑स्यसि स्व॒धा ऽस्य॑सि स्व॒धा अ॑सि । \newline
8. स्व॒धा अ॑स्यसि स्व॒धा स्व॒धा अ॒सीतीत्य॑सि स्व॒धा स्व॒धा अ॒सीति॑ । \newline
9. स्व॒धेति॑ स्व - धा । \newline
10. अ॒सीती त्य॑स्य॒सीति॑ योयुप्यते योयुप्यत॒ इत्य॑स्य॒सीति॑ योयुप्यते । \newline
11. इति॑ योयुप्यते योयुप्यत॒ इतीति॑ योयुप्यते॒ शान्त्यै॒ शान्त्यै॑ योयुप्यत॒ इतीति॑ योयुप्यते॒ शान्त्यै᳚ । \newline
12. यो॒यु॒प्य॒ते॒ शान्त्यै॒ शान्त्यै॑ योयुप्यते योयुप्यते॒ शान्त्यै॒ प्रोक्ष॑णीः॒ प्रोक्ष॑णीः॒ शान्त्यै॑ योयुप्यते योयुप्यते॒ शान्त्यै॒ प्रोक्ष॑णीः । \newline
13. शान्त्यै॒ प्रोक्ष॑णीः॒ प्रोक्ष॑णीः॒ शान्त्यै॒ शान्त्यै॒ प्रोक्ष॑णी॒रा प्रोक्ष॑णीः॒ शान्त्यै॒ शान्त्यै॒ प्रोक्ष॑णी॒रा । \newline
14. प्रोक्ष॑णी॒रा प्रोक्ष॑णीः॒ प्रोक्ष॑णी॒रा सा॑दयति सादय॒त्या प्रोक्ष॑णीः॒ प्रोक्ष॑णी॒रा सा॑दयति । \newline
15. प्रोक्ष॑णी॒रिति॑ प्र - उक्ष॑णीः । \newline
16. आ सा॑दयति सादय॒त्या सा॑दय॒ त्याप॒ आपः॑ सादय॒त्या सा॑दय॒ त्यापः॑ । \newline
17. सा॒द॒य॒ त्याप॒ आपः॑ सादयति सादय॒ त्यापो॒ वै वा आपः॑ सादयति सादय॒ त्यापो॒ वै । \newline
18. आपो॒ वै वा आप॒ आपो॒ वै र॑क्षो॒घ्नी र॑क्षो॒घ्नीर् वा आप॒ आपो॒ वै र॑क्षो॒घ्नीः । \newline
19. वै र॑क्षो॒घ्नी र॑क्षो॒घ्नीर् वै वै र॑क्षो॒घ्नी रक्ष॑साꣳ॒॒ रक्ष॑साꣳ रक्षो॒घ्नीर् वै वै र॑क्षो॒घ्नी रक्ष॑साम् । \newline
20. र॒क्षो॒घ्नी रक्ष॑साꣳ॒॒ रक्ष॑साꣳ रक्षो॒घ्नी र॑क्षो॒घ्नी रक्ष॑सा॒ मप॑हत्या॒ अप॑हत्यै॒ रक्ष॑साꣳ रक्षो॒घ्नी र॑क्षो॒घ्नी रक्ष॑सा॒ मप॑हत्यै । \newline
21. र॒क्षो॒घ्नीरिति॑ रक्षः - घ्नीः । \newline
22. रक्ष॑सा॒ मप॑हत्या॒ अप॑हत्यै॒ रक्ष॑साꣳ॒॒ रक्ष॑सा॒ मप॑हत्यै॒ स्फ्यस्य॒ स्फ्यस्या प॑हत्यै॒ रक्ष॑साꣳ॒॒ रक्ष॑सा॒ मप॑हत्यै॒ स्फ्यस्य॑ । \newline
23. अप॑हत्यै॒ स्फ्यस्य॒ स्फ्यस्या प॑हत्या॒ अप॑हत्यै॒ स्फ्यस्य॒ वर्त्म॒न्॒. वर्त्म॒न् थ्स्फ्यस्या प॑हत्या॒ अप॑हत्यै॒ स्फ्यस्य॒ वर्त्मन्न्॑ । \newline
24. अप॑हत्या॒ इत्यप॑ - ह॒त्यै॒ । \newline
25. स्फ्यस्य॒ वर्त्म॒न्॒. वर्त्म॒न् थ्स्फ्यस्य॒ स्फ्यस्य॒ वर्त्मन्᳚ थ्सादयति सादयति॒ वर्त्म॒न् थ्स्फ्यस्य॒ स्फ्यस्य॒ वर्त्मन्᳚ थ्सादयति । \newline
26. वर्त्मन्᳚ थ्सादयति सादयति॒ वर्त्म॒न्॒. वर्त्मन्᳚ थ्सादयति य॒ज्ञ्स्य॑ य॒ज्ञ्स्य॑ सादयति॒ वर्त्म॒न्॒. वर्त्मन्᳚ थ्सादयति य॒ज्ञ्स्य॑ । \newline
27. सा॒द॒य॒ति॒ य॒ज्ञ्स्य॑ य॒ज्ञ्स्य॑ सादयति सादयति य॒ज्ञ्स्य॒ सन्त॑त्यै॒ सन्त॑त्यै य॒ज्ञ्स्य॑ सादयति सादयति य॒ज्ञ्स्य॒ सन्त॑त्यै । \newline
28. य॒ज्ञ्स्य॒ सन्त॑त्यै॒ सन्त॑त्यै य॒ज्ञ्स्य॑ य॒ज्ञ्स्य॒ सन्त॑त्यै॒ यं ॅयꣳ सन्त॑त्यै य॒ज्ञ्स्य॑ य॒ज्ञ्स्य॒ सन्त॑त्यै॒ यम् । \newline
29. सन्त॑त्यै॒ यं ॅयꣳ सन्त॑त्यै॒ सन्त॑त्यै॒ यम् द्वि॒ष्याद् द्वि॒ष्याद् यꣳ सन्त॑त्यै॒ सन्त॑त्यै॒ यम् द्वि॒ष्यात् । \newline
30. सन्त॑त्या॒ इति॒ सं - त॒त्यै॒ । \newline
31. यम् द्वि॒ष्याद् द्वि॒ष्याद् यं ॅयम् द्वि॒ष्यात् तम् तम् द्वि॒ष्याद् यं ॅयम् द्वि॒ष्यात् तम् । \newline
32. द्वि॒ष्यात् तम् तम् द्वि॒ष्याद् द्वि॒ष्यात् तम् ध्या॑येद् ध्याये॒त् तम् द्वि॒ष्याद् द्वि॒ष्यात् तम् ध्या॑येत् । \newline
33. तम् ध्या॑येद् ध्याये॒त् तम् तम् ध्या॑येच्छु॒चा शु॒चा ध्या॑ये॒त् तम् तम् ध्या॑येच्छु॒चा । \newline
34. ध्या॒ये॒च्छु॒चा शु॒चा ध्या॑येद् ध्यायेच् छु॒चैवैव शु॒चा ध्या॑येद् ध्यायेच् छु॒चैव । \newline
35. शु॒चैवैव शु॒चा शु॒चैवैन॑ मेन मे॒व शु॒चा शु॒चैवैन᳚म् । \newline
36. ए॒वैन॑ मेन मे॒वैवैन॑ मर्पय त्यर्पयत्येन मे॒वैवैन॑ मर्पयति । \newline
37. ए॒न॒ म॒र्प॒य॒ त्य॒र्प॒य॒त्ये॒न॒ मे॒न॒ म॒र्प॒य॒ति॒ । \newline
38. अ॒र्प॒य॒तीत्य॑र्पयति । \newline
\pagebreak
\markright{ TS 2.6.5.1  \hfill https://www.vedavms.in \hfill}
\addcontentsline{toc}{section}{ TS 2.6.5.1 }
\section*{ TS 2.6.5.1 }

\textbf{TS 2.6.5.1 } \newline
\textbf{Samhita Paata} \newline

ब्र॒ह्म॒वा॒दिनो॑ वदन्त्य॒द्भिर्.-ह॒वीꣳषि॒ प्रौक्षीः॒ केना॒प इति॒ ब्रह्म॒णेति॑ ब्रूयाद॒द्भि-र्ह्ये॑व ह॒वीꣳषि॑ प्रो॒क्षति॒ ब्रह्म॑णा॒ऽप इ॒द्ध्माब॒र्॒.हिः प्रोक्ष॑ति॒ मेद्ध्य॑मे॒वैन॑त् करोति॒ वेदिं॒ प्रोक्ष॑त्यृ॒क्षा वा ॒षाऽलो॒मका॑ऽमे॒द्ध्या यद्-वेदि॒र्मेद्ध्या॑-मे॒वैनां᳚ करोति दि॒वे त्वा॒ऽन्तरि॑क्षाय त्वा पृथि॒व्यै त्वेति॑ ब॒र॒.हिरा॒साद्य॒ प्रो - [  ] \newline

\textbf{Pada Paata} \newline

ब्र॒ह्म॒वा॒दिन॒ इति॑ ब्रह्म - वा॒दिनः॑ । व॒द॒न्ति॒ । अ॒द्भिरित्य॑त् - भिः । ह॒वीꣳषि॑ । प्रेति॑ । औ॒क्षीः॒ । केन॑ । अ॒पः । इति॑ । ब्रह्म॑णा । इति॑ । ब्रू॒या॒त् । अ॒द्भिरित्य॑त् - भिः । हि । ए॒व । ह॒वीꣳषि॑ । प्रो॒क्षतीति॑ प्र - उ॒क्षति॑ । ब्रह्म॑णा । अ॒पः । इ॒द्ध्माब॒र्॒.हिरिती॒द्ध्मा - ब॒र्॒.हिः । प्रेति॑ । उ॒क्ष॒ति॒ । मेद्ध्य᳚म् । ए॒व । ए॒न॒त् । क॒रो॒ति॒ । वेदि᳚म् । प्रेति॑ । उ॒क्ष॒ति॒ । ऋ॒क्षा । वै । ए॒षा । अ॒लो॒मका᳚ । अ॒मे॒द्ध्या । यत् । वेदिः॑ । मेद्ध्या᳚म् । ए॒व । ए॒ना॒म् । क॒रो॒ति॒ । दि॒वे । त्वा॒ । अ॒न्तरि॑क्षाय । त्वा॒ । पृ॒थि॒व्यै । त्वा॒ । इति॑ । ब॒र्॒.हिः । आ॒साद्येत्या᳚ - साद्य॑ । प्रेति॑ ।  \newline


\textbf{Krama Paata} \newline

ब्र॒ह्म॒वा॒दिनो॑ वदन्ति । ब्र॒ह्म॒वा॒दिन॒ इति॑ ब्रह्म - वा॒दिनः॑ । व॒द॒न्त्य॒द्भिः । अ॒द्भिर्. ह॒वीꣳषि॑ । अ॒द्भिरित्य॑त् - भिः । ह॒वीꣳषि॒ प्र । प्रौक्षीः᳚ । औ॒क्षीः॒ केन॑ । केना॒पः । अ॒प इति॑ । इति॒ ब्रह्म॑णा । ब्रह्म॒णेति॑ । इति॑ ब्रूयात् । ब्रू॒या॒द॒द्भिः । अ॒द्भिर्. हि । अ॒द्भिरित्य॑त् - भिः । ह्ये॑व । ए॒व ह॒वीꣳषि॑ । ह॒वीꣳषि॑ प्रो॒क्षति॑ । प्रो॒क्षति॒ ब्रह्म॑णा । प्रो॒क्षतीति॑ प्र - उ॒क्षति॑ । ब्रह्म॑णा॒ऽपः । अ॒प इ॒ध्माब॒र्.॒हिः । इ॒ध्माब॒र्.॒हिः प्र । इ॒ध्माब॒र्.॒हिरिती॒ध्मा - ब॒र्.॒हिः । प्रोक्ष॑ति । उ॒क्ष॒ति॒ मेद्ध्य᳚म् । मेद्ध्य॑मे॒व । ए॒वैन॑त् । ए॒न॒त् क॒रो॒ति॒ । क॒रो॒ति॒ वेदि᳚म् । वेदि॒म् प्र । प्रोक्ष॑ति । उ॒क्ष॒त्यृ॒क्षा । ऋ॒क्षा वै । वा ए॒षा । ए॒षाऽलो॒मका᳚ । अ॒लो॒मका॑ऽमे॒द्ध्या । अ॒मे॒द्ध्या यत् । यद् वेदिः॑ । वेदि॒र् मेद्ध्या᳚म् । मेद्ध्या॑मे॒व । ए॒वैना᳚म् । ए॒ना॒म् क॒रो॒ति॒ । क॒रो॒ति॒ दि॒वे । दि॒वे त्वा᳚ । त्वा॒ ऽन्तरि॑क्षाय । अ॒न्तरिक्षा॑य त्वा । त्वा॒ पृ॒थि॒व्यै । पृ॒थि॒व्यै त्वा᳚ । त्वेति॑ । इति॑ ब॒र्.॒हिः । ब॒र्.॒हिरा॒साद्य॑ । आ॒साद्य॒ प्र । आ॒साद्येत्या᳚ - साद्य॑ । प्रोक्ष॑ति \newline

\textbf{Jatai Paata} \newline

1. ब्र॒ह्म॒वा॒दिनो॑ वदन्ति वदन्ति ब्रह्मवा॒दिनो᳚ ब्रह्मवा॒दिनो॑ वदन्ति । \newline
2. ब्र॒ह्म॒वा॒दिन॒ इति॑ ब्रह्म - वा॒दिनः॑ । \newline
3. व॒द॒ न्त्य॒द्भि र॒द्भिर् व॑दन्ति वद न्त्य॒द्भिः । \newline
4. अ॒द्भिर्. ह॒वीꣳषि॑ ह॒वीꣳ ष्य॒द्भि र॒द्भिर्. ह॒वीꣳषि॑ । \newline
5. अ॒द्भिरित्य॑त् - भिः । \newline
6. ह॒वीꣳषि॒ प्र प्र ह॒वीꣳषि॑ ह॒वीꣳषि॒ प्र । \newline
7. प्रौक्षी॑ रौक्षीः॒ प्र प्रौक्षीः᳚ । \newline
8. औ॒क्षीः॒ केन॒ केनौ᳚क्षी रौक्षीः॒ केन॑ । \newline
9. केना॒पो॑ ऽपः केन॒ केना॒पः । \newline
10. अ॒प इतीत्य॒पो॑ ऽप इति॑ । \newline
11. इति॒ ब्रह्म॑णा॒ ब्रह्म॒णेतीति॒ ब्रह्म॑णा । \newline
12. ब्रह्म॒णेतीति॒ ब्रह्म॑णा॒ ब्रह्म॒णेति॑ । \newline
13. इति॑ ब्रूयाद् ब्रूया॒ दितीति॑ ब्रूयात् । \newline
14. ब्रू॒या॒ द॒द्भि र॒द्भिर् ब्रू॑याद् ब्रूया द॒द्भिः । \newline
15. अ॒द्भिर्. हि ह्य॑द्भि र॒द्भिर्. हि । \newline
16. अ॒द्भिरित्य॑त् - भिः । \newline
17. ह्ये॑वैव हि ह्ये॑व । \newline
18. ए॒व ह॒वीꣳषि॑ ह॒वीꣳ ष्ये॒वैव ह॒वीꣳषि॑ । \newline
19. ह॒वीꣳषि॑ प्रो॒क्षति॑ प्रो॒क्षति॑ ह॒वीꣳषि॑ ह॒वीꣳषि॑ प्रो॒क्षति॑ । \newline
20. प्रो॒क्षति॒ ब्रह्म॑णा॒ ब्रह्म॑णा प्रो॒क्षति॑ प्रो॒क्षति॒ ब्रह्म॑णा । \newline
21. प्रो॒क्षतीति॑ प्र - उ॒क्षति॑ । \newline
22. ब्रह्म॑णा॒ ऽपो॑ ऽपो ब्रह्म॑णा॒ ब्रह्म॑णा॒ ऽपः । \newline
23. अ॒प इ॒द्ध्माब॒र्॒.हि रि॒द्ध्माब॒र्॒.हि र॒पो॑ ऽप इ॒द्ध्माब॒र्॒.हिः । \newline
24. इ॒द्ध्माब॒र्॒.हिः प्र प्रे द्ध्माब॒र्॒.हि रि॒द्ध्माब॒र्॒.हिः प्र । \newline
25. इ॒द्ध्माब॒र्॒.हिरिती॒द्ध्मा - ब॒र्॒.हिः । \newline
26. प्रोक्ष॑ त्युक्षति॒ प्र प्रोक्ष॑ति । \newline
27. उ॒क्ष॒ति॒ मेद्ध्य॒म् मेद्ध्य॑ मुक्ष त्युक्षति॒ मेद्ध्य᳚म् । \newline
28. मेद्ध्य॑ मे॒वैव मेद्ध्य॒म् मेद्ध्य॑ मे॒व । \newline
29. ए॒वैन॑ देन दे॒वैवैन॑त् । \newline
30. ए॒न॒त् क॒रो॒ति॒ क॒रो॒ त्ये॒न॒ दे॒न॒त् क॒रो॒ति॒ । \newline
31. क॒रो॒ति॒ वेदिं॒ ॅवेदि॑म् करोति करोति॒ वेदि᳚म् । \newline
32. वेदि॒म् प्र प्र वेदिं॒ ॅवेदि॒म् प्र । \newline
33. प्रोक्ष॑ त्युक्षति॒ प्र प्रोक्ष॑ति । \newline
34. उ॒क्ष॒ त्यृ॒क्ष र्‌क्षोक्ष॑ त्युक्ष त्यृ॒क्षा । \newline
35. ऋ॒क्षा वै वा ऋ॒क्ष र्‌क्षा वै । \newline
36. वा ए॒षैषा वै वा ए॒षा । \newline
37. ए॒षा ऽलो॒मका॑ ऽलो॒म कै॒षैषा ऽलो॒मका᳚ । \newline
38. अ॒लो॒मका॑ ऽमे॒द्ध्या ऽमे॒द्ध्या ऽलो॒मका॑ ऽलो॒मका॑ ऽमे॒द्ध्या । \newline
39. अ॒मे॒द्ध्या यद् यद॑मे॒द्ध्या ऽमे॒द्ध्या यत् । \newline
40. यद् वेदि॒र् वेदि॒र् यद् यद् वेदिः॑ । \newline
41. वेदि॒र् मेद्ध्या॒म् मेद्ध्यां॒ ॅवेदि॒र् वेदि॒र् मेद्ध्या᳚म् । \newline
42. मेद्ध्या॑ मे॒वैव मेद्ध्या॒म् मेद्ध्या॑ मे॒व । \newline
43. ए॒वैना॑ मेना मे॒वैवैना᳚म् । \newline
44. ए॒ना॒म् क॒रो॒ति॒ क॒रो॒ त्ये॒ना॒ मे॒ना॒म् क॒रो॒ति॒ । \newline
45. क॒रो॒ति॒ दि॒वे दि॒वे क॑रोति करोति दि॒वे । \newline
46. दि॒वे त्वा᳚ त्वा दि॒वे दि॒वे त्वा᳚ । \newline
47. त्वा॒ ऽन्तरि॑क्षाया॒ न्तरि॑क्षाय त्वा त्वा॒ ऽन्तरि॑क्षाय । \newline
48. अ॒न्तरि॑क्षाय त्वा त्वा॒ ऽन्तरि॑क्षाया॒ न्तरि॑क्षाय त्वा । \newline
49. त्वा॒ पृ॒थि॒व्यै पृ॑थि॒व्यै त्वा᳚ त्वा पृथि॒व्यै । \newline
50. पृ॒थि॒व्यै त्वा᳚ त्वा पृथि॒व्यै पृ॑थि॒व्यै त्वा᳚ । \newline
51. त्वेतीति॑ त्वा॒ त्वेति॑ । \newline
52. इति॑ ब॒र्॒.हिर् ब॒र्॒.हि रितीति॑ ब॒र्॒.हिः । \newline
53. ब॒र्॒.हि रा॒साद्या॒ साद्य॑ ब॒र्॒.हिर् ब॒र्॒.हि रा॒साद्य॑ । \newline
54. आ॒साद्य॒ प्र प्रासाद्या॒ साद्य॒ प्र । \newline
55. आ॒साद्येत्या᳚ - साद्य॑ । \newline
56. प्रोक्ष॑ त्युक्षति॒ प्र प्रोक्ष॑ति । \newline

\textbf{Ghana Paata } \newline

1. ब्र॒ह्म॒वा॒दिनो॑ वदन्ति वदन्ति ब्रह्मवा॒दिनो᳚ ब्रह्मवा॒दिनो॑ वद न्त्य॒द्भि र॒द्भिर् व॑दन्ति ब्रह्मवा॒दिनो᳚ ब्रह्मवा॒दिनो॑ वदन्त्य॒द्भिः । \newline
2. ब्र॒ह्म॒वा॒दिन॒ इति॑ ब्रह्म - वा॒दिनः॑ । \newline
3. व॒द॒ न्त्य॒द्भि र॒द्भिर् व॑दन्ति वदन् त्य॒द्भिर्. ह॒वीꣳषि॑ ह॒वीꣳ ष्य॒द्भिर् व॑दन्ति वद न्त्य॒द्भिर्. ह॒वीꣳषि॑ । \newline
4. अ॒द्भिर्. ह॒वीꣳषि॑ ह॒वीꣳ ष्य॒द्भि र॒द्भिर्. ह॒वीꣳषि॒ प्र प्र ह॒वीꣳ ष्य॒द्भि र॒द्भिर्. ह॒वीꣳषि॒ प्र । \newline
5. अ॒द्भिरित्य॑त् - भिः । \newline
6. ह॒वीꣳषि॒ प्र प्र ह॒वीꣳषि॑ ह॒वीꣳषि॒ प्रौक्षी॑ रौक्षीः॒ प्र ह॒वीꣳषि॑ ह॒वीꣳषि॒ प्रौक्षीः᳚ । \newline
7. प्रौक्षी॑ रौक्षीः॒ प्र प्रौक्षीः॒ केन॒ केनौ᳚क्षीः॒ प्र प्रौक्षीः॒ केन॑ । \newline
8. औ॒क्षीः॒ केन॒ केनौ᳚क्षी रौक्षीः॒ केना॒पो॑ ऽपः केनौ᳚क्षी रौक्षीः॒ केना॒पः । \newline
9. केना॒पो॑ ऽपः केन॒ केना॒प इतीत्य॒पः केन॒ केना॒प इति॑ । \newline
10. अ॒प इतीत्य॒पो॑ ऽप इति॒ ब्रह्म॑णा॒ ब्रह्म॒णेत्य॒पो॑ ऽप इति॒ ब्रह्म॑णा । \newline
11. इति॒ ब्रह्म॑णा॒ ब्रह्म॒णेतीति॒ ब्रह्म॒णेतीति॒ ब्रह्म॒णेतीति॒ ब्रह्म॒णेति॑ । \newline
12. ब्रह्म॒णेतीति॒ ब्रह्म॑णा॒ ब्रह्म॒णेति॑ ब्रूयाद् ब्रूया॒दिति॒ ब्रह्म॑णा॒ ब्रह्म॒णेति॑ ब्रूयात् । \newline
13. इति॑ ब्रूयाद् ब्रूया॒ दितीति॑ ब्रूया द॒द्भि र॒द्भिर् ब्रू॑या॒ दितीति॑ ब्रूयाद॒द्भिः । \newline
14. ब्रू॒या॒ द॒द्भि र॒द्भिर् ब्रू॑याद् ब्रूया द॒द्भिर्. हि ह्य॑द्भिर् ब्रू॑याद् ब्रूया द॒द्भिर्. हि । \newline
15. अ॒द्भिर्. हि ह्य॑द्भि र॒द्भिर् ह्ये॑वैव ह्य॑द्भि र॒द्भिर् ह्ये॑व । \newline
16. अ॒द्भिरित्य॑त् - भिः । \newline
17. ह्ये॑वैव हि ह्ये॑व ह॒वीꣳषि॑ ह॒वीꣳ ष्ये॒व हि ह्ये॑व ह॒वीꣳषि॑ । \newline
18. ए॒व ह॒वीꣳषि॑ ह॒वीꣳ ष्ये॒वैव ह॒वीꣳषि॑ प्रो॒क्षति॑ प्रो॒क्षति॑ ह॒वीꣳ ष्ये॒वैव ह॒वीꣳषि॑ प्रो॒क्षति॑ । \newline
19. ह॒वीꣳषि॑ प्रो॒क्षति॑ प्रो॒क्षति॑ ह॒वीꣳषि॑ ह॒वीꣳषि॑ प्रो॒क्षति॒ ब्रह्म॑णा॒ ब्रह्म॑णा प्रो॒क्षति॑ ह॒वीꣳषि॑ ह॒वीꣳषि॑ प्रो॒क्षति॒ ब्रह्म॑णा । \newline
20. प्रो॒क्षति॒ ब्रह्म॑णा॒ ब्रह्म॑णा प्रो॒क्षति॑ प्रो॒क्षति॒ ब्रह्म॑णा॒ ऽपो॑ ऽपो ब्रह्म॑णा प्रो॒क्षति॑ प्रो॒क्षति॒ ब्रह्म॑णा॒ ऽपः । \newline
21. प्रो॒क्षतीति॑ प्र - उ॒क्षति॑ । \newline
22. ब्रह्म॑णा॒ ऽपो॑ ऽपो ब्रह्म॑णा॒ ब्रह्म॑णा॒ ऽप इ॒द्ध्माब॒र्॒.हि रि॒द्ध्माब॒र्॒.हि र॒पो ब्रह्म॑णा॒ ब्रह्म॑णा॒ ऽप इ॒द्ध्माब॒र्॒.हिः । \newline
23. अ॒प इ॒द्ध्माब॒र्॒.हि रि॒द्ध्माब॒र्॒.हि र॒पो॑ ऽप इ॒द्ध्माब॒र्॒.हिः प्र प्रे द्ध्माब॒र्॒.हिर॒पो॑ ऽप इ॒द्ध्माब॒र्॒.हिः प्र । \newline
24. इ॒द्ध्माब॒र्॒.हिः प्र प्रे द्ध्माब॒र्॒.हि रि॒द्ध्माब॒र्॒.हिः प्रोक्ष॑त्युक्षति॒ प्रे द्ध्माब॒र्॒.हि रि॒द्ध्माब॒र्॒.हिः प्रोक्ष॑ति । \newline
25. इ॒द्ध्माब॒र्॒.हिरिती॒द्ध्मा - ब॒र्॒.हिः । \newline
26. प्रोक्ष॑त्युक्षति॒ प्र प्रोक्ष॑ति॒ मेद्ध्य॒म् मेद्ध्य॑ मुक्षति॒ प्र प्रोक्ष॑ति॒ मेद्ध्य᳚म् । \newline
27. उ॒क्ष॒ति॒ मेद्ध्य॒म् मेद्ध्य॑ मुक्षत्युक्षति॒ मेद्ध्य॑ मे॒वैव मेद्ध्य॑ मुक्षत्युक्षति॒ मेद्ध्य॑ मे॒व । \newline
28. मेद्ध्य॑ मे॒वैव मेद्ध्य॒म् मेद्ध्य॑ मे॒वैन॑ देनदे॒व मेद्ध्य॒म् मेद्ध्य॑ मे॒वैन॑त् । \newline
29. ए॒वैन॑ देन दे॒वैवैन॑त् करोति करो त्येन दे॒वैवैन॑त् करोति । \newline
30. ए॒न॒त् क॒रो॒ति॒ क॒रो॒ त्ये॒न॒ दे॒न॒त् क॒रो॒ति॒ वेदिं॒ ॅवेदि॑म् करो त्येन देनत् करोति॒ वेदि᳚म् । \newline
31. क॒रो॒ति॒ वेदिं॒ ॅवेदि॑म् करोति करोति॒ वेदि॒म् प्र प्र वेदि॑म् करोति करोति॒ वेदि॒म् प्र । \newline
32. वेदि॒म् प्र प्र वेदिं॒ ॅवेदि॒म् प्रोक्ष॑ त्युक्षति॒ प्र वेदिं॒ ॅवेदि॒म् प्रोक्ष॑ति । \newline
33. प्रोक्ष॑ त्युक्षति॒ प्र प्रोक्ष॑ त्यृ॒क्षर्‌क्षोक्ष॑ति॒ प्र प्रोक्ष॑त्यृ॒क्षा । \newline
34. उ॒क्ष॒ त्यृ॒क्षर्‌क्षोक्ष॑ त्युक्ष त्यृ॒क्षा वै वा ऋ॒क्षोक्ष॑ त्युक्ष त्यृ॒क्षा वै । \newline
35. ऋ॒क्षा वै वा ऋ॒क्ष र्‌क्षा वा ए॒षैषा वा ऋ॒क्ष र्‌क्षा वा ए॒षा । \newline
36. वा ए॒षैषा वै वा ए॒षा ऽलो॒मका॑ ऽलो॒मकै॒षा वै वा ए॒षा ऽलो॒मका᳚ । \newline
37. ए॒षा ऽलो॒मका॑ ऽलो॒मकै॒षैषा ऽलो॒मका॑ ऽमे॒द्ध्या ऽमे॒द्ध्या ऽलो॒मकै॒षैषा ऽलो॒मका॑ ऽमे॒द्ध्या । \newline
38. अ॒लो॒मका॑ ऽमे॒द्ध्या ऽमे॒द्ध्या ऽलो॒मका॑ ऽलो॒मका॑ ऽमे॒द्ध्या यद् यद॑मे॒द्ध्या ऽलो॒मका॑ ऽलो॒मका॑ ऽमे॒द्ध्या यत् । \newline
39. अ॒मे॒द्ध्या यद् यद॑मे॒द्ध्या ऽमे॒द्ध्या यद् वेदि॒र् वेदि॒र् यद॑मे॒द्ध्या ऽमे॒द्ध्या यद् वेदिः॑ । \newline
40. यद् वेदि॒र् वेदि॒र् यद् यद् वेदि॒र् मेद्ध्या॒म् मेद्ध्यां॒ ॅवेदि॒र् यद् यद् वेदि॒र् मेद्ध्या᳚म् । \newline
41. वेदि॒र् मेद्ध्या॒म् मेद्ध्यां॒ ॅवेदि॒र् वेदि॒र् मेद्ध्या॑ मे॒वैव मेद्ध्यां॒ ॅवेदि॒र् वेदि॒र् मेद्ध्या॑ मे॒व । \newline
42. मेद्ध्या॑ मे॒वैव मेद्ध्या॒म् मेद्ध्या॑ मे॒वैना॑ मेना मे॒व मेद्ध्या॒म् मेद्ध्या॑ मे॒वैना᳚म् । \newline
43. ए॒वैना॑ मेना मे॒वैवैना᳚म् करोति करोत्येना मे॒वैवैना᳚म् करोति । \newline
44. ए॒ना॒म् क॒रो॒ति॒ क॒रो॒त्ये॒ना॒ मे॒ना॒म् क॒रो॒ति॒ दि॒वे दि॒वे क॑रोत्येना मेनाम् करोति दि॒वे । \newline
45. क॒रो॒ति॒ दि॒वे दि॒वे क॑रोति करोति दि॒वे त्वा᳚ त्वा दि॒वे क॑रोति करोति दि॒वे त्वा᳚ । \newline
46. दि॒वे त्वा᳚ त्वा दि॒वे दि॒वे त्वा॒ ऽन्तरि॑क्षाया॒ न्तरि॑क्षाय त्वा दि॒वे दि॒वे त्वा॒ ऽन्तरि॑क्षाय । \newline
47. त्वा॒ ऽन्तरि॑क्षाया॒ न्तरि॑क्षाय त्वा त्वा॒ ऽन्तरि॑क्षाय त्वा त्वा॒ ऽन्तरि॑क्षाय त्वा त्वा॒ ऽन्तरि॑क्षाय त्वा । \newline
48. अ॒न्तरि॑क्षाय त्वा त्वा॒ ऽन्तरि॑क्षाया॒ न्तरि॑क्षाय त्वा पृथि॒व्यै पृ॑थि॒व्यै त्वा॒ ऽन्तरि॑क्षाया॒ न्तरि॑क्षाय त्वा पृथि॒व्यै । \newline
49. त्वा॒ पृ॒थि॒व्यै पृ॑थि॒व्यै त्वा᳚ त्वा पृथि॒व्यै त्वा᳚ त्वा पृथि॒व्यै त्वा᳚ त्वा पृथि॒व्यै त्वा᳚ । \newline
50. पृ॒थि॒व्यै त्वा᳚ त्वा पृथि॒व्यै पृ॑थि॒व्यै त्वेतीति॑ त्वा पृथि॒व्यै पृ॑थि॒व्यै त्वेति॑ । \newline
51. त्वेतीति॑ त्वा॒ त्वेति॑ ब॒र्॒.हिर् ब॒र्॒.हिरिति॑ त्वा॒ त्वेति॑ ब॒र्॒.हिः । \newline
52. इति॑ ब॒र्॒.हिर् ब॒र्॒.हि रितीति॑ ब॒र्॒.हि रा॒साद्या॒ साद्य॑ ब॒र्॒.हि रितीति॑ ब॒र्॒.हिरा॒साद्य॑ । \newline
53. ब॒र्॒.हि रा॒साद्या॒ साद्य॑ ब॒र्॒.हिर् ब॒र्॒.हि रा॒साद्य॒ प्र प्रासाद्य॑ ब॒र्॒.हिर् ब॒र्॒.हि रा॒साद्य॒ प्र । \newline
54. आ॒साद्य॒ प्र प्रासाद्या॒ साद्य॒ प्रोक्ष॑ त्युक्षति॒ प्रासाद्या॒ साद्य॒ प्रोक्ष॑ति । \newline
55. आ॒साद्येत्या᳚ - साद्य॑ । \newline
56. प्रोक्ष॑ त्युक्षति॒ प्र प्रोक्ष॑ त्ये॒भ्य ए॒भ्य उ॑क्षति॒ प्र प्रोक्ष॑ त्ये॒भ्यः । \newline
\pagebreak
\markright{ TS 2.6.5.2  \hfill https://www.vedavms.in \hfill}
\addcontentsline{toc}{section}{ TS 2.6.5.2 }
\section*{ TS 2.6.5.2 }

\textbf{TS 2.6.5.2 } \newline
\textbf{Samhita Paata} \newline

-क्ष॑त्ये॒भ्य ए॒वैन॑ल्लो॒केभ्यः॒ प्रोक्ष॑ति क्रू॒रमि॑व॒ वा ए॒तत् क॑रोति॒ यत् खन॑त्य॒पो निन॑यति॒ शान्त्यै॑ पु॒रस्ता᳚त् प्रस्त॒रं गृ॑ह्णाति॒ मुख्य॑मे॒वैनं॑ करो॒तीय॑न्तं गृह्णाति प्र॒जाप॑तिना यज्ञ्मु॒खेन॒ संमि॑तं ब॒र्॒.हिः स्तृ॑णाति प्र॒जा वै ब॒र्॒.हिः पृ॑थि॒वी वेदिः॑ प्र॒जा ए॒व पृ॑थि॒व्यां प्रति॑ष्ठापय॒त्यन॑तिदृश्नꣳ स्तृणाति प्र॒जयै॒वैनं॑ प॒शुभि॒रन॑तिदृश्नं करो॒- [  ] \newline

\textbf{Pada Paata} \newline

उ॒क्ष॒ति॒ । ए॒भ्यः । ए॒व । ए॒न॒त् । लो॒केभ्यः॑ । प्रेति॑ । उ॒क्ष॒ति॒ । क्रू॒रम् । इ॒व॒ । वै । ए॒तत् । क॒रो॒ति॒ । यत् । खन॑ति । अ॒पः । नीति॑ । न॒य॒ति॒ । शान्त्यै᳚ । पु॒रस्ता᳚त् । प्र॒स्त॒रमिति॑ प्र - स्त॒रम् । गृ॒ह्णा॒ति॒ । मुख्य᳚म् । ए॒व । ए॒न॒म् । क॒रो॒ति॒ । इय॑न्तम् । गृ॒ह्णा॒ति॒ । प्र॒जाप॑ति॒नेति॑ प्र॒जा - प॒ति॒ना॒ । य॒ज्ञ्॒मु॒खेनेति॑ यज्ञ् - मु॒खेन॑ । संमि॑त॒मिति॒ सं - मि॒त॒म् । ब॒र्॒.हिः । स्तृ॒णा॒ति॒ । प्र॒जा इति॑ प्र - जाः । वै । ब॒र्॒.हिः । पृ॒थि॒वी । वेदिः॑ । प्र॒जा इति॑ प्र - जाः । ए॒व । पृ॒थि॒व्याम् । प्रतीति॑ । स्थ॒प॒य॒ति॒ । अन॑तिदृश्न॒मित्यन॑ति - दृ॒श्न॒म् । स्तृ॒णा॒ति॒ । प्र॒जयेति॑ प्र - जया᳚ । ए॒व । ए॒न॒म् । प॒शुभि॒रिति॑ प॒शु - भिः॒ । अन॑तिदृश्न॒मित्यन॑ति - दृ॒श्न॒म् । क॒रो॒ति॒ ।  \newline


\textbf{Krama Paata} \newline

उ॒क्ष॒त्ये॒भ्यः । ए॒भ्य ए॒व । ए॒वैन॑त् । ए॒न॒ल्लो॒केभ्यः॑ । लो॒केभ्यः॒ प्र । प्रोक्ष॑ति । उ॒क्ष॒ति॒ क्रू॒रम् । क्रू॒रमि॑व । इ॒व॒ वै । वा ए॒तत् । ए॒तत् क॑रोति । क॒रो॒ति॒ यत् । यत् खन॑ति । खन॑त्य॒पः । अ॒पो नि । नि न॑यति । न॒य॒ति॒ शान्त्यै᳚ । शान्त्यै॑ पु॒रस्ता᳚त् । पु॒रस्ता᳚त् प्रस्त॒रम् । प्र॒स्त॒रम् गृ॑ह्णाति । प्र॒स्त॒रमिति॑ प्र - स्त॒रम् । गृ॒ह्णा॒ति॒ मुख्य᳚म् । मुख्य॑मे॒व । ए॒वैन᳚म् । ए॒न॒म् क॒रो॒ति॒ । क॒रो॒तीय॑न्तम् । इय॑न्तम् गृह्णाति । गृ॒ह्णा॒ति॒ प्र॒जाप॑तिना । प्र॒जाप॑तिना यज्ञ्मु॒खेन॑ । प्र॒जाप॑ति॒नेति॑ प्र॒जा - प॒ति॒ना॒ । य॒ज्ञ्॒मु॒खेन॒ सम्मि॑तम् । य॒ज्ञ्॒मु॒खेनेति॑ यज्ञ् - मु॒खेन॑ । सम्मि॑तम् ब॒र्.॒हिः । सम्मि॑त॒मिति॒ सं - मि॒त॒म् । ब॒र्.॒हिः स्तृ॑णाति । स्तृ॒णा॒ति॒ प्र॒जाः । प्र॒जा वै । प्र॒जा इति॑ प्र - जाः । वै ब॒र्.॒हिः । ब॒र्.॒हिः पृ॑थि॒वी । पृ॒थि॒वी वेदिः॑ । वेदिः॑ प्र॒जाः । प्र॒जा ए॒व । प्र॒जा इति॑ प्र - जाः । ए॒व पृ॑थि॒व्याम् । पृ॒थि॒व्याम् प्रति॑ । प्रति॑ ष्ठापयति । स्था॒प॒य॒त्यन॑तिदृश्ञम् । अन॑तिदृश्ञꣳ स्तृणाति । अन॑तिदृश्ञ॒मित्यन॑ति - दृ॒श्ञ॒म् । स्तृ॒णा॒ति॒ प्र॒जया᳚ । प्र॒जयै॒व । प्र॒जयेति॑ प्र - जया᳚ । ए॒वैन᳚म् । ए॒न॒म् प॒शुभिः॑ । प॒शुभि॒रन॑तिदृश्ञम् । प॒शुभि॒रिति॑ प॒शु - भिः॒ । अन॑तिदृश्ञम् करोति । अन॑तिदृश्ञ॒मित्यन॑ति - दृ॒श्ञ॒म् । क॒रो॒त्युत्त॑रम् \newline

\textbf{Jatai Paata} \newline

1. उ॒क्ष॒ त्ये॒भ्य ए॒भ्य उ॑क्ष त्युक्ष त्ये॒भ्यः । \newline
2. ए॒भ्य ए॒वैवैभ्य ए॒भ्य ए॒व । \newline
3. ए॒वैन॑ देन दे॒वैवैन॑त् । \newline
4. ए॒न॒ ल्लो॒केभ्यो॑ लो॒केभ्य॑ एन देन ल्लो॒केभ्यः॑ । \newline
5. लो॒केभ्यः॒ प्र प्र लो॒केभ्यो॑ लो॒केभ्यः॒ प्र । \newline
6. प्रोक्ष॑ त्युक्षति॒ प्र प्रोक्ष॑ति । \newline
7. उ॒क्ष॒ति॒ क्रू॒रम् क्रू॒र मु॑क्ष त्युक्षति क्रू॒रम् । \newline
8. क्रू॒र मि॑वे व क्रू॒रम् क्रू॒र मि॑व । \newline
9. इ॒व॒ वै वा इ॑वे व॒ वै । \newline
10. वा ए॒त दे॒तद् वै वा ए॒तत् । \newline
11. ए॒तत् क॑रोति करो त्ये॒त दे॒तत् क॑रोति । \newline
12. क॒रो॒ति॒ यद् यत् क॑रोति करोति॒ यत् । \newline
13. यत् खन॑ति॒ खन॑ति॒ यद् यत् खन॑ति । \newline
14. खन॑त्य॒पो॑ ऽपः खन॑ति॒ खन॑त्य॒पः । \newline
15. अ॒पो नि न्या᳚(1॒)पो॑ ऽपो नि । \newline
16. नि न॑यति नयति॒ नि नि न॑यति । \newline
17. न॒य॒ति॒ शान्त्यै॒ शान्त्यै॑ नयति नयति॒ शान्त्यै᳚ । \newline
18. शान्त्यै॑ पु॒रस्ता᳚त् पु॒रस्ता॒च् छान्त्यै॒ शान्त्यै॑ पु॒रस्ता᳚त् । \newline
19. पु॒रस्ता᳚त् प्रस्त॒रम् प्र॑स्त॒रम् पु॒रस्ता᳚त् पु॒रस्ता᳚त् प्रस्त॒रम् । \newline
20. प्र॒स्त॒रम् गृ॑ह्णाति गृह्णाति प्रस्त॒रम् प्र॑स्त॒रम् गृ॑ह्णाति । \newline
21. प्र॒स्त॒रमिति॑ प्र - स्त॒रम् । \newline
22. गृ॒ह्णा॒ति॒ मुख्य॒म् मुख्य॑म् गृह्णाति गृह्णाति॒ मुख्य᳚म् । \newline
23. मुख्य॑ मे॒वैव मुख्य॒म् मुख्य॑ मे॒व । \newline
24. ए॒वैन॑ मेन मे॒वैवैन᳚म् । \newline
25. ए॒न॒म् क॒रो॒ति॒ क॒रो॒ त्ये॒न॒ मे॒न॒म् क॒रो॒ति॒ । \newline
26. क॒रो॒तीय॑न्त॒ मिय॑न्तम् करोति करो॒तीय॑न्तम् । \newline
27. इय॑न्तम् गृह्णाति गृह्णा॒तीय॑न्त॒ मिय॑न्तम् गृह्णाति । \newline
28. गृ॒ह्णा॒ति॒ प्र॒जाप॑तिना प्र॒जाप॑तिना गृह्णाति गृह्णाति प्र॒जाप॑तिना । \newline
29. प्र॒जाप॑तिना यज्ञ्मु॒खेन॑ यज्ञ्मु॒खेन॑ प्र॒जाप॑तिना प्र॒जाप॑तिना यज्ञ्मु॒खेन॑ । \newline
30. प्र॒जाप॑ति॒नेति॑ प्र॒जा - प॒ति॒ना॒ । \newline
31. य॒ज्ञ्॒मु॒खेन॒ सम्मि॑तꣳ॒॒ सम्मि॑तं ॅयज्ञ्मु॒खेन॑ यज्ञ्मु॒खेन॒ सम्मि॑तम् । \newline
32. य॒ज्ञ्॒मु॒खेनेति॑ यज्ञ् - मु॒खेन॑ । \newline
33. सम्मि॑तम् ब॒र्॒.हिर् ब॒र्॒.हिः सम्मि॑तꣳ॒॒ सम्मि॑तम् ब॒र्॒.हिः । \newline
34. सम्मि॑त॒मिति॒ सं - मि॒त॒म् । \newline
35. ब॒र्॒.हिः स्तृ॑णाति स्तृणाति ब॒र्॒.हिर् ब॒र्॒.हिः स्तृ॑णाति । \newline
36. स्तृ॒णा॒ति॒ प्र॒जाः प्र॒जाः स्तृ॑णाति स्तृणाति प्र॒जाः । \newline
37. प्र॒जा वै वै प्र॒जाः प्र॒जा वै । \newline
38. प्र॒जा इति॑ प्र - जाः । \newline
39. वै ब॒र्॒.हिर् ब॒र्॒.हिर् वै वै ब॒र्॒.हिः । \newline
40. ब॒र्॒.हिः पृ॑थि॒वी पृ॑थि॒वी ब॒र्॒.हिर् ब॒र्॒.हिः पृ॑थि॒वी । \newline
41. पृ॒थि॒वी वेदि॒र् वेदिः॑ पृथि॒वी पृ॑थि॒वी वेदिः॑ । \newline
42. वेदिः॑ प्र॒जाः प्र॒जा वेदि॒र् वेदिः॑ प्र॒जाः । \newline
43. प्र॒जा ए॒वैव प्र॒जाः प्र॒जा ए॒व । \newline
44. प्र॒जा इति॑ प्र - जाः । \newline
45. ए॒व पृ॑थि॒व्याम् पृ॑थि॒व्या मे॒वैव पृ॑थि॒व्याम् । \newline
46. पृ॒थि॒व्याम् प्रति॒ प्रति॑ पृथि॒व्याम् पृ॑थि॒व्याम् प्रति॑ । \newline
47. प्रति॑ ष्ठापयति स्थापयति॒ प्रति॒ प्रति॑ ष्ठापयति । \newline
48. स्था॒प॒य॒ त्यन॑तिदृश्ञ॒ मन॑तिदृश्ञꣳ स्थापयति स्थापय॒ त्यन॑तिदृश्ञम् । \newline
49. अन॑तिदृश्ञꣳ स्तृणाति स्तृणा॒ त्यन॑तिदृश्ञ॒ मन॑तिदृश्ञꣳ स्तृणाति । \newline
50. अन॑तिदृश्ञ॒मित्यन॑ति - दृ॒श्ञ॒म् । \newline
51. स्तृ॒णा॒ति॒ प्र॒जया᳚ प्र॒जया᳚ स्तृणाति स्तृणाति प्र॒जया᳚ । \newline
52. प्र॒जयै॒वैव प्र॒जया᳚ प्र॒जयै॒व । \newline
53. प्र॒जयेति॑ प्र - जया᳚ । \newline
54. ए॒वैन॑ मेन मे॒वैवैन᳚म् । \newline
55. ए॒न॒म् प॒शुभिः॑ प॒शुभि॑ रेन मेनम् प॒शुभिः॑ । \newline
56. प॒शुभि॒ रन॑तिदृश्ञ॒ मन॑तिदृश्ञम् प॒शुभिः॑ प॒शुभि॒ रन॑तिदृश्ञम् । \newline
57. प॒शुभि॒रिति॑ प॒शु - भिः॒ । \newline
58. अन॑तिदृश्ञम् करोति करो॒ त्यन॑तिदृश्ञ॒ मन॑तिदृश्ञम् करोति । \newline
59. अन॑तिदृश्ञ॒मित्यन॑ति - दृ॒श्ञ॒म् । \newline
60. क॒रो॒ त्युत्त॑र॒ मुत्त॑रम् करोति करो॒ त्युत्त॑रम् । \newline

\textbf{Ghana Paata } \newline

1. उ॒क्ष॒त्ये॒भ्य ए॒भ्य उ॑क्ष त्युक्ष त्ये॒भ्य ए॒वैवैभ्य उ॑क्ष त्युक्ष त्ये॒भ्य ए॒व । \newline
2. ए॒भ्य ए॒वैवैभ्य ए॒भ्य ए॒वैन॑ देन दे॒वैभ्य ए॒भ्य ए॒वैन॑त् । \newline
3. ए॒वैन॑ देन दे॒वैवैन॑ ल्लो॒केभ्यो॑ लो॒केभ्य॑ एन दे॒वैवैन॑ ल्लो॒केभ्यः॑ । \newline
4. ए॒न॒ ल्लो॒केभ्यो॑ लो॒केभ्य॑ एन देन ल्लो॒केभ्यः॒ प्र प्र लो॒केभ्य॑ एन देन ल्लो॒केभ्यः॒ प्र । \newline
5. लो॒केभ्यः॒ प्र प्र लो॒केभ्यो॑ लो॒केभ्यः॒ प्रोक्ष॑ त्युक्षति॒ प्र लो॒केभ्यो॑ लो॒केभ्यः॒ प्रोक्ष॑ति । \newline
6. प्रोक्ष॑ त्युक्षति॒ प्र प्रोक्ष॑ति क्रू॒रम् क्रू॒र मु॑क्षति॒ प्र प्रोक्ष॑ति क्रू॒रम् । \newline
7. उ॒क्ष॒ति॒ क्रू॒रम् क्रू॒र मु॑क्ष त्युक्षति क्रू॒र मि॑वे व क्रू॒र मु॑क्ष त्युक्षति क्रू॒र मि॑व । \newline
8. क्रू॒र मि॑वे व क्रू॒रम् क्रू॒र मि॑व॒ वै वा इ॑व क्रू॒रम् क्रू॒र मि॑व॒ वै । \newline
9. इ॒व॒ वै वा इ॑वे व॒ वा ए॒त दे॒तद् वा इ॑वे व॒ वा ए॒तत् । \newline
10. वा ए॒त दे॒तद् वै वा ए॒तत् क॑रोति करो त्ये॒तद् वै वा ए॒तत् क॑रोति । \newline
11. ए॒तत् क॑रोति करो त्ये॒त दे॒तत् क॑रोति॒ यद् यत् क॑रो त्ये॒त दे॒तत् क॑रोति॒ यत् । \newline
12. क॒रो॒ति॒ यद् यत् क॑रोति करोति॒ यत् खन॑ति॒ खन॑ति॒ यत् क॑रोति करोति॒ यत् खन॑ति । \newline
13. यत् खन॑ति॒ खन॑ति॒ यद् यत् खन॑त्य॒पो॑ ऽपः खन॑ति॒ यद् यत् खन॑त्य॒पः । \newline
14. खन॑त्य॒पो॑ ऽपः खन॑ति॒ खन॑त्य॒पो नि न्य॑पः खन॑ति॒ खन॑त्य॒पो नि । \newline
15. अ॒पो नि न्या᳚(1॒)पो॑ ऽपो नि न॑यति नयति॒ न्या᳚(1॒)पो॑ ऽपो नि न॑यति । \newline
16. नि न॑यति नयति॒ नि नि न॑यति॒ शान्त्यै॒ शान्त्यै॑ नयति॒ नि नि न॑यति॒ शान्त्यै᳚ । \newline
17. न॒य॒ति॒ शान्त्यै॒ शान्त्यै॑ नयति नयति॒ शान्त्यै॑ पु॒रस्ता᳚त् पु॒रस्ता॒च् छान्त्यै॑ नयति नयति॒ शान्त्यै॑ पु॒रस्ता᳚त् । \newline
18. शान्त्यै॑ पु॒रस्ता᳚त् पु॒रस्ता॒च् छान्त्यै॒ शान्त्यै॑ पु॒रस्ता᳚त् प्रस्त॒रम् प्र॑स्त॒रम् पु॒रस्ता॒च् छान्त्यै॒ शान्त्यै॑ पु॒रस्ता᳚त् प्रस्त॒रम् । \newline
19. पु॒रस्ता᳚त् प्रस्त॒रम् प्र॑स्त॒रम् पु॒रस्ता᳚त् पु॒रस्ता᳚त् प्रस्त॒रम् गृ॑ह्णाति गृह्णाति प्रस्त॒रम् पु॒रस्ता᳚त् पु॒रस्ता᳚त् प्रस्त॒रम् गृ॑ह्णाति । \newline
20. प्र॒स्त॒रम् गृ॑ह्णाति गृह्णाति प्रस्त॒रम् प्र॑स्त॒रम् गृ॑ह्णाति॒ मुख्य॒म् मुख्य॑म् गृह्णाति प्रस्त॒रम् प्र॑स्त॒रम् गृ॑ह्णाति॒ मुख्य᳚म् । \newline
21. प्र॒स्त॒रमिति॑ प्र - स्त॒रम् । \newline
22. गृ॒ह्णा॒ति॒ मुख्य॒म् मुख्य॑म् गृह्णाति गृह्णाति॒ मुख्य॑ मे॒वैव मुख्य॑म् गृह्णाति गृह्णाति॒ मुख्य॑ मे॒व । \newline
23. मुख्य॑ मे॒वैव मुख्य॒म् मुख्य॑ मे॒वैन॑ मेन मे॒व मुख्य॒म् मुख्य॑ मे॒वैन᳚म् । \newline
24. ए॒वैन॑ मेन मे॒वैवैन॑म् करोति करोत्येन मे॒वैवैन॑म् करोति । \newline
25. ए॒न॒म् क॒रो॒ति॒ क॒रो॒त्ये॒न॒ मे॒न॒म् क॒रो॒तीय॑न्त॒ मिय॑न्तम् करोत्येन मेनम् करो॒तीय॑न्तम् । \newline
26. क॒रो॒तीय॑न्त॒ मिय॑न्तम् करोति करो॒तीय॑न्तम् गृह्णाति गृह्णा॒तीय॑न्तम् करोति करो॒तीय॑न्तम् गृह्णाति । \newline
27. इय॑न्तम् गृह्णाति गृह्णा॒तीय॑न्त॒ मिय॑न्तम् गृह्णाति प्र॒जाप॑तिना प्र॒जाप॑तिना गृह्णा॒तीय॑न्त॒ मिय॑न्तम् गृह्णाति प्र॒जाप॑तिना । \newline
28. गृ॒ह्णा॒ति॒ प्र॒जाप॑तिना प्र॒जाप॑तिना गृह्णाति गृह्णाति प्र॒जाप॑तिना यज्ञ्मु॒खेन॑ यज्ञ्मु॒खेन॑ प्र॒जाप॑तिना गृह्णाति गृह्णाति प्र॒जाप॑तिना यज्ञ्मु॒खेन॑ । \newline
29. प्र॒जाप॑तिना यज्ञ्मु॒खेन॑ यज्ञ्मु॒खेन॑ प्र॒जाप॑तिना प्र॒जाप॑तिना यज्ञ्मु॒खेन॒ सम्मि॑तꣳ॒॒ सम्मि॑तं ॅयज्ञ्मु॒खेन॑ प्र॒जाप॑तिना प्र॒जाप॑तिना यज्ञ्मु॒खेन॒ सम्मि॑तम् । \newline
30. प्र॒जाप॑ति॒नेति॑ प्र॒जा - प॒ति॒ना॒ । \newline
31. य॒ज्ञ्॒मु॒खेन॒ सम्मि॑तꣳ॒॒ सम्मि॑तं ॅयज्ञ्मु॒खेन॑ यज्ञ्मु॒खेन॒ सम्मि॑तम् ब॒र्॒.हिर् ब॒र्॒.हिः सम्मि॑तं ॅयज्ञ्मु॒खेन॑ यज्ञ्मु॒खेन॒ सम्मि॑तम् ब॒र्॒.हिः । \newline
32. य॒ज्ञ्॒मु॒खेनेति॑ यज्ञ् - मु॒खेन॑ । \newline
33. सम्मि॑तम् ब॒र्॒.हिर् ब॒र्॒.हिः सम्मि॑तꣳ॒॒ सम्मि॑तम् ब॒र्॒.हिः स्तृ॑णाति स्तृणाति ब॒र्॒.हिः सम्मि॑तꣳ॒॒ सम्मि॑तम् ब॒र्॒.हिः स्तृ॑णाति । \newline
34. सम्मि॑त॒मिति॒ सं - मि॒त॒म् । \newline
35. ब॒र्॒.हिः स्तृ॑णाति स्तृणाति ब॒र्॒.हिर् ब॒र्॒.हिः स्तृ॑णाति प्र॒जाः प्र॒जाः स्तृ॑णाति ब॒र्॒.हिर् ब॒र्॒.हिः स्तृ॑णाति प्र॒जाः । \newline
36. स्तृ॒णा॒ति॒ प्र॒जाः प्र॒जाः स्तृ॑णाति स्तृणाति प्र॒जा वै वै प्र॒जाः स्तृ॑णाति स्तृणाति प्र॒जा वै । \newline
37. प्र॒जा वै वै प्र॒जाः प्र॒जा वै ब॒र्॒.हिर् ब॒र्॒.हिर् वै प्र॒जाः प्र॒जा वै ब॒र्॒.हिः । \newline
38. प्र॒जा इति॑ प्र - जाः । \newline
39. वै ब॒र्॒.हिर् ब॒र्॒.हिर् वै वै ब॒र्॒.हिः पृ॑थि॒वी पृ॑थि॒वी ब॒र्॒.हिर् वै वै ब॒र्॒.हिः पृ॑थि॒वी । \newline
40. ब॒र्॒.हिः पृ॑थि॒वी पृ॑थि॒वी ब॒र्॒.हिर् ब॒र्॒.हिः पृ॑थि॒वी वेदि॒र् वेदिः॑ पृथि॒वी ब॒र्॒.हिर् ब॒र्॒.हिः पृ॑थि॒वी वेदिः॑ । \newline
41. पृ॒थि॒वी वेदि॒र् वेदिः॑ पृथि॒वी पृ॑थि॒वी वेदिः॑ प्र॒जाः प्र॒जा वेदिः॑ पृथि॒वी पृ॑थि॒वी वेदिः॑ प्र॒जाः । \newline
42. वेदिः॑ प्र॒जाः प्र॒जा वेदि॒र् वेदिः॑ प्र॒जा ए॒वैव प्र॒जा वेदि॒र् वेदिः॑ प्र॒जा ए॒व । \newline
43. प्र॒जा ए॒वैव प्र॒जाः प्र॒जा ए॒व पृ॑थि॒व्याम् पृ॑थि॒व्या मे॒व प्र॒जाः प्र॒जा ए॒व पृ॑थि॒व्याम् । \newline
44. प्र॒जा इति॑ प्र - जाः । \newline
45. ए॒व पृ॑थि॒व्याम् पृ॑थि॒व्या मे॒वैव पृ॑थि॒व्याम् प्रति॒ प्रति॑ पृथि॒व्या मे॒वैव पृ॑थि॒व्याम् प्रति॑ । \newline
46. पृ॒थि॒व्याम् प्रति॒ प्रति॑ पृथि॒व्याम् पृ॑थि॒व्याम् प्रति॑ ष्ठापयति स्थापयति॒ प्रति॑ पृथि॒व्याम् पृ॑थि॒व्याम् प्रति॑ ष्ठापयति । \newline
47. प्रति॑ ष्ठापयति स्थापयति॒ प्रति॒ प्रति॑ ष्ठापय॒ त्यन॑तिदृश्ञ॒ मन॑तिदृश्ञꣳ स्थापयति॒ प्रति॒ प्रति॑ ष्ठापय॒ त्यन॑तिदृश्ञम् । \newline
48. स्था॒प॒य॒ त्यन॑तिदृश्ञ॒ मन॑तिदृश्ञꣳ स्थापयति स्थापय॒ त्यन॑तिदृश्ञꣳ स्तृणाति स्तृणा॒ त्यन॑तिदृश्ञꣳ स्थापयति स्थापय॒ त्यन॑तिदृश्ञꣳ स्तृणाति । \newline
49. अन॑तिदृश्ञꣳ स्तृणाति स्तृणा॒ त्यन॑तिदृश्ञ॒ मन॑तिदृश्ञꣳ स्तृणाति प्र॒जया᳚ प्र॒जया᳚ स्तृणा॒ त्यन॑तिदृश्ञ॒ मन॑तिदृश्ञꣳ स्तृणाति प्र॒जया᳚ । \newline
50. अन॑तिदृश्ञ॒मित्यन॑ति - दृ॒श्ञ॒म् । \newline
51. स्तृ॒णा॒ति॒ प्र॒जया᳚ प्र॒जया᳚ स्तृणाति स्तृणाति प्र॒जयै॒वैव प्र॒जया᳚ स्तृणाति स्तृणाति प्र॒जयै॒व । \newline
52. प्र॒जयै॒वैव प्र॒जया᳚ प्र॒जयै॒वैन॑ मेन मे॒व प्र॒जया᳚ प्र॒जयै॒वैन᳚म् । \newline
53. प्र॒जयेति॑ प्र - जया᳚ । \newline
54. ए॒वैन॑ मेन मे॒वैवैन॑म् प॒शुभिः॑ प॒शुभि॑रेन मे॒वैवैन॑म् प॒शुभिः॑ । \newline
55. ए॒न॒म् प॒शुभिः॑ प॒शुभि॑रेन मेनम् प॒शुभि॒ रन॑तिदृश्ञ॒ मन॑तिदृश्ञम् प॒शुभि॑रेन मेनम् प॒शुभि॒ रन॑तिदृश्ञम् । \newline
56. प॒शुभि॒ रन॑तिदृश्ञ॒ मन॑तिदृश्ञम् प॒शुभिः॑ प॒शुभि॒ अन॑तिदृश्ञम् करोति करो॒ त्यन॑तिदृश्ञम् प॒शुभिः॑ प॒शुभि॒ रन॑तिदृश्ञम् करोति । \newline
57. प॒शुभि॒रिति॑ प॒शु - भिः॒ । \newline
58. अन॑तिदृश्ञम् करोति करो॒ त्यन॑तिदृश्ञ॒ मन॑तिदृश्ञम् करो॒त्युत्त॑र॒ मुत्त॑रम् करो॒ त्यन॑तिदृश्ञ॒ मन॑तिदृश्ञम् करो॒त्युत्त॑रम् । \newline
59. अन॑तिदृश्ञ॒मित्यन॑ति - दृ॒श्ञ॒म् । \newline
60. क॒रो॒ त्युत्त॑र॒ मुत्त॑रम् करोति करो॒ त्युत्त॑रम् ब॒र्॒.हिषो॑ ब॒र्॒.हिष॒ उत्त॑रम् करोति करो॒ त्युत्त॑रम् ब॒र्॒.हिषः॑ । \newline
\pagebreak
\markright{ TS 2.6.5.3  \hfill https://www.vedavms.in \hfill}
\addcontentsline{toc}{section}{ TS 2.6.5.3 }
\section*{ TS 2.6.5.3 }

\textbf{TS 2.6.5.3 } \newline
\textbf{Samhita Paata} \newline

-त्युत्त॑रं ब॒र्॒.हिषः॑ प्रस्त॒रꣳ सा॑दयति प्र॒जा वै ब॒र्॒.हि र्यज॑मानः प्रस्त॒रोयज॑मान-मे॒वाय॑जमाना॒-दुत्त॑रं करोति॒ तस्मा॒द्-यज॑मा॒नोऽय॑जमाना॒दुत्त॑रो॒ऽन्तर्द॑धाति॒ व्यावृ॑त्त्या अ॒नक्ति॑ ह॒विष्कृ॑तमे॒वैनꣳ॑ सुव॒र्गं ॅलो॒कं ग॑मयतित्रे॒धाऽन॑क्ति॒ त्रय॑ इ॒मे लो॒का ए॒भ्य ए॒वैनं॑ ॅलो॒केभ्यो॑ऽनक्ति॒ न प्रति॑ शृणाति॒यत् प्र॑तिशृणी॒यादनू᳚र्द्ध्वं भावुकं॒ ॅयज॑मानस्य स्यादु॒परी॑व॒ प्र ह॑र - [  ] \newline

\textbf{Pada Paata} \newline

उत्त॑र॒मित्युत् - त॒र॒म् । ब॒र॒.हिषः॑ । प्र॒स्त॒रमिति॑ प्र - स्त॒रम् । सा॒द॒य॒ति॒ । प्र॒जा इति॑ प्र - जाः । वै । ब॒र्॒.हिः । यज॑मानः । प्र॒स्त॒र इति॑ प्र - स्त॒रः । यज॑मानम् । ए॒व । अय॑जमानात् । उत्त॑र॒मित्युत् - त॒र॒म् । क॒रो॒ति॒ । तस्मा᳚त् । यज॑मानः । अय॑जमानात् । उत्त॑र॒ इत्युत् - त॒रः॒ । अ॒न्तः । द॒धा॒ति॒ । व्यावृ॑त्त्या॒ इति॑ वि-आवृ॑त्त्यै । अ॒नक्ति॑ । ह॒विष्कृ॑त॒मिति॑ ह॒विः - कृ॒त॒म् । ए॒व । ए॒न॒म् । सु॒व॒र्गमिति॑ सुवः - गम् । लो॒कम् । ग॒म॒य॒ति॒ । त्रे॒धा । अ॒न॒क्ति॒ । त्रयः॑ । इ॒मे । लो॒काः । ए॒भ्यः । ए॒व । ए॒न॒म् । लो॒केभ्यः॑ । अ॒न॒क्ति॒ । न । प्रतीति॑ । शृ॒णा॒ति॒ । यत् । प्र॒ति॒शृ॒णी॒यादिति॑ प्रति - शृ॒णी॒यात् । अनू᳚र्द्ध्वं भावुक॒मित्यनू᳚र्द्ध्वं-भा॒वु॒क॒म् । यज॑मानस्य । स्या॒त् । उ॒परि॑ । इ॒व॒ । प्रेति॑ । ह॒र॒ति॒ ।  \newline


\textbf{Krama Paata} \newline

उत्त॑रम् ब॒र्.॒हिषः॑ । उत्त॑र॒मित्युत् - त॒र॒म् । ब॒र्.॒हिषः॑ प्रस्त॒रम् । प्र॒स्त॒रꣳ सा॑दयति । प्र॒स्त॒रमिति॑ प्र - स्त॒रम् । सा॒द॒य॒ति॒ प्र॒जाः । प्र॒जा वै । प्र॒जा इति॑ प्र - जाः । वै ब॒र्.॒हिः । ब॒र्.॒हिर् यज॑मानः । यज॑मानः प्रस्त॒रः । प्र॒स्त॒रो यज॑मानम् । प्र॒स्त॒र इति॑ प्र - स्त॒रः । यज॑मानमे॒व । ए॒वाय॑जमानात् । अय॑जमाना॒दुत्त॑रम् । उत्त॑रम् करोति । उत्त॑र॒मित्युत् - त॒र॒म् । क॒रो॒ति॒ तस्मा᳚त् । तस्मा॒द् यज॑मानः । यज॑मा॒नो ऽय॑जमानात् । अय॑जमाना॒दुत्त॑रः । उत्त॑रो॒ ऽन्तः । उत्त॑र॒ इत्युत् - त॒रः॒ । अ॒न्तर् द॑धाति । द॒धा॒ति॒ व्यावृ॑त्यै । व्यावृ॑त्या अ॒नक्ति॑ । व्यावृ॑त्या॒ इति॑ वि - आवृ॑त्यै । अ॒नक्ति॑ ह॒विष्कृ॑तम् । ह॒विष्कृ॑तमे॒व । ह॒विष्कृ॑त॒मिति॑ ह॒विः - कृ॒त॒म् । ए॒वैन᳚म् । ए॒नꣳ॒॒ सु॒र्व॒र्गम् । सु॒व॒र्गं ॅलो॒कम् । सु॒व॒र्गमिति॑ सुवः - गम् । लो॒कम् ग॑मयति । ग॒म॒य॒ति॒ त्रे॒धा । त्रे॒धा ऽन॑क्ति । अ॒न॒क्ति॒ त्रयः॑ । त्रय॑ इ॒मे । इ॒मे लो॒काः । लो॒का ए॒भ्यः । ए॒भ्य ए॒व । ए॒वैन᳚म् । ए॒नं॒ ॅलो॒केभ्यः॑ । लो॒केभ्यो॑ ऽनक्ति । अ॒न॒क्ति॒ न । न प्रति॑ । प्रति॑ शृणाति । शृ॒णा॒ति॒ यत् । यत् प्र॑तिशृणी॒यात् । प्र॒ति॒शृ॒णी॒यादनू᳚र्द्ध्वम्भावुकम् । प्र॒ति॒शृ॒णी॒यादिति॑ प्रति - शृ॒णी॒यात् । अनू᳚र्द्ध्वंभावुकं॒ ॅयज॑मानस्य । अनू᳚र्द्ध्वंभावुक॒मित्यनू᳚र्द्ध्वम् - भा॒वु॒क॒म् । यज॑मानस्य स्यात् । स्या॒दु॒परि॑ । उ॒परी॑व । इ॒व॒ प्र । प्र ह॑रति । ह॒र॒त्यु॒परि॑ \newline

\textbf{Jatai Paata} \newline

1. उत्त॑रम् ब॒र्॒.हिषो॑ ब॒र्॒.हिष॒ उत्त॑र॒ मुत्त॑रम् ब॒र्॒.हिषः॑ । \newline
2. उत्त॑र॒मित्युत् - त॒र॒म् । \newline
3. ब॒र्॒.हिषः॑ प्रस्त॒रम् प्र॑स्त॒रम् ब॒र्॒.हिषो॑ ब॒र्॒.हिषः॑ प्रस्त॒रम् । \newline
4. प्र॒स्त॒रꣳ सा॑दयति सादयति प्रस्त॒रम् प्र॑स्त॒रꣳ सा॑दयति । \newline
5. प्र॒स्त॒रमिति॑ प्र - स्त॒रम् । \newline
6. सा॒द॒य॒ति॒ प्र॒जाः प्र॒जाः सा॑दयति सादयति प्र॒जाः । \newline
7. प्र॒जा वै वै प्र॒जाः प्र॒जा वै । \newline
8. प्र॒जा इति॑ प्र - जाः । \newline
9. वै ब॒र्॒.हिर् ब॒र्॒.हिर् वै वै ब॒र्॒.हिः । \newline
10. ब॒र्॒.हिर् यज॑मानो॒ यज॑मानो ब॒र्॒.हिर् ब॒र्॒.हिर् यज॑मानः । \newline
11. यज॑मानः प्रस्त॒रः प्र॑स्त॒रो यज॑मानो॒ यज॑मानः प्रस्त॒रः । \newline
12. प्र॒स्त॒रो यज॑मानं॒ ॅयज॑मानम् प्रस्त॒रः प्र॑स्त॒रो यज॑मानम् । \newline
13. प्र॒स्त॒र इति॑ प्र - स्त॒रः । \newline
14. यज॑मान मे॒वैव यज॑मानं॒ ॅयज॑मान मे॒व । \newline
15. ए॒वा य॑जमाना॒ दय॑जमाना दे॒वैवा य॑जमानात् । \newline
16. अय॑जमाना॒ दुत्त॑र॒ मुत्त॑र॒ मय॑जमाना॒ दय॑जमाना॒ दुत्त॑रम् । \newline
17. उत्त॑रम् करोति करो॒ त्युत्त॑र॒ मुत्त॑रम् करोति । \newline
18. उत्त॑र॒मित्युत् - त॒र॒म् । \newline
19. क॒रो॒ति॒ तस्मा॒त् तस्मा᳚त् करोति करोति॒ तस्मा᳚त् । \newline
20. तस्मा॒द् यज॑मानो॒ यज॑मान॒ स्तस्मा॒त् तस्मा॒द् यज॑मानः । \newline
21. यज॑मा॒नो ऽय॑जमाना॒ दय॑जमाना॒द् यज॑मानो॒ यज॑मा॒नो ऽय॑जमानात् । \newline
22. अय॑जमाना॒ दुत्त॑र॒ उत्त॒रो ऽय॑जमाना॒ दय॑जमाना॒ दुत्त॑रः । \newline
23. उत्त॑रो॒ ऽन्त र॒न्त रुत्त॑र॒ उत्त॑रो॒ ऽन्तः । \newline
24. उत्त॑र॒ इत्युत् - त॒रः॒ । \newline
25. अ॒न्तर् द॑धाति दधा त्य॒न्त र॒न्तर् द॑धाति । \newline
26. द॒धा॒ति॒ व्यावृ॑त्त्यै॒ व्यावृ॑त्त्यै दधाति दधाति॒ व्यावृ॑त्त्यै । \newline
27. व्यावृ॑त्त्या अ॒न क्त्य॒नक्ति॒ व्यावृ॑त्त्यै॒ व्यावृ॑त्त्या अ॒नक्ति॑ । \newline
28. व्यावृ॑त्त्या॒ इति॑ वि - आवृ॑त्त्यै । \newline
29. अ॒नक्ति॑ ह॒विष्कृ॑तꣳ ह॒विष्कृ॑त म॒नक्त्य॒नक्ति॑ ह॒विष्कृ॑तम् । \newline
30. ह॒विष्कृ॑त मे॒वैव ह॒विष्कृ॑तꣳ ह॒विष्कृ॑त मे॒व । \newline
31. ह॒विष्कृ॑त॒मिति॑ ह॒विः - कृ॒त॒म् । \newline
32. ए॒वैन॑ मेन मे॒वैवैन᳚म् । \newline
33. ए॒नꣳ॒॒ सु॒व॒र्गꣳ सु॑व॒र्ग मे॑न मेनꣳ सुव॒र्गम् । \newline
34. सु॒व॒र्गम् ॅलो॒कम् ॅलो॒कꣳ सु॑व॒र्गꣳ सु॑व॒र्गम् ॅलो॒कम् । \newline
35. सु॒व॒र्गमिति॑ सुवः - गम् । \newline
36. लो॒कम् ग॑मयति गमयति लो॒कम् ॅलो॒कम् ग॑मयति । \newline
37. ग॒म॒य॒ति॒ त्रे॒धा त्रे॒धा ग॑मयति गमयति त्रे॒धा । \newline
38. त्रे॒धा ऽन॑क्त्यनक्ति त्रे॒धा त्रे॒धा ऽन॑क्ति । \newline
39. अ॒न॒क्ति॒ त्रय॒ स्त्रयो॑ ऽनक्त्यनक्ति॒ त्रयः॑ । \newline
40. त्रय॑ इ॒म इ॒मे त्रय॒ स्त्रय॑ इ॒मे । \newline
41. इ॒मे लो॒का लो॒का इ॒म इ॒मे लो॒काः । \newline
42. लो॒का ए॒भ्य ए॒भ्यो लो॒का लो॒का ए॒भ्यः । \newline
43. ए॒भ्य ए॒वैवैभ्य ए॒भ्य ए॒व । \newline
44. ए॒वैन॑ मेन मे॒वैवैन᳚म् । \newline
45. ए॒न॒म् ॅलो॒केभ्यो॑ लो॒केभ्य॑ एन मेनम् ॅलो॒केभ्यः॑ । \newline
46. लो॒केभ्यो॑ ऽनक्त्यनक्ति लो॒केभ्यो॑ लो॒केभ्यो॑ ऽनक्ति । \newline
47. अ॒न॒क्ति॒ न नान॑ क्त्यनक्ति॒ न । \newline
48. न प्रति॒ प्रति॒ न न प्रति॑ । \newline
49. प्रति॑ शृणाति शृणाति॒ प्रति॒ प्रति॑ शृणाति । \newline
50. शृ॒णा॒ति॒ यद् यच् छृ॑णाति शृणाति॒ यत् । \newline
51. यत् प्र॑तिशृणी॒यात् प्र॑तिशृणी॒याद् यद् यत् प्र॑तिशृणी॒यात् । \newline
52. प्र॒ति॒शृ॒णी॒या दनू᳚र्द्ध्वम्भावुक॒ मनू᳚र्द्ध्वम्भावुकम् प्रतिशृणी॒यात् प्र॑तिशृणी॒या दनू᳚र्द्ध्वम्भावुकम् । \newline
53. प्र॒ति॒शृ॒णी॒यादिति॑ प्रति - शृ॒णी॒यात् । \newline
54. अनू᳚र्द्ध्वम्भावुकं॒ ॅयज॑मानस्य॒ यज॑मान॒स्या नू᳚र्द्ध्वम्भावुक॒ मनू᳚र्द्ध्वम्भावुकं॒ ॅयज॑मानस्य । \newline
55. अनू᳚र्द्ध्वम्भावुक॒मित्यनू᳚र्द्ध्वं - भा॒वु॒क॒म् । \newline
56. यज॑मानस्य स्याथ् स्या॒द् यज॑मानस्य॒ यज॑मानस्य स्यात् । \newline
57. स्या॒ दु॒पर्यु॒परि॑ स्याथ् स्या दु॒परि॑ । \newline
58. उ॒परी॑वे वो॒पर्यु॒प री॑व । \newline
59. इ॒व॒ प्र प्रे वे॑ व॒ प्र । \newline
60. प्र ह॑रति हरति॒ प्र प्र ह॑रति । \newline
61. ह॒र॒ त्यु॒पर्यु॒परि॑ हरति हर त्यु॒परि॑ । \newline

\textbf{Ghana Paata } \newline

1. उत्त॑रम् ब॒र्॒.हिषो॑ ब॒र्॒.हिष॒ उत्त॑र॒ मुत्त॑रम् ब॒र्॒.हिषः॑ प्रस्त॒रम् प्र॑स्त॒रम् ब॒र्॒.हिष॒ उत्त॑र॒ मुत्त॑रम् ब॒र्॒.हिषः॑ प्रस्त॒रम् । \newline
2. उत्त॑र॒मित्युत् - त॒र॒म् । \newline
3. ब॒र्॒.हिषः॑ प्रस्त॒रम् प्र॑स्त॒रम् ब॒र्॒.हिषो॑ ब॒र्॒.हिषः॑ प्रस्त॒रꣳ सा॑दयति सादयति प्रस्त॒रम् ब॒र्॒.हिषो॑ ब॒र्॒.हिषः॑ प्रस्त॒रꣳ सा॑दयति । \newline
4. प्र॒स्त॒रꣳ सा॑दयति सादयति प्रस्त॒रम् प्र॑स्त॒रꣳ सा॑दयति प्र॒जाः प्र॒जाः सा॑दयति प्रस्त॒रम् प्र॑स्त॒रꣳ सा॑दयति प्र॒जाः । \newline
5. प्र॒स्त॒रमिति॑ प्र - स्त॒रम् । \newline
6. सा॒द॒य॒ति॒ प्र॒जाः प्र॒जाः सा॑दयति सादयति प्र॒जा वै वै प्र॒जाः सा॑दयति सादयति प्र॒जा वै । \newline
7. प्र॒जा वै वै प्र॒जाः प्र॒जा वै ब॒र्॒.हिर् ब॒र्॒.हिर् वै प्र॒जाः प्र॒जा वै ब॒र्॒.हिः । \newline
8. प्र॒जा इति॑ प्र - जाः । \newline
9. वै ब॒र्॒.हिर् ब॒र्॒.हिर् वै वै ब॒र्॒.हिर् यज॑मानो॒ यज॑मानो ब॒र्॒.हिर् वै वै ब॒र्॒.हिर् यज॑मानः । \newline
10. ब॒र्॒.हिर् यज॑मानो॒ यज॑मानो ब॒र्॒.हिर् ब॒र्॒.हिर् यज॑मानः प्रस्त॒रः प्र॑स्त॒रो यज॑मानो ब॒र्॒.हिर् ब॒र्॒.हिर् यज॑मानः प्रस्त॒रः । \newline
11. यज॑मानः प्रस्त॒रः प्र॑स्त॒रो यज॑मानो॒ यज॑मानः प्रस्त॒रो यज॑मानं॒ ॅयज॑मानम् प्रस्त॒रो यज॑मानो॒ यज॑मानः प्रस्त॒रो यज॑मानम् । \newline
12. प्र॒स्त॒रो यज॑मानं॒ ॅयज॑मानम् प्रस्त॒रः प्र॑स्त॒रो यज॑मान मे॒वैव यज॑मानम् प्रस्त॒रः प्र॑स्त॒रो यज॑मान मे॒व । \newline
13. प्र॒स्त॒र इति॑ प्र - स्त॒रः । \newline
14. यज॑मान मे॒वैव यज॑मानं॒ ॅयज॑मान मे॒वाय॑जमाना॒ दय॑जमानादे॒व यज॑मानं॒ ॅयज॑मान मे॒वाय॑जमानात् । \newline
15. ए॒वाय॑जमाना॒ दय॑जमाना दे॒वैवा य॑जमाना॒ दुत्त॑र॒ मुत्त॑र॒ मय॑जमाना दे॒वैवा य॑जमाना॒ दुत्त॑रम् । \newline
16. अय॑जमाना॒ दुत्त॑र॒ मुत्त॑र॒ मय॑जमाना॒ दय॑जमाना॒ दुत्त॑रम् करोति करो॒ त्युत्त॑र॒ मय॑जमाना॒ दय॑जमाना॒ दुत्त॑रम् करोति । \newline
17. उत्त॑रम् करोति करो॒ त्युत्त॑र॒ मुत्त॑रम् करोति॒ तस्मा॒त् तस्मा᳚त् करो॒ त्युत्त॑र॒ मुत्त॑रम् करोति॒ तस्मा᳚त् । \newline
18. उत्त॑र॒मित्युत् - त॒र॒म् । \newline
19. क॒रो॒ति॒ तस्मा॒त् तस्मा᳚त् करोति करोति॒ तस्मा॒द् यज॑मानो॒ यज॑मान॒ स्तस्मा᳚त् करोति करोति॒ तस्मा॒द् यज॑मानः । \newline
20. तस्मा॒द् यज॑मानो॒ यज॑मान॒ स्तस्मा॒त् तस्मा॒द् यज॑मा॒नो ऽय॑जमाना॒ दय॑जमाना॒द् यज॑मान॒ स्तस्मा॒त् तस्मा॒द् यज॑मा॒नो ऽय॑जमानात् । \newline
21. यज॑मा॒नो ऽय॑जमाना॒ दय॑जमाना॒द् यज॑मानो॒ यज॑मा॒नो ऽय॑जमाना॒ दुत्त॑र॒ उत्त॒रो ऽय॑जमाना॒द् यज॑मानो॒ यज॑मा॒नो ऽय॑जमाना॒ दुत्त॑रः । \newline
22. अय॑जमाना॒ दुत्त॑र॒ उत्त॒रो ऽय॑जमाना॒ दय॑जमाना॒ दुत्त॑रो॒ ऽन्त र॒न्त रुत्त॒रो ऽय॑जमाना॒ दय॑जमाना॒ दुत्त॑रो॒ ऽन्तः । \newline
23. उत्त॑रो॒ ऽन्त र॒न्त रुत्त॑र॒ उत्त॑रो॒ ऽन्तर् द॑धाति दधात्य॒न्त रुत्त॑र॒ उत्त॑रो॒ ऽन्तर् द॑धाति । \newline
24. उत्त॑र॒ इत्युत् - त॒रः॒ । \newline
25. अ॒न्तर् द॑धाति दधात्य॒न्त र॒न्तर् द॑धाति॒ व्यावृ॑त्त्यै॒ व्यावृ॑त्त्यै दधात्य॒न्त र॒न्तर् द॑धाति॒ व्यावृ॑त्त्यै । \newline
26. द॒धा॒ति॒ व्यावृ॑त्त्यै॒ व्यावृ॑त्त्यै दधाति दधाति॒ व्यावृ॑त्त्या अ॒नक्त्य॒नक्ति॒ व्यावृ॑त्त्यै दधाति दधाति॒ व्यावृ॑त्त्या अ॒नक्ति॑ । \newline
27. व्यावृ॑त्त्या अ॒नक्त्य॒नक्ति॒ व्यावृ॑त्त्यै॒ व्यावृ॑त्त्या अ॒नक्ति॑ ह॒विष्कृ॑तꣳ ह॒विष्कृ॑त म॒नक्ति॒ व्यावृ॑त्त्यै॒ व्यावृ॑त्त्या अ॒नक्ति॑ ह॒विष्कृ॑तम् । \newline
28. व्यावृ॑त्त्या॒ इति॑ वि - आवृ॑त्त्यै । \newline
29. अ॒नक्ति॑ ह॒विष्कृ॑तꣳ ह॒विष्कृ॑त म॒नक्त्य॒नक्ति॑ ह॒विष्कृ॑त मे॒वैव ह॒विष्कृ॑त म॒नक्त्य॒नक्ति॑ ह॒विष्कृ॑त मे॒व । \newline
30. ह॒विष्कृ॑त मे॒वैव ह॒विष्कृ॑तꣳ ह॒विष्कृ॑त मे॒वैन॑ मेन मे॒व ह॒विष्कृ॑तꣳ ह॒विष्कृ॑त मे॒वैन᳚म् । \newline
31. ह॒विष्कृ॑त॒मिति॑ ह॒विः - कृ॒त॒म् । \newline
32. ए॒वैन॑ मेन मे॒वैवैनꣳ॑ सुव॒र्गꣳ सु॑व॒र्ग मे॑न मे॒वैवैनꣳ॑ सुव॒र्गम् । \newline
33. ए॒नꣳ॒॒ सु॒व॒र्गꣳ सु॑व॒र्ग मे॑न मेनꣳ सुव॒र्गम् ॅलो॒कम् ॅलो॒कꣳ सु॑व॒र्ग मे॑न मेनꣳ सुव॒र्गम् ॅलो॒कम् । \newline
34. सु॒व॒र्गम् ॅलो॒कम् ॅलो॒कꣳ सु॑व॒र्गꣳ सु॑व॒र्गम् ॅलो॒कम् ग॑मयति गमयति लो॒कꣳ सु॑व॒र्गꣳ सु॑व॒र्गम् ॅलो॒कम् ग॑मयति । \newline
35. सु॒व॒र्गमिति॑ सुवः - गम् । \newline
36. लो॒कम् ग॑मयति गमयति लो॒कम् ॅलो॒कम् ग॑मयति त्रे॒धा त्रे॒धा ग॑मयति लो॒कम् ॅलो॒कम् ग॑मयति त्रे॒धा । \newline
37. ग॒म॒य॒ति॒ त्रे॒धा त्रे॒धा ग॑मयति गमयति त्रे॒धा ऽन॑क्त्यनक्ति त्रे॒धा ग॑मयति गमयति त्रे॒धा ऽन॑क्ति । \newline
38. त्रे॒धा ऽन॑क्त्यनक्ति त्रे॒धा त्रे॒धा ऽन॑क्ति॒ त्रय॒ स्त्रयो॑ ऽनक्ति त्रे॒धा त्रे॒धा ऽन॑क्ति॒ त्रयः॑ । \newline
39. अ॒न॒क्ति॒ त्रय॒ स्त्रयो॑ ऽनक्त्यनक्ति॒ त्रय॑ इ॒म इ॒मे त्रयो॑ ऽनक्त्यनक्ति॒ त्रय॑ इ॒मे । \newline
40. त्रय॑ इ॒म इ॒मे त्रय॒ स्त्रय॑ इ॒मे लो॒का लो॒का इ॒मे त्रय॒ स्त्रय॑ इ॒मे लो॒काः । \newline
41. इ॒मे लो॒का लो॒का इ॒म इ॒मे लो॒का ए॒भ्य ए॒भ्यो लो॒का इ॒म इ॒मे लो॒का ए॒भ्यः । \newline
42. लो॒का ए॒भ्य ए॒भ्यो लो॒का लो॒का ए॒भ्य ए॒वैवैभ्यो लो॒का लो॒का ए॒भ्य ए॒व । \newline
43. ए॒भ्य ए॒वैवैभ्य ए॒भ्य ए॒वैन॑ मेन मे॒वैभ्य ए॒भ्य ए॒वैन᳚म् । \newline
44. ए॒वैन॑ मेन मे॒वैवैन॑म् ॅलो॒केभ्यो॑ लो॒केभ्य॑ एन मे॒वैवैन॑म् ॅलो॒केभ्यः॑ । \newline
45. ए॒न॒म् ॅलो॒केभ्यो॑ लो॒केभ्य॑ एन मेनम् ॅलो॒केभ्यो॑ ऽनक्त्यनक्ति लो॒केभ्य॑ एन मेनम् ॅलो॒केभ्यो॑ ऽनक्ति । \newline
46. लो॒केभ्यो॑ ऽनक्त्यनक्ति लो॒केभ्यो॑ लो॒केभ्यो॑ ऽनक्ति॒ न नान॑क्ति लो॒केभ्यो॑ लो॒केभ्यो॑ ऽनक्ति॒ न । \newline
47. अ॒न॒क्ति॒ न नान॑क्त्यनक्ति॒ न प्रति॒ प्रति॒ नान॑क्त्यनक्ति॒ न प्रति॑ । \newline
48. न प्रति॒ प्रति॒ न न प्रति॑ शृणाति शृणाति॒ प्रति॒ न न प्रति॑ शृणाति । \newline
49. प्रति॑ शृणाति शृणाति॒ प्रति॒ प्रति॑ शृणाति॒ यद् यच् छृ॑णाति॒ प्रति॒ प्रति॑ शृणाति॒ यत् । \newline
50. शृ॒णा॒ति॒ यद् यच्छृ॑णाति शृणाति॒ यत् प्र॑तिशृणी॒यात् प्र॑तिशृणी॒याद् यच्छृ॑णाति शृणाति॒ यत् प्र॑तिशृणी॒यात् । \newline
51. यत् प्र॑तिशृणी॒यात् प्र॑तिशृणी॒याद् यद् यत् प्र॑तिशृणी॒या दनू᳚र्द्ध्वम्भावुक॒ मनू᳚र्द्ध्वम्भावुकम् प्रतिशृणी॒याद् यद् यत् प्र॑तिशृणी॒या दनू᳚र्द्ध्वम्भावुकम् । \newline
52. प्र॒ति॒शृ॒णी॒या दनू᳚र्द्ध्वम्भावुक॒ मनू᳚र्द्ध्वम्भावुकम् प्रतिशृणी॒यात् प्र॑तिशृणी॒या दनू᳚र्द्ध्वम्भावुकं॒ ॅयज॑मानस्य॒ यज॑मान॒स्या नू᳚र्द्ध्वम्भावुकम् प्रतिशृणी॒यात् प्र॑तिशृणी॒या दनू᳚र्द्ध्वम्भावुकं॒ ॅयज॑मानस्य । \newline
53. प्र॒ति॒शृ॒णी॒यादिति॑ प्रति - शृ॒णी॒यात् । \newline
54. अनू᳚र्द्ध्वम्भावुकं॒ ॅयज॑मानस्य॒ यज॑मान॒स्या नू᳚र्द्ध्वम्भावुक॒ मनू᳚र्द्ध्वम्भावुकं॒ ॅयज॑मानस्य स्याथ् स्या॒द् यज॑मान॒स्या नू᳚र्द्ध्वम्भावुक॒ मनू᳚र्द्ध्वम्भावुकं॒ ॅयज॑मानस्य स्यात् । \newline
55. अनू᳚र्द्ध्वम्भावुक॒मित्यनू᳚र्द्ध्वं - भा॒वु॒क॒म् । \newline
56. यज॑मानस्य स्याथ् स्या॒द् यज॑मानस्य॒ यज॑मानस्य स्या दु॒पर्यु॒परि॑ स्या॒द् यज॑मानस्य॒ यज॑मानस्य स्या दु॒परि॑ । \newline
57. स्या॒ दु॒पर्यु॒परि॑ स्याथ् स्या दु॒परी॑वे वो॒परि॑ स्याथ् स्या दु॒परी॑व । \newline
58. उ॒परी॑वे वो॒ पर्यु॒परी॑व॒ प्र प्रे वो॒ पर्यु॒परी॑व॒ प्र । \newline
59. इ॒व॒ प्र प्रे वे॑ व॒ प्र ह॑रति हरति॒ प्रे वे॑ व॒ प्र ह॑रति । \newline
60. प्र ह॑रति हरति॒ प्र प्र ह॑र त्यु॒पर्यु॒परि॑ हरति॒ प्र प्र ह॑र त्यु॒परि॑ । \newline
61. ह॒र॒ त्यु॒पर्यु॒परि॑ हरति हर त्यु॒परी॑वे वो॒परि॑ हरति हर त्यु॒परी॑व । \newline
\pagebreak
\markright{ TS 2.6.5.4  \hfill https://www.vedavms.in \hfill}
\addcontentsline{toc}{section}{ TS 2.6.5.4 }
\section*{ TS 2.6.5.4 }

\textbf{TS 2.6.5.4 } \newline
\textbf{Samhita Paata} \newline

-त्यु॒परी॑व॒ हि सु॑व॒र्गो लो॒को निय॑च्छति॒ वृष्टि॑मे॒वास्मै॒ निय॑च्छति॒ नात्य॑ग्रं॒ प्र ह॑रे॒द्यदत्य॑ग्रं प्र॒हरे॑द-त्यासा॒रिण्य॑द्ध्व॒र्यो-र्नाशु॑का स्या॒न्न पु॒रस्ता॒त् प्रत्य॑स्ये॒द्यत् पु॒रस्ता᳚त् प्र॒त्यस्ये᳚थ् सुव॒र्गाल्लो॒काद्-यज॑मानं॒ प्रति॑ नुदे॒त् प्राञ्चं॒ प्रह॑रति॒ यज॑मानमे॒व सु॑व॒र्गं ॅलो॒कं ग॑मयति॒ न विष्व॑ञ्चं॒ ॅवि यु॑या॒द्-यद्-विष्व॑ञ्चं ॅवियु॒याथ् - [  ] \newline

\textbf{Pada Paata} \newline

उ॒परि॑ । इ॒व॒ । हि । सु॒व॒र्ग इति॑ सुवः - गः । लो॒कः । नीति॑ । य॒च्छ॒ति॒ । वृष्टि᳚म् । ए॒व । अ॒स्मै॒ । नीति॑ । य॒च्छ॒ति॒ । न । अत्य॑ग्र॒मित्यति॑ - अ॒ग्र॒म् । प्रेति॑ । ह॒रे॒त् । यत् । अत्य॑ग्र॒मित्यति॑ - अ॒ग्र॒म् । प्र॒हरे॒दिति॑ प्र - हरे᳚त् । अ॒त्या॒सा॒रिणीत्य॑ति-आ॒सा॒रिणी᳚ । अ॒द्ध्व॒र्योः । नाशु॑का । स्या॒त् । न । पु॒रस्ता᳚त् । प्रतीति॑ । अ॒स्ये॒त् । यत् । पु॒रस्ता᳚त् । प्र॒त्यस्ये॒दिति॑ प्रति - अस्ये᳚त् । सु॒व॒र्गादिति॑ सुवः - गात् । लो॒कात् । यज॑मानम् । प्रतीति॑ । नु॒दे॒त् । प्राञ्च᳚म् । प्रेति॑ । ह॒र॒ति॒ । यज॑मानम् । ए॒व । सु॒व॒र्गमिति॑ सुवः - गम् । लो॒कम् । ग॒म॒य॒ति॒ । न । विष्व॑ञ्चम् । वीति॑ । यु॒या॒त् । यत् । विष्व॑ञ्चम् । वि॒यु॒यादिति॑ वि-यु॒यात् ।  \newline


\textbf{Krama Paata} \newline

उ॒परी॑व । इ॒व॒ हि । हि सु॑व॒र्गः । सु॒र्व॒र्गो लो॒कः । सु॒व॒र्ग इति॑ सुवः - गः । लो॒को नि । नि य॑च्छति । य॒च्छ॒ति॒ वृष्टि᳚म् । वृष्टि॑मे॒व । ए॒वास्मै᳚ । अ॒स्मै॒ नि । नि य॑च्छति । य॒च्छ॒ति॒ न । नात्य॑ग्रम् । अत्य॑ग्र॒म् प्र । अत्य॑ग्र॒मित्यति॑ - अ॒ग्र॒म् । प्र ह॑रेत् । ह॒रे॒द् यत् । यदत्य॑ग्रम् । अत्य॑ग्रम् प्र॒हरे᳚त् । अत्य॑ग्र॒मित्यति॑ - अ॒ग्र॒म् । प्र॒हरे॑दत्यासा॒रिणी᳚ । प्र॒हरे॒दिति॑ प्र - हरे᳚त् । अ॒त्या॒सा॒रिण्य॑द्ध्व॒र्योः । अ॒त्या॒सा॒रिणीत्य॑ति - आ॒सा॒रिणी᳚ । अ॒द्ध्व॒र्योर् नाशु॑का । नाशु॑का स्यात् । स्या॒न् न । न पु॒रस्ता᳚त् । पु॒रस्ता॒त् प्रति॑ । प्रत्य॑स्येत् । अ॒स्ये॒द् यत् । यत् पु॒रस्ता᳚त् । पु॒रस्ता᳚त् प्र॒त्यस्ये᳚त् । प्र॒त्यस्ये᳚थ् सुव॒र्गात् । प्र॒त्यस्ये॒दिति॑ प्रति - अस्ये᳚त् । सु॒व॒र्गाल्लो॒कात् । सु॒व॒र्गादिति॑ सुवः - गात् । लो॒काद् यज॑मानम् । यज॑मान॒म् प्रति॑ । प्रति॑ नुदेत् । नु॒दे॒त् प्राञ्च᳚म् । प्राञ्च॒म् प्र । प्र ह॑रति । ह॒र॒ति॒ यज॑मानम् । यज॑मानमे॒व । ए॒व सु॑व॒र्गम् । सु॒व॒र्गं ॅलो॒कम् । सु॒व॒र्गमिति॑ सुवः - गम् । लो॒कम् ग॑मयति । ग॒म॒य॒ति॒ न । न विष्व॑ञ्चम् । विष्व॑ञ्च॒म् ॅवि । वि यु॑यात् । यु॒या॒द् यत् । यद् विष्व॑ञ्चम् । विष्व॑ञ्चम् ॅवियु॒यात् । वि॒यु॒याथ् स्त्री । वि॒यु॒यादिति॑ वि - यु॒यात् \newline

\textbf{Jatai Paata} \newline

1. उ॒परी॑वे वो॒पर्यु॒प री॑व । \newline
2. इ॒व॒ हि हीवे॑ व॒ हि । \newline
3. हि सु॑व॒र्गः सु॑व॒र्गो हि हि सु॑व॒र्गः । \newline
4. सु॒व॒र्गो लो॒को लो॒कः सु॑व॒र्गः सु॑व॒र्गो लो॒कः । \newline
5. सु॒व॒र्ग इति॑ सुवः - गः । \newline
6. लो॒को नि नि लो॒को लो॒को नि । \newline
7. नि य॑च्छति यच्छति॒ नि नि य॑च्छति । \newline
8. य॒च्छ॒ति॒ वृष्टिं॒ ॅवृष्टिं॑ ॅयच्छति यच्छति॒ वृष्टि᳚म् । \newline
9. वृष्टि॑ मे॒वैव वृष्टिं॒ ॅवृष्टि॑ मे॒व । \newline
10. ए॒वास्मा॑ अस्मा ए॒वैवास्मै᳚ । \newline
11. अ॒स्मै॒ नि न्य॑स्मा अस्मै॒ नि । \newline
12. नि य॑च्छति यच्छति॒ नि नि य॑च्छति । \newline
13. य॒च्छ॒ति॒ न न य॑च्छति यच्छति॒ न । \newline
14. नात्य॑ग्र॒ मत्य॑ग्र॒म् न नात्य॑ग्रम् । \newline
15. अत्य॑ग्र॒म् प्र प्रात्य॑ग्र॒ मत्य॑ग्र॒म् प्र । \newline
16. अत्य॑ग्र॒मित्यति॑ - अ॒ग्र॒म् । \newline
17. प्र ह॑रे द्धरे॒त् प्र प्र ह॑रेत् । \newline
18. ह॒रे॒द् यद् य द्ध॑रे द्धरे॒द् यत् । \newline
19. यद त्य॑ग्र॒ मत्य॑ग्रं॒ ॅयद् यद त्य॑ग्रम् । \newline
20. अत्य॑ग्रम् प्र॒हरे᳚त् प्र॒हरे॒ दत्य॑ग्र॒ मत्य॑ग्रम् प्र॒हरे᳚त् । \newline
21. अत्य॑ग्र॒मित्यति॑ - अ॒ग्र॒म् । \newline
22. प्र॒हरे॑ दत्यासा॒रि ण्य॑त्यासा॒रिणी᳚ प्र॒हरे᳚त् प्र॒हरे॑ दत्यासा॒रिणी᳚ । \newline
23. प्र॒हरे॒दिति॑ प्र - हरे᳚त् । \newline
24. अ॒त्या॒सा॒रि ण्य॑द्ध्व॒र्यो र॑द्ध्व॒र्यो र॑त्यासा॒रि ण्य॑त्यासा॒रि ण्य॑द्ध्व॒र्योः । \newline
25. अ॒त्या॒सा॒रिणीत्य॑ति - आ॒सा॒रिणी᳚ । \newline
26. अ॒द्ध्व॒र्योर् नाशु॑का॒ नाशु॑का ऽद्ध्व॒र्यो र॑द्ध्व॒र्योर् नाशु॑का । \newline
27. नाशु॑का स्याथ् स्या॒न् नाशु॑का॒ नाशु॑का स्यात् । \newline
28. स्या॒न् न न स्या᳚थ् स्या॒न् न । \newline
29. न पु॒रस्ता᳚त् पु॒रस्ता॒न् न न पु॒रस्ता᳚त् । \newline
30. पु॒रस्ता॒त् प्रति॒ प्रति॑ पु॒रस्ता᳚त् पु॒रस्ता॒त् प्रति॑ । \newline
31. प्रत्य॑स्ये दस्ये॒त् प्रति॒ प्रत्य॑स्येत् । \newline
32. अ॒स्ये॒द् यद् यद॑स्ये दस्ये॒द् यत् । \newline
33. यत् पु॒रस्ता᳚त् पु॒रस्ता॒द् यद् यत् पु॒रस्ता᳚त् । \newline
34. पु॒रस्ता᳚त् प्र॒त्यस्ये᳚त् प्र॒त्यस्ये᳚त् पु॒रस्ता᳚त् पु॒रस्ता᳚त् प्र॒त्यस्ये᳚त् । \newline
35. प्र॒त्यस्ये᳚थ् सुव॒र्गाथ् सु॑व॒र्गात् प्र॒त्यस्ये᳚त् प्र॒त्यस्ये᳚थ् सुव॒र्गात् । \newline
36. प्र॒त्यस्ये॒दिति॑ प्रति - अस्ये᳚त् । \newline
37. सु॒व॒र्गा ल्लो॒का ल्लो॒काथ् सु॑व॒र्गाथ् सु॑व॒र्गा ल्लो॒कात् । \newline
38. सु॒व॒र्गादिति॑ सुवः - गात् । \newline
39. लो॒काद् यज॑मानं॒ ॅयज॑मानम् ॅलो॒का ल्लो॒काद् यज॑मानम् । \newline
40. यज॑मान॒म् प्रति॒ प्रति॒ यज॑मानं॒ ॅयज॑मान॒म् प्रति॑ । \newline
41. प्रति॑ नुदेन् नुदे॒त् प्रति॒ प्रति॑ नुदेत् । \newline
42. नु॒दे॒त् प्राञ्च॒म् प्राञ्च॑म् नुदेन् नुदे॒त् प्राञ्च᳚म् । \newline
43. प्राञ्च॒म् प्र प्र प्राञ्च॒म् प्राञ्च॒म् प्र । \newline
44. प्र ह॑रति हरति॒ प्र प्र ह॑रति । \newline
45. ह॒र॒ति॒ यज॑मानं॒ ॅयज॑मानꣳ हरति हरति॒ यज॑मानम् । \newline
46. यज॑मान मे॒वैव यज॑मानं॒ ॅयज॑मान मे॒व । \newline
47. ए॒व सु॑व॒र्गꣳ सु॑व॒र्ग मे॒वैव सु॑व॒र्गम् । \newline
48. सु॒व॒र्गम् ॅलो॒कम् ॅलो॒कꣳ सु॑व॒र्गꣳ सु॑व॒र्गम् ॅलो॒कम् । \newline
49. सु॒व॒र्गमिति॑ सुवः - गम् । \newline
50. लो॒कम् ग॑मयति गमयति लो॒कम् ॅलो॒कम् ग॑मयति । \newline
51. ग॒म॒य॒ति॒ न न ग॑मयति गमयति॒ न । \newline
52. न विष्व॑ञ्चं॒ ॅविष्व॑ञ्च॒म् न न विष्व॑ञ्चम् । \newline
53. विष्व॑ञ्चं॒ ॅवि वि विष्व॑ञ्चं॒ ॅविष्व॑ञ्चं॒ ॅवि । \newline
54. वि यु॑याद् युया॒द् वि वि यु॑यात् । \newline
55. यु॒या॒द् यद् यद् यु॑याद् युया॒द् यत् । \newline
56. यद् विष्व॑ञ्चं॒ ॅविष्व॑ञ्चं॒ ॅयद् यद् विष्व॑ञ्चम् । \newline
57. विष्व॑ञ्चं ॅवियु॒याद् वि॑यु॒याद् विष्व॑ञ्चं॒ ॅविष्व॑ञ्चं ॅवियु॒यात् । \newline
58. वि॒यु॒याथ् स्त्री स्त्री वि॑यु॒याद् वि॑यु॒याथ् स्त्री । \newline
59. वि॒यु॒यादिति॑ वि - यु॒यात् । \newline

\textbf{Ghana Paata } \newline

1. उ॒परी॑वे वो॒ पर्यु॒परी॑व॒ हि हीवो॒ पर्यु॒परी॑व॒ हि । \newline
2. इ॒व॒ हि हीवे॑ व॒ हि सु॑व॒र्गः सु॑व॒र्गो हीवे॑ व॒ हि सु॑व॒र्गः । \newline
3. हि सु॑व॒र्गः सु॑व॒र्गो हि हि सु॑व॒र्गो लो॒को लो॒कः सु॑व॒र्गो हि हि सु॑व॒र्गो लो॒कः । \newline
4. सु॒व॒र्गो लो॒को लो॒कः सु॑व॒र्गः सु॑व॒र्गो लो॒को नि नि लो॒कः सु॑व॒र्गः सु॑व॒र्गो लो॒को नि । \newline
5. सु॒व॒र्ग इति॑ सुवः - गः । \newline
6. लो॒को नि नि लो॒को लो॒को नि य॑च्छति यच्छति॒ नि लो॒को लो॒को नि य॑च्छति । \newline
7. नि य॑च्छति यच्छति॒ नि नि य॑च्छति॒ वृष्टिं॒ ॅवृष्टिं॑ ॅयच्छति॒ नि नि य॑च्छति॒ वृष्टि᳚म् । \newline
8. य॒च्छ॒ति॒ वृष्टिं॒ ॅवृष्टिं॑ ॅयच्छति यच्छति॒ वृष्टि॑ मे॒वैव वृष्टिं॑ ॅयच्छति यच्छति॒ वृष्टि॑ मे॒व । \newline
9. वृष्टि॑ मे॒वैव वृष्टिं॒ ॅवृष्टि॑ मे॒वास्मा॑ अस्मा ए॒व वृष्टिं॒ ॅवृष्टि॑ मे॒वास्मै᳚ । \newline
10. ए॒वास्मा॑ अस्मा ए॒वैवास्मै॒ नि न्य॑स्मा ए॒वैवास्मै॒ नि । \newline
11. अ॒स्मै॒ नि न्य॑स्मा अस्मै॒ नि य॑च्छति यच्छति॒ न्य॑स्मा अस्मै॒ नि य॑च्छति । \newline
12. नि य॑च्छति यच्छति॒ नि नि य॑च्छति॒ न न य॑च्छति॒ नि नि य॑च्छति॒ न । \newline
13. य॒च्छ॒ति॒ न न य॑च्छति यच्छति॒ नात्य॑ग्र॒ मत्य॑ग्र॒म् न य॑च्छति यच्छति॒ नात्य॑ग्रम् । \newline
14. नात्य॑ग्र॒ मत्य॑ग्र॒म् न नात्य॑ग्र॒म् प्र प्रात्य॑ग्र॒म् न नात्य॑ग्र॒म् प्र । \newline
15. अत्य॑ग्र॒म् प्र प्रात्य॑ग्र॒ मत्य॑ग्र॒म् प्र ह॑रे द्धरे॒त् प्रात्य॑ग्र॒ मत्य॑ग्र॒म् प्र ह॑रेत् । \newline
16. अत्य॑ग्र॒मित्यति॑ - अ॒ग्र॒म् । \newline
17. प्र ह॑रे द्धरे॒त् प्र प्र ह॑रे॒द् यद् यद्ध॑रे॒त् प्र प्र ह॑रे॒द् यत् । \newline
18. ह॒रे॒द् यद् यद्ध॑रे द्धरे॒द् यदत्य॑ग्र॒ मत्य॑ग्रं॒ ॅयद्ध॑रे द्धरे॒द् यदत्य॑ग्रम् । \newline
19. यदत्य॑ग्र॒ मत्य॑ग्रं॒ ॅयद् यदत्य॑ग्रम् प्र॒हरे᳚त् प्र॒हरे॒ दत्य॑ग्रं॒ ॅयद् यदत्य॑ग्रम् प्र॒हरे᳚त् । \newline
20. अत्य॑ग्रम् प्र॒हरे᳚त् प्र॒हरे॒ दत्य॑ग्र॒ मत्य॑ग्रम् प्र॒हरे॑ दत्यासा॒रि ण्य॑त्यासा॒रिणी᳚ प्र॒हरे॒ दत्य॑ग्र॒ मत्य॑ग्रम् प्र॒हरे॑दत्यासा॒रिणी᳚ । \newline
21. अत्य॑ग्र॒मित्यति॑ - अ॒ग्र॒म् । \newline
22. प्र॒हरे॑ दत्यासा॒रि ण्य॑त्यासा॒रिणी᳚ प्र॒हरे᳚त् प्र॒हरे॑ दत्यासा॒रि ण्य॑द्ध्व॒र्यो र॑द्ध्व॒र्यो र॑त्यासा॒रिणी᳚ प्र॒हरे᳚त् प्र॒हरे॑ दत्यासा॒रि ण्य॑द्ध्व॒र्योः । \newline
23. प्र॒हरे॒दिति॑ प्र - हरे᳚त् । \newline
24. अ॒त्या॒सा॒रि ण्य॑द्ध्व॒र्यो र॑द्ध्व॒र्यो र॑त्यासा॒रि ण्य॑त्यासा॒रि ण्य॑द्ध्व॒र्योर् नाशु॑का॒ नाशु॑का ऽद्ध्व॒र्यो र॑त्यासा॒रि ण्य॑त्यासा॒रि ण्य॑द्ध्व॒र्योर् नाशु॑का । \newline
25. अ॒त्या॒सा॒रिणीत्य॑ति - आ॒सा॒रिणी᳚ । \newline
26. अ॒द्ध्व॒र्योर् नाशु॑का॒ नाशु॑का ऽद्ध्व॒र्यो र॑द्ध्व॒र्योर् नाशु॑का स्याथ् स्या॒न् नाशु॑का ऽद्ध्व॒र्यो र॑द्ध्व॒र्योर् नाशु॑का स्यात् । \newline
27. नाशु॑का स्याथ् स्या॒न् नाशु॑का॒ नाशु॑का स्या॒न् न न स्या॒न् नाशु॑का॒ नाशु॑का स्या॒न् न । \newline
28. स्या॒न् न न स्या᳚थ् स्या॒न् न पु॒रस्ता᳚त् पु॒रस्ता॒न् न स्या᳚थ् स्या॒न् न पु॒रस्ता᳚त् । \newline
29. न पु॒रस्ता᳚त् पु॒रस्ता॒न् न न पु॒रस्ता॒त् प्रति॒ प्रति॑ पु॒रस्ता॒न् न न पु॒रस्ता॒त् प्रति॑ । \newline
30. पु॒रस्ता॒त् प्रति॒ प्रति॑ पु॒रस्ता᳚त् पु॒रस्ता॒त् प्रत्य॑स्ये दस्ये॒त् प्रति॑ पु॒रस्ता᳚त् पु॒रस्ता॒त् प्रत्य॑स्येत् । \newline
31. प्रत्य॑स्ये दस्ये॒त् प्रति॒ प्रत्य॑स्ये॒द् यद् यद॑स्ये॒त् प्रति॒ प्रत्य॑स्ये॒द् यत् । \newline
32. अ॒स्ये॒द् यद् यद॑स्ये दस्ये॒द् यत् पु॒रस्ता᳚त् पु॒रस्ता॒द् यद॑स्ये दस्ये॒द् यत् पु॒रस्ता᳚त् । \newline
33. यत् पु॒रस्ता᳚त् पु॒रस्ता॒द् यद् यत् पु॒रस्ता᳚त् प्र॒त्यस्ये᳚त् प्र॒त्यस्ये᳚त् पु॒रस्ता॒द् यद् यत् पु॒रस्ता᳚त् प्र॒त्यस्ये᳚त् । \newline
34. पु॒रस्ता᳚त् प्र॒त्यस्ये᳚त् प्र॒त्यस्ये᳚त् पु॒रस्ता᳚त् पु॒रस्ता᳚त् प्र॒त्यस्ये᳚थ् सुव॒र्गाथ् सु॑व॒र्गात् प्र॒त्यस्ये᳚त् पु॒रस्ता᳚त् पु॒रस्ता᳚त् प्र॒त्यस्ये᳚थ् सुव॒र्गात् । \newline
35. प्र॒त्यस्ये᳚थ् सुव॒र्गाथ् सु॑व॒र्गात् प्र॒त्यस्ये᳚त् प्र॒त्यस्ये᳚थ् सुव॒र्गा ल्लो॒का ल्लो॒काथ् सु॑व॒र्गात् प्र॒त्यस्ये᳚त् प्र॒त्यस्ये᳚थ् सुव॒र्गा ल्लो॒कात् । \newline
36. प्र॒त्यस्ये॒दिति॑ प्रति - अस्ये᳚त् । \newline
37. सु॒व॒र्गा ल्लो॒का ल्लो॒काथ् सु॑व॒र्गाथ् सु॑व॒र्गा ल्लो॒काद् यज॑मानं॒ ॅयज॑मानम् ॅलो॒काथ् सु॑व॒र्गाथ् सु॑व॒र्गा ल्लो॒काद् यज॑मानम् । \newline
38. सु॒व॒र्गादिति॑ सुवः - गात् । \newline
39. लो॒काद् यज॑मानं॒ ॅयज॑मानम् ॅलो॒का ल्लो॒काद् यज॑मान॒म् प्रति॒ प्रति॒ यज॑मानम् ॅलो॒का ल्लो॒काद् यज॑मान॒म् प्रति॑ । \newline
40. यज॑मान॒म् प्रति॒ प्रति॒ यज॑मानं॒ ॅयज॑मान॒म् प्रति॑ नुदेन् नुदे॒त् प्रति॒ यज॑मानं॒ ॅयज॑मान॒म् प्रति॑ नुदेत् । \newline
41. प्रति॑ नुदेन् नुदे॒त् प्रति॒ प्रति॑ नुदे॒त् प्राञ्च॒म् प्राञ्च॑म् नुदे॒त् प्रति॒ प्रति॑ नुदे॒त् प्राञ्च᳚म् । \newline
42. नु॒दे॒त् प्राञ्च॒म् प्राञ्च॑म् नुदेन् नुदे॒त् प्राञ्च॒म् प्र प्र प्राञ्च॑म् नुदेन् नुदे॒त् प्राञ्च॒म् प्र । \newline
43. प्राञ्च॒म् प्र प्र प्राञ्च॒म् प्राञ्च॒म् प्र ह॑रति हरति॒ प्र प्राञ्च॒म् प्राञ्च॒म् प्र ह॑रति । \newline
44. प्र ह॑रति हरति॒ प्र प्र ह॑रति॒ यज॑मानं॒ ॅयज॑मानꣳ हरति॒ प्र प्र ह॑रति॒ यज॑मानम् । \newline
45. ह॒र॒ति॒ यज॑मानं॒ ॅयज॑मानꣳ हरति हरति॒ यज॑मान मे॒वैव यज॑मानꣳ हरति हरति॒ यज॑मान मे॒व । \newline
46. यज॑मान मे॒वैव यज॑मानं॒ ॅयज॑मान मे॒व सु॑व॒र्गꣳ सु॑व॒र्ग मे॒व यज॑मानं॒ ॅयज॑मान मे॒व सु॑व॒र्गम् । \newline
47. ए॒व सु॑व॒र्गꣳ सु॑व॒र्ग मे॒वैव सु॑व॒र्गम् ॅलो॒कम् ॅलो॒कꣳ सु॑व॒र्ग मे॒वैव सु॑व॒र्गम् ॅलो॒कम् । \newline
48. सु॒व॒र्गम् ॅलो॒कम् ॅलो॒कꣳ सु॑व॒र्गꣳ सु॑व॒र्गम् ॅलो॒कम् ग॑मयति गमयति लो॒कꣳ सु॑व॒र्गꣳ सु॑व॒र्गम् ॅलो॒कम् ग॑मयति । \newline
49. सु॒व॒र्गमिति॑ सुवः - गम् । \newline
50. लो॒कम् ग॑मयति गमयति लो॒कम् ॅलो॒कम् ग॑मयति॒ न न ग॑मयति लो॒कम् ॅलो॒कम् ग॑मयति॒ न । \newline
51. ग॒म॒य॒ति॒ न न ग॑मयति गमयति॒ न विष्व॑ञ्चं॒ ॅविष्व॑ञ्च॒म् न ग॑मयति गमयति॒ न विष्व॑ञ्चम् । \newline
52. न विष्व॑ञ्चं॒ ॅविष्व॑ञ्च॒म् न न विष्व॑ञ्चं॒ ॅवि वि विष्व॑ञ्च॒म् न न विष्व॑ञ्चं॒ ॅवि । \newline
53. विष्व॑ञ्चं॒ ॅवि वि विष्व॑ञ्चं॒ ॅविष्व॑ञ्चं॒ ॅवि यु॑याद् युया॒द् वि विष्व॑ञ्चं॒ ॅविष्व॑ञ्चं॒ ॅवि यु॑यात् । \newline
54. वि यु॑याद् युया॒द् वि वि यु॑या॒द् यद् यद् यु॑या॒द् वि वि यु॑या॒द् यत् । \newline
55. यु॒या॒द् यद् यद् यु॑याद् युया॒द् यद् विष्व॑ञ्चं॒ ॅविष्व॑ञ्चं॒ ॅयद् यु॑याद् युया॒द् यद् विष्व॑ञ्चम् । \newline
56. यद् विष्व॑ञ्चं॒ ॅविष्व॑ञ्चं॒ ॅयद् यद् विष्व॑ञ्चं ॅवियु॒याद् वि॑यु॒याद् विष्व॑ञ्चं॒ ॅयद् यद् विष्व॑ञ्चं ॅवियु॒यात् । \newline
57. विष्व॑ञ्चं ॅवियु॒याद् वि॑यु॒याद् विष्व॑ञ्चं॒ ॅविष्व॑ञ्चं ॅवियु॒याथ् स्त्री स्त्री वि॑यु॒याद् विष्व॑ञ्चं॒ ॅविष्व॑ञ्चं ॅवियु॒याथ् स्त्री । \newline
58. वि॒यु॒याथ् स्त्री स्त्री वि॑यु॒याद् वि॑यु॒याथ् स्त्र्य॑स्यास्य॒ स्त्री वि॑यु॒याद् वि॑यु॒याथ् स्त्र्य॑स्य । \newline
59. वि॒यु॒यादिति॑ वि - यु॒यात् । \newline
\pagebreak
\markright{ TS 2.6.5.5  \hfill https://www.vedavms.in \hfill}
\addcontentsline{toc}{section}{ TS 2.6.5.5 }
\section*{ TS 2.6.5.5 }

\textbf{TS 2.6.5.5 } \newline
\textbf{Samhita Paata} \newline

स्त्र्य॑स्य जायेतो॒र्द्ध्वमुद्यौ᳚त्यू॒र्द्ध्वमि॑व॒ हि पुꣳ॒॒सः पुमा॑ने॒वास्य॑ जायते॒ यथ् स्फ्येन॑ वोपवे॒षेण॑ वा योयु॒प्येत॒ स्तृति॑रे॒वास्य॒ सा हस्ते॑न योयुप्यते॒ यज॑मानस्य गोपी॒थाय॑ ब्रह्मवा॒दिनो॑ वदन्ति॒ किं ॅय॒ज्ञ्स्य॒ यज॑मान॒ इति॑ प्रस्त॒र इति॒ तस्य॒ क्व॑ सुव॒र्गो लो॒क इत्या॑हव॒नीय॒ इति॑ ब्रूया॒द्यत् प्र॑स्त॒रमा॑हव॒नीये᳚ प्र॒हर॑ति॒ यज॑मानमे॒व - [  ] \newline

\textbf{Pada Paata} \newline

स्त्री । अ॒स्य॒ । जा॒ये॒त॒ । ऊ॒र्द्ध्वम् । उदिति॑ । यौ॒ति॒ । ऊ॒र्द्ध्वम् । इ॒व॒ । हि । पुꣳ॒॒सः । पुमान्॑ । ए॒व । अ॒स्य॒ । जा॒य॒ते॒ । यत् । स्फ्येन॑ । वा॒ । उ॒प॒वे॒षेणेत्यु॑प-वे॒षेण॑ । वा॒ । यो॒यु॒प्येत॑ । स्तृतिः॑ । ए॒व । अ॒स्य॒ । सा । हस्ते॑न । यो॒यु॒प्य॒ते॒ । यज॑मानस्य । गो॒पी॒थाय॑ । ब्र॒ह्म॒वा॒दिन॒ इति॑ ब्रह्म - वा॒दिनः॑ । व॒द॒न्ति॒ । किम् । य॒ज्ञ्स्य॑ । यज॑मानः । इति॑ । प्र॒स्त॒र इति॑ प्र-स्त॒रः । इति॑ । तस्य॑ । क्व॑ । सु॒व॒र्ग इति॑ सुवः- गः । लो॒कः । इति॑ । आ॒ह॒व॒नीय॒ इत्या᳚ - ह॒व॒नीयः॑ । इति॑ । ब्रू॒या॒त् । यत् । प्र॒स्त॒रमिति॑ प्र - स्त॒रम् । आ॒ह॒व॒नीय॒ इत्या᳚ - ह॒व॒नीये᳚ । प्र॒हर॒तीति॑ प्र - हर॑ति । यज॑मानम् । ए॒व ।  \newline


\textbf{Krama Paata} \newline

स्त्य्र॑स्य । अ॒स्य॒ जा॒ये॒त॒ । जा॒ये॒तो॒र्द्ध्वम् । ऊ॒र्द्ध्वमुत् । उद् यौ॑ति । यौ॒त्यू॒र्द्ध्वम् । ऊ॒र्द्ध्वमि॑व । इ॒व॒ हि । हि पुꣳ॒॒सः । पुꣳ॒॒सः पुमान्॑ । पुमा॑ने॒व । ए॒वास्य॑ । अ॒स्य॒ जा॒य॒ते॒ । जा॒य॒ते॒ यत् । यथ् स्फ्येन॑ । स्फ्येन॑ वा । वो॒प॒वे॒षेण॑ । उ॒प॒वे॒षेण॑ वा । उ॒प॒वे॒षेणेत्यु॑प - वे॒षेण॑ । वा॒ यो॒यु॒प्येत॑ । यो॒यु॒प्येत॒ स्तृतिः॑ । स्तृति॑रे॒व । ए॒वास्य॑ । अ॒स्य॒ सा । सा हस्ते॑न । हस्ते॑न योयुप्यते । यो॒यु॒प्य॒ते॒ यज॑मानस्य । यज॑मानस्य गोपी॒थाय॑ । गो॒पी॒थाय॑ ब्रह्मवा॒दिनः॑ । ब्र॒ह्म॒वा॒दिनो॑ वदन्ति । ब्र॒ह्म॒वा॒दिन॒ इति॑ ब्रह्म - वा॒दिनः॑ । व॒द॒न्ति॒ किम् । किं ॅय॒ज्ञ्स्य॑ । य॒ज्ञ्स्य॒ यज॑मानः । यज॑मान॒ इति॑ । इति॑ प्रस्त॒रः । प्र॒स्त॒र इति॑ । प्र॒स्त॒र इति॑ प्र - स्त॒रः । इति॒ तस्य॑ । तस्य॒ क्व॑ । क्व॑ सुव॒र्गः । सु॒व॒र्गो लो॒कः । सु॒व॒र्ग इति॑ सुवः - गः । लो॒क इति॑ । इत्या॑हव॒नीयः॑ । आ॒ह॒व॒नीय॒ इति॑ । आ॒ह॒व॒नीय॒ इत्या᳚ - ह॒व॒नीयः॑ । इति॑ ब्रूयात् । ब्रू॒या॒द् यत् । यत् प्र॑स्त॒रम् । प्र॒स्त॒रमा॑हव॒नीये᳚ । प्र॒स्त॒रमिति॑ प्र - स्त॒रम् । आ॒ह॒व॒नीये᳚ प्र॒हर॑ति । आ॒ह॒व॒नीय॒ इत्या᳚ - ह॒व॒नीये᳚ । प्र॒हर॑ति॒ यज॑मानम् । प्र॒हर॒तीति॑ प्र - हर॑ति । यज॑मानमे॒व । ए॒व सु॑व॒र्गम् \newline

\textbf{Jatai Paata} \newline

1. स्त्र्य॑स्यास्य॒ स्त्री स्त्र्य॑स्य । \newline
2. अ॒स्य॒ जा॒ये॒त॒ जा॒ये॒ता॒ स्या॒स्य॒ जा॒ये॒त॒ । \newline
3. जा॒ये॒ तो॒र्द्ध्व मू॒र्द्ध्वम् जा॑येत जाये तो॒र्द्ध्वम् । \newline
4. ऊ॒र्द्ध्व मुदु दू॒र्द्ध्व मू॒र्द्ध्व मुत् । \newline
5. उद् यौ॑ति यौ॒ त्युदुद् यौ॑ति । \newline
6. यौ॒त्यू॒र्द्ध्व मू॒र्द्ध्वं ॅयौ॑ति यौत्यू॒र्द्ध्वम् । \newline
7. ऊ॒र्द्ध्व मि॑वे वो॒र्द्ध्व मू॒र्द्ध्व मि॑व । \newline
8. इ॒व॒ हि हीवे॑ व॒ हि । \newline
9. हि पुꣳ॒॒सः पुꣳ॒॒सो हि हि पुꣳ॒॒सः । \newline
10. पुꣳ॒॒सः पुमा॒न् पुमा᳚न् पुꣳ॒॒सः पुꣳ॒॒सः पुमान्॑ । \newline
11. पुमा॑ ने॒वैव पुमा॒न् पुमा॑ ने॒व । \newline
12. ए॒वास्या᳚ स्यै॒वै वास्य॑ । \newline
13. अ॒स्य॒ जा॒य॒ते॒ जा॒य॒ते॒ ऽस्या॒स्य॒ जा॒य॒ते॒ । \newline
14. जा॒य॒ते॒ यद् यज् जा॑यते जायते॒ यत् । \newline
15. यथ् स्फ्येन॒ स्फ्येन॒ यद् यथ् स्फ्येन॑ । \newline
16. स्फ्येन॑ वा वा॒ स्फ्येन॒ स्फ्येन॑ वा । \newline
17. वो॒प॒वे॒षे णो॑पवे॒षेण॑ वा वोपवे॒षेण॑ । \newline
18. उ॒प॒वे॒षेण॑ वा वोपवे॒षे णो॑पवे॒षेण॑ वा । \newline
19. उ॒प॒वे॒षेणेत्यु॑प - वे॒षेण॑ । \newline
20. वा॒ यो॒यु॒प्येत॑ योयु॒प्येत॑ वा वा योयु॒प्येत॑ । \newline
21. यो॒यु॒प्येत॒ स्तृतिः॒ स्तृति॑र् योयु॒प्येत॑ योयु॒प्येत॒ स्तृतिः॑ । \newline
22. स्तृति॑ रे॒वैव स्तृतिः॒ स्तृति॑ रे॒व । \newline
23. ए॒वास्या᳚ स्यै॒वै वास्य॑ । \newline
24. अ॒स्य॒ सा सा ऽस्या᳚स्य॒ सा । \newline
25. सा हस्ते॑न॒ हस्ते॑न॒ सा सा हस्ते॑न । \newline
26. हस्ते॑न योयुप्यते योयुप्यते॒ हस्ते॑न॒ हस्ते॑न योयुप्यते । \newline
27. यो॒यु॒प्य॒ते॒ यज॑मानस्य॒ यज॑मानस्य योयुप्यते योयुप्यते॒ यज॑मानस्य । \newline
28. यज॑मानस्य गोपी॒थाय॑ गोपी॒थाय॒ यज॑मानस्य॒ यज॑मानस्य गोपी॒थाय॑ । \newline
29. गो॒पी॒थाय॑ ब्रह्मवा॒दिनो᳚ ब्रह्मवा॒दिनो॑ गोपी॒थाय॑ गोपी॒थाय॑ ब्रह्मवा॒दिनः॑ । \newline
30. ब्र॒ह्म॒वा॒दिनो॑ वदन्ति वदन्ति ब्रह्मवा॒दिनो᳚ ब्रह्मवा॒दिनो॑ वदन्ति । \newline
31. ब्र॒ह्म॒वा॒दिन॒ इति॑ ब्रह्म - वा॒दिनः॑ । \newline
32. व॒द॒न्ति॒ किम् किं ॅव॑दन्ति वदन्ति॒ किम् । \newline
33. किं ॅय॒ज्ञ्स्य॑ य॒ज्ञ्स्य॒ किम् किं ॅय॒ज्ञ्स्य॑ । \newline
34. य॒ज्ञ्स्य॒ यज॑मानो॒ यज॑मानो य॒ज्ञ्स्य॑ य॒ज्ञ्स्य॒ यज॑मानः । \newline
35. यज॑मान॒ इतीति॒ यज॑मानो॒ यज॑मान॒ इति॑ । \newline
36. इति॑ प्रस्त॒रः प्र॑स्त॒र इतीति॑ प्रस्त॒रः । \newline
37. प्र॒स्त॒र इतीति॑ प्रस्त॒रः प्र॑स्त॒र इति॑ । \newline
38. प्र॒स्त॒र इति॑ प्र - स्त॒रः । \newline
39. इति॒ तस्य॒ तस्ये तीति॒ तस्य॑ । \newline
40. तस्य॒ क्वा᳚(1॒) क्व॑ तस्य॒ तस्य॒ क्व॑ । \newline
41. क्व॑ सुव॒र्गः सु॑व॒र्गः क्वा᳚(1॒) क्व॑ सुव॒र्गः । \newline
42. सु॒व॒र्गो लो॒को लो॒कः सु॑व॒र्गः सु॑व॒र्गो लो॒कः । \newline
43. सु॒व॒र्ग इति॑ सुवः - गः । \newline
44. लो॒क इतीति॑ लो॒को लो॒क इति॑ । \newline
45. इत्या॑हव॒नीय॑ आहव॒नीय॒ इती त्या॑हव॒नीयः॑ । \newline
46. आ॒ह॒व॒नीय॒ इती त्या॑हव॒नीय॑ आहव॒नीय॒ इति॑ । \newline
47. आ॒ह॒व॒नीय॒ इत्या᳚ - ह॒व॒नीयः॑ । \newline
48. इति॑ ब्रूयाद् ब्रूया॒ दितीति॑ ब्रूयात् । \newline
49. ब्रू॒या॒द् यद् यद् ब्रू॑याद् ब्रूया॒द् यत् । \newline
50. यत् प्र॑स्त॒रम् प्र॑स्त॒रं ॅयद् यत् प्र॑स्त॒रम् । \newline
51. प्र॒स्त॒र मा॑हव॒नीय॑ आहव॒नीये᳚ प्रस्त॒रम् प्र॑स्त॒र मा॑हव॒नीये᳚ । \newline
52. प्र॒स्त॒रमिति॑ प्र - स्त॒रम् । \newline
53. आ॒ह॒व॒नीये᳚ प्र॒हर॑ति प्र॒हर॑ त्याहव॒नीय॑ आहव॒नीये᳚ प्र॒हर॑ति । \newline
54. आ॒ह॒व॒नीय॒ इत्या᳚ - ह॒व॒नीये᳚ । \newline
55. प्र॒हर॑ति॒ यज॑मानं॒ ॅयज॑मानम् प्र॒हर॑ति प्र॒हर॑ति॒ यज॑मानम् । \newline
56. प्र॒हर॒तीति॑ प्र - हर॑ति । \newline
57. यज॑मान मे॒वैव यज॑मानं॒ ॅयज॑मान मे॒व । \newline
58. ए॒व सु॑व॒र्गꣳ सु॑व॒र्ग मे॒वैव सु॑व॒र्गम् । \newline

\textbf{Ghana Paata } \newline

1. स्त्र्य॑स्यास्य॒ स्त्री स्त्र्य॑स्य जायेत जायेतास्य॒ स्त्री स्त्र्य॑स्य जायेत । \newline
2. अ॒स्य॒ जा॒ये॒त॒ जा॒ये॒ता॒स्या॒स्य॒ जा॒ये॒तो॒र्द्ध्व मू॒र्द्ध्वम् जा॑येता स्यास्य जायेतो॒र्द्ध्वम् । \newline
3. जा॒ये॒तो॒र्द्ध्व मू॒र्द्ध्वम् जा॑येत जायेतो॒र्द्ध्व मुदुदू॒र्द्ध्वम् जा॑येत जायेतो॒र्द्ध्व मुत् । \newline
4. ऊ॒र्द्ध्व मुदुदू॒र्द्ध्व मू॒र्द्ध्व मुद् यौ॑ति यौ॒त्युदू॒र्द्ध्व मू॒र्द्ध्व मुद् यौ॑ति । \newline
5. उद् यौ॑ति यौ॒त्युदुद् यौ᳚त्यू॒र्द्ध्व मू॒र्द्ध्वं ॅयौ॒त्युदुद् यौ᳚त्यू॒र्द्ध्वम् । \newline
6. यौ॒त्यू॒र्द्ध्व मू॒र्द्ध्वं ॅयौ॑ति यौत्यू॒र्द्ध्व मि॑वे वो॒र्द्ध्वं ॅयौ॑ति यौत्यू॒र्द्ध्व मि॑व । \newline
7. ऊ॒र्द्ध्व मि॑वे वो॒र्द्ध्व मू॒र्द्ध्व मि॑व॒ हि हीवो॒र्द्ध्व मू॒र्द्ध्व मि॑व॒ हि । \newline
8. इ॒व॒ हि हीवे॑ व॒ हि पुꣳ॒॒सः पुꣳ॒॒सो हीवे॑ व॒ हि पुꣳ॒॒सः । \newline
9. हि पुꣳ॒॒सः पुꣳ॒॒सो हि हि पुꣳ॒॒सः पुमा॒न् पुमा᳚न् पुꣳ॒॒सो हि हि पुꣳ॒॒सः पुमान्॑ । \newline
10. पुꣳ॒॒सः पुमा॒न् पुमा᳚न् पुꣳ॒॒सः पुꣳ॒॒सः पुमा॑ ने॒वैव पुमा᳚न् पुꣳ॒॒सः पुꣳ॒॒सः पुमा॑ ने॒व । \newline
11. पुमा॑ ने॒वैव पुमा॒न् पुमा॑ ने॒वास्या᳚स्यै॒व पुमा॒न् पुमा॑ ने॒वास्य॑ । \newline
12. ए॒वास्या᳚ स्यै॒वैवास्य॑ जायते जायते ऽस्यै॒वैवास्य॑ जायते । \newline
13. अ॒स्य॒ जा॒य॒ते॒ जा॒य॒ते॒ ऽस्या॒स्य॒ जा॒य॒ते॒ यद् यज् जा॑यते ऽस्यास्य जायते॒ यत् । \newline
14. जा॒य॒ते॒ यद् यज् जा॑यते जायते॒ यथ् स्फ्येन॒ स्फ्येन॒ यज् जा॑यते जायते॒ यथ् स्फ्येन॑ । \newline
15. यथ् स्फ्येन॒ स्फ्येन॒ यद् यथ् स्फ्येन॑ वा वा॒ स्फ्येन॒ यद् यथ् स्फ्येन॑ वा । \newline
16. स्फ्येन॑ वा वा॒ स्फ्येन॒ स्फ्येन॑ वोपवे॒षेणो॑ पवे॒षेण॑ वा॒ स्फ्येन॒ स्फ्येन॑ वोपवे॒षेण॑ । \newline
17. वो॒प॒वे॒षेणो॑ पवे॒षेण॑ वा वोपवे॒षेण॑ वा वोपवे॒षेण॑ वा वोपवे॒षेण॑ वा । \newline
18. उ॒प॒वे॒षेण॑ वा वोपवे॒षेणो॑ पवे॒षेण॑ वा योयु॒प्येत॑ योयु॒प्येत॑ वोपवे॒षेणो॑ पवे॒षेण॑ वा योयु॒प्येत॑ । \newline
19. उ॒प॒वे॒षेणेत्यु॑प - वे॒षेण॑ । \newline
20. वा॒ यो॒यु॒प्येत॑ योयु॒प्येत॑ वा वा योयु॒प्येत॒ स्तृतिः॒ स्तृति॑र् योयु॒प्येत॑ वा वा योयु॒प्येत॒ स्तृतिः॑ । \newline
21. यो॒यु॒प्येत॒ स्तृतिः॒ स्तृति॑र् योयु॒प्येत॑ योयु॒प्येत॒ स्तृति॑ रे॒वैव स्तृति॑र् योयु॒प्येत॑ योयु॒प्येत॒ स्तृति॑रे॒व । \newline
22. स्तृति॑ रे॒वैव स्तृतिः॒ स्तृति॑ रे॒वास्या᳚ स्यै॒व स्तृतिः॒ स्तृति॑ रे॒वास्य॑ । \newline
23. ए॒वास्या᳚ स्यै॒वैवास्य॒ सा सा ऽस्यै॒वैवास्य॒ सा । \newline
24. अ॒स्य॒ सा सा ऽस्या᳚स्य॒ सा हस्ते॑न॒ हस्ते॑न॒ सा ऽस्या᳚स्य॒ सा हस्ते॑न । \newline
25. सा हस्ते॑न॒ हस्ते॑न॒ सा सा हस्ते॑न योयुप्यते योयुप्यते॒ हस्ते॑न॒ सा सा हस्ते॑न योयुप्यते । \newline
26. हस्ते॑न योयुप्यते योयुप्यते॒ हस्ते॑न॒ हस्ते॑न योयुप्यते॒ यज॑मानस्य॒ यज॑मानस्य योयुप्यते॒ हस्ते॑न॒ हस्ते॑न योयुप्यते॒ यज॑मानस्य । \newline
27. यो॒यु॒प्य॒ते॒ यज॑मानस्य॒ यज॑मानस्य योयुप्यते योयुप्यते॒ यज॑मानस्य गोपी॒थाय॑ गोपी॒थाय॒ यज॑मानस्य योयुप्यते योयुप्यते॒ यज॑मानस्य गोपी॒थाय॑ । \newline
28. यज॑मानस्य गोपी॒थाय॑ गोपी॒थाय॒ यज॑मानस्य॒ यज॑मानस्य गोपी॒थाय॑ ब्रह्मवा॒दिनो᳚ ब्रह्मवा॒दिनो॑ गोपी॒थाय॒ यज॑मानस्य॒ यज॑मानस्य गोपी॒थाय॑ ब्रह्मवा॒दिनः॑ । \newline
29. गो॒पी॒थाय॑ ब्रह्मवा॒दिनो᳚ ब्रह्मवा॒दिनो॑ गोपी॒थाय॑ गोपी॒थाय॑ ब्रह्मवा॒दिनो॑ वदन्ति वदन्ति ब्रह्मवा॒दिनो॑ गोपी॒थाय॑ गोपी॒थाय॑ ब्रह्मवा॒दिनो॑ वदन्ति । \newline
30. ब्र॒ह्म॒वा॒दिनो॑ वदन्ति वदन्ति ब्रह्मवा॒दिनो᳚ ब्रह्मवा॒दिनो॑ वदन्ति॒ किम् किं ॅव॑दन्ति ब्रह्मवा॒दिनो᳚ ब्रह्मवा॒दिनो॑ वदन्ति॒ किम् । \newline
31. ब्र॒ह्म॒वा॒दिन॒ इति॑ ब्रह्म - वा॒दिनः॑ । \newline
32. व॒द॒न्ति॒ किम् किं ॅव॑दन्ति वदन्ति॒ किं ॅय॒ज्ञ्स्य॑ य॒ज्ञ्स्य॒ किं ॅव॑दन्ति वदन्ति॒ किं ॅय॒ज्ञ्स्य॑ । \newline
33. किं ॅय॒ज्ञ्स्य॑ य॒ज्ञ्स्य॒ किम् किं ॅय॒ज्ञ्स्य॒ यज॑मानो॒ यज॑मानो य॒ज्ञ्स्य॒ किम् किं ॅय॒ज्ञ्स्य॒ यज॑मानः । \newline
34. य॒ज्ञ्स्य॒ यज॑मानो॒ यज॑मानो य॒ज्ञ्स्य॑ य॒ज्ञ्स्य॒ यज॑मान॒ इतीति॒ यज॑मानो य॒ज्ञ्स्य॑ य॒ज्ञ्स्य॒ यज॑मान॒ इति॑ । \newline
35. यज॑मान॒ इतीति॒ यज॑मानो॒ यज॑मान॒ इति॑ प्रस्त॒रः प्र॑स्त॒र इति॒ यज॑मानो॒ यज॑मान॒ इति॑ प्रस्त॒रः । \newline
36. इति॑ प्रस्त॒रः प्र॑स्त॒र इतीति॑ प्रस्त॒र इतीति॑ प्रस्त॒र इतीति॑ प्रस्त॒र इति॑ । \newline
37. प्र॒स्त॒र इतीति॑ प्रस्त॒रः प्र॑स्त॒र इति॒ तस्य॒ तस्ये ति॑ प्रस्त॒रः प्र॑स्त॒र इति॒ तस्य॑ । \newline
38. प्र॒स्त॒र इति॑ प्र - स्त॒रः । \newline
39. इति॒ तस्य॒ तस्ये तीति॒ तस्य॒ क्वा᳚(1॒) क्व॑ तस्ये तीति॒ तस्य॒ क्व॑ । \newline
40. तस्य॒ क्वा᳚(1॒) क्व॑ तस्य॒ तस्य॒ क्व॑ सुव॒र्गः सु॑व॒र्गः क्व॑ तस्य॒ तस्य॒ क्व॑ सुव॒र्गः । \newline
41. क्व॑ सुव॒र्गः सु॑व॒र्गः क्वा᳚(1॒) क्व॑ सुव॒र्गो लो॒को लो॒कः सु॑व॒र्गः क्वा᳚(1॒) क्व॑ सुव॒र्गो लो॒कः । \newline
42. सु॒व॒र्गो लो॒को लो॒कः सु॑व॒र्गः सु॑व॒र्गो लो॒क इतीति॑ लो॒कः सु॑व॒र्गः सु॑व॒र्गो लो॒क इति॑ । \newline
43. सु॒व॒र्ग इति॑ सुवः - गः । \newline
44. लो॒क इतीति॑ लो॒को लो॒क इत्या॑हव॒नीय॑ आहव॒नीय॒ इति॑ लो॒को लो॒क इत्या॑हव॒नीयः॑ । \newline
45. इत्या॑हव॒नीय॑ आहव॒नीय॒ इती त्या॑हव॒नीय॒ इती त्या॑हव॒नीय॒ इती त्या॑हव॒नीय॒ इति॑ । \newline
46. आ॒ह॒व॒नीय॒ इती त्या॑हव॒नीय॑ आहव॒नीय॒ इति॑ ब्रूयाद् ब्रूया॒ दित्या॑हव॒नीय॑ आहव॒नीय॒ इति॑ ब्रूयात् । \newline
47. आ॒ह॒व॒नीय॒ इत्या᳚ - ह॒व॒नीयः॑ । \newline
48. इति॑ ब्रूयाद् ब्रूया॒ दितीति॑ ब्रूया॒द् यद् यद् ब्रू॑या॒ दितीति॑ ब्रूया॒द् यत् । \newline
49. ब्रू॒या॒द् यद् यद् ब्रू॑याद् ब्रूया॒द् यत् प्र॑स्त॒रम् प्र॑स्त॒रं ॅयद् ब्रू॑याद् ब्रूया॒द् यत् प्र॑स्त॒रम् । \newline
50. यत् प्र॑स्त॒रम् प्र॑स्त॒रं ॅयद् यत् प्र॑स्त॒र मा॑हव॒नीय॑ आहव॒नीये᳚ प्रस्त॒रं ॅयद् यत् प्र॑स्त॒र मा॑हव॒नीये᳚ । \newline
51. प्र॒स्त॒र मा॑हव॒नीय॑ आहव॒नीये᳚ प्रस्त॒रम् प्र॑स्त॒र मा॑हव॒नीये᳚ प्र॒हर॑ति प्र॒हर॑ त्याहव॒नीये᳚ प्रस्त॒रम् प्र॑स्त॒र मा॑हव॒नीये᳚ प्र॒हर॑ति । \newline
52. प्र॒स्त॒रमिति॑ प्र - स्त॒रम् । \newline
53. आ॒ह॒व॒नीये᳚ प्र॒हर॑ति प्र॒हर॑ त्याहव॒नीय॑ आहव॒नीये᳚ प्र॒हर॑ति॒ यज॑मानं॒ ॅयज॑मानम् प्र॒हर॑ त्याहव॒नीय॑ आहव॒नीये᳚ प्र॒हर॑ति॒ यज॑मानम् । \newline
54. आ॒ह॒व॒नीय॒ इत्या᳚ - ह॒व॒नीये᳚ । \newline
55. प्र॒हर॑ति॒ यज॑मानं॒ ॅयज॑मानम् प्र॒हर॑ति प्र॒हर॑ति॒ यज॑मान मे॒वैव यज॑मानम् प्र॒हर॑ति प्र॒हर॑ति॒ यज॑मान मे॒व । \newline
56. प्र॒हर॒तीति॑ प्र - हर॑ति । \newline
57. यज॑मान मे॒वैव यज॑मानं॒ ॅयज॑मान मे॒व सु॑व॒र्गꣳ सु॑व॒र्ग मे॒व यज॑मानं॒ ॅयज॑मान मे॒व सु॑व॒र्गम् । \newline
58. ए॒व सु॑व॒र्गꣳ सु॑व॒र्ग मे॒वैव सु॑व॒र्गम् ॅलो॒कम् ॅलो॒कꣳ सु॑व॒र्ग मे॒वैव सु॑व॒र्गम् ॅलो॒कम् । \newline
\pagebreak
\markright{ TS 2.6.5.6  \hfill https://www.vedavms.in \hfill}
\addcontentsline{toc}{section}{ TS 2.6.5.6 }
\section*{ TS 2.6.5.6 }

\textbf{TS 2.6.5.6 } \newline
\textbf{Samhita Paata} \newline

सु॑व॒र्गं ॅलो॒कं ग॑मयति॒ वि वा ए॒तद्-यज॑मानो लिशते॒ यत् प्र॑स्त॒रं ॅयो॑यु॒प्यन्ते॑ ब॒र्॒.हिरनु॒ प्रह॑रति॒ शान्त्या॑ अनारम्भ॒ण इ॑व॒ वा ए॒तर्ह्य॑द्ध्व॒र्युः स ई᳚श्व॒रो वे॑प॒नो भवि॑तोर्द्ध्रु॒वा ऽसीती॒माम॒भि मृ॑शती॒यं ॅवै ध्रु॒वाऽस्यामे॒व प्रति॑तिष्ठति॒ न वे॑प॒नो भ॑व॒त्यगा(3)न॑ग्नी॒दित्या॑ह॒ यद्ब्रू॒याद-ग॑न्न॒ग्निरित्य॒ ( ) -ग्नाव॒ग्निं ग॑मये॒न्नि र्यज॑मानꣳ सुव॒र्गाल्लो॒काद्-भ॑जे॒दग॒न्नित्ये॒व ब्रू॑या॒द्-यज॑मानमे॒व सु॑व॒र्गं ॅलो॒कं ग॑मयति ॥ \newline

\textbf{Pada Paata} \newline

सु॒व॒र्गमिति॑ सुवः - गम् । लो॒कम् । ग॒म॒य॒ति॒ । वीति॑ । वै । ए॒तत् । यज॑मानः । लि॒श॒ते॒ । यत् । प्र॒स्त॒रमिति॑ प्र - स्त॒रम् । यो॒यु॒प्यन्ते᳚ । ब॒र्॒.हिः । अनु॑ । प्रेति॑ । ह॒र॒ति॒ । शान्त्यै᳚ । अ॒ना॒र॒भं॒ण इत्य॑ना - र॒भं॒णः । इ॒व॒ । वै । ए॒तर्.हि॑ । अ॒द्ध्व॒र्युः । सः । ई॒श्व॒रः । वे॒प॒नः । भवि॑तोः । ध्रु॒वा । अ॒सि॒ । इति॑ । इ॒माम् । अ॒भीति॑ । मृ॒श॒ति॒ । इ॒यम् । वै । ध्रु॒वा । अ॒स्याम् । ए॒व । प्रतीति॑ । ति॒ष्ठ॒ति॒ । न । वे॒प॒नः । भ॒व॒ति॒ । अगा(3)न् । अ॒ग्नी॒दित्य॑ग्नि - इ॒त् । इति॑ । आ॒ह॒ । यत् । ब्रू॒या॒त् । अगन्न्॑ । अ॒ग्निः । इति॑ ( ) । अ॒ग्नौ । अ॒ग्निम् । ग॒म॒ये॒त् । निरिति॑ । यज॑मानम् । सु॒व॒र्गादिति॑ सुवः-गात् । लो॒कात् । भ॒जे॒त् । अगन्न्॑ । इति॑ । ए॒व । ब्रू॒या॒त् । यज॑मानम् । ए॒व । सु॒व॒र्गमिति॑ सुवः - गम् । लो॒कम् । ग॒म॒य॒ति॒ ॥  \newline


\textbf{Krama Paata} \newline

सु॒व॒र्गं ॅलो॒कम् । सु॒व॒र्गमिति॑ सुवः - गम् । लो॒कम् ग॑मयति । ग॒म॒य॒ति॒ वि । वि वै । वा ए॒तत् । ए॒तद् यज॑मानः । यज॑मानो लिशते । लि॒श॒ते॒ यत् । यत् प्र॑स्त॒रम् । प्र॒स्त॒रं ॅयो॑यु॒प्यन्ते᳚ । प्र॒स्त॒रमिति॑ प्र - स्त॒रम् । यो॒यु॒प्यन्ते॑ ब॒र्.॒हिः । ब॒र्.॒हिरनु॑ । अनु॒ प्र । प्र ह॑रति । ह॒र॒ति॒ शान्त्यै᳚ । शान्त्या॑ अनारम्भ॒णः । अ॒ना॒र॒म्भ॒ण इ॑व । अ॒ना॒र॒म्भ॒ण इत्य॑ना - र॒म्भ॒णः । इ॒व॒ वै । वा ए॒तर्.हि॑ । ए॒तर्.ह्य॑द्ध्व॒र्युः । अ॒द्ध्व॒र्युः सः । स ई᳚श्व॒रः । ई॒श्व॒रो वे॑प॒नः । वे॒प॒नो भवि॑तोः । भवि॑तोर् ध्रु॒वा । ध्रु॒वा ऽसि॑ । अ॒सीति॑ । इती॒माम् । इ॒माम॒भि । अ॒भि मृ॑शति । मृ॒श॒ती॒यम् । इ॒यं ॅवै । वै ध्रु॒वा । ध्रु॒वा ऽस्याम् । अ॒स्यामे॒व । ए॒व प्रति॑ । प्रति॑ तिष्ठति । ति॒ष्ठ॒ति॒ न । न वे॑प॒नः । वे॒प॒नो भ॑वति । भ॒व॒त्यगा(3)न् । अगा(3)न॑ग्नीत् । अ॒ग्नी॒दिति॑ । अ॒ग्नी॒दित्य॑ग्नि - इ॒त्॒ । इत्या॑ह । आ॒ह॒ यत् । यद् ब्रू॒यात् । बू॒यादगन्न्॑ । अग॑न्न॒ग्निः । अ॒ग्निरिति॑ ( ) । इत्य॒ग्नौ । अ॒ग्नाव॒ग्निम् । अ॒ग्निम् ग॑मयेत् । ग॒म॒ये॒न् निः । निर् यज॑मानम् । यज॑मानꣳ सुव॒र्गात् । सु॒व॒र्गाल्लो॒कात् । सु॒व॒र्गादिति॑ सुवः - गात् । लो॒काद् भ॑जेत् । भ॒जे॒दगन्न्॑ । अग॒न्निति॑ । इत्ये॒व । ए॒व ब्रू॑यात् । ब्रू॒या॒द् यज॑मानम् । यज॑मानमे॒व । ए॒व सु॑व॒र्गम् । सु॒व॒र्गं ॅलो॒कम् । सु॒व॒र्गमिति॑ सुवः - गम् । लो॒कम् ग॑मयति । ग॒म॒य॒तीति॑ गमयति । \newline

\textbf{Jatai Paata} \newline

1. सु॒व॒र्गम् ॅलो॒कम् ॅलो॒कꣳ सु॑व॒र्गꣳ सु॑व॒र्गम् ॅलो॒कम् । \newline
2. सु॒व॒र्गमिति॑ सुवः - गम् । \newline
3. लो॒कम् ग॑मयति गमयति लो॒कम् ॅलो॒कम् ग॑मयति । \newline
4. ग॒म॒य॒ति॒ वि वि ग॑मयति गमयति॒ वि । \newline
5. वि वै वै वि वि वै । \newline
6. वा ए॒त दे॒तद् वै वा ए॒तत् । \newline
7. ए॒तद् यज॑मानो॒ यज॑मान ए॒त दे॒तद् यज॑मानः । \newline
8. यज॑मानो लिशते लिशते॒ यज॑मानो॒ यज॑मानो लिशते । \newline
9. लि॒श॒ते॒ यद् य ल्लि॑शते लिशते॒ यत् । \newline
10. यत् प्र॑स्त॒रम् प्र॑स्त॒रं ॅयद् यत् प्र॑स्त॒रम् । \newline
11. प्र॒स्त॒रं ॅयो॑यु॒प्यन्ते॑ योयु॒प्यन्ते᳚ प्रस्त॒रम् प्र॑स्त॒रं ॅयो॑यु॒प्यन्ते᳚ । \newline
12. प्र॒स्त॒रमिति॑ प्र - स्त॒रम् । \newline
13. यो॒यु॒प्यन्ते॑ ब॒र्॒.हिर् ब॒र्॒.हिर् यो॑यु॒प्यन्ते॑ योयु॒प्यन्ते॑ ब॒र्॒.हिः । \newline
14. ब॒र्॒.हि रन्वनु॑ ब॒र्॒.हिर् ब॒र्॒.हि रनु॑ । \newline
15. अनु॒ प्र प्रा ण्वनु॒ प्र । \newline
16. प्र ह॑रति हरति॒ प्र प्र ह॑रति । \newline
17. ह॒र॒ति॒ शान्त्यै॒ शान्त्यै॑ हरति हरति॒ शान्त्यै᳚ । \newline
18. शान्त्या॑ अनारम्भ॒णो॑ ऽनारम्भ॒णः शान्त्यै॒ शान्त्या॑ अनारम्भ॒णः । \newline
19. अ॒ना॒र॒म्भ॒ण इ॑वे वानारम्भ॒णो॑ ऽनारम्भ॒ण इ॑व । \newline
20. अ॒ना॒र॒म्भ॒ण इत्य॑ना - र॒म्भ॒णः । \newline
21. इ॒व॒ वै वा इ॑वे व॒ वै । \newline
22. वा ए॒तर् ह्ये॒तर्.हि॒ वै वा ए॒तर्.हि॑ । \newline
23. ए॒तर् ह्य॑द्ध्व॒र्यु र॑द्ध्व॒र्यु रे॒तर् ह्ये॒तर् ह्य॑द्ध्व॒र्युः । \newline
24. अ॒द्ध्व॒र्युः स सो᳚ ऽद्ध्व॒र्यु र॑द्ध्व॒र्युः सः । \newline
25. स ई᳚श्व॒र ई᳚श्व॒रः स स ई᳚श्व॒रः । \newline
26. ई॒श्व॒रो वे॑प॒नो वे॑प॒न ई᳚श्व॒र ई᳚श्व॒रो वे॑प॒नः । \newline
27. वे॒प॒नो भवि॑तो॒र् भवि॑तोर् वेप॒नो वे॑प॒नो भवि॑तोः । \newline
28. भवि॑तोर् ध्रु॒वा ध्रु॒वा भवि॑तो॒र् भवि॑तोर् ध्रु॒वा । \newline
29. ध्रु॒वा ऽस्य॑सि ध्रु॒वा ध्रु॒वा ऽसि॑ । \newline
30. अ॒सीतीत्य॑स्य॒सीति॑ । \newline
31. इती॒मा मि॒मा मितीती॒माम् । \newline
32. इ॒मा म॒भ्य॑भीमा मि॒मा म॒भि । \newline
33. अ॒भि मृ॑शति मृश त्य॒भ्य॑भि मृ॑शति । \newline
34. मृ॒श॒ती॒य मि॒यम् मृ॑शति मृशती॒यम् । \newline
35. इ॒यं ॅवै वा इ॒य मि॒यं ॅवै । \newline
36. वै ध्रु॒वा ध्रु॒वा वै वै ध्रु॒वा । \newline
37. ध्रु॒वा ऽस्या म॒स्याम् ध्रु॒वा ध्रु॒वा ऽस्याम् । \newline
38. अ॒स्या मे॒वैवास्या म॒स्या मे॒व । \newline
39. ए॒व प्रति॒ प्रत्ये॒वैव प्रति॑ । \newline
40. प्रति॑ तिष्ठति तिष्ठति॒ प्रति॒ प्रति॑ तिष्ठति । \newline
41. ति॒ष्ठ॒ति॒ न न ति॑ष्ठति तिष्ठति॒ न । \newline
42. न वे॑प॒नो वे॑प॒नो न न वे॑प॒नः । \newline
43. वे॒प॒नो भ॑वति भवति वेप॒नो वे॑प॒नो भ॑वति । \newline
44. भ॒व॒त्यगा(3) नगा(3)न् भ॑वति भव॒त्यगा(3)न् । \newline
45. अगा(3) न॑ग्नी दग्नी॒ दगा(3) नगा(3) न॑ग्नीत् । \newline
46. अ॒ग्नी॒ दिती त्य॑ग्नी दग्नी॒ दिति॑ । \newline
47. अ॒ग्नी॒दित्य॑ग्नि - इ॒त् । \newline
48. इत्या॑हा॒हे तीत्या॑ह । \newline
49. आ॒ह॒ यद् यदा॑हाह॒ यत् । \newline
50. यद् ब्रू॑याद् ब्रूया॒द् यद् यद् ब्रू॑यात् । \newline
51. ब्रू॒या॒ दग॒न् नग॑न् ब्रूयाद् ब्रूया॒ दगन्न्॑ । \newline
52. अग॑न् न॒ग्नि र॒ग्नि रग॒न् नग॑न् न॒ग्निः । \newline
53. अ॒ग्नि रिती त्य॒ग्नि र॒ग्नि रिति॑ । \newline
54. इत्य॒ग्ना व॒ग्ना विती त्य॒ग्नौ । \newline
55. अ॒ग्ना व॒ग्नि म॒ग्नि म॒ग्ना व॒ग्ना व॒ग्निम् । \newline
56. अ॒ग्निम् ग॑मयेद् गमये द॒ग्नि म॒ग्निम् ग॑मयेत् । \newline
57. ग॒म॒ये॒न् निर् णिर् ग॑मयेद् गमये॒न् निः । \newline
58. निर् यज॑मानं॒ ॅयज॑मान॒म् निर् णिर् यज॑मानम् । \newline
59. यज॑मानꣳ सुव॒र्गाथ् सु॑व॒र्गाद् यज॑मानं॒ ॅयज॑मानꣳ सुव॒र्गात् । \newline
60. सु॒व॒र्गा ल्लो॒का ल्लो॒काथ् सु॑व॒र्गाथ् सु॑व॒र्गा ल्लो॒कात् । \newline
61. सु॒व॒र्गादिति॑ सुवः - गात् । \newline
62. लो॒काद् भ॑जेद् भजे ल्लो॒का ल्लो॒काद् भ॑जेत् । \newline
63. भ॒जे॒ दग॒न् नग॑न् भजेद् भजे॒ दगन्न्॑ । \newline
64. अग॒न् निती त्यग॒न् नग॒न् निति॑ । \newline
65. इत्ये॒वैवे तीत्ये॒व । \newline
66. ए॒व ब्रू॑याद् ब्रूया दे॒वैव ब्रू॑यात् । \newline
67. ब्रू॒या॒द् यज॑मानं॒ ॅयज॑मानम् ब्रूयाद् ब्रूया॒द् यज॑मानम् । \newline
68. यज॑मान मे॒वैव यज॑मानं॒ ॅयज॑मान मे॒व । \newline
69. ए॒व सु॑व॒र्गꣳ सु॑व॒र्ग मे॒वैव सु॑व॒र्गम् । \newline
70. सु॒व॒र्गम् ॅलो॒कम् ॅलो॒कꣳ सु॑व॒र्गꣳ सु॑व॒र्गम् ॅलो॒कम् । \newline
71. सु॒व॒र्गमिति॑ सुवः - गम् । \newline
72. लो॒कम् ग॑मयति गमयति लो॒कम् ॅलो॒कम् ग॑मयति । \newline
73. ग॒म॒य॒तीति॑ गमयति । \newline

\textbf{Ghana Paata } \newline

1. सु॒व॒र्गम् ॅलो॒कम् ॅलो॒कꣳ सु॑व॒र्गꣳ सु॑व॒र्गम् ॅलो॒कम् ग॑मयति गमयति लो॒कꣳ सु॑व॒र्गꣳ सु॑व॒र्गम् ॅलो॒कम् ग॑मयति । \newline
2. सु॒व॒र्गमिति॑ सुवः - गम् । \newline
3. लो॒कम् ग॑मयति गमयति लो॒कम् ॅलो॒कम् ग॑मयति॒ वि वि ग॑मयति लो॒कम् ॅलो॒कम् ग॑मयति॒ वि । \newline
4. ग॒म॒य॒ति॒ वि वि ग॑मयति गमयति॒ वि वै वै वि ग॑मयति गमयति॒ वि वै । \newline
5. वि वै वै वि वि वा ए॒त दे॒तद् वै वि वि वा ए॒तत् । \newline
6. वा ए॒त दे॒तद् वै वा ए॒तद् यज॑मानो॒ यज॑मान ए॒तद् वै वा ए॒तद् यज॑मानः । \newline
7. ए॒तद् यज॑मानो॒ यज॑मान ए॒त दे॒तद् यज॑मानो लिशते लिशते॒ यज॑मान ए॒त दे॒तद् यज॑मानो लिशते । \newline
8. यज॑मानो लिशते लिशते॒ यज॑मानो॒ यज॑मानो लिशते॒ यद् यल्लि॑शते॒ यज॑मानो॒ यज॑मानो लिशते॒ यत् । \newline
9. लि॒श॒ते॒ यद् यल्लि॑शते लिशते॒ यत् प्र॑स्त॒रम् प्र॑स्त॒रं ॅयल्लि॑शते लिशते॒ यत् प्र॑स्त॒रम् । \newline
10. यत् प्र॑स्त॒रम् प्र॑स्त॒रं ॅयद् यत् प्र॑स्त॒रं ॅयो॑यु॒प्यन्ते॑ योयु॒प्यन्ते᳚ प्रस्त॒रं ॅयद् यत् प्र॑स्त॒रं ॅयो॑यु॒प्यन्ते᳚ । \newline
11. प्र॒स्त॒रं ॅयो॑यु॒प्यन्ते॑ योयु॒प्यन्ते᳚ प्रस्त॒रम् प्र॑स्त॒रं ॅयो॑यु॒प्यन्ते॑ ब॒र्॒.हिर् ब॒र्॒.हिर् यो॑यु॒प्यन्ते᳚ प्रस्त॒रम् प्र॑स्त॒रं ॅयो॑यु॒प्यन्ते॑ ब॒र्॒.हिः । \newline
12. प्र॒स्त॒रमिति॑ प्र - स्त॒रम् । \newline
13. यो॒यु॒प्यन्ते॑ ब॒र्॒.हिर् ब॒र्॒.हिर् यो॑यु॒प्यन्ते॑ योयु॒प्यन्ते॑ ब॒र्॒.हिरन्वनु॑ ब॒र्॒.हिर् यो॑यु॒प्यन्ते॑ योयु॒प्यन्ते॑ ब॒र्॒.हिरनु॑ । \newline
14. ब॒र्॒.हि रन्वनु॑ ब॒र्॒.हिर् ब॒र्॒.हिरनु॒ प्र प्राणु॑ ब॒र्॒.हिर् ब॒र्॒.हिरनु॒ प्र । \newline
15. अनु॒ प्र प्राण्वनु॒ प्र ह॑रति हरति॒ प्राण्वनु॒ प्र ह॑रति । \newline
16. प्र ह॑रति हरति॒ प्र प्र ह॑रति॒ शान्त्यै॒ शान्त्यै॑ हरति॒ प्र प्र ह॑रति॒ शान्त्यै᳚ । \newline
17. ह॒र॒ति॒ शान्त्यै॒ शान्त्यै॑ हरति हरति॒ शान्त्या॑ अनारम्भ॒णो॑ ऽनारम्भ॒णः शान्त्यै॑ हरति हरति॒ शान्त्या॑ अनारम्भ॒णः । \newline
18. शान्त्या॑ अनारम्भ॒णो॑ ऽनारम्भ॒णः शान्त्यै॒ शान्त्या॑ अनारम्भ॒ण इ॑वे वानारम्भ॒णः शान्त्यै॒ शान्त्या॑ अनारम्भ॒ण इ॑व । \newline
19. अ॒ना॒र॒म्भ॒ण इ॑वे वानारम्भ॒णो॑ ऽनारम्भ॒ण इ॑व॒ वै वा इ॑वानारम्भ॒णो॑ ऽनारम्भ॒ण इ॑व॒ वै । \newline
20. अ॒ना॒र॒म्भ॒ण इत्य॑ना - र॒म्भ॒णः । \newline
21. इ॒व॒ वै वा इ॑वे व॒ वा ए॒तर् ह्ये॒तर्.हि॒ वा इ॑वे व॒ वा ए॒तर्.हि॑ । \newline
22. वा ए॒तर् ह्ये॒तर्.हि॒ वै वा ए॒तर् ह्य॑द्ध्व॒र्यु र॑द्ध्व॒र्यु रे॒तर्.हि॒ वै वा ए॒तर् ह्य॑द्ध्व॒र्युः । \newline
23. ए॒तर् ह्य॑द्ध्व॒र्यु र॑द्ध्व॒र्यु रे॒तर् ह्ये॒तर् ह्य॑द्ध्व॒र्युः स सो᳚ ऽद्ध्व॒र्यु रे॒तर् ह्ये॒तर् ह्य॑द्ध्व॒र्युः सः । \newline
24. अ॒द्ध्व॒र्युः स सो᳚ ऽद्ध्व॒र्यु र॑द्ध्व॒र्युः स ई᳚श्व॒र ई᳚श्व॒रः सो᳚ ऽद्ध्व॒र्यु र॑द्ध्व॒र्युः स ई᳚श्व॒रः । \newline
25. स ई᳚श्व॒र ई᳚श्व॒रः स स ई᳚श्व॒रो वे॑प॒नो वे॑प॒न ई᳚श्व॒रः स स ई᳚श्व॒रो वे॑प॒नः । \newline
26. ई॒श्व॒रो वे॑प॒नो वे॑प॒न ई᳚श्व॒र ई᳚श्व॒रो वे॑प॒नो भवि॑तो॒र् भवि॑तोर् वेप॒न ई᳚श्व॒र ई᳚श्व॒रो वे॑प॒नो भवि॑तोः । \newline
27. वे॒प॒नो भवि॑तो॒र् भवि॑तोर् वेप॒नो वे॑प॒नो भवि॑तोर् ध्रु॒वा ध्रु॒वा भवि॑तोर् वेप॒नो वे॑प॒नो भवि॑तोर् ध्रु॒वा । \newline
28. भवि॑तोर् ध्रु॒वा ध्रु॒वा भवि॑तो॒र् भवि॑तोर् ध्रु॒वा ऽस्य॑सि ध्रु॒वा भवि॑तो॒र् भवि॑तोर् ध्रु॒वा ऽसि॑ । \newline
29. ध्रु॒वा ऽस्य॑सि ध्रु॒वा ध्रु॒वा ऽसीती त्य॑सि ध्रु॒वा ध्रु॒वा ऽसीति॑ । \newline
30. अ॒सीती त्य॑स्य॒सीती॒मा मि॒मा मित्य॑स्य॒सीती॒माम् । \newline
31. इती॒मा मि॒मा मितीती॒मा म॒भ्य॑भीमा मितीती॒मा म॒भि । \newline
32. इ॒मा म॒भ्य॑भीमा मि॒मा म॒भि मृ॑शति मृश त्य॒भीमा मि॒मा म॒भि मृ॑शति । \newline
33. अ॒भि मृ॑शति मृश त्य॒भ्य॑भि मृ॑शती॒य मि॒यम् मृ॑श त्य॒भ्य॑भि मृ॑शती॒यम् । \newline
34. मृ॒श॒ती॒य मि॒यम् मृ॑शति मृशती॒यं ॅवै वा इ॒यम् मृ॑शति मृशती॒यं ॅवै । \newline
35. इ॒यं ॅवै वा इ॒य मि॒यं ॅवै ध्रु॒वा ध्रु॒वा वा इ॒य मि॒यं ॅवै ध्रु॒वा । \newline
36. वै ध्रु॒वा ध्रु॒वा वै वै ध्रु॒वा ऽस्या म॒स्याम् ध्रु॒वा वै वै ध्रु॒वा ऽस्याम् । \newline
37. ध्रु॒वा ऽस्या म॒स्याम् ध्रु॒वा ध्रु॒वा ऽस्या मे॒वैवास्याम् ध्रु॒वा ध्रु॒वा ऽस्या मे॒व । \newline
38. अ॒स्या मे॒वैवास्या म॒स्या मे॒व प्रति॒ प्रत्ये॒वास्या म॒स्या मे॒व प्रति॑ । \newline
39. ए॒व प्रति॒ प्रत्ये॒वैव प्रति॑ तिष्ठति तिष्ठति॒ प्रत्ये॒वैव प्रति॑ तिष्ठति । \newline
40. प्रति॑ तिष्ठति तिष्ठति॒ प्रति॒ प्रति॑ तिष्ठति॒ न न ति॑ष्ठति॒ प्रति॒ प्रति॑ तिष्ठति॒ न । \newline
41. ति॒ष्ठ॒ति॒ न न ति॑ष्ठति तिष्ठति॒ न वे॑प॒नो वे॑प॒नो न ति॑ष्ठति तिष्ठति॒ न वे॑प॒नः । \newline
42. न वे॑प॒नो वे॑प॒नो न न वे॑प॒नो भ॑वति भवति वेप॒नो न न वे॑प॒नो भ॑वति । \newline
43. वे॒प॒नो भ॑वति भवति वेप॒नो वे॑प॒नो भ॑व॒त्यगा(3) नगा(3)न् भ॑वति वेप॒नो वे॑प॒नो भ॑व॒त्यगा(3)न् । \newline
44. भ॒व॒त्यगा(3) नगा(3)न् भ॑वति भव॒त्यगा(3) न॑ग्नी दग्नी॒ दगा(3)न् भ॑वति भव॒त्यगा(3) न॑ग्नीत् । \newline
45. अगा(3) न॑ग्नी दग्नी॒ दगा(3) नगा(3) न॑ग्नी॒ दिती त्य॑ग्नी॒ दगा(3) नगा(3) न॑ग्नी॒दिति॑ । \newline
46. अ॒ग्नी॒ दिती त्य॑ग्नी दग्नी॒ दित्या॑हा॒हे त्य॑ग्नी दग्नी॒ दित्या॑ह । \newline
47. अ॒ग्नी॒दित्य॑ग्नि - इ॒त् । \newline
48. इत्या॑हा॒हे तीत्या॑ह॒ यद् यदा॒हे तीत्या॑ह॒ यत् । \newline
49. आ॒ह॒ यद् यदा॑हाह॒ यद् ब्रू॑याद् ब्रूया॒द् यदा॑हाह॒ यद् ब्रू॑यात् । \newline
50. यद् ब्रू॑याद् ब्रूया॒द् यद् यद् ब्रू॑या॒ दग॒न् नग॑न् ब्रूया॒द् यद् यद् ब्रू॑या॒ दगन्न्॑ । \newline
51. ब्रू॒या॒ दग॒न् नग॑न् ब्रूयाद् ब्रूया॒ दग॑न् न॒ग्नि र॒ग्नि रग॑न् ब्रूयाद् ब्रूया॒ दग॑न् न॒ग्निः । \newline
52. अग॑न् न॒ग्नि र॒ग्नि रग॒न् नग॑न् न॒ग्नि रिती त्य॒ग्नि रग॒न् नग॑न् न॒ग्निरिति॑ । \newline
53. अ॒ग्नि रिती त्य॒ग्नि र॒ग्नि रित्य॒ग्ना व॒ग्ना वित्य॒ग्नि र॒ग्नि रित्य॒ग्नौ । \newline
54. इत्य॒ग्ना व॒ग्ना विती त्य॒ग्ना व॒ग्नि म॒ग्नि म॒ग्ना विती त्य॒ग्ना व॒ग्निम् । \newline
55. अ॒ग्ना व॒ग्नि म॒ग्नि म॒ग्ना व॒ग्ना व॒ग्निम् ग॑मयेद् गमये द॒ग्नि म॒ग्ना व॒ग्ना व॒ग्निम् ग॑मयेत् । \newline
56. अ॒ग्निम् ग॑मयेद् गमये द॒ग्नि म॒ग्निम् ग॑मये॒न् निर् णिर् ग॑मये द॒ग्नि म॒ग्निम् ग॑मये॒न् निः । \newline
57. ग॒म॒ये॒न् निर् णिर् ग॑मयेद् गमये॒न् निर् यज॑मानं॒ ॅयज॑मान॒म् निर् ग॑मयेद् गमये॒न् निर् यज॑मानम् । \newline
58. निर् यज॑मानं॒ ॅयज॑मान॒म् निर् णिर् यज॑मानꣳ सुव॒र्गाथ् सु॑व॒र्गाद् यज॑मान॒म् निर् णिर् यज॑मानꣳ सुव॒र्गात् । \newline
59. यज॑मानꣳ सुव॒र्गाथ् सु॑व॒र्गाद् यज॑मानं॒ ॅयज॑मानꣳ सुव॒र्गा ल्लो॒का ल्लो॒काथ् सु॑व॒र्गाद् यज॑मानं॒ ॅयज॑मानꣳ सुव॒र्गा ल्लो॒कात् । \newline
60. सु॒व॒र्गा ल्लो॒का ल्लो॒काथ् सु॑व॒र्गाथ् सु॑व॒र्गा ल्लो॒काद् भ॑जेद् भजे ल्लो॒काथ् सु॑व॒र्गाथ् सु॑व॒र्गा ल्लो॒काद् भ॑जेत् । \newline
61. सु॒व॒र्गादिति॑ सुवः - गात् । \newline
62. लो॒काद् भ॑जेद् भजे ल्लो॒का ल्लो॒काद् भ॑जे॒ दग॒न् नग॑न् भजे ल्लो॒का ल्लो॒काद् भ॑जे॒ दगन्न्॑ । \newline
63. भ॒जे॒ दग॒न् नग॑न् भजेद् भजे॒ दग॒न् निती त्यग॑न् भजेद् भजे॒ दग॒न् निति॑ । \newline
64. अग॒न् निती त्यग॒न् नग॒न् नित्ये॒वैवे त्यग॒न् नग॒न् नित्ये॒व । \newline
65. इत्ये॒वैवे तीत्ये॒व ब्रू॑याद् ब्रूयादे॒वे तीत्ये॒व ब्रू॑यात् । \newline
66. ए॒व ब्रू॑याद् ब्रूया दे॒वैव ब्रू॑या॒द् यज॑मानं॒ ॅयज॑मानम् ब्रूया दे॒वैव ब्रू॑या॒द् यज॑मानम् । \newline
67. ब्रू॒या॒द् यज॑मानं॒ ॅयज॑मानम् ब्रूयाद् ब्रूया॒द् यज॑मान मे॒वैव यज॑मानम् ब्रूयाद् ब्रूया॒द् यज॑मान मे॒व । \newline
68. यज॑मान मे॒वैव यज॑मानं॒ ॅयज॑मान मे॒व सु॑व॒र्गꣳ सु॑व॒र्ग मे॒व यज॑मानं॒ ॅयज॑मान मे॒व सु॑व॒र्गम् । \newline
69. ए॒व सु॑व॒र्गꣳ सु॑व॒र्ग मे॒वैव सु॑व॒र्गम् ॅलो॒कम् ॅलो॒कꣳ सु॑व॒र्ग मे॒वैव सु॑व॒र्गम् ॅलो॒कम् । \newline
70. सु॒व॒र्गम् ॅलो॒कम् ॅलो॒कꣳ सु॑व॒र्गꣳ सु॑व॒र्गम् ॅलो॒कम् ग॑मयति गमयति लो॒कꣳ सु॑व॒र्गꣳ सु॑व॒र्गम् ॅलो॒कम् ग॑मयति । \newline
71. सु॒व॒र्गमिति॑ सुवः - गम् । \newline
72. लो॒कम् ग॑मयति गमयति लो॒कम् ॅलो॒कम् ग॑मयति । \newline
73. ग॒म॒य॒तीति॑ गमयति । \newline
\pagebreak
\markright{ TS 2.6.6.1  \hfill https://www.vedavms.in \hfill}
\addcontentsline{toc}{section}{ TS 2.6.6.1 }
\section*{ TS 2.6.6.1 }

\textbf{TS 2.6.6.1 } \newline
\textbf{Samhita Paata} \newline

अ॒ग्नेस्त्रयो॒ ज्यायाꣳ॑सो॒ भ्रात॑र आस॒न् ते दे॒वेभ्यो॑ ह॒व्यं ॅवह॑न्तः॒ प्रामी॑यन्त॒ सो᳚ऽग्निर॑बिभेदि॒त्थं ॅवाव स्य आर्ति॒माऽरि॑ष्य॒तीति॒ स निला॑यत॒ सो॑ऽपः प्रावि॑श॒त् तं दे॒वताः॒ प्रैष॑मैच्छ॒न् तं मथ्स्यः॒ प्राब्र॑वी॒त् तम॑शपद्धि॒याधि॑या त्वा वद्ध्यासु॒र्यो मा॒ प्रावो॑च॒ इति॒ तस्मा॒न्मथ्स्यं॑ धि॒याधि॑या घ्नन्ति श॒प्तो - [  ] \newline

\textbf{Pada Paata} \newline

अ॒ग्नेः । त्रयः॑ । ज्यायाꣳ॑सः । भ्रात॑रः । आ॒स॒न्न् । ते । दे॒वेभ्यः॑ । ह॒व्यम् । वह॑न्तः । प्रेति॑ । अ॒मी॒य॒न्त॒ । सः । अ॒ग्निः । अ॒बि॒भे॒त् । इ॒त्थम् । वाव । स्यः । आर्ति᳚म् । एति॑ । अ॒रि॒ष्य॒ति॒ । इति॑ । सः । निला॑यत । सः । अ॒पः । प्रेति॑ । अ॒वि॒श॒त् । तम् । दे॒वताः᳚ । प्रैष॒मिति॑ प्र - एष᳚म् । ऐ॒च्छ॒न्न् । तम् । मथ्स्यः॑ । प्रेति॑ । अ॒ब्र॒वी॒त् । तम् । अ॒श॒प॒त् । धि॒याधि॒येति॑ धि॒या-धि॒या॒ । त्वा॒ । व॒द्ध्या॒सुः॒ । यः । मा॒ । प्रेति॑ । अवो॑चः । इति॑ । तस्मा᳚त् । मथ्स्य᳚म् । धि॒याधि॒येति॑ धि॒या - धि॒या॒ । घ्न॒न्ति॒ । श॒प्तः ।  \newline


\textbf{Krama Paata} \newline

अ॒ग्नेस्त्रयः॑ । त्रयो॒ ज्यायाꣳ॑सः । ज्यायाꣳ॑सो॒ भ्रात॑रः । भ्रात॑र आसन्न् । आ॒स॒न् ते । ते दे॒वेभ्यः॑ । दे॒वेभ्यो॑ ह॒व्यम् । ह॒व्यं ॅवह॑न्तः । वह॑न्तः॒ प्र । प्रामी॑यन्त । अ॒मी॒य॒न्त॒ सः । सो᳚ऽग्निः । अ॒ग्निर॑बिभेत् । अ॒बि॒भे॒दि॒त्थम् । इ॒त्थं ॅवाव । वाव स्यः । स्य आर्ति᳚म् । आर्ति॒मा । आऽरि॑ष्यति । अ॒रि॒ष्य॒तीति॑ । इति॒ सः । स निला॑यत । निला॑यत॒ सः । सो॑ऽपः । अ॒पः प्र । प्रावि॑शत् । अ॒वि॒श॒त् तम् । तम् दे॒वताः᳚ । दे॒वताः॒ प्रैष᳚म् । प्रैष॑मैच्छन्न् । प्रैष॒मिति॑ प्र - एष᳚म् । ऐ॒च्छ॒न् तम् । तम् मथ्स्यः॑ । मथ्स्यः॒ प्र । प्राब्र॑वीत् । अ॒ब्र॒वी॒त् तम् । तम॑शपत् । अ॒श॒प॒द् धि॒याधि॑या । धि॒याधि॑या त्वा । धि॒याधि॒येति॑ धि॒या - धि॒या॒ । त्वा॒ व॒द्ध्या॒सुः॒ । व॒द्ध्या॒सु॒र् यः । यो मा᳚ । मा॒ प्र । प्रावो॑चः । अवो॑च॒ इति॑ । इति॒ तस्मा᳚त् । तस्मा॒न् मथ्स्य᳚म् । मथ्स्य॑म् धि॒याधि॑या । धि॒याधि॑या घ्नन्ति । धि॒याधि॒येति॑ धि॒या - धि॒या॒ । घ्न॒न्ति॒ श॒प्तः । श॒प्तो हि \newline

\textbf{Jatai Paata} \newline

1. अ॒ग्ने स्त्रय॒ स्त्रयो॒ ऽग्ने र॒ग्ने स्त्रयः॑ । \newline
2. त्रयो॒ ज्यायाꣳ॑सो॒ ज्यायाꣳ॑स॒ स्त्रय॒ स्त्रयो॒ ज्यायाꣳ॑सः । \newline
3. ज्यायाꣳ॑सो॒ भ्रात॑रो॒ भ्रात॑रो॒ ज्यायाꣳ॑सो॒ ज्यायाꣳ॑सो॒ भ्रात॑रः । \newline
4. भ्रात॑र आसन् नास॒न् भ्रात॑रो॒ भ्रात॑र आसन्न् । \newline
5. आ॒स॒न् ते त आ॑सन् नास॒न् ते । \newline
6. ते दे॒वेभ्यो॑ दे॒वेभ्य॒ स्ते ते दे॒वेभ्यः॑ । \newline
7. दे॒वेभ्यो॑ ह॒व्यꣳ ह॒व्यम् दे॒वेभ्यो॑ दे॒वेभ्यो॑ ह॒व्यम् । \newline
8. ह॒व्यं ॅवह॑न्तो॒ वह॑न्तो ह॒व्यꣳ ह॒व्यं ॅवह॑न्तः । \newline
9. वह॑न्तः॒ प्र प्र वह॑न्तो॒ वह॑न्तः॒ प्र । \newline
10. प्रामी॑यन्ता मीयन्त॒ प्र प्रामी॑यन्त । \newline
11. अ॒मी॒य॒न्त॒ स सो॑ ऽमीयन्ता मीयन्त॒ सः । \newline
12. सो᳚ ऽग्नि र॒ग्निः स सो᳚ ऽग्निः । \newline
13. अ॒ग्नि र॑बिभे दबिभे द॒ग्नि र॒ग्नि र॑बिभेत् । \newline
14. अ॒बि॒भे॒ दि॒त्थ मि॒त्थ म॑बिभे दबिभे दि॒त्थम् । \newline
15. इ॒त्थं ॅवाव वावे त्थ मि॒त्थं ॅवाव । \newline
16. वाव स्य स्य वाव वाव स्यः । \newline
17. स्य आर्ति॒ मार्तिꣳ॒॒ स्य स्य आर्ति᳚म् । \newline
18. आर्ति॒ मा ऽऽर्ति॒ मार्ति॒ मा । \newline
19. आ ऽरि॑ष्य त्यरिष्य॒त्या ऽरि॑ष्यति । \newline
20. अ॒रि॒ष्य॒तीती त्य॑रिष्य त्यरिष्य॒तीति॑ । \newline
21. इति॒ स स इतीति॒ सः । \newline
22. स निला॑यत॒ निला॑यत॒ स स निला॑यत । \newline
23. निला॑यत॒ स स निला॑यत॒ निला॑यत॒ सः । \newline
24. सो᳚(ओ1॒) ऽपो॑ ऽपः स सो॑ ऽपः । \newline
25. अ॒पः प्र प्रापो॑ ऽपः प्र । \newline
26. प्रावि॑श दविश॒त् प्र प्रावि॑शत् । \newline
27. अ॒वि॒श॒त् तम् त म॑विश दविश॒त् तम् । \newline
28. तम् दे॒वता॑ दे॒वता॒ स्तम् तम् दे॒वताः᳚ । \newline
29. दे॒वताः॒ प्रैष॒म् प्रैष॑म् दे॒वता॑ दे॒वताः॒ प्रैष᳚म् । \newline
30. प्रैष॑ मैच्छन् नैच्छ॒न् प्रैष॒म् प्रैष॑ मैच्छन्न् । \newline
31. प्रैष॒मिति॑ प्र - एष᳚म् । \newline
32. ऐ॒च्छ॒न् तम् त मै᳚च्छन् नैच्छ॒न् तम् । \newline
33. तम् मथ्स्यो॒ मथ्स्य॒ स्तम् तम् मथ्स्यः॑ । \newline
34. मथ्स्यः॒ प्र प्र मथ्स्यो॒ मथ्स्यः॒ प्र । \newline
35. प्राब्र॑वी दब्रवी॒त् प्र प्राब्र॑वीत् । \newline
36. अ॒ब्र॒वी॒त् तम् त म॑ब्रवी दब्रवी॒त् तम् । \newline
37. त म॑शप दशप॒त् तम् त म॑शपत् । \newline
38. अ॒श॒प॒द् धि॒याधि॑या धि॒याधि॑या ऽशप दशपद् धि॒याधि॑या । \newline
39. धि॒याधि॑या त्वा त्वा धि॒याधि॑या धि॒याधि॑या त्वा । \newline
40. धि॒याधि॒येति॑ धि॒या - धि॒या॒ । \newline
41. त्वा॒ व॒द्ध्या॒सु॒र् व॒द्ध्या॒सु॒ स्त्वा॒ त्वा॒ व॒द्ध्या॒सुः॒ । \newline
42. व॒द्ध्या॒सु॒र् यो यो व॑द्ध्यासुर् वद्ध्यासु॒र् यः । \newline
43. यो मा॑ मा॒ यो यो मा᳚ । \newline
44. मा॒ प्र प्र मा॑ मा॒ प्र । \newline
45. प्रावो॒चो ऽवो॑चः॒ प्र प्रावो॑चः । \newline
46. अवो॑च॒ इती त्यवो॒चो ऽवो॑च॒ इति॑ । \newline
47. इति॒ तस्मा॒त् तस्मा॒ दितीति॒ तस्मा᳚त् । \newline
48. तस्मा॒न् मथ्स्य॒म् मथ्स्य॒म् तस्मा॒त् तस्मा॒न् मथ्स्य᳚म् । \newline
49. मथ्स्य॑म् धि॒याधि॑या धि॒याधि॑या॒ मथ्स्य॒म् मथ्स्य॑म् धि॒याधि॑या । \newline
50. धि॒याधि॑या घ्नन्ति घ्नन्ति धि॒याधि॑या धि॒याधि॑या घ्नन्ति । \newline
51. धि॒याधि॒येति॑ धि॒या - धि॒या॒ । \newline
52. घ्न॒न्ति॒ श॒प्तः श॒प्तो घ्न॑न्ति घ्नन्ति श॒प्तः । \newline
53. श॒प्तो हि हि श॒प्तः श॒प्तो हि । \newline

\textbf{Ghana Paata } \newline

1. अ॒ग्ने स्त्रय॒ स्त्रयो॒ ऽग्ने र॒ग्ने स्त्रयो॒ ज्यायाꣳ॑सो॒ ज्यायाꣳ॑स॒ स्त्रयो॒ ऽग्ने र॒ग्ने स्त्रयो॒ ज्यायाꣳ॑सः । \newline
2. त्रयो॒ ज्यायाꣳ॑सो॒ ज्यायाꣳ॑स॒ स्त्रय॒ स्त्रयो॒ ज्यायाꣳ॑सो॒ भ्रात॑रो॒ भ्रात॑रो॒ ज्यायाꣳ॑स॒ स्त्रय॒ स्त्रयो॒ ज्यायाꣳ॑सो॒ भ्रात॑रः । \newline
3. ज्यायाꣳ॑सो॒ भ्रात॑रो॒ भ्रात॑रो॒ ज्यायाꣳ॑सो॒ ज्यायाꣳ॑सो॒ भ्रात॑र आसन् नास॒न् भ्रात॑रो॒ ज्यायाꣳ॑सो॒ ज्यायाꣳ॑सो॒ भ्रात॑र आसन्न् । \newline
4. भ्रात॑र आसन् नास॒न् भ्रात॑रो॒ भ्रात॑र आस॒न् ते त आ॑स॒न् भ्रात॑रो॒ भ्रात॑र आस॒न् ते । \newline
5. आ॒स॒न् ते त आ॑सन् नास॒न् ते दे॒वेभ्यो॑ दे॒वेभ्य॒ स्त आ॑सन् नास॒न् ते दे॒वेभ्यः॑ । \newline
6. ते दे॒वेभ्यो॑ दे॒वेभ्य॒ स्ते ते दे॒वेभ्यो॑ ह॒व्यꣳ ह॒व्यम् दे॒वेभ्य॒ स्ते ते दे॒वेभ्यो॑ ह॒व्यम् । \newline
7. दे॒वेभ्यो॑ ह॒व्यꣳ ह॒व्यम् दे॒वेभ्यो॑ दे॒वेभ्यो॑ ह॒व्यं ॅवह॑न्तो॒ वह॑न्तो ह॒व्यम् दे॒वेभ्यो॑ दे॒वेभ्यो॑ ह॒व्यं ॅवह॑न्तः । \newline
8. ह॒व्यं ॅवह॑न्तो॒ वह॑न्तो ह॒व्यꣳ ह॒व्यं ॅवह॑न्तः॒ प्र प्र वह॑न्तो ह॒व्यꣳ ह॒व्यं ॅवह॑न्तः॒ प्र । \newline
9. वह॑न्तः॒ प्र प्र वह॑न्तो॒ वह॑न्तः॒ प्रामी॑यन्ता मीयन्त॒ प्र वह॑न्तो॒ वह॑न्तः॒ प्रामी॑यन्त । \newline
10. प्रामी॑यन्ता मीयन्त॒ प्र प्रामी॑यन्त॒ स सो॑ ऽमीयन्त॒ प्र प्रामी॑यन्त॒ सः । \newline
11. अ॒मी॒य॒न्त॒ स सो॑ ऽमीयन्ता मीयन्त॒ सो᳚ ऽग्निर॒ग्निः सो॑ ऽमीयन्ता मीयन्त॒ सो᳚ ऽग्निः । \newline
12. सो᳚ ऽग्नि र॒ग्निः स सो᳚ ऽग्नि र॑बिभे दबिभे द॒ग्निः स सो᳚ ऽग्नि र॑बिभेत् । \newline
13. अ॒ग्नि र॑बिभे दबिभे द॒ग्नि र॒ग्नि र॑बिभे दि॒त्थ मि॒त्थ म॑बिभे द॒ग्नि र॒ग्नि र॑बिभे दि॒त्थम् । \newline
14. अ॒बि॒भे॒ दि॒त्थ मि॒त्थ म॑बिभे दबिभे दि॒त्थं ॅवाव वावे त्थ म॑बिभे दबिभे दि॒त्थं ॅवाव । \newline
15. इ॒त्थं ॅवाव वावे त्थ मि॒त्थं ॅवाव स्य स्य वावे त्थ मि॒त्थं ॅवाव स्यः । \newline
16. वाव स्य स्य वाव वाव स्य आर्ति॒ मार्तिꣳ॒॒ स्य वाव वाव स्य आर्ति᳚म् । \newline
17. स्य आर्ति॒ मार्तिꣳ॒॒ स्य स्य आर्ति॒ मा ऽऽर्तिꣳ॒॒ स्य स्य आर्ति॒ मा । \newline
18. आर्ति॒ मा ऽऽर्ति॒ मार्ति॒ मा ऽरि॑ष्य त्यरिष्य॒ त्या ऽऽर्ति॒ मार्ति॒ मा ऽरि॑ष्यति । \newline
19. आ ऽरि॑ष्य त्यरिष्य॒ त्या ऽरि॑ष्य॒तीती त्य॑रिष्य॒ त्या ऽरि॑ष्य॒तीति॑ । \newline
20. अ॒रि॒ष्य॒तीती त्य॑रिष्य त्यरिष्य॒तीति॒ स स इत्य॑रिष्य त्यरिष्य॒तीति॒ सः । \newline
21. इति॒ स स इतीति॒ स निला॑यत॒ निला॑यत॒ स इतीति॒ स निला॑यत । \newline
22. स निला॑यत॒ निला॑यत॒ स स निला॑यत॒ स स निला॑यत॒ स स निला॑यत॒ सः । \newline
23. निला॑यत॒ स स निला॑यत॒ निला॑यत॒ सो᳚(ओ1॒) ऽपो॑ ऽपः स निला॑यत॒ निला॑यत॒ सो॑ ऽपः । \newline
24. सो᳚(ओ1॒) ऽपो॑ ऽपः स सो॑ ऽपः प्र प्रापः स सो॑ ऽपः प्र । \newline
25. अ॒पः प्र प्रापो॑ ऽपः प्रावि॑श दविश॒त् प्रापो॑ ऽपः प्रावि॑शत् । \newline
26. प्रावि॑श दविश॒त् प्र प्रावि॑श॒त् तम् त म॑विश॒त् प्र प्रावि॑श॒त् तम् । \newline
27. अ॒वि॒श॒त् तम् त म॑विश दविश॒त् तम् दे॒वता॑ दे॒वता॒ स्त म॑विश दविश॒त् तम् दे॒वताः᳚ । \newline
28. तम् दे॒वता॑ दे॒वता॒ स्तम् तम् दे॒वताः॒ प्रैष॒म् प्रैष॑म् दे॒वता॒ स्तम् तम् दे॒वताः॒ प्रैष᳚म् । \newline
29. दे॒वताः॒ प्रैष॒म् प्रैष॑म् दे॒वता॑ दे॒वताः॒ प्रैष॑ मैच्छन् नैच्छ॒न् प्रैष॑म् दे॒वता॑ दे॒वताः॒ प्रैष॑ मैच्छन्न् । \newline
30. प्रैष॑ मैच्छन् नैच्छ॒न् प्रैष॒म् प्रैष॑ मैच्छ॒न् तम् त मै᳚च्छ॒न् प्रैष॒म् प्रैष॑ मैच्छ॒न् तम् । \newline
31. प्रैष॒मिति॑ प्र - एष᳚म् । \newline
32. ऐ॒च्छ॒न् तम् त मै᳚च्छन् नैच्छ॒न् तम् मथ्स्यो॒ मथ्स्य॒स्त मै᳚च्छन् नैच्छ॒न् तम् मथ्स्यः॑ । \newline
33. तम् मथ्स्यो॒ मथ्स्य॒ स्तम् तम् मथ्स्यः॒ प्र प्र मथ्स्य॒ स्तम् तम् मथ्स्यः॒ प्र । \newline
34. मथ्स्यः॒ प्र प्र मथ्स्यो॒ मथ्स्यः॒ प्राब्र॑वी दब्रवी॒त् प्र मथ्स्यो॒ मथ्स्यः॒ प्राब्र॑वीत् । \newline
35. प्राब्र॑वी दब्रवी॒त् प्र प्राब्र॑वी॒त् तम् त म॑ब्रवी॒त् प्र प्राब्र॑वी॒त् तम् । \newline
36. अ॒ब्र॒वी॒त् तम् त म॑ब्रवी दब्रवी॒त् त म॑शप दशप॒त् त म॑ब्रवी दब्रवी॒त् त म॑शपत् । \newline
37. त म॑शप दशप॒त् तम् त म॑शपद् धि॒याधि॑या धि॒याधि॑या ऽशप॒त् तम् त म॑शपद् धि॒याधि॑या । \newline
38. अ॒श॒प॒द् धि॒याधि॑या धि॒याधि॑या ऽशप दशपद् धि॒याधि॑या त्वा त्वा धि॒याधि॑या ऽशप दशपद् धि॒याधि॑या त्वा । \newline
39. धि॒याधि॑या त्वा त्वा धि॒याधि॑या धि॒याधि॑या त्वा वद्ध्यासुर् वद्ध्यासु स्त्वा धि॒याधि॑या धि॒याधि॑या त्वा वद्ध्यासुः । \newline
40. धि॒याधि॒येति॑ धि॒या - धि॒या॒ । \newline
41. त्वा॒ व॒द्ध्या॒सु॒र् व॒द्ध्या॒सु॒ स्त्वा॒ त्वा॒ व॒द्ध्या॒सु॒र् यो यो व॑द्ध्यासु स्त्वा त्वा वद्ध्यासु॒र् यः । \newline
42. व॒द्ध्या॒सु॒र् यो यो व॑द्ध्यासुर् वद्ध्यासु॒र् यो मा॑ मा॒ यो व॑द्ध्यासुर् वद्ध्यासु॒र् यो मा᳚ । \newline
43. यो मा॑ मा॒ यो यो मा॒ प्र प्र मा॒ यो यो मा॒ प्र । \newline
44. मा॒ प्र प्र मा॑ मा॒ प्रावो॒चो ऽवो॑चः॒ प्र मा॑ मा॒ प्रावो॑चः । \newline
45. प्रावो॒चो ऽवो॑चः॒ प्र प्रावो॑च॒ इती त्यवो॑चः॒ प्र प्रावो॑च॒ इति॑ । \newline
46. अवो॑च॒ इती त्यवो॒चो ऽवो॑च॒ इति॒ तस्मा॒त् तस्मा॒ दित्यवो॒चो ऽवो॑च॒ इति॒ तस्मा᳚त् । \newline
47. इति॒ तस्मा॒त् तस्मा॒ दितीति॒ तस्मा॒न् मथ्स्य॒म् मथ्स्य॒म् तस्मा॒ दितीति॒ तस्मा॒न् मथ्स्य᳚म् । \newline
48. तस्मा॒न् मथ्स्य॒म् मथ्स्य॒म् तस्मा॒त् तस्मा॒न् मथ्स्य॑म् धि॒याधि॑या धि॒याधि॑या॒ मथ्स्य॒म् तस्मा॒त् तस्मा॒न् मथ्स्य॑म् धि॒याधि॑या । \newline
49. मथ्स्य॑म् धि॒याधि॑या धि॒याधि॑या॒ मथ्स्य॒म् मथ्स्य॑म् धि॒याधि॑या घ्नन्ति घ्नन्ति धि॒याधि॑या॒ मथ्स्य॒म् मथ्स्य॑म् धि॒याधि॑या घ्नन्ति । \newline
50. धि॒याधि॑या घ्नन्ति घ्नन्ति धि॒याधि॑या धि॒याधि॑या घ्नन्ति श॒प्तः श॒प्तो घ्न॑न्ति धि॒याधि॑या धि॒याधि॑या घ्नन्ति श॒प्तः । \newline
51. धि॒याधि॒येति॑ धि॒या - धि॒या॒ । \newline
52. घ्न॒न्ति॒ श॒प्तः श॒प्तो घ्न॑न्ति घ्नन्ति श॒प्तो हि हि श॒प्तो घ्न॑न्ति घ्नन्ति श॒प्तो हि । \newline
53. श॒प्तो हि हि श॒प्तः श॒प्तो हि तम् तꣳ हि श॒प्तः श॒प्तो हि तम् । \newline
\pagebreak
\markright{ TS 2.6.6.2  \hfill https://www.vedavms.in \hfill}
\addcontentsline{toc}{section}{ TS 2.6.6.2 }
\section*{ TS 2.6.6.2 }

\textbf{TS 2.6.6.2 } \newline
\textbf{Samhita Paata} \newline

हि तमन्व॑विन्द॒न् तम॑ ब्रुव॒न्नुप॑ न॒ आ व॑र्तस्व ह॒व्यं नो॑ व॒हेति॒ सो᳚ऽब्रवी॒द्वरं॑ ॅवृणै॒ यदे॒व गृ॑ही॒तस्याहु॑तस्यबहिः परि॒धि स्कन्दा॒त् तन्मे॒ भ्रातृ॑णां भाग॒धेय॑मस॒दिति॒ तस्मा॒द्यद् गृ॑ही॒तस्याहु॑तस्य बहिः परि॒धि स्कन्द॑ति॒ तेषां॒ तद्-भा॑ग॒धेयं॒ ताने॒व तेन॑ प्रीणाति परि॒धीन् परि॑ दधाति॒ रक्ष॑सा॒मप॑हत्यै॒ सꣳ स्प॑र्.शयति॒ - [  ] \newline

\textbf{Pada Paata} \newline

हि । तम् । अन्विति॑ । अ॒वि॒न्द॒न्न् । तम् । अ॒ब्रु॒व॒न्न् । उपेति॑ । नः॒ । एति॑ । व॒र्त॒स्व॒ । ह॒व्यम् । नः॒ । व॒ह॒ । इति॑ । सः । अ॒ब्र॒वी॒त् । वर᳚म् । वृ॒णै॒ । यत् । ए॒व । गृ॒ही॒तस्य॑ । अहु॑तस्य । ब॒हिः॒ प॒रि॒धीति॑ बहिः - प॒रि॒धि । स्कन्दा᳚त् । तत् । मे॒ । भ्रातृ॑णाम् । भा॒ग॒धेय॒मिति॑ भाग - धेय᳚म् । अ॒स॒त् । इति॑ । तस्मा᳚त् । यत् । गृ॒ही॒तस्य॑ । अहु॑तस्य । ब॒हिः॒ प॒रि॒धीति॑ बहिः - प॒रि॒धि । स्कन्द॑ति । तेषा᳚म् । तत् । भा॒ग॒धेय॒मिति॑ भाग - धेय᳚म् । तान् । ए॒व । तेन॑ । प्री॒णा॒ति॒ । प॒रि॒धीनिति॑ परि - धीन् । परीति॑ । द॒धा॒ति॒ । रक्ष॑साम् । अप॑हत्या॒ इत्यप॑ - ह॒त्यै॒ । समिति॑ । स्प॒र्॒.श॒य॒ति॒ ।  \newline


\textbf{Krama Paata} \newline

हि तम् । तमनु॑ । अन्व॑विन्दन्न् । अ॒वि॒न्द॒न् तम् । तम॑ब्रुवन्न् । अ॒ब्रु॒व॒न्नुप॑ । उप॑ नः । न॒ आ । आ व॑र्तस्व । व॒र्त॒स्व॒ ह॒व्यम् । ह॒व्यम् नः॑ । नो॒ व॒ह॒ । व॒हेति॑ । इति॒ सः । सो᳚ऽब्रवीत् । अ॒ब्र॒वी॒द् वर᳚म् । वरं॑ ॅवृणै । वृ॒णै॒ यत् । यदे॒व । ए॒व गृ॑ही॒तस्य॑ । गृ॒ही॒तस्याहु॑तस्य । अहु॑तस्य बहिःपरि॒धि । ब॒हिः॒प॒रि॒धि स्कन्दा᳚त् । ब॒हिः॒प॒रि॒धीति॑ बहिः - प॒रि॒धि । स्कन्दा॒त् तत् । तन् मे᳚ । मे॒ भ्रातृ॑णाम् । भ्रातृ॑णाम् भाग॒धेय᳚म् । भा॒ग॒धेय॑मसत् । भा॒ग॒धेय॒मिति॑ भाग - धेय᳚म् । अ॒स॒दिति॑ । इति॒ तस्मा᳚त् । तस्मा॒द् यत् । यद् गृ॑ही॒तस्य॑ । गृ॒ही॒तस्याहु॑तस्य । अहु॑तस्य बहिःपरि॒धि । ब॒हिः॒प॒रि॒धि स्कन्द॑ति । ब॒हिः॒प॒रि॒धीति॑ बहिः - प॒रि॒धि । स्कन्द॑ति॒ तेषा᳚म् । तेषा॒म् तत् । तद् भा॑ग॒धेय᳚म् । भा॒ग॒धेय॒म् तान् । भा॒ग॒धेय॒मिति॑ भाग - धेय᳚म् । ताने॒व । ए॒व तेन॑ । तेन॑ प्रीणाति । प्री॒णा॒ति॒ प॒रि॒धीन् । प॒रि॒धीन् परि॑ । प॒रि॒धीनिति॑ परि - धीन् । परि॑ दधाति । द॒धा॒ति॒ रक्ष॑साम् । रक्ष॑सा॒मप॑हत्यै । अप॑हत्यै॒ सम् । अप॑हत्या॒ इत्यप॑ - ह॒त्यै॒ । सꣳ स्प॑र्.शयति । स्प॒र्.॒श॒य॒ति॒ रक्ष॑साम् \newline

\textbf{Jatai Paata} \newline

1. हि तम् तꣳ हि हि तम् । \newline
2. त मन्वनु॒ तम् त मनु॑ । \newline
3. अन्व॑विन्दन् नविन्द॒न् नन्वन्व॑ विन्दन्न् । \newline
4. अ॒वि॒न्द॒न् तम् त म॑विन्दन् नविन्द॒न् तम् । \newline
5. त म॑ब्रुवन् नब्रुव॒न् तम् त म॑ब्रुवन्न् । \newline
6. अ॒ब्रु॒व॒न् नुपोपा᳚ब्रुवन् नब्रुव॒न् नुप॑ । \newline
7. उप॑ नो न॒ उपोप॑ नः । \newline
8. न॒ आ नो॑ न॒ आ । \newline
9. आ व॑र्तस्व वर्त॒स्वा व॑र्तस्व । \newline
10. व॒र्त॒स्व॒ ह॒व्यꣳ ह॒व्यं ॅव॑र्तस्व वर्तस्व ह॒व्यम् । \newline
11. ह॒व्यम् नो॑ नो ह॒व्यꣳ ह॒व्यम् नः॑ । \newline
12. नो॒ व॒ह॒ व॒ह॒ नो॒ नो॒ व॒ह॒ । \newline
13. व॒हे तीति॑ वह व॒हे ति॑ । \newline
14. इति॒ स स इतीति॒ सः । \newline
15. सो᳚ ऽब्रवी दब्रवी॒थ् स सो᳚ ऽब्रवीत् । \newline
16. अ॒ब्र॒वी॒द् वरं॒ ॅवर॑ मब्रवी दब्रवी॒द् वर᳚म् । \newline
17. वरं॑ ॅवृणै वृणै॒ वरं॒ ॅवरं॑ ॅवृणै । \newline
18. वृ॒णै॒ यद् यद् वृ॑णै वृणै॒ यत् । \newline
19. यदे॒वैव यद् यदे॒व । \newline
20. ए॒व गृ॑ही॒तस्य॑ गृही॒त स्यै॒वैव गृ॑ही॒तस्य॑ । \newline
21. गृ॒ही॒तस्या हु॑त॒स्या हु॑तस्य गृही॒तस्य॑ गृही॒तस्या हु॑तस्य । \newline
22. अहु॑तस्य बहिःपरि॒धि ब॑हिःपरि॒ ध्यहु॑त॒स्या हु॑तस्य बहिःपरि॒धि । \newline
23. ब॒हिः॒प॒रि॒धि स्कन्दा॒थ् स्कन्दा᳚द् बहिःपरि॒धि ब॑हिःपरि॒धि स्कन्दा᳚त् । \newline
24. ब॒हिः॒प॒रि॒धीति॑ बहिः - प॒रि॒धि । \newline
25. स्कन्दा॒त् तत् तथ् स्कन्दा॒थ् स्कन्दा॒त् तत् । \newline
26. तन् मे॑ मे॒ तत् तन् मे᳚ । \newline
27. मे॒ भ्रातृ॑णा॒म् भ्रातृ॑णाम् मे मे॒ भ्रातृ॑णाम् । \newline
28. भ्रातृ॑णाम् भाग॒धेय॑म् भाग॒धेय॒म् भ्रातृ॑णा॒म् भ्रातृ॑णाम् भाग॒धेय᳚म् । \newline
29. भा॒ग॒धेय॑ मसदसद् भाग॒धेय॑म् भाग॒धेय॑ मसत् । \newline
30. भा॒ग॒धेय॒मिति॑ भाग - धेय᳚म् । \newline
31. अ॒स॒ दिती त्य॑सदस॒ दिति॑ । \newline
32. इति॒ तस्मा॒त् तस्मा॒ दितीति॒ तस्मा᳚त् । \newline
33. तस्मा॒द् यद् यत् तस्मा॒त् तस्मा॒द् यत् । \newline
34. यद् गृ॑ही॒तस्य॑ गृही॒तस्य॒ यद् यद् गृ॑ही॒तस्य॑ । \newline
35. गृ॒ही॒तस्या हु॑त॒स्या हु॑तस्य गृही॒तस्य॑ गृही॒तस्या हु॑तस्य । \newline
36. अहु॑तस्य बहिःपरि॒धि ब॑हिःपरि॒ ध्यहु॑त॒स्या हु॑तस्य बहिःपरि॒धि । \newline
37. ब॒हिः॒प॒रि॒धि स्कन्द॑ति॒ स्कन्द॑ति बहिःपरि॒धि ब॑हिःपरि॒धि स्कन्द॑ति । \newline
38. ब॒हिः॒प॒रि॒धीति॑ बहिः - प॒रि॒धि । \newline
39. स्कन्द॑ति॒ तेषा॒म् तेषाꣳ॒॒ स्कन्द॑ति॒ स्कन्द॑ति॒ तेषा᳚म् । \newline
40. तेषा॒म् तत् तत् तेषा॒म् तेषा॒म् तत् । \newline
41. तद् भा॑ग॒धेय॑म् भाग॒धेय॒म् तत् तद् भा॑ग॒धेय᳚म् । \newline
42. भा॒ग॒धेय॒म् ताꣳ स्तान् भा॑ग॒धेय॑म् भाग॒धेय॒म् तान् । \newline
43. भा॒ग॒धेय॒मिति॑ भाग - धेय᳚म् । \newline
44. ता ने॒वैव ताꣳ स्ता ने॒व । \newline
45. ए॒व तेन॒ तेनै॒वैव तेन॑ । \newline
46. तेन॑ प्रीणाति प्रीणाति॒ तेन॒ तेन॑ प्रीणाति । \newline
47. प्री॒णा॒ति॒ प॒रि॒धीन् प॑रि॒धीन् प्री॑णाति प्रीणाति परि॒धीन् । \newline
48. प॒रि॒धीन् परि॒ परि॑ परि॒धीन् प॑रि॒धीन् परि॑ । \newline
49. प॒रि॒धीनिति॑ परि - धीन् । \newline
50. परि॑ दधाति दधाति॒ परि॒ परि॑ दधाति । \newline
51. द॒धा॒ति॒ रक्ष॑साꣳ॒॒ रक्ष॑साम् दधाति दधाति॒ रक्ष॑साम् । \newline
52. रक्ष॑सा॒ मप॑हत्या॒ अप॑हत्यै॒ रक्ष॑साꣳ॒॒ रक्ष॑सा॒ मप॑हत्यै । \newline
53. अप॑हत्यै॒ सꣳ स मप॑हत्या॒ अप॑हत्यै॒ सम् । \newline
54. अप॑हत्या॒ इत्यप॑ - ह॒त्यै॒ । \newline
55. सꣳ स्प॑र्.शयति स्पर्.शयति॒ सꣳ सꣳ स्प॑र्.शयति । \newline
56. स्प॒र्॒.श॒य॒ति॒ रक्ष॑साꣳ॒॒ रक्ष॑साꣳ स्पर्.शयति स्पर्.शयति॒ रक्ष॑साम् । \newline

\textbf{Ghana Paata } \newline

1. हि तम् तꣳ हि हि त मन्वनु॒ तꣳ हि हि त मनु॑ । \newline
2. त मन्वनु॒ तम् त मन्व॑विन्दन् नविन्द॒न् ननु॒ तम् त मन्व॑विन्दन्न् । \newline
3. अन्व॑विन्दन् नविन्द॒न् नन्वन्व॑ विन्द॒न् तम् त म॑विन्द॒न् नन्वन्व॑ विन्द॒न् तम् । \newline
4. अ॒वि॒न्द॒न् तम् त म॑विन्दन् नविन्द॒न् त म॑ब्रुवन् नब्रुव॒न् त म॑विन्दन् नविन्द॒न् त म॑ब्रुवन्न् । \newline
5. त म॑ब्रुवन् नब्रुव॒न् तम् त म॑ब्रुव॒न् नुपोपा᳚ब्रुव॒न् तम् त म॑ब्रुव॒न् नुप॑ । \newline
6. अ॒ब्रु॒व॒न् नुपोपा᳚ब्रुवन् नब्रुव॒न् नुप॑ नो न॒ उपा᳚ब्रुवन् नब्रुव॒न् नुप॑ नः । \newline
7. उप॑ नो न॒ उपोप॑ न॒ आ न॒ उपोप॑ न॒ आ । \newline
8. न॒ आ नो॑ न॒ आ व॑र्तस्व वर्त॒स्वा नो॑ न॒ आ व॑र्तस्व । \newline
9. आ व॑र्तस्व वर्त॒स्वा व॑र्तस्व ह॒व्यꣳ ह॒व्यं ॅव॑र्त॒स्वा व॑र्तस्व ह॒व्यम् । \newline
10. व॒र्त॒स्व॒ ह॒व्यꣳ ह॒व्यं ॅव॑र्तस्व वर्तस्व ह॒व्यम् नो॑ नो ह॒व्यं ॅव॑र्तस्व वर्तस्व ह॒व्यम् नः॑ । \newline
11. ह॒व्यम् नो॑ नो ह॒व्यꣳ ह॒व्यम् नो॑ वह वह नो ह॒व्यꣳ ह॒व्यम् नो॑ वह । \newline
12. नो॒ व॒ह॒ व॒ह॒ नो॒ नो॒ व॒हे तीति॑ वह नो नो व॒हे ति॑ । \newline
13. व॒हे तीति॑ वह व॒हे ति॒ स स इति॑ वह व॒हे ति॒ सः । \newline
14. इति॒ स स इतीति॒ सो᳚ ऽब्रवी दब्रवी॒थ् स इतीति॒ सो᳚ ऽब्रवीत् । \newline
15. सो᳚ ऽब्रवी दब्रवी॒थ् स सो᳚ ऽब्रवी॒द् वरं॒ ॅवर॑ मब्रवी॒थ् स सो᳚ ऽब्रवी॒द् वर᳚म् । \newline
16. अ॒ब्र॒वी॒द् वरं॒ ॅवर॑ मब्रवी दब्रवी॒द् वरं॑ ॅवृणै वृणै॒ वर॑ मब्रवी दब्रवी॒द् वरं॑ ॅवृणै । \newline
17. वरं॑ ॅवृणै वृणै॒ वरं॒ ॅवरं॑ ॅवृणै॒ यद् यद् वृ॑णै॒ वरं॒ ॅवरं॑ ॅवृणै॒ यत् । \newline
18. वृ॒णै॒ यद् यद् वृ॑णै वृणै॒ यदे॒वैव यद् वृ॑णै वृणै॒ यदे॒व । \newline
19. यदे॒वैव यद् यदे॒व गृ॑ही॒तस्य॑ गृही॒तस्यै॒व यद् यदे॒व गृ॑ही॒तस्य॑ । \newline
20. ए॒व गृ॑ही॒तस्य॑ गृही॒तस्यै॒वैव गृ॑ही॒तस्या हु॑त॒स्या हु॑तस्य गृही॒तस्यै॒वैव गृ॑ही॒तस्या हु॑तस्य । \newline
21. गृ॒ही॒तस्या हु॑त॒स्या हु॑तस्य गृही॒तस्य॑ गृही॒तस्या हु॑तस्य बहिःपरि॒धि ब॑हिःपरि॒ ध्यहु॑तस्य गृही॒तस्य॑ गृही॒तस्या हु॑तस्य बहिःपरि॒धि । \newline
22. अहु॑तस्य बहिःपरि॒धि ब॑हिःपरि॒ ध्यहु॑त॒स्या हु॑तस्य बहिःपरि॒धि स्कन्दा॒थ् स्कन्दा᳚द् बहिःपरि॒ ध्यहु॑त॒स्या हु॑तस्य 
बहिःपरि॒धि स्कन्दा᳚त् । \newline
23. ब॒हिः॒प॒रि॒धि स्कन्दा॒थ् स्कन्दा᳚द् बहिःपरि॒धि ब॑हिःपरि॒धि स्कन्दा॒त् तत् तथ् स्कन्दा᳚द् बहिःपरि॒धि ब॑हिःपरि॒धि स्कन्दा॒त् तत् । \newline
24. ब॒हिः॒प॒रि॒धीति॑ बहिः - प॒रि॒धि । \newline
25. स्कन्दा॒त् तत् तथ् स्कन्दा॒थ् स्कन्दा॒त् तन् मे॑ मे॒ तथ् स्कन्दा॒थ् स्कन्दा॒त् तन् मे᳚ । \newline
26. तन् मे॑ मे॒ तत् तन् मे॒ भ्रातृ॑णा॒म् भ्रातृ॑णाम् मे॒ तत् तन् मे॒ भ्रातृ॑णाम् । \newline
27. मे॒ भ्रातृ॑णा॒म् भ्रातृ॑णाम् मे मे॒ भ्रातृ॑णाम् भाग॒धेय॑म् भाग॒धेय॒म् भ्रातृ॑णाम् मे मे॒ भ्रातृ॑णाम् भाग॒धेय᳚म् । \newline
28. भ्रातृ॑णाम् भाग॒धेय॑म् भाग॒धेय॒म् भ्रातृ॑णा॒म् भ्रातृ॑णाम् भाग॒धेय॑ मस दसद् भाग॒धेय॒म् भ्रातृ॑णा॒म् भ्रातृ॑णाम् भाग॒धेय॑ मसत् । \newline
29. भा॒ग॒धेय॑ मस दसद् भाग॒धेय॑म् भाग॒धेय॑ मस॒दिती त्य॑सद् भाग॒धेय॑म् भाग॒धेय॑ मस॒दिति॑ । \newline
30. भा॒ग॒धेय॒मिति॑ भाग - धेय᳚म् । \newline
31. अ॒स॒ दितीत्य॑स दस॒दिति॒ तस्मा॒त् तस्मा॒ दित्य॑स दस॒दिति॒ तस्मा᳚त् । \newline
32. इति॒ तस्मा॒त् तस्मा॒ दितीति॒ तस्मा॒द् यद् यत् तस्मा॒ दितीति॒ तस्मा॒द् यत् । \newline
33. तस्मा॒द् यद् यत् तस्मा॒त् तस्मा॒द् यद् गृ॑ही॒तस्य॑ गृही॒तस्य॒ यत् तस्मा॒त् तस्मा॒द् यद् गृ॑ही॒तस्य॑ । \newline
34. यद् गृ॑ही॒तस्य॑ गृही॒तस्य॒ यद् यद् गृ॑ही॒तस्या हु॑त॒स्या हु॑तस्य गृही॒तस्य॒ यद् यद् गृ॑ही॒तस्या हु॑तस्य । \newline
35. गृ॒ही॒तस्या हु॑त॒स्या हु॑तस्य गृही॒तस्य॑ गृही॒तस्या हु॑तस्य बहिःपरि॒धि ब॑हिःपरि॒ ध्यहु॑तस्य गृही॒तस्य॑ गृही॒तस्या हु॑तस्य बहिःपरि॒धि । \newline
36. अहु॑तस्य बहिःपरि॒धि ब॑हिःपरि॒ ध्यहु॑त॒स्या हु॑तस्य बहिःपरि॒धि स्कन्द॑ति॒ स्कन्द॑ति बहिःपरि॒ ध्यहु॑त॒स्या हु॑तस्य 
बहिःपरि॒धि स्कन्द॑ति । \newline
37. ब॒हिः॒प॒रि॒धि स्कन्द॑ति॒ स्कन्द॑ति बहिःपरि॒धि ब॑हिःपरि॒धि स्कन्द॑ति॒ तेषा॒म् तेषाꣳ॒॒ स्कन्द॑ति बहिःपरि॒धि 
ब॑हिःपरि॒धि स्कन्द॑ति॒ तेषा᳚म् । \newline
38. ब॒हिः॒प॒रि॒धीति॑ बहिः - प॒रि॒धि । \newline
39. स्कन्द॑ति॒ तेषा॒म् तेषाꣳ॒॒ स्कन्द॑ति॒ स्कन्द॑ति॒ तेषा॒म् तत् तत् तेषाꣳ॒॒ स्कन्द॑ति॒ स्कन्द॑ति॒ तेषा॒म् तत् । \newline
40. तेषा॒म् तत् तत् तेषा॒म् तेषा॒म् तद् भा॑ग॒धेय॑म् भाग॒धेय॒म् तत् तेषा॒म् तेषा॒म् तद् भा॑ग॒धेय᳚म् । \newline
41. तद् भा॑ग॒धेय॑म् भाग॒धेय॒म् तत् तद् भा॑ग॒धेय॒म् ताꣳ स्तान् भा॑ग॒धेय॒म् तत् तद् भा॑ग॒धेय॒म् तान् । \newline
42. भा॒ग॒धेय॒म् ताꣳ स्तान् भा॑ग॒धेय॑म् भाग॒धेय॒म् ता ने॒वैव तान् भा॑ग॒धेय॑म् भाग॒धेय॒म् ता ने॒व । \newline
43. भा॒ग॒धेय॒मिति॑ भाग - धेय᳚म् । \newline
44. ता ने॒वैव ताꣳ स्ता ने॒व तेन॒ तेनै॒व ताꣳ स्ता ने॒व तेन॑ । \newline
45. ए॒व तेन॒ तेनै॒वैव तेन॑ प्रीणाति प्रीणाति॒ तेनै॒वैव तेन॑ प्रीणाति । \newline
46. तेन॑ प्रीणाति प्रीणाति॒ तेन॒ तेन॑ प्रीणाति परि॒धीन् प॑रि॒धीन् प्री॑णाति॒ तेन॒ तेन॑ प्रीणाति परि॒धीन् । \newline
47. प्री॒णा॒ति॒ प॒रि॒धीन् प॑रि॒धीन् प्री॑णाति प्रीणाति परि॒धीन् परि॒ परि॑ परि॒धीन् प्री॑णाति प्रीणाति परि॒धीन् परि॑ । \newline
48. प॒रि॒धीन् परि॒ परि॑ परि॒धीन् प॑रि॒धीन् परि॑ दधाति दधाति॒ परि॑ परि॒धीन् प॑रि॒धीन् परि॑ दधाति । \newline
49. प॒रि॒धीनिति॑ परि - धीन् । \newline
50. परि॑ दधाति दधाति॒ परि॒ परि॑ दधाति॒ रक्ष॑साꣳ॒॒ रक्ष॑साम् दधाति॒ परि॒ परि॑ दधाति॒ रक्ष॑साम् । \newline
51. द॒धा॒ति॒ रक्ष॑साꣳ॒॒ रक्ष॑साम् दधाति दधाति॒ रक्ष॑सा॒ मप॑हत्या॒ अप॑हत्यै॒ रक्ष॑साम् दधाति दधाति॒ रक्ष॑सा॒ मप॑हत्यै । \newline
52. रक्ष॑सा॒ मप॑हत्या॒ अप॑हत्यै॒ रक्ष॑साꣳ॒॒ रक्ष॑सा॒ मप॑हत्यै॒ सꣳ स मप॑हत्यै॒ रक्ष॑साꣳ॒॒ रक्ष॑सा॒ मप॑हत्यै॒ सम् । \newline
53. अप॑हत्यै॒ सꣳ स मप॑हत्या॒ अप॑हत्यै॒ सꣳ स्प॑र्.शयति स्पर्.शयति॒ स मप॑हत्या॒ अप॑हत्यै॒ सꣳ स्प॑र्.शयति । \newline
54. अप॑हत्या॒ इत्यप॑ - ह॒त्यै॒ । \newline
55. सꣳ स्प॑र्.शयति स्पर्.शयति॒ सꣳ सꣳ स्प॑र्.शयति॒ रक्ष॑साꣳ॒॒ रक्ष॑साꣳ स्पर्.शयति॒ सꣳ सꣳ स्प॑र्.शयति॒ रक्ष॑साम् । \newline
56. स्प॒र्॒.श॒य॒ति॒ रक्ष॑साꣳ॒॒ रक्ष॑साꣳ स्पर्.शयति स्पर्.शयति॒ रक्ष॑सा॒ मन॑न्ववचारा॒या न॑न्ववचाराय॒ रक्ष॑साꣳ स्पर्.शयति स्पर्.शयति॒ रक्ष॑सा॒ मन॑न्ववचाराय । \newline
\pagebreak
\markright{ TS 2.6.6.3  \hfill https://www.vedavms.in \hfill}
\addcontentsline{toc}{section}{ TS 2.6.6.3 }
\section*{ TS 2.6.6.3 }

\textbf{TS 2.6.6.3 } \newline
\textbf{Samhita Paata} \newline

रक्ष॑सा॒मन॑न्ववचाराय॒ न पु॒रस्ता॒त् परि॑ दधात्यादि॒त्यो ह्ये॑वोद्यन् पु॒रस्ता॒द्-रक्षाꣳ॑स्यप॒हन्त्यू॒र्द्ध्वे स॒मिधा॒वा द॑धात्यु॒परि॑ष्टादे॒व रक्षाꣳ॒॒स्यप॑ हन्ति॒ यजु॑षा॒ऽन्यां तू॒ष्णीम॒न्यां मि॑थुन॒त्वाय॒ द्वे आ द॑धाति द्वि॒पाद्-यज॑मानः॒ प्रति॑ष्ठित्यै ब्रह्मवा॒दिनो॑ वदन्ति॒ स त्वै य॑जेत॒ यो य॒ज्ञ्स्याऽऽर्त्या॒ वसी॑या॒न्थ् स्यादिति॒ भूप॑तये॒ स्वाहा॒ भुव॑नपतये॒ स्वाहा॑ भू॒तानां॒ - [  ] \newline

\textbf{Pada Paata} \newline

रक्ष॑साम् । अन॑न्ववचारा॒येत्यन॑नु - अ॒व॒चा॒रा॒य॒ । न । पु॒रस्ता᳚त् । परीति॑ । द॒धा॒ति॒ । आ॒दि॒त्यः । हि । ए॒व । उ॒द्यन्नित्यु॑त् - यन्न् । पु॒रस्ता᳚त् । रक्षाꣳ॑सि । अ॒प॒हन्तीत्य॑प - हन्ति॑ । ऊ॒र्द्ध्वे इति॑ । स॒मिधा॒विति॑ सं - इधौ᳚ । एति॑ । द॒धा॒ति॒ । उ॒परि॑ष्टात् । ए॒व । रक्षाꣳ॑सि । अपेति॑ । ह॒न्ति॒ । यजु॑षा । अ॒न्याम् । तू॒ष्णीम् । अ॒न्याम् । मि॒थु॒न॒त्वायेति॑ मिथुन - त्वाय॑ । द्वे इति॑ । एति॑ । द॒धा॒ति॒ । द्वि॒पादिति॑ द्वि - पात् । यज॑मानः । प्रति॑ष्ठित्या॒ इति॒ प्रति॑ - स्थि॒त्यै॒ । ब्र॒ह्म॒वा॒दिन॒ इति॑ ब्रह्म - वा॒दिनः॑ । व॒द॒न्ति॒ । सः । तु । वै । य॒जे॒त॒ । यः । य॒ज्ञ्स्य॑ । आर्त्या᳚ । वसी॑यान् । स्यात् । इति॑ । भूप॑तय॒ इति॒ भू - प॒त॒ये॒ । स्वाहा᳚ । भुव॑नपतय॒ इति॒ भुव॑न - प॒त॒ये॒ । स्वाहा᳚ । भू॒ताना᳚म् ।  \newline


\textbf{Krama Paata} \newline

रक्ष॑सा॒मन॑न्ववचाराय । अन॑न्ववचाराय॒ न । अन॑न्ववचारा॒येत्यन॑नु - अ॒व॒चा॒रा॒य॒ । न पु॒रस्ता᳚त् । पु॒रस्ता॒त् परि॑ । परि॑ दधाति । द॒धा॒त्या॒दि॒त्यः । आ॒दि॒त्यो हि । ह्ये॑व । ए॒वोद्यन्न् । उ॒द्यन् पु॒रस्ता᳚त् । उ॒द्यन्नित्यु॑त् - यन्न् । पु॒रस्ता॒द् रक्षाꣳ॑सि । रक्षाꣳ॑स्यप॒हन्ति॑ । अ॒प॒हन्त्यू॒र्द्ध्वे । अ॒प॒हन्तीत्य॑प - हन्ति॑ । ऊ॒र्द्ध्वे स॒मिधौ᳚ । ऊ॒र्द्ध्वे इत्यू॒र्द्ध्वे । स॒मिधा॒वा । स॒मिधा॒विति॑ सं - इधौ᳚ । आ द॑धाति । द॒धा॒त्यु॒परि॑ष्टात् । उ॒परि॑ष्टादे॒व । ए॒व रक्षाꣳ॑सि । रक्षाꣳ॒॒स्यप॑ । अप॑ हन्ति । ह॒न्ति॒ यजु॑षा । यजु॑षा॒ऽन्याम् । अ॒न्याम् तू॒ष्णीम् । तू॒ष्णीम॒न्याम् । अ॒न्याम् मि॑थुन॒त्वाय॑ । मि॒थु॒न॒त्वाय॒ द्वे । मि॒थु॒न॒त्वायेति॑ मिथुन - त्वाय॑ । द्वे आ । द्वे इति॒ द्वे । आ द॑धाति । द॒धा॒ति॒ द्वि॒पात् । द्वि॒पाद् यज॑मानः । द्वि॒पादिति॑ द्वि - पात् । यज॑मानः॒ प्रति॑ष्ठित्यै । प्रति॑ष्ठित्यै ब्रह्मवा॒दिनः॑ । प्रति॑ष्ठित्या॒ इति॒ प्रति॑ - स्थि॒त्यै॒ । ब्र॒ह्म॒वा॒दिनो॑ वदन्ति । ब्र॒ह्म॒वा॒दिन॒ इति॑ ब्रह्म - वा॒दिनः॑ । व॒द॒न्ति॒ सः । स तु । त्वै । वै य॑जेत । य॒जे॒त॒ यः । यो य॒ज्ञ्स्य॑ । य॒ज्ञ्स्यार्त्या᳚ । आर्त्या॒ वसी॑यान् । वसी॑या॒न्थ् स्यात् । स्या॒दिति॑ । इति॒ भूप॑तये । भूप॑तये॒ स्वाहा᳚ । भूप॑तय॒ इति॒ भू - प॒त॒ये॒ । स्वाहा॒ भुव॑नपतये । भुव॑नपतये॒ स्वाहा᳚ । भुव॑नपतय॒ इति॒ भुव॑न - प॒त॒ये॒ । स्वाहा॑ भू॒ताना᳚म् । भू॒ताना॒म् पत॑ये \newline

\textbf{Jatai Paata} \newline

1. रक्ष॑सा॒ मन॑न्ववचारा॒या न॑न्ववचाराय॒ रक्ष॑साꣳ॒॒ रक्ष॑सा॒ मन॑न्ववचाराय । \newline
2. अन॑न्ववचाराय॒ न नान॑न्ववचारा॒या न॑न्ववचाराय॒ न । \newline
3. अन॑न्ववचारा॒येत्यन॑नु - अ॒व॒चा॒रा॒य॒ । \newline
4. न पु॒रस्ता᳚त् पु॒रस्ता॒न् न न पु॒रस्ता᳚त् । \newline
5. पु॒रस्ता॒त् परि॒ परि॑ पु॒रस्ता᳚त् पु॒रस्ता॒त् परि॑ । \newline
6. परि॑ दधाति दधाति॒ परि॒ परि॑ दधाति । \newline
7. द॒धा॒ त्या॒दि॒त्य आ॑दि॒त्यो द॑धाति दधा त्यादि॒त्यः । \newline
8. आ॒दि॒त्यो हि ह्या॑दि॒त्य आ॑दि॒त्यो हि । \newline
9. ह्ये॑वैव हि ह्ये॑व । \newline
10. ए॒वोद्यन् नु॒द्यन् ने॒वैवोद्यन्न् । \newline
11. उ॒द्यन् पु॒रस्ता᳚त् पु॒रस्ता॑ दु॒द्यन् नु॒द्यन् पु॒रस्ता᳚त् । \newline
12. उ॒द्यन्नित्यु॑त् - यन्न् । \newline
13. पु॒रस्ता॒द् रक्षाꣳ॑सि॒ रक्षाꣳ॑सि पु॒रस्ता᳚त् पु॒रस्ता॒द् रक्षाꣳ॑सि । \newline
14. रक्षाꣳ॑ स्यप॒ह न्त्य॑प॒हन्ति॒ रक्षाꣳ॑सि॒ रक्षाꣳ॑ स्यप॒हन्ति॑ । \newline
15. अ॒प॒ह न्त्यू॒र्द्ध्वे ऊ॒र्द्ध्वे अ॑प॒ह न्त्य॑प॒ह न्त्यू॒र्द्ध्वे । \newline
16. अ॒प॒हन्तीत्य॑प - हन्ति॑ । \newline
17. ऊ॒र्द्ध्वे स॒मिधौ॑ स॒मिधा॑ वू॒र्द्ध्वे ऊ॒र्द्ध्वे स॒मिधौ᳚ । \newline
18. ऊ॒र्द्ध्वे इत्यू॒र्द्ध्वे । \newline
19. स॒मिधा॒ वा स॒मिधौ॑ स॒मिधा॒ वा । \newline
20. स॒मिधा॒विति॑ सं - इधौ᳚ । \newline
21. आ द॑धाति दधा॒ त्या द॑धाति । \newline
22. द॒धा॒ त्यु॒परि॑ष्टा दु॒परि॑ष्टाद् दधाति दधा त्यु॒परि॑ष्टात् । \newline
23. उ॒परि॑ष्टा दे॒वै वोपरि॑ष्टा दु॒परि॑ष्टा दे॒व । \newline
24. ए॒व रक्षाꣳ॑सि॒ रक्षाꣳ॑ स्ये॒वैव रक्षाꣳ॑सि । \newline
25. रक्षाꣳ॒॒ स्यपाप॒ रक्षाꣳ॑सि॒ रक्षाꣳ॒॒ स्यप॑ । \newline
26. अप॑ हन्ति ह॒ न्त्यपाप॑ हन्ति । \newline
27. ह॒न्ति॒ यजु॑षा॒ यजु॑षा हन्ति हन्ति॒ यजु॑षा । \newline
28. यजु॑षा॒ ऽन्या म॒न्यां ॅयजु॑षा॒ यजु॑षा॒ ऽन्याम् । \newline
29. अ॒न्याम् तू॒ष्णीम् तू॒ष्णी म॒न्या म॒न्याम् तू॒ष्णीम् । \newline
30. तू॒ष्णी म॒न्या म॒न्याम् तू॒ष्णीम् तू॒ष्णी म॒न्याम् । \newline
31. अ॒न्याम् मि॑थुन॒त्वाय॑ मिथुन॒त्वाया॒ न्या म॒न्याम् मि॑थुन॒त्वाय॑ । \newline
32. मि॒थु॒न॒त्वाय॒ द्वे द्वे मि॑थुन॒त्वाय॑ मिथुन॒त्वाय॒ द्वे । \newline
33. मि॒थु॒न॒त्वायेति॑ मिथुन - त्वाय॑ । \newline
34. द्वे आ द्वे द्वे आ । \newline
35. द्वे इति॒ द्वे । \newline
36. आ द॑धाति दधा॒ त्या द॑धाति । \newline
37. द॒धा॒ति॒ द्वि॒पाद् द्वि॒पाद् द॑धाति दधाति द्वि॒पात् । \newline
38. द्वि॒पाद् यज॑मानो॒ यज॑मानो द्वि॒पाद् द्वि॒पाद् यज॑मानः । \newline
39. द्वि॒पादिति॑ द्वि - पात् । \newline
40. यज॑मानः॒ प्रति॑ष्ठित्यै॒ प्रति॑ष्ठित्यै॒ यज॑मानो॒ यज॑मानः॒ प्रति॑ष्ठित्यै । \newline
41. प्रति॑ष्ठित्यै ब्रह्मवा॒दिनो᳚ ब्रह्मवा॒दिनः॒ प्रति॑ष्ठित्यै॒ प्रति॑ष्ठित्यै ब्रह्मवा॒दिनः॑ । \newline
42. प्रति॑ष्ठित्या॒ इति॒ प्रति॑ - स्थि॒त्यै॒ । \newline
43. ब्र॒ह्म॒वा॒दिनो॑ वदन्ति वदन्ति ब्रह्मवा॒दिनो᳚ ब्रह्मवा॒दिनो॑ वदन्ति । \newline
44. ब्र॒ह्म॒वा॒दिन॒ इति॑ ब्रह्म - वा॒दिनः॑ । \newline
45. व॒द॒न्ति॒ स स व॑दन्ति वदन्ति॒ सः । \newline
46. स तु तु स स तु । \newline
47. त्वै वै तु त्वै । \newline
48. वै य॑जेत यजेत॒ वै वै य॑जेत । \newline
49. य॒जे॒त॒ यो यो य॑जेत यजेत॒ यः । \newline
50. यो य॒ज्ञ्स्य॑ य॒ज्ञ्स्य॒ यो यो य॒ज्ञ्स्य॑ । \newline
51. य॒ज्ञ्स्या र्त्या ऽऽर्त्या॑ य॒ज्ञ्स्य॑ य॒ज्ञ्स्या र्त्या᳚ । \newline
52. आर्त्या॒ वसी॑या॒न्॒. वसी॑या॒ नार्त्या ऽऽर्त्या॒ वसी॑यान् । \newline
53. वसी॑या॒न् थ्स्याथ् स्याद् वसी॑या॒न्॒. वसी॑या॒न् थ्स्यात् । \newline
54. स्या दितीति॒ स्याथ् स्या दिति॑ । \newline
55. इति॒ भूप॑तये॒ भूप॑तय॒ इतीति॒ भूप॑तये । \newline
56. भूप॑तये॒ स्वाहा॒ स्वाहा॒ भूप॑तये॒ भूप॑तये॒ स्वाहा᳚ । \newline
57. भूप॑तय॒ इति॒ भू - प॒त॒ये॒ । \newline
58. स्वाहा॒ भुव॑नपतये॒ भुव॑नपतये॒ स्वाहा॒ स्वाहा॒ भुव॑नपतये । \newline
59. भुव॑नपतये॒ स्वाहा॒ स्वाहा॒ भुव॑नपतये॒ भुव॑नपतये॒ स्वाहा᳚ । \newline
60. भुव॑नपतय॒ इति॒ भुव॑न - प॒त॒ये॒ । \newline
61. स्वाहा॑ भू॒ताना᳚म् भू॒तानाꣳ॒॒ स्वाहा॒ स्वाहा॑ भू॒ताना᳚म् । \newline
62. भू॒ताना॒म् पत॑ये॒ पत॑ये भू॒ताना᳚म् भू॒ताना॒म् पत॑ये । \newline

\textbf{Ghana Paata } \newline

1. रक्ष॑सा॒ मन॑न्ववचारा॒या न॑न्ववचाराय॒ रक्ष॑साꣳ॒॒ रक्ष॑सा॒ मन॑न्ववचाराय॒ न नान॑न्ववचाराय॒ रक्ष॑साꣳ॒॒ रक्ष॑सा॒ मन॑न्ववचाराय॒ न । \newline
2. अन॑न्ववचाराय॒ न नान॑न्ववचारा॒या न॑न्ववचाराय॒ न पु॒रस्ता᳚त् पु॒रस्ता॒न् नान॑न्ववचारा॒या न॑न्ववचाराय॒ न पु॒रस्ता᳚त् । \newline
3. अन॑न्ववचारा॒येत्यन॑नु - अ॒व॒चा॒रा॒य॒ । \newline
4. न पु॒रस्ता᳚त् पु॒रस्ता॒न् न न पु॒रस्ता॒त् परि॒ परि॑ पु॒रस्ता॒न् न न पु॒रस्ता॒त् परि॑ । \newline
5. पु॒रस्ता॒त् परि॒ परि॑ पु॒रस्ता᳚त् पु॒रस्ता॒त् परि॑ दधाति दधाति॒ परि॑ पु॒रस्ता᳚त् पु॒रस्ता॒त् परि॑ दधाति । \newline
6. परि॑ दधाति दधाति॒ परि॒ परि॑ दधा त्यादि॒त्य आ॑दि॒त्यो द॑धाति॒ परि॒ परि॑ दधा त्यादि॒त्यः । \newline
7. द॒धा॒ त्या॒दि॒त्य आ॑दि॒त्यो द॑धाति दधा त्यादि॒त्यो हि ह्या॑दि॒त्यो द॑धाति दधा त्यादि॒त्यो हि । \newline
8. आ॒दि॒त्यो हि ह्या॑दि॒त्य आ॑दि॒त्यो ह्ये॑वैव ह्या॑दि॒त्य आ॑दि॒त्यो ह्ये॑व । \newline
9. ह्ये॑वैव हि ह्ये॑वोद्यन् नु॒द्यन् ने॒व हि ह्ये॑वोद्यन्न् । \newline
10. ए॒वोद्यन् नु॒द्यन् ने॒वैवोद्यन् पु॒रस्ता᳚त् पु॒रस्ता॑ दु॒द्यन् ने॒वैवोद्यन् पु॒रस्ता᳚त् । \newline
11. उ॒द्यन् पु॒रस्ता᳚त् पु॒रस्ता॑ दु॒द्यन् नु॒द्यन् पु॒रस्ता॒द् रक्षाꣳ॑सि॒ रक्षाꣳ॑सि पु॒रस्ता॑ दु॒द्यन् नु॒द्यन् पु॒रस्ता॒द् रक्षाꣳ॑सि । \newline
12. उ॒द्यन्नित्यु॑त् - यन्न् । \newline
13. पु॒रस्ता॒द् रक्षाꣳ॑सि॒ रक्षाꣳ॑सि पु॒रस्ता᳚त् पु॒रस्ता॒द् रक्षाꣳ॑ स्यप॒ह न्त्य॑प॒हन्ति॒ रक्षाꣳ॑सि पु॒रस्ता᳚त् पु॒रस्ता॒द् रक्षाꣳ॑ स्यप॒हन्ति॑ । \newline
14. रक्षाꣳ॑ स्यप॒ह न्त्य॑प॒हन्ति॒ रक्षाꣳ॑सि॒ रक्षाꣳ॑ स्यप॒ह न्त्यू॒र्द्ध्वे ऊ॒र्द्ध्वे अ॑प॒हन्ति॒ रक्षाꣳ॑सि॒ रक्षाꣳ॑ स्यप॒हन्त्यू॒र्द्ध्वे । \newline
15. अ॒प॒ह न्त्यू॒र्द्ध्वे ऊ॒र्द्ध्वे अ॑प॒ह न्त्य॑प॒ह न्त्यू॒र्द्ध्वे स॒मिधौ॑ स॒मिधा॑ वू॒र्द्ध्वे अ॑प॒ह न्त्य॑प॒ह न्त्यू॒र्द्ध्वे स॒मिधौ᳚ । \newline
16. अ॒प॒हन्तीत्य॑प - हन्ति॑ । \newline
17. ऊ॒र्द्ध्वे स॒मिधौ॑ स॒मिधा॑ वू॒र्द्ध्वे ऊ॒र्द्ध्वे स॒मिधा॒ वा स॒मिधा॑ वू॒र्द्ध्वे ऊ॒र्द्ध्वे स॒मिधा॒ वा । \newline
18. ऊ॒र्द्ध्वे इत्यू॒र्द्ध्वे । \newline
19. स॒मिधा॒ वा स॒मिधौ॑ स॒मिधा॒ वा द॑धाति दधा॒त्या स॒मिधौ॑ स॒मिधा॒ वा द॑धाति । \newline
20. स॒मिधा॒विति॑ सं - इधौ᳚ । \newline
21. आ द॑धाति दधा॒त्या द॑धा त्यु॒परि॑ष्टा दु॒परि॑ष्टाद् दधा॒त्या द॑धा त्यु॒परि॑ष्टात् । \newline
22. द॒धा॒ त्यु॒परि॑ष्टा दु॒परि॑ष्टाद् दधाति दधा त्यु॒परि॑ष्टा दे॒वैवो परि॑ष्टाद् दधाति दधा त्यु॒परि॑ष्टा दे॒व । \newline
23. उ॒परि॑ष्टा दे॒वैवोपरि॑ष्टा दु॒परि॑ष्टा दे॒व रक्षाꣳ॑सि॒ रक्षाꣳ॑ स्ये॒वोपरि॑ष्टा दु॒परि॑ष्टा दे॒व रक्षाꣳ॑सि । \newline
24. ए॒व रक्षाꣳ॑सि॒ रक्षाꣳ॑ स्ये॒वैव रक्षाꣳ॒॒ स्यपाप॒ रक्षाꣳ॑ स्ये॒वैव रक्षाꣳ॒॒ स्यप॑ । \newline
25. रक्षाꣳ॒॒ स्यपाप॒ रक्षाꣳ॑सि॒ रक्षाꣳ॒॒स्यप॑ हन्ति ह॒न्त्यप॒ रक्षाꣳ॑सि॒ रक्षाꣳ॒॒स्यप॑ हन्ति । \newline
26. अप॑ हन्ति ह॒न्त्यपाप॑ हन्ति॒ यजु॑षा॒ यजु॑षा ह॒न्त्यपाप॑ हन्ति॒ यजु॑षा । \newline
27. ह॒न्ति॒ यजु॑षा॒ यजु॑षा हन्ति हन्ति॒ यजु॑षा॒ ऽन्या म॒न्यां ॅयजु॑षा हन्ति हन्ति॒ यजु॑षा॒ ऽन्याम् । \newline
28. यजु॑षा॒ ऽन्या म॒न्यां ॅयजु॑षा॒ यजु॑षा॒ ऽन्याम् तू॒ष्णीम् तू॒ष्णी म॒न्यां ॅयजु॑षा॒ यजु॑षा॒ ऽन्याम् तू॒ष्णीम् । \newline
29. अ॒न्याम् तू॒ष्णीम् तू॒ष्णी म॒न्या म॒न्याम् तू॒ष्णी म॒न्या म॒न्याम् तू॒ष्णी म॒न्या म॒न्याम् तू॒ष्णी म॒न्याम् । \newline
30. तू॒ष्णी म॒न्या म॒न्याम् तू॒ष्णीम् तू॒ष्णी म॒न्याम् मि॑थुन॒त्वाय॑ मिथुन॒त्वाया॒ न्याम् तू॒ष्णीम् तू॒ष्णी म॒न्याम् मि॑थुन॒त्वाय॑ । \newline
31. अ॒न्याम् मि॑थुन॒त्वाय॑ मिथुन॒त्वाया॒न्या म॒न्याम् मि॑थुन॒त्वाय॒ द्वे द्वे मि॑थुन॒त्वाया॒न्या म॒न्याम् मि॑थुन॒त्वाय॒ द्वे । \newline
32. मि॒थु॒न॒त्वाय॒ द्वे द्वे मि॑थुन॒त्वाय॑ मिथुन॒त्वाय॒ द्वे आ द्वे मि॑थुन॒त्वाय॑ मिथुन॒त्वाय॒ द्वे आ । \newline
33. मि॒थु॒न॒त्वायेति॑ मिथुन - त्वाय॑ । \newline
34. द्वे आ द्वे द्वे आ द॑धाति दधा॒त्या द्वे द्वे आ द॑धाति । \newline
35. द्वे इति॒ द्वे । \newline
36. आ द॑धाति दधा॒त्या द॑धाति द्वि॒पाद् द्वि॒पाद् द॑धा॒त्या द॑धाति द्वि॒पात् । \newline
37. द॒धा॒ति॒ द्वि॒पाद् द्वि॒पाद् द॑धाति दधाति द्वि॒पाद् यज॑मानो॒ यज॑मानो द्वि॒पाद् द॑धाति दधाति द्वि॒पाद् यज॑मानः । \newline
38. द्वि॒पाद् यज॑मानो॒ यज॑मानो द्वि॒पाद् द्वि॒पाद् यज॑मानः॒ प्रति॑ष्ठित्यै॒ प्रति॑ष्ठित्यै॒ यज॑मानो द्वि॒पाद् द्वि॒पाद् यज॑मानः॒ प्रति॑ष्ठित्यै । \newline
39. द्वि॒पादिति॑ द्वि - पात् । \newline
40. यज॑मानः॒ प्रति॑ष्ठित्यै॒ प्रति॑ष्ठित्यै॒ यज॑मानो॒ यज॑मानः॒ प्रति॑ष्ठित्यै ब्रह्मवा॒दिनो᳚ ब्रह्मवा॒दिनः॒ प्रति॑ष्ठित्यै॒ यज॑मानो॒ यज॑मानः॒ प्रति॑ष्ठित्यै ब्रह्मवा॒दिनः॑ । \newline
41. प्रति॑ष्ठित्यै ब्रह्मवा॒दिनो᳚ ब्रह्मवा॒दिनः॒ प्रति॑ष्ठित्यै॒ प्रति॑ष्ठित्यै ब्रह्मवा॒दिनो॑ वदन्ति वदन्ति ब्रह्मवा॒दिनः॒ प्रति॑ष्ठित्यै॒ प्रति॑ष्ठित्यै ब्रह्मवा॒दिनो॑ वदन्ति । \newline
42. प्रति॑ष्ठित्या॒ इति॒ प्रति॑ - स्थि॒त्यै॒ । \newline
43. ब्र॒ह्म॒वा॒दिनो॑ वदन्ति वदन्ति ब्रह्मवा॒दिनो᳚ ब्रह्मवा॒दिनो॑ वदन्ति॒ स स व॑दन्ति ब्रह्मवा॒दिनो᳚ ब्रह्मवा॒दिनो॑ वदन्ति॒ सः । \newline
44. ब्र॒ह्म॒वा॒दिन॒ इति॑ ब्रह्म - वा॒दिनः॑ । \newline
45. व॒द॒न्ति॒ स स व॑दन्ति वदन्ति॒ स तु तु स व॑दन्ति वदन्ति॒ स तु । \newline
46. स तु तु स स त्वै वै तु स स त्वै । \newline
47. त्वै वै तु त्वै य॑जेत यजेत॒ वै तु त्वै य॑जेत । \newline
48. वै य॑जेत यजेत॒ वै वै य॑जेत॒ यो यो य॑जेत॒ वै वै य॑जेत॒ यः । \newline
49. य॒जे॒त॒ यो यो य॑जेत यजेत॒ यो य॒ज्ञ्स्य॑ य॒ज्ञ्स्य॒ यो य॑जेत यजेत॒ यो य॒ज्ञ्स्य॑ । \newline
50. यो य॒ज्ञ्स्य॑ य॒ज्ञ्स्य॒ यो यो य॒ज्ञ्स्यार्त्या ऽऽर्त्या॑ य॒ज्ञ्स्य॒ यो यो य॒ज्ञ्स्यार्त्या᳚ । \newline
51. य॒ज्ञ्स्यार्त्या ऽऽर्त्या॑ य॒ज्ञ्स्य॑ य॒ज्ञ्स्यार्त्या॒ वसी॑या॒न्॒. वसी॑या॒ नार्त्या॑ य॒ज्ञ्स्य॑ य॒ज्ञ्स्यार्त्या॒ वसी॑यान् । \newline
52. आर्त्या॒ वसी॑या॒न्॒. वसी॑या॒ नार्त्या ऽऽर्त्या॒ वसी॑या॒न् थ्स्याथ् स्याद् वसी॑या॒ नार्त्या ऽऽर्त्या॒ वसी॑या॒न् थ्स्यात् । \newline
53. वसी॑या॒न् थ्स्याथ् स्याद् वसी॑या॒न्॒. वसी॑या॒न् थ्स्या दितीति॒ स्याद् वसी॑या॒न्॒. वसी॑या॒न् थ्स्यादिति॑ । \newline
54. स्या दितीति॒ स्याथ् स्यादिति॒ भूप॑तये॒ भूप॑तय॒ इति॒ स्याथ् स्यादिति॒ भूप॑तये । \newline
55. इति॒ भूप॑तये॒ भूप॑तय॒ इतीति॒ भूप॑तये॒ स्वाहा॒ स्वाहा॒ भूप॑तय॒ इतीति॒ भूप॑तये॒ स्वाहा᳚ । \newline
56. भूप॑तये॒ स्वाहा॒ स्वाहा॒ भूप॑तये॒ भूप॑तये॒ स्वाहा॒ भुव॑नपतये॒ भुव॑नपतये॒ स्वाहा॒ भूप॑तये॒ भूप॑तये॒ स्वाहा॒ भुव॑नपतये । \newline
57. भूप॑तय॒ इति॒ भू - प॒त॒ये॒ । \newline
58. स्वाहा॒ भुव॑नपतये॒ भुव॑नपतये॒ स्वाहा॒ स्वाहा॒ भुव॑नपतये॒ स्वाहा॒ स्वाहा॒ भुव॑नपतये॒ स्वाहा॒ स्वाहा॒ भुव॑नपतये॒ स्वाहा᳚ । \newline
59. भुव॑नपतये॒ स्वाहा॒ स्वाहा॒ भुव॑नपतये॒ भुव॑नपतये॒ स्वाहा॑ भू॒ताना᳚म् भू॒तानाꣳ॒॒ स्वाहा॒ भुव॑नपतये॒ भुव॑नपतये॒ स्वाहा॑ भू॒ताना᳚म् । \newline
60. भुव॑नपतय॒ इति॒ भुव॑न - प॒त॒ये॒ । \newline
61. स्वाहा॑ भू॒ताना᳚म् भू॒तानाꣳ॒॒ स्वाहा॒ स्वाहा॑ भू॒ताना॒म् पत॑ये॒ पत॑ये भू॒तानाꣳ॒॒ स्वाहा॒ स्वाहा॑ भू॒ताना॒म् पत॑ये । \newline
62. भू॒ताना॒म् पत॑ये॒ पत॑ये भू॒ताना᳚म् भू॒ताना॒म् पत॑ये॒ स्वाहा॒ स्वाहा॒ पत॑ये भू॒ताना᳚म् भू॒ताना॒म् पत॑ये॒ स्वाहा᳚ । \newline
\pagebreak
\markright{ TS 2.6.6.4  \hfill https://www.vedavms.in \hfill}
\addcontentsline{toc}{section}{ TS 2.6.6.4 }
\section*{ TS 2.6.6.4 }

\textbf{TS 2.6.6.4 } \newline
\textbf{Samhita Paata} \newline

पत॑ये॒ स्वाहेति॑ स्क॒न्नमनु॑ मन्त्रयेत य॒ज्ञ्स्यै॒व तदार्त्या॒ यज॑मानो॒ वसी॑यान् भवति॒ भूय॑सी॒र्॒.हि दे॒वताः᳚ प्री॒णाति॑ जा॒मि वा ए॒तद्-य॒ज्ञ्स्य॑ क्रियते॒ यद॒न्वञ्चौ॑ पुरो॒डाशा॑ वुपाꣳशुया॒जम॑न्त॒रा य॑ज॒त्यजा॑मित्वा॒याथो॑ मिथुन॒त्वाया॒ग्निर॒मुष्मि॑न् ॅलो॒क आसी᳚द्-य॒मो᳚ऽस्मिन् ते दे॒वा अ॑ब्रुव॒न्नेते॒मौ वि पर्यू॑हा॒मेत्य॒न्नाद्ये॑न दे॒वा अ॒ग्नि - [  ] \newline

\textbf{Pada Paata} \newline

पत॑ये । स्वाहा᳚ । इति॑ । स्क॒न्नम् । अन्विति॑ ।   म॒न्त्र॒ये॒त॒ । य॒ज्ञ्स्य॑ । ए॒व । तत् । आर्त्या᳚ । यज॑मानः । वसी॑यान् । भ॒व॒ति॒ । भूय॑सीः । हि । दे॒वताः᳚ । प्री॒णाति॑ । जा॒मि । वै । ए॒तत् । य॒ज्ञ्स्य॑ । क्रि॒य॒ते॒ । यत् । अ॒न्वञ्चौ᳚ । पु॒रो॒डाशौ᳚ । उ॒पाꣳ॒॒शु॒या॒जमित्यु॑पाꣳशु-या॒जम् । अ॒न्त॒रा । य॒ज॒ति॒ । अजा॑मित्वा॒येत्यजा॑मि - त्वा॒य॒ । अथो॒ इति॑ । मि॒थु॒न॒त्वायेति॑ मिथुन - त्वाय॑ । अ॒ग्निः । अ॒मुष्मिन्न्॑ । लो॒के । आसी᳚त् । य॒मः । अ॒स्मिन्न् । ते । दे॒वाः । अ॒ब्रु॒व॒न्न् । एति॑ । इ॒त॒ । इ॒मौ । वि । परीति॑ । ऊ॒हा॒म॒ । इति॑ । अ॒न्नाद्ये॒नेत्य॑न्न - अद्ये॑न ।   दे॒वाः । अ॒ग्निम् ।  \newline


\textbf{Krama Paata} \newline

पत॑ये॒ स्वाहा᳚ । स्वाहेति॑ । इति॑ स्क॒न्नम् । स्क॒न्नमनु॑ । अनु॑ मन्त्रयेत । म॒न्त्र॒ये॒त॒ य॒ज्ञ्स्य॑ । य॒ज्ञ्स्यै॒व । ए॒व तत् । तदार्त्या᳚ । आर्त्या॒ यज॑मानः । यज॑मानो॒ वसी॑यान् । वसी॑यान् भवति । भ॒व॒ति॒ भूय॑सीः । भूय॑सी॒र्.॒ हि । हि दे॒वताः᳚ । दे॒वताः᳚ प्री॒णाति॑ । प्री॒णाति॑ जा॒मि । जा॒मि वै । वा ए॒तत् । ए॒तद् य॒ज्ञ्स्य॑ । य॒ज्ञ्स्य॑ क्रियते । क्रि॒य॒ते॒ यत् । यद॒न्वञ्चौ᳚ । अ॒न्वञ्चौ॑ पुरो॒डाशौ᳚ । पु॒रो॒डाशा॑वुपाꣳशुया॒जम् । उ॒पाꣳ॒॒शु॒या॒जम॑न्त॒रा । उ॒पाꣳ॒॒शु॒या॒जमित्यु॑पाꣳशु - या॒जम् । अ॒न्त॒रा य॑जति । य॒ज॒त्यजा॑मित्वाय । अजा॑मित्वा॒याथो᳚ । अजा॑मित्वा॒येत्यजा॑मि - त्वा॒य॒ । अथो॑ मिथुन॒त्वाय॑ । अथो॒ इत्यथो᳚ । मि॒थु॒न॒त्वाया॒ग्निः । मि॒थु॒न॒त्वायेति॑ मिथुन - त्वाय॑ । अ॒ग्निर॒मुष्मिन्न्॑ । अ॒मुष्मि॑न् ॅलो॒के । लो॒क आसी᳚त् । आसी᳚द् य॒मः । य॒मो᳚ऽस्मिन्न् । अ॒स्मिन् ते । ते दे॒वाः । दे॒वा अ॑ब्रुवन्न् । अ॒ब्रु॒व॒न्ना । एत॑ । इ॒ते॒मौ । इ॒मौ वि । वि परि॑ । पर्यू॑हाम । ऊ॒हा॒मेति॑ । इत्य॒न्नाद्ये॑न । अ॒न्नाद्ये॑न दे॒वाः । अ॒न्नाद्ये॒नेत्य॑न्न - अद्ये॑न । दे॒वा अ॒ग्निम् । अ॒ग्निमु॒पाम॑न्त्रयन्त \newline

\textbf{Jatai Paata} \newline

1. पत॑ये॒ स्वाहा॒ स्वाहा॒ पत॑ये॒ पत॑ये॒ स्वाहा᳚ । \newline
2. स्वाहेतीति॒ स्वाहा॒ स्वाहेति॑ । \newline
3. इति॑ स्क॒न्नꣳ स्क॒न्न मितीति॑ स्क॒न्नम् । \newline
4. स्क॒न्न मन्वनु॑ स्क॒न्नꣳ स्क॒न्न मनु॑ । \newline
5. अनु॑ मन्त्रयेत मन्त्रये॒ता न्वनु॑ मन्त्रयेत । \newline
6. म॒न्त्र॒ये॒त॒ य॒ज्ञ्स्य॑ य॒ज्ञ्स्य॑ मन्त्रयेत मन्त्रयेत य॒ज्ञ्स्य॑ । \newline
7. य॒ज्ञ् स्यै॒वैव य॒ज्ञ्स्य॑ य॒ज्ञ् स्यै॒व । \newline
8. ए॒व तत् तदे॒वैव तत् । \newline
9. तदा र्त्या ऽऽर्त्या॒ तत् तदा र्त्या᳚ । \newline
10. आर्त्या॒ यज॑मानो॒ यज॑मान॒ आर्त्या ऽऽर्त्या॒ यज॑मानः । \newline
11. यज॑मानो॒ वसी॑या॒न्॒. वसी॑या॒न्॒. यज॑मानो॒ यज॑मानो॒ वसी॑यान् । \newline
12. वसी॑यान् भवति भवति॒ वसी॑या॒न्॒. वसी॑यान् भवति । \newline
13. भ॒व॒ति॒ भूय॑सी॒र् भूय॑सीर् भवति भवति॒ भूय॑सीः । \newline
14. भूय॑सी॒र्॒. हि हि भूय॑सी॒र् भूय॑सी॒र्॒. हि । \newline
15. हि दे॒वता॑ दे॒वता॒ हि हि दे॒वताः᳚ । \newline
16. दे॒वताः᳚ प्री॒णाति॑ प्री॒णाति॑ दे॒वता॑ दे॒वताः᳚ प्री॒णाति॑ । \newline
17. प्री॒णाति॑ जा॒मि जा॒मि प्री॒णाति॑ प्री॒णाति॑ जा॒मि । \newline
18. जा॒मि वै वै जा॒मि जा॒मि वै । \newline
19. वा ए॒त दे॒तद् वै वा ए॒तत् । \newline
20. ए॒तद् य॒ज्ञ्स्य॑ य॒ज्ञ् स्यै॒त दे॒तद् य॒ज्ञ्स्य॑ । \newline
21. य॒ज्ञ्स्य॑ क्रियते क्रियते य॒ज्ञ्स्य॑ य॒ज्ञ्स्य॑ क्रियते । \newline
22. क्रि॒य॒ते॒ यद् यत् क्रि॑यते क्रियते॒ यत् । \newline
23. यद॒न्वञ्चा॑ व॒न्वञ्चौ॒ यद् यद॒न्वञ्चौ᳚ । \newline
24. अ॒न्वञ्चौ॑ पुरो॒डाशौ॑ पुरो॒डाशा॑ व॒न्वञ्चा॑ व॒न्वञ्चौ॑ पुरो॒डाशौ᳚ । \newline
25. पु॒रो॒डाशा॑ वुपाꣳशुया॒ज मु॑पाꣳशुया॒जम् पु॑रो॒डाशौ॑ पुरो॒डाशा॑ वुपाꣳशुया॒जम् । \newline
26. उ॒पाꣳ॒॒शु॒या॒ज म॑न्त॒रा ऽन्त॒रोपाꣳ॑शुया॒ज मु॑पाꣳशुया॒ज म॑न्त॒रा । \newline
27. उ॒पाꣳ॒॒शु॒या॒जमित्यु॑पाꣳशु - या॒जम् । \newline
28. अ॒न्त॒रा य॑जति यज त्यन्त॒रा ऽन्त॒रा य॑जति । \newline
29. य॒ज॒ त्यजा॑मित्वा॒या जा॑मित्वाय यजति यज॒ त्यजा॑मित्वाय । \newline
30. अजा॑मित्वा॒या थो॒ अथो॒ अजा॑मित्वा॒या जा॑मित्वा॒या थो᳚ । \newline
31. अजा॑मित्वा॒येत्यजा॑मि - त्वा॒य॒ । \newline
32. अथो॑ मिथुन॒त्वाय॑ मिथुन॒त्वाया थो॒ अथो॑ मिथुन॒त्वाय॑ । \newline
33. अथो॒ इत्यथो᳚ । \newline
34. मि॒थु॒न॒त्वाया॒ ग्नि र॒ग्निर् मि॑थुन॒त्वाय॑ मिथुन॒त्वाया॒ ग्निः । \newline
35. मि॒थु॒न॒त्वायेति॑ मिथुन - त्वाय॑ । \newline
36. अ॒ग्नि र॒मुष्मि॑न् न॒मुष्मि॑न् न॒ग्नि र॒ग्नि र॒मुष्मिन्न्॑ । \newline
37. अ॒मुष्मि॑न् ॅलो॒के लो॒के॑ ऽमुष्मि॑न् न॒मुष्मि॑न् ॅलो॒के । \newline
38. लो॒क आसी॒ दासी᳚ ल्लो॒के लो॒क आसी᳚त् । \newline
39. आसी᳚द् य॒मो य॒म आसी॒ दासी᳚द् य॒मः । \newline
40. य॒मो᳚ ऽस्मिन् न॒स्मिन्. य॒मो य॒मो᳚ ऽस्मिन्न् । \newline
41. अ॒स्मिन् ते ते᳚ ऽस्मिन् न॒स्मिन् ते । \newline
42. ते दे॒वा दे॒वा स्ते ते दे॒वाः । \newline
43. दे॒वा अ॑ब्रुवन् नब्रुवन् दे॒वा दे॒वा अ॑ब्रुवन्न् । \newline
44. अ॒ब्रु॒व॒न् ना ऽब्रु॑वन् नब्रुव॒न् ना । \newline
45. एते॒ तेत॑ । \newline
46. इ॒ते॒ मा वि॒मा वि॑ते ते॒ मौ । \newline
47. इ॒मौ वि वीमा वि॒मौ वि । \newline
48. वि परि॒ परि॒ वि वि परि॑ । \newline
49. पर्यू॑हा मोहाम॒ परि॒ पर्यू॑हाम । \newline
50. ऊ॒हा॒मे ती त्यू॑हा मोहा॒मे ति॑ । \newline
51. इत्य॒न्नाद्ये॑ना॒ न्नाद्ये॒ने ती त्य॒न्नाद्ये॑न । \newline
52. अ॒न्नाद्ये॑न दे॒वा दे॒वा अ॒न्नाद्ये॑ना॒ न्नाद्ये॑न दे॒वाः । \newline
53. अ॒न्नाद्ये॒नेत्य॑न्न - अद्ये॑न । \newline
54. दे॒वा अ॒ग्नि म॒ग्निम् दे॒वा दे॒वा अ॒ग्निम् । \newline
55. अ॒ग्नि मु॒पाम॑न्त्रय न्तो॒पाम॑न्त्रयन्ता॒ ग्नि म॒ग्नि मु॒पाम॑न्त्रयन्त । \newline

\textbf{Ghana Paata } \newline

1. पत॑ये॒ स्वाहा॒ स्वाहा॒ पत॑ये॒ पत॑ये॒ स्वाहेतीति॒ स्वाहा॒ पत॑ये॒ पत॑ये॒ स्वाहेति॑ । \newline
2. स्वाहेतीति॒ स्वाहा॒ स्वाहेति॑ स्क॒न्नꣳ स्क॒न्न मिति॒ स्वाहा॒ स्वाहेति॑ स्क॒न्नम् । \newline
3. इति॑ स्क॒न्नꣳ स्क॒न्न मितीति॑ स्क॒न्न मन्वनु॑ स्क॒न्न मितीति॑ स्क॒न्न मनु॑ । \newline
4. स्क॒न्न मन्वनु॑ स्क॒न्नꣳ स्क॒न्न मनु॑ मन्त्रयेत मन्त्रये॒तानु॑ स्क॒न्नꣳ स्क॒न्न मनु॑ मन्त्रयेत । \newline
5. अनु॑ मन्त्रयेत मन्त्रये॒तान्वनु॑ मन्त्रयेत य॒ज्ञ्स्य॑ य॒ज्ञ्स्य॑ मन्त्रये॒तान्वनु॑ मन्त्रयेत य॒ज्ञ्स्य॑ । \newline
6. म॒न्त्र॒ये॒त॒ य॒ज्ञ्स्य॑ य॒ज्ञ्स्य॑ मन्त्रयेत मन्त्रयेत य॒ज्ञ्स्यै॒वैव य॒ज्ञ्स्य॑ मन्त्रयेत मन्त्रयेत य॒ज्ञ्स्यै॒व । \newline
7. य॒ज्ञ्स्यै॒वैव य॒ज्ञ्स्य॑ य॒ज्ञ्स्यै॒व तत् तदे॒व य॒ज्ञ्स्य॑ य॒ज्ञ्स्यै॒व तत् । \newline
8. ए॒व तत् तदे॒वैव तदार्त्या ऽऽर्त्या॒ तदे॒वैव तदार्त्या᳚ । \newline
9. तदार्त्या ऽऽर्त्या॒ तत् तदार्त्या॒ यज॑मानो॒ यज॑मान॒ आर्त्या॒ तत् तदार्त्या॒ यज॑मानः । \newline
10. आर्त्या॒ यज॑मानो॒ यज॑मान॒ आर्त्या ऽऽर्त्या॒ यज॑मानो॒ वसी॑या॒न्॒. वसी॑या॒न्॒. यज॑मान॒ आर्त्या ऽऽर्त्या॒ यज॑मानो॒ वसी॑यान् । \newline
11. यज॑मानो॒ वसी॑या॒न्॒. वसी॑या॒न्॒. यज॑मानो॒ यज॑मानो॒ वसी॑यान् भवति भवति॒ वसी॑या॒न्॒. यज॑मानो॒ यज॑मानो॒ वसी॑यान् भवति । \newline
12. वसी॑यान् भवति भवति॒ वसी॑या॒न्॒. वसी॑यान् भवति॒ भूय॑सी॒र् भूय॑सीर् भवति॒ वसी॑या॒न्॒. वसी॑यान् भवति॒ भूय॑सीः । \newline
13. भ॒व॒ति॒ भूय॑सी॒र् भूय॑सीर् भवति भवति॒ भूय॑सी॒र्॒. हि हि भूय॑सीर् भवति भवति॒ भूय॑सी॒र्॒. हि । \newline
14. भूय॑सी॒र्॒. हि हि भूय॑सी॒र् भूय॑सी॒र्॒. हि दे॒वता॑ दे॒वता॒ हि भूय॑सी॒र् भूय॑सी॒र्॒. हि दे॒वताः᳚ । \newline
15. हि दे॒वता॑ दे॒वता॒ हि हि दे॒वताः᳚ प्री॒णाति॑ प्री॒णाति॑ दे॒वता॒ हि हि दे॒वताः᳚ प्री॒णाति॑ । \newline
16. दे॒वताः᳚ प्री॒णाति॑ प्री॒णाति॑ दे॒वता॑ दे॒वताः᳚ प्री॒णाति॑ जा॒मि जा॒मि प्री॒णाति॑ दे॒वता॑ दे॒वताः᳚ प्री॒णाति॑ जा॒मि । \newline
17. प्री॒णाति॑ जा॒मि जा॒मि प्री॒णाति॑ प्री॒णाति॑ जा॒मि वै वै जा॒मि प्री॒णाति॑ प्री॒णाति॑ जा॒मि वै । \newline
18. जा॒मि वै वै जा॒मि जा॒मि वा ए॒त दे॒तद् वै जा॒मि जा॒मि वा ए॒तत् । \newline
19. वा ए॒त दे॒तद् वै वा ए॒तद् य॒ज्ञ्स्य॑ य॒ज्ञ्स्यै॒तद् वै वा ए॒तद् य॒ज्ञ्स्य॑ । \newline
20. ए॒तद् य॒ज्ञ्स्य॑ य॒ज्ञ्स्यै॒त दे॒तद् य॒ज्ञ्स्य॑ क्रियते क्रियते य॒ज्ञ्स्यै॒त दे॒तद् य॒ज्ञ्स्य॑ क्रियते । \newline
21. य॒ज्ञ्स्य॑ क्रियते क्रियते य॒ज्ञ्स्य॑ य॒ज्ञ्स्य॑ क्रियते॒ यद् यत् क्रि॑यते य॒ज्ञ्स्य॑ य॒ज्ञ्स्य॑ क्रियते॒ यत् । \newline
22. क्रि॒य॒ते॒ यद् यत् क्रि॑यते क्रियते॒ यद॒न्वञ्चा॑ व॒न्वञ्चौ॒ यत् क्रि॑यते क्रियते॒ यद॒न्वञ्चौ᳚ । \newline
23. यद॒न्वञ्चा॑ व॒न्वञ्चौ॒ यद् यद॒न्वञ्चौ॑ पुरो॒डाशौ॑ पुरो॒डाशा॑ व॒न्वञ्चौ॒ यद् यद॒न्वञ्चौ॑ पुरो॒डाशौ᳚ । \newline
24. अ॒न्वञ्चौ॑ पुरो॒डाशौ॑ पुरो॒डाशा॑ व॒न्वञ्चा॑ व॒न्वञ्चौ॑ पुरो॒डाशा॑ वुपाꣳशुया॒ज मु॑पाꣳशुया॒जम् पु॑रो॒डाशा॑ व॒न्वञ्चा॑ व॒न्वञ्चौ॑ पुरो॒डाशा॑ वुपाꣳशुया॒जम् । \newline
25. पु॒रो॒डाशा॑ वुपाꣳशुया॒ज मु॑पाꣳशुया॒जम् पु॑रो॒डाशौ॑ पुरो॒डाशा॑ वुपाꣳशुया॒ज म॑न्त॒रा ऽन्त॒रोपाꣳ॑शुया॒जम् पु॑रो॒डाशौ॑ पुरो॒डाशा॑ वुपाꣳशुया॒ज म॑न्त॒रा । \newline
26. उ॒पाꣳ॒॒शु॒या॒ज म॑न्त॒रा ऽन्त॒रोपाꣳ॑शुया॒ज मु॑पाꣳशुया॒ज म॑न्त॒रा य॑जति यज त्यन्त॒रोपाꣳ॑शुया॒ज मु॑पाꣳशुया॒ज म॑न्त॒रा य॑जति । \newline
27. उ॒पाꣳ॒॒शु॒या॒जमित्यु॑पाꣳशु - या॒जम् । \newline
28. अ॒न्त॒रा य॑जति यजत्यन्त॒रा ऽन्त॒रा य॑ज॒ त्यजा॑मित्वा॒या जा॑मित्वाय यजत्यन्त॒रा ऽन्त॒रा य॑ज॒ त्यजा॑मित्वाय । \newline
29. य॒ज॒ त्यजा॑मित्वा॒या जा॑मित्वाय यजति यज॒ त्यजा॑मित्वा॒याथो॒ अथो॒ अजा॑मित्वाय यजति यज॒ त्यजा॑मित्वा॒याथो᳚ । \newline
30. अजा॑मित्वा॒याथो॒ अथो॒ अजा॑मित्वा॒या जा॑मित्वा॒याथो॑ मिथुन॒त्वाय॑ मिथुन॒त्वायाथो॒ अजा॑मित्वा॒या जा॑मित्वा॒याथो॑ मिथुन॒त्वाय॑ । \newline
31. अजा॑मित्वा॒येत्यजा॑मि - त्वा॒य॒ । \newline
32. अथो॑ मिथुन॒त्वाय॑ मिथुन॒त्वायाथो॒ अथो॑ मिथुन॒त्वाया॒ग्नि र॒ग्निर् मि॑थुन॒त्वायाथो॒ अथो॑ मिथुन॒त्वाया॒ग्निः । \newline
33. अथो॒ इत्यथो᳚ । \newline
34. मि॒थु॒न॒त्वाया॒ ग्नि र॒ग्निर् मि॑थुन॒त्वाय॑ मिथुन॒त्वाया॒ ग्नि र॒मुष्मि॑न् न॒मुष्मि॑न् न॒ग्निर् मि॑थुन॒त्वाय॑ मिथुन॒त्वाया॒ ग्नि र॒मुष्मिन्न्॑ । \newline
35. मि॒थु॒न॒त्वायेति॑ मिथुन - त्वाय॑ । \newline
36. अ॒ग्नि र॒मुष्मि॑न् न॒मुष्मि॑न् न॒ग्नि र॒ग्नि र॒मुष्मि॑न् ॅलो॒के लो॒के॑ ऽमुष्मि॑न् न॒ग्नि र॒ग्नि र॒मुष्मि॑न् ॅलो॒के । \newline
37. अ॒मुष्मि॑न् ॅलो॒के लो॒के॑ ऽमुष्मि॑न् न॒मुष्मि॑न् ॅलो॒क आसी॒ दासी᳚ ल्लो॒के॑ ऽमुष्मि॑न् न॒मुष्मि॑न् ॅलो॒क आसी᳚त् । \newline
38. लो॒क आसी॒ दासी᳚ ल्लो॒के लो॒क आसी᳚द् य॒मो य॒म आसी᳚ ल्लो॒के लो॒क आसी᳚द् य॒मः । \newline
39. आसी᳚द् य॒मो य॒म आसी॒ दासी᳚द् य॒मो᳚ ऽस्मिन् न॒स्मिन्. य॒म आसी॒ दासी᳚द् य॒मो᳚ ऽस्मिन्न् । \newline
40. य॒मो᳚ ऽस्मिन् न॒स्मिन्. य॒मो य॒मो᳚ ऽस्मिन् ते ते᳚ ऽस्मिन्. य॒मो य॒मो᳚ ऽस्मिन् ते । \newline
41. अ॒स्मिन् ते ते᳚ ऽस्मिन् न॒स्मिन् ते दे॒वा दे॒वा स्ते᳚ ऽस्मिन् न॒स्मिन् ते दे॒वाः । \newline
42. ते दे॒वा दे॒वा स्ते ते दे॒वा अ॑ब्रुवन् नब्रुवन् दे॒वा स्ते ते दे॒वा अ॑ब्रुवन्न् । \newline
43. दे॒वा अ॑ब्रुवन् नब्रुवन् दे॒वा दे॒वा अ॑ब्रुव॒न् ना ऽब्रु॑वन् दे॒वा दे॒वा अ॑ब्रुव॒न् ना । \newline
44. अ॒ब्रु॒व॒न् ना ऽब्रु॑वन् नब्रुव॒न् नेते॒ ता ऽब्रु॑वन् नब्रुव॒न् नेत॑ । \newline
45. एते॒ तेते॒ मा वि॒मा वि॒तेते॒ मौ । \newline
46. इ॒ते॒ मा वि॒मा वि॑ते ते॒ मौ वि वीमा वि॑ते ते॒ मौ वि । \newline
47. इ॒मौ वि वीमा वि॒मौ वि परि॒ परि॒ वीमा वि॒मौ वि परि॑ । \newline
48. वि परि॒ परि॒ वि वि पर्यू॑हामो हाम॒ परि॒ वि वि पर्यू॑हाम । \newline
49. पर्यू॑हामो हाम॒ परि॒ पर्यू॑हा॒मे तीत्यू॑हाम॒ परि॒ पर्यू॑हा॒मे ति॑ । \newline
50. ऊ॒हा॒मे तीत्यू॑हामो हा॒मे त्य॒न्नाद्ये॑ना॒ न्नाद्ये॒ने त्यू॑हामो हा॒मे त्य॒न्नाद्ये॑न । \newline
51. इत्य॒ न्नाद्ये॑ना॒ न्नाद्ये॒ने तीत्य॒न्नाद्ये॑न दे॒वा दे॒वा अ॒न्नाद्ये॒ने तीत्य॒न्नाद्ये॑न दे॒वाः । \newline
52. अ॒न्नाद्ये॑न दे॒वा दे॒वा अ॒न्नाद्ये॑ना॒ न्नाद्ये॑न दे॒वा अ॒ग्नि म॒ग्निम् दे॒वा अ॒न्नाद्ये॑ना॒ न्नाद्ये॑न दे॒वा अ॒ग्निम् । \newline
53. अ॒न्नाद्ये॒नेत्य॑न्न - अद्ये॑न । \newline
54. दे॒वा अ॒ग्नि म॒ग्निम् दे॒वा दे॒वा अ॒ग्नि मु॒पाम॑न्त्रयन् तो॒पाम॑न्त्रयन्ता॒ ग्निम् दे॒वा दे॒वा अ॒ग्नि मु॒पाम॑न्त्रयन्त । \newline
55. अ॒ग्नि मु॒पा म॑न्त्रय न्तो॒पाम॑न्त्रयन्ता॒ग्नि म॒ग्नि मु॒पाम॑न्त्रयन्त रा॒ज्येन॑ रा॒ज्येनो॒पा म॑न्त्रयन्ता॒ ग्नि म॒ग्नि मु॒पाम॑न्त्रयन्त रा॒ज्येन॑ । \newline
\pagebreak
\markright{ TS 2.6.6.5  \hfill https://www.vedavms.in \hfill}
\addcontentsline{toc}{section}{ TS 2.6.6.5 }
\section*{ TS 2.6.6.5 }

\textbf{TS 2.6.6.5 } \newline
\textbf{Samhita Paata} \newline

-मु॒पाम॑न्त्रयन्त रा॒ज्येन॑ पि॒तरो॑ य॒मं तस्मा॑द॒ग्नि र्दे॒वाना॑मन्ना॒दो य॒मः पि॑तृ॒णाꣳ राजा॒ य ए॒वं ॅवेद॒ प्ररा॒ज्यम॒न्नाद्य॑-माप्नोति॒ तस्मा॑ ए॒तद्-भा॑ग॒धेयं॒ प्राय॑च्छ॒न॒. यद॒ग्नये᳚ स्विष्ट॒कृते॑ऽव॒द्यन्ति॒ यद॒ग्नये᳚ स्विष्ट॒कृते॑ ऽव॒द्यति॑ भाग॒धेये॑नै॒व तद्-रु॒द्रꣳ सम॑र्द्धयति स॒कृथ् स॑कृ॒दव॑ द्यति स॒कृदि॑व॒ हि रु॒द्र उ॑त्तरा॒र्द्धादव॑ द्यत्ये॒षा वै रु॒द्रस्य॒ - [  ] \newline

\textbf{Pada Paata} \newline

उ॒पाम॑न्त्रय॒न्तेत्यु॑प-अम॑न्त्रयन्त । रा॒ज्येन॑ । पि॒तरः॑ । य॒मम् । तस्मा᳚त् । अ॒ग्निः । दे॒वाना᳚म् । अ॒न्ना॒द इत्य॑न्न - अ॒दः । य॒मः । पि॒तृ॒णाम् । राजा᳚ । यः । ए॒वम् । वेद॑ । प्रेति॑ । रा॒ज्यम् । अ॒न्नाद्य॒मित्य॑न्न-अद्य᳚म् । आ॒प्नो॒ति॒ । तस्मै᳚ । ए॒तत् । भा॒ग॒धेय॒मिति॑ भाग - धेय᳚म् । प्रेति॑ । अ॒य॒च्छ॒न्न् । यत् । अ॒ग्नये᳚ । स्वि॒ष्ट॒कृत॒ इति॑ स्विष्ट - कृते᳚ । अ॒व॒द्यन्तीत्य॑व - द्यन्ति॑ । यत् । अ॒ग्नये᳚ । स्वि॒ष्ट॒कृत॒ इति॑ स्विष्ट - कृते᳚ । अ॒व॒द्यतीत्य॑व - द्यति॑ । भा॒ग॒धेये॒नेति॑ भाग-धेये॑न । ए॒व । तत् । रु॒द्रम् । समिति॑ । अ॒र्द्ध॒य॒ति॒ । स॒कृथ्स॑कृ॒दिति॑ स॒कृत् - स॒कृ॒त् । अवेति॑ । द्य॒ति॒ । स॒कृत् । इ॒व॒ । हि । रु॒द्रः । उ॒त्त॒रा॒र्द्धादित्यु॑त्तर - अ॒र्द्धात् । अवेति॑ । द्य॒ति॒ । ए॒षा । वै । रु॒द्रस्य॑ ।  \newline


\textbf{Krama Paata} \newline

उ॒पाम॑न्त्रयन्त रा॒ज्येन॑ । उ॒पाम॑न्त्रय॒न्तेत्यु॑प - अम॑न्त्रयन्त । रा॒ज्येन॑ पि॒तरः॑ । पि॒तरो॑ य॒मम् । य॒मम् तस्मा᳚त् । तस्मा॑द॒ग्निः । अ॒ग्निर् दे॒वाना᳚म् । दे॒वाना॑मन्ना॒दः । अ॒न्ना॒दो य॒मः । अ॒न्ना॒द इत्य॑न्न - अ॒दः । य॒मः पि॑तृ॒णाम् । पि॒तृ॒णाꣳ राजा᳚ । राजा॒ यः । य ए॒वम् । ए॒वं ॅवेद॑ । वेद॒ प्र । प्र रा॒ज्यम् । रा॒ज्यम॒न्नाद्य᳚म् । अ॒न्नाद्य॑माप्नोति । अ॒न्नाद्य॒मित्य॑न्न - अद्य᳚म् । आ॒प्नो॒ति॒ तस्मै᳚ । तस्मा॑ ए॒तत् । ए॒तद् भा॑ग॒धेय᳚म् । भा॒ग॒धेय॒म् प्र । भा॒ग॒धेय॒मिति॑ भाग - धेय᳚म् । प्राय॑च्छन्न् । अ॒य॒च्छ॒न्न्॒. यत् । यद॒ग्नये᳚ । अ॒ग्नये᳚ स्विष्ट॒कृते᳚ । स्वि॒ष्ट॒कृते॑ऽव॒द्यन्ति॑ । स्वि॒ष्ट॒कृत॒ इति॑ स्विष्ट - कृते᳚ । अ॒व॒द्यन्ति॒ यत् । अ॒व॒द्यन्तीत्य॑व - द्यन्ति॑ । यद॒ग्नये᳚ । अ॒ग्नये᳚ स्विष्ट॒कृते᳚ । स्वि॒ष्ट॒कृते॑ऽव॒द्यति॑ । स्वि॒ष्ट॒कृत॒ इति॑ स्विष्ट - कृते᳚ । अ॒व॒द्यति॑ भाग॒धेये॑न । अ॒व॒द्यतीत्य॑व - द्यति॑ । भा॒ग॒धेये॑नै॒व । भा॒ग॒धेये॒नेति॑ भाग - धेये॑न । ए॒व तत् । तद् रु॒द्रम् । रु॒द्रꣳ सम् । सम॑र्द्धयति । अ॒र्द्ध॒य॒ति॒ स॒कृथ्स॑कृत् । स॒कृथ्स॑कृ॒दव॑ । स॒कृथ्स॑कृ॒दिति॑ स॒कृत् - स॒कृ॒त्॒ । अव॑ द्यति । द्य॒ति॒ स॒कृत् । स॒कृदि॑व । इ॒व॒ हि । हि रु॒द्रः । रु॒द्र उ॑त्तरा॒र्द्धात् । उ॒त्त॒रा॒र्द्धादव॑ । उ॒त्त॒रा॒र्द्धादित्यु॑त्तर - अ॒र्द्धात् । अव॑ द्यति । द्य॒त्ये॒षा । ए॒षा वै । वै रु॒द्रस्य॑ ( ) । रु॒दस्य॒ दिक् \newline

\textbf{Jatai Paata} \newline

1. उ॒पाम॑न्त्रयन्त रा॒ज्येन॑ रा॒ज्ये नो॒पाम॑न्त्रय न्तो॒पाम॑न्त्रयन्त रा॒ज्येन॑ । \newline
2. उ॒पाम॑न्त्रय॒न्तेत्यु॑प - अम॑न्त्रयन्त । \newline
3. रा॒ज्येन॑ पि॒तरः॑ पि॒तरो॑ रा॒ज्येन॑ रा॒ज्येन॑ पि॒तरः॑ । \newline
4. पि॒तरो॑ य॒मं ॅय॒मम् पि॒तरः॑ पि॒तरो॑ य॒मम् । \newline
5. य॒मम् तस्मा॒त् तस्मा᳚द् य॒मं ॅय॒मम् तस्मा᳚त् । \newline
6. तस्मा॑ द॒ग्नि र॒ग्नि स्तस्मा॒त् तस्मा॑ द॒ग्निः । \newline
7. अ॒ग्निर् दे॒वाना᳚म् दे॒वाना॑ म॒ग्नि र॒ग्निर् दे॒वाना᳚म् । \newline
8. दे॒वाना॑ मन्ना॒दो᳚ ऽन्ना॒दो दे॒वाना᳚म् दे॒वाना॑ मन्ना॒दः । \newline
9. अ॒न्ना॒दो य॒मो य॒मो᳚ ऽन्ना॒दो᳚ ऽन्ना॒दो य॒मः । \newline
10. अ॒न्ना॒द इत्य॑न्न - अ॒दः । \newline
11. य॒मः पि॑तृ॒णाम् पि॑तृ॒णां ॅय॒मो य॒मः पि॑तृ॒णाम् । \newline
12. पि॒तृ॒णाꣳ राजा॒ राजा॑ पितृ॒णाम् पि॑तृ॒णाꣳ राजा᳚ । \newline
13. राजा॒ यो यो राजा॒ राजा॒ यः । \newline
14. य ए॒व मे॒वं ॅयो य ए॒वम् । \newline
15. ए॒वं ॅवेद॒ वेदै॒व मे॒वं ॅवेद॑ । \newline
16. वेद॒ प्र प्र वेद॒ वेद॒ प्र । \newline
17. प्र रा॒ज्यꣳ रा॒ज्यम् प्र प्र रा॒ज्यम् । \newline
18. रा॒ज्य म॒न्नाद्य॑ म॒न्नाद्यꣳ॑ रा॒ज्यꣳ रा॒ज्य म॒न्नाद्य᳚म् । \newline
19. अ॒न्नाद्य॑ माप्नो त्याप्नो त्य॒न्नाद्य॑ म॒न्नाद्य॑ माप्नोति । \newline
20. अ॒न्नाद्य॒मित्य॑न्न - अद्य᳚म् । \newline
21. आ॒प्नो॒ति॒ तस्मै॒ तस्मा॑ आप्नो त्याप्नोति॒ तस्मै᳚ । \newline
22. तस्मा॑ ए॒त दे॒तत् तस्मै॒ तस्मा॑ ए॒तत् । \newline
23. ए॒तद् भा॑ग॒धेय॑म् भाग॒धेय॑ मे॒त दे॒तद् भा॑ग॒धेय᳚म् । \newline
24. भा॒ग॒धेय॒म् प्र प्र भा॑ग॒धेय॑म् भाग॒धेय॒म् प्र । \newline
25. भा॒ग॒धेय॒मिति॑ भाग - धेय᳚म् । \newline
26. प्राय॑च्छन् नयच्छ॒न् प्र प्राय॑च्छन्न् । \newline
27. अ॒य॒च्छ॒न्॒. यद् यद॑यच्छन् नयच्छ॒न्॒. यत् । \newline
28. यद॒ग्नये॒ ऽग्नये॒ यद् यद॒ग्नये᳚ । \newline
29. अ॒ग्नये᳚ स्विष्ट॒कृते᳚ स्विष्ट॒कृते॒ ऽग्नये॒ ऽग्नये᳚ स्विष्ट॒कृते᳚ । \newline
30. स्वि॒ष्ट॒कृते॑ ऽव॒द्य न्त्य॑व॒द्यन्ति॑ स्विष्ट॒कृते᳚ स्विष्ट॒कृते॑ ऽव॒द्यन्ति॑ । \newline
31. स्वि॒ष्ट॒कृत॒ इति॑ स्विष्ट - कृते᳚ । \newline
32. अ॒व॒द्यन्ति॒ यद् यद॑व॒द्य न्त्य॑व॒द्यन्ति॒ यत् । \newline
33. अ॒व॒द्यन्तीत्य॑व - द्यन्ति॑ । \newline
34. यद॒ग्नये॒ ऽग्नये॒ यद् यद॒ग्नये᳚ । \newline
35. अ॒ग्नये᳚ स्विष्ट॒कृते᳚ स्विष्ट॒कृते॒ ऽग्नये॒ ऽग्नये᳚ स्विष्ट॒कृते᳚ । \newline
36. स्वि॒ष्ट॒कृते॑ ऽव॒द्य त्य॑व॒द्यति॑ स्विष्ट॒कृते᳚ स्विष्ट॒कृते॑ ऽव॒द्यति॑ । \newline
37. स्वि॒ष्ट॒कृत॒ इति॑ स्विष्ट - कृते᳚ । \newline
38. अ॒व॒द्यति॑ भाग॒धेये॑न भाग॒धेये॑ना व॒द्य त्य॑व॒द्यति॑ भाग॒धेये॑न । \newline
39. अ॒व॒द्यतीत्य॑व - द्यति॑ । \newline
40. भा॒ग॒धेये॑ नै॒वैव भा॑ग॒धेये॑न भाग॒धेये॑ नै॒व । \newline
41. भा॒ग॒धेये॒नेति॑ भाग - धेये॑न । \newline
42. ए॒व तत् त दे॒वैव तत् । \newline
43. तद् रु॒द्रꣳ रु॒द्रम् तत् तद् रु॒द्रम् । \newline
44. रु॒द्रꣳ सꣳ सꣳ रु॒द्रꣳ रु॒द्रꣳ सम् । \newline
45. स म॑र्द्धय त्यर्द्धयति॒ सꣳ स म॑र्द्धयति । \newline
46. अ॒र्द्ध॒य॒ति॒ स॒कृथ्स॑कृथ् स॒कृथ्स॑कृ दर्द्धय त्यर्द्धयति स॒कृथ्स॑कृत् । \newline
47. स॒कृथ्स॑कृ॒ दवाव॑ स॒कृथ्स॑कृथ् स॒कृथ्स॑कृ॒ दव॑ । \newline
48. स॒कृथ्स॑कृ॒दिति॑ स॒कृत् - स॒कृ॒त् । \newline
49. अव॑ द्यति द्य॒त्यवाव॑ द्यति । \newline
50. द्य॒ति॒ स॒कृथ् स॒कृद् द्य॑ति द्यति स॒कृत् । \newline
51. स॒कृ दि॑वे व स॒कृथ् स॒कृ दि॑व । \newline
52. इ॒व॒ हि हीवे॑ व॒ हि । \newline
53. हि रु॒द्रो रु॒द्रो हि हि रु॒द्रः । \newline
54. रु॒द्र उ॑त्तरा॒र्द्धा दु॑त्तरा॒र्द्धाद् रु॒द्रो रु॒द्र उ॑त्तरा॒र्द्धात् । \newline
55. उ॒त्त॒रा॒र्द्धा दवावो᳚त्तरा॒र्द्धा दु॑त्तरा॒र्द्धा दव॑ । \newline
56. उ॒त्त॒रा॒र्द्धादित्यु॑त्तर - अ॒र्द्धात् । \newline
57. अव॑ द्यति द्य॒ त्यवाव॑ द्यति । \newline
58. द्य॒त्ये॒षैषा द्य॑ति द्यत्ये॒षा । \newline
59. ए॒षा वै वा ए॒षैषा वै । \newline
60. वै रु॒द्रस्य॑ रु॒द्रस्य॒ वै वै रु॒द्रस्य॑ । \newline
61. रु॒द्रस्य॒ दिग् दिग् रु॒द्रस्य॑ रु॒द्रस्य॒ दिक् । \newline

\textbf{Ghana Paata } \newline

1. उ॒पाम॑न्त्रयन्त रा॒ज्येन॑ रा॒ज्येनो॒ पाम॑न्त्रय न्तो॒पाम॑न्त्रयन्त रा॒ज्येन॑ पि॒तरः॑ पि॒तरो॑ रा॒ज्येनो॒पा 
म॑न्त्रय न्तो॒पाम॑न्त्रयन्त रा॒ज्येन॑ पि॒तरः॑ । \newline
2. उ॒पाम॑न्त्रय॒न्तेत्यु॑प - अम॑न्त्रयन्त । \newline
3. रा॒ज्येन॑ पि॒तरः॑ पि॒तरो॑ रा॒ज्येन॑ रा॒ज्येन॑ पि॒तरो॑ य॒मं ॅय॒मम् पि॒तरो॑ रा॒ज्येन॑ रा॒ज्येन॑ पि॒तरो॑ य॒मम् । \newline
4. पि॒तरो॑ य॒मं ॅय॒मम् पि॒तरः॑ पि॒तरो॑ य॒मम् तस्मा॒त् तस्मा᳚द् य॒मम् पि॒तरः॑ पि॒तरो॑ य॒मम् तस्मा᳚त् । \newline
5. य॒मम् तस्मा॒त् तस्मा᳚द् य॒मं ॅय॒मम् तस्मा॑ द॒ग्नि र॒ग्नि स्तस्मा᳚द् य॒मं ॅय॒मम् तस्मा॑ द॒ग्निः । \newline
6. तस्मा॑ द॒ग्नि र॒ग्नि स्तस्मा॒त् तस्मा॑ द॒ग्निर् दे॒वाना᳚म् दे॒वाना॑ म॒ग्नि स्तस्मा॒त् तस्मा॑ द॒ग्निर् दे॒वाना᳚म् । \newline
7. अ॒ग्निर् दे॒वाना᳚म् दे॒वाना॑ म॒ग्नि र॒ग्निर् दे॒वाना॑ मन्ना॒दो᳚ ऽन्ना॒दो दे॒वाना॑ म॒ग्नि र॒ग्निर् दे॒वाना॑ मन्ना॒दः । \newline
8. दे॒वाना॑ मन्ना॒दो᳚ ऽन्ना॒दो दे॒वाना᳚म् दे॒वाना॑ मन्ना॒दो य॒मो य॒मो᳚ ऽन्ना॒दो दे॒वाना᳚म् दे॒वाना॑ मन्ना॒दो य॒मः । \newline
9. अ॒न्ना॒दो य॒मो य॒मो᳚ ऽन्ना॒दो᳚ ऽन्ना॒दो य॒मः पि॑तृ॒णाम् पि॑तृ॒णां ॅय॒मो᳚ ऽन्ना॒दो᳚ ऽन्ना॒दो य॒मः पि॑तृ॒णाम् । \newline
10. अ॒न्ना॒द इत्य॑न्न - अ॒दः । \newline
11. य॒मः पि॑तृ॒णाम् पि॑तृ॒णां ॅय॒मो य॒मः पि॑तृ॒णाꣳ राजा॒ राजा॑ पितृ॒णां ॅय॒मो य॒मः पि॑तृ॒णाꣳ राजा᳚ । \newline
12. पि॒तृ॒णाꣳ राजा॒ राजा॑ पितृ॒णाम् पि॑तृ॒णाꣳ राजा॒ यो यो राजा॑ पितृ॒णाम् पि॑तृ॒णाꣳ राजा॒ यः । \newline
13. राजा॒ यो यो राजा॒ राजा॒ य ए॒व मे॒वं ॅयो राजा॒ राजा॒ य ए॒वम् । \newline
14. य ए॒व मे॒वं ॅयो य ए॒वं ॅवेद॒ वेदै॒वं ॅयो य ए॒वं ॅवेद॑ । \newline
15. ए॒वं ॅवेद॒ वेदै॒व मे॒वं ॅवेद॒ प्र प्र वेदै॒व मे॒वं ॅवेद॒ प्र । \newline
16. वेद॒ प्र प्र वेद॒ वेद॒ प्र रा॒ज्यꣳ रा॒ज्यम् प्र वेद॒ वेद॒ प्र रा॒ज्यम् । \newline
17. प्र रा॒ज्यꣳ रा॒ज्यम् प्र प्र रा॒ज्य म॒न्नाद्य॑ म॒न्नाद्यꣳ॑ रा॒ज्यम् प्र प्र रा॒ज्य म॒न्नाद्य᳚म् । \newline
18. रा॒ज्य म॒न्नाद्य॑ म॒न्नाद्यꣳ॑ रा॒ज्यꣳ रा॒ज्य म॒न्नाद्य॑ माप्नो त्याप्नो त्य॒न्नाद्यꣳ॑ रा॒ज्यꣳ रा॒ज्य म॒न्नाद्य॑ माप्नोति । \newline
19. अ॒न्नाद्य॑ माप्नो त्याप्नो त्य॒न्नाद्य॑ म॒न्नाद्य॑ माप्नोति॒ तस्मै॒ तस्मा॑ आप्नो त्य॒न्नाद्य॑ म॒न्नाद्य॑ माप्नोति॒ तस्मै᳚ । \newline
20. अ॒न्नाद्य॒मित्य॑न्न - अद्य᳚म् । \newline
21. आ॒प्नो॒ति॒ तस्मै॒ तस्मा॑ आप्नो त्याप्नोति॒ तस्मा॑ ए॒तदे॒तत् तस्मा॑ आप्नो त्याप्नोति॒ तस्मा॑ ए॒तत् । \newline
22. तस्मा॑ ए॒त दे॒तत् तस्मै॒ तस्मा॑ ए॒तद् भा॑ग॒धेय॑म् भाग॒धेय॑ मे॒तत् तस्मै॒ तस्मा॑ ए॒तद् भा॑ग॒धेय᳚म् । \newline
23. ए॒तद् भा॑ग॒धेय॑म् भाग॒धेय॑ मे॒त दे॒तद् भा॑ग॒धेय॒म् प्र प्र भा॑ग॒धेय॑ मे॒त दे॒तद् भा॑ग॒धेय॒म् प्र । \newline
24. भा॒ग॒धेय॒म् प्र प्र भा॑ग॒धेय॑म् भाग॒धेय॒म् प्राय॑च्छन् नयच्छ॒न् प्र भा॑ग॒धेय॑म् भाग॒धेय॒म् प्राय॑च्छन्न् । \newline
25. भा॒ग॒धेय॒मिति॑ भाग - धेय᳚म् । \newline
26. प्राय॑च्छन् नयच्छ॒न् प्र प्राय॑च्छ॒न्॒. यद् यद॑यच्छ॒न् प्र प्राय॑च्छ॒न्॒. यत् । \newline
27. अ॒य॒च्छ॒न्॒. यद् यद॑यच्छन् नयच्छ॒न्॒. यद॒ग्नये॒ ऽग्नये॒ यद॑यच्छन् नयच्छ॒न्॒. यद॒ग्नये᳚ । \newline
28. यद॒ग्नये॒ ऽग्नये॒ यद् यद॒ग्नये᳚ स्विष्ट॒कृते᳚ स्विष्ट॒कृते॒ ऽग्नये॒ यद् यद॒ग्नये᳚ स्विष्ट॒कृते᳚ । \newline
29. अ॒ग्नये᳚ स्विष्ट॒कृते᳚ स्विष्ट॒कृते॒ ऽग्नये॒ ऽग्नये᳚ स्विष्ट॒कृते॑ ऽव॒द्य न्त्य॑व॒द्यन्ति॑ स्विष्ट॒कृते॒ ऽग्नये॒ ऽग्नये᳚ स्विष्ट॒कृते॑ ऽव॒द्यन्ति॑ । \newline
30. स्वि॒ष्ट॒कृते॑ ऽव॒द्य न्त्य॑व॒द्यन्ति॑ स्विष्ट॒कृते᳚ स्विष्ट॒कृते॑ ऽव॒द्यन्ति॒ यद् यद॑व॒द्यन्ति॑ स्विष्ट॒कृते᳚ स्विष्ट॒कृते॑ ऽव॒द्यन्ति॒ यत् । \newline
31. स्वि॒ष्ट॒कृत॒ इति॑ स्विष्ट - कृते᳚ । \newline
32. अ॒व॒द्यन्ति॒ यद् यद॑व॒द्य न्त्य॑व॒द्यन्ति॒ यद॒ग्नये॒ ऽग्नये॒ यद॑व॒द्य न्त्य॑व॒द्यन्ति॒ यद॒ग्नये᳚ । \newline
33. अ॒व॒द्यन्तीत्य॑व - द्यन्ति॑ । \newline
34. यद॒ग्नये॒ ऽग्नये॒ यद् यद॒ग्नये᳚ स्विष्ट॒कृते᳚ स्विष्ट॒कृते॒ ऽग्नये॒ यद् यद॒ग्नये᳚ स्विष्ट॒कृते᳚ । \newline
35. अ॒ग्नये᳚ स्विष्ट॒कृते᳚ स्विष्ट॒कृते॒ ऽग्नये॒ ऽग्नये᳚ स्विष्ट॒कृते॑ ऽव॒द्य त्य॑व॒द्यति॑ स्विष्ट॒कृते॒ ऽग्नये॒ ऽग्नये᳚ स्विष्ट॒कृते॑ ऽव॒द्यति॑ । \newline
36. स्वि॒ष्ट॒कृते॑ ऽव॒ द्यत्य॑व॒द्यति॑ स्विष्ट॒कृते᳚ स्विष्ट॒कृते॑ ऽव॒द्यति॑ भाग॒धेये॑न भाग॒धेये॑ना व॒द्यति॑ स्विष्ट॒कृते᳚ स्विष्ट॒कृते॑ ऽव॒द्यति॑ भाग॒धेये॑न । \newline
37. स्वि॒ष्ट॒कृत॒ इति॑ स्विष्ट - कृते᳚ । \newline
38. अ॒व॒द्यति॑ भाग॒धेये॑न भाग॒धेये॑ना व॒द्य त्य॑व॒द्यति॑ भाग॒धेये॑ नै॒वैव भा॑ग॒धेये॑ना व॒द्य त्य॑व॒द्यति॑ भाग॒धेये॑नै॒व । \newline
39. अ॒व॒द्यतीत्य॑व - द्यति॑ । \newline
40. भा॒ग॒धेये॑ नै॒वैव भा॑ग॒धेये॑न भाग॒धेये॑नै॒व तत् तदे॒व भा॑ग॒धेये॑न भाग॒धेये॑नै॒व तत् । \newline
41. भा॒ग॒धेये॒नेति॑ भाग - धेये॑न । \newline
42. ए॒व तत् तदे॒वैव तद् रु॒द्रꣳ रु॒द्रम् तदे॒वैव तद् रु॒द्रम् । \newline
43. तद् रु॒द्रꣳ रु॒द्रम् तत् तद् रु॒द्रꣳ सꣳ सꣳ रु॒द्रम् तत् तद् रु॒द्रꣳ सम् । \newline
44. रु॒द्रꣳ सꣳ सꣳ रु॒द्रꣳ रु॒द्रꣳ स म॑र्द्धय त्यर्द्धयति॒ सꣳ रु॒द्रꣳ रु॒द्रꣳ स म॑र्द्धयति । \newline
45. स म॑र्द्धय त्यर्द्धयति॒ सꣳ स म॑र्द्धयति स॒कृथ्स॑कृथ् स॒कृथ्स॑कृ दर्द्धयति॒ सꣳ स म॑र्द्धयति स॒कृथ्स॑कृत् । \newline
46. अ॒र्द्ध॒य॒ति॒ स॒कृथ्स॑कृथ् स॒कृथ्स॑कृ दर्द्धय त्यर्द्धयति स॒कृथ्स॑कृ॒ दवाव॑ स॒कृथ्स॑कृ दर्द्धय त्यर्द्धयति स॒कृथ्स॑कृ॒ दव॑ । \newline
47. स॒कृथ्स॑कृ॒ दवाव॑ स॒कृथ्स॑कृथ् स॒कृथ्स॑कृ॒ दव॑ द्यति द्य॒त्यव॑ स॒कृथ्स॑कृथ् स॒कृथ्स॑कृ॒ दव॑ द्यति । \newline
48. स॒कृथ्स॑कृ॒दिति॑ स॒कृत् - स॒कृ॒त् । \newline
49. अव॑ द्यति द्य॒त्यवाव॑ द्यति स॒कृथ् स॒कृद् द्य॒त्यवाव॑ द्यति स॒कृत् । \newline
50. द्य॒ति॒ स॒कृथ् स॒कृद् द्य॑ति द्यति स॒कृदि॑वे व स॒कृद् द्य॑ति द्यति स॒कृदि॑व । \newline
51. स॒कृदि॑वे व स॒कृथ् स॒कृदि॑व॒ हि हीव॑ स॒कृथ् स॒कृदि॑व॒ हि । \newline
52. इ॒व॒ हि हीवे॑ व॒ हि रु॒द्रो रु॒द्रो हीवे॑ व॒ हि रु॒द्रः । \newline
53. हि रु॒द्रो रु॒द्रो हि हि रु॒द्र उ॑त्तरा॒र्द्धा दु॑त्तरा॒र्द्धाद् रु॒द्रो हि हि रु॒द्र उ॑त्तरा॒र्द्धात् । \newline
54. रु॒द्र उ॑त्तरा॒र्द्धा दु॑त्तरा॒र्द्धाद् रु॒द्रो रु॒द्र उ॑त्तरा॒र्द्धा दवावो᳚ त्तरा॒र्द्धाद् रु॒द्रो रु॒द्र उ॑त्तरा॒र्द्धा दव॑ । \newline
55. उ॒त्त॒रा॒र्द्धा दवावो᳚ त्तरा॒र्द्धा दु॑त्तरा॒र्द्धा दव॑ द्यति द्य॒त्यवो᳚त्तरा॒र्द्धा दु॑त्तरा॒र्द्धा दव॑ द्यति । \newline
56. उ॒त्त॒रा॒र्द्धादित्यु॑त्तर - अ॒र्द्धात् । \newline
57. अव॑ द्यति द्य॒त्यवाव॑ द्यत्ये॒षैषा द्य॒त्यवाव॑ द्यत्ये॒षा । \newline
58. द्य॒त्ये॒षैषा द्य॑ति द्यत्ये॒षा वै वा ए॒षा द्य॑ति द्यत्ये॒षा वै । \newline
59. ए॒षा वै वा ए॒षैषा वै रु॒द्रस्य॑ रु॒द्रस्य॒ वा ए॒षैषा वै रु॒द्रस्य॑ । \newline
60. वै रु॒द्रस्य॑ रु॒द्रस्य॒ वै वै रु॒द्रस्य॒ दिग् दिग् रु॒द्रस्य॒ वै वै रु॒द्रस्य॒ दिक् । \newline
61. रु॒द्रस्य॒ दिग् दिग् रु॒द्रस्य॑ रु॒द्रस्य॒ दिख् स्वायाꣳ॒॒ स्वाया॒म् दिग् रु॒द्रस्य॑ रु॒द्रस्य॒ दिख् स्वाया᳚म् । \newline
\pagebreak
\markright{ TS 2.6.6.6  \hfill https://www.vedavms.in \hfill}
\addcontentsline{toc}{section}{ TS 2.6.6.6 }
\section*{ TS 2.6.6.6 }

\textbf{TS 2.6.6.6 } \newline
\textbf{Samhita Paata} \newline

दिख् स्वाया॑मे॒व दि॒शि रु॒द्रं नि॒रव॑दयते॒ द्विर॒भि घा॑रयति चतुरव॒त्तस्याऽऽप्त्यै॑प॒शवो॒ वै पूर्वा॒ आहु॑तय ए॒ष रु॒द्रो यद॒ग्निर्यत् पूर्वा॒ आहु॑तीर॒भि जु॑हु॒याद्-रु॒द्राय॑ प॒शूनपि॑ दध्यादप॒शुर्यज॑मानः स्यादति॒हाय॒ पूर्वा॒ आहु॑तीर्जुहोति पशू॒नां गो॑पी॒थाय॑ ॥ \newline

\textbf{Pada Paata} \newline

दिक् । स्वाया᳚म् । ए॒व । दि॒शि । रु॒द्रम् । नि॒रव॑दयत॒ इति॑ निः-अव॑दयते । द्विः । अ॒भीति॑ । घा॒र॒य॒ति॒ । च॒तु॒र॒व॒त्तस्येति॑ चतुः - अ॒व॒त्तस्य॑ । आप्त्यै᳚ । प॒शवः॑ । वै । पूर्वाः᳚ । आहु॑तय॒ इत्या - हु॒त॒यः॒ । ए॒षः । रु॒द्रः । यत् । अ॒ग्निः । यत् । पूर्वाः᳚ । आहु॑ती॒रित्या - हु॒तीः॒ । अ॒भीति॑ । जु॒हु॒यात् । रु॒द्राय॑ । प॒शून् । अपीति॑ । द॒द्ध्या॒त् । अ॒प॒शुः । यज॑मानः । स्या॒त् । अ॒ति॒हायेत्य॑ति - हाय॑ । पूर्वाः᳚ । आहु॑ती॒रित्या - हु॒तीः॒ । जु॒हो॒ति॒ । प॒शू॒नाम् । गो॒पी॒थाय॑ ॥  \newline


\textbf{Krama Paata} \newline

दिख् स्वाया᳚म् । स्वाया॑मे॒व । ए॒व दि॒शि । दि॒शि रु॒द्रम् । रु॒द्रम् नि॒रव॑दयते । नि॒रव॑दयते॒ द्विः । नि॒रव॑दयत॒ इति॑ निः - अव॑दयते । द्विर॒भि । अ॒भि घा॑रयति । घा॒र॒य॒ति॒ च॒तु॒र॒व॒त्तस्य॑ । च॒तु॒र॒व॒त्तस्याप्त्यै᳚ । च॒तु॒र॒व॒त्तस्येति॑ चतुः - अ॒व॒त्तस्य॑ । आप्त्यै॑ प॒शवः॑ । प॒शवो॒ वै । वै पूर्वाः᳚ । पूर्वा॒ आहु॑तयः । आहु॑तय ए॒षः । आहु॑तय॒ इत्या - हु॒त॒यः॒ । ए॒ष रु॒द्रः । रु॒द्रो यत् । यद॒ग्निः । अ॒ग्निर् यत् । यत् पूर्वाः᳚ । पूर्वा॒ आहु॑तीः । आहु॑तीर॒भि । आहु॑ती॒रित्या - हु॒तीः॒ । अ॒भि जु॑हु॒यात् । जु॒हु॒याद् रु॒द्राय॑ । रु॒द्राय॑ प॒शून् । प॒शूनपि॑ । अपि॑ दद्ध्यात् । द॒द्ध्या॒द॒प॒शुः । अ॒प॒शुर् यज॑मानः । यज॑मानः स्यात् । स्या॒द॒ति॒हाय॑ । अ॒ति॒हाय॒ पूर्वाः᳚ । अ॒ति॒हायेत्य॑ति - हाय॑ । पूर्वा॒ आहु॑तीः । आहु॑तीर् जुहोति । आहु॑ती॒रित्या - हु॒तीः॒ । जु॒हो॒ति॒ प॒शू॒नाम् । प॒शू॒नाम् गो॑पी॒थाय॑ । गो॒पी॒थायेति॑ गोपी॒थाय॑ । \newline

\textbf{Jatai Paata} \newline

1. दिख् स्वायाꣳ॒॒ स्वाया॒म् दिग् दिख् स्वाया᳚म् । \newline
2. स्वाया॑ मे॒वैव स्वायाꣳ॒॒ स्वाया॑ मे॒व । \newline
3. ए॒व दि॒शि दि॒श्ये॑वैव दि॒शि । \newline
4. दि॒शि रु॒द्रꣳ रु॒द्रम् दि॒शि दि॒शि रु॒द्रम् । \newline
5. रु॒द्रम् नि॒रव॑दयते नि॒रव॑दयते रु॒द्रꣳ रु॒द्रम् नि॒रव॑दयते । \newline
6. नि॒रव॑दयते॒ द्विर् द्विर् नि॒रव॑दयते नि॒रव॑दयते॒ द्विः । \newline
7. नि॒रव॑दयत॒ इति॑ निः - अव॑दयते । \newline
8. द्विर॒भ्य॑भि द्विर् द्विर॒भि । \newline
9. अ॒भि घा॑रयति घारय त्य॒भ्य॑भि घा॑रयति । \newline
10. घा॒र॒य॒ति॒ च॒तु॒र॒व॒त्तस्य॑ चतुरव॒त्तस्य॑ घारयति घारयति चतुरव॒त्तस्य॑ । \newline
11. च॒तु॒र॒व॒त्त स्याप्त्या॒ आप्त्यै॑ चतुरव॒त्तस्य॑ चतुरव॒त्त स्याप्त्यै᳚ । \newline
12. च॒तु॒र॒व॒त्तस्येति॑ चतुः - अ॒व॒त्तस्य॑ । \newline
13. आप्त्यै॑ प॒शवः॑ प॒शव॒ आप्त्या॒ आप्त्यै॑ प॒शवः॑ । \newline
14. प॒शवो॒ वै वै प॒शवः॑ प॒शवो॒ वै । \newline
15. वै पूर्वाः॒ पूर्वा॒ वै वै पूर्वाः᳚ । \newline
16. पूर्वा॒ आहु॑तय॒ आहु॑तयः॒ पूर्वाः॒ पूर्वा॒ आहु॑तयः । \newline
17. आहु॑तय ए॒ष ए॒ष आहु॑तय॒ आहु॑तय ए॒षः । \newline
18. आहु॑तय॒ इत्या - हु॒त॒यः॒ । \newline
19. ए॒ष रु॒द्रो रु॒द्र ए॒ष ए॒ष रु॒द्रः । \newline
20. रु॒द्रो यद् यद् रु॒द्रो रु॒द्रो यत् । \newline
21. यद॒ग्नि र॒ग्निर् यद् यद॒ग्निः । \newline
22. अ॒ग्निर् यद् यद॒ग्नि र॒ग्निर् यत् । \newline
23. यत् पूर्वाः॒ पूर्वा॒ यद् यत् पूर्वाः᳚ । \newline
24. पूर्वा॒ आहु॑ती॒ राहु॑तीः॒ पूर्वाः॒ पूर्वा॒ आहु॑तीः । \newline
25. आहु॑ती र॒भ्य॑भ्याहु॑ती॒ राहु॑ती र॒भि । \newline
26. आहु॑ती॒रित्या - हु॒तीः॒ । \newline
27. अ॒भि जु॑हु॒याज् जु॑हु॒या द॒भ्य॑भि जु॑हु॒यात् । \newline
28. जु॒हु॒याद् रु॒द्राय॑ रु॒द्राय॑ जुहु॒याज् जु॑हु॒याद् रु॒द्राय॑ । \newline
29. रु॒द्राय॑ प॒शून् प॒शून् रु॒द्राय॑ रु॒द्राय॑ प॒शून् । \newline
30. प॒शू नप्यपि॑ प॒शून् प॒शू नपि॑ । \newline
31. अपि॑ दद्ध्याद् दद्ध्या॒ दप्यपि॑ दद्ध्यात् । \newline
32. द॒द्ध्या॒ द॒प॒शु र॑प॒शुर् द॑द्ध्याद् दद्ध्या दप॒शुः । \newline
33. अ॒प॒शुर् यज॑मानो॒ यज॑मानो ऽप॒शु र॑प॒शुर् यज॑मानः । \newline
34. यज॑मानः स्याथ् स्या॒द् यज॑मानो॒ यज॑मानः स्यात् । \newline
35. स्या॒ द॒ति॒हाया॑ ति॒हाय॑ स्याथ् स्या दति॒हाय॑ । \newline
36. अ॒ति॒हाय॒ पूर्वाः॒ पूर्वा॑ अति॒हाया॑ ति॒हाय॒ पूर्वाः᳚ । \newline
37. अ॒ति॒हायेत्य॑ति - हाय॑ । \newline
38. पूर्वा॒ आहु॑ती॒ राहु॑तीः॒ पूर्वाः॒ पूर्वा॒ आहु॑तीः । \newline
39. आहु॑तीर् जुहोति जुहो॒ त्याहु॑ती॒ राहु॑तीर् जुहोति । \newline
40. आहु॑ती॒रित्या - हु॒तीः॒ । \newline
41. जु॒हो॒ति॒ प॒शू॒नाम् प॑शू॒नाम् जु॑होति जुहोति पशू॒नाम् । \newline
42. प॒शू॒नाम् गो॑पी॒थाय॑ गोपी॒थाय॑ पशू॒नाम् प॑शू॒नाम् गो॑पी॒थाय॑ । \newline
43. गो॒पी॒थायेति॑ गोपी॒थाय॑ । \newline

\textbf{Ghana Paata } \newline

1. दिख् स्वायाꣳ॒॒ स्वाया॒म् दिग् दिख् स्वाया॑ मे॒वैव स्वाया॒म् दिग् दिख् स्वाया॑ मे॒व । \newline
2. स्वाया॑ मे॒वैव स्वायाꣳ॒॒ स्वाया॑ मे॒व दि॒शि दि॒श्ये॑व स्वायाꣳ॒॒ स्वाया॑ मे॒व दि॒शि । \newline
3. ए॒व दि॒शि दि॒श्ये॑वैव दि॒शि रु॒द्रꣳ रु॒द्रम् दि॒श्ये॑वैव दि॒शि रु॒द्रम् । \newline
4. दि॒शि रु॒द्रꣳ रु॒द्रम् दि॒शि दि॒शि रु॒द्रम् नि॒रव॑दयते नि॒रव॑दयते रु॒द्रम् दि॒शि दि॒शि रु॒द्रम् नि॒रव॑दयते । \newline
5. रु॒द्रम् नि॒रव॑दयते नि॒रव॑दयते रु॒द्रꣳ रु॒द्रम् नि॒रव॑दयते॒ द्विर् द्विर् नि॒रव॑दयते रु॒द्रꣳ रु॒द्रम् नि॒रव॑दयते॒ द्विः । \newline
6. नि॒रव॑दयते॒ द्विर् द्विर् नि॒रव॑दयते नि॒रव॑दयते॒ द्विर॒भ्य॑भि द्विर् नि॒रव॑दयते नि॒रव॑दयते॒ द्विर॒भि । \newline
7. नि॒रव॑दयत॒ इति॑ निः - अव॑दयते । \newline
8. द्विर॒भ्य॑भि द्विर् द्विर॒भि घा॑रयति घारय त्य॒भि द्विर् द्विर॒भि घा॑रयति । \newline
9. अ॒भि घा॑रयति घारय त्य॒भ्य॑भि घा॑रयति चतुरव॒त्तस्य॑ चतुरव॒त्तस्य॑ घारय त्य॒भ्य॑भि घा॑रयति चतुरव॒त्तस्य॑ । \newline
10. घा॒र॒य॒ति॒ च॒तु॒र॒व॒त्तस्य॑ चतुरव॒त्तस्य॑ घारयति घारयति चतुरव॒त्तस्या प्त्या॒ आप्त्यै॑ चतुरव॒त्तस्य॑ घारयति घारयति चतुरव॒त्तस्या प्त्यै᳚ । \newline
11. च॒तु॒र॒व॒त्तस्या प्त्या॒ आप्त्यै॑ चतुरव॒त्तस्य॑ चतुरव॒त्तस्या प्त्यै॑ प॒शवः॑ प॒शव॒ आप्त्यै॑ चतुरव॒त्तस्य॑ चतुरव॒त्तस्या प्त्यै॑ प॒शवः॑ । \newline
12. च॒तु॒र॒व॒त्तस्येति॑ चतुः - अ॒व॒त्तस्य॑ । \newline
13. आप्त्यै॑ प॒शवः॑ प॒शव॒ आप्त्या॒ आप्त्यै॑ प॒शवो॒ वै वै प॒शव॒ आप्त्या॒ आप्त्यै॑ प॒शवो॒ वै । \newline
14. प॒शवो॒ वै वै प॒शवः॑ प॒शवो॒ वै पूर्वाः॒ पूर्वा॒ वै प॒शवः॑ प॒शवो॒ वै पूर्वाः᳚ । \newline
15. वै पूर्वाः॒ पूर्वा॒ वै वै पूर्वा॒ आहु॑तय॒ आहु॑तयः॒ पूर्वा॒ वै वै पूर्वा॒ आहु॑तयः । \newline
16. पूर्वा॒ आहु॑तय॒ आहु॑तयः॒ पूर्वाः॒ पूर्वा॒ आहु॑तय ए॒ष ए॒ष आहु॑तयः॒ पूर्वाः॒ पूर्वा॒ आहु॑तय ए॒षः । \newline
17. आहु॑तय ए॒ष ए॒ष आहु॑तय॒ आहु॑तय ए॒ष रु॒द्रो रु॒द्र ए॒ष आहु॑तय॒ आहु॑तय ए॒ष रु॒द्रः । \newline
18. आहु॑तय॒ इत्या - हु॒त॒यः॒ । \newline
19. ए॒ष रु॒द्रो रु॒द्र ए॒ष ए॒ष रु॒द्रो यद् यद् रु॒द्र ए॒ष ए॒ष रु॒द्रो यत् । \newline
20. रु॒द्रो यद् यद् रु॒द्रो रु॒द्रो यद॒ग्नि र॒ग्निर् यद् रु॒द्रो रु॒द्रो यद॒ग्निः । \newline
21. यद॒ग्नि र॒ग्निर् यद् यद॒ग्निर् यद् यद॒ग्निर् यद् यद॒ग्निर् यत् । \newline
22. अ॒ग्निर् यद् यद॒ग्नि र॒ग्निर् यत् पूर्वाः॒ पूर्वा॒ यद॒ग्नि र॒ग्निर् यत् पूर्वाः᳚ । \newline
23. यत् पूर्वाः॒ पूर्वा॒ यद् यत् पूर्वा॒ आहु॑ती॒ राहु॑तीः॒ पूर्वा॒ यद् यत् पूर्वा॒ आहु॑तीः । \newline
24. पूर्वा॒ आहु॑ती॒ राहु॑तीः॒ पूर्वाः॒ पूर्वा॒ आहु॑ती र॒भ्य॑भ्या हु॑तीः॒ पूर्वाः॒ पूर्वा॒ आहु॑ती र॒भि । \newline
25. आहु॑ती र॒भ्य॑भ्या हु॑ती॒ राहु॑ती र॒भि जु॑हु॒याज् जु॑हु॒या द॒भ्याहु॑ती॒ राहु॑ती र॒भि जु॑हु॒यात् । \newline
26. आहु॑ती॒रित्या - हु॒तीः॒ । \newline
27. अ॒भि जु॑हु॒याज् जु॑हु॒या द॒भ्य॑भि जु॑हु॒याद् रु॒द्राय॑ रु॒द्राय॑ जुहु॒या द॒भ्य॑भि जु॑हु॒याद् रु॒द्राय॑ । \newline
28. जु॒हु॒याद् रु॒द्राय॑ रु॒द्राय॑ जुहु॒याज् जु॑हु॒याद् रु॒द्राय॑ प॒शून् प॒शून् रु॒द्राय॑ जुहु॒याज् जु॑हु॒याद् रु॒द्राय॑ प॒शून् । \newline
29. रु॒द्राय॑ प॒शून् प॒शून् रु॒द्राय॑ रु॒द्राय॑ प॒शू नप्यपि॑ प॒शून् रु॒द्राय॑ रु॒द्राय॑ प॒शू नपि॑ । \newline
30. प॒शू नप्यपि॑ प॒शून् प॒शू नपि॑ दद्ध्याद् दद्ध्या॒ दपि॑ प॒शून् प॒शू नपि॑ दद्ध्यात् । \newline
31. अपि॑ दद्ध्याद् दद्ध्या॒ दप्यपि॑ दद्ध्या दप॒शु र॑प॒शुर् द॑द्ध्या॒ दप्यपि॑ दद्ध्या दप॒शुः । \newline
32. द॒द्ध्या॒ द॒प॒शु र॑प॒शुर् द॑द्ध्याद् दद्ध्या दप॒शुर् यज॑मानो॒ यज॑मानो ऽप॒शुर् द॑द्ध्याद् दद्ध्या दप॒शुर् यज॑मानः । \newline
33. अ॒प॒शुर् यज॑मानो॒ यज॑मानो ऽप॒शु र॑प॒शुर् यज॑मानः स्याथ् स्या॒द् यज॑मानो ऽप॒शु र॑प॒शुर् यज॑मानः स्यात् । \newline
34. यज॑मानः स्याथ् स्या॒द् यज॑मानो॒ यज॑मानः स्या दति॒हाया॑ ति॒हाय॑ स्या॒द् यज॑मानो॒ यज॑मानः स्यादति॒हाय॑ । \newline
35. स्या॒ द॒ति॒हाया॑ ति॒हाय॑ स्याथ् स्या दति॒हाय॒ पूर्वाः॒ पूर्वा॑ अति॒हाय॑ स्याथ् स्या दति॒हाय॒ पूर्वाः᳚ । \newline
36. अ॒ति॒हाय॒ पूर्वाः॒ पूर्वा॑ अति॒हाया॑ ति॒हाय॒ पूर्वा॒ आहु॑ती॒ राहु॑तीः॒ पूर्वा॑ अति॒हाया॑ ति॒हाय॒ पूर्वा॒ आहु॑तीः । \newline
37. अ॒ति॒हायेत्य॑ति - हाय॑ । \newline
38. पूर्वा॒ आहु॑ती॒ राहु॑तीः॒ पूर्वाः॒ पूर्वा॒ आहु॑तीर् जुहोति जुहो॒ त्याहु॑तीः॒ पूर्वाः॒ पूर्वा॒ आहु॑तीर् जुहोति । \newline
39. आहु॑तीर् जुहोति जुहो॒ त्याहु॑ती॒ राहु॑तीर् जुहोति पशू॒नाम् प॑शू॒नाम् जु॑हो॒ त्याहु॑ती॒ राहु॑तीर् जुहोति पशू॒नाम् । \newline
40. आहु॑ती॒रित्या - हु॒तीः॒ । \newline
41. जु॒हो॒ति॒ प॒शू॒नाम् प॑शू॒नाम् जु॑होति जुहोति पशू॒नाम् गो॑पी॒थाय॑ गोपी॒थाय॑ पशू॒नाम् जु॑होति जुहोति पशू॒नाम् गो॑पी॒थाय॑ । \newline
42. प॒शू॒नाम् गो॑पी॒थाय॑ गोपी॒थाय॑ पशू॒नाम् प॑शू॒नाम् गो॑पी॒थाय॑ । \newline
43. गो॒पी॒थायेति॑ गोपी॒थाय॑ । \newline
\pagebreak
\markright{ TS 2.6.7.1  \hfill https://www.vedavms.in \hfill}
\addcontentsline{toc}{section}{ TS 2.6.7.1 }
\section*{ TS 2.6.7.1 }

\textbf{TS 2.6.7.1 } \newline
\textbf{Samhita Paata} \newline

मनुः॑ पृथि॒व्या य॒ज्ञिय॑मैच्छ॒थ् स घृ॒तं निषि॑क्तमविन्द॒थ् सो᳚ऽब्रवी॒त् को᳚ऽस्येश्व॒रो य॒ज्ञेऽपि॒ कर्तो॒रिति॒ ताव॑ब्रूतां मि॒त्रावरु॑णौ॒ गोरे॒वाऽऽवमी᳚श्व॒रौ कर्तोः᳚ स्व॒ इति॒ तौ ततो॒ गाꣳ समै॑रयताꣳ॒॒ सा यत्र॑ यत्र॒ न्यक्रा॑म॒त् ततो॑ घृ॒तम॑पीड्यत॒ तस्मा᳚द्-घृ॒तप॑द्युच्यते॒ तद॑स्यै॒ जन्मोप॑हूतꣳ रथन्त॒रꣳ स॒ह पृ॑थि॒व्येत्या॑हे॒ - [  ] \newline

\textbf{Pada Paata} \newline

मनुः॑ । पृ॒थि॒व्याः । य॒ज्ञिय᳚म् । ऐ॒च्छ॒त् । सः । घृ॒तम् । निषि॑क्त॒मिति॒ नि - सि॒क्त॒म् । अ॒वि॒न्द॒त् । सः । अ॒ब्र॒वी॒त् । कः । अ॒स्य । ई॒श्व॒रः । य॒ज्ञे । अपीति॑ । कर्तोः᳚ । इति॑ । तौ । अ॒ब्रू॒ता॒म् । मि॒त्रावरु॑णा॒विति॑ मि॒त्रा - वरु॑णौ । गोः । ए॒व । आ॒वम् । ई॒श्व॒रौ । कर्तोः᳚ । स्वः॒ । इति॑ । तौ । ततः॑ । गाम् । समिति॑ । ऐ॒र॒य॒ता॒म् । सा । यत्र॑य॒त्रेति॒ यत्र॑ - य॒त्र॒ । न्यक्रा॑म॒दिति॑ नि - अक्रा॑मत् । ततः॑ । घृ॒तम् । अ॒पी॒ड्य॒त॒ । तस्मा᳚त् । घृ॒तप॒दीति॑ घृ॒त - प॒दी॒ । उ॒च्य॒ते॒ । तत् । अ॒स्यै॒ । जन्म॑ । उप॑हूत॒मित्युप॑ - हू॒त॒म् । र॒थ॒न्त॒रमिति॑ रथं - त॒रम् । स॒ह । पृ॒थि॒व्या । इति॑ । आ॒ह॒ ।  \newline


\textbf{Krama Paata} \newline

मनुः॑ पृथि॒व्याः । पृ॒थि॒व्या य॒ज्ञिय᳚म् । य॒ज्ञिय॑मैच्छत् । ऐ॒च्छ॒थ् सः । स घृ॒तम् । घृ॒तम् निषि॑क्तम् । निषि॑क्तमविन्दत् । निषि॑क्त॒मिति॒ नि - सि॒क्त॒म् । अ॒वि॒न्द॒थ् सः । सो᳚ऽब्रवीत् । अ॒ब्र॒वी॒त् कः । को᳚ऽस्य । अ॒स्येश्व॒रः । ई॒श्व॒रो य॒ज्ञे । य॒ज्ञे ऽपि॑ । अपि॒ कर्तोः᳚ । कर्तो॒रिति॑ । इति॒ तौ । ताव॑ब्रूताम् । अ॒ब्रू॒ता॒म् मि॒त्रावरु॑णौ । मि॒त्रावरु॑णौ॒ गोः । मि॒त्रावरु॑णा॒विति॑ मि॒त्रा - वरु॑णौ । गोरे॒व । ए॒वावम् । आ॒वमी᳚श्व॒रौ । ई॒श्व॒रौ कर्तोः᳚ । कर्तोः᳚ स्वः । स्व॒ इति॑ । इति॒ तौ । तौ ततः॑ । ततो॒ गाम् । गाꣳ सम् । समै॑रयताम् । ऐ॒र॒य॒ताꣳ॒॒ सा । सा यत्र॑यत्र । यत्र॑यत्र॒ न्यक्रा॑मत् । यत्र॑य॒त्रेति॒ यत्र॑ - य॒त्र॒ । न्यक्रा॑म॒त् ततः॑ । न्यक्रा॑म॒दिति॑ नि - अक्रा॑मत् । ततो॑ घृ॒तम् । घृ॒तम॑पीड्यत । अ॒पी॒ड्य॒त॒ तस्मा᳚त् । तस्मा᳚द् घृ॒तप॑दी । घृ॒तप॑द्युच्यते । घृ॒तप॒दीति॑ घृ॒त - प॒दी॒ । उ॒च्य॒ते॒ तत् । तद॑स्यै । अ॒स्यै॒ जन्म॑ । जन्मोप॑हूतम् । उप॑हूतꣳ रथन्त॒रम् । उप॑हूत॒मित्युप॑ - हू॒त॒म् । र॒थ॒न्त॒रꣳ स॒ह । र॒थ॒न्त॒रमिति॑ रथम् - त॒रम् । स॒ह पृ॑थि॒व्या । पृ॒थि॒व्येति॑ । इत्या॑ह । आ॒हे॒यम् \newline

\textbf{Jatai Paata} \newline

1. मनुः॑ पृथि॒व्याः पृ॑थि॒व्या मनु॒र् मनुः॑ पृथि॒व्याः । \newline
2. पृ॒थि॒व्या य॒ज्ञियं॑ ॅय॒ज्ञिय॑म् पृथि॒व्याः पृ॑थि॒व्या य॒ज्ञिय᳚म् । \newline
3. य॒ज्ञिय॑ मैच्छ दैच्छद् य॒ज्ञियं॑ ॅय॒ज्ञिय॑ मैच्छत् । \newline
4. ऐ॒च्छ॒थ् स स ऐ᳚च्छ दैच्छ॒थ् सः । \newline
5. स घृ॒तम् घृ॒तꣳ स स घृ॒तम् । \newline
6. घृ॒तम् निषि॑क्त॒म् निषि॑क्तम् घृ॒तम् घृ॒तम् निषि॑क्तम् । \newline
7. निषि॑क्त मविन्द दविन्द॒न् निषि॑क्त॒म् निषि॑क्त मविन्दत् । \newline
8. निषि॑क्त॒मिति॒ नि - सि॒क्त॒म् । \newline
9. अ॒वि॒न्द॒थ् स सो॑ ऽविन्द दविन्द॒थ् सः । \newline
10. सो᳚ ऽब्रवी दब्रवी॒थ् स सो᳚ ऽब्रवीत् । \newline
11. अ॒ब्र॒वी॒त् कः को᳚ ऽब्रवी दब्रवी॒त् कः । \newline
12. को᳚ ऽस्यास्य कः को᳚ ऽस्य । \newline
13. अ॒स्येश्व॒र ई᳚श्व॒रो᳚ ऽस्या स्येश्व॒रः । \newline
14. ई॒श्व॒रो य॒ज्ञे य॒ज्ञ् ई᳚श्व॒र ई᳚श्व॒रो य॒ज्ञे । \newline
15. य॒ज्ञे ऽप्यपि॑ य॒ज्ञे य॒ज्ञे ऽपि॑ । \newline
16. अपि॒ कर्तोः॒ कर्तो॒ रप्यपि॒ कर्तोः᳚ । \newline
17. कर्तो॒ रितीति॒ कर्तोः॒ कर्तो॒ रिति॑ । \newline
18. इति॒ तौ ता वितीति॒ तौ । \newline
19. ता व॑ब्रूता मब्रूता॒म् तौ ता व॑ब्रूताम् । \newline
20. अ॒ब्रू॒ता॒म् मि॒त्रावरु॑णौ मि॒त्रावरु॑णा वब्रूता मब्रूताम् मि॒त्रावरु॑णौ । \newline
21. मि॒त्रावरु॑णौ॒ गोर् गोर् मि॒त्रावरु॑णौ मि॒त्रावरु॑णौ॒ गोः । \newline
22. मि॒त्रावरु॑णा॒विति॑ मि॒त्रा - वरु॑णौ । \newline
23. गो रे॒वैव गोर् गो रे॒व । \newline
24. ए॒वाव मा॒व मे॒वैवावम् । \newline
25. आ॒व मी᳚श्व॒रा वी᳚श्व॒रा वा॒व मा॒व मी᳚श्व॒रौ । \newline
26. ई॒श्व॒रौ कर्तोः॒ कर्तो॑ रीश्व॒रा वी᳚श्व॒रौ कर्तोः᳚ । \newline
27. कर्तोः᳚ स्वः स्वः॒ कर्तोः॒ कर्तोः᳚ स्वः । \newline
28. स्व॒ इतीति॑ स्वः स्व॒ इति॑ । \newline
29. इति॒ तौ ता वितीति॒ तौ । \newline
30. तौ तत॒ स्तत॒ स्तौ तौ ततः॑ । \newline
31. ततो॒ गाम् गाम् तत॒ स्ततो॒ गाम् । \newline
32. गाꣳ सꣳ सम् गाम् गाꣳ सम् । \newline
33. स मै॑रयता मैरयताꣳ॒॒ सꣳ स मै॑रयताम् । \newline
34. ऐ॒र॒य॒ताꣳ॒॒ सा सैर॑यता मैरयताꣳ॒॒ सा । \newline
35. सा यत्र॑यत्र॒ यत्र॑यत्र॒ सा सा यत्र॑यत्र । \newline
36. यत्र॑यत्र॒ न्यक्रा॑म॒न् न्यक्रा॑म॒द् यत्र॑यत्र॒ यत्र॑यत्र॒ न्यक्रा॑मत् । \newline
37. यत्र॑य॒त्रेति॒ यत्र॑ - य॒त्र॒ । \newline
38. न्यक्रा॑म॒त् तत॒ स्ततो॒ न्यक्रा॑म॒न् न्यक्रा॑म॒त् ततः॑ । \newline
39. न्यक्रा॑म॒दिति॑ नि - अक्रा॑मत् । \newline
40. ततो॑ घृ॒तम् घृ॒तम् तत॒ स्ततो॑ घृ॒तम् । \newline
41. घृ॒त म॑पीड्यता पीड्यत घृ॒तम् घृ॒त म॑पीड्यत । \newline
42. अ॒पी॒ड्य॒त॒ तस्मा॒त् तस्मा॑ दपीड्यता पीड्यत॒ तस्मा᳚त् । \newline
43. तस्मा᳚द् घृ॒तप॑दी घृ॒तप॑दी॒ तस्मा॒त् तस्मा᳚द् घृ॒तप॑दी । \newline
44. घृ॒तप॑ द्युच्यत उच्यते घृ॒तप॑दी घृ॒तप॑ द्युच्यते । \newline
45. घृ॒तप॒दीति॑ घृ॒त - प॒दी॒ । \newline
46. उ॒च्य॒ते॒ तत् तदु॑च्यत उच्यते॒ तत् । \newline
47. तद॑स्या अस्यै॒ तत् तद॑स्यै । \newline
48. अ॒स्यै॒ जन्म॒ जन्मा᳚स्या अस्यै॒ जन्म॑ । \newline
49. जन्मोप॑हूत॒ मुप॑हूत॒म् जन्म॒ जन्मोप॑हूतम् । \newline
50. उप॑हूतꣳ रथन्त॒रꣳ र॑थन्त॒र मुप॑हूत॒ मुप॑हूतꣳ रथन्त॒रम् । \newline
51. उप॑हूत॒मित्युप॑ - हू॒त॒म् । \newline
52. र॒थ॒न्त॒रꣳ स॒ह स॒ह र॑थन्त॒रꣳ र॑थन्त॒रꣳ स॒ह । \newline
53. र॒थ॒न्त॒रमिति॑ रथं - त॒रम् । \newline
54. स॒ह पृ॑थि॒व्या पृ॑थि॒व्या स॒ह स॒ह पृ॑थि॒व्या । \newline
55. पृ॒थि॒व्येतीति॑ पृथि॒व्या पृ॑थि॒व्येति॑ । \newline
56. इत्या॑हा॒हे तीत्या॑ह । \newline
57. आ॒हे॒ य मि॒य मा॑हाहे॒ यम् । \newline

\textbf{Ghana Paata } \newline

1. मनुः॑ पृथि॒व्याः पृ॑थि॒व्या मनु॒र् मनुः॑ पृथि॒व्या य॒ज्ञियं॑ ॅय॒ज्ञिय॑म् पृथि॒व्या मनु॒र् मनुः॑ पृथि॒व्या य॒ज्ञिय᳚म् । \newline
2. पृ॒थि॒व्या य॒ज्ञियं॑ ॅय॒ज्ञिय॑म् पृथि॒व्याः पृ॑थि॒व्या य॒ज्ञिय॑ मैच्छ दैच्छद् य॒ज्ञिय॑म् पृथि॒व्याः पृ॑थि॒व्या य॒ज्ञिय॑ मैच्छत् । \newline
3. य॒ज्ञिय॑ मैच्छ दैच्छद् य॒ज्ञियं॑ ॅय॒ज्ञिय॑ मैच्छ॒थ् स स ऐ᳚च्छद् य॒ज्ञियं॑ ॅय॒ज्ञिय॑ मैच्छ॒थ् सः । \newline
4. ऐ॒च्छ॒थ् स स ऐ᳚च्छ दैच्छ॒थ् स घृ॒तम् घृ॒तꣳ स ऐ᳚च्छ दैच्छ॒थ् स घृ॒तम् । \newline
5. स घृ॒तम् घृ॒तꣳ स स घृ॒तम् निषि॑क्त॒म् निषि॑क्तम् घृ॒तꣳ स स घृ॒तम् निषि॑क्तम् । \newline
6. घृ॒तम् निषि॑क्त॒म् निषि॑क्तम् घृ॒तम् घृ॒तम् निषि॑क्त मविन्द दविन्द॒न् निषि॑क्तम् घृ॒तम् घृ॒तम् निषि॑क्त मविन्दत् । \newline
7. निषि॑क्त मविन् ददविन्द॒न् निषि॑क्त॒म् निषि॑क्त मविन्द॒थ् स सो॑ ऽविन्द॒न् निषि॑क्त॒म् निषि॑क्त मविन्द॒थ् सः । \newline
8. निषि॑क्त॒मिति॒ नि - सि॒क्त॒म् । \newline
9. अ॒वि॒न्द॒थ् स सो॑ ऽविन्द दविन्द॒थ् सो᳚ ऽब्रवी दब्रवी॒थ् सो॑ ऽविन्द दविन्द॒थ् सो᳚ ऽब्रवीत् । \newline
10. सो᳚ ऽब्रवी दब्रवी॒थ् स सो᳚ ऽब्रवी॒त् कः को᳚ ऽब्रवी॒थ् स सो᳚ ऽब्रवी॒त् कः । \newline
11. अ॒ब्र॒वी॒त् कः को᳚ ऽब्रवी दब्रवी॒त् को᳚ ऽस्यास्य को᳚ ऽब्रवी दब्रवी॒त् को᳚ ऽस्य । \newline
12. को᳚ ऽस्यास्य कः को᳚ ऽस्येश्व॒र ई᳚श्व॒रो᳚ ऽस्य कः को᳚ ऽस्येश्व॒रः । \newline
13. अ॒स्येश्व॒र ई᳚श्व॒रो᳚ ऽस्यास्येश्व॒रो य॒ज्ञे य॒ज्ञ् ई᳚श्व॒रो᳚ ऽस्यास्येश्व॒रो य॒ज्ञे । \newline
14. ई॒श्व॒रो य॒ज्ञे य॒ज्ञ् ई᳚श्व॒र ई᳚श्व॒रो य॒ज्ञे ऽप्यपि॑ य॒ज्ञ् ई᳚श्व॒र ई᳚श्व॒रो य॒ज्ञे ऽपि॑ । \newline
15. य॒ज्ञे ऽप्यपि॑ य॒ज्ञे य॒ज्ञे ऽपि॒ कर्तोः॒ कर्तो॒ रपि॑ य॒ज्ञे य॒ज्ञे ऽपि॒ कर्तोः᳚ । \newline
16. अपि॒ कर्तोः॒ कर्तो॒ रप्यपि॒ कर्तो॒ रितीति॒ कर्तो॒ रप्यपि॒ कर्तो॒ रिति॑ । \newline
17. कर्तो॒ रितीति॒ कर्तोः॒ कर्तो॒ रिति॒ तौ ता विति॒ कर्तोः॒ कर्तो॒ रिति॒ तौ । \newline
18. इति॒ तौ ता वितीति॒ ता व॑ब्रूता मब्रूता॒म् ता वितीति॒ ता व॑ब्रूताम् । \newline
19. ता व॑ब्रूता मब्रूता॒म् तौ ता व॑ब्रूताम् मि॒त्रावरु॑णौ मि॒त्रावरु॑णा वब्रूता॒म् तौ ता व॑ब्रूताम् मि॒त्रावरु॑णौ । \newline
20. अ॒ब्रू॒ता॒म् मि॒त्रावरु॑णौ मि॒त्रावरु॑णा वब्रूता मब्रूताम् मि॒त्रावरु॑णौ॒ गोर् गोर् मि॒त्रावरु॑णा वब्रूता मब्रूताम् मि॒त्रावरु॑णौ॒ गोः । \newline
21. मि॒त्रावरु॑णौ॒ गोर् गोर् मि॒त्रावरु॑णौ मि॒त्रावरु॑णौ॒ गोरे॒वैव गोर् मि॒त्रावरु॑णौ मि॒त्रावरु॑णौ॒ गोरे॒व । \newline
22. मि॒त्रावरु॑णा॒विति॑ मि॒त्रा - वरु॑णौ । \newline
23. गोरे॒वैव गोर् गोरे॒वाव मा॒व मे॒व गोर् गोरे॒वावम् । \newline
24. ए॒वाव मा॒व मे॒वैवाव मी᳚श्व॒रा वी᳚श्व॒रा वा॒व मे॒वैवाव मी᳚श्व॒रौ । \newline
25. आ॒व मी᳚श्व॒रा वी᳚श्व॒रा वा॒व मा॒व मी᳚श्व॒रौ कर्तोः॒ कर्तो॑ रीश्व॒रा वा॒व मा॒व मी᳚श्व॒रौ कर्तोः᳚ । \newline
26. ई॒श्व॒रौ कर्तोः॒ कर्तो॑ रीश्व॒रा वी᳚श्व॒रौ कर्तोः᳚ स्वः स्वः॒ कर्तो॑ रीश्व॒रा वी᳚श्व॒रौ कर्तोः᳚ स्वः । \newline
27. कर्तोः᳚ स्वः स्वः॒ कर्तोः॒ कर्तोः᳚ स्व॒ इतीति॑ स्वः॒ कर्तोः॒ कर्तोः᳚ स्व॒ इति॑ । \newline
28. स्व॒ इतीति॑ स्वः स्व॒ इति॒ तौ ता विति॑ स्वः स्व॒ इति॒ तौ । \newline
29. इति॒ तौ ता वितीति॒ तौ तत॒ स्तत॒ स्ता वितीति॒ तौ ततः॑ । \newline
30. तौ तत॒ स्तत॒ स्तौ तौ ततो॒ गाम् गाम् तत॒ स्तौ तौ ततो॒ गाम् । \newline
31. ततो॒ गाम् गाम् तत॒ स्ततो॒ गाꣳ सꣳ सम् गाम् तत॒ स्ततो॒ गाꣳ सम् । \newline
32. गाꣳ सꣳ सम् गाम् गाꣳ स मै॑रयता मैरयताꣳ॒॒ सम् गाम् गाꣳ स मै॑रयताम् । \newline
33. स मै॑रयता मैरयताꣳ॒॒ सꣳ स मै॑रयताꣳ॒॒ सा सैर॑यताꣳ॒॒ सꣳ स मै॑रयताꣳ॒॒ सा । \newline
34. ऐ॒र॒य॒ताꣳ॒॒ सा सैर॑यता मैरयताꣳ॒॒ सा यत्र॑यत्र॒ यत्र॑यत्र॒ सैर॑यता मैरयताꣳ॒॒ सा यत्र॑यत्र । \newline
35. सा यत्र॑यत्र॒ यत्र॑यत्र॒ सा सा यत्र॑यत्र॒ न्यक्रा॑म॒न् न्यक्रा॑म॒द् यत्र॑यत्र॒ सा सा यत्र॑यत्र॒ न्यक्रा॑मत् । \newline
36. यत्र॑यत्र॒ न्यक्रा॑म॒न् न्यक्रा॑म॒द् यत्र॑यत्र॒ यत्र॑यत्र॒ न्यक्रा॑म॒त् तत॒ स्ततो॒ न्यक्रा॑म॒द् यत्र॑यत्र॒ यत्र॑यत्र॒ न्यक्रा॑म॒त् ततः॑ । \newline
37. यत्र॑य॒त्रेति॒ यत्र॑ - य॒त्र॒ । \newline
38. न्यक्रा॑म॒त् तत॒ स्ततो॒ न्यक्रा॑म॒न् न्यक्रा॑म॒त् ततो॑ घृ॒तम् घृ॒तम् ततो॒ न्यक्रा॑म॒न् न्यक्रा॑म॒त् ततो॑ घृ॒तम् । \newline
39. न्यक्रा॑म॒दिति॑ नि - अक्रा॑मत् । \newline
40. ततो॑ घृ॒तम् घृ॒तम् तत॒ स्ततो॑ घृ॒त म॑पीड्यता पीड्यत घृ॒तम् तत॒ स्ततो॑ घृ॒त म॑पीड्यत । \newline
41. घृ॒त म॑पीड्यता पीड्यत घृ॒तम् घृ॒त म॑पीड्यत॒ तस्मा॒त् तस्मा॑ दपीड्यत घृ॒तम् घृ॒त म॑पीड्यत॒ तस्मा᳚त् । \newline
42. अ॒पी॒ड्य॒त॒ तस्मा॒त् तस्मा॑ दपीड्यता पीड्यत॒ तस्मा᳚द् घृ॒तप॑दी घृ॒तप॑दी॒ तस्मा॑ दपीड्यता पीड्यत॒ तस्मा᳚द् घृ॒तप॑दी । \newline
43. तस्मा᳚द् घृ॒तप॑दी घृ॒तप॑दी॒ तस्मा॒त् तस्मा᳚द् घृ॒तप॑द्युच्यत उच्यते घृ॒तप॑दी॒ तस्मा॒त् तस्मा᳚द् घृ॒तप॑द्युच्यते । \newline
44. घृ॒तप॑द्युच्यत उच्यते घृ॒तप॑दी घृ॒तप॑द्युच्यते॒ तत् तदु॑च्यते घृ॒तप॑दी घृ॒तप॑द्युच्यते॒ तत् । \newline
45. घृ॒तप॒दीति॑ घृ॒त - प॒दी॒ । \newline
46. उ॒च्य॒ते॒ तत् तदु॑च्यत उच्यते॒ तद॑स्या अस्यै॒ तदु॑च्यत उच्यते॒ तद॑स्यै । \newline
47. तद॑स्या अस्यै॒ तत् तद॑स्यै॒ जन्म॒ जन्मा᳚स्यै॒ तत् तद॑स्यै॒ जन्म॑ । \newline
48. अ॒स्यै॒ जन्म॒ जन्मा᳚स्या अस्यै॒ जन्मोप॑हूत॒ मुप॑हूत॒म् जन्मा᳚स्या अस्यै॒ जन्मोप॑हूतम् । \newline
49. जन्मोप॑हूत॒ मुप॑हूत॒म् जन्म॒ जन्मोप॑हूतꣳ रथन्त॒रꣳ र॑थन्त॒र मुप॑हूत॒म् जन्म॒ जन्मोप॑हूतꣳ रथन्त॒रम् । \newline
50. उप॑हूतꣳ रथन्त॒रꣳ र॑थन्त॒र मुप॑हूत॒ मुप॑हूतꣳ रथन्त॒रꣳ स॒ह स॒ह र॑थन्त॒र मुप॑हूत॒ मुप॑हूतꣳ रथन्त॒रꣳ स॒ह । \newline
51. उप॑हूत॒मित्युप॑ - हू॒त॒म् । \newline
52. र॒थ॒न्त॒रꣳ स॒ह स॒ह र॑थन्त॒रꣳ र॑थन्त॒रꣳ स॒ह पृ॑थि॒व्या पृ॑थि॒व्या स॒ह र॑थन्त॒रꣳ र॑थन्त॒रꣳ स॒ह पृ॑थि॒व्या । \newline
53. र॒थ॒न्त॒रमिति॑ रथं - त॒रम् । \newline
54. स॒ह पृ॑थि॒व्या पृ॑थि॒व्या स॒ह स॒ह पृ॑थि॒व्येतीति॑ पृथि॒व्या स॒ह स॒ह पृ॑थि॒व्येति॑ । \newline
55. पृ॒थि॒व्येतीति॑ पृथि॒व्या पृ॑थि॒व्ये त्या॑हा॒हे ति॑ पृथि॒व्या पृ॑थि॒व्ये त्या॑ह । \newline
56. इत्या॑हा॒हे तीत्या॑हे॒ य मि॒य मा॒हे तीत्या॑हे॒ यम् । \newline
57. आ॒हे॒ य मि॒य मा॑हाहे॒ यं ॅवै वा इ॒य मा॑हाहे॒ यं ॅवै । \newline
\pagebreak
\markright{ TS 2.6.7.2  \hfill https://www.vedavms.in \hfill}
\addcontentsline{toc}{section}{ TS 2.6.7.2 }
\section*{ TS 2.6.7.2 }

\textbf{TS 2.6.7.2 } \newline
\textbf{Samhita Paata} \newline

-यं ॅवै र॑थन्त॒रमि॒मामे॒व स॒हान्नाद्-ये॒नोप॑ ह्वयत॒ उप॑हूतं ॅवामदे॒व्यꣳ स॒हान्तरि॑क्षे॒णेत्या॑ह प॒शवो॒ वै वा॑मदे॒व्यं प॒शूने॒व स॒हान्तरि॑क्षे॒णोप॑ ह्वयत॒ उप॑हूतं बृ॒हथ् स॒ह दि॒वेत्या॑है॒रं ॅवै बृ॒हदिरा॑मे॒व स॒ह दि॒वोप॑ ह्वयत॒ उप॑हूताः स॒प्त होत्रा॒ इत्या॑ह॒ होत्रा॑ ए॒वोप॑ ह्वयत॒ उप॑हूता धे॒नुः-  [  ] \newline

\textbf{Pada Paata} \newline

इ॒यम् । वै । र॒थ॒न्त॒रमिति॑ रथं - त॒रम् । इ॒माम् । ए॒व । स॒ह । अ॒न्नाद्ये॒नेत्य॑न्न - अद्ये॑न । उपेति॑ । ह्व॒य॒ते॒ । उप॑हूत॒मित्युप॑ - हू॒त॒म् । वा॒म॒दे॒व्यमिति॑ वाम - दे॒व्यम् । स॒ह । अ॒न्तरि॑क्षेण । इति॑ । आ॒ह॒ । प॒शवः॑ । वै । वा॒म॒दे॒व्यमिति॑ वाम - दे॒व्यम् । प॒शून् । ए॒व । स॒ह । अ॒न्तरि॑क्षेण । उपेति॑ । ह्व॒य॒ते॒ । उप॑हूत॒मित्युप॑-हू॒त॒म् । बृ॒हत् । स॒ह । दि॒वा । इति॑ । आ॒ह॒ । ऐ॒रम् । वै । बृ॒हत् । इरा᳚म् । ए॒व । स॒ह । दि॒वा । उपेति॑ । ह्व॒य॒ते॒ । उप॑हूता॒ इत्युप॑ - हू॒ताः॒ । स॒प्त । होत्राः᳚ । इति॑ । आ॒ह॒ । होत्राः᳚ । ए॒व । उपेति॑ । ह्व॒य॒ते॒ । उप॑हू॒तेत्युप॑-हू॒ता॒ । धे॒नुः ।  \newline


\textbf{Krama Paata} \newline

इ॒यं ॅवै । वै र॑थन्त॒रम् । र॒थ॒न्त॒रमि॒माम् । र॒थ॒न्त॒रमिति॑ रथम् - त॒रम् । इ॒मामे॒व । ए॒व स॒ह । स॒हान्नाद्ये॑न । अ॒न्नाद्ये॒नोप॑ । अ॒न्नाद्ये॒नेत्य॑न्न - अद्ये॑न । उप॑ ह्वयते । ह्व॒य॒त॒ उप॑हूतम् । उप॑हूतं ॅवामदे॒व्यम् । उप॑हूत॒मित्युप॑ - हू॒त॒म् । वा॒म॒दे॒व्यꣳ स॒ह । वा॒म॒दे॒व्यमिति॑ वाम - दे॒व्यम् । स॒हान्तरि॑क्षेण । अ॒न्तरि॑क्षे॒णेति॑ । इत्या॑ह । आ॒ह॒ प॒शवः॑ । प॒शवो॒ वै । वै वा॑मदे॒व्यम् । वा॒म॒दे॒व्यम् प॒शून् । वा॒म॒दे॒व्यमिति॑ वाम - दे॒व्यम् । प॒शूने॒व । ए॒व स॒ह । स॒हान्तरि॑क्षेण । अ॒न्तरि॑क्षे॒णोप॑ । उप॑ ह्वयते । ह्व॒य॒त॒ उप॑हूतम् । उप॑हूतम् बृ॒हत् । उप॑हूत॒मित्युप॑ - हू॒त॒म् । बृ॒हथ् स॒ह । स॒ह दि॒वा । दि॒वेति॑ । इत्या॑ह । आ॒है॒रम् । ऐ॒रं ॅवै । वै बृ॒हत् । बृ॒हदिरा᳚म् । इरा॑मे॒व । ए॒व स॒ह । स॒ह दि॒वा । दि॒वोप॑ । उप॑ ह्वयते । ह्व॒य॒त॒ उप॑हूताः । उप॑हूताः स॒प्त । उप॑हूता॒ इत्युप॑ - हू॒ताः॒ । स॒प्त होत्राः᳚ । होत्रा॒ इति॑ । इत्या॑ह । आ॒ह॒ होत्राः᳚ । होत्रा॑ ए॒व । ए॒वोप॑ । उप॑ ह्वयते । ह्व॒य॒त॒ उप॑हूता । उप॑हूता धे॒नुः । उप॑हू॒तेत्युप॑ - हू॒ता॒ । धे॒नुः स॒हर्.ष॑भा \newline

\textbf{Jatai Paata} \newline

1. इ॒यं ॅवै वा इ॒य मि॒यं ॅवै । \newline
2. वै र॑थन्त॒रꣳ र॑थन्त॒रं ॅवै वै र॑थन्त॒रम् । \newline
3. र॒थ॒न्त॒र मि॒मा मि॒माꣳ र॑थन्त॒रꣳ र॑थन्त॒र मि॒माम् । \newline
4. र॒थ॒न्त॒रमिति॑ रथं - त॒रम् । \newline
5. इ॒मा मे॒वैवे मा मि॒मा मे॒व । \newline
6. ए॒व स॒ह स॒हैवैव स॒ह । \newline
7. स॒हा न्नाद्ये॑ना॒ न्नाद्ये॑न स॒ह स॒हा न्नाद्ये॑न । \newline
8. अ॒न्नाद्ये॒ नोपोपा॒ न्नाद्ये॑ना॒ न्नाद्ये॒नोप॑ । \newline
9. अ॒न्नाद्ये॒नेत्य॑न्न - अद्ये॑न । \newline
10. उप॑ ह्वयते ह्वयत॒ उपोप॑ ह्वयते । \newline
11. ह्व॒य॒त॒ उप॑हूत॒ मुप॑हूतꣳ ह्वयते ह्वयत॒ उप॑हूतम् । \newline
12. उप॑हूतं ॅवामदे॒व्यं ॅवा॑मदे॒व्य मुप॑हूत॒ मुप॑हूतं ॅवामदे॒व्यम् । \newline
13. उप॑हूत॒मित्युप॑ - हू॒त॒म् । \newline
14. वा॒म॒दे॒व्यꣳ स॒ह स॒ह वा॑मदे॒व्यं ॅवा॑मदे॒व्यꣳ स॒ह । \newline
15. वा॒म॒दे॒व्यमिति॑ वाम - दे॒व्यम् । \newline
16. स॒हान्त रि॑क्षेणा॒ न्तरि॑क्षेण स॒ह स॒हा न्तरि॑क्षेण । \newline
17. अ॒न्तरि॑क्षे॒णे तीत्य॒न्त रि॑क्षेणा॒ न्तरि॑क्षे॒णे ति॑ । \newline
18. इत्या॑हा॒हे तीत्या॑ह । \newline
19. आ॒ह॒ प॒शवः॑ प॒शव॑ आहाह प॒शवः॑ । \newline
20. प॒शवो॒ वै वै प॒शवः॑ प॒शवो॒ वै । \newline
21. वै वा॑मदे॒व्यं ॅवा॑मदे॒व्यं ॅवै वै वा॑मदे॒व्यम् । \newline
22. वा॒म॒दे॒व्यम् प॒शून् प॒शून्. वा॑मदे॒व्यं ॅवा॑मदे॒व्यम् प॒शून् । \newline
23. वा॒म॒दे॒व्यमिति॑ वाम - दे॒व्यम् । \newline
24. प॒शू ने॒वैव प॒शून् प॒शू ने॒व । \newline
25. ए॒व स॒ह स॒हैवैव स॒ह । \newline
26. स॒हा न्तरि॑क्षेणा॒ न्तरि॑क्षेण स॒ह स॒हा न्तरि॑क्षेण । \newline
27. अ॒न्तरि॑क्षे॒ णोपोपा॒ न्तरि॑क्षेणा॒ न्तरि॑क्षे॒णोप॑ । \newline
28. उप॑ ह्वयते ह्वयत॒ उपोप॑ ह्वयते । \newline
29. ह्व॒य॒त॒ उप॑हूत॒ मुप॑हूतꣳ ह्वयते ह्वयत॒ उप॑हूतम् । \newline
30. उप॑हूतम् बृ॒हद् बृ॒ह दुप॑हूत॒ मुप॑हूतम् बृ॒हत् । \newline
31. उप॑हूत॒मित्युप॑ - हू॒त॒म् । \newline
32. बृ॒हथ् स॒ह स॒ह बृ॒हद् बृ॒हथ् स॒ह । \newline
33. स॒ह दि॒वा दि॒वा स॒ह स॒ह दि॒वा । \newline
34. दि॒वेतीति॑ दि॒वा दि॒वेति॑ । \newline
35. इत्या॑हा॒हे तीत्या॑ह । \newline
36. आ॒है॒र मै॒र मा॑हाहै॒रम् । \newline
37. ऐ॒रं ॅवै वा ऐ॒र मै॒रं ॅवै । \newline
38. वै बृ॒हद् बृ॒हद् वै वै बृ॒हत् । \newline
39. बृ॒हदिरा॒ मिरा᳚म् बृ॒हद् बृ॒हदिरा᳚म् । \newline
40. इरा॑ मे॒वैवे रा॒ मिरा॑ मे॒व । \newline
41. ए॒व स॒ह स॒हैवैव स॒ह । \newline
42. स॒ह दि॒वा दि॒वा स॒ह स॒ह दि॒वा । \newline
43. दि॒वोपोप॑ दि॒वा दि॒वोप॑ । \newline
44. उप॑ ह्वयते ह्वयत॒ उपोप॑ ह्वयते । \newline
45. ह्व॒य॒त॒ उप॑हूता॒ उप॑हूता ह्वयते ह्वयत॒ उप॑हूताः । \newline
46. उप॑हूताः स॒प्त स॒प्तो प॑हूता॒ उप॑हूताः स॒प्त । \newline
47. उप॑हूता॒ इत्युप॑ - हू॒ताः॒ । \newline
48. स॒प्त होत्रा॒ होत्राः᳚ स॒प्त स॒प्त होत्राः᳚ । \newline
49. होत्रा॒ इतीति॒ होत्रा॒ होत्रा॒ इति॑ । \newline
50. इत्या॑हा॒हे तीत्या॑ह । \newline
51. आ॒ह॒ होत्रा॒ होत्रा॑ आहाह॒ होत्राः᳚ । \newline
52. होत्रा॑ ए॒वैव होत्रा॒ होत्रा॑ ए॒व । \newline
53. ए॒वोपो पै॒वैवोप॑ । \newline
54. उप॑ ह्वयते ह्वयत॒ उपोप॑ ह्वयते । \newline
55. ह्व॒य॒त॒ उप॑हू॒तो प॑हूता ह्वयते ह्वयत॒ उप॑हूता । \newline
56. उप॑हूता धे॒नुर् धे॒नु रुप॑हू॒तो प॑हूता धे॒नुः । \newline
57. उप॑हू॒तेत्युप॑ - हू॒ता॒ । \newline
58. धे॒नुः स॒हर्.ष॑भा स॒हर्.ष॑भा धे॒नुर् धे॒नुः स॒हर्.ष॑भा । \newline

\textbf{Ghana Paata } \newline

1. इ॒यं ॅवै वा इ॒य मि॒यं ॅवै र॑थन्त॒रꣳ र॑थन्त॒रं ॅवा इ॒य मि॒यं ॅवै र॑थन्त॒रम् । \newline
2. वै र॑थन्त॒रꣳ र॑थन्त॒रं ॅवै वै र॑थन्त॒र मि॒मा मि॒माꣳ र॑थन्त॒रं ॅवै वै र॑थन्त॒र मि॒माम् । \newline
3. र॒थ॒न्त॒र मि॒मा मि॒माꣳ र॑थन्त॒रꣳ र॑थन्त॒र मि॒मा मे॒वैवे माꣳ र॑थन्त॒रꣳ र॑थन्त॒र मि॒मा मे॒व । \newline
4. र॒थ॒न्त॒रमिति॑ रथं - त॒रम् । \newline
5. इ॒मा मे॒वैवे मा मि॒मा मे॒व स॒ह स॒हैवे मा मि॒मा मे॒व स॒ह । \newline
6. ए॒व स॒ह स॒हैवैव स॒हा न्नाद्ये॑ना॒ न्नाद्ये॑न स॒हैवैव स॒हा न्नाद्ये॑न । \newline
7. स॒हा न्नाद्ये॑ना॒ न्नाद्ये॑न स॒ह स॒हा न्नाद्ये॒नोपोपा॒ न्नाद्ये॑न स॒ह स॒हा न्नाद्ये॒नोप॑ । \newline
8. अ॒न्नाद्ये॒नोपोपा॒ न्नाद्ये॑ना॒ न्नाद्ये॒नोप॑ ह्वयते ह्वयत॒ उपा॒न्नाद्ये॑ना॒ न्नाद्ये॒नोप॑ ह्वयते । \newline
9. अ॒न्नाद्ये॒नेत्य॑न्न - अद्ये॑न । \newline
10. उप॑ ह्वयते ह्वयत॒ उपोप॑ ह्वयत॒ उप॑हूत॒ मुप॑हूतꣳ ह्वयत॒ उपोप॑ ह्वयत॒ उप॑हूतम् । \newline
11. ह्व॒य॒त॒ उप॑हूत॒ मुप॑हूतꣳ ह्वयते ह्वयत॒ उप॑हूतं ॅवामदे॒व्यं ॅवा॑मदे॒व्य मुप॑हूतꣳ ह्वयते ह्वयत॒ उप॑हूतं ॅवामदे॒व्यम् । \newline
12. उप॑हूतं ॅवामदे॒व्यं ॅवा॑मदे॒व्य मुप॑हूत॒ मुप॑हूतं ॅवामदे॒व्यꣳ स॒ह स॒ह वा॑मदे॒व्य मुप॑हूत॒ मुप॑हूतं ॅवामदे॒व्यꣳ स॒ह । \newline
13. उप॑हूत॒मित्युप॑ - हू॒त॒म् । \newline
14. वा॒म॒दे॒व्यꣳ स॒ह स॒ह वा॑मदे॒व्यं ॅवा॑मदे॒व्यꣳ स॒हा न्तरि॑क्षेणा॒ न्तरि॑क्षेण स॒ह वा॑मदे॒व्यं ॅवा॑मदे॒व्यꣳ स॒हान्तरि॑क्षेण । \newline
15. वा॒म॒दे॒व्यमिति॑ वाम - दे॒व्यम् । \newline
16. स॒हान्तरि॑क्षेणा॒ न्तरि॑क्षेण स॒ह स॒हा न्तरि॑क्षे॒णे तीत्य॒न्तरि॑क्षेण स॒ह स॒हा न्तरि॑क्षे॒णे ति॑ । \newline
17. अ॒न्तरि॑क्षे॒णे तीत्य॒न्तरि॑क्षेणा॒ न्तरि॑क्षे॒णे त्या॑हा॒हे त्य॒न्तरि॑क्षेणा॒ न्तरि॑क्षे॒णे त्या॑ह । \newline
18. इत्या॑हा॒हे तीत्या॑ह प॒शवः॑ प॒शव॑ आ॒हे तीत्या॑ह प॒शवः॑ । \newline
19. आ॒ह॒ प॒शवः॑ प॒शव॑ आहाह प॒शवो॒ वै वै प॒शव॑ आहाह प॒शवो॒ वै । \newline
20. प॒शवो॒ वै वै प॒शवः॑ प॒शवो॒ वै वा॑मदे॒व्यं ॅवा॑मदे॒व्यं ॅवै प॒शवः॑ प॒शवो॒ वै वा॑मदे॒व्यम् । \newline
21. वै वा॑मदे॒व्यं ॅवा॑मदे॒व्यं ॅवै वै वा॑मदे॒व्यम् प॒शून् प॒शून्. वा॑मदे॒व्यं ॅवै वै वा॑मदे॒व्यम् प॒शून् । \newline
22. वा॒म॒दे॒व्यम् प॒शून् प॒शून्. वा॑मदे॒व्यं ॅवा॑मदे॒व्यम् प॒शू ने॒वैव प॒शून्. वा॑मदे॒व्यं ॅवा॑मदे॒व्यम् प॒शू ने॒व । \newline
23. वा॒म॒दे॒व्यमिति॑ वाम - दे॒व्यम् । \newline
24. प॒शू ने॒वैव प॒शून् प॒शू ने॒व स॒ह स॒हैव प॒शून् प॒शू ने॒व स॒ह । \newline
25. ए॒व स॒ह स॒हैवैव स॒हा न्तरि॑क्षेणा॒ न्तरि॑क्षेण स॒हैवैव स॒हा न्तरि॑क्षेण । \newline
26. स॒हा न्तरि॑क्षेणा॒ न्तरि॑क्षेण स॒ह स॒हा न्तरि॑क्षे॒णोपोपा॒ न्तरि॑क्षेण स॒ह स॒हा न्तरि॑क्षे॒णोप॑ । \newline
27. अ॒न्तरि॑क्षे॒णोपोपा॒ न्तरि॑क्षेणा॒ न्तरि॑क्षे॒णोप॑ ह्वयते ह्वयत॒ उपा॒न्तरि॑क्षेणा॒ न्तरि॑क्षे॒णोप॑ ह्वयते । \newline
28. उप॑ ह्वयते ह्वयत॒ उपोप॑ ह्वयत॒ उप॑हूत॒ मुप॑हूतꣳ ह्वयत॒ उपोप॑ ह्वयत॒ उप॑हूतम् । \newline
29. ह्व॒य॒त॒ उप॑हूत॒ मुप॑हूतꣳ ह्वयते ह्वयत॒ उप॑हूतम् बृ॒हद् बृ॒ह दुप॑हूतꣳ ह्वयते ह्वयत॒ उप॑हूतम् बृ॒हत् । \newline
30. उप॑हूतम् बृ॒हद् बृ॒ह दुप॑हूत॒ मुप॑हूतम् बृ॒हथ् स॒ह स॒ह बृ॒ह दुप॑हूत॒ मुप॑हूतम् बृ॒हथ् स॒ह । \newline
31. उप॑हूत॒मित्युप॑ - हू॒त॒म् । \newline
32. बृ॒हथ् स॒ह स॒ह बृ॒हद् बृ॒हथ् स॒ह दि॒वा दि॒वा स॒ह बृ॒हद् बृ॒हथ् स॒ह दि॒वा । \newline
33. स॒ह दि॒वा दि॒वा स॒ह स॒ह दि॒वेतीति॑ दि॒वा स॒ह स॒ह दि॒वेति॑ । \newline
34. दि॒वेतीति॑ दि॒वा दि॒वे त्या॑हा॒हे ति॑ दि॒वा दि॒वेत्या॑ह । \newline
35. इत्या॑हा॒हे तीत्या॑है॒र मै॒र मा॒हे तीत्या॑है॒रम् । \newline
36. आ॒है॒र मै॒र मा॑हाहै॒रं ॅवै वा ऐ॒र मा॑हाहै॒रं ॅवै । \newline
37. ऐ॒रं ॅवै वा ऐ॒र मै॒रं ॅवै बृ॒हद् बृ॒हद् वा ऐ॒र मै॒रं ॅवै बृ॒हत् । \newline
38. वै बृ॒हद् बृ॒हद् वै वै बृ॒हदिरा॒ मिरा᳚म् बृ॒हद् वै वै बृ॒हदिरा᳚म् । \newline
39. बृ॒हदिरा॒ मिरा᳚म् बृ॒हद् बृ॒हदिरा॑ मे॒वैवे रा᳚म् बृ॒हद् बृ॒हदिरा॑ मे॒व । \newline
40. इरा॑ मे॒वैवे रा॒ मिरा॑ मे॒व स॒ह स॒हैवे रा॒ मिरा॑ मे॒व स॒ह । \newline
41. ए॒व स॒ह स॒हैवैव स॒ह दि॒वा दि॒वा स॒हैवैव स॒ह दि॒वा । \newline
42. स॒ह दि॒वा दि॒वा स॒ह स॒ह दि॒वोपोप॑ दि॒वा स॒ह स॒ह दि॒वोप॑ । \newline
43. दि॒वोपोप॑ दि॒वा दि॒वोप॑ ह्वयते ह्वयत॒ उप॑ दि॒वा दि॒वोप॑ ह्वयते । \newline
44. उप॑ ह्वयते ह्वयत॒ उपोप॑ ह्वयत॒ उप॑हूता॒ उप॑हूता ह्वयत॒ उपोप॑ ह्वयत॒ उप॑हूताः । \newline
45. ह्व॒य॒त॒ उप॑हूता॒ उप॑हूता ह्वयते ह्वयत॒ उप॑हूताः स॒प्त स॒प्तोप॑हूता ह्वयते ह्वयत॒ उप॑हूताः स॒प्त । \newline
46. उप॑हूताः स॒प्त स॒प्तोप॑हूता॒ उप॑हूताः स॒प्त होत्रा॒ होत्राः᳚ स॒प्तोप॑हूता॒ उप॑हूताः स॒प्त होत्राः᳚ । \newline
47. उप॑हूता॒ इत्युप॑ - हू॒ताः॒ । \newline
48. स॒प्त होत्रा॒ होत्राः᳚ स॒प्त स॒प्त होत्रा॒ इतीति॒ होत्राः᳚ स॒प्त स॒प्त होत्रा॒ इति॑ । \newline
49. होत्रा॒ इतीति॒ होत्रा॒ होत्रा॒ इत्या॑हा॒हे ति॒ होत्रा॒ होत्रा॒ इत्या॑ह । \newline
50. इत्या॑हा॒हे तीत्या॑ह॒ होत्रा॒ होत्रा॑ आ॒हे तीत्या॑ह॒ होत्राः᳚ । \newline
51. आ॒ह॒ होत्रा॒ होत्रा॑ आहाह॒ होत्रा॑ ए॒वैव होत्रा॑ आहाह॒ होत्रा॑ ए॒व । \newline
52. होत्रा॑ ए॒वैव होत्रा॒ होत्रा॑ ए॒वोपोपै॒व होत्रा॒ होत्रा॑ ए॒वोप॑ । \newline
53. ए॒वोपो पै॒वैवोप॑ ह्वयते ह्वयत॒ उपै॒वैवोप॑ ह्वयते । \newline
54. उप॑ ह्वयते ह्वयत॒ उपोप॑ ह्वयत॒ उप॑हू॒तो प॑हूता ह्वयत॒ उपोप॑ ह्वयत॒ उप॑हूता । \newline
55. ह्व॒य॒त॒ उप॑हू॒तो प॑हूता ह्वयते ह्वयत॒ उप॑हूता धे॒नुर् धे॒नु रुप॑हूता ह्वयते ह्वयत॒ उप॑हूता धे॒नुः । \newline
56. उप॑हूता धे॒नुर् धे॒नु रुप॑हू॒तो प॑हूता धे॒नुः स॒हर्.ष॑भा स॒हर्.ष॑भा धे॒नु रुप॑हू॒तो प॑हूता धे॒नुः स॒हर्.ष॑भा । \newline
57. उप॑हू॒तेत्युप॑ - हू॒ता॒ । \newline
58. धे॒नुः स॒हर्.ष॑भा स॒हर्.ष॑भा धे॒नुर् धे॒नुः स॒हर्.ष॒भेतीति॑ स॒हर्.ष॑भा धे॒नुर् धे॒नुः स॒हर्.ष॒भेति॑ । \newline
\pagebreak
\markright{ TS 2.6.7.3  \hfill https://www.vedavms.in \hfill}
\addcontentsline{toc}{section}{ TS 2.6.7.3 }
\section*{ TS 2.6.7.3 }

\textbf{TS 2.6.7.3 } \newline
\textbf{Samhita Paata} \newline

स॒हर्.ष॒भेत्या॑ह मिथु॒नमे॒वोप॑ ह्वयत॒ उप॑हूतो भ॒क्षः सखेत्या॑ह सोमपी॒थमे॒वोप॑ ह्वयत॒ उप॑हू॒ताॅ(4) हो इत्या॑हा॒ऽऽत्मान॑मे॒वोप॑ ह्वयत आ॒त्मा ह्युप॑हूतानां॒ ॅवसि॑ष्ठ॒ इडा॒मुप॑ ह्वयते प॒शवो॒ वा इडा॑ प॒शूने॒वोप॑ ह्वयते च॒तुरुप॑ ह्वयते॒ चतु॑ष्पादो॒ हि प॒शवो॑ मान॒वीत्या॑ह॒ मनु॒र्ह्ये॑ता- [  ] \newline

\textbf{Pada Paata} \newline

स॒हर्.ष॒भेति॑ स॒ह-ऋ॒ष॒भा॒ । इति॑ । आ॒ह॒ । मि॒थु॒नम् । ए॒व । उपेति॑ । ह्व॒य॒ते॒ । उप॑हूत॒ इत्युप॑ - हू॒तः॒ । भ॒क्षः । सखा᳚ । इति॑ । आ॒ह॒ । सो॒म॒पी॒थमिति॑ सोम - पी॒थम् । ए॒व । उपेति॑ । ह्व॒य॒ते॒ । उप॑हू॒ताॅ(4)इत्युप॑ - हू॒ता(3)ॅ । हो इति॑ । इति॑ । आ॒ह॒ । आ॒त्मान᳚म् । ए॒व । उपेति॑ । ह्व॒य॒ते॒ । आ॒त्मा । हि । उप॑हूताना॒मित्युप॑ - हू॒ता॒ना॒म् । वसि॑ष्ठः । इडा᳚म् । उपेति॑ । ह्व॒य॒ते॒ । प॒शवः॑ । वै । इडा᳚ । प॒शून् । ए॒व । उपेति॑ । ह्व॒य॒ते॒ । च॒तुः । उपेति॑ । ह्व॒य॒ते॒ । चतु॑ष्पाद॒ इति॒ चतुः॑-पा॒दः॒ । हि । प॒शवः॑ । मा॒न॒वी । इति॑ । आ॒ह॒ । मनुः॑ । हि । ए॒ताम् ।  \newline


\textbf{Krama Paata} \newline

स॒हर्.ष॒भेति॑ । स॒हर्.ष॒भेति॑ स॒ह - ऋ॒ष॒भा॒ । इत्या॑ह । आ॒ह॒ मि॒थु॒नम् । मि॒थु॒नमे॒व । ए॒वोप॑ । उप॑ ह्वयते । ह्व॒य॒त॒ उप॑हूतः । उप॑हूतो भ॒क्षः । उप॑हूत॒ इत्युप॑ - हू॒तः॒ । भ॒क्षः सखा᳚ । सखेति॑ । इत्या॑ह । आ॒ह॒ सो॒म॒पी॒थम् । सो॒म॒पी॒थमे॒व । सो॒म॒पी॒थमिति॑ सोम - पी॒थम् । ए॒वोप॑ । उप॑ ह्वयते । ह्व॒य॒त॒ उप॑हू॒ता(4)म् । उप॑हू॒ता(4)म् हो । उप॑हू॒ता(4)म् इत्युप॑ - हू॒ता(4)म् । हो इति॑ । हो इति॒ हो । इत्या॑ह । आ॒हा॒त्मान᳚म् । आ॒त्मान॑मे॒व । ए॒वोप॑ । उप॑ ह्वयते । ह्व॒य॒त॒ आ॒त्मा । आ॒त्मा हि । ह्युप॑हूतानाम् । उप॑हूतानां॒ ॅवसि॑ष्ठः । उप॑हूताना॒मित्युप॑ - हू॒ता॒ना॒म् । वसि॑ष्ठ॒ इडा᳚म् । इडा॒मुप॑ । उप॑ ह्वयते । ह्व॒य॒ते॒ प॒शवः॑ । प॒शवो॒ वै । वा इडा᳚ । इडा॑ प॒शून् । प॒शूने॒व । ए॒वोप॑ । उप॑ ह्वयते । ह्व॒य॒ते॒ च॒तुः । च॒तुरुप॑ । उप॑ ह्वयते । ह्व॒य॒ते॒ चतु॑ष्पादः । चतु॑ष्पादो॒ हि । चतु॑ष्पाद॒ इति॒ चतुः॑ - पा॒दः॒ । हि प॒शवः॑ । प॒शवो॑ मान॒वी । मा॒न॒वीति॑ । इत्या॑ह । आ॒ह॒ मनुः॑ । मनु॒र्.॒ हि । ह्ये॑ताम् । ए॒तामग्रे᳚ \newline

\textbf{Jatai Paata} \newline

1. स॒हर्.ष॒भे तीति॑ स॒हर्.ष॑भा स॒हर्.ष॒भेति॑ । \newline
2. स॒हर्.ष॒भेति॑ स॒ह - ऋ॒ष॒भा॒ । \newline
3. इत्या॑हा॒हे तीत्या॑ह । \newline
4. आ॒ह॒ मि॒थु॒नम् मि॑थु॒न मा॑हाह मिथु॒नम् । \newline
5. मि॒थु॒न मे॒वैव मि॑थु॒नम् मि॑थु॒न मे॒व । \newline
6. ए॒वो पोपै॒ वैवोप॑ । \newline
7. उप॑ ह्वयते ह्वयत॒ उपोप॑ ह्वयते । \newline
8. ह्व॒य॒त॒ उप॑हूत॒ उप॑हूतो ह्वयते ह्वयत॒ उप॑हूतः । \newline
9. उप॑हूतो भ॒क्षो भ॒क्ष उप॑हूत॒ उप॑हूतो भ॒क्षः । \newline
10. उप॑हूत॒ इत्युप॑ - हू॒तः॒ । \newline
11. भ॒क्षः सखा॒ सखा॑ भ॒क्षो भ॒क्षः सखा᳚ । \newline
12. सखेतीति॒ सखा॒ सखेति॑ । \newline
13. इत्या॑हा॒हे तीत्या॑ह । \newline
14. आ॒ह॒ सो॒म॒पी॒थꣳ सो॑मपी॒थ मा॑हाह सोमपी॒थम् । \newline
15. सो॒म॒पी॒थ मे॒वैव सो॑मपी॒थꣳ सो॑मपी॒थ मे॒व । \newline
16. सो॒म॒पी॒थमिति॑ सोम - पी॒थम् । \newline
17. ए॒वो पोपै॒ वैवोप॑ । \newline
18. उप॑ ह्वयते ह्वयत॒ उपोप॑ ह्वयते । \newline
19. ह्व॒य॒त॒ उप॑हू॒ताॅ(4) उप॑हू॒ताॅ(4) ह्व॑यते ह्वयत॒ उप॑हू॒ताॅ(4) । \newline
20. उप॑हू॒ताॅ(4) हो हो उप॑हू॒ताॅ(4) उप॑हू॒ताॅ(4) हो । \newline
21. उप॑हू॒ताॅ(4)इत्युप॑ - हू॒ता(3)ॅ । \newline
22. हो इतीति॒ हो हो इति॑ । \newline
23. हो इति॒ हो । \newline
24. इत्या॑हा॒हे तीत्या॑ह । \newline
25. आ॒हा॒ त्मान॑ मा॒त्मान॑ माहा हा॒त्मान᳚म् । \newline
26. आ॒त्मान॑ मे॒वै वात्मान॑ मा॒त्मान॑ मे॒व । \newline
27. ए॒वोपो पै॒वैवोप॑ । \newline
28. उप॑ ह्वयते ह्वयत॒ उपोप॑ ह्वयते । \newline
29. ह्व॒य॒त॒ आ॒त्मा ऽऽत्मा ह्व॑यते ह्वयत आ॒त्मा । \newline
30. आ॒त्मा हि ह्या᳚त्मा ऽऽत्मा हि । \newline
31. ह्युप॑हूताना॒ मुप॑हूतानाꣳ॒॒ हि ह्युप॑हूतानाम् । \newline
32. उप॑हूतानां॒ ॅवसि॑ष्ठो॒ वसि॑ष्ठ॒ उप॑हूताना॒ मुप॑हूतानां॒ ॅवसि॑ष्ठः । \newline
33. उप॑हूताना॒मित्युप॑ - हू॒ता॒ना॒म् । \newline
34. वसि॑ष्ठ॒ इडा॒ मिडां॒ ॅवसि॑ष्ठो॒ वसि॑ष्ठ॒ इडा᳚म् । \newline
35. इडा॒ मुपोपे डा॒ मिडा॒ मुप॑ । \newline
36. उप॑ ह्वयते ह्वयत॒ उपोप॑ ह्वयते । \newline
37. ह्व॒य॒ते॒ प॒शवः॑ प॒शवो᳚ ह्वयते ह्वयते प॒शवः॑ । \newline
38. प॒शवो॒ वै वै प॒शवः॑ प॒शवो॒ वै । \newline
39. वा इडेडा॒ वै वा इडा᳚ । \newline
40. इडा॑ प॒शून् प॒शू निडेडा॑ प॒शून् । \newline
41. प॒शू ने॒वैव प॒शून् प॒शू ने॒व । \newline
42. ए॒वोपो पै॒वैवोप॑ । \newline
43. उप॑ ह्वयते ह्वयत॒ उपोप॑ ह्वयते । \newline
44. ह्व॒य॒ते॒ च॒तु श्च॒तुर् ह्व॑यते ह्वयते च॒तुः । \newline
45. च॒तु रुपोप॑ च॒तु श्च॒तु रुप॑ । \newline
46. उप॑ ह्वयते ह्वयत॒ उपोप॑ ह्वयते । \newline
47. ह्व॒य॒ते॒ चतु॑ष्पाद॒ श्चतु॑ष्पादो ह्वयते ह्वयते॒ चतु॑ष्पादः । \newline
48. चतु॑ष्पादो॒ हि हि चतु॑ष्पाद॒ श्चतु॑ष्पादो॒ हि । \newline
49. चतु॑ष्पाद॒ इति॒ चतुः॑ - पा॒दः॒ । \newline
50. हि प॒शवः॑ प॒शवो॒ हि हि प॒शवः॑ । \newline
51. प॒शवो॑ मान॒वी मा॑न॒वी प॒शवः॑ प॒शवो॑ मान॒वी । \newline
52. मा॒न॒वीतीति॑ मान॒वी मा॑न॒वीति॑ । \newline
53. इत्या॑हा॒हे तीत्या॑ह । \newline
54. आ॒ह॒ मनु॒र् मनु॑ राहाह॒ मनुः॑ । \newline
55. मनु॒र्॒. हि हि मनु॒र् मनु॒र्॒. हि । \newline
56. ह्ये॑ता मे॒ताꣳ हि ह्ये॑ताम् । \newline
57. ए॒ता मग्रे ऽग्र॑ ए॒ता मे॒ता मग्रे᳚ । \newline

\textbf{Ghana Paata } \newline

1. स॒हर्.ष॒भेतीति॑ स॒हर्.ष॑भा स॒हर्.ष॒भे त्या॑हा॒हे ति॑ स॒हर्.ष॑भा स॒हर्.ष॒भेत्या॑ह । \newline
2. स॒हर्.ष॒भेति॑ स॒ह - ऋ॒ष॒भा॒ । \newline
3. इत्या॑हा॒हे तीत्या॑ह मिथु॒नम् मि॑थु॒न मा॒हे तीत्या॑ह मिथु॒नम् । \newline
4. आ॒ह॒ मि॒थु॒नम् मि॑थु॒न मा॑हाह मिथु॒न मे॒वैव मि॑थु॒न मा॑हाह मिथु॒न मे॒व । \newline
5. मि॒थु॒न मे॒वैव मि॑थु॒नम् मि॑थु॒न मे॒वोपोपै॒व मि॑थु॒नम् मि॑थु॒न मे॒वोप॑ । \newline
6. ए॒वोपो पै॒वैवोप॑ ह्वयते ह्वयत॒ उपै॒वै वोप॑ ह्वयते । \newline
7. उप॑ ह्वयते ह्वयत॒ उपोप॑ ह्वयत॒ उप॑हूत॒ उप॑हूतो ह्वयत॒ उपोप॑ ह्वयत॒ उप॑हूतः । \newline
8. ह्व॒य॒त॒ उप॑हूत॒ उप॑हूतो ह्वयते ह्वयत॒ उप॑हूतो भ॒क्षो भ॒क्ष उप॑हूतो ह्वयते ह्वयत॒ उप॑हूतो भ॒क्षः । \newline
9. उप॑हूतो भ॒क्षो भ॒क्ष उप॑हूत॒ उप॑हूतो भ॒क्षः सखा॒ सखा॑ भ॒क्ष उप॑हूत॒ उप॑हूतो भ॒क्षः सखा᳚ । \newline
10. उप॑हूत॒ इत्युप॑ - हू॒तः॒ । \newline
11. भ॒क्षः सखा॒ सखा॑ भ॒क्षो भ॒क्षः सखेतीति॒ सखा॑ भ॒क्षो भ॒क्षः सखेति॑ । \newline
12. सखेतीति॒ सखा॒ सखे त्या॑हा॒हे ति॒ सखा॒ सखे त्या॑ह । \newline
13. इत्या॑हा॒हे तीत्या॑ह सोमपी॒थꣳ सो॑मपी॒थ मा॒हे तीत्या॑ह सोमपी॒थम् । \newline
14. आ॒ह॒ सो॒म॒पी॒थꣳ सो॑मपी॒थ मा॑हाह सोमपी॒थ मे॒वैव सो॑मपी॒थ मा॑हाह सोमपी॒थ मे॒व । \newline
15. सो॒म॒पी॒थ मे॒वैव सो॑मपी॒थꣳ सो॑मपी॒थ मे॒वोपोपै॒व सो॑मपी॒थꣳ सो॑मपी॒थ मे॒वोप॑ । \newline
16. सो॒म॒पी॒थमिति॑ सोम - पी॒थम् । \newline
17. ए॒वोपो पै॒वैवोप॑ ह्वयते ह्वयत॒ उपै॒वै वोप॑ ह्वयते । \newline
18. उप॑ ह्वयते ह्वयत॒ उपोप॑ ह्वयत॒ उप॑हू॒ताॅ(4) उप॑हू॒ताॅ(4) ह्व॑यत॒ उपोप॑ ह्वयत॒ उप॑हू॒ताॅ(4) । \newline
19. ह्व॒य॒त॒ उप॑हू॒ताॅ(4) उप॑हू॒ताॅ(4) ह्व॑यते ह्वयत॒ उप॑हू॒ताॅ(4) हो हो उप॑हू॒ताॅ(4) ह्व॑यते ह्वयत॒ उप॑हू॒ताॅ(4) हो । \newline
20. उप॑हू॒ताॅ(4) हो हो उप॑हू॒ताॅ(4) उप॑हू॒ताॅ(4) हो इतीति॒ हो उप॑हू॒ताॅ(4) उप॑हू॒ताॅ(4) हो इति॑ । \newline
21. उप॑हू॒ताॅ(4)इत्युप॑ - हू॒ता(3)ॅ । \newline
22. हो इतीति॒ हो हो इत्या॑हा॒हे ति॒ हो हो इत्या॑ह । \newline
23. हो इति॒ हो । \newline
24. इत्या॑हा॒हे तीत्या॑हा॒ त्मान॑ मा॒त्मान॑ मा॒हे तीत्या॑ हा॒त्मान᳚म् । \newline
25. आ॒हा॒त्मान॑ मा॒त्मान॑ माहा हा॒त्मान॑ मे॒वैवा त्मान॑ माहा हा॒त्मान॑ मे॒व । \newline
26. आ॒त्मान॑ मे॒वैवा त्मान॑ मा॒त्मान॑ मे॒वोपोपै॒वा त्मान॑ मा॒त्मान॑ मे॒वोप॑ । \newline
27. ए॒वोपो पै॒वैवोप॑ ह्वयते ह्वयत॒ उपै॒वै वोप॑ ह्वयते । \newline
28. उप॑ ह्वयते ह्वयत॒ उपोप॑ ह्वयत आ॒त्मा ऽऽत्मा ह्व॑यत॒ उपोप॑ ह्वयत आ॒त्मा । \newline
29. ह्व॒य॒त॒ आ॒त्मा ऽऽत्मा ह्व॑यते ह्वयत आ॒त्मा हि ह्या᳚त्मा ह्व॑यते ह्वयत आ॒त्मा हि । \newline
30. आ॒त्मा हि ह्या᳚त्मा ऽऽत्मा ह्युप॑हूताना॒ मुप॑हूतानाꣳ॒॒ ह्या᳚त्मा ऽऽत्मा ह्युप॑हूतानाम् । \newline
31. ह्युप॑हूताना॒ मुप॑हूतानाꣳ॒॒ हि ह्युप॑हूतानां॒ ॅवसि॑ष्ठो॒ वसि॑ष्ठ॒ उप॑हूतानाꣳ॒॒ हि ह्युप॑हूतानां॒ ॅवसि॑ष्ठः । \newline
32. उप॑हूतानां॒ ॅवसि॑ष्ठो॒ वसि॑ष्ठ॒ उप॑हूताना॒ मुप॑हूतानां॒ ॅवसि॑ष्ठ॒ इडा॒ मिडां॒ ॅवसि॑ष्ठ॒ उप॑हूताना॒ मुप॑हूतानां॒ ॅवसि॑ष्ठ॒ इडा᳚म् । \newline
33. उप॑हूताना॒मित्युप॑ - हू॒ता॒ना॒म् । \newline
34. वसि॑ष्ठ॒ इडा॒ मिडां॒ ॅवसि॑ष्ठो॒ वसि॑ष्ठ॒ इडा॒ मुपोपे डां॒ ॅवसि॑ष्ठो॒ वसि॑ष्ठ॒ इडा॒ मुप॑ । \newline
35. इडा॒ मुपोपे डा॒ मिडा॒ मुप॑ ह्वयते ह्वयत॒ उपे डा॒ मिडा॒ मुप॑ ह्वयते । \newline
36. उप॑ ह्वयते ह्वयत॒ उपोप॑ ह्वयते प॒शवः॑ प॒शवो᳚ ह्वयत॒ उपोप॑ ह्वयते प॒शवः॑ । \newline
37. ह्व॒य॒ते॒ प॒शवः॑ प॒शवो᳚ ह्वयते ह्वयते प॒शवो॒ वै वै प॒शवो᳚ ह्वयते ह्वयते प॒शवो॒ वै । \newline
38. प॒शवो॒ वै वै प॒शवः॑ प॒शवो॒ वा इडेडा॒ वै प॒शवः॑ प॒शवो॒ वा इडा᳚ । \newline
39. वा इडेडा॒ वै वा इडा॑ प॒शून् प॒शू निडा॒ वै वा इडा॑ प॒शून् । \newline
40. इडा॑ प॒शून् प॒शू निडेडा॑ प॒शू ने॒वैव प॒शू निडेडा॑ प॒शू ने॒व । \newline
41. प॒शू ने॒वैव प॒शून् प॒शू ने॒वोपो पै॒व प॒शून् प॒शू ने॒वोप॑ । \newline
42. ए॒वोपो पै॒वैवोप॑ ह्वयते ह्वयत॒ उपै॒ वैवोप॑ ह्वयते । \newline
43. उप॑ ह्वयते ह्वयत॒ उपोप॑ ह्वयते च॒तु श्च॒तुर् ह्व॑यत॒ उपोप॑ ह्वयते च॒तुः । \newline
44. ह्व॒य॒ते॒ च॒तु श्च॒तुर् ह्व॑यते ह्वयते च॒तु रुपोप॑ च॒तुर् ह्व॑यते ह्वयते च॒तुरुप॑ । \newline
45. च॒तु रुपोप॑ च॒तु श्च॒तुरुप॑ ह्वयते ह्वयत॒ उप॑ च॒तु श्च॒तुरुप॑ ह्वयते । \newline
46. उप॑ ह्वयते ह्वयत॒ उपोप॑ ह्वयते॒ चतु॑ष्पाद॒ श्चतु॑ष्पादो ह्वयत॒ उपोप॑ ह्वयते॒ चतु॑ष्पादः । \newline
47. ह्व॒य॒ते॒ चतु॑ष्पाद॒ श्चतु॑ष्पादो ह्वयते ह्वयते॒ चतु॑ष्पादो॒ हि हि चतु॑ष्पादो ह्वयते ह्वयते॒ चतु॑ष्पादो॒ हि । \newline
48. चतु॑ष्पादो॒ हि हि चतु॑ष्पाद॒ श्चतु॑ष्पादो॒ हि प॒शवः॑ प॒शवो॒ हि चतु॑ष्पाद॒ श्चतु॑ष्पादो॒ हि प॒शवः॑ । \newline
49. चतु॑ष्पाद॒ इति॒ चतुः॑ - पा॒दः॒ । \newline
50. हि प॒शवः॑ प॒शवो॒ हि हि प॒शवो॑ मान॒वी मा॑न॒वी प॒शवो॒ हि हि प॒शवो॑ मान॒वी । \newline
51. प॒शवो॑ मान॒वी मा॑न॒वी प॒शवः॑ प॒शवो॑ मान॒वीतीति॑ मान॒वी प॒शवः॑ प॒शवो॑ मान॒वीति॑ । \newline
52. मा॒न॒वीतीति॑ मान॒वी मा॑न॒वी त्या॑हा॒हे ति॑ मान॒वी मा॑न॒वी त्या॑ह । \newline
53. इत्या॑हा॒हे तीत्या॑ह॒ मनु॒र् मनु॑रा॒हे तीत्या॑ह॒ मनुः॑ । \newline
54. आ॒ह॒ मनु॒र् मनु॑ राहाह॒ मनु॒र्॒. हि हि मनु॑ राहाह॒ मनु॒र्॒. हि । \newline
55. मनु॒र्॒. हि हि मनु॒र् मनु॒र् ह्ये॑ता मे॒ताꣳ हि मनु॒र् मनु॒र् ह्ये॑ताम् । \newline
56. ह्ये॑ता मे॒ताꣳ हि ह्ये॑ता मग्रे ऽग्र॑ ए॒ताꣳ हि ह्ये॑ता मग्रे᳚ । \newline
57. ए॒ता मग्रे ऽग्र॑ ए॒ता मे॒ता मग्रे ऽप॑श्य॒ दप॑श्य॒ दग्र॑ ए॒ता मे॒ता मग्रे ऽप॑श्यत् । \newline
\pagebreak
\markright{ TS 2.6.7.4  \hfill https://www.vedavms.in \hfill}
\addcontentsline{toc}{section}{ TS 2.6.7.4 }
\section*{ TS 2.6.7.4 }

\textbf{TS 2.6.7.4 } \newline
\textbf{Samhita Paata} \newline

-मग्रे ऽप॑श्यद्-घृ॒तप॒दीत्या॑ह॒ य दे॒वास्यै॑ प॒दाद्-घृ॒तमपी᳚ड्यत॒ तस्मा॑दे॒वमा॑ह मैत्रावरु॒णीत्या॑ह मि॒त्रावरु॑णौ॒ ह्ये॑नाꣳ स॒मैर॑यतां॒ ब्रह्म॑ दे॒वकृ॑त॒-मुप॑हूत॒मित्या॑ह॒ ब्रह्मै॒वोप॑ ह्वयते॒ दैव्या॑ अद्ध्व॒र्यव॒ उप॑हूता॒ उप॑हूता मनु॒ष्या॑ इत्या॑ह देवमनु॒ष्याने॒वोप॑ ह्वयते॒ य इ॒मं ॅय॒ज्ञ्मवा॒न्॒ ये य॒ज्ञ्प॑तिं॒ ॅवर्द्धा॒नित्या॑ह - [  ] \newline

\textbf{Pada Paata} \newline

अग्रे᳚ । अप॑श्यत् । घृ॒तप॒दीति॑ घृ॒त-प॒दी॒ । इति॑ । आ॒ह॒ । यत् । ए॒व । अ॒स्यै॒ । प॒दात् । घृ॒तम् । अपी᳚ड्यत । तस्मा᳚त् । ए॒वम् । आ॒ह॒ । मै॒त्रा॒व॒रु॒णीति॑ मैत्रा - व॒रु॒णी । इति॑ । आ॒ह॒ । मि॒त्रावरु॑णा॒विति॑ मि॒त्रा - वरु॑णौ । हि । ए॒ना॒म् । स॒मैर॑यता॒मिति॑ सं - ऐर॑यताम् । ब्रह्म॑ । दे॒वकृ॑त॒मिति॑ दे॒व - कृ॒त॒म् । उप॑हूत॒मित्युप॑ - हू॒त॒म् । इति॑ । आ॒ह॒ । ब्रह्म॑ । ए॒व । उपेति॑ । ह्व॒य॒ते॒ । दैव्याः᳚ । अ॒द्ध्व॒र्यवः॑ । उप॑हूता॒ इत्युप॑ - हू॒ताः॒ । उप॑हूता॒ इत्युप॑ - हू॒ताः॒ । म॒नु॒ष्याः᳚ । इति॑ । आ॒ह॒ । दे॒व॒म॒नु॒ष्यानिति॑ देव-म॒नु॒ष्यान् । ए॒व । उपेति॑ । ह्व॒य॒ते॒ । ये । इ॒मम् । य॒ज्ञ्म् । अवान्॑ । ये । य॒ज्ञ्प॑ति॒मिति॑ य॒ज्ञ् - प॒ति॒म् । वर्द्धान्॑ । इति॑ । आ॒ह॒ ।  \newline


\textbf{Krama Paata} \newline

अग्रे ऽप॑श्यत् । अप॑श्यद् घृ॒तप॑दी । घृ॒तप॒दीति॑ । घृ॒तप॒दीति॑ घृ॒त - प॒दी॒ । इत्या॑ह । आ॒ह॒ यत् । यदे॒व । ए॒वास्यै᳚ । अ॒स्यै॒ प॒दात् । प॒दाद् घृ॒तम् । घृ॒तमपी᳚ड्यत । अपी᳚ड्यत॒ तस्मा᳚त् । तस्मा॑दे॒वम् । ए॒वमा॑ह । आ॒ह॒ मै॒त्रा॒व॒रु॒णी । मै॒त्रा॒व॒रु॒णीति॑ । मै॒त्रा॒व॒रु॒णीति॑ मैत्रा - व॒रु॒णी । इत्या॑ह । आ॒ह॒ मि॒त्रावरु॑णौ । मि॒त्राव॑रुणौ॒ हि । मि॒त्रावरु॑णा॒विति॑ मि॒त्रा - वरु॑णौ । ह्ये॑नाम् । ए॒नाꣳ॒॒ स॒मैर॑यताम् । स॒मैर॑यता॒म् ब्रह्म॑ । स॒मैर॑यता॒मिति॑ सं - ऐर॑यताम् । ब्रह्म॑ दे॒वकृ॑तम् । दे॒वकृ॑त॒मुप॑हूतम् । दे॒वकृ॑त॒मिति॑ दे॒व - कृ॒त॒म् । उप॑हूत॒मिति॑ । उप॑हूत॒मित्युप॑ - हू॒त॒म् । इत्या॑ह । आ॒ह॒ ब्रह्म॑ । ब्रह्मै॒व । ए॒वोप॑ । उप॑ ह्वयते । ह्व॒य॒ते॒ दैव्याः᳚ । दैव्या॑ अद्ध्व॒र्यवः॑ । अ॒द्ध्व॒र्यव॒ उप॑हूताः । उप॑हूता॒ उप॑हूताः । उप॑हूता॒ इत्युप॑ - हू॒ताः॒ । उप॑हूता मनु॒ष्याः᳚ । उप॑हूता॒ इत्युप॑ - हू॒ताः॒ । म॒नु॒ष्या॑ इति॑ । इत्या॑ह । आ॒ह॒ दे॒व॒म॒नु॒ष्यान् । दे॒व॒म॒नु॒ष्याने॒व । दे॒व॒म॒नु॒ष्यानिति॑ देव - म॒नु॒ष्यान् । ए॒वोप॑ । उप॑ ह्वयते । ह्व॒य॒ते॒ ये । य इ॒मम् । इ॒मं ॅय॒ज्ञ्म् । य॒ज्ञ्मवान्॑ । अवा॒न्॒. ये । ये य॒ज्ञ्प॑तिम् । य॒ज्ञ्प॑तिं॒ ॅवर्द्धान्॑ । य॒ज्ञ्प॑ति॒मिति॑ य॒ज्ञ् - प॒ति॒म् । वर्द्धा॒निति॑ । इत्या॑ह ।  आ॒ह॒ य॒ज्ञाय॑ \newline

\textbf{Jatai Paata} \newline

1. अग्रे ऽप॑श्य॒ दप॑श्य॒ दग्रे ऽग्रे ऽप॑श्यत् । \newline
2. अप॑श्यद् घृ॒तप॑दी घृ॒तप॒ द्यप॑श्य॒ दप॑श्यद् घृ॒तप॑दी । \newline
3. घृ॒तप॒दीतीति॑ घृ॒तप॑दी घृ॒तप॒दीति॑ । \newline
4. घृ॒तप॒दीति॑ घृ॒त - प॒दी॒ । \newline
5. इत्या॑हा॒हे तीत्या॑ह । \newline
6. आ॒ह॒ यद् यदा॑हाह॒ यत् । \newline
7. यदे॒वैव यद् यदे॒व । \newline
8. ए॒वास्या॑ अस्या ए॒वैवास्यै᳚ । \newline
9. अ॒स्यै॒ प॒दात् प॒दाद॑स्या अस्यै प॒दात् । \newline
10. प॒दाद् घृ॒तम् घृ॒तम् प॒दात् प॒दाद् घृ॒तम् । \newline
11. घृ॒त मपी᳚ड्य॒ता पी᳚ड्यत घृ॒तम् घृ॒त मपी᳚ड्यत । \newline
12. अपी᳚ड्यत॒ तस्मा॒त् तस्मा॒ दपी᳚ड्य॒ता पी᳚ड्यत॒ तस्मा᳚त् । \newline
13. तस्मा॑ दे॒व मे॒वम् तस्मा॒त् तस्मा॑ दे॒वम् । \newline
14. ए॒व मा॑हाहै॒व मे॒व मा॑ह । \newline
15. आ॒ह॒ मै॒त्रा॒व॒रु॒णी मै᳚त्रावरु॒ ण्या॑हाह मैत्रावरु॒णी । \newline
16. मै॒त्रा॒व॒रु॒णी तीति॑ मैत्रावरु॒णी मै᳚त्रावरु॒णीति॑ । \newline
17. मै॒त्रा॒व॒रु॒णीति॑ मैत्रा - व॒रु॒णी । \newline
18. इत्या॑हा॒हे तीत्या॑ह । \newline
19. आ॒ह॒ मि॒त्रावरु॑णौ मि॒त्रावरु॑णा वाहाह मि॒त्रावरु॑णौ । \newline
20. मि॒त्रावरु॑णौ॒ हि हि मि॒त्रावरु॑णौ मि॒त्रावरु॑णौ॒ हि । \newline
21. मि॒त्रावरु॑णा॒विति॑ मि॒त्रा - वरु॑णौ । \newline
22. ह्ये॑ना मेनाꣳ॒॒ हि ह्ये॑नाम् । \newline
23. ए॒नाꣳ॒॒ स॒मैर॑यताꣳ स॒मैर॑यता मेना मेनाꣳ स॒मैर॑यताम् । \newline
24. स॒मैर॑यता॒म् ब्रह्म॒ ब्रह्म॑ स॒मैर॑यताꣳ स॒मैर॑यता॒म् ब्रह्म॑ । \newline
25. स॒मैर॑यता॒मिति॑ सं - ऐर॑यताम् । \newline
26. ब्रह्म॑ दे॒वकृ॑तम् दे॒वकृ॑त॒म् ब्रह्म॒ ब्रह्म॑ दे॒वकृ॑तम् । \newline
27. दे॒वकृ॑त॒ मुप॑हूत॒ मुप॑हूतम् दे॒वकृ॑तम् दे॒वकृ॑त॒ मुप॑हूतम् । \newline
28. दे॒वकृ॑त॒मिति॑ दे॒व - कृ॒त॒म् । \newline
29. उप॑हूत॒ मिती त्युप॑हूत॒ मुप॑हूत॒ मिति॑ । \newline
30. उप॑हूत॒मित्युप॑ - हू॒त॒म् । \newline
31. इत्या॑हा॒हे तीत्या॑ह । \newline
32. आ॒ह॒ ब्रह्म॒ ब्रह्मा॑हाह॒ ब्रह्म॑ । \newline
33. ब्रह्मै॒वैव ब्रह्म॒ ब्रह्मै॒व । \newline
34. ए॒वोपो पै॒वैवोप॑ । \newline
35. उप॑ ह्वयते ह्वयत॒ उपोप॑ ह्वयते । \newline
36. ह्व॒य॒ते॒ दैव्या॒ दैव्या᳚ ह्वयते ह्वयते॒ दैव्याः᳚ । \newline
37. दैव्या॑ अद्ध्व॒र्यवो᳚ ऽद्ध्व॒र्यवो॒ दैव्या॒ दैव्या॑ अद्ध्व॒र्यवः॑ । \newline
38. अ॒द्ध्व॒र्यव॒ उप॑हूता॒ उप॑हूता अद्ध्व॒र्यवो᳚ ऽद्ध्व॒र्यव॒ उप॑हूताः । \newline
39. उप॑हूता॒ उप॑हूताः । \newline
40. उप॑हूता॒ इत्युप॑ - हू॒ताः॒ । \newline
41. उप॑हूता मनु॒ष्या॑ मनु॒ष्या॑ उप॑हूता॒ उप॑हूता मनु॒ष्याः᳚ । \newline
42. उप॑हूता॒ इत्युप॑ - हू॒ताः॒ । \newline
43. म॒नु॒ष्या॑ इतीति॑ मनु॒ष्या॑ मनु॒ष्या॑ इति॑ । \newline
44. इत्या॑हा॒हे तीत्या॑ह । \newline
45. आ॒ह॒ दे॒व॒म॒नु॒ष्यान् दे॑वमनु॒ष्या ना॑हाह देवमनु॒ष्यान् । \newline
46. दे॒व॒म॒नु॒ष्या ने॒वैव दे॑वमनु॒ष्यान् दे॑वमनु॒ष्या ने॒व । \newline
47. दे॒व॒म॒नु॒ष्यानिति॑ देव - म॒नु॒ष्यान् । \newline
48. ए॒वोपो पै॒वैवोप॑ । \newline
49. उप॑ ह्वयते ह्वयत॒ उपोप॑ ह्वयते । \newline
50. ह्व॒य॒ते॒ ये ये ह्व॑यते ह्वयते॒ ये । \newline
51. य इ॒म मि॒मं ॅये य इ॒मम् । \newline
52. इ॒मं ॅय॒ज्ञ्ं ॅय॒ज्ञ् मि॒म मि॒मं ॅय॒ज्ञ्म् । \newline
53. य॒ज्ञ् मवा॒ नवान्॑. य॒ज्ञ्ं ॅय॒ज्ञ् मवान्॑ । \newline
54. अवा॒न्॒. ये ये ऽवा॒ नवा॒न्॒. ये । \newline
55. ये य॒ज्ञ्प॑तिं ॅय॒ज्ञ्प॑तिं॒ ॅये ये य॒ज्ञ्प॑तिम् । \newline
56. य॒ज्ञ्प॑तिं॒ ॅवर्द्धा॒न्॒. वर्द्धा॑न्. य॒ज्ञ्प॑तिं ॅय॒ज्ञ्प॑तिं॒ ॅवर्द्धान्॑ । \newline
57. य॒ज्ञ्प॑ति॒मिति॑ य॒ज्ञ् - प॒ति॒म् । \newline
58. वर्द्धा॒ नितीति॒ वर्द्धा॒न्॒. वर्द्धा॒ निति॑ । \newline
59. इत्या॑हा॒हे तीत्या॑ह । \newline
60. आ॒ह॒ य॒ज्ञाय॑ य॒ज्ञाया॑ हाह य॒ज्ञाय॑ । \newline

\textbf{Ghana Paata } \newline

1. अग्रे ऽप॑श्य॒ दप॑श्य॒ दग्रे ऽग्रे ऽप॑श्यद् घृ॒तप॑दी घृ॒तप॒ द्यप॑श्य॒ दग्रे ऽग्रे ऽप॑श्यद् घृ॒तप॑दी । \newline
2. अप॑श्यद् घृ॒तप॑दी घृ॒तप॒ द्यप॑श्य॒ दप॑श्यद् घृ॒तप॒ दीतीति॑ घृ॒तप॒ द्यप॑श्य॒ दप॑श्यद् घृ॒तप॒दीति॑ । \newline
3. घृ॒तप॒ दीतीति॑ घृ॒तप॑दी घृ॒तप॒दी त्या॑हा॒हे ति॑ घृ॒तप॑दी घृ॒तप॒दी त्या॑ह । \newline
4. घृ॒तप॒दीति॑ घृ॒त - प॒दी॒ । \newline
5. इत्या॑हा॒हे तीत्या॑ह॒ यद् यदा॒हे तीत्या॑ह॒ यत् । \newline
6. आ॒ह॒ यद् यदा॑हाह॒ यदे॒वैव यदा॑हाह॒ यदे॒व । \newline
7. यदे॒वैव यद् यदे॒वास्या॑ अस्या ए॒व यद् यदे॒वास्यै᳚ । \newline
8. ए॒वास्या॑ अस्या ए॒वैवास्यै॑ प॒दात् प॒दाद॑स्या ए॒वैवास्यै॑ प॒दात् । \newline
9. अ॒स्यै॒ प॒दात् प॒दाद॑स्या अस्यै प॒दाद् घृ॒तम् घृ॒तम् प॒दाद॑स्या अस्यै प॒दाद् घृ॒तम् । \newline
10. प॒दाद् घृ॒तम् घृ॒तम् प॒दात् प॒दाद् घृ॒त मपी᳚ड्य॒ता पी᳚ड्यत घृ॒तम् प॒दात् प॒दाद् घृ॒त मपी᳚ड्यत । \newline
11. घृ॒त मपी᳚ड्य॒ता पी᳚ड्यत घृ॒तम् घृ॒त मपी᳚ड्यत॒ तस्मा॒त् तस्मा॒ दपी᳚ड्यत घृ॒तम् घृ॒त मपी᳚ड्यत॒ तस्मा᳚त् । \newline
12. अपी᳚ड्यत॒ तस्मा॒त् तस्मा॒ दपी᳚ड्य॒ता पी᳚ड्यत॒ तस्मा॑ दे॒व मे॒वम् तस्मा॒ दपी᳚ड्य॒ता पी᳚ड्यत॒ तस्मा॑ दे॒वम् । \newline
13. तस्मा॑दे॒व मे॒वम् तस्मा॒त् तस्मा॑ दे॒व मा॑हाहै॒वम् तस्मा॒त् तस्मा॑ दे॒व मा॑ह । \newline
14. ए॒व मा॑हाहै॒व मे॒व मा॑ह मैत्रावरु॒णी मै᳚त्रावरु॒ण्या॑है॒व मे॒व मा॑ह मैत्रावरु॒णी । \newline
15. आ॒ह॒ मै॒त्रा॒व॒रु॒णी मै᳚त्रावरु॒ण्या॑ हाह मैत्रावरु॒णीतीति॑ मैत्रावरु॒ण्या॑ हाह मैत्रावरु॒णीति॑ । \newline
16. मै॒त्रा॒व॒रु॒णीतीति॑ मैत्रावरु॒णी मै᳚त्रावरु॒णी त्या॑हा॒हे ति॑ मैत्रावरु॒णी मै᳚त्रावरु॒णी त्या॑ह । \newline
17. मै॒त्रा॒व॒रु॒णीति॑ मैत्रा - व॒रु॒णी । \newline
18. इत्या॑हा॒हे तीत्या॑ह मि॒त्रावरु॑णौ मि॒त्रावरु॑णा वा॒हे तीत्या॑ह मि॒त्रावरु॑णौ । \newline
19. आ॒ह॒ मि॒त्रावरु॑णौ मि॒त्रावरु॑णा वाहाह मि॒त्रावरु॑णौ॒ हि हि मि॒त्रावरु॑णा वाहाह मि॒त्रावरु॑णौ॒ हि । \newline
20. मि॒त्रावरु॑णौ॒ हि हि मि॒त्रावरु॑णौ मि॒त्रावरु॑णौ॒ ह्ये॑ना मेनाꣳ॒॒ हि मि॒त्रावरु॑णौ मि॒त्रावरु॑णौ॒ ह्ये॑नाम् । \newline
21. मि॒त्रावरु॑णा॒विति॑ मि॒त्रा - वरु॑णौ । \newline
22. ह्ये॑ना मेनाꣳ॒॒ हि ह्ये॑नाꣳ स॒मैर॑यताꣳ स॒मैर॑यता मेनाꣳ॒॒ हि ह्ये॑नाꣳ स॒मैर॑यताम् । \newline
23. ए॒नाꣳ॒॒ स॒मैर॑यताꣳ स॒मैर॑यता मेना मेनाꣳ स॒मैर॑यता॒म् ब्रह्म॒ ब्रह्म॑ स॒मैर॑यता मेना मेनाꣳ स॒मैर॑यता॒म् ब्रह्म॑ । \newline
24. स॒मैर॑यता॒म् ब्रह्म॒ ब्रह्म॑ स॒मैर॑यताꣳ स॒मैर॑यता॒म् ब्रह्म॑ दे॒वकृ॑तम् दे॒वकृ॑त॒म् ब्रह्म॑ स॒मैर॑यताꣳ स॒मैर॑यता॒म् ब्रह्म॑ दे॒वकृ॑तम् । \newline
25. स॒मैर॑यता॒मिति॑ सं - ऐर॑यताम् । \newline
26. ब्रह्म॑ दे॒वकृ॑तम् दे॒वकृ॑त॒म् ब्रह्म॒ ब्रह्म॑ दे॒वकृ॑त॒ मुप॑हूत॒ मुप॑हूतम् दे॒वकृ॑त॒म् ब्रह्म॒ ब्रह्म॑ दे॒वकृ॑त॒ मुप॑हूतम् । \newline
27. दे॒वकृ॑त॒ मुप॑हूत॒ मुप॑हूतम् दे॒वकृ॑तम् दे॒वकृ॑त॒ मुप॑हूत॒ मिती त्युप॑हूतम् दे॒वकृ॑तम् दे॒वकृ॑त॒ मुप॑हूत॒ मिति॑ । \newline
28. दे॒वकृ॑त॒मिति॑ दे॒व - कृ॒त॒म् । \newline
29. उप॑हूत॒ मिती त्युप॑हूत॒ मुप॑हूत॒ मित्या॑हा॒हे त्युप॑हूत॒ मुप॑हूत॒ मित्या॑ह । \newline
30. उप॑हूत॒मित्युप॑ - हू॒त॒म् । \newline
31. इत्या॑हा॒हे तीत्या॑ह॒ ब्रह्म॒ ब्रह्मा॒हे तीत्या॑ह॒ ब्रह्म॑ । \newline
32. आ॒ह॒ ब्रह्म॒ ब्रह्मा॑हाह॒ ब्रह्मै॒वैव ब्रह्मा॑हाह॒ ब्रह्मै॒व । \newline
33. ब्रह्मै॒वैव ब्रह्म॒ ब्रह्मै॒वोपो पै॒व ब्रह्म॒ ब्रह्मै॒वोप॑ । \newline
34. ए॒वोपो पै॒वैवोप॑ ह्वयते ह्वयत॒ उपै॒ वैवोप॑ ह्वयते । \newline
35. उप॑ ह्वयते ह्वयत॒ उपोप॑ ह्वयते॒ दैव्या॒ दैव्या᳚ ह्वयत॒ उपोप॑ ह्वयते॒ दैव्याः᳚ । \newline
36. ह्व॒य॒ते॒ दैव्या॒ दैव्या᳚ ह्वयते ह्वयते॒ दैव्या॑ अद्ध्व॒र्यवो᳚ ऽद्ध्व॒र्यवो॒ दैव्या᳚ ह्वयते ह्वयते॒ दैव्या॑ अद्ध्व॒र्यवः॑ । \newline
37. दैव्या॑ अद्ध्व॒र्यवो᳚ ऽद्ध्व॒र्यवो॒ दैव्या॒ दैव्या॑ अद्ध्व॒र्यव॒ उप॑हूता॒ उप॑हूता अद्ध्व॒र्यवो॒ दैव्या॒ दैव्या॑ अद्ध्व॒र्यव॒ उप॑हूताः । \newline
38. अ॒द्ध्व॒र्यव॒ उप॑हूता॒ उप॑हूता अद्ध्व॒र्यवो᳚ ऽद्ध्व॒र्यव॒ उप॑हूताः । \newline
39. उप॑हूता॒ उप॑हूताः । \newline
40. उप॑हूता॒ इत्युप॑ - हू॒ताः॒ । \newline
41. उप॑हूता मनु॒ष्या॑ मनु॒ष्या॑ उप॑हूता॒ उप॑हूता मनु॒ष्या॑ इतीति॑ मनु॒ष्या॑ उप॑हूता॒ उप॑हूता मनु॒ष्या॑ इति॑ । \newline
42. उप॑हूता॒ इत्युप॑ - हू॒ताः॒ । \newline
43. म॒नु॒ष्या॑ इतीति॑ मनु॒ष्या॑ मनु॒ष्या॑ इत्या॑हा॒हे ति॑ मनु॒ष्या॑ मनु॒ष्या॑ इत्या॑ह । \newline
44. इत्या॑हा॒हे तीत्या॑ह देवमनु॒ष्यान् दे॑वमनु॒ष्या ना॒हे तीत्या॑ह देवमनु॒ष्यान् । \newline
45. आ॒ह॒ दे॒व॒म॒नु॒ष्यान् दे॑वमनु॒ष्या ना॑हाह देवमनु॒ष्या ने॒वैव दे॑वमनु॒ष्या ना॑हाह देवमनु॒ष्या ने॒व । \newline
46. दे॒व॒म॒नु॒ष्या ने॒वैव दे॑वमनु॒ष्यान् दे॑वमनु॒ष्या ने॒वोपो पै॒व दे॑वमनु॒ष्यान् दे॑वमनु॒ष्या ने॒वोप॑ । \newline
47. दे॒व॒म॒नु॒ष्यानिति॑ देव - म॒नु॒ष्यान् । \newline
48. ए॒वोपो पै॒वैवोप॑ ह्वयते ह्वयत॒ उपै॒वै वोप॑ ह्वयते । \newline
49. उप॑ ह्वयते ह्वयत॒ उपोप॑ ह्वयते॒ ये ये ह्व॑यत॒ उपोप॑ ह्वयते॒ ये । \newline
50. ह्व॒य॒ते॒ ये ये ह्व॑यते ह्वयते॒ य इ॒म मि॒मं ॅये ह्व॑यते ह्वयते॒ य इ॒मम् । \newline
51. य इ॒म मि॒मं ॅये य इ॒मं ॅय॒ज्ञ्ं ॅय॒ज्ञ् मि॒मं ॅये य इ॒मं ॅय॒ज्ञ्म् । \newline
52. इ॒मं ॅय॒ज्ञ्ं ॅय॒ज्ञ् मि॒म मि॒मं ॅय॒ज्ञ् मवा॒ नवान्॑. य॒ज्ञ् मि॒म मि॒मं ॅय॒ज्ञ् मवान्॑ । \newline
53. य॒ज्ञ् मवा॒ नवान्॑. य॒ज्ञ्ं ॅय॒ज्ञ् मवा॒न्॒. ये ये ऽवान्॑. य॒ज्ञ्ं ॅय॒ज्ञ् मवा॒न्॒. ये । \newline
54. अवा॒न्॒. ये ये ऽवा॒ नवा॒न्॒. ये य॒ज्ञ्प॑तिं ॅय॒ज्ञ्प॑तिं॒ ॅये ऽवा॒ नवा॒न्॒. ये य॒ज्ञ्प॑तिम् । \newline
55. ये य॒ज्ञ्प॑तिं ॅय॒ज्ञ्प॑तिं॒ ॅये ये य॒ज्ञ्प॑तिं॒ ॅवर्द्धा॒न्॒. वर्द्धान्॑. य॒ज्ञ्प॑तिं॒ ॅये ये य॒ज्ञ्प॑तिं॒ ॅवर्द्धान्॑ । \newline
56. य॒ज्ञ्प॑तिं॒ ॅवर्द्धा॒न्॒. वर्द्धान्॑. य॒ज्ञ्प॑तिं ॅय॒ज्ञ्प॑तिं॒ ॅवर्द्धा॒ नितीति॒ वर्द्धान्॑. य॒ज्ञ्प॑तिं ॅय॒ज्ञ्प॑तिं॒ ॅवर्द्धा॒ निति॑ । \newline
57. य॒ज्ञ्प॑ति॒मिति॑ य॒ज्ञ् - प॒ति॒म् । \newline
58. वर्द्धा॒ नितीति॒ वर्द्धा॒न्॒. वर्द्धा॒ नित्या॑हा॒हे ति॒ वर्द्धा॒न्॒. वर्द्धा॒ नित्या॑ह । \newline
59. इत्या॑हा॒हे तीत्या॑ह य॒ज्ञाय॑ य॒ज्ञाया॒हे तीत्या॑ह य॒ज्ञाय॑ । \newline
60. आ॒ह॒ य॒ज्ञाय॑ य॒ज्ञाया॑ हाह य॒ज्ञाय॑ च च य॒ज्ञाया॑ हाह य॒ज्ञाय॑ च । \newline
\pagebreak
\markright{ TS 2.6.7.5  \hfill https://www.vedavms.in \hfill}
\addcontentsline{toc}{section}{ TS 2.6.7.5 }
\section*{ TS 2.6.7.5 }

\textbf{TS 2.6.7.5 } \newline
\textbf{Samhita Paata} \newline

य॒ज्ञाय॑ चै॒व यज॑मानाय चा॒ ऽऽशिष॒मा शा᳚स्त॒ उप॑हूते॒ द्यावा॑पृथि॒वी इत्या॑ह॒ द्यावा॑पृथि॒वी ए॒वोप॑ ह्वयते पूर्व॒जे ऋ॒ताव॑री॒ इत्या॑ह पूर्व॒जे ह्ये॑ते ऋ॒ताव॑री दे॒वी दे॒वपु॑त्रे॒ इत्या॑ह दे॒वी ह्ये॑ते दे॒वपु॑त्रे॒ उप॑हूतो॒ऽयं ॅयज॑मान॒ इत्या॑ह॒ यज॑मानमे॒वोप॑ ह्वयत॒ उत्त॑रस्यां देवय॒ज्याया॒मुप॑हूतो॒ भूय॑सि हवि॒ष्कर॑ण॒ उप॑हूतो दि॒व्ये धाम॒न्नुप॑हूत॒ - [  ] \newline

\textbf{Pada Paata} \newline

य॒ज्ञाय॑ । च॒ । ए॒व । यज॑मानाय । च॒ । आ॒शिष॒मित्या᳚-शिष᳚म् । एति॑ । शा॒स्ते॒ । उप॑हूते॒ इत्युप॑ - हू॒ते॒ । द्यावा॑पृथि॒वी इति॒ द्यावा᳚ - पृ॒थि॒वी । इति॑ । आ॒ह॒ । द्यावा॑पृथि॒वी इति॒ द्यावा᳚ - पृ॒थि॒वी । ए॒व । उपेति॑ । ह्व॒य॒ते॒ । पू॒र्व॒जे इति॑ पूर्व - जे । ऋ॒ताव॑री॒ इत्यृ॒त - व॒री॒ । इति॑ । आ॒ह॒ । पू॒र्व॒जे इति॑ पूर्व - जे । हि । ए॒ते इति॑ । ऋ॒ताव॑री॒ इत्यृ॒त - व॒री॒ । दे॒वी इति॑ । दे॒वपु॑त्रे॒ इति॑ दे॒व - पु॒त्रे॒ । इति॑ । आ॒ह॒ । दे॒वी इति॑ । हि । ए॒ते इति॑ । दे॒वपु॑त्रे॒ इति॑ दे॒व - पु॒त्रे॒ । उप॑हूत॒ इत्युप॑-हू॒तः॒ । अ॒यम् । यज॑मानः । इति॑ । आ॒ह॒ । यज॑मानम् । ए॒व । उपेति॑ । ह्व॒य॒ते॒ । उत्त॑रस्या॒मित्युत् - त॒र॒स्या॒म् । दे॒व॒य॒ज्याया॒मिति॑ देव - य॒ज्याया᳚म् । उप॑हूत॒ इत्युप॑ - हू॒तः॒ । भूय॑सि । ह॒वि॒ष्कर॑ण॒ इति॑ हविः - कर॑णे । उप॑हूत॒ इत्युप॑ - हू॒तः॒ । दि॒व्ये । धामन्न्॑ । उप॑हूत॒ इत्युप॑ - हू॒तः॒ ।  \newline


\textbf{Krama Paata} \newline

य॒ज्ञाय॑ च । चै॒व । ए॒व यज॑मानाय । यज॑मानाय च । चा॒शिष᳚म् । आ॒शिष॒मा । आ॒शिष॒मित्या᳚ - शिष᳚म् । आ शा᳚स्ते । शा॒स्त॒ उप॑हूते । उप॑हूते॒ द्यावा॑पृथि॒वी । उप॑हूते॒ इत्युप॑ - हू॒ते॒ । द्यावा॑पृथि॒वी इति॑ । द्यावा॑पृथि॒वी इति॒ द्यावा᳚ - पृ॒थि॒वी । इत्या॑ह । आ॒ह॒ द्यावा॑पृथि॒वी । द्यावा॑पृथि॒वी ए॒व । द्यावा॑पृथि॒वी इति॒ द्यावा᳚ - पृ॒थि॒वी । ए॒वोप॑ । उप॑ ह्वयते । ह्व॒य॒ते॒ पू॒र्व॒जे । पू॒र्व॒जे ऋ॒ताव॑री । पू॒र्व॒जे इति॑ पूर्व - जे । ऋ॒ताव॑री॒ इति॑ । ऋ॒ताव॑री॒ इत्यृ॒त - व॒री॒ । इत्या॑ह । आ॒ह॒ पू॒र्व॒जे । पू॒र्व॒जे हि । पू॒र्व॒जे इति॑ पूर्व - जे । ह्ये॑ते । ए॒ते ऋ॒ताव॑री । ए॒ते इत्ये॒ते । ऋ॒ताव॑री दे॒वी । ऋ॒ताव॑री॒ इत्यृ॒त - व॒री॒ । दे॒वी दे॒वपु॑त्रे । दे॒वी इति॑ दे॒वी । दे॒वपु॑त्रे॒ इति॑ । दे॒वपु॑त्रे॒ इति॑ दे॒व - पु॒त्रे॒ । इत्या॑ह । आ॒ह॒ दे॒वी । दे॒वी हि । दे॒वी इति॑ दे॒वी । ह्ये॑ते । ए॒ते दे॒वपु॑त्रे । ए॒ते इत्ये॒ते । दे॒वपु॑त्रे॒ उप॑हूतः । दे॒वपु॑त्रे॒ इति॑ दे॒व - पु॒त्रे॒ । उप॑हूतो॒ऽयम् । उप॑हूत॒ इत्युप॑ - हू॒तः॒ । अ॒यं ॅयज॑मानः । यज॑मान॒ इति॑ । इत्या॑ह । आ॒ह॒ यज॑मानम् । यज॑मानमे॒व । ए॒वोप॑ । उप॑ ह्वयते । ह्व॒य॒त॒ उत्त॑रस्याम् । उत्त॑रस्याम् देवय॒ज्याया᳚म् । उत्त॑रस्या॒मित्युत् - त॒र॒स्या॒म् । दे॒व॒य॒ज्याया॒मुप॑हूतः । दे॒व॒य॒ज्याया॒मिति॑ देव - य॒ज्याया᳚म् । उप॑हूतो॒ भूय॑सि । उप॑हूत॒ इत्युप॑ - हू॒तः॒ । भूय॑सि हवि॒ष्कर॑णे । ह॒वि॒ष्कर॑ण॒ उप॑हूतः । ह॒वि॒ष्कर॑ण॒ इति॑ हविः - कर॑णे । उप॑हूतो दि॒व्ये । उप॑हूत॒ इत्युप॑ - हू॒तः॒ । दि॒व्ये धामन्न्॑ । धाम॒न्नुप॑हूतः ( ) । उप॑हूत॒ इति॑ । उप॑हूत॒ इत्युप॑ - हू॒तः॒ \newline

\textbf{Jatai Paata} \newline

1. य॒ज्ञाय॑ च च य॒ज्ञाय॑ य॒ज्ञाय॑ च । \newline
2. चै॒वैव च॑ चै॒व । \newline
3. ए॒व यज॑मानाय॒ यज॑माना यै॒वैव यज॑मानाय । \newline
4. यज॑मानाय च च॒ यज॑मानाय॒ यज॑मानाय च । \newline
5. चा॒शिष॑ मा॒शिष॑म् च चा॒शिष᳚म् । \newline
6. आ॒शिष॒ मा ऽऽशिष॑ मा॒शिष॒ मा । \newline
7. आ॒शिष॒मित्या᳚ - शिष᳚म् । \newline
8. आ शा᳚स्ते शास्त॒ आ शा᳚स्ते । \newline
9. शा॒स्त॒ उप॑हूते॒ उप॑हूते शास्ते शास्त॒ उप॑हूते । \newline
10. उप॑हूते॒ द्यावा॑पृथि॒वी द्यावा॑पृथि॒वी उप॑हूते॒ उप॑हूते॒ द्यावा॑पृथि॒वी । \newline
11. उप॑हूते॒ इत्युप॑ - हू॒ते॒ । \newline
12. द्यावा॑पृथि॒वी इतीति॒ द्यावा॑पृथि॒वी द्यावा॑पृथि॒वी इति॑ । \newline
13. द्यावा॑पृथि॒वी इति॒ द्यावा᳚ - पृ॒थि॒वी । \newline
14. इत्या॑हा॒हे तीत्या॑ह । \newline
15. आ॒ह॒ द्यावा॑पृथि॒वी द्यावा॑पृथि॒वी आ॑हाह॒ द्यावा॑पृथि॒वी । \newline
16. द्यावा॑पृथि॒वी ए॒वैव द्यावा॑पृथि॒वी द्यावा॑पृथि॒वी ए॒व । \newline
17. द्यावा॑पृथि॒वी इति॒ द्यावा᳚ - पृ॒थि॒वी । \newline
18. ए॒वोपो पै॒वैवोप॑ । \newline
19. उप॑ ह्वयते ह्वयत॒ उपोप॑ ह्वयते । \newline
20. ह्व॒य॒ते॒ पू॒र्व॒जे पू᳚र्व॒जे ह्व॑यते ह्वयते पूर्व॒जे । \newline
21. पू॒र्व॒जे ऋ॒ताव॑री ऋ॒ताव॑री पूर्व॒जे पू᳚र्व॒जे ऋ॒ताव॑री । \newline
22. पू॒र्व॒जे इति॑ पूर्व - जे । \newline
23. ऋ॒ताव॑री॒ इती त्यृ॒ताव॑री ऋ॒ताव॑री॒ इति॑ । \newline
24. ऋ॒ताव॑री॒ इत्यृ॒त - व॒री॒ । \newline
25. इत्या॑हा॒हे तीत्या॑ह । \newline
26. आ॒ह॒ पू॒र्व॒जे पू᳚र्व॒जे आ॑हाह पूर्व॒जे । \newline
27. पू॒र्व॒जे हि हि पू᳚र्व॒जे पू᳚र्व॒जे हि । \newline
28. पू॒र्व॒जे इति॑ पूर्व - जे । \newline
29. ह्ये॑ते ए॒ते हि ह्ये॑ते । \newline
30. ए॒ते ऋ॒ताव॑री ऋ॒ताव॑री ए॒ते ए॒ते ऋ॒ताव॑री । \newline
31. ए॒ते इत्ये॒ते । \newline
32. ऋ॒ताव॑री दे॒वी दे॒वी ऋ॒ताव॑री ऋ॒ताव॑री दे॒वी । \newline
33. ऋ॒ताव॑री॒ इत्यृ॒त - व॒री॒ । \newline
34. दे॒वी दे॒वपु॑त्रे दे॒वपु॑त्रे दे॒वी दे॒वी दे॒वपु॑त्रे । \newline
35. दे॒वी इति॑ दे॒वी । \newline
36. दे॒वपु॑त्रे॒ इतीति॑ दे॒वपु॑त्रे दे॒वपु॑त्रे॒ इति॑ । \newline
37. दे॒वपु॑त्रे॒ इति॑ दे॒व - पु॒त्रे॒ । \newline
38. इत्या॑हा॒हे तीत्या॑ह । \newline
39. आ॒ह॒ दे॒वी दे॒वी आ॑हाह दे॒वी । \newline
40. दे॒वी हि हि दे॒वी दे॒वी हि । \newline
41. दे॒वी इति॑ दे॒वी । \newline
42. ह्ये॑ते ए॒ते हि ह्ये॑ते । \newline
43. ए॒ते दे॒वपु॑त्रे दे॒वपु॑त्रे ए॒ते ए॒ते दे॒वपु॑त्रे । \newline
44. ए॒ते इत्ये॒ते । \newline
45. दे॒वपु॑त्रे॒ उप॑हूत॒ उप॑हूतो दे॒वपु॑त्रे दे॒वपु॑त्रे॒ उप॑हूतः । \newline
46. दे॒वपु॑त्रे॒ इति॑ दे॒व - पु॒त्रे॒ । \newline
47. उप॑हूतो॒ ऽय म॒य मुप॑हूत॒ उप॑हूतो॒ ऽयम् । \newline
48. उप॑हूत॒ इत्युप॑ - हू॒तः॒ । \newline
49. अ॒यं ॅयज॑मानो॒ यज॑मानो॒ ऽय म॒यं ॅयज॑मानः । \newline
50. यज॑मान॒ इतीति॒ यज॑मानो॒ यज॑मान॒ इति॑ । \newline
51. इत्या॑हा॒हे तीत्या॑ह । \newline
52. आ॒ह॒ यज॑मानं॒ ॅयज॑मान माहाह॒ यज॑मानम् । \newline
53. यज॑मान मे॒वैव यज॑मानं॒ ॅयज॑मान मे॒व । \newline
54. ए॒वोपो पै॒वैवोप॑ । \newline
55. उप॑ ह्वयते ह्वयत॒ उपोप॑ ह्वयते । \newline
56. ह्व॒य॒त॒ उत्त॑रस्या॒ मुत्त॑रस्याꣳ ह्वयते ह्वयत॒ उत्त॑रस्याम् । \newline
57. उत्त॑रस्याम् देवय॒ज्याया᳚म् देवय॒ज्याया॒ मुत्त॑रस्या॒ मुत्त॑रस्याम् देवय॒ज्याया᳚म् । \newline
58. उत्त॑रस्या॒मित्युत् - त॒र॒स्या॒म् । \newline
59. दे॒व॒य॒ज्याया॒ मुप॑हूत॒ उप॑हूतो देवय॒ज्याया᳚म् देवय॒ज्याया॒ मुप॑हूतः । \newline
60. दे॒व॒य॒ज्याया॒मिति॑ देव - य॒ज्याया᳚म् । \newline
61. उप॑हूतो॒ भूय॑सि॒ भूय॒ स्युप॑हूत॒ उप॑हूतो॒ भूय॑सि । \newline
62. उप॑हूत॒ इत्युप॑ - हू॒तः॒ । \newline
63. भूय॑सि हवि॒ष्कर॑णे हवि॒ष्कर॑णे॒ भूय॑सि॒ भूय॑सि हवि॒ष्कर॑णे । \newline
64. ह॒वि॒ष्कर॑ण॒ उप॑हूत॒ उप॑हूतो हवि॒ष्कर॑णे हवि॒ष्कर॑ण॒ उप॑हूतः । \newline
65. ह॒वि॒ष्कर॑ण॒ इति॑ हविः - कर॑णे । \newline
66. उप॑हूतो दि॒व्ये दि॒व्य उप॑हूत॒ उप॑हूतो दि॒व्ये । \newline
67. उप॑हूत॒ इत्युप॑ - हू॒तः॒ । \newline
68. दि॒व्ये धाम॒न् धाम॑न् दि॒व्ये दि॒व्ये धामन्न्॑ । \newline
69. धाम॒न् नुप॑हूत॒ उप॑हूतो॒ धाम॒न् धाम॒न् नुप॑हूतः । \newline
70. उप॑हूत॒ इतीत्युप॑हूत॒ उप॑हूत॒ इति॑ । \newline
71. उप॑हूत॒ इत्युप॑ - हू॒तः॒ । \newline

\textbf{Ghana Paata } \newline

1. य॒ज्ञाय॑ च च य॒ज्ञाय॑ य॒ज्ञाय॑ चै॒वैव च॑ य॒ज्ञाय॑ य॒ज्ञाय॑ चै॒व । \newline
2. चै॒वैव च॑ चै॒व यज॑मानाय॒ यज॑मानायै॒व च॑ चै॒व यज॑मानाय । \newline
3. ए॒व यज॑मानाय॒ यज॑माना यै॒वैव यज॑मानाय च च॒ यज॑माना यै॒वैव यज॑मानाय च । \newline
4. यज॑मानाय च च॒ यज॑मानाय॒ यज॑मानाय चा॒शिष॑ मा॒शिष॑म् च॒ यज॑मानाय॒ यज॑मानाय चा॒शिष᳚म् । \newline
5. चा॒शिष॑ मा॒शिष॑म् च चा॒शिष॒ मा ऽऽशिष॑म् च चा॒शिष॒ मा । \newline
6. आ॒शिष॒ मा ऽऽशिष॑ मा॒शिष॒ मा शा᳚स्ते शास्त॒ आ ऽऽशिष॑ मा॒शिष॒ मा शा᳚स्ते । \newline
7. आ॒शिष॒मित्या᳚ - शिष᳚म् । \newline
8. आ शा᳚स्ते शास्त॒ आ शा᳚स्त॒ उप॑हूते॒ उप॑हूते शास्त॒ आ शा᳚स्त॒ उप॑हूते । \newline
9. शा॒स्त॒ उप॑हूते॒ उप॑हूते शास्ते शास्त॒ उप॑हूते॒ द्यावा॑पृथि॒वी द्यावा॑पृथि॒वी उप॑हूते शास्ते शास्त॒ उप॑हूते॒ द्यावा॑पृथि॒वी । \newline
10. उप॑हूते॒ द्यावा॑पृथि॒वी द्यावा॑पृथि॒वी उप॑हूते॒ उप॑हूते॒ द्यावा॑पृथि॒वी इतीति॒ द्यावा॑पृथि॒वी उप॑हूते॒ उप॑हूते॒ द्यावा॑पृथि॒वी इति॑ । \newline
11. उप॑हूते॒ इत्युप॑ - हू॒ते॒ । \newline
12. द्यावा॑पृथि॒वी इतीति॒ द्यावा॑पृथि॒वी द्यावा॑पृथि॒वी इत्या॑ हा॒हे ति॒ द्यावा॑पृथि॒वी द्यावा॑पृथि॒वी इत्या॑ह । \newline
13. द्यावा॑पृथि॒वी इति॒ द्यावा᳚ - पृ॒थि॒वी । \newline
14. इत्या॑हा॒हे तीत्या॑ह॒ द्यावा॑पृथि॒वी द्यावा॑पृथि॒वी आ॒हे तीत्या॑ह॒ द्यावा॑पृथि॒वी । \newline
15. आ॒ह॒ द्यावा॑पृथि॒वी द्यावा॑पृथि॒वी आ॑हाह॒ द्यावा॑पृथि॒वी ए॒वैव द्यावा॑पृथि॒वी आ॑हाह॒ द्यावा॑पृथि॒वी ए॒व । \newline
16. द्यावा॑पृथि॒वी ए॒वैव द्यावा॑पृथि॒वी द्यावा॑पृथि॒वी ए॒वोपोपै॒व द्यावा॑पृथि॒वी द्यावा॑पृथि॒वी ए॒वोप॑ । \newline
17. द्यावा॑पृथि॒वी इति॒ द्यावा᳚ - पृ॒थि॒वी । \newline
18. ए॒वोपो पै॒वैवोप॑ ह्वयते ह्वयत॒ उपै॒वै वोप॑ ह्वयते । \newline
19. उप॑ ह्वयते ह्वयत॒ उपोप॑ ह्वयते पूर्व॒जे पू᳚र्व॒जे ह्व॑यत॒ उपोप॑ ह्वयते पूर्व॒जे । \newline
20. ह्व॒य॒ते॒ पू॒र्व॒जे पू᳚र्व॒जे ह्व॑यते ह्वयते पूर्व॒जे ऋ॒ताव॑री ऋ॒ताव॑री पूर्व॒जे ह्व॑यते ह्वयते पूर्व॒जे ऋ॒ताव॑री । \newline
21. पू॒र्व॒जे ऋ॒ताव॑री ऋ॒ताव॑री पूर्व॒जे पू᳚र्व॒जे ऋ॒ताव॑री॒ इती त्यृ॒ताव॑री पूर्व॒जे पू᳚र्व॒जे ऋ॒ताव॑री॒ इति॑ । \newline
22. पू॒र्व॒जे इति॑ पूर्व - जे । \newline
23. ऋ॒ताव॑री॒ इती त्यृ॒ताव॑री ऋ॒ताव॑री॒ इत्या॑हा॒हे त्यृ॒ताव॑री ऋ॒ताव॑री॒ इत्या॑ह । \newline
24. ऋ॒ताव॑री॒ इत्यृ॒त - व॒री॒ । \newline
25. इत्या॑हा॒हे तीत्या॑ह पूर्व॒जे पू᳚र्व॒जे आ॒हे तीत्या॑ह पूर्व॒जे । \newline
26. आ॒ह॒ पू॒र्व॒जे पू᳚र्व॒जे आ॑हाह पूर्व॒जे हि हि पू᳚र्व॒जे आ॑हाह पूर्व॒जे हि । \newline
27. पू॒र्व॒जे हि हि पू᳚र्व॒जे पू᳚र्व॒जे ह्ये॑ते ए॒ते हि पू᳚र्व॒जे पू᳚र्व॒जे ह्ये॑ते । \newline
28. पू॒र्व॒जे इति॑ पूर्व - जे । \newline
29. ह्ये॑ते ए॒ते हि ह्ये॑ते ऋ॒ताव॑री ऋ॒ताव॑री ए॒ते हि ह्ये॑ते ऋ॒ताव॑री । \newline
30. ए॒ते ऋ॒ताव॑री ऋ॒ताव॑री ए॒ते ए॒ते ऋ॒ताव॑री दे॒वी दे॒वी ऋ॒ताव॑री ए॒ते ए॒ते ऋ॒ताव॑री दे॒वी । \newline
31. ए॒ते इत्ये॒ते । \newline
32. ऋ॒ताव॑री दे॒वी दे॒वी ऋ॒ताव॑री ऋ॒ताव॑री दे॒वी दे॒वपु॑त्रे दे॒वपु॑त्रे दे॒वी ऋ॒ताव॑री ऋ॒ताव॑री दे॒वी दे॒वपु॑त्रे । \newline
33. ऋ॒ताव॑री॒ इत्यृ॒त - व॒री॒ । \newline
34. दे॒वी दे॒वपु॑त्रे दे॒वपु॑त्रे दे॒वी दे॒वी दे॒वपु॑त्रे॒ इतीति॑ दे॒वपु॑त्रे दे॒वी दे॒वी दे॒वपु॑त्रे॒ इति॑ । \newline
35. दे॒वी इति॑ दे॒वी । \newline
36. दे॒वपु॑त्रे॒ इतीति॑ दे॒वपु॑त्रे दे॒वपु॑त्रे॒ इत्या॑हा॒हे ति॑ दे॒वपु॑त्रे दे॒वपु॑त्रे॒ इत्या॑ह । \newline
37. दे॒वपु॑त्रे॒ इति॑ दे॒व - पु॒त्रे॒ । \newline
38. इत्या॑हा॒हे तीत्या॑ह दे॒वी दे॒वी आ॒हे तीत्या॑ह दे॒वी । \newline
39. आ॒ह॒ दे॒वी दे॒वी आ॑हाह दे॒वी हि हि दे॒वी आ॑हाह दे॒वी हि । \newline
40. दे॒वी हि हि दे॒वी दे॒वी ह्ये॑ते ए॒ते हि दे॒वी दे॒वी ह्ये॑ते । \newline
41. दे॒वी इति॑ दे॒वी । \newline
42. ह्ये॑ते ए॒ते हि ह्ये॑ते दे॒वपु॑त्रे दे॒वपु॑त्रे ए॒ते हि ह्ये॑ते दे॒वपु॑त्रे । \newline
43. ए॒ते दे॒वपु॑त्रे दे॒वपु॑त्रे ए॒ते ए॒ते दे॒वपु॑त्रे॒ उप॑हूत॒ उप॑हूतो दे॒वपु॑त्रे ए॒ते ए॒ते दे॒वपु॑त्रे॒ उप॑हूतः । \newline
44. ए॒ते इत्ये॒ते । \newline
45. दे॒वपु॑त्रे॒ उप॑हूत॒ उप॑हूतो दे॒वपु॑त्रे दे॒वपु॑त्रे॒ उप॑हूतो॒ ऽय म॒य मुप॑हूतो दे॒वपु॑त्रे दे॒वपु॑त्रे॒ उप॑हूतो॒ ऽयम् । \newline
46. दे॒वपु॑त्रे॒ इति॑ दे॒व - पु॒त्रे॒ । \newline
47. उप॑हूतो॒ ऽय म॒य मुप॑हूत॒ उप॑हूतो॒ ऽयं ॅयज॑मानो॒ यज॑मानो॒ ऽय मुप॑हूत॒ उप॑हूतो॒ ऽयं ॅयज॑मानः । \newline
48. उप॑हूत॒ इत्युप॑ - हू॒तः॒ । \newline
49. अ॒यं ॅयज॑मानो॒ यज॑मानो॒ ऽय म॒यं ॅयज॑मान॒ इतीति॒ यज॑मानो॒ ऽय म॒यं ॅयज॑मान॒ इति॑ । \newline
50. यज॑मान॒ इतीति॒ यज॑मानो॒ यज॑मान॒ इत्या॑हा॒हे ति॒ यज॑मानो॒ यज॑मान॒ इत्या॑ह । \newline
51. इत्या॑हा॒हे तीत्या॑ह॒ यज॑मानं॒ ॅयज॑मान मा॒हे तीत्या॑ह॒ यज॑मानम् । \newline
52. आ॒ह॒ यज॑मानं॒ ॅयज॑मान माहाह॒ यज॑मान मे॒वैव यज॑मान माहाह॒ यज॑मान मे॒व । \newline
53. यज॑मान मे॒वैव यज॑मानं॒ ॅयज॑मान मे॒वोपो पै॒व यज॑मानं॒ ॅयज॑मान मे॒वोप॑ । \newline
54. ए॒वोपो पै॒वैवोप॑ ह्वयते ह्वयत॒ उपै॒वै वोप॑ ह्वयते । \newline
55. उप॑ ह्वयते ह्वयत॒ उपोप॑ ह्वयत॒ उत्त॑रस्या॒ मुत्त॑रस्याꣳ ह्वयत॒ उपोप॑ ह्वयत॒ उत्त॑रस्याम् । \newline
56. ह्व॒य॒त॒ उत्त॑रस्या॒ मुत्त॑रस्याꣳ ह्वयते ह्वयत॒ उत्त॑रस्याम् देवय॒ज्याया᳚म् देवय॒ज्याया॒ मुत्त॑रस्याꣳ ह्वयते ह्वयत॒ उत्त॑रस्याम् देवय॒ज्याया᳚म् । \newline
57. उत्त॑रस्याम् देवय॒ज्याया᳚म् देवय॒ज्याया॒ मुत्त॑रस्या॒ मुत्त॑रस्याम् देवय॒ज्याया॒ मुप॑हूत॒ उप॑हूतो देवय॒ज्याया॒ मुत्त॑रस्या॒ मुत्त॑रस्याम् देवय॒ज्याया॒ मुप॑हूतः । \newline
58. उत्त॑रस्या॒मित्युत् - त॒र॒स्या॒म् । \newline
59. दे॒व॒य॒ज्याया॒ मुप॑हूत॒ उप॑हूतो देवय॒ज्याया᳚म् देवय॒ज्याया॒ मुप॑हूतो॒ भूय॑सि॒ भूय॒ स्युप॑हूतो देवय॒ज्याया᳚म् देवय॒ज्याया॒ मुप॑हूतो॒ भूय॑सि । \newline
60. दे॒व॒य॒ज्याया॒मिति॑ देव - य॒ज्याया᳚म् । \newline
61. उप॑हूतो॒ भूय॑सि॒ भूय॒ स्युप॑हूत॒ उप॑हूतो॒ भूय॑सि हवि॒ष्कर॑णे हवि॒ष्कर॑णे॒ भूय॒ स्युप॑हूत॒ उप॑हूतो॒ भूय॑सि हवि॒ष्कर॑णे । \newline
62. उप॑हूत॒ इत्युप॑ - हू॒तः॒ । \newline
63. भूय॑सि हवि॒ष्कर॑णे हवि॒ष्कर॑णे॒ भूय॑सि॒ भूय॑सि हवि॒ष्कर॑ण॒ उप॑हूत॒ उप॑हूतो हवि॒ष्कर॑णे॒ भूय॑सि॒ भूय॑सि हवि॒ष्कर॑ण॒ उप॑हूतः । \newline
64. ह॒वि॒ष्कर॑ण॒ उप॑हूत॒ उप॑हूतो हवि॒ष्कर॑णे हवि॒ष्कर॑ण॒ उप॑हूतो दि॒व्ये दि॒व्य उप॑हूतो हवि॒ष्कर॑णे हवि॒ष्कर॑ण॒ उप॑हूतो दि॒व्ये । \newline
65. ह॒वि॒ष्कर॑ण॒ इति॑ हविः - कर॑णे । \newline
66. उप॑हूतो दि॒व्ये दि॒व्य उप॑हूत॒ उप॑हूतो दि॒व्ये धाम॒न् धाम॑न् दि॒व्य उप॑हूत॒ उप॑हूतो दि॒व्ये धामन्न्॑ । \newline
67. उप॑हूत॒ इत्युप॑ - हू॒तः॒ । \newline
68. दि॒व्ये धाम॒न् धाम॑न् दि॒व्ये दि॒व्ये धाम॒न् नुप॑हूत॒ उप॑हूतो॒ धाम॑न् दि॒व्ये दि॒व्ये धाम॒न् नुप॑हूतः । \newline
69. धाम॒न् नुप॑हूत॒ उप॑हूतो॒ धाम॒न् धाम॒न् नुप॑हूत॒ इती त्युप॑हूतो॒ धाम॒न् धाम॒न् नुप॑हूत॒ इति॑ । \newline
70. उप॑हूत॒ इती त्युप॑हूत॒ उप॑हूत॒ इत्या॑हा॒हे त्युप॑हूत॒ उप॑हूत॒ इत्या॑ह । \newline
71. उप॑हूत॒ इत्युप॑ - हू॒तः॒ । \newline
\pagebreak
\markright{ TS 2.6.7.6  \hfill https://www.vedavms.in \hfill}
\addcontentsline{toc}{section}{ TS 2.6.7.6 }
\section*{ TS 2.6.7.6 }

\textbf{TS 2.6.7.6 } \newline
\textbf{Samhita Paata} \newline

इत्या॑ह प्र॒जा वा उत्त॑रा देवय॒ज्या प॒शवो॒ भूयो॑ हवि॒ष्कर॑णꣳ सुव॒र्गो लो॒को दि॒व्यं धामे॒दम॑-सी॒दम॒सीत्ये॒व य॒ज्ञ्स्य॑ प्रि॒यं धामोप॑ ह्वयते॒ विश्व॑मस्य प्रि॒य-मुप॑हूत॒मित्या॒हा-छ॑बंट्कारमे॒वोप॑ ह्वयते ॥ \newline

\textbf{Pada Paata} \newline

इति॑ । आ॒ह॒ । प्र॒जेति॑ प्र - जा । वै । उत्त॒रेत्युत् - त॒रा॒ । दे॒व॒य॒ज्येति॑ देव - य॒ज्या । प॒शवः॑ । भूयः॑ । ह॒वि॒ष्कर॑ण॒मिति॑ हविः - कर॑णम् । सु॒व॒र्ग इति॑ सुवः-गः । लो॒कः । दि॒व्यम् । धाम॑ । इ॒दम् । अ॒सि॒ । इ॒दम् । अ॒सि॒ । इति॑ । ए॒व । य॒ज्ञ्स्य॑ । प्रि॒यम् । धाम॑ । उपेति॑ । ह्व॒य॒ते॒ । विश्व᳚म् । अ॒स्य॒ । प्रि॒यम् । उप॑हूत॒मित्युप॑ - हू॒त॒म् । इति॑ । आ॒ह॒ । अछ॑बंट्कार॒मित्यछ॑बंट् - का॒र॒म् । ए॒व । उपेति॑ । ह्व॒य॒ते॒ ॥  \newline


\textbf{Krama Paata} \newline

इत्या॑ह । आ॒ह॒ प्र॒जा । प्र॒जा वै । प्र॒जेति॑ प्र - जा । वा उत्त॑रा । उत्त॑रा देवय॒ज्या । उत्त॒रेत्युत् - त॒रा॒ । दे॒व॒य॒ज्या प॒शवः॑ । दे॒व॒य॒ज्येति॑ देव - य॒ज्या । प॒शवो॒ भूयः॑ । भूयो॑ हवि॒ष्कर॑णम् । ह॒वि॒ष्कर॑णꣳ सुव॒र्गः । ह॒वि॒ष्कर॑ण॒मिति॑ हविः - कर॑णम् । सु॒व॒र्गो लो॒कः । सु॒व॒र्ग इति॑ सुवः - गः । लो॒को दि॒व्यम् । दि॒व्यम् धाम॑ । धामे॒दम् । इ॒दम॑सि । अ॒सी॒दम् । इ॒दम॑सि । अ॒सीति॑ । इत्ये॒व । ए॒व य॒ज्ञ्स्य॑ । य॒ज्ञ्स्य॑ प्रि॒यम् । प्रि॒यम् धाम॑ । धामोप॑ । उप॑ ह्वयते । ह्व॒य॒ते॒ विश्व᳚म् । विश्व॑मस्य । अ॒स्य॒ प्रि॒यम् । प्रि॒यमुप॑हूतम् । उप॑हूत॒मिति॑ । उप॑हूत॒मित्युप॑ - हू॒त॒म् । इत्या॑ह । आ॒हाछ॑म्बट्कारम् । अछ॑म्बट्कारमे॒व । अछ॑म्बट्कार॒मित्यछ॑म्बट् - का॒र॒म् । ए॒वोप॑ । उप॑ ह्वयते । ह्व॒य॒त॒ इति॑ ह्वयते । \newline

\textbf{Jatai Paata} \newline

1. इत्या॑हा॒हे तीत्या॑ह । \newline
2. आ॒ह॒ प्र॒जा प्र॒जा ऽऽहा॑ह प्र॒जा । \newline
3. प्र॒जा वै वै प्र॒जा प्र॒जा वै । \newline
4. प्र॒जेति॑ प्र - जा । \newline
5. वा उत्त॒रोत्त॑रा॒ वै वा उत्त॑रा । \newline
6. उत्त॑रा देवय॒ज्या दे॑वय॒ ज्योत्त॒ रोत्त॑रा देवय॒ज्या । \newline
7. उत्त॒रेत्युत् - त॒रा॒ । \newline
8. दे॒व॒य॒ज्या प॒शवः॑ प॒शवो॑ देवय॒ज्या दे॑वय॒ज्या प॒शवः॑ । \newline
9. दे॒व॒य॒ज्येति॑ देव - य॒ज्या । \newline
10. प॒शवो॒ भूयो॒ भूयः॑ प॒शवः॑ प॒शवो॒ भूयः॑ । \newline
11. भूयो॑ हवि॒ष्कर॑णꣳ हवि॒ष्कर॑ण॒म् भूयो॒ भूयो॑ हवि॒ष्कर॑णम् । \newline
12. ह॒वि॒ष्कर॑णꣳ सुव॒र्गः सु॑व॒र्गो ह॑वि॒ष्कर॑णꣳ हवि॒ष्कर॑णꣳ सुव॒र्गः । \newline
13. ह॒वि॒ष्कर॑ण॒मिति॑ हविः - कर॑णम् । \newline
14. सु॒व॒र्गो लो॒को लो॒कः सु॑व॒र्गः सु॑व॒र्गो लो॒कः । \newline
15. सु॒व॒र्ग इति॑ सुवः - गः । \newline
16. लो॒को दि॒व्यम् दि॒व्यम् ॅलो॒को लो॒को दि॒व्यम् । \newline
17. दि॒व्यम् धाम॒ धाम॑ दि॒व्यम् दि॒व्यम् धाम॑ । \newline
18. धामे॒ द मि॒दम् धाम॒ धामे॒ दम् । \newline
19. इ॒द म॑स्यसी॒द मि॒द म॑सि । \newline
20. अ॒सी॒द मि॒द म॑स्यसी॒दम् । \newline
21. इ॒द म॑स्यसी॒द मि॒द म॑सि । \newline
22. अ॒सीतीत्य॑स्य॒सीति॑ । \newline
23. इत्ये॒वैवे तीत्ये॒व । \newline
24. ए॒व य॒ज्ञ्स्य॑ य॒ज्ञ् स्यै॒वैव य॒ज्ञ्स्य॑ । \newline
25. य॒ज्ञ्स्य॑ प्रि॒यम् प्रि॒यं ॅय॒ज्ञ्स्य॑ य॒ज्ञ्स्य॑ प्रि॒यम् । \newline
26. प्रि॒यम् धाम॒ धाम॑ प्रि॒यम् प्रि॒यम् धाम॑ । \newline
27. धामोपोप॒ धाम॒ धामोप॑ । \newline
28. उप॑ ह्वयते ह्वयत॒ उपोप॑ ह्वयते । \newline
29. ह्व॒य॒ते॒ विश्वं॒ ॅविश्वꣳ॑ ह्वयते ह्वयते॒ विश्व᳚म् । \newline
30. विश्व॑ मस्यास्य॒ विश्वं॒ ॅविश्व॑ मस्य । \newline
31. अ॒स्य॒ प्रि॒यम् प्रि॒य म॑स्यास्य प्रि॒यम् । \newline
32. प्रि॒य मुप॑हूत॒ मुप॑हूतम् प्रि॒यम् प्रि॒य मुप॑हूतम् । \newline
33. उप॑हूत॒ मिती त्युप॑हूत॒ मुप॑हूत॒ मिति॑ । \newline
34. उप॑हूत॒मित्युप॑ - हू॒त॒म् । \newline
35. इत्या॑हा॒हे तीत्या॑ह । \newline
36. आ॒हा छं॑बट्कार॒ मछं॑बट्कार माहा॒हा छं॑बट्कारम् । \newline
37. अछं॑बट्कार मे॒वैवा छं॑बट्कार॒ मछं॑बट्कार मे॒व । \newline
38. अछ॑म्बट्कार॒मित्यछ॑म्बट् - का॒र॒म् । \newline
39. ए॒वोपो पै॒वैवोप॑ । \newline
40. उप॑ ह्वयते ह्वयत॒ उपोप॑ ह्वयते । \newline
41. ह्व॒य॒त॒ इति॑ ह्वयते । \newline

\textbf{Ghana Paata } \newline

1. इत्या॑हा॒हे तीत्या॑ह प्र॒जा प्र॒जा ऽऽहे तीत्या॑ह प्र॒जा । \newline
2. आ॒ह॒ प्र॒जा प्र॒जा ऽऽहा॑ह प्र॒जा वै वै प्र॒जा ऽऽहा॑ह प्र॒जा वै । \newline
3. प्र॒जा वै वै प्र॒जा प्र॒जा वा उत्त॒रोत्त॑रा॒ वै प्र॒जा प्र॒जा वा उत्त॑रा । \newline
4. प्र॒जेति॑ प्र - जा । \newline
5. वा उत्त॒रोत्त॑रा॒ वै वा उत्त॑रा देवय॒ज्या दे॑वय॒ज्यो त्त॑रा॒ वै वा उत्त॑रा देवय॒ज्या । \newline
6. उत्त॑रा देवय॒ज्या दे॑वय॒ज्यो त्त॒रोत्त॑रा देवय॒ज्या प॒शवः॑ प॒शवो॑ देवय॒ज्यो त्त॒रोत्त॑रा देवय॒ज्या प॒शवः॑ । \newline
7. उत्त॒रेत्युत् - त॒रा॒ । \newline
8. दे॒व॒य॒ज्या प॒शवः॑ प॒शवो॑ देवय॒ज्या दे॑वय॒ज्या प॒शवो॒ भूयो॒ भूयः॑ प॒शवो॑ देवय॒ज्या दे॑वय॒ज्या प॒शवो॒ भूयः॑ । \newline
9. दे॒व॒य॒ज्येति॑ देव - य॒ज्या । \newline
10. प॒शवो॒ भूयो॒ भूयः॑ प॒शवः॑ प॒शवो॒ भूयो॑ हवि॒ष्कर॑णꣳ हवि॒ष्कर॑ण॒म् भूयः॑ प॒शवः॑ प॒शवो॒ भूयो॑ हवि॒ष्कर॑णम् । \newline
11. भूयो॑ हवि॒ष्कर॑णꣳ हवि॒ष्कर॑ण॒म् भूयो॒ भूयो॑ हवि॒ष्कर॑णꣳ सुव॒र्गः सु॑व॒र्गो ह॑वि॒ष्कर॑ण॒म् भूयो॒ भूयो॑ हवि॒ष्कर॑णꣳ सुव॒र्गः । \newline
12. ह॒वि॒ष्कर॑णꣳ सुव॒र्गः सु॑व॒र्गो ह॑वि॒ष्कर॑णꣳ हवि॒ष्कर॑णꣳ सुव॒र्गो लो॒को लो॒कः सु॑व॒र्गो ह॑वि॒ष्कर॑णꣳ हवि॒ष्कर॑णꣳ सुव॒र्गो लो॒कः । \newline
13. ह॒वि॒ष्कर॑ण॒मिति॑ हविः - कर॑णम् । \newline
14. सु॒व॒र्गो लो॒को लो॒कः सु॑व॒र्गः सु॑व॒र्गो लो॒को दि॒व्यम् दि॒व्यम् ॅलो॒कः सु॑व॒र्गः सु॑व॒र्गो लो॒को दि॒व्यम् । \newline
15. सु॒व॒र्ग इति॑ सुवः - गः । \newline
16. लो॒को दि॒व्यम् दि॒व्यम् ॅलो॒को लो॒को दि॒व्यम् धाम॒ धाम॑ दि॒व्यम् ॅलो॒को लो॒को दि॒व्यम् धाम॑ । \newline
17. दि॒व्यम् धाम॒ धाम॑ दि॒व्यम् दि॒व्यम् धामे॒ द मि॒दम् धाम॑ दि॒व्यम् दि॒व्यम् धामे॒ दम् । \newline
18. धामे॒ द मि॒दम् धाम॒ धामे॒ द म॑स्यसी॒दम् धाम॒ धामे॒ द म॑सि । \newline
19. इ॒द म॑स्यसी॒द मि॒द म॑सी॒द मि॒द म॑सी॒द मि॒द म॑सी॒दम् । \newline
20. अ॒सी॒द मि॒द म॑स्यसी॒द म॑स्यसी॒द म॑स्यसी॒द म॑सि । \newline
21. इ॒द म॑स्यसी॒द मि॒द म॒सीती त्य॑सी॒द मि॒द म॒सीति॑ । \newline
22. अ॒सीती त्य॑स्य॒सी त्ये॒वैवे त्य॑स्य॒सी त्ये॒व । \newline
23. इत्ये॒वैवे तीत्ये॒व य॒ज्ञ्स्य॑ य॒ज्ञ्स्यै॒वे तीत्ये॒व य॒ज्ञ्स्य॑ । \newline
24. ए॒व य॒ज्ञ्स्य॑ य॒ज्ञ्स्यै॒वैव य॒ज्ञ्स्य॑ प्रि॒यम् प्रि॒यं ॅय॒ज्ञ्स्यै॒वैव य॒ज्ञ्स्य॑ प्रि॒यम् । \newline
25. य॒ज्ञ्स्य॑ प्रि॒यम् प्रि॒यं ॅय॒ज्ञ्स्य॑ य॒ज्ञ्स्य॑ प्रि॒यम् धाम॒ धाम॑ प्रि॒यं ॅय॒ज्ञ्स्य॑ य॒ज्ञ्स्य॑ प्रि॒यम् धाम॑ । \newline
26. प्रि॒यम् धाम॒ धाम॑ प्रि॒यम् प्रि॒यम् धामोपोप॒ धाम॑ प्रि॒यम् प्रि॒यम् धामोप॑ । \newline
27. धामोपोप॒ धाम॒ धामोप॑ ह्वयते ह्वयत॒ उप॒ धाम॒ धामोप॑ ह्वयते । \newline
28. उप॑ ह्वयते ह्वयत॒ उपोप॑ ह्वयते॒ विश्वं॒ ॅविश्वꣳ॑ ह्वयत॒ उपोप॑ ह्वयते॒ विश्व᳚म् । \newline
29. ह्व॒य॒ते॒ विश्वं॒ ॅविश्वꣳ॑ ह्वयते ह्वयते॒ विश्व॑ मस्यास्य॒ विश्वꣳ॑ ह्वयते ह्वयते॒ विश्व॑ मस्य । \newline
30. विश्व॑ मस्यास्य॒ विश्वं॒ ॅविश्व॑ मस्य प्रि॒यम् प्रि॒य म॑स्य॒ विश्वं॒ ॅविश्व॑ मस्य प्रि॒यम् । \newline
31. अ॒स्य॒ प्रि॒यम् प्रि॒य म॑स्यास्य प्रि॒य मुप॑हूत॒ मुप॑हूतम् प्रि॒य म॑स्यास्य प्रि॒य मुप॑हूतम् । \newline
32. प्रि॒य मुप॑हूत॒ मुप॑हूतम् प्रि॒यम् प्रि॒य मुप॑हूत॒ मिती त्युप॑हूतम् प्रि॒यम् प्रि॒य मुप॑हूत॒ मिति॑ । \newline
33. उप॑हूत॒ मिती त्युप॑हूत॒ मुप॑हूत॒ मित्या॑हा॒हे त्युप॑हूत॒ मुप॑हूत॒ मित्या॑ह । \newline
34. उप॑हूत॒मित्युप॑ - हू॒त॒म् । \newline
35. इत्या॑हा॒हे तीत्या॒हाछं॑बट्कार॒ मछं॑बट्कार मा॒हे तीत्या॒हाछं॑बट्कारम् । \newline
36. आ॒हाछं॑बट्कार॒ मछं॑बट्कार माहा॒हा छं॑बट्कार मे॒वैवा छं॑बट्कार माहा॒हा छं॑बट्कार मे॒व । \newline
37. अछं॑बट्कार मे॒वैवा छं॑बट्कार॒ मछं॑बट्कार मे॒वोपोपै॒वा छं॑बट्कार॒ मछं॑बट्कार मे॒वोप॑ । \newline
38. अछ॑म्बट्कार॒मित्यछ॑म्बट् - का॒र॒म् । \newline
39. ए॒वोपो पै॒वैवोप॑ ह्वयते ह्वयत॒ उपै॒वै वोप॑ ह्वयते । \newline
40. उप॑ ह्वयते ह्वयत॒ उपोप॑ ह्वयते । \newline
41. ह्व॒य॒त॒ इति॑ ह्वयते । \newline
\pagebreak
\markright{ TS 2.6.8.1  \hfill https://www.vedavms.in \hfill}
\addcontentsline{toc}{section}{ TS 2.6.8.1 }
\section*{ TS 2.6.8.1 }

\textbf{TS 2.6.8.1 } \newline
\textbf{Samhita Paata} \newline

प॒शवो॒ वा इडा᳚ स्व॒यमा द॑त्ते॒ काम॑मे॒वाऽऽत्मना॑ पशू॒नामा द॑त्ते॒ न ह्य॑न्यः कामं॑ पशू॒नां प्र॒यच्छ॑ति वा॒चस्पत॑ये त्वा हु॒तं प्राश्ना॒मीत्या॑ह॒ वाच॑मे॒व भा॑ग॒धेये॑न प्रीणाति॒ सद॑स॒स्पत॑ये त्वा हु॒तं प्राश्ना॒मीत्या॑ह स्व॒गाकृ॑त्यै चतुरव॒त्तं भ॑वति ह॒विर्वै च॑तुरव॒त्तं प॒शव॑श्चतुरव॒त्तं ॅयद्धोता᳚ प्राश्नी॒याद्धोता - [  ] \newline

\textbf{Pada Paata} \newline

प॒शवः॑ । वै । इडा᳚ । स्व॒यम् । एति॑ । द॒त्ते॒ । काम᳚म् । ए॒व । आ॒त्मना᳚ । प॒शू॒नाम् । एति॑ । द॒त्ते॒ । न । हि । अ॒न्यः । काम᳚म् । प॒शू॒नाम् । प्र॒यच्छ॒तीति॑ प्र - यच्छ॑ति । वा॒चः । पत॑ये । त्वा॒ । हु॒तम् । प्रेति॑ । अ॒श्ना॒मि॒ । इति॑ । आ॒ह॒ । वाच᳚म् । ए॒व । भा॒ग॒धेये॒नेति॑ भाग - धेये॑न । प्री॒णा॒ति॒ । सद॑सः । पत॑ये । त्वा॒ । हु॒तम् । प्रेति॑ । अ॒श्ना॒मि॒ । इति॑ । आ॒ह॒ । स्व॒गाकृ॑त्या॒ इति॑ स्व॒गा - कृ॒त्यै॒ । च॒तु॒र॒व॒त्तमिति॑ चतुः - अ॒व॒त्तम् । भ॒व॒ति॒ । ह॒विः । वै । च॒तु॒र॒व॒त्तमिति॑ चतुः - अ॒व॒त्तम् । प॒शवः॑ । च॒तु॒र॒व॒त्तमिति॑ चतुः - अ॒व॒त्तम् । यत् । होता᳚ । प्रा॒श्नी॒यादिति॑ प्र - अ॒श्नी॒यात् । होता᳚ ।  \newline


\textbf{Krama Paata} \newline

प॒शवो॒ वै । वा इडा᳚ । इडा᳚ स्व॒यम् । स्व॒यमा । आ द॑त्ते । द॒त्ते॒ काम᳚म् । काम॑मे॒व । ए॒वात्मना᳚ । आ॒त्मना॑ पशू॒नाम् । प॒शू॒नामा । आ द॑त्ते । द॒त्ते॒ न । न हि । ह्य॑न्यः । अ॒न्यः काम᳚म् । काम॑म् पशू॒नाम् । प॒शू॒नाम् प्र॒यच्छ॑ति । प्र॒यच्छ॑ति वा॒चः । प्र॒यच्छ॒तीति॑ प्र - यच्छ॑ति । वा॒चस्पत॑ये । पत॑ये त्वा । त्वा॒ हु॒तम् । हु॒तम् प्र । प्राश्ञा॑मि । अ॒श्ञा॒मीति॑ । इत्या॑ह । आ॒ह॒ वाच᳚म् । वाच॑मे॒व । ए॒व भा॑ग॒धेये॑न । भा॒ग॒धेये॑न प्रीणाति । भा॒ग॒धेये॒नेति॑ भाग - धेये॑न । प्री॒णा॒ति॒ सद॑सः । सद॑स॒स्पत॑ये । पत॑ये त्वा । त्वा॒ हु॒तम् । हु॒तम् प्र । प्राश्ञा॑मि । अ॒श्ञा॒मीति॑ । इत्या॑ह । आ॒ह॒ स्व॒गाकृ॑त्यै । स्व॒गाकृ॑त्यै चतुरव॒त्तम् । स्व॒गाकृ॑त्या॒ इति॑ स्व॒गा - कृ॒त्यै॒ । च॒तु॒र॒व॒त्तम् भ॑वति । च॒तु॒र॒व॒त्तमिति॑ चतुः - अ॒व॒त्तम् । भ॒व॒ति॒ ह॒विः । ह॒विर् वै । वै च॑तुरव॒त्तम् । च॒तु॒र॒व॒त्तम् प॒शवः॑ । च॒तु॒र॒व॒त्तमिति॑ चतुः - अ॒व॒त्तम् । प॒शव॑श्चतुरव॒त्तम् । च॒तु॒र॒व॒त्तं ॅयत् । च॒तु॒र॒व॒त्तमिति॑ चतुः - अ॒व॒त्तम् । यद्धोता᳚ । होता᳚ प्राश्ञी॒यात् । प्रा॒श्ञी॒याद्धोता᳚ । प्रा॒श्ञी॒यादिति॑ प्र - अ॒श्ञी॒यात् । होता ऽऽर्ति᳚म् \newline

\textbf{Jatai Paata} \newline

1. प॒शवो॒ वै वै प॒शवः॑ प॒शवो॒ वै । \newline
2. वा इडेडा॒ वै वा इडा᳚ । \newline
3. इडा᳚ स्व॒यꣳ स्व॒य मिडेडा᳚ स्व॒यम् । \newline
4. स्व॒य मा स्व॒यꣳ स्व॒य मा । \newline
5. आ द॑त्ते दत्त॒ आ द॑त्ते । \newline
6. द॒त्ते॒ काम॒म् काम॑म् दत्ते दत्ते॒ काम᳚म् । \newline
7. काम॑ मे॒वैव काम॒म् काम॑ मे॒व । \newline
8. ए॒वा त्मना॒ ऽऽत्मनै॒वै वात्मना᳚ । \newline
9. आ॒त्मना॑ पशू॒नाम् प॑शू॒ना मा॒त्मना॒ ऽऽत्मना॑ पशू॒नाम् । \newline
10. प॒शू॒ना मा प॑शू॒नाम् प॑शू॒ना मा । \newline
11. आ द॑त्ते दत्त॒ आ द॑त्ते । \newline
12. द॒त्ते॒ न न द॑त्ते दत्ते॒ न । \newline
13. न हि हि न न हि । \newline
14. ह्या᳚(1॒)न्यो᳚ ऽन्यो हि ह्य॑न्यः । \newline
15. अ॒न्यः काम॒म् काम॑ म॒न्यो᳚ ऽन्यः काम᳚म् । \newline
16. काम॑म् पशू॒नाम् प॑शू॒नाम् काम॒म् काम॑म् पशू॒नाम् । \newline
17. प॒शू॒नाम् प्र॒यच्छ॑ति प्र॒यच्छ॑ति पशू॒नाम् प॑शू॒नाम् प्र॒यच्छ॑ति । \newline
18. प्र॒यच्छ॑ति वा॒चो वा॒चः प्र॒यच्छ॑ति प्र॒यच्छ॑ति वा॒चः । \newline
19. प्र॒यच्छ॒तीति॑ प्र - यच्छ॑ति । \newline
20. वा॒चस् पत॑ये॒ पत॑ये वा॒चो वा॒चस् पत॑ये । \newline
21. पत॑ये त्वा त्वा॒ पत॑ये॒ पत॑ये त्वा । \newline
22. त्वा॒ हु॒तꣳ हु॒तम् त्वा᳚ त्वा हु॒तम् । \newline
23. हु॒तम् प्र प्र हु॒तꣳ हु॒तम् प्र । \newline
24. प्राश्ञा᳚ म्यश्ञामि॒ प्र प्राश्ञा॑मि । \newline
25. अ॒श्ञा॒मीती त्य॑श्ञा म्यश्ञा॒मीति॑ । \newline
26. इत्या॑हा॒हे तीत्या॑ह । \newline
27. आ॒ह॒ वाचं॒ ॅवाच॑ माहाह॒ वाच᳚म् । \newline
28. वाच॑ मे॒वैव वाचं॒ ॅवाच॑ मे॒व । \newline
29. ए॒व भा॑ग॒धेये॑न भाग॒धेये॑ नै॒वैव भा॑ग॒धेये॑न । \newline
30. भा॒ग॒धेये॑न प्रीणाति प्रीणाति भाग॒धेये॑न भाग॒धेये॑न प्रीणाति । \newline
31. भा॒ग॒धेये॒नेति॑ भाग - धेये॑न । \newline
32. प्री॒णा॒ति॒ सद॑सः॒ सद॑सः प्रीणाति प्रीणाति॒ सद॑सः । \newline
33. सद॑स॒स् पत॑ये॒ पत॑ये॒ सद॑सः॒ सद॑स॒स् पत॑ये । \newline
34. पत॑ये त्वा त्वा॒ पत॑ये॒ पत॑ये त्वा । \newline
35. त्वा॒ हु॒तꣳ हु॒तम् त्वा᳚ त्वा हु॒तम् । \newline
36. हु॒तम् प्र प्र हु॒तꣳ हु॒तम् प्र । \newline
37. प्राश्ञा᳚ म्यश्ञामि॒ प्र प्राश्ञा॑मि । \newline
38. अ॒श्ञा॒मीती त्य॑श्ञा म्यश्ञा॒मीति॑ । \newline
39. इत्या॑हा॒हे तीत्या॑ह । \newline
40. आ॒ह॒ स्व॒गाकृ॑त्यै स्व॒गाकृ॑त्या आहाह स्व॒गाकृ॑त्यै । \newline
41. स्व॒गाकृ॑त्यै चतुरव॒त्तम् च॑तुरव॒त्तꣳ स्व॒गाकृ॑त्यै स्व॒गाकृ॑त्यै चतुरव॒त्तम् । \newline
42. स्व॒गाकृ॑त्या॒ इति॑ स्व॒गा - कृ॒त्यै॒ । \newline
43. च॒तु॒र॒व॒त्तम् भ॑वति भवति चतुरव॒त्तम् च॑तुरव॒त्तम् भ॑वति । \newline
44. च॒तु॒र॒व॒त्तमिति॑ चतुः - अ॒व॒त्तम् । \newline
45. भ॒व॒ति॒ ह॒विर्. ह॒विर् भ॑वति भवति ह॒विः । \newline
46. ह॒विर् वै वै ह॒विर्. ह॒विर् वै । \newline
47. वै च॑तुरव॒त्तम् च॑तुरव॒त्तं ॅवै वै च॑तुरव॒त्तम् । \newline
48. च॒तु॒र॒व॒त्तम् प॒शवः॑ प॒शव॑ श्चतुरव॒त्तम् च॑तुरव॒त्तम् प॒शवः॑ । \newline
49. च॒तु॒र॒व॒त्तमिति॑ चतुः - अ॒व॒त्तम् । \newline
50. प॒शव॑ श्चतुरव॒त्तम् च॑तुरव॒त्तम् प॒शवः॑ प॒शव॑ श्चतुरव॒त्तम् । \newline
51. च॒तु॒र॒व॒त्तं ॅयद् यच् च॑तुरव॒त्तम् च॑तुरव॒त्तं ॅयत् । \newline
52. च॒तु॒र॒व॒त्तमिति॑ चतुः - अ॒व॒त्तम् । \newline
53. यद्धोता॒ होता॒ यद् यद्धोता᳚ । \newline
54. होता᳚ प्राश्ञी॒यात् प्रा᳚श्ञी॒या द्धोता॒ होता᳚ प्राश्ञी॒यात् । \newline
55. प्रा॒श्ञी॒या द्धोता॒ होता᳚ प्राश्ञी॒यात् प्रा᳚श्ञी॒या द्धोता᳚ । \newline
56. प्रा॒श्ञी॒यादिति॑ प्र - अ॒श्ञी॒यात् । \newline
57. होता ऽऽर्ति॒ मार्तिꣳ॒॒ होता॒ होता ऽऽर्ति᳚म् । \newline

\textbf{Ghana Paata } \newline

1. प॒शवो॒ वै वै प॒शवः॑ प॒शवो॒ वा इडेडा॒ वै प॒शवः॑ प॒शवो॒ वा इडा᳚ । \newline
2. वा इडेडा॒ वै वा इडा᳚ स्व॒यꣳ स्व॒य मिडा॒ वै वा इडा᳚ स्व॒यम् । \newline
3. इडा᳚ स्व॒यꣳ स्व॒य मिडेडा᳚ स्व॒य मा स्व॒य मिडेडा᳚ स्व॒य मा । \newline
4. स्व॒य मा स्व॒यꣳ स्व॒य मा द॑त्ते दत्त॒ आ स्व॒यꣳ स्व॒य मा द॑त्ते । \newline
5. आ द॑त्ते दत्त॒ आ द॑त्ते॒ काम॒म् काम॑म् दत्त॒ आ द॑त्ते॒ काम᳚म् । \newline
6. द॒त्ते॒ काम॒म् काम॑म् दत्ते दत्ते॒ काम॑ मे॒वैव काम॑म् दत्ते दत्ते॒ काम॑ मे॒व । \newline
7. काम॑ मे॒वैव काम॒म् काम॑ मे॒वात्मना॒ ऽऽत्मनै॒व काम॒म् काम॑ मे॒वात्मना᳚ । \newline
8. ए॒वात्मना॒ ऽऽत्मनै॒वै वात्मना॑ पशू॒नाम् प॑शू॒ना मा॒त्मनै॒वै वात्मना॑ पशू॒नाम् । \newline
9. आ॒त्मना॑ पशू॒नाम् प॑शू॒ना मा॒त्मना॒ ऽऽत्मना॑ पशू॒ना मा प॑शू॒ना मा॒त्मना॒ ऽऽत्मना॑ पशू॒ना मा । \newline
10. प॒शू॒ना मा प॑शू॒नाम् प॑शू॒ना मा द॑त्ते दत्त॒ आ प॑शू॒नाम् प॑शू॒ना मा द॑त्ते । \newline
11. आ द॑त्ते दत्त॒ आ द॑त्ते॒ न न द॑त्त॒ आ द॑त्ते॒ न । \newline
12. द॒त्ते॒ न न द॑त्ते दत्ते॒ न हि हि न द॑त्ते दत्ते॒ न हि । \newline
13. न हि हि न न ह्या᳚(1॒)न्यो᳚ ऽन्यो हि न न ह्य॑न्यः । \newline
14. ह्या᳚(1॒)न्यो᳚ ऽन्यो हि ह्य॑न्यः काम॒म् काम॑ म॒न्यो हि ह्य॑न्यः काम᳚म् । \newline
15. अ॒न्यः काम॒म् काम॑ म॒न्यो᳚ ऽन्यः काम॑म् पशू॒नाम् प॑शू॒नाम् काम॑ म॒न्यो᳚ ऽन्यः काम॑म् पशू॒नाम् । \newline
16. काम॑म् पशू॒नाम् प॑शू॒नाम् काम॒म् काम॑म् पशू॒नाम् प्र॒यच्छ॑ति प्र॒यच्छ॑ति पशू॒नाम् काम॒म् काम॑म् पशू॒नाम् प्र॒यच्छ॑ति । \newline
17. प॒शू॒नाम् प्र॒यच्छ॑ति प्र॒यच्छ॑ति पशू॒नाम् प॑शू॒नाम् प्र॒यच्छ॑ति वा॒चो वा॒चः प्र॒यच्छ॑ति पशू॒नाम् प॑शू॒नाम् प्र॒यच्छ॑ति वा॒चः । \newline
18. प्र॒यच्छ॑ति वा॒चो वा॒चः प्र॒यच्छ॑ति प्र॒यच्छ॑ति वा॒च स्पत॑ये॒ पत॑ये वा॒चः प्र॒यच्छ॑ति प्र॒यच्छ॑ति वा॒च स्पत॑ये । \newline
19. प्र॒यच्छ॒तीति॑ प्र - यच्छ॑ति । \newline
20. वा॒च स्पत॑ये॒ पत॑ये वा॒चो वा॒च स्पत॑ये त्वा त्वा॒ पत॑ये वा॒चो वा॒च स्पत॑ये त्वा । \newline
21. पत॑ये त्वा त्वा॒ पत॑ये॒ पत॑ये त्वा हु॒तꣳ हु॒तम् त्वा॒ पत॑ये॒ पत॑ये त्वा हु॒तम् । \newline
22. त्वा॒ हु॒तꣳ हु॒तम् त्वा᳚ त्वा हु॒तम् प्र प्र हु॒तम् त्वा᳚ त्वा हु॒तम् प्र । \newline
23. हु॒तम् प्र प्र हु॒तꣳ हु॒तम् प्राश्ञा᳚ म्यश्ञामि॒ प्र हु॒तꣳ हु॒तम् प्राश्ञा॑मि । \newline
24. प्राश्ञा᳚ म्यश्ञामि॒ प्र प्राश्ञा॒मीती त्य॑श्ञामि॒ प्र प्राश्ञा॒मीति॑ । \newline
25. अ॒श्ञा॒मीती त्य॑श्ञा म्यश्ञा॒मी त्या॑हा॒हे त्य॑श्ञा म्यश्ञा॒मीत्या॑ह । \newline
26. इत्या॑हा॒हे तीत्या॑ह॒ वाचं॒ ॅवाच॑ मा॒हे तीत्या॑ह॒ वाच᳚म् । \newline
27. आ॒ह॒ वाचं॒ ॅवाच॑ माहाह॒ वाच॑ मे॒वैव वाच॑ माहाह॒ वाच॑ मे॒व । \newline
28. वाच॑ मे॒वैव वाचं॒ ॅवाच॑ मे॒व भा॑ग॒धेये॑न भाग॒धेये॑नै॒व वाचं॒ ॅवाच॑ मे॒व भा॑ग॒धेये॑न । \newline
29. ए॒व भा॑ग॒धेये॑न भाग॒धेये॑ नै॒वैव भा॑ग॒धेये॑न प्रीणाति प्रीणाति भाग॒धेये॑ नै॒वैव भा॑ग॒धेये॑न प्रीणाति । \newline
30. भा॒ग॒धेये॑न प्रीणाति प्रीणाति भाग॒धेये॑न भाग॒धेये॑न प्रीणाति॒ सद॑सः॒ सद॑सः प्रीणाति भाग॒धेये॑न भाग॒धेये॑न प्रीणाति॒ सद॑सः । \newline
31. भा॒ग॒धेये॒नेति॑ भाग - धेये॑न । \newline
32. प्री॒णा॒ति॒ सद॑सः॒ सद॑सः प्रीणाति प्रीणाति॒ सद॑स॒ स्पत॑ये॒ पत॑ये॒ सद॑सः प्रीणाति प्रीणाति॒ सद॑स॒ स्पत॑ये । \newline
33. सद॑स॒ स्पत॑ये॒ पत॑ये॒ सद॑सः॒ सद॑स॒ स्पत॑ये त्वा त्वा॒ पत॑ये॒ सद॑सः॒ सद॑स॒ स्पत॑ये त्वा । \newline
34. पत॑ये त्वा त्वा॒ पत॑ये॒ पत॑ये त्वा हु॒तꣳ हु॒तम् त्वा॒ पत॑ये॒ पत॑ये त्वा हु॒तम् । \newline
35. त्वा॒ हु॒तꣳ हु॒तम् त्वा᳚ त्वा हु॒तम् प्र प्र हु॒तम् त्वा᳚ त्वा हु॒तम् प्र । \newline
36. हु॒तम् प्र प्र हु॒तꣳ हु॒तम् प्राश्ञा᳚ म्यश्ञामि॒ प्र हु॒तꣳ हु॒तम् प्राश्ञा॑मि । \newline
37. प्राश्ञा᳚ म्यश्ञामि॒ प्र प्राश्ञा॒मीती त्य॑श्ञामि॒ प्र प्राश्ञा॒मीति॑ । \newline
38. अ॒श्ञा॒मीती त्य॑श्ञा म्यश्ञा॒मी त्या॑हा॒हे त्य॑श्ञा म्यश्ञा॒मी त्या॑ह । \newline
39. इत्या॑हा॒हे तीत्या॑ह स्व॒गाकृ॑त्यै स्व॒गाकृ॑त्या आ॒हे तीत्या॑ह स्व॒गाकृ॑त्यै । \newline
40. आ॒ह॒ स्व॒गाकृ॑त्यै स्व॒गाकृ॑त्या आहाह स्व॒गाकृ॑त्यै चतुरव॒त्तम् च॑तुरव॒त्तꣳ स्व॒गाकृ॑त्या आहाह स्व॒गाकृ॑त्यै चतुरव॒त्तम् । \newline
41. स्व॒गाकृ॑त्यै चतुरव॒त्तम् च॑तुरव॒त्तꣳ स्व॒गाकृ॑त्यै स्व॒गाकृ॑त्यै चतुरव॒त्तम् भ॑वति भवति चतुरव॒त्तꣳ स्व॒गाकृ॑त्यै स्व॒गाकृ॑त्यै चतुरव॒त्तम् भ॑वति । \newline
42. स्व॒गाकृ॑त्या॒ इति॑ स्व॒गा - कृ॒त्यै॒ । \newline
43. च॒तु॒र॒व॒त्तम् भ॑वति भवति चतुरव॒त्तम् च॑तुरव॒त्तम् भ॑वति ह॒विर्. ह॒विर् भ॑वति चतुरव॒त्तम् च॑तुरव॒त्तम् भ॑वति ह॒विः । \newline
44. च॒तु॒र॒व॒त्तमिति॑ चतुः - अ॒व॒त्तम् । \newline
45. भ॒व॒ति॒ ह॒विर्. ह॒विर् भ॑वति भवति ह॒विर् वै वै ह॒विर् भ॑वति भवति ह॒विर् वै । \newline
46. ह॒विर् वै वै ह॒विर्. ह॒विर् वै च॑तुरव॒त्तम् च॑तुरव॒त्तं ॅवै ह॒विर्. ह॒विर् वै च॑तुरव॒त्तम् । \newline
47. वै च॑तुरव॒त्तम् च॑तुरव॒त्तं ॅवै वै च॑तुरव॒त्तम् प॒शवः॑ प॒शव॑ श्चतुरव॒त्तं ॅवै वै च॑तुरव॒त्तम् प॒शवः॑ । \newline
48. च॒तु॒र॒व॒त्तम् प॒शवः॑ प॒शव॑ श्चतुरव॒त्तम् च॑तुरव॒त्तम् प॒शव॑ श्चतुरव॒त्तम् च॑तुरव॒त्तम् प॒शव॑ श्चतुरव॒त्तम् च॑तुरव॒त्तम् प॒शव॑ श्चतुरव॒त्तम् । \newline
49. च॒तु॒र॒व॒त्तमिति॑ चतुः - अ॒व॒त्तम् । \newline
50. प॒शव॑ श्चतुरव॒त्तम् च॑तुरव॒त्तम् प॒शवः॑ प॒शव॑ श्चतुरव॒त्तं ॅयद् यच् च॑तुरव॒त्तम् प॒शवः॑ प॒शव॑ श्चतुरव॒त्तं ॅयत् । \newline
51. च॒तु॒र॒व॒त्तं ॅयद् यच् च॑तुरव॒त्तम् च॑तुरव॒त्तं ॅयद्धोता॒ होता॒ यच् च॑तुरव॒त्तम् च॑तुरव॒त्तं ॅयद्धोता᳚ । \newline
52. च॒तु॒र॒व॒त्तमिति॑ चतुः - अ॒व॒त्तम् । \newline
53. यद्धोता॒ होता॒ यद् यद्धोता᳚ प्राश्ञी॒यात् प्रा᳚श्ञी॒या द्धोता॒ यद् यद्धोता᳚ प्राश्ञी॒यात् । \newline
54. होता᳚ प्राश्ञी॒यात् प्रा᳚श्ञी॒या द्धोता॒ होता᳚ प्राश्ञी॒या द्धोता॒ होता᳚ प्राश्ञी॒या द्धोता॒ होता᳚ प्राश्ञी॒या द्धोता᳚ । \newline
55. प्रा॒श्ञी॒या द्धोता॒ होता᳚ प्राश्ञी॒यात् प्रा᳚श्ञी॒या द्धोता ऽऽर्ति॒ मार्तिꣳ॒॒ होता᳚ प्राश्ञी॒यात् प्रा᳚श्ञी॒या द्धोता ऽऽर्ति᳚म् । \newline
56. प्रा॒श्ञी॒यादिति॑ प्र - अ॒श्ञी॒यात् । \newline
57. होता ऽऽर्ति॒ मार्तिꣳ॒॒ होता॒ होता ऽऽर्ति॒ मा ऽऽर्तिꣳ॒॒ होता॒ होता ऽऽर्ति॒ मा । \newline
\pagebreak
\markright{ TS 2.6.8.2  \hfill https://www.vedavms.in \hfill}
\addcontentsline{toc}{section}{ TS 2.6.8.2 }
\section*{ TS 2.6.8.2 }

\textbf{TS 2.6.8.2 } \newline
\textbf{Samhita Paata} \newline

ऽऽर्ति॒मार्च्छे॒द्-यद॒ग्नौ जु॑हु॒याद्-रु॒द्राय॑ प॒शूनपि॑ दद्ध्यादप॒शुर्यज॑मानः स्याद्-वा॒चस्पत॑ये त्वा हु॒तं प्राश्ना॒मीत्या॑ह प॒रोक्ष॑मे॒वैन॑-ज्जुहोति॒ सद॑स॒स्पत॑ये त्वा हु॒तं प्राश्ना॒मीत्या॑ह स्व॒गाकृ॑त्यै॒ प्राश्न॑न्ति ती॒र्थ ए॒व प्राश्न॑न्ति॒ दक्षि॑णां ददाति ती॒र्थ ए॒व दक्षि॑णां ददाति॒ वि वा ए॒तद्य॒ज्ञ्ं - [  ] \newline

\textbf{Pada Paata} \newline

आर्ति᳚म् । एति॑ । ऋ॒च्छे॒त् । यत् । अ॒ग्नौ । जु॒हु॒यात् । रु॒द्राय॑ । प॒शून् । अपीति॑ । द॒द्ध्या॒त् । अ॒प॒शुः । यज॑मानः । स्या॒त् । वा॒चः । पत॑ये । त्वा॒ । हु॒तम् । प्रेति॑ । अ॒श्ना॒मि॒ । इति॑ । आ॒ह॒ । प॒रोक्ष॒मिति॑ परः - अक्ष᳚म् । ए॒व । ए॒न॒त् । जु॒हो॒ति॒ । सद॑सः । पत॑ये । त्वा॒ । हु॒तम् । प्रेति॑ । अ॒श्ना॒मि॒ । इति॑ । आ॒ह॒ । स्व॒गाकृ॑त्या॒ इति॑ स्व॒गा - कृ॒त्यै॒ । प्रेति॑ । अ॒श्न॒न्ति॒ । ती॒र्त्थे । ए॒व । प्रेति॑ । अ॒श्न॒न्ति॒ । दक्षि॑णाम् । द॒दा॒ति॒ । ती॒र्त्थे । ए॒व । दक्षि॑णाम् । द॒दा॒ति॒ । वीति॑ । वै । ए॒तत् । य॒ज्ञ्म् ।  \newline


\textbf{Krama Paata} \newline

आर्ति॒मा । आर्च्छे᳚त् । ऋ॒च्छे॒द् यत् । यद॒ग्नौ । अ॒ग्नौ जु॑हु॒यात् । जु॒हु॒याद् रु॒द्राय॑ । रु॒द्राय॑ प॒शून् । प॒शूनपि॑ । अपि॑ दद्ध्यात् । द॒द्ध्या॒द॒प॒शुः । अ॒प॒शुर् यज॑मानः । यज॑मानः स्यात् । स्या॒द् वा॒चः । वा॒चस्पत॑ये । पत॑ये त्वा । त्वा॒ हु॒तम् । हु॒तम् प्र । प्राश्ञा॑मि । अ॒श्ञा॒मीति॑ । इत्या॑ह । आ॒ह॒ प॒रोक्ष᳚म् । प॒रोक्ष॑मे॒व । प॒रोक्ष॒मिति॑ परः - अक्ष᳚म् । ए॒वैन॑त् । ए॒न॒ज् जु॒हो॒ति॒ । जु॒हो॒ति॒ सद॑सः । सद॑स॒स्पत॑ये । पत॑ये त्वा । त्वा॒ हु॒तम् । हु॒तम् प्र । प्राश्ञा॑मि । अ॒श्ञा॒मीति॑ । इत्या॑ह । आ॒ह॒ स्व॒गाकृ॑त्यै । स्व॒गाकृ॑त्यै॒ प्र । स्व॒गाकृ॑त्या॒ इति॑ स्व॒गा - कृ॒त्यै॒ । प्राश्ञ॑न्ति । अ॒श्ञ॒न्ति॒ ती॒र्त्थे । ती॒र्त्थ ए॒व । ए॒व प्र । प्राश्ञ॑न्ति । अ॒श्ञ॒न्ति॒ दक्षि॑णाम् । दक्षि॑णाम् ददाति । द॒दा॒ति॒ ती॒र्त्थे । ती॒र्त्थ ए॒व । ए॒व दक्षि॑णाम् । दक्षि॑णाम् ददाति । द॒दा॒ति॒ वि । वि वै । वा ए॒तत् । ए॒तद् य॒ज्ञ्म् । य॒ज्ञ्म् छि॑न्दन्ति \newline

\textbf{Jatai Paata} \newline

1. आर्ति॒ मा ऽऽर्ति॒ मार्ति॒ मा । \newline
2. आर्च्छे॑ दृच्छे॒ दार्च्छे᳚त् । \newline
3. ऋ॒च्छे॒द् यद् यदृ॑च्छे दृच्छे॒द् यत् । \newline
4. यद॒ग्ना व॒ग्नौ यद् यद॒ग्नौ । \newline
5. अ॒ग्नौ जु॑हु॒याज् जु॑हु॒या द॒ग्ना व॒ग्नौ जु॑हु॒यात् । \newline
6. जु॒हु॒याद् रु॒द्राय॑ रु॒द्राय॑ जुहु॒याज् जु॑हु॒याद् रु॒द्राय॑ । \newline
7. रु॒द्राय॑ प॒शून् प॒शून् रु॒द्राय॑ रु॒द्राय॑ प॒शून् । \newline
8. प॒शू नप्यपि॑ प॒शून् प॒शू नपि॑ । \newline
9. अपि॑ दद्ध्याद् दद्ध्या॒ दप्यपि॑ दद्ध्यात् । \newline
10. द॒द्ध्या॒ द॒प॒शु र॑प॒शुर् द॑द्ध्याद् दद्ध्या दप॒शुः । \newline
11. अ॒प॒शुर् यज॑मानो॒ यज॑मानो ऽप॒शु र॑प॒शुर् यज॑मानः । \newline
12. यज॑मानः स्याथ् स्या॒द् यज॑मानो॒ यज॑मानः स्यात् । \newline
13. स्या॒द् वा॒चो वा॒चः स्या᳚थ् स्याद् वा॒चः । \newline
14. वा॒च स्पत॑ये॒ पत॑ये वा॒चो वा॒च स्पत॑ये । \newline
15. पत॑ये त्वा त्वा॒ पत॑ये॒ पत॑ये त्वा । \newline
16. त्वा॒ हु॒तꣳ हु॒तम् त्वा᳚ त्वा हु॒तम् । \newline
17. हु॒तम् प्र प्र हु॒तꣳ हु॒तम् प्र । \newline
18. प्राश्ञा᳚ म्यश्ञामि॒ प्र प्राश्ञा॑मि । \newline
19. अ॒श्ञा॒मीती त्य॑श्ञा म्यश्ञा॒मीति॑ । \newline
20. इत्या॑हा॒हे तीत्या॑ह । \newline
21. आ॒ह॒ प॒रोक्ष॑म् प॒रोक्ष॑ माहाह प॒रोक्ष᳚म् । \newline
22. प॒रोक्ष॑ मे॒वैव प॒रोक्ष॑म् प॒रोक्ष॑ मे॒व । \newline
23. प॒रोक्ष॒मिति॑ परः - अक्ष᳚म् । \newline
24. ए॒वैन॑ देन दे॒वैवैन॑त् । \newline
25. ए॒न॒ज् जु॒हो॒ति॒ जु॒हो॒ त्ये॒न॒ दे॒न॒ज् जु॒हो॒ति॒ । \newline
26. जु॒हो॒ति॒ सद॑सः॒ सद॑सो जुहोति जुहोति॒ सद॑सः । \newline
27. सद॑स॒स् पत॑ये॒ पत॑ये॒ सद॑सः॒ सद॑स॒स् पत॑ये । \newline
28. पत॑ये त्वा त्वा॒ पत॑ये॒ पत॑ये त्वा । \newline
29. त्वा॒ हु॒तꣳ हु॒तम् त्वा᳚ त्वा हु॒तम् । \newline
30. हु॒तम् प्र प्र हु॒तꣳ हु॒तम् प्र । \newline
31. प्राश्ञा᳚ म्यश्ञामि॒ प्र प्राश्ञा॑मि । \newline
32. अ॒श्ञा॒मीती त्य॑श्ञा म्यश्ञा॒मीति॑ । \newline
33. इत्या॑हा॒हे तीत्या॑ह । \newline
34. आ॒ह॒ स्व॒गाकृ॑त्यै स्व॒गाकृ॑त्या आहाह स्व॒गाकृ॑त्यै । \newline
35. स्व॒गाकृ॑त्यै॒ प्र प्र स्व॒गाकृ॑त्यै स्व॒गाकृ॑त्यै॒ प्र । \newline
36. स्व॒गाकृ॑त्या॒ इति॑ स्व॒गा - कृ॒त्यै॒ । \newline
37. प्राश्ञ॑ न्त्यश्ञन्ति॒ प्र प्राश्ञ॑न्ति । \newline
38. अ॒श्ञ॒न्ति॒ ती॒र्त्थे ती॒र्त्थे᳚ ऽश्ञ न्त्यश्ञन्ति ती॒र्त्थे । \newline
39. ती॒र्त्थ ए॒वैव ती॒र्त्थे ती॒र्त्थ ए॒व । \newline
40. ए॒व प्र प्रैवैव प्र । \newline
41. प्राश्ञ॑ न्त्यश्ञन्ति॒ प्र प्राश्ञ॑न्ति । \newline
42. अ॒श्ञ॒न्ति॒ दक्षि॑णा॒म् दक्षि॑णा मश्ञ न्त्यश्ञन्ति॒ दक्षि॑णाम् । \newline
43. दक्षि॑णाम् ददाति ददाति॒ दक्षि॑णा॒म् दक्षि॑णाम् ददाति । \newline
44. द॒दा॒ति॒ ती॒र्त्थे ती॒र्त्थे द॑दाति ददाति ती॒र्त्थे । \newline
45. ती॒र्त्थ ए॒वैव ती॒र्त्थे ती॒र्त्थ ए॒व । \newline
46. ए॒व दक्षि॑णा॒म् दक्षि॑णा मे॒वैव दक्षि॑णाम् । \newline
47. दक्षि॑णाम् ददाति ददाति॒ दक्षि॑णा॒म् दक्षि॑णाम् ददाति । \newline
48. द॒दा॒ति॒ वि वि द॑दाति ददाति॒ वि । \newline
49. वि वै वै वि वि वै । \newline
50. वा ए॒त दे॒तद् वै वा ए॒तत् । \newline
51. ए॒तद् य॒ज्ञ्ं ॅय॒ज्ञ् मे॒त दे॒तद् य॒ज्ञ्म् । \newline
52. य॒ज्ञ्म् छि॑न्दन्ति छिन्दन्ति य॒ज्ञ्ं ॅय॒ज्ञ्म् छि॑न्दन्ति । \newline

\textbf{Ghana Paata } \newline

1. आर्ति॒ मा ऽऽर्ति॒ मार्ति॒ मार्च्छे॑ दृच्छे॒दा ऽऽर्ति॒ मार्ति॒ मार्च्छे᳚त् । \newline
2. आर्च्छे॑ दृच्छे॒ दार्च्छे॒द् यद् यदृ॑च्छे॒ दार्च्छे॒द् यत् । \newline
3. ऋ॒च्छे॒द् यद् यदृ॑च्छे दृच्छे॒द् यद॒ग्ना व॒ग्नौ यदृ॑च्छे दृच्छे॒द् यद॒ग्नौ । \newline
4. यद॒ग्ना व॒ग्नौ यद् यद॒ग्नौ जु॑हु॒याज् जु॑हु॒या द॒ग्नौ यद् यद॒ग्नौ जु॑हु॒यात् । \newline
5. अ॒ग्नौ जु॑हु॒याज् जु॑हु॒या द॒ग्ना व॒ग्नौ जु॑हु॒याद् रु॒द्राय॑ रु॒द्राय॑ जुहु॒या द॒ग्ना व॒ग्नौ जु॑हु॒याद् रु॒द्राय॑ । \newline
6. जु॒हु॒याद् रु॒द्राय॑ रु॒द्राय॑ जुहु॒याज् जु॑हु॒याद् रु॒द्राय॑ प॒शून् प॒शून् रु॒द्राय॑ जुहु॒याज् जु॑हु॒याद् रु॒द्राय॑ प॒शून् । \newline
7. रु॒द्राय॑ प॒शून् प॒शून् रु॒द्राय॑ रु॒द्राय॑ प॒शू नप्यपि॑ प॒शून् रु॒द्राय॑ रु॒द्राय॑ प॒शू नपि॑ । \newline
8. प॒शू नप्यपि॑ प॒शून् प॒शू नपि॑ दद्ध्याद् दद्ध्या॒ दपि॑ प॒शून् प॒शू नपि॑ दद्ध्यात् । \newline
9. अपि॑ दद्ध्याद् दद्ध्या॒ दप्यपि॑ दद्ध्या दप॒शु र॑प॒शुर् द॑द्ध्या॒ दप्यपि॑ दद्ध्या दप॒शुः । \newline
10. द॒द्ध्या॒ द॒प॒शु र॑प॒शुर् द॑द्ध्याद् दद्ध्या दप॒शुर् यज॑मानो॒ यज॑मानो ऽप॒शुर् द॑द्ध्याद् दद्ध्या दप॒शुर् यज॑मानः । \newline
11. अ॒प॒शुर् यज॑मानो॒ यज॑मानो ऽप॒शु र॑प॒शुर् यज॑मानः स्याथ् स्या॒द् यज॑मानो ऽप॒शु र॑प॒शुर् यज॑मानः स्यात् । \newline
12. यज॑मानः स्याथ् स्या॒द् यज॑मानो॒ यज॑मानः स्याद् वा॒चो वा॒चः स्या॒द् यज॑मानो॒ यज॑मानः स्याद् वा॒चः । \newline
13. स्या॒द् वा॒चो वा॒चः स्या᳚थ् स्याद् वा॒च स्पत॑ये॒ पत॑ये वा॒चः स्या᳚थ् स्याद् वा॒च स्पत॑ये । \newline
14. वा॒च स्पत॑ये॒ पत॑ये वा॒चो वा॒च स्पत॑ये त्वा त्वा॒ पत॑ये वा॒चो वा॒च स्पत॑ये त्वा । \newline
15. पत॑ये त्वा त्वा॒ पत॑ये॒ पत॑ये त्वा हु॒तꣳ हु॒तम् त्वा॒ पत॑ये॒ पत॑ये त्वा हु॒तम् । \newline
16. त्वा॒ हु॒तꣳ हु॒तम् त्वा᳚ त्वा हु॒तम् प्र प्र हु॒तम् त्वा᳚ त्वा हु॒तम् प्र । \newline
17. हु॒तम् प्र प्र हु॒तꣳ हु॒तम् प्राश्ञा᳚ म्यश्ञामि॒ प्र हु॒तꣳ हु॒तम् प्राश्ञा॑मि । \newline
18. प्राश्ञा᳚ म्यश्ञामि॒ प्र प्राश्ञा॒मीती त्य॑श्ञामि॒ प्र प्राश्ञा॒मीति॑ । \newline
19. अ॒श्ञा॒मीती त्य॑श्ञा म्यश्ञा॒मी त्या॑हा॒हे त्य॑श्ञा म्यश्ञा॒मी त्या॑ह । \newline
20. इत्या॑हा॒हे तीत्या॑ह प॒रोक्ष॑म् प॒रोक्ष॑ मा॒हे तीत्या॑ह प॒रोक्ष᳚म् । \newline
21. आ॒ह॒ प॒रोक्ष॑म् प॒रोक्ष॑ माहाह प॒रोक्ष॑ मे॒वैव प॒रोक्ष॑ माहाह प॒रोक्ष॑ मे॒व । \newline
22. प॒रोक्ष॑ मे॒वैव प॒रोक्ष॑म् प॒रोक्ष॑ मे॒वैन॑ देनदे॒व प॒रोक्ष॑म् प॒रोक्ष॑ मे॒वैन॑त् । \newline
23. प॒रोक्ष॒मिति॑ परः - अक्ष᳚म् । \newline
24. ए॒वैन॑ देन दे॒वैवैन॑ज् जुहोति जुहो त्येन दे॒वैवैन॑ज् जुहोति । \newline
25. ए॒न॒ज् जु॒हो॒ति॒ जु॒हो॒ त्ये॒न॒ दे॒न॒ज् जु॒हो॒ति॒ सद॑सः॒ सद॑सो जुहो त्येन देनज् जुहोति॒ सद॑सः । \newline
26. जु॒हो॒ति॒ सद॑सः॒ सद॑सो जुहोति जुहोति॒ सद॑स॒ स्पत॑ये॒ पत॑ये॒ सद॑सो जुहोति जुहोति॒ सद॑स॒ स्पत॑ये । \newline
27. सद॑स॒ स्पत॑ये॒ पत॑ये॒ सद॑सः॒ सद॑स॒ स्पत॑ये त्वा त्वा॒ पत॑ये॒ सद॑सः॒ सद॑स॒ स्पत॑ये त्वा । \newline
28. पत॑ये त्वा त्वा॒ पत॑ये॒ पत॑ये त्वा हु॒तꣳ हु॒तम् त्वा॒ पत॑ये॒ पत॑ये त्वा हु॒तम् । \newline
29. त्वा॒ हु॒तꣳ हु॒तम् त्वा᳚ त्वा हु॒तम् प्र प्र हु॒तम् त्वा᳚ त्वा हु॒तम् प्र । \newline
30. हु॒तम् प्र प्र हु॒तꣳ हु॒तम् प्राश्ञा᳚ म्यश्ञामि॒ प्र हु॒तꣳ हु॒तम् प्राश्ञा॑मि । \newline
31. प्राश्ञा᳚ म्यश्ञामि॒ प्र प्राश्ञा॒मीती त्य॑श्ञामि॒ प्र प्राश्ञा॒मीति॑ । \newline
32. अ॒श्ञा॒मीती त्य॑श्ञा म्यश्ञा॒मी त्या॑हा॒हे त्य॑श्ञा म्यश्ञा॒मीत्या॑ह । \newline
33. इत्या॑हा॒हे तीत्या॑ह स्व॒गाकृ॑त्यै स्व॒गाकृ॑त्या आ॒हे तीत्या॑ह स्व॒गाकृ॑त्यै । \newline
34. आ॒ह॒ स्व॒गाकृ॑त्यै स्व॒गाकृ॑त्या आहाह स्व॒गाकृ॑त्यै॒ प्र प्र स्व॒गाकृ॑त्या आहाह स्व॒गाकृ॑त्यै॒ प्र । \newline
35. स्व॒गाकृ॑त्यै॒ प्र प्र स्व॒गाकृ॑त्यै स्व॒गाकृ॑त्यै॒ प्राश्ञ॑ न्त्यश्ञन्ति॒ प्र स्व॒गाकृ॑त्यै स्व॒गाकृ॑त्यै॒ प्राश्ञ॑न्ति । \newline
36. स्व॒गाकृ॑त्या॒ इति॑ स्व॒गा - कृ॒त्यै॒ । \newline
37. प्राश्ञ॑ न्त्यश्ञन्ति॒ प्र प्राश्ञ॑न्ति ती॒र्त्थे ती॒र्त्थे᳚ ऽश्ञन्ति॒ प्र प्राश्ञ॑न्ति ती॒र्त्थे । \newline
38. अ॒श्ञ॒न्ति॒ ती॒र्त्थे ती॒र्त्थे᳚ ऽश्ञ न्त्यश्ञन्ति ती॒र्त्थ ए॒वैव ती॒र्त्थे᳚ ऽश्ञ न्त्यश्ञन्ति ती॒र्त्थ ए॒व । \newline
39. ती॒र्त्थ ए॒वैव ती॒र्त्थे ती॒र्त्थ ए॒व प्र प्रैव ती॒र्त्थे ती॒र्त्थ ए॒व प्र । \newline
40. ए॒व प्र प्रैवैव प्राश्ञ॑ न्त्यश्ञन्ति॒ प्रैवैव प्राश्ञ॑न्ति । \newline
41. प्राश्ञ॑ न्त्यश्ञन्ति॒ प्र प्राश्ञ॑न्ति॒ दक्षि॑णा॒म् दक्षि॑णा मश्ञन्ति॒ प्र प्राश्ञ॑न्ति॒ दक्षि॑णाम् । \newline
42. अ॒श्ञ॒न्ति॒ दक्षि॑णा॒म् दक्षि॑णा मश्ञ न्त्यश्ञन्ति॒ दक्षि॑णाम् ददाति ददाति॒ दक्षि॑णा मश्ञ न्त्यश्ञन्ति॒ दक्षि॑णाम् ददाति । \newline
43. दक्षि॑णाम् ददाति ददाति॒ दक्षि॑णा॒म् दक्षि॑णाम् ददाति ती॒र्त्थे ती॒र्त्थे द॑दाति॒ दक्षि॑णा॒म् दक्षि॑णाम् ददाति ती॒र्त्थे । \newline
44. द॒दा॒ति॒ ती॒र्त्थे ती॒र्त्थे द॑दाति ददाति ती॒र्त्थ ए॒वैव ती॒र्त्थे द॑दाति ददाति ती॒र्त्थ ए॒व । \newline
45. ती॒र्त्थ ए॒वैव ती॒र्त्थे ती॒र्त्थ ए॒व दक्षि॑णा॒म् दक्षि॑णा मे॒व ती॒र्त्थे ती॒र्त्थ ए॒व दक्षि॑णाम् । \newline
46. ए॒व दक्षि॑णा॒म् दक्षि॑णा मे॒वैव दक्षि॑णाम् ददाति ददाति॒ दक्षि॑णा मे॒वैव दक्षि॑णाम् ददाति । \newline
47. दक्षि॑णाम् ददाति ददाति॒ दक्षि॑णा॒म् दक्षि॑णाम् ददाति॒ वि वि द॑दाति॒ दक्षि॑णा॒म् दक्षि॑णाम् ददाति॒ वि । \newline
48. द॒दा॒ति॒ वि वि द॑दाति ददाति॒ वि वै वै वि द॑दाति ददाति॒ वि वै । \newline
49. वि वै वै वि वि वा ए॒त दे॒तद् वै वि वि वा ए॒तत् । \newline
50. वा ए॒त दे॒तद् वै वा ए॒तद् य॒ज्ञ्ं ॅय॒ज्ञ् मे॒तद् वै वा ए॒तद् य॒ज्ञ्म् । \newline
51. ए॒तद् य॒ज्ञ्ं ॅय॒ज्ञ् मे॒त दे॒तद् य॒ज्ञ्म् छि॑न्दन्ति छिन्दन्ति य॒ज्ञ् मे॒त दे॒तद् य॒ज्ञ्म् छि॑न्दन्ति । \newline
52. य॒ज्ञ्म् छि॑न्दन्ति छिन्दन्ति य॒ज्ञ्ं ॅय॒ज्ञ्म् छि॑न्दन्ति॒ यद् यच् छि॑न्दन्ति य॒ज्ञ्ं ॅय॒ज्ञ्म् छि॑न्दन्ति॒ यत् । \newline
\pagebreak
\markright{ TS 2.6.8.3  \hfill https://www.vedavms.in \hfill}
\addcontentsline{toc}{section}{ TS 2.6.8.3 }
\section*{ TS 2.6.8.3 }

\textbf{TS 2.6.8.3 } \newline
\textbf{Samhita Paata} \newline

छि॑न्दन्ति॒ यन्म॑द्ध्य॒तः प्रा॒श्नन्त्य॒द्भि-र्मा᳚र्जयन्त॒ आपो॒ वै सर्वा॑ दे॒वता॑ दे॒वता॑भिरे॒व य॒ज्ञ्ꣳ सं त॑न्वन्ति दे॒वा वै य॒ज्ञाद्-रु॒द्रम॒न्तरा॑य॒न्थ्स य॒ज्ञ्म॑विद्ध्य॒त् तं दे॒वा अ॒भि सम॑गच्छन्त॒ कल्प॑तां न इ॒दमिति॒ ते᳚ऽब्रुव॒न्थ् स्वि॑ष्टं॒ ॅवै न॑ इ॒दं भ॑विष्यति॒ यदि॒मꣳ रा॑धयि॒ष्याम॒ इति॒ तथ् स्वि॑ष्ट॒कृतः॑ स्विष्टकृ॒त्त्वं तस्या ऽऽवि॑द्धं॒ नि - [  ] \newline

\textbf{Pada Paata} \newline

छि॒न्द॒न्ति॒ । यत् । म॒द्ध्य॒तः । प्रा॒श्नन्तीति॑ प्र - अ॒श्नन्ति॑ । अ॒द्भिरित्य॑त् - भिः । मा॒र्ज॒य॒न्ते॒ । आपः॑ । वै । सर्वाः᳚ । दे॒वताः᳚ । दे॒वता॑भिः । ए॒व । य॒ज्ञ्म् । समिति॑ । त॒न्व॒न्ति॒ । दे॒वाः । वै । य॒ज्ञात् । रु॒द्रम् । अ॒न्तः । आ॒य॒न्न् । सः । य॒ज्ञ्म् । अ॒वि॒द्ध्य॒त् । तम् । दे॒वाः । अ॒भि । समिति॑ । अ॒ग॒च्छ॒न्त॒ । कल्प॑ताम् । नः॒ । इ॒दम् । इति॑ । ते । अ॒ब्रु॒व॒न्न् । स्वि॑ष्ट॒मिति॒ सु - इ॒ष्ट॒म् । वै । नः॒ । इ॒दम् । भ॒वि॒ष्य॒ति॒ । यत् । इ॒मम् । रा॒ध॒यि॒ष्यामः॑ । इति॑ । तत् । स्वि॒ष्ट॒कृत॒ इति॑ स्विष्ट - कृतः॑ । स्वि॒ष्ट॒कृ॒त्त्वमिति॑ स्विष्टकृत् - त्वम् । तस्य॑ । आवि॑द्ध॒मित्या - वि॒द्ध॒म् । निरिति॑ ।  \newline


\textbf{Krama Paata} \newline

छि॒न्द॒न्ति॒ यत् । यन् म॑द्ध्य॒तः । म॒द्ध्य॒तः प्रा॒श्ञन्ति॑ । प्रा॒श्ञन्त्य॒द्भिः । प्रा॒श्ञन्तीति॑ प्र - अ॒श्ञन्ति॑ । अ॒द्भिर् मा᳚र्जयन्ते । अ॒द्भिरित्य॑त् - भिः । मा॒र्ज॒य॒न्त॒ आपः॑ । आपो॒ वै । वै सर्वाः᳚ । सर्वा॑ दे॒वताः᳚ । दे॒वता॑ दे॒वता॑भिः । दे॒वता॑भिरे॒व । ए॒व य॒ज्ञ्म् । य॒ज्ञ्ꣳ सम् । सम् त॑न्वन्ति । त॒न्व॒न्ति॒ दे॒वाः । दे॒वा वै । वै य॒ज्ञात् । य॒ज्ञाद् रु॒द्रम् । रु॒द्रम॒न्तः । अ॒न्तरा॑यन्न् । आ॒य॒न्थ् सः । स य॒ज्ञ्म् । य॒ज्ञ्म॑विद्ध्यत् । अ॒वि॒द्ध्य॒त् तम् । तम् दे॒वाः । दे॒वा अ॒भि । अ॒भि सम् । सम॑गच्छन्त । अ॒ग॒च्छ॒न्त॒ कल्प॑ताम् । कल्प॑ताम् नः । न॒ इ॒दम् । इ॒दमिति॑ । इति॒ ते । ते᳚ऽब्रुवन्न् । अ॒ब्रु॒व॒न्थ् स्वि॑ष्टम् । स्वि॑ष्टं॒ ॅवै । स्वि॑ष्ट॒मिति॒ सु - इ॒ष्ट॒म् । वै नः॑ । न॒ इ॒दम् । इ॒दम् भ॑विष्यति । भ॒वि॒ष्य॒ति॒ यत् । यदि॒मम् । इ॒मꣳ रा॑धयि॒ष्यामः॑ । रा॒ध॒यि॒ष्याम॒ इति॑ । इति॒ तत् । तथ् स्वि॑ष्ट॒कृतः॑ । स्वि॒ष्ट॒कृतः॑ स्विष्टकृ॒त्वम् । स्वि॒ष्ट॒कृत॒ इति॑ स्विष्ट - कृतः॑ । स्वि॒ष्ट॒कृ॒त्वम् तस्य॑ । स्वि॒ष्ट॒कृ॒त्वमिति॑ स्विष्टकृत् - त्वम् । तस्यावि॑द्धम् । आवि॑द्ध॒म् निः । आवि॑द्ध॒मित्या - वि॒द्ध॒म् । निर॑कृन्तन्न् \newline

\textbf{Jatai Paata} \newline

1. छि॒न्द॒न्ति॒ यद् यच् छि॑न्दन्ति छिन्दन्ति॒ यत् । \newline
2. यन् म॑द्ध्य॒तो म॑द्ध्य॒तो यद् यन् म॑द्ध्य॒तः । \newline
3. म॒द्ध्य॒तः प्रा॒श्ञन्ति॑ प्रा॒श्ञन्ति॑ मद्ध्य॒तो म॑द्ध्य॒तः प्रा॒श्ञन्ति॑ । \newline
4. प्रा॒श्ञ न्त्य॒द्भि र॒द्भिः प्रा॒श्ञन्ति॑ प्रा॒श्ञ न्त्य॒द्भिः । \newline
5. प्रा॒श्ञन्तीति॑ प्र - अ॒श्ञन्ति॑ । \newline
6. अ॒द्भिर् मा᳚र्जयन्ते मार्जयन्ते॒ ऽद्भि र॒द्भिर् मा᳚र्जयन्ते । \newline
7. अ॒द्भिरित्य॑त् - भिः । \newline
8. मा॒र्ज॒य॒न्त॒ आप॒ आपो॑ मार्जयन्ते मार्जयन्त॒ आपः॑ । \newline
9. आपो॒ वै वा आप॒ आपो॒ वै । \newline
10. वै सर्वाः॒ सर्वा॒ वै वै सर्वाः᳚ । \newline
11. सर्वा॑ दे॒वता॑ दे॒वताः॒ सर्वाः॒ सर्वा॑ दे॒वताः᳚ । \newline
12. दे॒वता॑ दे॒वता॑भिर् दे॒वता॑भिर् दे॒वता॑ दे॒वता॑ दे॒वता॑भिः । \newline
13. दे॒वता॑भि रे॒वैव दे॒वता॑भिर् दे॒वता॑भि रे॒व । \newline
14. ए॒व य॒ज्ञ्ं ॅय॒ज्ञ् मे॒वैव य॒ज्ञ्म् । \newline
15. य॒ज्ञ्ꣳ सꣳ सं ॅय॒ज्ञ्ं ॅय॒ज्ञ्ꣳ सम् । \newline
16. सम् त॑न्वन्ति तन्वन्ति॒ सꣳ सम् त॑न्वन्ति । \newline
17. त॒न्व॒न्ति॒ दे॒वा दे॒वा स्त॑न्वन्ति तन्वन्ति दे॒वाः । \newline
18. दे॒वा वै वै दे॒वा दे॒वा वै । \newline
19. वै य॒ज्ञाद् य॒ज्ञाद् वै वै य॒ज्ञात् । \newline
20. य॒ज्ञाद् रु॒द्रꣳ रु॒द्रं ॅय॒ज्ञाद् य॒ज्ञाद् रु॒द्रम् । \newline
21. रु॒द्र म॒न्त र॒न्ता रु॒द्रꣳ रु॒द्र म॒न्तः । \newline
22. अ॒न्त रा॑यन् नायन् न॒न्त र॒न्त रा॑यन्न् । \newline
23. आ॒य॒न् थ्स स आ॑यन् नाय॒न् थ्सः । \newline
24. स य॒ज्ञ्ं ॅय॒ज्ञ्ꣳ स स य॒ज्ञ्म् । \newline
25. य॒ज्ञ् म॑विद्ध्य दविद्ध्यद् य॒ज्ञ्ं ॅय॒ज्ञ् म॑विद्ध्यत् । \newline
26. अ॒वि॒द्ध्य॒त् तम् त म॑विद्ध्य दविद्ध्य॒त् तम् । \newline
27. तम् दे॒वा दे॒वा स्तम् तम् दे॒वाः । \newline
28. दे॒वा अ॒भ्य॑भि दे॒वा दे॒वा अ॒भि । \newline
29. अ॒भि सꣳ स म॒भ्य॑भि सम् । \newline
30. स म॑गच्छन्ता गच्छन्त॒ सꣳ स म॑गच्छन्त । \newline
31. अ॒ग॒च्छ॒न्त॒ कल्प॑ता॒म् कल्प॑ता मगच्छन्ता गच्छन्त॒ कल्प॑ताम् । \newline
32. कल्प॑ताम् नो नः॒ कल्प॑ता॒म् कल्प॑ताम् नः । \newline
33. न॒ इ॒द मि॒दम् नो॑ न इ॒दम् । \newline
34. इ॒द मितीती॒द मि॒द मिति॑ । \newline
35. इति॒ ते त इतीति॒ ते । \newline
36. ते᳚ ऽब्रुवन् नब्रुव॒न् ते ते᳚ ऽब्रुवन्न् । \newline
37. अ॒ब्रु॒व॒न् थ्स्वि॑ष्टꣳ॒॒ स्वि॑ष्ट मब्रुवन् नब्रुव॒न् थ्स्वि॑ष्टम् । \newline
38. स्वि॑ष्टं॒ ॅवै वै स्वि॑ष्टꣳ॒॒ स्वि॑ष्टं॒ ॅवै । \newline
39. स्वि॑ष्ट॒मिति॒ सु - इ॒ष्ट॒म् । \newline
40. वै नो॑ नो॒ वै वै नः॑ । \newline
41. न॒ इ॒द मि॒दम् नो॑ न इ॒दम् । \newline
42. इ॒दम् भ॑विष्यति भविष्यती॒द मि॒दम् भ॑विष्यति । \newline
43. भ॒वि॒ष्य॒ति॒ यद् यद् भ॑विष्यति भविष्यति॒ यत् । \newline
44. यदि॒म मि॒मं ॅयद् यदि॒मम् । \newline
45. इ॒मꣳ रा॑धयि॒ष्यामो॑ राधयि॒ष्याम॑ इ॒म मि॒मꣳ रा॑धयि॒ष्यामः॑ । \newline
46. रा॒ध॒यि॒ष्याम॒ इतीति॑ राधयि॒ष्यामो॑ राधयि॒ष्याम॒ इति॑ । \newline
47. इति॒ तत् तदितीति॒ तत् । \newline
48. तथ् स्वि॑ष्ट॒कृतः॑ स्विष्ट॒कृत॒ स्तत् तथ् स्वि॑ष्ट॒कृतः॑ । \newline
49. स्वि॒ष्ट॒कृतः॑ स्विष्टकृ॒त्त्वꣳ स्वि॑ष्टकृ॒त्त्वꣳ स्वि॑ष्ट॒कृतः॑ स्विष्ट॒कृतः॑ स्विष्टकृ॒त्त्वम् । \newline
50. स्वि॒ष्ट॒कृत॒ इति॑ स्विष्ट - कृतः॑ । \newline
51. स्वि॒ष्ट॒कृ॒त्त्वम् तस्य॒ तस्य॑ स्विष्टकृ॒त्त्वꣳ स्वि॑ष्टकृ॒त्त्वम् तस्य॑ । \newline
52. स्वि॒ष्ट॒कृ॒त्त्वमिति॑ स्विष्टकृत् - त्वम् । \newline
53. तस्यावि॑द्ध॒ मावि॑द्ध॒म् तस्य॒ तस्यावि॑द्धम् । \newline
54. आवि॑द्ध॒म् निर् णिरावि॑द्ध॒ मावि॑द्ध॒म् निः । \newline
55. आवि॑द्ध॒मित्या - वि॒द्ध॒म् । \newline
56. निर॑कृन्तन् नकृन्त॒न् निर् णिर॑कृन्तन्न् । \newline

\textbf{Ghana Paata } \newline

1. छि॒न्द॒न्ति॒ यद् यच् छि॑न्दन्ति छिन्दन्ति॒ यन् म॑द्ध्य॒तो म॑द्ध्य॒तो यच् छि॑न्दन्ति छिन्दन्ति॒ यन् म॑द्ध्य॒तः । \newline
2. यन् म॑द्ध्य॒तो म॑द्ध्य॒तो यद् यन् म॑द्ध्य॒तः प्रा॒श्ञन्ति॑ प्रा॒श्ञन्ति॑ मद्ध्य॒तो यद् यन् म॑द्ध्य॒तः प्रा॒श्ञन्ति॑ । \newline
3. म॒द्ध्य॒तः प्रा॒श्ञन्ति॑ प्रा॒श्ञन्ति॑ मद्ध्य॒तो म॑द्ध्य॒तः प्रा॒श्ञ न्त्य॒द्भि र॒द्भिः प्रा॒श्ञन्ति॑ मद्ध्य॒तो म॑द्ध्य॒तः प्रा॒श्ञन्त्य॒द्भिः । \newline
4. प्रा॒श्ञ न्त्य॒द्भि र॒द्भिः प्रा॒श्ञन्ति॑ प्रा॒श्ञ न्त्य॒द्भिर् मा᳚र्जयन्ते मार्जयन्ते॒ ऽद्भिः प्रा॒श्ञन्ति॑ प्रा॒श्ञ न्त्य॒द्भिर् मा᳚र्जयन्ते । \newline
5. प्रा॒श्ञन्तीति॑ प्र - अ॒श्ञन्ति॑ । \newline
6. अ॒द्भिर् मा᳚र्जयन्ते मार्जयन्ते॒ ऽद्भि र॒द्भिर् मा᳚र्जयन्त॒ आप॒ आपो॑ मार्जयन्ते॒ ऽद्भि र॒द्भिर् मा᳚र्जयन्त॒ आपः॑ । \newline
7. अ॒द्भिरित्य॑त् - भिः । \newline
8. मा॒र्ज॒य॒न्त॒ आप॒ आपो॑ मार्जयन्ते मार्जयन्त॒ आपो॒ वै वा आपो॑ मार्जयन्ते मार्जयन्त॒ आपो॒ वै । \newline
9. आपो॒ वै वा आप॒ आपो॒ वै सर्वाः॒ सर्वा॒ वा आप॒ आपो॒ वै सर्वाः᳚ । \newline
10. वै सर्वाः॒ सर्वा॒ वै वै सर्वा॑ दे॒वता॑ दे॒वताः॒ सर्वा॒ वै वै सर्वा॑ दे॒वताः᳚ । \newline
11. सर्वा॑ दे॒वता॑ दे॒वताः॒ सर्वाः॒ सर्वा॑ दे॒वता॑ दे॒वता॑भिर् दे॒वता॑भिर् दे॒वताः॒ सर्वाः॒ सर्वा॑ दे॒वता॑ दे॒वता॑भिः । \newline
12. दे॒वता॑ दे॒वता॑भिर् दे॒वता॑भिर् दे॒वता॑ दे॒वता॑ दे॒वता॑भि रे॒वैव दे॒वता॑भिर् दे॒वता॑ दे॒वता॑ दे॒वता॑भिरे॒व । \newline
13. दे॒वता॑भि रे॒वैव दे॒वता॑भिर् दे॒वता॑भि रे॒व य॒ज्ञ्ं ॅय॒ज्ञ् मे॒व दे॒वता॑भिर् दे॒वता॑भि रे॒व य॒ज्ञ्म् । \newline
14. ए॒व य॒ज्ञ्ं ॅय॒ज्ञ् मे॒वैव य॒ज्ञ्ꣳ सꣳ सं ॅय॒ज्ञ् मे॒वैव य॒ज्ञ्ꣳ सम् । \newline
15. य॒ज्ञ्ꣳ सꣳ सं ॅय॒ज्ञ्ं ॅय॒ज्ञ्ꣳ सम् त॑न्वन्ति तन्वन्ति॒ सं ॅय॒ज्ञ्ं ॅय॒ज्ञ्ꣳ सम् त॑न्वन्ति । \newline
16. सम् त॑न्वन्ति तन्वन्ति॒ सꣳ सम् त॑न्वन्ति दे॒वा दे॒वा स्त॑न्वन्ति॒ सꣳ सम् त॑न्वन्ति दे॒वाः । \newline
17. त॒न्व॒न्ति॒ दे॒वा दे॒वा स्त॑न्वन्ति तन्वन्ति दे॒वा वै वै दे॒वा स्त॑न्वन्ति तन्वन्ति दे॒वा वै । \newline
18. दे॒वा वै वै दे॒वा दे॒वा वै य॒ज्ञाद् य॒ज्ञाद् वै दे॒वा दे॒वा वै य॒ज्ञात् । \newline
19. वै य॒ज्ञाद् य॒ज्ञाद् वै वै य॒ज्ञाद् रु॒द्रꣳ रु॒द्रं ॅय॒ज्ञाद् वै वै य॒ज्ञाद् रु॒द्रम् । \newline
20. य॒ज्ञाद् रु॒द्रꣳ रु॒द्रं ॅय॒ज्ञाद् य॒ज्ञाद् रु॒द्र म॒न्त र॒न्ता रु॒द्रं ॅय॒ज्ञाद् य॒ज्ञाद् रु॒द्र म॒न्तः । \newline
21. रु॒द्र म॒न्त र॒न्ता रु॒द्रꣳ रु॒द्र म॒न्त रा॑यन् नायन् न॒न्ता रु॒द्रꣳ रु॒द्र म॒न्त रा॑यन्न् । \newline
22. अ॒न्त रा॑यन् नायन् न॒न्त र॒न्त रा॑य॒न् थ्स स आ॑यन् न॒न्त र॒न्त रा॑य॒न् थ्सः । \newline
23. आ॒य॒न् थ्स स आ॑यन् नाय॒न् थ्स य॒ज्ञ्ं ॅय॒ज्ञ्ꣳ स आ॑यन् नाय॒न् थ्स य॒ज्ञ्म् । \newline
24. स य॒ज्ञ्ं ॅय॒ज्ञ्ꣳ स स य॒ज्ञ् म॑विद्ध्य दविद्ध्यद् य॒ज्ञ्ꣳ स स य॒ज्ञ् म॑विद्ध्यत् । \newline
25. य॒ज्ञ् म॑विद्ध्य दविद्ध्यद् य॒ज्ञ्ं ॅय॒ज्ञ् म॑विद्ध्य॒त् तम् त म॑विद्ध्यद् य॒ज्ञ्ं ॅय॒ज्ञ् म॑विद्ध्य॒त् तम् । \newline
26. अ॒वि॒द्ध्य॒त् तम् त म॑विद्ध्य दविद्ध्य॒त् तम् दे॒वा दे॒वा स्त म॑विद्ध्य दविद्ध्य॒त् तम् दे॒वाः । \newline
27. तम् दे॒वा दे॒वा स्तम् तम् दे॒वा अ॒भ्य॑भि दे॒वा स्तम् तम् दे॒वा अ॒भि । \newline
28. दे॒वा अ॒भ्य॑भि दे॒वा दे॒वा अ॒भि सꣳ स म॒भि दे॒वा दे॒वा अ॒भि सम् । \newline
29. अ॒भि सꣳ स म॒भ्य॑भि स म॑गच्छन्ता गच्छन्त॒ स म॒भ्य॑भि स म॑गच्छन्त । \newline
30. स म॑गच्छन्ता गच्छन्त॒ सꣳ स म॑गच्छन्त॒ कल्प॑ता॒म् कल्प॑ता मगच्छन्त॒ सꣳ स म॑गच्छन्त॒ कल्प॑ताम् । \newline
31. अ॒ग॒च्छ॒न्त॒ कल्प॑ता॒म् कल्प॑ता मगच्छन्ता गच्छन्त॒ कल्प॑तान्नो नः॒ कल्प॑ता मगच्छन्ता गच्छन्त॒ कल्प॑तान्नः । \newline
32. कल्प॑ताम् नो नः॒ कल्प॑ता॒म् कल्प॑ताम् न इ॒द मि॒दम् नः॒ कल्प॑ता॒म् कल्प॑ताम् न इ॒दम् । \newline
33. न॒ इ॒द मि॒दम् नो॑ न इ॒द मितीती॒दम् नो॑ न इ॒द मिति॑ । \newline
34. इ॒द मितीती॒द मि॒द मिति॒ ते त इती॒द मि॒द मिति॒ ते । \newline
35. इति॒ ते त इतीति॒ ते᳚ ऽब्रुवन् नब्रुव॒न् त इतीति॒ ते᳚ ऽब्रुवन्न् । \newline
36. ते᳚ ऽब्रुवन् नब्रुव॒न् ते ते᳚ ऽब्रुव॒न् थ्स्वि॑ष्टꣳ॒॒ स्वि॑ष्ट मब्रुव॒न् ते ते᳚ ऽब्रुव॒न् थ्स्वि॑ष्टम् । \newline
37. अ॒ब्रु॒व॒न् थ्स्वि॑ष्टꣳ॒॒ स्वि॑ष्ट मब्रुवन् नब्रुव॒न् थ्स्वि॑ष्टं॒ ॅवै वै स्वि॑ष्ट मब्रुवन् नब्रुव॒न् थ्स्वि॑ष्टं॒ ॅवै । \newline
38. स्वि॑ष्टं॒ ॅवै वै स्वि॑ष्टꣳ॒॒ स्वि॑ष्टं॒ ॅवै नो॑ नो॒ वै स्वि॑ष्टꣳ॒॒ स्वि॑ष्टं॒ ॅवै नः॑ । \newline
39. स्वि॑ष्ट॒मिति॒ सु - इ॒ष्ट॒म् । \newline
40. वै नो॑ नो॒ वै वै न॑ इ॒द मि॒दम् नो॒ वै वै न॑ इ॒दम् । \newline
41. न॒ इ॒द मि॒दम् नो॑ न इ॒दम् भ॑विष्यति भविष्यती॒दम् नो॑ न इ॒दम् भ॑विष्यति । \newline
42. इ॒दम् भ॑विष्यति भविष्यती॒द मि॒दम् भ॑विष्यति॒ यद् यद् भ॑विष्यती॒द मि॒दम् भ॑विष्यति॒ यत् । \newline
43. भ॒वि॒ष्य॒ति॒ यद् यद् भ॑विष्यति भविष्यति॒ यदि॒म मि॒मं ॅयद् भ॑विष्यति भविष्यति॒ यदि॒मम् । \newline
44. यदि॒म मि॒मं ॅयद् यदि॒मꣳ रा॑धयि॒ष्यामो॑ राधयि॒ष्याम॑ इ॒मं ॅयद् यदि॒मꣳ रा॑धयि॒ष्यामः॑ । \newline
45. इ॒मꣳ रा॑धयि॒ष्यामो॑ राधयि॒ष्याम॑ इ॒म मि॒मꣳ रा॑धयि॒ष्याम॒ इतीति॑ राधयि॒ष्याम॑ इ॒म मि॒मꣳ रा॑धयि॒ष्याम॒ इति॑ । \newline
46. रा॒ध॒यि॒ष्याम॒ इतीति॑ राधयि॒ष्यामो॑ राधयि॒ष्याम॒ इति॒ तत् तदिति॑ राधयि॒ष्यामो॑ राधयि॒ष्याम॒ इति॒ तत् । \newline
47. इति॒ तत् तदितीति॒ तथ् स्वि॑ष्ट॒कृतः॑ स्विष्ट॒कृत॒ स्तदितीति॒ तथ् स्वि॑ष्ट॒कृतः॑ । \newline
48. तथ् स्वि॑ष्ट॒कृतः॑ स्विष्ट॒कृत॒ स्तत् तथ् स्वि॑ष्ट॒कृतः॑ स्विष्टकृ॒त्त्वꣳ स्वि॑ष्टकृ॒त्त्वꣳ स्वि॑ष्ट॒कृत॒ स्तत् तथ् स्वि॑ष्ट॒कृतः॑ स्विष्टकृ॒त्त्वम् । \newline
49. स्वि॒ष्ट॒कृतः॑ स्विष्टकृ॒त्त्वꣳ स्वि॑ष्टकृ॒त्त्वꣳ स्वि॑ष्ट॒कृतः॑ स्विष्ट॒कृतः॑ स्विष्टकृ॒त्त्वम् तस्य॒ तस्य॑ स्विष्टकृ॒त्त्वꣳ स्वि॑ष्ट॒कृतः॑ स्विष्ट॒कृतः॑ स्विष्टकृ॒त्त्वम् तस्य॑ । \newline
50. स्वि॒ष्ट॒कृत॒ इति॑ स्विष्ट - कृतः॑ । \newline
51. स्वि॒ष्ट॒कृ॒त्त्वम् तस्य॒ तस्य॑ स्विष्टकृ॒त्त्वꣳ स्वि॑ष्टकृ॒त्त्वम् तस्या वि॑द्ध॒ मावि॑द्ध॒म् तस्य॑ स्विष्टकृ॒त्त्वꣳ स्वि॑ष्टकृ॒त्त्वम् तस्या वि॑द्धम् । \newline
52. स्वि॒ष्ट॒कृ॒त्त्वमिति॑ स्विष्टकृत् - त्वम् । \newline
53. तस्या वि॑द्ध॒ मावि॑द्ध॒म् तस्य॒ तस्या वि॑द्ध॒म् निर् णिरा वि॑द्ध॒म् तस्य॒ तस्या वि॑द्ध॒म् निः । \newline
54. आवि॑द्ध॒म् निर् णिरावि॑द्ध॒ मावि॑द्ध॒म् निर॑कृन्तन् नकृन्त॒न् निरावि॑द्ध॒ मावि॑द्ध॒म् निर॑कृन्तन्न् । \newline
55. आवि॑द्ध॒मित्या - वि॒द्ध॒म् । \newline
56. निर॑कृन्तन् नकृन्त॒न् निर् णिर॑कृन्त॒न्॒. यवे॑न॒ यवे॑नाकृन्त॒न् निर् णिर॑कृन्त॒न्॒. यवे॑न । \newline
\pagebreak
\markright{ TS 2.6.8.4  \hfill https://www.vedavms.in \hfill}
\addcontentsline{toc}{section}{ TS 2.6.8.4 }
\section*{ TS 2.6.8.4 }

\textbf{TS 2.6.8.4 } \newline
\textbf{Samhita Paata} \newline

-र॑कृन्त॒न॒. यवे॑न॒ सम्मि॑तं॒ तस्मा᳚द्-यवमा॒त्रमव॑ द्ये॒द्-यज्ज्यायो॑ऽव॒द्-येद्-रो॒पये॒त् तद्-य॒ज्ञ्स्य॒ यदुप॑ च स्तृणी॒याद॒भि च॑ घा॒रये॑दुभयतः सꣳश्वा॒यि कु॑र्यादव॒दाया॒भि घा॑रयति॒ द्विः संप॑द्यते द्वि॒पाद्-यज॑मानः॒ प्रति॑ष्ठित्यै॒ यत् ति॑र॒श्चीन॑-मति॒-हरे॒दन॑भि-विद्धं ॅय॒ज्ञ्स्या॒भि वि॑द्ध्ये॒दग्रे॑ण॒ परि॑ हरति ती॒र्थेनै॒व परि॑ हरति॒ तत् पू॒ष्णे पर्य॑हर॒न् तत् - [  ] \newline

\textbf{Pada Paata} \newline

अ॒कृ॒न्त॒न्न् । यवे॑न । संमि॑त॒मिति॒ सं - मि॒त॒म् । तस्मा᳚त् । य॒व॒मा॒त्रमिति॑ यव - मा॒त्रम् । अवेति॑ । द्ये॒त् । यत् । ज्यायः॑ । अ॒व॒द्येदित्य॑व - द्येत् । रो॒पये᳚त् । तत् । य॒ज्ञ्स्य॑ । यत् । उपेति॑ । च॒ । स्तृ॒णी॒यात् । अ॒भीति॑ । च॒ । घा॒रये᳚त् । उ॒भ॒य॒तः॒, सꣳ॒॒श्वा॒यीत्यु॑भयतः - सꣳ॒॒श्वा॒यि । कु॒र्या॒त् । अ॒व॒दायेत्य॑व - दाय॑ । अ॒भीति॑ । घा॒र॒य॒ति॒ । द्विः । समिति॑ । प॒द्य॒ते॒ । द्वि॒पादिति॑ द्वि - पात् । यज॑मानः । प्रति॑ष्ठित्या॒ इति॒ प्रति॑ - स्थि॒त्यै॒ । यत् । ति॒र॒श्चीन᳚म् । अ॒ति॒हरे॒दित्य॑ति - हरे᳚त् । अन॑भिविद्ध॒मित्यन॑भि - वि॒द्ध॒म् । य॒ज्ञ्स्य॑ । अ॒भीति॑ । वि॒द्ध्ये॒त् । अग्रे॑ण । परीति॑ । ह॒र॒ति॒ । ती॒र्त्थेन॑ । ए॒व । परीति॑ । ह॒र॒ति॒ । तत् । पू॒ष्णे । परीति॑ । अ॒ह॒र॒न्न् । तत् ।  \newline


\textbf{Krama Paata} \newline

अ॒कृ॒न्त॒न्.॒ यवे॑न । यवे॑न॒ सम्मि॑तम् । सम्मि॑त॒म् तस्मा᳚त् । सम्मि॑त॒मिति॒ सं - मि॒त॒म् । तस्मा᳚द् यवमा॒त्रम् । य॒व॒मा॒त्रमव॑ । य॒व॒मा॒त्रमिति॑ यव - मा॒त्रम् । अव॑ द्येत् । द्ये॒द् यत् । यज् ज्यायः॑ । ज्यायो॑ ऽव॒द्येत् । अ॒व॒द्येद् रो॒पये᳚त् । अ॒व॒द्येदित्य॑व - द्येत् । रो॒पये॒त् तत् । तद् य॒ज्ञ्स्य॑ । य॒ज्ञ्स्य॒ यत् । यदुप॑ । उप॑ च । च॒ स्तृ॒णी॒यात् । स्तृ॒णी॒याद॒भि । अ॒भि च॑ । च॒ घा॒रये᳚त् । घा॒रये॑दुभयतस्सꣳश्वा॒यि । उ॒भ॒य॒त॒स्सꣳ॒॒श्वा॒यि कु॑र्यात् । उ॒भ॒य॒त॒स्सꣳ॒॒श्वा॒यीत्यु॑भयतः - सꣳ॒॒श्वा॒यि । कु॒र्या॒द॒व॒दाय॑ । अ॒व॒दाया॒भि । अ॒व॒दायेत्य॑व - दाय॑ । अ॒भि घा॑रयति । घा॒र॒य॒ति॒ द्विः । द्विः सम् । सं प॑द्यते । प॒द्य॒ते॒ द्वि॒पात् । द्वि॒पाद् यज॑मानः । द्वि॒पादिति॑ द्वि - पात् । यज॑मानः॒ प्रति॑ष्ठित्यै । प्रति॑ष्ठित्यै॒ यत् । प्रति॑ष्ठित्या॒ इति॒ प्रति॑ - स्थि॒त्यै॒ । यत् ति॑र॒श्चीन᳚म् । ति॒र॒श्चीन॑मति॒हरे᳚त् । अ॒ति॒हरे॒दन॑भिविद्धम् । अ॒ति॒हरे॒दित्य॑ति - हरे᳚त् । अन॑भिविद्धं ॅय॒ज्ञ्स्य॑ । अन॑भिविद्ध॒मित्यन॑भि - वि॒द्ध॒म् । य॒ज्ञ्स्या॒भि । अ॒भि वि॑द्ध्येत् । वि॒द्ध्ये॒दग्रे॑ण । अग्रे॑ण॒ परि॑ । परि॑ हरति । ह॒र॒ति॒ ती॒र्त्थेन॑ । ती॒र्त्थेनै॒व । ए॒व परि॑ । परि॑ हरति । ह॒र॒ति॒ तत् । तत् पू॒ष्णे । पू॒ष्णे परि॑ । पर्य॑हरन्न् । अ॒ह॒र॒न् तत् । तत् पू॒षा \newline

\textbf{Jatai Paata} \newline

1. अ॒कृ॒न्त॒न्॒. यवे॑न॒ यवे॑नाकृन्तन् नकृन्त॒न्॒. यवे॑न । \newline
2. यवे॑न॒ सम्मि॑तꣳ॒॒ सम्मि॑तं॒ ॅयवे॑न॒ यवे॑न॒ सम्मि॑तम् । \newline
3. सम्मि॑त॒म् तस्मा॒त् तस्मा॒थ् सम्मि॑तꣳ॒॒ सम्मि॑त॒म् तस्मा᳚त् । \newline
4. सम्मि॑त॒मिति॒ सं - मि॒त॒म् । \newline
5. तस्मा᳚द् यवमा॒त्रं ॅय॑वमा॒त्रम् तस्मा॒त् तस्मा᳚द् यवमा॒त्रम् । \newline
6. य॒व॒मा॒त्र मवाव॑ यवमा॒त्रं ॅय॑वमा॒त्र मव॑ । \newline
7. य॒व॒मा॒त्रमिति॑ यव - मा॒त्रम् । \newline
8. अव॑ द्येद् द्ये॒ दवाव॑ द्येत् । \newline
9. द्ये॒द् यद् यद् द्ये᳚द् द्ये॒द् यत् । \newline
10. यज् ज्यायो॒ ज्यायो॒ यद् यज् ज्यायः॑ । \newline
11. ज्यायो॑ ऽव॒द्ये द॑व॒द्येज् ज्यायो॒ ज्यायो॑ ऽव॒द्येत् । \newline
12. अ॒व॒द्येद् रो॒पये᳚द् रो॒पये॑ दव॒द्ये द॑व॒द्येद् रो॒पये᳚त् । \newline
13. अ॒व॒द्येदित्य॑व - द्येत् । \newline
14. रो॒पये॒त् तत् तद् रो॒पये᳚द् रो॒पये॒त् तत् । \newline
15. तद् य॒ज्ञ्स्य॑ य॒ज्ञ्स्य॒ तत् तद् य॒ज्ञ्स्य॑ । \newline
16. य॒ज्ञ्स्य॒ यद् यद् य॒ज्ञ्स्य॑ य॒ज्ञ्स्य॒ यत् । \newline
17. यदुपोप॒ यद् यदुप॑ । \newline
18. उप॑ च॒ चोपोप॑ च । \newline
19. च॒ स्तृ॒णी॒याथ् स्तृ॑णी॒याच् च॑ च स्तृणी॒यात् । \newline
20. स्तृ॒णी॒या द॒भ्य॑भि स्तृ॑णी॒याथ् स्तृ॑णी॒या द॒भि । \newline
21. अ॒भि च॑ चा॒भ्य॑भि च॑ । \newline
22. च॒ घा॒रये᳚द् घा॒रये᳚च् च च घा॒रये᳚त् । \newline
23. घा॒रये॑ दुभयतःसꣳश्वा॒ य्यु॑भयतःसꣳश्वा॒यि घा॒रये᳚द् घा॒रये॑ दुभयतःसꣳश्वा॒यि । \newline
24. उ॒भ॒य॒तः॒सꣳ॒॒श्वा॒यि कु॑र्यात् कुर्या दुभयतःसꣳश्वा॒ य्यु॑भयतःसꣳश्वा॒यि कु॑र्यात् । \newline
25. उ॒भ॒य॒तः॒सꣳ॒॒श्वा॒यीत्यु॑भयतः - सꣳ॒॒श्वा॒यि । \newline
26. कु॒र्या॒ द॒व॒दाया॑ व॒दाय॑ कुर्यात् कुर्या दव॒दाय॑ । \newline
27. अ॒व॒दाया॒ भ्या᳚(1॒)भ्य॑ व॒दाया॑ व॒दाया॒भि । \newline
28. अ॒व॒दायेत्य॑व - दाय॑ । \newline
29. अ॒भि घा॑रयति घारय त्य॒भ्य॑भि घा॑रयति । \newline
30. घा॒र॒य॒ति॒ द्विर् द्विर् घा॑रयति घारयति॒ द्विः । \newline
31. द्विः सꣳ सम् द्विर् द्विः सम् । \newline
32. सम् प॑द्यते पद्यते॒ सꣳ सम् प॑द्यते । \newline
33. प॒द्य॒ते॒ द्वि॒पाद् द्वि॒पात् प॑द्यते पद्यते द्वि॒पात् । \newline
34. द्वि॒पाद् यज॑मानो॒ यज॑मानो द्वि॒पाद् द्वि॒पाद् यज॑मानः । \newline
35. द्वि॒पादिति॑ द्वि - पात् । \newline
36. यज॑मानः॒ प्रति॑ष्ठित्यै॒ प्रति॑ष्ठित्यै॒ यज॑मानो॒ यज॑मानः॒ प्रति॑ष्ठित्यै । \newline
37. प्रति॑ष्ठित्यै॒ यद् यत् प्रति॑ष्ठित्यै॒ प्रति॑ष्ठित्यै॒ यत् । \newline
38. प्रति॑ष्ठित्या॒ इति॒ प्रति॑ - स्थि॒त्यै॒ । \newline
39. यत् ति॑र॒श्चीन॑म् तिर॒श्चीनं॒ ॅयद् यत् ति॑र॒श्चीन᳚म् । \newline
40. ति॒र॒श्चीन॑ मति॒हरे॑ दति॒हरे᳚त् तिर॒श्चीन॑म् तिर॒श्चीन॑ मति॒हरे᳚त् । \newline
41. अ॒ति॒हरे॒ दन॑भिविद्ध॒ मन॑भिविद्ध मति॒हरे॑ दति॒हरे॒ दन॑भिविद्धम् । \newline
42. अ॒ति॒हरे॒दित्य॑ति - हरे᳚त् । \newline
43. अन॑भिविद्धं ॅय॒ज्ञ्स्य॑ य॒ज्ञ्स्या न॑भिविद्ध॒ मन॑भिविद्धं ॅय॒ज्ञ्स्य॑ । \newline
44. अन॑भिविद्ध॒मित्यन॑भि - वि॒द्ध॒म् । \newline
45. य॒ज्ञ्स्या॒भ्य॑भि य॒ज्ञ्स्य॑ य॒ज्ञ्स्या॒भि । \newline
46. अ॒भि वि॑द्ध्येद् विद्ध्ये द॒भ्य॑भि वि॑द्ध्येत् । \newline
47. वि॒द्ध्ये॒ दग्रे॒णाग्रे॑ण विद्ध्येद् विद्ध्ये॒ दग्रे॑ण । \newline
48. अग्रे॑ण॒ परि॒ पर्यग्रे॒णाग्रे॑ण॒ परि॑ । \newline
49. परि॑ हरति हरति॒ परि॒ परि॑ हरति । \newline
50. ह॒र॒ति॒ ती॒र्त्थेन॑ ती॒र्त्थेन॑ हरति हरति ती॒र्त्थेन॑ । \newline
51. ती॒र्त्थेनै॒वैव ती॒र्त्थेन॑ ती॒र्त्थेनै॒व । \newline
52. ए॒व परि॒ पर्ये॒वैव परि॑ । \newline
53. परि॑ हरति हरति॒ परि॒ परि॑ हरति । \newline
54. ह॒र॒ति॒ तत् त द्ध॑रति हरति॒ तत् । \newline
55. तत् पू॒ष्णे पू॒ष्णे तत् तत् पू॒ष्णे । \newline
56. पू॒ष्णे परि॒ परि॑ पू॒ष्णे पू॒ष्णे परि॑ । \newline
57. पर्य॑हरन् नहर॒न् परि॒ पर्य॑हरन्न् । \newline
58. अ॒ह॒र॒न् तत् तद॑हरन् नहर॒न् तत् । \newline
59. तत् पू॒षा पू॒षा तत् तत् पू॒षा । \newline

\textbf{Ghana Paata } \newline

1. अ॒कृ॒न्त॒न्॒. यवे॑न॒ यवे॑नाकृन्तन् नकृन्त॒न्॒. यवे॑न॒ सम्मि॑तꣳ॒॒ सम्मि॑तं॒ ॅयवे॑नाकृन्तन् नकृन्त॒न्॒. यवे॑न॒ सम्मि॑तम् । \newline
2. यवे॑न॒ सम्मि॑तꣳ॒॒ सम्मि॑तं॒ ॅयवे॑न॒ यवे॑न॒ सम्मि॑त॒म् तस्मा॒त् तस्मा॒थ् सम्मि॑तं॒ ॅयवे॑न॒ यवे॑न॒ सम्मि॑त॒म् तस्मा᳚त् । \newline
3. सम्मि॑त॒म् तस्मा॒त् तस्मा॒थ् सम्मि॑तꣳ॒॒ सम्मि॑त॒म् तस्मा᳚द् यवमा॒त्रं ॅय॑वमा॒त्रम् तस्मा॒थ् सम्मि॑तꣳ॒॒ सम्मि॑त॒म् तस्मा᳚द् यवमा॒त्रम् । \newline
4. सम्मि॑त॒मिति॒ सं - मि॒त॒म् । \newline
5. तस्मा᳚द् यवमा॒त्रं ॅय॑वमा॒त्रम् तस्मा॒त् तस्मा᳚द् यवमा॒त्र मवाव॑ यवमा॒त्रम् तस्मा॒त् तस्मा᳚द् यवमा॒त्र मव॑ । \newline
6. य॒व॒मा॒त्र मवाव॑ यवमा॒त्रं ॅय॑वमा॒त्र मव॑ द्येद् द्ये॒दव॑ यवमा॒त्रं ॅय॑वमा॒त्र मव॑ द्येत् । \newline
7. य॒व॒मा॒त्रमिति॑ यव - मा॒त्रम् । \newline
8. अव॑ द्येद् द्ये॒दवाव॑ द्ये॒द् यद् यद् द्ये॒दवाव॑ द्ये॒द् यत् । \newline
9. द्ये॒द् यद् यद् द्ये᳚द् द्ये॒द् यज् ज्यायो॒ ज्यायो॒ यद् द्ये᳚द् द्ये॒द् यज् ज्यायः॑ । \newline
10. यज् ज्यायो॒ ज्यायो॒ यद् यज् ज्यायो॑ ऽव॒द्ये द॑व॒द्येज् ज्यायो॒ यद् यज् ज्यायो॑ ऽव॒द्येत् । \newline
11. ज्यायो॑ ऽव॒द्ये द॑व॒द्येज् ज्यायो॒ ज्यायो॑ ऽव॒द्येद् रो॒पये᳚द् रो॒पये॑ दव॒द्येज् ज्यायो॒ ज्यायो॑ ऽव॒द्येद् रो॒पये᳚त् । \newline
12. अ॒व॒द्येद् रो॒पये᳚द् रो॒पये॑ दव॒द्ये द॑व॒द्येद् रो॒पये॒त् तत् तद् रो॒पये॑ दव॒द्ये द॑व॒द्येद् रो॒पये॒त् तत् । \newline
13. अ॒व॒द्येदित्य॑व - द्येत् । \newline
14. रो॒पये॒त् तत् तद् रो॒पये᳚द् रो॒पये॒त् तद् य॒ज्ञ्स्य॑ य॒ज्ञ्स्य॒ तद् रो॒पये᳚द् रो॒पये॒त् तद् य॒ज्ञ्स्य॑ । \newline
15. तद् य॒ज्ञ्स्य॑ य॒ज्ञ्स्य॒ तत् तद् य॒ज्ञ्स्य॒ यद् यद् य॒ज्ञ्स्य॒ तत् तद् य॒ज्ञ्स्य॒ यत् । \newline
16. य॒ज्ञ्स्य॒ यद् यद् य॒ज्ञ्स्य॑ य॒ज्ञ्स्य॒ यदुपोप॒ यद् य॒ज्ञ्स्य॑ य॒ज्ञ्स्य॒ यदुप॑ । \newline
17. यदुपोप॒ यद् यदुप॑ च॒ चोप॒ यद् यदुप॑ च । \newline
18. उप॑ च॒ चोपोप॑ च स्तृणी॒याथ् स्तृ॑णी॒याच् चोपोप॑ च स्तृणी॒यात् । \newline
19. च॒ स्तृ॒णी॒याथ् स्तृ॑णी॒याच् च॑ च स्तृणी॒या द॒भ्य॑भि स्तृ॑णी॒याच् च॑ च स्तृणी॒या द॒भि । \newline
20. स्तृ॒णी॒या द॒भ्य॑भि स्तृ॑णी॒याथ् स्तृ॑णी॒या द॒भि च॑ चा॒भि स्तृ॑णी॒याथ् स्तृ॑णी॒या द॒भि च॑ । \newline
21. अ॒भि च॑ चा॒भ्य॑भि च॑ घा॒रये᳚द् घा॒रये᳚च् चा॒भ्य॑भि च॑ घा॒रये᳚त् । \newline
22. च॒ घा॒रये᳚द् घा॒रये᳚च् च च घा॒रये॑ दुभयतःसꣳश्वा॒ य्यु॑भयतःसꣳश्वा॒यि घा॒रये᳚च् च च घा॒रये॑ 
दुभयतःसꣳश्वा॒यि । \newline
23. घा॒रये॑ दुभयतःसꣳश्वा॒ य्यु॑भयतःसꣳश्वा॒यि घा॒रये᳚द् घा॒रये॑ दुभयतःसꣳश्वा॒यि कु॑र्यात् कुर्या 
दुभयतःसꣳश्वा॒यि घा॒रये᳚द् घा॒रये॑ दुभयतःसꣳश्वा॒यि कु॑र्यात् । \newline
24. उ॒भ॒य॒तः॒सꣳ॒॒श्वा॒यि कु॑र्यात् कुर्या दुभयतःसꣳश्वा॒ य्यु॑भयतःसꣳश्वा॒यि कु॑र्या दव॒दाया॑ व॒दाय॑ कुर्या दुभयतःसꣳश्वा॒ य्यु॑भयतःसꣳश्वा॒यि कु॑र्यादव॒दाय॑ । \newline
25. उ॒भ॒य॒तः॒सꣳ॒॒श्वा॒यीत्यु॑भयतः - सꣳ॒॒श्वा॒यि । \newline
26. कु॒र्या॒ द॒व॒दाया॑ व॒दाय॑ कुर्यात् कुर्या दव॒दाया॒भ्या᳚(1॒)भ्य॑व॒दाय॑ कुर्यात् कुर्या दव॒दाया॒भि । \newline
27. अ॒व॒दाया॒भ्या᳚(1॒)भ्य॑ व॒दाया॑ व॒दाया॒भि घा॑रयति घारय त्य॒भ्य॑व॒दाया॑ व॒दाया॒भि घा॑रयति । \newline
28. अ॒व॒दायेत्य॑व - दाय॑ । \newline
29. अ॒भि घा॑रयति घारय त्य॒भ्य॑भि घा॑रयति॒ द्विर् द्विर् घा॑रय त्य॒भ्य॑भि घा॑रयति॒ द्विः । \newline
30. घा॒र॒य॒ति॒ द्विर् द्विर् घा॑रयति घारयति॒ द्विः सꣳ सम् द्विर् घा॑रयति घारयति॒ द्विः सम् । \newline
31. द्विः सꣳ सम् द्विर् द्विः सम् प॑द्यते पद्यते॒ सम् द्विर् द्विः सम् प॑द्यते । \newline
32. सम् प॑द्यते पद्यते॒ सꣳ सम् प॑द्यते द्वि॒पाद् द्वि॒पात् प॑द्यते॒ सꣳ सम् प॑द्यते द्वि॒पात् । \newline
33. प॒द्य॒ते॒ द्वि॒पाद् द्वि॒पात् प॑द्यते पद्यते द्वि॒पाद् यज॑मानो॒ यज॑मानो द्वि॒पात् प॑द्यते पद्यते द्वि॒पाद् यज॑मानः । \newline
34. द्वि॒पाद् यज॑मानो॒ यज॑मानो द्वि॒पाद् द्वि॒पाद् यज॑मानः॒ प्रति॑ष्ठित्यै॒ प्रति॑ष्ठित्यै॒ यज॑मानो द्वि॒पाद् द्वि॒पाद् यज॑मानः॒ प्रति॑ष्ठित्यै । \newline
35. द्वि॒पादिति॑ द्वि - पात् । \newline
36. यज॑मानः॒ प्रति॑ष्ठित्यै॒ प्रति॑ष्ठित्यै॒ यज॑मानो॒ यज॑मानः॒ प्रति॑ष्ठित्यै॒ यद् यत् प्रति॑ष्ठित्यै॒ यज॑मानो॒ यज॑मानः॒ प्रति॑ष्ठित्यै॒ यत् । \newline
37. प्रति॑ष्ठित्यै॒ यद् यत् प्रति॑ष्ठित्यै॒ प्रति॑ष्ठित्यै॒ यत् ति॑र॒श्चीन॑म् तिर॒श्चीनं॒ ॅयत् प्रति॑ष्ठित्यै॒ प्रति॑ष्ठित्यै॒ यत् ति॑र॒श्चीन᳚म् । \newline
38. प्रति॑ष्ठित्या॒ इति॒ प्रति॑ - स्थि॒त्यै॒ । \newline
39. यत् ति॑र॒श्चीन॑म् तिर॒श्चीनं॒ ॅयद् यत् ति॑र॒श्चीन॑ मति॒हरे॑ दति॒हरे᳚त् तिर॒श्चीनं॒ ॅयद् यत् ति॑र॒श्चीन॑ मति॒हरे᳚त् । \newline
40. ति॒र॒श्चीन॑ मति॒हरे॑ दति॒हरे᳚त् तिर॒श्चीन॑म् तिर॒श्चीन॑ मति॒हरे॒ दन॑भिविद्ध॒ मन॑भिविद्ध मति॒हरे᳚त् तिर॒श्चीन॑म् तिर॒श्चीन॑ मति॒हरे॒ दन॑भिविद्धम् । \newline
41. अ॒ति॒हरे॒ दन॑भिविद्ध॒ मन॑भिविद्ध मति॒हरे॑ दति॒हरे॒ दन॑भिविद्धं ॅय॒ज्ञ्स्य॑ य॒ज्ञ्स्या न॑भिविद्ध मति॒हरे॑ दति॒हरे॒ दन॑भिविद्धं ॅय॒ज्ञ्स्य॑ । \newline
42. अ॒ति॒हरे॒दित्य॑ति - हरे᳚त् । \newline
43. अन॑भिविद्धं ॅय॒ज्ञ्स्य॑ य॒ज्ञ्स्यान॑ भिविद्ध॒ मन॑भिविद्धं ॅय॒ज्ञ्स्या॒ भ्य॑भि य॒ज्ञ्स्यान॑ भिविद्ध॒ मन॑भिविद्धं ॅय॒ज्ञ्स्या॒भि । \newline
44. अन॑भिविद्ध॒मित्यन॑भि - वि॒द्ध॒म् । \newline
45. य॒ज्ञ्स्या॒भ्य॑भि य॒ज्ञ्स्य॑ य॒ज्ञ्स्या॒भि वि॑द्ध्येद् विद्ध्येद॒भि य॒ज्ञ्स्य॑ य॒ज्ञ्स्या॒भि वि॑द्ध्येत् । \newline
46. अ॒भि वि॑द्ध्येद् विद्ध्ये द॒भ्य॑भि वि॑द्ध्ये॒ दग्रे॒णाग्रे॑ण विद्ध्ये द॒भ्य॑भि वि॑द्ध्ये॒ दग्रे॑ण । \newline
47. वि॒द्ध्ये॒ दग्रे॒णाग्रे॑ण विद्ध्येद् विद्ध्ये॒ दग्रे॑ण॒ परि॒ पर्यग्रे॑ण विद्ध्येद् विद्ध्ये॒ दग्रे॑ण॒ परि॑ । \newline
48. अग्रे॑ण॒ परि॒ पर्यग्रे॒णा ग्रे॑ण॒ परि॑ हरति हरति॒ पर्यग्रे॒णा ग्रे॑ण॒ परि॑ हरति । \newline
49. परि॑ हरति हरति॒ परि॒ परि॑ हरति ती॒र्त्थेन॑ ती॒र्त्थेन॑ हरति॒ परि॒ परि॑ हरति ती॒र्त्थेन॑ । \newline
50. ह॒र॒ति॒ ती॒र्त्थेन॑ ती॒र्त्थेन॑ हरति हरति ती॒र्त्थेनै॒वैव ती॒र्त्थेन॑ हरति हरति ती॒र्त्थेनै॒व । \newline
51. ती॒र्त्थेनै॒वैव ती॒र्त्थेन॑ ती॒र्त्थेनै॒व परि॒ पर्ये॒व ती॒र्त्थेन॑ ती॒र्त्थेनै॒व परि॑ । \newline
52. ए॒व परि॒ पर्ये॒वैव परि॑ हरति हरति॒ पर्ये॒वैव परि॑ हरति । \newline
53. परि॑ हरति हरति॒ परि॒ परि॑ हरति॒ तत् तद्ध॑रति॒ परि॒ परि॑ हरति॒ तत् । \newline
54. ह॒र॒ति॒ तत् तद्ध॑रति हरति॒ तत् पू॒ष्णे पू॒ष्णे तद्ध॑रति हरति॒ तत् पू॒ष्णे । \newline
55. तत् पू॒ष्णे पू॒ष्णे तत् तत् पू॒ष्णे परि॒ परि॑ पू॒ष्णे तत् तत् पू॒ष्णे परि॑ । \newline
56. पू॒ष्णे परि॒ परि॑ पू॒ष्णे पू॒ष्णे पर्य॑हरन् नहर॒न् परि॑ पू॒ष्णे पू॒ष्णे पर्य॑हरन्न् । \newline
57. पर्य॑हरन् नहर॒न् परि॒ पर्य॑हर॒न् तत् तद॑हर॒न् परि॒ पर्य॑हर॒न् तत् । \newline
58. अ॒ह॒र॒न् तत् तद॑हरन् नहर॒न् तत् पू॒षा पू॒षा तद॑हरन् नहर॒न् तत् पू॒षा । \newline
59. तत् पू॒षा पू॒षा तत् तत् पू॒षा प्राश्य॒ प्राश्य॑ पू॒षा तत् तत् पू॒षा प्राश्य॑ । \newline
\pagebreak
\markright{ TS 2.6.8.5  \hfill https://www.vedavms.in \hfill}
\addcontentsline{toc}{section}{ TS 2.6.8.5 }
\section*{ TS 2.6.8.5 }

\textbf{TS 2.6.8.5 } \newline
\textbf{Samhita Paata} \newline

पू॒षा प्राश्य॑ द॒तो॑ऽरुण॒त् तस्मा᳚त् पू॒षा प्र॑पि॒ष्टभा॑गोऽद॒न्तको॒ हि तं दे॒वा अ॑ब्रुव॒न् वि वा अ॒यमा᳚र्द्ध्यप्राशित्रि॒यो वा अ॒यम॑भू॒दिति॒ तद्-बृह॒स्पत॑ये॒ पर्य॑हर॒न्थ् सो॑ऽबिभे॒द्-बृह॒स्पति॑रि॒त्थं ॅवाव स्य आर्ति॒माऽरि॑ष्य॒तीति॒ स ए॒तं मन्त्र॑मपश्य॒थ् सूर्य॑स्य त्वा॒ चक्षु॑षा॒ प्रति॑ पश्या॒मीत्य॑ब्रवी॒न्न हि सूर्य॑स्य॒ चक्षुः॒ - [  ] \newline

\textbf{Pada Paata} \newline

पू॒षा । प्राश्येति॑ प्र - अश्य॑ । द॒तः । अ॒रु॒ण॒त् । तस्मा᳚त् । पू॒षा । प्र॒पि॒ष्टभा॑ग॒ इति॑ प्रपि॒ष्ट - भा॒गः॒ । अ॒द॒न्तकः॑ । हि । तम् । दे॒वाः । अ॒ब्रु॒व॒न्न् । वीति॑ । वै । अ॒यम् । आ॒र्द्धि॒ । अ॒प्रा॒शि॒त्रि॒य इत्य॑प्र - अ॒शि॒त्रि॒यः । वै । अ॒यम् । अ॒भू॒त् । इति॑ । तत् । बृह॒स्पत॑ये । परीति॑ । अ॒ह॒र॒न्न् । सः । अ॒बि॒भे॒त् । बृह॒स्पतिः॑ । इ॒त्थम् । वाव । स्यः । आर्ति᳚म् । एति॑ । अ॒रि॒ष्य॒ति॒ । इति॑ । सः । ए॒तम् । मन्त्र᳚म् । अ॒प॒श्य॒त् । सूर्य॑स्य । त्वा॒ । चक्षु॑षा । प्रतीति॑ । प॒श्या॒मि॒ । इति॑ । अ॒ब्र॒वी॒त् । न । हि । सूर्य॑स्य । चक्षुः॑ ।  \newline


\textbf{Krama Paata} \newline

पू॒षा प्राश्य॑ । प्राश्य॑ द॒तः । प्राश्येति॑ प्र - अश्य॑ । द॒तो॑ऽरुणत् । अ॒रु॒ण॒त् तस्मा᳚त् । तस्मा᳚त् पू॒षा । पू॒षा प्र॑पि॒ष्टभा॑गः । प्र॒पि॒ष्टभा॑गो ऽद॒न्तकः॑ । प्र॒पि॒ष्टभा॑ग॒ इति॑ प्रपि॒ष्ट - भा॒गः॒ । अ॒द॒न्तको॒ हि । हि तम् । तम् दे॒वाः । दे॒वा अ॑ब्रुवन्न् । अ॒ब्रु॒व॒न् वि । वि वै । वा अ॒यम् । अ॒यमा᳚र्द्धि । आ॒र्द्ध्य॒प्रा॒शि॒त्रि॒यः । अ॒प्रा॒शि॒त्रि॒यो वै । अ॒प्रा॒शि॒त्रि॒य इत्य॑प्र - अ॒शि॒त्रि॒यः । वा अ॒यम् । अ॒यम॑भूत् । अ॒भू॒दिति॑ । इति॒ तत् । तद् बृह॒स्पत॑ये । बृह॒स्पत॑ये॒ परि॑ । पर्य॑हरन्न् । अ॒ह॒र॒न्थ् सः । सो॑ऽबिभेत् । अ॒बि॒भे॒द् बृह॒स्पतिः॑ । बृह॒स्पति॑रि॒त्थम् । इ॒त्थं ॅवाव । वाव स्यः । स्य आर्ति᳚म् । आर्ति॒मा । आ ऽरि॑ष्यति । अ॒रि॒ष्य॒तीति॑ । इति॒ सः । स ए॒तम् । ए॒तम् मन्त्र᳚म् । मन्त्र॑मपश्यत् । अ॒प॒श्य॒थ् सूर्य॑स्य । सूर्य॑स्य त्वा । त्वा॒ चक्षु॑षा । चक्षु॑षा॒ प्रति॑ । प्रति॑ पश्यामि । प॒श्या॒मीति॑ । इत्य॑ब्रवीत् । अ॒ब्र॒वी॒न् न । न हि । हि सूर्य॑स्य । सूर्य॑स्य॒ चक्षुः॑ । चक्षुः॒ किम् \newline

\textbf{Jatai Paata} \newline

1. पू॒षा प्राश्य॒ प्राश्य॑ पू॒षा पू॒षा प्राश्य॑ । \newline
2. प्राश्य॑ द॒तो द॒तः प्राश्य॒ प्राश्य॑ द॒तः । \newline
3. प्राश्येति॑ प्र - अश्य॑ । \newline
4. द॒तो॑ ऽरुण दरुणद् द॒तो द॒तो॑ ऽरुणत् । \newline
5. अ॒रु॒ण॒त् तस्मा॒त् तस्मा॑ दरुण दरुण॒त् तस्मा᳚त् । \newline
6. तस्मा᳚त् पू॒षा पू॒षा तस्मा॒त् तस्मा᳚त् पू॒षा । \newline
7. पू॒षा प्र॑पि॒ष्टभा॑गः प्रपि॒ष्टभा॑गः पू॒षा पू॒षा प्र॑पि॒ष्टभा॑गः । \newline
8. प्र॒पि॒ष्टभा॑गो ऽद॒न्तको॑ ऽद॒न्तकः॑ प्रपि॒ष्टभा॑गः प्रपि॒ष्टभा॑गो ऽद॒न्तकः॑ । \newline
9. प्र॒पि॒ष्टभा॑ग॒ इति॑ प्रपि॒ष्ट - भा॒गः॒ । \newline
10. अ॒द॒न्तको॒ हि ह्य॑द॒न्तको॑ ऽद॒न्तको॒ हि । \newline
11. हि तम् तꣳ हि हि तम् । \newline
12. तम् दे॒वा दे॒वा स्तम् तम् दे॒वाः । \newline
13. दे॒वा अ॑ब्रुवन् नब्रुवन् दे॒वा दे॒वा अ॑ब्रुवन्न् । \newline
14. अ॒ब्रु॒व॒न् वि व्य॑ब्रुवन् नब्रुव॒न् वि । \newline
15. वि वै वै वि वि वै । \newline
16. वा अ॒य म॒यं ॅवै वा अ॒यम् । \newline
17. अ॒य मा᳚र्द्ध्या र्द्ध्य॒य म॒य मा᳚र्द्धि । \newline
18. आ॒र्द्ध्य॒प्रा॒शि॒त्रि॒यो᳚ ऽप्राशित्रि॒य आ᳚र्द्ध्या र्द्ध्य प्राशित्रि॒यः । \newline
19. अ॒प्रा॒शि॒त्रि॒यो वै वा अ॑प्राशित्रि॒यो᳚ ऽप्राशित्रि॒यो वै । \newline
20. अ॒प्रा॒शि॒त्रि॒य इत्य॑प्र - अ॒शि॒त्रि॒यः । \newline
21. वा अ॒य म॒यं ॅवै वा अ॒यम् । \newline
22. अ॒य म॑भू दभू द॒य म॒य म॑भूत् । \newline
23. अ॒भू॒ दिती त्य॑भू दभू॒ दिति॑ । \newline
24. इति॒ तत् तदितीति॒ तत् । \newline
25. तद् बृह॒स्पत॑ये॒ बृह॒स्पत॑ये॒ तत् तद् बृह॒स्पत॑ये । \newline
26. बृह॒स्पत॑ये॒ परि॒ परि॒ बृह॒स्पत॑ये॒ बृह॒स्पत॑ये॒ परि॑ । \newline
27. पर्य॑हरन् नहर॒न् परि॒ पर्य॑हरन्न् । \newline
28. अ॒ह॒र॒न् थ्स सो॑ ऽहरन् नहर॒न् थ्सः । \newline
29. सो॑ ऽबिभे दबिभे॒थ् स सो॑ ऽबिभेत् । \newline
30. अ॒बि॒भे॒द् बृह॒स्पति॒र् बृह॒स्पति॑ रबिभे दबिभे॒द् बृह॒स्पतिः॑ । \newline
31. बृह॒स्पति॑ रि॒त्थ मि॒त्थम् बृह॒स्पति॒र् बृह॒स्पति॑ रि॒त्थम् । \newline
32. इ॒त्थं ॅवाव वावे त्थ मि॒त्थं ॅवाव । \newline
33. वाव स्य स्य वाव वाव स्यः । \newline
34. स्य आर्ति॒ मार्तिꣳ॒॒ स्य स्य आर्ति᳚म् । \newline
35. आर्ति॒ मा ऽऽर्ति॒ मार्ति॒ मा । \newline
36. आ ऽरि॑ष्य त्यरिष्य॒त्या ऽरि॑ष्यति । \newline
37. अ॒रि॒ष्य॒तीती त्य॑रिष्य त्यरिष्य॒तीति॑ । \newline
38. इति॒ स स इतीति॒ सः । \newline
39. स ए॒त मे॒तꣳ स स ए॒तम् । \newline
40. ए॒तम् मन्त्र॒म् मन्त्र॑ मे॒त मे॒तम् मन्त्र᳚म् । \newline
41. मन्त्र॑ मपश्य दपश्य॒न् मन्त्र॒म् मन्त्र॑ मपश्यत् । \newline
42. अ॒प॒श्य॒थ् सूर्य॑स्य॒ सूर्य॑स्या पश्य दपश्य॒थ् सूर्य॑स्य । \newline
43. सूर्य॑स्य त्वा त्वा॒ सूर्य॑स्य॒ सूर्य॑स्य त्वा । \newline
44. त्वा॒ चक्षु॑षा॒ चक्षु॑षा त्वा त्वा॒ चक्षु॑षा । \newline
45. चक्षु॑षा॒ प्रति॒ प्रति॒ चक्षु॑षा॒ चक्षु॑षा॒ प्रति॑ । \newline
46. प्रति॑ पश्यामि पश्यामि॒ प्रति॒ प्रति॑ पश्यामि । \newline
47. प॒श्या॒मीतीति॑ पश्यामि पश्या॒मीति॑ । \newline
48. इत्य॑ब्रवी दब्रवी॒ दिती त्य॑ब्रवीत् । \newline
49. अ॒ब्र॒वी॒न् न नाब्र॑वी दब्रवी॒न् न । \newline
50. न हि हि न न हि । \newline
51. हि सूर्य॑स्य॒ सूर्य॑स्य॒ हि हि सूर्य॑स्य । \newline
52. सूर्य॑स्य॒ चक्षु॒ श्चक्षुः॒ सूर्य॑स्य॒ सूर्य॑स्य॒ चक्षुः॑ । \newline
53. चक्षुः॒ किम् किम् चक्षु॒ श्चक्षुः॒ किम् । \newline

\textbf{Ghana Paata } \newline

1. पू॒षा प्राश्य॒ प्राश्य॑ पू॒षा पू॒षा प्राश्य॑ द॒तो द॒तः प्राश्य॑ पू॒षा पू॒षा प्राश्य॑ द॒तः । \newline
2. प्राश्य॑ द॒तो द॒तः प्राश्य॒ प्राश्य॑ द॒तो॑ ऽरुण दरुणद् द॒तः प्राश्य॒ प्राश्य॑ द॒तो॑ ऽरुणत् । \newline
3. प्राश्येति॑ प्र - अश्य॑ । \newline
4. द॒तो॑ ऽरुण दरुणद् द॒तो द॒तो॑ ऽरुण॒त् तस्मा॒त् तस्मा॑ दरुणद् द॒तो द॒तो॑ ऽरुण॒त् तस्मा᳚त् । \newline
5. अ॒रु॒ण॒त् तस्मा॒त् तस्मा॑ दरुण दरुण॒त् तस्मा᳚त् पू॒षा पू॒षा तस्मा॑ दरुण दरुण॒त् तस्मा᳚त् पू॒षा । \newline
6. तस्मा᳚त् पू॒षा पू॒षा तस्मा॒त् तस्मा᳚त् पू॒षा प्र॑पि॒ष्टभा॑गः प्रपि॒ष्टभा॑गः पू॒षा तस्मा॒त् तस्मा᳚त् पू॒षा प्र॑पि॒ष्टभा॑गः । \newline
7. पू॒षा प्र॑पि॒ष्टभा॑गः प्रपि॒ष्टभा॑गः पू॒षा पू॒षा प्र॑पि॒ष्टभा॑गो ऽद॒न्तको॑ ऽद॒न्तकः॑ प्रपि॒ष्टभा॑गः पू॒षा पू॒षा प्र॑पि॒ष्टभा॑गो ऽद॒न्तकः॑ । \newline
8. प्र॒पि॒ष्टभा॑गो ऽद॒न्तको॑ ऽद॒न्तकः॑ प्रपि॒ष्टभा॑गः प्रपि॒ष्टभा॑गो ऽद॒न्तको॒ हि ह्य॑द॒न्तकः॑ प्रपि॒ष्टभा॑गः प्रपि॒ष्टभा॑गो ऽद॒न्तको॒ हि । \newline
9. प्र॒पि॒ष्टभा॑ग॒ इति॑ प्रपि॒ष्ट - भा॒गः॒ । \newline
10. अ॒द॒न्तको॒ हि ह्य॑द॒न्तको॑ ऽद॒न्तको॒ हि तम् तꣳ ह्य॑द॒न्तको॑ ऽद॒न्तको॒ हि तम् । \newline
11. हि तम् तꣳ हि हि तम् दे॒वा दे॒वा स्तꣳ हि हि तम् दे॒वाः । \newline
12. तम् दे॒वा दे॒वा स्तम् तम् दे॒वा अ॑ब्रुवन् नब्रुवन् दे॒वा स्तम् तम् दे॒वा अ॑ब्रुवन्न् । \newline
13. दे॒वा अ॑ब्रुवन् नब्रुवन् दे॒वा दे॒वा अ॑ब्रुव॒न् वि व्य॑ब्रुवन् दे॒वा दे॒वा अ॑ब्रुव॒न् वि । \newline
14. अ॒ब्रु॒व॒न् वि व्य॑ब्रुवन् नब्रुव॒न् वि वै वै व्य॑ब्रुवन् नब्रुव॒न् वि वै । \newline
15. वि वै वै वि वि वा अ॒य म॒यं ॅवै वि वि वा अ॒यम् । \newline
16. वा अ॒य म॒यं ॅवै वा अ॒य मा᳚र्द्ध्यार्द्ध्य॒यं ॅवै वा अ॒य मा᳚र्द्धि । \newline
17. अ॒य मा᳚र्द्ध्या र्द्ध्य॒य म॒य मा᳚र्द्ध्यप्राशित्रि॒यो᳚ ऽप्राशित्रि॒य आ᳚र्द्ध्य॒य म॒य मा᳚र्द्ध्यप्राशित्रि॒यः । \newline
18. आ॒र्द्ध्य॒प्रा॒शि॒त्रि॒यो᳚ ऽप्राशित्रि॒य आ᳚र्द्ध्या र्द्ध्यप्राशित्रि॒यो वै वा अ॑प्राशित्रि॒य आ᳚र्द्ध्या र्द्ध्यप्राशित्रि॒यो वै । \newline
19. अ॒प्रा॒शि॒त्रि॒यो वै वा अ॑प्राशित्रि॒यो᳚ ऽप्राशित्रि॒यो वा अ॒य म॒यं ॅवा अ॑प्राशित्रि॒यो᳚ ऽप्राशित्रि॒यो वा अ॒यम् । \newline
20. अ॒प्रा॒शि॒त्रि॒य इत्य॑प्र - अ॒शि॒त्रि॒यः । \newline
21. वा अ॒य म॒यं ॅवै वा अ॒य म॑भू दभू द॒यं ॅवै वा अ॒य म॑भूत् । \newline
22. अ॒य म॑भू दभू द॒य म॒य म॑भू॒ दिती त्य॑भू द॒य म॒य म॑भू॒दिति॑ । \newline
23. अ॒भू॒ दिती त्य॑भू दभू॒ दिति॒ तत् तदि त्य॑भू दभू॒ दिति॒ तत् । \newline
24. इति॒ तत् तदितीति॒ तद् बृह॒स्पत॑ये॒ बृह॒स्पत॑ये॒ तदितीति॒ तद् बृह॒स्पत॑ये । \newline
25. तद् बृह॒स्पत॑ये॒ बृह॒स्पत॑ये॒ तत् तद् बृह॒स्पत॑ये॒ परि॒ परि॒ बृह॒स्पत॑ये॒ तत् तद् बृह॒स्पत॑ये॒ परि॑ । \newline
26. बृह॒स्पत॑ये॒ परि॒ परि॒ बृह॒स्पत॑ये॒ बृह॒स्पत॑ये॒ पर्य॑हरन् नहर॒न् परि॒ बृह॒स्पत॑ये॒ बृह॒स्पत॑ये॒ पर्य॑हरन्न् । \newline
27. पर्य॑हरन् नहर॒न् परि॒ पर्य॑हर॒न् थ्स सो॑ ऽहर॒न् परि॒ पर्य॑हर॒न् थ्सः । \newline
28. अ॒ह॒र॒न् थ्स सो॑ ऽहरन् नहर॒न् थ्सो॑ ऽबिभे दबिभे॒थ् सो॑ ऽहरन् नहर॒न् थ्सो॑ ऽबिभेत् । \newline
29. सो॑ ऽबिभे दबिभे॒थ् स सो॑ ऽबिभे॒द् बृह॒स्पति॒र् बृह॒स्पति॑ रबिभे॒थ् स सो॑ ऽबिभे॒द् बृह॒स्पतिः॑ । \newline
30. अ॒बि॒भे॒द् बृह॒स्पति॒र् बृह॒स्पति॑ रबिभे दबिभे॒द् बृह॒स्पति॑रि॒त्थ मि॒त्थम् बृह॒स्पति॑ रबिभे दबिभे॒द् बृह॒स्पति॑ रि॒त्थम् । \newline
31. बृह॒स्पति॑ रि॒त्थ मि॒त्थम् बृह॒स्पति॒र् बृह॒स्पति॑ रि॒त्थं ॅवाव वावे त्थम् बृह॒स्पति॒र् बृह॒स्पति॑ रि॒त्थं ॅवाव । \newline
32. इ॒त्थं ॅवाव वावे त्थ मि॒त्थं ॅवाव स्य स्य वावे त्थ मि॒त्थं ॅवाव स्यः । \newline
33. वाव स्य स्य वाव वाव स्य आर्ति॒ मार्तिꣳ॒॒ स्य वाव वाव स्य आर्ति᳚म् । \newline
34. स्य आर्ति॒ मार्तिꣳ॒॒ स्य स्य आर्ति॒ मा ऽऽर्तिꣳ॒॒ स्य स्य आर्ति॒ मा । \newline
35. आर्ति॒ मा ऽऽर्ति॒ मार्ति॒ मा ऽरि॑ष्य त्यरिष्य॒त्या ऽऽर्ति॒ मार्ति॒ मा ऽरि॑ष्यति । \newline
36. आ ऽरि॑ष्य त्यरिष्य॒त्या ऽरि॑ष्य॒तीती त्य॑रिष्य॒त्या ऽरि॑ष्य॒तीति॑ । \newline
37. अ॒रि॒ष्य॒तीती त्य॑रिष्य त्यरिष्य॒तीति॒ स स इत्य॑रिष्य त्यरिष्य॒तीति॒ सः । \newline
38. इति॒ स स इतीति॒ स ए॒त मे॒तꣳ स इतीति॒ स ए॒तम् । \newline
39. स ए॒त मे॒तꣳ स स ए॒तम् मन्त्र॒म् मन्त्र॑ मे॒तꣳ स स ए॒तम् मन्त्र᳚म् । \newline
40. ए॒तम् मन्त्र॒म् मन्त्र॑ मे॒त मे॒तम् मन्त्र॑ मपश्य दपश्य॒न् मन्त्र॑ मे॒त मे॒तम् मन्त्र॑ मपश्यत् । \newline
41. मन्त्र॑ मपश्य दपश्य॒न् मन्त्र॒म् मन्त्र॑ मपश्य॒थ् सूर्य॑स्य॒ सूर्य॑स्या पश्य॒न् मन्त्र॒म् मन्त्र॑ मपश्य॒थ् सूर्य॑स्य । \newline
42. अ॒प॒श्य॒थ् सूर्य॑स्य॒ सूर्य॑स्या पश्य दपश्य॒थ् सूर्य॑स्य त्वा त्वा॒ सूर्य॑स्या पश्य दपश्य॒थ् सूर्य॑स्य त्वा । \newline
43. सूर्य॑स्य त्वा त्वा॒ सूर्य॑स्य॒ सूर्य॑स्य त्वा॒ चक्षु॑षा॒ चक्षु॑षा त्वा॒ सूर्य॑स्य॒ सूर्य॑स्य त्वा॒ चक्षु॑षा । \newline
44. त्वा॒ चक्षु॑षा॒ चक्षु॑षा त्वा त्वा॒ चक्षु॑षा॒ प्रति॒ प्रति॒ चक्षु॑षा त्वा त्वा॒ चक्षु॑षा॒ प्रति॑ । \newline
45. चक्षु॑षा॒ प्रति॒ प्रति॒ चक्षु॑षा॒ चक्षु॑षा॒ प्रति॑ पश्यामि पश्यामि॒ प्रति॒ चक्षु॑षा॒ चक्षु॑षा॒ प्रति॑ पश्यामि । \newline
46. प्रति॑ पश्यामि पश्यामि॒ प्रति॒ प्रति॑ पश्या॒मीतीति॑ पश्यामि॒ प्रति॒ प्रति॑ पश्या॒मीति॑ । \newline
47. प॒श्या॒मीतीति॑ पश्यामि पश्या॒मी त्य॑ब्रवी दब्रवी॒ दिति॑ पश्यामि पश्या॒मी त्य॑ब्रवीत् । \newline
48. इत्य॑ब्रवी दब्रवी॒ दिती त्य॑ब्रवी॒न् न नाब्र॑वी॒ दिती त्य॑ब्रवी॒न् न । \newline
49. अ॒ब्र॒वी॒न् न नाब्र॑वी दब्रवी॒न् न हि हि नाब्र॑वी दब्रवी॒न् न हि । \newline
50. न हि हि न न हि सूर्य॑स्य॒ सूर्य॑स्य॒ हि न न हि सूर्य॑स्य । \newline
51. हि सूर्य॑स्य॒ सूर्य॑स्य॒ हि हि सूर्य॑स्य॒ चक्षु॒ श्चक्षुः॒ सूर्य॑स्य॒ हि हि सूर्य॑स्य॒ चक्षुः॑ । \newline
52. सूर्य॑स्य॒ चक्षु॒ श्चक्षुः॒ सूर्य॑स्य॒ सूर्य॑स्य॒ चक्षुः॒ किम् किम् चक्षुः॒ सूर्य॑स्य॒ सूर्य॑स्य॒ चक्षुः॒ किम् । \newline
53. चक्षुः॒ किम् किम् चक्षु॒ श्चक्षुः॒ किम् च॒न च॒न किम् चक्षु॒ श्चक्षुः॒ किम् च॒न । \newline
\pagebreak
\markright{ TS 2.6.8.6  \hfill https://www.vedavms.in \hfill}
\addcontentsline{toc}{section}{ TS 2.6.8.6 }
\section*{ TS 2.6.8.6 }

\textbf{TS 2.6.8.6 } \newline
\textbf{Samhita Paata} \newline

किं च॒न हि॒नस्ति॒ सो॑ऽबिभेत् प्रतिगृ॒ह्णन्तं॑ मा हिꣳसिष्य॒तीति॑ दे॒वस्य॑ त्वा सवि॒तुः प्र॑स॒वे᳚ऽश्विनो᳚ र्बा॒हुभ्यां᳚ पू॒ष्णो हस्ता᳚भ्यां॒ प्रति॑ गृह्णा॒मीत्य॑ब्रवीथ् सवि॒तृप्र॑सूत ए॒वैन॒द्ब्रह्म॑णा दे॒वता॑भिः॒ प्रत्य॑गृह्णा॒थ् सो॑ऽबिभेत् प्रा॒श्नन्तं॑ मा हिꣳसिष्य॒तीत्य॒ग्नेस्त्वा॒ ऽऽस्ये॑न॒ प्राश्ना॒मीत्य॑ब्रवी॒न्न ह्य॑ग्नेरा॒स्यं॑ किञ्च॒न हि॒नस्ति॒ सो॑ऽबिभे॒त् - [  ] \newline

\textbf{Pada Paata} \newline

किम् । च॒न । हि॒नस्ति॑ । सः । अ॒बि॒भे॒त् । प्र॒ति॒गृ॒ह्णन्त॒मिति॑ प्रति - गृ॒ह्णन्त᳚म् । मा॒ । हिꣳ॒॒सि॒ष्य॒ति॒ । इति॑ । दे॒वस्य॑ । त्वा॒ । स॒वि॒तुः । प्र॒स॒व इति॑ प्र - स॒वे । अ॒श्विनोः᳚ । बा॒हुभ्या॒मिति॑ बा॒हु - भ्या॒म् । पू॒ष्णः । हस्ता᳚भ्याम् । प्रतीति॑ । गृ॒ह्णा॒मि॒ । इति॑ । अ॒ब्र॒वी॒त् । स॒वि॒तृप्र॑सूत॒ इति॑ सवि॒तृ - प्र॒सू॒तः॒ । ए॒व । ए॒न॒त् । ब्रह्म॑णा । दे॒वता॑भिः । प्रतीति॑ । अ॒गृ॒ह्णा॒त् । सः । अ॒बि॒भे॒त् । प्रा॒श्नन्त॒मिति॑ प्र - अ॒श्नन्त᳚म् । मा॒ । हिꣳ॒॒सि॒ष्य॒ति॒ । इति॑ । अ॒ग्नेः । त्वा॒ । आ॒स्ये॑न । प्रेति॑ । अ॒श्ना॒मि॒ । इति॑ । अ॒ब्र॒वी॒त् । न । हि । अ॒ग्नेः । आ॒स्य᳚म् । किम् । च॒न । हि॒नस्ति॑ । सः । अ॒बि॒भे॒त् ।  \newline


\textbf{Krama Paata} \newline

किम् च॒न । च॒न हि॒नस्ति॑ । हि॒नस्ति॒ सः । सो॑ऽबिभेत् । अ॒बि॒भे॒त् प्र॒ति॒गृ॒ह्णन्त᳚म् । प्र॒ति॒गृ॒ह्णन्त॑म् मा । प्र॒ति॒गृ॒ह्णन्त॒मिति॑ प्रति - गृ॒ह्णन्त᳚म् । मा॒ हिꣳ॒॒सि॒ष्य॒ति॒ । हिꣳ॒॒सि॒ष्य॒तीति॑ । इति॑ दे॒वस्य॑ । दे॒वस्य॑ त्वा । त्वा॒ स॒वि॒तुः । स॒वि॒तुः प्र॑स॒वे । प्र॒स॒वे᳚ ऽश्विनोः᳚ । प्र॒स॒व इति॑ प्र - स॒वे । अ॒श्विनो᳚र् बा॒हुभ्या᳚म् । बा॒हुभ्या᳚म् पू॒ष्णः । बा॒हुभ्या॒मिति॑ बा॒हु - भ्या॒म् । पू॒ष्णो हस्ता᳚भ्याम् । हस्ता᳚भ्या॒म् प्रति॑ । प्रति॑ गृह्णामि । गृ॒ह्णा॒मीति॑ । इत्य॑ब्रवीत् । अ॒ब्र॒वी॒थ् स॒वि॒तृप्र॑सूतः । स॒वि॒तृप्र॑सूत ए॒व । स॒वि॒तृप्र॑सूत॒ इति॑ सवि॒तृ - प्र॒सू॒तः॒ । ए॒वैन॑त् । ए॒न॒द् ब्रह्म॑णा । ब्रह्म॑णा दे॒वता॑भिः । दे॒वता॑भिः॒ प्रति॑ । प्रत्य॑गृह्णात् । अ॒गृ॒ह्णा॒थ् सः । सो॑ऽबिभेत् । अ॒बि॒भे॒त् प्रा॒श्ञन्त᳚म् । प्रा॒श्ञन्त॑म् मा । प्रा॒श्ञन्त॒मिति॑ प्र - अ॒श्ञन्त᳚म् । मा॒ हिꣳ॒॒सि॒ष्य॒ति॒ । हिꣳ॒॒सि॒ष्य॒तीति॑ । इत्य॒ग्नेः । अ॒ग्नेस्त्वा᳚ । त्वा॒ ऽऽस्ये॑न । आ॒स्ये॑न॒ प्र । प्राश्ञा॑मि । अ॒श्ञा॒मीति॑ । इत्य॑ब्रवीत् । अ॒ब्र॒वी॒न् न । न हि । ह्य॑ग्नेः । अ॒ग्नेरा॒स्य᳚म् । आ॒स्य॑म् किम् । किम् च॒न । च॒न हि॒नस्ति॑ । हि॒नस्ति॒ सः । सो॑ऽबिभेत् ( ) । अ॒बि॒भे॒त् प्राशि॑तम् \newline

\textbf{Jatai Paata} \newline

1. किम् च॒न च॒न किम् किम् च॒न । \newline
2. च॒न हि॒नस्ति॑ हि॒नस्ति॑ च॒न च॒न हि॒नस्ति॑ । \newline
3. हि॒नस्ति॒ स स हि॒नस्ति॑ हि॒नस्ति॒ सः । \newline
4. सो॑ ऽबिभे दबिभे॒थ् स सो॑ ऽबिभेत् । \newline
5. अ॒बि॒भे॒त् प्र॒ति॒गृ॒ह्णन्त॑म् प्रतिगृ॒ह्णन्त॑ मबिभे दबिभेत् प्रतिगृ॒ह्णन्त᳚म् । \newline
6. प्र॒ति॒गृ॒ह्णन्त॑म् मा मा प्रतिगृ॒ह्णन्त॑म् प्रतिगृ॒ह्णन्त॑म् मा । \newline
7. प्र॒ति॒गृ॒ह्णन्त॒मिति॑ प्रति - गृ॒ह्णन्त᳚म् । \newline
8. मा॒ हिꣳ॒॒सि॒ष्य॒ति॒ हिꣳ॒॒सि॒ष्य॒ति॒ मा॒ मा॒ हिꣳ॒॒सि॒ष्य॒ति॒ । \newline
9. हिꣳ॒॒सि॒ष्य॒तीतीति॑ हिꣳसिष्यति हिꣳसिष्य॒तीति॑ । \newline
10. इति॑ दे॒वस्य॑ दे॒वस्ये तीति॑ दे॒वस्य॑ । \newline
11. दे॒वस्य॑ त्वा त्वा दे॒वस्य॑ दे॒वस्य॑ त्वा । \newline
12. त्वा॒ स॒वि॒तुः स॑वि॒तु स्त्वा᳚ त्वा सवि॒तुः । \newline
13. स॒वि॒तुः प्र॑स॒वे प्र॑स॒वे स॑वि॒तुः स॑वि॒तुः प्र॑स॒वे । \newline
14. प्र॒स॒वे᳚ ऽश्विनो॑ र॒श्विनोः᳚ प्रस॒वे प्र॑स॒वे᳚ ऽश्विनोः᳚ । \newline
15. प्र॒स॒व इति॑ प्र - स॒वे । \newline
16. अ॒श्विनो᳚र् बा॒हुभ्या᳚म् बा॒हुभ्या॑ म॒श्विनो॑ र॒श्विनो᳚र् बा॒हुभ्या᳚म् । \newline
17. बा॒हुभ्या᳚म् पू॒ष्णः पू॒ष्णो बा॒हुभ्या᳚म् बा॒हुभ्या᳚म् पू॒ष्णः । \newline
18. बा॒हुभ्या॒मिति॑ बा॒हु - भ्या॒म् । \newline
19. पू॒ष्णो हस्ता᳚भ्याꣳ॒॒ हस्ता᳚भ्याम् पू॒ष्णः पू॒ष्णो हस्ता᳚भ्याम् । \newline
20. हस्ता᳚भ्या॒म् प्रति॒ प्रति॒ हस्ता᳚भ्याꣳ॒॒ हस्ता᳚भ्या॒म् प्रति॑ । \newline
21. प्रति॑ गृह्णामि गृह्णामि॒ प्रति॒ प्रति॑ गृह्णामि । \newline
22. गृ॒ह्णा॒मीतीति॑ गृह्णामि गृह्णा॒मीति॑ । \newline
23. इत्य॑ब्रवी दब्रवी॒ दिती त्य॑ब्रवीत् । \newline
24. अ॒ब्र॒वी॒थ् स॒वि॒तृप्र॑सूतः सवि॒तृप्र॑सूतो ऽब्रवी दब्रवीथ् सवि॒तृप्र॑सूतः । \newline
25. स॒वि॒तृप्र॑सूत ए॒वैव स॑वि॒तृप्र॑सूतः सवि॒तृप्र॑सूत ए॒व । \newline
26. स॒वि॒तृप्र॑सूत॒ इति॑ सवि॒तृ - प्र॒सू॒तः॒ । \newline
27. ए॒वैन॑ देन दे॒वैवैन॑त् । \newline
28. ए॒न॒द् ब्रह्म॑णा॒ ब्रह्म॑णैन देन॒द् ब्रह्म॑णा । \newline
29. ब्रह्म॑णा दे॒वता॑भिर् दे॒वता॑भि॒र् ब्रह्म॑णा॒ ब्रह्म॑णा दे॒वता॑भिः । \newline
30. दे॒वता॑भिः॒ प्रति॒ प्रति॑ दे॒वता॑भिर् दे॒वता॑भिः॒ प्रति॑ । \newline
31. प्रत्य॑गृह्णा दगृह्णा॒त् प्रति॒ प्रत्य॑गृह्णात् । \newline
32. अ॒गृ॒ह्णा॒थ् स सो॑ ऽगृह्णा दगृह्णा॒थ् सः । \newline
33. सो॑ ऽबिभे दबिभे॒थ् स सो॑ ऽबिभेत् । \newline
34. अ॒बि॒भे॒त् प्रा॒श्ञन्त॑म् प्रा॒श्ञन्त॑ मबिभे दबिभेत् प्रा॒श्ञन्त᳚म् । \newline
35. प्रा॒श्ञन्त॑म् मा मा प्रा॒श्ञन्त॑म् प्रा॒श्ञन्त॑म् मा । \newline
36. प्रा॒श्ञन्त॒मिति॑ प्र - अ॒श्ञन्त᳚म् । \newline
37. मा॒ हिꣳ॒॒सि॒ष्य॒ति॒ हिꣳ॒॒सि॒ष्य॒ति॒ मा॒ मा॒ हिꣳ॒॒सि॒ष्य॒ति॒ । \newline
38. हिꣳ॒॒सि॒ष्य॒तीतीति॑ हिꣳसिष्यति हिꣳसिष्य॒तीति॑ । \newline
39. इत्य॒ग्ने र॒ग्ने रिती त्य॒ग्नेः । \newline
40. अ॒ग्ने स्त्वा᳚ त्वा॒ ऽग्ने र॒ग्ने स्त्वा᳚ । \newline
41. त्वा॒ ऽऽस्ये॑ना॒स्ये॑न त्वा त्वा॒ ऽऽस्ये॑न । \newline
42. आ॒स्ये॑न॒ प्र प्रास्ये॑ना॒ स्ये॑न॒ प्र । \newline
43. प्राश्ञा᳚ म्यश्ञामि॒ प्र प्राश्ञा॑मि । \newline
44. अ॒श्ञा॒मीती त्य॑श्ञा म्यश्ञा॒मीति॑ । \newline
45. इत्य॑ब्रवी दब्रवी॒ दिती त्य॑ब्रवीत् । \newline
46. अ॒ब्र॒वी॒न् न नाब्र॑वी दब्रवी॒न् न । \newline
47. न हि हि न न हि । \newline
48. ह्य॑ग्ने र॒ग्नेर्. हि ह्य॑ग्नेः । \newline
49. अ॒ग्ने रा॒स्य॑ मा॒स्य॑ म॒ग्ने र॒ग्ने रा॒स्य᳚म् । \newline
50. आ॒स्य॑म् किम् कि मा॒स्य॑ मा॒स्य॑म् किम् । \newline
51. किम् च॒न च॒न किम् किम् च॒न । \newline
52. च॒न हि॒नस्ति॑ हि॒नस्ति॑ च॒न च॒न हि॒नस्ति॑ । \newline
53. हि॒नस्ति॒ स स हि॒नस्ति॑ हि॒नस्ति॒ सः । \newline
54. सो॑ ऽबिभे दबिभे॒थ् स सो॑ ऽबिभेत् । \newline
55. अ॒बि॒भे॒त् प्राशि॑त॒म् प्राशि॑त मबिभे दबिभे॒त् प्राशि॑तम् । \newline

\textbf{Ghana Paata } \newline

1. किम् च॒न च॒न किम् किम् च॒न हि॒नस्ति॑ हि॒नस्ति॑ च॒न किम् किम् च॒न हि॒नस्ति॑ । \newline
2. च॒न हि॒नस्ति॑ हि॒नस्ति॑ च॒न च॒न हि॒नस्ति॒ स स हि॒नस्ति॑ च॒न च॒न हि॒नस्ति॒ सः । \newline
3. हि॒नस्ति॒ स स हि॒नस्ति॑ हि॒नस्ति॒ सो॑ ऽबिभे दबिभे॒थ् स हि॒नस्ति॑ हि॒नस्ति॒ सो॑ ऽबिभेत् । \newline
4. सो॑ ऽबिभे दबिभे॒थ् स सो॑ ऽबिभेत् प्रतिगृ॒ह्णन्त॑म् प्रतिगृ॒ह्णन्त॑ मबिभे॒थ् स सो॑ ऽबिभेत् प्रतिगृ॒ह्णन्त᳚म् । \newline
5. अ॒बि॒भे॒त् प्र॒ति॒गृ॒ह्णन्त॑म् प्रतिगृ॒ह्णन्त॑ मबिभे दबिभेत् प्रतिगृ॒ह्णन्त॑म् मा मा प्रतिगृ॒ह्णन्त॑ मबिभे दबिभेत् प्रतिगृ॒ह्णन्त॑म् मा । \newline
6. प्र॒ति॒गृ॒ह्णन्त॑म् मा मा प्रतिगृ॒ह्णन्त॑म् प्रतिगृ॒ह्णन्त॑म् मा हिꣳसिष्यति हिꣳसिष्यति मा प्रतिगृ॒ह्णन्त॑म् प्रतिगृ॒ह्णन्त॑म् मा हिꣳसिष्यति । \newline
7. प्र॒ति॒गृ॒ह्णन्त॒मिति॑ प्रति - गृ॒ह्णन्त᳚म् । \newline
8. मा॒ हिꣳ॒॒सि॒ष्य॒ति॒ हिꣳ॒॒सि॒ष्य॒ति॒ मा॒ मा॒ हिꣳ॒॒सि॒ष्य॒तीतीति॑ हिꣳसिष्यति मा मा हिꣳसिष्य॒तीति॑ । \newline
9. हिꣳ॒॒सि॒ष्य॒तीतीति॑ हिꣳसिष्यति हिꣳसिष्य॒तीति॑ दे॒वस्य॑ दे॒वस्ये ति॑ हिꣳसिष्यति हिꣳसिष्य॒तीति॑ दे॒वस्य॑ । \newline
10. इति॑ दे॒वस्य॑ दे॒वस्ये तीति॑ दे॒वस्य॑ त्वा त्वा दे॒वस्ये तीति॑ दे॒वस्य॑ त्वा । \newline
11. दे॒वस्य॑ त्वा त्वा दे॒वस्य॑ दे॒वस्य॑ त्वा सवि॒तुः स॑वि॒तु स्त्वा॑ दे॒वस्य॑ दे॒वस्य॑ त्वा सवि॒तुः । \newline
12. त्वा॒ स॒वि॒तुः स॑वि॒तु स्त्वा᳚ त्वा सवि॒तुः प्र॑स॒वे प्र॑स॒वे स॑वि॒तु स्त्वा᳚ त्वा सवि॒तुः प्र॑स॒वे । \newline
13. स॒वि॒तुः प्र॑स॒वे प्र॑स॒वे स॑वि॒तुः स॑वि॒तुः प्र॑स॒वे᳚ ऽश्विनो॑ र॒श्विनोः᳚ प्रस॒वे स॑वि॒तुः स॑वि॒तुः प्र॑स॒वे᳚ ऽश्विनोः᳚ । \newline
14. प्र॒स॒वे᳚ ऽश्विनो॑ र॒श्विनोः᳚ प्रस॒वे प्र॑स॒वे᳚ ऽश्विनो᳚र् बा॒हुभ्या᳚म् बा॒हुभ्या॑ म॒श्विनोः᳚ प्रस॒वे प्र॑स॒वे᳚ ऽश्विनो᳚र् बा॒हुभ्या᳚म् । \newline
15. प्र॒स॒व इति॑ प्र - स॒वे । \newline
16. अ॒श्विनो᳚र् बा॒हुभ्या᳚म् बा॒हुभ्या॑ म॒श्विनो॑ र॒श्विनो᳚र् बा॒हुभ्या᳚म् पू॒ष्णः पू॒ष्णो बा॒हुभ्या॑ म॒श्विनो॑ र॒श्विनो᳚र् बा॒हुभ्या᳚म् पू॒ष्णः । \newline
17. बा॒हुभ्या᳚म् पू॒ष्णः पू॒ष्णो बा॒हुभ्या᳚म् बा॒हुभ्या᳚म् पू॒ष्णो हस्ता᳚भ्याꣳ॒॒ हस्ता᳚भ्याम् पू॒ष्णो बा॒हुभ्या᳚म् बा॒हुभ्या᳚म् पू॒ष्णो हस्ता᳚भ्याम् । \newline
18. बा॒हुभ्या॒मिति॑ बा॒हु - भ्या॒म् । \newline
19. पू॒ष्णो हस्ता᳚भ्याꣳ॒॒ हस्ता᳚भ्याम् पू॒ष्णः पू॒ष्णो हस्ता᳚भ्या॒म् प्रति॒ प्रति॒ हस्ता᳚भ्याम् पू॒ष्णः पू॒ष्णो हस्ता᳚भ्या॒म् प्रति॑ । \newline
20. हस्ता᳚भ्या॒म् प्रति॒ प्रति॒ हस्ता᳚भ्याꣳ॒॒ हस्ता᳚भ्या॒म् प्रति॑ गृह्णामि गृह्णामि॒ प्रति॒ हस्ता᳚भ्याꣳ॒॒ हस्ता᳚भ्या॒म् प्रति॑ गृह्णामि । \newline
21. प्रति॑ गृह्णामि गृह्णामि॒ प्रति॒ प्रति॑ गृह्णा॒मीतीति॑ गृह्णामि॒ प्रति॒ प्रति॑ गृह्णा॒मीति॑ । \newline
22. गृ॒ह्णा॒मीतीति॑ गृह्णामि गृह्णा॒मी त्य॑ब्रवी दब्रवी॒ दिति॑ गृह्णामि गृह्णा॒मी त्य॑ब्रवीत् । \newline
23. इत्य॑ब्रवी दब्रवी॒ दिती त्य॑ब्रवीथ् सवि॒तृप्र॑सूतः सवि॒तृप्र॑सूतो ऽब्रवी॒दिती त्य॑ब्रवीथ् सवि॒तृप्र॑सूतः । \newline
24. अ॒ब्र॒वी॒थ् स॒वि॒तृप्र॑सूतः सवि॒तृप्र॑सूतो ऽब्रवी दब्रवीथ् सवि॒तृप्र॑सूत ए॒वैव स॑वि॒तृप्र॑सूतो ऽब्रवी दब्रवीथ् सवि॒तृप्र॑सूत ए॒व । \newline
25. स॒वि॒तृप्र॑सूत ए॒वैव स॑वि॒तृप्र॑सूतः सवि॒तृप्र॑सूत ए॒वैन॑ देन दे॒व स॑वि॒तृप्र॑सूतः सवि॒तृप्र॑सूत ए॒वैन॑त् । \newline
26. स॒वि॒तृप्र॑सूत॒ इति॑ सवि॒तृ - प्र॒सू॒तः॒ । \newline
27. ए॒वैन॑ देन दे॒वैवैन॒द् ब्रह्म॑णा॒ ब्रह्म॑णैन दे॒वैवैन॒द् ब्रह्म॑णा । \newline
28. ए॒न॒द् ब्रह्म॑णा॒ ब्रह्म॑णैन देन॒द् ब्रह्म॑णा दे॒वता॑भिर् दे॒वता॑भि॒र् ब्रह्म॑णैन देन॒द् ब्रह्म॑णा दे॒वता॑भिः । \newline
29. ब्रह्म॑णा दे॒वता॑भिर् दे॒वता॑भि॒र् ब्रह्म॑णा॒ ब्रह्म॑णा दे॒वता॑भिः॒ प्रति॒ प्रति॑ दे॒वता॑भि॒र् ब्रह्म॑णा॒ ब्रह्म॑णा दे॒वता॑भिः॒ प्रति॑ । \newline
30. दे॒वता॑भिः॒ प्रति॒ प्रति॑ दे॒वता॑भिर् दे॒वता॑भिः॒ प्रत्य॑गृह्णा दगृह्णा॒त् प्रति॑ दे॒वता॑भिर् दे॒वता॑भिः॒ प्रत्य॑गृह्णात् । \newline
31. प्रत्य॑गृह्णा दगृह्णा॒त् प्रति॒ प्रत्य॑गृह्णा॒थ् स सो॑ ऽगृह्णा॒त् प्रति॒ प्रत्य॑गृह्णा॒थ् सः । \newline
32. अ॒गृ॒ह्णा॒थ् स सो॑ ऽगृह्णा दगृह्णा॒थ् सो॑ ऽबिभे दबिभे॒थ् सो॑ ऽगृह्णा दगृह्णा॒थ् सो॑ ऽबिभेत् । \newline
33. सो॑ ऽबिभे दबिभे॒थ् स सो॑ ऽबिभेत् प्रा॒श्ञन्त॑म् प्रा॒श्ञन्त॑ मबिभे॒थ् स सो॑ ऽबिभेत् प्रा॒श्ञन्त᳚म् । \newline
34. अ॒बि॒भे॒त् प्रा॒श्ञन्त॑म् प्रा॒श्ञन्त॑ मबिभे दबिभेत् प्रा॒श्ञन्त॑म् मा मा प्रा॒श्ञन्त॑ मबिभे दबिभेत् प्रा॒श्ञन्त॑म् मा । \newline
35. प्रा॒श्ञन्त॑म् मा मा प्रा॒श्ञन्त॑म् प्रा॒श्ञन्त॑म् मा हिꣳसिष्यति हिꣳसिष्यति मा प्रा॒श्ञन्त॑म् प्रा॒श्ञन्त॑म् मा हिꣳसिष्यति । \newline
36. प्रा॒श्ञन्त॒मिति॑ प्र - अ॒श्ञन्त᳚म् । \newline
37. मा॒ हिꣳ॒॒सि॒ष्य॒ति॒ हिꣳ॒॒सि॒ष्य॒ति॒ मा॒ मा॒ हिꣳ॒॒सि॒ष्य॒तीतीति॑ हिꣳसिष्यति मा मा हिꣳसिष्य॒तीति॑ । \newline
38. हिꣳ॒॒सि॒ष्य॒तीतीति॑ हिꣳसिष्यति हिꣳसिष्य॒ती त्य॒ग्ने र॒ग्नेरिति॑ हिꣳसिष्यति हिꣳसिष्य॒ती त्य॒ग्नेः । \newline
39. इत्य॒ग्ने र॒ग्ने रिती त्य॒ग्ने स्त्वा᳚ त्वा॒ ऽग्नेरिती त्य॒ग्ने स्त्वा᳚ । \newline
40. अ॒ग्ने स्त्वा᳚ त्वा॒ ऽग्ने र॒ग्ने स्त्वा॒ ऽऽस्ये॑ना॒स्ये॑न त्वा॒ ऽग्ने र॒ग्ने स्त्वा॒ ऽऽस्ये॑न । \newline
41. त्वा॒ ऽऽस्ये॑ना॒ स्ये॑न त्वा त्वा॒ ऽऽस्ये॑न॒ प्र प्रास्ये॑न त्वा त्वा॒ ऽऽस्ये॑न॒ प्र । \newline
42. आ॒स्ये॑न॒ प्र प्रास्ये॑ना॒ स्ये॑न॒ प्राश्ञा᳚ म्यश्ञामि॒ प्रास्ये॑ना॒ स्ये॑न॒ प्राश्ञा॑मि । \newline
43. प्राश्ञा᳚ म्यश्ञामि॒ प्र प्राश्ञा॒मीती त्य॑श्ञामि॒ प्र प्राश्ञा॒मीति॑ । \newline
44. अ॒श्ञा॒मीती त्य॑श्ञा म्यश्ञा॒मी त्य॑ब्रवी दब्रवी॒ दित्य॑श्ञा म्यश्ञा॒मी त्य॑ब्रवीत् । \newline
45. इत्य॑ब्रवी दब्रवी॒ दिती त्य॑ब्रवी॒न् न नाब्र॑वी॒दिती त्य॑ब्रवी॒न् न । \newline
46. अ॒ब्र॒वी॒न् न नाब्र॑वी दब्रवी॒न् न हि हि नाब्र॑वी दब्रवी॒न् न हि । \newline
47. न हि हि न न ह्य॑ग्ने र॒ग्नेर्. हि न न ह्य॑ग्नेः । \newline
48. ह्य॑ग्ने र॒ग्नेर्. हि ह्य॑ग्ने रा॒स्य॑ मा॒स्य॑ म॒ग्नेर्. हि ह्य॑ग्ने रा॒स्य᳚म् । \newline
49. अ॒ग्ने रा॒स्य॑ मा॒स्य॑ म॒ग्ने र॒ग्ने रा॒स्य॑म् किम् कि मा॒स्य॑ म॒ग्ने र॒ग्ने रा॒स्य॑म् किम् । \newline
50. आ॒स्य॑म् किम् कि मा॒स्य॑ मा॒स्य॑म् किम् च॒न च॒न कि मा॒स्य॑ मा॒स्य॑म् किम् च॒न । \newline
51. किम् च॒न च॒न किम् किम् च॒न हि॒नस्ति॑ हि॒नस्ति॑ च॒न किम् किम् च॒न हि॒नस्ति॑ । \newline
52. च॒न हि॒नस्ति॑ हि॒नस्ति॑ च॒न च॒न हि॒नस्ति॒ स स हि॒नस्ति॑ च॒न च॒न हि॒नस्ति॒ सः । \newline
53. हि॒नस्ति॒ स स हि॒नस्ति॑ हि॒नस्ति॒ सो॑ ऽबिभे दबिभे॒थ् स हि॒नस्ति॑ हि॒नस्ति॒ सो॑ ऽबिभेत् । \newline
54. सो॑ ऽबिभे दबिभे॒थ् स सो॑ ऽबिभे॒त् प्राशि॑त॒म् प्राशि॑त मबिभे॒थ् स सो॑ ऽबिभे॒त् प्राशि॑तम् । \newline
55. अ॒बि॒भे॒त् प्राशि॑त॒म् प्राशि॑त मबिभे दबिभे॒त् प्राशि॑तम् मा मा॒ प्राशि॑त मबिभे दबिभे॒त् प्राशि॑तम् मा । \newline
\pagebreak
\markright{ TS 2.6.8.7  \hfill https://www.vedavms.in \hfill}
\addcontentsline{toc}{section}{ TS 2.6.8.7 }
\section*{ TS 2.6.8.7 }

\textbf{TS 2.6.8.7 } \newline
\textbf{Samhita Paata} \newline

प्राशि॑तं माहिꣳसिष्य॒तीति॑ ब्राह्म॒णस्यो॒दरे॒णेत्य॑ ब्रवी॒न्न हि ब्रा᳚ह्म॒णस्यो॒दरं॒ किं च॒न हि॒नस्ति॒ बृह॒स्पते॒र्ब्रह्म॒णेति॒ स हि ब्रह्मि॒ष्ठोऽप॒ वा ए॒तस्मा᳚त् प्रा॒णाः क्रा॑मन्ति॒ यः प्रा॑शि॒त्रं प्रा॒श्नात्य॒द्भि-र्मा᳚र्जयि॒त्वा प्रा॒णान्थ् सं मृ॑शते॒ऽमृतं॒ ॅवै प्रा॒णा अ॒मृत॒मापः॑ प्रा॒णाने॒व य॑थास्था॒नमुप॑ ह्वयते ॥ \newline

\textbf{Pada Paata} \newline

प्राशि॑त॒मिति॒ प्र-अ॒शि॒त॒म् । मा॒ । हिꣳ॒॒सि॒ष्य॒ति॒ । इति॑ । ब्रा॒ह्म॒णस्य॑ । उ॒दरे॑ण । इति॑ । अ॒ब्र॒वी॒त् । न । हि । ब्रा॒ह्म॒णस्य॑ । उ॒दर᳚म् । किम् । च॒न । हि॒नस्ति॑ । बृह॒स्पतेः᳚ । ब्रह्म॑णा । इति॑ । सः । हि । ब्रह्मि॑ष्ठः । अपेति॑ । वै । ए॒तस्मा᳚त् । प्रा॒णा इति॑ प्र - अ॒नाः । क्रा॒म॒न्ति॒ । यः । प्रा॒शि॒त्रमिति॑ प्र - अ॒शि॒त्रम् । प्रा॒श्नातीति॑ प्र - अ॒श्नाति॑ । अ॒द्भिरित्य॑त् - भिः । मा॒र्ज॒यि॒त्वा । प्रा॒णानिति॑ प्र-अ॒नान् । समिति॑ । मृ॒श॒ते॒ । अ॒मृत᳚म् । वै । प्रा॒णा इति॑ प्र - अ॒नाः । अ॒मृत᳚म् । आपः॑ । प्रा॒णानिति॑ प्र - अ॒नान् । ए॒व । य॒था॒स्था॒नमिति॑ यथा - स्था॒नम् । उपेति॑ । ह्व॒य॒ते॒ ॥  \newline


\textbf{Krama Paata} \newline

प्राशि॑तम् मा । प्राशि॑त॒मिति॒ प्र - अ॒शि॒त॒म् । मा॒ हिꣳ॒॒सि॒ष्य॒ति॒ । हिꣳ॒॒सि॒ष्य॒तीति॑ । इति॑ ब्राह्म॒णस्य॑ । ब्रा॒ह्म॒णस्यो॒दरे॑ण । उ॒दरे॒णेति॑ । इत्य॑ब्रवीत् । अ॒ब्र॒वी॒न् न । न हि । हि ब्रा᳚ह्म॒णस्य॑ । ब्रा॒ह्म॒णस्यो॒दर᳚म् । उ॒दर॒म् किम् । किम् च॒न । च॒न हि॒नस्ति॑ । हि॒नस्ति॒ बृह॒स्पतेः᳚ । बृह॒स्पते॒र् ब्रह्म॑णा । ब्रह्म॒णेति॑ । इति॒ सः । स हि । हि ब्रह्मि॑ष्ठः । ब्रह्मि॒ष्ठो ऽप॑ । अप॒ वै । वा ए॒तस्मा᳚त् । ए॒तस्मा᳚त् प्रा॒णाः । प्रा॒णाः क्रा॑मन्ति । प्रा॒णा इति॑ प्र - अ॒नाः । क्रा॒म॒न्ति॒ यः । यः प्रा॑शि॒त्रम् । प्रा॒शि॒त्रम् प्रा॒श्ञाति॑ । प्रा॒शि॒त्रमिति॑ प्र - अ॒शि॒त्रम् । प्रा॒श्ञात्य॒द्भिः । प्रा॒श्ञातीति॑ प्र - अ॒श्ञाति॑ । अ॒द्भिर् मा᳚र्जयि॒त्वा । अ॒द्भिरित्य॑त् - भिः । मा॒र्ज॒यि॒त्वा प्रा॒णान् । प्रा॒णान्थ् सम् । प्रा॒णानिति॑ प्र - अ॒नान् । सं मृ॑शते । मृ॒श॒ते॒ऽमृत᳚म् । अ॒मृतं॒ ॅवै । वै प्रा॒णाः । प्रा॒णा अ॒मृत᳚म् । प्रा॒णा इति॑ प्र - अ॒नाः । अ॒मृत॒मापः॑ । आपः॑ प्रा॒णान् । प्रा॒णाने॒व । प्रा॒णानिति॑ प्र - अ॒नान् । ए॒व य॑थास्था॒नम् । य॒था॒स्था॒नमुप॑ । य॒था॒स्था॒नमिति॑ यथा - स्था॒नम् । उप॑ ह्वयते । ह्व॒य॒त॒ इति॑ ह्वयते । \newline

\textbf{Jatai Paata} \newline

1. प्राशि॑तम् मा मा॒ प्राशि॑त॒म् प्राशि॑तम् मा । \newline
2. प्राशि॑त॒मिति॒ प्र - अ॒शि॒त॒म् । \newline
3. मा॒ हिꣳ॒॒सि॒ष्य॒ति॒ हिꣳ॒॒सि॒ष्य॒ति॒ मा॒ मा॒ हिꣳ॒॒सि॒ष्य॒ति॒ । \newline
4. हिꣳ॒॒सि॒ष्य॒तीतीति॑ हिꣳसिष्यति हिꣳसिष्य॒तीति॑ । \newline
5. इति॑ ब्राह्म॒णस्य॑ ब्राह्म॒णस्ये तीति॑ ब्राह्म॒णस्य॑ । \newline
6. ब्रा॒ह्म॒ण स्यो॒दरे॑ णो॒दरे॑ण ब्राह्म॒णस्य॑ ब्राह्म॒ण स्यो॒दरे॑ण । \newline
7. उ॒दरे॒णे तीत्यु॒दरे॑ णो॒दरे॒णे ति॑ । \newline
8. इत्य॑ब्रवी दब्रवी॒ दिती त्य॑ब्रवीत् । \newline
9. अ॒ब्र॒वी॒न् न नाब्र॑वी दब्रवी॒न् न । \newline
10. न हि हि न न हि । \newline
11. हि ब्रा᳚ह्म॒णस्य॑ ब्राह्म॒णस्य॒ हि हि ब्रा᳚ह्म॒णस्य॑ । \newline
12. ब्रा॒ह्म॒ण स्यो॒दर॑ मु॒दर॑म् ब्राह्म॒णस्य॑ ब्राह्म॒ण स्यो॒दर᳚म् । \newline
13. उ॒दर॒म् किम् कि मु॒दर॑ मु॒दर॒म् किम् । \newline
14. किम् च॒न च॒न किम् किम् च॒न । \newline
15. च॒न हि॒नस्ति॑ हि॒नस्ति॑ च॒न च॒न हि॒नस्ति॑ । \newline
16. हि॒नस्ति॒ बृह॒स्पते॒र् बृह॒स्पतेर्॑. हि॒नस्ति॑ हि॒नस्ति॒ बृह॒स्पतेः᳚ । \newline
17. बृह॒स्पते॒र् ब्रह्म॑णा॒ ब्रह्म॑णा॒ बृह॒स्पते॒र् बृह॒स्पते॒र् ब्रह्म॑णा । \newline
18. ब्रह्म॒णेतीति॒ ब्रह्म॑णा॒ ब्रह्म॒णेति॑ । \newline
19. इति॒ स स इतीति॒ सः । \newline
20. स हि हि स स हि । \newline
21. हि ब्रह्मि॑ष्ठो॒ ब्रह्मि॑ष्ठो॒ हि हि ब्रह्मि॑ष्ठः । \newline
22. ब्रह्मि॒ष्ठो ऽपाप॒ ब्रह्मि॑ष्ठो॒ ब्रह्मि॒ष्ठो ऽप॑ । \newline
23. अप॒ वै वा अपाप॒ वै । \newline
24. वा ए॒तस्मा॑ दे॒तस्मा॒द् वै वा ए॒तस्मा᳚त् । \newline
25. ए॒तस्मा᳚त् प्रा॒णाः प्रा॒णा ए॒तस्मा॑ दे॒तस्मा᳚त् प्रा॒णाः । \newline
26. प्रा॒णाः क्रा॑मन्ति क्रामन्ति प्रा॒णाः प्रा॒णाः क्रा॑मन्ति । \newline
27. प्रा॒णा इति॑ प्र - अ॒नाः । \newline
28. क्रा॒म॒न्ति॒ यो यः क्रा॑मन्ति क्रामन्ति॒ यः । \newline
29. यः प्रा॑शि॒त्रम् प्रा॑शि॒त्रं ॅयो यः प्रा॑शि॒त्रम् । \newline
30. प्रा॒शि॒त्रम् प्रा॒श्ञाति॑ प्रा॒श्ञाति॑ प्राशि॒त्रम् प्रा॑शि॒त्रम् प्रा॒श्ञाति॑ । \newline
31. प्रा॒शि॒त्रमिति॑ प्र - अ॒शि॒त्रम् । \newline
32. प्रा॒श्ञा त्य॒द्भि र॒द्भिः प्रा॒श्ञाति॑ प्रा॒श्ञा त्य॒द्भिः । \newline
33. प्रा॒श्ञातीति॑ प्र - अ॒श्ञाति॑ । \newline
34. अ॒द्भिर् मा᳚र्जयि॒त्वा मा᳚र्जयि॒त्वा ऽद्भि र॒द्भिर् मा᳚र्जयि॒त्वा । \newline
35. अ॒द्भिरित्य॑त् - भिः । \newline
36. मा॒र्ज॒यि॒त्वा प्रा॒णान् प्रा॒णान् मा᳚र्जयि॒त्वा मा᳚र्जयि॒त्वा प्रा॒णान् । \newline
37. प्रा॒णान् थ्सꣳ सम् प्रा॒णान् प्रा॒णान् थ्सम् । \newline
38. प्रा॒णानिति॑ प्र - अ॒नान् । \newline
39. सम् मृ॑शते मृशते॒ सꣳ सम् मृ॑शते । \newline
40. मृ॒श॒ते॒ ऽमृत॑ म॒मृत॑म् मृशते मृशते॒ ऽमृत᳚म् । \newline
41. अ॒मृतं॒ ॅवै वा अ॒मृत॑ म॒मृतं॒ ॅवै । \newline
42. वै प्रा॒णाः प्रा॒णा वै वै प्रा॒णाः । \newline
43. प्रा॒णा अ॒मृत॑ म॒मृत॑म् प्रा॒णाः प्रा॒णा अ॒मृत᳚म् । \newline
44. प्रा॒णा इति॑ प्र - अ॒नाः । \newline
45. अ॒मृत॒ माप॒ आपो॒ ऽमृत॑ म॒मृत॒ मापः॑ । \newline
46. आपः॑ प्रा॒णान् प्रा॒णा नाप॒ आपः॑ प्रा॒णान् । \newline
47. प्रा॒णा ने॒वैव प्रा॒णान् प्रा॒णा ने॒व । \newline
48. प्रा॒णानिति॑ प्र - अ॒नान् । \newline
49. ए॒व य॑थास्था॒नं ॅय॑थास्था॒न मे॒वैव य॑थास्था॒नम् । \newline
50. य॒था॒स्था॒न मुपोप॑ यथास्था॒नं ॅय॑थास्था॒न मुप॑ । \newline
51. य॒था॒स्था॒नमिति॑ यथा - स्था॒नम् । \newline
52. उप॑ ह्वयते ह्वयत॒ उपोप॑ ह्वयते । \newline
53. ह्व॒य॒त॒ इति॑ ह्वयते । \newline

\textbf{Ghana Paata } \newline

1. प्राशि॑तम् मा मा॒ प्राशि॑त॒म् प्राशि॑तम् मा हिꣳसिष्यति हिꣳसिष्यति मा॒ प्राशि॑त॒म् प्राशि॑तम् मा हिꣳसिष्यति । \newline
2. प्राशि॑त॒मिति॒ प्र - अ॒शि॒त॒म् । \newline
3. मा॒ हिꣳ॒॒सि॒ष्य॒ति॒ हिꣳ॒॒सि॒ष्य॒ति॒ मा॒ मा॒ हिꣳ॒॒सि॒ष्य॒तीतीति॑ हिꣳसिष्यति मा मा हिꣳसिष्य॒तीति॑ । \newline
4. हिꣳ॒॒सि॒ष्य॒तीतीति॑ हिꣳसिष्यति हिꣳसिष्य॒तीति॑ ब्राह्म॒णस्य॑ ब्राह्म॒णस्ये ति॑ हिꣳसिष्यति हिꣳसिष्य॒तीति॑ ब्राह्म॒णस्य॑ । \newline
5. इति॑ ब्राह्म॒णस्य॑ ब्राह्म॒णस्ये तीति॑ ब्राह्म॒ण स्यो॒दरे॑णो॒ दरे॑ण ब्राह्म॒णस्ये तीति॑ ब्राह्म॒णस्यो॒दरे॑ण । \newline
6. ब्रा॒ह्म॒ण स्यो॒दरे॑णो॒ दरे॑ण ब्राह्म॒णस्य॑ ब्राह्म॒ण स्यो॒दरे॒णे तीत्यु॒दरे॑ण ब्राह्म॒णस्य॑ ब्राह्म॒ण स्यो॒दरे॒णे ति॑ । \newline
7. उ॒दरे॒णे तीत्यु॒दरे॑णो॒ दरे॒णे त्य॑ब्रवी दब्रवी॒ दित्यु॒दरे॑णो॒ दरे॒णे त्य॑ब्रवीत् । \newline
8. इत्य॑ब्रवी दब्रवी॒ दिती त्य॑ब्रवी॒न् न नाब्र॑वी॒दिती त्य॑ब्रवी॒न् न । \newline
9. अ॒ब्र॒वी॒न् न नाब्र॑वी दब्रवी॒न् न हि हि नाब्र॑वी दब्रवी॒न् न हि । \newline
10. न हि हि न न हि ब्रा᳚ह्म॒णस्य॑ ब्राह्म॒णस्य॒ हि न न हि ब्रा᳚ह्म॒णस्य॑ । \newline
11. हि ब्रा᳚ह्म॒णस्य॑ ब्राह्म॒णस्य॒ हि हि ब्रा᳚ह्म॒णस्यो॒दर॑ मु॒दर॑म् ब्राह्म॒णस्य॒ हि हि ब्रा᳚ह्म॒णस्यो॒दर᳚म् । \newline
12. ब्रा॒ह्म॒णस्यो॒दर॑ मु॒दर॑म् ब्राह्म॒णस्य॑ ब्राह्म॒णस्यो॒दर॒म् किम् कि मु॒दर॑म् ब्राह्म॒णस्य॑ ब्राह्म॒णस्यो॒दर॒म् किम् । \newline
13. उ॒दर॒म् किम् कि मु॒दर॑ मु॒दर॒म् किम् च॒न च॒न कि मु॒दर॑ मु॒दर॒म् किम् च॒न । \newline
14. किम् च॒न च॒न किम् किम् च॒न हि॒नस्ति॑ हि॒नस्ति॑ च॒न किम् किम् च॒न हि॒नस्ति॑ । \newline
15. च॒न हि॒नस्ति॑ हि॒नस्ति॑ च॒न च॒न हि॒नस्ति॒ बृह॒स्पते॒र् बृह॒स्पतेर्॑. हि॒नस्ति॑ च॒न च॒न हि॒नस्ति॒ बृह॒स्पतेः᳚ । \newline
16. हि॒नस्ति॒ बृह॒स्पते॒र् बृह॒स्पतेर्॑. हि॒नस्ति॑ हि॒नस्ति॒ बृह॒स्पते॒र् ब्रह्म॑णा॒ ब्रह्म॑णा॒ बृह॒स्पतेर्॑. हि॒नस्ति॑ हि॒नस्ति॒ बृह॒स्पते॒र् ब्रह्म॑णा । \newline
17. बृह॒स्पते॒र् ब्रह्म॑णा॒ ब्रह्म॑णा॒ बृह॒स्पते॒र् बृह॒स्पते॒र् ब्रह्म॒णेतीति॒ ब्रह्म॑णा॒ बृह॒स्पते॒र् बृह॒स्पते॒र् ब्रह्म॒णेति॑ । \newline
18. ब्रह्म॒णेतीति॒ ब्रह्म॑णा॒ ब्रह्म॒णेति॒ स स इति॒ ब्रह्म॑णा॒ ब्रह्म॒णेति॒ सः । \newline
19. इति॒ स स इतीति॒ स हि हि स इतीति॒ स हि । \newline
20. स हि हि स स हि ब्रह्मि॑ष्ठो॒ ब्रह्मि॑ष्ठो॒ हि स स हि ब्रह्मि॑ष्ठः । \newline
21. हि ब्रह्मि॑ष्ठो॒ ब्रह्मि॑ष्ठो॒ हि हि ब्रह्मि॒ष्ठो ऽपाप॒ ब्रह्मि॑ष्ठो॒ हि हि ब्रह्मि॒ष्ठो ऽप॑ । \newline
22. ब्रह्मि॒ष्ठो ऽपाप॒ ब्रह्मि॑ष्ठो॒ ब्रह्मि॒ष्ठो ऽप॒ वै वा अप॒ ब्रह्मि॑ष्ठो॒ ब्रह्मि॒ष्ठो ऽप॒ वै । \newline
23. अप॒ वै वा अपाप॒ वा ए॒तस्मा॑ दे॒तस्मा॒द् वा अपाप॒ वा ए॒तस्मा᳚त् । \newline
24. वा ए॒तस्मा॑ दे॒तस्मा॒द् वै वा ए॒तस्मा᳚त् प्रा॒णाः प्रा॒णा ए॒तस्मा॒द् वै वा ए॒तस्मा᳚त् प्रा॒णाः । \newline
25. ए॒तस्मा᳚त् प्रा॒णाः प्रा॒णा ए॒तस्मा॑ दे॒तस्मा᳚त् प्रा॒णाः क्रा॑मन्ति क्रामन्ति प्रा॒णा ए॒तस्मा॑ दे॒तस्मा᳚त् प्रा॒णाः क्रा॑मन्ति । \newline
26. प्रा॒णाः क्रा॑मन्ति क्रामन्ति प्रा॒णाः प्रा॒णाः क्रा॑मन्ति॒ यो यः क्रा॑मन्ति प्रा॒णाः प्रा॒णाः क्रा॑मन्ति॒ यः । \newline
27. प्रा॒णा इति॑ प्र - अ॒नाः । \newline
28. क्रा॒म॒न्ति॒ यो यः क्रा॑मन्ति क्रामन्ति॒ यः प्रा॑शि॒त्रम् प्रा॑शि॒त्रं ॅयः क्रा॑मन्ति क्रामन्ति॒ यः प्रा॑शि॒त्रम् । \newline
29. यः प्रा॑शि॒त्रम् प्रा॑शि॒त्रं ॅयो यः प्रा॑शि॒त्रम् प्रा॒श्ञाति॑ प्रा॒श्ञाति॑ प्राशि॒त्रं ॅयो यः प्रा॑शि॒त्रम् प्रा॒श्ञाति॑ । \newline
30. प्रा॒शि॒त्रम् प्रा॒श्ञाति॑ प्रा॒श्ञाति॑ प्राशि॒त्रम् प्रा॑शि॒त्रम् प्रा॒श्ञा त्य॒द्भि र॒द्भिः प्रा॒श्ञाति॑ प्राशि॒त्रम् प्रा॑शि॒त्रम् प्रा॒श्ञा त्य॒द्भिः । \newline
31. प्रा॒शि॒त्रमिति॑ प्र - अ॒शि॒त्रम् । \newline
32. प्रा॒श्ञा त्य॒द्भि र॒द्भिः प्रा॒श्ञाति॑ प्रा॒श्ञा त्य॒द्भिर् मा᳚र्जयि॒त्वा मा᳚र्जयि॒त्वा ऽद्भिः प्रा॒श्ञाति॑ प्रा॒श्ञा त्य॒द्भिर् मा᳚र्जयि॒त्वा । \newline
33. प्रा॒श्ञातीति॑ प्र - अ॒श्ञाति॑ । \newline
34. अ॒द्भिर् मा᳚र्जयि॒त्वा मा᳚र्जयि॒त्वा ऽद्भिर॒द्भिर् मा᳚र्जयि॒त्वा प्रा॒णान् प्रा॒णान् मा᳚र्जयि॒त्वा ऽद्भि र॒द्भिर् मा᳚र्जयि॒त्वा प्रा॒णान् । \newline
35. अ॒द्भिरित्य॑त् - भिः । \newline
36. मा॒र्ज॒यि॒त्वा प्रा॒णान् प्रा॒णान् मा᳚र्जयि॒त्वा मा᳚र्जयि॒त्वा प्रा॒णान् थ्सꣳ सम् प्रा॒णान् मा᳚र्जयि॒त्वा मा᳚र्जयि॒त्वा प्रा॒णान् थ्सम् । \newline
37. प्रा॒णान् थ्सꣳ सम् प्रा॒णान् प्रा॒णान् थ्सम् मृ॑शते मृशते॒ सम् प्रा॒णान् प्रा॒णान् थ्सम् मृ॑शते । \newline
38. प्रा॒णानिति॑ प्र - अ॒नान् । \newline
39. सम् मृ॑शते मृशते॒ सꣳ सम् मृ॑शते॒ ऽमृत॑ म॒मृत॑म् मृशते॒ सꣳ सम् मृ॑शते॒ ऽमृत᳚म् । \newline
40. मृ॒श॒ते॒ ऽमृत॑ म॒मृत॑म् मृशते मृशते॒ ऽमृतं॒ ॅवै वा अ॒मृत॑म् मृशते मृशते॒ ऽमृतं॒ ॅवै । \newline
41. अ॒मृतं॒ ॅवै वा अ॒मृत॑ म॒मृतं॒ ॅवै प्रा॒णाः प्रा॒णा वा अ॒मृत॑ म॒मृतं॒ ॅवै प्रा॒णाः । \newline
42. वै प्रा॒णाः प्रा॒णा वै वै प्रा॒णा अ॒मृत॑ म॒मृत॑म् प्रा॒णा वै वै प्रा॒णा अ॒मृत᳚म् । \newline
43. प्रा॒णा अ॒मृत॑ म॒मृत॑म् प्रा॒णाः प्रा॒णा अ॒मृत॒ माप॒ आपो॒ ऽमृत॑म् प्रा॒णाः प्रा॒णा अ॒मृत॒ मापः॑ । \newline
44. प्रा॒णा इति॑ प्र - अ॒नाः । \newline
45. अ॒मृत॒ माप॒ आपो॒ ऽमृत॑ म॒मृत॒ मापः॑ प्रा॒णान् प्रा॒णा नापो॒ ऽमृत॑ म॒मृत॒ मापः॑ प्रा॒णान् । \newline
46. आपः॑ प्रा॒णान् प्रा॒णा नाप॒ आपः॑ प्रा॒णा ने॒वैव प्रा॒णा नाप॒ आपः॑ प्रा॒णा ने॒व । \newline
47. प्रा॒णा ने॒वैव प्रा॒णान् प्रा॒णा ने॒व य॑थास्था॒नं ॅय॑थास्था॒न मे॒व प्रा॒णान् प्रा॒णा ने॒व य॑थास्था॒नम् । \newline
48. प्रा॒णानिति॑ प्र - अ॒नान् । \newline
49. ए॒व य॑थास्था॒नं ॅय॑थास्था॒न मे॒वैव य॑थास्था॒न मुपोप॑ यथास्था॒न मे॒वैव य॑थास्था॒न मुप॑ । \newline
50. य॒था॒स्था॒न मुपोप॑ यथास्था॒नं ॅय॑थास्था॒न मुप॑ ह्वयते ह्वयत॒ उप॑ यथास्था॒नं ॅय॑थास्था॒न मुप॑ ह्वयते । \newline
51. य॒था॒स्था॒नमिति॑ यथा - स्था॒नम् । \newline
52. उप॑ ह्वयते ह्वयत॒ उपोप॑ ह्वयते । \newline
53. ह्व॒य॒त॒ इति॑ ह्वयते । \newline
\pagebreak
\markright{ TS 2.6.9.1  \hfill https://www.vedavms.in \hfill}
\addcontentsline{toc}{section}{ TS 2.6.9.1 }
\section*{ TS 2.6.9.1 }

\textbf{TS 2.6.9.1 } \newline
\textbf{Samhita Paata} \newline

अ॒ग्नीध॒ आ द॑धा-त्य॒ग्निमु॑खा-ने॒वर्तून् प्री॑णाति स॒मिध॒मा द॑धा॒त्युत्त॑रासा॒-माहु॑तीनां॒ प्रति॑ष्ठित्या॒ अथो॑ स॒मिद्व॑त्ये॒व जु॑होति परि॒धीन्थ् सं मा᳚र्ष्टि पु॒नात्ये॒वैना᳚न्थ् स॒कृथ् स॑कृ॒थ् सं मा᳚र्ष्टि॒ परा॑ङिव॒ ह्ये॑तर्.हि॑ य॒ज्ञ्श्च॒तुः संप॑द्यते॒ चतु॑ष्पादः प॒शवः॑ प॒शूने॒वाव॑ रुन्धे॒ ब्रह्म॒न् प्रस्था᳚स्याम॒ इत्या॒हात्र॒ वा ए॒तर्.हि॑ य॒ज्ञ्ः श्रि॒तो - [  ] \newline

\textbf{Pada Paata} \newline

अ॒ग्नीध॒ इत्य॑ग्नि - इधे᳚ । एति॑ । द॒धा॒ति॒ । अ॒ग्निमु॑खा॒नित्य॒ग्नि - मु॒खा॒न् । ए॒व । ऋ॒तून् । प्री॒णा॒ति॒ । स॒मिध॒मिति॑ सं - इध᳚म् । एति॑ । द॒धा॒ति॒ । उत्त॑रासा॒मित्युत् - त॒रा॒सा॒म् । आहु॑तीना॒मित्या - हु॒ती॒ना॒म् । प्रति॑ष्ठित्या॒ इति॒ प्रति॑ - स्थि॒त्यै॒ । अथो॒ इति॑ । स॒मिद्व॒तीति॑ स॒मित् - व॒ति॒ । ए॒व । जु॒हो॒ति॒ । प॒रि॒धीनिति॑ परि - धीन् । समिति॑ । मा॒र्ष्टि॒ । पु॒नाति॑ । ए॒व । ए॒ना॒न् । स॒कृथ्स॑कृ॒दिति॑ स॒कृत् - स॒कृ॒त् । समिति॑ । मा॒र्ष्टि॒ । पराङ्॑ । इ॒व॒ । हि । ए॒तर्.हि॑ । य॒ज्ञ्ः । च॒तुः । समिति॑ । प॒द्य॒ते॒ । चतु॑ष्पाद॒ इति॒ चतुः॑ - पा॒दः॒ । प॒शवः॑ । प॒शून् । ए॒व । अवेति॑ । रु॒न्धे॒ । ब्रह्मन्न्॑ । प्रेति॑ । स्था॒स्या॒मः॒ । इति॑ । आ॒ह॒ । अत्र॑ । वै । ए॒तर्.हि॑ । य॒ज्ञ्ः । श्रि॒तः ।  \newline


\textbf{Krama Paata} \newline

अ॒ग्नीध॒ आ । अ॒ग्नीध॒ इत्य॑ग्नि - इधे᳚ । आ द॑धाति । द॒धा॒त्य॒ग्निमु॑खान् । अ॒ग्निमु॑खाने॒व । अ॒ग्निमु॑खा॒नित्य॒ग्नि - मु॒खा॒न्॒ । ए॒वर्तून् । ऋ॒तून् प्री॑णाति । प्री॒णा॒ति॒ स॒मिध᳚म् । स॒मिध॒मा । स॒मिध॒मिति॑ सं - इध᳚म् । आ द॑धाति । द॒धा॒त्युत्त॑रासाम् । उत्त॑रासा॒माहु॑तीनाम् । उत्त॑रासा॒मित्युत् - त॒रा॒सा॒म् । आहु॑तीना॒म् प्रति॑ष्ठित्यै । आहु॑तीना॒मित्या - हु॒ती॒ना॒म् । प्रति॑ष्ठित्या॒ अथो᳚ । प्रति॑ष्ठित्या॒ इति॒ प्रति॑ - स्थि॒त्यै॒ । अथो॑ स॒मिद्व॑ति । अथो॒ इत्यथो᳚ । स॒मिद्व॑त्ये॒व । स॒मिद्व॒तीति॑ स॒मित् - व॒ति॒ । ए॒व जु॑होति । जु॒हो॒ति॒ प॒रि॒धीन् । प॒रि॒धीन्थ् सम् । प॒रि॒धीनिति॑ परि - धीन् । सम् मा᳚र्ष्टि । मा॒र्ष्टि॒ पु॒नाति॑ । पु॒नात्ये॒व । ए॒वैनान्॑ । ए॒ना॒न्थ् स॒कृथ्स॑कृत् । स॒कृथ्स॑कृ॒थ् सम् । स॒कृथ्स॑कृ॒दिति॑ स॒कृत् - स॒कृ॒त्॒ । सम् मा᳚र्ष्टि । मा॒र्ष्टि॒ पराङ्॑ । परा॑ङिव । इ॒व॒ हि । ह्ये॑तर्.हि॑ । ए॒तर्.हि॑ य॒ज्ञ्ः । य॒ज्ञ्श्च॒तुः । च॒तुः सम् । सं प॑द्यते । प॒द्य॒ते॒ चतु॑ष्पादः । चतु॑ष्पादः प॒शवः॑ । चतु॑ष्टाद॒ इति॒ चतुः॑ - पा॒दः॒ । प॒शवः॑ प॒शून् । प॒शूने॒व । ए॒वाव॑ । अव॑ रुन्धे । रु॒न्धे॒ ब्रह्मन्न्॑ । ब्रह्म॒न् प्र । प्र स्था᳚स्यामः । स्था॒स्या॒म॒ इति॑ । इत्या॑ह । आ॒हात्र॑ । अत्र॒ वै । वा ए॒तर्.हि॑ । ए॒तर्.हि॑ य॒ज्ञ्ः । य॒ज्ञ्ः श्रि॒तः । श्रि॒तो यत्र॑ \newline

\textbf{Jatai Paata} \newline

1. अ॒ग्नीध॒ आ ऽग्नीधे॒ ऽग्नीध॒ आ । \newline
2. अ॒ग्नीध॒ इत्य॑ग्नि - इधे᳚ । \newline
3. आ द॑धाति दधा॒त्या द॑धाति । \newline
4. द॒धा॒ त्य॒ग्निमु॑खा न॒ग्निमु॑खान् दधाति दधा त्य॒ग्निमु॑खान् । \newline
5. अ॒ग्निमु॑खा ने॒वैवाग्निमु॑खा न॒ग्निमु॑खा ने॒व । \newline
6. अ॒ग्निमु॑खा॒नित्य॒ग्नि - मु॒खा॒न् । \newline
7. ए॒व र्‌तू नृ॒तू ने॒वैव र्‌तून् । \newline
8. ऋ॒तून् प्री॑णाति प्रीणा त्यृ॒तू नृ॒तून् प्री॑णाति । \newline
9. प्री॒णा॒ति॒ स॒मिधꣳ॑ स॒मिध॑म् प्रीणाति प्रीणाति स॒मिध᳚म् । \newline
10. स॒मिध॒ मा स॒मिधꣳ॑ स॒मिध॒ मा । \newline
11. स॒मिध॒मिति॑ सं - इध᳚म् । \newline
12. आ द॑धाति दधा॒ त्या द॑धाति । \newline
13. द॒धा॒ त्युत्त॑रासा॒ मुत्त॑रासाम् दधाति दधा॒ त्युत्त॑रासाम् । \newline
14. उत्त॑रासा॒ माहु॑तीना॒ माहु॑तीना॒ मुत्त॑रासा॒ मुत्त॑रासा॒ माहु॑तीनाम् । \newline
15. उत्त॑रासा॒मित्युत् - त॒रा॒सा॒म् । \newline
16. आहु॑तीना॒म् प्रति॑ष्ठित्यै॒ प्रति॑ष्ठित्या॒ आहु॑तीना॒ माहु॑तीना॒म् प्रति॑ष्ठित्यै । \newline
17. आहु॑तीना॒मित्या - हु॒ती॒ना॒म् । \newline
18. प्रति॑ष्ठित्या॒ अथो॒ अथो॒ प्रति॑ष्ठित्यै॒ प्रति॑ष्ठित्या॒ अथो᳚ । \newline
19. प्रति॑ष्ठित्या॒ इति॒ प्रति॑ - स्थि॒त्यै॒ । \newline
20. अथो॑ स॒मिद्व॑ति स॒मिद्व॒ त्यथो॒ अथो॑ स॒मिद्व॑ति । \newline
21. अथो॒ इत्यथो᳚ । \newline
22. स॒मिद्व॑ त्ये॒वैव स॒मिद्व॑ति स॒मिद्व॑ त्ये॒व । \newline
23. स॒मिद्व॒तीति॑ स॒मित् - व॒ति॒ । \newline
24. ए॒व जु॑होति जुहो त्ये॒वैव जु॑होति । \newline
25. जु॒हो॒ति॒ प॒रि॒धीन् प॑रि॒धीन् जु॑होति जुहोति परि॒धीन् । \newline
26. प॒रि॒धीन् थ्सꣳ सम् प॑रि॒धीन् प॑रि॒धीन् थ्सम् । \newline
27. प॒रि॒धीनिति॑ परि - धीन् । \newline
28. सम् मा᳚र्ष्टि मार्ष्टि॒ सꣳ सम् मा᳚र्ष्टि । \newline
29. मा॒र्ष्टि॒ पु॒नाति॑ पु॒नाति॑ मार्ष्टि मार्ष्टि पु॒नाति॑ । \newline
30. पु॒ना त्ये॒वैव पु॒नाति॑ पु॒ना त्ये॒व । \newline
31. ए॒वैना॑ नेना ने॒वैवैनान्॑ । \newline
32. ए॒ना॒न् थ्स॒कृथ्स॑कृथ् स॒कृथ्स॑कृ देना नेनान् थ्स॒कृथ्स॑कृत् । \newline
33. स॒कृथ्स॑कृ॒थ् सꣳ सꣳ स॒कृथ्स॑कृथ् स॒कृथ्स॑कृ॒थ् सम् । \newline
34. स॒कृथ्स॑कृ॒दिति॑ स॒कृत् - स॒कृ॒त् । \newline
35. सम् मा᳚र्ष्टि मार्ष्टि॒ सꣳ सम् मा᳚र्ष्टि । \newline
36. मा॒र्ष्टि॒ परा॒ङ् परा᳚ङ् मार्ष्टि मार्ष्टि॒ पराङ्॑ । \newline
37. परा॑ ङिवे व॒ परा॒ङ् परा॑ ङिव । \newline
38. इ॒व॒ हि हीवे॑ व॒ हि । \newline
39. ह्ये॑तर् ह्ये॒तर्.हि॒ हि ह्ये॑तर्.हि॑ । \newline
40. ए॒तर्.हि॑ य॒ज्ञो य॒ज्ञ् ए॒तर् ह्ये॒तर्.हि॑ य॒ज्ञ्ः । \newline
41. य॒ज्ञ् श्च॒तु श्च॒तुर् य॒ज्ञो य॒ज्ञ् श्च॒तुः । \newline
42. च॒तुः सꣳ सम् च॒तु श्च॒तुः सम् । \newline
43. सम् प॑द्यते पद्यते॒ सꣳ सम् प॑द्यते । \newline
44. प॒द्य॒ते॒ चतु॑ष्पाद॒ श्चतु॑ष्पादः पद्यते पद्यते॒ चतु॑ष्पादः । \newline
45. चतु॑ष्पादः प॒शवः॑ प॒शव॒ श्चतु॑ष्पाद॒ श्चतु॑ष्पादः प॒शवः॑ । \newline
46. चतु॑ष्पाद॒ इति॒ चतुः॑ - पा॒दः॒ । \newline
47. प॒शवः॑ प॒शून् प॒शून् प॒शवः॑ प॒शवः॑ प॒शून् । \newline
48. प॒शू ने॒वैव प॒शून् प॒शू ने॒व । \newline
49. ए॒वावा वै॒वैवाव॑ । \newline
50. अव॑ रुन्धे रु॒न्धे ऽवाव॑ रुन्धे । \newline
51. रु॒न्धे॒ ब्रह्म॒न् ब्रह्म॑न् रुन्धे रुन्धे॒ ब्रह्मन्न्॑ । \newline
52. ब्रह्म॒न् प्र प्र ब्रह्म॒न् ब्रह्म॒न् प्र । \newline
53. प्र स्था᳚स्यामः स्थास्यामः॒ प्र प्र स्था᳚स्यामः । \newline
54. स्था॒स्या॒म॒ इतीति॑ स्थास्यामः स्थास्याम॒ इति॑ । \newline
55. इत्या॑हा॒हे तीत्या॑ह । \newline
56. आ॒हा त्रात्रा॑ हा॒हात्र॑ । \newline
57. अत्र॒ वै वा अत्रात्र॒ वै । \newline
58. वा ए॒तर् ह्ये॒तर्.हि॒ वै वा ए॒तर्.हि॑ । \newline
59. ए॒तर्.हि॑ य॒ज्ञो य॒ज्ञ् ए॒तर् ह्ये॒तर्.हि॑ य॒ज्ञ्ः । \newline
60. य॒ज्ञ्ः श्रि॒तः श्रि॒तो य॒ज्ञो य॒ज्ञ्ः श्रि॒तः । \newline
61. श्रि॒तो यत्र॒ यत्र॑ श्रि॒तः श्रि॒तो यत्र॑ । \newline

\textbf{Ghana Paata } \newline

1. अ॒ग्नीध॒ आ ऽग्नीधे॒ ऽग्नीध॒ आ द॑धाति दधा॒त्या ऽग्नीधे॒ ऽग्नीध॒ आ द॑धाति । \newline
2. अ॒ग्नीध॒ इत्य॑ग्नि - इधे᳚ । \newline
3. आ द॑धाति दधा॒त्या द॑धा त्य॒ग्निमु॑खा न॒ग्निमु॑खान् दधा॒त्या द॑धा त्य॒ग्निमु॑खान् । \newline
4. द॒धा॒ त्य॒ग्निमु॑खा न॒ग्निमु॑खान् दधाति दधा त्य॒ग्निमु॑खा ने॒वैवा ग्निमु॑खान् दधाति दधा त्य॒ग्निमु॑खा ने॒व । \newline
5. अ॒ग्निमु॑खा ने॒वैवाग्निमु॑खा न॒ग्निमु॑खा ने॒व र्तू नृ॒तू ने॒वाग्निमु॑खा न॒ग्निमु॑खा ने॒व र्तून् । \newline
6. अ॒ग्निमु॑खा॒नित्य॒ग्नि - मु॒खा॒न् । \newline
7. ए॒व र्तू नृ॒तू ने॒वैव र्तून् प्री॑णाति प्रीणा त्यृ॒तू ने॒वैव र्तून् प्री॑णाति । \newline
8. ऋ॒तून् प्री॑णाति प्रीणात्यृ॒तू नृ॒तून् प्री॑णाति स॒मिधꣳ॑ स॒मिध॑म् प्रीणात्यृ॒तू नृ॒तून् प्री॑णाति स॒मिध᳚म् । \newline
9. प्री॒णा॒ति॒ स॒मिधꣳ॑ स॒मिध॑म् प्रीणाति प्रीणाति स॒मिध॒ मा स॒मिध॑म् प्रीणाति प्रीणाति स॒मिध॒ मा । \newline
10. स॒मिध॒ मा स॒मिधꣳ॑ स॒मिध॒ मा द॑धाति दधा॒त्या स॒मिधꣳ॑ स॒मिध॒ मा द॑धाति । \newline
11. स॒मिध॒मिति॑ सं - इध᳚म् । \newline
12. आ द॑धाति दधा॒त्या द॑धा॒ त्युत्त॑रासा॒ मुत्त॑रासाम् दधा॒त्या द॑धा॒ त्युत्त॑रासाम् । \newline
13. द॒धा॒ त्युत्त॑रासा॒ मुत्त॑रासाम् दधाति दधा॒ त्युत्त॑रासा॒ माहु॑तीना॒ माहु॑तीना॒ मुत्त॑रासाम् दधाति दधा॒ त्युत्त॑रासा॒ माहु॑तीनाम् । \newline
14. उत्त॑रासा॒ माहु॑तीना॒ माहु॑तीना॒ मुत्त॑रासा॒ मुत्त॑रासा॒ माहु॑तीना॒म् प्रति॑ष्ठित्यै॒ प्रति॑ष्ठित्या॒ आहु॑तीना॒ मुत्त॑रासा॒ मुत्त॑रासा॒ माहु॑तीना॒म् प्रति॑ष्ठित्यै । \newline
15. उत्त॑रासा॒मित्युत् - त॒रा॒सा॒म् । \newline
16. आहु॑तीना॒म् प्रति॑ष्ठित्यै॒ प्रति॑ष्ठित्या॒ आहु॑तीना॒ माहु॑तीना॒म् प्रति॑ष्ठित्या॒ अथो॒ अथो॒ प्रति॑ष्ठित्या॒ आहु॑तीना॒ माहु॑तीना॒म् प्रति॑ष्ठित्या॒ अथो᳚ । \newline
17. आहु॑तीना॒मित्या - हु॒ती॒ना॒म् । \newline
18. प्रति॑ष्ठित्या॒ अथो॒ अथो॒ प्रति॑ष्ठित्यै॒ प्रति॑ष्ठित्या॒ अथो॑ स॒मिद्व॑ति स॒मिद्व॒ त्यथो॒ प्रति॑ष्ठित्यै॒ प्रति॑ष्ठित्या॒ अथो॑ स॒मिद्व॑ति । \newline
19. प्रति॑ष्ठित्या॒ इति॒ प्रति॑ - स्थि॒त्यै॒ । \newline
20. अथो॑ स॒मिद्व॑ति स॒मिद्व॒ त्यथो॒ अथो॑ स॒मिद्व॑ त्ये॒वैव स॒मिद्व॒त्यथो॒ अथो॑ स॒मिद्व॑त्ये॒व । \newline
21. अथो॒ इत्यथो᳚ । \newline
22. स॒मिद्व॑ त्ये॒वैव स॒मिद्व॑ति स॒मिद्व॑ त्ये॒व जु॑होति जुहोत्ये॒व स॒मिद्व॑ति स॒मिद्व॑ त्ये॒व जु॑होति । \newline
23. स॒मिद्व॒तीति॑ स॒मित् - व॒ति॒ । \newline
24. ए॒व जु॑होति जुहो त्ये॒वैव जु॑होति परि॒धीन् प॑रि॒धीन् जु॑हो त्ये॒वैव जु॑होति परि॒धीन् । \newline
25. जु॒हो॒ति॒ प॒रि॒धीन् प॑रि॒धीन् जु॑होति जुहोति परि॒धीन् थ्सꣳ सम् प॑रि॒धीन् जु॑होति जुहोति परि॒धीन् थ्सम् । \newline
26. प॒रि॒धीन् थ्सꣳ सम् प॑रि॒धीन् प॑रि॒धीन् थ्सम् मा᳚र्ष्टि मार्ष्टि॒ सम् प॑रि॒धीन् प॑रि॒धीन् थ्सम् मा᳚र्ष्टि । \newline
27. प॒रि॒धीनिति॑ परि - धीन् । \newline
28. सम् मा᳚र्ष्टि मार्ष्टि॒ सꣳ सम् मा᳚र्ष्टि पु॒नाति॑ पु॒नाति॑ मार्ष्टि॒ सꣳ सम् मा᳚र्ष्टि पु॒नाति॑ । \newline
29. मा॒र्ष्टि॒ पु॒नाति॑ पु॒नाति॑ मार्ष्टि मार्ष्टि पु॒ना त्ये॒वैव पु॒नाति॑ मार्ष्टि मार्ष्टि पु॒ना त्ये॒व । \newline
30. पु॒ना त्ये॒वैव पु॒नाति॑ पु॒ना त्ये॒वैना॑ नेना ने॒व पु॒नाति॑ पु॒ना त्ये॒वैनान्॑ । \newline
31. ए॒वैना॑ नेना ने॒वैवैना᳚न् थ्स॒कृथ्स॑कृथ् स॒कृथ्स॑कृ देना ने॒वैवैना᳚न् थ्स॒कृथ्स॑कृत् । \newline
32. ए॒ना॒न् थ्स॒कृथ्स॑कृथ् स॒कृथ्स॑कृ देना नेनान् थ्स॒कृथ्स॑कृ॒थ् सꣳ सꣳ स॒कृथ्स॑कृ देना नेनान् थ्स॒कृथ्स॑कृ॒थ् सम् । \newline
33. स॒कृथ्स॑कृ॒थ् सꣳ सꣳ स॒कृथ्स॑कृथ् स॒कृथ्स॑कृ॒थ् सम् मा᳚र्ष्टि मार्ष्टि॒ सꣳ स॒कृथ्स॑कृथ् स॒कृथ्स॑कृ॒थ् सम् मा᳚र्ष्टि । \newline
34. स॒कृथ्स॑कृ॒दिति॑ स॒कृत् - स॒कृ॒त् । \newline
35. सम् मा᳚र्ष्टि मार्ष्टि॒ सꣳ सम् मा᳚र्ष्टि॒ परा॒ङ् परा᳚ङ् मार्ष्टि॒ सꣳ सम् मा᳚र्ष्टि॒ पराङ्॑ । \newline
36. मा॒र्ष्टि॒ परा॒ङ् परा᳚ङ् मार्ष्टि मार्ष्टि॒ परा॑ङिवे व॒ परा᳚ङ् मार्ष्टि मार्ष्टि॒ परा॑ङिव । \newline
37. परा॑ङिवे व॒ परा॒ङ् परा॑ङिव॒ हि हीव॒ परा॒ङ् परा॑ङिव॒ हि । \newline
38. इ॒व॒ हि हीवे॑ व॒ ह्ये॑तर् ह्ये॒तर्.हि॒ हीवे॑ व॒ ह्ये॑तर्.हि॑ । \newline
39. ह्ये॑तर् ह्ये॒तर्.हि॒ हि ह्ये॑तर्.हि॑ य॒ज्ञो य॒ज्ञ् ए॒तर्.हि॒ हि ह्ये॑तर्.हि॑ य॒ज्ञ्ः । \newline
40. ए॒तर्.हि॑ य॒ज्ञो य॒ज्ञ् ए॒तर् ह्ये॒तर्.हि॑ य॒ज्ञ् श्च॒तु श्च॒तुर् य॒ज्ञ् ए॒तर् ह्ये॒तर्.हि॑ य॒ज्ञ् श्च॒तुः । \newline
41. य॒ज्ञ् श्च॒तु श्च॒तुर् य॒ज्ञो य॒ज्ञ् श्च॒तुः सꣳ सम् च॒तुर् य॒ज्ञो य॒ज्ञ् श्च॒तुः सम् । \newline
42. च॒तुः सꣳ सम् च॒तु श्च॒तुः सम् प॑द्यते पद्यते॒ सम् च॒तु श्च॒तुः सम् प॑द्यते । \newline
43. सम् प॑द्यते पद्यते॒ सꣳ सम् प॑द्यते॒ चतु॑ष्पाद॒ श्चतु॑ष्पादः पद्यते॒ सꣳ सम् प॑द्यते॒ चतु॑ष्पादः । \newline
44. प॒द्य॒ते॒ चतु॑ष्पाद॒ श्चतु॑ष्पादः पद्यते पद्यते॒ चतु॑ष्पादः प॒शवः॑ प॒शव॒ श्चतु॑ष्पादः पद्यते पद्यते॒ चतु॑ष्पादः प॒शवः॑ । \newline
45. चतु॑ष्पादः प॒शवः॑ प॒शव॒ श्चतु॑ष्पाद॒ श्चतु॑ष्पादः प॒शवः॑ प॒शून् प॒शून् प॒शव॒ श्चतु॑ष्पाद॒ श्चतु॑ष्पादः प॒शवः॑ प॒शून् । \newline
46. चतु॑ष्पाद॒ इति॒ चतुः॑ - पा॒दः॒ । \newline
47. प॒शवः॑ प॒शून् प॒शून् प॒शवः॑ प॒शवः॑ प॒शू ने॒वैव प॒शून् प॒शवः॑ प॒शवः॑ प॒शू ने॒व । \newline
48. प॒शू ने॒वैव प॒शून् प॒शू ने॒वावावै॒व प॒शून् प॒शू ने॒वाव॑ । \newline
49. ए॒वावा वै॒वैवाव॑ रुन्धे रु॒न्धे ऽवै॒वैवाव॑ रुन्धे । \newline
50. अव॑ रुन्धे रु॒न्धे ऽवाव॑ रुन्धे॒ ब्रह्म॒न् ब्रह्म॑न् रु॒न्धे ऽवाव॑ रुन्धे॒ ब्रह्मन्न्॑ । \newline
51. रु॒न्धे॒ ब्रह्म॒न् ब्रह्म॑न् रुन्धे रुन्धे॒ ब्रह्म॒न् प्र प्र ब्रह्म॑न् रुन्धे रुन्धे॒ ब्रह्म॒न् प्र । \newline
52. ब्रह्म॒न् प्र प्र ब्रह्म॒न् ब्रह्म॒न् प्र स्था᳚स्यामः स्थास्यामः॒ प्र ब्रह्म॒न् ब्रह्म॒न् प्र स्था᳚स्यामः । \newline
53. प्र स्था᳚स्यामः स्थास्यामः॒ प्र प्र स्था᳚स्याम॒ इतीति॑ स्थास्यामः॒ प्र प्र स्था᳚स्याम॒ इति॑ । \newline
54. स्था॒स्या॒म॒ इतीति॑ स्थास्यामः स्थास्याम॒ इत्या॑हा॒हे ति॑ स्थास्यामः स्थास्याम॒ इत्या॑ह । \newline
55. इत्या॑हा॒हे तीत्या॒हा त्रात्रा॒हे तीत्या॒हात्र॑ । \newline
56. आ॒हा त्रात्रा॑ हा॒हात्र॒ वै वा अत्रा॑ हा॒हात्र॒ वै । \newline
57. अत्र॒ वै वा अत्रात्र॒ वा ए॒तर् ह्ये॒तर्.हि॒ वा अत्रात्र॒ वा ए॒तर्.हि॑ । \newline
58. वा ए॒तर् ह्ये॒तर्.हि॒ वै वा ए॒तर्.हि॑ य॒ज्ञो य॒ज्ञ् ए॒तर्.हि॒ वै वा ए॒तर्.हि॑ य॒ज्ञ्ः । \newline
59. ए॒तर्.हि॑ य॒ज्ञो य॒ज्ञ् ए॒तर् ह्ये॒तर्.हि॑ य॒ज्ञ्ः श्रि॒तः श्रि॒तो य॒ज्ञ् ए॒तर् ह्ये॒तर्.हि॑ य॒ज्ञ्ः श्रि॒तः । \newline
60. य॒ज्ञ्ः श्रि॒तः श्रि॒तो य॒ज्ञो य॒ज्ञ्ः श्रि॒तो यत्र॒ यत्र॑ श्रि॒तो य॒ज्ञो य॒ज्ञ्ः श्रि॒तो यत्र॑ । \newline
61. श्रि॒तो यत्र॒ यत्र॑ श्रि॒तः श्रि॒तो यत्र॑ ब्र॒ह्मा ब्र॒ह्मा यत्र॑ श्रि॒तः श्रि॒तो यत्र॑ ब्र॒ह्मा । \newline
\pagebreak
\markright{ TS 2.6.9.2  \hfill https://www.vedavms.in \hfill}
\addcontentsline{toc}{section}{ TS 2.6.9.2 }
\section*{ TS 2.6.9.2 }

\textbf{TS 2.6.9.2 } \newline
\textbf{Samhita Paata} \newline

यत्र॑ ब्र॒ह्मा यत्रै॒व य॒ज्ञ्ः श्रि॒तस्तत॑ ए॒वैन॒मा र॑भते॒ यद्धस्ते॑न प्र॒मीवे᳚द्वेप॒नः स्या॒द्यच्छी॒र्ष्णा शी॑र्.षक्ति॒मान्थ्-स्या॒द्यत् तू॒ष्णीमासी॒ता ऽस॑प्रंत्तो य॒ज्ञ्ः स्या॒त् प्रति॒ष्ठेत्ये॒व ब्रू॑याद्-वा॒चि वै य॒ज्ञ्ः श्रि॒तो यत्रै॒व य॒ज्ञ्ः श्रि॒तस्तत॑ ए॒वैनꣳ॒॒ सं प्र य॑च्छति॒ देव॑ सवितरे॒तत् ते॒ प्रा - [  ] \newline

\textbf{Pada Paata} \newline

यत्र॑ । ब्र॒ह्मा । यत्र॑ । ए॒व । य॒ज्ञ्ः । श्रि॒तः । ततः॑ । ए॒व । ए॒न॒म् । एति॑ । र॒भ॒ते॒ । यत् । हस्ते॑न । प्र॒मीवे॒दिति॑ प्र - मीवे᳚त् । वे॒प॒नः । स्या॒त् । यत् । शी॒र्ष्णा । शी॒र्॒.ष॒क्ति॒मानिति॑ शीर्.षक्ति-मान् । स्या॒त् । यत् । तू॒ष्णीम् । आसी॑त । अस॑प्रंत्त॒ इत्यसं᳚-प्र॒त्तः॒ । य॒ज्ञ्ः । स्या॒त् । प्रति॑ । ति॒ष्ठ॒ । इति॑ । ए॒व । ब्रू॒या॒त् । वा॒चि । वै । य॒ज्ञ्ः । श्रि॒तः । यत्र॑ । ए॒व । य॒ज्ञ्ः । श्रि॒तः । ततः॑ । ए॒व । ए॒न॒म् । सम् । प्रेति॑ । य॒च्छ॒ति॒ । देव॑ । स॒वि॒तः॒ । ए॒तत् । ते॒ । प्रेति॑ ।  \newline


\textbf{Krama Paata} \newline

यत्र॑ ब्र॒ह्मा । ब्र॒ह्मा यत्र॑ । यत्रै॒व । ए॒व य॒ज्ञ्ः । य॒ज्ञ्ः श्रि॒तः । श्रि॒तस्ततः॑ । तत॑ ए॒व । ए॒वैन᳚म् । ए॒न॒मा । आ र॑भते । र॒भ॒ते॒ यत् । यद्धस्ते॑न । हस्ते॑न प्र॒मीवे᳚त् । प्र॒मीवे᳚द् वेप॒नः । प्र॒मीवे॒दिति॑ प्र - मीवे᳚त् । वे॒प॒नः स्या᳚त् । स्या॒द् यत् । यच्छी॒र्.॒ष्णा । शी॒र्.॒ष्णा शी॑र्.षक्ति॒मान् । शी॒र्.॒ष॒क्ति॒मान्थ् स्या᳚त् । शी॒र्॒.ष॒क्ति॒मानिति॑ शीर्.षक्ति - मान् । स्या॒द् यत् । यत् तू॒ष्णीम् । तू॒ष्णीमासी॑त । आसी॒तास॑म्प्रत्तः । अस॑म्प्रत्तो य॒ज्ञ्ः । अस॑म्प्रत्त॒ इत्यस᳚म् - प्र॒त्तः॒ । य॒ज्ञ्ः स्या᳚त् । स्या॒त् प्र । प्र ति॑ष्ठ । ति॒ष्ठेति॑ । इत्ये॒व । ए॒व ब्रू॑यात् । ब्रू॒या॒द् वा॒चि । वा॒चि वै । वै य॒ज्ञ्ः । य॒ज्ञ्ः श्रि॒तः । श्रि॒तो यत्र॑ । यत्रै॒व । ए॒व य॒ज्ञ्ः । य॒ज्ञ्ः श्रि॒तः । श्रि॒तस्ततः॑ । तत॑ ए॒व । ए॒वैन᳚म् । ए॒नꣳ॒॒ सम् । सम् प्र । प्र य॑च्छति । य॒च्छ॒ति॒ देव॑ । देव॑ सवितः । स॒वि॒त॒रे॒तत् । ए॒तत् ते᳚ । ते॒ प्र । प्राह॑ \newline

\textbf{Jatai Paata} \newline

1. यत्र॑ ब्र॒ह्मा ब्र॒ह्मा यत्र॒ यत्र॑ ब्र॒ह्मा । \newline
2. ब्र॒ह्मा यत्र॒ यत्र॑ ब्र॒ह्मा ब्र॒ह्मा यत्र॑ । \newline
3. यत्रै॒वैव यत्र॒ यत्रै॒व । \newline
4. ए॒व य॒ज्ञो य॒ज्ञ् ए॒वैव य॒ज्ञ्ः । \newline
5. य॒ज्ञ्ः श्रि॒तः श्रि॒तो य॒ज्ञो य॒ज्ञ्ः श्रि॒तः । \newline
6. श्रि॒त स्तत॒ स्ततः॑ श्रि॒तः श्रि॒त स्ततः॑ । \newline
7. तत॑ ए॒वैव तत॒ स्तत॑ ए॒व । \newline
8. ए॒वैन॑ मेन मे॒वैवैन᳚म् । \newline
9. ए॒न॒ मैन॑ मेन॒ मा । \newline
10. आ र॑भते रभत॒ आ र॑भते । \newline
11. र॒भ॒ते॒ यद् यद् र॑भते रभते॒ यत् । \newline
12. य द्धस्ते॑न॒ हस्ते॑न॒ यद् य द्धस्ते॑न । \newline
13. हस्ते॑न प्र॒मीवे᳚त् प्र॒मीवे॒ द्धस्ते॑न॒ हस्ते॑न प्र॒मीवे᳚त् । \newline
14. प्र॒मीवे᳚द् वेप॒नो वे॑प॒नः प्र॒मीवे᳚त् प्र॒मीवे᳚द् वेप॒नः । \newline
15. प्र॒मीवे॒दिति॑ प्र - मीवे᳚त् । \newline
16. वे॒प॒नः स्या᳚थ् स्याद् वेप॒नो वे॑प॒नः स्या᳚त् । \newline
17. स्या॒द् यद् यथ् स्या᳚थ् स्या॒द् यत् । \newline
18. यच् छी॒र्ष्णा शी॒र्ष्णा यद् यच् छी॒र्ष्णा । \newline
19. शी॒र्ष्णा शी॑र्.षक्ति॒माञ् छी॑र्.षक्ति॒माञ् छी॒र्ष्णा शी॒र्ष्णा शी॑र्.षक्ति॒मान् । \newline
20. शी॒र्॒.ष॒क्ति॒मान् थ्स्या᳚थ् स्याच्छीर्.षक्ति॒माञ् छी॑र्.षक्ति॒मान् थ्स्या᳚त् । \newline
21. शी॒र्॒.ष॒क्ति॒मानिति॑ शीर्.षक्ति - मान् । \newline
22. स्या॒द् यद् यथ् स्या᳚थ् स्या॒द् यत् । \newline
23. यत् तू॒ष्णीम् तू॒ष्णीं ॅयद् यत् तू॒ष्णीम् । \newline
24. तू॒ष्णी मासी॒ तासी॑त तू॒ष्णीम् तू॒ष्णी मासी॑त । \newline
25. आसी॒ता स॑म्प्र॒त्तो ऽस॑म्प्रत्त॒ आसी॒ता सी॒ता स॑म्प्रत्तः । \newline
26. अस॑म्प्रत्तो य॒ज्ञो य॒ज्ञो ऽस॑म्प्र॒त्तो ऽस॑म्प्रत्तो य॒ज्ञ्ः । \newline
27. अस॑म्प्रत्त॒ इत्यसं᳚ - प्र॒त्तः॒ । \newline
28. य॒ज्ञ्ः स्या᳚थ् स्याद् य॒ज्ञो य॒ज्ञ्ः स्या᳚त् । \newline
29. स्या॒त् प्र प्र स्या᳚थ् स्या॒त् प्र । \newline
30. प्र ति॑ष्ठ तिष्ठ॒ प्र प्र ति॑ष्ठ । \newline
31. ति॒ष्ठे तीति॑ तिष्ठ ति॒ष्ठे ति॑ । \newline
32. इत्ये॒वैवे तीत्ये॒व । \newline
33. ए॒व ब्रू॑याद् ब्रूया दे॒वैव ब्रू॑यात् । \newline
34. ब्रू॒या॒द् वा॒चि वा॒चि ब्रू॑याद् ब्रूयाद् वा॒चि । \newline
35. वा॒चि वै वै वा॒चि वा॒चि वै । \newline
36. वै य॒ज्ञो य॒ज्ञो वै वै य॒ज्ञ्ः । \newline
37. य॒ज्ञ्ः श्रि॒तः श्रि॒तो य॒ज्ञो य॒ज्ञ्ः श्रि॒तः । \newline
38. श्रि॒तो यत्र॒ यत्र॑ श्रि॒तः श्रि॒तो यत्र॑ । \newline
39. यत्रै॒वैव यत्र॒ यत्रै॒व । \newline
40. ए॒व य॒ज्ञो य॒ज्ञ् ए॒वैव य॒ज्ञ्ः । \newline
41. य॒ज्ञ्ः श्रि॒तः श्रि॒तो य॒ज्ञो य॒ज्ञ्ः श्रि॒तः । \newline
42. श्रि॒त स्तत॒ स्ततः॑ श्रि॒तः श्रि॒त स्ततः॑ । \newline
43. तत॑ ए॒वैव तत॒ स्तत॑ ए॒व । \newline
44. ए॒वैन॑ मेन मे॒वैवैन᳚म् । \newline
45. ए॒नꣳ॒॒ सꣳ स मे॑न मेनꣳ॒॒ सम् । \newline
46. सम् प्र प्र सꣳ सम् प्र । \newline
47. प्र य॑च्छति यच्छति॒ प्र प्र य॑च्छति । \newline
48. य॒च्छ॒ति॒ देव॒ देव॑ यच्छति यच्छति॒ देव॑ । \newline
49. देव॑ सवितः सवित॒र् देव॒ देव॑ सवितः । \newline
50. स॒वि॒त॒ रे॒त दे॒तथ् स॑वितः सवित रे॒तत् । \newline
51. ए॒तत् ते॑ त ए॒त दे॒तत् ते᳚ । \newline
52. ते॒ प्र प्र ते॑ ते॒ प्र । \newline
53. प्राहा॑ह॒ प्र प्राह॑ । \newline

\textbf{Ghana Paata } \newline

1. यत्र॑ ब्र॒ह्मा ब्र॒ह्मा यत्र॒ यत्र॑ ब्र॒ह्मा यत्र॒ यत्र॑ ब्र॒ह्मा यत्र॒ यत्र॑ ब्र॒ह्मा यत्र॑ । \newline
2. ब्र॒ह्मा यत्र॒ यत्र॑ ब्र॒ह्मा ब्र॒ह्मा यत्रै॒वैव यत्र॑ ब्र॒ह्मा ब्र॒ह्मा यत्रै॒व । \newline
3. यत्रै॒वैव यत्र॒ यत्रै॒व य॒ज्ञो य॒ज्ञ् ए॒व यत्र॒ यत्रै॒व य॒ज्ञ्ः । \newline
4. ए॒व य॒ज्ञो य॒ज्ञ् ए॒वैव य॒ज्ञ्ः श्रि॒तः श्रि॒तो य॒ज्ञ् ए॒वैव य॒ज्ञ्ः श्रि॒तः । \newline
5. य॒ज्ञ्ः श्रि॒तः श्रि॒तो य॒ज्ञो य॒ज्ञ्ः श्रि॒त स्तत॒ स्ततः॑ श्रि॒तो य॒ज्ञो य॒ज्ञ्ः श्रि॒त स्ततः॑ । \newline
6. श्रि॒त स्तत॒ स्ततः॑ श्रि॒तः श्रि॒त स्तत॑ ए॒वैव ततः॑ श्रि॒तः श्रि॒त स्तत॑ ए॒व । \newline
7. तत॑ ए॒वैव तत॒ स्तत॑ ए॒वैन॑ मेन मे॒व तत॒ स्तत॑ ए॒वैन᳚म् । \newline
8. ए॒वैन॑ मेन मे॒वैवैन॒ मैन॑ मे॒वैवैन॒ मा । \newline
9. ए॒न॒ मैन॑ मेन॒ मा र॑भते रभत॒ ऐन॑ मेन॒ मा र॑भते । \newline
10. आ र॑भते रभत॒ आ र॑भते॒ यद् यद् र॑भत॒ आ र॑भते॒ यत् । \newline
11. र॒भ॒ते॒ यद् यद् र॑भते रभते॒ यद्धस्ते॑न॒ हस्ते॑न॒ यद् र॑भते रभते॒ यद्धस्ते॑न । \newline
12. यद्धस्ते॑न॒ हस्ते॑न॒ यद् यद्धस्ते॑न प्र॒मीवे᳚त् प्र॒मीवे॒ द्धस्ते॑न॒ यद् यद्धस्ते॑न प्र॒मीवे᳚त् । \newline
13. हस्ते॑न प्र॒मीवे᳚त् प्र॒मीवे॒ द्धस्ते॑न॒ हस्ते॑न प्र॒मीवे᳚द् वेप॒नो वे॑प॒नः प्र॒मीवे॒ द्धस्ते॑न॒ हस्ते॑न प्र॒मीवे᳚द् वेप॒नः । \newline
14. प्र॒मीवे᳚द् वेप॒नो वे॑प॒नः प्र॒मीवे᳚त् प्र॒मीवे᳚द् वेप॒नः स्या᳚थ् स्याद् वेप॒नः प्र॒मीवे᳚त् प्र॒मीवे᳚द् वेप॒नः स्या᳚त् । \newline
15. प्र॒मीवे॒दिति॑ प्र - मीवे᳚त् । \newline
16. वे॒प॒नः स्या᳚थ् स्याद् वेप॒नो वे॑प॒नः स्या॒द् यद् यथ् स्या᳚द् वेप॒नो वे॑प॒नः स्या॒द् यत् । \newline
17. स्या॒द् यद् यथ् स्या᳚थ् स्या॒द् यच्छी॒र्ष्णा शी॒र्ष्णा यथ् स्या᳚थ् स्या॒द् यच्छी॒र्ष्णा । \newline
18. यच्छी॒र्ष्णा शी॒र्ष्णा यद् यच्छी॒र्ष्णा शी॑र्.षक्ति॒माञ् छी॑र्.षक्ति॒माञ् छी॒र्ष्णा यद् यच्छी॒र्ष्णा शी॑र्.षक्ति॒मान् । \newline
19. शी॒र्ष्णा शी॑र्.षक्ति॒माञ् छी॑र्.षक्ति॒माञ् छी॒र्ष्णा शी॒र्ष्णा शी॑र्.षक्ति॒मान् थ्स्या᳚थ् स्याच् छीर्.षक्ति॒माञ् छी॒र्ष्णा शी॒र्ष्णा शी॑र्.षक्ति॒मान् थ्स्या᳚त् । \newline
20. शी॒र्॒.ष॒क्ति॒मान् थ्स्या᳚थ् स्याच् छीर्.षक्ति॒माञ् छी॑र्.षक्ति॒मान् थ्स्या॒द् यद् यथ् स्या᳚च् छीर्.षक्ति॒माञ् छी॑र्.षक्ति॒मान् थ्स्या॒द् यत् । \newline
21. शी॒र्॒.ष॒क्ति॒मानिति॑ शीर्.षक्ति - मान् । \newline
22. स्या॒द् यद् यथ् स्या᳚थ् स्या॒द् यत् तू॒ष्णीम् तू॒ष्णीं ॅयथ् स्या᳚थ् स्या॒द् यत् तू॒ष्णीम् । \newline
23. यत् तू॒ष्णीम् तू॒ष्णीं ॅयद् यत् तू॒ष्णी मासी॒ तासी॑त तू॒ष्णीं ॅयद् यत् तू॒ष्णी मासी॑त । \newline
24. तू॒ष्णी मासी॒ तासी॑त तू॒ष्णीम् तू॒ष्णी मासी॒ता स॑म्प्र॒त्तो ऽस॑म्प्रत्त॒ आसी॑त तू॒ष्णीम् तू॒ष्णी मासी॒ता स॑म्प्रत्तः । \newline
25. आसी॒ता स॑म्प्र॒त्तो ऽस॑म्प्रत्त॒ आसी॒ता सी॒ता स॑म्प्रत्तो य॒ज्ञो य॒ज्ञो ऽस॑म्प्रत्त॒ आसी॒ता सी॒ता स॑म्प्रत्तो य॒ज्ञ्ः । \newline
26. अस॑म्प्रत्तो य॒ज्ञो य॒ज्ञो ऽस॑म्प्र॒त्तो ऽस॑म्प्रत्तो य॒ज्ञ्ः स्या᳚थ् स्याद् य॒ज्ञो ऽस॑म्प्र॒त्तो ऽस॑म्प्रत्तो य॒ज्ञ्ः स्या᳚त् । \newline
27. अस॑म्प्रत्त॒ इत्यसं᳚ - प्र॒त्तः॒ । \newline
28. य॒ज्ञ्ः स्या᳚थ् स्याद् य॒ज्ञो य॒ज्ञ्ः स्या॒त् प्र प्र स्या᳚द् य॒ज्ञो य॒ज्ञ्ः स्या॒त् प्र । \newline
29. स्या॒त् प्र प्र स्या᳚थ् स्या॒त् प्र ति॑ष्ठ तिष्ठ॒ प्र स्या᳚थ् स्या॒त् प्र ति॑ष्ठ । \newline
30. प्र ति॑ष्ठ तिष्ठ॒ प्र प्र ति॒ष्ठे तीति॑ तिष्ठ॒ प्र प्र ति॒ष्ठे ति॑ । \newline
31. ति॒ष्ठे तीति॑ तिष्ठ ति॒ष्ठे त्ये॒वैवे ति॑ तिष्ठ ति॒ष्ठे त्ये॒व । \newline
32. इत्ये॒वैवे तीत्ये॒व ब्रू॑याद् ब्रूयादे॒वे तीत्ये॒व ब्रू॑यात् । \newline
33. ए॒व ब्रू॑याद् ब्रूया दे॒वैव ब्रू॑याद् वा॒चि वा॒चि ब्रू॑या दे॒वैव ब्रू॑याद् वा॒चि । \newline
34. ब्रू॒या॒द् वा॒चि वा॒चि ब्रू॑याद् ब्रूयाद् वा॒चि वै वै वा॒चि ब्रू॑याद् ब्रूयाद् वा॒चि वै । \newline
35. वा॒चि वै वै वा॒चि वा॒चि वै य॒ज्ञो य॒ज्ञो वै वा॒चि वा॒चि वै य॒ज्ञ्ः । \newline
36. वै य॒ज्ञो य॒ज्ञो वै वै य॒ज्ञ्ः श्रि॒तः श्रि॒तो य॒ज्ञो वै वै य॒ज्ञ्ः श्रि॒तः । \newline
37. य॒ज्ञ्ः श्रि॒तः श्रि॒तो य॒ज्ञो य॒ज्ञ्ः श्रि॒तो यत्र॒ यत्र॑ श्रि॒तो य॒ज्ञो य॒ज्ञ्ः श्रि॒तो यत्र॑ । \newline
38. श्रि॒तो यत्र॒ यत्र॑ श्रि॒तः श्रि॒तो यत्रै॒वैव यत्र॑ श्रि॒तः श्रि॒तो यत्रै॒व । \newline
39. यत्रै॒वैव यत्र॒ यत्रै॒व य॒ज्ञो य॒ज्ञ् ए॒व यत्र॒ यत्रै॒व य॒ज्ञ्ः । \newline
40. ए॒व य॒ज्ञो य॒ज्ञ् ए॒वैव य॒ज्ञ्ः श्रि॒तः श्रि॒तो य॒ज्ञ् ए॒वैव य॒ज्ञ्ः श्रि॒तः । \newline
41. य॒ज्ञ्ः श्रि॒तः श्रि॒तो य॒ज्ञो य॒ज्ञ्ः श्रि॒त स्तत॒ स्ततः॑ श्रि॒तो य॒ज्ञो य॒ज्ञ्ः श्रि॒त स्ततः॑ । \newline
42. श्रि॒त स्तत॒ स्ततः॑ श्रि॒तः श्रि॒त स्तत॑ ए॒वैव ततः॑ श्रि॒तः श्रि॒त स्तत॑ ए॒व । \newline
43. तत॑ ए॒वैव तत॒ स्तत॑ ए॒वैन॑ मेन मे॒व तत॒ स्तत॑ ए॒वैन᳚म् । \newline
44. ए॒वैन॑ मेन मे॒वैवैनꣳ॒॒ सꣳ स मे॑न मे॒वैवैनꣳ॒॒ सम् । \newline
45. ए॒नꣳ॒॒ सꣳ स मे॑न मेनꣳ॒॒ सम् प्र प्र स मे॑न मेनꣳ॒॒ सम् प्र । \newline
46. सम् प्र प्र सꣳ सम् प्र य॑च्छति यच्छति॒ प्र सꣳ सम् प्र य॑च्छति । \newline
47. प्र य॑च्छति यच्छति॒ प्र प्र य॑च्छति॒ देव॒ देव॑ यच्छति॒ प्र प्र य॑च्छति॒ देव॑ । \newline
48. य॒च्छ॒ति॒ देव॒ देव॑ यच्छति यच्छति॒ देव॑ सवितः सवित॒र् देव॑ यच्छति यच्छति॒ देव॑ सवितः । \newline
49. देव॑ सवितः सवित॒र् देव॒ देव॑ सवित रे॒त दे॒तथ् स॑वित॒र् देव॒ देव॑ सवित रे॒तत् । \newline
50. स॒वि॒त॒ रे॒त दे॒तथ् स॑वितः सवित रे॒तत् ते॑ त ए॒तथ् स॑वितः सवित रे॒तत् ते᳚ । \newline
51. ए॒तत् ते॑ त ए॒त दे॒तत् ते॒ प्र प्र त॑ ए॒त दे॒तत् ते॒ प्र । \newline
52. ते॒ प्र प्र ते॑ ते॒ प्राहा॑ह॒ प्र ते॑ ते॒ प्राह॑ । \newline
53. प्राहा॑ह॒ प्र प्राहे तीत्या॑ह॒ प्र प्राहे ति॑ । \newline
\pagebreak
\markright{ TS 2.6.9.3  \hfill https://www.vedavms.in \hfill}
\addcontentsline{toc}{section}{ TS 2.6.9.3 }
\section*{ TS 2.6.9.3 }

\textbf{TS 2.6.9.3 } \newline
\textbf{Samhita Paata} \newline

-ऽऽहेत्या॑ह॒ प्रसू᳚त्यै॒ बृह॒स्पति॑ र्ब्र॒ह्मेत्या॑ह॒ स हि ब्रह्मि॑ष्ठः॒ स य॒ज्ञ्ं पा॑हि॒ स य॒ज्ञ्प॑तिं पाहि॒ स मां पा॒हीत्या॑ह य॒ज्ञाय॒ यज॑मानाया॒ऽऽत्मने॒ तेभ्य॑ ए॒वाऽऽशिष॒मा शा॒स्तेऽना᳚र्त्या आ॒श्राव्या॑ऽऽह दे॒वान्. य॒जेति॑ ब्रह्मवा॒दिनो॑ वदन्ती॒ष्टा दे॒वता॒ अथ॑ कत॒म ए॒ते दे॒वा इति॒ छन्दाꣳ॒॒सीति॑ ब्रूयाद्-गाय॒त्रीं त्रि॒ष्टुभं॒ - [  ] \newline

\textbf{Pada Paata} \newline

आ॒ह॒ । इति॑ । आ॒ह॒ । प्रसू᳚त्या॒ इति॒ प्र - सू॒त्यै॒ । बृह॒स्पतिः॑ । ब्र॒ह्मा । इति॑ । आ॒ह॒ । सः । हि । ब्रह्मि॑ष्ठः । सः । य॒ज्ञ्म् । पा॒हि॒ । सः । य॒ज्ञ्प॑ति॒मिति॑ य॒ज्ञ् - प॒ति॒म् । पा॒हि॒ । सः । माम् । पा॒हि॒ । इति॑ । आ॒ह॒ । य॒ज्ञाय॑ । यज॑मानाय । आ॒त्मने᳚ । तेभ्यः॑ । ए॒व । आ॒शिष॒मित्या᳚ - शिष᳚म् । एति॑ । शा॒स्ते॒ । अना᳚र्त्यै । आ॒श्राव्येत्या᳚ - श्राव्य॑ । आ॒ह॒ । दे॒वान् । य॒ज॒ । इति॑ । ब्र॒ह्म॒वा॒दिन॒ इति॑ ब्रह्म - वा॒दिनः॑ । व॒द॒न्ति॒ । इ॒ष्टाः । दे॒वताः᳚ । अथ॑ । क॒त॒मे । ए॒ते । दे॒वाः । इति॑ । छन्दाꣳ॑सि । इति॑ । ब्रू॒या॒त् । गा॒य॒त्रीम् । त्रि॒ष्टुभ᳚म् ।  \newline


\textbf{Krama Paata} \newline

आ॒हेति॑ । इत्या॑ह । आ॒ह॒ प्रसू᳚त्यै । प्रसू᳚त्यै॒ बृह॒स्पतिः॑ । प्रसू᳚त्या॒ इति॒ प्र - सू॒त्यै॒ । बृह॒स्पति॑र् ब्र॒ह्मा । ब्र॒ह्मेति॑ । इत्या॑ह । आ॒ह॒ सः । स हि । हि ब्रह्मि॑ष्ठः । ब्रह्मि॑ष्ठः॒ सः । स य॒ज्ञ्म् । य॒ज्ञ्म् पा॑हि । पा॒हि॒ सः । स य॒ज्ञ्प॑तिम् । य॒ज्ञ्प॑तिम् पाहि । य॒ज्ञ्प॑ति॒मिति॑ य॒ज्ञ् - प॒ति॒म् । पा॒हि॒ सः । स माम् । माम् पा॑हि । पा॒हीति॑ । इत्या॑ह । आ॒ह॒ य॒ज्ञाय॑ । य॒ज्ञाय॒ यज॑मानाय । यज॑मानाया॒त्मने᳚ । आ॒त्मने॒ तेभ्यः॑ । तेभ्य॑ ए॒व । ए॒वाशिष᳚म् । आ॒शिष॒मा । आ॒शिष॒मित्या᳚ - शिष᳚म् । आ शा᳚स्ते । शा॒स्तेऽना᳚र्त्यै । अना᳚र्त्या आ॒श्राव्य॑ । आ॒श्राव्या॑ह । आ॒श्राव्येत्या᳚ - श्राव्य॑ । आ॒ह॒ दे॒वान् । दे॒वान्. य॑ज । य॒जेति॑ । इति॑ ब्रह्मवा॒दिनः॑ । ब्र॒ह्म॒वा॒दिनो॑ वदन्ति । ब्र॒ह्म॒वा॒दिन॒ इति॑ ब्रह्म - वा॒दिनः॑ । व॒द॒न्ती॒ष्टाः । इ॒ष्टा दे॒वताः᳚ । दे॒वता॒ अथ॑ । अथ॑ कत॒मे । क॒त॒म ए॒ते । ए॒ते दे॒वाः । दे॒वा इति॑ । इति॒ छन्दाꣳ॑सि । छन्दाꣳ॒॒सीति॑ । इति॑ ब्रूयात् । ब्रू॒या॒द् गा॒य॒त्रीम् । गा॒य॒त्रीम् त्रि॒ष्टुभ᳚म् । त्रि॒ष्टुभ॒म् जग॑तीम् \newline

\textbf{Jatai Paata} \newline

1. आ॒हे ती त्या॑हा॒हे ति॑ । \newline
2. इत्या॑हा॒हे तीत्या॑ह । \newline
3. आ॒ह॒ प्रसू᳚त्यै॒ प्रसू᳚त्या आहाह॒ प्रसू᳚त्यै । \newline
4. प्रसू᳚त्यै॒ बृह॒स्पति॒र् बृह॒स्पतिः॒ प्रसू᳚त्यै॒ प्रसू᳚त्यै॒ बृह॒स्पतिः॑ । \newline
5. प्रसू᳚त्या॒ इति॒ प्र - सू॒त्यै॒ । \newline
6. बृह॒स्पति॑र् ब्र॒ह्मा ब्र॒ह्मा बृह॒स्पति॒र् बृह॒स्पति॑र् ब्र॒ह्मा । \newline
7. ब्र॒ह्मेतीति॑ ब्र॒ह्मा ब्र॒ह्मेति॑ । \newline
8. इत्या॑हा॒हे तीत्या॑ह । \newline
9. आ॒ह॒ स स आ॑हाह॒ सः । \newline
10. स हि हि स स हि । \newline
11. हि ब्रह्मि॑ष्ठो॒ ब्रह्मि॑ष्ठो॒ हि हि ब्रह्मि॑ष्ठः । \newline
12. ब्रह्मि॑ष्ठः॒ स स ब्रह्मि॑ष्ठो॒ ब्रह्मि॑ष्ठः॒ सः । \newline
13. स य॒ज्ञ्ं ॅय॒ज्ञ्ꣳ स स य॒ज्ञ्म् । \newline
14. य॒ज्ञ्म् पा॑हि पाहि य॒ज्ञ्ं ॅय॒ज्ञ्म् पा॑हि । \newline
15. पा॒हि॒ स स पा॑हि पाहि॒ सः । \newline
16. स य॒ज्ञ्प॑तिं ॅय॒ज्ञ्प॑तिꣳ॒॒ स स य॒ज्ञ्प॑तिम् । \newline
17. य॒ज्ञ्प॑तिम् पाहि पाहि य॒ज्ञ्प॑तिं ॅय॒ज्ञ्प॑तिम् पाहि । \newline
18. य॒ज्ञ्प॑ति॒मिति॑ य॒ज्ञ् - प॒ति॒म् । \newline
19. पा॒हि॒ स स पा॑हि पाहि॒ सः । \newline
20. स माम् माꣳ स स माम् । \newline
21. माम् पा॑हि पाहि॒ माम् माम् पा॑हि । \newline
22. पा॒हीतीति॑ पाहि पा॒हीति॑ । \newline
23. इत्या॑हा॒हे तीत्या॑ह । \newline
24. आ॒ह॒ य॒ज्ञाय॑ य॒ज्ञाया॑ हाह य॒ज्ञाय॑ । \newline
25. य॒ज्ञाय॒ यज॑मानाय॒ यज॑मानाय य॒ज्ञाय॑ य॒ज्ञाय॒ यज॑मानाय । \newline
26. यज॑मानाया॒ त्मन॑ आ॒त्मने॒ यज॑मानाय॒ यज॑मानाया॒ त्मने᳚ । \newline
27. आ॒त्मने॒ तेभ्य॒ स्तेभ्य॑ आ॒त्मन॑ आ॒त्मने॒ तेभ्यः॑ । \newline
28. तेभ्य॑ ए॒वैव तेभ्य॒ स्तेभ्य॑ ए॒व । \newline
29. ए॒वाशिष॑ मा॒शिष॑ मे॒वैवाशिष᳚म् । \newline
30. आ॒शिष॒ मा ऽऽशिष॑ मा॒शिष॒ मा । \newline
31. आ॒शिष॒मित्या᳚ - शिष᳚म् । \newline
32. आ शा᳚स्ते शास्त॒ आ शा᳚स्ते । \newline
33. शा॒स्ते ऽना᳚र्त्या॒ अना᳚र्त्यै शास्ते शा॒स्ते ऽना᳚र्त्यै । \newline
34. अना᳚र्त्या आ॒श्राव्या॒ श्राव्या ना᳚र्त्या॒ अना᳚र्त्या आ॒श्राव्य॑ । \newline
35. आ॒श्राव्या॑ हाहा॒ श्राव्या॒ श्राव्या॑ह । \newline
36. आ॒श्राव्येत्या᳚ - श्राव्य॑ । \newline
37. आ॒ह॒ दे॒वान् दे॒वा ना॑हाह दे॒वान् । \newline
38. दे॒वान्. य॑ज यज दे॒वान् दे॒वान्. य॑ज । \newline
39. य॒जे तीति॑ यज य॒जे ति॑ । \newline
40. इति॑ ब्रह्मवा॒दिनो᳚ ब्रह्मवा॒दिन॒ इतीति॑ ब्रह्मवा॒दिनः॑ । \newline
41. ब्र॒ह्म॒वा॒दिनो॑ वदन्ति वदन्ति ब्रह्मवा॒दिनो᳚ ब्रह्मवा॒दिनो॑ वदन्ति । \newline
42. ब्र॒ह्म॒वा॒दिन॒ इति॑ ब्रह्म - वा॒दिनः॑ । \newline
43. व॒द॒न्ती॒ष्टा इ॒ष्टा व॑दन्ति वदन्ती॒ष्टाः । \newline
44. इ॒ष्टा दे॒वता॑ दे॒वता॑ इ॒ष्टा इ॒ष्टा दे॒वताः᳚ । \newline
45. दे॒वता॒ अथाथ॑ दे॒वता॑ दे॒वता॒ अथ॑ । \newline
46. अथ॑ कत॒मे क॑त॒मे ऽथाथ॑ कत॒मे । \newline
47. क॒त॒म ए॒त ए॒ते क॑त॒मे क॑त॒म ए॒ते । \newline
48. ए॒ते दे॒वा दे॒वा ए॒त ए॒ते दे॒वाः । \newline
49. दे॒वा इतीति॑ दे॒वा दे॒वा इति॑ । \newline
50. इति॒ छन्दाꣳ॑सि॒ छन्दाꣳ॒॒ सीतीति॒ छन्दाꣳ॑सि । \newline
51. छन्दाꣳ॒॒ सीतीति॒ छन्दाꣳ॑सि॒ छन्दाꣳ॒॒ सीति॑ । \newline
52. इति॑ ब्रूयाद् ब्रूया॒ दितीति॑ ब्रूयात् । \newline
53. ब्रू॒या॒द् गा॒य॒त्रीम् गा॑य॒त्रीम् ब्रू॑याद् ब्रूयाद् गाय॒त्रीम् । \newline
54. गा॒य॒त्रीम् त्रि॒ष्टुभ॑म् त्रि॒ष्टुभ॑म् गाय॒त्रीम् गा॑य॒त्रीम् त्रि॒ष्टुभ᳚म् । \newline
55. त्रि॒ष्टुभ॒म् जग॑ती॒म् जग॑तीम् त्रि॒ष्टुभ॑म् त्रि॒ष्टुभ॒म् जग॑तीम् । \newline

\textbf{Ghana Paata } \newline

1. आ॒हे तीत्या॑हा॒हे त्या॑हा॒हे त्या॑हा॒हे त्या॑ह । \newline
2. इत्या॑हा॒हे तीत्या॑ह॒ प्रसू᳚त्यै॒ प्रसू᳚त्या आ॒हे तीत्या॑ह॒ प्रसू᳚त्यै । \newline
3. आ॒ह॒ प्रसू᳚त्यै॒ प्रसू᳚त्या आहाह॒ प्रसू᳚त्यै॒ बृह॒स्पति॒र् बृह॒स्पतिः॒ प्रसू᳚त्या आहाह॒ प्रसू᳚त्यै॒ बृह॒स्पतिः॑ । \newline
4. प्रसू᳚त्यै॒ बृह॒स्पति॒र् बृह॒स्पतिः॒ प्रसू᳚त्यै॒ प्रसू᳚त्यै॒ बृह॒स्पति॑र् ब्र॒ह्मा ब्र॒ह्मा बृह॒स्पतिः॒ प्रसू᳚त्यै॒ प्रसू᳚त्यै॒ बृह॒स्पति॑र् ब्र॒ह्मा । \newline
5. प्रसू᳚त्या॒ इति॒ प्र - सू॒त्यै॒ । \newline
6. बृह॒स्पति॑र् ब्र॒ह्मा ब्र॒ह्मा बृह॒स्पति॒र् बृह॒स्पति॑र् ब्र॒ह्मेतीति॑ ब्र॒ह्मा बृह॒स्पति॒र् बृह॒स्पति॑र् ब्र॒ह्मेति॑ । \newline
7. ब्र॒ह्मेतीति॑ ब्र॒ह्मा ब्र॒ह्मेत्या॑हा॒हे ति॑ ब्र॒ह्मा ब्र॒ह्मेत्या॑ह । \newline
8. इत्या॑हा॒हे तीत्या॑ह॒ स स आ॒हे तीत्या॑ह॒ सः । \newline
9. आ॒ह॒ स स आ॑हाह॒ स हि हि स आ॑हाह॒ स हि । \newline
10. स हि हि स स हि ब्रह्मि॑ष्ठो॒ ब्रह्मि॑ष्ठो॒ हि स स हि ब्रह्मि॑ष्ठः । \newline
11. हि ब्रह्मि॑ष्ठो॒ ब्रह्मि॑ष्ठो॒ हि हि ब्रह्मि॑ष्ठः॒ स स ब्रह्मि॑ष्ठो॒ हि हि ब्रह्मि॑ष्ठः॒ सः । \newline
12. ब्रह्मि॑ष्ठः॒ स स ब्रह्मि॑ष्ठो॒ ब्रह्मि॑ष्ठः॒ स य॒ज्ञ्ं ॅय॒ज्ञ्ꣳ स ब्रह्मि॑ष्ठो॒ ब्रह्मि॑ष्ठः॒ स य॒ज्ञ्म् । \newline
13. स य॒ज्ञ्ं ॅय॒ज्ञ्ꣳ स स य॒ज्ञ्म् पा॑हि पाहि य॒ज्ञ्ꣳ स स य॒ज्ञ्म् पा॑हि । \newline
14. य॒ज्ञ्म् पा॑हि पाहि य॒ज्ञ्ं ॅय॒ज्ञ्म् पा॑हि॒ स स पा॑हि य॒ज्ञ्ं ॅय॒ज्ञ्म् पा॑हि॒ सः । \newline
15. पा॒हि॒ स स पा॑हि पाहि॒ स य॒ज्ञ्प॑तिं ॅय॒ज्ञ्प॑तिꣳ॒॒ स पा॑हि पाहि॒ स य॒ज्ञ्प॑तिम् । \newline
16. स य॒ज्ञ्प॑तिं ॅय॒ज्ञ्प॑तिꣳ॒॒ स स य॒ज्ञ्प॑तिम् पाहि पाहि य॒ज्ञ्प॑तिꣳ॒॒ स स य॒ज्ञ्प॑तिम् पाहि । \newline
17. य॒ज्ञ्प॑तिम् पाहि पाहि य॒ज्ञ्प॑तिं ॅय॒ज्ञ्प॑तिम् पाहि॒ स स पा॑हि य॒ज्ञ्प॑तिं ॅय॒ज्ञ्प॑तिम् पाहि॒ सः । \newline
18. य॒ज्ञ्प॑ति॒मिति॑ य॒ज्ञ् - प॒ति॒म् । \newline
19. पा॒हि॒ स स पा॑हि पाहि॒ स माम् माꣳ स पा॑हि पाहि॒ स माम् । \newline
20. स माम् माꣳ स स माम् पा॑हि पाहि॒ माꣳ स स माम् पा॑हि । \newline
21. माम् पा॑हि पाहि॒ माम् माम् पा॒हीतीति॑ पाहि॒ माम् माम् पा॒हीति॑ । \newline
22. पा॒हीतीति॑ पाहि पा॒ही त्या॑हा॒हे ति॑ पाहि पा॒हीत्या॑ह । \newline
23. इत्या॑हा॒हे तीत्या॑ह य॒ज्ञाय॑ य॒ज्ञाया॒हे तीत्या॑ह य॒ज्ञाय॑ । \newline
24. आ॒ह॒ य॒ज्ञाय॑ य॒ज्ञाया॑ हाह य॒ज्ञाय॒ यज॑मानाय॒ यज॑मानाय य॒ज्ञाया॑ हाह य॒ज्ञाय॒ यज॑मानाय । \newline
25. य॒ज्ञाय॒ यज॑मानाय॒ यज॑मानाय य॒ज्ञाय॑ य॒ज्ञाय॒ यज॑मानाया॒ त्मन॑ आ॒त्मने॒ यज॑मानाय य॒ज्ञाय॑ य॒ज्ञाय॒ यज॑मानाया॒त्मने᳚ । \newline
26. यज॑मानाया॒ त्मन॑ आ॒त्मने॒ यज॑मानाय॒ यज॑मानाया॒ त्मने॒ तेभ्य॒ स्तेभ्य॑ आ॒त्मने॒ यज॑मानाय॒ यज॑मानाया॒ त्मने॒ तेभ्यः॑ । \newline
27. आ॒त्मने॒ तेभ्य॒ स्तेभ्य॑ आ॒त्मन॑ आ॒त्मने॒ तेभ्य॑ ए॒वैव तेभ्य॑ आ॒त्मन॑ आ॒त्मने॒ तेभ्य॑ ए॒व । \newline
28. तेभ्य॑ ए॒वैव तेभ्य॒ स्तेभ्य॑ ए॒वाशिष॑ मा॒शिष॑ मे॒व तेभ्य॒ स्तेभ्य॑ ए॒वाशिष᳚म् । \newline
29. ए॒वाशिष॑ मा॒शिष॑ मे॒वैवाशिष॒ मा ऽऽशिष॑ मे॒वैवाशिष॒ मा । \newline
30. आ॒शिष॒ मा ऽऽशिष॑ मा॒शिष॒ मा शा᳚स्ते शास्त॒ आ ऽऽशिष॑ मा॒शिष॒ मा शा᳚स्ते । \newline
31. आ॒शिष॒मित्या᳚ - शिष᳚म् । \newline
32. आ शा᳚स्ते शास्त॒ आ शा॒स्ते ऽना᳚र्त्या॒ अना᳚र्त्यै शास्त॒ आ शा॒स्ते ऽना᳚र्त्यै । \newline
33. शा॒स्ते ऽना᳚र्त्या॒ अना᳚र्त्यै शास्ते शा॒स्ते ऽना᳚र्त्या आ॒श्राव्या॒ श्राव्याना᳚र्त्यै शास्ते शा॒स्ते ऽना᳚र्त्या आ॒श्राव्य॑ । \newline
34. अना᳚र्त्या आ॒श्राव्या॒ श्राव्याना᳚र्त्या॒ अना᳚र्त्या आ॒श्राव्या॑ हाहा॒ श्राव्याना᳚र्त्या॒ अना᳚र्त्या आ॒श्राव्या॑ह । \newline
35. आ॒श्राव्या॑ हाहा॒ श्राव्या॒ श्राव्या॑ह दे॒वान् दे॒वा ना॑हा॒ श्राव्या॒ श्राव्या॑ह दे॒वान् । \newline
36. आ॒श्राव्येत्या᳚ - श्राव्य॑ । \newline
37. आ॒ह॒ दे॒वान् दे॒वा ना॑हाह दे॒वान्. य॑ज यज दे॒वा ना॑हाह दे॒वान्. य॑ज । \newline
38. दे॒वान्. य॑ज यज दे॒वान् दे॒वान्. य॒जे तीति॑ यज दे॒वान् दे॒वान्. य॒जे ति॑ । \newline
39. य॒जे तीति॑ यज य॒जे ति॑ ब्रह्मवा॒दिनो᳚ ब्रह्मवा॒दिन॒ इति॑ यज य॒जे ति॑ ब्रह्मवा॒दिनः॑ । \newline
40. इति॑ ब्रह्मवा॒दिनो᳚ ब्रह्मवा॒दिन॒ इतीति॑ ब्रह्मवा॒दिनो॑ वदन्ति वदन्ति ब्रह्मवा॒दिन॒ इतीति॑ ब्रह्मवा॒दिनो॑ वदन्ति । \newline
41. ब्र॒ह्म॒वा॒दिनो॑ वदन्ति वदन्ति ब्रह्मवा॒दिनो᳚ ब्रह्मवा॒दिनो॑ वदन्ती॒ष्टा इ॒ष्टा व॑दन्ति ब्रह्मवा॒दिनो᳚ ब्रह्मवा॒दिनो॑ वदन्ती॒ष्टाः । \newline
42. ब्र॒ह्म॒वा॒दिन॒ इति॑ ब्रह्म - वा॒दिनः॑ । \newline
43. व॒द॒न्ती॒ष्टा इ॒ष्टा व॑दन्ति वदन्ती॒ष्टा दे॒वता॑ दे॒वता॑ इ॒ष्टा व॑दन्ति वदन्ती॒ष्टा दे॒वताः᳚ । \newline
44. इ॒ष्टा दे॒वता॑ दे॒वता॑ इ॒ष्टा इ॒ष्टा दे॒वता॒ अथाथ॑ दे॒वता॑ इ॒ष्टा इ॒ष्टा दे॒वता॒ अथ॑ । \newline
45. दे॒वता॒ अथाथ॑ दे॒वता॑ दे॒वता॒ अथ॑ कत॒मे क॑त॒मे ऽथ॑ दे॒वता॑ दे॒वता॒ अथ॑ कत॒मे । \newline
46. अथ॑ कत॒मे क॑त॒मे ऽथाथ॑ कत॒म ए॒त ए॒ते क॑त॒मे ऽथाथ॑ कत॒म ए॒ते । \newline
47. क॒त॒म ए॒त ए॒ते क॑त॒मे क॑त॒म ए॒ते दे॒वा दे॒वा ए॒ते क॑त॒मे क॑त॒म ए॒ते दे॒वाः । \newline
48. ए॒ते दे॒वा दे॒वा ए॒त ए॒ते दे॒वा इतीति॑ दे॒वा ए॒त ए॒ते दे॒वा इति॑ । \newline
49. दे॒वा इतीति॑ दे॒वा दे॒वा इति॒ छन्दाꣳ॑सि॒ छन्दाꣳ॒॒सीति॑ दे॒वा दे॒वा इति॒ छन्दाꣳ॑सि । \newline
50. इति॒ छन्दाꣳ॑सि॒ छन्दाꣳ॒॒सीतीति॒ छन्दाꣳ॒॒सीतीति॒ छन्दाꣳ॒॒सीतीति॒ छन्दाꣳ॒॒सीति॑ । \newline
51. छन्दाꣳ॒॒सीतीति॒ छन्दाꣳ॑सि॒ छन्दाꣳ॒॒सीति॑ ब्रूयाद् ब्रूया॒दिति॒ छन्दाꣳ॑सि॒ छन्दाꣳ॒॒सीति॑ ब्रूयात् । \newline
52. इति॑ ब्रूयाद् ब्रूया॒ दितीति॑ ब्रूयाद् गाय॒त्रीम् गा॑य॒त्रीम् ब्रू॑या॒ दितीति॑ ब्रूयाद् गाय॒त्रीम् । \newline
53. ब्रू॒या॒द् गा॒य॒त्रीम् गा॑य॒त्रीम् ब्रू॑याद् ब्रूयाद् गाय॒त्रीम् त्रि॒ष्टुभ॑म् त्रि॒ष्टुभ॑म् गाय॒त्रीम् ब्रू॑याद् ब्रूयाद् गाय॒त्रीम् त्रि॒ष्टुभ᳚म् । \newline
54. गा॒य॒त्रीम् त्रि॒ष्टुभ॑म् त्रि॒ष्टुभ॑म् गाय॒त्रीम् गा॑य॒त्रीम् त्रि॒ष्टुभ॒म् जग॑ती॒म् जग॑तीम् त्रि॒ष्टुभ॑म् गाय॒त्रीम् गा॑य॒त्रीम् त्रि॒ष्टुभ॒म् जग॑तीम् । \newline
55. त्रि॒ष्टुभ॒म् जग॑ती॒म् जग॑तीम् त्रि॒ष्टुभ॑म् त्रि॒ष्टुभ॒म् जग॑ती॒ मितीति॒ जग॑तीम् त्रि॒ष्टुभ॑म् त्रि॒ष्टुभ॒म् जग॑ती॒ मिति॑ । \newline
\pagebreak
\markright{ TS 2.6.9.4  \hfill https://www.vedavms.in \hfill}
\addcontentsline{toc}{section}{ TS 2.6.9.4 }
\section*{ TS 2.6.9.4 }

\textbf{TS 2.6.9.4 } \newline
\textbf{Samhita Paata} \newline

जग॑ती॒मित्यथो॒ खल्वा॑हुर्ब्राह्म॒णा वै छन्दाꣳ॒॒सीति॒ ताने॒व तद्-य॑जति दे॒वानां॒ ॅवा इ॒ष्टा दे॒वता॒ आस॒न्नथा॒ग्निर्नोद॑ज्वल॒त् तं दे॒वा आहु॑तीभि-रनूया॒जेष्वन्व॑-विन्द॒न्॒. यद॑नूया॒जान्. यज॑त्य॒ग्निमे॒व तथ् समि॑न्ध ए॒तदु॒र्वै नामा॑ऽऽ*सु॒र आ॑सी॒थ् स ए॒तर्.हि॑ य॒ज्ञ्स्या॒ ऽऽशिष॑मवृङ्क्त॒ यद् ब्रू॒यादे॒त - [  ] \newline

\textbf{Pada Paata} \newline

जग॑तीम् । इति॑ । अथो॒ इति॑ । खलु॑ । आ॒हुः॒ । ब्रा॒ह्म॒णाः । वै । छन्दाꣳ॑सि । इति॑ । तान् । ए॒व । तत् । य॒ज॒ति॒ । दे॒वाना᳚म् । वै । इ॒ष्टाः । दे॒वताः᳚ । आसन्न्॑ । अथ॑ । अ॒ग्निः । न । उदिति॑ । अ॒ज्व॒ल॒त् । तम् । दे॒वाः । आहु॑तीभि॒रित्याहु॑ति - भिः॒ । अ॒नू॒या॒जेष्वित्य॑नु - या॒जेषु॑ । अन्विति॑ । अ॒वि॒न्द॒न्न् । यत् । अ॒नू॒या॒जानित्य॑नु-या॒जान् । यज॑ति । अ॒ग्निम् । ए॒व । तत् । समिति॑ । इ॒न्धे॒ । ए॒तदुः॑ । वै । नाम॑ । आ॒सु॒रः । आ॒सी॒त् । सः । ए॒तर्.हि॑ । य॒ज्ञ्स्य॑ । आ॒शिष॒मित्या᳚ - शिष᳚म् । अ॒वृ॒ङ्क्त॒ । यत् । ब्रू॒यात् । ए॒तत् ।  \newline


\textbf{Krama Paata} \newline

जग॑ती॒मिति॑ । इत्यथो᳚ । अथो॒ खलु॑ । अथो॒ इत्यथो᳚ । खल्वा॑हुः । आ॒हु॒र् ब्रा॒ह्म॒णाः । ब्रा॒ह्म॒णा वै । वै छन्दाꣳ॑सि । छन्दाꣳ॒॒सीति॑ । इति॒ तान् । ताने॒व । ए॒व तत् । तद् य॑जति । य॒ज॒ति॒ दे॒वाना᳚म् । दे॒वानां॒ ॅवै । वा इ॒ष्टाः । इ॒ष्टा दे॒वताः᳚ । दे॒वता॒ आसन्न्॑ । आस॒न्नथ॑ । अथा॒ग्निः । अ॒ग्निर् न । नोत् । उद॑ज्वलत् । अ॒ज्व॒ल॒त् तम् । तम् दे॒वाः । दे॒वा आहु॑तीभिः । आहु॑तीभिरनूया॒जेषु॑ । आहु॑तीभि॒रित्याहु॑ति - भिः॒ । अ॒नू॒या॒जेष्वनु॑ । अ॒नू॒या॒जेष्वित्य॑नु - या॒जेषु॑ । अन्व॑विन्दन्न् । अ॒वि॒न्द॒न्न्॒. यत् । यद॑नूया॒जान् । अ॒नू॒या॒जान्. यज॑ति । अ॒नू॒या॒जानित्य॑नु - या॒जान् । यज॑त्य॒ग्निम् । अ॒ग्निमे॒व । ए॒व तत् । तथ् सम् । समि॑न्धे । इ॒न्ध॒ ए॒तदुः॑ । ए॒तदु॒र् वै । वै नाम॑ । नामा॑सु॒रः । आ॒सु॒र आ॑सीत् । आ॒सी॒थ् सः । स ए॒तर्.हि॑ । ए॒तर्.हि॑ य॒ज्ञ्स्य॑ । य॒ज्ञ्स्या॒शिष᳚म् । आ॒शिष॑मवृङ्क्त । आ॒शिष॒मित्या᳚ - शिष᳚म् । अ॒वृ॒ङ्क्त॒ यत् । यद् ब्रू॒यात् । ब्रू॒यादे॒तत् । ए॒तदु॑ \newline

\textbf{Jatai Paata} \newline

1. जग॑ती॒ मितीति॒ जग॑ती॒म् जग॑ती॒ मिति॑ । \newline
2. इत्यथो॒ अथो॒ इतीत्यथो᳚ । \newline
3. अथो॒ खलु॒ खल्वथो॒ अथो॒ खलु॑ । \newline
4. अथो॒ इत्यथो᳚ । \newline
5. खल्वा॑हु राहुः॒ खलु॒ खल्वा॑हुः । \newline
6. आ॒हु॒र् ब्रा॒ह्म॒णा ब्रा᳚ह्म॒णा आ॑हु राहुर् ब्राह्म॒णाः । \newline
7. ब्रा॒ह्म॒णा वै वै ब्रा᳚ह्म॒णा ब्रा᳚ह्म॒णा वै । \newline
8. वै छन्दाꣳ॑सि॒ छन्दाꣳ॑सि॒ वै वै छन्दाꣳ॑सि । \newline
9. छन्दाꣳ॒॒ सीतीति॒ छन्दाꣳ॑सि॒ छन्दाꣳ॒॒ सीति॑ । \newline
10. इति॒ ताꣳ स्ता नितीति॒ तान् । \newline
11. ता ने॒वैव ताꣳ स्ता ने॒व । \newline
12. ए॒व तत् तदे॒वैव तत् । \newline
13. तद् य॑जति यजति॒ तत् तद् य॑जति । \newline
14. य॒ज॒ति॒ दे॒वाना᳚म् दे॒वानां᳚ ॅयजति यजति दे॒वाना᳚म् । \newline
15. दे॒वानां॒ ॅवै वै दे॒वाना᳚म् दे॒वानां॒ ॅवै । \newline
16. वा इ॒ष्टा इ॒ष्टा वै वा इ॒ष्टाः । \newline
17. इ॒ष्टा दे॒वता॑ दे॒वता॑ इ॒ष्टा इ॒ष्टा दे॒वताः᳚ । \newline
18. दे॒वता॒ आस॒न् नास॑न् दे॒वता॑ दे॒वता॒ आसन्न्॑ । \newline
19. आस॒न् नथाथा स॒न् नास॒न् नथ॑ । \newline
20. अथा॒ग्नि र॒ग्नि रथाथा॒ग्निः । \newline
21. अ॒ग्निर् न नाग्नि र॒ग्निर् न । \newline
22. नोदुन् न नोत् । \newline
23. उद॑ज्वल दज्वल॒ दुदु द॑ज्वलत् । \newline
24. अ॒ज्व॒ल॒त् तम् त म॑ज्वल दज्वल॒त् तम् । \newline
25. तम् दे॒वा दे॒वा स्तम् तम् दे॒वाः । \newline
26. दे॒वा आहु॑तीभि॒ राहु॑तीभिर् दे॒वा दे॒वा आहु॑तीभिः । \newline
27. आहु॑तीभि रनूया॒जे ष्व॑नूया॒जे ष्वाहु॑तीभि॒ राहु॑तीभि रनूया॒जेषु॑ । \newline
28. आहु॑तीभि॒रित्याहु॑ति - भिः॒ । \newline
29. अ॒नू॒या॒जे ष्वन्वन्व॑ नूया॒जे ष्व॑नूया॒जे ष्वनु॑ । \newline
30. अ॒नू॒या॒जेष्वित्य॑नु - या॒जेषु॑ । \newline
31. अन्व॑विन्दन् नविन्द॒न् नन्व न्व॑विन्दन्न् । \newline
32. अ॒वि॒न्द॒न्॒. यद् यद॑विन्दन् नविन्द॒न्॒. यत् । \newline
33. यद॑नूया॒जा न॑नूया॒जान्. यद् यद॑नूया॒जान् । \newline
34. अ॒नू॒या॒जान्. यज॑ति॒ यज॑ त्यनूया॒जा न॑नूया॒जान्. यज॑ति । \newline
35. अ॒नू॒या॒जानित्य॑नु - या॒जान् । \newline
36. यज॑ त्य॒ग्नि म॒ग्निं ॅयज॑ति॒ यज॑ त्य॒ग्निम् । \newline
37. अ॒ग्नि मे॒वैवाग्नि म॒ग्नि मे॒व । \newline
38. ए॒व तत् तदे॒वैव तत् । \newline
39. तथ् सꣳ सम् तत् तथ् सम् । \newline
40. स मि॑न्ध इन्धे॒ सꣳ स मि॑न्धे । \newline
41. इ॒न्ध॒ ए॒तदु॑ रे॒तदु॑ रिन्ध इन्ध ए॒तदुः॑ । \newline
42. ए॒तदु॒र् वै वा ए॒तदु॑ रे॒तदु॒र् वै । \newline
43. वै नाम॒ नाम॒ वै वै नाम॑ । \newline
44. नामा॑सु॒र आ॑सु॒रो नाम॒ नामा॑सु॒रः । \newline
45. आ॒सु॒र आ॑सी दासी दासु॒र आ॑सु॒र आ॑सीत् । \newline
46. आ॒सी॒थ् स स आ॑सी दासी॒थ् सः । \newline
47. स ए॒तर् ह्ये॒तर्.हि॒ स स ए॒तर्.हि॑ । \newline
48. ए॒तर्.हि॑ य॒ज्ञ्स्य॑ य॒ज्ञ् स्यै॒तर् ह्ये॒तर्.हि॑ य॒ज्ञ्स्य॑ । \newline
49. य॒ज्ञ्स्या॒शिष॑ मा॒शिषं॑ ॅय॒ज्ञ्स्य॑ य॒ज्ञ्स्या॒शिष᳚म् । \newline
50. आ॒शिष॑ मवृङ्क्ता वृङ्क्ता॒शिष॑ मा॒शिष॑ मवृङ्क्त । \newline
51. आ॒शिष॒मित्या᳚ - शिष᳚म् । \newline
52. अ॒वृ॒ङ्क्त॒ यद् यद॑वृङ्क्ता वृङ्क्त॒ यत् । \newline
53. यद् ब्रू॒याद् ब्रू॒याद् यद् यद् ब्रू॒यात् । \newline
54. ब्रू॒या दे॒त दे॒तद् ब्रू॒याद् ब्रू॒या दे॒तत् । \newline
55. ए॒तदु॑ वु वे॒त दे॒तदु॑ । \newline

\textbf{Ghana Paata } \newline

1. जग॑ती॒ मितीति॒ जग॑ती॒म् जग॑ती॒ मित्यथो॒ अथो॒ इति॒ जग॑ती॒म् जग॑ती॒ मित्यथो᳚ । \newline
2. इत्यथो॒ अथो॒ इतीत्यथो॒ खलु॒ खल्वथो॒ इतीत्यथो॒ खलु॑ । \newline
3. अथो॒ खलु॒ खल्वथो॒ अथो॒ खल्वा॑हु राहुः॒ खल्वथो॒ अथो॒ खल्वा॑हुः । \newline
4. अथो॒ इत्यथो᳚ । \newline
5. खल्वा॑हु राहुः॒ खलु॒ खल्वा॑हुर् ब्राह्म॒णा ब्रा᳚ह्म॒णा आ॑हुः॒ खलु॒ खल्वा॑हुर् ब्राह्म॒णाः । \newline
6. आ॒हु॒र् ब्रा॒ह्म॒णा ब्रा᳚ह्म॒णा आ॑हु राहुर् ब्राह्म॒णा वै वै ब्रा᳚ह्म॒णा आ॑हु राहुर् ब्राह्म॒णा वै । \newline
7. ब्रा॒ह्म॒णा वै वै ब्रा᳚ह्म॒णा ब्रा᳚ह्म॒णा वै छन्दाꣳ॑सि॒ छन्दाꣳ॑सि॒ वै ब्रा᳚ह्म॒णा ब्रा᳚ह्म॒णा वै छन्दाꣳ॑सि । \newline
8. वै छन्दाꣳ॑सि॒ छन्दाꣳ॑सि॒ वै वै छन्दाꣳ॒॒सीतीति॒ छन्दाꣳ॑सि॒ वै वै छन्दाꣳ॒॒सीति॑ । \newline
9. छन्दाꣳ॒॒सीतीति॒ छन्दाꣳ॑सि॒ छन्दाꣳ॒॒सीति॒ ताꣳ स्ता निति॒ छन्दाꣳ॑सि॒ छन्दाꣳ॒॒सीति॒ तान् । \newline
10. इति॒ ताꣳ स्ता नितीति॒ ता ने॒वैव ता नितीति॒ ता ने॒व । \newline
11. ता ने॒वैव ताꣳ स्ता ने॒व तत् तदे॒व ताꣳ स्ता ने॒व तत् । \newline
12. ए॒व तत् तदे॒वैव तद् य॑जति यजति॒ तदे॒वैव तद् य॑जति । \newline
13. तद् य॑जति यजति॒ तत् तद् य॑जति दे॒वाना᳚म् दे॒वानां᳚ ॅयजति॒ तत् तद् य॑जति दे॒वाना᳚म् । \newline
14. य॒ज॒ति॒ दे॒वाना᳚म् दे॒वानां᳚ ॅयजति यजति दे॒वानां॒ ॅवै वै दे॒वानां᳚ ॅयजति यजति दे॒वानां॒ ॅवै । \newline
15. दे॒वानां॒ ॅवै वै दे॒वाना᳚म् दे॒वानां॒ ॅवा इ॒ष्टा इ॒ष्टा वै दे॒वाना᳚म् दे॒वानां॒ ॅवा इ॒ष्टाः । \newline
16. वा इ॒ष्टा इ॒ष्टा वै वा इ॒ष्टा दे॒वता॑ दे॒वता॑ इ॒ष्टा वै वा इ॒ष्टा दे॒वताः᳚ । \newline
17. इ॒ष्टा दे॒वता॑ दे॒वता॑ इ॒ष्टा इ॒ष्टा दे॒वता॒ आस॒न् नास॑न् दे॒वता॑ इ॒ष्टा इ॒ष्टा दे॒वता॒ आसन्न्॑ । \newline
18. दे॒वता॒ आस॒न् नास॑न् दे॒वता॑ दे॒वता॒ आस॒न् नथाथास॑न् दे॒वता॑ दे॒वता॒ आस॒न् नथ॑ । \newline
19. आस॒न् नथाथास॒न् नास॒न् नथा॒ग्नि र॒ग्नि रथास॒न् नास॒न् नथा॒ग्निः । \newline
20. अथा॒ग्नि र॒ग्नि रथाथा॒ग्निर् न नाग्नि रथाथा॒ग्निर् न । \newline
21. अ॒ग्निर् न नाग्नि र॒ग्निर् नोदुन् नाग्नि र॒ग्निर् नोत् । \newline
22. नोदुन् न नोद॑ज्वल दज्वल॒ दुन् न नोद॑ज्वलत् । \newline
23. उद॑ज्वल दज्वल॒ दुदु द॑ज्वल॒त् तम् त म॑ज्वल॒ दुदु द॑ज्वल॒त् तम् । \newline
24. अ॒ज्व॒ल॒त् तम् त म॑ज्वल दज्वल॒त् तम् दे॒वा दे॒वा स्त म॑ज्वल दज्वल॒त् तम् दे॒वाः । \newline
25. तम् दे॒वा दे॒वा स्तम् तम् दे॒वा आहु॑तीभि॒ राहु॑तीभिर् दे॒वा स्तम् तम् दे॒वा आहु॑तीभिः । \newline
26. दे॒वा आहु॑तीभि॒ राहु॑तीभिर् दे॒वा दे॒वा आहु॑तीभि रनूया॒जे ष्व॑नूया॒जे ष्वाहु॑तीभिर् दे॒वा दे॒वा आहु॑तीभि रनूया॒जेषु॑ । \newline
27. आहु॑तीभि रनूया॒जे ष्व॑नूया॒जे ष्वाहु॑तीभि॒ राहु॑तीभि रनूया॒जे ष्वन्वन्व॑नूया॒जे ष्वाहु॑तीभि॒ राहु॑तीभि रनूया॒जेष्वनु॑ । \newline
28. आहु॑तीभि॒रित्याहु॑ति - भिः॒ । \newline
29. अ॒नू॒या॒जे ष्वन्वन्व॑नूया॒जे ष्व॑नूया॒जे ष्वन्व॑विन्दन् नविन्द॒न् नन्व॑नूया॒जे ष्व॑नूया॒जे ष्वन्व॑विन्दन्न् । \newline
30. अ॒नू॒या॒जेष्वित्य॑नु - या॒जेषु॑ । \newline
31. अन्व॑विन्दन् नविन्द॒न् नन्वन्व॑ विन्द॒न्॒. यद् यद॑विन्द॒न् नन्वन्व॑ विन्द॒न्॒. यत् । \newline
32. अ॒वि॒न्द॒न्॒. यद् यद॑विन्दन् नविन्द॒न्॒. यद॑नूया॒जा न॑नूया॒जान्. यद॑विन्दन् नविन्द॒न्॒. यद॑नूया॒जान् । \newline
33. यद॑नूया॒जा न॑नूया॒जान्. यद् यद॑नूया॒जान्. यज॑ति॒ यज॑ त्यनूया॒जान्. यद् यद॑नूया॒जान्. यज॑ति । \newline
34. अ॒नू॒या॒जान्. यज॑ति॒ यज॑ त्यनूया॒जा न॑नूया॒जान्. यज॑त्य॒ग्नि म॒ग्निं ॅयज॑ त्यनूया॒जा न॑नूया॒जान्. यज॑त्य॒ग्निम् । \newline
35. अ॒नू॒या॒जानित्य॑नु - या॒जान् । \newline
36. यज॑त्य॒ग्नि म॒ग्निं ॅयज॑ति॒ यज॑त्य॒ग्नि मे॒वैवाग्निं ॅयज॑ति॒ यज॑त्य॒ग्नि मे॒व । \newline
37. अ॒ग्नि मे॒वैवाग्नि म॒ग्नि मे॒व तत् तदे॒वाग्नि म॒ग्नि मे॒व तत् । \newline
38. ए॒व तत् तदे॒वैव तथ् सꣳ सम् तदे॒वैव तथ् सम् । \newline
39. तथ् सꣳ सम् तत् तथ् स मि॑न्ध इन्धे॒ सम् तत् तथ् स मि॑न्धे । \newline
40. स मि॑न्ध इन्धे॒ सꣳ स मि॑न्ध ए॒तदु॑ रे॒तदु॑ रिन्धे॒ सꣳ स मि॑न्ध ए॒तदुः॑ । \newline
41. इ॒न्ध॒ ए॒तदु॑ रे॒तदु॑ रिन्ध इन्ध ए॒तदु॒र् वै वा ए॒तदु॑ रिन्ध इन्ध ए॒तदु॒र् वै । \newline
42. ए॒तदु॒र् वै वा ए॒तदु॑ रे॒तदु॒र् वै नाम॒ नाम॒ वा ए॒तदु॑ रे॒तदु॒र् वै नाम॑ । \newline
43. वै नाम॒ नाम॒ वै वै नामा॑सु॒र आ॑सु॒रो नाम॒ वै वै नामा॑सु॒रः । \newline
44. नामा॑सु॒र आ॑सु॒रो नाम॒ नामा॑सु॒र आ॑सी दासी दासु॒रो नाम॒ नामा॑सु॒र आ॑सीत् । \newline
45. आ॒सु॒र आ॑सी दासी दासु॒र आ॑सु॒र आ॑सी॒थ् स स आ॑सी दासु॒र आ॑सु॒र आ॑सी॒थ् सः । \newline
46. आ॒सी॒थ् स स आ॑सी दासी॒थ् स ए॒तर् ह्ये॒तर्.हि॒ स आ॑सी दासी॒थ् स ए॒तर्.हि॑ । \newline
47. स ए॒तर् ह्ये॒तर्.हि॒ स स ए॒तर्.हि॑ य॒ज्ञ्स्य॑ य॒ज्ञ्स्यै॒तर्.हि॒ स स ए॒तर्.हि॑ य॒ज्ञ्स्य॑ । \newline
48. ए॒तर्.हि॑ य॒ज्ञ्स्य॑ य॒ज्ञ्स्यै॒तर् ह्ये॒तर्.हि॑ य॒ज्ञ्स्या॒शिष॑ मा॒शिषं॑ ॅय॒ज्ञ्स्यै॒तर् ह्ये॒तर्.हि॑ य॒ज्ञ्स्या॒शिष᳚म् । \newline
49. य॒ज्ञ्स्या॒शिष॑ मा॒शिषं॑ ॅय॒ज्ञ्स्य॑ य॒ज्ञ्स्या॒शिष॑ मवृङ्क्ता वृङ्क्ता॒शिषं॑ ॅय॒ज्ञ्स्य॑ य॒ज्ञ्स्या॒शिष॑ मवृङ्क्त । \newline
50. आ॒शिष॑ मवृङ्क्ता वृङ्क्ता॒शिष॑ मा॒शिष॑ मवृङ्क्त॒ यद् यद॑वृङ्क्ता॒ शिष॑ मा॒शिष॑ मवृङ्क्त॒ यत् । \newline
51. आ॒शिष॒मित्या᳚ - शिष᳚म् । \newline
52. अ॒वृ॒ङ्क्त॒ यद् यद॑वृङ्क्ता वृङ्क्त॒ यद् ब्रू॒याद् ब्रू॒याद् यद॑वृङ्क्ता वृङ्क्त॒ यद् ब्रू॒यात् । \newline
53. यद् ब्रू॒याद् ब्रू॒याद् यद् यद् ब्रू॒या दे॒त दे॒तद् ब्रू॒याद् यद् यद् ब्रू॒या दे॒तत् । \newline
54. ब्रू॒या दे॒त दे॒तद् ब्रू॒याद् ब्रू॒या दे॒तदु॑ वु वे॒तद् ब्रू॒याद् ब्रू॒या दे॒तदु॑ । \newline
55. ए॒तदु॑ वु वे॒त दे॒तदु॑ द्यावापृथिवी द्यावापृथिवी उ वे॒त दे॒तदु॑ द्यावापृथिवी । \newline
\pagebreak
\markright{ TS 2.6.9.5  \hfill https://www.vedavms.in \hfill}
\addcontentsline{toc}{section}{ TS 2.6.9.5 }
\section*{ TS 2.6.9.5 }

\textbf{TS 2.6.9.5 } \newline
\textbf{Samhita Paata} \newline

-दु॑ द्यावापृथिवी भ॒द्र-म॑भू॒दित्ये॒तदु॑-मे॒वाऽऽसु॒रं ॅय॒ज्ञ्स्या॒ऽऽशिषं॑ गमयेदि॒दं द्या॑वापृथिवी भ॒द्रम॑भू॒दित्ये॒व ब्रू॑या॒द्-यज॑मानमे॒व य॒ज्ञ्स्या॒ऽऽशिषं॑ गमय॒त्यार्द्ध्म॑ सूक्तवा॒कमु॒त न॑मोवा॒कमि-त्या॑हे॒दम॑रा॒थ्-स्मेति॒ वावैतदा॒होप॑श्रितो दि॒वः पृ॑थि॒व्योरित्या॑ह॒ द्यावा॑पृथि॒व्योर्.हि य॒ज्ञ् उप॑श्रित॒ ओम॑न्वती ते॒ऽस्मिन्. य॒ज्ञे य॑जमान॒ द्यावा॑पृथि॒वी - [  ] \newline

\textbf{Pada Paata} \newline

उ॒ । द्या॒वा॒पृ॒थि॒वी॒ इति॑ द्यावा - पृ॒थि॒वी॒ । भ॒द्रम् । अ॒भू॒त् । इति॑ । ए॒तदु᳚म् । ए॒व । आ॒सु॒रम् । य॒ज्ञ्स्य॑ । आ॒शिष॒मित्या᳚ - शिष᳚म् । ग॒म॒ये॒त् । इ॒दम् । द्या॒वा॒पृ॒थि॒वी॒ इति॑ द्यावा - पृ॒थि॒वी॒ । भ॒द्रम् । अ॒भू॒त् । इति॑ । ए॒व । ब्रू॒या॒त् । यज॑मानम् । ए॒व । य॒ज्ञ्स्य॑ । आ॒शिष॒मित्या᳚- शिष᳚म् । ग॒म॒य॒ति॒ । आर्द्ध्म॑ । सू॒क्त॒वा॒कमिति॑ सूक्त - वा॒कम् । उ॒त । न॒मो॒वा॒कमिति॑ नमः-वा॒कम् । इति॑ । आ॒ह॒ । इ॒दम् । अ॒रा॒थ्स्म॒ । इति॑ । वाव । ए॒तत् । आ॒ह॒ । उप॑श्रित॒ इत्युप॑ - श्रि॒तः॒ । दि॒वः । पृ॒थि॒व्योः । इति॑ । आ॒ह॒ । द्यावा॑पृथि॒व्योरिति॒ द्यावा᳚ - पृ॒थि॒व्योः । हि । य॒ज्ञ्ः । उप॑श्रित॒ इत्युप॑ - श्रि॒तः॒ । ओम॑न्वती॒ इत्योमन्न्॑ - व॒ती॒ । ते॒ । अ॒स्मिन्न् । य॒ज्ञे । य॒ज॒मा॒न॒ । द्यावा॑पृथि॒वी इति॒ द्यावा᳚ - पृ॒थि॒वी ।  \newline


\textbf{Krama Paata} \newline

उ॒ द्या॒वा॒पृ॒थि॒वी॒ । द्या॒वा॒पृ॒थि॒वी॒ भ॒द्रम् । द्या॒वा॒पृ॒थि॒वी॒ इति॑ द्यावा - पृ॒थि॒वी॒ । भ॒द्रम॑भूत् । अ॒भू॒दिति॑ । इत्ये॒तदु᳚म् । ए॒तदु॑मे॒व । ए॒वासु॒रम् । आ॒सु॒रं ॅय॒ज्ञ्स्य॑ । य॒ज्ञ्स्या॒शिष᳚म् । आ॒शिष॑म् गमयेत् । आ॒शिष॒मित्या᳚ - शिष᳚म् । ग॒म॒ये॒दि॒दम् । इ॒दम् द्या॑वापृथिवी । द्या॒वा॒पृ॒थि॒वी॒ भ॒द्रम् । द्या॒वा॒पृ॒थि॒वी॒ इति॑ द्यावा - पृ॒थि॒वी॒ । भ॒द्रम॑भूत् । अ॒भू॒दिति॑ । इत्ये॒व । ए॒व ब्रू॑यात् । ब्रू॒या॒द् यज॑मानम् । यज॑मानमे॒व । ए॒व य॒ज्ञ्स्य॑ । य॒ज्ञ्स्या॒शिष᳚म् । आ॒शिष॑म् गमयति । आ॒शिष॒मित्या᳚ - शिष᳚म् । ग॒म॒य॒त्यार्द्ध्म॑ । आर्द्ध्म॑ सूक्तवा॒कम् । सू॒क्त॒वा॒कमु॒त । सू॒क्त॒वा॒कमिति॑ सूक्त - वा॒कम् । उ॒त न॑मोवा॒कम् । न॒मो॒वा॒कमिति॑ । न॒मो॒वा॒कमिति॑ नमः - वा॒कम् । इत्या॑ह । आ॒हे॒दम् । इ॒दम॑राथ्स्म । अ॒रा॒थ्स्मेति॑ । इति॒ वाव । वावैतत् । ए॒तदा॑ह । आ॒होप॑श्रितः । उप॑श्रितो दि॒वः । उप॑श्रित॒ इत्युप॑ - श्रि॒तः॒ । दि॒वः पृ॑थि॒व्योः । पृ॒थि॒व्योरिति॑ । इत्या॑ह । आ॒ह॒ द्यावा॑पृथि॒व्योः । द्यावा॑पृथि॒व्योर्. हि । द्यावा॑पृथि॒व्योरिति॒ द्यावा᳚ - पृ॒थि॒व्योः । हि य॒ज्ञ्ः । य॒ज्ञ् उप॑श्रितः । उप॑श्रित॒ ओम॑न्वती । उप॑श्रित॒ इत्युप॑ - श्रि॒तः॒ । ओम॑न्वती ते । ओम॑न्वती॒ इत्योमन्न्॑ - व॒ती॒ । ते॒ऽस्मिन्न् । अ॒स्मिन्न्. य॒ज्ञे । य॒ज्ञे य॑जमान । य॒ज॒मा॒न॒ द्यावा॑पृथि॒वी । द्यावा॑पृथि॒वी स्ता᳚म् । द्यावा॑पृथि॒वी इति॒ द्यावा᳚ - पृ॒थि॒वी \newline

\textbf{Jatai Paata} \newline

1. उ॒ द्या॒वा॒पृ॒थि॒वी॒ द्या॒वा॒पृ॒थि॒वी॒ उ॒ वु॒ द्या॒वा॒पृ॒थि॒वी॒ । \newline
2. द्या॒वा॒पृ॒थि॒वी॒ भ॒द्रम् भ॒द्रम् द्या॑वापृथिवी द्यावापृथिवी भ॒द्रम् । \newline
3. द्या॒वा॒पृ॒थि॒वी॒ इति॑ द्यावा - पृ॒थि॒वी॒ । \newline
4. भ॒द्र म॑भू दभूद् भ॒द्रम् भ॒द्र म॑भूत् । \newline
5. अ॒भू॒ दिती त्य॑भू दभू॒ दिति॑ । \newline
6. इत्ये॒तदु॑ मे॒तदु॒ मिती त्ये॒तदु᳚म् । \newline
7. ए॒तदु॑ मे॒वैवैतदु॑ मे॒तदु॑ मे॒व । \newline
8. ए॒वासु॒र मा॑सु॒र मे॒वैवासु॒रम् । \newline
9. आ॒सु॒रं ॅय॒ज्ञ्स्य॑ य॒ज्ञ्स्या॑सु॒र मा॑सु॒रं ॅय॒ज्ञ्स्य॑ । \newline
10. य॒ज्ञ्स्या॒शिष॑ मा॒शिषं॑ ॅय॒ज्ञ्स्य॑ य॒ज्ञ्स्या॒शिष᳚म् । \newline
11. आ॒शिष॑म् गमयेद् गमये दा॒शिष॑ मा॒शिष॑म् गमयेत् । \newline
12. आ॒शिष॒मित्या᳚ - शिष᳚म् । \newline
13. ग॒म॒ये॒ दि॒द मि॒दम् ग॑मयेद् गमये दि॒दम् । \newline
14. इ॒दम् द्या॑वापृथिवी द्यावापृथिवी इ॒द मि॒दम् द्या॑वापृथिवी । \newline
15. द्या॒वा॒पृ॒थि॒वी॒ भ॒द्रम् भ॒द्रम् द्या॑वापृथिवी द्यावापृथिवी भ॒द्रम् । \newline
16. द्या॒वा॒पृ॒थि॒वी॒ इति॑ द्यावा - पृ॒थि॒वी॒ । \newline
17. भ॒द्र म॑भू दभूद् भ॒द्रम् भ॒द्र म॑भूत् । \newline
18. अ॒भू॒ दिती त्य॑भू दभू॒ दिति॑ । \newline
19. इत्ये॒वैवे तीत्ये॒व । \newline
20. ए॒व ब्रू॑याद् ब्रूया दे॒वैव ब्रू॑यात् । \newline
21. ब्रू॒या॒द् यज॑मानं॒ ॅयज॑मानम् ब्रूयाद् ब्रूया॒द् यज॑मानम् । \newline
22. यज॑मान मे॒वैव यज॑मानं॒ ॅयज॑मान मे॒व । \newline
23. ए॒व य॒ज्ञ्स्य॑ य॒ज्ञ् स्यै॒वैव य॒ज्ञ्स्य॑ । \newline
24. य॒ज्ञ्स्या॒शिष॑ मा॒शिषं॑ ॅय॒ज्ञ्स्य॑ य॒ज्ञ्स्या॒शिष᳚म् । \newline
25. आ॒शिष॑म् गमयति गमय त्या॒शिष॑ मा॒शिष॑म् गमयति । \newline
26. आ॒शिष॒मित्या᳚ - शिष᳚म् । \newline
27. ग॒म॒य॒ त्यार्द्ध्मा र्द्ध्म॑ गमयति गमय॒ त्यार्द्ध्म॑ । \newline
28. आर्द्ध्म॑ सूक्तवा॒कꣳ सू᳚क्तवा॒क मार्द्ध्मा र्द्ध्म॑ सूक्तवा॒कम् । \newline
29. सू॒क्त॒वा॒क मु॒तोत सू᳚क्तवा॒कꣳ सू᳚क्तवा॒क मु॒त । \newline
30. सू॒क्त॒वा॒कमिति॑ सूक्त - वा॒कम् । \newline
31. उ॒त न॑मोवा॒कम् न॑मोवा॒क मु॒तोत न॑मोवा॒कम् । \newline
32. न॒मो॒वा॒क मितीति॑ नमोवा॒कम् न॑मोवा॒क मिति॑ । \newline
33. न॒मो॒वा॒कमिति॑ नमः - वा॒कम् । \newline
34. इत्या॑हा॒हे तीत्या॑ह । \newline
35. आ॒हे॒ द मि॒द मा॑हाहे॒ दम् । \newline
36. इ॒द म॑राथ्स्मा राथ्स्मे॒ द मि॒द म॑राथ्स्म । \newline
37. अ॒रा॒थ्स्मे ती त्य॑राथ्स्मा रा॒थ्स्मे ति॑ । \newline
38. इति॒ वाव वावे तीति॒ वाव । \newline
39. वावैत दे॒तद् वाव वावैतत् । \newline
40. ए॒त दा॑हाहै॒त दे॒तदा॑ह । \newline
41. आ॒होप॑श्रित॒ उप॑श्रित आहा॒ होप॑श्रितः । \newline
42. उप॑श्रितो दि॒वो दि॒व उप॑श्रित॒ उप॑श्रितो दि॒वः । \newline
43. उप॑श्रित॒ इत्युप॑ - श्रि॒तः॒ । \newline
44. दि॒वः पृ॑थि॒व्योः पृ॑थि॒व्योर् दि॒वो दि॒वः पृ॑थि॒व्योः । \newline
45. पृ॒थि॒व्यो रितीति॑ पृथि॒व्योः पृ॑थि॒व्यो रिति॑ । \newline
46. इत्या॑हा॒हे तीत्या॑ह । \newline
47. आ॒ह॒ द्यावा॑पृथि॒व्योर् द्यावा॑पृथि॒ व्योरा॑हाह॒ द्यावा॑पृथि॒व्योः । \newline
48. द्यावा॑पृथि॒व्योर्. हि हि द्यावा॑पृथि॒व्योर् द्यावा॑पृथि॒व्योर्. हि । \newline
49. द्यावा॑पृथि॒व्योरिति॒ द्यावा᳚ - पृ॒थि॒व्योः । \newline
50. हि य॒ज्ञो य॒ज्ञो हि हि य॒ज्ञ्ः । \newline
51. य॒ज्ञ् उप॑श्रित॒ उप॑श्रितो य॒ज्ञो य॒ज्ञ् उप॑श्रितः । \newline
52. उप॑श्रित॒ ओम॑न्वती॒ ओम॑न्वती॒ उप॑श्रित॒ उप॑श्रित॒ ओम॑न्वती । \newline
53. उप॑श्रित॒ इत्युप॑ - श्रि॒तः॒ । \newline
54. ओम॑न्वती ते त॒ ओम॑न्वती॒ ओम॑न्वती ते । \newline
55. ओम॑न्वती॒ इत्योमन्न्॑ - व॒ती॒ । \newline
56. ते॒ ऽस्मिन् न॒स्मिन् ते॑ ते॒ ऽस्मिन्न् । \newline
57. अ॒स्मिन्. य॒ज्ञे य॒ज्ञे᳚ ऽस्मिन् न॒स्मिन्. य॒ज्ञे । \newline
58. य॒ज्ञे य॑जमान यजमान य॒ज्ञे य॒ज्ञे य॑जमान । \newline
59. य॒ज॒मा॒न॒ द्यावा॑पृथि॒वी द्यावा॑पृथि॒वी य॑जमान यजमान॒ द्यावा॑पृथि॒वी । \newline
60. द्यावा॑पृथि॒वी स्ताꣳ॑ स्ता॒म् द्यावा॑पृथि॒वी द्यावा॑पृथि॒वी स्ता᳚म् । \newline
61. द्यावा॑पृथि॒वी इति॒ द्यावा᳚ - पृ॒थि॒वी । \newline

\textbf{Ghana Paata } \newline

1. उ॒ द्या॒वा॒पृ॒थि॒वी॒ द्या॒वा॒पृ॒थि॒वी॒ उ॒ वु॒ द्या॒वा॒पृ॒थि॒वी॒ भ॒द्रम् भ॒द्रम् द्या॑वापृथिवी उ वु द्यावापृथिवी भ॒द्रम् । \newline
2. द्या॒वा॒पृ॒थि॒वी॒ भ॒द्रम् भ॒द्रम् द्या॑वापृथिवी द्यावापृथिवी भ॒द्र म॑भू दभूद् भ॒द्रम् द्या॑वापृथिवी द्यावापृथिवी भ॒द्र म॑भूत् । \newline
3. द्या॒वा॒पृ॒थि॒वी॒ इति॑ द्यावा - पृ॒थि॒वी॒ । \newline
4. भ॒द्र म॑भू दभूद् भ॒द्रम् भ॒द्र म॑भू॒दिती त्य॑भूद् भ॒द्रम् भ॒द्र म॑भू॒दिति॑ । \newline
5. अ॒भू॒दिती त्य॑भू दभू॒ दित्ये॒तदु॑ मे॒तदु॒ मित्य॑भू दभू॒ दित्ये॒तदु᳚म् । \newline
6. इत्ये॒तदु॑ मे॒तदु॒ मिती त्ये॒तदु॑ मे॒वैवैतदु॒ मिती त्ये॒तदु॑ मे॒व । \newline
7. ए॒तदु॑ मे॒वैवैतदु॑ मे॒तदु॑ मे॒वासु॒र मा॑सु॒र मे॒वैतदु॑ मे॒तदु॑ मे॒वासु॒रम् । \newline
8. ए॒वासु॒र मा॑सु॒र मे॒वैवासु॒रं ॅय॒ज्ञ्स्य॑ य॒ज्ञ्स्या॑सु॒र मे॒वैवासु॒रं ॅय॒ज्ञ्स्य॑ । \newline
9. आ॒सु॒रं ॅय॒ज्ञ्स्य॑ य॒ज्ञ्स्या॑ सु॒र मा॑सु॒रं ॅय॒ज्ञ्स्या॒ शिष॑ मा॒शिषं॑ ॅय॒ज्ञ्स्या॑ सु॒र मा॑सु॒रं ॅय॒ज्ञ्स्या॒शिष᳚म् । \newline
10. य॒ज्ञ्स्या॒ शिष॑ मा॒शिषं॑ ॅय॒ज्ञ्स्य॑ य॒ज्ञ्स्या॒ शिष॑म् गमयेद् गमये दा॒शिषं॑ ॅय॒ज्ञ्स्य॑ य॒ज्ञ्स्या॒ शिष॑म् गमयेत् । \newline
11. आ॒शिष॑म् गमयेद् गमये दा॒शिष॑ मा॒शिष॑म् गमये दि॒द मि॒दम् ग॑मये दा॒शिष॑ मा॒शिष॑म् गमये दि॒दम् । \newline
12. आ॒शिष॒मित्या᳚ - शिष᳚म् । \newline
13. ग॒म॒ये॒ दि॒द मि॒दम् ग॑मयेद् गमये दि॒दम् द्या॑वापृथिवी द्यावापृथिवी इ॒दम् ग॑मयेद् गमये दि॒दम् द्या॑वापृथिवी । \newline
14. इ॒दम् द्या॑वापृथिवी द्यावापृथिवी इ॒द मि॒दम् द्या॑वापृथिवी भ॒द्रम् भ॒द्रम् द्या॑वापृथिवी इ॒द मि॒दम् द्या॑वापृथिवी भ॒द्रम् । \newline
15. द्या॒वा॒पृ॒थि॒वी॒ भ॒द्रम् भ॒द्रम् द्या॑वापृथिवी द्यावापृथिवी भ॒द्र म॑भू दभूद् भ॒द्रम् द्या॑वापृथिवी द्यावापृथिवी भ॒द्र म॑भूत् । \newline
16. द्या॒वा॒पृ॒थि॒वी॒ इति॑ द्यावा - पृ॒थि॒वी॒ । \newline
17. भ॒द्र म॑भू दभूद् भ॒द्रम् भ॒द्र म॑भू॒ दिती त्य॑भूद् भ॒द्रम् भ॒द्र म॑भू॒दिति॑ । \newline
18. अ॒भू॒ दिती त्य॑भू दभू॒ दित्ये॒वैवे त्य॑भू दभू॒ दित्ये॒व । \newline
19. इत्ये॒वैवे तीत्ये॒व ब्रू॑याद् ब्रूया दे॒वे तीत्ये॒व ब्रू॑यात् । \newline
20. ए॒व ब्रू॑याद् ब्रूया दे॒वैव ब्रू॑या॒द् यज॑मानं॒ ॅयज॑मानम् ब्रूया दे॒वैव ब्रू॑या॒द् यज॑मानम् । \newline
21. ब्रू॒या॒द् यज॑मानं॒ ॅयज॑मानम् ब्रूयाद् ब्रूया॒द् यज॑मान मे॒वैव यज॑मानम् ब्रूयाद् ब्रूया॒द् यज॑मान मे॒व । \newline
22. यज॑मान मे॒वैव यज॑मानं॒ ॅयज॑मान मे॒व य॒ज्ञ्स्य॑ य॒ज्ञ्स्यै॒व यज॑मानं॒ ॅयज॑मान मे॒व य॒ज्ञ्स्य॑ । \newline
23. ए॒व य॒ज्ञ्स्य॑ य॒ज्ञ्स्यै॒वैव य॒ज्ञ्स्या॒ शिष॑ मा॒शिषं॑ ॅय॒ज्ञ्स्यै॒वैव य॒ज्ञ्स्या॒शिष᳚म् । \newline
24. य॒ज्ञ्स्या॒ शिष॑ मा॒शिषं॑ ॅय॒ज्ञ्स्य॑ य॒ज्ञ्स्या॒ शिष॑म् गमयति गमय त्या॒शिषं॑ ॅय॒ज्ञ्स्य॑ य॒ज्ञ्स्या॒ शिष॑म् गमयति । \newline
25. आ॒शिष॑म् गमयति गमय त्या॒शिष॑ मा॒शिष॑म् गमय॒ त्यार्द्ध्मा र्द्ध्म॑ गमय त्या॒शिष॑ मा॒शिष॑म् गमय॒त्या र्द्ध्म॑ । \newline
26. आ॒शिष॒मित्या᳚ - शिष᳚म् । \newline
27. ग॒म॒य॒ त्यार्द्ध्मा र्द्ध्म॑ गमयति गमय॒ त्यार्द्ध्म॑ सूक्तवा॒कꣳ सू᳚क्तवा॒क मार्द्ध्म॑ गमयति गमय॒ त्यार्द्ध्म॑ सूक्तवा॒कम् । \newline
28. आर्द्ध्म॑ सूक्तवा॒कꣳ सू᳚क्तवा॒क मार्द्ध्मा र्द्ध्म॑ सूक्तवा॒क मु॒तोत सू᳚क्तवा॒क मार्द्ध्मा र्द्ध्म॑ सूक्तवा॒क मु॒त । \newline
29. सू॒क्त॒वा॒क मु॒तोत सू᳚क्तवा॒कꣳ सू᳚क्तवा॒क मु॒त न॑मोवा॒कम् न॑मोवा॒क मु॒त सू᳚क्तवा॒कꣳ सू᳚क्तवा॒क मु॒त न॑मोवा॒कम् । \newline
30. सू॒क्त॒वा॒कमिति॑ सूक्त - वा॒कम् । \newline
31. उ॒त न॑मोवा॒कम् न॑मोवा॒क मु॒तोत न॑मोवा॒क मितीति॑ नमोवा॒क मु॒तोत न॑मोवा॒क मिति॑ । \newline
32. न॒मो॒वा॒क मितीति॑ नमोवा॒कम् न॑मोवा॒क मित्या॑हा॒हे ति॑ नमोवा॒कम् न॑मोवा॒क मित्या॑ह । \newline
33. न॒मो॒वा॒कमिति॑ नमः - वा॒कम् । \newline
34. इत्या॑हा॒हे तीत्या॑हे॒ द मि॒द मा॒हे तीत्या॑हे॒ दम् । \newline
35. आ॒हे॒ द मि॒द मा॑हाहे॒ द म॑राथ्स्मा राथ्स्मे॒ द मा॑हाहे॒ द म॑राथ्स्म । \newline
36. इ॒द म॑राथ्स्मा राथ्स्मे॒ द मि॒द म॑रा॒थ्स्मे तीत्य॑राथ्स्मे॒ द मि॒द म॑रा॒थ्स्मे ति॑ । \newline
37. अ॒रा॒थ्स्मे तीत्य॑राथ्स्मा रा॒थ्स्मे ति॒ वाव वावे त्य॑राथ्स्मा रा॒थ्स्मे ति॒ वाव । \newline
38. इति॒ वाव वावे तीति॒ वावैत दे॒तद् वावे तीति॒ वावैतत् । \newline
39. वावैत दे॒तद् वाव वावैत दा॑हाहै॒तद् वाव वावैतदा॑ह । \newline
40. ए॒त दा॑हाहै॒त दे॒तदा॒हो प॑श्रित॒ उप॑श्रित आहै॒त दे॒तदा॒हो प॑श्रितः । \newline
41. आ॒होप॑श्रित॒ उप॑श्रित आहा॒हो प॑श्रितो दि॒वो दि॒व उप॑श्रित आहा॒हो प॑श्रितो दि॒वः । \newline
42. उप॑श्रितो दि॒वो दि॒व उप॑श्रित॒ उप॑श्रितो दि॒वः पृ॑थि॒व्योः पृ॑थि॒व्योर् दि॒व उप॑श्रित॒ उप॑श्रितो दि॒वः पृ॑थि॒व्योः । \newline
43. उप॑श्रित॒ इत्युप॑ - श्रि॒तः॒ । \newline
44. दि॒वः पृ॑थि॒व्योः पृ॑थि॒व्योर् दि॒वो दि॒वः पृ॑थि॒व्यो रितीति॑ पृथि॒व्योर् दि॒वो दि॒वः पृ॑थि॒व्यो रिति॑ । \newline
45. पृ॒थि॒व्यो रितीति॑ पृथि॒व्योः पृ॑थि॒व्यो रित्या॑हा॒हे ति॑ पृथि॒व्योः पृ॑थि॒व्यो रित्या॑ह । \newline
46. इत्या॑हा॒हे तीत्या॑ह॒ द्यावा॑पृथि॒व्योर् द्यावा॑पृथि॒व्यो रा॒हे तीत्या॑ह॒ द्यावा॑पृथि॒व्योः । \newline
47. आ॒ह॒ द्यावा॑पृथि॒व्योर् द्यावा॑पृथि॒व्यो रा॑हाह॒ द्यावा॑पृथि॒व्योर्. हि हि द्यावा॑पृथि॒व्यो रा॑हाह॒ द्यावा॑पृथि॒व्योर्. हि । \newline
48. द्यावा॑पृथि॒व्योर्. हि हि द्यावा॑पृथि॒व्योर् द्यावा॑पृथि॒व्योर्. हि य॒ज्ञो य॒ज्ञो हि द्यावा॑पृथि॒व्योर् द्यावा॑पृथि॒व्योर्. हि य॒ज्ञ्ः । \newline
49. द्यावा॑पृथि॒व्योरिति॒ द्यावा᳚ - पृ॒थि॒व्योः । \newline
50. हि य॒ज्ञो य॒ज्ञो हि हि य॒ज्ञ् उप॑श्रित॒ उप॑श्रितो य॒ज्ञो हि हि य॒ज्ञ् उप॑श्रितः । \newline
51. य॒ज्ञ् उप॑श्रित॒ उप॑श्रितो य॒ज्ञो य॒ज्ञ् उप॑श्रित॒ ओम॑न्वती॒ ओम॑न्वती॒ उप॑श्रितो य॒ज्ञो य॒ज्ञ् उप॑श्रित॒ ओम॑न्वती । \newline
52. उप॑श्रित॒ ओम॑न्वती॒ ओम॑न्वती॒ उप॑श्रित॒ उप॑श्रित॒ ओम॑न्वती ते त॒ ओम॑न्वती॒ उप॑श्रित॒ उप॑श्रित॒ ओम॑न्वती ते । \newline
53. उप॑श्रित॒ इत्युप॑ - श्रि॒तः॒ । \newline
54. ओम॑न्वती ते त॒ ओम॑न्वती॒ ओम॑न्वती ते॒ ऽस्मिन् न॒स्मिन् त॒ ओम॑न्वती॒ ओम॑न्वती ते॒ ऽस्मिन्न् । \newline
55. ओम॑न्वती॒ इत्योमन्न्॑ - व॒ती॒ । \newline
56. ते॒ ऽस्मिन् न॒स्मिन् ते॑ ते॒ ऽस्मिन्. य॒ज्ञे य॒ज्ञे᳚ ऽस्मिन् ते॑ ते॒ ऽस्मिन्. य॒ज्ञे । \newline
57. अ॒स्मिन्. य॒ज्ञे य॒ज्ञे᳚ ऽस्मिन् न॒स्मिन्. य॒ज्ञे य॑जमान यजमान य॒ज्ञे᳚ ऽस्मिन् न॒स्मिन्. य॒ज्ञे य॑जमान । \newline
58. य॒ज्ञे य॑जमान यजमान य॒ज्ञे य॒ज्ञे य॑जमान॒ द्यावा॑पृथि॒वी द्यावा॑पृथि॒वी य॑जमान य॒ज्ञे य॒ज्ञे य॑जमान॒ द्यावा॑पृथि॒वी । \newline
59. य॒ज॒मा॒न॒ द्यावा॑पृथि॒वी द्यावा॑पृथि॒वी य॑जमान यजमान॒ द्यावा॑पृथि॒वी स्ताꣳ॑ स्ता॒म् द्यावा॑पृथि॒वी य॑जमान यजमान॒ द्यावा॑पृथि॒वी स्ता᳚म् । \newline
60. द्यावा॑पृथि॒वी स्ताꣳ॑ स्ता॒म् द्यावा॑पृथि॒वी द्यावा॑पृथि॒वी स्ता॒ मितीति॑ स्ता॒म् द्यावा॑पृथि॒वी द्यावा॑पृथि॒वी स्ता॒ मिति॑ । \newline
61. द्यावा॑पृथि॒वी इति॒ द्यावा᳚ - पृ॒थि॒वी । \newline
\pagebreak
\markright{ TS 2.6.9.6  \hfill https://www.vedavms.in \hfill}
\addcontentsline{toc}{section}{ TS 2.6.9.6 }
\section*{ TS 2.6.9.6 }

\textbf{TS 2.6.9.6 } \newline
\textbf{Samhita Paata} \newline

स्ता॒मित्या॑हा॒ ऽऽशिष॑मे॒वैतामा शा᳚स्ते॒ यद्ब्रू॒याथ् सू॑पावसा॒ना च॑ स्वद्ध्यवसा॒ना चेति॑ प्र॒मायु॑को॒ यज॑मानः स्याद्य॒दा हि प्र॒मीय॒ते ऽथे॒मामु॑पाव॒स्यति॑ सूपचर॒णा च॑ स्वधिचर॒णा चेत्ये॒व ब्रू॑या॒द्-वरी॑यसीमे॒वास्मै॒ गव्यू॑ति॒मा शा᳚स्ते॒ न प्र॒मायु॑को भवति॒ तयो॑रा॒विद्य॒ग्निरि॒दꣳ ह॒विर॑जुष॒तेत्या॑ह॒ या अया᳚क्ष्म - [  ] \newline

\textbf{Pada Paata} \newline

स्ता॒म् । इति॑ । आ॒ह॒ । आ॒शिष॒मित्या᳚-शिष᳚म् । ए॒व । ए॒ताम् । एति॑ । शा॒स्ते॒ । यत् । ब्रू॒यात् । सू॒पा॒व॒सा॒नेति॑ सु - उ॒पा॒व॒सा॒ना । च॒ । स्व॒द्ध्य॒व॒सा॒नेति॑ सु - अ॒द्ध्य॒व॒सा॒ना । च॒ । इति॑ । प्र॒मायु॑क॒ इति॑ प्र - मायु॑कः । यज॑मानः । स्या॒त् । य॒दा । हि । प्र॒मीय॑त॒ इति॑ प्र - मीय॑ते । अथ॑ । इ॒माम् । उ॒पा॒व॒स्यतीत्यु॑प - अ॒व॒स्यति॑ । सू॒प॒च॒र॒णेति॑ सु - उ॒प॒च॒र॒णा । च॒ । स्व॒धि॒च॒र॒णेति॑ सु - अ॒धि॒च॒र॒णा । च॒ । इति॑ । ए॒व । ब्रू॒या॒त् । वरी॑यसीम् । ए॒व । अ॒स्मै॒ । गव्यू॑तिम् । एति॑ । शा॒स्ते॒ । न । प्र॒मायु॑क॒ इति॑ प्र - मायु॑कः । भ॒व॒ति॒ । तयोः᳚ । आ॒विदीत्या᳚-विदि॑ । अ॒ग्निः । इ॒दम् । ह॒विः । अ॒जु॒ष॒त॒ । इति॑ । आ॒ह॒ । याः । अया᳚क्ष्म ।  \newline


\textbf{Krama Paata} \newline

स्ता॒मिति॑ । इत्या॑ह । आ॒हा॒शिष᳚म् । आ॒शिष॑मे॒व । आ॒शिष॒मित्या᳚ - शिष᳚म् । ए॒वैताम् । ए॒तामा । आ शा᳚स्ते । शा॒स्ते॒ यत् । यद् ब्रू॒यात् । ब्रू॒याथ् सू॑पावसा॒ना । सू॒पा॒व॒सा॒ना च॑ । सू॒पा॒व॒सा॒नेति॑ सु - उ॒पा॒व॒सा॒ना । च॒ स्व॒द्ध्य॒व॒सा॒ना । स्व॒द्ध्य॒व॒सा॒ना च॑ । स्व॒द्ध्य॒व॒सा॒नेति॑ सु - अ॒द्ध्य॒व॒सा॒ना । चेति॑ । इति॑ प्र॒मायु॑कः । प्र॒मायु॑को॒ यज॑मानः । प्र॒मायु॑क॒ इति॑ प्र - मायु॑कः । यज॑मानः स्यात् । स्या॒द् य॒दा । य॒दा हि । हि प्र॒मीय॑ते । प्र॒मीय॒ते ऽथ॑ । प्र॒मीय॑त॒ इति॑ प्र - मीय॑ते । अथे॒माम् । इ॒मामु॑पाव॒स्यति॑ । उ॒पा॒व॒स्यति॑ सूपचर॒णा । उ॒पा॒व॒स्यतीत्यु॑प - अ॒व॒स्यति॑ । सू॒प॒च॒र॒णा च॑ । सू॒प॒च॒र॒णेति॑ सु - उ॒प॒च॒र॒णा । च॒ स्व॒धि॒च॒र॒णा । स्व॒धि॒च॒र॒णा च॑ । स्व॒धि॒च॒र॒णेति॑ सु - अ॒धि॒च॒र॒णा । चेति॑ । इत्ये॒व । ए॒व ब्रू॑यात् । ब्रू॒या॒द् वरी॑यसीम् । वरी॑यसीमे॒व । ए॒वास्मै᳚ । अ॒स्मै॒ गव्यू॑तिम् । गव्यू॑ति॒मा । आ शा᳚स्ते । शा॒स्ते॒ न । न प्र॒मायु॑कः । प्र॒मायु॑को भवति । प्र॒मायु॑क॒ इति॑ प्र - मायु॑कः । भ॒व॒ति॒ तयोः᳚ । तयो॑रा॒विदि॑ । आ॒विद्य॒ग्निः । आ॒विदीत्या᳚ - विदि॑ । अ॒ग्निरि॒दम् । इ॒दꣳ ह॒विः । ह॒विर॑जुषत । अ॒जु॒ष॒तेति॑ । इत्या॑ह । आ॒ह॒ याः । या अया᳚क्ष्म । अया᳚क्ष्म दे॒वताः᳚ \newline

\textbf{Jatai Paata} \newline

1. स्ता॒ मितीति॑ स्ताꣳ स्ता॒ मिति॑ । \newline
2. इत्या॑हा॒हे तीत्या॑ह । \newline
3. आ॒हा॒शिष॑ मा॒शिष॑ माहाहा॒ शिष᳚म् । \newline
4. आ॒शिष॑ मे॒वैवाशिष॑ मा॒शिष॑ मे॒व । \newline
5. आ॒शिष॒मित्या᳚ - शिष᳚म् । \newline
6. ए॒वैता मे॒ता मे॒वैवैताम् । \newline
7. ए॒ता मैता मे॒ता मा । \newline
8. आ शा᳚स्ते शास्त॒ आ शा᳚स्ते । \newline
9. शा॒स्ते॒ यद् यच् छा᳚स्ते शास्ते॒ यत् । \newline
10. यद् ब्रू॒याद् ब्रू॒याद् यद् यद् ब्रू॒यात् । \newline
11. ब्रू॒याथ् सू॑पावसा॒ना सू॑पावसा॒ना ब्रू॒याद् ब्रू॒याथ् सू॑पावसा॒ना । \newline
12. सू॒पा॒व॒सा॒ना च॑ च सूपावसा॒ना सू॑पावसा॒ना च॑ । \newline
13. सू॒पा॒व॒सा॒नेति॑ सु - उ॒पा॒व॒सा॒ना । \newline
14. च॒ स्व॒द्ध्य॒व॒सा॒ना स्व॑द्ध्यवसा॒ना च॑ च स्वद्ध्यवसा॒ना । \newline
15. स्व॒द्ध्य॒व॒सा॒ना च॑ च स्वद्ध्यवसा॒ना स्व॑द्ध्यवसा॒ना च॑ । \newline
16. स्व॒द्ध्य॒व॒सा॒नेति॑ सु - अ॒द्ध्य॒व॒सा॒ना । \newline
17. चे तीति॑ च॒ चे ति॑ । \newline
18. इति॑ प्र॒मायु॑कः प्र॒मायु॑क॒ इतीति॑ प्र॒मायु॑कः । \newline
19. प्र॒मायु॑को॒ यज॑मानो॒ यज॑मानः प्र॒मायु॑कः प्र॒मायु॑को॒ यज॑मानः । \newline
20. प्र॒मायु॑क॒ इति॑ प्र - मायु॑कः । \newline
21. यज॑मानः स्याथ् स्या॒द् यज॑मानो॒ यज॑मानः स्यात् । \newline
22. स्या॒द् य॒दा य॒दा स्या᳚थ् स्याद् य॒दा । \newline
23. य॒दा हि हि य॒दा य॒दा हि । \newline
24. हि प्र॒मीय॑ते प्र॒मीय॑ते॒ हि हि प्र॒मीय॑ते । \newline
25. प्र॒मीय॒ते ऽथाथ॑ प्र॒मीय॑ते प्र॒मीय॒ते ऽथ॑ । \newline
26. प्र॒मीय॑त॒ इति॑ प्र - मीय॑ते । \newline
27. अथे॒ मा मि॒मा मथाथे॒ माम् । \newline
28. इ॒मा मु॑पाव॒स्य त्यु॑पाव॒स्यती॒मा मि॒मा मु॑पाव॒स्यति॑ । \newline
29. उ॒पा॒व॒स्यति॑ सूपचर॒णा सू॑पचर॒ णोपा॑व॒स्य त्यु॑पाव॒स्यति॑ सूपचर॒णा । \newline
30. उ॒पा॒व॒स्यतीत्यु॑प - अ॒व॒स्यति॑ । \newline
31. सू॒प॒च॒र॒णा च॑ च सूपचर॒णा सू॑पचर॒णा च॑ । \newline
32. सू॒प॒च॒र॒णेति॑ सु - उ॒प॒च॒र॒णा । \newline
33. च॒ स्व॒धि॒च॒र॒णा स्व॑धिचर॒णा च॑ च स्वधिचर॒णा । \newline
34. स्व॒धि॒च॒र॒णा च॑ च स्वधिचर॒णा स्व॑धिचर॒णा च॑ । \newline
35. स्व॒धि॒च॒र॒णेति॑ सु - अ॒धि॒च॒र॒णा । \newline
36. चे तीति॑ च॒ चे ति॑ । \newline
37. इत्ये॒वैवे तीत्ये॒व । \newline
38. ए॒व ब्रू॑याद् ब्रूया दे॒वैव ब्रू॑यात् । \newline
39. ब्रू॒या॒द् वरी॑यसीं॒ ॅवरी॑यसीम् ब्रूयाद् ब्रूया॒द् वरी॑यसीम् । \newline
40. वरी॑यसी मे॒वैव वरी॑यसीं॒ ॅवरी॑यसी मे॒व । \newline
41. ए॒वास्मा॑ अस्मा ए॒वैवास्मै᳚ । \newline
42. अ॒स्मै॒ गव्यू॑ति॒म् गव्यू॑ति मस्मा अस्मै॒ गव्यू॑तिम् । \newline
43. गव्यू॑ति॒ मा गव्यू॑ति॒म् गव्यू॑ति॒ मा । \newline
44. आ शा᳚स्ते शास्त॒ आ शा᳚स्ते । \newline
45. शा॒स्ते॒ न न शा᳚स्ते शास्ते॒ न । \newline
46. न प्र॒मायु॑कः प्र॒मायु॑को॒ न न प्र॒मायु॑कः । \newline
47. प्र॒मायु॑को भवति भवति प्र॒मायु॑कः प्र॒मायु॑को भवति । \newline
48. प्र॒मायु॑क॒ इति॑ प्र - मायु॑कः । \newline
49. भ॒व॒ति॒ तयो॒ स्तयो᳚र् भवति भवति॒ तयोः᳚ । \newline
50. तयो॑ रा॒वि द्या॒विदि॒ तयो॒ स्तयो॑ रा॒विदि॑ । \newline
51. आ॒वि द्य॒ग्नि र॒ग्नि रा॒वि द्या॒वि द्य॒ग्निः । \newline
52. आ॒विदीत्या᳚ - विदि॑ । \newline
53. अ॒ग्नि रि॒द मि॒द म॒ग्नि र॒ग्नि रि॒दम् । \newline
54. इ॒दꣳ ह॒विर्. ह॒विरि॒द मि॒दꣳ ह॒विः । \newline
55. ह॒वि र॑जुषता जुषत ह॒विर्. ह॒वि र॑जुषत । \newline
56. अ॒जु॒ष॒ते ती त्य॑जुषता जुष॒ते ति॑ । \newline
57. इत्या॑हा॒हे तीत्या॑ह । \newline
58. आ॒ह॒ या या आ॑हाह॒ याः । \newline
59. या अया॒क्ष्मा या᳚क्ष्म॒ या या अया᳚क्ष्म । \newline
60. अया᳚क्ष्म दे॒वता॑ दे॒वता॒ अया॒क्ष्मा या᳚क्ष्म दे॒वताः᳚ । \newline

\textbf{Ghana Paata } \newline

1. स्ता॒ मितीति॑ स्ताꣳ स्ता॒ मित्या॑हा॒हे ति॑ स्ताꣳ स्ता॒ मित्या॑ह । \newline
2. इत्या॑हा॒हे तीत्या॑हा॒ शिष॑ मा॒शिष॑ मा॒हे तीत्या॑हा॒ शिष᳚म् । \newline
3. आ॒हा॒शिष॑ मा॒शिष॑ माहाहा॒शिष॑ मे॒वैवाशिष॑ माहाहा॒शिष॑ मे॒व । \newline
4. आ॒शिष॑ मे॒वैवाशिष॑ मा॒शिष॑ मे॒वैता मे॒ता मे॒वाशिष॑ मा॒शिष॑ मे॒वैताम् । \newline
5. आ॒शिष॒मित्या᳚ - शिष᳚म् । \newline
6. ए॒वैता मे॒ता मे॒वैवैता मैता मे॒वैवैता मा । \newline
7. ए॒ता मैता मे॒ता मा शा᳚स्ते शास्त॒ ऐता मे॒ता मा शा᳚स्ते । \newline
8. आ शा᳚स्ते शास्त॒ आ शा᳚स्ते॒ यद् यच्छा᳚स्त॒ आ शा᳚स्ते॒ यत् । \newline
9. शा॒स्ते॒ यद् यच्छा᳚स्ते शास्ते॒ यद् ब्रू॒याद् ब्रू॒याद् यच् छा᳚स्ते शास्ते॒ यद् ब्रू॒यात् । \newline
10. यद् ब्रू॒याद् ब्रू॒याद् यद् यद् ब्रू॒याथ् सू॑पावसा॒ना सू॑पावसा॒ना ब्रू॒याद् यद् यद् ब्रू॒याथ् सू॑पावसा॒ना । \newline
11. ब्रू॒याथ् सू॑पावसा॒ना सू॑पावसा॒ना ब्रू॒याद् ब्रू॒याथ् सू॑पावसा॒ना च॑ च सूपावसा॒ना ब्रू॒याद् ब्रू॒याथ् सू॑पावसा॒ना च॑ । \newline
12. सू॒पा॒व॒सा॒ना च॑ च सूपावसा॒ना सू॑पावसा॒ना च॑ स्वद्ध्यवसा॒ना स्व॑द्ध्यवसा॒ना च॑ सूपावसा॒ना सू॑पावसा॒ना च॑ स्वद्ध्यवसा॒ना । \newline
13. सू॒पा॒व॒सा॒नेति॑ सु - उ॒पा॒व॒सा॒ना । \newline
14. च॒ स्व॒द्ध्य॒व॒सा॒ना स्व॑द्ध्यवसा॒ना च॑ च स्वद्ध्यवसा॒ना च॑ च स्वद्ध्यवसा॒ना च॑ च स्वद्ध्यवसा॒ना च॑ । \newline
15. स्व॒द्ध्य॒व॒सा॒ना च॑ च स्वद्ध्यवसा॒ना स्व॑द्ध्यवसा॒ना चे तीति॑ च स्वद्ध्यवसा॒ना स्व॑द्ध्यवसा॒ना चे ति॑ । \newline
16. स्व॒द्ध्य॒व॒सा॒नेति॑ सु - अ॒द्ध्य॒व॒सा॒ना । \newline
17. चे तीति॑ च॒ चे ति॑ प्र॒मायु॑कः प्र॒मायु॑क॒ इति॑ च॒ चे ति॑ प्र॒मायु॑कः । \newline
18. इति॑ प्र॒मायु॑कः प्र॒मायु॑क॒ इतीति॑ प्र॒मायु॑को॒ यज॑मानो॒ यज॑मानः प्र॒मायु॑क॒ इतीति॑ प्र॒मायु॑को॒ यज॑मानः । \newline
19. प्र॒मायु॑को॒ यज॑मानो॒ यज॑मानः प्र॒मायु॑कः प्र॒मायु॑को॒ यज॑मानः स्याथ् स्या॒द् यज॑मानः प्र॒मायु॑कः प्र॒मायु॑को॒ यज॑मानः स्यात् । \newline
20. प्र॒मायु॑क॒ इति॑ प्र - मायु॑कः । \newline
21. यज॑मानः स्याथ् स्या॒द् यज॑मानो॒ यज॑मानः स्याद् य॒दा य॒दा स्या॒द् यज॑मानो॒ यज॑मानः स्याद् य॒दा । \newline
22. स्या॒द् य॒दा य॒दा स्या᳚थ् स्याद् य॒दा हि हि य॒दा स्या᳚थ् स्याद् य॒दा हि । \newline
23. य॒दा हि हि य॒दा य॒दा हि प्र॒मीय॑ते प्र॒मीय॑ते॒ हि य॒दा य॒दा हि प्र॒मीय॑ते । \newline
24. हि प्र॒मीय॑ते प्र॒मीय॑ते॒ हि हि प्र॒मीय॒ते ऽथाथ॑ प्र॒मीय॑ते॒ हि हि प्र॒मीय॒ते ऽथ॑ । \newline
25. प्र॒मीय॒ते ऽथाथ॑ प्र॒मीय॑ते प्र॒मीय॒ते ऽथे॒ मा मि॒मा मथ॑ प्र॒मीय॑ते प्र॒मीय॒ते ऽथे॒ माम् । \newline
26. प्र॒मीय॑त॒ इति॑ प्र - मीय॑ते । \newline
27. अथे॒ मा मि॒मा मथाथे॒ मा मु॑पाव॒स्य त्यु॑पाव॒स्यती॒मा मथाथे॒ मा मु॑पाव॒स्यति॑ । \newline
28. इ॒मा मु॑पाव॒स्य त्यु॑पाव॒स्यती॒मा मि॒मा मु॑पाव॒स्यति॑ सूपचर॒णा सू॑पचर॒ णोपा॑व॒स्यती॒मा मि॒मा मु॑पाव॒स्यति॑ सूपचर॒णा । \newline
29. उ॒पा॒व॒स्यति॑ सूपचर॒णा सू॑पचर॒ णोपा॑व॒स्य त्यु॑पाव॒स्यति॑ सूपचर॒णा च॑ च सूपचर॒ णोपा॑व॒स्य त्यु॑पाव॒स्यति॑ सूपचर॒णा च॑ । \newline
30. उ॒पा॒व॒स्यतीत्यु॑प - अ॒व॒स्यति॑ । \newline
31. सू॒प॒च॒र॒णा च॑ च सूपचर॒णा सू॑पचर॒णा च॑ स्वधिचर॒णा स्व॑धिचर॒णा च॑ सूपचर॒णा सू॑पचर॒णा च॑ स्वधिचर॒णा । \newline
32. सू॒प॒च॒र॒णेति॑ सु - उ॒प॒च॒र॒णा । \newline
33. च॒ स्व॒धि॒च॒र॒णा स्व॑धिचर॒णा च॑ च स्वधिचर॒णा च॑ च स्वधिचर॒णा च॑ च स्वधिचर॒णा च॑ । \newline
34. स्व॒धि॒च॒र॒णा च॑ च स्वधिचर॒णा स्व॑धिचर॒णा चे तीति॑ च स्वधिचर॒णा स्व॑धिचर॒णा चे ति॑ । \newline
35. स्व॒धि॒च॒र॒णेति॑ सु - अ॒धि॒च॒र॒णा । \newline
36. चे तीति॑ च॒ चे त्ये॒वैवे ति॑ च॒ चे त्ये॒व । \newline
37. इत्ये॒वैवे तीत्ये॒व ब्रू॑याद् ब्रूयादे॒वे तीत्ये॒व ब्रू॑यात् । \newline
38. ए॒व ब्रू॑याद् ब्रूया दे॒वैव ब्रू॑या॒द् वरी॑यसीं॒ ॅवरी॑यसीम् ब्रूया दे॒वैव ब्रू॑या॒द् वरी॑यसीम् । \newline
39. ब्रू॒या॒द् वरी॑यसीं॒ ॅवरी॑यसीम् ब्रूयाद् ब्रूया॒द् वरी॑यसी मे॒वैव वरी॑यसीम् ब्रूयाद् ब्रूया॒द् वरी॑यसी मे॒व । \newline
40. वरी॑यसी मे॒वैव वरी॑यसीं॒ ॅवरी॑यसी मे॒वास्मा॑ अस्मा ए॒व वरी॑यसीं॒ ॅवरी॑यसी मे॒वास्मै᳚ । \newline
41. ए॒वास्मा॑ अस्मा ए॒वैवास्मै॒ गव्यू॑ति॒म् गव्यू॑ति मस्मा ए॒वैवास्मै॒ गव्यू॑तिम् । \newline
42. अ॒स्मै॒ गव्यू॑ति॒म् गव्यू॑ति मस्मा अस्मै॒ गव्यू॑ति॒ मा गव्यू॑ति मस्मा अस्मै॒ गव्यू॑ति॒ मा । \newline
43. गव्यू॑ति॒ मा गव्यू॑ति॒म् गव्यू॑ति॒ मा शा᳚स्ते शास्त॒ आ गव्यू॑ति॒म् गव्यू॑ति॒ मा शा᳚स्ते । \newline
44. आ शा᳚स्ते शास्त॒ आ शा᳚स्ते॒ न न शा᳚स्त॒ आ शा᳚स्ते॒ न । \newline
45. शा॒स्ते॒ न न शा᳚स्ते शास्ते॒ न प्र॒मायु॑कः प्र॒मायु॑को॒ न शा᳚स्ते शास्ते॒ न प्र॒मायु॑कः । \newline
46. न प्र॒मायु॑कः प्र॒मायु॑को॒ न न प्र॒मायु॑को भवति भवति प्र॒मायु॑को॒ न न प्र॒मायु॑को भवति । \newline
47. प्र॒मायु॑को भवति भवति प्र॒मायु॑कः प्र॒मायु॑को भवति॒ तयो॒ स्तयो᳚र् भवति प्र॒मायु॑कः प्र॒मायु॑को भवति॒ तयोः᳚ । \newline
48. प्र॒मायु॑क॒ इति॑ प्र - मायु॑कः । \newline
49. भ॒व॒ति॒ तयो॒ स्तयो᳚र् भवति भवति॒ तयो॑ रा॒विद्या॒ विदि॒ तयो᳚र् भवति भवति॒ तयो॑ रा॒विदि॑ । \newline
50. तयो॑रा॒विद्या॒ विदि॒ तयो॒ स्तयो॑ रा॒वि द्य॒ग्नि र॒ग्नि रा॒विदि॒ तयो॒ स्तयो॑ रा॒विद्य॒ग्निः । \newline
51. आ॒विद्य॒ग्नि र॒ग्नि रा॒विद्या॒ विद्य॒ग्नि रि॒द मि॒द म॒ग्नि रा॒विद्या॒ विद्य॒ग्नि रि॒दम् । \newline
52. आ॒विदीत्या᳚ - विदि॑ । \newline
53. अ॒ग्निरि॒द मि॒द म॒ग्नि र॒ग्नि रि॒दꣳ ह॒विर्. ह॒विरि॒द म॒ग्नि र॒ग्नि रि॒दꣳ ह॒विः । \newline
54. इ॒दꣳ ह॒विर्. ह॒विरि॒द मि॒दꣳ ह॒वि र॑जुषता जुषत ह॒विरि॒द मि॒दꣳ ह॒वि र॑जुषत । \newline
55. ह॒वि र॑जुषता जुषत ह॒विर्. ह॒वि र॑जुष॒ते तीत्य॑जुषत ह॒विर्. ह॒वि र॑जुष॒ते ति॑ । \newline
56. अ॒जु॒ष॒ते तीत्य॑जुषता जुष॒ते त्या॑हा॒हे त्य॑जुषता जुष॒ते त्या॑ह । \newline
57. इत्या॑हा॒हे तीत्या॑ह॒ या या आ॒हे तीत्या॑ह॒ याः । \newline
58. आ॒ह॒ या या आ॑हाह॒ या अया॒क्ष्मा या᳚क्ष्म॒ या आ॑हाह॒ या अया᳚क्ष्म । \newline
59. या अया॒क्ष्मा या᳚क्ष्म॒ या या अया᳚क्ष्म दे॒वता॑ दे॒वता॒ अया᳚क्ष्म॒ या या अया᳚क्ष्म दे॒वताः᳚ । \newline
60. अया᳚क्ष्म दे॒वता॑ दे॒वता॒ अया॒क्ष्मा या᳚क्ष्म दे॒वता॒ स्ता स्ता दे॒वता॒ अया॒क्ष्मा या᳚क्ष्म दे॒वता॒ स्ताः । \newline
\pagebreak
\markright{ TS 2.6.9.7  \hfill https://www.vedavms.in \hfill}
\addcontentsline{toc}{section}{ TS 2.6.9.7 }
\section*{ TS 2.6.9.7 }

\textbf{TS 2.6.9.7 } \newline
\textbf{Samhita Paata} \newline

दे॒वता॒स्ता अ॑रीरधा॒मेति॒ वावैतदा॑ह॒ यन्न नि॑र्दि॒शेत् प्रति॑वेशं ॅय॒ज्ञ्स्या॒ ऽऽशीर्ग॑च्छे॒दा शा᳚स्ते॒ऽयं ॅयज॑मानो॒ऽसावित्या॑ह नि॒र्दिश्यै॒वैनꣳ॑ सुव॒र्गं ॅलो॒कं ग॑मय॒त्यायु॒रा शा᳚स्ते सुप्रजा॒स्त्वमा शा᳚स्त॒ इत्या॑हा॒ ऽशिष॑मे॒वै तामा शा᳚स्ते सजातवन॒स्यामा शा᳚स्त॒ इत्या॑ह प्रा॒णा वै स॑जा॒ताः प्रा॒णाने॒व - [  ] \newline

\textbf{Pada Paata} \newline

दे॒वताः᳚ । ताः । अ॒री॒र॒धा॒म॒ । इति॑ । वाव । ए॒तत् । आ॒ह॒ । यत् । न । नि॒र्दि॒शेदिति॑ निः - दि॒शेत् । प्रति॑वेश॒मिति॒ प्रति॑ - वे॒श॒म् । य॒ज्ञ्स्य॑ । आ॒शीरित्या᳚ - शीः । ग॒च्छे॒त् । एति॑ । शा॒स्ते॒ । अ॒यम् । यज॑मानः । अ॒सौ । इति॑ । आ॒ह॒ । नि॒र्दिश्येति॑ निः - दिश्य॑ । ए॒व । ए॒न॒म् । सु॒व॒र्गमिति॑ सुवः-गम् । लो॒कम् । ग॒म॒य॒ति॒ । आयुः॑ । एति॑ । शा॒स्ते॒ । सु॒प्र॒जा॒स्त्वमिति॑ सुप्रजाः - त्वम् । एति॑ । शा॒स्ते॒ । इति॑ । आ॒ह॒ । आ॒शिष॒मित्या᳚-शिष᳚म् । ए॒व । ए॒ताम् । एति॑ । शा॒स्ते॒ । स॒जा॒त॒व॒न॒स्यामिति॑ सजात - व॒न॒स्याम् । एति॑ । शा॒स्ते॒ । इति॑ । आ॒ह॒ । प्रा॒णा इति॑ प्र - अ॒नाः । वै । स॒जा॒ता इति॑ स - जा॒ताः । प्रा॒णानिति॑ प्र - अ॒नान् । ए॒व ।  \newline


\textbf{Krama Paata} \newline

दे॒वता॒स्ताः । ता अ॑रीरधा॒म । अ॒री॒र॒धा॒मेति॑ । इति॒ वाव । वावैतत् । ए॒तदा॑ह । आ॒ह॒ यत् । यन् न । न नि॑र्दि॒शेत् । नि॒र्दि॒शेत् प्रति॑वेशम् । नि॒र्दि॒शेदिति॑ निः - दि॒शेत् । प्रति॑वेशं ॅय॒ज्ञ्स्य॑ । प्रति॑वेश॒मिति॒ प्रति॑ - वे॒श॒म् । य॒ज्ञ्स्या॒शीः । आ॒शीर् ग॑च्छेत् । आ॒शीरित्या᳚ - शीः । ग॒च्छे॒दा । आ शा᳚स्ते । शा॒स्ते॒ऽयम् । अ॒यं ॅयज॑मानः । यज॑मानो॒ऽसौ । अ॒साविति॑ । इत्या॑ह । आ॒ह॒ नि॒र्दिश्य॑ । नि॒र्दिश्यै॒व । नि॒र्दिश्येति॑ निः - दिश्य॑ । ए॒वैन᳚म् । ए॒नꣳ॒॒ सु॒व॒र्गम् । सु॒व॒र्गं ॅलो॒कम् । सु॒व॒र्गमिति॑ सुवः - गम् । लो॒कम् ग॑मयति । ग॒म॒य॒त्यायुः॑ । आयु॒रा । आ शा᳚स्ते । शा॒स्ते॒ सु॒प्र॒जा॒स्त्वम् । सु॒प्र॒जा॒स्त्वमा । सु॒प्र॒जा॒स्त्वमिति॑ सुप्रजाः - त्वम् । आ शा᳚स्ते । शा॒स्त॒ इति॑ । इत्या॑ह । आ॒हा॒शिष᳚म् । आ॒शिष॑मे॒व । आ॒शिष॒मित्या᳚ - शिष᳚म् । ए॒वैताम् । ए॒तामा । आ शा᳚स्ते । शा॒स्ते॒ स॒जा॒त॒व॒न॒स्याम् । स॒जा॒त॒व॒न॒स्यामा । स॒जा॒त॒व॒न॒स्यामिति॑ सजात - व॒न॒स्याम् । आ शा᳚स्ते । शा॒स्त॒ इति॑ । इत्या॑ह । आ॒ह॒ प्रा॒णाः । प्रा॒णा वै । प्रा॒णा इति॑ प्र - अ॒नाः । वै स॑जा॒ताः । स॒जा॒ताः प्रा॒णान् । स॒जा॒ता इति॑ स - जा॒ताः । प्रा॒णाने॒व ( ) । प्रा॒णानिति॑ प्र - अ॒नान् । ए॒व न \newline

\textbf{Jatai Paata} \newline

1. दे॒वता॒ स्ता स्ता दे॒वता॑ दे॒वता॒ स्ताः । \newline
2. ता अ॑रीरधामा रीरधाम॒ तास्ता अ॑रीरधाम । \newline
3. अ॒री॒र॒धा॒मे तीत्य॑रीरधामा रीरधा॒मे ति॑ । \newline
4. इति॒ वाव वावे तीति॒ वाव । \newline
5. वावैत दे॒तद् वाव वावैतत् । \newline
6. ए॒त दा॑हाहै॒त दे॒तदा॑ह । \newline
7. आ॒ह॒ यद् यदा॑हाह॒ यत् । \newline
8. यन् न न यद् यन् न । \newline
9. न नि॑र्दि॒शेन् नि॑र्दि॒शेन् न न नि॑र्दि॒शेत् । \newline
10. नि॒र्दि॒शेत् प्रति॑वेश॒म् प्रति॑वेशम् निर्दि॒शेन् नि॑र्दि॒शेत् प्रति॑वेशम् । \newline
11. नि॒र्दि॒शेदिति॑ निः - दि॒शेत् । \newline
12. प्रति॑वेशं ॅय॒ज्ञ्स्य॑ य॒ज्ञ्स्य॒ प्रति॑वेश॒म् प्रति॑वेशं ॅय॒ज्ञ्स्य॑ । \newline
13. प्रति॑वेश॒मिति॒ प्रति॑ - वे॒श॒म् । \newline
14. य॒ज्ञ्स्या॒शी रा॒शीर् य॒ज्ञ्स्य॑ य॒ज्ञ्स्या॒शीः । \newline
15. आ॒शीर् ग॑च्छेद् गच्छे दा॒शी रा॒शीर् ग॑च्छेत् । \newline
16. आ॒शीरित्या᳚ - शीः । \newline
17. ग॒च्छे॒दा ग॑च्छेद् गच्छे॒दा । \newline
18. आ शा᳚स्ते शास्त॒ आ शा᳚स्ते । \newline
19. शा॒स्ते॒ ऽय म॒यꣳ शा᳚स्ते शास्ते॒ ऽयम् । \newline
20. अ॒यं ॅयज॑मानो॒ यज॑मानो॒ ऽय म॒यं ॅयज॑मानः । \newline
21. यज॑मानो॒ ऽसा व॒सौ यज॑मानो॒ यज॑मानो॒ ऽसौ । \newline
22. अ॒सा विती त्य॒सा व॒सा विति॑ । \newline
23. इत्या॑हा॒हे तीत्या॑ह । \newline
24. आ॒ह॒ नि॒र्दिश्य॑ नि॒र्दिश्या॑ हाह नि॒र्दिश्य॑ । \newline
25. नि॒र्दिश्यै॒ वैव नि॒र्दिश्य॑ नि॒र्दिश्यै॒व । \newline
26. नि॒र्दिश्येति॑ निः - दिश्य॑ । \newline
27. ए॒वैन॑ मेन मे॒वैवैन᳚म् । \newline
28. ए॒नꣳ॒॒ सु॒व॒र्गꣳ सु॑व॒र्ग मे॑न मेनꣳ सुव॒र्गम् । \newline
29. सु॒व॒र्गम् ॅलो॒कम् ॅलो॒कꣳ सु॑व॒र्गꣳ सु॑व॒र्गम् ॅलो॒कम् । \newline
30. सु॒व॒र्गमिति॑ सुवः - गम् । \newline
31. लो॒कम् ग॑मयति गमयति लो॒कम् ॅलो॒कम् ग॑मयति । \newline
32. ग॒म॒य॒ त्यायु॒ रायु॑र् गमयति गमय॒ त्यायुः॑ । \newline
33. आयु॒रा ऽऽयु॒ रायु॒रा । \newline
34. आ शा᳚स्ते शास्त॒ आ शा᳚स्ते । \newline
35. शा॒स्ते॒ सु॒प्र॒जा॒स्त्वꣳ सु॑प्रजा॒स्त्वꣳ शा᳚स्ते शास्ते सुप्रजा॒स्त्वम् । \newline
36. सु॒प्र॒जा॒स्त्व मा सु॑प्रजा॒स्त्वꣳ सु॑प्रजा॒स्त्व मा । \newline
37. सु॒प्र॒जा॒स्त्वमिति॑ सुप्रजाः - त्वम् । \newline
38. आ शा᳚स्ते शास्त॒ आ शा᳚स्ते । \newline
39. शा॒स्त॒ इतीति॑ शास्ते शास्त॒ इति॑ । \newline
40. इत्या॑हा॒हे तीत्या॑ह । \newline
41. आ॒हा॒शिष॑ मा॒शिष॑ माहाहा॒ शिष᳚म् । \newline
42. आ॒शिष॑ मे॒वैवाशिष॑ मा॒शिष॑ मे॒व । \newline
43. आ॒शिष॒मित्या᳚ - शिष᳚म् । \newline
44. ए॒वैता मे॒ता मे॒वैवैताम् । \newline
45. ए॒ता मैता मे॒ता मा । \newline
46. आ शा᳚स्ते शास्त॒ आ शा᳚स्ते । \newline
47. शा॒स्ते॒ स॒जा॒त॒व॒न॒स्याꣳ स॑जातवन॒स्याꣳ शा᳚स्ते शास्ते सजातवन॒स्याम् । \newline
48. स॒जा॒त॒व॒न॒स्या मा स॑जातवन॒स्याꣳ स॑जातवन॒स्या मा । \newline
49. स॒जा॒त॒व॒न॒स्यामिति॑ सजात - व॒न॒स्याम् । \newline
50. आ शा᳚स्ते शास्त॒ आ शा᳚स्ते । \newline
51. शा॒स्त॒ इतीति॑ शास्ते शास्त॒ इति॑ । \newline
52. इत्या॑हा॒हे तीत्या॑ह । \newline
53. आ॒ह॒ प्रा॒णाः प्रा॒णा आ॑हाह प्रा॒णाः । \newline
54. प्रा॒णा वै वै प्रा॒णाः प्रा॒णा वै । \newline
55. प्रा॒णा इति॑ प्र - अ॒नाः । \newline
56. वै स॑जा॒ताः स॑जा॒ता वै वै स॑जा॒ताः । \newline
57. स॒जा॒ताः प्रा॒णान् प्रा॒णान् थ्स॑जा॒ताः स॑जा॒ताः प्रा॒णान् । \newline
58. स॒जा॒ता इति॑ स - जा॒ताः । \newline
59. प्रा॒णा ने॒वैव प्रा॒णान् प्रा॒णा ने॒व । \newline
60. प्रा॒णानिति॑ प्र - अ॒नान् । \newline
61. ए॒व न नैवैव न । \newline

\textbf{Ghana Paata } \newline

1. दे॒वता॒ स्तास्ता दे॒वता॑ दे॒वता॒ स्ता अ॑रीरधामा रीरधाम॒ ता दे॒वता॑ दे॒वता॒ स्ता अ॑रीरधाम । \newline
2. ता अ॑रीरधामा रीरधाम॒ ता स्ता अ॑रीरधा॒मे तीत्य॑रीरधाम॒ ता स्ता अ॑रीरधा॒मे ति॑ । \newline
3. अ॒री॒र॒धा॒मे तीत्य॑रीरधामा रीरधा॒मे ति॒ वाव वावे त्य॑रीरधामा रीरधा॒मे ति॒ वाव । \newline
4. इति॒ वाव वावे तीति॒ वावैत दे॒तद् वावे तीति॒ वावैतत् । \newline
5. वावैत दे॒तद् वाव वावैत दा॑हा है॒तद् वाव वावैतदा॑ह । \newline
6. ए॒त दा॑हाहै॒त दे॒तदा॑ह॒ यद् यदा॑है॒त दे॒तदा॑ह॒ यत् । \newline
7. आ॒ह॒ यद् यदा॑हाह॒ यन् न न यदा॑हाह॒ यन् न । \newline
8. यन् न न यद् यन् न नि॑र्दि॒शेन् नि॑र्दि॒शेन् न यद् यन् न नि॑र्दि॒शेत् । \newline
9. न नि॑र्दि॒शेन् नि॑र्दि॒शेन् न न नि॑र्दि॒शेत् प्रति॑वेश॒म् प्रति॑वेशम् निर्दि॒शेन् न न नि॑र्दि॒शेत् प्रति॑वेशम् । \newline
10. नि॒र्दि॒शेत् प्रति॑वेश॒म् प्रति॑वेशम् निर्दि॒शेन् नि॑र्दि॒शेत् प्रति॑वेशं ॅय॒ज्ञ्स्य॑ य॒ज्ञ्स्य॒ प्रति॑वेशम् निर्दि॒शेन् नि॑र्दि॒शेत् प्रति॑वेशं ॅय॒ज्ञ्स्य॑ । \newline
11. नि॒र्दि॒शेदिति॑ निः - दि॒शेत् । \newline
12. प्रति॑वेशं ॅय॒ज्ञ्स्य॑ य॒ज्ञ्स्य॒ प्रति॑वेश॒म् प्रति॑वेशं ॅय॒ज्ञ्स्या॒ शीरा॒ शीर् य॒ज्ञ्स्य॒ प्रति॑वेश॒म् प्रति॑वेशं ॅय॒ज्ञ्स्या॒शीः । \newline
13. प्रति॑वेश॒मिति॒ प्रति॑ - वे॒श॒म् । \newline
14. य॒ज्ञ्स्या॒ शीरा॒ शीर् य॒ज्ञ्स्य॑ य॒ज्ञ्स्या॒शीर् ग॑च्छेद् गच्छे दा॒शीर् य॒ज्ञ्स्य॑ य॒ज्ञ्स्या॒शीर् ग॑च्छेत् । \newline
15. आ॒शीर् ग॑च्छेद् गच्छे दा॒शी रा॒शीर् ग॑च्छे॒दा ग॑च्छे दा॒शी रा॒शीर् ग॑च्छे॒दा । \newline
16. आ॒शीरित्या᳚ - शीः । \newline
17. ग॒च्छे॒दा ग॑च्छेद् गच्छे॒दा शा᳚स्ते शास्त॒ आ ग॑च्छेद् गच्छे॒दा शा᳚स्ते । \newline
18. आ शा᳚स्ते शास्त॒ आ शा᳚स्ते॒ ऽय म॒यꣳ शा᳚स्त॒ आ शा᳚स्ते॒ ऽयम् । \newline
19. शा॒स्ते॒ ऽय म॒यꣳ शा᳚स्ते शास्ते॒ ऽयं ॅयज॑मानो॒ यज॑मानो॒ ऽयꣳ शा᳚स्ते शास्ते॒ ऽयं ॅयज॑मानः । \newline
20. अ॒यं ॅयज॑मानो॒ यज॑मानो॒ ऽय म॒यं ॅयज॑मानो॒ ऽसा व॒सौ यज॑मानो॒ ऽय म॒यं ॅयज॑मानो॒ ऽसौ । \newline
21. यज॑मानो॒ ऽसा व॒सौ यज॑मानो॒ यज॑मानो॒ ऽसा विती त्य॒सौ यज॑मानो॒ यज॑मानो॒ ऽसा विति॑ । \newline
22. अ॒सा विती त्य॒सा व॒सा वित्या॑हा॒हे त्य॒सा व॒सा वित्या॑ह । \newline
23. इत्या॑हा॒हे तीत्या॑ह नि॒र्दिश्य॑ नि॒र्दिश्या॒हे तीत्या॑ह नि॒र्दिश्य॑ । \newline
24. आ॒ह॒ नि॒र्दिश्य॑ नि॒र्दिश्या॑हाह नि॒र्दिश्यै॒वैव नि॒र्दिश्या॑हाह नि॒र्दिश्यै॒व । \newline
25. नि॒र्दिश्यै॒वैव नि॒र्दिश्य॑ नि॒र्दिश्यै॒वैन॑ मेन मे॒व नि॒र्दिश्य॑ नि॒र्दिश्यै॒वैन᳚म् । \newline
26. नि॒र्दिश्येति॑ निः - दिश्य॑ । \newline
27. ए॒वैन॑ मेन मे॒वैवैनꣳ॑ सुव॒र्गꣳ सु॑व॒र्ग मे॑न मे॒वैवैनꣳ॑ सुव॒र्गम् । \newline
28. ए॒नꣳ॒॒ सु॒व॒र्गꣳ सु॑व॒र्ग मे॑न मेनꣳ सुव॒र्गम् ॅलो॒कम् ॅलो॒कꣳ सु॑व॒र्ग मे॑न मेनꣳ सुव॒र्गम् ॅलो॒कम् । \newline
29. सु॒व॒र्गम् ॅलो॒कम् ॅलो॒कꣳ सु॑व॒र्गꣳ सु॑व॒र्गम् ॅलो॒कम् ग॑मयति गमयति लो॒कꣳ सु॑व॒र्गꣳ सु॑व॒र्गम् ॅलो॒कम् ग॑मयति । \newline
30. सु॒व॒र्गमिति॑ सुवः - गम् । \newline
31. लो॒कम् ग॑मयति गमयति लो॒कम् ॅलो॒कम् ग॑मय॒ त्यायु॒रायु॑र् गमयति लो॒कम् ॅलो॒कम् ग॑मय॒ त्यायुः॑ । \newline
32. ग॒म॒य॒ त्यायु॒ रायु॑र् गमयति गमय॒ त्यायु॒रा ऽऽयु॑र् गमयति गमय॒ त्यायु॒रा । \newline
33. आयु॒रा ऽऽयु॒ रायु॒रा शा᳚स्ते शास्त॒ आ ऽऽयु॒ रायु॒रा शा᳚स्ते । \newline
34. आ शा᳚स्ते शास्त॒ आ शा᳚स्ते सुप्रजा॒स्त्वꣳ सु॑प्रजा॒स्त्वꣳ शा᳚स्त॒ आ शा᳚स्ते सुप्रजा॒स्त्वम् । \newline
35. शा॒स्ते॒ सु॒प्र॒जा॒स्त्वꣳ सु॑प्रजा॒स्त्वꣳ शा᳚स्ते शास्ते सुप्रजा॒स्त्व मा सु॑प्रजा॒स्त्वꣳ शा᳚स्ते शास्ते सुप्रजा॒स्त्व मा । \newline
36. सु॒प्र॒जा॒स्त्व मा सु॑प्रजा॒स्त्वꣳ सु॑प्रजा॒स्त्व मा शा᳚स्ते शास्त॒ आ सु॑प्रजा॒स्त्वꣳ सु॑प्रजा॒स्त्व मा शा᳚स्ते । \newline
37. सु॒प्र॒जा॒स्त्वमिति॑ सुप्रजाः - त्वम् । \newline
38. आ शा᳚स्ते शास्त॒ आ शा᳚स्त॒ इतीति॑ शास्त॒ आ शा᳚स्त॒ इति॑ । \newline
39. शा॒स्त॒ इतीति॑ शास्ते शास्त॒ इत्या॑हा॒हे ति॑ शास्ते शास्त॒ इत्या॑ह । \newline
40. इत्या॑हा॒हे तीत्या॑हा॒शिष॑ मा॒शिष॑ मा॒हे तीत्या॑हा॒शिष᳚म् । \newline
41. आ॒हा॒शिष॑ मा॒शिष॑ माहाहा॒शिष॑ मे॒वैवाशिष॑ माहाहा॒ शिष॑ मे॒व । \newline
42. आ॒शिष॑ मे॒वैवा शिष॑ मा॒शिष॑ मे॒वैता मे॒ता मे॒वाशिष॑ मा॒शिष॑ मे॒वैताम् । \newline
43. आ॒शिष॒मित्या᳚ - शिष᳚म् । \newline
44. ए॒वैता मे॒ता मे॒वैवैता मैता मे॒वैवैता मा । \newline
45. ए॒ता मैता मे॒ता मा शा᳚स्ते शास्त॒ ऐता मे॒ता मा शा᳚स्ते । \newline
46. आ शा᳚स्ते शास्त॒ आ शा᳚स्ते सजातवन॒स्याꣳ स॑जातवन॒स्याꣳ शा᳚स्त॒ आ शा᳚स्ते सजातवन॒स्याम् । \newline
47. शा॒स्ते॒ स॒जा॒त॒व॒न॒स्याꣳ स॑जातवन॒स्याꣳ शा᳚स्ते शास्ते सजातवन॒स्या मा स॑जातवन॒स्याꣳ शा᳚स्ते शास्ते सजातवन॒स्या मा । \newline
48. स॒जा॒त॒व॒न॒स्या मा स॑जातवन॒स्याꣳ स॑जातवन॒स्या मा शा᳚स्ते शास्त॒ आ स॑जातवन॒स्याꣳ स॑जातवन॒स्या मा शा᳚स्ते । \newline
49. स॒जा॒त॒व॒न॒स्यामिति॑ सजात - व॒न॒स्याम् । \newline
50. आ शा᳚स्ते शास्त॒ आ शा᳚स्त॒ इतीति॑ शास्त॒ आ शा᳚स्त॒ इति॑ । \newline
51. शा॒स्त॒ इतीति॑ शास्ते शास्त॒ इत्या॑हा॒हे ति॑ शास्ते शास्त॒ इत्या॑ह । \newline
52. इत्या॑हा॒हे तीत्या॑ह प्रा॒णाः प्रा॒णा आ॒हे तीत्या॑ह प्रा॒णाः । \newline
53. आ॒ह॒ प्रा॒णाः प्रा॒णा आ॑हाह प्रा॒णा वै वै प्रा॒णा आ॑हाह प्रा॒णा वै । \newline
54. प्रा॒णा वै वै प्रा॒णाः प्रा॒णा वै स॑जा॒ताः स॑जा॒ता वै प्रा॒णाः प्रा॒णा वै स॑जा॒ताः । \newline
55. प्रा॒णा इति॑ प्र - अ॒नाः । \newline
56. वै स॑जा॒ताः स॑जा॒ता वै वै स॑जा॒ताः प्रा॒णान् प्रा॒णान् थ्स॑जा॒ता वै वै स॑जा॒ताः प्रा॒णान् । \newline
57. स॒जा॒ताः प्रा॒णान् प्रा॒णान् थ्स॑जा॒ताः स॑जा॒ताः प्रा॒णा ने॒वैव प्रा॒णान् थ्स॑जा॒ताः स॑जा॒ताः प्रा॒णा ने॒व । \newline
58. स॒जा॒ता इति॑ स - जा॒ताः । \newline
59. प्रा॒णा ने॒वैव प्रा॒णान् प्रा॒णा ने॒व न नैव प्रा॒णान् प्रा॒णा ने॒व न । \newline
60. प्रा॒णानिति॑ प्र - अ॒नान् । \newline
61. ए॒व न नैवैव नान्त र॒न्तर् नैवैव नान्तः । \newline
\pagebreak
\markright{ TS 2.6.9.8  \hfill https://www.vedavms.in \hfill}
\addcontentsline{toc}{section}{ TS 2.6.9.8 }
\section*{ TS 2.6.9.8 }

\textbf{TS 2.6.9.8 } \newline
\textbf{Samhita Paata} \newline

नान्तरे॑ति॒ तद॒ग्निर्दे॒वो दे॒वेभ्यो॒ वन॑ते व॒यम॒ग्नेर्मानु॑षा॒ इत्या॑हा॒ग्निर्दे॒वेभ्यो॑ वनु॒ते व॒यं म॑नु॒ष्ये᳚भ्य॒ इति॒ वावैतदा॑हे॒ह गति॑र्वा॒मस्ये॒दं च॒ नमो॑ दे॒वेभ्य॒ इत्या॑ह॒ याश्चै॒व दे॒वता॒ यज॑ति॒ याश्च॒ न ताभ्य॑ ए॒वोभयी᳚भ्यो॒ नम॑स्करोत्या॒त्मनोऽना᳚र्त्यै ॥ \newline

\textbf{Pada Paata} \newline

न । अ॒न्तः । ए॒ति॒ । तत् । अ॒ग्निः । दे॒वः । दे॒वेभ्यः॑ । वन॑ते । व॒यम् । अ॒ग्नेः । मानु॑षाः । इति॑ । आ॒ह॒ । अ॒ग्निः । दे॒वेभ्यः॑ । व॒नु॒ते । व॒यम् । म॒नु॒ष्ये᳚भ्यः । इति॑ । वाव । ए॒तत् । आ॒ह॒ । इ॒ह । गतिः॑ । वा॒मस्य॑ । इ॒दम् । च॒ । नमः॑ । दे॒वेभ्यः॑ । इति॑ । आ॒ह॒ । याः । च॒ । ए॒व । दे॒वताः᳚ । यज॑ति । याः । च॒ । न । ताभ्यः॑ । ए॒व । उ॒भयी᳚भ्यः । नमः॑ । क॒रो॒ति॒ । आ॒त्मनः॑ । अना᳚र्त्यै ॥  \newline


\textbf{Krama Paata} \newline

नान्तः । अ॒न्तरे॑ति । ए॒ति॒ तत् । तद॒ग्निः । अ॒ग्निर् दे॒वः । दे॒वो दे॒वेभ्यः॑ । दे॒वेभ्यो॒ वन॑ते । वन॑ते व॒यम् । व॒यम॒ग्नेः । अ॒ग्नेर् मानु॑षाः । मानु॑षा॒ इति॑ । इत्या॑ह । आ॒हा॒ग्निः । अ॒ग्निर् दे॒वेभ्यः॑ । दे॒वेभ्यो॑ वनु॒ते । व॒नु॒ते व॒यम् । व॒यम् म॑नु॒ष्ये᳚भ्यः । म॒नु॒ष्ये᳚भ्य॒ इति॑ । इति॒ वाव । वावैतत् । ए॒तदा॑ह । आ॒हे॒ह । इ॒ह गतिः॑ । गति॑र् वा॒मस्य॑ । वा॒मस्ये॒दम् । इ॒दम् च॑ । च॒ नमः॑ । नमो॑ दे॒वेभ्यः॑ । दे॒वेभ्य॒ इति॑ । इत्या॑ह । आ॒ह॒ याः । याश्च॑ । चै॒व । ए॒व दे॒वताः᳚ । दे॒वता॒ यज॑ति । यज॑ति॒ याः । याश्च॑ । च॒ न । न ताभ्यः॑ । ताभ्य॑ ए॒व । ए॒वोभयी᳚भ्यः । उ॒भयी᳚भ्यो॒ नमः॑ । नम॑स्करोति । क॒रो॒त्या॒त्मनः॑ । आ॒त्मनोऽना᳚र्त्यै । अना᳚र्त्या॒ इत्यना᳚र्त्यै । \newline

\textbf{Jatai Paata} \newline

1. नान्त र॒न्तर् न नान्तः । \newline
2. अ॒न्त रे᳚त्ये त्य॒न्त र॒न्त रे॑ति । \newline
3. ए॒ति॒ तत् तदे᳚ त्येति॒ तत् । \newline
4. तद॒ग्नि र॒ग्नि स्तत् तद॒ग्निः । \newline
5. अ॒ग्निर् दे॒वो दे॒वो᳚ ऽग्नि र॒ग्निर् दे॒वः । \newline
6. दे॒वो दे॒वेभ्यो॑ दे॒वेभ्यो॑ दे॒वो दे॒वो दे॒वेभ्यः॑ । \newline
7. दे॒वेभ्यो॒ वन॑ते॒ वन॑ते दे॒वेभ्यो॑ दे॒वेभ्यो॒ वन॑ते । \newline
8. वन॑ते व॒यं ॅव॒यं ॅवन॑ते॒ वन॑ते व॒यम् । \newline
9. व॒य म॒ग्ने र॒ग्नेर् व॒यं ॅव॒य म॒ग्नेः । \newline
10. अ॒ग्नेर् मानु॑षा॒ मानु॑षा अ॒ग्ने र॒ग्नेर् मानु॑षाः । \newline
11. मानु॑षा॒ इतीति॒ मानु॑षा॒ मानु॑षा॒ इति॑ । \newline
12. इत्या॑हा॒हे तीत्या॑ह । \newline
13. आ॒हा॒ ग्नि र॒ग्नि रा॑हाहा॒ ग्निः । \newline
14. अ॒ग्निर् दे॒वेभ्यो॑ दे॒वेभ्यो॒ ऽग्नि र॒ग्निर् दे॒वेभ्यः॑ । \newline
15. दे॒वेभ्यो॑ वनु॒ते व॑नु॒ते दे॒वेभ्यो॑ दे॒वेभ्यो॑ वनु॒ते । \newline
16. व॒नु॒ते व॒यं ॅव॒यं ॅव॑नु॒ते व॑नु॒ते व॒यम् । \newline
17. व॒यम् म॑नु॒ष्ये᳚भ्यो मनु॒ष्ये᳚भ्यो व॒यं ॅव॒यम् म॑नु॒ष्ये᳚भ्यः । \newline
18. म॒नु॒ष्ये᳚भ्य॒ इतीति॑ मनु॒ष्ये᳚भ्यो मनु॒ष्ये᳚भ्य॒ इति॑ । \newline
19. इति॒ वाव वावे तीति॒ वाव । \newline
20. वावैत दे॒तद् वाव वावैतत् । \newline
21. ए॒त दा॑हाहै॒त दे॒त दा॑ह । \newline
22. आ॒हे॒ हे हाहा॑हे॒ ह । \newline
23. इ॒ह गति॒र् गति॑ रि॒हे ह गतिः॑ । \newline
24. गति॑र् वा॒मस्य॑ वा॒मस्य॒ गति॒र् गति॑र् वा॒मस्य॑ । \newline
25. वा॒मस्ये॒ द मि॒दं ॅवा॒मस्य॑ वा॒मस्ये॒ दम् । \newline
26. इ॒दम् च॑ चे॒ द मि॒दम् च॑ । \newline
27. च॒ नमो॒ नम॑श्च च॒ नमः॑ । \newline
28. नमो॑ दे॒वेभ्यो॑ दे॒वेभ्यो॒ नमो॒ नमो॑ दे॒वेभ्यः॑ । \newline
29. दे॒वेभ्य॒ इतीति॑ दे॒वेभ्यो॑ दे॒वेभ्य॒ इति॑ । \newline
30. इत्या॑हा॒हे तीत्या॑ह । \newline
31. आ॒ह॒ या या आ॑हाह॒ याः । \newline
32. याश्च॑ च॒ या याश्च॑ । \newline
33. चै॒वैव च॑ चै॒व । \newline
34. ए॒व दे॒वता॑ दे॒वता॑ ए॒वैव दे॒वताः᳚ । \newline
35. दे॒वता॒ यज॑ति॒ यज॑ति दे॒वता॑ दे॒वता॒ यज॑ति । \newline
36. यज॑ति॒ या या यज॑ति॒ यज॑ति॒ याः । \newline
37. याश्च॑ च॒ या याश्च॑ । \newline
38. च॒ न न च॑ च॒ न । \newline
39. न ताभ्य॒ स्ताभ्यो॒ न न ताभ्यः॑ । \newline
40. ताभ्य॑ ए॒वैव ताभ्य॒ स्ताभ्य॑ ए॒व । \newline
41. ए॒वोभयी᳚भ्य उ॒भयी᳚भ्य ए॒वै वोभयी᳚भ्यः । \newline
42. उ॒भयी᳚भ्यो॒ नमो॒ नम॑ उ॒भयी᳚भ्य उ॒भयी᳚भ्यो॒ नमः॑ । \newline
43. नम॑स् करोति करोति॒ नमो॒ नम॑स् करोति । \newline
44. क॒रो॒ त्या॒त्मन॑ आ॒त्मनः॑ करोति करो त्या॒त्मनः॑ । \newline
45. आ॒त्मनो ऽना᳚र्त्या॒ अना᳚र्त्या आ॒त्मन॑ आ॒त्मनो ऽना᳚र्त्यै । \newline
46. अना᳚र्त्या॒ इत्यना᳚र्त्यै । \newline

\textbf{Ghana Paata } \newline

1. नान्त र॒न्तर् न नान्त रे᳚त्येत्य॒न्तर् न नान्त रे॑ति । \newline
2. अ॒न्त रे᳚त्ये त्य॒न्त र॒न्त रे॑ति॒ तत् तदे᳚ त्य॒न्त र॒न्त रे॑ति॒ तत् । \newline
3. ए॒ति॒ तत् तदे᳚त्येति॒ तद॒ग्नि र॒ग्नि स्तदे᳚त्येति॒ तद॒ग्निः । \newline
4. तद॒ग्नि र॒ग्नि स्तत् तद॒ग्निर् दे॒वो दे॒वो᳚ ऽग्नि स्तत् तद॒ग्निर् दे॒वः । \newline
5. अ॒ग्निर् दे॒वो दे॒वो᳚ ऽग्नि र॒ग्निर् दे॒वो दे॒वेभ्यो॑ दे॒वेभ्यो॑ दे॒वो᳚ ऽग्नि र॒ग्निर् दे॒वो दे॒वेभ्यः॑ । \newline
6. दे॒वो दे॒वेभ्यो॑ दे॒वेभ्यो॑ दे॒वो दे॒वो दे॒वेभ्यो॒ वन॑ते॒ वन॑ते दे॒वेभ्यो॑ दे॒वो दे॒वो दे॒वेभ्यो॒ वन॑ते । \newline
7. दे॒वेभ्यो॒ वन॑ते॒ वन॑ते दे॒वेभ्यो॑ दे॒वेभ्यो॒ वन॑ते व॒यं ॅव॒यं ॅवन॑ते दे॒वेभ्यो॑ दे॒वेभ्यो॒ वन॑ते व॒यम् । \newline
8. वन॑ते व॒यं ॅव॒यं ॅवन॑ते॒ वन॑ते व॒य म॒ग्ने र॒ग्नेर् व॒यं ॅवन॑ते॒ वन॑ते व॒य म॒ग्नेः । \newline
9. व॒य म॒ग्ने र॒ग्नेर् व॒यं ॅव॒य म॒ग्नेर् मानु॑षा॒ मानु॑षा अ॒ग्नेर् व॒यं ॅव॒य म॒ग्नेर् मानु॑षाः । \newline
10. अ॒ग्नेर् मानु॑षा॒ मानु॑षा अ॒ग्ने र॒ग्नेर् मानु॑षा॒ इतीति॒ मानु॑षा अ॒ग्ने र॒ग्नेर् मानु॑षा॒ इति॑ । \newline
11. मानु॑षा॒ इतीति॒ मानु॑षा॒ मानु॑षा॒ इत्या॑हा॒हे ति॒ मानु॑षा॒ मानु॑षा॒ इत्या॑ह । \newline
12. इत्या॑हा॒हे तीत्या॑हा॒ग्नि र॒ग्निरा॒हे तीत्या॑हा॒ग्निः । \newline
13. आ॒हा॒ग्नि र॒ग्नि रा॑हाहा॒ग्निर् दे॒वेभ्यो॑ दे॒वेभ्यो॒ ऽग्नि रा॑हाहा॒ग्निर् दे॒वेभ्यः॑ । \newline
14. अ॒ग्निर् दे॒वेभ्यो॑ दे॒वेभ्यो॒ ऽग्नि र॒ग्निर् दे॒वेभ्यो॑ वनु॒ते व॑नु॒ते दे॒वेभ्यो॒ ऽग्नि र॒ग्निर् दे॒वेभ्यो॑ वनु॒ते । \newline
15. दे॒वेभ्यो॑ वनु॒ते व॑नु॒ते दे॒वेभ्यो॑ दे॒वेभ्यो॑ वनु॒ते व॒यं ॅव॒यं ॅव॑नु॒ते दे॒वेभ्यो॑ दे॒वेभ्यो॑ वनु॒ते व॒यम् । \newline
16. व॒नु॒ते व॒यं ॅव॒यं ॅव॑नु॒ते व॑नु॒ते व॒यम् म॑नु॒ष्ये᳚भ्यो मनु॒ष्ये᳚भ्यो व॒यं ॅव॑नु॒ते व॑नु॒ते व॒यम् म॑नु॒ष्ये᳚भ्यः । \newline
17. व॒यम् म॑नु॒ष्ये᳚भ्यो मनु॒ष्ये᳚भ्यो व॒यं ॅव॒यम् म॑नु॒ष्ये᳚भ्य॒ इतीति॑ मनु॒ष्ये᳚भ्यो व॒यं ॅव॒यम् म॑नु॒ष्ये᳚भ्य॒ इति॑ । \newline
18. म॒नु॒ष्ये᳚भ्य॒ इतीति॑ मनु॒ष्ये᳚भ्यो मनु॒ष्ये᳚भ्य॒ इति॒ वाव वावे ति॑ मनु॒ष्ये᳚भ्यो मनु॒ष्ये᳚भ्य॒ इति॒ वाव । \newline
19. इति॒ वाव वावे तीति॒ वावैत दे॒तद् वावे तीति॒ वावैतत् । \newline
20. वावैत दे॒तद् वाव वावै तदा॑हा है॒तद् वाव वावै तदा॑ह । \newline
21. ए॒त दा॑हा है॒त दे॒त दा॑हे॒ हे हाहै॒ तदे॒त दा॑हे॒ ह । \newline
22. आ॒हे॒ हे हाहा॑हे॒ ह गति॒र् गति॑ रि॒हाहा॑हे॒ ह गतिः॑ । \newline
23. इ॒ह गति॒र् गति॑रि॒हे ह गति॑र् वा॒मस्य॑ वा॒मस्य॒ गति॑रि॒हे ह गति॑र् वा॒मस्य॑ । \newline
24. गति॑र् वा॒मस्य॑ वा॒मस्य॒ गति॒र् गति॑र् वा॒मस्ये॒ द मि॒दं ॅवा॒मस्य॒ गति॒र् गति॑र् वा॒मस्ये॒ दम् । \newline
25. वा॒मस्ये॒ द मि॒दं ॅवा॒मस्य॑ वा॒मस्ये॒ दम् च॑ चे॒ दं ॅवा॒मस्य॑ वा॒मस्ये॒ दम् च॑ । \newline
26. इ॒दम् च॑ चे॒ द मि॒दम् च॒ नमो॒ नम॑श्चे॒ द मि॒दम् च॒ नमः॑ । \newline
27. च॒ नमो॒ नम॑श्च च॒ नमो॑ दे॒वेभ्यो॑ दे॒वेभ्यो॒ नम॑श्च च॒ नमो॑ दे॒वेभ्यः॑ । \newline
28. नमो॑ दे॒वेभ्यो॑ दे॒वेभ्यो॒ नमो॒ नमो॑ दे॒वेभ्य॒ इतीति॑ दे॒वेभ्यो॒ नमो॒ नमो॑ दे॒वेभ्य॒ इति॑ । \newline
29. दे॒वेभ्य॒ इतीति॑ दे॒वेभ्यो॑ दे॒वेभ्य॒ इत्या॑हा॒हे ति॑ दे॒वेभ्यो॑ दे॒वेभ्य॒ इत्या॑ह । \newline
30. इत्या॑हा॒हे तीत्या॑ह॒ या या आ॒हे तीत्या॑ह॒ याः । \newline
31. आ॒ह॒ या या आ॑हाह॒ याश्च॑ च॒ या आ॑हाह॒ याश्च॑ । \newline
32. याश्च॑ च॒ या याश्चै॒वैव च॒ या याश्चै॒व । \newline
33. चै॒वैव च॑ चै॒व दे॒वता॑ दे॒वता॑ ए॒व च॑ चै॒व दे॒वताः᳚ । \newline
34. ए॒व दे॒वता॑ दे॒वता॑ ए॒वैव दे॒वता॒ यज॑ति॒ यज॑ति दे॒वता॑ ए॒वैव दे॒वता॒ यज॑ति । \newline
35. दे॒वता॒ यज॑ति॒ यज॑ति दे॒वता॑ दे॒वता॒ यज॑ति॒ या या यज॑ति दे॒वता॑ दे॒वता॒ यज॑ति॒ याः । \newline
36. यज॑ति॒ या या यज॑ति॒ यज॑ति॒ याश्च॑ च॒ या यज॑ति॒ यज॑ति॒ याश्च॑ । \newline
37. याश्च॑ च॒ या याश्च॒ न न च॒ या याश्च॒ न । \newline
38. च॒ न न च॑ च॒ न ताभ्य॒ स्ताभ्यो॒ न च॑ च॒ न ताभ्यः॑ । \newline
39. न ताभ्य॒ स्ताभ्यो॒ न न ताभ्य॑ ए॒वैव ताभ्यो॒ न न ताभ्य॑ ए॒व । \newline
40. ताभ्य॑ ए॒वैव ताभ्य॒ स्ताभ्य॑ ए॒वोभयी᳚भ्य उ॒भयी᳚भ्य ए॒व ताभ्य॒ स्ताभ्य॑ ए॒वोभयी᳚भ्यः । \newline
41. ए॒वोभयी᳚भ्य उ॒भयी᳚भ्य ए॒वैवो भयी᳚भ्यो॒ नमो॒ नम॑ उ॒भयी᳚भ्य ए॒वैवो भयी᳚भ्यो॒ नमः॑ । \newline
42. उ॒भयी᳚भ्यो॒ नमो॒ नम॑ उ॒भयी᳚भ्य उ॒भयी᳚भ्यो॒ नम॑ स्करोति करोति॒ नम॑ उ॒भयी᳚भ्य उ॒भयी᳚भ्यो॒ नम॑ स्करोति । \newline
43. नम॑ स्करोति करोति॒ नमो॒ नम॑ स्करो त्या॒त्मन॑ आ॒त्मनः॑ करोति॒ नमो॒ नम॑ स्करो त्या॒त्मनः॑ । \newline
44. क॒रो॒त्या॒त्मन॑ आ॒त्मनः॑ करोति करो त्या॒त्मनो ऽना᳚र्त्या॒ अना᳚र्त्या आ॒त्मनः॑ करोति करो त्या॒त्मनो ऽना᳚र्त्यै । \newline
45. आ॒त्मनो ऽना᳚र्त्या॒ अना᳚र्त्या आ॒त्मन॑ आ॒त्मनो ऽना᳚र्त्यै । \newline
46. अना᳚र्त्या॒ इत्यना᳚र्त्यै । \newline
\pagebreak
\markright{ TS 2.6.10.1  \hfill https://www.vedavms.in \hfill}
\addcontentsline{toc}{section}{ TS 2.6.10.1 }
\section*{ TS 2.6.10.1 }

\textbf{TS 2.6.10.1 } \newline
\textbf{Samhita Paata} \newline

दे॒वा वै य॒ज्ञ्स्य॑ स्वगाक॒र्तारं॒ नावि॑न्द॒न् ते शं॒ॅयुं बा॑र्.हस्प॒त्यम॑ब्रुवन्नि॒मं नो॑ य॒ज्ञ्ꣳ स्व॒गा कु॒र्विति॒ सो᳚ऽब्रवी॒द्वरं॑ ॅवृणै॒ यदे॒वा-ब्रा᳚ह्मणो॒क्तो-ऽश्र॑द्दधानो॒ यजा॑तै॒ सा मे॑ य॒ज्ञ्स्या॒ऽऽशीर॑स॒दिति॒ तस्मा॒द्-यद्-ब्रा᳚ह्मणो॒क्तोऽश्र॑द्दधानो॒ यज॑ते शं॒ॅयुमे॒व तस्य॑ बार्.हस्प॒त्यं ॅय॒ज्ञ्स्या॒ ऽऽशीर्ग॑च्छत्ये॒तन्ममेत्य॑ब्रवी॒त् किं मे᳚ प्र॒जाया॒ - [  ] \newline

\textbf{Pada Paata} \newline

दे॒वाः । वै । य॒ज्ञ्स्य॑ । स्व॒गा॒क॒र्तार॒मिति॑ स्वगा - क॒र्तार᳚म् । न । अ॒वि॒न्द॒न्न् । ते । शं॒ॅयुमिति॑ शं - युम् । बा॒र्॒.ह॒स्प॒त्य॒म् । अ॒ब्रु॒व॒न्न् । इ॒मम् । नः॒ । य॒ज्ञ्म् । स्व॒गेति॑ स्व - गा । कु॒रु॒ । इति॑ । सः । अ॒ब्र॒वी॒त् । वर᳚म् । वृ॒णै॒ । यत् । ए॒व । अब्रा᳚ह्मणोक्त॒ इत्यब्रा᳚ह्मण - उ॒क्तः॒ । अश्र॑द्दधान॒ इत्यश्र॑त् - द॒धा॒नः॒ । यजा॑तै । सा । मे॒ । य॒ज्ञ्स्य॑ । आ॒शीरित्या᳚ - शीः । अ॒स॒त् । इति॑ । तस्मा᳚त् । यत् । अब्रा᳚ह्मणोक्त॒ इत्यब्रा᳚ह्मण - उ॒क्तः॒ । अश्र॑द्दधान॒ इत्यश्र॑त् - द॒धा॒नः॒ । यज॑ते । शं॒ॅयुमिति॑ शं-युम् । ए॒व । तस्य॑ । बा॒र्॒.ह॒स्प॒त्यम् । य॒ज्ञ्स्य॑ । आ॒शीरित्या᳚ - शीः । ग॒च्छ॒ति॒ । ए॒तत् । मम॑ । इति॑ । अ॒ब्र॒वी॒त् । किम् । मे॒ । प्र॒जाया॒ इति॑ प्र - जायाः᳚ ।  \newline


\textbf{Krama Paata} \newline

दे॒वा वै । वै य॒ज्ञ्स्य॑ । य॒ज्ञ्स्य॑ स्वगाक॒र्तार᳚म् । स्व॒गा॒क॒र्तार॒म् न । स्व॒गा॒क॒र्तार॒मिति॑ स्वगा - क॒र्तार᳚म् । नावि॑न्दन्न् । अ॒वि॒न्द॒न् ते । ते श॒म्ॅयुम् । श॒म्ॅयुम् बा॑र्.हस्प॒त्यम् । श॒म्ॅयुमिति॑ शं - युम् । बा॒र्॒.ह॒स्प॒त्यम॑ब्रुवन्न् । अ॒ब्रु॒व॒न्नि॒मम् । इ॒मम् नः॑ । नो॒ य॒ज्ञ्म् । य॒ज्ञ्ꣳ स्व॒गा । स्व॒गा कु॑रु । स्व॒गेति॑ स्व - गा । कु॒र्विति॑ । इति॒ सः । सो᳚ऽब्रवीत् । अ॒ब्र॒वी॒द् वर᳚म् । वरं॑ ॅवृणै । वृ॒णै॒ यत् । यदे॒व । ए॒वाब्रा᳚ह्मणोक्तः । अब्रा᳚ह्मणो॒क्तो ऽश्र॑द्दधानः । अब्रा᳚ह्मणोक्त॒ इत्यब्रा᳚ह्मण - उ॒क्तः॒ । अश्र॑द्दधानो॒ यजा॑तै । अश्र॑द्दधान॒ इत्यश्र॑त् - द॒धा॒नः॒ । यजा॑तै॒ सा । सा मे᳚ । मे॒ य॒ज्ञ्स्य॑ । य॒ज्ञ्स्या॒शीः । आ॒शीर॑सत् । आ॒शीरित्या᳚ - शीः । अ॒स॒दिति॑ । इति॒ तस्मा᳚त् । तस्मा॒द् यत् । यदब्रा᳚ह्मणोक्तः । अब्रा᳚णो॒क्तो ऽश्र॑द्दधानः । अब्रा᳚ह्मणोक्त॒ इत्यब्रा᳚ह्मण - उ॒क्तः॒ । अश्र॑द्दधानो॒ यज॑ते । अश्र॑द्दधान॒ इत्यश्र॑त् - द॒धा॒नः॒ । यज॑ते श॒म्ॅयुम् । श॒म्ॅयुमे॒व । श॒म्ॅयुमिति॑ शम् - युम् । ए॒व तस्य॑ । तस्य॑ बार्.हस्प॒त्यम् । बा॒र्.॒ह॒स्प॒त्यं ॅय॒ज्ञ्स्य॑ । य॒ज्ञ्स्या॒शीः । आ॒शीर् ग॑च्छति । आ॒शीरित्या᳚ - शीः । ग॒च्छ॒त्ये॒तत् । ए॒तन् मम॑ । ममेति॑ । इत्य॑ब्रवीत् । अ॒ब्र॒वी॒त् किम् । किम् मे᳚ । मे॒ प्र॒जायाः᳚ । प्र॒जाया॒ इति॑ । प्र॒जाया॒ इति॑ प्र - जायाः᳚ \newline

\textbf{Jatai Paata} \newline

1. दे॒वा वै वै दे॒वा दे॒वा वै । \newline
2. वै य॒ज्ञ्स्य॑ य॒ज्ञ्स्य॒ वै वै य॒ज्ञ्स्य॑ । \newline
3. य॒ज्ञ्स्य॑ स्वगाक॒र्तारꣳ॑ स्वगाक॒र्तारं॑ ॅय॒ज्ञ्स्य॑ य॒ज्ञ्स्य॑ स्वगाक॒र्तार᳚म् । \newline
4. स्व॒गा॒क॒र्तार॒म् न न स्व॑गाक॒र्तारꣳ॑ स्वगाक॒र्तार॒म् न । \newline
5. स्व॒गा॒क॒र्तार॒मिति॑ स्वगा - क॒र्तार᳚म् । \newline
6. नावि॑न्दन् नविन्द॒न् न नावि॑न्दन्न् । \newline
7. अ॒वि॒न्द॒न् ते ते॑ ऽविन्दन् नविन्द॒न् ते । \newline
8. ते शं॒ॅयुꣳ शं॒ॅयुम् ते ते शं॒ॅयुम् । \newline
9. शं॒ॅयुम् बा॑र्.हस्प॒त्यम् बा॑र्.हस्प॒त्यꣳ शं॒ॅयुꣳ शं॒ॅयुम् बा॑र्.हस्प॒त्यम् । \newline
10. शं॒ॅयुमिति॑ शं - युम् । \newline
11. बा॒र्॒.ह॒स्प॒त्य म॑ब्रुवन् नब्रुवन् बार्.हस्प॒त्यम् बा॑र्.हस्प॒त्य म॑ब्रुवन्न् । \newline
12. अ॒ब्रु॒व॒न् नि॒म मि॒म म॑ब्रुवन् नब्रुवन् नि॒मम् । \newline
13. इ॒मम् नो॑ न इ॒म मि॒मम् नः॑ । \newline
14. नो॒ य॒ज्ञ्ं ॅय॒ज्ञ्म् नो॑ नो य॒ज्ञ्म् । \newline
15. य॒ज्ञ्ꣳ स्व॒गा स्व॒गा य॒ज्ञ्ं ॅय॒ज्ञ्ꣳ स्व॒गा । \newline
16. स्व॒गा कु॑रु कुरु स्व॒गा स्व॒गा कु॑रु । \newline
17. स्व॒गेति॑ स्व - गा । \newline
18. कु॒र्वितीति॑ कुरु कु॒र्विति॑ । \newline
19. इति॒ स स इतीति॒ सः । \newline
20. सो᳚ ऽब्रवी दब्रवी॒थ् स सो᳚ ऽब्रवीत् । \newline
21. अ॒ब्र॒वी॒द् वरं॒ ॅवर॑ मब्रवी दब्रवी॒द् वर᳚म् । \newline
22. वरं॑ ॅवृणै वृणै॒ वरं॒ ॅवरं॑ ॅवृणै । \newline
23. वृ॒णै॒ यद् यद् वृ॑णै वृणै॒ यत् । \newline
24. यदे॒वैव यद् यदे॒व । \newline
25. ए॒वाब्रा᳚ह्मणो॒क्तो ऽब्रा᳚ह्मणोक्त ए॒वैवाब्रा᳚ह्मणोक्तः । \newline
26. अब्रा᳚ह्मणो॒क्तो ऽश्र॑द्दधा॒नो ऽश्र॑द्दधा॒नो ऽब्रा᳚ह्मणो॒क्तो ऽब्रा᳚ह्मणो॒क्तो ऽश्र॑द्दधानः । \newline
27. अब्रा᳚ह्मणोक्त॒ इत्यब्रा᳚ह्मण - उ॒क्तः॒ । \newline
28. अश्र॑द्दधानो॒ यजा॑तै॒ यजा॑ता॒ अश्र॑द्दधा॒नो ऽश्र॑द्दधानो॒ यजा॑तै । \newline
29. अश्र॑द्दधान॒ इत्यश्र॑त् - द॒धा॒नः॒ । \newline
30. यजा॑तै॒ सा सा यजा॑तै॒ यजा॑तै॒ सा । \newline
31. सा मे॑ मे॒ सा सा मे᳚ । \newline
32. मे॒ य॒ज्ञ्स्य॑ य॒ज्ञ्स्य॑ मे मे य॒ज्ञ्स्य॑ । \newline
33. य॒ज्ञ्स्या॒ शी रा॒शीर् य॒ज्ञ्स्य॑ य॒ज्ञ्स्या॒शीः । \newline
34. आ॒शी र॑स दस दा॒शी रा॒शी र॑सत् । \newline
35. आ॒शीरित्या᳚ - शीः । \newline
36. अ॒स॒ दिती त्य॑स दस॒ दिति॑ । \newline
37. इति॒ तस्मा॒त् तस्मा॒ दितीति॒ तस्मा᳚त् । \newline
38. तस्मा॒द् यद् यत् तस्मा॒त् तस्मा॒द् यत् । \newline
39. यदब्रा᳚ह्मणो॒क्तो ऽब्रा᳚ह्मणोक्तो॒ यद् यदब्रा᳚ह्मणोक्तः । \newline
40. अब्रा᳚ह्मणो॒क्तो ऽश्र॑द्दधा॒नो ऽश्र॑द्दधा॒नो ऽब्रा᳚ह्मणो॒क्तो ऽब्रा᳚ह्मणो॒क्तो ऽश्र॑द्दधानः । \newline
41. अब्रा᳚ह्मणोक्त॒ इत्यब्रा᳚ह्मण - उ॒क्तः॒ । \newline
42. अश्र॑द्दधानो॒ यज॑ते॒ यज॒ते ऽश्र॑द्दधा॒नो ऽश्र॑द्दधानो॒ यज॑ते । \newline
43. अश्र॑द्दधान॒ इत्यश्र॑त् - द॒धा॒नः॒ । \newline
44. यज॑ते शं॒ॅयुꣳ शं॒ॅयुं ॅयज॑ते॒ यज॑ते शं॒ॅयुम् । \newline
45. शं॒ॅयु मे॒वैव शं॒ॅयुꣳ शं॒ॅयु मे॒व । \newline
46. शं॒ॅयुमिति॑ शं - युम् । \newline
47. ए॒व तस्य॒ तस्यै॒वैव तस्य॑ । \newline
48. तस्य॑ बार्.हस्प॒त्यम् बा॑र्.हस्प॒त्यम् तस्य॒ तस्य॑ बार्.हस्प॒त्यम् । \newline
49. बा॒र्॒.ह॒स्प॒त्यं ॅय॒ज्ञ्स्य॑ य॒ज्ञ्स्य॑ बार्.हस्प॒त्यम् बा॑र्.हस्प॒त्यं ॅय॒ज्ञ्स्य॑ । \newline
50. य॒ज्ञ्स्या॒शी रा॒शीर् य॒ज्ञ्स्य॑ य॒ज्ञ्स्या॒शीः । \newline
51. आ॒शीर् ग॑च्छति गच्छ त्या॒शी रा॒शीर् ग॑च्छति । \newline
52. आ॒शीरित्या᳚ - शीः । \newline
53. ग॒च्छ॒ त्ये॒त दे॒तद् ग॑च्छति गच्छ त्ये॒तत् । \newline
54. ए॒तन् मम॒ ममै॒त दे॒तन् मम॑ । \newline
55. ममे तीति॒ मम॒ ममे ति॑ । \newline
56. इत्य॑ब्रवी दब्रवी॒ दिती त्य॑ब्रवीत् । \newline
57. अ॒ब्र॒वी॒त् किम् कि म॑ब्रवी दब्रवी॒त् किम् । \newline
58. किम् मे॑ मे॒ किम् किम् मे᳚ । \newline
59. मे॒ प्र॒जायाः᳚ प्र॒जाया॑ मे मे प्र॒जायाः᳚ । \newline
60. प्र॒जाया॒ इतीति॑ प्र॒जायाः᳚ प्र॒जाया॒ इति॑ । \newline
61. प्र॒जाया॒ इति॑ प्र - जायाः᳚ । \newline

\textbf{Ghana Paata } \newline

1. दे॒वा वै वै दे॒वा दे॒वा वै य॒ज्ञ्स्य॑ य॒ज्ञ्स्य॒ वै दे॒वा दे॒वा वै य॒ज्ञ्स्य॑ । \newline
2. वै य॒ज्ञ्स्य॑ य॒ज्ञ्स्य॒ वै वै य॒ज्ञ्स्य॑ स्वगाक॒र्तारꣳ॑ स्वगाक॒र्तारं॑ ॅय॒ज्ञ्स्य॒ वै वै य॒ज्ञ्स्य॑ स्वगाक॒र्तार᳚म् । \newline
3. य॒ज्ञ्स्य॑ स्वगाक॒र्तारꣳ॑ स्वगाक॒र्तारं॑ ॅय॒ज्ञ्स्य॑ य॒ज्ञ्स्य॑ स्वगाक॒र्तार॒म् न न स्व॑गाक॒र्तारं॑ ॅय॒ज्ञ्स्य॑ य॒ज्ञ्स्य॑ स्वगाक॒र्तार॒म् न । \newline
4. स्व॒गा॒क॒र्तार॒म् न न स्व॑गाक॒र्तारꣳ॑ स्वगाक॒र्तार॒म् नावि॑न्दन् नविन्द॒न् न स्व॑गाक॒र्तारꣳ॑ स्वगाक॒र्तार॒म् नावि॑न्दन्न् । \newline
5. स्व॒गा॒क॒र्तार॒मिति॑ स्वगा - क॒र्तार᳚म् । \newline
6. नावि॑न्दन् नविन्द॒न् न नावि॑न्द॒न् ते ते॑ ऽविन्द॒न् न नावि॑न्द॒न् ते । \newline
7. अ॒वि॒न्द॒न् ते ते॑ ऽविन्दन् नविन्द॒न् ते शं॒ॅयुꣳ शं॒ॅयुम् ते॑ ऽविन्दन् नविन्द॒न् ते शं॒ॅयुम् । \newline
8. ते शं॒ॅयुꣳ शं॒ॅयुम् ते ते शं॒ॅयुम् बा॑र्.हस्प॒त्यम् बा॑र्.हस्प॒त्यꣳ शं॒ॅयुम् ते ते शं॒ॅयुम् बा॑र्.हस्प॒त्यम् । \newline
9. शं॒ॅयुम् बा॑र्.हस्प॒त्यम् बा॑र्.हस्प॒त्यꣳ शं॒ॅयुꣳ शं॒ॅयुम् बा॑र्.हस्प॒त्य म॑ब्रुवन् नब्रुवन् बार्.हस्प॒त्यꣳ शं॒ॅयुꣳ शं॒ॅयुम् बा॑र्.हस्प॒त्य म॑ब्रुवन्न् । \newline
10. शं॒ॅयुमिति॑ शं - युम् । \newline
11. बा॒र्॒.ह॒स्प॒त्य म॑ब्रुवन् नब्रुवन् बार्.हस्प॒त्यम् बा॑र्.हस्प॒त्य म॑ब्रुवन् नि॒म मि॒म म॑ब्रुवन् बार्.हस्प॒त्यम् बा॑र्.हस्प॒त्य म॑ब्रुवन् नि॒मम् । \newline
12. अ॒ब्रु॒व॒न् नि॒म मि॒म म॑ब्रुवन् नब्रुवन् नि॒मम् नो॑ न इ॒म म॑ब्रुवन् नब्रुवन् नि॒मन्नः॑ । \newline
13. इ॒मम् नो॑ न इ॒म मि॒मम् नो॑ य॒ज्ञ्ं ॅय॒ज्ञ्म् न॑ इ॒म मि॒मम् नो॑ य॒ज्ञ्म् । \newline
14. नो॒ य॒ज्ञ्ं ॅय॒ज्ञ्म् नो॑ नो य॒ज्ञ्ꣳ स्व॒गा स्व॒गा य॒ज्ञ्म् नो॑ नो य॒ज्ञ्ꣳ स्व॒गा । \newline
15. य॒ज्ञ्ꣳ स्व॒गा स्व॒गा य॒ज्ञ्ं ॅय॒ज्ञ्ꣳ स्व॒गा कु॑रु कुरु स्व॒गा य॒ज्ञ्ं ॅय॒ज्ञ्ꣳ स्व॒गा कु॑रु । \newline
16. स्व॒गा कु॑रु कुरु स्व॒गा स्व॒गा कु॒र्वितीति॑ कुरु स्व॒गा स्व॒गा कु॒र्विति॑ । \newline
17. स्व॒गेति॑ स्व - गा । \newline
18. कु॒र्वितीति॑ कुरु कु॒र्विति॒ स स इति॑ कुरु कु॒र्विति॒ सः । \newline
19. इति॒ स स इतीति॒ सो᳚ ऽब्रवी दब्रवी॒थ् स इतीति॒ सो᳚ ऽब्रवीत् । \newline
20. सो᳚ ऽब्रवी दब्रवी॒थ् स सो᳚ ऽब्रवी॒द् वरं॒ ॅवर॑ मब्रवी॒थ् स सो᳚ ऽब्रवी॒द् वर᳚म् । \newline
21. अ॒ब्र॒वी॒द् वरं॒ ॅवर॑ मब्रवी दब्रवी॒द् वरं॑ ॅवृणै वृणै॒ वर॑ मब्रवी दब्रवी॒द् वरं॑ ॅवृणै । \newline
22. वरं॑ ॅवृणै वृणै॒ वरं॒ ॅवरं॑ ॅवृणै॒ यद् यद् वृ॑णै॒ वरं॒ ॅवरं॑ ॅवृणै॒ यत् । \newline
23. वृ॒णै॒ यद् यद् वृ॑णै वृणै॒ यदे॒वैव यद् वृ॑णै वृणै॒ यदे॒व । \newline
24. यदे॒वैव यद् यदे॒वाब्रा᳚ह्मणो॒क्तो ऽब्रा᳚ह्मणोक्त ए॒व यद् यदे॒वाब्रा᳚ह्मणोक्तः । \newline
25. ए॒वा ब्रा᳚ह्मणो॒क्तो ऽब्रा᳚ह्मणोक्त ए॒वैवा ब्रा᳚ह्मणो॒क्तो ऽश्र॑द्दधा॒नो ऽश्र॑द्दधा॒नो ऽब्रा᳚ह्मणोक्त ए॒वैवाब्रा᳚ह्मणो॒क्तो ऽश्र॑द्दधानः । \newline
26. अब्रा᳚ह्मणो॒क्तो ऽश्र॑द्दधा॒नो ऽश्र॑द्दधा॒नो ऽब्रा᳚ह्मणो॒क्तो ऽब्रा᳚ह्मणो॒क्तो ऽश्र॑द्दधानो॒ यजा॑तै॒ यजा॑ता॒ अश्र॑द्दधा॒नो ऽब्रा᳚ह्मणो॒क्तो ऽब्रा᳚ह्मणो॒क्तो ऽश्र॑द्दधानो॒ यजा॑तै । \newline
27. अब्रा᳚ह्मणोक्त॒ इत्यब्रा᳚ह्मण - उ॒क्तः॒ । \newline
28. अश्र॑द्दधानो॒ यजा॑तै॒ यजा॑ता॒ अश्र॑द्दधा॒नो ऽश्र॑द्दधानो॒ यजा॑तै॒ सा सा यजा॑ता॒ अश्र॑द्दधा॒नो ऽश्र॑द्दधानो॒ यजा॑तै॒ सा । \newline
29. अश्र॑द्दधान॒ इत्यश्र॑त् - द॒धा॒नः॒ । \newline
30. यजा॑तै॒ सा सा यजा॑तै॒ यजा॑तै॒ सा मे॑ मे॒ सा यजा॑तै॒ यजा॑तै॒ सा मे᳚ । \newline
31. सा मे॑ मे॒ सा सा मे॑ य॒ज्ञ्स्य॑ य॒ज्ञ्स्य॑ मे॒ सा सा मे॑ य॒ज्ञ्स्य॑ । \newline
32. मे॒ य॒ज्ञ्स्य॑ य॒ज्ञ्स्य॑ मे मे य॒ज्ञ्स्या॒शी रा॒शीर् य॒ज्ञ्स्य॑ मे मे य॒ज्ञ्स्या॒शीः । \newline
33. य॒ज्ञ्स्या॒शी रा॒शीर् य॒ज्ञ्स्य॑ य॒ज्ञ्स्या॒शी र॑सदस दा॒शीर् य॒ज्ञ्स्य॑ य॒ज्ञ्स्या॒शी र॑सत् । \newline
34. आ॒शी र॑स दस दा॒शी रा॒शी र॑स॒दिती त्य॑सदा॒शी रा॒शी र॑स॒ दिति॑ । \newline
35. आ॒शीरित्या᳚ - शीः । \newline
36. अ॒स॒दिती त्य॑स दस॒ दिति॒ तस्मा॒त् तस्मा॒ दित्य॑स दस॒ दिति॒ तस्मा᳚त् । \newline
37. इति॒ तस्मा॒त् तस्मा॒ दितीति॒ तस्मा॒द् यद् यत् तस्मा॒ दितीति॒ तस्मा॒द् यत् । \newline
38. तस्मा॒द् यद् यत् तस्मा॒त् तस्मा॒द् यदब्रा᳚ह्मणो॒क्तो ऽब्रा᳚ह्मणोक्तो॒ यत् तस्मा॒त् तस्मा॒द् यदब्रा᳚ह्मणोक्तः । \newline
39. यदब्रा᳚ह्मणो॒क्तो ऽब्रा᳚ह्मणोक्तो॒ यद् यदब्रा᳚ह्मणो॒क्तो ऽश्र॑द्दधा॒नो ऽश्र॑द्दधा॒नो ऽब्रा᳚ह्मणोक्तो॒ यद् यदब्रा᳚ह्मणो॒क्तो ऽश्र॑द्दधानः । \newline
40. अब्रा᳚ह्मणो॒क्तो ऽश्र॑द्दधा॒नो ऽश्र॑द्दधा॒नो ऽब्रा᳚ह्मणो॒क्तो ऽब्रा᳚ह्मणो॒क्तो ऽश्र॑द्दधानो॒ यज॑ते॒ यज॒ते ऽश्र॑द्दधा॒नो ऽब्रा᳚ह्मणो॒क्तो ऽब्रा᳚ह्मणो॒क्तो ऽश्र॑द्दधानो॒ यज॑ते । \newline
41. अब्रा᳚ह्मणोक्त॒ इत्यब्रा᳚ह्मण - उ॒क्तः॒ । \newline
42. अश्र॑द्दधानो॒ यज॑ते॒ यज॒ते ऽश्र॑द्दधा॒नो ऽश्र॑द्दधानो॒ यज॑ते शं॒ॅयुꣳ शं॒ॅयुं ॅयज॒ते ऽश्र॑द्दधा॒नो ऽश्र॑द्दधानो॒ यज॑ते शं॒ॅयुम् । \newline
43. अश्र॑द्दधान॒ इत्यश्र॑त् - द॒धा॒नः॒ । \newline
44. यज॑ते शं॒ॅयुꣳ शं॒ॅयुं ॅयज॑ते॒ यज॑ते शं॒ॅयु मे॒वैव शं॒ॅयुं ॅयज॑ते॒ यज॑ते शं॒ॅयु मे॒व । \newline
45. शं॒ॅयु मे॒वैव शं॒ॅयुꣳ शं॒ॅयु मे॒व तस्य॒ तस्यै॒व शं॒ॅयुꣳ शं॒ॅयु मे॒व तस्य॑ । \newline
46. शं॒ॅयुमिति॑ शं - युम् । \newline
47. ए॒व तस्य॒ तस्यै॒वैव तस्य॑ बार्.हस्प॒त्यम् बा॑र्.हस्प॒त्यम् तस्यै॒वैव तस्य॑ बार्.हस्प॒त्यम् । \newline
48. तस्य॑ बार्.हस्प॒त्यम् बा॑र्.हस्प॒त्यम् तस्य॒ तस्य॑ बार्.हस्प॒त्यं ॅय॒ज्ञ्स्य॑ य॒ज्ञ्स्य॑ बार्.हस्प॒त्यम् तस्य॒ तस्य॑ बार्.हस्प॒त्यं ॅय॒ज्ञ्स्य॑ । \newline
49. बा॒र्॒.ह॒स्प॒त्यं ॅय॒ज्ञ्स्य॑ य॒ज्ञ्स्य॑ बार्.हस्प॒त्यम् बा॑र्.हस्प॒त्यं ॅय॒ज्ञ्स्या॒शी रा॒शीर् य॒ज्ञ्स्य॑ बार्.हस्प॒त्यम् बा॑र्.हस्प॒त्यं ॅय॒ज्ञ्स्या॒शीः । \newline
50. य॒ज्ञ्स्या॒शी रा॒शीर् य॒ज्ञ्स्य॑ य॒ज्ञ्स्या॒शीर् ग॑च्छति गच्छ त्या॒शीर् य॒ज्ञ्स्य॑ य॒ज्ञ्स्या॒शीर् ग॑च्छति । \newline
51. आ॒शीर् ग॑च्छति गच्छत्या॒शी रा॒शीर् ग॑च्छ त्ये॒त दे॒तद् ग॑च्छत्या॒शी रा॒शीर् ग॑च्छ त्ये॒तत् । \newline
52. आ॒शीरित्या᳚ - शीः । \newline
53. ग॒च्छ॒ त्ये॒त दे॒तद् ग॑च्छति गच्छ त्ये॒तन् मम॒ ममै॒तद् ग॑च्छति गच्छ त्ये॒तन् मम॑ । \newline
54. ए॒तन् मम॒ ममै॒त दे॒तन् ममे तीति॒ ममै॒त दे॒तन् ममे ति॑ । \newline
55. ममे तीति॒ मम॒ ममे त्य॑ब्रवी दब्रवी॒ दिति॒ मम॒ ममे त्य॑ब्रवीत् । \newline
56. इत्य॑ब्रवी दब्रवी॒ दिती त्य॑ब्रवी॒त् किम् कि म॑ब्रवी॒दिती त्य॑ब्रवी॒त् किम् । \newline
57. अ॒ब्र॒वी॒त् किम् कि म॑ब्रवी दब्रवी॒त् किम् मे॑ मे॒ कि म॑ब्रवी दब्रवी॒त् किम् मे᳚ । \newline
58. किम् मे॑ मे॒ किम् किम् मे᳚ प्र॒जायाः᳚ प्र॒जाया॑ मे॒ किम् किम् मे᳚ प्र॒जायाः᳚ । \newline
59. मे॒ प्र॒जायाः᳚ प्र॒जाया॑ मे मे प्र॒जाया॒ इतीति॑ प्र॒जाया॑ मे मे प्र॒जाया॒ इति॑ । \newline
60. प्र॒जाया॒ इतीति॑ प्र॒जायाः᳚ प्र॒जाया॒ इति॒ यो य इति॑ प्र॒जायाः᳚ प्र॒जाया॒ इति॒ यः । \newline
61. प्र॒जाया॒ इति॑ प्र - जायाः᳚ । \newline
\pagebreak
\markright{ TS 2.6.10.2  \hfill https://www.vedavms.in \hfill}
\addcontentsline{toc}{section}{ TS 2.6.10.2 }
\section*{ TS 2.6.10.2 }

\textbf{TS 2.6.10.2 } \newline
\textbf{Samhita Paata} \newline

इति॒ यो॑ऽपगु॒रातै॑ श॒तेन॑ यातया॒द्यो नि॒हन॑थ् स॒हस्रे॑ण यातया॒द्यो लोहि॑तं क॒रव॒द्याव॑तः प्र॒स्कद्य॑ पाꣳ॒॒सून्थ् सं॑ गृ॒ह्णात् ताव॑तः संॅवथ्स॒रान् पि॑तृलो॒कं न प्रजा॑ना॒दिति॒ तस्मा᳚द्-ब्राह्म॒णाय॒ नाप॑ गुरेत॒ न नि ह॑न्या॒न्न लोहि॑तं कुर्यादे॒ताव॑ता॒ हैन॑सा भवति॒ तच्छं॒ॅयोरा वृ॑णीमह॒ इत्या॑ह य॒ज्ञ्मे॒व तथ् स्व॒गा क॑रोति॒ त - [  ] \newline

\textbf{Pada Paata} \newline

इति॑ । यः । अ॒प॒गु॒राता॒ इत्य॑प - गु॒रातै᳚ । श॒तेन॑ । या॒त॒या॒त् । यः । नि॒हन॒दिति॑ नि - हन॑त् । स॒हस्रे॑ण । या॒त॒या॒त् । यः । लोहि॑तम् । क॒रव॑त् । याव॑तः । प्र॒स्कद्येति॑ प्र - स्कद्य॑ । पाꣳ॒॒सून् । सं॒गृ॒ह्णादिति॑ सं - गृ॒ह्णात् । ताव॑तः । सं॒ॅव॒थ्स॒रानिति॑ सं - व॒थ्स॒रान् । पि॒तृ॒लो॒कमिति॑ पितृ-लो॒कम् । न । प्रेति॑ । जा॒ना॒त् । इति॑ । तस्मा᳚त् । ब्रा॒ह्म॒णाय॑ । न । अपेति॑ । गु॒रे॒त॒ । न । नीति॑ । ह॒न्या॒त् । न । लोहि॑तम् । कु॒र्या॒त् । ए॒ताव॑ता । ह॒ । एन॑सा । भ॒व॒ति॒ । तत् । शं॒ॅयोरिति॑ शं-योः । एति॑ । वृ॒णी॒म॒हे॒ । इति॑ । आ॒ह॒ । य॒ज्ञ्म् । ए॒व । तत् । स्व॒गेति॑ स्व - गा । क॒रो॒ति॒ । तत् ।  \newline


\textbf{Krama Paata} \newline

इति॒ यः । यो॑ऽपगु॒रातै᳚ । अ॒प॒गु॒रातै॑ श॒तेन॑ । अ॒प॒गुराता॒ इत्य॑प - गु॒रातै᳚ । श॒तेन॑ यातयात् । या॒त॒या॒द् यः । यो नि॒हन॑त् । नि॒हन॑थ् स॒हस्रे॑ण । नि॒हन॒दिति॑ नि - हन॑त् । स॒हस्रे॑ण यातयात् । या॒त॒या॒द् यः । यो लोहि॑तम् । लोहि॑तम् क॒रव॑त् । क॒रव॒द् याव॑तः । याव॑तः प्र॒स्कद्य॑ । प्र॒स्कद्य॑ पाꣳ॒॒सून् । प्र॒स्कद्येति॑ प्र - स्कद्य॑ । पाꣳ॒॒सून्थ् स॑ङ्गृ॒ह्णात् । स॒ङ्गृ॒ह्णात् ताव॑तः । स॒ङ्गृ॒ह्णादिति॑ सं - गृ॒ह्णात् । ताव॑तः सम्ॅवथ्स॒रान् । स॒म्ॅव॒थ्स॒रान् पि॑तृलो॒कम् । स॒म्ॅव॒थ्स॒रानिति॑ सं - व॒थ्स॒रान् । पि॒तृ॒लो॒कम् न । पि॒तृ॒लो॒कमिति॑ पितृ - लो॒कम् । न प्र । प्र जा॑नात् । जा॒ना॒दिति॑ । इति॒ तस्मा᳚त् । तस्मा᳚द् ब्राह्म॒णाय॑ । ब्रा॒ह्म॒णाय॒ न । नाप॑ । अप॑ गुरेत । गु॒रे॒त॒ न । न नि । नि ह॑न्यात् । ह॒न्या॒न् न । न लोहि॑तम् । लोहि॑तम् कुर्यात् । कु॒र्या॒दे॒ताव॑ता । ए॒ताव॑ता ह । हैन॑सा । एन॑सा भवति । भ॒व॒ति॒ तत् । 
तच्छं॒ॅयोः । शं॒ॅयोरा । श॒म्ॅयोरिति॑ शं - योः । आ वृ॑णीमहे । वृ॒णी॒म॒ह॒ इति॑ । इत्या॑ह । आ॒ह॒ य॒ज्ञ्म् । य॒ज्ञ्मे॒व । ए॒व तत् । तथ् स्व॒गा । स्व॒गा क॑रोति । स्व॒गेति॑ स्व - गा । क॒रो॒ति॒ तत् । तच्छं॒ॅयोः \newline

\textbf{Jatai Paata} \newline

1. इति॒ यो य इतीति॒ यः । \newline
2. यो॑ ऽपगु॒राता॑ अपगु॒रातै॒ यो यो॑ ऽपगु॒रातै᳚ । \newline
3. अ॒प॒गु॒रातै॑ श॒तेन॑ श॒तेना॑ पगु॒राता॑ अपगु॒रातै॑ श॒तेन॑ । \newline
4. अ॒प॒गु॒राता॒ इत्य॑प - गु॒रातै᳚ । \newline
5. श॒तेन॑ यातयाद् यातयाच् छ॒तेन॑ श॒तेन॑ यातयात् । \newline
6. या॒त॒या॒द् यो यो या॑तयाद् यातया॒द् यः । \newline
7. यो नि॒हन॑न् नि॒हन॒द् यो यो नि॒हन॑त् । \newline
8. नि॒हन॑थ् स॒हस्रे॑ण स॒हस्रे॑ण नि॒हन॑न् नि॒हन॑थ् स॒हस्रे॑ण । \newline
9. नि॒हन॒दिति॑ नि - हन॑त् । \newline
10. स॒हस्रे॑ण यातयाद् यातयाथ् स॒हस्रे॑ण स॒हस्रे॑ण यातयात् । \newline
11. या॒त॒या॒द् यो यो या॑तयाद् यातया॒द् यः । \newline
12. यो लोहि॑त॒म् ॅलोहि॑तं॒ ॅयो यो लोहि॑तम् । \newline
13. लोहि॑तम् क॒रव॑त् क॒रव॒ ल्लोहि॑त॒म् ॅलोहि॑तम् क॒रव॑त् । \newline
14. क॒रव॒द् याव॑तो॒ याव॑तः क॒रव॑त् क॒रव॒द् याव॑तः । \newline
15. याव॑तः प्र॒स्कद्य॑ प्र॒स्कद्य॒ याव॑तो॒ याव॑तः प्र॒स्कद्य॑ । \newline
16. प्र॒स्कद्य॑ पाꣳ॒॒सून् पाꣳ॒॒सून् प्र॒स्कद्य॑ प्र॒स्कद्य॑ पाꣳ॒॒सून् । \newline
17. प्र॒स्कद्येति॑ प्र - स्कद्य॑ । \newline
18. पाꣳ॒॒सून् थ्स॑ङ्गृ॒ह्णाथ् स॑ङ्गृ॒ह्णात् पाꣳ॒॒सून् पाꣳ॒॒सून् थ्स॑ङ्गृ॒ह्णात् । \newline
19. स॒ङ्गृ॒ह्णात् ताव॑त॒ स्ताव॑तः सङ्गृ॒ह्णाथ् स॑ङ्गृ॒ह्णात् ताव॑तः । \newline
20. स॒ङ्गृ॒ह्णादिति॑ सं - गृ॒ह्णात् । \newline
21. ताव॑तः संॅवथ्स॒रान् थ्सं॑ॅवथ्स॒रान् ताव॑त॒स्ताव॑तः संॅवथ्स॒रान् । \newline
22. सं॒ॅव॒थ्स॒रान् पि॑तृलो॒कम् पि॑तृलो॒कꣳ सं॑ॅवथ्स॒रान् थ्सं॑ॅवथ्स॒रान् पि॑तृलो॒कम् । \newline
23. सं॒ॅव॒थ्स॒रानिति॑ सं - व॒थ्स॒रान् । \newline
24. पि॒तृ॒लो॒कम् न न पि॑तृलो॒कम् पि॑तृलो॒कम् न । \newline
25. पि॒तृ॒लो॒कमिति॑ पितृ - लो॒कम् । \newline
26. न प्र प्र ण न प्र । \newline
27. प्र जा॑नाज् जाना॒त् प्र प्र जा॑नात् । \newline
28. जा॒ना॒ दितीति॑ जानाज् जाना॒ दिति॑ । \newline
29. इति॒ तस्मा॒त् तस्मा॒ दितीति॒ तस्मा᳚त् । \newline
30. तस्मा᳚द् ब्राह्म॒णाय॑ ब्राह्म॒णाय॒ तस्मा॒त् तस्मा᳚द् ब्राह्म॒णाय॑ । \newline
31. ब्रा॒ह्म॒णाय॒ न न ब्रा᳚ह्म॒णाय॑ ब्राह्म॒णाय॒ न । \newline
32. नापाप॒ न नाप॑ । \newline
33. अप॑ गुरेत गुरे॒ता पाप॑ गुरेत । \newline
34. गु॒रे॒त॒ न न गु॑रेत गुरेत॒ न । \newline
35. न नि नि न न नि । \newline
36. नि ह॑न्या द्धन्या॒न् नि नि ह॑न्यात् । \newline
37. ह॒न्या॒न् न न ह॑न्या द्धन्या॒न् न । \newline
38. न लोहि॑त॒म् ॅलोहि॑त॒म् न न लोहि॑तम् । \newline
39. लोहि॑तम् कुर्यात् कुर्या॒ ल्लोहि॑त॒म् ॅलोहि॑तम् कुर्यात् । \newline
40. कु॒र्या॒ दे॒ताव॑ तै॒ताव॑ता कुर्यात् कुर्या दे॒ताव॑ता । \newline
41. ए॒ताव॑ता ह है॒ताव॑ तै॒ताव॑ता ह । \newline
42. हैन॒सैन॑सा ह॒ हैन॑सा । \newline
43. एन॑सा भवति भव॒ त्येन॒सैन॑सा भवति । \newline
44. भ॒व॒ति॒ तत् तद् भ॑वति भवति॒ तत् । \newline
45. तच्छं॒ॅयोः शं॒ॅयो स्तत् तच्छं॒ॅयोः । \newline
46. शं॒ॅयोरा शं॒ॅयोः शं॒ॅयोरा । \newline
47. शं॒ॅयोरिति॑ शं - योः । \newline
48. आ वृ॑णीमहे वृणीमह॒ आ वृ॑णीमहे । \newline
49. वृ॒णी॒म॒ह॒ इतीति॑ वृणीमहे वृणीमह॒ इति॑ । \newline
50. इत्या॑हा॒हे तीत्या॑ह । \newline
51. आ॒ह॒ य॒ज्ञ्ं ॅय॒ज्ञ् मा॑हाह य॒ज्ञ्म् । \newline
52. य॒ज्ञ् मे॒वैव य॒ज्ञ्ं ॅय॒ज्ञ् मे॒व । \newline
53. ए॒व तत् तदे॒वैव तत् । \newline
54. तथ् स्व॒गा स्व॒गा तत् तथ् स्व॒गा । \newline
55. स्व॒गा क॑रोति करोति स्व॒गा स्व॒गा क॑रोति । \newline
56. स्व॒गेति॑ स्व - गा । \newline
57. क॒रो॒ति॒ तत् तत् क॑रोति करोति॒ तत् । \newline
58. तच्छं॒ॅयोः शं॒ॅयो स्तत् तच्छं॒ॅयोः । \newline

\textbf{Ghana Paata } \newline

1. इति॒ यो य इतीति॒ यो॑ ऽपगु॒राता॑ अपगु॒रातै॒ य इतीति॒ यो॑ ऽपगु॒रातै᳚ । \newline
2. यो॑ ऽपगु॒राता॑ अपगु॒रातै॒ यो यो॑ ऽपगु॒रातै॑ श॒तेन॑ श॒तेना॑ पगु॒रातै॒ यो यो॑ ऽपगु॒रातै॑ श॒तेन॑ । \newline
3. अ॒प॒गु॒रातै॑ श॒तेन॑ श॒तेना॑ पगु॒राता॑ अपगु॒रातै॑ श॒तेन॑ यातयाद् यातयाच् छ॒तेना॑ पगु॒राता॑ अपगु॒रातै॑ श॒तेन॑ यातयात् । \newline
4. अ॒प॒गु॒राता॒ इत्य॑प - गु॒रातै᳚ । \newline
5. श॒तेन॑ यातयाद् यातयाच् छ॒तेन॑ श॒तेन॑ यातया॒द् यो यो या॑तयाच् छ॒तेन॑ श॒तेन॑ यातया॒द् यः । \newline
6. या॒त॒या॒द् यो यो या॑तयाद् यातया॒द् यो नि॒हन॑न् नि॒हन॒द् यो या॑तयाद् यातया॒द् यो नि॒हन॑त् । \newline
7. यो नि॒हन॑न् नि॒हन॒द् यो यो नि॒हन॑थ् स॒हस्रे॑ण स॒हस्रे॑ण नि॒हन॒द् यो यो नि॒हन॑थ् स॒हस्रे॑ण । \newline
8. नि॒हन॑थ् स॒हस्रे॑ण स॒हस्रे॑ण नि॒हन॑न् नि॒हन॑थ् स॒हस्रे॑ण यातयाद् यातयाथ् स॒हस्रे॑ण नि॒हन॑न् नि॒हन॑थ् स॒हस्रे॑ण यातयात् । \newline
9. नि॒हन॒दिति॑ नि - हन॑त् । \newline
10. स॒हस्रे॑ण यातयाद् यातयाथ् स॒हस्रे॑ण स॒हस्रे॑ण यातया॒द् यो यो या॑तयाथ् स॒हस्रे॑ण स॒हस्रे॑ण यातया॒द् यः । \newline
11. या॒त॒या॒द् यो यो या॑तयाद् यातया॒द् यो लोहि॑त॒म् ॅलोहि॑तं॒ ॅयो या॑तयाद् यातया॒द् यो लोहि॑तम् । \newline
12. यो लोहि॑त॒म् ॅलोहि॑तं॒ ॅयो यो लोहि॑तम् क॒रव॑त् क॒रव॒ ल्लोहि॑तं॒ ॅयो यो लोहि॑तम् क॒रव॑त् । \newline
13. लोहि॑तम् क॒रव॑त् क॒रव॒ ल्लोहि॑त॒म् ॅलोहि॑तम् क॒रव॒द् याव॑तो॒ याव॑तः क॒रव॒ ल्लोहि॑त॒म् ॅलोहि॑तम् क॒रव॒द् याव॑तः । \newline
14. क॒रव॒द् याव॑तो॒ याव॑तः क॒रव॑त् क॒रव॒द् याव॑तः प्र॒स्कद्य॑ प्र॒स्कद्य॒ याव॑तः क॒रव॑त् क॒रव॒द् याव॑तः प्र॒स्कद्य॑ । \newline
15. याव॑तः प्र॒स्कद्य॑ प्र॒स्कद्य॒ याव॑तो॒ याव॑तः प्र॒स्कद्य॑ पाꣳ॒॒सून् पाꣳ॒॒सून् प्र॒स्कद्य॒ याव॑तो॒ याव॑तः प्र॒स्कद्य॑ पाꣳ॒॒सून् । \newline
16. प्र॒स्कद्य॑ पाꣳ॒॒सून् पाꣳ॒॒सून् प्र॒स्कद्य॑ प्र॒स्कद्य॑ पाꣳ॒॒सून् थ्स॑ङ्गृ॒ह्णाथ् स॑ङ्गृ॒ह्णात् पाꣳ॒॒सून् प्र॒स्कद्य॑ प्र॒स्कद्य॑ पाꣳ॒॒सून् थ्स॑ङ्गृ॒ह्णात् । \newline
17. प्र॒स्कद्येति॑ प्र - स्कद्य॑ । \newline
18. पाꣳ॒॒सून् थ्स॑ङ्गृ॒ह्णाथ् स॑ङ्गृ॒ह्णात् पाꣳ॒॒सून् पाꣳ॒॒सून् थ्स॑ङ्गृ॒ह्णात् ताव॑त॒ स्ताव॑तः सङ्गृ॒ह्णात् पाꣳ॒॒सून् पाꣳ॒॒सून् थ्स॑ङ्गृ॒ह्णात् ताव॑तः । \newline
19. स॒ङ्गृ॒ह्णात् ताव॑त॒ स्ताव॑तः सङ्गृ॒ह्णाथ् स॑ङ्गृ॒ह्णात् ताव॑तः संॅवथ्स॒रान् थ्सं॑ॅवथ्स॒रान् ताव॑तः सङ्गृ॒ह्णाथ् स॑ङ्गृ॒ह्णात् ताव॑तः संॅवथ्स॒रान् । \newline
20. स॒ङ्गृ॒ह्णादिति॑ सं - गृ॒ह्णात् । \newline
21. ताव॑तः संॅवथ्स॒रान् थ्सं॑ॅवथ्स॒रान् ताव॑त॒ स्ताव॑तः संॅवथ्स॒रान् पि॑तृलो॒कम् पि॑तृलो॒कꣳ सं॑ॅवथ्स॒रान् ताव॑त॒ स्ताव॑तः संॅवथ्स॒रान् पि॑तृलो॒कम् । \newline
22. सं॒ॅव॒थ्स॒रान् पि॑तृलो॒कम् पि॑तृलो॒कꣳ सं॑ॅवथ्स॒रान् थ्सं॑ॅवथ्स॒रान् पि॑तृलो॒कम् न न पि॑तृलो॒कꣳ सं॑ॅवथ्स॒रान् थ्सं॑ॅवथ्स॒रान् पि॑तृलो॒कम् न । \newline
23. सं॒ॅव॒थ्स॒रानिति॑ सं - व॒थ्स॒रान् । \newline
24. पि॒तृ॒लो॒कम् न न पि॑तृलो॒कम् पि॑तृलो॒कम् न प्र प्र ण पि॑तृलो॒कम् पि॑तृलो॒कम् न प्र । \newline
25. पि॒तृ॒लो॒कमिति॑ पितृ - लो॒कम् । \newline
26. न प्र प्र ण न प्र जा॑नाज् जाना॒त् प्र ण न प्र जा॑नात् । \newline
27. प्र जा॑नाज् जाना॒त् प्र प्र जा॑ना॒ दितीति॑ जाना॒त् प्र प्र जा॑ना॒ दिति॑ । \newline
28. जा॒ना॒ दितीति॑ जानाज् जाना॒ दिति॒ तस्मा॒त् तस्मा॒ दिति॑ जानाज् जाना॒ दिति॒ तस्मा᳚त् । \newline
29. इति॒ तस्मा॒त् तस्मा॒ दितीति॒ तस्मा᳚द् ब्राह्म॒णाय॑ ब्राह्म॒णाय॒ तस्मा॒ दितीति॒ तस्मा᳚द् ब्राह्म॒णाय॑ । \newline
30. तस्मा᳚द् ब्राह्म॒णाय॑ ब्राह्म॒णाय॒ तस्मा॒त् तस्मा᳚द् ब्राह्म॒णाय॒ न न ब्रा᳚ह्म॒णाय॒ तस्मा॒त् तस्मा᳚द् ब्राह्म॒णाय॒ न । \newline
31. ब्रा॒ह्म॒णाय॒ न न ब्रा᳚ह्म॒णाय॑ ब्राह्म॒णाय॒ नापाप॒ न ब्रा᳚ह्म॒णाय॑ ब्राह्म॒णाय॒ नाप॑ । \newline
32. नापाप॒ न नाप॑ गुरेत गुरे॒ताप॒ न नाप॑ गुरेत । \newline
33. अप॑ गुरेत गुरे॒तापाप॑ गुरेत॒ न न गु॑रे॒तापाप॑ गुरेत॒ न । \newline
34. गु॒रे॒त॒ न न गु॑रेत गुरेत॒ न नि नि न गु॑रेत गुरेत॒ न नि । \newline
35. न नि नि न न नि ह॑न्या द्धन्या॒न् नि न न नि ह॑न्यात् । \newline
36. नि ह॑न्या द्धन्या॒न् नि नि ह॑न्या॒न् न न ह॑न्या॒न् नि नि ह॑न्या॒न् न । \newline
37. ह॒न्या॒न् न न ह॑न्या द्धन्या॒न् न लोहि॑त॒म् ॅलोहि॑त॒म् न ह॑न्या द्धन्या॒न् न लोहि॑तम् । \newline
38. न लोहि॑त॒म् ॅलोहि॑त॒म् न न लोहि॑तम् कुर्यात् कुर्या॒ ल्लोहि॑त॒म् न न लोहि॑तम् कुर्यात् । \newline
39. लोहि॑तम् कुर्यात् कुर्या॒ ल्लोहि॑त॒म् ॅलोहि॑तम् कुर्या दे॒ताव॑ तै॒ताव॑ता कुर्या॒ ल्लोहि॑त॒म् ॅलोहि॑तम् कुर्या दे॒ताव॑ता । \newline
40. कु॒र्या॒ दे॒ताव॑ तै॒ताव॑ता कुर्यात् कुर्या दे॒ताव॑ता ह है॒ताव॑ता कुर्यात् कुर्या दे॒ताव॑ता ह । \newline
41. ए॒ताव॑ता ह है॒ताव॑ तै॒ताव॑ता॒ हैन॒सैन॑सा है॒ताव॑ तै॒ताव॑ता॒ हैन॑सा । \newline
42. हैन॒सैन॑सा ह॒ हैन॑सा भवति भव॒ त्येन॑सा ह॒ हैन॑सा भवति । \newline
43. एन॑सा भवति भव॒ त्येन॒सैन॑सा भवति॒ तत् तद् भ॑व॒ त्येन॒सैन॑सा भवति॒ तत् । \newline
44. भ॒व॒ति॒ तत् तद् भ॑वति भवति॒ तच्छं॒ॅयोः शं॒ॅयो स्तद् भ॑वति भवति॒ तच्छं॒ॅयोः । \newline
45. तच्छं॒ॅयोः शं॒ॅयो स्तत् तच्छं॒ॅयोरा शं॒ॅयो स्तत् तच्छं॒ॅयोरा । \newline
46. शं॒ॅयोरा शं॒ॅयोः शं॒ॅयोरा वृ॑णीमहे वृणीमह॒ आ शं॒ॅयोः शं॒ॅयोरा वृ॑णीमहे । \newline
47. शं॒ॅयोरिति॑ शं - योः । \newline
48. आ वृ॑णीमहे वृणीमह॒ आ वृ॑णीमह॒ इतीति॑ वृणीमह॒ आ वृ॑णीमह॒ इति॑ । \newline
49. वृ॒णी॒म॒ह॒ इतीति॑ वृणीमहे वृणीमह॒ इत्या॑हा॒हे ति॑ वृणीमहे वृणीमह॒ इत्या॑ह । \newline
50. इत्या॑हा॒हे तीत्या॑ह य॒ज्ञ्ं ॅय॒ज्ञ् मा॒हे तीत्या॑ह य॒ज्ञ्म् । \newline
51. आ॒ह॒ य॒ज्ञ्ं ॅय॒ज्ञ् मा॑हाह य॒ज्ञ् मे॒वैव य॒ज्ञ् मा॑हाह य॒ज्ञ् मे॒व । \newline
52. य॒ज्ञ् मे॒वैव य॒ज्ञ्ं ॅय॒ज्ञ् मे॒व तत् तदे॒व य॒ज्ञ्ं ॅय॒ज्ञ् मे॒व तत् । \newline
53. ए॒व तत् तदे॒वैव तथ् स्व॒गा स्व॒गा तदे॒वैव तथ् स्व॒गा । \newline
54. तथ् स्व॒गा स्व॒गा तत् तथ् स्व॒गा क॑रोति करोति स्व॒गा तत् तथ् स्व॒गा क॑रोति । \newline
55. स्व॒गा क॑रोति करोति स्व॒गा स्व॒गा क॑रोति॒ तत् तत् क॑रोति स्व॒गा स्व॒गा क॑रोति॒ तत् । \newline
56. स्व॒गेति॑ स्व - गा । \newline
57. क॒रो॒ति॒ तत् तत् क॑रोति करोति॒ तच्छं॒ॅयोः शं॒ॅयो स्तत् क॑रोति करोति॒ तच्छं॒ॅयोः । \newline
58. तच्छं॒ॅयोः शं॒ॅयो स्तत् तच्छं॒ॅयोरा शं॒ॅयो स्तत् तच्छं॒ॅयोरा । \newline
\pagebreak
\markright{ TS 2.6.10.3  \hfill https://www.vedavms.in \hfill}
\addcontentsline{toc}{section}{ TS 2.6.10.3 }
\section*{ TS 2.6.10.3 }

\textbf{TS 2.6.10.3 } \newline
\textbf{Samhita Paata} \newline

-च्छं॒ॅयोरा वृ॑णीमह॒ इत्या॑ह शं॒ॅयुमे॒व बा॑र्.हस्प॒त्यं भा॑ग॒धेये॑न॒ सम॑र्द्धयति गा॒तुं ॅय॒ज्ञाय॑ गा॒तुं ॅय॒ज्ञ्प॑तय॒ इत्या॑हा॒ ऽऽशिष॑मे॒वै तामा शा᳚स्ते॒ सोमं॑ ॅयजति॒ रेत॑ ए॒व तद्-द॑धाति॒ त्वष्टा॑रं ॅयजति॒ रेत॑ ए॒व हि॒तं त्वष्टा॑ रू॒पाणि॒ वि क॑रोति दे॒वानां॒ पत्नी᳚र्यजति मिथुन॒त्वाया॒ग्निं गृ॒हप॑तिं ॅयजति॒ प्रति॑ष्ठित्यै जा॒मि वा ए॒तद्-य॒ज्ञ्स्य॑ क्रियते॒ - [  ] \newline

\textbf{Pada Paata} \newline

शं॒ॅयोरिति॑ शं - योः । एति॑ । वृ॒णी॒म॒हे॒ । इति॑ । आ॒ह॒ । शं॒ॅयुमिति॑ शं - युम् । ए॒व । ब॒र्.॒ह॒स्प॒त्यम् । भा॒ग॒धेये॒नेति॑ भाग - धेये॑न । समिति॑ । अ॒र्द्ध॒य॒ति॒ । गा॒तुम् । य॒ज्ञाय॑ । गा॒तुम् । य॒ज्ञ्प॑तय॒ इति॑ य॒ज्ञ्-प॒त॒ये॒ । इति॑ । आ॒ह॒ । आ॒शिष॒मित्या᳚ - शिष᳚म् । ए॒व । ए॒ताम् । एति॑ । शा॒स्ते॒ । सोम᳚म् । य॒ज॒ति॒ । रेतः॑ । ए॒व । तत् । द॒धा॒ति॒ । त्वष्टा॑रम् । य॒ज॒ति॒ । रेतः॑ । ए॒व । हि॒तम् । त्वष्टा᳚ । रू॒पाणि॑ । वीति॑ । क॒रो॒ति॒ । दे॒वाना᳚म् । पत्नीः᳚ । य॒ज॒ति॒ । मि॒थु॒न॒त्वायेति॑ मिथुन-त्वाय॑ । अ॒ग्निम् । गृ॒हप॑ति॒मिति॑ गृ॒ह - प॒ति॒म् । य॒ज॒ति॒ । प्रति॑ष्ठित्या॒ इति॒ प्रति॑ - स्थि॒त्यै॒ । जा॒मि । वै । ए॒तत् । य॒ज्ञ्स्य॑ । क्रि॒य॒ते॒ ।  \newline


\textbf{Krama Paata} \newline

शं॒ॅयोरा । श॒म्ॅयोरिति॑ शम् - योः । आ वृ॑णीमहे । वृ॒णी॒म॒ह॒ इति॑ । इत्या॑ह । आ॒ह॒ श॒म्ॅयुम् । श॒म्ॅयुमे॒व । श॒म्ॅयुमिति॑ शम् - युम् । ए॒व बा॑र्.हस्प॒त्यम् । बा॒र्॒.ह॒स्प॒त्यम् भा॑ग॒धेये॑न । भा॒ग॒धेये॑न॒ सम् । भा॒ग॒धेये॒नेति॑ भाग - धेये॑न । सम॑र्द्धयति । अ॒र्द्ध॒य॒ति॒ गा॒तुम् । गा॒तुं ॅय॒ज्ञाय॑ । य॒ज्ञाय॑ गा॒तुम् । गा॒तुं ॅय॒ज्ञ्प॑तये । य॒ज्ञ्प॑तय॒ इति॑ । य॒ज्ञ्प॑तय॒ इति॑ य॒ज्ञ् - प॒त॒ये॒ । इत्या॑ह । आ॒हा॒शिष᳚म् । आ॒शिष॑मे॒व । आ॒शिष॒मित्या᳚ - शिष᳚म् । ए॒वैताम् । ए॒तामा । आ शा᳚स्ते । शा॒स्ते॒ सोम᳚म् । सोमं॑ ॅयजति । य॒ज॒ति॒ रेतः॑ । रेत॑ ए॒व । ए॒व तत् । तद् द॑धाति । द॒धा॒ति॒ त्वष्टा॑रम् । त्वष्टा॑रं ॅयजति । य॒ज॒ति॒ रेतः॑ । रेत॑ ए॒व । ए॒व हि॒तम् । हि॒तम् त्वष्टा᳚ । त्वष्टा॑ रू॒पाणि॑ । रू॒पाणि॒ वि । वि क॑रोति । क॒रो॒ति॒ दे॒वाना᳚म् । दे॒वाना॒म् पत्नीः᳚ । पत्नी᳚र् यजति । य॒ज॒ति॒ मि॒थु॒न॒त्वाय॑ । मि॒थु॒न॒त्वाया॒ग्निम् । मि॒थु॒न॒त्वायेति॑ मिथुन - त्वाय॑ । अ॒ग्निम् गृ॒हप॑तिम् । गृ॒हप॑तिं ॅयजति । गृ॒हप॑ति॒मिति॑ गृ॒ह - प॒ति॒म् । य॒ज॒ति॒ प्रति॑ष्ठित्यै । प्रति॑ष्ठित्यै जा॒मि । प्रति॑ष्ठित्या॒ इति॒ प्रति॑ - स्थि॒त्यै॒ । जा॒मि वै । वा ए॒तत् । ए॒तद् य॒ज्ञ्स्य॑ । य॒ज्ञ्स्य॑ क्रियते ( ) । क्रि॒य॒ते॒ यत् \newline

\textbf{Jatai Paata} \newline

1. शं॒ॅयोरा शं॒ॅयोः शं॒ॅयोरा । \newline
2. शं॒ॅयोरिति॑ शं - योः । \newline
3. आ वृ॑णीमहे वृणीमह॒ आ वृ॑णीमहे । \newline
4. वृ॒णी॒म॒ह॒ इतीति॑ वृणीमहे वृणीमह॒ इति॑ । \newline
5. इत्या॑हा॒हे तीत्या॑ह । \newline
6. आ॒ह॒ शं॒ॅयुꣳ शं॒ॅयु मा॑हाह शं॒ॅयुम् । \newline
7. शं॒ॅयु मे॒वैव शं॒ॅयुꣳ शं॒ॅयु मे॒व । \newline
8. शं॒ॅयुमिति॑ शं - युम् । \newline
9. ए॒व बा॑र्.हस्प॒त्यम् बा॑र्.हस्प॒त्य मे॒वैव बा॑र्.हस्प॒त्यम् । \newline
10. बा॒र्॒.ह॒स्प॒त्यम् भा॑ग॒धेये॑न भाग॒धेये॑न बार्.हस्प॒त्यम् बा॑र्.हस्प॒त्यम् भा॑ग॒धेये॑न । \newline
11. भा॒ग॒धेये॑न॒ सꣳ सम् भा॑ग॒धेये॑न भाग॒धेये॑न॒ सम् । \newline
12. भा॒ग॒धेये॒नेति॑ भाग - धेये॑न । \newline
13. स म॑र्द्धय त्यर्द्धयति॒ सꣳ स म॑र्द्धयति । \newline
14. अ॒र्द्ध॒य॒ति॒ गा॒तुम् गा॒तु म॑र्द्धय त्यर्द्धयति गा॒तुम् । \newline
15. गा॒तुं ॅय॒ज्ञाय॑ य॒ज्ञाय॑ गा॒तुम् गा॒तुं ॅय॒ज्ञाय॑ । \newline
16. य॒ज्ञाय॑ गा॒तुम् गा॒तुं ॅय॒ज्ञाय॑ य॒ज्ञाय॑ गा॒तुम् । \newline
17. गा॒तुं ॅय॒ज्ञ्प॑तये य॒ज्ञ्प॑तये गा॒तुम् गा॒तुं ॅय॒ज्ञ्प॑तये । \newline
18. य॒ज्ञ्प॑तय॒ इतीति॑ य॒ज्ञ्प॑तये य॒ज्ञ्प॑तय॒ इति॑ । \newline
19. य॒ज्ञ्प॑तय॒ इति॑ य॒ज्ञ् - प॒त॒ये॒ । \newline
20. इत्या॑हा॒हे तीत्या॑ह । \newline
21. आ॒हा॒शिष॑ मा॒शिष॑ माहा हा॒शिष᳚म् । \newline
22. आ॒शिष॑ मे॒वैवाशिष॑ मा॒शिष॑ मे॒व । \newline
23. आ॒शिष॒मित्या᳚ - शिष᳚म् । \newline
24. ए॒वैता मे॒ता मे॒वैवैताम् । \newline
25. ए॒ता मैता मे॒ता मा । \newline
26. आ शा᳚स्ते शास्त॒ आ शा᳚स्ते । \newline
27. शा॒स्ते॒ सोमꣳ॒॒ सोमꣳ॑ शास्ते शास्ते॒ सोम᳚म् । \newline
28. सोमं॑ ॅयजति यजति॒ सोमꣳ॒॒ सोमं॑ ॅयजति । \newline
29. य॒ज॒ति॒ रेतो॒ रेतो॑ यजति यजति॒ रेतः॑ । \newline
30. रेत॑ ए॒वैव रेतो॒ रेत॑ ए॒व । \newline
31. ए॒व तत् तदे॒वैव तत् । \newline
32. तद् द॑धाति दधाति॒ तत् तद् द॑धाति । \newline
33. द॒धा॒ति॒ त्वष्टा॑र॒म् त्वष्टा॑रम् दधाति दधाति॒ त्वष्टा॑रम् । \newline
34. त्वष्टा॑रं ॅयजति यजति॒ त्वष्टा॑र॒म् त्वष्टा॑रं ॅयजति । \newline
35. य॒ज॒ति॒ रेतो॒ रेतो॑ यजति यजति॒ रेतः॑ । \newline
36. रेत॑ ए॒वैव रेतो॒ रेत॑ ए॒व । \newline
37. ए॒व हि॒तꣳ हि॒त मे॒वैव हि॒तम् । \newline
38. हि॒तम् त्वष्टा॒ त्वष्टा॑ हि॒तꣳ हि॒तम् त्वष्टा᳚ । \newline
39. त्वष्टा॑ रू॒पाणि॑ रू॒पाणि॒ त्वष्टा॒ त्वष्टा॑ रू॒पाणि॑ । \newline
40. रू॒पाणि॒ वि वि रू॒पाणि॑ रू॒पाणि॒ वि । \newline
41. वि क॑रोति करोति॒ वि वि क॑रोति । \newline
42. क॒रो॒ति॒ दे॒वाना᳚म् दे॒वाना᳚म् करोति करोति दे॒वाना᳚म् । \newline
43. दे॒वाना॒म् पत्नीः॒ पत्नी᳚र् दे॒वाना᳚म् दे॒वाना॒म् पत्नीः᳚ । \newline
44. पत्नी᳚र् यजति यजति॒ पत्नीः॒ पत्नी᳚र् यजति । \newline
45. य॒ज॒ति॒ मि॒थु॒न॒त्वाय॑ मिथुन॒त्वाय॑ यजति यजति मिथुन॒त्वाय॑ । \newline
46. मि॒थु॒न॒त्वाया॒ग्नि म॒ग्निम् मि॑थुन॒त्वाय॑ मिथुन॒त्वाया॒ग्निम् । \newline
47. मि॒थु॒न॒त्वायेति॑ मिथुन - त्वाय॑ । \newline
48. अ॒ग्निम् गृ॒हप॑तिम् गृ॒हप॑ति म॒ग्नि म॒ग्निम् गृ॒हप॑तिम् । \newline
49. गृ॒हप॑तिं ॅयजति यजति गृ॒हप॑तिम् गृ॒हप॑तिं ॅयजति । \newline
50. गृ॒हप॑ति॒मिति॑ गृ॒ह - प॒ति॒म् । \newline
51. य॒ज॒ति॒ प्रति॑ष्ठित्यै॒ प्रति॑ष्ठित्यै यजति यजति॒ प्रति॑ष्ठित्यै । \newline
52. प्रति॑ष्ठित्यै जा॒मि जा॒मि प्रति॑ष्ठित्यै॒ प्रति॑ष्ठित्यै जा॒मि । \newline
53. प्रति॑ष्ठित्या॒ इति॒ प्रति॑ - स्थि॒त्यै॒ । \newline
54. जा॒मि वै वै जा॒मि जा॒मि वै । \newline
55. वा ए॒त दे॒तद् वै वा ए॒तत् । \newline
56. ए॒तद् य॒ज्ञ्स्य॑ य॒ज्ञ् स्यै॒त दे॒तद् य॒ज्ञ्स्य॑ । \newline
57. य॒ज्ञ्स्य॑ क्रियते क्रियते य॒ज्ञ्स्य॑ य॒ज्ञ्स्य॑ क्रियते । \newline
58. क्रि॒य॒ते॒ यद् यत् क्रि॑यते क्रियते॒ यत् । \newline

\textbf{Ghana Paata } \newline

1. शं॒ॅयोरा शं॒ॅयोः शं॒ॅयोरा वृ॑णीमहे वृणीमह॒ आ शं॒ॅयोः शं॒ॅयोरा वृ॑णीमहे । \newline
2. शं॒ॅयोरिति॑ शं - योः । \newline
3. आ वृ॑णीमहे वृणीमह॒ आ वृ॑णीमह॒ इतीति॑ वृणीमह॒ आ वृ॑णीमह॒ इति॑ । \newline
4. वृ॒णी॒म॒ह॒ इतीति॑ वृणीमहे वृणीमह॒ इत्या॑हा॒हे ति॑ वृणीमहे वृणीमह॒ इत्या॑ह । \newline
5. इत्या॑हा॒हे तीत्या॑ह शं॒ॅयुꣳ शं॒ॅयु मा॒हे तीत्या॑ह शं॒ॅयुम् । \newline
6. आ॒ह॒ शं॒ॅयुꣳ शं॒ॅयु मा॑हाह शं॒ॅयु मे॒वैव शं॒ॅयु मा॑हाह शं॒ॅयु मे॒व । \newline
7. शं॒ॅयु मे॒वैव शं॒ॅयुꣳ शं॒ॅयु मे॒व बा॑र्.हस्प॒त्यम् बा॑र्.हस्प॒त्य मे॒व शं॒ॅयुꣳ शं॒ॅयु मे॒व बा॑र्.हस्प॒त्यम् । \newline
8. शं॒ॅयुमिति॑ शं - युम् । \newline
9. ए॒व बा॑र्.हस्प॒त्यम् बा॑र्.हस्प॒त्य मे॒वैव बा॑र्.हस्प॒त्यम् भा॑ग॒धेये॑न भाग॒धेये॑न बार्.हस्प॒त्य मे॒वैव बा॑र्.हस्प॒त्यम् भा॑ग॒धेये॑न । \newline
10. बा॒र्॒.ह॒स्प॒त्यम् भा॑ग॒धेये॑न भाग॒धेये॑न बार्.हस्प॒त्यम् बा॑र्.हस्प॒त्यम् भा॑ग॒धेये॑न॒ सꣳ सम् भा॑ग॒धेये॑न बार्.हस्प॒त्यम् बा॑र्.हस्प॒त्यम् भा॑ग॒धेये॑न॒ सम् । \newline
11. भा॒ग॒धेये॑न॒ सꣳ सम् भा॑ग॒धेये॑न भाग॒धेये॑न॒ स म॑र्द्धय त्यर्द्धयति॒ सम् भा॑ग॒धेये॑न भाग॒धेये॑न॒ स म॑र्द्धयति । \newline
12. भा॒ग॒धेये॒नेति॑ भाग - धेये॑न । \newline
13. स म॑र्द्धय त्यर्द्धयति॒ सꣳ स म॑र्द्धयति गा॒तुम् गा॒तु म॑र्द्धयति॒ सꣳ स म॑र्द्धयति गा॒तुम् । \newline
14. अ॒र्द्ध॒य॒ति॒ गा॒तुम् गा॒तु म॑र्द्धय त्यर्द्धयति गा॒तुं ॅय॒ज्ञाय॑ य॒ज्ञाय॑ गा॒तु म॑र्द्धय त्यर्द्धयति गा॒तुं ॅय॒ज्ञाय॑ । \newline
15. गा॒तुं ॅय॒ज्ञाय॑ य॒ज्ञाय॑ गा॒तुम् गा॒तुं ॅय॒ज्ञाय॑ गा॒तुम् गा॒तुं ॅय॒ज्ञाय॑ गा॒तुम् गा॒तुं ॅय॒ज्ञाय॑ गा॒तुम् । \newline
16. य॒ज्ञाय॑ गा॒तुम् गा॒तुं ॅय॒ज्ञाय॑ य॒ज्ञाय॑ गा॒तुं ॅय॒ज्ञ्प॑तये य॒ज्ञ्प॑तये गा॒तुं ॅय॒ज्ञाय॑ य॒ज्ञाय॑ गा॒तुं ॅय॒ज्ञ्प॑तये । \newline
17. गा॒तुं ॅय॒ज्ञ्प॑तये य॒ज्ञ्प॑तये गा॒तुम् गा॒तुं ॅय॒ज्ञ्प॑तय॒ इतीति॑ य॒ज्ञ्प॑तये गा॒तुम् गा॒तुं ॅय॒ज्ञ्प॑तय॒ इति॑ । \newline
18. य॒ज्ञ्प॑तय॒ इतीति॑ य॒ज्ञ्प॑तये य॒ज्ञ्प॑तय॒ इत्या॑हा॒हे ति॑ य॒ज्ञ्प॑तये य॒ज्ञ्प॑तय॒ इत्या॑ह । \newline
19. य॒ज्ञ्प॑तय॒ इति॑ य॒ज्ञ् - प॒त॒ये॒ । \newline
20. इत्या॑हा॒हे तीत्या॑ हा॒शिष॑ मा॒शिष॑ मा॒हे तीत्या॑हा॒ शिष᳚म् । \newline
21. आ॒हा॒शिष॑ मा॒शिष॑ माहाहा॒ शिष॑ मे॒वैवा शिष॑ माहाहा॒ शिष॑ मे॒व । \newline
22. आ॒शिष॑ मे॒वैवाशिष॑ मा॒शिष॑ मे॒वैता मे॒ता मे॒वाशिष॑ मा॒शिष॑ मे॒वैताम् । \newline
23. आ॒शिष॒मित्या᳚ - शिष᳚म् । \newline
24. ए॒वैता मे॒ता मे॒वैवैता मैता मे॒वैवैता मा । \newline
25. ए॒ता मैता मे॒ता मा शा᳚स्ते शास्त॒ ऐता मे॒ता मा शा᳚स्ते । \newline
26. आ शा᳚स्ते शास्त॒ आ शा᳚स्ते॒ सोमꣳ॒॒ सोमꣳ॑ शास्त॒ आ शा᳚स्ते॒ सोम᳚म् । \newline
27. शा॒स्ते॒ सोमꣳ॒॒ सोमꣳ॑ शास्ते शास्ते॒ सोमं॑ ॅयजति यजति॒ सोमꣳ॑ शास्ते शास्ते॒ सोमं॑ ॅयजति । \newline
28. सोमं॑ ॅयजति यजति॒ सोमꣳ॒॒ सोमं॑ ॅयजति॒ रेतो॒ रेतो॑ यजति॒ सोमꣳ॒॒ सोमं॑ ॅयजति॒ रेतः॑ । \newline
29. य॒ज॒ति॒ रेतो॒ रेतो॑ यजति यजति॒ रेत॑ ए॒वैव रेतो॑ यजति यजति॒ रेत॑ ए॒व । \newline
30. रेत॑ ए॒वैव रेतो॒ रेत॑ ए॒व तत् तदे॒व रेतो॒ रेत॑ ए॒व तत् । \newline
31. ए॒व तत् तदे॒वैव तद् द॑धाति दधाति॒ तदे॒वैव तद् द॑धाति । \newline
32. तद् द॑धाति दधाति॒ तत् तद् द॑धाति॒ त्वष्टा॑र॒म् त्वष्टा॑रम् दधाति॒ तत् तद् द॑धाति॒ त्वष्टा॑रम् । \newline
33. द॒धा॒ति॒ त्वष्टा॑र॒म् त्वष्टा॑रम् दधाति दधाति॒ त्वष्टा॑रं ॅयजति यजति॒ त्वष्टा॑रम् दधाति दधाति॒ त्वष्टा॑रं ॅयजति । \newline
34. त्वष्टा॑रं ॅयजति यजति॒ त्वष्टा॑र॒म् त्वष्टा॑रं ॅयजति॒ रेतो॒ रेतो॑ यजति॒ त्वष्टा॑र॒म् त्वष्टा॑रं ॅयजति॒ रेतः॑ । \newline
35. य॒ज॒ति॒ रेतो॒ रेतो॑ यजति यजति॒ रेत॑ ए॒वैव रेतो॑ यजति यजति॒ रेत॑ ए॒व । \newline
36. रेत॑ ए॒वैव रेतो॒ रेत॑ ए॒व हि॒तꣳ हि॒त मे॒व रेतो॒ रेत॑ ए॒व हि॒तम् । \newline
37. ए॒व हि॒तꣳ हि॒त मे॒वैव हि॒तम् त्वष्टा॒ त्वष्टा॑ हि॒त मे॒वैव हि॒तम् त्वष्टा᳚ । \newline
38. हि॒तम् त्वष्टा॒ त्वष्टा॑ हि॒तꣳ हि॒तम् त्वष्टा॑ रू॒पाणि॑ रू॒पाणि॒ त्वष्टा॑ हि॒तꣳ हि॒तम् त्वष्टा॑ रू॒पाणि॑ । \newline
39. त्वष्टा॑ रू॒पाणि॑ रू॒पाणि॒ त्वष्टा॒ त्वष्टा॑ रू॒पाणि॒ वि वि रू॒पाणि॒ त्वष्टा॒ त्वष्टा॑ रू॒पाणि॒ वि । \newline
40. रू॒पाणि॒ वि वि रू॒पाणि॑ रू॒पाणि॒ वि क॑रोति करोति॒ वि रू॒पाणि॑ रू॒पाणि॒ वि क॑रोति । \newline
41. वि क॑रोति करोति॒ वि वि क॑रोति दे॒वाना᳚म् दे॒वाना᳚म् करोति॒ वि वि क॑रोति दे॒वाना᳚म् । \newline
42. क॒रो॒ति॒ दे॒वाना᳚म् दे॒वाना᳚म् करोति करोति दे॒वाना॒म् पत्नीः॒ पत्नी᳚र् दे॒वाना᳚म् करोति करोति दे॒वाना॒म् पत्नीः᳚ । \newline
43. दे॒वाना॒म् पत्नीः॒ पत्नी᳚र् दे॒वाना᳚म् दे॒वाना॒म् पत्नी᳚र् यजति यजति॒ पत्नी᳚र् दे॒वाना᳚म् दे॒वाना॒म् पत्नी᳚र् यजति । \newline
44. पत्नी᳚र् यजति यजति॒ पत्नीः॒ पत्नी᳚र् यजति मिथुन॒त्वाय॑ मिथुन॒त्वाय॑ यजति॒ पत्नीः॒ पत्नी᳚र् यजति मिथुन॒त्वाय॑ । \newline
45. य॒ज॒ति॒ मि॒थु॒न॒त्वाय॑ मिथुन॒त्वाय॑ यजति यजति मिथुन॒त्वाया॒ ग्नि म॒ग्निम् मि॑थुन॒त्वाय॑ यजति यजति मिथुन॒त्वाया॒ ग्निम् । \newline
46. मि॒थु॒न॒त्वाया॒ग्नि म॒ग्निम् मि॑थुन॒त्वाय॑ मिथुन॒त्वाया॒ग्निम् गृ॒हप॑तिम् गृ॒हप॑ति म॒ग्निम् मि॑थुन॒त्वाय॑ मिथुन॒त्वाया॒ग्निम् गृ॒हप॑तिम् । \newline
47. मि॒थु॒न॒त्वायेति॑ मिथुन - त्वाय॑ । \newline
48. अ॒ग्निम् गृ॒हप॑तिम् गृ॒हप॑ति म॒ग्नि म॒ग्निम् गृ॒हप॑तिं ॅयजति यजति गृ॒हप॑ति म॒ग्नि म॒ग्निम् गृ॒हप॑तिं ॅयजति । \newline
49. गृ॒हप॑तिं ॅयजति यजति गृ॒हप॑तिम् गृ॒हप॑तिं ॅयजति॒ प्रति॑ष्ठित्यै॒ प्रति॑ष्ठित्यै यजति गृ॒हप॑तिम् गृ॒हप॑तिं ॅयजति॒ प्रति॑ष्ठित्यै । \newline
50. गृ॒हप॑ति॒मिति॑ गृ॒ह - प॒ति॒म् । \newline
51. य॒ज॒ति॒ प्रति॑ष्ठित्यै॒ प्रति॑ष्ठित्यै यजति यजति॒ प्रति॑ष्ठित्यै जा॒मि जा॒मि प्रति॑ष्ठित्यै यजति यजति॒ प्रति॑ष्ठित्यै जा॒मि । \newline
52. प्रति॑ष्ठित्यै जा॒मि जा॒मि प्रति॑ष्ठित्यै॒ प्रति॑ष्ठित्यै जा॒मि वै वै जा॒मि प्रति॑ष्ठित्यै॒ प्रति॑ष्ठित्यै जा॒मि वै । \newline
53. प्रति॑ष्ठित्या॒ इति॒ प्रति॑ - स्थि॒त्यै॒ । \newline
54. जा॒मि वै वै जा॒मि जा॒मि वा ए॒त दे॒तद् वै जा॒मि जा॒मि वा ए॒तत् । \newline
55. वा ए॒त दे॒तद् वै वा ए॒तद् य॒ज्ञ्स्य॑ य॒ज्ञ्स्यै॒तद् वै वा ए॒तद् य॒ज्ञ्स्य॑ । \newline
56. ए॒तद् य॒ज्ञ्स्य॑ य॒ज्ञ् स्यै॒त दे॒तद् य॒ज्ञ्स्य॑ क्रियते क्रियते य॒ज्ञ् स्यै॒त दे॒तद् य॒ज्ञ्स्य॑ क्रियते । \newline
57. य॒ज्ञ्स्य॑ क्रियते क्रियते य॒ज्ञ्स्य॑ य॒ज्ञ्स्य॑ क्रियते॒ यद् यत् क्रि॑यते य॒ज्ञ्स्य॑ य॒ज्ञ्स्य॑ क्रियते॒ यत् । \newline
58. क्रि॒य॒ते॒ यद् यत् क्रि॑यते क्रियते॒ यदाज्ये॒ नाज्ये॑न॒ यत् क्रि॑यते क्रियते॒ यदाज्ये॑न । \newline
\pagebreak
\markright{ TS 2.6.10.4  \hfill https://www.vedavms.in \hfill}
\addcontentsline{toc}{section}{ TS 2.6.10.4 }
\section*{ TS 2.6.10.4 }

\textbf{TS 2.6.10.4 } \newline
\textbf{Samhita Paata} \newline

यदाज्ये॑न प्रया॒जा इ॒ज्यन्त॒ आज्ये॑न पत्नीसंॅया॒जा ऋच॑म॒नूच्य॑ पत्नीसंॅया॒जाना॑मृ॒चा य॑ज॒त्यजा॑मित्वा॒याथो॑ मिथुन॒त्वाय॑ प॒ङ्क्ति प्रा॑यणो॒ वै य॒ज्ञ्ः प॒ङ्क्त्यु॑दयनः॒ पञ्च॑ प्रया॒जा इ॑ज्यन्ते च॒त्वारः॑ पत्नीसंॅया॒जाः स॑मिष्टय॒जुः प॑ञ्च॒मं प॒ङ्क्तिमे॒वानु॑ प्र॒यन्ति॑ प॒ङ्क्तिमनूद्य॑न्ति ॥ \newline

\textbf{Pada Paata} \newline

यत् । आज्ये॑न । प्र॒या॒जा इति॑ प्र - या॒जाः । इ॒ज्यन्ते᳚ । आज्ये॑न । प॒त्नी॒सं॒ॅया॒जा इति॑ पत्नी - सं॒ॅया॒जाः । ऋच᳚म् । अ॒नूच्येत्य॑नु - उच्य॑ । प॒त्नी॒सं॒ॅया॒जाना॒मिति॑ पत्नी - सं॒ॅया॒जाना᳚म् । ऋ॒चा । य॒ज॒ति॒ । अजा॑मित्वा॒येत्यजा॑मि - त्वा॒य॒ । अथो॒ इति॑ । मि॒थु॒न॒त्वायेति॑ मिथुन - त्वाय॑ । प॒ङ्क्तिप्रा॑यण॒ इति॑ प॒ङ्क्ति - प्रा॒य॒णः॒ । वै । य॒ज्ञ्ः । प॒ङ्क्त्यु॑दयन॒ इति॑ प॒ङ्क्ति - उ॒द॒य॒नः॒ । पञ्च॑ । प्र॒या॒जा इति॑ प्र - या॒जाः । इ॒ज्य॒न्ते॒ । च॒त्वारः॑ । प॒त्नी॒सं॒ॅया॒जा इति॑ पत्नी - सं॒ॅया॒जाः । स॒मि॒ष्ट॒य॒जुरिति॑ समिष्ट - य॒जुः । प॒ञ्च॒मम् । प॒ङ्क्तिम् । ए॒व । अन्विति॑ । प्र॒यन्तीति॑ प्र - यन्ति॑ । प॒ङ्क्तिम् । अनु॑ । उदिति॑ । य॒न्ति॒ ॥  \newline


\textbf{Krama Paata} \newline

यदाज्ये॑न । आज्ये॑न प्रया॒जाः । प्र॒या॒जा इ॒ज्यन्ते᳚ । प्र॒या॒जा इति॑ प्र - या॒जाः । इ॒ज्यन्त॒ आज्ये॑न । आज्ये॑न पत्नीसम्ॅया॒जाः । प॒त्नी॒स॒म्ॅया॒जा ऋच᳚म् । प॒त्नी॒स॒म्ॅया॒जा इति॑ पत्नी - स॒म्ॅया॒जाः । ऋच॑म॒नूच्य॑ । अ॒नूच्य॑ पत्नीसम्ॅया॒जाना᳚म् । अ॒नूच्येत्य॑नु - उच्य॑ । प॒त्नी॒स॒म्ॅया॒जाना॑मृ॒चा । प॒त्नी॒स॒म्ॅया॒जाना॒मिति॑ पत्नी - स॒म्ॅया॒जाना᳚म् । ऋ॒चा य॑जति । य॒ज॒त्यजा॑मित्वाय । अजा॑मित्वा॒याथो᳚ । अजा॑मित्वा॒येत्यजा॑मि - त्वा॒य॒ । अथो॑ मिथुन॒त्वाय॑ । अथो॒ इत्यथो᳚ । मि॒थु॒न॒त्वाय॑ प॒ङ्क्तिप्रा॑यणः । मि॒थु॒न॒त्वायेति॑ मिथुन - त्वाय॑ । प॒ङ्क्तिप्रा॑यणो॒ वै । प॒ङ्क्तिप्रा॑यण॒ इति॑ प॒ङ्क्ति - प्रा॒य॒णः॒ । वै य॒ज्ञ्ः । य॒ज्ञ्ः प॒ङ्क्त्यु॑दयनः । प॒ङ्क्त्यु॑दयनः॒ पञ्च॑ । प॒ङ्क्त्यु॑दयन॒ इति॑ प॒ङ्क्ति - उ॒द॒य॒नः॒ । पञ्च॑ प्रया॒जाः । प्र॒या॒जा इ॑ज्यन्ते । प्र॒या॒जा इति॑ प्र - या॒जाः । इ॒ज्य॒न्ते॒ च॒त्वारः॑ । च॒त्वारः॑ पत्नीसम्ॅया॒जाः । प॒त्नी॒स॒म्ॅया॒जाः स॑मिष्टय॒जुः । प॒त्नी॒स॒म्ॅया॒जा इति॑ पत्नी - स॒म्ॅया॒जाः । स॒मि॒ष्ट॒य॒जुः प॑ञ्च॒मम् । स॒मि॒ष्ट॒य॒जुरिति॑ समिष्ट - य॒जुः । प॒ञ्च॒मम् प॒ङ्क्तिम् । प॒ङ्क्तिमे॒व । ए॒वानु॑ । अनु॑ प्र॒यन्ति॑ । प्र॒यन्ति॑ प॒ङ्क्तिम् । प्र॒यन्तीति॑ प्र - यन्ति॑ । प॒ङ्क्तिमनु॑ । अनूत् । उद् य॑न्ति । य॒न्तीति॑ यन्ति । \newline

\textbf{Jatai Paata} \newline

1. यदाज्ये॒ नाज्ये॑न॒ यद् यदाज्ये॑न । \newline
2. आज्ये॑न प्रया॒जाः प्र॑या॒जा आज्ये॒ नाज्ये॑न प्रया॒जाः । \newline
3. प्र॒या॒जा इ॒ज्यन्त॑ इ॒ज्यन्ते᳚ प्रया॒जाः प्र॑या॒जा इ॒ज्यन्ते᳚ । \newline
4. प्र॒या॒जा इति॑ प्र - या॒जाः । \newline
5. इ॒ज्यन्त॒ आज्ये॒ नाज्ये॑ने॒ ज्यन्त॑ इ॒ज्यन्त॒ आज्ये॑न । \newline
6. आज्ये॑न पत्नीसंॅया॒जाः प॑त्नीसंॅया॒जा आज्ये॒ नाज्ये॑न पत्नीसंॅया॒जाः । \newline
7. प॒त्नी॒सं॒ॅया॒जा ऋच॒ मृच॑म् पत्नीसंॅया॒जाः प॑त्नीसंॅया॒जा ऋच᳚म् । \newline
8. प॒त्नी॒सं॒ॅया॒जा इति॑ पत्नी - सं॒ॅया॒जाः । \newline
9. ऋच॑ म॒नूच्या॒ नूच्य र्च॒ मृच॑ म॒नूच्य॑ । \newline
10. अ॒नूच्य॑ पत्नीसंॅया॒जाना᳚म् पत्नीसंॅया॒जाना॑ म॒नूच्या॒ नूच्य॑ पत्नीसंॅया॒जाना᳚म् । \newline
11. अ॒नूच्येत्य॑नु - उच्य॑ । \newline
12. प॒त्नी॒सं॒ॅया॒जाना॑ मृ॒चर्चा प॑त्नीसंॅया॒जाना᳚म् पत्नीसंॅया॒जाना॑ मृ॒चा । \newline
13. प॒त्नी॒सं॒ॅया॒जाना॒मिति॑ पत्नी - सं॒ॅया॒जाना᳚म् । \newline
14. ऋ॒चा य॑जति यज त्यृ॒चर्चा य॑जति । \newline
15. य॒ज॒ त्यजा॑मित्वा॒या जा॑मित्वाय यजति यज॒ त्यजा॑मित्वाय । \newline
16. अजा॑मित्वा॒या थो॒ अथो॒ अजा॑मित्वा॒या जा॑मित्वा॒या थो᳚ । \newline
17. अजा॑मित्वा॒येत्यजा॑मि - त्वा॒य॒ । \newline
18. अथो॑ मिथुन॒त्वाय॑ मिथुन॒त्वाया थो॒ अथो॑ मिथुन॒त्वाय॑ । \newline
19. अथो॒ इत्यथो᳚ । \newline
20. मि॒थु॒न॒त्वाय॑ प॒ङ्क्तिप्रा॑यणः प॒ङ्क्तिप्रा॑यणो मिथुन॒त्वाय॑ मिथुन॒त्वाय॑ प॒ङ्क्तिप्रा॑यणः । \newline
21. मि॒थु॒न॒त्वायेति॑ मिथुन - त्वाय॑ । \newline
22. प॒ङ्क्तिप्रा॑यणो॒ वै वै प॒ङ्क्तिप्रा॑यणः प॒ङ्क्तिप्रा॑यणो॒ वै । \newline
23. प॒ङ्क्तिप्रा॑यण॒ इति॑ प॒ङ्क्ति - प्रा॒य॒णः॒ । \newline
24. वै य॒ज्ञो य॒ज्ञो वै वै य॒ज्ञ्ः । \newline
25. य॒ज्ञ्ः प॒ङ्क्त्यु॑दयनः प॒ङ्क्त्यु॑दयनो य॒ज्ञो य॒ज्ञ्ः प॒ङ्क्त्यु॑दयनः । \newline
26. प॒ङ्क्त्यु॑दयनः॒ पञ्च॒ पञ्च॑ प॒ङ्क्त्यु॑दयनः प॒ङ्क्त्यु॑दयनः॒ पञ्च॑ । \newline
27. प॒ङ्क्त्यु॑दयन॒ इति॑ प॒ङ्क्ति - उ॒द॒य॒नः॒ । \newline
28. पञ्च॑ प्रया॒जाः प्र॑या॒जाः पञ्च॒ पञ्च॑ प्रया॒जाः । \newline
29. प्र॒या॒जा इ॑ज्यन्त इज्यन्ते प्रया॒जाः प्र॑या॒जा इ॑ज्यन्ते । \newline
30. प्र॒या॒जा इति॑ प्र - या॒जाः । \newline
31. इ॒ज्य॒न्ते॒ च॒त्वार॑ श्च॒त्वार॑ इज्यन्त इज्यन्ते च॒त्वारः॑ । \newline
32. च॒त्वारः॑ पत्नीसंॅया॒जाः प॑त्नीसंॅया॒जा श्च॒त्वार॑ श्च॒त्वारः॑ पत्नीसंॅया॒जाः । \newline
33. प॒त्नी॒सं॒ॅया॒जाः स॑मिष्टय॒जुः स॑मिष्टय॒जुः प॑त्नीसंॅया॒जाः प॑त्नीसंॅया॒जाः स॑मिष्टय॒जुः । \newline
34. प॒त्नी॒सं॒ॅया॒जा इति॑ पत्नी - सं॒ॅया॒जाः । \newline
35. स॒मि॒ष्ट॒य॒जुः प॑ञ्च॒मम् प॑ञ्च॒मꣳ स॑मिष्टय॒जुः स॑मिष्टय॒जुः प॑ञ्च॒मम् । \newline
36. स॒मि॒ष्ट॒य॒जुरिति॑ समिष्ट - य॒जुः । \newline
37. प॒ञ्च॒मम् प॒ङ्क्तिम् प॒ङ्क्तिम् प॑ञ्च॒मम् प॑ञ्च॒मम् प॒ङ्क्तिम् । \newline
38. प॒ङ्क्ति मे॒वैव प॒ङ्क्तिम् प॒ङ्क्ति मे॒व । \newline
39. ए॒वा न्वन्वे॒ वैवानु॑ । \newline
40. अनु॑ प्र॒यन्ति॑ प्र॒य न्त्यन्वनु॑ प्र॒यन्ति॑ । \newline
41. प्र॒यन्ति॑ प॒ङ्क्तिम् प॒ङ्क्तिम् प्र॒यन्ति॑ प्र॒यन्ति॑ प॒ङ्क्तिम् । \newline
42. प्र॒यन्तीति॑ प्र - यन्ति॑ । \newline
43. प॒ङ्क्ति मन्वनु॑ प॒ङ्क्तिम् प॒ङ्क्ति मनु॑ । \newline
44. अनू दुद न्वनूत् । \newline
45. उद् य॑न्ति य॒ न्त्युदुद् य॑न्ति । \newline
46. य॒न्तीति॑ यन्ति । \newline

\textbf{Ghana Paata } \newline

1. यदाज्ये॒ नाज्ये॑न॒ यद् यदाज्ये॑न प्रया॒जाः प्र॑या॒जा आज्ये॑न॒ यद् यदाज्ये॑न प्रया॒जाः । \newline
2. आज्ये॑न प्रया॒जाः प्र॑या॒जा आज्ये॒ नाज्ये॑न प्रया॒जा इ॒ज्यन्त॑ इ॒ज्यन्ते᳚ प्रया॒जा आज्ये॒ नाज्ये॑न प्रया॒जा इ॒ज्यन्ते᳚ । \newline
3. प्र॒या॒जा इ॒ज्यन्त॑ इ॒ज्यन्ते᳚ प्रया॒जाः प्र॑या॒जा इ॒ज्यन्त॒ आज्ये॒ नाज्ये॑ने॒ ज्यन्ते᳚ प्रया॒जाः प्र॑या॒जा इ॒ज्यन्त॒ आज्ये॑न । \newline
4. प्र॒या॒जा इति॑ प्र - या॒जाः । \newline
5. इ॒ज्यन्त॒ आज्ये॒ नाज्ये॑ने॒ ज्यन्त॑ इ॒ज्यन्त॒ आज्ये॑न पत्नीसंॅया॒जाः प॑त्नीसंॅया॒जा आज्ये॑ने॒ ज्यन्त॑ इ॒ज्यन्त॒ आज्ये॑न पत्नीसंॅया॒जाः । \newline
6. आज्ये॑न पत्नीसंॅया॒जाः प॑त्नीसंॅया॒जा आज्ये॒ नाज्ये॑न पत्नीसंॅया॒जा ऋच॒ मृच॑म् पत्नीसंॅया॒जा आज्ये॒ नाज्ये॑न पत्नीसंॅया॒जा ऋच᳚म् । \newline
7. प॒त्नी॒सं॒ॅया॒जा ऋच॒ मृच॑म् पत्नीसंॅया॒जाः प॑त्नीसंॅया॒जा ऋच॑ म॒नूच्या॒ नूच्य र्च॑म् पत्नीसंॅया॒जाः प॑त्नीसंॅया॒जा ऋच॑ म॒नूच्य॑ । \newline
8. प॒त्नी॒सं॒ॅया॒जा इति॑ पत्नी - सं॒ॅया॒जाः । \newline
9. ऋच॑ म॒नूच्या॒ नूच्य र्च॒ मृच॑ म॒नूच्य॑ पत्नीसंॅया॒जाना᳚म् पत्नीसंॅया॒जाना॑ म॒नूच्य र्च॒ मृच॑ म॒नूच्य॑ पत्नीसंॅया॒जाना᳚म् । \newline
10. अ॒नूच्य॑ पत्नीसंॅया॒जाना᳚म् पत्नीसंॅया॒जाना॑ म॒नूच्या॒ नूच्य॑ पत्नीसंॅया॒जाना॑ मृ॒चर्चा प॑त्नीसंॅया॒जाना॑ म॒नूच्या॒ नूच्य॑ पत्नीसंॅया॒जाना॑ मृ॒चा । \newline
11. अ॒नूच्येत्य॑नु - उच्य॑ । \newline
12. प॒त्नी॒सं॒ॅया॒जाना॑ मृ॒चर्चा प॑त्नीसंॅया॒जाना᳚म् पत्नीसंॅया॒जाना॑ मृ॒चा य॑जति यज त्यृ॒चा प॑त्नीसंॅया॒जाना᳚म् पत्नीसंॅया॒जाना॑ मृ॒चा य॑जति । \newline
13. प॒त्नी॒सं॒ॅया॒जाना॒मिति॑ पत्नी - सं॒ॅया॒जाना᳚म् । \newline
14. ऋ॒चा य॑जति यजत्यृ॒चर्चा य॑ज॒ त्यजा॑मित्वा॒या जा॑मित्वाय यजत्यृ॒चर्चा य॑ज॒ त्यजा॑मित्वाय । \newline
15. य॒ज॒ त्यजा॑मित्वा॒या जा॑मित्वाय यजति यज॒ त्यजा॑मित्वा॒याथो॒ अथो॒ अजा॑मित्वाय यजति यज॒ त्यजा॑मित्वा॒याथो᳚ । \newline
16. अजा॑मित्वा॒याथो॒ अथो॒ अजा॑मित्वा॒या जा॑मित्वा॒याथो॑ मिथुन॒त्वाय॑ मिथुन॒त्वायाथो॒ अजा॑मित्वा॒या जा॑मित्वा॒याथो॑ मिथुन॒त्वाय॑ । \newline
17. अजा॑मित्वा॒येत्यजा॑मि - त्वा॒य॒ । \newline
18. अथो॑ मिथुन॒त्वाय॑ मिथुन॒त्वायाथो॒ अथो॑ मिथुन॒त्वाय॑ प॒ङ्क्तिप्रा॑यणः प॒ङ्क्तिप्रा॑यणो मिथुन॒त्वायाथो॒ अथो॑ मिथुन॒त्वाय॑ प॒ङ्क्तिप्रा॑यणः । \newline
19. अथो॒ इत्यथो᳚ । \newline
20. मि॒थु॒न॒त्वाय॑ प॒ङ्क्तिप्रा॑यणः प॒ङ्क्तिप्रा॑यणो मिथुन॒त्वाय॑ मिथुन॒त्वाय॑ प॒ङ्क्तिप्रा॑यणो॒ वै वै प॒ङ्क्तिप्रा॑यणो मिथुन॒त्वाय॑ मिथुन॒त्वाय॑ प॒ङ्क्तिप्रा॑यणो॒ वै । \newline
21. मि॒थु॒न॒त्वायेति॑ मिथुन - त्वाय॑ । \newline
22. प॒ङ्क्तिप्रा॑यणो॒ वै वै प॒ङ्क्तिप्रा॑यणः प॒ङ्क्तिप्रा॑यणो॒ वै य॒ज्ञो य॒ज्ञो वै प॒ङ्क्तिप्रा॑यणः प॒ङ्क्तिप्रा॑यणो॒ वै य॒ज्ञ्ः । \newline
23. प॒ङ्क्तिप्रा॑यण॒ इति॑ प॒ङ्क्ति - प्रा॒य॒णः॒ । \newline
24. वै य॒ज्ञो य॒ज्ञो वै वै य॒ज्ञ्ः प॒ङ्क्त्यु॑दयनः प॒ङ्क्त्यु॑दयनो य॒ज्ञो वै वै य॒ज्ञ्ः प॒ङ्क्त्यु॑दयनः । \newline
25. य॒ज्ञ्ः प॒ङ्क्त्यु॑दयनः प॒ङ्क्त्यु॑दयनो य॒ज्ञो य॒ज्ञ्ः प॒ङ्क्त्यु॑दयनः॒ पञ्च॒ पञ्च॑ प॒ङ्क्त्यु॑दयनो य॒ज्ञो य॒ज्ञ्ः प॒ङ्क्त्यु॑दयनः॒ पञ्च॑ । \newline
26. प॒ङ्क्त्यु॑दयनः॒ पञ्च॒ पञ्च॑ प॒ङ्क्त्यु॑दयनः प॒ङ्क्त्यु॑दयनः॒ पञ्च॑ प्रया॒जाः प्र॑या॒जाः पञ्च॑ प॒ङ्क्त्यु॑दयनः प॒ङ्क्त्यु॑दयनः॒ पञ्च॑ प्रया॒जाः । \newline
27. प॒ङ्क्त्यु॑दयन॒ इति॑ प॒ङ्क्ति - उ॒द॒य॒नः॒ । \newline
28. पञ्च॑ प्रया॒जाः प्र॑या॒जाः पञ्च॒ पञ्च॑ प्रया॒जा इ॑ज्यन्त इज्यन्ते प्रया॒जाः पञ्च॒ पञ्च॑ प्रया॒जा इ॑ज्यन्ते । \newline
29. प्र॒या॒जा इ॑ज्यन्त इज्यन्ते प्रया॒जाः प्र॑या॒जा इ॑ज्यन्ते च॒त्वार॑ श्च॒त्वार॑ इज्यन्ते प्रया॒जाः प्र॑या॒जा इ॑ज्यन्ते च॒त्वारः॑ । \newline
30. प्र॒या॒जा इति॑ प्र - या॒जाः । \newline
31. इ॒ज्य॒न्ते॒ च॒त्वार॑ श्च॒त्वार॑ इज्यन्त इज्यन्ते च॒त्वारः॑ पत्नीसंॅया॒जाः प॑त्नीसंॅया॒जा श्च॒त्वार॑ इज्यन्त इज्यन्ते च॒त्वारः॑ पत्नीसंॅया॒जाः । \newline
32. च॒त्वारः॑ पत्नीसंॅया॒जाः प॑त्नीसंॅया॒जा श्च॒त्वार॑ श्च॒त्वारः॑ पत्नीसंॅया॒जाः स॑मिष्टय॒जुः स॑मिष्टय॒जुः प॑त्नीसंॅया॒जा श्च॒त्वार॑ श्च॒त्वारः॑ पत्नीसंॅया॒जाः स॑मिष्टय॒जुः । \newline
33. प॒त्नी॒सं॒ॅया॒जाः स॑मिष्टय॒जुः स॑मिष्टय॒जुः प॑त्नीसंॅया॒जाः प॑त्नीसंॅया॒जाः स॑मिष्टय॒जुः प॑ञ्च॒मम् प॑ञ्च॒मꣳ स॑मिष्टय॒जुः प॑त्नीसंॅया॒जाः प॑त्नीसंॅया॒जाः स॑मिष्टय॒जुः प॑ञ्च॒मम् । \newline
34. प॒त्नी॒सं॒ॅया॒जा इति॑ पत्नी - सं॒ॅया॒जाः । \newline
35. स॒मि॒ष्ट॒य॒जुः प॑ञ्च॒मम् प॑ञ्च॒मꣳ स॑मिष्टय॒जुः स॑मिष्टय॒जुः प॑ञ्च॒मम् प॒ङ्क्तिम् प॒ङ्क्तिम् प॑ञ्च॒मꣳ स॑मिष्टय॒जुः स॑मिष्टय॒जुः प॑ञ्च॒मम् प॒ङ्क्तिम् । \newline
36. स॒मि॒ष्ट॒य॒जुरिति॑ समिष्ट - य॒जुः । \newline
37. प॒ञ्च॒मम् प॒ङ्क्तिम् प॒ङ्क्तिम् प॑ञ्च॒मम् प॑ञ्च॒मम् प॒ङ्क्ति मे॒वैव प॒ङ्क्तिम् प॑ञ्च॒मम् प॑ञ्च॒मम् प॒ङ्क्ति मे॒व । \newline
38. प॒ङ्क्ति मे॒वैव प॒ङ्क्तिम् प॒ङ्क्ति मे॒वा न्वन्वे॒व प॒ङ्क्तिम् प॒ङ्क्ति मे॒वानु॑ । \newline
39. ए॒वा न्वन्वे॒ वैवानु॑ प्र॒यन्ति॑ प्र॒य न्त्यन्वे॒ वैवानु॑ प्र॒यन्ति॑ । \newline
40. अनु॑ प्र॒यन्ति॑ प्र॒य न्त्यन्वनु॑ प्र॒यन्ति॑ प॒ङ्क्तिम् प॒ङ्क्तिम् प्र॒य न्त्यन्वनु॑ प्र॒यन्ति॑ प॒ङ्क्तिम् । \newline
41. प्र॒यन्ति॑ प॒ङ्क्तिम् प॒ङ्क्तिम् प्र॒यन्ति॑ प्र॒यन्ति॑ प॒ङ्क्ति मन्वनु॑ प॒ङ्क्तिम् प्र॒यन्ति॑ प्र॒यन्ति॑ प॒ङ्क्ति मनु॑ । \newline
42. प्र॒यन्तीति॑ प्र - यन्ति॑ । \newline
43. प॒ङ्क्ति मन्वनु॑ प॒ङ्क्तिम् प॒ङ्क्ति मनू दुदनु॑ प॒ङ्क्तिम् प॒ङ्क्ति मनूत् । \newline
44. अनू दुदन्वनूद् य॑न्ति य॒न्त्यु दन्वनूद् य॑न्ति । \newline
45. उद् य॑न्ति य॒न्त्यु दुद् य॑न्ति । \newline
46. य॒न्तीति॑ यन्ति । \newline
\pagebreak
\markright{ TS 2.6.11.1  \hfill https://www.vedavms.in \hfill}
\addcontentsline{toc}{section}{ TS 2.6.11.1 }
\section*{ TS 2.6.11.1 }

\textbf{TS 2.6.11.1 } \newline
\textbf{Samhita Paata} \newline

यु॒क्ष्वाहि दे॑व॒हूत॑माꣳ॒॒ अश्वाꣳ॑ अग्ने र॒थीरि॑व । नि होता॑ पू॒र्व्यः स॑दः ॥उ॒त नो॑ देव दे॒वाꣳ अच्छा॑ वोचो वि॒दुष्ट॑रः । श्रद्विश्वा॒ वार्या॑ कृधि ॥त्वꣳ ह॒ यद्य॑विष्ठ्॒य सह॑सः सूनवाहुत । ऋ॒तावा॑ य॒ज्ञियो॒ भुवः॑ ॥अ॒यम॒ग्निः स॑ह॒स्रिणो॒ वाज॑स्य श॒तिन॒स्पतिः॑ । मू॒र्द्धा क॒वी र॑यी॒णां ॥तं ने॒मिमृ॒भवो॑ य॒था ऽऽन॑मस्व॒ सहू॑तिभिः । नेदी॑यो य॒ज्ञ् - [  ] \newline

\textbf{Pada Paata} \newline

यु॒क्ष्व । हि । दे॒व॒हूत॑मा॒निति॑ देव-हूत॑मान् । अश्वान्॑ । अ॒ग्ने॒ । र॒थीः । इ॒व॒ ॥ नीति॑ । होता᳚ । पू॒र्व्यः । स॒दः॒ ॥ उ॒त । नः॒ । दे॒व॒ । दे॒वान् । अच्छ॑ । वो॒चः॒ । वि॒दुष्ट॑र॒ इति॑ वि॒दुः - त॒रः॒ ॥ श्रत् । विश्वा᳚ । वार्या᳚ । कृ॒धि॒ ॥ त्वम् । ह॒ । यत् । य॒वि॒ष्ठ्य॒ । सह॑सः । सू॒नो॒ । आ॒हु॒तेत्या᳚ - हु॒त॒ ॥ ऋ॒तावेत्यृ॒त - वा॒ । य॒ज्ञियः॑ । भुवः॑ ॥ अ॒यम् । अ॒ग्निः । स॒ह॒स्रिणः॑ । वाज॑स्य । श॒तिनः॑ । पतिः॑ ॥ मू॒र्द्धा । क॒विः । र॒यी॒णाम् ॥ तम् । ने॒मिम् । ऋ॒भवः॑ । य॒था॒ । एति॑ । न॒म॒स्व॒ । सहू॑तिभि॒रिति॒ सहू॑ति - भिः॒ ॥ नेदी॑यः । य॒ज्ञ्म् ।  \newline


\textbf{Krama Paata} \newline

यु॒क्ष्वा हि । हि दे॑व॒हूत॑मान् । दे॒व॒हूत॑माꣳ॒॒ अश्वान्॑ । दे॒व॒हूत॑मा॒निति॑ देव - हूत॑मान् । अश्वाꣳ॑ अग्ने । अ॒ग्ने॒ र॒थीः । र॒थीरि॑व । इ॒वे॒ती॑व ॥ नि होता᳚ । होता॑ पू॒र्व्यः । पू॒र्व्यः स॑दः । स॒द॒ इति॑ सदः ॥ उ॒त नः॑ । नो॒ दे॒व॒ । दे॒व॒ दे॒वान् । दे॒वाꣳ अच्छ॑ । अच्छा॑ वोचः । वो॒चो॒ वि॒दुष्ट॑रः । वि॒दुष्ट॑र॒ इति॑ वि॒दुः - त॒रः॒ ॥ श्रद् विश्वा᳚ । विश्वा॒ वार्या᳚ । वार्या॑ कृधि । कृ॒धीति॑ कृधि ॥ त्वꣳ ह॑ । ह॒ यत् । यद् य॑विष्ठ्य । य॒वि॒ष्ठ्य॒ सह॑सः । सह॑सः सूनो । सू॒न॒वा॒हु॒त॒ । आ॒हु॒तेत्या᳚ - हु॒त॒ ॥ ऋ॒तावा॑ य॒ज्ञियः॑ । ऋ॒तावेत्यृ॒त - वा॒ । य॒ज्ञियो॒ भुवः॑ । भुव॒ इति॒ भुवः॑ ॥ अ॒यम॒ग्निः । अ॒ग्निः स॑ह॒स्रिणः॑ । स॒ह॒स्रिणो॒ वाज॑स्य । वाज॑स्य श॒तिनः॑ । श॒तिन॒स्पतिः॑ । पति॒रिति॒ पतिः॑ ॥ मू॒र्द्धा क॒विः । क॒वी र॑यी॒णाम् । र॒यी॒णामिति॑ रयी॒णाम् ॥ तम् ने॒मिम् । ने॒मिमृ॒भवः॑ । ऋ॒भवो॑ यथा । य॒था । आ न॑मस्व । न॒म॒स्व॒ सहू॑तिभिः । सहू॑तिभि॒रिति॒ सहू॑ति - भिः॒ ॥ नेदी॑यो य॒ज्ञ्म् । य॒ज्ञ्म॑ङ्गिरः \newline

\textbf{Jatai Paata} \newline

1. यु॒क्ष्वा हि हि यु॒क्ष्व यु॒क्ष्वा हि । \newline
2. हि दे॑व॒हूत॑मान् देव॒हूत॑मा॒न्॒. हि हि दे॑व॒हूत॑मान् । \newline
3. दे॒व॒हूत॑माꣳ॒॒ अश्वाꣳ॒॒ अश्वा᳚न् देव॒हूत॑मान् देव॒हूत॑माꣳ॒॒ अश्वान्॑ । \newline
4. दे॒व॒हूत॑मा॒निति॑ देव - हूत॑मान् । \newline
5. अश्वाꣳ॑ अग्ने अ॒ग्ने ऽश्वाꣳ॒॒ अश्वाꣳ॑ अग्ने । \newline
6. अ॒ग्ने॒ र॒थी र॒थी र॑ग्ने अग्ने र॒थीः । \newline
7. र॒थी रि॑वे व र॒थी र॒थी रि॑व । \newline
8. इ॒वेती॑व । \newline
9. नि होता॒ होता॒ नि नि होता᳚ । \newline
10. होता॑ पू॒र्व्यः पू॒र्व्यो होता॒ होता॑ पू॒र्व्यः । \newline
11. पू॒र्व्यः स॑दः सदः पू॒र्व्यः पू॒र्व्यः स॑दः । \newline
12. स॒द॒ इति॑ सदः । \newline
13. उ॒त नो॑ न उ॒तोत नः॑ । \newline
14. नो॒ दे॒व॒ दे॒व॒ नो॒ नो॒ दे॒व॒ । \newline
15. दे॒व॒ दे॒वान् दे॒वान् दे॑व देव दे॒वान् । \newline
16. दे॒वाꣳ अच्छाच्छ॑ दे॒वान् दे॒वाꣳ अच्छ॑ । \newline
17. अच्छा॑ वोचो वो॒चो अच्छाच्छा॑ वोचः । \newline
18. वो॒चो॒ वि॒दुष्ट॑रो वि॒दुष्ट॑रो वोचो वोचो वि॒दुष्ट॑रः । \newline
19. वि॒दुष्ट॑र॒ इति॑ वि॒दुः - त॒रः॒ । \newline
20. श्रद् विश्वा॒ विश्वा॒ श्रच्छ्रद् विश्वा᳚ । \newline
21. विश्वा॒ वार्या॒ वार्या॒ विश्वा॒ विश्वा॒ वार्या᳚ । \newline
22. वार्या॑ कृधि कृधि॒ वार्या॒ वार्या॑ कृधि । \newline
23. कृ॒धीति॑ कृधि । \newline
24. त्वꣳ ह॑ ह॒ त्वम् त्वꣳ ह॑ । \newline
25. ह॒ यद् यद्ध॑ ह॒ यत् । \newline
26. यद् य॑विष्ठ्य यविष्ठ्य॒ यद् यद् य॑विष्ठ्य । \newline
27. य॒वि॒ष्ठ्य॒ सह॑सः॒ सह॑सो यविष्ठ्य यविष्ठ्य॒ सह॑सः । \newline
28. सह॑सः सूनो सूनो॒ सह॑सः॒ सह॑सः सूनो । \newline
29. सू॒न॒ वा॒हु॒ता॒ हु॒त॒ सू॒नो॒ सू॒न॒वा॒ हु॒त॒ । \newline
30. आ॒हु॒तेत्या᳚ - हु॒त॒ । \newline
31. ऋ॒तावा॑ य॒ज्ञियो॑ य॒ज्ञिय॑ ऋ॒ताव॒ र्‌तावा॑ य॒ज्ञियः॑ । \newline
32. ऋ॒तावेत्यृ॒त - वा॒ । \newline
33. य॒ज्ञियो॒ भुवो॒ भुवो॑ य॒ज्ञियो॑ य॒ज्ञियो॒ भुवः॑ । \newline
34. भुव॒ इति॒ भुवः॑ । \newline
35. अ॒य म॒ग्नि र॒ग्नि र॒य म॒य म॒ग्निः । \newline
36. अ॒ग्निः स॑ह॒स्रिणः॑ सह॒स्रिणो॑ अ॒ग्नि र॒ग्निः स॑ह॒स्रिणः॑ । \newline
37. स॒ह॒स्रिणो॒ वाज॑स्य॒ वाज॑स्य सह॒स्रिणः॑ सह॒स्रिणो॒ वाज॑स्य । \newline
38. वाज॑स्य श॒तिनः॑ श॒तिनो॒ वाज॑स्य॒ वाज॑स्य श॒तिनः॑ । \newline
39. श॒तिन॒ स्पति॒ ष्पतिः॑ श॒तिनः॑ श॒तिन॒ स्पतिः॑ । \newline
40. पति॒रिति॒ पतिः॑ । \newline
41. मू॒र्द्धा क॒विः क॒विर् मू॒र्द्धा मू॒र्द्धा क॒विः । \newline
42. क॒वी र॑यी॒णाꣳ र॑यी॒णाम् क॒विः क॒वी र॑यी॒णाम् । \newline
43. र॒यी॒णामिति॑ रयी॒णाम् । \newline
44. तम् ने॒मिम् ने॒मिम् तम् तम् ने॒मिम् । \newline
45. ने॒मि मृ॒भव॑ ऋ॒भवो॑ ने॒मिम् ने॒मि मृ॒भवः॑ । \newline
46. ऋ॒भवो॑ यथा यथ॒ र्‌भव॑ ऋ॒भवो॑ यथा । \newline
47. य॒था ऽऽय॑था य॒था । \newline
48. आ न॑मस्व नम॒स्वा न॑मस्व । \newline
49. न॒म॒स्व॒ सहू॑तिभिः॒ सहू॑तिभिर् नमस्व नमस्व॒ सहू॑तिभिः । \newline
50. सहू॑तिभि॒रिति॒ सहू॑ति - भिः॒ । \newline
51. नेदी॑यो य॒ज्ञ्ं ॅय॒ज्ञ्म् नेदी॑यो॒ नेदी॑यो य॒ज्ञ्म् । \newline
52. य॒ज्ञ् म॑ङ्गिरो अङ्गिरो य॒ज्ञ्ं ॅय॒ज्ञ् म॑ङ्गिरः । \newline

\textbf{Ghana Paata } \newline

1. यु॒क्ष्वा हि हि यु॒क्ष्व यु॒क्ष्वा हि दे॑व॒हूत॑मान् देव॒हूत॑मा॒न्॒. हि यु॒क्ष्व यु॒क्ष्वा हि दे॑व॒हूत॑मान् । \newline
2. हि दे॑व॒हूत॑मान् देव॒हूत॑मा॒न्॒. हि हि दे॑व॒हूत॑माꣳ॒॒ अश्वाꣳ॒॒ अश्वा᳚न् देव॒हूत॑मा॒न्॒. हि हि दे॑व॒हूत॑माꣳ॒॒ अश्वान्॑ । \newline
3. दे॒व॒हूत॑माꣳ॒॒ अश्वाꣳ॒॒ अश्वा᳚न् देव॒हूत॑मान् देव॒हूत॑माꣳ॒॒ अश्वाꣳ॑ अग्ने अ॒ग्ने ऽश्वा᳚न् देव॒हूत॑मान् देव॒हूत॑माꣳ॒॒ अश्वाꣳ॑ अग्ने । \newline
4. दे॒व॒हूत॑मा॒निति॑ देव - हूत॑मान् । \newline
5. अश्वाꣳ॑ अग्ने अ॒ग्ने ऽश्वाꣳ॒॒ अश्वाꣳ॑ अग्ने र॒थी र॒थी र॒ग्ने ऽश्वाꣳ॒॒ अश्वाꣳ॑ अग्ने र॒थीः । \newline
6. अ॒ग्ने॒ र॒थी र॒थी र॑ग्ने अग्ने र॒थी रि॑वे व र॒थीर॑ग्ने अग्ने र॒थीरि॑व । \newline
7. र॒थीरि॑वे व र॒थी र॒थीरि॑व । \newline
8. इ॒वेती॑व । \newline
9. नि होता॒ होता॒ नि नि होता॑ पू॒र्व्यः पू॒र्व्यो होता॒ नि नि होता॑ पू॒र्व्यः । \newline
10. होता॑ पू॒र्व्यः पू॒र्व्यो होता॒ होता॑ पू॒र्व्यः स॑दः सदः पू॒र्व्यो होता॒ होता॑ पू॒र्व्यः स॑दः । \newline
11. पू॒र्व्यः स॑दः सदः पू॒र्व्यः पू॒र्व्यः स॑दः । \newline
12. स॒द॒ इति॑ सदः । \newline
13. उ॒त नो॑ न उ॒तोत नो॑ देव देव न उ॒तोत नो॑ देव । \newline
14. नो॒ दे॒व॒ दे॒व॒ नो॒ नो॒ दे॒व॒ दे॒वान् दे॒वान् दे॑व नो नो देव दे॒वान् । \newline
15. दे॒व॒ दे॒वान् दे॒वान् दे॑व देव दे॒वाꣳ अच्छाच्छ॑ दे॒वान् दे॑व देव दे॒वाꣳ अच्छ॑ । \newline
16. दे॒वाꣳ अच्छाच्छ॑ दे॒वान् दे॒वाꣳ अच्छा॑ वोचो वो॒चो अच्छ॑ दे॒वान् दे॒वाꣳ अच्छा॑ वोचः । \newline
17. अच्छा॑ वोचो वो॒चो अच्छाच्छा॑ वोचो वि॒दुष्ट॑रो वि॒दुष्ट॑रो वो॒चो अच्छाच्छा॑ वोचो वि॒दुष्ट॑रः । \newline
18. वो॒चो॒ वि॒दुष्ट॑रो वि॒दुष्ट॑रो वोचो वोचो वि॒दुष्ट॑रः । \newline
19. वि॒दुष्ट॑र॒ इति॑ वि॒दुः - त॒रः॒ । \newline
20. श्रद् विश्वा॒ विश्वा॒ श्रच्छ्रद् विश्वा॒ वार्या॒ वार्या॒ विश्वा॒ श्रच्छ्रद् विश्वा॒ वार्या᳚ । \newline
21. विश्वा॒ वार्या॒ वार्या॒ विश्वा॒ विश्वा॒ वार्या॑ कृधि कृधि॒ वार्या॒ विश्वा॒ विश्वा॒ वार्या॑ कृधि । \newline
22. वार्या॑ कृधि कृधि॒ वार्या॒ वार्या॑ कृधि । \newline
23. कृ॒धीति॑ कृधि । \newline
24. त्वꣳ ह॑ ह॒ त्वम् त्वꣳ ह॒ यद् यद्ध॒ त्वम् त्वꣳ ह॒ यत् । \newline
25. ह॒ यद् यद्ध॑ ह॒ यद् य॑विष्ठ्य यविष्ठ्य॒ यद्ध॑ ह॒ यद् य॑विष्ठ्य । \newline
26. यद् य॑विष्ठ्य यविष्ठ्य॒ यद् यद् य॑विष्ठ्य॒ सह॑सः॒ सह॑सो यविष्ठ्य॒ यद् यद् य॑विष्ठ्य॒ सह॑सः । \newline
27. य॒वि॒ष्ठ्य॒ सह॑सः॒ सह॑सो यविष्ठ्य यविष्ठ्य॒ सह॑सः सूनो सूनो॒ सह॑सो यविष्ठ्य यविष्ठ्य॒ सह॑सः सूनो । \newline
28. सह॑सः सूनो सूनो॒ सह॑सः॒ सह॑सः सून वाहुताहुत सूनो॒ सह॑सः॒ सह॑सः सूनवाहुत । \newline
29. सू॒न॒ वा॒हु॒ता॒हु॒त॒ सू॒नो॒ सू॒न॒वा॒हु॒त॒ । \newline
30. आ॒हु॒तेत्या᳚ - हु॒त॒ । \newline
31. ऋ॒तावा॑ य॒ज्ञियो॑ य॒ज्ञिय॑ ऋ॒ताव॒ र्‌तावा॑ य॒ज्ञियो॒ भुवो॒ भुवो॑ य॒ज्ञिय॑ ऋ॒ताव॒ र्‌तावा॑ य॒ज्ञियो॒ भुवः॑ । \newline
32. ऋ॒तावेत्यृ॒त - वा॒ । \newline
33. य॒ज्ञियो॒ भुवो॒ भुवो॑ य॒ज्ञियो॑ य॒ज्ञियो॒ भुवः॑ । \newline
34. भुव॒ इति॒ भुवः॑ । \newline
35. अ॒य म॒ग्नि र॒ग्नि र॒य म॒य म॒ग्निः स॑ह॒स्रिणः॑ सह॒स्रिणो॑ अ॒ग्नि र॒य म॒य म॒ग्निः स॑ह॒स्रिणः॑ । \newline
36. अ॒ग्निः स॑ह॒स्रिणः॑ सह॒स्रिणो॑ अ॒ग्नि र॒ग्निः स॑ह॒स्रिणो॒ वाज॑स्य॒ वाज॑स्य सह॒स्रिणो॑ अ॒ग्नि र॒ग्निः स॑ह॒स्रिणो॒ वाज॑स्य । \newline
37. स॒ह॒स्रिणो॒ वाज॑स्य॒ वाज॑स्य सह॒स्रिणः॑ सह॒स्रिणो॒ वाज॑स्य श॒तिनः॑ श॒तिनो॒ वाज॑स्य सह॒स्रिणः॑ सह॒स्रिणो॒ वाज॑स्य श॒तिनः॑ । \newline
38. वाज॑स्य श॒तिनः॑ श॒तिनो॒ वाज॑स्य॒ वाज॑स्य श॒तिन॒ स्पति॒ष्पतिः॑ श॒तिनो॒ वाज॑स्य॒ वाज॑स्य श॒तिन॒ स्पतिः॑ । \newline
39. श॒तिन॒ स्पति॒ष्पतिः॑ श॒तिनः॑ श॒तिन॒ स्पतिः॑ । \newline
40. पति॒रिति॒ पतिः॑ । \newline
41. मू॒र्द्धा क॒विः क॒विर् मू॒र्द्धा मू॒र्द्धा क॒वी र॑यी॒णाꣳ र॑यी॒णाम् क॒विर् मू॒र्द्धा मू॒र्द्धा क॒वी र॑यी॒णाम् । \newline
42. क॒वी र॑यी॒णाꣳ र॑यी॒णाम् क॒विः क॒वी र॑यी॒णाम् । \newline
43. र॒यी॒णामिति॑ रयी॒णाम् । \newline
44. तम् ने॒मिम् ने॒मिम् तम् तम् ने॒मि मृ॒भव॑ ऋ॒भवो॑ ने॒मिम् तम् तम् ने॒मि मृ॒भवः॑ । \newline
45. ने॒मि मृ॒भव॑ ऋ॒भवो॑ ने॒मिम् ने॒मि मृ॒भवो॑ यथा यथ॒र्भवो॑ ने॒मिम् ने॒मि मृ॒भवो॑ यथा । \newline
46. ऋ॒भवो॑ यथा यथ॒र्भव॑ ऋ॒भवो॑ य॒था ऽऽय॑थ॒र्भव॑ ऋ॒भवो॑ य॒था । \newline
47. य॒था ऽऽय॑था य॒था ऽऽन॑मस्व नम॒स्वा य॑था य॒था ऽऽन॑मस्व । \newline
48. आ न॑मस्व नम॒स्वा न॑मस्व॒ सहू॑तिभिः॒ सहू॑तिभिर् नम॒स्वा न॑मस्व॒ सहू॑तिभिः । \newline
49. न॒म॒स्व॒ सहू॑तिभिः॒ सहू॑तिभिर् नमस्व नमस्व॒ सहू॑तिभिः । \newline
50. सहू॑तिभि॒रिति॒ सहू॑ति - भिः॒ । \newline
51. नेदी॑यो य॒ज्ञ्ं ॅय॒ज्ञ्म् नेदी॑यो॒ नेदी॑यो य॒ज्ञ् म॑ङ्गिरो अङ्गिरो य॒ज्ञ्म् नेदी॑यो॒ नेदी॑यो य॒ज्ञ् म॑ङ्गिरः । \newline
52. य॒ज्ञ् म॑ङ्गिरो अङ्गिरो य॒ज्ञ्ं ॅय॒ज्ञ् म॑ङ्गिरः । \newline
\pagebreak
\markright{ TS 2.6.11.2  \hfill https://www.vedavms.in \hfill}
\addcontentsline{toc}{section}{ TS 2.6.11.2 }
\section*{ TS 2.6.11.2 }

\textbf{TS 2.6.11.2 } \newline
\textbf{Samhita Paata} \newline

-म॑ङ्गिरः ॥ तस्मै॑ नू॒ नम॒भिद्य॑वे वा॒चा वि॑रूप॒ नित्य॑या । वृष्णे॑ चोदस्व सुष्टु॒तिं ॥ कमु॑ ष्विदस्य॒ सेन॑या॒ऽग्नेरपा॑कचक्षसः । प॒णिं गोषु॑ स्तरामहे ॥मा नो॑ दे॒वानां॒ ॅविशः॑ प्रस्ना॒तीरि॑वो॒स्राः । कृ॒शं न हा॑सु॒रघ्नि॑याः ॥मा नः॑ समस्य दू॒ढ्यः॑ परि॑द्वेषसो अꣳ ह॒तिः । ऊ॒र्मिर्न नाव॒मा व॑धीत् ॥ नम॑स्ते अग्न॒ ओज॑से गृ॒णन्ति॑ देव कृ॒ष्टयः॑ । अमै॑ - [  ] \newline

\textbf{Pada Paata} \newline

अ॒ङ्गि॒रः॒ ॥ तस्मै᳚ । नू॒नम् । अ॒भिद्य॑व॒ इत्य॒भि - द्य॒वे॒ । वा॒चा । वि॒रू॒पेति॑ वि - रू॒प॒ । नित्य॑या ॥ वृष्णे᳚ । चो॒द॒स्व॒ । सु॒ष्टु॒तिमिति॑ सु - स्तु॒तिम् ॥ कम् । उ॒ । स्वि॒त् । अ॒स्य॒ । सेन॑या । अ॒ग्नेः । अपा॑कचक्षस॒ इत्यपा॑क-च॒क्ष॒सः॒ ॥ प॒णिम् । गोषु॑ । स्त॒रा॒म॒हे॒ ॥ मा । नः॒ । दे॒वाना᳚म् । विशः॑ । प्र॒स्ना॒तीरिति॑ प्र - स्ना॒तीः । इ॒व॒ । उ॒स्राः ॥ कृ॒शम् । न । हा॒सुः॒ । अघ्नि॑याः ॥ मा । नः॒ । स॒म॒स्य॒ । दू॒ढ्यः॑ । परि॑द्वेषस॒ इति॒ परि॑ - द्वे॒ष॒सः॒ । अꣳ॒॒ह॒तिः ॥ ऊ॒र्मिः । न । नाव᳚म् । एति॑ । व॒धी॒त् ॥ नमः॑ । ते॒ । अ॒ग्ने॒ । ओज॑से । गृ॒णन्ति॑ । दे॒व॒ । कृ॒ष्टयः॑ ॥ अमैः᳚ ।  \newline


\textbf{Krama Paata} \newline

अ॒ङ्गि॒र॒ इत्य॑ङ्गिरः ॥ तस्मै॑ नू॒नम् । नू॒नम॒भिद्य॑वे । अ॒भिद्य॑वे वा॒चा । अ॒भिद्य॑व॒ इत्य॒भि - द्य॒वे॒ । वा॒चा वि॑रूप । वि॒रू॒प॒ नित्य॑या । वि॒रू॒पेति॑ वि - रू॒प॒ । नित्य॒येति॒ नित्य॑या ॥ वृष्णे॑ चोदस्व । चो॒द॒स्व॒ सु॒ष्टु॒तिम् । सु॒ष्टु॒तिमिति॑ सु - स्तु॒तिम् ॥  कमु॑ । उ॒ स्वि॒त्॒ । स्वि॒द॒स्य॒ । अ॒स्य॒ सेन॑या । सेन॑या॒ऽग्नेः । अ॒ग्नेर् अपा॑कचक्षसः । अपा॑कचक्षस॒ इत्यपा॑क - च॒क्ष॒सः॒ ॥ प॒णिम् गोषु॑ । गोषु॑ स्तरामहे । स्त॒रा॒म॒ह॒ इति॑ स्तरामहे ॥ मा नः॑ । नो॒ दे॒वाना᳚म् । दे॒वानां॒ ॅविशः॑ । विशः॑ प्रस्ना॒तीः । प्र॒स्ना॒तीरि॑व । प्र॒स्ना॒तीरिति॑ प्र - स्ना॒तीः । इ॒वो॒स्राः । उ॒स्रा इत्यु॒स्राः ॥ कृ॒शम् न । न हा॑सुः । हा॒सु॒रघ्नि॑याः । अघ्नि॑या॒ इत्यघ्नि॑याः ॥ मा नः॑ । नः॒ स॒म॒स्य॒ । स॒म॒स्य॒ दू॒ढ्यः॑ । दू॒ढ्यः॑ परि॑द्वेषसः । परि॑द्वेषसो अꣳह॒तिः । परि॑द्वेषस॒ इति॒ परि॑ - द्वे॒ष॒सः॒ । अꣳ॒॒ह॒तिरित्यꣳ॑ह॒तिः ॥ ऊ॒र्मिर् न । न नाव᳚म् । नाव॒मा । आ व॑धीत् । व॒धी॒दिति॑ वधीत् ॥ नम॑स्ते । ते॒ अ॒ग्ने॒ । अ॒ग्न॒ ओज॑से । ओज॑से गृ॒णन्ति॑ । गृ॒णन्ति॑ देव । दे॒व॒ कृ॒ष्टयः॑ । कृ॒ष्टय॒ इति॑ कृ॒ष्टयः॑ ॥ अमै॑र॒मित्र᳚म् \newline

\textbf{Jatai Paata} \newline

1. अ॒ङ्‍गि॒र॒ इत्य॑ङ्‍गिरः । \newline
2. तस्मै॑ नू॒नम् नू॒नम् तस्मै॒ तस्मै॑ नू॒नम् । \newline
3. नू॒न म॒भिद्य॑वे अ॒भिद्य॑वे नू॒नम् नू॒न म॒भिद्य॑वे । \newline
4. अ॒भिद्य॑वे वा॒चा वा॒चा ऽभिद्य॑वे अ॒भिद्य॑वे वा॒चा । \newline
5. अ॒भिद्य॑व॒ इत्य॒भि - द्य॒वे॒ । \newline
6. वा॒चा वि॑रूप विरूप वा॒चा वा॒चा वि॑रूप । \newline
7. वि॒रू॒प॒ नित्य॑या॒ नित्य॑या विरूप विरूप॒ नित्य॑या । \newline
8. वि॒रू॒पेति॑ वि - रू॒प॒ । \newline
9. नित्य॒येति॒ नित्य॑या । \newline
10. वृष्णे॑ चोदस्व चोदस्व॒ वृष्णे॒ वृष्णे॑ चोदस्व । \newline
11. चो॒द॒स्व॒ सु॒ष्टु॒तिꣳ सु॑ष्टु॒तिम् चो॑दस्व चोदस्व सुष्टु॒तिम् । \newline
12. सु॒ष्टु॒तिमिति॑ सु - स्तु॒तिम् । \newline
13. क मु॑ वु॒ कम् क मु॑ । \newline
14. उ॒ ष्वि॒थ् स्वि॒दु॒ वु॒ ष्वि॒त् । \newline
15. स्वि॒द॒स्या॒स्य॒ स्वि॒थ् स्वि॒द॒स्य॒ । \newline
16. अ॒स्य॒ सेन॑या॒ सेन॑या ऽस्यास्य॒ सेन॑या । \newline
17. सेन॑या॒ ऽग्ने र॒ग्नेः सेन॑या॒ सेन॑या॒ ऽग्नेः । \newline
18. अ॒ग्ने रपा॑कचक्षसो॒ अपा॑कचक्षसो अ॒ग्ने र॒ग्ने रपा॑कचक्षसः । \newline
19. अपा॑कचक्षस॒ इत्यपा॑क - च॒क्ष॒सः॒ । \newline
20. प॒णिम् गोषु॒ गोषु॑ प॒णिम् प॒णिम् गोषु॑ । \newline
21. गोषु॑ स्तरामहे स्तरामहे॒ गोषु॒ गोषु॑ स्तरामहे । \newline
22. स्त॒रा॒म॒ह॒ इति॑ स्तरामहे । \newline
23. मा नो॑ नो॒ मा मा नः॑ । \newline
24. नो॒ दे॒वाना᳚म् दे॒वाना᳚म् नो नो दे॒वाना᳚म् । \newline
25. दे॒वानां॒ ॅविशो॒ विशो॑ दे॒वाना᳚म् दे॒वानां॒ ॅविशः॑ । \newline
26. विशः॑ प्रस्ना॒तीः प्र॑स्ना॒तीर् विशो॒ विशः॑ प्रस्ना॒तीः । \newline
27. प्र॒स्ना॒ती रि॑वे व प्रस्ना॒तीः प्र॑स्ना॒ती रि॑व । \newline
28. प्र॒स्ना॒तीरिति॑ प्र - स्ना॒तीः । \newline
29. इ॒वो॒स्रा उ॒स्रा इ॑वे वो॒स्राः । \newline
30. उ॒स्रा इत्यु॒स्राः । \newline
31. कृ॒शम् न न कृ॒शम् कृ॒शम् न । \newline
32. न हा॑सुर्. हासु॒र् न न हा॑सुः । \newline
33. हा॒सु॒ रघ्नि॑या॒ अघ्नि॑या हासुर्. हासु॒ रघ्नि॑याः । \newline
34. अघ्नि॑या॒ इत्यघ्नि॑याः । \newline
35. मा नो॑ नो॒ मा मा नः॑ । \newline
36. नः॒ स॒म॒स्य॒ स॒म॒स्य॒ नो॒ नः॒ स॒म॒स्य॒ । \newline
37. स॒म॒स्य॒ दू॒ढ्यो॑ दू॒ढ्यः॑ समस्य समस्य दू॒ढ्यः॑ । \newline
38. दू॒ढ्यः॑ परि॑द्वेषसः॒ परि॑द्वेषसो दू॒ढ्यो॑ दू॒ढ्यः॑ परि॑द्वेषसः । \newline
39. परि॑द्वेषसो अꣳह॒ति रꣳ॑ह॒तिः परि॑द्वेषसः॒ परि॑द्वेषसो अꣳह॒तिः । \newline
40. परि॑द्वेषस॒ इति॒ परि॑ - द्वे॒ष॒सः॒ । \newline
41. अꣳ॒॒ह॒तिरित्यꣳ॑ह॒तिः । \newline
42. ऊ॒र्मिर् न नोर्मि रू॒र्मिर् न । \newline
43. न नाव॒म् नाव॒म् न न नाव᳚म् । \newline
44. नाव॒ मा नाव॒म् नाव॒ मा । \newline
45. आ व॑धीद् वधी॒दा व॑धीत् । \newline
46. व॒धी॒दिति॑ वधीत् । \newline
47. नम॑ स्ते ते॒ नमो॒ नम॑ स्ते । \newline
48. ते॒ अ॒ग्ने॒ अ॒ग्ने॒ ते॒ ते॒ अ॒ग्ने॒ । \newline
49. अ॒ग्न॒ ओज॑स॒ ओज॑से अग्ने अग्न॒ ओज॑से । \newline
50. ओज॑से गृ॒णन्ति॑ गृ॒ण न्त्योज॑स॒ ओज॑से गृ॒णन्ति॑ । \newline
51. गृ॒णन्ति॑ देव देव गृ॒णन्ति॑ गृ॒णन्ति॑ देव । \newline
52. दे॒व॒ कृ॒ष्टयः॑ कृ॒ष्टयो॑ देव देव कृ॒ष्टयः॑ । \newline
53. कृ॒ष्टय॒ इति॑ कृ॒ष्टयः॑ । \newline
54. अमै॑ र॒मित्र॑ म॒मित्र॒ ममै॒ रमै॑ र॒मित्र᳚म् । \newline

\textbf{Ghana Paata } \newline

1. अ॒ङ्‍गि॒र॒ इत्य॑ङ्‍गिरः । \newline
2. तस्मै॑ नू॒नम् नू॒नम् तस्मै॒ तस्मै॑ नू॒न म॒भिद्य॑वे अ॒भिद्य॑वे नू॒नम् तस्मै॒ तस्मै॑ नू॒न म॒भिद्य॑वे । \newline
3. नू॒न म॒भिद्य॑वे अ॒भिद्य॑वे नू॒नम् नू॒न म॒भिद्य॑वे वा॒चा वा॒चा ऽभिद्य॑वे नू॒नम् नू॒न म॒भिद्य॑वे वा॒चा । \newline
4. अ॒भिद्य॑वे वा॒चा वा॒चा ऽभिद्य॑वे अ॒भिद्य॑वे वा॒चा वि॑रूप विरूप वा॒चा ऽभिद्य॑वे अ॒भिद्य॑वे वा॒चा वि॑रूप । \newline
5. अ॒भिद्य॑व॒ इत्य॒भि - द्य॒वे॒ । \newline
6. वा॒चा वि॑रूप विरूप वा॒चा वा॒चा वि॑रूप॒ नित्य॑या॒ नित्य॑या विरूप वा॒चा वा॒चा वि॑रूप॒ नित्य॑या । \newline
7. वि॒रू॒प॒ नित्य॑या॒ नित्य॑या विरूप विरूप॒ नित्य॑या । \newline
8. वि॒रू॒पेति॑ वि - रू॒प॒ । \newline
9. नित्य॒येति॒ नित्य॑या । \newline
10. वृष्णे॑ चोदस्व चोदस्व॒ वृष्णे॒ वृष्णे॑ चोदस्व सुष्टु॒तिꣳ सु॑ष्टु॒तिम् चो॑दस्व॒ वृष्णे॒ वृष्णे॑ चोदस्व सुष्टु॒तिम् । \newline
11. चो॒द॒स्व॒ सु॒ष्टु॒तिꣳ सु॑ष्टु॒तिम् चो॑दस्व चोदस्व सुष्टु॒तिम् । \newline
12. सु॒ष्टु॒तिमिति॑ सु - स्तु॒तिम् । \newline
13. क मु॑ वु॒ कम् क मु॑ ष्विथ् स्विदु॒ कम् क मु॑ ष्वित् । \newline
14. उ॒ ष्वि॒थ् स्वि॒दु॒ वु॒ ष्वि॒द॒स्या॒स्य॒ स्वि॒दु॒ वु॒ ष्वि॒द॒स्य॒ । \newline
15. स्वि॒द॒स्या॒स्य॒ स्वि॒थ् स्वि॒द॒स्य॒ सेन॑या॒ सेन॑या ऽस्य स्विथ् स्विदस्य॒ सेन॑या । \newline
16. अ॒स्य॒ सेन॑या॒ सेन॑या ऽस्यास्य॒ सेन॑या॒ ऽग्ने र॒ग्नेः सेन॑या ऽस्यास्य॒ सेन॑या॒ ऽग्नेः । \newline
17. सेन॑या॒ ऽग्ने र॒ग्नेः सेन॑या॒ सेन॑या॒ ऽग्ने रपा॑कचक्षसो॒ अपा॑कचक्षसो अ॒ग्नेः सेन॑या॒ सेन॑या॒ ऽग्नेरपा॑कचक्षसः । \newline
18. अ॒ग्ने रपा॑कचक्षसो॒ अपा॑कचक्षसो अ॒ग्ने र॒ग्ने रपा॑कचक्षसः । \newline
19. अपा॑कचक्षस॒ इत्यपा॑क - च॒क्ष॒सः॒ । \newline
20. प॒णिम् गोषु॒ गोषु॑ प॒णिम् प॒णिम् गोषु॑ स्तरामहे स्तरामहे॒ गोषु॑ प॒णिम् प॒णिम् गोषु॑ स्तरामहे । \newline
21. गोषु॑ स्तरामहे स्तरामहे॒ गोषु॒ गोषु॑ स्तरामहे । \newline
22. स्त॒रा॒म॒ह॒ इति॑ स्तरामहे । \newline
23. मा नो॑ नो॒ मा मा नो॑ दे॒वाना᳚म् दे॒वाना᳚म् नो॒ मा मा नो॑ दे॒वाना᳚म् । \newline
24. नो॒ दे॒वाना᳚म् दे॒वाना᳚म् नो नो दे॒वानां॒ ॅविशो॒ विशो॑ दे॒वाना᳚म् नो नो दे॒वानां॒ ॅविशः॑ । \newline
25. दे॒वानां॒ ॅविशो॒ विशो॑ दे॒वाना᳚म् दे॒वानां॒ ॅविशः॑ प्रस्ना॒तीः प्र॑स्ना॒तीर् विशो॑ दे॒वाना᳚म् दे॒वानां॒ ॅविशः॑ प्रस्ना॒तीः । \newline
26. विशः॑ प्रस्ना॒तीः प्र॑स्ना॒तीर् विशो॒ विशः॑ प्रस्ना॒ती रि॑वे व प्रस्ना॒तीर् विशो॒ विशः॑ प्रस्ना॒ती रि॑व । \newline
27. प्र॒स्ना॒ती रि॑वे व प्रस्ना॒तीः प्र॑स्ना॒ती रि॑वो॒स्रा उ॒स्रा इ॑व प्रस्ना॒तीः प्र॑स्ना॒ती रि॑वो॒स्राः । \newline
28. प्र॒स्ना॒तीरिति॑ प्र - स्ना॒तीः । \newline
29. इ॒वो॒स्रा उ॒स्रा इ॑वे वो॒स्राः । \newline
30. उ॒स्रा इत्यु॒स्राः । \newline
31. कृ॒शम् न न कृ॒शम् कृ॒शम् न हा॑सुर्. हासु॒र् न कृ॒शम् कृ॒शम् न हा॑सुः । \newline
32. न हा॑सुर्. हासु॒र् न न हा॑सु॒ रघ्नि॑या॒ अघ्नि॑या हासु॒र् न न हा॑सु॒ रघ्नि॑याः । \newline
33. हा॒सु॒ रघ्नि॑या॒ अघ्नि॑या हासुर्. हासु॒ रघ्नि॑याः । \newline
34. अघ्नि॑या॒ इत्यघ्नि॑याः । \newline
35. मा नो॑ नो॒ मा मा नः॑ समस्य समस्य नो॒ मा मा नः॑ समस्य । \newline
36. नः॒ स॒म॒स्य॒ स॒म॒स्य॒ नो॒ नः॒ स॒म॒स्य॒ दू॒ढ्यो॑ दू॒ढ्यः॑ समस्य नो नः समस्य दू॒ढ्यः॑ । \newline
37. स॒म॒स्य॒ दू॒ढ्यो॑ दू॒ढ्यः॑ समस्य समस्य दू॒ढ्यः॑ परि॑द्वेषसः॒ परि॑द्वेषसो दू॒ढ्यः॑ समस्य समस्य दू॒ढ्यः॑ परि॑द्वेषसः । \newline
38. दू॒ढ्यः॑ परि॑द्वेषसः॒ परि॑द्वेषसो दू॒ढ्यो॑ दू॒ढ्यः॑ परि॑द्वेषसो अꣳह॒ति रꣳ॑ह॒तिः परि॑द्वेषसो दू॒ढ्यो॑ दू॒ढ्यः॑ परि॑द्वेषसो अꣳह॒तिः । \newline
39. परि॑द्वेषसो अꣳह॒ति रꣳ॑ह॒तिः परि॑द्वेषसः॒ परि॑द्वेषसो अꣳह॒तिः । \newline
40. परि॑द्वेषस॒ इति॒ परि॑ - द्वे॒ष॒सः॒ । \newline
41. अꣳ॒॒ह॒तिरित्यꣳ॑ह॒तिः । \newline
42. ऊ॒र्मिर् न नोर्म् इरू॒र्मिर् न नाव॒म् नाव॒म् नोर्मि रू॒र्मिर् न नाव᳚म् । \newline
43. न नाव॒म् नाव॒म् न न नाव॒ मा नाव॒म् न न नाव॒ मा । \newline
44. नाव॒ मा नाव॒म् नाव॒ मा व॑धीद् वधी॒दा नाव॒म् नाव॒ मा व॑धीत् । \newline
45. आ व॑धीद् वधी॒दा व॑धीत् । \newline
46. व॒धी॒दिति॑ वधीत् । \newline
47. नम॑स्ते ते॒ नमो॒ नम॑स्ते अग्ने अग्ने ते॒ नमो॒ नम॑स्ते अग्ने । \newline
48. ते॒ अ॒ग्ने॒ अ॒ग्ने॒ ते॒ ते॒ अ॒ग्न॒ ओज॑स॒ ओज॑से अग्ने ते ते अग्न॒ ओज॑से । \newline
49. अ॒ग्न॒ ओज॑स॒ ओज॑से अग्ने अग्न॒ ओज॑से गृ॒णन्ति॑ गृ॒ण न्त्योज॑से अग्ने अग्न॒ ओज॑से गृ॒णन्ति॑ । \newline
50. ओज॑से गृ॒णन्ति॑ गृ॒ण न्त्योज॑स॒ ओज॑से गृ॒णन्ति॑ देव देव गृ॒ण न्त्योज॑स॒ ओज॑से गृ॒णन्ति॑ देव । \newline
51. गृ॒णन्ति॑ देव देव गृ॒णन्ति॑ गृ॒णन्ति॑ देव कृ॒ष्टयः॑ कृ॒ष्टयो॑ देव गृ॒णन्ति॑ गृ॒णन्ति॑ देव कृ॒ष्टयः॑ । \newline
52. दे॒व॒ कृ॒ष्टयः॑ कृ॒ष्टयो॑ देव देव कृ॒ष्टयः॑ । \newline
53. कृ॒ष्टय॒ इति॑ कृ॒ष्टयः॑ । \newline
54. अमै॑ र॒मित्र॑ म॒मित्र॒ ममै॒ रमै॑ र॒मित्र॑ मर्दयार् दया॒मित्र॒ ममै॒ रमै॑ र॒मित्र॑ मर्दय । \newline
\pagebreak
\markright{ TS 2.6.11.3  \hfill https://www.vedavms.in \hfill}
\addcontentsline{toc}{section}{ TS 2.6.11.3 }
\section*{ TS 2.6.11.3 }

\textbf{TS 2.6.11.3 } \newline
\textbf{Samhita Paata} \newline

-र॒मित्र॑मर्दय ॥ कु॒विथ्‌सुनो॒ गवि॑ष्ट॒येऽग्ने॑ सं॒ॅवेषि॑षो र॒यिं । उरु॑कृदु॒रु ण॑स्कृधि ॥ मा नो॑ अ॒स्मिन् म॑हाध॒ने परा॑ वर्ग्भार॒भृद्य॑था । सं॒ॅवर्गꣳ॒॒ सꣳ र॒यिं ज॑य ॥अ॒न्यम॒स्मद्भि॒या इ॒यमग्ने॒ सिष॑क्तु दु॒च्छुना᳚ । वर्द्धा॑ नो॒ अम॑व॒च्छवः॑ ॥ यस्याजु॑षन्नम॒स्विनः॒ शमी॒मदु॑र्मखस्यवा । तं घेद॒ग्निर्वृ॒धाऽव॑ति ॥ पर॑स्या॒ अधि॑ - [  ] \newline

\textbf{Pada Paata} \newline

अ॒मित्र᳚म् । अ॒र्द॒य॒ ॥ कु॒वित् । स्विति॑ । नः॒ । गवि॑ष्टय॒ इति॒ गो-इ॒ष्ट॒ये॒ । अग्ने᳚ । सं॒ॅवेषि॑ष॒ इति॑ सं - वेषि॑षः । र॒यिम् ॥ उरु॑कृ॒दित्युरु॑-कृ॒त् । उ॒रु । नः॒ ॥ कृ॒धि॒ । मा । नः॒ । अ॒स्मिन्न् । म॒हा॒ध॒न इति॑ महा-ध॒ने । परेति॑ । व॒र्क् । भा॒र॒भृदिति॑ भार - भृत् । य॒था॒ ॥ सं॒ॅवर्ग॒मिति॑ सं - वर्ग᳚म् । समिति॑ । र॒यिम् । ज॒य॒ ॥ अ॒न्यम् । अ॒स्मत् । भि॒यै । इ॒यम् । अग्ने᳚ । सिष॑क्तु । दु॒च्छुना᳚ ॥ वर्द्ध॑ । नः॒ । अम॑व॒दित्यम॑-व॒त् । शवः॑ ॥ यस्य॑ । अजु॑षत् । न॒म॒स्विनः॑ । शमी᳚म् । अदु॑र्मख॒स्येत्यदुः॑-म॒ख॒स्य॒ । वा॒ ॥ तम् । घ॒ । इत् । अ॒ग्निः । वृ॒धा । अ॒व॒ति॒ ॥ पर॑स्याः । अधीति॑ ।  \newline


\textbf{Krama Paata} \newline

अ॒मित्र॑मर्दय । अ॒र्द॒येत्य॑र्दय ॥ कु॒विथ् सु । सु नः॑ । नो॒ गवि॑ष्टये । गवि॑ष्ट॒येऽग्ने᳚ । गवि॑ष्टय॒ इति॒ गो - इ॒ष्ट॒ये॒ । अग्ने॑ स॒म्ॅवेषि॑षः । स॒म्ॅवेषि॑षो र॒यिम् । स॒म्ॅवेषि॑ष॒ इति॑ सं - वेषि॑षः । र॒यिमिति॑ र॒यिम् ॥ उरु॑कृदु॒रु । उरु॑कृ॒दित्युरु॑ - कृ॒त्॒ । उ॒रु णः॑ । न॒स्कृ॒धि॒ । कृ॒धीति॑ कृधि ॥ मा नः॑ । नो॒ अ॒स्मिन्न् । अ॒स्मिन् म॑हाध॒ने । म॒हा॒ध॒ने परा᳚ । म॒हा॒ध॒न इति॑ महा - ध॒ने । परा॑ वर्क् । व॒र्ग् भा॒र॒भृत् । भा॒र॒भृद् य॑था । भा॒र॒भृदिति॑ भार - भृत् । य॒थेति॑ यथा ॥ स॒म्ॅवर्गꣳ॒॒ सम् । स॒म्ॅवर्ग॒मिति॑ सं - वर्ग᳚म् । सꣳ र॒यिम् । र॒यिम् ज॑य । ज॒येति॑ जय ॥ अ॒न्यम॒स्मत् । अ॒स्मद् भि॒यै । भि॒या इ॒यम् । इ॒यमग्ने᳚ । अग्ने॒ सिष॑क्तु । सिष॑क्तु दु॒च्छुना᳚ । दु॒च्छुनेति॑ दु॒च्छुना᳚ ॥ वर्द्धा॑ नः । नो॒ अम॑वत् । अम॑व॒च्छवः॑ । अम॑व॒दित्यम॑ - व॒त्॒ । शव॒ इति॒ शवः॑ ॥ यस्याजु॑षत् । अजु॑षन् नम॒स्विनः॑ । न॒म॒स्विनः॒ शमी᳚म् । शमी॒मदु॑र्मखस्य । अदु॑र्मखस्य वा । अदु॑र्मख॒स्येत्यदुः॑ - म॒ख॒स्य॒ । वेति॑ वा ॥ तम् घ॑ । घेत् । इद॒ग्निः । अ॒ग्निर् वृ॒धा । वृ॒धा ऽव॑ति । अ॒व॒तीत्य॑वति ॥ पर॑स्या॒ अधि॑ । अधि॑ स॒म्ॅवतः॑ \newline

\textbf{Jatai Paata} \newline

1. अ॒मित्र॑ मर्दया र्दया॒ मित्र॑ म॒मित्र॑ मर्दय । \newline
2. अ॒र्द॒येत्य॑र्दय । \newline
3. कु॒विथ् सु सु कु॒वित् कु॒विथ् सु । \newline
4. सु नो॑ नः॒ सु सु नः॑ । \newline
5. नो॒ गवि॑ष्टये॒ गवि॑ष्टये नो नो॒ गवि॑ष्टये । \newline
6. गवि॑ष्ट॒ये ऽग्ने ऽग्ने॒ गवि॑ष्टये॒ गवि॑ष्ट॒ये ऽग्ने᳚ । \newline
7. गवि॑ष्टय॒ इति॒ गो - इ॒ष्ट॒ये॒ । \newline
8. अग्ने॑ सं॒ॅवेषि॑षः सं॒ॅवेषि॒षो ऽग्ने ऽग्ने॑ सं॒ॅवेषि॑षः । \newline
9. सं॒ॅवेषि॑षो र॒यिꣳ र॒यिꣳ सं॒ॅवेषि॑षः सं॒ॅवेषि॑षो र॒यिम् । \newline
10. सं॒ॅवेषि॑ष॒ इति॑ सं - वेषि॑षः । \newline
11. र॒यिमिति॑ र॒यिम् । \newline
12. उरु॑कृ दु॒रू॑रू रु॑कृ॒ दुरु॑कृ दु॒रु । \newline
13. उरु॑कृ॒दित्युरु॑ - कृ॒त् । \newline
14. उ॒रु णो॑ न उ॒रू॑रु णः॑ । \newline
15. न॒ स्कृ॒धि॒ कृ॒धि॒ नो॒ न॒ स्कृ॒धि॒ । \newline
16. कृ॒धीति॑ कृधि । \newline
17. मा नो॑ नो॒ मा मा नः॑ । \newline
18. नो॒ अ॒स्मिन् न॒स्मिन् नो॑ नो अ॒स्मिन्न् । \newline
19. अ॒स्मिन् म॑हाध॒ने म॑हाध॒ने अ॒स्मिन् न॒स्मिन् म॑हाध॒ने । \newline
20. म॒हा॒ध॒ने परा॒ परा॑ महाध॒ने म॑हाध॒ने परा᳚ । \newline
21. म॒हा॒ध॒न इति॑ महा - ध॒ने । \newline
22. परा॑ वर्ग् व॒र्क् परा॒ परा॑ वर्क् । \newline
23. व॒र्ग् भा॒र॒भृद् भा॑र॒भृद् व॑र्ग् वर्ग् भार॒भृत् । \newline
24. भा॒र॒भृद् य॑था यथा भार॒भृद् भा॑र॒भृद् य॑था । \newline
25. भा॒र॒भृदिति॑ भार - भृत् । \newline
26. य॒थेति॑ यथा । \newline
27. सं॒ॅवर्गꣳ॒॒ सꣳ सꣳ सं॒ॅवर्गꣳ॑ सं॒ॅवर्गꣳ॒॒ सम् । \newline
28. सं॒ॅवर्ग॒मिति॑ सं - वर्ग᳚म् । \newline
29. सꣳ र॒यिꣳ र॒यिꣳ सꣳ सꣳ र॒यिम् । \newline
30. र॒यिम् ज॑य जय र॒यिꣳ र॒यिम् ज॑य । \newline
31. ज॒येति॑ जय । \newline
32. अ॒न्य म॒स्म द॒स्म द॒न्य म॒न्य म॒स्मत् । \newline
33. अ॒स्मद् भि॒यै भि॒या अ॒स्म द॒स्मद् भि॒यै । \newline
34. भि॒या इ॒य मि॒यम् भि॒यै भि॒या इ॒यम् । \newline
35. इ॒य मग्ने ऽग्न॑ इ॒य मि॒य मग्ने᳚ । \newline
36. अग्ने॒ सिष॑क्तु॒ सिष॒क्त्वग्ने ऽग्ने॒ सिष॑क्तु । \newline
37. सिष॑क्तु दु॒च्छुना॑ दु॒च्छुना॒ सिष॑क्तु॒ सिष॑क्तु दु॒च्छुना᳚ । \newline
38. दु॒च्छुनेति॑ दु॒च्छुना᳚ । \newline
39. वर्द्धा॑ नो नो॒ वर्द्ध॒ वर्द्धा॑ नः । \newline
40. नो॒ अम॑व॒ दम॑वन् नो नो॒ अम॑वत् । \newline
41. अम॑व॒च् छवः॒ शवो॒ अम॑व॒ दम॑व॒च् छवः॑ । \newline
42. अम॑व॒दित्यम॑ - व॒त् । \newline
43. शव॒ इति॒ शवः॑ । \newline
44. यस्या जु॑ष॒ दजु॑ष॒द् यस्य॒ यस्या जु॑षत् । \newline
45. अजु॑षन् नम॒स्विनो॑ नम॒स्विनो ऽजु॑ष॒ दजु॑षन् नम॒स्विनः॑ । \newline
46. न॒म॒स्विनः॒ शमीꣳ॒॒ शमी᳚म् नम॒स्विनो॑ नम॒स्विनः॒ शमी᳚म् । \newline
47. शमी॒ मदु॑र्मख॒स्या दु॑र्मखस्य॒ शमीꣳ॒॒ शमी॒ मदु॑र्मखस्य । \newline
48. अदु॑र्मखस्य वा॒ वा ऽदु॑र्मख॒स्या दु॑र्मखस्य वा । \newline
49. अदु॑र्मख॒स्येत्यदुः॑ - म॒ख॒स्य॒ । \newline
50. वेति॑ वा । \newline
51. तम् घ॑ घ॒ तम् तम् घ॑ । \newline
52. घे दिद् घ॒ घे त् । \newline
53. इद॒ग्नि र॒ग्नि रिदि द॒ग्निः । \newline
54. अ॒ग्निर् वृ॒धा वृ॒धा ऽग्नि र॒ग्निर् वृ॒धा । \newline
55. वृ॒धा ऽव॑ त्यवति वृ॒धा वृ॒धा ऽव॑ति । \newline
56. अ॒व॒तीत्य॑वति । \newline
57. पर॑स्या॒ अध्यधि॒ पर॑स्याः॒ पर॑स्या॒ अधि॑ । \newline
58. अधि॑ सं॒ॅवतः॑ सं॒ॅवतो॒ अध्यधि॑ सं॒ॅवतः॑ । \newline

\textbf{Ghana Paata } \newline

1. अ॒मित्र॑ मर्दया र्दया॒मित्र॑ म॒मित्र॑ मर्दय । \newline
2. अ॒र्द॒येत्य॑र्दय । \newline
3. कु॒विथ् सु सु कु॒वित् कु॒विथ् सु नो॑ नः॒ सु कु॒वित् कु॒विथ् सु नः॑ । \newline
4. सु नो॑ नः॒ सु सु नो॒ गवि॑ष्टये॒ गवि॑ष्टये नः॒ सु सु नो॒ गवि॑ष्टये । \newline
5. नो॒ गवि॑ष्टये॒ गवि॑ष्टये नो नो॒ गवि॑ष्ट॒ये ऽग्ने ऽग्ने॒ गवि॑ष्टये नो नो॒ गवि॑ष्ट॒ये ऽग्ने᳚ । \newline
6. गवि॑ष्ट॒ये ऽग्ने ऽग्ने॒ गवि॑ष्टये॒ गवि॑ष्ट॒ये ऽग्ने॑ सं॒ॅवेषि॑षः सं॒ॅवेषि॒षो ऽग्ने॒ गवि॑ष्टये॒ गवि॑ष्ट॒ये ऽग्ने॑ सं॒ॅवेषि॑षः । \newline
7. गवि॑ष्टय॒ इति॒ गो - इ॒ष्ट॒ये॒ । \newline
8. अग्ने॑ सं॒ॅवेषि॑षः सं॒ॅवेषि॒षो ऽग्ने ऽग्ने॑ सं॒ॅवेषि॑षो र॒यिꣳ र॒यिꣳ सं॒ॅवेषि॒षो ऽग्ने ऽग्ने॑ सं॒ॅवेषि॑षो र॒यिम् । \newline
9. सं॒ॅवेषि॑षो र॒यिꣳ र॒यिꣳ सं॒ॅवेषि॑षः सं॒ॅवेषि॑षो र॒यिम् । \newline
10. सं॒ॅवेषि॑ष॒ इति॑ सं - वेषि॑षः । \newline
11. र॒यिमिति॑ र॒यिम् । \newline
12. उरु॑ कृदु॒रू॑ रूरु॑ कृ॒दुरु॑ कृदु॒रु णो॑ न उ॒रूरु॑ कृ॒दुरु॑ कृदु॒रु णः॑ । \newline
13. उरु॑कृ॒दित्युरु॑ - कृ॒त् । \newline
14. उ॒रु णो॑ न उ॒रू॑रु ण॑ स्कृधि कृधि न उ॒रू॑रु ण॑ स्कृधि । \newline
15. न॒ स्कृ॒धि॒ कृ॒धि॒ नो॒ न॒ स्कृ॒धि॒ । \newline
16. कृ॒धीति॑ कृधि । \newline
17. मा नो॑ नो॒ मा मा नो॑ अ॒स्मिन् न॒स्मिन् नो॒ मा मा नो॑ अ॒स्मिन्न् । \newline
18. नो॒ अ॒स्मिन् न॒स्मिन् नो॑ नो अ॒स्मिन् म॑हाध॒ने म॑हाध॒ने अ॒स्मिन् नो॑ नो अ॒स्मिन् म॑हाध॒ने । \newline
19. अ॒स्मिन् म॑हाध॒ने म॑हाध॒ने अ॒स्मिन् न॒स्मिन् म॑हाध॒ने परा॒ परा॑ महाध॒ने अ॒स्मिन् न॒स्मिन् म॑हाध॒ने परा᳚ । \newline
20. म॒हा॒ध॒ने परा॒ परा॑ महाध॒ने म॑हाध॒ने परा॑ वर्ग् व॒र्क् परा॑ महाध॒ने म॑हाध॒ने परा॑ वर्क् । \newline
21. म॒हा॒ध॒न इति॑ महा - ध॒ने । \newline
22. परा॑ वर्ग् व॒र्क् परा॒ परा॑ वर्ग् भार॒भृद् भा॑र॒भृद् व॒र्क् परा॒ परा॑ वर्ग् भार॒भृत् । \newline
23. व॒र्ग् भा॒र॒भृद् भा॑र॒भृद् व॑र्ग् वर्ग् भार॒भृद् य॑था यथा भार॒भृद् व॑र्ग् वर्ग् भार॒भृद् य॑था । \newline
24. भा॒र॒भृद् य॑था यथा भार॒भृद् भा॑र॒भृद् य॑था । \newline
25. भा॒र॒भृदिति॑ भार - भृत् । \newline
26. य॒थेति॑ यथा । \newline
27. सं॒ॅवर्गꣳ॒॒ सꣳ सꣳ सं॒ॅवर्गꣳ॑ सं॒ॅवर्गꣳ॒॒ सꣳ र॒यिꣳ र॒यिꣳ सꣳ सं॒ॅवर्गꣳ॑ सं॒ॅवर्गꣳ॒॒ सꣳ र॒यिम् । \newline
28. सं॒ॅवर्ग॒मिति॑ सं - वर्ग᳚म् । \newline
29. सꣳ र॒यिꣳ र॒यिꣳ सꣳ सꣳ र॒यिम् ज॑य जय र॒यिꣳ सꣳ सꣳ र॒यिम् ज॑य । \newline
30. र॒यिम् ज॑य जय र॒यिꣳ र॒यिम् ज॑य । \newline
31. ज॒येति॑ जय । \newline
32. अ॒न्य म॒स्म द॒स्म द॒न्य म॒न्य म॒स्मद् भि॒यै भि॒या अ॒स्म द॒न्य म॒न्य म॒स्मद् भि॒यै । \newline
33. अ॒स्मद् भि॒यै भि॒या अ॒स्म द॒स्मद् भि॒या इ॒य मि॒यम् भि॒या अ॒स्म द॒स्मद् भि॒या इ॒यम् । \newline
34. भि॒या इ॒य मि॒यम् भि॒यै भि॒या इ॒य मग्ने ऽग्न॑ इ॒यम् भि॒यै भि॒या इ॒य मग्ने᳚ । \newline
35. इ॒य मग्ने ऽग्न॑ इ॒य मि॒य मग्ने॒ सिष॑क्तु॒ सिष॒क्त्वग्न॑ इ॒य मि॒य मग्ने॒ सिष॑क्तु । \newline
36. अग्ने॒ सिष॑क्तु॒ सिष॒क्त्वग्ने ऽग्ने॒ सिष॑क्तु दु॒च्छुना॑ दु॒च्छुना॒ सिष॒क्त्वग्ने ऽग्ने॒ सिष॑क्तु दु॒च्छुना᳚ । \newline
37. सिष॑क्तु दु॒च्छुना॑ दु॒च्छुना॒ सिष॑क्तु॒ सिष॑क्तु दु॒च्छुना᳚ । \newline
38. दु॒च्छुनेति॑ दु॒च्छुना᳚ । \newline
39. वर्द्धा॑ नो नो॒ वर्द्ध॒ वर्द्धा॑ नो॒ अम॑व॒ दम॑वन् नो॒ वर्द्ध॒ वर्द्धा॑ नो॒ अम॑वत् । \newline
40. नो॒ अम॑व॒ दम॑वन् नो नो॒ अम॑व॒ च्छवः॒ शवो॒ अम॑वन् नो नो॒ अम॑व॒ च्छवः॑ । \newline
41. अम॑व॒ च्छवः॒ शवो॒ अम॑व॒ दम॑व॒ च्छवः॑ । \newline
42. अम॑व॒दित्यम॑ - व॒त् । \newline
43. शव॒ इति॒ शवः॑ । \newline
44. यस्या जु॑ष॒ दजु॑ष॒द् यस्य॒ यस्या जु॑षन् नम॒स्विनो॑ नम॒स्विनो ऽजु॑ष॒द् यस्य॒ यस्या जु॑षन् नम॒स्विनः॑ । \newline
45. अजु॑षन् नम॒स्विनो॑ नम॒स्विनो ऽजु॑ष॒ दजु॑षन् नम॒स्विनः॒ शमीꣳ॒॒ शमी᳚म् नम॒स्विनो ऽजु॑ष॒ दजु॑षन् नम॒स्विनः॒ शमी᳚म् । \newline
46. न॒म॒स्विनः॒ शमीꣳ॒॒ शमी᳚म् नम॒स्विनो॑ नम॒स्विनः॒ शमी॒ मदु॑र्मख॒स्या दु॑र्मखस्य॒ शमी᳚म् नम॒स्विनो॑ नम॒स्विनः॒ शमी॒ मदु॑र्मखस्य । \newline
47. शमी॒ मदु॑र्मख॒स्या दु॑र्मखस्य॒ शमीꣳ॒॒ शमी॒ मदु॑र्मखस्य वा॒ वा ऽदु॑र्मखस्य॒ शमीꣳ॒॒ शमी॒ मदु॑र्मखस्य वा । \newline
48. अदु॑र्मखस्य वा॒ वा ऽदु॑र्मख॒स्या दु॑र्मखस्य वा । \newline
49. अदु॑र्मख॒स्येत्यदुः॑ - म॒ख॒स्य॒ । \newline
50. वेति॑ वा । \newline
51. तम् घ॑ घ॒ तम् तम् घे दिद् घ॒ तम् तम् घे त् । \newline
52. घे दिद् घ॒ घे द॒ग्नि र॒ग्नि रिद् घ॒ घे द॒ग्निः । \newline
53. इद॒ग्नि र॒ग्नि रिदि द॒ग्निर् वृ॒धा वृ॒धा ऽग्निरिदि द॒ग्निर् वृ॒धा । \newline
54. अ॒ग्निर् वृ॒धा वृ॒धा ऽग्नि र॒ग्निर् वृ॒धा ऽव॑त्यवति वृ॒धा ऽग्नि र॒ग्निर् वृ॒धा ऽव॑ति । \newline
55. वृ॒धा ऽव॑त्यवति वृ॒धा वृ॒धा ऽव॑ति । \newline
56. अ॒व॒तीत्य॑वति । \newline
57. पर॑स्या॒ अध्यधि॒ पर॑स्याः॒ पर॑स्या॒ अधि॑ सं॒ॅवतः॑ सं॒ॅवतो॒ अधि॒ पर॑स्याः॒ पर॑स्या॒ अधि॑ सं॒ॅवतः॑ । \newline
58. अधि॑ सं॒ॅवतः॑ सं॒ॅवतो॒ अध्यधि॑ सं॒ॅवतो ऽव॑राꣳ॒॒ अव॑रान् थ्सं॒ॅवतो॒ अध्यधि॑ सं॒ॅवतो ऽव॑रान् । \newline
\pagebreak
\markright{ TS 2.6.11.4  \hfill https://www.vedavms.in \hfill}
\addcontentsline{toc}{section}{ TS 2.6.11.4 }
\section*{ TS 2.6.11.4 }

\textbf{TS 2.6.11.4 } \newline
\textbf{Samhita Paata} \newline

सं॒ॅवतोऽव॑राꣳ अ॒भ्या त॑र । यत्रा॒हमस्मि॒ ताꣳ अ॑व ॥ वि॒द्मा हि ते॑ पु॒रा व॒यमग्ने॑ पि॒तुर्यथाव॑सः । अधा॑ ते सु॒म्नमी॑महे ॥ य उ॒ग्र इ॑व शर्य॒हा ति॒ग्मशृ॑ङ्गो॒ न वꣳस॑गः । अग्ने॒ पुरो॑ रु॒रोजि॑थ ॥ सखा॑यः॒ सं ॅवः॑ सं॒यञ्च॒मिषꣳ॒॒ स्तोमं॑ चा॒ग्नये᳚ । वर्.षि॑ष्ठाय क्षिती॒नामू॒र्जो नप्त्रे॒ सह॑स्वते ॥ सꣳ स॒मिद्यु॑वसे वृष॒न्न ( ) -ग्ने॒ विश्वा᳚न्य॒र्य आ । इ॒डस्प॒दे समि॑ध्यसे॒ स नो॒ वसू॒न्या भ॑र ।प्रजा॑पते॒ >1, स वे॑द॒>2, सोमा॑ पूषणे॒>3, मौ दे॒वौ>4 ॥ \newline

\textbf{Pada Paata} \newline

सं॒ॅवत॒ इति॑ सं-वतः॑ । अव॑रान् । अ॒भि । एति॑ । त॒र॒ ॥ यत्र॑ । अ॒हम् । अस्मि॑ । तान् । अ॒व॒ ॥ वि॒द्म । हि । ते॒ । पु॒रा । व॒यम् । अग्ने᳚ । पि॒तुः । यथा᳚ । अव॑सः ॥ अध॑ । ते॒ । सु॒म्नम् । ई॒म॒हे॒ ॥ यः । उ॒ग्रः । इ॒व॒ । श॒र्य॒हेति॑ शर्य - हा । ति॒ग्मशृ॑ङ्ग॒ इति॑ ति॒ग्म - शृ॒ङ्गः॒ । न । वꣳस॑गः ॥ अग्ने᳚ । पुरः॑ । रु॒रोजि॑थ ॥ सखा॑यः । समिति॑ । वः॒ । स॒म्यञ्च᳚म् । इष᳚म् । स्तोम᳚म् । च॒ । अ॒ग्नये᳚ ॥ वर्.षि॑ष्ठाय । क्षि॒ती॒नाम् । ऊ॒र्जः । नप्त्रे᳚ । सह॑स्वते ॥ सꣳस॒मिति॒ सं-स॒म् । इत् । यु॒व॒से॒ । वृ॒ष॒न्न् ( ) । अग्ने᳚ । विश्वा॑नि । अ॒र्यः । आ ॥ इ॒डः । प॒दे । समिति॑ । इ॒द्ध्य॒से॒ । सः । नः॒ । वसू॑नि । एति॑ । भ॒र॒ ॥ प्रजा॑पत॒ इति॒ प्रजा᳚ - प॒ते॒ । सः । वे॒द॒ । सोमा॑पूष॒णेति॒ सोमा᳚ - पू॒ष॒णा॒ । इ॒मौ । दे॒वौ ॥  \newline


\textbf{Krama Paata} \newline

स॒म्ॅवतो ऽव॑रान् । स॒म्ॅवत॒ इति॑ सं - वतः॑ । अव॑राꣳ अ॒भि । अ॒भ्या । आ त॑र । त॒रेति॑ तर ॥ यत्रा॒हम् । अ॒हमस्मि॑ । अस्मि॒ तान् । ताꣳ अ॑व । अ॒वेत्य॑व ॥ वि॒द्मा हि । हि ते᳚ । ते॒ पु॒रा । पु॒रा व॒यम् । व॒यमग्ने᳚ । अग्ने॑ पि॒तुः । पि॒तुर् यथा᳚ । यथा ऽव॑सः । अव॑स॒ इत्यव॑सः ॥ अधा॑ ते । ते॒ सु॒म्नम् । सु॒म्नमी॑महे । ई॒म॒ह॒ इती॑महे ॥ य उ॒ग्रः । उ॒ग्र इ॑व । इ॒व॒ श॒र्य॒हा । श॒र्य॒हा ति॒ग्मशृ॑ङ्गः । श॒र्य॒हेति॑ शर्य - हा । ति॒ग्मशृ॑ङ्गो॒ न । ति॒ग्मशृ॑ङ्ग॒ इति॑ ति॒ग्म - शृ॒ङ्गः॒ । न वꣳस॑गः । वꣳस॑ग॒ इति॒ वꣳस॑गः ॥ अग्ने॒ पुरः॑ । पुरो॑ रु॒रोजि॑थ । रु॒रोजि॒थेति॑ रु॒रोजि॑थ ॥ सखा॑यः॒ सम् । सम् ॅवः॑ । वः॒ स॒म्ॅयञ्च᳚म् । स॒म्ॅयञ्च॒मिष᳚म् । इषꣳ॒॒ स्तोम᳚म् । स्तोम॑म् च । चा॒ग्नये᳚ । अ॒ग्नय॒ इत्य॒ग्नये᳚ ॥ वर्.षि॑ष्ठाय क्षिती॒नाम् । क्षि॒ती॒नामू॒र्जः । ऊ॒र्जो नप्त्रे᳚ । नप्त्रे॒ सह॑स्वते । सह॑स्वत॒ इति॒ सह॑स्वते ॥ सꣳस॒मित् । सꣳस॒मिति॒ सं - स॒म् । इद् यु॑वसे । यु॒व॒से॒ वृ॒ष॒न्न्॒ ( ) । वृ॒ष॒न्नग्ने᳚ । अग्ने॒ विश्वा॑नि । विश्वा᳚न्य॒र्यः । अ॒र्य आ । एत्या ॥ इ॒डस्प॒दे । प॒दे सम् । समि॑द्ध्यसे । इ॒द्ध्य॒से॒ सः । स नः॑ । नो॒ वसू॑नि । वसू॒न्या । आ भ॑र । भ॒रेति॑ भर ॥ प्रजा॑पते॒ सः । प्रजा॑पत॒ इति॒ प्रजा᳚ - प॒ते॒ । स वे॑द । वे॒द॒ सोमा॑पूषणा । सोमा॑पूषणे॒मौ । सोमा॑पूष॒णेति॒ सोमा᳚ - पू॒ष॒णा॒ । इ॒मौ दे॒वौ । दे॒वाविति॑ दे॒वौ । \newline

\textbf{Jatai Paata} \newline

1. सं॒ॅवतो ऽव॑राꣳ॒॒ अव॑रान् थ्सं॒ॅवतः॑ सं॒ॅवतो ऽव॑रान् । \newline
2. सं॒ॅवत॒ इति॑ सं - वतः॑ । \newline
3. अव॑राꣳ अ॒भ्य॑भ्य व॑राꣳ॒॒ अव॑राꣳ अ॒भि । \newline
4. अ॒भ्या ऽभ्य॑भ्या । \newline
5. आ त॑र त॒रा त॑र । \newline
6. त॒रेति॑ तर । \newline
7. यत्रा॒ह म॒हं ॅयत्र॒ यत्रा॒हम् । \newline
8. अ॒ह मस्म्य स्म्य॒ह म॒ह मस्मि॑ । \newline
9. अस्मि॒ ताꣳ स्ताꣳ अस्म्यस्मि॒ तान् । \newline
10. ताꣳ अ॑वाव॒ ताꣳ स्ताꣳ अ॑व । \newline
11. अ॒वेत्य॑व । \newline
12. वि॒द्मा हि हि वि॒द्म वि॒द्मा हि । \newline
13. हि ते॑ ते॒ हि हि ते᳚ । \newline
14. ते॒ पु॒रा पु॒रा ते॑ ते पु॒रा । \newline
15. पु॒रा व॒यं ॅव॒यम् पु॒रा पु॒रा व॒यम् । \newline
16. व॒य मग्ने ऽग्ने॑ व॒यं ॅव॒य मग्ने᳚ । \newline
17. अग्ने॑ पि॒तुः पि॒तु रग्ने ऽग्ने॑ पि॒तुः । \newline
18. पि॒तुर् यथा॒ यथा॑ पि॒तुः पि॒तुर् यथा᳚ । \newline
19. यथा ऽव॒सो ऽव॑सो॒ यथा॒ यथा ऽव॑सः । \newline
20. अव॑स॒ इत्यव॑सः । \newline
21. अधा॑ ते॒ ते अधाधा॑ ते । \newline
22. ते॒ सु॒म्नꣳ सु॒म्नम् ते॑ ते सु॒म्नम् । \newline
23. सु॒म्न मी॑मह ईमहे सु॒म्नꣳ सु॒म्न मी॑महे । \newline
24. ई॒म॒ह॒ इती॑महे । \newline
25. य उ॒ग्र उ॒ग्रो यो य उ॒ग्रः । \newline
26. उ॒ग्र इ॑वे वो॒ग्र उ॒ग्र इ॑व । \newline
27. इ॒व॒ श॒र्य॒हा श॑र्य॒हेवे॑ व शर्य॒हा । \newline
28. श॒र्य॒हा ति॒ग्मशृ॑ङ्ग स्ति॒ग्मशृ॑ङ्गः शर्य॒हा श॑र्य॒हा ति॒ग्मशृ॑ङ्गः । \newline
29. श॒र्य॒हेति॑ शर्य - हा । \newline
30. ति॒ग्मशृ॑ङ्गो॒ न न ति॒ग्मशृ॑ङ्ग स्ति॒ग्मशृ॑ङ्गो॒ न । \newline
31. ति॒ग्मशृ॑ङ्ग॒ इति॑ ति॒ग्म - शृ॒ङ्गः॒ । \newline
32. न वꣳस॑गो॒ वꣳस॑गो॒ न न वꣳस॑गः । \newline
33. वꣳस॑ग॒ इति॒ वꣳस॑गः । \newline
34. अग्ने॒ पुरः॒ पुरो ऽग्ने ऽग्ने॒ पुरः॑ । \newline
35. पुरो॑ रु॒रोजि॑थ रु॒रोजि॑थ॒ पुरः॒ पुरो॑ रु॒रोजि॑थ । \newline
36. रु॒रोजि॒थेति॑ रु॒रोजि॑थ । \newline
37. सखा॑यः॒ सꣳ सꣳ सखा॑यः॒ सखा॑यः॒ सम् । \newline
38. सं ॅवो॑ वः॒ सꣳ सं ॅवः॑ । \newline
39. वः॒ स॒म्यञ्चꣳ॑ स॒म्यञ्चं॑ ॅवो वः स॒म्यञ्च᳚म् । \newline
40. स॒म्यञ्च॒ मिष॒ मिषꣳ॑ स॒म्यञ्चꣳ॑ स॒म्यञ्च॒ मिष᳚म् । \newline
41. इषꣳ॒॒ स्तोमꣳ॒॒ स्तोम॒ मिष॒ मिषꣳ॒॒ स्तोम᳚म् । \newline
42. स्तोम॑म् च च॒ स्तोमꣳ॒॒ स्तोम॑म् च । \newline
43. चा॒ग्नये॑ अ॒ग्नये॑ च चा॒ग्नये᳚ । \newline
44. अ॒ग्नय॒ इत्य॒ग्नये᳚ । \newline
45. वर्.षि॑ष्ठाय क्षिती॒नाम् क्षि॑ती॒नां ॅवर्.षि॑ष्ठाय॒ वर्.षि॑ष्ठाय क्षिती॒नाम् । \newline
46. क्षि॒ती॒ना मू॒र्ज ऊ॒र्जः क्षि॑ती॒नाम् क्षि॑ती॒ना मू॒र्जः । \newline
47. ऊ॒र्जो नप्त्रे॒ नप्त्र॑ ऊ॒र्ज ऊ॒र्जो नप्त्रे᳚ । \newline
48. नप्त्रे॒ सह॑स्वते॒ सह॑स्वते॒ नप्त्रे॒ नप्त्रे॒ सह॑स्वते । \newline
49. सह॑स्वत॒ इति॒ सह॑स्वते । \newline
50. सꣳस॒ मिदिथ् सꣳसꣳ॒॒ सꣳस॒ मित् । \newline
51. सꣳस॒मिति॒ सं - स॒म् । \newline
52. इद् यु॑वसे युवस॒ इदिद् यु॑वसे । \newline
53. यु॒व॒से॒ वृ॒ष॒न्॒. वृ॒ष॒न्॒. यु॒व॒से॒ यु॒व॒से॒ वृ॒ष॒न्न् । \newline
54. वृ॒ष॒न् नग्ने ऽग्ने॑ वृषन् वृष॒न् नग्ने᳚ । \newline
55. अग्ने॒ विश्वा॑नि॒ विश्वा॒ न्यग्ने ऽग्ने॒ विश्वा॑नि । \newline
56. विश्वा᳚ न्य॒र्यो अ॒र्यो विश्वा॑नि॒ विश्वा᳚ न्य॒र्यः । \newline
57. अ॒र्य आ ऽर्यो अ॒र्य आ । \newline
58. एत्या । \newline
59. इ॒ड स्प॒दे प॒द इ॒ड इ॒ड स्प॒दे । \newline
60. प॒दे सꣳ सम् प॒दे प॒दे सम् । \newline
61. स मि॑द्ध्यस इद्ध्यसे॒ सꣳ स मि॑द्ध्यसे । \newline
62. इ॒द्ध्य॒से॒ स स इ॑द्ध्यस इद्ध्यसे॒ सः । \newline
63. स नो॑ नः॒ स स नः॑ । \newline
64. नो॒ वसू॑नि॒ वसू॑नि नो नो॒ वसू॑नि । \newline
65. वसू॒न्या वसू॑नि॒ वसू॒न्या । \newline
66. आ भ॑र भ॒रा भ॑र । \newline
67. भ॒रेति॑ भर । \newline
68. प्रजा॑पते॒ स स प्रजा॑पते॒ प्रजा॑पते॒ सः । \newline
69. प्रजा॑पत॒ इति॒ प्रजा᳚ - प॒ते॒ । \newline
70. स वे॑द वेद॒ स स वे॑द । \newline
71. वे॒द॒ सोमा॑पूषणा॒ सोमा॑पूषणा वेद वेद॒ सोमा॑पूषणा । \newline
72. सोमा॑पूषणे॒मा वि॒मौ सोमा॑पूषणा॒ सोमा॑पूषणे॒मौ । \newline
73. सोमा॑पूष॒णेति॒ सोमा᳚ - पू॒ष॒णा॒ । \newline
74. इ॒मौ दे॒वौ दे॒वा वि॒मा वि॒मौ दे॒वौ । \newline
75. दे॒वाविति॑ दे॒वौ । \newline

\textbf{Ghana Paata } \newline

1. सं॒ॅवतो ऽव॑राꣳ॒॒ अव॑रान् थ्सं॒ॅवतः॑ सं॒ॅवतो ऽव॑राꣳ अ॒भ्य॑भ्य व॑रान् थ्सं॒ॅवतः॑ सं॒ॅवतो ऽव॑राꣳ अ॒भि । \newline
2. सं॒ॅवत॒ इति॑ सं - वतः॑ । \newline
3. अव॑राꣳ अ॒भ्य॑भ्य व॑राꣳ॒॒ अव॑राꣳ अ॒भ्या ऽभ्यव॑राꣳ॒॒ अव॑राꣳ अ॒भ्या । \newline
4. अ॒भ्या ऽभ्य॑ भ्या त॑र त॒रा ऽभ्य॑  भ्या त॑र । \newline
5. आ त॑र त॒रा त॑र । \newline
6. त॒रेति॑ तर । \newline
7. यत्रा॒ह म॒हं ॅयत्र॒ यत्रा॒ह मस्म्य स्म्य॒हं ॅयत्र॒ यत्रा॒ह मस्मि॑ । \newline
8. अ॒ह मस्म्य स्म्य॒ह म॒ह मस्मि॒ ताꣳ स्ताꣳ अस्म्य॒ह म॒ह मस्मि॒ तान् । \newline
9. अस्मि॒ ताꣳ स्ताꣳ अस्म्यस्मि॒ ताꣳ अ॑वाव॒ ताꣳ अस्म्यस्मि॒ ताꣳ अ॑व । \newline
10. ताꣳ अ॑वाव॒ ताꣳ स्ताꣳ अ॑व । \newline
11. अ॒वेत्य॑व । \newline
12. वि॒द्मा हि हि वि॒द्म वि॒द्मा हि ते॑ ते॒ हि वि॒द्म वि॒द्मा हि ते᳚ । \newline
13. हि ते॑ ते॒ हि हि ते॑ पु॒रा पु॒रा ते॒ हि हि ते॑ पु॒रा । \newline
14. ते॒ पु॒रा पु॒रा ते॑ ते पु॒रा व॒यं ॅव॒यम् पु॒रा ते॑ ते पु॒रा व॒यम् । \newline
15. पु॒रा व॒यं ॅव॒यम् पु॒रा पु॒रा व॒य मग्ने ऽग्ने॑ व॒यम् पु॒रा पु॒रा व॒य मग्ने᳚ । \newline
16. व॒य मग्ने ऽग्ने॑ व॒यं ॅव॒य मग्ने॑ पि॒तुः पि॒तु रग्ने॑ व॒यं ॅव॒य मग्ने॑ पि॒तुः । \newline
17. अग्ने॑ पि॒तुः पि॒तु रग्ने ऽग्ने॑ पि॒तुर् यथा॒ यथा॑ पि॒तु रग्ने ऽग्ने॑ पि॒तुर् यथा᳚ । \newline
18. पि॒तुर् यथा॒ यथा॑ पि॒तुः पि॒तुर् यथा ऽव॒सो ऽव॑सो॒ यथा॑ पि॒तुः पि॒तुर् यथा ऽव॑सः । \newline
19. यथा ऽव॒सो ऽव॑सो॒ यथा॒ यथा ऽव॑सः । \newline
20. अव॑स॒ इत्यव॑सः । \newline
21. अधा॑ ते॒ ते अधाधा॑ ते सु॒म्नꣳ सु॒म्नम् ते अधाधा॑ ते सु॒म्नम् । \newline
22. ते॒ सु॒म्नꣳ सु॒म्नम् ते॑ ते सु॒म्न मी॑मह ईमहे सु॒म्नम् ते॑ ते सु॒म्न मी॑महे । \newline
23. सु॒म्न मी॑मह ईमहे सु॒म्नꣳ सु॒म्न मी॑महे । \newline
24. ई॒म॒ह॒ इती॑महे । \newline
25. य उ॒ग्र उ॒ग्रो यो य उ॒ग्र इ॑वे वो॒ग्रो यो य उ॒ग्र इ॑व । \newline
26. उ॒ग्र इ॑वे वो॒ग्र उ॒ग्र इ॑व शर्य॒हा श॑र्य॒हेवो॒ग्र उ॒ग्र इ॑व शर्य॒हा । \newline
27. इ॒व॒ श॒र्य॒हा श॑र्य॒हेवे॑ व शर्य॒हा ति॒ग्मशृ॑ङ्ग स्ति॒ग्मशृ॑ङ्गः शर्य॒हेवे॑ व शर्य॒हा ति॒ग्मशृ॑ङ्गः । \newline
28. श॒र्य॒हा ति॒ग्मशृ॑ङ्ग स्ति॒ग्मशृ॑ङ्गः शर्य॒हा श॑र्य॒हा ति॒ग्मशृ॑ङ्गो॒ न न ति॒ग्मशृ॑ङ्गः शर्य॒हा श॑र्य॒हा ति॒ग्मशृ॑ङ्गो॒ न । \newline
29. श॒र्य॒हेति॑ शर्य - हा । \newline
30. ति॒ग्मशृ॑ङ्गो॒ न न ति॒ग्मशृ॑ङ्ग स्ति॒ग्मशृ॑ङ्गो॒ न वꣳस॑गो॒ वꣳस॑गो॒ न ति॒ग्मशृ॑ङ्ग स्ति॒ग्मशृ॑ङ्गो॒ न वꣳस॑गः । \newline
31. ति॒ग्मशृ॑ङ्ग॒ इति॑ ति॒ग्म - शृ॒ङ्गः॒ । \newline
32. न वꣳस॑गो॒ वꣳस॑गो॒ न न वꣳस॑गः । \newline
33. वꣳस॑ग॒ इति॒ वꣳस॑गः । \newline
34. अग्ने॒ पुरः॒ पुरो ऽग्ने ऽग्ने॒ पुरो॑ रु॒रोजि॑थ रु॒रोजि॑थ॒ पुरो ऽग्ने ऽग्ने॒ पुरो॑ रु॒रोजि॑थ । \newline
35. पुरो॑ रु॒रोजि॑थ रु॒रोजि॑थ॒ पुरः॒ पुरो॑ रु॒रोजि॑थ । \newline
36. रु॒रोजि॒थेति॑ रु॒रोजि॑थ । \newline
37. सखा॑यः॒ सꣳ सꣳ सखा॑यः॒ सखा॑यः॒ सं ॅवो॑ वः॒ सꣳ सखा॑यः॒ सखा॑यः॒ सं ॅवः॑ । \newline
38. सं ॅवो॑ वः॒ सꣳ सं ॅवः॑ स॒म्यञ्चꣳ॑ स॒म्यञ्चं॑ ॅवः॒ सꣳ सं ॅवः॑ स॒म्यञ्च᳚म् । \newline
39. वः॒ स॒म्यञ्चꣳ॑ स॒म्यञ्चं॑ ॅवो वः स॒म्यञ्च॒ मिष॒ मिषꣳ॑ स॒म्यञ्चं॑ ॅवो वः स॒म्यञ्च॒ मिष᳚म् । \newline
40. स॒म्यञ्च॒ मिष॒ मिषꣳ॑ स॒म्यञ्चꣳ॑ स॒म्यञ्च॒ मिषꣳ॒॒ स्तोमꣳ॒॒ स्तोम॒ मिषꣳ॑ स॒म्यञ्चꣳ॑ स॒म्यञ्च॒ मिषꣳ॒॒ स्तोम᳚म् । \newline
41. इषꣳ॒॒ स्तोमꣳ॒॒ स्तोम॒ मिष॒ मिषꣳ॒॒ स्तोम॑म् च च॒ स्तोम॒ मिष॒ मिषꣳ॒॒ स्तोम॑म् च । \newline
42. स्तोम॑म् च च॒ स्तोमꣳ॒॒ स्तोम॑म् चा॒ग्नये॑ अ॒ग्नये॑ च॒ स्तोमꣳ॒॒ स्तोम॑म् चा॒ग्नये᳚ । \newline
43. चा॒ग्नये॑ अ॒ग्नये॑ च चा॒ग्नये᳚ । \newline
44. अ॒ग्नय॒ इत्य॒ग्नये᳚ । \newline
45. वर्.षि॑ष्ठाय क्षिती॒नाम् क्षि॑ती॒नां ॅवर्.षि॑ष्ठाय॒ वर्.षि॑ष्ठाय क्षिती॒ना मू॒र्ज ऊ॒र्जः क्षि॑ती॒नां ॅवर्.षि॑ष्ठाय॒ वर्.षि॑ष्ठाय क्षिती॒ना मू॒र्जः । \newline
46. क्षि॒ती॒ना मू॒र्ज ऊ॒र्जः क्षि॑ती॒नाम् क्षि॑ती॒ना मू॒र्जो नप्त्रे॒ नप्त्र॑ ऊ॒र्जः क्षि॑ती॒नाम् क्षि॑ती॒ना मू॒र्जो नप्त्रे᳚ । \newline
47. ऊ॒र्जो नप्त्रे॒ नप्त्र॑ ऊ॒र्ज ऊ॒र्जो नप्त्रे॒ सह॑स्वते॒ सह॑स्वते॒ नप्त्र॑ ऊ॒र्ज ऊ॒र्जो नप्त्रे॒ सह॑स्वते । \newline
48. नप्त्रे॒ सह॑स्वते॒ सह॑स्वते॒ नप्त्रे॒ नप्त्रे॒ सह॑स्वते । \newline
49. सह॑स्वत॒ इति॒ सह॑स्वते । \newline
50. सꣳस॒ मिदिथ् सꣳसꣳ॒॒ सꣳस॒ मिद् यु॑वसे युवस॒ इथ् सꣳसꣳ॒॒ सꣳस॒ मिद् यु॑वसे । \newline
51. सꣳस॒मिति॒ सं - स॒म् । \newline
52. इद् यु॑वसे युवस॒ इदिद् यु॑वसे वृषन् वृषन्. युवस॒ इदिद् यु॑वसे वृषन्न् । \newline
53. यु॒व॒से॒ वृ॒ष॒न्॒ वृ॒ष॒न्॒. यु॒व॒से॒ यु॒व॒से॒ वृ॒ष॒न् नग्ने ऽग्ने॑ वृषन्. युवसे युवसे वृष॒न् नग्ने᳚ । \newline
54. वृ॒ष॒न् नग्ने ऽग्ने॑ वृषन् वृष॒न् नग्ने॒ विश्वा॑नि॒ विश्वा॒ न्यग्ने॑ वृषन् वृष॒न् नग्ने॒ विश्वा॑नि । \newline
55. अग्ने॒ विश्वा॑नि॒ विश्वा॒ न्यग्ने ऽग्ने॒ विश्वा᳚न्य॒र्यो अ॒र्यो विश्वा॒ न्यग्ने ऽग्ने॒ विश्वा᳚ न्य॒र्यः । \newline
56. विश्वा᳚ न्य॒र्यो अ॒र्यो विश्वा॑नि॒ विश्वा᳚ न्य॒र्य आ ऽर्यो विश्वा॑नि॒ विश्वा᳚ न्य॒र्य आ । \newline
57. अ॒र्य आ ऽर्यो अ॒र्य आ । \newline
58. एत्या । \newline
59. इ॒ड स्प॒दे प॒द इ॒ड इ॒ड स्प॒दे सꣳ सम् प॒द इ॒ड इ॒ड स्प॒दे सम् । \newline
60. प॒दे सꣳ सम् प॒दे प॒दे स मि॑द्ध्यस इद्ध्यसे॒ सम् प॒दे प॒दे स मि॑द्ध्यसे । \newline
61. स मि॑द्ध्यस इद्ध्यसे॒ सꣳ स मि॑द्ध्यसे॒ स स इ॑द्ध्यसे॒ सꣳ स मि॑द्ध्यसे॒ सः । \newline
62. इ॒द्ध्य॒से॒ स स इ॑द्ध्यस इद्ध्यसे॒ स नो॑ नः॒ स इ॑द्ध्यस इद्ध्यसे॒ स नः॑ । \newline
63. स नो॑ नः॒ स स नो॒ वसू॑नि॒ वसू॑नि नः॒ स स नो॒ वसू॑नि । \newline
64. नो॒ वसू॑नि॒ वसू॑नि नो नो॒ वसू॒न्या वसू॑नि नो नो॒ वसू॒न्या । \newline
65. वसू॒न्या वसू॑नि॒ वसू॒न्या भ॑र भ॒रा वसू॑नि॒ वसू॒न्या भ॑र । \newline
66. आ भ॑र भ॒रा भ॑र । \newline
67. भ॒रेति॑ भर । \newline
68. प्रजा॑पते॒ स स प्रजा॑पते॒ प्रजा॑पते॒ स वे॑द वेद॒ स प्रजा॑पते॒ प्रजा॑पते॒ स वे॑द । \newline
69. प्रजा॑पत॒ इति॒ प्रजा᳚ - प॒ते॒ । \newline
70. स वे॑द वेद॒ स स वे॑द॒ सोमा॑पूषणा॒ सोमा॑पूषणा वेद॒ स स वे॑द॒ सोमा॑पूषणा । \newline
71. वे॒द॒ सोमा॑पूषणा॒ सोमा॑पूषणा वेद वेद॒ सोमा॑पूषणे॒मा वि॒मौ सोमा॑पूषणा वेद वेद॒ सोमा॑पूषणे॒मौ । \newline
72. सोमा॑पूषणे॒मा वि॒मौ सोमा॑पूषणा॒ सोमा॑पूषणे॒मौ दे॒वौ दे॒वा वि॒मौ सोमा॑पूषणा॒ सोमा॑पूषणे॒मौ दे॒वौ । \newline
73. सोमा॑पूष॒णेति॒ सोमा᳚ - पू॒ष॒णा॒ । \newline
74. इ॒मौ दे॒वौ दे॒वा वि॒मा वि॒मौ दे॒वौ । \newline
75. दे॒वाविति॑ दे॒वौ । \newline
\pagebreak
\markright{ TS 2.6.12.1  \hfill https://www.vedavms.in \hfill}
\addcontentsline{toc}{section}{ TS 2.6.12.1 }
\section*{ TS 2.6.12.1 }

\textbf{TS 2.6.12.1 } \newline
\textbf{Samhita Paata} \newline

उ॒शन्त॑स्त्वा हवामह उ॒शन्तः॒ समि॑धीमहि । उ॒शन्नु॑श॒त आ व॑ह पि॒तॄन्. ह॒विषे॒ अत्त॑वे ॥ त्वꣳ सो॑म॒ प्रचि॑कितो मनी॒षा त्वꣳ रजि॑ष्ठ॒मनु॑ नेषि॒ पन्थां᳚ । तव॒ प्रणी॑ती पि॒तरो॑ न इन्दो दे॒वेषु॒ रत्न॑म भजन्त॒ धीराः᳚ ॥त्वया॒ हि नः॑ पि॒तरः॑ सोम॒ पूर्वे॒ कर्मा॑णि च॒क्रुः प॑वमान॒ धीराः᳚ । व॒न्वन्नवा॑तः परि॒धीꣳ रपो᳚र्णु वी॒रेभि॒रश्वै᳚र्म॒घवा॑ भवा - [  ] \newline

\textbf{Pada Paata} \newline

उ॒शन्तः॑ । त्वा॒ । ह॒वा॒म॒हे॒ । उ॒शन्तः॑ । समिति॑ । इ॒धी॒म॒हि॒ ॥ उ॒शन्न् । उ॒श॒तः । एति॑ । व॒ह॒ । पि॒तॄन् । ह॒विषे᳚ । अत्त॑वे ॥ त्वम् । सो॒म॒ । प्रचि॑कित॒ इति॒ प्र - चि॒कि॒तः॒ । म॒नी॒षा । त्वम् । रजि॑ष्ठम् । अन्विति॑ । ने॒षि॒ । पन्था᳚म् ॥ तव॑ । प्रणी॒तीति॒ प्र - नी॒ती॒ । पि॒तरः॑ । नः॒ । इ॒न्दो॒ इति॑ । दे॒वेषु॑ । रत्न᳚म् । अ॒भ॒ज॒न्त॒ । धीराः᳚ ॥ त्वया᳚ । हि । नः॒ । पि॒तरः॑ । सो॒म॒ । पूर्वे᳚ । कर्मा॑णि । च॒क्रुः । प॒व॒मा॒न॒ । धीराः᳚ ॥ व॒न्वन्न् । अवा॑तः । प॒रि॒धीनिति॑ परि - धीन् । अपेति॑ । ऊ॒र्णु॒ । वी॒रेभिः॑ । अश्वैः᳚ । म॒घवेति॑ म॒घ - वा॒ । भ॒व॒ ।  \newline


\textbf{Krama Paata} \newline

उ॒शन्त॑स्त्वा । त्वा॒ ह॒वा॒म॒हे॒ । ह॒वा॒म॒ह॒ उ॒शन्तः॑ । उ॒शन्तः॒ सम् । समि॑धीमहि । इ॒धी॒म॒हीती॑धीमहि ॥ उ॒शन्नु॑श॒तः । उ॒श॒त आ । आ व॑ह । व॒ह॒ पि॒तॄन् । पि॒तॄन्. ह॒विषे᳚ । ह॒विषे॒ अत्त॑वे । अत्त॑व॒ इत्यत्त॑वे ॥ त्वꣳ सो॑म । सो॒म॒ प्रचि॑कितः । प्रचि॑कितो मनी॒षा । प्रचि॑कित॒ इति॒ प्र - चि॒कि॒तः॒ । म॒नी॒षा त्वम् । त्वꣳ रजि॑ष्ठम् । रजि॑ष्ठ॒मनु॑ । अनु॑ नेषि । ने॒षि॒ पन्था᳚म् । पन्था॒मिति॒ पन्था᳚म् ॥ तव॒ प्रणी॑ती । प्रणी॑ती पि॒तरः॑ । प्रणी॒तीति॒ प्र - नी॒ती॒ । पि॒तरो॑ नः । न॒ इ॒न्दो॒ । इ॒न्दो॒ दे॒वेषु॑ । इ॒न्दो॒ इती᳚न्दो । दे॒वेषु॒ रत्न᳚म् । रत्न॑मभजन्त । अ॒भ॒ज॒न्त॒ धीराः᳚ । धीरा॒ इति॒ धीराः᳚ ॥ त्वया॒ हि । हि नः॑ । नः॒ पि॒तरः॑ । पि॒तरः॑ सोम । सो॒म॒ पूर्वे᳚ । पूर्वे॒ कर्मा॑णि । कर्मा॑णि च॒क्रुः । च॒क्रुः प॑वमान । प॒व॒मा॒न॒ धीराः᳚ । धीरा॒ इति॒ धीराः᳚ ॥ व॒न्वन्नवा॑तः । अवा॑तः परि॒धीन् । प॒रि॒धीꣳरप॑ । प॒रि॒धीनिति॑ परि - धीन् । अपो᳚र्णु । ऊ॒र्णु॒ वी॒रेभिः॑ । वि॒रेभि॒रश्वैः᳚ । अश्वै᳚र् म॒घवा᳚ । म॒घवा॑ भव । म॒घवेति॑ म॒घ - वा॒ । भ॒वा॒ नः॒ \newline

\textbf{Jatai Paata} \newline

1. उ॒शन्त॑ स्त्वा त्वो॒शन्त॑ उ॒शन्त॑ स्त्वा । \newline
2. त्वा॒ ह॒वा॒म॒हे॒ ह॒वा॒म॒हे॒ त्वा॒ त्वा॒ ह॒वा॒म॒हे॒ । \newline
3. ह॒वा॒म॒ह॒ उ॒शन्त॑ उ॒शन्तो॑ हवामहे हवामह उ॒शन्तः॑ । \newline
4. उ॒शन्तः॒ सꣳ स मु॒शन्त॑ उ॒शन्तः॒ सम् । \newline
5. स मि॑धीमही धीमहि॒ सꣳ स मि॑धीमहि । \newline
6. इ॒धी॒म॒हीती॑धीमहि । \newline
7. उ॒शन् नु॑श॒त उ॑श॒त उ॒शन् नु॒शन् नु॑श॒तः । \newline
8. उ॒श॒त ओश॒त उ॑श॒त आ । \newline
9. आ व॑ह व॒हा व॑ह । \newline
10. व॒ह॒ पि॒तॄन् पि॒तॄन्. व॑ह वह पि॒तॄन् । \newline
11. पि॒तॄन्. ह॒विषे॑ ह॒विषे॑ पि॒तॄन् पि॒तॄन्. ह॒विषे᳚ । \newline
12. ह॒विषे॒ अत्त॑वे॒ अत्त॑वे ह॒विषे॑ ह॒विषे॒ अत्त॑वे । \newline
13. अत्त॑व॒ इत्यत्त॑वे । \newline
14. त्वꣳ सो॑म सोम॒ त्वम् त्वꣳ सो॑म । \newline
15. सो॒म॒ प्रचि॑कितः॒ प्रचि॑कितः सोम सोम॒ प्रचि॑कितः । \newline
16. प्रचि॑कितो मनी॒षा म॑नी॒षा प्रचि॑कितः॒ प्रचि॑कितो मनी॒षा । \newline
17. प्रचि॑कित॒ इति॒ प्र - चि॒कि॒तः॒ । \newline
18. म॒नी॒षा त्वम् त्वम् म॑नी॒षा म॑नी॒षा त्वम् । \newline
19. त्वꣳ रजि॑ष्ठꣳ॒॒ रजि॑ष्ठ॒म् त्वम् त्वꣳ रजि॑ष्ठम् । \newline
20. रजि॑ष्ठ॒ मन्वनु॒ रजि॑ष्ठꣳ॒॒ रजि॑ष्ठ॒ मनु॑ । \newline
21. अनु॑ नेषि ने॒ष्यन्वनु॑ नेषि । \newline
22. ने॒षि॒ पन्था॒म् पन्था᳚म् नेषि नेषि॒ पन्था᳚म् । \newline
23. पन्था॒मिति॒ पन्था᳚म् । \newline
24. तव॒ प्रणी॑ती॒ प्रणी॑ती॒ तव॒ तव॒ प्रणी॑ती । \newline
25. प्रणी॑ती पि॒तरः॑ पि॒तरः॒ प्रणी॑ती॒ प्रणी॑ती पि॒तरः॑ । \newline
26. प्रणी॒तीति॒ प्र - नी॒ती॒ । \newline
27. पि॒तरो॑ नो नः पि॒तरः॑ पि॒तरो॑ नः । \newline
28. न॒ इ॒न्दो॒ इ॒न्दो॒ नो॒ न॒ इ॒न्दो॒ । \newline
29. इ॒न्दो॒ दे॒वेषु॑ दे॒वे ष्वि॑न्दो इन्दो दे॒वेषु॑ । \newline
30. इ॒न्दो॒ इती᳚न्दो । \newline
31. दे॒वेषु॒ रत्नꣳ॒॒ रत्न॑म् दे॒वेषु॑ दे॒वेषु॒ रत्न᳚म् । \newline
32. रत्न॑ मभजन्ता भजन्त॒ रत्नꣳ॒॒ रत्न॑ मभजन्त । \newline
33. अ॒भ॒ज॒न्त॒ धीरा॒ धीरा॑ अभजन्ता भजन्त॒ धीराः᳚ । \newline
34. धीरा॒ इति॒ धीराः᳚ । \newline
35. त्वया॒ हि हि त्वया॒ त्वया॒ हि । \newline
36. हि नो॑ नो॒ हि हि नः॑ । \newline
37. नः॒ पि॒तरः॑ पि॒तरो॑ नो नः पि॒तरः॑ । \newline
38. पि॒तरः॑ सोम सोम पि॒तरः॑ पि॒तरः॑ सोम । \newline
39. सो॒म॒ पूर्वे॒ पूर्वे॑ सोम सोम॒ पूर्वे᳚ । \newline
40. पूर्वे॒ कर्मा॑णि॒ कर्मा॑णि॒ पूर्वे॒ पूर्वे॒ कर्मा॑णि । \newline
41. कर्मा॑णि च॒क्रु श्च॒क्रुः कर्मा॑णि॒ कर्मा॑णि च॒क्रुः । \newline
42. च॒क्रुः प॑वमान पवमान च॒क्रु श्च॒क्रुः प॑वमान । \newline
43. प॒व॒मा॒न॒ धीरा॒ धीराः᳚ पवमान पवमान॒ धीराः᳚ । \newline
44. धीरा॒ इति॒ धीराः᳚ । \newline
45. व॒न्वन् नवा॒तो ऽवा॑तो व॒न्वन्. व॒न्वन् नवा॑तः । \newline
46. अवा॑तः परि॒धीन् प॑रि॒धीꣳ रवा॒तो ऽवा॑तः परि॒धीन् । \newline
47. प॒रि॒धीꣳ रपाप॑ परि॒धीन् प॑रि॒धीꣳ रप॑ । \newline
48. प॒रि॒धीनिति॑ परि - धीन् । \newline
49. अपो᳚र्णू॒र् ण्वपापो᳚र्णु । \newline
50. ऊ॒र्णु॒ वी॒रेभि॑र् वी॒रेभि॑ रूर्णूर्णु वी॒रेभिः॑ । \newline
51. वी॒रेभि॒ रश्वै॒ रश्वै᳚र् वी॒रेभि॑र् वी॒रेभि॒ रश्वैः᳚ । \newline
52. अश्वै᳚र् म॒घवा॑ म॒घवा ऽश्वै॒ रश्वै᳚र् म॒घवा᳚ । \newline
53. म॒घवा॑ भव भव म॒घवा॑ म॒घवा॑ भव । \newline
54. म॒घवेति॑ म॒घ - वा॒ । \newline
55. भ॒वा॒ नो॒ नो॒ भ॒व॒ भ॒वा॒ नः॒ । \newline

\textbf{Ghana Paata } \newline

1. उ॒शन्त॑ स्त्वा त्वो॒शन्त॑ उ॒शन्त॑ स्त्वा हवामहे हवामहे त्वो॒शन्त॑ उ॒शन्त॑ स्त्वा हवामहे । \newline
2. त्वा॒ ह॒वा॒म॒हे॒ ह॒वा॒म॒हे॒ त्वा॒ त्वा॒ ह॒वा॒म॒ह॒ उ॒शन्त॑ उ॒शन्तो॑ हवामहे त्वा त्वा हवामह उ॒शन्तः॑ । \newline
3. ह॒वा॒म॒ह॒ उ॒शन्त॑ उ॒शन्तो॑ हवामहे हवामह उ॒शन्तः॒ सꣳ स मु॒शन्तो॑ हवामहे हवामह उ॒शन्तः॒ सम् । \newline
4. उ॒शन्तः॒ सꣳ स मु॒शन्त॑ उ॒शन्तः॒ स मि॑धीमही धीमहि॒ स मु॒शन्त॑ उ॒शन्तः॒ स मि॑धीमहि । \newline
5. स मि॑धीमही धीमहि॒ सꣳ स मि॑धीमहि । \newline
6. इ॒धी॒म॒हीती॑धीमहि । \newline
7. उ॒शन् नु॑श॒त उ॑श॒त उ॒शन् नु॒शन् नु॑श॒त ओश॒त उ॒शन् नु॒शन् नु॑श॒त आ । \newline
8. उ॒श॒त ओश॒त उ॑श॒त आ व॑ह व॒होश॒त उ॑श॒त आ व॑ह । \newline
9. आ व॑ह व॒हा व॑ह पि॒तॄन् पि॒तॄन्. व॒हा व॑ह पि॒तॄन् । \newline
10. व॒ह॒ पि॒तॄन् पि॒तॄन्. व॑ह वह पि॒तॄन्. ह॒विषे॑ ह॒विषे॑ पि॒तॄन्. व॑ह वह पि॒तॄन्. ह॒विषे᳚ । \newline
11. पि॒तॄन्. ह॒विषे॑ ह॒विषे॑ पि॒तॄन् पि॒तॄन्. ह॒विषे॒ अत्त॑वे॒ अत्त॑वे ह॒विषे॑ पि॒तॄन् पि॒तॄन्. ह॒विषे॒ अत्त॑वे । \newline
12. ह॒विषे॒ अत्त॑वे॒ अत्त॑वे ह॒विषे॑ ह॒विषे॒ अत्त॑वे । \newline
13. अत्त॑व॒ इत्यत्त॑वे । \newline
14. त्वꣳ सो॑म सोम॒ त्वम् त्वꣳ सो॑म॒ प्रचि॑कितः॒ प्रचि॑कितः सोम॒ त्वम् त्वꣳ सो॑म॒ प्रचि॑कितः । \newline
15. सो॒म॒ प्रचि॑कितः॒ प्रचि॑कितः सोम सोम॒ प्रचि॑कितो मनी॒षा म॑नी॒षा प्रचि॑कितः सोम सोम॒ प्रचि॑कितो मनी॒षा । \newline
16. प्रचि॑कितो मनी॒षा म॑नी॒षा प्रचि॑कितः॒ प्रचि॑कितो मनी॒षा त्वम् त्वम् म॑नी॒षा प्रचि॑कितः॒ प्रचि॑कितो मनी॒षा त्वम् । \newline
17. प्रचि॑कित॒ इति॒ प्र - चि॒कि॒तः॒ । \newline
18. म॒नी॒षा त्वम् त्वम् म॑नी॒षा म॑नी॒षा त्वꣳ रजि॑ष्ठꣳ॒॒ रजि॑ष्ठ॒म् त्वम् म॑नी॒षा म॑नी॒षा त्वꣳ रजि॑ष्ठम् । \newline
19. त्वꣳ रजि॑ष्ठꣳ॒॒ रजि॑ष्ठ॒म् त्वम् त्वꣳ रजि॑ष्ठ॒ मन्वनु॒ रजि॑ष्ठ॒म् त्वम् त्वꣳ रजि॑ष्ठ॒ मनु॑ । \newline
20. रजि॑ष्ठ॒ मन्वनु॒ रजि॑ष्ठꣳ॒॒ रजि॑ष्ठ॒ मनु॑ नेषि ने॒ष्यनु॒ रजि॑ष्ठꣳ॒॒ रजि॑ष्ठ॒ मनु॑ नेषि । \newline
21. अनु॑ नेषि ने॒ष्यन्वनु॑ नेषि॒ पन्था॒म् पन्था᳚म् ने॒ष्यन्वनु॑ नेषि॒ पन्था᳚म् । \newline
22. ने॒षि॒ पन्था॒म् पन्था᳚म् नेषि नेषि॒ पन्था᳚म् । \newline
23. पन्था॒मिति॒ पन्था᳚म् । \newline
24. तव॒ प्रणी॑ती॒ प्रणी॑ती॒ तव॒ तव॒ प्रणी॑ती पि॒तरः॑ पि॒तरः॒ प्रणी॑ती॒ तव॒ तव॒ प्रणी॑ती पि॒तरः॑ । \newline
25. प्रणी॑ती पि॒तरः॑ पि॒तरः॒ प्रणी॑ती॒ प्रणी॑ती पि॒तरो॑ नो नः पि॒तरः॒ प्रणी॑ती॒ प्रणी॑ती पि॒तरो॑ नः । \newline
26. प्रणी॒तीति॒ प्र - नी॒ती॒ । \newline
27. पि॒तरो॑ नो नः पि॒तरः॑ पि॒तरो॑ न इन्दो इन्दो नः पि॒तरः॑ पि॒तरो॑ न इन्दो । \newline
28. न॒ इ॒न्दो॒ इ॒न्दो॒ नो॒ न॒ इ॒न्दो॒ दे॒वेषु॑ दे॒वेष्वि॑न्दो नो न इन्दो दे॒वेषु॑ । \newline
29. इ॒न्दो॒ दे॒वेषु॑ दे॒वे ष्वि॑न्दो इन्दो दे॒वेषु॒ रत्नꣳ॒॒ रत्न॑म् दे॒वे ष्वि॑न्दो इन्दो दे॒वेषु॒ रत्न᳚म् । \newline
30. इ॒न्दो॒ इती᳚न्दो । \newline
31. दे॒वेषु॒ रत्नꣳ॒॒ रत्न॑म् दे॒वेषु॑ दे॒वेषु॒ रत्न॑ मभजन्ता भजन्त॒ रत्न॑म् दे॒वेषु॑ दे॒वेषु॒ रत्न॑ मभजन्त । \newline
32. रत्न॑ मभजन्ता भजन्त॒ रत्नꣳ॒॒ रत्न॑ मभजन्त॒ धीरा॒ धीरा॑ अभजन्त॒ रत्नꣳ॒॒ रत्न॑ मभजन्त॒ धीराः᳚ । \newline
33. अ॒भ॒ज॒न्त॒ धीरा॒ धीरा॑ अभजन्ता भजन्त॒ धीराः᳚ । \newline
34. धीरा॒ इति॒ धीराः᳚ । \newline
35. त्वया॒ हि हि त्वया॒ त्वया॒ हि नो॑ नो॒ हि त्वया॒ त्वया॒ हि नः॑ । \newline
36. हि नो॑ नो॒ हि हि नः॑ पि॒तरः॑ पि॒तरो॑ नो॒ हि हि नः॑ पि॒तरः॑ । \newline
37. नः॒ पि॒तरः॑ पि॒तरो॑ नो नः पि॒तरः॑ सोम सोम पि॒तरो॑ नो नः पि॒तरः॑ सोम । \newline
38. पि॒तरः॑ सोम सोम पि॒तरः॑ पि॒तरः॑ सोम॒ पूर्वे॒ पूर्वे॑ सोम पि॒तरः॑ पि॒तरः॑ सोम॒ पूर्वे᳚ । \newline
39. सो॒म॒ पूर्वे॒ पूर्वे॑ सोम सोम॒ पूर्वे॒ कर्मा॑णि॒ कर्मा॑णि॒ पूर्वे॑ सोम सोम॒ पूर्वे॒ कर्मा॑णि । \newline
40. पूर्वे॒ कर्मा॑णि॒ कर्मा॑णि॒ पूर्वे॒ पूर्वे॒ कर्मा॑णि च॒क्रु श्च॒क्रुः कर्मा॑णि॒ पूर्वे॒ पूर्वे॒ कर्मा॑णि च॒क्रुः । \newline
41. कर्मा॑णि च॒क्रु श्च॒क्रुः कर्मा॑णि॒ कर्मा॑णि च॒क्रुः प॑वमान पवमान च॒क्रुः कर्मा॑णि॒ कर्मा॑णि च॒क्रुः प॑वमान । \newline
42. च॒क्रुः प॑वमान पवमान च॒क्रु श्च॒क्रुः प॑वमान॒ धीरा॒ धीराः᳚ पवमान च॒क्रु श्च॒क्रुः प॑वमान॒ धीराः᳚ । \newline
43. प॒व॒मा॒न॒ धीरा॒ धीराः᳚ पवमान पवमान॒ धीराः᳚ । \newline
44. धीरा॒ इति॒ धीराः᳚ । \newline
45. व॒न्वन् नवा॒तो ऽवा॑तो व॒न्वन्. व॒न्वन् नवा॑तः परि॒धीन् प॑रि॒धीꣳ रवा॑तो व॒न्वन्. व॒न्वन् नवा॑तः परि॒धीन् । \newline
46. अवा॑तः परि॒धीन् प॑रि॒धीꣳ रवा॒तो ऽवा॑तः परि॒धीꣳ रपाप॑ परि॒धीꣳ रवा॒तो ऽवा॑तः परि॒धीꣳ रप॑ । \newline
47. प॒रि॒धीꣳ रपाप॑ परि॒धीन् प॑रि॒धीꣳ रपो᳚र्णू॒र्ण्वप॑ परि॒धीन् प॑रि॒धीꣳ रपो᳚र्णु । \newline
48. प॒रि॒धीनिति॑ परि - धीन् । \newline
49. अपो᳚ र्णू॒र्ण्वपा पो᳚र्णु वी॒रेभि॑र् वी॒रेभि॑ रू॒र्ण्वपा पो᳚र्णु वी॒रेभिः॑ । \newline
50. ऊ॒र्णु॒ वी॒रेभि॑र् वी॒रेभि॑ रूर्णूर्णु वी॒रेभि॒ रश्वै॒ रश्वै᳚र् वी॒रेभि॑ रूर्णूर्णु वी॒रेभि॒ रश्वैः᳚ । \newline
51. वी॒रेभि॒ रश्वै॒ रश्वै᳚र् वी॒रेभि॑र् वी॒रेभि॒ रश्वै᳚र् म॒घवा॑ म॒घवा ऽश्वै᳚र् वी॒रेभि॑र् वी॒रेभि॒ रश्वै᳚र् म॒घवा᳚ । \newline
52. अश्वै᳚र् म॒घवा॑ म॒घवा ऽश्वै॒ रश्वै᳚र् म॒घवा॑ भव भव म॒घवा ऽश्वै॒ रश्वै᳚र् म॒घवा॑ भव । \newline
53. म॒घवा॑ भव भव म॒घवा॑ म॒घवा॑ भवा नो नो भव म॒घवा॑ म॒घवा॑ भवा नः । \newline
54. म॒घवेति॑ म॒घ - वा॒ । \newline
55. भ॒वा॒ नो॒ नो॒ भ॒व॒ भ॒वा॒ नः॒ । \newline
\pagebreak
\markright{ TS 2.6.12.2  \hfill https://www.vedavms.in \hfill}
\addcontentsline{toc}{section}{ TS 2.6.12.2 }
\section*{ TS 2.6.12.2 }

\textbf{TS 2.6.12.2 } \newline
\textbf{Samhita Paata} \newline

नः ॥ त्वꣳ सो॑म पि॒तृभिः॑ संॅविदा॒नोऽनु॒ द्यावा॑पृथि॒वी आ त॑तन्थ । तस्मै॑ त इन्दो ह॒विषा॑ विधेम व॒यꣳ स्या॑म॒ पत॑यो रयी॒णां ॥ अग्नि॑ष्वात्ताः पितर॒ एह ग॑च्छत॒ सदः॑ सदः सदत सुप्रणीतयः । अ॒त्ता ह॒वीꣳषि॒ प्रय॑तानि ब॒र्॒.हिष्यथा॑ र॒यिꣳ सर्व॑वीरं दधातन ॥ बर्.हि॑षदः पितर ऊ॒त्य॑र्वागि॒मा वो॑ ह॒व्या च॑कृमा जु॒षद्ध्वं᳚ । त आ ग॒ताऽ*व॑सा॒ शं त॑मे॒नाथा॒स्मभ्यꣳ॒॒ - [  ] \newline

\textbf{Pada Paata} \newline

नः॒ ॥ त्वम् । सो॒म॒ । पि॒तृभि॒रिति॑ पि॒तृ - भिः॒ । सं॒ॅवि॒दा॒न इति॑ सं - वि॒दा॒नः । अन्विति॑ । द्यावा॑पृथि॒वी इति॒ द्यावा᳚-पृ॒थि॒वी । एति॑ । त॒त॒न्थ॒ ॥ तस्मै᳚ । ते॒ । इ॒न्दो॒ इति॑ । ह॒विषा᳚ । वि॒धे॒म॒ । व॒यम् । स्या॒म॒ । पत॑यः । र॒यी॒णाम् ॥ अग्नि॑ष्वात्ता॒ इत्यग्नि॑- स्वा॒त्ताः॒ । पि॒त॒रः॒ । एति॑ । इ॒ह । ग॒च्छ॒त॒ । सदः॑ सद॒ इति॒ सदः॑ - स॒दः॒ । स॒द॒त॒ । सु॒प्र॒णी॒त॒य॒ इति॑ सु - प्र॒णी॒त॒यः॒ ॥ अ॒त्त । ह॒वीꣳषि॑ । प्रय॑ता॒नीति॒ प्र - य॒ता॒नि॒ । ब॒र्॒.हिषि॑ । अथ॑ । र॒यिम् । सर्व॑वीर॒मिति॒ सर्व॑ - वी॒र॒म् । द॒धा॒त॒न॒ ॥ बर्.हि॑षद॒ इति॒ बर्.हि॑ - स॒दः॒ । पि॒त॒रः॒ । ऊ॒ती । अ॒र्वाक् । इ॒मा । वः॒ । ह॒व्या । च॒कृ॒म॒ । जु॒षद्ध्व᳚म् ॥ ते । एति॑ । ग॒त॒ । अव॑सा । शंत॑मे॒नेति॒ शं - त॒मे॒न॒ । अथ॑ । अ॒स्मभ्य॒मित्य॒स्म - भ्य॒म् ।  \newline


\textbf{Krama Paata} \newline

न॒ इति॑ नः ॥ त्वꣳ सो॑म । सो॒म॒ पि॒तृभिः॑ । पि॒तृभिः॑ सम्ॅविदा॒नः । पि॒तृभि॒रिति॑ पि॒तृ - भिः॒ । स॒म्ॅवि॒दा॒नोऽनु॑ । स॒म्ॅवि॒दा॒न इति॑ सं - वि॒दा॒नः । अनु॒ द्यावा॑पृथि॒वी । द्यावा॑पृथि॒वी आ । द्यावा॑पृथि॒वी इति॒ द्यावा᳚ - पृ॒थि॒वी । आ त॑तन्थ । त॒त॒न्थेति॑ ततन्थ ॥ तस्मै॑ ते । त॒ इ॒न्दो॒ । इ॒न्दो॒ ह॒विषा᳚ । इ॒न्दो॒ इती᳚न्दो । ह॒विषा॑ विधेम । वि॒धे॒म॒ व॒यम् । व॒यꣳ स्या॑म । स्या॒म॒ पत॑यः । पत॑यो रयी॒णाम् । र॒यी॒णामिति॑ रयी॒णाम् ॥ अग्नि॑ष्वात्ताः पितरः । अग्नि॑ष्वात्ता॒ इत्यग्नि॑ - स्वा॒त्ताः॒ । पि॒त॒र॒ आ । एह । इ॒ह ग॑च्छत । ग॒च्छ॒त॒ सद॑स्सदः । सद॑स्सदः सदत । सद॑स्सद॒ इति॒ सदः॑ - स॒दः॒ । स॒द॒त॒ सु॒प्र॒णी॒त॒यः॒ । सु॒प्र॒णी॒त॒य॒ इति॑ सु - प्र॒णी॒त॒यः॒ ॥ अ॒त्ता ह॒वीꣳषि॑ । ह॒वीꣳषि॒ प्रय॑तानि । प्रय॑तानि ब॒र्.॒हिषि॑ । प्रय॑ता॒नीति॒ प्र - य॒ता॒नि॒ । ब॒र्.॒हिष्यथ॑ । अथा॑ र॒यिम् । र॒यिꣳ सर्व॑वीरम् । सर्व॑वीरम् दधातन । सर्व॑वीर॒मिति॒ सर्व॑ - वी॒र॒म् । द॒धा॒त॒नेति॑ दधातन ॥ बर्.हि॑षदः पितरः । बर्.हि॑षद॒ इति॒ बर्.हि॑ - स॒दः॒ । पि॒त॒र॒ ऊ॒ती । ऊ॒त्य॑र्वाक् । अ॒र्वागि॒मा । इ॒मा वः॑ । वो॒ ह॒व्या । ह॒व्या च॑कृम । च॒कृ॒मा॒ जु॒षद्ध्व᳚म् । जु॒षद्ध्व॒मिति॑ जु॒षद्ध्व᳚म् ॥ त आ । आ ग॑त । ग॒ताव॑सा । अव॑सा॒ शन्त॑मेन । शन्त॑मे॒नाथ॑ । शन्त॑मे॒नेति॒ शम् - त॒मे॒न॒ । अथा॒स्मभ्य᳚म् । अ॒स्मभ्यꣳ॒॒ शम् । अ॒स्मभ्य॒मित्य॒स्म - भ्य॒म् \newline

\textbf{Jatai Paata} \newline

1. न॒ इति॑ नः । \newline
2. त्वꣳ सो॑म सोम॒ त्वम् त्वꣳ सो॑म । \newline
3. सो॒म॒ पि॒तृभिः॑ पि॒तृभिः॑ सोम सोम पि॒तृभिः॑ । \newline
4. पि॒तृभिः॑ संॅविदा॒नः सं॑ॅविदा॒नः पि॒तृभिः॑ पि॒तृभिः॑ संॅविदा॒नः । \newline
5. पि॒तृभि॒रिति॑ पि॒तृ - भिः॒ । \newline
6. सं॒ॅवि॒दा॒नो ऽन्वनु॑ संॅविदा॒नः सं॑ॅविदा॒नो ऽनु॑ । \newline
7. सं॒ॅवि॒दा॒न इति॑ सं - वि॒दा॒नः । \newline
8. अनु॒ द्यावा॑पृथि॒वी द्यावा॑पृथि॒वी अन्वनु॒ द्यावा॑पृथि॒वी । \newline
9. द्यावा॑पृथि॒वी आ द्यावा॑पृथि॒वी द्यावा॑पृथि॒वी आ । \newline
10. द्यावा॑पृथि॒वी इति॒ द्यावा᳚ - पृ॒थि॒वी । \newline
11. आ त॑तन्थ तत॒न्था त॑तन्थ । \newline
12. त॒त॒न्थेति॑ ततन्थ । \newline
13. तस्मै॑ ते ते॒ तस्मै॒ तस्मै॑ ते । \newline
14. त॒ इ॒न्दो॒ इ॒न्दो॒ ते॒ त॒ इ॒न्दो॒ । \newline
15. इ॒न्दो॒ ह॒विषा॑ ह॒विषे᳚न्दो इन्दो ह॒विषा᳚ । \newline
16. इ॒न्दो॒ इती᳚न्दो । \newline
17. ह॒विषा॑ विधेम विधेम ह॒विषा॑ ह॒विषा॑ विधेम । \newline
18. वि॒धे॒म॒ व॒यं ॅव॒यं ॅवि॑धेम विधेम व॒यम् । \newline
19. व॒यꣳ स्या॑म स्याम व॒यं ॅव॒यꣳ स्या॑म । \newline
20. स्या॒म॒ पत॑यः॒ पत॑यः स्याम स्याम॒ पत॑यः । \newline
21. पत॑यो रयी॒णाꣳ र॑यी॒णाम् पत॑यः॒ पत॑यो रयी॒णाम् । \newline
22. र॒यी॒णामिति॑ रयी॒णाम् । \newline
23. अग्नि॑ष्वात्ताः पितरः पित॒रो ऽग्नि॑ष्वात्ता॒ अग्नि॑ष्वात्ताः पितरः । \newline
24. अग्नि॑ष्वात्ता॒ इत्यग्नि॑ - स्वा॒त्ताः॒ । \newline
25. पि॒त॒र॒ आ पि॑तरः पितर॒ आ । \newline
26. एहे हेह । \newline
27. इ॒ह ग॑च्छत गच्छते॒ हे ह ग॑च्छत । \newline
28. ग॒च्छ॒त॒ सदः॑ सदः॒सदः॑ सदो गच्छत गच्छत॒ सदः॑सदः । \newline
29. सदः॑सदः सदत सदत॒ सदः॑सदः॒ सदः॑सदः सदत । \newline
30. सदः॑सद॒ इति॒ सदः॑ - स॒दः॒ । \newline
31. स॒द॒त॒ सु॒प्र॒णी॒त॒यः॒ सु॒प्र॒णी॒त॒यः॒ स॒द॒त॒ स॒द॒त॒ सु॒प्र॒णी॒त॒यः॒ । \newline
32. सु॒प्र॒णी॒त॒य॒ इति॑ सु - प्र॒णी॒त॒यः॒ । \newline
33. अ॒त्ता ह॒वीꣳषि॑ ह॒वीꣳ ष्य॒त्तात्ता ह॒वीꣳषि॑ । \newline
34. ह॒वीꣳषि॒ प्रय॑तानि॒ प्रय॑तानि ह॒वीꣳषि॑ ह॒वीꣳषि॒ प्रय॑तानि । \newline
35. प्रय॑तानि ब॒र्॒.हिषि॑ ब॒र्॒.हिषि॒ प्रय॑तानि॒ प्रय॑तानि ब॒र्॒.हिषि॑ । \newline
36. प्रय॑ता॒नीति॒ प्र - य॒ता॒नि॒ । \newline
37. ब॒र्॒.हि ष्यथाथ॑ ब॒र्॒.हिषि॑ ब॒र्॒.हिष्यथ॑ । \newline
38. अथा॑ र॒यिꣳ र॒यि मथाथा॑ र॒यिम् । \newline
39. र॒यिꣳ सर्व॑वीरꣳ॒॒ सर्व॑वीरꣳ र॒यिꣳ र॒यिꣳ सर्व॑वीरम् । \newline
40. सर्व॑वीरम् दधातन दधातन॒ सर्व॑वीरꣳ॒॒ सर्व॑वीरम् दधातन । \newline
41. सर्व॑वीर॒मिति॒ सर्व॑ - वी॒र॒म् । \newline
42. द॒धा॒त॒नेति॑ दधातन । \newline
43. बर्.हि॑षदः पितरः पितरो॒ बर्.हि॑षदो॒ बर्.हि॑षदः पितरः । \newline
44. बर्.हि॑षद॒ इति॒ बर्.हि॑ - स॒दः॒ । \newline
45. पि॒त॒र॒ ऊ॒त्यू॑ती पि॑तरः पितर ऊ॒ती । \newline
46. ऊ॒ त्य॑र्वा ग॒र्वा गू॒त्यू᳚(उ1॒)त्य॑र्वाक् । \newline
47. अ॒र्वा गि॒मेमा ऽर्वा ग॒र्वा गि॒मा । \newline
48. इ॒मा वो॑ व इ॒मेमा वः॑ । \newline
49. वो॒ ह॒व्या ह॒व्या वो॑ वो ह॒व्या । \newline
50. ह॒व्या च॑कृम चकृम ह॒व्या ह॒व्या च॑कृम । \newline
51. च॒कृ॒मा॒ जु॒षद्ध्व॑म् जु॒षद्ध्व॑म् चकृम चकृमा जु॒षद्ध्व᳚म् । \newline
52. जु॒षद्ध्व॒मिति॑ जु॒षद्ध्व᳚म् । \newline
53. त आ ते त आ । \newline
54. आ ग॑त ग॒ता ग॑त । \newline
55. ग॒ताव॒सा ऽव॑सा गत ग॒ताव॑सा । \newline
56. अव॑सा॒ शन्त॑मेन॒ शन्त॑मे॒नाव॒सा ऽव॑सा॒ शन्त॑मेन । \newline
57. शन्त॑मे॒ना थाथ॒ शन्त॑मेन॒ शन्त॑मे॒ नाथ॑ । \newline
58. शन्त॑मे॒नेति॒ शं - त॒मे॒न॒ । \newline
59. अथा॒स्मभ्य॑ म॒स्मभ्य॒ मथाथा॒ स्मभ्य᳚म् । \newline
60. अ॒स्मभ्यꣳ॒॒ शꣳ श म॒स्मभ्य॑ म॒स्मभ्यꣳ॒॒ शम् । \newline
61. अ॒स्मभ्य॒मित्य॒स्म - भ्य॒म् । \newline

\textbf{Ghana Paata } \newline

1. न॒ इति॑ नः । \newline
2. त्वꣳ सो॑म सोम॒ त्वम् त्वꣳ सो॑म पि॒तृभिः॑ पि॒तृभिः॑ सोम॒ त्वम् त्वꣳ सो॑म पि॒तृभिः॑ । \newline
3. सो॒म॒ पि॒तृभिः॑ पि॒तृभिः॑ सोम सोम पि॒तृभिः॑ संॅविदा॒नः सं॑ॅविदा॒नः पि॒तृभिः॑ सोम सोम पि॒तृभिः॑ संॅविदा॒नः । \newline
4. पि॒तृभिः॑ संॅविदा॒नः सं॑ॅविदा॒नः पि॒तृभिः॑ पि॒तृभिः॑ संॅविदा॒नो ऽन्वनु॑ संॅविदा॒नः पि॒तृभिः॑ पि॒तृभिः॑ संॅविदा॒नो ऽनु॑ । \newline
5. पि॒तृभि॒रिति॑ पि॒तृ - भिः॒ । \newline
6. सं॒ॅवि॒दा॒नो ऽन्वनु॑ संॅविदा॒नः सं॑ॅविदा॒नो ऽनु॒ द्यावा॑पृथि॒वी द्यावा॑पृथि॒वी अनु॑ संॅविदा॒नः सं॑ॅविदा॒नो ऽनु॒ द्यावा॑पृथि॒वी । \newline
7. सं॒ॅवि॒दा॒न इति॑ सं - वि॒दा॒नः । \newline
8. अनु॒ द्यावा॑पृथि॒वी द्यावा॑पृथि॒वी अन्वनु॒ द्यावा॑पृथि॒वी आ द्यावा॑पृथि॒वी अन्वनु॒ द्यावा॑पृथि॒वी आ । \newline
9. द्यावा॑पृथि॒वी आ द्यावा॑पृथि॒वी द्यावा॑पृथि॒वी आ त॑तन्थ तत॒न्था द्यावा॑पृथि॒वी द्यावा॑पृथि॒वी आ त॑तन्थ । \newline
10. द्यावा॑पृथि॒वी इति॒ द्यावा᳚ - पृ॒थि॒वी । \newline
11. आ त॑तन्थ तत॒न्था त॑तन्थ । \newline
12. त॒त॒न्थेति॑ ततन्थ । \newline
13. तस्मै॑ ते ते॒ तस्मै॒ तस्मै॑ त इन्दो इन्दो ते॒ तस्मै॒ तस्मै॑ त इन्दो । \newline
14. त॒ इ॒न्दो॒ इ॒न्दो॒ ते॒ त॒ इ॒न्दो॒ ह॒विषा॑ ह॒विषे᳚न्दो ते त इन्दो ह॒विषा᳚ । \newline
15. इ॒न्दो॒ ह॒विषा॑ ह॒विषे᳚न्दो इन्दो ह॒विषा॑ विधेम विधेम ह॒विषे᳚न्दो इन्दो ह॒विषा॑ विधेम । \newline
16. इ॒न्दो॒ इती᳚न्दो । \newline
17. ह॒विषा॑ विधेम विधेम ह॒विषा॑ ह॒विषा॑ विधेम व॒यं ॅव॒यं ॅवि॑धेम ह॒विषा॑ ह॒विषा॑ विधेम व॒यम् । \newline
18. वि॒धे॒म॒ व॒यं ॅव॒यं ॅवि॑धेम विधेम व॒यꣳ स्या॑म स्याम व॒यं ॅवि॑धेम विधेम व॒यꣳ स्या॑म । \newline
19. व॒यꣳ स्या॑म स्याम व॒यं ॅव॒यꣳ स्या॑म॒ पत॑यः॒ पत॑यः स्याम व॒यं ॅव॒यꣳ स्या॑म॒ पत॑यः । \newline
20. स्या॒म॒ पत॑यः॒ पत॑यः स्याम स्याम॒ पत॑यो रयी॒णाꣳ र॑यी॒णाम् पत॑यः स्याम स्याम॒ पत॑यो रयी॒णाम् । \newline
21. पत॑यो रयी॒णाꣳ र॑यी॒णाम् पत॑यः॒ पत॑यो रयी॒णाम् । \newline
22. र॒यी॒णामिति॑ रयी॒णाम् । \newline
23. अग्नि॑ष्वात्ताः पितरः पित॒रो ऽग्नि॑ष्वात्ता॒ अग्नि॑ष्वात्ताः पितर॒ आ पि॑त॒रो ऽग्नि॑ष्वात्ता॒ अग्नि॑ष्वात्ताः पितर॒ आ । \newline
24. अग्नि॑ष्वात्ता॒ इत्यग्नि॑ - स्वा॒त्ताः॒ । \newline
25. पि॒त॒र॒ आ पि॑तरः पितर॒ एहे हा पि॑तरः पितर॒ एह । \newline
26. एहे हेह ग॑च्छत गच्छते॒ हेह ग॑च्छत । \newline
27. इ॒ह ग॑च्छत गच्छते॒ हे ह ग॑च्छत॒ सदः॑सदः॒ सदः॑सदो गच्छते॒ हे ह ग॑च्छत॒ सदः॑सदः । \newline
28. ग॒च्छ॒त॒ सदः॑सदः॒ सदः॑सदो गच्छत गच्छत॒ सदः॑सदः सदत सदत॒ सदः॑सदो गच्छत गच्छत॒ सदः॑सदः सदत । \newline
29. सदः॑सदः सदत सदत॒ सदः॑सदः॒ सदः॑सदः सदत सुप्रणीतयः सुप्रणीतयः सदत॒ सदः॑सदः॒ सदः॑सदः सदत सुप्रणीतयः । \newline
30. सदः॑सद॒ इति॒ सदः॑ - स॒दः॒ । \newline
31. स॒द॒त॒ सु॒प्र॒णी॒त॒यः॒ सु॒प्र॒णी॒त॒यः॒ स॒द॒त॒ स॒द॒त॒ सु॒प्र॒णी॒त॒यः॒ । \newline
32. सु॒प्र॒णी॒त॒य॒ इति॑ सु - प्र॒णी॒त॒यः॒ । \newline
33. अ॒त्ता ह॒वीꣳषि॑ ह॒वीꣳष्य॒त्तात्ता ह॒वीꣳषि॒ प्रय॑तानि॒ प्रय॑तानि ह॒वीꣳ ष्य॒त्तात्ता ह॒वीꣳषि॒ प्रय॑तानि । \newline
34. ह॒वीꣳषि॒ प्रय॑तानि॒ प्रय॑तानि ह॒वीꣳषि॑ ह॒वीꣳषि॒ प्रय॑तानि ब॒र्॒.हिषि॑ ब॒र्॒.हिषि॒ प्रय॑तानि ह॒वीꣳषि॑ ह॒वीꣳषि॒ प्रय॑तानि ब॒र्॒.हिषि॑ । \newline
35. प्रय॑तानि ब॒र्॒.हिषि॑ ब॒र्॒.हिषि॒ प्रय॑तानि॒ प्रय॑तानि ब॒र्॒.हिष्यथाथ॑ ब॒र्॒.हिषि॒ प्रय॑तानि॒ प्रय॑तानि ब॒र्॒.हिष्यथ॑ । \newline
36. प्रय॑ता॒नीति॒ प्र - य॒ता॒नि॒ । \newline
37. ब॒र्॒.हिष्यथाथ॑ ब॒र्॒.हिषि॑ ब॒र्॒.हिष्यथा॑ र॒यिꣳ र॒यि मथ॑ ब॒र्॒.हिषि॑ ब॒र्॒.हिष्यथा॑ र॒यिम् । \newline
38. अथा॑ र॒यिꣳ र॒यि मथाथा॑ र॒यिꣳ सर्व॑वीरꣳ॒॒ सर्व॑वीरꣳ र॒यि मथाथा॑ र॒यिꣳ सर्व॑वीरम् । \newline
39. र॒यिꣳ सर्व॑वीरꣳ॒॒ सर्व॑वीरꣳ र॒यिꣳ र॒यिꣳ सर्व॑वीरम् दधातन दधातन॒ सर्व॑वीरꣳ र॒यिꣳ र॒यिꣳ सर्व॑वीरम् दधातन । \newline
40. सर्व॑वीरम् दधातन दधातन॒ सर्व॑वीरꣳ॒॒ सर्व॑वीरम् दधातन । \newline
41. सर्व॑वीर॒मिति॒ सर्व॑ - वी॒र॒म् । \newline
42. द॒धा॒त॒नेति॑ दधातन । \newline
43. बर्.हि॑षदः पितरः पितरो॒ बर्.हि॑षदो॒ बर्.हि॑षदः पितर ऊ॒त्यू॑ती पि॑तरो॒ बर्.हि॑षदो॒ बर्.हि॑षदः पितर ऊ॒ती । \newline
44. बर्.हि॑षद॒ इति॒ बर्.हि॑ - स॒दः॒ । \newline
45. पि॒त॒र॒ ऊ॒त्यू॑ती पि॑तरः पितर ऊ॒त्य॑र्वा ग॒र्वा गू॒ती पि॑तरः पितर ऊ॒त्य॑र्वाक् । \newline
46. ऊ॒त्य॑र्वा ग॒र्वागू॒ त्यू᳚(उ1॒)त्य॑र्वा गि॒मेमा ऽर्वागू॒ त्यू᳚(उ1॒)त्य॑र्वा गि॒मा । \newline
47. अ॒र्वा गि॒मेमा ऽर्वाग॒र्वा गि॒मा वो॑ व इ॒मा ऽर्वाग॒र्वा गि॒मा वः॑ । \newline
48. इ॒मा वो॑ व इ॒मेमा वो॑ ह॒व्या ह॒व्या व॑ इ॒मेमा वो॑ ह॒व्या । \newline
49. वो॒ ह॒व्या ह॒व्या वो॑ वो ह॒व्या च॑कृम चकृम ह॒व्या वो॑ वो ह॒व्या च॑कृम । \newline
50. ह॒व्या च॑कृम चकृम ह॒व्या ह॒व्या च॑कृमा जु॒षद्ध्व॑म् जु॒षद्ध्व॑म् चकृम ह॒व्या ह॒व्या च॑कृमा जु॒षद्ध्व᳚म् । \newline
51. च॒कृ॒मा॒ जु॒षद्ध्व॑म् जु॒षद्ध्व॑म् चकृम चकृमा जु॒षद्ध्व᳚म् । \newline
52. जु॒षद्ध्व॒मिति॑ जु॒षद्ध्व᳚म् । \newline
53. त आ ते त आ ग॑त ग॒ता ते त आ ग॑त । \newline
54. आ ग॑त ग॒ता ग॒ताव॒सा ऽव॑सा ग॒ता ग॒ताव॑सा । \newline
55. ग॒ताव॒सा ऽव॑सा गत ग॒ताव॑सा॒ शन्त॑मेन॒ शन्त॑मे॒ना व॑सा गत ग॒ताव॑सा॒ शन्त॑मेन । \newline
56. अव॑सा॒ शन्त॑मेन॒ शन्त॑मे॒ना व॒सा ऽव॑सा॒ शन्त॑मे॒नाथाथ॒ शन्त॑मे॒ना व॒सा ऽव॑सा॒ शन्त॑मे॒नाथ॑ । \newline
57. शन्त॑मे॒नाथाथ॒ शन्त॑मेन॒ शन्त॑मे॒ना था॒स्मभ्य॑ म॒स्मभ्य॒ मथ॒ शन्त॑मेन॒ शन्त॑मे॒ना था॒स्मभ्य᳚म् । \newline
58. शन्त॑मे॒नेति॒ शं - त॒मे॒न॒ । \newline
59. अथा॒स्मभ्य॑ म॒स्मभ्य॒ मथाथा॒ स्मभ्यꣳ॒॒ शꣳ श म॒स्मभ्य॒ मथाथा॒ स्मभ्यꣳ॒॒ शम् । \newline
60. अ॒स्मभ्यꣳ॒॒ शꣳ श म॒स्मभ्य॑ म॒स्मभ्यꣳ॒॒ शं ॅयोर् योः श म॒स्मभ्य॑ म॒स्मभ्यꣳ॒॒ शं ॅयोः । \newline
61. अ॒स्मभ्य॒मित्य॒स्म - भ्य॒म् । \newline
\pagebreak
\markright{ TS 2.6.12.3  \hfill https://www.vedavms.in \hfill}
\addcontentsline{toc}{section}{ TS 2.6.12.3 }
\section*{ TS 2.6.12.3 }

\textbf{TS 2.6.12.3 } \newline
\textbf{Samhita Paata} \newline

शं ॅयोर॑र॒पो द॑धात ॥ आऽहं पि॒तॄन्थ् सु॑वि॒दत्राꣳ॑ अविथ्सि॒ नपा॑तं च वि॒क्रम॑णं च॒ विष्णोः᳚ । ब॒र्॒.हि॒षदो॒ ये स्व॒धया॑ सु॒तस्य॒ भज॑न्त पि॒त्वस्त इ॒हाऽऽ ग॑मिष्ठाः ॥ उप॑हूताः पि॒तरः॑ सो॒म्यासो॑ बर्.हि॒ष्ये॑षु नि॒धिषु॑ प्रि॒येषु॑ । त आ ग॑मन्तु॒ त इ॒ह श्रु॑व॒न्त्वधि॑ ब्रुवन्तु॒ ते अ॑वन्त्व॒स्मान् ॥ उदी॑रता॒मव॑र॒ उत् परा॑स॒ उन्म॑द्ध्य॒माः पि॒तरः॑ सो॒म्यासः॑ । असुं॒ - [  ] \newline

\textbf{Pada Paata} \newline

शम् । योः । अ॒र॒पः । द॒धा॒त॒ ॥ एति॑ । अ॒हम् । पि॒तॄन् । सु॒वि॒दत्रा॒निति॑ सु - वि॒दत्रान्॑ । अ॒वि॒थ्सि॒ । नपा॑तम् । च॒ । वि॒क्रम॑ण॒मिति॑ वि - क्रम॑णम् । च॒ । विष्णोः᳚ ॥ ब॒र्॒.हि॒षद॒ इति॑ बर्.हि - सदः॑ । ये । स्व॒धयेति॑ स्व - धया᳚ । सु॒तस्य॑ । भज॑न्त । पि॒त्वः । ते । इ॒ह । आग॑मिष्ठा॒ इत्या - ग॒मि॒ष्ठाः॒ ॥ उप॑हूता॒ इत्युप॑ - हू॒ताः॒ । पि॒तरः॑ । सो॒म्यासः॑ । ब॒र्॒.हि॒ष्ये॑षु । नि॒धिष्विति॑ नि-धिषु॑ । प्रि॒येषु॑ ॥ ते । एति॑ । ग॒म॒न्तु॒ । ते । इ॒ह । श्रु॒व॒न्तु॒ । अधीति॑ । ब्रु॒व॒न्तु॒ । ते । अ॒व॒न्तु॒ । अ॒स्मान् ॥ उदिति॑ । ई॒र॒ता॒म् । अव॑रे । उदिति॑ । परा॑सः । उदिति॑ । म॒द्ध्य॒माः । पि॒तरः॑ । सो॒म्यासः॑ ॥ असु᳚म् ।  \newline


\textbf{Krama Paata} \newline

शं ॅयोः । योर॑र॒पः । अ॒र॒पो द॑धात । द॒धा॒तेति॑ दधात ॥ आऽहम् । अ॒हम् पि॒तॄन् । पि॒तॄन्थ् सु॑वि॒दत्रान्॑ । सु॒वि॒दत्राꣳ॑ अविथ्सि । सु॒वि॒दत्रा॒निति॑ सु - वि॒दत्रान्॑ । अ॒वि॒थ्सि॒ नपा॑तम् । नपा॑तम् च । च॒ वि॒क्रम॑णम् । वि॒क्रम॑णम् च । वि॒क्रम॑ण॒मिति॑ वि - क्रम॑णम् । च॒ विष्णोः᳚ । विष्णो॒रिति॒ विष्णोः᳚ ॥ ब॒र्॒.हि॒षदो॒ ये । ब॒र्॒.हि॒षद॒ इति॑ बर्.हि - सदः॑ । ये स्व॒धया᳚ । स्व॒धया॑ सु॒तस्य॑ । स्व॒धयेति॑ स्व - धया᳚ । सु॒तस्य॒ भज॑न्त । भज॑न्त पि॒त्वः । पि॒त्वस्ते । त इ॒ह । इ॒हाग॑मिष्ठाः । आग॑मिष्ठा॒ इत्या - ग॒मि॒ष्ठाः॒ ॥ उप॑हूताः पि॒तरः॑ । उप॑हूता॒ इत्युप॑ - हू॒ताः॒ । पि॒तरः॑ सो॒म्यासः॑ । सो॒म्यासो॑ ब॒र्.॒हि॒ष्ये॑षु । बर्.हि॒ष्ये॑षु नि॒धिषु॑ । नि॒धिषु॑ प्रि॒येषु॑ । नि॒धिष्विति॑ नि - धिषु॑ । प्रि॒येष्विति॑ प्रि॒येषु॑ ॥ त आ । आ ग॑मन्तु । ग॒म॒न्तु॒ ते । त इ॒ह । इ॒ह श्रु॑वन्तु । श्रु॒व॒न्त्वधि॑ । अधि॑ ब्रुवन्तु । ब्रु॒व॒न्तु॒ ते । ते अ॑वन्तु । अ॒व॒न्त्व॒स्मान् । अ॒स्मानित्य॒स्मान् ॥ उदी॑रताम् । ई॒र॒ता॒मव॑रे । अव॑र॒ उत् । उत् परा॑सः । परा॑स॒ उत् । उन् म॑द्ध्य॒माः । म॒द्ध्य॒माः पि॒तरः॑ । पि॒तरः॑ सो॒म्यासः॑ । सो॒म्यास॒ इति॑ सो॒म्यासः॑ ॥ असुं॒ ॅये \newline

\textbf{Jatai Paata} \newline

1. शं ॅयोर् योः शꣳ शं ॅयोः । \newline
2. योर॑र॒पो अ॑र॒पो योर् योर॑र॒पः । \newline
3. अ॒र॒पो द॑धात दधाता र॒पो अ॑र॒पो द॑धात । \newline
4. द॒धा॒तेति॑ दधात । \newline
5. आ ऽह म॒ह मा ऽहम् । \newline
6. अ॒हम् पि॒तॄन् पि॒तॄ न॒ह म॒हम् पि॒तॄन् । \newline
7. पि॒तॄन् थ्सु॑वि॒दत्रा᳚न् थ्सुवि॒दत्रा᳚न् पि॒तॄन् पि॒तॄन् थ्सु॑वि॒दत्रान्॑ । \newline
8. सु॒वि॒दत्राꣳ॑ अवि थ्स्यविथ्सि सुवि॒दत्रा᳚न् थ्सुवि॒दत्राꣳ॑ अविथ्सि । \newline
9. सु॒वि॒दत्रा॒निति॑ सु - वि॒दत्रान्॑ । \newline
10. अ॒वि॒थ्सि॒ नपा॑त॒म् नपा॑त मविथ् स्यविथ्सि॒ नपा॑तम् । \newline
11. नपा॑तम् च च॒ नपा॑त॒म् नपा॑तम् च । \newline
12. च॒ वि॒क्रम॑णं ॅवि॒क्रम॑णम् च च वि॒क्रम॑णम् । \newline
13. वि॒क्रम॑णम् च च वि॒क्रम॑णं ॅवि॒क्रम॑णम् च । \newline
14. वि॒क्रम॑ण॒मिति॑ वि - क्रम॑णम् । \newline
15. च॒ विष्णो॒र् विष्णो᳚श्च च॒ विष्णोः᳚ । \newline
16. विष्णो॒रिति॒ विष्णोः᳚ । \newline
17. ब॒र्॒.हि॒षदो॒ ये ये ब॑र्.हि॒षदो॑ बर्.हि॒षदो॒ ये । \newline
18. ब॒र्॒.हि॒षद॒ इति॑ बर्.हि - सदः॑ । \newline
19. ये स्व॒धया᳚ स्व॒धया॒ ये ये स्व॒धया᳚ । \newline
20. स्व॒धया॑ सु॒तस्य॑ सु॒तस्य॑ स्व॒धया᳚ स्व॒धया॑ सु॒तस्य॑ । \newline
21. स्व॒धयेति॑ स्व - धया᳚ । \newline
22. सु॒तस्य॒ भज॑न्त॒ भज॑न्त सु॒तस्य॑ सु॒तस्य॒ भज॑न्त । \newline
23. भज॑न्त पि॒त्वः पि॒त्वो भज॑न्त॒ भज॑न्त पि॒त्वः । \newline
24. पि॒त्व स्ते ते पि॒त्वः पि॒त्व स्ते । \newline
25. त इ॒हे ह ते त इ॒ह । \newline
26. इ॒हाग॑मिष्ठा॒ आग॑मिष्ठा इ॒हे हाग॑मिष्ठाः । \newline
27. आग॑मिष्ठा॒ इत्या - ग॒मि॒ष्ठाः॒ । \newline
28. उप॑हूताः पि॒तरः॑ पि॒तर॒ उप॑हूता॒ उप॑हूताः पि॒तरः॑ । \newline
29. उप॑हूता॒ इत्युप॑ - हू॒ताः॒ । \newline
30. पि॒तरः॑ सो॒म्यासः॑ सो॒म्यासः॑ पि॒तरः॑ पि॒तरः॑ सो॒म्यासः॑ । \newline
31. सो॒म्यासो॑ बर्.हि॒ष्ये॑षु बर्.हि॒ष्ये॑षु सो॒म्यासः॑ सो॒म्यासो॑ बर्.हि॒ष्ये॑षु । \newline
32. ब॒र्॒.हि॒ष्ये॑षु नि॒धिषु॑ नि॒धिषु॑ बर्.हि॒ष्ये॑षु बर्.हि॒ष्ये॑षु नि॒धिषु॑ । \newline
33. नि॒धिषु॑ प्रि॒येषु॑ प्रि॒येषु॑ नि॒धिषु॑ नि॒धिषु॑ प्रि॒येषु॑ । \newline
34. नि॒धिष्विति॑ नि - धिषु॑ । \newline
35. प्रि॒येष्विति॑ प्रि॒येषु॑ । \newline
36. त आ ते त आ । \newline
37. आ ग॑मन्तु गम॒न्त्वा ग॑मन्तु । \newline
38. ग॒म॒न्तु॒ ते ते ग॑मन्तु गमन्तु॒ ते । \newline
39. त इ॒हे ह ते त इ॒ह । \newline
40. इ॒ह श्रु॑वन्तु श्रुव न्त्वि॒हे ह श्रु॑वन्तु । \newline
41. श्रु॒व॒ न्त्वध्यधि॑ श्रुवन्तु श्रुव॒ न्त्वधि॑ । \newline
42. अधि॑ ब्रुवन्तु ब्रुव॒ न्त्वध्यधि॑ ब्रुवन्तु । \newline
43. ब्रु॒व॒न्तु॒ ते ते ब्रु॑वन्तु ब्रुवन्तु॒ ते । \newline
44. ते अ॑व न्त्ववन्तु॒ ते ते अ॑वन्तु । \newline
45. अ॒व॒ न्त्व॒स्मा न॒स्मा न॑व न्त्वव न्त्व॒स्मान् । \newline
46. अ॒स्मानित्य॒स्मान् । \newline
47. उदी॑रता मीरता॒ मुदु दी॑रताम् । \newline
48. ई॒र॒ता॒ मव॒रे ऽव॑र ईरता मीरता॒ मव॑रे । \newline
49. अव॑र॒ उदु दव॒रे ऽव॑र॒ उत् । \newline
50. उत् परा॑सः॒ परा॑स॒ उदुत् परा॑सः । \newline
51. परा॑स॒ उदुत् परा॑सः॒ परा॑स॒ उत् । \newline
52. उन् म॑द्ध्य॒मा म॑द्ध्य॒मा उदुन् म॑द्ध्य॒माः । \newline
53. म॒द्ध्य॒माः पि॒तरः॑ पि॒तरो॑ मद्ध्य॒मा म॑द्ध्य॒माः पि॒तरः॑ । \newline
54. पि॒तरः॑ सो॒म्यासः॑ सो॒म्यासः॑ पि॒तरः॑ पि॒तरः॑ सो॒म्यासः॑ । \newline
55. सो॒म्यास॒ इति॑ सो॒म्यासः॑ । \newline
56. असुं॒ ॅये ये असु॒ मसुं॒ ॅये । \newline

\textbf{Ghana Paata } \newline

1. शं ॅयोर् योः शꣳ शं ॅयोर॑र॒पो अ॑र॒पो योः शꣳ शं ॅयोर॑र॒पः । \newline
2. योर॑र॒पो अ॑र॒पो योर् योर॑र॒पो द॑धात दधातार॒पो योर् योर॑र॒पो द॑धात । \newline
3. अ॒र॒पो द॑धात दधातार॒पो अ॑र॒पो द॑धात । \newline
4. द॒धा॒तेति॑ दधात । \newline
5. आ ऽह म॒ह मा ऽहम् पि॒तॄन् पि॒तॄ न॒ह मा ऽहम् पि॒तॄन् । \newline
6. अ॒हम् पि॒तॄन् पि॒तॄ न॒ह म॒हम् पि॒तॄन् थ्सु॑वि॒दत्रा᳚न् थ्सुवि॒दत्रा᳚न् पि॒तॄ न॒ह म॒हम् पि॒तॄन् थ्सु॑वि॒दत्रान्॑ । \newline
7. पि॒तॄन् थ्सु॑वि॒दत्रा᳚न् थ्सुवि॒दत्रा᳚न् पि॒तॄन् पि॒तॄन् थ्सु॑वि॒दत्राꣳ॑ अविथ् स्यविथ्सि सुवि॒दत्रा᳚न् पि॒तॄन् पि॒तॄन् थ्सु॑वि॒दत्राꣳ॑ अविथ्सि । \newline
8. सु॒वि॒दत्राꣳ॑ अविथ् स्यविथ्सि सुवि॒दत्रा᳚न् थ्सुवि॒दत्राꣳ॑ अविथ्सि॒ नपा॑त॒म् नपा॑त मविथ्सि सुवि॒दत्रा᳚न् थ्सुवि॒दत्राꣳ॑ अविथ्सि॒ नपा॑तम् । \newline
9. सु॒वि॒दत्रा॒निति॑ सु - वि॒दत्रान्॑ । \newline
10. अ॒वि॒थ्सि॒ नपा॑त॒म् नपा॑त मवि थ्स्यविथ्सि॒ नपा॑तम् च च॒ नपा॑त मवि थ्स्यविथ्सि॒ नपा॑तम् च । \newline
11. नपा॑तम् च च॒ नपा॑त॒म् नपा॑तम् च वि॒क्रम॑णं ॅवि॒क्रम॑णम् च॒ नपा॑त॒म् नपा॑तम् च वि॒क्रम॑णम् । \newline
12. च॒ वि॒क्रम॑णं ॅवि॒क्रम॑णम् च च वि॒क्रम॑णम् च च वि॒क्रम॑णम् च च वि॒क्रम॑णम् च । \newline
13. वि॒क्रम॑णम् च च वि॒क्रम॑णं ॅवि॒क्रम॑णम् च॒ विष्णो॒र् विष्णो᳚श्च वि॒क्रम॑णं ॅवि॒क्रम॑णम् च॒ विष्णोः᳚ । \newline
14. वि॒क्रम॑ण॒मिति॑ वि - क्रम॑णम् । \newline
15. च॒ विष्णो॒र् विष्णो᳚श्च च॒ विष्णोः᳚ । \newline
16. विष्णो॒रिति॒ विष्णोः᳚ । \newline
17. ब॒र्॒.हि॒षदो॒ ये ये ब॑र्.हि॒षदो॑ बर्.हि॒षदो॒ ये स्व॒धया᳚ स्व॒धया॒ ये ब॑र्.हि॒षदो॑ बर्.हि॒षदो॒ ये स्व॒धया᳚ । \newline
18. ब॒र्॒.हि॒षद॒ इति॑ बर्.हि - सदः॑ । \newline
19. ये स्व॒धया᳚ स्व॒धया॒ ये ये स्व॒धया॑ सु॒तस्य॑ सु॒तस्य॑ स्व॒धया॒ ये ये स्व॒धया॑ सु॒तस्य॑ । \newline
20. स्व॒धया॑ सु॒तस्य॑ सु॒तस्य॑ स्व॒धया᳚ स्व॒धया॑ सु॒तस्य॒ भज॑न्त॒ भज॑न्त सु॒तस्य॑ स्व॒धया᳚ स्व॒धया॑ सु॒तस्य॒ भज॑न्त । \newline
21. स्व॒धयेति॑ स्व - धया᳚ । \newline
22. सु॒तस्य॒ भज॑न्त॒ भज॑न्त सु॒तस्य॑ सु॒तस्य॒ भज॑न्त पि॒त्वः पि॒त्वो भज॑न्त सु॒तस्य॑ सु॒तस्य॒ भज॑न्त पि॒त्वः । \newline
23. भज॑न्त पि॒त्वः पि॒त्वो भज॑न्त॒ भज॑न्त पि॒त्व स्ते ते पि॒त्वो भज॑न्त॒ भज॑न्त पि॒त्व स्ते । \newline
24. पि॒त्व स्ते ते पि॒त्वः पि॒त्व स्त इ॒हे ह ते पि॒त्वः पि॒त्व स्त इ॒ह । \newline
25. त इ॒हे ह ते त इ॒हाग॑मिष्ठा॒ आग॑मिष्ठा इ॒ह ते त इ॒हाग॑मिष्ठाः । \newline
26. इ॒हाग॑मिष्ठा॒ आग॑मिष्ठा इ॒हे हाग॑मिष्ठाः । \newline
27. आग॑मिष्ठा॒ इत्या - ग॒मि॒ष्ठाः॒ । \newline
28. उप॑हूताः पि॒तरः॑ पि॒तर॒ उप॑हूता॒ उप॑हूताः पि॒तरः॑ सो॒म्यासः॑ सो॒म्यासः॑ पि॒तर॒ उप॑हूता॒ उप॑हूताः पि॒तरः॑ सो॒म्यासः॑ । \newline
29. उप॑हूता॒ इत्युप॑ - हू॒ताः॒ । \newline
30. पि॒तरः॑ सो॒म्यासः॑ सो॒म्यासः॑ पि॒तरः॑ पि॒तरः॑ सो॒म्यासो॑ बर्.हि॒ष्ये॑षु बर्.हि॒ष्ये॑षु सो॒म्यासः॑ पि॒तरः॑ पि॒तरः॑ सो॒म्यासो॑ बर्.हि॒ष्ये॑षु । \newline
31. सो॒म्यासो॑ बर्.हि॒ष्ये॑षु बर्.हि॒ष्ये॑षु सो॒म्यासः॑ सो॒म्यासो॑ बर्.हि॒ष्ये॑षु नि॒धिषु॑ नि॒धिषु॑ बर्.हि॒ष्ये॑षु सो॒म्यासः॑ सो॒म्यासो॑ बर्.हि॒ष्ये॑षु नि॒धिषु॑ । \newline
32. ब॒र्॒.हि॒ष्ये॑षु नि॒धिषु॑ नि॒धिषु॑ बर्.हि॒ष्ये॑षु बर्.हि॒ष्ये॑षु नि॒धिषु॑ प्रि॒येषु॑ प्रि॒येषु॑ नि॒धिषु॑ बर्.हि॒ष्ये॑षु बर्.हि॒ष्ये॑षु नि॒धिषु॑ प्रि॒येषु॑ । \newline
33. नि॒धिषु॑ प्रि॒येषु॑ प्रि॒येषु॑ नि॒धिषु॑ नि॒धिषु॑ प्रि॒येषु॑ । \newline
34. नि॒धिष्विति॑ नि - धिषु॑ । \newline
35. प्रि॒येष्विति॑ प्रि॒येषु॑ । \newline
36. त आ ते त आ ग॑मन्तु गम॒न्त्वा ते त आ ग॑मन्तु । \newline
37. आ ग॑मन्तु गम॒न्त्वा ग॑मन्तु॒ ते ते ग॑म॒न्त्वा ग॑मन्तु॒ ते । \newline
38. ग॒म॒न्तु॒ ते ते ग॑मन्तु गमन्तु॒ त इ॒हे ह ते ग॑मन्तु गमन्तु॒ त इ॒ह । \newline
39. त इ॒हे ह ते त इ॒ह श्रु॑वन्तु श्रुव न्त्वि॒ह ते त इ॒ह श्रु॑वन्तु । \newline
40. इ॒ह श्रु॑वन्तु श्रुव न्त्वि॒हे ह श्रु॑व॒ न्त्वध्यधि॑ श्रुवन्त्वि॒हे ह श्रु॑व॒ न्त्वधि॑ । \newline
41. श्रु॒व॒ न्त्वध्यधि॑ श्रुवन्तु श्रुव॒ न्त्वधि॑ ब्रुवन्तु ब्रुव॒ न्त्वधि॑ श्रुवन्तु श्रुव॒ न्त्वधि॑ ब्रुवन्तु । \newline
42. अधि॑ ब्रुवन्तु ब्रुव॒ न्त्वध्यधि॑ ब्रुवन्तु॒ ते ते ब्रु॑व॒ न्त्वध्यधि॑ ब्रुवन्तु॒ ते । \newline
43. ब्रु॒व॒न्तु॒ ते ते ब्रु॑वन्तु ब्रुवन्तु॒ ते अ॑व न्त्ववन्तु॒ ते ब्रु॑वन्तु ब्रुवन्तु॒ ते अ॑वन्तु । \newline
44. ते अ॑व न्त्ववन्तु॒ ते ते अ॑व न्त्व॒स्मा न॒स्मा न॑वन्तु॒ ते ते अ॑व न्त्व॒स्मान् । \newline
45. अ॒व॒ न्त्व॒स्मा न॒स्मा न॑व न्त्वव न्त्व॒स्मान् । \newline
46. अ॒स्मानित्य॒स्मान् । \newline
47. उदी॑रता मीरता॒ मुदु दी॑रता॒ मव॒रे ऽव॑र ईरता॒ मुदु दी॑रता॒ मव॑रे । \newline
48. ई॒र॒ता॒ मव॒रे ऽव॑र ईरता मीरता॒ मव॑र॒ उदुदव॑र ईरता मीरता॒ मव॑र॒ उत् । \newline
49. अव॑र॒ उदुदव॒रे ऽव॑र॒ उत् परा॑सः॒ परा॑स॒ उदव॒रे ऽव॑र॒ उत् परा॑सः । \newline
50. उत् परा॑सः॒ परा॑स॒ उदुत् परा॑स॒ उदुत् परा॑स॒ उदुत् परा॑स॒ उत् । \newline
51. परा॑स॒ उदुत् परा॑सः॒ परा॑स॒ उन् म॑द्ध्य॒मा म॑द्ध्य॒मा उत् परा॑सः॒ परा॑स॒ उन् म॑द्ध्य॒माः । \newline
52. उन् म॑द्ध्य॒मा म॑द्ध्य॒मा उदुन् म॑द्ध्य॒माः पि॒तरः॑ पि॒तरो॑ मद्ध्य॒मा उदुन् म॑द्ध्य॒माः पि॒तरः॑ । \newline
53. म॒द्ध्य॒माः पि॒तरः॑ पि॒तरो॑ मद्ध्य॒मा म॑द्ध्य॒माः पि॒तरः॑ सो॒म्यासः॑ सो॒म्यासः॑ पि॒तरो॑ मद्ध्य॒मा म॑द्ध्य॒माः पि॒तरः॑ सो॒म्यासः॑ । \newline
54. पि॒तरः॑ सो॒म्यासः॑ सो॒म्यासः॑ पि॒तरः॑ पि॒तरः॑ सो॒म्यासः॑ । \newline
55. सो॒म्यास॒ इति॑ सो॒म्यासः॑ । \newline
56. असुं॒ ॅये ये असु॒ मसुं॒ ॅय ई॒युरी॒युर् ये असु॒ मसुं॒ ॅय ई॒युः । \newline
\pagebreak
\markright{ TS 2.6.12.4  \hfill https://www.vedavms.in \hfill}
\addcontentsline{toc}{section}{ TS 2.6.12.4 }
\section*{ TS 2.6.12.4 }

\textbf{TS 2.6.12.4 } \newline
\textbf{Samhita Paata} \newline

ॅय ई॒युर॑ वृ॒का ऋ॑त॒ज्ञास्ते नो॑ऽवन्तु पि॒तरो॒ हवे॑षु ॥ इ॒दं पि॒तृभ्यो॒ नमो॑ अस्त्व॒द्य ये पूर्वा॑सो॒ य उप॑रास ई॒युः । ये पार्थि॑वे॒ रज॒स्या निष॑त्ता॒ ये वा॑ नू॒नꣳ सु॑वृ॒जना॑सु वि॒क्षु ॥ अधा॒ यथा॑ नः पि॒तरः॒ परा॑सः प्र॒त्नासो॑ अग्न ऋ॒तमा॑शुषा॒णाः । शुचीद॑य॒न् दीधि॑ति मुक्थ॒शासः॒ क्षामा॑ भि॒न्दन्तो॑ अरु॒णीरप॑ व्रन्न् ॥ यद॑ग्ने - [  ] \newline

\textbf{Pada Paata} \newline

ये । ई॒युः । अ॒वृ॒काः । ऋ॒त॒ज्ञा इत्यृ॑त - ज्ञाः । ते । नः॒ । अ॒व॒न्तु॒ । पि॒तरः॑ । हवे॑षु ॥ इ॒दम् । पि॒तृभ्य॒ इति॑ पि॒तृ - भ्यः॒ । नमः॑ । अ॒स्तु॒ । अ॒द्य । ये । पूर्वा॑सः । ये । उप॑रासः । ई॒युः ॥ ये । पार्थि॑वे । रज॑सि । एति॑ । निष॑त्ता॒ इति॒ नि - स॒त्ताः॒ । ये । वा॒ । नू॒नम् । सु॒वृ॒जना॒स्विति॑ सु-वृ॒जना॑सु । वि॒क्षु ॥ अध॑ । यथा᳚ । नः॒ । पि॒तरः॑ । परा॑सः । प्र॒त्नासः॑ । अ॒ग्ने॒ । ऋ॒तम् । आ॒शु॒षा॒णाः ॥ शुचि॑ । इत् । अ॒य॒न्न् । दीधि॑तिम् । उ॒क्थ॒शास॒ इत्यु॑क्थ - शासः॑ । क्षाम॑ । भि॒न्दन्तः॑ । अ॒रु॒णीः । अपेति॑ । व्र॒न्न् ॥ यत् । अ॒ग्ने॒ ।  \newline


\textbf{Krama Paata} \newline

य ई॒युः । ई॒युर॑वृ॒काः । अ॒वृ॒का ऋ॑त॒ज्ञाः । ऋ॒त॒ज्ञास्ते । ऋ॒त॒ज्ञा इत्यृ॑त - ज्ञाः । ते नः॑ । नो॒ ऽव॒न्तु॒ । अ॒व॒न्तु॒ पि॒तरः॑ । पि॒तरो॒ हवे॑षु । हवे॒ष्विति॒ हवे॑षु ॥ इ॒दम् पि॒तृभ्यः॑ । पि॒तृभ्यो॒ नमः॑ । पि॒तृभ्य॒ इति॑ पि॒तृ - भ्यः॒ । नमो॑ अस्तु । अस्त्व॒द्य । अ॒द्य ये । ये पूर्वा॑सः । पूर्वा॑सो॒ ये । य उप॑रासः । उप॑रास ई॒युः । ई॒युरिती॒युः ॥ ये पार्त्थि॑वे । पार्त्थि॑वे॒ रज॑सि । रज॒स्या । आ निष॑त्ताः । निष॑त्ता॒ ये । निष॑त्ता॒ इति॒ नि - स॒त्ताः॒ । ये वा᳚ । वा॒ नू॒नम् । नू॒नꣳ सु॑वृ॒जना॑सु । सु॒वृ॒जना॑सु वि॒क्षु । सु॒वृ॒जना॒स्विति॑ सु - वृ॒जना॑सु । वि॒क्ष्विति॑ वि॒क्षु ॥ अधा॒ यथा᳚ । यथा॑ नः । नः॒ पि॒तरः॑ । पि॒तरः॒ परा॑सः । परा॑सः प्र॒त्नासः॑ । प्र॒त्नासो॑ अग्ने । अ॒ग्न॒ ऋ॒तम् । ऋ॒तमा॑शुषा॒णाः । आ॒शु॒षा॒णा इत्या॑शुषा॒णाः ॥ शुचीत् । इद॑यन्न् । अ॒य॒न् दीधि॑तिम् । दीधि॑तिमुक्थ॒शासः॑ । उ॒क्थ॒शासः॒ क्षाम॑ । उ॒क्थ॒शास॒ इत्यु॑क्थ - शासः॑ । क्षामा॑ भि॒न्दन्तः॑ । भि॒न्दन्तो॑ अरु॒णीः । अ॒रु॒णीरप॑ । अप॑ व्रन्न् । व्र॒न्निति॑ व्रन्न् ॥ यद॑ग्ने । अ॒ग्ने॒ क॒व्य॒वा॒ह॒न॒ \newline

\textbf{Jatai Paata} \newline

1. य ई॒यु री॒युर् ये य ई॒युः । \newline
2. ई॒यु र॑वृ॒का अ॑वृ॒का ई॒यु री॒यु र॑वृ॒काः । \newline
3. अ॒वृ॒का ऋ॑त॒ज्ञा ऋ॑त॒ज्ञा अ॑वृ॒का अ॑वृ॒का ऋ॑त॒ज्ञाः । \newline
4. ऋ॒त॒ज्ञा स्ते त ऋ॑त॒ज्ञा ऋ॑त॒ज्ञा स्ते । \newline
5. ऋ॒त॒ज्ञा इत्यृ॑त - ज्ञाः । \newline
6. ते नो॑ न॒ स्ते ते नः॑ । \newline
7. नो॒ ऽव॒ न्त्व॒व॒न्तु॒ नो॒ नो॒ ऽव॒न्तु॒ । \newline
8. अ॒व॒न्तु॒ पि॒तरः॑ पि॒तरो॑ ऽव न्त्ववन्तु पि॒तरः॑ । \newline
9. पि॒तरो॒ हवे॑षु॒ हवे॑षु पि॒तरः॑ पि॒तरो॒ हवे॑षु । \newline
10. हवे॒ष्विति॒ हवे॑षु । \newline
11. इ॒दम् पि॒तृभ्यः॑ पि॒तृभ्य॑ इ॒द मि॒दम् पि॒तृभ्यः॑ । \newline
12. पि॒तृभ्यो॒ नमो॒ नमः॑ पि॒तृभ्यः॑ पि॒तृभ्यो॒ नमः॑ । \newline
13. पि॒तृभ्य॒ इति॑ पि॒तृ - भ्यः॒ । \newline
14. नमो॑ अस्त्वस्तु॒ नमो॒ नमो॑ अस्तु । \newline
15. अ॒स्त्व॒द्या द्या स्त्व॑ स्त्व॒द्य । \newline
16. अ॒द्य ये ये अ॒द्याद्य ये । \newline
17. ये पूर्वा॑सः॒ पूर्वा॑सो॒ ये ये पूर्वा॑सः । \newline
18. पूर्वा॑सो॒ ये ये पूर्वा॑सः॒ पूर्वा॑सो॒ ये । \newline
19. य उप॑रास॒ उप॑रासो॒ ये य उप॑रासः । \newline
20. उप॑रास ई॒यु री॒यु रुप॑रास॒ उप॑रास ई॒युः । \newline
21. ई॒युरिती॒युः । \newline
22. ये पार्थि॑वे॒ पार्थि॑वे॒ ये ये पार्थि॑वे । \newline
23. पार्थि॑वे॒ रज॑सि॒ रज॑सि॒ पार्थि॑वे॒ पार्थि॑वे॒ रज॑सि । \newline
24. रज॒स्या रज॑सि॒ रज॒स्या । \newline
25. आ निष॑त्ता॒ निष॑त्ता॒ आ निष॑त्ताः । \newline
26. निष॑त्ता॒ ये ये निष॑त्ता॒ निष॑त्ता॒ ये । \newline
27. निष॑त्ता॒ इति॒ नि - स॒त्ताः॒ । \newline
28. ये वा॑ वा॒ ये ये वा᳚ । \newline
29. वा॒ नू॒नम् नू॒नं ॅवा॑ वा नू॒नम् । \newline
30. नू॒नꣳ सु॑वृ॒जना॑सु सुवृ॒जना॑सु नू॒नम् नू॒नꣳ सु॑वृ॒जना॑सु । \newline
31. सु॒वृ॒जना॑सु वि॒क्षु वि॒क्षु सु॑वृ॒जना॑सु सुवृ॒जना॑सु वि॒क्षु । \newline
32. सु॒वृ॒जना॒स्विति॑ सु - वृ॒जना॑सु । \newline
33. वि॒क्ष्विति॑ वि॒क्षु । \newline
34. अधा॒ यथा॒ यथा ऽधाधा॒ यथा᳚ । \newline
35. यथा॑ नो नो॒ यथा॒ यथा॑ नः । \newline
36. नः॒ पि॒तरः॑ पि॒तरो॑ नो नः पि॒तरः॑ । \newline
37. पि॒तरः॒ परा॑सः॒ परा॑सः पि॒तरः॑ पि॒तरः॒ परा॑सः । \newline
38. परा॑सः प्र॒त्नासः॑ प्र॒त्नासः॒ परा॑सः॒ परा॑सः प्र॒त्नासः॑ । \newline
39. प्र॒त्नासो॑ अग्ने अग्ने प्र॒त्नासः॑ प्र॒त्नासो॑ अग्ने । \newline
40. अ॒ग्न॒ ऋ॒त मृ॒त म॑ग्ने अग्न ऋ॒तम् । \newline
41. ऋ॒त मा॑शुषा॒णा आ॑शुषा॒णा ऋ॒त मृ॒त मा॑शुषा॒णाः । \newline
42. आ॒शु॒षा॒णा इत्या॑शुषा॒णाः । \newline
43. शुची दिच् छुचि॒ शुचीत् । \newline
44. इद॑यन् नय॒न् निदि द॑यन्न् । \newline
45. अ॒य॒न् दीधि॑ति॒म् दीधि॑ति मयन् नय॒न् दीधि॑तिम् । \newline
46. दीधि॑ति मुक्थ॒शास॑ उक्थ॒शासो॒ दीधि॑ति॒म् दीधि॑ति मुक्थ॒शासः॑ । \newline
47. उ॒क्थ॒शासः॒ क्षाम॒ क्षामो᳚क्थ॒शास॑ उक्थ॒शासः॒ क्षाम॑ । \newline
48. उ॒क्थ॒शास॒ इत्यु॑क्थ - शासः॑ । \newline
49. क्षामा॑ भि॒न्दन्तो॑ भि॒न्दन्तः॒ क्षाम॒ क्षामा॑ भि॒न्दन्तः॑ । \newline
50. भि॒न्दन्तो॑ अरु॒णी र॑रु॒णीर् भि॒न्दन्तो॑ भि॒न्दन्तो॑ अरु॒णीः । \newline
51. अ॒रु॒णी रपापा॑ रु॒णी र॑रु॒णी रप॑ । \newline
52. अप॑ व्रन् व्र॒न् नपाप॑ व्रन्न् । \newline
53. व्र॒न्निति॑ व्रन्न् । \newline
54. यद॑ग्ने अग्ने॒ यद् यद॑ग्ने । \newline
55. अ॒ग्ने॒ क॒व्य॒वा॒ह॒न॒ क॒व्य॒वा॒ह॒ना॒ग्ने॒ अ॒ग्ने॒ क॒व्य॒वा॒ह॒न॒ । \newline

\textbf{Ghana Paata } \newline

1. य ई॒यु री॒युर् ये य ई॒यु र॑वृ॒का अ॑वृ॒का ई॒युर् ये य ई॒यु र॑वृ॒काः । \newline
2. ई॒यु र॑वृ॒का अ॑वृ॒का ई॒यु री॒यु र॑वृ॒का ऋ॑त॒ज्ञा ऋ॑त॒ज्ञा अ॑वृ॒का ई॒यु री॒यु र॑वृ॒का ऋ॑त॒ज्ञाः । \newline
3. अ॒वृ॒का ऋ॑त॒ज्ञा ऋ॑त॒ज्ञा अ॑वृ॒का अ॑वृ॒का ऋ॑त॒ज्ञा स्ते त ऋ॑त॒ज्ञा अ॑वृ॒का अ॑वृ॒का ऋ॑त॒ज्ञा स्ते । \newline
4. ऋ॒त॒ज्ञा स्ते त ऋ॑त॒ज्ञा ऋ॑त॒ज्ञा स्ते नो॑ न॒स्त ऋ॑त॒ज्ञा ऋ॑त॒ज्ञा स्ते नः॑ । \newline
5. ऋ॒त॒ज्ञा इत्यृ॑त - ज्ञाः । \newline
6. ते नो॑ न॒स्ते ते नो॑ ऽव न्त्ववन्तु न॒स्ते ते नो॑ ऽवन्तु । \newline
7. नो॒ ऽव॒ न्त्व॒व॒न्तु॒ नो॒ नो॒ ऽव॒न्तु॒ पि॒तरः॑ पि॒तरो॑ ऽवन्तु नो नो ऽवन्तु पि॒तरः॑ । \newline
8. अ॒व॒न्तु॒ पि॒तरः॑ पि॒तरो॑ ऽव न्त्ववन्तु पि॒तरो॒ हवे॑षु॒ हवे॑षु पि॒तरो॑ ऽव न्त्ववन्तु पि॒तरो॒ हवे॑षु । \newline
9. पि॒तरो॒ हवे॑षु॒ हवे॑षु पि॒तरः॑ पि॒तरो॒ हवे॑षु । \newline
10. हवे॒ष्विति॒ हवे॑षु । \newline
11. इ॒दम् पि॒तृभ्यः॑ पि॒तृभ्य॑ इ॒द मि॒दम् पि॒तृभ्यो॒ नमो॒ नमः॑ पि॒तृभ्य॑ इ॒द मि॒दम् पि॒तृभ्यो॒ नमः॑ । \newline
12. पि॒तृभ्यो॒ नमो॒ नमः॑ पि॒तृभ्यः॑ पि॒तृभ्यो॒ नमो॑ अस्त्व स्तु॒ नमः॑ पि॒तृभ्यः॑ पि॒तृभ्यो॒ नमो॑ अस्तु । \newline
13. पि॒तृभ्य॒ इति॑ पि॒तृ - भ्यः॒ । \newline
14. नमो॑ अस्त्वस्तु॒ नमो॒ नमो॑ अस्त्व॒ द्याद्यास्तु॒ नमो॒ नमो॑ अस्त्व॒द्य । \newline
15. अ॒स्त्व॒ द्याद्या स्त्व॑स्त्व॒द्य ये ये अ॒द्या स्त्व॑ स्त्व॒द्य ये । \newline
16. अ॒द्य ये ये अ॒द्याद्य ये पूर्वा॑सः॒ पूर्वा॑सो॒ ये अ॒द्याद्य ये पूर्वा॑सः । \newline
17. ये पूर्वा॑सः॒ पूर्वा॑सो॒ ये ये पूर्वा॑सो॒ ये ये पूर्वा॑सो॒ ये ये पूर्वा॑सो॒ ये । \newline
18. पूर्वा॑सो॒ ये ये पूर्वा॑सः॒ पूर्वा॑सो॒ य उप॑रास॒ उप॑रासो॒ ये पूर्वा॑सः॒ पूर्वा॑सो॒ य उप॑रासः । \newline
19. य उप॑रास॒ उप॑रासो॒ ये य उप॑रास ई॒यु री॒यु रुप॑रासो॒ ये य उप॑रास ई॒युः । \newline
20. उप॑रास ई॒यु री॒यु रुप॑रास॒ उप॑रास ई॒युः । \newline
21. ई॒युरिती॒युः । \newline
22. ये पार्थि॑वे॒ पार्थि॑वे॒ ये ये पार्थि॑वे॒ रज॑सि॒ रज॑सि॒ पार्थि॑वे॒ ये ये पार्थि॑वे॒ रज॑सि । \newline
23. पार्थि॑वे॒ रज॑सि॒ रज॑सि॒ पार्थि॑वे॒ पार्थि॑वे॒ रज॒स्या रज॑सि॒ पार्थि॑वे॒ पार्थि॑वे॒ रज॒स्या । \newline
24. रज॒स्या रज॑सि॒ रज॒स्या निष॑त्ता॒ निष॑त्ता॒ आ रज॑सि॒ रज॒स्या निष॑त्ताः । \newline
25. आ निष॑त्ता॒ निष॑त्ता॒ आ निष॑त्ता॒ ये ये निष॑त्ता॒ आ निष॑त्ता॒ ये । \newline
26. निष॑त्ता॒ ये ये निष॑त्ता॒ निष॑त्ता॒ ये वा॑ वा॒ ये निष॑त्ता॒ निष॑त्ता॒ ये वा᳚ । \newline
27. निष॑त्ता॒ इति॒ नि - स॒त्ताः॒ । \newline
28. ये वा॑ वा॒ ये ये वा॑ नू॒नम् नू॒नं ॅवा॒ ये ये वा॑ नू॒नम् । \newline
29. वा॒ नू॒नम् नू॒नं ॅवा॑ वा नू॒नꣳ सु॑वृ॒जना॑सु सुवृ॒जना॑सु नू॒नं ॅवा॑ वा नू॒नꣳ सु॑वृ॒जना॑सु । \newline
30. नू॒नꣳ सु॑वृ॒जना॑सु सुवृ॒जना॑सु नू॒नम् नू॒नꣳ सु॑वृ॒जना॑सु वि॒क्षु वि॒क्षु सु॑वृ॒जना॑सु नू॒नम् नू॒नꣳ सु॑वृ॒जना॑सु वि॒क्षु । \newline
31. सु॒वृ॒जना॑सु वि॒क्षु वि॒क्षु सु॑वृ॒जना॑सु सुवृ॒जना॑सु वि॒क्षु । \newline
32. सु॒वृ॒जना॒स्विति॑ सु - वृ॒जना॑सु । \newline
33. वि॒क्ष्विति॑ वि॒क्षु । \newline
34. अधा॒ यथा॒ यथा ऽधाधा॒ यथा॑ नो नो॒ यथा ऽधाधा॒ यथा॑ नः । \newline
35. यथा॑ नो नो॒ यथा॒ यथा॑ नः पि॒तरः॑ पि॒तरो॑ नो॒ यथा॒ यथा॑ नः पि॒तरः॑ । \newline
36. नः॒ पि॒तरः॑ पि॒तरो॑ नो नः पि॒तरः॒ परा॑सः॒ परा॑सः पि॒तरो॑ नो नः पि॒तरः॒ परा॑सः । \newline
37. पि॒तरः॒ परा॑सः॒ परा॑सः पि॒तरः॑ पि॒तरः॒ परा॑सः प्र॒त्नासः॑ प्र॒त्नासः॒ परा॑सः पि॒तरः॑ पि॒तरः॒ परा॑सः प्र॒त्नासः॑ । \newline
38. परा॑सः प्र॒त्नासः॑ प्र॒त्नासः॒ परा॑सः॒ परा॑सः प्र॒त्नासो॑ अग्ने अग्ने प्र॒त्नासः॒ परा॑सः॒ परा॑सः प्र॒त्नासो॑ अग्ने । \newline
39. प्र॒त्नासो॑ अग्ने अग्ने प्र॒त्नासः॑ प्र॒त्नासो॑ अग्न ऋ॒त मृ॒त म॑ग्ने प्र॒त्नासः॑ प्र॒त्नासो॑ अग्न ऋ॒तम् । \newline
40. अ॒ग्न॒ ऋ॒त मृ॒त म॑ग्ने अग्न ऋ॒त मा॑शुषा॒णा आ॑शुषा॒णा ऋ॒त म॑ग्ने अग्न ऋ॒त मा॑शुषा॒णाः । \newline
41. ऋ॒त मा॑शुषा॒णा आ॑शुषा॒णा ऋ॒त मृ॒त मा॑शुषा॒णाः । \newline
42. आ॒शु॒षा॒णा इत्या॑शुषा॒णाः । \newline
43. शुची दिच्छुचि॒ शुची द॑यन् नय॒न् निच्छुचि॒ शुची द॑यन्न् । \newline
44. इद॑यन् नय॒न् निदि द॑य॒न् दीधि॑ति॒म् दीधि॑ति मय॒न् निदि द॑य॒न् दीधि॑तिम् । \newline
45. अ॒य॒न् दीधि॑ति॒म् दीधि॑ति मयन् नय॒न् दीधि॑ति मुक्थ॒शास॑ उक्थ॒शासो॒ दीधि॑ति मयन् नय॒न् दीधि॑ति मुक्थ॒शासः॑ । \newline
46. दीधि॑ति मुक्थ॒शास॑ उक्थ॒शासो॒ दीधि॑ति॒म् दीधि॑ति मुक्थ॒शासः॒ क्षाम॒ क्षामो᳚ क्थ॒शासो॒ दीधि॑ति॒म् दीधि॑ति मुक्थ॒शासः॒ क्षाम॑ । \newline
47. उ॒क्थ॒शासः॒ क्षाम॒ क्षामो᳚क्थ॒शास॑ उक्थ॒शासः॒ क्षामा॑ भि॒न्दन्तो॑ भि॒न्दन्तः॒ क्षामो᳚ क्थ॒शास॑ उक्थ॒शासः॒ क्षामा॑ भि॒न्दन्तः॑ । \newline
48. उ॒क्थ॒शास॒ इत्यु॑क्थ - शासः॑ । \newline
49. क्षामा॑ भि॒न्दन्तो॑ भि॒न्दन्तः॒ क्षाम॒ क्षामा॑ भि॒न्दन्तो॑ अरु॒णी र॑रु॒णीर् भि॒न्दन्तः॒ क्षाम॒ क्षामा॑ भि॒न्दन्तो॑ अरु॒णीः । \newline
50. भि॒न्दन्तो॑ अरु॒णी र॑रु॒णीर् भि॒न्दन्तो॑ भि॒न्दन्तो॑ अरु॒णी रपापा॑ रु॒णीर् भि॒न्दन्तो॑ भि॒न्दन्तो॑ अरु॒णीरप॑ । \newline
51. अ॒रु॒णी रपापा॑ रु॒णी र॑रु॒णी रप॑ व्रन् व्र॒न् नपा॑ रु॒णी र॑रु॒णी रप॑ व्रन्न् । \newline
52. अप॑ व्रन् व्र॒न् नपाप॑ व्रन्न् । \newline
53. व्र॒न्निति॑ व्रन्न् । \newline
54. यद॑ग्ने अग्ने॒ यद् यद॑ग्ने कव्यवाहन कव्यवाहनाग्ने॒ यद् यद॑ग्ने कव्यवाहन । \newline
55. अ॒ग्ने॒ क॒व्य॒वा॒ह॒न॒ क॒व्य॒वा॒ह॒ना॒ग्ने॒ अ॒ग्ने॒ क॒व्य॒वा॒ह॒न॒ पि॒तॄन् पि॒तॄन् क॑व्यवाहनाग्ने अग्ने कव्यवाहन पि॒तॄन् । \newline
\pagebreak
\markright{ TS 2.6.12.5  \hfill https://www.vedavms.in \hfill}
\addcontentsline{toc}{section}{ TS 2.6.12.5 }
\section*{ TS 2.6.12.5 }

\textbf{TS 2.6.12.5 } \newline
\textbf{Samhita Paata} \newline

कव्यवाहन पि॒तॄन्. यक्ष्यृ॑ता॒वृधः॑ । प्र च॑ ह॒व्यानि॑ वक्ष्यसि दे॒वेभ्य॑श्च पि॒तृभ्य॒ आ ॥ त्वम॑ग्न ईडि॒तो जा॑तवे॒दोऽवा᳚ड्ढ॒व्यानि॑ सुर॒भीणि॑ कृ॒त्वा । प्रादाः᳚ पि॒तृभ्यः॑ स्व॒धया॒ ते अ॑क्षन्न॒द्धि त्वं दे॑व॒ प्रय॑ता ह॒वीꣳषि॑ ॥ मात॑ली क॒व्यैर्य॒मो अङ्गि॑रोभि॒ र्बृह॒स्पति॒र्॒. ऋक्व॑भि र्वावृधा॒नः । याꣳश्च॑ दे॒वा वा॑वृ॒धुर्ये च॑ दे॒वान्थ् स्वाहा॒ऽन्ये स्व॒धया॒ऽन्ये म॑दन्ति ॥ \newline

\textbf{Pada Paata} \newline

क॒व्य॒वा॒ह॒नेति॑ कव्य - वा॒ह॒न॒ । पि॒तॄन् । यक्षि॑ । ऋ॒ता॒वृध॒ इत्यृ॑त - वृधः॑ ॥ प्रेति॑ । च॒ । ह॒व्यानि॑ । व॒क्ष्य॒सि॒ । दे॒वेभ्यः॑ । च॒ । पि॒तृभ्य॒ इति॑ पि॒तृ - भ्यः॒ । आ ॥ त्वम् । अ॒ग्ने॒ । ई॒डि॒तः । जा॒त॒वे॒द॒ इति॑ जात - वे॒दः॒ । अवा᳚ट् । ह॒व्यानि॑ । सु॒र॒भीणि॑ । कृ॒त्वा ॥ प्रेति॑ । अ॒दाः॒ । पि॒तृभ्य॒ इति॑ पि॒तृ - भ्यः॒ । स्व॒धयेति॑ स्व - धया᳚ । ते । अ॒क्ष॒न्न् । अ॒द्धि । त्वम् । दे॒व॒ । प्रय॒तेति॒ प्र - य॒ता॒ । ह॒वीꣳषि॑ ॥ मात॑ली । क॒व्यैः । य॒मः । अङ्गि॑रोभि॒रित्यङ्गि॑रः - भिः॒ । बृह॒स्पतिः॑ । ऋक्व॑भि॒रित्यृक्व॑-भिः॒ । वा॒वृ॒धा॒नः ॥ यान् । च॒ । दे॒वाः । वा॒वृ॒धुः । ये । च॒ । दे॒वान् । स्वाहा᳚ । अ॒न्ये । स्व॒धयेति॑ स्व-धया᳚ । अ॒न्ये । म॒द॒न्ति॒ ॥  \newline


\textbf{Krama Paata} \newline

क॒व्य॒वा॒ह॒न॒ पि॒तॄन् । क॒व्य॒वा॒ह॒नेति॑ कव्य - वा॒ह॒न॒ । पि॒तॄन्. यक्षि॑ । यक्ष्यृ॑ता॒वृधः॑ । ऋ॒ता॒वृध॒ इत्यृ॑त - वृधः॑ ॥ प्र च॑ । च॒ ह॒व्यानि॑ । ह॒व्यानि॑ वक्ष्यसि । व॒क्ष्य॒सि॒ दे॒वेभ्यः॑ । दे॒वेभ्य॑श्च । च॒ पि॒तृभ्यः॑ । पि॒तृभ्य॒ आ । पि॒तृभ्य॒ इति॑ पि॒तृ - भ्यः । एत्या ॥ त्वम॑ग्ने । अ॒ग्न॒ ई॒डि॒तः । ई॒डि॒तो जा॑तवेदः । जा॒त॒वे॒दोऽवा᳚ट् । जा॒त॒वे॒द॒ इति॑ जात - वे॒दः॒ । अवा᳚ड्ढ॒व्यानि॑ । ह॒व्यानि॑ सुर॒भीणि॑ । सु॒र॒भीणि॑ कृ॒त्वा । कृ॒त्वेति॑ कृ॒त्वा ॥ प्रादाः᳚ । अ॒दाः॒ पि॒तृभ्यः॑ । पि॒तृभ्यः॑ स्व॒धया᳚ । पि॒तृभ्य॒ इति॑ पि॒तृ - भ्यः॒ । स्व॒धया॒ ते । स्व॒धयेति॑ स्व - धया᳚ । ते अ॑क्षन्न् । अ॒क्ष॒न्न॒द्धि । अ॒द्धि त्वम् । त्वम् दे॑व । दे॒व॒ प्रय॑ता । प्रय॑ता ह॒वीꣳषि॑ । प्रय॒तेति॒ प्र - य॒ता॒ । ह॒वीꣳषीति॑ ह॒वीꣳषि॑ ॥ मात॑ली क॒व्यैः । क॒व्यैर् य॒मः । य॒मो अङ्गि॑रोभिः । अङ्गि॑रोभि॒र् बृह॒स्पतिः॑ । अङ्गि॑रोभि॒रित्यङ्गि॑रः - भिः॒ । बृह॒स्पति॒र्.॒ ऋक्व॑भिः । ऋक्व॑भिर् वावृधा॒नः । ऋक्व॑भि॒रित्यृक्व॑ - भिः॒ । वा॒वृ॒धा॒न इति॑ वावृधा॒नः ॥ याꣳश्च॑ । च॒ दे॒वाः । दे॒वा वा॑वृ॒धुः । वा॒वृ॒धुर् ये । ये च॑ । च॒ दे॒वान् । दे॒वान्थ् स्वाहा᳚ । स्वाहा॒ऽन्ये । अ॒न्ये स्व॒धया᳚ । स्व॒धया॒ऽन्ये । स्व॒धयेति॑ स्व - धया᳚ । अ॒न्ये म॑दन्ति । म॒द॒न्तीति॑ मदन्ति । \newline

\textbf{Jatai Paata} \newline

1. क॒व्य॒वा॒ह॒न॒ पि॒तॄन् पि॒तॄन् क॑व्यवाहन कव्यवाहन पि॒तॄन् । \newline
2. क॒व्य॒वा॒ह॒नेति॑ कव्य - वा॒ह॒न॒ । \newline
3. पि॒तॄन्. यक्षि॒ यक्षि॑ पि॒तॄन् पि॒तॄन्. यक्षि॑ । \newline
4. यक्ष्यृ॑ता॒वृध॑ ऋता॒वृधो॒ यक्षि॒ यक्ष्यृ॑ता॒वृधः॑ । \newline
5. ऋ॒ता॒वृध॒ इत्यृ॑त - वृधः॑ । \newline
6. प्र च॑ च॒ प्र प्र च॑ । \newline
7. च॒ ह॒व्यानि॑ ह॒व्यानि॑ च च ह॒व्यानि॑ । \newline
8. ह॒व्यानि॑ वक्ष्यसि वक्ष्यसि ह॒व्यानि॑ ह॒व्यानि॑ वक्ष्यसि । \newline
9. व॒क्ष्य॒सि॒ दे॒वेभ्यो॑ दे॒वेभ्यो॑ वक्ष्यसि वक्ष्यसि दे॒वेभ्यः॑ । \newline
10. दे॒वेभ्य॑श्च च दे॒वेभ्यो॑ दे॒वेभ्य॑श्च । \newline
11. च॒ पि॒तृभ्यः॑ पि॒तृभ्य॑श्च च पि॒तृभ्यः॑ । \newline
12. पि॒तृभ्य॒ आ पि॒तृभ्यः॑ पि॒तृभ्य॒ आ । \newline
13. पि॒तृभ्य॒ इति॑ पि॒तृ - भ्यः॒ । \newline
14. एत्या । \newline
15. त्व म॑ग्ने अग्ने॒ त्वम् त्व म॑ग्ने । \newline
16. अ॒ग्न॒ ई॒डि॒त ई॑डि॒तो अ॑ग्ने अग्न ईडि॒तः । \newline
17. ई॒डि॒तो जा॑तवेदो जातवेद ईडि॒त ई॑डि॒तो जा॑तवेदः । \newline
18. जा॒त॒वे॒दो ऽवा॒डवा᳚ड् जातवेदो जातवे॒दो ऽवा᳚ट् । \newline
19. जा॒त॒वे॒द॒ इति॑ जात - वे॒दः॒ । \newline
20. अवा᳚ ड्ढ॒व्यानि॑ ह॒व्या न्यवा॒डवा᳚ ड्ढ॒व्यानि॑ । \newline
21. ह॒व्यानि॑ सुर॒भीणि॑ सुर॒भीणि॑ ह॒व्यानि॑ ह॒व्यानि॑ सुर॒भीणि॑ । \newline
22. सु॒र॒भीणि॑ कृ॒त्वा कृ॒त्वा सु॑र॒भीणि॑ सुर॒भीणि॑ कृ॒त्वा । \newline
23. कृ॒त्वेति॑ कृ॒त्वा । \newline
24. प्रादा॑ अदाः॒ प्र प्रादाः᳚ । \newline
25. अ॒दाः॒ पि॒तृभ्यः॑ पि॒तृभ्यो॑ अदा अदाः पि॒तृभ्यः॑ । \newline
26. पि॒तृभ्यः॑ स्व॒धया᳚ स्व॒धया॑ पि॒तृभ्यः॑ पि॒तृभ्यः॑ स्व॒धया᳚ । \newline
27. पि॒तृभ्य॒ इति॑ पि॒तृ - भ्यः॒ । \newline
28. स्व॒धया॒ ते ते स्व॒धया᳚ स्व॒धया॒ ते । \newline
29. स्व॒धयेति॑ स्व - धया᳚ । \newline
30. ते अ॑क्षन् नक्ष॒न् ते ते अ॑क्षन्न् । \newline
31. अ॒क्ष॒न् न॒द्ध्या᳚(1॒)द्ध्य॑क्षन् नक्षन् न॒द्धि । \newline
32. अ॒द्धि त्वम् त्व म॒द्ध्य॑द्धि त्वम् । \newline
33. त्वम् दे॑व देव॒ त्वम् त्वम् दे॑व । \newline
34. दे॒व॒ प्रय॑ता॒ प्रय॑ता देव देव॒ प्रय॑ता । \newline
35. प्रय॑ता ह॒वीꣳषि॑ ह॒वीꣳषि॒ प्रय॑ता॒ प्रय॑ता ह॒वीꣳषि॑ । \newline
36. प्रय॒तेति॒ प्र - य॒ता॒ । \newline
37. ह॒वीꣳषीति॑ ह॒वीꣳषि॑ । \newline
38. मात॑ली क॒व्यैः क॒व्यैर् मात॑ली॒ मात॑ली क॒व्यैः । \newline
39. क॒व्यैर् य॒मो य॒मः क॒व्यैः क॒व्यैर् य॒मः । \newline
40. य॒मो अङ्गि॑रोभि॒ रङ्गि॑रोभिर् य॒मो य॒मो अङ्गि॑रोभिः । \newline
41. अङ्गि॑रोभि॒र् बृह॒स्पति॒र् बृह॒स्पति॒ रङ्गि॑रोभि॒ रङ्गि॑रोभि॒र् बृह॒स्पतिः॑ । \newline
42. अङ्गि॑रोभि॒रित्यङ्गि॑रः - भिः॒ । \newline
43. बृह॒स्पति॒र्॒. ऋक्व॑भि॒र्॒. ऋक्व॑भि॒र् बृह॒स्पति॒र् बृह॒स्पति॒र्॒. ऋक्व॑भिः । \newline
44. ऋक्व॑भिर् वावृधा॒नो वा॑वृधा॒न ऋक्व॑भि॒र्॒. ऋक्व॑भिर् वावृधा॒नः । \newline
45. ऋक्व॑भि॒रित्यृक्व॑ - भिः॒ । \newline
46. वा॒वृ॒धा॒न इति॑ वावृधा॒नः । \newline
47. याꣳ श्च॑ च॒ यान्. याꣳ श्च॑ । \newline
48. च॒ दे॒वा दे॒वाश्च॑ च दे॒वाः । \newline
49. दे॒वा वा॑वृ॒धुर् वा॑वृ॒धुर् दे॒वा दे॒वा वा॑वृ॒धुः । \newline
50. वा॒वृ॒धुर् ये ये वा॑वृ॒धुर् वा॑वृ॒धुर् ये । \newline
51. ये च॑ च॒ ये ये च॑ । \newline
52. च॒ दे॒वान् दे॒वाꣳ श्च॑ च दे॒वान् । \newline
53. दे॒वान् थ्स्वाहा॒ स्वाहा॑ दे॒वान् दे॒वान् थ्स्वाहा᳚ । \newline
54. स्वाहा॒ ऽन्ये अ॒न्ये स्वाहा॒ स्वाहा॒ ऽन्ये । \newline
55. अ॒न्ये स्व॒धया᳚ स्व॒धया॒ ऽन्ये अ॒न्ये स्व॒धया᳚ । \newline
56. स्व॒धया॒ ऽन्ये अ॒न्ये स्व॒धया᳚ स्व॒धया॒ ऽन्ये । \newline
57. स्व॒धयेति॑ स्व - धया᳚ । \newline
58. अ॒न्ये म॑दन्ति मदन्त्य॒न्ये अ॒न्ये म॑दन्ति । \newline
59. म॒द॒न्तीति॑ मदन्ति । \newline

\textbf{Ghana Paata } \newline

1. क॒व्य॒वा॒ह॒न॒ पि॒तॄन् पि॒तॄन् क॑व्यवाहन कव्यवाहन पि॒तॄन्. यक्षि॒ यक्षि॑ पि॒तॄन् क॑व्यवाहन कव्यवाहन पि॒तॄन्. यक्षि॑ । \newline
2. क॒व्य॒वा॒ह॒नेति॑ कव्य - वा॒ह॒न॒ । \newline
3. पि॒तॄन्. यक्षि॒ यक्षि॑ पि॒तॄन् पि॒तॄन्. यक्ष्यृ॑ता॒वृध॑ ऋता॒वृधो॒ यक्षि॑ पि॒तॄन् पि॒तॄन्. यक्ष्यृ॑ता॒वृधः॑ । \newline
4. यक्ष्यृ॑ता॒वृध॑ ऋता॒वृधो॒ यक्षि॒ यक्ष्यृ॑ता॒वृधः॑ । \newline
5. ऋ॒ता॒वृध॒ इत्यृ॑त - वृधः॑ । \newline
6. प्र च॑ च॒ प्र प्र च॑ ह॒व्यानि॑ ह॒व्यानि॑ च॒ प्र प्र च॑ ह॒व्यानि॑ । \newline
7. च॒ ह॒व्यानि॑ ह॒व्यानि॑ च च ह॒व्यानि॑ वक्ष्यसि वक्ष्यसि ह॒व्यानि॑ च च ह॒व्यानि॑ वक्ष्यसि । \newline
8. ह॒व्यानि॑ वक्ष्यसि वक्ष्यसि ह॒व्यानि॑ ह॒व्यानि॑ वक्ष्यसि दे॒वेभ्यो॑ दे॒वेभ्यो॑ वक्ष्यसि ह॒व्यानि॑ ह॒व्यानि॑ वक्ष्यसि दे॒वेभ्यः॑ । \newline
9. व॒क्ष्य॒सि॒ दे॒वेभ्यो॑ दे॒वेभ्यो॑ वक्ष्यसि वक्ष्यसि दे॒वेभ्य॑श्च च दे॒वेभ्यो॑ वक्ष्यसि वक्ष्यसि दे॒वेभ्य॑श्च । \newline
10. दे॒वेभ्य॑श्च च दे॒वेभ्यो॑ दे॒वेभ्य॑श्च पि॒तृभ्यः॑ पि॒तृभ्य॑श्च दे॒वेभ्यो॑ दे॒वेभ्य॑श्च पि॒तृभ्यः॑ । \newline
11. च॒ पि॒तृभ्यः॑ पि॒तृभ्य॑श्च च पि॒तृभ्य॒ आ पि॒तृभ्य॑श्च च पि॒तृभ्य॒ आ । \newline
12. पि॒तृभ्य॒ आ पि॒तृभ्यः॑ पि॒तृभ्य॒ आ । \newline
13. पि॒तृभ्य॒ इति॑ पि॒तृ - भ्यः॒ । \newline
14. एत्या । \newline
15. त्व म॑ग्ने अग्ने॒ त्वम् त्व म॑ग्न ईडि॒त ई॑डि॒तो अ॑ग्ने॒ त्वम् त्व म॑ग्न ईडि॒तः । \newline
16. अ॒ग्न॒ ई॒डि॒त ई॑डि॒तो अ॑ग्ने अग्न ईडि॒तो जा॑तवेदो जातवेद ईडि॒तो अ॑ग्ने अग्न ईडि॒तो जा॑तवेदः । \newline
17. ई॒डि॒तो जा॑तवेदो जातवेद ईडि॒त ई॑डि॒तो जा॑तवे॒दो ऽवा॒डवा᳚ड् जातवेद ईडि॒त ई॑डि॒तो जा॑तवे॒दो ऽवा᳚ट् । \newline
18. जा॒त॒वे॒दो ऽवा॒डवा᳚ड् जातवेदो जातवे॒दो ऽवा᳚ड्ढ॒व्यानि॑ ह॒व्या न्यवा᳚ड् जातवेदो जातवे॒दो ऽवा᳚ड्ढ॒व्यानि॑ । \newline
19. जा॒त॒वे॒द॒ इति॑ जात - वे॒दः॒ । \newline
20. अवा᳚ड्ढ॒व्यानि॑ ह॒व्या न्यवा॒डवा᳚ ड्ढ॒व्यानि॑ सुर॒भीणि॑ सुर॒भीणि॑ ह॒व्या न्यवा॒डवा᳚ ड्ढ॒व्यानि॑ सुर॒भीणि॑ । \newline
21. ह॒व्यानि॑ सुर॒भीणि॑ सुर॒भीणि॑ ह॒व्यानि॑ ह॒व्यानि॑ सुर॒भीणि॑ कृ॒त्वा कृ॒त्वा सु॑र॒भीणि॑ ह॒व्यानि॑ ह॒व्यानि॑ सुर॒भीणि॑ कृ॒त्वा । \newline
22. सु॒र॒भीणि॑ कृ॒त्वा कृ॒त्वा सु॑र॒भीणि॑ सुर॒भीणि॑ कृ॒त्वा । \newline
23. कृ॒त्वेति॑ कृ॒त्वा । \newline
24. प्रादा॑ अदाः॒ प्र प्रादाः᳚ पि॒तृभ्यः॑ पि॒तृभ्यो॑ अदाः॒ प्र प्रादाः᳚ पि॒तृभ्यः॑ । \newline
25. अ॒दाः॒ पि॒तृभ्यः॑ पि॒तृभ्यो॑ अदा अदाः पि॒तृभ्यः॑ स्व॒धया᳚ स्व॒धया॑ पि॒तृभ्यो॑ अदा अदाः पि॒तृभ्यः॑ स्व॒धया᳚ । \newline
26. पि॒तृभ्यः॑ स्व॒धया᳚ स्व॒धया॑ पि॒तृभ्यः॑ पि॒तृभ्यः॑ स्व॒धया॒ ते ते स्व॒धया॑ पि॒तृभ्यः॑ पि॒तृभ्यः॑ स्व॒धया॒ ते । \newline
27. पि॒तृभ्य॒ इति॑ पि॒तृ - भ्यः॒ । \newline
28. स्व॒धया॒ ते ते स्व॒धया᳚ स्व॒धया॒ ते अ॑क्षन् नक्ष॒न् ते स्व॒धया᳚ स्व॒धया॒ ते अ॑क्षन्न् । \newline
29. स्व॒धयेति॑ स्व - धया᳚ । \newline
30. ते अ॑क्षन् नक्ष॒न् ते ते अ॑क्षन् न॒द्ध्या᳚(1॒)द्ध्य॑क्ष॒न् ते ते अ॑क्षन् न॒द्धि । \newline
31. अ॒क्ष॒न् न॒द्ध्या᳚(1॒)द्ध्य॑क्षन् नक्षन् न॒द्धि त्वम् त्व म॒द्ध्य॑क्षन् नक्षन् न॒द्धि त्वम् । \newline
32. अ॒द्धि त्वम् त्व म॒द्ध्य॑द्धि त्वम् दे॑व देव॒ त्व म॒द्ध्य॑द्धि त्वम् दे॑व । \newline
33. त्वम् दे॑व देव॒ त्वम् त्वम् दे॑व॒ प्रय॑ता॒ प्रय॑ता देव॒ त्वम् त्वम् दे॑व॒ प्रय॑ता । \newline
34. दे॒व॒ प्रय॑ता॒ प्रय॑ता देव देव॒ प्रय॑ता ह॒वीꣳषि॑ ह॒वीꣳषि॒ प्रय॑ता देव देव॒ प्रय॑ता ह॒वीꣳषि॑ । \newline
35. प्रय॑ता ह॒वीꣳषि॑ ह॒वीꣳषि॒ प्रय॑ता॒ प्रय॑ता ह॒वीꣳषि॑ । \newline
36. प्रय॒तेति॒ प्र - य॒ता॒ । \newline
37. ह॒वीꣳषीति॑ ह॒वीꣳषि॑ । \newline
38. मात॑ली क॒व्यैः क॒व्यैर् मात॑ली॒ मात॑ली क॒व्यैर् य॒मो य॒मः क॒व्यैर् मात॑ली॒ मात॑ली क॒व्यैर् य॒मः । \newline
39. क॒व्यैर् य॒मो य॒मः क॒व्यैः क॒व्यैर् य॒मो अङ्गि॑रोभि॒ रङ्गि॑रोभिर् य॒मः क॒व्यैः क॒व्यैर् य॒मो अङ्गि॑रोभिः । \newline
40. य॒मो अङ्गि॑रोभि॒ रङ्गि॑रोभिर् य॒मो य॒मो अङ्गि॑रोभि॒र् बृह॒स्पति॒र् बृह॒स्पति॒ रङ्गि॑रोभिर् य॒मो य॒मो 
अङ्गि॑रोभि॒र् बृह॒स्पतिः॑ । \newline
41. अङ्गि॑रोभि॒र् बृह॒स्पति॒र् बृह॒स्पति॒ रङ्गि॑रोभि॒ रङ्गि॑रोभि॒र् बृह॒स्पति॒र्॒. ऋक्व॑भि॒र्॒. ऋक्व॑भि॒र् बृह॒स्पति॒ 
रङ्गि॑रोभि॒ रङ्गि॑रोभि॒र् बृह॒स्पति॒र्॒. ऋक्व॑भिः । \newline
42. अङ्गि॑रोभि॒रित्यङ्गि॑रः - भिः॒ । \newline
43. बृह॒स्पति॒र्॒. ऋक्व॑भि॒र्॒. ऋक्व॑भि॒र् बृह॒स्पति॒र् बृह॒स्पति॒र्॒. ऋक्व॑भिर् वावृधा॒नो वा॑वृधा॒न ऋक्व॑भि॒र् बृह॒स्पति॒र् बृह॒स्पति॒र्॒. ऋक्व॑भिर् वावृधा॒नः । \newline
44. ऋक्व॑भिर् वावृधा॒नो वा॑वृधा॒न ऋक्व॑भि॒र्॒. ऋक्व॑भिर् वावृधा॒नः । \newline
45. ऋक्व॑भि॒रित्यृक्व॑ - भिः॒ । \newline
46. वा॒वृ॒धा॒न इति॑ वावृधा॒नः । \newline
47. याꣳ श्च॑ च॒ यान्. याꣳ श्च॑ दे॒वा दे॒वाश्च॒ यान्. याꣳ श्च॑ दे॒वाः । \newline
48. च॒ दे॒वा दे॒वाश्च॑ च दे॒वा वा॑वृ॒धुर् वा॑वृ॒धुर् दे॒वाश्च॑ च दे॒वा वा॑वृ॒धुः । \newline
49. दे॒वा वा॑वृ॒धुर् वा॑वृ॒धुर् दे॒वा दे॒वा वा॑वृ॒धुर् ये ये वा॑वृ॒धुर् दे॒वा दे॒वा वा॑वृ॒धुर् ये । \newline
50. वा॒वृ॒धुर् ये ये वा॑वृ॒धुर् वा॑वृ॒धुर् ये च॑ च॒ ये वा॑वृ॒धुर् वा॑वृ॒धुर् ये च॑ । \newline
51. ये च॑ च॒ ये ये च॑ दे॒वान् दे॒वाꣳश्च॒ ये ये च॑ दे॒वान् । \newline
52. च॒ दे॒वान् दे॒वाꣳश्च॑ च दे॒वान् थ्स्वाहा॒ स्वाहा॑ दे॒वाꣳश्च॑ च दे॒वान् थ्स्वाहा᳚ । \newline
53. दे॒वान् थ्स्वाहा॒ स्वाहा॑ दे॒वान् दे॒वान् थ्स्वाहा॒ ऽन्ये अ॒न्ये स्वाहा॑ दे॒वान् दे॒वान् थ्स्वाहा॒ ऽन्ये । \newline
54. स्वाहा॒ ऽन्ये अ॒न्ये स्वाहा॒ स्वाहा॒ ऽन्ये स्व॒धया᳚ स्व॒धया॒ ऽन्ये स्वाहा॒ स्वाहा॒ ऽन्ये स्व॒धया᳚ । \newline
55. अ॒न्ये स्व॒धया᳚ स्व॒धया॒ ऽन्ये अ॒न्ये स्व॒धया॒ ऽन्ये अ॒न्ये स्व॒धया॒ ऽन्ये अ॒न्ये स्व॒धया॒ ऽन्ये । \newline
56. स्व॒धया॒ ऽन्ये अ॒न्ये स्व॒धया᳚ स्व॒धया॒ ऽन्ये म॑दन्ति मद न्त्य॒ न्ये स्व॒धया᳚ स्व॒धया॒ ऽन्ये म॑दन्ति । \newline
57. स्व॒धयेति॑ स्व - धया᳚ । \newline
58. अ॒न्ये म॑दन्ति मद न्त्य॒न्ये अ॒न्ये म॑दन्ति । \newline
59. म॒द॒न्तीति॑ मदन्ति । \newline
\pagebreak
\markright{ TS 2.6.12.6  \hfill https://www.vedavms.in \hfill}
\addcontentsline{toc}{section}{ TS 2.6.12.6 }
\section*{ TS 2.6.12.6 }

\textbf{TS 2.6.12.6 } \newline
\textbf{Samhita Paata} \newline

इ॒मं ॅय॑म प्रस्त॒रमाहि सीदाङ्गि॑रोभिः पि॒तृभिः॑ संॅविदा॒नः । आत्वा॒ मन्त्राः᳚ कविश॒स्ता व॑हन्त्वे॒ना रा॑जन्. ह॒विषा॑ मादयस्व ॥ अङ्गि॑रोभि॒रा ग॑हि य॒ज्ञिये॑भि॒र्यम॑ वैरू॒पैरि॒ह मा॑दयस्व । विव॑स्वन्तꣳ हुवे॒ यः पि॒ता ते॒ऽस्मिन्. य॒ज्ञे ब॒र्॒.हिष्या नि॒षद्य॑ ॥ अङ्गि॑रसो नः पि॒तरो॒ नव॑ग्वा॒ अथ॑र्वाणो॒ भृग॑वः सो॒म्यासः॑ । तेषां᳚ ॅव॒यꣳ सु॑म॒तौ य॒ज्ञिया॑ना॒मपि॑ भ॒द्रे सौ॑मन॒से ( ) स्या॑म ॥ \newline

\textbf{Pada Paata} \newline

इ॒मम् । य॒म॒ । प्र॒स्त॒रमिति॑ प्र - स्त॒रम् । एति॑ । हि । सीद॑ । अङ्गि॑रोभि॒रित्यङ्गि॑रः - भिः॒ । पि॒तृभि॒रिति॑ पि॒तृ - भिः॒ । सं॒ॅवि॒दा॒न इति॑ सं - वि॒दा॒नः ॥ एति॑ । त्वा॒ । मन्त्राः᳚ । क॒वि॒श॒स्ता इति॑ कवि - श॒स्ताः । व॒ह॒न्तु॒ । ए॒ना । रा॒ज॒न्न् । ह॒विषा᳚ । मा॒द॒य॒स्व॒ ॥ अङ्गि॑रोभि॒रित्यङ्गि॑रः - भिः॒ । एति॑ । ग॒हि॒ । य॒ज्ञिये॑भिः । यम॑ । वै॒रू॒पैः । इ॒ह । मा॒द॒य॒स्व॒ ॥ विव॑स्वन्तम् । हु॒वे॒ । यः । पि॒ता । ते॒ । अ॒स्मिन्न् । य॒ज्ञे । ब॒र्॒.हिषि॑ । एति॑ । नि॒षद्येति॑ नि - सद्य॑ ॥ अङ्गि॑रसः । नः॒ । पि॒तरः॑ । नव॑ग्वाः । अथ॑र्वाणः । भृग॑वः । सो॒म्यासः॑ ॥ तेषा᳚म् । व॒यम् । सु॒म॒ताविति॑ सु - म॒तौ । य॒ज्ञिया॑नाम् । अपीति॑ । भ॒द्रे । सौ॒म॒न॒से ( ) । स्या॒म॒ ॥  \newline


\textbf{Krama Paata} \newline

इ॒मं ॅय॑म । य॒म॒ प्र॒स्त॒रम् । प्र॒स्त॒रमा । प्र॒स्त॒रमिति॑ प्र - स्त॒रम् । आ हि । हि सीद॑ । सीदाङ्गि॑रोभिः । अङ्गि॑रोभिः पि॒तृभिः॑ । अङ्गि॑रोभि॒रित्यङ्गि॑रः - भिः॒ । पि॒तृभिः॑ सम्ॅविदा॒नः । पि॒तृभि॒रिति॑ पि॒तृ - भिः॒ । स॒म्ॅवि॒दा॒न इति॑ सं - वि॒दा॒नः ॥ आ त्वा᳚ । त्वा॒ मन्त्राः᳚ । मन्त्राः᳚ कविश॒स्ताः । क॒वि॒श॒स्ता व॑हन्तु । क॒वि॒श॒स्ता इति॑ कवि - श॒स्ताः । व॒ह॒न्त्वे॒ना । ए॒ना रा॑जन्न् । रा॒ज॒न्॒. ह॒विषा᳚ । ह॒विषा॑ मादयस्व । मा॒द॒य॒स्वेति॑ मादयस्व ॥ अङ्गि॑रोभि॒रा । अङ्गि॑रोभि॒रित्यङ्गि॑रः - भिः॒ । आ ग॑हि । ग॒हि॒ य॒ज्ञिये॑भिः । य॒ज्ञिये॑भि॒र् यम॑ । यम॑ वैरू॒पैः । वै॒रू॒पैरि॒ह । इ॒ह मा॑दयस्व । मा॒द॒य॒स्वेति॑ मादयस्व ॥ विव॑स्वन्तꣳ हुवे । हु॒वे॒ यः । यः पि॒ता । पि॒ता ते᳚ । ते॒ऽस्मिन्न् । अ॒स्मिन्न्. य॒ज्ञे । य॒ज्ञे ब॒र्.॒हिषि॑ । ब॒र्.॒हिष्या । आ नि॒षद्य॑ । नि॒षद्येति॑ नि - सद्य॑ ॥ अङ्गि॑रसो नः । नः॒ पि॒तरः॑ । पि॒तरो॒ नव॑ग्वाः । नव॑ग्वा॒ अथ॑र्वाणः । अथ॑र्वाणो॒ भृग॑वः । भृग॑वः सो॒म्यासः॑ । सो॒म्यास॒ इति॑ सो॒म्यासः॑ ॥ तेषां᳚ ॅव॒यम् । व॒यꣳ सु॑म॒तौ । सु॒म॒तौ य॒ज्ञिया॑नाम् । सु॒म॒ताविति॑ सु - म॒तौ । य॒ज्ञिया॑ना॒मपि॑ । अपि॑ भ॒द्रे । भ॒द्रे सौ॑मन॒से ( ) । सौ॒म॒न॒से स्या॑म । स्या॒मेति॑ स्याम । \newline

\textbf{Jatai Paata} \newline

1. इ॒मं ॅय॑म यमे॒ म मि॒मं ॅय॑म । \newline
2. य॒म॒ प्र॒स्त॒रम् प्र॑स्त॒रं ॅय॑म यम प्रस्त॒रम् । \newline
3. प्र॒स्त॒र मा प्र॑स्त॒रम् प्र॑स्त॒र मा । \newline
4. प्र॒स्त॒रमिति॑ प्र - स्त॒रम् । \newline
5. आ हि ह्या हि । \newline
6. हि सीद॒ सीद॒ हि हि सीद॑ । \newline
7. सीदाङ्गि॑रोभि॒ रङ्गि॑रोभिः॒ सीद॒ सीदाङ्गि॑रोभिः । \newline
8. अङ्गि॑रोभिः पि॒तृभिः॑ पि॒तृभि॒ रङ्गि॑रोभि॒ रङ्गि॑रोभिः पि॒तृभिः॑ । \newline
9. अङ्गि॑रोभि॒रित्यङ्गि॑रः - भिः॒ । \newline
10. पि॒तृभिः॑ संॅविदा॒नः सं॑ॅविदा॒नः पि॒तृभिः॑ पि॒तृभिः॑ संॅविदा॒नः । \newline
11. पि॒तृभि॒रिति॑ पि॒तृ - भिः॒ । \newline
12. सं॒ॅवि॒दा॒न इति॑ सं - वि॒दा॒नः । \newline
13. आ त्वा॒ त्वा ऽऽत्वा᳚ । \newline
14. त्वा॒ मन्त्रा॒ मन्त्रा᳚ स्त्वा त्वा॒ मन्त्राः᳚ । \newline
15. मन्त्राः᳚ कविश॒स्ताः क॑विश॒स्ता मन्त्रा॒ मन्त्राः᳚ कविश॒स्ताः । \newline
16. क॒वि॒श॒स्ता व॑हन्तु वहन्तु कविश॒स्ताः क॑विश॒स्ता व॑हन्तु । \newline
17. क॒वि॒श॒स्ता इति॑ कवि - श॒स्ताः । \newline
18. व॒ह॒ न्त्वे॒नैना व॑हन्तु वह न्त्वे॒ना । \newline
19. ए॒ना रा॑जन् राजन् ने॒नैना रा॑जन्न् । \newline
20. रा॒ज॒न्॒. ह॒विषा॑ ह॒विषा॑ राजन् राजन्. ह॒विषा᳚ । \newline
21. ह॒विषा॑ मादयस्व मादयस्व ह॒विषा॑ ह॒विषा॑ मादयस्व । \newline
22. मा॒द॒य॒स्वेति॑ मादयस्व । \newline
23. अङ्गि॑रोभि॒रा ऽङ्गि॑रोभि॒ रङ्गि॑रोभि॒रा । \newline
24. अङ्गि॑रोभि॒रित्यङ्गि॑रः - भिः॒ । \newline
25. आ ग॑हि ग॒ह्या ग॑हि । \newline
26. ग॒हि॒ य॒ज्ञिये॑भिर् य॒ज्ञिये॑भिर् गहि गहि य॒ज्ञिये॑भिः । \newline
27. य॒ज्ञिये॑भि॒र् यम॒ यम॑ य॒ज्ञिये॑भिर् य॒ज्ञिये॑भि॒र् यम॑ । \newline
28. यम॑ वैरू॒पैर् वै॑रू॒पैर् यम॒ यम॑ वैरू॒पैः । \newline
29. वै॒रू॒पैरि॒हे ह वै॑रू॒पैर् वै॑रू॒पैरि॒ह । \newline
30. इ॒ह मा॑दयस्व मादयस्वे॒ हे ह मा॑दयस्व । \newline
31. मा॒द॒य॒स्वेति॑ मादयस्व । \newline
32. विव॑स्वन्तꣳ हुवे हुवे॒ विव॑स्वन्तं॒ ॅविव॑स्वन्तꣳ हुवे । \newline
33. हु॒वे॒ यो यो हु॑वे हुवे॒ यः । \newline
34. यः पि॒ता पि॒ता यो यः पि॒ता । \newline
35. पि॒ता ते॑ ते पि॒ता पि॒ता ते᳚ । \newline
36. ते॒ ऽस्मिन् न॒स्मिन् ते॑ ते॒ ऽस्मिन्न् । \newline
37. अ॒स्मिन्. य॒ज्ञे य॒ज्ञे अ॒स्मिन् न॒स्मिन्. य॒ज्ञे । \newline
38. य॒ज्ञे ब॒र्॒.हिषि॑ ब॒र्॒.हिषि॑ य॒ज्ञे य॒ज्ञे ब॒र्॒.हिषि॑ । \newline
39. ब॒र्॒.हिष्या ब॒र्॒.हिषि॑ ब॒र्॒.हिष्या । \newline
40. आ नि॒षद्य॑ नि॒षद्या नि॒षद्य॑ । \newline
41. नि॒षद्येति॑ नि - सद्य॑ । \newline
42. अङ्गि॑रसो नो नो॒ अङ्गि॑रसो॒ अङ्गि॑रसो नः । \newline
43. नः॒ पि॒तरः॑ पि॒तरो॑ नो नः पि॒तरः॑ । \newline
44. पि॒तरो॒ नव॑ग्वा॒ नव॑ग्वाः पि॒तरः॑ पि॒तरो॒ नव॑ग्वाः । \newline
45. नव॑ग्वा॒ अथ॑र्वाणो॒ अथ॑र्वाणो॒ नव॑ग्वा॒ नव॑ग्वा॒ अथ॑र्वाणः । \newline
46. अथ॑र्वाणो॒ भृग॑वो॒ भृग॑वो॒ अथ॑र्वाणो॒ अथ॑र्वाणो॒ भृग॑वः । \newline
47. भृग॑वः सो॒म्यासः॑ सो॒म्यासो॒ भृग॑वो॒ भृग॑वः सो॒म्यासः॑ । \newline
48. सो॒म्यास॒ इति॑ सो॒म्यासः॑ । \newline
49. तेषां᳚ ॅव॒यं ॅव॒यम् तेषा॒म् तेषां᳚ ॅव॒यम् । \newline
50. व॒यꣳ सु॑म॒तौ सु॑म॒तौ व॒यं ॅव॒यꣳ सु॑म॒तौ । \newline
51. सु॒म॒तौ य॒ज्ञिया॑नां ॅय॒ज्ञिया॑नाꣳ सुम॒तौ सु॑म॒तौ य॒ज्ञिया॑नाम् । \newline
52. सु॒म॒ताविति॑ सु - म॒तौ । \newline
53. य॒ज्ञिया॑ना॒ मप्यपि॑ य॒ज्ञिया॑नां ॅय॒ज्ञिया॑ना॒ मपि॑ । \newline
54. अपि॑ भ॒द्रे भ॒द्रे अप्यपि॑ भ॒द्रे । \newline
55. भ॒द्रे सौ॑मन॒से सौ॑मन॒से भ॒द्रे भ॒द्रे सौ॑मन॒से । \newline
56. सौ॒म॒न॒से स्या॑म स्याम सौमन॒से सौ॑मन॒से स्या॑म । \newline
57. स्या॒मेति॑ स्याम । \newline

\textbf{Ghana Paata } \newline

1. इ॒मं ॅय॑म यमे॒ म मि॒मं ॅय॑म प्रस्त॒रम् प्र॑स्त॒रं ॅय॑मे॒ म मि॒मं ॅय॑म प्रस्त॒रम् । \newline
2. य॒म॒ प्र॒स्त॒रम् प्र॑स्त॒रं ॅय॑म यम प्रस्त॒र मा प्र॑स्त॒रं ॅय॑म यम प्रस्त॒र मा । \newline
3. प्र॒स्त॒र मा प्र॑स्त॒रम् प्र॑स्त॒र मा हि ह्या प्र॑स्त॒रम् प्र॑स्त॒र मा हि । \newline
4. प्र॒स्त॒रमिति॑ प्र - स्त॒रम् । \newline
5. आ हि ह्या हि सीद॒ सीद॒ ह्या हि सीद॑ । \newline
6. हि सीद॒ सीद॒ हि हि सीदा ङ्गि॑रोभि॒ रङ्गि॑रोभिः॒ सीद॒ हि हि सीदा ङ्गि॑रोभिः । \newline
7. सीदा ङ्गि॑रोभि॒ रङ्गि॑रोभिः॒ सीद॒ सीदाङ्गि॑रोभिः पि॒तृभिः॑ पि॒तृभि॒ रङ्गि॑रोभिः॒ सीद॒ सीदाङ्गि॑रोभिः पि॒तृभिः॑ । \newline
8. अङ्गि॑रोभिः पि॒तृभिः॑ पि॒तृभि॒ रङ्गि॑रोभि॒ रङ्गि॑रोभिः पि॒तृभिः॑ संॅविदा॒नः सं॑ॅविदा॒नः पि॒तृभि॒ रङ्गि॑रोभि॒ रङ्गि॑रोभिः पि॒तृभिः॑ संॅविदा॒नः । \newline
9. अङ्गि॑रोभि॒रित्यङ्गि॑रः - भिः॒ । \newline
10. पि॒तृभिः॑ संॅविदा॒नः सं॑ॅविदा॒नः पि॒तृभिः॑ पि॒तृभिः॑ संॅविदा॒नः । \newline
11. पि॒तृभि॒रिति॑ पि॒तृ - भिः॒ । \newline
12. सं॒ॅवि॒दा॒न इति॑ सं - वि॒दा॒नः । \newline
13. आ त्वा॒ त्वा ऽऽत्वा॒ मन्त्रा॒ मन्त्रा॒ स्त्वा ऽऽत्वा॒ मन्त्राः᳚ । \newline
14. त्वा॒ मन्त्रा॒ मन्त्रा᳚ स्त्वा त्वा॒ मन्त्राः᳚ कविश॒स्ताः क॑विश॒स्ता मन्त्रा᳚ स्त्वा त्वा॒ मन्त्राः᳚ कविश॒स्ताः । \newline
15. मन्त्राः᳚ कविश॒स्ताः क॑विश॒स्ता मन्त्रा॒ मन्त्राः᳚ कविश॒स्ता व॑हन्तु वहन्तु कविश॒स्ता मन्त्रा॒ मन्त्राः᳚ कविश॒स्ता व॑हन्तु । \newline
16. क॒वि॒श॒स्ता व॑हन्तु वहन्तु कविश॒स्ताः क॑विश॒स्ता व॑ह न्त्वे॒नैना व॑हन्तु कविश॒स्ताः क॑विश॒स्ता व॑ह न्त्वे॒ना । \newline
17. क॒वि॒श॒स्ता इति॑ कवि - श॒स्ताः । \newline
18. व॒ह॒ न्त्वे॒नैना व॑हन्तु वह न्त्वे॒ना रा॑जन् राजन् ने॒ना व॑हन्तु वह न्त्वे॒ना रा॑जन्न् । \newline
19. ए॒ना रा॑जन् राजन् ने॒नैना रा॑जन्. ह॒विषा॑ ह॒विषा॑ राजन् ने॒नैना रा॑जन्. ह॒विषा᳚ । \newline
20. रा॒ज॒न्॒. ह॒विषा॑ ह॒विषा॑ राजन् राजन्. ह॒विषा॑ मादयस्व मादयस्व ह॒विषा॑ राजन् राजन्. ह॒विषा॑ मादयस्व । \newline
21. ह॒विषा॑ मादयस्व मादयस्व ह॒विषा॑ ह॒विषा॑ मादयस्व । \newline
22. मा॒द॒य॒स्वेति॑ मादयस्व । \newline
23. अङ्गि॑रोभि॒रा ऽङ्गि॑रोभि॒ रङ्गि॑रोभि॒रा ग॑हि ग॒ह्या ऽङ्गि॑रोभि॒ रङ्गि॑रोभि॒रा ग॑हि । \newline
24. अङ्गि॑रोभि॒रित्यङ्गि॑रः - भिः॒ । \newline
25. आ ग॑हि ग॒ह्या ग॑हि य॒ज्ञिये॑भिर् य॒ज्ञिये॑भिर् ग॒ह्या ग॑हि य॒ज्ञिये॑भिः । \newline
26. ग॒हि॒ य॒ज्ञिये॑भिर् य॒ज्ञिये॑भिर् गहि गहि य॒ज्ञिये॑भि॒र् यम॒ यम॑ य॒ज्ञिये॑भिर् गहि गहि य॒ज्ञिये॑भि॒र् यम॑ । \newline
27. य॒ज्ञिये॑भि॒र् यम॒ यम॑ य॒ज्ञिये॑भिर् य॒ज्ञिये॑भि॒र् यम॑ वैरू॒पैर् वै॑रू॒पैर् यम॑ य॒ज्ञिये॑भिर् य॒ज्ञिये॑भि॒र् यम॑ वैरू॒पैः । \newline
28. यम॑ वैरू॒पैर् वै॑रू॒पैर् यम॒ यम॑ वैरू॒पैरि॒हे ह वै॑रू॒पैर् यम॒ यम॑ वैरू॒पैरि॒ह । \newline
29. वै॒रू॒पैरि॒हे ह वै॑रू॒पैर् वै॑रू॒पैरि॒ह मा॑दयस्व मादयस्वे॒ ह वै॑रू॒पैर् वै॑रू॒पैरि॒ह मा॑दयस्व । \newline
30. इ॒ह मा॑दयस्व मादयस्वे॒ हे ह मा॑दयस्व । \newline
31. मा॒द॒य॒स्वेति॑ मादयस्व । \newline
32. विव॑स्वन्तꣳ हुवे हुवे॒ विव॑स्वन्तं॒ ॅविव॑स्वन्तꣳ हुवे॒ यो यो हु॑वे॒ विव॑स्वन्तं॒ ॅविव॑स्वन्तꣳ हुवे॒ यः । \newline
33. हु॒वे॒ यो यो हु॑वे हुवे॒ यः पि॒ता पि॒ता यो हु॑वे हुवे॒ यः पि॒ता । \newline
34. यः पि॒ता पि॒ता यो यः पि॒ता ते॑ ते पि॒ता यो यः पि॒ता ते᳚ । \newline
35. पि॒ता ते॑ ते पि॒ता पि॒ता ते॒ ऽस्मिन् न॒स्मिन् ते॑ पि॒ता पि॒ता ते॒ ऽस्मिन्न् । \newline
36. ते॒ ऽस्मिन् न॒स्मिन् ते॑ ते॒ ऽस्मिन्. य॒ज्ञे य॒ज्ञे अ॒स्मिन् ते॑ ते॒ ऽस्मिन्. य॒ज्ञे । \newline
37. अ॒स्मिन्. य॒ज्ञे य॒ज्ञे अ॒स्मिन् न॒स्मिन्. य॒ज्ञे ब॒र्॒.हिषि॑ ब॒र्॒.हिषि॑ य॒ज्ञे अ॒स्मिन् न॒स्मिन्. य॒ज्ञे ब॒र्॒.हिषि॑ । \newline
38. य॒ज्ञे ब॒र्॒.हिषि॑ ब॒र्॒.हिषि॑ य॒ज्ञे य॒ज्ञे ब॒र्॒.हिष्या ब॒र्॒.हिषि॑ य॒ज्ञे य॒ज्ञे ब॒र्॒.हिष्या । \newline
39. ब॒र्॒.हिष्या ब॒र्॒.हिषि॑ ब॒र्॒.हिष्या नि॒षद्य॑ नि॒षद्या ब॒र्॒.हिषि॑ ब॒र्॒.हिष्या नि॒षद्य॑ । \newline
40. आ नि॒षद्य॑ नि॒षद्या नि॒षद्य॑ । \newline
41. नि॒षद्येति॑ नि - सद्य॑ । \newline
42. अङ्गि॑रसो नो नो॒ अङ्गि॑रसो॒ अङ्गि॑रसो नः पि॒तरः॑ पि॒तरो॑ नो॒ अङ्गि॑रसो॒ अङ्गि॑रसो नः पि॒तरः॑ । \newline
43. नः॒ पि॒तरः॑ पि॒तरो॑ नो नः पि॒तरो॒ नव॑ग्वा॒ नव॑ग्वाः पि॒तरो॑ नो नः पि॒तरो॒ नव॑ग्वाः । \newline
44. पि॒तरो॒ नव॑ग्वा॒ नव॑ग्वाः पि॒तरः॑ पि॒तरो॒ नव॑ग्वा॒ अथ॑र्वाणो॒ अथ॑र्वाणो॒ नव॑ग्वाः पि॒तरः॑ पि॒तरो॒ नव॑ग्वा॒ अथ॑र्वाणः । \newline
45. नव॑ग्वा॒ अथ॑र्वाणो॒ अथ॑र्वाणो॒ नव॑ग्वा॒ नव॑ग्वा॒ अथ॑र्वाणो॒ भृग॑वो॒ भृग॑वो॒ अथ॑र्वाणो॒ नव॑ग्वा॒ नव॑ग्वा॒ अथ॑र्वाणो॒ भृग॑वः । \newline
46. अथ॑र्वाणो॒ भृग॑वो॒ भृग॑वो॒ अथ॑र्वाणो॒ अथ॑र्वाणो॒ भृग॑वः सो॒म्यासः॑ सो॒म्यासो॒ भृग॑वो॒ अथ॑र्वाणो॒ अथ॑र्वाणो॒ भृग॑वः सो॒म्यासः॑ । \newline
47. भृग॑वः सो॒म्यासः॑ सो॒म्यासो॒ भृग॑वो॒ भृग॑वः सो॒म्यासः॑ । \newline
48. सो॒म्यास॒ इति॑ सो॒म्यासः॑ । \newline
49. तेषां᳚ ॅव॒यं ॅव॒यम् तेषा॒म् तेषां᳚ ॅव॒यꣳ सु॑म॒तौ सु॑म॒तौ व॒यम् तेषा॒म् तेषां᳚ ॅव॒यꣳ सु॑म॒तौ । \newline
50. व॒यꣳ सु॑म॒तौ सु॑म॒तौ व॒यं ॅव॒यꣳ सु॑म॒तौ य॒ज्ञिया॑नां ॅय॒ज्ञिया॑नाꣳ सुम॒तौ व॒यं ॅव॒यꣳ सु॑म॒तौ य॒ज्ञिया॑नाम् । \newline
51. सु॒म॒तौ य॒ज्ञिया॑नां ॅय॒ज्ञिया॑नाꣳ सुम॒तौ सु॑म॒तौ य॒ज्ञिया॑ना॒ मप्यपि॑ य॒ज्ञिया॑नाꣳ सुम॒तौ सु॑म॒तौ य॒ज्ञिया॑ना॒ मपि॑ । \newline
52. सु॒म॒ताविति॑ सु - म॒तौ । \newline
53. य॒ज्ञिया॑ना॒ मप्यपि॑ य॒ज्ञिया॑नां ॅय॒ज्ञिया॑ना॒ मपि॑ भ॒द्रे भ॒द्रे अपि॑ य॒ज्ञिया॑नां ॅय॒ज्ञिया॑ना॒ मपि॑ भ॒द्रे । \newline
54. अपि॑ भ॒द्रे भ॒द्रे अप्यपि॑ भ॒द्रे सौ॑मन॒से सौ॑मन॒से भ॒द्रे अप्यपि॑ भ॒द्रे सौ॑मन॒से । \newline
55. भ॒द्रे सौ॑मन॒से सौ॑मन॒से भ॒द्रे भ॒द्रे सौ॑मन॒से स्या॑म स्याम सौमन॒से भ॒द्रे भ॒द्रे सौ॑मन॒से स्या॑म । \newline
56. सौ॒म॒न॒से स्या॑म स्याम सौमन॒से सौ॑मन॒से स्या॑म । \newline
57. स्या॒मेति॑ स्याम । \newline
\pagebreak


\end{document}