\documentclass[17pt]{extarticle}
\usepackage{babel}
\usepackage{fontspec}
\usepackage{polyglossia}
\usepackage{extsizes}

\usepackage{color}   %May be necessary if you want to color links
\usepackage{hyperref}
\hypersetup{
    colorlinks=true, %set true if you want colored links
    linktoc=all,     %set to all if you want both sections and subsections linked
    linkcolor=black,  %choose some color if you want links to stand out
}

\setmainlanguage{sanskrit}
\setotherlanguages{english} %% or other languages
\setlength{\parindent}{0pt}
\pagestyle{myheadings}
\newfontfamily\devanagarifont[Script=Devanagari]{AdishilaVedic}
\renewcommand{\theHsection}{\thepart.section.\thesection}

\newcommand{\VAR}[1]{}
\newcommand{\BLOCK}[1]{}




\begin{document}
\begin{titlepage}
    \begin{center}
 
\begin{sanskrit}
    { \Large
    कृष्ण यजुर्वेदीय तैत्तिरीय संहिता,पद,जटा,घन पाठः 
    }
    \\
    \vspace{2.5cm}
    \mbox{ \Large
    6.5      षष्ठकाण्डे पञ्चमः प्रश्नः - सोममन्त्रब्राह्मणनिरूपणं   }
\end{sanskrit}
\end{center}

\end{titlepage}
\tableofcontents
\phantomsection
\pagebreak

\markright{ TS 6.5.1.1  \hfill https://www.vedavms.in \hfill}

\section{ TS 6.5.1.1 }

\textbf{TS 6.5.1.1 } \newline
\textbf{Samhita Paata} \newline

इन्द्रो॑ वृ॒त्राय॒ वज्र॒मुद॑यच्छ॒थ् स वृ॒त्रो वज्रा॒दुद्य॑तादबिभे॒थ् सो᳚ऽब्रवी॒न्मा मे॒ प्र हा॒रस्ति॒ वा इ॒दं मयि॑ वी॒र्यं॑ तत् ते॒ प्र दा᳚स्या॒मीति॒ तस्मा॑ उ॒क्थ्यं॑ प्राय॑च्छ॒त् तस्मै᳚ द्वि॒तीय॒मुद॑यच्छ॒थ् सो᳚ऽब्रवी॒न्मा म॒ प्र हा॒रस्ति॒ वा इ॒दं मयि॑ वी॒र्यं॑ तत् ते॒ प्र दा᳚स्या॒मीति॒- [  ] \newline

\textbf{Pada Paata} \newline

इन्द्रः॑ । वृ॒त्राय॑ । वज्र᳚म् । उदिति॑ । अ॒य॒च्छ॒त् । सः । वृ॒त्रः । वज्रा᳚त् । उद्य॑ता॒दित्युत्-य॒ता॒त् । अ॒बि॒भे॒त् । सः । अ॒ब्र॒वी॒त् । मा । मे॒ । प्रेति॑ । हाः॒ । अस्ति॑ । वै । इ॒दम् । मयि॑ । वी॒र्य᳚म् । तत् । ते॒ । प्रेति॑ । दा॒स्या॒मि॒ । इति॑ । तस्मै᳚ । उ॒क्थ्य᳚म् । प्रेति॑ । अ॒य॒च्छ॒त् । तस्मै᳚ । द्वि॒तीय᳚म् । उदिति॑ । अ॒य॒च्छ॒त् । सः । अ॒ब्र॒वी॒त् । मा । मे॒ । प्रेति॑ । हाः॒ । अस्ति॑ । वै । इ॒दम् । मयि॑ । वी॒र्य᳚म् । तत् । ते॒ । प्रेति॑ । दा॒स्या॒मि॒ । इति॑ ।  \newline


\textbf{Krama Paata} \newline

इन्द्रो॑ वृ॒त्राय॑ । वृ॒त्राय॒ वज्र᳚म् । वज्र॒मुत् । उद॑यच्छत् । अ॒य॒च्छ॒थ् सः । स वृ॒त्रः । वृ॒त्रो वज्रा᳚त् । वज्रा॒दुद्य॑तात् । उद्य॑तादबिभेत् । उद्य॑ता॒दित्युत् - य॒ता॒त्॒ । अ॒बि॒भे॒थ् सः । सो᳚ऽब्रवीत् । अ॒ब्र॒वी॒न् मा । मा मे᳚ । मे॒ प्र । प्र हाः᳚ । हा॒रस्ति॑ । अस्ति॒ वै । वा इ॒दम् । इ॒दम् मयि॑ । मयि॑ वी॒र्य᳚म् । वी॒र्य॑म् तत् । तत् ते᳚ । ते॒ प्र । प्र दा᳚स्यामि । दा॒स्या॒मीति॑ । इति॒ तस्मै᳚ । तस्मा॑ उ॒क्थ्य᳚म् । उ॒क्थ्य॑म् प्र । प्राय॑च्छत् । अ॒य॒च्छ॒त् तस्मै᳚ । तस्मै᳚ द्वि॒तीय᳚म् । द्वि॒तीय॒मुत् । उद॑यच्छत् । अ॒य॒च्छ॒थ् सः । सो᳚ऽब्रवीत् । अ॒ब्र॒वी॒न् मा । मा मे᳚ । मे॒ प्र । प्र हाः᳚ । हा॒रस्ति॑ । अस्ति॒ वै । वा इ॒दम् । इ॒दम् मयि॑ । मयि॑ वी॒र्य᳚म् । वी॒र्य॑म् तत् । तत् ते᳚ । ते॒ प्र । प्र दा᳚स्यामि । दा॒स्या॒मीति॑ । इति॒ तस्मै᳚ \newline

\textbf{Jatai Paata} \newline

1. इन्द्रो॑ वृ॒त्राय॑ वृ॒त्रायेन्द्र॒ इन्द्रो॑ वृ॒त्राय॑ । \newline
2. वृ॒त्राय॒ वज्रं॒ ॅवज्रं॑ ॅवृ॒त्राय॑ वृ॒त्राय॒ वज्र᳚म् । \newline
3. वज्र॒ मुदुद् वज्रं॒ ॅवज्र॒ मुत् । \newline
4. उद॑यच्छ दयच्छ॒ दुदु द॑यच्छत् । \newline
5. अ॒य॒च्छ॒थ् स सो॑ ऽयच्छ दयच्छ॒थ् सः । \newline
6. स वृ॒त्रो वृ॒त्रः स स वृ॒त्रः । \newline
7. वृ॒त्रो वज्रा॒द् वज्रा᳚द् वृ॒त्रो वृ॒त्रो वज्रा᳚त् । \newline
8. वज्रा॒ दुद्य॑ता॒ दुद्य॑ता॒द् वज्रा॒द् वज्रा॒ दुद्य॑तात् । \newline
9. उद्य॑ता दबिभे दबिभे॒ दुद्य॑ता॒ दुद्य॑ता दबिभेत् । \newline
10. उद्य॑ता॒दित्युत् - य॒ता॒त् । \newline
11. अ॒बि॒भे॒थ् स सो॑ ऽबिभे दबिभे॒थ् सः । \newline
12. सो᳚ ऽब्रवी दब्रवी॒थ् स सो᳚ ऽब्रवीत् । \newline
13. अ॒ब्र॒वी॒न् मा मा ऽब्र॑वी दब्रवी॒न् मा । \newline
14. मा मे॑ मे॒ मा मा मे᳚ । \newline
15. मे॒ प्र प्र मे॑ मे॒ प्र । \newline
16. प्र हार्॑. हाः॒ प्र प्र हाः᳚ । \newline
17. हा॒ रस्त्यस्ति॑ हार्. हा॒ रस्ति॑ । \newline
18. अस्ति॒ वै वा अस्त्यस्ति॒ वै । \newline
19. वा इ॒द मि॒दं ॅवै वा इ॒दम् । \newline
20. इ॒दम् मयि॒ मयी॒द मि॒दम् मयि॑ । \newline
21. मयि॑ वी॒र्यं॑ ॅवी॒र्य॑म् मयि॒ मयि॑ वी॒र्य᳚म् । \newline
22. वी॒र्य॑म् तत् तद् वी॒र्यं॑ ॅवी॒र्य॑म् तत् । \newline
23. तत् ते॑ ते॒ तत् तत् ते᳚ । \newline
24. ते॒ प्र प्र ते॑ ते॒ प्र । \newline
25. प्र दा᳚स्यामि दास्यामि॒ प्र प्र दा᳚स्यामि । \newline
26. दा॒स्या॒ मीतीति॑ दास्यामि दास्या॒ मीति॑ । \newline
27. इति॒ तस्मै॒ तस्मा॒ इतीति॒ तस्मै᳚ । \newline
28. तस्मा॑ उ॒क्थ्य॑ मु॒क्थ्य॑म् तस्मै॒ तस्मा॑ उ॒क्थ्य᳚म् । \newline
29. उ॒क्थ्य॑म् प्र प्रोक्थ्य॑ मु॒क्थ्य॑म् प्र । \newline
30. प्राय॑च्छ दयच्छ॒त् प्र प्राय॑च्छत् । \newline
31. अ॒य॒च्छ॒त् तस्मै॒ तस्मा॑ अयच्छ दयच्छ॒त् तस्मै᳚ । \newline
32. तस्मै᳚ द्वि॒तीय॑म् द्वि॒तीय॒म् तस्मै॒ तस्मै᳚ द्वि॒तीय᳚म् । \newline
33. द्वि॒तीय॒ मुदुद् द्वि॒तीय॑म् द्वि॒तीय॒ मुत् । \newline
34. उद॑यच्छ दयच्छ॒ दुदु द॑यच्छत् । \newline
35. अ॒य॒च्छ॒थ् स सो॑ ऽयच्छ दयच्छ॒थ् सः । \newline
36. सो᳚ ऽब्रवी दब्रवी॒थ् स सो᳚ ऽब्रवीत् । \newline
37. अ॒ब्र॒वी॒न् मा मा ऽब्र॑वी दब्रवी॒न् मा । \newline
38. मा मे॑ मे॒ मा मा मे᳚ । \newline
39. मे॒ प्र प्र मे॑ मे॒ प्र । \newline
40. प्र हार्॑. हाः॒ प्र प्र हाः᳚ । \newline
41. हा॒ रस्त्यस्ति॑ हार्. हा॒ रस्ति॑ । \newline
42. अस्ति॒ वै वा अस्त्यस्ति॒ वै । \newline
43. वा इ॒द मि॒दं ॅवै वा इ॒दम् । \newline
44. इ॒दम् मयि॒ मयी॒द मि॒दम् मयि॑ । \newline
45. मयि॑ वी॒र्यं॑ ॅवी॒र्य॑म् मयि॒ मयि॑ वी॒र्य᳚म् । \newline
46. वी॒र्य॑म् तत् तद् वी॒र्यं॑ ॅवी॒र्य॑म् तत् । \newline
47. तत् ते॑ ते॒ तत् तत् ते᳚ । \newline
48. ते॒ प्र प्र ते॑ ते॒ प्र । \newline
49. प्र दा᳚स्यामि दास्यामि॒ प्र प्र दा᳚स्यामि । \newline
50. दा॒स्या॒ मीतीति॑ दास्यामि दास्या॒ मीति॑ । \newline
51. इति॒ तस्मै॒ तस्मा॒ इतीति॒ तस्मै᳚ । \newline

\textbf{Ghana Paata } \newline

1. इन्द्रो॑ वृ॒त्राय॑ वृ॒त्रायेन्द्र॒ इन्द्रो॑ वृ॒त्राय॒ वज्रं॒ ॅवज्रं॑ ॅवृ॒त्रायेन्द्र॒ इन्द्रो॑ वृ॒त्राय॒ वज्र᳚म् । \newline
2. वृ॒त्राय॒ वज्रं॒ ॅवज्रं॑ ॅवृ॒त्राय॑ वृ॒त्राय॒ वज्र॒ मुदुद् वज्रं॑ ॅवृ॒त्राय॑ वृ॒त्राय॒ वज्र॒ मुत् । \newline
3. वज्र॒ मुदुद् वज्रं॒ ॅवज्र॒ मुद॑यच्छ दयच्छ॒ दुद् वज्रं॒ ॅवज्र॒ मुद॑यच्छत् । \newline
4. उद॑यच्छ दयच्छ॒ दुदु द॑यच्छ॒थ् स सो॑ ऽयच्छ॒ दुदु द॑यच्छ॒थ् सः । \newline
5. अ॒य॒च्छ॒थ् स सो॑ ऽयच्छ दयच्छ॒थ् स वृ॒त्रो वृ॒त्रः सो॑ ऽयच्छ दयच्छ॒थ् स वृ॒त्रः । \newline
6. स वृ॒त्रो वृ॒त्रः स स वृ॒त्रो वज्रा॒द् वज्रा᳚द् वृ॒त्रः स स वृ॒त्रो वज्रा᳚त् । \newline
7. वृ॒त्रो वज्रा॒द् वज्रा᳚द् वृ॒त्रो वृ॒त्रो वज्रा॒ दुद्य॑ता॒ दुद्य॑ता॒द् वज्रा᳚द् वृ॒त्रो वृ॒त्रो वज्रा॒ दुद्य॑तात् । \newline
8. वज्रा॒ दुद्य॑ता॒ दुद्य॑ता॒द् वज्रा॒द् वज्रा॒ दुद्य॑ता दबिभे दबिभे॒ दुद्य॑ता॒द् वज्रा॒द् वज्रा॒ दुद्य॑ता दबिभेत् । \newline
9. उद्य॑ता दबिभे दबिभे॒ दुद्य॑ता॒ दुद्य॑ता दबिभे॒थ् स सो॑ ऽबिभे॒ दुद्य॑ता॒ दुद्य॑ता दबिभे॒थ् सः । \newline
10. उद्य॑ता॒दित्युत् - य॒ता॒त् । \newline
11. अ॒बि॒भे॒थ् स सो॑ ऽबिभे दबिभे॒थ् सो᳚ ऽब्रवी दब्रवी॒थ् सो॑ ऽबिभे दबिभे॒थ् सो᳚ ऽब्रवीत् । \newline
12. सो᳚ ऽब्रवी दब्रवी॒थ् स सो᳚ ऽब्रवी॒न् मा मा ऽब्र॑वी॒थ् स सो᳚ ऽब्रवी॒न् मा । \newline
13. अ॒ब्र॒वी॒न् मा मा ऽब्र॑वी दब्रवी॒न् मा मे॑ मे॒ मा ऽब्र॑वी दब्रवी॒न् मा मे᳚ । \newline
14. मा मे॑ मे॒ मा मा मे॒ प्र प्र मे॒ मा मा मे॒ प्र । \newline
15. मे॒ प्र प्र मे॑ मे॒ प्र हार्॑. हाः॒ प्र मे॑ मे॒ प्र हाः᳚ । \newline
16. प्र हार्॑. हाः॒ प्र प्र हा॒ रस्त्यस्ति॑ हाः॒ प्र प्र हा॒ रस्ति॑ । \newline
17. हा॒ रस्त्यस्ति॑ हार्. हा॒ रस्ति॒ वै वा अस्ति॑ हार्. हा॒ रस्ति॒ वै । \newline
18. अस्ति॒ वै वा अस्त्यस्ति॒ वा इ॒द मि॒दं ॅवा अस्त्यस्ति॒ वा इ॒दम् । \newline
19. वा इ॒द मि॒दं ॅवै वा इ॒दम् मयि॒ मयी॒दं ॅवै वा इ॒दम् मयि॑ । \newline
20. इ॒दम् मयि॒ मयी॒द मि॒दम् मयि॑ वी॒र्यं॑ ॅवी॒र्य॑म् मयी॒द मि॒दम् मयि॑ वी॒र्य᳚म् । \newline
21. मयि॑ वी॒र्यं॑ ॅवी॒र्य॑म् मयि॒ मयि॑ वी॒र्य॑म् तत् तद् वी॒र्य॑म् मयि॒ मयि॑ वी॒र्य॑म् तत् । \newline
22. वी॒र्य॑म् तत् तद् वी॒र्यं॑ ॅवी॒र्य॑म् तत् ते॑ ते॒ तद् वी॒र्यं॑ ॅवी॒र्य॑म् तत् ते᳚ । \newline
23. तत् ते॑ ते॒ तत् तत् ते॒ प्र प्र ते॒ तत् तत् ते॒ प्र । \newline
24. ते॒ प्र प्र ते॑ ते॒ प्र दा᳚स्यामि दास्यामि॒ प्र ते॑ ते॒ प्र दा᳚स्यामि । \newline
25. प्र दा᳚स्यामि दास्यामि॒ प्र प्र दा᳚स्या॒ मीतीति॑ दास्यामि॒ प्र प्र दा᳚स्या॒ मीति॑ । \newline
26. दा॒स्या॒ मीतीति॑ दास्यामि दास्या॒ मीति॒ तस्मै॒ तस्मा॒ इति॑ दास्यामि दास्या॒ मीति॒ तस्मै᳚ । \newline
27. इति॒ तस्मै॒ तस्मा॒ इतीति॒ तस्मा॑ उ॒क्थ्य॑ मु॒क्थ्य॑म् तस्मा॒ इतीति॒ तस्मा॑ उ॒क्थ्य᳚म् । \newline
28. तस्मा॑ उ॒क्थ्य॑ मु॒क्थ्य॑म् तस्मै॒ तस्मा॑ उ॒क्थ्य॑म् प्र प्रोक्थ्य॑म् तस्मै॒ तस्मा॑ उ॒क्थ्य॑म् प्र । \newline
29. उ॒क्थ्य॑म् प्र प्रोक्थ्य॑ मु॒क्थ्य॑म् प्राय॑च्छ दयच्छ॒त् प्रोक्थ्य॑ मु॒क्थ्य॑म् प्राय॑च्छत् । \newline
30. प्राय॑च्छ दयच्छ॒त् प्र प्राय॑च्छ॒त् तस्मै॒ तस्मा॑ अयच्छ॒त् प्र प्राय॑च्छ॒त् तस्मै᳚ । \newline
31. अ॒य॒च्छ॒त् तस्मै॒ तस्मा॑ अयच्छ दयच्छ॒त् तस्मै᳚ द्वि॒तीय॑म् द्वि॒तीय॒म् तस्मा॑ अयच्छ दयच्छ॒त् तस्मै᳚ द्वि॒तीय᳚म् । \newline
32. तस्मै᳚ द्वि॒तीय॑म् द्वि॒तीय॒म् तस्मै॒ तस्मै᳚ द्वि॒तीय॒ मुदुद् द्वि॒तीय॒म् तस्मै॒ तस्मै᳚ द्वि॒तीय॒ मुत् । \newline
33. द्वि॒तीय॒ मुदुद् द्वि॒तीय॑म् द्वि॒तीय॒ मुद॑यच्छ दयच्छ॒ दुद् द्वि॒तीय॑म् द्वि॒तीय॒ मुद॑यच्छत् । \newline
34. उद॑यच्छ दयच्छ॒ दुदु द॑यच्छ॒थ् स सो॑ ऽयच्छ॒ दुदु द॑यच्छ॒थ् सः । \newline
35. अ॒य॒च्छ॒थ् स सो॑ ऽयच्छ दयच्छ॒थ् सो᳚ ऽब्रवी दब्रवी॒थ् सो॑ ऽयच्छ दयच्छ॒थ् सो᳚ ऽब्रवीत् । \newline
36. सो᳚ ऽब्रवी दब्रवी॒थ् स सो᳚ ऽब्रवी॒न् मा मा ऽब्र॑वी॒थ् स सो᳚ ऽब्रवी॒न् मा । \newline
37. अ॒ब्र॒वी॒न् मा मा ऽब्र॑वी दब्रवी॒न् मा मे॑ मे॒ मा ऽब्र॑वी दब्रवी॒न् मा मे᳚ । \newline
38. मा मे॑ मे॒ मा मा मे॒ प्र प्र मे॒ मा मा मे॒ प्र । \newline
39. मे॒ प्र प्र मे॑ मे॒ प्र हार्॑. हाः॒ प्र मे॑ मे॒ प्र हाः᳚ । \newline
40. प्र हार्॑. हाः॒ प्र प्र हा॒ रस्त्यस्ति॑ हाः॒ प्र प्र हा॒ रस्ति॑ । \newline
41. हा॒ रस्त्यस्ति॑ हार्. हा॒ रस्ति॒ वै वा अस्ति॑ हार्. हा॒ रस्ति॒ वै । \newline
42. अस्ति॒ वै वा अस्त्यस्ति॒ वा इ॒द मि॒दं ॅवा अस्त्यस्ति॒ वा इ॒दम् । \newline
43. वा इ॒द मि॒दं ॅवै वा इ॒दम् मयि॒ मयी॒दं ॅवै वा इ॒दम् मयि॑ । \newline
44. इ॒दम् मयि॒ मयी॒द मि॒दम् मयि॑ वी॒र्यं॑ ॅवी॒र्य॑म् मयी॒द मि॒दम् मयि॑ वी॒र्य᳚म् । \newline
45. मयि॑ वी॒र्यं॑ ॅवी॒र्य॑म् मयि॒ मयि॑ वी॒र्य॑म् तत् तद् वी॒र्य॑म् मयि॒ मयि॑ वी॒र्य॑म् तत् । \newline
46. वी॒र्य॑म् तत् तद् वी॒र्यं॑ ॅवी॒र्य॑म् तत् ते॑ ते॒ तद् वी॒र्यं॑ ॅवी॒र्य॑म् तत् ते᳚ । \newline
47. तत् ते॑ ते॒ तत् तत् ते॒ प्र प्र ते॒ तत् तत् ते॒ प्र । \newline
48. ते॒ प्र प्र ते॑ ते॒ प्र दा᳚स्यामि दास्यामि॒ प्र ते॑ ते॒ प्र दा᳚स्यामि । \newline
49. प्र दा᳚स्यामि दास्यामि॒ प्र प्र दा᳚स्या॒ मीतीति॑ दास्यामि॒ प्र प्र दा᳚स्या॒ मीति॑ । \newline
50. दा॒स्या॒ मीतीति॑ दास्यामि दास्या॒ मीति॒ तस्मै॒ तस्मा॒ इति॑ दास्यामि दास्या॒ मीति॒ तस्मै᳚ । \newline
51. इति॒ तस्मै॒ तस्मा॒ इतीति॒ तस्मा॑ उ॒क्थ्य॑ मु॒क्थ्य॑म् तस्मा॒ इतीति॒ तस्मा॑ उ॒क्थ्य᳚म् । \newline
\pagebreak
\markright{ TS 6.5.1.2  \hfill https://www.vedavms.in \hfill}

\section{ TS 6.5.1.2 }

\textbf{TS 6.5.1.2 } \newline
\textbf{Samhita Paata} \newline

तस्मा॑ उ॒क्थ्य॑मे॒व प्राय॑च्छ॒त् तस्मै॑ तृ॒तीय॒मुद॑यच्छ॒त् तं ॅविष्णु॒रन्व॑तिष्ठत ज॒हीति॒ सो᳚ऽब्रवी॒न्मा मे॒ प्र हा॒रस्ति॒ वा इ॒दं मयि॑ वी॒र्यं॑ तत् ते॒ प्र दा᳚स्या॒मीति॒ तस्मा॑ उ॒क्थ्य॑मे॒व प्राय॑च्छ॒त् तं निर्मा॑यं भू॒तम॑हन्. य॒ज्ञो हि तस्य॑ मा॒याऽऽसी॒द्-यदु॒क्थ्यो॑ गृ॒ह्यत॑ इन्द्रि॒यमे॒व- [  ] \newline

\textbf{Pada Paata} \newline

तस्मै᳚ । उ॒क्थ्य᳚म् । ए॒व । प्रेति॑ । अ॒य॒च्छ॒त् । तस्मै᳚ । तृ॒तीय᳚म् । उदिति॑ । अ॒य॒च्छ॒त् । तम् । विष्णुः॑ । अन्विति॑ । अ॒ति॒ष्ठ॒त॒ । ज॒हि । इति॑ । सः । अ॒ब्र॒वी॒त् । मा । मे॒ । प्रेति॑ । हाः॒ । अस्ति॑ । वै । इ॒दम् । मयि॑ । वी॒र्य᳚म् । तत् । ते॒ । प्रेति॑ । दा॒स्या॒मि॒ । इति॑ । तस्मै᳚ । उ॒क्थ्य᳚म् । ए॒व । प्रेति॑ । अ॒य॒च्छ॒त् । तम् । निर्मा॑य॒मिति॒ निः-मा॒य॒म् । भू॒तम् । अ॒ह॒न्न् । य॒ज्ञ्ः । हि । तस्य॑ । मा॒या । आसी᳚त् । यत् । उ॒क्थ्यः॑ । गृ॒ह्यते᳚ । इ॒न्द्रि॒यम् । ए॒व ।  \newline


\textbf{Krama Paata} \newline

तस्मा॑ उ॒क्थ्य᳚म् । उ॒क्थ्य॑मे॒व । ए॒व प्र । प्राय॑च्छत् । अ॒य॒च्छ॒त् तस्मै᳚ । तस्मै॑ तृ॒तीय᳚म् । तृ॒तीय॒मुत् । उद॑यच्छत् । अ॒य॒च्छ॒त् तम् । तम् ॅविष्णुः॑ । विष्णु॒रनु॑ । अन्व॑तिष्ठत । अ॒ति॒ष्ठ॒त॒ ज॒हि । ज॒हीति॑ । इति॒ सः । सो᳚ऽब्रवीत् । अ॒ब्र॒वी॒न् मा । मा मे᳚ । मे॒ प्र । प्र हाः᳚ । हा॒रस्ति॑ । अस्ति॒ वै । वा इ॒दम् । इ॒दम् मयि॑ । मयि॑ वी॒र्य᳚म् । वी॒र्य॑म् तत् । तत्ते᳚ । ते॒ प्र । प्र दा᳚स्यामि । दा॒स्या॒मीति॑ । इति॒ तस्मै᳚ । तस्मा॑ उ॒क्थ्य᳚म् । उ॒क्थ्य॑मे॒व । ए॒व प्र । प्राय॑च्छत् । अ॒य॒च्छ॒त् तम् । तम् निर्मा॑यम् । निर्मा॑यम् भू॒तम् । निर्मा॑य॒मिति॒ निः - मा॒य॒म् । भू॒तम॑हन्न् । अ॒ह॒न्॒. य॒ज्ञ्ः । य॒ज्ञो हि । हि तस्य॑ । तस्य॑ मा॒या । मा॒याऽऽसी᳚त् । आसी॒द् यत् । यदु॒क्थ्यः॑ । उ॒क्थ्यो॑ गृ॒ह्यते᳚ । गृ॒ह्यत॑ इन्द्रि॒यम् । इ॒न्द्रि॒यमे॒व । ए॒व तत् \newline

\textbf{Jatai Paata} \newline

1. तस्मा॑ उ॒क्थ्य॑ मु॒क्थ्य॑म् तस्मै॒ तस्मा॑ उ॒क्थ्य᳚म् । \newline
2. उ॒क्थ्य॑ मे॒वै वोक्थ्य॑ मु॒क्थ्य॑ मे॒व । \newline
3. ए॒व प्र प्रैवैव प्र । \newline
4. प्राय॑च्छ दयच्छ॒त् प्र प्राय॑च्छत् । \newline
5. अ॒य॒च्छ॒त् तस्मै॒ तस्मा॑ अयच्छ दयच्छ॒त् तस्मै᳚ । \newline
6. तस्मै॑ तृ॒तीय॑म् तृ॒तीय॒म् तस्मै॒ तस्मै॑ तृ॒तीय᳚म् । \newline
7. तृ॒तीय॒ मुदुत् तृ॒तीय॑म् तृ॒तीय॒ मुत् । \newline
8. उद॑यच्छ दयच्छ॒ दुदु द॑यच्छत् । \newline
9. अ॒य॒च्छ॒त् तम् त म॑यच्छ दयच्छ॒त् तम् । \newline
10. तं ॅविष्णु॒र् विष्णु॒ स्तम् तं ॅविष्णुः॑ । \newline
11. विष्णु॒ रन्वनु॒ विष्णु॒र् विष्णु॒ रनु॑ । \newline
12. अन्व॑तिष्ठता तिष्ठ॒ता न्वन् व॑तिष्ठत । \newline
13. अ॒ति॒ष्ठ॒त॒ ज॒हि ज॒ह्य॑ तिष्ठता तिष्ठत ज॒हि । \newline
14. ज॒ही तीति॑ ज॒हि ज॒हीति॑ । \newline
15. इति॒ स स इतीति॒ सः । \newline
16. सो᳚ ऽब्रवी दब्रवी॒थ् स सो᳚ ऽब्रवीत् । \newline
17. अ॒ब्र॒वी॒न् मा मा ऽब्र॑वी दब्रवी॒न् मा । \newline
18. मा मे॑ मे॒ मा मा मे᳚ । \newline
19. मे॒ प्र प्र मे॑ मे॒ प्र । \newline
20. प्र हार्॑. हाः॒ प्र प्र हाः᳚ । \newline
21. हा॒ रस्त्यस्ति॑ हार्. हा॒ रस्ति॑ । \newline
22. अस्ति॒ वै वा अस्त्यस्ति॒ वै । \newline
23. वा इ॒द मि॒दं ॅवै वा इ॒दम् । \newline
24. इ॒दम् मयि॒ मयी॒द मि॒दम् मयि॑ । \newline
25. मयि॑ वी॒र्यं॑ ॅवी॒र्य॑म् मयि॒ मयि॑ वी॒र्य᳚म् । \newline
26. वी॒र्य॑म् तत् तद् वी॒र्यं॑ ॅवी॒र्य॑म् तत् । \newline
27. तत् ते॑ ते॒ तत् तत् ते᳚ । \newline
28. ते॒ प्र प्र ते॑ ते॒ प्र । \newline
29. प्र दा᳚स्यामि दास्यामि॒ प्र प्र दा᳚स्यामि । \newline
30. दा॒स्या॒ मीतीति॑ दास्यामि दास्या॒ मीति॑ । \newline
31. इति॒ तस्मै॒ तस्मा॒ इतीति॒ तस्मै᳚ । \newline
32. तस्मा॑ उ॒क्थ्य॑ मु॒क्थ्य॑म् तस्मै॒ तस्मा॑ उ॒क्थ्य᳚म् । \newline
33. उ॒क्थ्य॑ मे॒वै वोक्थ्य॑ मु॒क्थ्य॑ मे॒व । \newline
34. ए॒व प्र प्रैवैव प्र । \newline
35. प्राय॑च्छ दयच्छ॒त् प्र प्राय॑च्छत् । \newline
36. अ॒य॒च्छ॒त् तम् त म॑यच्छ दयच्छ॒त् तम् । \newline
37. तन् निर्मा॑य॒म् निर्मा॑य॒म् तम् तन् निर्मा॑यम् । \newline
38. निर्मा॑यम् भू॒तम् भू॒तन् निर्मा॑य॒म् निर्मा॑यम् भू॒तम् । \newline
39. निर्मा॑य॒मिति॒ निः - मा॒य॒म् । \newline
40. भू॒त म॑हन् नहन् भू॒तम् भू॒त म॑हन्न् । \newline
41. अ॒ह॒न्॒. य॒ज्ञो य॒ज्ञो॑ ऽहन् नहन्. य॒ज्ञ्ः । \newline
42. य॒ज्ञो हि हि य॒ज्ञो य॒ज्ञो हि । \newline
43. हि तस्य॒ तस्य॒ हि हि तस्य॑ । \newline
44. तस्य॑ मा॒या मा॒या तस्य॒ तस्य॑ मा॒या । \newline
45. मा॒या ऽऽसी॒ दासी᳚न् मा॒या मा॒या ऽऽसी᳚त् । \newline
46. आसी॒द् यद् यदासी॒ दासी॒द् यत् । \newline
47. यदु॒क्थ्य॑ उ॒क्थ्यो॑ यद् यदु॒क्थ्यः॑ । \newline
48. उ॒क्थ्यो॑ गृ॒ह्यते॑ गृ॒ह्यत॑ उ॒क्थ्य॑ उ॒क्थ्यो॑ गृ॒ह्यते᳚ । \newline
49. गृ॒ह्यत॑ इन्द्रि॒य मि॑न्द्रि॒यम् गृ॒ह्यते॑ गृ॒ह्यत॑ इन्द्रि॒यम् । \newline
50. इ॒न्द्रि॒य मे॒वै वेन्द्रि॒य मि॑न्द्रि॒य मे॒व । \newline
51. ए॒व तत् तदे॒ वैव तत् । \newline

\textbf{Ghana Paata } \newline

1. तस्मा॑ उ॒क्थ्य॑ मु॒क्थ्य॑म् तस्मै॒ तस्मा॑ उ॒क्थ्य॑ मे॒वै वोक्थ्य॑म् तस्मै॒ तस्मा॑ उ॒क्थ्य॑ मे॒व । \newline
2. उ॒क्थ्य॑ मे॒वै वोक्थ्य॑ मु॒क्थ्य॑ मे॒व प्र प्रैवोक्थ्य॑ मु॒क्थ्य॑ मे॒व प्र । \newline
3. ए॒व प्र प्रैवैव प्राय॑च्छ दयच्छ॒त् प्रैवैव प्राय॑च्छत् । \newline
4. प्राय॑च्छ दयच्छ॒त् प्र प्राय॑च्छ॒त् तस्मै॒ तस्मा॑ अयच्छ॒त् प्र प्राय॑च्छ॒त् तस्मै᳚ । \newline
5. अ॒य॒च्छ॒त् तस्मै॒ तस्मा॑ अयच्छ दयच्छ॒त् तस्मै॑ तृ॒तीय॑म् तृ॒तीय॒म् तस्मा॑ अयच्छ दयच्छ॒त् तस्मै॑ तृ॒तीय᳚म् । \newline
6. तस्मै॑ तृ॒तीय॑म् तृ॒तीय॒म् तस्मै॒ तस्मै॑ तृ॒तीय॒ मुदुत् तृ॒तीय॒म् तस्मै॒ तस्मै॑ तृ॒तीय॒ मुत् । \newline
7. तृ॒तीय॒ मुदुत् तृ॒तीय॑म् तृ॒तीय॒ मुद॑यच्छ दयच्छ॒ दुत् तृ॒तीय॑म् तृ॒तीय॒ मुद॑यच्छत् । \newline
8. उद॑यच्छ दयच्छ॒ दुदु द॑यच्छ॒त् तम् त म॑यच्छ॒ दुदु द॑यच्छ॒त् तम् । \newline
9. अ॒य॒च्छ॒त् तम् त म॑यच्छ दयच्छ॒त् तं ॅविष्णु॒र् विष्णु॒ स्त म॑यच्छ दयच्छ॒त् तं ॅविष्णुः॑ । \newline
10. तं ॅविष्णु॒र् विष्णु॒ स्तम् तं ॅविष्णु॒ रन्वनु॒ विष्णु॒ स्तम् तं ॅविष्णु॒ रनु॑ । \newline
11. विष्णु॒ रन्वनु॒ विष्णु॒र् विष्णु॒ रन् व॑तिष्ठ तातिष्ठ॒ तानु॒ विष्णु॒र् विष्णु॒ रन् व॑तिष्ठत । \newline
12. अन् व॑तिष्ठता तिष्ठ॒ तान् वन् व॑तिष्ठत ज॒हि ज॒ह्य॑तिष्ठ॒ तान् वन् व॑तिष्ठत ज॒हि । \newline
13. अ॒ति॒ष्ठ॒त॒ ज॒हि ज॒ह्य॑तिष्ठता तिष्ठत ज॒हीतीति॑ ज॒ह्य॑तिष्ठता तिष्ठत ज॒हीति॑ । \newline
14. ज॒हीतीति॑ ज॒हि ज॒हीति॒ स स इति॑ ज॒हि ज॒हीति॒ सः । \newline
15. इति॒ स स इतीति॒ सो᳚ ऽब्रवी दब्रवी॒थ् स इतीति॒ सो᳚ ऽब्रवीत् । \newline
16. सो᳚ ऽब्रवी दब्रवी॒थ् स सो᳚ ऽब्रवी॒न् मा मा ऽब्र॑वी॒थ् स सो᳚ ऽब्रवी॒न् मा । \newline
17. अ॒ब्र॒वी॒न् मा मा ऽब्र॑वी दब्रवी॒न् मा मे॑ मे॒ मा ऽब्र॑वी दब्रवी॒न् मा मे᳚ । \newline
18. मा मे॑ मे॒ मा मा मे॒ प्र प्र मे॒ मा मा मे॒ प्र । \newline
19. मे॒ प्र प्र मे॑ मे॒ प्र हार्॑. हाः॒ प्र मे॑ मे॒ प्र हाः᳚ । \newline
20. प्र हार्॑. हाः॒ प्र प्र हा॒ रस्त्यस्ति॑ हाः॒ प्र प्र हा॒ रस्ति॑ । \newline
21. हा॒ रस्त्यस्ति॑ हार्. हा॒ रस्ति॒ वै वा अस्ति॑ हार्. हा॒ रस्ति॒ वै । \newline
22. अस्ति॒ वै वा अस्त्यस्ति॒ वा इ॒द मि॒दं ॅवा अस्त्यस्ति॒ वा इ॒दम् । \newline
23. वा इ॒द मि॒दं ॅवै वा इ॒दम् मयि॒ मयी॒दं ॅवै वा इ॒दम् मयि॑ । \newline
24. इ॒दम् मयि॒ मयी॒द मि॒दम् मयि॑ वी॒र्यं॑ ॅवी॒र्य॑म् मयी॒द मि॒दम् मयि॑ वी॒र्य᳚म् । \newline
25. मयि॑ वी॒र्यं॑ ॅवी॒र्य॑म् मयि॒ मयि॑ वी॒र्य॑म् तत् तद् वी॒र्य॑म् मयि॒ मयि॑ वी॒र्य॑म् तत् । \newline
26. वी॒र्य॑म् तत् तद् वी॒र्यं॑ ॅवी॒र्य॑म् तत् ते॑ ते॒ तद् वी॒र्यं॑ ॅवी॒र्य॑म् तत् ते᳚ । \newline
27. तत् ते॑ ते॒ तत् तत् ते॒ प्र प्र ते॒ तत् तत् ते॒ प्र । \newline
28. ते॒ प्र प्र ते॑ ते॒ प्र दा᳚स्यामि दास्यामि॒ प्र ते॑ ते॒ प्र दा᳚स्यामि । \newline
29. प्र दा᳚स्यामि दास्यामि॒ प्र प्र दा᳚स्या॒ मीतीति॑ दास्यामि॒ प्र प्र दा᳚स्या॒ मीति॑ । \newline
30. दा॒स्या॒ मीतीति॑ दास्यामि दास्या॒ मीति॒ तस्मै॒ तस्मा॒ इति॑ दास्यामि दास्या॒ मीति॒ तस्मै᳚ । \newline
31. इति॒ तस्मै॒ तस्मा॒ इतीति॒ तस्मा॑ उ॒क्थ्य॑ मु॒क्थ्य॑म् तस्मा॒ इतीति॒ तस्मा॑ उ॒क्थ्य᳚म् । \newline
32. तस्मा॑ उ॒क्थ्य॑ मु॒क्थ्य॑म् तस्मै॒ तस्मा॑ उ॒क्थ्य॑ मे॒वै वोक्थ्य॑म् तस्मै॒ तस्मा॑ उ॒क्थ्य॑ मे॒व । \newline
33. उ॒क्थ्य॑ मे॒वै वोक्थ्य॑ मु॒क्थ्य॑ मे॒व प्र प्रैवोक्थ्य॑ मु॒क्थ्य॑ मे॒व प्र । \newline
34. ए॒व प्र प्रैवैव प्राय॑च्छ दयच्छ॒त् प्रैवैव प्राय॑च्छत् । \newline
35. प्राय॑च्छ दयच्छ॒त् प्र प्राय॑च्छ॒त् तम् त म॑यच्छ॒त् प्र प्राय॑च्छ॒त् तम् । \newline
36. अ॒य॒च्छ॒त् तम् त म॑यच्छ दयच्छ॒त् तन् निर्मा॑य॒म् निर्मा॑य॒म् त म॑यच्छ दयच्छ॒त् तन् निर्मा॑यम् । \newline
37. तन् निर्मा॑य॒म् निर्मा॑य॒म् तम् तन् निर्मा॑यम् भू॒तम् भू॒तन् निर्मा॑य॒म् तम् तन् निर्मा॑यम् भू॒तम् । \newline
38. निर्मा॑यम् भू॒तम् भू॒तम् निर्मा॑य॒म् निर्मा॑यम् भू॒त म॑हन्-नहन् भू॒तम् निर्मा॑य॒म् निर्मा॑यम् भू॒त म॑हन्न् । \newline
39. निर्मा॑य॒मिति॒ निः - मा॒य॒म् । \newline
40. भू॒त म॑हन्-नहन् भू॒तम् भू॒त म॑हन्. य॒ज्ञो य॒ज्ञो॑ ऽहन् भू॒तम् भू॒त म॑हन्. य॒ज्ञ्ः । \newline
41. अ॒ह॒न्॒. य॒ज्ञो य॒ज्ञो॑ ऽहन्-नहन्. य॒ज्ञो हि हि य॒ज्ञो॑ ऽहन्-नहन्. य॒ज्ञो हि । \newline
42. य॒ज्ञो हि हि य॒ज्ञो य॒ज्ञो हि तस्य॒ तस्य॒ हि य॒ज्ञो य॒ज्ञो हि तस्य॑ । \newline
43. हि तस्य॒ तस्य॒ हि हि तस्य॑ मा॒या मा॒या तस्य॒ हि हि तस्य॑ मा॒या । \newline
44. तस्य॑ मा॒या मा॒या तस्य॒ तस्य॑ मा॒या ऽऽसी॒ दासी᳚न् मा॒या तस्य॒ तस्य॑ मा॒या ऽऽसी᳚त् । \newline
45. मा॒या ऽऽसी॒ दासी᳚न् मा॒या मा॒या ऽऽसी॒द् यद् यदासी᳚न् मा॒या मा॒या ऽऽसी॒द् यत् । \newline
46. आसी॒द् यद् यदासी॒ दासी॒द् यदु॒क्थ्य॑ उ॒क्थ्यो॑ यदासी॒ दासी॒द् यदु॒क्थ्यः॑ । \newline
47. यदु॒क्थ्य॑ उ॒क्थ्यो॑ यद् यदु॒क्थ्यो॑ गृ॒ह्यते॑ गृ॒ह्यत॑ उ॒क्थ्यो॑ यद् यदु॒क्थ्यो॑ गृ॒ह्यते᳚ । \newline
48. उ॒क्थ्यो॑ गृ॒ह्यते॑ गृ॒ह्यत॑ उ॒क्थ्य॑ उ॒क्थ्यो॑ गृ॒ह्यत॑ इन्द्रि॒य मि॑न्द्रि॒यम् गृ॒ह्यत॑ उ॒क्थ्य॑ उ॒क्थ्यो॑ गृ॒ह्यत॑ इन्द्रि॒यम् । \newline
49. गृ॒ह्यत॑ इन्द्रि॒य मि॑न्द्रि॒यम् गृ॒ह्यते॑ गृ॒ह्यत॑ इन्द्रि॒य मे॒वै वेन्द्रि॒यम् गृ॒ह्यते॑ गृ॒ह्यत॑ इन्द्रि॒य मे॒व । \newline
50. इ॒न्द्रि॒य मे॒वैवेन्द्रि॒य मि॑न्द्रि॒य मे॒व तत् तदे॒ वेन्द्रि॒य मि॑न्द्रि॒य मे॒व तत् । \newline
51. ए॒व तत् तदे॒ वैव तद् वी॒र्यं॑ ॅवी॒र्य॑म् तदे॒ वैव तद् वी॒र्य᳚म् । \newline
\pagebreak
\markright{ TS 6.5.1.3  \hfill https://www.vedavms.in \hfill}

\section{ TS 6.5.1.3 }

\textbf{TS 6.5.1.3 } \newline
\textbf{Samhita Paata} \newline

तद् वी॒र्यं॑ ॅयज॑मानो॒ भ्रातृ॑व्यस्य वृङ्क्त॒ इन्द्रा॑य त्वा बृ॒हद्-व॑ते॒ वय॑स्वत॒ इत्या॒हेन्द्रा॑य॒ हि स तं प्राय॑च्छ॒त् तस्मै᳚ त्वा॒ विष्ण॑वे॒ त्वेत्या॑ह॒ यदे॒व विष्णु॑र॒न्वति॑ष्ठत ज॒हीति॒ तस्मा॒द्-विष्णु॑म॒न्वाभ॑जति॒ त्रिर्निर्गृ॑ह्णाति॒ त्रिर्.हि स तं तस्मै॒ प्राय॑च्छदे॒ष ते॒ योनिः॒ पुन॑र्.हविर॒सीत्या॑ह॒ पुनः॑पुन॒- [  ] \newline

\textbf{Pada Paata} \newline

तत् । वी॒र्य᳚म् । यज॑मानः । भ्रातृ॑व्यस्य । वृ॒ङ्क्ते॒ । इन्द्रा॑य । त्वा॒ । बृ॒हद्व॑त॒ इति॑ बृ॒हत् - व॒ते॒ । वय॑स्वते । इति॑ । आ॒ह॒ । इन्द्रा॑य । हि । सः । तम् । प्रेति॑ । अय॑च्छत् । तस्मै᳚ । त्वा॒ । विष्ण॑वे । त्वा॒ । इति॑ । आ॒ह॒ । यत् । ए॒व । विष्णुः॑ । अ॒न्वति॑ष्ठ॒तेत्य॑नु - अति॑ष्ठत । ज॒हि । इति॑ । तस्मा᳚त् । विष्णु᳚म् । अ॒न्वाभ॑ज॒तीत्य॑नु - आभ॑जति । त्रिः । निरिति॑ । गृ॒ह्णा॒ति॒ । त्रिः । हि । सः । तम् । तस्मै᳚ । प्रेति॑ । अय॑च्छत् । ए॒षः । ते॒ । योनिः॑ । पुन॑र्.हवि॒रिति॒ पुनः॑ - ह॒विः॒ । अ॒सि॒ । इति॑ । आ॒ह॒ । पुनः॑पुन॒रिति॒ पुनः॑ - पु॒नः॒ ।  \newline


\textbf{Krama Paata} \newline

तद् वी॒र्य᳚म् । वी॒र्य॑म् ॅयज॑मानः । यज॑मानो॒ भ्रातृ॑व्यस्य । भ्रातृ॑व्यस्य वृङ्‍क्ते । वृ॒ङ्‍क्त॒ इन्द्रा॑य । इन्द्रा॑य त्वा । त्वा॒ बृ॒हद्व॑ते । बृ॒हद्‍व॑ते॒ वय॑स्वते । बृ॒हद्‍व॑त॒ इति॑ बृ॒हत् - व॒ते॒ । वय॑स्वत॒ इति॑ । इत्या॑ह । आ॒हेन्द्रा॑य । इन्द्रा॑य॒ हि । हि सः । स तम् । तम् प्र । प्राय॑च्छत् । अय॑च्छ॒त् तस्मै᳚ । तस्मै᳚ त्वा । त्वा॒ विष्ण॑वे । विष्ण॑वे त्वा । त्वेति॑ । इत्या॑ह । आ॒ह॒ यत् । यदे॒व । ए॒व विष्णुः॑ । विष्णु॑र॒न्वति॑ष्ठत । अ॒न्वति॑ष्ठत ज॒हि । अ॒न्वति॑ष्ठ॒तेत्य॑नु - अति॑ष्ठत । ज॒हीति॑ । इति॒ तस्मा᳚त् । तस्मा॒द् विष्णु᳚म् । विष्णु॑म॒न्वाभ॑जति । अ॒न्वाभ॑जति॒ त्रिः । अ॒न्वाभ॑ज॒तीत्य॑नु - आभ॑जति । त्रिर् निः । निर् गृ॑ह्णाति । गृ॒ह्णा॒ति॒ त्रिः । त्रिर्. हि । हि सः । स तम् । तम् तस्मै᳚ । तस्मै॒ प्र । प्राय॑च्छत् । अय॑च्छदे॒षः । ए॒ष ते᳚ । ते॒ योनिः॑ । योनिः॒ पुन॑र्.हविः । पुन॑र्.हविरसि । पुन॑र्.हवि॒रिति॒ पुनः॑ - ह॒विः॒ । अ॒सीति॑ । इत्या॑ह । आ॒ह॒ पुनः॑पुनः । पुनः॑पुन॒र्.॒ हि । पुनः॑पुन॒रिति॒ पुनः॑ - पु॒नः॒ \newline

\textbf{Jatai Paata} \newline

1. तद् वी॒र्यं॑ ॅवी॒र्य॑म् तत् तद् वी॒र्य᳚म् । \newline
2. वी॒र्यं॑ ॅयज॑मानो॒ यज॑मानो वी॒र्यं॑ ॅवी॒र्यं॑ ॅयज॑मानः । \newline
3. यज॑मानो॒ भ्रातृ॑व्यस्य॒ भ्रातृ॑व्यस्य॒ यज॑मानो॒ यज॑मानो॒ भ्रातृ॑व्यस्य । \newline
4. भ्रातृ॑व्यस्य वृङ्क्ते वृङ्क्ते॒ भ्रातृ॑व्यस्य॒ भ्रातृ॑व्यस्य वृङ्क्ते । \newline
5. वृ॒ङ्क्त॒ इन्द्रा॒ येन्द्रा॑य वृङ्क्ते वृङ्क्त॒ इन्द्रा॑य । \newline
6. इन्द्रा॑य त्वा॒ त्वेन्द्रा॒ येन्द्रा॑य त्वा । \newline
7. त्वा॒ बृ॒हद्व॑ते बृ॒हद्व॑ते त्वा त्वा बृ॒हद्व॑ते । \newline
8. बृ॒हद्व॑ते॒ वय॑स्वते॒ वय॑स्वते बृ॒हद्व॑ते बृ॒हद्व॑ते॒ वय॑स्वते । \newline
9. बृ॒हद्व॑त॒ इति॑ बृ॒हत् - व॒ते॒ । \newline
10. वय॑स्वत॒ इतीति॒ वय॑स्वते॒ वय॑स्वत॒ इति॑ । \newline
11. इत्या॑हा॒हे तीत्या॑ह । \newline
12. आ॒हेन्द्रा॒ येन्द्रा॑या हा॒हेन्द्रा॑य । \newline
13. इन्द्रा॑य॒ हि हीन्द्रा॒ येन्द्रा॑य॒ हि । \newline
14. हि स स हि हि सः । \newline
15. स तम् तꣳ स स तम् । \newline
16. तम् प्र प्र तम् तम् प्र । \newline
17. प्राय॑च्छ॒ दय॑च्छ॒त् प्र प्राय॑च्छत् । \newline
18. अय॑च्छ॒त् तस्मै॒ तस्मा॒ अय॑च्छ॒ दय॑च्छ॒त् तस्मै᳚ । \newline
19. तस्मै᳚ त्वा त्वा॒ तस्मै॒ तस्मै᳚ त्वा । \newline
20. त्वा॒ विष्ण॑वे॒ विष्ण॑वे त्वा त्वा॒ विष्ण॑वे । \newline
21. विष्ण॑वे त्वा त्वा॒ विष्ण॑वे॒ विष्ण॑वे त्वा । \newline
22. त्वेतीति॑ त्वा॒ त्वेति॑ । \newline
23. इत्या॑हा॒हे तीत्या॑ह । \newline
24. आ॒ह॒ यद् यदा॑हाह॒ यत् । \newline
25. यदे॒ वैव यद् यदे॒व । \newline
26. ए॒व विष्णु॒र् विष्णु॑ रे॒वैव विष्णुः॑ । \newline
27. विष्णु॑ र॒न्वति॑ष्ठता॒ न्वति॑ष्ठत॒ विष्णु॒र् विष्णु॑ र॒न्वति॑ष्ठत । \newline
28. अ॒न्वति॑ष्ठत ज॒हि ज॒ह्य॑ न्वति॑ष्ठता॒ न्वति॑ष्ठत ज॒हि । \newline
29. अ॒न्वति॑ष्ठ॒तेत्य॑नु - अति॑ष्ठत । \newline
30. ज॒ही तीति॑ ज॒हि ज॒हीति॑ । \newline
31. इति॒ तस्मा॒त् तस्मा॒ दितीति॒ तस्मा᳚त् । \newline
32. तस्मा॒द् विष्णुं॒ ॅविष्णु॒म् तस्मा॒त् तस्मा॒द् विष्णु᳚म् । \newline
33. विष्णु॑ म॒न्वाभ॑ज त्य॒न्वाभ॑जति॒ विष्णुं॒ ॅविष्णु॑ म॒न्वाभ॑जति । \newline
34. अ॒न्वाभ॑जति॒ त्रि स्त्रि र॒न्वाभ॑ज त्य॒न्वाभ॑जति॒ त्रिः । \newline
35. अ॒न्वाभ॑ज॒तीत्य॑नु - आभ॑जति । \newline
36. त्रिर् निर् णिष् ट्रि स्त्रिर् निः । \newline
37. निर् गृ॑ह्णाति गृह्णाति॒ निर् णिर् गृ॑ह्णाति । \newline
38. गृ॒ह्णा॒ति॒ त्रि स्त्रिर् गृ॑ह्णाति गृह्णाति॒ त्रिः । \newline
39. त्रिर्. हि हि त्रि स्त्रिर्. हि । \newline
40. हि स स हि हि सः । \newline
41. स तम् तꣳ स स तम् । \newline
42. तम् तस्मै॒ तस्मै॒ तम् तम् तस्मै᳚ । \newline
43. तस्मै॒ प्र प्र तस्मै॒ तस्मै॒ प्र । \newline
44. प्राय॑च्छ॒ दय॑च्छ॒त् प्र प्राय॑च्छत् । \newline
45. अय॑च्छ दे॒ष ए॒षो ऽय॑च्छ॒ दय॑च्छ दे॒षः । \newline
46. ए॒ष ते॑ त ए॒ष ए॒ष ते᳚ । \newline
47. ते॒ योनि॒र् योनि॑ स्ते ते॒ योनिः॑ । \newline
48. योनिः॒ पुन॑र्.हविः॒ पुन॑र्.हवि॒र् योनि॒र् योनिः॒ पुन॑र्.हविः । \newline
49. पुन॑र्.हवि रस्यसि॒ पुन॑र्.हविः॒ पुन॑र्.हवि रसि । \newline
50. पुन॑र्.हवि॒रिति॒ पुनः॑ - ह॒विः॒ । \newline
51. अ॒सीती त्य॑स्य॒ सीति॑ । \newline
52. इत्या॑हा॒हे तीत्या॑ह । \newline
53. आ॒ह॒ पुनः॑पुनः॒ पुनः॑पुन राहाह॒ पुनः॑पुनः । \newline
54. पुनः॑पुन॒र्॒. हि हि पुनः॑पुनः॒ पुनः॑पुन॒र्॒. हि । \newline
55. पुनः॑पुन॒रिति॒ पुनः॑ - पु॒नः॒ । \newline

\textbf{Ghana Paata } \newline

1. तद् वी॒र्यं॑ ॅवी॒र्य॑म् तत् तद् वी॒र्यं॑ ॅयज॑मानो॒ यज॑मानो वी॒र्य॑म् तत् तद् वी॒र्यं॑ ॅयज॑मानः । \newline
2. वी॒र्यं॑ ॅयज॑मानो॒ यज॑मानो वी॒र्यं॑ ॅवी॒र्यं॑ ॅयज॑मानो॒ भ्रातृ॑व्यस्य॒ भ्रातृ॑व्यस्य॒ यज॑मानो वी॒र्यं॑ ॅवी॒र्यं॑ ॅयज॑मानो॒ भ्रातृ॑व्यस्य । \newline
3. यज॑मानो॒ भ्रातृ॑व्यस्य॒ भ्रातृ॑व्यस्य॒ यज॑मानो॒ यज॑मानो॒ भ्रातृ॑व्यस्य वृङ्क्ते वृङ्क्ते॒ भ्रातृ॑व्यस्य॒ यज॑मानो॒ यज॑मानो॒ भ्रातृ॑व्यस्य वृङ्क्ते । \newline
4. भ्रातृ॑व्यस्य वृङ्क्ते वृङ्क्ते॒ भ्रातृ॑व्यस्य॒ भ्रातृ॑व्यस्य वृङ्क्त॒ इन्द्रा॒ येन्द्रा॑य वृङ्क्ते॒ भ्रातृ॑व्यस्य॒ भ्रातृ॑व्यस्य वृङ्क्त॒ इन्द्रा॑य । \newline
5. वृ॒ङ्क्त॒ इन्द्रा॒ येन्द्रा॑य वृङ्क्ते वृङ्क्त॒ इन्द्रा॑य त्वा॒ त्वेन्द्रा॑य वृङ्क्ते वृङ्क्त॒ इन्द्रा॑य त्वा । \newline
6. इन्द्रा॑य त्वा॒ त्वेन्द्रा॒ येन्द्रा॑य त्वा बृ॒हद्व॑ते बृ॒हद्व॑ते॒ त्वेन्द्रा॒ येन्द्रा॑य त्वा बृ॒हद्व॑ते । \newline
7. त्वा॒ बृ॒हद्व॑ते बृ॒हद्व॑ते त्वा त्वा बृ॒हद्व॑ते॒ वय॑स्वते॒ वय॑स्वते बृ॒हद्व॑ते त्वा त्वा बृ॒हद्व॑ते॒ वय॑स्वते । \newline
8. बृ॒हद्व॑ते॒ वय॑स्वते॒ वय॑स्वते बृ॒हद्व॑ते बृ॒हद्व॑ते॒ वय॑स्वत॒ इतीति॒ वय॑स्वते बृ॒हद्व॑ते बृ॒हद्व॑ते॒ वय॑स्वत॒ इति॑ । \newline
9. बृ॒हद्व॑त॒ इति॑ बृ॒हत् - व॒ते॒ । \newline
10. वय॑स्वत॒ इतीति॒ वय॑स्वते॒ वय॑स्वत॒ इत्या॑हा॒हेति॒ वय॑स्वते॒ वय॑स्वत॒ इत्या॑ह । \newline
11. इत्या॑हा॒हे तीत्या॒ हेन्द्रा॒ येन्द्रा॑या॒हे तीत्या॒ हेन्द्रा॑य । \newline
12. आ॒हेन्द्रा॒ येन्द्रा॑या हा॒हेन्द्रा॑य॒ हि हीन्द्रा॑या हा॒हेन्द्रा॑य॒ हि । \newline
13. इन्द्रा॑य॒ हि हीन्द्रा॒ येन्द्रा॑य॒ हि स स हीन्द्रा॒ येन्द्रा॑य॒ हि सः । \newline
14. हि स स हि हि स तम् तꣳ स हि हि स तम् । \newline
15. स तम् तꣳ स स तम् प्र प्र तꣳ स स तम् प्र । \newline
16. तम् प्र प्र तम् तम् प्राय॑च्छ॒ दय॑च्छ॒त् प्र तम् तम् प्राय॑च्छत् । \newline
17. प्राय॑च्छ॒ दय॑च्छ॒त् प्र प्राय॑च्छ॒त् तस्मै॒ तस्मा॒ अय॑च्छ॒त् प्र प्राय॑च्छ॒त् तस्मै᳚ । \newline
18. अय॑च्छ॒त् तस्मै॒ तस्मा॒ अय॑च्छ॒ दय॑च्छ॒त् तस्मै᳚ त्वा त्वा॒ तस्मा॒ अय॑च्छ॒ दय॑च्छ॒त् तस्मै᳚ त्वा । \newline
19. तस्मै᳚ त्वा त्वा॒ तस्मै॒ तस्मै᳚ त्वा॒ विष्ण॑वे॒ विष्ण॑वे त्वा॒ तस्मै॒ तस्मै᳚ त्वा॒ विष्ण॑वे । \newline
20. त्वा॒ विष्ण॑वे॒ विष्ण॑वे त्वा त्वा॒ विष्ण॑वे त्वा त्वा॒ विष्ण॑वे त्वा त्वा॒ विष्ण॑वे त्वा । \newline
21. विष्ण॑वे त्वा त्वा॒ विष्ण॑वे॒ विष्ण॑वे॒ त्वेतीति॑ त्वा॒ विष्ण॑वे॒ विष्ण॑वे॒ त्वेति॑ । \newline
22. त्वेतीति॑ त्वा॒ त्वेत्या॑हा॒ हेति॑ त्वा॒ त्वेत्या॑ह । \newline
23. इत्या॑हा॒हे तीत्या॑ह॒ यद् यदा॒हे तीत्या॑ह॒ यत् । \newline
24. आ॒ह॒ यद् यदा॑हाह॒ यदे॒ वैव यदा॑हाह॒ यदे॒व । \newline
25. यदे॒ वैव यद् यदे॒व विष्णु॒र् विष्णु॑ रे॒व यद् यदे॒व विष्णुः॑ । \newline
26. ए॒व विष्णु॒र् विष्णु॑ रे॒वैव विष्णु॑ र॒न्वति॑ष्ठता॒ न्वति॑ष्ठत॒ विष्णु॑ रे॒वैव विष्णु॑ र॒न्वति॑ष्ठत । \newline
27. विष्णु॑ र॒न्वति॑ष्ठता॒ न्वति॑ष्ठत॒ विष्णु॒र् विष्णु॑ र॒न्वति॑ष्ठत ज॒हि ज॒ह्य॑न्वति॑ष्ठत॒ विष्णु॒र् विष्णु॑ र॒न्वति॑ष्ठत ज॒हि । \newline
28. अ॒न्वति॑ष्ठत ज॒हि ज॒ह्य॑न्वति॑ष्ठता॒ न्वति॑ष्ठत ज॒ही तीति॑ ज॒ह्य॑न्वति॑ष्ठता॒ न्वति॑ष्ठत ज॒हीति॑ । \newline
29. अ॒न्वति॑ष्ठ॒तेत्य॑नु - अति॑ष्ठत । \newline
30. ज॒हीतीति॑ ज॒हि ज॒हीति॒ तस्मा॒त् तस्मा॒ दिति॑ ज॒हि ज॒हीति॒ तस्मा᳚त् । \newline
31. इति॒ तस्मा॒त् तस्मा॒ दितीति॒ तस्मा॒द् विष्णुं॒ ॅविष्णु॒म् तस्मा॒ दितीति॒ तस्मा॒द् विष्णु᳚म् । \newline
32. तस्मा॒द् विष्णुं॒ ॅविष्णु॒म् तस्मा॒त् तस्मा॒द् विष्णु॑ म॒न्वाभ॑ज त्य॒न्वाभ॑जति॒ विष्णु॒म् तस्मा॒त् तस्मा॒द् विष्णु॑ म॒न्वाभ॑जति । \newline
33. विष्णु॑ म॒न्वाभ॑ज त्य॒न्वाभ॑जति॒ विष्णुं॒ ॅविष्णु॑ म॒न्वाभ॑जति॒ त्रि स्त्रि र॒न्वाभ॑जति॒ विष्णुं॒ ॅविष्णु॑ म॒न्वाभ॑जति॒ त्रिः । \newline
34. अ॒न्वाभ॑जति॒ त्रि स्त्रि र॒न्वाभ॑ज त्य॒न्वाभ॑जति॒ त्रिर् निर् णिष् ट्रि र॒न्वाभ॑ज त्य॒न्वाभ॑जति॒ त्रिर् निः । \newline
35. अ॒न्वाभ॑ज॒तीत्य॑नु - आभ॑जति । \newline
36. त्रिर् निर् णिष् ट्रि स्त्रिर् निर् गृ॑ह्णाति गृह्णाति॒ निष् ट्रि स्त्रिर् निर् गृ॑ह्णाति । \newline
37. निर् गृ॑ह्णाति गृह्णाति॒ निर् णिर् गृ॑ह्णाति॒ त्रि स्त्रिर् गृ॑ह्णाति॒ निर् णिर् गृ॑ह्णाति॒ त्रिः । \newline
38. गृ॒ह्णा॒ति॒ त्रि स्त्रिर् गृ॑ह्णाति गृह्णाति॒ त्रिर्. हि हि त्रिर् गृ॑ह्णाति गृह्णाति॒ त्रिर्. हि । \newline
39. त्रिर्. हि हि त्रि स्त्रिर्. हि स स हि त्रि स्त्रिर्. हि सः । \newline
40. हि स स हि हि स तम् तꣳ स हि हि स तम् । \newline
41. स तम् तꣳ स स तम् तस्मै॒ तस्मै॒ तꣳ स स तम् तस्मै᳚ । \newline
42. तम् तस्मै॒ तस्मै॒ तम् तम् तस्मै॒ प्र प्र तस्मै॒ तम् तम् तस्मै॒ प्र । \newline
43. तस्मै॒ प्र प्र तस्मै॒ तस्मै॒ प्राय॑च्छ॒ दय॑च्छ॒त् प्र तस्मै॒ तस्मै॒ प्राय॑च्छत् । \newline
44. प्राय॑च्छ॒ दय॑च्छ॒त् प्र प्राय॑च्छ दे॒ष ए॒षो ऽय॑च्छ॒त् प्र प्राय॑च्छ दे॒षः । \newline
45. अय॑च्छ दे॒ष ए॒षो ऽय॑च्छ॒ दय॑च्छ दे॒ष ते॑ त ए॒षो ऽय॑च्छ॒ दय॑च्छ दे॒ष ते᳚ । \newline
46. ए॒ष ते॑ त ए॒ष ए॒ष ते॒ योनि॒र् योनि॑ स्त ए॒ष ए॒ष ते॒ योनिः॑ । \newline
47. ते॒ योनि॒र् योनि॑ स्ते ते॒ योनिः॒ पुन॑र्.हविः॒ पुन॑र्.हवि॒र् योनि॑ स्ते ते॒ योनिः॒ पुन॑र्.हविः । \newline
48. योनिः॒ पुन॑र्.हविः॒ पुन॑र्.हवि॒र् योनि॒र् योनिः॒ पुन॑र्.हवि रस्यसि॒ पुन॑र्.हवि॒र् योनि॒र् योनिः॒ पुन॑र्.हवि रसि । \newline
49. पुन॑र्.हवि रस्यसि॒ पुन॑र्.हविः॒ पुन॑र्.हवि र॒सीती त्य॑सि॒ पुन॑र्.हविः॒ पुन॑र्.हवि र॒सीति॑ । \newline
50. पुन॑र्.हवि॒रिति॒ पुनः॑ - ह॒विः॒ । \newline
51. अ॒सीती त्य॑स्य॒ सीत्या॑हा॒हे त्य॑स्य॒ सीत्या॑ह । \newline
52. इत्या॑हा॒हे तीत्या॑ह॒ पुनः॑पुनः॒ पुनः॑पुन रा॒हे तीत्या॑ह॒ पुनः॑पुनः । \newline
53. आ॒ह॒ पुनः॑पुनः॒ पुनः॑पुन राहाह॒ पुनः॑पुन॒र्॒. हि हि पुनः॑पुन राहाह॒ पुनः॑पुन॒र्॒. हि । \newline
54. पुनः॑पुन॒र्॒. हि हि पुनः॑पुनः॒ पुनः॑पुन॒र् ह्य॑स्मा दस्मा॒द्धि पुनः॑पुनः॒ पुनः॑पुन॒र् ह्य॑स्मात् । \newline
55. पुनः॑पुन॒रिति॒ पुनः॑ - पु॒नः॒ । \newline
\pagebreak
\markright{ TS 6.5.1.4  \hfill https://www.vedavms.in \hfill}

\section{ TS 6.5.1.4 }

\textbf{TS 6.5.1.4 } \newline
\textbf{Samhita Paata} \newline

-र्ह्य॑स्मा-न्निर्गृ॒ह्णाति॒ चक्षु॒र्वा ए॒तद्-य॒ज्ञ्स्य॒ यदु॒क्थ्य॑-स्तस्मा॑दु॒क्थ्यꣳ॑ हु॒तꣳ सोमा॑ अ॒न्वाय॑न्ति॒ तस्मा॑दा॒त्मा चक्षु॒रन्वे॑ति॒ तस्मा॒देकं॒ ॅयन्तं॑ ब॒हवोऽनु॑ यन्ति॒ तस्मा॒देको॑ बहू॒नां भ॒द्रो भ॑वति॒ तस्मा॒देको॑ ब॒ह्वीर्जा॒या वि॑न्दते॒ यदि॑ का॒मये॑ता-द्ध्व॒र्यु-रा॒त्मानं॑ ॅयज्ञ् यश॒सेना᳚-र्पयेय॒-मित्य॑न्त॒राऽऽह॑व॒नीयं॑ च हवि॒र्द्धानं॑ च॒ तिष्ठ॒न्नव॑ नये- [  ] \newline

\textbf{Pada Paata} \newline

हि । अ॒स्मा॒त् । नि॒र्गृ॒ह्णातीति॑ निः - गृ॒ह्णाति॑ । चक्षुः॑ । वै । ए॒तत् । य॒ज्ञ्स्य॑ । यत् । उ॒क्थ्यः॑ । तस्मा᳚त् । उ॒क्थ्य᳚म् । हु॒तम् । सोमाः᳚ । अ॒न्वाय॒न्तीत्य॑नु - आय॑न्ति । तस्मा᳚त् । आ॒त्मा । चक्षुः॑ । अन्विति॑ । ए॒ति॒ । तस्मा᳚त् । एक᳚म् । यन्त᳚म् । ब॒हवः॑ । अन्विति॑ । य॒न्ति॒ । तस्मा᳚त् । एकः॑ । ब॒हू॒नाम् । भ॒द्रः । भ॒व॒ति॒ । तस्मा᳚त् । एकः॑ । ब॒ह्वीः । जा॒याः । वि॒न्द॒ते॒ । यदि॑ । का॒मये॑त । अ॒द्ध्व॒र्युः । आ॒त्मान᳚म् । य॒ज्ञ्॒य॒श॒सेनेति॑ यज्ञ् - य॒श॒सेन॑ । अ॒र्प॒ये॒य॒म् । इति॑ । अ॒न्त॒रा । आ॒ह॒व॒नीय॒मित्या᳚ - ह॒व॒नीय᳚म् । च॒ । ह॒वि॒द्‌र्धान॒मिति॑ हविः - धान᳚म् । च॒ । तिष्ठन्न्॑ । अवेति॑ । न॒ये॒त् ।  \newline


\textbf{Krama Paata} \newline

ह्य॑स्मात् । अ॒स्मा॒न् नि॒र्गृ॒ह्णाति॑ । नि॒र्गृ॒ह्णाति॒ चक्षुः॑ । नि॒र्गृ॒ह्णातीति॑ निः - गृ॒ह्णाति॑ । चक्षु॒र् वै । वा ए॒तत् । ए॒तद् य॒ज्ञ्स्य॑ । य॒ज्ञ्स्य॒ यत् । यदु॒क्थ्यः॑ । उ॒क्थ्य॑स्तस्मा᳚त् । तस्मा॑दु॒क्थ्य᳚म् । उ॒क्थ्यꣳ॑ हु॒तम् । हु॒तꣳ सोमाः᳚ । सोमा॑ अ॒न्वाय॑न्ति । अ॒न्वाय॑न्ति॒ तस्मा᳚त् । अ॒न्वाय॒न्तीत्य॑नु - आय॑न्ति । तस्मा॑दा॒त्मा । आ॒त्मा चक्षुः॑ । चक्षु॒रनु॑ । अन्वे॑ति । ए॒ति॒ तस्मा᳚त् । तस्मा॒देक᳚म् । एक॒म् ॅयन्त᳚म् । यन्त॑म् ब॒हवः॑ । ब॒हवोऽनु॑ । अनु॑ यन्ति । य॒न्ति॒ तस्मा᳚त् । तस्मा॒देकः॑ । एको॑ बहू॒नाम् । ब॒हू॒नाम् भ॒द्रः । भ॒द्रो भ॑वति । भ॒व॒ति॒ तस्मा᳚त् । तस्मा॒देकः॑ । एको॑ ब॒ह्वीः । ब॒ह्वीर् जा॒याः । जा॒या वि॑न्दते । वि॒न्द॒ते॒ यदि॑ । यदि॑ का॒मये॑त । का॒मये॑ताद्ध्व॒र्युः । अ॒द्ध्व॒र्युरा॒त्मान᳚म् । आ॒त्मान॑म् ॅयज्ञ्यश॒सेन॑ । य॒ज्ञ्॒य॒श॒सेना᳚र्पयेयम् । य॒ज्ञ्॒य॒श॒सेनेति॑ यज्ञ् - य॒श॒सेन॑ । अ॒र्प॒ये॒य॒मिति॑ । इत्य॑न्त॒रा । अ॒न्त॒राऽऽह॑व॒नीय᳚म् । आ॒ह॒व॒नीय॑म् च । आ॒ह॒व॒नीय॒मित्या᳚ - ह॒व॒नीय᳚म् । च॒ ह॒वि॒र्द्धान᳚म् । ह॒वि॒र्द्धान॑म् च । ह॒वि॒र्द्धान॒मिति॑ हविः - धान᳚म् । च॒ तिष्ठन्न्॑ । तिष्ठ॒न्नव॑ । अव॑ नयेत् ( ) । न॒ये॒दा॒त्मान᳚म् \newline

\textbf{Jatai Paata} \newline

1. ह्य॑स्मा दस्मा॒ द्धि ह्य॑स्मात् । \newline
2. अ॒स्मा॒न् नि॒र्गृ॒ह्णाति॑ निर्गृ॒ह्णा त्य॑स्मा दस्मान् निर्गृ॒ह्णाति॑ । \newline
3. नि॒र्गृ॒ह्णाति॒ चक्षु॒ श्चक्षु॑र् निर्गृ॒ह्णाति॑ निर्गृ॒ह्णाति॒ चक्षुः॑ । \newline
4. नि॒र्गृ॒ह्णातीति॑ निः - गृ॒ह्णाति॑ । \newline
5. चक्षु॒र् वै वै चक्षु॒ श्चक्षु॒र् वै । \newline
6. वा ए॒त दे॒तद् वै वा ए॒तत् । \newline
7. ए॒तद् य॒ज्ञ्स्य॑ य॒ज्ञ् स्यै॒त दे॒तद् य॒ज्ञ्स्य॑ । \newline
8. य॒ज्ञ्स्य॒ यद् यद् य॒ज्ञ्स्य॑ य॒ज्ञ्स्य॒ यत् । \newline
9. यदु॒क्थ्य॑ उ॒क्थ्यो॑ यद् यदु॒क्थ्यः॑ । \newline
10. उ॒क्थ्य॑ स्तस्मा॒त् तस्मा॑ दु॒क्थ्य॑ उ॒क्थ्य॑ स्तस्मा᳚त् । \newline
11. तस्मा॑ दु॒क्थ्य॑ मु॒क्थ्य॑म् तस्मा॒त् तस्मा॑ दु॒क्थ्य᳚म् । \newline
12. उ॒क्थ्यꣳ॑ हु॒तꣳ हु॒त मु॒क्थ्य॑ मु॒क्थ्यꣳ॑ हु॒तम् । \newline
13. हु॒तꣳ सोमाः॒ सोमा॑ हु॒तꣳ हु॒तꣳ सोमाः᳚ । \newline
14. सोमा॑ अ॒न्वाय॑ न्त्य॒न्वाय॑न्ति॒ सोमाः॒ सोमा॑ अ॒न्वाय॑न्ति । \newline
15. अ॒न्वाय॑न्ति॒ तस्मा॒त् तस्मा॑ द॒न्वाय॑ न्त्य॒न्वाय॑न्ति॒ तस्मा᳚त् । \newline
16. अ॒न्वाय॒न्तीत्य॑नु - आय॑न्ति । \newline
17. तस्मा॑ दा॒त्मा ऽऽत्मा तस्मा॒त् तस्मा॑ दा॒त्मा । \newline
18. आ॒त्मा चक्षु॒ श्चक्षु॑ रा॒त्मा ऽऽत्मा चक्षुः॑ । \newline
19. चक्षु॒ रन्वनु॒ चक्षु॒ श्चक्षु॒ रनु॑ । \newline
20. अन्वे᳚त्ये॒ त्यन् वन् वे॑ति । \newline
21. ए॒ति॒ तस्मा॒त् तस्मा॑ देत्येति॒ तस्मा᳚त् । \newline
22. तस्मा॒ देक॒ मेक॒म् तस्मा॒त् तस्मा॒ देक᳚म् । \newline
23. एकं॒ ॅयन्तं॒ ॅयन्त॒ मेक॒ मेकं॒ ॅयन्त᳚म् । \newline
24. यन्त॑म् ब॒हवो॑ ब॒हवो॒ यन्तं॒ ॅयन्त॑म् ब॒हवः॑ । \newline
25. ब॒हवो ऽन्वनु॑ ब॒हवो॑ ब॒हवो ऽनु॑ । \newline
26. अनु॑ यन्ति य॒न्त्य न्वनु॑ यन्ति । \newline
27. य॒न्ति॒ तस्मा॒त् तस्मा᳚द् यन्ति यन्ति॒ तस्मा᳚त् । \newline
28. तस्मा॒ देक॒ एक॒ स्तस्मा॒त् तस्मा॒ देकः॑ । \newline
29. एको॑ बहू॒नाम् ब॑हू॒ना मेक॒ एको॑ बहू॒नाम् । \newline
30. ब॒हू॒नाम् भ॒द्रो भ॒द्रो ब॑हू॒नाम् ब॑हू॒नाम् भ॒द्रः । \newline
31. भ॒द्रो भ॑वति भवति भ॒द्रो भ॒द्रो भ॑वति । \newline
32. भ॒व॒ति॒ तस्मा॒त् तस्मा᳚द् भवति भवति॒ तस्मा᳚त् । \newline
33. तस्मा॒ देक॒ एक॒ स्तस्मा॒त् तस्मा॒ देकः॑ । \newline
34. एको॑ ब॒ह्वीर् ब॒ह्वी रेक॒ एको॑ ब॒ह्वीः । \newline
35. ब॒ह्वीर् जा॒या जा॒या ब॒ह्वीर् ब॒ह्वीर् जा॒याः । \newline
36. जा॒या वि॑न्दते विन्दते जा॒या जा॒या वि॑न्दते । \newline
37. वि॒न्द॒ते॒ यदि॒ यदि॑ विन्दते विन्दते॒ यदि॑ । \newline
38. यदि॑ का॒मये॑त का॒मये॑त॒ यदि॒ यदि॑ का॒मये॑त । \newline
39. का॒मये॑ता द्ध्व॒र्यु र॑द्ध्व॒र्युः का॒मये॑त का॒मये॑ता द्ध्व॒र्युः । \newline
40. अ॒द्ध्व॒र्यु रा॒त्मान॑ मा॒त्मान॑ मद्ध्व॒र्यु र॑द्ध्व॒र्यु रा॒त्मान᳚म् । \newline
41. आ॒त्मानं॑ ॅयज्ञ्यश॒सेन॑ यज्ञ्यश॒से ना॒त्मान॑ मा॒त्मानं॑ ॅयज्ञ्यश॒सेन॑ । \newline
42. य॒ज्ञ्॒य॒श॒सेना᳚ र्पयेय मर्पयेयं ॅयज्ञ्यश॒सेन॑ यज्ञ्यश॒सेना᳚ र्पयेयम् । \newline
43. य॒ज्ञ्॒य॒श॒सेनेति॑ यज्ञ् - य॒श॒सेन॑ । \newline
44. अ॒र्प॒ये॒य॒ मिती त्य॑र्पयेय मर्पयेय॒ मिति॑ । \newline
45. इत्य॑न्त॒रा ऽन्त॒ रेती त्य॑न्त॒रा । \newline
46. अ॒न्त॒रा ऽऽह॑व॒नीय॑ माहव॒नीय॑ मन्त॒रा ऽन्त॒रा ऽऽह॑व॒नीय᳚म् । \newline
47. आ॒ह॒व॒नीय॑म् च चाहव॒नीय॑ माहव॒नीय॑म् च । \newline
48. आ॒ह॒व॒नीय॒मित्या᳚ - ह॒व॒नीय᳚म् । \newline
49. च॒ ह॒वि॒र्द्धानꣳ॑ हवि॒र्द्धान॑म् च च हवि॒र्द्धान᳚म् । \newline
50. ह॒वि॒र्द्धान॑म् च च हवि॒र्द्धानꣳ॑ हवि॒र्द्धान॑म् च । \newline
51. ह॒वि॒र्द्धान॒मिति॑ हविः - धान᳚म् । \newline
52. च॒ तिष्ठꣳ॒॒ स्तिष्ठꣳ॑ श्च च॒ तिष्ठन्न्॑ । \newline
53. तिष्ठ॒ न्नवाव॒ तिष्ठꣳ॒॒ स्तिष्ठ॒न्नव॑ । \newline
54. अव॑ नयेन् नये॒ दवाव॑ नयेत् । \newline
55. न॒ये॒ दा॒त्मान॑ मा॒त्मान॑म् नयेन् नये दा॒त्मान᳚म् । \newline

\textbf{Ghana Paata } \newline

1. ह्य॑स्मा दस्मा॒द्धि ह्य॑स्मान् निर्गृ॒ह्णाति॑ निर्गृ॒ह्णा त्य॑स्मा॒द्धि ह्य॑स्मान् निर्गृ॒ह्णाति॑ । \newline
2. अ॒स्मा॒न् नि॒र्गृ॒ह्णाति॑ निर्गृ॒ह्णा त्य॑स्मा दस्मान् निर्गृ॒ह्णाति॒ चक्षु॒ श्चक्षु॑र् निर्गृ॒ह्णा त्य॑स्मा दस्मान् निर्गृ॒ह्णाति॒ चक्षुः॑ । \newline
3. नि॒र्गृ॒ह्णाति॒ चक्षु॒ श्चक्षु॑र् निर्गृ॒ह्णाति॑ निर्गृ॒ह्णाति॒ चक्षु॒र् वै वै चक्षु॑र् निर्गृ॒ह्णाति॑ निर्गृ॒ह्णाति॒ चक्षु॒र् वै । \newline
4. नि॒र्गृ॒ह्णातीति॑ निः - गृ॒ह्णाति॑ । \newline
5. चक्षु॒र् वै वै चक्षु॒ श्चक्षु॒र् वा ए॒त दे॒तद् वै चक्षु॒ श्चक्षु॒र् वा ए॒तत् । \newline
6. वा ए॒त दे॒तद् वै वा ए॒तद् य॒ज्ञ्स्य॑ य॒ज्ञ् स्यै॒तद् वै वा ए॒तद् य॒ज्ञ्स्य॑ । \newline
7. ए॒तद् य॒ज्ञ्स्य॑ य॒ज्ञ् स्यै॒त दे॒तद् य॒ज्ञ्स्य॒ यद् यद् य॒ज्ञ् स्यै॒त दे॒तद् य॒ज्ञ्स्य॒ यत् । \newline
8. य॒ज्ञ्स्य॒ यद् यद् य॒ज्ञ्स्य॑ य॒ज्ञ्स्य॒ यदु॒क्थ्य॑ उ॒क्थ्यो॑ यद् य॒ज्ञ्स्य॑ य॒ज्ञ्स्य॒ यदु॒क्थ्यः॑ । \newline
9. यदु॒क्थ्य॑ उ॒क्थ्यो॑ यद् यदु॒क्थ्य॑ स्तस्मा॒त् तस्मा॑ दु॒क्थ्यो॑ यद् यदु॒क्थ्य॑ स्तस्मा᳚त् । \newline
10. उ॒क्थ्य॑ स्तस्मा॒त् तस्मा॑ दु॒क्थ्य॑ उ॒क्थ्य॑ स्तस्मा॑ दु॒क्थ्य॑ मु॒क्थ्य॑म् तस्मा॑ दु॒क्थ्य॑ उ॒क्थ्य॑ स्तस्मा॑ दु॒क्थ्य᳚म् । \newline
11. तस्मा॑ दु॒क्थ्य॑ मु॒क्थ्य॑म् तस्मा॒त् तस्मा॑ दु॒क्थ्यꣳ॑ हु॒तꣳ हु॒त मु॒क्थ्य॑म् तस्मा॒त् तस्मा॑ दु॒क्थ्यꣳ॑ हु॒तम् । \newline
12. उ॒क्थ्यꣳ॑ हु॒तꣳ हु॒त मु॒क्थ्य॑ मु॒क्थ्यꣳ॑ हु॒तꣳ सोमाः॒ सोमा॑ हु॒त मु॒क्थ्य॑ मु॒क्थ्यꣳ॑ हु॒तꣳ सोमाः᳚ । \newline
13. हु॒तꣳ सोमाः॒ सोमा॑ हु॒तꣳ हु॒तꣳ सोमा॑ अ॒न्वाय॑ न्त्य॒न्वाय॑न्ति॒ सोमा॑ हु॒तꣳ हु॒तꣳ सोमा॑ अ॒न्वाय॑न्ति । \newline
14. सोमा॑ अ॒न्वाय॑ न्त्य॒न्वाय॑न्ति॒ सोमाः॒ सोमा॑ अ॒न्वाय॑न्ति॒ तस्मा॒त् तस्मा॑ द॒न्वाय॑न्ति॒ सोमाः॒ सोमा॑ अ॒न्वाय॑न्ति॒ तस्मा᳚त् । \newline
15. अ॒न्वाय॑न्ति॒ तस्मा॒त् तस्मा॑ द॒न्वाय॑ न्त्य॒न्वाय॑न्ति॒ तस्मा॑ दा॒त्मा ऽऽत्मा तस्मा॑ द॒न्वाय॑ न्त्य॒न्वाय॑न्ति॒ तस्मा॑ दा॒त्मा । \newline
16. अ॒न्वाय॒न्तीत्य॑नु - आय॑न्ति । \newline
17. तस्मा॑ दा॒त्मा ऽऽत्मा तस्मा॒त् तस्मा॑ दा॒त्मा चक्षु॒ श्चक्षु॑ रा॒त्मा तस्मा॒त् तस्मा॑ दा॒त्मा चक्षुः॑ । \newline
18. आ॒त्मा चक्षु॒ श्चक्षु॑ रा॒त्मा ऽऽत्मा चक्षु॒ रन्वनु॒ चक्षु॑ रा॒त्मा ऽऽत्मा चक्षु॒ रनु॑ । \newline
19. चक्षु॒ रन्वनु॒ चक्षु॒ श्चक्षु॒ रन्वे᳚ त्ये॒ त्यनु॒ चक्षु॒ श्चक्षु॒ रन्वे॑ति । \newline
20. अन्वे᳚ त्ये॒ त्यन् वन् वे॑ति॒ तस्मा॒त् तस्मा॑ दे॒त्यन् वन् वे॑ति॒ तस्मा᳚त् । \newline
21. ए॒ति॒ तस्मा॒त् तस्मा॑ देत्येति॒ तस्मा॒ देक॒ मेक॒म् तस्मा॑ देत्येति॒ तस्मा॒ देक᳚म् । \newline
22. तस्मा॒ देक॒ मेक॒म् तस्मा॒त् तस्मा॒ देकं॒ ॅयन्तं॒ ॅयन्त॒ मेक॒म् तस्मा॒त् तस्मा॒ देकं॒ ॅयन्त᳚म् । \newline
23. एकं॒ ॅयन्तं॒ ॅयन्त॒ मेक॒ मेकं॒ ॅयन्त॑म् ब॒हवो॑ ब॒हवो॒ यन्त॒ मेक॒ मेकं॒ ॅयन्त॑म् ब॒हवः॑ । \newline
24. यन्त॑म् ब॒हवो॑ ब॒हवो॒ यन्तं॒ ॅयन्त॑म् ब॒हवो ऽन्वनु॑ ब॒हवो॒ यन्तं॒ ॅयन्त॑म् ब॒हवो ऽनु॑ । \newline
25. ब॒हवो ऽन्वनु॑ ब॒हवो॑ ब॒हवो ऽनु॑ यन्ति य॒न्त्यनु॑ ब॒हवो॑ ब॒हवो ऽनु॑ यन्ति । \newline
26. अनु॑ यन्ति य॒न्त्यन् वनु॑ यन्ति॒ तस्मा॒त् तस्मा᳚द् य॒न्त्यन् वनु॑ यन्ति॒ तस्मा᳚त् । \newline
27. य॒न्ति॒ तस्मा॒त् तस्मा᳚द् यन्ति यन्ति॒ तस्मा॒ देक॒ एक॒ स्तस्मा᳚द् यन्ति यन्ति॒ तस्मा॒ देकः॑ । \newline
28. तस्मा॒ देक॒ एक॒ स्तस्मा॒त् तस्मा॒ देको॑ बहू॒नाम् ब॑हू॒ना मेक॒ स्तस्मा॒त् तस्मा॒ देको॑ बहू॒नाम् । \newline
29. एको॑ बहू॒नाम् ब॑हू॒ना मेक॒ एको॑ बहू॒नाम् भ॒द्रो भ॒द्रो ब॑हू॒ना मेक॒ एको॑ बहू॒नाम् भ॒द्रः । \newline
30. ब॒हू॒नाम् भ॒द्रो भ॒द्रो ब॑हू॒नाम् ब॑हू॒नाम् भ॒द्रो भ॑वति भवति भ॒द्रो ब॑हू॒नाम् ब॑हू॒नाम् भ॒द्रो भ॑वति । \newline
31. भ॒द्रो भ॑वति भवति भ॒द्रो भ॒द्रो भ॑वति॒ तस्मा॒त् तस्मा᳚द् भवति भ॒द्रो भ॒द्रो भ॑वति॒ तस्मा᳚त् । \newline
32. भ॒व॒ति॒ तस्मा॒त् तस्मा᳚द् भवति भवति॒ तस्मा॒ देक॒ एक॒ स्तस्मा᳚द् भवति भवति॒ तस्मा॒ देकः॑ । \newline
33. तस्मा॒ देक॒ एक॒ स्तस्मा॒त् तस्मा॒ देको॑ ब॒ह्वीर् ब॒ह्वी रेक॒ स्तस्मा॒त् तस्मा॒ देको॑ ब॒ह्वीः । \newline
34. एको॑ ब॒ह्वीर् ब॒ह्वी रेक॒ एको॑ ब॒ह्वीर् जा॒या जा॒या ब॒ह्वी रेक॒ एको॑ ब॒ह्वीर् जा॒याः । \newline
35. ब॒ह्वीर् जा॒या जा॒या ब॒ह्वीर् ब॒ह्वीर् जा॒या वि॑न्दते विन्दते जा॒या ब॒ह्वीर् ब॒ह्वीर् जा॒या वि॑न्दते । \newline
36. जा॒या वि॑न्दते विन्दते जा॒या जा॒या वि॑न्दते॒ यदि॒ यदि॑ विन्दते जा॒या जा॒या वि॑न्दते॒ यदि॑ । \newline
37. वि॒न्द॒ते॒ यदि॒ यदि॑ विन्दते विन्दते॒ यदि॑ का॒मये॑त का॒मये॑त॒ यदि॑ विन्दते विन्दते॒ यदि॑ का॒मये॑त । \newline
38. यदि॑ का॒मये॑त का॒मये॑त॒ यदि॒ यदि॑ का॒मये॑ता ध्व॒र्यु र॑द्ध्व॒र्युः का॒मये॑त॒ यदि॒ यदि॑ का॒मये॑ता ध्व॒र्युः । \newline
39. का॒मये॑ता ध्व॒र्यु र॑द्ध्व॒र्युः का॒मये॑त का॒मये॑ता ध्व॒र्यु रा॒त्मान॑ मा॒त्मान॑ मद्ध्व॒र्युः का॒मये॑त का॒मये॑ता ध्व॒र्यु रा॒त्मान᳚म् । \newline
40. अ॒द्ध्व॒र्यु रा॒त्मान॑ मा॒त्मान॑ मद्ध्व॒र्यु र॑द्ध्व॒र्यु रा॒त्मानं॑ ॅयज्ञ्यश॒सेन॑ यज्ञ्यश॒से ना॒त्मान॑ मद्ध्व॒र्यु र॑द्ध्व॒र्यु रा॒त्मानं॑ ॅयज्ञ्यश॒सेन॑ । \newline
41. आ॒त्मानं॑ ॅयज्ञ्यश॒सेन॑ यज्ञ्यश॒सेना॒त्मान॑ मा॒त्मानं॑ ॅयज्ञ्यश॒सेना᳚ र्पयेय मर्पयेयं ॅयज्ञ्यश॒सेना॒त्मान॑ मा॒त्मानं॑ ॅयज्ञ्यश॒सेना᳚ र्पयेयम् । \newline
42. य॒ज्ञ्॒य॒श॒सेना᳚ र्पयेय मर्पयेयं ॅयज्ञ्यश॒सेन॑ यज्ञ्यश॒सेना᳚ र्पयेय॒ मिती त्य॑र्पयेयं ॅयज्ञ्यश॒सेन॑ यज्ञ्यश॒सेना᳚ र्पयेय॒ मिति॑ । \newline
43. य॒ज्ञ्॒य॒श॒सेनेति॑ यज्ञ् - य॒श॒सेन॑ । \newline
44. अ॒र्प॒ये॒य॒ मिती त्य॑र्पयेय मर्पयेय॒ मित्य॑न्त॒रा ऽन्त॒रे त्य॑र्पयेय मर्पयेय॒ मित्य॑न्त॒रा । \newline
45. इत्य॑न्त॒रा ऽन्त॒रे तीत्य॑न्त॒रा ऽऽह॑व॒नीय॑ माहव॒नीय॑ मन्त॒रे तीत्य॑न्त॒रा ऽऽह॑व॒नीय᳚म् । \newline
46. अ॒न्त॒रा ऽऽह॑व॒नीय॑ माहव॒नीय॑ मन्त॒रा ऽन्त॒रा ऽऽह॑व॒नीय॑म् च चाहव॒नीय॑ मन्त॒रा ऽन्त॒रा ऽऽह॑व॒नीय॑म् च । \newline
47. आ॒ह॒व॒नीय॑म् च चाहव॒नीय॑ माहव॒नीय॑म् च हवि॒र्द्धानꣳ॑ हवि॒र्द्धान॑म् चाहव॒नीय॑ माहव॒नीय॑म् च हवि॒र्द्धान᳚म् । \newline
48. आ॒ह॒व॒नीय॒मित्या᳚ - ह॒व॒नीय᳚म् । \newline
49. च॒ ह॒वि॒र्द्धानꣳ॑ हवि॒र्द्धान॑म् च च हवि॒र्द्धान॑म् च च हवि॒र्द्धान॑म् च च हवि॒र्द्धान॑म् च । \newline
50. ह॒वि॒र्द्धान॑म् च च हवि॒र्द्धानꣳ॑ हवि॒र्द्धान॑म् च॒ तिष्ठꣳ॒॒ स्तिष्ठꣳ॑ श्च हवि॒र्द्धानꣳ॑ हवि॒र्द्धान॑म् च॒ तिष्ठन्न्॑ । \newline
51. ह॒वि॒र्द्धान॒मिति॑ हविः - धान᳚म् । \newline
52. च॒ तिष्ठꣳ॒॒ स्तिष्ठꣳ॑ श्च च॒ तिष्ठ॒न्-नवाव॒ तिष्ठꣳ॑श्च च॒ तिष्ठ॒न्-नव॑ । \newline
53. तिष्ठ॒न्-नवाव॒ तिष्ठꣳ॒॒ स्तिष्ठ॒न्-नव॑ नयेन् नये॒ दव॒ तिष्ठꣳ॒॒ स्तिष्ठ॒न्-नव॑ नयेत् । \newline
54. अव॑ नयेन् नये॒ दवाव॑ नये दा॒त्मान॑ मा॒त्मान॑म् नये॒ दवाव॑ नये दा॒त्मान᳚म् । \newline
55. न॒ये॒ दा॒त्मान॑ मा॒त्मान॑म् नयेन् नये दा॒त्मान॑ मे॒वै वात्मान॑म् नयेन् नये दा॒त्मान॑ मे॒व । \newline
\pagebreak
\markright{ TS 6.5.1.5  \hfill https://www.vedavms.in \hfill}

\section{ TS 6.5.1.5 }

\textbf{TS 6.5.1.5 } \newline
\textbf{Samhita Paata} \newline

दा॒त्मान॑मे॒व य॑ज्ञ्यश॒सेना᳚र्पयति॒ यदि॑ का॒मये॑त॒ यज॑मानं ॅयज्ञ् यश॒सेना᳚र्पयेय॒-मित्य॑न्त॒रा स॑दोहविर्द्धा॒ने तिष्ठ॒न्नव॑ नये॒द्-यज॑मानमे॒व य॑ज्ञ्यश॒सेना᳚-र्पयति॒ यदि॑ का॒मये॑त सद॒स्यान्॑ यज्ञ् यश॒सेना᳚-र्पयेय॒मिति॒ सद॑ आ॒लभ्याव॑ नयेथ् सद॒स्या॑ने॒व य॑ज्ञ्यश॒सेना᳚र्पयति ॥ \newline

\textbf{Pada Paata} \newline

आ॒त्मान᳚म् । ए॒व । य॒ज्ञ्॒य॒श॒सेनेति॑ यज्ञ्-य॒श॒सेन॑ । अ॒र्प॒य॒ति॒ । यदि॑ । का॒मये॑त । यज॑मानम् । य॒ज्ञ्॒य॒श॒सेनेति॑ यज्ञ् - य॒श॒सेन॑ । अ॒र्प॒ये॒य॒म् । इति॑ । अ॒न्त॒रा । स॒दो॒ह॒वि॒द्‌र्धा॒ने इति॑ सदः - ह॒वि॒द्‌र्धा॒ने । तिष्ठन्न्॑ । अवेति॑ । न॒ये॒त् । यज॑मानम् । ए॒व । य॒ज्ञ्॒य॒श॒सेनेति॑ यज्ञ् - य॒श॒सेन॑ । अ॒र्प॒य॒ति॒ । यदि॑ । का॒मये॑त । स॒द॒स्यान्॑ । य॒ज्ञ्॒य॒श॒सेनेति॑ यज्ञ् - य॒श॒सेन॑ । अ॒र्प॒ये॒य॒म् । इति॑ । सदः॑ । आ॒लभ्येत्या᳚ - लभ्य॑ । अवेति॑ । न॒ये॒त् । स॒द॒स्यान्॑ । ए॒व । य॒ज्ञ्॒य॒श॒सेनेति॑ यज्ञ् - य॒श॒सेन॑ । अ॒र्प॒य॒ति॒ ॥  \newline


\textbf{Krama Paata} \newline

आ॒त्मान॑मे॒व । ए॒व य॑ज्ञ्यश॒सेन॑ । य॒ज्ञ्॒य॒श॒सेना᳚र्पयति । य॒ज्ञ्॒य॒श॒सेनेति॑ यज्ञ् - य॒श॒सेन॑ । अ॒र्प॒य॒ति॒ यदि॑ । यदि॑ का॒मये॑त । का॒मये॑त॒ यज॑मानम् । यज॑मानम् ॅयज्ञ्यश॒सेन॑ । य॒ज्ञ्॒य॒श॒सेना᳚र्पयेयम् । य॒ज्ञ्॒य॒श॒सेनेति॑ यज्ञ् - य॒श॒सेन॑ । अ॒र्प॒ये॒य॒मिति॑ । इत्य॑न्त॒रा । अ॒न्त॒रा स॑दोहविर्द्धा॒ने । स॒दो॒ह॒वि॒र्द्धा॒ने तिष्ठन्न्॑ । स॒दो॒ह॒वि॒र्द्धा॒ने इति॑ सदः - ह॒वि॒र्द्धा॒ने । तिष्ठ॒न्नव॑ । अव॑ नयेत् । न॒ये॒द् यज॑मानम् । यज॑मानमे॒व । ए॒व य॑ज्ञ्यश॒सेन॑ । य॒ज्ञ्॒य॒श॒सेना᳚र्पयति । य॒ज्ञ्॒य॒श॒सेनेति॑ यज्ञ् - य॒श॒सेन॑ । अ॒र्प॒य॒ति॒ यदि॑ । यदि॑ का॒मये॑त । का॒मये॑त सद॒स्यान्॑ । स॒द॒स्यान्॑. यज्ञ्यश॒सेन॑ । य॒ज्ञ्॒य॒श॒सेना᳚र्पयेयम् । य॒ज्ञ्॒य॒श॒सेनेति॑ यज्ञ् - य॒श॒सेन॑ । अ॒र्प॒ये॒य॒मिति॑ । इति॒ सदः॑ । सद॑ आ॒लभ्य॑ । आ॒लभ्याव॑ । आ॒लभ्येत्या᳚ - लभ्य॑ । अव॑ नयेत् । न॒ये॒थ् स॒द॒स्यान्॑ । स॒द॒स्या॑ने॒व । ए॒व य॑ज्ञ्यश॒सेन॑ । य॒ज्ञ्॒य॒श॒सेना᳚र्पयति । य॒ज्ञ्॒य॒श॒सेनेति॑ यज्ञ् - य॒श॒सेन॑ । अ॒र्प॒य॒तीत्य॑र्पयति । \newline

\textbf{Jatai Paata} \newline

1. आ॒त्मान॑ मे॒वै वात्मान॑ मा॒त्मान॑ मे॒व । \newline
2. ए॒व य॑ज्ञ्यश॒सेन॑ यज्ञ्यश॒से नै॒वैव य॑ज्ञ्यश॒सेन॑ । \newline
3. य॒ज्ञ्॒य॒श॒सेना᳚ र्पय त्यर्पयति यज्ञ्यश॒सेन॑ यज्ञ्यश॒सेना᳚ र्पयति । \newline
4. य॒ज्ञ्॒य॒श॒सेनेति॑ यज्ञ् - य॒श॒सेन॑ । \newline
5. अ॒र्प॒य॒ति॒ यदि॒ यद्य॑र्पय त्यर्पयति॒ यदि॑ । \newline
6. यदि॑ का॒मये॑त का॒मये॑त॒ यदि॒ यदि॑ का॒मये॑त । \newline
7. का॒मये॑त॒ यज॑मानं॒ ॅयज॑मानम् का॒मये॑त का॒मये॑त॒ यज॑मानम् । \newline
8. यज॑मानं ॅयज्ञ्यश॒सेन॑ यज्ञ्यश॒सेन॒ यज॑मानं॒ ॅयज॑मानं ॅयज्ञ्यश॒सेन॑ । \newline
9. य॒ज्ञ्॒य॒श॒सेना᳚ र्पयेय मर्पयेयं ॅयज्ञ्यश॒सेन॑ यज्ञ्यश॒सेना᳚ र्पयेयम् । \newline
10. य॒ज्ञ्॒य॒श॒सेनेति॑ यज्ञ् - य॒श॒सेन॑ । \newline
11. अ॒र्प॒ये॒य॒ मिती त्य॑र्पयेय मर्पयेय॒ मिति॑ । \newline
12. इत्य॑न्त॒रा ऽन्त॒ रेतीत्य॑न्त॒रा । \newline
13. अ॒न्त॒रा स॑दोहविर्द्धा॒ने स॑दोहविर्द्धा॒ने अ॑न्त॒रा ऽन्त॒रा स॑दोहविर्द्धा॒ने । \newline
14. स॒दो॒ह॒वि॒र्द्धा॒ने तिष्ठꣳ॒॒ स्तिष्ठन्᳚ थ्सदोहविर्द्धा॒ने स॑दोहविर्द्धा॒ने तिष्ठन्न्॑ । \newline
15. स॒दो॒ह॒वि॒र्द्धा॒ने इति॑ सदः - ह॒वि॒र्द्धा॒ने । \newline
16. तिष्ठ॒ न्नवाव॒ तिष्ठꣳ॒॒ स्तिष्ठ॒न्नव॑ । \newline
17. अव॑ नयेन् नये॒ दवाव॑ नयेत् । \newline
18. न॒ये॒द् यज॑मानं॒ ॅयज॑मानम् नयेन् नये॒द् यज॑मानम् । \newline
19. यज॑मान मे॒वैव यज॑मानं॒ ॅयज॑मान मे॒व । \newline
20. ए॒व य॑ज्ञ्यश॒सेन॑ यज्ञ्यश॒से नै॒वैव य॑ज्ञ्यश॒सेन॑ । \newline
21. य॒ज्ञ्॒य॒श॒सेना᳚ र्पय त्यर्पयति यज्ञ्यश॒सेन॑ यज्ञ्यश॒सेना᳚ र्पयति । \newline
22. य॒ज्ञ्॒य॒श॒सेनेति॑ यज्ञ् - य॒श॒सेन॑ । \newline
23. अ॒र्प॒य॒ति॒ यदि॒ यद्य॑र्पय त्यर्पयति॒ यदि॑ । \newline
24. यदि॑ का॒मये॑त का॒मये॑त॒ यदि॒ यदि॑ का॒मये॑त । \newline
25. का॒मये॑त सद॒स्या᳚न् थ्सद॒स्या᳚न् का॒मये॑त का॒मये॑त सद॒स्यान्॑ । \newline
26. स॒द॒स्यान्॑. यज्ञ्यश॒सेन॑ यज्ञ्यश॒सेन॑ सद॒स्या᳚न् थ्सद॒स्यान्॑. यज्ञ्यश॒सेन॑ । \newline
27. य॒ज्ञ्॒य॒श॒सेना᳚ र्पयेय मर्पयेयं ॅयज्ञ्यश॒सेन॑ यज्ञ्यश॒सेना᳚ र्पयेयम् । \newline
28. य॒ज्ञ्॒य॒श॒सेनेति॑ यज्ञ् - य॒श॒सेन॑ । \newline
29. अ॒र्प॒ये॒य॒ मिती त्य॑र्पयेय मर्पयेय॒ मिति॑ । \newline
30. इति॒ सदः॒ सद॒ इतीति॒ सदः॑ । \newline
31. सद॑ आ॒लभ्या॒ लभ्य॒ सदः॒ सद॑ आ॒लभ्य॑ । \newline
32. आ॒लभ्या वावा॒ लभ्या॒ लभ्याव॑ । \newline
33. आ॒लभ्येत्या᳚ - लभ्य॑ । \newline
34. अव॑ नयेन् नये॒ दवाव॑ नयेत् । \newline
35. न॒ये॒थ् स॒द॒स्या᳚न् थ्सद॒स्या᳚न् नयेन् नयेथ् सद॒स्यान्॑ । \newline
36. स॒द॒स्या॑ ने॒वैव स॑द॒स्या᳚न् थ्सद॒स्या॑ ने॒व । \newline
37. ए॒व य॑ज्ञ्यश॒सेन॑ यज्ञ्यश॒से नै॒वैव य॑ज्ञ्यश॒सेन॑ । \newline
38. य॒ज्ञ्॒य॒श॒सेना᳚ र्पय त्यर्पयति यज्ञ्यश॒सेन॑ यज्ञ्यश॒सेना᳚ र्पयति । \newline
39. य॒ज्ञ्॒य॒श॒सेनेति॑ यज्ञ् - य॒श॒सेन॑ । \newline
40. अ॒र्प॒य॒तीत्य॑र्पयति । \newline

\textbf{Ghana Paata } \newline

1. आ॒त्मान॑ मे॒वै वात्मान॑ मा॒त्मान॑ मे॒व य॑ज्ञ्यश॒सेन॑ यज्ञ्यश॒से नै॒वात्मान॑ मा॒त्मान॑ मे॒व य॑ज्ञ्यश॒सेन॑ । \newline
2. ए॒व य॑ज्ञ्यश॒सेन॑ यज्ञ्यश॒से नै॒वैव य॑ज्ञ्यश॒सेना᳚ र्पय त्यर्पयति यज्ञ्यश॒से नै॒वैव य॑ज्ञ्यश॒सेना᳚ र्पयति । \newline
3. य॒ज्ञ्॒य॒श॒सेना᳚ र्पय त्यर्पयति यज्ञ्यश॒सेन॑ यज्ञ्यश॒सेना᳚ र्पयति॒ यदि॒ यद्य॑र्पयति यज्ञ्यश॒सेन॑ यज्ञ्यश॒सेना᳚ र्पयति॒ यदि॑ । \newline
4. य॒ज्ञ्॒य॒श॒सेनेति॑ यज्ञ् - य॒श॒सेन॑ । \newline
5. अ॒र्प॒य॒ति॒ यदि॒ यद्य॑र्पय त्यर्पयति॒ यदि॑ का॒मये॑त का॒मये॑त॒ यद्य॑र्पय त्यर्पयति॒ यदि॑ का॒मये॑त । \newline
6. यदि॑ का॒मये॑त का॒मये॑त॒ यदि॒ यदि॑ का॒मये॑त॒ यज॑मानं॒ ॅयज॑मानम् का॒मये॑त॒ यदि॒ यदि॑ का॒मये॑त॒ यज॑मानम् । \newline
7. का॒मये॑त॒ यज॑मानं॒ ॅयज॑मानम् का॒मये॑त का॒मये॑त॒ यज॑मानं ॅयज्ञ्यश॒सेन॑ यज्ञ्यश॒सेन॒ यज॑मानम् का॒मये॑त का॒मये॑त॒ यज॑मानं ॅयज्ञ्यश॒सेन॑ । \newline
8. यज॑मानं ॅयज्ञ्यश॒सेन॑ यज्ञ्यश॒सेन॒ यज॑मानं॒ ॅयज॑मानं ॅयज्ञ्यश॒सेना᳚ र्पयेय मर्पयेयं ॅयज्ञ्यश॒सेन॒ यज॑मानं॒ ॅयज॑मानं ॅयज्ञ्यश॒सेना᳚ र्पयेयम् । \newline
9. य॒ज्ञ्॒य॒श॒सेना᳚र् पयेय मर्पयेयं ॅयज्ञ्यश॒सेन॑ यज्ञ्यश॒सेना᳚र् पयेय॒ मिती त्य॑र्पयेयं ॅयज्ञ्यश॒सेन॑ यज्ञ्यश॒सेना᳚र् पयेय॒ मिति॑ । \newline
10. य॒ज्ञ्॒य॒श॒सेनेति॑ यज्ञ् - य॒श॒सेन॑ । \newline
11. अ॒र्प॒ये॒य॒ मिती त्य॑र्पयेय मर्पयेय॒ मित्य॑न्त॒रा ऽन्त॒रे त्य॑र्पयेय मर्पयेय॒ मित्य॑न्त॒रा । \newline
12. इत्य॑न्त॒रा ऽन्त॒रे तीत्य॑न्त॒रा स॑दोहविर्द्धा॒ने स॑दोहविर्द्धा॒ने अ॑न्त॒रे तीत्य॑न्त॒रा स॑दोहविर्द्धा॒ने । \newline
13. अ॒न्त॒रा स॑दोहविर्द्धा॒ने स॑दोहविर्द्धा॒ने अ॑न्त॒रा ऽन्त॒रा स॑दोहविर्द्धा॒ने तिष्ठꣳ॒॒ 
स्तिष्ठन्᳚ थ्सदोहविर्द्धा॒ने अ॑न्त॒रा ऽन्त॒रा स॑दोहविर्द्धा॒ने तिष्ठन्न्॑ । \newline
14. स॒दो॒ह॒वि॒र्द्धा॒ने तिष्ठꣳ॒॒ स्तिष्ठन्᳚ थ्सदोहविर्द्धा॒ने स॑दोहविर्द्धा॒ने तिष्ठ॒न्-नवाव॒ तिष्ठन्᳚ थ्सदोहविर्द्धा॒ने स॑दोहविर्द्धा॒ने तिष्ठ॒न्-नव॑ । \newline
15. स॒दो॒ह॒वि॒र्द्धा॒ने इति॑ सदः - ह॒वि॒र्द्धा॒ने । \newline
16. तिष्ठ॒न्-नवाव॒ तिष्ठꣳ॒॒ स्तिष्ठ॒न्-नव॑ नयेन् नये॒ दव॒ तिष्ठꣳ॒॒ स्तिष्ठ॒न्-नव॑ नयेत् । \newline
17. अव॑ नयेन् नये॒ दवाव॑ नये॒द् यज॑मानं॒ ॅयज॑मानम् नये॒ दवाव॑ नये॒द् यज॑मानम् । \newline
18. न॒ये॒द् यज॑मानं॒ ॅयज॑मानम् नयेन् नये॒द् यज॑मान मे॒वैव यज॑मानम् नयेन् नये॒द् यज॑मान मे॒व । \newline
19. यज॑मान मे॒वैव यज॑मानं॒ ॅयज॑मान मे॒व य॑ज्ञ्यश॒सेन॑ यज्ञ्यश॒से नै॒व यज॑मानं॒ ॅयज॑मान मे॒व य॑ज्ञ्यश॒सेन॑ । \newline
20. ए॒व य॑ज्ञ्यश॒सेन॑ यज्ञ्यश॒से नै॒वैव य॑ज्ञ्यश॒से ना᳚र्पय त्यर्पयति यज्ञ्यश॒से नै॒वैव य॑ज्ञ्यश॒से ना᳚र्पयति । \newline
21. य॒ज्ञ्॒य॒श॒से ना᳚र्पय त्यर्पयति यज्ञ्यश॒सेन॑ यज्ञ्यश॒से ना᳚र्पयति॒ यदि॒ यद्य॑र्पयति यज्ञ्यश॒सेन॑ यज्ञ्यश॒से ना᳚र्पयति॒ यदि॑ । \newline
22. य॒ज्ञ्॒य॒श॒सेनेति॑ यज्ञ् - य॒श॒सेन॑ । \newline
23. अ॒र्प॒य॒ति॒ यदि॒ यद्य॑र्पय त्यर्पयति॒ यदि॑ का॒मये॑त का॒मये॑त॒ यद्य॑र्पय त्यर्पयति॒ यदि॑ का॒मये॑त । \newline
24. यदि॑ का॒मये॑त का॒मये॑त॒ यदि॒ यदि॑ का॒मये॑त सद॒स्या᳚न् थ्सद॒स्या᳚न् का॒मये॑त॒ यदि॒ यदि॑ का॒मये॑त सद॒स्यान्॑ । \newline
25. का॒मये॑त सद॒स्या᳚न् थ्सद॒स्या᳚न् का॒मये॑त का॒मये॑त सद॒स्यान्॑. यज्ञ्यश॒सेन॑ यज्ञ्यश॒सेन॑ सद॒स्या᳚न् का॒मये॑त का॒मये॑त सद॒स्यान्॑. यज्ञ्यश॒सेन॑ । \newline
26. स॒द॒स्यान्॑. यज्ञ्यश॒सेन॑ यज्ञ्यश॒सेन॑ सद॒स्या᳚न् थ्सद॒स्यान्॑. यज्ञ्यश॒से ना᳚र्पयेय मर्पयेयं ॅयज्ञ्यश॒सेन॑ सद॒स्या᳚न् थ्सद॒स्यान्॑. यज्ञ्यश॒से ना᳚र्पयेयम् । \newline
27. य॒ज्ञ्॒य॒श॒से ना᳚र्पयेय मर्पयेयं ॅयज्ञ्यश॒सेन॑ यज्ञ्यश॒से ना᳚र्पयेय॒ मिती त्य॑र्पयेयं ॅयज्ञ्यश॒सेन॑ यज्ञ्यश॒से ना᳚र्पयेय॒ मिति॑ । \newline
28. य॒ज्ञ्॒य॒श॒सेनेति॑ यज्ञ् - य॒श॒सेन॑ । \newline
29. अ॒र्प॒ये॒य॒ मिती त्य॑र्पयेय मर्पयेय॒ मिति॒ सदः॒ सद॒ इत्य॑र्पयेय मर्पयेय॒ मिति॒ सदः॑ । \newline
30. इति॒ सदः॒ सद॒ इतीति॒ सद॑ आ॒लभ्या॒ लभ्य॒ सद॒ इतीति॒ सद॑ आ॒लभ्य॑ । \newline
31. सद॑ आ॒लभ्या॒ लभ्य॒ सदः॒ सद॑ आ॒लभ्या वावा॒ लभ्य॒ सदः॒ सद॑ आ॒लभ्याव॑ । \newline
32. आ॒लभ्या वावा॒ लभ्या॒ लभ्याव॑ नयेन् नये॒ दवा॒ लभ्या॒ लभ्याव॑ नयेत् । \newline
33. आ॒लभ्येत्या᳚ - लभ्य॑ । \newline
34. अव॑ नयेन् नये॒ दवाव॑ नयेथ् सद॒स्या᳚न् थ्सद॒स्या᳚न् नये॒ दवाव॑ नयेथ् सद॒स्यान्॑ । \newline
35. न॒ये॒थ् स॒द॒स्या᳚न् थ्सद॒स्या᳚न् नयेन् नयेथ् सद॒स्या॑ ने॒वैव स॑द॒स्या᳚न् नयेन् नयेथ् सद॒स्या॑ने॒व । \newline
36. स॒द॒स्या॑ ने॒वैव स॑द॒स्या᳚न् थ्सद॒स्या॑ने॒व य॑ज्ञ्यश॒सेन॑ यज्ञ्यश॒सेनै॒व स॑द॒स्या᳚न् थ्सद॒स्या॑-ने॒व य॑ज्ञ्यश॒सेन॑ । \newline
37. ए॒व य॑ज्ञ्यश॒सेन॑ यज्ञ्यश॒से नै॒वैव य॑ज्ञ्यश॒से ना᳚र्पय त्यर्पयति यज्ञ्यश॒से नै॒वैव य॑ज्ञ्यश॒से ना᳚र्पयति । \newline
38. य॒ज्ञ्॒य॒श॒से ना᳚र्पय त्यर्पयति यज्ञ्यश॒सेन॑ यज्ञ्यश॒से ना᳚र्पयति । \newline
39. य॒ज्ञ्॒य॒श॒सेनेति॑ यज्ञ् - य॒श॒सेन॑ । \newline
40. अ॒र्प॒य॒तीत्य॑र्पयति । \newline
\pagebreak
\markright{ TS 6.5.2.1  \hfill https://www.vedavms.in \hfill}

\section{ TS 6.5.2.1 }

\textbf{TS 6.5.2.1 } \newline
\textbf{Samhita Paata} \newline

आयु॒र्वा ए॒तद्-य॒ज्ञ्स्य॒ यद् ध्रु॒व उ॑त्त॒मो ग्रहा॑णां गृह्यते॒ तस्मा॒दायुः॑ प्रा॒णाना॑मुत्त॒मं मू॒र्द्धानं॑ दि॒वो अ॑र॒तिं पृ॑थि॒व्या इत्या॑ह मू॒र्द्धान॑मे॒वैनꣳ॑ समा॒नानां᳚ करोति वैश्वान॒रमृ॒ताय॑ जा॒तम॒ग्निमित्या॑ह वैश्वान॒रꣳ हि दे॒वत॒याऽऽयु॑रुभ॒यतो॑ वैश्वानरो गृह्यते॒ तस्मा॑दुभ॒यतः॑ प्रा॒णा अ॒धस्ता᳚-च्चो॒परि॑ष्टा-च्चा॒र्द्धिनो॒ऽन्ये ग्रहा॑ गृ॒ह्यन्ते॒ऽर्द्धी ध्रु॒वस्तस्मा॑- [  ] \newline

\textbf{Pada Paata} \newline

आयुः॑ । वै । ए॒तत् । य॒ज्ञ्स्य॑ । यत् । ध्रु॒वः । उ॒त्त॒म इत्यु॑त् - त॒मः । ग्रहा॑णाम् । गृ॒ह्य॒ते॒ । तस्मा᳚त् । आयुः॑ । प्रा॒णाना॒मिति॑ प्र - अ॒नाना᳚म् । उ॒त्त॒ममित्यु॑त्-त॒मम् । मू॒द्‌र्धान᳚म् । दि॒वः । अ॒र॒तिम् । पृ॒थि॒व्याः । इति॑ । आ॒ह॒ । मू॒द्‌र्धान᳚म् । ए॒व । ए॒न॒म् । स॒मा॒नाना᳚म् । क॒रो॒ति॒ । वै॒श्वा॒न॒रम् । ऋ॒ताय॑ । जा॒तम् । अ॒ग्निम् । इति॑ । आ॒ह॒ । वै॒श्वा॒न॒रम् । हि । दे॒वत॑या । आयुः॑ । उ॒भ॒यतो॑वैश्वानर॒ इत्यु॑भ॒यतः॑ - वै॒श्वा॒न॒रः॒ । गृ॒ह्य॒ते॒ । तस्मा᳚त् । उ॒भ॒यतः॑ । प्रा॒णा इति॑ प्र - अ॒नाः । अ॒धस्ता᳚त् । च॒ । उ॒परि॑ष्टात् । च॒ । अ॒र्द्धिनः॑ । अ॒न्ये । ग्रहाः᳚ । गृ॒ह्यन्ते᳚ । अ॒द्‌र्धी । ध्रु॒वः । तस्मा᳚त् ।  \newline


\textbf{Krama Paata} \newline

आयु॒र् वै । वा ए॒तत् । ए॒तद् य॒ज्ञ्स्य॑ । य॒ज्ञ्स्य॒ यत् । यद् ध्रु॒वः । ध्रु॒व उ॑त्त॒मः । उ॒त्त॒मो ग्रहा॑णाम् । उ॒त्त॒म इत्यु॑त् - त॒मः । ग्रहा॑णाम् गृह्यते । गृ॒ह्य॒ते॒ तस्मा᳚त् । तस्मा॒दायुः॑ । आयुः॑ प्रा॒णाना᳚म् । प्रा॒णाना॑मुत्त॒मम् । प्रा॒णाना॒मिति॑ प्र - अ॒नाना᳚म् । उ॒त्त॒मम् मू॒र्द्धान᳚म् । उ॒त्त॒ममित्यु॑त् - त॒मम् । मू॒र्द्धान॑म् दि॒वः । दि॒वो अ॑र॒तिम् । अ॒र॒तिम् पृ॑थि॒व्याः । पृ॒थि॒व्या इति॑ । इत्या॑ह । आ॒ह॒ मू॒र्द्धान᳚म् । मू॒र्द्धान॑मे॒व । ए॒वैन᳚म् । ए॒नꣳ॒॒ स॒मा॒नाना᳚म् । स॒मा॒नाना᳚म् करोति । क॒रो॒ति॒ वै॒श्वा॒न॒रम् । वै॒श्वा॒न॒रमृ॒ताय॑ । ऋ॒ताय॑ जा॒तम् । जा॒तम॒ग्निम् । अ॒ग्निमिति॑ । इत्या॑ह । आ॒ह॒ वै॒श्वा॒न॒रम् । वै॒श्वा॒न॒रꣳ हि । हि दे॒वत॑या । दे॒वत॒याऽऽयुः॑ । आयु॑रुभ॒यतो॑वैश्वानरः । उ॒भ॒यतो॑वैश्वानरो गृह्यते । उ॒भ॒यतो॑वैश्वानर॒ इत्यु॑भ॒यतः॑ - वै॒श्वा॒न॒रः॒ । गृ॒ह्य॒ते॒ तस्मा᳚त् । तस्मा॑दुभ॒यतः॑ । उ॒भ॒यतः॑ प्रा॒णाः । प्रा॒णा अ॒धस्ता᳚त् । प्रा॒णा इति॑ प्र - अ॒नाः । अ॒धस्ता᳚च् च । चो॒परि॑ष्टात् । उ॒परि॑ष्टाच् च । चा॒र्द्धिनः॑ । अ॒र्द्धिनो॒ऽन्ये । अ॒न्ये ग्रहाः᳚ । ग्रहा॑ गृ॒ह्यन्ते᳚ । गृ॒ह्यन्ते॒ऽर्द्धी । अ॒र्द्धी ध्रु॒वः । ध्रु॒व स्तस्मा᳚त् । तस्मा॑द॒र्द्धी \newline

\textbf{Jatai Paata} \newline

1. आयु॒र् वै वा आयु॒ रायु॒र् वै । \newline
2. वा ए॒त दे॒तद् वै वा ए॒तत् । \newline
3. ए॒तद् य॒ज्ञ्स्य॑ य॒ज्ञ्स्यै॒त दे॒तद् य॒ज्ञ्स्य॑ । \newline
4. य॒ज्ञ्स्य॒ यद् यद् य॒ज्ञ्स्य॑ य॒ज्ञ्स्य॒ यत् । \newline
5. यद् ध्रु॒वो ध्रु॒वो यद् यद् ध्रु॒वः । \newline
6. ध्रु॒व उ॑त्त॒म उ॑त्त॒मो ध्रु॒वो ध्रु॒व उ॑त्त॒मः । \newline
7. उ॒त्त॒मो ग्रहा॑णा॒म् ग्रहा॑णा मुत्त॒म उ॑त्त॒मो ग्रहा॑णाम् । \newline
8. उ॒त्त॒म इत्यु॑त् - त॒मः । \newline
9. ग्रहा॑णाम् गृह्यते गृह्यते॒ ग्रहा॑णा॒म् ग्रहा॑णाम् गृह्यते । \newline
10. गृ॒ह्य॒ते॒ तस्मा॒त् तस्मा᳚द् गृह्यते गृह्यते॒ तस्मा᳚त् । \newline
11. तस्मा॒ दायु॒ रायु॒ष् टस्मा॒त् तस्मा॒ दायुः॑ । \newline
12. आयुः॑ प्रा॒णाना᳚म् प्रा॒णाना॒ मायु॒ रायुः॑ प्रा॒णाना᳚म् । \newline
13. प्रा॒णाना॑ मुत्त॒म मु॑त्त॒मम् प्रा॒णाना᳚म् प्रा॒णाना॑ मुत्त॒मम् । \newline
14. प्रा॒णाना॒मिति॑ प्र - अ॒नाना᳚म् । \newline
15. उ॒त्त॒मम् मू॒र्द्धान॑म् मू॒र्द्धान॑ मुत्त॒म मु॑त्त॒मम् मू॒र्द्धान᳚म् । \newline
16. उ॒त्त॒ममित्यु॑त् - त॒मम् । \newline
17. मू॒र्द्धान॑म् दि॒वो दि॒वो मू॒र्द्धान॑म् मू॒र्द्धान॑म् दि॒वः । \newline
18. दि॒वो अ॑र॒ति म॑र॒तिम् दि॒वो दि॒वो अ॑र॒तिम् । \newline
19. अ॒र॒तिम् पृ॑थि॒व्याः पृ॑थि॒व्या अ॑र॒ति म॑र॒तिम् पृ॑थि॒व्याः । \newline
20. पृ॒थि॒व्या इतीति॑ पृथि॒व्याः पृ॑थि॒व्या इति॑ । \newline
21. इत्या॑हा॒हे तीत्या॑ह । \newline
22. आ॒ह॒ मू॒र्द्धान॑म् मू॒र्द्धान॑ माहाह मू॒र्द्धान᳚म् । \newline
23. मू॒र्द्धान॑ मे॒वैव मू॒र्द्धान॑म् मू॒र्द्धान॑ मे॒व । \newline
24. ए॒वैन॑ मेन मे॒वै वैन᳚म् । \newline
25. ए॒नꣳ॒॒ स॒मा॒नानाꣳ॑ समा॒नाना॑ मेन मेनꣳ समा॒नाना᳚म् । \newline
26. स॒मा॒नाना᳚म् करोति करोति समा॒नानाꣳ॑ समा॒नाना᳚म् करोति । \newline
27. क॒रो॒ति॒ वै॒श्वा॒न॒रं ॅवै᳚श्वान॒रम् क॑रोति करोति वैश्वान॒रम् । \newline
28. वै॒श्वा॒न॒र मृ॒ताय॒ र्‌ताय॑ वैश्वान॒रं ॅवै᳚श्वान॒र मृ॒ताय॑ । \newline
29. ऋ॒ताय॑ जा॒तम् जा॒त मृ॒ताय॒ र्‌ताय॑ जा॒तम् । \newline
30. जा॒त म॒ग्नि म॒ग्निम् जा॒तम् जा॒त म॒ग्निम् । \newline
31. अ॒ग्नि मिती त्य॒ग्नि म॒ग्नि मिति॑ । \newline
32. इत्या॑हा॒हे तीत्या॑ह । \newline
33. आ॒ह॒ वै॒श्वा॒न॒रं ॅवै᳚श्वान॒र मा॑हाह वैश्वान॒रम् । \newline
34. वै॒श्वा॒न॒रꣳ हि हि वै᳚श्वान॒रं ॅवै᳚श्वान॒रꣳ हि । \newline
35. हि दे॒वत॑या दे॒वत॑या॒ हि हि दे॒वत॑या । \newline
36. दे॒वत॒या ऽऽयु॒ रायु॑र् दे॒वत॑या दे॒वत॒या ऽऽयुः॑ । \newline
37. आयु॑ रुभ॒यतो॑वैश्वानर उभ॒यतो॑वैश्वानर॒ आयु॒ रायु॑ रुभ॒यतो॑वैश्वानरः । \newline
38. उ॒भ॒यतो॑वैश्वानरो गृह्यते गृह्यत उभ॒यतो॑वैश्वानर उभ॒यतो॑वैश्वानरो गृह्यते । \newline
39. उ॒भ॒यतो॑वैश्वानर॒ इत्यु॑भ॒यतः॑ - वै॒श्वा॒न॒रः॒ । \newline
40. गृ॒ह्य॒ते॒ तस्मा॒त् तस्मा᳚द् गृह्यते गृह्यते॒ तस्मा᳚त् । \newline
41. तस्मा॑ दुभ॒यत॑ उभ॒यत॒ स्तस्मा॒त् तस्मा॑ दुभ॒यतः॑ । \newline
42. उ॒भ॒यतः॑ प्रा॒णाः प्रा॒णा उ॑भ॒यत॑ उभ॒यतः॑ प्रा॒णाः । \newline
43. प्रा॒णा अ॒धस्ता॑ द॒धस्ता᳚त् प्रा॒णाः प्रा॒णा अ॒धस्ता᳚त् । \newline
44. प्रा॒णा इति॑ प्र - अ॒नाः । \newline
45. अ॒धस्ता᳚च् च चा॒धस्ता॑ द॒धस्ता᳚च् च । \newline
46. चो॒परि॑ष्टा दु॒परि॑ष्टाच् च चो॒परि॑ष्टात् । \newline
47. उ॒परि॑ष्टाच् च चो॒परि॑ष्टा दु॒परि॑ष्टाच् च । \newline
48. चा॒र्द्धिनो॒ ऽर्द्धिन॑श्च चा॒र्द्धिनः॑ । \newline
49. अ॒र्द्धिनो॒ ऽन्ये᳚(1॒) ऽन्ये᳚ ऽर्द्धिनो॒ ऽर्द्धिनो॒ ऽन्ये । \newline
50. अ॒न्ये ग्रहा॒ ग्रहा॑ अ॒न्ये᳚ ऽन्ये ग्रहाः᳚ । \newline
51. ग्रहा॑ गृ॒ह्यन्ते॑ गृ॒ह्यन्ते॒ ग्रहा॒ ग्रहा॑ गृ॒ह्यन्ते᳚ । \newline
52. गृ॒ह्यन्ते॒ ऽर्द्ध्य॑र्द्धी गृ॒ह्यन्ते॑ गृ॒ह्यन्ते॒ ऽर्द्धी । \newline
53. अ॒र्द्धी ध्रु॒वो ध्रु॒वो᳚(1॒) ऽर्द्ध्य॑र्द्धी ध्रु॒वः । \newline
54. ध्रु॒व स्तस्मा॒त् तस्मा᳚द् ध्रु॒वो ध्रु॒व स्तस्मा᳚त् । \newline
55. तस्मा॑ द॒र्द्ध्य॑र्द्धी तस्मा॒त् तस्मा॑ द॒र्द्धी । \newline

\textbf{Ghana Paata } \newline

1. आयु॒र् वै वा आयु॒ रायु॒र् वा ए॒त दे॒तद् वा आयु॒ रायु॒र् वा ए॒तत् । \newline
2. वा ए॒त दे॒तद् वै वा ए॒तद् य॒ज्ञ्स्य॑ य॒ज्ञ् स्यै॒तद् वै वा ए॒तद् य॒ज्ञ्स्य॑ । \newline
3. ए॒तद् य॒ज्ञ्स्य॑ य॒ज्ञ् स्यै॒त दे॒तद् य॒ज्ञ्स्य॒ यद् यद् य॒ज्ञ् स्यै॒त दे॒तद् य॒ज्ञ्स्य॒ यत् । \newline
4. य॒ज्ञ्स्य॒ यद् यद् य॒ज्ञ्स्य॑ य॒ज्ञ्स्य॒ यद् ध्रु॒वो ध्रु॒वो यद् य॒ज्ञ्स्य॑ य॒ज्ञ्स्य॒ यद् ध्रु॒वः । \newline
5. यद् ध्रु॒वो ध्रु॒वो यद् यद् ध्रु॒व उ॑त्त॒म उ॑त्त॒मो ध्रु॒वो यद् यद् ध्रु॒व उ॑त्त॒मः । \newline
6. ध्रु॒व उ॑त्त॒म उ॑त्त॒मो ध्रु॒वो ध्रु॒व उ॑त्त॒मो ग्रहा॑णा॒म् ग्रहा॑णा मुत्त॒मो ध्रु॒वो ध्रु॒व उ॑त्त॒मो ग्रहा॑णाम् । \newline
7. उ॒त्त॒मो ग्रहा॑णा॒म् ग्रहा॑णा मुत्त॒म उ॑त्त॒मो ग्रहा॑णाम् गृह्यते गृह्यते॒ ग्रहा॑णा मुत्त॒म उ॑त्त॒मो ग्रहा॑णाम् गृह्यते । \newline
8. उ॒त्त॒म इत्यु॑त् - त॒मः । \newline
9. ग्रहा॑णाम् गृह्यते गृह्यते॒ ग्रहा॑णा॒म् ग्रहा॑णाम् गृह्यते॒ तस्मा॒त् तस्मा᳚द् गृह्यते॒ ग्रहा॑णा॒म् ग्रहा॑णाम् गृह्यते॒ तस्मा᳚त् । \newline
10. गृ॒ह्य॒ते॒ तस्मा॒त् तस्मा᳚द् गृह्यते गृह्यते॒ तस्मा॒ दायु॒ रायु॒ष् टस्मा᳚द् गृह्यते गृह्यते॒ तस्मा॒ दायुः॑ । \newline
11. तस्मा॒ दायु॒ रायु॒ष् टस्मा॒त् तस्मा॒ दायुः॑ प्रा॒णाना᳚म् प्रा॒णाना॒ मायु॒ष् टस्मा॒त् तस्मा॒ दायुः॑ प्रा॒णाना᳚म् । \newline
12. आयुः॑ प्रा॒णाना᳚म् प्रा॒णाना॒ मायु॒ रायुः॑ प्रा॒णाना॑ मुत्त॒म मु॑त्त॒मम् प्रा॒णाना॒ मायु॒ रायुः॑ प्रा॒णाना॑ मुत्त॒मम् । \newline
13. प्रा॒णाना॑ मुत्त॒म मु॑त्त॒मम् प्रा॒णाना᳚म् प्रा॒णाना॑ मुत्त॒मम् मू॒र्द्धान॑म् मू॒र्द्धान॑ मुत्त॒मम् प्रा॒णाना᳚म् प्रा॒णाना॑ मुत्त॒मम् मू॒र्द्धान᳚म् । \newline
14. प्रा॒णाना॒मिति॑ प्र - अ॒नाना᳚म् । \newline
15. उ॒त्त॒मम् मू॒र्द्धान॑म् मू॒र्द्धान॑ मुत्त॒म मु॑त्त॒मम् मू॒र्द्धान॑म् दि॒वो दि॒वो मू॒र्द्धान॑ मुत्त॒म मु॑त्त॒मम् मू॒र्द्धान॑म् दि॒वः । \newline
16. उ॒त्त॒ममित्यु॑त् - त॒मम् । \newline
17. मू॒र्द्धान॑म् दि॒वो दि॒वो मू॒र्द्धान॑म् मू॒र्द्धान॑म् दि॒वो अ॑र॒ति म॑र॒तिम् दि॒वो मू॒र्द्धान॑म् मू॒र्द्धान॑म् दि॒वो अ॑र॒तिम् । \newline
18. दि॒वो अ॑र॒ति म॑र॒तिम् दि॒वो दि॒वो अ॑र॒तिम् पृ॑थि॒व्याः पृ॑थि॒व्या अ॑र॒तिम् दि॒वो दि॒वो अ॑र॒तिम् पृ॑थि॒व्याः । \newline
19. अ॒र॒तिम् पृ॑थि॒व्याः पृ॑थि॒व्या अ॑र॒ति म॑र॒तिम् पृ॑थि॒व्या इतीति॑ पृथि॒व्या अ॑र॒ति म॑र॒तिम् पृ॑थि॒व्या इति॑ । \newline
20. पृ॒थि॒व्या इतीति॑ पृथि॒व्याः पृ॑थि॒व्या इत्या॑हा॒हेति॑ पृथि॒व्याः पृ॑थि॒व्या इत्या॑ह । \newline
21. इत्या॑हा॒हे तीत्या॑ह मू॒र्द्धान॑म् मू॒र्द्धान॑ मा॒हे तीत्या॑ह मू॒र्द्धान᳚म् । \newline
22. आ॒ह॒ मू॒र्द्धान॑म् मू॒र्द्धान॑ माहाह मू॒र्द्धान॑ मे॒वैव मू॒र्द्धान॑ माहाह मू॒र्द्धान॑ मे॒व । \newline
23. मू॒र्द्धान॑ मे॒वैव मू॒र्द्धान॑म् मू॒र्द्धान॑ मे॒वैन॑ मेन मे॒व मू॒र्द्धान॑म् मू॒र्द्धान॑ मे॒वैन᳚म् । \newline
24. ए॒वैन॑ मेन मे॒वै वैनꣳ॑ समा॒नानाꣳ॑ समा॒नाना॑ मेन मे॒वै वैनꣳ॑ समा॒नाना᳚म् । \newline
25. ए॒नꣳ॒॒ स॒मा॒नानाꣳ॑ समा॒नाना॑ मेन मेनꣳ समा॒नाना᳚म् करोति करोति समा॒नाना॑ मेन मेनꣳ समा॒नाना᳚म् करोति । \newline
26. स॒मा॒नाना᳚म् करोति करोति समा॒नानाꣳ॑ समा॒नाना᳚म् करोति वैश्वान॒रं ॅवै᳚श्वान॒रम् क॑रोति समा॒नानाꣳ॑ समा॒नाना᳚म् करोति वैश्वान॒रम् । \newline
27. क॒रो॒ति॒ वै॒श्वा॒न॒रं ॅवै᳚श्वान॒रम् क॑रोति करोति वैश्वान॒र मृ॒ताय॒ र्‌ताय॑ वैश्वान॒रम् क॑रोति करोति वैश्वान॒र मृ॒ताय॑ । \newline
28. वै॒श्वा॒न॒र मृ॒ताय॒ र्‌ताय॑ वैश्वान॒रं ॅवै᳚श्वान॒र मृ॒ताय॑ जा॒तम् जा॒त मृ॒ताय॑ वैश्वान॒रं ॅवै᳚श्वान॒र मृ॒ताय॑ जा॒तम् । \newline
29. ऋ॒ताय॑ जा॒तम् जा॒त मृ॒ताय॒ र्‌ताय॑ जा॒त म॒ग्नि म॒ग्निम् जा॒त मृ॒ताय॒ र्‌ताय॑ जा॒त म॒ग्निम् । \newline
30. जा॒त म॒ग्नि म॒ग्निम् जा॒तम् जा॒त म॒ग्नि मिती त्य॒ग्निम् जा॒तम् जा॒त म॒ग्नि मिति॑ । \newline
31. अ॒ग्नि मिती त्य॒ग्नि म॒ग्नि मित्या॑हा॒ हेत्य॒ग्नि म॒ग्नि मित्या॑ह । \newline
32. इत्या॑हा॒हे तीत्या॑ह वैश्वान॒रं ॅवै᳚श्वान॒र मा॒हे तीत्या॑ह वैश्वान॒रम् । \newline
33. आ॒ह॒ वै॒श्वा॒न॒रं ॅवै᳚श्वान॒र मा॑हाह वैश्वान॒रꣳ हि हि वै᳚श्वान॒र मा॑हाह वैश्वान॒रꣳ हि । \newline
34. वै॒श्वा॒न॒रꣳ हि हि वै᳚श्वान॒रं ॅवै᳚श्वान॒रꣳ हि दे॒वत॑या दे॒वत॑या॒ हि वै᳚श्वान॒रं ॅवै᳚श्वान॒रꣳ हि दे॒वत॑या । \newline
35. हि दे॒वत॑या दे॒वत॑या॒ हि हि दे॒वत॒या ऽऽयु॒ रायु॑र् दे॒वत॑या॒ हि हि दे॒वत॒या ऽऽयुः॑ । \newline
36. दे॒वत॒या ऽऽयु॒ रायु॑र् दे॒वत॑या दे॒वत॒या ऽऽयु॑ रुभ॒यतो॑वैश्वानर उभ॒यतो॑वैश्वानर॒ आयु॑र् दे॒वत॑या दे॒वत॒या ऽऽयु॑ रुभ॒यतो॑वैश्वानरः । \newline
37. आयु॑ रुभ॒यतो॑वैश्वानर उभ॒यतो॑वैश्वानर॒ आयु॒ रायु॑ रुभ॒यतो॑वैश्वानरो गृह्यते गृह्यत उभ॒यतो॑वैश्वानर॒ आयु॒ रायु॑ रुभ॒यतो॑वैश्वानरो गृह्यते । \newline
38. उ॒भ॒यतो॑वैश्वानरो गृह्यते गृह्यत उभ॒यतो॑वैश्वानर उभ॒यतो॑वैश्वानरो गृह्यते॒ तस्मा॒त् तस्मा᳚द् गृह्यत उभ॒यतो॑वैश्वानर उभ॒यतो॑वैश्वानरो गृह्यते॒ तस्मा᳚त् । \newline
39. उ॒भ॒यतो॑वैश्वानर॒ इत्यु॑भ॒यतः॑ - वै॒श्वा॒न॒रः॒ । \newline
40. गृ॒ह्य॒ते॒ तस्मा॒त् तस्मा᳚द् गृह्यते गृह्यते॒ तस्मा॑ दुभ॒यत॑ उभ॒यत॒ स्तस्मा᳚द् गृह्यते गृह्यते॒ तस्मा॑ दुभ॒यतः॑ । \newline
41. तस्मा॑ दुभ॒यत॑ उभ॒यत॒ स्तस्मा॒त् तस्मा॑ दुभ॒यतः॑ प्रा॒णाः प्रा॒णा उ॑भ॒यत॒ स्तस्मा॒त् तस्मा॑ दुभ॒यतः॑ प्रा॒णाः । \newline
42. उ॒भ॒यतः॑ प्रा॒णाः प्रा॒णा उ॑भ॒यत॑ उभ॒यतः॑ प्रा॒णा अ॒धस्ता॑ द॒धस्ता᳚त् प्रा॒णा उ॑भ॒यत॑ उभ॒यतः॑ प्रा॒णा अ॒धस्ता᳚त् । \newline
43. प्रा॒णा अ॒धस्ता॑ द॒धस्ता᳚त् प्रा॒णाः प्रा॒णा अ॒धस्ता᳚च् च चा॒धस्ता᳚त् प्रा॒णाः प्रा॒णा अ॒धस्ता᳚च् च । \newline
44. प्रा॒णा इति॑ प्र - अ॒नाः । \newline
45. अ॒धस्ता᳚च् च चा॒धस्ता॑ द॒धस्ता᳚च् चो॒परि॑ष्टा दु॒परि॑ष्टाच् चा॒धस्ता॑ द॒धस्ता᳚च् चो॒परि॑ष्टात् । \newline
46. चो॒परि॑ष्टा दु॒परि॑ष्टाच् च चो॒परि॑ष्टाच् च चो॒परि॑ष्टाच् च चो॒परि॑ष्टाच् च । \newline
47. उ॒परि॑ष्टाच् च चो॒परि॑ष्टा दु॒परि॑ष्टाच् चा॒र्द्धिनो॒ ऽर्द्धिन॑ श्चो॒परि॑ष्टा दु॒परि॑ष्टाच् चा॒र्द्धिनः॑ । \newline
48. चा॒र्द्धिनो॒ ऽर्द्धिन॑श्च चा॒र्द्धिनो॒ ऽन्ये᳚(1॒) ऽन्ये᳚ ऽर्द्धिन॑श्च चा॒र्द्धिनो॒ ऽन्ये । \newline
49. अ॒र्द्धिनो॒ ऽन्ये᳚(1॒) ऽन्ये᳚ ऽर्द्धिनो॒ ऽर्द्धिनो॒ ऽन्ये ग्रहा॒ ग्रहा॑ अ॒न्ये᳚ ऽर्द्धिनो॒ ऽर्द्धिनो॒ ऽन्ये ग्रहाः᳚ । \newline
50. अ॒न्ये ग्रहा॒ ग्रहा॑ अ॒न्ये᳚ ऽन्ये ग्रहा॑ गृ॒ह्यन्ते॑ गृ॒ह्यन्ते॒ ग्रहा॑ अ॒न्ये᳚ ऽन्ये ग्रहा॑ गृ॒ह्यन्ते᳚ । \newline
51. ग्रहा॑ गृ॒ह्यन्ते॑ गृ॒ह्यन्ते॒ ग्रहा॒ ग्रहा॑ गृ॒ह्यन्ते॒ ऽर्द्ध्य॑र्द्धी गृ॒ह्यन्ते॒ ग्रहा॒ ग्रहा॑ गृ॒ह्यन्ते॒ ऽर्द्धी । \newline
52. गृ॒ह्यन्ते॒ ऽर्द्ध्य॑र्द्धी गृ॒ह्यन्ते॑ गृ॒ह्यन्ते॒ ऽर्द्धी ध्रु॒वो ध्रु॒वो᳚ ऽर्द्धी गृ॒ह्यन्ते॑ गृ॒ह्यन्ते॒ ऽर्द्धी ध्रु॒वः । \newline
53. अ॒र्द्धी ध्रु॒वो ध्रु॒वो᳚(1॒) ऽर्द्ध्य॑र्द्धी ध्रु॒व स्तस्मा॒त् तस्मा᳚द् ध्रु॒वो᳚(1॒) ऽर्द्ध्य॑र्द्धी ध्रु॒व स्तस्मा᳚त् । \newline
54. ध्रु॒व स्तस्मा॒त् तस्मा᳚द् ध्रु॒वो ध्रु॒व स्तस्मा॑ द॒र्द्ध्य॑र्द्धी तस्मा᳚द् ध्रु॒वो ध्रु॒व स्तस्मा॑ द॒र्द्धी । \newline
55. तस्मा॑ द॒र्द्ध्य॑र्द्धी तस्मा॒त् तस्मा॑ द॒र्द्ध्यवा॒ ङवा॑ ङ॒र्द्धी तस्मा॒त् तस्मा॑ द॒र्द्ध्यवाङ्॑ । \newline
\pagebreak
\markright{ TS 6.5.2.2  \hfill https://www.vedavms.in \hfill}

\section{ TS 6.5.2.2 }

\textbf{TS 6.5.2.2 } \newline
\textbf{Samhita Paata} \newline

-द॒र्द्ध्यवा᳚ङ् प्रा॒णो᳚ऽन्येषां᳚ प्रा॒णाना॒मुपो᳚प्ते॒ऽन्ये ग्रहाः᳚ सा॒द्यन्तेऽनु॑पोप्ते ध्रु॒वस्तस्मा॑-द॒स्थ्नान्याः प्र॒जाः प्र॑ति॒तिष्ठ॑न्ति माꣳ॒॒सेना॒न्या असु॑रा॒ वा उ॑त्तर॒तः पृ॑थि॒वीं प॒र्याचि॑कीर्.ष॒न् तां दे॒वा ध्रु॒वेणा॑दृꣳह॒न् तद् ध्रु॒वस्य॑ ध्रुव॒त्वं ॅयद् ध्रु॒व उ॑त्तर॒तः सा॒द्यते॒ धृत्या॒ आयु॒र्वा ए॒तद्-य॒ज्ञ्स्य॒ यद् ध्रु॒व आ॒त्मा होता॒ यद्धो॑तृचम॒से ध्रु॒व-म॑व॒नय॑त्या॒त्मन्ने॒व य॒ज्ञ्स्या- [  ] \newline

\textbf{Pada Paata} \newline

अ॒र्द्धि । अवाङ्॑ । प्रा॒ण इति॑ प्र - अ॒नः । अ॒न्येषा᳚म् । प्रा॒णाना॒मिति॑ प्र - अ॒नाना᳚म् । उपो᳚प्त॒ इत्युप॑ - उ॒प्ते॒ । अ॒न्ये । ग्रहाः᳚ । सा॒द्यन्ते᳚ । अनु॑पोप्त॒ इत्यनु॑प - उ॒प्ते॒ । ध्रु॒वः । तस्मा᳚त् । अ॒स्थ्ना । अ॒न्याः । प्र॒जा इति॑ प्र - जाः । प्र॒ति॒तिष्ठ॒न्तीति॑ प्रति - तिष्ठ॑न्ति । माꣳ॒॒सेन॑ । अ॒न्याः । असु॑राः । वै । उ॒त्त॒र॒त इत्यु॑त् - त॒र॒तः । पृ॒थि॒वीम् । प॒र्याचि॑कीर्.ष॒न्निति॑ परि - आचि॑कीर्.षन्न् । ताम् । दे॒वाः । ध्रु॒वेण॑ । अ॒दृ॒ह॒न्न् । तत् । ध्रु॒वस्य॑ । ध्रु॒व॒त्वमिति॑ ध्रुव - त्वम् । यत् । ध्रु॒वः । उ॒त्त॒र॒त इत्यु॑त् - त॒र॒तः । सा॒द्यते᳚ । धृत्यै᳚ । आयुः॑ । वै । ए॒तत् । य॒ज्ञ्स्य॑ । यत् । ध्रु॒वः । आ॒त्मा । होता᳚ । यत् । हो॒तृ॒च॒म॒स इति॑ होतृ - च॒म॒से । ध्रु॒वम् । अ॒व॒नय॒तीत्य॑व - नय॑ति । आ॒त्मन्न् । ए॒व । य॒ज्ञ्स्य॑ ।  \newline


\textbf{Krama Paata} \newline

अ॒र्द्ध्यवाङ्॑ । अवा᳚ङ्‍ प्रा॒णः । प्रा॒णो᳚ऽन्येषा᳚म् । प्रा॒ण इति॑ प्र - अ॒नः । अ॒न्येषा᳚म् प्रा॒णाना᳚म् । प्रा॒णाना॒मुपो᳚प्ते । प्रा॒णाना॒मिति॑ प्र - अ॒नाना᳚म् । उपो᳚प्ते॒ऽन्ये । उपो᳚प्त॒ इत्युप॑ - उ॒प्ते॒ । अ॒न्ये ग्रहाः᳚ । ग्रहाः᳚ सा॒द्यन्ते᳚ । सा॒द्यन्तेऽनु॑पोप्ते । अनु॑पोप्ते ध्रु॒वः । अनु॑पोप्त॒ इत्यनु॑प - उ॒प्ते॒ । ध्रु॒वस्तस्मा᳚त् । तस्मा॑द॒स्थ्ना । अ॒स्थ्नाऽन्याः । अ॒न्याः प्र॒जाः । प्र॒जाः प्र॑ति॒तिष्ठ॑न्ति । प्र॒जा इति॑ प्र - जाः । प्र॒ति॒तिष्ठ॑न्ति माꣳ॒॒सेन॑ । प्र॒ति॒तिष्ठ॒न्तीति॑ प्रति - तिष्ठ॑न्ति । माꣳ॒॒सेना॒न्याः । अ॒न्या असु॑राः । असु॑रा॒ वै । वा उ॑त्तर॒तः । उ॒त्त॒र॒तः पृ॑थि॒वीम् । उ॒त्त॒र॒त इत्यु॑त् - त॒र॒तः । पृ॒थि॒वीम् प॒र्याचि॑कीर्.षन्न् । प॒र्याचि॑कीर्.ष॒न् ताम् । प॒र्याचि॑कीर्.ष॒न्निति॑ परि - आचि॑कीर्.षन्न् । ताम् दे॒वाः । दे॒वा ध्रु॒वेण॑ । ध्रु॒वेणा॑दृꣳहन्न् । अ॒दृꣳ॒॒ह॒न् तत् । तद् ध्रु॒वस्य॑ । ध्रु॒वस्य॑ ध्रुव॒त्वम् । धु॒व॒त्वम् ॅयत् । ध्रु॒व॒त्वमिति॑ ध्रुव - त्वम् । यद् ध्रु॒वः । ध्रु॒व उ॑त्तर॒तः । उ॒त्त॒र॒तः सा॒द्यते᳚ । उ॒त्त॒र॒त इत्यु॑त् - त॒र॒तः । सा॒द्यते॒ धृत्यै᳚ । धृत्या॒ आयुः॑ । आयु॒र् वै । वा ए॒तत् । ए॒तद् य॒ज्ञ्स्य॑ । य॒ज्ञ्स्य॒ यत् । यद् ध्रु॒वः । ध्रु॒व आ॒त्मा । आ॒त्मा होता᳚ । होता॒ यत् । यद्‍धो॑तृचम॒से । हो॒तृ॒च॒म॒से ध्रु॒वम् । हो॒तृ॒च॒म॒स इति॑ होतृ - च॒म॒से । ध्रु॒वम॑व॒नय॑ति । अ॒व॒नय॑त्या॒त्मन्न् । अ॒व॒नय॒तीत्य॑व - नय॑ति । आ॒त्मन्ने॒व । ए॒व य॒ज्ञ्स्य॑ ( ) । य॒ज्ञ्स्यायुः॑ \newline

\textbf{Jatai Paata} \newline

1. अ॒र्द्ध्यवा॒ ङवा॑ ङ॒र्द्ध्य॑ र्ध्यवाङ्॑ । \newline
2. अवा᳚ङ् प्रा॒णः प्रा॒णो ऽवा॒ ङवा᳚ङ् प्रा॒णः । \newline
3. प्रा॒णो᳚ ऽन्येषा॑ म॒न्येषा᳚म् प्रा॒णः प्रा॒णो᳚ ऽन्येषा᳚म् । \newline
4. प्रा॒ण इति॑ प्र - अ॒नः । \newline
5. अ॒न्येषा᳚म् प्रा॒णाना᳚म् प्रा॒णाना॑ म॒न्येषा॑ म॒न्येषा᳚म् प्रा॒णाना᳚म् । \newline
6. प्रा॒णाना॒ मुपो᳚प्त॒ उपो᳚प्ते प्रा॒णाना᳚म् प्रा॒णाना॒ मुपो᳚प्ते । \newline
7. प्रा॒णाना॒मिति॑ प्र - अ॒नाना᳚म् । \newline
8. उपो᳚प्ते॒ ऽन्ये᳚ ऽन्य उपो᳚प्त॒ उपो᳚प्ते॒ ऽन्ये । \newline
9. उपो᳚प्त॒ इत्युप॑ - उ॒प्ते॒ । \newline
10. अ॒न्ये ग्रहा॒ ग्रहा॑ अ॒न्ये᳚ ऽन्ये ग्रहाः᳚ । \newline
11. ग्रहाः᳚ सा॒द्यन्ते॑ सा॒द्यन्ते॒ ग्रहा॒ ग्रहाः᳚ सा॒द्यन्ते᳚ । \newline
12. सा॒द्यन्ते ऽनु॑पो॒प्ते ऽनु॑पोप्ते सा॒द्यन्ते॑ सा॒द्यन्ते ऽनु॑पोप्ते । \newline
13. अनु॑पोप्ते ध्रु॒वो ध्रु॒वो ऽनु॑पो॒प्ते ऽनु॑पोप्ते ध्रु॒वः । \newline
14. अनु॑पोप्त॒ इत्यनु॑प - उ॒प्ते॒ । \newline
15. ध्रु॒व स्तस्मा॒त् तस्मा᳚द् ध्रु॒वो ध्रु॒व स्तस्मा᳚त् । \newline
16. तस्मा॑ द॒स्थ्ना ऽस्थ्ना तस्मा॒त् तस्मा॑ द॒स्थ्ना । \newline
17. अ॒स्थ्ना ऽन्या अ॒न्या अ॒स्थ्ना ऽस्थ्ना ऽन्याः । \newline
18. अ॒न्याः प्र॒जाः प्र॒जा अ॒न्या अ॒न्याः प्र॒जाः । \newline
19. प्र॒जाः प्र॑ति॒तिष्ठ॑न्ति प्रति॒तिष्ठ॑न्ति प्र॒जाः प्र॒जाः प्र॑ति॒तिष्ठ॑न्ति । \newline
20. प्र॒जा इति॑ प्र - जाः । \newline
21. प्र॒ति॒तिष्ठ॑न्ति माꣳ॒॒सेन॑ माꣳ॒॒सेन॑ प्रति॒तिष्ठ॑न्ति प्रति॒तिष्ठ॑न्ति माꣳ॒॒सेन॑ । \newline
22. प्र॒ति॒तिष्ठ॒न्तीति॑ प्रति - तिष्ठ॑न्ति । \newline
23. माꣳ॒॒से ना॒न्या अ॒न्या माꣳ॒॒सेन॑ माꣳ॒॒से ना॒न्याः । \newline
24. अ॒न्या असु॑रा॒ असु॑रा अ॒न्या अ॒न्या असु॑राः । \newline
25. असु॑रा॒ वै वा असु॑रा॒ असु॑रा॒ वै । \newline
26. वा उ॑त्तर॒त उ॑त्तर॒तो वै वा उ॑त्तर॒तः । \newline
27. उ॒त्त॒र॒तः पृ॑थि॒वीम् पृ॑थि॒वी मु॑त्तर॒त उ॑त्तर॒तः पृ॑थि॒वीम् । \newline
28. उ॒त्त॒र॒त इत्यु॑त् - त॒र॒तः । \newline
29. पृ॒थि॒वीम् प॒र्याचि॑कीर्.षन् प॒र्याचि॑कीर्.षन् पृथि॒वीम् पृ॑थि॒वीम् प॒र्याचि॑कीर्.षन्न् । \newline
30. प॒र्याचि॑कीर्.ष॒न् ताम् ताम् प॒र्याचि॑कीर्.षन् प॒र्याचि॑कीर्.ष॒न् ताम् । \newline
31. प॒र्याचि॑कीर्.ष॒न्निति॑ परि - आचि॑कीर्.षन्न् । \newline
32. ताम् दे॒वा दे॒वा स्ताम् ताम् दे॒वाः । \newline
33. दे॒वा ध्रु॒वेण॑ ध्रु॒वेण॑ दे॒वा दे॒वा ध्रु॒वेण॑ । \newline
34. ध्रु॒वेणा॑ दृहन् नदृहन् ध्रु॒वेण॑ ध्रु॒वेणा॑ दृहन्न् । \newline
35. अ॒दृ॒ह॒न् तत् तद॑दृहन् नदृह॒न् तत् । \newline
36. तद् ध्रु॒वस्य॑ ध्रु॒वस्य॒ तत् तद् ध्रु॒वस्य॑ । \newline
37. ध्रु॒वस्य॑ ध्रुव॒त्वम् ध्रु॑व॒त्वम् ध्रु॒वस्य॑ ध्रु॒वस्य॑ ध्रुव॒त्वम् । \newline
38. ध्रु॒व॒त्वं ॅयद् यद् ध्रु॑व॒त्वम् ध्रु॑व॒त्वं ॅयत् । \newline
39. ध्रु॒व॒त्वमिति॑ ध्रुव - त्वम् । \newline
40. यद् ध्रु॒वो ध्रु॒वो यद् यद् ध्रु॒वः । \newline
41. ध्रु॒व उ॑त्तर॒त उ॑त्तर॒तो ध्रु॒वो ध्रु॒व उ॑त्तर॒तः । \newline
42. उ॒त्त॒र॒तः सा॒द्यते॑ सा॒द्यत॑ उत्तर॒त उ॑त्तर॒तः सा॒द्यते᳚ । \newline
43. उ॒त्त॒र॒त इत्यु॑त् - त॒र॒तः । \newline
44. सा॒द्यते॒ धृत्यै॒ धृत्यै॑ सा॒द्यते॑ सा॒द्यते॒ धृत्यै᳚ । \newline
45. धृत्या॒ आयु॒ रायु॒र् धृत्यै॒ धृत्या॒ आयुः॑ । \newline
46. आयु॒र् वै वा आयु॒ रायु॒र् वै । \newline
47. वा ए॒त दे॒तद् वै वा ए॒तत् । \newline
48. ए॒तद् य॒ज्ञ्स्य॑ य॒ज्ञ् स्यै॒त दे॒तद् य॒ज्ञ्स्य॑ । \newline
49. य॒ज्ञ्स्य॒ यद् यद् य॒ज्ञ्स्य॑ य॒ज्ञ्स्य॒ यत् । \newline
50. यद् ध्रु॒वो ध्रु॒वो यद् यद् ध्रु॒वः । \newline
51. ध्रु॒व आ॒त्मा ऽऽत्मा ध्रु॒वो ध्रु॒व आ॒त्मा । \newline
52. आ॒त्मा होता॒ होता॒ ऽऽत्मा ऽऽत्मा होता᳚ । \newline
53. होता॒ यद् यद्धोता॒ होता॒ यत् । \newline
54. यद्धो॑तृचम॒से हो॑तृचम॒से यद् यद्धो॑तृचम॒से । \newline
55. हो॒तृ॒च॒म॒से ध्रु॒वम् ध्रु॒वꣳ हो॑तृचम॒से हो॑तृचम॒से ध्रु॒वम् । \newline
56. हो॒तृ॒च॒म॒स इति॑ होतृ - च॒म॒से । \newline
57. ध्रु॒व म॑व॒नय॑ त्यव॒नय॑ति ध्रु॒वम् ध्रु॒व म॑व॒नय॑ति । \newline
58. अ॒व॒नय॑ त्या॒त्मन् ना॒त्मन् न॑व॒नय॑ त्यव॒नय॑ त्या॒त्मन्न् । \newline
59. अ॒व॒नय॒तीत्य॑व - नय॑ति । \newline
60. आ॒त्मन् ने॒वै वात्मन् ना॒त्मन् ने॒व । \newline
61. ए॒व य॒ज्ञ्स्य॑ य॒ज्ञ् स्यै॒वैव य॒ज्ञ्स्य॑ । \newline
62. य॒ज्ञ् स्यायु॒ रायु॑र् य॒ज्ञ्स्य॑ य॒ज्ञ् स्यायुः॑ । \newline

\textbf{Ghana Paata } \newline

1. अ॒र्द्ध्यवा॒ ङवा॑ ङ॒र्ध्य॑ र्द्ध्यवा᳚ङ् प्रा॒णः प्रा॒णो ऽवा॑ ङ॒र्ध्य॑ र्द्ध्यवा᳚ङ् प्रा॒णः । \newline
2. अवा᳚ङ् प्रा॒णः प्रा॒णो ऽवा॒ ङवा᳚ङ् प्रा॒णो᳚ ऽन्येषा॑ म॒न्येषा᳚म् प्रा॒णो ऽवा॒ ङवा᳚ङ् प्रा॒णो᳚ ऽन्येषा᳚म् । \newline
3. प्रा॒णो᳚ ऽन्येषा॑ म॒न्येषा᳚म् प्रा॒णः प्रा॒णो᳚ ऽन्येषा᳚म् प्रा॒णाना᳚म् प्रा॒णाना॑ म॒न्येषा᳚म् प्रा॒णः प्रा॒णो᳚ ऽन्येषा᳚म् प्रा॒णाना᳚म् । \newline
4. प्रा॒ण इति॑ प्र - अ॒नः । \newline
5. अ॒न्येषा᳚म् प्रा॒णाना᳚म् प्रा॒णाना॑ म॒न्येषा॑ म॒न्येषा᳚म् प्रा॒णाना॒ मुपो᳚प्त॒ उपो᳚प्ते प्रा॒णाना॑ म॒न्येषा॑ म॒न्येषा᳚म् प्रा॒णाना॒ मुपो᳚प्ते । \newline
6. प्रा॒णाना॒ मुपो᳚प्त॒ उपो᳚प्ते प्रा॒णाना᳚म् प्रा॒णाना॒ मुपो᳚प्ते॒ ऽन्ये᳚ ऽन्य उपो᳚प्ते प्रा॒णाना᳚म् प्रा॒णाना॒ मुपो᳚प्ते॒ ऽन्ये । \newline
7. प्रा॒णाना॒मिति॑ प्र - अ॒नाना᳚म् । \newline
8. उपो᳚प्ते॒ ऽन्ये᳚ ऽन्य उपो᳚प्त॒ उपो᳚प्ते॒ ऽन्ये ग्रहा॒ ग्रहा॑ अ॒न्य उपो᳚प्त॒ उपो᳚प्ते॒ ऽन्ये ग्रहाः᳚ । \newline
9. उपो᳚प्त॒ इत्युप॑ - उ॒प्ते॒ । \newline
10. अ॒न्ये ग्रहा॒ ग्रहा॑ अ॒न्ये᳚ ऽन्ये ग्रहाः᳚ सा॒द्यन्ते॑ सा॒द्यन्ते॒ ग्रहा॑ अ॒न्ये᳚ ऽन्ये ग्रहाः᳚ सा॒द्यन्ते᳚ । \newline
11. ग्रहाः᳚ सा॒द्यन्ते॑ सा॒द्यन्ते॒ ग्रहा॒ ग्रहाः᳚ सा॒द्यन्ते ऽनु॑पो॒प्ते ऽनु॑पोप्ते सा॒द्यन्ते॒ ग्रहा॒ ग्रहाः᳚ सा॒द्यन्ते ऽनु॑पोप्ते । \newline
12. सा॒द्यन्ते ऽनु॑पो॒प्ते ऽनु॑पोप्ते सा॒द्यन्ते॑ सा॒द्यन्ते ऽनु॑पोप्ते ध्रु॒वो ध्रु॒वो ऽनु॑पोप्ते सा॒द्यन्ते॑ सा॒द्यन्ते ऽनु॑पोप्ते ध्रु॒वः । \newline
13. अनु॑पोप्ते ध्रु॒वो ध्रु॒वो ऽनु॑पो॒प्ते ऽनु॑पोप्ते ध्रु॒व स्तस्मा॒त् तस्मा᳚द् ध्रु॒वो ऽनु॑पो॒प्ते ऽनु॑पोप्ते ध्रु॒व स्तस्मा᳚त् । \newline
14. अनु॑पोप्त॒ इत्यनु॑प - उ॒प्ते॒ । \newline
15. ध्रु॒व स्तस्मा॒त् तस्मा᳚द् ध्रु॒वो ध्रु॒व स्तस्मा॑ द॒स्थ्ना ऽस्थ्ना तस्मा᳚द् ध्रु॒वो ध्रु॒व स्तस्मा॑ द॒स्थ्ना । \newline
16. तस्मा॑ द॒स्थ्ना ऽस्थ्ना तस्मा॒त् तस्मा॑ द॒स्थ्ना ऽन्या अ॒न्या अ॒स्थ्ना तस्मा॒त् तस्मा॑ द॒स्थ्ना ऽन्याः । \newline
17. अ॒स्थ्ना ऽन्या अ॒न्या अ॒स्थ्ना ऽस्थ्ना ऽन्याः प्र॒जाः प्र॒जा अ॒न्या अ॒स्थ्ना ऽस्थ्ना ऽन्याः प्र॒जाः । \newline
18. अ॒न्याः प्र॒जाः प्र॒जा अ॒न्या अ॒न्याः प्र॒जाः प्र॑ति॒तिष्ठ॑न्ति प्रति॒तिष्ठ॑न्ति प्र॒जा अ॒न्या अ॒न्याः प्र॒जाः प्र॑ति॒तिष्ठ॑न्ति । \newline
19. प्र॒जाः प्र॑ति॒तिष्ठ॑न्ति प्रति॒तिष्ठ॑न्ति प्र॒जाः प्र॒जाः प्र॑ति॒तिष्ठ॑न्ति माꣳ॒॒सेन॑ माꣳ॒॒सेन॑ प्रति॒तिष्ठ॑न्ति प्र॒जाः प्र॒जाः प्र॑ति॒तिष्ठ॑न्ति माꣳ॒॒सेन॑ । \newline
20. प्र॒जा इति॑ प्र - जाः । \newline
21. प्र॒ति॒तिष्ठ॑न्ति माꣳ॒॒सेन॑ माꣳ॒॒सेन॑ प्रति॒तिष्ठ॑न्ति प्रति॒तिष्ठ॑न्ति माꣳ॒॒सेना॒न्या अ॒न्या माꣳ॒॒सेन॑ प्रति॒तिष्ठ॑न्ति प्रति॒तिष्ठ॑न्ति माꣳ॒॒सेना॒न्याः । \newline
22. प्र॒ति॒तिष्ठ॒न्तीति॑ प्रति - तिष्ठ॑न्ति । \newline
23. माꣳ॒॒से ना॒न्या अ॒न्या माꣳ॒॒सेन॑ माꣳ॒॒सेना॒न्या असु॑रा॒ असु॑रा अ॒न्या माꣳ॒॒सेन॑ माꣳ॒॒सेना॒न्या असु॑राः । \newline
24. अ॒न्या असु॑रा॒ असु॑रा अ॒न्या अ॒न्या असु॑रा॒ वै वा असु॑रा अ॒न्या अ॒न्या असु॑रा॒ वै । \newline
25. असु॑रा॒ वै वा असु॑रा॒ असु॑रा॒ वा उ॑त्तर॒त उ॑त्तर॒तो वा असु॑रा॒ असु॑रा॒ वा उ॑त्तर॒तः । \newline
26. वा उ॑त्तर॒त उ॑त्तर॒तो वै वा उ॑त्तर॒तः पृ॑थि॒वीम् पृ॑थि॒वी मु॑त्तर॒तो वै वा उ॑त्तर॒तः पृ॑थि॒वीम् । \newline
27. उ॒त्त॒र॒तः पृ॑थि॒वीम् पृ॑थि॒वी मु॑त्तर॒त उ॑त्तर॒तः पृ॑थि॒वीम् प॒र्याचि॑कीर्.षन् प॒र्याचि॑कीर्.षन् पृथि॒वी मु॑त्तर॒त उ॑त्तर॒तः पृ॑थि॒वीम् प॒र्याचि॑कीर्.षन्न् । \newline
28. उ॒त्त॒र॒त इत्यु॑त् - त॒र॒तः । \newline
29. पृ॒थि॒वीम् प॒र्याचि॑कीर्.षन् प॒र्याचि॑कीर्.षन् पृथि॒वीम् पृ॑थि॒वीम् प॒र्याचि॑कीर्.ष॒न् ताम् ताम् प॒र्याचि॑कीर्.षन् पृथि॒वीम् पृ॑थि॒वीम् प॒र्याचि॑कीर्.ष॒न् ताम् । \newline
30. प॒र्याचि॑कीर्.ष॒न् ताम् ताम् प॒र्याचि॑कीर्.षन् प॒र्याचि॑कीर्.ष॒न् ताम् दे॒वा दे॒वा स्ताम् प॒र्याचि॑कीर्.षन् प॒र्याचि॑कीर्.ष॒न् ताम् दे॒वाः । \newline
31. प॒र्याचि॑कीर्.ष॒न्निति॑ परि - आचि॑कीर्.षन्न् । \newline
32. ताम् दे॒वा दे॒वा स्ताम् ताम् दे॒वा ध्रु॒वेण॑ ध्रु॒वेण॑ दे॒वा स्ताम् ताम् दे॒वा ध्रु॒वेण॑ । \newline
33. दे॒वा ध्रु॒वेण॑ ध्रु॒वेण॑ दे॒वा दे॒वा ध्रु॒व्णा॑ दृहन्-नदृहन् ध्रु॒वेण॑ दे॒वा दे॒वा ध्रु॒वेणा॑ दृहन्न् । \newline
34. ध्रु॒वेणा॑ दृहन्-नदृहन् ध्रु॒वेण॑ ध्रु॒वेणा॑ दृह॒न् तत् तद॑दृहन् ध्रु॒वेण॑ ध्रु॒वेणा॑ दृह॒न् तत् । \newline
35. अ॒दृ॒ह॒न् तत् तद॑दृहन्-नदृह॒न् तद् ध्रु॒वस्य॑ ध्रु॒वस्य॒ तद॑दृहन्-नदृह॒न् तद् ध्रु॒वस्य॑ । \newline
36. तद् ध्रु॒वस्य॑ ध्रु॒वस्य॒ तत् तद् ध्रु॒वस्य॑ ध्रुव॒त्वम् ध्रु॑व॒त्वम् ध्रु॒वस्य॒ तत् तद् ध्रु॒वस्य॑ ध्रुव॒त्वम् । \newline
37. ध्रु॒वस्य॑ ध्रुव॒त्वम् ध्रु॑व॒त्वम् ध्रु॒वस्य॑ ध्रु॒वस्य॑ ध्रुव॒त्वं ॅयद् यद् ध्रु॑व॒त्वम् ध्रु॒वस्य॑ ध्रु॒वस्य॑ ध्रुव॒त्वं ॅयत् । \newline
38. ध्रु॒व॒त्वं ॅयद् यद् ध्रु॑व॒त्वम् ध्रु॑व॒त्वं ॅयद् ध्रु॒वो ध्रु॒वो यद् ध्रु॑व॒त्वम् ध्रु॑व॒त्वं ॅयद् ध्रु॒वः । \newline
39. ध्रु॒व॒त्वमिति॑ ध्रुव - त्वम् । \newline
40. यद् ध्रु॒वो ध्रु॒वो यद् यद् ध्रु॒व उ॑त्तर॒त उ॑त्तर॒तो ध्रु॒वो यद् यद् ध्रु॒व उ॑त्तर॒तः । \newline
41. ध्रु॒व उ॑त्तर॒त उ॑त्तर॒तो ध्रु॒वो ध्रु॒व उ॑त्तर॒तः सा॒द्यते॑ सा॒द्यत॑ उत्तर॒तो ध्रु॒वो ध्रु॒व उ॑त्तर॒तः सा॒द्यते᳚ । \newline
42. उ॒त्त॒र॒तः सा॒द्यते॑ सा॒द्यत॑ उत्तर॒त उ॑त्तर॒तः सा॒द्यते॒ धृत्यै॒ धृत्यै॑ सा॒द्यत॑ उत्तर॒त उ॑त्तर॒तः सा॒द्यते॒ धृत्यै᳚ । \newline
43. उ॒त्त॒र॒त इत्यु॑त् - त॒र॒तः । \newline
44. सा॒द्यते॒ धृत्यै॒ धृत्यै॑ सा॒द्यते॑ सा॒द्यते॒ धृत्या॒ आयु॒ रायु॒र् धृत्यै॑ सा॒द्यते॑ सा॒द्यते॒ धृत्या॒ आयुः॑ । \newline
45. धृत्या॒ आयु॒ रायु॒र् धृत्यै॒ धृत्या॒ आयु॒र् वै वा आयु॒र् धृत्यै॒ धृत्या॒ आयु॒र् वै । \newline
46. आयु॒र् वै वा आयु॒ रायु॒र् वा ए॒त दे॒तद् वा आयु॒ रायु॒र् वा ए॒तत् । \newline
47. वा ए॒त दे॒तद् वै वा ए॒तद् य॒ज्ञ्स्य॑ य॒ज्ञ् स्यै॒तद् वै वा ए॒तद् य॒ज्ञ्स्य॑ । \newline
48. ए॒तद् य॒ज्ञ्स्य॑ य॒ज्ञ् स्यै॒त दे॒तद् य॒ज्ञ्स्य॒ यद् यद् य॒ज्ञ् स्यै॒त दे॒तद् य॒ज्ञ्स्य॒ यत् । \newline
49. य॒ज्ञ्स्य॒ यद् यद् य॒ज्ञ्स्य॑ य॒ज्ञ्स्य॒ यद् ध्रु॒वो ध्रु॒वो यद् य॒ज्ञ्स्य॑ य॒ज्ञ्स्य॒ यद् ध्रु॒वः । \newline
50. यद् ध्रु॒वो ध्रु॒वो यद् यद् ध्रु॒व आ॒त्मा ऽऽत्मा ध्रु॒वो यद् यद् ध्रु॒व आ॒त्मा । \newline
51. ध्रु॒व आ॒त्मा ऽऽत्मा ध्रु॒वो ध्रु॒व आ॒त्मा होता॒ होता॒ ऽऽत्मा ध्रु॒वो ध्रु॒व आ॒त्मा होता᳚ । \newline
52. आ॒त्मा होता॒ होता॒ ऽऽत्मा ऽऽत्मा होता॒ यद् यद्धोता॒ ऽऽत्मा ऽऽत्मा होता॒ यत् । \newline
53. होता॒ यद् यद्धोता॒ होता॒ यद्धो॑तृचम॒से हो॑तृचम॒से यद्धोता॒ होता॒ यद्धो॑तृचम॒से । \newline
54. यद्धो॑तृचम॒से हो॑तृचम॒से यद् यद्धो॑तृचम॒से ध्रु॒वम् ध्रु॒वꣳ हो॑तृचम॒से यद् यद्धो॑तृचम॒से ध्रु॒वम् । \newline
55. हो॒तृ॒च॒म॒से ध्रु॒वम् ध्रु॒वꣳ हो॑तृचम॒से हो॑तृचम॒से ध्रु॒व म॑व॒नय॑ त्यव॒नय॑ति ध्रु॒वꣳ हो॑तृचम॒से हो॑तृचम॒से ध्रु॒व म॑व॒नय॑ति । \newline
56. हो॒तृ॒च॒म॒स इति॑ होतृ - च॒म॒से । \newline
57. ध्रु॒व म॑व॒नय॑ त्यव॒नय॑ति ध्रु॒वम् ध्रु॒व म॑व॒नय॑ त्या॒त्मन्-ना॒त्मन्-न॑व॒नय॑ति ध्रु॒वम् ध्रु॒व म॑व॒नय॑ त्या॒त्मन्न् । \newline
58. अ॒व॒नय॑ त्या॒त्मन्-ना॒त्मन्-न॑व॒नय॑ त्यव॒नय॑ त्या॒त्मन्-ने॒वैवात्मन्-न॑व॒नय॑ त्यव॒नय॑ त्या॒त्मन्-ने॒व । \newline
59. अ॒व॒नय॒तीत्य॑व - नय॑ति । \newline
60. आ॒त्मन्-ने॒वै वात्मन्-ना॒त्मन्-ने॒व य॒ज्ञ्स्य॑ य॒ज्ञ् स्यै॒वात्मन्-ना॒त्मन्-ने॒व य॒ज्ञ्स्य॑ । \newline
61. ए॒व य॒ज्ञ्स्य॑ य॒ज्ञ् स्यै॒वैव य॒ज्ञ्स्यायु॒ रायु॑र् य॒ज्ञ्स्यै॒वैव य॒ज्ञ्स्यायुः॑ । \newline
62. य॒ज्ञ्स्यायु॒ रायु॑र् य॒ज्ञ्स्य॑ य॒ज्ञ्स्यायु॑र् दधाति दधा॒ त्यायु॑र् य॒ज्ञ्स्य॑ य॒ज्ञ्स्यायु॑र् दधाति । \newline
\pagebreak
\markright{ TS 6.5.2.3  \hfill https://www.vedavms.in \hfill}

\section{ TS 6.5.2.3 }

\textbf{TS 6.5.2.3 } \newline
\textbf{Samhita Paata} \newline

-यु॑र्दधाति पु॒रस्ता॑दु॒क्थस्या॑व॒नीय॒ इत्या॑हुः पु॒रस्ता॒द्ध्यायु॑षो भु॒ङ्क्ते म॑द्ध्य॒तो॑ऽव॒नीय॒ इत्या॑हुर्मद्ध्य॒मेन॒ ह्यायु॑षो भु॒ङ्क्त उ॑त्तरा॒र्द्धे॑ऽव॒नीय॒ इत्या॑हुरुत्त॒मेन॒ ह्यायु॑षो भु॒ङ्क्ते वै᳚श्वदे॒व्यामृ॒चि श॒स्यमा॑नाया॒मव॑ नयति वैश्वदे॒व्यो॑ वै प्र॒जाः प्र॒जास्वे॒वाऽऽ*यु॑र्दधाति ॥ \newline

\textbf{Pada Paata} \newline

आयुः॑ । द॒धा॒ति॒ । पु॒रस्ता᳚त् । उ॒क्थस्य॑ । अ॒व॒नीय॒ इत्य॑व - नीयः॑ । इति॑ । आ॒हुः॒ । पु॒रस्ता᳚त् । हि । आयु॑षः । भु॒ङ्क्ते । म॒द्ध्य॒तः । अ॒व॒नीय॒ इत्य॑व - नीयः॑ । इति॑ । आ॒हुः॒ । म॒द्ध्य॒मेन॑ । हि । आयु॑षः । भु॒ङ्क्ते । उ॒त्त॒रा॒द्‌र्ध इत्यु॑त्तर - अ॒द्‌र्धे । अ॒व॒नीय॒ इत्य॑व - नीयः॑ । इति॑ । आ॒हुः॒ । उ॒त्त॒मेनेत्यु॑त् - त॒मेन॑ । हि । आयु॑षः । भु॒ङ्क्ते । वै॒श्व॒दे॒व्यामिति॑ वैश्व - दे॒व्याम् । ऋ॒चि । श॒स्यमा॑नायाम् । अवेति॑ । न॒य॒ति॒ । वै॒श्व॒दे॒व्य॑ इति॑ वैश्व - दे॒व्यः॑ । वै । प्र॒जा इति॑ प्र - जाः । प्र॒जास्विति॑ प्र - जासु॑ । ए॒व । आयुः॑ । द॒धा॒ति॒ ॥  \newline


\textbf{Krama Paata} \newline

आयु॑र् दधाति । द॒धा॒ति॒ पु॒रस्ता᳚त् । पु॒रस्ता॑दु॒क्थस्य॑ । उ॒क्थस्या॑व॒नीयः॑ । अ॒व॒नीय॒ इति॑ । अ॒व॒नीय॒ इत्य॑व - नीयः॑ । इत्या॑हुः । आ॒हुः॒ पु॒रस्ता᳚त् । पु॒रस्ता॒द्‌धि । ह्यायु॑षः । आयु॑षो भु॒ङ्‍क्ते । भु॒ङ्‍क्ते म॑द्ध्य॒तः । म॒द्ध्य॒तो॑ऽव॒नीयः॑ । अ॒व॒नीय॒ इति॑ । अ॒व॒नीय॒ इत्य॑व - नीयः॑ । इत्या॑हुः । आ॒हु॒र् म॒द्ध्य॒मेन॑ । म॒द्ध्य॒मेन॒ हि । ह्यायु॑षः । आयु॑षो भु॒ङ्‍क्ते । भु॒ङ्‍क्त उ॑त्तरा॒र्द्धे । उ॒त्त॒रा॒र्द्धे॑ऽव॒नीयः॑ । उ॒त्त॒रा॒र्द्ध इत्यु॑त्तर - अ॒र्द्धे । अ॒व॒नीय॒ इति॑ । अ॒व॒नीय॒ इत्य॑व - नीयः॑ । इत्या॑हुः । आ॒हु॒रु॒त्त॒मेन॑ । उ॒त्त॒मेन॒ हि । उ॒त्त॒मेनेत्यु॑त् - त॒मेन॑ । ह्यायु॑षः । आयु॑षो भु॒ङ्‍क्ते । भु॒ङ्‍क्ते वै᳚श्वदे॒व्याम् । वै॒श्व॒दे॒व्यामृ॒चि । वै॒श्व॒दे॒व्यामिति॑ वैश्व - दे॒व्याम् । ऋ॒चि श॒स्यमा॑नायाम् । श॒स्यमा॑नाया॒मव॑ । अव॑ नयति । न॒य॒ति॒ वै॒श्व॒दे॒व्यः॑ । वै॒श्व॒दे॒व्यो॑ वै । वै॒श्व॒दे॒व्य॑ इति॑ वैश्व - दे॒व्यः॑ । वै प्र॒जाः । प्र॒जाः प्र॒जासु॑ । प्र॒जा इति॑ प्र - जाः । प्र॒जास्वे॒व । प्र॒जास्विति॑ प्र - जासु॑ । ए॒वायुः॑ । आयु॑र् दधाति । द॒धा॒तीति॑ दधाति । \newline

\textbf{Jatai Paata} \newline

1. आयु॑र् दधाति दधा॒ त्यायु॒ रायु॑र् दधाति । \newline
2. द॒धा॒ति॒ पु॒रस्ता᳚त् पु॒रस्ता᳚द् दधाति दधाति पु॒रस्ता᳚त् । \newline
3. पु॒रस्ता॑ दु॒क्थ स्यो॒क्थस्य॑ पु॒रस्ता᳚त् पु॒रस्ता॑ दु॒क्थस्य॑ । \newline
4. उ॒क्थस्या॑ व॒नीयो॑ ऽव॒नीय॑ उ॒क्थ स्यो॒क्थस्या॑ व॒नीयः॑ । \newline
5. अ॒व॒नीय॒ इती त्य॑व॒नीयो॑ ऽव॒नीय॒ इति॑ । \newline
6. अ॒व॒नीय॒ इत्य॑व - नीयः॑ । \newline
7. इत्या॑हु राहु॒ रिती त्या॑हुः । \newline
8. आ॒हुः॒ पु॒रस्ता᳚त् पु॒रस्ता॑ दाहु राहुः पु॒रस्ता᳚त् । \newline
9. पु॒रस्ता॒ द्धि हि पु॒रस्ता᳚त् पु॒रस्ता॒ द्धि । \newline
10. ह्यायु॑ष॒ आयु॑षो॒ हि ह्यायु॑षः । \newline
11. आयु॑षो भु॒ङ्क्ते भु॒ङ्क्त आयु॑ष॒ आयु॑षो भु॒ङ्क्ते । \newline
12. भु॒ङ्क्ते म॑द्ध्य॒तो म॑द्ध्य॒तो भु॒ङ्क्ते भु॒ङ्क्ते म॑द्ध्य॒तः । \newline
13. म॒द्ध्य॒तो॑ ऽव॒नीयो॑ ऽव॒नीयो॑ मद्ध्य॒तो म॑द्ध्य॒तो॑ ऽव॒नीयः॑ । \newline
14. अ॒व॒नीय॒ इती त्य॑व॒नीयो॑ ऽव॒नीय॒ इति॑ । \newline
15. अ॒व॒नीय॒ इत्य॑व - नीयः॑ । \newline
16. इत्या॑हु राहु॒ रिती त्या॑हुः । \newline
17. आ॒हु॒र् म॒द्ध्य॒मेन॑ मद्ध्य॒मे ना॑हु राहुर् मद्ध्य॒मेन॑ । \newline
18. म॒द्ध्य॒मेन॒ हि हि म॑द्ध्य॒मेन॑ मद्ध्य॒मेन॒ हि । \newline
19. ह्यायु॑ष॒ आयु॑षो॒ हि ह्यायु॑षः । \newline
20. आयु॑षो भु॒ङ्क्ते भु॒ङ्क्त आयु॑ष॒ आयु॑षो भु॒ङ्क्ते । \newline
21. भु॒ङ्क्त उ॑त्तरा॒र्द्ध उ॑त्तरा॒र्द्धे भु॒ङ्क्ते भु॒ङ्क्त उ॑त्तरा॒र्द्धे । \newline
22. उ॒त्त॒रा॒र्द्धे॑ ऽव॒नीयो॑ ऽव॒नीय॑ उत्तरा॒र्द्ध उ॑त्तरा॒र्द्धे॑ ऽव॒नीयः॑ । \newline
23. उ॒त्त॒रा॒र्द्ध इत्यु॑त्तर - अ॒र्द्धे । \newline
24. अ॒व॒नीय॒ इती त्य॑व॒नीयो॑ ऽव॒नीय॒ इति॑ । \newline
25. अ॒व॒नीय॒ इत्य॑व - नीयः॑ । \newline
26. इत्या॑हु राहु॒ रिती त्या॑हुः । \newline
27. आ॒हु॒ रु॒त्त॒मे नो᳚त्त॒मेना॑हु राहु रुत्त॒मेन॑ । \newline
28. उ॒त्त॒मेन॒ हि ह्यु॑त्त॒मे नो᳚त्त॒मेन॒ हि । \newline
29. उ॒त्त॒मेनेत्यु॑त् - त॒मेन॑ । \newline
30. ह्यायु॑ष॒ आयु॑षो॒ हि ह्यायु॑षः । \newline
31. आयु॑षो भु॒ङ्क्ते भु॒ङ्क्त आयु॑ष॒ आयु॑षो भु॒ङ्क्ते । \newline
32. भु॒ङ्क्ते वै᳚श्वदे॒व्यां ॅवै᳚श्वदे॒व्याम् भु॒ङ्क्ते भु॒ङ्क्ते वै᳚श्वदे॒व्याम् । \newline
33. वै॒श्व॒दे॒व्या मृ॒च्यृ॑चि वै᳚श्वदे॒व्यां ॅवै᳚श्वदे॒व्या मृ॒चि । \newline
34. वै॒श्व॒दे॒व्यामिति॑ वैश्व - दे॒व्याम् । \newline
35. ऋ॒चि श॒स्यमा॑नायाꣳ श॒स्यमा॑नाया मृ॒च्यृ॑चि श॒स्यमा॑नायाम् । \newline
36. श॒स्यमा॑नाया॒ मवाव॑ श॒स्यमा॑नायाꣳ श॒स्यमा॑नाया॒ मव॑ । \newline
37. अव॑ नयति नय॒ त्यवाव॑ नयति । \newline
38. न॒य॒ति॒ वै॒श्व॒दे॒व्यो॑ वैश्वदे॒व्यो॑ नयति नयति वैश्वदे॒व्यः॑ । \newline
39. वै॒श्व॒दे॒व्यो॑ वै वै वै᳚श्वदे॒व्यो॑ वैश्वदे॒व्यो॑ वै । \newline
40. वै॒श्व॒दे॒व्य॑ इति॑ वैश्व - दे॒व्यः॑ । \newline
41. वै प्र॒जाः प्र॒जा वै वै प्र॒जाः । \newline
42. प्र॒जाः प्र॒जासु॑ प्र॒जासु॑ प्र॒जाः प्र॒जाः प्र॒जासु॑ । \newline
43. प्र॒जा इति॑ प्र - जाः । \newline
44. प्र॒जा स्वे॒वैव प्र॒जासु॑ प्र॒जा स्वे॒व । \newline
45. प्र॒जास्विति॑ प्र - जासु॑ । \newline
46. ए॒वायु॒ रायु॑ रे॒वै वायुः॑ । \newline
47. आयु॑र् दधाति दधा॒ त्यायु॒ रायु॑र् दधाति । \newline
48. द॒धा॒तीति॑ दधाति । \newline

\textbf{Ghana Paata } \newline

1. आयु॑र् दधाति दधा॒ त्यायु॒ रायु॑र् दधाति पु॒रस्ता᳚त् पु॒रस्ता᳚द् दधा॒ त्यायु॒ रायु॑र् दधाति पु॒रस्ता᳚त् । \newline
2. द॒धा॒ति॒ पु॒रस्ता᳚त् पु॒रस्ता᳚द् दधाति दधाति पु॒रस्ता॑ दु॒क्थ स्यो॒क्थस्य॑ पु॒रस्ता᳚द् दधाति दधाति पु॒रस्ता॑ दु॒क्थस्य॑ । \newline
3. पु॒रस्ता॑ दु॒क्थ स्यो॒क्थस्य॑ पु॒रस्ता᳚त् पु॒रस्ता॑ दु॒क्थ स्या॑व॒नीयो॑ ऽव॒नीय॑ उ॒क्थस्य॑ पु॒रस्ता᳚त् पु॒रस्ता॑ दु॒क्थ स्या॑व॒नीयः॑ । \newline
4. उ॒क्थ स्या॑व॒नीयो॑ ऽव॒नीय॑ उ॒क्थ स्यो॒क्थ स्या॑व॒नीय॒ इती त्य॑व॒नीय॑ उ॒क्थ स्यो॒क्थ स्या॑व॒नीय॒ इति॑ । \newline
5. अ॒व॒नीय॒ इती त्य॑व॒नीयो॑ ऽव॒नीय॒ इत्या॑हु राहु॒ रित्य॑व॒नीयो॑ ऽव॒नीय॒ इत्या॑हुः । \newline
6. अ॒व॒नीय॒ इत्य॑व - नीयः॑ । \newline
7. इत्या॑हु राहु॒ रिती त्या॑हुः पु॒रस्ता᳚त् पु॒रस्ता॑ दाहु॒ रिती त्या॑हुः पु॒रस्ता᳚त् । \newline
8. आ॒हुः॒ पु॒रस्ता᳚त् पु॒रस्ता॑ दाहु राहुः पु॒रस्ता॒द्धि हि पु॒रस्ता॑ दाहु राहुः पु॒रस्ता॒द्धि । \newline
9. पु॒रस्ता॒द्धि हि पु॒रस्ता᳚त् पु॒रस्ता॒ द्ध्यायु॑ष॒ आयु॑षो॒ हि पु॒रस्ता᳚त् पु॒रस्ता॒ द्ध्यायु॑षः । \newline
10. ह्यायु॑ष॒ आयु॑षो॒ हि ह्यायु॑षो भु॒ङ्क्ते भु॒ङ्क्त आयु॑षो॒ हि ह्यायु॑षो भु॒ङ्क्ते । \newline
11. आयु॑षो भु॒ङ्क्ते भु॒ङ्क्त आयु॑ष॒ आयु॑षो भु॒ङ्क्ते म॑द्ध्य॒तो म॑द्ध्य॒तो भु॒ङ्क्त आयु॑ष॒ आयु॑षो भु॒ङ्क्ते म॑द्ध्य॒तः । \newline
12. भु॒ङ्क्ते म॑द्ध्य॒तो म॑द्ध्य॒तो भु॒ङ्क्ते भु॒ङ्क्ते म॑द्ध्य॒तो॑ ऽव॒नीयो॑ ऽव॒नीयो॑ मद्ध्य॒तो भु॒ङ्क्ते भु॒ङ्क्ते म॑द्ध्य॒तो॑ ऽव॒नीयः॑ । \newline
13. म॒द्ध्य॒तो॑ ऽव॒नीयो॑ ऽव॒नीयो॑ मद्ध्य॒तो म॑द्ध्य॒तो॑ ऽव॒नीय॒ इती त्य॑व॒नीयो॑ मद्ध्य॒तो म॑द्ध्य॒तो॑ ऽव॒नीय॒ इति॑ । \newline
14. अ॒व॒नीय॒ इती त्य॑व॒नीयो॑ ऽव॒नीय॒ इत्या॑हु राहु॒ रित्य॑व॒नीयो॑ ऽव॒नीय॒ इत्या॑हुः । \newline
15. अ॒व॒नीय॒ इत्य॑व - नीयः॑ । \newline
16. इत्या॑हु राहु॒ रिती त्या॑हुर् मद्ध्य॒मेन॑ मद्ध्य॒मेना॑हु॒ रिती त्या॑हुर् मद्ध्य॒मेन॑ । \newline
17. आ॒हु॒र् म॒द्ध्य॒मेन॑ मद्ध्य॒मेना॑हु राहुर् मद्ध्य॒मेन॒ हि हि म॑द्ध्य॒मेना॑हु राहुर् मद्ध्य॒मेन॒ हि । \newline
18. म॒द्ध्य॒मेन॒ हि हि म॑द्ध्य॒मेन॑ मद्ध्य॒मेन॒ ह्यायु॑ष॒ आयु॑षो॒ हि म॑द्ध्य॒मेन॑ मद्ध्य॒मेन॒ ह्यायु॑षः । \newline
19. ह्यायु॑ष॒ आयु॑षो॒ हि ह्यायु॑षो भु॒ङ्क्ते भु॒ङ्क्त आयु॑षो॒ हि ह्यायु॑षो भु॒ङ्क्ते । \newline
20. आयु॑षो भु॒ङ्क्ते भु॒ङ्क्त आयु॑ष॒ आयु॑षो भु॒ङ्क्त उ॑त्तरा॒र्द्ध उ॑त्तरा॒र्द्धे भु॒ङ्क्त आयु॑ष॒ आयु॑षो भु॒ङ्क्त उ॑त्तरा॒र्द्धे । \newline
21. भु॒ङ्क्त उ॑त्तरा॒र्द्ध उ॑त्तरा॒र्द्धे भु॒ङ्क्ते भु॒ङ्क्त उ॑त्तरा॒र्द्धे॑ ऽव॒नीयो॑ ऽव॒नीय॑ उत्तरा॒र्द्धे भु॒ङ्क्ते भु॒ङ्क्त उ॑त्तरा॒र्द्धे॑ ऽव॒नीयः॑ । \newline
22. उ॒त्त॒रा॒र्द्धे॑ ऽव॒नीयो॑ ऽव॒नीय॑ उत्तरा॒र्द्ध उ॑त्तरा॒र्द्धे॑ ऽव॒नीय॒ इती त्य॑व॒नीय॑ उत्तरा॒र्द्ध उ॑त्तरा॒र्द्धे॑ ऽव॒नीय॒ इति॑ । \newline
23. उ॒त्त॒रा॒र्द्ध इत्यु॑त्तर - अ॒र्द्धे । \newline
24. अ॒व॒नीय॒ इती त्य॑व॒नीयो॑ ऽव॒नीय॒ इत्या॑हु राहु॒ रित्य॑व॒नीयो॑ ऽव॒नीय॒ इत्या॑हुः । \newline
25. अ॒व॒नीय॒ इत्य॑व - नीयः॑ । \newline
26. इत्या॑हु राहु॒ रिती त्या॑हु रुत्त॒मेनो᳚ त्त॒मेना॑हु॒ रिती त्या॑हु रुत्त॒मेन॑ । \newline
27. आ॒हु॒ रु॒त्त॒मेनो᳚ त्त॒मेना॑हु राहु रुत्त॒मेन॒ हि ह्यु॑त्त॒मेना॑हु राहु रुत्त॒मेन॒ हि । \newline
28. उ॒त्त॒मेन॒ हि ह्यु॑त्त॒मेनो᳚ त्त॒मेन॒ ह्यायु॑ष॒ आयु॑षो॒ ह्यु॑त्त॒मेनो᳚ त्त॒मेन॒ ह्यायु॑षः । \newline
29. उ॒त्त॒मेनेत्यु॑त् - त॒मेन॑ । \newline
30. ह्यायु॑ष॒ आयु॑षो॒ हि ह्यायु॑षो भु॒ङ्क्ते भु॒ङ्क्त आयु॑षो॒ हि ह्यायु॑षो भु॒ङ्क्ते । \newline
31. आयु॑षो भु॒ङ्क्ते भु॒ङ्क्त आयु॑ष॒ आयु॑षो भु॒ङ्क्ते वै᳚श्वदे॒व्यां ॅवै᳚श्वदे॒व्याम् भु॒ङ्क्त आयु॑ष॒ आयु॑षो भु॒ङ्क्ते वै᳚श्वदे॒व्याम् । \newline
32. भु॒ङ्क्ते वै᳚श्वदे॒व्यां ॅवै᳚श्वदे॒व्याम् भु॒ङ्क्ते भु॒ङ्क्ते वै᳚श्वदे॒व्या मृ॒च्यृ॑चि वै᳚श्वदे॒व्याम् भु॒ङ्क्ते भु॒ङ्क्ते वै᳚श्वदे॒व्या मृ॒चि । \newline
33. वै॒श्व॒दे॒व्या मृ॒च्यृ॑चि वै᳚श्वदे॒व्यां ॅवै᳚श्वदे॒व्या मृ॒चि श॒स्यमा॑नायाꣳ श॒स्यमा॑नाया मृ॒चि वै᳚श्वदे॒व्यां ॅवै᳚श्वदे॒व्या मृ॒चि श॒स्यमा॑नायाम् । \newline
34. वै॒श्व॒दे॒व्यामिति॑ वैश्व - दे॒व्याम् । \newline
35. ऋ॒चि श॒स्यमा॑नायाꣳ श॒स्यमा॑नाया मृ॒च्यृ॑चि श॒स्यमा॑नाया॒ मवाव॑ श॒स्यमा॑नाया मृ॒च्यृ॑चि श॒स्यमा॑नाया॒ मव॑ । \newline
36. श॒स्यमा॑नाया॒ मवाव॑ श॒स्यमा॑नायाꣳ श॒स्यमा॑नाया॒ मव॑ नयति नय॒ त्यव॑ श॒स्यमा॑नायाꣳ श॒स्यमा॑नाया॒ मव॑ नयति । \newline
37. अव॑ नयति नय॒ त्यवाव॑ नयति वैश्वदे॒व्यो॑ वैश्वदे॒व्यो॑ नय॒ त्यवाव॑ नयति वैश्वदे॒व्यः॑ । \newline
38. न॒य॒ति॒ वै॒श्व॒दे॒व्यो॑ वैश्वदे॒व्यो॑ नयति नयति वैश्वदे॒व्यो॑ वै वै वै᳚श्वदे॒व्यो॑ नयति नयति वैश्वदे॒व्यो॑ वै । \newline
39. वै॒श्व॒दे॒व्यो॑ वै वै वै᳚श्वदे॒व्यो॑ वैश्वदे॒व्यो॑ वै प्र॒जाः प्र॒जा वै वै᳚श्वदे॒व्यो॑ वैश्वदे॒व्यो॑ वै प्र॒जाः । \newline
40. वै॒श्व॒दे॒व्य॑ इति॑ वैश्व - दे॒व्यः॑ । \newline
41. वै प्र॒जाः प्र॒जा वै वै प्र॒जाः प्र॒जासु॑ प्र॒जासु॑ प्र॒जा वै वै प्र॒जाः प्र॒जासु॑ । \newline
42. प्र॒जाः प्र॒जासु॑ प्र॒जासु॑ प्र॒जाः प्र॒जाः प्र॒जा स्वे॒वैव प्र॒जासु॑ प्र॒जाः प्र॒जाः प्र॒जा स्वे॒व । \newline
43. प्र॒जा इति॑ प्र - जाः । \newline
44. प्र॒जा स्वे॒वैव प्र॒जासु॑ प्र॒जा स्वे॒वायु॒ रायु॑ रे॒व प्र॒जासु॑ प्र॒जा स्वे॒वायुः॑ । \newline
45. प्र॒जास्विति॑ प्र - जासु॑ । \newline
46. ए॒वायु॒ रायु॑ रे॒वै वायु॑र् दधाति दधा॒ त्यायु॑ रे॒वै वायु॑र् दधाति । \newline
47. आयु॑र् दधाति दधा॒ त्यायु॒ रायु॑र् दधाति । \newline
48. द॒धा॒तीति॑ दधाति । \newline
\pagebreak
\markright{ TS 6.5.3.1  \hfill https://www.vedavms.in \hfill}

\section{ TS 6.5.3.1 }

\textbf{TS 6.5.3.1 } \newline
\textbf{Samhita Paata} \newline

य॒ज्ञेन॒ वै दे॒वाः सु॑व॒र्गं ॅलो॒कमा॑य॒न् ते॑ऽमन्यन्त मनु॒ष्या॑ नो॒ऽन्वाभ॑विष्य॒न्तीति॒ ते सं॑ॅवथ्स॒रेण॑ योपयि॒त्वा सु॑व॒र्गं ॅलो॒कमा॑य॒न् तमृष॑य ऋतुग्र॒हैरे॒वानु॒ प्राजा॑न॒न॒. यदृ॑तुग्र॒हा गृ॒ह्यन्ते॑ सुव॒र्गस्य॑ लो॒कस्य॒ प्रज्ञा᳚त्यै॒ द्वाद॑श गृह्यन्ते॒ द्वाद॑श॒ मासाः᳚ संॅवथ्स॒रः सं॑ॅवथ्स॒रस्य॒ प्रज्ञा᳚त्यै स॒ह प्र॑थ॒मौ गृ॑ह्येते स॒होत्त॒मौ तस्मा॒द् द्वौद्वा॑वृ॒तू उ॑भ॒यतो॑मुख-मृतुपा॒त्रं भ॑वति॒ को- [  ] \newline

\textbf{Pada Paata} \newline

य॒ज्ञेन॑ । वै । दे॒वाः । सु॒व॒र्गमिति॑ सुवः - गम् । लो॒कम् । आ॒य॒न्न् । ते । अ॒म॒न्य॒न्त॒ । म॒नु॒ष्याः᳚ । नः॒ । अ॒न्वाभ॑विष्य॒न्तीत्य॑नु-आभ॑विष्यन्ति । इति॑ । ते । सं॒ॅव॒थ्स॒रे॒णेति॑ सं - व॒थ्स॒रेण॑ । यो॒प॒यि॒त्वा । सु॒व॒र्गमिति॑ सुवः - गम् । लो॒कम् । आ॒य॒न्न् । तम् । ऋष॑यः । ऋ॒तु॒ग्र॒हैरित्यृ॑तु - ग्र॒हैः । ए॒व । अनु॑ । प्रेति॑ । अ॒जा॒न॒न्न् । यत् । ऋ॒तु॒ग्र॒हा इत्यृ॑तु - ग्र॒हाः । गृ॒ह्यन्ते᳚ । सु॒व॒र्गसेति॑ सुवः - गस्य॑ । लो॒कस्य॑ । प्रज्ञा᳚त्या॒ इति॒ प्र - ज्ञा॒त्यै॒ । द्वाद॑श । गृ॒ह्य॒न्ते॒ । द्वाद॑श । मासाः᳚ । सं॒ॅव॒थ्स॒र इति॑ सं - व॒थ्स॒रः । सं॒ॅव॒थ्स॒रस्येति॑ सं - व॒थ्स॒रस्य॑ । प्रज्ञा᳚त्या॒ इति॒ प्र - ज्ञा॒त्यै॒ । स॒ह । प्र॒थ॒मौ । गृ॒ह्ये॒ते॒ इति॑ । स॒ह । उ॒त्त॒मावितु॑त् - त॒मौ । तस्मा᳚त् । द्वौद्वा॒विति॒ द्वौ - द्वौ॒ । ऋ॒तू इति॑ । उ॒भ॒यतो॑मुख॒मित्यु॑भ॒यतः॑ - मु॒ख॒म् । ऋ॒तु॒पा॒त्रमित्यृ॑तु - पा॒त्रम् । भ॒व॒ति॒ । कः ।  \newline


\textbf{Krama Paata} \newline

य॒ज्ञेन॒ वै । वै दे॒वाः । दे॒वाः सु॑व॒र्गम् । सु॒व॒र्गम् ॅलो॒कम् । सु॒व॒र्गमिति॑ सुवः - गम् । लो॒कमा॑यन्न् । आ॒य॒न् ते । ते॑ऽमन्यन्त । अ॒म॒न्य॒न्त॒ म॒नु॒ष्याः᳚ । म॒नु॒ष्या॑ नः । नो॒ऽन्वाभ॑विष्यन्ति । अ॒न्वाभ॑विष्य॒न्तीति॑ । अ॒न्वाभ॑विष्य॒न्तीत्य॑नु - आभ॑विष्यन्ति । इति॒ ते । ते स॑म्ॅवथ्स॒रेण॑ । स॒म्ॅव॒थ्स॒रेण॑ योपयि॒त्वा । स॒म्ॅव॒थ्स॒रेणेति॑ सम् - व॒थ्स॒रेण॑ । यो॒प॒यि॒त्वा सु॑व॒र्गम् । सु॒व॒र्गम् ॅलो॒कम् । सु॒व॒र्गमिति॑ सुवः - गम् । लो॒कमा॑यन्न् । आ॒य॒न् तम् । तमृष॑यः । ऋष॑य ऋतुग्र॒हैः । ऋ॒तु॒ग्र॒हैरे॒व । ऋ॒तु॒ग्र॒हैरित्यृ॑तु - ग्र॒हैः । ए॒वानु॑ । अनु॒ प्र । प्राजा॑नन्न् । अ॒जा॒न॒न्॒. यत् । यदृ॑तुग्र॒हाः । ऋ॒तु॒ग्र॒हा गृ॒ह्यन्ते᳚ । ऋ॒तु॒ग्र॒हा इत्यृ॑तु - ग्र॒हाः । गृ॒ह्यन्ते॑ सुव॒र्गस्य॑ । सु॒व॒र्गस्य॑ लो॒कस्य॑ । सु॒व॒र्गस्येति॑ सुवः - गस्य॑ । लो॒कस्य॒ प्रज्ञा᳚त्यै । प्रज्ञा᳚त्यै॒ द्वाद॑श । प्रज्ञा᳚त्या॒ इति॒ प्र - ज्ञा॒त्यै॒ । द्वाद॑श गृह्यन्ते । गृ॒ह्य॒न्ते॒ द्वाद॑श । द्वाद॑श॒ मासाः᳚ । मासाः᳚ सम्ॅवथ्स॒रः । स॒म्ॅव॒थ्स॒रः स॑म्ॅवथ्स॒रस्य॑ । स॒म्ॅव॒थ्स॒र इति॑ सम् - व॒थ्स॒रः । स॒म्ॅव॒थ्स॒रस्य॒ प्रज्ञा᳚त्यै । स॒म्ॅव॒थ्स॒रस्येति॑ सम् - व॒थ्स॒रस्य॑ । प्रज्ञा᳚त्यै स॒ह । प्रज्ञा᳚त्या॒ इति॒ प्र - ज्ञा॒त्यै॒ । स॒ह प्र॑थ॒मौ । प्र॒थ॒मौ गृ॑ह्येते । गृ॒ह्ये॒ते॒ स॒ह । गृ॒ह्ये॒ते॒ इति॑ गृह्येते । स॒होत्त॒मौ । उ॒त्त॒मौ तस्मा᳚त् । उ॒त्त॒मावित्यु॑त् - त॒मौ । तस्मा॒द् द्वौद्वौ᳚ । द्वौद्वा॑वृ॒तू । द्वौद्वा॒विति॒ द्वौ - द्वौ॒ । ऋ॒तू उ॑भ॒यतो॑मुखम् । ऋ॒तू इत्यृ॒तू । उ॒भ॒यतो॑मुखमृतुपा॒त्रम् । उ॒भ॒यतो॑मुख॒मित्यु॑भ॒यतः॑ - मु॒ख॒म् । ऋ॒तु॒पा॒त्रम् भ॑वति । ऋ॒तु॒पा॒त्रमित्यृ॑तु - पा॒त्रम् । भ॒व॒ति॒ कः । को हि \newline

\textbf{Jatai Paata} \newline

1. य॒ज्ञेन॒ वै वै य॒ज्ञेन॑ य॒ज्ञेन॒ वै । \newline
2. वै दे॒वा दे॒वा वै वै दे॒वाः । \newline
3. दे॒वाः सु॑व॒र्गꣳ सु॑व॒र्गम् दे॒वा दे॒वाः सु॑व॒र्गम् । \newline
4. सु॒व॒र्गम् ॅलो॒कम् ॅलो॒कꣳ सु॑व॒र्गꣳ सु॑व॒र्गम् ॅलो॒कम् । \newline
5. सु॒व॒र्गमिति॑ सुवः - गम् । \newline
6. लो॒क मा॑यन् नायन् ॅलो॒कम् ॅलो॒क मा॑यन्न् । \newline
7. आ॒य॒न् ते त आ॑यन् नाय॒न् ते । \newline
8. ते॑ ऽमन्यन्ता मन्यन्त॒ ते ते॑ ऽमन्यन्त । \newline
9. अ॒म॒न्य॒न्त॒ म॒नु॒ष्या॑ मनु॒ष्या॑ अमन्यन्ता मन्यन्त मनु॒ष्याः᳚ । \newline
10. म॒नु॒ष्या॑ नो नो मनु॒ष्या॑ मनु॒ष्या॑ नः । \newline
11. नो॒ ऽन्वाभ॑विष्य न्त्य॒न्वाभ॑विष्यन्ति नो नो॒ ऽन्वाभ॑विष्यन्ति । \newline
12. अ॒न्वाभ॑विष्य॒ न्तीती त्य॒न्वाभ॑विष्य न्त्य॒न्वाभ॑विष्य॒ न्तीति॑ । \newline
13. अ॒न्वाभ॑विष्य॒न्तीत्य॑नु - आभ॑विष्यन्ति । \newline
14. इति॒ ते त इतीति॒ ते । \newline
15. ते सं॑ॅवथ्स॒रेण॑ संॅवथ्स॒रेण॒ ते ते सं॑ॅवथ्स॒रेण॑ । \newline
16. सं॒ॅव॒थ्स॒रेण॑ योपयि॒त्वा यो॑पयि॒त्वा सं॑ॅवथ्स॒रेण॑ संॅवथ्स॒रेण॑ योपयि॒त्वा । \newline
17. सं॒ॅव॒थ्स॒रेणेति॑ सं - व॒थ्स॒रेण॑ । \newline
18. यो॒प॒यि॒त्वा सु॑व॒र्गꣳ सु॑व॒र्गं ॅयो॑पयि॒त्वा यो॑पयि॒त्वा सु॑व॒र्गम् । \newline
19. सु॒व॒र्गम् ॅलो॒कम् ॅलो॒कꣳ सु॑व॒र्गꣳ सु॑व॒र्गम् ॅलो॒कम् । \newline
20. सु॒व॒र्गमिति॑ सुवः - गम् । \newline
21. लो॒क मा॑यन् नायन् ॅलो॒कम् ॅलो॒क मा॑यन्न् । \newline
22. आ॒य॒न् तम् त मा॑यन् नाय॒न् तम् । \newline
23. त मृष॑य॒ ऋष॑य॒ स्तम् त मृष॑यः । \newline
24. ऋष॑य ऋतुग्र॒हैर्. ऋ॑तुग्र॒हैर्. ऋष॑य॒ ऋष॑य ऋतुग्र॒हैः । \newline
25. ऋ॒तु॒ग्र॒है रे॒वैव र्तु॑ग्र॒हैर्. ऋ॑तुग्र॒है रे॒व । \newline
26. ऋ॒तु॒ग्र॒हैरित्यृ॑तु - ग्र॒हैः । \newline
27. ए॒वान् वन् वे॒वैवानु॑ । \newline
28. अनु॒ प्र प्राण्वनु॒ प्र । \newline
29. प्राजा॑नन् नजान॒न् प्र प्राजा॑नन्न् । \newline
30. अ॒जा॒न॒न्॒. यद् यद॑जानन् नजान॒न्॒. यत् । \newline
31. यदृ॑तुग्र॒हा ऋ॑तुग्र॒हा यद् यदृ॑तुग्र॒हाः । \newline
32. ऋ॒तु॒ग्र॒हा गृ॒ह्यन्ते॑ गृ॒ह्यन्त॑ ऋतुग्र॒हा ऋ॑तुग्र॒हा गृ॒ह्यन्ते᳚ । \newline
33. ऋ॒तु॒ग्र॒हा इत्यृ॑तु - ग्र॒हाः । \newline
34. गृ॒ह्यन्ते॑ सुव॒र्गस्य॑ सुव॒र्गस्य॑ गृ॒ह्यन्ते॑ गृ॒ह्यन्ते॑ सुव॒र्गस्य॑ । \newline
35. सु॒व॒र्गस्य॑ लो॒कस्य॑ लो॒कस्य॑ सुव॒र्गस्य॑ सुव॒र्गस्य॑ लो॒कस्य॑ । \newline
36. सु॒व॒र्गस्येति॑ सुवः - गस्य॑ । \newline
37. लो॒कस्य॒ प्रज्ञा᳚त्यै॒ प्रज्ञा᳚त्यै लो॒कस्य॑ लो॒कस्य॒ प्रज्ञा᳚त्यै । \newline
38. प्रज्ञा᳚त्यै॒ द्वाद॑श॒ द्वाद॑श॒ प्रज्ञा᳚त्यै॒ प्रज्ञा᳚त्यै॒ द्वाद॑श । \newline
39. प्रज्ञा᳚त्या॒ इति॒ प्र - ज्ञा॒त्यै॒ । \newline
40. द्वाद॑श गृह्यन्ते गृह्यन्ते॒ द्वाद॑श॒ द्वाद॑श गृह्यन्ते । \newline
41. गृ॒ह्य॒न्ते॒ द्वाद॑श॒ द्वाद॑श गृह्यन्ते गृह्यन्ते॒ द्वाद॑श । \newline
42. द्वाद॑श॒ मासा॒ मासा॒ द्वाद॑श॒ द्वाद॑श॒ मासाः᳚ । \newline
43. मासाः᳚ संॅवथ्स॒रः सं॑ॅवथ्स॒रो मासा॒ मासाः᳚ संॅवथ्स॒रः । \newline
44. सं॒ॅव॒थ्स॒रः सं॑ॅवथ्स॒रस्य॑ संॅवथ्स॒रस्य॑ संॅवथ्स॒रः सं॑ॅवथ्स॒रः सं॑ॅवथ्स॒रस्य॑ । \newline
45. सं॒ॅव॒थ्स॒र इति॑ सं - व॒थ्स॒रः । \newline
46. सं॒ॅव॒थ्स॒रस्य॒ प्रज्ञा᳚त्यै॒ प्रज्ञा᳚त्यै संॅवथ्स॒रस्य॑ संॅवथ्स॒रस्य॒ प्रज्ञा᳚त्यै । \newline
47. सं॒ॅव॒थ्स॒रस्येति॑ सं - व॒थ्स॒रस्य॑ । \newline
48. प्रज्ञा᳚त्यै स॒ह स॒ह प्रज्ञा᳚त्यै॒ प्रज्ञा᳚त्यै स॒ह । \newline
49. प्रज्ञा᳚त्या॒ इति॒ प्र - ज्ञा॒त्यै॒ । \newline
50. स॒ह प्र॑थ॒मौ प्र॑थ॒मौ स॒ह स॒ह प्र॑थ॒मौ । \newline
51. प्र॒थ॒मौ गृ॑ह्येते गृह्येते प्रथ॒मौ प्र॑थ॒मौ गृ॑ह्येते । \newline
52. गृ॒ह्ये॒ते॒ स॒ह स॒ह गृ॑ह्येते गृह्येते स॒ह । \newline
53. गृ॒ह्ये॒ते॒ इति॑ गृह्येते । \newline
54. स॒होत्त॒मा वु॑त्त॒मौ स॒ह स॒होत्त॒मौ । \newline
55. उ॒त्त॒मौ तस्मा॒त् तस्मा॑ दुत्त॒मा वु॑त्त॒मौ तस्मा᳚त् । \newline
56. उ॒त्त॒मावित्यु॑त् - त॒मौ । \newline
57. तस्मा॒द् द्वौद्वौ॒ द्वौद्वौ॒ तस्मा॒त् तस्मा॒द् द्वौद्वौ᳚ । \newline
58. द्वौद्वा॑ वृ॒तू ऋ॒तू द्वौद्वौ॒ द्वौद्वा॑ वृ॒तू । \newline
59. द्वौद्वा॒विति॒ द्वौ - द्वौ॒ । \newline
60. ऋ॒तू उ॑भ॒यतो॑मुख मुभ॒यतो॑मुख मृ॒तू ऋ॒तू उ॑भ॒यतो॑मुखम् । \newline
61. ऋ॒तू इतृ॒तू । \newline
62. उ॒भ॒यतो॑मुख मृतुपा॒त्र मृ॑तुपा॒त्र मु॑भ॒यतो॑मुख मुभ॒यतो॑मुख मृतुपा॒त्रम् । \newline
63. उ॒भ॒यतो॑मुख॒मित्यु॑भ॒यतः॑ - मु॒ख॒म् । \newline
64. ऋ॒तु॒पा॒त्रम् भ॑वति भव त्यृतुपा॒त्र मृ॑तुपा॒त्रम् भ॑वति । \newline
65. ऋ॒तु॒पा॒त्रमित्यृ॑तु - पा॒त्रम् । \newline
66. भ॒व॒ति॒ कः को भ॑वति भवति॒ कः । \newline
67. को हि हि कः को हि । \newline

\textbf{Ghana Paata } \newline

1. य॒ज्ञेन॒ वै वै य॒ज्ञेन॑ य॒ज्ञेन॒ वै दे॒वा दे॒वा वै य॒ज्ञेन॑ य॒ज्ञेन॒ वै दे॒वाः । \newline
2. वै दे॒वा दे॒वा वै वै दे॒वाः सु॑व॒र्गꣳ सु॑व॒र्गम् दे॒वा वै वै दे॒वाः सु॑व॒र्गम् । \newline
3. दे॒वाः सु॑व॒र्गꣳ सु॑व॒र्गम् दे॒वा दे॒वाः सु॑व॒र्गम् ॅलो॒कम् ॅलो॒कꣳ सु॑व॒र्गम् दे॒वा दे॒वाः सु॑व॒र्गम् ॅलो॒कम् । \newline
4. सु॒व॒र्गम् ॅलो॒कम् ॅलो॒कꣳ सु॑व॒र्गꣳ सु॑व॒र्गम् ॅलो॒क मा॑यन्-नायन् ॅलो॒कꣳ सु॑व॒र्गꣳ सु॑व॒र्गम् ॅलो॒क मा॑यन्न् । \newline
5. सु॒व॒र्गमिति॑ सुवः - गम् । \newline
6. लो॒क मा॑यन्-नायन् ॅलो॒कम् ॅलो॒क मा॑य॒न् ते त आ॑यन् ॅलो॒कम् ॅलो॒क मा॑य॒न् ते । \newline
7. आ॒य॒न् ते त आ॑यन्-नाय॒न् ते॑ ऽमन्यन्ता मन्यन्त॒ त आ॑यन्-नाय॒न् ते॑ ऽमन्यन्त । \newline
8. ते॑ ऽमन्यन्ता मन्यन्त॒ ते ते॑ ऽमन्यन्त मनु॒ष्या॑ मनु॒ष्या॑ अमन्यन्त॒ ते ते॑ ऽमन्यन्त मनु॒ष्याः᳚ । \newline
9. अ॒म॒न्य॒न्त॒ म॒नु॒ष्या॑ मनु॒ष्या॑ अमन्यन्ता मन्यन्त मनु॒ष्या॑ नो नो मनु॒ष्या॑ अमन्यन्ता मन्यन्त मनु॒ष्या॑ नः । \newline
10. म॒नु॒ष्या॑ नो नो मनु॒ष्या॑ मनु॒ष्या॑ नो॒ ऽन्वाभ॑विष्य न्त्य॒न्वाभ॑विष्यन्ति नो मनु॒ष्या॑ मनु॒ष्या॑ नो॒ ऽन्वाभ॑विष्यन्ति । \newline
11. नो॒ ऽन्वाभ॑विष्य न्त्य॒न्वाभ॑विष्यन्ति नो नो॒ ऽन्वाभ॑विष्य॒न्तीती त्य॒न्वाभ॑विष्यन्ति नो नो॒ ऽन्वाभ॑विष्य॒न्तीति॑ । \newline
12. अ॒न्वाभ॑विष्य॒न्तीती त्य॒न्वाभ॑विष्य न्त्य॒न्वाभ॑विष्य॒न्तीति॒ ते त इत्य॒न्वाभ॑विष्य न्त्य॒न्वाभ॑विष्य॒न्तीति॒ ते । \newline
13. अ॒न्वाभ॑विष्य॒न्तीत्य॑नु - आभ॑विष्यन्ति । \newline
14. इति॒ ते त इतीति॒ ते सं॑ॅवथ्स॒रेण॑ संॅवथ्स॒रेण॒ त इतीति॒ ते सं॑ॅवथ्स॒रेण॑ । \newline
15. ते सं॑ॅवथ्स॒रेण॑ संॅवथ्स॒रेण॒ ते ते सं॑ॅवथ्स॒रेण॑ योपयि॒त्वा यो॑पयि॒त्वा सं॑ॅवथ्स॒रेण॒ ते ते सं॑ॅवथ्स॒रेण॑ योपयि॒त्वा । \newline
16. सं॒ॅव॒थ्स॒रेण॑ योपयि॒त्वा यो॑पयि॒त्वा सं॑ॅवथ्स॒रेण॑ संॅवथ्स॒रेण॑ योपयि॒त्वा सु॑व॒र्गꣳ सु॑व॒र्गं ॅयो॑पयि॒त्वा सं॑ॅवथ्स॒रेण॑ संॅवथ्स॒रेण॑ योपयि॒त्वा सु॑व॒र्गम् । \newline
17. सं॒ॅव॒थ्स॒रेणेति॑ सं - व॒थ्स॒रेण॑ । \newline
18. यो॒प॒यि॒त्वा सु॑व॒र्गꣳ सु॑व॒र्गं ॅयो॑पयि॒त्वा यो॑पयि॒त्वा सु॑व॒र्गम् ॅलो॒कम् ॅलो॒कꣳ सु॑व॒र्गं ॅयो॑पयि॒त्वा यो॑पयि॒त्वा सु॑व॒र्गम् ॅलो॒कम् । \newline
19. सु॒व॒र्गम् ॅलो॒कम् ॅलो॒कꣳ सु॑व॒र्गꣳ सु॑व॒र्गम् ॅलो॒क मा॑यन्-नायन् ॅलो॒कꣳ सु॑व॒र्गꣳ सु॑व॒र्गम् ॅलो॒क मा॑यन्न् । \newline
20. सु॒व॒र्गमिति॑ सुवः - गम् । \newline
21. लो॒क मा॑यन्-नायन् ॅलो॒कम् ॅलो॒क मा॑य॒न् तम् त मा॑यन् ॅलो॒कम् ॅलो॒क मा॑य॒न् तम् । \newline
22. आ॒य॒न् तम् त मा॑यन्-नाय॒न् त मृष॑य॒ ऋष॑य॒ स्त मा॑यन्-नाय॒न् त मृष॑यः । \newline
23. त मृष॑य॒ ऋष॑य॒ स्तम् त मृष॑य ऋतुग्र॒हैर्. ऋ॑तुग्र॒हैर्. ऋष॑य॒ स्तम् त मृष॑य ऋतुग्र॒हैः । \newline
24. ऋष॑य ऋतुग्र॒हैर्. ऋ॑तुग्र॒हैर्. ऋष॑य॒ ऋष॑य ऋतुग्र॒है रे॒वैव र्‌तु॑ग्र॒हैर्. ऋष॑य॒ ऋष॑य ऋतुग्र॒है रे॒व । \newline
25. ऋ॒तु॒ग्र॒है रे॒वैव र्‌तु॑ग्र॒हैर्. ऋ॑तुग्र॒है रे॒वान् वन् वे॒व र्‌तु॑ग्र॒हैर्. ऋ॑तुग्र॒है रे॒वानु॑ । \newline
26. ऋ॒तु॒ग्र॒हैरित्यृ॑तु - ग्र॒हैः । \newline
27. ए॒वान् वन् वे॒वैवानु॒ प्र प्राण्वे॒वैवानु॒ प्र । \newline
28. अनु॒ प्र प्राण्वनु॒ प्राजा॑नन्-नजान॒न् प्राण्वनु॒ प्राजा॑नन्न् । \newline
29. प्राजा॑नन्-नजान॒न् प्र प्राजा॑न॒न्॒. यद् यद॑जान॒न् प्र प्राजा॑न॒न्॒. यत् । \newline
30. अ॒जा॒न॒न्॒. यद् यद॑जानन्-नजान॒न्॒. यदृ॑तुग्र॒हा ऋ॑तुग्र॒हा यद॑जानन्-नजान॒न्॒. यदृ॑तुग्र॒हाः । \newline
31. यदृ॑तुग्र॒हा ऋ॑तुग्र॒हा यद् यदृ॑तुग्र॒हा गृ॒ह्यन्ते॑ गृ॒ह्यन्त॑ ऋतुग्र॒हा यद् यदृ॑तुग्र॒हा गृ॒ह्यन्ते᳚ । \newline
32. ऋ॒तु॒ग्र॒हा गृ॒ह्यन्ते॑ गृ॒ह्यन्त॑ ऋतुग्र॒हा ऋ॑तुग्र॒हा गृ॒ह्यन्ते॑ सुव॒र्गस्य॑ सुव॒र्गस्य॑ गृ॒ह्यन्त॑ ऋतुग्र॒हा ऋ॑तुग्र॒हा गृ॒ह्यन्ते॑ सुव॒र्गस्य॑ । \newline
33. ऋ॒तु॒ग्र॒हा इत्यृ॑तु - ग्र॒हाः । \newline
34. गृ॒ह्यन्ते॑ सुव॒र्गस्य॑ सुव॒र्गस्य॑ गृ॒ह्यन्ते॑ गृ॒ह्यन्ते॑ सुव॒र्गस्य॑ लो॒कस्य॑ लो॒कस्य॑ सुव॒र्गस्य॑ गृ॒ह्यन्ते॑ गृ॒ह्यन्ते॑ सुव॒र्गस्य॑ लो॒कस्य॑ । \newline
35. सु॒व॒र्गस्य॑ लो॒कस्य॑ लो॒कस्य॑ सुव॒र्गस्य॑ सुव॒र्गस्य॑ लो॒कस्य॒ प्रज्ञा᳚त्यै॒ प्रज्ञा᳚त्यै लो॒कस्य॑ सुव॒र्गस्य॑ सुव॒र्गस्य॑ लो॒कस्य॒ प्रज्ञा᳚त्यै । \newline
36. सु॒व॒र्गस्येति॑ सुवः - गस्य॑ । \newline
37. लो॒कस्य॒ प्रज्ञा᳚त्यै॒ प्रज्ञा᳚त्यै लो॒कस्य॑ लो॒कस्य॒ प्रज्ञा᳚त्यै॒ द्वाद॑श॒ द्वाद॑श॒ प्रज्ञा᳚त्यै लो॒कस्य॑ लो॒कस्य॒ प्रज्ञा᳚त्यै॒ द्वाद॑श । \newline
38. प्रज्ञा᳚त्यै॒ द्वाद॑श॒ द्वाद॑श॒ प्रज्ञा᳚त्यै॒ प्रज्ञा᳚त्यै॒ द्वाद॑श गृह्यन्ते गृह्यन्ते॒ द्वाद॑श॒ प्रज्ञा᳚त्यै॒ प्रज्ञा᳚त्यै॒ द्वाद॑श गृह्यन्ते । \newline
39. प्रज्ञा᳚त्या॒ इति॒ प्र - ज्ञा॒त्यै॒ । \newline
40. द्वाद॑श गृह्यन्ते गृह्यन्ते॒ द्वाद॑श॒ द्वाद॑श गृह्यन्ते॒ द्वाद॑श॒ द्वाद॑श गृह्यन्ते॒ द्वाद॑श॒ द्वाद॑श गृह्यन्ते॒ द्वाद॑श । \newline
41. गृ॒ह्य॒न्ते॒ द्वाद॑श॒ द्वाद॑श गृह्यन्ते गृह्यन्ते॒ द्वाद॑श॒ मासा॒ मासा॒ द्वाद॑श गृह्यन्ते गृह्यन्ते॒ द्वाद॑श॒ मासाः᳚ । \newline
42. द्वाद॑श॒ मासा॒ मासा॒ द्वाद॑श॒ द्वाद॑श॒ मासाः᳚ संॅवथ्स॒रः सं॑ॅवथ्स॒रो मासा॒ द्वाद॑श॒ द्वाद॑श॒ मासाः᳚ संॅवथ्स॒रः । \newline
43. मासाः᳚ संॅवथ्स॒रः सं॑ॅवथ्स॒रो मासा॒ मासाः᳚ संॅवथ्स॒रः सं॑ॅवथ्स॒रस्य॑ संॅवथ्स॒रस्य॑ संॅवथ्स॒रो मासा॒ मासाः᳚ संॅवथ्स॒रः सं॑ॅवथ्स॒रस्य॑ । \newline
44. सं॒ॅव॒थ्स॒रः सं॑ॅवथ्स॒रस्य॑ संॅवथ्स॒रस्य॑ संॅवथ्स॒रः सं॑ॅवथ्स॒रः सं॑ॅवथ्स॒रस्य॒ प्रज्ञा᳚त्यै॒ प्रज्ञा᳚त्यै संॅवथ्स॒रस्य॑ संॅवथ्स॒रः सं॑ॅवथ्स॒रः सं॑ॅवथ्स॒रस्य॒ प्रज्ञा᳚त्यै । \newline
45. सं॒ॅव॒थ्स॒र इति॑ सं - व॒थ्स॒रः । \newline
46. सं॒ॅव॒थ्स॒रस्य॒ प्रज्ञा᳚त्यै॒ प्रज्ञा᳚त्यै संॅवथ्स॒रस्य॑ संॅवथ्स॒रस्य॒ प्रज्ञा᳚त्यै स॒ह स॒ह प्रज्ञा᳚त्यै संॅवथ्स॒रस्य॑ संॅवथ्स॒रस्य॒ प्रज्ञा᳚त्यै स॒ह । \newline
47. सं॒ॅव॒थ्स॒रस्येति॑ सं - व॒थ्स॒रस्य॑ । \newline
48. प्रज्ञा᳚त्यै स॒ह स॒ह प्रज्ञा᳚त्यै॒ प्रज्ञा᳚त्यै स॒ह प्र॑थ॒मौ प्र॑थ॒मौ स॒ह प्रज्ञा᳚त्यै॒ प्रज्ञा᳚त्यै स॒ह प्र॑थ॒मौ । \newline
49. प्रज्ञा᳚त्या॒ इति॒ प्र - ज्ञा॒त्यै॒ । \newline
50. स॒ह प्र॑थ॒मौ प्र॑थ॒मौ स॒ह स॒ह प्र॑थ॒मौ गृ॑ह्येते गृह्येते प्रथ॒मौ स॒ह स॒ह प्र॑थ॒मौ गृ॑ह्येते । \newline
51. प्र॒थ॒मौ गृ॑ह्येते गृह्येते प्रथ॒मौ प्र॑थ॒मौ गृ॑ह्येते स॒ह स॒ह गृ॑ह्येते प्रथ॒मौ प्र॑थ॒मौ गृ॑ह्येते स॒ह । \newline
52. गृ॒ह्ये॒ते॒ स॒ह स॒ह गृ॑ह्येते गृह्येते स॒होत्त॒मा वु॑त्त॒मौ स॒ह गृ॑ह्येते गृह्येते स॒होत्त॒मौ । \newline
53. गृ॒ह्ये॒ते॒ इति॑ गृह्येते । \newline
54. स॒होत्त॒मा वु॑त्त॒मौ स॒ह स॒होत्त॒मौ तस्मा॒त् तस्मा॑ दुत्त॒मौ स॒ह स॒होत्त॒मौ तस्मा᳚त् । \newline
55. उ॒त्त॒मौ तस्मा॒त् तस्मा॑ दुत्त॒मा वु॑त्त॒मौ तस्मा॒द् द्वौद्वौ॒ द्वौद्वौ॒ तस्मा॑ दुत्त॒मा वु॑त्त॒मौ तस्मा॒द् द्वौद्वौ᳚ । \newline
56. उ॒त्त॒मावित्यु॑त् - त॒मौ । \newline
57. तस्मा॒द् द्वौद्वौ॒ द्वौद्वौ॒ तस्मा॒त् तस्मा॒द् द्वौद्वा॑ वृ॒तू ऋ॒तू द्वौद्वौ॒ तस्मा॒त् तस्मा॒द् द्वौद्वा॑ वृ॒तू । \newline
58. द्वौद्वा॑ वृ॒तू ऋ॒तू द्वौद्वौ॒ द्वौद्वा॑ वृ॒तू उ॑भ॒यतो॑मुख मुभ॒यतो॑मुख मृ॒तू द्वौद्वौ॒ द्वौद्वा॑ वृ॒तू उ॑भ॒यतो॑मुखम् । \newline
59. द्वौद्वा॒विति॒ द्वौ - द्वौ॒ । \newline
60. ऋ॒तू उ॑भ॒यतो॑मुख मुभ॒यतो॑मुख मृ॒तू ऋ॒तू उ॑भ॒यतो॑मुख मृतुपा॒त्र मृ॑तुपा॒त्र मु॑भ॒यतो॑मुख मृ॒तू ऋ॒तू उ॑भ॒यतो॑मुख मृतुपा॒त्रम् । \newline
61. ऋ॒तू इतृ॒तू । \newline
62. उ॒भ॒यतो॑मुख मृतुपा॒त्र मृ॑तुपा॒त्र मु॑भ॒यतो॑मुख मुभ॒यतो॑मुख मृतुपा॒त्रम् भ॑वति भव त्यृतुपा॒त्र मु॑भ॒यतो॑मुख मुभ॒यतो॑मुख मृतुपा॒त्रम् भ॑वति । \newline
63. उ॒भ॒यतो॑मुख॒मित्यु॑भ॒यतः॑ - मु॒ख॒म् । \newline
64. ऋ॒तु॒पा॒त्रम् भ॑वति भव त्यृतुपा॒त्र मृ॑तुपा॒त्रम् भ॑वति॒ कः को भ॑व त्यृतुपा॒त्र मृ॑तुपा॒त्रम् भ॑वति॒ कः । \newline
65. ऋ॒तु॒पा॒त्रमित्यृ॑तु - पा॒त्रम् । \newline
66. भ॒व॒ति॒ कः को भ॑वति भवति॒ को हि हि को भ॑वति भवति॒ को हि । \newline
67. को हि हि कः को हि तत् तद्धि कः को हि तत् । \newline
\pagebreak
\markright{ TS 6.5.3.2  \hfill https://www.vedavms.in \hfill}

\section{ TS 6.5.3.2 }

\textbf{TS 6.5.3.2 } \newline
\textbf{Samhita Paata} \newline

हि तद्-वेद॒ यत॑ ऋतू॒नां मुख॑मृ॒तुना॒ प्रेष्येति॒ षट् कृत्व॑ आह॒ षड्वा ऋ॒तव॑ ऋ॒तूने॒व प्री॑णात्यृ॒तुभि॒रिति॑ च॒तुश्चतु॑ष्पद ए॒व प॒शून् प्री॑णाति॒ द्विः पुन॑र्.ऋ॒तुना॑ऽऽह द्वि॒पद॑ ए॒व प्री॑णात्यृ॒तुना॒ प्रेष्येति॒ षट् कृत्व॑ आह॒र्तुभि॒रिति॑ च॒तुस्तस्मा॒-च्चतु॑ष्पादः प॒शव॑ ऋ॒तूनुप॑ जीवन्ति॒ द्विः- [  ] \newline

\textbf{Pada Paata} \newline

हि । तत् । वेद॑ । यतः॑ । ऋ॒तू॒नाम् । मुख᳚म् । ऋ॒तुना᳚ । प्रेति॑ । इ॒ष्य॒ । इति॑ । षट् । कृत्वः॑ । आ॒ह॒ । षट् । वै । ऋ॒तवः॑ । ऋ॒तून् । ए॒व । प्री॒णा॒ति॒ । ऋ॒तुभि॒रित्यृ॒तु - भिः॒ । इति॑ । च॒तुः । चतु॑ष्पद॒ इति॒ चतुः॑ - प॒दः॒ । ए॒व । प॒शून् । प्री॒णा॒ति॒ । द्विः । पुनः॑ । ऋ॒तुना᳚ । आ॒ह॒ । द्वि॒पद॒ इति॑ द्वि - पदः॑ । ए॒व । प्री॒णा॒ति॒ । ऋ॒तुना᳚ । प्रेति॑ । इ॒ष्य॒ । इति॑ । षट् । कृत्वः॑ । आ॒ह॒ । ऋ॒तुभि॒रित्यृ॒तु - भिः॒ । इति॑ । च॒तुः । तस्मा᳚त् । चतु॑ष्पाद॒ इति॒ चतुः॑ - पा॒दः॒ । प॒शवः॑ । ऋ॒तून् । उपेति॑ । जी॒व॒न्ति॒ । द्विः ।  \newline


\textbf{Krama Paata} \newline

हि तत् । तद् वेद॑ । वेद॒ यतः॑ । यत॑ ऋतू॒नाम् । ऋ॒तू॒नाम् मुख᳚म् । मुख॑मृ॒तुना᳚ । ऋ॒तुना॒ प्र । प्रेष्य॑ । इ॒ष्येति॑ । इति॒ षट् । षट् कृत्वः॑ । कृत्व॑ आह । आ॒ह॒ षट् । षड् वै । वा ऋ॒तवः॑ । ऋ॒तव॑ ऋ॒तून् । ऋ॒तूने॒व । ए॒व प्री॑णाति । प्री॒णा॒त्यृ॒तुभिः॑ । ऋ॒तभि॒रिति॑ । ऋ॒तुभि॒रित्यृ॒तु - भिः॒ । इति॑ च॒तुः । च॒तुश्चतु॑ष्पदः । चतु॑ष्पद ए॒व । चतु॑ष्पद॒ इति॒ चतुः॑ - प॒दः॒ । ए॒व प॒शून् । प॒शून् प्री॑णाति । प्री॒णा॒ति॒ द्विः । द्विः पुनः॑ । पुन॑र्. ऋ॒तुना᳚ । ऋ॒तुना॑ऽऽह । आ॒ह॒ द्वि॒पदः॑ । द्वि॒पद॑ ए॒व । द्वि॒पद॒ इति॑ द्वि - पदः॑ । ए॒व प्री॑णाति । प्री॒णा॒त्यृ॒तुना᳚ । ऋ॒तुना॒ प्र । प्रेष्य॑ । इ॒ष्येति॑ । इति॒ षट् । षट् कृत्वः॑ । कृत्व॑ आह । आ॒ह॒र्तुभिः॑ । ऋ॒तुभि॒रिति॑ । ऋ॒तुभि॒रित्यृ॒तु - भिः॒ । इति॑ च॒तुः । च॒तुस्तस्मा᳚त् । तस्मा॒च् चतु॑ष्पादः । चतु॑ष्पादः प॒शवः॑ । चतु॑ष्पाद॒ इति॒ चतुः॑ - पा॒दः॒ । प॒शव॑ ऋ॒तून् । ऋ॒तूनुप॑ । उप॑ जीवन्ति । जी॒व॒न्ति॒ द्विः । द्विः पुनः॑ \newline

\textbf{Jatai Paata} \newline

1. हि तत् तद्धि हि तत् । \newline
2. तद् वेद॒ वेद॒ तत् तद् वेद॑ । \newline
3. वेद॒ यतो॒ यतो॒ वेद॒ वेद॒ यतः॑ । \newline
4. यत॑ ऋतू॒ना मृ॑तू॒नां ॅयतो॒ यत॑ ऋतू॒नाम् । \newline
5. ऋ॒तू॒नाम् मुख॒म् मुख॑ मृतू॒ना मृ॑तू॒नाम् मुख᳚म् । \newline
6. मुख॑ मृ॒तुन॒ र्‌तुना॒ मुख॒म् मुख॑ मृ॒तुना᳚ । \newline
7. ऋ॒तुना॒ प्र प्रा र्‌तुन॒ र्‌तुना॒ प्र । \newline
8. प्रे ह्ये᳚ष्य॒ प्र प्रेष्य॑ । \newline
9. इ॒ष्ये तीती᳚ष्ये॒ ष्येति॑ । \newline
10. इति॒ षट् थ्षडितीति॒ षट् । \newline
11. षट् कृत्वः॒ कृत्व॒ ष्षट् थ्षट् कृत्वः॑ । \newline
12. कृत्व॑ आहाह॒ कृत्वः॒ कृत्व॑ आह । \newline
13. आ॒ह॒ षट् थ्षडा॑ हाह॒ षट् । \newline
14. षड् वै वै षट् थ्षड् वै । \newline
15. वा ऋ॒तव॑ ऋ॒तवो॒ वै वा ऋ॒तवः॑ । \newline
16. ऋ॒तव॑ ऋ॒तू नृ॒तू नृ॒तव॑ ऋ॒तव॑ ऋ॒तून् । \newline
17. ऋ॒तू ने॒वैव र्‌तू नृ॒तू ने॒व । \newline
18. ए॒व प्री॑णाति प्रीणा त्ये॒वैव प्री॑णाति । \newline
19. प्री॒णा॒ त्यृ॒तुभिर्॑. ऋ॒तुभिः॑ प्रीणाति प्रीणा त्यृ॒तुभिः॑ । \newline
20. ऋ॒तुभि॒ रिती त्यृ॒तुभिर्॑. ऋ॒तुभि॒ रिति॑ । \newline
21. ऋ॒तुभि॒रित्यृ॒तु - भिः॒ । \newline
22. इति॑ च॒तु श्च॒तु रितीति॑ च॒तुः । \newline
23. च॒तु श्चतु॑ष्पद॒ श्चतु॑ष्पद श्च॒तु श्च॒तु श्चतु॑ष्पदः । \newline
24. चतु॑ष्पद ए॒वैव चतु॑ष्पद॒ श्चतु॑ष्पद ए॒व । \newline
25. चतु॑ष्पद॒ इति॒ चतुः॑ - प॒दः॒ । \newline
26. ए॒व प॒शून् प॒शू ने॒वैव प॒शून् । \newline
27. प॒शून् प्री॑णाति प्रीणाति प॒शून् प॒शून् प्री॑णाति । \newline
28. प्री॒णा॒ति॒ द्विर् द्विः प्री॑णाति प्रीणाति॒ द्विः । \newline
29. द्विः पुनः॒ पुन॒र् द्विर् द्विः पुनः॑ । \newline
30. पुनर्॑. ऋ॒तुन॒ र्‌तुना॒ पुनः॒ पुनर्॑. ऋ॒तुना᳚ । \newline
31. ऋ॒तुना॑ ऽऽहाह॒ र्‌तुन॒ र्‌तुना॑ ऽऽह । \newline
32. आ॒ह॒ द्वि॒पदो᳚ द्वि॒पद॑ आहाह द्वि॒पदः॑ । \newline
33. द्वि॒पद॑ ए॒वैव द्वि॒पदो᳚ द्वि॒पद॑ ए॒व । \newline
34. द्वि॒पद॒ इति॑ द्वि - पदः॑ । \newline
35. ए॒व प्री॑णाति प्रीणा त्ये॒वैव प्री॑णाति । \newline
36. प्री॒णा॒ त्यृ॒तुन॒ र्‌तुना᳚ प्रीणाति प्रीणा त्यृ॒तुना᳚ । \newline
37. ऋ॒तुना॒ प्र प्रा र्‌तुन॒ र्‌तुना॒ प्र । \newline
38. प्रेष्ये᳚ष्य॒ प्र प्रेष्य॑ । \newline
39. इ॒ष्ये तीती᳚ष्ये॒ ष्येति॑ । \newline
40. इति॒ षट् थ्षडितीति॒ षट् । \newline
41. षट् कृत्वः॒ कृत्व॒ ष्षट् थ्षट् कृत्वः॑ । \newline
42. कृत्व॑ आहाह॒ कृत्वः॒ कृत्व॑ आह । \newline
43. आ॒ह॒ र्‌तुभिर्॑. ऋ॒तुभि॑ राहाह॒ र्तुभिः॑ । \newline
44. ऋ॒तुभि॒ रिती त्यृ॒तुभिर्॑. ऋ॒तुभि॒ रिति॑ । \newline
45. ऋ॒तुभि॒रित्यृ॒तु - भिः॒ । \newline
46. इति॑ च॒तु श्च॒तु रितीति॑ च॒तुः । \newline
47. च॒तु स्तस्मा॒त् तस्मा᳚च् च॒तु श्च॒तु स्तस्मा᳚त् । \newline
48. तस्मा॒च् चतु॑ष्पाद॒ श्चतु॑ष्पाद॒ स्तस्मा॒त् तस्मा॒च् चतु॑ष्पादः । \newline
49. चतु॑ष्पादः प॒शवः॑ प॒शव॒ श्चतु॑ष्पाद॒ श्चतु॑ष्पादः प॒शवः॑ । \newline
50. चतु॑ष्पाद॒ इति॒ चतुः॑ - पा॒दः॒ । \newline
51. प॒शव॑ ऋ॒तू नृ॒तून् प॒शवः॑ प॒शव॑ ऋ॒तून् । \newline
52. ऋ॒तू नुपोपा॒ र्‌तू नृ॒तू नुप॑ । \newline
53. उप॑ जीवन्ति जीव॒ न्त्युपोप॑ जीवन्ति । \newline
54. जी॒व॒न्ति॒ द्विर् द्विर् जी॑वन्ति जीवन्ति॒ द्विः । \newline
55. द्विः पुनः॒ पुन॒र् द्विर् द्विः पुनः॑ । \newline

\textbf{Ghana Paata } \newline

1. हि तत् तद्धि हि तद् वेद॒ वेद॒ तद्धि हि तद् वेद॑ । \newline
2. तद् वेद॒ वेद॒ तत् तद् वेद॒ यतो॒ यतो॒ वेद॒ तत् तद् वेद॒ यतः॑ । \newline
3. वेद॒ यतो॒ यतो॒ वेद॒ वेद॒ यत॑ ऋतू॒ना मृ॑तू॒नां ॅयतो॒ वेद॒ वेद॒ यत॑ ऋतू॒नाम् । \newline
4. यत॑ ऋतू॒ना मृ॑तू॒नां ॅयतो॒ यत॑ ऋतू॒नाम् मुख॒म् मुख॑ मृतू॒नां ॅयतो॒ यत॑ ऋतू॒नाम् मुख᳚म् । \newline
5. ऋ॒तू॒नाम् मुख॒म् मुख॑ मृतू॒ना मृ॑तू॒नाम् मुख॑ मृ॒तुन॒ र्‌तुना॒ मुख॑ मृतू॒ना मृ॑तू॒नाम् मुख॑ मृ॒तुना᳚ । \newline
6. मुख॑ मृ॒तुन॒ र्‌तुना॒ मुख॒म् मुख॑ मृ॒तुना॒ प्र प्रा र्‌तुना॒ मुख॒म् मुख॑ मृ॒तुना॒ प्र । \newline
7. ऋ॒तुना॒ प्र प्रा र्‌तुन॒ र्‌तुना॒ प्रेष्ये᳚ ष्य॒ प्रा र्‌तुन॒ र्‌तुना॒ प्रेष्य॑ । \newline
8. प्रेष्ये᳚ ष्य॒ प्र प्रेष्ये तीती᳚ष्य॒ प्र प्रेष्येति॑ । \newline
9. इ॒ष्ये तीती᳚ष्ये॒ ष्येति॒ षट् थ्षडिती᳚ ष्ये॒ ष्येति॒ षट् । \newline
10. इति॒ षट् थ्षडि तीति॒ षट् कृत्वः॒ कृत्व॒ ष्षडितीति॒ षट् कृत्वः॑ । \newline
11. षट् कृत्वः॒ कृत्व॒ ष्षट् थ्षट् कृत्व॑ आहाह॒ कृत्व॒ ष्षट् थ्षट् कृत्व॑ आह । \newline
12. कृत्व॑ आहाह॒ कृत्वः॒ कृत्व॑ आह॒ षट् थ्षडा॑ह॒ कृत्वः॒ कृत्व॑ आह॒ षट् । \newline
13. आ॒ह॒ षट् थ्षडा॑ हाह॒ षड् वै वै षडा॑ हाह॒ षड् वै । \newline
14. षड् वै वै षट् थ्षड् वा ऋ॒तव॑ ऋ॒तवो॒ वै षट् थ्षड् वा ऋ॒तवः॑ । \newline
15. वा ऋ॒तव॑ ऋ॒तवो॒ वै वा ऋ॒तव॑ ऋ॒तू-नृ॒तू-नृ॒तवो॒ वै वा ऋ॒तव॑ ऋ॒तून् । \newline
16. ऋ॒तव॑ ऋ॒तू-नृ॒तू-नृ॒तव॑ ऋ॒तव॑ ऋ॒तू-ने॒वैव र्‌तू-नृ॒तव॑ ऋ॒तव॑ ऋ॒तू-ने॒व । \newline
17. ऋ॒तू-ने॒वैव र्‌तू-नृ॒तू-ने॒व प्री॑णाति प्रीणा त्ये॒व र्‌तू-नृ॒तू-ने॒व प्री॑णाति । \newline
18. ए॒व प्री॑णाति प्रीणा त्ये॒वैव प्री॑णा त्यृ॒तुभिर्॑. ऋ॒तुभिः॑ प्रीणा त्ये॒वैव प्री॑णा त्यृ॒तुभिः॑ । \newline
19. प्री॒णा॒ त्यृ॒तुभिर्॑. ऋ॒तुभिः॑ प्रीणाति प्रीणा त्यृ॒तुभि॒ रिती त्यृ॒तुभिः॑ प्रीणाति प्रीणा त्यृ॒तुभि॒रिति॑ । \newline
20. ऋ॒तुभि॒ रिती त्यृ॒तुभिर्॑. ऋ॒तुभि॒रिति॑ च॒तु श्च॒तुरि त्यृ॒तुभिर्॑. ऋ॒तुभि॒रिति॑ च॒तुः । \newline
21. ऋ॒तुभि॒रित्यृ॒तु - भिः॒ । \newline
22. इति॑ च॒तु श्च॒तु रितीति॑ च॒तु श्चतु॑ष्पद॒ श्चतु॑ष्पद श्च॒तु रितीति॑ च॒तु श्चतु॑ष्पदः । \newline
23. च॒तु श्चतु॑ष्पद॒ श्चतु॑ष्पद श्च॒तु श्च॒तु श्चतु॑ष्पद ए॒वैव चतु॑ष्पद श्च॒तु श्च॒तु श्चतु॑ष्पद ए॒व । \newline
24. चतु॑ष्पद ए॒वैव चतु॑ष्पद॒ श्चतु॑ष्पद ए॒व प॒शून् प॒शूने॒व चतु॑ष्पद॒ श्चतु॑ष्पद ए॒व प॒शून् । \newline
25. चतु॑ष्पद॒ इति॒ चतुः॑ - प॒दः॒ । \newline
26. ए॒व प॒शून् प॒शू ने॒वैव प॒शून् प्री॑णाति प्रीणाति प॒शूने॒वैव प॒शून् प्री॑णाति । \newline
27. प॒शून् प्री॑णाति प्रीणाति प॒शून् प॒शून् प्री॑णाति॒ द्विर् द्विः प्री॑णाति प॒शून् प॒शून् प्री॑णाति॒ द्विः । \newline
28. प्री॒णा॒ति॒ द्विर् द्विः प्री॑णाति प्रीणाति॒ द्विः पुनः॒ पुन॒र् द्विः प्री॑णाति प्रीणाति॒ द्विः पुनः॑ । \newline
29. द्विः पुनः॒ पुन॒र् द्विर् द्विः पुनर्॑. ऋ॒तुन॒ र्‌तुना॒ पुन॒र् द्विर् द्विः पुनर्॑. ऋ॒तुना᳚ । \newline
30. पुनर्॑. ऋ॒तुन॒ र्‌तुना॒ पुनः॒ पुनर्॑. ऋ॒तुना॑ ऽऽहाह॒ र्‌तुना॒ पुनः॒ पुनर्॑. ऋ॒तुना॑ ऽऽह । \newline
31. ऋ॒तुना॑ ऽऽहाह॒ र्‌तुन॒ र्‌तुना॑ ऽऽह द्वि॒पदो᳚ द्वि॒पद॑ आह॒ र्‌तुन॒ र्‌तुना॑ ऽऽह द्वि॒पदः॑ । \newline
32. आ॒ह॒ द्वि॒पदो᳚ द्वि॒पद॑ आहाह द्वि॒पद॑ ए॒वैव द्वि॒पद॑ आहाह द्वि॒पद॑ ए॒व । \newline
33. द्वि॒पद॑ ए॒वैव द्वि॒पदो᳚ द्वि॒पद॑ ए॒व प्री॑णाति प्रीणा त्ये॒व द्वि॒पदो᳚ द्वि॒पद॑ ए॒व प्री॑णाति । \newline
34. द्वि॒पद॒ इति॑ द्वि - पदः॑ । \newline
35. ए॒व प्री॑णाति प्रीणा त्ये॒वैव प्री॑णा त्यृ॒तुन॒ र्‌तुना᳚ प्रीणा त्ये॒वैव प्री॑णा त्यृ॒तुना᳚ । \newline
36. प्री॒णा॒ त्यृ॒तुन॒ र्‌तुना᳚ प्रीणाति प्रीणा त्यृ॒तुना॒ प्र प्रा र्‌तुना᳚ प्रीणाति प्रीणा त्यृ॒तुना॒ प्र । \newline
37. ऋ॒तुना॒ प्र प्रा र्‌तुन॒ र्‌तुना॒ प्रेष्ये᳚ ष्य॒ प्रा र्‌तुन॒ र्‌तुना॒ प्रेष्य॑ । \newline
38. प्रेष्ये᳚ ष्य॒ प्र प्रेष्ये तीती᳚ष्य॒ प्र प्रेष्येति॑ । \newline
39. इ॒ष्ये तीती᳚ष्ये॒ ष्येति॒ षट् थ्षडिती᳚ ष्ये॒ ष्येति॒ षट् । \newline
40. इति॒ षट् थ्षडितीति॒ षट् कृत्वः॒ कृत्व॒ ष्षडितीति॒ षट् कृत्वः॑ । \newline
41. षट् कृत्वः॒ कृत्व॒ ष्षट् थ्षट् कृत्व॑ आहाह॒ कृत्व॒ ष्षट् थ्षट् कृत्व॑ आह । \newline
42. कृत्व॑ आहाह॒ कृत्वः॒ कृत्व॑ आह॒ र्‌तुभिर्॑. ऋ॒तुभि॑ राह॒ कृत्वः॒ कृत्व॑ आह॒ र्‌तुभिः॑ । \newline
43. आ॒ह॒ र्‌तुभिर्॑. ऋ॒तुभि॑ राहाह॒ र्‌तुभि॒रिती त्यृ॒तुभि॑ राहाह॒ र्‌तुभि॒ रिति॑ । \newline
44. ऋ॒तुभि॒रिती त्यृ॒तुभिर्॑. ऋ॒तुभि॒रिति॑ च॒तु श्च॒तुरि त्यृ॒तुभिर्॑. ऋ॒तुभि॒रिति॑ च॒तुः । \newline
45. ऋ॒तुभि॒रित्यृ॒तु - भिः॒ । \newline
46. इति॑ च॒तु श्च॒तु रितीति॑ च॒तु स्तस्मा॒त् तस्मा᳚च् च॒तु रितीति॑ च॒तु स्तस्मा᳚त् । \newline
47. च॒तु स्तस्मा॒त् तस्मा᳚च् च॒तु श्च॒तु स्तस्मा॒च् चतु॑ष्पाद॒ श्चतु॑ष्पाद॒ स्तस्मा᳚च् च॒तु श्च॒तु स्तस्मा॒च् चतु॑ष्पादः । \newline
48. तस्मा॒च् चतु॑ष्पाद॒ श्चतु॑ष्पाद॒ स्तस्मा॒त् तस्मा॒च् चतु॑ष्पादः प॒शवः॑ प॒शव॒ श्चतु॑ष्पाद॒ स्तस्मा॒त् तस्मा॒च् चतु॑ष्पादः प॒शवः॑ । \newline
49. चतु॑ष्पादः प॒शवः॑ प॒शव॒ श्चतु॑ष्पाद॒ श्चतु॑ष्पादः प॒शव॑ ऋ॒तू-नृ॒तून् प॒शव॒ श्चतु॑ष्पाद॒ श्चतु॑ष्पादः प॒शव॑ ऋ॒तून् । \newline
50. चतु॑ष्पाद॒ इति॒ चतुः॑ - पा॒दः॒ । \newline
51. प॒शव॑ ऋ॒तू-नृ॒तून् प॒शवः॑ प॒शव॑ ऋ॒तू-नुपोपा॒ र्‌तून् प॒शवः॑ प॒शव॑ ऋ॒तू-नुप॑ । \newline
52. ऋ॒तू-नुपोपा॒ र्‌तू-नृ॒तू-नुप॑ जीवन्ति जीव॒ न्त्युपा॒ र्‌तू-नृ॒तू-नुप॑ जीवन्ति । \newline
53. उप॑ जीवन्ति जीव॒ न्त्युपोप॑ जीवन्ति॒ द्विर् द्विर् जी॑व॒ न्त्युपोप॑ जीवन्ति॒ द्विः । \newline
54. जी॒व॒न्ति॒ द्विर् द्विर् जी॑वन्ति जीवन्ति॒ द्विः पुनः॒ पुन॒र् द्विर् जी॑वन्ति जीवन्ति॒ द्विः पुनः॑ । \newline
55. द्विः पुनः॒ पुन॒र् द्विर् द्विः पुनर्॑. ऋ॒तुन॒ र्‌तुना॒ पुन॒र् द्विर् द्विः पुनर्॑. ऋ॒तुना᳚ । \newline
\pagebreak
\markright{ TS 6.5.3.3  \hfill https://www.vedavms.in \hfill}

\section{ TS 6.5.3.3 }

\textbf{TS 6.5.3.3 } \newline
\textbf{Samhita Paata} \newline

पुन॑र्.ऋ॒तुना॑ऽऽह॒ तस्मा᳚द्-द्वि॒पाद॒श्चतु॑ष्पदः प॒शूनुप॑ जीवन्त्यृ॒तुना॒ प्रेष्येति॒ षट् कृत्व॑ आह॒र्तुभि॒रिति॑ च॒तुर्द्विः पुन॑र् ऋ॒तुना॑ऽऽहा॒ ऽऽक्रम॑णमे॒व तथ् सेतुं॒ ॅयज॑मानः कुरुते सुव॒र्गस्य॑ लो॒कस्य॒ सम॑ष्ट्यै॒ नान्यो᳚ऽन्यमनु॒ प्रप॑द्येत॒ यद॒न्यो᳚ऽन्यम॑नु प्र॒पद्ये॑त॒र्तुर्. ऋ॒तुमनु॒ प्रप॑द्येत॒र्तवो॒ मोहु॑काः स्युः॒- [  ] \newline

\textbf{Pada Paata} \newline

पुनः॑ । ऋ॒तुना᳚ । आ॒ह॒ । तस्मा᳚त् । द्वि॒पाद॒ इति॑ द्वि - पादः॑ । चतु॑ष्पद॒ इति॒ चतुः॑ - प॒दः॒ । प॒शून् । उपेति॑ । जी॒व॒न्ति॒ । ऋ॒तुना᳚ । प्रेति॑ । इ॒ष्य॒ । इति॑ । षट् । कृत्वः॑ । आ॒ह॒ । ऋ॒तुभि॒रित्यृ॒तु - भिः॒ । इति॑ । च॒तुः । द्विः । पुनः॑ । ऋ॒तुना᳚ । आ॒ह॒ । आ॒क्रम॑ण॒मित्या᳚ - क्रम॑णम् । ए॒व । तत् । सेतु᳚म् । यज॑मानः । कु॒रु॒ते॒ । सु॒व॒र्गसेति॑ सुवः - गस्य॑ । लो॒कस्य॑ । सम॑ष्ट्या॒ इति॒ सं - अ॒ष्ट्यै॒ । न । अ॒न्यः । अ॒न्यम् । अनु॑ । प्रेति॑ । प॒द्ये॒त॒ । यत् । अ॒न्यः । अ॒न्यम् । अ॒नु॒प्र॒पद्ये॒तेत्य॑नु - प्र॒पद्ये॑त । ऋ॒तुः । ऋ॒तुम् । अनु॑ । प्रेति॑ । प॒द्ये॒त॒ । ऋ॒तवः॑ । मोहु॑काः । स्युः॒ ।  \newline


\textbf{Krama Paata} \newline

पुन॑र्. ऋ॒तुना᳚ । ऋ॒तुना॑ऽऽह । आ॒ह॒ तस्मा᳚त् । तस्मा᳚द् द्वि॒पादः॑ । द्वि॒पाद॒श्चतु॑ष्पदः । द्वि॒पाद॒ इति॑ द्वि - पादः॑ । चतु॑ष्पदः प॒शून् । चतु॑ष्पद॒ इति॒ चतुः॑ - प॒दः॒ । प॒शूनुप॑ । उप॑ जीवन्ति । जी॒व॒न्त्यृ॒तुना᳚ । ऋ॒तुना॒ प्र । प्रेष्य॑ । इ॒ष्येति॑ । इति॒ षट् । षट् कृत्वः॑ । कृत्व॑ आह । आ॒ह॒र्तुभिः॑ । ऋ॒तुभि॒रिति॑ । ऋ॒तुभि॒रित्यृ॒तु - भिः॒ । इति॑ च॒तुः । च॒तुर् द्विः । द्विः पुनः॑ । पुन॑र्. ऋ॒तुना᳚ । ऋ॒तुना॑ऽऽह । आ॒हा॒क्रम॑णम् । आ॒क्रम॑णमे॒व । आ॒क्रम॑ण॒मित्या᳚ - क्रम॑णम् । ए॒व तत् । तथ् सेतु᳚म् । सेतु॒म् ॅयज॑मानः । यज॑मानः कुरुते । कु॒रु॒ते॒ सु॒व॒र्गस्य॑ । सु॒व॒र्गस्य॑ लो॒कस्य॑ । सु॒व॒र्गस्येति॑ सुवः - गस्य॑ । लो॒कस्य॒ सम॑ष्ट्यै । सम॑ष्ट्यै॒ न । सम॑ष्ट्या॒ इति॒ सम् - अ॒ष्ट्यै॒ । नान्यः । अ॒न्यो᳚ऽन्यम् । अ॒न्यमनु॑ । अनु॒ प्र । प्र प॑द्येत । प॒द्ये॒त॒ यत् । यद॒न्यः । अ॒न्यो᳚ऽन्यम् । अ॒न्यम॑नुप्र॒पद्ये॑त । अ॒नु॒प्र॒पद्ये॑त॒र्तुः । अ॒नु॒प्र॒पद्ये॒तेत्य॑नु - प्र॒पद्ये॑त । ऋ॒तुर्. ऋ॒तुम् । ऋ॒तुमनु॑ । अनु॒ प्र । प्र प॑द्येत । प॒द्ये॒त॒र्तवः॑ । ऋ॒तवो॒ मोहु॑काः । मोहु॑काः स्युः ( ) । स्युः॒ प्रसि॑द्धम् \newline

\textbf{Jatai Paata} \newline

1. पुनर्॑. ऋ॒तुन॒ र्‌तुना॒ पुनः॒ पुनर्॑. ऋ॒तुना᳚ । \newline
2. ऋ॒तुना॑ ऽऽहाह॒ र्‌तुन॒ र्‌तुना॑ ऽऽह । \newline
3. आ॒ह॒ तस्मा॒त् तस्मा॑ दाहाह॒ तस्मा᳚त् । \newline
4. तस्मा᳚द् द्वि॒पादो᳚ द्वि॒पाद॒ स्तस्मा॒त् तस्मा᳚द् द्वि॒पादः॑ । \newline
5. द्वि॒पाद॒ श्चतु॑ष्पद॒ श्चतु॑ष्पदो द्वि॒पादो᳚ द्वि॒पाद॒ श्चतु॑ष्पदः । \newline
6. द्वि॒पाद॒ इति॑ द्वि - पादः॑ । \newline
7. चतु॑ष्पदः प॒शून् प॒शूꣳ श्चतु॑ष्पद॒ श्चतु॑ष्पदः प॒शून् । \newline
8. चतु॑ष्पद॒ इति॒ चतुः॑ - प॒दः॒ । \newline
9. प॒शू नुपोप॑ प॒शून् प॒शू नुप॑ । \newline
10. उप॑ जीवन्ति जीव॒ न्त्युपोप॑ जीवन्ति । \newline
11. जी॒व॒ न्त्यृ॒तुन॒ र्‌तुना॑ जीवन्ति जीव न्त्यृ॒तुना᳚ । \newline
12. ऋ॒तुना॒ प्र प्रा र्‌तुन॒ र्‌तुना॒ प्र । \newline
13. प्रेष्ये᳚ष्य॒ प्र प्रेष्य॑ । \newline
14. इ॒ष्ये तीती᳚ष्ये॒ ष्येति॑ । \newline
15. इति॒ षट् थ्षडितीति॒ षट् । \newline
16. षट् कृत्वः॒ कृत्व॒ ष्षट् थ्षट् कृत्वः॑ । \newline
17. कृत्व॑ आहाह॒ कृत्वः॒ कृत्व॑ आह । \newline
18. आ॒ह॒ र्‌तुभिर्॑. ऋ॒तुभि॑ राहाह॒ र्‌तुभिः॑ । \newline
19. ऋ॒तुभि॒ रिती त्यृ॒तुभिर्॑. ऋ॒तुभि॒ रिति॑ । \newline
20. ऋ॒तुभि॒रित्यृ॒तु - भिः॒ । \newline
21. इति॑ च॒तु श्च॒तु रितीति॑ च॒तुः । \newline
22. च॒तुर् द्विर् द्वि श्च॒तु श्च॒तुर् द्विः । \newline
23. द्विः पुनः॒ पुन॒र् द्विर् द्विः पुनः॑ । \newline
24. पुनर्॑. ऋ॒तुन॒ र्‌तुना॒ पुनः॒ पुनर्॑. ऋ॒तुना᳚ । \newline
25. ऋ॒तुना॑ ऽऽहाह॒ र्‌तुन॒ र्‌तुना॑ ऽऽह । \newline
26. आ॒हा॒ क्रम॑ण मा॒क्रम॑ण माहा हा॒क्रम॑णम् । \newline
27. आ॒क्रम॑ण मे॒वैवा क्रम॑ण मा॒क्रम॑ण मे॒व । \newline
28. आ॒क्रम॑ण॒मित्या᳚ - क्रम॑णम् । \newline
29. ए॒व तत् तदे॒ वैव तत् । \newline
30. तथ् सेतुꣳ॒॒ सेतु॒म् तत् तथ् सेतु᳚म् । \newline
31. सेतुं॒ ॅयज॑मानो॒ यज॑मानः॒ सेतुꣳ॒॒ सेतुं॒ ॅयज॑मानः । \newline
32. यज॑मानः कुरुते कुरुते॒ यज॑मानो॒ यज॑मानः कुरुते । \newline
33. कु॒रु॒ते॒ सु॒व॒र्गस्य॑ सुव॒र्गस्य॑ कुरुते कुरुते सुव॒र्गस्य॑ । \newline
34. सु॒व॒र्गस्य॑ लो॒कस्य॑ लो॒कस्य॑ सुव॒र्गस्य॑ सुव॒र्गस्य॑ लो॒कस्य॑ । \newline
35. सु॒व॒र्गस्येति॑ सुवः - गस्य॑ । \newline
36. लो॒कस्य॒ सम॑ष्ट्यै॒ सम॑ष्ट्यै लो॒कस्य॑ लो॒कस्य॒ सम॑ष्ट्यै । \newline
37. सम॑ष्ट्यै॒ न न सम॑ष्ट्यै॒ सम॑ष्ट्यै॒ न । \newline
38. सम॑ष्ट्या॒ इति॒ सं - अ॒ष्ट्यै॒ । \newline
39. नान्यो᳚ ऽन्यो न नान्यः । \newline
40. अ॒न्यो᳚ ऽन्य म॒न्य म॒न्यो᳚(1॒) ऽन्यो᳚ ऽन्यम् । \newline
41. अ॒न्य मन्वन् व॒न्य म॒न्य मनु॑ । \newline
42. अनु॒ प्र प्राण्वनु॒ प्र । \newline
43. प्र प॑द्येत पद्येत॒ प्र प्र प॑द्येत । \newline
44. प॒द्ये॒त॒ यद् यत् प॑द्येत पद्येत॒ यत् । \newline
45. यद॒न्यो᳚ ऽन्यो यद् यद॒न्यः । \newline
46. अ॒न्यो᳚ ऽन्य म॒न्य म॒न्यो᳚(1॒) ऽन्यो᳚ ऽन्यम् । \newline
47. अ॒न्य म॑नुप्र॒पद्ये॑ता नुप्र॒पद्ये॑ता॒ न्य म॒न्य म॑नुप्र॒पद्ये॑त । \newline
48. अ॒नु॒प्र॒पद्ये॑त॒ र्‌तुर्. ऋ॒तु र॑नुप्र॒पद्ये॑ता नुप्र॒पद्ये॑त॒ र्‌तुः । \newline
49. अ॒नु॒प्र॒पद्ये॒तेत्य॑नु - प्र॒पद्ये॑त । \newline
50. ऋ॒तुर्. ऋ॒तु मृ॒तु मृ॒तुर्. ऋ॒तुर्. ऋ॒तुम् । \newline
51. ऋ॒तु मन्वन् वृ॒तु मृ॒तु मनु॑ । \newline
52. अनु॒ प्र प्राण्वनु॒ प्र । \newline
53. प्र प॑द्येत पद्येत॒ प्र प्र प॑द्येत । \newline
54. प॒द्ये॒त॒ र्‌तव॑ ऋ॒तवः॑ पद्येत पद्येत॒ र्‌तवः॑ । \newline
55. ऋ॒तवो॒ मोहु॑का॒ मोहु॑का ऋ॒तव॑ ऋ॒तवो॒ मोहु॑काः । \newline
56. मोहु॑काः स्युः स्यु॒र् मोहु॑का॒ मोहु॑काः स्युः । \newline
57. स्युः॒ प्रसि॑द्ध॒म् प्रसि॑द्धꣳ स्युः स्युः॒ प्रसि॑द्धम् । \newline

\textbf{Ghana Paata } \newline

1. पुनर्॑. ऋ॒तुन॒ र्‌तुना॒ पुनः॒ पुनर्॑. ऋ॒तुना॑ ऽऽहाह॒ र्‌तुना॒ पुनः॒ पुनर्॑. ऋ॒तुना॑ ऽऽह । \newline
2. ऋ॒तुना॑ ऽऽहाह॒ र्‌तुन॒ र्‌तुना॑ ऽऽह॒ तस्मा॒त् तस्मा॑ दाह॒ र्‌तुन॒ र्‌तुना॑ ऽऽह॒ तस्मा᳚त् । \newline
3. आ॒ह॒ तस्मा॒त् तस्मा॑ दाहाह॒ तस्मा᳚द् द्वि॒पादो᳚ द्वि॒पाद॒ स्तस्मा॑ दाहाह॒ तस्मा᳚द् द्वि॒पादः॑ । \newline
4. तस्मा᳚द् द्वि॒पादो᳚ द्वि॒पाद॒ स्तस्मा॒त् तस्मा᳚द् द्वि॒पाद॒ श्चतु॑ष्पद॒ श्चतु॑ष्पदो द्वि॒पाद॒ स्तस्मा॒त् तस्मा᳚द् द्वि॒पाद॒ श्चतु॑ष्पदः । \newline
5. द्वि॒पाद॒ श्चतु॑ष्पद॒ श्चतु॑ष्पदो द्वि॒पादो᳚ द्वि॒पाद॒ श्चतु॑ष्पदः प॒शून् प॒शूꣳ 
श्चतु॑ष्पदो द्वि॒पादो᳚ द्वि॒पाद॒ श्चतु॑ष्पदः प॒शून् । \newline
6. द्वि॒पाद॒ इति॑ द्वि - पादः॑ । \newline
7. चतु॑ष्पदः प॒शून् प॒शूꣳ श्चतु॑ष्पद॒ श्चतु॑ष्पदः प॒शूनुपोप॑ प॒शूꣳ
श्चतु॑ष्पद॒ श्चतु॑ष्पदः प॒शूनुप॑ । \newline
8. चतु॑ष्पद॒ इति॒ चतुः॑ - प॒दः॒ । \newline
9. प॒शूनु पोप॑ प॒शून् प॒शूनुप॑ जीवन्ति जीव॒ न्त्युप॑ प॒शून् प॒शूनुप॑ जीवन्ति । \newline
10. उप॑ जीवन्ति जीव॒ न्त्युपोप॑ जीव न्त्यृ॒तुन॒ र्‌तुना॑ जीव॒ न्त्युपोप॑ जीव न्त्यृ॒तुना᳚ । \newline
11. जी॒व॒ न्त्यृ॒तुन॒ र्‌तुना॑ जीवन्ति जीव न्त्यृ॒तुना॒ प्र प्रा र्‌तुना॑ जीवन्ति जीव न्त्यृ॒तुना॒ प्र । \newline
12. ऋ॒तुना॒ प्र प्रा र्‌तुन॒ र्‌तुना॒ प्रेष्ये᳚ष्य॒ प्रा र्‌तुन॒ र्‌तुना॒ प्रेष्य॑ । \newline
13. प्रेष्ये᳚ष्य॒ प्र प्रेष्ये तीती᳚ष्य॒ प्र प्रेष्येति॑ । \newline
14. इ॒ष्ये तीती᳚ष्ये॒ ष्येति॒ षट् थ्षडिती᳚ ष्ये॒ ष्येति॒ षट् । \newline
15. इति॒ षट् थ्षडितीति॒ षट् कृत्वः॒ कृत्व॒ ष्षडितीति॒ षट् कृत्वः॑ । \newline
16. षट् कृत्वः॒ कृत्व॒ ष्षट् थ्षट् कृत्व॑ आहाह॒ कृत्व॒ ष्षट् थ्षट् कृत्व॑ आह । \newline
17. कृत्व॑ आहाह॒ कृत्वः॒ कृत्व॑ आह॒ र्‌तुभिर्॑. ऋ॒तुभि॑ राह॒ कृत्वः॒ कृत्व॑ आह॒ र्‌तुभिः॑ । \newline
18. आ॒ह॒ र्‌तुभिर्॑. ऋ॒तुभि॑ राहाह॒ र्‌तुभि॒ रिती त्यृ॒तुभि॑ राहाह॒ र्‌तुभि॒ रिति॑ । \newline
19. ऋ॒तुभि॒ रिती त्यृ॒तुभिर्॑. ऋ॒तुभि॒ रिति॑ च॒तु श्च॒तुरि त्यृ॒तुभिर्॑. ऋ॒तुभि॒ रिति॑ च॒तुः । \newline
20. ऋ॒तुभि॒रित्यृ॒तु - भिः॒ । \newline
21. इति॑ च॒तु श्च॒तु रितीति॑ च॒तुर् द्विर् द्वि श्च॒तु रितीति॑ च॒तुर् द्विः । \newline
22. च॒तुर् द्विर् द्वि श्च॒तु श्च॒तुर् द्विः पुनः॒ पुन॒र् द्वि श्च॒तु श्च॒तुर् द्विः पुनः॑ । \newline
23. द्विः पुनः॒ पुन॒र् द्विर् द्विः पुनर्॑. ऋ॒तुन॒ र्‌तुना॒ पुन॒र् द्विर् द्विः पुनर्॑. ऋ॒तुना᳚ । \newline
24. पुनर्॑. ऋ॒तुन॒ र्‌तुना॒ पुनः॒ पुनर्॑. ऋ॒तुना॑ ऽऽहाह॒ र्‌तुना॒ पुनः॒ पुनर्॑. ऋ॒तुना॑ ऽऽह । \newline
25. ऋ॒तुना॑ ऽऽहाह॒ र्‌तुन॒ र्‌तुना॑ ऽऽहा॒क्रम॑ण मा॒क्रम॑ण माह॒ र्‌तुन॒ र्‌तुना॑ ऽऽहा॒क्रम॑णम् । \newline
26. आ॒हा॒ क्रम॑ण मा॒क्रम॑ण माहाहा॒ क्रम॑ण मे॒वै वाक्रम॑ण माहाहा॒ क्रम॑ण मे॒व । \newline
27. आ॒क्रम॑ण मे॒वैवा क्रम॑ण मा॒क्रम॑ण मे॒व तत् तदे॒वाक्रम॑ण मा॒क्रम॑ण मे॒व तत् । \newline
28. आ॒क्रम॑ण॒मित्या᳚ - क्रम॑णम् । \newline
29. ए॒व तत् तदे॒ वैव तथ् सेतुꣳ॒॒ सेतु॒म् तदे॒ वैव तथ् सेतु᳚म् । \newline
30. तथ् सेतुꣳ॒॒ सेतु॒म् तत् तथ् सेतुं॒ ॅयज॑मानो॒ यज॑मानः॒ सेतु॒म् तत् तथ् सेतुं॒ ॅयज॑मानः । \newline
31. सेतुं॒ ॅयज॑मानो॒ यज॑मानः॒ सेतुꣳ॒॒ सेतुं॒ ॅयज॑मानः कुरुते कुरुते॒ यज॑मानः॒ सेतुꣳ॒॒ सेतुं॒ ॅयज॑मानः कुरुते । \newline
32. यज॑मानः कुरुते कुरुते॒ यज॑मानो॒ यज॑मानः कुरुते सुव॒र्गस्य॑ सुव॒र्गस्य॑ कुरुते॒ यज॑मानो॒ यज॑मानः कुरुते सुव॒र्गस्य॑ । \newline
33. कु॒रु॒ते॒ सु॒व॒र्गस्य॑ सुव॒र्गस्य॑ कुरुते कुरुते सुव॒र्गस्य॑ लो॒कस्य॑ लो॒कस्य॑ सुव॒र्गस्य॑ कुरुते कुरुते सुव॒र्गस्य॑ लो॒कस्य॑ । \newline
34. सु॒व॒र्गस्य॑ लो॒कस्य॑ लो॒कस्य॑ सुव॒र्गस्य॑ सुव॒र्गस्य॑ लो॒कस्य॒ सम॑ष्ट्यै॒ सम॑ष्ट्यै लो॒कस्य॑ सुव॒र्गस्य॑ सुव॒र्गस्य॑ लो॒कस्य॒ सम॑ष्ट्यै । \newline
35. सु॒व॒र्गस्येति॑ सुवः - गस्य॑ । \newline
36. लो॒कस्य॒ सम॑ष्ट्यै॒ सम॑ष्ट्यै लो॒कस्य॑ लो॒कस्य॒ सम॑ष्ट्यै॒ न न सम॑ष्ट्यै लो॒कस्य॑ लो॒कस्य॒ सम॑ष्ट्यै॒ न । \newline
37. सम॑ष्ट्यै॒ न न सम॑ष्ट्यै॒ सम॑ष्ट्यै॒ नान्यो᳚ ऽन्यो न सम॑ष्ट्यै॒ सम॑ष्ट्यै॒ नान्यः । \newline
38. सम॑ष्ट्या॒ इति॒ सं - अ॒ष्ट्यै॒ । \newline
39. नान्यो᳚ ऽन्यो न नान्यो᳚ ऽन्य म॒न्य म॒न्यो न नान्यो᳚ ऽन्यम् । \newline
40. अ॒न्यो᳚ ऽन्य म॒न्य म॒न्यो᳚(1॒) ऽन्यो᳚ ऽन्य मन् वन् व॒न्य म॒न्यो᳚(1॒) ऽन्यो᳚ ऽन्य मनु॑ । \newline
41. अ॒न्य मन् वन् व॒न्य म॒न्य मनु॒ प्र प्राण् व॒न्य म॒न्य मनु॒ प्र । \newline
42. अनु॒ प्र प्राण्वनु॒ प्र प॑द्येत पद्येत॒ प्राण्वनु॒ प्र प॑द्येत । \newline
43. प्र प॑द्येत पद्येत॒ प्र प्र प॑द्येत॒ यद् यत् प॑द्येत॒ प्र प्र प॑द्येत॒ यत् । \newline
44. प॒द्ये॒त॒ यद् यत् प॑द्येत पद्येत॒ यद॒न्यो᳚ ऽन्यो यत् प॑द्येत पद्येत॒ यद॒न्यः । \newline
45. यद॒न्यो᳚ ऽन्यो यद् यद॒न्यो᳚ ऽन्य म॒न्य म॒न्यो यद् यद॒न्यो᳚ ऽन्यम् । \newline
46. अ॒न्यो᳚ ऽन्य म॒न्य म॒न्यो᳚(1॒) ऽन्यो᳚ ऽन्य म॑नुप्र॒पद्ये॑ता नुप्र॒पद्ये॑ता॒न्य म॒न्यो᳚(1॒) ऽन्यो᳚ ऽन्य म॑नुप्र॒पद्ये॑त । \newline
47. अ॒न्य म॑नुप्र॒पद्ये॑ता नुप्र॒पद्ये॑ता॒न्य म॒न्य म॑नुप्र॒पद्ये॑त॒ र्‌तुर्. ऋ॒तु र॑नुप्र॒पद्ये॑ता॒न्य म॒न्य म॑नुप्र॒पद्ये॑त॒ र्‌तुः । \newline
48. अ॒नु॒प्र॒पद्ये॑त॒ र्‌तुर्. ऋ॒तु र॑नुप्र॒पद्ये॑ता नुप्र॒पद्ये॑त॒ र्‌तुर्. ऋ॒तु मृ॒तु मृ॒तु र॑नुप्र॒पद्ये॑ता नुप्र॒पद्ये॑त॒ र्‌तुर्. ऋ॒तुम् । \newline
49. अ॒नु॒प्र॒पद्ये॒तेत्य॑नु - प्र॒पद्ये॑त । \newline
50. ऋ॒तुर्. ऋ॒तु मृ॒तु मृ॒तुर्. ऋ॒तुर्. ऋ॒तु मन् वन् वृ॒तु मृ॒तुर्. ऋ॒तुर्. ऋ॒तु मनु॑ । \newline
51. ऋ॒तु मन् वन् वृ॒तु मृ॒तु मनु॒ प्र प्राण्वृ॒तु मृ॒तु मनु॒ प्र । \newline
52. अनु॒ प्र प्राण्वनु॒ प्र प॑द्येत पद्येत॒ प्राण्वनु॒ प्र प॑द्येत । \newline
53. प्र प॑द्येत पद्येत॒ प्र प्र प॑द्येत॒ र्‌तव॑ ऋ॒तवः॑ पद्येत॒ प्र प्र प॑द्येत॒ र्‌तवः॑ । \newline
54. प॒द्ये॒त॒ र्‌तव॑ ऋ॒तवः॑ पद्येत पद्येत॒ र्‌तवो॒ मोहु॑का॒ मोहु॑का ऋ॒तवः॑ पद्येत पद्येत॒ र्‌तवो॒ मोहु॑काः । \newline
55. ऋ॒तवो॒ मोहु॑का॒ मोहु॑का ऋ॒तव॑ ऋ॒तवो॒ मोहु॑काः स्युः स्यु॒र् मोहु॑का ऋ॒तव॑ ऋ॒तवो॒ मोहु॑काः स्युः । \newline
56. मोहु॑काः स्युः स्यु॒र् मोहु॑का॒ मोहु॑काः स्युः॒ प्रसि॑द्ध॒म् प्रसि॑द्धꣳ स्यु॒र् मोहु॑का॒ मोहु॑काः स्युः॒ प्रसि॑द्धम् । \newline
57. स्युः॒ प्रसि॑द्ध॒म् प्रसि॑द्धꣳ स्युः स्युः॒ प्रसि॑द्ध मे॒वैव प्रसि॑द्धꣳ स्युः स्युः॒ प्रसि॑द्ध मे॒व । \newline
\pagebreak
\markright{ TS 6.5.3.4  \hfill https://www.vedavms.in \hfill}

\section{ TS 6.5.3.4 }

\textbf{TS 6.5.3.4 } \newline
\textbf{Samhita Paata} \newline

प्रसि॑द्धमे॒वा-द्ध्व॒र्युर्दक्षि॑णेन॒ प्रप॑द्यते॒ प्रसि॑द्धं प्रतिप्रस्था॒तोत्त॑रेण॒ तस्मा॑दादि॒त्यः षण्मा॒सो दक्षि॑णेनैति॒ षडुत्त॑रेणो-पया॒मगृ॑हीतोऽसि सꣳ॒॒ सर्पो᳚ऽस्यꣳहस्प॒त्याय॒ त्वेत्या॒हास्ति॑ त्रयोद॒शो मास॒ इत्या॑हु॒स्तमे॒व तत् प्री॑णाति ॥ \newline

\textbf{Pada Paata} \newline

प्रसि॑द्ध॒मिति॒ प्र - सि॒द्ध॒म् । ए॒व । अ॒द्ध्व॒र्युः । दक्षि॑णेन । प्रेति॑ । प॒द्य॒ते॒ । प्रसि॑द्ध॒मिति॒ प्र - सि॒द्ध॒म् । प्र॒ति॒प्र॒स्था॒तेति॑ प्रति-प्र॒स्था॒ता । उत्त॑रे॒णेत्युत्-त॒रे॒ण॒ । तस्मा᳚त् । आ॒दि॒त्यः । षट् । मा॒सः । दक्षि॑णेन । ए॒ति॒ । षट् । उत्त॑रे॒णेत्युत्-त॒रे॒ण॒ । उ॒प॒या॒मगृ॑हीत॒ इत्यु॑पया॒म-गृ॒ही॒तः॒ । अ॒सि॒ । सꣳ॒॒सर्प॒ इति॑ सं - सर्पः॑ । अ॒सि॒ । अꣳ॒॒ह॒स्प॒त्यायेत्यꣳ॑हः - प॒त्याय॑ । त्वा॒ । इति॑ । आ॒ह॒ । अस्ति॑ । त्र॒यो॒द॒श इति॑ त्रयः - द॒शः । मासः॑ । इति॑ । आ॒हुः॒ । तम् । ए॒व । तत् । प्री॒णा॒ति॒ ॥  \newline


\textbf{Krama Paata} \newline

प्रसि॑द्धमे॒व । प्रसि॑द्ध॒मिति॒ प्र - सि॒द्ध॒म् । ए॒वाद्ध्व॒र्युः । अ॒द्ध्व॒र्युर् दक्षि॑णेन । दक्षि॑णेन॒ प्र । प्र प॑द्यते । प॒द्य॒ते॒ प्रसि॑द्धम् । प्रसि॑द्धम् प्रतिप्रस्था॒ता । प्रसि॑द्ध॒मिति॒ प्र - सि॒द्ध॒म् । प्र॒ति॒प्र॒स्था॒तोत्त॑रेण । प्र॒ति॒प्र॒स्था॒तेति॑ प्रति - प्र॒स्था॒ता । उत्त॑रेण॒ तस्मा᳚त् । उत्त॑रे॒णेत्युत् - त॒रे॒ण॒ । तस्मा॑दादि॒त्यः । आ॒दि॒त्यः षट् । षण्मा॒सः । मा॒सो दक्षि॑णेन । दक्षि॑णेनैति । ए॒ति॒ षट् । षडुत्त॑रेण । उत्त॑रेणोपया॒मगृ॑हीतः । उत्त॑रे॒णेत्युत् - त॒रे॒ण॒ । उ॒प॒या॒मगृ॑हीतोऽसि । उ॒प॒या॒मगृ॑हीत॒ इत्यु॑पया॒म - गृ॒ही॒तः॒ । अ॒सि॒ सꣳ॒॒सर्पः॑ । सꣳ॒॒सर्पो॑ऽसि । सꣳ॒॒सर्प॒ इति॑ सम् - सर्पः॑ । अ॒स्यꣳ॒॒ह॒स्प॒त्याय॑ । अꣳ॒॒ह॒स्प॒त्याय॑ त्वा । अꣳ॒॒ह॒स्प॒त्यायेत्यꣳ॑हः - प॒त्याय॑ । त्वेति॑ । इत्या॑ह । आ॒हास्ति॑ । अस्ति॑ त्रयोद॒शः । त्र॒यो॒द॒शो मासः॑ । त्र॒यो॒द॒श इति॑ त्रयः - द॒शः । मास॒ इति॑ । इत्या॑हुः । आ॒हु॒स्तम् । तमे॒व । ए॒व तत् । तत् प्री॑णाति । प्री॒णा॒तीति॑ प्रीणाति । \newline

\textbf{Jatai Paata} \newline

1. प्रसि॑द्ध मे॒वैव प्रसि॑द्ध॒म् प्रसि॑द्ध मे॒व । \newline
2. प्रसि॑द्ध॒मिति॒ प्र - सि॒द्ध॒म् । \newline
3. ए॒वाद्ध्व॒र्यु र॑द्ध्व॒र्यु रे॒वै वाद्ध्व॒र्युः । \newline
4. अ॒द्ध्व॒र्युर् दक्षि॑णेन॒ दक्षि॑णे नाद्ध्व॒र्यु र॑द्ध्व॒र्युर् दक्षि॑णेन । \newline
5. दक्षि॑णेन॒ प्र प्र दक्षि॑णेन॒ दक्षि॑णेन॒ प्र । \newline
6. प्र प॑द्यते पद्यते॒ प्र प्र प॑द्यते । \newline
7. प॒द्य॒ते॒ प्रसि॑द्ध॒म् प्रसि॑द्धम् पद्यते पद्यते॒ प्रसि॑द्धम् । \newline
8. प्रसि॑द्धम् प्रतिप्रस्था॒ता प्र॑तिप्रस्था॒ता प्रसि॑द्ध॒म् प्रसि॑द्धम् प्रतिप्रस्था॒ता । \newline
9. प्रसि॑द्ध॒मिति॒ प्र - सि॒द्ध॒म् । \newline
10. प्र॒ति॒प्र॒स्था॒ तोत्त॑रे॒ णोत्त॑रेण प्रतिप्रस्था॒ता प्र॑तिप्रस्था॒ तोत्त॑रेण । \newline
11. प्र॒ति॒प्र॒स्था॒तेति॑ प्रति - प्र॒स्था॒ता । \newline
12. उत्त॑रेण॒ तस्मा॒त् तस्मा॒ दुत्त॑रे॒ णोत्त॑रेण॒ तस्मा᳚त् । \newline
13. उत्त॑रे॒णेत्युत् - त॒रे॒ण॒ । \newline
14. तस्मा॑ दादि॒त्य आ॑दि॒त्य स्तस्मा॒त् तस्मा॑ दादि॒त्यः । \newline
15. आ॒दि॒त्य ष्षट् थ्षडा॑दि॒त्य आ॑दि॒त्य ष्षट् । \newline
16. षण् मा॒सो मा॒स ष्षट् थ्षण् मा॒सः । \newline
17. मा॒सो दक्षि॑णेन॒ दक्षि॑णेन मा॒सो मा॒सो दक्षि॑णेन । \newline
18. दक्षि॑णे नैत्येति॒ दक्षि॑णेन॒ दक्षि॑णेनैति । \newline
19. ए॒ति॒ षट् थ्षडे᳚ त्येति॒ षट् । \newline
20. षडुत्त॑रे॒ णोत्त॑रेण॒ षट् थ्ष डुत्त॑रेण । \newline
21. उत्त॑रे णोपया॒मगृ॑हीत उपया॒मगृ॑हीत॒ उत्त॑रे॒ णोत्त॑रे णोपया॒मगृ॑हीतः । \newline
22. उत्त॑रे॒णेत्युत् - त॒रे॒ण॒ । \newline
23. उ॒प॒या॒मगृ॑हीतो ऽस्य स्युपया॒मगृ॑हीत उपया॒मगृ॑हीतो ऽसि । \newline
24. उ॒प॒या॒मगृ॑हीत॒ इत्यु॑पया॒म - गृ॒ही॒तः॒ । \newline
25. अ॒सि॒ सꣳ॒॒सर्पः॑ सꣳ॒॒सर्पो᳚ ऽस्यसि सꣳ॒॒सर्पः॑ । \newline
26. सꣳ॒॒सर्पो᳚ ऽस्यसि सꣳ॒॒सर्पः॑ सꣳ॒॒सर्पो॑ ऽसि । \newline
27. सꣳ॒॒सर्प॒ इति॑ सं - सर्पः॑ । \newline
28. अ॒स्यꣳ॒॒ह॒स्प॒त्या याꣳ॑हस्प॒त्याया᳚ स्य स्यꣳहस्प॒त्याय॑ । \newline
29. अꣳ॒॒ह॒स्प॒त्याय॑ त्वा त्वा ऽꣳहस्प॒त्या याꣳ॑हस्प॒त्याय॑ त्वा । \newline
30. अꣳ॒॒ह॒स्प॒त्यायेत्यꣳ॑हः - प॒त्याय॑ । \newline
31. त्वेतीति॑ त्वा॒ त्वेति॑ । \newline
32. इत्या॑हा॒हे तीत्या॑ह । \newline
33. आ॒हा स्त्यस्त्या॑ हा॒हास्ति॑ । \newline
34. अस्ति॑ त्रयोद॒श स्त्र॑योद॒शो ऽस्त्यस्ति॑ त्रयोद॒शः । \newline
35. त्र॒यो॒द॒शो मासो॒ मास॑ स्त्रयोद॒श स्त्र॑योद॒शो मासः॑ । \newline
36. त्र॒यो॒द॒श इति॑ त्रयः - द॒शः । \newline
37. मास॒ इतीति॒ मासो॒ मास॒ इति॑ । \newline
38. इत्या॑हु राहु॒ रिती त्या॑हुः । \newline
39. आ॒हु॒ स्तम् तमा॑हु राहु॒ स्तम् । \newline
40. तमे॒वैव तम् तमे॒व । \newline
41. ए॒व तत् तदे॒ वैव तत् । \newline
42. तत् प्री॑णाति प्रीणाति॒ तत् तत् प्री॑णाति । \newline
43. प्री॒णा॒तीति॑ प्रीणाति । \newline

\textbf{Ghana Paata } \newline

1. प्रसि॑द्ध मे॒वैव प्रसि॑द्ध॒म् प्रसि॑द्ध मे॒वाद्ध्व॒र्यु र॑द्ध्व॒र्यु रे॒व प्रसि॑द्ध॒म् प्रसि॑द्ध मे॒वाद्ध्व॒र्युः । \newline
2. प्रसि॑द्ध॒मिति॒ प्र - सि॒द्ध॒म् । \newline
3. ए॒वाद्ध्व॒र्यु र॑द्ध्व॒र्यु रे॒वै वाद्ध्व॒र्युर् दक्षि॑णेन॒ दक्षि॑णे नाद्ध्व॒र्यु रे॒वै वाद्ध्व॒र्युर् दक्षि॑णेन । \newline
4. अ॒द्ध्व॒र्युर् दक्षि॑णेन॒ दक्षि॑णेना द्ध्व॒र्यु र॑द्ध्व॒र्युर् दक्षि॑णेन॒ प्र प्र दक्षि॑णेना द्ध्व॒र्यु र॑द्ध्व॒र्युर् दक्षि॑णेन॒ प्र । \newline
5. दक्षि॑णेन॒ प्र प्र दक्षि॑णेन॒ दक्षि॑णेन॒ प्र प॑द्यते पद्यते॒ प्र दक्षि॑णेन॒ दक्षि॑णेन॒ प्र प॑द्यते । \newline
6. प्र प॑द्यते पद्यते॒ प्र प्र प॑द्यते॒ प्रसि॑द्ध॒म् प्रसि॑द्धम् पद्यते॒ प्र प्र प॑द्यते॒ प्रसि॑द्धम् । \newline
7. प॒द्य॒ते॒ प्रसि॑द्ध॒म् प्रसि॑द्धम् पद्यते पद्यते॒ प्रसि॑द्धम् प्रतिप्रस्था॒ता प्र॑तिप्रस्था॒ता प्रसि॑द्धम् पद्यते पद्यते॒ प्रसि॑द्धम् प्रतिप्रस्था॒ता । \newline
8. प्रसि॑द्धम् प्रतिप्रस्था॒ता प्र॑तिप्रस्था॒ता प्रसि॑द्ध॒म् प्रसि॑द्धम् प्रतिप्रस्था॒ तोत्त॑रे॒ णोत्त॑रेण प्रतिप्रस्था॒ता प्रसि॑द्ध॒म् प्रसि॑द्धम् प्रतिप्रस्था॒ तोत्त॑रेण । \newline
9. प्रसि॑द्ध॒मिति॒ प्र - सि॒द्ध॒म् । \newline
10. प्र॒ति॒प्र॒स्था॒ तोत्त॑रे॒ णोत्त॑रेण प्रतिप्रस्था॒ता प्र॑तिप्रस्था॒ तोत्त॑रेण॒ तस्मा॒त् तस्मा॒ दुत्त॑रेण प्रतिप्रस्था॒ता प्र॑तिप्रस्था॒ तोत्त॑रेण॒ तस्मा᳚त् । \newline
11. प्र॒ति॒प्र॒स्था॒तेति॑ प्रति - प्र॒स्था॒ता । \newline
12. उत्त॑रेण॒ तस्मा॒त् तस्मा॒ दुत्त॑रे॒ णोत्त॑रेण॒ तस्मा॑ दादि॒त्य आ॑दि॒त्य स्तस्मा॒ दुत्त॑रे॒ णोत्त॑रेण॒ तस्मा॑ दादि॒त्यः । \newline
13. उत्त॑रे॒णेत्युत् - त॒रे॒ण॒ । \newline
14. तस्मा॑ दादि॒त्य आ॑दि॒त्य स्तस्मा॒त् तस्मा॑ दादि॒त्य ष्षट् थ्षडा॑दि॒त्य स्तस्मा॒त् तस्मा॑ दादि॒त्य ष्षट् । \newline
15. आ॒दि॒त्य ष्षट् थ्षडा॑दि॒त्य आ॑दि॒त्य ष्षण् मा॒सो मा॒स ष्षडा॑दि॒त्य आ॑दि॒त्य ष्षण् मा॒सः । \newline
16. षण् मा॒सो मा॒स ष्षट् थ्षण् मा॒सो दक्षि॑णेन॒ दक्षि॑णेन मा॒स ष्षट् थ्षण् मा॒सो दक्षि॑णेन । \newline
17. मा॒सो दक्षि॑णेन॒ दक्षि॑णेन मा॒सो मा॒सो दक्षि॑ णेनैत्येति॒ दक्षि॑णेन मा॒सो मा॒सो दक्षि॑ णेनैति । \newline
18. दक्षि॑ णेनैत्येति॒ दक्षि॑णेन॒ दक्षि॑ णेनैति॒ षट् थ्षडे॑ति॒ दक्षि॑णेन॒ दक्षि॑ णेनैति॒ षट् । \newline
19. ए॒ति॒ षट् थ्षडे᳚ त्येति॒ षडुत्त॑रे॒ णोत्त॑रेण॒ षडे᳚त्येति॒ षडुत्त॑रेण । \newline
20. षडुत्त॑रे॒ णोत्त॑रेण॒ षट् थ्ष डुत्त॑रे णोपया॒मगृ॑हीत उपया॒मगृ॑हीत॒ उत्त॑रेण॒ षट् थ्षडुत्त॑रे णोपया॒मगृ॑हीतः । \newline
21. उत्त॑रे णोपया॒मगृ॑हीत उपया॒मगृ॑हीत॒ उत्त॑रे॒ णोत्त॑रे णोपया॒मगृ॑हीतो ऽस्य स्युपया॒मगृ॑हीत॒ उत्त॑रे॒ णोत्त॑रे णोपया॒मगृ॑हीतो ऽसि । \newline
22. उत्त॑रे॒णेत्युत् - त॒रे॒ण॒ । \newline
23. उ॒प॒या॒मगृ॑हीतो ऽस्य स्युपया॒मगृ॑हीत उपया॒मगृ॑हीतो ऽसि सꣳ॒॒सर्पः॑ सꣳ॒॒सर्पो᳚ ऽस्युपया॒मगृ॑हीत उपया॒मगृ॑हीतो ऽसि सꣳ॒॒सर्पः॑ । \newline
24. उ॒प॒या॒मगृ॑हीत॒ इत्यु॑पया॒म - गृ॒ही॒तः॒ । \newline
25. अ॒सि॒ सꣳ॒॒सर्पः॑ सꣳ॒॒सर्पो᳚ ऽस्यसि सꣳ॒॒सर्पो᳚ ऽस्यसि सꣳ॒॒सर्पो᳚ ऽस्यसि सꣳ॒॒सर्पो॑ ऽसि । \newline
26. सꣳ॒॒सर्पो᳚ ऽस्यसि सꣳ॒॒सर्पः॑ सꣳ॒॒सर्पो᳚ ऽस्यꣳहस्प॒त्याया ꣳ॑हस्प॒त्या या॑सि सꣳ॒॒सर्पः॑ सꣳ॒॒सर्पो᳚ ऽस्यꣳहस्प॒त्याय॑ । \newline
27. सꣳ॒॒सर्प॒ इति॑ सं - सर्पः॑ । \newline
28. अ॒स्यꣳ॒॒ह॒स्प॒त्याया ꣳ॑हस्प॒त्या या᳚स्य स्यꣳहस्प॒त्याय॑ त्वा त्वा ऽꣳहस्प॒त्याया᳚स्य स्यꣳहस्प॒त्याय॑ त्वा । \newline
29. अꣳ॒॒ह॒स्प॒त्याय॑ त्वा त्वा ऽꣳहस्प॒त्याया ꣳ॑हस्प॒त्याय॒ त्वेतीति॑ त्वा ऽꣳहस्प॒त्याया ꣳ॑हस्प॒त्याय॒ त्वेति॑ । \newline
30. अꣳ॒॒ह॒स्प॒त्यायेत्यꣳ॑हः - प॒त्याय॑ । \newline
31. त्वेतीति॑ त्वा॒ त्वेत्या॑ हा॒हेति॑ त्वा॒ त्वेत्या॑ह । \newline
32. इत्या॑हा॒हे तीत्या॒ हास्त्य स्त्या॒हे तीत्या॒ हास्ति॑ । \newline
33. आ॒हास्त्य स्त्या॑हा॒ हास्ति॑ त्रयोद॒श स्त्र॑योद॒शो ऽस्त्या॑हा॒ हास्ति॑ त्रयोद॒शः । \newline
34. अस्ति॑ त्रयोद॒श स्त्र॑योद॒शो ऽस्त्यस्ति॑ त्रयोद॒शो मासो॒ मास॑ स्त्रयोद॒शो ऽस्त्यस्ति॑ त्रयोद॒शो मासः॑ । \newline
35. त्र॒यो॒द॒शो मासो॒ मास॑ स्त्रयोद॒श स्त्र॑योद॒शो मास॒ इतीति॒ मास॑ स्त्रयोद॒श स्त्र॑योद॒शो मास॒ इति॑ । \newline
36. त्र॒यो॒द॒श इति॑ त्रयः - द॒शः । \newline
37. मास॒ इतीति॒ मासो॒ मास॒ इत्या॑हु राहु॒ रिति॒ मासो॒ मास॒ इत्या॑हुः । \newline
38. इत्या॑हु राहु॒ रितीत्या॑हु॒ स्तम् त मा॑हु॒ रिती त्या॑हु॒ स्तम् । \newline
39. आ॒हु॒ स्तम् त मा॑हु राहु॒ स्त मे॒वैव त मा॑हु राहु॒ स्त मे॒व । \newline
40. त मे॒वैव तम् त मे॒व तत् तदे॒व तम् त मे॒व तत् । \newline
41. ए॒व तत् तदे॒ वैव तत् प्री॑णाति प्रीणाति॒ तदे॒ वैव तत् प्री॑णाति । \newline
42. तत् प्री॑णाति प्रीणाति॒ तत् तत् प्री॑णाति । \newline
43. प्री॒णा॒तीति॑ प्रीणाति । \newline
\pagebreak
\markright{ TS 6.5.4.1  \hfill https://www.vedavms.in \hfill}

\section{ TS 6.5.4.1 }

\textbf{TS 6.5.4.1 } \newline
\textbf{Samhita Paata} \newline

सु॒व॒र्गाय॒ वा ए॒ते लो॒काय॑ गृह्यन्ते॒ यदृ॑तुग्र॒हा ज्योति॑रिन्द्रा॒ग्नी यदै᳚न्द्रा॒ग्नमृ॑तुपा॒त्रेण॑ गृ॒ह्णाति॒ ज्योति॑रे॒वास्मा॑ उ॒परि॑ष्टाद्-दधाति सुव॒र्गस्य॑ लो॒कस्यानु॑ख्यात्या ओजो॒भृतौ॒ वा ए॒तौ दे॒वानां॒ ॅयदि॑न्द्रा॒ग्नी यदै᳚न्द्रा॒ग्नो गृ॒ह्यत॒ ओज॑ ए॒वाव॑ रुन्धे वैश्वदे॒वꣳ शु॑क्रपा॒त्रेण॑ गृह्णाति वैश्वदे॒व्यो॑ वै प्र॒जा अ॒सावा॑दि॒त्यः शु॒क्रो यद्-वै᳚श्वदे॒वꣳशु॑क्रपा॒त्रेण॑ गृ॒ह्णाति॒ तस्मा॑द॒सावा॑दि॒त्यः- [  ] \newline

\textbf{Pada Paata} \newline

सु॒व॒र्गायेति॑ सुवः - गाय॑ । वै । ए॒ते । लो॒काय॑ । गृ॒ह्य॒न्ते॒ । यत् । ऋ॒तु॒ग्र॒हा इत्यृ॑तु - ग्र॒हाः । ज्योतिः॑ । इ॒न्द्रा॒ग्नी इती᳚न्द्र - अ॒ग्नी । यत् । ऐ॒न्द्रा॒ग्नमित्यै᳚न्द्र - अ॒ग्नम् । ऋ॒तु॒पा॒त्रेणेत्यृ॑तु - पा॒त्रेण॑ । गृ॒ह्णाति॑ । ज्योतिः॑ । ए॒व । अ॒स्मै॒ । उ॒परि॑ष्टात् । द॒धा॒ति॒ । सु॒व॒र्गसेति॑ सुवः-गस्य॑ । लो॒कस्य॑ । अनु॑ख्यात्या॒ इत्यनु॑ - ख्या॒त्यै॒ । ओ॒जो॒भृता॒वित्यो॑जः - भृतौ᳚ । वै । ए॒तौ । दे॒वाना᳚म् । यत् । इ॒न्द्रा॒ग्नी इती᳚न्द्र - अ॒ग्नी । यत् । ऐ॒न्द्रा॒ग्न इत्यै᳚न्द्र-अ॒ग्नः । गृ॒ह्यते᳚ । ओजः॑ । ए॒व । अवेति॑ । रु॒न्धे॒ । वै॒श्व॒दे॒वमिति॑ वैश्व - दे॒वम् । शु॒क्र॒पा॒त्रेणेति॑ शुक्र - पा॒त्रेण॑ । गृ॒ह्णा॒ति॒ । वै॒श्व॒दे॒व्य॑ इति॑ वैश्व - दे॒व्यः॑ । वै । प्र॒जा इति॑ प्र - जाः । अ॒सौ । आ॒दि॒त्यः । शु॒क्रः । यत् । वै॒श्व॒दे॒वमिति॑ वैश्व - दे॒वम् । शु॒क्र॒पा॒त्रेणेति॑ शुक्र - पा॒त्रेण॑ । गृ॒ह्णाति॑ । तस्मा᳚त् । अ॒सौ । आ॒दि॒त्यः ।  \newline


\textbf{Krama Paata} \newline

सु॒व॒र्गाय॒ वै । सु॒व॒र्गायेति॑ सुवः - गाय॑ । वा ए॒ते । ए॒ते लो॒काय॑ । लो॒काय॑ गृह्यन्ते । गृ॒ह्य॒न्ते॒ यत् । यदृ॑तुग्र॒हाः । ऋ॒तु॒ग्र॒हा ज्योतिः॑ । ऋ॒तु॒ग्र॒हा इत्यृ॑तु - ग्र॒हाः । ज्योति॑रिन्द्रा॒ग्नी । इ॒न्द्रा॒ग्नी यत् । इ॒न्द्रा॒ग्नी इती᳚न्द्र - अ॒ग्नी । यदै᳚न्द्रा॒ग्नम् । ऐ॒न्द्रा॒ग्न,मृ॑तुपा॒त्रेण॑ । ऐ॒न्द्रा॒ग्नमित्यै᳚न्द्र - अ॒ग्नम् । ऋ॒तु॒पा॒त्रेण॑ गृ॒ह्णाति॑ । ऋ॒तु॒पा॒त्रेणेत्यृ॑तु - पा॒त्रेण॑ । गृ॒ह्णाति॒ ज्योतिः॑ । ज्योति॑रे॒व । ए॒वास्मै᳚ । अ॒स्मा॒ उ॒परि॑ष्टात् । उ॒परि॑ष्टाद् दधाति । द॒धा॒ति॒ सु॒व॒र्गस्य॑ । सु॒व॒र्गस्य॑ लो॒कस्य॑ । सु॒व॒र्गस्येति॑ सुवः - गस्य॑ । लो॒कस्यानु॑ख्यात्यै । अनु॑ख्यात्या ओजो॒भृतौ᳚ । अनु॑ख्यात्या॒ इत्यनु॑ - ख्या॒त्यै॒ । ओ॒जो॒भृतौ॒ वै । ओ॒जो॒भृता॒वित्यो॑जः - भृतौ᳚ । वा ए॒तौ । ए॒तौ दे॒वाना᳚म् । दे॒वाना॒म् ॅयत् । यदि॑न्द्रा॒ग्नी । इ॒न्द्रा॒ग्नी यत् । इ॒न्द्रा॒ग्नी इती᳚न्द्र - अ॒ग्नी । यदै᳚न्द्रा॒ग्नः । ऐ॒न्द्रा॒ग्नो गृ॒ह्यते᳚ । ऐ॒न्द्रा॒ग्न इत्यै᳚न्द्र - अ॒ग्नः । गृ॒ह्यत॒ ओजः॑ । ओज॑ ए॒व । ए॒वाव॑ । अव॑ रुन्धे । रु॒न्धे॒ वै॒श्व॒दे॒वम् । वै॒श्व॒दे॒वꣳ शु॑क्रपा॒त्रेण॑ । वै॒श्व॒दे॒वमिति॑ वैश्व - दे॒वम् । शु॒क्र॒पा॒त्रेण॑ गृह्णाति । शु॒क्र॒पा॒त्रेणेति॑ शुक्र - पा॒त्रेण॑ । गृ॒ह्णा॒ति॒ वै॒श्व॒दे॒व्यः॑ । वै॒श्व॒दे॒व्यो॑ वै । वै॒श्व॒दे॒व्य॑ इति॑ वैश्व - दे॒व्यः॑ । वै प्र॒जाः । प्र॒जा अ॒सौ । प्र॒जा इति॑ प्र - जाः । अ॒सावा॑दि॒त्यः । आ॒दि॒त्यः शु॒क्रः । शु॒क्रो यत् । यद् वै᳚श्वदे॒वम् । वै॒श्व॒दे॒वꣳ शु॑क्रपा॒त्रेण॑ । वै॒श्व॒दे॒वमिति॑ वैश्व - दे॒वम् । शु॒क्र॒पा॒त्रेण॑ गृ॒ह्णाति॑ । शु॒क्र॒पा॒त्रेणेति॑ शुक्र - पा॒त्रेण॑ । गृ॒ह्णाति॒ तस्मा᳚त् । तस्मा॑द॒सौ । अ॒सावा॑दि॒त्यः ( ) । आ॒दि॒त्यः सर्वाः᳚ \newline

\textbf{Jatai Paata} \newline

1. सु॒व॒र्गाय॒ वै वै सु॑व॒र्गाय॑ सुव॒र्गाय॒ वै । \newline
2. सु॒व॒र्गायेति॑ सुवः - गाय॑ । \newline
3. वा ए॒त ए॒ते वै वा ए॒ते । \newline
4. ए॒ते लो॒काय॑ लो॒कायै॒त ए॒ते लो॒काय॑ । \newline
5. लो॒काय॑ गृह्यन्ते गृह्यन्ते लो॒काय॑ लो॒काय॑ गृह्यन्ते । \newline
6. गृ॒ह्य॒न्ते॒ यद् यद् गृ॑ह्यन्ते गृह्यन्ते॒ यत् । \newline
7. यदृ॑तुग्र॒हा ऋ॑तुग्र॒हा यद् यदृ॑तुग्र॒हाः । \newline
8. ऋ॒तु॒ग्र॒हा ज्योति॒र् ज्योतिर्॑. ऋतुग्र॒हा ऋ॑तुग्र॒हा ज्योतिः॑ । \newline
9. ऋ॒तु॒ग्र॒हा इत्यृ॑तु - ग्र॒हाः । \newline
10. ज्योति॑ रिन्द्रा॒ग्नी इ॑न्द्रा॒ग्नी ज्योति॒र् ज्योति॑ रिन्द्रा॒ग्नी । \newline
11. इ॒न्द्रा॒ग्नी यद् यदि॑न्द्रा॒ग्नी इ॑न्द्रा॒ग्नी यत् । \newline
12. इ॒न्द्रा॒ग्नी इती᳚न्द्र - अ॒ग्नी । \newline
13. यदै᳚न्द्रा॒ग्न मै᳚न्द्रा॒ग्नं ॅयद् यदै᳚न्द्रा॒ग्नम् । \newline
14. ऐ॒न्द्रा॒ग्न मृ॑तुपा॒त्रेण॑ र्‌तुपा॒त्रेणै᳚ न्द्रा॒ग्न मै᳚न्द्रा॒ग्न मृ॑तुपा॒त्रेण॑ । \newline
15. ऐ॒न्द्रा॒ग्नमित्यै᳚न्द्र - अ॒ग्नम् । \newline
16. ऋ॒तु॒पा॒त्रेण॑ गृ॒ह्णाति॑ गृ॒ह्णा त्यृ॑तुपा॒त्रेण॑ र्‌तुपा॒त्रेण॑ गृ॒ह्णाति॑ । \newline
17. ऋ॒तु॒पा॒त्रेणेत्यृ॑तु - पा॒त्रेण॑ । \newline
18. गृ॒ह्णाति॒ ज्योति॒र् ज्योति॑र् गृ॒ह्णाति॑ गृ॒ह्णाति॒ ज्योतिः॑ । \newline
19. ज्योति॑ रे॒वैव ज्योति॒र् ज्योति॑ रे॒व । \newline
20. ए॒वास्मा॑ अस्मा ए॒वै वास्मै᳚ । \newline
21. अ॒स्मा॒ उ॒परि॑ष्टा दु॒परि॑ष्टा दस्मा अस्मा उ॒परि॑ष्टात् । \newline
22. उ॒परि॑ष्टाद् दधाति दधा त्यु॒परि॑ष्टा दु॒परि॑ष्टाद् दधाति । \newline
23. द॒धा॒ति॒ सु॒व॒र्गस्य॑ सुव॒र्गस्य॑ दधाति दधाति सुव॒र्गस्य॑ । \newline
24. सु॒व॒र्गस्य॑ लो॒कस्य॑ लो॒कस्य॑ सुव॒र्गस्य॑ सुव॒र्गस्य॑ लो॒कस्य॑ । \newline
25. सु॒व॒र्गसेति॑ सुवः - गस्य॑ । \newline
26. लो॒कस्या नु॑ख्यात्या॒ अनु॑ख्यात्यै लो॒कस्य॑ लो॒कस्या नु॑ख्यात्यै । \newline
27. अनु॑ख्यात्या ओजो॒भृता॑ वोजो॒भृता॒ वनु॑ख्यात्या॒ अनु॑ख्यात्या ओजो॒भृतौ᳚ । \newline
28. अनु॑ख्यात्या॒ इत्यनु॑ - ख्या॒त्यै॒ । \newline
29. ओ॒जो॒भृतौ॒ वै वा ओ॑जो॒भृता॑ वोजो॒भृतौ॒ वै । \newline
30. ओ॒जो॒भृता॒वित्यो॑जः - भृतौ᳚ । \newline
31. वा ए॒ता वे॒तौ वै वा ए॒तौ । \newline
32. ए॒तौ दे॒वाना᳚म् दे॒वाना॑ मे॒ता वे॒तौ दे॒वाना᳚म् । \newline
33. दे॒वानां॒ ॅयद् यद् दे॒वाना᳚म् दे॒वानां॒ ॅयत् । \newline
34. यदि॑न्द्रा॒ग्नी इ॑न्द्रा॒ग्नी यद् यदि॑न्द्रा॒ग्नी । \newline
35. इ॒न्द्रा॒ग्नी यद् यदि॑न्द्रा॒ग्नी इ॑न्द्रा॒ग्नी यत् । \newline
36. इ॒न्द्रा॒ग्नी इती᳚न्द्र - अ॒ग्नी । \newline
37. यदै᳚न्द्रा॒ग्न ऐ᳚न्द्रा॒ग्नो यद् यदै᳚न्द्रा॒ग्नः । \newline
38. ऐ॒न्द्रा॒ग्नो गृ॒ह्यते॑ गृ॒ह्यत॑ ऐन्द्रा॒ग्न ऐ᳚न्द्रा॒ग्नो गृ॒ह्यते᳚ । \newline
39. ऐ॒न्द्रा॒ग्न इत्यै᳚न्द्र - अ॒ग्नः । \newline
40. गृ॒ह्यत॒ ओज॒ ओजो॑ गृ॒ह्यते॑ गृ॒ह्यत॒ ओजः॑ । \newline
41. ओज॑ ए॒वै वौज॒ ओज॑ ए॒व । \newline
42. ए॒वावा वै॒वै वाव॑ । \newline
43. अव॑ रुन्धे रु॒न्धे ऽवाव॑ रुन्धे । \newline
44. रु॒न्धे॒ वै॒श्व॒दे॒वं ॅवै᳚श्वदे॒वꣳ रु॑न्धे रुन्धे वैश्वदे॒वम् । \newline
45. वै॒श्व॒दे॒वꣳ शु॑क्रपा॒त्रेण॑ शुक्रपा॒त्रेण॑ वैश्वदे॒वं ॅवै᳚श्वदे॒वꣳ शु॑क्रपा॒त्रेण॑ । \newline
46. वै॒श्व॒दे॒वमिति॑ वैश्व - दे॒वम् । \newline
47. शु॒क्र॒पा॒त्रेण॑ गृह्णाति गृह्णाति शुक्रपा॒त्रेण॑ शुक्रपा॒त्रेण॑ गृह्णाति । \newline
48. शु॒क्र॒पा॒त्रेणेति॑ शुक्र - पा॒त्रेण॑ । \newline
49. गृ॒ह्णा॒ति॒ वै॒श्व॒दे॒व्यो॑ वैश्वदे॒व्यो॑ गृह्णाति गृह्णाति वैश्वदे॒व्यः॑ । \newline
50. वै॒श्व॒दे॒व्यो॑ वै वै वै᳚श्वदे॒व्यो॑ वैश्वदे॒व्यो॑ वै । \newline
51. वै॒श्व॒दे॒व्य॑ इति॑ वैश्व - दे॒व्यः॑ । \newline
52. वै प्र॒जाः प्र॒जा वै वै प्र॒जाः । \newline
53. प्र॒जा अ॒सा व॒सौ प्र॒जाः प्र॒जा अ॒सौ । \newline
54. प्र॒जा इति॑ प्र - जाः । \newline
55. अ॒सा वा॑दि॒त्य आ॑दि॒त्यो॑ ऽसा व॒सा वा॑दि॒त्यः । \newline
56. आ॒दि॒त्यः शु॒क्रः शु॒क्र आ॑दि॒त्य आ॑दि॒त्यः शु॒क्रः । \newline
57. शु॒क्रो यद् यच्छु॒क्रः शु॒क्रो यत् । \newline
58. यद् वै᳚श्वदे॒वं ॅवै᳚श्वदे॒वं ॅयद् यद् वै᳚श्वदे॒वम् । \newline
59. वै॒श्व॒दे॒वꣳ शु॑क्रपा॒त्रेण॑ शुक्रपा॒त्रेण॑ वैश्वदे॒वं ॅवै᳚श्वदे॒वꣳ शु॑क्रपा॒त्रेण॑ । \newline
60. वै॒श्व॒दे॒वमिति॑ वैश्व - दे॒वम् । \newline
61. शु॒क्र॒पा॒त्रेण॑ गृ॒ह्णाति॑ गृ॒ह्णाति॑ शुक्रपा॒त्रेण॑ शुक्रपा॒त्रेण॑ गृ॒ह्णाति॑ । \newline
62. शु॒क्र॒पा॒त्रेणेति॑ शुक्र - पा॒त्रेण॑ । \newline
63. गृ॒ह्णाति॒ तस्मा॒त् तस्मा᳚द् गृ॒ह्णाति॑ गृ॒ह्णाति॒ तस्मा᳚त् । \newline
64. तस्मा॑ द॒सा व॒सौ तस्मा॒त् तस्मा॑ द॒सौ । \newline
65. अ॒सा वा॑दि॒त्य आ॑दि॒त्यो॑ ऽसा व॒सा वा॑दि॒त्यः । \newline
66. आ॒दि॒त्यः सर्वाः॒ सर्वा॑ आदि॒त्य आ॑दि॒त्यः सर्वाः᳚ । \newline

\textbf{Ghana Paata } \newline

1. सु॒व॒र्गाय॒ वै वै सु॑व॒र्गाय॑ सुव॒र्गाय॒ वा ए॒त ए॒ते वै सु॑व॒र्गाय॑ सुव॒र्गाय॒ वा ए॒ते । \newline
2. सु॒व॒र्गायेति॑ सुवः - गाय॑ । \newline
3. वा ए॒त ए॒ते वै वा ए॒ते लो॒काय॑ लो॒कायै॒ते वै वा ए॒ते लो॒काय॑ । \newline
4. ए॒ते लो॒काय॑ लो॒कायै॒त ए॒ते लो॒काय॑ गृह्यन्ते गृह्यन्ते लो॒कायै॒त ए॒ते लो॒काय॑ गृह्यन्ते । \newline
5. लो॒काय॑ गृह्यन्ते गृह्यन्ते लो॒काय॑ लो॒काय॑ गृह्यन्ते॒ यद् यद् गृ॑ह्यन्ते लो॒काय॑ लो॒काय॑ गृह्यन्ते॒ यत् । \newline
6. गृ॒ह्य॒न्ते॒ यद् यद् गृ॑ह्यन्ते गृह्यन्ते॒ यदृ॑तुग्र॒हा ऋ॑तुग्र॒हा यद् गृ॑ह्यन्ते गृह्यन्ते॒ यदृ॑तुग्र॒हाः । \newline
7. यदृ॑तुग्र॒हा ऋ॑तुग्र॒हा यद् यदृ॑तुग्र॒हा ज्योति॒र् ज्योतिर्॑. ऋतुग्र॒हा यद् यदृ॑तुग्र॒हा ज्योतिः॑ । \newline
8. ऋ॒तु॒ग्र॒हा ज्योति॒र् ज्योतिर्॑. ऋतुग्र॒हा ऋ॑तुग्र॒हा ज्योति॑ रिन्द्रा॒ग्नी इ॑न्द्रा॒ग्नी ज्योतिर्॑. ऋतुग्र॒हा ऋ॑तुग्र॒हा ज्योति॑ रिन्द्रा॒ग्नी । \newline
9. ऋ॒तु॒ग्र॒हा इत्यृ॑तु - ग्र॒हाः । \newline
10. ज्योति॑ रिन्द्रा॒ग्नी इ॑न्द्रा॒ग्नी ज्योति॒र् ज्योति॑ रिन्द्रा॒ग्नी यद् यदि॑न्द्रा॒ग्नी ज्योति॒र् ज्योति॑ रिन्द्रा॒ग्नी यत् । \newline
11. इ॒न्द्रा॒ग्नी यद् यदि॑न्द्रा॒ग्नी इ॑न्द्रा॒ग्नी यदै᳚न्द्रा॒ग्न मै᳚न्द्रा॒ग्नं ॅयदि॑न्द्रा॒ग्नी इ॑न्द्रा॒ग्नी यदै᳚न्द्रा॒ग्नम् । \newline
12. इ॒न्द्रा॒ग्नी इती᳚न्द्र - अ॒ग्नी । \newline
13. यदै᳚न्द्रा॒ग्न मै᳚न्द्रा॒ग्नं ॅयद् यदै᳚न्द्रा॒ग्न मृ॑तुपा॒त्रेण॑ र्‌तुपा॒त्रे णै᳚न्द्रा॒ग्नं ॅयद् 
यदै᳚न्द्रा॒ग्न मृ॑तुपा॒त्रेण॑ । \newline
14. ऐ॒न्द्रा॒ग्न मृ॑तुपा॒त्रेण॑ र्‌तुपा॒त्रे णै᳚न्द्रा॒ग्न मै᳚न्द्रा॒ग्न मृ॑तुपा॒त्रेण॑ गृ॒ह्णाति॑ गृ॒ह्णा त्यृ॑तुपा॒त्रे णै᳚न्द्रा॒ग्न मै᳚न्द्रा॒ग्न मृ॑तुपा॒त्रेण॑ गृ॒ह्णाति॑ । \newline
15. ऐ॒न्द्रा॒ग्नमित्यै᳚न्द्र - अ॒ग्नम् । \newline
16. ऋ॒तु॒पा॒त्रेण॑ गृ॒ह्णाति॑ गृ॒ह्णा त्यृ॑तुपा॒त्रेण॑ र्‌तुपा॒त्रेण॑ गृ॒ह्णाति॒ ज्योति॒र् ज्योति॑र् गृ॒ह्णा त्यृ॑तुपा॒त्रेण॑ र्‌तुपा॒त्रेण॑ गृ॒ह्णाति॒ ज्योतिः॑ । \newline
17. ऋ॒तु॒पा॒त्रेणेत्यृ॑तु - पा॒त्रेण॑ । \newline
18. गृ॒ह्णाति॒ ज्योति॒र् ज्योति॑र् गृ॒ह्णाति॑ गृ॒ह्णाति॒ ज्योति॑ रे॒वैव ज्योति॑र् गृ॒ह्णाति॑ गृ॒ह्णाति॒ ज्योति॑ रे॒व । \newline
19. ज्योति॑ रे॒वैव ज्योति॒र् ज्योति॑ रे॒वास्मा॑ अस्मा ए॒व ज्योति॒र् ज्योति॑ रे॒वास्मै᳚ । \newline
20. ए॒वास्मा॑ अस्मा ए॒वै वास्मा॑ उ॒परि॑ष्टा दु॒परि॑ष्टा दस्मा ए॒वैवास्मा॑ उ॒परि॑ष्टात् । \newline
21. अ॒स्मा॒ उ॒परि॑ष्टा दु॒परि॑ष्टा दस्मा अस्मा उ॒परि॑ष्टाद् दधाति दधा त्यु॒परि॑ष्टा दस्मा अस्मा उ॒परि॑ष्टाद् दधाति । \newline
22. उ॒परि॑ष्टाद् दधाति दधा त्यु॒परि॑ष्टा दु॒परि॑ष्टाद् दधाति सुव॒र्गस्य॑ सुव॒र्गस्य॑ दधा त्यु॒परि॑ष्टा दु॒परि॑ष्टाद् दधाति सुव॒र्गस्य॑ । \newline
23. द॒धा॒ति॒ सु॒व॒र्गस्य॑ सुव॒र्गस्य॑ दधाति दधाति सुव॒र्गस्य॑ लो॒कस्य॑ लो॒कस्य॑ सुव॒र्गस्य॑ दधाति दधाति सुव॒र्गस्य॑ लो॒कस्य॑ । \newline
24. सु॒व॒र्गस्य॑ लो॒कस्य॑ लो॒कस्य॑ सुव॒र्गस्य॑ सुव॒र्गस्य॑ लो॒कस्या नु॑ख्यात्या॒ अनु॑ख्यात्यै लो॒कस्य॑ सुव॒र्गस्य॑ सुव॒र्गस्य॑ लो॒कस्या नु॑ख्यात्यै । \newline
25. सु॒व॒र्गसेति॑ सुवः - गस्य॑ । \newline
26. लो॒कस्या नु॑ख्यात्या॒ अनु॑ख्यात्यै लो॒कस्य॑ लो॒कस्या नु॑ख्यात्या ओजो॒भृता॑ वोजो॒भृता॒ वनु॑ख्यात्यै लो॒कस्य॑ लो॒कस्या नु॑ख्यात्या ओजो॒भृतौ᳚ । \newline
27. अनु॑ख्यात्या ओजो॒भृता॑ वोजो॒भृता॒ वनु॑ख्यात्या॒ अनु॑ख्यात्या ओजो॒भृतौ॒ वै वा ओ॑जो॒भृता॒ वनु॑ख्यात्या॒ अनु॑ख्यात्या ओजो॒भृतौ॒ वै । \newline
28. अनु॑ख्यात्या॒ इत्यनु॑ - ख्या॒त्यै॒ । \newline
29. ओ॒जो॒भृतौ॒ वै वा ओ॑जो॒भृता॑ वोजो॒भृतौ॒ वा ए॒ता वे॒तौ वा ओ॑जो॒भृता॑ वोजो॒भृतौ॒ वा ए॒तौ । \newline
30. ओ॒जो॒भृता॒वित्यो॑जः - भृतौ᳚ । \newline
31. वा ए॒ता वे॒तौ वै वा ए॒तौ दे॒वाना᳚म् दे॒वाना॑ मे॒तौ वै वा ए॒तौ दे॒वाना᳚म् । \newline
32. ए॒तौ दे॒वाना᳚म् दे॒वाना॑ मे॒ता वे॒तौ दे॒वानां॒ ॅयद् यद् दे॒वाना॑ मे॒ता वे॒तौ दे॒वानां॒ ॅयत् । \newline
33. दे॒वानां॒ ॅयद् यद् दे॒वाना᳚म् दे॒वानां॒ ॅयदि॑न्द्रा॒ग्नी इ॑न्द्रा॒ग्नी यद् दे॒वाना᳚म् दे॒वानां॒ ॅयदि॑न्द्रा॒ग्नी । \newline
34. यदि॑न्द्रा॒ग्नी इ॑न्द्रा॒ग्नी यद् यदि॑न्द्रा॒ग्नी यद् यदि॑न्द्रा॒ग्नी यद् यदि॑न्द्रा॒ग्नी यत् । \newline
35. इ॒न्द्रा॒ग्नी यद् यदि॑न्द्रा॒ग्नी इ॑न्द्रा॒ग्नी यदै᳚न्द्रा॒ग्न ऐ᳚न्द्रा॒ग्नो यदि॑न्द्रा॒ग्नी इ॑न्द्रा॒ग्नी यदै᳚न्द्रा॒ग्नः । \newline
36. इ॒न्द्रा॒ग्नी इती᳚न्द्र - अ॒ग्नी । \newline
37. यदै᳚न्द्रा॒ग्न ऐ᳚न्द्रा॒ग्नो यद् यदै᳚न्द्रा॒ग्नो गृ॒ह्यते॑ गृ॒ह्यत॑ ऐन्द्रा॒ग्नो यद् यदै᳚न्द्रा॒ग्नो गृ॒ह्यते᳚ । \newline
38. ऐ॒न्द्रा॒ग्नो गृ॒ह्यते॑ गृ॒ह्यत॑ ऐन्द्रा॒ग्न ऐ᳚न्द्रा॒ग्नो गृ॒ह्यत॒ ओज॒ ओजो॑ गृ॒ह्यत॑ ऐन्द्रा॒ग्न ऐ᳚न्द्रा॒ग्नो गृ॒ह्यत॒ ओजः॑ । \newline
39. ऐ॒न्द्रा॒ग्न इत्यै᳚न्द्र - अ॒ग्नः । \newline
40. गृ॒ह्यत॒ ओज॒ ओजो॑ गृ॒ह्यते॑ गृ॒ह्यत॒ ओज॑ ए॒वै वौजो॑ गृ॒ह्यते॑ गृ॒ह्यत॒ ओज॑ ए॒व । \newline
41. ओज॑ ए॒वै वौज॒ ओज॑ ए॒वावा वै॒वौज॒ ओज॑ ए॒वाव॑ । \newline
42. ए॒वावा वै॒वै वाव॑ रुन्धे रु॒न्धे ऽवै॒वै वाव॑ रुन्धे । \newline
43. अव॑ रुन्धे रु॒न्धे ऽवाव॑ रुन्धे वैश्वदे॒वं ॅवै᳚श्वदे॒वꣳ रु॒न्धे ऽवाव॑ रुन्धे वैश्वदे॒वम् । \newline
44. रु॒न्धे॒ वै॒श्व॒दे॒वं ॅवै᳚श्वदे॒वꣳ रु॑न्धे रुन्धे वैश्वदे॒वꣳ शु॑क्रपा॒त्रेण॑ शुक्रपा॒त्रेण॑ वैश्वदे॒वꣳ रु॑न्धे रुन्धे वैश्वदे॒वꣳ शु॑क्रपा॒त्रेण॑ । \newline
45. वै॒श्व॒दे॒वꣳ शु॑क्रपा॒त्रेण॑ शुक्रपा॒त्रेण॑ वैश्वदे॒वं ॅवै᳚श्वदे॒वꣳ शु॑क्रपा॒त्रेण॑ गृह्णाति गृह्णाति शुक्रपा॒त्रेण॑ वैश्वदे॒वं ॅवै᳚श्वदे॒वꣳ शु॑क्रपा॒त्रेण॑ गृह्णाति । \newline
46. वै॒श्व॒दे॒वमिति॑ वैश्व - दे॒वम् । \newline
47. शु॒क्र॒पा॒त्रेण॑ गृह्णाति गृह्णाति शुक्रपा॒त्रेण॑ शुक्रपा॒त्रेण॑ गृह्णाति वैश्वदे॒व्यो॑ वैश्वदे॒व्यो॑ गृह्णाति शुक्रपा॒त्रेण॑ शुक्रपा॒त्रेण॑ गृह्णाति वैश्वदे॒व्यः॑ । \newline
48. शु॒क्र॒पा॒त्रेणेति॑ शुक्र - पा॒त्रेण॑ । \newline
49. गृ॒ह्णा॒ति॒ वै॒श्व॒दे॒व्यो॑ वैश्वदे॒व्यो॑ गृह्णाति गृह्णाति वैश्वदे॒व्यो॑ वै वै वै᳚श्वदे॒व्यो॑ गृह्णाति गृह्णाति वैश्वदे॒व्यो॑ वै । \newline
50. वै॒श्व॒दे॒व्यो॑ वै वै वै᳚श्वदे॒व्यो॑ वैश्वदे॒व्यो॑ वै प्र॒जाः प्र॒जा वै वै᳚श्वदे॒व्यो॑ वैश्वदे॒व्यो॑ वै प्र॒जाः । \newline
51. वै॒श्व॒दे॒व्य॑ इति॑ वैश्व - दे॒व्यः॑ । \newline
52. वै प्र॒जाः प्र॒जा वै वै प्र॒जा अ॒सा व॒सौ प्र॒जा वै वै प्र॒जा अ॒सौ । \newline
53. प्र॒जा अ॒सा व॒सौ प्र॒जाः प्र॒जा अ॒सा वा॑दि॒त्य आ॑दि॒त्यो॑ ऽसौ प्र॒जाः प्र॒जा अ॒सा वा॑दि॒त्यः । \newline
54. प्र॒जा इति॑ प्र - जाः । \newline
55. अ॒सा वा॑दि॒त्य आ॑दि॒त्यो॑ ऽसा व॒सा वा॑दि॒त्यः शु॒क्रः शु॒क्र आ॑दि॒त्यो॑ ऽसा व॒सा वा॑दि॒त्यः शु॒क्रः । \newline
56. आ॒दि॒त्यः शु॒क्रः शु॒क्र आ॑दि॒त्य आ॑दि॒त्यः शु॒क्रो यद् यच्छु॒क्र आ॑दि॒त्य आ॑दि॒त्यः शु॒क्रो यत् । \newline
57. शु॒क्रो यद् यच्छु॒क्रः शु॒क्रो यद् वै᳚श्वदे॒वं ॅवै᳚श्वदे॒वं ॅयच्छु॒क्रः शु॒क्रो यद् वै᳚श्वदे॒वम् । \newline
58. यद् वै᳚श्वदे॒वं ॅवै᳚श्वदे॒वं ॅयद् यद् वै᳚श्वदे॒वꣳ शु॑क्रपा॒त्रेण॑ शुक्रपा॒त्रेण॑ वैश्वदे॒वं ॅयद् यद् वै᳚श्वदे॒वꣳ शु॑क्रपा॒त्रेण॑ । \newline
59. वै॒श्व॒दे॒वꣳ शु॑क्रपा॒त्रेण॑ शुक्रपा॒त्रेण॑ वैश्वदे॒वं ॅवै᳚श्वदे॒वꣳ शु॑क्रपा॒त्रेण॑ गृ॒ह्णाति॑ गृ॒ह्णाति॑ शुक्रपा॒त्रेण॑ वैश्वदे॒वं ॅवै᳚श्वदे॒वꣳ शु॑क्रपा॒त्रेण॑ गृ॒ह्णाति॑ । \newline
60. वै॒श्व॒दे॒वमिति॑ वैश्व - दे॒वम् । \newline
61. शु॒क्र॒पा॒त्रेण॑ गृ॒ह्णाति॑ गृ॒ह्णाति॑ शुक्रपा॒त्रेण॑ शुक्रपा॒त्रेण॑ गृ॒ह्णाति॒ तस्मा॒त् तस्मा᳚द् गृ॒ह्णाति॑ शुक्रपा॒त्रेण॑ शुक्रपा॒त्रेण॑ गृ॒ह्णाति॒ तस्मा᳚त् । \newline
62. शु॒क्र॒पा॒त्रेणेति॑ शुक्र - पा॒त्रेण॑ । \newline
63. गृ॒ह्णाति॒ तस्मा॒त् तस्मा᳚द् गृ॒ह्णाति॑ गृ॒ह्णाति॒ तस्मा॑ द॒सा व॒सौ तस्मा᳚द् गृ॒ह्णाति॑ गृ॒ह्णाति॒ तस्मा॑ द॒सौ । \newline
64. तस्मा॑ द॒सा व॒सौ तस्मा॒त् तस्मा॑ द॒सा वा॑दि॒त्य आ॑दि॒त्यो॑ ऽसौ तस्मा॒त् तस्मा॑ द॒सा वा॑दि॒त्यः । \newline
65. अ॒सा वा॑दि॒त्य आ॑दि॒त्यो॑ ऽसा व॒सा वा॑दि॒त्यः सर्वाः॒ सर्वा॑ आदि॒त्यो॑ ऽसा व॒सा वा॑दि॒त्यः सर्वाः᳚ । \newline
66. आ॒दि॒त्यः सर्वाः॒ सर्वा॑ आदि॒त्य आ॑दि॒त्यः सर्वाः᳚ प्र॒जाः प्र॒जाः सर्वा॑ आदि॒त्य आ॑दि॒त्यः सर्वाः᳚ प्र॒जाः । \newline
\pagebreak
\markright{ TS 6.5.4.2  \hfill https://www.vedavms.in \hfill}

\section{ TS 6.5.4.2 }

\textbf{TS 6.5.4.2 } \newline
\textbf{Samhita Paata} \newline

सर्वाः᳚ प्र॒जाः प्र॒त्यङ्ङुदे॑ति॒ तस्मा॒थ् सर्व॑ ए॒व म॑न्यते॒ मां प्रत्युद॑गा॒दिति॑ वैश्वदे॒वꣳ शु॑क्रपा॒त्रेण॑ गृह्णाति वैश्वदे॒व्यो॑ वै प्र॒जास्तेजः॑ शु॒क्रो यद्-वै᳚श्वदे॒वꣳ शु॑क्रपा॒त्रेण॑ गृ॒ह्णाति॑ प्र॒जास्वे॒व तेजो॑ दधाति ॥ \newline

\textbf{Pada Paata} \newline

सर्वाः᳚ । प्र॒जा इति॑ प्र - जाः । प्र॒त्यङ् । उदिति॑ । ए॒ति॒ । तस्मा᳚त् । सर्वः॑ । ए॒व । म॒न्य॒ते॒ । माम् । प्रति॑ । उदिति॑ । अ॒गा॒त् । इति॑ । वै॒श्व॒दे॒वमिति॑ वैश्व - दे॒वम् । शु॒क्र॒पा॒त्रेणेति॑ शुक्र-पा॒त्रेण॑ । गृ॒ह्णा॒ति॒ । वै॒श्व॒दे॒व्य॑ इति॑ वैश्व - दे॒व्यः॑ । वै । प्र॒जा इति॑ प्र - जाः । तेजः॑ । शु॒क्रः । यत् । वै॒श्व॒दे॒वमिति॑ वैश्व - दे॒वम् । शु॒क्र॒पा॒त्रेणेति॑ शुक्र - पा॒त्रेण॑ । गृ॒ह्णाति॑ । प्र॒जास्विति॑ प्र - जासु॑ । ए॒व । तेजः॑ । द॒धा॒ति॒ ॥  \newline


\textbf{Krama Paata} \newline

सर्वाः᳚ प्र॒जाः । प्र॒जाः प्र॒त्यङ्‍ङ् । प्र॒जा इति॑ प्र - जाः । प्र॒त्यङ्‍ङुत् । उदे॑ति । ए॒ति॒ तस्मा᳚त् । तस्मा॒थ् सर्वः॑ । सर्व॑ ए॒व । ए॒व म॑न्यते । म॒न्य॒ते॒ माम् । माम् प्रति॑ । प्रत्युत् । उद॑गात् । अ॒गा॒दिति॑ । इति॑ वैश्वदे॒वम् । वै॒श्व॒दे॒वꣳ शु॑क्रपा॒त्रेण॑ । वै॒श्व॒दे॒वमिति॑ वैश्व - दे॒वम् । शु॒क्र॒पा॒त्रेण॑ गृह्णाति । शु॒क्र॒पा॒त्रेणेति॑ शुक्र - पा॒त्रेण॑ । गृ॒ह्णा॒ति॒ वै॒श्व॒दे॒व्यः॑ । वै॒श्व॒दे॒व्यो॑ वै । वै॒श्व॒दे॒व्य॑ इति॑ वैश्व - दे॒व्यः॑ । वै प्र॒जाः । प्र॒जास्तेजः॑ । प्र॒जा इति॑ प्र - जाः । तेजः॑ शु॒क्रः । शु॒क्रो यत् । यद् वै᳚श्वदे॒वम् । वै॒श्व॒दे॒वꣳ शु॑क्रपा॒त्रेण॑ । वै॒श्व॒दे॒वमिति॑ वैश्व - दे॒वम् । शु॒क्र॒पा॒त्रेण॑ गृ॒ह्णाति॑ । शु॒क्र॒पा॒त्रेणेति॑ शुक्र - पा॒त्रेण॑ । गृ॒ह्णाति॑ प्र॒जासु॑ । प्र॒जास्वे॒व । प्र॒जास्विति॑ प्र - जासु॑ । ए॒व तेजः॑ । तेजो॑ दधाति । द॒धा॒तीति॑ दधाति । \newline

\textbf{Jatai Paata} \newline

1. सर्वाः᳚ प्र॒जाः प्र॒जाः सर्वाः॒ सर्वाः᳚ प्र॒जाः । \newline
2. प्र॒जाः प्र॒त्यङ् प्र॒त्यङ् प्र॒जाः प्र॒जाः प्र॒त्यङ् । \newline
3. प्र॒जा इति॑ प्र - जाः । \newline
4. प्र॒त्यङ् ङुदुत् प्र॒त्यङ् प्र॒त्यङ् ङुत् । \newline
5. उदे᳚ त्ये॒ त्युदु दे॑ति । \newline
6. ए॒ति॒ तस्मा॒त् तस्मा॑ देत्येति॒ तस्मा᳚त् । \newline
7. तस्मा॒थ् सर्वः॒ सर्व॒ स्तस्मा॒त् तस्मा॒थ् सर्वः॑ । \newline
8. सर्व॑ ए॒वैव सर्वः॒ सर्व॑ ए॒व । \newline
9. ए॒व म॑न्यते मन्यत ए॒वैव म॑न्यते । \newline
10. म॒न्य॒ते॒ माम् माम् म॑न्यते मन्यते॒ माम् । \newline
11. माम् प्रति॒ प्रति॒ माम् माम् प्रति॑ । \newline
12. प्रत्युदुत् प्रति॒ प्रत्युत् । \newline
13. उद॑गा दगा॒ दुदु द॑गात् । \newline
14. अ॒गा॒ दिती त्य॑गा दगा॒ दिति॑ । \newline
15. इति॑ वैश्वदे॒वं ॅवै᳚श्वदे॒व मितीति॑ वैश्वदे॒वम् । \newline
16. वै॒श्व॒दे॒वꣳ शु॑क्रपा॒त्रेण॑ शुक्रपा॒त्रेण॑ वैश्वदे॒वं ॅवै᳚श्वदे॒वꣳ शु॑क्रपा॒त्रेण॑ । \newline
17. वै॒श्व॒दे॒वमिति॑ वैश्व - दे॒वम् । \newline
18. शु॒क्र॒पा॒त्रेण॑ गृह्णाति गृह्णाति शुक्रपा॒त्रेण॑ शुक्रपा॒त्रेण॑ गृह्णाति । \newline
19. शु॒क्र॒पा॒त्रेणेति॑ शुक्र - पा॒त्रेण॑ । \newline
20. गृ॒ह्णा॒ति॒ वै॒श्व॒दे॒व्यो॑ वैश्वदे॒व्यो॑ गृह्णाति गृह्णाति वैश्वदे॒व्यः॑ । \newline
21. वै॒श्व॒दे॒व्यो॑ वै वै वै᳚श्वदे॒व्यो॑ वैश्वदे॒व्यो॑ वै । \newline
22. वै॒श्व॒दे॒व्य॑ इति॑ वैश्व - दे॒व्यः॑ । \newline
23. वै प्र॒जाः प्र॒जा वै वै प्र॒जाः । \newline
24. प्र॒जा स्तेज॒ स्तेजः॑ प्र॒जाः प्र॒जा स्तेजः॑ । \newline
25. प्र॒जा इति॑ प्र - जाः । \newline
26. तेजः॑ शु॒क्रः शु॒क्र स्तेज॒ स्तेजः॑ शु॒क्रः । \newline
27. शु॒क्रो यद् यच्छु॒क्रः शु॒क्रो यत् । \newline
28. यद् वै᳚श्वदे॒वं ॅवै᳚श्वदे॒वं ॅयद् यद् वै᳚श्वदे॒वम् । \newline
29. वै॒श्व॒दे॒वꣳ शु॑क्रपा॒त्रेण॑ शुक्रपा॒त्रेण॑ वैश्वदे॒वं ॅवै᳚श्वदे॒वꣳ शु॑क्रपा॒त्रेण॑ । \newline
30. वै॒श्व॒दे॒वमिति॑ वैश्व - दे॒वम् । \newline
31. शु॒क्र॒पा॒त्रेण॑ गृ॒ह्णाति॑ गृ॒ह्णाति॑ शुक्रपा॒त्रेण॑ शुक्रपा॒त्रेण॑ गृ॒ह्णाति॑ । \newline
32. शु॒क्र॒पा॒त्रेणेति॑ शुक्र - पा॒त्रेण॑ । \newline
33. गृ॒ह्णाति॑ प्र॒जासु॑ प्र॒जासु॑ गृ॒ह्णाति॑ गृ॒ह्णाति॑ प्र॒जासु॑ । \newline
34. प्र॒जा स्वे॒वैव प्र॒जासु॑ प्र॒जा स्वे॒व । \newline
35. प्र॒जास्विति॑ प्र - जासु॑ । \newline
36. ए॒व तेज॒ स्तेज॑ ए॒वैव तेजः॑ । \newline
37. तेजो॑ दधाति दधाति॒ तेज॒ स्तेजो॑ दधाति । \newline
38. द॒धा॒तीति॑ दधाति । \newline

\textbf{Ghana Paata } \newline

1. सर्वाः᳚ प्र॒जाः प्र॒जाः सर्वाः॒ सर्वाः᳚ प्र॒जाः प्र॒त्यङ् प्र॒त्यङ् प्र॒जाः सर्वाः॒ सर्वाः᳚ प्र॒जाः प्र॒त्यङ् । \newline
2. प्र॒जाः प्र॒त्यङ् प्र॒त्यङ् प्र॒जाः प्र॒जाः प्र॒त्यङ् ङुदुत् प्र॒त्यङ् प्र॒जाः प्र॒जाः प्र॒त्यङ्
ङुत् । \newline
3. प्र॒जा इति॑ प्र - जाः । \newline
4. प्र॒त्यङ् ङुदुत् प्र॒त्यङ् प्र॒त्यङ् ङुदे᳚त्ये॒ त्युत् प्र॒त्यङ् प्र॒त्यङ् ङुदे॑ति । \newline
5. उदे᳚त्ये॒ त्युदु दे॑ति॒ तस्मा॒त् तस्मा॑ दे॒त्यु दुदे॑ति॒ तस्मा᳚त् । \newline
6. ए॒ति॒ तस्मा॒त् तस्मा॑ देत्येति॒ तस्मा॒थ् सर्वः॒ सर्व॒ स्तस्मा॑ देत्येति॒ तस्मा॒थ् सर्वः॑ । \newline
7. तस्मा॒थ् सर्वः॒ सर्व॒ स्तस्मा॒त् तस्मा॒थ् सर्व॑ ए॒वैव सर्व॒ स्तस्मा॒त् तस्मा॒थ् सर्व॑ ए॒व । \newline
8. सर्व॑ ए॒वैव सर्वः॒ सर्व॑ ए॒व म॑न्यते मन्यत ए॒व सर्वः॒ सर्व॑ ए॒व म॑न्यते । \newline
9. ए॒व म॑न्यते मन्यत ए॒वैव म॑न्यते॒ माम् माम् म॑न्यत ए॒वैव म॑न्यते॒ माम् । \newline
10. म॒न्य॒ते॒ माम् माम् म॑न्यते मन्यते॒ माम् प्रति॒ प्रति॒ माम् म॑न्यते मन्यते॒ माम् प्रति॑ । \newline
11. माम् प्रति॒ प्रति॒ माम् माम् प्रत्युदुत् प्रति॒ माम् माम् प्रत्युत् । \newline
12. प्रत्युदुत् प्रति॒ प्रत्यु द॑गा दगा॒ दुत् प्रति॒ प्रत्यु द॑गात् । \newline
13. उद॑गा दगा॒ दुदु द॑गा॒ दितीत्य॑गा॒ दुदु द॑गा॒ दिति॑ । \newline
14. अ॒गा॒ दितीत्य॑गा दगा॒ दिति॑ वैश्वदे॒वं ॅवै᳚श्वदे॒व मित्य॑गा दगा॒ दिति॑ वैश्वदे॒वम् । \newline
15. इति॑ वैश्वदे॒वं ॅवै᳚श्वदे॒व मितीति॑ वैश्वदे॒वꣳ शु॑क्रपा॒त्रेण॑ शुक्रपा॒त्रेण॑ वैश्वदे॒व मितीति॑ वैश्वदे॒वꣳ शु॑क्रपा॒त्रेण॑ । \newline
16. वै॒श्व॒दे॒वꣳ शु॑क्रपा॒त्रेण॑ शुक्रपा॒त्रेण॑ वैश्वदे॒वं ॅवै᳚श्वदे॒वꣳ शु॑क्रपा॒त्रेण॑ गृह्णाति गृह्णाति शुक्रपा॒त्रेण॑ वैश्वदे॒वं ॅवै᳚श्वदे॒वꣳ शु॑क्रपा॒त्रेण॑ गृह्णाति । \newline
17. वै॒श्व॒दे॒वमिति॑ वैश्व - दे॒वम् । \newline
18. शु॒क्र॒पा॒त्रेण॑ गृह्णाति गृह्णाति शुक्रपा॒त्रेण॑ शुक्रपा॒त्रेण॑ गृह्णाति वैश्वदे॒व्यो॑ वैश्वदे॒व्यो॑ गृह्णाति शुक्रपा॒त्रेण॑ शुक्रपा॒त्रेण॑ गृह्णाति वैश्वदे॒व्यः॑ । \newline
19. शु॒क्र॒पा॒त्रेणेति॑ शुक्र - पा॒त्रेण॑ । \newline
20. गृ॒ह्णा॒ति॒ वै॒श्व॒दे॒व्यो॑ वैश्वदे॒व्यो॑ गृह्णाति गृह्णाति वैश्वदे॒व्यो॑ वै वै वै᳚श्वदे॒व्यो॑ गृह्णाति गृह्णाति वैश्वदे॒व्यो॑ वै । \newline
21. वै॒श्व॒दे॒व्यो॑ वै वै वै᳚श्वदे॒व्यो॑ वैश्वदे॒व्यो॑ वै प्र॒जाः प्र॒जा वै वै᳚श्वदे॒व्यो॑ वैश्वदे॒व्यो॑ वै प्र॒जाः । \newline
22. वै॒श्व॒दे॒व्य॑ इति॑ वैश्व - दे॒व्यः॑ । \newline
23. वै प्र॒जाः प्र॒जा वै वै प्र॒जा स्तेज॒ स्तेजः॑ प्र॒जा वै वै प्र॒जा स्तेजः॑ । \newline
24. प्र॒जा स्तेज॒ स्तेजः॑ प्र॒जाः प्र॒जा स्तेजः॑ शु॒क्रः शु॒क्र स्तेजः॑ प्र॒जाः प्र॒जा स्तेजः॑ शु॒क्रः । \newline
25. प्र॒जा इति॑ प्र - जाः । \newline
26. तेजः॑ शु॒क्रः शु॒क्र स्तेज॒ स्तेजः॑ शु॒क्रो यद् यच्छु॒क्र स्तेज॒ स्तेजः॑ शु॒क्रो यत् । \newline
27. शु॒क्रो यद् यच्छु॒क्रः शु॒क्रो यद् वै᳚श्वदे॒वं ॅवै᳚श्वदे॒वं ॅयच्छु॒क्रः शु॒क्रो यद् वै᳚श्वदे॒वम् । \newline
28. यद् वै᳚श्वदे॒वं ॅवै᳚श्वदे॒वं ॅयद् यद् वै᳚श्वदे॒वꣳ शु॑क्रपा॒त्रेण॑ शुक्रपा॒त्रेण॑ वैश्वदे॒वं ॅयद् यद् वै᳚श्वदे॒वꣳ शु॑क्रपा॒त्रेण॑ । \newline
29. वै॒श्व॒दे॒वꣳ शु॑क्रपा॒त्रेण॑ शुक्रपा॒त्रेण॑ वैश्वदे॒वं ॅवै᳚श्वदे॒वꣳ शु॑क्रपा॒त्रेण॑ गृ॒ह्णाति॑ गृ॒ह्णाति॑ शुक्रपा॒त्रेण॑ वैश्वदे॒वं ॅवै᳚श्वदे॒वꣳ शु॑क्रपा॒त्रेण॑ गृ॒ह्णाति॑ । \newline
30. वै॒श्व॒दे॒वमिति॑ वैश्व - दे॒वम् । \newline
31. शु॒क्र॒पा॒त्रेण॑ गृ॒ह्णाति॑ गृ॒ह्णाति॑ शुक्रपा॒त्रेण॑ शुक्रपा॒त्रेण॑ गृ॒ह्णाति॑ प्र॒जासु॑ प्र॒जासु॑ गृ॒ह्णाति॑ शुक्रपा॒त्रेण॑ शुक्रपा॒त्रेण॑ गृ॒ह्णाति॑ प्र॒जासु॑ । \newline
32. शु॒क्र॒पा॒त्रेणेति॑ शुक्र - पा॒त्रेण॑ । \newline
33. गृ॒ह्णाति॑ प्र॒जासु॑ प्र॒जासु॑ गृ॒ह्णाति॑ गृ॒ह्णाति॑ प्र॒जा स्वे॒वैव प्र॒जासु॑ गृ॒ह्णाति॑ गृ॒ह्णाति॑ प्र॒जा स्वे॒व । \newline
34. प्र॒जा स्वे॒वैव प्र॒जासु॑ प्र॒जा स्वे॒व तेज॒ स्तेज॑ ए॒व प्र॒जासु॑ प्र॒जा स्वे॒व तेजः॑ । \newline
35. प्र॒जास्विति॑ प्र - जासु॑ । \newline
36. ए॒व तेज॒ स्तेज॑ ए॒वैव तेजो॑ दधाति दधाति॒ तेज॑ ए॒वैव तेजो॑ दधाति । \newline
37. तेजो॑ दधाति दधाति॒ तेज॒ स्तेजो॑ दधाति । \newline
38. द॒धा॒तीति॑ दधाति । \newline
\pagebreak
\markright{ TS 6.5.5.1  \hfill https://www.vedavms.in \hfill}

\section{ TS 6.5.5.1 }

\textbf{TS 6.5.5.1 } \newline
\textbf{Samhita Paata} \newline

इन्द्रो॑ म॒रुद्भिः॒ सांॅवि॑द्येन॒ माद्ध्य॑न्दिने॒ सव॑ने वृ॒त्रम॑ह॒न्॒. यन्माद्ध्य॑न्दिने॒ सव॑ने मरुत्व॒तीया॑ गृ॒ह्यन्ते॒ वार्त्र॑घ्ना ए॒व ते यज॑मानस्य गृह्यन्ते॒ तस्य॑ वृ॒त्रं ज॒घ्नुष॑ ऋ॒तवो॑ऽमुह्य॒न्थ्स ऋ॑तुपा॒त्रेण॑ मरुत्व॒तीया॑नगृह्णा॒त् ततो॒ वै स ऋ॒तून् प्राजा॑ना॒द्-यदृ॑तुपा॒त्रेण॑ मरुत्व॒तीया॑ गृ॒ह्यन्त॑ ऋतू॒नां प्रज्ञा᳚त्यै॒ वज्रं॒ ॅवा ए॒तं ॅयज॑मानो॒ भ्रातृ॑व्याय॒ प्र ह॑रति॒ यन्म॑रुत्व॒तीया॒ उदे॒व प्र॑थ॒मेन॑- [  ] \newline

\textbf{Pada Paata} \newline

इन्द्रः॑ । म॒रुद्भि॒रिति॑ म॒रुत् - भिः॒ । सांॅवि॑द्ये॒नेति॒ सां - वि॒द्ये॒न॒ । माद्ध्य॑न्दिने । सव॑ने । वृ॒त्रम् । अ॒ह॒न्न् । यत् । माद्ध्य॑न्दिने । सव॑ने । म॒रु॒त्व॒तीयाः᳚ । गृ॒ह्यन्ते᳚ । वार्त्र॑घ्ना॒ इति॒ वार्त्र॑ - घ्नाः॒ । ए॒व । ते । यज॑मानस्य । गृ॒ह्य॒न्ते॒ । तस्य॑ । वृ॒त्रम् । ज॒घ्नुषः॑ । ऋ॒तवः॑ । अ॒मु॒ह्य॒न्न् । सः । ऋ॒तु॒पा॒त्रेणेत्यृ॑तु - पा॒त्रेण॑ । म॒रु॒त्व॒तीयान्॑ । अ॒गृ॒ह्णा॒त् । ततः॑ । वै । सः । ऋ॒तून् । प्रेति॑ । अ॒जा॒ना॒त् । यत् । ऋ॒तु॒पा॒त्रेणेत्यृ॑तु-पा॒त्रेण॑ । म॒रु॒त्व॒तीयाः᳚ । गृ॒ह्यन्ते᳚ । ऋ॒तू॒नाम् । प्रज्ञा᳚त्या॒ इति॒ प्र-ज्ञा॒त्यै॒ । वज्र᳚म् । वै । ए॒तम् । यज॑मानः । भ्रातृ॑व्याय । प्रेति॑ । ह॒र॒ति॒ । यत् । म॒रु॒त्व॒तीयाः᳚ । उदिति॑ । ए॒व । प्र॒थ॒मेन॑ ।  \newline


\textbf{Krama Paata} \newline

इन्द्रो॑ म॒रुद्‌भिः॑ । म॒रुद्‌भिः॒ साम्ॅवि॑द्येन । म॒रुद्‌भि॒रिति॑ म॒रुत् - भिः॒ । साम्ॅवि॑द्येन॒ माद्ध्य॑न्दिने । साम्ॅवि॑द्ये॒नेति॒ साम् - वि॒द्ये॒न॒ । माद्ध्य॑न्दिने॒ सव॑ने । सव॑ने वृ॒त्रम् । वृ॒त्रम॑हन्न् । अ॒ह॒न्॒. यत् । यन् माद्ध्य॑न्दिने । माद्ध्य॑न्दिने॒ सव॑ने । सव॑ने मरुत्व॒तीयाः᳚ । म॒रु॒त्व॒तीया॑ गृ॒ह्यन्ते᳚ । गृ॒ह्यन्ते॒ वार्त्र॑घ्नाः । वार्त्र॑घ्ना ए॒व । वार्त्र॑घ्ना॒ इति॒ वार्त्र॑ - घ्नाः॒ । ए॒व ते । ते यज॑मानस्य । यज॑मानस्य गृह्यन्ते । गृ॒ह्य॒न्ते॒ तस्य॑ । तस्य॑ वृ॒त्रम् । वृ॒त्रम् ज॒घ्नुषः॑ । ज॒घ्नुष॑ ऋ॒तवः॑ । ऋ॒तवो॑ऽमुह्यन्न् । अ॒मु॒ह्य॒न्थ् सः । स ऋ॑तुपा॒त्रेण॑ । ऋ॒तु॒पा॒त्रेण॑ मरुत्व॒तीयान्॑ । ऋ॒तु॒पा॒त्रेणेत्यृ॑तु - पा॒त्रेण॑ । म॒रु॒त्व॒तीया॑नगृह्णात् । अ॒गृ॒ह्णा॒त् ततः॑ । ततो॒ वै । वै सः । स ऋ॒तून् । ऋ॒तून् प्र । प्राजा॑नात् । अ॒जा॒ना॒द् यत् । यदृ॑तुपा॒त्रेण॑ । ऋ॒तु॒पा॒त्रेण॑ मरुत्व॒तीयाः᳚ । ऋ॒तु॒पा॒त्रेणेत्यृ॑तु - पा॒त्रेण॑ । म॒रु॒त्व॒तीया॑ गृ॒ह्यन्ते᳚ । गृ॒ह्यन्त॑ ऋतू॒नाम् । ऋ॒तू॒नाम् प्रज्ञा᳚त्यै । प्रज्ञा᳚त्यै॒ वज्र᳚म् । प्रज्ञा᳚त्या॒ इति॒ प्र - ज्ञा॒त्यै॒ । वज्र॒म् ॅवै । वा ए॒तम् । ए॒तम् ॅयज॑मानः । यज॑मानो॒ भ्रातृ॑व्याय । भ्रातृ॑व्याय॒ प्र । प्र ह॑रति । ह॒र॒ति॒ यत् । यन् म॑रुत्व॒तीयाः᳚ । म॒रु॒त्व॒तीया॒ उत् । उदे॒व । ए॒व प्र॑थ॒मेन॑ । प्र॒थ॒मेन॑ यच्छति \newline

\textbf{Jatai Paata} \newline

1. इन्द्रो॑ म॒रुद्भि॑र् म॒रुद्भि॒ रिन्द्र॒ इन्द्रो॑ म॒रुद्भिः॑ । \newline
2. म॒रुद्भिः॒ सांॅवि॑द्येन॒ सांॅवि॑द्येन म॒रुद्भि॑र् म॒रुद्भिः॒ सांॅवि॑द्येन । \newline
3. म॒रुद्भि॒रिति॑ म॒रुत् - भिः॒ । \newline
4. सांॅवि॑द्येन॒ माद्ध्य॑न्दिने॒ माद्ध्य॑न्दिने॒ सांॅवि॑द्येन॒ सांॅवि॑द्येन॒ माद्ध्य॑न्दिने । \newline
5. सांॅवि॑द्ये॒नेति॒ सां - वि॒द्ये॒न॒ । \newline
6. माद्ध्य॑न्दिने॒ सव॑ने॒ सव॑ने॒ माद्ध्य॑न्दिने॒ माद्ध्य॑न्दिने॒ सव॑ने । \newline
7. सव॑ने वृ॒त्रं ॅवृ॒त्रꣳ सव॑ने॒ सव॑ने वृ॒त्रम् । \newline
8. वृ॒त्र म॑हन् नहन् वृ॒त्रं ॅवृ॒त्र म॑हन्न् । \newline
9. अ॒ह॒न्॒. यद् यद॑हन् नह॒न्॒. यत् । \newline
10. यन् माद्ध्य॑न्दिने॒ माद्ध्य॑न्दिने॒ यद् यन् माद्ध्य॑न्दिने । \newline
11. माद्ध्य॑न्दिने॒ सव॑ने॒ सव॑ने॒ माद्ध्य॑न्दिने॒ माद्ध्य॑न्दिने॒ सव॑ने । \newline
12. सव॑ने मरुत्व॒तीया॑ मरुत्व॒तीयाः॒ सव॑ने॒ सव॑ने मरुत्व॒तीयाः᳚ । \newline
13. म॒रु॒त्व॒तीया॑ गृ॒ह्यन्ते॑ गृ॒ह्यन्ते॑ मरुत्व॒तीया॑ मरुत्व॒तीया॑ गृ॒ह्यन्ते᳚ । \newline
14. गृ॒ह्यन्ते॒ वार्त्र॑घ्ना॒ वार्त्र॑घ्ना गृ॒ह्यन्ते॑ गृ॒ह्यन्ते॒ वार्त्र॑घ्नाः । \newline
15. वार्त्र॑घ्ना ए॒वैव वार्त्र॑घ्ना॒ वार्त्र॑घ्ना ए॒व । \newline
16. वार्त्र॑घ्ना॒ इति॒ वार्त्र॑ - घ्नाः॒ । \newline
17. ए॒व ते त ए॒वैव ते । \newline
18. ते यज॑मानस्य॒ यज॑मानस्य॒ ते ते यज॑मानस्य । \newline
19. यज॑मानस्य गृह्यन्ते गृह्यन्ते॒ यज॑मानस्य॒ यज॑मानस्य गृह्यन्ते । \newline
20. गृ॒ह्य॒न्ते॒ तस्य॒ तस्य॑ गृह्यन्ते गृह्यन्ते॒ तस्य॑ । \newline
21. तस्य॑ वृ॒त्रं ॅवृ॒त्रम् तस्य॒ तस्य॑ वृ॒त्रम् । \newline
22. वृ॒त्रम् ज॒घ्नुषो॑ ज॒घ्नुषो॑ वृ॒त्रं ॅवृ॒त्रम् ज॒घ्नुषः॑ । \newline
23. ज॒घ्नुष॑ ऋ॒तव॑ ऋ॒तवो॑ ज॒घ्नुषो॑ ज॒घ्नुष॑ ऋ॒तवः॑ । \newline
24. ऋ॒तवो॑ ऽमुह्यन् नमुह्यन् नृ॒तव॑ ऋ॒तवो॑ ऽमुह्यन्न् । \newline
25. अ॒मु॒ह्य॒न् थ्स सो॑ ऽमुह्यन् नमुह्य॒न् थ्सः । \newline
26. स ऋ॑तुपा॒त्रेण॑ र्‌तुपा॒त्रेण॒ स स ऋ॑तुपा॒त्रेण॑ । \newline
27. ऋ॒तु॒पा॒त्रेण॑ मरुत्व॒तीया᳚न् मरुत्व॒तीया॑ नृतुपा॒त्रेण॑ र्‌तुपा॒त्रेण॑ मरुत्व॒तीयान्॑ । \newline
28. ऋ॒तु॒पा॒त्रेणेत्यृ॑तु - पा॒त्रेण॑ । \newline
29. म॒रु॒त्व॒तीया॑ नगृह्णा दगृह्णान् मरुत्व॒तीया᳚न् मरुत्व॒तीया॑ नगृह्णात् । \newline
30. अ॒गृ॒ह्णा॒त् तत॒ स्ततो॑ ऽगृह्णा दगृह्णा॒त् ततः॑ । \newline
31. ततो॒ वै वै तत॒ स्ततो॒ वै । \newline
32. वै स स वै वै सः । \newline
33. स ऋ॒तू नृ॒तून् थ्स स ऋ॒तून् । \newline
34. ऋ॒तून् प्र प्रा र्‌तू नृ॒तून् प्र । \newline
35. प्राजा॑ना दजाना॒त् प्र प्राजा॑नात् । \newline
36. अ॒जा॒ना॒द् यद् यद॑जाना दजाना॒द् यत् । \newline
37. यदृ॑तुपा॒त्रेण॑ र्‌तुपा॒त्रेण॒ यद् यदृ॑तुपा॒त्रेण॑ । \newline
38. ऋ॒तु॒पा॒त्रेण॑ मरुत्व॒तीया॑ मरुत्व॒तीया॑ ऋतुपा॒त्रेण॑ र्‌तुपा॒त्रेण॑ मरुत्व॒तीयाः᳚ । \newline
39. ऋ॒तु॒पा॒त्रेणेत्यृ॑तु - पा॒त्रेण॑ । \newline
40. म॒रु॒त्व॒तीया॑ गृ॒ह्यन्ते॑ गृ॒ह्यन्ते॑ मरुत्व॒तीया॑ मरुत्व॒तीया॑ गृ॒ह्यन्ते᳚ । \newline
41. गृ॒ह्यन्त॑ ऋतू॒ना मृ॑तू॒नाम् गृ॒ह्यन्ते॑ गृ॒ह्यन्त॑ ऋतू॒नाम् । \newline
42. ऋ॒तू॒नाम् प्रज्ञा᳚त्यै॒ प्रज्ञा᳚त्या ऋतू॒ना मृ॑तू॒नाम् प्रज्ञा᳚त्यै । \newline
43. प्रज्ञा᳚त्यै॒ वज्रं॒ ॅवज्र॒म् प्रज्ञा᳚त्यै॒ प्रज्ञा᳚त्यै॒ वज्र᳚म् । \newline
44. प्रज्ञा᳚त्या॒ इति॒ प्र - ज्ञा॒त्यै॒ । \newline
45. वज्रं॒ ॅवै वै वज्रं॒ ॅवज्रं॒ ॅवै । \newline
46. वा ए॒त मे॒तं ॅवै वा ए॒तम् । \newline
47. ए॒तं ॅयज॑मानो॒ यज॑मान ए॒त मे॒तं ॅयज॑मानः । \newline
48. यज॑मानो॒ भ्रातृ॑व्याय॒ भ्रातृ॑व्याय॒ यज॑मानो॒ यज॑मानो॒ भ्रातृ॑व्याय । \newline
49. भ्रातृ॑व्याय॒ प्र प्र भ्रातृ॑व्याय॒ भ्रातृ॑व्याय॒ प्र । \newline
50. प्र ह॑रति हरति॒ प्र प्र ह॑रति । \newline
51. ह॒र॒ति॒ यद् यद्ध॑रति हरति॒ यत् । \newline
52. यन् म॑रुत्व॒तीया॑ मरुत्व॒तीया॒ यद् यन् म॑रुत्व॒तीयाः᳚ । \newline
53. म॒रु॒त्व॒तीया॒ उदुन् म॑रुत्व॒तीया॑ मरुत्व॒तीया॒ उत् । \newline
54. उदे॒ वैवोदु दे॒व । \newline
55. ए॒व प्र॑थ॒मेन॑ प्रथ॒मे नै॒वैव प्र॑थ॒मेन॑ । \newline
56. प्र॒थ॒मेन॑ यच्छति यच्छति प्रथ॒मेन॑ प्रथ॒मेन॑ यच्छति । \newline

\textbf{Ghana Paata } \newline

1. इन्द्रो॑ म॒रुद्भि॑र् म॒रुद्भि॒ रिन्द्र॒ इन्द्रो॑ म॒रुद्भिः॒ सांॅवि॑द्येन॒ सांॅवि॑द्येन म॒रुद्भि॒ रिन्द्र॒ इन्द्रो॑ म॒रुद्भिः॒ सांॅवि॑द्येन । \newline
2. म॒रुद्भिः॒ सांॅवि॑द्येन॒ सांॅवि॑द्येन म॒रुद्भि॑र् म॒रुद्भिः॒ सांॅवि॑द्येन॒ माद्ध्य॑न्दिने॒ माद्ध्य॑न्दिने॒ सांॅवि॑द्येन म॒रुद्भि॑र् म॒रुद्भिः॒ सांॅवि॑द्येन॒ माद्ध्य॑न्दिने । \newline
3. म॒रुद्भि॒रिति॑ म॒रुत् - भिः॒ । \newline
4. सांॅवि॑द्येन॒ माद्ध्य॑न्दिने॒ माद्ध्य॑न्दिने॒ सांॅवि॑द्येन॒ सांॅवि॑द्येन॒ माद्ध्य॑न्दिने॒ सव॑ने॒ सव॑ने॒ माद्ध्य॑न्दिने॒ सांॅवि॑द्येन॒ सांॅवि॑द्येन॒ माद्ध्य॑न्दिने॒ सव॑ने । \newline
5. सांॅवि॑द्ये॒नेति॒ सां - वि॒द्ये॒न॒ । \newline
6. माद्ध्य॑न्दिने॒ सव॑ने॒ सव॑ने॒ माद्ध्य॑न्दिने॒ माद्ध्य॑न्दिने॒ सव॑ने वृ॒त्रं ॅवृ॒त्रꣳ सव॑ने॒ माद्ध्य॑न्दिने॒ माद्ध्य॑न्दिने॒ सव॑ने वृ॒त्रम् । \newline
7. सव॑ने वृ॒त्रं ॅवृ॒त्रꣳ सव॑ने॒ सव॑ने वृ॒त्र म॑हन्-नहन्. वृ॒त्रꣳ सव॑ने॒ सव॑ने वृ॒त्र म॑हन्न् । \newline
8. वृ॒त्र म॑हन्-नहन्. वृ॒त्रं ॅवृ॒त्र म॑ह॒न्॒. यद् यद॑हन्. वृ॒त्रं ॅवृ॒त्र म॑ह॒न्॒. यत् । \newline
9. अ॒ह॒न्॒. यद् यद॑हन्-नह॒न्॒. यन् माद्ध्य॑न्दिने॒ माद्ध्य॑न्दिने॒ यद॑हन्-नह॒न्॒. यन् माद्ध्य॑न्दिने । \newline
10. यन् माद्ध्य॑न्दिने॒ माद्ध्य॑न्दिने॒ यद् यन् माद्ध्य॑न्दिने॒ सव॑ने॒ सव॑ने॒ माद्ध्य॑न्दिने॒ यद् यन् माद्ध्य॑न्दिने॒ सव॑ने । \newline
11. माद्ध्य॑न्दिने॒ सव॑ने॒ सव॑ने॒ माद्ध्य॑न्दिने॒ माद्ध्य॑न्दिने॒ सव॑ने मरुत्व॒तीया॑ मरुत्व॒तीयाः॒ सव॑ने॒ माद्ध्य॑न्दिने॒ माद्ध्य॑न्दिने॒ सव॑ने मरुत्व॒तीयाः᳚ । \newline
12. सव॑ने मरुत्व॒तीया॑ मरुत्व॒तीयाः॒ सव॑ने॒ सव॑ने मरुत्व॒तीया॑ गृ॒ह्यन्ते॑ गृ॒ह्यन्ते॑ मरुत्व॒तीयाः॒ सव॑ने॒ सव॑ने मरुत्व॒तीया॑ गृ॒ह्यन्ते᳚ । \newline
13. म॒रु॒त्व॒तीया॑ गृ॒ह्यन्ते॑ गृ॒ह्यन्ते॑ मरुत्व॒तीया॑ मरुत्व॒तीया॑ गृ॒ह्यन्ते॒ वार्त्र॑घ्ना॒ वार्त्र॑घ्ना गृ॒ह्यन्ते॑ मरुत्व॒तीया॑ मरुत्व॒तीया॑ गृ॒ह्यन्ते॒ वार्त्र॑घ्नाः । \newline
14. गृ॒ह्यन्ते॒ वार्त्र॑घ्ना॒ वार्त्र॑घ्ना गृ॒ह्यन्ते॑ गृ॒ह्यन्ते॒ वार्त्र॑घ्ना ए॒वैव वार्त्र॑घ्ना गृ॒ह्यन्ते॑ गृ॒ह्यन्ते॒ वार्त्र॑घ्ना ए॒व । \newline
15. वार्त्र॑घ्ना ए॒वैव वार्त्र॑घ्ना॒ वार्त्र॑घ्ना ए॒व ते त ए॒व वार्त्र॑घ्ना॒ वार्त्र॑घ्ना ए॒व ते । \newline
16. वार्त्र॑घ्ना॒ इति॒ वार्त्र॑ - घ्नाः॒ । \newline
17. ए॒व ते त ए॒वैव ते यज॑मानस्य॒ यज॑मानस्य॒ त ए॒वैव ते यज॑मानस्य । \newline
18. ते यज॑मानस्य॒ यज॑मानस्य॒ ते ते यज॑मानस्य गृह्यन्ते गृह्यन्ते॒ यज॑मानस्य॒ ते ते यज॑मानस्य गृह्यन्ते । \newline
19. यज॑मानस्य गृह्यन्ते गृह्यन्ते॒ यज॑मानस्य॒ यज॑मानस्य गृह्यन्ते॒ तस्य॒ तस्य॑ गृह्यन्ते॒ यज॑मानस्य॒ यज॑मानस्य गृह्यन्ते॒ तस्य॑ । \newline
20. गृ॒ह्य॒न्ते॒ तस्य॒ तस्य॑ गृह्यन्ते गृह्यन्ते॒ तस्य॑ वृ॒त्रं ॅवृ॒त्रम् तस्य॑ गृह्यन्ते गृह्यन्ते॒ तस्य॑ वृ॒त्रम् । \newline
21. तस्य॑ वृ॒त्रं ॅवृ॒त्रम् तस्य॒ तस्य॑ वृ॒त्रम् ज॒घ्नुषो॑ ज॒घ्नुषो॑ वृ॒त्रम् तस्य॒ तस्य॑ वृ॒त्रम् ज॒घ्नुषः॑ । \newline
22. वृ॒त्रम् ज॒घ्नुषो॑ ज॒घ्नुषो॑ वृ॒त्रं ॅवृ॒त्रम् ज॒घ्नुष॑ ऋ॒तव॑ ऋ॒तवो॑ ज॒घ्नुषो॑ वृ॒त्रं ॅवृ॒त्रम् ज॒घ्नुष॑ ऋ॒तवः॑ । \newline
23. ज॒घ्नुष॑ ऋ॒तव॑ ऋ॒तवो॑ ज॒घ्नुषो॑ ज॒घ्नुष॑ ऋ॒तवो॑ ऽमुह्यन्-नमुह्यन्-नृ॒तवो॑ ज॒घ्नुषो॑ ज॒घ्नुष॑ ऋ॒तवो॑ ऽमुह्यन्न् । \newline
24. ऋ॒तवो॑ ऽमुह्यन्-नमुह्यन्-नृ॒तव॑ ऋ॒तवो॑ ऽमुह्य॒न् थ्स सो॑ ऽमुह्यन्-नृ॒तव॑ ऋ॒तवो॑ ऽमुह्य॒न् थ्सः । \newline
25. अ॒मु॒ह्य॒न् थ्स सो॑ ऽमुह्यन्-नमुह्य॒न् थ्स ऋ॑तुपा॒त्रेण॑ र्‌तुपा॒त्रेण॒ सो॑ ऽमुह्यन्-नमुह्य॒न् थ्स ऋ॑तुपा॒त्रेण॑ । \newline
26. स ऋ॑तुपा॒त्रेण॑ र्‌तुपा॒त्रेण॒ स स ऋ॑तुपा॒त्रेण॑ मरुत्व॒तीया᳚न् मरुत्व॒तीया॑-नृतुपा॒त्रेण॒ स स ऋ॑तुपा॒त्रेण॑ मरुत्व॒तीयान्॑ । \newline
27. ऋ॒तु॒पा॒त्रेण॑ मरुत्व॒तीया᳚न् मरुत्व॒तीया॑-नृतुपा॒त्रेण॑ र्‌तुपा॒त्रेण॑ मरुत्व॒तीया॑-नगृह्णा दगृह्णान् मरुत्व॒तीया॑-नृतुपा॒त्रेण॑ र्‌तुपा॒त्रेण॑ मरुत्व॒तीया॑-नगृह्णात् । \newline
28. ऋ॒तु॒पा॒त्रेणेत्यृ॑तु - पा॒त्रेण॑ । \newline
29. म॒रु॒त्व॒तीया॑-नगृह्णा दगृह्णान् मरुत्व॒तीया᳚न् मरुत्व॒तीया॑-नगृह्णा॒त् तत॒ स्ततो॑ ऽगृह्णान् मरुत्व॒तीया᳚न् मरुत्व॒तीया॑-नगृह्णा॒त् ततः॑ । \newline
30. अ॒गृ॒ह्णा॒त् तत॒ स्ततो॑ ऽगृह्णा दगृह्णा॒त् ततो॒ वै वै ततो॑ ऽगृह्णा दगृह्णा॒त् ततो॒ वै । \newline
31. ततो॒ वै वै तत॒ स्ततो॒ वै स स वै तत॒ स्ततो॒ वै सः । \newline
32. वै स स वै वै स ऋ॒तू-नृ॒तून् थ्स वै वै स ऋ॒तून् । \newline
33. स ऋ॒तू-नृ॒तून् थ्स स ऋ॒तून् प्र प्रा र्‌तून् थ्स स ऋ॒तून् प्र । \newline
34. ऋ॒तून् प्र प्रा र्‌तू-नृ॒तून् प्राजा॑ना दजाना॒त् प्रा र्‌तू-नृ॒तून् प्राजा॑नात् । \newline
35. प्राजा॑ना दजाना॒त् प्र प्राजा॑ना॒द् यद् यद॑जाना॒त् प्र प्राजा॑ना॒द् यत् । \newline
36. अ॒जा॒ना॒द् यद् यद॑जाना दजाना॒द् यदृ॑तुपा॒त्रेण॑ र्‌तुपा॒त्रेण॒ यद॑जाना दजाना॒द् यदृ॑तुपा॒त्रेण॑ । \newline
37. यदृ॑तुपा॒त्रेण॑ र्‌तुपा॒त्रेण॒ यद् यदृ॑तुपा॒त्रेण॑ मरुत्व॒तीया॑ मरुत्व॒तीया॑ ऋतुपा॒त्रेण॒ यद् यदृ॑तुपा॒त्रेण॑ मरुत्व॒तीयाः᳚ । \newline
38. ऋ॒तु॒पा॒त्रेण॑ मरुत्व॒तीया॑ मरुत्व॒तीया॑ ऋतुपा॒त्रेण॑ र्‌तुपा॒त्रेण॑ मरुत्व॒तीया॑ गृ॒ह्यन्ते॑ गृ॒ह्यन्ते॑ मरुत्व॒तीया॑ ऋतुपा॒त्रेण॑ र्‌तुपा॒त्रेण॑ मरुत्व॒तीया॑ गृ॒ह्यन्ते᳚ । \newline
39. ऋ॒तु॒पा॒त्रेणेत्यृ॑तु - पा॒त्रेण॑ । \newline
40. म॒रु॒त्व॒तीया॑ गृ॒ह्यन्ते॑ गृ॒ह्यन्ते॑ मरुत्व॒तीया॑ मरुत्व॒तीया॑ गृ॒ह्यन्त॑ ऋतू॒ना मृ॑तू॒नाम् गृ॒ह्यन्ते॑ मरुत्व॒तीया॑ मरुत्व॒तीया॑ गृ॒ह्यन्त॑ ऋतू॒नाम् । \newline
41. गृ॒ह्यन्त॑ ऋतू॒ना मृ॑तू॒नाम् गृ॒ह्यन्ते॑ गृ॒ह्यन्त॑ ऋतू॒नाम् प्रज्ञा᳚त्यै॒ प्रज्ञा᳚त्या ऋतू॒नाम् गृ॒ह्यन्ते॑ गृ॒ह्यन्त॑ ऋतू॒नाम् प्रज्ञा᳚त्यै । \newline
42. ऋ॒तू॒नाम् प्रज्ञा᳚त्यै॒ प्रज्ञा᳚त्या ऋतू॒ना मृ॑तू॒नाम् प्रज्ञा᳚त्यै॒ वज्रं॒ ॅवज्र॒म् प्रज्ञा᳚त्या ऋतू॒ना मृ॑तू॒नाम् प्रज्ञा᳚त्यै॒ वज्र᳚म् । \newline
43. प्रज्ञा᳚त्यै॒ वज्रं॒ ॅवज्र॒म् प्रज्ञा᳚त्यै॒ प्रज्ञा᳚त्यै॒ वज्रं॒ ॅवै वै वज्र॒म् प्रज्ञा᳚त्यै॒ प्रज्ञा᳚त्यै॒ वज्रं॒ ॅवै । \newline
44. प्रज्ञा᳚त्या॒ इति॒ प्र - ज्ञा॒त्यै॒ । \newline
45. वज्रं॒ ॅवै वै वज्रं॒ ॅवज्रं॒ ॅवा ए॒त मे॒तं ॅवै वज्रं॒ ॅवज्रं॒ ॅवा ए॒तम् । \newline
46. वा ए॒त मे॒तं ॅवै वा ए॒तं ॅयज॑मानो॒ यज॑मान ए॒तं ॅवै वा ए॒तं ॅयज॑मानः । \newline
47. ए॒तं ॅयज॑मानो॒ यज॑मान ए॒त मे॒तं ॅयज॑मानो॒ भ्रातृ॑व्याय॒ भ्रातृ॑व्याय॒ यज॑मान ए॒त मे॒तं ॅयज॑मानो॒ भ्रातृ॑व्याय । \newline
48. यज॑मानो॒ भ्रातृ॑व्याय॒ भ्रातृ॑व्याय॒ यज॑मानो॒ यज॑मानो॒ भ्रातृ॑व्याय॒ प्र प्र भ्रातृ॑व्याय॒ यज॑मानो॒ यज॑मानो॒ भ्रातृ॑व्याय॒ प्र । \newline
49. भ्रातृ॑व्याय॒ प्र प्र भ्रातृ॑व्याय॒ भ्रातृ॑व्याय॒ प्र ह॑रति हरति॒ प्र भ्रातृ॑व्याय॒ भ्रातृ॑व्याय॒ प्र ह॑रति । \newline
50. प्र ह॑रति हरति॒ प्र प्र ह॑रति॒ यद् यद्ध॑रति॒ प्र प्र ह॑रति॒ यत् । \newline
51. ह॒र॒ति॒ यद् यद्ध॑रति हरति॒ यन् म॑रुत्व॒तीया॑ मरुत्व॒तीया॒ यद्ध॑रति हरति॒ यन् म॑रुत्व॒तीयाः᳚ । \newline
52. यन् म॑रुत्व॒तीया॑ मरुत्व॒तीया॒ यद् यन् म॑रुत्व॒तीया॒ उदुन् म॑रुत्व॒तीया॒ यद् यन् म॑रुत्व॒तीया॒ उत् । \newline
53. म॒रु॒त्व॒तीया॒ उदुन् म॑रुत्व॒तीया॑ मरुत्व॒तीया॒ उदे॒ वैवोन् म॑रुत्व॒तीया॑ मरुत्व॒तीया॒ उदे॒व । \newline
54. उदे॒वैवो दुदे॒व प्र॑थ॒मेन॑ प्रथ॒मे नै॒वो दुदे॒व प्र॑थ॒मेन॑ । \newline
55. ए॒व प्र॑थ॒मेन॑ प्रथ॒मे नै॒वैव प्र॑थ॒मेन॑ यच्छति यच्छति प्रथ॒मे नै॒वैव प्र॑थ॒मेन॑ यच्छति । \newline
56. प्र॒थ॒मेन॑ यच्छति यच्छति प्रथ॒मेन॑ प्रथ॒मेन॑ यच्छति॒ प्र प्र य॑च्छति प्रथ॒मेन॑ प्रथ॒मेन॑ यच्छति॒ प्र । \newline
\pagebreak
\markright{ TS 6.5.5.2  \hfill https://www.vedavms.in \hfill}

\section{ TS 6.5.5.2 }

\textbf{TS 6.5.5.2 } \newline
\textbf{Samhita Paata} \newline

यच्छति॒ प्र ह॑रति द्वि॒तीये॑न स्तृणु॒ते तृ॒तीये॒नाऽऽ*यु॑धं॒ ॅवा ए॒तद्-यज॑मानः॒ सꣳ स्कु॑रुते॒ यन्म॑रुत्व॒तीया॒ धनु॑रे॒व प्र॑थ॒मो ज्या द्वि॒तीय॒ इषु॑स्तृ॒तीयः॒ प्रत्ये॒व प्र॑थ॒मेन॑ धत्ते॒ विसृ॑जति द्वि॒तीये॑न॒ विद्ध्य॑ति तृ॒तीये॒नेन्द्रो॑ वृ॒त्रꣳ ह॒त्वा परां᳚ परा॒वत॑-मगच्छ॒-दपा॑राध॒मिति॒ मन्य॑मानः॒ स हरि॑तोऽभव॒थ् स ए॒तान् म॑रुत्व॒तीया॑-नात्म॒स्पर॑णा-नपश्य॒त् तान॑गृह्णीत- [  ] \newline

\textbf{Pada Paata} \newline

य॒च्छ॒ति॒ । प्रेति॑ । ह॒र॒ति॒ । द्वि॒तीये॑न । स्तृ॒णु॒ते । तृ॒तीये॑न । आयु॑धम् । वै । ए॒तत् । यज॑मानः । समिति॑ । कु॒रु॒ते॒ । यत् । म॒रु॒त्व॒तीयाः᳚ । धनुः॑ । ए॒व । प्र॒थ॒मः । ज्या । द्वि॒तीयः॑ । इषुः॑ । तृ॒तीयः॑ । प्रतीति॑ । ए॒व । प्र॒थ॒मेन॑ । ध॒त्ते॒ । वीति॑ । सृ॒ज॒ति॒ । द्वि॒तीये॑न । विद्ध्य॑ति । तृ॒तीये॑न । इन्द्रः॑ । वृ॒त्रम् । ह॒त्वा । परा᳚म् । प॒रा॒वत॒मिति॑ परा - वत᳚म् । अ॒ग॒च्छ॒त् । अपेति॑ । अ॒रा॒ध॒म् । इति॑ । मन्य॑मानः । सः । हरि॑तः । अ॒भ॒व॒त् । सः । ए॒तान् । म॒रु॒त्व॒तीयान्॑ । आ॒त्म॒स्पर॑णा॒नित्या᳚त्म - स्पर॑णान् । अ॒प॒श्य॒त् । तान् । अ॒गृ॒ह्णी॒त॒ ।  \newline


\textbf{Krama Paata} \newline

य॒च्छ॒ति॒ प्र । प्र ह॑रति । ह॒र॒ति॒ द्वि॒तीये॑न । द्वि॒तीये॑न स्तृणु॒ते । स्तृ॒णु॒ते तृ॒तीये॑न । तृ॒तीये॒नायु॑धम् । आयु॑ध॒म् ॅवै । वा ए॒तत् । ए॒तद् यज॑मानः । यज॑मानः॒ सम् । सꣳ स्कु॑रुते । कु॒रु॒ते॒ यत् । यन् म॑रुत्व॒तीयाः᳚ । म॒रु॒त्व॒तीया॒ धनुः॑ । धनु॑रे॒व । ए॒व प्र॑थ॒मः । प्र॒थ॒मो ज्या । ज्या द्वि॒तीयः॑ । द्वि॒तीय॒ इषुः॑ । इषु॑स्तृ॒तीयः॑ । तृ॒तीयः॒ प्रति॑ । प्रत्ये॒व । ए॒व प्र॑थ॒मेन॑ । प्र॒थ॒मेन॑ धत्ते । ध॒त्ते॒ वि । वि सृ॑जति । सृ॒ज॒ति॒ द्वि॒तीये॑न । द्वि॒तीये॑न॒ विद्ध्य॑ति । विद्ध्य॑ति तृ॒तीये॑न । तृ॒तीये॒नेन्द्रः॑ । इन्द्रो॑ वृ॒त्रम् । वृ॒त्रꣳ ह॒त्वा । ह॒त्वा परा᳚म् । परा᳚म् परा॒वत᳚म् । प॒रा॒वत॑मगच्छत् । प॒रा॒वत॒मिति॑ परा - वत᳚म् । अ॒ग॒च्छ॒दप॑ । अपा॑राधम् । अ॒रा॒ध॒मिति॑ । इति॒ मन्य॑मानः । मन्य॑मानः॒ सः । स हरि॑तः । हरि॑तोऽभवत् । अ॒भ॒व॒थ् सः । स ए॒तान् । ए॒तान् म॑रुत्व॒तीयान्॑ । म॒रु॒त्व॒तीया॑नात्म॒स्पर॑णान् । आ॒त्म॒स्पर॑णानपश्यत् । आ॒त्म॒स्पर॑णा॒नित्या᳚त्म - स्पर॑णान् । अ॒प॒श्य॒त् तान् । तान॑गृह्णीत । अ॒गृ॒ह्णी॒त॒ प्रा॒णम् \newline

\textbf{Jatai Paata} \newline

1. य॒च्छ॒ति॒ प्र प्र य॑च्छति यच्छति॒ प्र । \newline
2. प्र ह॑रति हरति॒ प्र प्र ह॑रति । \newline
3. ह॒र॒ति॒ द्वि॒तीये॑न द्वि॒तीये॑न हरति हरति द्वि॒तीये॑न । \newline
4. द्वि॒तीये॑न स्तृणु॒ते स्तृ॑णु॒ते द्वि॒तीये॑न द्वि॒तीये॑न स्तृणु॒ते । \newline
5. स्तृ॒णु॒ते तृ॒तीये॑न तृ॒तीये॑न स्तृणु॒ते स्तृ॑णु॒ते तृ॒तीये॑न । \newline
6. तृ॒तीये॒ना यु॑ध॒ मायु॑धम् तृ॒तीये॑न तृ॒तीये॒ना यु॑धम् । \newline
7. आयु॑धं॒ ॅवै वा आयु॑ध॒ मायु॑धं॒ ॅवै । \newline
8. वा ए॒त दे॒तद् वै वा ए॒तत् । \newline
9. ए॒तद् यज॑मानो॒ यज॑मान ए॒त दे॒तद् यज॑मानः । \newline
10. यज॑मानः॒ सꣳ सं ॅयज॑मानो॒ यज॑मानः॒ सम् । \newline
11. सꣳ स्कु॑रुते कुरुते॒ सꣳ सꣳ स्कु॑रुते । \newline
12. कु॒रु॒ते॒ यद् यत् कु॑रुते कुरुते॒ यत् । \newline
13. यन् म॑रुत्व॒तीया॑ मरुत्व॒तीया॒ यद् यन् म॑रुत्व॒तीयाः᳚ । \newline
14. म॒रु॒त्व॒तीया॒ धनु॒र् धनु॑र् मरुत्व॒तीया॑ मरुत्व॒तीया॒ धनुः॑ । \newline
15. धनु॑ रे॒वैव धनु॒र् धनु॑ रे॒व । \newline
16. ए॒व प्र॑थ॒मः प्र॑थ॒म ए॒वैव प्र॑थ॒मः । \newline
17. प्र॒थ॒मो ज्या ज्या प्र॑थ॒मः प्र॑थ॒मो ज्या । \newline
18. ज्या द्वि॒तीयो᳚ द्वि॒तीयो॒ ज्या ज्या द्वि॒तीयः॑ । \newline
19. द्वि॒तीय॒ इषु॒ रिषु॑र् द्वि॒तीयो᳚ द्वि॒तीय॒ इषुः॑ । \newline
20. इषु॑ स्तृ॒तीय॑ स्तृ॒तीय॒ इषु॒ रिषु॑ स्तृ॒तीयः॑ । \newline
21. तृ॒तीयः॒ प्रति॒ प्रति॑ तृ॒तीय॑ स्तृ॒तीयः॒ प्रति॑ । \newline
22. प्रत्ये॒ वैव प्रति॒ प्रत्ये॒व । \newline
23. ए॒व प्र॑थ॒मेन॑ प्रथ॒मे नै॒वैव प्र॑थ॒मेन॑ । \newline
24. प्र॒थ॒मेन॑ धत्ते धत्ते प्रथ॒मेन॑ प्रथ॒मेन॑ धत्ते । \newline
25. ध॒त्ते॒ वि वि ध॑त्ते धत्ते॒ वि । \newline
26. वि सृ॑जति सृजति॒ वि वि सृ॑जति । \newline
27. सृ॒ज॒ति॒ द्वि॒तीये॑न द्वि॒तीये॑न सृजति सृजति द्वि॒तीये॑न । \newline
28. द्वि॒तीये॑न॒ विद्ध्य॑ति॒ विद्ध्य॑ति द्वि॒तीये॑न द्वि॒तीये॑न॒ विद्ध्य॑ति । \newline
29. विद्ध्य॑ति तृ॒तीये॑न तृ॒तीये॑न॒ विद्ध्य॑ति॒ विद्ध्य॑ति तृ॒तीये॑न । \newline
30. तृ॒तीये॒नेन्द्र॒ इन्द्र॑ स्तृ॒तीये॑न तृ॒तीये॒नेन्द्रः॑ । \newline
31. इन्द्रो॑ वृ॒त्रं ॅवृ॒त्र मिन्द्र॒ इन्द्रो॑ वृ॒त्रम् । \newline
32. वृ॒त्रꣳ ह॒त्वा ह॒त्वा वृ॒त्रं ॅवृ॒त्रꣳ ह॒त्वा । \newline
33. ह॒त्वा परा॒म् पराꣳ॑ ह॒त्वा ह॒त्वा परा᳚म् । \newline
34. परा᳚म् परा॒वत॑म् परा॒वत॒म् परा॒म् परा᳚म् परा॒वत᳚म् । \newline
35. प॒रा॒वत॑ मगच्छ दगच्छत् परा॒वत॑म् परा॒वत॑ मगच्छत् । \newline
36. प॒रा॒वत॒मिति॑ परा - वत᳚म् । \newline
37. अ॒ग॒च्छ॒ दपापा॑ गच्छ दगच्छ॒ दप॑ । \newline
38. अपा॑ राध मराध॒ मपापा॑ राधम् । \newline
39. अ॒रा॒ध॒ मिती त्य॑राध मराध॒ मिति॑ । \newline
40. इति॒ मन्य॑मानो॒ मन्य॑मान॒ इतीति॒ मन्य॑मानः । \newline
41. मन्य॑मानः॒ स स मन्य॑मानो॒ मन्य॑मानः॒ सः । \newline
42. स हरि॑तो॒ हरि॑तः॒ स स हरि॑तः । \newline
43. हरि॑तो ऽभव दभव॒ द्धरि॑तो॒ हरि॑तो ऽभवत् । \newline
44. अ॒भ॒व॒थ् स सो॑ ऽभव दभव॒थ् सः । \newline
45. स ए॒ता ने॒तान् थ्स स ए॒तान् । \newline
46. ए॒तान् म॑रुत्व॒तीया᳚न् मरुत्व॒तीया॑ ने॒ता ने॒तान् म॑रुत्व॒तीयान्॑ । \newline
47. म॒रु॒त्व॒तीया॑ नात्म॒स्पर॑णा नात्म॒स्पर॑णान् मरुत्व॒तीया᳚न् मरुत्व॒तीया॑ नात्म॒स्पर॑णान् । \newline
48. आ॒त्म॒स्पर॑णा नपश्य दपश्य दात्म॒स्पर॑णा नात्म॒स्पर॑णा नपश्यत् । \newline
49. आ॒त्म॒स्पर॑णा॒नित्या᳚त्म - स्पर॑णान् । \newline
50. अ॒प॒श्य॒त् ताꣳ स्ता न॑पश्य दपश्य॒त् तान् । \newline
51. तान॑गृह्णीता गृह्णीत॒ ताꣳ स्ता न॑गृह्णीत । \newline
52. अ॒गृ॒ह्णी॒त॒ प्रा॒णम् प्रा॒ण म॑गृह्णीता गृह्णीत प्रा॒णम् । \newline

\textbf{Ghana Paata } \newline

1. य॒च्छ॒ति॒ प्र प्र य॑च्छति यच्छति॒ प्र ह॑रति हरति॒ प्र य॑च्छति यच्छति॒ प्र ह॑रति । \newline
2. प्र ह॑रति हरति॒ प्र प्र ह॑रति द्वि॒तीये॑न द्वि॒तीये॑न हरति॒ प्र प्र ह॑रति द्वि॒तीये॑न । \newline
3. ह॒र॒ति॒ द्वि॒तीये॑न द्वि॒तीये॑न हरति हरति द्वि॒तीये॑न स्तृणु॒ते स्तृ॑णु॒ते द्वि॒तीये॑न हरति हरति द्वि॒तीये॑न स्तृणु॒ते । \newline
4. द्वि॒तीये॑न स्तृणु॒ते स्तृ॑णु॒ते द्वि॒तीये॑न द्वि॒तीये॑न स्तृणु॒ते तृ॒तीये॑न तृ॒तीये॑न स्तृणु॒ते द्वि॒तीये॑न द्वि॒तीये॑न स्तृणु॒ते तृ॒तीये॑न । \newline
5. स्तृ॒णु॒ते तृ॒तीये॑न तृ॒तीये॑न स्तृणु॒ते स्तृ॑णु॒ते तृ॒तीये॒ नायु॑ध॒ मायु॑धम् तृ॒तीये॑न स्तृणु॒ते स्तृ॑णु॒ते तृ॒तीये॒ नायु॑धम् । \newline
6. तृ॒तीये॒ नायु॑ध॒ मायु॑धम् तृ॒तीये॑न तृ॒तीये॒ नायु॑धं॒ ॅवै वा आयु॑धम् तृ॒तीये॑न तृ॒तीये॒ नायु॑धं॒ ॅवै । \newline
7. आयु॑धं॒ ॅवै वा आयु॑ध॒ मायु॑धं॒ ॅवा ए॒त दे॒तद् वा आयु॑ध॒ मायु॑धं॒ ॅवा ए॒तत् । \newline
8. वा ए॒त दे॒तद् वै वा ए॒तद् यज॑मानो॒ यज॑मान ए॒तद् वै वा ए॒तद् यज॑मानः । \newline
9. ए॒तद् यज॑मानो॒ यज॑मान ए॒त दे॒तद् यज॑मानः॒ सꣳ सं ॅयज॑मान ए॒त दे॒तद् यज॑मानः॒ सम् । \newline
10. यज॑मानः॒ सꣳ सं ॅयज॑मानो॒ यज॑मानः॒ सꣳ स्कु॑रुते कुरुते॒ सं ॅयज॑मानो॒ यज॑मानः॒ 
सꣳ स्कु॑रुते । \newline
11. सꣳ स्कु॑रुते कुरुते॒ सꣳ सꣳ स्कु॑रुते॒ यद् यत् कु॑रुते॒ सꣳ सꣳ स्कु॑रुते॒ यत् । \newline
12. कु॒रु॒ते॒ यद् यत् कु॑रुते कुरुते॒ यन् म॑रुत्व॒तीया॑ मरुत्व॒तीया॒ यत् कु॑रुते कुरुते॒ यन् म॑रुत्व॒तीयाः᳚ । \newline
13. यन् म॑रुत्व॒तीया॑ मरुत्व॒तीया॒ यद् यन् म॑रुत्व॒तीया॒ धनु॒र् धनु॑र् मरुत्व॒तीया॒ यद् यन् म॑रुत्व॒तीया॒ धनुः॑ । \newline
14. म॒रु॒त्व॒तीया॒ धनु॒र् धनु॑र् मरुत्व॒तीया॑ मरुत्व॒तीया॒ धनु॑ रे॒वैव धनु॑र् मरुत्व॒तीया॑ मरुत्व॒तीया॒ धनु॑रे॒व । \newline
15. धनु॑ रे॒वैव धनु॒र् धनु॑ रे॒व प्र॑थ॒मः प्र॑थ॒म ए॒व धनु॒र् धनु॑ रे॒व प्र॑थ॒मः । \newline
16. ए॒व प्र॑थ॒मः प्र॑थ॒म ए॒वैव प्र॑थ॒मो ज्या ज्या प्र॑थ॒म ए॒वैव प्र॑थ॒मो ज्या । \newline
17. प्र॒थ॒मो ज्या ज्या प्र॑थ॒मः प्र॑थ॒मो ज्या द्वि॒तीयो᳚ द्वि॒तीयो॒ ज्या प्र॑थ॒मः प्र॑थ॒मो ज्या द्वि॒तीयः॑ । \newline
18. ज्या द्वि॒तीयो᳚ द्वि॒तीयो॒ ज्या ज्या द्वि॒तीय॒ इषु॒ रिषु॑र् द्वि॒तीयो॒ ज्या ज्या द्वि॒तीय॒ इषुः॑ । \newline
19. द्वि॒तीय॒ इषु॒ रिषु॑र् द्वि॒तीयो᳚ द्वि॒तीय॒ इषु॑ स्तृ॒तीय॑ स्तृ॒तीय॒ इषु॑र् द्वि॒तीयो᳚ द्वि॒तीय॒ इषु॑ स्तृ॒तीयः॑ । \newline
20. इषु॑ स्तृ॒तीय॑ स्तृ॒तीय॒ इषु॒ रिषु॑ स्तृ॒तीयः॒ प्रति॒ प्रति॑ तृ॒तीय॒ इषु॒ रिषु॑ स्तृ॒तीयः॒ प्रति॑ । \newline
21. तृ॒तीयः॒ प्रति॒ प्रति॑ तृ॒तीय॑ स्तृ॒तीयः॒ प्रत्ये॒ वैव प्रति॑ तृ॒तीय॑ स्तृ॒तीयः॒ प्रत्ये॒व । \newline
22. प्रत्ये॒ वैव प्रति॒ प्रत्ये॒व प्र॑थ॒मेन॑ प्रथ॒मे नै॒व प्रति॒ प्रत्ये॒व प्र॑थ॒मेन॑ । \newline
23. ए॒व प्र॑थ॒मेन॑ प्रथ॒मे नै॒वैव प्र॑थ॒मेन॑ धत्ते धत्ते प्रथ॒मे नै॒वैव प्र॑थ॒मेन॑ धत्ते । \newline
24. प्र॒थ॒मेन॑ धत्ते धत्ते प्रथ॒मेन॑ प्रथ॒मेन॑ धत्ते॒ वि वि ध॑त्ते प्रथ॒मेन॑ प्रथ॒मेन॑ धत्ते॒ वि । \newline
25. ध॒त्ते॒ वि वि ध॑त्ते धत्ते॒ वि सृ॑जति सृजति॒ वि ध॑त्ते धत्ते॒ वि सृ॑जति । \newline
26. वि सृ॑जति सृजति॒ वि वि सृ॑जति द्वि॒तीये॑न द्वि॒तीये॑न सृजति॒ वि वि सृ॑जति द्वि॒तीये॑न । \newline
27. सृ॒ज॒ति॒ द्वि॒तीये॑न द्वि॒तीये॑न सृजति सृजति द्वि॒तीये॑न॒ विद्ध्य॑ति॒ विद्ध्य॑ति द्वि॒तीये॑न सृजति सृजति द्वि॒तीये॑न॒ विद्ध्य॑ति । \newline
28. द्वि॒तीये॑न॒ विद्ध्य॑ति॒ विद्ध्य॑ति द्वि॒तीये॑न द्वि॒तीये॑न॒ विद्ध्य॑ति तृ॒तीये॑न तृ॒तीये॑न॒ विद्ध्य॑ति द्वि॒तीये॑न द्वि॒तीये॑न॒ विद्ध्य॑ति तृ॒तीये॑न । \newline
29. विद्ध्य॑ति तृ॒तीये॑न तृ॒तीये॑न॒ विद्ध्य॑ति॒ विद्ध्य॑ति तृ॒तीये॒नेन्द्र॒ इन्द्र॑ स्तृ॒तीये॑न॒ विद्ध्य॑ति॒ विद्ध्य॑ति तृ॒तीये॒नेन्द्रः॑ । \newline
30. तृ॒तीये॒नेन्द्र॒ इन्द्र॑ स्तृ॒तीये॑न तृ॒तीये॒नेन्द्रो॑ वृ॒त्रं ॅवृ॒त्र मिन्द्र॑ स्तृ॒तीये॑न तृ॒तीये॒नेन्द्रो॑ वृ॒त्रम् । \newline
31. इन्द्रो॑ वृ॒त्रं ॅवृ॒त्र मिन्द्र॒ इन्द्रो॑ वृ॒त्रꣳ ह॒त्वा ह॒त्वा वृ॒त्र मिन्द्र॒ इन्द्रो॑ वृ॒त्रꣳ ह॒त्वा । \newline
32. वृ॒त्रꣳ ह॒त्वा ह॒त्वा वृ॒त्रं ॅवृ॒त्रꣳ ह॒त्वा परा॒म् पराꣳ॑ ह॒त्वा वृ॒त्रं ॅवृ॒त्रꣳ ह॒त्वा परा᳚म् । \newline
33. ह॒त्वा परा॒म् पराꣳ॑ ह॒त्वा ह॒त्वा परा᳚म् परा॒वत॑म् परा॒वत॒म् पराꣳ॑ ह॒त्वा ह॒त्वा परा᳚म् परा॒वत᳚म् । \newline
34. परा᳚म् परा॒वत॑म् परा॒वत॒म् परा॒म् परा᳚म् परा॒वत॑ मगच्छ दगच्छत् परा॒वत॒म् परा॒म् परा᳚म् परा॒वत॑ मगच्छत् । \newline
35. प॒रा॒वत॑ मगच्छ दगच्छत् परा॒वत॑म् परा॒वत॑ मगच्छ॒ दपापा॑ गच्छत् परा॒वत॑म् परा॒वत॑ मगच्छ॒ दप॑ । \newline
36. प॒रा॒वत॒मिति॑ परा - वत᳚म् । \newline
37. अ॒ग॒च्छ॒ दपापा॑ गच्छ दगच्छ॒ दपा॑राध मराध॒ मपा॑गच्छ दगच्छ॒ दपा॑राधम् । \newline
38. अपा॑ राध मराध॒ मपापा॑ राध॒ मिती त्य॑राध॒ मपापा॑ राध॒ मिति॑ । \newline
39. अ॒रा॒ध॒ मिती त्य॑राध मराध॒ मिति॒ मन्य॑मानो॒ मन्य॑मान॒ इत्य॑राध मराध॒ मिति॒ मन्य॑मानः । \newline
40. इति॒ मन्य॑मानो॒ मन्य॑मान॒ इतीति॒ मन्य॑मानः॒ स स मन्य॑मान॒ इतीति॒ मन्य॑मानः॒ सः । \newline
41. मन्य॑मानः॒ स स मन्य॑मानो॒ मन्य॑मानः॒ स हरि॑तो॒ हरि॑तः॒ स मन्य॑मानो॒ मन्य॑मानः॒ स हरि॑तः । \newline
42. स हरि॑तो॒ हरि॑तः॒ स स हरि॑तो ऽभव दभव॒ द्धरि॑तः॒ स स हरि॑तो ऽभवत् । \newline
43. हरि॑तो ऽभव दभव॒ द्धरि॑तो॒ हरि॑तो ऽभव॒थ् स सो॑ ऽभव॒ द्धरि॑तो॒ हरि॑तो ऽभव॒थ् सः । \newline
44. अ॒भ॒व॒थ् स सो॑ ऽभव दभव॒थ् स ए॒ता-ने॒तान् थ्सो॑ ऽभव-दभव॒थ् स ए॒तान् । \newline
45. स ए॒ता-ने॒तान् थ्स स ए॒तान् म॑रुत्व॒तीया᳚न् मरुत्व॒तीया॑-ने॒तान् थ्स स ए॒तान् म॑रुत्व॒तीयान्॑ । \newline
46. ए॒तान् म॑रुत्व॒तीया᳚न् मरुत्व॒तीया॑-ने॒ता-ने॒तान् म॑रुत्व॒तीया॑-नात्म॒स्पर॑णा-नात्म॒स्पर॑णान् मरुत्व॒तीया॑-ने॒ता-ने॒तान् म॑रुत्व॒तीया॑-नात्म॒स्पर॑णान् । \newline
47. म॒रु॒त्व॒तीया॑-नात्म॒स्पर॑णा-नात्म॒स्पर॑णान् मरुत्व॒तीया᳚न् मरुत्व॒तीया॑-नात्म॒स्पर॑णा-नपश्य दपश्य दात्म॒स्पर॑णान् मरुत्व॒तीया᳚न् मरुत्व॒तीया॑-नात्म॒स्पर॑णा-नपश्यत् । \newline
48. आ॒त्म॒स्पर॑णा-नपश्य दपश्य दात्म॒स्पर॑णा-नात्म॒स्पर॑णा-नपश्य॒त् ताꣳ स्ता-न॑पश्य दात्म॒स्पर॑णा-नात्म॒स्पर॑णा-नपश्य॒त् तान् । \newline
49. आ॒त्म॒स्पर॑णा॒नित्या᳚त्म - स्पर॑णान् । \newline
50. अ॒प॒श्य॒त् ताꣳ स्तान॑पश्य दपश्य॒त् तान॑गृह्णीता गृह्णीत॒ तान॑पश्य दपश्य॒त् तान॑गृह्णीत । \newline
51. तान॑गृह्णीता गृह्णीत॒ ताꣳ स्तान॑गृह्णीत प्रा॒णम् प्रा॒ण म॑गृह्णीत॒ ताꣳ स्तान॑गृह्णीत प्रा॒णम् । \newline
52. अ॒गृ॒ह्णी॒त॒ प्रा॒णम् प्रा॒ण म॑गृह्णीता गृह्णीत प्रा॒ण मे॒वैव प्रा॒ण म॑गृह्णीता गृह्णीत प्रा॒ण मे॒व । \newline
\pagebreak
\markright{ TS 6.5.5.3  \hfill https://www.vedavms.in \hfill}

\section{ TS 6.5.5.3 }

\textbf{TS 6.5.5.3 } \newline
\textbf{Samhita Paata} \newline

प्रा॒णमे॒व प्र॑थ॒मेना᳚-स्पृणुतापा॒नं द्वि॒तीये॑ना॒ऽऽत्मानं॑ तृ॒तीये॑नाऽऽत्म॒स्पर॑णा॒ वा ए॒ते यज॑मानस्य गृह्यन्ते॒ यन्म॑रुत्व॒तीयाः᳚ प्रा॒णमे॒व प्र॑थ॒मेन॑ स्पृणुतेऽपा॒नं द्वि॒तीये॑ना॒ऽऽ*त्मानं॑ तृ॒तीये॒नेन्द्रो॑ वृ॒त्रम॑ह॒न् तं दे॒वा अ॑ब्रुवन् म॒हान्. वा अ॒यम॑भू॒द्यो वृ॒त्रमव॑धी॒दिति॒ तन्म॑हे॒न्द्रस्य॑ महेन्द्र॒त्वꣳ स ए॒तं मा॑हे॒न्द्र-मु॑द्धा॒र-मुद॑हरत वृ॒त्रꣳ ह॒त्वाऽन्यासु॑ दे॒वता॒स्व ( ) धि॒ यन्मा॑हे॒न्द्रो गृ॒ह्यत॑ उद्धा॒रमे॒व तं ॅयज॑मान॒ उद्ध॑रते॒ऽन्यासु॑ प्र॒जास्वधि॑ शुक्रपा॒त्रेण॑ गृह्णाति यजमानदेव॒त्यो॑ वै मा॑हे॒न्द्रस्तेजः॑ शु॒क्रो यन्मा॑हे॒न्द्रꣳ शु॑क्रपा॒त्रेण॑ गृ॒ह्णाति॒ यज॑मान ए॒व तेजो॑ दधाति ॥ \newline

\textbf{Pada Paata} \newline

प्रा॒णमिति॑ प्र - अ॒नम् । ए॒व । प्र॒थ॒मेन॑ । अ॒स्पृ॒णु॒त॒ । अ॒पा॒नमित्य॑प - अ॒नम् । द्वि॒तीये॑न । आ॒त्मान᳚म् । तृ॒तीये॑न । आ॒त्म॒स्पर॑णा॒ इत्या᳚त्म - स्पर॑णाः । वै । ए॒ते । यज॑मानस्य । गृ॒ह्य॒न्ते॒ । यत् । म॒रु॒त्व॒तीयाः᳚ । प्रा॒णमिति॑ प्र - अ॒नम् । ए॒व । प्र॒थ॒मेन॑ । स्पृ॒णु॒ते॒ । अ॒पा॒नमित्य॑प-अ॒नम् । द्वि॒तीये॑न । आ॒त्मान᳚म् । तृ॒तीये॑न । इन्द्रः॑ । वृ॒त्रम् । अ॒ह॒न्न् । तम् । दे॒वाः । अ॒ब्रु॒व॒न्न् । म॒हान् । वै । अ॒यम् । अ॒भू॒त् । यः । वृ॒त्रम् । अव॑धीत् । इति॑ । तत् । म॒हे॒न्द्रस्येति॑ महा - इ॒न्द्रस्य॑ । म॒हे॒न्द्र॒त्वमिति॑ महेन्द्र - त्वम् । सः । ए॒तम् । मा॒हे॒न्द्रमिति॑ माहा - इ॒न्द्रम् । उ॒द्धा॒रमित्यु॑त् - हा॒रम् । उदिति॑ । अ॒ह॒र॒त॒ । वृ॒त्रम् । ह॒त्वा । अ॒न्यासु॑ । दे॒वता॑सु ( ) । अधीति॑ । यत् । मा॒हे॒न्द्र इति॑ माहा - इ॒न्द्रः । गृ॒ह्यते᳚ । उ॒द्धा॒रमित्यु॑त् - हा॒रम् । ए॒व । तम् । यज॑मानः । उदिति॑ । ह॒र॒ते॒ । अ॒न्यासु॑ । प्र॒जास्विति॑ प्र-जासु॑ । अधीति॑ । शु॒क्र॒पा॒त्रेणेति॑ शुक्र - पा॒त्रेण॑ । गृ॒ह्णा॒ति॒ । य॒ज॒मा॒न॒दे॒व॒त्य॑ इति॑ यजमान - दे॒व॒त्यः॑ । वै । मा॒हे॒न्द्र इति॑ माहा - इ॒न्द्रः । तेजः॑ । शु॒क्रः । यत् । मा॒हे॒न्द्रमिति॑ माहा - इ॒न्द्रम् । शु॒क्र॒पा॒त्रेणेति॑ शुक्र - पा॒त्रेण॑ । गृ॒ह्णाति॑ । यज॑माने । ए॒व । तेजः॑ । द॒धा॒ति॒ ॥  \newline


\textbf{Krama Paata} \newline

प्रा॒णमे॒व । प्रा॒णमिति॑ प्र - अ॒नम् । ए॒व प्र॑थ॒मेन॑ । प्र॒थ॒मेना᳚स्पृणुत । अ॒स्पृ॒णु॒ता॒पा॒नम् । अ॒पा॒नम् द्वि॒तीये॑न । अ॒पा॒नमित्य॑प - अ॒नम् । द्वि॒तीये॑ना॒त्मान᳚म् । आ॒त्मान॑म् तृ॒तीये॑न । तृ॒तीये॑नात्म॒स्पर॑णाः । आ॒त्म॒स्पर॑णा॒ वै । आ॒त्म॒स्पर॑णा॒ इत्या᳚त्म - स्पर॑णाः । वा ए॒ते । ए॒ते यज॑मानस्य । यज॑मानस्य गृह्यन्ते । गृ॒ह्य॒न्ते॒ यत् । यन् म॑रुत्व॒तीयाः᳚ । म॒रु॒त्व॒तीयाः᳚ प्रा॒णम् । प्रा॒णमे॒व । प्रा॒णमिति॑ प्र - अ॒नम् । ए॒व प्र॑थ॒मेन॑ । प्र॒थ॒मेन॑ स्पृणुते । स्पृ॒णु॒ते॒ऽपा॒नम् । अ॒पा॒नम् द्वि॒तीये॑न । अ॒पा॒नमित्य॑प - अ॒नम् । द्वि॒तीये॑ना॒त्मान᳚म् । आ॒त्मान॑म् तृ॒तीये॑न । तृ॒तीये॒नेन्द्रः॑ । इन्द्रो॑ वृ॒त्रम् । वृ॒त्रम॑हन्न् । अ॒ह॒न् तम् । तम् दे॒वाः । दे॒वा अ॑ब्रुवन्न् । अ॒ब्रु॒व॒न् म॒हान् । म॒हान्. वै । वा अ॒यम् । अ॒यम॑भूत् । अ॒भू॒द् यः । यो वृ॒त्रम् । वृ॒त्रमव॑धीत् । अव॑धी॒दिति॑ । इति॒ तत् । तन् म॑हे॒न्द्रस्य॑ । म॒हे॒न्द्रस्य॑ महेन्द्र॒त्वम् । म॒हे॒न्द्रस्येति॑ महा - इ॒न्द्रस्य॑ । म॒हे॒न्द्र॒त्वꣳ सः । म॒हे॒न्द्र॒त्वमिति॑ महेन्द्र - त्वम् । स ए॒तम् । ए॒तम् मा॑हे॒न्द्रम् । मा॒हे॒न्द्रमु॑द्धा॒रम् । मा॒हे॒न्द्रमिति॑ माहा - इ॒न्द्रम् । उ॒द्धा॒रमुत् । उ॒द्धा॒रमित्यु॑त् - हा॒रम् । उद॑हरत । अ॒ह॒र॒त॒ वृ॒त्रम् । वृ॒त्रꣳ ह॒त्वा । ह॒त्वाऽन्यासु॑ । अ॒न्यासु॑ दे॒वता॑सु ( ) । दे॒वता॒स्वधि॑ । अधि॒ यत् । यन् मा॑हे॒न्द्रः । मा॒हे॒न्द्रो गृ॒ह्यते᳚ । मा॒हे॒न्द्र इति॑ माहा - इ॒न्द्रः । गृ॒ह्यत॑ उद्धा॒रम् । उ॒द्धा॒रमे॒व । उ॒द्धा॒रमित्यु॑त् - हा॒रम् । ए॒व तम् । तम् ॅयज॑मानः । यज॑मान॒ उत् । उद्‌ध॑रते । ह॒र॒ते॒ऽन्यासु॑ । अ॒न्यासु॑ प्र॒जासु॑ । प्र॒जास्वधि॑ । प्र॒जास्विति॑ प्र - जासु॑ । अधि॑ शुक्रपा॒त्रेण॑ । शु॒क्र॒पा॒त्रेण॑ गृह्णाति । शु॒क्र॒पा॒त्रेणेति॑ शुक्र - पा॒त्रेण॑ । गृ॒ह्णा॒ति॒ य॒ज॒मा॒न॒दे॒व॒त्यः॑ । य॒ज॒मा॒न॒दे॒व॒त्यो॑ वै । य॒ज॒मा॒न॒दे॒व॒त्य॑ इति॑ यजमान - दे॒व॒त्यः॑ । वै मा॑हे॒न्द्रः । मा॒हे॒न्द्रस्तेजः॑ । मा॒हे॒न्द्र इति॑ माहा - इ॒न्द्रः । तेजः॑ शु॒क्रः । शु॒क्रो यत् । यन् मा॑हे॒न्द्रम् । मा॒हे॒न्द्रꣳ शु॑क्रपा॒त्रेण॑ । मा॒हे॒न्द्रमिति॑ माहा - इ॒न्द्रम् । शु॒क्र॒पा॒त्रेण॑ गृ॒ह्णाति॑ । शु॒क्र॒पा॒त्रेणेति॑ शुक्र - पा॒त्रेण॑ । गृ॒ह्णाति॒ यज॑माने । यज॑मान ए॒व । ए॒व तेजः॑ । तेजो॑ दधाति । द॒धा॒तीति॑ दधाति । \newline

\textbf{Jatai Paata} \newline

1. प्रा॒ण मे॒वैव प्रा॒णम् प्रा॒ण मे॒व । \newline
2. प्रा॒णमिति॑ प्र - अ॒नम् । \newline
3. ए॒व प्र॑थ॒मेन॑ प्रथ॒मेनै॒ वैव प्र॑थ॒मेन॑ । \newline
4. प्र॒थ॒मेना᳚ स्पृणुता स्पृणुत प्रथ॒मेन॑ प्रथ॒मेना᳚ स्पृणुत । \newline
5. अ॒स्पृ॒णु॒ता॒ पा॒न म॑पा॒न म॑स्पृणुता स्पृणुता पा॒नम् । \newline
6. अ॒पा॒नम् द्वि॒तीये॑न द्वि॒तीये॑ना पा॒न म॑पा॒नम् द्वि॒तीये॑न । \newline
7. अ॒पा॒नमित्य॑प - अ॒नम् । \newline
8. द्वि॒तीये॑ ना॒त्मान॑ मा॒त्मान॑म् द्वि॒तीये॑न द्वि॒तीये॑ ना॒त्मान᳚म् । \newline
9. आ॒त्मान॑म् तृ॒तीये॑न तृ॒तीये॑ ना॒त्मान॑ मा॒त्मान॑म् तृ॒तीये॑न । \newline
10. तृ॒तीये॑ नात्म॒स्पर॑णा आत्म॒स्पर॑णा स्तृ॒तीये॑न तृ॒तीये॑ नात्म॒स्पर॑णाः । \newline
11. आ॒त्म॒स्पर॑णा॒ वै वा आ᳚त्म॒स्पर॑णा आत्म॒स्पर॑णा॒ वै । \newline
12. आ॒त्म॒स्पर॑णा॒ इत्या᳚त्म - स्पर॑णाः । \newline
13. वा ए॒त ए॒ते वै वा ए॒ते । \newline
14. ए॒ते यज॑मानस्य॒ यज॑मान स्यै॒त ए॒ते यज॑मानस्य । \newline
15. यज॑मानस्य गृह्यन्ते गृह्यन्ते॒ यज॑मानस्य॒ यज॑मानस्य गृह्यन्ते । \newline
16. गृ॒ह्य॒न्ते॒ यद् यद् गृ॑ह्यन्ते गृह्यन्ते॒ यत् । \newline
17. यन् म॑रुत्व॒तीया॑ मरुत्व॒तीया॒ यद् यन् म॑रुत्व॒तीयाः᳚ । \newline
18. म॒रु॒त्व॒तीयाः᳚ प्रा॒णम् प्रा॒णम् म॑रुत्व॒तीया॑ मरुत्व॒तीयाः᳚ प्रा॒णम् । \newline
19. प्रा॒ण मे॒वैव प्रा॒णम् प्रा॒ण मे॒व । \newline
20. प्रा॒णमिति॑ प्र - अ॒नम् । \newline
21. ए॒व प्र॑थ॒मेन॑ प्रथ॒मे नै॒वैव प्र॑थ॒मेन॑ । \newline
22. प्र॒थ॒मेन॑ स्पृणुते स्पृणुते प्रथ॒मेन॑ प्रथ॒मेन॑ स्पृणुते । \newline
23. स्पृ॒णु॒ते॒ ऽपा॒न म॑पा॒नꣳ स्पृ॑णुते स्पृणुते ऽपा॒नम् । \newline
24. अ॒पा॒नम् द्वि॒तीये॑न द्वि॒तीये॑ना पा॒न म॑पा॒नम् द्वि॒तीये॑न । \newline
25. अ॒पा॒नमित्य॑प - अ॒नम् । \newline
26. द्वि॒तीये॑ ना॒त्मान॑ मा॒त्मान॑म् द्वि॒तीये॑न द्वि॒तीये॑ ना॒त्मान᳚म् । \newline
27. आ॒त्मान॑म् तृ॒तीये॑न तृ॒तीये॑ ना॒त्मान॑ मा॒त्मान॑म् तृ॒तीये॑न । \newline
28. तृ॒तीये॒नेन्द्र॒ इन्द्र॑ स्तृ॒तीये॑न तृ॒तीये॒नेन्द्रः॑ । \newline
29. इन्द्रो॑ वृ॒त्रं ॅवृ॒त्र मिन्द्र॒ इन्द्रो॑ वृ॒त्रम् । \newline
30. वृ॒त्र म॑हन् नहन् वृ॒त्रं ॅवृ॒त्र म॑हन्न् । \newline
31. अ॒ह॒न् तम् तम॑हन् नह॒न् तम् । \newline
32. तम् दे॒वा दे॒वा स्तम् तम् दे॒वाः । \newline
33. दे॒वा अ॑ब्रुवन् नब्रुवन् दे॒वा दे॒वा अ॑ब्रुवन्न् । \newline
34. अ॒ब्रु॒व॒न् म॒हान् म॒हा न॑ब्रुवन् नब्रुवन् म॒हान् । \newline
35. म॒हान्. वै वै म॒हान् म॒हान्. वै । \newline
36. वा अ॒य म॒यं ॅवै वा अ॒यम् । \newline
37. अ॒य म॑भू दभू द॒य म॒य म॑भूत् । \newline
38. अ॒भू॒द् यो यो॑ ऽभू दभू॒द् यः । \newline
39. यो वृ॒त्रं ॅवृ॒त्रं ॅयो यो वृ॒त्रम् । \newline
40. वृ॒त्र मव॑धी॒ दव॑धीद् वृ॒त्रं ॅवृ॒त्र मव॑धीत् । \newline
41. अव॑धी॒ दितीत्य व॑धी॒ दव॑धी॒ दिति॑ । \newline
42. इति॒ तत् तदि तीति॒ तत् । \newline
43. तन् म॑हे॒न्द्रस्य॑ महे॒न्द्रस्य॒ तत् तन् म॑हे॒न्द्रस्य॑ । \newline
44. म॒हे॒न्द्रस्य॑ महेन्द्र॒त्वम् म॑हेन्द्र॒त्वम् म॑हे॒न्द्रस्य॑ महे॒न्द्रस्य॑ महेन्द्र॒त्वम् । \newline
45. म॒हे॒न्द्रस्येति॑ महा - इ॒न्द्रस्य॑ । \newline
46. म॒हे॒न्द्र॒त्वꣳ स स म॑हेन्द्र॒त्वम् म॑हेन्द्र॒त्वꣳ सः । \newline
47. म॒हे॒न्द्र॒त्वमिति॑ महेन्द्र - त्वम् । \newline
48. स ए॒त मे॒तꣳ स स ए॒तम् । \newline
49. ए॒तम् मा॑हे॒न्द्रम् मा॑हे॒न्द्र मे॒त मे॒तम् मा॑हे॒न्द्रम् । \newline
50. मा॒हे॒न्द्र मु॑द्धा॒र मु॑द्धा॒रम् मा॑हे॒न्द्रम् मा॑हे॒न्द्र मु॑द्धा॒रम् । \newline
51. मा॒हे॒न्द्रमिति॑ माहा - इ॒न्द्रम् । \newline
52. उ॒द्धा॒र मुदु दु॑द्धा॒र मु॑द्धा॒र मुत् । \newline
53. उ॒द्धा॒रमित्यु॑त् - हा॒रम् । \newline
54. उद॑हरता हर॒तो दुद॑ हरत । \newline
55. अ॒ह॒र॒त॒ वृ॒त्रं ॅवृ॒त्र म॑हरता हरत वृ॒त्रम् । \newline
56. वृ॒त्रꣳ ह॒त्वा ह॒त्वा वृ॒त्रं ॅवृ॒त्रꣳ ह॒त्वा । \newline
57. ह॒त्वा ऽन्या स्व॒न्यासु॑ ह॒त्वा ह॒त्वा ऽन्यासु॑ । \newline
58. अ॒न्यासु॑ दे॒वता॑सु दे॒वता᳚ स्व॒न्या स्व॒न्यासु॑ दे॒वता॑सु । \newline
59. दे॒वता॒ स्वध्यधि॑ दे॒वता॑सु दे॒वता॒ स्वधि॑ । \newline
60. अधि॒ यद् यदध्यधि॒ यत् । \newline
61. यन् मा॑हे॒न्द्रो मा॑हे॒न्द्रो यद् यन् मा॑हे॒न्द्रः । \newline
62. मा॒हे॒न्द्रो गृ॒ह्यते॑ गृ॒ह्यते॑ माहे॒न्द्रो मा॑हे॒न्द्रो गृ॒ह्यते᳚ । \newline
63. मा॒हे॒न्द्र इति॑ माहा - इ॒न्द्रः । \newline
64. गृ॒ह्यत॑ उद्धा॒र मु॑द्धा॒रम् गृ॒ह्यते॑ गृ॒ह्यत॑ उद्धा॒रम् । \newline
65. उ॒द्धा॒र मे॒वै वोद्धा॒र मु॑द्धा॒र मे॒व । \newline
66. उ॒द्धा॒रमित्यु॑त् - हा॒रम् । \newline
67. ए॒व तम् त मे॒वैव तम् । \newline
68. तं ॅयज॑मानो॒ यज॑मान॒ स्तम् तं ॅयज॑मानः । \newline
69. यज॑मान॒ उदुद् यज॑मानो॒ यज॑मान॒ उत् । \newline
70. उद्ध॑रते हरत॒ उदुद्ध॑रते । \newline
71. ह॒र॒ते॒ ऽन्या स्व॒न्यासु॑ हरते हरते॒ ऽन्यासु॑ । \newline
72. अ॒न्यासु॑ प्र॒जासु॑ प्र॒जा स्व॒न्या स्व॒न्यासु॑ प्र॒जासु॑ । \newline
73. प्र॒जा स्वध्यधि॑ प्र॒जासु॑ प्र॒जा स्वधि॑ । \newline
74. प्र॒जास्विति॑ प्र - जासु॑ । \newline
75. अधि॑ शुक्रपा॒त्रेण॑ शुक्रपा॒त्रेणा ध्यधि॑ शुक्रपा॒त्रेण॑ । \newline
76. शु॒क्र॒पा॒त्रेण॑ गृह्णाति गृह्णाति शुक्रपा॒त्रेण॑ शुक्रपा॒त्रेण॑ गृह्णाति । \newline
77. शु॒क्र॒पा॒त्रेणेति॑ शुक्र - पा॒त्रेण॑ । \newline
78. गृ॒ह्णा॒ति॒ य॒ज॒मा॒न॒दे॒व॒त्यो॑ यजमानदेव॒त्यो॑ गृह्णाति गृह्णाति यजमानदेव॒त्यः॑ । \newline
79. य॒ज॒मा॒न॒दे॒व॒त्यो॑ वै वै य॑जमानदेव॒त्यो॑ यजमानदेव॒त्यो॑ वै । \newline
80. य॒ज॒मा॒न॒दे॒व॒त्य॑ इति॑ यजमान - दे॒व॒त्यः॑ । \newline
81. वै मा॑हे॒न्द्रो मा॑हे॒न्द्रो वै वै मा॑हे॒न्द्रः । \newline
82. मा॒हे॒न्द्र स्तेज॒ स्तेजो॑ माहे॒न्द्रो मा॑हे॒न्द्र स्तेजः॑ । \newline
83. मा॒हे॒न्द्र इति॑ माहा - इ॒न्द्रः । \newline
84. तेजः॑ शु॒क्रः शु॒क्र स्तेज॒ स्तेजः॑ शु॒क्रः । \newline
85. शु॒क्रो यद् यच् छु॒क्रः शु॒क्रो यत् । \newline
86. यन् मा॑हे॒न्द्रम् मा॑हे॒न्द्रं ॅयद् यन् मा॑हे॒न्द्रम् । \newline
87. मा॒हे॒न्द्रꣳ शु॑क्रपा॒त्रेण॑ शुक्रपा॒त्रेण॑ माहे॒न्द्रम् मा॑हे॒न्द्रꣳ शु॑क्रपा॒त्रेण॑ । \newline
88. मा॒हे॒न्द्रमिति॑ माहा - इ॒न्द्रम् । \newline
89. शु॒क्र॒पा॒त्रेण॑ गृ॒ह्णाति॑ गृ॒ह्णाति॑ शुक्रपा॒त्रेण॑ शुक्रपा॒त्रेण॑ गृ॒ह्णाति॑ । \newline
90. शु॒क्र॒पा॒त्रेणेति॑ शुक्र - पा॒त्रेण॑ । \newline
91. गृ॒ह्णाति॒ यज॑माने॒ यज॑माने गृ॒ह्णाति॑ गृ॒ह्णाति॒ यज॑माने । \newline
92. यज॑मान ए॒वैव यज॑माने॒ यज॑मान ए॒व । \newline
93. ए॒व तेज॒ स्तेज॑ ए॒वैव तेजः॑ । \newline
94. तेजो॑ दधाति दधाति॒ तेज॒ स्तेजो॑ दधाति । \newline
95. द॒धा॒तीति॑ दधाति । \newline

\textbf{Ghana Paata } \newline

1. प्रा॒ण मे॒वैव प्रा॒णम् प्रा॒ण मे॒व प्र॑थ॒मेन॑ प्रथ॒मेनै॒व प्रा॒णम् प्रा॒ण मे॒व प्र॑थ॒मेन॑ । \newline
2. प्रा॒णमिति॑ प्र - अ॒नम् । \newline
3. ए॒व प्र॑थ॒मेन॑ प्रथ॒मे नै॒वैव प्र॑थ॒मेना᳚ स्पृणुता स्पृणुत प्रथ॒मे नै॒वैव प्र॑थ॒मेना᳚ स्पृणुत । \newline
4. प्र॒थ॒मेना᳚ स्पृणुता स्पृणुत प्रथ॒मेन॑ प्रथ॒मेना᳚ स्पृणुता पा॒न म॑पा॒न म॑स्पृणुत प्रथ॒मेन॑ प्रथ॒मेना᳚ स्पृणुता पा॒नम् । \newline
5. अ॒स्पृ॒णु॒ता॒ पा॒न म॑पा॒न म॑स्पृणुता स्पृणुता पा॒नम् द्वि॒तीये॑न द्वि॒तीये॑ना पा॒न म॑स्पृणुता स्पृणुता पा॒नम् द्वि॒तीये॑न । \newline
6. अ॒पा॒नम् द्वि॒तीये॑न द्वि॒तीये॑ना पा॒न म॑पा॒नम् द्वि॒तीये॑ ना॒त्मान॑ मा॒त्मान॑म् द्वि॒तीये॑ना पा॒न म॑पा॒नम् द्वि॒तीये॑ ना॒त्मान᳚म् । \newline
7. अ॒पा॒नमित्य॑प - अ॒नम् । \newline
8. द्वि॒तीये॑ ना॒त्मान॑ मा॒त्मान॑म् द्वि॒तीये॑न द्वि॒तीये॑ ना॒त्मान॑म् तृ॒तीये॑न तृ॒तीये॑ ना॒त्मान॑म् द्वि॒तीये॑न द्वि॒तीये॑ ना॒त्मान॑म् तृ॒तीये॑न । \newline
9. आ॒त्मान॑म् तृ॒तीये॑न तृ॒तीये॑ ना॒त्मान॑ मा॒त्मान॑म् तृ॒तीये॑ नात्म॒स्पर॑णा आत्म॒स्पर॑णा स्तृ॒तीये॑ ना॒त्मान॑ मा॒त्मान॑म् तृ॒तीये॑ नात्म॒स्पर॑णाः । \newline
10. तृ॒तीये॑ नात्म॒स्पर॑णा आत्म॒स्पर॑णा स्तृ॒तीये॑न तृ॒तीये॑ नात्म॒स्पर॑णा॒ वै वा आ᳚त्म॒स्पर॑णा स्तृ॒तीये॑न तृ॒तीये॑ नात्म॒स्पर॑णा॒ वै । \newline
11. आ॒त्म॒स्पर॑णा॒ वै वा आ᳚त्म॒स्पर॑णा आत्म॒स्पर॑णा॒ वा ए॒त ए॒ते वा आ᳚त्म॒स्पर॑णा आत्म॒स्पर॑णा॒ वा ए॒ते । \newline
12. आ॒त्म॒स्पर॑णा॒ इत्या᳚त्म - स्पर॑णाः । \newline
13. वा ए॒त ए॒ते वै वा ए॒ते यज॑मानस्य॒ यज॑मान स्यै॒ते वै वा ए॒ते यज॑मानस्य । \newline
14. ए॒ते यज॑मानस्य॒ यज॑मान स्यै॒त ए॒ते यज॑मानस्य गृह्यन्ते गृह्यन्ते॒ यज॑मान स्यै॒त ए॒ते यज॑मानस्य गृह्यन्ते । \newline
15. यज॑मानस्य गृह्यन्ते गृह्यन्ते॒ यज॑मानस्य॒ यज॑मानस्य गृह्यन्ते॒ यद् यद् गृ॑ह्यन्ते॒ यज॑मानस्य॒ यज॑मानस्य गृह्यन्ते॒ यत् । \newline
16. गृ॒ह्य॒न्ते॒ यद् यद् गृ॑ह्यन्ते गृह्यन्ते॒ यन् म॑रुत्व॒तीया॑ मरुत्व॒तीया॒ यद् गृ॑ह्यन्ते गृह्यन्ते॒ यन् म॑रुत्व॒तीयाः᳚ । \newline
17. यन् म॑रुत्व॒तीया॑ मरुत्व॒तीया॒ यद् यन् म॑रुत्व॒तीयाः᳚ प्रा॒णम् प्रा॒णम् म॑रुत्व॒तीया॒ यद् यन् म॑रुत्व॒तीयाः᳚ प्रा॒णम् । \newline
18. म॒रु॒त्व॒तीयाः᳚ प्रा॒णम् प्रा॒णम् म॑रुत्व॒तीया॑ मरुत्व॒तीयाः᳚ प्रा॒ण मे॒वैव प्रा॒णम् म॑रुत्व॒तीया॑ मरुत्व॒तीयाः᳚ प्रा॒ण मे॒व । \newline
19. प्रा॒ण मे॒वैव प्रा॒णम् प्रा॒ण मे॒व प्र॑थ॒मेन॑ प्रथ॒मे नै॒व प्रा॒णम् प्रा॒ण मे॒व प्र॑थ॒मेन॑ । \newline
20. प्रा॒णमिति॑ प्र - अ॒नम् । \newline
21. ए॒व प्र॑थ॒मेन॑ प्रथ॒मे नै॒वैव प्र॑थ॒मेन॑ स्पृणुते स्पृणुते प्रथ॒मे नै॒वैव प्र॑थ॒मेन॑ स्पृणुते । \newline
22. प्र॒थ॒मेन॑ स्पृणुते स्पृणुते प्रथ॒मेन॑ प्रथ॒मेन॑ स्पृणुते ऽपा॒न म॑पा॒नꣳ स्पृ॑णुते प्रथ॒मेन॑ प्रथ॒मेन॑ स्पृणुते ऽपा॒नम् । \newline
23. स्पृ॒णु॒ते॒ ऽपा॒न म॑पा॒नꣳ स्पृ॑णुते स्पृणुते ऽपा॒नम् द्वि॒तीये॑न द्वि॒तीये॑ना पा॒नꣳ स्पृ॑णुते स्पृणुते ऽपा॒नम् द्वि॒तीये॑न । \newline
24. अ॒पा॒नम् द्वि॒तीये॑न द्वि॒तीये॑ना पा॒न म॑पा॒नम् द्वि॒तीये॑ ना॒त्मान॑ मा॒त्मान॑म् द्वि॒तीये॑ना पा॒न म॑पा॒नम् द्वि॒तीये॑ ना॒त्मान᳚म् । \newline
25. अ॒पा॒नमित्य॑प - अ॒नम् । \newline
26. द्वि॒तीये॑ ना॒त्मान॑ मा॒त्मान॑म् द्वि॒तीये॑न द्वि॒तीये॑ ना॒त्मान॑म् तृ॒तीये॑न तृ॒तीये॑ ना॒त्मान॑म् द्वि॒तीये॑न द्वि॒तीये॑ ना॒त्मान॑म् तृ॒तीये॑न । \newline
27. आ॒त्मान॑म् तृ॒तीये॑न तृ॒तीये॑ ना॒त्मान॑ मा॒त्मान॑म् तृ॒तीये॒नेन्द्र॒ इन्द्र॑ स्तृ॒तीये॑ ना॒त्मान॑ मा॒त्मान॑म् तृ॒तीये॒नेन्द्रः॑ । \newline
28. तृ॒तीये॒नेन्द्र॒ इन्द्र॑ स्तृ॒तीये॑न तृ॒तीये॒नेन्द्रो॑ वृ॒त्रं ॅवृ॒त्र मिन्द्र॑ स्तृ॒तीये॑न तृ॒तीये॒नेन्द्रो॑ वृ॒त्रम् । \newline
29. इन्द्रो॑ वृ॒त्रं ॅवृ॒त्र मिन्द्र॒ इन्द्रो॑ वृ॒त्र म॑हन्-नहन् वृ॒त्र मिन्द्र॒ इन्द्रो॑ वृ॒त्र म॑हन्न् । \newline
30. वृ॒त्र म॑हन्-नहन् वृ॒त्रं ॅवृ॒त्र म॑ह॒न् तम् त म॑हन् वृ॒त्रं ॅवृ॒त्र म॑ह॒न् तम् । \newline
31. अ॒ह॒न् तम् त म॑हन्-नह॒न् तम् दे॒वा दे॒वा स्त म॑हन्-नह॒न् तम् दे॒वाः । \newline
32. तम् दे॒वा दे॒वा स्तम् तम् दे॒वा अ॑ब्रुवन्-नब्रुवन् दे॒वा स्तम् तम् दे॒वा अ॑ब्रुवन्न् । \newline
33. दे॒वा अ॑ब्रुवन्-नब्रुवन् दे॒वा दे॒वा अ॑ब्रुवन् म॒हान् म॒हा-न॑ब्रुवन् दे॒वा दे॒वा अ॑ब्रुवन् म॒हान् । \newline
34. अ॒ब्रु॒व॒न् म॒हान् म॒हा-न॑ब्रुवन्-नब्रुवन् म॒हान्. वै वै म॒हा-न॑ब्रुवन्-नब्रुवन् म॒हान्. वै । \newline
35. म॒हान्. वै वै म॒हान् म॒हान्. वा अ॒य म॒यं ॅवै म॒हान् म॒हान्. वा अ॒यम् । \newline
36. वा अ॒य म॒यं ॅवै वा अ॒य म॑भू दभू द॒यं ॅवै वा अ॒य म॑भूत् । \newline
37. अ॒य म॑भू दभू द॒य म॒य म॑भू॒द् यो यो॑ ऽभू द॒य म॒य म॑भू॒द् यः । \newline
38. अ॒भू॒द् यो यो॑ ऽभू दभू॒द् यो वृ॒त्रं ॅवृ॒त्रं ॅयो॑ ऽभू दभू॒द् यो वृ॒त्रम् । \newline
39. यो वृ॒त्रं ॅवृ॒त्रं ॅयो यो वृ॒त्र मव॑धी॒ दव॑धीद् वृ॒त्रं ॅयो यो वृ॒त्र मव॑धीत् । \newline
40. वृ॒त्र मव॑धी॒ दव॑धीद् वृ॒त्रं ॅवृ॒त्र मव॑धी॒ दिती त्यव॑धीद् वृ॒त्रं ॅवृ॒त्र मव॑धी॒ दिति॑ । \newline
41. अव॑धी॒ दिती त्यव॑धी॒ दव॑धी॒ दिति॒ तत् तदित्यव॑धी॒ दव॑धी॒ दिति॒ तत् । \newline
42. इति॒ तत् तदितीति॒ तन् म॑हे॒न्द्रस्य॑ महे॒न्द्रस्य॒ तदितीति॒ तन् म॑हे॒न्द्रस्य॑ । \newline
43. तन् म॑हे॒न्द्रस्य॑ महे॒न्द्रस्य॒ तत् तन् म॑हे॒न्द्रस्य॑ महेन्द्र॒त्वम् म॑हेन्द्र॒त्वम् म॑हे॒न्द्रस्य॒ तत् तन् म॑हे॒न्द्रस्य॑ महेन्द्र॒त्वम् । \newline
44. म॒हे॒न्द्रस्य॑ महेन्द्र॒त्वम् म॑हेन्द्र॒त्वम् म॑हे॒न्द्रस्य॑ महे॒न्द्रस्य॑ महेन्द्र॒त्वꣳ स स म॑हेन्द्र॒त्वम् म॑हे॒न्द्रस्य॑ महे॒न्द्रस्य॑ महेन्द्र॒त्वꣳ सः । \newline
45. म॒हे॒न्द्रस्येति॑ महा - इ॒न्द्रस्य॑ । \newline
46. म॒हे॒न्द्र॒त्वꣳ स स म॑हेन्द्र॒त्वम् म॑हेन्द्र॒त्वꣳ स ए॒त मे॒तꣳ स म॑हेन्द्र॒त्वम् म॑हेन्द्र॒त्वꣳ स ए॒तम् । \newline
47. म॒हे॒न्द्र॒त्वमिति॑ महेन्द्र - त्वम् । \newline
48. स ए॒त मे॒तꣳ स स ए॒तम् मा॑हे॒न्द्रम् मा॑हे॒न्द्र मे॒तꣳ स स ए॒तम् मा॑हे॒न्द्रम् । \newline
49. ए॒तम् मा॑हे॒न्द्रम् मा॑हे॒न्द्र मे॒त मे॒तम् मा॑हे॒न्द्र मु॑द्धा॒र मु॑द्धा॒रम् मा॑हे॒न्द्र मे॒त मे॒तम् मा॑हे॒न्द्र मु॑द्धा॒रम् । \newline
50. मा॒हे॒न्द्र मु॑द्धा॒र मु॑द्धा॒रम् मा॑हे॒न्द्रम् मा॑हे॒न्द्र मु॑द्धा॒र मुदु दु॑द्धा॒रम् मा॑हे॒न्द्रम् मा॑हे॒न्द्र मु॑द्धा॒र मुत् । \newline
51. मा॒हे॒न्द्रमिति॑ माहा - इ॒न्द्रम् । \newline
52. उ॒द्धा॒र मुदु दु॑द्धा॒र मु॑द्धा॒र मुद॑हरता हर॒तो दु॑द्धा॒र मु॑द्धा॒र मुद॑हरत । \newline
53. उ॒द्धा॒रमित्यु॑त् - हा॒रम् । \newline
54. उद॑हरता हर॒तोदु द॑हरत वृ॒त्रं ॅवृ॒त्र म॑हर॒तोदु द॑हरत वृ॒त्रम् । \newline
55. अ॒ह॒र॒त॒ वृ॒त्रं ॅवृ॒त्र म॑हरता हरत वृ॒त्रꣳ ह॒त्वा ह॒त्वा वृ॒त्र म॑हरता हरत वृ॒त्रꣳ ह॒त्वा । \newline
56. वृ॒त्रꣳ ह॒त्वा ह॒त्वा वृ॒त्रं ॅवृ॒त्रꣳ ह॒त्वा ऽन्यास्व॒ न्यासु॑ ह॒त्वा वृ॒त्रं ॅवृ॒त्रꣳ ह॒त्वा ऽन्यासु॑ । \newline
57. ह॒त्वा ऽन्या स्व॒न्यासु॑ ह॒त्वा ह॒त्वा ऽन्यासु॑ दे॒वता॑सु दे॒वता᳚ स्व॒न्यासु॑ ह॒त्वा ह॒त्वा ऽन्यासु॑ दे॒वता॑सु । \newline
58. अ॒न्यासु॑ दे॒वता॑सु दे॒वता᳚ स्व॒न्या स्व॒न्यासु॑ दे॒वता॒ स्वध्यधि॑ दे॒वता᳚ स्व॒न्या स्व॒न्यासु॑ दे॒वता॒ स्वधि॑ । \newline
59. दे॒वता॒ स्वध्यधि॑ दे॒वता॑सु दे॒वता॒ स्वधि॒ यद् यदधि॑ दे॒वता॑सु दे॒वता॒ स्वधि॒ यत् । \newline
60. अधि॒ यद् यदध्यधि॒ यन् मा॑हे॒न्द्रो मा॑हे॒न्द्रो यदध्यधि॒ यन् मा॑हे॒न्द्रः । \newline
61. यन् मा॑हे॒न्द्रो मा॑हे॒न्द्रो यद् यन् मा॑हे॒न्द्रो गृ॒ह्यते॑ गृ॒ह्यते॑ माहे॒न्द्रो यद् यन् मा॑हे॒न्द्रो गृ॒ह्यते᳚ । \newline
62. मा॒हे॒न्द्रो गृ॒ह्यते॑ गृ॒ह्यते॑ माहे॒न्द्रो मा॑हे॒न्द्रो गृ॒ह्यत॑ उद्धा॒र मु॑द्धा॒रम् गृ॒ह्यते॑ माहे॒न्द्रो मा॑हे॒न्द्रो गृ॒ह्यत॑ उद्धा॒रम् । \newline
63. मा॒हे॒न्द्र इति॑ माहा - इ॒न्द्रः । \newline
64. गृ॒ह्यत॑ उद्धा॒र मु॑द्धा॒रम् गृ॒ह्यते॑ गृ॒ह्यत॑ उद्धा॒र मे॒वै वोद्धा॒रम् गृ॒ह्यते॑ गृ॒ह्यत॑ उद्धा॒र मे॒व । \newline
65. उ॒द्धा॒र मे॒वै वोद्धा॒र मु॑द्धा॒र मे॒व तम् त मे॒वोद्धा॒र मु॑द्धा॒र मे॒व तम् । \newline
66. उ॒द्धा॒रमित्यु॑त् - हा॒रम् । \newline
67. ए॒व तम् त मे॒वैव तं ॅयज॑मानो॒ यज॑मान॒ स्त मे॒वैव तं ॅयज॑मानः । \newline
68. तं ॅयज॑मानो॒ यज॑मान॒ स्तम् तं ॅयज॑मान॒ उदुद् यज॑मान॒ स्तम् तं ॅयज॑मान॒ उत् । \newline
69. यज॑मान॒ उदुद् यज॑मानो॒ यज॑मान॒ उद्ध॑रते हरत॒ उद् यज॑मानो॒ यज॑मान॒ उद्ध॑रते । \newline
70. उद्ध॑रते हरत॒ उदुद्ध॑रते॒ ऽन्या स्व॒न्यासु॑ हरत॒ उदु द्ध॑रते॒ ऽन्यासु॑ । \newline
71. ह॒र॒ते॒ ऽन्या स्व॒न्यासु॑ हरते हरते॒ ऽन्यासु॑ प्र॒जासु॑ प्र॒जा स्व॒न्यासु॑ हरते हरते॒ ऽन्यासु॑ प्र॒जासु॑ । \newline
72. अ॒न्यासु॑ प्र॒जासु॑ प्र॒जा स्व॒न्या स्व॒न्यासु॑ प्र॒जास्व ध्यधि॑ प्र॒जा स्व॒न्या स्व॒न्यासु॑ प्र॒जा स्वधि॑ । \newline
73. प्र॒जा स्वध्यधि॑ प्र॒जासु॑ प्र॒जा स्वधि॑ शुक्रपा॒त्रेण॑ शुक्रपा॒त्रेणाधि॑ प्र॒जासु॑ प्र॒जा स्वधि॑ शुक्रपा॒त्रेण॑ । \newline
74. प्र॒जास्विति॑ प्र - जासु॑ । \newline
75. अधि॑ शुक्रपा॒त्रेण॑ शुक्रपा॒त्रेणा ध्यधि॑ शुक्रपा॒त्रेण॑ गृह्णाति गृह्णाति शुक्रपा॒त्रेणा ध्यधि॑ शुक्रपा॒त्रेण॑ गृह्णाति । \newline
76. शु॒क्र॒पा॒त्रेण॑ गृह्णाति गृह्णाति शुक्रपा॒त्रेण॑ शुक्रपा॒त्रेण॑ गृह्णाति यजमानदेव॒त्यो॑ यजमानदेव॒त्यो॑ गृह्णाति शुक्रपा॒त्रेण॑ शुक्रपा॒त्रेण॑ गृह्णाति यजमानदेव॒त्यः॑ । \newline
77. शु॒क्र॒पा॒त्रेणेति॑ शुक्र - पा॒त्रेण॑ । \newline
78. गृ॒ह्णा॒ति॒ य॒ज॒मा॒न॒दे॒व॒त्यो॑ यजमानदेव॒त्यो॑ गृह्णाति गृह्णाति यजमानदेव॒त्यो॑ वै वै य॑जमानदेव॒त्यो॑ गृह्णाति गृह्णाति यजमानदेव॒त्यो॑ वै । \newline
79. य॒ज॒मा॒न॒दे॒व॒त्यो॑ वै वै य॑जमानदेव॒त्यो॑ यजमानदेव॒त्यो॑ वै मा॑हे॒न्द्रो मा॑हे॒न्द्रो वै य॑जमानदेव॒त्यो॑ यजमानदेव॒त्यो॑ वै मा॑हे॒न्द्रः । \newline
80. य॒ज॒मा॒न॒दे॒व॒त्य॑ इति॑ यजमान - दे॒व॒त्यः॑ । \newline
81. वै मा॑हे॒न्द्रो मा॑हे॒न्द्रो वै वै मा॑हे॒न्द्र स्तेज॒ स्तेजो॑ माहे॒न्द्रो वै वै मा॑हे॒न्द्र स्तेजः॑ । \newline
82. मा॒हे॒न्द्र स्तेज॒ स्तेजो॑ माहे॒न्द्रो मा॑हे॒न्द्र स्तेजः॑ शु॒क्रः शु॒क्र स्तेजो॑ माहे॒न्द्रो मा॑हे॒न्द्र स्तेजः॑ शु॒क्रः । \newline
83. मा॒हे॒न्द्र इति॑ माहा - इ॒न्द्रः । \newline
84. तेजः॑ शु॒क्रः शु॒क्र स्तेज॒ स्तेजः॑ शु॒क्रो यद् यच्छु॒क्र स्तेज॒ स्तेजः॑ शु॒क्रो यत् । \newline
85. शु॒क्रो यद् यच्छु॒क्रः शु॒क्रो यन् मा॑हे॒न्द्रम् मा॑हे॒न्द्रं ॅयच्छु॒क्रः शु॒क्रो यन् मा॑हे॒न्द्रम् । \newline
86. यन् मा॑हे॒न्द्रम् मा॑हे॒न्द्रं ॅयद् यन् मा॑हे॒न्द्रꣳ शु॑क्रपा॒त्रेण॑ शुक्रपा॒त्रेण॑ माहे॒न्द्रं ॅयद् यन् मा॑हे॒न्द्रꣳ शु॑क्रपा॒त्रेण॑ । \newline
87. मा॒हे॒न्द्रꣳ शु॑क्रपा॒त्रेण॑ शुक्रपा॒त्रेण॑ माहे॒न्द्रम् मा॑हे॒न्द्रꣳ शु॑क्रपा॒त्रेण॑ गृ॒ह्णाति॑ गृ॒ह्णाति॑ शुक्रपा॒त्रेण॑ माहे॒न्द्रम् मा॑हे॒न्द्रꣳ शु॑क्रपा॒त्रेण॑ गृ॒ह्णाति॑ । \newline
88. मा॒हे॒न्द्रमिति॑ माहा - इ॒न्द्रम् । \newline
89. शु॒क्र॒पा॒त्रेण॑ गृ॒ह्णाति॑ गृ॒ह्णाति॑ शुक्रपा॒त्रेण॑ शुक्रपा॒त्रेण॑ गृ॒ह्णाति॒ यज॑माने॒ यज॑माने गृ॒ह्णाति॑ शुक्रपा॒त्रेण॑ शुक्रपा॒त्रेण॑ गृ॒ह्णाति॒ यज॑माने । \newline
90. शु॒क्र॒पा॒त्रेणेति॑ शुक्र - पा॒त्रेण॑ । \newline
91. गृ॒ह्णाति॒ यज॑माने॒ यज॑माने गृ॒ह्णाति॑ गृ॒ह्णाति॒ यज॑मान ए॒वैव यज॑माने गृ॒ह्णाति॑ गृ॒ह्णाति॒ यज॑मान ए॒व । \newline
92. यज॑मान ए॒वैव यज॑माने॒ यज॑मान ए॒व तेज॒ स्तेज॑ ए॒व यज॑माने॒ यज॑मान ए॒व तेजः॑ । \newline
93. ए॒व तेज॒ स्तेज॑ ए॒वैव तेजो॑ दधाति दधाति॒ तेज॑ ए॒वैव तेजो॑ दधाति । \newline
94. तेजो॑ दधाति दधाति॒ तेज॒ स्तेजो॑ दधाति । \newline
95. द॒धा॒तीति॑ दधाति । \newline
\pagebreak
\markright{ TS 6.5.6.1  \hfill https://www.vedavms.in \hfill}

\section{ TS 6.5.6.1 }

\textbf{TS 6.5.6.1 } \newline
\textbf{Samhita Paata} \newline

अदि॑तिः पु॒त्रका॑मा सा॒द्ध्येभ्यो॑ दे॒वेभ्यो᳚ ब्रह्मौद॒नम॑पच॒त् तस्या॑ उ॒च्छेष॑णमददु॒स्तत् प्राश्ना॒थ् सा रेतो॑ऽधत्त॒ तस्यै॑ च॒त्वार॑ आदि॒त्या अ॑जायन्त॒ सा द्वि॒तीय॑मपच॒थ् साऽम॑न्यतो॒च्छेष॑णान्म इ॒मे᳚ऽज्ञ्त॒ यदग्रे᳚ प्राशि॒ष्यामी॒तो मे॒ वसी॑याꣳसो जनिष्यन्त॒ इति॒ साऽग्रे॒ प्राश्ना॒थ् सा रेतो॑ऽधत्त॒ तस्यै॒ व्यृ॑द्धमा॒ण्डम॑जायत॒ साऽऽदि॒त्येभ्य॑ ए॒व- [  ] \newline

\textbf{Pada Paata} \newline

अदि॑तिः । पु॒त्रका॒मेति॑ पु॒त्र - का॒मा॒ । सा॒द्ध्येभ्यः॑ । दे॒वेभ्यः॑ । ब्र॒ह्मौ॒द॒नमिति॑ ब्रह्म - ओ॒द॒नम् । अ॒प॒च॒त् । तस्यै᳚ । उ॒च्छेष॑ण॒मित्यु॑त्-शेष॑णम् । अ॒द॒दुः॒ । तत् । प्रेति॑ । अ॒श्ना॒त् । सा । रेतः॑ । अ॒ध॒त्त॒ । तस्यै᳚ । च॒त्वारः॑ । आ॒दि॒त्याः । अ॒जा॒य॒न्त॒ । सा । द्वि॒तीय᳚म् । अ॒प॒च॒त् । सा । अ॒म॒न्य॒त॒ । उ॒च्छेष॑णा॒दित्यु॑त् - शेष॑णात् । मे॒ । इ॒मे । अ॒ज्ञ्॒त॒ । यत् । अग्रे᳚ । प्रा॒शि॒ष्यामीति॑ प्र - अ॒शि॒ष्यामि॑ । इ॒तः । मे॒ । वसी॑याꣳसः । ज॒नि॒ष्य॒न्ते॒ । इति॑ । सा । अग्रे᳚ । प्रेति॑ । अ॒श्ना॒त् । सा । रेतः॑ । अ॒ध॒त्त॒ । तस्यै᳚ । व्यृ॑द्ध॒मिति॒ वि - ऋ॒द्ध॒म् । आ॒ण्डम् । अ॒जा॒य॒त॒ । सा । आ॒दि॒त्येभ्यः॑ । ए॒व ।  \newline


\textbf{Krama Paata} \newline

अदि॑तिः पु॒त्रका॑मा । पु॒त्रका॑मा सा॒द्ध्येभ्यः॑ । पु॒त्रका॒मेति॑ पु॒त्र - का॒मा॒ । सा॒द्ध्येभ्यो॑ दे॒वेभ्यः॑ । दे॒वेभ्यो᳚ ब्रह्मौद॒नम् । ब्र॒ह्मौ॒द॒नम॑पचत् । ब्र॒ह्मौ॒द॒नमिति॑ ब्रह्म - ओ॒द॒नम् । अ॒प॒च॒त् तस्यै᳚ । तस्या॑ उ॒च्छेष॑णम् । उ॒च्छेष॑णमददुः । उ॒च्छेष॑ण॒मित्यु॑त् - शेष॑णम् । अ॒द॒दु॒स्तत् । तत् प्र । प्राश्ञा᳚त् । आ॒श्ञा॒थ् सा । सा रेतः॑ । रेतो॑ऽधत्त । अ॒ध॒त्त॒ तस्यै᳚ । तस्यै॑ च॒त्वारः॑ । च॒त्वार॑ आदि॒त्याः । आ॒दि॒त्या अ॑जायन्त । अ॒जा॒य॒न्त॒ सा । सा द्वि॒तीय᳚म् । द्वि॒तीय॑मपचत् । अ॒प॒च॒थ् सा । साऽम॑न्यत । अ॒म॒न्य॒तो॒च्छेष॑णात् । उ॒च्छेष॑णान् मे । उ॒च्छेष॑णा॒दित्यु॑त् - शेष॑णात् । म॒ इ॒मे । इ॒मे᳚ऽज्ञ्त । अ॒ज्ञ्॒त॒ यत् । यदग्रे᳚ । अग्रे᳚ प्राशि॒ष्यामि॑ । प्रा॒शि॒ष्यामी॒तः । प्रा॒शि॒ष्यामीति॑ प्र - अ॒शि॒ष्यामि॑ । इ॒तो मे᳚ । मे॒ वसी॑याꣳसः । वसी॑याꣳसो जनिष्यन्ते । ज॒नि॒ष्य॒न्त॒ इति॑ । इति॒ सा । साऽग्रे᳚ । अग्रे॒ प्र । प्राश्ञा᳚त् । आ॒श्ञा॒थ् सा । सा रेतः॑ । रेतो॑ऽधत्त । अ॒ध॒त्त॒ तस्यै᳚ । तस्यै॒ व्यृ॑द्धम् । व्यृ॑द्धमा॒ण्डम् । व्यृ॑द्ध॒मिति॒ वि - ऋ॒द्ध॒म् । आ॒ण्डम॑जायत । अ॒जा॒य॒त॒ सा । साऽऽदि॒त्येभ्यः॑ । आ॒दि॒त्येभ्य॑ ए॒व । ए॒व तृ॒तीय᳚म् \newline

\textbf{Jatai Paata} \newline

1. अदि॑तिः पु॒त्रका॑मा पु॒त्रका॒मा ऽदि॑ति॒ रदि॑तिः पु॒त्रका॑मा । \newline
2. पु॒त्रका॑मा सा॒द्ध्येभ्यः॑ सा॒द्ध्येभ्यः॑ पु॒त्रका॑मा पु॒त्रका॑मा सा॒द्ध्येभ्यः॑ । \newline
3. पु॒त्रका॒मेति॑ पु॒त्र - का॒मा॒ । \newline
4. सा॒द्ध्येभ्यो॑ दे॒वेभ्यो॑ दे॒वेभ्यः॑ सा॒द्ध्येभ्यः॑ सा॒द्ध्येभ्यो॑ दे॒वेभ्यः॑ । \newline
5. दे॒वेभ्यो᳚ ब्रह्मौद॒नम् ब्र॑ह्मौद॒नम् दे॒वेभ्यो॑ दे॒वेभ्यो᳚ ब्रह्मौद॒नम् । \newline
6. ब्र॒ह्मौ॒द॒न म॑पच दपचद् ब्रह्मौद॒नम् ब्र॑ह्मौद॒न म॑पचत् । \newline
7. ब्र॒ह्मौ॒द॒नमिति॑ ब्रह्म - ओ॒द॒नम् । \newline
8. अ॒प॒च॒त् तस्यै॒ तस्या॑ अपच दपच॒त् तस्यै᳚ । \newline
9. तस्या॑ उ॒च्छेष॑ण मु॒च्छेष॑ण॒म् तस्यै॒ तस्या॑ उ॒च्छेष॑णम् । \newline
10. उ॒च्छेष॑ण मददु रददु रु॒च्छेष॑ण मु॒च्छेष॑ण मददुः । \newline
11. उ॒च्छेष॑ण॒मित्यु॑त् - शेष॑णम् । \newline
12. अ॒द॒दु॒ स्तत् तद॑ ददु रददु॒ स्तत् । \newline
13. तत् प्र प्र तत् तत् प्र । \newline
14. प्राश्ञा॑ दाश्ञा॒त् प्र प्राश्ञा᳚त् । \newline
15. आ॒श्ञा॒थ् सा सा ऽऽश्ञा॑ दाश्ञा॒थ् सा । \newline
16. सा रेतो॒ रेतः॒ सा सा रेतः॑ । \newline
17. रेतो॑ ऽधत्ता धत्त॒ रेतो॒ रेतो॑ ऽधत्त । \newline
18. अ॒ध॒त्त॒ तस्यै॒ तस्या॑ अधत्ता धत्त॒ तस्यै᳚ । \newline
19. तस्यै॑ च॒त्वार॑ श्च॒त्वार॒ स्तस्यै॒ तस्यै॑ च॒त्वारः॑ । \newline
20. च॒त्वार॑ आदि॒त्या आ॑दि॒त्या श्च॒त्वार॑ श्च॒त्वार॑ आदि॒त्याः । \newline
21. आ॒दि॒त्या अ॑जायन्ता जायन्ता दि॒त्या आ॑दि॒त्या अ॑जायन्त । \newline
22. अ॒जा॒य॒न्त॒ सा सा ऽजा॑यन्ता जायन्त॒ सा । \newline
23. सा द्वि॒तीय॑म् द्वि॒तीयꣳ॒॒ सा सा द्वि॒तीय᳚म् । \newline
24. द्वि॒तीय॑ मपच दपचद् द्वि॒तीय॑म् द्वि॒तीय॑ मपचत् । \newline
25. अ॒प॒च॒थ् सा सा ऽप॑च दपच॒थ् सा । \newline
26. सा ऽम॑न्यता मन्यत॒ सा सा ऽम॑न्यत । \newline
27. अ॒म॒न्य॒ तो॒च्छेष॑णा दु॒च्छेष॑णा दमन्यता मन्य तो॒च्छेष॑णात् । \newline
28. उ॒च्छेष॑णान् मे म उ॒च्छेष॑णा दु॒च्छेष॑णान् मे । \newline
29. उ॒च्छेष॑णा॒दित्यु॑त् - शेष॑णात् । \newline
30. म॒ इ॒म इ॒मे मे॑ म इ॒मे । \newline
31. इ॒मे᳚ ऽज्ञ्ता ज्ञ्ते॒ म इ॒मे᳚ ऽज्ञ्त । \newline
32. अ॒ज्ञ्॒त॒ यद् यद॑ज्ञ्ता ज्ञ्त॒ यत् । \newline
33. यदग्रे ऽग्रे॒ यद् यदग्रे᳚ । \newline
34. अग्रे᳚ प्राशि॒ष्यामि॑ प्राशि॒ष्या म्यग्रे ऽग्रे᳚ प्राशि॒ष्यामि॑ । \newline
35. प्रा॒शि॒ष्या मी॒त इ॒तः प्रा॑शि॒ष्यामि॑ प्राशि॒ष्या मी॒तः । \newline
36. प्रा॒शि॒ष्यामीति॑ प्र - अ॒शि॒ष्यामि॑ । \newline
37. इ॒तो मे॑ म इ॒त इ॒तो मे᳚ । \newline
38. मे॒ वसी॑याꣳसो॒ वसी॑याꣳसो मे मे॒ वसी॑याꣳसः । \newline
39. वसी॑याꣳसो जनिष्यन्ते जनिष्यन्ते॒ वसी॑याꣳसो॒ वसी॑याꣳसो जनिष्यन्ते । \newline
40. ज॒नि॒ष्य॒न्त॒ इतीति॑ जनिष्यन्ते जनिष्यन्त॒ इति॑ । \newline
41. इति॒ सा सेतीति॒ सा । \newline
42. सा ऽग्रे ऽग्रे॒ सा सा ऽग्रे᳚ । \newline
43. अग्रे॒ प्र प्राग्रे ऽग्रे॒ प्र । \newline
44. प्राश्ञा॑ दाश्ञा॒त् प्र प्राश्ञा᳚त् । \newline
45. आ॒श्ञा॒थ् सा सा ऽऽश्ञा॑ दाश्ञा॒थ् सा । \newline
46. सा रेतो॒ रेतः॒ सा सा रेतः॑ । \newline
47. रेतो॑ ऽधत्ता धत्त॒ रेतो॒ रेतो॑ ऽधत्त । \newline
48. अ॒ध॒त्त॒ तस्यै॒ तस्या॑ अधत्ता धत्त॒ तस्यै᳚ । \newline
49. तस्यै॒ व्यृ॑द्धं॒ ॅव्यृ॑द्ध॒म् तस्यै॒ तस्यै॒ व्यृ॑द्धम् । \newline
50. व्यृ॑द्ध मा॒ण्ड मा॒ण्डं ॅव्यृ॑द्धं॒ ॅव्यृ॑द्ध मा॒ण्डम् । \newline
51. व्यृ॑द्ध॒मिति॒ वि - ऋ॒द्ध॒म् । \newline
52. आ॒ण्ड म॑जायता जायता॒ ण्ड मा॒ण्ड म॑जायत । \newline
53. अ॒जा॒य॒त॒ सा सा ऽजा॑यता जायत॒ सा । \newline
54. सा ऽऽदि॒त्येभ्य॑ आदि॒त्येभ्यः॒ सा सा ऽऽदि॒त्येभ्यः॑ । \newline
55. आ॒दि॒त्येभ्य॑ ए॒वै वादि॒त्येभ्य॑ आदि॒त्येभ्य॑ ए॒व । \newline
56. ए॒व तृ॒तीय॑म् तृ॒तीय॑ मे॒वैव तृ॒तीय᳚म् । \newline

\textbf{Ghana Paata } \newline

1. अदि॑तिः पु॒त्रका॑मा पु॒त्रका॒मा ऽदि॑ति॒ रदि॑तिः पु॒त्रका॑मा सा॒द्ध्येभ्यः॑ सा॒द्ध्येभ्यः॑ पु॒त्रका॒मा ऽदि॑ति॒ रदि॑तिः पु॒त्रका॑मा सा॒द्ध्येभ्यः॑ । \newline
2. पु॒त्रका॑मा सा॒द्ध्येभ्यः॑ सा॒द्ध्येभ्यः॑ पु॒त्रका॑मा पु॒त्रका॑मा सा॒द्ध्येभ्यो॑ दे॒वेभ्यो॑ दे॒वेभ्यः॑ सा॒द्ध्येभ्यः॑ पु॒त्रका॑मा पु॒त्रका॑मा सा॒द्ध्येभ्यो॑ दे॒वेभ्यः॑ । \newline
3. पु॒त्रका॒मेति॑ पु॒त्र - का॒मा॒ । \newline
4. सा॒द्ध्येभ्यो॑ दे॒वेभ्यो॑ दे॒वेभ्यः॑ सा॒द्ध्येभ्यः॑ सा॒द्ध्येभ्यो॑ दे॒वेभ्यो᳚ ब्रह्मौद॒नम् ब्र॑ह्मौद॒नम् दे॒वेभ्यः॑ सा॒द्ध्येभ्यः॑ सा॒द्ध्येभ्यो॑ दे॒वेभ्यो᳚ ब्रह्मौद॒नम् । \newline
5. दे॒वेभ्यो᳚ ब्रह्मौद॒नम् ब्र॑ह्मौद॒नम् दे॒वेभ्यो॑ दे॒वेभ्यो᳚ ब्रह्मौद॒न म॑पच दपचद् ब्रह्मौद॒नम् दे॒वेभ्यो॑ दे॒वेभ्यो᳚ ब्रह्मौद॒न म॑पचत् । \newline
6. ब्र॒ह्मौ॒द॒न म॑पच दपचद् ब्रह्मौद॒नम् ब्र॑ह्मौद॒न म॑पच॒त् तस्यै॒ तस्या॑ अपचद् ब्रह्मौद॒नम् ब्र॑ह्मौद॒न म॑पच॒त् तस्यै᳚ । \newline
7. ब्र॒ह्मौ॒द॒नमिति॑ ब्रह्म - ओ॒द॒नम् । \newline
8. अ॒प॒च॒त् तस्यै॒ तस्या॑ अपच दपच॒त् तस्या॑ उ॒च्छेष॑ण मु॒च्छेष॑ण॒म् तस्या॑ अपच दपच॒त् तस्या॑ उ॒च्छेष॑णम् । \newline
9. तस्या॑ उ॒च्छेष॑ण मु॒च्छेष॑ण॒म् तस्यै॒ तस्या॑ उ॒च्छेष॑ण मददु रददु रु॒च्छेष॑ण॒म् तस्यै॒ तस्या॑ उ॒च्छेष॑ण मददुः । \newline
10. उ॒च्छेष॑ण मददु रददु रु॒च्छेष॑ण मु॒च्छेष॑ण मददु॒ स्तत् तद॑ददु रु॒च्छेष॑ण मु॒च्छेष॑ण मददु॒ स्तत् । \newline
11. उ॒च्छेष॑ण॒मित्यु॑त् - शेष॑णम् । \newline
12. अ॒द॒दु॒ स्तत् तद॑ददु रददु॒ स्तत् प्र प्र तद॑ददु रददु॒ स्तत् प्र । \newline
13. तत् प्र प्र तत् तत् प्राश्ञा॑ दाश्ञा॒त् प्र तत् तत् प्राश्ञा᳚त् । \newline
14. प्राश्ञा॑ दाश्ञा॒त् प्र प्राश्ञा॒थ् सा सा ऽऽश्ञा॒त् प्र प्राश्ञा॒थ् सा । \newline
15. आ॒श्ञा॒थ् सा सा ऽऽश्ञा॑ दाश्ञा॒थ् सा रेतो॒ रेतः॒ सा ऽऽश्ञा॑ दाश्ञा॒थ् सा रेतः॑ । \newline
16. सा रेतो॒ रेतः॒ सा सा रेतो॑ ऽधत्ता धत्त॒ रेतः॒ सा सा रेतो॑ ऽधत्त । \newline
17. रेतो॑ ऽधत्ता धत्त॒ रेतो॒ रेतो॑ ऽधत्त॒ तस्यै॒ तस्या॑ अधत्त॒ रेतो॒ रेतो॑ ऽधत्त॒ तस्यै᳚ । \newline
18. अ॒ध॒त्त॒ तस्यै॒ तस्या॑ अधत्ता धत्त॒ तस्यै॑ च॒त्वार॑ श्च॒त्वार॒ स्तस्या॑ अधत्ता धत्त॒ तस्यै॑ च॒त्वारः॑ । \newline
19. तस्यै॑ च॒त्वार॑ श्च॒त्वार॒ स्तस्यै॒ तस्यै॑ च॒त्वार॑ आदि॒त्या आ॑दि॒त्या श्च॒त्वार॒ स्तस्यै॒ तस्यै॑ च॒त्वार॑ आदि॒त्याः । \newline
20. च॒त्वार॑ आदि॒त्या आ॑दि॒त्या श्च॒त्वार॑ श्च॒त्वार॑ आदि॒त्या अ॑जायन्ता जायन्ता दि॒त्या श्च॒त्वार॑ श्च॒त्वार॑ आदि॒त्या अ॑जायन्त । \newline
21. आ॒दि॒त्या अ॑जायन्ता जायन्ता दि॒त्या आ॑दि॒त्या अ॑जायन्त॒ सा सा ऽजा॑यन्ता दि॒त्या आ॑दि॒त्या अ॑जायन्त॒ सा । \newline
22. अ॒जा॒य॒न्त॒ सा सा ऽजा॑यन्ता जायन्त॒ सा द्वि॒तीय॑म् द्वि॒तीयꣳ॒॒ सा ऽजा॑यन्ता जायन्त॒ सा द्वि॒तीय᳚म् । \newline
23. सा द्वि॒तीय॑म् द्वि॒तीयꣳ॒॒ सा सा द्वि॒तीय॑ मपच दपचद् द्वि॒तीयꣳ॒॒ सा सा द्वि॒तीय॑ मपचत् । \newline
24. द्वि॒तीय॑ मपच दपचद् द्वि॒तीय॑म् द्वि॒तीय॑ मपच॒थ् सा सा ऽप॑चद् द्वि॒तीय॑म् द्वि॒तीय॑ मपच॒थ् सा । \newline
25. अ॒प॒च॒थ् सा सा ऽप॑च दपच॒थ् सा ऽम॑न्यता मन्यत॒ सा ऽप॑च दपच॒थ् सा ऽम॑न्यत । \newline
26. सा ऽम॑न्यता मन्यत॒ सा सा ऽम॑न्य तो॒च्छेष॑णा दु॒च्छेष॑णा दमन्यत॒ सा सा ऽम॑न्य तो॒च्छेष॑णात् । \newline
27. अ॒म॒न्य॒ तो॒च्छेष॑णा दु॒च्छेष॑णा दमन्यता मन्य तो॒च्छेष॑णान् मे म उ॒च्छेष॑णा दमन्यता मन्य
तो॒च्छेष॑णान् मे । \newline
28. उ॒च्छेष॑णान् मे म उ॒च्छेष॑णा दु॒च्छेष॑णान् म इ॒म इ॒मे म॑ उ॒च्छेष॑णा दु॒च्छेष॑णान् म इ॒मे । \newline
29. उ॒च्छेष॑णा॒दित्यु॑त् - शेष॑णात् । \newline
30. म॒ इ॒म इ॒मे मे॑ म इ॒मे᳚ ऽज्ञ्ता ज्ञ्ते॒ मे मे॑ म इ॒मे᳚ ऽज्ञ्त । \newline
31. इ॒मे᳚ ऽज्ञ्ता ज्ञ्ते॒ म इ॒मे᳚ ऽज्ञ्त॒ यद् यद॑ज्ञ्ते॒ म इ॒मे᳚ ऽज्ञ्त॒ यत् । \newline
32. अ॒ज्ञ्॒त॒ यद् यद॑ज्ञ्ता ज्ञ्त॒ यदग्रे ऽग्रे॒ यद॑ज्ञ्ता ज्ञ्त॒ यदग्रे᳚ । \newline
33. यदग्रे ऽग्रे॒ यद् यदग्रे᳚ प्राशि॒ष्यामि॑ प्राशि॒ष्या म्यग्रे॒ यद् यदग्रे᳚ प्राशि॒ष्यामि॑ । \newline
34. अग्रे᳚ प्राशि॒ष्यामि॑ प्राशि॒ष्या म्यग्रे ऽग्रे᳚ प्राशि॒ष्या मी॒त इ॒तः प्रा॑शि॒ष्या म्यग्रे ऽग्रे᳚ प्राशि॒ष्या मी॒तः । \newline
35. प्रा॒शि॒ष्या मी॒त इ॒तः प्रा॑शि॒ष्यामि॑ प्राशि॒ष्या मी॒तो मे॑ म इ॒तः प्रा॑शि॒ष्यामि॑ प्राशि॒ष्या मी॒तो मे᳚ । \newline
36. प्रा॒शि॒ष्यामीति॑ प्र - अ॒शि॒ष्यामि॑ । \newline
37. इ॒तो मे॑ म इ॒त इ॒तो मे॒ वसी॑याꣳसो॒ वसी॑याꣳसो म इ॒त इ॒तो मे॒ वसी॑याꣳसः । \newline
38. मे॒ वसी॑याꣳसो॒ वसी॑याꣳसो मे मे॒ वसी॑याꣳसो जनिष्यन्ते जनिष्यन्ते॒ वसी॑याꣳसो मे मे॒ वसी॑याꣳसो जनिष्यन्ते । \newline
39. वसी॑याꣳसो जनिष्यन्ते जनिष्यन्ते॒ वसी॑याꣳसो॒ वसी॑याꣳसो जनिष्यन्त॒ इतीति॑ जनिष्यन्ते॒ वसी॑याꣳसो॒ वसी॑याꣳसो जनिष्यन्त॒ इति॑ । \newline
40. ज॒नि॒ष्य॒न्त॒ इतीति॑ जनिष्यन्ते जनिष्यन्त॒ इति॒ सा सेति॑ जनिष्यन्ते जनिष्यन्त॒ इति॒ सा । \newline
41. इति॒ सा सेतीति॒ सा ऽग्रे ऽग्रे॒ सेतीति॒ सा ऽग्रे᳚ । \newline
42. सा ऽग्रे ऽग्रे॒ सा सा ऽग्रे॒ प्र प्राग्रे॒ सा सा ऽग्रे॒ प्र । \newline
43. अग्रे॒ प्र प्राग्रे ऽग्रे॒ प्राश्ञा॑ दाश्ञा॒त् प्राग्रे ऽग्रे॒ प्राश्ञा᳚त् । \newline
44. प्राश्ञा॑ दाश्ञा॒त् प्र प्राश्ञा॒थ् सा सा ऽऽश्ञा॒त् प्र प्राश्ञा॒थ् सा । \newline
45. आ॒श्ञा॒थ् सा सा ऽऽश्ञा॑ दाश्ञा॒थ् सा रेतो॒ रेतः॒ सा ऽऽश्ञा॑ दाश्ञा॒थ् सा रेतः॑ । \newline
46. सा रेतो॒ रेतः॒ सा सा रेतो॑ ऽधत्ता धत्त॒ रेतः॒ सा सा रेतो॑ ऽधत्त । \newline
47. रेतो॑ ऽधत्ता धत्त॒ रेतो॒ रेतो॑ ऽधत्त॒ तस्यै॒ तस्या॑ अधत्त॒ रेतो॒ रेतो॑ ऽधत्त॒ तस्यै᳚ । \newline
48. अ॒ध॒त्त॒ तस्यै॒ तस्या॑ अधत्ता धत्त॒ तस्यै॒ व्यृ॑द्धं॒ ॅव्यृ॑द्ध॒म् तस्या॑ अधत्ता धत्त॒ तस्यै॒ व्यृ॑द्धम् । \newline
49. तस्यै॒ व्यृ॑द्धं॒ ॅव्यृ॑द्ध॒म् तस्यै॒ तस्यै॒ व्यृ॑द्ध मा॒ण्ड मा॒ण्डं ॅव्यृ॑द्ध॒म् तस्यै॒ तस्यै॒ व्यृ॑द्ध मा॒ण्डम् । \newline
50. व्यृ॑द्ध मा॒ण्ड मा॒ण्डं ॅव्यृ॑द्धं॒ ॅव्यृ॑द्ध मा॒ण्ड म॑जायता जायता॒ण्डं ॅव्यृ॑द्धं॒ ॅव्यृ॑द्ध मा॒ण्ड म॑जायत । \newline
51. व्यृ॑द्ध॒मिति॒ वि - ऋ॒द्ध॒म् । \newline
52. आ॒ण्ड म॑जायता जायता॒ण्ड मा॒ण्ड म॑जायत॒ सा सा ऽजा॑यता॒ ण्ड मा॒ण्ड म॑जायत॒ सा । \newline
53. अ॒जा॒य॒त॒ सा सा ऽजा॑यता जायत॒ सा ऽऽदि॒त्येभ्य॑ आदि॒त्येभ्यः॒ सा ऽजा॑यता जायत॒ सा ऽऽदि॒त्येभ्यः॑ । \newline
54. सा ऽऽदि॒त्येभ्य॑ आदि॒त्येभ्यः॒ सा सा ऽऽदि॒त्येभ्य॑ ए॒वै वादि॒त्येभ्यः॒ सा सा ऽऽदि॒त्येभ्य॑ ए॒व । \newline
55. आ॒दि॒त्येभ्य॑ ए॒वै वादि॒त्येभ्य॑ आदि॒त्येभ्य॑ ए॒व तृ॒तीय॑म् तृ॒तीय॑ मे॒वादि॒त्येभ्य॑ आदि॒त्येभ्य॑ ए॒व तृ॒तीय᳚म् । \newline
56. ए॒व तृ॒तीय॑म् तृ॒तीय॑ मे॒वैव तृ॒तीय॑ मपच दपचत् तृ॒तीय॑ मे॒वैव तृ॒तीय॑ मपचत् । \newline
\pagebreak
\markright{ TS 6.5.6.2  \hfill https://www.vedavms.in \hfill}

\section{ TS 6.5.6.2 }

\textbf{TS 6.5.6.2 } \newline
\textbf{Samhita Paata} \newline

तृ॒तीय॑मपच॒द्-भोगा॑य म इ॒दꣳ श्रा॒न्तम॒स्त्विति॒ ते᳚ऽब्रुव॒न् वरं॑ ॅवृणामहै॒ योऽतो॒ जाया॑ता अ॒स्माकꣳ॒॒ स एको॑ऽस॒द्यो᳚ऽस्य प्र॒जाया॒मृद्ध्या॑ता अ॒स्माकं॒ भोगा॑य भवा॒दिति॒ ततो॒ विव॑स्वानादि॒त्यो॑ ऽजायत॒ तस्य॒ वा इ॒यं प्र॒जा यन्म॑नु॒ष्या᳚स्तास्वेक॑ ए॒वर्द्धो यो यज॑ते॒ स दे॒वानां॒ भोगा॑य भवति दे॒वा वै य॒ज्ञा- [  ] \newline

\textbf{Pada Paata} \newline

तृ॒तीय᳚म् । अ॒प॒च॒त् । भोगा॑य । मे॒ । इ॒दम् । श्रा॒न्तम् । अ॒स्तु॒ । इति॑ । ते । अ॒ब्रु॒व॒न्न् । वर᳚म् । वृ॒णा॒म॒है॒ । यः । अतः॑ । जाया॑तै । अ॒स्माक᳚म् । सः । एकः॑ । अ॒स॒त् । यः । अ॒स्य॒ । प्र॒जाया॒मिति॑ प्र - जाया᳚म् । ऋद्ध्या॑तै । अ॒स्माक᳚म् । भोगा॑य । भ॒वा॒त् । इति॑ । ततः॑ । विव॑स्वान् । आ॒दि॒त्यः । अ॒जा॒य॒त॒ । तस्य॑ । वै । इ॒यम् । प्र॒जेति॑ प्र - जा । यत् । म॒नु॒ष्याः᳚ । तासु॑ । एकः॑ । ए॒व । ऋ॒द्धः । यः । यज॑ते । सः । दे॒वाना᳚म् । भोगा॑य । भ॒व॒ति॒ । दे॒वाः । वै । य॒ज्ञात् ।  \newline


\textbf{Krama Paata} \newline

तृ॒तीय॑मपचत् । अ॒प॒च॒द् भोगा॑य । भोगा॑य मे । म॒ इ॒दम् । इ॒दꣳ श्रा॒न्तम् । श्रा॒न्तम॑स्तु । अ॒स्त्विति॑ । इति॒ ते । ते᳚ऽब्रुवन्न् । अ॒ब्रु॒व॒न् वर᳚म् । वर॑म् ॅवृणामहै । वृ॒णा॒म॒है॒ यः । योऽतः॑ । अतो॒ जाया॑तै । जाया॑ता अ॒स्माक᳚म् । अ॒स्माकꣳ॒॒ सः । स एकः॑ । एको॑ऽसत् । अ॒स॒द् यः । यो᳚ऽस्य । अ॒स्य॒ प्र॒जाया᳚म् । प्र॒जाया॒मृद्ध्या॑तै । प्र॒जाया॒मिति॑ प्र - जाया᳚म् । ऋद्ध्या॑ता अ॒स्माक᳚म् । अ॒स्माक॒म् भोगा॑य । भोगा॑य भवात् । भ॒वा॒दिति॑ । इति॒ ततः॑ । ततो॒ विव॑स्वान् । विव॑स्वानादि॒त्यः । आ॒दि॒त्यो॑ऽजायत । अ॒जा॒य॒त॒ तस्य॑ । तस्य॒ वै । वा इ॒यम् । इ॒यम् प्र॒जा । प्र॒जा यत् । प्र॒जेति॑ प्र - जा । यन् म॑नु॒ष्याः᳚ । म॒नु॒ष्या᳚स्तासु॑ । तास्वेकः॑ । एक॑ ए॒व । ए॒वर्द्धः । ऋ॒द्धो यः । यो यज॑ते । यज॑ते॒ सः । स दे॒वाना᳚म् । दे॒वाना॒म् भोगा॑य । भोगा॑य भवति । भ॒व॒ति॒ दे॒वाः । दे॒वा वै । वै य॒ज्ञात् । य॒ज्ञाद् रु॒द्रम् \newline

\textbf{Jatai Paata} \newline

1. तृ॒तीय॑ मपच दपचत् तृ॒तीय॑म् तृ॒तीय॑ मपचत् । \newline
2. अ॒प॒च॒द् भोगा॑य॒ भोगा॑या पच दपच॒द् भोगा॑य । \newline
3. भोगा॑य मे मे॒ भोगा॑य॒ भोगा॑य मे । \newline
4. म॒ इ॒द मि॒दम् मे॑ म इ॒दम् । \newline
5. इ॒दꣳ श्रा॒न्तꣳ श्रा॒न्त मि॒द मि॒दꣳ श्रा॒न्तम् । \newline
6. श्रा॒न्त म॑स्त्वस्तु श्रा॒न्तꣳ श्रा॒न्त म॑स्तु । \newline
7. अ॒स्त्वि तीत्य॑स्त्व॒ स्त्विति॑ । \newline
8. इति॒ ते त इतीति॒ ते । \newline
9. ते᳚ ऽब्रुवन् नब्रुव॒न् ते ते᳚ ऽब्रुवन्न् । \newline
10. अ॒ब्रु॒व॒न्॒. वरं॒ ॅवर॑ मब्रुवन् नब्रुव॒न्॒. वर᳚म् । \newline
11. वरं॑ ॅवृणामहै वृणामहै॒ वरं॒ ॅवरं॑ ॅवृणामहै । \newline
12. वृ॒णा॒म॒है॒ यो यो वृ॑णामहै वृणामहै॒ यः । \newline
13. यो ऽतो ऽतो॒ यो यो ऽतः॑ । \newline
14. अतो॒ जाया॑तै॒ जाया॑ता॒ अतो ऽतो॒ जाया॑तै । \newline
15. जाया॑ता अ॒स्माक॑ म॒स्माक॒म् जाया॑तै॒ जाया॑ता अ॒स्माक᳚म् । \newline
16. अ॒स्माकꣳ॒॒ स सो᳚ ऽस्माक॑ म॒स्माकꣳ॒॒ सः । \newline
17. स एक॒ एकः॒ स स एकः॑ । \newline
18. एको॑ ऽस दस॒ देक॒ एको॑ ऽसत् । \newline
19. अ॒स॒द् यो यो॑ ऽस दस॒द् यः । \newline
20. यो᳚ ऽस्यास्य॒ यो यो᳚ ऽस्य । \newline
21. अ॒स्य॒ प्र॒जाया᳚म् प्र॒जाया॑ मस्यास्य प्र॒जाया᳚म् । \newline
22. प्र॒जाया॒ मृद्ध्या॑ता॒ ऋद्ध्या॑तै प्र॒जाया᳚म् प्र॒जाया॒ मृद्ध्या॑तै । \newline
23. प्र॒जाया॒मिति॑ प्र - जाया᳚म् । \newline
24. ऋद्ध्या॑ता अ॒स्माक॑ म॒स्माक॒ मृद्ध्या॑ता॒ ऋद्ध्या॑ता अ॒स्माक᳚म् । \newline
25. अ॒स्माक॒म् भोगा॑य॒ भोगा॑या॒ स्माक॑ म॒स्माक॒म् भोगा॑य । \newline
26. भोगा॑य भवाद् भवा॒द् भोगा॑य॒ भोगा॑य भवात् । \newline
27. भ॒वा॒ दितीति॑ भवाद् भवा॒ दिति॑ । \newline
28. इति॒ तत॒ स्तत॒ इतीति॒ ततः॑ । \newline
29. ततो॒ विव॑स्वा॒न्॒. विव॑स्वा॒न् तत॒ स्ततो॒ विव॑स्वान् । \newline
30. विव॑स्वा नादि॒त्य आ॑दि॒त्यो विव॑स्वा॒न्॒. विव॑स्वा नादि॒त्यः । \newline
31. आ॒दि॒त्यो॑ ऽजायता जायता दि॒त्य आ॑दि॒त्यो॑ ऽजायत । \newline
32. अ॒जा॒य॒त॒ तस्य॒ तस्या॑ जायता जायत॒ तस्य॑ । \newline
33. तस्य॒ वै वै तस्य॒ तस्य॒ वै । \newline
34. वा इ॒य मि॒यं ॅवै वा इ॒यम् । \newline
35. इ॒यम् प्र॒जा प्र॒जेय मि॒यम् प्र॒जा । \newline
36. प्र॒जा यद् यत् प्र॒जा प्र॒जा यत् । \newline
37. प्र॒जेति॑ प्र - जा । \newline
38. यन् म॑नु॒ष्या॑ मनु॒ष्या॑ यद् यन् म॑नु॒ष्याः᳚ । \newline
39. म॒नु॒ष्या᳚ स्तासु॒ तासु॑ मनु॒ष्या॑ मनु॒ष्या᳚ स्तासु॑ । \newline
40. तास्वेक॒ एक॒ स्तासु॒ तास्वेकः॑ । \newline
41. एक॑ ए॒वै वैक॒ एक॑ ए॒व । \newline
42. ए॒व र्‌द्ध ऋ॒द्ध ए॒वैव र्‌द्धः । \newline
43. ऋ॒द्धो यो य ऋ॒द्ध ऋ॒द्धो यः । \newline
44. यो यज॑ते॒ यज॑ते॒ यो यो यज॑ते । \newline
45. यज॑ते॒ स स यज॑ते॒ यज॑ते॒ सः । \newline
46. स दे॒वाना᳚म् दे॒वानाꣳ॒॒ स स दे॒वाना᳚म् । \newline
47. दे॒वाना॒म् भोगा॑य॒ भोगा॑य दे॒वाना᳚म् दे॒वाना॒म् भोगा॑य । \newline
48. भोगा॑य भवति भवति॒ भोगा॑य॒ भोगा॑य भवति । \newline
49. भ॒व॒ति॒ दे॒वा दे॒वा भ॑वति भवति दे॒वाः । \newline
50. दे॒वा वै वै दे॒वा दे॒वा वै । \newline
51. वै य॒ज्ञाद् य॒ज्ञाद् वै वै य॒ज्ञात् । \newline
52. य॒ज्ञाद् रु॒द्रꣳ रु॒द्रं ॅय॒ज्ञाद् य॒ज्ञाद् रु॒द्रम् । \newline

\textbf{Ghana Paata } \newline

1. तृ॒तीय॑ मपच दपचत् तृ॒तीय॑म् तृ॒तीय॑ मपच॒द् भोगा॑य॒ भोगा॑या पचत् तृ॒तीय॑म् तृ॒तीय॑ मपच॒द् भोगा॑य । \newline
2. अ॒प॒च॒द् भोगा॑य॒ भोगा॑या पच दपच॒द् भोगा॑य मे मे॒ भोगा॑या पच दपच॒द् भोगा॑य मे । \newline
3. भोगा॑य मे मे॒ भोगा॑य॒ भोगा॑य म इ॒द मि॒दम् मे॒ भोगा॑य॒ भोगा॑य म इ॒दम् । \newline
4. म॒ इ॒द मि॒दम् मे॑ म इ॒दꣳ श्रा॒न्तꣳ श्रा॒न्त मि॒दम् मे॑ म इ॒दꣳ श्रा॒न्तम् । \newline
5. इ॒दꣳ श्रा॒न्तꣳ श्रा॒न्त मि॒द मि॒दꣳ श्रा॒न्त म॑स्त्वस्तु श्रा॒न्त मि॒द मि॒दꣳ श्रा॒न्त म॑स्तु । \newline
6. श्रा॒न्त म॑स्त्वस्तु श्रा॒न्तꣳ श्रा॒न्त म॒स्त्विती त्य॑स्तु श्रा॒न्तꣳ श्रा॒न्त म॒स्त्विति॑ । \newline
7. अ॒स्त्वि तीत्य॑स्त्व॒ स्त्विति॒ ते त इत्य॑स्त्व॒ स्त्विति॒ ते । \newline
8. इति॒ ते त इतीति॒ ते᳚ ऽब्रुवन्-नब्रुव॒न् त इतीति॒ ते᳚ ऽब्रुवन्न् । \newline
9. ते᳚ ऽब्रुवन्-नब्रुव॒न् ते ते᳚ ऽब्रुव॒न्॒. वरं॒ ॅवर॑ मब्रुव॒न् ते ते᳚ ऽब्रुव॒न्॒. वर᳚म् । \newline
10. अ॒ब्रु॒व॒न्॒. वरं॒ ॅवर॑ मब्रुवन्-नब्रुव॒न्॒. वरं॑ ॅवृणामहै वृणामहै॒ वर॑ मब्रुवन्-नब्रुव॒न्॒. वरं॑ ॅवृणामहै । \newline
11. वरं॑ ॅवृणामहै वृणामहै॒ वरं॒ ॅवरं॑ ॅवृणामहै॒ यो यो वृ॑णामहै॒ वरं॒ ॅवरं॑ ॅवृणामहै॒ यः । \newline
12. वृ॒णा॒म॒है॒ यो यो वृ॑णामहै वृणामहै॒ यो ऽतो ऽतो॒ यो वृ॑णामहै वृणामहै॒ यो ऽतः॑ । \newline
13. यो ऽतो ऽतो॒ यो यो ऽतो॒ जाया॑तै॒ जाया॑ता॒ अतो॒ यो यो ऽतो॒ जाया॑तै । \newline
14. अतो॒ जाया॑तै॒ जाया॑ता॒ अतो ऽतो॒ जाया॑ता अ॒स्माक॑ म॒स्माक॒म् जाया॑ता॒ अतो ऽतो॒ जाया॑ता अ॒स्माक᳚म् । \newline
15. जाया॑ता अ॒स्माक॑ म॒स्माक॒म् जाया॑तै॒ जाया॑ता अ॒स्माकꣳ॒॒ स सो᳚ ऽस्माक॒म् जाया॑तै॒ जाया॑ता अ॒स्माकꣳ॒॒ सः । \newline
16. अ॒स्माकꣳ॒॒ स सो᳚ ऽस्माक॑ म॒स्माकꣳ॒॒ स एक॒ एकः॒ सो᳚ ऽस्माक॑ म॒स्माकꣳ॒॒ स एकः॑ । \newline
17. स एक॒ एकः॒ स स एको॑ ऽस दस॒ देकः॒ स स एको॑ ऽसत् । \newline
18. एको॑ ऽस दस॒ देक॒ एको॑ ऽस॒द् यो यो॑ ऽस॒ देक॒ एको॑ ऽस॒द् यः । \newline
19. अ॒स॒द् यो यो॑ ऽस दस॒द् यो᳚ ऽस्यास्य॒ यो॑ ऽस दस॒द् यो᳚ ऽस्य । \newline
20. यो᳚ ऽस्यास्य॒ यो यो᳚ ऽस्य प्र॒जाया᳚म् प्र॒जाया॑ मस्य॒ यो यो᳚ ऽस्य प्र॒जाया᳚म् । \newline
21. अ॒स्य॒ प्र॒जाया᳚म् प्र॒जाया॑ मस्यास्य प्र॒जाया॒ मृद्ध्या॑ता॒ ऋद्ध्या॑तै प्र॒जाया॑ मस्यास्य प्र॒जाया॒ मृद्ध्या॑तै । \newline
22. प्र॒जाया॒ मृद्ध्या॑ता॒ ऋद्ध्या॑तै प्र॒जाया᳚म् प्र॒जाया॒ मृद्ध्या॑ता अ॒स्माक॑ म॒स्माक॒ मृद्ध्या॑तै प्र॒जाया᳚म् प्र॒जाया॒ मृद्ध्या॑ता अ॒स्माक᳚म् । \newline
23. प्र॒जाया॒मिति॑ प्र - जाया᳚म् । \newline
24. ऋद्ध्या॑ता अ॒स्माक॑ म॒स्माक॒ मृद्ध्या॑ता॒ ऋद्ध्या॑ता अ॒स्माक॒म् भोगा॑य॒ भोगा॑या॒स्माक॒ मृद्ध्या॑ता॒ ऋद्ध्या॑ता अ॒स्माक॒म् भोगा॑य । \newline
25. अ॒स्माक॒म् भोगा॑य॒ भोगा॑या॒ स्माक॑ म॒स्माक॒म् भोगा॑य भवाद् भवा॒द् भोगा॑या॒ स्माक॑ म॒स्माक॒म् भोगा॑य भवात् । \newline
26. भोगा॑य भवाद् भवा॒द् भोगा॑य॒ भोगा॑य भवा॒ दितीति॑ भवा॒द् भोगा॑य॒ भोगा॑य भवा॒ दिति॑ । \newline
27. भ॒वा॒ दितीति॑ भवाद् भवा॒ दिति॒ तत॒ स्तत॒ इति॑ भवाद् भवा॒ दिति॒ ततः॑ । \newline
28. इति॒ तत॒ स्तत॒ इतीति॒ ततो॒ विव॑स्वा॒न्॒. विव॑स्वा॒न् तत॒ इतीति॒ ततो॒ विव॑स्वान् । \newline
29. ततो॒ विव॑स्वा॒न्॒. विव॑स्वा॒न् तत॒ स्ततो॒ विव॑स्वा-नादि॒त्य आ॑दि॒त्यो विव॑स्वा॒न् तत॒ स्ततो॒ विव॑स्वा-नादि॒त्यः । \newline
30. विव॑स्वा-नादि॒त्य आ॑दि॒त्यो विव॑स्वा॒न्॒. विव॑स्वा-नादि॒त्यो॑ ऽजायता जायता दि॒त्यो विव॑स्वा॒न्॒. विव॑स्वा-नादि॒त्यो॑ ऽजायत । \newline
31. आ॒दि॒त्यो॑ ऽजायता जायता दि॒त्य आ॑दि॒त्यो॑ ऽजायत॒ तस्य॒ तस्या॑ जायता दि॒त्य आ॑दि॒त्यो॑ ऽजायत॒ तस्य॑ । \newline
32. अ॒जा॒य॒त॒ तस्य॒ तस्या॑ जायता जायत॒ तस्य॒ वै वै तस्या॑ जायता जायत॒ तस्य॒ वै । \newline
33. तस्य॒ वै वै तस्य॒ तस्य॒ वा इ॒य मि॒यं ॅवै तस्य॒ तस्य॒ वा इ॒यम् । \newline
34. वा इ॒य मि॒यं ॅवै वा इ॒यम् प्र॒जा प्र॒जेयं ॅवै वा इ॒यम् प्र॒जा । \newline
35. इ॒यम् प्र॒जा प्र॒जेय मि॒यम् प्र॒जा यद् यत् प्र॒जेय मि॒यम् प्र॒जा यत् । \newline
36. प्र॒जा यद् यत् प्र॒जा प्र॒जा यन् म॑नु॒ष्या॑ मनु॒ष्या॑ यत् प्र॒जा प्र॒जा यन् म॑नु॒ष्याः᳚ । \newline
37. प्र॒जेति॑ प्र - जा । \newline
38. यन् म॑नु॒ष्या॑ मनु॒ष्या॑ यद् यन् म॑नु॒ष्या᳚ स्तासु॒ तासु॑ मनु॒ष्या॑ यद् यन् म॑नु॒ष्या᳚ स्तासु॑ । \newline
39. म॒नु॒ष्या᳚ स्तासु॒ तासु॑ मनु॒ष्या॑ मनु॒ष्या᳚ स्तास्वेक॒ एक॒ स्तासु॑ मनु॒ष्या॑ मनु॒ष्या᳚ स्तास्वेकः॑ । \newline
40. तास्वेक॒ एक॒ स्तासु॒ तास्वेक॑ ए॒वैवैक॒ स्तासु॒ तास्वेक॑ ए॒व । \newline
41. एक॑ ए॒वैवैक॒ एक॑ ए॒व र्‌द्ध ऋ॒द्ध ए॒वैक॒ एक॑ ए॒व र्‌द्धः । \newline
42. ए॒व र्‌द्ध ऋ॒द्ध ए॒वैव र्‌द्धो यो य ऋ॒द्ध ए॒वैव र्‌द्धो यः । \newline
43. ऋ॒द्धो यो य ऋ॒द्ध ऋ॒द्धो यो यज॑ते॒ यज॑ते॒ य ऋ॒द्ध ऋ॒द्धो यो यज॑ते । \newline
44. यो यज॑ते॒ यज॑ते॒ यो यो यज॑ते॒ स स यज॑ते॒ यो यो यज॑ते॒ सः । \newline
45. यज॑ते॒ स स यज॑ते॒ यज॑ते॒ स दे॒वाना᳚म् दे॒वानाꣳ॒॒ स यज॑ते॒ यज॑ते॒ स दे॒वाना᳚म् । \newline
46. स दे॒वाना᳚म् दे॒वानाꣳ॒॒ स स दे॒वाना॒म् भोगा॑य॒ भोगा॑य दे॒वानाꣳ॒॒ स स दे॒वाना॒म् भोगा॑य । \newline
47. दे॒वाना॒म् भोगा॑य॒ भोगा॑य दे॒वाना᳚म् दे॒वाना॒म् भोगा॑य भवति भवति॒ भोगा॑य दे॒वाना᳚म् दे॒वाना॒म् भोगा॑य भवति । \newline
48. भोगा॑य भवति भवति॒ भोगा॑य॒ भोगा॑य भवति दे॒वा दे॒वा भ॑वति॒ भोगा॑य॒ भोगा॑य भवति दे॒वाः । \newline
49. भ॒व॒ति॒ दे॒वा दे॒वा भ॑वति भवति दे॒वा वै वै दे॒वा भ॑वति भवति दे॒वा वै । \newline
50. दे॒वा वै वै दे॒वा दे॒वा वै य॒ज्ञाद् य॒ज्ञाद् वै दे॒वा दे॒वा वै य॒ज्ञात् । \newline
51. वै य॒ज्ञाद् य॒ज्ञाद् वै वै य॒ज्ञाद् रु॒द्रꣳ रु॒द्रं ॅय॒ज्ञाद् वै वै य॒ज्ञाद् रु॒द्रम् । \newline
52. य॒ज्ञाद् रु॒द्रꣳ रु॒द्रं ॅय॒ज्ञाद् य॒ज्ञाद् रु॒द्र म॒न्त र॒न्ता रु॒द्रं ॅय॒ज्ञाद् य॒ज्ञाद् रु॒द्र म॒न्तः । \newline
\pagebreak
\markright{ TS 6.5.6.3  \hfill https://www.vedavms.in \hfill}

\section{ TS 6.5.6.3 }

\textbf{TS 6.5.6.3 } \newline
\textbf{Samhita Paata} \newline

-द्रु॒द्र-म॒न्तरा॑य॒न्थ् स आ॑दि॒त्यान॒न्वाक्र॑मत॒ ते द्वि॑देव॒त्या᳚न् प्राप॑द्यन्त॒ तान् न प्रति॒ प्राय॑च्छ॒न् तस्मा॒दपि॒ वद्ध्यं॒ प्रप॑न्नं॒ न प्रति॒ प्रय॑च्छन्ति॒ तस्मा᳚द् द्विदेव॒त्ये᳚भ्य आदि॒त्यो निर्गृ॑ह्यते॒ यदु॒च्छेष॑णा॒-दजा॑यन्त॒ तस्मा॑दु॒च्छेष॑णाद्-गृह्यते ति॒सृभि॑र्. ऋ॒ग्भिर्गृ॑ह्णाति मा॒ता पि॒ता पु॒त्रस्तदे॒व तन्मि॑थु॒नमुल्बं॒ गर्भो॑ ज॒रायु॒ तदे॒व तन्- [  ] \newline

\textbf{Pada Paata} \newline

रु॒द्रम् । अ॒न्तः । आ॒य॒न्न् । सः । आ॒दि॒त्यान् । अ॒न्वाक्र॑म॒तेत्य॑नु - आक्र॑मत । ते । द्वि॒दे॒व॒त्या॑निति॑ द्वि - दे॒व॒त्यान्॑ । प्रेति॑ । अ॒प॒द्य॒न्त॒ । तान् । न । प्रति॑ । प्रेति॑ । अ॒य॒च्छ॒न्न् । तस्मा᳚त् । अपीति॑ । वद्ध्य᳚म् । प्रप॑न्न॒मिति॒ प्र - प॒न्न॒म् । न । प्रति॑ । प्रेति॑ । य॒च्छ॒न्ति॒ । तस्मा᳚त् । द्वि॒दे॒व॒त्ये᳚भ्य॒ इति॑ द्वि - दे॒व॒त्ये᳚भ्यः । आ॒दि॒त्यः । निरिति॑ । गृ॒ह्य॒ते॒ । यत् । उ॒च्छेष॑णा॒दित्यु॑त् - शेष॑णात् । अजा॑यन्त । तस्मा᳚त् । उ॒च्छेष॑णा॒दित्यु॑त् - शेष॑णात् । गृ॒ह्य॒ते॒ । ति॒सृभि॒रिति॑ ति॒सृ - भिः॒ । ऋ॒ग्भिरित्यृ॑क् - भिः । गृ॒ह्णा॒ति॒ । मा॒ता । पि॒ता । पु॒त्रः । तत् । ए॒व । तत् । मि॒थु॒नम् । उल्ब᳚म् । गर्भः॑ । ज॒रायु॑ । तत् । ए॒व । तत् ।  \newline


\textbf{Krama Paata} \newline

रु॒द्रम॒न्तः । अ॒न्तरा॑यन्न् । आ॒य॒न्थ् सः । स आ॑दि॒त्यान् । आ॒दि॒त्या,न॒न्वाक्र॑मत । अ॒न्वाक्र॑मत॒ ते । अ॒न्वाक्र॑म॒तेत्य॑नु - आक्र॑मत । ते द्वि॑देव॒त्यान्॑ । द्वि॒दे॒व॒त्या᳚न् प्र । द्वि॒दे॒व॒त्या॑निति॑ द्वि - दे॒व॒त्यान्॑ । प्राप॑द्यन्त । अ॒प॒द्य॒न्त॒ तान् । तान् न । न प्रति॑ । प्रति॒ प्र । प्राय॑च्छन्न् । अ॒य॒च्छ॒न् तस्मा᳚त् । तस्मा॒दपि॑ । अपि॒ वद्ध्य᳚म् । वद्ध्य॒म् प्रप॑न्नम् । प्रप॑न्न॒म् न । प्रप॑न्न॒मिति॒ प्र - प॒न्न॒म् । न प्रति॑ । प्रति॒ प्र । प्र य॑च्छन्ति । य॒च्छ॒न्ति॒ तस्मा᳚त् । तस्मा᳚द् द्विदेव॒त्ये᳚भ्यः । द्वि॒दे॒व॒त्ये᳚भ्य आदि॒त्यः । द्वि॒दे॒व॒त्ये᳚भ्य॒ इति॑ द्वि - दे॒व॒त्ये᳚भ्यः । आ॒दि॒त्यो निः । निर् गृ॑ह्यते । गृ॒ह्य॒ते॒ यत् । यदु॒च्छेष॑णात् । उ॒च्छेष॑णा॒दजा॑यन्त । उ॒च्छेष॑णा॒दित्यु॑त् - शेष॑णात् । अजा॑यन्त॒ तस्मा᳚त् । तस्मा॑दु॒च्छेष॑णात् । उ॒च्छेष॑णाद् गृह्यते । उ॒च्छेष॑णा॒दित्यु॑त् - शेष॑णात् । गृ॒ह्य॒ते॒ ति॒सृभिः॑ । ति॒सृभि॑र्. ऋ॒ग्भिः । ति॒सृभि॒रिति॑ ति॒सृ - भिः॒ । ऋ॒ग्भिर् गृ॑ह्णाति । ऋ॒ग्भिरित्यृ॑क् - भिः । गृ॒ह्णा॒ति॒ मा॒ता । मा॒ता पि॒ता । पि॒ता पु॒त्रः । पु॒त्रस्तत् । तदे॒व । ए॒व तत् । तन् मि॑थु॒नम् । मि॒थु॒नमुल्ब᳚म् । उल्ब॒म् गर्भः॑ । गर्भो॑ ज॒रायु॑ । ज॒रायु॒ तत् । तदे॒व । ए॒व तत् । तन् मि॑थु॒नम् \newline

\textbf{Jatai Paata} \newline

1. रु॒द्र म॒न्त र॒न्ता रु॒द्रꣳ रु॒द्र म॒न्तः । \newline
2. अ॒न्त रा॑यन् नायन् न॒न्त र॒न्त रा॑यन्न् । \newline
3. आ॒य॒न् थ्स स आ॑यन् नाय॒न् थ्सः । \newline
4. स आ॑दि॒त्या ना॑दि॒त्यान् थ्स स आ॑दि॒त्यान् । \newline
5. आ॒दि॒त्या न॒न्वाक्र॑मता॒ न्वाक्र॑म तादि॒त्या ना॑दि॒त्या न॒न्वाक्र॑मत । \newline
6. अ॒न्वाक्र॑मत॒ ते ते᳚ ऽन्वाक्र॑मता॒ न्वाक्र॑मत॒ ते । \newline
7. अ॒न्वाक्र॑म॒तेत्य॑नु - आक्र॑मत । \newline
8. ते द्वि॑देव॒त्या᳚न् द्विदेव॒त्या᳚न् ते ते द्वि॑देव॒त्यान्॑ । \newline
9. द्वि॒दे॒व॒त्या᳚न् प्र प्र द्वि॑देव॒त्या᳚न् द्विदेव॒त्या᳚न् प्र । \newline
10. द्वि॒दे॒व॒त्या॑निति॑ द्वि - दे॒व॒त्यान्॑ । \newline
11. प्राप॑द्यन्ता पद्यन्त॒ प्र प्राप॑द्यन्त । \newline
12. अ॒प॒द्य॒न्त॒ ताꣳ स्ता न॑पद्यन्ता पद्यन्त॒ तान् । \newline
13. तान् न न ताꣳ स्तान् न । \newline
14. न प्रति॒ प्रति॒ न न प्रति॑ । \newline
15. प्रति॒ प्र प्र प्रति॒ प्रति॒ प्र । \newline
16. प्राय॑च्छन् नयच्छ॒न् प्र प्राय॑च्छन्न् । \newline
17. अ॒य॒च्छ॒न् तस्मा॒त् तस्मा॑ दयच्छन् नयच्छ॒न् तस्मा᳚त् । \newline
18. तस्मा॒ दप्यपि॒ तस्मा॒त् तस्मा॒ दपि॑ । \newline
19. अपि॒ वद्ध्यं॒ ॅवद्ध्य॒ मप्यपि॒ वद्ध्य᳚म् । \newline
20. वद्ध्य॒म् प्रप॑न्न॒म् प्रप॑न्नं॒ ॅवद्ध्यं॒ ॅवद्ध्य॒म् प्रप॑न्नम् । \newline
21. प्रप॑न्न॒न् न न प्रप॑न्न॒म् प्रप॑न्न॒न् न । \newline
22. प्रप॑न्न॒मिति॒ प्र - प॒न्न॒म् । \newline
23. न प्रति॒ प्रति॒ न न प्रति॑ । \newline
24. प्रति॒ प्र प्र प्रति॒ प्रति॒ प्र । \newline
25. प्र य॑च्छन्ति यच्छन्ति॒ प्र प्र य॑च्छन्ति । \newline
26. य॒च्छ॒न्ति॒ तस्मा॒त् तस्मा᳚द् यच्छन्ति यच्छन्ति॒ तस्मा᳚त् । \newline
27. तस्मा᳚द् द्विदेव॒त्ये᳚भ्यो द्विदेव॒त्ये᳚भ्य॒ स्तस्मा॒त् तस्मा᳚द् द्विदेव॒त्ये᳚भ्यः । \newline
28. द्वि॒दे॒व॒त्ये᳚भ्य आदि॒त्य आ॑दि॒त्यो द्वि॑देव॒त्ये᳚भ्यो द्विदेव॒त्ये᳚भ्य आदि॒त्यः । \newline
29. द्वि॒दे॒व॒त्ये᳚भ्य॒ इति॑ द्वि - दे॒व॒त्ये᳚भ्यः । \newline
30. आ॒दि॒त्यो निर् णिरा॑दि॒त्य आ॑दि॒त्यो निः । \newline
31. निर् गृ॑ह्यते गृह्यते॒ निर् णिर् गृ॑ह्यते । \newline
32. गृ॒ह्य॒ते॒ यद् यद् गृ॑ह्यते गृह्यते॒ यत् । \newline
33. यदु॒च्छेष॑णा दु॒च्छेष॑णा॒द् यद् यदु॒च्छेष॑णात् । \newline
34. उ॒च्छेष॑णा॒ दजा॑य॒न्ता जा॑य न्तो॒च्छेष॑णा दु॒च्छेष॑णा॒ दजा॑यन्त । \newline
35. उ॒च्छेष॑णा॒दित्यु॑त् - शेष॑णात् । \newline
36. अजा॑यन्त॒ तस्मा॒त् तस्मा॒ दजा॑य॒न्ता जा॑यन्त॒ तस्मा᳚त् । \newline
37. तस्मा॑ दु॒च्छेष॑णा दु॒च्छेष॑णा॒त् तस्मा॒त् तस्मा॑ दु॒च्छेष॑णात् । \newline
38. उ॒च्छेष॑णाद् गृह्यते गृह्यत उ॒च्छेष॑णा दु॒च्छेष॑णाद् गृह्यते । \newline
39. उ॒च्छेष॑णा॒दित्यु॑त् - शेष॑णात् । \newline
40. गृ॒ह्य॒ते॒ ति॒सृभि॑ स्ति॒सृभि॑र् गृह्यते गृह्यते ति॒सृभिः॑ । \newline
41. ति॒सृभिर्॑. ऋ॒ग्भिर्. ऋ॒ग्भि स्ति॒सृभि॑ स्ति॒सृभिर्॑. ऋ॒ग्भिः । \newline
42. ति॒सृभि॒रिति॑ ति॒सृ - भिः॒ । \newline
43. ऋ॒ग्भिर् गृ॑ह्णाति गृह्णा त्यृ॒ग्भिर्. ऋ॒ग्भिर् गृ॑ह्णाति । \newline
44. ऋ॒ग्भिरित्यृ॑क् - भिः । \newline
45. गृ॒ह्णा॒ति॒ मा॒ता मा॒ता गृ॑ह्णाति गृह्णाति मा॒ता । \newline
46. मा॒ता पि॒ता पि॒ता मा॒ता मा॒ता पि॒ता । \newline
47. पि॒ता पु॒त्रः पु॒त्रः पि॒ता पि॒ता पु॒त्रः । \newline
48. पु॒त्र स्तत् तत् पु॒त्रः पु॒त्र स्तत् । \newline
49. तदे॒ वैव तत् तदे॒व । \newline
50. ए॒व तत् तदे॒ वैव तत् । \newline
51. तन् मि॑थु॒नम् मि॑थु॒नम् तत् तन् मि॑थु॒नम् । \newline
52. मि॒थु॒न मुल्ब॒ मुल्ब॑म् मिथु॒नम् मि॑थु॒न मुल्ब᳚म् । \newline
53. उल्ब॒म् गर्भो॒ गर्भ॒ उल्ब॒ मुल्ब॒म् गर्भः॑ । \newline
54. गर्भो॑ ज॒रायु॑ ज॒रायु॒ गर्भो॒ गर्भो॑ ज॒रायु॑ । \newline
55. ज॒रायु॒ तत् तज् ज॒रायु॑ ज॒रायु॒ तत् । \newline
56. तदे॒ वैव तत् तदे॒व । \newline
57. ए॒व तत् तदे॒ वैव तत् । \newline
58. तन् मि॑थु॒नम् मि॑थु॒नम् तत् तन् मि॑थु॒नम् । \newline

\textbf{Ghana Paata } \newline

1. रु॒द्र म॒न्त र॒न्ता रु॒द्रꣳ रु॒द्र म॒न्त रा॑यन्-नायन्-न॒न्ता रु॒द्रꣳ रु॒द्र म॒न्त रा॑यन्न् । \newline
2. अ॒न्त रा॑यन्-नायन्-न॒न्त र॒न्त रा॑य॒न् थ्स स आ॑यन्-न॒न्त र॒न्त रा॑य॒न् थ्सः । \newline
3. आ॒य॒न् थ्स स आ॑यन्-नाय॒न् थ्स आ॑दि॒त्या-ना॑दि॒त्यान् थ्स आ॑यन्-नाय॒न् थ्स आ॑दि॒त्यान् । \newline
4. स आ॑दि॒त्या-ना॑दि॒त्यान् थ्स स आ॑दि॒त्या-न॒न्वाक्र॑मता॒ न्वाक्र॑म तादि॒त्यान् थ्स स आ॑दि॒त्या-न॒न्वाक्र॑मत । \newline
5. आ॒दि॒त्या-न॒न्वाक्र॑मता॒ न्वाक्र॑म तादि॒त्या-ना॑दि॒त्या-न॒न्वाक्र॑मत॒ ते ते᳚ ऽन्वाक्र॑म तादि॒त्या-ना॑दि॒त्या-न॒न्वाक्र॑मत॒ ते । \newline
6. अ॒न्वाक्र॑मत॒ ते ते᳚ ऽन्वाक्र॑मता॒ न्वाक्र॑मत॒ ते द्वि॑देव॒त्या᳚न् द्विदेव॒त्या᳚न् (1॒) ते᳚ ऽन्वाक्र॑मता॒ न्वाक्र॑मत॒ ते द्वि॑देव॒त्यान्॑ । \newline
7. अ॒न्वाक्र॑म॒तेत्य॑नु - आक्र॑मत । \newline
8. ते द्वि॑देव॒त्या᳚न् द्विदेव॒त्या᳚न् ते ते द्वि॑देव॒त्या᳚न् प्र प्र द्वि॑देव॒त्या᳚न् ते ते द्वि॑देव॒त्या᳚न् प्र । \newline
9. द्वि॒दे॒व॒त्या᳚न् प्र प्र द्वि॑देव॒त्या᳚न् द्विदेव॒त्या᳚न् प्राप॑द्यन्ता पद्यन्त॒ प्र द्वि॑देव॒त्या᳚न् द्विदेव॒त्या᳚न् प्राप॑द्यन्त । \newline
10. द्वि॒दे॒व॒त्या॑निति॑ द्वि - दे॒व॒त्यान्॑ । \newline
11. प्राप॑द्यन्ता पद्यन्त॒ प्र प्राप॑द्यन्त॒ ताꣳ स्तान॑पद्यन्त॒ प्र प्राप॑द्यन्त॒ तान् । \newline
12. अ॒प॒द्य॒न्त॒ ताꣳ स्तान॑पद्यन्त् आपद्यन्त॒ तान् न न तान॑पद्यन्ता पद्यन्त॒ तान् न । \newline
13. तान् न न ताꣳ स्तान् न प्रति॒ प्रति॒ न ताꣳ स्तान् न प्रति॑ । \newline
14. न प्रति॒ प्रति॒ न न प्रति॒ प्र प्र प्रति॒ न न प्रति॒ प्र । \newline
15. प्रति॒ प्र प्र प्रति॒ प्रति॒ प्राय॑च्छन्-नयच्छ॒न् प्र प्रति॒ प्रति॒ प्राय॑च्छन्न् । \newline
16. प्राय॑च्छन्-नयच्छ॒न् प्र प्राय॑च्छ॒न् तस्मा॒त् तस्मा॑ दयच्छ॒न् प्र प्राय॑च्छ॒न् तस्मा᳚त् । \newline
17. अ॒य॒च्छ॒न् तस्मा॒त् तस्मा॑ दयच्छन्-नयच्छ॒न् तस्मा॒ दप्यपि॒ तस्मा॑ दयच्छन्-नयच्छ॒न् तस्मा॒ दपि॑ । \newline
18. तस्मा॒ दप्यपि॒ तस्मा॒त् तस्मा॒ दपि॒ वद्ध्यं॒ ॅवद्ध्य॒ मपि॒ तस्मा॒त् तस्मा॒ दपि॒ वद्ध्य᳚म् । \newline
19. अपि॒ वद्ध्यं॒ ॅवद्ध्य॒ मप्यपि॒ वद्ध्य॒म् प्रप॑न्न॒म् प्रप॑न्नं॒ ॅवद्ध्य॒ मप्यपि॒ वद्ध्य॒म् प्रप॑न्नम् । \newline
20. वद्ध्य॒म् प्रप॑न्न॒म् प्रप॑न्नं॒ ॅवद्ध्यं॒ ॅवद्ध्य॒म् प्रप॑न्न॒न् न न प्रप॑न्नं॒ ॅवद्ध्यं॒ ॅवद्ध्य॒म् प्रप॑न्न॒न् न । \newline
21. प्रप॑न्न॒न् न न प्रप॑न्न॒म् प्रप॑न्न॒न् न प्रति॒ प्रति॒ न प्रप॑न्न॒म् प्रप॑न्न॒न् न प्रति॑ । \newline
22. प्रप॑न्न॒मिति॒ प्र - प॒न्न॒म् । \newline
23. न प्रति॒ प्रति॒ न न प्रति॒ प्र प्र प्रति॒ न न प्रति॒ प्र । \newline
24. प्रति॒ प्र प्र प्रति॒ प्रति॒ प्र य॑च्छन्ति यच्छन्ति॒ प्र प्रति॒ प्रति॒ प्र य॑च्छन्ति । \newline
25. प्र य॑च्छन्ति यच्छन्ति॒ प्र प्र य॑च्छन्ति॒ तस्मा॒त् तस्मा᳚द् यच्छन्ति॒ प्र प्र य॑च्छन्ति॒ तस्मा᳚त् । \newline
26. य॒च्छ॒न्ति॒ तस्मा॒त् तस्मा᳚द् यच्छन्ति यच्छन्ति॒ तस्मा᳚द् द्विदेव॒त्ये᳚भ्यो द्विदेव॒त्ये᳚भ्य॒ स्तस्मा᳚द् यच्छन्ति यच्छन्ति॒ तस्मा᳚द् द्विदेव॒त्ये᳚भ्यः । \newline
27. तस्मा᳚द् द्विदेव॒त्ये᳚भ्यो द्विदेव॒त्ये᳚भ्य॒ स्तस्मा॒त् तस्मा᳚द् द्विदेव॒त्ये᳚भ्य आदि॒त्य आ॑दि॒त्यो द्वि॑देव॒त्ये᳚भ्य॒ स्तस्मा॒त् तस्मा᳚द् द्विदेव॒त्ये᳚भ्य आदि॒त्यः । \newline
28. द्वि॒दे॒व॒त्ये᳚भ्य आदि॒त्य आ॑दि॒त्यो द्वि॑देव॒त्ये᳚भ्यो द्विदेव॒त्ये᳚भ्य आदि॒त्यो निर् णिरा॑दि॒त्यो द्वि॑देव॒त्ये᳚भ्यो द्विदेव॒त्ये᳚भ्य आदि॒त्यो निः । \newline
29. द्वि॒दे॒व॒त्ये᳚भ्य॒ इति॑ द्वि - दे॒व॒त्ये᳚भ्यः । \newline
30. आ॒दि॒त्यो निर् णिरा॑दि॒त्य आ॑दि॒त्यो निर् गृ॑ह्यते गृह्यते॒ निरा॑दि॒त्य आ॑दि॒त्यो निर् गृ॑ह्यते । \newline
31. निर् गृ॑ह्यते गृह्यते॒ निर् णिर् गृ॑ह्यते॒ यद् यद् गृ॑ह्यते॒ निर् णिर् गृ॑ह्यते॒ यत् । \newline
32. गृ॒ह्य॒ते॒ यद् यद् गृ॑ह्यते गृह्यते॒ यदु॒च्छेष॑णा दु॒च्छेष॑णा॒द् यद् गृ॑ह्यते गृह्यते॒ यदु॒च्छेष॑णात् । \newline
33. यदु॒च्छेष॑णा दु॒च्छेष॑णा॒द् यद् यदु॒च्छेष॑णा॒ दजा॑य॒न्ता जा॑य न्तो॒च्छेष॑णा॒द् यद् यदु॒च्छेष॑णा॒ दजा॑यन्त । \newline
34. उ॒च्छेष॑णा॒ दजा॑य॒न्ता जा॑यन्तो॒च्छेष॑णा दु॒च्छेष॑णा॒ दजा॑यन्त॒ तस्मा॒त् तस्मा॒ दजा॑यन्तो॒च्छेष॑णा दु॒च्छेष॑णा॒ दजा॑यन्त॒ तस्मा᳚त् । \newline
35. उ॒च्छेष॑णा॒दित्यु॑त् - शेष॑णात् । \newline
36. अजा॑यन्त॒ तस्मा॒त् तस्मा॒ दजा॑य॒न्ता जा॑यन्त॒ तस्मा॑ दु॒च्छेष॑णा दु॒च्छेष॑णा॒त् तस्मा॒ दजा॑य॒न्ता जा॑यन्त॒ तस्मा॑ दु॒च्छेष॑णात् । \newline
37. तस्मा॑दु॒ च्छेष॑णा दु॒च्छेष॑णा॒त् तस्मा॒त् तस्मा॑ दु॒च्छेष॑णाद् गृह्यते गृह्यत उ॒च्छेष॑णा॒त् तस्मा॒त् तस्मा॑ दु॒च्छेष॑णाद् गृह्यते । \newline
38. उ॒च्छेष॑णाद् गृह्यते गृह्यत उ॒च्छेष॑णा दु॒च्छेष॑णाद् गृह्यते ति॒सृभि॑ स्ति॒सृभि॑र् गृह्यत उ॒च्छेष॑णा दु॒च्छेष॑णाद् गृह्यते ति॒सृभिः॑ । \newline
39. उ॒च्छेष॑णा॒दित्यु॑त् - शेष॑णात् । \newline
40. गृ॒ह्य॒ते॒ ति॒सृभि॑ स्ति॒सृभि॑र् गृह्यते गृह्यते ति॒सृभिर्॑. ऋ॒ग्भिर्. ऋ॒ग्भि स्ति॒सृभि॑र् गृह्यते गृह्यते ति॒सृभिर्॑. ऋ॒ग्भिः । \newline
41. ति॒सृभिर्॑. ऋ॒ग्भिर्. ऋ॒ग्भि स्ति॒सृभि॑ स्ति॒सृभिर्॑. ऋ॒ग्भिर् गृ॑ह्णाति गृह्णा त्यृ॒ग्भि स्ति॒सृभि॑ स्ति॒सृभिर्॑. ऋ॒ग्भिर् गृ॑ह्णाति । \newline
42. ति॒सृभि॒रिति॑ ति॒सृ - भिः॒ । \newline
43. ऋ॒ग्भिर् गृ॑ह्णाति गृह्णा त्यृ॒ग्भिर्. ऋ॒ग्भिर् गृ॑ह्णाति मा॒ता मा॒ता गृ॑ह्णा त्यृ॒ग्भिर्. ऋ॒ग्भिर् गृ॑ह्णाति मा॒ता । \newline
44. ऋ॒ग्भिरित्यृ॑क् - भिः । \newline
45. गृ॒ह्णा॒ति॒ मा॒ता मा॒ता गृ॑ह्णाति गृह्णाति मा॒ता पि॒ता पि॒ता मा॒ता गृ॑ह्णाति गृह्णाति मा॒ता पि॒ता । \newline
46. मा॒ता पि॒ता पि॒ता मा॒ता मा॒ता पि॒ता पु॒त्रः पु॒त्रः पि॒ता मा॒ता मा॒ता पि॒ता पु॒त्रः । \newline
47. पि॒ता पु॒त्रः पु॒त्रः पि॒ता पि॒ता पु॒त्र स्तत् तत् पु॒त्रः पि॒ता पि॒ता पु॒त्र स्तत् । \newline
48. पु॒त्र स्तत् तत् पु॒त्रः पु॒त्र स्त दे॒वैव तत् पु॒त्रः पु॒त्र स्तदे॒व । \newline
49. तदे॒ वैव तत् तदे॒व तत् तदे॒व तत् तदे॒व तत् । \newline
50. ए॒व तत् तदे॒ वैव तन् मि॑थु॒नम् मि॑थु॒नम् तदे॒ वैव तन् मि॑थु॒नम् । \newline
51. तन् मि॑थु॒नम् मि॑थु॒नम् तत् तन् मि॑थु॒न मुल्ब॒ मुल्ब॑म् मिथु॒नम् तत् तन् मि॑थु॒न मुल्ब᳚म् । \newline
52. मि॒थु॒न मुल्ब॒ मुल्ब॑म् मिथु॒नम् मि॑थु॒न मुल्ब॒म् गर्भो॒ गर्भ॒ उल्ब॑म् मिथु॒नम् मि॑थु॒न मुल्ब॒म् गर्भः॑ । \newline
53. उल्ब॒म् गर्भो॒ गर्भ॒ उल्ब॒ मुल्ब॒म् गर्भो॑ ज॒रायु॑ ज॒रायु॒ गर्भ॒ उल्ब॒ मुल्ब॒म् गर्भो॑ ज॒रायु॑ । \newline
54. गर्भो॑ ज॒रायु॑ ज॒रायु॒ गर्भो॒ गर्भो॑ ज॒रायु॒ तत् तज् ज॒रायु॒ गर्भो॒ गर्भो॑ ज॒रायु॒ तत् । \newline
55. ज॒रायु॒ तत् तज् ज॒रायु॑ ज॒रायु॒ तदे॒ वैव तज् ज॒रायु॑ ज॒रायु॒ तदे॒व । \newline
56. तदे॒ वैव तत् तदे॒व तत् तदे॒व तत् तदे॒व तत् । \newline
57. ए॒व तत् तदे॒ वैव तन् मि॑थु॒नम् मि॑थु॒नम् तदे॒ वैव तन् मि॑थु॒नम् । \newline
58. तन् मि॑थु॒नम् मि॑थु॒नम् तत् तन् मि॑थु॒नम् प॒शवः॑ प॒शवो॑ मिथु॒नम् तत् तन् मि॑थु॒नम् प॒शवः॑ । \newline
\pagebreak
\markright{ TS 6.5.6.4  \hfill https://www.vedavms.in \hfill}

\section{ TS 6.5.6.4 }

\textbf{TS 6.5.6.4 } \newline
\textbf{Samhita Paata} \newline

-मि॑थु॒नं प॒शवो॒ वा ए॒ते यदा॑दि॒त्य ऊर्ग्दधि॑ द॒द्ध्ना म॑द्ध्य॒तः श्री॑णा॒त्यूर्ज॑मे॒व प॑शू॒नां म॑द्ध्य॒तो द॑धाति शृतात॒ङ्क्ये॑न मेद्ध्य॒त्वाय॒ तस्मा॑दा॒मा प॒क्वं दु॑हे प॒शवो॒ वा ए॒ते यदा॑दि॒त्यः प॑रि॒श्रित्य॑ गृह्णाति प्रति॒रुद्ध्यै॒वास्मै॑ प॒शून् गृ॑ह्णाति प॒शवो॒ वा ए॒ते यदा॑दि॒त्य ए॒ष रु॒द्रो यद॒ग्निः प॑रि॒श्रित्य॑ गृह्णाति रु॒द्रादे॒व प॒शून॒न्तर्द॑धा- [  ] \newline

\textbf{Pada Paata} \newline

मि॒थु॒नम् । प॒शवः॑ । वै । ए॒ते । यत् । आ॒दि॒त्यः । ऊर्क् । दधि॑ । द॒द्ध्ना । म॒द्ध्य॒तः । श्री॒णा॒ति॒ । ऊर्ज᳚म् । ए॒व । प॒शू॒नाम् । म॒द्ध्य॒तः । द॒धा॒ति॒ । शृ॒ता॒त॒ङ्क्ये॑नेति॑ शृत - आ॒त॒ङ्क्ये॑न । मे॒द्ध्य॒त्वायेति॑ मेद्ध्य - त्वाय॑ । तस्मा᳚त् । आ॒मा । प॒क्वम् । दु॒हे॒ । प॒शवः॑ । वै । ए॒ते । यत् । आ॒दि॒त्यः । प॒रि॒श्रित्येति॑ परि - श्रित्य॑ । गृ॒ह्णा॒ति॒ । प्र॒ति॒रुद्ध्येति॑ प्रति - रुद्ध्य॑ । ए॒व । अ॒स्मै॒ । प॒शून् । गृ॒ह्णा॒ति॒ । प॒शवः॑ । वै । ए॒ते । यत् । आ॒दि॒त्यः । ए॒षः । रु॒द्रः । यत् । अ॒ग्निः । प॒रि॒श्रित्येति॑ परि - श्रित्य॑ । गृ॒ह्णा॒ति॒ । रु॒द्रात् । ए॒व । प॒शून् । अ॒न्तः । द॒धा॒ति॒ ।  \newline


\textbf{Krama Paata} \newline

मि॒थु॒नम् प॒शवः॑ । प॒शवो॒ वै । वा ए॒ते । ए॒ते यत् । यदा॑दि॒त्यः । आ॒दि॒त्य ऊर्क् । ऊर्ग् दधि॑ । दधि॑ द॒द्ध्ना । द॒द्ध्ना म॑द्ध्य॒तः । म॒द्ध्य॒तः श्री॑णाति । श्री॒णा॒त्यूर्ज᳚म् । ऊर्ज॑मे॒व । ए॒व प॑शू॒नाम् । प॒शू॒नाम् म॑द्ध्य॒तः । म॒द्ध्य॒तो द॑धाति । द॒धा॒ति॒ शृ॒ता॒त॒ङ्‍क्ये॑न । शृ॒ता॒त॒ङ्‍क्ये॑न मेद्ध्य॒त्वाय॑ । शृ॒ता॒त॒ङ्‍क्ये॑नेति॑ शृत - आ॒त॒ङ्‍क्ये॑न । मे॒द्ध्य॒त्वाय॒ तस्मा᳚त् । मे॒द्ध्य॒त्वायेति॑ मेद्ध्य - त्वाय॑ । तस्मा॑दा॒मा । आ॒मा प॒क्वम् । प॒क्वम् दु॑हे । दु॒हे॒ प॒शवः॑ । प॒शवो॒ वै । वा ए॒ते । ए॒ते यत् । यदा॑दि॒त्यः । आ॒दि॒त्यः प॑रि॒श्रित्य॑ । प॒रि॒श्रित्य॑ गृह्णाति । प॒रि॒श्रित्येति॑ परि - श्रित्य॑ । गृ॒ह्णा॒ति॒ प्र॒ति॒रुद्ध्य॑ । प्र॒ति॒रुद्ध्यै॒व । प्र॒ति॒रुद्ध्येति॑ प्रति - रुद्ध्य॑ । ए॒वास्मै᳚ । अ॒स्मै॒ प॒शून् । प॒शून् गृ॑ह्णाति । गृ॒ह्णा॒ति॒ प॒शवः॑ । प॒शवो॒ वै । वा ए॒ते । ए॒ते यत् । यदा॑दि॒त्यः । आ॒दि॒त्य ए॒षः । ए॒ष रु॒द्रः । रु॒द्रो यत् । यद॒ग्निः । अ॒ग्निः प॑रि॒श्रित्य॑ । प॒रि॒श्रित्य॑ गृह्णाति । प॒रि॒श्रित्येति॑ परि - श्रित्य॑ । गृ॒ह्णा॒ति॒ रु॒द्रात्  । रु॒द्रादे॒व । ए॒व प॒शून् । प॒शून॒न्तः । अ॒न्तर् द॑धाति । द॒धा॒त्ये॒षः \newline

\textbf{Jatai Paata} \newline

1. मि॒थु॒नम् प॒शवः॑ प॒शवो॑ मिथु॒नम् मि॑थु॒नम् प॒शवः॑ । \newline
2. प॒शवो॒ वै वै प॒शवः॑ प॒शवो॒ वै । \newline
3. वा ए॒त ए॒ते वै वा ए॒ते । \newline
4. ए॒ते यद् यदे॒त ए॒ते यत् । \newline
5. यदा॑दि॒त्य आ॑दि॒त्यो यद् यदा॑दि॒त्यः । \newline
6. आ॒दि॒त्य ऊर्गूर् गा॑दि॒त्य आ॑दि॒त्य ऊर्क् । \newline
7. ऊर्ग् दधि॒ दध्यूर् गूर्ग् दधि॑ । \newline
8. दधि॑ द॒द्ध्ना द॒द्ध्ना दधि॒ दधि॑ द॒द्ध्ना । \newline
9. द॒द्ध्ना म॑द्ध्य॒तो म॑द्ध्य॒तो द॒द्ध्ना द॒द्ध्ना म॑द्ध्य॒तः । \newline
10. म॒द्ध्य॒तः श्री॑णाति श्रीणाति मद्ध्य॒तो म॑द्ध्य॒तः श्री॑णाति । \newline
11. श्री॒णा॒ त्यूर्ज॒ मूर्जꣳ॑ श्रीणाति श्रीणा॒ त्यूर्ज᳚म् । \newline
12. ऊर्ज॑ मे॒वै वोर्ज॒ मूर्ज॑ मे॒व । \newline
13. ए॒व प॑शू॒नाम् प॑शू॒ना मे॒वैव प॑शू॒नाम् । \newline
14. प॒शू॒नाम् म॑द्ध्य॒तो म॑द्ध्य॒तः प॑शू॒नाम् प॑शू॒नाम् म॑द्ध्य॒तः । \newline
15. म॒द्ध्य॒तो द॑धाति दधाति मद्ध्य॒तो म॑द्ध्य॒तो द॑धाति । \newline
16. द॒धा॒ति॒ शृ॒ता॒त॒ङ्क्ये॑न शृतात॒ङ्क्ये॑न दधाति दधाति शृतात॒ङ्क्ये॑न । \newline
17. शृ॒ता॒त॒ङ्क्ये॑न मेद्ध्य॒त्वाय॑ मेद्ध्य॒त्वाय॑ शृतात॒ङ्क्ये॑न शृतात॒ङ्क्ये॑न मेद्ध्य॒त्वाय॑ । \newline
18. शृ॒ता॒त॒ङ्क्ये॑नेति॑ शृत - आ॒त॒ङ्क्ये॑न । \newline
19. मे॒द्ध्य॒त्वाय॒ तस्मा॒त् तस्मा᳚न् मेद्ध्य॒त्वाय॑ मेद्ध्य॒त्वाय॒ तस्मा᳚त् । \newline
20. मे॒द्ध्य॒त्वायेति॑ मेद्ध्य - त्वाय॑ । \newline
21. तस्मा॑ दा॒मा ऽऽमा तस्मा॒त् तस्मा॑ दा॒मा । \newline
22. आ॒मा प॒क्वम् प॒क्व मा॒मा ऽऽमा प॒क्वम् । \newline
23. प॒क्वम् दु॑हे दुहे प॒क्वम् प॒क्वम् दु॑हे । \newline
24. दु॒हे॒ प॒शवः॑ प॒शवो॑ दुहे दुहे प॒शवः॑ । \newline
25. प॒शवो॒ वै वै प॒शवः॑ प॒शवो॒ वै । \newline
26. वा ए॒त ए॒ते वै वा ए॒ते । \newline
27. ए॒ते यद् यदे॒त ए॒ते यत् । \newline
28. यदा॑दि॒त्य आ॑दि॒त्यो यद् यदा॑दि॒त्यः । \newline
29. आ॒दि॒त्यः प॑रि॒श्रित्य॑ परि॒श्रि त्या॑दि॒त्य आ॑दि॒त्यः प॑रि॒श्रित्य॑ । \newline
30. प॒रि॒श्रित्य॑ गृह्णाति गृह्णाति परि॒श्रित्य॑ परि॒श्रित्य॑ गृह्णाति । \newline
31. प॒रि॒श्रित्येति॑ परि - श्रित्य॑ । \newline
32. गृ॒ह्णा॒ति॒ प्र॒ति॒रुद्ध्य॑ प्रति॒रुद्ध्य॑ गृह्णाति गृह्णाति प्रति॒रुद्ध्य॑ । \newline
33. प्र॒ति॒रुद्ध्यै॒वैव प्र॑ति॒रुद्ध्य॑ प्रति॒रुद्ध् यै॒व । \newline
34. प्र॒ति॒रुद्ध्येति॑ प्रति - रुद्ध्य॑ । \newline
35. ए॒वास्मा॑ अस्मा ए॒वै वास्मै᳚ । \newline
36. अ॒स्मै॒ प॒शून् प॒शू न॑स्मा अस्मै प॒शून् । \newline
37. प॒शून् गृ॑ह्णाति गृह्णाति प॒शून् प॒शून् गृ॑ह्णाति । \newline
38. गृ॒ह्णा॒ति॒ प॒शवः॑ प॒शवो॑ गृह्णाति गृह्णाति प॒शवः॑ । \newline
39. प॒शवो॒ वै वै प॒शवः॑ प॒शवो॒ वै । \newline
40. वा ए॒त ए॒ते वै वा ए॒ते । \newline
41. ए॒ते यद् यदे॒त ए॒ते यत् । \newline
42. यदा॑दि॒त्य आ॑दि॒त्यो यद् यदा॑दि॒त्यः । \newline
43. आ॒दि॒त्य ए॒ष ए॒ष आ॑दि॒त्य आ॑दि॒त्य ए॒षः । \newline
44. ए॒ष रु॒द्रो रु॒द्र ए॒ष ए॒ष रु॒द्रः । \newline
45. रु॒द्रो यद् यद् रु॒द्रो रु॒द्रो यत् । \newline
46. यद॒ग्नि र॒ग्निर् यद् यद॒ग्निः । \newline
47. अ॒ग्निः प॑रि॒श्रित्य॑ परि॒श्रि त्या॒ग्नि र॒ग्निः प॑रि॒श्रित्य॑ । \newline
48. प॒रि॒श्रित्य॑ गृह्णाति गृह्णाति परि॒श्रित्य॑ परि॒श्रित्य॑ गृह्णाति । \newline
49. प॒रि॒श्रित्येति॑ परि - श्रित्य॑ । \newline
50. गृ॒ह्णा॒ति॒ रु॒द्राद् रु॒द्राद् गृ॑ह्णाति गृह्णाति रु॒द्रात् । \newline
51. रु॒द्रा दे॒वैव रु॒द्राद् रु॒द्रा दे॒व । \newline
52. ए॒व प॒शून् प॒शूने॒ वैव प॒शून् । \newline
53. प॒शू न॒न्त र॒न्तः प॒शून् प॒शू न॒न्तः । \newline
54. अ॒न्तर् द॑धाति दधा त्य॒न्त र॒न्तर् द॑धाति । \newline
55. द॒धा॒ त्ये॒ष ए॒ष द॑धाति दधा त्ये॒षः । \newline

\textbf{Ghana Paata } \newline

1. मि॒थु॒नम् प॒शवः॑ प॒शवो॑ मिथु॒नम् मि॑थु॒नम् प॒शवो॒ वै वै प॒शवो॑ मिथु॒नम् मि॑थु॒नम् प॒शवो॒ वै । \newline
2. प॒शवो॒ वै वै प॒शवः॑ प॒शवो॒ वा ए॒त ए॒ते वै प॒शवः॑ प॒शवो॒ वा ए॒ते । \newline
3. वा ए॒त ए॒ते वै वा ए॒ते यद् यदे॒ते वै वा ए॒ते यत् । \newline
4. ए॒ते यद् यदे॒त ए॒ते यदा॑दि॒त्य आ॑दि॒त्यो यदे॒त ए॒ते यदा॑दि॒त्यः । \newline
5. यदा॑दि॒त्य आ॑दि॒त्यो यद् यदा॑दि॒त्य ऊर् गूर् गा॑दि॒त्यो यद् यदा॑दि॒त्य ऊर्क् । \newline
6. आ॒दि॒त्य ऊर् गूर् गा॑दि॒त्य आ॑दि॒त्य ऊर्ग् दधि॒ दध्यूर् गा॑दि॒त्य आ॑दि॒त्य ऊर्ग् दधि॑ । \newline
7. ऊर्ग् दधि॒ दध्यूर् गूर्ग् दधि॑ द॒द्ध्ना द॒द्ध्ना दध्यूर् गूर्ग् दधि॑ द॒द्ध्ना । \newline
8. दधि॑ द॒द्ध्ना द॒द्ध्ना दधि॒ दधि॑ द॒द्ध्ना म॑द्ध्य॒तो म॑द्ध्य॒तो द॒द्ध्ना दधि॒ दधि॑ द॒द्ध्ना म॑द्ध्य॒तः । \newline
9. द॒द्ध्ना म॑द्ध्य॒तो म॑द्ध्य॒तो द॒द्ध्ना द॒द्ध्ना म॑द्ध्य॒तः श्री॑णाति श्रीणाति मद्ध्य॒तो द॒द्ध्ना द॒द्ध्ना म॑द्ध्य॒तः श्री॑णाति । \newline
10. म॒द्ध्य॒तः श्री॑णाति श्रीणाति मद्ध्य॒तो म॑द्ध्य॒तः श्री॑णा॒ त्यूर्ज॒ मूर्जꣳ॑ श्रीणाति मद्ध्य॒तो म॑द्ध्य॒तः श्री॑णा॒ त्यूर्ज᳚म् । \newline
11. श्री॒णा॒ त्यूर्ज॒ मूर्जꣳ॑ श्रीणाति श्रीणा॒ त्यूर्ज॑ मे॒वै वोर्जꣳ॑ श्रीणाति श्रीणा॒ त्यूर्ज॑ मे॒व । \newline
12. ऊर्ज॑ मे॒वै वोर्ज॒ मूर्ज॑ मे॒व प॑शू॒नाम् प॑शू॒ना मे॒वोर्ज॒ मूर्ज॑ मे॒व प॑शू॒नाम् । \newline
13. ए॒व प॑शू॒नाम् प॑शू॒ना मे॒वैव प॑शू॒नाम् म॑द्ध्य॒तो म॑द्ध्य॒तः प॑शू॒ना मे॒वैव प॑शू॒नाम् म॑द्ध्य॒तः । \newline
14. प॒शू॒नाम् म॑द्ध्य॒तो म॑द्ध्य॒तः प॑शू॒नाम् प॑शू॒नाम् म॑द्ध्य॒तो द॑धाति दधाति मद्ध्य॒तः प॑शू॒नाम् प॑शू॒नाम् म॑द्ध्य॒तो द॑धाति । \newline
15. म॒द्ध्य॒तो द॑धाति दधाति मद्ध्य॒तो म॑द्ध्य॒तो द॑धाति शृतात॒ङ्क्ये॑न शृतात॒ङ्क्ये॑न दधाति मद्ध्य॒तो म॑द्ध्य॒तो द॑धाति शृतात॒ङ्क्ये॑न । \newline
16. द॒धा॒ति॒ शृ॒ता॒त॒ङ्क्ये॑न शृतात॒ङ्क्ये॑न दधाति दधाति शृतात॒ङ्क्ये॑न मेद्ध्य॒त्वाय॑ मेद्ध्य॒त्वाय॑ शृतात॒ङ्क्ये॑न दधाति दधाति शृतात॒ङ्क्ये॑न मेद्ध्य॒त्वाय॑ । \newline
17. शृ॒ता॒त॒ङ्क्ये॑न मेद्ध्य॒त्वाय॑ मेद्ध्य॒त्वाय॑ शृतात॒ङ्क्ये॑न शृतात॒ङ्क्ये॑न मेद्ध्य॒त्वाय॒ तस्मा॒त् तस्मा᳚न् मेद्ध्य॒त्वाय॑ शृतात॒ङ्क्ये॑न शृतात॒ङ्क्ये॑न मेद्ध्य॒त्वाय॒ तस्मा᳚त् । \newline
18. शृ॒ता॒त॒ङ्क्ये॑नेति॑ शृत - आ॒त॒ङ्क्ये॑न । \newline
19. मे॒द्ध्य॒त्वाय॒ तस्मा॒त् तस्मा᳚न् मेद्ध्य॒त्वाय॑ मेद्ध्य॒त्वाय॒ तस्मा॑ दा॒मा ऽऽमा तस्मा᳚न् मेद्ध्य॒त्वाय॑ मेद्ध्य॒त्वाय॒ तस्मा॑ दा॒मा । \newline
20. मे॒द्ध्य॒त्वायेति॑ मेद्ध्य - त्वाय॑ । \newline
21. तस्मा॑ दा॒मा ऽऽमा तस्मा॒त् तस्मा॑ दा॒मा प॒क्वम् प॒क्व मा॒मा तस्मा॒त् तस्मा॑ दा॒मा प॒क्वम् । \newline
22. आ॒मा प॒क्वम् प॒क्व मा॒मा ऽऽमा प॒क्वम् दु॑हे दुहे प॒क्व मा॒मा ऽऽमा प॒क्वम् दु॑हे । \newline
23. प॒क्वम् दु॑हे दुहे प॒क्वम् प॒क्वम् दु॑हे प॒शवः॑ प॒शवो॑ दुहे प॒क्वम् प॒क्वम् दु॑हे प॒शवः॑ । \newline
24. दु॒हे॒ प॒शवः॑ प॒शवो॑ दुहे दुहे प॒शवो॒ वै वै प॒शवो॑ दुहे दुहे प॒शवो॒ वै । \newline
25. प॒शवो॒ वै वै प॒शवः॑ प॒शवो॒ वा ए॒त ए॒ते वै प॒शवः॑ प॒शवो॒ वा ए॒ते । \newline
26. वा ए॒त ए॒ते वै वा ए॒ते यद् यदे॒ते वै वा ए॒ते यत् । \newline
27. ए॒ते यद् यदे॒त ए॒ते यदा॑दि॒त्य आ॑दि॒त्यो यदे॒त ए॒ते यदा॑दि॒त्यः । \newline
28. यदा॑दि॒त्य आ॑दि॒त्यो यद् यदा॑दि॒त्यः प॑रि॒श्रित्य॑ परि॒श्रि त्या॑दि॒त्यो यद् यदा॑दि॒त्यः प॑रि॒श्रित्य॑ । \newline
29. आ॒दि॒त्यः प॑रि॒श्रित्य॑ परि॒श्रि त्या॑दि॒त्य आ॑दि॒त्यः प॑रि॒श्रित्य॑ गृह्णाति गृह्णाति परि॒श्रि त्या॑दि॒त्य आ॑दि॒त्यः प॑रि॒श्रित्य॑ गृह्णाति । \newline
30. प॒रि॒श्रित्य॑ गृह्णाति गृह्णाति परि॒श्रित्य॑ परि॒श्रित्य॑ गृह्णाति प्रति॒रुद्ध्य॑ प्रति॒रुद्ध्य॑ गृह्णाति परि॒श्रित्य॑ परि॒श्रित्य॑ गृह्णाति प्रति॒रुद्ध्य॑ । \newline
31. प॒रि॒श्रित्येति॑ परि - श्रित्य॑ । \newline
32. गृ॒ह्णा॒ति॒ प्र॒ति॒रुद्ध्य॑ प्रति॒रुद्ध्य॑ गृह्णाति गृह्णाति प्रति॒रुद्ध्यै॒ वैव प्र॑ति॒रुद्ध्य॑ गृह्णाति गृह्णाति प्रति॒रुद्ध्यै॒व । \newline
33. प्र॒ति॒रुद्ध्यै॒ वैव प्र॑ति॒रुद्ध्य॑ प्रति॒रुद्ध्यै॒ वास्मा॑ अस्मा ए॒व प्र॑ति॒रुद्ध्य॑ प्रति॒रुद्ध्यै॒ वास्मै᳚ । \newline
34. प्र॒ति॒रुद्ध्येति॑ प्रति - रुद्ध्य॑ । \newline
35. ए॒वास्मा॑ अस्मा ए॒वै वास्मै॑ प॒शून् प॒शून॑स्मा ए॒वै वास्मै॑ प॒शून् । \newline
36. अ॒स्मै॒ प॒शून् प॒शून॑स्मा अस्मै प॒शून् गृ॑ह्णाति गृह्णाति प॒शून॑स्मा अस्मै प॒शून् गृ॑ह्णाति । \newline
37. प॒शून् गृ॑ह्णाति गृह्णाति प॒शून् प॒शून् गृ॑ह्णाति प॒शवः॑ प॒शवो॑ गृह्णाति प॒शून् प॒शून् गृ॑ह्णाति प॒शवः॑ । \newline
38. गृ॒ह्णा॒ति॒ प॒शवः॑ प॒शवो॑ गृह्णाति गृह्णाति प॒शवो॒ वै वै प॒शवो॑ गृह्णाति गृह्णाति प॒शवो॒ वै । \newline
39. प॒शवो॒ वै वै प॒शवः॑ प॒शवो॒ वा ए॒त ए॒ते वै प॒शवः॑ प॒शवो॒ वा ए॒ते । \newline
40. वा ए॒त ए॒ते वै वा ए॒ते यद् यदे॒ते वै वा ए॒ते यत् । \newline
41. ए॒ते यद् यदे॒त ए॒ते यदा॑दि॒त्य आ॑दि॒त्यो यदे॒त ए॒ते यदा॑दि॒त्यः । \newline
42. यदा॑दि॒त्य आ॑दि॒त्यो यद् यदा॑दि॒त्य ए॒ष ए॒ष आ॑दि॒त्यो यद् यदा॑दि॒त्य ए॒षः । \newline
43. आ॒दि॒त्य ए॒ष ए॒ष आ॑दि॒त्य आ॑दि॒त्य ए॒ष रु॒द्रो रु॒द्र ए॒ष आ॑दि॒त्य आ॑दि॒त्य ए॒ष रु॒द्रः । \newline
44. ए॒ष रु॒द्रो रु॒द्र ए॒ष ए॒ष रु॒द्रो यद् यद् रु॒द्र ए॒ष ए॒ष रु॒द्रो यत् । \newline
45. रु॒द्रो यद् यद् रु॒द्रो रु॒द्रो यद॒ग्नि र॒ग्निर् यद् रु॒द्रो रु॒द्रो यद॒ग्निः । \newline
46. यद॒ग्नि र॒ग्निर् यद् यद॒ग्निः प॑रि॒श्रित्य॑ परि॒श्रि त्या॒ग्निर् यद् यद॒ग्निः प॑रि॒श्रित्य॑ । \newline
47. अ॒ग्निः प॑रि॒श्रित्य॑ परि॒श्रि त्या॒ग्नि र॒ग्निः प॑रि॒श्रित्य॑ गृह्णाति गृह्णाति परि॒श्रि त्या॒ग्नि र॒ग्निः प॑रि॒श्रित्य॑ गृह्णाति । \newline
48. प॒रि॒श्रित्य॑ गृह्णाति गृह्णाति परि॒श्रित्य॑ परि॒श्रित्य॑ गृह्णाति रु॒द्राद् रु॒द्राद् गृ॑ह्णाति परि॒श्रित्य॑ परि॒श्रित्य॑ गृह्णाति रु॒द्रात् । \newline
49. प॒रि॒श्रित्येति॑ परि - श्रित्य॑ । \newline
50. गृ॒ह्णा॒ति॒ रु॒द्राद् रु॒द्राद् गृ॑ह्णाति गृह्णाति रु॒द्रा दे॒वैव रु॒द्राद् गृ॑ह्णाति गृह्णाति रु॒द्रा दे॒व । \newline
51. रु॒द्रा दे॒वैव रु॒द्राद् रु॒द्रा दे॒व प॒शून् प॒शूने॒व रु॒द्राद् रु॒द्रा दे॒व प॒शून् । \newline
52. ए॒व प॒शून् प॒शूने॒वैव प॒शून॒न्त र॒न्तः प॒शूने॒वैव प॒शून॒न्तः । \newline
53. प॒शून॒न्त र॒न्तः प॒शून् प॒शून॒न्तर् द॑धाति दधा त्य॒न्तः प॒शून् प॒शून॒न्तर् द॑धाति । \newline
54. अ॒न्तर् द॑धाति दधा त्य॒न्त र॒न्तर् द॑धा त्ये॒ष ए॒ष द॑धा त्य॒न्त र॒न्तर् द॑धा त्ये॒षः । \newline
55. द॒धा॒ त्ये॒ष ए॒ष द॑धाति दधा त्ये॒ष वै वा ए॒ष द॑धाति दधा त्ये॒ष वै । \newline
\pagebreak
\markright{ TS 6.5.6.5  \hfill https://www.vedavms.in \hfill}

\section{ TS 6.5.6.5 }

\textbf{TS 6.5.6.5 } \newline
\textbf{Samhita Paata} \newline

-त्ये॒ष वै विव॑स्वानादि॒त्यो यदु॑पाꣳ शु॒सव॑नः॒ स ए॒तमे॒व सो॑मपी॒थं परि॑ शय॒ आ तृ॑तीयसव॒नाद्-विव॑स्व आदित्यै॒ष ते॑ सोमपी॒थ इत्या॑ह॒ विव॑स्वन्तमे॒वाऽऽ*दि॒त्यꣳ सो॑मपी॒थेन॒ सम॑र्द्धयति॒ या दि॒व्या वृष्टि॒स्तया᳚ त्वा श्रीणा॒मीति॒ वृष्टि॑कामस्य श्रीणीया॒द्-वृष्टि॑मे॒वाव॑ रुन्धे॒ यदि॑ ता॒जक् प्र॒स्कन्दे॒द्-वर्.षु॑कः प॒र्जन्यः॑ स्या॒द्यदि॑ चि॒रमव॑र्.षुको॒ न ( ) सा॑दय॒त्यस॑न्ना॒द्धि प्र॒जाः प्र॒जाय॑न्ते॒ नानु॒ वष॑ट् करोति॒ यद॑नुवषट्कु॒र्याद् -रु॒द्रं प्र॒जा अ॒न्वव॑सृजे॒न्न हु॒त्वान्वी᳚क्षेत॒ यद॒न्वीक्षे॑त॒ चक्षु॑रस्य प्र॒मायु॑कꣳ स्या॒त् तस्मा॒न्नान्वीक्ष्यः॑ ॥ \newline

\textbf{Pada Paata} \newline

ए॒षः । वै । विव॑स्वान् । आ॒दि॒त्यः । यत् । उ॒पाꣳ॒॒शु॒सव॑न॒ इत्यु॑पाꣳशु - सव॑नः । सः । ए॒तम् । ए॒व । सो॒म॒पी॒थमिति॑ सोम - पी॒थम् । परीति॑ । श॒ये॒ । एति॑ । तृ॒ती॒य॒स॒व॒नादिति॑ तृतीय-स॒व॒नात् । विव॑स्वः । आ॒दि॒त्य॒ । ए॒षः । ते॒ । सो॒म॒पी॒थ इति॑ सोम - पी॒थः । इति॑ । आ॒ह॒ । विव॑स्वन्तम् । ए॒व । आ॒दि॒त्यम् । सो॒म॒पी॒थेनेति॑ सोम - पी॒थेन॑ । समिति॑ । अ॒द्‌र्ध॒य॒ति॒ । या । दि॒व्या । वृष्टिः॑ । तया᳚ । त्वा॒ । श्री॒णा॒मि॒ । इति॑ । वृष्टि॑काम॒स्येति॒ वृष्टि॑ - का॒म॒स्य॒ । श्री॒णी॒या॒त् । वृष्टि᳚म् । ए॒व । अवेति॑ । रु॒न्धे॒ । यदि॑ । ता॒जक् । प्र॒स्कन्दे॒दिति॑ प्र - स्कन्दे᳚त् । वर्.षु॑कः । प॒र्जन्यः॑ । स्या॒त् । यदि॑ । चि॒रम् । अव॑र्.षुकः । न ( ) । सा॒द॒य॒ति॒ । अस॑न्नात् । हि । प्र॒जा इति॑ प्र - जाः । प्र॒जाय॑न्त॒ इति॑ प्र - जाय॑न्ते । न । अन्विति॑ । वष॑ट् । क॒रो॒ति॒ । यत् । अ॒नु॒व॒ष॒ट्कु॒र्यादित्य॑नु - व॒ष॒ट्कु॒र्यात् । रु॒द्रम् । प्र॒जा इति॑ प्र - जाः । अ॒न्वव॑सृजे॒दित्य॑नु - अव॑सृजेत् । न । हु॒त्वा । अन्विति॑ । ई॒क्षे॒त॒ । यत् । अ॒न्वीक्षे॒तेत्य॑नु - ईक्षे॑त । चक्षुः॑ । अ॒स्य॒ । प्र॒मायु॑क॒मिति॑ प्र - मायु॑कम् । स्या॒त् । तस्मा᳚त् । न । अ॒न्वीक्ष्य॒ इत्य॑नु - ईक्ष्यः॑ ॥  \newline


\textbf{Krama Paata} \newline

ए॒ष वै । वै विव॑स्वान् । विव॑स्वानादि॒त्यः । आ॒दि॒त्यो यत् । यदु॑पाꣳशु॒सव॑नः । उ॒पाꣳ॒॒शु॒सव॑नः॒ सः । उ॒पाꣳ॒॒शु॒सव॑न॒ इत्यु॑पाꣳशु - सव॑नः । स ए॒तम् । ए॒तमे॒व । ए॒व सो॑मपी॒थम् । सो॒म॒पी॒थम् परि॑ । सो॒म॒पी॒थमिति॑ सोम - पी॒थम् । परि॑ शये । श॒य॒ आ । आ तृ॑तीयसव॒नात् । तृ॒ती॒य॒स॒व॒नाद् विव॑स्वः । तृ॒ती॒य॒स॒व॒नादिति॑ तृतीय - स॒व॒नात् । विव॑स्व आदित्य । आ॒दि॒त्यै॒षः । ए॒ष ते᳚ । ते॒ सो॒म॒पी॒थः । सो॒म॒पी॒थ इति॑ । सो॒म॒पी॒थ इति॑ सोम - पी॒थः । इत्या॑ह । आ॒ह॒ विव॑स्वन्तम् । विव॑स्वन्तमे॒व । ए॒वादि॒त्यम् । आ॒दि॒त्यꣳ सो॑मपी॒थेन॑ । सो॒म॒पी॒थेन॒ सम् । सो॒म॒पी॒थेनेति॑ सोम - पी॒थेन॑ । सम॑र्द्धयति । अ॒र्द्ध॒य॒ति॒ या । या दि॒व्या । दि॒व्या वृष्टिः॑ । वृष्टि॒स्तया᳚ । तया᳚ त्वा । त्वा॒ श्री॒णा॒मि॒ । श्री॒णा॒मीति॑ । इति॒ वृष्टि॑कामस्य । वृष्टि॑कामस्य श्रीणीयात् । वृष्टि॑काम॒स्येति॒ वृष्टि॑ - का॒म॒स्य॒ । श्री॒णी॒या॒द् वृष्टि᳚म् । वृष्टि॑मे॒व । ए॒वाव॑ । अव॑ रुन्धे । रु॒न्धे॒ यदि॑ । यदि॑ ता॒जक् । ता॒जक् प्र॒स्कन्दे᳚त् । प्र॒स्कन्दे॒द् वर्.षु॑कः । प्र॒स्कन्दे॒दिति॑ प्र - स्कन्दे᳚त् । वर्.षु॑कः प॒र्जन्यः॑ । प॒र्जन्यः॑ स्यात् । स्या॒द् यदि॑ । यदि॑ चि॒रम् । चि॒रमव॑र्.षुकः । अव॑र्.षुको॒ न ( ) । न सा॑दयति । सा॒द॒य॒त्यस॑न्नात् । अस॑न्ना॒द्‌धि । हि प्र॒जाः । प्र॒जाः प्र॒जाय॑न्ते । प्र॒जा इति॑ प्र - जाः । प्र॒जाय॑न्ते॒ न । प्र॒जाय॑न्त॒ इति॑ प्र - जाय॑न्ते । नानु॑ । अनु॒ वष॑ट् । वष॑ट् करोति । क॒रो॒ति॒ यत् । यद॑नुवषट्कु॒र्यात् । अ॒नु॒व॒ष॒ट्॒कु॒र्याद् रु॒द्रम् । अ॒नु॒व॒ष॒ट्॒कु॒र्यादित्य॑नु - व॒ष॒ट्॒कु॒र्यात् । रु॒द्रम् प्र॒जाः । प्र॒जा अ॒न्वव॑सृजेत् । प्र॒जा इति॑ प्र - जाः । अ॒न्वव॑सृजे॒न् न । अ॒न्वव॑सृजे॒दित्य॑नु - अव॑सृजेत् । न हु॒त्वा । हु॒त्वाऽनु॑ । अन्वी᳚क्षेत । ई॒क्षे॒त॒ यत् । यद॒न्वीक्षे॑त । अ॒न्वीक्षे॑त॒ चक्षुः॑ । अ॒न्वीक्षे॒तेत्य॑नु - ईक्षे॑त । चक्षु॑रस्य । अ॒स्य॒ प्र॒मायु॑कम् । प्र॒मायु॑कꣳ स्यात् । प्र॒मायु॑क॒मिति॑ प्र - मायु॑कम् । स्या॒त् तस्मा᳚त् । तस्मा॒न् न । नान्वीक्ष्यः॑ । अ॒न्वीक्ष्य॒ इत्य॑नु - ईक्ष्यः॑ । \newline

\textbf{Jatai Paata} \newline

1. ए॒ष वै वा ए॒ष ए॒ष वै । \newline
2. वै विव॑स्वा॒न्॒. विव॑स्वा॒न्॒. वै वै विव॑स्वान् । \newline
3. विव॑स्वा नादि॒त्य आ॑दि॒त्यो विव॑स्वा॒न्॒. विव॑स्वा नादि॒त्यः । \newline
4. आ॒दि॒त्यो यद् यदा॑दि॒त्य आ॑दि॒त्यो यत् । \newline
5. यदु॑पाꣳशु॒सव॑न उपाꣳशु॒सव॑नो॒ यद् यदु॑पाꣳशु॒सव॑नः । \newline
6. उ॒पाꣳ॒॒शु॒सव॑नः॒ स स उ॑पाꣳशु॒सव॑न उपाꣳशु॒सव॑नः॒ सः । \newline
7. उ॒पाꣳ॒॒शु॒सव॑न॒ इत्यु॑पाꣳशु - सव॑नः । \newline
8. स ए॒त मे॒तꣳ स स ए॒तम् । \newline
9. ए॒त मे॒वै वैत मे॒त मे॒व । \newline
10. ए॒व सो॑मपी॒थꣳ सो॑मपी॒थ मे॒वैव सो॑मपी॒थम् । \newline
11. सो॒म॒पी॒थम् परि॒ परि॑ सोमपी॒थꣳ सो॑मपी॒थम् परि॑ । \newline
12. सो॒म॒पी॒थमिति॑ सोम - पी॒थम् । \newline
13. परि॑ शये शये॒ परि॒ परि॑ शये । \newline
14. श॒य॒ आ श॑ये शय॒ आ । \newline
15. आ तृ॑तीयसव॒नात् तृ॑तीयसव॒नादा तृ॑तीयसव॒नात् । \newline
16. तृ॒ती॒य॒स॒व॒नाद् विव॑स्वो॒ विव॑स्व स्तृतीयसव॒नात् तृ॑तीयसव॒नाद् विव॑स्वः । \newline
17. तृ॒ती॒य॒स॒व॒नादिति॑ तृतीय - स॒व॒नात् । \newline
18. विव॑स्व आदित्या दित्य॒ विव॑स्वो॒ विव॑स्व आदित्य । \newline
19. आ॒दि॒ त्यै॒ष ए॒ष आ॑दित्या दित्यै॒षः । \newline
20. ए॒ष ते॑ त ए॒ष ए॒ष ते᳚ । \newline
21. ते॒ सो॒म॒पी॒थः सो॑मपी॒थ स्ते॑ ते सोमपी॒थः । \newline
22. सो॒म॒पी॒थ इतीति॑ सोमपी॒थः सो॑मपी॒थ इति॑ । \newline
23. सो॒म॒पी॒थ इति॑ सोम - पी॒थः । \newline
24. इत्या॑हा॒हे तीत्या॑ह । \newline
25. आ॒ह॒ विव॑स्वन्तं॒ ॅविव॑स्वन्त माहाह॒ विव॑स्वन्तम् । \newline
26. विव॑स्वन्त मे॒वैव विव॑स्वन्तं॒ ॅविव॑स्वन्त मे॒व । \newline
27. ए॒वादि॒त्य मा॑दि॒त्य मे॒वै वादि॒त्यम् । \newline
28. आ॒दि॒त्यꣳ सो॑मपी॒थेन॑ सोमपी॒थे ना॑दि॒त्य मा॑दि॒त्यꣳ सो॑मपी॒थेन॑ । \newline
29. सो॒म॒पी॒थेन॒ सꣳ सꣳ सो॑मपी॒थेन॑ सोमपी॒थेन॒ सम् । \newline
30. सो॒म॒पी॒थेनेति॑ सोम - पी॒थेन॑ । \newline
31. स म॑र्द्धय त्यर्द्धयति॒ सꣳ स म॑र्द्धयति । \newline
32. अ॒र्द्ध॒य॒ति॒ या या ऽर्द्ध॑य त्यर्द्धयति॒ या । \newline
33. या दि॒व्या दि॒व्या या या दि॒व्या । \newline
34. दि॒व्या वृष्टि॒र् वृष्टि॑र् दि॒व्या दि॒व्या वृष्टिः॑ । \newline
35. वृष्टि॒ स्तया॒ तया॒ वृष्टि॒र् वृष्टि॒ स्तया᳚ । \newline
36. तया᳚ त्वा त्वा॒ तया॒ तया᳚ त्वा । \newline
37. त्वा॒ श्री॒णा॒मि॒ श्री॒णा॒मि॒ त्वा॒ त्वा॒ श्री॒णा॒मि॒ । \newline
38. श्री॒णा॒ मीतीति॑ श्रीणामि श्रीणा॒ मीति॑ । \newline
39. इति॒ वृष्टि॑कामस्य॒ वृष्टि॑काम॒ स्येतीति॒ वृष्टि॑कामस्य । \newline
40. वृष्टि॑कामस्य श्रीणीयाच् छ्रीणीया॒द् वृष्टि॑कामस्य॒ वृष्टि॑कामस्य श्रीणीयात् । \newline
41. वृष्टि॑काम॒स्येति॒ वृष्टि॑ - का॒म॒स्य॒ । \newline
42. श्री॒णी॒या॒द् वृष्टिं॒ ॅवृष्टिꣳ॑ श्रीणीयाच् छ्रीणीया॒द् वृष्टि᳚म् । \newline
43. वृष्टि॑ मे॒वैव वृष्टिं॒ ॅवृष्टि॑ मे॒व । \newline
44. ए॒वावा वै॒वै वाव॑ । \newline
45. अव॑ रुन्धे रु॒न्धे ऽवाव॑ रुन्धे । \newline
46. रु॒न्धे॒ यदि॒ यदि॑ रुन्धे रुन्धे॒ यदि॑ । \newline
47. यदि॑ ता॒जक् ता॒जग् यदि॒ यदि॑ ता॒जक् । \newline
48. ता॒जक् प्र॒स्कन्दे᳚त् प्र॒स्कन्दे᳚त् ता॒जक् ता॒जक् प्र॒स्कन्दे᳚त् । \newline
49. प्र॒स्कन्दे॒द् वर्.षु॑को॒ वर्.षु॑कः प्र॒स्कन्दे᳚त् प्र॒स्कन्दे॒द् वर्.षु॑कः । \newline
50. प्र॒स्कन्दे॒दिति॑ प्र - स्कन्दे᳚त् । \newline
51. वर्.षु॑कः प॒र्जन्यः॑ प॒र्जन्यो॒ वर्.षु॑को॒ वर्.षु॑कः प॒र्जन्यः॑ । \newline
52. प॒र्जन्यः॑ स्याथ् स्यात् प॒र्जन्यः॑ प॒र्जन्यः॑ स्यात् । \newline
53. स्या॒द् यदि॒ यदि॑ स्याथ् स्या॒द् यदि॑ । \newline
54. यदि॑ चि॒रम् चि॒रं ॅयदि॒ यदि॑ चि॒रम् । \newline
55. चि॒र मव॑र्.षु॒को ऽव॑र्.षुक श्चि॒रम् चि॒र मव॑र्.षुकः । \newline
56. अव॑र्.षुको॒ न नाव॑र्.षु॒को ऽव॑र्.षुको॒ न । \newline
57. न सा॑दयति सादयति॒ न न सा॑दयति । \newline
58. सा॒द॒य॒ त्यस॑न्ना॒ दस॑न्नाथ् सादयति सादय॒ त्यस॑न्नात् । \newline
59. अस॑न्ना॒द्धि ह्यस॑न्ना॒ दस॑न्ना॒द्धि । \newline
60. हि प्र॒जाः प्र॒जा हि हि प्र॒जाः । \newline
61. प्र॒जाः प्र॒जाय॑न्ते प्र॒जाय॑न्ते प्र॒जाः प्र॒जाः प्र॒जाय॑न्ते । \newline
62. प्र॒जा इति॑ प्र - जाः । \newline
63. प्र॒जाय॑न्ते॒ न न प्र॒जाय॑न्ते प्र॒जाय॑न्ते॒ न । \newline
64. प्र॒जाय॑न्त॒ इति॑ प्र - जाय॑न्ते । \newline
65. नान् वनु॒ न नानु॑ । \newline
66. अनु॒ वष॒ड् वष॒ डन् वनु॒ वष॑ट् । \newline
67. वष॑ट् करोति करोति॒ वष॒ड् वष॑ट् करोति । \newline
68. क॒रो॒ति॒ यद् यत् क॑रोति करोति॒ यत् । \newline
69. यद॑नुवषट्कु॒र्या द॑नुवषट्कु॒र्याद् यद् यद॑नुवषट्कु॒र्यात् । \newline
70. अ॒नु॒व॒ष॒ट्कु॒र्याद् रु॒द्रꣳ रु॒द्र म॑नुवषट्कु॒र्या द॑नुवषट्कु॒र्याद् रु॒द्रम् । \newline
71. अ॒नु॒व॒ष॒ट्कु॒र्यादित्य॑नु - व॒ष॒ट्कु॒र्यात् । \newline
72. रु॒द्रम् प्र॒जाः प्र॒जा रु॒द्रꣳ रु॒द्रम् प्र॒जाः । \newline
73. प्र॒जा अ॒न्वव॑सृजे द॒न्वव॑सृजेत् प्र॒जाः प्र॒जा अ॒न्वव॑सृजेत् । \newline
74. प्र॒जा इति॑ प्र - जाः । \newline
75. अ॒न्वव॑सृजे॒न् न नान्वव॑सृजे द॒न्वव॑सृजे॒न् न । \newline
76. अ॒न्वव॑सृजे॒दित्य॑नु - अव॑सृजेत् । \newline
77. न हु॒त्वा हु॒त्वा न न हु॒त्वा । \newline
78. हु॒त्वा ऽन्वनु॑ हु॒त्वा हु॒त्वा ऽनु॑ । \newline
79. अन्वी᳚क्षेते क्षे॒ता न्वन्वी᳚ क्षेत । \newline
80. ई॒क्षे॒त॒ यद् यदी᳚क्षेते क्षेत॒ यत् । \newline
81. यद॒न्वीक्षे॑ता॒ न्वीक्षे॑त॒ यद् यद॒न्वीक्षे॑त । \newline
82. अ॒न्वीक्षे॑त॒ चक्षु॒ श्चक्षु॑ र॒न्वीक्षे॑ता॒ न्वीक्षे॑त॒ चक्षुः॑ । \newline
83. अ॒न्वीक्षे॒तेत्य॑नु - ईक्षे॑त । \newline
84. चक्षु॑ रस्यास्य॒ चक्षु॒ श्चक्षु॑ रस्य । \newline
85. अ॒स्य॒ प्र॒मायु॑कम् प्र॒मायु॑क मस्यास्य प्र॒मायु॑कम् । \newline
86. प्र॒मायु॑कꣳ स्याथ् स्यात् प्र॒मायु॑कम् प्र॒मायु॑कꣳ स्यात् । \newline
87. प्र॒मायु॑क॒मिति॑ प्र - मायु॑कम् । \newline
88. स्या॒त् तस्मा॒त् तस्मा᳚थ् स्याथ् स्या॒त् तस्मा᳚त् । \newline
89. तस्मा॒न् न न तस्मा॒त् तस्मा॒न् न । \newline
90. नान्वीक्ष्यो॒ ऽन्वीक्ष्यो॒ न नान्वीक्ष्यः॑ । \newline
91. अ॒न्वीक्ष्य॒ इत्य॑नु - ईक्ष्यः॑ । \newline

\textbf{Ghana Paata } \newline

1. ए॒ष वै वा ए॒ष ए॒ष वै विव॑स्वा॒न्॒. विव॑स्वा॒न्॒. वा ए॒ष ए॒ष वै विव॑स्वान् । \newline
2. वै विव॑स्वा॒न्॒. विव॑स्वा॒न्॒. वै वै विव॑स्वा-नादि॒त्य आ॑दि॒त्यो विव॑स्वा॒न्॒. वै वै विव॑स्वा-नादि॒त्यः । \newline
3. विव॑स्वा-नादि॒त्य आ॑दि॒त्यो विव॑स्वा॒न्॒. विव॑स्वा-नादि॒त्यो यद् यदा॑दि॒त्यो विव॑स्वा॒न्॒. विव॑स्वा-नादि॒त्यो यत् । \newline
4. आ॒दि॒त्यो यद् यदा॑दि॒त्य आ॑दि॒त्यो यदु॑पाꣳशु॒सव॑न उपाꣳशु॒सव॑नो॒ यदा॑दि॒त्य आ॑दि॒त्यो यदु॑पाꣳशु॒सव॑नः । \newline
5. यदु॑पाꣳशु॒सव॑न उपाꣳशु॒सव॑नो॒ यद् यदु॑पाꣳशु॒सव॑नः॒ स स उ॑पाꣳशु॒सव॑नो॒ यद् यदु॑पाꣳशु॒सव॑नः॒ सः । \newline
6. उ॒पाꣳ॒॒शु॒सव॑नः॒ स स उ॑पाꣳशु॒सव॑न उपाꣳशु॒सव॑नः॒ स ए॒त मे॒तꣳ स उ॑पाꣳशु॒सव॑न उपाꣳशु॒सव॑नः॒ स ए॒तम् । \newline
7. उ॒पाꣳ॒॒शु॒सव॑न॒ इत्यु॑पाꣳशु - सव॑नः । \newline
8. स ए॒त मे॒तꣳ स स ए॒त मे॒वै वैतꣳ स स ए॒त मे॒व । \newline
9. ए॒त मे॒वै वैत मे॒त मे॒व सो॑मपी॒थꣳ सो॑मपी॒थ मे॒वैत मे॒त मे॒व सो॑मपी॒थम् । \newline
10. ए॒व सो॑मपी॒थꣳ सो॑मपी॒थ मे॒वैव सो॑मपी॒थम् परि॒ परि॑ सोमपी॒थ मे॒वैव सो॑मपी॒थम् परि॑ । \newline
11. सो॒म॒पी॒थम् परि॒ परि॑ सोमपी॒थꣳ सो॑मपी॒थम् परि॑ शये शये॒ परि॑ सोमपी॒थꣳ सो॑मपी॒थम् परि॑ शये । \newline
12. सो॒म॒पी॒थमिति॑ सोम - पी॒थम् । \newline
13. परि॑ शये शये॒ परि॒ परि॑ शय॒ आ श॑ये॒ परि॒ परि॑ शय॒ आ । \newline
14. श॒य॒ आ श॑ये शय॒ आ तृ॑तीयसव॒नात् तृ॑तीयसव॒नादा श॑ये शय॒ आ तृ॑तीयसव॒नात् । \newline
15. आ तृ॑तीयसव॒नात् तृ॑तीयसव॒नादा तृ॑तीयसव॒नाद् विव॑स्वो॒ विव॑स्व स्तृतीयसव॒नादा तृ॑तीयसव॒नाद् विव॑स्वः । \newline
16. तृ॒ती॒य॒स॒व॒नाद् विव॑स्वो॒ विव॑स्व स्तृतीयसव॒नात् तृ॑तीयसव॒नाद् विव॑स्व आदित्या दित्य॒ विव॑स्व स्तृतीयसव॒नात् तृ॑तीयसव॒नाद् विव॑स्व आदित्य । \newline
17. तृ॒ती॒य॒स॒व॒नादिति॑ तृतीय - स॒व॒नात् । \newline
18. विव॑स्व आदि त्यादित्य॒ विव॑स्वो॒ विव॑स्व आदित्यै॒ष ए॒ष आ॑दित्य॒ विव॑स्वो॒ विव॑स्व आदित्यै॒षः । \newline
19. आ॒दि॒त्यै॒ष ए॒ष आ॑दि त्यादि त्यै॒ष ते॑ त ए॒ष आ॑दि त्यादि त्यै॒ष ते᳚ । \newline
20. ए॒ष ते॑ त ए॒ष ए॒ष ते॑ सोमपी॒थः सो॑मपी॒थ स्त॑ ए॒ष ए॒ष ते॑ सोमपी॒थः । \newline
21. ते॒ सो॒म॒पी॒थः सो॑मपी॒थ स्ते॑ ते सोमपी॒थ इतीति॑ सोमपी॒थ स्ते॑ ते सोमपी॒थ इति॑ । \newline
22. सो॒म॒पी॒थ इतीति॑ सोमपी॒थः सो॑मपी॒थ इत्या॑हा॒हेति॑ सोमपी॒थः सो॑मपी॒थ इत्या॑ह । \newline
23. सो॒म॒पी॒थ इति॑ सोम - पी॒थः । \newline
24. इत्या॑हा॒हे तीत्या॑ह॒ विव॑स्वन्तं॒ ॅविव॑स्वन्त मा॒हे तीत्या॑ह॒ विव॑स्वन्तम् । \newline
25. आ॒ह॒ विव॑स्वन्तं॒ ॅविव॑स्वन्त माहाह॒ विव॑स्वन्त मे॒वैव विव॑स्वन्त माहाह॒ विव॑स्वन्त मे॒व । \newline
26. विव॑स्वन्त मे॒वैव विव॑स्वन्तं॒ ॅविव॑स्वन्त मे॒वादि॒त्य मा॑दि॒त्य मे॒व विव॑स्वन्तं॒ ॅविव॑स्वन्त मे॒वादि॒त्यम् । \newline
27. ए॒वादि॒त्य मा॑दि॒त्य मे॒वै वादि॒त्यꣳ सो॑मपी॒थेन॑ सोमपी॒थे ना॑दि॒त्य मे॒वै वादि॒त्यꣳ सो॑मपी॒थेन॑ । \newline
28. आ॒दि॒त्यꣳ सो॑मपी॒थेन॑ सोमपी॒थे ना॑दि॒त्य मा॑दि॒त्यꣳ सो॑मपी॒थेन॒ सꣳ सꣳ सो॑मपी॒थे ना॑दि॒त्य मा॑दि॒त्यꣳ सो॑मपी॒थेन॒ सम् । \newline
29. सो॒म॒पी॒थेन॒ सꣳ सꣳ सो॑मपी॒थेन॑ सोमपी॒थेन॒ स म॑र्द्धय त्यर्द्धयति॒ सꣳ सो॑मपी॒थेन॑ सोमपी॒थेन॒ स म॑र्द्धयति । \newline
30. सो॒म॒पी॒थेनेति॑ सोम - पी॒थेन॑ । \newline
31. स म॑र्द्धय त्यर्द्धयति॒ सꣳ स म॑र्द्धयति॒ या या ऽर्द्ध॑यति॒ सꣳ स म॑र्द्धयति॒ या । \newline
32. अ॒र्द्ध॒य॒ति॒ या या ऽर्द्ध॑य त्यर्द्धयति॒ या दि॒व्या दि॒व्या या ऽर्द्ध॑य त्यर्द्धयति॒ या दि॒व्या । \newline
33. या दि॒व्या दि॒व्या या या दि॒व्या वृष्टि॒र् वृष्टि॑र् दि॒व्या या या दि॒व्या वृष्टिः॑ । \newline
34. दि॒व्या वृष्टि॒र् वृष्टि॑र् दि॒व्या दि॒व्या वृष्टि॒ स्तया॒ तया॒ वृष्टि॑र् दि॒व्या दि॒व्या वृष्टि॒ स्तया᳚ । \newline
35. वृष्टि॒ स्तया॒ तया॒ वृष्टि॒र् वृष्टि॒ स्तया᳚ त्वा त्वा॒ तया॒ वृष्टि॒र् वृष्टि॒ स्तया᳚ त्वा । \newline
36. तया᳚ त्वा त्वा॒ तया॒ तया᳚ त्वा श्रीणामि श्रीणामि त्वा॒ तया॒ तया᳚ त्वा श्रीणामि । \newline
37. त्वा॒ श्री॒णा॒मि॒ श्री॒णा॒मि॒ त्वा॒ त्वा॒ श्री॒णा॒ मीतीति॑ श्रीणामि त्वा त्वा श्रीणा॒ मीति॑ । \newline
38. श्री॒णा॒ मीतीति॑ श्रीणामि श्रीणा॒ मीति॒ वृष्टि॑कामस्य॒ वृष्टि॑काम॒स्येति॑ श्रीणामि श्रीणा॒ मीति॒ वृष्टि॑कामस्य । \newline
39. इति॒ वृष्टि॑कामस्य॒ वृष्टि॑काम॒स्येतीति॒ वृष्टि॑कामस्य श्रीणीयाच् छ्रीणीया॒द् वृष्टि॑काम॒स्येतीति॒ वृष्टि॑कामस्य श्रीणीयात् । \newline
40. वृष्टि॑कामस्य श्रीणीयाच् छ्रीणीया॒द् वृष्टि॑कामस्य॒ वृष्टि॑कामस्य श्रीणीया॒द् वृष्टिं॒ ॅवृष्टिꣳ॑ श्रीणीया॒द् वृष्टि॑कामस्य॒ वृष्टि॑कामस्य श्रीणीया॒द् वृष्टि᳚म् । \newline
41. वृष्टि॑काम॒स्येति॒ वृष्टि॑ - का॒म॒स्य॒ । \newline
42. श्री॒णी॒या॒द् वृष्टिं॒ ॅवृष्टिꣳ॑ श्रीणीयाच् छ्रीणीया॒द् वृष्टि॑ मे॒वैव वृष्टिꣳ॑ श्रीणीयाच् छ्रीणीया॒द् वृष्टि॑ मे॒व । \newline
43. वृष्टि॑ मे॒वैव वृष्टिं॒ ॅवृष्टि॑ मे॒वा वावै॒व वृष्टिं॒ ॅवृष्टि॑ मे॒वाव॑ । \newline
44. ए॒वावा वै॒वै वाव॑ रुन्धे रु॒न्धे ऽवै॒वै वाव॑ रुन्धे । \newline
45. अव॑ रुन्धे रु॒न्धे ऽवाव॑ रुन्धे॒ यदि॒ यदि॑ रु॒न्धे ऽवाव॑ रुन्धे॒ यदि॑ । \newline
46. रु॒न्धे॒ यदि॒ यदि॑ रुन्धे रुन्धे॒ यदि॑ ता॒जक् ता॒जग् यदि॑ रुन्धे रुन्धे॒ यदि॑ ता॒जक् । \newline
47. यदि॑ ता॒जक् ता॒जग् यदि॒ यदि॑ ता॒जक् प्र॒स्कन्दे᳚त् प्र॒स्कन्दे᳚त् ता॒जग् यदि॒ यदि॑ ता॒जक् प्र॒स्कन्दे᳚त् । \newline
48. ता॒जक् प्र॒स्कन्दे᳚त् प्र॒स्कन्दे᳚त् ता॒जक् ता॒जक् प्र॒स्कन्दे॒द् वर्.षु॑को॒ वर्.षु॑कः प्र॒स्कन्दे᳚त् ता॒जक् ता॒जक् प्र॒स्कन्दे॒द् वर्.षु॑कः । \newline
49. प्र॒स्कन्दे॒द् वर्.षु॑को॒ वर्.षु॑कः प्र॒स्कन्दे᳚त् प्र॒स्कन्दे॒द् वर्.षु॑कः प॒र्जन्यः॑ प॒र्जन्यो॒ वर्.षु॑कः प्र॒स्कन्दे᳚त् प्र॒स्कन्दे॒द् वर्.षु॑कः प॒र्जन्यः॑ । \newline
50. प्र॒स्कन्दे॒दिति॑ प्र - स्कन्दे᳚त् । \newline
51. वर्.षु॑कः प॒र्जन्यः॑ प॒र्जन्यो॒ वर्.षु॑को॒ वर्.षु॑कः प॒र्जन्यः॑ स्याथ् स्यात् प॒र्जन्यो॒ वर्.षु॑को॒ वर्.षु॑कः प॒र्जन्यः॑ स्यात् । \newline
52. प॒र्जन्यः॑ स्याथ् स्यात् प॒र्जन्यः॑ प॒र्जन्यः॑ स्या॒द् यदि॒ यदि॑ स्यात् प॒र्जन्यः॑ प॒र्जन्यः॑ स्या॒द् यदि॑ । \newline
53. स्या॒द् यदि॒ यदि॑ स्याथ् स्या॒द् यदि॑ चि॒रम् चि॒रं ॅयदि॑ स्याथ् स्या॒द् यदि॑ चि॒रम् । \newline
54. यदि॑ चि॒रम् चि॒रं ॅयदि॒ यदि॑ चि॒र मव॑र्.षु॒को ऽव॑र्.षुक श्चि॒रं ॅयदि॒ यदि॑ चि॒र मव॑र्.षुकः । \newline
55. चि॒र मव॑र्.षु॒को ऽव॑र्.षुक श्चि॒रम् चि॒र मव॑र्.षुको॒ न नाव॑र्.षुक श्चि॒रम् चि॒र मव॑र्.षुको॒ न । \newline
56. अव॑र्.षुको॒ न नाव॑र्.षु॒को ऽव॑र्.षुको॒ न सा॑दयति सादयति॒ नाव॑र्.षु॒को ऽव॑र्.षुको॒ न सा॑दयति । \newline
57. न सा॑दयति सादयति॒ न न सा॑दय॒ त्यस॑न्ना॒ दस॑न्नाथ् सादयति॒ न न सा॑दय॒ त्यस॑न्नात् । \newline
58. सा॒द॒य॒ त्यस॑न्ना॒ दस॑न्नाथ् सादयति सादय॒ त्यस॑न्ना॒द्धि ह्यस॑न्नाथ् सादयति सादय॒ त्यस॑न्ना॒द्धि । \newline
59. अस॑न्ना॒द्धि ह्यस॑न्ना॒ दस॑न्ना॒द्धि प्र॒जाः प्र॒जा ह्यस॑न्ना॒ दस॑न्ना॒द्धि प्र॒जाः । \newline
60. हि प्र॒जाः प्र॒जा हि हि प्र॒जाः प्र॒जाय॑न्ते प्र॒जाय॑न्ते प्र॒जा हि हि प्र॒जाः प्र॒जाय॑न्ते । \newline
61. प्र॒जाः प्र॒जाय॑न्ते प्र॒जाय॑न्ते प्र॒जाः प्र॒जाः प्र॒जाय॑न्ते॒ न न प्र॒जाय॑न्ते प्र॒जाः प्र॒जाः प्र॒जाय॑न्ते॒ न । \newline
62. प्र॒जा इति॑ प्र - जाः । \newline
63. प्र॒जाय॑न्ते॒ न न प्र॒जाय॑न्ते प्र॒जाय॑न्ते॒ नान् वनु॒ न प्र॒जाय॑न्ते प्र॒जाय॑न्ते॒ नानु॑ । \newline
64. प्र॒जाय॑न्त॒ इति॑ प्र - जाय॑न्ते । \newline
65. नान् वनु॒ न नानु॒ वष॒ड् वष॒ डनु॒ न नानु॒ वष॑ट् । \newline
66. अनु॒ वष॒ड् वष॒ डन्वनु॒ वष॑ट् करोति करोति॒ वष॒ डन्वनु॒ वष॑ट् करोति । \newline
67. वष॑ट् करोति करोति॒ वष॒ड् वष॑ट् करोति॒ यद् यत् क॑रोति॒ वष॒ड् वष॑ट् करोति॒ यत् । \newline
68. क॒रो॒ति॒ यद् यत् क॑रोति करोति॒ यद॑नुवषट्कु॒र्या द॑नुवषट्कु॒र्याद् यत् क॑रोति करोति॒ यद॑नुवषट्कु॒र्यात् । \newline
69. यद॑नुवषट्कु॒र्या द॑नुवषट्कु॒र्याद् यद् यद॑नुवषट्कु॒र्याद् रु॒द्रꣳ रु॒द्र म॑नुवषट्कु॒र्याद् यद् यद॑नुवषट्कु॒र्याद् रु॒द्रम् । \newline
70. अ॒नु॒व॒ष॒ट्कु॒र्याद् रु॒द्रꣳ रु॒द्र म॑नुवषट्कु॒र्या द॑नुवषट्कु॒र्याद् रु॒द्रम् प्र॒जाः प्र॒जा रु॒द्र म॑नुवषट्कु॒र्या द॑नुवषट्कु॒र्याद् रु॒द्रम् प्र॒जाः । \newline
71. अ॒नु॒व॒ष॒ट्कु॒र्यादित्य॑नु - व॒ष॒ट्कु॒र्यात् । \newline
72. रु॒द्रम् प्र॒जाः प्र॒जा रु॒द्रꣳ रु॒द्रम् प्र॒जा अ॒न्वव॑सृजे द॒न्वव॑सृजेत् प्र॒जा रु॒द्रꣳ रु॒द्रम् प्र॒जा अ॒न्वव॑सृजेत् । \newline
73. प्र॒जा अ॒न्वव॑सृजे द॒न्वव॑सृजेत् प्र॒जाः प्र॒जा अ॒न्वव॑सृजे॒न् न नान्वव॑सृजेत् प्र॒जाः प्र॒जा अ॒न्वव॑सृजे॒न् न । \newline
74. प्र॒जा इति॑ प्र - जाः । \newline
75. अ॒न्वव॑सृजे॒न् न नान्वव॑सृजे द॒न्वव॑सृजे॒न् न हु॒त्वा हु॒त्वा नान्वव॑सृजे द॒न्वव॑सृजे॒न् न हु॒त्वा । \newline
76. अ॒न्वव॑सृजे॒दित्य॑नु - अव॑सृजेत् । \newline
77. न हु॒त्वा हु॒त्वा न न हु॒त्वा ऽन्वनु॑ हु॒त्वा न न हु॒त्वा ऽनु॑ । \newline
78. हु॒त्वा ऽन्वनु॑ हु॒त्वा हु॒त्वा ऽन्वी᳚क्षेते क्षे॒तानु॑ हु॒त्वा हु॒त्वा ऽन्वी᳚क्षेत । \newline
79. अन्वी᳚क्षेते क्षे॒तान् वन् वी᳚क्षेत॒ यद् यदी᳚क्षे॒तान् वन् वी᳚क्षेत॒ यत् । \newline
80. ई॒क्षे॒त॒ यद् यदी᳚क्षेते क्षेत॒ यद॒न्वीक्षे॑ता॒ न्वीक्षे॑त॒ यदी᳚क्षेते क्षेत॒ यद॒न्वीक्षे॑त । \newline
81. यद॒न्वीक्षे॑ता॒ न्वीक्षे॑त॒ यद् यद॒न्वीक्षे॑त॒ चक्षु॒ श्चक्षु॑ र॒न्वीक्षे॑त॒ यद् यद॒न्वीक्षे॑त॒ चक्षुः॑ । \newline
82. अ॒न्वीक्षे॑त॒ चक्षु॒ श्चक्षु॑ र॒न्वीक्षे॑ता॒ न्वीक्षे॑त॒ चक्षु॑ रस्यास्य॒ चक्षु॑ र॒न्वीक्षे॑ता॒ न्वीक्षे॑त॒ चक्षु॑रस्य । \newline
83. अ॒न्वीक्षे॒तेत्य॑नु - ईक्षे॑त । \newline
84. चक्षु॑ रस्यास्य॒ चक्षु॒ श्चक्षु॑रस्य प्र॒मायु॑कम् प्र॒मायु॑क मस्य॒ चक्षु॒ श्चक्षु॑रस्य प्र॒मायु॑कम् । \newline
85. अ॒स्य॒ प्र॒मायु॑कम् प्र॒मायु॑क मस्यास्य प्र॒मायु॑कꣳ स्याथ् स्यात् प्र॒मायु॑क मस्यास्य प्र॒मायु॑कꣳ स्यात् । \newline
86. प्र॒मायु॑कꣳ स्याथ् स्यात् प्र॒मायु॑कम् प्र॒मायु॑कꣳ स्या॒त् तस्मा॒त् तस्मा᳚थ् स्यात् प्र॒मायु॑कम् प्र॒मायु॑कꣳ स्या॒त् तस्मा᳚त् । \newline
87. प्र॒मायु॑क॒मिति॑ प्र - मायु॑कम् । \newline
88. स्या॒त् तस्मा॒त् तस्मा᳚थ् स्याथ् स्या॒त् तस्मा॒न् न न तस्मा᳚थ् स्याथ् स्या॒त् तस्मा॒न् न । \newline
89. तस्मा॒न् न न तस्मा॒त् तस्मा॒न् नान्वीक्ष्यो॒ ऽन्वीक्ष्यो॒ न तस्मा॒त् तस्मा॒न् नान्वीक्ष्यः॑ । \newline
90. नान्वीक्ष्यो॒ ऽन्वीक्ष्यो॒ न नान्वीक्ष्यः॑ । \newline
91. अ॒न्वीक्ष्य॒ इत्य॑नु - ईक्ष्यः॑ । \newline
\pagebreak
\markright{ TS 6.5.7.1  \hfill https://www.vedavms.in \hfill}

\section{ TS 6.5.7.1 }

\textbf{TS 6.5.7.1 } \newline
\textbf{Samhita Paata} \newline

अ॒न्त॒र्या॒म॒पा॒त्रेण॑ सावि॒त्रमा᳚ग्रय॒णाद्-गृ॑ह्णाति प्र॒जाप॑ति॒र्वा ए॒ष यदा᳚ग्रय॒णः प्र॒जानां᳚ प्र॒जन॑नाय॒ न सा॑दय॒त्यस॑न्ना॒द्धि प्र॒जाः प्र॒जाय॑न्ते॒ नानु॒ वष॑ट् करोति॒ यद॑नुवषट्कु॒र्याद्-रु॒द्रं प्र॒जा अ॒न्वव॑सृजेदे॒ष वै गा॑य॒त्रो दे॒वानां॒ ॅयथ् स॑वि॒तैष गा॑यत्रि॒यै लो॒के गृ॑ह्यते॒ यदा᳚ग्रय॒णो यद॑न्तर्यामपा॒त्रेण॑ सावि॒त्रमा᳚ग्रय॒णाद्-गृ॒ह्णाति॒ स्वादे॒वैनं॒ ॅयोने॒र्निर्गृ॑ह्णाति॒ विश्वे॑- [  ] \newline

\textbf{Pada Paata} \newline

अ॒न्त॒र्या॒म॒पा॒त्रेणेत्य॑न्तर्याम - पा॒त्रेण॑ । सा॒वि॒त्रम् । आ॒ग्र॒य॒णात् । गृ॒ह्णा॒ति॒ । प्र॒जाप॑ति॒रिति॑ प्र॒जा-प॒तिः॒ । वै । ए॒षः । यत् । आ॒ग्र॒य॒णः । प्र॒जाना॒मिति॑ प्र - जाना᳚म् । प्र॒जन॑ना॒येति॑ प्र - जन॑नाय । न । सा॒द॒य॒ति॒ । अस॑न्नात् । हि । प्र॒जा इति॑ प्र - जाः । प्र॒जाय॑न्त॒ इति॑ प्र - जाय॑न्ते । न । अन्विति॑ । वष॑ट् । क॒रो॒ति॒ । यत् । अ॒नु॒व॒ष॒ट्कु॒र्यादित्य॑नु - व॒ष॒ट्कु॒र्यात् । रु॒द्रम् । प्र॒जा इति॑ प्र - जाः । अ॒न्वव॑सृजे॒दित्य॑नु - अव॑सृजेत् । ए॒षः । वै । गा॒य॒त्रः । दे॒वाना᳚म् । यत् । स॒वि॒ता । ए॒षः । गा॒य॒त्रि॒यै । लो॒के । गृ॒ह्य॒ते॒ । यत् । आ॒ग्र॒य॒णः । यत् । अ॒न्त॒र्या॒म॒पा॒त्रेणेत्य॑न्तर्याम - पा॒त्रेण॑ । सा॒वि॒त्रम् । आ॒ग्र॒य॒णात् । गृ॒ह्णाति॑ । स्वात् । ए॒व । ए॒न॒म् । योनेः᳚ । निरिति॑ । गृ॒ह्णा॒ति॒ । विश्वे᳚ ।  \newline


\textbf{Krama Paata} \newline

अ॒न्त॒र्या॒म॒पा॒त्रेण॑ सावि॒त्रम् । अ॒न्त॒र्या॒म॒पा॒त्रेणेत्य॑न्तर्याम - पा॒त्रेण॑ । सा॒वि॒त्रमा᳚ग्रय॒णात् । आ॒ग्र॒य॒णाद् गृ॑ह्णाति । गृ॒ह्णा॒ति॒ प्र॒जाप॑तिः । प्र॒जाप॑ति॒र् वै । प्र॒जाप॑ति॒रिति॑ प्र॒जा - प॒तिः॒ । वा ए॒षः । ए॒ष यत् । यदा᳚ग्रय॒णः । आ॒ग्र॒य॒णः प्र॒जाना᳚म् । प्र॒जाना᳚म् प्र॒जन॑नाय । प्र॒जाना॒मिति॑ प्र - जाना᳚म् । प्र॒जन॑नाय॒ न । प्र॒जन॑ना॒येति॑ प्र - जन॑नाय । न सा॑दयति । सा॒द॒य॒त्यस॑न्नात् । अस॑न्ना॒द्‌धि । हि प्र॒जाः । प्र॒जाः प्र॒जाय॑न्ते । प्र॒जा इति॑ प्र - जाः । प्र॒जाय॑न्ते॒ न । प्र॒जाय॑न्त॒ इति॑ प्र - जाय॑न्ते । नानु॑ । अनु॒ वष॑ट् । वष॑ट् करोति । क॒रो॒ति॒ यत् । यद॑नुवषट्कु॒र्यात् । अ॒नु॒व॒ष॒ट्॒कु॒र्याद् रु॒द्रम् । अ॒नु॒व॒ष॒ट्॒कु॒र्यादित्य॑नु - व॒ष॒ट्कु॒र्यात् । रु॒द्रम् प्र॒जाः । प्र॒जा अ॒न्वव॑सृजेत् । प्र॒जा इति॑ प्र - जाः । अ॒न्वव॑सृजेदे॒षः । अ॒न्वव॑सृजे॒दित्य॑नु - अव॑सृजेत् । ए॒ष वै । वै गा॑य॒त्रः । गा॒य॒त्रो दे॒वाना᳚म् । दे॒वाना॒म् ॅयत् । यथ् स॑वि॒ता । स॒वि॒तैषः । ए॒ष गा॑यत्रि॒यै । गा॒य॒त्रि॒यै लो॒के । लो॒के गृ॑ह्यते । गृ॒ह्य॒ते॒ यत् । यदा᳚ग्रय॒णः । आ॒ग्र॒य॒णो यत् । यद॑न्तर्यामपा॒त्रेण॑ । अ॒न्त॒र्या॒म॒पा॒त्रेण॑ सावि॒त्रम् । अ॒न्त॒र्या॒म॒पा॒त्रेणेत्य॑न्तर्याम - पा॒त्रेण॑ । सा॒वि॒त्रमा᳚ग्रय॒णात् । आ॒ग्र॒य॒णाद् गृ॒ह्णाति॑ । गृ॒ह्णाति॒ स्वात् । स्वादे॒व । ए॒वैन᳚म् । ए॒न॒म् ॅयोनेः᳚ । योने॒र् निः । निर् गृ॑ह्णाति । गृ॒ह्णा॒ति॒ विश्वे᳚ । विश्वे॑ दे॒वाः \newline

\textbf{Jatai Paata} \newline

1. अ॒न्त॒र्या॒म॒पा॒त्रेण॑ सावि॒त्रꣳ सा॑वि॒त्र म॑न्तर्यामपा॒त्रेणा᳚ न्तर्यामपा॒त्रेण॑ सावि॒त्रम् । \newline
2. अ॒न्त॒र्या॒म॒पा॒त्रेणेत्य॑न्तर्याम - पा॒त्रेण॑ । \newline
3. सा॒वि॒त्र मा᳚ग्रय॒णा दा᳚ग्रय॒णाथ् सा॑वि॒त्रꣳ सा॑वि॒त्र मा᳚ग्रय॒णात् । \newline
4. आ॒ग्र॒य॒णाद् गृ॑ह्णाति गृह्णा त्याग्रय॒णा दा᳚ग्रय॒णाद् गृ॑ह्णाति । \newline
5. गृ॒ह्णा॒ति॒ प्र॒जाप॑तिः प्र॒जाप॑तिर् गृह्णाति गृह्णाति प्र॒जाप॑तिः । \newline
6. प्र॒जाप॑ति॒र् वै वै प्र॒जाप॑तिः प्र॒जाप॑ति॒र् वै । \newline
7. प्र॒जाप॑ति॒रिति॑ प्र॒जा - प॒तिः॒ । \newline
8. वा ए॒ष ए॒ष वै वा ए॒षः । \newline
9. ए॒ष यद् यदे॒ष ए॒ष यत् । \newline
10. यदा᳚ग्रय॒ण आ᳚ग्रय॒णो यद् यदा᳚ग्रय॒णः । \newline
11. आ॒ग्र॒य॒णः प्र॒जाना᳚म् प्र॒जाना॑ माग्रय॒ण आ᳚ग्रय॒णः प्र॒जाना᳚म् । \newline
12. प्र॒जाना᳚म् प्र॒जन॑नाय प्र॒जन॑नाय प्र॒जाना᳚म् प्र॒जाना᳚म् प्र॒जन॑नाय । \newline
13. प्र॒जाना॒मिति॑ प्र - जाना᳚म् । \newline
14. प्र॒जन॑नाय॒ न न प्र॒जन॑नाय प्र॒जन॑नाय॒ न । \newline
15. प्र॒जन॑ना॒येति॑ प्र - जन॑नाय । \newline
16. न सा॑दयति सादयति॒ न न सा॑दयति । \newline
17. सा॒द॒य॒ त्यस॑न्ना॒ दस॑न्नाथ् सादयति सादय॒ त्यस॑न्नात् । \newline
18. अस॑न्ना॒द्धि ह्यस॑न्ना॒ दस॑न्ना॒द्धि । \newline
19. हि प्र॒जाः प्र॒जा हि हि प्र॒जाः । \newline
20. प्र॒जाः प्र॒जाय॑न्ते प्र॒जाय॑न्ते प्र॒जाः प्र॒जाः प्र॒जाय॑न्ते । \newline
21. प्र॒जा इति॑ प्र - जाः । \newline
22. प्र॒जाय॑न्ते॒ न न प्र॒जाय॑न्ते प्र॒जाय॑न्ते॒ न । \newline
23. प्र॒जाय॑न्त॒ इति॑ प्र - जाय॑न्ते । \newline
24. नान् वनु॒ न नानु॑ । \newline
25. अनु॒ वष॒ड् वष॒ डन् वनु॒ वष॑ट् । \newline
26. वष॑ट् करोति करोति॒ वष॒ड् वष॑ट् करोति । \newline
27. क॒रो॒ति॒ यद् यत् क॑रोति करोति॒ यत् । \newline
28. यद॑नुवषट्कु॒र्या द॑नुवषट्कु॒र्याद् यद् यद॑नुवषट्कु॒र्यात् । \newline
29. अ॒नु॒व॒ष॒ट्कु॒र्याद् रु॒द्रꣳ रु॒द्र म॑नुवषट्कु॒र्या द॑नुवषट्कु॒र्याद् रु॒द्रम् । \newline
30. अ॒नु॒व॒ष॒ट्कु॒र्यादित्य॑नु - व॒ष॒ट्कु॒र्यात् । \newline
31. रु॒द्रम् प्र॒जाः प्र॒जा रु॒द्रꣳ रु॒द्रम् प्र॒जाः । \newline
32. प्र॒जा अ॒न्वव॑सृजे द॒न्वव॑सृजेत् प्र॒जाः प्र॒जा अ॒न्वव॑सृजेत् । \newline
33. प्र॒जा इति॑ प्र - जाः । \newline
34. अ॒न्वव॑सृजे दे॒ष ए॒षो᳚ ऽन्वव॑सृजे द॒न्वव॑सृजे दे॒षः । \newline
35. अ॒न्वव॑सृजे॒दित्य॑नु - अव॑सृजेत् । \newline
36. ए॒ष वै वा ए॒ष ए॒ष वै । \newline
37. वै गा॑य॒त्रो गा॑य॒त्रो वै वै गा॑य॒त्रः । \newline
38. गा॒य॒त्रो दे॒वाना᳚म् दे॒वाना᳚म् गाय॒त्रो गा॑य॒त्रो दे॒वाना᳚म् । \newline
39. दे॒वानां॒ ॅयद् यद् दे॒वाना᳚म् दे॒वानां॒ ॅयत् । \newline
40. यथ् स॑वि॒ता स॑वि॒ता यद् यथ् स॑वि॒ता । \newline
41. स॒वि॒ तैष ए॒ष स॑वि॒ता स॑वि॒ तैषः । \newline
42. ए॒ष गा॑यत्रि॒यै गा॑यत्रि॒या ए॒ष ए॒ष गा॑यत्रि॒यै । \newline
43. गा॒य॒त्रि॒यै लो॒के लो॒के गा॑यत्रि॒यै गा॑यत्रि॒यै लो॒के । \newline
44. लो॒के गृ॑ह्यते गृह्यते लो॒के लो॒के गृ॑ह्यते । \newline
45. गृ॒ह्य॒ते॒ यद् यद् गृ॑ह्यते गृह्यते॒ यत् । \newline
46. यदा᳚ग्रय॒ण आ᳚ग्रय॒णो यद् यदा᳚ग्रय॒णः । \newline
47. आ॒ग्र॒य॒णो यद् यदा᳚ग्रय॒ण आ᳚ग्रय॒णो यत् । \newline
48. यद॑न्तर्यामपा॒त्रेणा᳚ न्तर्यामपा॒त्रेण॒ यद् यद॑न्तर्यामपा॒त्रेण॑ । \newline
49. अ॒न्त॒र्या॒म॒पा॒त्रेण॑ सावि॒त्रꣳ सा॑वि॒त्र म॑न्तर्यामपा॒त्रेणा᳚ न्तर्यामपा॒त्रेण॑ सावि॒त्रम् । \newline
50. अ॒न्त॒र्या॒म॒पा॒त्रेणेत्य॑न्तर्याम - पा॒त्रेण॑ । \newline
51. सा॒वि॒त्र मा᳚ग्रय॒णा दा᳚ग्रय॒णाथ् सा॑वि॒त्रꣳ सा॑वि॒त्र मा᳚ग्रय॒णात् । \newline
52. आ॒ग्र॒य॒णाद् गृ॒ह्णाति॑ गृ॒ह्णा त्या᳚ग्रय॒णा दा᳚ग्रय॒णाद् गृ॒ह्णाति॑ । \newline
53. गृ॒ह्णाति॒ स्वाथ् स्वाद् गृ॒ह्णाति॑ गृ॒ह्णाति॒ स्वात् । \newline
54. स्वादे॒ वैव स्वाथ् स्वादे॒व । \newline
55. ए॒वैन॑ मेन मे॒वै वैन᳚म् । \newline
56. ए॒नं॒ ॅयोने॒र् योने॑ रेन मेनं॒ ॅयोनेः᳚ । \newline
57. योने॒र् निर् णिर् योने॒र् योने॒र् निः । \newline
58. निर् गृ॑ह्णाति गृह्णाति॒ निर् णिर् गृ॑ह्णाति । \newline
59. गृ॒ह्णा॒ति॒ विश्वे॒ विश्वे॑ गृह्णाति गृह्णाति॒ विश्वे᳚ । \newline
60. विश्वे॑ दे॒वा दे॒वा विश्वे॒ विश्वे॑ दे॒वाः । \newline

\textbf{Ghana Paata } \newline

1. अ॒न्त॒र्या॒म॒पा॒त्रेण॑ सावि॒त्रꣳ सा॑वि॒त्र म॑न्तर्यामपा॒त्रेणा᳚ न्तर्यामपा॒त्रेण॑ सावि॒त्र मा᳚ग्रय॒णा दा᳚ग्रय॒णाथ् सा॑वि॒त्र म॑न्तर्यामपा॒त्रेणा᳚ न्तर्यामपा॒त्रेण॑ सावि॒त्र मा᳚ग्रय॒णात् । \newline
2. अ॒न्त॒र्या॒म॒पा॒त्रेणेत्य॑न्तर्याम - पा॒त्रेण॑ । \newline
3. सा॒वि॒त्र मा᳚ग्रय॒णा दा᳚ग्रय॒णाथ् सा॑वि॒त्रꣳ सा॑वि॒त्र मा᳚ग्रय॒णाद् गृ॑ह्णाति गृह्णा त्याग्रय॒णाथ् सा॑वि॒त्रꣳ सा॑वि॒त्र मा᳚ग्रय॒णाद् गृ॑ह्णाति । \newline
4. आ॒ग्र॒य॒णाद् गृ॑ह्णाति गृह्णा त्याग्रय॒णा दा᳚ग्रय॒णाद् गृ॑ह्णाति प्र॒जाप॑तिः प्र॒जाप॑तिर् गृह्णा त्याग्रय॒णा दा᳚ग्रय॒णाद् गृ॑ह्णाति प्र॒जाप॑तिः । \newline
5. गृ॒ह्णा॒ति॒ प्र॒जाप॑तिः प्र॒जाप॑तिर् गृह्णाति गृह्णाति प्र॒जाप॑ति॒र् वै वै प्र॒जाप॑तिर् गृह्णाति गृह्णाति प्र॒जाप॑ति॒र् वै । \newline
6. प्र॒जाप॑ति॒र् वै वै प्र॒जाप॑तिः प्र॒जाप॑ति॒र् वा ए॒ष ए॒ष वै प्र॒जाप॑तिः प्र॒जाप॑ति॒र् वा ए॒षः । \newline
7. प्र॒जाप॑ति॒रिति॑ प्र॒जा - प॒तिः॒ । \newline
8. वा ए॒ष ए॒ष वै वा ए॒ष यद् यदे॒ष वै वा ए॒ष यत् । \newline
9. ए॒ष यद् यदे॒ष ए॒ष यदा᳚ग्रय॒ण आ᳚ग्रय॒णो यदे॒ष ए॒ष यदा᳚ग्रय॒णः । \newline
10. यदा᳚ग्रय॒ण आ᳚ग्रय॒णो यद् यदा᳚ग्रय॒णः प्र॒जाना᳚म् प्र॒जाना॑ माग्रय॒णो यद् यदा᳚ग्रय॒णः प्र॒जाना᳚म् । \newline
11. आ॒ग्र॒य॒णः प्र॒जाना᳚म् प्र॒जाना॑ माग्रय॒ण आ᳚ग्रय॒णः प्र॒जाना᳚म् प्र॒जन॑नाय प्र॒जन॑नाय प्र॒जाना॑ माग्रय॒ण आ᳚ग्रय॒णः प्र॒जाना᳚म् प्र॒जन॑नाय । \newline
12. प्र॒जाना᳚म् प्र॒जन॑नाय प्र॒जन॑नाय प्र॒जाना᳚म् प्र॒जाना᳚म् प्र॒जन॑नाय॒ न न प्र॒जन॑नाय प्र॒जाना᳚म् प्र॒जाना᳚म् प्र॒जन॑नाय॒ न । \newline
13. प्र॒जाना॒मिति॑ प्र - जाना᳚म् । \newline
14. प्र॒जन॑नाय॒ न न प्र॒जन॑नाय प्र॒जन॑नाय॒ न सा॑दयति सादयति॒ न प्र॒जन॑नाय प्र॒जन॑नाय॒ न सा॑दयति । \newline
15. प्र॒जन॑ना॒येति॑ प्र - जन॑नाय । \newline
16. न सा॑दयति सादयति॒ न न सा॑दय॒ त्यस॑न्ना॒ दस॑न्नाथ् सादयति॒ न न सा॑दय॒ त्यस॑न्नात् । \newline
17. सा॒द॒य॒ त्यस॑न्ना॒ दस॑न्नाथ् सादयति सादय॒ त्यस॑न्ना॒द्धि ह्यस॑न्नाथ् सादयति सादय॒ त्यस॑न्ना॒द्धि । \newline
18. अस॑न्ना॒द्धि ह्यस॑न्ना॒ दस॑न्ना॒द्धि प्र॒जाः प्र॒जा ह्यस॑न्ना॒ दस॑न्ना॒द्धि प्र॒जाः । \newline
19. हि प्र॒जाः प्र॒जा हि हि प्र॒जाः प्र॒जाय॑न्ते प्र॒जाय॑न्ते प्र॒जा हि हि प्र॒जाः प्र॒जाय॑न्ते । \newline
20. प्र॒जाः प्र॒जाय॑न्ते प्र॒जाय॑न्ते प्र॒जाः प्र॒जाः प्र॒जाय॑न्ते॒ न न प्र॒जाय॑न्ते प्र॒जाः प्र॒जाः प्र॒जाय॑न्ते॒ न । \newline
21. प्र॒जा इति॑ प्र - जाः । \newline
22. प्र॒जाय॑न्ते॒ न न प्र॒जाय॑न्ते प्र॒जाय॑न्ते॒ नान्वनु॒ न प्र॒जाय॑न्ते प्र॒जाय॑न्ते॒ नानु॑ । \newline
23. प्र॒जाय॑न्त॒ इति॑ प्र - जाय॑न्ते । \newline
24. नान्वनु॒ न नानु॒ वष॒ड् वष॒ डनु॒ न नानु॒ वष॑ट् । \newline
25. अनु॒ वष॒ड् वष॒ डन्वनु॒ वष॑ट् करोति करोति॒ वष॒ डन्वनु॒ वष॑ट् करोति । \newline
26. वष॑ट् करोति करोति॒ वष॒ड् वष॑ट् करोति॒ यद् यत् क॑रोति॒ वष॒ड् वष॑ट् करोति॒ यत् । \newline
27. क॒रो॒ति॒ यद् यत् क॑रोति करोति॒ यद॑नुवषट्कु॒र्या द॑नुवषट्कु॒र्याद् यत् क॑रोति करोति॒ यद॑नुवषट्कु॒र्यात् । \newline
28. यद॑नुवषट्कु॒र्या द॑नुवषट्कु॒र्याद् यद् यद॑नुवषट्कु॒र्याद् रु॒द्रꣳ रु॒द्र म॑नुवषट्कु॒र्याद् यद् यद॑नुवषट्कु॒र्याद् रु॒द्रम् । \newline
29. अ॒नु॒व॒ष॒ट्कु॒र्याद् रु॒द्रꣳ रु॒द्र म॑नुवषट्कु॒र्या द॑नुवषट्कु॒र्याद् रु॒द्रम् प्र॒जाः प्र॒जा रु॒द्र म॑नुवषट्कु॒र्या द॑नुवषट्कु॒र्याद् रु॒द्रम् प्र॒जाः । \newline
30. अ॒नु॒व॒ष॒ट्कु॒र्यादित्य॑नु - व॒ष॒ट्कु॒र्यात् । \newline
31. रु॒द्रम् प्र॒जाः प्र॒जा रु॒द्रꣳ रु॒द्रम् प्र॒जा अ॒न्वव॑सृजे द॒न्वव॑सृजेत् प्र॒जा रु॒द्रꣳ रु॒द्रम् प्र॒जा अ॒न्वव॑सृजेत् । \newline
32. प्र॒जा अ॒न्वव॑सृजे द॒न्वव॑सृजेत् प्र॒जाः प्र॒जा अ॒न्वव॑सृजे दे॒ष ए॒षो᳚ ऽन्वव॑सृजेत् प्र॒जाः प्र॒जा अ॒न्वव॑सृजे दे॒षः । \newline
33. प्र॒जा इति॑ प्र - जाः । \newline
34. अ॒न्वव॑सृजे दे॒ष ए॒षो᳚ ऽन्वव॑सृजे द॒न्वव॑सृजे दे॒ष वै वा ए॒षो᳚ ऽन्वव॑सृजे द॒न्वव॑सृजे दे॒ष वै । \newline
35. अ॒न्वव॑सृजे॒दित्य॑नु - अव॑सृजेत् । \newline
36. ए॒ष वै वा ए॒ष ए॒ष वै गा॑य॒त्रो गा॑य॒त्रो वा ए॒ष ए॒ष वै गा॑य॒त्रः । \newline
37. वै गा॑य॒त्रो गा॑य॒त्रो वै वै गा॑य॒त्रो दे॒वाना᳚म् दे॒वाना᳚म् गाय॒त्रो वै वै गा॑य॒त्रो दे॒वाना᳚म् । \newline
38. गा॒य॒त्रो दे॒वाना᳚म् दे॒वाना᳚म् गाय॒त्रो गा॑य॒त्रो दे॒वानां॒ ॅयद् यद् दे॒वाना᳚म् गाय॒त्रो गा॑य॒त्रो दे॒वानां॒ ॅयत् । \newline
39. दे॒वानां॒ ॅयद् यद् दे॒वाना᳚म् दे॒वानां॒ ॅयथ् स॑वि॒ता स॑वि॒ता यद् दे॒वाना᳚म् दे॒वानां॒ ॅयथ् स॑वि॒ता । \newline
40. यथ् स॑वि॒ता स॑वि॒ता यद् यथ् स॑वि॒तैष ए॒ष स॑वि॒ता यद् यथ् स॑वि॒तैषः । \newline
41. स॒वि॒तैष ए॒ष स॑वि॒ता स॑वि॒तैष गा॑यत्रि॒यै गा॑यत्रि॒या ए॒ष स॑वि॒ता स॑वि॒तैष गा॑यत्रि॒यै । \newline
42. ए॒ष गा॑यत्रि॒यै गा॑यत्रि॒या ए॒ष ए॒ष गा॑यत्रि॒यै लो॒के लो॒के गा॑यत्रि॒या ए॒ष ए॒ष गा॑यत्रि॒यै लो॒के । \newline
43. गा॒य॒त्रि॒यै लो॒के लो॒के गा॑यत्रि॒यै गा॑यत्रि॒यै लो॒के गृ॑ह्यते गृह्यते लो॒के गा॑यत्रि॒यै गा॑यत्रि॒यै लो॒के गृ॑ह्यते । \newline
44. लो॒के गृ॑ह्यते गृह्यते लो॒के लो॒के गृ॑ह्यते॒ यद् यद् गृ॑ह्यते लो॒के लो॒के गृ॑ह्यते॒ यत् । \newline
45. गृ॒ह्य॒ते॒ यद् यद् गृ॑ह्यते गृह्यते॒ यदा᳚ग्रय॒ण आ᳚ग्रय॒णो यद् गृ॑ह्यते गृह्यते॒ यदा᳚ग्रय॒णः । \newline
46. यदा᳚ग्रय॒ण आ᳚ग्रय॒णो यद् यदा᳚ग्रय॒णो यद् यदा᳚ग्रय॒णो यद् यदा᳚ग्रय॒णो यत् । \newline
47. आ॒ग्र॒य॒णो यद् यदा᳚ग्रय॒ण आ᳚ग्रय॒णो यद॑न्तर्यामपा॒त्रेणा᳚ न्तर्यामपा॒त्रेण॒ यदा᳚ग्रय॒ण आ᳚ग्रय॒णो यद॑न्तर्यामपा॒त्रेण॑ । \newline
48. यद॑न्तर्यामपा॒त्रेणा᳚ न्तर्यामपा॒त्रेण॒ यद् यद॑न्तर्यामपा॒त्रेण॑ सावि॒त्रꣳ सा॑वि॒त्र म॑न्तर्यामपा॒त्रेण॒ यद् यद॑न्तर्यामपा॒त्रेण॑ सावि॒त्रम् । \newline
49. अ॒न्त॒र्या॒म॒पा॒त्रेण॑ सावि॒त्रꣳ सा॑वि॒त्र म॑न्तर्यामपा॒त्रेणा᳚ न्तर्यामपा॒त्रेण॑ सावि॒त्र मा᳚ग्रय॒णा दा᳚ग्रय॒णाथ् सा॑वि॒त्र म॑न्तर्यामपा॒त्रेणा᳚ न्तर्यामपा॒त्रेण॑ सावि॒त्र मा᳚ग्रय॒णात् । \newline
50. अ॒न्त॒र्या॒म॒पा॒त्रेणेत्य॑न्तर्याम - पा॒त्रेण॑ । \newline
51. सा॒वि॒त्र मा᳚ग्रय॒णा दा᳚ग्रय॒णाथ् सा॑वि॒त्रꣳ सा॑वि॒त्र मा᳚ग्रय॒णाद् गृ॒ह्णाति॑ गृ॒ह्णा त्या᳚ग्रय॒णाथ् सा॑वि॒त्रꣳ सा॑वि॒त्र मा᳚ग्रय॒णाद् गृ॒ह्णाति॑ । \newline
52. आ॒ग्र॒य॒णाद् गृ॒ह्णाति॑ गृ॒ह्णा त्या᳚ग्रय॒णा दा᳚ग्रय॒णाद् गृ॒ह्णाति॒ स्वाथ् स्वाद् गृ॒ह्णा त्या᳚ग्रय॒णा
दा᳚ग्रय॒णाद् गृ॒ह्णाति॒ स्वात् । \newline
53. गृ॒ह्णाति॒ स्वाथ् स्वाद् गृ॒ह्णाति॑ गृ॒ह्णाति॒ स्वादे॒वैव स्वाद् गृ॒ह्णाति॑ गृ॒ह्णाति॒ स्वादे॒व । \newline
54. स्वा दे॒वैव स्वाथ् स्वा दे॒वैन॑ मेन मे॒व स्वाथ् स्वा दे॒वैन᳚म् । \newline
55. ए॒वैन॑ मेन मे॒वै वैनं॒ ॅयोने॒र् योने॑ रेन मे॒वै वैनं॒ ॅयोनेः᳚ । \newline
56. ए॒नं॒ ॅयोने॒र् योने॑ रेन मेनं॒ ॅयोने॒र् निर् णिर् योने॑ रेन मेनं॒ ॅयोने॒र् निः । \newline
57. योने॒र् निर् णिर् योने॒र् योने॒र् निर् गृ॑ह्णाति गृह्णाति॒ निर् योने॒र् योने॒र् निर् गृ॑ह्णाति । \newline
58. निर् गृ॑ह्णाति गृह्णाति॒ निर् णिर् गृ॑ह्णाति॒ विश्वे॒ विश्वे॑ गृह्णाति॒ निर् णिर् गृ॑ह्णाति॒ विश्वे᳚ । \newline
59. गृ॒ह्णा॒ति॒ विश्वे॒ विश्वे॑ गृह्णाति गृह्णाति॒ विश्वे॑ दे॒वा दे॒वा विश्वे॑ गृह्णाति गृह्णाति॒ विश्वे॑ दे॒वाः । \newline
60. विश्वे॑ दे॒वा दे॒वा विश्वे॒ विश्वे॑ दे॒वा स्तृ॒तीय॑म् तृ॒तीय॑म् दे॒वा विश्वे॒ विश्वे॑ दे॒वा स्तृ॒तीय᳚म् । \newline
\pagebreak
\markright{ TS 6.5.7.2  \hfill https://www.vedavms.in \hfill}

\section{ TS 6.5.7.2 }

\textbf{TS 6.5.7.2 } \newline
\textbf{Samhita Paata} \newline

दे॒वास्तृ॒तीयꣳ॒॒ सव॑नं॒ नोद॑यच्छ॒न् ते स॑वि॒तारं॑ प्रातस्सव॒नभा॑गꣳ॒॒ सन्तं॑ तृतीयसव॒नम॒भि पर्य॑णय॒न् ततो॒ वै ते तृ॒तीयꣳ॒॒ सव॑न॒मुद॑यच्छ॒न्॒. यत् तृ॑तीयसव॒ने सा॑वि॒त्रो गृ॒ह्यते॑ तृ॒तीय॑स्य॒ सव॑न॒स्योद्य॑त्यै सवितृपा॒त्रेण॑ वैश्वदे॒वं क॒लशा᳚द्-गृह्णाति वैश्वदे॒व्यो॑ वै प्र॒जा वै᳚श्वदे॒वः क॒लशः॑ सवि॒ता प्र॑स॒वाना॑मीशे॒ यथ् स॑वितृपा॒त्रेण॑ वैश्वदे॒वं क॒लशा᳚द्-गृ॒ह्णाति॑ सवि॒तृप्र॑सूत ए॒वास्मै᳚ प्र॒जाः प्र- [  ] \newline

\textbf{Pada Paata} \newline

दे॒वाः । तृ॒तीय᳚म् । सव॑नम् । न । उदिति॑ । अ॒य॒च्छ॒न् । ते । स॒वि॒तार᳚म् । प्रा॒त॒स्स॒व॒नभा॑ग॒मिति॑ प्रातस्सव॒न - भा॒ग॒म् । सन्त᳚म् । तृ॒ती॒य॒स॒व॒नमिति॑ तृतीय-स॒व॒नम् । अ॒भि । परीति॑ । अ॒न॒य॒न्न् । ततः॑ । वै । ते । तृ॒तीय᳚म् । सव॑नम् । उदिति॑ । अ॒य॒च्छ॒न्न् । यत् । तृ॒ती॒य॒स॒व॒न इति॑ तृतीय - स॒व॒ने । सा॒वि॒त्रः । गृ॒ह्यते᳚ । तृ॒तीय॑स्य । सव॑नस्य । उद्य॑त्या॒ इत्युत् - य॒त्यै॒ । स॒वि॒तृ॒पा॒त्रेणेति॑ सवितृ-पा॒त्रेण॑ । वै॒श्व॒दे॒वमिति॑ वैश्व - दे॒वम् । क॒लशा᳚त् । गृ॒ह्णा॒ति॒ । वै॒श्व॒दे॒व्य॑ इति॑ वैश्व - दे॒व्यः॑ । वै । प्र॒जा इति॑ प्र-जाः । वै॒श्व॒दे॒व इति॑ वैश्व-दे॒वः । क॒लशः॑ । स॒वि॒ता । प्र॒स॒वाना॒मिति॑ प्र - स॒वाना᳚म् । ई॒शे॒ । यत् । स॒वि॒तृ॒पा॒त्रेणेति॑ सवितृ - पा॒त्रेण॑ । वै॒श्व॒दे॒वमिति॑ वैश्व - दे॒वम् । क॒लशा᳚त् । गृ॒ह्णाति॑ । स॒वि॒तृप्र॑सूत॒ इति॑ सवि॒तृ - प्र॒सू॒तः॒ । ए॒व । अ॒स्मै॒ । प्र॒जा इति॑ प्र - जाः । प्रेति॑ ।  \newline


\textbf{Krama Paata} \newline

दे॒वास्तृ॒तीय᳚म् । तृ॒तीयꣳ॒॒ सव॑नम् । सव॑न॒म् न । नोत् । उद॑यच्छन्न् । अ॒य॒च्छ॒न् ते । ते स॑वि॒तार᳚म् । स॒वि॒तार॑म् प्रातस्सव॒नभा॑गम् । प्रा॒त॒स्स॒व॒नभा॑गꣳ॒॒ सन्त᳚म् । प्रा॒त॒स्स॒व॒नभा॑ग॒मिति॑ प्रातस्सव॒न - भा॒ग॒म् । सन्त॑म् तृतीयसव॒नम् । तृ॒ती॒य॒स॒व॒नम॒भि । तृ॒ती॒य॒स॒व॒नमिति॑ तृतीय - स॒व॒नम् । अ॒भि परि॑ । पर्य॑णयन्न् । अ॒न॒य॒न् ततः॑ । ततो॒ वै । वै ते । ते तृ॒तीय᳚म् । तृ॒तीयꣳ॒॒ सव॑नम् । सव॑न॒मुत् । उद॑यच्छन्न् । अ॒य॒च्छ॒न्॒. यत् । यत् तृ॑तीयसव॒ने । तृ॒ती॒य॒स॒व॒ने सा॑वि॒त्रः । तृ॒ती॒य॒स॒व॒न इति॑ तृतीय - स॒व॒ने । सा॒वि॒त्रो गृ॒ह्यते᳚ । गृ॒ह्यते॑ तृ॒तीय॑स्य । तृ॒तीय॑स्य॒ सव॑नस्य । सव॑न॒स्योद्य॑त्यै । उद्य॑त्यै सवितृपा॒त्रेण॑ । उद्य॑त्या॒ इत्युत् - य॒त्यै॒ । स॒वि॒तृ॒पा॒त्रेण॑ वैश्वदे॒वम् । स॒वि॒तृ॒पा॒त्रेणेति॑ सवितृ - पा॒त्रेण॑ । वै॒श्व॒दे॒वम् क॒लशा᳚त् । वै॒श्व॒दे॒वमिति॑ वैश्व - दे॒वम् । क॒लशा᳚द् गृह्णाति । गृ॒ह्णा॒ति॒ वै॒श्व॒दे॒व्यः॑ । वै॒श्व॒दे॒व्यो॑ वै । वै॒श्व॒दे॒व्य॑ इति॑ वैश्व - दे॒व्यः॑ । वै प्र॒जाः । प्र॒जा वै᳚श्वदे॒वः । प्र॒जा इति॑ प्र - जाः । वै॒श्व॒दे॒वः क॒लशः॑ । वै॒श्व॒दे॒व इति॑ वैश्व - दे॒वः । क॒लशः॑ सवि॒ता । स॒वि॒ता प्र॑स॒वाना᳚म् । प्र॒स॒वाना॑मीशे । प्र॒स॒वाना॒मिति॑ प्र - स॒वाना᳚म् । ई॒शे॒ यत् । यथ् स॑वितृपा॒त्रेण॑ । स॒वि॒तृ॒पा॒त्रेण॑ वैश्वदे॒वम् । स॒वि॒तृ॒पा॒त्रेणेति॑ सवितृ - पा॒त्रेण॑ । वै॒श्व॒दे॒वम् क॒लशा᳚त् । वै॒श्व॒दे॒वमिति॑ वैश्व - दे॒वम् । क॒लशा᳚द् गृ॒ह्णाति॑ । गृ॒ह्णाति॑ सवि॒तृप्र॑सूतः । स॒वि॒तृप्र॑सूत ए॒व । 
स॒वि॒तृप्र॑सूत॒ इति॑ सवि॒तृ - प्र॒सू॒तः॒ । ए॒वास्मै᳚ । अ॒स्मै॒ प्र॒जाः । प्र॒जाः प्र । प्र॒जा इति॑ प्र - जाः । प्र ज॑नयति \newline

\textbf{Jatai Paata} \newline

1. दे॒वा स्तृ॒तीय॑म् तृ॒तीय॑म् दे॒वा दे॒वा स्तृ॒तीय᳚म् । \newline
2. तृ॒तीयꣳ॒॒ सव॑नꣳ॒॒ सव॑नम् तृ॒तीय॑म् तृ॒तीयꣳ॒॒ सव॑नम् । \newline
3. सव॑न॒न् न न सव॑नꣳ॒॒ सव॑न॒न् न । \newline
4. नोदुन् न नोत् । \newline
5. उद॑यच्छन् नयच्छ॒न् नुदु द॑यच्छन्न् । \newline
6. अ॒य॒च्छ॒न् ते ते॑ ऽयच्छन् नयच्छ॒न् ते । \newline
7. ते स॑वि॒तारꣳ॑ सवि॒तार॒म् ते ते स॑वि॒तार᳚म् । \newline
8. स॒वि॒तार॑म् प्रातस्सव॒नभा॑गम् प्रातस्सव॒नभा॑गꣳ सवि॒तारꣳ॑ सवि॒तार॑म् प्रातस्सव॒नभा॑गम् । \newline
9. प्रा॒त॒स्स॒व॒नभा॑गꣳ॒॒ सन्तꣳ॒॒ सन्त॑म् प्रातस्सव॒नभा॑गम् प्रातस्सव॒नभा॑गꣳ॒॒ सन्त᳚म् । \newline
10. प्रा॒त॒स्स॒व॒नभा॑ग॒मिति॑ प्रातस्सव॒न - भा॒ग॒म् । \newline
11. सन्त॑म् तृतीयसव॒नम् तृ॑तीयसव॒नꣳ सन्तꣳ॒॒ सन्त॑म् तृतीयसव॒नम् । \newline
12. तृ॒ती॒य॒स॒व॒न म॒भ्य॑भि तृ॑तीयसव॒नम् तृ॑तीयसव॒न म॒भि । \newline
13. तृ॒ती॒य॒स॒व॒नमिति॑ तृतीय - स॒व॒नम् । \newline
14. अ॒भि परि॒ पर्य॒ भ्य॑भि परि॑ । \newline
15. पर्य॑णयन् ननय॒न् परि॒ पर्य॑णयन्न् । \newline
16. अ॒न॒य॒न् तत॒ स्ततो॑ ऽनयन् ननय॒न् ततः॑ । \newline
17. ततो॒ वै वै तत॒ स्ततो॒ वै । \newline
18. वै ते ते वै वै ते । \newline
19. ते तृ॒तीय॑म् तृ॒तीय॒म् ते ते तृ॒तीय᳚म् । \newline
20. तृ॒तीयꣳ॒॒ सव॑नꣳ॒॒ सव॑नम् तृ॒तीय॑म् तृ॒तीयꣳ॒॒ सव॑नम् । \newline
21. सव॑न॒ मुदुथ् सव॑नꣳ॒॒ सव॑न॒ मुत् । \newline
22. उद॑यच्छन् नयच्छ॒न् नुदु द॑यच्छन्न् । \newline
23. अ॒य॒च्छ॒न्॒. यद् यद॑यच्छन् नयच्छ॒न्॒. यत् । \newline
24. यत् तृ॑तीयसव॒ने तृ॑तीयसव॒ने यद् यत् तृ॑तीयसव॒ने । \newline
25. तृ॒ती॒य॒स॒व॒ने सा॑वि॒त्रः सा॑वि॒त्र स्तृ॑तीयसव॒ने तृ॑तीयसव॒ने सा॑वि॒त्रः । \newline
26. तृ॒ती॒य॒स॒व॒न इति॑ तृतीय - स॒व॒ने । \newline
27. सा॒वि॒त्रो गृ॒ह्यते॑ गृ॒ह्यते॑ सावि॒त्रः सा॑वि॒त्रो गृ॒ह्यते᳚ । \newline
28. गृ॒ह्यते॑ तृ॒तीय॑स्य तृ॒तीय॑स्य गृ॒ह्यते॑ गृ॒ह्यते॑ तृ॒तीय॑स्य । \newline
29. तृ॒तीय॑स्य॒ सव॑नस्य॒ सव॑नस्य तृ॒तीय॑स्य तृ॒तीय॑स्य॒ सव॑नस्य । \newline
30. सव॑न॒ स्योद्य॑त्या॒ उद्य॑त्यै॒ सव॑नस्य॒ सव॑न॒ स्योद्य॑त्यै । \newline
31. उद्य॑त्यै सवितृपा॒त्रेण॑ सवितृपा॒त्रे णोद्य॑त्या॒ उद्य॑त्यै सवितृपा॒त्रेण॑ । \newline
32. उद्य॑त्या॒ इत्युत् - य॒त्यै॒ । \newline
33. स॒वि॒तृ॒पा॒त्रेण॑ वैश्वदे॒वं ॅवै᳚श्वदे॒वꣳ स॑वितृपा॒त्रेण॑ सवितृपा॒त्रेण॑ वैश्वदे॒वम् । \newline
34. स॒वि॒तृ॒पा॒त्रेणेति॑ सवितृ - पा॒त्रेण॑ । \newline
35. वै॒श्व॒दे॒वम् क॒लशा᳚त् क॒लशा᳚द् वैश्वदे॒वं ॅवै᳚श्वदे॒वम् क॒लशा᳚त् । \newline
36. वै॒श्व॒दे॒वमिति॑ वैश्व - दे॒वम् । \newline
37. क॒लशा᳚द् गृह्णाति गृह्णाति क॒लशा᳚त् क॒लशा᳚द् गृह्णाति । \newline
38. गृ॒ह्णा॒ति॒ वै॒श्व॒दे॒व्यो॑ वैश्वदे॒व्यो॑ गृह्णाति गृह्णाति वैश्वदे॒व्यः॑ । \newline
39. वै॒श्व॒दे॒व्यो॑ वै वै वै᳚श्वदे॒व्यो॑ वैश्वदे॒व्यो॑ वै । \newline
40. वै॒श्व॒दे॒व्य॑ इति॑ वैश्व - दे॒व्यः॑ । \newline
41. वै प्र॒जाः प्र॒जा वै वै प्र॒जाः । \newline
42. प्र॒जा वै᳚श्वदे॒वो वै᳚श्वदे॒वः प्र॒जाः प्र॒जा वै᳚श्वदे॒वः । \newline
43. प्र॒जा इति॑ प्र - जाः । \newline
44. वै॒श्व॒दे॒वः क॒लशः॑ क॒लशो॑ वैश्वदे॒वो वै᳚श्वदे॒वः क॒लशः॑ । \newline
45. वै॒श्व॒दे॒व इति॑ वैश्व - दे॒वः । \newline
46. क॒लशः॑ सवि॒ता स॑वि॒ता क॒लशः॑ क॒लशः॑ सवि॒ता । \newline
47. स॒वि॒ता प्र॑स॒वाना᳚म् प्रस॒वानाꣳ॑ सवि॒ता स॑वि॒ता प्र॑स॒वाना᳚म् । \newline
48. प्र॒स॒वाना॑ मीश ईशे प्रस॒वाना᳚म् प्रस॒वाना॑ मीशे । \newline
49. प्र॒स॒वाना॒मिति॑ प्र - स॒वाना᳚म् । \newline
50. ई॒शे॒ यद् यदी॑श ईशे॒ यत् । \newline
51. यथ् स॑वितृपा॒त्रेण॑ सवितृपा॒त्रेण॒ यद् यथ् स॑वितृपा॒त्रेण॑ । \newline
52. स॒वि॒तृ॒पा॒त्रेण॑ वैश्वदे॒वं ॅवै᳚श्वदे॒वꣳ स॑वितृपा॒त्रेण॑ सवितृपा॒त्रेण॑ वैश्वदे॒वम् । \newline
53. स॒वि॒तृ॒पा॒त्रेणेति॑ सवितृ - पा॒त्रेण॑ । \newline
54. वै॒श्व॒दे॒वम् क॒लशा᳚त् क॒लशा᳚द् वैश्वदे॒वं ॅवै᳚श्वदे॒वम् क॒लशा᳚त् । \newline
55. वै॒श्व॒दे॒वमिति॑ वैश्व - दे॒वम् । \newline
56. क॒लशा᳚द् गृ॒ह्णाति॑ गृ॒ह्णाति॑ क॒लशा᳚त् क॒लशा᳚द् गृ॒ह्णाति॑ । \newline
57. गृ॒ह्णाति॑ सवि॒तृप्र॑सूतः सवि॒तृप्र॑सूतो गृ॒ह्णाति॑ गृ॒ह्णाति॑ सवि॒तृप्र॑सूतः । \newline
58. स॒वि॒तृप्र॑सूत ए॒वैव स॑वि॒तृप्र॑सूतः सवि॒तृप्र॑सूत ए॒व । \newline
59. स॒वि॒तृप्र॑सूत॒ इति॑ सवि॒तृ - प्र॒सू॒तः॒ । \newline
60. ए॒वास्मा॑ अस्मा ए॒वै वास्मै᳚ । \newline
61. अ॒स्मै॒ प्र॒जाः प्र॒जा अ॑स्मा अस्मै प्र॒जाः । \newline
62. प्र॒जाः प्र प्र प्र॒जाः प्र॒जाः प्र । \newline
63. प्र॒जा इति॑ प्र - जाः । \newline
64. प्र ज॑नयति जनयति॒ प्र प्र ज॑नयति । \newline

\textbf{Ghana Paata } \newline

1. दे॒वा स्तृ॒तीय॑म् तृ॒तीय॑म् दे॒वा दे॒वा स्तृ॒तीयꣳ॒॒ सव॑नꣳ॒॒ सव॑नम् तृ॒तीय॑म् दे॒वा दे॒वा स्तृ॒तीयꣳ॒॒ सव॑नम् । \newline
2. तृ॒तीयꣳ॒॒ सव॑नꣳ॒॒ सव॑नम् तृ॒तीय॑म् तृ॒तीयꣳ॒॒ सव॑न॒न् न न सव॑नम् तृ॒तीय॑म् तृ॒तीयꣳ॒॒ सव॑न॒न् न । \newline
3. सव॑न॒न् न न सव॑नꣳ॒॒ सव॑न॒म् नोदुन् न सव॑नꣳ॒॒ सव॑न॒म् नोत् । \newline
4. नोदुन् न नोद॑यच्छन्-नयच्छ॒न् नुन् न नोद॑यच्छन्न् । \newline
5. उद॑यच्छन्-नयच्छ॒न्-नुदु द॑यच्छ॒न् ते ते॑ ऽयच्छ॒न्-नुदु द॑यच्छ॒न् ते । \newline
6. अ॒य॒च्छ॒न् ते ते॑ ऽयच्छन्-नयच्छ॒न् ते स॑वि॒तारꣳ॑ सवि॒तार॒म् ते॑ ऽयच्छन्-नयच्छ॒न् ते स॑वि॒तार᳚म् । \newline
7. ते स॑वि॒तारꣳ॑ सवि॒तार॒म् ते ते स॑वि॒तार॑म् प्रातस्सव॒नभा॑गम् प्रातस्सव॒नभा॑गꣳ सवि॒तार॒म् ते ते स॑वि॒तार॑म् प्रातस्सव॒नभा॑गम् । \newline
8. स॒वि॒तार॑म् प्रातस्सव॒नभा॑गम् प्रातस्सव॒नभा॑गꣳ सवि॒तारꣳ॑ सवि॒तार॑म् प्रातस्सव॒नभा॑गꣳ॒॒ सन्तꣳ॒॒ सन्त॑म् प्रातस्सव॒नभा॑गꣳ सवि॒तारꣳ॑ सवि॒तार॑म् प्रातस्सव॒नभा॑गꣳ॒॒ सन्त᳚म् । \newline
9. प्रा॒त॒स्स॒व॒नभा॑गꣳ॒॒ सन्तꣳ॒॒ सन्त॑म् प्रातस्सव॒नभा॑गम् प्रातस्सव॒नभा॑गꣳ॒॒ सन्त॑म् तृतीयसव॒नम् तृ॑तीयसव॒नꣳ सन्त॑म् प्रातस्सव॒नभा॑गम् प्रातस्सव॒नभा॑गꣳ॒॒ सन्त॑म् तृतीयसव॒नम् । \newline
10. प्रा॒त॒स्स॒व॒नभा॑ग॒मिति॑ प्रातस्सव॒न - भा॒ग॒म् । \newline
11. सन्त॑म् तृतीयसव॒नम् तृ॑तीयसव॒नꣳ सन्तꣳ॒॒ सन्त॑म् तृतीयसव॒न म॒भ्य॑भि तृ॑तीयसव॒नꣳ सन्तꣳ॒॒ सन्त॑म् तृतीयसव॒न म॒भि । \newline
12. तृ॒ती॒य॒स॒व॒न म॒भ्य॑भि तृ॑तीयसव॒नम् तृ॑तीयसव॒न म॒भि परि॒ पर्य॒भि तृ॑तीयसव॒नम् तृ॑तीयसव॒न म॒भि परि॑ । \newline
13. तृ॒ती॒य॒स॒व॒नमिति॑ तृतीय - स॒व॒नम् । \newline
14. अ॒भि परि॒ पर्य॒ भ्य॑भि पर्य॑णयन्-ननय॒न् पर्य॒ भ्य॑भि पर्य॑णयन्न् । \newline
15. पर्य॑णयन्-ननय॒न् परि॒ पर्य॑णय॒न् तत॒ स्ततो॑ ऽनय॒न् परि॒ पर्य॑णय॒न् ततः॑ । \newline
16. अ॒न॒य॒न् तत॒ स्ततो॑ ऽनयन्-ननय॒न् ततो॒ वै वै ततो॑ ऽनयन्-ननय॒न् ततो॒ वै । \newline
17. ततो॒ वै वै तत॒ स्ततो॒ वै ते ते वै तत॒ स्ततो॒ वै ते । \newline
18. वै ते ते वै वै ते तृ॒तीय॑म् तृ॒तीय॒म् ते वै वै ते तृ॒तीय᳚म् । \newline
19. ते तृ॒तीय॑म् तृ॒तीय॒म् ते ते तृ॒तीयꣳ॒॒ सव॑नꣳ॒॒ सव॑नम् तृ॒तीय॒म् ते ते तृ॒तीयꣳ॒॒ सव॑नम् । \newline
20. तृ॒तीयꣳ॒॒ सव॑नꣳ॒॒ सव॑नम् तृ॒तीय॑म् तृ॒तीयꣳ॒॒ सव॑न॒ मुदुथ् सव॑नम् तृ॒तीय॑म् तृ॒तीयꣳ॒॒ सव॑न॒ मुत् । \newline
21. सव॑न॒ मुदुथ् सव॑नꣳ॒॒ सव॑न॒ मुद॑यच्छन्-नयच्छ॒न्-नुथ् सव॑नꣳ॒॒ सव॑न॒ मुद॑यच्छन्न् । \newline
22. उद॑यच्छन्-नयच्छ॒न्-नुदु द॑यच्छ॒न्॒. यद् यद॑यच्छ॒न्-नुदु द॑यच्छ॒न्॒. यत् । \newline
23. अ॒य॒च्छ॒न्॒. यद् यद॑यच्छन्-नयच्छ॒न्॒. यत् तृ॑तीयसव॒ने तृ॑तीयसव॒ने यद॑यच्छन्-नयच्छ॒न्॒. यत् तृ॑तीयसव॒ने । \newline
24. यत् तृ॑तीयसव॒ने तृ॑तीयसव॒ने यद् यत् तृ॑तीयसव॒ने सा॑वि॒त्रः सा॑वि॒त्र स्तृ॑तीयसव॒ने यद् यत् तृ॑तीयसव॒ने सा॑वि॒त्रः । \newline
25. तृ॒ती॒य॒स॒व॒ने सा॑वि॒त्रः सा॑वि॒त्र स्तृ॑तीयसव॒ने तृ॑तीयसव॒ने सा॑वि॒त्रो गृ॒ह्यते॑ गृ॒ह्यते॑ सावि॒त्र स्तृ॑तीयसव॒ने तृ॑तीयसव॒ने सा॑वि॒त्रो गृ॒ह्यते᳚ । \newline
26. तृ॒ती॒य॒स॒व॒न इति॑ तृतीय - स॒व॒ने । \newline
27. सा॒वि॒त्रो गृ॒ह्यते॑ गृ॒ह्यते॑ सावि॒त्रः सा॑वि॒त्रो गृ॒ह्यते॑ तृ॒तीय॑स्य तृ॒तीय॑स्य गृ॒ह्यते॑ सावि॒त्रः सा॑वि॒त्रो गृ॒ह्यते॑ तृ॒तीय॑स्य । \newline
28. गृ॒ह्यते॑ तृ॒तीय॑स्य तृ॒तीय॑स्य गृ॒ह्यते॑ गृ॒ह्यते॑ तृ॒तीय॑स्य॒ सव॑नस्य॒ सव॑नस्य तृ॒तीय॑स्य गृ॒ह्यते॑ गृ॒ह्यते॑ तृ॒तीय॑स्य॒ सव॑नस्य । \newline
29. तृ॒तीय॑स्य॒ सव॑नस्य॒ सव॑नस्य तृ॒तीय॑स्य तृ॒तीय॑स्य॒ सव॑न॒ स्योद्य॑त्या॒ उद्य॑त्यै॒ सव॑नस्य तृ॒तीय॑स्य तृ॒तीय॑स्य॒ सव॑न॒ स्योद्य॑त्यै । \newline
30. सव॑न॒ स्योद्य॑त्या॒ उद्य॑त्यै॒ सव॑नस्य॒ सव॑न॒ स्योद्य॑त्यै सवितृपा॒त्रेण॑ सवितृपा॒त्रे णोद्य॑त्यै॒ सव॑नस्य॒ सव॑न॒ स्योद्य॑त्यै सवितृपा॒त्रेण॑ । \newline
31. उद्य॑त्यै सवितृपा॒त्रेण॑ सवितृपा॒त्रे णोद्य॑त्या॒ उद्य॑त्यै सवितृपा॒त्रेण॑ वैश्वदे॒वं ॅवै᳚श्वदे॒वꣳ स॑वितृपा॒त्रे णोद्य॑त्या॒ उद्य॑त्यै सवितृपा॒त्रेण॑ वैश्वदे॒वम् । \newline
32. उद्य॑त्या॒ इत्युत् - य॒त्यै॒ । \newline
33. स॒वि॒तृ॒पा॒त्रेण॑ वैश्वदे॒वं ॅवै᳚श्वदे॒वꣳ स॑वितृपा॒त्रेण॑ सवितृपा॒त्रेण॑ वैश्वदे॒वम् क॒लशा᳚त् क॒लशा᳚द् वैश्वदे॒वꣳ स॑वितृपा॒त्रेण॑ सवितृपा॒त्रेण॑ वैश्वदे॒वम् क॒लशा᳚त् । \newline
34. स॒वि॒तृ॒पा॒त्रेणेति॑ सवितृ - पा॒त्रेण॑ । \newline
35. वै॒श्व॒दे॒वम् क॒लशा᳚त् क॒लशा᳚द् वैश्वदे॒वं ॅवै᳚श्वदे॒वम् क॒लशा᳚द् गृह्णाति गृह्णाति क॒लशा᳚द् वैश्वदे॒वं ॅवै᳚श्वदे॒वम् क॒लशा᳚द् गृह्णाति । \newline
36. वै॒श्व॒दे॒वमिति॑ वैश्व - दे॒वम् । \newline
37. क॒लशा᳚द् गृह्णाति गृह्णाति क॒लशा᳚त् क॒लशा᳚द् गृह्णाति वैश्वदे॒व्यो॑ वैश्वदे॒व्यो॑ गृह्णाति क॒लशा᳚त् क॒लशा᳚द् गृह्णाति वैश्वदे॒व्यः॑ । \newline
38. गृ॒ह्णा॒ति॒ वै॒श्व॒दे॒व्यो॑ वैश्वदे॒व्यो॑ गृह्णाति गृह्णाति वैश्वदे॒व्यो॑ वै वै वै᳚श्वदे॒व्यो॑ गृह्णाति गृह्णाति वैश्वदे॒व्यो॑ वै । \newline
39. वै॒श्व॒दे॒व्यो॑ वै वै वै᳚श्वदे॒व्यो॑ वैश्वदे॒व्यो॑ वै प्र॒जाः प्र॒जा वै वै᳚श्वदे॒व्यो॑ वैश्वदे॒व्यो॑ वै प्र॒जाः । \newline
40. वै॒श्व॒दे॒व्य॑ इति॑ वैश्व - दे॒व्यः॑ । \newline
41. वै प्र॒जाः प्र॒जा वै वै प्र॒जा वै᳚श्वदे॒वो वै᳚श्वदे॒वः प्र॒जा वै वै प्र॒जा वै᳚श्वदे॒वः । \newline
42. प्र॒जा वै᳚श्वदे॒वो वै᳚श्वदे॒वः प्र॒जाः प्र॒जा वै᳚श्वदे॒वः क॒लशः॑ क॒लशो॑ वैश्वदे॒वः प्र॒जाः प्र॒जा वै᳚श्वदे॒वः क॒लशः॑ । \newline
43. प्र॒जा इति॑ प्र - जाः । \newline
44. वै॒श्व॒दे॒वः क॒लशः॑ क॒लशो॑ वैश्वदे॒वो वै᳚श्वदे॒वः क॒लशः॑ सवि॒ता स॑वि॒ता क॒लशो॑ वैश्वदे॒वो वै᳚श्वदे॒वः क॒लशः॑ सवि॒ता । \newline
45. वै॒श्व॒दे॒व इति॑ वैश्व - दे॒वः । \newline
46. क॒लशः॑ सवि॒ता स॑वि॒ता क॒लशः॑ क॒लशः॑ सवि॒ता प्र॑स॒वाना᳚म् प्रस॒वानाꣳ॑ सवि॒ता क॒लशः॑ क॒लशः॑ सवि॒ता प्र॑स॒वाना᳚म् । \newline
47. स॒वि॒ता प्र॑स॒वाना᳚म् प्रस॒वानाꣳ॑ सवि॒ता स॑वि॒ता प्र॑स॒वाना॑ मीश ईशे प्रस॒वानाꣳ॑ सवि॒ता स॑वि॒ता प्र॑स॒वाना॑ मीशे । \newline
48. प्र॒स॒वाना॑ मीश ईशे प्रस॒वाना᳚म् प्रस॒वाना॑ मीशे॒ यद् यदी॑शे प्रस॒वाना᳚म् प्रस॒वाना॑ मीशे॒ यत् । \newline
49. प्र॒स॒वाना॒मिति॑ प्र - स॒वाना᳚म् । \newline
50. ई॒शे॒ यद् यदी॑श ईशे॒ यथ् स॑वितृपा॒त्रेण॑ सवितृपा॒त्रेण॒ यदी॑श ईशे॒ यथ् स॑वितृपा॒त्रेण॑ । \newline
51. यथ् स॑वितृपा॒त्रेण॑ सवितृपा॒त्रेण॒ यद् यथ् स॑वितृपा॒त्रेण॑ वैश्वदे॒वं ॅवै᳚श्वदे॒वꣳ स॑वितृपा॒त्रेण॒ यद् यथ् स॑वितृपा॒त्रेण॑ वैश्वदे॒वम् । \newline
52. स॒वि॒तृ॒पा॒त्रेण॑ वैश्वदे॒वं ॅवै᳚श्वदे॒वꣳ स॑वितृपा॒त्रेण॑ सवितृपा॒त्रेण॑ वैश्वदे॒वम् क॒लशा᳚त् क॒लशा᳚द् वैश्वदे॒वꣳ स॑वितृपा॒त्रेण॑ सवितृपा॒त्रेण॑ वैश्वदे॒वम् क॒लशा᳚त् । \newline
53. स॒वि॒तृ॒पा॒त्रेणेति॑ सवितृ - पा॒त्रेण॑ । \newline
54. वै॒श्व॒दे॒वम् क॒लशा᳚त् क॒लशा᳚द् वैश्वदे॒वं ॅवै᳚श्वदे॒वम् क॒लशा᳚द् गृ॒ह्णाति॑ गृ॒ह्णाति॑ क॒लशा᳚द् वैश्वदे॒वं ॅवै᳚श्वदे॒वम् क॒लशा᳚द् गृ॒ह्णाति॑ । \newline
55. वै॒श्व॒दे॒वमिति॑ वैश्व - दे॒वम् । \newline
56. क॒लशा᳚द् गृ॒ह्णाति॑ गृ॒ह्णाति॑ क॒लशा᳚त् क॒लशा᳚द् गृ॒ह्णाति॑ सवि॒तृप्र॑सूतः सवि॒तृप्र॑सूतो गृ॒ह्णाति॑ क॒लशा᳚त् क॒लशा᳚द् गृ॒ह्णाति॑ सवि॒तृप्र॑सूतः । \newline
57. गृ॒ह्णाति॑ सवि॒तृप्र॑सूतः सवि॒तृप्र॑सूतो गृ॒ह्णाति॑ गृ॒ह्णाति॑ सवि॒तृप्र॑सूत ए॒वैव स॑वि॒तृप्र॑सूतो गृ॒ह्णाति॑ गृ॒ह्णाति॑ सवि॒तृप्र॑सूत ए॒व । \newline
58. स॒वि॒तृप्र॑सूत ए॒वैव स॑वि॒तृप्र॑सूतः सवि॒तृप्र॑सूत ए॒वास्मा॑ अस्मा ए॒व स॑वि॒तृप्र॑सूतः सवि॒तृप्र॑सूत ए॒वास्मै᳚ । \newline
59. स॒वि॒तृप्र॑सूत॒ इति॑ सवि॒तृ - प्र॒सू॒तः॒ । \newline
60. ए॒वास्मा॑ अस्मा ए॒वै वास्मै᳚ प्र॒जाः प्र॒जा अ॑स्मा ए॒वै वास्मै᳚ प्र॒जाः । \newline
61. अ॒स्मै॒ प्र॒जाः प्र॒जा अ॑स्मा अस्मै प्र॒जाः प्र प्र प्र॒जा अ॑स्मा अस्मै प्र॒जाः प्र । \newline
62. प्र॒जाः प्र प्र प्र॒जाः प्र॒जाः प्र ज॑नयति जनयति॒ प्र प्र॒जाः प्र॒जाः प्र ज॑नयति । \newline
63. प्र॒जा इति॑ प्र - जाः । \newline
64. प्र ज॑नयति जनयति॒ प्र प्र ज॑नयति॒ सोमे॒ सोमे॑ जनयति॒ प्र प्र ज॑नयति॒ सोमे᳚ । \newline
\pagebreak
\markright{ TS 6.5.7.3  \hfill https://www.vedavms.in \hfill}

\section{ TS 6.5.7.3 }

\textbf{TS 6.5.7.3 } \newline
\textbf{Samhita Paata} \newline

ज॑नयति॒ सोमे॒ सोम॑म॒भि गृ॑ह्णाति॒ रेत॑ ए॒व तद्-द॑धाति सु॒शर्मा॑ऽसि सुप्रतिष्ठा॒न इत्या॑ह॒ सोमे॒ हि सोम॑मभिगृ॒ह्णाति॒ प्रति॑ष्ठित्या ए॒तस्मि॒न् वा अपि॒ ग्रहे॑ मनु॒ष्ये᳚भ्यो दे॒वेभ्यः॑ पि॒तृभ्यः॑ क्रियते सु॒शर्मा॑ऽसि सुप्रतिष्ठा॒न इत्या॑ह मनु॒ष्ये᳚भ्य ए॒वैतेन॑ करोति बृ॒हदित्या॑ह दे॒वेभ्य॑ ए॒वैतेन॑ करोति॒ नम॒ इत्या॑ह पि॒तृभ्य॑ ए॒वैतेन॑ करोत्ये॒ ( ) ताव॑ती॒ र्वै दे॒वता॒स्ताभ्य॑ ए॒वैनꣳ॒॒ सर्वा᳚भ्यो गृह्णात्ये॒ष ते॒ योनि॒र्विश्वे᳚भ्यस्त्वा दे॒वेभ्य॒ इत्या॑ह वैश्वदे॒वो ह्ये॑षः ॥ \newline

\textbf{Pada Paata} \newline

ज॒न॒य॒ति॒ । सोमे᳚ । सोम᳚म् । अ॒भीति॑ । गृ॒ह्णा॒ति॒ । रेतः॑ । ए॒व । तत् । द॒धा॒ति॒ । सु॒शर्मेति॑ सु - शर्मा᳚ । अ॒सि॒ । सु॒प्र॒ति॒ष्ठा॒न इति॑ सु - प्र॒ति॒ष्ठा॒नः । इति॑ । आ॒ह॒ । सोमे᳚ । हि । सोम᳚म् । अ॒भि॒गृ॒ह्णातीत्य॑भि - गृ॒ह्णाति॑ । प्रति॑ष्ठित्या॒ इति॒ प्रति॑ - स्थि॒त्यै॒ । ए॒तस्मिन्न्॑ । वै । अपीति॑ । ग्रहे᳚ । म॒नु॒ष्ये᳚भ्यः । दे॒वेभ्यः॑ । पि॒तृभ्य॒ इति॑ पि॒तृ - भ्यः॒ । क्रि॒य॒ते॒ । सु॒शर्मेति॑ सु - शर्मा᳚ । अ॒सि॒ । सु॒प्र॒ति॒ष्ठा॒न इति॑ सु - प्र॒ति॒ष्ठा॒नः । इति॑ । आ॒ह॒ । म॒नु॒ष्ये᳚भ्यः । ए॒व । ए॒तेन॑ । क॒रो॒ति॒ । बृ॒हत् । इति॑ । आ॒ह॒ । दे॒वेभ्यः॑ । ए॒व । ए॒तेन॑ । क॒रो॒ति॒ । नमः॑ । इति॑ । आ॒ह॒ । पि॒तृभ्य॒ इति॑ पि॒तृ - भ्यः॒ । ए॒व । ए॒तेन॑ । क॒रो॒ति॒ ( ) । ए॒ताव॑तीः । वै । दे॒वताः᳚ । ताभ्यः॑ । ए॒व । ए॒न॒म् । सर्वा᳚भ्यः । गृ॒ह्णा॒ति॒ । ए॒षः । ते॒ । योनिः॑ । विश्वे᳚भ्यः । त्वा॒ । दे॒वेभ्यः॑ । इति॑ । आ॒ह॒ । वै॒श्व॒दे॒व इति॑ वैश्व - दे॒वः । हि । ए॒षः ॥  \newline


\textbf{Krama Paata} \newline

ज॒न॒य॒ति॒ सोमे᳚ । सोमे॒ सोम᳚म् । सोम॑म॒भि । अ॒भि गृ॑ह्णाति । गृ॒ह्णा॒ति॒ रेतः॑ । रेत॑ ए॒व । ए॒व तत् । तद् द॑धाति । द॒धा॒ति॒ सु॒शर्मा᳚ । सु॒शर्मा॑ऽसि । सु॒शर्मेति॑ सु - शर्मा᳚ । अ॒सि॒ सु॒प्र॒ति॒ष्ठा॒नः । सु॒प्र॒ति॒ष्ठा॒न इति॑ । सु॒प्र॒ति॒ष्ठा॒न इति॑ सु - प्र॒ति॒ष्ठा॒नः । इत्या॑ह । आ॒ह॒ सोमे᳚ । सोमे॒ हि । हि सोम᳚म् । सोम॑मभिगृ॒ह्णाति॑ । अ॒भि॒गृ॒ह्णाति॒ प्रति॑ष्ठित्यै । अ॒भि॒गृ॒ह्णातीत्य॑भि - गृ॒ह्णाति॑ । प्रति॑ष्ठित्या ए॒तस्मिन्न्॑ । प्रति॑ष्ठित्या॒ इति॒ प्रति॑ - स्थि॒त्यै॒ । ए॒तस्मि॒न् वै । वा अपि॑ । अपि॒ ग्रहे᳚ । ग्रहे॑ मनु॒ष्ये᳚भ्यः । म॒नु॒ष्ये᳚भ्यो दे॒वेभ्यः॑ । दे॒वेभ्यः॑ पि॒तृभ्यः॑ । पि॒तृभ्यः॑ क्रियते । पि॒तृभ्य॒ इति॑ पि॒तृ - भ्यः॒ । क्रि॒य॒ते॒ सु॒शर्मा᳚ । सु॒शर्मा॑ऽसि । सु॒शर्मेति॑ सु - शर्मा᳚ । अ॒सि॒ सु॒प्र॒ति॒ष्ठा॒नः । सु॒प्र॒ति॒ष्ठा॒न इति॑ । सु॒प्र॒ति॒ष्ठा॒न इति॑ सु - प्र॒ति॒ष्ठा॒नः । इत्या॑ह । आ॒ह॒ म॒नु॒ष्ये᳚भ्यः । म॒नु॒ष्ये᳚भ्य ए॒व । ए॒वैतेन॑ । ए॒तेन॑ करोति । क॒रो॒ति॒ बृ॒हत् । बृ॒हदिति॑ । इत्या॑ह । आ॒ह॒ दे॒वेभ्यः॑ । दे॒वेभ्य॑ ए॒व । ए॒वैतेन॑ । ए॒तेन॑ करोति । क॒रो॒ति॒ नमः॑ । नम॒ इति॑ । इत्या॑ह । आ॒ह॒ पि॒तृभ्यः॑ । पि॒तृभ्य॑ ए॒व । पि॒तृभ्य॒ इति॑ पि॒तृ - भ्यः॒ । ए॒वैतेन॑ । ए॒तेन॑ करोति ( ) । क॒रो॒त्ये॒ताव॑तीः । ए॒ताव॑ती॒र् वै । वै दे॒वताः᳚ । दे॒वता॒स्ताभ्यः॑ । ताभ्य॑ ए॒व । ए॒वैन᳚म् । ए॒नꣳ॒॒ सर्वा᳚भ्यः । सर्वा᳚भ्यो गृह्णाति । गृ॒ह्णा॒त्ये॒षः । ए॒ष ते᳚ । ते॒ योनिः॑ । योनि॒र् विश्वे᳚भ्यः । विश्वे᳚भ्यस्त्वा । त्वा॒ दे॒वेभ्यः॑ । दे॒वेभ्य॒ इति॑ । इत्या॑ह । आ॒ह॒ वै॒श्व॒दे॒वः । वै॒श्व॒दे॒वो हि । वै॒श्व॒दे॒व इति॑ वैश्व - दे॒वः । ह्ये॑षः । ए॒ष इत्ये॒षः । \newline

\textbf{Jatai Paata} \newline

1. ज॒न॒य॒ति॒ सोमे॒ सोमे॑ जनयति जनयति॒ सोमे᳚ । \newline
2. सोमे॒ सोमꣳ॒॒ सोमꣳ॒॒ सोमे॒ सोमे॒ सोम᳚म् । \newline
3. सोम॑ म॒भ्य॑भि सोमꣳ॒॒ सोम॑ म॒भि । \newline
4. अ॒भि गृ॑ह्णाति गृह्णा त्य॒भ्य॑भि गृ॑ह्णाति । \newline
5. गृ॒ह्णा॒ति॒ रेतो॒ रेतो॑ गृह्णाति गृह्णाति॒ रेतः॑ । \newline
6. रेत॑ ए॒वैव रेतो॒ रेत॑ ए॒व । \newline
7. ए॒व तत् तदे॒ वैव तत् । \newline
8. तद् द॑धाति दधाति॒ तत् तद् द॑धाति । \newline
9. द॒धा॒ति॒ सु॒शर्मा॑ सु॒शर्मा॑ दधाति दधाति सु॒शर्मा᳚ । \newline
10. सु॒शर्मा᳚ ऽस्यसि सु॒शर्मा॑ सु॒शर्मा॑ ऽसि । \newline
11. सु॒शर्मेति॑ सु - शर्मा᳚ । \newline
12. अ॒सि॒ सु॒प्र॒ति॒ष्ठा॒नः सु॑प्रतिष्ठा॒नो᳚ ऽस्यसि सुप्रतिष्ठा॒नः । \newline
13. सु॒प्र॒ति॒ष्ठा॒न इतीति॑ सुप्रतिष्ठा॒नः सु॑प्रतिष्ठा॒न इति॑ । \newline
14. सु॒प्र॒ति॒ष्ठा॒न इति॑ सु - प्र॒ति॒ष्ठा॒नः । \newline
15. इत्या॑हा॒हे तीत्या॑ह । \newline
16. आ॒ह॒ सोमे॒ सोम॑ आहाह॒ सोमे᳚ । \newline
17. सोमे॒ हि हि सोमे॒ सोमे॒ हि । \newline
18. हि सोमꣳ॒॒ सोमꣳ॒॒ हि हि सोम᳚म् । \newline
19. सोम॑ मभिगृ॒ह्णा त्य॑भिगृ॒ह्णाति॒ सोमꣳ॒॒ सोम॑ मभिगृ॒ह्णाति॑ । \newline
20. अ॒भि॒गृ॒ह्णाति॒ प्रति॑ष्ठित्यै॒ प्रति॑ष्ठित्या अभिगृ॒ह्णा त्य॑भिगृ॒ह्णाति॒ प्रति॑ष्ठित्यै । \newline
21. अ॒भि॒गृ॒ह्णातीत्य॑भि - गृ॒ह्णाति॑ । \newline
22. प्रति॑ष्ठित्या ए॒तस्मि॑न् ने॒तस्मि॒न् प्रति॑ष्ठित्यै॒ प्रति॑ष्ठित्या ए॒तस्मिन्न्॑ । \newline
23. प्रति॑ष्ठित्या॒ इति॒ प्रति॑ - स्थि॒त्यै॒ । \newline
24. ए॒तस्मि॒न्॒. वै वा ए॒तस्मि॑न् ने॒तस्मि॒न्॒. वै । \newline
25. वा अप्यपि॒ वै वा अपि॑ । \newline
26. अपि॒ ग्रहे॒ ग्रहे ऽप्यपि॒ ग्रहे᳚ । \newline
27. ग्रहे॑ मनु॒ष्ये᳚भ्यो मनु॒ष्ये᳚भ्यो॒ ग्रहे॒ ग्रहे॑ मनु॒ष्ये᳚भ्यः । \newline
28. म॒नु॒ष्ये᳚भ्यो दे॒वेभ्यो॑ दे॒वेभ्यो॑ मनु॒ष्ये᳚भ्यो मनु॒ष्ये᳚भ्यो दे॒वेभ्यः॑ । \newline
29. दे॒वेभ्यः॑ पि॒तृभ्यः॑ पि॒तृभ्यो॑ दे॒वेभ्यो॑ दे॒वेभ्यः॑ पि॒तृभ्यः॑ । \newline
30. पि॒तृभ्यः॑ क्रियते क्रियते पि॒तृभ्यः॑ पि॒तृभ्यः॑ क्रियते । \newline
31. पि॒तृभ्य॒ इति॑ पि॒तृ - भ्यः॒ । \newline
32. क्रि॒य॒ते॒ सु॒शर्मा॑ सु॒शर्मा᳚ क्रियते क्रियते सु॒शर्मा᳚ । \newline
33. सु॒शर्मा᳚ ऽस्यसि सु॒शर्मा॑ सु॒शर्मा॑ ऽसि । \newline
34. सु॒शर्मेति॑ सु - शर्मा᳚ । \newline
35. अ॒सि॒ सु॒प्र॒ति॒ष्ठा॒नः सु॑प्रतिष्ठा॒नो᳚ ऽस्यसि सुप्रतिष्ठा॒नः । \newline
36. सु॒प्र॒ति॒ष्ठा॒न इतीति॑ सुप्रतिष्ठा॒नः सु॑प्रतिष्ठा॒न इति॑ । \newline
37. सु॒प्र॒ति॒ष्ठा॒न इति॑ सु - प्र॒ति॒ष्ठा॒नः । \newline
38. इत्या॑हा॒हे तीत्या॑ह । \newline
39. आ॒ह॒ म॒नु॒ष्ये᳚भ्यो मनु॒ष्ये᳚भ्य आहाह मनु॒ष्ये᳚भ्यः । \newline
40. म॒नु॒ष्ये᳚भ्य ए॒वैव म॑नु॒ष्ये᳚भ्यो मनु॒ष्ये᳚भ्य ए॒व । \newline
41. ए॒वैते नै॒ते नै॒वैवैतेन॑ । \newline
42. ए॒तेन॑ करोति करो त्ये॒ते नै॒तेन॑ करोति । \newline
43. क॒रो॒ति॒ बृ॒हद् बृ॒हत् क॑रोति करोति बृ॒हत् । \newline
44. बृ॒ह दितीति॑ बृ॒हद् बृ॒ह दिति॑ । \newline
45. इत्या॑हा॒हे तीत्या॑ह । \newline
46. आ॒ह॒ दे॒वेभ्यो॑ दे॒वेभ्य॑ आहाह दे॒वेभ्यः॑ । \newline
47. दे॒वेभ्य॑ ए॒वैव दे॒वेभ्यो॑ दे॒वेभ्य॑ ए॒व । \newline
48. ए॒वैते नै॒ते नै॒वैवैतेन॑ । \newline
49. ए॒तेन॑ करोति करो त्ये॒ते नै॒तेन॑ करोति । \newline
50. क॒रो॒ति॒ नमो॒ नम॑ स्करोति करोति॒ नमः॑ । \newline
51. नम॒ इतीति॒ नमो॒ नम॒ इति॑ । \newline
52. इत्या॑हा॒हे तीत्या॑ह । \newline
53. आ॒ह॒ पि॒तृभ्यः॑ पि॒तृभ्य॑ आहाह पि॒तृभ्यः॑ । \newline
54. पि॒तृभ्य॑ ए॒वैव पि॒तृभ्यः॑ पि॒तृभ्य॑ ए॒व । \newline
55. पि॒तृभ्य॒ इति॑ पि॒तृ - भ्यः॒ । \newline
56. ए॒वैते नै॒ते नै॒वैवैतेन॑ । \newline
57. ए॒तेन॑ करोति करोत्ये॒ तेनै॒तेन॑ करोति । \newline
58. क॒रो॒ त्ये॒ताव॑ती रे॒ताव॑तीः करोति करो त्ये॒ताव॑तीः । \newline
59. ए॒ताव॑ती॒र् वै वा ए॒ताव॑ती रे॒ताव॑ती॒र् वै । \newline
60. वै दे॒वता॑ दे॒वता॒ वै वै दे॒वताः᳚ । \newline
61. दे॒वता॒स् ताभ्य॒ स्ताभ्यो॑ दे॒वता॑ दे॒वता॒ स्ताभ्यः॑ । \newline
62. ताभ्य॑ ए॒वैव ताभ्य॒ स्ताभ्य॑ ए॒व । \newline
63. ए॒वैन॑ मेन मे॒वै वैन᳚म् । \newline
64. ए॒नꣳ॒॒ सर्वा᳚भ्यः॒ सर्वा᳚भ्य एन मेनꣳ॒॒ सर्वा᳚भ्यः । \newline
65. सर्वा᳚भ्यो गृह्णाति गृह्णाति॒ सर्वा᳚भ्यः॒ सर्वा᳚भ्यो गृह्णाति । \newline
66. गृ॒ह्णा॒ त्ये॒ष ए॒ष गृ॑ह्णाति गृह्णा त्ये॒षः । \newline
67. ए॒ष ते॑ त ए॒ष ए॒ष ते᳚ । \newline
68. ते॒ योनि॒र् योनि॑ स्ते ते॒ योनिः॑ । \newline
69. योनि॒र् विश्वे᳚भ्यो॒ विश्वे᳚भ्यो॒ योनि॒र् योनि॒र् विश्वे᳚भ्यः । \newline
70. विश्वे᳚भ्य स्त्वा त्वा॒ विश्वे᳚भ्यो॒ विश्वे᳚भ्य स्त्वा । \newline
71. त्वा॒ दे॒वेभ्यो॑ दे॒वेभ्य॑ स्त्वा त्वा दे॒वेभ्यः॑ । \newline
72. दे॒वेभ्य॒ इतीति॑ दे॒वेभ्यो॑ दे॒वेभ्य॒ इति॑ । \newline
73. इत्या॑हा॒हे तीत्या॑ह । \newline
74. आ॒ह॒ वै॒श्व॒दे॒वो वै᳚श्वदे॒व आ॑हाह वैश्वदे॒वः । \newline
75. वै॒श्व॒दे॒वो हि हि वै᳚श्वदे॒वो वै᳚श्वदे॒वो हि । \newline
76. वै॒श्व॒दे॒व इति॑ वैश्व - दे॒वः । \newline
77. ह्ये॑ष ए॒ष हि ह्ये॑षः । \newline
78. ए॒ष इत्ये॒षः । \newline

\textbf{Ghana Paata } \newline

1. ज॒न॒य॒ति॒ सोमे॒ सोमे॑ जनयति जनयति॒ सोमे॒ सोमꣳ॒॒ सोमꣳ॒॒ सोमे॑ जनयति जनयति॒ सोमे॒ सोम᳚म् । \newline
2. सोमे॒ सोमꣳ॒॒ सोमꣳ॒॒ सोमे॒ सोमे॒ सोम॑ म॒भ्य॑भि सोमꣳ॒॒ सोमे॒ सोमे॒ सोम॑ म॒भि । \newline
3. सोम॑ म॒भ्य॑भि सोमꣳ॒॒ सोम॑ म॒भि गृ॑ह्णाति गृह्णा त्य॒भि सोमꣳ॒॒ सोम॑ म॒भि गृ॑ह्णाति । \newline
4. अ॒भि गृ॑ह्णाति गृह्णा त्य॒भ्य॑भि गृ॑ह्णाति॒ रेतो॒ रेतो॑ गृह्णा त्य॒भ्य॑भि गृ॑ह्णाति॒ रेतः॑ । \newline
5. गृ॒ह्णा॒ति॒ रेतो॒ रेतो॑ गृह्णाति गृह्णाति॒ रेत॑ ए॒वैव रेतो॑ गृह्णाति गृह्णाति॒ रेत॑ ए॒व । \newline
6. रेत॑ ए॒वैव रेतो॒ रेत॑ ए॒व तत् तदे॒व रेतो॒ रेत॑ ए॒व तत् । \newline
7. ए॒व तत् तदे॒वैव तद् द॑धाति दधाति॒ तदे॒वैव तद् द॑धाति । \newline
8. तद् द॑धाति दधाति॒ तत् तद् द॑धाति सु॒शर्मा॑ सु॒शर्मा॑ दधाति॒ तत् तद् द॑धाति सु॒शर्मा᳚ । \newline
9. द॒धा॒ति॒ सु॒शर्मा॑ सु॒शर्मा॑ दधाति दधाति सु॒शर्मा᳚ ऽस्यसि सु॒शर्मा॑ दधाति दधाति सु॒शर्मा॑ ऽसि । \newline
10. सु॒शर्मा᳚ ऽस्यसि सु॒शर्मा॑ सु॒शर्मा॑ ऽसि सुप्रतिष्ठा॒नः सु॑प्रतिष्ठा॒नो॑ ऽसि सु॒शर्मा॑ सु॒शर्मा॑ ऽसि सुप्रतिष्ठा॒नः । \newline
11. सु॒शर्मेति॑ सु - शर्मा᳚ । \newline
12. अ॒सि॒ सु॒प्र॒ति॒ष्ठा॒नः सु॑प्रतिष्ठा॒नो᳚ ऽस्यसि सुप्रतिष्ठा॒न इतीति॑ सुप्रतिष्ठा॒नो᳚ ऽस्यसि सुप्रतिष्ठा॒न इति॑ । \newline
13. सु॒प्र॒ति॒ष्ठा॒न इतीति॑ सुप्रतिष्ठा॒नः सु॑प्रतिष्ठा॒न इत्या॑हा॒ हेति॑ सुप्रतिष्ठा॒नः सु॑प्रतिष्ठा॒न इत्या॑ह । \newline
14. सु॒प्र॒ति॒ष्ठा॒न इति॑ सु - प्र॒ति॒ष्ठा॒नः । \newline
15. इत्या॑हा॒हे तीत्या॑ह॒ सोमे॒ सोम॑ आ॒हे तीत्या॑ह॒ सोमे᳚ । \newline
16. आ॒ह॒ सोमे॒ सोम॑ आहाह॒ सोमे॒ हि हि सोम॑ आहाह॒ सोमे॒ हि । \newline
17. सोमे॒ हि हि सोमे॒ सोमे॒ हि सोमꣳ॒॒ सोमꣳ॒॒ हि सोमे॒ सोमे॒ हि सोम᳚म् । \newline
18. हि सोमꣳ॒॒ सोमꣳ॒॒ हि हि सोम॑ मभिगृ॒ह्णा त्य॑भिगृ॒ह्णाति॒ सोमꣳ॒॒ हि हि सोम॑ मभिगृ॒ह्णाति॑ । \newline
19. सोम॑ मभिगृ॒ह्णा त्य॑भिगृ॒ह्णाति॒ सोमꣳ॒॒ सोम॑ मभिगृ॒ह्णाति॒ प्रति॑ष्ठित्यै॒ प्रति॑ष्ठित्या अभिगृ॒ह्णाति॒ सोमꣳ॒॒ सोम॑ मभिगृ॒ह्णाति॒ प्रति॑ष्ठित्यै । \newline
20. अ॒भि॒गृ॒ह्णाति॒ प्रति॑ष्ठित्यै॒ प्रति॑ष्ठित्या अभिगृ॒ह्णा त्य॑भिगृ॒ह्णाति॒ प्रति॑ष्ठित्या ए॒तस्मि॑न्-ने॒तस्मि॒न् प्रति॑ष्ठित्या अभिगृ॒ह्णा त्य॑भिगृ॒ह्णाति॒ प्रति॑ष्ठित्या ए॒तस्मिन्न्॑ । \newline
21. अ॒भि॒गृ॒ह्णातीत्य॑भि - गृ॒ह्णाति॑ । \newline
22. प्रति॑ष्ठित्या ए॒तस्मि॑न्-ने॒तस्मि॒न् प्रति॑ष्ठित्यै॒ प्रति॑ष्ठित्या ए॒तस्मि॒न्॒. वै वा ए॒तस्मि॒न् प्रति॑ष्ठित्यै॒ प्रति॑ष्ठित्या ए॒तस्मि॒न्॒. वै । \newline
23. प्रति॑ष्ठित्या॒ इति॒ प्रति॑ - स्थि॒त्यै॒ । \newline
24. ए॒तस्मि॒न्॒. वै वा ए॒तस्मि॑न्-ने॒तस्मि॒न्॒. वा अप्यपि॒ वा ए॒तस्मि॑न्-ने॒तस्मि॒न्॒. वा अपि॑ । \newline
25. वा अप्यपि॒ वै वा अपि॒ ग्रहे॒ ग्रहे ऽपि॒ वै वा अपि॒ ग्रहे᳚ । \newline
26. अपि॒ ग्रहे॒ ग्रहे ऽप्यपि॒ ग्रहे॑ मनु॒ष्ये᳚भ्यो मनु॒ष्ये᳚भ्यो॒ ग्रहे ऽप्यपि॒ ग्रहे॑ मनु॒ष्ये᳚भ्यः । \newline
27. ग्रहे॑ मनु॒ष्ये᳚भ्यो मनु॒ष्ये᳚भ्यो॒ ग्रहे॒ ग्रहे॑ मनु॒ष्ये᳚भ्यो दे॒वेभ्यो॑ दे॒वेभ्यो॑ मनु॒ष्ये᳚भ्यो॒ ग्रहे॒ ग्रहे॑ मनु॒ष्ये᳚भ्यो दे॒वेभ्यः॑ । \newline
28. म॒नु॒ष्ये᳚भ्यो दे॒वेभ्यो॑ दे॒वेभ्यो॑ मनु॒ष्ये᳚भ्यो मनु॒ष्ये᳚भ्यो दे॒वेभ्यः॑ पि॒तृभ्यः॑ पि॒तृभ्यो॑ दे॒वेभ्यो॑ मनु॒ष्ये᳚भ्यो मनु॒ष्ये᳚भ्यो दे॒वेभ्यः॑ पि॒तृभ्यः॑ । \newline
29. दे॒वेभ्यः॑ पि॒तृभ्यः॑ पि॒तृभ्यो॑ दे॒वेभ्यो॑ दे॒वेभ्यः॑ पि॒तृभ्यः॑ क्रियते क्रियते पि॒तृभ्यो॑ दे॒वेभ्यो॑ दे॒वेभ्यः॑ पि॒तृभ्यः॑ क्रियते । \newline
30. पि॒तृभ्यः॑ क्रियते क्रियते पि॒तृभ्यः॑ पि॒तृभ्यः॑ क्रियते सु॒शर्मा॑ सु॒शर्मा᳚ क्रियते पि॒तृभ्यः॑ पि॒तृभ्यः॑ क्रियते सु॒शर्मा᳚ । \newline
31. पि॒तृभ्य॒ इति॑ पि॒तृ - भ्यः॒ । \newline
32. क्रि॒य॒ते॒ सु॒शर्मा॑ सु॒शर्मा᳚ क्रियते क्रियते सु॒शर्मा᳚ ऽस्यसि सु॒शर्मा᳚ क्रियते क्रियते सु॒शर्मा॑ ऽसि । \newline
33. सु॒शर्मा᳚ ऽस्यसि सु॒शर्मा॑ सु॒शर्मा॑ ऽसि सुप्रतिष्ठा॒नः सु॑प्रतिष्ठा॒नो॑ ऽसि सु॒शर्मा॑ सु॒शर्मा॑ ऽसि सुप्रतिष्ठा॒नः । \newline
34. सु॒शर्मेति॑ सु - शर्मा᳚ । \newline
35. अ॒सि॒ सु॒प्र॒ति॒ष्ठा॒नः सु॑प्रतिष्ठा॒नो᳚ ऽस्यसि सुप्रतिष्ठा॒न इतीति॑ सुप्रतिष्ठा॒नो᳚ ऽस्यसि सुप्रतिष्ठा॒न इति॑ । \newline
36. सु॒प्र॒ति॒ष्ठा॒न इतीति॑ सुप्रतिष्ठा॒नः सु॑प्रतिष्ठा॒न इत्या॑हा॒हेति॑ सुप्रतिष्ठा॒नः सु॑प्रतिष्ठा॒न इत्या॑ह । \newline
37. सु॒प्र॒ति॒ष्ठा॒न इति॑ सु - प्र॒ति॒ष्ठा॒नः । \newline
38. इत्या॑हा॒हे तीत्या॑ह मनु॒ष्ये᳚भ्यो मनु॒ष्ये᳚भ्य आ॒हे तीत्या॑ह मनु॒ष्ये᳚भ्यः । \newline
39. आ॒ह॒ म॒नु॒ष्ये᳚भ्यो मनु॒ष्ये᳚भ्य आहाह मनु॒ष्ये᳚भ्य ए॒वैव म॑नु॒ष्ये᳚भ्य आहाह मनु॒ष्ये᳚भ्य ए॒व । \newline
40. म॒नु॒ष्ये᳚भ्य ए॒वैव म॑नु॒ष्ये᳚भ्यो मनु॒ष्ये᳚भ्य ए॒वैते नै॒ते नै॒व म॑नु॒ष्ये᳚भ्यो मनु॒ष्ये᳚भ्य ए॒वैतेन॑ । \newline
41. ए॒वैते नै॒ते नै॒वैवैतेन॑ करोति करो त्ये॒ते नै॒वैवैतेन॑ करोति । \newline
42. ए॒तेन॑ करोति करो त्ये॒ते नै॒तेन॑ करोति बृ॒हद् बृ॒हत् क॑रो त्ये॒ते नै॒तेन॑ करोति बृ॒हत् । \newline
43. क॒रो॒ति॒ बृ॒हद् बृ॒हत् क॑रोति करोति बृ॒ह दितीति॑ बृ॒हत् क॑रोति करोति बृ॒ह दिति॑ । \newline
44. बृ॒ह दितीति॑ बृ॒हद् बृ॒ह दित्या॑हा॒ हेति॑ बृ॒हद् बृ॒ह दित्या॑ह । \newline
45. इत्या॑हा॒हे तीत्या॑ह दे॒वेभ्यो॑ दे॒वेभ्य॑ आ॒हे तीत्या॑ह दे॒वेभ्यः॑ । \newline
46. आ॒ह॒ दे॒वेभ्यो॑ दे॒वेभ्य॑ आहाह दे॒वेभ्य॑ ए॒वैव दे॒वेभ्य॑ आहाह दे॒वेभ्य॑ ए॒व । \newline
47. दे॒वेभ्य॑ ए॒वैव दे॒वेभ्यो॑ दे॒वेभ्य॑ ए॒वैते नै॒तेनै॒व दे॒वेभ्यो॑ दे॒वेभ्य॑ ए॒वै तेन॑ । \newline
48. ए॒वैते नै॒तेनै॒वै वैतेन॑ करोति करो त्ये॒ते नै॒वैवैतेन॑ करोति । \newline
49. ए॒तेन॑ करोति करो त्ये॒ते नै॒तेन॑ करोति॒ नमो॒ नम॑ स्करो त्ये॒ते नै॒तेन॑ करोति॒ नमः॑ । \newline
50. क॒रो॒ति॒ नमो॒ नम॑ स्करोति करोति॒ नम॒ इतीति॒ नम॑ स्करोति करोति॒ नम॒ इति॑ । \newline
51. नम॒ इतीति॒ नमो॒ नम॒ इत्या॑हा॒ हेति॒ नमो॒ नम॒ इत्या॑ह । \newline
52. इत्या॑हा॒हे तीत्या॑ह पि॒तृभ्यः॑ पि॒तृभ्य॑ आ॒हे तीत्या॑ह पि॒तृभ्यः॑ । \newline
53. आ॒ह॒ पि॒तृभ्यः॑ पि॒तृभ्य॑ आहाह पि॒तृभ्य॑ ए॒वैव पि॒तृभ्य॑ आहाह पि॒तृभ्य॑ ए॒व । \newline
54. पि॒तृभ्य॑ ए॒वैव पि॒तृभ्यः॑ पि॒तृभ्य॑ ए॒वैते नै॒तेनै॒व पि॒तृभ्यः॑ पि॒तृभ्य॑ ए॒वैतेन॑ । \newline
55. पि॒तृभ्य॒ इति॑ पि॒तृ - भ्यः॒ । \newline
56. ए॒वैते नै॒तेनै॒वै वैतेन॑ करोति करो त्ये॒ते नै॒वैवैतेन॑ करोति । \newline
57. ए॒तेन॑ करोति करो त्ये॒तेनै॒ तेन॑ करो त्ये॒ताव॑ती रे॒ताव॑तीः करो त्ये॒ते नै॒तेन॑ करो त्ये॒ताव॑तीः । \newline
58. क॒रो॒ त्ये॒ताव॑ती रे॒ताव॑तीः करोति करो त्ये॒ताव॑ती॒र् वै वा ए॒ताव॑तीः करोति करो त्ये॒ताव॑ती॒र् वै । \newline
59. ए॒ताव॑ती॒र् वै वा ए॒ताव॑ती रे॒ताव॑ती॒र् वै दे॒वता॑ दे॒वता॒ वा ए॒ताव॑ती रे॒ताव॑ती॒र् वै दे॒वताः᳚ । \newline
60. वै दे॒वता॑ दे॒वता॒ वै वै दे॒वता॒ स्ताभ्य॒ स्ताभ्यो॑ दे॒वता॒ वै वै दे॒वता॒ स्ताभ्यः॑ । \newline
61. दे॒वता॒ स्ताभ्य॒ स्ताभ्यो॑ दे॒वता॑ दे॒वता॒ स्ताभ्य॑ ए॒वैव ताभ्यो॑ दे॒वता॑ दे॒वता॒ स्ताभ्य॑ ए॒व । \newline
62. ताभ्य॑ ए॒वैव ताभ्य॒ स्ताभ्य॑ ए॒वैन॑ मेन मे॒व ताभ्य॒ स्ताभ्य॑ ए॒वैन᳚म् । \newline
63. ए॒वैन॑ मेन मे॒वै वैनꣳ॒॒ सर्वा᳚भ्यः॒ सर्वा᳚भ्य एन मे॒वै वैनꣳ॒॒ सर्वा᳚भ्यः । \newline
64. ए॒नꣳ॒॒ सर्वा᳚भ्यः॒ सर्वा᳚भ्य एन मेनꣳ॒॒ सर्वा᳚भ्यो गृह्णाति गृह्णाति॒ सर्वा᳚भ्य एन मेनꣳ॒॒ सर्वा᳚भ्यो गृह्णाति । \newline
65. सर्वा᳚भ्यो गृह्णाति गृह्णाति॒ सर्वा᳚भ्यः॒ सर्वा᳚भ्यो गृह्णा त्ये॒ष ए॒ष गृ॑ह्णाति॒ सर्वा᳚भ्यः॒ सर्वा᳚भ्यो गृह्णा त्ये॒षः । \newline
66. गृ॒ह्णा॒ त्ये॒ष ए॒ष गृ॑ह्णाति गृह्णा त्ये॒ष ते॑ त ए॒ष गृ॑ह्णाति गृह्णा त्ये॒ष ते᳚ । \newline
67. ए॒ष ते॑ त ए॒ष ए॒ष ते॒ योनि॒र् योनि॑ स्त ए॒ष ए॒ष ते॒ योनिः॑ । \newline
68. ते॒ योनि॒र् योनि॑ स्ते ते॒ योनि॒र् विश्वे᳚भ्यो॒ विश्वे᳚भ्यो॒ योनि॑ स्ते ते॒ योनि॒र् विश्वे᳚भ्यः । \newline
69. योनि॒र् विश्वे᳚भ्यो॒ विश्वे᳚भ्यो॒ योनि॒र् योनि॒र् विश्वे᳚भ्य स्त्वा त्वा॒ विश्वे᳚भ्यो॒ योनि॒र् योनि॒र् विश्वे᳚भ्य स्त्वा । \newline
70. विश्वे᳚भ्य स्त्वा त्वा॒ विश्वे᳚भ्यो॒ विश्वे᳚भ्य स्त्वा दे॒वेभ्यो॑ दे॒वेभ्य॑ स्त्वा॒ विश्वे᳚भ्यो॒ विश्वे᳚भ्य स्त्वा दे॒वेभ्यः॑ । \newline
71. त्वा॒ दे॒वेभ्यो॑ दे॒वेभ्य॑ स्त्वा त्वा दे॒वेभ्य॒ इतीति॑ दे॒वेभ्य॑ स्त्वा त्वा दे॒वेभ्य॒ इति॑ । \newline
72. दे॒वेभ्य॒ इतीति॑ दे॒वेभ्यो॑ दे॒वेभ्य॒ इत्या॑हा॒ हेति॑ दे॒वेभ्यो॑ दे॒वेभ्य॒ इत्या॑ह । \newline
73. इत्या॑हा॒हे तीत्या॑ह वैश्वदे॒वो वै᳚श्वदे॒व आ॒हे तीत्या॑ह वैश्वदे॒वः । \newline
74. आ॒ह॒ वै॒श्व॒दे॒वो वै᳚श्वदे॒व आ॑हाह वैश्वदे॒वो हि हि वै᳚श्वदे॒व आ॑हाह वैश्वदे॒वो हि । \newline
75. वै॒श्व॒दे॒वो हि हि वै᳚श्वदे॒वो वै᳚श्वदे॒वो ह्ये॑ष ए॒ष हि वै᳚श्वदे॒वो वै᳚श्वदे॒वो ह्ये॑षः । \newline
76. वै॒श्व॒दे॒व इति॑ वैश्व - दे॒वः । \newline
77. ह्ये॑ष ए॒ष हि ह्ये॑षः । \newline
78. ए॒ष इत्ये॒षः । \newline
\pagebreak
\markright{ TS 6.5.8.1  \hfill https://www.vedavms.in \hfill}

\section{ TS 6.5.8.1 }

\textbf{TS 6.5.8.1 } \newline
\textbf{Samhita Paata} \newline

प्रा॒णो वा ए॒ष यदु॑पाꣳ॒॒शु-र्यदु॑पाꣳशुपा॒त्रेण॑ प्रथ॒मश्चो᳚त्त॒मश्च॒ ग्रहौ॑ गृ॒ह्येते᳚ प्रा॒णमे॒वानु॑ प्र॒यन्ति॑ प्रा॒णमनूद्य॑न्ति प्र॒जाप॑ति॒र्वा ए॒ष यदा᳚ग्रय॒णः प्रा॒ण उ॑पाꣳ॒॒शुः पत्नीः᳚ प्र॒जाः प्र ज॑नयन्ति॒ यदु॑पाꣳशुपा॒त्रेण॑ पात्नीव॒तमा᳚ग्रय॒णाद्-गृ॒ह्णाति॑ प्र॒जानां᳚ प्र॒जन॑नाय॒ तस्मा᳚त् प्रा॒णं प्र॒जा अनु॒ प्र जा॑यन्ते दे॒वा वा इ॒त इ॑तः॒ पत्नीः᳚ सुव॒र्गं- [  ] \newline

\textbf{Pada Paata} \newline

प्रा॒ण इति॑ प्र - अ॒नः । वै । ए॒षः । यत् । उ॒पाꣳ॒॒शुरित्यु॑प-अꣳ॒॒शुः । यत् । उ॒पाꣳ॒॒शु॒पा॒त्रेणेत्यु॑पाꣳशु - पा॒त्रेण॑ । प्र॒थ॒मः । च॒ । उ॒त्त॒म इत्यु॑त् - त॒मः । च॒ । ग्रहौ᳚ । गृ॒ह्येते॒ इति॑ । प्रा॒णमिति॑ प्र - अ॒नम् । ए॒व । अन्विति॑ । प्र॒यन्तीति॑ प्र - यन्ति॑ । प्रा॒णमिति॑ प्र - अ॒नम् । अनु॑ । उदिति॑ । य॒न्ति॒ । प्र॒जाप॑ति॒रिति॑ प्र॒जा - प॒तिः॒ । वै । ए॒षः । यत् । आ॒ग्र॒य॒णः । प्रा॒ण इति॑ प्र - अ॒नः । उ॒पाꣳ॒॒शुरित्यु॑प-अꣳ॒॒शुः । पत्नीः᳚ । प्र॒जा इति॑ प्र - जाः । प्रेति॑ । ज॒न॒य॒न्ति॒ । यत् । उ॒पाꣳ॒॒शु॒पा॒त्रेणेत्यु॑पाꣳशु - पा॒त्रेण॑ । पा॒त्नी॒व॒तमिति॑ पात्नी - व॒तम् । आ॒ग्र॒य॒णात् । गृ॒ह्णाति॑ । प्र॒जाना॒मिति॑ प्र - जाना᳚म् । प्र॒जन॑ना॒येति॑ प्र - जन॑नाय । तस्मा᳚त् । प्रा॒णमिति॑ प्र - अ॒नम् । प्र॒जा इति॑ प्र-जाः । अनु॑ । प्रेति॑ । जा॒य॒न्ते॒ । दे॒वाः । वै । इ॒त इ॑त॒ इती॒तः - इ॒तः॒ । पत्नीः᳚ । सु॒व॒र्गमिति॑ सुवः-गम् ।  \newline


\textbf{Krama Paata} \newline

प्रा॒णो वै । प्रा॒ण इति॑ प्र - अ॒नः । वा ए॒षः । ए॒ष यत् । यदु॑पाꣳ॒॒शुः । उ॒पाꣳ॒॒शुर् यत् । उ॒पाꣳ॒॒शुरित्यु॑प - अꣳ॒॒शुः । यदु॑पाꣳशुपा॒त्रेण॑ । उ॒पाꣳ॒॒शु॒पा॒त्रेण॑ प्रथ॒मः । उ॒पाꣳ॒॒शु॒पा॒त्रेणेत्यु॑पाꣳशु - पा॒त्रेण॑ । प्र॒थ॒मश्च॑ । चो॒त्त॒मः । उ॒त्त॒मश्च॑ । उ॒त्त॒म इत्यु॑त् - त॒मः । च॒ ग्रहौ᳚ । ग्रहौ॑ गृ॒ह्येते᳚ । गृ॒ह्येते᳚ प्रा॒णम् । गृ॒ह्येते॒ इति॑ गृ॒ह्येते᳚ । प्रा॒णमे॒व । प्रा॒णमिति॑ प्र - अ॒नम् । ए॒वानु॑ । अनु॑ प्र॒यन्ति॑ । प्र॒यन्ति॑ प्रा॒णम् । प्र॒यन्तीति॑ प्र - यन्ति॑ । प्रा॒णमनु॑ । प्रा॒णमिति॑ प्र - अ॒नम् । अनूत् । उद् य॑न्ति । य॒न्ति॒ प्र॒जाप॑तिः । प्र॒जाप॑ति॒र् वै । प्र॒जाप॑ति॒रिति॑ प्र॒जा - प॒तिः॒ । वा ए॒षः । ए॒ष यत् । यदा᳚ग्रय॒णः । आ॒ग्र॒य॒णः प्रा॒णः । प्रा॒ण उ॑पाꣳ॒॒शुः । प्रा॒ण इति॑ प्र - अ॒नः । उ॒पाꣳ॒॒शुः पत्नीः᳚ । उ॒पाꣳ॒॒शुरित्यु॑प - अꣳ॒॒शुः । पत्नीः᳚ प्र॒जाः । प्र॒जाः प्र । प्र॒जा इति॑ प्र - जाः । प्र ज॑नयन्ति । ज॒न॒य॒न्ति॒ यत् । यदु॑पाꣳशुपा॒त्रेण॑ । उ॒पाꣳ॒॒शु॒पा॒त्रेण॑ पात्नीव॒तम् । उ॒पाꣳ॒॒शु॒पा॒त्रेणेत्यु॑पाꣳशु - पा॒त्रेण॑ । पा॒त्नी॒व॒तमा᳚ग्रय॒णात् । पा॒त्नी॒व॒तमिति॑ पात्नी - व॒तम् । आ॒ग्र॒य॒णाद् गृ॒ह्णाति॑ । गृ॒ह्णाति॑ प्र॒जाना᳚म् । प्र॒जाना᳚म् प्र॒जन॑नाय । प्र॒जाना॒मिति॑ प्र - जाना᳚म् । प्र॒जन॑नाय॒ तस्मा᳚त् । प्र॒जन॑ना॒येति॑ प्र - जन॑नाय । तस्मा᳚त् प्रा॒णम् । प्रा॒णम् प्र॒जाः । प्रा॒णमिति॑ प्र - अ॒नम् । प्र॒जा अनु॑ । प्र॒जा इति॑ प्र - जाः । अनु॒ प्र । प्र जा॑यन्ते । जा॒य॒न्ते॒ दे॒वाः । दे॒वा वै । वा इ॒त‍इ॑तः । इ॒त‍इ॑तः॒ पत्नीः᳚ । इ॒त इ॑त॒ इती॒तः - इ॒तः॒ । पत्नीः᳚ सुव॒र्गम् । सु॒व॒र्गम् ॅलो॒कम् । सु॒व॒र्गमिति॑ सुवः - गम् \newline

\textbf{Jatai Paata} \newline

1. प्रा॒णो वै वै प्रा॒णः प्रा॒णो वै । \newline
2. प्रा॒ण इति॑ प्र - अ॒नः । \newline
3. वा ए॒ष ए॒ष वै वा ए॒षः । \newline
4. ए॒ष यद् यदे॒ष ए॒ष यत् । \newline
5. यदु॑पाꣳ॒॒शु रु॑पाꣳ॒॒शुर् यद् यदु॑पाꣳ॒॒शुः । \newline
6. उ॒पाꣳ॒॒शुर् यद् यदु॑पाꣳ॒॒शु रु॑पाꣳ॒॒शुर् यत् । \newline
7. उ॒पाꣳ॒॒शुरित्यु॑प - अꣳ॒॒शुः । \newline
8. यदु॑पाꣳशुपा॒त्रेणो॑ पाꣳशुपा॒त्रेण॒ यद् यदु॑पाꣳशुपा॒त्रेण॑ । \newline
9. उ॒पाꣳ॒॒शु॒पा॒त्रेण॑ प्रथ॒मः प्र॑थ॒म उ॑पाꣳशुपा॒त्रेणो॑ पाꣳशुपा॒त्रेण॑ प्रथ॒मः । \newline
10. उ॒पाꣳ॒॒शु॒पा॒त्रेणेत्यु॑पाꣳशु - पा॒त्रेण॑ । \newline
11. प्र॒थ॒म श्च॑ च प्रथ॒मः प्र॑थ॒म श्च॑ । \newline
12. चो॒त्त॒म उ॑त्त॒मश्च॑ चोत्त॒मः । \newline
13. उ॒त्त॒म श्च॑ चोत्त॒म उ॑त्त॒म श्च॑ । \newline
14. उ॒त्त॒म इत्यु॑त् - त॒मः । \newline
15. च॒ ग्रहौ॒ ग्रहौ॑ च च॒ ग्रहौ᳚ । \newline
16. ग्रहौ॑ गृ॒ह्येते॑ गृ॒ह्येते॒ ग्रहौ॒ ग्रहौ॑ गृ॒ह्येते᳚ । \newline
17. गृ॒ह्येते᳚ प्रा॒णम् प्रा॒णम् गृ॒ह्येते॑ गृ॒ह्येते᳚ प्रा॒णम् । \newline
18. गृ॒ह्येते॒ इति॑ गृ॒ह्येते᳚ । \newline
19. प्रा॒ण मे॒वैव प्रा॒णम् प्रा॒ण मे॒व । \newline
20. प्रा॒णमिति॑ प्र - अ॒नम् । \newline
21. ए॒वान् वन् वे॒वै वानु॑ । \newline
22. अनु॑ प्र॒यन्ति॑ प्र॒यन् त्यन् वनु॑ प्र॒यन्ति॑ । \newline
23. प्र॒यन्ति॑ प्रा॒णम् प्रा॒णम् प्र॒यन्ति॑ प्र॒यन्ति॑ प्रा॒णम् । \newline
24. प्र॒यन्तीति॑ प्र - यन्ति॑ । \newline
25. प्रा॒ण मन् वनु॑ प्रा॒णम् प्रा॒ण मनु॑ । \newline
26. प्रा॒णमिति॑ प्र - अ॒नम् । \newline
27. अनू दुदन् वनूत् । \newline
28. उद् य॑न्ति य॒न् त्युदुद् य॑न्ति । \newline
29. य॒न्ति॒ प्र॒जाप॑तिः प्र॒जाप॑तिर् यन्ति यन्ति प्र॒जाप॑तिः । \newline
30. प्र॒जाप॑ति॒र् वै वै प्र॒जाप॑तिः प्र॒जाप॑ति॒र् वै । \newline
31. प्र॒जाप॑ति॒रिति॑ प्र॒जा - प॒तिः॒ । \newline
32. वा ए॒ष ए॒ष वै वा ए॒षः । \newline
33. ए॒ष यद् यदे॒ष ए॒ष यत् । \newline
34. यदा᳚ग्रय॒ण आ᳚ग्रय॒णो यद् यदा᳚ग्रय॒णः । \newline
35. आ॒ग्र॒य॒णः प्रा॒णः प्रा॒ण आ᳚ग्रय॒ण आ᳚ग्रय॒णः प्रा॒णः । \newline
36. प्रा॒ण उ॑पाꣳ॒॒शु रु॑पाꣳ॒॒शुः प्रा॒णः प्रा॒ण उ॑पाꣳ॒॒शुः । \newline
37. प्रा॒ण इति॑ प्र - अ॒नः । \newline
38. उ॒पाꣳ॒॒शुः पत्नीः॒ पत्नी॑ रुपाꣳ॒॒शु रु॑पाꣳ॒॒शुः पत्नीः᳚ । \newline
39. उ॒पाꣳ॒॒शुरित्यु॑प - अꣳ॒॒शुः । \newline
40. पत्नीः᳚ प्र॒जाः प्र॒जाः पत्नीः॒ पत्नीः᳚ प्र॒जाः । \newline
41. प्र॒जाः प्र प्र प्र॒जाः प्र॒जाः प्र । \newline
42. प्र॒जा इति॑ प्र - जाः । \newline
43. प्र ज॑नयन्ति जनयन्ति॒ प्र प्र ज॑नयन्ति । \newline
44. ज॒न॒य॒न्ति॒ यद् यज् ज॑नयन्ति जनयन्ति॒ यत् । \newline
45. यदु॑पाꣳशुपा॒त्रे णो॑पाꣳशुपा॒त्रेण॒ यद् यदु॑पाꣳशुपा॒त्रेण॑ । \newline
46. उ॒पाꣳ॒॒शु॒पा॒त्रेण॑ पात्नीव॒तम् पा᳚त्नीव॒त मु॑पाꣳशुपा॒त्रे णो॑पाꣳशुपा॒त्रेण॑ पात्नीव॒तम् । \newline
47. उ॒पाꣳ॒॒शु॒पा॒त्रेणेत्यु॑पाꣳशु - पा॒त्रेण॑ । \newline
48. पा॒त्नी॒व॒त मा᳚ग्रय॒णा दा᳚ग्रय॒णात् पा᳚त्नीव॒तम् पा᳚त्नीव॒त मा᳚ग्रय॒णात् । \newline
49. पा॒त्नी॒व॒तमिति॑ पात्नी - व॒तम् । \newline
50. आ॒ग्र॒य॒णाद् गृ॒ह्णाति॑ गृ॒ह्णा त्या᳚ग्रय॒णा दा᳚ग्रय॒णाद् गृ॒ह्णाति॑ । \newline
51. गृ॒ह्णाति॑ प्र॒जाना᳚म् प्र॒जाना᳚म् गृ॒ह्णाति॑ गृ॒ह्णाति॑ प्र॒जाना᳚म् । \newline
52. प्र॒जाना᳚म् प्र॒जन॑नाय प्र॒जन॑नाय प्र॒जाना᳚म् प्र॒जाना᳚म् प्र॒जन॑नाय । \newline
53. प्र॒जाना॒मिति॑ प्र - जाना᳚म् । \newline
54. प्र॒जन॑नाय॒ तस्मा॒त् तस्मा᳚त् प्र॒जन॑नाय प्र॒जन॑नाय॒ तस्मा᳚त् । \newline
55. प्र॒जन॑ना॒येति॑ प्र - जन॑नाय । \newline
56. तस्मा᳚त् प्रा॒णम् प्रा॒णम् तस्मा॒त् तस्मा᳚त् प्रा॒णम् । \newline
57. प्रा॒णम् प्र॒जाः प्र॒जाः प्रा॒णम् प्रा॒णम् प्र॒जाः । \newline
58. प्रा॒णमिति॑ प्र - अ॒नम् । \newline
59. प्र॒जा अन्वनु॑ प्र॒जाः प्र॒जा अनु॑ । \newline
60. प्र॒जा इति॑ प्र - जाः । \newline
61. अनु॒ प्र प्राण्वनु॒ प्र । \newline
62. प्र जा॑यन्ते जायन्ते॒ प्र प्र जा॑यन्ते । \newline
63. जा॒य॒न्ते॒ दे॒वा दे॒वा जा॑यन्ते जायन्ते दे॒वाः । \newline
64. दे॒वा वै वै दे॒वा दे॒वा वै । \newline
65. वा इ॒त‌इ॑त इ॒त‌इ॑तो॒ वै वा इ॒त‌इ॑तः । \newline
66. इ॒त‌इ॑तः॒ पत्नीः॒ पत्नी॑ रि॒त‌इ॑त इ॒त‌इ॑तः॒ पत्नीः᳚ । \newline
67. इ॒त‌इ॑त॒ इती॒तः - इ॒तः॒ । \newline
68. पत्नीः᳚ सुव॒र्गꣳ सु॑व॒र्गम् पत्नीः॒ पत्नीः᳚ सुव॒र्गम् । \newline
69. सु॒व॒र्गम् ॅलो॒कम् ॅलो॒कꣳ सु॑व॒र्गꣳ सु॑व॒र्गम् ॅलो॒कम् । \newline
70. सु॒व॒र्गमिति॑ सुवः - गम् । \newline

\textbf{Ghana Paata } \newline

1. प्रा॒णो वै वै प्रा॒णः प्रा॒णो वा ए॒ष ए॒ष वै प्रा॒णः प्रा॒णो वा ए॒षः । \newline
2. प्रा॒ण इति॑ प्र - अ॒नः । \newline
3. वा ए॒ष ए॒ष वै वा ए॒ष यद् यदे॒ष वै वा ए॒ष यत् । \newline
4. ए॒ष यद् यदे॒ष ए॒ष यदु॑पाꣳ॒॒शु रु॑पाꣳ॒॒शुर् यदे॒ष ए॒ष यदु॑पाꣳ॒॒शुः । \newline
5. यदु॑पाꣳ॒॒शु रु॑पाꣳ॒॒शुर् यद् यदु॑पाꣳ॒॒शुर् यद् यदु॑पाꣳ॒॒शुर् यद् यदु॑पाꣳ॒॒शुर् यत् । \newline
6. उ॒पाꣳ॒॒शुर् यद् यदु॑पाꣳ॒॒शु रु॑पाꣳ॒॒शुर् यदु॑पाꣳशुपा॒त्रे णो॑पाꣳशुपा॒त्रेण॒ यदु॑पाꣳ॒॒शु रु॑पाꣳ॒॒शुर् यदु॑पाꣳशुपा॒त्रेण॑ । \newline
7. उ॒पाꣳ॒॒शुरित्यु॑प - अꣳ॒॒शुः । \newline
8. यदु॑पाꣳशुपा॒त्रे णो॑पाꣳशुपा॒त्रेण॒ यद् यदु॑पाꣳशुपा॒त्रेण॑ प्रथ॒मः प्र॑थ॒म उ॑पाꣳशुपा॒त्रेण॒ यद् यदु॑पाꣳशुपा॒त्रेण॑ प्रथ॒मः । \newline
9. उ॒पाꣳ॒॒शु॒पा॒त्रेण॑ प्रथ॒मः प्र॑थ॒म उ॑पाꣳशुपा॒त्रे णो॑पाꣳशुपा॒त्रेण॑ प्रथ॒मश्च॑ च प्रथ॒म उ॑पाꣳशुपा॒त्रे णो॑पाꣳशुपा॒त्रेण॑ प्रथ॒मश्च॑ । \newline
10. उ॒पाꣳ॒॒शु॒पा॒त्रेणेत्यु॑पाꣳशु - पा॒त्रेण॑ । \newline
11. प्र॒थ॒मश्च॑ च प्रथ॒मः प्र॑थ॒म श्चो᳚त्त॒म उ॑त्त॒मश्च॑ प्रथ॒मः प्र॑थ॒म श्चो᳚त्त॒मः । \newline
12. चो॒त्त॒म उ॑त्त॒म श्च॑ चोत्त॒म श्च॑ चोत्त॒म श्च॑ चोत्त॒म श्च॑ । \newline
13. उ॒त्त॒म श्च॑ चोत्त॒म उ॑त्त॒म श्च॒ ग्रहौ॒ ग्रहौ॑ चोत्त॒म उ॑त्त॒म श्च॒ ग्रहौ᳚ । \newline
14. उ॒त्त॒म इत्यु॑त् - त॒मः । \newline
15. च॒ ग्रहौ॒ ग्रहौ॑ च च॒ ग्रहौ॑ गृ॒ह्येते॑ गृ॒ह्येते॒ ग्रहौ॑ च च॒ ग्रहौ॑ गृ॒ह्येते᳚ । \newline
16. ग्रहौ॑ गृ॒ह्येते॑ गृ॒ह्येते॒ ग्रहौ॒ ग्रहौ॑ गृ॒ह्येते᳚ प्रा॒णम् प्रा॒णम् गृ॒ह्येते॒ ग्रहौ॒ ग्रहौ॑ गृ॒ह्येते᳚ प्रा॒णम् । \newline
17. गृ॒ह्येते᳚ प्रा॒णम् प्रा॒णम् गृ॒ह्येते॑ गृ॒ह्येते᳚ प्रा॒ण मे॒वैव प्रा॒णम् गृ॒ह्येते॑ गृ॒ह्येते᳚ प्रा॒ण मे॒व । \newline
18. गृ॒ह्येते॒ इति॑ गृ॒ह्येते᳚ । \newline
19. प्रा॒ण मे॒वैव प्रा॒णम् प्रा॒ण मे॒वान् वन् वे॒व प्रा॒णम् प्रा॒ण मे॒वानु॑ । \newline
20. प्रा॒णमिति॑ प्र - अ॒नम् । \newline
21. ए॒वान् वन् वे॒वै वानु॑ प्र॒यन्ति॑ प्र॒यन् त्यन् वे॒वै वानु॑ प्र॒यन्ति॑ । \newline
22. अनु॑ प्र॒यन्ति॑ प्र॒यन् त्यन् वनु॑ प्र॒यन्ति॑ प्रा॒णम् प्रा॒णम् प्र॒यन् त्यन् वनु॑ प्र॒यन्ति॑ प्रा॒णम् । \newline
23. प्र॒यन्ति॑ प्रा॒णम् प्रा॒णम् प्र॒यन्ति॑ प्र॒यन्ति॑ प्रा॒ण मन्वनु॑ प्रा॒णम् प्र॒यन्ति॑ प्र॒यन्ति॑ प्रा॒ण मनु॑ । \newline
24. प्र॒यन्तीति॑ प्र - यन्ति॑ । \newline
25. प्रा॒ण मन्वनु॑ प्रा॒णम् प्रा॒ण मनू दुदनु॑ प्रा॒णम् प्रा॒ण मनूत् । \newline
26. प्रा॒णमिति॑ प्र - अ॒नम् । \newline
27. अनूदु दन् वनूद् य॑न्ति य॒न् त्युदन् वनूद् य॑न्ति । \newline
28. उद् य॑न्ति य॒न्त्युदुद् य॑न्ति प्र॒जाप॑तिः प्र॒जाप॑तिर् य॒न् त्युदुद् य॑न्ति प्र॒जाप॑तिः । \newline
29. य॒न्ति॒ प्र॒जाप॑तिः प्र॒जाप॑तिर् यन्ति यन्ति प्र॒जाप॑ति॒र् वै वै प्र॒जाप॑तिर् यन्ति यन्ति प्र॒जाप॑ति॒र् वै । \newline
30. प्र॒जाप॑ति॒र् वै वै प्र॒जाप॑तिः प्र॒जाप॑ति॒र् वा ए॒ष ए॒ष वै प्र॒जाप॑तिः प्र॒जाप॑ति॒र् वा ए॒षः । \newline
31. प्र॒जाप॑ति॒रिति॑ प्र॒जा - प॒तिः॒ । \newline
32. वा ए॒ष ए॒ष वै वा ए॒ष यद् यदे॒ष वै वा ए॒ष यत् । \newline
33. ए॒ष यद् यदे॒ष ए॒ष यदा᳚ग्रय॒ण आ᳚ग्रय॒णो यदे॒ष ए॒ष यदा᳚ग्रय॒णः । \newline
34. यदा᳚ग्रय॒ण आ᳚ग्रय॒णो यद् यदा᳚ग्रय॒णः प्रा॒णः प्रा॒ण आ᳚ग्रय॒णो यद् यदा᳚ग्रय॒णः प्रा॒णः । \newline
35. आ॒ग्र॒य॒णः प्रा॒णः प्रा॒ण आ᳚ग्रय॒ण आ᳚ग्रय॒णः प्रा॒ण उ॑पाꣳ॒॒शु रु॑पाꣳ॒॒शुः प्रा॒ण आ᳚ग्रय॒ण आ᳚ग्रय॒णः प्रा॒ण उ॑पाꣳ॒॒शुः । \newline
36. प्रा॒ण उ॑पाꣳ॒॒शु रु॑पाꣳ॒॒शुः प्रा॒णः प्रा॒ण उ॑पाꣳ॒॒शुः पत्नीः॒ पत्नी॑ रुपाꣳ॒॒शुः प्रा॒णः प्रा॒ण उ॑पाꣳ॒॒शुः पत्नीः᳚ । \newline
37. प्रा॒ण इति॑ प्र - अ॒नः । \newline
38. उ॒पाꣳ॒॒शुः पत्नीः॒ पत्नी॑ रुपाꣳ॒॒शु रु॑पाꣳ॒॒शुः पत्नीः᳚ प्र॒जाः प्र॒जाः पत्नी॑ रुपाꣳ॒॒शु रु॑पाꣳ॒॒शुः पत्नीः᳚ प्र॒जाः । \newline
39. उ॒पाꣳ॒॒शुरित्यु॑प - अꣳ॒॒शुः । \newline
40. पत्नीः᳚ प्र॒जाः प्र॒जाः पत्नीः॒ पत्नीः᳚ प्र॒जाः प्र प्र प्र॒जाः पत्नीः॒ पत्नीः᳚ प्र॒जाः प्र । \newline
41. प्र॒जाः प्र प्र प्र॒जाः प्र॒जाः प्र ज॑नयन्ति जनयन्ति॒ प्र प्र॒जाः प्र॒जाः प्र ज॑नयन्ति । \newline
42. प्र॒जा इति॑ प्र - जाः । \newline
43. प्र ज॑नयन्ति जनयन्ति॒ प्र प्र ज॑नयन्ति॒ यद् यज् ज॑नयन्ति॒ प्र प्र ज॑नयन्ति॒ यत् । \newline
44. ज॒न॒य॒न्ति॒ यद् यज् ज॑नयन्ति जनयन्ति॒ यदु॑पाꣳशुपा॒त्रे णो॑पाꣳशुपा॒त्रेण॒ यज् ज॑नयन्ति जनयन्ति॒ यदु॑पाꣳशुपा॒त्रेण॑ । \newline
45. यदु॑पाꣳशुपा॒त्रे णो॑पाꣳशुपा॒त्रेण॒ यद् यदु॑पाꣳशुपा॒त्रेण॑ पात्नीव॒तम् पा᳚त्नीव॒त मु॑पाꣳशुपा॒त्रेण॒ यद् यदु॑पाꣳशुपा॒त्रेण॑ पात्नीव॒तम् । \newline
46. उ॒पाꣳ॒॒शु॒पा॒त्रेण॑ पात्नीव॒तम् पा᳚त्नीव॒त मु॑पाꣳशुपा॒त्रे णो॑पाꣳशुपा॒त्रेण॑ पात्नीव॒त मा᳚ग्रय॒णा दा᳚ग्रय॒णात् पा᳚त्नीव॒त मु॑पाꣳशुपा॒त्रे णो॑पाꣳशुपा॒त्रेण॑ पात्नीव॒त मा᳚ग्रय॒णात् । \newline
47. उ॒पाꣳ॒॒शु॒पा॒त्रेणेत्यु॑पाꣳशु - पा॒त्रेण॑ । \newline
48. पा॒त्नी॒व॒त मा᳚ग्रय॒णा दा᳚ग्रय॒णात् पा᳚त्नीव॒तम् पा᳚त्नीव॒त मा᳚ग्रय॒णाद् गृ॒ह्णाति॑ गृ॒ह्णा त्या᳚ग्रय॒णात् पा᳚त्नीव॒तम् पा᳚त्नीव॒त मा᳚ग्रय॒णाद् गृ॒ह्णाति॑ । \newline
49. पा॒त्नी॒व॒तमिति॑ पात्नी - व॒तम् । \newline
50. आ॒ग्र॒य॒णाद् गृ॒ह्णाति॑ गृ॒ह्णा त्या᳚ग्रय॒णा दा᳚ग्रय॒णाद् गृ॒ह्णाति॑ प्र॒जाना᳚म् प्र॒जाना᳚म् गृ॒ह्णा त्या᳚ग्रय॒णा दा᳚ग्रय॒णाद् गृ॒ह्णाति॑ प्र॒जाना᳚म् । \newline
51. गृ॒ह्णाति॑ प्र॒जाना᳚म् प्र॒जाना᳚म् गृ॒ह्णाति॑ गृ॒ह्णाति॑ प्र॒जाना᳚म् प्र॒जन॑नाय प्र॒जन॑नाय प्र॒जाना᳚म् गृ॒ह्णाति॑ गृ॒ह्णाति॑ प्र॒जाना᳚म् प्र॒जन॑नाय । \newline
52. प्र॒जाना᳚म् प्र॒जन॑नाय प्र॒जन॑नाय प्र॒जाना᳚म् प्र॒जाना᳚म् प्र॒जन॑नाय॒ तस्मा॒त् तस्मा᳚त् प्र॒जन॑नाय प्र॒जाना᳚म् प्र॒जाना᳚म् प्र॒जन॑नाय॒ तस्मा᳚त् । \newline
53. प्र॒जाना॒मिति॑ प्र - जाना᳚म् । \newline
54. प्र॒जन॑नाय॒ तस्मा॒त् तस्मा᳚त् प्र॒जन॑नाय प्र॒जन॑नाय॒ तस्मा᳚त् प्रा॒णम् प्रा॒णम् तस्मा᳚त् प्र॒जन॑नाय प्र॒जन॑नाय॒ तस्मा᳚त् प्रा॒णम् । \newline
55. प्र॒जन॑ना॒येति॑ प्र - जन॑नाय । \newline
56. तस्मा᳚त् प्रा॒णम् प्रा॒णम् तस्मा॒त् तस्मा᳚त् प्रा॒णम् प्र॒जाः प्र॒जाः प्रा॒णम् तस्मा॒त् तस्मा᳚त् प्रा॒णम् प्र॒जाः । \newline
57. प्रा॒णम् प्र॒जाः प्र॒जाः प्रा॒णम् प्रा॒णम् प्र॒जा अन्वनु॑ प्र॒जाः प्रा॒णम् प्रा॒णम् प्र॒जा अनु॑ । \newline
58. प्रा॒णमिति॑ प्र - अ॒नम् । \newline
59. प्र॒जा अन्वनु॑ प्र॒जाः प्र॒जा अनु॒ प्र प्राणु॑ प्र॒जाः प्र॒जा अनु॒ प्र । \newline
60. प्र॒जा इति॑ प्र - जाः । \newline
61. अनु॒ प्र प्राण्वनु॒ प्र जा॑यन्ते जायन्ते॒ प्राण्वनु॒ प्र जा॑यन्ते । \newline
62. प्र जा॑यन्ते जायन्ते॒ प्र प्र जा॑यन्ते दे॒वा दे॒वा जा॑यन्ते॒ प्र प्र जा॑यन्ते दे॒वाः । \newline
63. जा॒य॒न्ते॒ दे॒वा दे॒वा जा॑यन्ते जायन्ते दे॒वा वै वै दे॒वा जा॑यन्ते जायन्ते दे॒वा वै । \newline
64. दे॒वा वै वै दे॒वा दे॒वा वा इ॒त‌इ॑त इ॒त‌इ॑तो॒ वै दे॒वा दे॒वा वा इ॒त‌इ॑तः । \newline
65. वा इ॒त‌इ॑त इ॒त‌इ॑तो॒ वै वा इ॒त‌इ॑तः॒ पत्नीः॒ पत्नी॑ रि॒त‌इ॑तो॒ वै वा इ॒त‌इ॑तः॒ पत्नीः᳚ । \newline
66. इ॒त‌इ॑तः॒ पत्नीः॒ पत्नी॑ रि॒त‌इ॑त इ॒त‌इ॑तः॒ पत्नीः᳚ सुव॒र्गꣳ सु॑व॒र्गम् पत्नी॑ रि॒त‌इ॑त इ॒त‌इ॑तः॒ पत्नीः᳚ सुव॒र्गम् । \newline
67. इ॒त‌इ॑त॒ इती॒तः - इ॒तः॒ । \newline
68. पत्नीः᳚ सुव॒र्गꣳ सु॑व॒र्गम् पत्नीः॒ पत्नीः᳚ सुव॒र्गम् ॅलो॒कम् ॅलो॒कꣳ सु॑व॒र्गम् पत्नीः॒ पत्नीः᳚ सुव॒र्गम् ॅलो॒कम् । \newline
69. सु॒व॒र्गम् ॅलो॒कम् ॅलो॒कꣳ सु॑व॒र्गꣳ सु॑व॒र्गम् ॅलो॒क म॑जिगाꣳसन्-नजिगाꣳसन् ॅलो॒कꣳ सु॑व॒र्गꣳ सु॑व॒र्गम् ॅलो॒क म॑जिगाꣳसन्न् । \newline
70. सु॒व॒र्गमिति॑ सुवः - गम् । \newline
\pagebreak
\markright{ TS 6.5.8.2  \hfill https://www.vedavms.in \hfill}

\section{ TS 6.5.8.2 }

\textbf{TS 6.5.8.2 } \newline
\textbf{Samhita Paata} \newline

ॅलो॒कम॑जिगाꣳस॒न् ते सु॑व॒र्गं ॅलो॒कं न प्राजा॑न॒न् त ए॒तं पा᳚त्नीव॒तम॑पश्य॒न् तम॑गृह्णत॒ ततो॒ वै ते सु॑व॒र्गं ॅलो॒कं प्राजा॑न॒न्॒. यत् पा᳚त्नीव॒तो गृ॒ह्यते॑ सुव॒र्गस्य॑ लो॒कस्य॒ प्रज्ञा᳚त्यै॒ स सोमो॒ नाति॑ष्ठत स्त्री॒भ्यो गृ॒ह्यमा॑ण॒स्तं घृ॒तं ॅवज्रं॑ कृ॒त्वाऽघ्न॒न् तं निरि॑न्द्रियं भू॒तम॑गृह्ण॒न् तस्मा॒थ् स्त्रियो॒ निरि॑न्द्रिया॒ अदा॑यादी॒रपि॑ पा॒पात् पुꣳ॒॒स उप॑स्तितरं- [  ] \newline

\textbf{Pada Paata} \newline

लो॒कम् । अ॒जि॒गाꣳ॒॒स॒न्न् । ते । सु॒व॒र्गमिति॑ सुवः - गम् । लो॒कम् । न । प्रेति॑ । अ॒जा॒न॒न्न् । ते । ए॒तम् । पा॒त्नी॒व॒तमिति॑ पात्नी - व॒तम् । अ॒प॒श्य॒न्न् । तम् । अ॒गृ॒ह्ण॒त । ततः॑ । वै । ते । सु॒व॒र्गमिति॑ सुवः - गम् । लो॒कम् । प्रेति॑ । अ॒जा॒न॒न्न् । यत् । पा॒त्नी॒व॒त इति॑ पात्नी - व॒तः । गृ॒ह्यते᳚ । सु॒व॒र्गसेति॑ सुवः - गस्य॑ । लो॒कस्य॑ । प्रज्ञा᳚त्या॒ इति॒ प्र - ज्ञा॒त्यै॒ । सः । सोमः॑ । न । अ॒ति॒ष्ठ॒त॒ । स्त्री॒भ्यः । गृ॒ह्यमा॑णः । तम् । घृ॒तम् । वज्र᳚म् । कृ॒त्वा । अ॒घ्न॒न्न् । तम् । निरि॑न्द्रिय॒मिति॒ निः - इ॒न्द्रि॒य॒म् । भू॒तम् । अ॒गृ॒ह्ण॒न्न् । तस्मा᳚त् । स्त्रियः॑ । निरि॑न्द्रिया॒ इति॒ निः-इ॒न्द्रि॒याः॒ । अदा॑यादी॒रित्यदा॑य-अ॒दीः॒ । अपीति॑ । पा॒पात् । पुꣳ॒॒सः । उप॑स्तितर॒मित्युप॑स्ति - त॒र॒म् ।  \newline


\textbf{Krama Paata} \newline

लो॒कम॑जिगाꣳसन्न् । अ॒जि॒गाꣳ॒॒स॒न् ते । ते सु॑व॒र्गम् । सु॒व॒र्गम् ॅलो॒कम् । सु॒व॒र्गमिति॑ सुवः - गम् । लो॒कम् न । न प्र । प्राजा॑नन्न् । अ॒जा॒न॒न् ते । त ए॒तम् । ए॒तम् पा᳚त्नीव॒तम् । पा॒त्नी॒व॒तम॑पश्यन्न् । पा॒त्नी॒व॒तमिति॑ पात्नी - व॒तम् । अ॒प॒श्य॒न् तम् । तम॑गृह्णत । अ॒गृ॒ह्ण॒त॒ ततः॑ । ततो॒ वै । वै ते । ते सु॑व॒र्गम् । सु॒व॒र्गम् ॅलो॒कम् । सु॒व॒र्गमिति॑ सुवः - गम् । लो॒कम् प्र । प्राजा॑नन्न् । अ॒जा॒न॒न्॒. यत् । यत् पा᳚त्नीव॒तः । पा॒त्नी॒व॒तो गृ॒ह्यते᳚ । पा॒त्नी॒व॒त इति॑ पात्नी - व॒तः । गृ॒ह्यते॑ सुव॒र्गस्य॑ । सु॒व॒र्गस्य॑ लो॒कस्य॑ । सु॒व॒र्गस्येति॑ सुवः - गस्य॑ । लो॒कस्य॒ प्रज्ञा᳚त्यै । प्रज्ञा᳚त्यै॒ सः । प्रज्ञा᳚त्या॒ इति॒ प्र - ज्ञा॒त्यै॒ । स सोमः॑ । सोमो॒ न । नाति॑ष्ठत । अ॒ति॒ष्ठ॒त॒ स्त्री॒भ्यः । स्त्री॒भ्यो गृ॒ह्यमा॑णः । गृ॒ह्यमा॑ण॒स्तम् । तम् घृ॒तम् । घृ॒तम् ॅवज्र᳚म् । वज्र॑म् कृ॒त्वा । कृ॒त्वाऽघ्नन्न्॑ । अ॒घ्न॒न् तम् । तम् निरि॑न्द्रियम् । निरि॑न्द्रियम् भू॒तम् । निरि॑न्द्रिय॒मिति॒ निः - इ॒न्द्रि॒य॒म् । भू॒तम॑गृह्णन्न् । अ॒गृ॒ह्ण॒न् तस्मा᳚त् । तस्मा॒थ् स्त्रियः॑ । स्त्रियो॒ निरि॑न्द्रियाः । निरि॑न्द्रिया॒ अदा॑यादीः । निरि॑न्द्रिया॒ इति॒ निः - इ॒न्द्रि॒याः॒ । अदा॑यादी॒रपि॑ । अदा॑यादी॒रित्यदा॑य - अ॒दीः॒ । अपि॑ पा॒पात् । पा॒पात् पुꣳ॒॒सः । पुꣳ॒॒स उप॑स्तितरम् । उप॑स्तितरम् ॅवदन्ति । उप॑स्तितर॒मित्युप॑स्ति - त॒र॒म् \newline

\textbf{Jatai Paata} \newline

1. लो॒क म॑जिगाꣳसन् नजिगाꣳसन् ॅलो॒कम् ॅलो॒क म॑जिगाꣳसन्न् । \newline
2. अ॒जि॒गाꣳ॒॒स॒न् ते ते॑ ऽजिगाꣳसन् नजिगाꣳस॒न् ते । \newline
3. ते सु॑व॒र्गꣳ सु॑व॒र्गम् ते ते सु॑व॒र्गम् । \newline
4. सु॒व॒र्गम् ॅलो॒कम् ॅलो॒कꣳ सु॑व॒र्गꣳ सु॑व॒र्गम् ॅलो॒कम् । \newline
5. सु॒व॒र्गमिति॑ सुवः - गम् । \newline
6. लो॒कन् न न लो॒कम् ॅलो॒कन् न । \newline
7. न प्र प्र ण न प्र । \newline
8. प्राजा॑नन् नजान॒न् प्र प्राजा॑नन्न् । \newline
9. अ॒जा॒न॒न् ते ते॑ ऽजानन् नजान॒न् ते । \newline
10. त ए॒त मे॒तम् ते त ए॒तम् । \newline
11. ए॒तम् पा᳚त्नीव॒तम् पा᳚त्नीव॒त मे॒त मे॒तम् पा᳚त्नीव॒तम् । \newline
12. पा॒त्नी॒व॒त म॑पश्यन् नपश्यन् पात्नीव॒तम् पा᳚त्नीव॒त म॑पश्यन्न् । \newline
13. पा॒त्नी॒व॒तमिति॑ पात्नी - व॒तम् । \newline
14. अ॒प॒श्य॒न् तम् त म॑पश्यन् नपश्य॒न् तम् । \newline
15. त म॑गृह्ण॒ता गृ॑ह्ण॒त तम् त म॑गृह्ण॒त । \newline
16. अ॒गृ॒ह्ण॒त तत॒ स्ततो॑ ऽगृह्ण॒ता गृ॑ह्ण॒त ततः॑ । \newline
17. ततो॒ वै वै तत॒ स्ततो॒ वै । \newline
18. वै ते ते वै वै ते । \newline
19. ते सु॑व॒र्गꣳ सु॑व॒र्गम् ते ते सु॑व॒र्गम् । \newline
20. सु॒व॒र्गम् ॅलो॒कम् ॅलो॒कꣳ सु॑व॒र्गꣳ सु॑व॒र्गम् ॅलो॒कम् । \newline
21. सु॒व॒र्गमिति॑ सुवः - गम् । \newline
22. लो॒कम् प्र प्र लो॒कम् ॅलो॒कम् प्र । \newline
23. प्राजा॑नन् नजान॒न् प्र प्राजा॑नन्न् । \newline
24. अ॒जा॒न॒न्॒. यद् यद॑जानन् नजान॒न्॒. यत् । \newline
25. यत् पा᳚त्नीव॒तः पा᳚त्नीव॒तो यद् यत् पा᳚त्नीव॒तः । \newline
26. पा॒त्नी॒व॒तो गृ॒ह्यते॑ गृ॒ह्यते॑ पात्नीव॒तः पा᳚त्नीव॒तो गृ॒ह्यते᳚ । \newline
27. पा॒त्नी॒व॒त इति॑ पात्नी - व॒तः । \newline
28. गृ॒ह्यते॑ सुव॒र्गस्य॑ सुव॒र्गस्य॑ गृ॒ह्यते॑ गृ॒ह्यते॑ सुव॒र्गस्य॑ । \newline
29. सु॒व॒र्गस्य॑ लो॒कस्य॑ लो॒कस्य॑ सुव॒र्गस्य॑ सुव॒र्गस्य॑ लो॒कस्य॑ । \newline
30. सु॒व॒र्गसेति॑ सुवः - गस्य॑ । \newline
31. लो॒कस्य॒ प्रज्ञा᳚त्यै॒ प्रज्ञा᳚त्यै लो॒कस्य॑ लो॒कस्य॒ प्रज्ञा᳚त्यै । \newline
32. प्रज्ञा᳚त्यै॒ स स प्रज्ञा᳚त्यै॒ प्रज्ञा᳚त्यै॒ सः । \newline
33. प्रज्ञा᳚त्या॒ इति॒ प्र - ज्ञा॒त्यै॒ । \newline
34. स सोमः॒ सोमः॒ स स सोमः॑ । \newline
35. सोमो॒ न न सोमः॒ सोमो॒ न । \newline
36. नाति॑ष्ठता तिष्ठत॒ न नाति॑ष्ठत । \newline
37. अ॒ति॒ष्ठ॒त॒ स्त्री॒भ्यः स्त्री॒भ्यो॑ ऽतिष्ठता तिष्ठत स्त्री॒भ्यः । \newline
38. स्त्री॒भ्यो गृ॒ह्यमा॑णो गृ॒ह्यमा॑णः स्त्री॒भ्यः स्त्री॒भ्यो गृ॒ह्यमा॑णः । \newline
39. गृ॒ह्यमा॑ण॒ स्तम् तम् गृ॒ह्यमा॑णो गृ॒ह्यमा॑ण॒ स्तम् । \newline
40. तम् घृ॒तम् घृ॒तम् तम् तम् घृ॒तम् । \newline
41. घृ॒तं ॅवज्रं॒ ॅवज्र॑म् घृ॒तम् घृ॒तं ॅवज्र᳚म् । \newline
42. वज्र॑म् कृ॒त्वा कृ॒त्वा वज्रं॒ ॅवज्र॑म् कृ॒त्वा । \newline
43. कृ॒त्वा ऽघ्न॑न् नघ्नन् कृ॒त्वा कृ॒त्वा ऽघ्नन्न्॑ । \newline
44. अ॒घ्न॒न् तम् त म॑घ्नन् नघ्न॒न् तम् । \newline
45. तन् निरि॑न्द्रिय॒न् निरि॑न्द्रिय॒म् तम् तन् निरि॑न्द्रियम् । \newline
46. निरि॑न्द्रियम् भू॒तम् भू॒तन् निरि॑न्द्रिय॒न् निरि॑न्द्रियम् भू॒तम् । \newline
47. निरि॑न्द्रिय॒मिति॒ निः - इ॒न्द्रि॒य॒म् । \newline
48. भू॒त म॑गृह्णन् नगृह्णन् भू॒तम् भू॒त म॑गृह्णन्न् । \newline
49. अ॒गृ॒ह्ण॒न् तस्मा॒त् तस्मा॑ दगृह्णन् नगृह्ण॒न् तस्मा᳚त् । \newline
50. तस्मा॒थ् स्त्रियः॒ स्त्रिय॒ स्तस्मा॒त् तस्मा॒थ् स्त्रियः॑ । \newline
51. स्त्रियो॒ निरि॑न्द्रिया॒ निरि॑न्द्रियाः॒ स्त्रियः॒ स्त्रियो॒ निरि॑न्द्रियाः । \newline
52. निरि॑न्द्रिया॒ अदा॑यादी॒ रदा॑यादी॒र् निरि॑न्द्रिया॒ निरि॑न्द्रिया॒ अदा॑यादीः । \newline
53. निरि॑न्द्रिया॒ इति॒ निः - इ॒न्द्रि॒याः॒ । \newline
54. अदा॑यादी॒ रप्यप्य दा॑यादी॒ रदा॑यादी॒ रपि॑ । \newline
55. अदा॑यादी॒रित्यदा॑य - अ॒दीः॒ । \newline
56. अपि॑ पा॒पात् पा॒पा दप्यपि॑ पा॒पात् । \newline
57. पा॒पात् पुꣳ॒॒सः पुꣳ॒॒सः पा॒पात् पा॒पात् पुꣳ॒॒सः । \newline
58. पुꣳ॒॒स उप॑स्तितर॒ मुप॑स्तितरम् पुꣳ॒॒सः पुꣳ॒॒स उप॑स्तितरम् । \newline
59. उप॑स्तितरं ॅवदन्ति वद॒न् त्युप॑स्तितर॒ मुप॑स्तितरं ॅवदन्ति । \newline
60. उप॑स्तितर॒मित्युप॑स्ति - त॒र॒म् । \newline

\textbf{Ghana Paata } \newline

1. लो॒क म॑जिगाꣳसन्-नजिगाꣳसन् ॅलो॒कम् ॅलो॒क म॑जिगाꣳस॒न् ते ते॑ ऽजिगाꣳसन् ॅलो॒कम् ॅलो॒क म॑जिगाꣳस॒न् ते । \newline
2. अ॒जि॒गाꣳ॒॒स॒न् ते ते॑ ऽजिगाꣳसन्-नजिगाꣳस॒न् ते सु॑व॒र्गꣳ सु॑व॒र्गम् ते॑ ऽजिगाꣳसन्-नजिगाꣳस॒न् ते सु॑व॒र्गम् । \newline
3. ते सु॑व॒र्गꣳ सु॑व॒र्गम् ते ते सु॑व॒र्गम् ॅलो॒कम् ॅलो॒कꣳ सु॑व॒र्गम् ते ते सु॑व॒र्गम् ॅलो॒कम् । \newline
4. सु॒व॒र्गम् ॅलो॒कम् ॅलो॒कꣳ सु॑व॒र्गꣳ सु॑व॒र्गम् ॅलो॒कन्न न लो॒कꣳ सु॑व॒र्गꣳ सु॑व॒र्गम् ॅलो॒कन्न । \newline
5. सु॒व॒र्गमिति॑ सुवः - गम् । \newline
6. लो॒कन् न न लो॒कम् ॅलो॒कन् न प्र प्र ण लो॒कम् ॅलो॒कन् न प्र । \newline
7. न प्र प्र ण न प्राजा॑नन्-नजान॒न् प्र ण न प्राजा॑नन्न् । \newline
8. प्राजा॑नन्-नजान॒न् प्र प्राजा॑न॒न् ते ते॑ ऽजान॒न् प्र प्राजा॑न॒न् ते । \newline
9. अ॒जा॒न॒न् ते ते॑ ऽजानन्-नजान॒न् त ए॒त मे॒तम् ते॑ ऽजानन्-नजान॒न् त ए॒तम् । \newline
10. त ए॒त मे॒तम् ते त ए॒तम् पा᳚त्नीव॒तम् पा᳚त्नीव॒त मे॒तम् ते त ए॒तम् पा᳚त्नीव॒तम् । \newline
11. ए॒तम् पा᳚त्नीव॒तम् पा᳚त्नीव॒त मे॒त मे॒तम् पा᳚त्नीव॒त म॑पश्यन्-नपश्यन् पात्नीव॒त मे॒त मे॒तम् पा᳚त्नीव॒त म॑पश्यन्न् । \newline
12. पा॒त्नी॒व॒त म॑पश्यन्-नपश्यन् पात्नीव॒तम् पा᳚त्नीव॒त म॑पश्य॒न् तम् त म॑पश्यन् पात्नीव॒तम् पा᳚त्नीव॒त म॑पश्य॒न् तम् । \newline
13. पा॒त्नी॒व॒तमिति॑ पात्नी - व॒तम् । \newline
14. अ॒प॒श्य॒न् तम् त म॑पश्यन्-नपश्य॒न् त म॑गृह्ण॒ता गृ॑ह्ण॒त त म॑पश्यन्-नपश्य॒न् त म॑गृह्ण॒त । \newline
15. त म॑गृह्ण॒ता गृ॑ह्ण॒त तम् त म॑गृह्ण॒त तत॒ स्ततो॑ ऽगृह्ण॒त तम् त म॑गृह्ण॒त ततः॑ । \newline
16. अ॒गृ॒ह्ण॒त तत॒ स्ततो॑ ऽगृह्ण॒ता गृ॑ह्ण॒त ततो॒ वै वै ततो॑ ऽगृह्ण॒ता गृ॑ह्ण॒त ततो॒ वै । \newline
17. ततो॒ वै वै तत॒ स्ततो॒ वै ते ते वै तत॒ स्ततो॒ वै ते । \newline
18. वै ते ते वै वै ते सु॑व॒र्गꣳ सु॑व॒र्गम् ते वै वै ते सु॑व॒र्गम् । \newline
19. ते सु॑व॒र्गꣳ सु॑व॒र्गम् ते ते सु॑व॒र्गम् ॅलो॒कम् ॅलो॒कꣳ सु॑व॒र्गम् ते ते सु॑व॒र्गम् ॅलो॒कम् । \newline
20. सु॒व॒र्गम् ॅलो॒कम् ॅलो॒कꣳ सु॑व॒र्गꣳ सु॑व॒र्गम् ॅलो॒कम् प्र प्र लो॒कꣳ सु॑व॒र्गꣳ सु॑व॒र्गम् ॅलो॒कम् प्र । \newline
21. सु॒व॒र्गमिति॑ सुवः - गम् । \newline
22. लो॒कम् प्र प्र लो॒कम् ॅलो॒कम् प्राजा॑नन्-नजान॒न् प्र लो॒कम् ॅलो॒कम् प्राजा॑नन्न् । \newline
23. प्राजा॑नन्-नजान॒न् प्र प्राजा॑न॒न्॒. यद् यद॑जान॒न् प्र प्राजा॑न॒न्॒. यत् । \newline
24. अ॒जा॒न॒न्॒. यद् यद॑जानन्-नजान॒न्॒. यत् पा᳚त्नीव॒तः पा᳚त्नीव॒तो यद॑जानन्-नजान॒न्॒. यत् पा᳚त्नीव॒तः । \newline
25. यत् पा᳚त्नीव॒तः पा᳚त्नीव॒तो यद् यत् पा᳚त्नीव॒तो गृ॒ह्यते॑ गृ॒ह्यते॑ पात्नीव॒तो यद् यत् पा᳚त्नीव॒तो गृ॒ह्यते᳚ । \newline
26. पा॒त्नी॒व॒तो गृ॒ह्यते॑ गृ॒ह्यते॑ पात्नीव॒तः पा᳚त्नीव॒तो गृ॒ह्यते॑ सुव॒र्गस्य॑ सुव॒र्गस्य॑ गृ॒ह्यते॑ पात्नीव॒तः पा᳚त्नीव॒तो गृ॒ह्यते॑ सुव॒र्गस्य॑ । \newline
27. पा॒त्नी॒व॒त इति॑ पात्नी - व॒तः । \newline
28. गृ॒ह्यते॑ सुव॒र्गस्य॑ सुव॒र्गस्य॑ गृ॒ह्यते॑ गृ॒ह्यते॑ सुव॒र्गस्य॑ लो॒कस्य॑ लो॒कस्य॑ सुव॒र्गस्य॑ गृ॒ह्यते॑ गृ॒ह्यते॑ सुव॒र्गस्य॑ लो॒कस्य॑ । \newline
29. सु॒व॒र्गस्य॑ लो॒कस्य॑ लो॒कस्य॑ सुव॒र्गस्य॑ सुव॒र्गस्य॑ लो॒कस्य॒ प्रज्ञा᳚त्यै॒ प्रज्ञा᳚त्यै लो॒कस्य॑ सुव॒र्गस्य॑ सुव॒र्गस्य॑ लो॒कस्य॒ प्रज्ञा᳚त्यै । \newline
30. सु॒व॒र्गसेति॑ सुवः - गस्य॑ । \newline
31. लो॒कस्य॒ प्रज्ञा᳚त्यै॒ प्रज्ञा᳚त्यै लो॒कस्य॑ लो॒कस्य॒ प्रज्ञा᳚त्यै॒ स स प्रज्ञा᳚त्यै लो॒कस्य॑ लो॒कस्य॒ प्रज्ञा᳚त्यै॒ सः । \newline
32. प्रज्ञा᳚त्यै॒ स स प्रज्ञा᳚त्यै॒ प्रज्ञा᳚त्यै॒ स सोमः॒ सोमः॒ स प्रज्ञा᳚त्यै॒ प्रज्ञा᳚त्यै॒ स सोमः॑ । \newline
33. प्रज्ञा᳚त्या॒ इति॒ प्र - ज्ञा॒त्यै॒ । \newline
34. स सोमः॒ सोमः॒ स स सोमो॒ न न सोमः॒ स स सोमो॒ न । \newline
35. सोमो॒ न न सोमः॒ सोमो॒ नाति॑ष्ठता तिष्ठत॒ न सोमः॒ सोमो॒ नाति॑ष्ठत । \newline
36. नाति॑ष्ठता तिष्ठत॒ न नाति॑ष्ठत स्त्री॒भ्यः स्त्री॒भ्यो॑ ऽतिष्ठत॒ न नाति॑ष्ठत स्त्री॒भ्यः । \newline
37. अ॒ति॒ष्ठ॒त॒ स्त्री॒भ्यः स्त्री॒भ्यो॑ ऽतिष्ठता तिष्ठत स्त्री॒भ्यो गृ॒ह्यमा॑णो गृ॒ह्यमा॑णः स्त्री॒भ्यो॑ ऽतिष्ठता तिष्ठत स्त्री॒भ्यो गृ॒ह्यमा॑णः । \newline
38. स्त्री॒भ्यो गृ॒ह्यमा॑णो गृ॒ह्यमा॑णः स्त्री॒भ्यः स्त्री॒भ्यो गृ॒ह्यमा॑ण॒ स्तम् तम् गृ॒ह्यमा॑णः स्त्री॒भ्यः स्त्री॒भ्यो गृ॒ह्यमा॑ण॒ स्तम् । \newline
39. गृ॒ह्यमा॑ण॒ स्तम् तम् गृ॒ह्यमा॑णो गृ॒ह्यमा॑ण॒ स्तम् घृ॒तम् घृ॒तम् तम् गृ॒ह्यमा॑णो गृ॒ह्यमा॑ण॒ स्तम् घृ॒तम् । \newline
40. तम् घृ॒तम् घृ॒तम् तम् तम् घृ॒तं ॅवज्रं॒ ॅवज्र॑म् घृ॒तम् तम् तम् घृ॒तं ॅवज्र᳚म् । \newline
41. घृ॒तं ॅवज्रं॒ ॅवज्र॑म् घृ॒तम् घृ॒तं ॅवज्र॑म् कृ॒त्वा कृ॒त्वा वज्र॑म् घृ॒तम् घृ॒तं ॅवज्र॑म् कृ॒त्वा । \newline
42. वज्र॑म् कृ॒त्वा कृ॒त्वा वज्रं॒ ॅवज्र॑म् कृ॒त्वा ऽघ्न॑न्-नघ्नन् कृ॒त्वा वज्रं॒ ॅवज्र॑म् कृ॒त्वा ऽघ्नन्न्॑ । \newline
43. कृ॒त्वा ऽघ्न॑न्-नघ्नन् कृ॒त्वा कृ॒त्वा ऽघ्न॒न् तम् त म॑घ्नन् कृ॒त्वा कृ॒त्वा ऽघ्न॒न् तम् । \newline
44. अ॒घ्न॒न् तम् त म॑घ्नन्-नघ्न॒न् तन् निरि॑न्द्रिय॒म् निरि॑न्द्रिय॒म् त म॑घ्नन्-नघ्न॒न् तन् निरि॑न्द्रियम् । \newline
45. तन् निरि॑न्द्रिय॒म् निरि॑न्द्रिय॒म् तम् तन् निरि॑न्द्रियम् भू॒तम् भू॒तम् निरि॑न्द्रिय॒म् तम् तन् निरि॑न्द्रियम् भू॒तम् । \newline
46. निरि॑न्द्रियम् भू॒तम् भू॒तम् निरि॑न्द्रिय॒म् निरि॑न्द्रियम् भू॒त म॑गृह्णन्-नगृह्णन् भू॒तम् निरि॑न्द्रिय॒म् निरि॑न्द्रियम् भू॒त म॑गृह्णन्न् । \newline
47. निरि॑न्द्रिय॒मिति॒ निः - इ॒न्द्रि॒य॒म् । \newline
48. भू॒त म॑गृह्णन्-नगृह्णन् भू॒तम् भू॒त म॑गृह्ण॒न् तस्मा॒त् तस्मा॑ दगृह्णन् भू॒तम् भू॒त म॑गृह्ण॒न् तस्मा᳚त् । \newline
49. अ॒गृ॒ह्ण॒न् तस्मा॒त् तस्मा॑ दगृह्णन्-नगृह्ण॒न् तस्मा॒थ् स्त्रियः॒ स्त्रिय॒ स्तस्मा॑ दगृह्णन्-नगृह्ण॒न् तस्मा॒थ् स्त्रियः॑ । \newline
50. तस्मा॒थ् स्त्रियः॒ स्त्रिय॒ स्तस्मा॒त् तस्मा॒थ् स्त्रियो॒ निरि॑न्द्रिया॒ निरि॑न्द्रियाः॒ स्त्रिय॒ स्तस्मा॒त् तस्मा॒थ् स्त्रियो॒ निरि॑न्द्रियाः । \newline
51. स्त्रियो॒ निरि॑न्द्रिया॒ निरि॑न्द्रियाः॒ स्त्रियः॒ स्त्रियो॒ निरि॑न्द्रिया॒ अदा॑यादी॒ रदा॑यादी॒र् निरि॑न्द्रियाः॒ स्त्रियः॒ स्त्रियो॒ निरि॑न्द्रिया॒ अदा॑यादीः । \newline
52. निरि॑न्द्रिया॒ अदा॑यादी॒ रदा॑यादी॒र् निरि॑न्द्रिया॒ निरि॑न्द्रिया॒ अदा॑यादी॒ रप्य प्यदा॑यादी॒र् निरि॑न्द्रिया॒ निरि॑न्द्रिया॒ अदा॑यादी॒ रपि॑ । \newline
53. निरि॑न्द्रिया॒ इति॒ निः - इ॒न्द्रि॒याः॒ । \newline
54. अदा॑यादी॒ रप्य प्यदा॑यादी॒ रदा॑यादी॒ रपि॑ पा॒पात् पा॒पा दप्यदा॑यादी॒ रदा॑यादी॒ रपि॑ पा॒पात् । \newline
55. अदा॑यादी॒रित्यदा॑य - अ॒दीः॒ । \newline
56. अपि॑ पा॒पात् पा॒पा दप्यपि॑ पा॒पात् पुꣳ॒॒सः पुꣳ॒॒सः पा॒पा दप्यपि॑ पा॒पात् पुꣳ॒॒सः । \newline
57. पा॒पात् पुꣳ॒॒सः पुꣳ॒॒सः पा॒पात् पा॒पात् पुꣳ॒॒स उप॑स्तितर॒ मुप॑स्तितरम् पुꣳ॒॒सः पा॒पात् पा॒पात् पुꣳ॒॒स उप॑स्तितरम् । \newline
58. पुꣳ॒॒स उप॑स्तितर॒ मुप॑स्तितरम् पुꣳ॒॒सः पुꣳ॒॒स उप॑स्तितरं ॅवदन्ति वद॒ न्त्युप॑स्तितरम् पुꣳ॒॒सः पुꣳ॒॒स उप॑स्तितरं ॅवदन्ति । \newline
59. उप॑स्तितरं ॅवदन्ति वद॒ न्त्युप॑स्तितर॒ मुप॑स्तितरं ॅवदन्ति॒ यद् यद् व॑द॒ न्त्युप॑स्तितर॒ मुप॑स्तितरं ॅवदन्ति॒ यत् । \newline
60. उप॑स्तितर॒मित्युप॑स्ति - त॒र॒म् । \newline
\pagebreak
\markright{ TS 6.5.8.3  \hfill https://www.vedavms.in \hfill}

\section{ TS 6.5.8.3 }

\textbf{TS 6.5.8.3 } \newline
\textbf{Samhita Paata} \newline

ॅवदन्ति॒ यद्-घृ॒तेन॑ पात्नीव॒तꣳ श्री॒णाति॒ वज्रे॑णै॒वैनं॒ ॅवशे॑ कृ॒त्वा गृ॑ह्णा-त्युपया॒मगृ॑हीतो॒ऽसीत्या॑हे॒यं ॅवा उ॑पया॒मस्तस्मा॑दि॒मां प्र॒जा अनु॒ प्र जा॑यन्ते॒ बृह॒स्पति॑सुतस्य त॒ इत्या॑ह॒ ब्रह्म॒ वै दे॒वानां॒ बृह॒स्पति॒र्ब्रह्म॑णै॒वास्मै᳚ प्र॒जाः प्र ज॑नयतीन्दो॒ इत्या॑ह॒ रेतो॒ वा इन्दू॒ रेत॑ ए॒व तद्-द॑धातीन्द्रियाव॒ इ- [  ] \newline

\textbf{Pada Paata} \newline

व॒द॒न्ति॒ । यत् । घृ॒तेन॑ । पा॒त्नी॒व॒तमिति॑ पात्नी - व॒तम् । श्री॒णाति॑ । वज्रे॑ण । ए॒व । ए॒न॒म् । वशे᳚ । कृ॒त्वा । गृ॒ह्णा॒ति॒ । उ॒प॒या॒मगृ॑हीत॒ इत्यु॑पया॒म - गृ॒ही॒तः॒ । अ॒सि॒ । इति॑ । आ॒ह॒ । इ॒यम् । वै । उ॒प॒या॒म इत्यु॑प - या॒मः । तस्मा᳚त् । इ॒माम् । प्र॒जा इति॑ प्र - जाः । अनु॑ । प्रेति॑ । जा॒य॒न्ते॒ । बृह॒स्पति॑सुत॒स्येति॒ बृह॒स्पति॑-सु॒त॒स्य॒ । ते॒ । इति॑ । आ॒ह॒ । ब्रह्म॑ । वै । दे॒वाना᳚म् । बृह॒स्पतिः॑ । ब्रह्म॑णा । ए॒व । अ॒स्मै॒ । प्र॒जा इति॑ प्र - जाः । प्रेति॑ । ज॒न॒य॒ति॒ । इ॒न्दो॒ इति॑ । इति॑ । आ॒ह॒ । रेतः॑ । वै । इन्दुः॑ । रेतः॑ । ए॒व । तत् । द॒धा॒ति॒ । इ॒न्द्रि॒या॒व॒ इती᳚न्द्रिय - वः॒ । इति॑ ।  \newline


\textbf{Krama Paata} \newline

व॒द॒न्ति॒ यत् । यद् घृ॒तेन॑ । घृ॒तेन॑ पात्नीव॒तम् । पा॒त्नी॒व॒तꣳ श्री॒णाति॑ । पा॒त्नी॒व॒तमिति॑ पात्नी - व॒तम् । श्री॒णाति॒ वज्रे॑ण । वज्रे॑णै॒व । ए॒वैन᳚म् । ए॒न॒म् ॅवशे᳚ । वशे॑ कृ॒त्वा । कृ॒त्वा गृ॑ह्णाति । गृ॒ह्णा॒त्यु॒प॒या॒मगृ॑हीतः । उ॒प॒या॒मगृ॑हीतोऽसि । उ॒प॒या॒मगृ॑हीत॒ इत्यु॑पया॒म - गृ॒ही॒तः॒ । अ॒सीति॑ । इत्या॑ह । आ॒हे॒यम् । इ॒यम् ॅवै । वा उ॑पया॒मः । उ॒प॒या॒मस्तस्मा᳚त् । उ॒प॒या॒म इत्यु॑प - या॒मः । तस्मा॑दि॒माम् । इ॒माम् प्र॒जाः । प्र॒जा अनु॑ । प्र॒जा इति॑ प्र - जाः । अनु॒ प्र । प्र जा॑यन्ते । जा॒य॒न्ते॒ बृह॒स्पति॑सुतस्य । बृह॒स्पति॑सुतस्य ते । बृह॒स्पति॑सुत॒स्येति॒ बृह॒स्पति॑ - सु॒त॒स्य॒ । त॒ इति॑ । इत्या॑ह । आ॒ह॒ ब्रह्म॑ । ब्रह्म॒ वै । वै दे॒वाना᳚म् । दे॒वाना॒म् बृह॒स्पतिः॑ । बृह॒स्पति॒र् ब्रह्म॑णा । ब्रह्म॑णै॒व । ए॒वास्मै᳚ । अ॒स्मै॒ प्र॒जाः । प्र॒जाः प्र । प्र॒जा इति॑ प्र - जाः । प्र ज॑नयति । ज॒न॒य॒ती॒न्दो॒ । इ॒न्दो॒ इति॑ । 
इ॒न्दो॒ इती᳚न्दो । इत्या॑ह । आ॒ह॒ रेतः॑ । रेतो॒ वै । वा इन्दुः॑ । इन्दू॒ रेतः॑ । रेत॑ ए॒व । ए॒व तत् । तद् द॑धाति । द॒धा॒ती॒न्द्रि॒या॒वः॒ । इ॒न्द्रि॒या॒व॒ इति॑। इ॒न्द्रि॒या॒व॒ इत् ई᳚न्द्रिय - वः॒ । इत्या॑ह \newline

\textbf{Jatai Paata} \newline

1. व॒द॒न्ति॒ यद् यद् व॑दन्ति वदन्ति॒ यत् । \newline
2. यद् घृ॒तेन॑ घृ॒तेन॒ यद् यद् घृ॒तेन॑ । \newline
3. घृ॒तेन॑ पात्नीव॒तम् पा᳚त्नीव॒तम् घृ॒तेन॑ घृ॒तेन॑ पात्नीव॒तम् । \newline
4. पा॒त्नी॒व॒तꣳ श्री॒णाति॑ श्री॒णाति॑ पात्नीव॒तम् पा᳚त्नीव॒तꣳ श्री॒णाति॑ । \newline
5. पा॒त्नी॒व॒तमिति॑ पात्नी - व॒तम् । \newline
6. श्री॒णाति॒ वज्रे॑ण॒ वज्रे॑ण श्री॒णाति॑ श्री॒णाति॒ वज्रे॑ण । \newline
7. वज्रे॑णै॒ वैव वज्रे॑ण॒ वज्रे॑णै॒व । \newline
8. ए॒वैन॑ मेन मे॒वै वैन᳚म् । \newline
9. ए॒नं॒ ॅवशे॒ वश॑ एन मेनं॒ ॅवशे᳚ । \newline
10. वशे॑ कृ॒त्वा कृ॒त्वा वशे॒ वशे॑ कृ॒त्वा । \newline
11. कृ॒त्वा गृ॑ह्णाति गृह्णाति कृ॒त्वा कृ॒त्वा गृ॑ह्णाति । \newline
12. गृ॒ह्णा॒ त्यु॒प॒या॒मगृ॑हीत उपया॒मगृ॑हीतो गृह्णाति गृह्णा त्युपया॒मगृ॑हीतः । \newline
13. उ॒प॒या॒मगृ॑हीतो ऽस्यस्यु पया॒मगृ॑हीत उपया॒मगृ॑हीतो ऽसि । \newline
14. उ॒प॒या॒मगृ॑हीत॒ इत्यु॑पया॒म - गृ॒ही॒तः॒ । \newline
15. अ॒सी तीत्य॑स्य॒ सीति॑ । \newline
16. इत्या॑हा॒हे तीत्या॑ह । \newline
17. आ॒हे॒य मि॒य मा॑हाहे॒ यम् । \newline
18. इ॒यं ॅवै वा इ॒य मि॒यं ॅवै । \newline
19. वा उ॑पया॒म उ॑पया॒मो वै वा उ॑पया॒मः । \newline
20. उ॒प॒या॒म स्तस्मा॒त् तस्मा॑ दुपया॒म उ॑पया॒म स्तस्मा᳚त् । \newline
21. उ॒प॒या॒म इत्यु॑प - या॒मः । \newline
22. तस्मा॑ दि॒मा मि॒माम् तस्मा॒त् तस्मा॑ दि॒माम् । \newline
23. इ॒माम् प्र॒जाः प्र॒जा इ॒मा मि॒माम् प्र॒जाः । \newline
24. प्र॒जा अन्वनु॑ प्र॒जाः प्र॒जा अनु॑ । \newline
25. प्र॒जा इति॑ प्र - जाः । \newline
26. अनु॒ प्र प्राण्वनु॒ प्र । \newline
27. प्र जा॑यन्ते जायन्ते॒ प्र प्र जा॑यन्ते । \newline
28. जा॒य॒न्ते॒ बृह॒स्पति॑सुतस्य॒ बृह॒स्पति॑सुतस्य जायन्ते जायन्ते॒ बृह॒स्पति॑सुतस्य । \newline
29. बृह॒स्पति॑सुतस्य ते ते॒ बृह॒स्पति॑सुतस्य॒ बृह॒स्पति॑सुतस्य ते । \newline
30. बृह॒स्पति॑सुत॒स्येति॒ बृह॒स्पति॑ - सु॒त॒स्य॒ । \newline
31. त॒ इतीति॑ ते त॒ इति॑ । \newline
32. इत्या॑हा॒हे तीत्या॑ह । \newline
33. आ॒ह॒ ब्रह्म॒ ब्रह्मा॑ हाह॒ ब्रह्म॑ । \newline
34. ब्रह्म॒ वै वै ब्रह्म॒ ब्रह्म॒ वै । \newline
35. वै दे॒वाना᳚म् दे॒वानां॒ ॅवै वै दे॒वाना᳚म् । \newline
36. दे॒वाना॒म् बृह॒स्पति॒र् बृह॒स्पति॑र् दे॒वाना᳚म् दे॒वाना॒म् बृह॒स्पतिः॑ । \newline
37. बृह॒स्पति॒र् ब्रह्म॑णा॒ ब्रह्म॑णा॒ बृह॒स्पति॒र् बृह॒स्पति॒र् ब्रह्म॑णा । \newline
38. ब्रह्म॑ णै॒वैव ब्रह्म॑णा॒ ब्रह्म॑णै॒व । \newline
39. ए॒वास्मा॑ अस्मा ए॒वै वास्मै᳚ । \newline
40. अ॒स्मै॒ प्र॒जाः प्र॒जा अ॑स्मा अस्मै प्र॒जाः । \newline
41. प्र॒जाः प्र प्र प्र॒जाः प्र॒जाः प्र । \newline
42. प्र॒जा इति॑ प्र - जाः । \newline
43. प्र ज॑नयति जनयति॒ प्र प्र ज॑नयति । \newline
44. ज॒न॒य॒ ती॒न्दो॒ इ॒न्दो॒ ज॒न॒य॒ति॒ ज॒न॒य॒ ती॒न्दो॒ । \newline
45. इ॒न्दो॒ इतीती᳚न्दो इन्दो॒ इति॑ । \newline
46. इ॒न्दो॒ इती᳚न्दो । \newline
47. इत्या॑हा॒हे तीत्या॑ह । \newline
48. आ॒ह॒ रेतो॒ रेत॑ आहाह॒ रेतः॑ । \newline
49. रेतो॒ वै वै रेतो॒ रेतो॒ वै । \newline
50. वा इन्दु॒ रिन्दु॒र् वै वा इन्दुः॑ । \newline
51. इन्दू॒ रेतो॒ रेत॒ इन्दु॒ रिन्दू॒ रेतः॑ । \newline
52. रेत॑ ए॒वैव रेतो॒ रेत॑ ए॒व । \newline
53. ए॒व तत् तदे॒ वैव तत् । \newline
54. तद् द॑धाति दधाति॒ तत् तद् द॑धाति । \newline
55. द॒धा॒ ती॒न्द्रि॒या॒व॒ इ॒न्द्रि॒या॒वो॒ द॒धा॒ति॒ द॒धा॒ ती॒न्द्रि॒या॒वः॒ । \newline
56. इ॒न्द्रि॒या॒व॒ इतीती᳚न्द्रियाव इन्द्रियाव॒ इति॑ । \newline
57. इ॒न्द्रि॒या॒व॒ इती᳚न्द्रिय - वः॒ । \newline
58. इत्या॑हा॒हे तीत्या॑ह । \newline

\textbf{Ghana Paata } \newline

1. व॒द॒न्ति॒ यद् यद् व॑दन्ति वदन्ति॒ यद् घृ॒तेन॑ घृ॒तेन॒ यद् व॑दन्ति वदन्ति॒ यद् घृ॒तेन॑ । \newline
2. यद् घृ॒तेन॑ घृ॒तेन॒ यद् यद् घृ॒तेन॑ पात्नीव॒तम् पा᳚त्नीव॒तम् घृ॒तेन॒ यद् यद् घृ॒तेन॑ पात्नीव॒तम् । \newline
3. घृ॒तेन॑ पात्नीव॒तम् पा᳚त्नीव॒तम् घृ॒तेन॑ घृ॒तेन॑ पात्नीव॒तꣳ श्री॒णाति॑ श्री॒णाति॑ पात्नीव॒तम् घृ॒तेन॑ घृ॒तेन॑ पात्नीव॒तꣳ श्री॒णाति॑ । \newline
4. पा॒त्नी॒व॒तꣳ श्री॒णाति॑ श्री॒णाति॑ पात्नीव॒तम् पा᳚त्नीव॒तꣳ श्री॒णाति॒ वज्रे॑ण॒ वज्रे॑ण श्री॒णाति॑ पात्नीव॒तम् पा᳚त्नीव॒तꣳ श्री॒णाति॒ वज्रे॑ण । \newline
5. पा॒त्नी॒व॒तमिति॑ पात्नी - व॒तम् । \newline
6. श्री॒णाति॒ वज्रे॑ण॒ वज्रे॑ण श्री॒णाति॑ श्री॒णाति॒ वज्रे॑णै॒वैव वज्रे॑ण श्री॒णाति॑ श्री॒णाति॒ वज्रे॑णै॒व । \newline
7. वज्रे॑णै॒वैव वज्रे॑ण॒ वज्रे॑णै॒वैन॑ मेन मे॒व वज्रे॑ण॒ वज्रे॑णै॒वैन᳚म् । \newline
8. ए॒वैन॑ मेन मे॒वै वैनं॒ ॅवशे॒ वश॑ एन मे॒वै वैनं॒ ॅवशे᳚ । \newline
9. ए॒नं॒ ॅवशे॒ वश॑ एन मेनं॒ ॅवशे॑ कृ॒त्वा कृ॒त्वा वश॑ एन मेनं॒ ॅवशे॑ कृ॒त्वा । \newline
10. वशे॑ कृ॒त्वा कृ॒त्वा वशे॒ वशे॑ कृ॒त्वा गृ॑ह्णाति गृह्णाति कृ॒त्वा वशे॒ वशे॑ कृ॒त्वा गृ॑ह्णाति । \newline
11. कृ॒त्वा गृ॑ह्णाति गृह्णाति कृ॒त्वा कृ॒त्वा गृ॑ह्णा त्युपया॒मगृ॑हीत उपया॒मगृ॑हीतो गृह्णाति कृ॒त्वा कृ॒त्वा गृ॑ह्णा त्युपया॒मगृ॑हीतः । \newline
12. गृ॒ह्णा॒ त्यु॒प॒या॒मगृ॑हीत उपया॒मगृ॑हीतो गृह्णाति गृह्णा त्युपया॒मगृ॑हीतो ऽस्य स्युपया॒मगृ॑हीतो गृह्णाति गृह्णा त्युपया॒मगृ॑हीतो ऽसि । \newline
13. उ॒प॒या॒मगृ॑हीतो ऽस्य स्युपया॒मगृ॑हीत उपया॒मगृ॑हीतो॒ ऽसीती त्य॑स्युपया॒मगृ॑हीत उपया॒मगृ॑हीतो॒ ऽसीति॑ । \newline
14. उ॒प॒या॒मगृ॑हीत॒ इत्यु॑पया॒म - गृ॒ही॒तः॒ । \newline
15. अ॒सीती त्य॑स्य॒ सीत्या॑ हा॒हे त्य॑स्य॒ सीत्या॑ह । \newline
16. इत्या॑हा॒हे तीत्या॑ हे॒य मि॒य मा॒हे तीत्या॑हे॒यम् । \newline
17. आ॒हे॒य मि॒य मा॑हाहे॒यं ॅवै वा इ॒य मा॑हाहे॒यं ॅवै । \newline
18. इ॒यं ॅवै वा इ॒य मि॒यं ॅवा उ॑पया॒म उ॑पया॒मो वा इ॒य मि॒यं ॅवा उ॑पया॒मः । \newline
19. वा उ॑पया॒म उ॑पया॒मो वै वा उ॑पया॒म स्तस्मा॒त् तस्मा॑ दुपया॒मो वै वा उ॑पया॒म स्तस्मा᳚त् । \newline
20. उ॒प॒या॒म स्तस्मा॒त् तस्मा॑ दुपया॒म उ॑पया॒म स्तस्मा॑ दि॒मा मि॒माम् तस्मा॑ दुपया॒म उ॑पया॒म स्तस्मा॑ दि॒माम् । \newline
21. उ॒प॒या॒म इत्यु॑प - या॒मः । \newline
22. तस्मा॑ दि॒मा मि॒माम् तस्मा॒त् तस्मा॑ दि॒माम् प्र॒जाः प्र॒जा इ॒माम् तस्मा॒त् तस्मा॑ दि॒माम् प्र॒जाः । \newline
23. इ॒माम् प्र॒जाः प्र॒जा इ॒मा मि॒माम् प्र॒जा अन्वनु॑ प्र॒जा इ॒मा मि॒माम् प्र॒जा अनु॑ । \newline
24. प्र॒जा अन्वनु॑ प्र॒जाः प्र॒जा अनु॒ प्र प्राणु॑ प्र॒जाः प्र॒जा अनु॒ प्र । \newline
25. प्र॒जा इति॑ प्र - जाः । \newline
26. अनु॒ प्र प्राण्वनु॒ प्र जा॑यन्ते जायन्ते॒ प्राण्वनु॒ प्र जा॑यन्ते । \newline
27. प्र जा॑यन्ते जायन्ते॒ प्र प्र जा॑यन्ते॒ बृह॒स्पति॑सुतस्य॒ बृह॒स्पति॑सुतस्य जायन्ते॒ प्र प्र जा॑यन्ते॒ बृह॒स्पति॑सुतस्य । \newline
28. जा॒य॒न्ते॒ बृह॒स्पति॑सुतस्य॒ बृह॒स्पति॑सुतस्य जायन्ते जायन्ते॒ बृह॒स्पति॑सुतस्य ते ते॒ बृह॒स्पति॑सुतस्य जायन्ते जायन्ते॒ बृह॒स्पति॑सुतस्य ते । \newline
29. बृह॒स्पति॑सुतस्य ते ते॒ बृह॒स्पति॑सुतस्य॒ बृह॒स्पति॑सुतस्य त॒ इतीति॑ ते॒ बृह॒स्पति॑सुतस्य॒ बृह॒स्पति॑सुतस्य त॒ इति॑ । \newline
30. बृह॒स्पति॑सुत॒स्येति॒ बृह॒स्पति॑ - सु॒त॒स्य॒ । \newline
31. त॒ इतीति॑ ते त॒ इत्या॑हा॒ हेति॑ ते त॒ इत्या॑ह । \newline
32. इत्या॑हा॒हे तीत्या॑ह॒ ब्रह्म॒ ब्रह्मा॒हे तीत्या॑ह॒ ब्रह्म॑ । \newline
33. आ॒ह॒ ब्रह्म॒ ब्रह्मा॑हाह॒ ब्रह्म॒ वै वै ब्रह्मा॑हाह॒ ब्रह्म॒ वै । \newline
34. ब्रह्म॒ वै वै ब्रह्म॒ ब्रह्म॒ वै दे॒वाना᳚म् दे॒वानां॒ ॅवै ब्रह्म॒ ब्रह्म॒ वै दे॒वाना᳚म् । \newline
35. वै दे॒वाना᳚म् दे॒वानां॒ ॅवै वै दे॒वाना॒म् बृह॒स्पति॒र् बृह॒स्पति॑र् दे॒वानां॒ ॅवै वै दे॒वाना॒म् बृह॒स्पतिः॑ । \newline
36. दे॒वाना॒म् बृह॒स्पति॒र् बृह॒स्पति॑र् दे॒वाना᳚म् दे॒वाना॒म् बृह॒स्पति॒र् ब्रह्म॑णा॒ ब्रह्म॑णा॒ बृह॒स्पति॑र् दे॒वाना᳚म् दे॒वाना॒म् बृह॒स्पति॒र् ब्रह्म॑णा । \newline
37. बृह॒स्पति॒र् ब्रह्म॑णा॒ ब्रह्म॑णा॒ बृह॒स्पति॒र् बृह॒स्पति॒र् ब्रह्म॑णै॒वैव ब्रह्म॑णा॒ बृह॒स्पति॒र् बृह॒स्पति॒र् ब्रह्म॑णै॒व । \newline
38. ब्रह्म॑णै॒वैव ब्रह्म॑णा॒ ब्रह्म॑णै॒ वास्मा॑ अस्मा ए॒व ब्रह्म॑णा॒ ब्रह्म॑णै॒ वास्मै᳚ । \newline
39. ए॒वास्मा॑ अस्मा ए॒वै वास्मै᳚ प्र॒जाः प्र॒जा अ॑स्मा ए॒वै वास्मै᳚ प्र॒जाः । \newline
40. अ॒स्मै॒ प्र॒जाः प्र॒जा अ॑स्मा अस्मै प्र॒जाः प्र प्र प्र॒जा अ॑स्मा अस्मै प्र॒जाः प्र । \newline
41. प्र॒जाः प्र प्र प्र॒जाः प्र॒जाः प्र ज॑नयति जनयति॒ प्र प्र॒जाः प्र॒जाः प्र ज॑नयति । \newline
42. प्र॒जा इति॑ प्र - जाः । \newline
43. प्र ज॑नयति जनयति॒ प्र प्र ज॑नय तीन्दो इन्दो जनयति॒ प्र प्र ज॑नय तीन्दो । \newline
44. ज॒न॒य॒ ती॒न्दो॒ इ॒न्दो॒ ज॒न॒य॒ति॒ ज॒न॒य॒ ती॒न्दो॒ इतीती᳚न्दो जनयति जनय तीन्दो॒ इति॑ । \newline
45. इ॒न्दो॒ इतीती᳚न्दो इन्दो॒ इत्या॑हा॒ हेती᳚न्दो इन्दो॒ इत्या॑ह । \newline
46. इ॒न्दो॒ इती᳚न्दो । \newline
47. इत्या॑हा॒हे तीत्या॑ह॒ रेतो॒ रेत॑ आ॒हे तीत्या॑ह॒ रेतः॑ । \newline
48. आ॒ह॒ रेतो॒ रेत॑ आहाह॒ रेतो॒ वै वै रेत॑ आहाह॒ रेतो॒ वै । \newline
49. रेतो॒ वै वै रेतो॒ रेतो॒ वा इन्दु॒ रिन्दु॒र् वै रेतो॒ रेतो॒ वा इन्दुः॑ । \newline
50. वा इन्दु॒ रिन्दु॒र् वै वा इन्दू॒ रेतो॒ रेत॒ इन्दु॒र् वै वा इन्दू॒ रेतः॑ । \newline
51. इन्दू॒ रेतो॒ रेत॒ इन्दु॒ रिन्दू॒ रेत॑ ए॒वैव रेत॒ इन्दु॒ रिन्दू॒ रेत॑ ए॒व । \newline
52. रेत॑ ए॒वैव रेतो॒ रेत॑ ए॒व तत् तदे॒व रेतो॒ रेत॑ ए॒व तत् । \newline
53. ए॒व तत् तदे॒ वैव तद् द॑धाति दधाति॒ तदे॒ वैव तद् द॑धाति । \newline
54. तद् द॑धाति दधाति॒ तत् तद् द॑धातीन्द्रियाव इन्द्रियावो दधाति॒ तत् तद् द॑धातीन्द्रियावः । \newline
55. द॒धा॒ती॒न्द्रि॒या॒व॒ इ॒न्द्रि॒या॒वो॒ द॒धा॒ति॒ द॒धा॒ती॒न्द्रि॒या॒व॒ इतीती᳚न्द्रियावो दधाति दधातीन्द्रियाव॒ इति॑ । \newline
56. इ॒न्द्रि॒या॒व॒ इतीती᳚न्द्रियाव इन्द्रियाव॒ इत्या॑हा॒हे ती᳚न्द्रियाव इन्द्रियाव॒ इत्या॑ह । \newline
57. इ॒न्द्रि॒या॒व॒ इती᳚न्द्रिय - वः॒ । \newline
58. इत्या॑हा॒हे तीत्या॑ह प्र॒जाः प्र॒जा आ॒हे तीत्या॑ह प्र॒जाः । \newline
\pagebreak
\markright{ TS 6.5.8.4  \hfill https://www.vedavms.in \hfill}

\section{ TS 6.5.8.4 }

\textbf{TS 6.5.8.4 } \newline
\textbf{Samhita Paata} \newline

-त्या॑ह प्र॒जा वा इ॑न्द्रि॒यं प्र॒जा ए॒वास्मै॒ प्र ज॑नय॒त्यग्ना(3) इत्या॑हा॒ग्निर्वै रे॑तो॒धाः पत्नी॑व॒ इत्या॑ह मिथुन॒त्वाय॑ स॒जूर्दे॒वेन॒ त्वष्ट्रा॒ सोमं॑ पि॒बेत्या॑ह॒ त्वष्टा॒ वै प॑शू॒नां मि॑थु॒नानाꣳ॑ रूप॒कृद्-रू॒पमे॒व प॒शुषु॑ दधाति दे॒वा वै त्वष्टा॑रमजिघाꣳस॒न्थ् स पत्नीः॒ प्राप॑द्यत॒ तं न प्रति॒ प्राय॑च्छ॒न् तस्मा॒दपि॒- [  ] \newline

\textbf{Pada Paata} \newline

आ॒ह॒ । प्र॒जा इति॑ प्र - जाः । वै । इ॒न्द्रि॒यम् । प्र॒जा इति॑ प्र-जाः । ए॒व । अ॒स्मै॒ । प्रेति॑ । ज॒न॒य॒ति॒ । अग्ना(3) इ । इति॑ । आ॒ह॒ । अ॒ग्निः । वै । रे॒तो॒धा इति॑ रेतः - धाः । पत्नी॑व॒ इति॒ पत्नी᳚ - वः॒ । इति॑ । आ॒ह॒ । मि॒थु॒न॒त्वायेति॑ मिथुन-त्वाय॑ । स॒जूरिति॑ स - जूः । दे॒वेन॑ । त्वष्ट्रा᳚ । सोम᳚म् । पि॒ब॒ । इति॑ । आ॒ह॒ । त्वष्टा᳚ । वै । प॒शू॒नाम् । मि॒थु॒नाना᳚म् । रू॒प॒कृदिति॑ रूप - कृत् । रू॒पम् । ए॒व । प॒शुषु॑ । द॒धा॒ति॒ । दे॒वाः । वै । त्वष्टा॑रम् । अ॒जि॒घाꣳ॒॒स॒न्न् । सः । पत्नीः᳚ । प्रेति॑ । अ॒प॒द्य॒त॒ । तम् । न । प्रति॑ । प्रेति॑ । अ॒य॒च्छ॒न्न् । तस्मा᳚त् । अपीति॑ ।  \newline


\textbf{Krama Paata} \newline

आ॒ह॒ प्र॒जाः । प्र॒जा वै । प्र॒जा इति॑ प्र - जाः । वा इ॑न्द्रि॒यम् । इ॒न्द्रि॒यम् प्र॒जाः । प्र॒जा ए॒व । प्र॒जा इति॑ प्र - जाः । ए॒वास्मै᳚ । अ॒स्मै॒ प्र । प्र ज॑नयति । ज॒न॒य॒त्यग्ना(3) इ । अग्ना(3) इति॑ । इत्या॑ह । आ॒हा॒ग्निः । अ॒ग्निर् वै । वै रे॑तो॒धाः । रे॒तो॒धाः पत्नी॑वः । रे॒तो॒धा इति॑ रेतः - धाः । पत्नी॑व॒ इति॑ । पत्नी॑व॒ इति॒ पत्नी᳚ - वः॒ । इत्या॑ह । आ॒ह॒ मि॒थु॒न॒त्वाय॑ । मि॒थु॒न॒त्वाय॑ स॒जूः । मि॒थु॒न॒त्वायेति॑ मिथुन - त्वाय॑ । स॒जूर् दे॒वेन॑ । स॒जूरिति॑ स - जूः । दे॒वेन॒ त्वष्ट्रा᳚ । त्वष्ट्रा॒ सोम᳚म् । सोम॑म् पिब । पि॒बेति॑ । इत्या॑ह । आ॒ह॒ त्वष्टा᳚ । त्वष्टा॒ वै । वै प॑शू॒नाम् । प॒शू॒नाम् मि॑थु॒नाना᳚म् । मि॒थु॒नानाꣳ॑ रूप॒कृत् । रू॒प॒कृद् रू॒पम् । रू॒प॒कृदिति॑ रूप - कृत् । रू॒पमे॒व । ए॒व प॒शुषु॑ । प॒शुषु॑ दधाति । द॒धा॒ति॒ दे॒वाः । दे॒वा वै । वै त्वष्टा॑रम् । त्वष्टा॑रमजिघाꣳसन्न् । अ॒जि॒घाꣳ॒॒स॒न्थ् सः । स पत्नीः᳚ । पत्नीः॒ प्र । प्राप॑द्यत । अ॒प॒द्य॒त॒ तम् । तम् न । न प्रति॑ । प्रति॒ प्र । प्राय॑च्छन्न् । अ॒य॒च्छ॒न् तस्मा᳚त् । तस्मा॒दपि॑ । अपि॒ वद्ध्य᳚म् \newline

\textbf{Jatai Paata} \newline

1. आ॒ह॒ प्र॒जाः प्र॒जा आ॑हाह प्र॒जाः । \newline
2. प्र॒जा वै वै प्र॒जाः प्र॒जा वै । \newline
3. प्र॒जा इति॑ प्र - जाः । \newline
4. वा इ॑न्द्रि॒य मि॑न्द्रि॒यं ॅवै वा इ॑न्द्रि॒यम् । \newline
5. इ॒न्द्रि॒यम् प्र॒जाः प्र॒जा इ॑न्द्रि॒य मि॑न्द्रि॒यम् प्र॒जाः । \newline
6. प्र॒जा ए॒वैव प्र॒जाः प्र॒जा ए॒व । \newline
7. प्र॒जा इति॑ प्र - जाः । \newline
8. ए॒वास्मा॑ अस्मा ए॒वै वास्मै᳚ । \newline
9. अ॒स्मै॒ प्र प्रास्मा॑ अस्मै॒ प्र । \newline
10. प्र ज॑नयति जनयति॒ प्र प्र ज॑नयति । \newline
11. ज॒न॒य॒ त्यग्ना(3) अग्ना(3) इज॑नयति जनय॒ त्यग्ना(3) इ । \newline
12. अग्ना(3) इतीत्यग्ना(3) अग्ना(3) इति॑ । \newline
13. इत्या॑हा॒हे तीत्या॑ह । \newline
14. आ॒हा॒ग्नि र॒ग्नि रा॑हा हा॒ग्निः । \newline
15. अ॒ग्निर् वै वा अ॒ग्नि र॒ग्निर् वै । \newline
16. वै रे॑तो॒धा रे॑तो॒धा वै वै रे॑तो॒धाः । \newline
17. रे॒तो॒धाः पत्नी॑वः॒ पत्नी॑वो रेतो॒धा रे॑तो॒धाः पत्नी॑वः । \newline
18. रे॒तो॒धा इति॑ रेतः - धाः । \newline
19. पत्नी॑व॒ इतीति॒ पत्नी॑वः॒ पत्नी॑व॒ इति॑ । \newline
20. पत्नी॑व॒ इति॒ पत्नी᳚ - वः॒ । \newline
21. इत्या॑हा॒हे तीत्या॑ह । \newline
22. आ॒ह॒ मि॒थु॒न॒त्वाय॑ मिथुन॒त्वा या॑हाह मिथुन॒त्वाय॑ । \newline
23. मि॒थु॒न॒त्वाय॑ स॒जूः स॒जूर् मि॑थुन॒त्वाय॑ मिथुन॒त्वाय॑ स॒जूः । \newline
24. मि॒थु॒न॒त्वायेति॑ मिथुन - त्वाय॑ । \newline
25. स॒जूर् दे॒वेन॑ दे॒वेन॑ स॒जूः स॒जूर् दे॒वेन॑ । \newline
26. स॒जूरिति॑ स - जूः । \newline
27. दे॒वेन॒ त्वष्ट्रा॒ त्वष्ट्रा॑ दे॒वेन॑ दे॒वेन॒ त्वष्ट्रा᳚ । \newline
28. त्वष्ट्रा॒ सोमꣳ॒॒ सोम॒म् त्वष्ट्रा॒ त्वष्ट्रा॒ सोम᳚म् । \newline
29. सोम॑म् पिब पिब॒ सोमꣳ॒॒ सोम॑म् पिब । \newline
30. पि॒बे तीति॑ पिब पि॒बेति॑ । \newline
31. इत्या॑हा॒हे तीत्या॑ह । \newline
32. आ॒ह॒ त्वष्टा॒ त्वष्टा॑ ऽऽहाह॒ त्वष्टा᳚ । \newline
33. त्वष्टा॒ वै वै त्वष्टा॒ त्वष्टा॒ वै । \newline
34. वै प॑शू॒नाम् प॑शू॒नां ॅवै वै प॑शू॒नाम् । \newline
35. प॒शू॒नाम् मि॑थु॒नाना᳚म् मिथु॒नाना᳚म् पशू॒नाम् प॑शू॒नाम् मि॑थु॒नाना᳚म् । \newline
36. मि॒थु॒नानाꣳ॑ रूप॒कृद् रू॑प॒कृन् मि॑थु॒नाना᳚म् मिथु॒नानाꣳ॑ रूप॒कृत् । \newline
37. रू॒प॒कृद् रू॒पꣳ रू॒पꣳ रू॑प॒कृद् रू॑प॒कृद् रू॒पम् । \newline
38. रू॒प॒कृदिति॑ रूप - कृत् । \newline
39. रू॒प मे॒वैव रू॒पꣳ रू॒प मे॒व । \newline
40. ए॒व प॒शुषु॑ प॒शु ष्वे॒वैव प॒शुषु॑ । \newline
41. प॒शुषु॑ दधाति दधाति प॒शुषु॑ प॒शुषु॑ दधाति । \newline
42. द॒धा॒ति॒ दे॒वा दे॒वा द॑धाति दधाति दे॒वाः । \newline
43. दे॒वा वै वै दे॒वा दे॒वा वै । \newline
44. वै त्वष्टा॑र॒म् त्वष्टा॑रं॒ ॅवै वै त्वष्टा॑रम् । \newline
45. त्वष्टा॑र मजिघाꣳसन् नजिघाꣳस॒न् त्वष्टा॑र॒म् त्वष्टा॑र मजिघाꣳसन्न् । \newline
46. अ॒जि॒घाꣳ॒॒स॒न् थ्स सो॑ ऽजिघाꣳसन् नजिघाꣳस॒न् थ्सः । \newline
47. स पत्नीः॒ पत्नीः॒ स स पत्नीः᳚ । \newline
48. पत्नीः॒ प्र प्र पत्नीः॒ पत्नीः॒ प्र । \newline
49. प्राप॑द्यता पद्यत॒ प्र प्राप॑द्यत । \newline
50. अ॒प॒द्य॒त॒ तम् त म॑पद्यता पद्यत॒ तम् । \newline
51. तन् न न तम् तन् न । \newline
52. न प्रति॒ प्रति॒ न न प्रति॑ । \newline
53. प्रति॒ प्र प्र प्रति॒ प्रति॒ प्र । \newline
54. प्राय॑च्छन् नयच्छ॒न् प्र प्राय॑च्छन्न् । \newline
55. अ॒य॒च्छ॒न् तस्मा॒त् तस्मा॑ दयच्छन् नयच्छ॒न् तस्मा᳚त् । \newline
56. तस्मा॒ दप्यपि॒ तस्मा॒त् तस्मा॒ दपि॑ । \newline
57. अपि॒ वद्ध्यं॒ ॅवद्ध्य॒ मप्यपि॒ वद्ध्य᳚म् । \newline

\textbf{Ghana Paata } \newline

1. आ॒ह॒ प्र॒जाः प्र॒जा आ॑हाह प्र॒जा वै वै प्र॒जा आ॑हाह प्र॒जा वै । \newline
2. प्र॒जा वै वै प्र॒जाः प्र॒जा वा इ॑न्द्रि॒य मि॑न्द्रि॒यं ॅवै प्र॒जाः प्र॒जा वा इ॑न्द्रि॒यम् । \newline
3. प्र॒जा इति॑ प्र - जाः । \newline
4. वा इ॑न्द्रि॒य मि॑न्द्रि॒यं ॅवै वा इ॑न्द्रि॒यम् प्र॒जाः प्र॒जा इ॑न्द्रि॒यं ॅवै वा इ॑न्द्रि॒यम् प्र॒जाः । \newline
5. इ॒न्द्रि॒यम् प्र॒जाः प्र॒जा इ॑न्द्रि॒य मि॑न्द्रि॒यम् प्र॒जा ए॒वैव प्र॒जा इ॑न्द्रि॒य मि॑न्द्रि॒यम् प्र॒जा ए॒व । \newline
6. प्र॒जा ए॒वैव प्र॒जाः प्र॒जा ए॒वास्मा॑ अस्मा ए॒व प्र॒जाः प्र॒जा ए॒वास्मै᳚ । \newline
7. प्र॒जा इति॑ प्र - जाः । \newline
8. ए॒वास्मा॑ अस्मा ए॒वै वास्मै॒ प्र प्रास्मा॑ ए॒वै वास्मै॒ प्र । \newline
9. अ॒स्मै॒ प्र प्रास्मा॑ अस्मै॒ प्र ज॑नयति जनयति॒ प्रास्मा॑ अस्मै॒ प्र ज॑नयति । \newline
10. प्र ज॑नयति जनयति॒ प्र प्र ज॑नय॒ त्यग्ना(3) अग्ना(3) इज॑नयति॒ प्र प्र ज॑नय॒ त्यग्ना(3) इ । \newline
11. ज॒न॒य॒ त्यग्ना(3) अग्ना(3) इज॑नयति जनय॒ त्यग्ना(3) इतीत्यग्ना(3) इज॑नयति जनय॒ त्यग्ना(3) इति॑ । \newline
12. अग्ना(3) इतीत्यग्ना(3) अग्ना(3) इत्या॑हा॒हे त्यग्ना(3) अग्ना(3) इत्या॑ह । \newline
13. इत्या॑हा॒ हेती त्या॑हा॒ग्नि र॒ग्नि रा॒हे तीत्या॑ हा॒ग्निः । \newline
14. आ॒हा॒ग्नि र॒ग्नि रा॑हा हा॒ग्निर् वै वा अ॒ग्नि रा॑हा हा॒ग्निर् वै । \newline
15. अ॒ग्निर् वै वा अ॒ग्नि र॒ग्निर् वै रे॑तो॒धा रे॑तो॒धा वा अ॒ग्नि र॒ग्निर् वै रे॑तो॒धाः । \newline
16. वै रे॑तो॒धा रे॑तो॒धा वै वै रे॑तो॒धाः पत्नी॑वः॒ पत्नी॑वो रेतो॒धा वै वै रे॑तो॒धाः पत्नी॑वः । \newline
17. रे॒तो॒धाः पत्नी॑वः॒ पत्नी॑वो रेतो॒धा रे॑तो॒धाः पत्नी॑व॒ इतीति॒ पत्नी॑वो रेतो॒धा रे॑तो॒धाः पत्नी॑व॒ इति॑ । \newline
18. रे॒तो॒धा इति॑ रेतः - धाः । \newline
19. पत्नी॑व॒ इतीति॒ पत्नी॑वः॒ पत्नी॑व॒ इत्या॑हा॒हेति॒ पत्नी॑वः॒ पत्नी॑व॒ इत्या॑ह । \newline
20. पत्नी॑व॒ इति॒ पत्नी᳚ - वः॒ । \newline
21. इत्या॑हा॒हे तीत्या॑ह मिथुन॒त्वाय॑ मिथुन॒त्वाया॒ हेतीत्या॑ह मिथुन॒त्वाय॑ । \newline
22. आ॒ह॒ मि॒थु॒न॒त्वाय॑ मिथुन॒त्वाया॑ हाह मिथुन॒त्वाय॑ स॒जूः स॒जूर् मि॑थुन॒त्वाया॑ हाह मिथुन॒त्वाय॑ स॒जूः । \newline
23. मि॒थु॒न॒त्वाय॑ स॒जूः स॒जूर् मि॑थुन॒त्वाय॑ मिथुन॒त्वाय॑ स॒जूर् दे॒वेन॑ दे॒वेन॑ स॒जूर् मि॑थुन॒त्वाय॑ मिथुन॒त्वाय॑ स॒जूर् दे॒वेन॑ । \newline
24. मि॒थु॒न॒त्वायेति॑ मिथुन - त्वाय॑ । \newline
25. स॒जूर् दे॒वेन॑ दे॒वेन॑ स॒जूः स॒जूर् दे॒वेन॒ त्वष्ट्रा॒ त्वष्ट्रा॑ दे॒वेन॑ स॒जूः स॒जूर् दे॒वेन॒ त्वष्ट्रा᳚ । \newline
26. स॒जूरिति॑ स - जूः । \newline
27. दे॒वेन॒ त्वष्ट्रा॒ त्वष्ट्रा॑ दे॒वेन॑ दे॒वेन॒ त्वष्ट्रा॒ सोमꣳ॒॒ सोम॒म् त्वष्ट्रा॑ दे॒वेन॑ दे॒वेन॒ त्वष्ट्रा॒ सोम᳚म् । \newline
28. त्वष्ट्रा॒ सोमꣳ॒॒ सोम॒म् त्वष्ट्रा॒ त्वष्ट्रा॒ सोम॑म् पिब पिब॒ सोम॒म् त्वष्ट्रा॒ त्वष्ट्रा॒ सोम॑म् पिब । \newline
29. सोम॑म् पिब पिब॒ सोमꣳ॒॒ सोम॑म् पि॒बेतीति॑ पिब॒ सोमꣳ॒॒ सोम॑म् पि॒बेति॑ । \newline
30. पि॒बेतीति॑ पिब पि॒बे त्या॑हा॒हेति॑ पिब पि॒बे त्या॑ह । \newline
31. इत्या॑हा॒हे तीत्या॑ह॒ त्वष्टा॒ त्वष्टा॒ ऽऽहे तीत्या॑ह॒ त्वष्टा᳚ । \newline
32. आ॒ह॒ त्वष्टा॒ त्वष्टा॑ ऽऽहाह॒ त्वष्टा॒ वै वै त्वष्टा॑ ऽऽहाह॒ त्वष्टा॒ वै । \newline
33. त्वष्टा॒ वै वै त्वष्टा॒ त्वष्टा॒ वै प॑शू॒नाम् प॑शू॒नां ॅवै त्वष्टा॒ त्वष्टा॒ वै प॑शू॒नाम् । \newline
34. वै प॑शू॒नाम् प॑शू॒नां ॅवै वै प॑शू॒नाम् मि॑थु॒नाना᳚म् मिथु॒नाना᳚म् पशू॒नां ॅवै वै प॑शू॒नाम् मि॑थु॒नाना᳚म् । \newline
35. प॒शू॒नाम् मि॑थु॒नाना᳚म् मिथु॒नाना᳚म् पशू॒नाम् प॑शू॒नाम् मि॑थु॒नानाꣳ॑ रूप॒कृद् रू॑प॒कृन् मि॑थु॒नाना᳚म् पशू॒नाम् प॑शू॒नाम् मि॑थु॒नानाꣳ॑ रूप॒कृत् । \newline
36. मि॒थु॒नानाꣳ॑ रूप॒कृद् रू॑प॒कृन् मि॑थु॒नाना᳚म् मिथु॒नानाꣳ॑ रूप॒कृद् रू॒पꣳ रू॒पꣳ रू॑प॒कृन् मि॑थु॒नाना᳚म् मिथु॒नानाꣳ॑ रूप॒कृद् रू॒पम् । \newline
37. रू॒प॒कृद् रू॒पꣳ रू॒पꣳ रू॑प॒कृद् रू॑प॒कृद् रू॒प मे॒वैव रू॒पꣳ रू॑प॒कृद् रू॑प॒कृद् रू॒प मे॒व । \newline
38. रू॒प॒कृदिति॑ रूप - कृत् । \newline
39. रू॒प मे॒वैव रू॒पꣳ रू॒प मे॒व प॒शुषु॑ प॒शुष्वे॒व रू॒पꣳ रू॒प मे॒व प॒शुषु॑ । \newline
40. ए॒व प॒शुषु॑ प॒शु ष्वे॒वैव प॒शुषु॑ दधाति दधाति प॒शु ष्वे॒वैव प॒शुषु॑ दधाति । \newline
41. प॒शुषु॑ दधाति दधाति प॒शुषु॑ प॒शुषु॑ दधाति दे॒वा दे॒वा द॑धाति प॒शुषु॑ प॒शुषु॑ दधाति दे॒वाः । \newline
42. द॒धा॒ति॒ दे॒वा दे॒वा द॑धाति दधाति दे॒वा वै वै दे॒वा द॑धाति दधाति दे॒वा वै । \newline
43. दे॒वा वै वै दे॒वा दे॒वा वै त्वष्टा॑र॒म् त्वष्टा॑रं॒ ॅवै दे॒वा दे॒वा वै त्वष्टा॑रम् । \newline
44. वै त्वष्टा॑र॒म् त्वष्टा॑रं॒ ॅवै वै त्वष्टा॑र मजिघाꣳसन्-नजिघाꣳस॒न् त्वष्टा॑रं॒ ॅवै वै त्वष्टा॑र मजिघाꣳसन्न् । \newline
45. त्वष्टा॑र मजिघाꣳसन्-नजिघाꣳस॒न् त्वष्टा॑र॒म् त्वष्टा॑र मजिघाꣳस॒न् थ्स सो॑ ऽजिघाꣳस॒न् त्वष्टा॑र॒म् त्वष्टा॑र मजिघाꣳस॒न् थ्सः । \newline
46. अ॒जि॒घाꣳ॒॒स॒न् थ्स सो॑ ऽजिघाꣳसन्-नजिघाꣳस॒न् थ्स पत्नीः॒ पत्नीः॒ सो॑ ऽजिघाꣳसन्-नजिघाꣳस॒न् थ्स पत्नीः᳚ । \newline
47. स पत्नीः॒ पत्नीः॒ स स पत्नीः॒ प्र प्र पत्नीः॒ स स पत्नीः॒ प्र । \newline
48. पत्नीः॒ प्र प्र पत्नीः॒ पत्नीः॒ प्राप॑द्यता पद्यत॒ प्र पत्नीः॒ पत्नीः॒ प्राप॑द्यत । \newline
49. प्राप॑द्यता पद्यत॒ प्र प्राप॑द्यत॒ तम् त म॑पद्यत॒ प्र प्राप॑द्यत॒ तम् । \newline
50. अ॒प॒द्य॒त॒ तम् त म॑पद्यता पद्यत॒ तन् न न त म॑पद्यता पद्यत॒ तन् न । \newline
51. तन् न न तम् तन् न प्रति॒ प्रति॒ न तम् तन् न प्रति॑ । \newline
52. न प्रति॒ प्रति॒ न न प्रति॒ प्र प्र प्रति॒ न न प्रति॒ प्र । \newline
53. प्रति॒ प्र प्र प्रति॒ प्रति॒ प्राय॑च्छन्-नयच्छ॒न् प्र प्रति॒ प्रति॒ प्राय॑च्छन्न् । \newline
54. प्राय॑च्छन्-नयच्छ॒न् प्र प्राय॑च्छ॒न् तस्मा॒त् तस्मा॑ दयच्छ॒न् प्र प्राय॑च्छ॒न् तस्मा᳚त् । \newline
55. अ॒य॒च्छ॒न् तस्मा॒त् तस्मा॑ दयच्छन्-नयच्छ॒न् तस्मा॒ दप्यपि॒ तस्मा॑ दयच्छन्-नयच्छ॒न् तस्मा॒ दपि॑ । \newline
56. तस्मा॒ दप्यपि॒ तस्मा॒त् तस्मा॒ दपि॒ वद्ध्यं॒ ॅवद्ध्य॒ मपि॒ तस्मा॒त् तस्मा॒ दपि॒ वद्ध्य᳚म् । \newline
57. अपि॒ वद्ध्यं॒ ॅवद्ध्य॒ मप्यपि॒ वद्ध्य॒म् प्रप॑न्न॒म् प्रप॑न्नं॒ ॅवद्ध्य॒ मप्यपि॒ वद्ध्य॒म् प्रप॑न्नम् । \newline
\pagebreak
\markright{ TS 6.5.8.5  \hfill https://www.vedavms.in \hfill}

\section{ TS 6.5.8.5 }

\textbf{TS 6.5.8.5 } \newline
\textbf{Samhita Paata} \newline

वद्ध्यं॒ प्रप॑न्नं॒ न प्रति॒ प्रय॑च्छन्ति॒ तस्मा᳚त् पात्नीव॒ते त्वष्ट्रेऽपि॑ गृह्यते॒ न सा॑दय॒त्यस॑न्ना॒द्धि प्र॒जाः प्र॒जाय॑न्ते॒ नानु॒ वष॑ट् करोति॒ यद॑नुवषट् कु॒र्याद्-रु॒द्रं प्र॒जा अ॒न्वव॑सृजे॒द्-यन्नानु॑वषट्कु॒र्या-दशा᳚न्त-म॒ग्नीथ् सोमं॑ भक्षयेदुपाꣳ॒॒श्वनु॒ वष॑ट् करोति॒ न रु॒द्रं प्र॒जा अ॑न्ववसृ॒जति॑ शा॒न्तम॒ग्नीथ् सोमं॑ भक्षय॒त्यग्नी॒-न्नेष्टु॑रु॒पस्थ॒मा सी॑द॒- [  ] \newline

\textbf{Pada Paata} \newline

वद्ध्य᳚म् । प्रप॑न्न॒मिति॒ प्र - प॒न्न॒म् । न । प्रति॑ । प्रेति॑ । य॒च्छ॒न्ति॒ । तस्मा᳚त् । पा॒त्नी॒व॒त इति॑ पात्नी - व॒ते । त्वष्ट्रे᳚ । अपीति॑ । गृ॒ह्य॒ते॒ । न । सा॒द॒य॒ति॒ । अस॑न्नात् । हि । प्र॒जा इति॑ प्र - जाः । प्र॒जाय॑न्त॒ इति॑ प्र - जाय॑न्ते । न । अन्विति॑ । वष॑ट् । क॒रो॒ति॒ । यत् । अ॒नु॒व॒ष॒ट्कु॒र्यादित्य॑नु - व॒ष॒ट्कु॒र्यात् । रु॒द्रम् । प्र॒जा इति॑ प्र - जाः । अ॒न्वव॑सृजे॒दित्य॑नु - अव॑सृजेत् । यत् । न । अ॒नु॒व॒ष॒ट्कु॒र्यादित्य॑नु-व॒ष॒ट्कु॒र्यात् । अशा᳚न्तम् । अ॒ग्नीदित्य॑ग्नि-इत् । सोम᳚म् । भ॒क्ष॒ये॒त् ।  उ॒पाꣳ॒॒श्वित्यु॑प - अꣳ॒॒शु । अन्विति॑ । वष॑ट् । क॒रो॒ति॒ । न । रु॒द्रम् । प्र॒जा इति॑ प्र - जाः । अ॒न्व॒व॒सृ॒जतीत्य॑नु - अ॒व॒सृ॒जति॑ । शा॒न्तम् । अ॒ग्नीदित्य॑ग्नि - इत् । सोम᳚म् । भ॒क्ष॒य॒ति॒ । अग्नी॒दित्यग्नि॑ - इ॒त् । नेष्टुः॑ । उ॒पस्थ॒मित्यु॒प - स्थ॒म् । एति॑ । सी॒द॒ ।  \newline


\textbf{Krama Paata} \newline

वद्ध्य॒म् प्रप॑न्नम् । प्रप॑न्न॒म् न । प्रप॑न्न॒मिति॒ प्र - प॒न्न॒म् । न प्रति॑ । प्रति॒ प्र । प्र य॑च्छन्ति । य॒च्छ॒न्ति॒ तस्मा᳚त् । तस्मा᳚त् पात्नीव॒ते । पा॒त्नी॒व॒ते त्वष्ट्रे᳚ । पा॒त्नी॒व॒त इति॑ पात्नी - व॒ते । त्वष्टेऽपि॑ । अपि॑ गृह्यते । गृ॒ह्य॒ते॒ न । न सा॑दयति । सा॒द॒य॒त्यस॑न्नात् । अस॑न्ना॒द्‌धि । हि प्र॒जाः । प्र॒जाः प्र॒जाय॑न्ते । प्र॒जा इति॑ प्र - जाः । प्र॒जाय॑न्ते॒ न । प्र॒जाय॑न्त॒ इति॑ प्र - जाय॑न्ते । नानु॑ । अनु॒ वष॑ट् । वष॑ट् करोति । क॒रो॒ति॒ यत् । यद॑नुवषट्कु॒र्यात् । अ॒नु॒व॒ष॒ट्॒कु॒र्याद् रु॒द्रम् । अ॒नु॒व॒ष॒ट्॒कु॒र्यादित्य॑नु - व॒ष॒ट्॒कु॒र्यात् । रु॒द्रम् प्र॒जाः । प्र॒जा अ॒न्वव॑सृजेत् । प्र॒जा इति॑ प्र - जाः । अ॒न्वव॑सृजे॒द् यत् । अ॒न्वव॑सृजे॒दित्य॑नु - अव॑सृजेत् । यन् न । नानु॑वषट्कु॒र्यात् । अ॒नु॒व॒ष॒ट्॒कु॒र्यादशा᳚न्तम् । अ॒नु॒व॒ष॒ट्॒कु॒र्यादित्य॑नु - व॒ष॒ट्॒कु॒र्यात् । अशा᳚न्तम॒ग्नीत् । अ॒ग्नीथ् सोम᳚म् । अ॒ग्नीदित्य॑ग्नि - इत् । सोम॑म् भक्षयेत् । भ॒क्ष॒ये॒दु॒पाꣳ॒॒शु । उ॒पाꣳ॒॒श्वनु॑ । उ॒पाꣳ॒॒श्वित्यु॑प - अꣳ॒॒शु । अनु॒ वष॑ट् । वष॑ट् करोति । क॒रो॒ति॒ न । न रु॒द्रम् । रु॒द्रम् प्र॒जाः । प्र॒जा अ॑न्ववसृ॒जति॑ । प्र॒जा इति॑ प्र - जाः । अ॒न्व॒व॒सृ॒जति॑ शा॒न्तम् । अ॒न्व॒व॒सृ॒जतीत्य॑नु - अ॒व॒सृ॒जति॑ । शा॒न्तम॒ग्नीत् । अ॒ग्नीथ् सोम᳚म् । अ॒ग्नीदित्य॑ग्नि - इत् । सोम॑म् भक्षयति । भ॒क्ष॒य॒त्यग्नी᳚त् । अग्नी॒न् नेष्टुः॑ । अग्नी॒दित्यग्नि॑ - इ॒त्॒ । नेष्टु॑रु॒पस्थ᳚म् । उ॒पस्थ॒मा । उ॒पस्थ॒मित्यु॒प - स्थ॒म् । आ सी॑द । सी॒द॒ नेष्टः॑ \newline

\textbf{Jatai Paata} \newline

1. वद्ध्य॒म् प्रप॑न्न॒म् प्रप॑न्नं॒ ॅवद्ध्यं॒ ॅवद्ध्य॒म् प्रप॑न्नम् । \newline
2. प्रप॑न्न॒न् न न प्रप॑न्न॒म् प्रप॑न्न॒न् न । \newline
3. प्रप॑न्न॒मिति॒ प्र - प॒न्न॒म् । \newline
4. न प्रति॒ प्रति॒ न न प्रति॑ । \newline
5. प्रति॒ प्र प्र प्रति॒ प्रति॒ प्र । \newline
6. प्र य॑च्छन्ति यच्छन्ति॒ प्र प्र य॑च्छन्ति । \newline
7. य॒च्छ॒न्ति॒ तस्मा॒त् तस्मा᳚द् यच्छन्ति यच्छन्ति॒ तस्मा᳚त् । \newline
8. तस्मा᳚त् पात्नीव॒ते पा᳚त्नीव॒ते तस्मा॒त् तस्मा᳚त् पात्नीव॒ते । \newline
9. पा॒त्नी॒व॒ते त्वष्ट्रे॒ त्वष्ट्रे॑ पात्नीव॒ते पा᳚त्नीव॒ते त्वष्ट्रे᳚ । \newline
10. पा॒त्नी॒व॒त इति॑ पात्नी - व॒ते । \newline
11. त्वष्ट्रे ऽप्यपि॒ त्वष्ट्रे॒ त्वष्ट्रे ऽपि॑ । \newline
12. अपि॑ गृह्यते गृह्य॒ते ऽप्यपि॑ गृह्यते । \newline
13. गृ॒ह्य॒ते॒ न न गृ॑ह्यते गृह्यते॒ न । \newline
14. न सा॑दयति सादयति॒ न न सा॑दयति । \newline
15. सा॒द॒य॒ त्यस॑न्ना॒ दस॑न्नाथ् सादयति सादय॒ त्यस॑न्नात् । \newline
16. अस॑न्ना॒द्धि ह्यस॑न्ना॒ दस॑न्ना॒द्धि । \newline
17. हि प्र॒जाः प्र॒जा हि हि प्र॒जाः । \newline
18. प्र॒जाः प्र॒जाय॑न्ते प्र॒जाय॑न्ते प्र॒जाः प्र॒जाः प्र॒जाय॑न्ते । \newline
19. प्र॒जा इति॑ प्र - जाः । \newline
20. प्र॒जाय॑न्ते॒ न न प्र॒जाय॑न्ते प्र॒जाय॑न्ते॒ न । \newline
21. प्र॒जाय॑न्त॒ इति॑ प्र - जाय॑न्ते । \newline
22. नान् वनु॒ न नानु॑ । \newline
23. अनु॒ वष॒ड् वष॒ डन्वनु॒ वष॑ट् । \newline
24. वष॑ट् करोति करोति॒ वष॒ड् वष॑ट् करोति । \newline
25. क॒रो॒ति॒ यद् यत् क॑रोति करोति॒ यत् । \newline
26. यद॑नुवषट्कु॒र्या द॑नुवषट्कु॒र्याद् यद् यद॑नुवषट्कु॒र्यात् । \newline
27. अ॒नु॒व॒ष॒ट्कु॒र्याद् रु॒द्रꣳ रु॒द्र म॑नुवषट्कु॒र्या द॑नुवषट्कु॒र्याद् रु॒द्रम् । \newline
28. अ॒नु॒व॒ष॒ट्कु॒र्यादित्य॑नु - व॒ष॒ट्कु॒र्यात् । \newline
29. रु॒द्रम् प्र॒जाः प्र॒जा रु॒द्रꣳ रु॒द्रम् प्र॒जाः । \newline
30. प्र॒जा अ॒न्वव॑सृजे द॒न्वव॑सृजेत् प्र॒जाः प्र॒जा अ॒न्वव॑सृजेत् । \newline
31. प्र॒जा इति॑ प्र - जाः । \newline
32. अ॒न्वव॑सृजे॒द् यद् यद॒न्वव॑सृजे द॒न्वव॑सृजे॒द् यत् । \newline
33. अ॒न्वव॑सृजे॒दित्य॑नु - अव॑सृजेत् । \newline
34. यन् न न यद् यन् न । \newline
35. नानु॑वषट्कु॒र्या द॑नुवषट्कु॒र्यान् न नानु॑वषट्कु॒र्यात् । \newline
36. अ॒नु॒व॒ष॒ट्कु॒र्या दशा᳚न्त॒ मशा᳚न्त मनुवषट्कु॒र्या द॑नुवषट्कु॒र्या दशा᳚न्तम् । \newline
37. अ॒नु॒व॒ष॒ट्कु॒र्यादित्य॑नु - व॒ष॒ट्कु॒र्यात् । \newline
38. अशा᳚न्त म॒ग्नी द॒ग्नी दशा᳚न्त॒ मशा᳚न्त म॒ग्नीत् । \newline
39. अ॒ग्नीथ् सोमꣳ॒॒ सोम॑ म॒ग्नी द॒ग्नीथ् सोम᳚म् । \newline
40. अ॒ग्नीदित्य॑ग्नि - इत् । \newline
41. सोम॑म् भक्षयेद् भक्षये॒थ् सोमꣳ॒॒ सोम॑म् भक्षयेत् । \newline
42. भ॒क्ष॒ये॒ दु॒पाꣳ॒॒शू॑ पाꣳ॒॒शु भ॑क्षयेद् भक्षये दुपाꣳ॒॒शु । \newline
43. उ॒पाꣳ॒॒श्वन् वनू॑पाꣳ॒॒शू॑ पाꣳ॒॒श्वनु॑ । \newline
44. उ॒पाꣳ॒॒श्वित्यु॑प - अꣳ॒॒शु । \newline
45. अनु॒ वष॒ड् वष॒ डन्वनु॒ वष॑ट् । \newline
46. वष॑ट् करोति करोति॒ वष॒ड् वष॑ट् करोति । \newline
47. क॒रो॒ति॒ न न क॑रोति करोति॒ न । \newline
48. न रु॒द्रꣳ रु॒द्रन्न न रु॒द्रम् । \newline
49. रु॒द्रम् प्र॒जाः प्र॒जा रु॒द्रꣳ रु॒द्रम् प्र॒जाः । \newline
50. प्र॒जा अ॑न्ववसृ॒ज त्य॑न्ववसृ॒जति॑ प्र॒जाः प्र॒जा अ॑न्ववसृ॒जति॑ । \newline
51. प्र॒जा इति॑ प्र - जाः । \newline
52. अ॒न्व॒व॒सृ॒जति॑ शा॒न्तꣳ शा॒न्त म॑न्ववसृ॒ज त्य॑न्ववसृ॒जति॑ शा॒न्तम् । \newline
53. अ॒न्व॒व॒सृ॒जतीत्य॑नु - अ॒व॒सृ॒जति॑ । \newline
54. शा॒न्त म॒ग्नी द॒ग्नीच् छा॒न्तꣳ शा॒न्त म॒ग्नीत् । \newline
55. अ॒ग्नीथ् सोमꣳ॒॒ सोम॑ म॒ग्नी द॒ग्नीथ् सोम᳚म् । \newline
56. अ॒ग्नीदित्य॑ग्नि - इत् । \newline
57. सोम॑म् भक्षयति भक्षयति॒ सोमꣳ॒॒ सोम॑म् भक्षयति । \newline
58. भ॒क्ष॒य॒ त्यग्नी॒ दग्नी᳚द् भक्षयति भक्षय॒ त्यग्नी᳚त् । \newline
59. अग्नी॒न् नेष्टु॒र् नेष्टु॒ रग्नी॒ दग्नी॒न् नेष्टुः॑ । \newline
60. अग्नी॒दित्यग्नि॑ - इ॒त् । \newline
61. नेष्टु॑ रु॒पस्थ॑ मु॒पस्थ॒न् नेष्टु॒र् नेष्टु॑ रु॒पस्थ᳚म् । \newline
62. उ॒पस्थ॒ मोपस्थ॑ मु॒पस्थ॒ मा । \newline
63. उ॒पस्थ॒मित्यु॒प - स्थ॒म् । \newline
64. आ सी॑द सी॒दा सी॑द । \newline
65. सी॒द॒ नेष्ट॒र् नेष्टः॑ सीद सीद॒ नेष्टः॑ । \newline

\textbf{Ghana Paata } \newline

1. वद्ध्य॒म् प्रप॑न्न॒म् प्रप॑न्नं॒ ॅवद्ध्यं॒ ॅवद्ध्य॒म् प्रप॑न्न॒न् न न प्रप॑न्नं॒ ॅवद्ध्यं॒ ॅवद्ध्य॒म् प्रप॑न्न॒न् न । \newline
2. प्रप॑न्न॒न् न न प्रप॑न्न॒म् प्रप॑न्न॒न् न प्रति॒ प्रति॒ न प्रप॑न्न॒म् प्रप॑न्न॒न् न प्रति॑ । \newline
3. प्रप॑न्न॒मिति॒ प्र - प॒न्न॒म् । \newline
4. न प्रति॒ प्रति॒ न न प्रति॒ प्र प्र प्रति॒ न न प्रति॒ प्र । \newline
5. प्रति॒ प्र प्र प्रति॒ प्रति॒ प्र य॑च्छन्ति यच्छन्ति॒ प्र प्रति॒ प्रति॒ प्र य॑च्छन्ति । \newline
6. प्र य॑च्छन्ति यच्छन्ति॒ प्र प्र य॑च्छन्ति॒ तस्मा॒त् तस्मा᳚द् यच्छन्ति॒ प्र प्र य॑च्छन्ति॒ तस्मा᳚त् । \newline
7. य॒च्छ॒न्ति॒ तस्मा॒त् तस्मा᳚द् यच्छन्ति यच्छन्ति॒ तस्मा᳚त् पात्नीव॒ते पा᳚त्नीव॒ते तस्मा᳚द् यच्छन्ति यच्छन्ति॒ तस्मा᳚त् पात्नीव॒ते । \newline
8. तस्मा᳚त् पात्नीव॒ते पा᳚त्नीव॒ते तस्मा॒त् तस्मा᳚त् पात्नीव॒ते त्वष्ट्रे॒ त्वष्ट्रे॑ पात्नीव॒ते तस्मा॒त् तस्मा᳚त् पात्नीव॒ते त्वष्ट्रे᳚ । \newline
9. पा॒त्नी॒व॒ते त्वष्ट्रे॒ त्वष्ट्रे॑ पात्नीव॒ते पा᳚त्नीव॒ते त्वष्ट्रे ऽप्यपि॒ त्वष्ट्रे॑ पात्नीव॒ते पा᳚त्नीव॒ते त्वष्ट्रे ऽपि॑ । \newline
10. पा॒त्नी॒व॒त इति॑ पात्नी - व॒ते । \newline
11. त्वष्ट्रे ऽप्यपि॒ त्वष्ट्रे॒ त्वष्ट्रे ऽपि॑ गृह्यते गृह्य॒ते ऽपि॒ त्वष्ट्रे॒ त्वष्ट्रे ऽपि॑ गृह्यते । \newline
12. अपि॑ गृह्यते गृह्य॒ते ऽप्यपि॑ गृह्यते॒ न न गृ॑ह्य॒ते ऽप्यपि॑ गृह्यते॒ न । \newline
13. गृ॒ह्य॒ते॒ न न गृ॑ह्यते गृह्यते॒ न सा॑दयति सादयति॒ न गृ॑ह्यते गृह्यते॒ न सा॑दयति । \newline
14. न सा॑दयति सादयति॒ न न सा॑दय॒ त्यस॑न्ना॒ दस॑न्नाथ् सादयति॒ न न सा॑दय॒ त्यस॑न्नात् । \newline
15. सा॒द॒य॒ त्यस॑न्ना॒ दस॑न्नाथ् सादयति सादय॒ त्यस॑न्ना॒द्धि ह्यस॑न्नाथ् सादयति सादय॒ त्यस॑न्ना॒द्धि । \newline
16. अस॑न्ना॒द्धि ह्यस॑न्ना॒ दस॑न्ना॒द्धि प्र॒जाः प्र॒जा ह्यस॑न्ना॒ दस॑न्ना॒द्धि प्र॒जाः । \newline
17. हि प्र॒जाः प्र॒जा हि हि प्र॒जाः प्र॒जाय॑न्ते प्र॒जाय॑न्ते प्र॒जा हि हि प्र॒जाः प्र॒जाय॑न्ते । \newline
18. प्र॒जाः प्र॒जाय॑न्ते प्र॒जाय॑न्ते प्र॒जाः प्र॒जाः प्र॒जाय॑न्ते॒ न न प्र॒जाय॑न्ते प्र॒जाः प्र॒जाः प्र॒जाय॑न्ते॒ न । \newline
19. प्र॒जा इति॑ प्र - जाः । \newline
20. प्र॒जाय॑न्ते॒ न न प्र॒जाय॑न्ते प्र॒जाय॑न्ते॒ नान्वनु॒ न प्र॒जाय॑न्ते प्र॒जाय॑न्ते॒ नानु॑ । \newline
21. प्र॒जाय॑न्त॒ इति॑ प्र - जाय॑न्ते । \newline
22. नान्वनु॒ न नानु॒ वष॒ड् वष॒ डनु॒ न नानु॒ वष॑ट् । \newline
23. अनु॒ वष॒ड् वष॒ डन्वनु॒ वष॑ट् करोति करोति॒ वष॒ डन्वनु॒ वष॑ट् करोति । \newline
24. वष॑ट् करोति करोति॒ वष॒ड् वष॑ट् करोति॒ यद् यत् क॑रोति॒ वष॒ड् वष॑ट् करोति॒ यत् । \newline
25. क॒रो॒ति॒ यद् यत् क॑रोति करोति॒ यद॑नुवषट्कु॒र्या द॑नुवषट्कु॒र्याद् यत् क॑रोति करोति॒ यद॑नुवषट्कु॒र्यात् । \newline
26. यद॑नुवषट्कु॒र्या द॑नुवषट्कु॒र्याद् यद् यद॑नुवषट्कु॒र्याद् रु॒द्रꣳ रु॒द्र म॑नुवषट्कु॒र्याद् यद् यद॑नुवषट्कु॒र्याद् रु॒द्रम् । \newline
27. अ॒नु॒व॒ष॒ट्कु॒र्याद् रु॒द्रꣳ रु॒द्र म॑नुवषट्कु॒र्या द॑नुवषट्कु॒र्याद् रु॒द्रम् प्र॒जाः प्र॒जा रु॒द्र म॑नुवषट्कु॒र्या द॑नुवषट्कु॒र्याद् रु॒द्रम् प्र॒जाः । \newline
28. अ॒नु॒व॒ष॒ट्कु॒र्यादित्य॑नु - व॒ष॒ट्कु॒र्यात् । \newline
29. रु॒द्रम् प्र॒जाः प्र॒जा रु॒द्रꣳ रु॒द्रम् प्र॒जा अ॒न्वव॑सृजे द॒न्वव॑सृजेत् प्र॒जा रु॒द्रꣳ रु॒द्रम् प्र॒जा अ॒न्वव॑सृजेत् । \newline
30. प्र॒जा अ॒न्वव॑सृजे द॒न्वव॑सृजेत् प्र॒जाः प्र॒जा अ॒न्वव॑सृजे॒द् यद् यद॒न्वव॑सृजेत् प्र॒जाः प्र॒जा अ॒न्वव॑सृजे॒द् यत् । \newline
31. प्र॒जा इति॑ प्र - जाः । \newline
32. अ॒न्वव॑सृजे॒द् यद् यद॒न्वव॑सृजे द॒न्वव॑सृजे॒द् यन् न न यद॒न्वव॑सृजे द॒न्वव॑सृजे॒द् यन् न । \newline
33. अ॒न्वव॑सृजे॒दित्य॑नु - अव॑सृजेत् । \newline
34. यन् न न यद् यन् नानु॑वषट्कु॒र्या द॑नुवषट्कु॒र्यान् न यद् यन् नानु॑वषट्कु॒र्यात् । \newline
35. नानु॑वषट्कु॒र्या द॑नुवषट्कु॒र्यान् न नानु॑वषट्कु॒र्या दशा᳚न्त॒ मशा᳚न्त मनुवषट्कु॒र्यान् न नानु॑वषट्कु॒र्या दशा᳚न्तम् । \newline
36. अ॒नु॒व॒ष॒ट्कु॒र्या दशा᳚न्त॒ मशा᳚न्त मनुवषट्कु॒र्या द॑नुवषट्कु॒र्या दशा᳚न्त म॒ग्नी द॒ग्नी दशा᳚न्त मनुवषट्कु॒र्या द॑नुवषट्कु॒र्या दशा᳚न्त म॒ग्नीत् । \newline
37. अ॒नु॒व॒ष॒ट्कु॒र्यादित्य॑नु - व॒ष॒ट्कु॒र्यात् । \newline
38. अशा᳚न्त म॒ग्नी द॒ग्नी दशा᳚न्त॒ मशा᳚न्त म॒ग्नीथ् सोमꣳ॒॒ सोम॑ म॒ग्नी दशा᳚न्त॒ मशा᳚न्त म॒ग्नीथ् सोम᳚म् । \newline
39. अ॒ग्नीथ् सोमꣳ॒॒ सोम॑ म॒ग्नी द॒ग्नीथ् सोम॑म् भक्षयेद् भक्षये॒थ् सोम॑ म॒ग्नी द॒ग्नीथ् सोम॑म् भक्षयेत् । \newline
40. अ॒ग्नीदित्य॑ग्नि - इत् । \newline
41. सोम॑म् भक्षयेद् भक्षये॒थ् सोमꣳ॒॒ सोम॑म् भक्षये दुपाꣳ॒॒शू॑ पाꣳ॒॒शु भ॑क्षये॒थ् सोमꣳ॒॒ सोम॑म् भक्षये दुपाꣳ॒॒शु । \newline
42. भ॒क्ष॒ये॒ दु॒पाꣳ॒॒शू॑ पाꣳ॒॒शु भ॑क्षयेद् भक्षयेदुपाꣳ॒॒ श्वन्वनू॑ पाꣳ॒॒शु भ॑क्षयेद् भक्षयेदुपाꣳ॒॒ श्वनु॑ । \newline
43. उ॒पाꣳ॒॒ श्वन्वनू॑ पाꣳ॒॒शू॑ पाꣳ॒॒ श्वनु॒ वष॒ड् वष॒ डनू॑पाꣳ॒॒शू॑ पाꣳ॒॒ श्वनु॒ वष॑ट् । \newline
44. उ॒पाꣳ॒॒श्वित्यु॑प - अꣳ॒॒शु । \newline
45. अनु॒ वष॒ड् वष॒ डन्वनु॒ वष॑ट् करोति करोति॒ वष॒ डन्वनु॒ वष॑ट् करोति । \newline
46. वष॑ट् करोति करोति॒ वष॒ड् वष॑ट् करोति॒ न न क॑रोति॒ वष॒ड् वष॑ट् करोति॒ न । \newline
47. क॒रो॒ति॒ न न क॑रोति करोति॒ न रु॒द्रꣳ रु॒द्रन् न क॑रोति करोति॒ न रु॒द्रम् । \newline
48. न रु॒द्रꣳ रु॒द्रन् न न रु॒द्रम् प्र॒जाः प्र॒जा रु॒द्रन् न न रु॒द्रम् प्र॒जाः । \newline
49. रु॒द्रम् प्र॒जाः प्र॒जा रु॒द्रꣳ रु॒द्रम् प्र॒जा अ॑न्ववसृ॒ज त्य॑न्ववसृ॒जति॑ प्र॒जा रु॒द्रꣳ रु॒द्रम् प्र॒जा अ॑न्ववसृ॒जति॑ । \newline
50. प्र॒जा अ॑न्ववसृ॒ज त्य॑न्ववसृ॒जति॑ प्र॒जाः प्र॒जा अ॑न्ववसृ॒जति॑ शा॒न्तꣳ शा॒न्त म॑न्ववसृ॒जति॑ प्र॒जाः प्र॒जा अ॑न्ववसृ॒जति॑ शा॒न्तम् । \newline
51. प्र॒जा इति॑ प्र - जाः । \newline
52. अ॒न्व॒व॒सृ॒जति॑ शा॒न्तꣳ शा॒न्त म॑न्ववसृ॒ज त्य॑न्ववसृ॒जति॑ शा॒न्त म॒ग्नी द॒ग्नीच् छा॒न्त म॑न्ववसृ॒ज त्य॑न्ववसृ॒जति॑ शा॒न्त म॒ग्नीत् । \newline
53. अ॒न्व॒व॒सृ॒जतीत्य॑नु - अ॒व॒सृ॒जति॑ । \newline
54. शा॒न्त म॒ग्नी द॒ग्नीच् छा॒न्तꣳ शा॒न्त म॒ग्नीथ् सोमꣳ॒॒ सोम॑ म॒ग्नीच् छा॒न्तꣳ शा॒न्त म॒ग्नीथ् सोम᳚म् । \newline
55. अ॒ग्नीथ् सोमꣳ॒॒ सोम॑ म॒ग्नी द॒ग्नीथ् सोम॑म् भक्षयति भक्षयति॒ सोम॑ म॒ग्नी द॒ग्नीथ् सोम॑म् भक्षयति । \newline
56. अ॒ग्नीदित्य॑ग्नि - इत् । \newline
57. सोम॑म् भक्षयति भक्षयति॒ सोमꣳ॒॒ सोम॑म् भक्षय॒ त्यग्नी॒ दग्नी᳚द् भक्षयति॒ सोमꣳ॒॒ सोम॑म् भक्षय॒ त्यग्नी᳚त् । \newline
58. भ॒क्ष॒य॒ त्यग्नी॒ दग्नी᳚द् भक्षयति भक्षय॒ त्यग्नी॒न् नेष्टु॒र् नेष्टु॒ रग्नी᳚द् भक्षयति भक्षय॒ त्यग्नी॒न् नेष्टुः॑ । \newline
59. अग्नी॒न् नेष्टु॒र् नेष्टु॒ रग्नी॒ दग्नी॒न् नेष्टु॑ रु॒पस्थ॑ मु॒पस्थ॒म् नेष्टु॒ रग्नी॒ दग्नी॒न् नेष्टु॑ रु॒पस्थ᳚म् । \newline
60. अग्नी॒दित्यग्नि॑ - इ॒त् । \newline
61. नेष्टु॑ रु॒पस्थ॑ मु॒पस्थ॒म् नेष्टु॒र् नेष्टु॑ रु॒पस्थ॒ मोपस्थ॒म् नेष्टु॒र् नेष्टु॑ रु॒पस्थ॒ मा । \newline
62. उ॒पस्थ॒ मोपस्थ॑ मु॒पस्थ॒ मा सी॑द सीदो॒पस्थ॑ मु॒पस्थ॒ मा सी॑द । \newline
63. उ॒पस्थ॒मित्यु॒प - स्थ॒म् । \newline
64. आ सी॑द सी॒दा सी॑द॒ नेष्ट॒र् नेष्टः॑ सी॒दा सी॑द॒ नेष्टः॑ । \newline
65. सी॒द॒ नेष्ट॒र् नेष्टः॑ सीद सीद॒ नेष्टः॒ पत्नी॒म् पत्नी॒म् नेष्टः॑ सीद सीद॒ नेष्टः॒ पत्नी᳚म् । \newline
\pagebreak
\markright{ TS 6.5.8.6  \hfill https://www.vedavms.in \hfill}

\section{ TS 6.5.8.6 }

\textbf{TS 6.5.8.6 } \newline
\textbf{Samhita Paata} \newline

नेष्टः॒ पत्नी॑-मु॒दान॒येत्या॑हा॒-ग्नीदे॒व नेष्ट॑रि॒ रेतो॒ दधा॑ति॒ नेष्टा॒ पत्नि॑यामुद्-गा॒त्रा सं ख्या॑पयति प्र॒जाप॑ति॒र्वा ए॒ष यदु॑द्गा॒ता प्र॒जानां᳚ प्र॒जन॑नाया॒प उप॒ प्र व॑र्तयति॒ रेत॑ ए॒व तथ् सि॑ञ्चत्यू॒रुणोप॒ प्र व॑र्तयत्यू॒रुणा॒ हि रेतः॑ सि॒च्यते॑ नग्नं॒ कृत्यो॒-रुमुप॒ प्र व॑र्तयति य॒दा हि न॒ग्न ऊ॒रुर्भव॒त्यथ॑ मिथु॒नी ( ) भ॑व॒तोऽथ॒ रेतः॑ सिच्य॒तेऽथ॑ प्र॒जाः प्र जा॑यन्ते ॥ \newline

\textbf{Pada Paata} \newline

नेष्टः॑ । पत्नी᳚म् । उ॒दान॒येत्यु॑त् - आन॑य । इति॑ । आ॒ह॒ । अ॒ग्नीदित्य॑ग्नि-इत् । ए॒व । नेष्ट॑रि । रेतः॑ । दधा॑ति । नेष्टा᳚ । पत्नि॑याम् । उ॒द्गा॒त्रेत्यु॑त्-गा॒त्रा । समिति॑ । ख्या॒प॒य॒ति॒ । प्र॒जाप॑ति॒रिति॑ प्र॒जा-प॒तिः॒ । वै । ए॒षः । यत् । उ॒द्गा॒तेत्यु॑त् - गा॒ता । प्र॒जाना॒मिति॑ प्र - जाना᳚म् । प्र॒जन॑ना॒येति॑ प्र - जन॑नाय । अ॒पः । उप॑ । प्रेति॑ । व॒र्त॒य॒ति॒ । रेतः॑ । ए॒व । तत् । सि॒ञ्च॒ति॒ । ऊ॒रुणा᳚ । उप॑ । प्रेति॑ । व॒र्त॒य॒ति॒ । ऊ॒रुणा᳚ । हि । रेतः॑ । सि॒च्यते᳚ । न॒ग्न॒ङ्कृत्येति॑ नग्नम् - कृत्य॑ । ऊ॒रुम् । उप॑ । प्रेति॑ । व॒र्त॒य॒ति॒ । य॒दा । हि । न॒ग्नः । ऊ॒रुः । भव॑ति । अथ॑ । मि॒थु॒नी ( ) । भ॒व॒तः॒ । अथ॑ । रेतः॑ । सि॒च्य॒ते॒ । अथ॑ । प्र॒जा इति॑ प्र - जाः । प्रेति॑ । जा॒य॒न्ते॒ ॥  \newline


\textbf{Krama Paata} \newline

नेष्टः॒ पत्नी᳚म् । पत्नी॑मु॒दान॑य । उ॒दान॒येति॑ । उ॒दान॒येत्यु॑त् - आन॑य । इत्या॑ह । आ॒हा॒ग्नीत् । अ॒ग्नीदे॒व । अ॒ग्नीदित्य॑ग्नि - इत् । ए॒व नेष्ट॑रि । नेष्ट॑रि॒ रेतः॑ । रेतो॒ दधा॑ति । दधा॑ति॒ नेष्टा᳚ । नेष्टा॒ पत्नि॑याम् । पत्नि॑यामुद्गा॒त्रा । उ॒द्‍गा॒त्रा सम् । उ॒द्‍गा॒त्रेत्यु॑त् - गा॒त्रा । सम् ख्या॑पयति । ख्या॒प॒य॒ति॒ प्र॒जाप॑तिः । प्र॒जाप॑ति॒र् वै । प्र॒जाप॑ति॒रिति॑ प्र॒जा - प॒तिः॒ । वा ए॒षः । ए॒ष यत् । यदु॑द्‍गा॒ता । उ॒द्‍गा॒ता प्र॒जाना᳚म् । उ॒द्‍गा॒तेत्यु॑त् - गा॒ता । प्र॒जाना᳚म् प्र॒जन॑नाय । प्र॒जाना॒मिति॑ प्र - जाना᳚म् । प्र॒जन॑नाया॒पः । प्र॒जन॑ना॒येति॑ प्र - जन॑नाय । अ॒प उप॑ । उप॒ प्र । प्र व॑र्तयति । व॒र्त॒य॒ति॒ रेतः॑ । रेत॑ ए॒व । ए॒व तत् । तथ् सि॑ञ्चति । सि॒ञ्च॒त्यू॒रुणा᳚ । ऊ॒रुणोप॑ । उप॒ प्र । प्र व॑र्तयति । व॒र्त॒य॒त्यू॒रुणा᳚ । ऊ॒रुणा॒ हि । हि रेतः॑ । रेतः॑ सि॒च्यते᳚ । सि॒च्यते॑ नग्न॒ङ्‍कृत्य॑ । न॒ग्न॒ङ्‍कृत्यो॒रुम् । न॒ग्न॒ङ्‍कृत्येति॑ नग्नम् - कृत्य॑ । ऊ॒रुमुप॑ । उप॒ प्र । प्र व॑र्तयति । व॒र्त॒य॒ति॒ य॒दा । य॒दा हि । हि न॒ग्नः । न॒ग्न ऊ॒रुः । ऊ॒रुर् भव॑ति । भव॒त्यथ॑ । अथ॑ मिथु॒नी ( ) । मि॒थु॒नी भ॑वतः । भ॒व॒तोऽथ॑ । अथ॒ रेतः॑ । रेतः॑ सिच्यते । सि॒च्य॒तेऽथ॑ । अथ॑ प्र॒जाः । प्र॒जाः प्र । प्र॒जा इति॑ प्र - जाः । प्र जा॑यन्ते । जा॒य॒न्त॒ इति॑ जायन्ते । \newline

\textbf{Jatai Paata} \newline

1. नेष्टः॒ पत्नी॒म् पत्नी॒न् नेष्ट॒र् नेष्टः॒ पत्नी᳚म् । \newline
2. पत्नी॑ मु॒दान॑यो॒ दान॑य॒ पत्नी॒म् पत्नी॑ मु॒दान॑य । \newline
3. उ॒दान॒ये तीत्यु॒दान॑ यो॒दान॒येति॑ । \newline
4. उ॒दान॒येत्यु॑त् - आन॑य । \newline
5. इत्या॑हा॒हे तीत्या॑ह । \newline
6. आ॒हा॒ग्नी द॒ग्नी दा॑हा हा॒ग्नीत् । \newline
7. अ॒ग्नी दे॒वै वाग्नी द॒ग्नी दे॒व । \newline
8. अ॒ग्नीदित्य॑ग्नि - इत् । \newline
9. ए॒व नेष्ट॑रि॒ नेष्ट॑र्ये॒वैव नेष्ट॑रि । \newline
10. नेष्ट॑रि॒ रेतो॒ रेतो॒ नेष्ट॑रि॒ नेष्ट॑रि॒ रेतः॑ । \newline
11. रेतो॒ दधा॑ति॒ दधा॑ति॒ रेतो॒ रेतो॒ दधा॑ति । \newline
12. दधा॑ति॒ नेष्टा॒ नेष्टा॒ दधा॑ति॒ दधा॑ति॒ नेष्टा᳚ । \newline
13. नेष्टा॒ पत्नि॑या॒म् पत्नि॑या॒म् नेष्टा॒ नेष्टा॒ पत्नि॑याम् । \newline
14. पत्नि॑या मुद्‌गा॒ त्रोद्‌गा॒त्रा पत्नि॑या॒म् पत्नि॑या मुद्‌गा॒त्रा । \newline
15. उ॒द्‌गा॒त्रा सꣳ स मु॑द्‌गा॒ त्रोद्‌गा॒त्रा सम् । \newline
16. उ॒द्‌गा॒त्रेत्यु॑त् - गा॒त्रा । \newline
17. सम् ख्या॑पयति ख्यापयति॒ सꣳ सम् ख्या॑पयति । \newline
18. ख्या॒प॒य॒ति॒ प्र॒जाप॑तिः प्र॒जाप॑तिः ख्यापयति ख्यापयति प्र॒जाप॑तिः । \newline
19. प्र॒जाप॑ति॒र् वै वै प्र॒जाप॑तिः प्र॒जाप॑ति॒र् वै । \newline
20. प्र॒जाप॑ति॒रिति॑ प्र॒जा - प॒तिः॒ । \newline
21. वा ए॒ष ए॒ष वै वा ए॒षः । \newline
22. ए॒ष यद् यदे॒ष ए॒ष यत् । \newline
23. यदु॑द्‌गा॒ तोद्‌गा॒ता यद् यदु॑द्‌गा॒ता । \newline
24. उ॒द्‌गा॒ता प्र॒जाना᳚म् प्र॒जाना॑ मुद्‌गा॒ तोद्‌गा॒ता प्र॒जाना᳚म् । \newline
25. उ॒द्‌गा॒तेत्यु॑त् - गा॒ता । \newline
26. प्र॒जाना᳚म् प्र॒जन॑नाय प्र॒जन॑नाय प्र॒जाना᳚म् प्र॒जाना᳚म् प्र॒जन॑नाय । \newline
27. प्र॒जाना॒मिति॑ प्र - जाना᳚म् । \newline
28. प्र॒जन॑ना या॒पो॑ ऽपः प्र॒जन॑नाय प्र॒जन॑ना या॒पः । \newline
29. प्र॒जन॑ना॒येति॑ प्र - जन॑नाय । \newline
30. अ॒प उपोपा॒ पो॑ऽप उप॑ । \newline
31. उप॒ प्र प्रोपोप॒ प्र । \newline
32. प्र व॑र्तयति वर्तयति॒ प्र प्र व॑र्तयति । \newline
33. व॒र्त॒य॒ति॒ रेतो॒ रेतो॑ वर्तयति वर्तयति॒ रेतः॑ । \newline
34. रेत॑ ए॒वैव रेतो॒ रेत॑ ए॒व । \newline
35. ए॒व तत् तदे॒ वैव तत् । \newline
36. तथ् सि॑ञ्चति सिञ्चति॒ तत् तथ् सि॑ञ्चति । \newline
37. सि॒ञ्च॒ त्यू॒रुणो॒ रुणा॑ सिञ्चति सिञ्च त्यू॒रुणा᳚ । \newline
38. ऊ॒रुणो पोपो॒ रुणो॒ रुणोप॑ । \newline
39. उप॒ प्र प्रोपोप॒ प्र । \newline
40. प्र व॑र्तयति वर्तयति॒ प्र प्र व॑र्तयति । \newline
41. व॒र्त॒य॒ त्यू॒रुणो॒ रुणा॑ वर्तयति वर्तय त्यू॒रुणा᳚ । \newline
42. ऊ॒रुणा॒ हि ह्यू॑रुणो॒ रुणा॒ हि । \newline
43. हि रेतो॒ रेतो॒ हि हि रेतः॑ । \newline
44. रेतः॑ सि॒च्यते॑ सि॒च्यते॒ रेतो॒ रेतः॑ सि॒च्यते᳚ । \newline
45. सि॒च्यते॑ नग्न॒ङ्कृत्य॑ नग्न॒ङ्कृत्य॑ सि॒च्यते॑ सि॒च्यते॑ नग्न॒ङ्कृत्य॑ । \newline
46. न॒ग्न॒ङ्कृ त्यो॒रु मू॒रुम् न॑ग्न॒ङ्कृत्य॑ नग्न॒ङ्कृ त्यो॒रुम् । \newline
47. न॒ग्न॒ङ्कृत्येति॑ नग्नम् - कृत्य॑ । \newline
48. ऊ॒रु मुपो पो॒रु मू॒रु मुप॑ । \newline
49. उप॒ प्र प्रोपोप॒ प्र । \newline
50. प्र व॑र्तयति वर्तयति॒ प्र प्र व॑र्तयति । \newline
51. व॒र्त॒य॒ति॒ य॒दा य॒दा व॑र्तयति वर्तयति य॒दा । \newline
52. य॒दा हि हि य॒दा य॒दा हि । \newline
53. हि न॒ग्नो न॒ग्नो हि हि न॒ग्नः । \newline
54. न॒ग्न ऊ॒रु रू॒रुर् न॒ग्नो न॒ग्न ऊ॒रुः । \newline
55. ऊ॒रुर् भव॑ति॒ भव॑ त्यू॒रु रू॒रुर् भव॑ति । \newline
56. भव॒ त्यथाथ॒ भव॑ति॒ भव॒ त्यथ॑ । \newline
57. अथ॑ मिथु॒नी मि॑थु॒ न्यथाथ॑ मिथु॒नी । \newline
58. मि॒थु॒नी भ॑वतो भवतो मिथु॒नी मि॑थु॒नी भ॑वतः । \newline
59. भ॒व॒तो ऽथाथ॑ भवतो भव॒तो ऽथ॑ । \newline
60. अथ॒ रेतो॒ रेतो ऽथाथ॒ रेतः॑ । \newline
61. रेतः॑ सिच्यते सिच्यते॒ रेतो॒ रेतः॑ सिच्यते । \newline
62. सि॒च्य॒ते ऽथाथ॑ सिच्यते सिच्य॒ते ऽथ॑ । \newline
63. अथ॑ प्र॒जाः प्र॒जा अथाथ॑ प्र॒जाः । \newline
64. प्र॒जाः प्र प्र प्र॒जाः प्र॒जाः प्र । \newline
65. प्र॒जा इति॑ प्र - जाः । \newline
66. प्र जा॑यन्ते जायन्ते॒ प्र प्र जा॑यन्ते । \newline
67. जा॒य॒न्त॒ इति॑ जायन्ते । \newline

\textbf{Ghana Paata } \newline

1. नेष्टः॒ पत्नी॒म् पत्नी॒म् नेष्ट॒र् नेष्टः॒ पत्नी॑ मु॒दान॑ यो॒दान॑य॒ पत्नी॒म् नेष्ट॒र् नेष्टः॒ पत्नी॑ मु॒दान॑य । \newline
2. पत्नी॑ मु॒दान॑ यो॒दान॑य॒ पत्नी॒म् पत्नी॑ मु॒दान॒ये तीत्यु॒दान॑य॒ पत्नी॒म् पत्नी॑ मु॒दान॒येति॑ । \newline
3. उ॒दान॒ये तीत्यु॒दान॑ यो॒दान॒ये त्या॑हा॒हे त्यु॒दान॑ यो॒दान॒ येत्या॑ह । \newline
4. उ॒दान॒येत्यु॑त् - आन॑य । \newline
5. इत्या॑हा॒हे तीत्या॑ हा॒ग्नी द॒ग्नी दा॒हे तीत्या॑ हा॒ग्नीत् । \newline
6. आ॒हा॒ग्नी द॒ग्नी दा॑हा हा॒ग्नी दे॒वै वाग्नी दा॑हा हा॒ग्नी दे॒व । \newline
7. अ॒ग्नी दे॒वै वाग्नी द॒ग्नी दे॒व नेष्ट॑रि॒ नेष्ट॑ र्ये॒वाग्नी द॒ग्नी दे॒व नेष्ट॑रि । \newline
8. अ॒ग्नीदित्य॑ग्नि - इत् । \newline
9. ए॒व नेष्ट॑रि॒ नेष्ट॑र् ये॒वैव नेष्ट॑रि॒ रेतो॒ रेतो॒ नेष्ट॑र् ये॒वैव नेष्ट॑रि॒ रेतः॑ । \newline
10. नेष्ट॑रि॒ रेतो॒ रेतो॒ नेष्ट॑रि॒ नेष्ट॑रि॒ रेतो॒ दधा॑ति॒ दधा॑ति॒ रेतो॒ नेष्ट॑रि॒ नेष्ट॑रि॒ रेतो॒ दधा॑ति । \newline
11. रेतो॒ दधा॑ति॒ दधा॑ति॒ रेतो॒ रेतो॒ दधा॑ति॒ नेष्टा॒ नेष्टा॒ दधा॑ति॒ रेतो॒ रेतो॒ दधा॑ति॒ नेष्टा᳚ । \newline
12. दधा॑ति॒ नेष्टा॒ नेष्टा॒ दधा॑ति॒ दधा॑ति॒ नेष्टा॒ पत्नि॑या॒म् पत्नि॑या॒म् नेष्टा॒ दधा॑ति॒ दधा॑ति॒ नेष्टा॒ पत्नि॑याम् । \newline
13. नेष्टा॒ पत्नि॑या॒म् पत्नि॑या॒म् नेष्टा॒ नेष्टा॒ पत्नि॑या मुद्‍गा॒ त्रोद्‍गा॒त्रा पत्नि॑या॒म् नेष्टा॒ नेष्टा॒ पत्नि॑या मुद्‍गा॒त्रा । \newline
14. पत्नि॑या मुद्‍गा॒ त्रोद्‍गा॒त्रा पत्नि॑या॒म् पत्नि॑या मुद्‍गा॒त्रा सꣳ स मु॑द्‍गा॒त्रा पत्नि॑या॒म् पत्नि॑या मुद्‍गा॒त्रा सम् । \newline
15. उ॒द्‍गा॒त्रा सꣳ स मु॑द्‍गा॒ त्रोद्‍गा॒त्रा सम् ख्या॑पयति ख्यापयति॒ स मु॑द्‍गा॒ त्रोद्‍गा॒त्रा सम् ख्या॑पयति । \newline
16. उ॒द्‍गा॒त्रेत्यु॑त् - गा॒त्रा । \newline
17. सम् ख्या॑पयति ख्यापयति॒ सꣳ सम् ख्या॑पयति प्र॒जाप॑तिः प्र॒जाप॑तिः ख्यापयति॒ सꣳ सम् ख्या॑पयति प्र॒जाप॑तिः । \newline
18. ख्या॒प॒य॒ति॒ प्र॒जाप॑तिः प्र॒जाप॑तिः ख्यापयति ख्यापयति प्र॒जाप॑ति॒र् वै वै प्र॒जाप॑तिः ख्यापयति ख्यापयति प्र॒जाप॑ति॒र् वै । \newline
19. प्र॒जाप॑ति॒र् वै वै प्र॒जाप॑तिः प्र॒जाप॑ति॒र् वा ए॒ष ए॒ष वै प्र॒जाप॑तिः प्र॒जाप॑ति॒र् वा ए॒षः । \newline
20. प्र॒जाप॑ति॒रिति॑ प्र॒जा - प॒तिः॒ । \newline
21. वा ए॒ष ए॒ष वै वा ए॒ष यद् यदे॒ष वै वा ए॒ष यत् । \newline
22. ए॒ष यद् यदे॒ष ए॒ष यदु॑द्‍गा॒ तोद्‍गा॒ता यदे॒ष ए॒ष यदु॑द्‍गा॒ता । \newline
23. यदु॑द्‍गा॒ तोद्‍गा॒ता यद् यदु॑द्‍गा॒ता प्र॒जाना᳚म् प्र॒जाना॑ मुद्‍गा॒ता यद् यदु॑द्‍गा॒ता प्र॒जाना᳚म् । \newline
24. उ॒द्‍गा॒ता प्र॒जाना᳚म् प्र॒जाना॑ मुद्‍गा॒ तोद्‍गा॒ता प्र॒जाना᳚म् प्र॒जन॑नाय प्र॒जन॑नाय प्र॒जाना॑ मुद्‍गा॒ 
तोद्‍गा॒ता प्र॒जाना᳚म् प्र॒जन॑नाय । \newline
25. उ॒द्‍गा॒तेत्यु॑त् - गा॒ता । \newline
26. प्र॒जाना᳚म् प्र॒जन॑नाय प्र॒जन॑नाय प्र॒जाना᳚म् प्र॒जाना᳚म् प्र॒जन॑ना या॒पो॑ ऽपः प्र॒जन॑नाय प्र॒जाना᳚म् प्र॒जाना᳚म् प्र॒जन॑ना या॒पः । \newline
27. प्र॒जाना॒मिति॑ प्र - जाना᳚म् । \newline
28. प्र॒जन॑ना या॒पो॑ ऽपः प्र॒जन॑नाय प्र॒जन॑ना या॒प उपोपा॒पः प्र॒जन॑नाय प्र॒जन॑ना या॒प उप॑ । \newline
29. प्र॒जन॑ना॒येति॑ प्र - जन॑नाय । \newline
30. अ॒प उपोपा॒पो॑ ऽप उप॒ प्र प्रोपा॒पो॑ ऽप उप॒ प्र । \newline
31. उप॒ प्र प्रोपोप॒ प्र व॑र्तयति वर्तयति॒ प्रोपोप॒ प्र व॑र्तयति । \newline
32. प्र व॑र्तयति वर्तयति॒ प्र प्र व॑र्तयति॒ रेतो॒ रेतो॑ वर्तयति॒ प्र प्र व॑र्तयति॒ रेतः॑ । \newline
33. व॒र्त॒य॒ति॒ रेतो॒ रेतो॑ वर्तयति वर्तयति॒ रेत॑ ए॒वैव रेतो॑ वर्तयति वर्तयति॒ रेत॑ ए॒व । \newline
34. रेत॑ ए॒वैव रेतो॒ रेत॑ ए॒व तत् तदे॒व रेतो॒ रेत॑ ए॒व तत् । \newline
35. ए॒व तत् तदे॒ वैव तथ् सि॑ञ्चति सिञ्चति॒ तदे॒ वैव तथ् सि॑ञ्चति । \newline
36. तथ् सि॑ञ्चति सिञ्चति॒ तत् तथ् सि॑ञ्च त्यू॒रुणो॒ रुणा॑ सिञ्चति॒ तत् तथ् सि॑ञ्च त्यू॒रुणा᳚ । \newline
37. सि॒ञ्च॒ त्यू॒रुणो॒ रुणा॑ सिञ्चति सिञ्च त्यू॒रुणो पोपो॒ रुणा॑ सिञ्चति सिञ्च त्यू॒रुणोप॑ । \newline
38. ऊ॒रुणो पोपो॒ रुणो॒ रुणोप॒ प्र प्रोपो॒ रुणो॒ रुणोप॒ प्र । \newline
39. उप॒ प्र प्रोपोप॒ प्र व॑र्तयति वर्तयति॒ प्रोपोप॒ प्र व॑र्तयति । \newline
40. प्र व॑र्तयति वर्तयति॒ प्र प्र व॑र्तय त्यू॒रुणो॒ रुणा॑ वर्तयति॒ प्र प्र व॑र्तय त्यू॒रुणा᳚ । \newline
41. व॒र्त॒य॒ त्यू॒रुणो॒ रुणा॑ वर्तयति वर्तय त्यू॒रुणा॒ हि ह्यू॑रुणा॑ वर्तयति वर्तय त्यू॒रुणा॒ हि । \newline
42. ऊ॒रुणा॒ हि ह्यू॑रुणो॒ रुणा॒ हि रेतो॒ रेतो॒ ह्यू॑रुणो॒ रुणा॒ हि रेतः॑ । \newline
43. हि रेतो॒ रेतो॒ हि हि रेतः॑ सि॒च्यते॑ सि॒च्यते॒ रेतो॒ हि हि रेतः॑ सि॒च्यते᳚ । \newline
44. रेतः॑ सि॒च्यते॑ सि॒च्यते॒ रेतो॒ रेतः॑ सि॒च्यते॑ नग्न॒ङ्कृत्य॑ नग्न॒ङ्कृत्य॑ सि॒च्यते॒ रेतो॒ रेतः॑ सि॒च्यते॑ नग्न॒ङ्कृत्य॑ । \newline
45. सि॒च्यते॑ नग्न॒ङ्कृत्य॑ नग्न॒ङ्कृत्य॑ सि॒च्यते॑ सि॒च्यते॑ नग्न॒ङ्कृत्यो॒रु मू॒रुम् न॑ग्न॒ङ्कृत्य॑ सि॒च्यते॑ सि॒च्यते॑ नग्न॒ङ्कृत्यो॒रुम् । \newline
46. न॒ग्न॒ङ्कृत्यो॒रु मू॒रुम् न॑ग्न॒ङ्कृत्य॑ नग्न॒ङ्कृत्यो॒रु मुपोपो॒ रुम् न॑ग्न॒ङ्कृत्य॑ नग्न॒ङ्कृ
त्यो॒रु मुप॑ । \newline
47. न॒ग्न॒ङ्कृत्येति॑ नग्नम् - कृत्य॑ । \newline
48. ऊ॒रु मुपोपो॒रु मू॒रु मुप॒ प्र प्रोपो॒रु मू॒रु मुप॒ प्र । \newline
49. उप॒ प्र प्रोपोप॒ प्र व॑र्तयति वर्तयति॒ प्रोपोप॒ प्र व॑र्तयति । \newline
50. प्र व॑र्तयति वर्तयति॒ प्र प्र व॑र्तयति य॒दा य॒दा व॑र्तयति॒ प्र प्र व॑र्तयति य॒दा । \newline
51. व॒र्त॒य॒ति॒ य॒दा य॒दा व॑र्तयति वर्तयति य॒दा हि हि य॒दा व॑र्तयति वर्तयति य॒दा हि । \newline
52. य॒दा हि हि य॒दा य॒दा हि न॒ग्नो न॒ग्नो हि य॒दा य॒दा हि न॒ग्नः । \newline
53. हि न॒ग्नो न॒ग्नो हि हि न॒ग्न ऊ॒रु रू॒रुर् न॒ग्नो हि हि न॒ग्न ऊ॒रुः । \newline
54. न॒ग्न ऊ॒रु रू॒रुर् न॒ग्नो न॒ग्न ऊ॒रुर् भव॑ति॒ भव॑ त्यू॒रुर् न॒ग्नो न॒ग्न ऊ॒रुर् भव॑ति । \newline
55. ऊ॒रुर् भव॑ति॒ भव॑ त्यू॒रु रू॒रुर् भव॒ त्यथाथ॒ भव॑ त्यू॒रु रू॒रुर् भव॒ त्यथ॑ । \newline
56. भव॒ त्यथाथ॒ भव॑ति॒ भव॒ त्यथ॑ मिथु॒नी मि॑थु॒ न्यथ॒ भव॑ति॒ भव॒ त्यथ॑ मिथु॒नी । \newline
57. अथ॑ मिथु॒नी मि॑थु॒ न्यथाथ॑ मिथु॒नी भ॑वतो भवतो मिथु॒ न्यथाथ॑ मिथु॒नी भ॑वतः । \newline
58. मि॒थु॒नी भ॑वतो भवतो मिथु॒नी मि॑थु॒नी भ॑व॒तो ऽथाथ॑ भवतो मिथु॒नी मि॑थु॒नी भ॑व॒तो ऽथ॑ । \newline
59. भ॒व॒तो ऽथाथ॑ भवतो भव॒तो ऽथ॒ रेतो॒ रेतो ऽथ॑ भवतो भव॒तो ऽथ॒ रेतः॑ । \newline
60. अथ॒ रेतो॒ रेतो ऽथाथ॒ रेतः॑ सिच्यते सिच्यते॒ रेतो ऽथाथ॒ रेतः॑ सिच्यते । \newline
61. रेतः॑ सिच्यते सिच्यते॒ रेतो॒ रेतः॑ सिच्य॒ते ऽथाथ॑ सिच्यते॒ रेतो॒ रेतः॑ सिच्य॒ते ऽथ॑ । \newline
62. सि॒च्य॒ते ऽथाथ॑ सिच्यते सिच्य॒ते ऽथ॑ प्र॒जाः प्र॒जा अथ॑ सिच्यते सिच्य॒ते ऽथ॑ प्र॒जाः । \newline
63. अथ॑ प्र॒जाः प्र॒जा अथाथ॑ प्र॒जाः प्र प्र प्र॒जा अथाथ॑ प्र॒जाः प्र । \newline
64. प्र॒जाः प्र प्र प्र॒जाः प्र॒जाः प्र जा॑यन्ते जायन्ते॒ प्र प्र॒जाः प्र॒जाः प्र जा॑यन्ते । \newline
65. प्र॒जा इति॑ प्र - जाः । \newline
66. प्र जा॑यन्ते जायन्ते॒ प्र प्र जा॑यन्ते । \newline
67. जा॒य॒न्त॒ इति॑ जायन्ते । \newline
\pagebreak
\markright{ TS 6.5.9.1  \hfill https://www.vedavms.in \hfill}

\section{ TS 6.5.9.1 }

\textbf{TS 6.5.9.1 } \newline
\textbf{Samhita Paata} \newline

इन्द्रो॑ वृ॒त्रम॑ह॒न् तस्य॑ शीर्.षकपा॒ल-मुदौ᳚ब्ज॒थ् स द्रो॑णक॒लशो॑ऽभव॒त् तस्मा॒थ् सोमः॒ सम॑स्रव॒थ् स हा॑रियोज॒नो॑ऽभव॒त् तं ॅव्य॑चिकिथ्स-ज्जु॒हवा॒नी(3) मा हौ॒षा(3)मिति॒ सो॑ऽमन्यत॒ यद्धो॒ष्याम्या॒मꣳ हो᳚ष्यामि॒ यन्न हो॒ष्यामि॑ यज्ञ्वेश॒सं क॑रिष्या॒मीति॒ तम॑द्ध्रियत॒ होतुꣳ॒॒ सो᳚ऽग्निर॑ब्रवी॒न्न मय्या॒मꣳ हो᳚ष्य॒सीति॒ तं धा॒नाभि॑रश्रीणा॒त्- [  ] \newline

\textbf{Pada Paata} \newline

इन्द्रः॑ । वृ॒त्रम् । अ॒ह॒न्न् । तस्य॑ । शी॒र्॒.ष॒क॒पा॒लमिति॑ शीर्.ष-क॒पा॒लम् । उदिति॑ । औ॒ब्ज॒त् । सः । द्रो॒ण॒क॒ल॒श इति॑ द्रोण-क॒ल॒शः । अ॒भ॒व॒त् । तस्मा᳚त् । सोमः॑ । समिति॑ । अ॒स्र॒व॒त् । सः । हा॒रि॒यो॒ज॒न इति॑ हारि - यो॒ज॒नः । अ॒भ॒व॒त् । तम् । वीति॑ । अ॒चि॒कि॒थ्स॒त् । जु॒हवा॒नी(3) । मा । हौ॒षा(3)म् । इति॑ । सः । अ॒म॒न्य॒त॒ । यत् । हो॒ष्यामि॑ । आ॒मम् । हो॒ष्या॒मि॒ । यत् । न । हो॒ष्यामि॑ । य॒ज्ञ्॒वे॒श॒समिति॑ यज्ञ् - वे॒श॒सम् । क॒रि॒ष्या॒मि॒ । इति॑ । तम् । अ॒द्ध्रि॒य॒त॒ । होतु᳚म् । सः । अ॒ग्निः । अ॒ब्र॒वी॒त् । न । मयि॑ । आ॒मम् । हो॒ष्य॒सि॒ । इति॑ । तम् । धा॒नाभिः॑ । अ॒श्री॒णा॒त् ।  \newline


\textbf{Krama Paata} \newline

इन्द्रो॑ वृ॒त्रम् । वृ॒त्रम॑हन्न् । अ॒ह॒न् तस्य॑ । तस्य॑ शीर्.षकपा॒लम् । शी॒र्॒.ष॒क॒पा॒लमुत् । शी॒र्॒.ष॒क॒पा॒लमिति॑ शीर्.ष - क॒पा॒लम् । उदौ᳚ब्जत् । औ॒ब्ज॒थ् सः । स द्रो॑णकल॒शः । द्रो॒ण॒क॒ल॒शो॑ऽभवत् । द्रो॒ण॒क॒ल॒श इति॑ द्रोण - क॒ल॒शः । अ॒भ॒व॒त् तस्मा᳚त् । तस्मा॒थ् सोमः॑ । सोमः॒ सम् । सम॑स्रवत् । अ॒स्र॒व॒थ् सः । स हा॑रियोज॒नः । हा॒रि॒यो॒ज॒नो॑ऽभवत् । हा॒रि॒यो॒ज॒न इति॑ हारि - यो॒ज॒नः । अ॒भ॒व॒त् तम् । तम् ॅवि । व्य॑चिकिथ्सत् । अ॒चि॒कि॒थ्स॒ज् जु॒हवा॒नी(3) । जु॒हवा॒नी(3) मा । मा हौ॒षा(3)म् । हौ॒षा(3)मिति॑ । इति॒ सः । सो॑ऽमन्यत । अ॒म॒न्य॒त॒ यत् । यद्‌धो॒ष्यामि॑ । हो॒ष्याम्या॒मम् । आ॒मꣳ हो᳚ष्यामि । हो॒ष्या॒मि॒ यत् । यन् न । न हो॒ष्यामि॑ । हो॒ष्यामि॑ यज्ञ्वेश॒सम् । य॒ज्ञ्॒वे॒श॒सम् क॑रिष्यामि । य॒ज्ञ्॒वे॒श॒समिति॑ यज्ञ् - वे॒श॒सम् । क॒रि॒ष्या॒मीति॑ । इति॒ तम् । तम॑द्ध्रियत । अ॒द्ध्रि॒य॒त॒ होतु᳚म् । होतुꣳ॒॒ सः । सो᳚ऽग्निः । अ॒ग्निर॑ब्रवीत् । अ॒ब्र॒वी॒न् न । न मयि॑ । मय्या॒मम् । आ॒मꣳ हो᳚ष्यसि । हो॒ष्य॒सीति॑ । इति॒ तम् । तम् धा॒नाभिः॑ । धा॒नाभि॑रश्रीणात् । अ॒श्री॒णा॒त् तम् \newline

\textbf{Jatai Paata} \newline

1. इन्द्रो॑ वृ॒त्रं ॅवृ॒त्र मिन्द्र॒ इन्द्रो॑ वृ॒त्रम् । \newline
2. वृ॒त्र म॑हन् नहन् वृ॒त्रं ॅवृ॒त्र म॑हन्न् । \newline
3. अ॒ह॒न् तस्य॒ तस्या॑हन् नह॒न् तस्य॑ । \newline
4. तस्य॑ शीर्.षकपा॒लꣳ शी॑र्.षकपा॒लम् तस्य॒ तस्य॑ शीर्.षकपा॒लम् । \newline
5. शी॒र्॒.ष॒क॒पा॒ल मुदुच् छी॑र्.षकपा॒लꣳ शी॑र्.षकपा॒ल मुत् । \newline
6. शी॒र्॒.ष॒क॒पा॒लमिति॑ शीर्.ष - क॒पा॒लम् । \newline
7. उदौ᳚ब्ज दौब्ज॒ दुदु दौ᳚ब्जत् । \newline
8. औ॒ब्ज॒थ् स स औ᳚ब्ज दौब्ज॒थ् सः । \newline
9. स द्रो॑णकल॒शो द्रो॑णकल॒शः स स द्रो॑णकल॒शः । \newline
10. द्रो॒ण॒क॒ल॒शो॑ ऽभव दभवद् द्रोणकल॒शो द्रो॑णकल॒शो॑ ऽभवत् । \newline
11. द्रो॒ण॒क॒ल॒श इति॑ द्रोण - क॒ल॒शः । \newline
12. अ॒भ॒व॒त् तस्मा॒त् तस्मा॑ दभव दभव॒त् तस्मा᳚त् । \newline
13. तस्मा॒थ् सोमः॒ सोम॒ स्तस्मा॒त् तस्मा॒थ् सोमः॑ । \newline
14. सोमः॒ सꣳ सꣳ सोमः॒ सोमः॒ सम् । \newline
15. स म॑स्रव दस्रव॒थ् सꣳ स म॑स्रवत् । \newline
16. अ॒स्र॒व॒थ् स सो᳚ ऽस्रव दस्रव॒थ् सः । \newline
17. स हा॑रियोज॒नो हा॑रियोज॒नः स स हा॑रियोज॒नः । \newline
18. हा॒रि॒यो॒ज॒नो॑ ऽभव दभव द्धारियोज॒नो हा॑रियोज॒नो॑ ऽभवत् । \newline
19. हा॒रि॒यो॒ज॒न इति॑ हारि - यो॒ज॒नः । \newline
20. अ॒भ॒व॒त् तम् त म॑भव दभव॒त् तम् । \newline
21. तं ॅवि वि तम् तं ॅवि । \newline
22. व्य॑चिकिथ्स दचिकिथ्स॒द् वि व्य॑चिकिथ्सत् । \newline
23. अ॒चि॒कि॒थ्स॒ज् जु॒हवा॒नी(3) जु॒हवा॒नी(3) अ॑चिकिथ्स दचिकिथ्सज् जु॒हवा॒नी(3) । \newline
24. जु॒हवा॒नी(3) मा मा जु॒हवा॒नी(3) जु॒हवा॒नी(3) मा । \newline
25. मा हौ॒षा(3)ꣳ हौ॒षा(3)म् मा मा हौ॒षा(3)म् । \newline
26. हौ॒षा(3) मितीति॑ हौ॒षा(3)ꣳ हौ॒षा(3) मिति॑ । \newline
27. इति॒ स स इतीति॒ सः । \newline
28. सो॑ ऽमन्यता मन्यत॒ स सो॑ ऽमन्यत । \newline
29. अ॒म॒न्य॒त॒ यद् यद॑मन्यता मन्यत॒ यत् । \newline
30. यद्धो॒ष्यामि॑ हो॒ष्यामि॒ यद् यद्धो॒ष्यामि॑ । \newline
31. हो॒ष्या म्या॒म मा॒मꣳ हो॒ष्यामि॑ हो॒ष्या म्या॒मम् । \newline
32. आ॒मꣳ हो᳚ष्यामि होष्याम्या॒म मा॒मꣳ हो᳚ष्यामि । \newline
33. हो॒ष्या॒मि॒ यद् यद्धो᳚ष्यामि होष्यामि॒ यत् । \newline
34. यन् न न यद् यन् न । \newline
35. न हो॒ष्यामि॑ हो॒ष्यामि॒ न न हो॒ष्यामि॑ । \newline
36. हो॒ष्यामि॑ यज्ञ्वेश॒सं ॅय॑ज्ञ्वेश॒सꣳ हो॒ष्यामि॑ हो॒ष्यामि॑ यज्ञ्वेश॒सम् । \newline
37. य॒ज्ञ्॒वे॒श॒सम् क॑रिष्यामि करिष्यामि यज्ञ्वेश॒सं ॅय॑ज्ञ्वेश॒सम् क॑रिष्यामि । \newline
38. य॒ज्ञ्॒वे॒श॒समिति॑ यज्ञ् - वे॒श॒सम् । \newline
39. क॒रि॒ष्या॒मी तीति॑ करिष्यामि करिष्या॒ मीति॑ । \newline
40. इति॒ तम् तमितीति॒ तम् । \newline
41. त म॑द्ध्रियता द्ध्रियत॒ तम् त म॑द्ध्रियत । \newline
42. अ॒द्ध्रि॒य॒त॒ होतुꣳ॒॒ होतु॑ मद्ध्रियता द्ध्रियत॒ होतु᳚म् । \newline
43. होतुꣳ॒॒ स स होतुꣳ॒॒ होतुꣳ॒॒ सः । \newline
44. सो᳚ ऽग्नि र॒ग्निः स सो᳚ ऽग्निः । \newline
45. अ॒ग्नि र॑ब्रवी दब्रवी द॒ग्नि र॒ग्नि र॑ब्रवीत् । \newline
46. अ॒ब्र॒वी॒न् न नाब्र॑वी दब्रवी॒न् न । \newline
47. न मयि॒ मयि॒ न न मयि॑ । \newline
48. मय्या॒म मा॒मम् मयि॒ मय्या॒मम् । \newline
49. आ॒मꣳ हो᳚ष्यसि होष्यस्या॒म मा॒मꣳ हो᳚ष्यसि । \newline
50. हो॒ष्य॒सीतीति॑ होष्यसि होष्य॒सीति॑ । \newline
51. इति॒ तम् तमितीति॒ तम् । \newline
52. तम् धा॒नाभि॑र् धा॒नाभि॒ स्तम् तम् धा॒नाभिः॑ । \newline
53. धा॒नाभि॑ रश्रीणा दश्रीणाद् धा॒नाभि॑र् धा॒नाभि॑ रश्रीणात् । \newline
54. अ॒श्री॒णा॒त् तम् त म॑श्रीणा दश्रीणा॒त् तम् । \newline

\textbf{Ghana Paata } \newline

1. इन्द्रो॑ वृ॒त्रं ॅवृ॒त्र मिन्द्र॒ इन्द्रो॑ वृ॒त्र म॑हन्-नहन् वृ॒त्र मिन्द्र॒ इन्द्रो॑ वृ॒त्र म॑हन्न् । \newline
2. वृ॒त्र म॑हन्-नहन् वृ॒त्रं ॅवृ॒त्र म॑ह॒न् तस्य॒ तस्या॑हन् वृ॒त्रं ॅवृ॒त्र म॑ह॒न् तस्य॑ । \newline
3. अ॒ह॒न् तस्य॒ तस्या॑हन्-नह॒न् तस्य॑ शीर्.षकपा॒लꣳ शी॑र्.षकपा॒लम् तस्या॑हन्-नह॒न् तस्य॑ शीर्.षकपा॒लम् । \newline
4. तस्य॑ शीर्.षकपा॒लꣳ शी॑र्.षकपा॒लम् तस्य॒ तस्य॑ शीर्.षकपा॒ल मुदुच् छी॑र्.षकपा॒लम् तस्य॒ तस्य॑ शीर्.षकपा॒ल मुत् । \newline
5. शी॒र्॒.ष॒क॒पा॒ल मुदुच् छी॑र्.षकपा॒लꣳ शी॑र्.षकपा॒ल मुदौ᳚ब्ज दौब्ज॒ 
दुच् छी॑र्.षकपा॒लꣳ शी॑र्.षकपा॒ल मुदौ᳚ब्जत् । \newline
6. शी॒र्॒.ष॒क॒पा॒लमिति॑ शीर्.ष - क॒पा॒लम् । \newline
7. उदौ᳚ब्ज दौब्ज॒ दुदु दौ᳚ब्ज॒थ् स स औ᳚ब्ज॒ दुदु दौ᳚ब्ज॒थ् सः । \newline
8. औ॒ब्ज॒थ् स स औ᳚ब्ज दौब्ज॒थ् स द्रो॑णकल॒शो द्रो॑णकल॒शः स औ᳚ब्ज दौब्ज॒थ् स द्रो॑णकल॒शः । \newline
9. स द्रो॑णकल॒शो द्रो॑णकल॒शः स स द्रो॑णकल॒शो॑ ऽभव दभवद् द्रोणकल॒शः स स द्रो॑णकल॒शो॑ ऽभवत् । \newline
10. द्रो॒ण॒क॒ल॒शो॑ ऽभव दभवद् द्रोणकल॒शो द्रो॑णकल॒शो॑ ऽभव॒त् तस्मा॒त् तस्मा॑ दभवद् द्रोणकल॒शो द्रो॑णकल॒शो॑ ऽभव॒त् तस्मा᳚त् । \newline
11. द्रो॒ण॒क॒ल॒श इति॑ द्रोण - क॒ल॒शः । \newline
12. अ॒भ॒व॒त् तस्मा॒त् तस्मा॑ दभव दभव॒त् तस्मा॒थ् सोमः॒ सोम॒ स्तस्मा॑ दभव दभव॒त् तस्मा॒थ् सोमः॑ । \newline
13. तस्मा॒थ् सोमः॒ सोम॒ स्तस्मा॒त् तस्मा॒थ् सोमः॒ सꣳ सꣳ सोम॒ स्तस्मा॒त् तस्मा॒थ् सोमः॒ सम् । \newline
14. सोमः॒ सꣳ सꣳ सोमः॒ सोमः॒ स म॑स्रव दस्रव॒थ् सꣳ सोमः॒ सोमः॒ स म॑स्रवत् । \newline
15. स म॑स्रव दस्रव॒थ् सꣳ स म॑स्रव॒थ् स सो᳚ ऽस्रव॒थ् सꣳ स म॑स्रव॒थ् सः । \newline
16. अ॒स्र॒व॒थ् स सो᳚ ऽस्रव दस्रव॒थ् स हा॑रियोज॒नो हा॑रियोज॒नः सो᳚ ऽस्रव दस्रव॒थ् स हा॑रियोज॒नः । \newline
17. स हा॑रियोज॒नो हा॑रियोज॒नः स स हा॑रियोज॒नो॑ ऽभव दभव द्धारियोज॒नः स स हा॑रियोज॒नो॑ ऽभवत् । \newline
18. हा॒रि॒यो॒ज॒नो॑ ऽभव दभव द्धारियोज॒नो हा॑रियोज॒नो॑ ऽभव॒त् तम् त म॑भव द्धारियोज॒नो हा॑रियोज॒नो॑ ऽभव॒त् तम् । \newline
19. हा॒रि॒यो॒ज॒न इति॑ हारि - यो॒ज॒नः । \newline
20. अ॒भ॒व॒त् तम् त म॑भव दभव॒त् तं ॅवि वि त म॑भव दभव॒त् तं ॅवि । \newline
21. तं ॅवि वि तम् तं ॅव्य॑चिकिथ्स दचिकिथ्स॒द् वि तम् तं ॅव्य॑चिकिथ्सत् । \newline
22. व्य॑चिकिथ्स दचिकिथ्स॒द् वि व्य॑चिकिथ्सज् जु॒हवा॒नी(3) जु॒हवा॒नी(3) अ॑चिकिथ्स॒द् वि व्य॑चिकिथ्सज् जु॒हवा॒नी(3) । \newline
23. अ॒चि॒कि॒थ्स॒ज् जु॒हवा॒नी(3) जु॒हवा॒नी(3) अ॑चिकिथ्स दचिकिथ्सज् जु॒हवा॒नी(3) मा मा जु॒हवा॒नी(3) अ॑चिकिथ्स दचिकिथ्सज् जु॒हवा॒नी(3) मा । \newline
24. जु॒हवा॒नी(3) मा मा जु॒हवा॒नी(3) जु॒हवा॒नी(3) मा हौ॒षा(3)ꣳ हौ॒षा(3)म् मा जु॒हवा॒नी(3) जु॒हवा॒नी(3) मा हौ॒षा(3)म् । \newline
25. मा हौ॒षा(3)ꣳ हौ॒षा(3)म् मा मा हौ॒षा(3) मितीति॑ हौ॒षा(3)म् मा मा हौ॒षा(3) मिति॑ । \newline
26. हौ॒षा(3) मितीति॑ हौ॒षा(3)ꣳ हौ॒षा(3) मिति॒ स स इति॑ हौ॒षा(3)ꣳ हौ॒षा(3) मिति॒ सः । \newline
27. इति॒ स स इतीति॒ सो॑ ऽमन्यता मन्यत॒ स इतीति॒ सो॑ ऽमन्यत । \newline
28. सो॑ ऽमन्यता मन्यत॒ स सो॑ ऽमन्यत॒ यद् यद॑मन्यत॒ स सो॑ ऽमन्यत॒ यत् । \newline
29. अ॒म॒न्य॒त॒ यद् यद॑मन्यता मन्यत॒ यद्धो॒ष्यामि॑ हो॒ष्यामि॒ यद॑मन्यता मन्यत॒ यद्धो॒ष्यामि॑ । \newline
30. यद्धो॒ष्यामि॑ हो॒ष्यामि॒ यद् यद्धो॒ष्या म्या॒म मा॒मꣳ हो॒ष्यामि॒ यद् यद्धो॒ष्या म्या॒मम् । \newline
31. हो॒ष्या म्या॒म मा॒मꣳ हो॒ष्यामि॑ हो॒ष्या म्या॒मꣳ हो᳚ष्यामि होष्या म्या॒मꣳ हो॒ष्यामि॑ हो॒ष्या म्या॒मꣳ हो᳚ष्यामि । \newline
32. आ॒मꣳ हो᳚ष्यामि होष्या म्या॒म मा॒मꣳ हो᳚ष्यामि॒ यद् यद्धो᳚ष्या म्या॒म मा॒मꣳ हो᳚ष्यामि॒ यत् । \newline
33. हो॒ष्या॒मि॒ यद् यद्धो᳚ष्यामि होष्यामि॒ यन् न न यद्धो᳚ष्यामि होष्यामि॒ यन् न । \newline
34. यन् न न यद् यन् न हो॒ष्यामि॑ हो॒ष्यामि॒ न यद् यन् न हो॒ष्यामि॑ । \newline
35. न हो॒ष्यामि॑ हो॒ष्यामि॒ न न हो॒ष्यामि॑ यज्ञ्वेश॒सं ॅय॑ज्ञ्वेश॒सꣳ हो॒ष्यामि॒ न न हो॒ष्यामि॑ यज्ञ्वेश॒सम् । \newline
36. हो॒ष्यामि॑ यज्ञ्वेश॒सं ॅय॑ज्ञ्वेश॒सꣳ हो॒ष्यामि॑ हो॒ष्यामि॑ यज्ञ्वेश॒सम् क॑रिष्यामि करिष्यामि यज्ञ्वेश॒सꣳ हो॒ष्यामि॑ हो॒ष्यामि॑ यज्ञ्वेश॒सम् क॑रिष्यामि । \newline
37. य॒ज्ञ्॒वे॒श॒सम् क॑रिष्यामि करिष्यामि यज्ञ्वेश॒सं ॅय॑ज्ञ्वेश॒सम् क॑रिष्या॒ मीतीति॑ करिष्यामि यज्ञ्वेश॒सं ॅय॑ज्ञ्वेश॒सम् क॑रिष्या॒ मीति॑ । \newline
38. य॒ज्ञ्॒वे॒श॒समिति॑ यज्ञ् - वे॒श॒सम् । \newline
39. क॒रि॒ष्या॒मीतीति॑ करिष्यामि करिष्या॒मीति॒ तम् तमिति॑ करिष्यामि करिष्या॒मीति॒ तम् । \newline
40. इति॒ तम् त मितीति॒ त म॑द्ध्रियता द्ध्रियत॒ त मितीति॒ त म॑द्ध्रियत । \newline
41. त म॑द्ध्रियता द्ध्रियत॒ तम् त म॑द्ध्रियत॒ होतुꣳ॒॒ होतु॑ मद्ध्रियत॒ तम् त म॑द्ध्रियत॒ होतु᳚म् । \newline
42. अ॒द्ध्रि॒य॒त॒ होतुꣳ॒॒ होतु॑ मद्ध्रियता द्ध्रियत॒ होतुꣳ॒॒ स स होतु॑ मद्ध्रियता द्ध्रियत॒ होतुꣳ॒॒ सः । \newline
43. होतुꣳ॒॒ स स होतुꣳ॒॒ होतुꣳ॒॒ सो᳚ ऽग्नि र॒ग्निः स होतुꣳ॒॒ होतुꣳ॒॒ सो᳚ ऽग्निः । \newline
44. सो᳚ ऽग्नि र॒ग्निः स सो᳚ ऽग्नि र॑ब्रवी दब्रवी द॒ग्निः स सो᳚ ऽग्नि र॑ब्रवीत् । \newline
45. अ॒ग्नि र॑ब्रवी दब्रवी द॒ग्नि र॒ग्नि र॑ब्रवी॒न् न नाब्र॑वी द॒ग्नि र॒ग्नि र॑ब्रवी॒न् न । \newline
46. अ॒ब्र॒वी॒न् न नाब्र॑वी दब्रवी॒न् न मयि॒ मयि॒ नाब्र॑वी दब्रवी॒न् न मयि॑ । \newline
47. न मयि॒ मयि॒ न न मय्या॒म मा॒मम् मयि॒ न न मय्या॒मम् । \newline
48. मय्या॒म मा॒मम् मयि॒ मय्या॒मꣳ हो᳚ष्यसि होष्यस्या॒मम् मयि॒ मय्या॒मꣳ हो᳚ष्यसि । \newline
49. आ॒मꣳ हो᳚ष्यसि होष्य स्या॒म मा॒मꣳ हो᳚ष्य॒ सीतीति॑ होष्य स्या॒म मा॒मꣳ हो᳚ष्य॒सीति॑ । \newline
50. हो॒ष्य॒ सीतीति॑ होष्यसि होष्य॒ सीति॒ तम् त मिति॑ होष्यसि होष्य॒ सीति॒ तम् । \newline
51. इति॒ तम् त मितीति॒ तम् धा॒नाभि॑र् धा॒नाभि॒ स्त मितीति॒ तम् धा॒नाभिः॑ । \newline
52. तम् धा॒नाभि॑र् धा॒नाभि॒ स्तम् तम् धा॒नाभि॑ रश्रीणा दश्रीणाद् धा॒नाभि॒ स्तम् तम् धा॒नाभि॑ रश्रीणात् । \newline
53. धा॒नाभि॑ रश्रीणा दश्रीणाद् धा॒नाभि॑र् धा॒नाभि॑ रश्रीणा॒त् तम् त म॑श्रीणाद् धा॒नाभि॑र् धा॒नाभि॑ रश्रीणा॒त् तम् । \newline
54. अ॒श्री॒णा॒त् तम् त म॑श्रीणा दश्रीणा॒त् तꣳ शृ॒तꣳ शृ॒तम् त म॑श्रीणा दश्रीणा॒त् तꣳ शृ॒तम् । \newline
\pagebreak
\markright{ TS 6.5.9.2  \hfill https://www.vedavms.in \hfill}

\section{ TS 6.5.9.2 }

\textbf{TS 6.5.9.2 } \newline
\textbf{Samhita Paata} \newline

तꣳ शृ॒तं भू॒तम॑जुहो॒द्-यद्धा॒नाभि॑र्. हारियोज॒नꣳ श्री॒णाति॑ शृत॒त्वाय॑ शृ॒तमे॒वैनं॑ भू॒तं जु॑होति ब॒ह्वीभिः॑ श्रीणात्ये॒ताव॑ती-रे॒वास्या॒-मुष्मि॑न् ॅलो॒के का॑म॒दुघा॑ भव॒न्त्यथो॒ खल्वा॑हुरे॒ता वा इन्द्र॑स्य॒ पृश्न॑यः काम॒दुघा॒ यद्धा॑रियोज॒नीरिति॒ तस्मा᳚द्-ब॒ह्वीभिः॑ श्रीणीयादृख्सा॒मे वा इन्द्र॑स्य॒ हरी॑ सोम॒पानौ॒ तयोः᳚ परि॒धय॑ आ॒धानं॒ ॅयदप्र॑हृत्य परि॒धी-ञ्जु॑हु॒या-द॒न्तरा॑धानाभ्यां- [  ] \newline

\textbf{Pada Paata} \newline

तम् । शृ॒तम् । भू॒तम् । अ॒जु॒हो॒त् । यत् । धा॒नाभिः॑ । हा॒रि॒यो॒ज॒नमिति॑ हारि - यो॒ज॒नम् । श्री॒णाति॑ । शृ॒त॒त्वायेति॑ शृत - त्वाय॑ । शृ॒तम् । ए॒व । ए॒न॒म् । भू॒तम् । जु॒हो॒ति॒ । ब॒ह्वीभिः॑ । श्री॒णा॒ति॒ । ए॒ताव॑तीः । ए॒व । अ॒स्य॒ । अ॒मुष्मिन्न्॑ । लो॒के । का॒म॒दुघा॒ इति॑ काम - दुघाः᳚ । भ॒व॒न्ति॒ । अथो॒ इति॑ । खलु॑ । आ॒हुः॒ । ए॒ताः । वै । इन्द्र॑स्य । पृश्न॑यः । का॒म॒दुघा॒ इति॑ काम - दुघाः᳚ । यत् । हा॒रि॒यो॒ज॒नीरिति॑ हारि - यो॒ज॒नीः । इति॑ । तस्मा᳚त् । ब॒ह्वीभिः॑ । श्री॒णी॒या॒त् । ऋ॒ख्सा॒मे इत्यृ॑क् - सा॒मे । वै । इन्द्र॑स्य । हरी॒ इति॑ । सो॒म॒पाना॒विति॑ सोम - पानौ᳚ । तयोः᳚ । प॒रि॒धय॒ इति॑ परि - धयः॑ । आ॒धान॒मित्या᳚ - धान᳚म् । यत् । अप्र॑हृ॒त्येत्यप्र॑ - हृ॒त्य॒ । प॒रि॒धीनिति॑ परि - धीन् । जु॒हु॒यात् । अ॒न्तरा॑धानाभ्या॒मित्य॒न्तः-आ॒धा॒ना॒भ्या॒म् ।  \newline


\textbf{Krama Paata} \newline

तꣳ शृ॒तम् । शृ॒तम् भू॒तम् । भू॒तम॑जुहोत् । अ॒जु॒हो॒द् यत् । यद् धा॒नाभिः॑ । धा॒नाभि॑र्. हारियोज॒नम् । हा॒रि॒यो॒ज॒नꣳ श्री॒णाति॑ । हा॒रि॒यो॒ज॒नमिति॑ हारि - यो॒ज॒नम् । श्री॒णाति॑ शृत॒त्वाय॑ । शृ॒त॒त्वाय॑ शृ॒तम् । शृ॒त॒त्वायेति॑ शृत - त्वाय॑ । शृ॒तमे॒व । ए॒वैन᳚म् । ए॒न॒म् भू॒तम् । भू॒तम् जु॑होति । जु॒हो॒ति॒ ब॒ह्वीभिः॑ । ब॒ह्वीभिः॑ श्रीणाति । श्री॒णा॒त्ये॒ताव॑तीः । ए॒ताव॑तीरे॒व । ए॒वास्य॑ । अ॒स्या॒मुष्मिन्न्॑ । अ॒मुष्मि॑न् ॅलो॒के । लो॒के का॑म॒दुघाः᳚ । का॒म॒दुघा॑ भवन्ति । का॒म॒दुघा॒ इति॑ काम - दुघाः᳚ । भ॒व॒न्त्यथो᳚ । अथो॒ खलु॑ । अथो॒ इत्यथो᳚ । खल्वा॑हुः । आ॒हु॒रे॒ताः । ए॒ता वै । वा इन्द्र॑स्य । इन्द्र॑स्य॒ पृश्ञ॑यः । पृश्ञ॑यः काम॒दुघाः᳚ । का॒म॒दुघा॒ यत् । का॒म॒दुघा॒ इति॑ काम - दुघाः᳚ । यद्‌धा॑रियोज॒नीः । हा॒रि॒यो॒ज॒नीरिति॑ । हा॒रि॒यो॒ज॒नीरिति॑ हारि - यो॒ज॒नीः । इति॒ तस्मा᳚त् । तस्मा᳚द् ब॒ह्वीभिः॑ । ब॒ह्वीभिः॑ श्रीणीयात् । 
श्री॒णी॒या॒दृ॒ख्सा॒मे । ऋ॒ख्सा॒मे वै । ऋ॒ख्सा॒मे इत्यृ॑क् - सा॒मे । वा इन्द्र॑स्य । इन्द्र॑स्य॒ हरी᳚ । हरी॑ सोम॒पानौ᳚ । हरी॒ इति॒ हरी᳚ । सो॒म॒पानौ॒ तयोः᳚ । सो॒म॒पाना॒विति॑ सोम - पानौ᳚ । तयोः᳚ परि॒धयः॑ । प॒रि॒धय॑ आ॒धान᳚म् । प॒रि॒धय॒ इति॑ परि - धयः॑ । आ॒धान॒म् ॅयत् । आ॒धान॒मित्या᳚ - धान᳚म् । यदप्र॑हृत्य । अप्र॑हृत्य परि॒धीन् । अप्र॑हृ॒त्येत्यप्र॑ - हृ॒त्य॒ । प॒रि॒धीन् जु॑हु॒यात् । प॒रि॒धीनिति॑ परि - धीन् । जु॒हु॒याद॒न्तरा॑धानाभ्याम् । अ॒न्तरा॑धानाभ्याम् घा॒सम् । अ॒न्तरा॑धानाभ्या॒मित्य॒न्तः - आ॒धा॒ना॒भ्या॒म् \newline

\textbf{Jatai Paata} \newline

1. तꣳ शृ॒तꣳ शृ॒तम् तम् तꣳ शृ॒तम् । \newline
2. शृ॒तम् भू॒तम् भू॒तꣳ शृ॒तꣳ शृ॒तम् भू॒तम् । \newline
3. भू॒त म॑जुहो दजुहोद् भू॒तम् भू॒त म॑जुहोत् । \newline
4. अ॒जु॒हो॒द् यद् यद॑जुहो दजुहो॒द् यत् । \newline
5. यद् धा॒नाभि॑र् धा॒नाभि॒र् यद् यद् धा॒नाभिः॑ । \newline
6. धा॒नाभि॑र्. हारियोज॒नꣳ हा॑रियोज॒नम् धा॒नाभि॑र् धा॒नाभि॑र्. हारियोज॒नम् । \newline
7. हा॒रि॒यो॒ज॒नꣳ श्री॒णाति॑ श्री॒णाति॑ हारियोज॒नꣳ हा॑रियोज॒नꣳ श्री॒णाति॑ । \newline
8. हा॒रि॒यो॒ज॒नमिति॑ हारि - यो॒ज॒नम् । \newline
9. श्री॒णाति॑ शृत॒त्वाय॑ शृत॒त्वाय॑ श्री॒णाति॑ श्री॒णाति॑ शृत॒त्वाय॑ । \newline
10. शृ॒त॒त्वाय॑ शृ॒तꣳ शृ॒तꣳ शृ॑त॒त्वाय॑ शृत॒त्वाय॑ शृ॒तम् । \newline
11. शृ॒त॒त्वायेति॑ शृत - त्वाय॑ । \newline
12. शृ॒त मे॒वैव शृ॒तꣳ शृ॒त मे॒व । \newline
13. ए॒वैन॑ मेन मे॒वै वैन᳚म् । \newline
14. ए॒न॒म् भू॒तम् भू॒त मे॑न मेनम् भू॒तम् । \newline
15. भू॒तम् जु॑होति जुहोति भू॒तम् भू॒तम् जु॑होति । \newline
16. जु॒हो॒ति॒ ब॒ह्वीभि॑र् ब॒ह्वीभि॑र् जुहोति जुहोति ब॒ह्वीभिः॑ । \newline
17. ब॒ह्वीभिः॑ श्रीणाति श्रीणाति ब॒ह्वीभि॑र् ब॒ह्वीभिः॑ श्रीणाति । \newline
18. श्री॒णा॒ त्ये॒ताव॑ती रे॒ताव॑तीः श्रीणाति श्रीणा त्ये॒ताव॑तीः । \newline
19. ए॒ताव॑ती रे॒वै वैताव॑ती रे॒ताव॑ती रे॒व । \newline
20. ए॒वास्या᳚ स्यै॒वै वास्य॑ । \newline
21. अ॒स्या॒ मुष्मि॑न् न॒मुष्मि॑न् नस्यास्या॒ मुष्मिन्न्॑ । \newline
22. अ॒मुष्मि॑न् ॅलो॒के लो॒के॑ ऽमुष्मि॑न् न॒मुष्मि॑न् ॅलो॒के । \newline
23. लो॒के का॑म॒दुघाः᳚ काम॒दुघा॑ लो॒के लो॒के का॑म॒दुघाः᳚ । \newline
24. का॒म॒दुघा॑ भवन्ति भवन्ति काम॒दुघाः᳚ काम॒दुघा॑ भवन्ति । \newline
25. का॒म॒दुघा॒ इति॑ काम - दुघाः᳚ । \newline
26. भ॒व॒ न्त्यथो॒ अथो॑ भवन्ति भव॒ न्त्यथो᳚ । \newline
27. अथो॒ खलु॒ खल्वथो॒ अथो॒ खलु॑ । \newline
28. अथो॒ इत्यथो᳚ । \newline
29. खल्वा॑हु राहुः॒ खलु॒ खल्वा॑हुः । \newline
30. आ॒हु॒ रे॒ता ए॒ता आ॑हु राहु रे॒ताः । \newline
31. ए॒ता वै वा ए॒ता ए॒ता वै । \newline
32. वा इन्द्र॒ स्येन्द्र॑स्य॒ वै वा इन्द्र॑स्य । \newline
33. इन्द्र॑स्य॒ पृश्ञ॑यः॒ पृश्ञ॑य॒ इन्द्र॒ स्येन्द्र॑स्य॒ पृश्ञ॑यः । \newline
34. पृश्ञ॑यः काम॒दुघाः᳚ काम॒दुघाः॒ पृश्ञ॑यः॒ पृश्ञ॑यः काम॒दुघाः᳚ । \newline
35. का॒म॒दुघा॒ यद् यत् का॑म॒दुघाः᳚ काम॒दुघा॒ यत् । \newline
36. का॒म॒दुघा॒ इति॑ काम - दुघाः᳚ । \newline
37. यद्धा॑रियोज॒नीर्. हा॑रियोज॒नीर् यद् यद्धा॑रियोज॒नीः । \newline
38. हा॒रि॒यो॒ज॒नीरि तीति॑ हारियोज॒नीर्. हा॑रियोज॒नी रिति॑ । \newline
39. हा॒रि॒यो॒ज॒नीरिति॑ हारि - यो॒ज॒नीः । \newline
40. इति॒ तस्मा॒त् तस्मा॒ दितीति॒ तस्मा᳚त् । \newline
41. तस्मा᳚द् ब॒ह्वीभि॑र् ब॒ह्वीभि॒ स्तस्मा॒त् तस्मा᳚द् ब॒ह्वीभिः॑ । \newline
42. ब॒ह्वीभिः॑ श्रीणीयाच् छ्रीणीयाद् ब॒ह्वीभि॑र् ब॒ह्वीभिः॑ श्रीणीयात् । \newline
43. श्री॒णी॒या॒ दृ॒ख्सा॒मे ऋ॑ख्सा॒मे श्री॑णीयाच् छ्रीणीया दृख्सा॒मे । \newline
44. ऋ॒ख्सा॒मे वै वा ऋ॑ख्सा॒मे ऋ॑ख्सा॒मे वै । \newline
45. ऋ॒ख्सा॒मे इत्यृ॑क् - सा॒मे । \newline
46. वा इन्द्र॒ स्येन्द्र॑स्य॒ वै वा इन्द्र॑स्य । \newline
47. इन्द्र॑स्य॒ हरी॒ हरी॒ इन्द्र॒ स्येन्द्र॑स्य॒ हरी᳚ । \newline
48. हरी॑ सोम॒पानौ॑ सोम॒पानौ॒ हरी॒ हरी॑ सोम॒पानौ᳚ । \newline
49. हरी॒ इति॒ हरी᳚ । \newline
50. सो॒म॒पानौ॒ तयो॒ स्तयोः᳚ सोम॒पानौ॑ सोम॒पानौ॒ तयोः᳚ । \newline
51. सो॒म॒पाना॒विति॑ सोम - पानौ᳚ । \newline
52. तयोः᳚ परि॒धयः॑ परि॒धय॒ स्तयो॒ स्तयोः᳚ परि॒धयः॑ । \newline
53. प॒रि॒धय॑ आ॒धान॑ मा॒धान॑म् परि॒धयः॑ परि॒धय॑ आ॒धान᳚म् । \newline
54. प॒रि॒धय॒ इति॑ परि - धयः॑ । \newline
55. आ॒धानं॒ ॅयद् यदा॒धान॑ मा॒धानं॒ ॅयत् । \newline
56. आ॒धान॒मित्या᳚ - धान᳚म् । \newline
57. यदप्र॑हृ॒त्या प्र॑हृत्य॒ यद् यदप्र॑हृत्य । \newline
58. अप्र॑हृत्य परि॒धीन् प॑रि॒धी नप्र॑हृ॒त्या प्र॑हृत्य परि॒धीन् । \newline
59. अप्र॑हृ॒त्येत्यप्र॑ - हृ॒त्य॒ । \newline
60. प॒रि॒धीन् जु॑हु॒याज् जु॑हु॒यात् प॑रि॒धीन् प॑रि॒धीन् जु॑हु॒यात् । \newline
61. प॒रि॒धीनिति॑ परि - धीन् । \newline
62. जु॒हु॒या द॒न्तरा॑धानाभ्या म॒न्तरा॑धानाभ्याम् जुहु॒याज् जु॑हु॒या द॒न्तरा॑धानाभ्याम् । \newline
63. अ॒न्तरा॑धानाभ्याम् घा॒सम् घा॒स म॒न्तरा॑धानाभ्या म॒न्तरा॑धानाभ्याम् घा॒सम् । \newline
64. अ॒न्तरा॑धानाभ्या॒मित्य॒न्तः - आ॒धा॒ना॒भ्या॒म् । \newline

\textbf{Ghana Paata } \newline

1. तꣳ शृ॒तꣳ शृ॒तम् तम् तꣳ शृ॒तम् भू॒तम् भू॒तꣳ शृ॒तम् तम् तꣳ शृ॒तम् भू॒तम् । \newline
2. शृ॒तम् भू॒तम् भू॒तꣳ शृ॒तꣳ शृ॒तम् भू॒त म॑जुहो दजुहोद् भू॒तꣳ शृ॒तꣳ शृ॒तम् भू॒त म॑जुहोत् । \newline
3. भू॒त म॑जुहो दजुहोद् भू॒तम् भू॒त म॑जुहो॒द् यद् यद॑जुहोद् भू॒तम् भू॒त म॑जुहो॒द् यत् । \newline
4. अ॒जु॒हो॒द् यद् यद॑जुहो दजुहो॒द् यद् धा॒नाभि॑र् धा॒नाभि॒र् यद॑जुहो दजुहो॒द् यद् धा॒नाभिः॑ । \newline
5. यद् धा॒नाभि॑र् धा॒नाभि॒र् यद् यद् धा॒नाभि॑र्. हारियोज॒नꣳ हा॑रियोज॒नम् धा॒नाभि॒र् यद् यद् धा॒नाभि॑र्. हारियोज॒नम् । \newline
6. धा॒नाभि॑र्. हारियोज॒नꣳ हा॑रियोज॒नम् धा॒नाभि॑र् धा॒नाभि॑र्. हारियोज॒नꣳ श्री॒णाति॑ श्री॒णाति॑ हारियोज॒नम् धा॒नाभि॑र् धा॒नाभि॑र्. हारियोज॒नꣳ श्री॒णाति॑ । \newline
7. हा॒रि॒यो॒ज॒नꣳ श्री॒णाति॑ श्री॒णाति॑ हारियोज॒नꣳ हा॑रियोज॒नꣳ श्री॒णाति॑ शृत॒त्वाय॑ शृत॒त्वाय॑ श्री॒णाति॑ हारियोज॒नꣳ हा॑रियोज॒नꣳ श्री॒णाति॑ शृत॒त्वाय॑ । \newline
8. हा॒रि॒यो॒ज॒नमिति॑ हारि - यो॒ज॒नम् । \newline
9. श्री॒णाति॑ शृत॒त्वाय॑ शृत॒त्वाय॑ श्री॒णाति॑ श्री॒णाति॑ शृत॒त्वाय॑ शृ॒तꣳ शृ॒तꣳ शृ॑त॒त्वाय॑ श्री॒णाति॑ श्री॒णाति॑ शृत॒त्वाय॑ शृ॒तम् । \newline
10. शृ॒त॒त्वाय॑ शृ॒तꣳ शृ॒तꣳ शृ॑त॒त्वाय॑ शृत॒त्वाय॑ शृ॒त मे॒वैव शृ॒तꣳ शृ॑त॒त्वाय॑ शृत॒त्वाय॑ शृ॒त मे॒व । \newline
11. शृ॒त॒त्वायेति॑ शृत - त्वाय॑ । \newline
12. शृ॒त मे॒वैव शृ॒तꣳ शृ॒त मे॒वैन॑ मेन मे॒व शृ॒तꣳ शृ॒त मे॒वैन᳚म् । \newline
13. ए॒वैन॑ मेन मे॒वै वैन॑म् भू॒तम् भू॒त मे॑न मे॒वै वैन॑म् भू॒तम् । \newline
14. ए॒न॒म् भू॒तम् भू॒त मे॑न मेनम् भू॒तम् जु॑होति जुहोति भू॒त मे॑न मेनम् भू॒तम् जु॑होति । \newline
15. भू॒तम् जु॑होति जुहोति भू॒तम् भू॒तम् जु॑होति ब॒ह्वीभि॑र् ब॒ह्वीभि॑र् जुहोति भू॒तम् भू॒तम् जु॑होति ब॒ह्वीभिः॑ । \newline
16. जु॒हो॒ति॒ ब॒ह्वीभि॑र् ब॒ह्वीभि॑र् जुहोति जुहोति ब॒ह्वीभिः॑ श्रीणाति श्रीणाति ब॒ह्वीभि॑र् जुहोति जुहोति ब॒ह्वीभिः॑ श्रीणाति । \newline
17. ब॒ह्वीभिः॑ श्रीणाति श्रीणाति ब॒ह्वीभि॑र् ब॒ह्वीभिः॑ श्रीणा त्ये॒ताव॑ती रे॒ताव॑तीः श्रीणाति ब॒ह्वीभि॑र् ब॒ह्वीभिः॑ श्रीणा त्ये॒ताव॑तीः । \newline
18. श्री॒णा॒ त्ये॒ताव॑ती रे॒ताव॑तीः श्रीणाति श्रीणा त्ये॒ताव॑ती रे॒वै वैताव॑तीः श्रीणाति श्रीणा त्ये॒ताव॑ती रे॒व । \newline
19. ए॒ताव॑ती रे॒वै वैताव॑ती रे॒ताव॑ती रे॒वास्या᳚ स्यै॒वैताव॑ती रे॒ताव॑ती रे॒वास्य॑ । \newline
20. ए॒वास्या᳚ स्यै॒वै वास्या॒ मुष्मि॑न्-न॒मुष्मि॑न्-नस्यै॒ वैवास्या॒ मुष्मिन्न्॑ । \newline
21. अ॒स्या॒ मुष्मि॑न्-न॒मुष्मि॑न्-नस्यास्या॒ मुष्मि॑न् ॅलो॒के लो॒के॑ ऽमुष्मि॑न्-नस्यास्या॒ मुष्मि॑न् ॅलो॒के । \newline
22. अ॒मुष्मि॑न् ॅलो॒के लो॒के॑ ऽमुष्मि॑न्-न॒मुष्मि॑न् ॅलो॒के का॑म॒दुघाः᳚ काम॒दुघा॑ लो॒के॑ ऽमुष्मि॑न्-न॒मुष्मि॑न् ॅलो॒के का॑म॒दुघाः᳚ । \newline
23. लो॒के का॑म॒दुघाः᳚ काम॒दुघा॑ लो॒के लो॒के का॑म॒दुघा॑ भवन्ति भवन्ति काम॒दुघा॑ लो॒के लो॒के का॑म॒दुघा॑ भवन्ति । \newline
24. का॒म॒दुघा॑ भवन्ति भवन्ति काम॒दुघाः᳚ काम॒दुघा॑ भव॒न्त्यथो॒ अथो॑ भवन्ति काम॒दुघाः᳚ काम॒दुघा॑ भव॒न्त्यथो᳚ । \newline
25. का॒म॒दुघा॒ इति॑ काम - दुघाः᳚ । \newline
26. भ॒व॒न् त्यथो॒ अथो॑ भवन्ति भव॒न्त्यथो॒ खलु॒ खल्वथो॑ भवन्ति भव॒न्त्यथो॒ खलु॑ । \newline
27. अथो॒ खलु॒ खल्वथो॒ अथो॒ खल्वा॑हु राहुः॒ खल्वथो॒ अथो॒ खल्वा॑हुः । \newline
28. अथो॒ इत्यथो᳚ । \newline
29. खल्वा॑हु राहुः॒ खलु॒ खल्वा॑हु रे॒ता ए॒ता आ॑हुः॒ खलु॒ खल्वा॑हु रे॒ताः । \newline
30. आ॒हु॒ रे॒ता ए॒ता आ॑हु राहु रे॒ता वै वा ए॒ता आ॑हु राहु रे॒ता वै । \newline
31. ए॒ता वै वा ए॒ता ए॒ता वा इन्द्र॒ स्येन्द्र॑स्य॒ वा ए॒ता ए॒ता वा इन्द्र॑स्य । \newline
32. वा इन्द्र॒ स्येन्द्र॑स्य॒ वै वा इन्द्र॑स्य॒ पृश्ञ॑यः॒ पृश्ञ॑य॒ इन्द्र॑स्य॒ वै वा इन्द्र॑स्य॒ पृश्ञ॑यः । \newline
33. इन्द्र॑स्य॒ पृश्ञ॑यः॒ पृश्ञ॑य॒ इन्द्र॒ स्येन्द्र॑स्य॒ पृश्ञ॑यः काम॒दुघाः᳚ काम॒दुघाः॒ पृश्ञ॑य॒ इन्द्र॒ स्येन्द्र॑स्य॒ पृश्ञ॑यः काम॒दुघाः᳚ । \newline
34. पृश्ञ॑यः काम॒दुघाः᳚ काम॒दुघाः॒ पृश्ञ॑यः॒ पृश्ञ॑यः काम॒दुघा॒ यद् यत् का॑म॒दुघाः॒ पृश्ञ॑यः॒ पृश्ञ॑यः काम॒दुघा॒ यत् । \newline
35. का॒म॒दुघा॒ यद् यत् का॑म॒दुघाः᳚ काम॒दुघा॒ यद्धा॑रियोज॒नीर्. हा॑रियोज॒नीर् यत् का॑म॒दुघाः᳚ काम॒दुघा॒ यद्धा॑रियोज॒नीः । \newline
36. का॒म॒दुघा॒ इति॑ काम - दुघाः᳚ । \newline
37. यद्धा॑रियोज॒नीर्. हा॑रियोज॒नीर् यद् यद्धा॑रियोज॒नी रितीति॑ हारियोज॒नीर् यद् यद्धा॑रियोज॒नी रिति॑ । \newline
38. हा॒रि॒यो॒ज॒नी रितीति॑ हारियोज॒नीर्. हा॑रियोज॒नी रिति॒ तस्मा॒त् तस्मा॒ दिति॑ हारियोज॒नीर्. हा॑रियोज॒नी रिति॒ तस्मा᳚त् । \newline
39. हा॒रि॒यो॒ज॒नीरिति॑ हारि - यो॒ज॒नीः । \newline
40. इति॒ तस्मा॒त् तस्मा॒ दितीति॒ तस्मा᳚द् ब॒ह्वीभि॑र् ब॒ह्वीभि॒ स्तस्मा॒ दितीति॒ तस्मा᳚द् ब॒ह्वीभिः॑ । \newline
41. तस्मा᳚द् ब॒ह्वीभि॑र् ब॒ह्वीभि॒ स्तस्मा॒त् तस्मा᳚द् ब॒ह्वीभिः॑ श्रीणीयाच् छ्रीणीयाद् ब॒ह्वीभि॒ स्तस्मा॒त् तस्मा᳚द् ब॒ह्वीभिः॑ श्रीणीयात् । \newline
42. ब॒ह्वीभिः॑ श्रीणीयाच् छ्रीणीयाद् ब॒ह्वीभि॑र् ब॒ह्वीभिः॑ श्रीणीया दृख्सा॒मे ऋ॑ख्सा॒मे श्री॑णीयाद् ब॒ह्वीभि॑र् ब॒ह्वीभिः॑ श्रीणीया दृख्सा॒मे । \newline
43. श्री॒णी॒या॒ दृ॒ख्सा॒मे ऋ॑ख्सा॒मे श्री॑णीयाच् छ्रीणीया दृख्सा॒मे वै वा ऋ॑ख्सा॒मे श्री॑णीयाच् छ्रीणीया दृख्सा॒मे वै । \newline
44. ऋ॒ख्सा॒मे वै वा ऋ॑ख्सा॒मे ऋ॑ख्सा॒मे वा इन्द्र॒ स्येन्द्र॑स्य॒ वा ऋ॑ख्सा॒मे ऋ॑ख्सा॒मे वा इन्द्र॑स्य । \newline
45. ऋ॒ख्सा॒मे इत्यृ॑क् - सा॒मे । \newline
46. वा इन्द्र॒ स्येन्द्र॑स्य॒ वै वा इन्द्र॑स्य॒ हरी॒ हरी॒ इन्द्र॑स्य॒ वै वा इन्द्र॑स्य॒ हरी᳚ । \newline
47. इन्द्र॑स्य॒ हरी॒ हरी॒ इन्द्र॒ स्येन्द्र॑स्य॒ हरी॑ सोम॒पानौ॑ सोम॒पानौ॒ हरी॒ इन्द्र॒ स्येन्द्र॑स्य॒ हरी॑ सोम॒पानौ᳚ । \newline
48. हरी॑ सोम॒पानौ॑ सोम॒पानौ॒ हरी॒ हरी॑ सोम॒पानौ॒ तयो॒ स्तयोः᳚ सोम॒पानौ॒ हरी॒ हरी॑ सोम॒पानौ॒ तयोः᳚ । \newline
49. हरी॒ इति॒ हरी᳚ । \newline
50. सो॒म॒पानौ॒ तयो॒ स्तयोः᳚ सोम॒पानौ॑ सोम॒पानौ॒ तयोः᳚ परि॒धयः॑ परि॒धय॒ स्तयोः᳚ सोम॒पानौ॑ सोम॒पानौ॒ तयोः᳚ परि॒धयः॑ । \newline
51. सो॒म॒पाना॒विति॑ सोम - पानौ᳚ । \newline
52. तयोः᳚ परि॒धयः॑ परि॒धय॒ स्तयो॒ स्तयोः᳚ परि॒धय॑ आ॒धान॑ मा॒धान॑म् परि॒धय॒ स्तयो॒ स्तयोः᳚ परि॒धय॑ आ॒धान᳚म् । \newline
53. प॒रि॒धय॑ आ॒धान॑ मा॒धान॑म् परि॒धयः॑ परि॒धय॑ आ॒धानं॒ ॅयद् यदा॒धान॑म् परि॒धयः॑ परि॒धय॑ आ॒धानं॒ ॅयत् । \newline
54. प॒रि॒धय॒ इति॑ परि - धयः॑ । \newline
55. आ॒धानं॒ ॅयद् यदा॒धान॑ मा॒धानं॒ ॅयदप्र॑हृ॒त्या प्र॑हृत्य॒ यदा॒धान॑ मा॒धानं॒ ॅयदप्र॑हृत्य । \newline
56. आ॒धान॒मित्या᳚ - धान᳚म् । \newline
57. यदप्र॑हृ॒त्या प्र॑हृत्य॒ यद् यदप्र॑हृत्य परि॒धीन् प॑रि॒धी-नप्र॑हृत्य॒ यद् यदप्र॑हृत्य परि॒धीन् । \newline
58. अप्र॑हृत्य परि॒धीन् प॑रि॒धी-नप्र॑हृ॒त्या प्र॑हृत्य परि॒धीन् जु॑हु॒याज् जु॑हु॒यात् प॑रि॒धी-नप्र॑हृ॒त्या प्र॑हृत्य परि॒धीन् जु॑हु॒यात् । \newline
59. अप्र॑हृ॒त्येत्यप्र॑ - हृ॒त्य॒ । \newline
60. प॒रि॒धीन् जु॑हु॒याज् जु॑हु॒यात् प॑रि॒धीन् प॑रि॒धीन् जु॑हु॒या द॒न्तरा॑धानाभ्या म॒न्तरा॑धानाभ्याम् जुहु॒यात् प॑रि॒धीन् प॑रि॒धीन् जु॑हु॒या द॒न्तरा॑धानाभ्याम् । \newline
61. प॒रि॒धीनिति॑ परि - धीन् । \newline
62. जु॒हु॒या द॒न्तरा॑धानाभ्या म॒न्तरा॑धानाभ्याम् जुहु॒याज् जु॑हु॒या द॒न्तरा॑धानाभ्याम् घा॒सम् घा॒स म॒न्तरा॑धानाभ्याम् जुहु॒याज् जु॑हु॒या द॒न्तरा॑धानाभ्याम् घा॒सम् । \newline
63. अ॒न्तरा॑धानाभ्याम् घा॒सम् घा॒स म॒न्तरा॑धानाभ्या म॒न्तरा॑धानाभ्याम् घा॒सम् प्र प्र घा॒स म॒न्तरा॑धानाभ्या म॒न्तरा॑धानाभ्याम् घा॒सम् प्र । \newline
64. अ॒न्तरा॑धानाभ्या॒मित्य॒न्तः - आ॒धा॒ना॒भ्या॒म् । \newline
\pagebreak
\markright{ TS 6.5.9.3  \hfill https://www.vedavms.in \hfill}

\section{ TS 6.5.9.3 }

\textbf{TS 6.5.9.3 } \newline
\textbf{Samhita Paata} \newline

घा॒सं प्र य॑च्छेत् प्र॒हृत्य॑ परि॒धीञ्जु॑होति॒ निरा॑धानाभ्यामे॒व घा॒सं प्र य॑च्छत्युन्ने॒ता जु॑होति या॒तया॑मेव॒ ह्ये॑तर्ह्य॑द्ध्व॒र्युः स्व॒गाकृ॑तो॒ यद॑द्ध्व॒र्युर्जु॑हु॒याद्-यथा॒ विमु॑क्तं॒ पुन॑र्यु॒नक्ति॑ ता॒दृगे॒व तच्छी॒र्॒.ष-न्न॑धिनि॒धाय॑ जुहोति शीर्.ष॒तो हि स स॒मभ॑वद्-वि॒क्रम्य॑ जुहोति वि॒क्रम्य॒ हीन्द्रो॑ वृ॒त्रमह॒न्थ् समृ॑द्ध्यै प॒शवो॒ वै हा॑रियोज॒नीर्यथ् स॑भिं॒न्द्यादल्पा॑- [  ] \newline

\textbf{Pada Paata} \newline

घा॒सम् । प्रेति॑ । य॒च्छे॒त् । प्र॒हृत्येति॑ प्र - हृत्य॑ । प॒रि॒धीनिति॑ परि - धीन् । जु॒हो॒ति॒ । निरा॑धानाभ्या॒मिति॒ निः - आ॒धा॒ना॒भ्या॒म् । ए॒व । घा॒सम् । प्रेति॑ । य॒च्छ॒ति॒ । उ॒न्ने॒तेत्यु॑त् - ने॒ता । जु॒हो॒ति॒ । या॒तया॒मेति॑ या॒त - या॒मा॒ । इ॒व॒ । हि । ए॒तर्.हि॑ । अ॒द्ध्व॒र्युः । स्व॒गाकृ॑त॒ इति॑ स्व॒गा-कृ॒तः॒ । यत् । अ॒द्ध्व॒र्युः । जु॒हु॒यात् । यथा᳚ । विमु॑क्त॒मित्॒ वि - मु॒क्त॒म् । पुनः॑ । यु॒नक्ति॑ । ता॒दृक् । ए॒व । तत् । शी॒र्॒.षन्न् । अ॒धि॒नि॒धायेत्य॑धि - नि॒धाय॑ । जु॒हो॒ति॒ । शी॒र्॒.ष॒तः । हि । सः । स॒मभ॑व॒दिति॑ सं - अभ॑वत् । वि॒क्रम्येति॑ वि-क्रम्य॑ । जु॒हो॒ति॒ । वि॒क्रम्येति॑ वि - क्रम्य॑ । हि । इन्द्रः॑ । वृ॒त्रम् । अहन्न्॑ । समृ॑द्ध्या॒ इति॒ सम्-ऋ॒द्ध्यै॒ । प॒शवः॑ । वै । हा॒रि॒यो॒ज॒नीरिति॑ हारि-यो॒ज॒नीः । यत् । स॒भिं॒न्द्यादिति॑ सं - भि॒न्द्यात् । अल्पाः᳚ ।  \newline


\textbf{Krama Paata} \newline

घा॒सम् प्र । प्र य॑च्छेत् । य॒च्छे॒त् प्र॒हृत्य॑ । प्र॒हृत्य॑ परि॒धीन् । प्र॒हृत्येति॑ प्र - हृत्य॑ । प॒रि॒धीन् जु॑होति । प॒रि॒धीनिति॑ परि - धीन् । जु॒हो॒ति॒ निरा॑धानाभ्याम् । निरा॑धानाभ्यामे॒व । निरा॑धानाभ्या॒मिति॒ निः - आ॒धा॒ना॒भ्या॒म् । ए॒व घा॒सम् । घा॒सम् प्र । प्र य॑च्छति । य॒च्छ॒त्यु॒न्ने॒ता । उ॒न्ने॒ता जु॑होति । उ॒न्ने॒तेत्यु॑त् - ने॒ता । जु॒हो॒ति॒ या॒तया॑मा । या॒तया॑मेव । या॒तया॒मेति॑ या॒त - या॒मा॒ । इ॒व॒ हि । ह्ये॑तर्.हि॑ । ए॒तर्ह्य॑द्ध्व॒र्युः । अ॒द्ध्व॒र्युः स्व॒गाकृ॑तः । स्व॒गाकृ॑तो॒ यत् । स्व॒गाकृ॑त॒ इति॑ स्व॒गा - कृ॒तः॒ । यद॑द्ध्व॒र्युः । अ॒द्ध्व॒र्युर् जु॑हु॒यात् । जु॒हु॒याद् यथा᳚ । यथा॒ विमु॑क्तम् । विमु॑क्त॒म् पुनः॑ । विमु॑क्त॒मिति॒ वि - मु॒क्त॒म् । पुन॑र् यु॒नक्ति॑ । यु॒नक्ति॑ ता॒दृक् । ता॒दृगे॒व । ए॒व तत् । तच्छी॒र्.॒षन्न् । शी॒र्॒.षन्न॑धिनि॒धाय॑ । अ॒धि॒नि॒धाय॑ जुहोति । अ॒धि॒नि॒धायेत्य॑धि - नि॒धाय॑ । जु॒हो॒ति॒ शी॒र्॒.ष॒तः । शी॒र्॒.ष॒तो हि । हि सः । स स॒मभ॑वत् । स॒मभ॑वद् वि॒क्रम्य॑ । स॒मभ॑व॒दिति॑ सम् - अभ॑वत् । वि॒क्रम्य॑ जुहोति । वि॒क्रम्येति॑ वि - क्रम्य॑ । जु॒हो॒ति॒ वि॒क्रम्य॑ । वि॒क्रम्य॒ हि । वि॒क्रम्येति॑ वि - क्रम्य॑ । हीन्द्रः॑ । इन्द्रो॑ वृ॒त्रम् । वृ॒त्रमहन्न्॑ । अह॒न्थ् समृ॑द्ध्यै । समृ॑द्ध्यै प॒शवः॑ । समृ॑द्ध्या॒ इति॒ सम् - ऋ॒द्ध्यै॒ । प॒शवो॒ वै । वै हा॑रियोज॒नीः । हा॒रि॒यो॒ज॒नीर् यत् । हा॒रि॒यो॒ज॒नीरिति॑ हारि - यो॒ज॒नीः । यथ् स॑म्भि॒न्द्यात् । स॒म्भि॒न्द्यादल्पाः᳚ । स॒म्भि॒न्द्यादिति॑ सम् - भि॒न्द्यात् । अल्पा॑ एनम् \newline

\textbf{Jatai Paata} \newline

1. घा॒सम् प्र प्र घा॒सम् घा॒सम् प्र । \newline
2. प्र य॑च्छेद् यच्छे॒त् प्र प्र य॑च्छेत् । \newline
3. य॒च्छे॒त् प्र॒हृत्य॑ प्र॒हृत्य॑ यच्छेद् यच्छेत् प्र॒हृत्य॑ । \newline
4. प्र॒हृत्य॑ परि॒धीन् प॑रि॒धीन् प्र॒हृत्य॑ प्र॒हृत्य॑ परि॒धीन् । \newline
5. प्र॒हृत्येति॑ प्र - हृत्य॑ । \newline
6. प॒रि॒धीन् जु॑होति जुहोति परि॒धीन् प॑रि॒धीन् जु॑होति । \newline
7. प॒रि॒धीनिति॑ परि - धीन् । \newline
8. जु॒हो॒ति॒ निरा॑धानाभ्या॒म् निरा॑धानाभ्याम् जुहोति जुहोति॒ निरा॑धानाभ्याम् । \newline
9. निरा॑धानाभ्या मे॒वैव निरा॑धानाभ्या॒म् निरा॑धानाभ्या मे॒व । \newline
10. निरा॑धानाभ्या॒मिति॒ निः - आ॒धा॒ना॒भ्या॒म् । \newline
11. ए॒व घा॒सम् घा॒स मे॒वैव घा॒सम् । \newline
12. घा॒सम् प्र प्र घा॒सम् घा॒सम् प्र । \newline
13. प्र य॑च्छति यच्छति॒ प्र प्र य॑च्छति । \newline
14. य॒च्छ॒ त्यु॒न्ने॒ तोन्ने॒ता य॑च्छति यच्छ त्युन्ने॒ता । \newline
15. उ॒न्ने॒ता जु॑होति जुहो त्युन्ने॒ तोन्ने॒ता जु॑होति । \newline
16. उ॒न्ने॒तेत्यु॑त् - ने॒ता । \newline
17. जु॒हो॒ति॒ या॒तया॑मा या॒तया॑मा जुहोति जुहोति या॒तया॑मा । \newline
18. या॒तया॑ मेवेव या॒तया॑मा या॒तया॑मेव । \newline
19. या॒तया॒मेति॑ या॒त - या॒मा॒ । \newline
20. इ॒व॒ हि हीवे॑व॒ हि । \newline
21. ह्ये॑तर् ह्ये॒तर्.हि॒ हि ह्ये॑तर्.हि॑ । \newline
22. ए॒तर्. ह्य॑द्ध्व॒र्यु र॑द्ध्व॒र्यु रे॒तर् ह्ये॒तर् ह्य॑द्ध्व॒र्युः । \newline
23. अ॒द्ध्व॒र्युः स्व॒गाकृ॑तः स्व॒गाकृ॑तो ऽद्ध्व॒र्यु र॑द्ध्व॒र्युः स्व॒गाकृ॑तः । \newline
24. स्व॒गाकृ॑तो॒ यद् यथ् स्व॒गाकृ॑तः स्व॒गाकृ॑तो॒ यत् । \newline
25. स्व॒गाकृ॑त॒ इति॑ स्व॒गा - कृ॒तः॒ । \newline
26. यद॑द्ध्व॒र्यु र॑द्ध्व॒र्युर् यद् यद॑द्ध्व॒र्युः । \newline
27. अ॒द्ध्व॒र्युर् जु॑हु॒याज् जु॑हु॒या द॑द्ध्व॒र्यु र॑द्ध्व॒र्युर् जु॑हु॒यात् । \newline
28. जु॒हु॒याद् यथा॒ यथा॑ जुहु॒याज् जु॑हु॒याद् यथा᳚ । \newline
29. यथा॒ विमु॑क्तं॒ ॅविमु॑क्तं॒ ॅयथा॒ यथा॒ विमु॑क्तम् । \newline
30. विमु॑क्त॒म् पुनः॒ पुन॒र् विमु॑क्तं॒ ॅविमु॑क्त॒म् पुनः॑ । \newline
31. विमु॑क्त॒मिति॒ वि - मु॒क्त॒म् । \newline
32. पुन॑र् यु॒नक्ति॑ यु॒नक्ति॒ पुनः॒ पुन॑र् यु॒नक्ति॑ । \newline
33. यु॒नक्ति॑ ता॒दृक् ता॒दृग् यु॒नक्ति॑ यु॒नक्ति॑ ता॒दृक् । \newline
34. ता॒दृ गे॒वैव ता॒दृक् ता॒दृ गे॒व । \newline
35. ए॒व तत् तदे॒ वैव तत् । \newline
36. तच् छी॒र्॒.षञ् छी॒र्॒.षन् तत् तच् छी॒र्॒.षन्न् । \newline
37. शी॒र्॒.षन् न॑धिनि॒धाया॑ धिनि॒धाय॑ शी॒र्॒.षञ् छी॒र्॒.षन् न॑धिनि॒धाय॑ । \newline
38. अ॒धि॒नि॒धाय॑ जुहोति जुहो त्यधिनि॒धाया॑ धिनि॒धाय॑ जुहोति । \newline
39. अ॒धि॒नि॒धायेत्य॑धि - नि॒धाय॑ । \newline
40. जु॒हो॒ति॒ शी॒र्॒.ष॒तः शी॑र्.ष॒तो जु॑होति जुहोति शीर्.ष॒तः । \newline
41. शी॒र्॒.ष॒तो हि हि शी॑र्.ष॒तः शी॑र्.ष॒तो हि । \newline
42. हि स स हि हि सः । \newline
43. स स॒मभ॑वथ् स॒मभ॑व॒थ् स स स॒मभ॑वत् । \newline
44. स॒मभ॑वद् वि॒क्रम्य॑ वि॒क्रम्य॑ स॒मभ॑वथ् स॒मभ॑वद् वि॒क्रम्य॑ । \newline
45. स॒मभ॑व॒दिति॑ सं - अभ॑वत् । \newline
46. वि॒क्रम्य॑ जुहोति जुहोति वि॒क्रम्य॑ वि॒क्रम्य॑ जुहोति । \newline
47. वि॒क्रम्येति॑ वि - क्रम्य॑ । \newline
48. जु॒हो॒ति॒ वि॒क्रम्य॑ वि॒क्रम्य॑ जुहोति जुहोति वि॒क्रम्य॑ । \newline
49. वि॒क्रम्य॒ हि हि वि॒क्रम्य॑ वि॒क्रम्य॒ हि । \newline
50. वि॒क्रम्येति॑ वि - क्रम्य॑ । \newline
51. हीन्द्र॒ इन्द्रो॒ हि हीन्द्रः॑ । \newline
52. इन्द्रो॑ वृ॒त्रं ॅवृ॒त्र मिन्द्र॒ इन्द्रो॑ वृ॒त्रम् । \newline
53. वृ॒त्र मह॒न् नह॑न् वृ॒त्रं ॅवृ॒त्र महन्न्॑ । \newline
54. अह॒न् थ्समृ॑द्ध्यै॒ समृ॑द्ध्या॒ अह॒न् नह॒न् थ्समृ॑द्ध्यै । \newline
55. समृ॑द्ध्यै प॒शवः॑ प॒शवः॒ समृ॑द्ध्यै॒ समृ॑द्ध्यै प॒शवः॑ । \newline
56. समृ॑द्ध्या॒ इति॒ सम् - ऋ॒द्ध्यै॒ । \newline
57. प॒शवो॒ वै वै प॒शवः॑ प॒शवो॒ वै । \newline
58. वै हा॑रियोज॒नीर्. हा॑रियोज॒नीर् वै वै हा॑रियोज॒नीः । \newline
59. हा॒रि॒यो॒ज॒नीर् यद् यद्धा॑रियोज॒नीर्. हा॑रियोज॒नीर् यत् । \newline
60. हा॒रि॒यो॒ज॒नीरिति॑ हारि - यो॒ज॒नीः । \newline
61. यथ् सं॑भि॒न्द्याथ् सं॑भि॒न्द्याद् यद् यथ् सं॑भि॒न्द्यात् । \newline
62. सं॒भि॒न्द्या दल्पा॒ अल्पाः᳚ संभि॒न्द्याथ् सं॑भि॒न्द्या दल्पाः᳚ । \newline
63. सं॒भि॒न्द्यादिति॑ सं - भि॒न्द्यात् । \newline
64. अल्पा॑ एन मेन॒ मल्पा॒ अल्पा॑ एनम् । \newline

\textbf{Ghana Paata } \newline

1. घा॒सम् प्र प्र घा॒सम् घा॒सम् प्र य॑च्छेद् यच्छे॒त् प्र घा॒सम् घा॒सम् प्र य॑च्छेत् । \newline
2. प्र य॑च्छेद् यच्छे॒त् प्र प्र य॑च्छेत् प्र॒हृत्य॑ प्र॒हृत्य॑ यच्छे॒त् प्र प्र य॑च्छेत् प्र॒हृत्य॑ । \newline
3. य॒च्छे॒त् प्र॒हृत्य॑ प्र॒हृत्य॑ यच्छेद् यच्छेत् प्र॒हृत्य॑ परि॒धीन् प॑रि॒धीन् प्र॒हृत्य॑ यच्छेद् यच्छेत् प्र॒हृत्य॑ परि॒धीन् । \newline
4. प्र॒हृत्य॑ परि॒धीन् प॑रि॒धीन् प्र॒हृत्य॑ प्र॒हृत्य॑ परि॒धीन् जु॑होति जुहोति परि॒धीन् प्र॒हृत्य॑ प्र॒हृत्य॑ परि॒धीन् जु॑होति । \newline
5. प्र॒हृत्येति॑ प्र - हृत्य॑ । \newline
6. प॒रि॒धीन् जु॑होति जुहोति परि॒धीन् प॑रि॒धीन् जु॑होति॒ निरा॑धानाभ्या॒म् निरा॑धानाभ्याम् जुहोति परि॒धीन् प॑रि॒धीन् जु॑होति॒ निरा॑धानाभ्याम् । \newline
7. प॒रि॒धीनिति॑ परि - धीन् । \newline
8. जु॒हो॒ति॒ निरा॑धानाभ्या॒म् निरा॑धानाभ्याम् जुहोति जुहोति॒ निरा॑धानाभ्या मे॒वैव निरा॑धानाभ्याम् जुहोति जुहोति॒ निरा॑धानाभ्या मे॒व । \newline
9. निरा॑धानाभ्या मे॒वैव निरा॑धानाभ्या॒म् निरा॑धानाभ्या मे॒व घा॒सम् घा॒स मे॒व निरा॑धानाभ्या॒म् निरा॑धानाभ्या मे॒व घा॒सम् । \newline
10. निरा॑धानाभ्या॒मिति॒ निः - आ॒धा॒ना॒भ्या॒म् । \newline
11. ए॒व घा॒सम् घा॒स मे॒वैव घा॒सम् प्र प्र घा॒स मे॒वैव घा॒सम् प्र । \newline
12. घा॒सम् प्र प्र घा॒सम् घा॒सम् प्र य॑च्छति यच्छति॒ प्र घा॒सम् घा॒सम् प्र य॑च्छति । \newline
13. प्र य॑च्छति यच्छति॒ प्र प्र य॑च्छ त्युन्ने॒ तोन्ने॒ता य॑च्छति॒ प्र प्र य॑च्छ त्युन्ने॒ता । \newline
14. य॒च्छ॒ त्यु॒न्ने॒ तोन्ने॒ता य॑च्छति यच्छ त्युन्ने॒ता जु॑होति जुहो त्युन्ने॒ता य॑च्छति यच्छ त्युन्ने॒ता जु॑होति । \newline
15. उ॒न्ने॒ता जु॑होति जुहो त्युन्ने॒ तोन्ने॒ता जु॑होति या॒तया॑मा या॒तया॑मा जुहो त्युन्ने॒ तोन्ने॒ता जु॑होति या॒तया॑मा । \newline
16. उ॒न्ने॒तेत्यु॑त् - ने॒ता । \newline
17. जु॒हो॒ति॒ या॒तया॑मा या॒तया॑मा जुहोति जुहोति या॒तया॑मेवेव या॒तया॑मा जुहोति जुहोति या॒तया॑मेव । \newline
18. या॒तया॑मेवेव या॒तया॑मा या॒तया॑मेव॒ हि हीव॑ या॒तया॑मा या॒तया॑मेव॒ हि । \newline
19. या॒तया॒मेति॑ या॒त - या॒मा॒ । \newline
20. इ॒व॒ हि हीवे॑व॒ ह्ये॑तर् ह्ये॒तर्.हि॒ हीवे॑व॒ ह्ये॑तर्.हि॑ । \newline
21. ह्ये॑तर् ह्ये॒तर्.हि॒ हि ह्ये॑तर् ह्य॑द्ध्व॒र्यु र॑द्ध्व॒र्यु रे॒तर्.हि॒ हि ह्ये॑तर् ह्य॑द्ध्व॒र्युः । \newline
22. ए॒तर् ह्य॑द्ध्व॒र्यु र॑द्ध्व॒र्यु रे॒तर् ह्ये॒तर् ह्य॑द्ध्व॒र्युः स्व॒गाकृ॑तः स्व॒गाकृ॑तो ऽद्ध्व॒र्यु रे॒तर् ह्ये॒तर् ह्य॑द्ध्व॒र्युः स्व॒गाकृ॑तः । \newline
23. अ॒द्ध्व॒र्युः स्व॒गाकृ॑तः स्व॒गाकृ॑तो ऽद्ध्व॒र्यु र॑द्ध्व॒र्युः स्व॒गाकृ॑तो॒ यद् यथ् स्व॒गाकृ॑तो ऽद्ध्व॒र्यु र॑द्ध्व॒र्युः स्व॒गाकृ॑तो॒ यत् । \newline
24. स्व॒गाकृ॑तो॒ यद् यथ् स्व॒गाकृ॑तः स्व॒गाकृ॑तो॒ यद॑द्ध्व॒र्यु र॑द्ध्व॒र्युर् यथ् स्व॒गाकृ॑तः स्व॒गाकृ॑तो॒ यद॑द्ध्व॒र्युः । \newline
25. स्व॒गाकृ॑त॒ इति॑ स्व॒गा - कृ॒तः॒ । \newline
26. यद॑द्ध्व॒र्यु र॑द्ध्व॒र्युर् यद् यद॑द्ध्व॒र्युर् जु॑हु॒याज् जु॑हु॒या द॑द्ध्व॒र्युर् यद् यद॑द्ध्व॒र्युर् जु॑हु॒यात् । \newline
27. अ॒द्ध्व॒र्युर् जु॑हु॒याज् जु॑हु॒या द॑द्ध्व॒र्यु र॑द्ध्व॒र्युर् जु॑हु॒याद् यथा॒ यथा॑ जुहु॒या द॑द्ध्व॒र्यु र॑द्ध्व॒र्युर् जु॑हु॒याद् यथा᳚ । \newline
28. जु॒हु॒याद् यथा॒ यथा॑ जुहु॒याज् जु॑हु॒याद् यथा॒ विमु॑क्तं॒ ॅविमु॑क्तं॒ ॅयथा॑ जुहु॒याज् जु॑हु॒याद् यथा॒ विमु॑क्तम् । \newline
29. यथा॒ विमु॑क्तं॒ ॅविमु॑क्तं॒ ॅयथा॒ यथा॒ विमु॑क्त॒म् पुनः॒ पुन॒र् विमु॑क्तं॒ ॅयथा॒ यथा॒ विमु॑क्त॒म् पुनः॑ । \newline
30. विमु॑क्त॒म् पुनः॒ पुन॒र् विमु॑क्तं॒ ॅविमु॑क्त॒म् पुन॑र् यु॒नक्ति॑ यु॒नक्ति॒ पुन॒र् विमु॑क्तं॒ ॅविमु॑क्त॒म् पुन॑र् यु॒नक्ति॑ । \newline
31. विमु॑क्त॒मिति॒ वि - मु॒क्त॒म् । \newline
32. पुन॑र् यु॒नक्ति॑ यु॒नक्ति॒ पुनः॒ पुन॑र् यु॒नक्ति॑ ता॒दृक् ता॒दृग् यु॒नक्ति॒ पुनः॒ पुन॑र् यु॒नक्ति॑ ता॒दृक् । \newline
33. यु॒नक्ति॑ ता॒दृक् ता॒दृग् यु॒नक्ति॑ यु॒नक्ति॑ ता॒दृ गे॒वैव ता॒दृग् यु॒नक्ति॑ यु॒नक्ति॑ ता॒दृ गे॒व । \newline
34. ता॒दृ गे॒वैव ता॒दृक् ता॒दृ गे॒व तत् तदे॒व ता॒दृक् ता॒दृ गे॒व तत् । \newline
35. ए॒व तत् तदे॒ वैव तच्छी॒र्॒.षञ् छी॒र्॒.षन् तदे॒ वैव तच्छी॒र्॒.षन्न् । \newline
36. तच्छी॒र्॒.षञ् छी॒र्॒.षन् तत् तच्छी॒र्॒.षन्-न॑धिनि॒धाया॑ धिनि॒धाय॑ शी॒र्॒.षन् तत् तच्छी॒र्॒.षन्-न॑धिनि॒धाय॑ । \newline
37. शी॒र्॒.षन्-न॑धिनि॒धाया॑ धिनि॒धाय॑ शी॒र्॒.षञ् छी॒र्॒.षन्-न॑धिनि॒धाय॑ जुहोति जुहो त्यधिनि॒धाय॑ शी॒र्॒.षञ् छी॒र्॒.षन्-न॑धिनि॒धाय॑ जुहोति । \newline
38. अ॒धि॒नि॒धाय॑ जुहोति जुहो त्यधिनि॒धाया॑ धिनि॒धाय॑ जुहोति शीर्.ष॒तः शी॑र्.ष॒तो जु॑हो त्यधिनि॒धाया॑ धिनि॒धाय॑ जुहोति शीर्.ष॒तः । \newline
39. अ॒धि॒नि॒धायेत्य॑धि - नि॒धाय॑ । \newline
40. जु॒हो॒ति॒ शी॒र्॒.ष॒तः शी॑र्.ष॒तो जु॑होति जुहोति शीर्.ष॒तो हि हि शी॑र्.ष॒तो जु॑होति जुहोति शीर्.ष॒तो हि । \newline
41. शी॒र्॒.ष॒तो हि हि शी॑र्.ष॒तः शी॑र्.ष॒तो हि स स हि शी॑र्.ष॒तः शी॑र्.ष॒तो हि सः । \newline
42. हि स स हि हि स स॒मभ॑वथ् स॒मभ॑व॒थ् स हि हि स स॒मभ॑वत् । \newline
43. स स॒मभ॑वथ् स॒मभ॑व॒थ् स स स॒मभ॑वद् वि॒क्रम्य॑ वि॒क्रम्य॑ स॒मभ॑व॒थ् स स स॒मभ॑वद् वि॒क्रम्य॑ । \newline
44. स॒मभ॑वद् वि॒क्रम्य॑ वि॒क्रम्य॑ स॒मभ॑वथ् स॒मभ॑वद् वि॒क्रम्य॑ जुहोति जुहोति वि॒क्रम्य॑ स॒मभ॑वथ् स॒मभ॑वद् वि॒क्रम्य॑ जुहोति । \newline
45. स॒मभ॑व॒दिति॑ सं - अभ॑वत् । \newline
46. वि॒क्रम्य॑ जुहोति जुहोति वि॒क्रम्य॑ वि॒क्रम्य॑ जुहोति वि॒क्रम्य॑ वि॒क्रम्य॑ जुहोति वि॒क्रम्य॑ वि॒क्रम्य॑ जुहोति वि॒क्रम्य॑ । \newline
47. वि॒क्रम्येति॑ वि - क्रम्य॑ । \newline
48. जु॒हो॒ति॒ वि॒क्रम्य॑ वि॒क्रम्य॑ जुहोति जुहोति वि॒क्रम्य॒ हि हि वि॒क्रम्य॑ जुहोति जुहोति वि॒क्रम्य॒ हि । \newline
49. वि॒क्रम्य॒ हि हि वि॒क्रम्य॑ वि॒क्रम्य॒ हीन्द्र॒ इन्द्रो॒ हि वि॒क्रम्य॑ वि॒क्रम्य॒ हीन्द्रः॑ । \newline
50. वि॒क्रम्येति॑ वि - क्रम्य॑ । \newline
51. हीन्द्र॒ इन्द्रो॒ हि हीन्द्रो॑ वृ॒त्रं ॅवृ॒त्र मिन्द्रो॒ हि हीन्द्रो॑ वृ॒त्रम् । \newline
52. इन्द्रो॑ वृ॒त्रं ॅवृ॒त्र मिन्द्र॒ इन्द्रो॑ वृ॒त्र मह॒न्-नह॑न् वृ॒त्र मिन्द्र॒ इन्द्रो॑ वृ॒त्र महन्न्॑ । \newline
53. वृ॒त्र मह॒न्-नह॑न् वृ॒त्रं ॅवृ॒त्र मह॒न् थ्समृ॑द्ध्यै॒ समृ॑द्ध्या॒ अह॑न् वृ॒त्रं ॅवृ॒त्र मह॒न् थ्समृ॑द्ध्यै । \newline
54. अह॒न् थ्समृ॑द्ध्यै॒ समृ॑द्ध्या॒ अह॒न्-नह॒न् थ्समृ॑द्ध्यै प॒शवः॑ प॒शवः॒ समृ॑द्ध्या॒ अह॒न्-नह॒न् थ्समृ॑द्ध्यै प॒शवः॑ । \newline
55. समृ॑द्ध्यै प॒शवः॑ प॒शवः॒ समृ॑द्ध्यै॒ समृ॑द्ध्यै प॒शवो॒ वै वै प॒शवः॒ समृ॑द्ध्यै॒ समृ॑द्ध्यै प॒शवो॒ वै । \newline
56. समृ॑द्ध्या॒ इति॒ सम् - ऋ॒द्ध्यै॒ । \newline
57. प॒शवो॒ वै वै प॒शवः॑ प॒शवो॒ वै हा॑रियोज॒नीर्. हा॑रियोज॒नीर् वै प॒शवः॑ प॒शवो॒ वै हा॑रियोज॒नीः । \newline
58. वै हा॑रियोज॒नीर्. हा॑रियोज॒नीर् वै वै हा॑रियोज॒नीर् यद् यद्धा॑रियोज॒नीर् वै वै हा॑रियोज॒नीर् यत् । \newline
59. हा॒रि॒यो॒ज॒नीर् यद् यद्धा॑रियोज॒नीर्. हा॑रियोज॒नीर् यथ् सं॑भि॒न्द्याथ् सं॑भि॒न्द्याद् यद्धा॑रियोज॒नीर्. हा॑रियोज॒नीर् यथ् सं॑भि॒न्द्यात् । \newline
60. हा॒रि॒यो॒ज॒नीरिति॑ हारि - यो॒ज॒नीः । \newline
61. यथ् सं॑भि॒न्द्याथ् सं॑भि॒न्द्याद् यद् यथ् सं॑भि॒न्द्या दल्पा॒ अल्पाः᳚ संभि॒न्द्याद् यद् यथ् सं॑भि॒न्द्या दल्पाः᳚ । \newline
62. सं॒भि॒न्द्या दल्पा॒ अल्पाः᳚ संभि॒न्द्याथ् सं॑भि॒न्द्या दल्पा॑ एन मेन॒ मल्पाः᳚ संभि॒न्द्याथ् सं॑भि॒न्द्या दल्पा॑ एनम् । \newline
63. सं॒भि॒न्द्यादिति॑ सं - भि॒न्द्यात् । \newline
64. अल्पा॑ एन मेन॒ मल्पा॒ अल्पा॑ एनम् प॒शवः॑ प॒शव॑ एन॒ मल्पा॒ अल्पा॑ एनम् प॒शवः॑ । \newline
\pagebreak
\markright{ TS 6.5.9.4  \hfill https://www.vedavms.in \hfill}

\section{ TS 6.5.9.4 }

\textbf{TS 6.5.9.4 } \newline
\textbf{Samhita Paata} \newline

एनं प॒शवो॑ भु॒ञ्जन्त॒ उप॑तिष्ठेर॒न्॒. यन्न स॑भिं॒न्द्याद्-ब॒हव॑ एनं प॒शवोऽभु॑ञ्जन्त॒ उप॑ तिष्ठेर॒न् मन॑सा॒ सं बा॑धत उ॒भयं॑ करोति ब॒हव॑ ए॒वैनं॑ प॒शवो॑ भु॒ञ्जन्त॒ उप॑ तिष्ठन्त उन्ने॒तर्यु॑पह॒वमि॑च्छन्ते॒ य ए॒व तत्र॑ सोमपी॒थस्तमे॒वाव॑ रुन्धत उत्तरवे॒द्यां निव॑पति प॒शवो॒ वा उ॑त्तरवे॒दिः प॒शवो॑ हारियोज॒नीः प॒शुष्वे॒व प॒शून् प्रति॑ ष्ठापयन्ति ( ) ॥ \newline

\textbf{Pada Paata} \newline

ए॒न॒म् । प॒शवः॑ । भु॒ञ्जन्तः॑ । उपेति॑ । ति॒ष्ठे॒र॒न्न् । यत् । न । स॒भिं॒न्द्यादिति॑ सं - भि॒न्द्यात् । ब॒हवः॑ । ए॒न॒म् । प॒शवः॑ । अभु॑ञ्जन्तः । उपेति॑ । ति॒ष्ठे॒र॒न्न् । मन॑सा । समिति॑ । बा॒ध॒ते॒ । उ॒भय᳚म् । क॒रो॒ति॒ । ब॒हवः॑ । ए॒व । ए॒न॒म् । प॒शवः॑ । भु॒ञ्जन्तः॑ । उपेति॑ । ति॒ष्ठ॒न्ते॒ । उ॒न्ने॒तरीत्यु॑त् - ने॒तरि॑ । उ॒प॒ह॒वमित्यु॑प - ह॒वम् । इ॒च्छ॒न्ते॒ । यः । ए॒व । तत्र॑ । सो॒म॒पी॒थ इति॑ सोम - पी॒थः । तम् । ए॒व । अवेति॑ । रु॒न्ध॒ते॒ । उ॒त्त॒र॒वे॒द्यामित्यु॑त्तर - वे॒द्याम् । नीति॑ । व॒प॒ति॒ । प॒शवः॑ । वै । उ॒त्त॒र॒वे॒दिरित्यु॑त्तर - वे॒दिः । प॒शवः॑ । हा॒रि॒यो॒ज॒नीरिति॑ हारि - यो॒ज॒नीः । प॒शुषु॑ । ए॒व । प॒शून् । प्रतीति॑ । स्था॒प॒य॒न्ति॒ ( ) ॥  \newline


\textbf{Krama Paata} \newline

ए॒न॒म् प॒शवः॑ । प॒शवो॑ भु॒ञ्जन्तः॑ । भु॒ञ्जन्त॒ उप॑ । उप॑ तिष्ठेरन्न् । ति॒ष्ठे॒र॒न्॒. यत् । यन् न । न स॑म्भि॒न्द्यात् । स॒म्भि॒न्द्याद् ब॒हवः॑ । स॒म्भि॒न्द्यादि॑ति सम् - भि॒न्द्यात् । ब॒हव॑ एनम् । ए॒न॒म् प॒शवः॑ । प॒शवोऽभु॑ञ्जन्तः । अभु॑ञ्जन्त॒ उप॑ । उप॑ तिष्ठेरन्न् । ति॒ष्ठे॒र॒न् मन॑सा । मन॑सा॒ सम् । सम् बा॑धते । बा॒ध॒त॒ उ॒भय᳚म् । उ॒भय॑म् करोति । क॒रो॒ति॒ ब॒हवः॑ । ब॒हव॑ ए॒व । ए॒वैन᳚म् । ए॒न॒म् प॒शवः॑ । 
प॒शवो॑ भु॒ञ्जन्तः॑ । भु॒ञ्जन्त॒ उप॑ । उप॑ तिष्ठन्ते । ति॒ष्ठ॒न्त॒ उ॒न्ने॒तरि॑ । उ॒न्ने॒तर्यु॑पह॒वम् । उ॒न्ने॒तरीत्यु॑त् - ने॒तरि॑ । उ॒प॒ह॒वमि॑च्छन्ते । उ॒प॒ह॒वमित्यु॑प - ह॒वम् । इ॒च्छ॒न्ते॒ यः । य ए॒व । ए॒व तत्र॑ । तत्र॑ सोमपी॒थः । सो॒म॒पी॒थस्तम् । सो॒म॒पी॒थ इति॑ सोम - पी॒थः । तमे॒व । ए॒वाव॑ । अव॑ रुन्धते । रु॒न्ध॒त॒ उ॒त्त॒र॒वे॒द्याम् । उ॒त्त॒र॒वे॒द्याम् नि । उ॒त्त॒र॒वे॒द्यामित्यु॑त्तर - वे॒द्याम् । नि व॑पति । व॒प॒ति॒ प॒शवः॑ । प॒शवो॒ वै । वा उ॑त्तरवे॒दिः । उ॒त्त॒र॒वे॒दिः प॒शवः॑ । उ॒त्त॒र॒वे॒दिरित्यु॑त्तर - वे॒दिः । प॒शवो॑ हारियोज॒नीः । हा॒रि॒यो॒ज॒नीः प॒शुषु॑ । हा॒रि॒यो॒ज॒नीरिति॑ हारि - यो॒ज॒नीः । प॒शुष्वे॒व । ए॒व प॒शून् । प॒शून् प्रति॑ । प्रति॑ ष्ठापयन्ति ( ) । स्था॒प॒य॒न्तीति॑ स्थापयन्ति । \newline

\textbf{Jatai Paata} \newline

1. ए॒न॒म् प॒शवः॑ प॒शव॑ एन मेनम् प॒शवः॑ । \newline
2. प॒शवो॑ भु॒ञ्जन्तो॑ भु॒ञ्जन्तः॑ प॒शवः॑ प॒शवो॑ भु॒ञ्जन्तः॑ । \newline
3. भु॒ञ्जन्त॒ उपोप॑ भु॒ञ्जन्तो॑ भु॒ञ्जन्त॒ उप॑ । \newline
4. उप॑ तिष्ठेरन् तिष्ठेर॒न् नुपोप॑ तिष्ठेरन्न् । \newline
5. ति॒ष्ठे॒र॒न्॒. यद् यत् ति॑ष्ठेरन् तिष्ठेर॒न्॒. यत् । \newline
6. यन् न न यद् यन् न । \newline
7. न सं॑भि॒न्द्याथ् सं॑भि॒न्द्यान् न न सं॑भि॒न्द्यात् । \newline
8. सं॒भि॒न्द्याद् ब॒हवो॑ ब॒हवः॑ संभि॒न्द्याथ् सं॑भि॒न्द्याद् ब॒हवः॑ । \newline
9. सं॒भि॒न्द्यादिति॑ सं - भि॒न्द्यात् । \newline
10. ब॒हव॑ एन मेनम् ब॒हवो॑ ब॒हव॑ एनम् । \newline
11. ए॒न॒म् प॒शवः॑ प॒शव॑ एन मेनम् प॒शवः॑ । \newline
12. प॒शवो ऽभु॑ञ्ज॒न्तो ऽभु॑ञ्जन्तः प॒शवः॑ प॒शवो ऽभु॑ञ्जन्तः । \newline
13. अभु॑ञ्जन्त॒ उपोपा भु॑ञ्ज॒न्तो ऽभु॑ञ्जन्त॒ उप॑ । \newline
14. उप॑ तिष्ठेरन् तिष्ठेर॒न् नुपोप॑ तिष्ठेरन्न् । \newline
15. ति॒ष्ठे॒र॒न् मन॑सा॒ मन॑सा तिष्ठेरन् तिष्ठेर॒न् मन॑सा । \newline
16. मन॑सा॒ सꣳ सम् मन॑सा॒ मन॑सा॒ सम् । \newline
17. सम् बा॑धते बाधते॒ सꣳ सम् बा॑धते । \newline
18. बा॒ध॒त॒ उ॒भय॑ मु॒भय॑म् बाधते बाधत उ॒भय᳚म् । \newline
19. उ॒भय॑म् करोति करो त्यु॒भय॑ मु॒भय॑म् करोति । \newline
20. क॒रो॒ति॒ ब॒हवो॑ ब॒हवः॑ करोति करोति ब॒हवः॑ । \newline
21. ब॒हव॑ ए॒वैव ब॒हवो॑ ब॒हव॑ ए॒व । \newline
22. ए॒वैन॑ मेन मे॒वै वैन᳚म् । \newline
23. ए॒न॒म् प॒शवः॑ प॒शव॑ एन मेनम् प॒शवः॑ । \newline
24. प॒शवो॑ भु॒ञ्जन्तो॑ भु॒ञ्जन्तः॑ प॒शवः॑ प॒शवो॑ भु॒ञ्जन्तः॑ । \newline
25. भु॒ञ्जन्त॒ उपोप॑ भु॒ञ्जन्तो॑ भु॒ञ्जन्त॒ उप॑ । \newline
26. उप॑ तिष्ठन्ते तिष्ठन्त॒ उपोप॑ तिष्ठन्ते । \newline
27. ति॒ष्ठ॒न्त॒ उ॒न्ने॒तर् यु॑न्ने॒तरि॑ तिष्ठन्ते तिष्ठन्त उन्ने॒तरि॑ । \newline
28. उ॒न्ने॒तर् यु॑पह॒व मु॑पह॒व मु॑न्ने॒तर् यु॑न्ने॒तर् यु॑पह॒वम् । \newline
29. उ॒न्ने॒तरीत्यु॑त् - ने॒तरि॑ । \newline
30. उ॒प॒ह॒व मि॑च्छन्त इच्छन्त उपह॒व मु॑पह॒व मि॑च्छन्ते । \newline
31. उ॒प॒ह॒वमित्यु॑प - ह॒वम् । \newline
32. इ॒च्छ॒न्ते॒ यो य इ॑च्छन्त इच्छन्ते॒ यः । \newline
33. य ए॒वैव यो य ए॒व । \newline
34. ए॒व तत्र॒ तत्रै॒ वैव तत्र॑ । \newline
35. तत्र॑ सोमपी॒थः सो॑मपी॒थ स्तत्र॒ तत्र॑ सोमपी॒थः । \newline
36. सो॒म॒पी॒थ स्तम् तꣳ सो॑मपी॒थः सो॑मपी॒थ स्तम् । \newline
37. सो॒म॒पी॒थ इति॑ सोम - पी॒थः । \newline
38. तमे॒वैव तम् तमे॒व । \newline
39. ए॒वावा वै॒वै वाव॑ । \newline
40. अव॑ रुन्धते रुन्ध॒ते ऽवाव॑ रुन्धते । \newline
41. रु॒न्ध॒त॒ उ॒त्त॒र॒वे॒द्या मु॑त्तरवे॒द्याꣳ रु॑न्धते रुन्धत उत्तरवे॒द्याम् । \newline
42. उ॒त्त॒र॒वे॒द्यान् नि न्यु॑त्तरवे॒द्या मु॑त्तरवे॒द्यान् नि । \newline
43. उ॒त्त॒र॒वे॒द्यामित्यु॑त्तर - वे॒द्याम् । \newline
44. नि व॑पति वपति॒ नि नि व॑पति । \newline
45. व॒प॒ति॒ प॒शवः॑ प॒शवो॑ वपति वपति प॒शवः॑ । \newline
46. प॒शवो॒ वै वै प॒शवः॑ प॒शवो॒ वै । \newline
47. वा उ॑त्तरवे॒दि रु॑त्तरवे॒दिर् वै वा उ॑त्तरवे॒दिः । \newline
48. उ॒त्त॒र॒वे॒दिः प॒शवः॑ प॒शव॑ उत्तरवे॒दि रु॑त्तरवे॒दिः प॒शवः॑ । \newline
49. उ॒त्त॒र॒वे॒दिरित्यु॑त्तर - वे॒दिः । \newline
50. प॒शवो॑ हारियोज॒नीर्. हा॑रियोज॒नीः प॒शवः॑ प॒शवो॑ हारियोज॒नीः । \newline
51. हा॒रि॒यो॒ज॒नीः प॒शुषु॑ प॒शुषु॑ हारियोज॒नीर्. हा॑रियोज॒नीः प॒शुषु॑ । \newline
52. हा॒रि॒यो॒ज॒नीरिति॑ हारि - यो॒ज॒नीः । \newline
53. प॒शु ष्वे॒वैव प॒शुषु॑ प॒शुष्वे॒व । \newline
54. ए॒व प॒शून् प॒शू ने॒वैव प॒शून् । \newline
55. प॒शून् प्रति॒ प्रति॑ प॒शून् प॒शून् प्रति॑ । \newline
56. प्रति॑ ष्ठापयन्ति स्थापयन्ति॒ प्रति॒ प्रति॑ ष्ठापयन्ति । \newline
57. स्था॒प॒य॒न्तीति॑ स्थापयन्ति । \newline

\textbf{Ghana Paata } \newline

1. ए॒न॒म् प॒शवः॑ प॒शव॑ एन मेनम् प॒शवो॑ भु॒ञ्जन्तो॑ भु॒ञ्जन्तः॑ प॒शव॑ एन मेनम् प॒शवो॑ भु॒ञ्जन्तः॑ । \newline
2. प॒शवो॑ भु॒ञ्जन्तो॑ भु॒ञ्जन्तः॑ प॒शवः॑ प॒शवो॑ भु॒ञ्जन्त॒ उपोप॑ भु॒ञ्जन्तः॑ प॒शवः॑ प॒शवो॑ भु॒ञ्जन्त॒ उप॑ । \newline
3. भु॒ञ्जन्त॒ उपोप॑ भु॒ञ्जन्तो॑ भु॒ञ्जन्त॒ उप॑ तिष्ठेरन् तिष्ठेर॒न्-नुप॑ भु॒ञ्जन्तो॑ भु॒ञ्जन्त॒ उप॑ तिष्ठेरन्न् । \newline
4. उप॑ तिष्ठेरन् तिष्ठेर॒न्-नुपोप॑ तिष्ठेर॒न्॒. यद् यत् ति॑ष्ठेर॒न्-नुपोप॑ तिष्ठेर॒न्॒. यत् । \newline
5. ति॒ष्ठे॒र॒न्॒. यद् यत् ति॑ष्ठेरन् तिष्ठेर॒न्॒. यन् न न यत् ति॑ष्ठेरन् तिष्ठेर॒न्॒. यन् न । \newline
6. यन् न न यद् यन् न सं॑भि॒न्द्याथ् सं॑भि॒न्द्यान् न यद् यन् न सं॑भि॒न्द्यात् । \newline
7. न सं॑भि॒न्द्याथ् सं॑भि॒न्द्यान् न न सं॑भि॒न्द्याद् ब॒हवो॑ ब॒हवः॑ संभि॒न्द्यान् न न सं॑भि॒न्द्याद् ब॒हवः॑ । \newline
8. सं॒भि॒न्द्याद् ब॒हवो॑ ब॒हवः॑ संभि॒न्द्याथ् सं॑भि॒न्द्याद् ब॒हव॑ एन मेनम् ब॒हवः॑ संभि॒न्द्याथ् सं॑भि॒न्द्याद् ब॒हव॑ एनम् । \newline
9. सं॒भि॒न्द्यादिति॑ सं - भि॒न्द्यात् । \newline
10. ब॒हव॑ एन मेनम् ब॒हवो॑ ब॒हव॑ एनम् प॒शवः॑ प॒शव॑ एनम् ब॒हवो॑ ब॒हव॑ एनम् प॒शवः॑ । \newline
11. ए॒न॒म् प॒शवः॑ प॒शव॑ एन मेनम् प॒शवो ऽभु॑ञ्ज॒न्तो ऽभु॑ञ्जन्तः प॒शव॑ एन मेनम् प॒शवो ऽभु॑ञ्जन्तः । \newline
12. प॒शवो ऽभु॑ञ्ज॒न्तो ऽभु॑ञ्जन्तः प॒शवः॑ प॒शवो ऽभु॑ञ्जन्त॒ उपोपाभु॑ञ्जन्तः प॒शवः॑ प॒शवो ऽभु॑ञ्जन्त॒ उप॑ । \newline
13. अभु॑ञ्जन्त॒ उपोपाभु॑ञ्ज॒न्तो ऽभु॑ञ्जन्त॒ उप॑ तिष्ठेरन् तिष्ठेर॒न्-नुपाभु॑ञ्ज॒न्तो ऽभु॑ञ्जन्त॒ उप॑ तिष्ठेरन्न् । \newline
14. उप॑ तिष्ठेरन् तिष्ठेर॒न्-नुपोप॑ तिष्ठेर॒न् मन॑सा॒ मन॑सा तिष्ठेर॒न्-नुपोप॑ तिष्ठेर॒न् मन॑सा । \newline
15. ति॒ष्ठे॒र॒न् मन॑सा॒ मन॑सा तिष्ठेरन् तिष्ठेर॒न् मन॑सा॒ सꣳ सम् मन॑सा तिष्ठेरन् तिष्ठेर॒न् मन॑सा॒ सम् । \newline
16. मन॑सा॒ सꣳ सम् मन॑सा॒ मन॑सा॒ सम् बा॑धते बाधते॒ सम् मन॑सा॒ मन॑सा॒ सम् बा॑धते । \newline
17. सम् बा॑धते बाधते॒ सꣳ सम् बा॑धत उ॒भय॑ मु॒भय॑म् बाधते॒ सꣳ सम् बा॑धत उ॒भय᳚म् । \newline
18. बा॒ध॒त॒ उ॒भय॑ मु॒भय॑म् बाधते बाधत उ॒भय॑म् करोति करो त्यु॒भय॑म् बाधते बाधत उ॒भय॑म् करोति । \newline
19. उ॒भय॑म् करोति करो त्यु॒भय॑ मु॒भय॑म् करोति ब॒हवो॑ ब॒हवः॑ करो त्यु॒भय॑ मु॒भय॑म् करोति ब॒हवः॑ । \newline
20. क॒रो॒ति॒ ब॒हवो॑ ब॒हवः॑ करोति करोति ब॒हव॑ ए॒वैव ब॒हवः॑ करोति करोति ब॒हव॑ ए॒व । \newline
21. ब॒हव॑ ए॒वैव ब॒हवो॑ ब॒हव॑ ए॒वैन॑ मेन मे॒व ब॒हवो॑ ब॒हव॑ ए॒वैन᳚म् । \newline
22. ए॒वैन॑ मेन मे॒वै वैन॑म् प॒शवः॑ प॒शव॑ एन मे॒वै वैन॑म् प॒शवः॑ । \newline
23. ए॒न॒म् प॒शवः॑ प॒शव॑ एन मेनम् प॒शवो॑ भु॒ञ्जन्तो॑ भु॒ञ्जन्तः॑ प॒शव॑ एन मेनम् प॒शवो॑ भु॒ञ्जन्तः॑ । \newline
24. प॒शवो॑ भु॒ञ्जन्तो॑ भु॒ञ्जन्तः॑ प॒शवः॑ प॒शवो॑ भु॒ञ्जन्त॒ उपोप॑ भु॒ञ्जन्तः॑ प॒शवः॑ प॒शवो॑ भु॒ञ्जन्त॒ उप॑ । \newline
25. भु॒ञ्जन्त॒ उपोप॑ भु॒ञ्जन्तो॑ भु॒ञ्जन्त॒ उप॑ तिष्ठन्ते तिष्ठन्त॒ उप॑ भु॒ञ्जन्तो॑ भु॒ञ्जन्त॒ उप॑ तिष्ठन्ते । \newline
26. उप॑ तिष्ठन्ते तिष्ठन्त॒ उपोप॑ तिष्ठन्त उन्ने॒तर् यु॑न्ने॒तरि॑ तिष्ठन्त॒ उपोप॑ तिष्ठन्त उन्ने॒तरि॑ । \newline
27. ति॒ष्ठ॒न्त॒ उ॒न्ने॒तर् यु॑न्ने॒तरि॑ तिष्ठन्ते तिष्ठन्त उन्ने॒तर् यु॑पह॒व मु॑पह॒व मु॑न्ने॒तरि॑ तिष्ठन्ते तिष्ठन्त उन्ने॒तर् यु॑पह॒वम् । \newline
28. उ॒न्ने॒तर् यु॑पह॒व मु॑पह॒व मु॑न्ने॒तर् यु॑न्ने॒तर् यु॑पह॒व मि॑च्छन्त इच्छन्त उपह॒व मु॑न्ने॒तर् यु॑न्ने॒तर् यु॑पह॒व मि॑च्छन्ते । \newline
29. उ॒न्ने॒तरीत्यु॑त् - ने॒तरि॑ । \newline
30. उ॒प॒ह॒व मि॑च्छन्त इच्छन्त उपह॒व मु॑पह॒व मि॑च्छन्ते॒ यो य इ॑च्छन्त उपह॒व मु॑पह॒व मि॑च्छन्ते॒ यः । \newline
31. उ॒प॒ह॒वमित्यु॑प - ह॒वम् । \newline
32. इ॒च्छ॒न्ते॒ यो य इ॑च्छन्त इच्छन्ते॒ य ए॒वैव य इ॑च्छन्त इच्छन्ते॒ य ए॒व । \newline
33. य ए॒वैव यो य ए॒व तत्र॒ तत्रै॒व यो य ए॒व तत्र॑ । \newline
34. ए॒व तत्र॒ तत्रै॒ वैव तत्र॑ सोमपी॒थः सो॑मपी॒थ स्तत्रै॒ वैव तत्र॑ सोमपी॒थः । \newline
35. तत्र॑ सोमपी॒थः सो॑मपी॒थ स्तत्र॒ तत्र॑ सोमपी॒थ स्तम् तꣳ सो॑मपी॒थ स्तत्र॒ तत्र॑ सोमपी॒थ स्तम् । \newline
36. सो॒म॒पी॒थ स्तम् तꣳ सो॑मपी॒थः सो॑मपी॒थ स्त मे॒वैव तꣳ सो॑मपी॒थः सो॑मपी॒थ स्त मे॒व । \newline
37. सो॒म॒पी॒थ इति॑ सोम - पी॒थः । \newline
38. त मे॒वैव तम् त मे॒वावा वै॒व तम् त मे॒वाव॑ । \newline
39. ए॒वावा वै॒वै वाव॑ रुन्धते रुन्ध॒ते ऽवै॒वै वाव॑ रुन्धते । \newline
40. अव॑ रुन्धते रुन्ध॒ते ऽवाव॑ रुन्धत उत्तरवे॒द्या मु॑त्तरवे॒द्याꣳ रु॑न्ध॒ते ऽवाव॑ रुन्धत उत्तरवे॒द्याम् । \newline
41. रु॒न्ध॒त॒ उ॒त्त॒र॒वे॒द्या मु॑त्तरवे॒द्याꣳ रु॑न्धते रुन्धत उत्तरवे॒द्यान् नि न्यु॑त्तरवे॒द्याꣳ रु॑न्धते रुन्धत उत्तरवे॒द्यान् नि । \newline
42. उ॒त्त॒र॒वे॒द्यान् नि न्यु॑त्तरवे॒द्या मु॑त्तरवे॒द्यान् नि व॑पति वपति॒ न्यु॑त्तरवे॒द्या मु॑त्तरवे॒द्यान् नि व॑पति । \newline
43. उ॒त्त॒र॒वे॒द्यामित्यु॑त्तर - वे॒द्याम् । \newline
44. नि व॑पति वपति॒ नि नि व॑पति प॒शवः॑ प॒शवो॑ वपति॒ नि नि व॑पति प॒शवः॑ । \newline
45. व॒प॒ति॒ प॒शवः॑ प॒शवो॑ वपति वपति प॒शवो॒ वै वै प॒शवो॑ वपति वपति प॒शवो॒ वै । \newline
46. प॒शवो॒ वै वै प॒शवः॑ प॒शवो॒ वा उ॑त्तरवे॒दि रु॑त्तरवे॒दिर् वै प॒शवः॑ प॒शवो॒ वा उ॑त्तरवे॒दिः । \newline
47. वा उ॑त्तरवे॒दि रु॑त्तरवे॒दिर् वै वा उ॑त्तरवे॒दिः प॒शवः॑ प॒शव॑ उत्तरवे॒दिर् वै वा उ॑त्तरवे॒दिः प॒शवः॑ । \newline
48. उ॒त्त॒र॒वे॒दिः प॒शवः॑ प॒शव॑ उत्तरवे॒दि रु॑त्तरवे॒दिः प॒शवो॑ हारियोज॒नीर्. हा॑रियोज॒नीः प॒शव॑ उत्तरवे॒दि रु॑त्तरवे॒दिः प॒शवो॑ हारियोज॒नीः । \newline
49. उ॒त्त॒र॒वे॒दिरित्यु॑त्तर - वे॒दिः । \newline
50. प॒शवो॑ हारियोज॒नीर्. हा॑रियोज॒नीः प॒शवः॑ प॒शवो॑ हारियोज॒नीः प॒शुषु॑ प॒शुषु॑ हारियोज॒नीः प॒शवः॑ प॒शवो॑ हारियोज॒नीः प॒शुषु॑ । \newline
51. हा॒रि॒यो॒ज॒नीः प॒शुषु॑ प॒शुषु॑ हारियोज॒नीर्. हा॑रियोज॒नीः प॒शु ष्वे॒वैव प॒शुषु॑ हारियोज॒नीर्. हा॑रियोज॒नीः प॒शुष्वे॒व । \newline
52. हा॒रि॒यो॒ज॒नीरिति॑ हारि - यो॒ज॒नीः । \newline
53. प॒शु ष्वे॒वैव प॒शुषु॑ प॒शुष्वे॒व प॒शून् प॒शू-ने॒व प॒शुषु॑ प॒शुष्वे॒व प॒शून् । \newline
54. ए॒व प॒शून् प॒शूने॒ वैव प॒शून् प्रति॒ प्रति॑ प॒शूने॒ वैव प॒शून् प्रति॑ । \newline
55. प॒शून् प्रति॒ प्रति॑ प॒शून् प॒शून् प्रति॑ ष्ठापयन्ति स्थापयन्ति॒ प्रति॑ प॒शून् प॒शून् प्रति॑ ष्ठापयन्ति । \newline
56. प्रति॑ ष्ठापयन्ति स्थापयन्ति॒ प्रति॒ प्रति॑ ष्ठापयन्ति । \newline
57. स्था॒प॒य॒न्तीति॑ स्थापयन्ति । \newline
\pagebreak
\markright{ TS 6.5.10.1  \hfill https://www.vedavms.in \hfill}

\section{ TS 6.5.10.1 }

\textbf{TS 6.5.10.1 } \newline
\textbf{Samhita Paata} \newline

ग्रहा॒न्॒. वा अनु॑ प्र॒जाः प॒शवः॒ प्र जा॑यन्त उपाꣳश्वन्तर्या॒-माव॑जा॒वयः॑ शु॒क्राम॒न्थिनौ॒ पुरु॑षा ऋतुग्र॒हा-नेक॑शफा आदित्यग्र॒हं गाव॑ आदित्यग्र॒हो भूयि॑ष्ठाभिर्. ऋ॒ग्भिर्गृ॑ह्यते॒ तस्मा॒द्-गावः॑ पशू॒नां भूयि॑ष्ठा॒ यत् त्रिरु॑पाꣳ॒॒ शुꣳ हस्ते॑न विगृ॒ह्णाति॒ तस्मा॒द् द्वौ त्रीन॒जा ज॒नय॒त्यथाव॑यो॒ भूय॑सीः पि॒ता वा ए॒ष यदा᳚ग्रय॒णः पु॒त्रः क॒लशो॒ यदा᳚ग्रय॒ण उ॑प॒दस्ये᳚त् क॒लशा᳚द्-गृह्णीया॒द्-यथा॑ पि॒ता- [  ] \newline

\textbf{Pada Paata} \newline

ग्रहान्॑ । वै । अन्विति॑ । प्र॒जा इति॑ प्र - जाः । प॒शवः॑ । प्रेति॑ । जा॒य॒न्ते॒ । उ॒पाꣳ॒॒श्व॒न्त॒र्या॒मावित्यु॑पाꣳशु - अ॒न्त॒र्या॒मौ । अ॒जा॒वय॒ इत्य॑जा - अ॒वयः॑ । शु॒क्राम॒न्थिना॒विति॑ शु॒क्रा - म॒न्थिनौ᳚ । पुरु॑षाः । ऋ॒तु॒ग्र॒हानित्यृ॑तु - ग्र॒हान् । एक॑शफा॒ इत्येक॑ - श॒फाः॒ । आ॒दि॒त्य॒ग्र॒हमित्या॑दित्य - ग्र॒हम् । गावः॑ । आ॒दि॒त्य॒ग्र॒ह इत्या॑दित्य - ग्र॒हः । भूयि॑ष्ठाभिः । ऋ॒ग्भिरित्यृ॑क्-भिः । गृ॒ह्य॒ते॒ । तस्मा᳚त् । गावः॑ । प॒शू॒नाम् । भूयि॑ष्ठाः । यत् । त्रिः । उ॒पाꣳ॒॒शुमित्यु॑प - अ॒शुम् । हस्ते॑न । वि॒गृ॒ह्णातीति॑ वि - गृ॒ह्णाति॑ । तस्मा᳚त् । द्वौ । त्रीन् । अ॒जा । ज॒नय॑ति । अथ॑ । अव॑यः । भूय॑सीः । पि॒ता । वै । ए॒षः । यत् । आ॒ग्र॒य॒णः । पु॒त्रः । क॒लशः॑ । यत् । आ॒ग्र॒य॒णः । उ॒प॒दस्ये॒दित्यु॑प - दस्ये᳚त् । क॒लशा᳚त् । गृ॒ह्णी॒या॒त् । यथा᳚ । पि॒ता ।  \newline


\textbf{Krama Paata} \newline

ग्रहा॒न्॒. वै । वा अनु॑ । अनु॑ प्र॒जाः । प्र॒जाः प॒शवः॑ । प्र॒जा इति॑ प्र - जाः । प॒शवः॒ प्र । प्र जा॑यन्ते । जा॒य॒न्त॒ उ॒पाꣳ॒॒श्व॒न्त॒र्या॒मौ । उ॒पाꣳ॒॒श्व॒न्त॒र्या॒माव॑जा॒वयः॑ । उ॒पाꣳ॒॒श्व॒न्त॒र्या॒मावित्यु॑पाꣳशु - अ॒न्त॒र्या॒मौ । अ॒जा॒वयः॑ शु॒क्राम॒न्थिनौ᳚ । अ॒जा॒वय॒ इत्य॑जा - अ॒वयः॑ । शु॒क्राम॒न्थिनौ॒ पुरु॑षाः । शु॒क्राम॒न्थिना॒विति॑ शु॒क्रा - म॒न्थिनौ᳚ । पुरु॑षा ऋतुग्र॒हान् । ऋ॒तु॒ग्र॒हानेक॑शफाः । ऋ॒तु॒ग्र॒हानित्यृ॑तु - ग्र॒हान् । एक॑शफा आदित्यग्र॒हम् । एक॑शफा॒ इत्येक॑ - श॒फाः॒ । आ॒दि॒त्य॒ग्र॒हम् गावः॑ । आ॒दि॒त्य॒ग्र॒हमित्या॑दित्य - ग्र॒हम् । गाव॑ आदित्यग्र॒हः । आ॒दि॒त्य॒ग्र॒हो भूयि॑ष्ठाभिः । आ॒दि॒त्य॒ग्र॒ह इत्या॑दित्य - ग्र॒हः । भूयि॑ष्ठाभिर्. ऋ॒ग्भिः । ऋ॒ग्भिर् गृ॑ह्यते । ऋ॒ग्भिरित्यृ॑क् - भिः । गृ॒ह्य॒ते॒ तस्मा᳚त् । तस्मा॒द् गावः॑ । गावः॑ पशू॒नाम् । प॒शू॒नाम् भूयि॑ष्ठाः । भूयि॑ष्ठा॒ यत् । यत् त्रिः । त्रिरु॑पाꣳ॒॒शुम् । उ॒पाꣳ॒॒शुꣳ हस्ते॑न । उ॒पाꣳ॒॒ शुमित्यु॑प - अꣳ॒॒शुम् । हस्ते॑न विगृ॒ह्णाति॑ । वि॒गृ॒ह्णाति॒ तस्मा᳚त् । वि॒गृ॒ह्णातीति॑ वि - गृ॒ह्णाति॑ । तस्मा॒द् द्वौ । द्वौ त्रीन् । त्रीन॒जा । अ॒जा ज॒नय॑ति । ज॒नय॒त्यथ॑ । अथाव॑यः । अव॑यो॒ भूय॑सीः । भूय॑सीः पि॒ता । पि॒ता वै । वा ए॒षः । ए॒ष यत् । यदा᳚ग्रय॒णः । आ॒ग्र॒य॒णः पु॒त्रः । पु॒त्रः क॒लशः॑ । क॒लशो॒ यत् । यदा᳚ग्रय॒णः । आ॒ग्र॒य॒ण उ॑प॒दस्ये᳚त् । उ॒प॒दस्ये᳚त् क॒लशा᳚त् । उ॒प॒दस्ये॒दित्यु॑प - दस्ये᳚त् । क॒लशा᳚द् गृह्णीयात् । गृ॒ह्णी॒या॒द् यथा᳚ । यथा॑ पि॒ता । पि॒ता पु॒त्रम् \newline

\textbf{Jatai Paata} \newline

1. ग्रहा॒न्॒. वै वै ग्रहा॒न् ग्रहा॒न्॒. वै । \newline
2. वा अन् वनु॒ वै वा अनु॑ । \newline
3. अनु॑ प्र॒जाः प्र॒जा अन् वनु॑ प्र॒जाः । \newline
4. प्र॒जाः प॒शवः॑ प॒शवः॑ प्र॒जाः प्र॒जाः प॒शवः॑ । \newline
5. प्र॒जा इति॑ प्र - जाः । \newline
6. प॒शवः॒ प्र प्र प॒शवः॑ प॒शवः॒ प्र । \newline
7. प्र जा॑यन्ते जायन्ते॒ प्र प्र जा॑यन्ते । \newline
8. जा॒य॒न्त॒ उ॒पाꣳ॒॒श्व॒न्त॒र्या॒मा वु॑पाꣳश्वन्तर्या॒मौ जा॑यन्ते जायन्त उपाꣳश्वन्तर्या॒मौ । \newline
9. उ॒पाꣳ॒॒श्व॒न्त॒र्या॒मा व॑जा॒वयो॑ ऽजा॒वय॑ उपाꣳश्वन्तर्या॒मा वु॑पाꣳश्वन्तर्या॒मा व॑जा॒वयः॑ । \newline
10. उ॒पाꣳ॒॒श्व॒न्त॒र्या॒मावित्यु॑पाꣳशु - अ॒न्त॒र्या॒मौ । \newline
11. अ॒जा॒वयः॑ शु॒क्राम॒न्थिनौ॑ शु॒क्राम॒न्थिना॑ वजा॒वयो॑ ऽजा॒वयः॑ शु॒क्राम॒न्थिनौ᳚ । \newline
12. अ॒जा॒वय॒ इत्य॑जा - अ॒वयः॑ । \newline
13. शु॒क्राम॒न्थिनौ॒ पुरु॑षाः॒ पुरु॑षाः शु॒क्राम॒न्थिनौ॑ शु॒क्राम॒न्थिनौ॒ पुरु॑षाः । \newline
14. शु॒क्राम॒न्थिना॒विति॑ शु॒क्रा - म॒न्थिनौ᳚ । \newline
15. पुरु॑षा ऋतुग्र॒हा नृ॑तुग्र॒हान् पुरु॑षाः॒ पुरु॑षा ऋतुग्र॒हान् । \newline
16. ऋ॒तु॒ग्र॒हा नेक॑शफा॒ एक॑शफा ऋतुग्र॒हा नृ॑तुग्र॒हा नेक॑शफाः । \newline
17. ऋ॒तु॒ग्र॒हानित्यृ॑तु - ग्र॒हान् । \newline
18. एक॑शफा आदित्यग्र॒ह मा॑दित्यग्र॒ह मेक॑शफा॒ एक॑शफा आदित्यग्र॒हम् । \newline
19. एक॑शफा॒ इत्येक॑ - श॒फाः॒ । \newline
20. आ॒दि॒त्य॒ग्र॒हम् गावो॒ गाव॑ आदित्यग्र॒ह मा॑दित्यग्र॒हम् गावः॑ । \newline
21. आ॒दि॒त्य॒ग्र॒हमित्या॑दित्य - ग्र॒हम् । \newline
22. गाव॑ आदित्यग्र॒ह आ॑दित्यग्र॒हो गावो॒ गाव॑ आदित्यग्र॒हः । \newline
23. आ॒दि॒त्य॒ग्र॒हो भूयि॑ष्ठाभि॒र् भूयि॑ष्ठाभि रादित्यग्र॒ह आ॑दित्यग्र॒हो भूयि॑ष्ठाभिः । \newline
24. आ॒दि॒त्य॒ग्र॒ह इत्या॑दित्य - ग्र॒हः । \newline
25. भूयि॑ष्ठाभिर्. ऋ॒ग्भिर्. ऋ॒ग्भिर् भूयि॑ष्ठाभि॒र् भूयि॑ष्ठाभिर्. ऋ॒ग्भिः । \newline
26. ऋ॒ग्भिर् गृ॑ह्यते गृह्यत ऋ॒ग्भिर्. ऋ॒ग्भिर् गृ॑ह्यते । \newline
27. ऋ॒ग्भिरित्यृ॑क् - भिः । \newline
28. गृ॒ह्य॒ते॒ तस्मा॒त् तस्मा᳚द् गृह्यते गृह्यते॒ तस्मा᳚त् । \newline
29. तस्मा॒द् गावो॒ गाव॒ स्तस्मा॒त् तस्मा॒द् गावः॑ । \newline
30. गावः॑ पशू॒नाम् प॑शू॒नाम् गावो॒ गावः॑ पशू॒नाम् । \newline
31. प॒शू॒नाम् भूयि॑ष्ठा॒ भूयि॑ष्ठाः पशू॒नाम् प॑शू॒नाम् भूयि॑ष्ठाः । \newline
32. भूयि॑ष्ठा॒ यद् यद् भूयि॑ष्ठा॒ भूयि॑ष्ठा॒ यत् । \newline
33. यत् त्रि स्त्रिर् यद् यत् त्रिः । \newline
34. त्रि रु॑पाꣳ॒॒शु मु॑पाꣳ॒॒शुम् त्रि स्त्रि रु॑पाꣳ॒॒शुम् । \newline
35. उ॒पाꣳ॒॒शुꣳ हस्ते॑न॒ हस्ते॑ नोपाꣳ॒॒शु मु॑पाꣳ॒॒शुꣳ हस्ते॑न । \newline
36. उ॒पाꣳ॒॒शुमित्यु॑प - अ॒शुम् । \newline
37. हस्ते॑न विगृ॒ह्णाति॑ विगृ॒ह्णाति॒ हस्ते॑न॒ हस्ते॑न विगृ॒ह्णाति॑ । \newline
38. वि॒गृ॒ह्णाति॒ तस्मा॒त् तस्मा᳚द् विगृ॒ह्णाति॑ विगृ॒ह्णाति॒ तस्मा᳚त् । \newline
39. वि॒गृ॒ह्णातीति॑ वि - गृ॒ह्णाति॑ । \newline
40. तस्मा॒द् द्वौ द्वौ तस्मा॒त् तस्मा॒द् द्वौ । \newline
41. द्वौ त्रीꣳ स्त्रीन् द्वौ द्वौ त्रीन् । \newline
42. त्री न॒जा ऽजा त्रीꣳ स्त्रीन॒जा । \newline
43. अ॒जा ज॒नय॑ति ज॒नय॑ त्य॒जा ऽजा ज॒नय॑ति । \newline
44. ज॒नय॒ त्यथाथ॑ ज॒नय॑ति ज॒नय॒ त्यथ॑ । \newline
45. अथा व॒यो ऽव॒यो ऽथाथा व॑यः । \newline
46. अव॑यो॒ भूय॑सी॒र् भूय॑सी॒ रव॒यो ऽव॑यो॒ भूय॑सीः । \newline
47. भूय॑सीः पि॒ता पि॒ता भूय॑सी॒र् भूय॑सीः पि॒ता । \newline
48. पि॒ता वै वै पि॒ता पि॒ता वै । \newline
49. वा ए॒ष ए॒ष वै वा ए॒षः । \newline
50. ए॒ष यद् यदे॒ष ए॒ष यत् । \newline
51. यदा᳚ग्रय॒ण आ᳚ग्रय॒णो यद् यदा᳚ग्रय॒णः । \newline
52. आ॒ग्र॒य॒णः पु॒त्रः पु॒त्र आ᳚ग्रय॒ण आ᳚ग्रय॒णः पु॒त्रः । \newline
53. पु॒त्रः क॒लशः॑ क॒लशः॑ पु॒त्रः पु॒त्रः क॒लशः॑ । \newline
54. क॒लशो॒ यद् यत् क॒लशः॑ क॒लशो॒ यत् । \newline
55. यदा᳚ग्रय॒ण आ᳚ग्रय॒णो यद् यदा᳚ग्रय॒णः । \newline
56. आ॒ग्र॒य॒ण उ॑प॒दस्ये॑ दुप॒दस्ये॑ दाग्रय॒ण आ᳚ग्रय॒ण उ॑प॒दस्ये᳚त् । \newline
57. उ॒प॒दस्ये᳚त् क॒लशा᳚त् क॒लशा॑ दुप॒दस्ये॑ दुप॒दस्ये᳚त् क॒लशा᳚त् । \newline
58. उ॒प॒दस्ये॒दित्यु॑प - दस्ये᳚त् । \newline
59. क॒लशा᳚द् गृह्णीयाद् गृह्णीयात् क॒लशा᳚त् क॒लशा᳚द् गृह्णीयात् । \newline
60. गृ॒ह्णी॒या॒द् यथा॒ यथा॑ गृह्णीयाद् गृह्णीया॒द् यथा᳚ । \newline
61. यथा॑ पि॒ता पि॒ता यथा॒ यथा॑ पि॒ता । \newline
62. पि॒ता पु॒त्रम् पु॒त्रम् पि॒ता पि॒ता पु॒त्रम् । \newline

\textbf{Ghana Paata } \newline

1. ग्रहा॒न्॒. वै वै ग्रहा॒न् ग्रहा॒न्॒. वा अन्वनु॒ वै ग्रहा॒न् ग्रहा॒न्॒. वा अनु॑ । \newline
2. वा अन्वनु॒ वै वा अनु॑ प्र॒जाः प्र॒जा अनु॒ वै वा अनु॑ प्र॒जाः । \newline
3. अनु॑ प्र॒जाः प्र॒जा अन्वनु॑ प्र॒जाः प॒शवः॑ प॒शवः॑ प्र॒जा अन्वनु॑ प्र॒जाः प॒शवः॑ । \newline
4. प्र॒जाः प॒शवः॑ प॒शवः॑ प्र॒जाः प्र॒जाः प॒शवः॒ प्र प्र प॒शवः॑ प्र॒जाः प्र॒जाः प॒शवः॒ प्र । \newline
5. प्र॒जा इति॑ प्र - जाः । \newline
6. प॒शवः॒ प्र प्र प॒शवः॑ प॒शवः॒ प्र जा॑यन्ते जायन्ते॒ प्र प॒शवः॑ प॒शवः॒ प्र जा॑यन्ते । \newline
7. प्र जा॑यन्ते जायन्ते॒ प्र प्र जा॑यन्त उपाꣳश्वन्तर्या॒मा वु॑पाꣳश्वन्तर्या॒मौ जा॑यन्ते॒ प्र प्र जा॑यन्त उपाꣳश्वन्तर्या॒मौ । \newline
8. जा॒य॒न्त॒ उ॒पाꣳ॒॒श्व॒न्त॒र्या॒मा वु॑पाꣳश्वन्तर्या॒मौ जा॑यन्ते जायन्त उपाꣳश्वन्तर्या॒मा व॑जा॒वयो॑ ऽजा॒वय॑ उपाꣳश्वन्तर्या॒मौ जा॑यन्ते जायन्त उपाꣳश्वन्तर्या॒मा व॑जा॒वयः॑ । \newline
9. उ॒पाꣳ॒॒श्व॒न्त॒र्या॒मा व॑जा॒वयो॑ ऽजा॒वय॑ उपाꣳश्वन्तर्या॒मा वु॑पाꣳश्वन्तर्या॒मा व॑जा॒वयः॑ शु॒क्राम॒न्थिनौ॑ शु॒क्राम॒न्थिना॑ वजा॒वय॑ उपाꣳश्वन्तर्या॒मा वु॑पाꣳश्वन्तर्या॒मा व॑जा॒वयः॑ शु॒क्राम॒न्थिनौ᳚ । \newline
10. उ॒पाꣳ॒॒श्व॒न्त॒र्या॒मावित्यु॑पाꣳशु - अ॒न्त॒र्या॒मौ । \newline
11. अ॒जा॒वयः॑ शु॒क्राम॒न्थिनौ॑ शु॒क्राम॒न्थिना॑ वजा॒वयो॑ ऽजा॒वयः॑ शु॒क्राम॒न्थिनौ॒ पुरु॑षाः॒ पुरु॑षाः शु॒क्राम॒न्थिना॑ वजा॒वयो॑ ऽजा॒वयः॑ शु॒क्राम॒न्थिनौ॒ पुरु॑षाः । \newline
12. अ॒जा॒वय॒ इत्य॑जा - अ॒वयः॑ । \newline
13. शु॒क्राम॒न्थिनौ॒ पुरु॑षाः॒ पुरु॑षाः शु॒क्राम॒न्थिनौ॑ शु॒क्राम॒न्थिनौ॒ पुरु॑षा ऋतुग्र॒हा-नृ॑तुग्र॒हान् पुरु॑षाः शु॒क्राम॒न्थिनौ॑ शु॒क्राम॒न्थिनौ॒ पुरु॑षा ऋतुग्र॒हान् । \newline
14. शु॒क्राम॒न्थिना॒विति॑ शु॒क्रा - म॒न्थिनौ᳚ । \newline
15. पुरु॑षा ऋतुग्र॒हा-नृ॑तुग्र॒हान् पुरु॑षाः॒ पुरु॑षा ऋतुग्र॒हा-नेक॑शफा॒ एक॑शफा ऋतुग्र॒हान् पुरु॑षाः॒ पुरु॑षा ऋतुग्र॒हा-नेक॑शफाः । \newline
16. ऋ॒तु॒ग्र॒हा-नेक॑शफा॒ एक॑शफा ऋतुग्र॒हा-नृ॑तुग्र॒हा-नेक॑शफा आदित्यग्र॒ह मा॑दित्यग्र॒ह मेक॑शफा ऋतुग्र॒हा-नृ॑तुग्र॒हा-नेक॑शफा आदित्यग्र॒हम् । \newline
17. ऋ॒तु॒ग्र॒हानित्यृ॑तु - ग्र॒हान् । \newline
18. एक॑शफा आदित्यग्र॒ह मा॑दित्यग्र॒ह मेक॑शफा॒ एक॑शफा आदित्यग्र॒हम् गावो॒ गाव॑ आदित्यग्र॒ह मेक॑शफा॒ एक॑शफा आदित्यग्र॒हम् गावः॑ । \newline
19. एक॑शफा॒ इत्येक॑ - श॒फाः॒ । \newline
20. आ॒दि॒त्य॒ग्र॒हम् गावो॒ गाव॑ आदित्यग्र॒ह मा॑दित्यग्र॒हम् गाव॑ आदित्यग्र॒ह आ॑दित्यग्र॒हो गाव॑ आदित्यग्र॒ह मा॑दित्यग्र॒हम् गाव॑ आदित्यग्र॒हः । \newline
21. आ॒दि॒त्य॒ग्र॒हमित्या॑दित्य - ग्र॒हम् । \newline
22. गाव॑ आदित्यग्र॒ह आ॑दित्यग्र॒हो गावो॒ गाव॑ आदित्यग्र॒हो भूयि॑ष्ठाभि॒र् भूयि॑ष्ठाभि रादित्यग्र॒हो गावो॒ गाव॑ आदित्यग्र॒हो भूयि॑ष्ठाभिः । \newline
23. आ॒दि॒त्य॒ग्र॒हो भूयि॑ष्ठाभि॒र् भूयि॑ष्ठाभि रादित्यग्र॒ह आ॑दित्यग्र॒हो भूयि॑ष्ठाभिर्. ऋ॒ग्भिर्. ऋ॒ग्भिर् भूयि॑ष्ठाभि रादित्यग्र॒ह आ॑दित्यग्र॒हो भूयि॑ष्ठाभिर्. ऋ॒ग्भिः । \newline
24. आ॒दि॒त्य॒ग्र॒ह इत्या॑दित्य - ग्र॒हः । \newline
25. भूयि॑ष्ठाभिर्. ऋ॒ग्भिर्. ऋ॒ग्भिर् भूयि॑ष्ठाभि॒र् भूयि॑ष्ठाभिर्. ऋ॒ग्भिर् गृ॑ह्यते गृह्यत ऋ॒ग्भिर् भूयि॑ष्ठाभि॒र् भूयि॑ष्ठाभिर्. ऋ॒ग्भिर् गृ॑ह्यते । \newline
26. ऋ॒ग्भिर् गृ॑ह्यते गृह्यत ऋ॒ग्भिर्. ऋ॒ग्भिर् गृ॑ह्यते॒ तस्मा॒त् तस्मा᳚द् गृह्यत ऋ॒ग्भिर्. ऋ॒ग्भिर् गृ॑ह्यते॒ तस्मा᳚त् । \newline
27. ऋ॒ग्भिरित्यृ॑क् - भिः । \newline
28. गृ॒ह्य॒ते॒ तस्मा॒त् तस्मा᳚द् गृह्यते गृह्यते॒ तस्मा॒द् गावो॒ गाव॒ स्तस्मा᳚द् गृह्यते गृह्यते॒ तस्मा॒द् गावः॑ । \newline
29. तस्मा॒द् गावो॒ गाव॒ स्तस्मा॒त् तस्मा॒द् गावः॑ पशू॒नाम् प॑शू॒नाम् गाव॒ स्तस्मा॒त् तस्मा॒द् गावः॑ पशू॒नाम् । \newline
30. गावः॑ पशू॒नाम् प॑शू॒नाम् गावो॒ गावः॑ पशू॒नाम् भूयि॑ष्ठा॒ भूयि॑ष्ठाः पशू॒नाम् गावो॒ गावः॑ पशू॒नाम् भूयि॑ष्ठाः । \newline
31. प॒शू॒नाम् भूयि॑ष्ठा॒ भूयि॑ष्ठाः पशू॒नाम् प॑शू॒नाम् भूयि॑ष्ठा॒ यद् यद् भूयि॑ष्ठाः पशू॒नाम् प॑शू॒नाम् भूयि॑ष्ठा॒ यत् । \newline
32. भूयि॑ष्ठा॒ यद् यद् भूयि॑ष्ठा॒ भूयि॑ष्ठा॒ यत् त्रि स्त्रिर् यद् भूयि॑ष्ठा॒ भूयि॑ष्ठा॒ यत् त्रिः । \newline
33. यत् त्रि स्त्रिर् यद् यत् त्रिरु॑पाꣳ॒॒शु मु॑पाꣳ॒॒शुम् त्रिर् यद् यत् त्रिरु॑पाꣳ॒॒शुम् । \newline
34. त्रिरु॑पाꣳ॒॒शु मु॑पाꣳ॒॒शुम् त्रि स्त्रि रु॑पाꣳ॒॒शुꣳ हस्ते॑न॒ हस्ते॑ नोपाꣳ॒॒शुम् त्रि स्त्रि रु॑पाꣳ॒॒शुꣳ हस्ते॑न । \newline
35. उ॒पाꣳ॒॒शुꣳ हस्ते॑न॒ हस्ते॑ नोपाꣳ॒॒शु मु॑पाꣳ॒॒शुꣳ हस्ते॑न विगृ॒ह्णाति॑ विगृ॒ह्णाति॒ हस्ते॑
नोपाꣳ॒॒शु मु॑पाꣳ॒॒शुꣳ हस्ते॑न विगृ॒ह्णाति॑ । \newline
36. उ॒पाꣳ॒॒शुमित्यु॑प - अ॒शुम् । \newline
37. हस्ते॑न विगृ॒ह्णाति॑ विगृ॒ह्णाति॒ हस्ते॑न॒ हस्ते॑न विगृ॒ह्णाति॒ तस्मा॒त् तस्मा᳚द् विगृ॒ह्णाति॒ हस्ते॑न॒ हस्ते॑न विगृ॒ह्णाति॒ तस्मा᳚त् । \newline
38. वि॒गृ॒ह्णाति॒ तस्मा॒त् तस्मा᳚द् विगृ॒ह्णाति॑ विगृ॒ह्णाति॒ तस्मा॒द् द्वौ द्वौ तस्मा᳚द् विगृ॒ह्णाति॑ विगृ॒ह्णाति॒ तस्मा॒द् द्वौ । \newline
39. वि॒गृ॒ह्णातीति॑ वि - गृ॒ह्णाति॑ । \newline
40. तस्मा॒द् द्वौ द्वौ तस्मा॒त् तस्मा॒द् द्वौ त्रीꣳ स्त्रीन् द्वौ तस्मा॒त् तस्मा॒द् द्वौ त्रीन् । \newline
41. द्वौ त्रीꣳ स्त्रीन् द्वौ द्वौ त्रीन॒जा ऽजा त्रीन् द्वौ द्वौ त्री-न॒जा । \newline
42. त्रीन॒जा ऽजा त्रीꣳ स्त्रीन॒जा ज॒नय॑ति ज॒नय॑ त्य॒जा त्रीꣳ स्त्रीन॒जा ज॒नय॑ति । \newline
43. अ॒जा ज॒नय॑ति ज॒नय॑ त्य॒जा ऽजा ज॒नय॒ त्यथाथ॑ ज॒नय॑ त्य॒जा ऽजा ज॒नय॒ त्यथ॑ । \newline
44. ज॒नय॒ त्यथाथ॑ ज॒नय॑ति ज॒नय॒ त्यथा व॒यो ऽव॒यो ऽथ॑ ज॒नय॑ति ज॒नय॒ त्यथा व॑यः । \newline
45. अथा व॒यो ऽव॒यो ऽथाथा व॑यो॒ भूय॑सी॒र् भूय॑सी॒ रव॒यो ऽथाथा व॑यो॒ भूय॑सीः । \newline
46. अव॑यो॒ भूय॑सी॒र् भूय॑सी॒ रव॒यो ऽव॑यो॒ भूय॑सीः पि॒ता पि॒ता भूय॑सी॒ रव॒यो ऽव॑यो॒ भूय॑सीः पि॒ता । \newline
47. भूय॑सीः पि॒ता पि॒ता भूय॑सी॒र् भूय॑सीः पि॒ता वै वै पि॒ता भूय॑सी॒र् भूय॑सीः पि॒ता वै । \newline
48. पि॒ता वै वै पि॒ता पि॒ता वा ए॒ष ए॒ष वै पि॒ता पि॒ता वा ए॒षः । \newline
49. वा ए॒ष ए॒ष वै वा ए॒ष यद् यदे॒ष वै वा ए॒ष यत् । \newline
50. ए॒ष यद् यदे॒ष ए॒ष यदा᳚ग्रय॒ण आ᳚ग्रय॒णो यदे॒ष ए॒ष यदा᳚ग्रय॒णः । \newline
51. यदा᳚ग्रय॒ण आ᳚ग्रय॒णो यद् यदा᳚ग्रय॒णः पु॒त्रः पु॒त्र आ᳚ग्रय॒णो यद् यदा᳚ग्रय॒णः पु॒त्रः । \newline
52. आ॒ग्र॒य॒णः पु॒त्रः पु॒त्र आ᳚ग्रय॒ण आ᳚ग्रय॒णः पु॒त्रः क॒लशः॑ क॒लशः॑ पु॒त्र आ᳚ग्रय॒ण आ᳚ग्रय॒णः पु॒त्रः क॒लशः॑ । \newline
53. पु॒त्रः क॒लशः॑ क॒लशः॑ पु॒त्रः पु॒त्रः क॒लशो॒ यद् यत् क॒लशः॑ पु॒त्रः पु॒त्रः क॒लशो॒ यत् । \newline
54. क॒लशो॒ यद् यत् क॒लशः॑ क॒लशो॒ यदा᳚ग्रय॒ण आ᳚ग्रय॒णो यत् क॒लशः॑ क॒लशो॒ यदा᳚ग्रय॒णः । \newline
55. यदा᳚ग्रय॒ण आ᳚ग्रय॒णो यद् यदा᳚ग्रय॒ण उ॑प॒दस्ये॑ दुप॒दस्ये॑ दाग्रय॒णो यद् यदा᳚ग्रय॒ण उ॑प॒दस्ये᳚त् । \newline
56. आ॒ग्र॒य॒ण उ॑प॒दस्ये॑ दुप॒दस्ये॑ दाग्रय॒ण आ᳚ग्रय॒ण उ॑प॒दस्ये᳚त् क॒लशा᳚त् क॒लशा॑ दुप॒दस्ये॑ दाग्रय॒ण आ᳚ग्रय॒ण उ॑प॒दस्ये᳚त् क॒लशा᳚त् । \newline
57. उ॒प॒दस्ये᳚त् क॒लशा᳚त् क॒लशा॑ दुप॒दस्ये॑ दुप॒दस्ये᳚त् क॒लशा᳚द् गृह्णीयाद् गृह्णीयात् क॒लशा॑ दुप॒दस्ये॑ दुप॒दस्ये᳚त् क॒लशा᳚द् गृह्णीयात् । \newline
58. उ॒प॒दस्ये॒दित्यु॑प - दस्ये᳚त् । \newline
59. क॒लशा᳚द् गृह्णीयाद् गृह्णीयात् क॒लशा᳚त् क॒लशा᳚द् गृह्णीया॒द् यथा॒ यथा॑ गृह्णीयात् क॒लशा᳚त् क॒लशा᳚द् गृह्णीया॒द् यथा᳚ । \newline
60. गृ॒ह्णी॒या॒द् यथा॒ यथा॑ गृह्णीयाद् गृह्णीया॒द् यथा॑ पि॒ता पि॒ता यथा॑ गृह्णीयाद् गृह्णीया॒द् यथा॑ पि॒ता । \newline
61. यथा॑ पि॒ता पि॒ता यथा॒ यथा॑ पि॒ता पु॒त्रम् पु॒त्रम् पि॒ता यथा॒ यथा॑ पि॒ता पु॒त्रम् । \newline
62. पि॒ता पु॒त्रम् पु॒त्रम् पि॒ता पि॒ता पु॒त्रम् क्षि॒तः क्षि॒तः पु॒त्रम् पि॒ता पि॒ता पु॒त्रम् क्षि॒तः । \newline
\pagebreak
\markright{ TS 6.5.10.2  \hfill https://www.vedavms.in \hfill}

\section{ TS 6.5.10.2 }

\textbf{TS 6.5.10.2 } \newline
\textbf{Samhita Paata} \newline

पु॒त्रं क्षि॒त उ॑प॒धाव॑ति ता॒दृगे॒व तद्यत् क॒लश॑ उप॒दस्ये॑-दाग्रय॒णाद्-गृ॑ह्णीया॒द्-यथा॑ पु॒त्रः पि॒तरं॑ क्षि॒त उ॑प॒धाव॑ति ता॒दृगे॒व तदा॒त्मा वा ए॒ष य॒ज्ञ्स्य॒ यदा᳚ग्रय॒णो यद्ग्रहो॑ वा क॒लशो॑ वोप॒दस्ये॑-दाग्रय॒णाद्-गृ॑ह्णीयादा॒त्मन॑ ए॒वाधि॑ य॒ज्ञ्ं निष्क॑रो॒त्यवि॑ज्ञातो॒ वा ए॒ष गृ॑ह्यते॒ यदा᳚ग्रय॒णः स्था॒ल्या गृ॒ह्णाति॑ वाय॒व्ये॑न जुहोति॒ तस्मा॒द्- [  ] \newline

\textbf{Pada Paata} \newline

पु॒त्रम् । क्षि॒तः । उ॒प॒धाव॒तीत्यु॑प-धाव॑ति । ता॒दृक् । ए॒व । तत् । यत् । क॒लशः॑ । उ॒प॒दस्ये॒दित्यु॑प-दस्ये᳚त् । आ॒ग्र॒य॒णात् । गृ॒ह्णी॒या॒त् । यथा᳚ । पु॒त्रः । पि॒तर᳚म् । क्षि॒तः । उ॒प॒धाव॒तीत्यु॑प - धाव॑ति । ता॒दृक् । ए॒व । तत् । आ॒त्मा । वै । ए॒षः । य॒ज्ञ्स्य॑ । यत् । आ॒ग्र॒य॒णः । यत् । ग्रहः॑ । वा॒ । क॒लशः॑ । वा॒ । उ॒प॒दस्ये॒दित्यु॑प - दस्ये᳚त् । आ॒ग्र॒य॒णात् । गृ॒ह्णी॒या॒त् । आ॒त्मनः॑ । ए॒व । अधीति॑ । य॒ज्ञ्म् । निरिति॑ । क॒रो॒ति॒ । अवि॑ज्ञात॒ इत्यवि॑ - ज्ञा॒तः॒ । वै । ए॒षः । गृ॒ह्य॒ते॒ । यत् । आ॒ग्र॒य॒णः । स्था॒ल्या । गृ॒ह्णाति॑ । वा॒य॒व्ये॑न । जु॒हो॒ति॒ । तस्मा᳚त् ।  \newline


\textbf{Krama Paata} \newline

पु॒त्रम् क्षि॒तः । क्षि॒त उ॑प॒धाव॑ति । उ॒प॒धाव॑ति ता॒दृक् । उ॒प॒धाव॒तीत्यु॑प - धाव॑ति । ता॒दृगे॒व । ए॒व तत् । तद् यत् । यत् क॒लशः॑ । क॒लश॑ उप॒दस्ये᳚त् । उ॒प॒दस्ये॑दाग्रय॒णात् । उ॒प॒दस्ये॒दित्यु॑प - दस्ये᳚त् । आ॒ग्र॒य॒णाद् गृ॑ह्णीयात् । गृ॒ह्णी॒या॒द् यथा᳚ । यथा॑ पु॒त्रः । पु॒त्रः पि॒तर᳚म् । पि॒तर॑म् क्षि॒तः । क्षि॒त उ॑प॒धाव॑ति । उ॒प॒धाव॑ति ता॒दृक् । उ॒प॒धाव॒तीत्यु॑प - धाव॑ति । ता॒दृगे॒व । ए॒व तत् । तदा॒त्मा । आ॒त्मा वै । वा ए॒षः । ए॒ष य॒ज्ञ्स्य॑ । य॒ज्ञ्स्य॒ यत् । यदा᳚ग्रय॒णः । आ॒ग्र॒य॒णो यत् । यद् ग्रहः॑ । ग्रहो॑ वा । वा॒ क॒लशः॑ । क॒लशो॑ वा । वो॒प॒दस्ये᳚त् । उ॒प॒दस्ये॑दाग्रय॒णात् । उ॒प॒दस्ये॒दित्यु॑प - दस्ये᳚त् । आ॒ग्र॒य॒णाद् गृ॑ह्णीयात् । गृ॒ह्णी॒या॒दा॒त्मनः॑ । आ॒त्मन॑ ए॒व । ए॒वाधि॑ । अधि॑ य॒ज्ञ्म् । य॒ज्ञ्म् निः । निष्क॑रोति । क॒रो॒त्यवि॑ज्ञातः । अवि॑ज्ञातो॒ वै । अवि॑ज्ञात॒ इत्यवि॑ - ज्ञा॒तः॒ । वा ए॒षः । ए॒ष गृ॑ह्यते । गृ॒ह्य॒ते॒ यत् । यदा᳚ग्रय॒णः । आ॒ग्र॒य॒णः स्था॒ल्या । स्था॒ल्या गृ॒ह्णाति॑ । गृ॒ह्णाति॑ वाय॒व्ये॑न । वा॒य॒व्ये॑न जुहोति । जु॒हो॒ति॒ तस्मा᳚त् । तस्मा॒द् गर्भे॑ण \newline

\textbf{Jatai Paata} \newline

1. पु॒त्रम् क्षि॒तः क्षि॒तः पु॒त्रम् पु॒त्रम् क्षि॒तः । \newline
2. क्षि॒त उ॑प॒धाव॑ त्युप॒धाव॑ति क्षि॒तः क्षि॒त उ॑प॒धाव॑ति । \newline
3. उ॒प॒धाव॑ति ता॒दृक् ता॒दृ गु॑प॒धाव॑ त्युप॒धाव॑ति ता॒दृक् । \newline
4. उ॒प॒धाव॒तीत्यु॑प - धाव॑ति । \newline
5. ता॒दृ गे॒वैव ता॒दृक् ता॒दृ गे॒व । \newline
6. ए॒व तत् तदे॒ वैव तत् । \newline
7. तद् यद् यत् तत् तद् यत् । \newline
8. यत् क॒लशः॑ क॒लशो॒ यद् यत् क॒लशः॑ । \newline
9. क॒लश॑ उप॒दस्ये॑ दुप॒दस्ये᳚त् क॒लशः॑ क॒लश॑ उप॒दस्ये᳚त् । \newline
10. उ॒प॒दस्ये॑ दाग्रय॒णा दा᳚ग्रय॒णा दु॑प॒दस्ये॑ दुप॒दस्ये॑ दाग्रय॒णात् । \newline
11. उ॒प॒दस्ये॒दित्यु॑प - दस्ये᳚त् । \newline
12. आ॒ग्र॒य॒णाद् गृ॑ह्णीयाद् गृह्णीया दाग्रय॒णा दा᳚ग्रय॒णाद् गृ॑ह्णीयात् । \newline
13. गृ॒ह्णी॒या॒द् यथा॒ यथा॑ गृह्णीयाद् गृह्णीया॒द् यथा᳚ । \newline
14. यथा॑ पु॒त्रः पु॒त्रो यथा॒ यथा॑ पु॒त्रः । \newline
15. पु॒त्रः पि॒तर॑म् पि॒तर॑म् पु॒त्रः पु॒त्रः पि॒तर᳚म् । \newline
16. पि॒तर॑म् क्षि॒तः क्षि॒तः पि॒तर॑म् पि॒तर॑म् क्षि॒तः । \newline
17. क्षि॒त उ॑प॒धाव॑ त्युप॒धाव॑ति क्षि॒तः क्षि॒त उ॑प॒धाव॑ति । \newline
18. उ॒प॒धाव॑ति ता॒दृक् ता॒दृ गु॑प॒धाव॑ त्युप॒धाव॑ति ता॒दृक् । \newline
19. उ॒प॒धाव॒तीत्यु॑प - धाव॑ति । \newline
20. ता॒दृ गे॒वैव ता॒दृक् ता॒दृ गे॒व । \newline
21. ए॒व तत् तदे॒ वैव तत् । \newline
22. तदा॒त्मा ऽऽत्मा तत् तदा॒त्मा । \newline
23. आ॒त्मा वै वा आ॒त्मा ऽऽत्मा वै । \newline
24. वा ए॒ष ए॒ष वै वा ए॒षः । \newline
25. ए॒ष य॒ज्ञ्स्य॑ य॒ज्ञ् स्यै॒ष ए॒ष य॒ज्ञ्स्य॑ । \newline
26. य॒ज्ञ्स्य॒ यद् यद् य॒ज्ञ्स्य॑ य॒ज्ञ्स्य॒ यत् । \newline
27. यदा᳚ग्रय॒ण आ᳚ग्रय॒णो यद् यदा᳚ग्रय॒णः । \newline
28. आ॒ग्र॒य॒णो यद् यदा᳚ग्रय॒ण आ᳚ग्रय॒णो यत् । \newline
29. यद् ग्रहो॒ ग्रहो॒ यद् यद् ग्रहः॑ । \newline
30. ग्रहो॑ वा वा॒ ग्रहो॒ ग्रहो॑ वा । \newline
31. वा॒ क॒लशः॑ क॒लशो॑ वा वा क॒लशः॑ । \newline
32. क॒लशो॑ वा वा क॒लशः॑ क॒लशो॑ वा । \newline
33. वो॒प॒दस्ये॑ दुप॒दस्ये᳚द् वा वोप॒दस्ये᳚त् । \newline
34. उ॒प॒दस्ये॑ दाग्रय॒णा दा᳚ग्रय॒णा दु॑प॒दस्ये॑ दुप॒दस्ये॑ दाग्रय॒णात् । \newline
35. उ॒प॒दस्ये॒दित्यु॑प - दस्ये᳚त् । \newline
36. आ॒ग्र॒य॒णाद् गृ॑ह्णीयाद् गृह्णीया दाग्रय॒णा दा᳚ग्रय॒णाद् गृ॑ह्णीयात् । \newline
37. गृ॒ह्णी॒या॒ दा॒त्मन॑ आ॒त्मनो॑ गृह्णीयाद् गृह्णीया दा॒त्मनः॑ । \newline
38. आ॒त्मन॑ ए॒वै वात्मन॑ आ॒त्मन॑ ए॒व । \newline
39. ए॒वा ध्यध्ये॒ वैवाधि॑ । \newline
40. अधि॑ य॒ज्ञ्ं ॅय॒ज्ञ् मध्यधि॑ य॒ज्ञ्म् । \newline
41. य॒ज्ञ्न् निर् णिर् य॒ज्ञ्ं ॅय॒ज्ञ्न् निः । \newline
42. निष् क॑रोति करोति॒ निर् णिष् क॑रोति । \newline
43. क॒रो॒ त्यवि॑ज्ञा॒तो ऽवि॑ज्ञातः करोति करो॒ त्यवि॑ज्ञातः । \newline
44. अवि॑ज्ञातो॒ वै वा अवि॑ज्ञा॒तो ऽवि॑ज्ञातो॒ वै । \newline
45. अवि॑ज्ञात॒ इत्यवि॑ - ज्ञा॒तः॒ । \newline
46. वा ए॒ष ए॒ष वै वा ए॒षः । \newline
47. ए॒ष गृ॑ह्यते गृह्यत ए॒ष ए॒ष गृ॑ह्यते । \newline
48. गृ॒ह्य॒ते॒ यद् यद् गृ॑ह्यते गृह्यते॒ यत् । \newline
49. यदा᳚ग्रय॒ण आ᳚ग्रय॒णो यद् यदा᳚ग्रय॒णः । \newline
50. आ॒ग्र॒य॒णः स्था॒ल्या स्था॒ल्या ऽऽग्र॑य॒ण आ᳚ग्रय॒णः स्था॒ल्या । \newline
51. स्था॒ल्या गृ॒ह्णाति॑ गृ॒ह्णाति॑ स्था॒ल्या स्था॒ल्या गृ॒ह्णाति॑ । \newline
52. गृ॒ह्णाति॑ वाय॒व्ये॑न वाय॒व्ये॑न गृ॒ह्णाति॑ गृ॒ह्णाति॑ वाय॒व्ये॑न । \newline
53. वा॒य॒व्ये॑न जुहोति जुहोति वाय॒व्ये॑न वाय॒व्ये॑न जुहोति । \newline
54. जु॒हो॒ति॒ तस्मा॒त् तस्मा᳚ज् जुहोति जुहोति॒ तस्मा᳚त् । \newline
55. तस्मा॒द् गर्भे॑ण॒ गर्भे॑ण॒ तस्मा॒त् तस्मा॒द् गर्भे॑ण । \newline

\textbf{Ghana Paata } \newline

1. पु॒त्रम् क्षि॒तः क्षि॒तः पु॒त्रम् पु॒त्रम् क्षि॒त उ॑प॒धाव॑ त्युप॒धाव॑ति क्षि॒तः पु॒त्रम् पु॒त्रम् क्षि॒त उ॑प॒धाव॑ति । \newline
2. क्षि॒त उ॑प॒धाव॑ त्युप॒धाव॑ति क्षि॒तः क्षि॒त उ॑प॒धाव॑ति ता॒दृक् ता॒दृ गु॑प॒धाव॑ति क्षि॒तः क्षि॒त उ॑प॒धाव॑ति ता॒दृक् । \newline
3. उ॒प॒धाव॑ति ता॒दृक् ता॒दृ गु॑प॒धाव॑ त्युप॒धाव॑ति ता॒दृ गे॒वैव ता॒दृ गु॑प॒धाव॑ त्युप॒धाव॑ति ता॒दृ गे॒व । \newline
4. उ॒प॒धाव॒तीत्यु॑प - धाव॑ति । \newline
5. ता॒दृ गे॒वैव ता॒दृक् ता॒दृ गे॒व तत् तदे॒व ता॒दृक् ता॒दृ गे॒व तत् । \newline
6. ए॒व तत् तदे॒ वैव तद् यद् यत् तदे॒ वैव तद् यत् । \newline
7. तद् यद् यत् तत् तद् यत् क॒लशः॑ क॒लशो॒ यत् तत् तद् यत् क॒लशः॑ । \newline
8. यत् क॒लशः॑ क॒लशो॒ यद् यत् क॒लश॑ उप॒दस्ये॑ दुप॒दस्ये᳚त् क॒लशो॒ यद् यत् क॒लश॑ उप॒दस्ये᳚त् । \newline
9. क॒लश॑ उप॒दस्ये॑ दुप॒दस्ये᳚त् क॒लशः॑ क॒लश॑ उप॒दस्ये॑ दाग्रय॒णा दा᳚ग्रय॒णा दु॑प॒दस्ये᳚त् क॒लशः॑ क॒लश॑ उप॒दस्ये॑ दाग्रय॒णात् । \newline
10. उ॒प॒दस्ये॑ दाग्रय॒णा दा᳚ग्रय॒णा दु॑प॒दस्ये॑ दुप॒दस्ये॑ दाग्रय॒णाद् गृ॑ह्णीयाद् गृह्णीया दाग्रय॒णा दु॑प॒दस्ये॑ दुप॒दस्ये॑ दाग्रय॒णाद् गृ॑ह्णीयात् । \newline
11. उ॒प॒दस्ये॒दित्यु॑प - दस्ये᳚त् । \newline
12. आ॒ग्र॒य॒णाद् गृ॑ह्णीयाद् गृह्णीया दाग्रय॒णा दा᳚ग्रय॒णाद् गृ॑ह्णीया॒द् यथा॒ यथा॑ गृह्णीया दाग्रय॒णा दा᳚ग्रय॒णाद् गृ॑ह्णीया॒द् यथा᳚ । \newline
13. गृ॒ह्णी॒या॒द् यथा॒ यथा॑ गृह्णीयाद् गृह्णीया॒द् यथा॑ पु॒त्रः पु॒त्रो यथा॑ गृह्णीयाद् गृह्णीया॒द् यथा॑ पु॒त्रः । \newline
14. यथा॑ पु॒त्रः पु॒त्रो यथा॒ यथा॑ पु॒त्रः पि॒तर॑म् पि॒तर॑म् पु॒त्रो यथा॒ यथा॑ पु॒त्रः पि॒तर᳚म् । \newline
15. पु॒त्रः पि॒तर॑म् पि॒तर॑म् पु॒त्रः पु॒त्रः पि॒तर॑म् क्षि॒तः क्षि॒तः पि॒तर॑म् पु॒त्रः पु॒त्रः पि॒तर॑म् क्षि॒तः । \newline
16. पि॒तर॑म् क्षि॒तः क्षि॒तः पि॒तर॑म् पि॒तर॑म् क्षि॒त उ॑प॒धाव॑ त्युप॒धाव॑ति क्षि॒तः पि॒तर॑म् पि॒तर॑म् क्षि॒त उ॑प॒धाव॑ति । \newline
17. क्षि॒त उ॑प॒धाव॑ त्युप॒धाव॑ति क्षि॒तः क्षि॒त उ॑प॒धाव॑ति ता॒दृक् ता॒दृ गु॑प॒धाव॑ति क्षि॒तः क्षि॒त उ॑प॒धाव॑ति ता॒दृक् । \newline
18. उ॒प॒धाव॑ति ता॒दृक् ता॒दृ गु॑प॒धाव॑ त्युप॒धाव॑ति ता॒दृ गे॒वैव ता॒दृ गु॑प॒धाव॑ त्युप॒धाव॑ति ता॒दृ गे॒व । \newline
19. उ॒प॒धाव॒तीत्यु॑प - धाव॑ति । \newline
20. ता॒दृगे॒ वैव ता॒दृक् ता॒दृ गे॒व तत् तदे॒व ता॒दृक् ता॒दृ गे॒व तत् । \newline
21. ए॒व तत् तदे॒ वैव तदा॒त्मा ऽऽत्मा तदे॒ वैव तदा॒त्मा । \newline
22. तदा॒त्मा ऽऽत्मा तत् तदा॒त्मा वै वा आ॒त्मा तत् तदा॒त्मा वै । \newline
23. आ॒त्मा वै वा आ॒त्मा ऽऽत्मा वा ए॒ष ए॒ष वा आ॒त्मा ऽऽत्मा वा ए॒षः । \newline
24. वा ए॒ष ए॒ष वै वा ए॒ष य॒ज्ञ्स्य॑ य॒ज्ञ्स्यै॒ष वै वा ए॒ष य॒ज्ञ्स्य॑ । \newline
25. ए॒ष य॒ज्ञ्स्य॑ य॒ज्ञ्स्यै॒ष ए॒ष य॒ज्ञ्स्य॒ यद् यद् य॒ज्ञ्स्यै॒ष ए॒ष य॒ज्ञ्स्य॒ यत् । \newline
26. य॒ज्ञ्स्य॒ यद् यद् य॒ज्ञ्स्य॑ य॒ज्ञ्स्य॒ यदा᳚ग्रय॒ण आ᳚ग्रय॒णो यद् य॒ज्ञ्स्य॑ य॒ज्ञ्स्य॒ यदा᳚ग्रय॒णः । \newline
27. यदा᳚ग्रय॒ण आ᳚ग्रय॒णो यद् यदा᳚ग्रय॒णो यद् यदा᳚ग्रय॒णो यद् यदा᳚ग्रय॒णो यत् । \newline
28. आ॒ग्र॒य॒णो यद् यदा᳚ग्रय॒ण आ᳚ग्रय॒णो यद् ग्रहो॒ ग्रहो॒ यदा᳚ग्रय॒ण आ᳚ग्रय॒णो यद् ग्रहः॑ । \newline
29. यद् ग्रहो॒ ग्रहो॒ यद् यद् ग्रहो॑ वा वा॒ ग्रहो॒ यद् यद् ग्रहो॑ वा । \newline
30. ग्रहो॑ वा वा॒ ग्रहो॒ ग्रहो॑ वा क॒लशः॑ क॒लशो॑ वा॒ ग्रहो॒ ग्रहो॑ वा क॒लशः॑ । \newline
31. वा॒ क॒लशः॑ क॒लशो॑ वा वा क॒लशो॑ वा वा क॒लशो॑ वा वा क॒लशो॑ वा । \newline
32. क॒लशो॑ वा वा क॒लशः॑ क॒लशो॑ वोप॒दस्ये॑ दुप॒दस्ये᳚द् वा क॒लशः॑ क॒लशो॑ वोप॒दस्ये᳚त् । \newline
33. वो॒प॒दस्ये॑ दुप॒दस्ये᳚द् वा वोप॒दस्ये॑ दाग्रय॒णा दा᳚ग्रय॒णा दु॑प॒दस्ये᳚द् वा वोप॒दस्ये॑ दाग्रय॒णात् । \newline
34. उ॒प॒दस्ये॑ दाग्रय॒णा दा᳚ग्रय॒णा दु॑प॒दस्ये॑ दुप॒दस्ये॑ दाग्रय॒णाद् गृ॑ह्णीयाद् गृह्णीया दाग्रय॒णा दु॑प॒दस्ये॑ दुप॒दस्ये॑ दाग्रय॒णाद् गृ॑ह्णीयात् । \newline
35. उ॒प॒दस्ये॒दित्यु॑प - दस्ये᳚त् । \newline
36. आ॒ग्र॒य॒णाद् गृ॑ह्णीयाद् गृह्णीया दाग्रय॒णा दा᳚ग्रय॒णाद् गृ॑ह्णीया दा॒त्मन॑ आ॒त्मनो॑ गृह्णीया दाग्रय॒णा दा᳚ग्रय॒णाद् गृ॑ह्णीया दा॒त्मनः॑ । \newline
37. गृ॒ह्णी॒या॒ दा॒त्मन॑ आ॒त्मनो॑ गृह्णीयाद् गृह्णीया दा॒त्मन॑ ए॒वै वात्मनो॑ गृह्णीयाद् गृह्णीया दा॒त्मन॑ ए॒व । \newline
38. आ॒त्मन॑ ए॒वै वात्मन॑ आ॒त्मन॑ ए॒वा ध्य ध्ये॒वात्मन॑ आ॒त्मन॑ ए॒वाधि॑ । \newline
39. ए॒वा ध्यध्ये॒ वैवाधि॑ य॒ज्ञ्ं ॅय॒ज्ञ् मध्ये॒ वैवाधि॑ य॒ज्ञ्म् । \newline
40. अधि॑ य॒ज्ञ्ं ॅय॒ज्ञ् मध्यधि॑ य॒ज्ञ्न् निर् णिर् य॒ज्ञ् मध्यधि॑ य॒ज्ञ्न् निः । \newline
41. य॒ज्ञ्न् निर् णिर् य॒ज्ञ्ं ॅय॒ज्ञ्न् निष् क॑रोति करोति॒ निर् य॒ज्ञ्ं ॅय॒ज्ञ्न् निष् क॑रोति । \newline
42. निष् क॑रोति करोति॒ निर् णिष् क॑रो॒ त्यवि॑ज्ञा॒तो ऽवि॑ज्ञातः करोति॒ निर् णिष् क॑रो॒ त्यवि॑ज्ञातः । \newline
43. क॒रो॒ त्यवि॑ज्ञा॒तो ऽवि॑ज्ञातः करोति करो॒ त्यवि॑ज्ञातो॒ वै वा अवि॑ज्ञातः करोति करो॒ त्यवि॑ज्ञातो॒ वै । \newline
44. अवि॑ज्ञातो॒ वै वा अवि॑ज्ञा॒तो ऽवि॑ज्ञातो॒ वा ए॒ष ए॒ष वा अवि॑ज्ञा॒तो ऽवि॑ज्ञातो॒ वा ए॒षः । \newline
45. अवि॑ज्ञात॒ इत्यवि॑ - ज्ञा॒तः॒ । \newline
46. वा ए॒ष ए॒ष वै वा ए॒ष गृ॑ह्यते गृह्यत ए॒ष वै वा ए॒ष गृ॑ह्यते । \newline
47. ए॒ष गृ॑ह्यते गृह्यत ए॒ष ए॒ष गृ॑ह्यते॒ यद् यद् गृ॑ह्यत ए॒ष ए॒ष गृ॑ह्यते॒ यत् । \newline
48. गृ॒ह्य॒ते॒ यद् यद् गृ॑ह्यते गृह्यते॒ यदा᳚ग्रय॒ण आ᳚ग्रय॒णो यद् गृ॑ह्यते गृह्यते॒ यदा᳚ग्रय॒णः । \newline
49. यदा᳚ग्रय॒ण आ᳚ग्रय॒णो यद् यदा᳚ग्रय॒णः स्था॒ल्या स्था॒ल्या ऽऽग्र॑य॒णो यद् यदा᳚ग्रय॒णः स्था॒ल्या । \newline
50. आ॒ग्र॒य॒णः स्था॒ल्या स्था॒ल्या ऽऽग्र॑य॒ण आ᳚ग्रय॒णः स्था॒ल्या गृ॒ह्णाति॑ गृ॒ह्णाति॑ स्था॒ल्या ऽऽग्र॑य॒ण आ᳚ग्रय॒णः स्था॒ल्या गृ॒ह्णाति॑ । \newline
51. स्था॒ल्या गृ॒ह्णाति॑ गृ॒ह्णाति॑ स्था॒ल्या स्था॒ल्या गृ॒ह्णाति॑ वाय॒व्ये॑न वाय॒व्ये॑न गृ॒ह्णाति॑ स्था॒ल्या स्था॒ल्या गृ॒ह्णाति॑ वाय॒व्ये॑न । \newline
52. गृ॒ह्णाति॑ वाय॒व्ये॑न वाय॒व्ये॑न गृ॒ह्णाति॑ गृ॒ह्णाति॑ वाय॒व्ये॑न जुहोति जुहोति वाय॒व्ये॑न गृ॒ह्णाति॑ गृ॒ह्णाति॑ वाय॒व्ये॑न जुहोति । \newline
53. वा॒य॒व्ये॑न जुहोति जुहोति वाय॒व्ये॑न वाय॒व्ये॑न जुहोति॒ तस्मा॒त् तस्मा᳚ज् जुहोति वाय॒व्ये॑न वाय॒व्ये॑न जुहोति॒ तस्मा᳚त् । \newline
54. जु॒हो॒ति॒ तस्मा॒त् तस्मा᳚ज् जुहोति जुहोति॒ तस्मा॒द् गर्भे॑ण॒ गर्भे॑ण॒ तस्मा᳚ज् जुहोति जुहोति॒ तस्मा॒द् गर्भे॑ण । \newline
55. तस्मा॒द् गर्भे॑ण॒ गर्भे॑ण॒ तस्मा॒त् तस्मा॒द् गर्भे॒णा वि॑ज्ञाते॒ना वि॑ज्ञातेन॒ गर्भे॑ण॒ तस्मा॒त् तस्मा॒द् गर्भे॒णा वि॑ज्ञातेन । \newline
\pagebreak
\markright{ TS 6.5.10.3  \hfill https://www.vedavms.in \hfill}

\section{ TS 6.5.10.3 }

\textbf{TS 6.5.10.3 } \newline
\textbf{Samhita Paata} \newline

गर्भे॒णा वि॑ज्ञातेन ब्रह्म॒हा ऽव॑भृ॒थमव॑ यन्ति॒ परा᳚ स्था॒लीरस्य॒न्त्युद्-वा॑य॒व्या॑नि हरन्ति॒ तस्मा॒थ् स्त्रियं॑ जा॒तां परा᳚ऽस्य॒न्त्युत् पुमाꣳ॑ सꣳ हरन्ति॒ यत् पु॑रो॒रुच॒माह॒ यथा॒ वस्य॑स आ॒हर॑ति ता॒दृगे॒व तद्-यद्-ग्रहं॑ गृ॒ह्णाति॒ यथा॒ वस्य॑स आ॒हृत्य॒ प्राऽऽ*ह॑ ता॒दृगे॒व तद्-यथ् सा॒दय॑ति॒ यथा॒ वस्य॑स उपनि॒धाया॑-प॒क्राम॑ति ता॒दृगे॒व तद् य ( ) द्वै य॒ज्ञ्स्य॒ साम्ना॒ यजु॑षा क्रि॒यते॑ शिथि॒लं तद्-यदृ॒चा तद् दृ॒ढं पु॒रस्ता॑दुपयामा॒ यजु॑षा गृह्यन्त उ॒परि॑ष्टा-दुपयामा ऋ॒चा य॒ज्ञ्स्य॒ धृत्यै᳚ ॥ \newline

\textbf{Pada Paata} \newline

गभे॑र्ण । अवि॑ज्ञाते॒नेत्यवि॑ - ज्ञा॒ते॒न॒ । ब्र॒ह्म॒हेति॑ ब्रह्म - हा । अ॒व॒भृ॒थमित्यव॑ - भृ॒थम् । अवेति॑ । य॒न्ति॒ । परेति॑ । स्था॒लीः । अस्य॑न्ति । उदिति॑ । वा॒य॒व्या॑नि । ह॒र॒न्ति॒ । तस्मा᳚त् । स्त्रिय᳚म् । जा॒ताम् । परेति॑ । अ॒स्य॒न्ति॒ । उदिति॑ । पुमाꣳ॑सम् । ह॒र॒न्ति॒ । यत् । पु॒रो॒रुच॒मिति॑ पुरः - रुच᳚म् । आह॑ । यथा᳚ । वस्य॑से । आ॒हर॒तीत्या᳚-हर॑ति । ता॒दृक् । ए॒व । तत् । यत् । ग्रह᳚म् । गृ॒ह्णाति॑ । यथा᳚ । वस्य॑से । आ॒हृत्येत्या᳚ - हृत्य॑ । प्रेति॑ । आह॑ । ता॒दृक् । ए॒व । तत् । यत् । सा॒दय॑ति । यथा᳚ । वस्य॑से । उ॒प॒नि॒धायेत्यु॑प - नि॒धाय॑ । अ॒प॒क्राम॒तीत्य॑प - क्राम॑ति । ता॒दृक् । ए॒व । तत् । यत् ( ) । वै । य॒ज्ञ्स्य॑ । साम्ना᳚ । यजु॑षा । क्रि॒यते᳚ । शि॒थि॒लम् । तत् । यत् । ऋ॒चा । तत् । दृ॒ढम् । पु॒रस्ता॑दुपयामा॒ इति॑ पु॒रस्ता᳚त् - उ॒प॒या॒माः॒ । यजु॑षा । गृ॒ह्य॒न्ते॒ । उ॒परि॑ष्टादुपयामा॒ इत्यु॒परि॑ष्टात् - उ॒प॒या॒माः॒ । ऋ॒चा । य॒ज्ञ्स्य॑ । धृत्यै᳚ ॥  \newline


\textbf{Krama Paata} \newline

गर्भे॒णावि॑ज्ञातेन । अवि॑ज्ञातेन ब्रह्म॒हा । अवि॑ज्ञाते॒नेत्यवि॑ - ज्ञा॒ते॒न॒ । ब्र॒ह्म॒हाऽव॑भृ॒थम् । ब्र॒ह्म॒हेति॑ ब्रह्म - हा । अ॒व॒भृ॒थमव॑ । अ॒व॒भृ॒थमित्य॑व - भृ॒थम् । अव॑ यन्ति । य॒न्ति॒ परा᳚ । परा᳚ स्था॒लीः । स्था॒लीरस्य॑न्ति । अस्य॒न्त्युत् । उद् वा॑य॒व्या॑नि । वा॒य॒व्या॑नि हरन्ति । ह॒र॒न्ति॒ तस्मा᳚त् । तस्मा॒थ् स्त्रिय᳚म् । स्त्रिय॑म् जा॒ताम् । जा॒ताम् परा᳚ । परा᳚ऽस्यन्ति । अ॒स्य॒न्त्युत् । उत् पुमाꣳ॑सम् । पुमाꣳ॑सꣳ हरन्ति । ह॒र॒न्ति॒ यत् । यत् पु॑रो॒रुच᳚म् । पु॒रो॒रुच॒माह॑ । पु॒रो॒रुच॒मिति॑ पुरः - रुच᳚म् । आह॒ यथा᳚ । यथा॒ वस्य॑से । वस्य॑स आ॒हर॑ति । आ॒हर॑ति ता॒दृक् । आ॒हर॒तीत्या᳚ - हर॑ति । ता॒दृगे॒व । ए॒व तत् । तद् यत् । यद् ग्रह᳚म् । ग्रह॑म् गृ॒ह्णाति॑ । गृ॒ह्णाति॒ यथा᳚ । यथा॒ वस्य॑से । वस्य॑स आ॒हृत्य॑ । आ॒हृत्य॒ प्र । आ॒हृतेत्या᳚ - हृत्य॑ । प्राह॑ । आह॑ ता॒दृक् । ता॒दृगे॒व । ए॒व तत् । तद् यत् । यथ् सा॒दय॑ति । सा॒दय॑ति॒ यथा᳚ । यथा॒ वस्य॑से । वस्य॑स उपनि॒धाय॑ । उ॒प॒नि॒धाया॑प॒क्राम॑ति । उ॒प॒नि॒धायेत्यु॑प - नि॒धाय॑ । अ॒प॒क्राम॑ति ता॒दृक् । अ॒प॒क्राम॒तीत्य॑प - क्राम॑ति । ता॒दृगे॒व । ए॒व तत् । तद् यत् ( ) । यद् वै । वै य॒ज्ञ्स्य॑ । य॒ज्ञ्स्य॒ साम्ना᳚ । साम्ना॒ यजु॑षा । यजु॑षा क्रि॒यते᳚ । क्रि॒यते॑ शिथि॒लम् । शि॒थि॒लम् तत् । तद् यत् । यदृ॒चा । ऋ॒चा तत् । तद् दृ॒ढम् । दृ॒ढम् पु॒रस्ता॑दुपयामाः । पु॒रस्ता॑दुपयामा॒ यजु॑षा । पु॒रस्ता॑दुपयामा॒ इति॑ पु॒रस्ता᳚त् - उ॒प॒या॒माः॒ । यजु॑षा गृह्यन्ते । गृ॒ह्य॒न्त॒ उ॒परि॑ष्टादुपयामाः । उ॒परि॑ष्टादुपयामा ऋ॒चा । उ॒परि॑ष्टादुपयामा॒ इत्यु॒परि॑ष्टात् - उ॒प॒या॒माः॒ । ऋ॒चा य॒ज्ञ्स्य॑ । य॒ज्ञ्स्य॒ धृत्यै᳚ । धृत्या॒ इति॒ धृत्यै᳚ । \newline

\textbf{Jatai Paata} \newline

1. गर्भे॒णा वि॑ज्ञाते॒ना वि॑ज्ञातेन॒ गर्भे॑ण॒ गर्भे॒णा वि॑ज्ञातेन । \newline
2. अवि॑ज्ञातेन ब्रह्म॒हा ब्र॑ह्म॒हा ऽवि॑ज्ञाते॒ना वि॑ज्ञातेन ब्रह्म॒हा । \newline
3. अवि॑ज्ञाते॒नेत्यवि॑ - ज्ञा॒ते॒न॒ । \newline
4. ब्र॒ह्म॒हा ऽव॑भृ॒थ म॑वभृ॒थम् ब्र॑ह्म॒हा ब्र॑ह्म॒हा ऽव॑भृ॒थम् । \newline
5. ब्र॒ह्म॒हेति॑ ब्रह्म - हा । \newline
6. अ॒व॒भृ॒थ मवावा॑ वभृ॒थ म॑वभृ॒थ मव॑ । \newline
7. अ॒व॒भृ॒थमित्य॑व - भृ॒थम् । \newline
8. अव॑ यन्ति य॒न्त्यवाव॑ यन्ति । \newline
9. य॒न्ति॒ परा॒ परा॑ यन्ति यन्ति॒ परा᳚ । \newline
10. परा᳚ स्था॒लीः स्था॒लीः परा॒ परा᳚ स्था॒लीः । \newline
11. स्था॒ली रस्य॒ न्त्यस्य॑न्ति स्था॒लीः स्था॒ली रस्य॑न्ति । \newline
12. अस्य॒ न्त्युदु दस्य॒ न्त्यस्य॒ न्त्युत् । \newline
13. उद् वा॑य॒व्या॑नि वाय॒व्या᳚ न्युदुद् वा॑य॒व्या॑नि । \newline
14. वा॒य॒व्या॑नि हरन्ति हरन्ति वाय॒व्या॑नि वाय॒व्या॑नि हरन्ति । \newline
15. ह॒र॒न्ति॒ तस्मा॒त् तस्मा᳚ द्धरन्ति हरन्ति॒ तस्मा᳚त् । \newline
16. तस्मा॒थ् स्त्रियꣳ॒॒ स्त्रिय॒म् तस्मा॒त् तस्मा॒थ् स्त्रिय᳚म् । \newline
17. स्त्रिय॑म् जा॒ताम् जा॒ताꣳ स्त्रियꣳ॒॒ स्त्रिय॑म् जा॒ताम् । \newline
18. जा॒ताम् परा॒ परा॑ जा॒ताम् जा॒ताम् परा᳚ । \newline
19. परा᳚ ऽस्य न्त्यस्यन्ति॒ परा॒ परा᳚ ऽस्यन्ति । \newline
20. अ॒स्य॒ न्त्युदु द॑स्य न्त्यस्य॒ न्त्युत् । \newline
21. उत् पुमाꣳ॑स॒म् पुमाꣳ॑स॒ मुदुत् पुमाꣳ॑सम् । \newline
22. पुमाꣳ॑सꣳ हरन्ति हरन्ति॒ पुमाꣳ॑स॒म् पुमाꣳ॑सꣳ हरन्ति । \newline
23. ह॒र॒न्ति॒ यद् यद्ध॑रन्ति हरन्ति॒ यत् । \newline
24. यत् पु॑रो॒रुच॑म् पुरो॒रुचं॒ ॅयद् यत् पु॑रो॒रुच᳚म् । \newline
25. पु॒रो॒रुच॒ माहाह॑ पुरो॒रुच॑म् पुरो॒रुच॒ माह॑ । \newline
26. पु॒रो॒रुच॒मिति॑ पुरः - रुच᳚म् । \newline
27. आह॒ यथा॒ यथा ऽऽहाह॒ यथा᳚ । \newline
28. यथा॒ वस्य॑से॒ वस्य॑से॒ यथा॒ यथा॒ वस्य॑से । \newline
29. वस्य॑स आ॒हर॑ त्या॒हर॑ति॒ वस्य॑से॒ वस्य॑स आ॒हर॑ति । \newline
30. आ॒हर॑ति ता॒दृक् ता॒दृ गा॒हर॑ त्या॒हर॑ति ता॒दृक् । \newline
31. आ॒हर॒तीत्या᳚ - हर॑ति । \newline
32. ता॒दृ गे॒वैव ता॒दृक् ता॒दृ गे॒व । \newline
33. ए॒व तत् तदे॒ वैव तत् । \newline
34. तद् यद् यत् तत् तद् यत् । \newline
35. यद् ग्रह॒म् ग्रहं॒ ॅयद् यद् ग्रह᳚म् । \newline
36. ग्रह॑म् गृ॒ह्णाति॑ गृ॒ह्णाति॒ ग्रह॒म् ग्रह॑म् गृ॒ह्णाति॑ । \newline
37. गृ॒ह्णाति॒ यथा॒ यथा॑ गृ॒ह्णाति॑ गृ॒ह्णाति॒ यथा᳚ । \newline
38. यथा॒ वस्य॑से॒ वस्य॑से॒ यथा॒ यथा॒ वस्य॑से । \newline
39. वस्य॑स आ॒हृत्या॒ हृत्य॒ वस्य॑से॒ वस्य॑स आ॒हृत्य॑ । \newline
40. आ॒हृत्य॒ प्र प्राहृत्या॒ हृत्य॒ प्र । \newline
41. आ॒हृत्येत्या᳚ - हृत्य॑ । \newline
42. प्राहाह॒ प्र प्राह॑ । \newline
43. आह॑ ता॒दृक् ता॒दृ गाहाह॑ ता॒दृक् । \newline
44. ता॒दृ गे॒वैव ता॒दृक् ता॒दृ गे॒व । \newline
45. ए॒व तत् तदे॒ वैव तत् । \newline
46. तद् यद् यत् तत् तद् यत् । \newline
47. यथ् सा॒दय॑ति सा॒दय॑ति॒ यद् यथ् सा॒दय॑ति । \newline
48. सा॒दय॑ति॒ यथा॒ यथा॑ सा॒दय॑ति सा॒दय॑ति॒ यथा᳚ । \newline
49. यथा॒ वस्य॑से॒ वस्य॑से॒ यथा॒ यथा॒ वस्य॑से । \newline
50. वस्य॑स उपनि॒धायो॑ पनि॒धाय॒ वस्य॑से॒ वस्य॑स उपनि॒धाय॑ । \newline
51. उ॒प॒नि॒धाया॑ प॒क्राम॑ त्यप॒क्राम॑ त्युपनि॒धायो॑ पनि॒धाया॑ प॒क्राम॑ति । \newline
52. उ॒प॒नि॒धायेत्यु॑प - नि॒धाय॑ । \newline
53. अ॒प॒क्राम॑ति ता॒दृक् ता॒दृ ग॑प॒क्राम॑ त्यप॒क्राम॑ति ता॒दृक् । \newline
54. अ॒प॒क्राम॒तीत्य॑प - क्राम॑ति । \newline
55. ता॒दृ गे॒वैव ता॒दृक् ता॒दृ गे॒व । \newline
56. ए॒व तत् तदे॒ वैव तत् । \newline
57. तद् यद् यत् तत् तद् यत् । \newline
58. यद् वै वै यद् यद् वै । \newline
59. वै य॒ज्ञ्स्य॑ य॒ज्ञ्स्य॒ वै वै य॒ज्ञ्स्य॑ । \newline
60. य॒ज्ञ्स्य॒ साम्ना॒ साम्ना॑ य॒ज्ञ्स्य॑ य॒ज्ञ्स्य॒ साम्ना᳚ । \newline
61. साम्ना॒ यजु॑षा॒ यजु॑षा॒ साम्ना॒ साम्ना॒ यजु॑षा । \newline
62. यजु॑षा क्रि॒यते᳚ क्रि॒यते॒ यजु॑षा॒ यजु॑षा क्रि॒यते᳚ । \newline
63. क्रि॒यते॑ शिथि॒लꣳ शि॑थि॒लम् क्रि॒यते᳚ क्रि॒यते॑ शिथि॒लम् । \newline
64. शि॒थि॒लम् तत् तच् छि॑थि॒लꣳ शि॑थि॒लम् तत् । \newline
65. तद् यद् यत् तत् तद् यत् । \newline
66. यदृ॒च र्‌चा यद् यदृ॒चा । \newline
67. ऋ॒चा तत् तदृ॒च र्‌चा तत् । \newline
68. तद् दृ॒ढम् दृ॒ढम् तत् तद् दृ॒ढम् । \newline
69. दृ॒ढम् पु॒रस्ता॑दुपयामाः पु॒रस्ता॑दुपयामा दृ॒ढम् दृ॒ढम् पु॒रस्ता॑दुपयामाः । \newline
70. पु॒रस्ता॑दुपयामा॒ यजु॑षा॒ यजु॑षा पु॒रस्ता॑दुपयामाः पु॒रस्ता॑दुपयामा॒ यजु॑षा । \newline
71. पु॒रस्ता॑दुपयामा॒ इति॑ पु॒रस्ता᳚त् - उ॒प॒या॒माः॒ । \newline
72. यजु॑षा गृह्यन्ते गृह्यन्ते॒ यजु॑षा॒ यजु॑षा गृह्यन्ते । \newline
73. गृ॒ह्य॒न्त॒ उ॒परि॑ष्टादुपयामा उ॒परि॑ष्टादुपयामा गृह्यन्ते गृह्यन्त उ॒परि॑ष्टादुपयामाः । \newline
74. उ॒परि॑ष्टादुपयामा ऋ॒च र्‌चो परि॑ष्टा दुपयामा उ॒परि॑ष्टा दुपयामा ऋ॒चा । \newline
75. उ॒परि॑ष्टादुपयामा॒ इत्यु॒परि॑ष्टात् - उ॒प॒या॒माः॒ । \newline
76. ऋ॒चा य॒ज्ञ्स्य॑ य॒ज्ञ्स्य॒ र्‌च र्‌चा य॒ज्ञ्स्य॑ । \newline
77. य॒ज्ञ्स्य॒ धृत्यै॒ धृत्यै॑ य॒ज्ञ्स्य॑ य॒ज्ञ्स्य॒ धृत्यै᳚ । \newline
78. धृत्या॒ इति॒ धृत्यै᳚ । \newline

\textbf{Ghana Paata } \newline

1. गर्भे॒णा वि॑ज्ञाते॒ना वि॑ज्ञातेन॒ गर्भे॑ण॒ गर्भे॒णा वि॑ज्ञातेन ब्रह्म॒हा ब्र॑ह्म॒हा ऽवि॑ज्ञातेन॒ गर्भे॑ण॒ गर्भे॒णा वि॑ज्ञातेन ब्रह्म॒हा । \newline
2. अवि॑ज्ञातेन ब्रह्म॒हा ब्र॑ह्म॒हा ऽवि॑ज्ञाते॒ना वि॑ज्ञातेन ब्रह्म॒हा ऽव॑भृ॒थ म॑वभृ॒थम् ब्र॑ह्म॒हा ऽवि॑ज्ञाते॒ना वि॑ज्ञातेन ब्रह्म॒हा ऽव॑भृ॒थम् । \newline
3. अवि॑ज्ञाते॒नेत्यवि॑ - ज्ञा॒ते॒न॒ । \newline
4. ब्र॒ह्म॒हा ऽव॑भृ॒थ म॑वभृ॒थम् ब्र॑ह्म॒हा ब्र॑ह्म॒हा ऽव॑भृ॒थ मवावा॑ वभृ॒थम् ब्र॑ह्म॒हा ब्र॑ह्म॒हा ऽव॑भृ॒थ मव॑ । \newline
5. ब्र॒ह्म॒हेति॑ ब्रह्म - हा । \newline
6. अ॒व॒भृ॒थ मवावा॑ वभृ॒थ म॑वभृ॒थ मव॑ यन्ति य॒न्त्यवा॑ वभृ॒थ म॑वभृ॒थ मव॑ यन्ति । \newline
7. अ॒व॒भृ॒थमित्य॑व - भृ॒थम् । \newline
8. अव॑ यन्ति य॒न्त्य वाव॑ यन्ति॒ परा॒ परा॑ य॒न्त्य वाव॑ यन्ति॒ परा᳚ । \newline
9. य॒न्ति॒ परा॒ परा॑ यन्ति यन्ति॒ परा᳚ स्था॒लीः स्था॒लीः परा॑ यन्ति यन्ति॒ परा᳚ स्था॒लीः । \newline
10. परा᳚ स्था॒लीः स्था॒लीः परा॒ परा᳚ स्था॒ली रस्य॒न् त्यस्य॑न्ति स्था॒लीः परा॒ परा᳚ स्था॒ली रस्य॑न्ति । \newline
11. स्था॒ली रस्य॒न् त्यस्य॑न्ति स्था॒लीः स्था॒ली रस्य॒न् त्युदु दस्य॑न्ति स्था॒लीः स्था॒ली रस्य॒न्त्युत् । \newline
12. अस्य॒न् त्युदु दस्य॒ न्त्यस्य॒ न्त्युद् वा॑य॒व्या॑नि वाय॒व्या᳚ न्युदस्य॒ न्त्यस्य॒ न्त्युद् वा॑य॒व्या॑नि । \newline
13. उद् वा॑य॒व्या॑नि वाय॒व्या᳚ न्युदुद् वा॑य॒व्या॑नि हरन्ति हरन्ति वाय॒व्या᳚ न्युदुद् वा॑य॒व्या॑नि हरन्ति । \newline
14. वा॒य॒व्या॑नि हरन्ति हरन्ति वाय॒व्या॑नि वाय॒व्या॑नि हरन्ति॒ तस्मा॒त् तस्मा᳚ द्धरन्ति वाय॒व्या॑नि वाय॒व्या॑नि हरन्ति॒ तस्मा᳚त् । \newline
15. ह॒र॒न्ति॒ तस्मा॒त् तस्मा᳚ द्धरन्ति हरन्ति॒ तस्मा॒थ् स्त्रियꣳ॒॒ स्त्रिय॒म् तस्मा᳚ द्धरन्ति हरन्ति॒ तस्मा॒थ् स्त्रिय᳚म् । \newline
16. तस्मा॒थ् स्त्रियꣳ॒॒ स्त्रिय॒म् तस्मा॒त् तस्मा॒थ् स्त्रिय॑म् जा॒ताम् जा॒ताꣳ स्त्रिय॒म् तस्मा॒त् तस्मा॒थ् स्त्रिय॑म् जा॒ताम् । \newline
17. स्त्रिय॑म् जा॒ताम् जा॒ताꣳ स्त्रियꣳ॒॒ स्त्रिय॑म् जा॒ताम् परा॒ परा॑ जा॒ताꣳ स्त्रियꣳ॒॒ स्त्रिय॑म् जा॒ताम् परा᳚ । \newline
18. जा॒ताम् परा॒ परा॑ जा॒ताम् जा॒ताम् परा᳚ ऽस्य न्त्यस्यन्ति॒ परा॑ जा॒ताम् जा॒ताम् परा᳚ ऽस्यन्ति । \newline
19. परा᳚ ऽस्य न्त्यस्यन्ति॒ परा॒ परा᳚ ऽस्य॒ न्त्युदु द॑स्यन्ति॒ परा॒ परा᳚ ऽस्य॒न्त्युत् । \newline
20. अ॒स्य॒ न्त्युदु द॑स्य न्त्यस्य॒ न्त्युत् पुमाꣳ॑स॒म् पुमाꣳ॑स॒ मुद॑स्य न्त्यस्य॒ न्त्युत् पुमाꣳ॑सम् । \newline
21. उत् पुमाꣳ॑स॒म् पुमाꣳ॑स॒ मुदुत् पुमाꣳ॑सꣳ हरन्ति हरन्ति॒ पुमाꣳ॑स॒ मुदुत् पुमाꣳ॑सꣳ हरन्ति । \newline
22. पुमाꣳ॑सꣳ हरन्ति हरन्ति॒ पुमाꣳ॑स॒म् पुमाꣳ॑सꣳ हरन्ति॒ यद् यद्ध॑रन्ति॒ पुमाꣳ॑स॒म् पुमाꣳ॑सꣳ हरन्ति॒ यत् । \newline
23. ह॒र॒न्ति॒ यद् यद्ध॑रन्ति हरन्ति॒ यत् पु॑रो॒रुच॑म् पुरो॒रुचं॒ ॅयद्ध॑रन्ति हरन्ति॒ यत् पु॑रो॒रुच᳚म् । \newline
24. यत् पु॑रो॒रुच॑म् पुरो॒रुचं॒ ॅयद् यत् पु॑रो॒रुच॒ माहाह॑ पुरो॒रुचं॒ ॅयद् यत् पु॑रो॒रुच॒ माह॑ । \newline
25. पु॒रो॒रुच॒ माहाह॑ पुरो॒रुच॑म् पुरो॒रुच॒ माह॒ यथा॒ यथा ऽऽह॑ पुरो॒रुच॑म् पुरो॒रुच॒ माह॒ यथा᳚ । \newline
26. पु॒रो॒रुच॒मिति॑ पुरः - रुच᳚म् । \newline
27. आह॒ यथा॒ यथा ऽऽहाह॒ यथा॒ वस्य॑से॒ वस्य॑से॒ यथा ऽऽहाह॒ यथा॒ वस्य॑से । \newline
28. यथा॒ वस्य॑से॒ वस्य॑से॒ यथा॒ यथा॒ वस्य॑स आ॒हर॑ त्या॒हर॑ति॒ वस्य॑से॒ यथा॒ यथा॒ वस्य॑स आ॒हर॑ति । \newline
29. वस्य॑स आ॒हर॑ त्या॒हर॑ति॒ वस्य॑से॒ वस्य॑स आ॒हर॑ति ता॒दृक् ता॒दृ गा॒हर॑ति॒ वस्य॑से॒ वस्य॑स आ॒हर॑ति ता॒दृक् । \newline
30. आ॒हर॑ति ता॒दृक् ता॒दृ गा॒हर॑ त्या॒हर॑ति ता॒दृ गे॒वैव ता॒दृ गा॒हर॑ त्या॒हर॑ति ता॒दृ गे॒व । \newline
31. आ॒हर॒तीत्या᳚ - हर॑ति । \newline
32. ता॒दृ गे॒वैव ता॒दृक् ता॒दृ गे॒व तत् तदे॒व ता॒दृक् ता॒दृ गे॒व तत् । \newline
33. ए॒व तत् तदे॒ वैव तद् यद् यत् तदे॒ वैव तद् यत् । \newline
34. तद् यद् यत् तत् तद् यद् ग्रह॒म् ग्रहं॒ ॅयत् तत् तद् यद् ग्रह᳚म् । \newline
35. यद् ग्रह॒म् ग्रहं॒ ॅयद् यद् ग्रह॑म् गृ॒ह्णाति॑ गृ॒ह्णाति॒ ग्रहं॒ ॅयद् यद् ग्रह॑म् गृ॒ह्णाति॑ । \newline
36. ग्रह॑म् गृ॒ह्णाति॑ गृ॒ह्णाति॒ ग्रह॒म् ग्रह॑म् गृ॒ह्णाति॒ यथा॒ यथा॑ गृ॒ह्णाति॒ ग्रह॒म् ग्रह॑म् गृ॒ह्णाति॒ यथा᳚ । \newline
37. गृ॒ह्णाति॒ यथा॒ यथा॑ गृ॒ह्णाति॑ गृ॒ह्णाति॒ यथा॒ वस्य॑से॒ वस्य॑से॒ यथा॑ गृ॒ह्णाति॑ गृ॒ह्णाति॒ यथा॒ वस्य॑से । \newline
38. यथा॒ वस्य॑से॒ वस्य॑से॒ यथा॒ यथा॒ वस्य॑स आ॒हृत्या॒ हृत्य॒ वस्य॑से॒ यथा॒ यथा॒ वस्य॑स आ॒हृत्य॑ । \newline
39. वस्य॑स आ॒हृत्या॒ हृत्य॒ वस्य॑से॒ वस्य॑स आ॒हृत्य॒ प्र प्राहृत्य॒ वस्य॑से॒ वस्य॑स आ॒हृत्य॒ प्र । \newline
40. आ॒हृत्य॒ प्र प्राहृत्या॒ हृत्य॒ प्राहाह॒ प्राहृत्या॒ हृत्य॒ प्राह॑ । \newline
41. आ॒हृत्येत्या᳚ - हृत्य॑ । \newline
42. प्राहाह॒ प्र प्राह॑ ता॒दृक् ता॒दृ गाह॒ प्र प्राह॑ ता॒दृक् । \newline
43. आह॑ ता॒दृक् ता॒दृ गाहाह॑ ता॒दृ गे॒वैव ता॒दृ गाहाह॑ ता॒दृ गे॒व । \newline
44. ता॒दृ गे॒वैव ता॒दृक् ता॒दृ गे॒व तत् तदे॒व ता॒दृक् ता॒दृ गे॒व तत् । \newline
45. ए॒व तत् तदे॒ वैव तद् यद् यत् तदे॒ वैव तद् यत् । \newline
46. तद् यद् यत् तत् तद् यथ् सा॒दय॑ति सा॒दय॑ति॒ यत् तत् तद् यथ् सा॒दय॑ति । \newline
47. यथ् सा॒दय॑ति सा॒दय॑ति॒ यद् यथ् सा॒दय॑ति॒ यथा॒ यथा॑ सा॒दय॑ति॒ यद् यथ् सा॒दय॑ति॒ यथा᳚ । \newline
48. सा॒दय॑ति॒ यथा॒ यथा॑ सा॒दय॑ति सा॒दय॑ति॒ यथा॒ वस्य॑से॒ वस्य॑से॒ यथा॑ सा॒दय॑ति सा॒दय॑ति॒ यथा॒ वस्य॑से । \newline
49. यथा॒ वस्य॑से॒ वस्य॑से॒ यथा॒ यथा॒ वस्य॑स उपनि॒धा यो॑पनि॒धाय॒ वस्य॑से॒ यथा॒ यथा॒ वस्य॑स उपनि॒धाय॑ । \newline
50. वस्य॑स उपनि॒धा यो॑पनि॒धाय॒ वस्य॑से॒ वस्य॑स उपनि॒धाया॑ प॒क्राम॑ त्यप॒क्राम॑ त्युपनि॒धाय॒ वस्य॑से॒ वस्य॑स उपनि॒धाया॑ प॒क्राम॑ति । \newline
51. उ॒प॒नि॒धाया॑ प॒क्राम॑ त्यप॒क्राम॑ त्युपनि॒धा यो॑पनि॒धाया॑ प॒क्राम॑ति ता॒दृक् ता॒दृ ग॑प॒क्राम॑ त्युपनि॒धा यो॑पनि॒धाया॑ प॒क्राम॑ति ता॒दृक् । \newline
52. उ॒प॒नि॒धायेत्यु॑प - नि॒धाय॑ । \newline
53. अ॒प॒क्राम॑ति ता॒दृक् ता॒दृ ग॑प॒क्राम॑ त्यप॒क्राम॑ति ता॒दृ गे॒वैव ता॒दृ ग॑प॒क्राम॑ त्यप॒क्राम॑ति ता॒दृ गे॒व । \newline
54. अ॒प॒क्राम॒तीत्य॑प - क्राम॑ति । \newline
55. ता॒दृ गे॒वैव ता॒दृक् ता॒दृ गे॒व तत् तदे॒व ता॒दृक् ता॒दृ गे॒व तत् । \newline
56. ए॒व तत् तदे॒ वैव तद् यद् यत् तदे॒ वैव तद् यत् । \newline
57. तद् यद् यत् तत् तद् यद् वै वै यत् तत् तद् यद् वै । \newline
58. यद् वै वै यद् यद् वै य॒ज्ञ्स्य॑ य॒ज्ञ्स्य॒ वै यद् यद् वै य॒ज्ञ्स्य॑ । \newline
59. वै य॒ज्ञ्स्य॑ य॒ज्ञ्स्य॒ वै वै य॒ज्ञ्स्य॒ साम्ना॒ साम्ना॑ य॒ज्ञ्स्य॒ वै वै य॒ज्ञ्स्य॒ साम्ना᳚ । \newline
60. य॒ज्ञ्स्य॒ साम्ना॒ साम्ना॑ य॒ज्ञ्स्य॑ य॒ज्ञ्स्य॒ साम्ना॒ यजु॑षा॒ यजु॑षा॒ साम्ना॑ य॒ज्ञ्स्य॑ य॒ज्ञ्स्य॒ साम्ना॒ यजु॑षा । \newline
61. साम्ना॒ यजु॑षा॒ यजु॑षा॒ साम्ना॒ साम्ना॒ यजु॑षा क्रि॒यते᳚ क्रि॒यते॒ यजु॑षा॒ साम्ना॒ साम्ना॒ यजु॑षा क्रि॒यते᳚ । \newline
62. यजु॑षा क्रि॒यते᳚ क्रि॒यते॒ यजु॑षा॒ यजु॑षा क्रि॒यते॑ शिथि॒लꣳ शि॑थि॒लम् क्रि॒यते॒ यजु॑षा॒ यजु॑षा क्रि॒यते॑ शिथि॒लम् । \newline
63. क्रि॒यते॑ शिथि॒लꣳ शि॑थि॒लम् क्रि॒यते᳚ क्रि॒यते॑ शिथि॒लम् तत् तच्छि॑थि॒लम् क्रि॒यते᳚ क्रि॒यते॑ शिथि॒लम् तत् । \newline
64. शि॒थि॒लम् तत् तच्छि॑थि॒लꣳ शि॑थि॒लम् तद् यद् यत् तच्छि॑थि॒लꣳ शि॑थि॒लम् तद् यत् । \newline
65. तद् यद् यत् तत् तद् यदृ॒च र्‌चा यत् तत् तद् यदृ॒चा । \newline
66. यदृ॒च र्‌चा यद् यदृ॒चा तत् तदृ॒चा यद् यदृ॒चा तत् । \newline
67. ऋ॒चा तत् तदृ॒च र्‌चा तद् दृ॒ढम् दृ॒ढम् तदृ॒च र्‌चा तद् दृ॒ढम् । \newline
68. तद् दृ॒ढम् दृ॒ढम् तत् तद् दृ॒ढम् पु॒रस्ता॑दुपयामाः पु॒रस्ता॑दुपयामा दृ॒ढम् तत् तद् दृ॒ढम् पु॒रस्ता॑दुपयामाः । \newline
69. दृ॒ढम् पु॒रस्ता॑दुपयामाः पु॒रस्ता॑दुपयामा दृ॒ढम् दृ॒ढम् पु॒रस्ता॑दुपयामा॒ यजु॑षा॒ यजु॑षा पु॒रस्ता॑दुपयामा दृ॒ढम् दृ॒ढम् पु॒रस्ता॑दुपयामा॒ यजु॑षा । \newline
70. पु॒रस्ता॑दुपयामा॒ यजु॑षा॒ यजु॑षा पु॒रस्ता॑दुपयामाः पु॒रस्ता॑दुपयामा॒ यजु॑षा गृह्यन्ते गृह्यन्ते॒ यजु॑षा पु॒रस्ता॑दुपयामाः पु॒रस्ता॑दुपयामा॒ यजु॑षा गृह्यन्ते । \newline
71. पु॒रस्ता॑दुपयामा॒ इति॑ पु॒रस्ता᳚त् - उ॒प॒या॒माः॒ । \newline
72. यजु॑षा गृह्यन्ते गृह्यन्ते॒ यजु॑षा॒ यजु॑षा गृह्यन्त उ॒परि॑ष्टादुपयामा उ॒परि॑ष्टादुपयामा गृह्यन्ते॒ यजु॑षा॒ यजु॑षा गृह्यन्त उ॒परि॑ष्टादुपयामाः । \newline
73. गृ॒ह्य॒न्त॒ उ॒परि॑ष्टादुपयामा उ॒परि॑ष्टादुपयामा गृह्यन्ते गृह्यन्त उ॒परि॑ष्टादुपयामा ऋ॒च र्‌चोपरि॑ष्टादुपयामा गृह्यन्ते गृह्यन्त उ॒परि॑ष्टादुपयामा ऋ॒चा । \newline
74. उ॒परि॑ष्टादुपयामा ऋ॒च र्‌चोपरि॑ष्टादुपयामा उ॒परि॑ष्टादुपयामा ऋ॒चा य॒ज्ञ्स्य॑ य॒ज्ञ्स्य॒ र्‌चोपरि॑ष्टादुपयामा उ॒परि॑ष्टादुपयामा ऋ॒चा य॒ज्ञ्स्य॑ । \newline
75. उ॒परि॑ष्टादुपयामा॒ इत्यु॒परि॑ष्टात् - उ॒प॒या॒माः॒ । \newline
76. ऋ॒चा य॒ज्ञ्स्य॑ य॒ज्ञ्स्य॒ र्‌च र्‌चा य॒ज्ञ्स्य॒ धृत्यै॒ धृत्यै॑ य॒ज्ञ्स्य॒ र्‌च र्‌चा य॒ज्ञ्स्य॒ धृत्यै᳚ । \newline
77. य॒ज्ञ्स्य॒ धृत्यै॒ धृत्यै॑ य॒ज्ञ्स्य॑ य॒ज्ञ्स्य॒ धृत्यै᳚ । \newline
78. धृत्या॒ इति॒ धृत्यै᳚ । \newline
\pagebreak
\markright{ TS 6.5.11.1  \hfill https://www.vedavms.in \hfill}

\section{ TS 6.5.11.1 }

\textbf{TS 6.5.11.1 } \newline
\textbf{Samhita Paata} \newline

प्रान्यानि॒ पात्रा॑णि यु॒ज्यन्ते॒ नान्यानि॒ यानि॑ परा॒चीना॑नि प्रयु॒ज्यन्ते॒ऽमुमे॒व तैर्लो॒कम॒भि ज॑यति॒ परा॑ङिव॒ ह्य॑सौ लो॒को यानि॒ पु॑नःप्रयु॒ज्यन्त॑ इ॒ममे॒व तैर्लो॒कम॒भि ज॑यति॒ पुनः॑ पुनरिव॒ ह्य॑यं ॅलो॒कः प्रान्यानि॒ पात्रा॑णि यु॒ज्यन्ते॒ नान्यानि॒ यानि॑ परा॒चीना॑नि प्रयु॒ज्यन्ते॒ तान्यन्वोष॑धयः॒ परा॑ भवन्ति॒ यानि॒ पुनः॑- [  ] \newline

\textbf{Pada Paata} \newline

प्रेति॑ । अ॒न्यानि॑ । पात्रा॑णि । यु॒ज्यन्ते᳚ । न । अ॒न्यानि॑ । यानि॑ । प॒रा॒चीना॑नि । प्र॒यु॒ज्यन्त॒ इति॑ प्र - यु॒ज्यन्ते᳚ । अ॒मुम् । ए॒व । तैः । लो॒कम् । अ॒भीति॑ । ज॒य॒ति॒ । पराङ्॑ । इ॒व॒ । हि । अ॒सौ । लो॒कः । यानि॑ । पुनः॑ । प्र॒यु॒ज्यन्त॒ इति॑ प्र - यु॒ज्यन्ते᳚ । इ॒मम् । ए॒व । तैः । लो॒कम् । अ॒भीति॑ । ज॒य॒ति॒ । पुनः॑पुन॒रिति॒ पुनः॑ - पु॒नः॒ । इ॒व॒ । हि । अ॒यम् । लो॒कः । प्रेति॑ । अ॒न्यानि॑ । पात्रा॑णि । यु॒ज्यन्ते᳚ । न । अ॒न्यानि॑ । यानि॑ । प॒रा॒चीना॑नि । प्र॒यु॒ज्यन्त॒ इति॑ प्र - यु॒ज्यन्ते᳚ । तानि॑ । अन्विति॑ । ओष॑धयः । परेति॑ । भ॒व॒न्ति॒ । यानि॑ । पुनः॑ ।  \newline


\textbf{Krama Paata} \newline

प्रान्यानि॑ । अ॒न्यानि॒ पात्रा॑णि । पात्रा॑णि यु॒ज्यन्ते᳚ । यु॒ज्यन्ते॒ न । नान्यानि॑ । अ॒न्यानि॒ यानि॑ । यानि॑ परा॒चीना॑नि । प॒रा॒चीना॑नि प्रयु॒ज्यन्ते᳚ । प्र॒यु॒ज्यन्ते॒ऽमुम् । प्र॒यु॒ज्यन्त॒ इति॑ प्र - यु॒जन्ते᳚ । अ॒मुमे॒व । ए॒व तैः । तैर् लो॒कम् । लो॒कम॒भि । अ॒भि ज॑यति । ज॒य॒ति॒ पराङ्॑ । परा॑ङिव । इ॒व॒ हि । ह्य॑सौ । अ॒सौ लो॒कः । लो॒को यानि॑ । यानि॒ पुनः॑ । पुनः॑ प्रयु॒ज्यन्ते᳚ । प्र॒यु॒ज्यन्त॑ इ॒मम् । प्र॒यु॒ज्यन्त॒ इति॑ प्र - यु॒ज्यन्ते᳚ । इ॒ममे॒व । ए॒व तैः । तैर् लो॒कम् । लो॒कम॒भि । अ॒भि ज॑यति । ज॒य॒ति॒ पुनः॑पुनः । पुनः॑पुनरिव । पुनः॑पुन॒रिति॒ पुनः॑ - पु॒नः॒ । इ॒व॒ हि । ह्य॑यम् । अ॒यम् ॅलो॒कः । लो॒कः प्र । प्रान्यानि॑ । अ॒न्यानि॒ पात्रा॑णि । पात्रा॑णि यु॒ज्यन्ते᳚ । यु॒ज्यन्ते॒ न । नान्यानि॑ । अ॒न्यानि॒ यानि॑ । यानि॑ परा॒चीना॑नि । प॒रा॒चीना॑नि प्रयु॒ज्यन्ते᳚ । प्र॒यु॒ज्यन्ते॒ तानि॑ । प्र॒यु॒ज्यन्त॒ इति॑ प्र - यु॒ज्यन्ते᳚ । तान्यनु॑ । अन्वोष॑धयः । ओष॑धयः॒ परा᳚ । परा॑ भवन्ति । भ॒व॒न्ति॒ यानि॑ । यानि॒ पुनः॑ । पुनः॑ प्रयु॒ज्यन्ते᳚ \newline

\textbf{Jatai Paata} \newline

1. प्रा न्यान्य॒ न्यानि॒ प्र प्रा न्यानि॑ । \newline
2. अ॒न्यानि॒ पात्रा॑णि॒ पात्रा᳚ ण्य॒न्या न्य॒न्यानि॒ पात्रा॑णि । \newline
3. पात्रा॑णि यु॒ज्यन्ते॑ यु॒ज्यन्ते॒ पात्रा॑णि॒ पात्रा॑णि यु॒ज्यन्ते᳚ । \newline
4. यु॒ज्यन्ते॒ न न यु॒ज्यन्ते॑ यु॒ज्यन्ते॒ न । \newline
5. ना न्यान्य॒ न्यानि॒ न नान्यानि॑ । \newline
6. अ॒न्यानि॒ यानि॒ यान्य॒ न्यान्य॒ न्यानि॒ यानि॑ । \newline
7. यानि॑ परा॒चीना॑नि परा॒चीना॑नि॒ यानि॒ यानि॑ परा॒चीना॑नि । \newline
8. प॒रा॒चीना॑नि प्रयु॒ज्यन्ते᳚ प्रयु॒ज्यन्ते॑ परा॒चीना॑नि परा॒चीना॑नि प्रयु॒ज्यन्ते᳚ । \newline
9. प्र॒यु॒ज्यन्ते॒ ऽमु म॒मुम् प्र॑यु॒ज्यन्ते᳚ प्रयु॒ज्यन्ते॒ ऽमुम् । \newline
10. प्र॒यु॒ज्यन्त॒ इति॑ प्र - यु॒ज्यन्ते᳚ । \newline
11. अ॒मु मे॒वै वामु म॒मु मे॒व । \newline
12. ए॒व तै स्तै रे॒वैव तैः । \newline
13. तैर् लो॒कम् ॅलो॒कम् तै स्तैर् लो॒कम् । \newline
14. लो॒क म॒भ्य॑भि लो॒कम् ॅलो॒क म॒भि । \newline
15. अ॒भि ज॑यति जय त्य॒भ्य॑भि ज॑यति । \newline
16. ज॒य॒ति॒ परा॒ङ् परा᳚ङ् जयति जयति॒ पराङ्॑ । \newline
17. परा॑ ङिवेव॒ परा॒ङ् परा॑ ङिव । \newline
18. इ॒व॒ हि हीवे॑व॒ हि । \newline
19. ह्य॑सा व॒सौ हि ह्य॑सौ । \newline
20. अ॒सौ लो॒को लो॒को॑ ऽसा व॒सौ लो॒कः । \newline
21. लो॒को यानि॒ यानि॑ लो॒को लो॒को यानि॑ । \newline
22. यानि॒ पुनः॒ पुन॒र् यानि॒ यानि॒ पुनः॑ । \newline
23. पुनः॑ प्रयु॒ज्यन्ते᳚ प्रयु॒ज्यन्ते॒ पुनः॒ पुनः॑ प्रयु॒ज्यन्ते᳚ । \newline
24. प्र॒यु॒ज्यन्त॑ इ॒म मि॒मम् प्र॑यु॒ज्यन्ते᳚ प्रयु॒ज्यन्त॑ इ॒मम् । \newline
25. प्र॒यु॒ज्यन्त॒ इति॑ प्र - यु॒ज्यन्ते᳚ । \newline
26. इ॒म मे॒वैवे म मि॒म मे॒व । \newline
27. ए॒व तै स्तै रे॒वैव तैः । \newline
28. तैर् लो॒कम् ॅलो॒कम् तै स्तैर् लो॒कम् । \newline
29. लो॒क म॒भ्य॑भि लो॒कम् ॅलो॒क म॒भि । \newline
30. अ॒भि ज॑यति जय त्य॒भ्य॑भि ज॑यति । \newline
31. ज॒य॒ति॒ पुनः॑पुनः॒ पुनः॑पुनर् जयति जयति॒ पुनः॑पुनः । \newline
32. पुनः॑पुन रिवेव॒ पुनः॑पुनः॒ पुनः॑पुन रिव । \newline
33. पुनः॑पुन॒रिति॒ पुनः॑ - पु॒नः॒ । \newline
34. इ॒व॒ हि हीवे॑व॒ हि । \newline
35. ह्य॑य म॒यꣳ हि ह्य॑यम् । \newline
36. अ॒यम् ॅलो॒को लो॒को॑ ऽय म॒यम् ॅलो॒कः । \newline
37. लो॒कः प्र प्र लो॒को लो॒कः प्र । \newline
38. प्रान्या न्य॒न्यानि॒ प्र प्रा न्यानि॑ । \newline
39. अ॒न्यानि॒ पात्रा॑णि॒ पात्रा᳚ ण्य॒न्या न्य॒न्यानि॒ पात्रा॑णि । \newline
40. पात्रा॑णि यु॒ज्यन्ते॑ यु॒ज्यन्ते॒ पात्रा॑णि॒ पात्रा॑णि यु॒ज्यन्ते᳚ । \newline
41. यु॒ज्यन्ते॒ न न यु॒ज्यन्ते॑ यु॒ज्यन्ते॒ न । \newline
42. नान्या न्य॒न्यानि॒ न ना न्यानि॑ । \newline
43. अ॒न्यानि॒ यानि॒ या न्य॒न्या न्य॒न्यानि॒ यानि॑ । \newline
44. यानि॑ परा॒चीना॑नि परा॒चीना॑नि॒ यानि॒ यानि॑ परा॒चीना॑नि । \newline
45. प॒रा॒चीना॑नि प्रयु॒ज्यन्ते᳚ प्रयु॒ज्यन्ते॑ परा॒चीना॑नि परा॒चीना॑नि प्रयु॒ज्यन्ते᳚ । \newline
46. प्र॒यु॒ज्यन्ते॒ तानि॒ तानि॑ प्रयु॒ज्यन्ते᳚ प्रयु॒ज्यन्ते॒ तानि॑ । \newline
47. प्र॒यु॒ज्यन्त॒ इति॑ प्र - यु॒ज्यन्ते᳚ । \newline
48. तान्यन् वनु॒ तानि॒ तान् यनु॑ । \newline
49. अन्वोष॑धय॒ ओष॑ध॒यो ऽन्वन् वोष॑धयः । \newline
50. ओष॑धयः॒ परा॒ परौष॑धय॒ ओष॑धयः॒ परा᳚ । \newline
51. परा॑ भवन्ति भवन्ति॒ परा॒ परा॑ भवन्ति । \newline
52. भ॒व॒न्ति॒ यानि॒ यानि॑ भवन्ति भवन्ति॒ यानि॑ । \newline
53. यानि॒ पुनः॒ पुन॒र् यानि॒ यानि॒ पुनः॑ । \newline
54. पुनः॑ प्रयु॒ज्यन्ते᳚ प्रयु॒ज्यन्ते॒ पुनः॒ पुनः॑ प्रयु॒ज्यन्ते᳚ । \newline

\textbf{Ghana Paata } \newline

1. प्रान्या न्य॒न्यानि॒ प्र प्रान्यानि॒ पात्रा॑णि॒ पात्रा᳚ ण्य॒न्यानि॒ प्र प्रान्यानि॒ पात्रा॑णि । \newline
2. अ॒न्यानि॒ पात्रा॑णि॒ पात्रा᳚ ण्य॒न्या न्य॒न्यानि॒ पात्रा॑णि यु॒ज्यन्ते॑ यु॒ज्यन्ते॒ पात्रा᳚ ण्य॒न्या न्य॒न्यानि॒ पात्रा॑णि यु॒ज्यन्ते᳚ । \newline
3. पात्रा॑णि यु॒ज्यन्ते॑ यु॒ज्यन्ते॒ पात्रा॑णि॒ पात्रा॑णि यु॒ज्यन्ते॒ न न यु॒ज्यन्ते॒ पात्रा॑णि॒ पात्रा॑णि यु॒ज्यन्ते॒ न । \newline
4. यु॒ज्यन्ते॒ न न यु॒ज्यन्ते॑ यु॒ज्यन्ते॒ नान्या न्य॒न्यानि॒ न यु॒ज्यन्ते॑ यु॒ज्यन्ते॒ नान्यानि॑ । \newline
5. नान्या न्य॒न्यानि॒ न नान्यानि॒ यानि॒ यान्य॒ न्यानि॒ न नान्यानि॒ यानि॑ । \newline
6. अ॒न्यानि॒ यानि॒ यान्य॒ न्यान्य॒ न्यानि॒ यानि॑ परा॒चीना॑नि परा॒चीना॑नि॒ यान्य॒ न्यान्य॒ न्यानि॒ यानि॑ परा॒चीना॑नि । \newline
7. यानि॑ परा॒चीना॑नि परा॒चीना॑नि॒ यानि॒ यानि॑ परा॒चीना॑नि प्रयु॒ज्यन्ते᳚ प्रयु॒ज्यन्ते॑ परा॒चीना॑नि॒ यानि॒ यानि॑ परा॒चीना॑नि प्रयु॒ज्यन्ते᳚ । \newline
8. प॒रा॒चीना॑नि प्रयु॒ज्यन्ते᳚ प्रयु॒ज्यन्ते॑ परा॒चीना॑नि परा॒चीना॑नि प्रयु॒ज्यन्ते॒ ऽमु म॒मुम् प्र॑यु॒ज्यन्ते॑ परा॒चीना॑नि परा॒चीना॑नि प्रयु॒ज्यन्ते॒ ऽमुम् । \newline
9. प्र॒यु॒ज्यन्ते॒ ऽमु म॒मुम् प्र॑यु॒ज्यन्ते᳚ प्रयु॒ज्यन्ते॒ ऽमु मे॒वैवामुम् प्र॑यु॒ज्यन्ते᳚ प्रयु॒ज्यन्ते॒ ऽमु मे॒व । \newline
10. प्र॒यु॒ज्यन्त॒ इति॑ प्र - यु॒ज्यन्ते᳚ । \newline
11. अ॒मु मे॒वै वामु म॒मु मे॒व तै स्तै रे॒वामु म॒मु मे॒व तैः । \newline
12. ए॒व तै स्तै रे॒वैव तैर् लो॒कम् ॅलो॒कम् तै रे॒वैव तैर् लो॒कम् । \newline
13. तैर् लो॒कम् ॅलो॒कम् तै स्तैर् लो॒क म॒भ्य॑भि लो॒कम् तै स्तैर् लो॒क म॒भि । \newline
14. लो॒क म॒भ्य॑भि लो॒कम् ॅलो॒क म॒भि ज॑यति जय त्य॒भि लो॒कम् ॅलो॒क म॒भि ज॑यति । \newline
15. अ॒भि ज॑यति जय त्य॒भ्य॑भि ज॑यति॒ परा॒ङ् परा᳚ङ् जय त्य॒भ्य॑भि ज॑यति॒ पराङ्॑ । \newline
16. ज॒य॒ति॒ परा॒ङ् परा᳚ङ् जयति जयति॒ परा॑ ङिवेव॒ परा᳚ङ् जयति जयति॒ परा॑ ङिव । \newline
17. परा॑ ङिवेव॒ परा॒ङ् परा॑ ङिव॒ हि हीव॒ परा॒ङ् परा॑ ङिव॒ हि । \newline
18. इ॒व॒ हि हीवे॑व॒ ह्य॑सा व॒सौ हीवे॑व॒ ह्य॑सौ । \newline
19. ह्य॑सा व॒सौ हि ह्य॑सौ लो॒को लो॒को॑ ऽसौ हि ह्य॑सौ लो॒कः । \newline
20. अ॒सौ लो॒को लो॒को॑ ऽसा व॒सौ लो॒को यानि॒ यानि॑ लो॒को॑ ऽसा व॒सौ लो॒को यानि॑ । \newline
21. लो॒को यानि॒ यानि॑ लो॒को लो॒को यानि॒ पुनः॒ पुन॒र् यानि॑ लो॒को लो॒को यानि॒ पुनः॑ । \newline
22. यानि॒ पुनः॒ पुन॒र् यानि॒ यानि॒ पुनः॑ प्रयु॒ज्यन्ते᳚ प्रयु॒ज्यन्ते॒ पुन॒र् यानि॒ यानि॒ पुनः॑ प्रयु॒ज्यन्ते᳚ । \newline
23. पुनः॑ प्रयु॒ज्यन्ते᳚ प्रयु॒ज्यन्ते॒ पुनः॒ पुनः॑ प्रयु॒ज्यन्त॑ इ॒म मि॒मम् प्र॑यु॒ज्यन्ते॒ पुनः॒ पुनः॑ प्रयु॒ज्यन्त॑ इ॒मम् । \newline
24. प्र॒यु॒ज्यन्त॑ इ॒म मि॒मम् प्र॑यु॒ज्यन्ते᳚ प्रयु॒ज्यन्त॑ इ॒म मे॒वैवेमम् प्र॑यु॒ज्यन्ते᳚ प्रयु॒ज्यन्त॑ इ॒म मे॒व । \newline
25. प्र॒यु॒ज्यन्त॒ इति॑ प्र - यु॒ज्यन्ते᳚ । \newline
26. इ॒म मे॒वै वेम मि॒म मे॒व तै स्तै रे॒वेम मि॒म मे॒व तैः । \newline
27. ए॒व तै स्तै रे॒वैव तैर् लो॒कम् ॅलो॒कम् तै रे॒वैव तैर् लो॒कम् । \newline
28. तैर् लो॒कम् ॅलो॒कम् तै स्तैर् लो॒क म॒भ्य॑भि लो॒कम् तै स्तैर् लो॒क म॒भि । \newline
29. लो॒क म॒भ्य॑भि लो॒कम् ॅलो॒क म॒भि ज॑यति जय त्य॒भि लो॒कम् ॅलो॒क म॒भि ज॑यति । \newline
30. अ॒भि ज॑यति जय त्य॒भ्य॑भि ज॑यति॒ पुनः॑पुनः॒ पुनः॑पुनर् जय त्य॒भ्य॑भि ज॑यति॒ पुनः॑पुनः । \newline
31. ज॒य॒ति॒ पुनः॑पुनः॒ पुनः॑पुनर् जयति जयति॒ पुनः॑पुन रिवेव॒ पुनः॑पुनर् जयति जयति॒ पुनः॑पुन रिव । \newline
32. पुनः॑पुन रिवेव॒ पुनः॑पुनः॒ पुनः॑पुन रिव॒ हि हीव॒ पुनः॑पुनः॒ पुनः॑पुन रिव॒ हि । \newline
33. पुनः॑पुन॒रिति॒ पुनः॑ - पु॒नः॒ । \newline
34. इ॒व॒ हि हीवे॑व॒ ह्य॑य म॒यꣳ हीवे॑व॒ ह्य॑यम् । \newline
35. ह्य॑य म॒यꣳ हि ह्य॑यम् ॅलो॒को लो॒को॑ ऽयꣳ हि ह्य॑यम् ॅलो॒कः । \newline
36. अ॒यम् ॅलो॒को लो॒को॑ ऽय म॒यम् ॅलो॒कः प्र प्र लो॒को॑ ऽय म॒यम् ॅलो॒कः प्र । \newline
37. लो॒कः प्र प्र लो॒को लो॒कः प्रान्या न्य॒न्यानि॒ प्र लो॒को लो॒कः प्रान्यानि॑ । \newline
38. प्रान्या न्य॒न्यानि॒ प्र प्रान्यानि॒ पात्रा॑णि॒ पात्रा᳚ ण्य॒न्यानि॒ प्र प्रान्यानि॒ पात्रा॑णि । \newline
39. अ॒न्यानि॒ पात्रा॑णि॒ पात्रा᳚ ण्य॒न्या न्य॒न्यानि॒ पात्रा॑णि यु॒ज्यन्ते॑ यु॒ज्यन्ते॒ पात्रा᳚ ण्य॒न्या न्य॒न्यानि॒ पात्रा॑णि यु॒ज्यन्ते᳚ । \newline
40. पात्रा॑णि यु॒ज्यन्ते॑ यु॒ज्यन्ते॒ पात्रा॑णि॒ पात्रा॑णि यु॒ज्यन्ते॒ न न यु॒ज्यन्ते॒ पात्रा॑णि॒ पात्रा॑णि यु॒ज्यन्ते॒ न । \newline
41. यु॒ज्यन्ते॒ न न यु॒ज्यन्ते॑ यु॒ज्यन्ते॒ नान्या न्य॒न्यानि॒ न यु॒ज्यन्ते॑ यु॒ज्यन्ते॒ नान्यानि॑ । \newline
42. नान्या न्य॒न्यानि॒ न नान्यानि॒ यानि॒ यान्य॒न्यानि॒ न नान्यानि॒ यानि॑ । \newline
43. अ॒न्यानि॒ यानि॒ यान्य॒न्या न्य॒न्यानि॒ यानि॑ परा॒चीना॑नि परा॒चीना॑नि॒ यान्य॒न्या न्य॒न्यानि॒ यानि॑ परा॒चीना॑नि । \newline
44. यानि॑ परा॒चीना॑नि परा॒चीना॑नि॒ यानि॒ यानि॑ परा॒चीना॑नि प्रयु॒ज्यन्ते᳚ प्रयु॒ज्यन्ते॑ परा॒चीना॑नि॒ यानि॒ यानि॑ परा॒चीना॑नि प्रयु॒ज्यन्ते᳚ । \newline
45. प॒रा॒चीना॑नि प्रयु॒ज्यन्ते᳚ प्रयु॒ज्यन्ते॑ परा॒चीना॑नि परा॒चीना॑नि प्रयु॒ज्यन्ते॒ तानि॒ तानि॑ प्रयु॒ज्यन्ते॑ परा॒चीना॑नि परा॒चीना॑नि प्रयु॒ज्यन्ते॒ तानि॑ । \newline
46. प्र॒यु॒ज्यन्ते॒ तानि॒ तानि॑ प्रयु॒ज्यन्ते᳚ प्रयु॒ज्यन्ते॒ तान्यन्वनु॒ तानि॑ प्रयु॒ज्यन्ते᳚ प्रयु॒ज्यन्ते॒ तान्यनु॑ । \newline
47. प्र॒यु॒ज्यन्त॒ इति॑ प्र - यु॒ज्यन्ते᳚ । \newline
48. तान्यन् वनु॒ तानि॒ तान्यन् वोष॑धय॒ ओष॑ध॒यो ऽनु॒ तानि॒ तान् यन् वोष॑धयः । \newline
49. अन् वोष॑धय॒ ओष॑ध॒यो ऽन्वन् वोष॑धयः॒ परा॒ परौ ष॑ध॒यो ऽन्वन् वोष॑धयः॒ परा᳚ । \newline
50. ओष॑धयः॒ परा॒ परौष॑धय॒ ओष॑धयः॒ परा॑ भवन्ति भवन्ति॒ परौष॑धय॒ ओष॑धयः॒ परा॑ भवन्ति । \newline
51. परा॑ भवन्ति भवन्ति॒ परा॒ परा॑ भवन्ति॒ यानि॒ यानि॑ भवन्ति॒ परा॒ परा॑ भवन्ति॒ यानि॑ । \newline
52. भ॒व॒न्ति॒ यानि॒ यानि॑ भवन्ति भवन्ति॒ यानि॒ पुनः॒ पुन॒र् यानि॑ भवन्ति भवन्ति॒ यानि॒ पुनः॑ । \newline
53. यानि॒ पुनः॒ पुन॒र् यानि॒ यानि॒ पुनः॑ प्रयु॒ज्यन्ते᳚ प्रयु॒ज्यन्ते॒ पुन॒र् यानि॒ यानि॒ पुनः॑ प्रयु॒ज्यन्ते᳚ । \newline
54. पुनः॑ प्रयु॒ज्यन्ते᳚ प्रयु॒ज्यन्ते॒ पुनः॒ पुनः॑ प्रयु॒ज्यन्ते॒ तानि॒ तानि॑ प्रयु॒ज्यन्ते॒ पुनः॒ पुनः॑ प्रयु॒ज्यन्ते॒ तानि॑ । \newline
\pagebreak
\markright{ TS 6.5.11.2  \hfill https://www.vedavms.in \hfill}

\section{ TS 6.5.11.2 }

\textbf{TS 6.5.11.2 } \newline
\textbf{Samhita Paata} \newline

प्रयु॒ज्यन्ते॒ तान्यन्वोष॑धयः॒ पुन॒रा भ॑वन्ति॒ प्रान्यानि॒ पात्रा॑णि यु॒ज्यन्ते॒ नान्यानि॒ यानि॑ परा॒चीना॑नि प्रयु॒ज्यन्ते॒ तान्यन्वा॑र॒ण्याः प॒शवोऽर॑ण्य॒मप॑ यन्ति॒ यानि॒ पुनः॑ प्रयु॒ज्यन्ते॒ तान्यनु॑ ग्रा॒म्याः प॒शवो॒ ग्राम॑मु॒पाव॑यन्ति॒ यो वै ग्रहा॑णां नि॒दानं॒ ॅवेद॑ नि॒दान॑वान् भव॒त्याज्य॒मित्यु॒क्थं तद्वै ग्रहा॑णां नि॒दानं॒ ॅयदु॑पाꣳ॒॒शु शꣳस॑ति॒ त- [  ] \newline

\textbf{Pada Paata} \newline

प्र॒यु॒ज्यन्त॒ इति॑ प्र - यु॒ज्यन्ते᳚ । तानि॑ । अन्विति॑ । ओष॑धयः । पुनः॑ । एति॑ । भ॒व॒न्ति॒ । प्रेति॑ । अ॒न्यानि॑ । पात्रा॑णि । यु॒ज्यन्ते᳚ । न । अ॒न्यानि॑ । यानि॑ । प॒रा॒चीना॑नि । प्र॒यु॒ज्यन्त॒ इति॑ प्र-यु॒ज्यन्ते᳚ । तानि॑ । अन्विति॑ । आ॒र॒ण्याः । प॒शवः॑ । अर॑ण्यम् । अपेति॑ । य॒न्ति॒ । यानि॑ । पुनः॑ । प्र॒यु॒ज्यन्त॒ इति॑ प्र - यु॒ज्यन्ते᳚ । तानि॑ । अन्विति॑ । ग्रा॒म्याः । प॒शवः॑ । ग्राम᳚म् । उ॒पाव॑य॒न्तीत्यु॑प-अव॑यन्ति । यः । वै । ग्रहा॑णाम् । नि॒दान॒मिति॑ नि - दान᳚म् । वेद॑ । नि॒दान॑वा॒निति॑ नि॒दान॑ - वा॒न् । भ॒व॒ति॒ । आज्य᳚म् । इति॑ । उ॒क्थम् । तत् । वै । ग्रहा॑णाम् । नि॒दान॒मिति॑ नि - दान᳚म् । यत् । उ॒पाꣳ॒॒श्वित्यु॑प - अꣳ॒॒शु । शꣳस॑ति । तत् ।  \newline


\textbf{Krama Paata} \newline

प्र॒यु॒ज्यन्ते॒ तानि॑ । प्र॒यु॒ज्यन्त॒ इति॑ प्र - यु॒ज्यन्ते᳚ । तान्यनु॑ । अन्वोष॑धयः । ओष॑धयः॒ पुनः॑ । पुन॒रा । आ भ॑वन्ति । भ॒व॒न्ति॒ प्र । प्रान्यानि॑ । अ॒न्यानि॒ पात्रा॑णि । पात्रा॑णि यु॒ज्यन्ते᳚ । यु॒ज्यन्ते॒ न । नान्यानि॑ । अ॒न्यानि॒ यानि॑ । यानि॑ परा॒चीना॑नि । प॒रा॒चीना॑नि प्रयु॒ज्यन्ते᳚ । प्र॒यु॒ज्यन्ते॒ तानि॑ । प्र॒यु॒ज्यन्त॒ इति॑ प्र - यु॒ज्यन्ते᳚ । तान्यनु॑ । अन्वा॑र॒ण्याः । आ॒र॒ण्याः प॒शवः॑ । प॒शवोऽर॑ण्यम् । अर॑ण्य॒मप॑ । अप॑ यन्ति । य॒न्ति॒ यानि॑ । यानि॒ पुनः॑ । पुनः॑ प्रयु॒ज्यन्ते᳚ । प्र॒यु॒ज्यन्ते॒ तानि॑ । प्र॒यु॒ज्यन्त॒ इति॑ प्र - यु॒ज्यन्ते᳚ । तान्यनु॑ । अनु॑ ग्रा॒म्याः । ग्रा॒म्याः प॒शवः॑ । प॒शवो॒ ग्राम᳚म् । ग्राम॑मु॒पाव॑यन्ति । उ॒पाव॑यन्ति॒ यः । उ॒पाव॑य॒न्तीत्यु॑प - अव॑यन्ति । यो वै । वै ग्रहा॑णाम् । ग्रहा॑णाम् नि॒दान᳚म् । नि॒दान॒म् ॅवेद॑ । नि॒दान॒मिति॑ नि - दान᳚म् । वेद॑ नि॒दान॑वान् । नि॒दान॑वान् भवति । नि॒दान॑वा॒निति॑ नि॒दान॑ - वा॒न्॒ । भ॒व॒त्याज्य᳚म् । आज्य॒मिति॑ । इत्यु॒क्थम् । उ॒क्थम् तत् । तद् वै । वै ग्रहा॑णाम् । ग्रहा॑णाम् नि॒दान᳚म् । नि॒दान॒म् ॅयत् । नि॒दान॒मिति॑ नि - दान᳚म् । यदु॑पाꣳ॒॒शु । उ॒पाꣳ॒॒शु शꣳस॑ति । उ॒पाꣳ॒॒श्वित्यु॑प - अꣳ॒॒शु । शꣳस॑ति॒ तत् । तदु॑पाꣳश्वन्तर्या॒मयोः᳚ \newline

\textbf{Jatai Paata} \newline

1. प्र॒यु॒ज्यन्ते॒ तानि॒ तानि॑ प्रयु॒ज्यन्ते᳚ प्रयु॒ज्यन्ते॒ तानि॑ । \newline
2. प्र॒यु॒ज्यन्त॒ इति॑ प्र - यु॒ज्यन्ते᳚ । \newline
3. तान्य न्वनु॒ तानि॒ तान्यनु॑ । \newline
4. अन्वोष॑धय॒ ओष॑ध॒यो ऽन्वन्वोष॑धयः । \newline
5. ओष॑धयः॒ पुनः॒ पुन॒ रोष॑धय॒ ओष॑धयः॒ पुनः॑ । \newline
6. पुन॒ रा पुनः॒ पुन॒ रा । \newline
7. आ भ॑वन्ति भव॒न्त्या भ॑वन्ति । \newline
8. भ॒व॒न्ति॒ प्र प्र भ॑वन्ति भवन्ति॒ प्र । \newline
9. प्रा न्यान्य॒ न्यानि॒ प्र प्रा न्यानि॑ । \newline
10. अ॒न्यानि॒ पात्रा॑णि॒ पात्रा᳚ ण्य॒न्या न्य॒न्यानि॒ पात्रा॑णि । \newline
11. पात्रा॑णि यु॒ज्यन्ते॑ यु॒ज्यन्ते॒ पात्रा॑णि॒ पात्रा॑णि यु॒ज्यन्ते᳚ । \newline
12. यु॒ज्यन्ते॒ न न यु॒ज्यन्ते॑ यु॒ज्यन्ते॒ न । \newline
13. ना न्यान्य॒ न्यानि॒ न नान्यानि॑ । \newline
14. अ॒न्यानि॒ यानि॒ या न्य॒न्या न्य॒न्यानि॒ यानि॑ । \newline
15. यानि॑ परा॒चीना॑नि परा॒चीना॑नि॒ यानि॒ यानि॑ परा॒चीना॑नि । \newline
16. प॒रा॒चीना॑नि प्रयु॒ज्यन्ते᳚ प्रयु॒ज्यन्ते॑ परा॒चीना॑नि परा॒चीना॑नि प्रयु॒ज्यन्ते᳚ । \newline
17. प्र॒यु॒ज्यन्ते॒ तानि॒ तानि॑ प्रयु॒ज्यन्ते᳚ प्रयु॒ज्यन्ते॒ तानि॑ । \newline
18. प्र॒यु॒ज्यन्त॒ इति॑ प्र - यु॒ज्यन्ते᳚ । \newline
19. तान्यन् वनु॒ तानि॒ तान्यनु॑ । \newline
20. अन्वा॑ र॒ण्या आ॑र॒ण्या अन् वन् वा॑र॒ण्याः । \newline
21. आ॒र॒ण्याः प॒शवः॑ प॒शव॑ आर॒ण्या आ॑र॒ण्याः प॒शवः॑ । \newline
22. प॒शवो ऽर॑ण्य॒ मर॑ण्यम् प॒शवः॑ प॒शवो ऽर॑ण्यम् । \newline
23. अर॑ण्य॒ मपापा र॑ण्य॒ मर॑ण्य॒ मप॑ । \newline
24. अप॑ यन्ति य॒न्त्यपाप॑ यन्ति । \newline
25. य॒न्ति॒ यानि॒ यानि॑ यन्ति यन्ति॒ यानि॑ । \newline
26. यानि॒ पुनः॒ पुन॒र् यानि॒ यानि॒ पुनः॑ । \newline
27. पुनः॑ प्रयु॒ज्यन्ते᳚ प्रयु॒ज्यन्ते॒ पुनः॒ पुनः॑ प्रयु॒ज्यन्ते᳚ । \newline
28. प्र॒यु॒ज्यन्ते॒ तानि॒ तानि॑ प्रयु॒ज्यन्ते᳚ प्रयु॒ज्यन्ते॒ तानि॑ । \newline
29. प्र॒यु॒ज्यन्त॒ इति॑ प्र - यु॒ज्यन्ते᳚ । \newline
30. तान्यन् वनु॒ तानि॒ तान्यनु॑ । \newline
31. अनु॑ ग्रा॒म्या ग्रा॒म्या अन्वनु॑ ग्रा॒म्याः । \newline
32. ग्रा॒म्याः प॒शवः॑ प॒शवो᳚ ग्रा॒म्या ग्रा॒म्याः प॒शवः॑ । \newline
33. प॒शवो॒ ग्राम॒म् ग्राम॑म् प॒शवः॑ प॒शवो॒ ग्राम᳚म् । \newline
34. ग्राम॑ मु॒पाव॑य न्त्यु॒पाव॑यन्ति॒ ग्राम॒म् ग्राम॑ मु॒पाव॑यन्ति । \newline
35. उ॒पाव॑यन्ति॒ यो य उ॒पाव॑य न्त्यु॒पाव॑यन्ति॒ यः । \newline
36. उ॒पाव॑य॒न्तीत्यु॑प - अव॑यन्ति । \newline
37. यो वै वै यो यो वै । \newline
38. वै ग्रहा॑णा॒म् ग्रहा॑णां॒ ॅवै वै ग्रहा॑णाम् । \newline
39. ग्रहा॑णाम् नि॒दान॑म् नि॒दान॒म् ग्रहा॑णा॒म् ग्रहा॑णाम् नि॒दान᳚म् । \newline
40. नि॒दानं॒ ॅवेद॒ वेद॑ नि॒दान॑म् नि॒दानं॒ ॅवेद॑ । \newline
41. नि॒दान॒मिति॑ नि - दान᳚म् । \newline
42. वेद॑ नि॒दान॑वान् नि॒दान॑वा॒न्॒. वेद॒ वेद॑ नि॒दान॑वान् । \newline
43. नि॒दान॑वान् भवति भवति नि॒दान॑वान् नि॒दान॑वान् भवति । \newline
44. नि॒दान॑वा॒निति॑ नि॒दान॑ - वा॒न् । \newline
45. भ॒व॒ त्याज्य॒ माज्य॑म् भवति भव॒ त्याज्य᳚म् । \newline
46. आज्य॒ मितीत्याज्य॒ माज्य॒ मिति॑ । \newline
47. इत्यु॒क्थ मु॒क्थ मिती त्यु॒क्थम् । \newline
48. उ॒क्थम् तत् तदु॒क्थ मु॒क्थम् तत् । \newline
49. तद् वै वै तत् तद् वै । \newline
50. वै ग्रहा॑णा॒म् ग्रहा॑णां॒ ॅवै वै ग्रहा॑णाम् । \newline
51. ग्रहा॑णाम् नि॒दान॑म् नि॒दान॒म् ग्रहा॑णा॒म् ग्रहा॑णाम् नि॒दान᳚म् । \newline
52. नि॒दानं॒ ॅयद् यन् नि॒दान॑म् नि॒दानं॒ ॅयत् । \newline
53. नि॒दान॒मिति॑ नि - दान᳚म् । \newline
54. यदु॑पाꣳ॒॒शू॑ पाꣳ॒॒शु यद् यदु॑पाꣳ॒॒शु । \newline
55. उ॒पाꣳ॒॒शु शꣳस॑ति॒ शꣳस॑ त्युपाꣳ॒॒शू॑ पाꣳ॒॒शु शꣳस॑ति । \newline
56. उ॒पाꣳ॒॒श्वित्यु॑प - अꣳ॒॒शु । \newline
57. शꣳस॑ति॒ तत् तच् छꣳस॑ति॒ शꣳस॑ति॒ तत् । \newline
58. तदु॑पाꣳश्वन्तर्या॒मयो॑ रुपाꣳश्वन्तर्या॒मयो॒ स्तत् तदु॑पाꣳश्वन्तर्या॒मयोः᳚ । \newline

\textbf{Ghana Paata } \newline

1. प्र॒यु॒ज्यन्ते॒ तानि॒ तानि॑ प्रयु॒ज्यन्ते᳚ प्रयु॒ज्यन्ते॒ तान्यन् वनु॒ तानि॑ प्रयु॒ज्यन्ते᳚ प्रयु॒ज्यन्ते॒ तान्यनु॑ । \newline
2. प्र॒यु॒ज्यन्त॒ इति॑ प्र - यु॒ज्यन्ते᳚ । \newline
3. तान्यन् वनु॒ तानि॒ तान्यन्वो ष॑धय॒ ओष॑ध॒यो ऽनु॒ तानि॒ तान्यन् वोष॑धयः । \newline
4. अन्वोष॑धय॒ ओष॑ध॒यो ऽन्वन् वोष॑धयः॒ पुनः॒ पुन॒ रोष॑ध॒यो ऽन्वन् वोष॑धयः॒ पुनः॑ । \newline
5. ओष॑धयः॒ पुनः॒ पुन॒ रोष॑धय॒ ओष॑धयः॒ पुन॒ रा पुन॒ रोष॑धय॒ ओष॑धयः॒ पुन॒ रा । \newline
6. पुन॒ रा पुनः॒ पुन॒ रा भ॑वन्ति भव॒न्त्या पुनः॒ पुन॒ रा भ॑वन्ति । \newline
7. आ भ॑वन्ति भव॒न्त्या भ॑वन्ति॒ प्र प्र भ॑व॒न्त्या भ॑वन्ति॒ प्र । \newline
8. भ॒व॒न्ति॒ प्र प्र भ॑वन्ति भवन्ति॒ प्रान्या न्य॒न्यानि॒ प्र भ॑वन्ति भवन्ति॒ प्रान्यानि॑ । \newline
9. प्रान्या न्य॒न्यानि॒ प्र प्रान्यानि॒ पात्रा॑णि॒ पात्रा᳚ ण्य॒न्यानि॒ प्र प्रान्यानि॒ पात्रा॑णि । \newline
10. अ॒न्यानि॒ पात्रा॑णि॒ पात्रा᳚ ण्य॒न्या न्य॒न्यानि॒ पात्रा॑णि यु॒ज्यन्ते॑ यु॒ज्यन्ते॒ पात्रा᳚ ण्य॒न्या न्य॒न्यानि॒ पात्रा॑णि यु॒ज्यन्ते᳚ । \newline
11. पात्रा॑णि यु॒ज्यन्ते॑ यु॒ज्यन्ते॒ पात्रा॑णि॒ पात्रा॑णि यु॒ज्यन्ते॒ न न यु॒ज्यन्ते॒ पात्रा॑णि॒ पात्रा॑णि यु॒ज्यन्ते॒ न । \newline
12. यु॒ज्यन्ते॒ न न यु॒ज्यन्ते॑ यु॒ज्यन्ते॒ नान्या न्य॒न्यानि॒ न यु॒ज्यन्ते॑ यु॒ज्यन्ते॒ नान्यानि॑ । \newline
13. नान्या न्य॒न्यानि॒ न नान्यानि॒ यानि॒ यान्य॒ न्यानि॒ न नान्यानि॒ यानि॑ । \newline
14. अ॒न्यानि॒ यानि॒ यान्य॒न्या न्य॒न्यानि॒ यानि॑ परा॒चीना॑नि परा॒चीना॑नि॒ यान्य॒न्या न्य॒न्यानि॒ यानि॑ परा॒चीना॑नि । \newline
15. यानि॑ परा॒चीना॑नि परा॒चीना॑नि॒ यानि॒ यानि॑ परा॒चीना॑नि प्रयु॒ज्यन्ते᳚ प्रयु॒ज्यन्ते॑ परा॒चीना॑नि॒ यानि॒ यानि॑ परा॒चीना॑नि प्रयु॒ज्यन्ते᳚ । \newline
16. प॒रा॒चीना॑नि प्रयु॒ज्यन्ते᳚ प्रयु॒ज्यन्ते॑ परा॒चीना॑नि परा॒चीना॑नि प्रयु॒ज्यन्ते॒ तानि॒ तानि॑ प्रयु॒ज्यन्ते॑ परा॒चीना॑नि परा॒चीना॑नि प्रयु॒ज्यन्ते॒ तानि॑ । \newline
17. प्र॒यु॒ज्यन्ते॒ तानि॒ तानि॑ प्रयु॒ज्यन्ते᳚ प्रयु॒ज्यन्ते॒ तान्यन् वनु॒ तानि॑ प्रयु॒ज्यन्ते᳚ प्रयु॒ज्यन्ते॒ तान्यनु॑ । \newline
18. प्र॒यु॒ज्यन्त॒ इति॑ प्र - यु॒ज्यन्ते᳚ । \newline
19. तान्यन् वनु॒ तानि॒ तान्यन् वा॑र॒ण्या आ॑र॒ण्या अनु॒ तानि॒ तान्यन् वा॑र॒ण्याः । \newline
20. अन्वा॑र॒ण्या आ॑र॒ण्या अन्वन् वा॑र॒ण्याः प॒शवः॑ प॒शव॑ आर॒ण्या अन्वन् वा॑र॒ण्याः प॒शवः॑ । \newline
21. आ॒र॒ण्याः प॒शवः॑ प॒शव॑ आर॒ण्या आ॑र॒ण्याः प॒शवो ऽर॑ण्य॒ मर॑ण्यम् प॒शव॑ आर॒ण्या आ॑र॒ण्याः प॒शवो ऽर॑ण्यम् । \newline
22. प॒शवो ऽर॑ण्य॒ मर॑ण्यम् प॒शवः॑ प॒शवो ऽर॑ण्य॒ मपापा र॑ण्यम् प॒शवः॑ प॒शवो ऽर॑ण्य॒ मप॑ । \newline
23. अर॑ण्य॒ मपापा र॑ण्य॒ मर॑ण्य॒ मप॑ यन्ति य॒न्त्य पार॑ण्य॒ मर॑ण्य॒ मप॑ यन्ति । \newline
24. अप॑ यन्ति य॒न्त्य पाप॑ यन्ति॒ यानि॒ यानि॑ य॒न्त्य पाप॑ यन्ति॒ यानि॑ । \newline
25. य॒न्ति॒ यानि॒ यानि॑ यन्ति यन्ति॒ यानि॒ पुनः॒ पुन॒र् यानि॑ यन्ति यन्ति॒ यानि॒ पुनः॑ । \newline
26. यानि॒ पुनः॒ पुन॒र् यानि॒ यानि॒ पुनः॑ प्रयु॒ज्यन्ते᳚ प्रयु॒ज्यन्ते॒ पुन॒र् यानि॒ यानि॒ पुनः॑ प्रयु॒ज्यन्ते᳚ । \newline
27. पुनः॑ प्रयु॒ज्यन्ते᳚ प्रयु॒ज्यन्ते॒ पुनः॒ पुनः॑ प्रयु॒ज्यन्ते॒ तानि॒ तानि॑ प्रयु॒ज्यन्ते॒ पुनः॒ पुनः॑ प्रयु॒ज्यन्ते॒ तानि॑ । \newline
28. प्र॒यु॒ज्यन्ते॒ तानि॒ तानि॑ प्रयु॒ज्यन्ते᳚ प्रयु॒ज्यन्ते॒ तान्यन् वनु॒ तानि॑ प्रयु॒ज्यन्ते᳚ प्रयु॒ज्यन्ते॒ तान्यनु॑ । \newline
29. प्र॒यु॒ज्यन्त॒ इति॑ प्र - यु॒ज्यन्ते᳚ । \newline
30. तान्यन् वनु॒ तानि॒ तान्यनु॑ ग्रा॒म्या ग्रा॒म्या अनु॒ तानि॒ तान्यनु॑ ग्रा॒म्याः । \newline
31. अनु॑ ग्रा॒म्या ग्रा॒म्या अन्वनु॑ ग्रा॒म्याः प॒शवः॑ प॒शवो᳚ ग्रा॒म्या अन्वनु॑ ग्रा॒म्याः प॒शवः॑ । \newline
32. ग्रा॒म्याः प॒शवः॑ प॒शवो᳚ ग्रा॒म्या ग्रा॒म्याः प॒शवो॒ ग्राम॒म् ग्राम॑म् प॒शवो᳚ ग्रा॒म्या ग्रा॒म्याः प॒शवो॒ ग्राम᳚म् । \newline
33. प॒शवो॒ ग्राम॒म् ग्राम॑म् प॒शवः॑ प॒शवो॒ ग्राम॑ मु॒पाव॑य न्त्यु॒पाव॑यन्ति॒ ग्राम॑म् प॒शवः॑ प॒शवो॒ ग्राम॑ मु॒पाव॑यन्ति । \newline
34. ग्राम॑ मु॒पाव॑य न्त्यु॒पाव॑यन्ति॒ ग्राम॒म् ग्राम॑ मु॒पाव॑यन्ति॒ यो य उ॒पाव॑यन्ति॒ ग्राम॒म् ग्राम॑ मु॒पाव॑यन्ति॒ यः । \newline
35. उ॒पाव॑यन्ति॒ यो य उ॒पाव॑य न्त्यु॒पाव॑यन्ति॒ यो वै वै य उ॒पाव॑य न्त्यु॒पाव॑यन्ति॒ यो वै । \newline
36. उ॒पाव॑य॒न्तीत्यु॑प - अव॑यन्ति । \newline
37. यो वै वै यो यो वै ग्रहा॑णा॒म् ग्रहा॑णां॒ ॅवै यो यो वै ग्रहा॑णाम् । \newline
38. वै ग्रहा॑णा॒म् ग्रहा॑णां॒ ॅवै वै ग्रहा॑णाम् नि॒दान॑म् नि॒दान॒म् ग्रहा॑णां॒ ॅवै वै ग्रहा॑णाम् नि॒दान᳚म् । \newline
39. ग्रहा॑णाम् नि॒दान॑म् नि॒दान॒म् ग्रहा॑णा॒म् ग्रहा॑णाम् नि॒दानं॒ ॅवेद॒ वेद॑ नि॒दान॒म् ग्रहा॑णा॒म् ग्रहा॑णाम् नि॒दानं॒ ॅवेद॑ । \newline
40. नि॒दानं॒ ॅवेद॒ वेद॑ नि॒दान॑म् नि॒दानं॒ ॅवेद॑ नि॒दान॑वान् नि॒दान॑वा॒न्॒. वेद॑ नि॒दान॑म् नि॒दानं॒ ॅवेद॑ नि॒दान॑वान् । \newline
41. नि॒दान॒मिति॑ नि - दान᳚म् । \newline
42. वेद॑ नि॒दान॑वान् नि॒दान॑वा॒न्॒. वेद॒ वेद॑ नि॒दान॑वान् भवति भवति नि॒दान॑वा॒न्॒. वेद॒ वेद॑ नि॒दान॑वान् भवति । \newline
43. नि॒दान॑वान् भवति भवति नि॒दान॑वान् नि॒दान॑वान् भव॒त्याज्य॒ माज्य॑म् भवति नि॒दान॑वान् नि॒दान॑वान् भव॒ त्याज्य᳚म् । \newline
44. नि॒दान॑वा॒निति॑ नि॒दान॑ - वा॒न् । \newline
45. भ॒व॒ त्याज्य॒ माज्य॑म् भवति भव॒ त्याज्य॒ मिती त्याज्य॑म् भवति भव॒ त्याज्य॒ मिति॑ । \newline
46. आज्य॒ मिती त्याज्य॒ माज्य॒ मित्यु॒क्थ मु॒क्थ मित्याज्य॒ माज्य॒ मित्यु॒क्थम् । \newline
47. इत्यु॒क्थ मु॒क्थ मिती त्यु॒क्थम् तत् तदु॒क्थ मिती त्यु॒क्थम् तत् । \newline
48. उ॒क्थम् तत् तदु॒क्थ मु॒क्थम् तद् वै वै तदु॒क्थ मु॒क्थम् तद् वै । \newline
49. तद् वै वै तत् तद् वै ग्रहा॑णा॒म् ग्रहा॑णां॒ ॅवै तत् तद् वै ग्रहा॑णाम् । \newline
50. वै ग्रहा॑णा॒म् ग्रहा॑णां॒ ॅवै वै ग्रहा॑णाम् नि॒दान॑म् नि॒दान॒म् ग्रहा॑णां॒ ॅवै वै ग्रहा॑णाम् नि॒दान᳚म् । \newline
51. ग्रहा॑णाम् नि॒दान॑म् नि॒दान॒म् ग्रहा॑णा॒म् ग्रहा॑णाम् नि॒दानं॒ ॅयद् यन् नि॒दान॒म् ग्रहा॑णा॒म् ग्रहा॑णाम् नि॒दानं॒ ॅयत् । \newline
52. नि॒दानं॒ ॅयद् यन् नि॒दान॑म् नि॒दानं॒ ॅयदु॑पाꣳ॒॒शू॑ पाꣳ॒॒शु यन् नि॒दान॑म् नि॒दानं॒ ॅयदु॑पाꣳ॒॒शु । \newline
53. नि॒दान॒मिति॑ नि - दान᳚म् । \newline
54. यदु॑पाꣳ॒॒शू॑ पाꣳ॒॒शु यद् यदु॑पाꣳ॒॒शु शꣳस॑ति॒ शꣳस॑ त्युपाꣳ॒॒शु यद् यदु॑पाꣳ॒॒शु शꣳस॑ति । \newline
55. उ॒पाꣳ॒॒शु शꣳस॑ति॒ शꣳस॑ त्युपाꣳ॒॒शू॑ पाꣳ॒॒शु शꣳस॑ति॒ तत् तच्छꣳस॑ त्युपाꣳ॒॒शू॑ पाꣳ॒॒शु शꣳस॑ति॒ तत् । \newline
56. उ॒पाꣳ॒॒श्वित्यु॑प - अꣳ॒॒शु । \newline
57. शꣳस॑ति॒ तत् तच्छꣳस॑ति॒ शꣳस॑ति॒ तदु॑पाꣳश्वन्तर्या॒मयो॑ रुपाꣳश्वन्तर्या॒मयो॒ स्तच्छꣳस॑ति॒ शꣳस॑ति॒ तदु॑पाꣳश्वन्तर्या॒मयोः᳚ । \newline
58. तदु॑पाꣳश्वन्तर्या॒मयो॑ रुपाꣳश्वन्तर्या॒मयो॒ स्तत् तदु॑पाꣳश्वन्तर्या॒मयो॒र् यद् यदु॑पाꣳश्वन्तर्या॒मयो॒ स्तत् तदु॑पाꣳश्वन्तर्या॒मयो॒र् यत् । \newline
\pagebreak
\markright{ TS 6.5.11.3  \hfill https://www.vedavms.in \hfill}

\section{ TS 6.5.11.3 }

\textbf{TS 6.5.11.3 } \newline
\textbf{Samhita Paata} \newline

-दु॑पाꣳश्वन्तर्या॒मयो॒-र्यदु॒च्चै-स्तदित॑रेषां॒ ग्रहा॑णामे॒तद्वै ग्रहा॑णां नि॒दानं॒ ॅय ए॒वं ॅवेद॑ नि॒दान॑वान् भवति॒ यो वै ग्रहा॑णां मिथु॒नं ॅवेद॒ प्र प्र॒जया॑ प॒शुभि॑र्मिथु॒नैर्जा॑यते स्था॒लीभि॑र॒न्ये ग्रहा॑ गृ॒ह्यन्ते॑ वाय॒व्यै॑र॒न्य ए॒तद्वै ग्रहा॑णां मिथु॒नं ॅय ए॒वं ॅवेद॒ प्र प्र॒जया॑ प॒शुभि॑र्मिथु॒नैर्जा॑यत॒ इन्द्र॒स्त्वष्टुः॒ सोम॑मभी॒षहा॑ऽपिब॒थ् स विष्व॒ङ्- [  ] \newline

\textbf{Pada Paata} \newline

उ॒पाꣳ॒॒श्व॒न्त॒र्या॒मयो॒रित्यु॑पाꣳशु - अ॒न्त॒र्या॒मयोः᳚ । यत् । उ॒च्चैः । तत् । इत॑रेषाम् । ग्रहा॑णाम् । ए॒तत् । वै । ग्रहा॑णाम् । नि॒दान॒मिति॑ नि - दान᳚म् । यः । ए॒वम् । वेद॑ । नि॒दान॑वा॒निति॑ नि॒दान॑ - वा॒न् । भ॒व॒ति॒ । यः । वै । ग्रहा॑णाम् । मि॒थु॒नम् । वेद॑ । प्रेति॑ । प्र॒जयेति॑ प्र - जया᳚ । प॒शुभि॒रिति॑ प॒शु - भिः॒ । मि॒थु॒नैः । जा॒य॒ते॒ । स्था॒लीभिः॑ । अ॒न्ये । ग्रहाः᳚ । गृ॒ह्यन्ते᳚ । वा॒य॒व्यैः᳚ । अ॒न्ये । ए॒तत् । वै । ग्रहा॑णाम् । मि॒थु॒नम् । यः । ए॒वम् । वेद॑ । प्रेति॑ । प्र॒जयेति॑ प्र - जया᳚ । प॒शुभि॒रिति॑ प॒शु - भिः॒ । मि॒थु॒नैः । जा॒य॒ते॒ । इन्द्रः॑ । त्वष्टुः॑ । सोम᳚म् । अ॒भी॒षहेत्य॑भि - सहा᳚ । अ॒पि॒ब॒त् । सः । विष्वङ्॑ ।  \newline


\textbf{Krama Paata} \newline

उ॒पाꣳ॒॒श्व॒न्त॒र्या॒मयो॒र् यत् । उ॒पाꣳ॒॒श्व॒न्त॒र्या॒मयो॒रित्यु॑पाꣳशु - अ॒न्त॒र्या॒मयोः᳚ । यदु॒च्चैः । उ॒च्चैस्तत् । तदित॑रेषाम् । इत॑रेषा॒म् ग्रहा॑णाम् । ग्रहा॑णामे॒तत् । ए॒तद् वै । वै ग्रहा॑णाम् । ग्रहा॑णाम् नि॒दान᳚म् । नि॒दान॒म् ॅयः । नि॒दान॒मिति॑ नि - दान᳚म् । य ए॒वम् । ए॒वम् ॅवेद॑ । वेद॑ नि॒दान॑वान् । नि॒दान॑वान् भवति । नि॒दान॑वा॒निति॑ नि॒दान॑ - वा॒न्॒ । भ॒व॒ति॒ यः । यो वै । वै ग्रहा॑णाम् । ग्रहा॑णाम् मिथु॒नम् । मि॒थु॒नम् ॅवेद॑ । वेद॒ प्र । प्र प्र॒जया᳚ । प्र॒जया॑ प॒शुभिः॑ । प्र॒जयेति॑ प्र - जया᳚ । प॒शुभि॑र् मिथु॒नैः । प॒शुभि॒रिति॑ प॒शु - भिः॒ । मि॒थु॒नैर् जा॑यते । जा॒य॒ते॒ स्था॒लीभिः॑ । स्था॒लीभि॑र॒न्ये । अ॒न्ये ग्रहाः᳚ । ग्रहा॑ गृ॒ह्यन्ते᳚ । गृ॒ह्यन्ते॑ वाय॒व्यैः᳚ । वा॒य॒व्यै॑र॒न्ये । अ॒न्य ए॒तत् । ए॒तद् वै । वै ग्रहा॑णाम् । ग्रहा॑णाम् मिथु॒नम् । मि॒थु॒नम् ॅयः । य ए॒वम् । ए॒वम् ॅवेद॑ । वेद॒ प्र । प्र प्र॒जया᳚ । प्र॒जया॑ प॒शुभिः॑ । प्र॒जयेति॑ प्र - जया᳚ । प॒शुभि॑र् मिथु॒नैः । प॒शुभि॒रिति॑ प॒शु - भिः॒ । मि॒थु॒नैर्जा॑यते । जा॒य॒त॒ इन्द्रः॑ । इन्द्र॒स्त्वष्टुः॑ । त्वष्टुः॒ सोम᳚म् । सोम॑मभी॒षहा᳚ । अ॒भी॒षहा॑ऽपिबत् । अ॒भी॒षहेत्य॑भि - सहा᳚ । अ॒पि॒ब॒थ् सः । स विष्वङ्‍ङ्॑ । विष्व॒ङ्॒. वि \newline

\textbf{Jatai Paata} \newline

1. उ॒पाꣳ॒॒श्व॒न्त॒र्या॒मयो॒र् यद् यदु॑पाꣳश्वन्तर्या॒मयो॑ रुपाꣳश्वन्तर्या॒मयो॒र् यत् । \newline
2. उ॒पाꣳ॒॒श्व॒न्त॒र्या॒मयो॒रित्यु॑पाꣳशु - अ॒न्त॒र्या॒मयोः᳚ । \newline
3. यदु॒च्चै रु॒च्चैर् यद् यदु॒च्चैः । \newline
4. उ॒च्चै स्तत् तदु॒च्चै रु॒च्चै स्तत् । \newline
5. तदित॑रेषा॒ मित॑ रेषा॒म् तत् तदित॑रेषाम् । \newline
6. इत॑रेषा॒म् ग्रहा॑णा॒म् ग्रहा॑णा॒ मित॑रेषा॒ मित॑रेषा॒म् ग्रहा॑णाम् । \newline
7. ग्रहा॑णा मे॒त दे॒तद् ग्रहा॑णा॒म् ग्रहा॑णा मे॒तत् । \newline
8. ए॒तद् वै वा ए॒त दे॒तद् वै । \newline
9. वै ग्रहा॑णा॒म् ग्रहा॑णां॒ ॅवै वै ग्रहा॑णाम् । \newline
10. ग्रहा॑णाम् नि॒दान॑म् नि॒दान॒म् ग्रहा॑णा॒म् ग्रहा॑णाम् नि॒दान᳚म् । \newline
11. नि॒दानं॒ ॅयो यो नि॒दान॑म् नि॒दानं॒ ॅयः । \newline
12. नि॒दान॒मिति॑ नि - दान᳚म् । \newline
13. य ए॒व मे॒वं ॅयो य ए॒वम् । \newline
14. ए॒वं ॅवेद॒ वेदै॒व मे॒वं ॅवेद॑ । \newline
15. वेद॑ नि॒दान॑वान् नि॒दान॑वा॒न्॒. वेद॒ वेद॑ नि॒दान॑वान् । \newline
16. नि॒दान॑वान् भवति भवति नि॒दान॑वान् नि॒दान॑वान् भवति । \newline
17. नि॒दान॑वा॒निति॑ नि॒दान॑ - वा॒न् । \newline
18. भ॒व॒ति॒ यो यो भ॑वति भवति॒ यः । \newline
19. यो वै वै यो यो वै । \newline
20. वै ग्रहा॑णा॒म् ग्रहा॑णां॒ ॅवै वै ग्रहा॑णाम् । \newline
21. ग्रहा॑णाम् मिथु॒नम् मि॑थु॒नम् ग्रहा॑णा॒म् ग्रहा॑णाम् मिथु॒नम् । \newline
22. मि॒थु॒नं ॅवेद॒ वेद॑ मिथु॒नम् मि॑थु॒नं ॅवेद॑ । \newline
23. वेद॒ प्र प्र वेद॒ वेद॒ प्र । \newline
24. प्र प्र॒जया᳚ प्र॒जया॒ प्र प्र प्र॒जया᳚ । \newline
25. प्र॒जया॑ प॒शुभिः॑ प॒शुभिः॑ प्र॒जया᳚ प्र॒जया॑ प॒शुभिः॑ । \newline
26. प्र॒जयेति॑ प्र - जया᳚ । \newline
27. प॒शुभि॑र् मिथु॒नैर् मि॑थु॒नैः प॒शुभिः॑ प॒शुभि॑र् मिथु॒नैः । \newline
28. प॒शुभि॒रिति॑ प॒शु - भिः॒ । \newline
29. मि॒थु॒नैर् जा॑यते जायते मिथु॒नैर् मि॑थु॒नैर् जा॑यते । \newline
30. जा॒य॒ते॒ स्था॒लीभिः॑ स्था॒लीभि॑र् जायते जायते स्था॒लीभिः॑ । \newline
31. स्था॒लीभि॑ र॒न्ये᳚ ऽन्ये स्था॒लीभिः॑ स्था॒लीभि॑ र॒न्ये । \newline
32. अ॒न्ये ग्रहा॒ ग्रहा॑ अ॒न्ये᳚ ऽन्ये ग्रहाः᳚ । \newline
33. ग्रहा॑ गृ॒ह्यन्ते॑ गृ॒ह्यन्ते॒ ग्रहा॒ ग्रहा॑ गृ॒ह्यन्ते᳚ । \newline
34. गृ॒ह्यन्ते॑ वाय॒व्यै᳚र् वाय॒व्यै᳚र् गृ॒ह्यन्ते॑ गृ॒ह्यन्ते॑ वाय॒व्यैः᳚ । \newline
35. वा॒य॒व्यै॑ र॒न्ये᳚ ऽन्ये वा॑य॒व्यै᳚र् वाय॒व्यै॑ र॒न्ये । \newline
36. अ॒न्य ए॒त दे॒त द॒न्ये᳚ ऽन्य ए॒तत् । \newline
37. ए॒तद् वै वा ए॒त दे॒तद् वै । \newline
38. वै ग्रहा॑णा॒म् ग्रहा॑णां॒ ॅवै वै ग्रहा॑णाम् । \newline
39. ग्रहा॑णाम् मिथु॒नम् मि॑थु॒नम् ग्रहा॑णा॒म् ग्रहा॑णाम् मिथु॒नम् । \newline
40. मि॒थु॒नं ॅयो यो मि॑थु॒नम् मि॑थु॒नं ॅयः । \newline
41. य ए॒व मे॒वं ॅयो य ए॒वम् । \newline
42. ए॒वं ॅवेद॒ वेदै॒व मे॒वं ॅवेद॑ । \newline
43. वेद॒ प्र प्र वेद॒ वेद॒ प्र । \newline
44. प्र प्र॒जया᳚ प्र॒जया॒ प्र प्र प्र॒जया᳚ । \newline
45. प्र॒जया॑ प॒शुभिः॑ प॒शुभिः॑ प्र॒जया᳚ प्र॒जया॑ प॒शुभिः॑ । \newline
46. प्र॒जयेति॑ प्र - जया᳚ । \newline
47. प॒शुभि॑र् मिथु॒नैर् मि॑थु॒नैः प॒शुभिः॑ प॒शुभि॑र् मिथु॒नैः । \newline
48. प॒शुभि॒रिति॑ प॒शु - भिः॒ । \newline
49. मि॒थु॒नैर् जा॑यते जायते मिथु॒नैर् मि॑थु॒नैर् जा॑यते । \newline
50. जा॒य॒त॒ इन्द्र॒ इन्द्रो॑ जायते जायत॒ इन्द्रः॑ । \newline
51. इन्द्र॒ स्त्वष्टु॒ स्त्वष्टु॒ रिन्द्र॒ इन्द्र॒ स्त्वष्टुः॑ । \newline
52. त्वष्टुः॒ सोमꣳ॒॒ सोम॒म् त्वष्टु॒ स्त्वष्टुः॒ सोम᳚म् । \newline
53. सोम॑ मभी॒षहा॑ ऽभी॒षहा॒ सोमꣳ॒॒ सोम॑ मभी॒षहा᳚ । \newline
54. अ॒भी॒षहा॑ ऽपिब दपिब दभी॒षहा॑ ऽभी॒षहा॑ ऽपिबत् । \newline
55. अ॒भी॒षहेत्य॑भि - सहा᳚ । \newline
56. अ॒पि॒ब॒थ् स सो॑ ऽपिब दपिब॒थ् सः । \newline
57. स विष्व॒ङ्॒. विष्व॒ङ् ख्स स विष्वङ्॑ । \newline
58. विष्व॒ङ्॒. वि वि विष्व॒ङ्॒. विष्व॒ङ्॒. वि । \newline

\textbf{Ghana Paata } \newline

1. उ॒पाꣳ॒॒श्व॒न्त॒र्या॒मयो॒र् यद् यदु॑पाꣳश्वन्तर्या॒मयो॑ रुपाꣳश्वन्तर्या॒मयो॒र् यदु॒च्चै रु॒च्चैर् यदु॑पाꣳश्वन्तर्या॒मयो॑ रुपाꣳश्वन्तर्या॒मयो॒र् यदु॒च्चैः । \newline
2. उ॒पाꣳ॒॒श्व॒न्त॒र्या॒मयो॒रित्यु॑पाꣳशु - अ॒न्त॒र्या॒मयोः᳚ । \newline
3. यदु॒च्चै रु॒च्चैर् यद् यदु॒च्चै स्तत् तदु॒च्चैर् यद् यदु॒च्चै स्तत् । \newline
4. उ॒च्चै स्तत् तदु॒च्चै रु॒च्चै स्त दित॑रेषा॒ मित॑रेषा॒म् तदु॒च्चै रु॒च्चै स्त दित॑रेषाम् । \newline
5. तदित॑रेषा॒ मित॑रेषा॒म् तत् तदित॑रेषा॒म् ग्रहा॑णा॒म् ग्रहा॑णा॒ मित॑रेषा॒म् तत् तदित॑रेषा॒म् ग्रहा॑णाम् । \newline
6. इत॑रेषा॒म् ग्रहा॑णा॒म् ग्रहा॑णा॒ मित॑रेषा॒ मित॑रेषा॒म् ग्रहा॑णा मे॒त दे॒तद् ग्रहा॑णा॒ मित॑रेषा॒ मित॑रेषा॒म् ग्रहा॑णा मे॒तत् । \newline
7. ग्रहा॑णा मे॒त दे॒तद् ग्रहा॑णा॒म् ग्रहा॑णा मे॒तद् वै वा ए॒तद् ग्रहा॑णा॒म् ग्रहा॑णा मे॒तद् वै । \newline
8. ए॒तद् वै वा ए॒त दे॒तद् वै ग्रहा॑णा॒म् ग्रहा॑णां॒ ॅवा ए॒त दे॒तद् वै ग्रहा॑णाम् । \newline
9. वै ग्रहा॑णा॒म् ग्रहा॑णां॒ ॅवै वै ग्रहा॑णाम् नि॒दान॑म् नि॒दान॒म् ग्रहा॑णां॒ ॅवै वै ग्रहा॑णाम् नि॒दान᳚म् । \newline
10. ग्रहा॑णाम् नि॒दान॑म् नि॒दान॒म् ग्रहा॑णा॒म् ग्रहा॑णाम् नि॒दानं॒ ॅयो यो नि॒दान॒म् ग्रहा॑णा॒म् ग्रहा॑णाम् नि॒दानं॒ ॅयः । \newline
11. नि॒दानं॒ ॅयो यो नि॒दान॑म् नि॒दानं॒ ॅय ए॒व मे॒वं ॅयो नि॒दान॑म् नि॒दानं॒ ॅय ए॒वम् । \newline
12. नि॒दान॒मिति॑ नि - दान᳚म् । \newline
13. य ए॒व मे॒वं ॅयो य ए॒वं ॅवेद॒ वेदै॒वं ॅयो य ए॒वं ॅवेद॑ । \newline
14. ए॒वं ॅवेद॒ वेदै॒व मे॒वं ॅवेद॑ नि॒दान॑वान् नि॒दान॑वा॒न्॒. वेदै॒व मे॒वं ॅवेद॑ नि॒दान॑वान् । \newline
15. वेद॑ नि॒दान॑वान् नि॒दान॑वा॒न्॒. वेद॒ वेद॑ नि॒दान॑वान् भवति भवति नि॒दान॑वा॒न्॒. वेद॒ वेद॑ नि॒दान॑वान् भवति । \newline
16. नि॒दान॑वान् भवति भवति नि॒दान॑वान् नि॒दान॑वान् भवति॒ यो यो भ॑वति नि॒दान॑वान् नि॒दान॑वान् भवति॒ यः । \newline
17. नि॒दान॑वा॒निति॑ नि॒दान॑ - वा॒न् । \newline
18. भ॒व॒ति॒ यो यो भ॑वति भवति॒ यो वै वै यो भ॑वति भवति॒ यो वै । \newline
19. यो वै वै यो यो वै ग्रहा॑णा॒म् ग्रहा॑णां॒ ॅवै यो यो वै ग्रहा॑णाम् । \newline
20. वै ग्रहा॑णा॒म् ग्रहा॑णां॒ ॅवै वै ग्रहा॑णाम् मिथु॒नम् मि॑थु॒नम् ग्रहा॑णां॒ ॅवै वै ग्रहा॑णाम् मिथु॒नम् । \newline
21. ग्रहा॑णाम् मिथु॒नम् मि॑थु॒नम् ग्रहा॑णा॒म् ग्रहा॑णाम् मिथु॒नं ॅवेद॒ वेद॑ मिथु॒नम् ग्रहा॑णा॒म् ग्रहा॑णाम् मिथु॒नं ॅवेद॑ । \newline
22. मि॒थु॒नं ॅवेद॒ वेद॑ मिथु॒नम् मि॑थु॒नं ॅवेद॒ प्र प्र वेद॑ मिथु॒नम् मि॑थु॒नं ॅवेद॒ प्र । \newline
23. वेद॒ प्र प्र वेद॒ वेद॒ प्र प्र॒जया᳚ प्र॒जया॒ प्र वेद॒ वेद॒ प्र प्र॒जया᳚ । \newline
24. प्र प्र॒जया᳚ प्र॒जया॒ प्र प्र प्र॒जया॑ प॒शुभिः॑ प॒शुभिः॑ प्र॒जया॒ प्र प्र प्र॒जया॑ प॒शुभिः॑ । \newline
25. प्र॒जया॑ प॒शुभिः॑ प॒शुभिः॑ प्र॒जया᳚ प्र॒जया॑ प॒शुभि॑र् मिथु॒नैर् मि॑थु॒नैः प॒शुभिः॑ प्र॒जया᳚ प्र॒जया॑ प॒शुभि॑र् मिथु॒नैः । \newline
26. प्र॒जयेति॑ प्र - जया᳚ । \newline
27. प॒शुभि॑र् मिथु॒नैर् मि॑थु॒नैः प॒शुभिः॑ प॒शुभि॑र् मिथु॒नैर् जा॑यते जायते मिथु॒नैः प॒शुभिः॑ प॒शुभि॑र् मिथु॒नैर् जा॑यते । \newline
28. प॒शुभि॒रिति॑ प॒शु - भिः॒ । \newline
29. मि॒थु॒नैर् जा॑यते जायते मिथु॒नैर् मि॑थु॒नैर् जा॑यते स्था॒लीभिः॑ स्था॒लीभि॑र् जायते मिथु॒नैर् मि॑थु॒नैर् जा॑यते स्था॒लीभिः॑ । \newline
30. जा॒य॒ते॒ स्था॒लीभिः॑ स्था॒लीभि॑र् जायते जायते स्था॒लीभि॑ र॒न्ये᳚ ऽन्ये स्था॒लीभि॑र् जायते जायते स्था॒लीभि॑ र॒न्ये । \newline
31. स्था॒लीभि॑ र॒न्ये᳚ ऽन्ये स्था॒लीभिः॑ स्था॒लीभि॑ र॒न्ये ग्रहा॒ ग्रहा॑ अ॒न्ये स्था॒लीभिः॑ स्था॒लीभि॑ र॒न्ये ग्रहाः᳚ । \newline
32. अ॒न्ये ग्रहा॒ ग्रहा॑ अ॒न्ये᳚ ऽन्ये ग्रहा॑ गृ॒ह्यन्ते॑ गृ॒ह्यन्ते॒ ग्रहा॑ अ॒न्ये᳚ ऽन्ये ग्रहा॑ गृ॒ह्यन्ते᳚ । \newline
33. ग्रहा॑ गृ॒ह्यन्ते॑ गृ॒ह्यन्ते॒ ग्रहा॒ ग्रहा॑ गृ॒ह्यन्ते॑ वाय॒व्यै᳚र् वाय॒व्यै᳚र् गृ॒ह्यन्ते॒ ग्रहा॒ ग्रहा॑ गृ॒ह्यन्ते॑ वाय॒व्यैः᳚ । \newline
34. गृ॒ह्यन्ते॑ वाय॒व्यै᳚र् वाय॒व्यै᳚र् गृ॒ह्यन्ते॑ गृ॒ह्यन्ते॑ वाय॒व्यै॑ र॒न्ये᳚ ऽन्ये वा॑य॒व्यै᳚र् गृ॒ह्यन्ते॑ गृ॒ह्यन्ते॑ वाय॒व्यै॑ र॒न्ये । \newline
35. वा॒य॒व्यै॑ र॒न्ये᳚ ऽन्ये वा॑य॒व्यै᳚र् वाय॒व्यै॑ र॒न्य ए॒त दे॒त द॒न्ये वा॑य॒व्यै᳚र् वाय॒व्यै॑ र॒न्य ए॒तत् । \newline
36. अ॒न्य ए॒त दे॒त द॒न्ये᳚ ऽन्य ए॒तद् वै वा ए॒त द॒न्ये᳚ ऽन्य ए॒तद् वै । \newline
37. ए॒तद् वै वा ए॒त दे॒तद् वै ग्रहा॑णा॒म् ग्रहा॑णां॒ ॅवा ए॒त दे॒तद् वै ग्रहा॑णाम् । \newline
38. वै ग्रहा॑णा॒म् ग्रहा॑णां॒ ॅवै वै ग्रहा॑णाम् मिथु॒नम् मि॑थु॒नम् ग्रहा॑णां॒ ॅवै वै ग्रहा॑णाम् मिथु॒नम् । \newline
39. ग्रहा॑णाम् मिथु॒नम् मि॑थु॒नम् ग्रहा॑णा॒म् ग्रहा॑णाम् मिथु॒नं ॅयो यो मि॑थु॒नम् ग्रहा॑णा॒म् ग्रहा॑णाम् मिथु॒नं ॅयः । \newline
40. मि॒थु॒नं ॅयो यो मि॑थु॒नम् मि॑थु॒नं ॅय ए॒व मे॒वं ॅयो मि॑थु॒नम् मि॑थु॒नं ॅय ए॒वम् । \newline
41. य ए॒व मे॒वं ॅयो य ए॒वं ॅवेद॒ वेदै॒वं ॅयो य ए॒वं ॅवेद॑ । \newline
42. ए॒वं ॅवेद॒ वेदै॒व मे॒वं ॅवेद॒ प्र प्र वेदै॒व मे॒वं ॅवेद॒ प्र । \newline
43. वेद॒ प्र प्र वेद॒ वेद॒ प्र प्र॒जया᳚ प्र॒जया॒ प्र वेद॒ वेद॒ प्र प्र॒जया᳚ । \newline
44. प्र प्र॒जया᳚ प्र॒जया॒ प्र प्र प्र॒जया॑ प॒शुभिः॑ प॒शुभिः॑ प्र॒जया॒ प्र प्र प्र॒जया॑ प॒शुभिः॑ । \newline
45. प्र॒जया॑ प॒शुभिः॑ प॒शुभिः॑ प्र॒जया᳚ प्र॒जया॑ प॒शुभि॑र् मिथु॒नैर् मि॑थु॒नैः प॒शुभिः॑ प्र॒जया᳚ प्र॒जया॑ प॒शुभि॑र् मिथु॒नैः । \newline
46. प्र॒जयेति॑ प्र - जया᳚ । \newline
47. प॒शुभि॑र् मिथु॒नैर् मि॑थु॒नैः प॒शुभिः॑ प॒शुभि॑र् मिथु॒नैर् जा॑यते जायते मिथु॒नैः प॒शुभिः॑ प॒शुभि॑र् मिथु॒नैर् जा॑यते । \newline
48. प॒शुभि॒रिति॑ प॒शु - भिः॒ । \newline
49. मि॒थु॒नैर् जा॑यते जायते मिथु॒नैर् मि॑थु॒नैर् जा॑यत॒ इन्द्र॒ इन्द्रो॑ जायते मिथु॒नैर् मि॑थु॒नैर् जा॑यत॒ इन्द्रः॑ । \newline
50. जा॒य॒त॒ इन्द्र॒ इन्द्रो॑ जायते जायत॒ इन्द्र॒ स्त्वष्टु॒ स्त्वष्टु॒ रिन्द्रो॑ जायते जायत॒ इन्द्र॒ स्त्वष्टुः॑ । \newline
51. इन्द्र॒ स्त्वष्टु॒ स्त्वष्टु॒ रिन्द्र॒ इन्द्र॒ स्त्वष्टुः॒ सोमꣳ॒॒ सोम॒म् त्वष्टु॒ रिन्द्र॒ इन्द्र॒ स्त्वष्टुः॒ सोम᳚म् । \newline
52. त्वष्टुः॒ सोमꣳ॒॒ सोम॒म् त्वष्टु॒ स्त्वष्टुः॒ सोम॑ मभी॒षहा॑ ऽभी॒षहा॒ सोम॒म् त्वष्टु॒ स्त्वष्टुः॒ सोम॑ मभी॒षहा᳚ । \newline
53. सोम॑ मभी॒षहा॑ ऽभी॒षहा॒ सोमꣳ॒॒ सोम॑ मभी॒षहा॑ ऽपिब दपिब दभी॒षहा॒ सोमꣳ॒॒ सोम॑ मभी॒षहा॑ ऽपिबत् । \newline
54. अ॒भी॒षहा॑ ऽपिब दपिब दभी॒षहा॑ ऽभी॒षहा॑ ऽपिब॒थ् स सो॑ ऽपिब दभी॒षहा॑ ऽभी॒षहा॑ ऽपिब॒थ् सः । \newline
55. अ॒भी॒षहेत्य॑भि - सहा᳚ । \newline
56. अ॒पि॒ब॒थ् स सो॑ ऽपिब दपिब॒थ् स विष्व॒ङ्॒. विष्व॒ङ् ख्सो॑ ऽपिब दपिब॒थ् स विष्वङ्॑ । \newline
57. स विष्व॒ङ्॒. विष्व॒ङ् ख्स स विष्व॒ङ्॒. वि वि विष्व॒ङ् ख्स स विष्व॒ङ्॒. वि । \newline
58. विष्व॒ङ्॒. वि वि विष्व॒ङ्॒. विष्व॒ङ् व्या᳚र्च्छ दार्च्छ॒द् वि विष्व॒ङ्॒. विष्व॒ङ् व्या᳚र्च्छत् । \newline
\pagebreak
\markright{ TS 6.5.11.4  \hfill https://www.vedavms.in \hfill}

\section{ TS 6.5.11.4 }

\textbf{TS 6.5.11.4 } \newline
\textbf{Samhita Paata} \newline

व्या᳚र्च्छ॒थ् स आ॒त्मन्ना॒रम॑णं॒ नावि॑न्द॒थ् स ए॒तान॑नुसव॒नं पु॑रो॒डाशा॑नपश्य॒त् तान् निर॑वप॒त् तैर्वै स आ॒त्मन्ना॒रम॑ण-मकुरुत॒ तस्मा॑दनुसव॒नं पु॑रो॒डाशा॒ निरु॑प्यन्ते॒ तस्मा॑दनुसव॒नं पु॑रो॒डाशा॑नां॒ प्राश्नी॑यादा॒त्म-न्ने॒वाऽऽ*रम॑णं कुरुते॒ नैनꣳ॒॒ सोमोऽति॑ पवते ब्रह्मवा॒दिनो॑ वदन्ति॒ नर्चा न यजु॑षा प॒ङ्क्तिरा᳚प्य॒तेऽथ॒ किं ( ) ॅय॒ज्ञ्स्य॑ पाङ्क्त॒त्वमिति॑ धा॒नाः क॑र॒भंः प॑रिवा॒पः पु॑रो॒डाशः॑ पय॒स्या॑ तेन॑ प॒ङ्क्तिरा॑प्यते॒ तद्-य॒ज्ञ्स्य॑ पाङ्क्त॒त्वं ॥ \newline

\textbf{Pada Paata} \newline

वीति॑ । आ॒र्च्छ॒त् । सः । आ॒त्मन्न् । आ॒रम॑ण॒मित्या᳚ - रम॑णम् । न । अ॒वि॒न्द॒त् । सः । ए॒तान् । अ॒नु॒स॒व॒नमित्य॑नु-स॒व॒नम् । पु॒रो॒डाशान्॑ । अ॒प॒श्य॒त् । तान् । निरिति॑ । अ॒व॒प॒त् । तैः । वै । सः । आ॒त्मन्न् । आ॒रम॑ण॒मित्या᳚ - रम॑णम् । अ॒कु॒रु॒त॒ । तस्मा᳚त् । अ॒नु॒स॒व॒नमित्य॑नु - स॒व॒नम् । पु॒रो॒डाशाः᳚ । निरिति॑ । उ॒प्य॒न्ते॒ । तस्मा᳚त् । अ॒नु॒स॒व॒नमित्य॑नु - स॒व॒नम् । पु॒रो॒डाशा॑नाम् । प्रेति॑ । अ॒श्नी॒या॒त् । आ॒त्मन्न् । ए॒व । आ॒रम॑ण॒मित्या᳚ - रम॑णम् । कु॒रु॒ते॒ । न । ए॒न॒म् । सोमः॑ । अतीति॑ । प॒व॒ते॒ । ब्र॒ह्म॒वा॒दिन॒ इति॑ ब्रह्म - वा॒दिनः॑ । व॒द॒न्ति॒ । न । ऋ॒चा । न । यजु॑षा । प॒ङ्क्तिः । आ॒प्य॒ते॒ । अथ॑ । किम् ( ) । य॒ज्ञ्स्य॑ । पा॒ङ्क्त॒त्वमिति॑ पाङ्क्त - त्वम् । इति॑ । धा॒नाः । क॒र॒म्भः । प॒रि॒वा॒प इति॑ परि - वा॒पः । पु॒रो॒डाशः॑ । प॒य॒स्या᳚ । तेन॑ । प॒ङ्क्तिः । आ॒प्य॒ते॒ । तत् । य॒ज्ञ्स्य॑ । पा॒ङ्क्त॒त्वमिति॑ पाङ्क्त - त्वम् ॥  \newline


\textbf{Krama Paata} \newline

व्या᳚र्च्छत् । आ॒र्च्छ॒थ् सः । स आ॒त्मन्न् । आ॒त्मन्ना॒रम॑णम् । आ॒रम॑ण॒म् न । आ॒रम॑ण॒मित्या᳚ - रम॑णम् । नावि॑न्दत् । अ॒वि॒न्द॒थ् सः । स ए॒तान् । ए॒तान॑नुसव॒नम् । अ॒नु॒स॒व॒नम् पु॑रो॒डाशान्॑ । अ॒नु॒स॒व॒नमित्य॑नु - स॒व॒नम् । पु॒रो॒डाशा॑नपश्यत् । अ॒प॒श्य॒त् तान् । तान् निः । निर॑वपत् । अ॒व॒प॒त् तैः । तैर् वै । वै सः । स आ॒त्मन्न् । आ॒त्मन्ना॒रम॑णम् । आ॒रम॑णमकुरुत । आ॒रम॑ण॒मित्या᳚ - रम॑णम् । अ॒कु॒रु॒त॒ तस्मा᳚त् । तस्मा॑दनुसव॒नम् । अ॒नु॒स॒व॒नम् पु॑रो॒डाशाः᳚ । अ॒नु॒स॒व॒नमित्य॑नु - स॒व॒नम् । पु॒रो॒डाशा॒ निः । निरु॑प्यन्ते । उ॒प्य॒न्ते॒ तस्मा᳚त् । तस्मा॑दनुसव॒नम् । अ॒नु॒स॒व॒नम् पु॑रो॒डाशा॑नाम् । अ॒नु॒स॒व॒नमित्य॑नु - स॒व॒नम् । पु॒रो॒डाशा॑ना॒म् प्र । प्राश्ञी॑यात् । अ॒श्ञी॒या॒दा॒त्मन्न् । आ॒त्मन्ने॒व । ए॒वारम॑णम् । आ॒रम॑णम् कुरुते । आ॒रम॑ण॒मित्या᳚ - रम॑णम् । कु॒रु॒ते॒ न । नैन᳚म् । ए॒नꣳ॒॒ सोमः॑ । सोमोऽति॑ । अति॑ पवते । प॒व॒ते॒ ब्र॒ह्म॒वा॒दिनः॑ । ब्र॒ह्म॒वा॒दिनो॑ वदन्ति । ब्र॒ह्म॒वा॒दिन॒ इति॑ ब्रह्म - वा॒दिनः॑ । व॒द॒न्ति॒ न । नर्चा । ऋ॒चा न । न यजु॑षा । यजु॑षा प॒ङ्‍क्तिः । प॒ङ्‍क्तिरा᳚प्यते । आ॒प्य॒तेऽथ॑ । अथ॒ किम् ( ) । किम् ॅय॒ज्ञ्स्य॑ । य॒ज्ञ्स्य॑ पाङ्‍क्त॒त्वम् । पा॒ङ्‍क्त॒त्वमिति॑ । पा॒ङ्‍क्त॒त्वमिति॑ पाङ्‍क्त - त्वम् । इति॑ धा॒नाः । धा॒नाः क॑र॒म्भः । क॒र॒म्भः प॑रिवा॒पः । प॒रि॒वा॒पः पु॑रो॒डाशः॑ । प॒रि॒वा॒प इति॑ परि - वा॒पः । पु॒रो॒डाशः॑ पय॒स्या᳚ । प॒य॒स्या॑ तेन॑ । तेन॑ प॒ङ्‍क्तिः । प॒ङ्‍क्तिरा᳚प्यते । आ॒प्य॒ते॒ तत् । तद् य॒ज्ञ्स्य॑ । य॒ज्ञ्स्य॑ पाङ्‍क्त॒त्वम् । पा॒ङ्‍क्त॒त्वमिति॑ पाङ्‍क्त - त्वम् । \newline

\textbf{Jatai Paata} \newline

1. व्या᳚र्च्छ दार्च्छ॒द् वि व्या᳚र्च्छत् । \newline
2. आ॒र्च्छ॒थ् स स आ᳚र्च्छ दार्च्छ॒थ् सः । \newline
3. स आ॒त्मन् ना॒त्मन् थ्स स आ॒त्मन्न् । \newline
4. आ॒त्मन् ना॒रम॑ण मा॒रम॑ण मा॒त्मन् ना॒त्मन् ना॒रम॑णम् । \newline
5. आ॒रम॑ण॒न् न नारम॑ण मा॒रम॑ण॒न् न । \newline
6. आ॒रम॑ण॒मित्या᳚ - रम॑णम् । \newline
7. नावि॑न्द दविन्द॒न् न नावि॑न्दत् । \newline
8. अ॒वि॒न्द॒थ् स सो॑ ऽविन्द दविन्द॒थ् सः । \newline
9. स ए॒ता ने॒तान् थ्स स ए॒तान् । \newline
10. ए॒ता न॑नुसव॒न म॑नुसव॒न मे॒ता ने॒ता न॑नुसव॒नम् । \newline
11. अ॒नु॒स॒व॒नम् पु॑रो॒डाशा᳚न् पुरो॒डाशा॑ ननुसव॒न म॑नुसव॒नम् पु॑रो॒डाशान्॑ । \newline
12. अ॒नु॒स॒व॒नमित्य॑नु - स॒व॒नम् । \newline
13. पु॒रो॒डाशा॑ नपश्य दपश्यत् पुरो॒डाशा᳚न् पुरो॒डाशा॑ नपश्यत् । \newline
14. अ॒प॒श्य॒त् ताꣳ स्ता न॑पश्य दपश्य॒त् तान् । \newline
15. तान् निर् णिष् टाꣳ स्तान् निः । \newline
16. निर॑वप दवप॒न् निर् णिर॑वपत् । \newline
17. अ॒व॒प॒त् तै स्तै र॑वप दवप॒त् तैः । \newline
18. तैर् वै वै तै स्तैर् वै । \newline
19. वै स स वै वै सः । \newline
20. स आ॒त्मन् ना॒त्मन् थ्स स आ॒त्मन्न् । \newline
21. आ॒त्मन् ना॒रम॑ण मा॒रम॑ण मा॒त्मन् ना॒त्मन् ना॒रम॑णम् । \newline
22. आ॒रम॑ण मकुरुता कुरुता॒ रम॑ण मा॒रम॑ण मकुरुत । \newline
23. आ॒रम॑ण॒मित्या᳚ - रम॑णम् । \newline
24. अ॒कु॒रु॒त॒ तस्मा॒त् तस्मा॑ दकुरुता कुरुत॒ तस्मा᳚त् । \newline
25. तस्मा॑ दनुसव॒न म॑नुसव॒नम् तस्मा॒त् तस्मा॑ दनुसव॒नम् । \newline
26. अ॒नु॒स॒व॒नम् पु॑रो॒डाशाः᳚ पुरो॒डाशा॑ अनुसव॒न म॑नुसव॒नम् पु॑रो॒डाशाः᳚ । \newline
27. अ॒नु॒स॒व॒नमित्य॑नु - स॒व॒नम् । \newline
28. पु॒रो॒डाशा॒ निर् णिष् पु॑रो॒डाशाः᳚ पुरो॒डाशा॒ निः । \newline
29. निरु॑प्यन्त उप्यन्ते॒ निर् णिरु॑प्यन्ते । \newline
30. उ॒प्य॒न्ते॒ तस्मा॒त् तस्मा॑ दुप्यन्त उप्यन्ते॒ तस्मा᳚त् । \newline
31. तस्मा॑ दनुसव॒न म॑नुसव॒नम् तस्मा॒त् तस्मा॑ दनुसव॒नम् । \newline
32. अ॒नु॒स॒व॒नम् पु॑रो॒डाशा॑नाम् पुरो॒डाशा॑ना मनुसव॒न म॑नुसव॒नम् पु॑रो॒डाशा॑नाम् । \newline
33. अ॒नु॒स॒व॒नमित्य॑नु - स॒व॒नम् । \newline
34. पु॒रो॒डाशा॑ना॒म् प्र प्र पु॑रो॒डाशा॑नाम् पुरो॒डाशा॑ना॒म् प्र । \newline
35. प्राश्ञी॑या दश्ञीया॒त् प्र प्राश्ञी॑यात् । \newline
36. अ॒श्ञी॒या॒ दा॒त्मन् ना॒त्मन् न॑श्ञीया दश्ञीया दा॒त्मन्न् । \newline
37. आ॒त्मन् ने॒वै वात्मन् ना॒त्मन् ने॒व । \newline
38. ए॒वारम॑ण मा॒रम॑ण मे॒वै वारम॑णम् । \newline
39. आ॒रम॑णम् कुरुते कुरुत आ॒रम॑ण मा॒रम॑णम् कुरुते । \newline
40. आ॒रम॑ण॒मित्या᳚ - रम॑णम् । \newline
41. कु॒रु॒ते॒ न न कु॑रुते कुरुते॒ न । \newline
42. नैन॑ मेन॒न् न नैन᳚म् । \newline
43. ए॒नꣳ॒॒ सोमः॒ सोम॑ एन मेनꣳ॒॒ सोमः॑ । \newline
44. सोमो ऽत्यति॒ सोमः॒ सोमो ऽति॑ । \newline
45. अति॑ पवते पव॒ते ऽत्यति॑ पवते । \newline
46. प॒व॒ते॒ ब्र॒ह्म॒वा॒दिनो᳚ ब्रह्मवा॒दिनः॑ पवते पवते ब्रह्मवा॒दिनः॑ । \newline
47. ब्र॒ह्म॒वा॒दिनो॑ वदन्ति वदन्ति ब्रह्मवा॒दिनो᳚ ब्रह्मवा॒दिनो॑ वदन्ति । \newline
48. ब्र॒ह्म॒वा॒दिन॒ इति॑ ब्रह्म - वा॒दिनः॑ । \newline
49. व॒द॒न्ति॒ न न व॑दन्ति वदन्ति॒ न । \newline
50. न र्‌च र्‌चा न न र्‌चा । \newline
51. ऋ॒चा न न र्‌च र्‌चा न । \newline
52. न यजु॑षा॒ यजु॑षा॒ न न यजु॑षा । \newline
53. यजु॑षा प॒ङ्क्तिः प॒ङ्क्तिर् यजु॑षा॒ यजु॑षा प॒ङ्क्तिः । \newline
54. प॒ङ्क्ति रा᳚प्यत आप्यते प॒ङ्क्तिः प॒ङ्क्ति रा᳚प्यते । \newline
55. आ॒प्य॒ते ऽथाथा᳚ प्यत आप्य॒ते ऽथ॑ । \newline
56. अथ॒ किम् कि मथाथ॒ किम् । \newline
57. किं ॅय॒ज्ञ्स्य॑ य॒ज्ञ्स्य॒ किम् किं ॅय॒ज्ञ्स्य॑ । \newline
58. य॒ज्ञ्स्य॑ पाङ्क्त॒त्वम् पा᳚ङ्क्त॒त्वं ॅय॒ज्ञ्स्य॑ य॒ज्ञ्स्य॑ पाङ्क्त॒त्वम् । \newline
59. पा॒ङ्क्त॒त्व मितीति॑ पाङ्क्त॒त्वम् पा᳚ङ्क्त॒त्व मिति॑ । \newline
60. पा॒ङ्क्त॒त्वमिति॑ पाङ्क्त - त्वम् । \newline
61. इति॑ धा॒ना धा॒ना इतीति॑ धा॒नाः । \newline
62. धा॒नाः क॑र॒म्भः क॑र॒म्भो धा॒ना धा॒नाः क॑र॒म्भः । \newline
63. क॒र॒म्भः प॑रिवा॒पः प॑रिवा॒पः क॑र॒म्भः क॑र॒म्भः प॑रिवा॒पः । \newline
64. प॒रि॒वा॒पः पु॑रो॒डाशः॑ पुरो॒डाशः॑ परिवा॒पः प॑रिवा॒पः पु॑रो॒डाशः॑ । \newline
65. प॒रि॒वा॒प इति॑ परि - वा॒पः । \newline
66. पु॒रो॒डाशः॑ पय॒स्या॑ पय॒स्या॑ पुरो॒डाशः॑ पुरो॒डाशः॑ पय॒स्या᳚ । \newline
67. प॒य॒स्या॑ तेन॒ तेन॑ पय॒स्या॑ पय॒स्या॑ तेन॑ । \newline
68. तेन॑ प॒ङ्क्तिः प॒ङ्क्ति स्तेन॒ तेन॑ प॒ङ्क्तिः । \newline
69. प॒ङ्क्ति रा᳚प्यत आप्यते प॒ङ्क्तिः प॒ङ्क्ति रा᳚प्यते । \newline
70. आ॒प्य॒ते॒ तत् तदा᳚प्यत आप्यते॒ तत् । \newline
71. तद् य॒ज्ञ्स्य॑ य॒ज्ञ्स्य॒ तत् तद् य॒ज्ञ्स्य॑ । \newline
72. य॒ज्ञ्स्य॑ पाङ्क्त॒त्वम् पा᳚ङ्क्त॒त्वं ॅय॒ज्ञ्स्य॑ य॒ज्ञ्स्य॑ पाङ्क्त॒त्वम् । \newline
73. पा॒ङ्क्त॒त्वमिति॑ पाङ्क्त - त्वम् । \newline

\textbf{Ghana Paata } \newline

1. व्या᳚र्च्छ दार्च्छ॒द् वि व्या᳚र्च्छ॒थ् स स आ᳚र्च्छ॒द् वि व्या᳚र्च्छ॒थ् सः । \newline
2. आ॒र्च्छ॒थ् स स आ᳚र्च्छ दार्च्छ॒थ् स आ॒त्मन्-ना॒त्मन् थ्स आ᳚र्च्छ दार्च्छ॒थ् स आ॒त्मन्न् । \newline
3. स आ॒त्मन्-ना॒त्मन् थ्स स आ॒त्मन्-ना॒रम॑ण मा॒रम॑ण मा॒त्मन् थ्स स आ॒त्मन्-ना॒रम॑णम् । \newline
4. आ॒त्मन्-ना॒रम॑ण मा॒रम॑ण मा॒त्मन्-ना॒त्मन्-ना॒रम॑ण॒न् न नारम॑ण मा॒त्मन्-ना॒त्मन्-ना॒रम॑ण॒न् न । \newline
5. आ॒रम॑ण॒न् न नारम॑ण मा॒रम॑ण॒न् नावि॑न्द दविन्द॒न् नारम॑ण मा॒रम॑ण॒म् नावि॑न्दत् । \newline
6. आ॒रम॑ण॒मित्या᳚ - रम॑णम् । \newline
7. नावि॑न्द दविन्द॒न् न नावि॑न्द॒थ् स सो॑ ऽविन्द॒न् न नावि॑न्द॒थ् सः । \newline
8. अ॒वि॒न्द॒थ् स सो॑ ऽविन्द दविन्द॒थ् स ए॒ता ने॒तान् थ्सो॑ ऽविन्द दविन्द॒थ् स ए॒तान् । \newline
9. स ए॒ता-ने॒तान् थ्स स ए॒ता न॑नुसव॒न म॑नुसव॒न मे॒तान् थ्स स ए॒ता न॑नुसव॒नम् । \newline
10. ए॒ता न॑नुसव॒न म॑नुसव॒न मे॒ता-ने॒ता न॑नुसव॒नम् पु॑रो॒डाशा᳚न् पुरो॒डाशा॑-ननुसव॒न मे॒ता-ने॒ता न॑नुसव॒नम् पु॑रो॒डाशान्॑ । \newline
11. अ॒नु॒स॒व॒नम् पु॑रो॒डाशा᳚न् पुरो॒डाशा॑-ननुसव॒न म॑नुसव॒नम् पु॑रो॒डाशा॑-नपश्य दपश्यत् पुरो॒डाशा॑-ननुसव॒न म॑नुसव॒नम् पु॑रो॒डाशा॑-नपश्यत् । \newline
12. अ॒नु॒स॒व॒नमित्य॑नु - स॒व॒नम् । \newline
13. पु॒रो॒डाशा॑-नपश्य दपश्यत् पुरो॒डाशा᳚न् पुरो॒डाशा॑-नपश्य॒त् ताꣳ स्तान॑पश्यत् पुरो॒डाशा᳚न् पुरो॒डाशा॑-नपश्य॒त् तान् । \newline
14. अ॒प॒श्य॒त् ताꣳ स्ता-न॑पश्य दपश्य॒त् तान् निर् णिष् टान॑पश्य दपश्य॒त् तान् निः । \newline
15. तान् निर् णिष् टाꣳ स्तान् निर॑वप दवप॒न् निष् टाꣳ स्तान् निर॑वपत् । \newline
16. निर॑वप दवप॒न् निर् णिर॑वप॒त् तै स्तै र॑वप॒न् निर् णिर॑वप॒त् तैः । \newline
17. अ॒व॒प॒त् तै स्तै र॑वप दवप॒त् तैर् वै वै तै र॑वप दवप॒त् तैर् वै । \newline
18. तैर् वै वै तै स्तैर् वै स स वै तै स्तैर् वै सः । \newline
19. वै स स वै वै स आ॒त्मन्-ना॒त्मन् थ्स वै वै स आ॒त्मन्न् । \newline
20. स आ॒त्मन्-ना॒त्मन् थ्स स आ॒त्मन्-ना॒रम॑ण मा॒रम॑ण मा॒त्मन् थ्स स आ॒त्मन्-ना॒रम॑णम् । \newline
21. आ॒त्मन्-ना॒रम॑ण मा॒रम॑ण मा॒त्मन्-ना॒त्मन्-ना॒रम॑ण मकुरुता कुरुता॒ रम॑ण मा॒त्मन्-ना॒त्मन्-ना॒रम॑ण मकुरुत । \newline
22. आ॒रम॑ण मकुरुता कुरुता॒ रम॑ण मा॒रम॑ण मकुरुत॒ तस्मा॒त् तस्मा॑ दकुरुता॒ रम॑ण मा॒रम॑ण मकुरुत॒ तस्मा᳚त् । \newline
23. आ॒रम॑ण॒मित्या᳚ - रम॑णम् । \newline
24. अ॒कु॒रु॒त॒ तस्मा॒त् तस्मा॑ दकुरुता कुरुत॒ तस्मा॑ दनुसव॒न म॑नुसव॒नम् तस्मा॑ दकुरुता कुरुत॒ तस्मा॑ दनुसव॒नम् । \newline
25. तस्मा॑ दनुसव॒न म॑नुसव॒नम् तस्मा॒त् तस्मा॑ दनुसव॒नम् पु॑रो॒डाशाः᳚ पुरो॒डाशा॑ अनुसव॒नम् तस्मा॒त् तस्मा॑ दनुसव॒नम् पु॑रो॒डाशाः᳚ । \newline
26. अ॒नु॒स॒व॒नम् पु॑रो॒डाशाः᳚ पुरो॒डाशा॑ अनुसव॒न म॑नुसव॒नम् पु॑रो॒डाशा॒ निर् णिष् पु॑रो॒डाशा॑ अनुसव॒न म॑नुसव॒नम् पु॑रो॒डाशा॒ निः । \newline
27. अ॒नु॒स॒व॒नमित्य॑नु - स॒व॒नम् । \newline
28. पु॒रो॒डाशा॒ निर् णिष् पु॑रो॒डाशाः᳚ पुरो॒डाशा॒ निरु॑प्यन्त उप्यन्ते॒ निष् पु॑रो॒डाशाः᳚ पुरो॒डाशा॒ निरु॑प्यन्ते । \newline
29. निरु॑प्यन्त उप्यन्ते॒ निर् णिरु॑प्यन्ते॒ तस्मा॒त् तस्मा॑ दुप्यन्ते॒ निर् णिरु॑प्यन्ते॒ तस्मा᳚त् । \newline
30. उ॒प्य॒न्ते॒ तस्मा॒त् तस्मा॑ दुप्यन्त उप्यन्ते॒ तस्मा॑ दनुसव॒न म॑नुसव॒नम् तस्मा॑ दुप्यन्त उप्यन्ते॒ तस्मा॑ दनुसव॒नम् । \newline
31. तस्मा॑ दनुसव॒न म॑नुसव॒नम् तस्मा॒त् तस्मा॑ दनुसव॒नम् पु॑रो॒डाशा॑नाम् पुरो॒डाशा॑ना मनुसव॒नम् तस्मा॒त् तस्मा॑ दनुसव॒नम् पु॑रो॒डाशा॑नाम् । \newline
32. अ॒नु॒स॒व॒नम् पु॑रो॒डाशा॑नाम् पुरो॒डाशा॑ना मनुसव॒न म॑नुसव॒नम् पु॑रो॒डाशा॑ना॒म् प्र प्र पु॑रो॒डाशा॑ना मनुसव॒न म॑नुसव॒नम् पु॑रो॒डाशा॑ना॒म् प्र । \newline
33. अ॒नु॒स॒व॒नमित्य॑नु - स॒व॒नम् । \newline
34. पु॒रो॒डाशा॑ना॒म् प्र प्र पु॑रो॒डाशा॑नाम् पुरो॒डाशा॑ना॒म् प्राश्ञी॑या दश्ञीया॒त् प्र पु॑रो॒डाशा॑नाम् पुरो॒डाशा॑ना॒म् प्राश्ञी॑यात् । \newline
35. प्राश्ञी॑या दश्ञीया॒त् प्र प्राश्ञी॑या दा॒त्मन्-ना॒त्मन्-न॑श्ञीया॒त् प्र प्राश्ञी॑या दा॒त्मन्न् । \newline
36. अ॒श्ञी॒या॒ दा॒त्मन्-ना॒त्मन्-न॑श्ञीया दश्ञीया दा॒त्मन्-ने॒वै वात्मन्-न॑श्ञीया दश्ञीया दा॒त्मन्-ने॒व । \newline
37. आ॒त्मन्-ने॒वै वात्मन्-ना॒त्मन्-ने॒वारम॑ण मा॒रम॑ण मे॒वात्मन्-ना॒त्मन्-ने॒वारम॑णम् । \newline
38. ए॒वारम॑ण मा॒रम॑ण मे॒वै वारम॑णम् कुरुते कुरुत आ॒रम॑ण मे॒वै वारम॑णम् कुरुते । \newline
39. आ॒रम॑णम् कुरुते कुरुत आ॒रम॑ण मा॒रम॑णम् कुरुते॒ न न कु॑रुत आ॒रम॑ण मा॒रम॑णम् कुरुते॒ न । \newline
40. आ॒रम॑ण॒मित्या᳚ - रम॑णम् । \newline
41. कु॒रु॒ते॒ न न कु॑रुते कुरुते॒ नैन॑ मेन॒न् न कु॑रुते कुरुते॒ नैन᳚म् । \newline
42. नैन॑ मेन॒न् न नैनꣳ॒॒ सोमः॒ सोम॑ एन॒न् न नैनꣳ॒॒ सोमः॑ । \newline
43. ए॒नꣳ॒॒ सोमः॒ सोम॑ एन मेनꣳ॒॒ सोमो ऽत्यति॒ सोम॑ एन मेनꣳ॒॒ सोमो ऽति॑ । \newline
44. सोमो ऽत्यति॒ सोमः॒ सोमो ऽति॑ पवते पव॒ते ऽति॒ सोमः॒ सोमो ऽति॑ पवते । \newline
45. अति॑ पवते पव॒ते ऽत्यति॑ पवते ब्रह्मवा॒दिनो᳚ ब्रह्मवा॒दिनः॑ पव॒ते ऽत्यति॑ पवते ब्रह्मवा॒दिनः॑ । \newline
46. प॒व॒ते॒ ब्र॒ह्म॒वा॒दिनो᳚ ब्रह्मवा॒दिनः॑ पवते पवते ब्रह्मवा॒दिनो॑ वदन्ति वदन्ति ब्रह्मवा॒दिनः॑ पवते पवते ब्रह्मवा॒दिनो॑ वदन्ति । \newline
47. ब्र॒ह्म॒वा॒दिनो॑ वदन्ति वदन्ति ब्रह्मवा॒दिनो᳚ ब्रह्मवा॒दिनो॑ वदन्ति॒ न न व॑दन्ति ब्रह्मवा॒दिनो᳚ ब्रह्मवा॒दिनो॑ वदन्ति॒ न । \newline
48. ब्र॒ह्म॒वा॒दिन॒ इति॑ ब्रह्म - वा॒दिनः॑ । \newline
49. व॒द॒न्ति॒ न न व॑दन्ति वदन्ति॒ न र्‌च र्‌चा न व॑दन्ति वदन्ति॒ न र्‌चा । \newline
50. न र्‌च र्‌चा न न र्‌चा न न र्‌चा न न र्‌चा न । \newline
51. ऋ॒चा न न र्‌च र्‌चा न यजु॑षा॒ यजु॑षा॒ न र्‌च र्‌चा न यजु॑षा । \newline
52. न यजु॑षा॒ यजु॑षा॒ न न यजु॑षा प॒ङ्क्तिः प॒ङ्क्तिर् यजु॑षा॒ न न यजु॑षा प॒ङ्क्तिः । \newline
53. यजु॑षा प॒ङ्क्तिः प॒ङ्क्तिर् यजु॑षा॒ यजु॑षा प॒ङ्क्ति रा᳚प्यत आप्यते प॒ङ्क्तिर् यजु॑षा॒ यजु॑षा प॒ङ्क्ति रा᳚प्यते । \newline
54. प॒ङ्क्ति रा᳚प्यत आप्यते प॒ङ्क्तिः प॒ङ्क्ति रा᳚प्य॒ते ऽथाथा᳚प्यते प॒ङ्क्तिः प॒ङ्क्ति रा᳚प्य॒ते ऽथ॑ । \newline
55. आ॒प्य॒ते ऽथाथा᳚ प्यत आप्य॒ते ऽथ॒ किम् कि मथा᳚ प्यत आप्य॒ते ऽथ॒ किम् । \newline
56. अथ॒ किम् कि मथाथ॒ किं ॅय॒ज्ञ्स्य॑ य॒ज्ञ्स्य॒ कि मथाथ॒ किं ॅय॒ज्ञ्स्य॑ । \newline
57. किं ॅय॒ज्ञ्स्य॑ य॒ज्ञ्स्य॒ किम् किं ॅय॒ज्ञ्स्य॑ पाङ्क्त॒त्वम् पा᳚ङ्क्त॒त्वं ॅय॒ज्ञ्स्य॒ किम् किं ॅय॒ज्ञ्स्य॑ पाङ्क्त॒त्वम् । \newline
58. य॒ज्ञ्स्य॑ पाङ्क्त॒त्वम् पा᳚ङ्क्त॒त्वं ॅय॒ज्ञ्स्य॑ य॒ज्ञ्स्य॑ पाङ्क्त॒त्व मितीति॑ पाङ्क्त॒त्वं ॅय॒ज्ञ्स्य॑ य॒ज्ञ्स्य॑ पाङ्क्त॒त्व मिति॑ । \newline
59. पा॒ङ्क्त॒त्व मितीति॑ पाङ्क्त॒त्वम् पा᳚ङ्क्त॒त्व मिति॑ धा॒ना धा॒ना इति॑ पाङ्क्त॒त्वम् पा᳚ङ्क्त॒त्व मिति॑ धा॒नाः । \newline
60. पा॒ङ्क्त॒त्वमिति॑ पाङ्क्त - त्वम् । \newline
61. इति॑ धा॒ना धा॒ना इतीति॑ धा॒नाः क॑र॒म्भः क॑र॒म्भो धा॒ना इतीति॑ धा॒नाः क॑र॒म्भः । \newline
62. धा॒नाः क॑र॒म्भः क॑र॒म्भो धा॒ना धा॒नाः क॑र॒म्भः प॑रिवा॒पः प॑रिवा॒पः क॑र॒म्भो धा॒ना धा॒नाः क॑र॒म्भः प॑रिवा॒पः । \newline
63. क॒र॒म्भः प॑रिवा॒पः प॑रिवा॒पः क॑र॒म्भः क॑र॒म्भः प॑रिवा॒पः पु॑रो॒डाशः॑ पुरो॒डाशः॑ परिवा॒पः क॑र॒म्भः क॑र॒म्भः प॑रिवा॒पः पु॑रो॒डाशः॑ । \newline
64. प॒रि॒वा॒पः पु॑रो॒डाशः॑ पुरो॒डाशः॑ परिवा॒पः प॑रिवा॒पः पु॑रो॒डाशः॑ पय॒स्या॑ पय॒स्या॑ पुरो॒डाशः॑ परिवा॒पः प॑रिवा॒पः पु॑रो॒डाशः॑ पय॒स्या᳚ । \newline
65. प॒रि॒वा॒प इति॑ परि - वा॒पः । \newline
66. पु॒रो॒डाशः॑ पय॒स्या॑ पय॒स्या॑ पुरो॒डाशः॑ पुरो॒डाशः॑ पय॒स्या॑ तेन॒ तेन॑ पय॒स्या॑ पुरो॒डाशः॑ पुरो॒डाशः॑ पय॒स्या॑ तेन॑ । \newline
67. प॒य॒स्या॑ तेन॒ तेन॑ पय॒स्या॑ पय॒स्या॑ तेन॑ प॒ङ्क्तिः प॒ङ्क्ति स्तेन॑ पय॒स्या॑ पय॒स्या॑ तेन॑ प॒ङ्क्तिः । \newline
68. तेन॑ प॒ङ्क्तिः प॒ङ्क्ति स्तेन॒ तेन॑ प॒ङ्क्ति रा᳚प्यत आप्यते प॒ङ्क्ति स्तेन॒ तेन॑ प॒ङ्क्ति रा᳚प्यते । \newline
69. प॒ङ्क्ति रा᳚प्यत आप्यते प॒ङ्क्तिः प॒ङ्क्ति रा᳚प्यते॒ तत् तदा᳚प्यते प॒ङ्क्तिः प॒ङ्क्ति रा᳚प्यते॒ तत् । \newline
70. आ॒प्य॒ते॒ तत् तदा᳚प्यत आप्यते॒ तद् य॒ज्ञ्स्य॑ य॒ज्ञ्स्य॒ तदा᳚प्यत आप्यते॒ तद् य॒ज्ञ्स्य॑ । \newline
71. तद् य॒ज्ञ्स्य॑ य॒ज्ञ्स्य॒ तत् तद् य॒ज्ञ्स्य॑ पाङ्क्त॒त्वम् पा᳚ङ्क्त॒त्वं ॅय॒ज्ञ्स्य॒ तत् तद् य॒ज्ञ्स्य॑ पाङ्क्त॒त्वम् । \newline
72. य॒ज्ञ्स्य॑ पाङ्क्त॒त्वम् पा᳚ङ्क्त॒त्वं ॅय॒ज्ञ्स्य॑ य॒ज्ञ्स्य॑ पाङ्क्त॒त्वम् । \newline
73. पा॒ङ्क्त॒त्वमिति॑ पाङ्क्त - त्वम् । \newline
\pagebreak


\end{document}