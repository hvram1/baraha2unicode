\documentclass[17pt]{extarticle}
\usepackage{babel}
\usepackage{fontspec}
\usepackage{polyglossia}
\usepackage{extsizes}



\setmainlanguage{sanskrit}
\setotherlanguages{english} %% or other languages
\setlength{\parindent}{0pt}
\pagestyle{myheadings}
\newfontfamily\devanagarifont[Script=Devanagari]{AdishilaVedic}


\newcommand{\VAR}[1]{}
\newcommand{\BLOCK}[1]{}




\begin{document}
\begin{titlepage}
    \begin{center}
 
\begin{sanskrit}
    { \Huge
    कृष्ण यजुर्वेदीय तैत्तिरीय संहिता,पद,जटा,घन पाठः 
    }
    \\
    \vspace{2.5cm}
    \mbox{ \Huge
    3.4      तृतीयकाण्डे चतुर्थः प्रश्नः - इष्टिहोमाभिधानं   }
\end{sanskrit}
\end{center}

\end{titlepage}
\tableofcontents
\pagebreak

\markright{ TS 3.4.1.1  \hfill https://www.vedavms.in \hfill}
\addcontentsline{toc}{section}{ TS 3.4.1.1 }
\section*{ TS 3.4.1.1 }

\textbf{TS 3.4.1.1 } \newline
\textbf{Samhita Paata} \newline

वि वा ए॒तस्य॑ य॒ज्ञ् ऋ॑द्ध्यते॒ यस्य॑ ह॒विर॑ति॒रिच्य॑ते॒ सूर्यो॑ दे॒वो दि॑वि॒षद्भ्य॒ इत्या॑ह॒ बृह॒स्पति॑ना चै॒वास्य॑ प्र॒जाप॑तिना च य॒ज्ञ्स्य॒ व्यृ॑द्ध॒मपि॑ वपति॒ रक्षाꣳ॑सि॒ वा ए॒तत् प॒शुꣳ स॑चन्ते॒ यदे॑कदेव॒त्य॑ आल॑ब्धो॒ भूया॒न् भव॑ति॒ यस्या᳚स्ते॒ हरि॑तो॒ गर्भ॒ इत्या॑ह देव॒त्रैवैनां᳚ गमयति॒ रक्ष॑सा॒मप॑हत्या॒ आ व॑र्तन वर्त॒येत्या॑ह॒ - [  ] \newline

\textbf{Pada Paata} \newline

वीति॑ । वै । ए॒तस्य॑ । य॒ज्ञ्ः । ऋ॒द्ध्य॒ते॒ । यस्य॑ । ह॒विः । अ॒ति॒रिच्य॑त॒ इत्य॑ति - रिच्य॑ते । सूर्यः॑ । दे॒वः । दि॒वि॒षद्भ्य॒ इति॑ दिवि॒षत् - भ्यः॒ । इति॑ । आ॒ह॒ । बृह॒स्पति॑ना । च॒ । ए॒व । अ॒स्य॒ । प्र॒जाप॑ति॒नेति॑ प्र॒जा - प॒ति॒ना॒ । च॒ । य॒ज्ञ्स्य॑ । व्यृ॑द्ध॒मिति॒ वि - ऋ॒द्ध॒म् । अपीति॑ । व॒प॒ति॒ । रक्षाꣳ॑सि । वै । ए॒तत् । प॒शुम् । स॒च॒न्ते॒ । यत् । ए॒क॒दे॒व॒त्य॑ इत्ये॑क - दे॒व॒त्यः॑ । आल॑ब्ध॒ इत्या - ल॒ब्धः॒ । भूयान्॑ । भव॑ति । यस्याः᳚ । ते॒ । हरि॑तः । गर्भः॑ । इति॑ । आ॒ह॒ । दे॒व॒त्रेति॑ देव - त्रा । ए॒व । ए॒ना॒म् । ग॒म॒य॒ति॒ । रक्ष॑साम् । अप॑हत्या॒ इत्यप॑ - ह॒त्यै॒ । एति॑ । व॒र्त॒न॒ । व॒र्त॒य॒ । इति॑ । आ॒ह॒ ।  \newline




\markright{ TS 3.4.1.2  \hfill https://www.vedavms.in \hfill}
\addcontentsline{toc}{section}{ TS 3.4.1.2 }
\section*{ TS 3.4.1.2 }

\textbf{TS 3.4.1.2 } \newline
\textbf{Samhita Paata} \newline

ब्रह्म॑णै॒वैन॒मा व॑र्तयति॒ वि ते॑ भिनद्मि तक॒रीमित्या॑ह यथाय॒जुरे॒वैतदु॑- रुद्र॒फ्सो वि॒श्वरू॑प॒ इन्दु॒रित्या॑ह प्र॒जा वै प॒शव॒ इन्दुः॑ प्र॒जयै॒वैनं॑ प॒शुभिः॒ सम॑र्द्धयति॒ दिवं॒ ॅवै य॒ज्ञ्स्य॒ व्यृ॑द्धं गच्छति पृथि॒वीमति॑रिक्तं॒ तद्यन्न श॒मये॒दार्ति॒मार्च्छे॒द्-यज॑मानो म॒ही द्यौः पृ॑थि॒वीच॑ न॒ इत्या॑ - [  ] \newline

\textbf{Pada Paata} \newline

ब्रह्म॑णा । ए॒व । ए॒न॒म् । एति॑ । व॒र्त॒य॒ति॒ । वीति॑ । ते॒ । भि॒न॒द्मि॒ । त॒क॒रीम् । इति॑ । आ॒ह॒ । य॒था॒य॒जुरिति॑ यथा - य॒जुः । ए॒व । ए॒तत् । उ॒रु॒द्र॒फ्स इत्यु॑रु - द्र॒फ्सः । वि॒श्वरू॑प॒ इति॑ वि॒श्व - रू॒पः॒ । इन्दुः॑ । इति॑ । आ॒ह॒ । प्र॒जेति॑ प्र - जा । वै । प॒शवः॑ । इन्दुः॑ । प्र॒जयेति॑ प्र - जया᳚ । ए॒व । ए॒न॒म् । प॒शुभि॒रिति॑ प॒शु - भिः॒ । समिति॑ । अ॒र्द्ध॒य॒ति॒ । दिव᳚म् । वै । य॒ज्ञ्स्य॑ । व्यृ॑द्ध॒मिति॒ वि-ऋ॒द्ध॒म् । ग॒च्छ॒ति॒ । पृ॒थि॒वीम् । अति॑रिक्त॒मित्यति॑ - रि॒क्त॒म् । तत् । यत् । न । श॒मये᳚त् । आर्ति᳚म् । एति॑ । ऋ॒च्छे॒त् । यज॑मानः । म॒ही । द्यौः । पृ॒थि॒वी । च॒ । नः॒ । इति॑ ।  \newline




\markright{ TS 3.4.1.3  \hfill https://www.vedavms.in \hfill}
\addcontentsline{toc}{section}{ TS 3.4.1.3 }
\section*{ TS 3.4.1.3 }

\textbf{TS 3.4.1.3 } \newline
\textbf{Samhita Paata} \newline

ह॒ द्यावा॑पृथि॒वीभ्या॑मे॒व य॒ज्ञ्स्य॒ व्यृ॑द्धं॒ चाति॑रिक्तं च शमयति॒ नाऽऽ*र्ति॒मार्च्छ॑ति॒ यज॑मानो॒ भस्म॑ना॒ऽभि समू॑हति स्व॒गाकृ॑त्या॒ अथो॑ अ॒नयो॒र्वा ए॒ष गर्भो॒ऽनयो॑रे॒वैनं॑ दधाति॒ यद॑व॒द्येदति॒ तद्रे॑चये॒द्यन्नाव॒द्येत् प॒शोराल॑ब्धस्य॒ नाव॑ द्येत् पु॒रस्ता॒न्नाभ्या॑ अ॒न्यद॑व॒द्ये-दु॒परि॑ष्टाद॒न्यत् पु॒रस्ता॒द्वै नाभ्यै᳚ - [  ] \newline

\textbf{Pada Paata} \newline

आ॒ह॒ । द्यावा॑पृथि॒वीभ्या॒मिति॒ द्यावा᳚ - पृ॒थि॒वीभ्या᳚म् । ए॒व । य॒ज्ञ्स्य॑ । व्यृ॑द्ध॒मिति॒ वि - ऋ॒द्ध॒म् । च॒ । अति॑रिक्त॒मित्यति॑ - रि॒क्त॒म् । च॒ । श॒म॒य॒ति॒ । न । आर्ति᳚म् । एति॑ । ऋ॒च्छ॒ति॒ । यज॑मानः । भस्म॑ना । अ॒भि । समिति॑ । ऊ॒ह॒ति॒ । स्व॒गाकृ॑त्या॒ इति॑ स्व॒गा - कृ॒त्यै॒ । अथो॒ इति॑ । अ॒नयोः᳚ । वै । ए॒षः । गर्भः॑ । अ॒नयोः᳚ । ए॒व । ए॒न॒म् । द॒धा॒ति॒ । यत् । अ॒व॒द्येदित्य॑व-द्येत् । अतीति॑ । तत् । रे॒च॒ये॒त् । यत् । न । अ॒व॒द्येदित्य॑व - द्येत् । प॒शोः । आल॑ब्ध॒स्येत्या-ल॒ब्ध॒स्य॒ । न । अवेति॑ । द्ये॒त् । पु॒रस्ता᳚त् । नाभ्याः᳚ । अ॒न्यत् । अ॒व॒द्येदित्य॑व - द्येत् । उ॒परि॑ष्टात् । अ॒न्यत् । पु॒रस्ता᳚त् । वै । नाभ्यै᳚ ।  \newline




\markright{ TS 3.4.1.4  \hfill https://www.vedavms.in \hfill}
\addcontentsline{toc}{section}{ TS 3.4.1.4 }
\section*{ TS 3.4.1.4 }

\textbf{TS 3.4.1.4 } \newline
\textbf{Samhita Paata} \newline

प्रा॒ण उ॒परि॑ष्टादपा॒नो यावा॑ने॒व प॒शुस्तस्याव॑ द्यति॒ विष्ण॑वे शिपिवि॒ष्टाय॑ जुहोति॒ यद्वै य॒ज्ञ्स्या॑ति॒रिच्य॑ते॒ यः प॒शोर्भू॒मा या पुष्टि॒स्तद्-विष्णुः॑ शिपिवि॒ष्टो ऽति॑रिक्त ए॒वाति॑रिक्तं दधा॒त्यति॑रिक्तस्य॒ शान्त्या॑ अ॒ष्टाप्रू॒ड्ढिर॑ण्यं॒ दक्षि॑णा॒ऽष्टाप॑दी॒ ह्ये॑षा ऽऽत्मा न॑व॒मः प॒शोराप्त्या॑ अन्तरको॒श उ॒ष्णीषे॒णाऽऽ*वि॑ष्टितं भवत्ये॒वमि॑व॒ हि प॒शुरुल्ब॑मिव॒ ( ) चर्मे॑व माꣳ॒॒समि॒वास्थी॑व॒ यावा॑ने॒व प॒शुस्तमा॒प्त्वाऽव॑ रुन्धे॒यस्यै॒षा य॒ज्ञे प्राय॑श्चित्तिः क्रि॒यत॑ इ॒ष्ट्वा वसी॑यान् भवति ॥ \newline

\textbf{Pada Paata} \newline

प्रा॒ण इति॑ प्र - अ॒नः । उ॒परि॑ष्टात् । अ॒पा॒न इत्य॑प-अ॒नः । यावान्॑ । ए॒व । प॒शुः । तस्य॑ । अवेति॑ । द्य॒ति॒ । विष्ण॑वे । शि॒पि॒वि॒ष्टायेति॑ शिपि - वि॒ष्टाय॑ । जु॒हो॒ति॒ । यत् । वै । य॒ज्ञ्स्य॑ । अ॒ति॒रिच्य॑त॒ इत्य॑ति - रिच्य॑ते । यः । प॒शोः । भू॒मा । या । पुष्टिः॑ । तत् । विष्णुः॑ । शि॒पि॒वि॒ष्ट इति॑ शिपि - वि॒ष्टः । अति॑रिक्त॒ इत्यति॑ - रि॒क्ते॒ । ए॒व । अति॑रिक्त॒मित्यति॑ - रि॒क्त॒म् । द॒धा॒ति॒ । अति॑रिक्त॒स्येत्यति॑ - रि॒क्त॒स्य॒ । शान्त्यै᳚ । अ॒ष्टाप्रू॒डित्य॒ष्टा - प्रू॒ट् । हिर॑ण्यम् । दक्षि॑णा । अ॒ष्टाप॒दीत्य॒ष्टा - प॒दी॒ । हि । ए॒षा । आ॒त्मा । न॒व॒मः । प॒शोः । आप्त्यै᳚ । अ॒न्त॒र॒को॒श इत्य॑न्तर - को॒शे । उ॒ष्णीषे॑ण । आवि॑ष्टित॒मित्या - वि॒ष्टि॒त॒म् । भ॒व॒ति॒ । ए॒वम् । इ॒व॒ । हि । प॒शुः । उल्ब᳚म् । इ॒व॒ ( ) । चर्म॑ । इ॒व॒ । माꣳ॒॒सम् । इ॒व॒ । अस्थि॑ । इ॒व॒ । यावान्॑ । ए॒व । प॒शुः । तम् । आ॒प्त्वा । अवेति॑ । रु॒न्धे॒ । यस्य॑ । ए॒षा । य॒ज्ञे । प्राय॑श्चित्तिः । क्रि॒यते᳚ । इ॒ष्ट्वा । वसी॑यान् । भ॒व॒ति॒ ॥  \newline




\markright{ TS 3.4.2.1  \hfill https://www.vedavms.in \hfill}
\addcontentsline{toc}{section}{ TS 3.4.2.1 }
\section*{ TS 3.4.2.1 }

\textbf{TS 3.4.2.1 } \newline
\textbf{Samhita Paata} \newline

आ वा॑यो भूष शुचिपा॒ उप॑ नः स॒हस्रं॑ ते नि॒युतो॑ विश्ववार । उपो॑ ते॒ अन्धो॒ मद्य॑मयामि॒ यस्य॑ देव दधि॒षे पू᳚र्व॒पेयं᳚ ॥ आकू᳚त्यै त्वा॒ कामा॑य त्वा स॒मृधे᳚ त्वा किक्कि॒टा ते॒ मनः॑ प्र॒जाप॑तये॒ स्वाहा॑ किक्कि॒टा ते᳚ प्रा॒णं ॅवा॒यवे॒ स्वाहा॑ किक्कि॒टा ते॒ चक्षुः॒ सूर्या॑य॒ स्वाहा॑ किक्कि॒टा ते॒ श्रोत्रं॒ द्यावा॑पृथि॒वीभ्याꣳ॒॒ स्वाहा॑ किक्कि॒टा ते॒ वाचꣳ॒॒ सर॑स्वत्यै॒ स्वाहा॒- [  ] \newline

\textbf{Pada Paata} \newline

एति॑ । वा॒यो॒ इति॑ । भू॒ष॒ । शु॒चि॒पा॒ इति॑ शुचि - पाः॒ । उपेति॑ । नः॒ । स॒हस्र᳚म् । ते॒ । नि॒युत॒ इति॑ नि - युतः॑ । वि॒श्व॒वा॒रेति॑ विश्व - वा॒र॒ ॥ उपो॒ इति॑ । ते॒ । अन्धः॑ । मद्य᳚म् । अ॒या॒मि॒ । यस्य॑ । दे॒व॒ । द॒धि॒षे । पू॒र्व॒पेय॒मिति॑ पूर्व - पेय᳚म् ॥ आकू᳚त्या॒ इत्या-कू॒त्यै॒ । त्वा॒ । कामा॑य । त्वा॒ । स॒मृध॒ इति॑ सं - ऋधे᳚ । त्वा॒ । कि॒क्कि॒टा । ते॒ । मनः॑ । प्र॒जाप॑तय॒ इति॑ प्र॒जा - प॒त॒ये॒ । स्वाहा᳚ । कि॒क्कि॒टा । ते॒ । प्रा॒णमिति॑ प्र - अ॒नम् । वा॒यवे᳚ । स्वाहा᳚ । कि॒क्कि॒टा । ते॒ । चक्षुः॑ । सूर्या॑य । स्वाहा᳚ । कि॒क्कि॒टा । ते॒ । श्रोत्र᳚म् । द्यावा॑पृथि॒वीभ्या॒मिति॒ द्यावा᳚-पृ॒थि॒वीभ्या᳚म् । स्वाहा᳚ । कि॒क्कि॒टा । ते॒ । वाच᳚म् । सर॑स्वत्यै । स्वाहा᳚ ।  \newline




\markright{ TS 3.4.2.2  \hfill https://www.vedavms.in \hfill}
\addcontentsline{toc}{section}{ TS 3.4.2.2 }
\section*{ TS 3.4.2.2 }

\textbf{TS 3.4.2.2 } \newline
\textbf{Samhita Paata} \newline

त्वं तु॒रीया॑ व॒शिनी॑ व॒शाऽसि॑ स॒कृद्यत् त्वा॒ मन॑सा॒ गर्भ॒ आऽश॑यत् । व॒शा त्वं ॅव॒शिनी॑ गच्छ दे॒वान्थ्-स॒त्याः स॑न्तु॒ यज॑मानस्य॒ कामाः᳚ ॥ अ॒जाऽसि॑ रयि॒ष्ठा पृ॑थि॒व्याꣳ सी॑दो॒र्द्ध्वाऽन्तरि॑क्ष॒मुप॑ तिष्ठस्व दि॒वि ते॑ बृ॒हद्भाः ॥ तन्तुं॑ त॒न्वन् रज॑सो भा॒नुमन्वि॑हि॒ ज्योति॑ष्मतः प॒थो र॑क्ष धि॒या कृ॒तान् ॥ अ॒नु॒ल्ब॒णं ॅव॑यत॒ जोगु॑वा॒मपो॒ मनु॑ ( ) र्भव ज॒नया॒ दैव्यं॒ जनं᳚ ॥ मन॑सो ह॒विर॑सि प्र॒जाप॑ते॒र्वर्णो॒ गात्रा॑णां ते गात्र॒भाजो॑ भूयास्म ॥ \newline

\textbf{Pada Paata} \newline

त्वम् । तु॒रीया᳚ । व॒शिनी᳚ । व॒शा । अ॒सि॒ । स॒कृत् । यत् । त्वा॒ । मन॑सा । गर्भः॑ । एति॑ । अश॑यत् ॥ व॒शा । त्वम् । व॒शिनी᳚ । ग॒च्छ॒ । दे॒वान् । स॒त्याः । स॒न्तु॒ । यज॑मानस्य । कामाः᳚ ॥ अ॒जा । अ॒सि॒ । र॒यि॒ष्ठेति॑ रयि - स्था । पृ॒थि॒व्याम् । सी॒द॒ । ऊ॒र्द्ध्वा । अ॒न्तरि॑क्षम् । उपेति॑ । ति॒ष्ठ॒स्व॒ । दि॒वि । ते॒ । बृ॒हत् । भाः ॥ तन्तु᳚म् । त॒न्वन्न् । रज॑सः । भा॒नुम् । अन्विति॑ । इ॒हि॒ । ज्योति॑ष्मतः । प॒थः । र॒क्ष॒ । धि॒या । कृ॒तान् ॥ अ॒नु॒ल्ब॒णम् । व॒य॒त॒ । जोगु॑वाम् । अपः॑ । मनुः॑( ) । भ॒व॒ । ज॒नय॑ । दैव्य᳚म् । जन᳚म् ॥ मन॑सः । ह॒विः । अ॒सि॒ । प्र॒जाप॑ते॒रिति॑ प्र॒जा - प॒तेः॒ । वर्णः॑ । गात्रा॑णाम् । ते॒ । गा॒त्र॒भाज॒ इति॑ गात्र - भाजः॑ । भू॒या॒स्म॒ ॥  \newline




\markright{ TS 3.4.3.1  \hfill https://www.vedavms.in \hfill}
\addcontentsline{toc}{section}{ TS 3.4.3.1 }
\section*{ TS 3.4.3.1 }

\textbf{TS 3.4.3.1 } \newline
\textbf{Samhita Paata} \newline

इ॒मे वै स॒हाऽऽ*स्तां॒ ते वा॒युर्व्य॑वा॒त् ते गर्भ॑मदधातां॒ तꣳ सोमः॒ प्राज॑नय-द॒ग्निर॑ग्रसत॒ स ए॒तं प्र॒जाप॑तिराग्ने॒य-म॒ष्टाक॑पालमपश्य॒त् तं निर॑वप॒त् तेनै॒वैना॑म॒ग्नेरधि॒ निर॑क्रीणा॒त् तस्मा॒दप्य॑न्यदेव॒त्या॑मा॒लभ॑मान आग्ने॒यम॒ष्टाक॑पालं पु॒रस्ता॒न्निर्व॑पेद॒ग्नेरे॒वैना॒मधि॑ नि॒ष्क्रीयाऽऽल॑भते॒ यद् - [  ] \newline

\textbf{Pada Paata} \newline

इ॒मे इति॑ । वै । स॒ह । आ॒स्ता॒म् । ते इति॑ । वा॒युः । वीति॑ । अ॒वा॒त् । ते इति॑ । गर्भ᳚म् । अ॒द॒धा॒ता॒म् । तम् । सोमः॑ । प्रेति॑ । अ॒ज॒न॒य॒त् । अ॒ग्निः । अ॒ग्र॒स॒त॒ । सः । ए॒तम् । प्र॒जाप॑ति॒रिति॑ प्र॒जा - प॒तिः॒ । आ॒ग्ने॒यम् । अ॒ष्टाक॑पाल॒मित्य॒ष्टा - क॒पा॒ल॒म् । अ॒प॒श्य॒त् । तम् । निरिति॑ । अ॒व॒प॒त् । तेन॑ । ए॒व । ए॒ना॒म् । अ॒ग्नेः । अधि॑ । निरिति॑ । अ॒क्री॒णा॒त् । तस्मा᳚त् । अपीति॑ । अ॒न्य॒दे॒व॒त्या॑मित्य॑न्य - दे॒व॒त्या᳚म् । आ॒लभ॑मान॒ इत्या᳚ - लभ॑मानः । आ॒ग्ने॒यम् । अ॒ष्टाक॑पाल॒मित्य॒ष्टा-क॒पा॒ल॒म् । पु॒रस्ता᳚त् । निरिति॑ । व॒पे॒त् । अ॒ग्नेः । ए॒व । ए॒ना॒म् । अधीति॑ । नि॒ष्क्रीयेति॑ निः - क्रीय॑ । एति॑ । ल॒भ॒ते॒ । यत् ।  \newline




\markright{ TS 3.4.3.2  \hfill https://www.vedavms.in \hfill}
\addcontentsline{toc}{section}{ TS 3.4.3.2 }
\section*{ TS 3.4.3.2 }

\textbf{TS 3.4.3.2 } \newline
\textbf{Samhita Paata} \newline

वा॒युर्व्यवा॒त् तस्मा᳚द्-वाय॒व्या॑ यदि॒मे गर्भ॒मद॑धातां॒ तस्मा᳚द्-द्यावापृथि॒व्या॑ यथ् सोमः॒ प्राज॑नयद॒ग्निरग्र॑सत॒ तस्मा॑दग्नीषो॒मीया॒ यद॒नयो᳚र्विय॒त्योर्-वागव॑द॒त् तस्मा᳚थ् सारस्व॒ती यत् प्र॒जाप॑तिर॒ग्नेरधि॑ नि॒रक्री॑णा॒त् तस्मा᳚त् प्राजाप॒त्या सा वा ए॒षा स॑र्वदेव॒त्या॑ यद॒जा व॒शा वा॑य॒व्या॑मा ल॑भेत॒ भूति॑कामो वा॒युर्वै क्षेपि॑ष्ठा दे॒वता॑ वा॒युमे॒व स्वेन॑- [  ] \newline

\textbf{Pada Paata} \newline

वा॒युः । व्यवा॒दिति॑ वि - अवा᳚त् । तस्मा᳚त् । वा॒य॒व्या᳚ । यत् । इ॒मे इति॑ । गर्भ᳚म् । अद॑धाताम् । तस्मा᳚त् । द्या॒वा॒पृ॒थि॒व्येति॑ द्यावा - पृ॒थि॒व्या᳚ । यत् । सोमः॑ । प्रेति॑ । अज॑नयत् । अ॒ग्निः । अग्र॑सत । तस्मा᳚त् । अ॒ग्नी॒षो॒मीयेत्य॑ग्नी-सो॒मीया᳚ । यत् । अ॒नयोः᳚ । वि॒य॒त्योरिति॑ वि - य॒त्योः । वाक् । अव॑दत् । तस्मा᳚त् । सा॒र॒स्व॒ती । यत् । प्र॒जाप॑ति॒रिति॑ प्र॒जा - प॒तिः॒ । अ॒ग्नेः । अधीति॑ । नि॒रक्री॑णा॒दिति॑ निः - अक्री॑णात् । तस्मा᳚त् । प्रा॒जा॒प॒त्येति॑ प्राजा - प॒त्या । सा । वै । ए॒षा । स॒र्व॒दे॒व॒त्येति॑ सर्व - दे॒व॒त्या᳚ । यत् । अ॒जा । व॒शा । वा॒य॒व्या᳚म् । एति॑ । ल॒भे॒त॒ । भूति॑काम॒ इति॒ भूति॑ - का॒मः॒ । वा॒युः । वै । क्षेपि॑ष्ठा । दे॒वता᳚ । वा॒युम् । ए॒व । स्वेन॑ ।  \newline




\markright{ TS 3.4.3.3  \hfill https://www.vedavms.in \hfill}
\addcontentsline{toc}{section}{ TS 3.4.3.3 }
\section*{ TS 3.4.3.3 }

\textbf{TS 3.4.3.3 } \newline
\textbf{Samhita Paata} \newline

भाग॒धेये॒नोप॑ धावति॒ स ए॒वैनं॒ भूतिं॑ गमयति द्यावापृथि॒व्या॑मा ल॑भेत कृ॒षमा॑णः प्रति॒ष्ठाका॑मो दि॒व ए॒वास्मै॑ प॒र्जन्यो॑ वर्.षति॒ व्य॑स्यामोष॑धयो रोहन्ति स॒मर्द्धु॑कमस्य स॒स्यं भ॑वत्यग्नीषो॒मीया॒मा ल॑भेत॒ यः का॒मये॒तान्न॑वानन्ना॒दः स्या॒मित्य॒ग्निनै॒वान्न॒मव॑ रुन्धे॒ सोमे॑ना॒न्नाद्य॒-मन्न॑वाने॒वान्ना॒दो भ॑वति सारस्व॒तीमा ल॑भेत॒ य - [  ] \newline

\textbf{Pada Paata} \newline

भा॒ग॒धेये॒नेति॑ भाग - धेये॑न । उपेति॑ । धा॒व॒ति॒ । सः । ए॒व । ए॒न॒म् । भूति᳚म् । ग॒म॒य॒ति॒ । द्या॒वा॒पृ॒थि॒व्या॑मिति॑ द्यावा - पृ॒थि॒व्या᳚म् । एति॑ । ल॒भे॒त॒ । कृ॒षमा॑णः । प्र॒ति॒ष्ठाका॑म॒ इति॑ प्रति॒ष्ठा-का॒मः॒ । दि॒वः । ए॒व । अ॒स्मै॒ । प॒र्जन्यः॑ । व॒र्॒.ष॒ति॒ । वीति॑ । अ॒स्याम् । ओष॑धयः । रो॒ह॒न्ति॒ । स॒मर्द्धु॑क॒मिति॑ सं - अर्द्धु॑कम् । अ॒स्य॒ । स॒स्यम् । भ॒व॒ति॒ । अ॒ग्नी॒षो॒मीया॒मित्य॑ग्नी-सो॒मीया᳚म् । एति॑ । ल॒भे॒त॒ । यः । का॒मये॑त । अन्न॑वा॒न्नित्यन्न॑ - वा॒न् । अ॒न्ना॒द इत्य॑न्न - अ॒दः । स्या॒म् । इति॑ । अ॒ग्निना᳚ । ए॒व । अन्न᳚म् । अवेति॑ । रु॒न्धे॒ । सोमे॑न । अ॒न्नाद्य॒मित्य॑न्न - अद्य᳚म् । अन्न॑वा॒न्नित्यन्न॑ - वा॒न् । ए॒व । अ॒न्ना॒द इत्य॑न्न - अ॒दः । भ॒व॒ति॒ । सा॒र॒स्व॒तीम् । एति॑ । ल॒भे॒त॒ । यः ।  \newline




\markright{ TS 3.4.3.4  \hfill https://www.vedavms.in \hfill}
\addcontentsline{toc}{section}{ TS 3.4.3.4 }
\section*{ TS 3.4.3.4 }

\textbf{TS 3.4.3.4 } \newline
\textbf{Samhita Paata} \newline

ई᳚श्व॒रो वा॒चो वदि॑तोः॒ सन्. वाचं॒ नवदे॒द्-वाग्वै सर॑स्वती॒ सर॑स्वतीमे॒व स्वेन॑ भाग॒धेये॒नोप॑ धावति॒ सैवास्मि॒न्. वाचं॑ दधाति प्राजाप॒त्यामा ल॑भेत॒ यः का॒मये॒तान॑भिजितम॒भि ज॑येय॒मिति॑ प्र॒जाप॑तिः॒ सर्वा॑ दे॒वता॑ दे॒वता॑भिरे॒वा-न॑भिजितम॒भि ज॑यति वाय॒व्य॑यो॒पाक॑रोति वा॒योरे॒वैना॑मव॒रुद्ध्याऽऽ*ल॑भत॒ आकू᳚त्यै त्वा॒ कामा॑य॒ त्वे - [  ] \newline

\textbf{Pada Paata} \newline

ई॒श्व॒रः । वा॒चः । वदि॑तोः । सन्न् । वाच᳚म् । न । वदे᳚त् । वाक् । वै । सर॑स्वती । सर॑स्वतीम् । ए॒व । स्वेन॑ । भा॒ग॒धेये॒नेति॑ भाग - धेये॑न । उपेति॑ । धा॒व॒ति॒ । सा । ए॒व । अ॒स्मि॒न्न् । वाच᳚म् । द॒धा॒ति॒ । प्रा॒जा॒प॒त्यामिति॑ प्राजा - प॒त्याम् । एति॑ । ल॒भे॒त॒ । यः । का॒मये॑त । अन॑भिजित॒मित्यन॑भि - जि॒त॒म् । अ॒भीति॑ । ज॒ये॒य॒म् । इति॑ । प्र॒जाप॑ति॒रिति॑ प्र॒जा - प॒तिः॒ । सर्वाः᳚ । दे॒वताः᳚ । दे॒वता॑भिः । ए॒व । अन॑भिजित॒मित्यन॑भि - जि॒त॒म् । अ॒भीति॑ । ज॒य॒ति॒ । वा॒य॒व्य॑या । उ॒पाक॑रो॒तीयु॑प - आक॑रोति । वा॒योः । ए॒व । ए॒ना॒म् । अ॒व॒रुद्ध्येत्य॑व - रुद्ध्य॑ । एति॑ । ल॒भ॒ते॒ । आकू᳚त्या॒ इत्या - कू॒त्यै॒ । त्वा॒ । कामा॑य । त्वा॒ ।  \newline




\markright{ TS 3.4.3.5  \hfill https://www.vedavms.in \hfill}
\addcontentsline{toc}{section}{ TS 3.4.3.5 }
\section*{ TS 3.4.3.5 }

\textbf{TS 3.4.3.5 } \newline
\textbf{Samhita Paata} \newline

त्या॑ह यथाय॒जुरे॒वैतत् कि॑क्किटा॒कारं॑ जुहोति किक्किटाका॒रेण॒ वै ग्रा॒म्याः प॒शवो॑ रमन्ते॒ प्राऽऽ*र॒ण्याः प॑तन्ति॒ यत् कि॑क्किटा॒कारं॑ जु॒होति॑ ग्रा॒म्याणां᳚ पशू॒नां धृत्यै॒ पर्य॑ग्नौ क्रि॒यमा॑णे जुहोति॒ जीव॑न्तीमे॒वैनाꣳ॑ सुव॒र्गं ॅलो॒कं ग॑मयति॒ त्वं तु॒रीया॑ व॒शिनी॑ व॒शाऽसीत्या॑ह देव॒त्रैवैनां᳚ गमयति स॒त्याः स॑न्तु॒ यज॑मानस्य॒ कामा॒ इत्या॑है॒ष वै कामो॒ -  [  ] \newline

\textbf{Pada Paata} \newline

इति॑ । आ॒ह॒ । य॒था॒य॒जुरिति॑ यथा - य॒जुः । ए॒व । ए॒तत् । कि॒क्कि॒टा॒कार॒मिति॑ किक्किटा - कार᳚म् । जु॒हो॒ति॒ । कि॒क्कि॒टा॒का॒रेणेति॑ किक्किटा - का॒रेण॑ । वै । ग्रा॒म्याः । प॒शवः॑ । र॒म॒न्ते॒ । प्रेति॑ । आ॒र॒ण्याः । प॒त॒न्ति॒ । यत् । कि॒क्कि॒टा॒कार॒मिति॑ किक्किटा - कार᳚म् । जु॒होति॑ । ग्रा॒म्याणा᳚म् । प॒शू॒नाम् । धृत्यै᳚ । पर्य॑ग्ना॒विति॒ परि॑-अ॒ग्नौ॒ । क्रि॒यमा॑णे । जु॒हो॒ति॒ । जीव॑न्तीम् । ए॒व । ए॒ना॒म् । सु॒व॒र्गमिति॑ सुवः - गम् । लो॒कम् । ग॒म॒य॒ति॒ । त्वम् । तु॒रीया᳚ । व॒शिनी᳚ । व॒शा । अ॒सि॒ । इति॑ । आ॒ह॒ । दे॒व॒त्रेति॑ देव - त्रा । ए॒व । ए॒ना॒म् । ग॒म॒य॒ति॒ । स॒त्याः । स॒न्तु॒ । यज॑मानस्य । कामाः᳚ । इति॑ । आ॒ह॒ । ए॒षः । वै । कामः॑ ।  \newline




\markright{ TS 3.4.3.6  \hfill https://www.vedavms.in \hfill}
\addcontentsline{toc}{section}{ TS 3.4.3.6 }
\section*{ TS 3.4.3.6 }

\textbf{TS 3.4.3.6 } \newline
\textbf{Samhita Paata} \newline

यज॑मानस्य॒ यदना᳚र्त उ॒दृचं॒ गच्छ॑ति॒ तस्मा॑दे॒वमा॑हा॒ऽजाऽसि॑ रयि॒ष्ठेत्या॑है॒ ष्वे॑वैनां᳚ ॅलो॒केषु॒ प्रति॑ष्ठापयति दि॒वि ते॑ बृ॒हद्भा इत्या॑ह सुव॒र्ग ए॒वास्मै॑ लो॒के ज्योति॑र् दधाति॒ तन्तुं॑ त॒न्वन् रज॑सो भा॒नुमन्वि॒हीत्या॑हे॒माने॒वास्मै॑ लो॒कान् ज्योति॑ष्मतः करोत्यनुल्ब॒णं ॅव॑यत॒ जोगु॑वा॒मप॒ इत्या॑-  [  ] \newline

\textbf{Pada Paata} \newline

यज॑मानस्य । यत् । अना᳚र्तः । उ॒दृच॒मित्यु॑त् - ऋच᳚म् । गच्छ॑ति । तस्मा᳚त् । ए॒वम् । आ॒ह॒ । अ॒जा । अ॒सि॒ । र॒यि॒ष्ठेति॑ रयि - स्था । इति॑ । आ॒ह॒ । ए॒षु । ए॒व । ए॒ना॒म् । लो॒केषु॑ । प्रतीति॑ । स्था॒प॒य॒ति॒ । दि॒वि । ते॒ । बृ॒हत् । भाः । इति॑ । आ॒ह॒ । सु॒व॒र्ग इति॑ सुवः- गे । ए॒व । अ॒स्मै॒ । लो॒के । ज्योतिः॑ । द॒धा॒ति॒ । तन्तु᳚म् । त॒न्वन्न् । रज॑सः । भा॒नुम् । अन्विति॑ । इ॒हि॒ । इति॑ । आ॒ह॒ । इ॒मान् । ए॒व । अ॒स्मै॒ । लो॒कान् । ज्योति॑ष्मतः । क॒रो॒ति॒ । अ॒नु॒ल्ब॒णम् । व॒य॒त॒ । जोगु॑वाम् । अपः॑ । इति॑ ।  \newline




\markright{ TS 3.4.3.7  \hfill https://www.vedavms.in \hfill}
\addcontentsline{toc}{section}{ TS 3.4.3.7 }
\section*{ TS 3.4.3.7 }

\textbf{TS 3.4.3.7 } \newline
\textbf{Samhita Paata} \newline

-ह॒ यदे॒व य॒ज्ञ् उ॒ल्बणं॑ क्रि॒यते॒ तस्यै॒वैषा शान्ति॒र्मनु॑र्भव ज॒नया॒ दैव्यं॒ जन॒मित्या॑ह मान॒व्यो॑ वै प्र॒जास्ता ए॒वाऽऽ*द्याः᳚ कुरुते॒ मन॑सो ह॒विर॒सीत्या॑ह स्व॒गाकृ॑त्यै॒ गात्रा॑णां ते गात्र॒भाजो॑ भूया॒स्मेत्या॑हा॒ ऽऽ*शिष॑मे॒वैतामा शा᳚स्ते॒ तस्यै॒ वा ए॒तस्या॒ एक॑मे॒वा-दे॑वयजनं॒ ॅयदाल॑ब्धायाम॒भ्रो - [  ] \newline

\textbf{Pada Paata} \newline

आ॒ह॒ । यत् । ए॒व । य॒ज्ञे । उ॒ल्बण᳚म् । क्रि॒यते᳚ । तस्य॑ । ए॒व । ए॒षा । शान्तिः॑ । मनुः॑ । भ॒व॒ । ज॒नय॑ । दैव्य᳚म् । जन᳚म् । इति॑ । आ॒ह॒ । मा॒न॒व्यः॑ । वै । प्र॒जा इति॑ प्र - जाः । ताः । ए॒व । आ॒द्याः᳚ । कु॒रु॒ते॒ । मन॑सः । ह॒विः । अ॒सि॒ । इति॑ । आ॒ह॒ । स्व॒गाकृ॑त्या॒ इति॑ स्व॒गा - कृ॒त्यै॒ । गात्रा॑णाम् । ते॒ । गा॒त्र॒भाज॒ इति॑ गात्र - भाजः॑ । भू॒या॒स्म॒ । इति॑ । आ॒ह॒ । आ॒शिष॒मित्या᳚ - शिष᳚म् । ए॒व । ए॒ताम् । एति॑ । शा॒स्ते॒ । तस्यै᳚ । वै । ए॒तस्याः᳚ । एक᳚म् । ए॒व । अदे॑वयजन॒मित्यदे॑व - य॒ज॒न॒म् । यत् । आल॑ब्धाया॒मित्या - ल॒ब्धा॒या॒म् । अ॒भ्रः ।  \newline




\markright{ TS 3.4.3.8  \hfill https://www.vedavms.in \hfill}
\addcontentsline{toc}{section}{ TS 3.4.3.8 }
\section*{ TS 3.4.3.8 }

\textbf{TS 3.4.3.8 } \newline
\textbf{Samhita Paata} \newline

भव॑ति॒ यदाल॑ब्धायाम॒भ्रः स्याद॒फ्सु वा᳚प्रवे॒शये॒थ् सर्वां᳚ ॅवा॒ प्राश्ञी॑या॒द्यद॒फ्सु प्र॑वे॒शये᳚द्यज्ञ्वेश॒सं कु॑र्या॒थ् सर्वा॑मे॒व प्राश्ञी॑यादिन्द्रि॒यमे॒वाऽऽ*त्मन् ध॑त् ते॒ सा वा ए॒षा त्र॑या॒णामे॒वाव॑ रुद्धा संॅवथ्सर॒सदः॑ सहस्रया॒जिनो॑ गृहमे॒धिन॒स्त ए॒वैतया॑ यजेर॒न् तेषा॑मे॒वैषाऽऽप्ता ॥ \newline

\textbf{Pada Paata} \newline

भव॑ति । यत् । आल॑ब्धाया॒मित्या - ल॒ब्धा॒या॒म् । अ॒भ्रः । स्यात् । अ॒फ्स्वित्य॑प्-सु । वा॒ । प्र॒वे॒शये॒दिति॑ प्र - वे॒शये᳚त् । सर्वा᳚म् । वा॒ । प्रेति॑ । अ॒श्नी॒या॒त् । यत् । अ॒फ्स्वित्य॑प् - सु । प्र॒वे॒शये॒दिति॑ प्र - वे॒शये᳚त् । य॒ज्ञ्॒वे॒श॒समिति॑ यज्ञ् - वे॒श॒सम् । कु॒र्या॒त् । सर्वा᳚म् । ए॒व । प्रेति॑ । अ॒श्नी॒या॒त् । इ॒न्द्रि॒यम् । ए॒व । आ॒त्मन् । ध॒त्ते॒ । सा । वै । ए॒षा । त्र॒या॒णाम् । ए॒व । अव॑रु॒द्धेत्यव॑ - रु॒द्धा॒ । सं॒ॅव॒थ्स॒र॒सद॒ इति॑ संॅवथ्सर - सदः॑ । स॒ह॒स्र॒या॒जिन॒ इति॑ सहस्र - या॒जिनः॑ । गृ॒ह॒मे॒धिन॒ इति॑ गृह - मे॒धिनः॑ । ते । ए॒व । ए॒तया᳚ । य॒जे॒र॒न्न् । तेषा᳚म् । ए॒व । ए॒षा । आ॒प्ता ॥  \newline




\markright{ TS 3.4.4.1  \hfill https://www.vedavms.in \hfill}
\addcontentsline{toc}{section}{ TS 3.4.4.1 }
\section*{ TS 3.4.4.1 }

\textbf{TS 3.4.4.1 } \newline
\textbf{Samhita Paata} \newline

चि॒त्तं च॒ चित्ति॒श्चा ऽऽ*कू॑तं॒ चाऽऽ*कू॑तिश्च॒ विज्ञा॑तं च वि॒ज्ञानं॑ च॒ मन॑श्च॒ शक्व॑रीश्च॒ दर्.श॑श्च पू॒र्णमा॑सश्च बृ॒हच्च॑ रथन्त॒रं च॑ प्र॒जाप॑ति॒र्जया॒निन्द्रा॑य॒ वृष्णे॒ प्राय॑च्छदु॒ग्रः पृ॑त॒नाज्ये॑षु॒ तस्मै॒ विशः॒ सम॑नमन्त॒ सर्वाः॒ स उ॒ग्रः सहि हव्यो॑ ब॒भूव॑देवासु॒राः संॅय॑त्ता आस॒न्थ्स इन्द्रः॑ प्र॒जाप॑ति॒मुपा॑ ( ) धाव॒त् तस्मा॑ ए॒ताञ्जया॒न् प्राय॑च्छ॒त् तान॑जुहो॒त् ततो॒ वै दे॒वा असु॑रानजय॒न॒. यदज॑य॒न् तज्जया॑नां जय॒त्वꣳ स्पर्द्ध॑मानेनै॒ते हो॑त॒व्या॑ जय॑त्ये॒व तां पृत॑नां ॥ \newline

\textbf{Pada Paata} \newline

चि॒त्तम् । च॒ । चित्तिः॑ । च॒ । आकू॑त॒मित्या - कू॒त॒म् । च॒ । आकू॑ति॒रित्या - कू॒तिः॒ । च॒ । विज्ञा॑त॒मिति॒ वि - ज्ञा॒त॒म् । च॒ । वि॒ज्ञान॒मिति॑ वि-ज्ञान᳚म् । च॒ । मनः॑ । च॒ । शक्व॑रीः । च॒ । दर्.शः॑ । च॒ । पू॒र्णमा॑स॒ इति॑ पू॒र्ण - मा॒सः॒ । च॒ । बृ॒हत् । च॒ । र॒थ॒न्त॒रमिति॑ रथं - त॒रम् । च॒ । प्र॒जाप॑ति॒रिति॑ प्र॒जा - प॒तिः॒ । जयान्॑ । इन्द्रा॑य । वृष्णे᳚ । प्रेति॑ । अ॒य॒च्छ॒त् । उ॒ग्रः । पृ॒त॒नाज्ये॑षु । तस्मै᳚ । विशः॑ । समिति॑ । अ॒न॒म॒न्त॒ । सर्वाः᳚ । सः । उ॒ग्रः । सः । हि । हव्यः॑ । ब॒भूव॑ । दे॒वा॒सु॒रा इति॑ देव - अ॒सु॒राः । संॅय॑त्ता॒ इति॒ सं - य॒त्ताः॒ । आ॒स॒न्न् । सः । इन्द्रः॑ । प्र॒जाप॑ति॒मिति॑ प्र॒जा - प॒ति॒म् । उपेति॑ ( ) । अ॒धा॒व॒त् । तस्मै᳚ । ए॒तान् । जयान्॑ । प्रेति॑ । अ॒य॒च्छ॒त् । तान् । अ॒जु॒हो॒त् । ततः॑ । वै । दे॒वाः । असु॑रान् । अ॒ज॒य॒न्न् । यत् । अज॑यन्न् । तत् । जया॑नाम् । ज॒य॒त्वमिति॑ जय - त्वम् । स्पर्द्ध॑मानेन । ए॒ते । हो॒त॒व्याः᳚ । जय॑ति । ए॒व । ताम् । पृत॑नाम् ॥  \newline




\markright{ TS 3.4.5.1  \hfill https://www.vedavms.in \hfill}
\addcontentsline{toc}{section}{ TS 3.4.5.1 }
\section*{ TS 3.4.5.1 }

\textbf{TS 3.4.5.1 } \newline
\textbf{Samhita Paata} \newline

अ॒ग्निर्भू॒ताना॒मधि॑पतिः॒ समा॑ऽव॒त्विन्द्रो᳚ ज्ये॒ष्ठानां᳚ ॅय॒मः पृ॑थि॒व्या वा॒युर॒न्तरि॑क्षस्य॒ सूर्यो॑दि॒वश्च॒न्द्रमा॒ नक्ष॑त्राणां॒ बृह॒स्पति॒र्ब्रह्म॑णो मि॒त्रः स॒त्यानां॒ ॅवरु॑णो॒ऽपाꣳ स॑मु॒द्रः स्रो॒त्याना॒मन्नꣳ॒॒ साम्रा᳚ज्याना॒मधि॑पति॒ तन्मा॑ऽवतु॒ सोम॒ ओष॑धीनाꣳ सवि॒ता प्र॑स॒वानाꣳ॑ रु॒द्रः प॑शू॒नां त्वष्टा॑ रू॒पाणां॒ ॅविष्णुः॒ पर्व॑तानां म॒रुतो॑ ग॒णाना॒मधि॑पतय॒स्ते मा॑वन्तु॒ पित॑रः पितामहाः परेऽवरे॒ ( ) तता᳚स्ततामहा इ॒ह मा॑ऽवत । अ॒स्मिन् ब्रह्म॑न्न॒स्मिन् क्ष॒त्रे᳚ऽस्या-मा॒शिष्य॒स्यां पु॑रो॒धाया॑म॒स्मिन्-कर्म॑न्न॒स्यां दे॒वहू᳚त्यां ॥ \newline

\textbf{Pada Paata} \newline

अ॒ग्निः । भू॒ताना᳚म् । अधि॑पति॒रित्यधि॑- प॒तिः॒ । सः । मा॒ । अ॒व॒तु॒ । इन्द्रः॑ । ज्ये॒ष्ठाना᳚म् । य॒मः । पृ॒थि॒व्याः । वा॒युः । अ॒न्तरि॑क्षस्य । सूर्यः॑ । दि॒वः । च॒न्द्रमाः᳚ । नक्ष॑त्राणाम् । बृह॒स्पतिः॑ । ब्रह्म॑णः । मि॒त्रः । स॒त्याना᳚म् । वरु॑णः । अ॒पाम् । स॒मु॒द्रः । स्रो॒त्याना᳚म् । अन्न᳚म् । साम्रा᳚ज्याना॒मिति॒ सां - रा॒ज्या॒ना॒म् । अधि॑प॒तीत्यधि॑-प॒ति॒ । तत् । मा॒ । अ॒व॒तु॒ । सोमः॑ । ओष॑धीनाम् । स॒वि॒ता । प्र॒स॒वाना॒मिति॑ प्र - स॒वाना᳚म् । रु॒द्रः । प॒शू॒नाम् । त्वष्टा᳚ । रू॒पाणा᳚म् । विष्णुः॑ । पर्व॑तानाम् । म॒रुतः॑ । ग॒णाना᳚म् । अधि॑पतय॒ इत्यधि॑ - प॒त॒यः॒ । ते । मा॒ । अ॒व॒न्तु॒ । पित॑रः । पि॒ता॒म॒हाः॒ । प॒रे॒ । अ॒व॒रे॒ ( ) । तताः᳚ । त॒ता॒म॒हाः॒ । इ॒ह । मा॒ । अ॒व॒त॒ ॥ अ॒स्मिन्न् । ब्रह्मन्न्॑ । अ॒स्मिन्न् । क्ष॒त्रे । अ॒स्याम् । आ॒शिषीत्या᳚ - शिषि॑ । अ॒स्याम् । पु॒रो॒धाया॒मिति॑ पुरः - धाया᳚म् । अ॒स्मिन्न् । कर्मन्न्॑ । अ॒स्याम् । दे॒वहू᳚त्या॒मिति॑ दे॒व - हू॒त्या॒म् ॥  \newline




\markright{ TS 3.4.6.1  \hfill https://www.vedavms.in \hfill}
\addcontentsline{toc}{section}{ TS 3.4.6.1 }
\section*{ TS 3.4.6.1 }

\textbf{TS 3.4.6.1 } \newline
\textbf{Samhita Paata} \newline

दे॒वा वै यद्य॒ज्ञेऽकु॑र्वत॒ तदसु॑रा अकुर्वत॒ ते दे॒वा ए॒तान॑भ्याता॒नान॑पश्य॒न्- तान॒भ्यात॑न्वत॒ यद्दे॒वानां॒ कर्माऽऽ*सी॒दार्द्ध्य॑त॒ तद्यदसु॑राणां॒ न तदा᳚र्द्ध्यत॒ येन॒ कर्म॒णेर्थ्से॒त् तत्र॑ होत॒व्या॑ ऋ॒द्ध्नोत्ये॒व तेन॒ कर्म॑णा॒ यद्विश्वे॑ दे॒वाः स॒मभ॑र॒न् तस्मा॑-दभ्याता॒ना वै᳚श्वदे॒वायत्-प्र॒जाप॑ति॒र्जया॒न् प्राय॑च्छ॒त् तस्मा॒ज्जयाः᳚ प्राजाप॒त्या - [  ] \newline

\textbf{Pada Paata} \newline

दे॒वाः । वै । यत् । य॒ज्ञे । अकु॑र्वत । तत् । असु॑राः । अ॒कु॒र्व॒त॒ । ते । दे॒वाः । ए॒तान् । अ॒भ्या॒ता॒नानित्य॑भि - आ॒ता॒नान् । अ॒प॒श्य॒न्न् । तान् । अ॒भ्यात॑न्व॒तेत्य॑भि - आत॑न्वत । यत् । दे॒वाना᳚म् । कर्म॑ । आसी᳚त् । आर्द्ध्य॑त । तत् । यत् । असु॑राणाम् । न । तत् । आ॒र्द्ध्य॒त॒ । येन॑ । कर्म॑णा । ईर्थ्से᳚त् । तत्र॑ । हो॒त॒व्याः᳚ । ऋ॒द्ध्नोति॑ । ए॒व । तेन॑ । कर्म॑णा । यत् । विश्वे᳚ । दे॒वाः । स॒मभ॑र॒न्निति॑ सं - अभ॑रन्न् । तस्मा᳚त् । अ॒भ्या॒ता॒ना इत्य॑भि - आ॒ता॒नाः । वै॒श्व॒दे॒वा इति॑ वैश्व - दे॒वाः । यत् । प्र॒जाप॑ति॒रिति॑ प्र॒जा - प॒तिः॒ । जयान्॑ । प्रेति॑ । अय॑च्छत् । तस्मा᳚त् । जयाः᳚ । प्रा॒जा॒प॒त्या इति॑ प्राजा-प॒त्याः ।  \newline




\markright{ TS 3.4.6.2  \hfill https://www.vedavms.in \hfill}
\addcontentsline{toc}{section}{ TS 3.4.6.2 }
\section*{ TS 3.4.6.2 }

\textbf{TS 3.4.6.2 } \newline
\textbf{Samhita Paata} \newline

यद्-रा᳚ष्ट्र॒भृद्भी॑ रा॒ष्ट्रमाऽद॑दत॒ तद्-रा᳚ष्ट्र॒भृताꣳ॑ राष्ट्रभृ॒त्त्वं ते दे॒वा अ॑भ्याता॒नैरसु॑रान॒भ्यात॑न्वत॒ जयै॑रजयन्-राष्ट्र॒भृद्भी॑ रा॒ष्ट्रमाऽद॑दत॒ यद्दे॒वा अ॑भ्याता॒नैरसु॑रान॒भ्यात॑न्वत॒ तद॑भ्याता॒नाना॑मभ्यातान॒त्वं ॅयज्जयै॒रज॑य॒न् तज्जया॑नां जय॒त्वं ॅयद्-रा᳚ष्ट्र॒भृद्भी॑ रा॒ष्ट्रमाऽद॑दत॒ तद्-रा᳚ष्ट्र॒भृताꣳ॑ राष्ट्रभृ॒त्त्वं ततो॑ दे॒वा अभ॑व॒न् पराऽसु॑रा॒ यो भ्रातृ॑व्यवा॒न्थ् स्याथ् स ( ) ए॒तान् जु॑हुयादभ्याता॒नैरे॒व भ्रातृ॑व्यान॒भ्यात॑नुते॒ जयै᳚र्जयति राष्ट्र॒भृद्भी॑ रा॒ष्ट्रमा द॑त्ते॒ भव॑त्या॒त्मना॒ परा᳚ऽस्य॒ भ्रातृ॑व्यो भवति ॥ \newline

\textbf{Pada Paata} \newline

यत् । रा॒ष्ट्र॒भृद्भि॒रिति॑ राष्ट्र॒भृत् - भिः॒ । रा॒ष्ट्रम् । एति॑ । अद॑दत । तत् । रा॒ष्ट्र॒भृता॒मिति॑ राष्ट्र - भृता᳚म् । रा॒ष्ट्र॒भृ॒त्त्वमिति॑ राष्ट्रभृत् - त्वम् । ते । दे॒वाः । अ॒भ्या॒ता॒नैरित्य॑भि - आ॒ता॒नैः । असु॑रान् । अ॒भ्यात॑न्व॒तेत्य॑भि - आत॑न्वत । जयैः᳚ । अ॒ज॒य॒न् । रा॒ष्ट्र॒भृद्भि॒रिति॑ राष्ट्र॒भृत् - भिः॒ । रा॒ष्ट्रम् । एति॑ । अ॒द॒द॒त॒ । यत् । दे॒वाः । अ॒भ्या॒ता॒नैरित्य॑भि - आ॒ता॒नैः । असु॑रान् । अ॒भ्यात॑न्व॒तेत्य॑भि - आत॑न्वत । तत् । अ॒भ्या॒ता॒नाना॒मित्य॑भि - आ॒ता॒नाना᳚म् । अ॒भ्या॒ता॒न॒त्वमित्य॑भ्यातान - त्वम् । यत् । जयैः᳚ । अज॑यन्न् । तत् । जया॑नाम् । ज॒य॒त्वमिति॑ जय - त्वम् । यत् । रा॒ष्ट्र॒भृद्भि॒रिति॑ राष्ट्र॒भृत् - भिः॒ । रा॒ष्ट्रम् । एति॑ । अद॑दत । तत् । रा॒ष्ट्र॒भृता॒मिति॑ राष्ट्र - भृता᳚म् । रा॒ष्ट्र॒भृ॒त्त्वमिति॑ राष्ट्रभृत् - त्वम् । ततः॑ । दे॒वाः । अभ॑वन्न् । परेति॑ । असु॑राः । यः । भ्रातृ॑व्यवा॒निति॒ भ्रातृ॑व्य - वा॒न् । स्यात् । सः ( ) । ए॒तान् । जु॒हु॒या॒त् । अ॒भ्या॒ता॒नैरित्य॑भि-आ॒ता॒नैः । ए॒व । भ्रातृ॑व्यान् । अ॒भ्यात॑नुत॒ इत्य॑भि - आत॑नुते । जयैः᳚ । ज॒य॒ति॒ । रा॒ष्ट्र॒भृद्भि॒रिति॑ राष्ट्र॒भृत् - भिः॒ । रा॒ष्ट्रम् । एति॑ । द॒त्ते॒ । भव॑ति । आ॒त्मना᳚ । परेति॑ । अ॒स्य॒ । भ्रातृ॑व्यः । भ॒व॒ति॒ ॥  \newline




\markright{ TS 3.4.7.1  \hfill https://www.vedavms.in \hfill}
\addcontentsline{toc}{section}{ TS 3.4.7.1 }
\section*{ TS 3.4.7.1 }

\textbf{TS 3.4.7.1 } \newline
\textbf{Samhita Paata} \newline

ऋ॒ता॒षाड् ऋ॒तधा॑मा॒ऽग्नि-र्ग॑न्ध॒र्वस्त-स्यौष॑धयोऽफ्स॒रस॒ ऊर्जो॒ नाम॒ स इ॒दं ब्रह्म॑ क्ष॒त्रं पा॑तु॒ ता इ॒दं ब्रह्म॑ क्ष॒त्रं पा᳚न्तु॒ तस्मै॒ स्वाहा॒ ताभ्यः॒ स्वाहा॑ सꣳहि॒तो वि॒श्वसा॑मा॒ सूर्यो॑ गन्ध॒र्व-स्तस्य॒ मरी॑चयोऽफ्स॒रस॑ आ॒युवः॑ सुषु॒म्नः सूर्य॑ रश्मि-श्च॒न्द्रमा॑ गन्ध॒र्व-स्तस्य॒ नक्ष॑त्राण्य-फ्स॒रसो॑ बे॒कुर॑योभु॒ज्युः सु॑प॒र्णो य॒ज्ञो ग॑न्ध॒र्व-स्तस्य॒ दक्षि॑णा अप्स॒रस॑ स्त॒वाः प्र॒जाप॑ति-र्वि॒श्वक॑र्मा॒ मनो॑ - [  ] \newline

\textbf{Pada Paata} \newline

ऋ॒ता॒षाट् । ऋ॒तधा॒मेत्यृ॒त - धा॒मा॒ । अ॒ग्निः । ग॒न्ध॒र्वः । तस्य॑ । ओष॑धयः । अ॒फ्स॒रसः॑ । ऊर्जः॑ । नाम॑ । सः । इ॒दम् । ब्रह्म॑ । क्ष॒त्रम् । पा॒तु॒ । ताः । इ॒दम् । ब्रह्म॑ । क्ष॒त्रम् । पा॒न्तु॒ । तस्मै᳚ । स्वाहा᳚ । ताभ्यः॑ । स्वाहा᳚ । सꣳ॒॒हि॒त इति॑ सं - हि॒तः । वि॒श्वसा॒मेति॑ वि॒श्व - सा॒मा॒ । सूर्यः॑ । ग॒न्ध॒र्वः । तस्य॑ । मरी॑चयः । अ॒फ्स॒रसः॑ । आ॒युव॒ इत्या᳚ - युवः॑ । सु॒षु॒म्न इति॑ सु - सु॒म्नः । सूर्य॑रश्मि॒रिति॒ सूर्य॑-र॒श्मिः॒ । च॒न्द्रमाः᳚ । ग॒न्ध॒र्वः । तस्य॑ । नक्ष॑त्राणि । अ॒फ्स॒रसः॑ । बे॒कुर॑यः । भु॒ज्युः । सु॒प॒र्ण इति॑ सु - प॒र्णः । य॒ज्ञ्ः । ग॒न्ध॒र्वः । तस्य॑ । दक्षि॑णाः । अ॒फ्स॒रसः॑ । स्त॒वाः । प्र॒जाप॑ति॒रिति॑ प्र॒जा-प॒तिः॒ । वि॒श्वक॒र्मेति॑ वि॒श्व - क॒र्मा॒ । मनः॑ ।  \newline




\markright{ TS 3.4.7.2  \hfill https://www.vedavms.in \hfill}
\addcontentsline{toc}{section}{ TS 3.4.7.2 }
\section*{ TS 3.4.7.2 }

\textbf{TS 3.4.7.2 } \newline
\textbf{Samhita Paata} \newline

गन्ध॒र्वस्तस्य॑र्ख् सा॒मान्य॑-फ्स॒रसो॒ वह्न॑यैषि॒रो वि॒श्वव्य॑चा॒ वातो॑ गन्ध॒र्व-स्तस्याऽऽ*पो᳚ ऽफ्स॒रसो॑ मु॒दाभुव॑नस्य पते॒ यस्य॑त उ॒परि॑ गृ॒हा इ॒ह च॑ । स नो॑ रा॒स्वाज्या॑निꣳ रा॒यस्पोषꣳ॑ सु॒वीर्यꣳ॑ संवॅथ्स॒रीणाꣳ॑ स्व॒स्तिं ॥ प॒र॒मे॒ष्ठ्यधि॑पति-र्मृ॒त्युर् ग॑न्ध॒र्व-स्तस्य॒ विश्व॑मप्स॒रसो॒ भुवः॑ सुक्षि॒तिः- सुभू॑ति-र्भद्र॒कृथ् सुव॑र्वान् प॒र्जन्यो॑ गन्ध॒र्व-स्तस्य॑ वि॒द्युतो᳚ ऽफ्स॒रसो॒ रुचो॑ दू॒रे हे॑तिर-मृड॒यो - [  ] \newline

\textbf{Pada Paata} \newline

ग॒न्ध॒र्वः । तस्य॑ । ऋ॒ख्सा॒मानीत्यृ॑क्-सा॒मानि॑ । अ॒फ्स॒रसः॑ । वह्न॑यः । इ॒षि॒रः । वि॒श्वव्य॑चा॒ इति॑ वि॒श्व - व्य॒चाः॒ । वातः॑ । ग॒न्ध॒र्वः । तस्य॑ । आपः॑ । अ॒फ्स॒रसः॑ । मु॒दाः । भुव॑नस्य । प॒ते॒ । यस्य॑ । ते॒ । उ॒परि॑ । गृ॒हाः । इ॒ह । च॒ ॥ सः । नः॒ । रा॒स्व॒ । अज्या॑निम् । रा॒यः । पोष᳚म् । सु॒वीर्य॒मिति॑ सु - वीर्य᳚म् । सं॒ॅव॒थ्स॒रीणा॒मिति॑ सं - व॒थ्स॒रीणा᳚म् । स्व॒स्तिम् ॥ प॒र॒मे॒ष्ठी । अधि॑प॒तिरित्यधि॑ - प॒तिः॒ । मृ॒त्युः । ग॒न्ध॒र्वः । तस्य॑ । विश्व᳚म् । अ॒फ्स॒रसः॑ । भुवः॑ । सु॒क्षि॒तिरिति॑ सु - क्षि॒तिः । सुभू॑ति॒रिति॒ सु - भू॒तिः॒ । भ॒द्र॒कृदिति॑ भद्र - कृत् । सुव॑र्वा॒निति॒ सुवः॑ - वा॒न् । प॒र्जन्यः॑ । ग॒न्ध॒र्वः । तस्य॑ । वि॒द्युत॒ इति॑ वि-द्युतः॑ । अ॒फ्स॒रसः॑ । रुचः॑ । दू॒रेहे॑ति॒रिति॑ दू॒रे - हे॒तिः॒ । अ॒मृ॒ड॒यः ।  \newline




\markright{ TS 3.4.7.3  \hfill https://www.vedavms.in \hfill}
\addcontentsline{toc}{section}{ TS 3.4.7.3 }
\section*{ TS 3.4.7.3 }

\textbf{TS 3.4.7.3 } \newline
\textbf{Samhita Paata} \newline

मृ॒त्युर्ग॑न्ध॒र्व-स्तस्य॑ प्र॒जा अ॑फ्स॒रसो॑ भी॒रुव॒श्चरुः॑ कृपण का॒शी कामो॑ गन्ध॒र्व-स्तस्या॒धयो᳚ ऽफ्स॒रसः॑ शो॒चय॑न्ती॒र्नाम॒ स इ॒दं ब्रह्म॑ क्ष॒त्रं पा॑त॒ ता इ॒दं ब्रह्म॑ क्ष॒त्रं पा᳚न्तु॒ तस्मै॒ स्वाहा॒ ताभ्यः॒ स्वाहा॒ स नो॑ भुवनस्य पते॒ यस्य॑त उ॒परि॑ गृ॒हा इ॒ह च॑ । उ॒रु ब्र॒ह्म॑णे॒ऽस्मै क्ष॒त्राय॒ महि॒ शर्म॑ यच्छ ॥ \newline

\textbf{Pada Paata} \newline

मृ॒त्युः । ग॒न्ध॒र्वः । तस्य॑ । प्र॒जा इति॑ प्र - जाः । अ॒फ्स॒रसः॑ । भी॒रुवः॑ । चारुः॑ । कृ॒प॒ण॒का॒शीति॑ कृपण-का॒शी । कामः॑ । ग॒न्ध॒र्वः । तस्य॑ । आ॒धय॒ इत्या᳚ - धयः॑ । अ॒फ्स॒रसः॑ । शो॒चय॑न्तीः । नाम॑ । सः । इ॒दम् । ब्रह्म॑ । क्ष॒त्रम् । पा॒तु॒ । ताः । इ॒दम् । ब्रह्म॑ । क्ष॒त्रम् । पा॒न्तु॒ । तस्मै᳚ । स्वाहा᳚ । ताभ्यः॑ । स्वाहा᳚ । सः । नः॒ । भु॒व॒न॒स्य॒ । प॒ते॒ । यस्य॑ । ते॒ । उ॒परि॑ । गृ॒हाः । इ॒ह । च॒ ॥ उ॒रु । ब्रह्म॑णे । अ॒स्मै । क्ष॒त्राय॑ । महि॑ । शर्म॑ । य॒च्छ॒ ॥  \newline




\markright{ TS 3.4.8.1  \hfill https://www.vedavms.in \hfill}
\addcontentsline{toc}{section}{ TS 3.4.8.1 }
\section*{ TS 3.4.8.1 }

\textbf{TS 3.4.8.1 } \newline
\textbf{Samhita Paata} \newline

रा॒ष्ट्रका॑माय होत॒व्या॑ रा॒ष्ट्रं ॅवै रा᳚ष्ट्र॒भृतो॑ रा॒ष्ट्रेणै॒वास्मै॑ रा॒ष्ट्रमव॑ रुन्धे रा॒ष्ट्रमे॒व भ॑वत्या॒त्मने॑ होत॒व्या॑ रा॒ष्ट्रं ॅवै रा᳚ष्ट्र॒भृतो॑ रा॒ष्ट्रं प्र॒जा रा॒ष्ट्रं प॒शवो॑ रा॒ष्ट्रं ॅयच्छ्रेष्ठो॒ भव॑ति रा॒ष्ट्रेणै॒व रा॒ष्ट्रमव॑ रुन्धे॒ वसि॑ष्ठः समा॒नानां᳚ भवति॒ ग्राम॑कामाय होत॒व्या॑ रा॒ष्ट्रं ॅवै रा᳚ष्ट्र॒भृतो॑ रा॒ष्ट्रꣳ स॑जा॒ता रा॒ष्ट्रेणै॒वास्मै॑ रा॒ष्ट्रꣳ स॑जा॒तानव॑ रुन्धे ग्रा॒म्ये॑ - [  ] \newline

\textbf{Pada Paata} \newline

रा॒ष्ट्रका॑मा॒येति॑ रा॒ष्ट्र - का॒मा॒य॒ । हो॒त॒व्याः᳚ । रा॒ष्ट्रम् । वै । रा॒ष्ट्र॒भृत॒ इति॑ राष्ट्र - भृतः॑ । रा॒ष्ट्रेण॑ । ए॒व । अ॒स्मै॒ । रा॒ष्ट्रम् । अवेति॑ । रु॒न्धे॒ । रा॒ष्ट्रम् । ए॒व । भ॒व॒ति॒ । आ॒त्मने᳚ । हो॒त॒व्याः᳚ । रा॒ष्ट्रम् । वै । रा॒ष्ट्र॒भृत॒ इति॑ राष्ट्र - भृतः॑ । रा॒ष्ट्रम् । प्र॒जेति॑ प्र - जा । रा॒ष्ट्रम् । प॒शवः॑ । रा॒ष्ट्रम् । यत् । श्रेष्ठः॑ । भव॑ति । रा॒ष्ट्रेण॑ । ए॒व । रा॒ष्ट्रम् । अवेति॑ । रु॒न्धे॒ । वसि॑ष्ठः । स॒मा॒नाना᳚म् । भ॒व॒ति॒ । ग्राम॑कामा॒येति॒ ग्राम॑ - का॒मा॒य॒ । हो॒त॒व्याः᳚ । रा॒ष्ट्रम् । वै । रा॒ष्ट्र॒भृत॒ इति॑ राष्ट्र-भृतः॑ । रा॒ष्ट्रम् । स॒जा॒ता इति॑ स - जा॒ताः । रा॒ष्ट्रेण॑ । ए॒व । अ॒स्मै॒ । रा॒ष्ट्रम् । स॒जा॒तानिति॑ स - जा॒तान् । अवेति॑ । रु॒न्धे॒ । ग्रा॒मी ।  \newline




\markright{ TS 3.4.8.2  \hfill https://www.vedavms.in \hfill}
\addcontentsline{toc}{section}{ TS 3.4.8.2 }
\section*{ TS 3.4.8.2 }

\textbf{TS 3.4.8.2 } \newline
\textbf{Samhita Paata} \newline

-व भ॑वत्यधि॒देव॑ने जुहोत्यधि॒देव॑न ए॒वास्मै॑ सजा॒तानव॑ रुन्धे॒ त ए॑न॒मव॑रुद्धा॒ उप॑ तिष्ठन्ते रथमु॒ख ओज॑स्कामस्य होत॒व्या॑ ओजो॒ वै रा᳚ष्ट्र॒भृत॒ ओजो॒ रथ॒ ओज॑सै॒वास्मा॒ ओजोऽव॑ रुन्ध ओज॒स्व्ये॑व भ॑वति॒ यो रा॒ष्ट्रादप॑भूतः॒ स्यात् तस्मै॑ होत॒व्या॑ याव॑न्तोऽस्य॒ रथाः॒ स्युस्तान् ब्रू॑याद् यु॒ङ्ध्वमिति॑ रा॒ष्ट्रमे॒वास्मै॑ युन॒क्त्या - [  ] \newline

\textbf{Pada Paata} \newline

ए॒व । भ॒व॒ति॒ । अ॒धि॒देव॑न॒ इत्य॑धि - देव॑ने । जु॒हो॒ति॒ । अ॒धि॒देव॑न॒ इत्य॑धि - देव॑ने । ए॒व । अ॒स्मै॒ । स॒जा॒तानिति॑ स-जा॒तान् । अवेति॑ । रु॒न्धे॒ । ते । ए॒न॒म् । अव॑रुद्धा॒ इत्यव॑ - रु॒द्धाः॒ । उपेति॑ । ति॒ष्ठ॒न्ते॒ । र॒थ॒मु॒ख इति॑ रथ - मु॒खे । ओज॑स्काम॒स्येत्योजः॑ - का॒म॒स्य॒ । हो॒त॒व्याः᳚ । ओजः॑ । वै । रा॒ष्ट्र॒भृत॒ इति॑ राष्ट्र-भृतः॑ । ओजः॑ । रथः॑ । ओज॑सा । ए॒व । अ॒स्मै॒ । ओजः॑ । अवेति॑ । रु॒न्धे॒ । ओ॒ज॒स्वी । ए॒व । भ॒व॒ति॒ । यः । रा॒ष्ट्रात् । अप॑भूत॒ इत्यप॑ - भू॒तः॒ । स्यात् । तस्मै᳚ । हो॒त॒व्याः᳚ । याव॑न्तः । अ॒स्य॒ । रथाः᳚ । स्युः । तान् । ब्रू॒या॒त् । यु॒ङ्ध्वम् । इति॑ । रा॒ष्ट्रम् । ए॒व । अ॒स्मै॒ । यु॒न॒क्ति॒ ।  \newline




\markright{ TS 3.4.8.3  \hfill https://www.vedavms.in \hfill}
\addcontentsline{toc}{section}{ TS 3.4.8.3 }
\section*{ TS 3.4.8.3 }

\textbf{TS 3.4.8.3 } \newline
\textbf{Samhita Paata} \newline

हु॑तयो॒ वा ए॒तस्याक्लृ॑प्ता॒ यस्य॑ रा॒ष्ट्रं न कल्प॑ते स्वर॒थस्य॒ दक्षि॑णं च॒क्रं प्र॒वृह्य॑ ना॒डीम॒भि जु॑हुया॒दाहु॑तीरे॒वास्य॑ कल्पयति॒ ता अ॑स्य॒ कल्प॑माना रा॒ष्ट्रमनु॑ कल्पते संग्रा॒मे संॅय॑त्ते होत॒व्या॑ रा॒ष्ट्रं ॅवै रा᳚ष्ट्र॒भृतो॑ रा॒ष्ट्रे खलु॒ वा ए॒ते व्याय॑च्छन्ते॒ ये स॑ङ्ग्रा॒मꣳ सं॒ ॅयन्ति॒ यस्य॒ पूर्व॑स्य॒ जुह्व॑ति॒ स ए॒व भ॑वति॒ जय॑ति॒ तꣳ स॑ग्रां॒मं मा᳚न्धु॒क इ॒द्ध्मो- [  ] \newline

\textbf{Pada Paata} \newline

आहु॑तय॒ इत्या - हु॒त॒यः॒ । वै । ए॒तस्य॑ । अक्लृ॑प्ताः । यस्य॑ । रा॒ष्ट्रम् । न । कल्प॑ते । स्व॒र॒थस्येति॑ स्व - र॒थस्य॑ । दक्षि॑णम् । च॒क्रम् । प्र॒वृह्येति॑ प्र - वृह्य॑ । ना॒डीम् । अ॒भीति॑ । जु॒हु॒या॒त् । आहु॑ती॒रित्या - हु॒तीः॒ । ए॒व । अ॒स्य॒ । क॒ल्प॒य॒ति॒ । ताः । अ॒स्य॒ । कल्प॑मानाः । रा॒ष्ट्रम् । अन्विति॑ । क॒ल्प॒ते॒ । स॒ग्रां॒म इति॑ सं - ग्रा॒मे । संॅय॑त्त॒ इति॒ सं - य॒त्ते॒ । हो॒त॒व्याः᳚ । रा॒ष्ट्रम् । वै । रा॒ष्ट्र॒भृत॒ इति॑ राष्ट्र - भृतः॑ । रा॒ष्ट्रे । खलु॑ । वै । ए॒ते । व्याय॑च्छन्त॒ इति॑ वि - आय॑च्छन्ते । ये । स॒ग्रां॒ममिति॑ सं - ग्रा॒मम् । सं॒ॅयन्तीति॑ सं-यन्ति॑ । यस्य॑ । पूर्व॑स्य । जुह्व॑ति । सः । ए॒व । भ॒व॒ति॒ । जय॑ति । तम् । स॒ग्रां॒ममिति॑ सं - ग्रा॒मम् । मा॒न्धु॒कः । इ॒द्ध्मः ।  \newline




\markright{ TS 3.4.8.4  \hfill https://www.vedavms.in \hfill}
\addcontentsline{toc}{section}{ TS 3.4.8.4 }
\section*{ TS 3.4.8.4 }

\textbf{TS 3.4.8.4 } \newline
\textbf{Samhita Paata} \newline

भ॑व॒त्यङ्गा॑रा ए॒व प्र॑ति॒वेष्ट॑माना अ॒मित्रा॑णामस्य॒ सेनां॒ प्रति॑वेष्टयन्ति॒ य उ॒न्माद्ये॒त् तस्मै॑ होत॒व्या॑ गन्धर्वाफ्स॒रसो॒ वा ए॒तमुन्मा॑दयन्ति॒ य उ॒न्माद्य॑त्ये॒ते खलु॒ वै ग॑न्धर्वाफ्स॒रसो॒ यद्रा᳚ष्ट्र॒भृत॒स्तस्मै॒ स्वाहा॒ ताभ्यः॒ स्वाहेति॑ जुहोति॒ तेनै॒वैना᳚ञ्छमयति॒ नैय॑ग्रोध॒ औदु॑म्बर॒ आश्व॑त्थः॒ प्लाक्ष॒ इती॒द्ध्मो भ॑वत्ये॒ते वै ग॑न्धर्वाफ्स॒रसां᳚ गृ॒हाः स्व ए॒वैना॑ - [  ] \newline

\textbf{Pada Paata} \newline

भ॒व॒ति॒ । अङ्गा॑राः । ए॒व । प्र॒ति॒वेष्ट॑माना॒ इति॑ प्रति - वेष्ट॑मानाः । अ॒मित्रा॑णाम् । अ॒स्य॒ । सेना᳚म् । प्रतीति॑ । वे॒ष्ट॒य॒न्ति॒ । यः । उ॒न्माद्ये॒दित्यु॑त् - माद्ये᳚त् । तस्मै᳚ । हो॒त॒व्याः᳚ । ग॒न्ध॒र्वा॒फ्स॒रस॒ इति॑ गन्धर्व - अ॒फ्स॒रसः॑ । वै । ए॒तम् । उदिति॑ । मा॒द॒य॒न्ति॒ । यः । उ॒न्माद्य॒तीत्यु॑त् - माद्य॑ति । ए॒ते । खलु॑ । वै । ग॒न्ध॒र्वा॒फ्स॒रस॒ इति॑ गन्धर्व - अ॒फ्स॒रसः॑ । यत् । रा॒ष्ट्र॒भृत॒ इति॑ राष्ट्र - भृतः॑ । तस्मै᳚ । स्वाहा᳚ । ताभ्यः॑ । स्वाहा᳚ । इति॑ । जु॒हो॒ति॒ । तेन॑ । ए॒व । ए॒ना॒न् । श॒म॒य॒ति॒ । नैय॑ग्रोधः । औदु॑बंरः । आश्व॑त्थः । प्लाक्षः॑ । इति॑ । इ॒द्ध्मः । भ॒व॒ति॒ । ए॒ते । वै । ग॒न्ध॒र्वा॒फ्स॒रसा॒मिति॑ गन्धर्व - अ॒फ्स॒रसा᳚म् । गृ॒हाः । स्वे । ए॒व । ए॒ना॒न् ।  \newline




\markright{ TS 3.4.8.5  \hfill https://www.vedavms.in \hfill}
\addcontentsline{toc}{section}{ TS 3.4.8.5 }
\section*{ TS 3.4.8.5 }

\textbf{TS 3.4.8.5 } \newline
\textbf{Samhita Paata} \newline

ना॒यत॑ने शमयत्यभि॒चर॑ता प्रतिलो॒मꣳ हो॑त॒व्याः᳚ प्रा॒णाने॒वास्य॑ प्र॒तीचः॒ प्रति॑ यौति॒ तं ततो॒ येन॒ केन॑ च स्तृणुते॒ स्वकृ॑त॒ इरि॑णे जुहोति प्रद॒रे वै॒तद्वा अ॒स्यै निर्.ऋ॑तिगृहीतं॒ निर्.ऋ॑तिगृहीत ए॒वैनं॒ निर्.ऋ॑त्या ग्राहयति॒ यद्वा॒चः क्रू॒रं तेन॒ वष॑ट् करोति वा॒च ए॒वैनं॑ क्रू॒रेण॒ प्रवृ॑श्चति ता॒जगार्ति॒मार्च्छ॑ति॒ यस्य॑ का॒मये॑ता॒न्नाद्य॒ - [  ] \newline

\textbf{Pada Paata} \newline

आ॒यत॑न॒ इत्या᳚ - यत॑ने । श॒म॒य॒ति॒ । अ॒भि॒चर॒तेत्य॑भि - चर॑ता । प्र॒ति॒लो॒ममिति॑ प्रति - लो॒मम् । हो॒त॒व्याः᳚ । प्रा॒णानिति॑ प्र - अ॒नान् । ए॒व । अ॒स्य॒ । प्र॒तीचः॑ । प्रतीति॑ । यौ॒ति॒ । तम् । ततः॑ । येन॑ । केन॑ । च॒ । स्तृ॒णु॒ते॒ । स्वकृ॑त॒ इति॒ स्व - कृ॒ते॒ । इरि॑णे । जु॒हो॒ति॒ । प्र॒द॒र इति॑ प्र - द॒रे । वा॒ । ए॒तत् । वै । अ॒स्यै । निर्.ऋ॑तिगृहीत॒मिति॒ निर्.ऋ॑ति - गृ॒ही॒त॒म् । निर्.ऋ॑तिगृहीत॒ इति॒ निर्.ऋ॑ति-गृ॒ही॒ते॒ । ए॒व । ए॒न॒म् । निर्.ऋ॒त्येति॒ निः-ऋ॒त्या॒ । ग्रा॒ह॒य॒ति॒ । यत् । वा॒चः । क्रू॒रम् । तेन॑ । वष॑ट् । क॒रो॒ति॒ । वा॒चः । ए॒व । ए॒न॒म् । क्रू॒रेण॑ । प्रेति॑ । वृ॒श्च॒ति॒ । ता॒जक् । आर्ति᳚म् । एति॑ । ऋ॒च्छ॒ति॒ । यस्य॑ । का॒मये॑त । अ॒न्नाद्य॒मित्य॑न्न - अद्य᳚म् ।  \newline




\markright{ TS 3.4.8.6  \hfill https://www.vedavms.in \hfill}
\addcontentsline{toc}{section}{ TS 3.4.8.6 }
\section*{ TS 3.4.8.6 }

\textbf{TS 3.4.8.6 } \newline
\textbf{Samhita Paata} \newline

मा द॑दी॒येति॒ तस्य॑ स॒भाया॑मुत्ता॒नो नि॒पद्य॒ भुव॑नस्य पत॒ इति॒ तृणा॑नि॒ सं गृ॑ह्णीयात् प्र॒जाप॑ति॒र्वै भुव॑नस्य॒ पतिः॑ प्र॒जाप॑तिनै॒वास्या॒न्नाद्य॒मा द॑त्त इ॒दम॒हम॒मुष्या॑ ऽऽ*मुष्याय॒णस्या॒न्नाद्यꣳ॑ हरा॒मीत्या॑हा॒न्नाद्य॑मे॒वास्य॑ हरति ष॒ड्भिर्.ह॑रति॒ षड्वा ऋ॒तवः॑ प्र॒जाप॑तिनै॒वास्या॒-न्नाद्य॑मा॒दाय॒र्तवो᳚ ऽस्मा॒ अनु॒ प्रय॑च्छन्ति॒ - [  ] \newline

\textbf{Pada Paata} \newline

एति॑ । द॒दी॒य॒ । इति॑ । तस्य॑ । स॒भाया᳚म् । उ॒त्ता॒न इत्यु॑त् - ता॒नः । नि॒पद्येति॑ नि - पद्य॑ । भुव॑नस्य । प॒ते॒ । इति॑ । तृणा॑नि । समिति॑ । गृ॒ह्णी॒या॒त् । प्र॒जाप॑ति॒रिति॑ प्र॒जा - प॒तिः॒ । वै । भुव॑नस्य । पतिः॑ । प्र॒जाप॑ति॒नेति॑ प्र॒जा - प॒ति॒ना॒ । ए॒व । अ॒स्य॒ । अ॒न्नाद्य॒मित्य॑न्न - अद्य᳚म् । एति॑ । द॒त्ते॒ । इ॒दम् । अ॒हम् । अ॒मुष्य॑ । आ॒मु॒ष्या॒य॒णस्य॑ । अ॒न्नाद्य॒मित्य॑न्न - अद्य᳚म् । ह॒रा॒मि॒ । इति॑ । आ॒ह॒ । अ॒न्नाद्य॒मित्य॑न्न - अद्य᳚म् । ए॒व । अ॒स्य॒ । ह॒र॒ति॒ । ष॒ड्भिरिति॑ षट् - भिः । ह॒र॒ति॒ । षट् । वै । ऋ॒तवः॑ । प्र॒जाप॑ति॒नेति॑ प्र॒जा-प॒ति॒ना॒ । ए॒व । अ॒स्य॒ । अ॒न्नाद्य॒मित्य॑न्न -अद्य᳚म् । आ॒दायेत्या᳚-दाय॑ । ऋ॒तवः॑ । अ॒स्मै॒ । अनु॑ । प्रेति॑ । य॒च्छ॒न्ति॒ ।  \newline




\markright{ TS 3.4.8.7  \hfill https://www.vedavms.in \hfill}
\addcontentsline{toc}{section}{ TS 3.4.8.7 }
\section*{ TS 3.4.8.7 }

\textbf{TS 3.4.8.7 } \newline
\textbf{Samhita Paata} \newline

यो ज्ये॒ष्ठब॑न्धु॒रप॑ भूतः॒ स्यात् तꣳस्थले॑ऽव॒साय्य॑ ब्रह्मौद॒नं चतुः॑ शरावं प॒क्त्वा तस्मै॑ होत॒व्या॑ वर्ष्म॒ वै रा᳚ष्ट्र॒भृतो॒ वष्म॒ स्थलं॒ ॅवर्ष्म॑णै॒वैनं॒ ॅवष्म॑ समा॒नानां᳚ गमयति॒ चतुः॑ शरावो भवति दि॒क्ष्वे॑व प्रति॑तिष्ठति क्षी॒रे भ॑वति॒ रुच॑मे॒वास्मि॑न् दधा॒त्युद्ध॑रति शृत॒त्वाय॑ स॒र्पिष्वा᳚न् भवति मेद्ध्य॒त्वाय॑ च॒त्वार॑ आर्.षे॒याः प्राश्न॑न्ति दि॒शामे॒व ज्योति॑षि जुहोति ॥ \newline

\textbf{Pada Paata} \newline

यः । ज्ये॒ष्ठब॑न्धु॒रिति॑ ज्ये॒ष्ठ - ब॒न्धुः॒ । अप॑भूत॒ इत्यप॑ - भू॒तः॒ । स्यात् । तम् । स्थले᳚ । अ॒व॒साय्येत्य॑व - साय्य॑ । ब्र॒ह्मौ॒द॒नमिति॑ ब्रह्म - ओ॒द॒नम् । चतुः॑ शराव॒मिति॒ चतुः॑ - श॒रा॒व॒म् । प॒क्त्वा । तस्मै᳚ । हो॒त॒व्याः᳚ । वर्ष्म॑ । वै । रा॒ष्ट्र॒भृत॒ इति॑ राष्ट्र - भृतः॑ । वर्ष्म॑ । स्थल᳚म् । वर्ष्म॑णा । ए॒व । ए॒न॒म् । वर्ष्म॑ । स॒मा॒नाना᳚म् । ग॒म॒य॒ति॒ । चतुः॑ शराव॒ इति॒ चतुः॑ - श॒रा॒वः॒ । भ॒व॒ति॒ । दि॒क्षु । ए॒व । प्रतीति॑ । ति॒ष्ठ॒ति॒ । क्षी॒रे । भ॒व॒ति॒ । रुच᳚म् । ए॒व । अ॒स्मि॒न्न् । द॒धा॒ति॒ । उदिति॑ । ह॒र॒ति॒ । शृ॒त॒त्वायेति॑ शृत - त्वाय॑ । स॒र्पिष्वान्॑ । भ॒व॒ति॒ । मे॒द्ध्य॒त्वायेति॑ मेद्ध्य-त्वाय॑ । च॒त्वारः॑ । आ॒र्॒.षे॒याः । प्रेति॑ । अ॒श्न॒न्ति॒ । दि॒शाम् । ए॒व । ज्योति॑षि । जु॒हो॒ति॒ ॥  \newline




\markright{ TS 3.4.9.1  \hfill https://www.vedavms.in \hfill}
\addcontentsline{toc}{section}{ TS 3.4.9.1 }
\section*{ TS 3.4.9.1 }

\textbf{TS 3.4.9.1 } \newline
\textbf{Samhita Paata} \newline

देवि॑का॒ निव॑र्पेत् प्र॒जाका॑म॒श्छन्दाꣳ॑सि॒ वै देवि॑का॒श्छन्दाꣳ॑सीव॒ खलु॒ वै प्र॒जाश्छन्दो॑भिरे॒वास्मै᳚ प्र॒जाः प्रज॑नयति प्रथ॒मं धा॒तारं॑ करोति मिथु॒नी ए॒व तेन॑ करो॒त्यन्वे॒वास्मा॒ अनु॑मतिर्मन्यते रा॒ते रा॒का प्र सि॑नीवा॒ली ज॑नयति प्र॒जास्वे॒व प्रजा॑तासु कु॒ह्वा॑ वाचं॑ दधात्ये॒ता ए॒व निव॑र्पेत् प॒शुका॑म॒श्छन्दाꣳ॑सि॒ वै देवि॑का॒श्छन्दाꣳ॑सी - [  ] \newline

\textbf{Pada Paata} \newline

देवि॑काः । निरिति॑ । व॒पे॒त् । प्र॒जाका॑म॒ इति॑ प्र॒जा - का॒मः॒ । छन्दाꣳ॑सि । वै । देवि॑काः । छन्दाꣳ॑सि । इ॒व॒ । खलु॑ । वै । प्र॒जा इति॑ प्र-जाः । छन्दो॑भि॒रिति॒ छन्दः॑ - भिः॒ । ए॒व । अ॒स्मै॒ । प्र॒जा इति॑ प्र-जाः । प्रेति॑ । ज॒न॒य॒ति॒ । प्र॒थ॒मम् । धा॒तार᳚म् । क॒रो॒ति॒ । मि॒थु॒नी । ए॒व । तेन॑ । क॒रो॒ति॒ । अन्विति॑ । ए॒व । अ॒स्मै॒ । अनु॑मति॒रित्य॑नु - म॒तिः॒ । म॒न्य॒ते॒ । रा॒ते । रा॒का । प्रेति॑ । सि॒नी॒वा॒ली । ज॒न॒य॒ति॒ । प्र॒जास्विति॑ प्र - जासु॑ । ए॒व । प्रजा॑ता॒स्विति॒ प्र - जा॒ता॒सु॒ । कु॒ह्वा᳚ । वाच᳚म् । द॒धा॒ति॒ । ए॒ताः । ए॒व । निरिति॑ । व॒पे॒त् । प॒शुका॑म॒ इति॑ प॒शु - का॒मः॒ । छन्दाꣳ॑सि । वै । देवि॑काः । छन्दाꣳ॑सि ।  \newline




\markright{ TS 3.4.9.2  \hfill https://www.vedavms.in \hfill}
\addcontentsline{toc}{section}{ TS 3.4.9.2 }
\section*{ TS 3.4.9.2 }

\textbf{TS 3.4.9.2 } \newline
\textbf{Samhita Paata} \newline

व॒ खलु॒ वै प॒शव॒श्छन्दो॑भिरे॒वास्मै॑ प॒शून् प्रज॑नयति प्रथ॒मं धा॒तारं॑ करोति॒ प्रैव तेन॑ वापय॒त्यन्वे॒वास्मा॒ अनु॑मतिर्मन्यते रा॒ते रा॒का प्र सि॑नीवा॒ली ज॑नयति प॒शूने॒व प्रजा॑तान् कु॒ह्वा᳚ प्रति॑ष्ठापयत्ये॒ता ए॒व निर्व॑पे॒द्-ग्राम॑काम॒श्छन्दाꣳ॑सि॒ वै देवि॑का॒श्छन्दाꣳ॑सी व॒ खलु॒ वै ग्राम॒श्छन्दो॑भिरे॒वास्मै॒ ग्राम॒ - [  ] \newline

\textbf{Pada Paata} \newline

इ॒व॒ । खलु॑ । वै । प॒शवः॑ । छन्दो॑भि॒रिति॒ छन्दः॑ - भिः॒ । ए॒व । अ॒स्मै॒ । प॒शून् । प्रेति॑ । ज॒न॒य॒ति॒ । प्र॒थ॒मम् । धा॒तार᳚म् । क॒रो॒ति॒ । प्रेति॑ । ए॒व । तेन॑ । वा॒प॒य॒ति॒ । अन्विति॑ । ए॒व । अ॒स्मै॒ । अनु॑मति॒रित्य॑नु - म॒तिः॒ । म॒न्य॒ते॒ । रा॒ते । रा॒का । प्रेति॑ । सि॒नी॒वा॒ली । ज॒न॒य॒ति॒ । प॒शून् । ए॒व । प्रजा॑ता॒निति॒ प्र-जा॒ता॒न् । कु॒ह्वा᳚ । प्रतीति॑ । स्था॒प॒य॒ति॒ । ए॒ताः । ए॒व । निरिति॑ । व॒पे॒त् । ग्राम॑काम॒ इति॒ ग्राम॑ - का॒मः॒ । छन्दाꣳ॑सि । वै । देवि॑काः । छन्दाꣳ॑सि । इ॒व॒ । खलु॑ । वै । ग्रामः॑ । छन्दो॑भि॒रिति॒ छन्दः॑ - भिः॒ । ए॒व । अ॒स्मै॒ । ग्राम᳚म् ।  \newline




\markright{ TS 3.4.9.3  \hfill https://www.vedavms.in \hfill}
\addcontentsline{toc}{section}{ TS 3.4.9.3 }
\section*{ TS 3.4.9.3 }

\textbf{TS 3.4.9.3 } \newline
\textbf{Samhita Paata} \newline

मव॑ रुन्धे मद्ध्य॒तो धा॒तारं॑ करोति मद्ध्य॒त ए॒वैनं॒ ग्राम॑स्य दधात्ये॒ता ए॒व निर्व॑पे॒ज्ज्योगा॑मयावी॒ छन्दाꣳ॑सि॒ वै देवि॑का॒श्छन्दाꣳ॑सि॒ खलु॒ वा ए॒तम॒भि म॑न्यन्ते॒ यस्य॒ ज्योगा॒मय॑ति॒ छन्दो॑भिरे॒वैन॑-मग॒दं क॑रोति मद्ध्य॒तो धा॒तारं॑ करोति मद्ध्य॒तो वा ए॒तस्याक्लृ॑प्तं॒ ॅयस्य॒ ज्योगा॒मय॑ति मद्ध्य॒त ए॒वास्य॒ तेन॑ कल्पयत्ये॒ता ए॒व नि - [  ] \newline

\textbf{Pada Paata} \newline

अवेति॑ । रु॒न्धे॒ । म॒द्ध्य॒तः । धा॒तार᳚म् । क॒रो॒ति॒ । म॒द्ध्य॒तः । ए॒व । ए॒न॒म् । ग्राम॑स्य । द॒धा॒ति॒ । ए॒ताः । ए॒व । निरिति॑ । व॒पे॒त् । ज्योगा॑मया॒वीति॒ ज्योक् - आ॒म॒या॒वी॒ । छन्दाꣳ॑सि । वै । देवि॑काः । छन्दाꣳ॑सि । खलु॑ । वै । ए॒तम् । अ॒भीति॑ । म॒न्य॒न्ते॒ । यस्य॑ । ज्योक् । आ॒मय॑ति । छन्दो॑भि॒रिति॒ छन्दः॑ - भिः॒ । ए॒व । ए॒न॒म् । अ॒ग॒दम् । क॒रो॒ति॒ । म॒द्ध्य॒तः । धा॒तार᳚म् । क॒रो॒ति॒ । म॒द्ध्य॒तः । वै । ए॒तस्य॑ । अक्लृ॑प्तम् । यस्य॑ । ज्योक् । आ॒मय॑ति । म॒द्ध्य॒तः । ए॒व । अ॒स्य॒ । तेन॑ । क॒ल्प॒य॒ति॒ । ए॒ताः । ए॒व । निरिति॑ ।  \newline




\markright{ TS 3.4.9.4  \hfill https://www.vedavms.in \hfill}
\addcontentsline{toc}{section}{ TS 3.4.9.4 }
\section*{ TS 3.4.9.4 }

\textbf{TS 3.4.9.4 } \newline
\textbf{Samhita Paata} \newline

र्व॑पे॒द्यं ॅय॒ज्ञो नोप॒नमे॒च्छन्दाꣳ॑सि॒ वै देवि॑का॒श्छन्दाꣳ॑सि॒ खलु॒ वा ए॒तं नोप॑ नमन्ति॒ यं ॅय॒ज्ञो नोप॒नम॑ति प्रथ॒मं धा॒तारं॑ करोति मुख॒त ए॒वास्मै॒ छन्दाꣳ॑सि दधा॒त्युपै॑नं ॅय॒ज्ञो न॑मत्ये॒ता ए॒व निव॑र्पेदीजा॒नश्छन्दाꣳ॑सि॒ वै देवि॑का या॒तया॑मानीव॒ खलु॒ वा ए॒तस्य॒ छन्दाꣳ॑सि॒ य ई॑जा॒न उ॑त्त॒मं धा॒तारं॑ करो - [  ] \newline

\textbf{Pada Paata} \newline

व॒पे॒त् । यम् । य॒ज्ञ्ः । न । उ॒प॒नमे॒दित्यु॑प - नमे᳚त् । छन्दाꣳ॑सि । वै । देवि॑काः । छन्दाꣳ॑सि । खलु॑ । वै । ए॒तम् । न । उपेति॑ । न॒म॒न्ति॒ । यम् । य॒ज्ञ्ः । न । उ॒प॒नम॒तीत्यु॑प - नम॑ति । प्र॒थ॒मम् । धा॒तार᳚म् । क॒रो॒ति॒ । मु॒ख॒तः । ए॒व । अ॒स्मै॒ । छन्दाꣳ॑सि । द॒धा॒ति॒ । उपेति॑ । ए॒न॒म् । य॒ज्ञ्ः । न॒म॒ति॒ । ए॒ताः । ए॒व । निरिति॑ । व॒पे॒त् । ई॒जा॒नः । छन्दाꣳ॑सि । वै । देवि॑काः । या॒तया॑मा॒नीति॑ या॒त - या॒मा॒नि॒ । इ॒व॒ । खलु॑ । वै । ए॒तस्य॑ । छन्दाꣳ॑सि । यः । ई॒जा॒नः । उ॒त्त॒ममित्यु॑त् - त॒मम् । धा॒तार᳚म् । क॒रो॒ति॒ ।  \newline




\markright{ TS 3.4.9.5  \hfill https://www.vedavms.in \hfill}
\addcontentsline{toc}{section}{ TS 3.4.9.5 }
\section*{ TS 3.4.9.5 }

\textbf{TS 3.4.9.5 } \newline
\textbf{Samhita Paata} \newline

-त्यु॒परि॑ष्टादे॒वास्मै॒ छन्दाꣳ॒॒स्यया॑तयामा॒न्यव॑ रुन्ध॒ उपै॑न॒मुत्त॑रो य॒ज्ञो न॑मत्ये॒ता ए॒व निव॑र्पे॒द्यं मे॒धा नोप॒नमे॒च्छन्दाꣳ॑सि॒ वै देवि॑का॒श्छन्दाꣳ॑सि॒ खलु॒ वा ए॒तं नोप॑ नमन्ति॒ यं मे॒धा नोप॒नम॑ति प्रथ॒मं धा॒तारं॑ करोति मुख॒त ए॒वास्मै॒ छन्दाꣳ॑सि दधा॒त्युपै॑नं मे॒धा न॑मत्ये॒ता ए॒व निव॑र्पे॒ - [  ] \newline

\textbf{Pada Paata} \newline

उ॒परि॑ष्टात् । ए॒व । अ॒स्मै॒ । छन्दाꣳ॑सि । अया॑तयामा॒नीत्यया॑त - या॒मा॒नि॒ । अवेति॑ । रु॒न्धे॒ । उपेति॑ । ए॒न॒म् । उत्त॑र॒ इत्युत् - त॒रः॒ । य॒ज्ञ्ः । न॒म॒ति॒ । ए॒ताः । ए॒व । निरिति॑ । व॒पे॒त् । यम् । मे॒धा । न । उ॒प॒नमे॒दित्यु॑प - नमे᳚त् । छन्दाꣳ॑सि । वै । देवि॑काः । छन्दाꣳ॑सि । खलु॑ । वै । ए॒तम् । न । उपेति॑ । न॒म॒न्ति॒ । यम् । मे॒धा । न । उ॒प॒नम॒तीत्यु॑प - नम॑ति । प्र॒थ॒मम् । धा॒तार᳚म् । क॒रो॒ति॒ । मु॒ख॒तः । ए॒व । अ॒स्मै॒ । छन्दाꣳ॑सि । द॒धा॒ति॒ । उपेति॑ । ए॒न॒म् । मे॒धा । न॒म॒ति॒ । ए॒ताः । ए॒व । निरिति॑ । व॒पे॒त् ।  \newline




\markright{ TS 3.4.9.6  \hfill https://www.vedavms.in \hfill}
\addcontentsline{toc}{section}{ TS 3.4.9.6 }
\section*{ TS 3.4.9.6 }

\textbf{TS 3.4.9.6 } \newline
\textbf{Samhita Paata} \newline

द्रुक्का॑म॒श्छन्दाꣳ॑सि॒ वै देवि॑का॒श्छन्दाꣳ॑सीव॒ खलु॒ वै रुक् छन्दो॑भिरे॒वास्मि॒न् रुचं॑ दधातिक्षी॒रे भ॑वन्ति॒ रुच॑मे॒वास्मि॑न् दधति मद्ध्य॒तो धा॒तारं॑ करोति मद्ध्य॒त ए॒वैनꣳ॑ रु॒चो द॑धातिगाय॒त्री वा अनु॑मतिस्त्रि॒ष्टुग्रा॒का जग॑ती सिनीवा॒ल्य॑नु॒ष्टुप् कु॒हूर्द्धा॒ता व॑षट्का॒रः पू᳚र्वप॒क्षो रा॒काऽप॑रप॒क्षः कु॒हूर॑मावा॒स्या॑ सिनीवा॒ली पौ᳚र्णमा॒स्यनु॑मतिश्च॒न्द्रमा॑ धा॒ताऽष्टौ - [  ] \newline

\textbf{Pada Paata} \newline

रुक्का॑म॒ इति॒ रुक्-का॒मः॒ । छन्दाꣳ॑सि । वै । देवि॑काः । छन्दाꣳ॑सि । इ॒व॒ । खलु॑ । वै । रुक् । छन्दो॑भि॒रिति॒ छन्दः॑-भिः॒ । ए॒व । अ॒स्मि॒न्न् । रुच᳚म् । द॒धा॒ति॒ । क्षी॒रे । भ॒व॒न्ति॒ । रुच᳚म् । ए॒व । अ॒स्मि॒न्न् । द॒ध॒ति॒ । म॒द्ध्य॒तः । धा॒तार᳚म् । क॒रो॒ति॒ । म॒द्ध्य॒तः । ए॒व । ए॒न॒म् । रु॒चः । द॒धा॒ति॒ । गा॒य॒त्री । वै । अनु॑मति॒रित्य॑नु - म॒तिः॒ । त्रि॒ष्टुक् । रा॒का । जग॑ती । सि॒नी॒वा॒ली । अ॒नु॒ष्टुबित्य॑नु - स्तुप् । कु॒हूः । धा॒ता । व॒ष॒ट्का॒र इति॑ वषट् - का॒रः । पू॒र्व॒प॒क्ष इति॑ पूर्व - प॒क्षः । रा॒का । अ॒प॒र॒प॒क्ष इत्य॑पर - प॒क्षः । कु॒हूः । अ॒मा॒वा॒स्येत्य॑मा - वा॒स्या᳚ । सि॒नी॒वा॒ली । पौ॒र्ण॒मा॒सीति॑ पौर्ण - मा॒सी । अनु॑मति॒रित्य॑नु - म॒तिः॒ । च॒न्द्रमाः᳚ । धा॒ता । अ॒ष्टौ ।  \newline




\markright{ TS 3.4.9.7  \hfill https://www.vedavms.in \hfill}
\addcontentsline{toc}{section}{ TS 3.4.9.7 }
\section*{ TS 3.4.9.7 }

\textbf{TS 3.4.9.7 } \newline
\textbf{Samhita Paata} \newline

वस॑वो॒ऽष्टाक्ष॑रा गाय॒त्र्येका॑दश रु॒द्रा एका॑दशाक्षरा त्रि॒ष्टुब् द्वाद॑शाऽऽ*दि॒त्या द्वाद॑शाक्षरा॒ जग॑ती प्र॒जाप॑तिरनु॒ष्टुब् धा॒ता व॑षट्का॒र ए॒तद्वै देवि॑काः॒ सर्वा॑णि च॒ छन्दाꣳ॑सि॒ सर्वा᳚श्च दे॒वता॑ वषट्का॒रस्ता यथ् स॒ह सर्वा॑ नि॒र्वपे॑दीश्व॒रा ए॑नं प्र॒दहो॒ द्वे प्र॑थ॒मे नि॒रुप्य॑ धा॒तुस्तृ॒तीयं॒ निव॑र्पे॒त् तथो॑ ए॒वोत्त॑रे॒ निव॑र्पे॒त् तथै॑नं॒ न प्रद॑ह॒न्त्य ( ) थो॒ यस्मै॒ कामा॑य निरु॒प्यन्ते॒ तमे॒वाऽऽ*भि॒रुपा᳚ऽऽ*प्नोति ॥ \newline

\textbf{Pada Paata} \newline

वस॑वः । अ॒ष्टाक्ष॒रेत्य॒ष्टा - अ॒क्ष॒रा॒ । गा॒य॒त्री । एका॑दश । रु॒द्राः । एका॑दशाक्ष॒रेत्येका॑दश - अ॒क्ष॒रा॒ । त्रि॒ष्टुप् । द्वाद॑श । आ॒दि॒त्याः । द्वाद॑शाक्ष॒रेति॒ द्वाद॑श-अ॒क्ष॒रा॒ । जग॑ती । प्र॒जाप॑ति॒रिति॑ प्र॒जा - प॒तिः॒ । अ॒नु॒ष्टुबित्य॑नु-स्तुप् । धा॒ता । व॒ष॒ट्का॒र इति॑ वषट् - का॒रः । ए॒तत् । वै । देवि॑काः । सर्वा॑णि । च॒ । छन्दाꣳ॑सि । सर्वाः᳚ । च॒ । दे॒वताः᳚ । व॒ष॒ट्का॒र इति॑ वषट् - का॒रः । ताः । यत् । स॒ह । सर्वाः᳚ । नि॒र्वपे॒दिति॑ निः - वपे᳚त् । ई॒श्व॒राः । ए॒न॒म् । प्र॒दह॒ इति॑ प्र - दहः॑ । द्वे इति॑ । प्र॒थ॒मे इति॑ । नि॒रुप्येति॑ निः - उप्य॑ । धा॒तुः । तृ॒तीय᳚म् । निरिति॑ । व॒पे॒त् । तथो॒ इति॑ । ए॒व । उत्त॑रे॒ इत्युत् - त॒रे॒ । निरिति॑ । व॒पे॒त् । तथा᳚ । ए॒न॒म् । न । प्रेति॑ । द॒ह॒न्ति॒ ( ) । अथो॒ इति॑ । यस्मै᳚ । कामा॑य । नि॒रु॒प्यन्त॒ इति॑ निः - उ॒प्यन्ते᳚ । तम् । ए॒व । आ॒भिः॒ । उपेति॑ । आ॒प्नो॒ति॒ ॥  \newline




\markright{ TS 3.4.10.1  \hfill https://www.vedavms.in \hfill}
\addcontentsline{toc}{section}{ TS 3.4.10.1 }
\section*{ TS 3.4.10.1 }

\textbf{TS 3.4.10.1 } \newline
\textbf{Samhita Paata} \newline

वास्तो᳚ष्पते॒ प्रति॑ जानी ह्य॒स्मान्थ् स्वा॑वे॒शो अ॑नमी॒वो भ॑वानः । यत् त्वेम॑हे॒ प्रति॒तन्नो॑ जुषस्व॒ शन्न॑ एधि द्वि॒पदे॒ शञ्चतु॑ष्पदे ॥वास्तो᳚ष्पते श॒ग्मया॑ सꣳ॒॒ सदा॑ते सक्षी॒महि॑ र॒ण्वया॑ गातु॒मत्या᳚ । आवः॒ क्षेम॑ उ॒त योगे॒ वर॑न्नो यू॒यं पा॑त स्व॒स्तिभिः॒ सदा॑नः ॥ यथ् सा॒यं प्रा॑तरग्निहो॒त्रं जु॒होत्या॑हुतीष्ट॒का ए॒व ता उप॑ धत्ते॒ - [  ] \newline

\textbf{Pada Paata} \newline

वास्तोः᳚ । प॒ते॒ । प्रतीति॑ । जा॒नी॒हि॒ । अ॒स्मान् । स्वा॒वे॒श इति॑ सु-आ॒वे॒शः । अ॒न॒मी॒वः । भ॒व॒ । नः॒ ॥ यत् । त्वा॒ । ईम॑हे । प्रतीति॑ । तत् । नः॒ । जु॒ष॒स्व॒ । शम् । नः॒ । ए॒धि॒ । द्वि॒पद॒ इति॑ द्वि - पदे᳚ । शम् । चतु॑ष्पद॒ इति॒ चतुः॑ - प॒दे॒ ॥ वास्तोः᳚ । प॒ते॒ । श॒ग्मया᳚ । सꣳ॒॒सदेति॑ सं - सदा᳚ । ते॒ । स॒क्षी॒महि॑ । र॒ण्वया᳚ । गा॒तु॒मत्येति॑ गातु - मत्या᳚ ॥ आवः॑ । क्षेमे᳚ । उ॒त । योगे᳚ । वर᳚म् । नः॒ । यू॒यम् । पा॒त॒ । स्व॒स्तिभि॒रिति॑ स्व॒स्ति - भिः॒ । सदा᳚ । नः॒ ॥ यत् । सा॒यंप्रा॑त॒रिति॑ सा॒यं - प्रा॒तः॒ । अ॒ग्नि॒हो॒त्रमित्य॑ग्नि - हो॒त्रम् । जु॒होति॑ । आ॒हु॒ती॒ष्ट॒का इत्या॑हुति - इ॒ष्ट॒काः । ए॒व । ताः । उपेति॑ । ध॒त्ते॒ ।  \newline




\markright{ TS 3.4.10.2  \hfill https://www.vedavms.in \hfill}
\addcontentsline{toc}{section}{ TS 3.4.10.2 }
\section*{ TS 3.4.10.2 }

\textbf{TS 3.4.10.2 } \newline
\textbf{Samhita Paata} \newline

यज॑मानोऽहोरा॒त्राणि॒ वा ए॒तस्येष्ट॑का॒ य आहि॑ताग्नि॒र्यथ् सा॒यं प्रा॑तर्जु॒होत्य॑होरा॒त्राण्ये॒वा ऽऽ*प्त्वेष्ट॑काः कृ॒त्वोप॑ धत्ते॒ दश॑ समा॒नत्र॑ जुहोति॒ दशा᳚क्षरा वि॒राड् वि॒राज॑मे॒वाऽऽ*प्त्वेष्ट॑कां कृ॒त्वोप॑ ध॒त्तेऽथो॑ वि॒राज्ये॒व य॒ज्ञ्मा᳚प्नोति॒ चित्य॑श्चित्योऽस्य भवति॒ तस्मा॒द्यत्र॒ दशो॑षि॒त्वा प्र॒याति॒ तद्-य॑ज्ञ्वा॒स्त्ववा᳚स्त्वे॒व तद्यत् ततो᳚ऽर्वा॒चीनꣳ॑ - [  ] \newline

\textbf{Pada Paata} \newline

यज॑मानः । अ॒हो॒रा॒त्राणीत्य॑हः-रा॒त्राणि॑ । वै । ए॒तस्य॑ । इष्ट॑काः । यः । आहि॑ताग्नि॒रित्याहि॑त - अ॒ग्निः॒ । यत् । सा॒यंप्रा॑त॒रिति॑ सा॒यं-प्रा॒तः॒ । जु॒होति॑ । अ॒हो॒रा॒त्राणीत्य॑हः - रा॒त्राणि॑ । ए॒व । आ॒प्त्वा । इष्ट॑काः । कृ॒त्वा । उपेति॑ । ध॒त्ते॒ । दश॑ । स॒मा॒नत्र॑ । जु॒हो॒ति॒ । दशा᳚क्ष॒रेति॒ दश॑ - अ॒क्ष॒रा॒ । वि॒राडिति॑ वि - राट् । वि॒राज॒मिति॑ वि - राज᳚म् । ए॒व । आ॒प्त्वा । इष्ट॑काम् । कृ॒त्वा । उपेति॑ । ध॒त्ते॒ । अथो॒ इति॑ । वि॒राजीति॑ वि - राजि॑ । ए॒व । य॒ज्ञ्म् । आ॒प्नो॒ति॒ । चित्य॑श्चित्य॒ इति॒ चित्यः॑ - चि॒त्यः॒ । अ॒स्य॒ । भ॒व॒ति॒ । तस्मा᳚त् । यत्र॑ । दश॑ । उ॒षि॒त्वा । प्र॒यातीति॑ प्र - याति॑ । तत् । य॒ज्ञ्॒वा॒स्त्विति॑ यज्ञ्-वा॒स्तु । अवा᳚स्तु । ए॒व । तत् । यत् । ततः॑ । अ॒र्वा॒चीन᳚म् ।  \newline




\markright{ TS 3.4.10.3  \hfill https://www.vedavms.in \hfill}
\addcontentsline{toc}{section}{ TS 3.4.10.3 }
\section*{ TS 3.4.10.3 }

\textbf{TS 3.4.10.3 } \newline
\textbf{Samhita Paata} \newline

रु॒द्रः खलु॒ वै वा᳚स्तोष्प॒तिर्यदहु॑त्वा वास्तोष्प॒तीयं॑ प्रया॒याद् रु॒द्र ए॑नं भू॒त्वाऽग्निर॑नू॒त्थाय॑ हन्याद्वास्तोष्प॒तीयं॑ जुहोति भाग॒धेये॑नै॒वैनꣳ॑ शमयति॒ नाऽऽ*र्ति॒मार्च्छ॑ति॒ यज॑मानो॒ यद्यु॒क्ते जु॑हु॒याद्यथा॒ प्रया॑ते॒ वास्ता॒वाहु॑तिं जु॒होति॑ ता॒दृगे॒व तद्यदयु॑क्ते जुहु॒याद्यथा॒ क्षेम॒ आहु॑तिं जु॒होति॑ ता॒दृगे॒व तदहु॑तमस्य वास्तोष्प॒तीयꣳ॑ स्या॒ - [  ] \newline

\textbf{Pada Paata} \newline

रु॒द्रः । खलु॑ । वै । वा॒स्तो॒ष्प॒तिरिति॑ वास्तोः-प॒तिः । यत् । अहु॑त्वा । वा॒स्तो॒ष्प॒तीय॒मिति॑ वास्तोः - प॒तीय᳚म् । प्र॒या॒यादिति॑ प्र - या॒यात् । रु॒द्रः । ए॒न॒म् । भू॒त्वा । अ॒ग्निः । अ॒नू॒त्थायेत्य॑नु - उ॒त्थाय॑ । ह॒न्या॒त् । वा॒स्तो॒ष्प॒तीय॒मिति॑ वास्तोः - प॒तीय᳚म् । जु॒हो॒ति॒ । भा॒ग॒धेये॒नेति॑ भाग - धेये॑न । ए॒व । ए॒न॒म् । श॒म॒य॒ति॒ । न । आर्ति᳚म् । एति॑ । ऋ॒च्छ॒ति॒ । यज॑मानः । यत् । यु॒क्ते । जु॒हु॒यात् । यथा᳚ । प्रया॑त॒ इति॒ प्र - या॒ते॒ । वास्तौ᳚ । आहु॑ति॒मित्या-हु॒ति॒म् । जु॒होति॑ । ता॒दृक् । ए॒व । तत् । यत् । अयु॑क्ते । जु॒हु॒यात् । यथा᳚ । क्षेमे᳚ । आहु॑ति॒मित्या - हु॒ति॒म् । जु॒होति॑ । ता॒दृक् । ए॒व । तत् । अहु॑तम् । अ॒स्य॒ । वा॒स्तो॒ष्प॒तीय॒मिति॑ वास्तोः - प॒तीय᳚म् । स्यात् ।  \newline




\markright{ TS 3.4.10.4  \hfill https://www.vedavms.in \hfill}
\addcontentsline{toc}{section}{ TS 3.4.10.4 }
\section*{ TS 3.4.10.4 }

\textbf{TS 3.4.10.4 } \newline
\textbf{Samhita Paata} \newline

द्दक्षि॑णो यु॒क्तो भव॑ति स॒व्योऽयु॒क्तोऽथ॑ वास्तोष्प॒तीयं॑ जुहोत्यु॒भय॑मे॒वाऽ क॒रप॑रिवर्गमे॒वैनꣳ॑ शमयति॒ यदेक॑या जुहु॒याद्द॑र्विहो॒मं कु॑र्यात् पुरोऽनुवा॒क्या॑ म॒नूच्य॑ या॒ज्य॑या जुहोति सदेव॒त्वाय॒ यद्धु॒त आ॑द॒द्ध्याद् रु॒द्रं गृ॒हान॒न्वारो॑हये॒द्-यद॑व॒क्षाणा॒न्यसं॑ प्रक्षाप्य प्रया॒याद्यथा॑ यज्ञ्वेश॒सं ॅवा॒ऽऽदह॑नं ॅवा ता॒दृगे॒व तद॒यन्ते॒ योनि॑र्.ऋ॒त्विय॒ इत्य॒रण्योः᳚ स॒मारो॑हय - [  ] \newline

\textbf{Pada Paata} \newline

दक्षि॑णः । यु॒क्तः । भव॑ति । स॒व्यः । अयु॑क्तः । अथ॑ । वा॒स्तो॒ष्प॒तीय॒मिति॑ वास्तोः -प॒तीय᳚म् । जु॒हो॒ति॒ । उ॒भय᳚म् । ए॒व । अ॒कः॒ । अप॑रिवर्ग॒मित्यप॑रि - व॒र्ग॒म् । ए॒व । ए॒न॒म् । श॒म॒य॒ति॒ । यत् । एक॑या । जु॒हु॒यात् । द॒र्वि॒हो॒ममिति॑ दर्वि - हो॒मम् । कु॒र्या॒त् । पु॒रो॒नु॒वा॒क्या॑मिति॑ पुरः - अ॒नु॒वा॒क्या᳚म् । अ॒नूच्येत्य॑नु - उच्य॑ । या॒ज्य॑या । जु॒हो॒ति॒ । स॒दे॒व॒त्वायेति॑ सदेव - त्वाय॑ । यत् । हु॒ते । आ॒द॒द्ध्यादित्या᳚ - द॒द्ध्यात् । रु॒द्रम् । गृ॒हान् । अ॒न्वारो॑हये॒दित्य॑नु - आरो॑हयेत् । यत् । अ॒व॒क्षाणा॒नीत्य॑व - क्षाणा॑नि । अस॑प्रंक्षा॒प्येत्यसं᳚ - प्र॒क्षा॒प्य॒ । प्र॒या॒यादिति॑ प्र - या॒यात् । यथा᳚ । य॒ज्ञ्॒वे॒श॒समिति॑ यज्ञ् - वे॒श॒सम् । वा॒ । आ॒दह॑न॒मित्या᳚ - दह॑नम् । वा॒ । ता॒दृक् । ए॒व । तत् । अ॒यम् । ते॒ । योनिः॑ । ऋ॒त्वियः॑ । इति॑ । अ॒रण्योः᳚ । स॒मारो॑हय॒तीति॑ सं - आरो॑हयति ।  \newline




\markright{ TS 3.4.10.5  \hfill https://www.vedavms.in \hfill}
\addcontentsline{toc}{section}{ TS 3.4.10.5 }
\section*{ TS 3.4.10.5 }

\textbf{TS 3.4.10.5 } \newline
\textbf{Samhita Paata} \newline

-त्ये॒ष वा अ॒ग्नेर्योनिः॒ स्व ए॒वैनं॒ ॅयोनौ॑ स॒मारो॑हय॒त्यथो॒ खल्वा॑हु॒र्यद॒रण्योः᳚ स॒मारू॑ढो॒ नश्ये॒दुद॑स्या॒ग्निः सी॑देत् पुनरा॒धेयः॑ स्या॒दिति॒ या ते॑ अग्ने य॒ज्ञिया॑ त॒नूस्तयेह्या रो॒हेत्या॒त्मन्थ् स॒मारो॑हयते॒ यज॑मानो॒ वा अ॒ग्नेर्योनिः॒ स्वाया॑मे॒वैनं॒ ॅयोन्याꣳ॑ स॒मारो॑हयते ॥ \newline

\textbf{Pada Paata} \newline

ए॒षः । वै । अ॒ग्नेः । योनिः॑ । स्वे । ए॒व । ए॒न॒म् । योनौ᳚ । स॒मारो॑हय॒तीति॑ सं-आरो॑हयति । अथो॒ इति॑ । खलु॑ । आ॒हुः॒ । यत् । अ॒रण्योः᳚ । स॒मारू॑ढ॒ इति॑ सं-आरू॑ढः । नश्ये᳚त् । उदिति॑ । अ॒स्य॒ । अ॒ग्निः । सी॒दे॒त् । पु॒न॒रा॒धेय॒ इति॑ पुनः - आ॒धेयः॑ । स्या॒त् । इति॑ । या । ते॒ । अ॒ग्न॒ । य॒ज्ञिया᳚ । त॒नूः । तया᳚ । एति॑ । इ॒हि॒ । एति॑ । रो॒ह॒ । इति॑ । आ॒त्मन्न् । स॒मारो॑हयत॒ इति॑ सं-आरो॑हयते । यज॑मानः । वै । अ॒ग्नेः । योनिः॑ । स्वाया᳚म् । ए॒व । ए॒न॒म् । योन्या᳚म् । स॒मारो॑हयत॒ इति॑ सं - आरो॑हयते ॥  \newline




\markright{ TS 3.4.11.1  \hfill https://www.vedavms.in \hfill}
\addcontentsline{toc}{section}{ TS 3.4.11.1 }
\section*{ TS 3.4.11.1 }

\textbf{TS 3.4.11.1 } \newline
\textbf{Samhita Paata} \newline

त्वम॑ग्ने बृ॒हद्वयो॒ दधा॑सि देव दा॒शुषे᳚ । क॒विर्गृ॒हप॑ति॒र्युवा᳚ ॥ह॒व्य॒वाड॒ग्निर॒जरः॑ पि॒ता नो॑ वि॒भुर्वि॒भावा॑ सु॒दृशी॑को अ॒स्मे । सु॒गा॒र्॒.ह॒प॒त्याः समिषो॑ दिदीह्यस्म॒द्रिय॒ख्सं मि॑मीहि॒ श्रवाꣳ॑सि ॥ त्वं च॑ सोम नो॒ वशो॑ जी॒वातुं॒ न म॑रामहे । प्रि॒यस्तो᳚त्रो॒ वन॒स्पतिः॑ ॥ ब्र॒ह्मा दे॒वानां᳚ पद॒वीः क॑वी॒नामृषि॒र्विप्रा॑णां महि॒षो मृ॒गाणां᳚ । श्ये॒नो गृद्ध्रा॑णाꣳ॒॒ स्वधि॑ति॒ र्वना॑नाꣳ॒॒ सोमः॑ - [  ] \newline

\textbf{Pada Paata} \newline

त्वम् । अ॒ग्ने॒ । बृ॒हत् । वयः॑ । दधा॑सि । दे॒व॒ । दा॒शुषे᳚ ॥ क॒विः । गृ॒हप॑ति॒रिति॑ गृ॒ह - प॒तिः॒ । युवा᳚ ॥ ह॒व्य॒वाडिति॑ हव्य - वाट् । अ॒ग्निः । अ॒जरः॑ । पि॒ता । नः॒ । वि॒भुरिति॑ वि - भुः । वि॒भावेति॑ वि - भावा᳚ । सु॒दृशी॑क॒ इति॑ सु - दृशी॑कः । अ॒स्मे इति॑ ॥ सु॒गा॒र्॒.ह॒प॒त्या इति॑ सु - गा॒र्॒.ह॒प॒त्याः । समिति॑ । इषः॑ । दि॒दी॒हि॒ । ॒स्म॒द्रिय॒गित्य॑स्म - द्रिय॑क् । समिति॑ । मि॒मी॒हि॒ । श्रवाꣳ॑सि ॥ त्वम् । च॒ । सो॒म॒ । नः॒ । वशः॑ । जी॒वातु᳚म् । न । म॒रा॒म॒हे॒ ॥ प्रि॒यस्तो᳚त्र॒ इति॑ प्रि॒य - स्तो॒त्रः॒ । वन॒स्पतिः॑ ॥ ब्र॒ह्मा । दे॒वाना᳚म् । प॒द॒वीरिति॑ पद - वीः । क॒वी॒नाम् । ऋषिः॑ । विप्रा॑णाम् । म॒हि॒षः । मृ॒गाणा᳚म् ॥ श्ये॒नः । गृद्ध्रा॑णाम् । स्वधि॑ति॒रिति॒ स्व - धि॒तिः॒ । वना॑नाम् । सोमः॑ ।  \newline




\markright{ TS 3.4.11.2  \hfill https://www.vedavms.in \hfill}
\addcontentsline{toc}{section}{ TS 3.4.11.2 }
\section*{ TS 3.4.11.2 }

\textbf{TS 3.4.11.2 } \newline
\textbf{Samhita Paata} \newline

प॒वित्र॒मत्ये॑ति॒ रेभन्न्॑ ॥ आ वि॒श्वदे॑वꣳ॒॒ सत्प॑तिꣳ सू॒क्तैर॒द्या वृ॑णीमहे । स॒त्यस॑वꣳ सवि॒तारं᳚ ॥ आस॒त्येन॒ रज॑सा॒ वर्त॑मानो निवे॒शय॑न्न॒मृतं॒ मर्त्य॑ञ्च । हि॒र॒ण्यये॑न सवि॒ता रथे॒नाऽऽ* दे॒वोया॑ति॒ भुव॑ना वि॒पश्यन्न्॑ ॥ यथा॑ नो॒ अदि॑तिः॒ कर॒त् पश्वे॒ नृभ्यो॒ यथा॒ गवे᳚ । यथा॑ तो॒काय॑ रु॒द्रियं᳚ ॥ मा न॑स्तो॒के तन॑ये॒ मा न॒ आयु॑षि॒ मा नो॒ गोषु॒ मा- [  ] \newline

\textbf{Pada Paata} \newline

प॒वित्र᳚म् । अतीति॑ । ए॒ति॒ । रेभन्न्॑ ॥ एति॑ । वि॒श्वदे॑व॒मिति॑ वि॒श्व - दे॒व॒म् । सत्प॑ति॒मिति॒ सत् - प॒ति॒म् । सू॒क्तैरिति॑ सु-उ॒क्तैः । अ॒द्य । वृ॒णी॒म॒हे॒ ॥ स॒त्यस॑व॒मिति॑ स॒त्य-स॒व॒म् । स॒वि॒तार᳚म् ॥ एति॑ । स॒त्येन॑ । रज॑सा । वर्त॑मानः । नि॒वे॒शय॒न्निति॑ नि-वे॒शयन्न्॑ । अ॒मृत᳚म् । मर्त्य᳚म् । च॒ ॥ हि॒र॒ण्यये॑न । स॒वि॒ता । रथे॑न । एति॑ । दे॒वः । या॒ति॒ । भुव॑ना । वि॒पश्य॒न्निति॑ वि-पश्यन्न्॑ ॥ यथा᳚ । नः॒ । अदि॑तिः । कर॑त् । पश्वे᳚ । नृभ्य॒ इति॒ नृ - भ्यः॒ । यथा᳚ । गवे᳚ ॥ यथा᳚ । तो॒काय॑ । रु॒द्रिय᳚म् ॥ मा । नः॒ । तो॒के । तन॑ये । मा । नः॒ । आयु॑षि । मा । नः॒ । गोषु॑ । मा ।  \newline




\markright{ TS 3.4.11.3  \hfill https://www.vedavms.in \hfill}
\addcontentsline{toc}{section}{ TS 3.4.11.3 }
\section*{ TS 3.4.11.3 }

\textbf{TS 3.4.11.3 } \newline
\textbf{Samhita Paata} \newline

नो॒ अश्वे॑षु रीरिषः । वी॒रान् मानो॑ रुद्र भामि॒तो व॑धीर्.ह॒विष्म॑न्तो॒ नम॑सा विधेम ते ॥ उ॒द॒प्रुतो॒ न वयो॒ रक्ष॑माणा॒ वाव॑दतो अ॒भ्रिय॑स्येव॒ घोषाः᳚ । गि॒रि॒भ्रजो॒ नोर्मयो॒ मद॑न्तो॒ बृह॒स्पति॑म॒भ्य॑र्का अ॑नावन्न् ॥ हꣳ॒॒सैरि॑व॒ सखि॑भि॒र्वाव॑दद्भिरश्म॒न्- मया॑नि॒ नह॑ना॒ व्यस्यन्न्॑ । बृह॒स्पति॑रभि॒कनि॑क्रद॒द्गा उ॒त प्रास्तौ॒दुच्च॑ वि॒द्वाꣳ अ॑गायत् ॥एन्द्र॑ सान॒सिꣳ र॒यिꣳ - [  ] \newline

\textbf{Pada Paata} \newline

नः॒ । अश्वे॑षु । री॒रि॒षः॒ ॥ वी॒रान् । मा । नः॒ । रु॒द्र॒ । भा॒मि॒तः । व॒धीः॒ । ह॒विष्म॑न्तः । नम॑सा । वि॒धे॒म॒ । ते॒ ॥ उ॒द॒प्रुत॒ इत्यु॑द - प्रुतः॑ । न । वयः॑ । रक्ष॑माणाः । वाव॑दतः । अ॒भ्रिय॑स्य । इ॒व॒ । घोषाः᳚ ॥ गि॒रि॒भ्रज॒ इति॑ गिरि - भ्रजः॑ । न । ऊ॒र्मयः॑ । मद॑न्तः । बृह॒स्पति᳚म् । अ॒भीति॑ । अ॒र्काः । अ॒ना॒व॒न्न् ॥ हꣳ॒॒सैः । इ॒व॒ । सखि॑भि॒रिति॒ सखि॑ - भिः॒ । वाव॑दद्भि॒रिति॒ वाव॑दत्- भिः॒ । अ॒श्म॒न्मया॒नीत्य॑श्मन्न् - मया॑नि । नह॑ना । व्यस्य॒न्निति॑ वि-अस्यन्न्॑ ॥ बृह॒स्पतिः॑ । अ॒भि॒कनि॑क्रद॒दित्य॑भि - कनि॑क्रदत् । गाः । उ॒त । प्रेति॑ । अ॒स्तौ॒त् । उदिति॑ । च॒ । वि॒द्वान् । अ॒गा॒य॒त् ॥ एति॑ । इ॒न्द्र॒ । सा॒न॒सिम् । र॒यिम् ।  \newline




\markright{ TS 3.4.11.4  \hfill https://www.vedavms.in \hfill}
\addcontentsline{toc}{section}{ TS 3.4.11.4 }
\section*{ TS 3.4.11.4 }

\textbf{TS 3.4.11.4 } \newline
\textbf{Samhita Paata} \newline

स॒जित्वा॑नꣳ सदा॒सहं᳚ । वर्.षि॑ष्ठमू॒तये॑ भर ॥प्र स॑साहिषे पुरुहूत॒ शत्रू॒न् ज्येष्ठ॑स्ते॒ शुष्म॑ इ॒ह रा॒तिर॑स्तु । इन्द्राऽऽ* भ॑र॒ दक्षि॑णेना॒ वसू॑नि॒ पतिः॒ सिन्धू॑नामसि रे॒वती॑नां ॥ त्वꣳ सु॒तस्य॑ पी॒तये॑ स॒द्यो वृ॒द्धो अ॑जायथाः । इन्द्र॒ ज्यैष्ठ्या॑य सुक्रतो ॥ भुव॒स्त्वमि॑न्द्र॒ ब्रह्म॑णा म॒हान् भुवो॒ विश्वे॑षु॒ सव॑नेषु य॒ज्ञियः॑ । भुवो॒ नॄꣳश्च्यौ॒त्नो विश्व॑स्मि॒न् भरे॒ ज्येष्ठ॑श्च॒ मन्त्रो॑ - [  ] \newline

\textbf{Pada Paata} \newline

स॒जित्वा॑न॒मिति॑ स - जित्वा॑नम् । स॒दा॒सह॒मिति॑ सदा - सह᳚म् ॥ वर्.षि॑ष्ठम् । ऊ॒तये᳚ । भ॒र॒ ॥ प्रेति॑ । स॒सा॒हि॒षे॒ । पु॒रु॒हू॒तेति॑ पुरु-हू॒त॒ । शत्रून्॑ । ज्येष्ठः॑ । ते॒ । शुष्मः॑ । इ॒ह । रा॒तिः । अ॒स्तु॒ ॥ इन्द्र॑ । एति॑ । भ॒र॒ । दक्षि॑णेन । वसू॑नि । पतिः॑ । सिन्धू॑नाम् । अ॒सि॒ । रे॒वती॑नाम् ॥ त्वम् । सु॒तस्य॑ । पी॒तये᳚ । स॒द्यः । वृ॒द्धः । अ॒जा॒य॒थाः॒ । इन्द्र॑ । ज्यैष्ठ्या॑य । सु॒क्र॒तो॒ इति॑ सु-क्र॒तो॒ ॥ भुवः॑ । त्वम् । इ॒न्द्र॒ । ब्रह्म॑णा । म॒हान् । भुवः॑ । विश्वे॑षु । सव॑नेषु । य॒ज्ञियः॑ ॥ भुवः॑ । नॄन् । च्यौ॒त्नः । विश्व॑स्मिन्न् । भरे᳚ । ज्येष्ठः॑ । च॒ । मन्त्रः॑ ।  \newline




\markright{ TS 3.4.11.5  \hfill https://www.vedavms.in \hfill}
\addcontentsline{toc}{section}{ TS 3.4.11.5 }
\section*{ TS 3.4.11.5 }

\textbf{TS 3.4.11.5 } \newline
\textbf{Samhita Paata} \newline

विश्वचर्.षणे ॥ मि॒त्रस्य॑ चर्.षणी॒धृतः॒ श्रवो॑ दे॒वस्य॑ सान॒सिं । स॒त्यं चि॒त्र श्र॑वस्तमं ॥मि॒त्रो जनान्॑ यातयति प्रजा॒नन् मि॒त्रो दा॑धार पृथि॒वीमु॒त द्यां । मि॒त्रः कृ॒ष्टीरनि॑मिषा॒ऽभि च॑ष्टे स॒त्याय॑ ह॒व्यं घृ॒तव॑द्-विधेम ॥ प्रसमि॑त्र॒ मर्तो॑ अस्तु॒ प्रय॑स्वा॒न्॒. यस्त॑ आदित्य॒ शिक्ष॑ति व्र॒तेन॑ । न ह॑न्यते॒ न जी॑यते॒ त्वोतो॒ नैन॒मꣳहो॑ अश्नो॒त्यन्ति॑तो॒ न दू॒रात् ॥ य- [  ] \newline

\textbf{Pada Paata} \newline

वि॒श्व॒च॒र्॒.ष॒ण॒ इति॑ विश्व - च॒र्॒.ष॒णे॒ ॥ मि॒त्रस्य॑ । च॒र्॒.ष॒णी॒धृत॒ इति॑ चर्.षणी - धृतः॑ । श्रवः॑ । दे॒वस्य॑ । सा॒न॒सिम् ॥ स॒त्यम् । चि॒त्रश्र॑वस्तम॒मिति॑ चि॒त्रश्र॑वः - त॒म॒म् ॥ मि॒त्रः । जनान्॑ । या॒त॒य॒ति॒ । प्र॒जा॒नन्निति॑ प्र - जा॒नन् । मि॒त्रः । दा॒धा॒र॒ । पृ॒थि॒वीम् । उ॒त । द्याम् ॥ मि॒त्रः । कृ॒ष्टीः । अनि॑मि॒षेत्यनि॑ - मि॒षा॒ । अ॒भीति॑ । च॒ष्टे॒ । स॒त्याय॑ । ह॒व्यम् । घृ॒तव॒दिति॑ घृ॒त-व॒त् । वि॒धे॒म॒ ॥ प्रेति॑ । सः । मि॒त्र॒ । मर्तः॑ । अ॒स्तु॒ । प्रय॑स्वान् । यः । ते॒ । आ॒दि॒त्य॒ । शिक्ष॑ति । व्र॒तेन॑ ॥ न । ह॒न्य॒ते॒ । न । जी॒य॒ते॒ । त्वोतः॑ । न । ए॒न॒म् । अꣳहः॑ । अ॒श्नो॒ति॒ । अन्ति॑तः । न । दू॒रात् ॥ यत् ।  \newline




\markright{ TS 3.4.11.6  \hfill https://www.vedavms.in \hfill}
\addcontentsline{toc}{section}{ TS 3.4.11.6 }
\section*{ TS 3.4.11.6 }

\textbf{TS 3.4.11.6 } \newline
\textbf{Samhita Paata} \newline

च्चि॒द्धि ते॒ विशो॑ यथा॒ प्रदे॑व वरुण व्र॒तं । मि॒नी॒मसि॒ द्यवि॑द्यवि ॥यत् किञ्चे॒दं ॅव॑रुण॒ दैव्ये॒ जने॑ऽभिद्रो॒हं म॑नु॒ष्या᳚श्चरा॑मसि । अचि॑त्ती॒यत् तव॒ धर्मा॑ युयोपि॒ममा न॒स्तस्मा॒ देन॑सो देव रीरिषः ॥ कि॒त॒वासो॒ यद्रि॑ रि॒पुर्न दी॒वि यद्वा॑ घा स॒त्य मु॒तयन्न वि॒द्म । सर्वा॒ ता विष्य॑ शिथि॒रे ( ) व॑ दे॒वाथा॑ ते स्याम वरुण प्रि॒यासः॑ ॥ \newline

\textbf{Pada Paata} \newline

चि॒त् । हि । ते॒ । विशः॑ । य॒था॒ । प्रेति॑ । दे॒व॒ । व॒रु॒ण॒ । व्र॒तम् ॥ मि॒नी॒मसि॑ । द्यवि॑द्य॒वीति॒ द्यवि॑ - द्य॒वि॒ ॥ यत् । किम् । च॒ । इ॒दम् । व॒रु॒ण॒ । दैव्ये᳚ । जने᳚ । अ॒भि॒द्रो॒हमित्य॑भि - द्रो॒हम् । म॒नु॒ष्याः᳚ । चरा॑मसि ॥ अचि॑त्ती । यत् । तव॑ । धर्मा᳚ । यु॒यो॒पि॒म । मा । नः॒ । तस्मा᳚त् । एन॑सः । दे॒व॒ । री॒रि॒षः॒ ॥ कि॒त॒वासः॑ । यत् । रि॒रि॒पुः । न । दी॒वि । यत् । वा॒ । घ॒ । स॒त्यम् । उ॒त । यत् । न । वि॒द्म ॥ सर्वा᳚ । ता । वीति॑ । स्य॒ । शि॒थि॒रा ( ) । इ॒व॒ । दे॒व॒ । अथ॑ । ते॒ । स्या॒म॒ । व॒रु॒ण॒ । प्रि॒यासः॑ ॥  \newline






\end{document}