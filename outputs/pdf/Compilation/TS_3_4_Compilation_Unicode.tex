\documentclass[17pt]{extarticle}
\usepackage{babel}
\usepackage{fontspec}
\usepackage{polyglossia}
\usepackage{extsizes}

\usepackage{color}   %May be necessary if you want to color links
\usepackage{hyperref}
\hypersetup{
    colorlinks=true, %set true if you want colored links
    linktoc=all,     %set to all if you want both sections and subsections linked
    linkcolor=black,  %choose some color if you want links to stand out
}

\setmainlanguage{sanskrit}
\setotherlanguages{english} %% or other languages
\setlength{\parindent}{0pt}
\pagestyle{myheadings}
\newfontfamily\devanagarifont[Script=Devanagari]{AdishilaVedic}
\renewcommand{\theHsection}{\thepart.section.\thesection}

\newcommand{\VAR}[1]{}
\newcommand{\BLOCK}[1]{}




\begin{document}
\begin{titlepage}
    \begin{center}
 
\begin{sanskrit}
    { \Large
    कृष्ण यजुर्वेदीय तैत्तिरीय संहिता,पद,जटा,घन पाठः 
    }
    \\
    \vspace{2.5cm}
    \mbox{ \Large
    3.4      तृतीयकाण्डे चतुर्थः प्रश्नः - इष्टिहोमाभिधानं   }
\end{sanskrit}
\end{center}

\end{titlepage}
\tableofcontents
\phantomsection
\pagebreak

\markright{ TS 3.4.1.1  \hfill https://www.vedavms.in \hfill}

\section{ TS 3.4.1.1 }

\textbf{TS 3.4.1.1 } \newline
\textbf{Samhita Paata} \newline

वि वा ए॒तस्य॑ य॒ज्ञ् ऋ॑द्ध्यते॒ यस्य॑ ह॒विर॑ति॒रिच्य॑ते॒ सूर्यो॑ दे॒वो दि॑वि॒षद्भ्य॒ इत्या॑ह॒ बृह॒स्पति॑ना चै॒वास्य॑ प्र॒जाप॑तिना च य॒ज्ञ्स्य॒ व्यृ॑द्ध॒मपि॑ वपति॒ रक्षाꣳ॑सि॒ वा ए॒तत् प॒शुꣳ स॑चन्ते॒ यदे॑कदेव॒त्य॑ आल॑ब्धो॒ भूया॒न् भव॑ति॒ यस्या᳚स्ते॒ हरि॑तो॒ गर्भ॒ इत्या॑ह देव॒त्रैवैनां᳚ गमयति॒ रक्ष॑सा॒मप॑हत्या॒ आ व॑र्तन वर्त॒येत्या॑ह॒ - [  ] \newline

\textbf{Pada Paata} \newline

वीति॑ । वै । ए॒तस्य॑ । य॒ज्ञ्ः । ऋ॒द्ध्य॒ते॒ । यस्य॑ । ह॒विः । अ॒ति॒रिच्य॑त॒ इत्य॑ति - रिच्य॑ते । सूर्यः॑ । दे॒वः । दि॒वि॒षद्भ्य॒ इति॑ दिवि॒षत् - भ्यः॒ । इति॑ । आ॒ह॒ । बृह॒स्पति॑ना । च॒ । ए॒व । अ॒स्य॒ । प्र॒जाप॑ति॒नेति॑ प्र॒जा - प॒ति॒ना॒ । च॒ । य॒ज्ञ्स्य॑ । व्यृ॑द्ध॒मिति॒ वि - ऋ॒द्ध॒म् । अपीति॑ । व॒प॒ति॒ । रक्षाꣳ॑सि । वै । ए॒तत् । प॒शुम् । स॒च॒न्ते॒ । यत् । ए॒क॒दे॒व॒त्य॑ इत्ये॑क - दे॒व॒त्यः॑ । आल॑ब्ध॒ इत्या - ल॒ब्धः॒ । भूयान्॑ । भव॑ति । यस्याः᳚ । ते॒ । हरि॑तः । गर्भः॑ । इति॑ । आ॒ह॒ । दे॒व॒त्रेति॑ देव - त्रा । ए॒व । ए॒ना॒म् । ग॒म॒य॒ति॒ । रक्ष॑साम् । अप॑हत्या॒ इत्यप॑ - ह॒त्यै॒ । एति॑ । व॒र्त॒न॒ । व॒र्त॒य॒ । इति॑ । आ॒ह॒ ।  \newline


\textbf{Krama Paata} \newline

वि वै । वा ए॒तस्य॑ । ए॒तस्य॑ य॒ज्ञ्ः । य॒ज्ञ् ऋ॑द्ध्यते । ऋ॒द्ध्य॒ते॒ यस्य॑ । यस्य॑ ह॒विः । ह॒विर॑ति॒रिच्य॑ते । अ॒ति॒रिच्य॑ते॒ सूर्यः॑ । अ॒ति॒रिच्य॑त॒ इत्य॑ति - रिच्य॑ते । सूर्यो॑ दे॒वः । दे॒वो दि॑वि॒षद्भ्यः॑ । दि॒वि॒षद्भ्य॒ इति॑ । दि॒वि॒षद्भ्य॒ इति॑ दिवि॒षत् - भ्यः॒ । इत्या॑ह । आ॒ह॒ बृह॒स्पति॑ना । बृह॒स्पति॑ना च । चै॒व । ए॒वास्य॑ । अ॒स्य॒ प्र॒जाप॑तिना । प्र॒जाप॑तिना च । प्र॒जाप॑ति॒नेति॑ प्र॒जा - प॒ति॒ना॒ । च॒ य॒ज्ञ्स्य॑ । य॒ज्ञ्स्य॒ व्यृ॑द्धम् । व्यृ॑द्ध॒मपि॑ । व्यृ॑द्ध॒मिति॒ वि - ऋ॒द्ध॒म् । अपि॑ वपति । व॒प॒ति॒ रक्षाꣳ॑सि । रक्षाꣳ॑सि॒ वै । वा ए॒तत् । ए॒तत् प॒शुम् । प॒शुꣳ स॑चन्ते । स॒च॒न्ते॒ यत् । यदे॑कदेव॒त्यः॑ । ए॒क॒दे॒व॒त्य॑ आल॑ब्धः । ए॒क॒दे॒व॒त्य॑ इत्ये॑क - दे॒व॒त्यः॑ । आल॑ब्धो॒ भूयान्॑ । आल॑ब्ध॒ इत्या - ल॒ब्धः॒ । भूया॒न् भव॑ति । भव॑ति॒ यस्याः᳚ । यस्या᳚स्ते । ते॒ हरि॑तः । हरि॑तो॒ गर्भः॑ । गर्भ॒ इति॑ । इत्या॑ह । आ॒ह॒ दे॒व॒त्रा । दे॒व॒त्रैव । दे॒व॒त्रेति॑ 
देव - त्रा । ए॒वैना᳚म् । ए॒ना॒म् ग॒म॒य॒ति॒ । ग॒म॒य॒ति॒ रक्ष॑साम् । रक्ष॑सा॒मप॑हत्यै । अप॑हत्या॒ आ । अप॑हत्या॒ इत्यप॑ - ह॒त्यै॒ । 
आ व॑र्तन । व॒र्त॒न॒ व॒र्त॒य॒ । व॒र्त॒येति॑ । इत्या॑ह । 
आ॒ह॒ ब्रह्म॑णा \newline

\textbf{Jatai Paata} \newline

1. वि वै वै वि वि वै । \newline
2. वा ए॒त स्यै॒तस्य॒ वै वा ए॒तस्य॑ । \newline
3. ए॒तस्य॑ य॒ज्ञो य॒ज्ञ् ए॒त स्यै॒तस्य॑ य॒ज्ञ्ः । \newline
4. य॒ज्ञ् ऋ॑द्ध्यत ऋद्ध्यते य॒ज्ञो य॒ज्ञ् ऋ॑द्ध्यते । \newline
5. ऋ॒द्ध्य॒ते॒ यस्य॒ यस्य॑ र्द्ध्यत ऋद्ध्यते॒ यस्य॑ । \newline
6. यस्य॑ ह॒विर्. ह॒विर् यस्य॒ यस्य॑ ह॒विः । \newline
7. ह॒वि र॑ति॒रिच्य॑ते ऽति॒रिच्य॑ते ह॒विर्. ह॒वि र॑ति॒रिच्य॑ते । \newline
8. अ॒ति॒रिच्य॑ते॒ सूर्यः॒ सूर्यो॑ ऽति॒रिच्य॑ते ऽति॒रिच्य॑ते॒ सूर्यः॑ । \newline
9. अ॒ति॒रिच्य॑त॒ इत्य॑ति - रिच्य॑ते । \newline
10. सूर्यो॑ दे॒वो दे॒वः सूर्यः॒ सूर्यो॑ दे॒वः । \newline
11. दे॒वो दि॑वि॒षद्भ्यो॑ दिवि॒षद्भ्यो॑ दे॒वो दे॒वो दि॑वि॒षद्भ्यः॑ । \newline
12. दि॒वि॒षद्भ्य॒ इतीति॑ दिवि॒षद्भ्यो॑ दिवि॒षद्भ्य॒ इति॑ । \newline
13. दि॒वि॒षद्भ्य॒ इति॑ दिवि॒षत् - भ्यः॒ । \newline
14. इत्या॑ हा॒हे तीत्या॑ह । \newline
15. आ॒ह॒ बृह॒स्पति॑ना॒ बृह॒स्पति॑ना ऽऽहाह॒ बृह॒स्पति॑ना । \newline
16. बृह॒स्पति॑ना च च॒ बृह॒स्पति॑ना॒ बृह॒स्पति॑ना च । \newline
17. चै॒वैव च॑ चै॒व । \newline
18. ए॒वास्या᳚ स्यै॒वैवास्य॑ । \newline
19. अ॒स्य॒ प्र॒जाप॑तिना प्र॒जाप॑तिना ऽस्यास्य प्र॒जाप॑तिना । \newline
20. प्र॒जाप॑तिना च च प्र॒जाप॑तिना प्र॒जाप॑तिना च । \newline
21. प्र॒जाप॑ति॒नेति॑ प्र॒जा - प॒ति॒ना॒ । \newline
22. च॒ य॒ज्ञ्स्य॑ य॒ज्ञ्स्य॑ च च य॒ज्ञ्स्य॑ । \newline
23. य॒ज्ञ्स्य॒ व्यृ॑द्धं॒ ॅव्यृ॑द्धं ॅय॒ज्ञ्स्य॑ य॒ज्ञ्स्य॒ व्यृ॑द्धम् । \newline
24. व्यृ॑द्ध॒ मप्यपि॒ व्यृ॑द्धं॒ ॅव्यृ॑द्ध॒ मपि॑ । \newline
25. व्यृ॑द्ध॒मिति॒ वि - ऋ॒द्ध॒म् । \newline
26. अपि॑ वपति वप॒ त्यप्यपि॑ वपति । \newline
27. व॒प॒ति॒ रक्षाꣳ॑सि॒ रक्षाꣳ॑सि वपति वपति॒ रक्षाꣳ॑सि । \newline
28. रक्षाꣳ॑सि॒ वै वै रक्षाꣳ॑सि॒ रक्षाꣳ॑सि॒ वै । \newline
29. वा ए॒त दे॒तद् वै वा ए॒तत् । \newline
30. ए॒तत् प॒शुम् प॒शु मे॒त दे॒तत् प॒शुम् । \newline
31. प॒शुꣳ स॑चन्ते सचन्ते प॒शुम् प॒शुꣳ स॑चन्ते । \newline
32. स॒च॒न्ते॒ यद् यथ् स॑चन्ते सचन्ते॒ यत् । \newline
33. यदे॑कदेव॒त्य॑ एकदेव॒त्यो॑ यद् यदे॑कदेव॒त्यः॑ । \newline
34. ए॒क॒दे॒व॒त्य॑ आल॑ब्ध॒ आल॑ब्ध एकदेव॒त्य॑ एकदेव॒त्य॑ आल॑ब्धः । \newline
35. ए॒क॒दे॒व॒त्य॑ इत्ये॑क - दे॒व॒त्यः॑ । \newline
36. आल॑ब्धो॒ भूया॒न् भूया॒-नाल॑ब्ध॒ आल॑ब्धो॒ भूयान्॑ । \newline
37. आल॑ब्ध॒ इत्या - ल॒ब्धः॒ । \newline
38. भूया॒न् भव॑ति॒ भव॑ति॒ भूया॒न् भूया॒न् भव॑ति । \newline
39. भव॑ति॒ यस्या॒ यस्या॒ भव॑ति॒ भव॑ति॒ यस्याः᳚ । \newline
40. यस्या᳚ स्ते ते॒ यस्या॒ यस्या᳚ स्ते । \newline
41. ते॒ हरि॑तो॒ हरि॑त स्ते ते॒ हरि॑तः । \newline
42. हरि॑तो॒ गर्भो॒ गर्भो॒ हरि॑तो॒ हरि॑तो॒ गर्भः॑ । \newline
43. गर्भ॒ इतीति॒ गर्भो॒ गर्भ॒ इति॑ । \newline
44. इत्या॑ हा॒हे तीत्या॑ह । \newline
45. आ॒ह॒ दे॒व॒त्रा दे॑व॒त्रा ऽऽहा॑ह देव॒त्रा । \newline
46. दे॒व॒त्रैवैव दे॑व॒त्रा दे॑व॒त्रैव । \newline
47. दे॒व॒त्रेति॑ देव - त्रा । \newline
48. ए॒वैना॑ मेना मे॒वै वैना᳚म् । \newline
49. ए॒ना॒म् ग॒म॒य॒ति॒ ग॒म॒य॒ त्ये॒ना॒ मे॒ना॒म् ग॒म॒य॒ति॒ । \newline
50. ग॒म॒य॒ति॒ रक्ष॑साꣳ॒॒ रक्ष॑साम् गमयति गमयति॒ रक्ष॑साम् । \newline
51. रक्ष॑सा॒ मप॑हत्या॒ अप॑हत्यै॒ रक्ष॑साꣳ॒॒ रक्ष॑सा॒ मप॑हत्यै । \newline
52. अप॑हत्या॒ आ ऽप॑हत्या॒ अप॑हत्या॒ आ । \newline
53. अप॑हत्या॒ इत्यप॑ - ह॒त्यै॒ । \newline
54. आ व॑र्तन वर्त॒ना व॑र्तन । \newline
55. व॒र्त॒न॒ व॒र्त॒य॒ व॒र्त॒य॒ व॒र्त॒न॒ व॒र्त॒न॒ व॒र्त॒य॒ । \newline
56. व॒र्त॒येतीति॑ वर्तय वर्त॒येति॑ । \newline
57. इत्या॑ हा॒हे तीत्या॑ह । \newline
58. आ॒ह॒ ब्रह्म॑णा॒ ब्रह्म॑णा ऽऽहाह॒ ब्रह्म॑णा । \newline

\textbf{Ghana Paata } \newline

1. वि वै वै वि वि वा ए॒त स्यै॒तस्य॒ वै वि वि वा ए॒तस्य॑ । \newline
2. वा ए॒त स्यै॒तस्य॒ वै वा ए॒तस्य॑ य॒ज्ञो य॒ज्ञ् ए॒तस्य॒ वै वा ए॒तस्य॑ य॒ज्ञ्ः । \newline
3. ए॒तस्य॑ य॒ज्ञो य॒ज्ञ् ए॒त स्यै॒तस्य॑ य॒ज्ञ् ऋ॑द्ध्यत ऋद्ध्यते य॒ज्ञ् ए॒त स्यै॒तस्य॑ य॒ज्ञ् ऋ॑द्ध्यते । \newline
4. य॒ज्ञ् ऋ॑द्ध्यत ऋद्ध्यते य॒ज्ञो य॒ज्ञ् ऋ॑द्ध्यते॒ यस्य॒ यस्य॑ र्‌द्ध्यते य॒ज्ञो य॒ज्ञ् ऋ॑द्ध्यते॒ यस्य॑ । \newline
5. ऋ॒द्ध्य॒ते॒ यस्य॒ यस्य॑ र्‌द्ध्यत ऋद्ध्यते॒ यस्य॑ ह॒विर्. ह॒विर् यस्य॑ र्‌द्ध्यत ऋद्ध्यते॒ यस्य॑ ह॒विः । \newline
6. यस्य॑ ह॒विर्. ह॒विर् यस्य॒ यस्य॑ ह॒वि र॑ति॒रिच्य॑ते ऽति॒रिच्य॑ते ह॒विर् यस्य॒ यस्य॑ ह॒वि र॑ति॒रिच्य॑ते । \newline
7. ह॒वि र॑ति॒रिच्य॑ते ऽति॒रिच्य॑ते ह॒विर्. ह॒वि र॑ति॒रिच्य॑ते॒ सूर्यः॒ सूर्यो॑ ऽति॒रिच्य॑ते ह॒विर्. ह॒वि र॑ति॒रिच्य॑ते॒ सूर्यः॑ । \newline
8. अ॒ति॒रिच्य॑ते॒ सूर्यः॒ सूर्यो॑ ऽति॒रिच्य॑ते ऽति॒रिच्य॑ते॒ सूर्यो॑ दे॒वो दे॒वः सूर्यो॑ ऽति॒रिच्य॑ते ऽति॒रिच्य॑ते॒ सूर्यो॑ दे॒वः । \newline
9. अ॒ति॒रिच्य॑त॒ इत्य॑ति - रिच्य॑ते । \newline
10. सूर्यो॑ दे॒वो दे॒वः सूर्यः॒ सूर्यो॑ दे॒वो दि॑वि॒षद्भ्यो॑ दिवि॒षद्भ्यो॑ दे॒वः सूर्यः॒ सूर्यो॑ दे॒वो दि॑वि॒षद्भ्यः॑ । \newline
11. दे॒वो दि॑वि॒षद्भ्यो॑ दिवि॒षद्भ्यो॑ दे॒वो दे॒वो दि॑वि॒षद्भ्य॒ इतीति॑ दिवि॒षद्भ्यो॑ दे॒वो दे॒वो दि॑वि॒षद्भ्य॒ इति॑ । \newline
12. दि॒वि॒षद्भ्य॒ इतीति॑ दिवि॒षद्भ्यो॑ दिवि॒षद्भ्य॒ इत्या॑ हा॒हेति॑ दिवि॒षद्भ्यो॑ दिवि॒षद्भ्य॒ इत्या॑ह । \newline
13. दि॒वि॒षद्भ्य॒ इति॑ दिवि॒षत् - भ्यः॒ । \newline
14. इत्या॑हा॒हे तीत्या॑ह॒ बृह॒स्पति॑ना॒ बृह॒स्पति॑ना॒ ऽऽहे तीत्या॑ह॒ बृह॒स्पति॑ना । \newline
15. आ॒ह॒ बृह॒स्पति॑ना॒ बृह॒स्पति॑ना ऽऽहाह॒ बृह॒स्पति॑ना च च॒ बृह॒स्पति॑ना ऽऽहाह॒ बृह॒स्पति॑ना च । \newline
16. बृह॒स्पति॑ना च च॒ बृह॒स्पति॑ना॒ बृह॒स्पति॑ना चै॒वैव च॒ बृह॒स्पति॑ना॒ बृह॒स्पति॑ना चै॒व । \newline
17. चै॒वैव च॑ चै॒वा स्या᳚ स्यै॒व च॑ चै॒वा स्य॑ । \newline
18. ए॒वास्या᳚ स्यै॒ वैवा स्य॑ प्र॒जाप॑तिना प्र॒जाप॑तिना ऽस्यै॒ वैवास्य॑ प्र॒जाप॑तिना । \newline
19. अ॒स्य॒ प्र॒जाप॑तिना प्र॒जाप॑तिना ऽस्यास्य प्र॒जाप॑तिना च च प्र॒जाप॑तिना ऽस्यास्य प्र॒जाप॑तिना च । \newline
20. प्र॒जाप॑तिना च च प्र॒जाप॑तिना प्र॒जाप॑तिना च य॒ज्ञ्स्य॑ य॒ज्ञ्स्य॑ च प्र॒जाप॑तिना प्र॒जाप॑तिना च य॒ज्ञ्स्य॑ । \newline
21. प्र॒जाप॑ति॒नेति॑ प्र॒जा - प॒ति॒ना॒ । \newline
22. च॒ य॒ज्ञ्स्य॑ य॒ज्ञ्स्य॑ च च य॒ज्ञ्स्य॒ व्यृ॑द्ध॒म् ॅव्यृ॑द्धम् ॅय॒ज्ञ्स्य॑ च च य॒ज्ञ्स्य॒ व्यृ॑द्धम् । \newline
23. य॒ज्ञ्स्य॒ व्यृ॑द्ध॒म् ॅव्यृ॑द्धम् ॅय॒ज्ञ्स्य॑ य॒ज्ञ्स्य॒ व्यृ॑द्ध॒ मप्यपि॒ व्यृ॑द्धम् ॅय॒ज्ञ्स्य॑ य॒ज्ञ्स्य॒ व्यृ॑द्ध॒ मपि॑ । \newline
24. व्यृ॑द्ध॒ मप्यपि॒ व्यृ॑द्ध॒म् ॅव्यृ॑द्ध॒ मपि॑ वपति वप॒ त्यपि॒ व्यृ॑द्ध॒म् ॅव्यृ॑द्ध॒ मपि॑ वपति । \newline
25. व्यृ॑द्ध॒मिति॒ वि - ऋ॒द्ध॒म् । \newline
26. अपि॑ वपति वप॒ त्यप्यपि॑ वपति॒ रक्षाꣳ॑सि॒ रक्षाꣳ॑सि वप॒ त्यप्यपि॑ वपति॒ रक्षाꣳ॑सि । \newline
27. व॒प॒ति॒ रक्षाꣳ॑सि॒ रक्षाꣳ॑सि वपति वपति॒ रक्षाꣳ॑सि॒ वै वै रक्षाꣳ॑सि वपति वपति॒ रक्षाꣳ॑सि॒ वै । \newline
28. रक्षाꣳ॑सि॒ वै वै रक्षाꣳ॑सि॒ रक्षाꣳ॑सि॒ वा ए॒त दे॒तद् वै रक्षाꣳ॑सि॒ रक्षाꣳ॑सि॒ वा ए॒तत् । \newline
29. वा ए॒त दे॒तद् वै वा ए॒तत् प॒शुम् प॒शु मे॒तद् वै वा ए॒तत् प॒शुम् । \newline
30. ए॒तत् प॒शुम् प॒शु मे॒त दे॒तत् प॒शुꣳ स॑चन्ते सचन्ते प॒शु मे॒त दे॒तत् प॒शुꣳ स॑चन्ते । \newline
31. प॒शुꣳ स॑चन्ते सचन्ते प॒शुम् प॒शुꣳ स॑चन्ते॒ यद् यथ् स॑चन्ते प॒शुम् प॒शुꣳ स॑चन्ते॒ यत् । \newline
32. स॒च॒न्ते॒ यद् यथ् स॑चन्ते सचन्ते॒ यदे॑कदेव॒त्य॑ एकदेव॒त्यो॑ यथ् स॑चन्ते सचन्ते॒ यदे॑कदेव॒त्यः॑ । \newline
33. यदे॑कदेव॒त्य॑ एकदेव॒त्यो॑ यद् यदे॑कदेव॒त्य॑ आल॑ब्ध॒ आल॑ब्ध एकदेव॒त्यो॑ यद् यदे॑कदेव॒त्य॑ आल॑ब्धः । \newline
34. ए॒क॒दे॒व॒त्य॑ आल॑ब्ध॒ आल॑ब्ध एकदेव॒त्य॑ एकदेव॒त्य॑ आल॑ब्धो॒ भूया॒न् भूया॒ नाल॑ब्ध एकदेव॒त्य॑ एकदेव॒त्य॑ आल॑ब्धो॒ भूयान्॑ । \newline
35. ए॒क॒दे॒व॒त्य॑ इत्ये॑क - दे॒व॒त्यः॑ । \newline
36. आल॑ब्धो॒ भूया॒न् भूया॒ नाल॑ब्ध॒ आल॑ब्धो॒ भूया॒न् भव॑ति॒ भव॑ति॒ भूया॒ नाल॑ब्ध॒ आल॑ब्धो॒ भूया॒न् भव॑ति । \newline
37. आल॑ब्ध॒ इत्या - ल॒ब्धः॒ । \newline
38. भूया॒न् भव॑ति॒ भव॑ति॒ भूया॒न् भूया॒न् भव॑ति॒ यस्या॒ यस्या॒ भव॑ति॒ भूया॒न् भूया॒न् भव॑ति॒ यस्याः᳚ । \newline
39. भव॑ति॒ यस्या॒ यस्या॒ भव॑ति॒ भव॑ति॒ यस्या᳚ स्ते ते॒ यस्या॒ भव॑ति॒ भव॑ति॒ यस्या᳚ स्ते । \newline
40. यस्या᳚ स्ते ते॒ यस्या॒ यस्या᳚ स्ते॒ हरि॑तो॒ हरि॑त स्ते॒ यस्या॒ यस्या᳚ स्ते॒ हरि॑तः । \newline
41. ते॒ हरि॑तो॒ हरि॑त स्ते ते॒ हरि॑तो॒ गर्भो॒ गर्भो॒ हरि॑त स्ते ते॒ हरि॑तो॒ गर्भः॑ । \newline
42. हरि॑तो॒ गर्भो॒ गर्भो॒ हरि॑तो॒ हरि॑तो॒ गर्भ॒ इतीति॒ गर्भो॒ हरि॑तो॒ हरि॑तो॒ गर्भ॒ इति॑ । \newline
43. गर्भ॒ इतीति॒ गर्भो॒ गर्भ॒ इत्या॑ हा॒हेति॒ गर्भो॒ गर्भ॒ इत्या॑ह । \newline
44. इत्या॑हा॒हे तीत्या॑ह देव॒त्रा दे॑व॒त्रा ऽऽहेतीत्या॑ह देव॒त्रा । \newline
45. आ॒ह॒ दे॒व॒त्रा दे॑व॒त्रा ऽऽहा॑ह देव॒त्रैवैव दे॑व॒त्रा ऽऽहा॑ह देव॒त्रैव । \newline
46. दे॒व॒त्रै वैव दे॑व॒त्रा दे॑व॒त्रै वैना॑ मेना मे॒व दे॑व॒त्रा दे॑व॒त्रै वैना᳚म् । \newline
47. दे॒व॒त्रेति॑ देव - त्रा । \newline
48. ए॒वैना॑ मेना मे॒वैवैना᳚म् गमयति गमय त्येना मे॒वैवैना᳚म् गमयति । \newline
49. ए॒ना॒म् ग॒म॒य॒ति॒ ग॒म॒य॒ त्ये॒ना॒ मे॒ना॒म् ग॒म॒य॒ति॒ रक्ष॑साꣳ॒॒ रक्ष॑साम् गमय त्येना मेनाम् गमयति॒ रक्ष॑साम् । \newline
50. ग॒म॒य॒ति॒ रक्ष॑साꣳ॒॒ रक्ष॑साम् गमयति गमयति॒ रक्ष॑सा॒ मप॑हत्या॒ अप॑हत्यै॒ रक्ष॑साम् गमयति गमयति॒ रक्ष॑सा॒ मप॑हत्यै । \newline
51. रक्ष॑सा॒ मप॑हत्या॒ अप॑हत्यै॒ रक्ष॑साꣳ॒॒ रक्ष॑सा॒ मप॑हत्या॒ आ ऽप॑हत्यै॒ रक्ष॑साꣳ॒॒ रक्ष॑सा॒ मप॑हत्या॒ आ । \newline
52. अप॑हत्या॒ आ ऽप॑हत्या॒ अप॑हत्या॒ आ व॑र्तन वर्त॒ना ऽप॑हत्या॒ अप॑हत्या॒ आ व॑र्तन । \newline
53. अप॑हत्या॒ इत्यप॑ - ह॒त्यै॒ । \newline
54. आ व॑र्तन वर्त॒ना व॑र्तन वर्तय वर्तय वर्त॒ना व॑र्तन वर्तय । \newline
55. व॒र्त॒न॒ व॒र्त॒य॒ व॒र्त॒य॒ व॒र्त॒न॒ व॒र्त॒न॒ व॒र्त॒येतीति॑ वर्तय वर्तन वर्तन वर्त॒येति॑ । \newline
56. व॒र्त॒ये तीति॑ वर्तय वर्त॒ये त्या॑ हा॒हेति॑ वर्तय वर्त॒ये त्या॑ह । \newline
57. इत्या॑ हा॒हेती त्या॑ह॒ ब्रह्म॑णा॒ ब्रह्म॑णा॒ ऽऽहेती त्या॑ह॒ ब्रह्म॑णा । \newline
58. आ॒ह॒ ब्रह्म॑णा॒ ब्रह्म॑णा ऽऽहाह॒ ब्रह्म॑ णै॒वैव ब्रह्म॑णा ऽऽहाह॒ ब्रह्म॑णै॒व । \newline
\pagebreak
\markright{ TS 3.4.1.2  \hfill https://www.vedavms.in \hfill}

\section{ TS 3.4.1.2 }

\textbf{TS 3.4.1.2 } \newline
\textbf{Samhita Paata} \newline

ब्रह्म॑णै॒वैन॒मा व॑र्तयति॒ वि ते॑ भिनद्मि तक॒रीमित्या॑ह यथाय॒जुरे॒वैतदु॑- रुद्र॒फ्सो वि॒श्वरू॑प॒ इन्दु॒रित्या॑ह प्र॒जा वै प॒शव॒ इन्दुः॑ प्र॒जयै॒वैनं॑ प॒शुभिः॒ सम॑र्द्धयति॒ दिवं॒ ॅवै य॒ज्ञ्स्य॒ व्यृ॑द्धं गच्छति पृथि॒वीमति॑रिक्तं॒ तद्यन्न श॒मये॒दार्ति॒मार्च्छे॒द्-यज॑मानो म॒ही द्यौः पृ॑थि॒वीच॑ न॒ इत्या॑ - [  ] \newline

\textbf{Pada Paata} \newline

ब्रह्म॑णा । ए॒व । ए॒न॒म् । एति॑ । व॒र्त॒य॒ति॒ । वीति॑ । ते॒ । भि॒न॒द्मि॒ । त॒क॒रीम् । इति॑ । आ॒ह॒ । य॒था॒य॒जुरिति॑ यथा - य॒जुः । ए॒व । ए॒तत् । उ॒रु॒द्र॒फ्स इत्यु॑रु - द्र॒फ्सः । वि॒श्वरू॑प॒ इति॑ वि॒श्व - रू॒पः॒ । इन्दुः॑ । इति॑ । आ॒ह॒ । प्र॒जेति॑ प्र - जा । वै । प॒शवः॑ । इन्दुः॑ । प्र॒जयेति॑ प्र - जया᳚ । ए॒व । ए॒न॒म् । प॒शुभि॒रिति॑ प॒शु - भिः॒ । समिति॑ । अ॒र्द्ध॒य॒ति॒ । दिव᳚म् । वै । य॒ज्ञ्स्य॑ । व्यृ॑द्ध॒मिति॒ वि-ऋ॒द्ध॒म् । ग॒च्छ॒ति॒ । पृ॒थि॒वीम् । अति॑रिक्त॒मित्यति॑ - रि॒क्त॒म् । तत् । यत् । न । श॒मये᳚त् । आर्ति᳚म् । एति॑ । ऋ॒च्छे॒त् । यज॑मानः । म॒ही । द्यौः । पृ॒थि॒वी । च॒ । नः॒ । इति॑ ।  \newline


\textbf{Krama Paata} \newline

ब्रह्म॑णै॒व । ए॒वैन᳚म् । ए॒न॒मा । आ व॑र्तयति । व॒र्त॒य॒ति॒ वि । वि ते᳚ । ते॒ भि॒न॒द्मि॒ । भि॒न॒द्मि॒ त॒क॒रीम् । त॒क॒रीमिति॑ । इत्या॑ह । आ॒ह॒ य॒था॒य॒जुः । य॒था॒य॒जुरे॒व । य॒था॒य॒जुरिति॑ यथा - य॒जुः । ए॒वैतत् । ए॒तदु॑रुद्र॒फ्सः । उ॒रु॒द्र॒फ्सो वि॒श्वरू॑पः । उ॒रु॒द्र॒फ्स इत्यु॑रु - द्र॒फ्सः । वि॒श्वरू॑प॒ इन्दुः॑ । वि॒श्वरू॑प॒ इति॑ वि॒श्व - रू॒पः॒ । इन्दु॒रिति॑ । इत्या॑ह । आ॒ह॒ प्र॒जा । प्र॒जा वै । प्र॒जेति॑ प्र - जा । वै प॒शवः॑ । प॒शव॒ इन्दुः॑ । इन्दुः॑ प्र॒जया᳚ । प्र॒जयै॒व । प्र॒जयेति॑ प्र - जया᳚ । ए॒वैन᳚म् । ए॒न॒म् प॒शुभिः॑ । प॒शुभिः॒ सम् । प॒शुभि॒रति॑ प॒शु - भिः॒ । सम॑र्द्धयति । अ॒र्द्ध॒य॒ति॒ दिव᳚म् । दिवं॒ ॅवै । वै य॒ज्ञ्स्य॑ । य॒ज्ञ्स्य॒ व्यृ॑द्धम् । व्यृ॑द्धम् गच्छति । व्यृ॑द्ध॒मिति॒ वि - ऋ॒द्ध॒म् । ग॒च्छ॒ति॒ पृ॒थि॒वीम् । पृ॒थि॒वीमति॑रिक्तम् । अति॑रिक्त॒म् तत् । अति॑रिक्त॒मित्यति॑ - रि॒क्त॒म् । तद् यत् । यन् न । न श॒मये᳚त् । श॒मये॒दार्ति᳚म् । आर्ति॒मा । आर्च्छे᳚त् । ऋ॒च्छे॒द् यज॑मानः । यज॑मानो म॒ही । म॒ही द्यौः । द्यौः पृ॑थि॒वी । पृ॒थि॒वी च॑ । च॒ नः॒ । न॒ इति॑ । इत्या॑ह \newline

\textbf{Jatai Paata} \newline

1. ब्रह्म॑णै॒वैव ब्रह्म॑णा॒ ब्रह्म॑णै॒व । \newline
2. ए॒वैन॑ मेन मे॒वैवैन᳚म् । \newline
3. ए॒न॒ मैन॑ मेन॒ मा । \newline
4. आ व॑र्तयति वर्तय॒ त्याव॑र्तयति । \newline
5. व॒र्त॒य॒ति॒ वि वि व॑र्तयति वर्तयति॒ वि । \newline
6. वि ते॑ ते॒ वि वि ते᳚ । \newline
7. ते॒ भि॒न॒द्मि॒ भि॒न॒द्मि॒ ते॒ ते॒ भि॒न॒द्मि॒ । \newline
8. भि॒न॒द्मि॒ त॒क॒रीम् त॑क॒रीम् भि॑नद्मि भिनद्मि तक॒रीम् । \newline
9. त॒क॒री मितीति॑ तक॒रीम् त॑क॒री मिति॑ । \newline
10. इत्या॑ हा॒हे तीत्या॑ह । \newline
11. आ॒ह॒ य॒था॒य॒जुर् य॑थाय॒जु रा॑हाह यथाय॒जुः । \newline
12. य॒था॒य॒जु रे॒वैव य॑थाय॒जुर् य॑थाय॒जु रे॒व । \newline
13. य॒था॒य॒जुरिति॑ यथा - य॒जुः । \newline
14. ए॒वैत दे॒त दे॒वैवैतत् । \newline
15. ए॒त दु॑रुद्र॒फ्स उ॑रुद्र॒फ्स ए॒त दे॒त दु॑रुद्र॒फ्सः । \newline
16. उ॒रु॒द्र॒फ्सो वि॒श्वरू॑पो वि॒श्वरू॑प उरुद्र॒फ्स उ॑रुद्र॒फ्सो वि॒श्वरू॑पः । \newline
17. उ॒रु॒द्र॒फ्स इत्यु॑रु - द्र॒फ्सः । \newline
18. वि॒श्वरू॑प॒ इन्दु॒ रिन्दु॑र् वि॒श्वरू॑पो वि॒श्वरू॑प॒ इन्दुः॑ । \newline
19. वि॒श्वरू॑प॒ इति॑ वि॒श्व - रू॒पः॒ । \newline
20. इन्दु॒ रितीतीन्दु॒ रिन्दु॒ रिति॑ । \newline
21. इत्या॑ हा॒हे तीत्या॑ह । \newline
22. आ॒ह॒ प्र॒जा प्र॒जा ऽऽहा॑ह प्र॒जा । \newline
23. प्र॒जा वै वै प्र॒जा प्र॒जा वै । \newline
24. प्र॒जेति॑ प्र - जा । \newline
25. वै प॒शवः॑ प॒शवो॒ वै वै प॒शवः॑ । \newline
26. प॒शव॒ इन्दु॒ रिन्दुः॑ प॒शवः॑ प॒शव॒ इन्दुः॑ । \newline
27. इन्दुः॑ प्र॒जया᳚ प्र॒जयेन् दु॒रिन्दुः॑ प्र॒जया᳚ । \newline
28. प्र॒जयै॒ वैव प्र॒जया᳚ प्र॒जयै॒व । \newline
29. प्र॒जयेति॑ प्र - जया᳚ । \newline
30. ए॒वैन॑ मेन मे॒वैवैन᳚म् । \newline
31. ए॒न॒म् प॒शुभिः॑ प॒शुभि॑ रेन मेनम् प॒शुभिः॑ । \newline
32. प॒शुभिः॒ सꣳ सम् प॒शुभिः॑ प॒शुभिः॒ सम् । \newline
33. प॒शुभि॒रिति॑ प॒शु - भिः॒ । \newline
34. स म॑र्द्धय त्यर्द्धयति॒ सꣳ स म॑र्द्धयति । \newline
35. अ॒र्द्ध॒य॒ति॒ दिव॒म् दिव॑ मर्द्धय त्यर्द्धयति॒ दिव᳚म् । \newline
36. दिवं॒ ॅवै वै दिव॒म् दिवं॒ ॅवै । \newline
37. वै य॒ज्ञ्स्य॑ य॒ज्ञ्स्य॒ वै वै य॒ज्ञ्स्य॑ । \newline
38. य॒ज्ञ्स्य॒ व्यृ॑द्धं॒ ॅव्यृ॑द्धं ॅय॒ज्ञ्स्य॑ य॒ज्ञ्स्य॒ व्यृ॑द्धम् । \newline
39. व्यृ॑द्धम् गच्छति गच्छति॒ व्यृ॑द्धं॒ ॅव्यृ॑द्धम् गच्छति । \newline
40. व्यृ॑द्ध॒मिति॒ वि - ऋ॒द्ध॒म् । \newline
41. ग॒च्छ॒ति॒ पृ॒थि॒वीम् पृ॑थि॒वीम् ग॑च्छति गच्छति पृथि॒वीम् । \newline
42. पृ॒थि॒वी मति॑रिक्त॒ मति॑रिक्तम् पृथि॒वीम् पृ॑थि॒वी मति॑रिक्तम् । \newline
43. अति॑रिक्त॒म् तत् तदति॑रिक्त॒ मति॑रिक्त॒म् तत् । \newline
44. अति॑रिक्त॒मित्यति॑ - रि॒क्त॒म् । \newline
45. तद् यद् यत् तत् तद् यत् । \newline
46. यन् न न यद् यन् न । \newline
47. न श॒मये᳚ च्छ॒मये॒न् न न श॒मये᳚त् । \newline
48. श॒मये॒दार्ति॒ मार्तिꣳ॑ श॒मये᳚ च्छ॒मये॒ दार्ति᳚म् । \newline
49. आर्ति॒ मा ऽऽर्ति॒ मार्ति॒ मा । \newline
50. आर्च्छे॑ दृच्छे॒ दार्च्छे᳚त् । \newline
51. ऋ॒च्छे॒द् यज॑मानो॒ यज॑मान ऋच्छे दृच्छे॒द् यज॑मानः । \newline
52. यज॑मानो म॒ही म॒ही यज॑मानो॒ यज॑मानो म॒ही । \newline
53. म॒ही द्यौर् द्यौर् म॒ही म॒ही द्यौः । \newline
54. द्यौः पृ॑थि॒वी पृ॑थि॒वी द्यौर् द्यौः पृ॑थि॒वी । \newline
55. पृ॒थि॒वी च॑ च पृथि॒वी पृ॑थि॒वी च॑ । \newline
56. च॒ नो॒ न॒ श्च॒ च॒ नः॒ । \newline
57. न॒ इतीति॑ नो न॒ इति॑ । \newline
58. इत्या॑ हा॒हे तीत्या॑ह । \newline

\textbf{Ghana Paata } \newline

1. ब्रह्म॑ णै॒वैव ब्रह्म॑णा॒ ब्रह्म॑णै॒ वैन॑ मेन मे॒व ब्रह्म॑णा॒ ब्रह्म॑ णै॒वैन᳚म् । \newline
2. ए॒वैन॑ मेन मे॒वैवैन॒ मैन॑ मे॒वैवैन॒ मा । \newline
3. ए॒न॒ मैन॑ मेन॒ मा व॑र्तयति वर्तय॒ त्यैन॑ मेन॒ मा व॑र्तयति । \newline
4. आ व॑र्तयति वर्तय॒ त्याव॑र्तयति॒ वि वि व॑र्तय॒ त्याव॑र्तयति॒ वि । \newline
5. व॒र्त॒य॒ति॒ वि वि व॑र्तयति वर्तयति॒ वि ते॑ ते॒ वि व॑र्तयति वर्तयति॒ वि ते᳚ । \newline
6. वि ते॑ ते॒ वि वि ते॑ भिनद्मि भिनद्मि ते॒ वि वि ते॑ भिनद्मि । \newline
7. ते॒ भि॒न॒द्मि॒ भि॒न॒द्मि॒ ते॒ ते॒ भि॒न॒द्मि॒ त॒क॒रीम् त॑क॒रीम् भि॑नद्मि ते ते भिनद्मि तक॒रीम् । \newline
8. भि॒न॒द्मि॒ त॒क॒रीम् त॑क॒रीम् भि॑नद्मि भिनद्मि तक॒री मितीति॑ तक॒रीम् भि॑नद्मि भिनद्मि तक॒री मिति॑ । \newline
9. त॒क॒री मितीति॑ तक॒रीम् त॑क॒री मित्या॑ हा॒हेति॑ तक॒रीम् त॑क॒री मित्या॑ह । \newline
10. इत्या॑ हा॒हे तीत्या॑ह यथाय॒जुर् य॑थाय॒जु रा॒हे तीत्या॑ह यथाय॒जुः । \newline
11. आ॒ह॒ य॒था॒य॒जुर् य॑थाय॒जु रा॑हाह यथाय॒जु रे॒वै व य॑थाय॒जु रा॑हाह यथाय॒जु रे॒व । \newline
12. य॒था॒य॒जु रे॒वै व य॑थाय॒जुर् य॑थाय॒जु रे॒वैत दे॒त दे॒व य॑थाय॒जुर् य॑थाय॒जु रे॒वैतत् । \newline
13. य॒था॒य॒जुरिति॑ यथा - य॒जुः । \newline
14. ए॒वैत दे॒त दे॒वैवैत दु॑रुद्र॒फ्स उ॑रुद्र॒फ्स ए॒त दे॒वैवैत दु॑रुद्र॒फ्सः । \newline
15. ए॒त दु॑रुद्र॒फ्स उ॑रुद्र॒फ्स ए॒त दे॒तदु॑रुद्र॒फ्सो वि॒श्वरू॑पो वि॒श्वरू॑प उरुद्र॒फ्स ए॒तदे॒त दु॑रुद्र॒फ्सो वि॒श्वरू॑पः । \newline
16. उ॒रु॒द्र॒फ्सो वि॒श्वरू॑पो वि॒श्वरू॑प उरुद्र॒फ्स उ॑रुद्र॒फ्सो वि॒श्वरू॑प॒ इन्दु॒ रिन्दु॑र् वि॒श्वरू॑प उरुद्र॒फ्स उ॑रुद्र॒फ्सो वि॒श्वरू॑प॒ इन्दुः॑ । \newline
17. उ॒रु॒द्र॒फ्स इत्यु॑रु - द्र॒फ्सः । \newline
18. वि॒श्वरू॑प॒ इन्दु॒ रिन्दु॑र् वि॒श्वरू॑पो वि॒श्वरू॑प॒ इन्दु॒ रिती तीन्दु॑र् वि॒श्वरू॑पो वि॒श्वरू॑प॒ इन्दु॒रिति॑ । \newline
19. वि॒श्वरू॑प॒ इति॑ वि॒श्व - रू॒पः॒ । \newline
20. इन्दु॒ रितीतीन्दु॒ रिन्दु॒ रित्या॑हा॒हेतीन्दु॒ रिन्दु॒ रित्या॑ह । \newline
21. इत्या॑ हा॒हेतीत्या॑ह प्र॒जा प्र॒जा ऽऽहे तीत्या॑ह प्र॒जा । \newline
22. आ॒ह॒ प्र॒जा प्र॒जा ऽऽहा॑ह प्र॒जा वै वै प्र॒जा ऽऽहा॑ह प्र॒जा वै । \newline
23. प्र॒जा वै वै प्र॒जा प्र॒जा वै प॒शवः॑ प॒शवो॒ वै प्र॒जा प्र॒जा वै प॒शवः॑ । \newline
24. प्र॒जेति॑ प्र - जा । \newline
25. वै प॒शवः॑ प॒शवो॒ वै वै प॒शव॒ इन्दु॒ रिन्दुः॑ प॒शवो॒ वै वै प॒शव॒ इन्दुः॑ । \newline
26. प॒शव॒ इन्दु॒ रिन्दुः॑ प॒शवः॑ प॒शव॒ इन्दुः॑ प्र॒जया᳚ प्र॒जयेन्दुः॑ प॒शवः॑ प॒शव॒ इन्दुः॑ प्र॒जया᳚ । \newline
27. इन्दुः॑ प्र॒जया᳚ प्र॒जयेन्दु॒ रिन्दुः॑ प्र॒जयै॒वैव प्र॒जयेन्दु॒ रिन्दुः॑ प्र॒जयै॒व । \newline
28. प्र॒जयै॒ वैव प्र॒जया᳚ प्र॒जयै॒वैन॑ मेन मे॒व प्र॒जया᳚ प्र॒जयै॒वैन᳚म् । \newline
29. प्र॒जयेति॑ प्र - जया᳚ । \newline
30. ए॒वैन॑ मेन मे॒वैवैन॑म् प॒शुभिः॑ प॒शुभि॑ रेन मे॒वैवैन॑म् प॒शुभिः॑ । \newline
31. ए॒न॒म् प॒शुभिः॑ प॒शुभि॑ रेन मेनम् प॒शुभिः॒ सꣳ सम् प॒शुभि॑ रेन मेनम् प॒शुभिः॒ सम् । \newline
32. प॒शुभिः॒ सꣳ सम् प॒शुभिः॑ प॒शुभिः॒ स म॑र्द्धय त्यर्द्धयति॒ सम् प॒शुभिः॑ प॒शुभिः॒ स म॑र्द्धयति । \newline
33. प॒शुभि॒रिति॑ प॒शु - भिः॒ । \newline
34. स म॑र्द्धय त्यर्द्धयति॒ सꣳ स म॑र्द्धयति॒ दिव॒म् दिव॑ मर्द्धयति॒ सꣳ स म॑र्द्धयति॒ दिव᳚म् । \newline
35. अ॒र्द्ध॒य॒ति॒ दिव॒म् दिव॑ मर्द्धय त्यर्द्धयति॒ दिव॒म् ॅवै वै दिव॑ मर्द्धय त्यर्द्धयति॒ दिव॒म् ॅवै । \newline
36. दिव॒म् ॅवै वै दिव॒म् दिव॒म् ॅवै य॒ज्ञ्स्य॑ य॒ज्ञ्स्य॒ वै दिव॒म् दिव॒म् ॅवै य॒ज्ञ्स्य॑ । \newline
37. वै य॒ज्ञ्स्य॑ य॒ज्ञ्स्य॒ वै वै य॒ज्ञ्स्य॒ व्यृ॑द्ध॒म् ॅव्यृ॑द्धम् ॅय॒ज्ञ्स्य॒ वै वै य॒ज्ञ्स्य॒ व्यृ॑द्धम् । \newline
38. य॒ज्ञ्स्य॒ व्यृ॑द्ध॒म् ॅव्यृ॑द्धम् ॅय॒ज्ञ्स्य॑ य॒ज्ञ्स्य॒ व्यृ॑द्धम् गच्छति गच्छति॒ व्यृ॑द्धम् ॅय॒ज्ञ्स्य॑ य॒ज्ञ्स्य॒ व्यृ॑द्धम् गच्छति । \newline
39. व्यृ॑द्धम् गच्छति गच्छति॒ व्यृ॑द्ध॒म् ॅव्यृ॑द्धम् गच्छति पृथि॒वीम् पृ॑थि॒वीम् ग॑च्छति॒ व्यृ॑द्ध॒म् ॅव्यृ॑द्धम् गच्छति पृथि॒वीम् । \newline
40. व्यृ॑द्ध॒मिति॒ वि - ऋ॒द्ध॒म् । \newline
41. ग॒च्छ॒ति॒ पृ॒थि॒वीम् पृ॑थि॒वीम् ग॑च्छति गच्छति पृथि॒वी मति॑रिक्त॒ मति॑रिक्तम् पृथि॒वीम् ग॑च्छति गच्छति पृथि॒वी मति॑रिक्तम् । \newline
42. पृ॒थि॒वी मति॑रिक्त॒ मति॑रिक्तम् पृथि॒वीम् पृ॑थि॒वी मति॑रिक्त॒म् तत् तदति॑रिक्तम् पृथि॒वीम् पृ॑थि॒वी मति॑रिक्त॒म् तत् । \newline
43. अति॑रिक्त॒म् तत् तदति॑रिक्त॒ मति॑रिक्त॒म् तद् यद् यत् तदति॑रिक्त॒ मति॑रिक्त॒म् तद् यत् । \newline
44. अति॑रिक्त॒मित्यति॑ - रि॒क्त॒म् । \newline
45. तद् यद् यत् तत् तद् यन् न न यत् तत् तद् यन् न । \newline
46. यन् न न यद् यन् न श॒मये᳚ च्छ॒मये॒न् न यद् यन् न श॒मये᳚त् । \newline
47. न श॒मये᳚ च्छ॒मये॒न् न न श॒मये॒ दार्ति॒ मार्तिꣳ॑ श॒मये॒न् न न श॒मये॒ दार्ति᳚म् । \newline
48. श॒मये॒ दार्ति॒ मार्तिꣳ॑ श॒मये᳚ च्छ॒मये॒ दार्ति॒ मा ऽऽर्तिꣳ॑ श॒मये᳚ च्छ॒मये॒ दार्ति॒ मा । \newline
49. आर्ति॒ मा ऽऽर्ति॒ मार्ति॒ मार्च्छे॑ दृच्छे॒दा ऽऽर्ति॒ मार्ति॒ मार्च्छे᳚त् । \newline
50. आर्च्छे॑ दृच्छे॒ दार्च्छे॒द् यज॑मानो॒ यज॑मान ऋच्छे॒ दार्च्छे॒द् यज॑मानः । \newline
51. ऋ॒च्छे॒द् यज॑मानो॒ यज॑मान ऋच्छे दृच्छे॒द् यज॑मानो म॒ही म॒ही यज॑मान ऋच्छे दृच्छे॒द् यज॑मानो म॒ही । \newline
52. यज॑मानो म॒ही म॒ही यज॑मानो॒ यज॑मानो म॒ही द्यौर् द्यौर् म॒ही यज॑मानो॒ यज॑मानो म॒ही द्यौः । \newline
53. म॒ही द्यौर् द्यौर् म॒ही म॒ही द्यौः पृ॑थि॒वी पृ॑थि॒वी द्यौर् म॒ही म॒ही द्यौः पृ॑थि॒वी । \newline
54. द्यौः पृ॑थि॒वी पृ॑थि॒वी द्यौर् द्यौः पृ॑थि॒वी च॑ च पृथि॒वी द्यौर् द्यौः पृ॑थि॒वी च॑ । \newline
55. पृ॒थि॒वी च॑ च पृथि॒वी पृ॑थि॒वी च॑ नो नश्च पृथि॒वी पृ॑थि॒वी च॑ नः । \newline
56. च॒ नो॒ न॒श्च॒ च॒ न॒ इतीति॑ नश्च च न॒ इति॑ । \newline
57. न॒ इतीति॑ नो न॒ इत्या॑ हा॒हे ति॑ नो न॒ इत्या॑ह । \newline
58. इत्या॑ हा॒हे तीत्या॑ह॒ द्यावा॑पृथि॒वीभ्या॒म् द्यावा॑पृथि॒वीभ्या॑ मा॒हे तीत्या॑ह॒ द्यावा॑पृथि॒वीभ्या᳚म् । \newline
\pagebreak
\markright{ TS 3.4.1.3  \hfill https://www.vedavms.in \hfill}

\section{ TS 3.4.1.3 }

\textbf{TS 3.4.1.3 } \newline
\textbf{Samhita Paata} \newline

ह॒ द्यावा॑पृथि॒वीभ्या॑मे॒व य॒ज्ञ्स्य॒ व्यृ॑द्धं॒ चाति॑रिक्तं च शमयति॒ नाऽऽ*र्ति॒मार्च्छ॑ति॒ यज॑मानो॒ भस्म॑ना॒ऽभि समू॑हति स्व॒गाकृ॑त्या॒ अथो॑ अ॒नयो॒र्वा ए॒ष गर्भो॒ऽनयो॑रे॒वैनं॑ दधाति॒ यद॑व॒द्येदति॒ तद्रे॑चये॒द्यन्नाव॒द्येत् प॒शोराल॑ब्धस्य॒ नाव॑ द्येत् पु॒रस्ता॒न्नाभ्या॑ अ॒न्यद॑व॒द्ये-दु॒परि॑ष्टाद॒न्यत् पु॒रस्ता॒द्वै नाभ्यै᳚ - [  ] \newline

\textbf{Pada Paata} \newline

आ॒ह॒ । द्यावा॑पृथि॒वीभ्या॒मिति॒ द्यावा᳚ - पृ॒थि॒वीभ्या᳚म् । ए॒व । य॒ज्ञ्स्य॑ । व्यृ॑द्ध॒मिति॒ वि - ऋ॒द्ध॒म् । च॒ । अति॑रिक्त॒मित्यति॑ - रि॒क्त॒म् । च॒ । श॒म॒य॒ति॒ । न । आर्ति᳚म् । एति॑ । ऋ॒च्छ॒ति॒ । यज॑मानः । भस्म॑ना । अ॒भि । समिति॑ । ऊ॒ह॒ति॒ । स्व॒गाकृ॑त्या॒ इति॑ स्व॒गा - कृ॒त्यै॒ । अथो॒ इति॑ । अ॒नयोः᳚ । वै । ए॒षः । गर्भः॑ । अ॒नयोः᳚ । ए॒व । ए॒न॒म् । द॒धा॒ति॒ । यत् । अ॒व॒द्येदित्य॑व-द्येत् । अतीति॑ । तत् । रे॒च॒ये॒त् । यत् । न । अ॒व॒द्येदित्य॑व - द्येत् । प॒शोः । आल॑ब्ध॒स्येत्या-ल॒ब्ध॒स्य॒ । न । अवेति॑ । द्ये॒त् । पु॒रस्ता᳚त् । नाभ्याः᳚ । अ॒न्यत् । अ॒व॒द्येदित्य॑व - द्येत् । उ॒परि॑ष्टात् । अ॒न्यत् । पु॒रस्ता᳚त् । वै । नाभ्यै᳚ ।  \newline


\textbf{Krama Paata} \newline

आ॒ह॒ द्यावा॑पृथि॒वीभ्या᳚म् । द्यावा॑पृथि॒वीभ्या॑मे॒व । द्यावा॑पृथि॒वीभ्या॒मिति॒ द्यावा᳚ - पृ॒थि॒वीभ्या᳚म् । ए॒व य॒ज्ञ्स्य॑ । य॒ज्ञ्स्य॒ व्यृ॑द्धम् । व्यृ॑द्धम् च । व्यृ॑द्ध॒मिति॒ वि - ऋ॒द्ध॒म् । चाति॑रिक्तम् । अति॑रिक्तम् च । अति॑रिक्त॒मित्यति॑ - रि॒क्त॒म् । च॒ श॒म॒य॒ति॒ । श॒म॒य॒ति॒ न । नार्ति᳚म् । आर्ति॒मा । आर्च्छ॑ति । ऋ॒च्छ॒ति॒ यज॑मानः । यज॑मानो॒ भस्म॑ना । भस्म॑ना॒ ऽभि । अ॒भि सम् । समू॑हति । ऊ॒ह॒ति॒ स्व॒गाकृ॑त्यै । स्व॒गाकृ॑त्या॒ अथो᳚ । स्व॒गाकृ॑त्या॒ इति॑ स्व॒गा - कृ॒त्यै॒ । अथो॑ अ॒नयोः᳚ । अथो॒ इत्यथो᳚ । अ॒नयो॒र् वै । वा ए॒षः । ए॒ष गर्भः॑ । गर्भो॒ ऽनयोः᳚ । अ॒नयो॑रे॒व । ए॒वैन᳚म् । ए॒न॒म् द॒धा॒ति॒ । द॒धा॒ति॒ यत् । यद॑व॒द्येत् । अ॒व॒द्येदति॑ । अ॒व॒द्येदित्य॑व - द्येत् । अति॒ तत् । तद् रे॑चयेत् । रे॒च॒ये॒द् यत् । यन् न । नाव॒द्येत् । अ॒व॒द्येत् प॒शोः । अ॒व॒द्येदित्य॑व - द्येत् । प॒शोराल॑ब्धस्य । आल॑ब्धस्य॒ न । आल॑ब्ध॒स्येत्या - ल॒ब्ध॒स्य॒ । नाव॑ । 
अव॑ द्येत् । द्ये॒त् पु॒रस्ता᳚त् । पु॒रस्ता॒न् नाभ्याः᳚ । नाभ्या॑ अ॒न्यत् । अ॒न्यद॑व॒द्येत् । अ॒व॒द्येदु॒परि॑ष्टात् । अ॒व॒द्येदित्य॑व - द्येत् । उ॒परि॑ष्टाद॒न्यत् । अ॒न्यत् पु॒रस्ता᳚त् । पु॒रस्ता॒द् वै । वै नाभ्यै᳚ । नाभ्यै᳚ प्रा॒णः \newline

\textbf{Jatai Paata} \newline

1. आ॒ह॒ द्यावा॑पृथि॒वीभ्या॒म् द्यावा॑पृथि॒वीभ्या॑ माहाह॒ द्यावा॑पृथि॒वीभ्या᳚म् । \newline
2. द्यावा॑पृथि॒वीभ्या॑ मे॒वैव द्यावा॑पृथि॒वीभ्या॒म् द्यावा॑पृथि॒वीभ्या॑ मे॒व । \newline
3. द्यावा॑पृथि॒वीभ्या॒मिति॒ द्यावा᳚ - पृ॒थि॒वीभ्या᳚म् । \newline
4. ए॒व य॒ज्ञ्स्य॑ य॒ज्ञ्स्यै॒वैव य॒ज्ञ्स्य॑ । \newline
5. य॒ज्ञ्स्य॒ व्यृ॑द्धं॒ ॅव्यृ॑द्धं ॅय॒ज्ञ्स्य॑ य॒ज्ञ्स्य॒ व्यृ॑द्धम् । \newline
6. व्यृ॑द्धम् च च॒ व्यृ॑द्धं॒ ॅव्यृ॑द्धम् च । \newline
7. व्यृ॑द्ध॒मिति॒ वि - ऋ॒द्ध॒म् । \newline
8. चाति॑रिक्त॒ मति॑रिक्तम् च॒ चाति॑रिक्तम् । \newline
9. अति॑रिक्तम् च॒ चाति॑रिक्त॒ मति॑रिक्तम् च । \newline
10. अति॑रिक्त॒मित्यति॑ - रि॒क्त॒म् । \newline
11. च॒ श॒म॒य॒ति॒ श॒म॒य॒ति॒ च॒ च॒ श॒म॒य॒ति॒ । \newline
12. श॒म॒य॒ति॒ न न श॑मयति शमयति॒ न । \newline
13. नार्ति॒ मार्ति॒म् न नार्ति᳚म् । \newline
14. आर्ति॒ मा ऽऽर्ति॒ मार्ति॒ मा । \newline
15. आर्च्छ॑ त्यृच्छ॒ त्यार्च्छ॑ति । \newline
16. ऋ॒च्छ॒ति॒ यज॑मानो॒ यज॑मान ऋच्छ त्यृच्छति॒ यज॑मानः । \newline
17. यज॑मानो॒ भस्म॑ना॒ भस्म॑ना॒ यज॑मानो॒ यज॑मानो॒ भस्म॑ना । \newline
18. भस्म॑ना॒ ऽभ्य॑भि भस्म॑ना॒ भस्म॑ना॒ ऽभि । \newline
19. अ॒भि सꣳ स म॒भ्य॑भि सम् । \newline
20. स मू॑ह त्यूहति॒ सꣳ समू॑हति । \newline
21. ऊ॒ह॒ति॒ स्व॒गाकृ॑त्यै स्व॒गाकृ॑त्या ऊह त्यूहति स्व॒गाकृ॑त्यै । \newline
22. स्व॒गाकृ॑त्या॒ अथो॒ अथो᳚ स्व॒गाकृ॑त्यै स्व॒गाकृ॑त्या॒ अथो᳚ । \newline
23. स्व॒गाकृ॑त्या॒ इति॑ स्व॒गा - कृ॒त्यै॒ । \newline
24. अथो॑ अ॒नयो॑ र॒नयो॒ रथो॒ अथो॑ अ॒नयोः᳚ । \newline
25. अथो॒ इत्यथो᳚ । \newline
26. अ॒नयो॒र् वै वा अ॒नयो॑ र॒नयो॒र् वै । \newline
27. वा ए॒ष ए॒ष वै वा ए॒षः । \newline
28. ए॒ष गर्भो॒ गर्भ॑ ए॒ष ए॒ष गर्भः॑ । \newline
29. गर्भो॒ ऽनयो॑ र॒नयो॒र् गर्भो॒ गर्भो॒ ऽनयोः᳚ । \newline
30. अ॒नयो॑ रे॒वैवानयो॑ र॒नयो॑ रे॒व । \newline
31. ए॒वैन॑ मेन मे॒वैवैन᳚म् । \newline
32. ए॒न॒म् द॒धा॒ति॒ द॒धा॒त् ये॒न॒ मे॒न॒म् द॒धा॒ति॒ । \newline
33. द॒धा॒ति॒ यद् यद् द॑धाति दधाति॒ यत् । \newline
34. यद॑व॒द्ये द॑व॒द्येद् यद् यद॑व॒द्येत् । \newline
35. अ॒व॒द्ये दत्य त्य॑व॒द्ये द॑व॒द्ये दति॑ । \newline
36. अ॒व॒द्येदित्य॑व - द्येत् । \newline
37. अति॒ तत् तदत्यति॒ तत् । \newline
38. तद् रे॑चयेद् रेचये॒त् तत् तद् रे॑चयेत् । \newline
39. रे॒च॒ये॒द् यद् यद् रे॑चयेद् रेचये॒द् यत् । \newline
40. यन् न न यद् यन् न । \newline
41. ना व॒द्ये द॑व॒द्येन् न नाव॒द्येत् । \newline
42. अ॒व॒द्येत् प॒शोः प॒शो र॑व॒द्ये द॑व॒द्येत् प॒शोः । \newline
43. अ॒व॒द्येदित्य॑व - द्येत् । \newline
44. प॒शो राल॑ब्ध॒स्या ल॑ब्धस्य प॒शोः प॒शो राल॑ब्धस्य । \newline
45. आल॑ब्धस्य॒ न नाल॑ब्ध॒स्या ल॑ब्धस्य॒ न । \newline
46. आल॑ब्ध॒स्येत्या - ल॒ब्ध॒स्य॒ । \newline
47. नावाव॒ न नाव॑ । \newline
48. अव॑ द्येद् द्ये॒द वाव॑ द्येत् । \newline
49. द्ये॒त् पु॒रस्ता᳚त् पु॒रस्ता᳚द् द्येद् द्येत् पु॒रस्ता᳚त् । \newline
50. पु॒रस्ता॒न् नाभ्या॒ नाभ्याः᳚ पु॒रस्ता᳚त् पु॒रस्ता॒न् नाभ्याः᳚ । \newline
51. नाभ्या॑ अ॒न्य द॒न्यन् नाभ्या॒ नाभ्या॑ अ॒न्यत् । \newline
52. अ॒न्य द॑व॒द्ये द॑व॒द्ये द॒न्य द॒न्य द॑व॒द्येत् । \newline
53. अ॒व॒द्ये दु॒परि॑ष्टा दु॒परि॑ष्टा दव॒द्ये द॑व॒द्ये दु॒परि॑ष्टात् । \newline
54. अ॒व॒द्येदित्य॑व - द्येत् । \newline
55. उ॒परि॑ष्टा द॒न्य द॒न्य दु॒परि॑ष्टा दु॒परि॑ष्टा द॒न्यत् । \newline
56. अ॒न्यत् पु॒रस्ता᳚त् पु॒रस्ता॑ द॒न्य द॒न्यत् पु॒रस्ता᳚त् । \newline
57. पु॒रस्ता॒द् वै वै पु॒रस्ता᳚त् पु॒रस्ता॒द् वै । \newline
58. वै नाभ्यै॒ नाभ्यै॒ वै वै नाभ्यै᳚ । \newline
59. नाभ्यै᳚ प्रा॒णः प्रा॒णो नाभ्यै॒ नाभ्यै᳚ प्रा॒णः । \newline

\textbf{Ghana Paata } \newline

1. आ॒ह॒ द्यावा॑पृथि॒वीभ्या॒म् द्यावा॑पृथि॒वीभ्या॑ माहाह॒ द्यावा॑पृथि॒वीभ्या॑ मे॒वैव द्यावा॑पृथि॒वीभ्या॑ 
माहाह॒ द्यावा॑पृथि॒वीभ्या॑ मे॒व । \newline
2. द्यावा॑पृथि॒वीभ्या॑ मे॒वैव द्यावा॑पृथि॒वीभ्या॒म् द्यावा॑पृथि॒वीभ्या॑ मे॒व य॒ज्ञ्स्य॑ य॒ज्ञ्स्यै॒व द्यावा॑पृथि॒वीभ्या॒म् द्यावा॑पृथि॒वीभ्या॑ मे॒व य॒ज्ञ्स्य॑ । \newline
3. द्यावा॑पृथि॒वीभ्या॒मिति॒ द्यावा᳚ - पृ॒थि॒वीभ्या᳚म् । \newline
4. ए॒व य॒ज्ञ्स्य॑ य॒ज्ञ्स्यै॒ वैव य॒ज्ञ्स्य॒ व्यृ॑द्ध॒म् ॅव्यृ॑द्धम् ॅय॒ज्ञ्स्यै॒ वैव य॒ज्ञ्स्य॒ व्यृ॑द्धम् । \newline
5. य॒ज्ञ्स्य॒ व्यृ॑द्ध॒म् ॅव्यृ॑द्धम् ॅय॒ज्ञ्स्य॑ य॒ज्ञ्स्य॒ व्यृ॑द्धम् च च॒ व्यृ॑द्धम् ॅय॒ज्ञ्स्य॑ य॒ज्ञ्स्य॒ व्यृ॑द्धम् च । \newline
6. व्यृ॑द्धम् च च॒ व्यृ॑द्ध॒म् ॅव्यृ॑द्ध॒म् चाति॑रिक्त॒ मति॑रिक्तम् च॒ व्यृ॑द्ध॒म् ॅव्यृ॑द्ध॒म् चाति॑रिक्तम् । \newline
7. व्यृ॑द्ध॒मिति॒ वि - ऋ॒द्ध॒म् । \newline
8. चाति॑रिक्त॒ मति॑रिक्तम् च॒ चाति॑रिक्तम् च॒ चाति॑रिक्तम् च॒ चाति॑रिक्तम् च । \newline
9. अति॑रिक्तम् च॒ चाति॑रिक्त॒ मति॑रिक्तम् च शमयति शमयति॒ चाति॑रिक्त॒ मति॑रिक्तम् च शमयति । \newline
10. अति॑रिक्त॒मित्यति॑ - रि॒क्त॒म् । \newline
11. च॒ श॒म॒य॒ति॒ श॒म॒य॒ति॒ च॒ च॒ श॒म॒य॒ति॒ न न श॑मयति च च शमयति॒ न । \newline
12. श॒म॒य॒ति॒ न न श॑मयति शमयति॒ नार्ति॒ मार्ति॒म् न श॑मयति शमयति॒ नार्ति᳚म् । \newline
13. नार्ति॒ मार्ति॒म् न नार्ति॒ मा ऽऽर्ति॒म् न नार्ति॒ मा । \newline
14. आर्ति॒ मा ऽऽर्ति॒ मार्ति॒ मार्च्छ॑ त्यृच्छ॒ त्याऽऽर्ति॒ मार्ति॒ मार्च्छ॑ति । \newline
15. आर्च्छ॑ त्यृच्छ॒ त्यार्च्छ॑ति॒ यज॑मानो॒ यज॑मान ऋच्छ॒ त्यार्च्छ॑ति॒ यज॑मानः । \newline
16. ऋ॒च्छ॒ति॒ यज॑मानो॒ यज॑मान ऋच्छ त्यृच्छति॒ यज॑मानो॒ भस्म॑ना॒ भस्म॑ना॒ यज॑मान ऋच्छ त्यृच्छति॒ यज॑मानो॒ भस्म॑ना । \newline
17. यज॑मानो॒ भस्म॑ना॒ भस्म॑ना॒ यज॑मानो॒ यज॑मानो॒ भस्म॑ना॒ ऽभ्य॑भि भस्म॑ना॒ यज॑मानो॒ यज॑मानो॒ भस्म॑ना॒ ऽभि । \newline
18. भस्म॑ना॒ ऽभ्य॑भि भस्म॑ना॒ भस्म॑ना॒ ऽभि सꣳ स म॒भि भस्म॑ना॒ भस्म॑ना॒ ऽभि सम् । \newline
19. अ॒भि सꣳ स म॒भ्य॑भि स मू॑ह त्यूहति॒ स म॒भ्य॑भि स मू॑हति । \newline
20. स मू॑ह त्यूहति॒ सꣳ स मू॑हति स्व॒गाकृ॑त्यै स्व॒गाकृ॑त्या ऊहति॒ सꣳ स मू॑हति स्व॒गाकृ॑त्यै । \newline
21. ऊ॒ह॒ति॒ स्व॒गाकृ॑त्यै स्व॒गाकृ॑त्या ऊह त्यूहति स्व॒गाकृ॑त्या॒ अथो॒ अथो᳚ स्व॒गाकृ॑त्या ऊह त्यूहति स्व॒गाकृ॑त्या॒ अथो᳚ । \newline
22. स्व॒गाकृ॑त्या॒ अथो॒ अथो᳚ स्व॒गाकृ॑त्यै स्व॒गाकृ॑त्या॒ अथो॑ अ॒नयो॑ र॒नयो॒ रथो᳚ स्व॒गाकृ॑त्यै स्व॒गाकृ॑त्या॒ अथो॑ अ॒नयोः᳚ । \newline
23. स्व॒गाकृ॑त्या॒ इति॑ स्व॒गा - कृ॒त्यै॒ । \newline
24. अथो॑ अ॒नयो॑ र॒नयो॒ रथो॒ अथो॑ अ॒नयो॒र् वै वा अ॒नयो॒ रथो॒ अथो॑ अ॒नयो॒र् वै । \newline
25. अथो॒ इत्यथो᳚ । \newline
26. अ॒नयो॒र् वै वा अ॒नयो॑ र॒नयो॒र् वा ए॒ष ए॒ष वा अ॒नयो॑ र॒नयो॒र् वा ए॒षः । \newline
27. वा ए॒ष ए॒ष वै वा ए॒ष गर्भो॒ गर्भ॑ ए॒ष वै वा ए॒ष गर्भः॑ । \newline
28. ए॒ष गर्भो॒ गर्भ॑ ए॒ष ए॒ष गर्भो॒ ऽनयो॑ र॒नयो॒र् गर्भ॑ ए॒ष ए॒ष गर्भो॒ ऽनयोः᳚ । \newline
29. गर्भो॒ ऽनयो॑ र॒नयो॒र् गर्भो॒ गर्भो॒ ऽनयो॑ रे॒वैवा नयो॒र् गर्भो॒ गर्भो॒ ऽनयो॑ रे॒व । \newline
30. अ॒नयो॑ रे॒वैवा नयो॑ र॒नयो॑ रे॒वैन॑ मेन मे॒वा नयो॑ र॒नयो॑ रे॒वैन᳚म् । \newline
31. ए॒वैन॑मेन मे॒वैवैन॑म् दधाति दधा त्येन मे॒वैवैन॑म् दधाति । \newline
32. ए॒न॒म् द॒धा॒ति॒ द॒धा॒ त्ये॒न॒ मे॒न॒म् द॒धा॒ति॒ यद् यद् द॑धा त्येन मेनम् दधाति॒ यत् । \newline
33. द॒धा॒ति॒ यद् यद् द॑धाति दधाति॒ यद॑व॒द्ये द॑व॒द्येद् यद् द॑धाति दधाति॒ यद॑व॒द्येत् । \newline
34. यद॑व॒द्ये द॑व॒द्येद् यद् यद॑व॒द्ये दत्य त्य॑व॒द्येद् यद् यद॑व॒द्ये दति॑ । \newline
35. अ॒व॒द्ये दत्य त्य॑व॒द्ये द॑व॒द्ये दति॒ तत् तद त्य॑व॒द्ये द॑व॒द्ये दति॒ तत् । \newline
36. अ॒व॒द्येदित्य॑व - द्येत् । \newline
37. अति॒ तत् तदत्यति॒ तद् रे॑चयेद् रेचये॒त् तद त्यति॒ तद् रे॑चयेत् । \newline
38. तद् रे॑चयेद् रेचये॒त् तत् तद् रे॑चये॒द् यद् यद् रे॑चये॒त् तत् तद् रे॑चये॒द् यत् । \newline
39. रे॒च॒ये॒द् यद् यद् रे॑चयेद् रेचये॒द् यन् न न यद् रे॑चयेद् रेचये॒द् यन् न । \newline
40. यन् न न यद् यन् नाव॒द्ये द॑व॒द्येन् न यद् यन् नाव॒द्येत् । \newline
41. ना व॒द्ये द॑व॒द्येन् न ना व॒द्येत् प॒शोः प॒शो र॑व॒द्येन् न ना व॒द्येत् प॒शोः । \newline
42. अ॒व॒द्येत् प॒शोः प॒शो र॑व॒द्ये द॑व॒द्येत् प॒शो राल॑ब्ध॒स्या ल॑ब्धस्य प॒शोर् अ॑व॒द्ये द॑व॒द्येत् प॒शोरा ल॑ब्धस्य । \newline
43. अ॒व॒द्येदित्य॑व - द्येत् । \newline
44. प॒शो राल॑ब्ध॒स्या ल॑ब्धस्य प॒शोः प॒शो राल॑ब्धस्य॒ न नाल॑ब्ध॒स्य प॒शोः प॒शो राल॑ब्धस्य॒ न । \newline
45. आल॑ब्धस्य॒ न ना ल॑ब्ध॒स्या ल॑ब्धस्य॒ ना वाव॒ नाल॑ब्ध॒स्या ल॑ब्धस्य॒ नाव॑ । \newline
46. आल॑ब्ध॒स्येत्या - ल॒ब्ध॒स्य॒ । \newline
47. ना वाव॒ न नाव॑ द्येद् द्ये॒दव॒ न नाव॑ द्येत् । \newline
48. अव॑ द्येद् द्ये॒दवाव॑ द्येत् पु॒रस्ता᳚त् पु॒रस्ता᳚द् द्ये॒दवाव॑ द्येत् पु॒रस्ता᳚त् । \newline
49. द्ये॒त् पु॒रस्ता᳚त् पु॒रस्ता᳚द् द्येद् द्येत् पु॒रस्ता॒न् नाभ्या॒ नाभ्याः᳚ पु॒रस्ता᳚द् द्येद् द्येत् पु॒रस्ता॒न् नाभ्याः᳚ । \newline
50. पु॒रस्ता॒न् नाभ्या॒ नाभ्याः᳚ पु॒रस्ता᳚त् पु॒रस्ता॒न् नाभ्या॑ अ॒न्य द॒न्यन् नाभ्याः᳚ पु॒रस्ता᳚त् पु॒रस्ता॒न् नाभ्या॑ अ॒न्यत् । \newline
51. नाभ्या॑ अ॒न्य द॒न्यन् नाभ्या॒ नाभ्या॑ अ॒न्य द॑व॒द्ये द॑व॒द्ये द॒न्यन् नाभ्या॒ नाभ्या॑ अ॒न्य द॑व॒द्येत् । \newline
52. अ॒न्य द॑व॒द्ये द॑व॒द्ये द॒न्य द॒न्य द॑व॒द्ये दु॒परि॑ष्टा दु॒परि॑ष्टा दव॒द्ये द॒न्य द॒न्य द॑व॒द्ये दु॒परि॑ष्टात् । \newline
53. अ॒व॒द्ये दु॒परि॑ष्टा दु॒परि॑ष्टा दव॒द्ये द॑व॒द्ये दु॒परि॑ष्टा द॒न्य द॒न्य दु॒परि॑ष्टा दव॒द्ये द॑व॒द्ये दु॒परि॑ष्टा द॒न्यत् । \newline
54. अ॒व॒द्येदित्य॑व - द्येत् । \newline
55. उ॒परि॑ष्टा द॒न्य द॒न्य दु॒परि॑ष्टा दु॒परि॑ष्टा द॒न्यत् पु॒रस्ता᳚त् पु॒रस्ता॑ द॒न्य दु॒परि॑ष्टा दु॒परि॑ष्टा द॒न्यत् पु॒रस्ता᳚त् । \newline
56. अ॒न्यत् पु॒रस्ता᳚त् पु॒रस्ता॑ द॒न्य द॒न्यत् पु॒रस्ता॒द् वै वै पु॒रस्ता॑ द॒न्य द॒न्यत् पु॒रस्ता॒द् वै । \newline
57. पु॒रस्ता॒द् वै वै पु॒रस्ता᳚त् पु॒रस्ता॒द् वै नाभ्यै॒ नाभ्यै॒ वै पु॒रस्ता᳚त् पु॒रस्ता॒द् वै नाभ्यै᳚ । \newline
58. वै नाभ्यै॒ नाभ्यै॒ वै वै नाभ्यै᳚ प्रा॒णः प्रा॒णो नाभ्यै॒ वै वै नाभ्यै᳚ प्रा॒णः । \newline
59. नाभ्यै᳚ प्रा॒णः प्रा॒णो नाभ्यै॒ नाभ्यै᳚ प्रा॒ण उ॒परि॑ष्टा दु॒परि॑ष्टात् प्रा॒णो नाभ्यै॒ नाभ्यै᳚ प्रा॒ण उ॒परि॑ष्टात् । \newline
\pagebreak
\markright{ TS 3.4.1.4  \hfill https://www.vedavms.in \hfill}

\section{ TS 3.4.1.4 }

\textbf{TS 3.4.1.4 } \newline
\textbf{Samhita Paata} \newline

प्रा॒ण उ॒परि॑ष्टादपा॒नो यावा॑ने॒व प॒शुस्तस्याव॑ द्यति॒ विष्ण॑वे शिपिवि॒ष्टाय॑ जुहोति॒ यद्वै य॒ज्ञ्स्या॑ति॒रिच्य॑ते॒ यः प॒शोर्भू॒मा या पुष्टि॒स्तद्-विष्णुः॑ शिपिवि॒ष्टो ऽति॑रिक्त ए॒वाति॑रिक्तं दधा॒त्यति॑रिक्तस्य॒ शान्त्या॑ अ॒ष्टाप्रू॒ड्ढिर॑ण्यं॒ दक्षि॑णा॒ऽष्टाप॑दी॒ ह्ये॑षा ऽऽत्मा न॑व॒मः प॒शोराप्त्या॑ अन्तरको॒श उ॒ष्णीषे॒णाऽऽ*वि॑ष्टितं भवत्ये॒वमि॑व॒ हि प॒शुरुल्ब॑मिव॒ ( ) चर्मे॑व माꣳ॒॒समि॒वास्थी॑व॒ यावा॑ने॒व प॒शुस्तमा॒प्त्वाऽव॑ रुन्धे॒यस्यै॒षा य॒ज्ञे प्राय॑श्चित्तिः क्रि॒यत॑ इ॒ष्ट्वा वसी॑यान् भवति ॥ \newline

\textbf{Pada Paata} \newline

प्रा॒ण इति॑ प्र - अ॒नः । उ॒परि॑ष्टात् । अ॒पा॒न इत्य॑प-अ॒नः । यावान्॑ । ए॒व । प॒शुः । तस्य॑ । अवेति॑ । द्य॒ति॒ । विष्ण॑वे । शि॒पि॒वि॒ष्टायेति॑ शिपि - वि॒ष्टाय॑ । जु॒हो॒ति॒ । यत् । वै । य॒ज्ञ्स्य॑ । अ॒ति॒रिच्य॑त॒ इत्य॑ति - रिच्य॑ते । यः । प॒शोः । भू॒मा । या । पुष्टिः॑ । तत् । विष्णुः॑ । शि॒पि॒वि॒ष्ट इति॑ शिपि - वि॒ष्टः । अति॑रिक्त॒ इत्यति॑ - रि॒क्ते॒ । ए॒व । अति॑रिक्त॒मित्यति॑ - रि॒क्त॒म् । द॒धा॒ति॒ । अति॑रिक्त॒स्येत्यति॑ - रि॒क्त॒स्य॒ । शान्त्यै᳚ । अ॒ष्टाप्रू॒डित्य॒ष्टा - प्रू॒ट् । हिर॑ण्यम् । दक्षि॑णा । अ॒ष्टाप॒दीत्य॒ष्टा - प॒दी॒ । हि । ए॒षा । आ॒त्मा । न॒व॒मः । प॒शोः । आप्त्यै᳚ । अ॒न्त॒र॒को॒श इत्य॑न्तर - को॒शे । उ॒ष्णीषे॑ण । आवि॑ष्टित॒मित्या - वि॒ष्टि॒त॒म् । भ॒व॒ति॒ । ए॒वम् । इ॒व॒ । हि । प॒शुः । उल्ब᳚म् । इ॒व॒ ( ) । चर्म॑ । इ॒व॒ । माꣳ॒॒सम् । इ॒व॒ । अस्थि॑ । इ॒व॒ । यावान्॑ । ए॒व । प॒शुः । तम् । आ॒प्त्वा । अवेति॑ । रु॒न्धे॒ । यस्य॑ । ए॒षा । य॒ज्ञे । प्राय॑श्चित्तिः । क्रि॒यते᳚ । इ॒ष्ट्वा । वसी॑यान् । भ॒व॒ति॒ ॥  \newline


\textbf{Krama Paata} \newline

प्रा॒ण उ॒परि॑ष्टात् । प्रा॒ण इति॑ प्र - अ॒नः । उ॒परि॑ष्टादपा॒नः । अ॒पा॒नो यावान्॑ । अ॒पा॒न इत्य॑प - अ॒नः । यावा॑ने॒व । ए॒व प॒शुः । प॒शुस्तस्य॑ । तस्याव॑ । अव॑ द्यति । द्य॒ति॒ विष्ण॑वे । विष्ण॑वे शिपिवि॒ष्टाय॑ । शि॒पि॒वि॒ष्टाय॑ जुहोति । शि॒पि॒वि॒ष्टायेति॑ शिपि - वि॒ष्टाय॑ । जु॒हो॒ति॒ यत् । यद् वै । वै य॒ज्ञ्स्य॑ । य॒ज्ञ्स्या॑ति॒रिच्य॑ते । अ॒ति॒रिच्य॑ते॒ यः । अ॒ति॒रिच्य॑त॒ इत्य॑ति - रिच्य॑ते । यः प॒शोः । प॒शोर् भू॒मा । भू॒मा या । या पुष्टिः॑ । पुष्टि॒स्तत् । तद् विष्णुः॑ । विष्णुः॑ शिपिवि॒ष्टः । शि॒पि॒वि॒ष्टो ऽति॑रिक्ते । शि॒पि॒वि॒ष्ट इति॑ शिपि - वि॒ष्टः । अति॑रिक्त ए॒व । अति॑रिक्त॒ इत्यति॑ - रि॒क्ते॒ । ए॒वाति॑रिक्तम् । अति॑रिक्तम् दधाति । अति॑रिक्त॒मित्यति॑ - रि॒क्त॒म् । द॒धा॒त्यति॑रिक्तस्य । अति॑रिक्तस्य॒ शान्त्यै᳚ । अति॑रिक्त॒स्येत्यति॑ - रि॒क्त॒स्य॒ । शान्त्या॑ अ॒ष्टाप्रू᳚ट् । अ॒ष्टाप्रू॒ड्ढिर॑ण्यम् । अ॒ष्टाप्रू॒डित्य॒ष्टा - प्रू॒ट्॒ । हिर॑ण्य॒म् दक्षि॑णा । दक्षि॑णा॒ ऽष्टाप॑दी । अ॒ष्टाप॑दी॒ हि । अ॒ष्टाप॒दीत्य॒ष्टा - प॒दी॒ । ह्ये॑षा । ए॒षा ऽऽत्मा । आ॒त्मा न॑व॒मः । न॒व॒मः प॒शोः । प॒शोराप्त्यै᳚ । आप्त्या॑ अन्तरको॒शे । अ॒न्त॒र॒को॒श उ॒ष्णीषे॑ण । अ॒न्त॒र॒को॒श इत्य॑न्तर - को॒शे । उ॒ष्णीषे॒णाऽऽवि॑ष्टितम् । आवि॑ष्टितम् भवति । आवि॑ष्टित॒मित्या - वि॒ष्टि॒त॒म् । भ॒व॒त्ये॒वम् । ए॒वमि॑व । इ॒व॒ हि । हि प॒शुः । प॒शुरुल्ब᳚म् । उल्ब॑मिव ( ) । इ॒व॒ चर्म॑ । चर्मे॑व । इ॒व॒ माꣳ॒॒सम् । माꣳ॒॒समि॑व । इ॒वास्थि॑ । अस्थी॑व । इ॒व॒ यावान्॑ । यावा॑ने॒व । ए॒व प॒शुः । प॒शुस्तम् । तमा॒प्त्वा । आ॒प्त्वा ऽव॑ । अव॑ रुन्धे । रु॒न्धे॒ यस्य॑ । यस्यै॒षा । ए॒षा य॒ज्ञे । य॒ज्ञे प्राय॑श्चित्तिः । प्राय॑श्चित्तिः क्रि॒यते᳚ । क्रि॒यत॑ इ॒ष्ट्वा । इ॒ष्ट्वा वसी॑यान् । वसी॑यान् भवति । भ॒व॒तीति॑ भवति । \newline

\textbf{Jatai Paata} \newline

1. प्रा॒ण उ॒परि॑ष्टा दु॒परि॑ष्टात् प्रा॒णः प्रा॒ण उ॒परि॑ष्टात् । \newline
2. प्रा॒ण इति॑ प्र - अ॒नः । \newline
3. उ॒परि॑ष्टा दपा॒नो॑ ऽपा॒न उ॒परि॑ष्टा दु॒परि॑ष्टा दपा॒नः । \newline
4. अ॒पा॒नो यावा॒न्॒. यावा॑-नपा॒नो॑ ऽपा॒नो यावान्॑ । \newline
5. अ॒पा॒न इत्य॑प - अ॒नः । \newline
6. यावा॑ने॒वैव यावा॒न्॒. यावा॑-ने॒व । \newline
7. ए॒व प॒शुः प॒शु रे॒वैव प॒शुः । \newline
8. प॒शु स्तस्य॒ तस्य॑ प॒शुः प॒शु स्तस्य॑ । \newline
9. तस्या वाव॒ तस्य॒ तस्याव॑ । \newline
10. अव॑ द्यति द्य॒त्यवाव॑ द्यति । \newline
11. द्य॒ति॒ विष्ण॑वे॒ विष्ण॑वे द्यति द्यति॒ विष्ण॑वे । \newline
12. विष्ण॑वे शिपिवि॒ष्टाय॑ शिपिवि॒ष्टाय॒ विष्ण॑वे॒ विष्ण॑वे शिपिवि॒ष्टाय॑ । \newline
13. शि॒पि॒वि॒ष्टाय॑ जुहोति जुहोति शिपिवि॒ष्टाय॑ शिपिवि॒ष्टाय॑ जुहोति । \newline
14. शि॒पि॒वि॒ष्टायेति॑ शिपि - वि॒ष्टाय॑ । \newline
15. जु॒हो॒ति॒ यद् यज् जु॑होति जुहोति॒ यत् । \newline
16. यद् वै वै यद् यद् वै । \newline
17. वै य॒ज्ञ्स्य॑ य॒ज्ञ्स्य॒ वै वै य॒ज्ञ्स्य॑ । \newline
18. य॒ज्ञ्स्या॑ ति॒रिच्य॑ते ऽति॒रिच्य॑ते य॒ज्ञ्स्य॑ य॒ज्ञ्स्या॑ ति॒रिच्य॑ते । \newline
19. अ॒ति॒रिच्य॑ते॒ यो यो॑ ऽति॒रिच्य॑ते ऽति॒रिच्य॑ते॒ यः । \newline
20. अ॒ति॒रिच्य॑त॒ इत्य॑ति - रिच्य॑ते । \newline
21. यः प॒शोः प॒शोर् यो यः प॒शोः । \newline
22. प॒शोर् भू॒मा भू॒मा प॒शोः प॒शोर् भू॒मा । \newline
23. भू॒मा या या भू॒मा भू॒मा या । \newline
24. या पुष्टिः॒ पुष्टि॒र् या या पुष्टिः॑ । \newline
25. पुष्टि॒ स्तत् तत् पुष्टिः॒ पुष्टि॒ स्तत् । \newline
26. तद् विष्णु॒र् विष्णु॒ स्तत् तद् विष्णुः॑ । \newline
27. विष्णुः॑ शिपिवि॒ष्टः शि॑पिवि॒ष्टो विष्णु॒र् विष्णुः॑ शिपिवि॒ष्टः । \newline
28. शि॒पि॒वि॒ष्टो ऽति॑रि॒क्ते ऽति॑रिक्ते शिपिवि॒ष्टः शि॑पिवि॒ष्टो ऽति॑रिक्ते । \newline
29. शि॒पि॒वि॒ष्ट इति॑ शिपि - वि॒ष्टः । \newline
30. अति॑रिक्त ए॒वैवाति॑रि॒क्ते ऽति॑रिक्त ए॒व । \newline
31. अति॑रिक्त॒ इत्यति॑ - रि॒क्ते॒ । \newline
32. ए॒वा ति॑रिक्त॒ मति॑रिक्त मे॒वैवा ति॑रिक्तम् । \newline
33. अति॑रिक्तम् दधाति दधा॒ त्यति॑रिक्त॒ मति॑रिक्तम् दधाति । \newline
34. अति॑रिक्त॒मित्यति॑ - रि॒क्त॒म् । \newline
35. द॒धा॒ त्यति॑रिक्त॒स्या ति॑रिक्तस्य दधाति दधा॒ त्यति॑रिक्तस्य । \newline
36. अति॑रिक्तस्य॒ शान्त्यै॒ शान्त्या॒ अति॑रिक्त॒स्या ति॑रिक्तस्य॒ शान्त्यै᳚ । \newline
37. अति॑रिक्त॒स्येत्यति॑ - रि॒क्त॒स्य॒ । \newline
38. शान्त्या॑ अ॒ष्टाप्रू॑ ड॒ष्टाप्रू॒ट् छान्त्यै॒ शान्त्या॑ अ॒ष्टाप्रू᳚ट् । \newline
39. अ॒ष्टाप्रू॒ ड्ढिर॑ण्यꣳ॒॒ हिर॑ण्य म॒ष्टाप्रू॑ ड॒ष्टाप्रू॒ ड्ढिर॑ण्यम् । \newline
40. अ॒ष्टाप्रू॒डित्य॒ष्टा - प्रू॒ट् । \newline
41. हिर॑ण्य॒म् दक्षि॑णा॒ दक्षि॑णा॒ हिर॑ण्यꣳ॒॒ हिर॑ण्य॒म् दक्षि॑णा । \newline
42. दक्षि॑णा॒ ऽष्टाप॑द्य॒ ष्टाप॑दी॒ दक्षि॑णा॒ दक्षि॑णा॒ ऽष्टाप॑दी । \newline
43. अ॒ष्टाप॑दी॒ हि ह्य॑ष्टाप॑ द्य॒ष्टाप॑दी॒ हि । \newline
44. अ॒ष्टाप॒दीत्य॒ष्टा - प॒दी॒ । \newline
45. ह्ये॑षैषा हि ह्ये॑षा । \newline
46. ए॒षा ऽऽत्मा ऽऽत्मैषैषा ऽऽत्मा । \newline
47. आ॒त्मा न॑व॒मो न॑व॒म आ॒त्मा ऽऽत्मा न॑व॒मः । \newline
48. न॒व॒मः प॒शोः प॒शोर् न॑व॒मो न॑व॒मः प॒शोः । \newline
49. प॒शो राप्त्या॒ आप्त्यै॑ प॒शोः प॒शो राप्त्यै᳚ । \newline
50. आप्त्या॑ अन्तरको॒शे᳚ ऽन्तरको॒श आप्त्या॒ आप्त्या॑ अन्तरको॒शे । \newline
51. अ॒न्त॒र॒को॒श उ॒ष्णीषे॑ णो॒ष्णीषे॑णा न्तरको॒शे᳚ ऽन्तरको॒श उ॒ष्णीषे॑ण । \newline
52. अ॒न्त॒र॒को॒श इत्य॑न्तर - को॒शे । \newline
53. उ॒ष्णीषे॒णा वि॑ष्टित॒ मावि॑ष्टित मु॒ष्णीषे॑ णो॒ष्णीषे॒णा वि॑ष्टितम् । \newline
54. आवि॑ष्टितम् भवति भव॒ त्यावि॑ष्टित॒ मावि॑ष्टितम् भवति । \newline
55. आवि॑ष्टित॒मित्या - वि॒ष्टि॒त॒म् । \newline
56. भ॒व॒ त्ये॒व मे॒वम् भ॑वति भव त्ये॒वम् । \newline
57. ए॒व मि॑वेवै॒व मे॒व मि॑व । \newline
58. इ॒व॒ हि हीवे॑व॒ हि । \newline
59. हि प॒शुः प॒शुर्. हि हि प॒शुः । \newline
60. प॒शु रुल्ब॒ मुल्ब॑म् प॒शुः प॒शु रुल्ब᳚म् । \newline
61. उल्ब॑ मिवे॒ वोल्ब॒ मुल्ब॑ मिव । \newline
62. इ॒व॒ चर्म॒ चर्मे॑वेव॒ चर्म॑ । \newline
63. चर्मे॑वेव॒ चर्म॒ चर्मे॑व । \newline
64. इ॒व॒ माꣳ॒॒सम् माꣳ॒॒स मि॑वेव माꣳ॒॒सम् । \newline
65. माꣳ॒॒स मि॑वेव माꣳ॒॒सम् माꣳ॒॒स मि॑व । \newline
66. इ॒वास्थ्यस्थी॑ वे॒वास्थि॑ । \newline
67. अस्थी॑ वे॒वास्थ्य स्थी॑व । \newline
68. इ॒व॒ यावा॒न्॒. यावा॑-निवेव॒ यावान्॑ । \newline
69. यावा॑-ने॒वैव यावा॒न्॒. यावा॑-ने॒व । \newline
70. ए॒व प॒शुः प॒शु रे॒वैव प॒शुः । \newline
71. प॒शु स्तम् तम् प॒शुः प॒शु स्तम् । \newline
72. त मा॒प्त्वा ऽऽप्त्वा तम् तमा॒प्त्वा । \newline
73. आ॒प्त्वा ऽवावा॒प्त्वा ऽऽप्त्वा ऽव॑ । \newline
74. अव॑ रुन्धे रु॒न्धे ऽवाव॑ रुन्धे । \newline
75. रु॒न्धे॒ यस्य॒ यस्य॑ रुन्धे रुन्धे॒ यस्य॑ । \newline
76. यस्यै॒षैषा यस्य॒ यस्यै॒षा । \newline
77. ए॒षा य॒ज्ञे य॒ज्ञ् ए॒षैषा य॒ज्ञे । \newline
78. य॒ज्ञे प्राय॑श्चित्तिः॒ प्राय॑श्चित्तिर् य॒ज्ञे य॒ज्ञे प्राय॑श्चित्तिः । \newline
79. प्राय॑श्चित्तिः क्रि॒यते᳚ क्रि॒यते॒ प्राय॑श्चित्तिः॒ प्राय॑श्चित्तिः क्रि॒यते᳚ । \newline
80. क्रि॒यत॑ इ॒ष्ट्वे ष्ट्वा क्रि॒यते᳚ क्रि॒यत॑ इ॒ष्ट्वा । \newline
81. इ॒ष्ट्वा वसी॑या॒न्॒. वसी॑या-नि॒ष्ट्वे ष्ट्वा वसी॑यान् । \newline
82. वसी॑यान् भवति भवति॒ वसी॑या॒न्॒. वसी॑यान् भवति । \newline
83. भ॒व॒तीति॑ भवति । \newline

\textbf{Ghana Paata } \newline

1. प्रा॒ण उ॒परि॑ष्टा दु॒परि॑ष्टात् प्रा॒णः प्रा॒ण उ॒परि॑ष्टा दपा॒नो॑ ऽपा॒न उ॒परि॑ष्टात् प्रा॒णः प्रा॒ण उ॒परि॑ष्टा दपा॒नः । \newline
2. प्रा॒ण इति॑ प्र - अ॒नः । \newline
3. उ॒परि॑ष्टा दपा॒नो॑ ऽपा॒न उ॒परि॑ष्टा दु॒परि॑ष्टा दपा॒नो यावा॒न्॒. यावा॑ नपा॒न उ॒परि॑ष्टा दु॒परि॑ष्टा दपा॒नो यावान्॑ । \newline
4. अ॒पा॒नो यावा॒न्॒. यावा॑ नपा॒नो॑ ऽपा॒नो यावा॑ ने॒वैव यावा॑ नपा॒नो॑ ऽपा॒नो यावा॑ ने॒व । \newline
5. अ॒पा॒न इत्य॑प - अ॒नः । \newline
6. यावा॑ ने॒वैव यावा॒न्॒. यावा॑ने॒व प॒शुः प॒शु रे॒व यावा॒न्॒. यावा॑ ने॒व प॒शुः । \newline
7. ए॒व प॒शुः प॒शु रे॒वैव प॒शु स्तस्य॒ तस्य॑ प॒शु रे॒वैव प॒शु स्तस्य॑ । \newline
8. प॒शु स्तस्य॒ तस्य॑ प॒शुः प॒शु स्तस्या वाव॒ तस्य॑ प॒शुः प॒शु स्तस्याव॑ । \newline
9. तस्या वाव॒ तस्य॒ तस्या व॑द्यति द्य॒त्यव॒ तस्य॒ तस्या व॑द्यति । \newline
10. अव॑ द्यति द्य॒त्यवाव॑ द्यति॒ विष्ण॑वे॒ विष्ण॑वे द्य॒त्यवाव॑ द्यति॒ विष्ण॑वे । \newline
11. द्य॒ति॒ विष्ण॑वे॒ विष्ण॑वे द्यति द्यति॒ विष्ण॑वे शिपिवि॒ष्टाय॑ शिपिवि॒ष्टाय॒ विष्ण॑वे द्यति द्यति॒ विष्ण॑वे शिपिवि॒ष्टाय॑ । \newline
12. विष्ण॑वे शिपिवि॒ष्टाय॑ शिपिवि॒ष्टाय॒ विष्ण॑वे॒ विष्ण॑वे शिपिवि॒ष्टाय॑ जुहोति जुहोति शिपिवि॒ष्टाय॒ विष्ण॑वे॒ विष्ण॑वे शिपिवि॒ष्टाय॑ जुहोति । \newline
13. शि॒पि॒वि॒ष्टाय॑ जुहोति जुहोति शिपिवि॒ष्टाय॑ शिपिवि॒ष्टाय॑ जुहोति॒ यद् यज् जु॑होति शिपिवि॒ष्टाय॑ शिपिवि॒ष्टाय॑ जुहोति॒ यत् । \newline
14. शि॒पि॒वि॒ष्टायेति॑ शिपि - वि॒ष्टाय॑ । \newline
15. जु॒हो॒ति॒ यद् यज् जु॑होति जुहोति॒ यद् वै वै यज् जु॑होति जुहोति॒ यद् वै । \newline
16. यद् वै वै यद् यद् वै य॒ज्ञ्स्य॑ य॒ज्ञ्स्य॒ वै यद् यद् वै य॒ज्ञ्स्य॑ । \newline
17. वै य॒ज्ञ्स्य॑ य॒ज्ञ्स्य॒ वै वै य॒ज्ञ्स्या॑ ति॒रिच्य॑ते ऽति॒रिच्य॑ते य॒ज्ञ्स्य॒ वै वै य॒ज्ञ्स्या॑ ति॒रिच्य॑ते । \newline
18. य॒ज्ञ्स्या॑ ति॒रिच्य॑ते ऽति॒रिच्य॑ते य॒ज्ञ्स्य॑ य॒ज्ञ्स्या॑ ति॒रिच्य॑ते॒ यो यो॑ ऽति॒रिच्य॑ते य॒ज्ञ्स्य॑ य॒ज्ञ्स्या॑ ति॒रिच्य॑ते॒ यः । \newline
19. अ॒ति॒रिच्य॑ते॒ यो यो॑ ऽति॒रिच्य॑ते ऽति॒रिच्य॑ते॒ यः प॒शोः प॒शोर् यो॑ ऽति॒रिच्य॑ते ऽति॒रिच्य॑ते॒ यः प॒शोः । \newline
20. अ॒ति॒रिच्य॑त॒ इत्य॑ति - रिच्य॑ते । \newline
21. यः प॒शोः प॒शोर् यो यः प॒शोर् भू॒मा भू॒मा प॒शोर् यो यः प॒शोर् भू॒मा । \newline
22. प॒शोर् भू॒मा भू॒मा प॒शोः प॒शोर् भू॒मा या या भू॒मा प॒शोः प॒शोर् भू॒मा या । \newline
23. भू॒मा या या भू॒मा भू॒मा या पुष्टिः॒ पुष्टि॒र् या भू॒मा भू॒मा या पुष्टिः॑ । \newline
24. या पुष्टिः॒ पुष्टि॒र् या या पुष्टि॒ स्तत् तत् पुष्टि॒र् या या पुष्टि॒स्तत् । \newline
25. पुष्टि॒ स्तत् तत् पुष्टिः॒ पुष्टि॒ स्तद् विष्णु॒र् विष्णु॒ स्तत् पुष्टिः॒ पुष्टि॒ स्तद् विष्णुः॑ । \newline
26. तद् विष्णु॒र् विष्णु॒ स्तत् तद् विष्णुः॑ शिपिवि॒ष्टः शि॑पिवि॒ष्टो विष्णु॒ स्तत् तद् विष्णुः॑ शिपिवि॒ष्टः । \newline
27. विष्णुः॑ शिपिवि॒ष्टः शि॑पिवि॒ष्टो विष्णु॒र् विष्णुः॑ शिपिवि॒ष्टो ऽति॑रि॒क्ते ऽति॑रिक्ते शिपिवि॒ष्टो विष्णु॒र् विष्णुः॑ शिपिवि॒ष्टो ऽति॑रिक्ते । \newline
28. शि॒पि॒वि॒ष्टो ऽति॑रि॒क्ते ऽति॑रिक्ते शिपिवि॒ष्टः शि॑पिवि॒ष्टो ऽति॑रिक्त ए॒वै वाति॑रिक्ते शिपिवि॒ष्टः शि॑पिवि॒ष्टो ऽति॑रिक्त ए॒व । \newline
29. शि॒पि॒वि॒ष्ट इति॑ शिपि - वि॒ष्टः । \newline
30. अति॑रिक्त ए॒वै वाति॑रि॒क्ते ऽति॑रिक्त ए॒वा ति॑रिक्त॒ मति॑रिक्त मे॒वा ति॑रि॒क्ते ऽति॑रिक्त ए॒वा ति॑रिक्तम् । \newline
31. अति॑रिक्त॒ इत्यति॑ - रि॒क्ते॒ । \newline
32. ए॒वा ति॑रिक्त॒ मति॑रिक्त मे॒वैवा ति॑रिक्तम् दधाति दधा॒ त्यति॑रिक्त मे॒वै वाति॑रिक्तम् दधाति । \newline
33. अति॑रिक्तम् दधाति दधा॒ त्यति॑रिक्त॒ मति॑रिक्तम् दधा॒ त्यति॑रिक्त॒स्या ति॑रिक्तस्य दधा॒ त्यति॑रिक्त॒ मति॑रिक्तम् दधा॒ त्यति॑रिक्तस्य । \newline
34. अति॑रिक्त॒मित्यति॑ - रि॒क्त॒म् । \newline
35. द॒धा॒ त्यति॑रिक्त॒स्या ति॑रिक्तस्य दधाति दधा॒ त्यति॑रिक्तस्य॒ शान्त्यै॒ शान्त्या॒ अति॑रिक्तस्य दधाति दधा॒ त्यति॑रिक्तस्य॒ शान्त्यै᳚ । \newline
36. अति॑रिक्तस्य॒ शान्त्यै॒ शान्त्या॒ अति॑रिक्त॒स्या ति॑रिक्तस्य॒ शान्त्या॑ अ॒ष्टाप्रू॑ ड॒ष्टाप्रू॒ट् छान्त्या॒ अति॑रिक्त॒स्या ति॑रिक्तस्य॒ शान्त्या॑ अ॒ष्टाप्रू᳚ट् । \newline
37. अति॑रिक्त॒स्येत्यति॑ - रि॒क्त॒स्य॒ । \newline
38. शान्त्या॑ अ॒ष्टाप्रू॑ ड॒ष्टाप्रू॒ट् छान्त्यै॒ शान्त्या॑ अ॒ष्टाप्रू॒ ड्ढिर॑ण्यꣳ॒॒ हिर॑ण्य म॒ष्टाप्रू॒ट् छान्त्यै॒ शान्त्या॑ अ॒ष्टाप्रू॒ ड्ढिर॑ण्यम् । \newline
39. अ॒ष्टाप्रू॒ ड्ढिर॑ण्यꣳ॒॒ हिर॑ण्य म॒ष्टा प्रू॑ड॒ष्टाप्रू॒ ड्ढिर॑ण्य॒म् दक्षि॑णा॒ दक्षि॑णा॒ हिर॑ण्य म॒ष्टाप्रू॑ ड॒ष्टाप्रू॒ ड्ढिर॑ण्य॒म् दक्षि॑णा । \newline
40. अ॒ष्टाप्रू॒डित्य॒ष्टा - प्रू॒ट् । \newline
41. हिर॑ण्य॒म् दक्षि॑णा॒ दक्षि॑णा॒ हिर॑ण्यꣳ॒॒ हिर॑ण्य॒म् दक्षि॑णा॒ ऽष्टाप॑ द्य॒ष्टाप॑दी॒ दक्षि॑णा॒ हिर॑ण्यꣳ॒॒ हिर॑ण्य॒म् दक्षि॑णा॒ ऽष्टाप॑दी । \newline
42. दक्षि॑णा॒ ऽष्टाप॑ द्य॒ष्टाप॑दी॒ दक्षि॑णा॒ दक्षि॑णा॒ ऽष्टाप॑दी॒ हि ह्य॑ष्टाप॑दी॒ दक्षि॑णा॒ दक्षि॑णा॒ ऽष्टाप॑दी॒ हि । \newline
43. अ॒ष्टाप॑दी॒ हि ह्य॑ष्टाप॑ द्य॒ष्टाप॑दी॒ ह्ये॑षैषा ह्य॑ष्टाप॑ द्य॒ष्टाप॑दी॒ ह्ये॑षा । \newline
44. अ॒ष्टाप॒दीत्य॒ष्टा - प॒दी॒ । \newline
45. ह्ये॑षैषा हि ह्ये॑षा ऽऽत्मा ऽऽत्मैषा हि ह्ये॑षा ऽऽत्मा । \newline
46. ए॒षा ऽऽत्मा ऽऽत्मैषैषा ऽऽत्मा न॑व॒मो न॑व॒म आ॒त्मैषैषा ऽऽत्मा न॑व॒मः । \newline
47. आ॒त्मा न॑व॒मो न॑व॒म आ॒त्मा ऽऽत्मा न॑व॒मः प॒शोः प॒शोर् न॑व॒म आ॒त्मा ऽऽत्मा न॑व॒मः प॒शोः । \newline
48. न॒व॒मः प॒शोः प॒शोर् न॑व॒मो न॑व॒मः प॒शो राप्त्या॒ आप्त्यै॑ प॒शोर् न॑व॒मो न॑व॒मः प॒शो राप्त्यै᳚ । \newline
49. प॒शो राप्त्या॒ आप्त्यै॑ प॒शोः प॒शो राप्त्या॑ अन्तरको॒शे᳚ ऽन्तरको॒श आप्त्यै॑ प॒शोः प॒शो राप्त्या॑ अन्तरको॒शे । \newline
50. आप्त्या॑ अन्तरको॒शे᳚ ऽन्तरको॒श आप्त्या॒ आप्त्या॑ अन्तरको॒श उ॒ष्णीषे॑ णो॒ष्णीषे॑णा न्तरको॒श आप्त्या॒ आप्त्या॑ अन्तरको॒श उ॒ष्णीषे॑ण । \newline
51. अ॒न्त॒र॒को॒श उ॒ष्णीषे॑ णो॒ष्णीषे॑णा न्तरको॒शे᳚ ऽन्तरको॒श उ॒ष्णीषे॒णा वि॑ष्टित॒ मावि॑ष्टित मु॒ष्णीषे॑णा न्तरको॒शे᳚ ऽन्तरको॒श उ॒ष्णीषे॒णा वि॑ष्टितम् । \newline
52. अ॒न्त॒र॒को॒श इत्य॑न्तर - को॒शे । \newline
53. उ॒ष्णीषे॒णा वि॑ष्टित॒ मावि॑ष्टित मु॒ष्णीषे॑ णो॒ष्णीषे॒णा वि॑ष्टितम् भवति भव॒ त्यावि॑ष्टित मु॒ष्णीषे॑ णो॒ष्णीषे॒णा वि॑ष्टितम् भवति । \newline
54. आवि॑ष्टितम् भवति भव॒ त्यावि॑ष्टित॒ मावि॑ष्टितम् भवत्ये॒व मे॒वम् भ॑व॒ त्यावि॑ष्टित॒ मावि॑ष्टितम् भव त्ये॒वम् । \newline
55. आवि॑ष्टित॒मित्या - वि॒ष्टि॒त॒म् । \newline
56. भ॒व॒ त्ये॒व मे॒वम् भ॑वति भव त्ये॒व मि॑वेवै॒वम् भ॑वति भवत्ये॒व मि॑व । \newline
57. ए॒व मि॑वेवै॒व मे॒व मि॑व॒ हि हीवै॒व मे॒व मि॑व॒ हि । \newline
58. इ॒व॒ हि हीवे॑ व॒ हि प॒शुः प॒शुर्. ही वे॑ व॒ हि प॒शुः । \newline
59. हि प॒शुः प॒शुर्. हि हि प॒शु रुल्ब॒ मुल्ब॑म् प॒शुर्. हि हि प॒शु रुल्ब᳚म् । \newline
60. प॒शु रुल्ब॒ मुल्ब॑म् प॒शुः प॒शु रुल्ब॑ मिवे॒ वोल्ब॑म् प॒शुः प॒शु रुल्ब॑ मिव । \newline
61. उल्ब॑ मिवे॒ वोल्ब॒ मुल्ब॑ मिव॒ चर्म॒ चर्मे॒ वोल्ब॒ मुल्ब॑ मिव॒ चर्म॑ । \newline
62. इ॒व॒ चर्म॒ चर्मे॑ वे व॒ चर्मे॑ वे व॒ चर्मे॑ वे व॒ चर्मे॑ व । \newline
63. चर्मे॑ वे व॒ चर्म॒ चर्मे॑ व माꣳ॒॒सम् माꣳ॒॒स मि॑व॒ चर्म॒ चर्मे॑ व माꣳ॒॒सम् । \newline
64. इ॒व॒ माꣳ॒॒सम् माꣳ॒॒स मि॑वेव माꣳ॒॒स मि॑वेव माꣳ॒॒स मि॑वेव माꣳ॒॒स मि॑व । \newline
65. माꣳ॒॒स मि॑वेव माꣳ॒॒सम् माꣳ॒॒स मि॒वास्थ्य स्थी॑व माꣳ॒॒सम् माꣳ॒॒स मि॒वास्थि॑ । \newline
66. इ॒वास्थ्य स्थी॑वे॒ वास्थी॑ वे॒वास्थी॑ वे॒वास्थी॑व । \newline
67. अस्थी॑ वे॒वास्थ्य स्थी॑व॒ यावा॒न्॒. यावा॑ नि॒वास्थ्य स्थी॑व॒ यावान्॑ । \newline
68. इ॒व॒ यावा॒न्॒. यावा॑ निवेव॒ यावा॑ ने॒वैव यावा॑ निवेव॒ यावा॑ ने॒व । \newline
69. यावा॑ ने॒वैव यावा॒न्॒. यावा॑ ने॒व प॒शुः प॒शुरे॒व यावा॒न्॒. यावा॑ ने॒व प॒शुः । \newline
70. ए॒व प॒शुः प॒शु रे॒वै व प॒शु स्तम् तम् प॒शु रे॒वै व प॒शु स्तम् । \newline
71. प॒शु स्तम् तम् प॒शुः प॒शु स्त मा॒प्त्वा ऽऽप्त्वा तम् प॒शुः प॒शु स्त मा॒प्त्वा । \newline
72. त मा॒प्त्वा ऽऽप्त्वा तम् त मा॒प्त्वा ऽवावा॒ प्त्वा तम् त मा॒प्त्वा ऽव॑ । \newline
73. आ॒प्त्वा ऽवावा॒प्त्वा ऽऽप्त्वा ऽव॑ रुन्धे रु॒न्धे ऽवा॒प् त्वा ऽऽप्त्वा ऽव॑ रुन्धे । \newline
74. अव॑ रुन्धे रु॒न्धे ऽवाव॑ रुन्धे॒ यस्य॒ यस्य॑ रु॒न्धे ऽवाव॑ रुन्धे॒ यस्य॑ । \newline
75. रु॒न्धे॒ यस्य॒ यस्य॑ रुन्धे रुन्धे॒ यस्यै॒षैषा यस्य॑ रुन्धे रुन्धे॒ यस्यै॒षा । \newline
76. यस्यै॒षैषा यस्य॒ यस्यै॒षा य॒ज्ञे य॒ज्ञ् ए॒षा यस्य॒ यस्यै॒षा य॒ज्ञे । \newline
77. ए॒षा य॒ज्ञे य॒ज्ञ् ए॒षैषा य॒ज्ञे प्राय॑श्चित्तिः॒ प्राय॑श्चित्तिर् य॒ज्ञ् ए॒षैषा य॒ज्ञे प्राय॑श्चित्तिः । \newline
78. य॒ज्ञे प्राय॑श्चित्तिः॒ प्राय॑श्चित्तिर् य॒ज्ञे य॒ज्ञे प्राय॑श्चित्तिः क्रि॒यते᳚ क्रि॒यते॒ प्राय॑श्चित्तिर् य॒ज्ञे य॒ज्ञे प्राय॑श्चित्तिः क्रि॒यते᳚ । \newline
79. प्राय॑श्चित्तिः क्रि॒यते᳚ क्रि॒यते॒ प्राय॑श्चित्तिः॒ प्राय॑श्चित्तिः क्रि॒यत॑ इ॒ष्ट् वेष्ट्वा क्रि॒यते॒ प्राय॑श्चित्तिः॒ प्राय॑श्चित्तिः क्रि॒यत॑ इ॒ष्ट्वा । \newline
80. क्रि॒यत॑ इ॒ष्ट् वेष्ट्वा क्रि॒यते᳚ क्रि॒यत॑ इ॒ष्ट्वा वसी॑या॒न्॒. वसी॑या नि॒ष्ट्वा क्रि॒यते᳚ क्रि॒यत॑ इ॒ष्ट्वा वसी॑यान् । \newline
81. इ॒ष्ट्वा वसी॑या॒न्॒. वसी॑या नि॒ष्ट् वेष्ट्वा वसी॑यान् भवति भवति॒ वसी॑या नि॒ष्ट् वेष्ट्वा वसी॑यान् भवति । \newline
82. वसी॑यान् भवति भवति॒ वसी॑या॒न्॒. वसी॑यान् भवति । \newline
83. भ॒व॒तीति॑ भवति । \newline
\pagebreak
\markright{ TS 3.4.2.1  \hfill https://www.vedavms.in \hfill}

\section{ TS 3.4.2.1 }

\textbf{TS 3.4.2.1 } \newline
\textbf{Samhita Paata} \newline

आ वा॑यो भूष शुचिपा॒ उप॑ नः स॒हस्रं॑ ते नि॒युतो॑ विश्ववार । उपो॑ ते॒ अन्धो॒ मद्य॑मयामि॒ यस्य॑ देव दधि॒षे पू᳚र्व॒पेयं᳚ ॥ आकू᳚त्यै त्वा॒ कामा॑य त्वा स॒मृधे᳚ त्वा किक्कि॒टा ते॒ मनः॑ प्र॒जाप॑तये॒ स्वाहा॑ किक्कि॒टा ते᳚ प्रा॒णं ॅवा॒यवे॒ स्वाहा॑ किक्कि॒टा ते॒ चक्षुः॒ सूर्या॑य॒ स्वाहा॑ किक्कि॒टा ते॒ श्रोत्रं॒ द्यावा॑पृथि॒वीभ्याꣳ॒॒ स्वाहा॑ किक्कि॒टा ते॒ वाचꣳ॒॒ सर॑स्वत्यै॒ स्वाहा॒- [  ] \newline

\textbf{Pada Paata} \newline

एति॑ । वा॒यो॒ इति॑ । भू॒ष॒ । शु॒चि॒पा॒ इति॑ शुचि - पाः॒ । उपेति॑ । नः॒ । स॒हस्र᳚म् । ते॒ । नि॒युत॒ इति॑ नि - युतः॑ । वि॒श्व॒वा॒रेति॑ विश्व - वा॒र॒ ॥ उपो॒ इति॑ । ते॒ । अन्धः॑ । मद्य᳚म् । अ॒या॒मि॒ । यस्य॑ । दे॒व॒ । द॒धि॒षे । पू॒र्व॒पेय॒मिति॑ पूर्व - पेय᳚म् ॥ आकू᳚त्या॒ इत्या-कू॒त्यै॒ । त्वा॒ । कामा॑य । त्वा॒ । स॒मृध॒ इति॑ सं - ऋधे᳚ । त्वा॒ । कि॒क्कि॒टा । ते॒ । मनः॑ । प्र॒जाप॑तय॒ इति॑ प्र॒जा - प॒त॒ये॒ । स्वाहा᳚ । कि॒क्कि॒टा । ते॒ । प्रा॒णमिति॑ प्र - अ॒नम् । वा॒यवे᳚ । स्वाहा᳚ । कि॒क्कि॒टा । ते॒ । चक्षुः॑ । सूर्या॑य । स्वाहा᳚ । कि॒क्कि॒टा । ते॒ । श्रोत्र᳚म् । द्यावा॑पृथि॒वीभ्या॒मिति॒ द्यावा᳚-पृ॒थि॒वीभ्या᳚म् । स्वाहा᳚ । कि॒क्कि॒टा । ते॒ । वाच᳚म् । सर॑स्वत्यै । स्वाहा᳚ ।  \newline


\textbf{Krama Paata} \newline

आ वा॑यो । वा॒यो॒ भू॒ष॒ । वा॒यो॒ इति॑ वायो । भू॒ष॒ शु॒चि॒पाः॒ । शु॒चि॒पा॒ उप॑ । शु॒चि॒पा॒ इति॑ शुचि - पाः॒ । उप॑ नः । नः॒ स॒हस्र᳚म् । स॒हस्र॑म् ते । ते॒ नि॒युतः॑ । नि॒युतो॑ विश्ववार । नि॒युत॒ इति॑ नि - युतः॑ । वि॒श्व॒वा॒रेति॑ विश्व - वा॒र॒ ॥ उपो॑ ते । उपो॒ इत्युपो᳚ । ते॒ अन्धः॑ । अन्धो॒ मद्य᳚म् । मद्य॑मयामि । अ॒या॒मि॒ यस्य॑ । यस्य॑ देव । दे॒व॒ द॒धि॒षे । द॒धि॒षे पू᳚र्व॒पेय᳚म् । पू॒र्व॒पेय॒मिति॑ पूर्व - पेय᳚म् ॥  आकू᳚त्यै त्वा । आकू᳚त्या॒ इत्या - कू॒त्यै॒ । त्वा॒ कामा॑य । कामा॑य त्वा । त्वा॒ स॒मृधे᳚ । स॒मृधे᳚ त्वा । स॒मृध॒ इति॑ सम् - ऋधे᳚ । त्वा॒ कि॒क्कि॒टा । कि॒क्कि॒टा ते᳚ । ते॒ मनः॑ । मनः॑ प्र॒जाप॑तये । प्र॒जाप॑तये॒ स्वाहा᳚ । प्र॒जाप॑तय॒ इति॑ प्र॒जा - प॒त॒ये॒ । स्वाहा॑ किक्कि॒टा । कि॒क्कि॒टा ते᳚ । ते॒ प्रा॒णम् । प्रा॒णं ॅवा॒यवे᳚ । प्रा॒णमिति॑ प्र - अ॒नम् । वा॒यवे॒ स्वाहा᳚ । स्वाहा॑ किक्कि॒टा । कि॒क्कि॒टा ते᳚ । ते॒ चक्षुः॑ । चक्षुः॒ सूर्या॑य । सूर्या॑य॒ स्वाहा᳚ । स्वाहा॑ किक्कि॒टा । कि॒क्कि॒टा ते᳚ । ते॒ श्रोत्र᳚म् । श्रोत्र॒म् द्यावा॑पृथि॒वीभ्या᳚म् । द्यावा॑पृथि॒वीभ्याꣳ॒॒ स्वाहा᳚ । द्यावा॑पृथि॒वीभ्या॒मिति॒ द्यावा᳚ - पृ॒थि॒वीभ्या᳚म् । स्वाहा॑ किक्कि॒टा । कि॒क्कि॒टा ते᳚ । ते॒ वाच᳚म् । वाचꣳ॒॒ सर॑स्वत्यै । सर॑स्वत्यै॒ स्वाहा᳚ । स्वाहा॒ त्वम् \newline

\textbf{Jatai Paata} \newline

1. आ वा॑यो वायो॒ आ वा॑यो । \newline
2. वा॒यो॒ भू॒ष॒ भू॒ष॒ वा॒यो॒ वा॒यो॒ भू॒ष॒ । \newline
3. वा॒यो॒ इति॑ वायो । \newline
4. भू॒ष॒ शु॒चि॒पाः॒ शु॒चि॒पा॒ भू॒ष॒ भू॒ष॒ शु॒चि॒पाः॒ । \newline
5. शु॒चि॒पा॒ उपोप॑ शुचिपाः शुचिपा॒ उप॑ । \newline
6. शु॒चि॒पा॒ इति॑ शुचि - पाः॒ । \newline
7. उप॑ नो न॒ उपोप॑ नः । \newline
8. नः॒ स॒हस्रꣳ॑ स॒हस्र॑म् नो नः स॒हस्र᳚म् । \newline
9. स॒हस्र॑म् ते ते स॒हस्रꣳ॑ स॒हस्र॑म् ते । \newline
10. ते॒ नि॒युतो॑ नि॒युत॑ स्ते ते नि॒युतः॑ । \newline
11. नि॒युतो॑ विश्ववार विश्ववार नि॒युतो॑ नि॒युतो॑ विश्ववार । \newline
12. नि॒युत॒ इति॑ नि - युतः॑ । \newline
13. वि॒श्व॒वा॒रेति॑ विश्व - वा॒र॒ । \newline
14. उपो॑ ते त॒ उपो॒ उपो॑ ते । \newline
15. उपो॒ इत्युपो᳚ । \newline
16. ते॒ अन्धो ऽन्ध॑ स्ते ते॒ अन्धः॑ । \newline
17. अन्धो॒ मद्य॒म् मद्य॒ मन्धो ऽन्धो॒ मद्य᳚म् । \newline
18. मद्य॑ मया म्ययामि॒ मद्य॒म् मद्य॑ मयामि । \newline
19. अ॒या॒मि॒ यस्य॒ यस्या॑या म्ययामि॒ यस्य॑ । \newline
20. यस्य॑ देव देव॒ यस्य॒ यस्य॑ देव । \newline
21. दे॒व॒ द॒धि॒षे द॑धि॒षे दे॑व देव दधि॒षे । \newline
22. द॒धि॒षे पू᳚र्व॒पेय॑म् पूर्व॒पेय॑म् दधि॒षे द॑धि॒षे पू᳚र्व॒पेय᳚म् । \newline
23. पू॒र्व॒पेय॒मिति॑ पूर्व - पेय᳚म् । \newline
24. आकू᳚त्यै त्वा॒ त्वा ऽऽकू᳚त्या॒ आकू᳚त्यै त्वा । \newline
25. आकू᳚त्या॒ इत्या - कू॒त्यै॒ । \newline
26. त्वा॒ कामा॑य॒ कामा॑य त्वा त्वा॒ कामा॑य । \newline
27. कामा॑य त्वा त्वा॒ कामा॑य॒ कामा॑य त्वा । \newline
28. त्वा॒ स॒मृधे॑ स॒मृधे᳚ त्वा त्वा स॒मृधे᳚ । \newline
29. स॒मृधे᳚ त्वा त्वा स॒मृधे॑ स॒मृधे᳚ त्वा । \newline
30. स॒मृध॒ इति॑ सं - ऋधे᳚ । \newline
31. त्वा॒ कि॒क्कि॒टा कि॑क्कि॒टा त्वा᳚ त्वा किक्कि॒टा । \newline
32. कि॒क्कि॒टा ते॑ ते किक्कि॒टा कि॑क्कि॒टा ते᳚ । \newline
33. ते॒ मनो॒ मन॑ स्ते ते॒ मनः॑ । \newline
34. मनः॑ प्र॒जाप॑तये प्र॒जाप॑तये॒ मनो॒ मनः॑ प्र॒जाप॑तये । \newline
35. प्र॒जाप॑तये॒ स्वाहा॒ स्वाहा᳚ प्र॒जाप॑तये प्र॒जाप॑तये॒ स्वाहा᳚ । \newline
36. प्र॒जाप॑तय॒ इति॑ प्र॒जा - प॒त॒ये॒ । \newline
37. स्वाहा॑ किक्कि॒टा कि॑क्कि॒टा स्वाहा॒ स्वाहा॑ किक्कि॒टा । \newline
38. कि॒क्कि॒टा ते॑ ते किक्कि॒टा कि॑क्कि॒टा ते᳚ । \newline
39. ते॒ प्रा॒णम् प्रा॒णम् ते॑ ते प्रा॒णम् । \newline
40. प्रा॒णं ॅवा॒यवे॑ वा॒यवे᳚ प्रा॒णम् प्रा॒णं ॅवा॒यवे᳚ । \newline
41. प्रा॒णमिति॑ प्र - अ॒नम् । \newline
42. वा॒यवे॒ स्वाहा॒ स्वाहा॑ वा॒यवे॑ वा॒यवे॒ स्वाहा᳚ । \newline
43. स्वाहा॑ किक्कि॒टा कि॑क्कि॒टा स्वाहा॒ स्वाहा॑ किक्कि॒टा । \newline
44. कि॒क्कि॒टा ते॑ ते किक्कि॒टा कि॑क्कि॒टा ते᳚ । \newline
45. ते॒ चक्षु॒ श्चक्षु॑ स्ते ते॒ चक्षुः॑ । \newline
46. चक्षुः॒ सूर्या॑य॒ सूर्या॑य॒ चक्षु॒ श्चक्षुः॒ सूर्या॑य । \newline
47. सूर्या॑य॒ स्वाहा॒ स्वाहा॒ सूर्या॑य॒ सूर्या॑य॒ स्वाहा᳚ । \newline
48. स्वाहा॑ किक्कि॒टा कि॑क्कि॒टा स्वाहा॒ स्वाहा॑ किक्कि॒टा । \newline
49. कि॒क्कि॒टा ते॑ ते किक्कि॒टा कि॑क्कि॒टा ते᳚ । \newline
50. ते॒ श्रोत्रꣳ॒॒ श्रोत्र॑म् ते ते॒ श्रोत्र᳚म् । \newline
51. श्रोत्र॒म् द्यावा॑पृथि॒वीभ्या॒म् द्यावा॑पृथि॒वीभ्याꣳ॒॒ श्रोत्रꣳ॒॒ श्रोत्र॒म् द्यावा॑पृथि॒वीभ्या᳚म् । \newline
52. द्यावा॑पृथि॒वीभ्याꣳ॒॒ स्वाहा॒ स्वाहा॒ द्यावा॑पृथि॒वीभ्या॒म् द्यावा॑पृथि॒वीभ्याꣳ॒॒ स्वाहा᳚ । \newline
53. द्यावा॑पृथि॒वीभ्या॒मिति॒ द्यावा᳚ - पृ॒थि॒वीभ्या᳚म् । \newline
54. स्वाहा॑ किक्कि॒टा कि॑क्कि॒टा स्वाहा॒ स्वाहा॑ किक्कि॒टा । \newline
55. कि॒क्कि॒टा ते॑ ते किक्कि॒टा कि॑क्कि॒टा ते᳚ । \newline
56. ते॒ वाचं॒ ॅवाच॑म् ते ते॒ वाच᳚म् । \newline
57. वाचꣳ॒॒ सर॑स्वत्यै॒ सर॑स्वत्यै॒ वाचं॒ ॅवाचꣳ॒॒ सर॑स्वत्यै । \newline
58. सर॑स्वत्यै॒ स्वाहा॒ स्वाहा॒ सर॑स्वत्यै॒ सर॑स्वत्यै॒ स्वाहा᳚ । \newline
59. स्वाहा॒ त्वम् त्वꣳ स्वाहा॒ स्वाहा॒ त्वम् । \newline

\textbf{Ghana Paata } \newline

1. आ वा॑यो वायो॒ आ वा॑यो भूष भूष वायो॒ आ वा॑यो भूष । \newline
2. वा॒यो॒ भू॒ष॒ भू॒ष॒ वा॒यो॒ वा॒यो॒ भू॒ष॒ शु॒चि॒पाः॒ शु॒चि॒पा॒ भू॒ष॒ वा॒यो॒ वा॒यो॒ भू॒ष॒ शु॒चि॒पाः॒ । \newline
3. वा॒यो॒ इति॑ वायो । \newline
4. भू॒ष॒ शु॒चि॒पाः॒ शु॒चि॒पा॒ भू॒ष॒ भू॒ष॒ शु॒चि॒पा॒ उपोप॑ शुचिपा भूष भूष शुचिपा॒ उप॑ । \newline
5. शु॒चि॒पा॒ उपोप॑ शुचिपाः शुचिपा॒ उप॑ नो न॒ उप॑ शुचिपाः शुचिपा॒ उप॑ नः । \newline
6. शु॒चि॒पा॒ इति॑ शुचि - पाः॒ । \newline
7. उप॑ नो न॒ उपोप॑ नः स॒हस्रꣳ॑ स॒हस्र॑म् न॒ उपोप॑ नः स॒हस्र᳚म् । \newline
8. नः॒ स॒हस्रꣳ॑ स॒हस्र॑म् नो नः स॒हस्र॑म् ते ते स॒हस्र॑म् नो नः स॒हस्र॑म् ते । \newline
9. स॒हस्र॑म् ते ते स॒हस्रꣳ॑ स॒हस्र॑म् ते नि॒युतो॑ नि॒युत॑ स्ते स॒हस्रꣳ॑ स॒हस्र॑म् ते नि॒युतः॑ । \newline
10. ते॒ नि॒युतो॑ नि॒युत॑ स्ते ते नि॒युतो॑ विश्ववार विश्ववार नि॒युत॑ स्ते ते नि॒युतो॑ विश्ववार । \newline
11. नि॒युतो॑ विश्ववार विश्ववार नि॒युतो॑ नि॒युतो॑ विश्ववार । \newline
12. नि॒युत॒ इति॑ नि - युतः॑ । \newline
13. वि॒श्व॒वा॒रेति॑ विश्व - वा॒र॒ । \newline
14. उपो॑ ते त॒ उपो॒ उपो॑ ते॒ अन्धो ऽन्ध॑ स्त॒ उपो॒ उपो॑ ते॒ अन्धः॑ । \newline
15. उपो॒ इत्युपो᳚ । \newline
16. ते॒ अन्धो ऽन्ध॑ स्ते ते॒ अन्धो॒ मद्य॒म् मद्य॒ मन्ध॑ स्ते ते॒ अन्धो॒ मद्य᳚म् । \newline
17. अन्धो॒ मद्य॒म् मद्य॒ मन्धो ऽन्धो॒ मद्य॑ मयाम्ययामि॒ मद्य॒ मन्धो ऽन्धो॒ मद्य॑ मयामि । \newline
18. मद्य॑ मया म्ययामि॒ मद्य॒म् मद्य॑ मयामि॒ यस्य॒ यस्या॑ यामि॒ मद्य॒म् मद्य॑ मयामि॒ यस्य॑ । \newline
19. अ॒या॒मि॒ यस्य॒ यस्या॑ याम्ययामि॒ यस्य॑ देव देव॒ यस्या॑ याम्ययामि॒ यस्य॑ देव । \newline
20. यस्य॑ देव देव॒ यस्य॒ यस्य॑ देव दधि॒षे द॑धि॒षे दे॑व॒ यस्य॒ यस्य॑ देव दधि॒षे । \newline
21. दे॒व॒ द॒धि॒षे द॑धि॒षे दे॑व देव दधि॒षे पू᳚र्व॒पेय॑म् पूर्व॒पेय॑म् दधि॒षे दे॑व देव दधि॒षे 
पू᳚र्व॒पेय᳚म् । \newline
22. द॒धि॒षे पू᳚र्व॒पेय॑म् पूर्व॒पेय॑म् दधि॒षे द॑धि॒षे पू᳚र्व॒पेय᳚म् । \newline
23. पू॒र्व॒पेय॒मिति॑ पूर्व - पेय᳚म् । \newline
24. आकू᳚त्यै त्वा॒ त्वा ऽऽकू᳚त्या॒ आकू᳚त्यै त्वा॒ कामा॑य॒ कामा॑य॒ त्वा ऽऽकू᳚त्या॒ आकू᳚त्यै त्वा॒ कामा॑य । \newline
25. आकू᳚त्या॒ इत्या - कू॒त्यै॒ । \newline
26. त्वा॒ कामा॑य॒ कामा॑य त्वा त्वा॒ कामा॑य त्वा त्वा॒ कामा॑य त्वा त्वा॒ कामा॑य त्वा । \newline
27. कामा॑य त्वा त्वा॒ कामा॑य॒ कामा॑य त्वा स॒मृधे॑ स॒मृधे᳚ त्वा॒ कामा॑य॒ कामा॑य त्वा स॒मृधे᳚ । \newline
28. त्वा॒ स॒मृधे॑ स॒मृधे᳚ त्वा त्वा स॒मृधे᳚ त्वा त्वा स॒मृधे᳚ त्वा त्वा स॒मृधे᳚ त्वा । \newline
29. स॒मृधे᳚ त्वा त्वा स॒मृधे॑ स॒मृधे᳚ त्वा किक्कि॒टा कि॑क्कि॒टा त्वा॑ स॒मृधे॑ स॒मृधे᳚ त्वा किक्कि॒टा । \newline
30. स॒मृध॒ इति॑ सम् - ऋधे᳚ । \newline
31. त्वा॒ कि॒क्कि॒टा कि॑क्कि॒टा त्वा᳚ त्वा किक्कि॒टा ते॑ ते किक्कि॒टा त्वा᳚ त्वा किक्कि॒टा ते᳚ । \newline
32. कि॒क्कि॒टा ते॑ ते किक्कि॒टा कि॑क्कि॒टा ते॒ मनो॒ मन॑ स्ते किक्कि॒टा कि॑क्कि॒टा ते॒ मनः॑ । \newline
33. ते॒ मनो॒ मन॑ स्ते ते॒ मनः॑ प्र॒जाप॑तये प्र॒जाप॑तये॒ मन॑ स्ते ते॒ मनः॑ प्र॒जाप॑तये । \newline
34. मनः॑ प्र॒जाप॑तये प्र॒जाप॑तये॒ मनो॒ मनः॑ प्र॒जाप॑तये॒ स्वाहा॒ स्वाहा᳚ प्र॒जाप॑तये॒ मनो॒ मनः॑ प्र॒जाप॑तये॒ स्वाहा᳚ । \newline
35. प्र॒जाप॑तये॒ स्वाहा॒ स्वाहा᳚ प्र॒जाप॑तये प्र॒जाप॑तये॒ स्वाहा॑ किक्कि॒टा कि॑क्कि॒टा स्वाहा᳚ प्र॒जाप॑तये प्र॒जाप॑तये॒ स्वाहा॑ किक्कि॒टा । \newline
36. प्र॒जाप॑तय॒ इति॑ प्र॒जा - प॒त॒ये॒ । \newline
37. स्वाहा॑ किक्कि॒टा कि॑क्कि॒टा स्वाहा॒ स्वाहा॑ किक्कि॒टा ते॑ ते किक्कि॒टा स्वाहा॒ स्वाहा॑ किक्कि॒टा ते᳚ । \newline
38. कि॒क्कि॒टा ते॑ ते किक्कि॒टा कि॑क्कि॒टा ते᳚ प्रा॒णम् प्रा॒णम् ते॑ किक्कि॒टा कि॑क्कि॒टा ते᳚ प्रा॒णम् । \newline
39. ते॒ प्रा॒णम् प्रा॒णम् ते॑ ते प्रा॒णम् ॅवा॒यवे॑ वा॒यवे᳚ प्रा॒णम् ते॑ ते प्रा॒णम् ॅवा॒यवे᳚ । \newline
40. प्रा॒णम् ॅवा॒यवे॑ वा॒यवे᳚ प्रा॒णम् प्रा॒णम् ॅवा॒यवे॒ स्वाहा॒ स्वाहा॑ वा॒यवे᳚ प्रा॒णम् प्रा॒णम् ॅवा॒यवे॒ स्वाहा᳚ । \newline
41. प्रा॒णमिति॑ प्र - अ॒नम् । \newline
42. वा॒यवे॒ स्वाहा॒ स्वाहा॑ वा॒यवे॑ वा॒यवे॒ स्वाहा॑ किक्कि॒टा कि॑क्कि॒टा स्वाहा॑ वा॒यवे॑ वा॒यवे॒ स्वाहा॑ किक्कि॒टा । \newline
43. स्वाहा॑ किक्कि॒टा कि॑क्कि॒टा स्वाहा॒ स्वाहा॑ किक्कि॒टा ते॑ ते किक्कि॒टा स्वाहा॒ स्वाहा॑ किक्कि॒टा ते᳚ । \newline
44. कि॒क्कि॒टा ते॑ ते किक्कि॒टा कि॑क्कि॒टा ते॒ चक्षु॒ श्चक्षु॑ स्ते किक्कि॒टा कि॑क्कि॒टा ते॒ चक्षुः॑ । \newline
45. ते॒ चक्षु॒ श्चक्षु॑ स्ते ते॒ चक्षुः॒ सूर्या॑य॒ सूर्या॑य॒ चक्षु॑ स्ते ते॒ चक्षुः॒ सूर्या॑य । \newline
46. चक्षुः॒ सूर्या॑य॒ सूर्या॑य॒ चक्षु॒ श्चक्षुः॒ सूर्या॑य॒ स्वाहा॒ स्वाहा॒ सूर्या॑य॒ चक्षु॒ श्चक्षुः॒ सूर्या॑य॒ स्वाहा᳚ । \newline
47. सूर्या॑य॒ स्वाहा॒ स्वाहा॒ सूर्या॑य॒ सूर्या॑य॒ स्वाहा॑ किक्कि॒टा कि॑क्कि॒टा स्वाहा॒ सूर्या॑य॒ सूर्या॑य॒ स्वाहा॑ किक्कि॒टा । \newline
48. स्वाहा॑ किक्कि॒टा कि॑क्कि॒टा स्वाहा॒ स्वाहा॑ किक्कि॒टा ते॑ ते किक्कि॒टा स्वाहा॒ स्वाहा॑ किक्कि॒टा ते᳚ । \newline
49. कि॒क्कि॒टा ते॑ ते किक्कि॒टा कि॑क्कि॒टा ते॒ श्रोत्रꣳ॒॒ श्रोत्र॑म् ते किक्कि॒टा कि॑क्कि॒टा ते॒ श्रोत्र᳚म् । \newline
50. ते॒ श्रोत्रꣳ॒॒ श्रोत्र॑म् ते ते॒ श्रोत्र॒म् द्यावा॑पृथि॒वीभ्या॒म् द्यावा॑पृथि॒वीभ्याꣳ॒॒ श्रोत्र॑म् ते ते॒ श्रोत्र॒म् द्यावा॑पृथि॒वीभ्या᳚म् । \newline
51. श्रोत्र॒म् द्यावा॑पृथि॒वीभ्या॒म् द्यावा॑पृथि॒वीभ्याꣳ॒॒ श्रोत्रꣳ॒॒ श्रोत्र॒म् द्यावा॑पृथि॒वीभ्याꣳ॒॒ स्वाहा॒ स्वाहा॒ द्यावा॑पृथि॒वीभ्याꣳ॒॒ श्रोत्रꣳ॒॒ श्रोत्र॒म् द्यावा॑पृथि॒वीभ्याꣳ॒॒ स्वाहा᳚ । \newline
52. द्यावा॑पृथि॒वीभ्याꣳ॒॒ स्वाहा॒ स्वाहा॒ द्यावा॑पृथि॒वीभ्या॒म् द्यावा॑पृथि॒वीभ्याꣳ॒॒ स्वाहा॑ किक्कि॒टा कि॑क्कि॒टा स्वाहा॒ द्यावा॑पृथि॒वीभ्या॒म् द्यावा॑पृथि॒वीभ्याꣳ॒॒ स्वाहा॑ किक्कि॒टा । \newline
53. द्यावा॑पृथि॒वीभ्या॒मिति॒ द्यावा᳚ - पृ॒थि॒वीभ्या᳚म् । \newline
54. स्वाहा॑ किक्कि॒टा कि॑क्कि॒टा स्वाहा॒ स्वाहा॑ किक्कि॒टा ते॑ ते किक्कि॒टा स्वाहा॒ स्वाहा॑ किक्कि॒टा ते᳚ । \newline
55. कि॒क्कि॒टा ते॑ ते किक्कि॒टा कि॑क्कि॒टा ते॒ वाच॒म् ॅवाच॑म् ते किक्कि॒टा कि॑क्कि॒टा ते॒ वाच᳚म् । \newline
56. ते॒ वाच॒म् ॅवाच॑म् ते ते॒ वाचꣳ॒॒ सर॑स्वत्यै॒ सर॑स्वत्यै॒ वाच॑म् ते ते॒ वाचꣳ॒॒ सर॑स्वत्यै । \newline
57. वाचꣳ॒॒ सर॑स्वत्यै॒ सर॑स्वत्यै॒ वाच॒म् ॅवाचꣳ॒॒ सर॑स्वत्यै॒ स्वाहा॒ स्वाहा॒ सर॑स्वत्यै॒ वाच॒म् ॅवाचꣳ॒॒ सर॑स्वत्यै॒ स्वाहा᳚ । \newline
58. सर॑स्वत्यै॒ स्वाहा॒ स्वाहा॒ सर॑स्वत्यै॒ सर॑स्वत्यै॒ स्वाहा॒ त्वम् त्वꣳ स्वाहा॒ सर॑स्वत्यै॒ सर॑स्वत्यै॒ स्वाहा॒ त्वम् । \newline
59. स्वाहा॒ त्वम् त्वꣳ स्वाहा॒ स्वाहा॒ त्वम् तु॒रीया॑ तु॒रीया॒ त्वꣳ स्वाहा॒ स्वाहा॒ त्वम् तु॒रीया᳚ । \newline
\pagebreak
\markright{ TS 3.4.2.2  \hfill https://www.vedavms.in \hfill}

\section{ TS 3.4.2.2 }

\textbf{TS 3.4.2.2 } \newline
\textbf{Samhita Paata} \newline

त्वं तु॒रीया॑ व॒शिनी॑ व॒शाऽसि॑ स॒कृद्यत् त्वा॒ मन॑सा॒ गर्भ॒ आऽश॑यत् । व॒शा त्वं ॅव॒शिनी॑ गच्छ दे॒वान्थ्-स॒त्याः स॑न्तु॒ यज॑मानस्य॒ कामाः᳚ ॥ अ॒जाऽसि॑ रयि॒ष्ठा पृ॑थि॒व्याꣳ सी॑दो॒र्द्ध्वाऽन्तरि॑क्ष॒मुप॑ तिष्ठस्व दि॒वि ते॑ बृ॒हद्भाः ॥ तन्तुं॑ त॒न्वन् रज॑सो भा॒नुमन्वि॑हि॒ ज्योति॑ष्मतः प॒थो र॑क्ष धि॒या कृ॒तान् ॥ अ॒नु॒ल्ब॒णं ॅव॑यत॒ जोगु॑वा॒मपो॒ मनु॑ ( ) र्भव ज॒नया॒ दैव्यं॒ जनं᳚ ॥ मन॑सो ह॒विर॑सि प्र॒जाप॑ते॒र्वर्णो॒ गात्रा॑णां ते गात्र॒भाजो॑ भूयास्म ॥ \newline

\textbf{Pada Paata} \newline

त्वम् । तु॒रीया᳚ । व॒शिनी᳚ । व॒शा । अ॒सि॒ । स॒कृत् । यत् । त्वा॒ । मन॑सा । गर्भः॑ । एति॑ । अश॑यत् ॥ व॒शा । त्वम् । व॒शिनी᳚ । ग॒च्छ॒ । दे॒वान् । स॒त्याः । स॒न्तु॒ । यज॑मानस्य । कामाः᳚ ॥ अ॒जा । अ॒सि॒ । र॒यि॒ष्ठेति॑ रयि - स्था । पृ॒थि॒व्याम् । सी॒द॒ । ऊ॒र्द्ध्वा । अ॒न्तरि॑क्षम् । उपेति॑ । ति॒ष्ठ॒स्व॒ । दि॒वि । ते॒ । बृ॒हत् । भाः ॥ तन्तु᳚म् । त॒न्वन्न् । रज॑सः । भा॒नुम् । अन्विति॑ । इ॒हि॒ । ज्योति॑ष्मतः । प॒थः । र॒क्ष॒ । धि॒या । कृ॒तान् ॥ अ॒नु॒ल्ब॒णम् । व॒य॒त॒ । जोगु॑वाम् । अपः॑ । मनुः॑( ) । भ॒व॒ । ज॒नय॑ । दैव्य᳚म् । जन᳚म् ॥ मन॑सः । ह॒विः । अ॒सि॒ । प्र॒जाप॑ते॒रिति॑ प्र॒जा - प॒तेः॒ । वर्णः॑ । गात्रा॑णाम् । ते॒ । गा॒त्र॒भाज॒ इति॑ गात्र - भाजः॑ । भू॒या॒स्म॒ ॥  \newline


\textbf{Krama Paata} \newline

त्वम् तु॒रीया᳚ । तु॒रीया॑ व॒शिनी᳚ । व॒शिनी॑ व॒शा । व॒शा ऽसि॑ । अ॒सि॒ स॒कृत् । स॒कृद् यत् । यत् त्वा᳚ । त्वा॒ मन॑सा । मन॑सा॒ गर्भः॑ । गर्भ॒ आ । आ ऽश॑यत् । अश॑य॒दित्यश॑यत् ॥ व॒शा त्वम् । त्वं ॅव॒शिनी᳚ । व॒शिनी॑ गच्छ । ग॒च्छ॒ दे॒वान् । दे॒वान्थ् स॒त्याः । स॒त्याः स॑न्तु । स॒न्तु॒ यज॑मानस्य । यज॑मानस्य॒ कामाः᳚ । कामा॒ इति॒ कामाः᳚ ॥ अ॒जा ऽसि॑ । अ॒सि॒ र॒यि॒ष्ठा । र॒यि॒ष्ठा पृ॑थि॒व्याम् । र॒यि॒ष्ठेति॑ रयि - स्था । पृ॒थि॒व्याꣳ सी॑द । सी॒दो॒र्द्ध्वा । ऊ॒र्द्ध्वा ऽन्तरि॑क्षम् । अ॒न्तरि॑क्ष॒मुप॑ । उप॑ तिष्ठस्व । ति॒ष्ठ॒स्व॒ दि॒वि । दि॒वि ते᳚ । ते॒ बृ॒हत् । बृ॒हद् भाः । भा इति॒ भाः ॥ तन्तु॑म् त॒न्वन्न् । त॒न्वन् रज॑सः । रज॑सो भा॒नुम् । भा॒नुमनु॑ । अन्वि॑हि । इ॒हि॒ ज्योति॑ष्मतः । ज्योति॑ष्मतः प॒थः । प॒थो र॑क्ष । र॒क्ष॒ धि॒या । धि॒या कृ॒तान् । कृ॒तानिति॑ कृ॒तान् ॥ अ॒नु॒ल्ब॒णं ॅव॑यत । व॒य॒त॒ जोगु॑वाम् । जोगु॑वा॒मपः॑ । अपो॒ मनुः॑ ( ) । मनु॑र् भव । भ॒व॒ ज॒नय॑ । ज॒नया॒ दैव्य᳚म् । दैव्य॒म् जन᳚म् । जन॒मिति॒ जन᳚म् ॥ मन॑सो ह॒विः । ह॒विर॑सि । अ॒सि॒ प्र॒जाप॑तेः । प्र॒जाप॑ते॒र् वर्णः॑ । प्र॒जाप॑ते॒रिति॑ प्र॒जा - प॒तेः॒ । वर्णो॒ गात्रा॑णाम् । गात्रा॑णाम् ते । ते॒ गा॒त्र॒भाजः॑ । गा॒त्र॒भाजो॑ भूयास्म । गा॒त्र॒भाज॒ इति॑ गात्र - भाजः॑ । भू॒या॒स्मेति॑ भूयास्म । \newline

\textbf{Jatai Paata} \newline

1. त्वम् तु॒रीया॑ तु॒रीया॒ त्वम् त्वम् तु॒रीया᳚ । \newline
2. तु॒रीया॑ व॒शिनी॑ व॒शिनी॑ तु॒रीया॑ तु॒रीया॑ व॒शिनी᳚ । \newline
3. व॒शिनी॑ व॒शा व॒शा व॒शिनी॑ व॒शिनी॑ व॒शा । \newline
4. व॒शा ऽस्य॑सि व॒शा व॒शा ऽसि॑ । \newline
5. अ॒सि॒ स॒कृथ् स॒कृ द॑स्यसि स॒कृत् । \newline
6. स॒कृद् यद् यथ् स॒कृथ् स॒कृद् यत् । \newline
7. यत् त्वा᳚ त्वा॒ यद् यत् त्वा᳚ । \newline
8. त्वा॒ मन॑सा॒ मन॑सा त्वा त्वा॒ मन॑सा । \newline
9. मन॑सा॒ गर्भो॒ गर्भो॒ मन॑सा॒ मन॑सा॒ गर्भः॑ । \newline
10. गर्भ॒ आ गर्भो॒ गर्भ॒ आ । \newline
11. आ ऽश॑य॒ दश॑य॒दा ऽश॑यत् । \newline
12. अश॑य॒दित्यश॑यत् । \newline
13. व॒शा त्वम् त्वं ॅव॒शा व॒शा त्वम् । \newline
14. त्वं ॅव॒शिनी॑ व॒शिनी॒ त्वम् त्वं ॅव॒शिनी᳚ । \newline
15. व॒शिनी॑ गच्छ गच्छ व॒शिनी॑ व॒शिनी॑ गच्छ । \newline
16. ग॒च्छ॒ दे॒वान् दे॒वान् ग॑च्छ गच्छ दे॒वान् । \newline
17. दे॒वान् थ्स॒त्याः स॒त्या दे॒वान् दे॒वान् थ्स॒त्याः । \newline
18. स॒त्याः स॑न्तु सन्तु स॒त्याः स॒त्याः स॑न्तु । \newline
19. स॒न्तु॒ यज॑मानस्य॒ यज॑मानस्य सन्तु सन्तु॒ यज॑मानस्य । \newline
20. यज॑मानस्य॒ कामाः॒ कामा॒ यज॑मानस्य॒ यज॑मानस्य॒ कामाः᳚ । \newline
21. कामा॒ इति॒ कामाः᳚ । \newline
22. अ॒जा ऽस्य॑ स्य॒जा ऽजा ऽसि॑ । \newline
23. अ॒सि॒ र॒यि॒ष्ठा र॑यि॒ष्ठा ऽस्य॑सि रयि॒ष्ठा । \newline
24. र॒यि॒ष्ठा पृ॑थि॒व्याम् पृ॑थि॒व्याꣳ र॑यि॒ष्ठा र॑यि॒ष्ठा पृ॑थि॒व्याम् । \newline
25. र॒यि॒ष्ठेति॑ रयि - स्था । \newline
26. पृ॒थि॒व्याꣳ सी॑द सीद पृथि॒व्याम् पृ॑थि॒व्याꣳ सी॑द । \newline
27. सी॒दो॒र्द्ध्वो र्द्ध्वा सी॑द सीदो॒र्द्ध्वा । \newline
28. ऊ॒र्द्ध्वा ऽन्तरि॑क्ष म॒न्तरि॑क्ष मू॒र्द्ध्वोर्द्ध्वा ऽन्तरि॑क्षम् । \newline
29. अ॒न्तरि॑क्ष॒ मुपोपा॒ न्तरि॑क्ष म॒न्तरि॑क्ष॒ मुप॑ । \newline
30. उप॑ तिष्ठस्व तिष्ठ॒स्वो पोप॑ तिष्ठस्व । \newline
31. ति॒ष्ठ॒स्व॒ दि॒वि दि॒वि ति॑ष्ठस्व तिष्ठस्व दि॒वि । \newline
32. दि॒वि ते॑ ते दि॒वि दि॒वि ते᳚ । \newline
33. ते॒ बृ॒हद् बृ॒हत् ते॑ ते बृ॒हत् । \newline
34. बृ॒हद् भा भा बृ॒हद् बृ॒हद् भाः । \newline
35. भा इति॒ भाः । \newline
36. तन्तु॑म् त॒न्वन् त॒न्वन् तन्तु॒म् तन्तु॑म् त॒न्वन्न् । \newline
37. त॒न्वन् रज॑सो॒ रज॑स स्त॒न्वन् त॒न्वन् रज॑सः । \newline
38. रज॑सो भा॒नुम् भा॒नुꣳ रज॑सो॒ रज॑सो भा॒नुम् । \newline
39. भा॒नु मन्वनु॑ भा॒नुम् भा॒नु मनु॑ । \newline
40. अन्वि॑ ही॒ह्य न्वन्वि॑हि । \newline
41. इ॒हि॒ ज्योति॑ष्मतो॒ ज्योति॑ष्मत इहीहि॒ ज्योति॑ष्मतः । \newline
42. ज्योति॑ष्मतः प॒थः प॒थो ज्योति॑ष्मतो॒ ज्योति॑ष्मतः प॒थः । \newline
43. प॒थो र॑क्ष रक्ष प॒थः प॒थो र॑क्ष । \newline
44. र॒क्ष॒ धि॒या धि॒या र॑क्ष रक्ष धि॒या । \newline
45. धि॒या कृ॒तान् कृ॒तान् धि॒या धि॒या कृ॒तान् । \newline
46. कृ॒तानिति॑ कृ॒तान् । \newline
47. अ॒नु॒ल्ब॒णं ॅव॑यत वयता नुल्ब॒ण म॑नुल्ब॒णं ॅव॑यत । \newline
48. व॒य॒त॒ जोगु॑वा॒म् जोगु॑वां ॅवयत वयत॒ जोगु॑वाम् । \newline
49. जोगु॑वा॒ मपो ऽपो॒ जोगु॑वा॒म् जोगु॑वा॒ मपः॑ । \newline
50. अपो॒ मनु॒र् मनु॒ रपो ऽपो॒ मनुः॑ । \newline
51. मनु॑र् भव भव॒ मनु॒र् मनु॑र् भव । \newline
52. भ॒व॒ ज॒नय॑ ज॒नय॑ भव भव ज॒नय॑ । \newline
53. ज॒नया॒ दैव्य॒म् दैव्य॑म् ज॒नय॑ ज॒नया॒ दैव्य᳚म् । \newline
54. दैव्य॒म् जन॒म् जन॒म् दैव्य॒म् दैव्य॒म् जन᳚म् । \newline
55. जन॒मिति॒ जन᳚म् । \newline
56. मन॑सो ह॒विर्. ह॒विर् मन॑सो॒ मन॑सो ह॒विः । \newline
57. ह॒वि र॑स्यसि ह॒विर्. ह॒वि र॑सि । \newline
58. अ॒सि॒ प्र॒जाप॑तेः प्र॒जाप॑ते रस्यसि प्र॒जाप॑तेः । \newline
59. प्र॒जाप॑ते॒र् वर्णो॒ वर्णः॑ प्र॒जाप॑तेः प्र॒जाप॑ते॒र् वर्णः॑ । \newline
60. प्र॒जाप॑ते॒रिति॑ प्र॒जा - प॒तेः॒ । \newline
61. वर्णो॒ गात्रा॑णा॒म् गात्रा॑णां॒ ॅवर्णो॒ वर्णो॒ गात्रा॑णाम् । \newline
62. गात्रा॑णाम् ते ते॒ गात्रा॑णा॒म् गात्रा॑णाम् ते । \newline
63. ते॒ गा॒त्र॒भाजो॑ गात्र॒भाज॑ स्ते ते गात्र॒भाजः॑ । \newline
64. गा॒त्र॒भाजो॑ भूयास्म भूयास्म गात्र॒भाजो॑ गात्र॒भाजो॑ भूयास्म । \newline
65. गा॒त्र॒भाज॒ इति॑ गात्र - भाजः॑ । \newline
66. भू॒या॒स्मेति॑ भूयास्म । \newline

\textbf{Ghana Paata } \newline

1. त्वम् तु॒रीया॑ तु॒रीया॒ त्वम् त्वम् तु॒रीया॑ व॒शिनी॑ व॒शिनी॑ तु॒रीया॒ त्वम् त्वम् तु॒रीया॑ व॒शिनी᳚ । \newline
2. तु॒रीया॑ व॒शिनी॑ व॒शिनी॑ तु॒रीया॑ तु॒रीया॑ व॒शिनी॑ व॒शा व॒शा व॒शिनी॑ तु॒रीया॑ तु॒रीया॑ व॒शिनी॑ व॒शा । \newline
3. व॒शिनी॑ व॒शा व॒शा व॒शिनी॑ व॒शिनी॑ व॒शा ऽस्य॑सि व॒शा व॒शिनी॑ व॒शिनी॑ व॒शा ऽसि॑ । \newline
4. व॒शा ऽस्य॑सि व॒शा व॒शा ऽसि॑ स॒कृथ् स॒कृ द॑सि व॒शा व॒शा ऽसि॑ स॒कृत् । \newline
5. अ॒सि॒ स॒कृथ् स॒कृ द॑स्यसि स॒कृद् यद् यथ् स॒कृ द॑स्यसि स॒कृद् यत् । \newline
6. स॒कृद् यद् यथ् स॒कृथ् स॒कृद् यत् त्वा᳚ त्वा॒ यथ् स॒कृथ् स॒कृद् यत् त्वा᳚ । \newline
7. यत् त्वा᳚ त्वा॒ यद् यत् त्वा॒ मन॑सा॒ मन॑सा त्वा॒ यद् यत् त्वा॒ मन॑सा । \newline
8. त्वा॒ मन॑सा॒ मन॑सा त्वा त्वा॒ मन॑सा॒ गर्भो॒ गर्भो॒ मन॑सा त्वा त्वा॒ मन॑सा॒ गर्भः॑ । \newline
9. मन॑सा॒ गर्भो॒ गर्भो॒ मन॑सा॒ मन॑सा॒ गर्भ॒ आ गर्भो॒ मन॑सा॒ मन॑सा॒ गर्भ॒ आ । \newline
10. गर्भ॒ आ गर्भो॒ गर्भ॒ आ ऽश॑य॒ दश॑य॒ दा गर्भो॒ गर्भ॒ आ ऽश॑यत् । \newline
11. आ ऽश॑य॒ दश॑य॒ दा ऽश॑यत् । \newline
12. अश॑य॒दित्यश॑यत् । \newline
13. व॒शा त्वम् त्वम् ॅव॒शा व॒शा त्वम् ॅव॒शिनी॑ व॒शिनी॒ त्वम् ॅव॒शा व॒शा त्वम् ॅव॒शिनी᳚ । \newline
14. त्वम् ॅव॒शिनी॑ व॒शिनी॒ त्वम् त्वम् ॅव॒शिनी॑ गच्छ गच्छ व॒शिनी॒ त्वम् त्वम् ॅव॒शिनी॑ गच्छ । \newline
15. व॒शिनी॑ गच्छ गच्छ व॒शिनी॑ व॒शिनी॑ गच्छ दे॒वान् दे॒वान् ग॑च्छ व॒शिनी॑ व॒शिनी॑ गच्छ दे॒वान् । \newline
16. ग॒च्छ॒ दे॒वान् दे॒वान् ग॑च्छ गच्छ दे॒वान् थ्स॒त्याः स॒त्या दे॒वान् ग॑च्छ गच्छ दे॒वान् थ्स॒त्याः । \newline
17. दे॒वान् थ्स॒त्याः स॒त्या दे॒वान् दे॒वान् थ्स॒त्याः स॑न्तु सन्तु स॒त्या दे॒वान् दे॒वान् थ्स॒त्याः स॑न्तु । \newline
18. स॒त्याः स॑न्तु सन्तु स॒त्याः स॒त्याः स॑न्तु॒ यज॑मानस्य॒ यज॑मानस्य सन्तु स॒त्याः स॒त्याः स॑न्तु॒ यज॑मानस्य । \newline
19. स॒न्तु॒ यज॑मानस्य॒ यज॑मानस्य सन्तु सन्तु॒ यज॑मानस्य॒ कामाः॒ कामा॒ यज॑मानस्य सन्तु सन्तु॒ यज॑मानस्य॒ कामाः᳚ । \newline
20. यज॑मानस्य॒ कामाः॒ कामा॒ यज॑मानस्य॒ यज॑मानस्य॒ कामाः᳚ । \newline
21. कामा॒ इति॒ कामाः᳚ । \newline
22. अ॒जा ऽस्य॑स्य॒जा ऽजा ऽसि॑ रयि॒ष्ठा र॑यि॒ष्ठा ऽस्य॒जा ऽजा ऽसि॑ रयि॒ष्ठा । \newline
23. अ॒सि॒ र॒यि॒ष्ठा र॑यि॒ष्ठा ऽस्य॑सि रयि॒ष्ठा पृ॑थि॒व्याम् पृ॑थि॒व्याꣳ र॑यि॒ष्ठा ऽस्य॑सि रयि॒ष्ठा पृ॑थि॒व्याम् । \newline
24. र॒यि॒ष्ठा पृ॑थि॒व्याम् पृ॑थि॒व्याꣳ र॑यि॒ष्ठा र॑यि॒ष्ठा पृ॑थि॒व्याꣳ सी॑द सीद पृथि॒व्याꣳ र॑यि॒ष्ठा र॑यि॒ष्ठा पृ॑थि॒व्याꣳ सी॑द । \newline
25. र॒यि॒ष्ठेति॑ रयि - स्था । \newline
26. पृ॒थि॒व्याꣳ सी॑द सीद पृथि॒व्याम् पृ॑थि॒व्याꣳ सी॑दो॒र्द्ध्वोर् द्ध्वा सी॑द पृथि॒व्याम् पृ॑थि॒व्याꣳ सी॑दो॒ र्द्ध्वा । \newline
27. सी॒दो॒ र्द्ध्वो र्द्ध्वा सी॑द सीदो॒ र्द्ध्वा ऽन्तरि॑क्ष म॒न्तरि॑क्ष मू॒र्द्ध्वा सी॑द सीदो॒ र्द्ध्वा ऽन्तरि॑क्षम् । \newline
28. ऊ॒र्द्ध्वा ऽन्तरि॑क्ष म॒न्तरि॑क्ष मू॒र्द्ध्वो र्द्ध्वा ऽन्तरि॑क्ष॒ मुपोपा॒ न्तरि॑क्ष मू॒र्द्ध्वो र्द्ध्वा ऽन्तरि॑क्ष॒ मुप॑ । \newline
29. अ॒न्तरि॑क्ष॒ मुपोपा॒ न्तरि॑क्ष म॒न्तरि॑क्ष॒ मुप॑ तिष्ठस्व तिष्ठ॒स्वो पा॒न्तरि॑क्ष म॒न्तरि॑क्ष॒ मुप॑ तिष्ठस्व । \newline
30. उप॑ तिष्ठस्व तिष्ठ॒स्वो पोप॑ तिष्ठस्व दि॒वि दि॒वि ति॑ष्ठ॒स्वो पोप॑ तिष्ठस्व दि॒वि । \newline
31. ति॒ष्ठ॒स्व॒ दि॒वि दि॒वि ति॑ष्ठस्व तिष्ठस्व दि॒वि ते॑ ते दि॒वि ति॑ष्ठस्व तिष्ठस्व दि॒वि ते᳚ । \newline
32. दि॒वि ते॑ ते दि॒वि दि॒वि ते॑ बृ॒हद् बृ॒हत् ते॑ दि॒वि दि॒वि ते॑ बृ॒हत् । \newline
33. ते॒ बृ॒हद् बृ॒हत् ते॑ ते बृ॒हद् भा भा बृ॒हत् ते॑ ते बृ॒हद् भाः । \newline
34. बृ॒हद् भा भा बृ॒हद् बृ॒हद् भाः । \newline
35. भा इति॒ भाः । \newline
36. तन्तु॑म् त॒न्वन् त॒न्वन् तन्तु॒म् तन्तु॑म् त॒न्वन् रज॑सो॒ रज॑स स्त॒न्वन् तन्तु॒म् तन्तु॑म् त॒न्वन् रज॑सः । \newline
37. त॒न्वन् रज॑सो॒ रज॑स स्त॒न्वन् त॒न्वन् रज॑सो भा॒नुम् भा॒नुꣳ रज॑स स्त॒न्वन् त॒न्वन् रज॑सो भा॒नुम् । \newline
38. रज॑सो भा॒नुम् भा॒नुꣳ रज॑सो॒ रज॑सो भा॒नु मन्वनु॑ भा॒नुꣳ रज॑सो॒ रज॑सो भा॒नु मनु॑ । \newline
39. भा॒नु मन्वनु॑ भा॒नुम् भा॒नु मन्वि॑ही॒ ह्यनु॑ भा॒नुम् भा॒नु मन्वि॑हि । \newline
40. अन्वि॑ही॒ ह्यन्वन् वि॑हि॒ ज्योति॑ष्मतो॒ ज्योति॑ष्मत इ॒ह्यन् वन्वि॑हि॒ ज्योति॑ष्मतः । \newline
41. इ॒हि॒ ज्योति॑ष्मतो॒ ज्योति॑ष्मत इहीहि॒ ज्योति॑ष्मतः प॒थः प॒थो ज्योति॑ष्मत इहीहि॒ ज्योति॑ष्मतः प॒थः । \newline
42. ज्योति॑ष्मतः प॒थः प॒थो ज्योति॑ष्मतो॒ ज्योति॑ष्मतः प॒थो र॑क्ष रक्ष प॒थो ज्योति॑ष्मतो॒ ज्योति॑ष्मतः प॒थो र॑क्ष । \newline
43. प॒थो र॑क्ष रक्ष प॒थः प॒थो र॑क्ष धि॒या धि॒या र॑क्ष प॒थः प॒थो र॑क्ष धि॒या । \newline
44. र॒क्ष॒ धि॒या धि॒या र॑क्ष रक्ष धि॒या कृ॒तान् कृ॒तान् धि॒या र॑क्ष रक्ष धि॒या कृ॒तान् । \newline
45. धि॒या कृ॒तान् कृ॒तान् धि॒या धि॒या कृ॒तान् । \newline
46. कृ॒तानिति॑ कृ॒तान् । \newline
47. अ॒नु॒ल्ब॒णम् ॅव॑यत वयता नुल्ब॒ण म॑नुल्ब॒णम् ॅव॑यत॒ जोगु॑वा॒म् जोगु॑वाम् ॅवयता नुल्ब॒ण म॑नुल्ब॒णम् ॅव॑यत॒ जोगु॑वाम् । \newline
48. व॒य॒त॒ जोगु॑वा॒म् जोगु॑वाम् ॅवयत वयत॒ जोगु॑वा॒ मपो ऽपो॒ जोगु॑वाम् ॅवयत वयत॒ जोगु॑वा॒ मपः॑ । \newline
49. जोगु॑वा॒ मपो ऽपो॒ जोगु॑वा॒म् जोगु॑वा॒ मपो॒ मनु॒र् मनु॒ रपो॒ जोगु॑वा॒म् जोगु॑वा॒ मपो॒ मनुः॑ । \newline
50. अपो॒ मनु॒र् मनु॒रपो ऽपो॒ मनु॑र् भव भव॒ मनु॒रपो ऽपो॒ मनु॑र् भव । \newline
51. मनु॑र् भव भव॒ मनु॒र् मनु॑र् भव ज॒नय॑ ज॒नय॑ भव॒ मनु॒र् मनु॑र् भव ज॒नय॑ । \newline
52. भ॒व॒ ज॒नय॑ ज॒नय॑ भव भव ज॒नया॒ दैव्य॒म् दैव्य॑म् ज॒नय॑ भव भव ज॒नया॒ दैव्य᳚म् । \newline
53. ज॒नया॒ दैव्य॒म् दैव्य॑म् ज॒नय॑ ज॒नया॒ दैव्य॒म् जन॒म् जन॒म् दैव्य॑म् ज॒नय॑ ज॒नया॒ दैव्य॒म् जन᳚म् । \newline
54. दैव्य॒म् जन॒म् जन॒म् दैव्य॒म् दैव्य॒म् जन᳚म् । \newline
55. जन॒मिति॒ जन᳚म् । \newline
56. मन॑सो ह॒विर्. ह॒विर् मन॑सो॒ मन॑सो ह॒वि र॑स्यसि ह॒विर् मन॑सो॒ मन॑सो ह॒वि र॑सि । \newline
57. ह॒वि र॑स्यसि ह॒विर्. ह॒वि र॑सि प्र॒जाप॑तेः प्र॒जाप॑ते रसि ह॒विर्. ह॒वि र॑सि प्र॒जाप॑तेः । \newline
58. अ॒सि॒ प्र॒जाप॑तेः प्र॒जाप॑ते रस्यसि प्र॒जाप॑ते॒र् वर्णो॒ वर्णः॑ प्र॒जाप॑ते रस्यसि प्र॒जाप॑ते॒र् वर्णः॑ । \newline
59. प्र॒जाप॑ते॒र् वर्णो॒ वर्णः॑ प्र॒जाप॑तेः प्र॒जाप॑ते॒र् वर्णो॒ गात्रा॑णा॒म् गात्रा॑णा॒म् ॅवर्णः॑ प्र॒जाप॑तेः प्र॒जाप॑ते॒र् वर्णो॒ गात्रा॑णाम् । \newline
60. प्र॒जाप॑ते॒रिति॑ प्र॒जा - प॒तेः॒ । \newline
61. वर्णो॒ गात्रा॑णा॒म् गात्रा॑णा॒म् ॅवर्णो॒ वर्णो॒ गात्रा॑णाम् ते ते॒ गात्रा॑णा॒म् ॅवर्णो॒ वर्णो॒ गात्रा॑णाम् ते । \newline
62. गात्रा॑णाम् ते ते॒ गात्रा॑णा॒म् गात्रा॑णाम् ते गात्र॒भाजो॑ गात्र॒भाज॑ स्ते॒ गात्रा॑णा॒म् गात्रा॑णाम् ते गात्र॒भाजः॑ । \newline
63. ते॒ गा॒त्र॒भाजो॑ गात्र॒भाज॑ स्ते ते गात्र॒भाजो॑ भूयास्म भूयास्म गात्र॒भाज॑ स्ते ते गात्र॒भाजो॑ भूयास्म । \newline
64. गा॒त्र॒भाजो॑ भूयास्म भूयास्म गात्र॒भाजो॑ गात्र॒भाजो॑ भूयास्म । \newline
65. गा॒त्र॒भाज॒ इति॑ गात्र - भाजः॑ । \newline
66. भू॒या॒स्मेति॑ भूयास्म । \newline
\pagebreak
\markright{ TS 3.4.3.1  \hfill https://www.vedavms.in \hfill}

\section{ TS 3.4.3.1 }

\textbf{TS 3.4.3.1 } \newline
\textbf{Samhita Paata} \newline

इ॒मे वै स॒हाऽऽ*स्तां॒ ते वा॒युर्व्य॑वा॒त् ते गर्भ॑मदधातां॒ तꣳ सोमः॒ प्राज॑नय-द॒ग्निर॑ग्रसत॒ स ए॒तं प्र॒जाप॑तिराग्ने॒य-म॒ष्टाक॑पालमपश्य॒त् तं निर॑वप॒त् तेनै॒वैना॑म॒ग्नेरधि॒ निर॑क्रीणा॒त् तस्मा॒दप्य॑न्यदेव॒त्या॑मा॒लभ॑मान आग्ने॒यम॒ष्टाक॑पालं पु॒रस्ता॒न्निर्व॑पेद॒ग्नेरे॒वैना॒मधि॑ नि॒ष्क्रीयाऽऽल॑भते॒ यद् - [  ] \newline

\textbf{Pada Paata} \newline

इ॒मे इति॑ । वै । स॒ह । आ॒स्ता॒म् । ते इति॑ । वा॒युः । वीति॑ । अ॒वा॒त् । ते इति॑ । गर्भ᳚म् । अ॒द॒धा॒ता॒म् । तम् । सोमः॑ । प्रेति॑ । अ॒ज॒न॒य॒त् । अ॒ग्निः । अ॒ग्र॒स॒त॒ । सः । ए॒तम् । प्र॒जाप॑ति॒रिति॑ प्र॒जा - प॒तिः॒ । आ॒ग्ने॒यम् । अ॒ष्टाक॑पाल॒मित्य॒ष्टा - क॒पा॒ल॒म् । अ॒प॒श्य॒त् । तम् । निरिति॑ । अ॒व॒प॒त् । तेन॑ । ए॒व । ए॒ना॒म् । अ॒ग्नेः । अधि॑ । निरिति॑ । अ॒क्री॒णा॒त् । तस्मा᳚त् । अपीति॑ । अ॒न्य॒दे॒व॒त्या॑मित्य॑न्य - दे॒व॒त्या᳚म् । आ॒लभ॑मान॒ इत्या᳚ - लभ॑मानः । आ॒ग्ने॒यम् । अ॒ष्टाक॑पाल॒मित्य॒ष्टा-क॒पा॒ल॒म् । पु॒रस्ता᳚त् । निरिति॑ । व॒पे॒त् । अ॒ग्नेः । ए॒व । ए॒ना॒म् । अधीति॑ । नि॒ष्क्रीयेति॑ निः - क्रीय॑ । एति॑ । ल॒भ॒ते॒ । यत् ।  \newline


\textbf{Krama Paata} \newline

इ॒मे वै । इ॒मे इती॒मे । वै स॒ह । स॒हास्ता᳚म् । आ॒स्ता॒म् ते । ते वा॒युः । ते इति॒ ते । वा॒युर् वि । व्य॑वात् । अ॒वा॒त् ते । ते गर्भ᳚म् । ते इति॒ ते । गर्भ॑मदधाताम् । अ॒द॒धा॒ता॒म् तम् । तꣳ सोमः॑ । सोमः॒ प्र । प्राज॑नयत् । अ॒ज॒न॒य॒द॒ग्निः । अ॒ग्निर॑ग्रसत । अ॒ग्र॒स॒त॒ सः । स ए॒तम् । ए॒तम् प्र॒जाप॑तिः । प्र॒जाप॑तिराग्ने॒यम् । प्र॒जाप॑ति॒रिति॑ प्र॒जा - प॒तिः॒ । आ॒ग्ने॒यम॒ष्टाक॑पालम् । अ॒ष्टाक॑पालमपश्यत् । अ॒ष्टाक॑पाल॒मित्य॒ष्टा - क॒पा॒ल॒म् । अ॒प॒श्य॒त् तम् । तम् निः । निर॑वपत् । अ॒व॒प॒त् तेन॑ । तेनै॒व । ए॒वैना᳚म् । ए॒॒ना॒म॒ग्नेः । अ॒ग्नेरधि॑ । अधि॒ निः । निर॑क्रीणात् । अ॒क्री॒णा॒त् तस्मा᳚त् । तस्मा॒दपि॑ । अप्य॑न्यदेव॒त्या᳚म् । अ॒न्य॒दे॒व॒त्या॑मा॒लभ॑मानः । अ॒न्य॒दे॒व॒त्या॑मित्य॑न्य - दे॒व॒त्या᳚म् । आ॒लभ॑मान आग्ने॒यम् । आ॒लभ॑मान॒ इत्या᳚ - लभ॑मानः । आ॒ग्ने॒यम॒ष्टाक॑पालम् । अ॒ष्टाक॑पालम् पु॒रस्ता᳚त् । अ॒ष्टाक॑पाल॒मित्य॒ष्टा - क॒पा॒ल॒म् । पु॒रस्ता॒न् निः । निर् व॑पेत् । व॒पे॒द॒ग्नेः । अ॒ग्नेरे॒व । ए॒वैना᳚म् । ए॒ना॒मधि॑ । अधि॑ नि॒ष्क्रीय॑ । नि॒ष्क्रीया । नि॒ष्क्रीयेति॑ निः - क्रीय॑ । आ ल॑भते । ल॒भ॒ते॒ यत् । यद् वा॒युः \newline

\textbf{Jatai Paata} \newline

1. इ॒मे वै वा इ॒मे इ॒मे वै । \newline
2. इ॒मे इती॒मे । \newline
3. वै स॒ह स॒ह वै वै स॒ह । \newline
4. स॒हास्ता॑ मास्ताꣳ स॒ह स॒हास्ता᳚म् । \newline
5. आ॒स्ता॒म् ते ते आ᳚स्ता मास्ता॒म् ते । \newline
6. ते वा॒युर् वा॒यु स्ते ते वा॒युः । \newline
7. ते इति॒ ते । \newline
8. वा॒युर् वि वि वा॒युर् वा॒युर् वि । \newline
9. व्य॑वा दवा॒द् वि व्य॑वात् । \newline
10. अ॒वा॒त् ते ते अ॑वा दवा॒त् ते । \newline
11. ते गर्भ॒म् गर्भ॒म् ते ते गर्भ᳚म् । \newline
12. ते इति॒ ते । \newline
13. गर्भ॑ मदधाता मदधाता॒म् गर्भ॒म् गर्भ॑ मदधाताम् । \newline
14. अ॒द॒धा॒ता॒म् तम् त म॑दधाता मदधाता॒म् तम् । \newline
15. तꣳ सोमः॒ सोम॒ स्तम् तꣳ सोमः॑ । \newline
16. सोमः॒ प्र प्र सोमः॒ सोमः॒ प्र । \newline
17. प्रा ज॑नयद जनय॒त् प्र प्रा ज॑नयत् । \newline
18. अ॒ज॒न॒य॒ द॒ग्निर् अ॒ग्नि र॑जनय दजनय द॒ग्निः । \newline
19. आ॒ग्नि र॑ग्रसता ग्रसता॒ ग्नि र॒ग्नि र॑ग्रसत । \newline
20. अ॒ग्र॒स॒त॒ स सो᳚ ऽग्रसता ग्रसत॒ सः । \newline
21. स ए॒त मे॒तꣳ स स ए॒तम् । \newline
22. ए॒तम् प्र॒जाप॑तिः प्र॒जाप॑ति रे॒त मे॒तम् प्र॒जाप॑तिः । \newline
23. प्र॒जाप॑ति राग्ने॒य मा᳚ग्ने॒यम् प्र॒जाप॑तिः प्र॒जाप॑ति राग्ने॒यम् । \newline
24. प्र॒जाप॑ति॒रिति॑ प्र॒जा - प॒तिः॒ । \newline
25. आ॒ग्ने॒य म॒ष्टाक॑पाल म॒ष्टाक॑पाल माग्ने॒य मा᳚ग्ने॒य म॒ष्टाक॑पालम् । \newline
26. अ॒ष्टाक॑पाल मपश्य दपश्य द॒ष्टाक॑पाल म॒ष्टाक॑पाल मपश्यत् । \newline
27. अ॒ष्टाक॑पाल॒मित्य॒ष्टा - क॒पा॒ल॒म् । \newline
28. अ॒प॒श्य॒त् तम् त म॑पश्य दपश्य॒त् तम् । \newline
29. तम् निर् णिष् टम् तम् निः । \newline
30. निर॑वप दवप॒न् निर् णिर॑वपत् । \newline
31. अ॒व॒प॒त् तेन॒ तेना॑ वप दवप॒त् तेन॑ । \newline
32. तेनै॒वैव तेन॒ तेनै॒व । \newline
33. ए॒वैना॑ मेना मे॒वैवैना᳚म् । \newline
34. ए॒ना॒ म॒ग्ने र॒ग्ने रे॑ना मेना म॒ग्नेः । \newline
35. अ॒ग्ने रध्यध्य॒ ग्ने र॒ग्ने रधि॑ । \newline
36. अधि॒ निर् णि रध्यधि॒ निः । \newline
37. नि र॑क्रीणा दक्रीणा॒न् निर् णिर॑क्रीणात् । \newline
38. अ॒क्री॒णा॒त् तस्मा॒त् तस्मा॑ दक्रीणा दक्रीणा॒त् तस्मा᳚त् । \newline
39. तस्मा॒ दप्यपि॒ तस्मा॒त् तस्मा॒ दपि॑ । \newline
40. अप्य॑ न्यदेव॒त्या॑ मन्यदेव॒त्या॑ मप्यप्य॑ न्यदेव॒त्या᳚म् । \newline
41. अ॒न्य॒दे॒व॒त्या॑ मा॒लभ॑मान आ॒लभ॑मानो ऽन्यदेव॒त्या॑ मन्यदेव॒त्या॑ मा॒लभ॑मानः । \newline
42. अ॒न्य॒दे॒व॒त्या॑मित्य॑न्य - दे॒व॒त्या᳚म् । \newline
43. आ॒लभ॑मान आग्ने॒य मा᳚ग्ने॒य मा॒लभ॑मान आ॒लभ॑मान आग्ने॒यम् । \newline
44. आ॒लभ॑मान॒ इत्या᳚ - लभ॑मानः । \newline
45. आ॒ग्ने॒य म॒ष्टाक॑पाल म॒ष्टाक॑पाल माग्ने॒य मा᳚ग्ने॒य म॒ष्टाक॑पालम् । \newline
46. अ॒ष्टाक॑पालम् पु॒रस्ता᳚त् पु॒रस्ता॑ द॒ष्टाक॑पाल म॒ष्टाक॑पालम् पु॒रस्ता᳚त् । \newline
47. अ॒ष्टाक॑पाल॒मित्य॒ष्टा - क॒पा॒ल॒म् । \newline
48. पु॒रस्ता॒न् निर् णिष् पु॒रस्ता᳚त् पु॒रस्ता॒न् निः । \newline
49. निर् व॑पेद् वपे॒न् निर् णिर् व॑पेत् । \newline
50. व॒पे॒ द॒ग्ने र॒ग्नेर् व॑पेद् वपे द॒ग्नेः । \newline
51. अ॒ग्नेरे॒वै वाग्ने र॒ग्नेरे॒व । \newline
52. ए॒वैना॑ मेना मे॒वैवैना᳚म् । \newline
53. ए॒ना॒ मध्यध्ये॑ ना मेना॒ मधि॑ । \newline
54. अधि॑ नि॒ष्क्रीय॑ नि॒ष्क्रीया ध्यधि॑ नि॒ष्क्रीय॑ । \newline
55. नि॒ष्क्रीया नि॒ष्क्रीय॑ नि॒ष्क्रीया । \newline
56. नि॒ष्क्रीयेति॑ निः - क्रीय॑ । \newline
57. आ ल॑भते लभत॒ आ ल॑भते । \newline
58. ल॒भ॒ते॒ यद् यल्ल॑भते लभते॒ यत् । \newline
59. यद् वा॒युर् वा॒युर् यद् यद् वा॒युः । \newline

\textbf{Ghana Paata } \newline

1. इ॒मे वै वा इ॒मे इ॒मे वै स॒ह स॒ह वा इ॒मे इ॒मे वै स॒ह । \newline
2. इ॒मे इती॒मे । \newline
3. वै स॒ह स॒ह वै वै स॒हास्ता॑ मास्ताꣳ स॒ह वै वै स॒हास्ता᳚म् । \newline
4. स॒हास्ता॑ मास्ताꣳ स॒ह स॒हास्ता॒म् ते ते आ᳚स्ताꣳ स॒ह स॒हास्ता॒म् ते । \newline
5. आ॒स्ता॒म् ते ते आ᳚स्ता मास्ता॒म् ते वा॒युर् वा॒यु स्ते आ᳚स्ता मास्ता॒म् ते वा॒युः । \newline
6. ते वा॒युर् वा॒यु स्ते ते वा॒युर् वि वि वा॒यु स्ते ते वा॒युर् वि । \newline
7. ते इति॒ ते । \newline
8. वा॒युर् वि वि वा॒युर् वा॒युर् व्य॑वा दवा॒द् वि वा॒युर् वा॒युर् व्य॑वात् । \newline
9. व्य॑वा दवा॒द् वि व्य॑वा॒त् ते ते अ॑वा॒द् वि व्य॑वा॒त् ते । \newline
10. अ॒वा॒त् ते ते अ॑वा दवा॒त् ते गर्भ॒म् गर्भ॒म् ते अ॑वा दवा॒त् ते गर्भ᳚म् । \newline
11. ते गर्भ॒म् गर्भ॒म् ते ते गर्भ॑ मदधाता मदधाता॒म् गर्भ॒म् ते ते गर्भ॑ मदधाताम् । \newline
12. ते इति॒ ते । \newline
13. गर्भ॑ मदधाता मदधाता॒म् गर्भ॒म् गर्भ॑ मदधाता॒म् तम् तम॑दधाता॒म् गर्भ॒म् गर्भ॑ मदधाता॒म् तम् । \newline
14. अ॒द॒धा॒ता॒म् तम् त म॑दधाता मदधाता॒म् तꣳ सोमः॒ सोम॒ स्त म॑दधाता मदधाता॒म् तꣳ सोमः॑ । \newline
15. तꣳ सोमः॒ सोम॒ स्तम् तꣳ सोमः॒ प्र प्र सोम॒ स्तम् तꣳ सोमः॒ प्र । \newline
16. सोमः॒ प्र प्र सोमः॒ सोमः॒ प्रा ज॑नय दजनय॒त् प्र सोमः॒ सोमः॒ प्राज॑नयत् । \newline
17. प्रा ज॑नय दजनय॒त् प्र प्रा ज॑नय द॒ग्नि र॒ग्नि र॑जनय॒त् प्र प्राज॑नय द॒ग्निः । \newline
18. अ॒ज॒न॒य॒ द॒ग्नि र॒ग्नि र॑जनय दजनय द॒ग्नि र॑ग्रसता ग्रसता॒ ग्नि र॑जनय दजनय द॒ग्नि र॑ग्रसत । \newline
19. अ॒ग्नि र॑ग्रसता ग्रसता॒ ग्निर॒ग्नि र॑ग्रसत॒ स सो᳚ ऽग्रसता॒ ग्नि र॒ग्नि र॑ग्रसत॒ सः । \newline
20. अ॒ग्र॒स॒त॒ स सो᳚ ऽग्रसता ग्रसत॒ स ए॒त मे॒तꣳ सो᳚ ऽग्रसता ग्रसत॒ स ए॒तम् । \newline
21. स ए॒त मे॒तꣳ स स ए॒तम् प्र॒जाप॑तिः प्र॒जाप॑ति रे॒तꣳ स स ए॒तम् प्र॒जाप॑तिः । \newline
22. ए॒तम् प्र॒जाप॑तिः प्र॒जाप॑ति रे॒त मे॒तम् प्र॒जाप॑ति राग्ने॒य मा᳚ग्ने॒यम् प्र॒जाप॑तिरे॒त मे॒तम् प्र॒जाप॑ति राग्ने॒यम् । \newline
23. प्र॒जाप॑ति राग्ने॒य मा᳚ग्ने॒यम् प्र॒जाप॑तिः प्र॒जाप॑ति राग्ने॒य म॒ष्टाक॑पाल म॒ष्टाक॑पाल माग्ने॒यम् प्र॒जाप॑तिः प्र॒जाप॑ति राग्ने॒य म॒ष्टाक॑पालम् । \newline
24. प्र॒जाप॑ति॒रिति॑ प्र॒जा - प॒तिः॒ । \newline
25. आ॒ग्ने॒य म॒ष्टाक॑पाल म॒ष्टाक॑पाल माग्ने॒य मा᳚ग्ने॒य म॒ष्टाक॑पाल मपश्य दपश्य द॒ष्टाक॑पाल माग्ने॒य मा᳚ग्ने॒य म॒ष्टाक॑पाल मपश्यत् । \newline
26. अ॒ष्टाक॑पाल मपश्य दपश्य द॒ष्टाक॑पाल म॒ष्टाक॑पाल मपश्य॒त् तम् तम॑पश्य द॒ष्टाक॑पाल म॒ष्टाक॑पाल मपश्य॒त् तम् । \newline
27. अ॒ष्टाक॑पाल॒मित्य॒ष्टा - क॒पा॒ल॒म् । \newline
28. अ॒प॒श्य॒त् तम् तम॑पश्य दपश्य॒त् तम् निर् णिष्ट म॑पश्य दपश्य॒त् तम् निः । \newline
29. तम् निर् णिष्टम् तम् निर॑वप दवप॒न् निष्टम् तम् निर॑वपत् । \newline
30. निर॑वप दवप॒न् निर् णिर॑वप॒त् तेन॒ तेना॑ वप॒न् निर् णिर॑वप॒त् तेन॑ । \newline
31. अ॒व॒प॒त् तेन॒ तेना॑ वप दवप॒त् तेनै॒वैव तेना॑ वप दवप॒त् तेनै॒व । \newline
32. तेनै॒वैव तेन॒ तेनै॒वैना॑ मेना मे॒व तेन॒ तेनै॒वैना᳚म् । \newline
33. ए॒वैना॑ मेना मे॒वैवैना॑ म॒ग्ने र॒ग्ने रे॑ना मे॒वैवैना॑ म॒ग्नेः । \newline
34. ए॒ना॒ म॒ग्ने र॒ग्ने रे॑ना मेना म॒ग्ने रध्यध्य॒ग्ने रे॑ना मेना म॒ग्ने रधि॑ । \newline
35. अ॒ग्ने रध्यध्य॒ग्ने र॒ग्ने रधि॒ निर् णिरध्य॒ग्ने र॒ग्ने रधि॒ निः । \newline
36. अधि॒ निर् णिरध्यधि॒ निर॑क्रीणा दक्रीणा॒न् निरध्यधि॒ निर॑क्रीणात् । \newline
37. निर॑क्रीणा दक्रीणा॒न् निर् णि र॑क्रीणा॒त् तस्मा॒त् तस्मा॑ दक्रीणा॒न् निर् णिर॑क्रीणा॒त् तस्मा᳚त् । \newline
38. अ॒क्री॒णा॒त् तस्मा॒त् तस्मा॑ दक्रीणा दक्रीणा॒त् तस्मा॒ दप्यपि॒ तस्मा॑ दक्रीणा दक्रीणा॒त् तस्मा॒ दपि॑ । \newline
39. तस्मा॒ दप्यपि॒ तस्मा॒त् तस्मा॒ दप्य॑ न्यदेव॒त्या॑ मन्यदेव॒त्या॑ मपि॒ तस्मा॒त् तस्मा॒ दप्य॑ न्यदेव॒त्या᳚म् । \newline
40. अप्य॑ न्यदेव॒त्या॑ मन्य देव॒त्या॑ मप्यप्य॑ न्यदेव॒त्या॑ मा॒लभ॑मान आ॒लभ॑मानो ऽन्यदेव॒त्या॑ मप्यप्य॑ न्यदेव॒त्या॑ मा॒लभ॑मानः । \newline
41. अ॒न्य॒दे॒व॒त्या॑ मा॒लभ॑मान आ॒लभ॑मानो ऽन्यदेव॒त्या॑ मन्यदेव॒त्या॑ मा॒लभ॑मान आग्ने॒य मा᳚ग्ने॒य मा॒लभ॑मानो ऽन्यदेव॒त्या॑ मन्यदेव॒त्या॑ मा॒लभ॑मान आग्ने॒यम् । \newline
42. अ॒न्य॒दे॒व॒त्या॑मित्य॑न्य - दे॒व॒त्या᳚म् । \newline
43. आ॒लभ॑मान आग्ने॒य मा᳚ग्ने॒य मा॒लभ॑मान आ॒लभ॑मान आग्ने॒य म॒ष्टाक॑पाल म॒ष्टाक॑पाल माग्ने॒य मा॒लभ॑मान आ॒लभ॑मान आग्ने॒य म॒ष्टाक॑पालम् । \newline
44. आ॒लभ॑मान॒ इत्या᳚ - लभ॑मानः । \newline
45. आ॒ग्ने॒य म॒ष्टाक॑पाल म॒ष्टाक॑पाल माग्ने॒य मा᳚ग्ने॒य म॒ष्टाक॑पालम् पु॒रस्ता᳚त् पु॒रस्ता॑ द॒ष्टाक॑पाल माग्ने॒य मा᳚ग्ने॒य म॒ष्टाक॑पालम् पु॒रस्ता᳚त् । \newline
46. अ॒ष्टाक॑पालम् पु॒रस्ता᳚त् पु॒रस्ता॑ द॒ष्टाक॑पाल म॒ष्टाक॑पालम् पु॒रस्ता॒न् निर् णिष् पु॒रस्ता॑ द॒ष्टाक॑पाल म॒ष्टाक॑पालम् पु॒रस्ता॒न् निः । \newline
47. अ॒ष्टाक॑पाल॒मित्य॒ष्टा - क॒पा॒ल॒म् । \newline
48. पु॒रस्ता॒न् निर् णिष् पु॒रस्ता᳚त् पु॒रस्ता॒न् निर् व॑पेद् वपे॒न् निष् पु॒रस्ता᳚त् पु॒रस्ता॒न् निर् व॑पेत् । \newline
49. निर् व॑पेद् वपे॒न् निर् णिर् व॑पे द॒ग्ने र॒ग्नेर् व॑पे॒न् निर् णिर् व॑पे द॒ग्नेः । \newline
50. व॒पे॒ द॒ग्ने र॒ग्नेर् व॑पेद् वपेद॒ग्ने रे॒वै वाग्नेर् व॑पेद् वपे द॒ग्ने रे॒व । \newline
51. अ॒ग्ने रे॒वै वाग्ने र॒ग्ने रे॒वैना॑ मेना मे॒वाग्ने र॒ग्ने रे॒वैना᳚म् । \newline
52. ए॒वैना॑ मेना मे॒वैवैना॒ मध्यध्ये॑ना मे॒वैवैना॒ मधि॑ । \newline
53. ए॒ना॒ मध्यध्ये॑ना मेना॒ मधि॑ नि॒ष्क्रीय॑ नि॒ष्क्रीया ध्ये॑ना मेना॒ मधि॑ नि॒ष्क्रीय॑ । \newline
54. अधि॑ नि॒ष्क्रीय॑ नि॒ष्क्रीया ध्यधि॑ नि॒ष्क्रीया नि॒ष्क्रीया ध्यधि॑ नि॒ष्क्रीया । \newline
55. नि॒ष्क्रीया नि॒ष्क्रीय॑ नि॒ष्क्रीया ल॑भते लभत॒ आ नि॒ष्क्रीय॑ नि॒ष्क्रीया ल॑भते । \newline
56. नि॒ष्क्रीयेति॑ निः - क्रीय॑ । \newline
57. आ ल॑भते लभत॒ आ ल॑भते॒ यद् यल्ल॑भत॒ आ ल॑भते॒ यत् । \newline
58. ल॒भ॒ते॒ यद् यल्ल॑भते लभते॒ यद् वा॒युर् वा॒युर् यल्ल॑भते लभते॒ यद् वा॒युः । \newline
59. यद् वा॒युर् वा॒युर् यद् यद् वा॒युर् व्यवा॒द् व्यवा᳚द् वा॒युर् यद् यद् वा॒युर् व्यवा᳚त् । \newline
\pagebreak
\markright{ TS 3.4.3.2  \hfill https://www.vedavms.in \hfill}

\section{ TS 3.4.3.2 }

\textbf{TS 3.4.3.2 } \newline
\textbf{Samhita Paata} \newline

वा॒युर्व्यवा॒त् तस्मा᳚द्-वाय॒व्या॑ यदि॒मे गर्भ॒मद॑धातां॒ तस्मा᳚द्-द्यावापृथि॒व्या॑ यथ् सोमः॒ प्राज॑नयद॒ग्निरग्र॑सत॒ तस्मा॑दग्नीषो॒मीया॒ यद॒नयो᳚र्विय॒त्योर्-वागव॑द॒त् तस्मा᳚थ् सारस्व॒ती यत् प्र॒जाप॑तिर॒ग्नेरधि॑ नि॒रक्री॑णा॒त् तस्मा᳚त् प्राजाप॒त्या सा वा ए॒षा स॑र्वदेव॒त्या॑ यद॒जा व॒शा वा॑य॒व्या॑मा ल॑भेत॒ भूति॑कामो वा॒युर्वै क्षेपि॑ष्ठा दे॒वता॑ वा॒युमे॒व स्वेन॑- [  ] \newline

\textbf{Pada Paata} \newline

वा॒युः । व्यवा॒दिति॑ वि - अवा᳚त् । तस्मा᳚त् । वा॒य॒व्या᳚ । यत् । इ॒मे इति॑ । गर्भ᳚म् । अद॑धाताम् । तस्मा᳚त् । द्या॒वा॒पृ॒थि॒व्येति॑ द्यावा - पृ॒थि॒व्या᳚ । यत् । सोमः॑ । प्रेति॑ । अज॑नयत् । अ॒ग्निः । अग्र॑सत । तस्मा᳚त् । अ॒ग्नी॒षो॒मीयेत्य॑ग्नी-सो॒मीया᳚ । यत् । अ॒नयोः᳚ । वि॒य॒त्योरिति॑ वि - य॒त्योः । वाक् । अव॑दत् । तस्मा᳚त् । सा॒र॒स्व॒ती । यत् । प्र॒जाप॑ति॒रिति॑ प्र॒जा - प॒तिः॒ । अ॒ग्नेः । अधीति॑ । नि॒रक्री॑णा॒दिति॑ निः - अक्री॑णात् । तस्मा᳚त् । प्रा॒जा॒प॒त्येति॑ प्राजा - प॒त्या । सा । वै । ए॒षा । स॒र्व॒दे॒व॒त्येति॑ सर्व - दे॒व॒त्या᳚ । यत् । अ॒जा । व॒शा । वा॒य॒व्या᳚म् । एति॑ । ल॒भे॒त॒ । भूति॑काम॒ इति॒ भूति॑ - का॒मः॒ । वा॒युः । वै । क्षेपि॑ष्ठा । दे॒वता᳚ । वा॒युम् । ए॒व । स्वेन॑ ।  \newline


\textbf{Krama Paata} \newline

वा॒युर् व्यवा᳚त् । व्यवा॒त् तस्मा᳚त् । व्यवा॒दिति॑ वि - अवा᳚त् । तस्मा᳚द् वाय॒व्या᳚ । वा॒य॒व्या॑ यत् । यदि॒मे । इ॒मे गर्भ᳚म् । इ॒मे इती॒मे । गर्भ॒मद॑धाताम् । अद॑धाता॒म् तस्मा᳚त् । तस्मा᳚द् द्यावापृथि॒व्या᳚ । द्या॒वा॒पृ॒थि॒व्या॑ यत् । द्या॒व्या॒पृ॒थि॒व्येति॑ द्यावा - पृ॒थि॒व्या᳚ । यथ् सोमः॑ । सोमः॒ प्र । प्राज॑नयत् । अज॑नयद॒ग्निः । अ॒ग्निरग्र॑सत । अग्र॑सत॒ तस्मा᳚त् । तस्मा॑दग्नीषो॒मीया᳚ । अ॒ग्नी॒षो॒मीया॒ यत् । अ॒ग्नी॒षो॒मीयेत्य॑ग्नी - सो॒मीया᳚ । यद॒नयोः᳚ । अ॒नयो᳚र् विय॒त्योः । वि॒य॒त्योर् वाक् । वि॒य॒त्योरिति॑ वि - य॒त्योः । वागव॑दत् । अव॑द॒त् तस्मा᳚त् । तस्मा᳚थ् सारस्व॒ती । सा॒र॒स्व॒ती यत् । यत् प्र॒जाप॑तिः । प्र॒जाप॑तिर॒ग्नेः । प्र॒जाप॑ति॒रिति॑ प्र॒जा - प॒तिः॒ । अ॒ग्नेरधि॑ । अधि॑ नि॒रक्री॑णात् । नि॒रक्री॑णा॒त् तस्मा᳚त् । नि॒रक्री॑णा॒दिति॑ निः - अक्री॑णात् । तस्मा᳚त् प्राजाप॒त्या । प्रा॒जा॒प॒त्या सा । प्रा॒जा॒प॒त्येति॑ प्राजा - प॒त्या । सा वै । वा ए॒षा । ए॒षा स॑र्वदेव॒त्या᳚ । स॒र्व॒दे॒व॒त्या॑ यत् । स॒र्व॒दे॒व॒त्येति॑ सर्व - दे॒व॒त्या᳚ । यद॒जा । अ॒जा व॒शा । व॒शा वा॑य॒व्या᳚म् । वा॒य॒व्या॑मा । आ ल॑भेत । ल॒भे॒त॒ भूति॑कामः । भूति॑कामो वा॒युः । भूति॑काम॒ इति॒ भूति॑ - का॒मः॒ । वा॒युर् वै । वै क्षेपि॑ष्ठा । क्षेपि॑ष्ठा दे॒वता᳚ । दे॒वता॑ वा॒युम् । वा॒युमे॒व । ए॒व स्वेन॑ । स्वेन॑ भाग॒धेये॑न \newline

\textbf{Jatai Paata} \newline

1. वा॒युर् व्यवा॒द् व्यवा᳚द् वा॒युर् वा॒युर् व्यवा᳚त् । \newline
2. व्यवा॒त् तस्मा॒त् तस्मा॒द् व्यवा॒द् व्यवा॒त् तस्मा᳚त् । \newline
3. व्यवा॒दिति॑ वि - अवा᳚त् । \newline
4. तस्मा᳚द् वाय॒व्या॑ वाय॒व्या॑ तस्मा॒त् तस्मा᳚द् वाय॒व्या᳚ । \newline
5. वा॒य॒व्या॑ यद् यद् वा॑य॒व्या॑ वाय॒व्या॑ यत् । \newline
6. यदि॒मे इ॒मे यद् यदि॒मे । \newline
7. इ॒मे गर्भ॒म् गर्भ॑ मि॒मे इ॒मे गर्भ᳚म् । \newline
8. इ॒मे इती॒मे । \newline
9. गर्भ॒ मद॑धाता॒ मद॑धाता॒म् गर्भ॒म् गर्भ॒ मद॑धाताम् । \newline
10. अद॑धाता॒म् तस्मा॒त् तस्मा॒ दद॑धाता॒ मद॑धाता॒म् तस्मा᳚त् । \newline
11. तस्मा᳚द् द्यावापृथि॒व्या᳚ द्यावापृथि॒व्या॑ तस्मा॒त् तस्मा᳚द् द्यावापृथि॒व्या᳚ । \newline
12. द्या॒वा॒पृ॒थि॒व्या॑ यद् यद् द्या॑वापृथि॒व्या᳚ द्यावापृथि॒व्या॑ यत् । \newline
13. द्या॒वा॒पृ॒थि॒व्येति॑ द्यावा - पृ॒थि॒व्या᳚ । \newline
14. यथ् सोमः॒ सोमो॒ यद् यथ् सोमः॑ । \newline
15. सोमः॒ प्र प्र सोमः॒ सोमः॒ प्र । \newline
16. प्राज॑नय॒ दज॑नय॒त् प्र प्राज॑नयत् । \newline
17. अज॑नय द॒ग्नि र॒ग्नि रज॑नय॒ दज॑नय द॒ग्निः । \newline
18. अ॒ग्नि रग्र॑स॒ता ग्र॑सता॒ग्नि र॒ग्नि रग्र॑सत । \newline
19. अग्र॑सत॒ तस्मा॒त् तस्मा॒ दग्र॑स॒ता ग्र॑सत॒ तस्मा᳚त् । \newline
20. तस्मा॑ दग्नीषो॒मीया᳚ ऽग्नीषो॒मीया॒ तस्मा॒त् तस्मा॑ दग्नीषो॒मीया᳚ । \newline
21. अ॒ग्नी॒षो॒मीया॒ यद् यद॑ग्नीषो॒मीया᳚ ऽग्नीषो॒मीया॒ यत् । \newline
22. अ॒ग्नी॒षो॒मीयेत्य॑ग्नी - सो॒मीया᳚ । \newline
23. य द॒नयो॑ र॒नयो॒र् यद् य द॒नयोः᳚ । \newline
24. अ॒नयो᳚र् विय॒त्योर् वि॑य॒त्यो र॒नयो॑ र॒नयो᳚र् विय॒त्योः । \newline
25. वि॒य॒त्योर् वाग् वाग् वि॑य॒त्योर् वि॑य॒त्योर् वाक् । \newline
26. वि॒य॒त्योरिति॑ वि - य॒त्योः । \newline
27. वागव॑द॒ दव॑द॒द् वाग् वागव॑दत् । \newline
28. अव॑द॒त् तस्मा॒त् तस्मा॒ दव॑द॒ दव॑द॒त् तस्मा᳚त् । \newline
29. तस्मा᳚थ् सारस्व॒ती सा॑रस्व॒ती तस्मा॒त् तस्मा᳚थ् सारस्व॒ती । \newline
30. सा॒र॒स्व॒ती यद् यथ् सा॑रस्व॒ती सा॑रस्व॒ती यत् । \newline
31. यत् प्र॒जाप॑तिः प्र॒जाप॑ति॒र् यद् यत् प्र॒जाप॑तिः । \newline
32. प्र॒जाप॑ति र॒ग्ने र॒ग्नेः प्र॒जाप॑तिः प्र॒जाप॑ति र॒ग्नेः । \newline
33. प्र॒जाप॑ति॒रिति॑ प्र॒जा - प॒तिः॒ । \newline
34. अ॒ग्ने रध्यध्य॒ ग्ने र॒ग्ने रधि॑ । \newline
35. अधि॑ नि॒रक्री॑णान् नि॒रक्री॑णा॒ दध्यधि॑ नि॒रक्री॑णात् । \newline
36. नि॒रक्री॑णा॒त् तस्मा॒त् तस्मा᳚न् नि॒रक्री॑णान् नि॒रक्री॑णा॒त् तस्मा᳚त् । \newline
37. नि॒रक्री॑णा॒दिति॑ निः - अक्री॑णात् । \newline
38. तस्मा᳚त् प्राजाप॒त्या प्रा॑जाप॒त्या तस्मा॒त् तस्मा᳚त् प्राजाप॒त्या । \newline
39. प्रा॒जा॒प॒त्या सा सा प्रा॑जाप॒त्या प्रा॑जाप॒त्या सा । \newline
40. प्रा॒जा॒प॒त्येति॑ प्राजा - प॒त्या । \newline
41. सा वै वै सा सा वै । \newline
42. वा ए॒षैषा वै वा ए॒षा । \newline
43. ए॒षा स॑र्वदेव॒त्या॑ सर्वदेव॒ त्यै॑षैषा स॑र्वदेव॒त्या᳚ । \newline
44. स॒र्व॒दे॒व॒त्या॑ यद् यथ् स॑र्वदेव॒त्या॑ सर्वदेव॒त्या॑ यत् । \newline
45. स॒र्व॒दे॒व॒त्येति॑ सर्व - दे॒व॒त्या᳚ । \newline
46. यद॒जा ऽजा यद् यद॒जा । \newline
47. अ॒जा व॒शा व॒शा ऽजा ऽजा व॒शा । \newline
48. व॒शा वा॑य॒व्यां᳚ ॅवाय॒व्यां᳚ ॅव॒शा व॒शा वा॑य॒व्या᳚म् । \newline
49. वा॒य॒व्या॑ मा वा॑य॒व्यां᳚ ॅवाय॒व्या॑ मा । \newline
50. आ ल॑भेत लभे॒ता ल॑भेत । \newline
51. ल॒भे॒त॒ भूति॑कामो॒ भूति॑कामो लभेत लभेत॒ भूति॑कामः । \newline
52. भूति॑कामो वा॒युर् वा॒युर् भूति॑कामो॒ भूति॑कामो वा॒युः । \newline
53. भूति॑काम॒ इति॒ भूति॑ - का॒मः॒ । \newline
54. वा॒युर् वै वै वा॒युर् वा॒युर् वै । \newline
55. वै क्षेपि॑ष्ठा॒ क्षेपि॑ष्ठा॒ वै वै क्षेपि॑ष्ठा । \newline
56. क्षेपि॑ष्ठा दे॒वता॑ दे॒वता॒ क्षेपि॑ष्ठा॒ क्षेपि॑ष्ठा दे॒वता᳚ । \newline
57. दे॒वता॑ वा॒युं ॅवा॒युम् दे॒वता॑ दे॒वता॑ वा॒युम् । \newline
58. वा॒यु मे॒वैव वा॒युं ॅवा॒यु मे॒व । \newline
59. ए॒व स्वेन॒ स्वेनै॒वैव स्वेन॑ । \newline
60. स्वेन॑ भाग॒धेये॑न भाग॒धेये॑न॒ स्वेन॒ स्वेन॑ भाग॒धेये॑न । \newline

\textbf{Ghana Paata } \newline

1. वा॒युर् व्यवा॒द् व्यवा᳚द् वा॒युर् वा॒युर् व्यवा॒त् तस्मा॒त् तस्मा॒द् व्यवा᳚द् वा॒युर् वा॒युर् व्यवा॒त् तस्मा᳚त् । \newline
2. व्यवा॒त् तस्मा॒त् तस्मा॒द् व्यवा॒द् व्यवा॒त् तस्मा᳚द् वाय॒व्या॑ वाय॒व्या॑ तस्मा॒द् व्यवा॒द् व्यवा॒त् तस्मा᳚द् वाय॒व्या᳚ । \newline
3. व्यवा॒दिति॑ वि - अवा᳚त् । \newline
4. तस्मा᳚द् वाय॒व्या॑ वाय॒व्या॑ तस्मा॒त् तस्मा᳚द् वाय॒व्या॑ यद् यद् वा॑य॒व्या॑ तस्मा॒त् तस्मा᳚द् वाय॒व्या॑ यत् । \newline
5. वा॒य॒व्या॑ यद् यद् वा॑य॒व्या॑ वाय॒व्या॑ यदि॒मे इ॒मे यद् वा॑य॒व्या॑ वाय॒व्या॑ यदि॒मे । \newline
6. यदि॒मे इ॒मे यद् यदि॒मे गर्भ॒म् गर्भ॑ मि॒मे यद् यदि॒मे गर्भ᳚म् । \newline
7. इ॒मे गर्भ॒म् गर्भ॑ मि॒मे इ॒मे गर्भ॒ मद॑धाता॒ मद॑धाता॒म् गर्भ॑ मि॒मे इ॒मे गर्भ॒ मद॑धाताम् । \newline
8. इ॒मे इती॒मे । \newline
9. गर्भ॒ मद॑धाता॒ मद॑धाता॒म् गर्भ॒म् गर्भ॒ मद॑धाता॒म् तस्मा॒त् तस्मा॒ दद॑धाता॒म् गर्भ॒म् गर्भ॒ मद॑धाता॒म् तस्मा᳚त् । \newline
10. अद॑धाता॒म् तस्मा॒त् तस्मा॒ दद॑धाता॒ मद॑धाता॒म् तस्मा᳚द् द्यावापृथि॒व्या᳚ द्यावापृथि॒व्या॑ तस्मा॒ दद॑धाता॒ मद॑धाता॒म् तस्मा᳚द् द्यावापृथि॒व्या᳚ । \newline
11. तस्मा᳚द् द्यावापृथि॒व्या᳚ द्यावापृथि॒व्या॑ तस्मा॒त् तस्मा᳚द् द्यावापृथि॒व्या॑ यद् यद् द्या॑वापृथि॒व्या॑ तस्मा॒त् तस्मा᳚द् द्यावापृथि॒व्या॑ यत् । \newline
12. द्या॒वा॒पृ॒थि॒व्या॑ यद् यद् द्या॑वापृथि॒व्या᳚ द्यावापृथि॒व्या॑ यथ् सोमः॒ सोमो॒ यद् द्या॑वापृथि॒व्या᳚ द्यावापृथि॒व्या॑ यथ् सोमः॑ । \newline
13. द्या॒वा॒पृ॒थि॒व्येति॑ द्यावा - पृ॒थि॒व्या᳚ । \newline
14. यथ् सोमः॒ सोमो॒ यद् यथ् सोमः॒ प्र प्र सोमो॒ यद् यथ् सोमः॒ प्र । \newline
15. सोमः॒ प्र प्र सोमः॒ सोमः॒ प्रा ज॑नय॒ दज॑नय॒त् प्र सोमः॒ सोमः॒ प्राज॑नयत् । \newline
16. प्रा ज॑नय॒ दज॑नय॒त् प्र प्रा ज॑नय द॒ग्नि र॒ग्नि रज॑नय॒त् प्र प्रा ज॑नय द॒ग्निः । \newline
17. अज॑नय द॒ग्नि र॒ग्नि रज॑नय॒ दज॑नय द॒ग्नि रग्र॑स॒ता ग्र॑सता॒ ग्नि रज॑नय॒ दज॑नय द॒ग्नि रग्र॑सत । \newline
18. अ॒ग्नि रग्र॑स॒ता ग्र॑सता॒ ग्निर॒ग्नि रग्र॑सत॒ तस्मा॒त् तस्मा॒ दग्र॑सता॒ ग्निर॒ग्नि रग्र॑सत॒ तस्मा᳚त् । \newline
19. अग्र॑सत॒ तस्मा॒त् तस्मा॒ दग्र॑स॒ता ग्र॑सत॒ तस्मा॑ दग्नीषो॒मीया᳚ ऽग्नीषो॒मीया॒ तस्मा॒ दग्र॑स॒ता ग्र॑सत॒ तस्मा॑ दग्नीषो॒मीया᳚ । \newline
20. तस्मा॑ दग्नीषो॒मीया᳚ ऽग्नीषो॒मीया॒ तस्मा॒त् तस्मा॑ दग्नीषो॒मीया॒ यद् यद॑ग्नीषो॒मीया॒ तस्मा॒त् तस्मा॑ दग्नीषो॒मीया॒ यत् । \newline
21. अ॒ग्नी॒षो॒मीया॒ यद् यद॑ग्नीषो॒मीया᳚ ऽग्नीषो॒मीया॒ यद॒नयो॑ र॒नयो॒र् यद॑ग्नीषो॒मीया᳚ ऽग्नीषो॒मीया॒ यद॒नयोः᳚ । \newline
22. अ॒ग्नी॒षो॒मीयेत्य॑ग्नी - सो॒मीया᳚ । \newline
23. यद॒नयो॑ र॒नयो॒र् यद् यद॒नयो᳚र् विय॒त्योर् वि॑य॒त्यो र॒नयो॒र् यद् यद॒नयो᳚र् विय॒त्योः । \newline
24. अ॒नयो᳚र् विय॒त्योर् वि॑य॒त्यो र॒नयो॑ र॒नयो᳚र् विय॒त्योर् वाग् वाग् वि॑य॒त्यो र॒नयो॑ र॒नयो᳚र् विय॒त्योर् वाक् । \newline
25. वि॒य॒त्योर् वाग् वाग् वि॑य॒त्योर् वि॑य॒त्योर् वा गव॑द॒ दव॑द॒द् वाग् वि॑य॒त्योर् वि॑य॒त्योर् वा गव॑दत् । \newline
26. वि॒य॒त्योरिति॑ वि - य॒त्योः । \newline
27. वागव॑द॒ दव॑द॒द् वाग् वागव॑द॒त् तस्मा॒त् तस्मा॒ दव॑द॒द् वाग् वागव॑द॒त् तस्मा᳚त् । \newline
28. अव॑द॒त् तस्मा॒त् तस्मा॒ दव॑द॒ दव॑द॒त् तस्मा᳚थ् सारस्व॒ती सा॑रस्व॒ती तस्मा॒ दव॑द॒ दव॑द॒त् तस्मा᳚थ् सारस्व॒ती । \newline
29. तस्मा᳚थ् सारस्व॒ती सा॑रस्व॒ती तस्मा॒त् तस्मा᳚थ् सारस्व॒ती यद् यथ् सा॑रस्व॒ती तस्मा॒त् तस्मा᳚थ् सारस्व॒ती यत् । \newline
30. सा॒र॒स्व॒ती यद् यथ् सा॑रस्व॒ती सा॑रस्व॒ती यत् प्र॒जाप॑तिः प्र॒जाप॑ति॒र् यथ् सा॑रस्व॒ती सा॑रस्व॒ती यत् प्र॒जाप॑तिः । \newline
31. यत् प्र॒जाप॑तिः प्र॒जाप॑ति॒र् यद् यत् प्र॒जाप॑ति र॒ग्ने र॒ग्नेः प्र॒जाप॑ति॒र् यद् यत् प्र॒जाप॑ति र॒ग्नेः । \newline
32. प्र॒जाप॑ति र॒ग्ने र॒ग्नेः प्र॒जाप॑तिः प्र॒जाप॑ति र॒ग्ने रध्य ध्य॒ग्नेः प्र॒जाप॑तिः प्र॒जाप॑ति र॒ग्ने रधि॑ । \newline
33. प्र॒जाप॑ति॒रिति॑ प्र॒जा - प॒तिः॒ । \newline
34. अ॒ग्ने रध्य ध्य॒ग्ने र॒ग्ने रधि॑ नि॒रक्री॑णान् नि॒रक्री॑णा॒ दध्य॒ग्ने र॒ग्नेरधि॑ नि॒रक्री॑णात् । \newline
35. अधि॑ नि॒रक्री॑णान् नि॒रक्री॑णा॒ दध्यधि॑ नि॒रक्री॑णा॒त् तस्मा॒त् तस्मा᳚न् नि॒रक्री॑णा॒ दध्यधि॑ नि॒रक्री॑णा॒त् तस्मा᳚त् । \newline
36. नि॒रक्री॑णा॒त् तस्मा॒त् तस्मा᳚न् नि॒रक्री॑णान् नि॒रक्री॑णा॒त् तस्मा᳚त् प्राजाप॒त्या प्रा॑जाप॒त्या तस्मा᳚न् नि॒रक्री॑णान् नि॒रक्री॑णा॒त् तस्मा᳚त् प्राजाप॒त्या । \newline
37. नि॒रक्री॑णा॒दिति॑ निः - अक्री॑णात् । \newline
38. तस्मा᳚त् प्राजाप॒त्या प्रा॑जाप॒त्या तस्मा॒त् तस्मा᳚त् प्राजाप॒त्या सा सा प्रा॑जाप॒त्या तस्मा॒त् तस्मा᳚त् प्राजाप॒त्या सा । \newline
39. प्रा॒जा॒प॒त्या सा सा प्रा॑जाप॒त्या प्रा॑जाप॒त्या सा वै वै सा प्रा॑जाप॒त्या प्रा॑जाप॒त्या सा वै । \newline
40. प्रा॒जा॒प॒त्येति॑ प्राजा - प॒त्या । \newline
41. सा वै वै सा सा वा ए॒षैषा वै सा सा वा ए॒षा । \newline
42. वा ए॒षैषा वै वा ए॒षा स॑र्वदेव॒त्या॑ सर्वदेव॒त्यै॑षा वै वा ए॒षा स॑र्वदेव॒त्या᳚ । \newline
43. ए॒षा स॑र्वदेव॒त्या॑ सर्वदेव॒त्यै॑षैषा स॑र्वदेव॒त्या॑ यद् यथ् स॑र्वदेव॒ त्यै॑षैषा स॑र्वदेव॒त्या॑ यत् । \newline
44. स॒र्व॒दे॒व॒त्या॑ यद् यथ् स॑र्वदेव॒त्या॑ सर्वदेव॒त्या॑ यद॒जा ऽजा यथ् स॑र्वदेव॒त्या॑ सर्वदेव॒त्या॑ यद॒जा । \newline
45. स॒र्व॒दे॒व॒त्येति॑ सर्व - दे॒व॒त्या᳚ । \newline
46. यद॒जा ऽजा यद् यद॒जा व॒शा व॒शा ऽजा यद् यद॒जा व॒शा । \newline
47. अ॒जा व॒शा व॒शा ऽजा ऽजा व॒शा वा॑य॒व्या᳚म् ॅवाय॒व्या᳚म् ॅव॒शा ऽजा ऽजा व॒शा वा॑य॒व्या᳚म् । \newline
48. व॒शा वा॑य॒व्या᳚म् ॅवाय॒व्या᳚म् ॅव॒शा व॒शा वा॑य॒व्या॑ मा वा॑य॒व्या᳚म् ॅव॒शा व॒शा वा॑य॒व्या॑ मा । \newline
49. वा॒य॒व्या॑ मा वा॑य॒व्या᳚म् ॅवाय॒व्या॑ मा ल॑भेत लभे॒ता वा॑य॒व्या᳚म् ॅवाय॒व्या॑ मा ल॑भेत । \newline
50. आ ल॑भेत लभे॒ता ल॑भेत॒ भूति॑कामो॒ भूति॑कामो लभे॒ता ल॑भेत॒ भूति॑कामः । \newline
51. ल॒भे॒त॒ भूति॑कामो॒ भूति॑कामो लभेत लभेत॒ भूति॑कामो वा॒युर् वा॒युर् भूति॑कामो लभेत लभेत॒ भूति॑कामो वा॒युः । \newline
52. भूति॑कामो वा॒युर् वा॒युर् भूति॑कामो॒ भूति॑कामो वा॒युर् वै वै वा॒युर् भूति॑कामो॒ भूति॑कामो वा॒युर् वै । \newline
53. भूति॑काम॒ इति॒ भूति॑ - का॒मः॒ । \newline
54. वा॒युर् वै वै वा॒युर् वा॒युर् वै क्षेपि॑ष्ठा॒ क्षेपि॑ष्ठा॒ वै वा॒युर् वा॒युर् वै क्षेपि॑ष्ठा । \newline
55. वै क्षेपि॑ष्ठा॒ क्षेपि॑ष्ठा॒ वै वै क्षेपि॑ष्ठा दे॒वता॑ दे॒वता॒ क्षेपि॑ष्ठा॒ वै वै क्षेपि॑ष्ठा दे॒वता᳚ । \newline
56. क्षेपि॑ष्ठा दे॒वता॑ दे॒वता॒ क्षेपि॑ष्ठा॒ क्षेपि॑ष्ठा दे॒वता॑ वा॒युम् ॅवा॒युम् दे॒वता॒ क्षेपि॑ष्ठा॒ क्षेपि॑ष्ठा दे॒वता॑ वा॒युम् । \newline
57. दे॒वता॑ वा॒युम् ॅवा॒युम् दे॒वता॑ दे॒वता॑ वा॒यु मे॒वैव वा॒युम् दे॒वता॑ दे॒वता॑ वा॒यु मे॒व । \newline
58. वा॒यु मे॒वैव वा॒युम् ॅवा॒यु मे॒व स्वेन॒ स्वेनै॒व वा॒युम् ॅवा॒यु मे॒व स्वेन॑ । \newline
59. ए॒व स्वेन॒ स्वेनै॒वैव स्वेन॑ भाग॒धेये॑न भाग॒धेये॑न॒ स्वेनै॒वैव स्वेन॑ भाग॒धेये॑न । \newline
60. स्वेन॑ भाग॒धेये॑न भाग॒धेये॑न॒ स्वेन॒ स्वेन॑ भाग॒धेये॒नोपोप॑ भाग॒धेये॑न॒ स्वेन॒ स्वेन॑ भाग॒धेये॒नोप॑ । \newline
\pagebreak
\markright{ TS 3.4.3.3  \hfill https://www.vedavms.in \hfill}

\section{ TS 3.4.3.3 }

\textbf{TS 3.4.3.3 } \newline
\textbf{Samhita Paata} \newline

भाग॒धेये॒नोप॑ धावति॒ स ए॒वैनं॒ भूतिं॑ गमयति द्यावापृथि॒व्या॑मा ल॑भेत कृ॒षमा॑णः प्रति॒ष्ठाका॑मो दि॒व ए॒वास्मै॑ प॒र्जन्यो॑ वर्.षति॒ व्य॑स्यामोष॑धयो रोहन्ति स॒मर्द्धु॑कमस्य स॒स्यं भ॑वत्यग्नीषो॒मीया॒मा ल॑भेत॒ यः का॒मये॒तान्न॑वानन्ना॒दः स्या॒मित्य॒ग्निनै॒वान्न॒मव॑ रुन्धे॒ सोमे॑ना॒न्नाद्य॒-मन्न॑वाने॒वान्ना॒दो भ॑वति सारस्व॒तीमा ल॑भेत॒ य - [  ] \newline

\textbf{Pada Paata} \newline

भा॒ग॒धेये॒नेति॑ भाग - धेये॑न । उपेति॑ । धा॒व॒ति॒ । सः । ए॒व । ए॒न॒म् । भूति᳚म् । ग॒म॒य॒ति॒ । द्या॒वा॒पृ॒थि॒व्या॑मिति॑ द्यावा - पृ॒थि॒व्या᳚म् । एति॑ । ल॒भे॒त॒ । कृ॒षमा॑णः । प्र॒ति॒ष्ठाका॑म॒ इति॑ प्रति॒ष्ठा-का॒मः॒ । दि॒वः । ए॒व । अ॒स्मै॒ । प॒र्जन्यः॑ । व॒र्॒.ष॒ति॒ । वीति॑ । अ॒स्याम् । ओष॑धयः । रो॒ह॒न्ति॒ । स॒मर्द्धु॑क॒मिति॑ सं - अर्द्धु॑कम् । अ॒स्य॒ । स॒स्यम् । भ॒व॒ति॒ । अ॒ग्नी॒षो॒मीया॒मित्य॑ग्नी-सो॒मीया᳚म् । एति॑ । ल॒भे॒त॒ । यः । का॒मये॑त । अन्न॑वा॒न्नित्यन्न॑ - वा॒न् । अ॒न्ना॒द इत्य॑न्न - अ॒दः । स्या॒म् । इति॑ । अ॒ग्निना᳚ । ए॒व । अन्न᳚म् । अवेति॑ । रु॒न्धे॒ । सोमे॑न । अ॒न्नाद्य॒मित्य॑न्न - अद्य᳚म् । अन्न॑वा॒न्नित्यन्न॑ - वा॒न् । ए॒व । अ॒न्ना॒द इत्य॑न्न - अ॒दः । भ॒व॒ति॒ । सा॒र॒स्व॒तीम् । एति॑ । ल॒भे॒त॒ । यः ।  \newline


\textbf{Krama Paata} \newline

भा॒ग॒धेये॒नोप॑ । भा॒ग॒धेये॒नेति॑ भाग - धेये॑न । उप॑ धावति । धा॒व॒ति॒ सः । स ए॒व । ए॒वैन᳚म् । ए॒न॒म् भूति᳚म् । भूति॑म् गमयति । ग॒म॒य॒ति॒ द्या॒वा॒पृ॒थि॒व्या᳚म् । द्या॒वा॒पृ॒थि॒व्या॑मा । द्या॒वा॒पृ॒थि॒व्या॑मिति॑ द्यावा - पृ॒थि॒व्या᳚म् । आ ल॑भेत । ल॒भे॒त॒ कृ॒षमा॑णः । कृ॒षमा॑णः प्रति॒ष्ठाका॑मः । प्र॒ति॒ष्ठाका॑मो दि॒वः । प्र॒ति॒ष्ठाका॑म॒ इति॑ प्रति॒ष्ठा - का॒मः॒ । दि॒व ए॒व । ए॒वास्मै᳚ । अ॒स्मै॒ प॒र्जन्यः॑ । प॒र्जन्यो॑ वर्.षति । व॒र्॒.ष॒ति॒ वि । व्य॑स्याम् । अ॒स्यामोष॑धयः । ओष॑धयो रोहन्ति । रो॒ह॒न्ति॒ स॒मर्द्धु॑कम् । स॒मर्द्धु॑कमस्य । स॒मर्द्धु॑क॒मिति॑ सम् - अर्द्धु॑कम् । अ॒स्य॒ स॒स्यम् । स॒स्यम् भ॑वति । भ॒व॒त्य॒ग्नी॒षो॒मीया᳚म् । अ॒ग्नी॒षो॒मीया॒मा । अ॒ग्नी॒षो॒मीया॒मित्य॑ग्नी - सो॒मीया᳚म् । आ ल॑भेत । ल॒भे॒त॒ यः । यः का॒मये॑त । का॒मये॒तान्न॑वान् । अन्न॑वानन्ना॒दः । अन्न॑वा॒नित्यन्न॑ - वा॒न्॒ । अ॒न्ना॒दः स्या᳚म् । अ॒न्ना॒द इत्य॑न्न - अ॒दः । स्या॒मिति॑ । इत्य॒ग्निना᳚ । अ॒ग्निनै॒व । ए॒वान्न᳚म् । अन्न॒मव॑ । अव॑ रुन्धे । रु॒न्धे॒ सोमे॑न । सोमे॑ना॒न्नाद्य᳚म् । अ॒न्नाद्य॒मन्न॑वान् । अ॒न्नाद्य॒मित्य॑न्न - अद्य᳚म् । अन्न॑वाने॒व । अन्न॑वा॒नित्यन्न॑ - वा॒न्॒ । ए॒वान्ना॒दः । अ॒न्ना॒दो भ॑वति । अ॒न्ना॒द इत्य॑न्न - अ॒दः । भ॒व॒ति॒ सा॒र॒स्व॒तीम् । सा॒र॒स्व॒तीमा । आ ल॑भेत । ल॒भे॒त॒ यः । य ई᳚श्व॒रः \newline

\textbf{Jatai Paata} \newline

1. भा॒ग॒धेये॒नो पोप॑ भाग॒धेये॑न भाग॒धेये॒ नोप॑ । \newline
2. भा॒ग॒धेये॒नेति॑ भाग - धेये॑न । \newline
3. उप॑ धावति धाव॒ त्युपोप॑ धावति । \newline
4. धा॒व॒ति॒ स स धा॑वति धावति॒ सः । \newline
5. स ए॒वैव स स ए॒व । \newline
6. ए॒वैन॑ मेन मे॒वैवैन᳚म् । \newline
7. ए॒न॒म् भूति॒म् भूति॑ मेन मेन॒म् भूति᳚म् । \newline
8. भूति॑म् गमयति गमयति॒ भूति॒म् भूति॑म् गमयति । \newline
9. ग॒म॒य॒ति॒ द्या॒वा॒पृ॒थि॒व्या᳚म् द्यावापृथि॒व्या᳚म् गमयति गमयति द्यावापृथि॒व्या᳚म् । \newline
10. द्या॒वा॒पृ॒थि॒व्या॑ मा द्या॑वापृथि॒व्या᳚म् द्यावापृथि॒व्या॑ मा । \newline
11. द्या॒वा॒पृ॒थि॒व्या॑मिति॑ द्यावा - पृ॒थि॒व्या᳚म् । \newline
12. आ ल॑भेत लभे॒ता ल॑भेत । \newline
13. ल॒भे॒त॒ कृ॒षमा॑णः कृ॒षमा॑णो लभेत लभेत कृ॒षमा॑णः । \newline
14. कृ॒षमा॑णः प्रति॒ष्ठाका॑मः प्रति॒ष्ठाका॑मः कृ॒षमा॑णः कृ॒षमा॑णः प्रति॒ष्ठाका॑मः । \newline
15. प्र॒ति॒ष्ठाका॑मो दि॒वो दि॒वः प्र॑ति॒ष्ठाका॑मः प्रति॒ष्ठाका॑मो दि॒वः । \newline
16. प्र॒ति॒ष्ठाका॑म॒ इति॑ प्रति॒ष्ठा - का॒मः॒ । \newline
17. दि॒व ए॒वैव दि॒वो दि॒व ए॒व । \newline
18. ए॒वास्मा॑ अस्मा ए॒वैवास्मै᳚ । \newline
19. अ॒स्मै॒ प॒र्जन्यः॑ प॒र्जन्यो᳚ ऽस्मा अस्मै प॒र्जन्यः॑ । \newline
20. प॒र्जन्यो॑ वर्.षति वर्.षति प॒र्जन्यः॑ प॒र्जन्यो॑ वर्.षति । \newline
21. व॒र्॒.ष॒ति॒ वि वि व॑र्.षति वर्.षति॒ वि । \newline
22. व्य॑स्या म॒स्यां ॅवि व्य॑स्याम् । \newline
23. अ॒स्या मोष॑धय॒ ओष॑धयो॒ ऽस्या म॒स्या मोष॑धयः । \newline
24. ओष॑धयो रोहन्ति रोह॒ न्त्योष॑धय॒ ओष॑धयो रोहन्ति । \newline
25. रो॒ह॒न्ति॒ स॒मर्द्धु॑कꣳ स॒मर्द्धु॑कꣳ रोहन्ति रोहन्ति स॒मर्द्धु॑कम् । \newline
26. स॒मर्द्धु॑क मस्यास्य स॒मर्द्धु॑कꣳ स॒मर्द्धु॑क मस्य । \newline
27. स॒मर्द्धु॑क॒मिति॑ सं - अर्द्धु॑कम् । \newline
28. अ॒स्य॒ स॒स्यꣳ स॒स्य म॑स्यास्य स॒स्यम् । \newline
29. स॒स्यम् भ॑वति भवति स॒स्यꣳ स॒स्यम् भ॑वति । \newline
30. भ॒व॒ त्य॒ग्नी॒षो॒मीया॑ मग्नीषो॒मीया᳚म् भवति भव त्यग्नीषो॒मीया᳚म् । \newline
31. अ॒ग्नी॒षो॒मीया॒मा ऽग्नी॑षो॒मीया॑ मग्नीषो॒मीया॒मा । \newline
32. अ॒ग्नी॒षो॒मीया॒मित्य॑ग्नी - सो॒मीया᳚म् । \newline
33. आ ल॑भेत लभे॒ता ल॑भेत । \newline
34. ल॒भे॒त॒ यो यो ल॑भेत लभेत॒ यः । \newline
35. यः का॒मये॑त का॒मये॑त॒ यो यः का॒मये॑त । \newline
36. का॒मये॒ता न्न॑वा॒-नन्न॑वान् का॒मये॑त का॒मये॒ता न्न॑वान् । \newline
37. अन्न॑वा-नन्ना॒दो᳚ ऽन्ना॒दो ऽन्न॑वा॒-नन्न॑वा-नन्ना॒दः । \newline
38. अन्न॑वा॒न्नित्यन्न॑ - वा॒न् । \newline
39. अ॒न्ना॒दः स्याꣳ॑ स्या मन्ना॒दो᳚ ऽन्ना॒दः स्या᳚म् । \newline
40. अ॒न्ना॒द इत्य॑न्न - अ॒दः । \newline
41. स्या॒ मितीति॑ स्याꣳ स्या॒ मिति॑ । \newline
42. इत्य॒ग्निना॒ ऽग्निनेती त्य॒ग्निना᳚ । \newline
43. अ॒ग्निनै॒ वैवाग्निना॒ ऽग्निनै॒व । \newline
44. ए॒वान्न॒ मन्न॑ मे॒वैवान्न᳚म् । \newline
45. अन्न॒ मवावान्न॒ मन्न॒ मव॑ । \newline
46. अव॑ रुन्धे रु॒न्धे ऽवाव॑ रुन्धे । \newline
47. रु॒न्धे॒ सोमे॑न॒ सोमे॑न रुन्धे रुन्धे॒ सोमे॑न । \newline
48. सोमे॑ना॒ न्नाद्य॑ म॒न्नाद्यꣳ॒॒ सोमे॑न॒ सोमे॑ना॒ न्नाद्य᳚म् । \newline
49. अ॒न्नाद्य॒ मन्न॑वा॒-नन्न॑वा-न॒न्नाद्य॑ म॒न्नाद्य॒ मन्न॑वान् । \newline
50. अ॒न्नाद्य॒मित्य॑न्न - अद्य᳚म् । \newline
51. अन्न॑वा-ने॒वैवान्न॑वा॒-नन्न॑वा-ने॒व । \newline
52. अन्न॑वा॒न्नित्यन्न॑ - वा॒न् । \newline
53. ए॒वान्ना॒दो᳚ ऽन्ना॒द ए॒वैवान्ना॒दः । \newline
54. अ॒न्ना॒दो भ॑वति भव त्यन्ना॒दो᳚ ऽन्ना॒दो भ॑वति । \newline
55. अ॒न्ना॒द इत्य॑न्न - अ॒दः । \newline
56. भ॒व॒ति॒ सा॒र॒स्व॒तीꣳ सा॑रस्व॒तीम् भ॑वति भवति सारस्व॒तीम् । \newline
57. सा॒र॒स्व॒ती मा सा॑रस्व॒तीꣳ सा॑रस्व॒ती मा । \newline
58. आ ल॑भेत लभे॒ता ल॑भेत । \newline
59. ल॒भे॒त॒ यो यो ल॑भेत लभेत॒ यः । \newline
60. य ई᳚श्व॒र ई᳚श्व॒रो यो य ई᳚श्व॒रः । \newline

\textbf{Ghana Paata } \newline

1. भा॒ग॒धेये॒नोपोप॑ भाग॒धेये॑न भाग॒धेये॒नोप॑ धावति धाव॒ त्युप॑ भाग॒धेये॑न भाग॒धेये॒ नोप॑ धावति । \newline
2. भा॒ग॒धेये॒नेति॑ भाग - धेये॑न । \newline
3. उप॑ धावति धाव॒ त्युपोप॑ धावति॒ स स धा॑व॒ त्युपोप॑ धावति॒ सः । \newline
4. धा॒व॒ति॒ स स धा॑वति धावति॒ स ए॒वैव स धा॑वति धावति॒ स ए॒व । \newline
5. स ए॒वैव स स ए॒वैन॑ मेन मे॒व स स ए॒वैन᳚म् । \newline
6. ए॒वैन॑ मेन मे॒वैवैन॒म् भूति॒म् भूति॑ मेन मे॒वैवैन॒म् भूति᳚म् । \newline
7. ए॒न॒म् भूति॒म् भूति॑ मेन मेन॒म् भूति॑म् गमयति गमयति॒ भूति॑ मेन मेन॒म् भूति॑म् गमयति । \newline
8. भूति॑म् गमयति गमयति॒ भूति॒म् भूति॑म् गमयति द्यावापृथि॒व्या᳚म् द्यावापृथि॒व्या᳚म् गमयति॒ भूति॒म् भूति॑म् गमयति द्यावापृथि॒व्या᳚म् । \newline
9. ग॒म॒य॒ति॒ द्या॒वा॒पृ॒थि॒व्या᳚म् द्यावापृथि॒व्या᳚म् गमयति गमयति द्यावापृथि॒व्या॑ मा द्या॑वापृथि॒व्या᳚म् गमयति गमयति द्यावापृथि॒व्या॑ मा । \newline
10. द्या॒वा॒पृ॒थि॒व्या॑ मा द्या॑वापृथि॒व्या᳚म् द्यावापृथि॒व्या॑ मा ल॑भेत लभे॒ता द्या॑वापृथि॒व्या᳚म् द्यावापृथि॒व्या॑ मा ल॑भेत । \newline
11. द्या॒वा॒पृ॒थि॒व्या॑मिति॑ द्यावा - पृ॒थि॒व्या᳚म् । \newline
12. आ ल॑भेत लभे॒ता ल॑भेत कृ॒षमा॑णः कृ॒षमा॑णो लभे॒ता ल॑भेत कृ॒षमा॑णः । \newline
13. ल॒भे॒त॒ कृ॒षमा॑णः कृ॒षमा॑णो लभेत लभेत कृ॒षमा॑णः प्रति॒ष्ठाका॑मः प्रति॒ष्ठाका॑मः कृ॒षमा॑णो लभेत लभेत कृ॒षमा॑णः प्रति॒ष्ठाका॑मः । \newline
14. कृ॒षमा॑णः प्रति॒ष्ठाका॑मः प्रति॒ष्ठाका॑मः कृ॒षमा॑णः कृ॒षमा॑णः प्रति॒ष्ठाका॑मो दि॒वो दि॒वः प्र॑ति॒ष्ठाका॑मः कृ॒षमा॑णः कृ॒षमा॑णः प्रति॒ष्ठाका॑मो दि॒वः । \newline
15. प्र॒ति॒ष्ठाका॑मो दि॒वो दि॒वः प्र॑ति॒ष्ठाका॑मः प्रति॒ष्ठाका॑मो दि॒व ए॒वैव दि॒वः प्र॑ति॒ष्ठाका॑मः प्रति॒ष्ठाका॑मो दि॒व ए॒व । \newline
16. प्र॒ति॒ष्ठाका॑म॒ इति॑ प्रति॒ष्ठा - का॒मः॒ । \newline
17. दि॒व ए॒वैव दि॒वो दि॒व ए॒वास्मा॑ अस्मा ए॒व दि॒वो दि॒व ए॒वास्मै᳚ । \newline
18. ए॒वास्मा॑ अस्मा ए॒वैवास्मै॑ प॒र्जन्यः॑ प॒र्जन्यो᳚ ऽस्मा ए॒वैवास्मै॑ प॒र्जन्यः॑ । \newline
19. अ॒स्मै॒ प॒र्जन्यः॑ प॒र्जन्यो᳚ ऽस्मा अस्मै प॒र्जन्यो॑ वर्.षति वर्.षति प॒र्जन्यो᳚ ऽस्मा अस्मै प॒र्जन्यो॑ वर्.षति । \newline
20. प॒र्जन्यो॑ वर्.षति वर्.षति प॒र्जन्यः॑ प॒र्जन्यो॑ वर्.षति॒ वि वि व॑र्.षति प॒र्जन्यः॑ प॒र्जन्यो॑ वर्.षति॒ वि । \newline
21. व॒र्॒.ष॒ति॒ वि वि व॑र्.षति वर्.षति॒ व्य॑स्या म॒स्याम् ॅवि व॑र्.षति वर्.षति॒ व्य॑स्याम् । \newline
22. व्य॑स्या म॒स्याम् ॅवि व्य॑स्या मोष॑धय॒ ओष॑धयो॒ ऽस्याम् ॅवि व्य॑स्या मोष॑धयः । \newline
23. अ॒स्या मोष॑धय॒ ओष॑धयो॒ ऽस्या म॒स्या मोष॑धयो रोहन्ति रोह॒ न्त्योष॑धयो॒ ऽस्या म॒स्या मोष॑धयो रोहन्ति । \newline
24. ओष॑धयो रोहन्ति रोह॒ न्त्योष॑धय॒ ओष॑धयो रोहन्ति स॒मर्द्धु॑कꣳ स॒मर्द्धु॑कꣳ रोह॒
न्त्योष॑धय॒ ओष॑धयो रोहन्ति स॒मर्द्धु॑कम् । \newline
25. रो॒ह॒न्ति॒ स॒मर्द्धु॑कꣳ स॒मर्द्धु॑कꣳ रोहन्ति रोहन्ति स॒मर्द्धु॑क मस्यास्य स॒मर्द्धु॑कꣳ रोहन्ति रोहन्ति स॒मर्द्धु॑क मस्य । \newline
26. स॒मर्द्धु॑क मस्यास्य स॒मर्द्धु॑कꣳ स॒मर्द्धु॑क मस्य स॒स्यꣳ स॒स्य म॑स्य स॒मर्द्धु॑कꣳ स॒मर्द्धु॑क मस्य स॒स्यम् । \newline
27. स॒मद्‌र्धु॑क॒मिति॑ सम् - अर्द्धु॑कम् । \newline
28. अ॒स्य॒ स॒स्यꣳ स॒स्य म॑स्यास्य स॒स्यम् भ॑वति भवति स॒स्य म॑स्यास्य स॒स्यम् भ॑वति । \newline
29. स॒स्यम् भ॑वति भवति स॒स्यꣳ स॒स्यम् भ॑व त्यग्नीषो॒मीया॑ मग्नीषो॒मीया᳚म् भवति स॒स्यꣳ स॒स्यम् 
भ॑व त्यग्नीषो॒मीया᳚म् । \newline
30. भ॒व॒ त्य॒ग्नी॒षो॒मीया॑ मग्नीषो॒मीया᳚म् भवति भव त्यग्नीषो॒मीया॒मा ऽग्नी॑षो॒मीया᳚म् भवति भव त्यग्नीषो॒मीया॒ मा । \newline
31. अ॒ग्नी॒षो॒मीया॒ मा ऽग्नी॑षो॒मीया॑ मग्नीषो॒मीया॒ मा ल॑भेत लभे॒ता ऽग्नी॑षो॒मीया॑ मग्नीषो॒मीया॒ मा ल॑भेत । \newline
32. अ॒ग्नी॒षो॒मीया॒मित्य॑ग्नी - सो॒मीया᳚म् । \newline
33. आ ल॑भेत लभे॒ता ल॑भेत॒ यो यो ल॑भे॒ता ल॑भेत॒ यः । \newline
34. ल॒भे॒त॒ यो यो ल॑भेत लभेत॒ यः का॒मये॑त का॒मये॑त॒ यो ल॑भेत लभेत॒ यः का॒मये॑त । \newline
35. यः का॒मये॑त का॒मये॑त॒ यो यः का॒मये॒ता न्न॑वा॒,नन्न॑वान् का॒मये॑त॒ यो यः का॒मये॒ता,न्न॑वान् । \newline
36. का॒मये॒ता,न्न॑वा॒,नन्न॑वान् का॒मये॑त का॒मये॒ता न्न॑वा,नन्ना॒दो᳚ ऽन्ना॒दो ऽन्न॑वान् का॒मये॑त का॒मये॒ता,न्न॑वा-नन्ना॒दः । \newline
37. अन्न॑वा,नन्ना॒दो᳚ ऽन्ना॒दो ऽन्न॑वा॒,नन्न॑वा,नन्ना॒दः स्याꣳ॑ स्यामन्ना॒दो ऽन्न॑वा॒,नन्न॑वा,नन्ना॒दः स्या᳚म् । \newline
38. अन्न॑वा॒न्नित्यन्न॑ - वा॒न् । \newline
39. अ॒न्ना॒दः स्याꣳ॑ स्यामन्ना॒दो᳚ ऽन्ना॒दः स्या॒मितीति॑ स्यामन्ना॒दो᳚ ऽन्ना॒दः स्या॒मिति॑ । \newline
40. अ॒न्ना॒द इत्य॑न्न - अ॒दः । \newline
41. स्या॒ मितीति॑ स्याꣳ स्या॒मित्य॒ ग्निना॒ ऽग्निनेति॑ स्याꣳ स्या॒मित्य॒ ग्निना᳚ । \newline
42. इत्य॒ग्निना॒ ऽग्नि नेतीत्य॒ग्नि नै॒वैवाग्नि नेतीत्य॒ग्नि नै॒व । \newline
43. अ॒ग्नि नै॒वैवाग्निना॒ ऽग्निनै॒वान्न॒ मन्न॑ मे॒वाग्निना॒ ऽग्निनै॒वान्न᳚म् । \newline
44. ए॒वान्न॒ मन्न॑ मे॒वैवान्न॒ मवावान्न॑ मे॒वैवान्न॒ मव॑ । \newline
45. अन्न॒मवा वान्न॒ मन्न॒ मव॑ रुन्धे रु॒न्धे ऽवान्न॒ मन्न॒ मव॑ रुन्धे । \newline
46. अव॑ रुन्धे रु॒न्धे ऽवाव॑ रुन्धे॒ सोमे॑न॒ सोमे॑न रु॒न्धे ऽवाव॑ रुन्धे॒ सोमे॑न । \newline
47. रु॒न्धे॒ सोमे॑न॒ सोमे॑न रुन्धे रुन्धे॒ सोमे॑ना॒ न्नाद्य॑ म॒न्नाद्यꣳ॒॒ सोमे॑न रुन्धे रुन्धे॒ सोमे॑ना॒ न्नाद्य᳚म् । \newline
48. सोमे॑ना॒ न्नाद्य॑ म॒न्नाद्यꣳ॒॒ सोमे॑न॒ सोमे॑ना॒ न्नाद्य॒ मन्न॑वा॒,नन्न॑वा,न॒न्नाद्यꣳ॒॒ सोमे॑न॒ सोमे॑ना॒न्नाद्य॒ मन्न॑वान् । \newline
49. अ॒न्नाद्य॒ मन्न॑वा॒,नन्न॑वा,न॒न्नाद्य॑ म॒न्नाद्य॒ मन्न॑वा, ने॒वैवान्न॑वा,न॒न्नाद्य॑ म॒न्नाद्य॒ मन्न॑वा,ने॒व । \newline
50. अ॒न्नाद्य॒मित्य॑न्न - अद्य᳚म् । \newline
51. अन्न॑वा,ने॒वैवान्न॑वा॒,नन्न॑वा,ने॒वान्ना॒दो᳚ ऽन्ना॒द ए॒वान्न॑वा॒,नन्न॑वा,ने॒वान्ना॒दः । \newline
52. अन्न॑वा॒न्नित्यन्न॑ - वा॒न् । \newline
53. ए॒वान्ना॒दो᳚ ऽन्ना॒द ए॒वैवान्ना॒दो भ॑वति भवत्यन्ना॒द ए॒वैवान्ना॒दो भ॑वति । \newline
54. अ॒न्ना॒दो भ॑वति भवत्यन्ना॒दो᳚ ऽन्ना॒दो भ॑वति सारस्व॒तीꣳ सा॑रस्व॒तीम् भ॑वत्यन्ना॒दो᳚ ऽन्ना॒दो भ॑वति सारस्व॒तीम् । \newline
55. अ॒न्ना॒द इत्य॑न्न - अ॒दः । \newline
56. भ॒व॒ति॒ सा॒र॒स्व॒तीꣳ सा॑रस्व॒तीम् भ॑वति भवति सारस्व॒तीमा सा॑रस्व॒तीम् भ॑वति भवति सारस्व॒तीमा । \newline
57. सा॒र॒स्व॒तीमा सा॑रस्व॒तीꣳ सा॑रस्व॒तीमा ल॑भेत लभे॒ता सा॑रस्व॒तीꣳ सा॑रस्व॒तीमा ल॑भेत । \newline
58. आ ल॑भेत लभे॒ता ल॑भेत॒ यो यो ल॑भे॒ता ल॑भेत॒ यः । \newline
59. ल॒भे॒त॒ यो यो ल॑भेत लभेत॒ य ई᳚श्व॒र ई᳚श्व॒रो यो ल॑भेत लभेत॒ य ई᳚श्व॒रः । \newline
60. य ई᳚श्व॒र ई᳚श्व॒रो यो य ई᳚श्व॒रो वा॒चो वा॒च ई᳚श्व॒रो यो य ई᳚श्व॒रो वा॒चः । \newline
\pagebreak
\markright{ TS 3.4.3.4  \hfill https://www.vedavms.in \hfill}

\section{ TS 3.4.3.4 }

\textbf{TS 3.4.3.4 } \newline
\textbf{Samhita Paata} \newline

ई᳚श्व॒रो वा॒चो वदि॑तोः॒ सन्. वाचं॒ नवदे॒द्-वाग्वै सर॑स्वती॒ सर॑स्वतीमे॒व स्वेन॑ भाग॒धेये॒नोप॑ धावति॒ सैवास्मि॒न्. वाचं॑ दधाति प्राजाप॒त्यामा ल॑भेत॒ यः का॒मये॒तान॑भिजितम॒भि ज॑येय॒मिति॑ प्र॒जाप॑तिः॒ सर्वा॑ दे॒वता॑ दे॒वता॑भिरे॒वा-न॑भिजितम॒भि ज॑यति वाय॒व्य॑यो॒पाक॑रोति वा॒योरे॒वैना॑मव॒रुद्ध्याऽऽ*ल॑भत॒ आकू᳚त्यै त्वा॒ कामा॑य॒ त्वे - [  ] \newline

\textbf{Pada Paata} \newline

ई॒श्व॒रः । वा॒चः । वदि॑तोः । सन्न् । वाच᳚म् । न । वदे᳚त् । वाक् । वै । सर॑स्वती । सर॑स्वतीम् । ए॒व । स्वेन॑ । भा॒ग॒धेये॒नेति॑ भाग - धेये॑न । उपेति॑ । धा॒व॒ति॒ । सा । ए॒व । अ॒स्मि॒न्न् । वाच᳚म् । द॒धा॒ति॒ । प्रा॒जा॒प॒त्यामिति॑ प्राजा - प॒त्याम् । एति॑ । ल॒भे॒त॒ । यः । का॒मये॑त । अन॑भिजित॒मित्यन॑भि - जि॒त॒म् । अ॒भीति॑ । ज॒ये॒य॒म् । इति॑ । प्र॒जाप॑ति॒रिति॑ प्र॒जा - प॒तिः॒ । सर्वाः᳚ । दे॒वताः᳚ । दे॒वता॑भिः । ए॒व । अन॑भिजित॒मित्यन॑भि - जि॒त॒म् । अ॒भीति॑ । ज॒य॒ति॒ । वा॒य॒व्य॑या । उ॒पाक॑रो॒तीयु॑प - आक॑रोति । वा॒योः । ए॒व । ए॒ना॒म् । अ॒व॒रुद्ध्येत्य॑व - रुद्ध्य॑ । एति॑ । ल॒भ॒ते॒ । आकू᳚त्या॒ इत्या - कू॒त्यै॒ । त्वा॒ । कामा॑य । त्वा॒ ।  \newline


\textbf{Krama Paata} \newline

ई॒श्व॒रो वा॒चः । वा॒चो वदि॑तोः । वदि॑तोः॒ सन्न् । सन् वाच᳚म् । वाच॒म् न । न वदे᳚त् । वदे॒द् वाक् । वाग् वै । वै सर॑स्वती । सर॑स्वती॒ सर॑स्वतीम् । सर॑स्वतीमे॒व । ए॒व स्वेन॑ । स्वेन॑ भाग॒धेये॑न । भा॒ग॒धेये॒नोप॑ । भा॒ग॒धेये॒नेति॑ भाग - धेये॑न । उप॑ धावति । धा॒व॒ति॒ सा । सैव । ए॒वास्मिन्न्॑ । अ॒स्मि॒न् वाच᳚म् । वाच॑म् दधाति । द॒धा॒ति॒ प्रा॒जा॒प॒त्याम् । प्रा॒जा॒प॒त्यामा । प्रा॒जा॒प॒त्यामिति॑ प्राजा - प॒त्याम् । आ ल॑भेत । ल॒भे॒त॒ यः । यः का॒मये॑त । का॒मये॒तान॑भिजितम् । अन॑भिजितम॒भि । अन॑भिजित॒मित्यन॑भि - जि॒त॒म् । अ॒भि ज॑येयम् । ज॒ये॒य॒मिति॑ । इति॑ प्र॒जाप॑तिः । प्र॒जाप॑तिः॒ सर्वाः᳚ । प्र॒जाप॑ति॒रिति॑ प्र॒जा - प॒तिः॒ । सर्वा॑ दे॒वताः᳚ । दे॒वता॑ दे॒वता॑भिः । दे॒वता॑भिरे॒व । ए॒वान॑भिजितम् । अन॑भिजितम॒भि । अन॑भिजित॒मित्यन॑भि - जि॒त॒म् । अ॒भि ज॑यति । ज॒य॒ति॒ वा॒य॒व्य॑या । वा॒य॒व्य॑यो॒पाक॑रोति । उ॒पाक॑रोति वा॒योः । उ॒पाक॑रो॒तीत्यु॑प - आक॑रोति । वा॒योरे॒व । ए॒वैना᳚म् । ए॒ना॒म॒व॒रुद्ध्य॑ । अ॒व॒रुद्ध्या । अ॒व॒रुद्ध्येत्य॑व - रुद्ध्य॑ । आ ल॑भते । ल॒भ॒त॒ आकू᳚त्यै । आकू᳚त्यै त्वा । आकू᳚त्या॒ इत्या - कू॒त्यै॒ । त्वा॒ कामा॑य । कामा॑य त्वा । त्वेति॑ \newline

\textbf{Jatai Paata} \newline

1. ई॒श्व॒रो वा॒चो वा॒च ई᳚श्व॒र ई᳚श्व॒रो वा॒चः । \newline
2. वा॒चो वदि॑तो॒र् वदि॑तोर् वा॒चो वा॒चो वदि॑तोः । \newline
3. वदि॑तोः॒ सन् थ्सन्. वदि॑तो॒र् वदि॑तोः॒ सन्न् । \newline
4. सन्. वाचं॒ ॅवाचꣳ॒॒ सन् थ्सन्. वाच᳚म् । \newline
5. वाच॒म् न न वाचं॒ ॅवाच॒म् न । \newline
6. न वदे॒द् वदे॒न् न न वदे᳚त् । \newline
7. वदे॒द् वाग् वाग् वदे॒द् वदे॒द् वाक् । \newline
8. वाग् वै वै वाग् वाग् वै । \newline
9. वै सर॑स्वती॒ सर॑स्वती॒ वै वै सर॑स्वती । \newline
10. सर॑स्वती॒ सर॑स्वतीꣳ॒॒ सर॑स्वतीꣳ॒॒ सर॑स्वती॒ सर॑स्वती॒ सर॑स्वतीम् । \newline
11. सर॑स्वती मे॒वैव सर॑स्वतीꣳ॒॒ सर॑स्वती मे॒व । \newline
12. ए॒व स्वेन॒ स्वेनै॒वैव स्वेन॑ । \newline
13. स्वेन॑ भाग॒धेये॑न भाग॒धेये॑न॒ स्वेन॒ स्वेन॑ भाग॒धेये॑न । \newline
14. भा॒ग॒धेये॒नो पोप॑ भाग॒धेये॑न भाग॒धेये॒ नोप॑ । \newline
15. भा॒ग॒धेये॒नेति॑ भाग - धेये॑न । \newline
16. उप॑ धावति धाव॒ त्युपोप॑ धावति । \newline
17. धा॒व॒ति॒ सा सा धा॑वति धावति॒ सा । \newline
18. सैवैव सा सैव । \newline
19. ए॒वास्मि॑न्-नस्मिन्-ने॒वैवास्मिन्न्॑ । \newline
20. अ॒स्मि॒न्॒. वाचं॒ ॅवाच॑ मस्मिन्-नस्मि॒न्॒. वाच᳚म् । \newline
21. वाच॑म् दधाति दधाति॒ वाचं॒ ॅवाच॑म् दधाति । \newline
22. द॒धा॒ति॒ प्रा॒जा॒प॒त्याम् प्रा॑जाप॒त्याम् द॑धाति दधाति प्राजाप॒त्याम् । \newline
23. प्रा॒जा॒प॒त्या मा प्रा॑जाप॒त्याम् प्रा॑जाप॒त्या मा । \newline
24. प्रा॒जा॒प॒त्यामिति॑ प्राजा - प॒त्याम् । \newline
25. आ ल॑भेत लभे॒ता ल॑भेत । \newline
26. ल॒भे॒त॒ यो यो ल॑भेत लभेत॒ यः । \newline
27. यः का॒मये॑त का॒मये॑त॒ यो यः का॒मये॑त । \newline
28. का॒मये॒ता न॑भिजित॒ मन॑भिजितम् का॒मये॑त का॒मये॒ता न॑भिजितम् । \newline
29. अन॑भिजित म॒भ्य॑भ्य न॑भिजित॒ मन॑भिजित म॒भि । \newline
30. अन॑भिजित॒मित्यन॑भि - जि॒त॒म् । \newline
31. अ॒भि ज॑येयम् जयेय म॒भ्य॑भि ज॑येयम् । \newline
32. ज॒ये॒य॒ मितीति॑ जयेयम् जयेय॒ मिति॑ । \newline
33. इति॑ प्र॒जाप॑तिः प्र॒जाप॑ति॒ रितीति॑ प्र॒जाप॑तिः । \newline
34. प्र॒जाप॑तिः॒ सर्वाः॒ सर्वाः᳚ प्र॒जाप॑तिः प्र॒जाप॑तिः॒ सर्वाः᳚ । \newline
35. प्र॒जाप॑ति॒रिति॑ प्र॒जा - प॒तिः॒ । \newline
36. सर्वा॑ दे॒वता॑ दे॒वताः॒ सर्वाः॒ सर्वा॑ दे॒वताः᳚ । \newline
37. दे॒वता॑ दे॒वता॑भिर् दे॒वता॑भिर् दे॒वता॑ दे॒वता॑ दे॒वता॑भिः । \newline
38. दे॒वता॑भि रे॒वैव दे॒वता॑भिर् दे॒वता॑भि रे॒व । \newline
39. ए॒वा न॑भिजित॒ मन॑भिजित मे॒वैवा न॑भिजितम् । \newline
40. अन॑भिजित म॒भ्य॑भ्य न॑भिजित॒ मन॑भिजित म॒भि । \newline
41. अन॑भिजित॒मित्यन॑भि - जि॒त॒म् । \newline
42. अ॒भि ज॑यति जय त्य॒भ्य॑भि ज॑यति । \newline
43. ज॒य॒ति॒ वा॒य॒व्य॑या वाय॒व्य॑या जयति जयति वाय॒व्य॑या । \newline
44. वा॒य॒व्य॑यो॒ पाक॑रोत्यु॒ पाक॑रोति वाय॒व्य॑या वाय॒व्य॑यो॒ पाक॑रोति । \newline
45. उ॒पाक॑रोति वा॒योर् वा॒योरु॒ पाक॑रो त्यु॒पाक॑रोति वा॒योः । \newline
46. उ॒पाक॑रो॒तीयु॑प - आक॑रोति । \newline
47. वा॒यो रे॒वैव वा॒योर् वा॒यो रे॒व । \newline
48. ए॒वैना॑ मेना मे॒वैवैना᳚म् । \newline
49. ए॒ना॒ म॒व॒रुद्ध्या॑ व॒रुद्ध्यै॑ ना मेना मव॒रुद्ध्य॑ । \newline
50. अ॒व॒रुद्ध्या ऽव॒रुद्ध्या॑ व॒रुद्ध्या । \newline
51. अ॒व॒रुद्ध्येत्य॑व - रुद्ध्य॑ । \newline
52. आ ल॑भते लभत॒ आ ल॑भते । \newline
53. ल॒भ॒त॒ आकू᳚त्या॒ आकू᳚त्यै लभते लभत॒ आकू᳚त्यै । \newline
54. आकू᳚त्यै त्वा॒ त्वा ऽऽकू᳚त्या॒ आकू᳚त्यै त्वा । \newline
55. आकू᳚त्या॒ इत्या - कू॒त्यै॒ । \newline
56. त्वा॒ कामा॑य॒ कामा॑य त्वा त्वा॒ कामा॑य । \newline
57. कामा॑य त्वा त्वा॒ कामा॑य॒ कामा॑य त्वा । \newline
58. त्वेतीति॑ त्वा॒ त्वेति॑ । \newline

\textbf{Ghana Paata } \newline

1. ई॒श्व॒रो वा॒चो वा॒च ई᳚श्व॒र ई᳚श्व॒रो वा॒चो वदि॑तो॒र् वदि॑तोर् वा॒च ई᳚श्व॒र ई᳚श्व॒रो वा॒चो वदि॑तोः । \newline
2. वा॒चो वदि॑तो॒र् वदि॑तोर् वा॒चो वा॒चो वदि॑तोः॒ सन् थ्सन्. वदि॑तोर् वा॒चो वा॒चो वदि॑तोः॒ सन्न् । \newline
3. वदि॑तोः॒ सन् थ्सन्. वदि॑तो॒र् वदि॑तोः॒ सन्. वाच॒म् ॅवाचꣳ॒॒ सन्. वदि॑तो॒र् वदि॑तोः॒ सन्. वाच᳚म् । \newline
4. सन्. वाच॒म् ॅवाचꣳ॒॒ सन् थ्सन्. वाच॒म् न न वाचꣳ॒॒ सन् थ्सन्. वाच॒म् न । \newline
5. वाच॒म् न न वाच॒म् ॅवाच॒म् न वदे॒द् वदे॒न् न वाच॒म् ॅवाच॒म् न वदे᳚त् । \newline
6. न वदे॒द् वदे॒न् न न वदे॒द् वाग् वाग् वदे॒न् न न वदे॒द् वाक् । \newline
7. वदे॒द् वाग् वाग् वदे॒द् वदे॒द् वाग् वै वै वाग् वदे॒द् वदे॒द् वाग् वै । \newline
8. वाग् वै वै वाग् वाग् वै सर॑स्वती॒ सर॑स्वती॒ वै वाग् वाग् वै सर॑स्वती । \newline
9. वै सर॑स्वती॒ सर॑स्वती॒ वै वै सर॑स्वती॒ सर॑स्वतीꣳ॒॒ सर॑स्वतीꣳ॒॒ सर॑स्वती॒ वै वै सर॑स्वती॒ सर॑स्वतीम् । \newline
10. सर॑स्वती॒ सर॑स्वतीꣳ॒॒ सर॑स्वतीꣳ॒॒ सर॑स्वती॒ सर॑स्वती॒ सर॑स्वती मे॒वैव सर॑स्वतीꣳ॒॒ सर॑स्वती॒ सर॑स्वती॒ सर॑स्वती मे॒व । \newline
11. सर॑स्वती मे॒वैव सर॑स्वतीꣳ॒॒ सर॑स्वती मे॒व स्वेन॒ स्वेनै॒व सर॑स्वतीꣳ॒॒ सर॑स्वती मे॒व स्वेन॑ । \newline
12. ए॒व स्वेन॒ स्वेनै॒वैव स्वेन॑ भाग॒धेये॑न भाग॒धेये॑न॒ स्वेनै॒वैव स्वेन॑ भाग॒धेये॑न । \newline
13. स्वेन॑ भाग॒धेये॑न भाग॒धेये॑न॒ स्वेन॒ स्वेन॑ भाग॒धेये॒ नोपोप॑ भाग॒धेये॑न॒ स्वेन॒ स्वेन॑ भाग॒धेये॒ नोप॑ । \newline
14. भा॒ग॒धेये॒ नोपोप॑ भाग॒धेये॑न भाग॒धेये॒ नोप॑ धावति धाव॒ त्युप॑ भाग॒धेये॑न भाग॒धेये॒ नोप॑ धावति । \newline
15. भा॒ग॒धेये॒नेति॑ भाग - धेये॑न । \newline
16. उप॑ धावति धाव॒ त्युपोप॑ धावति॒ सा सा धा॑व॒ त्युपोप॑ धावति॒ सा । \newline
17. धा॒व॒ति॒ सा सा धा॑वति धावति॒ सैवैव सा धा॑वति धावति॒ सैव । \newline
18. सै वैव सा सै वास्मि॑न्,नस्मिन्,ने॒व सा सैवास्मिन्न्॑ । \newline
19. ए॒वास्मि॑न्,नस्मिन्,ने॒वै वास्मि॒न्॒. वाच॒म् ॅवाच॑ मस्मिन्,ने॒वैवास्मि॒न्॒. वाच᳚म् । \newline
20. अ॒स्मि॒न्॒. वाच॒म् ॅवाच॑ मस्मिन्,नस्मि॒न्॒. वाच॑म् दधाति दधाति॒ वाच॑ मस्मिन्,नस्मि॒न्॒. वाच॑म् दधाति । \newline
21. वाच॑म् दधाति दधाति॒ वाच॒म् ॅवाच॑म् दधाति प्राजाप॒त्याम् प्रा॑जाप॒त्याम् द॑धाति॒ वाच॒म् ॅवाच॑म् दधाति प्राजाप॒त्याम् । \newline
22. द॒धा॒ति॒ प्रा॒जा॒प॒त्याम् प्रा॑जाप॒त्याम् द॑धाति दधाति प्राजाप॒त्यामा प्रा॑जाप॒त्याम् द॑धाति दधाति प्राजाप॒त्यामा । \newline
23. प्रा॒जा॒प॒त्यामा प्रा॑जाप॒त्याम् प्रा॑जाप॒त्यामा ल॑भेत लभे॒ता प्रा॑जाप॒त्याम् प्रा॑जाप॒त्यामा ल॑भेत । \newline
24. प्रा॒जा॒प॒त्यामिति॑ प्राजा - प॒त्याम् । \newline
25. आ ल॑भेत लभे॒ता ल॑भेत॒ यो यो ल॑भे॒ता ल॑भेत॒ यः । \newline
26. ल॒भे॒त॒ यो यो ल॑भेत लभेत॒ यः का॒मये॑त का॒मये॑त॒ यो ल॑भेत लभेत॒ यः का॒मये॑त । \newline
27. यः का॒मये॑त का॒मये॑त॒ यो यः का॒मये॒ता न॑भिजित॒ मन॑भिजितम् का॒मये॑त॒ यो यः का॒मये॒ता न॑भिजितम् । \newline
28. का॒मये॒ता न॑भिजित॒ मन॑भिजितम् का॒मये॑त का॒मये॒ता न॑भिजित म॒भ्य॑भ्य न॑भिजितम् का॒मये॑त का॒मये॒ता न॑भिजित म॒भि । \newline
29. अन॑भिजित म॒भ्य॑भ्य न॑भिजित॒ मन॑भिजित म॒भि ज॑येयम् जयेय म॒भ्य न॑भिजित॒ मन॑भिजित म॒भि ज॑येयम् । \newline
30. अन॑भिजित॒मित्यन॑भि - जि॒त॒म् । \newline
31. अ॒भि ज॑येयम् जयेय म॒भ्य॑भि ज॑येय॒ मितीति॑ जयेय म॒भ्य॑भि ज॑येय॒ मिति॑ । \newline
32. ज॒ये॒य॒ मितीति॑ जयेयम् जयेय॒ मिति॑ प्र॒जाप॑तिः प्र॒जाप॑ति॒ रिति॑ जयेयम् जयेय॒ मिति॑ प्र॒जाप॑तिः । \newline
33. इति॑ प्र॒जाप॑तिः प्र॒जाप॑ति॒ रितीति॑ प्र॒जाप॑तिः॒ सर्वाः॒ सर्वाः᳚ प्र॒जाप॑ति॒ रितीति॑ प्र॒जाप॑तिः॒ सर्वाः᳚ । \newline
34. प्र॒जाप॑तिः॒ सर्वाः॒ सर्वाः᳚ प्र॒जाप॑तिः प्र॒जाप॑तिः॒ सर्वा॑ दे॒वता॑ दे॒वताः॒ सर्वाः᳚ प्र॒जाप॑तिः प्र॒जाप॑तिः॒ सर्वा॑ दे॒वताः᳚ । \newline
35. प्र॒जाप॑ति॒रिति॑ प्र॒जा - प॒तिः॒ । \newline
36. सर्वा॑ दे॒वता॑ दे॒वताः॒ सर्वाः॒ सर्वा॑ दे॒वता॑ दे॒वता॑भिर् दे॒वता॑भिर् दे॒वताः॒ सर्वाः॒ सर्वा॑ दे॒वता॑ दे॒वता॑भिः । \newline
37. दे॒वता॑ दे॒वता॑भिर् दे॒वता॑भिर् दे॒वता॑ दे॒वता॑ दे॒वता॑भि रे॒वैव दे॒वता॑भिर् दे॒वता॑ दे॒वता॑ दे॒वता॑भि रे॒व । \newline
38. दे॒वता॑भि रे॒वैव दे॒वता॑भिर् दे॒वता॑भि रे॒वा न॑भिजित॒ मन॑भिजित मे॒व दे॒वता॑भिर् दे॒वता॑भि रे॒वा न॑भिजितम् । \newline
39. ए॒वा न॑भिजित॒ मन॑भिजित मे॒वैवा न॑भिजित म॒भ्य॑भ्य न॑भिजित मे॒वैवा न॑भिजित म॒भि । \newline
40. अन॑भिजित म॒भ्य॑भ्य न॑भिजित॒ मन॑भिजित म॒भि ज॑यति जय त्य॒भ्य न॑भिजित॒ मन॑भिजित म॒भि ज॑यति । \newline
41. अन॑भिजित॒मित्यन॑भि - जि॒त॒म् । \newline
42. अ॒भि ज॑यति जय त्य॒भ्य॑भि ज॑यति वाय॒व्य॑या वाय॒व्य॑या जय त्य॒भ्य॑भि ज॑यति वाय॒व्य॑या । \newline
43. ज॒य॒ति॒ वा॒य॒व्य॑या वाय॒व्य॑या जयति जयति वाय॒व्य॑ यो॒पाक॑रो त्यु॒पाक॑रोति वाय॒व्य॑या जयति जयति वाय॒व्य॑ यो॒पाक॑रोति । \newline
44. वा॒य॒व्य॑ यो॒पाक॑रो त्यु॒पाक॑रोति वाय॒व्य॑या वाय॒व्य॑ यो॒पाक॑रोति वा॒योर् वा॒यो रु॒पाक॑रोति वाय॒व्य॑या वाय॒व्य॑यो॒ पाक॑रोति वा॒योः । \newline
45. उ॒पाक॑रोति वा॒योर् वा॒यो रु॒पाक॑रो त्यु॒पाक॑रोति वा॒यो रे॒वैव वा॒योरु॒पाक॑रो त्यु॒पाक॑रोति वा॒यो रे॒व । \newline
46. उ॒पाक॑रो॒तीयु॑प - आक॑रोति । \newline
47. वा॒यो रे॒वैव वा॒योर् वा॒यो रे॒वैना॑ मेना मे॒व वा॒योर् वा॒यो रे॒वै ना᳚म् । \newline
48. ए॒वैना॑ मेना मे॒वैवैना॑ मव॒रुद्ध्या॑ व॒रुद्ध्यै॑ना मे॒वैवैना॑ मव॒रुद्ध्य॑ । \newline
49. ए॒ना॒ म॒व॒रुद्ध्या॑ व॒रुद्ध्यै॑ना मेना मव॒रुद्ध्या ऽव॒रुद्ध्यै॑ना मेना मव॒रुद्ध्या । \newline
50. अ॒व॒रुद्ध्या ऽव॒रुद्ध्या॑ व॒रुद्ध्या ल॑भते लभत॒ आ ऽव॒रुद्ध्या॑ व॒रुद्ध्या ल॑भते । \newline
51. अ॒व॒रुद्ध्येत्य॑व - रुद्ध्य॑ । \newline
52. आ ल॑भते लभत॒ आ ल॑भत॒ आकू᳚त्या॒ आकू᳚त्यै लभत॒ आ ल॑भत॒ आकू᳚त्यै । \newline
53. ल॒भ॒त॒ आकू᳚त्या॒ आकू᳚त्यै लभते लभत॒ आकू᳚त्यै त्वा॒ त्वा ऽऽकू᳚त्यै लभते लभत॒ आकू᳚त्यै त्वा । \newline
54. आकू᳚त्यै त्वा॒ त्वा ऽऽकू᳚त्या॒ आकू᳚त्यै त्वा॒ कामा॑य॒ कामा॑य॒ त्वा ऽऽकू᳚त्या॒ आकू᳚त्यै त्वा॒ कामा॑य । \newline
55. आकू᳚त्या॒ इत्या - कू॒त्यै॒ । \newline
56. त्वा॒ कामा॑य॒ कामा॑य त्वा त्वा॒ कामा॑य त्वा त्वा॒ कामा॑य त्वा त्वा॒ कामा॑य त्वा । \newline
57. कामा॑य त्वा त्वा॒ कामा॑य॒ कामा॑य॒ त्वेतीति॑ त्वा॒ कामा॑य॒ कामा॑य॒ त्वेति॑ । \newline
58. त्वेतीति॑ त्वा॒ त्वेत्या॑ हा॒हेति॑ त्वा॒ त्वेत्या॑ह । \newline
\pagebreak
\markright{ TS 3.4.3.5  \hfill https://www.vedavms.in \hfill}

\section{ TS 3.4.3.5 }

\textbf{TS 3.4.3.5 } \newline
\textbf{Samhita Paata} \newline

त्या॑ह यथाय॒जुरे॒वैतत् कि॑क्किटा॒कारं॑ जुहोति किक्किटाका॒रेण॒ वै ग्रा॒म्याः प॒शवो॑ रमन्ते॒ प्राऽऽ*र॒ण्याः प॑तन्ति॒ यत् कि॑क्किटा॒कारं॑ जु॒होति॑ ग्रा॒म्याणां᳚ पशू॒नां धृत्यै॒ पर्य॑ग्नौ क्रि॒यमा॑णे जुहोति॒ जीव॑न्तीमे॒वैनाꣳ॑ सुव॒र्गं ॅलो॒कं ग॑मयति॒ त्वं तु॒रीया॑ व॒शिनी॑ व॒शाऽसीत्या॑ह देव॒त्रैवैनां᳚ गमयति स॒त्याः स॑न्तु॒ यज॑मानस्य॒ कामा॒ इत्या॑है॒ष वै कामो॒ -  [  ] \newline

\textbf{Pada Paata} \newline

इति॑ । आ॒ह॒ । य॒था॒य॒जुरिति॑ यथा - य॒जुः । ए॒व । ए॒तत् । कि॒क्कि॒टा॒कार॒मिति॑ किक्किटा - कार᳚म् । जु॒हो॒ति॒ । कि॒क्कि॒टा॒का॒रेणेति॑ किक्किटा - का॒रेण॑ । वै । ग्रा॒म्याः । प॒शवः॑ । र॒म॒न्ते॒ । प्रेति॑ । आ॒र॒ण्याः । प॒त॒न्ति॒ । यत् । कि॒क्कि॒टा॒कार॒मिति॑ किक्किटा - कार᳚म् । जु॒होति॑ । ग्रा॒म्याणा᳚म् । प॒शू॒नाम् । धृत्यै᳚ । पर्य॑ग्ना॒विति॒ परि॑-अ॒ग्नौ॒ । क्रि॒यमा॑णे । जु॒हो॒ति॒ । जीव॑न्तीम् । ए॒व । ए॒ना॒म् । सु॒व॒र्गमिति॑ सुवः - गम् । लो॒कम् । ग॒म॒य॒ति॒ । त्वम् । तु॒रीया᳚ । व॒शिनी᳚ । व॒शा । अ॒सि॒ । इति॑ । आ॒ह॒ । दे॒व॒त्रेति॑ देव - त्रा । ए॒व । ए॒ना॒म् । ग॒म॒य॒ति॒ । स॒त्याः । स॒न्तु॒ । यज॑मानस्य । कामाः᳚ । इति॑ । आ॒ह॒ । ए॒षः । वै । कामः॑ ।  \newline


\textbf{Krama Paata} \newline

इत्या॑ह । आ॒ह॒ य॒था॒य॒जुः । य॒था॒य॒जुरे॒व । य॒था॒य॒जुरिति॑ यथा - य॒जुः । ए॒वैतत् । ए॒तत् कि॑क्किटा॒कार᳚म् । कि॒क्कि॒टा॒कार॑म् जुहोति । कि॒क्कि॒टा॒कार॒मिति॑ किक्किटा - कार᳚म् । जु॒हो॒ति॒ कि॒क्कि॒टा॒का॒रेण॑ । कि॒क्कि॒टा॒का॒रेण॒ वै । कि॒क्कि॒टा॒का॒रेणेति॑ किक्किटा - का॒रेण॑ । वै ग्रा॒म्याः । ग्रा॒म्याः प॒शवः॑ । प॒शवो॑ रमन्ते । र॒म॒न्ते॒ प्र । प्रार॒ण्याः । आ॒र॒ण्याः प॑तन्ति । प॒त॒न्ति॒ यत् । यत् कि॑क्किटा॒कार᳚म् । कि॒क्कि॒टा॒कार॑म् जु॒होति॑ । कि॒क्कि॒टा॒कार॒मिति॑ किक्किटा - कार᳚म् । जु॒होति॑ ग्रा॒म्याणा᳚म् । ग्रा॒म्याणा᳚म् पशू॒नाम् । प॒शू॒नाम् धृत्यै᳚ । धृत्यै॒ पर्य॑ग्नौ । पर्य॑ग्नौ क्रि॒यमा॑णे । पर्य॑ग्ना॒विति॒ परि॑ - अ॒ग्नौ॒ । क्रि॒यमा॑णे जुहोति । जु॒हो॒ति॒ जीव॑न्तीम् । जीव॑न्तीमे॒व । ए॒वैना᳚म् । ए॒नाꣳ॒॒ सु॒व॒र्गम् । सु॒व॒र्गं ॅलो॒कम् । सु॒व॒र्गमिति॑ सुवः - गम् । लो॒कम् ग॑मयति । ग॒म॒य॒ति॒ त्वम् । त्वम् तु॒रीया᳚ । तु॒रीया॑ व॒शिनी᳚ । व॒शिनी॑ व॒शा । व॒शा ऽसि॑ । अ॒सीति॑ । इत्या॑ह । आ॒ह॒ दे॒व॒त्रा । दे॒व॒त्रैव । दे॒व॒त्रेति॑ देव - त्रा । ए॒वैना᳚म् । ए॒ना॒म् ग॒म॒य॒ति॒ । ग॒म॒य॒ति॒ स॒त्याः । स॒त्याः स॑न्तु । स॒न्तु॒ यज॑मानस्य । यज॑मानस्य॒ कामाः᳚ । कामा॒ इति॑ । इत्या॑ह । आ॒है॒षः । ए॒ष वै । वै कामः॑ । कामो॒ यज॑मानस्य \newline

\textbf{Jatai Paata} \newline

1. इत्या॑ हा॒हे तीत्या॑ह । \newline
2. आ॒ह॒ य॒था॒य॒जुर् य॑थाय॒जु रा॑हाह यथाय॒जुः । \newline
3. य॒था॒य॒जु रे॒वैव य॑थाय॒जुर् य॑थाय॒जु रे॒व । \newline
4. य॒था॒य॒जुरिति॑ यथा - य॒जुः । \newline
5. ए॒वैत दे॒त दे॒वैवैतत् । \newline
6. ए॒तत् कि॑क्किटा॒कार॑म् किक्किटा॒कार॑ मे॒त दे॒तत् कि॑क्किटा॒कार᳚म् । \newline
7. कि॒क्कि॒टा॒कार॑म् जुहोति जुहोति किक्किटा॒कार॑म् किक्किटा॒कार॑म् जुहोति । \newline
8. कि॒क्कि॒टा॒कार॒मिति॑ किक्किटा - कार᳚म् । \newline
9. जु॒हो॒ति॒ कि॒क्कि॒टा॒का॒रेण॑ किक्किटाका॒रेण॑ जुहोति जुहोति किक्किटाका॒रेण॑ । \newline
10. कि॒क्कि॒टा॒का॒रेण॒ वै वै कि॑क्किटाका॒रेण॑ किक्किटाका॒रेण॒ वै । \newline
11. कि॒क्कि॒टा॒का॒रेणेति॑ किक्किटा - का॒रेण॑ । \newline
12. वै ग्रा॒म्या ग्रा॒म्या वै वै ग्रा॒म्याः । \newline
13. ग्रा॒म्याः प॒शवः॑ प॒शवो᳚ ग्रा॒म्या ग्रा॒म्याः प॒शवः॑ । \newline
14. प॒शवो॑ रमन्ते रमन्ते प॒शवः॑ प॒शवो॑ रमन्ते । \newline
15. र॒म॒न्ते॒ प्र प्र र॑मन्ते रमन्ते॒ प्र । \newline
16. प्रार॒ण्या आ॑र॒ण्याः प्र प्रार॒ण्याः । \newline
17. आ॒र॒ण्याः प॑तन्ति पत न्त्यार॒ण्या आ॑र॒ण्याः प॑तन्ति । \newline
18. प॒त॒न्ति॒ यद् यत् प॑तन्ति पतन्ति॒ यत् । \newline
19. यत् कि॑क्किटा॒कार॑म् किक्किटा॒कारं॒ ॅयद् यत् कि॑क्किटा॒कार᳚म् । \newline
20. कि॒क्कि॒टा॒कार॑म् जु॒होति॑ जु॒होति॑ किक्किटा॒कार॑म् किक्किटा॒कार॑म् जु॒होति॑ । \newline
21. कि॒क्कि॒टा॒कार॒मिति॑ किक्किटा - कार᳚म् । \newline
22. जु॒होति॑ ग्रा॒म्याणा᳚म् ग्रा॒म्याणा᳚म् जु॒होति॑ जु॒होति॑ ग्रा॒म्याणा᳚म् । \newline
23. ग्रा॒म्याणा᳚म् पशू॒नाम् प॑शू॒नाम् ग्रा॒म्याणा᳚म् ग्रा॒म्याणा᳚म् पशू॒नाम् । \newline
24. प॒शू॒नाम् धृत्यै॒ धृत्यै॑ पशू॒नाम् प॑शू॒नाम् धृत्यै᳚ । \newline
25. धृत्यै॒ पर्य॑ग्नौ॒ पर्य॑ग्नौ॒ धृत्यै॒ धृत्यै॒ पर्य॑ग्नौ । \newline
26. पर्य॑ग्नौ क्रि॒यमा॑णे क्रि॒यमा॑णे॒ पर्य॑ग्नौ॒ पर्य॑ग्नौ क्रि॒यमा॑णे । \newline
27. पर्य॑ग्ना॒विति॒ परि॑ - अ॒ग्नौ॒ । \newline
28. क्रि॒यमा॑णे जुहोति जुहोति क्रि॒यमा॑णे क्रि॒यमा॑णे जुहोति । \newline
29. जु॒हो॒ति॒ जीव॑न्ती॒म् जीव॑न्तीम् जुहोति जुहोति॒ जीव॑न्तीम् । \newline
30. जीव॑न्ती मे॒वैव जीव॑न्ती॒म् जीव॑न्ती मे॒व । \newline
31. ए॒वैना॑ मेना मे॒वैवैना᳚म् । \newline
32. ए॒नाꣳ॒॒ सु॒व॒र्गꣳ सु॑व॒र्ग मे॑ना मेनाꣳ सुव॒र्गम् । \newline
33. सु॒व॒र्गम् ॅलो॒कम् ॅलो॒कꣳ सु॑व॒र्गꣳ सु॑व॒र्गम् ॅलो॒कम् । \newline
34. सु॒व॒र्गमिति॑ सुवः - गम् । \newline
35. लो॒कम् ग॑मयति गमयति लो॒कम् ॅलो॒कम् ग॑मयति । \newline
36. ग॒म॒य॒ति॒ त्वम् त्वम् ग॑मयति गमयति॒ त्वम् । \newline
37. त्वम् तु॒रीया॑ तु॒रीया॒ त्वम् त्वम् तु॒रीया᳚ । \newline
38. तु॒रीया॑ व॒शिनी॑ व॒शिनी॑ तु॒रीया॑ तु॒रीया॑ व॒शिनी᳚ । \newline
39. व॒शिनी॑ व॒शा व॒शा व॒शिनी॑ व॒शिनी॑ व॒शा । \newline
40. व॒शा ऽस्य॑सि व॒शा व॒शा ऽसि॑ । \newline
41. अ॒सी तीत्य॑ स्य॒सी ति॑ । \newline
42. इत्या॑ हा॒हे तीत्या॑ ह । \newline
43. आ॒ह॒ दे॒व॒त्रा दे॑व॒त्रा ऽऽहा॑ह देव॒त्रा । \newline
44. दे॒व॒त्रै वैव दे॑व॒त्रा दे॑व॒त्रैव । \newline
45. दे॒व॒त्रेति॑ देव - त्रा । \newline
46. ए॒वैना॑ मेना मे॒वैवैना᳚म् । \newline
47. ए॒ना॒म् ग॒म॒य॒ति॒ ग॒म॒य॒ त्ये॒ना॒ मे॒ना॒म् ग॒म॒य॒ति॒ । \newline
48. ग॒म॒य॒ति॒ स॒त्याः स॒त्या ग॑मयति गमयति स॒त्याः । \newline
49. स॒त्याः स॑न्तु सन्तु स॒त्याः स॒त्याः स॑न्तु । \newline
50. स॒न्तु॒ यज॑मानस्य॒ यज॑मानस्य सन्तु सन्तु॒ यज॑मानस्य । \newline
51. यज॑मानस्य॒ कामाः॒ कामा॒ यज॑मानस्य॒ यज॑मानस्य॒ कामाः᳚ । \newline
52. कामा॒ इतीति॒ कामाः॒ कामा॒ इति॑ । \newline
53. इत्या॑ हा॒हे तीत्या॑ह । \newline
54. आ॒है॒ष ए॒ष आ॑हा है॒षः । \newline
55. ए॒ष वै वा ए॒ष ए॒ष वै । \newline
56. वै कामः॒ कामो॒ वै वै कामः॑ । \newline
57. कामो॒ यज॑मानस्य॒ यज॑मानस्य॒ कामः॒ कामो॒ यज॑मानस्य । \newline

\textbf{Ghana Paata } \newline

1. इत्या॑ हा॒हेतीत्या॑ह यथाय॒जुर् य॑थाय॒जु रा॒हेतीत्या॑ह यथाय॒जुः । \newline
2. आ॒ह॒ य॒था॒य॒जुर् य॑थाय॒जु रा॑हाह यथाय॒जु रे॒वैव य॑थाय॒जु रा॑हाह यथाय॒जु रे॒व । \newline
3. य॒था॒य॒जु रे॒वैव य॑थाय॒जुर् य॑थाय॒जु रे॒वैत दे॒तदे॒व य॑थाय॒जुर् य॑थाय॒जु रे॒वैतत् । \newline
4. य॒था॒य॒जुरिति॑ यथा - य॒जुः । \newline
5. ए॒वैत दे॒त दे॒वैवैतत् कि॑क्किटा॒कार॑म् किक्किटा॒कार॑ मे॒त दे॒वैवैतत् कि॑क्किटा॒कार᳚म् । \newline
6. ए॒तत् कि॑क्किटा॒कार॑म् किक्किटा॒कार॑ मे॒त दे॒तत् कि॑क्किटा॒कार॑म् जुहोति जुहोति किक्किटा॒कार॑ मे॒त दे॒तत् कि॑क्किटा॒कार॑म् जुहोति । \newline
7. कि॒क्कि॒टा॒कार॑म् जुहोति जुहोति किक्किटा॒कार॑म् किक्किटा॒कार॑म् जुहोति किक्किटाका॒रेण॑ किक्किटाका॒रेण॑ जुहोति किक्किटा॒कार॑म् किक्किटा॒कार॑म् जुहोति किक्किटाका॒रेण॑ । \newline
8. कि॒क्कि॒टा॒कार॒मिति॑ किक्किटा - कार᳚म् । \newline
9. जु॒हो॒ति॒ कि॒क्कि॒टा॒का॒रेण॑ किक्किटाका॒रेण॑ जुहोति जुहोति किक्किटाका॒रेण॒ वै वै कि॑क्किटाका॒रेण॑ जुहोति जुहोति किक्किटाका॒रेण॒ वै । \newline
10. कि॒क्कि॒टा॒का॒रेण॒ वै वै कि॑क्किटाका॒रेण॑ किक्किटाका॒रेण॒ वै ग्रा॒म्या ग्रा॒म्या वै कि॑क्किटाका॒रेण॑ किक्किटाका॒रेण॒ वै ग्रा॒म्याः । \newline
11. कि॒क्कि॒टा॒का॒रेणेति॑ किक्किटा - का॒रेण॑ । \newline
12. वै ग्रा॒म्या ग्रा॒म्या वै वै ग्रा॒म्याः प॒शवः॑ प॒शवो᳚ ग्रा॒म्या वै वै ग्रा॒म्याः प॒शवः॑ । \newline
13. ग्रा॒म्याः प॒शवः॑ प॒शवो᳚ ग्रा॒म्या ग्रा॒म्याः प॒शवो॑ रमन्ते रमन्ते प॒शवो᳚ ग्रा॒म्या ग्रा॒म्याः प॒शवो॑ रमन्ते । \newline
14. प॒शवो॑ रमन्ते रमन्ते प॒शवः॑ प॒शवो॑ रमन्ते॒ प्र प्र र॑मन्ते प॒शवः॑ प॒शवो॑ रमन्ते॒ प्र । \newline
15. र॒म॒न्ते॒ प्र प्र र॑मन्ते रमन्ते॒ प्रार॒ण्या आ॑र॒ण्याः प्र र॑मन्ते रमन्ते॒ प्रार॒ण्याः । \newline
16. प्रार॒ण्या आ॑र॒ण्याः प्र प्रार॒ण्याः प॑तन्ति पत न्त्यार॒ण्याः प्र प्रार॒ण्याः प॑तन्ति । \newline
17. आ॒र॒ण्याः प॑तन्ति पत न्त्यार॒ण्या आ॑र॒ण्याः प॑तन्ति॒ यद् यत् प॑त न्त्यार॒ण्या आ॑र॒ण्याः प॑तन्ति॒ यत् । \newline
18. प॒त॒न्ति॒ यद् यत् प॑तन्ति पतन्ति॒ यत् कि॑क्किटा॒कार॑म् किक्किटा॒कार॒म् ॅयत् प॑तन्ति पतन्ति॒ यत् कि॑क्किटा॒कार᳚म् । \newline
19. यत् कि॑क्किटा॒कार॑म् किक्किटा॒कार॒म् ॅयद् यत् कि॑क्किटा॒कार॑म् जु॒होति॑ जु॒होति॑ किक्किटा॒कार॒म् ॅयद् यत् कि॑क्किटा॒कार॑म् जु॒होति॑ । \newline
20. कि॒क्कि॒टा॒कार॑म् जु॒होति॑ जु॒होति॑ किक्किटा॒कार॑म् किक्किटा॒कार॑म् जु॒होति॑ ग्रा॒म्याणा᳚म् ग्रा॒म्याणा᳚म् जु॒होति॑ किक्किटा॒कार॑म् किक्किटा॒कार॑म् जु॒होति॑ ग्रा॒म्याणा᳚म् । \newline
21. कि॒क्कि॒टा॒कार॒मिति॑ किक्किटा - कार᳚म् । \newline
22. जु॒होति॑ ग्रा॒म्याणा᳚म् ग्रा॒म्याणा᳚म् जु॒होति॑ जु॒होति॑ ग्रा॒म्याणा᳚म् पशू॒नाम् प॑शू॒नाम् ग्रा॒म्याणा᳚म् जु॒होति॑ 
जु॒होति॑ ग्रा॒म्याणा᳚म् पशू॒नाम् । \newline
23. ग्रा॒म्याणा᳚म् पशू॒नाम् प॑शू॒नाम् ग्रा॒म्याणा᳚म् ग्रा॒म्याणा᳚म् पशू॒नाम् धृत्यै॒ धृत्यै॑ पशू॒नाम् ग्रा॒म्याणा᳚म् ग्रा॒म्याणा᳚म् पशू॒नाम् धृत्यै᳚ । \newline
24. प॒शू॒नाम् धृत्यै॒ धृत्यै॑ पशू॒नाम् प॑शू॒नाम् धृत्यै॒ पर्य॑ग्नौ॒ पर्य॑ग्नौ॒ धृत्यै॑ पशू॒नाम् प॑शू॒नाम् धृत्यै॒ पर्य॑ग्नौ । \newline
25. धृत्यै॒ पर्य॑ग्नौ॒ पर्य॑ग्नौ॒ धृत्यै॒ धृत्यै॒ पर्य॑ग्नौ क्रि॒यमा॑णे क्रि॒यमा॑णे॒ पर्य॑ग्नौ॒ धृत्यै॒ धृत्यै॒ पर्य॑ग्नौ क्रि॒यमा॑णे । \newline
26. पर्य॑ग्नौ क्रि॒यमा॑णे क्रि॒यमा॑णे॒ पर्य॑ग्नौ॒ पर्य॑ग्नौ क्रि॒यमा॑णे जुहोति जुहोति क्रि॒यमा॑णे॒ पर्य॑ग्नौ॒ पर्य॑ग्नौ क्रि॒यमा॑णे जुहोति । \newline
27. पर्य॑ग्ना॒विति॒ परि॑ - अ॒ग्नौ॒ । \newline
28. क्रि॒यमा॑णे जुहोति जुहोति क्रि॒यमा॑णे क्रि॒यमा॑णे जुहोति॒ जीव॑न्ती॒म् जीव॑न्तीम् जुहोति क्रि॒यमा॑णे क्रि॒यमा॑णे जुहोति॒ जीव॑न्तीम् । \newline
29. जु॒हो॒ति॒ जीव॑न्ती॒म् जीव॑न्तीम् जुहोति जुहोति॒ जीव॑न्ती मे॒वैव जीव॑न्तीम् जुहोति जुहोति॒ जीव॑न्तीमे॒व । \newline
30. जीव॑न्ती मे॒वैव जीव॑न्ती॒म् जीव॑न्ती मे॒वैना॑ मेना मे॒व जीव॑न्ती॒म् जीव॑न्ती मे॒वैना᳚म् । \newline
31. ए॒वैना॑ मेना मे॒वैवैनाꣳ॑ सुव॒र्गꣳ सु॑व॒र्ग मे॑ना मे॒वैवैनाꣳ॑ सुव॒र्गम् । \newline
32. ए॒नाꣳ॒॒ सु॒व॒र्गꣳ सु॑व॒र्ग मे॑ना मेनाꣳ सुव॒र्गम् ॅलो॒कम् ॅलो॒कꣳ सु॑व॒र्ग मे॑ना मेनाꣳ सुव॒र्गम् ॅलो॒कम् । \newline
33. सु॒व॒र्गम् ॅलो॒कम् ॅलो॒कꣳ सु॑व॒र्गꣳ सु॑व॒र्गम् ॅलो॒कम् ग॑मयति गमयति लो॒कꣳ सु॑व॒र्गꣳ सु॑व॒र्गम् ॅलो॒कम् ग॑मयति । \newline
34. सु॒व॒र्गमिति॑ सुवः - गम् । \newline
35. लो॒कम् ग॑मयति गमयति लो॒कम् ॅलो॒कम् ग॑मयति॒ त्वम् त्वम् ग॑मयति लो॒कम् ॅलो॒कम् ग॑मयति॒ त्वम् । \newline
36. ग॒म॒य॒ति॒ त्वम् त्वम् ग॑मयति गमयति॒ त्वम् तु॒रीया॑ तु॒रीया॒ त्वम् ग॑मयति गमयति॒ त्वम् तु॒रीया᳚ । \newline
37. त्वम् तु॒रीया॑ तु॒रीया॒ त्वम् त्वम् तु॒रीया॑ व॒शिनी॑ व॒शिनी॑ तु॒रीया॒ त्वम् त्वम् तु॒रीया॑ व॒शिनी᳚ । \newline
38. तु॒रीया॑ व॒शिनी॑ व॒शिनी॑ तु॒रीया॑ तु॒रीया॑ व॒शिनी॑ व॒शा व॒शा व॒शिनी॑ तु॒रीया॑ तु॒रीया॑ व॒शिनी॑ व॒शा । \newline
39. व॒शिनी॑ व॒शा व॒शा व॒शिनी॑ व॒शिनी॑ व॒शा ऽस्य॑सि व॒शा व॒शिनी॑ व॒शिनी॑ व॒शा ऽसि॑ । \newline
40. व॒शा ऽस्य॑सि व॒शा व॒शा ऽसीती त्य॑सि व॒शा व॒शा ऽसीति॑ । \newline
41. अ॒सीती त्य॑स्य॒सी त्या॑हा॒हे त्य॑स्य॒सी त्या॑ह । \newline
42. इत्या॑हा॒हेती त्या॑ह देव॒त्रा दे॑व॒त्रा ऽऽहेतीत्या॑ह देव॒त्रा । \newline
43. आ॒ह॒ दे॒व॒त्रा दे॑व॒त्रा ऽऽहा॑ह देव॒त्रैवैव दे॑व॒त्रा ऽऽहा॑ह देव॒त्रैव । \newline
44. दे॒व॒त्रैवैव दे॑व॒त्रा दे॑व॒त्रैवैना॑ मेना मे॒व दे॑व॒त्रा दे॑व॒त्रैवैना᳚म् । \newline
45. दे॒व॒त्रेति॑ देव - त्रा । \newline
46. ए॒वैना॑ मेना मे॒वैवैना᳚म् गमयति गमय त्येना मे॒वैवैना᳚म् गमयति । \newline
47. ए॒ना॒म् ग॒म॒य॒ति॒ ग॒म॒य॒ त्ये॒ना॒ मे॒ना॒म् ग॒म॒य॒ति॒ स॒त्याः स॒त्या ग॑मय त्येना मेनाम् गमयति स॒त्याः । \newline
48. ग॒म॒य॒ति॒ स॒त्याः स॒त्या ग॑मयति गमयति स॒त्याः स॑न्तु सन्तु स॒त्या ग॑मयति गमयति स॒त्याः स॑न्तु । \newline
49. स॒त्याः स॑न्तु सन्तु स॒त्याः स॒त्याः स॑न्तु॒ यज॑मानस्य॒ यज॑मानस्य सन्तु स॒त्याः स॒त्याः स॑न्तु॒ यज॑मानस्य । \newline
50. स॒न्तु॒ यज॑मानस्य॒ यज॑मानस्य सन्तु सन्तु॒ यज॑मानस्य॒ कामाः॒ कामा॒ यज॑मानस्य सन्तु सन्तु॒ यज॑मानस्य॒ कामाः᳚ । \newline
51. यज॑मानस्य॒ कामाः॒ कामा॒ यज॑मानस्य॒ यज॑मानस्य॒ कामा॒ इतीति॒ कामा॒ यज॑मानस्य॒ यज॑मानस्य॒ कामा॒ इति॑ । \newline
52. कामा॒ इतीति॒ कामाः॒ कामा॒ इत्या॑हा॒हेति॒ कामाः॒ कामा॒ इत्या॑ह । \newline
53. इत्या॑हा॒हेती त्या॑है॒ष ए॒ष आ॒हेतीत्या॑ है॒षः । \newline
54. आ॒है॒ष ए॒ष आ॑हा है॒ष वैवा ए॒ष आ॑हा है॒षवै । \newline
55. ए॒ष वै वा ए॒ष ए॒ष वै कामः॒ कामो॒ वा ए॒ष ए॒ष वै कामः॑ । \newline
56. वै कामः॒ कामो॒ वै वै कामो॒ यज॑मानस्य॒ यज॑मानस्य॒ कामो॒ वै वै कामो॒ यज॑मानस्य । \newline
57. कामो॒ यज॑मानस्य॒ यज॑मानस्य॒ कामः॒ कामो॒ यज॑मानस्य॒ यद् यद् यज॑मानस्य॒ कामः॒ कामो॒ यज॑मानस्य॒ यत् । \newline
\pagebreak
\markright{ TS 3.4.3.6  \hfill https://www.vedavms.in \hfill}

\section{ TS 3.4.3.6 }

\textbf{TS 3.4.3.6 } \newline
\textbf{Samhita Paata} \newline

यज॑मानस्य॒ यदना᳚र्त उ॒दृचं॒ गच्छ॑ति॒ तस्मा॑दे॒वमा॑हा॒ऽजाऽसि॑ रयि॒ष्ठेत्या॑है॒ ष्वे॑वैनां᳚ ॅलो॒केषु॒ प्रति॑ष्ठापयति दि॒वि ते॑ बृ॒हद्भा इत्या॑ह सुव॒र्ग ए॒वास्मै॑ लो॒के ज्योति॑र् दधाति॒ तन्तुं॑ त॒न्वन् रज॑सो भा॒नुमन्वि॒हीत्या॑हे॒माने॒वास्मै॑ लो॒कान् ज्योति॑ष्मतः करोत्यनुल्ब॒णं ॅव॑यत॒ जोगु॑वा॒मप॒ इत्या॑-  [  ] \newline

\textbf{Pada Paata} \newline

यज॑मानस्य । यत् । अना᳚र्तः । उ॒दृच॒मित्यु॑त् - ऋच᳚म् । गच्छ॑ति । तस्मा᳚त् । ए॒वम् । आ॒ह॒ । अ॒जा । अ॒सि॒ । र॒यि॒ष्ठेति॑ रयि - स्था । इति॑ । आ॒ह॒ । ए॒षु । ए॒व । ए॒ना॒म् । लो॒केषु॑ । प्रतीति॑ । स्था॒प॒य॒ति॒ । दि॒वि । ते॒ । बृ॒हत् । भाः । इति॑ । आ॒ह॒ । सु॒व॒र्ग इति॑ सुवः- गे । ए॒व । अ॒स्मै॒ । लो॒के । ज्योतिः॑ । द॒धा॒ति॒ । तन्तु᳚म् । त॒न्वन्न् । रज॑सः । भा॒नुम् । अन्विति॑ । इ॒हि॒ । इति॑ । आ॒ह॒ । इ॒मान् । ए॒व । अ॒स्मै॒ । लो॒कान् । ज्योति॑ष्मतः । क॒रो॒ति॒ । अ॒नु॒ल्ब॒णम् । व॒य॒त॒ । जोगु॑वाम् । अपः॑ । इति॑ ।  \newline


\textbf{Krama Paata} \newline

यज॑मानस्य॒ यत् । यदना᳚र्तः । अना᳚र्त उ॒दृच᳚म् । उ॒दृच॒म् गच्छ॑ति । उ॒दृच॒मित्यु॑त् - ऋच᳚म् । गच्छ॑ति॒ तस्मा᳚त् । तस्मा॑दे॒वम् । ए॒वमा॑ह । आ॒हा॒जा । अ॒जा ऽसि॑ । अ॒सि॒ र॒यि॒ष्ठा । र॒यि॒ष्ठेति॑ । र॒यि॒ष्ठेति॑ रयि - स्था । इत्या॑ह । आ॒है॒षु । ए॒ष्वे॑व । ए॒वैना᳚म् । ए॒नां॒ ॅलो॒केषु॑ । लो॒केषु॒ प्रति॑ । प्रति॑ ष्ठापयति । स्था॒प॒य॒ति॒ दि॒वि । दि॒वि ते᳚ । ते॒ बृ॒हत् । बृ॒हद् भाः । भा इति॑ । इत्या॑ह । आ॒ह॒ सु॒व॒र्गे । सु॒व॒र्ग ए॒व । सु॒व॒र्ग इति॑ सुवः - गे । ए॒वास्मै᳚ । अ॒स्मै॒ लो॒के । लो॒के ज्योतिः॑ । ज्योति॑र् दधाति । द॒धा॒ति॒ तन्तु᳚म् । तन्तु॑म् त॒न्वन्न् । त॒न्वन् रज॑सः । रज॑सो भा॒नुम् । भा॒नुमनु॑ । अन्वि॒हि । इ॒हीति॑ । इत्या॑ह । आ॒हे॒मान् । इ॒माने॒व । ए॒वास्मै᳚ । अ॒स्मै॒ लो॒कान् । लो॒कान् ज्योति॑ष्मतः । ज्योति॑ष्मतः करोति । क॒रो॒त्य॒नु॒ल्ब॒णम् । अ॒नु॒ल्ब॒णं ॅव॑यत । व॒य॒त॒ जोगु॑वाम् । जोगु॑वा॒मपः॑ । अप॒ इति॑ । इत्या॑ह \newline

\textbf{Jatai Paata} \newline

1. यज॑मानस्य॒ यद् यद् यज॑मानस्य॒ यज॑मानस्य॒ यत् । \newline
2. यद ना॒र्तो ऽना᳚र्तो॒ यद् यद ना᳚र्तः । \newline
3. अना᳚र्त उ॒दृच॑ मु॒दृच॒ मना॒र्तो ऽना᳚र्त उ॒दृच᳚म् । \newline
4. उ॒दृच॒म् गच्छ॑ति॒ गच्छ॑ त्यु॒दृच॑ मु॒दृच॒म् गच्छ॑ति । \newline
5. उ॒दृच॒मित्यु॑त् - ऋच᳚म् । \newline
6. गच्छ॑ति॒ तस्मा॒त् तस्मा॒द् गच्छ॑ति॒ गच्छ॑ति॒ तस्मा᳚त् । \newline
7. तस्मा॑ दे॒व मे॒वम् तस्मा॒त् तस्मा॑ दे॒वम् । \newline
8. ए॒व मा॑हाहै॒व मे॒व मा॑ह । \newline
9. आ॒हा॒जा ऽजा ऽऽहा॑हा॒जा । \newline
10. अ॒जा ऽस्य॑ स्य॒जा ऽजा ऽसि॑ । \newline
11. अ॒सि॒ र॒यि॒ष्ठा र॑यि॒ष्ठा ऽस्य॑सि रयि॒ष्ठा । \newline
12. र॒यि॒ष्ठेतीति॑ रयि॒ष्ठा र॑यि॒ष्ठेति॑ । \newline
13. र॒यि॒ष्ठेति॑ रयि - स्था । \newline
14. इत्या॑ हा॒हे तीत्या॑ह । \newline
15. आ॒है॒ष्वे᳚(1॒)ष्वा॑हाहै॒षु । \newline
16. ए॒ष्वे॑ वैवै ष्वे᳚(1॒)ष्वे॑व । \newline
17. ए॒वैना॑ मेना मे॒वैवैना᳚म् । \newline
18. ए॒ना॒म् ॅलो॒केषु॑ लो॒के ष्वे॑ना मेनाम् ॅलो॒केषु॑ । \newline
19. लो॒केषु॒ प्रति॒ प्रति॑ लो॒केषु॑ लो॒केषु॒ प्रति॑ । \newline
20. प्रति॑ ष्ठापयति स्थापयति॒ प्रति॒ प्रति॑ ष्ठापयति । \newline
21. स्था॒प॒य॒ति॒ दि॒वि दि॒वि स्था॑पयति स्थापयति दि॒वि । \newline
22. दि॒वि ते॑ ते दि॒वि दि॒वि ते᳚ । \newline
23. ते॒ बृ॒हद् बृ॒हत् ते॑ ते बृ॒हत् । \newline
24. बृ॒हद् भा भा बृ॒हद् बृ॒हद् भाः । \newline
25. भा इतीति॒ भा भा इति॑ । \newline
26. इत्या॑ हा॒हेती त्या॑ह । \newline
27. आ॒ह॒ सु॒व॒र्गे सु॑व॒र्ग आ॑हाह सुव॒र्गे । \newline
28. सु॒व॒र्ग ए॒वैव सु॑व॒र्गे सु॑व॒र्ग ए॒व । \newline
29. सु॒व॒र्ग इति॑ सुवः - गे । \newline
30. ए॒वास्मा॑ अस्मा ए॒वैवास्मै᳚ । \newline
31. अ॒स्मै॒ लो॒के लो॒के᳚ ऽस्मा अस्मै लो॒के । \newline
32. लो॒के ज्योति॒र् ज्योति॑र् लो॒के लो॒के ज्योतिः॑ । \newline
33. ज्योति॑र् दधाति दधाति॒ ज्योति॒र् ज्योति॑र् दधाति । \newline
34. द॒धा॒ति॒ तन्तु॒म् तन्तु॑म् दधाति दधाति॒ तन्तु᳚म् । \newline
35. तन्तु॑म् त॒न्वन् त॒न्वन् तन्तु॒म् तन्तु॑म् त॒न्वन्न् । \newline
36. त॒न्वन् रज॑सो॒ रज॑स स्त॒न्वन् त॒न्वन् रज॑सः । \newline
37. रज॑सो भा॒नुम् भा॒नुꣳ रज॑सो॒ रज॑सो भा॒नुम् । \newline
38. भा॒नु मन्वनु॑ भा॒नुम् भा॒नु मनु॑ । \newline
39. अन्वि॑ही॒ ह्यन्वन् वि॑हि । \newline
40. इ॒ही तीती॑ही॒ हीति॑ । \newline
41. इत्या॑हा॒हे तीत्या॑ह । \newline
42. आ॒हे॒मा-नि॒मा-ना॑हाहे॒मान् । \newline
43. इ॒मा-ने॒वैवेमा-नि॒मा-ने॒व । \newline
44. ए॒वास्मा॑ अस्मा ए॒वैवास्मै᳚ । \newline
45. अ॒स्मै॒ लो॒कान् ॅलो॒का-न॑स्मा अस्मै लो॒कान् । \newline
46. लो॒कान् ज्योति॑ष्मतो॒ ज्योति॑ष्मतो लो॒कान् ॅलो॒कान् ज्योति॑ष्मतः । \newline
47. ज्योति॑ष्मतः करोति करोति॒ ज्योति॑ष्मतो॒ ज्योति॑ष्मतः करोति । \newline
48. क॒रो॒ त्य॒नु॒ल्ब॒ण म॑नुल्ब॒णम् क॑रोति करो त्यनुल्ब॒णम् । \newline
49. अ॒नु॒ल्ब॒णं ॅव॑यत वयता नुल्ब॒ण म॑नुल्ब॒णं ॅव॑यत । \newline
50. व॒य॒त॒ जोगु॑वा॒म् जोगु॑वां ॅवयत वयत॒ जोगु॑वाम् । \newline
51. जोगु॑वा॒ मपो ऽपो॒ जोगु॑वा॒म् जोगु॑वा॒ मपः॑ । \newline
52. अप॒ इती त्यपोऽप॒ इति॑ । \newline
53. इत्या॑ हा॒हेती त्या॑ह । \newline

\textbf{Ghana Paata } \newline

1. यज॑मानस्य॒ यद् यद् यज॑मानस्य॒ यज॑मानस्य॒ यदना॒र्तो ऽना᳚र्तो॒ यद् यज॑मानस्य॒ यज॑मानस्य॒ यदना᳚र्तः । \newline
2. यदना॒र्तो ऽना᳚र्तो॒ यद् यदना᳚र्त उ॒दृच॑ मु॒दृच॒ मना᳚र्तो॒ यद् यदना᳚र्त उ॒दृच᳚म् । \newline
3. अना᳚र्त उ॒दृच॑ मु॒दृच॒ मना॒र्तो ऽना᳚र्त उ॒दृच॒म् गच्छ॑ति॒ गच्छ॑ त्यु॒दृच॒ मना॒र्तो ऽना᳚र्त उ॒दृच॒म् गच्छ॑ति । \newline
4. उ॒दृच॒म् गच्छ॑ति॒ गच्छ॑ त्यु॒दृच॑ मु॒दृच॒म् गच्छ॑ति॒ तस्मा॒त् तस्मा॒द् गच्छ॑ त्यु॒दृच॑ मु॒दृच॒म् गच्छ॑ति॒ तस्मा᳚त् । \newline
5. उ॒दृच॒मित्यु॑त् - ऋच᳚म् । \newline
6. गच्छ॑ति॒ तस्मा॒त् तस्मा॒द् गच्छ॑ति॒ गच्छ॑ति॒ तस्मा॑ दे॒व मे॒वम् तस्मा॒द् गच्छ॑ति॒ गच्छ॑ति॒ तस्मा॑ दे॒वम् । \newline
7. तस्मा॑ दे॒व मे॒वम् तस्मा॒त् तस्मा॑ दे॒व मा॑हा है॒वम् तस्मा॒त् तस्मा॑ दे॒व मा॑ह । \newline
8. ए॒व मा॑हाहै॒व मे॒व मा॑हा॒जा ऽजा ऽऽहै॒व मे॒व मा॑हा॒जा । \newline
9. आ॒हा॒जा ऽजा ऽऽहा॑ हा॒जा ऽस्य॑ स्य॒जा ऽऽहा॑ हा॒जा ऽसि॑ । \newline
10. अ॒जा ऽस्य॑ स्य॒जा ऽजा ऽसि॑ रयि॒ष्ठा र॑यि॒ष्ठा ऽस्य॒जा ऽजा ऽसि॑ रयि॒ष्ठा । \newline
11. अ॒सि॒ र॒यि॒ष्ठा र॑यि॒ष्ठा ऽस्य॑सि रयि॒ष्ठेतीति॑ रयि॒ष्ठा ऽस्य॑सि रयि॒ष्ठेति॑ । \newline
12. र॒यि॒ष्ठेतीति॑ रयि॒ष्ठा र॑यि॒ष्ठे त्या॑हा॒हेति॑ रयि॒ष्ठा र॑यि॒ष्ठे त्या॑ह । \newline
13. र॒यि॒ष्ठेति॑ रयि - स्था । \newline
14. इत्या॑ हा॒हेती त्या॑है॒ष्वे᳚(1॒)ष्वा॑हेती त्या॑है॒षु । \newline
15. आ॒है॒ष्वे᳚(1॒)ष्वा॑हाहै॒ ष्वे॑वै वैष्वा॑ हाहै॒ ष्वे॑व । \newline
16. ए॒ष्वे॑वै वैष्वे᳚(1॒)ष्वे॑ वैना॑ मेना मे॒वैष्वे᳚(1॒)ष्वे॑ वैना᳚म् । \newline
17. ए॒वै ना॑ मेना मे॒वै वैना᳚म् ॅलो॒केषु॑ लो॒केष्वे॑ना मे॒वैवैना᳚म् ॅलो॒केषु॑ । \newline
18. ए॒ना॒म् ॅलो॒केषु॑ लो॒केष्वे॑ना मेनाम् ॅलो॒केषु॒ प्रति॒ प्रति॑ लो॒केष्वे॑ना मेनाम् ॅलो॒केषु॒ प्रति॑ । \newline
19. लो॒केषु॒ प्रति॒ प्रति॑ लो॒केषु॑ लो॒केषु॒ प्रति॑ ष्ठापयति स्थापयति॒ प्रति॑ लो॒केषु॑ लो॒केषु॒ प्रति॑ ष्ठापयति । \newline
20. प्रति॑ ष्ठापयति स्थापयति॒ प्रति॒ प्रति॑ ष्ठापयति दि॒वि दि॒वि स्था॑पयति॒ प्रति॒ प्रति॑ ष्ठापयति दि॒वि । \newline
21. स्था॒प॒य॒ति॒ दि॒वि दि॒वि स्था॑पयति स्थापयति दि॒वि ते॑ ते दि॒वि स्था॑पयति स्थापयति दि॒वि ते᳚ । \newline
22. दि॒वि ते॑ ते दि॒वि दि॒वि ते॑ बृ॒हद् बृ॒हत् ते॑ दि॒वि दि॒वि ते॑ बृ॒हत् । \newline
23. ते॒ बृ॒हद् बृ॒हत् ते॑ ते बृ॒हद् भा भा बृ॒हत् ते॑ ते बृ॒हद् भाः । \newline
24. बृ॒हद् भा भा बृ॒हद् बृ॒हद् भा इतीति॒ भा बृ॒हद् बृ॒हद् भा इति॑ । \newline
25. भा इतीति॒ भा भा इत्या॑ हा॒हेति॒ भा भा इत्या॑ह । \newline
26. इत्या॑ हा॒हेती त्या॑ह सुव॒र्गे सु॑व॒र्ग आ॒हे तीत्या॑ह सुव॒र्गे । \newline
27. आ॒ह॒ सु॒व॒र्गे सु॑व॒र्ग आ॑हाह सुव॒र्ग ए॒वैव सु॑व॒र्ग आ॑हाह सुव॒र्ग ए॒व । \newline
28. सु॒व॒र्ग ए॒वैव सु॑व॒र्गे सु॑व॒र्ग ए॒वास्मा॑ अस्मा ए॒व सु॑व॒र्गे सु॑व॒र्ग ए॒वास्मै᳚ । \newline
29. सु॒व॒र्ग इति॑ सुवः - गे । \newline
30. ए॒वास्मा॑ अस्मा ए॒वैवास्मै॑ लो॒के लो॒के᳚ ऽस्मा ए॒वैवास्मै॑ लो॒के । \newline
31. अ॒स्मै॒ लो॒के लो॒के᳚ ऽस्मा अस्मै लो॒के ज्योति॒र् ज्योति॑र् लो॒के᳚ ऽस्मा अस्मै लो॒के ज्योतिः॑ । \newline
32. लो॒के ज्योति॒र् ज्योति॑र् लो॒के लो॒के ज्योति॑र् दधाति दधाति॒ ज्योति॑र् लो॒के लो॒के ज्योति॑र् दधाति । \newline
33. ज्योति॑र् दधाति दधाति॒ ज्योति॒र् ज्योति॑र् दधाति॒ तन्तु॒म् तन्तु॑म् दधाति॒ ज्योति॒र् ज्योति॑र् दधाति॒ तन्तु᳚म् । \newline
34. द॒धा॒ति॒ तन्तु॒म् तन्तु॑म् दधाति दधाति॒ तन्तु॑म् त॒न्वन् त॒न्वन् तन्तु॑म् दधाति दधाति॒ तन्तु॑म् त॒न्वन्न् । \newline
35. तन्तु॑म् त॒न्वन् त॒न्वन् तन्तु॒म् तन्तु॑म् त॒न्वन् रज॑सो॒ रज॑स स्त॒न्वन् तन्तु॒म् तन्तु॑म् त॒न्वन् रज॑सः । \newline
36. त॒न्वन् रज॑सो॒ रज॑स स्त॒न्वन् त॒न्वन् रज॑सो भा॒नुम् भा॒नुꣳ रज॑स स्त॒न्वन् त॒न्वन् रज॑सो भा॒नुम् । \newline
37. रज॑सो भा॒नुम् भा॒नुꣳ रज॑सो॒ रज॑सो भा॒नु मन्वनु॑ भा॒नुꣳ रज॑सो॒ रज॑सो भा॒नु मनु॑ । \newline
38. भा॒नु मन्वनु॑ भा॒नुम् भा॒नु मन्वि॑ही॒ ह्यनु॑ भा॒नुम् भा॒नु मन्वि॑हि । \newline
39. अन्वि॑ही॒ ह्यन्व न्वि॒ही तीती॒ ह्यन्व न्वि॒हीति॑ । \newline
40. इ॒ही तीती॑ ही॒ही त्या॑हा॒हेती॑ ही॒ही त्या॑ह । \newline
41. इत्या॑ हा॒हेती त्या॑हे॒मा,नि॒मा,ना॒हेती त्या॑हे॒मान् । \newline
42. आ॒हे॒मा,नि॒मा,ना॑हा हे॒मा,ने॒वै वेमा,ना॑हा हे॒माने॒व । \newline
43. इ॒मा,ने॒वैवेमा,नि॒मा,ने॒वास्मा॑ अस्मा ए॒वेमा,नि॒मा,ने॒वास्मै᳚ । \newline
44. ए॒वास्मा॑ अस्मा ए॒वैवास्मै॑ लो॒कान् ॅलो॒का,न॑स्मा ए॒वैवास्मै॑ लो॒कान् । \newline
45. अ॒स्मै॒ लो॒कान् ॅलो॒का,न॑स्मा अस्मै लो॒कान् ज्योति॑ष्मतो॒ ज्योति॑ष्मतो लो॒का न॑स्मा अस्मै लो॒कान् ज्योति॑ष्मतः । \newline
46. लो॒कान् ज्योति॑ष्मतो॒ ज्योति॑ष्मतो लो॒कान् ॅलो॒कान् ज्योति॑ष्मतः करोति करोति॒ ज्योति॑ष्मतो लो॒कान् ॅलो॒कान् ज्योति॑ष्मतः करोति । \newline
47. ज्योति॑ष्मतः करोति करोति॒ ज्योति॑ष्मतो॒ ज्योति॑ष्मतः करो त्यनुल्ब॒ण म॑नुल्ब॒णम् क॑रोति॒ ज्योति॑ष्मतो॒ ज्योति॑ष्मतः करो त्यनुल्ब॒णम् । \newline
48. क॒रो॒ त्य॒नु॒ल्ब॒ण म॑नुल्ब॒णम् क॑रोति करो त्यनुल्ब॒णम् ॅव॑यत वयता नुल्ब॒णम् क॑रोति करो त्यनुल्ब॒णम् ॅव॑यत । \newline
49. अ॒नु॒ल्ब॒णम् ॅव॑यत वयता नुल्ब॒ण म॑नुल्ब॒णम् ॅव॑यत॒ जोगु॑वा॒म् जोगु॑वाम् ॅवयता नुल्ब॒ण म॑नुल्ब॒णम् ॅव॑यत॒ जोगु॑वाम् । \newline
50. व॒य॒त॒ जोगु॑वा॒म् जोगु॑वाम् ॅवयत वयत॒ जोगु॑वा॒ मपोऽपो॒ जोगु॑वाम् ॅवयत वयत॒ जोगु॑वा॒ मपः॑ । \newline
51. जोगु॑वा॒ मपो ऽपो॒ जोगु॑वा॒म् जोगु॑वा॒ मप॒ इती त्यपो॒ जोगु॑वा॒म् जोगु॑वा॒ मप॒ इति॑ । \newline
52. अप॒ इती त्यपोऽप॒ इत्या॑हा॒हे त्यपोऽप॒ इत्या॑ह । \newline
53. इत्या॑ हा॒हेती त्या॑ह॒ यद् यदा॒हेती त्या॑ह॒ यत् । \newline
\pagebreak
\markright{ TS 3.4.3.7  \hfill https://www.vedavms.in \hfill}

\section{ TS 3.4.3.7 }

\textbf{TS 3.4.3.7 } \newline
\textbf{Samhita Paata} \newline

-ह॒ यदे॒व य॒ज्ञ् उ॒ल्बणं॑ क्रि॒यते॒ तस्यै॒वैषा शान्ति॒र्मनु॑र्भव ज॒नया॒ दैव्यं॒ जन॒मित्या॑ह मान॒व्यो॑ वै प्र॒जास्ता ए॒वाऽऽ*द्याः᳚ कुरुते॒ मन॑सो ह॒विर॒सीत्या॑ह स्व॒गाकृ॑त्यै॒ गात्रा॑णां ते गात्र॒भाजो॑ भूया॒स्मेत्या॑हा॒ ऽऽ*शिष॑मे॒वैतामा शा᳚स्ते॒ तस्यै॒ वा ए॒तस्या॒ एक॑मे॒वा-दे॑वयजनं॒ ॅयदाल॑ब्धायाम॒भ्रो - [  ] \newline

\textbf{Pada Paata} \newline

आ॒ह॒ । यत् । ए॒व । य॒ज्ञे । उ॒ल्बण᳚म् । क्रि॒यते᳚ । तस्य॑ । ए॒व । ए॒षा । शान्तिः॑ । मनुः॑ । भ॒व॒ । ज॒नय॑ । दैव्य᳚म् । जन᳚म् । इति॑ । आ॒ह॒ । मा॒न॒व्यः॑ । वै । प्र॒जा इति॑ प्र - जाः । ताः । ए॒व । आ॒द्याः᳚ । कु॒रु॒ते॒ । मन॑सः । ह॒विः । अ॒सि॒ । इति॑ । आ॒ह॒ । स्व॒गाकृ॑त्या॒ इति॑ स्व॒गा - कृ॒त्यै॒ । गात्रा॑णाम् । ते॒ । गा॒त्र॒भाज॒ इति॑ गात्र - भाजः॑ । भू॒या॒स्म॒ । इति॑ । आ॒ह॒ । आ॒शिष॒मित्या᳚ - शिष᳚म् । ए॒व । ए॒ताम् । एति॑ । शा॒स्ते॒ । तस्यै᳚ । वै । ए॒तस्याः᳚ । एक᳚म् । ए॒व । अदे॑वयजन॒मित्यदे॑व - य॒ज॒न॒म् । यत् । आल॑ब्धाया॒मित्या - ल॒ब्धा॒या॒म् । अ॒भ्रः ।  \newline


\textbf{Krama Paata} \newline

आ॒ह॒ यत् । यदे॒व । ए॒व य॒ज्ञे । य॒ज्ञ् उ॒ल्बण᳚म् । उ॒ल्बण॑म् क्रि॒यते᳚ । क्रि॒यते॒ तस्य॑ । तस्यै॒व । ए॒वैषा । ए॒षा शान्तिः॑ । शान्ति॒र् मनुः॑ । मनु॑र् भव । भ॒व॒ ज॒नय॑ । ज॒नया॒ दैव्य᳚म् । दैव्य॒म् जन᳚म् । जन॒मिति॑ । इत्या॑ह । आ॒ह॒ मा॒न॒व्यः॑ । मा॒न॒व्यो॑ वै । वै प्र॒जाः । प्र॒जा स्ताः । प्र॒जा इति॑ प्र - जाः । ता ए॒व । ए॒वाद्याः᳚ । आ॒द्याः᳚ कुरुते । कु॒रु॒ते॒ मन॑सः । मन॑सो ह॒विः । ह॒विर॑सि । अ॒सीति॑ । इत्या॑ह । आ॒ह॒ स्व॒गाकृ॑त्यै । स्व॒गाकृ॑त्यै॒ गात्रा॑णाम् । स्व॒गाकृ॑त्या॒ इति॑ स्व॒गा - कृ॒त्यै॒ । गात्रा॑णाम् ते । ते॒ गा॒त्र॒भाजः॑ । गा॒त्र॒भाजो॑ भूयास्म । गा॒त्र॒भाज॒ इति॑ गात्र - भाजः॑ । भू॒या॒स्मेति॑ । इत्या॑ह । आ॒हा॒शिष᳚म् । आ॒शिष॑मे॒व । आ॒शिष॒मित्या᳚ - शिष᳚म् । ए॒वैताम् । ए॒तामा । आ शा᳚स्ते । शा॒स्ते॒ तस्यै᳚ । तस्यै॒ वै । वा ए॒तस्याः᳚ । ए॒तस्या॒ एक᳚म् । एक॑मे॒व । ए॒वादे॑वयजनम् । अदे॑वयजनं॒ ॅयत् । अदे॑वयजन॒मित्यदे॑व - य॒ज॒न॒म् । यदाल॑ब्धायाम् । आल॑ब्धायाम॒भ्रः ( ) । आल॑ब्धाया॒मित्या - ल॒ब्धा॒या॒म् । अ॒भ्रो भव॑ति \newline

\textbf{Jatai Paata} \newline

1. आ॒ह॒ यद् यदा॑हाह॒ यत् । \newline
2. यदे॒ वैव यद् यदे॒व । \newline
3. ए॒व य॒ज्ञे य॒ज्ञ् ए॒वैव य॒ज्ञे । \newline
4. य॒ज्ञ् उ॒ल्बण॑ मु॒ल्बणं॑ ॅय॒ज्ञे य॒ज्ञ् उ॒ल्बण᳚म् । \newline
5. उ॒ल्बण॑म् क्रि॒यते᳚ क्रि॒यत॑ उ॒ल्बण॑ मु॒ल्बण॑म् क्रि॒यते᳚ । \newline
6. क्रि॒यते॒ तस्य॒ तस्य॑ क्रि॒यते᳚ क्रि॒यते॒ तस्य॑ । \newline
7. तस्यै॒वैव तस्य॒ तस्यै॒व । \newline
8. ए॒वै षैषै वैवैषा । \newline
9. ए॒षा शान्तिः॒ शान्ति॑ रे॒षैषा शान्तिः॑ । \newline
10. शान्ति॒र् मनु॒र् मनुः॒ शान्तिः॒ शान्ति॒र् मनुः॑ । \newline
11. मनु॑र् भव भव॒ मनु॒र् मनु॑र् भव । \newline
12. भ॒व॒ ज॒नय॑ ज॒नय॑ भव भव ज॒नय॑ । \newline
13. ज॒नया॒ दैव्य॒म् दैव्य॑म् ज॒नय॑ ज॒नया॒ दैव्य᳚म् । \newline
14. दैव्य॒म् जन॒म् जन॒म् दैव्य॒म् दैव्य॒म् जन᳚म् । \newline
15. जन॒ मितीति॒ जन॒म् जन॒ मिति॑ । \newline
16. इत्या॑ हा॒हेती त्या॑ह । \newline
17. आ॒ह॒ मा॒न॒व्यो॑ मान॒व्य॑ आहाह मान॒व्यः॑ । \newline
18. मा॒न॒व्यो॑ वै वै मा॑न॒व्यो॑ मान॒व्यो॑ वै । \newline
19. वै प्र॒जाः प्र॒जा वै वै प्र॒जाः । \newline
20. प्र॒जा स्ता स्ताः प्र॒जाः प्र॒जा स्ताः । \newline
21. प्र॒जा इति॑ प्र - जाः । \newline
22. ता ए॒वैव ता स्ता ए॒व । \newline
23. ए॒वाद्या॑ आ॒द्या॑ ए॒वैवाद्याः᳚ । \newline
24. आ॒द्याः᳚ कुरुते कुरुत आ॒द्या॑ आ॒द्याः᳚ कुरुते । \newline
25. कु॒रु॒ते॒ मन॑सो॒ मन॑सः कुरुते कुरुते॒ मन॑सः । \newline
26. मन॑सो ह॒विर्. ह॒विर् मन॑सो॒ मन॑सो ह॒विः । \newline
27. ह॒वि र॑स्यसि ह॒विर्. ह॒वि र॑सि । \newline
28. अ॒सीती त्य॑स्य॒ सीति॑ । \newline
29. इत्या॑ हा॒हेती त्या॑ह । \newline
30. आ॒ह॒ स्व॒गाकृ॑त्यै स्व॒गाकृ॑त्या आहाह स्व॒गाकृ॑त्यै । \newline
31. स्व॒गाकृ॑त्यै॒ गात्रा॑णा॒म् गात्रा॑णाꣳ स्व॒गाकृ॑त्यै स्व॒गाकृ॑त्यै॒ गात्रा॑णाम् । \newline
32. स्व॒गाकृ॑त्या॒ इति॑ स्व॒गा - कृ॒त्यै॒ । \newline
33. गात्रा॑णाम् ते ते॒ गात्रा॑णा॒म् गात्रा॑णाम् ते । \newline
34. ते॒ गा॒त्र॒भाजो॑ गात्र॒भाज॑ स्ते ते गात्र॒भाजः॑ । \newline
35. गा॒त्र॒भाजो॑ भूयास्म भूयास्म गात्र॒भाजो॑ गात्र॒भाजो॑ भूयास्म । \newline
36. गा॒त्र॒भाज॒ इति॑ गात्र - भाजः॑ । \newline
37. भू॒या॒स्मेतीति॑ भूयास्म भूया॒स्मेति॑ । \newline
38. इत्या॑ हा॒हेती त्या॑ह । \newline
39. आ॒हा॒ शिष॑ मा॒शिष॑ माहा हा॒शिष᳚म् । \newline
40. आ॒शिष॑ मे॒वैवाशिष॑ मा॒शिष॑मे॒व । \newline
41. आ॒शिष॒मित्या᳚ - शिष᳚म् । \newline
42. ए॒वैता मे॒ता मे॒वैवैताम् । \newline
43. ए॒ता मैता मे॒तामा । \newline
44. आ शा᳚स्ते शास्त॒ आ शा᳚स्ते । \newline
45. शा॒स्ते॒ तस्यै॒ तस्यै॑ शास्ते शास्ते॒ तस्यै᳚ । \newline
46. तस्यै॒ वै वै तस्यै॒ तस्यै॒ वै । \newline
47. वा ए॒तस्या॑ ए॒तस्या॒ वै वा ए॒तस्याः᳚ । \newline
48. ए॒तस्या॒ एक॒ मेक॑ मे॒तस्या॑ ए॒तस्या॒ एक᳚म् । \newline
49. एक॑ मे॒वैवैक॒ मेक॑ मे॒व । \newline
50. ए॒वा दे॑वयजन॒ मदे॑वयजन मे॒वैवा दे॑वयजनम् । \newline
51. अदे॑वयजनं॒ ॅयद् यददे॑वयजन॒ मदे॑वयजनं॒ ॅयत् । \newline
52. अदे॑वयजन॒मित्यदे॑व - य॒ज॒न॒म् । \newline
53. यदाल॑ब्धाया॒ माल॑ब्धायां॒ ॅयद् यदाल॑ब्धायाम् । \newline
54. आल॑ब्धाया म॒भ्रो᳚ ऽभ्र आल॑ब्धाया॒ माल॑ब्धाया म॒भ्रः । \newline
55. आल॑ब्धाया॒मित्या - ल॒ब्धा॒या॒म् । \newline
56. अ॒भ्रो भव॑ति॒ भव॑ त्य॒भ्रो᳚ ऽभ्रो भव॑ति । \newline

\textbf{Ghana Paata } \newline

1. आ॒ह॒ यद् यदा॑ हाह॒ यदे॒वैव यदा॑हाह॒ यदे॒व । \newline
2. यदे॒वैव यद् यदे॒व य॒ज्ञे य॒ज्ञ् ए॒व यद् यदे॒व य॒ज्ञे । \newline
3. ए॒व य॒ज्ञे य॒ज्ञ् ए॒वैव य॒ज्ञ् उ॒ल्बण॑ मु॒ल्बण॑म् ॅय॒ज्ञ् ए॒वैव य॒ज्ञ् उ॒ल्बण᳚म् । \newline
4. य॒ज्ञ् उ॒ल्बण॑ मु॒ल्बण॑म् ॅय॒ज्ञे य॒ज्ञ् उ॒ल्बण॑म् क्रि॒यते᳚ क्रि॒यत॑ उ॒ल्बण॑म् ॅय॒ज्ञे य॒ज्ञ् उ॒ल्बण॑म् क्रि॒यते᳚ । \newline
5. उ॒ल्बण॑म् क्रि॒यते᳚ क्रि॒यत॑ उ॒ल्बण॑ मु॒ल्बण॑म् क्रि॒यते॒ तस्य॒ तस्य॑ क्रि॒यत॑ उ॒ल्बण॑ मु॒ल्बण॑म् क्रि॒यते॒ तस्य॑ । \newline
6. क्रि॒यते॒ तस्य॒ तस्य॑ क्रि॒यते᳚ क्रि॒यते॒ तस्यै॒वैव तस्य॑ क्रि॒यते᳚ क्रि॒यते॒ तस्यै॒व । \newline
7. तस्यै॒वैव तस्य॒ तस्यै॒वै षैषैव तस्य॒ तस्यै॒वैषा । \newline
8. ए॒वैषैषै वैवैषा शान्तिः॒ शान्ति॑ रे॒षै वैवैषा शान्तिः॑ । \newline
9. ए॒षा शान्तिः॒ शान्ति॑ रे॒षैषा शान्ति॒र् मनु॒र् मनुः॒ शान्ति॑ रे॒षैषा शान्ति॒र् मनुः॑ । \newline
10. शान्ति॒र् मनु॒र् मनुः॒ शान्तिः॒ शान्ति॒र् मनु॑र् भव भव॒ मनुः॒ शान्तिः॒ शान्ति॒र् मनु॑र् भव । \newline
11. मनु॑र् भव भव॒ मनु॒र् मनु॑र् भव ज॒नय॑ ज॒नय॑ भव॒ मनु॒र् मनु॑र् भव ज॒नय॑ । \newline
12. भ॒व॒ ज॒नय॑ ज॒नय॑ भव भव ज॒नया॒ दैव्य॒म् दैव्य॑म् ज॒नय॑ भव भव ज॒नया॒ दैव्य᳚म् । \newline
13. ज॒नया॒ दैव्य॒म् दैव्य॑म् ज॒नय॑ ज॒नया॒ दैव्य॒म् जन॒म् जन॒म् दैव्य॑म् ज॒नय॑ ज॒नया॒ दैव्य॒म् जन᳚म् । \newline
14. दैव्य॒म् जन॒म् जन॒म् दैव्य॒म् दैव्य॒म् जन॒ मितीति॒ जन॒म् दैव्य॒म् दैव्य॒म् जन॒ मिति॑ । \newline
15. जन॒ मितीति॒ जन॒म् जन॒ मित्या॑ हा॒हेति॒ जन॒म् जन॒ मित्या॑ह । \newline
16. इत्या॑ हा॒हेतीत्या॑ह मान॒व्यो॑ मान॒व्य॑ आ॒हेतीत्या॑ह मान॒व्यः॑ । \newline
17. आ॒ह॒ मा॒न॒व्यो॑ मान॒व्य॑ आहाह मान॒व्यो॑ वै वै मा॑न॒व्य॑ आहाह मान॒व्यो॑ वै । \newline
18. मा॒न॒व्यो॑ वै वै मा॑न॒व्यो॑ मान॒व्यो॑ वै प्र॒जाः प्र॒जा वै मा॑न॒व्यो॑ मान॒व्यो॑ वै प्र॒जाः । \newline
19. वै प्र॒जाः प्र॒जा वै वै प्र॒जा स्ता स्ताः प्र॒जा वै वै प्र॒जा स्ताः । \newline
20. प्र॒जा स्ता स्ताः प्र॒जाः प्र॒जा स्ता ए॒वैव ताः प्र॒जाः प्र॒जा स्ता ए॒व । \newline
21. प्र॒जा इति॑ प्र - जाः । \newline
22. ता ए॒वैव ता स्ता ए॒वाद्या॑ आ॒द्या॑ ए॒व ता स्ता ए॒वाद्याः᳚ । \newline
23. ए॒वाद्या॑ आ॒द्या॑ ए॒वैवाद्याः᳚ कुरुते कुरुत आ॒द्या॑ ए॒वैवाद्याः᳚ कुरुते । \newline
24. आ॒द्याः᳚ कुरुते कुरुत आ॒द्या॑ आ॒द्याः᳚ कुरुते॒ मन॑सो॒ मन॑सः कुरुत आ॒द्या॑ आ॒द्याः᳚ कुरुते॒ मन॑सः । \newline
25. कु॒रु॒ते॒ मन॑सो॒ मन॑सः कुरुते कुरुते॒ मन॑सो ह॒विर्. ह॒विर् मन॑सः कुरुते कुरुते॒ मन॑सो ह॒विः । \newline
26. मन॑सो ह॒विर्. ह॒विर् मन॑सो॒ मन॑सो ह॒वि र॑स्यसि ह॒विर् मन॑सो॒ मन॑सो ह॒विर॑सि । \newline
27. ह॒वि र॑स्यसि ह॒विर्. ह॒वि र॒सीती त्य॑सि ह॒विर्. ह॒वि र॒सीति॑ । \newline
28. अ॒सीती त्य॑स्य॒सी त्या॑हा॒हे त्य॑स्य॒सी त्या॑ह । \newline
29. इत्या॑ हा॒हेती त्या॑ह स्व॒गाकृ॑त्यै स्व॒गाकृ॑त्या आ॒हेती त्या॑ह स्व॒गाकृ॑त्यै । \newline
30. आ॒ह॒ स्व॒गाकृ॑त्यै स्व॒गाकृ॑त्या आहाह स्व॒गाकृ॑त्यै॒ गात्रा॑णा॒म् गात्रा॑णाꣳ स्व॒गाकृ॑त्या आहाह स्व॒गाकृ॑त्यै॒ गात्रा॑णाम् । \newline
31. स्व॒गाकृ॑त्यै॒ गात्रा॑णा॒म् गात्रा॑णाꣳ स्व॒गाकृ॑त्यै स्व॒गाकृ॑त्यै॒ गात्रा॑णाम् ते ते॒ गात्रा॑णाꣳ स्व॒गाकृ॑त्यै स्व॒गाकृ॑त्यै॒ गात्रा॑णाम् ते । \newline
32. स्व॒गाकृ॑त्या॒ इति॑ स्व॒गा - कृ॒त्यै॒ । \newline
33. गात्रा॑णाम् ते ते॒ गात्रा॑णा॒म् गात्रा॑णाम् ते गात्र॒भाजो॑ गात्र॒भाज॑ स्ते॒ गात्रा॑णा॒म् गात्रा॑णाम् ते गात्र॒भाजः॑ । \newline
34. ते॒ गा॒त्र॒भाजो॑ गात्र॒भाज॑ स्ते ते गात्र॒भाजो॑ भूयास्म भूयास्म गात्र॒भाज॑ स्ते ते गात्र॒भाजो॑ भूयास्म । \newline
35. गा॒त्र॒भाजो॑ भूयास्म भूयास्म गात्र॒भाजो॑ गात्र॒भाजो॑ भूया॒स्मेतीति॑ भूयास्म गात्र॒भाजो॑ गात्र॒भाजो॑ भूया॒स्मेति॑ । \newline
36. गा॒त्र॒भाज॒ इति॑ गात्र - भाजः॑ । \newline
37. भू॒या॒स्मेतीति॑ भूयास्म भूया॒स्मे त्या॑हा॒हेति॑ भूयास्म भूया॒स्मे त्या॑ह । \newline
38. इत्या॑ हा॒हेती त्या॑हा॒शिष॑ मा॒शिष॑ मा॒हेती त्या॑हा॒शिष᳚म् । \newline
39. आ॒हा॒शिष॑ मा॒शिष॑ माहा हा॒शिष॑ मे॒वै वाशिष॑ माहा हा॒शिष॑ मे॒व । \newline
40. आ॒शिष॑ मे॒वै वाशिष॑ मा॒शिष॑ मे॒वैता मे॒ता मे॒वा शिष॑ मा॒शिष॑ मे॒वै ताम् । \newline
41. आ॒शिष॒मित्या᳚ - शिष᳚म् । \newline
42. ए॒वैता मे॒ता मे॒वै वैता मैता मे॒वै वैता मा । \newline
43. ए॒ता मैता मे॒ता मा शा᳚स्ते शास्त॒ ऐता मे॒ता मा शा᳚स्ते । \newline
44. आ शा᳚स्ते शास्त॒ आ शा᳚स्ते॒ तस्यै॒ तस्यै॑ शास्त॒ आ शा᳚स्ते॒ तस्यै᳚ । \newline
45. शा॒स्ते॒ तस्यै॒ तस्यै॑ शास्ते शास्ते॒ तस्यै॒ वै वै तस्यै॑ शास्ते शास्ते॒ तस्यै॒ वै । \newline
46. तस्यै॒ वै वै तस्यै॒ तस्यै॒ वा ए॒तस्या॑ ए॒तस्या॒ वै तस्यै॒ तस्यै॒ वा ए॒तस्याः᳚ । \newline
47. वा ए॒तस्या॑ ए॒तस्या॒ वै वा ए॒तस्या॒ एक॒ मेक॑ मे॒तस्या॒ वै वा ए॒तस्या॒ एक᳚म् । \newline
48. ए॒तस्या॒ एक॒ मेक॑ मे॒तस्या॑ ए॒तस्या॒ एक॑ मे॒वैवैक॑ मे॒तस्या॑ ए॒तस्या॒ एक॑ मे॒व । \newline
49. एक॑ मे॒वैवैक॒ मेक॑ मे॒वा दे॑वयजन॒ मदे॑वयजन मे॒वैक॒ मेक॑ मे॒वा दे॑वयजनम् । \newline
50. ए॒वा दे॑वयजन॒ मदे॑वयजन मे॒वैवा दे॑वयजन॒म् ॅयद् यद दे॑वयजन मे॒वैवा दे॑वयजन॒म् ॅयत् । \newline
51. अदे॑वयजन॒म् ॅयद् यददे॑वयजन॒ मदे॑वयजन॒म् ॅयदाल॑ब्धाया॒ माल॑ब्धाया॒म् ॅयददे॑वयजन॒ मदे॑वयजन॒म् ॅयदाल॑ब्धायाम् । \newline
52. अदे॑वयजन॒मित्यदे॑व - य॒ज॒न॒म् । \newline
53. यदाल॑ब्धाया॒ माल॑ब्धाया॒म् ॅयद् यदाल॑ब्धाया म॒भ्रो᳚ ऽभ्र आल॑ब्धाया॒म् ॅयद् यदाल॑ब्धाया म॒भ्रः । \newline
54. आल॑ब्धाया म॒भ्रो᳚ ऽभ्र आल॑ब्धाया॒ माल॑ब्धाया म॒भ्रो भव॑ति॒ भव॑ त्य॒भ्र आल॑ब्धाया॒ माल॑ब्धाया म॒भ्रो भव॑ति । \newline
55. आल॑ब्धाया॒मित्या - ल॒ब्धा॒या॒म् । \newline
56. अ॒भ्रो भव॑ति॒ भव॑ त्य॒भ्रो᳚ ऽभ्रो भव॑ति॒ यद् यद् भव॑ त्य॒भ्रो᳚ ऽभ्रो भव॑ति॒ यत् । \newline
\pagebreak
\markright{ TS 3.4.3.8  \hfill https://www.vedavms.in \hfill}

\section{ TS 3.4.3.8 }

\textbf{TS 3.4.3.8 } \newline
\textbf{Samhita Paata} \newline

भव॑ति॒ यदाल॑ब्धायाम॒भ्रः स्याद॒फ्सु वा᳚प्रवे॒शये॒थ् सर्वां᳚ ॅवा॒ प्राश्ञी॑या॒द्यद॒फ्सु प्र॑वे॒शये᳚द्यज्ञ्वेश॒सं कु॑र्या॒थ् सर्वा॑मे॒व प्राश्ञी॑यादिन्द्रि॒यमे॒वाऽऽ*त्मन् ध॑त् ते॒ सा वा ए॒षा त्र॑या॒णामे॒वाव॑ रुद्धा संॅवथ्सर॒सदः॑ सहस्रया॒जिनो॑ गृहमे॒धिन॒स्त ए॒वैतया॑ यजेर॒न् तेषा॑मे॒वैषाऽऽप्ता ॥ \newline

\textbf{Pada Paata} \newline

भव॑ति । यत् । आल॑ब्धाया॒मित्या - ल॒ब्धा॒या॒म् । अ॒भ्रः । स्यात् । अ॒फ्स्वित्य॑प्-सु । वा॒ । प्र॒वे॒शये॒दिति॑ प्र - वे॒शये᳚त् । सर्वा᳚म् । वा॒ । प्रेति॑ । अ॒श्नी॒या॒त् । यत् । अ॒फ्स्वित्य॑प् - सु । प्र॒वे॒शये॒दिति॑ प्र - वे॒शये᳚त् । य॒ज्ञ्॒वे॒श॒समिति॑ यज्ञ् - वे॒श॒सम् । कु॒र्या॒त् । सर्वा᳚म् । ए॒व । प्रेति॑ । अ॒श्नी॒या॒त् । इ॒न्द्रि॒यम् । ए॒व । आ॒त्मन् । ध॒त्ते॒ । सा । वै । ए॒षा । त्र॒या॒णाम् । ए॒व । अव॑रु॒द्धेत्यव॑ - रु॒द्धा॒ । सं॒ॅव॒थ्स॒र॒सद॒ इति॑ संॅवथ्सर - सदः॑ । स॒ह॒स्र॒या॒जिन॒ इति॑ सहस्र - या॒जिनः॑ । गृ॒ह॒मे॒धिन॒ इति॑ गृह - मे॒धिनः॑ । ते । ए॒व । ए॒तया᳚ । य॒जे॒र॒न्न् । तेषा᳚म् । ए॒व । ए॒षा । आ॒प्ता ॥  \newline


\textbf{Krama Paata} \newline

भव॑ति॒ यत् । यदाल॑ब्धायाम् । आल॑ब्धायाम॒भ्रः । आल॑ब्धाया॒मित्या - ल॒ब्धा॒या॒म् । अ॒भ्रः स्यात् । स्याद॒फ्सु । अ॒फ्सु वा᳚ । अ॒फ्स्वित्य॑प् - सु । वा॒ प्र॒वे॒शये᳚त् । प्र॒वे॒शये॒थ् सर्वा᳚म् । प्र॒वे॒शये॒दिति॑ प्र - वे॒शये᳚त् । सर्वां᳚ ॅवा । वा॒ प्र । प्राश्ञी॑यात् । अ॒श्ञी॒या॒द् यत् । यद॒फ्सु । अ॒फ्सु प्र॑वे॒शये᳚त् । अ॒फ्स्वित्य॑प् - सु । प्र॒वे॒शये᳚द् यज्ञ्वेश॒सम् । प्र॒वे॒शये॒दिति॑ प्र - वे॒शये᳚त् । य॒ज्ञ्॒वे॒श॒सम् कु॑र्यात् । य॒ज्ञ्॒वे॒श॒समिति॑ यज्ञ् - वे॒श॒सम् । कु॒र्या॒थ् सर्वा᳚म् । सर्वा॑मे॒व । ए॒व प्र । प्राश्ञी॑यात् । अ॒श्ञी॒या॒दि॒न्द्रि॒यम् । इ॒न्द्रि॒यमे॒व । ए॒वात्मन्न् । आ॒त्मन् ध॑त्ते । ध॒त्ते॒ सा । सा वै । वा ए॒षा । ए॒षा त्र॑या॒णाम् । त्र॒या॒णामे॒व । ए॒वाव॑रुद्धा । अव॑रुद्धा सम्ॅवथ्सर॒सदः॑ । अव॑रु॒द्धेत्यव॑ - रु॒द्धा॒ । स॒म्ॅव॒थ्स॒र॒सदः॑ सहस्रया॒जिनः॑ । स॒म्ॅव॒थ्स॒र॒सद॒ इति॑ सम्ॅवथ्सर - सदः॑ । स॒ह॒स्र॒या॒जिनो॑ गृहमे॒धिनः॑ । स॒ह॒स्र॒या॒जिन॒ इति॑ सहस्र - या॒जिनः॑ । गृ॒ह॒मे॒धिन॒स्ते । गृ॒ह॒मे॒धिन॒ इति॑ गृह - मे॒धिनः॑ । त ए॒व । ए॒वैतया᳚ । ए॒तया॑ यजेरन्न् । य॒जे॒र॒न् तेषा᳚म् । तेषा॑मे॒व । ए॒वैषा । ए॒षा ऽऽप्ता । आ॒प्तेत्या॒प्ता । \newline

\textbf{Jatai Paata} \newline

1. भव॑ति॒ यद् यद् भव॑ति॒ भव॑ति॒ यत् । \newline
2. यदाल॑ब्धाया॒ माल॑ब्धायां॒ ॅयद् यदाल॑ब्धायाम् । \newline
3. आल॑ब्धाया म॒भ्रो᳚ ऽभ्र आल॑ब्धाया॒ माल॑ब्धाया म॒भ्रः । \newline
4. आल॑ब्धाया॒मित्या - ल॒ब्धा॒या॒म् । \newline
5. अ॒भ्रः स्याथ् स्या द॒भ्रो᳚ ऽभ्रः स्यात् । \newline
6. स्या द॒फ्स्व॑फ्सु स्याथ् स्या द॒फ्सु । \newline
7. अ॒फ्सु वा॑ वा॒ ऽफ्स्व॑फ्सु वा᳚ । \newline
8. अ॒फ्स्वित्य॑प् - सु । \newline
9. वा॒ प्र॒वे॒शये᳚त् प्रवे॒शये᳚द् वा वा प्रवे॒शये᳚त् । \newline
10. प्र॒वे॒शये॒थ् सर्वाꣳ॒॒ सर्वा᳚म् प्रवे॒शये᳚त् प्रवे॒शये॒थ् सर्वा᳚म् । \newline
11. प्र॒वे॒शये॒दिति॑ प्र - वे॒शये᳚त् । \newline
12. सर्वां᳚ ॅवा वा॒ सर्वाꣳ॒॒ सर्वां᳚ ॅवा । \newline
13. वा॒ प्र प्र वा॑ वा॒ प्र । \newline
14. प्राश्ञी॑याद श्ञीया॒त् प्र प्राश्ञी॑यात् । \newline
15. अ॒श्ञी॒या॒द् यद् यद॑श्ञीया दश्ञीया॒द् यत् । \newline
16. यद॒फ्स्व॑ फ्सु यद् यद॒फ्सु । \newline
17. अ॒फ्सु प्र॑वे॒शये᳚त् प्रवे॒शये॑ द॒फ्स्व॑ फ्सु प्र॑वे॒शये᳚त् । \newline
18. अ॒फ्स्वित्य॑प् - सु । \newline
19. प्र॒वे॒शये᳚द् यज्ञ्वेश॒सं ॅय॑ज्ञ्वेश॒सम् प्र॑वे॒शये᳚त् प्रवे॒शये᳚द् यज्ञ्वेश॒सम् । \newline
20. प्र॒वे॒शये॒दिति॑ प्र - वे॒शये᳚त् । \newline
21. य॒ज्ञ्॒वे॒श॒सम् कु॑र्यात् कुर्याद् यज्ञ्वेश॒सं ॅय॑ज्ञ्वेश॒सम् कु॑र्यात् । \newline
22. य॒ज्ञ्॒वे॒श॒समिति॑ यज्ञ् - वे॒श॒सम् । \newline
23. कु॒र्या॒थ् सर्वाꣳ॒॒ सर्वा᳚म् कुर्यात् कुर्या॒थ् सर्वा᳚म् । \newline
24. सर्वा॑ मे॒वैव सर्वाꣳ॒॒ सर्वा॑ मे॒व । \newline
25. ए॒व प्र प्रैवैव प्र । \newline
26. प्राश्ञी॑या दश्ञीया॒त् प्र प्रा श्ञी॑यात् । \newline
27. अ॒श्ञी॒या॒ दि॒न्द्रि॒य मि॑न्द्रि॒य म॑श्ञीया दश्ञीया दिन्द्रि॒यम् । \newline
28. इ॒न्द्रि॒य मे॒वैवेन्द्रि॒य मि॑न्द्रि॒य मे॒व । \newline
29. ए॒वात्मन्-ना॒त्म-ने॒वैवात्मन्न् । \newline
30. आ॒त्मन् ध॑त्ते धत्त आ॒त्मन्-ना॒त्मन् ध॑त्ते । \newline
31. ध॒त्ते॒ सा सा ध॑त्ते धत्ते॒ सा । \newline
32. सा वै वै सा सा वै । \newline
33. वा ए॒षैषा वै वा ए॒षा । \newline
34. ए॒षा त्र॑या॒णाम् त्र॑या॒णा मे॒षैषा त्र॑या॒णाम् । \newline
35. त्र॒या॒णा मे॒वैव त्र॑या॒णाम् त्र॑या॒णा मे॒व । \newline
36. ए॒वा व॑रु॒द्धा ऽव॑रुद्धै॒वैवा व॑रुद्धा । \newline
37. अव॑रुद्धा संॅवथ्सर॒सदः॑ संॅवथ्सर॒सदो ऽव॑रु॒द्धा ऽव॑रुद्धा संॅवथ्सर॒सदः॑ । \newline
38. अव॑रु॒द्धेत्यव॑ - रु॒द्धा॒ । \newline
39. सं॒ॅव॒थ्स॒र॒सदः॑ सहस्रया॒जिनः॑ सहस्रया॒जिनः॑ संॅवथ्सर॒सदः॑ संॅवथ्सर॒सदः॑ सहस्रया॒जिनः॑ । \newline
40. सं॒ॅव॒थ्स॒र॒सद॒ इति॑ संॅवथ्सर - सदः॑ । \newline
41. स॒ह॒स्र॒या॒जिनो॑ गृहमे॒धिनो॑ गृहमे॒धिनः॑ सहस्रया॒जिनः॑ सहस्रया॒जिनो॑ गृहमे॒धिनः॑ । \newline
42. स॒ह॒स्र॒या॒जिन॒ इति॑ सहस्र - या॒जिनः॑ । \newline
43. गृ॒ह॒मे॒धिन॒ स्ते ते गृ॑हमे॒धिनो॑ गृहमे॒धिन॒ स्ते । \newline
44. गृ॒ह॒मे॒धिन॒ इति॑ गृह - मे॒धिनः॑ । \newline
45. त ए॒वैव ते त ए॒व । \newline
46. ए॒वैतयै॒ तयै॒ वैवैतया᳚ । \newline
47. ए॒तया॑ यजेरन्. यजेरन्-ने॒तयै॒तया॑ यजेरन्न् । \newline
48. य॒जे॒र॒न् तेषा॒म् तेषां᳚ ॅयजेरन्. यजेर॒न् तेषा᳚म् । \newline
49. तेषा॑ मे॒वैव तेषा॒म् तेषा॑ मे॒व । \newline
50. ए॒वै षैषै वैवैषा । \newline
51. ए॒षा ऽऽप्ता ऽऽप्तै षैषा ऽऽप्ता । \newline
52. आ॒प्तेत्या॒प्ता । \newline

\textbf{Ghana Paata } \newline

1. भव॑ति॒ यद् यद् भव॑ति॒ भव॑ति॒ यदाल॑ब्धाया॒ माल॑ब्धाया॒म् ॅयद् भव॑ति॒ भव॑ति॒ यदाल॑ब्धायाम् । \newline
2. यदाल॑ब्धाया॒ माल॑ब्धाया॒म् ॅयद् यदाल॑ब्धाया म॒भ्रो᳚ ऽभ्र आल॑ब्धाया॒म् ॅयद् यदाल॑ब्धाया म॒भ्रः । \newline
3. आल॑ब्धाया म॒भ्रो᳚ ऽभ्र आल॑ब्धाया॒ माल॑ब्धाया म॒भ्रः स्याथ् स्या द॒भ्र आल॑ब्धाया॒ माल॑ब्धाया म॒भ्रः स्यात् । \newline
4. आल॑ब्धाया॒मित्या - ल॒ब्धा॒या॒म् । \newline
5. अ॒भ्रः स्याथ् स्या द॒भ्रो᳚ ऽभ्रः स्या द॒फ्स्व॑फ्सु स्या द॒भ्रो᳚ ऽभ्रः स्या द॒फ्सु । \newline
6. स्या द॒फ्स्व॑फ्सु स्याथ् स्या द॒फ्सु वा॑ वा॒ ऽफ्सु स्याथ् स्या द॒फ्सु वा᳚ । \newline
7. अ॒फ्सु वा॑ वा॒ ऽफ्स्व॑फ्सु वा᳚ प्रवे॒शये᳚त् प्रवे॒शये᳚द् वा॒ ऽफ्स्व॑फ्सु वा᳚ प्रवे॒शये᳚त् । \newline
8. अ॒फ्स्वित्य॑प् - सु । \newline
9. वा॒ प्र॒वे॒शये᳚त् प्रवे॒शये᳚द् वा वा प्रवे॒शये॒थ् सर्वाꣳ॒॒ सर्वा᳚म् प्रवे॒शये᳚द् वा वा प्रवे॒शये॒थ् सर्वा᳚म् । \newline
10. प्र॒वे॒शये॒थ् सर्वाꣳ॒॒ सर्वा᳚म् प्रवे॒शये᳚त् प्रवे॒शये॒थ् सर्वा᳚म् ॅवा वा॒ सर्वा᳚म् प्रवे॒शये᳚त् प्रवे॒शये॒थ् सर्वा᳚म् ॅवा । \newline
11. प्र॒वे॒शये॒दिति॑ प्र - वे॒शये᳚त् । \newline
12. सर्वा᳚म् ॅवा वा॒ सर्वाꣳ॒॒ सर्वा᳚म् ॅवा॒ प्र प्र वा॒ सर्वाꣳ॒॒ सर्वा᳚म् ॅवा॒ प्र । \newline
13. वा॒ प्र प्र वा॑ वा॒ प्राश्ञी॑या दश्ञीया॒त् प्र वा॑ वा॒ प्राश्ञी॑यात् । \newline
14. प्राश्ञी॑या दश्ञीया॒त् प्र प्राश्ञी॑या॒द् यद् यद॑श्ञीया॒त् प्र प्राश्ञी॑या॒द् यत् । \newline
15. अ॒श्ञी॒या॒द् यद् यद॑श्ञीया दश्ञीया॒द् यद॒ फ्स्व॑फ्सु यद॑श्ञीया दश्ञीया॒द् यद॒फ्सु । \newline
16. Yअद॒ फ्स्व॑फ्सु यद् यद॒फ्सु प्र॑वे॒शये᳚त् प्रवे॒शये॑ द॒फ्सु यद् यद॒फ्सु प्र॑वे॒शये᳚त् । \newline
17. अ॒फ्सु प्र॑वे॒शये᳚त् प्रवे॒शये॑द॒ फ्स्व॑फ्सु प्र॑वे॒शये᳚द् यज्ञ्वेश॒सम् ॅय॑ज्ञ्वेश॒सम् प्र॑वे॒शये॑
द॒फ्स्व॑फ्सु प्र॑वे॒शये᳚द् यज्ञ्वेश॒सम् । \newline
18. अ॒फ्स्वित्य॑प् - सु । \newline
19. प्र॒वे॒शये᳚द् यज्ञ्वेश॒सम् ॅय॑ज्ञ्वेश॒सम् प्र॑वे॒शये᳚त् प्रवे॒शये᳚द् यज्ञ्वेश॒सम् कु॑र्यात् कुर्याद् यज्ञ्वेश॒सम् प्र॑वे॒शये᳚त् प्रवे॒शये᳚द् यज्ञ्वेश॒सम् कु॑र्यात् । \newline
20. प्र॒वे॒शये॒दिति॑ प्र - वे॒शये᳚त् । \newline
21. य॒ज्ञ्॒वे॒श॒सम् कु॑र्यात् कुर्याद् यज्ञ्वेश॒सम् ॅय॑ज्ञ्वेश॒सम् कु॑र्या॒थ् सर्वाꣳ॒॒ सर्वा᳚म् कुर्याद् यज्ञ्वेश॒सम् ॅय॑ज्ञ्वेश॒सम् कु॑र्या॒थ् सर्वा᳚म् । \newline
22. य॒ज्ञ्॒वे॒श॒समिति॑ यज्ञ् - वे॒श॒सम् । \newline
23. कु॒र्या॒थ् सर्वाꣳ॒॒ सर्वा᳚म् कुर्यात् कुर्या॒थ् सर्वा॑ मे॒वैव सर्वा᳚म् कुर्यात् कुर्या॒थ् सर्वा॑ मे॒व । \newline
24. सर्वा॑ मे॒वैव सर्वाꣳ॒॒ सर्वा॑ मे॒व प्र प्रैव सर्वाꣳ॒॒ सर्वा॑ मे॒व प्र । \newline
25. ए॒व प्र प्रैवैव प्राश्ञी॑या दश्ञीया॒त् प्रैवैव प्राश्ञी॑यात् । \newline
26. प्राश्ञी॑या दश्ञीया॒त् प्र प्राश्ञी॑या दिन्द्रि॒य मि॑न्द्रि॒य म॑श्ञीया॒त् प्र प्राश्ञी॑या दिन्द्रि॒यम् । \newline
27. अ॒श्ञी॒या॒ दि॒न्द्रि॒य मि॑न्द्रि॒य म॑श्ञीया दश्ञीया दिन्द्रि॒य मे॒वैवेन्द्रि॒य म॑श्ञीया दश्ञीया दिन्द्रि॒य मे॒व । \newline
28. इ॒न्द्रि॒य मे॒वैवेन्द्रि॒य मि॑न्द्रि॒य मे॒वात्मन् ना॒त्म ने॒वेन्द्रि॒य मि॑न्द्रि॒य मे॒वात्मन्न् । \newline
29. ए॒वात्मन् ना॒त्मन् ने॒वैवात्मन् ध॑त्ते धत्त आ॒त्मन् ने॒वैवात्मन् ध॑त्ते । \newline
30. आ॒त्मन् ध॑त्ते धत्त आ॒त्मन् ना॒त्मन् ध॑त्ते॒ सा सा ध॑त्त आ॒त्मन् ना॒त्मन् ध॑त्ते॒ सा । \newline
31. ध॒त्ते॒ सा सा ध॑त्ते धत्ते॒ सा वै वै सा ध॑त्ते धत्ते॒ सा वै । \newline
32. सा वै वै सा सा वा ए॒षैषावै सा सा वा ए॒षा । \newline
33. वा ए॒षैषावै वा ए॒षा त्र॑या॒णाम् त्र॑या॒णा मे॒षावैवा ए॒षा त्र॑या॒णाम् । \newline
34. ए॒षा त्र॑या॒णाम् त्र॑या॒णा मे॒षैषा त्र॑या॒णा मे॒वैव त्र॑या॒णा मे॒षैषा त्र॑या॒णा मे॒व । \newline
35. त्र॒या॒णा मे॒वैव त्र॑या॒णाम् त्र॑या॒णा मे॒वा व॑रु॒द्धा ऽव॑रुद्धै॒व त्र॑या॒णाम् त्र॑या॒णा मे॒वा व॑रुद्धा । \newline
36. ए॒वा व॑रु॒द्धा ऽव॑रुद्धै॒ वैवा व॑रुद्धा सम्ॅवथ्सर॒सदः॑ सम्ॅवथ्सर॒सदो ऽव॑रुद्धै॒ वैवा व॑रुद्धा सम्ॅवथ्सर॒सदः॑ । \newline
37. अव॑रुद्धा सम्ॅवथ्सर॒सदः॑ सम्ॅवथ्सर॒सदो ऽव॑रु॒द्धा ऽव॑रुद्धा सम्ॅवथ्सर॒सदः॑ सहस्रया॒जिनः॑ सहस्रया॒जिनः॑ सम्ॅवथ्सर॒सदो ऽव॑रु॒द्धा ऽव॑रुद्धा सम्ॅवथ्सर॒सदः॑ सहस्रया॒जिनः॑ । \newline
38. अव॑रु॒द्धेत्यव॑ - रु॒द्धा॒ । \newline
39. स॒म्ॅव॒थ्स॒र॒सदः॑ सहस्रया॒जिनः॑ सहस्रया॒जिनः॑ सम्ॅवथ्सर॒सदः॑ सम्ॅवथ्सर॒सदः॑ सहस्रया॒जिनो॑ गृहमे॒धिनो॑ गृहमे॒धिनः॑ सहस्रया॒जिनः॑ सम्ॅवथ्सर॒सदः॑ सम्ॅवथ्सर॒सदः॑ सहस्रया॒जिनो॑ गृहमे॒धिनः॑ । \newline
40. स॒म्ॅव॒थ्स॒र॒सद॒ इति॑ सम्ॅवथ्सर - सदः॑ । \newline
41. स॒ह॒स्र॒या॒जिनो॑ गृहमे॒धिनो॑ गृहमे॒धिनः॑ सहस्रया॒जिनः॑ सहस्रया॒जिनो॑ गृहमे॒धिन॒ स्ते ते गृ॑हमे॒धिनः॑ सहस्रया॒जिनः॑ सहस्रया॒जिनो॑ गृहमे॒धिन॒ स्ते । \newline
42. स॒ह॒स्र॒या॒जिन॒ इति॑ सहस्र - या॒जिनः॑ । \newline
43. गृ॒ह॒मे॒धिन॒ स्ते ते गृ॑हमे॒धिनो॑ गृहमे॒धिन॒ स्त ए॒वैव ते गृ॑हमे॒धिनो॑ गृहमे॒धिन॒ स्त ए॒व । \newline
44. गृ॒ह॒मे॒धिन॒ इति॑ गृह - मे॒धिनः॑ । \newline
45. त ए॒वैव ते त ए॒वैत यै॒त यै॒व ते त ए॒वैतया᳚ । \newline
46. ए॒वैत यै॒त यै॒वैवैतया॑ यजेरन्. यजेरन्,ने॒त यै॒वैवैतया॑ यजेरन्न् । \newline
47. ए॒तया॑ यजेरन्. यजेरन्,ने॒त यै॒तया॑ यजेर॒न् तेषा॒म् तेषा᳚म् ॅयजेरन्,ने॒त यै॒तया॑ यजेर॒न् तेषा᳚म् । \newline
48. य॒जे॒र॒न् तेषा॒म् तेषा᳚म् ॅयजेरन्. यजेर॒न् तेषा॑ मे॒वैव तेषा᳚म् ॅयजेरन्. यजेर॒न् तेषा॑ मे॒व । \newline
49. तेषा॑ मे॒वैव तेषा॒म् तेषा॑ मे॒वैषै षैव तेषा॒म् तेषा॑ मे॒वैषा । \newline
50. ए॒वै षैषै वैवैषा ऽऽप्ता ऽऽप्तैषै वैवैषा ऽऽप्ता । \newline
51. ए॒षा ऽऽप्ता ऽऽप्तै षैषा ऽऽप्ता । \newline
52. आ॒प्तेत्या॒प्ता । \newline
\pagebreak
\markright{ TS 3.4.4.1  \hfill https://www.vedavms.in \hfill}

\section{ TS 3.4.4.1 }

\textbf{TS 3.4.4.1 } \newline
\textbf{Samhita Paata} \newline

चि॒त्तं च॒ चित्ति॒श्चा ऽऽ*कू॑तं॒ चाऽऽ*कू॑तिश्च॒ विज्ञा॑तं च वि॒ज्ञानं॑ च॒ मन॑श्च॒ शक्व॑रीश्च॒ दर्.श॑श्च पू॒र्णमा॑सश्च बृ॒हच्च॑ रथन्त॒रं च॑ प्र॒जाप॑ति॒र्जया॒निन्द्रा॑य॒ वृष्णे॒ प्राय॑च्छदु॒ग्रः पृ॑त॒नाज्ये॑षु॒ तस्मै॒ विशः॒ सम॑नमन्त॒ सर्वाः॒ स उ॒ग्रः सहि हव्यो॑ ब॒भूव॑देवासु॒राः संॅय॑त्ता आस॒न्थ्स इन्द्रः॑ प्र॒जाप॑ति॒मुपा॑ ( ) धाव॒त् तस्मा॑ ए॒ताञ्जया॒न् प्राय॑च्छ॒त् तान॑जुहो॒त् ततो॒ वै दे॒वा असु॑रानजय॒न॒. यदज॑य॒न् तज्जया॑नां जय॒त्वꣳ स्पर्द्ध॑मानेनै॒ते हो॑त॒व्या॑ जय॑त्ये॒व तां पृत॑नां ॥ \newline

\textbf{Pada Paata} \newline

चि॒त्तम् । च॒ । चित्तिः॑ । च॒ । आकू॑त॒मित्या - कू॒त॒म् । च॒ । आकू॑ति॒रित्या - कू॒तिः॒ । च॒ । विज्ञा॑त॒मिति॒ वि - ज्ञा॒त॒म् । च॒ । वि॒ज्ञान॒मिति॑ वि-ज्ञान᳚म् । च॒ । मनः॑ । च॒ । शक्व॑रीः । च॒ । दर्.शः॑ । च॒ । पू॒र्णमा॑स॒ इति॑ पू॒र्ण - मा॒सः॒ । च॒ । बृ॒हत् । च॒ । र॒थ॒न्त॒रमिति॑ रथं - त॒रम् । च॒ । प्र॒जाप॑ति॒रिति॑ प्र॒जा - प॒तिः॒ । जयान्॑ । इन्द्रा॑य । वृष्णे᳚ । प्रेति॑ । अ॒य॒च्छ॒त् । उ॒ग्रः । पृ॒त॒नाज्ये॑षु । तस्मै᳚ । विशः॑ । समिति॑ । अ॒न॒म॒न्त॒ । सर्वाः᳚ । सः । उ॒ग्रः । सः । हि । हव्यः॑ । ब॒भूव॑ । दे॒वा॒सु॒रा इति॑ देव - अ॒सु॒राः । संॅय॑त्ता॒ इति॒ सं - य॒त्ताः॒ । आ॒स॒न्न् । सः । इन्द्रः॑ । प्र॒जाप॑ति॒मिति॑ प्र॒जा - प॒ति॒म् । उपेति॑ ( ) । अ॒धा॒व॒त् । तस्मै᳚ । ए॒तान् । जयान्॑ । प्रेति॑ । अ॒य॒च्छ॒त् । तान् । अ॒जु॒हो॒त् । ततः॑ । वै । दे॒वाः । असु॑रान् । अ॒ज॒य॒न्न् । यत् । अज॑यन्न् । तत् । जया॑नाम् । ज॒य॒त्वमिति॑ जय - त्वम् । स्पर्द्ध॑मानेन । ए॒ते । हो॒त॒व्याः᳚ । जय॑ति । ए॒व । ताम् । पृत॑नाम् ॥  \newline


\textbf{Krama Paata} \newline

चि॒त्तम् च॑ । च॒ चित्तिः॑ । चित्ति॑श्च । चाकू॑तम् । आकू॑तम् च । आकू॑त॒मित्या - कू॒त॒म् । चाकू॑तिः । आकू॑तिश्च । आकू॑ति॒रित्या - कू॒तिः॒ । च॒ विज्ञा॑तम् । विज्ञा॑तम् च । विज्ञा॑त॒मिति॒ वि - ज्ञा॒त॒म् । च॒ वि॒ज्ञान᳚म् । वि॒ज्ञानं॑ च । वि॒ज्ञान॒मिति॑ वि - ज्ञान᳚म् । च॒ मनः॑ । मन॑श्च । च॒ शक्व॑रीः । शक्व॑रीश्च । च॒ दर्.शः॑ । दर्.श॑श्च । च॒ पू॒र्णमा॑सः । पू॒र्णमा॑सश्च । पू॒र्णमा॑स॒ इति॑ पू॒र्ण - मा॒सः॒ । च॒ बृ॒हत् । बृ॒हच् च॑ । च॒ र॒थ॒न्त॒रम् । र॒थ॒न्त॒रम् च॑ । र॒थ॒न्त॒रमिति॑ रथं - त॒रम् । च॒ प्र॒जाप॑तिः । प्र॒जाप॑ति॒र् जयान्॑ । प्र॒जाप॑ति॒रिति॑ प्र॒जा - प॒तिः॒ । जया॒निन्द्रा॑य । इन्द्रा॑य॒ वृष्णे᳚ । वृष्णे॒ प्र । प्राय॑च्छत् । अ॒य॒च्छ॒दु॒ग्रः । उ॒ग्रः पृ॑त॒नाज्ये॑षु । पृ॒त॒नाज्ये॑षु॒ तस्मै᳚ । तस्मै॒ विशः॑ । विशः॒ सम् । सम॑नमन्त । अ॒न॒म॒न्त॒ सर्वाः᳚ । सर्वाः॒ सः । स उ॒ग्रः । उ॒ग्रः सः । स हि । हि हव्यः॑ । हव्यो॑ ब॒भूव॑ । ब॒भूव॑ देवासु॒राः । दे॒वा॒सु॒राः सम्ॅय॑त्ताः । दे॒वा॒सु॒रा इति॑ देव - अ॒सु॒राः । सम्ॅय॑त्ता आसन्न् । सम्ॅय॑त्ता॒ इति॒ सं - य॒त्ताः॒ । आ॒स॒न्थ् सः । स इन्द्रः॑ । इन्द्रः॑ प्र॒जाप॑तिम् । प्र॒जाप॑ति॒मुप॑ ( ) । प्र॒जाप॑ति॒मिति॑ प्र॒जा - प॒ति॒म् । उपा॑धावत् । अ॒धा॒व॒त् तस्मै᳚ । तस्मा॑ ए॒तान् । ए॒तान् जयान्॑ । जया॒न् प्र । प्राय॑च्छत् । अ॒य॒च्छ॒त् तान् । तान॑जुहोत् । अ॒जु॒हो॒त् ततः॑ । ततो॒ वै । वै दे॒वाः । दे॒वा असु॑रान् । असु॑रानजयन्न् । अ॒ज॒य॒न्॒. यत् । यदज॑यन्न् । अज॑य॒न् तत् । तज् जया॑नाम् । जया॑नाम् जय॒त्वम् । ज॒य॒त्वꣳ स्पर्द्ध॑मानेन । ज॒य॒त्वमिति॑ जय - त्वम् । स्पर्द्ध॑मानेनै॒ते । ए॒ते हो॑त॒व्याः᳚ । हो॒त॒व्या॑ जय॑ति । जय॑त्ये॒व । ए॒व ताम् । ताम् पृत॑नाम् । पृत॑ना॒मिति॒ पृत॑नाम् । \newline

\textbf{Jatai Paata} \newline

1. चि॒त्तम् च॑ च चि॒त्तम् चि॒त्तम् च॑ । \newline
2. च॒ चित्ति॒ श्चित्ति॑श्च च॒ चित्तिः॑ । \newline
3. चित्ति॑श्च च॒ चित्ति॒ श्चित्ति॑श्च । \newline
4. चाकू॑त॒ माकू॑तम् च॒ चाकू॑तम् । \newline
5. आकू॑तम् च॒ चाकू॑त॒ माकू॑तम् च । \newline
6. आकू॑त॒मित्या - कू॒त॒म् । \newline
7. चाकू॑ति॒ राकू॑तिश्च॒ चाकू॑तिः । \newline
8. आकू॑तिश्च॒ चाकू॑ति॒ राकू॑तिश्च । \newline
9. आकू॑ति॒रित्या - कू॒तिः॒ । \newline
10. च॒ विज्ञा॑तं॒ ॅविज्ञा॑तम् च च॒ विज्ञा॑तम् । \newline
11. विज्ञा॑तम् च च॒ विज्ञा॑तं॒ ॅविज्ञा॑तम् च । \newline
12. विज्ञा॑त॒मिति॒ वि - ज्ञा॒त॒म् । \newline
13. च॒ वि॒ज्ञानं॑ ॅवि॒ज्ञान॑म् च च वि॒ज्ञान᳚म् । \newline
14. वि॒ज्ञान॑म् च च वि॒ज्ञानं॑ ॅवि॒ज्ञान॑म् च । \newline
15. वि॒ज्ञान॒मिति॑ वि - ज्ञान᳚म् । \newline
16. च॒ मनो॒ मन॑श्च च॒ मनः॑ । \newline
17. मन॑श्च च॒ मनो॒ मन॑श्च । \newline
18. च॒ शक्व॑रीः॒ शक्व॑रीश्च च॒ शक्व॑रीः । \newline
19. शक्व॑रीश्च च॒ शक्व॑रीः॒ शक्व॑रीश्च । \newline
20. च॒ दर्.शो॒ दर्.श॑श्च च॒ दर्.शः॑ । \newline
21. दर्.श॑श्च च॒ दर्.शो॒ दर्.श॑श्च । \newline
22. च॒ पू॒र्णमा॑सः पू॒र्णमा॑सश्च च पू॒र्णमा॑सः । \newline
23. पू॒र्णमा॑सश्च च पू॒र्णमा॑सः पू॒र्णमा॑सश्च । \newline
24. पू॒र्णमा॑स॒ इति॑ पू॒र्ण - मा॒सः॒ । \newline
25. च॒ बृ॒हद् बृ॒हच् च॑ च बृ॒हत् । \newline
26. बृ॒हच् च॑ च बृ॒हद् बृ॒हच् च॑ । \newline
27. च॒ र॒थ॒न्त॒रꣳ र॑थन्त॒रम् च॑ च रथन्त॒रम् । \newline
28. र॒थ॒न्त॒रम् च॑ च रथन्त॒रꣳ र॑थन्त॒रम् च॑ । \newline
29. र॒थ॒न्त॒रमिति॑ रथं - त॒रम् । \newline
30. च॒ प्र॒जाप॑तिः प्र॒जाप॑तिश्च च प्र॒जाप॑तिः । \newline
31. प्र॒जाप॑ति॒र् जया॒न् जया᳚न् प्र॒जाप॑तिः प्र॒जाप॑ति॒र् जयान्॑ । \newline
32. प्र॒जाप॑ति॒रिति॑ प्र॒जा - प॒तिः॒ । \newline
33. जया॒-निन्द्रा॒ येन्द्रा॑य॒ जया॒न् जया॒-निन्द्रा॑य । \newline
34. इन्द्रा॑य॒ वृष्णे॒ वृष्ण॒ इन्द्रा॒ येन्द्रा॑य॒ वृष्णे᳚ । \newline
35. वृष्णे॒ प्र प्र वृष्णे॒ वृष्णे॒ प्र । \newline
36. प्रा य॑च्छ दयच्छ॒त् प्र प्रा य॑च्छत् । \newline
37. अ॒य॒च्छ॒ दु॒ग्र उ॒ग्रो॑ ऽयच्छ दयच्छ दु॒ग्रः । \newline
38. उ॒ग्रः पृ॑त॒नाज्ये॑षु पृत॒नाज्ये॑षू॒ग्र उ॒ग्रः पृ॑त॒नाज्ये॑षु । \newline
39. पृ॒त॒नाज्ये॑षु॒ तस्मै॒ तस्मै॑ पृत॒नाज्ये॑षु पृत॒नाज्ये॑षु॒ तस्मै᳚ । \newline
40. तस्मै॒ विशो॒ विश॒ स्तस्मै॒ तस्मै॒ विशः॑ । \newline
41. विशः॒ सꣳ सं ॅविशो॒ विशः॒ सम् । \newline
42. स म॑नमन्ता नमन्त॒ सꣳ स म॑नमन्त । \newline
43. अ॒न॒म॒न्त॒ सर्वाः॒ सर्वा॑ अनमन्ता नमन्त॒ सर्वाः᳚ । \newline
44. सर्वाः॒ स स सर्वाः॒ सर्वाः॒ सः । \newline
45. स उ॒ग्र उ॒ग्रः स स उ॒ग्रः । \newline
46. उ॒ग्रः स स उ॒ग्र उ॒ग्रः सः । \newline
47. स हि हि स स हि । \newline
48. हि हव्यो॒ हव्यो॒ हि हि हव्यः॑ । \newline
49. हव्यो॑ ब॒भूव॑ ब॒भूव॒ हव्यो॒ हव्यो॑ ब॒भूव॑ । \newline
50. ब॒भूव॑ देवासु॒रा दे॑वासु॒रा ब॒भूव॑ ब॒भूव॑ देवासु॒राः । \newline
51. दे॒वा॒सु॒राः संॅय॑त्ताः॒ संॅय॑त्ता देवासु॒रा दे॑वासु॒राः संॅय॑त्ताः । \newline
52. दे॒वा॒सु॒रा इति॑ देव - अ॒सु॒राः । \newline
53. सम्ॅय॑त्ता आसन्-नास॒न् थ्संॅय॑त्ताः॒ सम्ॅय॑त्ता आसन्न् । \newline
54. संॅय॑त्ता॒ इति॒ सं - य॒त्ताः॒ । \newline
55. आ॒स॒न् थ्स स आ॑सन्-नास॒न् थ्सः । \newline
56. स इन्द्र॒ इन्द्रः॒ स स इन्द्रः॑ । \newline
57. इन्द्रः॑ प्र॒जाप॑तिम् प्र॒जाप॑ति॒ मिन्द्र॒ इन्द्रः॑ प्र॒जाप॑तिम् । \newline
58. प्र॒जाप॑ति॒ मुपोप॑ प्र॒जाप॑तिम् प्र॒जाप॑ति॒ मुप॑ । \newline
59. प्र॒जाप॑ति॒मिति॑ प्र॒जा - प॒ति॒म् । \newline
60. उपा॑ धाव दधाव॒ दुपोपा॑ धावत् । \newline
61. अ॒धा॒व॒त् तस्मै॒ तस्मा॑ अधाव दधाव॒त् तस्मै᳚ । \newline
62. तस्मा॑ ए॒ता-ने॒तान् तस्मै॒ तस्मा॑ ए॒तान् । \newline
63. ए॒तान् जया॒न् जया॑-ने॒ता-ने॒तान् जयान्॑ । \newline
64. जया॒न् प्र प्र जया॒न् जया॒न् प्र । \newline
65. प्राय॑च्छ दयच्छ॒त् प्र प्राय॑च्छत् । \newline
66. अ॒य॒च्छ॒त् ताꣳ स्ता,न॑यच्छ दयच्छ॒त् तान् । \newline
67. तान॑जुहो दजुहो॒त् ताꣳ स्ता-न॑जुहोत् । \newline
68. अ॒जु॒हो॒त् तत॒ स्ततो॑ ऽजुहो दजुहो॒त् ततः॑ । \newline
69. ततो॒ वै वै तत॒ स्ततो॒ वै । \newline
70. वै दे॒वा दे॒वा वै वै दे॒वाः । \newline
71. दे॒वा असु॑रा॒-नसु॑रान् दे॒वा दे॒वा असु॑रान् । \newline
72. असु॑रा-नजयन्-नजय॒न्-नसु॑रा॒-नसु॑रा-नजयन्न् । \newline
73. अ॒ज॒य॒न्॒. यद् यद॑जयन्-नजय॒न्॒. यत् । \newline
74. यदज॑य॒न्-नज॑य॒न्॒. यद् यदज॑यन्न् । \newline
75. अज॑य॒न् तत् तदज॑य॒न्-नज॑य॒न् तत् । \newline
76. तज् जया॑ना॒म् जया॑ना॒म् तत् तज् जया॑नाम् । \newline
77. जया॑नाम् जय॒त्वम् ज॑य॒त्वम् जया॑ना॒म् जया॑नाम् जय॒त्वम् । \newline
78. ज॒य॒त्वꣳ स्पर्द्ध॑मानेन॒ स्पर्द्ध॑मानेन जय॒त्वम् ज॑य॒त्वꣳ स्पर्द्ध॑मानेन । \newline
79. ज॒य॒त्वमिति॑ जय - त्वम् । \newline
80. स्पर्द्ध॑माने नै॒त ए॒ते स्पर्द्ध॑मानेन॒ स्पर्द्ध॑माने नै॒ते । \newline
81. ए॒ते हो॑त॒व्या॑ होत॒व्या॑ ए॒त ए॒ते हो॑त॒व्याः᳚ । \newline
82. हो॒त॒व्या॑ जय॑ति॒ जय॑ति होत॒व्या॑ होत॒व्या॑ जय॑ति । \newline
83. जय॑ त्ये॒वैव जय॑ति॒ जय॑ त्ये॒व । \newline
84. ए॒व ताम् ता मे॒वैव ताम् । \newline
85. ताम् पृत॑ना॒म् पृत॑ना॒म् ताम् ताम् पृत॑नाम् । \newline
86. पृत॑ना॒मिति॒ पृत॑नाम् । \newline

\textbf{Ghana Paata } \newline

1. चि॒त्तम् च॑ च चि॒त्तम् चि॒त्तम् च॒ चित्ति॒ श्चित्ति॑श्च चि॒त्तम् चि॒त्तम् च॒ चित्तिः॑ । \newline
2. च॒ चित्ति॒ श्चित्ति॑श्च च॒ चित्ति॑श्च च॒ चित्ति॑श्च च॒ चित्ति॑श्च । \newline
3. चित्ति॑श्च च॒ चित्ति॒ श्चित्ति॒ श्चाकू॑त॒ माकू॑तम् च॒ चित्ति॒ श्चित्ति॒ श्चाकू॑तम् । \newline
4. चाकू॑त॒ माकू॑तम् च॒ चाकू॑तम् च॒ चाकू॑तम् च॒ चाकू॑तम् च । \newline
5. आकू॑तम् च॒ चाकू॑त॒ माकू॑त॒म् चाकू॑ति॒ राकू॑ति॒ श्चाकू॑त॒ माकू॑त॒म् चाकू॑तिः । \newline
6. आकू॑त॒मित्या - कू॒त॒म् । \newline
7. चाकू॑ति॒ राकू॑तिश्च॒ चाकू॑तिश्च॒ चाकू॑तिश्च॒ चाकू॑तिश्च । \newline
8. आकू॑तिश्च॒ चाकू॑ति॒ राकू॑तिश्च॒ विज्ञा॑त॒म् ॅविज्ञा॑त॒म् चाकू॑ति॒ राकू॑तिश्च॒ विज्ञा॑तम् । \newline
9. आकू॑ति॒रित्या - कू॒तिः॒ । \newline
10. च॒ विज्ञा॑त॒म् ॅविज्ञा॑तम् च च॒ विज्ञा॑तम् च च॒ विज्ञा॑तम् च च॒ विज्ञा॑तम् च । \newline
11. विज्ञा॑तम् च च॒ विज्ञा॑त॒म् ॅविज्ञा॑तम् च वि॒ज्ञान॑म् ॅवि॒ज्ञान॑म् च॒ विज्ञा॑त॒म् ॅविज्ञा॑तम् च वि॒ज्ञान᳚म् । \newline
12. विज्ञा॑त॒मिति॒ वि - ज्ञा॒त॒म् । \newline
13. च॒ वि॒ज्ञान॑म् ॅवि॒ज्ञान॑म् च च वि॒ज्ञान॑म् च च वि॒ज्ञान॑म् च च वि॒ज्ञान॑म् च । \newline
14. वि॒ज्ञान॑म् च च वि॒ज्ञान॑म् ॅवि॒ज्ञान॑म् च॒ मनो॒ मन॑श्च वि॒ज्ञान॑म् ॅवि॒ज्ञान॑म् च॒ मनः॑ । \newline
15. वि॒ज्ञान॒मिति॑ वि - ज्ञान᳚म् । \newline
16. च॒ मनो॒ मन॑श्च च॒ मन॑श्च च॒ मन॑श्च च॒ मन॑श्च । \newline
17. मन॑श्च च॒ मनो॒ मन॑श्च॒ शक्व॑रीः॒ शक्व॑रीश्च॒ मनो॒ मन॑श्च॒ शक्व॑रीः । \newline
18. च॒ शक्व॑रीः॒ शक्व॑रीश्च च॒ शक्व॑रीश्च च॒ शक्व॑रीश्च च॒ शक्व॑रीश्च । \newline
19. शक्व॑रीश्च च॒ शक्व॑रीः॒ शक्व॑रीश्च॒ दर्.शो॒ दर्.श॑श्च॒ शक्व॑रीः॒ शक्व॑रीश्च॒ दर्.शः॑ । \newline
20. च॒ दर्.शो॒ दर्.श॑श्च च॒ दर्.श॑श्च च॒ दर्.श॑श्च च॒ दर्.श॑श्च । \newline
21. दर्.श॑श्च च॒ दर्.शो॒ दर्.श॑श्च पू॒र्णमा॑सः पू॒र्णमा॑सश्च॒ दर्.शो॒ दर्.श॑श्च पू॒र्णमा॑सः । \newline
22. च॒ पू॒र्णमा॑सः पू॒र्णमा॑सश्च च पू॒र्णमा॑सश्च च पू॒र्णमा॑सश्च च पू॒र्णमा॑सश्च । \newline
23. पू॒र्णमा॑सश्च च पू॒र्णमा॑सः पू॒र्णमा॑सश्च बृ॒हद् बृ॒हच् च॑ पू॒र्णमा॑सः पू॒र्णमा॑सश्च बृ॒हत् । \newline
24. पू॒र्णमा॑स॒ इति॑ पू॒र्ण - मा॒सः॒ । \newline
25. च॒ बृ॒हद् बृ॒हच् च॑ च बृ॒हच् च॑ च बृ॒हच् च॑ च बृ॒हच् च॑ । \newline
26. बृ॒हच् च॑ च बृ॒हद् बृ॒हच् च॑ रथन्त॒रꣳ र॑थन्त॒रम् च॑ बृ॒हद् बृ॒हच् च॑ रथन्त॒रम् । \newline
27. च॒ र॒थ॒न्त॒रꣳ र॑थन्त॒रम् च॑ च रथन्त॒रम् च॑ च रथन्त॒रम् च॑ च रथन्त॒रम् च॑ । \newline
28. र॒थ॒न्त॒रम् च॑ च रथन्त॒रꣳ र॑थन्त॒रम् च॑ प्र॒जाप॑तिः प्र॒जाप॑तिश्च रथन्त॒रꣳ र॑थन्त॒रम् च॑ प्र॒जाप॑तिः । \newline
29. र॒थ॒न्त॒रमिति॑ रथम् - त॒रम् । \newline
30. च॒ प्र॒जाप॑तिः प्र॒जाप॑तिश्च च प्र॒जाप॑ति॒र् जया॒न् जया᳚न् प्र॒जाप॑तिश्च च प्र॒जाप॑ति॒र् जयान्॑ । \newline
31. प्र॒जाप॑ति॒र् जया॒न् जया᳚न् प्र॒जाप॑तिः प्र॒जाप॑ति॒र् जया॒,निन्द्रा॒ येन्द्रा॑य॒ जया᳚न् प्र॒जाप॑तिः प्र॒जाप॑ति॒र् जया॒,निन्द्रा॑य । \newline
32. प्र॒जाप॑ति॒रिति॑ प्र॒जा - प॒तिः॒ । \newline
33. जया॒,निन्द्रा॒ येन्द्रा॑य॒ जया॒न् जया॒,निन्द्रा॑य॒ वृष्णे॒ वृष्ण॒ इन्द्रा॑य॒ जया॒न् जया॒,निन्द्रा॑य॒ वृष्णे᳚ । \newline
34. इन्द्रा॑य॒ वृष्णे॒ वृष्ण॒ इन्द्रा॒ये न्द्रा॑य॒ वृष्णे॒ प्र प्र वृष्ण॒ इन्द्रा॒ येन्द्रा॑य॒ वृष्णे॒ प्र । \newline
35. वृष्णे॒ प्र प्र वृष्णे॒ वृष्णे॒ प्राय॑च्छ दयच्छ॒त् प्र वृष्णे॒ वृष्णे॒ प्राय॑च्छत् । \newline
36. प्राय॑च्छ दयच्छ॒त् प्र प्राय॑च्छ दु॒ग्र उ॒ग्रो॑ ऽयच्छ॒त् प्र प्राय॑च्छ दु॒ग्रः । \newline
37. अ॒य॒च्छ॒ दु॒ग्र उ॒ग्रो॑ ऽयच्छ दयच्छ दु॒ग्रः पृ॑त॒नाज्ये॑षु पृत॒नाज्ये॑ षू॒ग्रो॑ ऽयच्छद यच्छ दु॒ग्रः पृ॑त॒नाज्ये॑षु । \newline
38. उ॒ग्रः पृ॑त॒नाज्ये॑षु पृत॒नाज्ये॑ षू॒ग्र उ॒ग्रः पृ॑त॒नाज्ये॑षु॒ तस्मै॒ तस्मै॑ पृत॒नाज्ये॑ षू॒ग्र उ॒ग्रः पृ॑त॒नाज्ये॑षु॒ तस्मै᳚ । \newline
39. पृ॒त॒नाज्ये॑षु॒ तस्मै॒ तस्मै॑ पृत॒नाज्ये॑षु पृत॒नाज्ये॑षु॒ तस्मै॒ विशो॒ विश॒ स्तस्मै॑ पृत॒नाज्ये॑षु पृत॒नाज्ये॑षु॒ तस्मै॒ विशः॑ । \newline
40. तस्मै॒ विशो॒ विश॒ स्तस्मै॒ तस्मै॒ विशः॒ सꣳ सम् ॅविश॒ स्तस्मै॒ तस्मै॒ विशः॒ सम् । \newline
41. विशः॒ सꣳ सम् ॅविशो॒ विशः॒ स म॑नमन्ता नमन्त॒ सम् ॅविशो॒ विशः॒ स म॑नमन्त । \newline
42. स म॑नमन्ता नमन्त॒ सꣳ स म॑नमन्त॒ सर्वाः॒ सर्वा॑ अनमन्त॒ सꣳ स म॑नमन्त॒ सर्वाः᳚ । \newline
43. अ॒न॒म॒न्त॒ सर्वाः॒ सर्वा॑ अनमन्ता नमन्त॒ सर्वाः॒ स स सर्वा॑ अनमन्ता नमन्त॒ सर्वाः॒ सः । \newline
44. सर्वाः॒ स स सर्वाः॒ सर्वाः॒ स उ॒ग्र उ॒ग्रः स सर्वाः॒ सर्वाः॒ स उ॒ग्रः । \newline
45. स उ॒ग्र उ॒ग्रः स स उ॒ग्रः स स उ॒ग्रः स स उ॒ग्रः सः । \newline
46. उ॒ग्रः स स उ॒ग्र उ॒ग्रः स हि हि स उ॒ग्र उ॒ग्रः स हि । \newline
47. स हि हि स स हि हव्यो॒ हव्यो॒ हि स स हि हव्यः॑ । \newline
48. हि हव्यो॒ हव्यो॒ हि हि हव्यो॑ ब॒भूव॑ ब॒भूव॒ हव्यो॒ हि हि हव्यो॑ ब॒भूव॑ । \newline
49. हव्यो॑ ब॒भूव॑ ब॒भूव॒ हव्यो॒ हव्यो॑ ब॒भूव॑ देवासु॒रा दे॑वासु॒रा ब॒भूव॒ हव्यो॒ हव्यो॑ ब॒भूव॑ देवासु॒राः । \newline
50. ब॒भूव॑ देवासु॒रा दे॑वासु॒रा ब॒भूव॑ ब॒भूव॑ देवासु॒राः सम्ॅय॑त्ताः॒ सम्ॅय॑त्ता देवासु॒रा ब॒भूव॑ 
ब॒भूव॑ देवासु॒राः सम्ॅय॑त्ताः । \newline
51. दे॒वा॒सु॒राः सम्ॅय॑त्ताः॒ सम्ॅय॑त्ता देवासु॒रा दे॑वासु॒राः सम्ॅय॑त्ता आसन्,नास॒न् थ्सम्ॅय॑त्ता देवासु॒रा दे॑वासु॒राः सम्ॅय॑त्ता आसन्न् । \newline
52. दे॒वा॒सु॒रा इति॑ देव - अ॒सु॒राः । \newline
53. सम्ॅय॑त्ता आसन्,नास॒न् थ्सम्ॅय॑त्ताः॒ सम्ॅय॑त्ता आस॒न् थ्स स आ॑स॒न् थ्सम्ॅय॑त्ताः॒ सम्ॅय॑त्ता आस॒न् थ्सः । \newline
54. सम्ॅय॑त्ता॒ इति॒ सम् - य॒त्ताः॒ । \newline
55. आ॒स॒न् थ्स स आ॑सन्,नास॒न् थ्स इन्द्र॒ इन्द्रः॒ स आ॑सन्,नास॒न् थ्स इन्द्रः॑ । \newline
56. स इन्द्र॒ इन्द्रः॒ स स इन्द्रः॑ प्र॒जाप॑तिम् प्र॒जाप॑ति॒ मिन्द्रः॒ स स इन्द्रः॑ प्र॒जाप॑तिम् । \newline
57. इन्द्रः॑ प्र॒जाप॑तिम् प्र॒जाप॑ति॒ मिन्द्र॒ इन्द्रः॑ प्र॒जाप॑ति॒ मुपोप॑ प्र॒जाप॑ति॒ मिन्द्र॒ इन्द्रः॑ प्र॒जाप॑ति॒ मुप॑ । \newline
58. प्र॒जाप॑ति॒ मुपोप॑ प्र॒जाप॑तिम् प्र॒जाप॑ति॒ मुपा॑धाव दधाव॒ दुप॑ प्र॒जाप॑तिम् प्र॒जाप॑ति॒ मुपा॑धावत् । \newline
59. प्र॒जाप॑ति॒मिति॑ प्र॒जा - प॒ति॒म् । \newline
60. उपा॑धाव दधाव॒ दुपोपा॑धाव॒त् तस्मै॒ तस्मा॑ अधाव॒ दुपोपा॑धाव॒त् तस्मै᳚ । \newline
61. अ॒धा॒व॒त् तस्मै॒ तस्मा॑ अधाव दधाव॒त् तस्मा॑ ए॒ता,ने॒तान् तस्मा॑ अधाव दधाव॒त् तस्मा॑ ए॒तान् । \newline
62. तस्मा॑ ए॒ता ने॒तान् तस्मै॒ तस्मा॑ ए॒तान् जया॒न् जया॑,ने॒तान् तस्मै॒ तस्मा॑ ए॒तान् जयान्॑ । \newline
63. ए॒तान् जया॒न् जया॑,ने॒ता,ने॒तान् जया॒न् प्र प्र जया॑,ने॒ता,ने॒तान् जया॒न् प्र । \newline
64. जया॒न् प्र प्र जया॒न् जया॒न् प्राय॑च्छ दयच्छ॒त् प्र जया॒न् जया॒न् प्राय॑च्छत् । \newline
65. प्राय॑च्छ दयच्छ॒त् प्र प्राय॑च्छ॒त् ताꣳ स्तान॑यच्छ॒त् प्र प्राय॑च्छ॒त् तान् । \newline
66. अ॒य॒च्छ॒त् ताꣳ स्तान॑यच्छ दयच्छ॒त् तान॑जुहो दजुहो॒त् तान॑ यच्छ दयच्छ॒त् तान॑जुहोत् । \newline
67. तान॑जुहो दजुहो॒त् ताꣳ स्ता,न॑जुहो॒त् तत॒ स्ततो॑ ऽजुहो॒त् ताꣳ स्तान॑जुहो॒त् ततः॑ । \newline
68. अ॒जु॒हो॒त् तत॒ स्ततो॑ ऽजुहो दजुहो॒त् ततो॒ वै वै ततो॑ ऽजुहो दजुहो॒त् ततो॒ वै । \newline
69. ततो॒ वै वै तत॒ स्ततो॒ वै दे॒वा दे॒वा वै तत॒ स्ततो॒ वै दे॒वाः । \newline
70. वै दे॒वा दे॒वा वै वै दे॒वा असु॑रा॒ नसु॑रान् दे॒वा वै वै दे॒वा असु॑रान् । \newline
71. दे॒वा असु॑रा॒ नसु॑रान् दे॒वा दे॒वा असु॑रा नजयन्,नजय॒न्,नसु॑रान् दे॒वा दे॒वा असु॑रा नजयन्न् । \newline
72. असु॑रा,नजयन्,नजय॒न्,नसु॑रा॒,नसु॑रा,नजय॒न्॒. यद् यद॑जय॒न्,नसु॑रा॒,नसु॑रा,नजय॒न्॒. यत् । \newline
73. अ॒ज॒य॒न्॒. यद् यद॑जयन्,नजय॒न्॒. यदज॑य॒न्,नज॑य॒न्॒. यद॑जयन्,नजय॒न्॒. यदज॑यन्न् । \newline
74. यदज॑य॒न्,नज॑य॒न्॒. यद् यदज॑य॒न् तत् तदज॑य॒न्॒. यद् यदज॑य॒न् तत् । \newline
75. अज॑य॒न् तत् तदज॑य॒न्,नज॑य॒न् तज् जया॑ना॒म् जया॑ना॒म् तदज॑य॒न्,नज॑य॒न् तज् जया॑नाम् । \newline
76. तज् जया॑ना॒म् जया॑ना॒म् तत् तज् जया॑नाम् जय॒त्वम् ज॑य॒त्वम् जया॑ना॒म् तत् तज् जया॑नाम् जय॒त्वम् । \newline
77. जया॑नाम् जय॒त्वम् ज॑य॒त्वम् जया॑ना॒म् जया॑नाम् जय॒त्वꣳ स्पर्द्ध॑मानेन॒ स्पर्द्ध॑मानेन जय॒त्वम् जया॑ना॒म् जया॑नाम् जय॒त्वꣳ स्पर्द्ध॑मानेन । \newline
78. ज॒य॒त्वꣳ स्पर्द्ध॑मानेन॒ स्पर्द्ध॑मानेन जय॒त्वम् ज॑य॒त्वꣳ स्पर्द्ध॑मानेनै॒त ए॒ते स्पर्द्ध॑मानेन जय॒त्वम् ज॑य॒त्वꣳ स्पर्द्ध॑मानेनै॒ते । \newline
79. ज॒य॒त्वमिति॑ जय - त्वम् । \newline
80. स्पर्द्ध॑माने नै॒त ए॒ते स्पर्द्ध॑मानेन॒ स्पर्द्ध॑माने नै॒ते हो॑त॒व्या॑ होत॒व्या॑ ए॒ते स्पर्द्ध॑मानेन॒ स्पर्द्ध॑माने नै॒ते हो॑त॒व्याः᳚ । \newline
81. ए॒ते हो॑त॒व्या॑ होत॒व्या॑ ए॒त ए॒ते हो॑त॒व्या॑ जय॑ति॒ जय॑ति होत॒व्या॑ ए॒त ए॒ते हो॑त॒व्या॑ जय॑ति । \newline
82. हो॒त॒व्या॑ जय॑ति॒ जय॑ति होत॒व्या॑ होत॒व्या॑ जय॑ त्ये॒वैव जय॑ति होत॒व्या॑ होत॒व्या॑ जय॑ त्ये॒व । \newline
83. जय॑ त्ये॒वैव जय॑ति॒ जय॑ त्ये॒व ताम् ता मे॒व जय॑ति॒ जय॑ त्ये॒व ताम् । \newline
84. ए॒व ताम् ता मे॒वैव ताम् पृत॑ना॒म् पृत॑ना॒म् ता मे॒वैव ताम् पृत॑नाम् । \newline
85. ताम् पृत॑ना॒म् पृत॑ना॒म् ताम् ताम् पृत॑नाम् । \newline
86. पृत॑ना॒मिति॒ पृत॑नाम् । \newline
\pagebreak
\markright{ TS 3.4.5.1  \hfill https://www.vedavms.in \hfill}

\section{ TS 3.4.5.1 }

\textbf{TS 3.4.5.1 } \newline
\textbf{Samhita Paata} \newline

अ॒ग्निर्भू॒ताना॒मधि॑पतिः॒ समा॑ऽव॒त्विन्द्रो᳚ ज्ये॒ष्ठानां᳚ ॅय॒मः पृ॑थि॒व्या वा॒युर॒न्तरि॑क्षस्य॒ सूर्यो॑दि॒वश्च॒न्द्रमा॒ नक्ष॑त्राणां॒ बृह॒स्पति॒र्ब्रह्म॑णो मि॒त्रः स॒त्यानां॒ ॅवरु॑णो॒ऽपाꣳ स॑मु॒द्रः स्रो॒त्याना॒मन्नꣳ॒॒ साम्रा᳚ज्याना॒मधि॑पति॒ तन्मा॑ऽवतु॒ सोम॒ ओष॑धीनाꣳ सवि॒ता प्र॑स॒वानाꣳ॑ रु॒द्रः प॑शू॒नां त्वष्टा॑ रू॒पाणां॒ ॅविष्णुः॒ पर्व॑तानां म॒रुतो॑ ग॒णाना॒मधि॑पतय॒स्ते मा॑वन्तु॒ पित॑रः पितामहाः परेऽवरे॒ ( ) तता᳚स्ततामहा इ॒ह मा॑ऽवत । अ॒स्मिन् ब्रह्म॑न्न॒स्मिन् क्ष॒त्रे᳚ऽस्या-मा॒शिष्य॒स्यां पु॑रो॒धाया॑म॒स्मिन्-कर्म॑न्न॒स्यां दे॒वहू᳚त्यां ॥ \newline

\textbf{Pada Paata} \newline

अ॒ग्निः । भू॒ताना᳚म् । अधि॑पति॒रित्यधि॑- प॒तिः॒ । सः । मा॒ । अ॒व॒तु॒ । इन्द्रः॑ । ज्ये॒ष्ठाना᳚म् । य॒मः । पृ॒थि॒व्याः । वा॒युः । अ॒न्तरि॑क्षस्य । सूर्यः॑ । दि॒वः । च॒न्द्रमाः᳚ । नक्ष॑त्राणाम् । बृह॒स्पतिः॑ । ब्रह्म॑णः । मि॒त्रः । स॒त्याना᳚म् । वरु॑णः । अ॒पाम् । स॒मु॒द्रः । स्रो॒त्याना᳚म् । अन्न᳚म् । साम्रा᳚ज्याना॒मिति॒ सां - रा॒ज्या॒ना॒म् । अधि॑प॒तीत्यधि॑-प॒ति॒ । तत् । मा॒ । अ॒व॒तु॒ । सोमः॑ । ओष॑धीनाम् । स॒वि॒ता । प्र॒स॒वाना॒मिति॑ प्र - स॒वाना᳚म् । रु॒द्रः । प॒शू॒नाम् । त्वष्टा᳚ । रू॒पाणा᳚म् । विष्णुः॑ । पर्व॑तानाम् । म॒रुतः॑ । ग॒णाना᳚म् । अधि॑पतय॒ इत्यधि॑ - प॒त॒यः॒ । ते । मा॒ । अ॒व॒न्तु॒ । पित॑रः । पि॒ता॒म॒हाः॒ । प॒रे॒ । अ॒व॒रे॒ ( ) । तताः᳚ । त॒ता॒म॒हाः॒ । इ॒ह । मा॒ । अ॒व॒त॒ ॥ अ॒स्मिन्न् । ब्रह्मन्न्॑ । अ॒स्मिन्न् । क्ष॒त्रे । अ॒स्याम् । आ॒शिषीत्या᳚ - शिषि॑ । अ॒स्याम् । पु॒रो॒धाया॒मिति॑ पुरः - धाया᳚म् । अ॒स्मिन्न् । कर्मन्न्॑ । अ॒स्याम् । दे॒वहू᳚त्या॒मिति॑ दे॒व - हू॒त्या॒म् ॥  \newline


\textbf{Krama Paata} \newline

अ॒ग्निर् भू॒ताना᳚म् । भू॒ताना॒मधि॑पतिः । अधि॑पतिः॒ सः । अधि॑पति॒रित्यधि॑ - प॒तिः॒ । स मा᳚ । मा॒ ऽव॒तु॒ । अ॒व॒त्विन्द्रः॑ । इन्द्रो᳚ ज्ये॒ष्ठाना᳚म् । ज्ये॒ष्ठानां᳚ ॅय॒मः । य॒मः पृ॑थि॒व्याः । पृ॒थि॒व्या वा॒युः । वा॒युर॒न्तरि॑क्षस्य । अ॒न्तरि॑क्षस्य॒ सूर्यः॑ । सूर्यो॑ दि॒वः । दि॒वश्च॒न्द्रमाः᳚ । च॒न्द्रमा॒ नक्ष॑त्राणाम् । नक्ष॑त्राणा॒म् बृह॒स्पतिः॑ । बृह॒स्पति॒र् ब्रह्म॑णः । ब्रह्म॑णो मि॒त्रः । मि॒त्रः स॒त्याना᳚म् । स॒त्यानां॒ ॅवरु॑णः । वरु॑णो॒ ऽपाम् । अ॒पाꣳ स॑मु॒द्रः । स॒मु॒द्रः स्रो॒त्याना᳚म् । स्रो॒त्याना॒मन्न᳚म् । अन्नꣳ॒॒ साम्रा᳚ज्यानाम् । साम्रा᳚ज्याना॒मधि॑पति । साम्रा᳚ज्याना॒मिति॒ साम् - रा॒ज्या॒ना॒म् । अधि॑पति॒ तत् । अधि॑प॒तीत्यधि॑ - प॒ति॒ । तन्मा᳚ । मा॒ ऽव॒तु॒ । अ॒व॒तु॒ सोमः॑ । सोम॒ ओष॑धीनाम् । ओष॑धीनाꣳ सवि॒ता । स॒वि॒ता प्र॑स॒वाना᳚म् । प्र॒स॒वानाꣳ॑ रु॒द्रः । प्र॒स॒वाना॒मिति॑ प्र - स॒वाना᳚म् । रु॒द्रः प॑शू॒नाम् । प॒शू॒नाम् त्वष्टा᳚ । त्वष्टा॑ रू॒पाणा᳚म् । रू॒पाणां॒ ॅविष्णुः॑ । विष्णुः॒ पर्व॑तानाम् । पर्व॑तानां म॒रुतः॑ । म॒रुतो॑ ग॒णाना᳚म् । ग॒णाना॒मधि॑पतयः । 
अधि॑पतय॒स्ते । अधि॑पतय॒ इत्यधि॑ - प॒त॒यः॒ । ते मा᳚ । 
मा॒ ऽव॒न्तु॒ । अ॒व॒न्तु॒ पित॑रः । पित॑रः पितामहाः । पि॒ता॒म॒हाः॒ प॒रे॒ । प॒रे॒ ऽव॒रे॒ ( ) । अ॒व॒रे॒ तताः᳚ । तता᳚स्ततामहाः । त॒ता॒म॒हा॒ इ॒ह । इ॒ह मा᳚ । मा॒ ऽव॒त॒ । अ॒व॒तेत्य॑वत ॥ अ॒स्मिन् ब्रह्मन्न्॑ । ब्रह्म॑न्न॒स्मिन्न् । अ॒स्मिन् क्ष॒त्रे । क्ष॒त्रे᳚ ऽस्याम् । अ॒स्यामा॒शिषि॑ । आ॒शिष्य॒स्याम् । आ॒शिषीत्या᳚ - शिषि॑ । अ॒स्याम् पु॑रो॒धाया᳚म् । पु॒रो॒धाया॑म॒स्मिन्न् । पु॒रो॒धाया॒मिति॑ पुरः - धाया᳚म् । अ॒स्मिन् कर्मन्न्॑ । कर्म॑न्न॒स्याम् । अ॒स्याम् दे॒वहू᳚त्याम् । दे॒वहू᳚त्या॒मिति॑ दे॒व - हू॒त्या॒म् । \newline

\textbf{Jatai Paata} \newline

1. अ॒ग्निर् भू॒ताना᳚म् भू॒ताना॑ म॒ग्नि र॒ग्निर् भू॒ताना᳚म् । \newline
2. भू॒ताना॒ मधि॑पति॒ रधि॑पतिर् भू॒ताना᳚म् भू॒ताना॒ मधि॑पतिः । \newline
3. अधि॑पतिः॒ स सो ऽधि॑पति॒ रधि॑पतिः॒ सः । \newline
4. अधि॑पति॒रित्यधि॑ - प॒तिः॒ । \newline
5. स मा॑ मा॒ स स मा᳚ । \newline
6. मा॒ ऽव॒ त्व॒व॒तु॒ मा॒ मा॒ ऽव॒तु॒ । \newline
7. अ॒व॒त्विन्द्र॒ इन्द्रो॑ ऽवत्व व॒त्विन्द्रः॑ । \newline
8. इन्द्रो᳚ ज्ये॒ष्ठाना᳚म् ज्ये॒ष्ठाना॒ मिन्द्र॒ इन्द्रो᳚ ज्ये॒ष्ठाना᳚म् । \newline
9. ज्ये॒ष्ठानां᳚ ॅय॒मो य॒मो ज्ये॒ष्ठाना᳚म् ज्ये॒ष्ठानां᳚ ॅय॒मः । \newline
10. य॒मः पृ॑थि॒व्याः पृ॑थि॒व्या य॒मो य॒मः पृ॑थि॒व्याः । \newline
11. पृ॒थि॒व्या वा॒युर् वा॒युः पृ॑थि॒व्याः पृ॑थि॒व्या वा॒युः । \newline
12. वा॒यु र॒न्तरि॑क्षस्या॒ न्तरि॑क्षस्य वा॒युर् वा॒यु र॒न्तरि॑क्षस्य । \newline
13. अ॒न्तरि॑क्षस्य॒ सूर्यः॒ सूर्यो॒ ऽन्तरि॑क्षस्या॒ न्तरि॑क्षस्य॒ सूर्यः॑ । \newline
14. सूर्यो॑ दि॒वो दि॒वः सूर्यः॒ सूर्यो॑ दि॒वः । \newline
15. दि॒व श्च॒न्द्रमा᳚ श्च॒न्द्रमा॑ दि॒वो दि॒व श्च॒न्द्रमाः᳚ । \newline
16. च॒न्द्रमा॒ नक्ष॑त्राणा॒म् नक्ष॑त्राणाम् च॒न्द्रमा᳚ श्च॒न्द्रमा॒ नक्ष॑त्राणाम् । \newline
17. नक्ष॑त्राणा॒म् बृह॒स्पति॒र् बृह॒स्पति॒र् नक्ष॑त्राणा॒म् नक्ष॑त्राणा॒म् बृह॒स्पतिः॑ । \newline
18. बृह॒स्पति॒र् ब्रह्म॑णो॒ ब्रह्म॑णो॒ बृह॒स्पति॒र् बृह॒स्पति॒र् ब्रह्म॑णः । \newline
19. ब्रह्म॑णो मि॒त्रो मि॒त्रो ब्रह्म॑णो॒ ब्रह्म॑णो मि॒त्रः । \newline
20. मि॒त्रः स॒त्यानाꣳ॑ स॒त्याना᳚म् मि॒त्रो मि॒त्रः स॒त्याना᳚म् । \newline
21. स॒त्यानां॒ ॅवरु॑णो॒ वरु॑णः स॒त्यानाꣳ॑ स॒त्यानां॒ ॅवरु॑णः । \newline
22. वरु॑णो॒ ऽपा म॒पां ॅवरु॑णो॒ वरु॑णो॒ ऽपाम् । \newline
23. अ॒पाꣳ स॑मु॒द्रः स॑मु॒द्रो॑ ऽपा म॒पाꣳ स॑मु॒द्रः । \newline
24. स॒मु॒द्रः स्रो॒त्यानाꣳ॑ स्रो॒त्यानाꣳ॑ समु॒द्रः स॑मु॒द्रः स्रो॒त्याना᳚म् । \newline
25. स्रो॒त्याना॒ मन्न॒ मन्नꣳ॑ स्रो॒त्यानाꣳ॑ स्रो॒त्याना॒ मन्न᳚म् । \newline
26. अन्नꣳ॒॒ साम्रा᳚ज्यानाꣳ॒॒ साम्रा᳚ज्याना॒ मन्न॒ मन्नꣳ॒॒ साम्रा᳚ज्यानाम् । \newline
27. साम्रा᳚ज्याना॒ मधि॑प॒त्य धि॑पति॒ साम्रा᳚ज्यानाꣳ॒॒ साम्रा᳚ज्याना॒ मधि॑पति । \newline
28. साम्रा᳚ज्याना॒मिति॒ सां - रा॒ज्या॒ना॒म् । \newline
29. अधि॑पति॒ तत् तदधि॑प॒त्य धि॑पति॒ तत् । \newline
30. अधि॑प॒तीत्यधि॑ - प॒ति॒ । \newline
31. तन् मा॑ मा॒ तत् तन् मा᳚ । \newline
32. मा॒ ऽव॒ त्व॒व॒तु॒ मा॒ मा॒ ऽव॒तु॒ । \newline
33. अ॒व॒तु॒ सोमः॒ सोमो॑ ऽवत्ववतु॒ सोमः॑ । \newline
34. सोम॒ ओष॑धीना॒ मोष॑धीनाꣳ॒॒ सोमः॒ सोम॒ ओष॑धीनाम् । \newline
35. ओष॑धीनाꣳ सवि॒ता स॑वि॒तौ ष॑धीना॒ मोष॑धीनाꣳ सवि॒ता । \newline
36. स॒वि॒ता प्र॑स॒वाना᳚म् प्रस॒वानाꣳ॑ सवि॒ता स॑वि॒ता प्र॑स॒वाना᳚म् । \newline
37. प्र॒स॒वानाꣳ॑ रु॒द्रो रु॒द्रः प्र॑स॒वाना᳚म् प्रस॒वानाꣳ॑ रु॒द्रः । \newline
38. प्र॒स॒वाना॒मिति॑ प्र - स॒वाना᳚म् । \newline
39. रु॒द्रः प॑शू॒नाम् प॑शू॒नाꣳ रु॒द्रो रु॒द्रः प॑शू॒नाम् । \newline
40. प॒शू॒नाम् त्वष्टा॒ त्वष्टा॑ पशू॒नाम् प॑शू॒नाम् त्वष्टा᳚ । \newline
41. त्वष्टा॑ रू॒पाणाꣳ॑ रू॒पाणा॒म् त्वष्टा॒ त्वष्टा॑ रू॒पाणा᳚म् । \newline
42. रू॒पाणां॒ ॅविष्णु॒र् विष्णू॑ रू॒पाणाꣳ॑ रू॒पाणां॒ ॅविष्णुः॑ । \newline
43. विष्णुः॒ पर्व॑ताना॒म् पर्व॑तानां॒ ॅविष्णु॒र् विष्णुः॒ पर्व॑तानाम् । \newline
44. पर्व॑तानाम् म॒रुतो॑ म॒रुतः॒ पर्व॑ताना॒म् पर्व॑तानाम् म॒रुतः॑ । \newline
45. म॒रुतो॑ ग॒णाना᳚म् ग॒णाना᳚म् म॒रुतो॑ म॒रुतो॑ ग॒णाना᳚म् । \newline
46. ग॒णाना॒ मधि॑पत॒यो ऽधि॑पतयो ग॒णाना᳚म् ग॒णाना॒ मधि॑पतयः । \newline
47. अधि॑पतय॒ स्ते ते ऽधि॑पत॒यो ऽधि॑पतय॒ स्ते । \newline
48. अधि॑पतय॒ इत्यधि॑ - प॒त॒यः॒ । \newline
49. ते मा॑ मा॒ ते ते मा᳚ । \newline
50. मा॒ ऽव॒न्त्व॒व॒न्तु॒ मा॒ मा॒ ऽव॒न्तु॒ । \newline
51. अ॒व॒न्तु॒ पित॑रः॒ पित॑रो ऽवन्त्ववन्तु॒ पित॑रः । \newline
52. पित॑रः पितामहाः पितामहाः॒ पित॑रः॒ पित॑रः पितामहाः । \newline
53. पि॒ता॒म॒हाः॒ प॒रे॒ प॒रे॒ पि॒ता॒म॒हाः॒ पि॒ता॒म॒हाः॒ प॒रे॒ । \newline
54. प॒रे॒ ऽव॒रे॒ ऽव॒रे॒ प॒रे॒ प॒रे॒ ऽव॒रे॒ । \newline
55. अ॒व॒रे॒ तता॒ स्तता॑ अवरे ऽवरे॒ तताः᳚ । \newline
56. तता᳚ स्ततामहा स्ततामहा॒ स्तता॒ स्तता᳚ स्ततामहाः । \newline
57. त॒ता॒म॒हा॒ इ॒हेह त॑तामहा स्ततामहा इ॒ह । \newline
58. इ॒ह मा॑ मे॒हेह मा᳚ । \newline
59. मा॒ ऽव॒ता॒ व॒त॒ मा॒ मा॒ ऽव॒त॒ । \newline
60. अ॒व॒तेत्य॑वत । \newline
61. अ॒स्मिन् ब्रह्म॒न् ब्रह्म॑न्-न॒स्मिन्-न॒स्मिन् ब्रह्मन्न्॑ । \newline
62. ब्रह्म॑न्-न॒स्मिन्-न॒स्मिन् ब्रह्म॒न् ब्रह्म॑न्-न॒स्मिन्न् । \newline
63. अ॒स्मिन् क्ष॒त्रे क्ष॒त्रे᳚ ऽस्मिन्-न॒स्मिन् क्ष॒त्रे । \newline
64. क्ष॒त्रे᳚ ऽस्या म॒स्याम् क्ष॒त्रे क्ष॒त्रे᳚ ऽस्याम् । \newline
65. अ॒स्या मा॒शि ष्या॒शि ष्य॒स्या म॒स्या मा॒शिषि॑ । \newline
66. आ॒शिष्य॒स्या म॒स्या मा॒शि ष्या॒शिष्य॒स्याम् । \newline
67. आ॒शिषीत्या᳚ - शिषि॑ । \newline
68. अ॒स्याम् पु॑रो॒धाया᳚म् पुरो॒धाया॑ म॒स्या म॒स्याम् पु॑रो॒धाया᳚म् । \newline
69. पु॒रो॒धाया॑ म॒स्मिन्-न॒स्मिन् पु॑रो॒धाया᳚म् पुरो॒धाया॑ म॒स्मिन्न् । \newline
70. पु॒रो॒धाया॒मिति॑ पुरः - धाया᳚म् । \newline
71. अ॒स्मिन् कर्म॒न् कर्म॑न्-न॒स्मिन्-न॒स्मिन् कर्मन्न्॑ । \newline
72. कर्म॑न्-न॒स्या म॒स्याम् कर्म॒न् कर्म॑न्-न॒स्याम् । \newline
73. अ॒स्याम् दे॒वहू᳚त्याम् दे॒वहू᳚त्या म॒स्या म॒स्याम् दे॒वहू᳚त्याम् । \newline
74. दे॒वहू᳚त्या॒मिति॑ दे॒व - हू॒त्या॒म् । \newline

\textbf{Ghana Paata } \newline

1. अ॒ग्निर् भू॒ताना᳚म् भू॒ताना॑ म॒ग्नि र॒ग्निर् भू॒ताना॒ मधि॑पति॒ रधि॑पतिर् भू॒ताना॑ म॒ग्नि र॒ग्निर् भू॒ताना॒ मधि॑पतिः । \newline
2. भू॒ताना॒ मधि॑पति॒ रधि॑पतिर् भू॒ताना᳚म् भू॒ताना॒ मधि॑पतिः॒ स सो ऽधि॑पतिर् भू॒ताना᳚म् भू॒ताना॒ मधि॑पतिः॒ सः । \newline
3. अधि॑पतिः॒ स सो ऽधि॑पति॒ रधि॑पतिः॒ स मा॑ मा॒ सो ऽधि॑पति॒ रधि॑पतिः॒ स मा᳚ । \newline
4. अधि॑पति॒रित्यधि॑ - प॒तिः॒ । \newline
5. स मा॑ मा॒ स स मा॑ ऽवत्व वतु मा॒ स स मा॑ ऽवतु । \newline
6. मा॒ ऽव॒त्व॒ व॒तु॒ मा॒ मा॒ ऽव॒त्विन्द्र॒ इन्द्रो॑ ऽवतु मा मा ऽव॒त्विन्द्रः॑ । \newline
7. अ॒व॒त्विन्द्र॒ इन्द्रो॑ ऽवत्वव॒ त्विन्द्रो᳚ ज्ये॒ष्ठाना᳚म् ज्ये॒ष्ठाना॒ मिन्द्रो॑ ऽवत्वव॒ त्विन्द्रो᳚ ज्ये॒ष्ठाना᳚म् । \newline
8. इन्द्रो᳚ ज्ये॒ष्ठाना᳚म् ज्ये॒ष्ठाना॒ मिन्द्र॒ इन्द्रो᳚ ज्ये॒ष्ठाना᳚म् ॅय॒मो य॒मो ज्ये॒ष्ठाना॒ मिन्द्र॒ इन्द्रो᳚ ज्ये॒ष्ठाना᳚म् ॅय॒मः । \newline
9. ज्ये॒ष्ठाना᳚म् ॅय॒मो य॒मो ज्ये॒ष्ठाना᳚म् ज्ये॒ष्ठाना᳚म् ॅय॒मः पृ॑थि॒व्याः पृ॑थि॒व्या य॒मो ज्ये॒ष्ठाना᳚म् ज्ये॒ष्ठाना᳚म् ॅय॒मः पृ॑थि॒व्याः । \newline
10. य॒मः पृ॑थि॒व्याः पृ॑थि॒व्या य॒मो य॒मः पृ॑थि॒व्या वा॒युर् वा॒युः पृ॑थि॒व्या य॒मो य॒मः पृ॑थि॒व्या वा॒युः । \newline
11. पृ॒थि॒व्या वा॒युर् वा॒युः पृ॑थि॒व्याः पृ॑थि॒व्या वा॒यु र॒न्तरि॑क्षस्या॒ न्तरि॑क्षस्य वा॒युः पृ॑थि॒व्याः 
पृ॑थि॒व्या वा॒यु र॒न्तरि॑क्षस्य । \newline
12. वा॒यु र॒न्तरि॑क्षस्या॒ न्तरि॑क्षस्य वा॒युर् वा॒यु र॒न्तरि॑क्षस्य॒ सूर्यः॒ सूर्यो॒ ऽन्तरि॑क्षस्य वा॒युर् वा॒यु
र॒न्तरि॑क्षस्य॒ सूर्यः॑ । \newline
13. अ॒न्तरि॑क्षस्य॒ सूर्यः॒ सूर्यो॒ ऽन्तरि॑क्षस्या॒ न्तरि॑क्षस्य॒ सूर्यो॑ दि॒वो दि॒वः सूर्यो॒ ऽन्तरि॑क्षस्या॒ न्तरि॑क्षस्य॒ सूर्यो॑ दि॒वः । \newline
14. सूर्यो॑ दि॒वो दि॒वः सूर्यः॒ सूर्यो॑ दि॒व श्च॒न्द्रमा᳚ श्च॒न्द्रमा॑ दि॒वः सूर्यः॒ सूर्यो॑ दि॒व श्च॒न्द्रमाः᳚ । \newline
15. दि॒व श्च॒न्द्रमा᳚ श्च॒न्द्रमा॑ दि॒वो दि॒व श्च॒न्द्रमा॒ नक्ष॑त्राणा॒न् नक्ष॑त्राणाम् च॒न्द्रमा॑ दि॒वो दि॒व श्च॒न्द्रमा॒ नक्ष॑त्राणाम् । \newline
16. च॒न्द्रमा॒ नक्ष॑त्राणा॒न् नक्ष॑त्राणाम् च॒न्द्रमा᳚ श्च॒न्द्रमा॒ नक्ष॑त्राणा॒म् बृह॒स्पति॒र् बृह॒स्पति॒र् नक्ष॑त्राणाम् च॒न्द्रमा᳚ श्च॒न्द्रमा॒ नक्ष॑त्राणा॒म् बृह॒स्पतिः॑ । \newline
17. नक्ष॑त्राणा॒म् बृह॒स्पति॒र् बृह॒स्पति॒र् नक्ष॑त्राणा॒म् नक्ष॑त्राणा॒म् बृह॒स्पति॒र् ब्रह्म॑णो॒ ब्रह्म॑णो॒ बृह॒स्पति॒र् नक्ष॑त्राणा॒म् नक्ष॑त्राणा॒म् बृह॒स्पति॒र् ब्रह्म॑णः । \newline
18. बृह॒स्पति॒र् ब्रह्म॑णो॒ ब्रह्म॑णो॒ बृह॒स्पति॒र् बृह॒स्पति॒र् ब्रह्म॑णो मि॒त्रो मि॒त्रो ब्रह्म॑णो॒ बृह॒स्पति॒र् बृह॒स्पति॒र् ब्रह्म॑णो मि॒त्रः । \newline
19. ब्रह्म॑णो मि॒त्रो मि॒त्रो ब्रह्म॑णो॒ ब्रह्म॑णो मि॒त्रः स॒त्यानाꣳ॑ स॒त्याना᳚म् मि॒त्रो ब्रह्म॑णो॒ ब्रह्म॑णो मि॒त्रः स॒त्याना᳚म् । \newline
20. मि॒त्रः स॒त्यानाꣳ॑ स॒त्याना᳚म् मि॒त्रो मि॒त्रः स॒त्याना॒म् ॅवरु॑णो॒ वरु॑णः स॒त्याना᳚म् मि॒त्रो मि॒त्रः स॒त्याना॒म् ॅवरु॑णः । \newline
21. स॒त्याना॒म् ॅवरु॑णो॒ वरु॑णः स॒त्यानाꣳ॑ स॒त्याना॒म् ॅवरु॑णो॒ ऽपा म॒पाम् ॅवरु॑णः स॒त्यानाꣳ॑ स॒त्याना॒म् ॅवरु॑णो॒ ऽपाम् । \newline
22. वरु॑णो॒ ऽपा म॒पाम् ॅवरु॑णो॒ वरु॑णो॒ ऽपाꣳ स॑मु॒द्रः स॑मु॒द्रो॑ ऽपाम् ॅवरु॑णो॒ वरु॑णो॒ ऽपाꣳ स॑मु॒द्रः । \newline
23. अ॒पाꣳ स॑मु॒द्रः स॑मु॒द्रो॑ ऽपा म॒पाꣳ स॑मु॒द्रः स्रो॒त्यानाꣳ॑ स्रो॒त्यानाꣳ॑ समु॒द्रो॑ ऽपा म॒पाꣳ 
स॑मु॒द्रः स्रो॒त्याना᳚म् । \newline
24. स॒मु॒द्रः स्रो॒त्यानाꣳ॑ स्रो॒त्यानाꣳ॑ समु॒द्रः स॑मु॒द्रः स्रो॒त्याना॒ मन्न॒ मन्नꣳ॑ स्रो॒त्यानाꣳ॑ समु॒द्रः स॑मु॒द्रः स्रो॒त्याना॒ मन्न᳚म् । \newline
25. स्रो॒त्याना॒ मन्न॒ मन्नꣳ॑ स्रो॒त्यानाꣳ॑ स्रो॒त्याना॒ मन्नꣳ॒॒ साम्रा᳚ज्यानाꣳ॒॒ साम्रा᳚ज्याना॒ मन्नꣳ॑ स्रो॒त्यानाꣳ॑ स्रो॒त्याना॒ मन्नꣳ॒॒ साम्रा᳚ज्यानाम् । \newline
26. अन्नꣳ॒॒ साम्रा᳚ज्यानाꣳ॒॒ साम्रा᳚ज्याना॒ मन्न॒ मन्नꣳ॒॒ साम्रा᳚ज्याना॒ मधि॑प॒ त्यधि॑पति॒ साम्रा᳚ज्याना॒ मन्न॒ मन्नꣳ॒॒ साम्रा᳚ज्याना॒ मधि॑पति । \newline
27. साम्रा᳚ज्याना॒ मधि॑प॒ त्यधि॑पति॒ साम्रा᳚ज्यानाꣳ॒॒ साम्रा᳚ज्याना॒ मधि॑पति॒ तत् तदधि॑पति॒ साम्रा᳚ज्यानाꣳ॒॒ साम्रा᳚ज्याना॒ मधि॑पति॒ तत् । \newline
28. साम्रा᳚ज्याना॒मिति॒ साम् - रा॒ज्या॒ना॒म् । \newline
29. अधि॑पति॒ तत् तदधि॑प॒ त्यधि॑पति॒ तन् मा॑ मा॒ तदधि॑प॒ त्यधि॑पति॒ तन् मा᳚ । \newline
30. अधि॑प॒तीत्यधि॑ - प॒ति॒ । \newline
31. तन् मा॑ मा॒ तत् तन् मा॑ ऽवत्व वतु मा॒ तत् तन् मा॑ ऽवतु । \newline
32. मा॒ ऽव॒त्व॒ व॒तु॒ मा॒ मा॒ ऽव॒तु॒ सोमः॒ सोमो॑ ऽवतु मा मा ऽवतु॒ सोमः॑ । \newline
33. अ॒व॒तु॒ सोमः॒ सोमो॑ ऽवत्व वतु॒ सोम॒ ओष॑धीना॒ मोष॑धीनाꣳ॒॒ सोमो॑ ऽवत्व वतु॒ सोम॒ ओष॑धीनाम् । \newline
34. सोम॒ ओष॑धीना॒ मोष॑धीनाꣳ॒॒ सोमः॒ सोम॒ ओष॑धीनाꣳ सवि॒ता स॑वि॒ तौष॑धीनाꣳ॒॒ सोमः॒ सोम॒ ओष॑धीनाꣳ सवि॒ता । \newline
35. ओष॑धीनाꣳ सवि॒ता स॑वि॒ तौष॑धीना॒ मोष॑धीनाꣳ सवि॒ता प्र॑स॒वाना᳚म् प्रस॒वानाꣳ॑ 
सवि॒ तौष॑धीना॒ मोष॑धीनाꣳ सवि॒ता प्र॑स॒वाना᳚म् । \newline
36. स॒वि॒ता प्र॑स॒वाना᳚म् प्रस॒वानाꣳ॑ सवि॒ता स॑वि॒ता प्र॑स॒वानाꣳ॑ रु॒द्रो रु॒द्रः प्र॑स॒वानाꣳ॑ सवि॒ता स॑वि॒ता प्र॑स॒वानाꣳ॑ रु॒द्रः । \newline
37. प्र॒स॒वानाꣳ॑ रु॒द्रो रु॒द्रः प्र॑स॒वाना᳚म् प्रस॒वानाꣳ॑ रु॒द्रः प॑शू॒नाम् प॑शू॒नाꣳ रु॒द्रः प्र॑स॒वाना᳚म् प्रस॒वानाꣳ॑ रु॒द्रः प॑शू॒नाम् । \newline
38. प्र॒स॒वाना॒मिति॑ प्र - स॒वाना᳚म् । \newline
39. रु॒द्रः प॑शू॒नाम् प॑शू॒नाꣳ रु॒द्रो रु॒द्रः प॑शू॒नाम् त्वष्टा॒ त्वष्टा॑ पशू॒नाꣳ रु॒द्रो रु॒द्रः प॑शू॒नाम् त्वष्टा᳚ । \newline
40. प॒शू॒नाम् त्वष्टा॒ त्वष्टा॑ पशू॒नाम् प॑शू॒नाम् त्वष्टा॑ रू॒पाणाꣳ॑ रू॒पाणा॒म् त्वष्टा॑ पशू॒नाम् प॑शू॒नाम् त्वष्टा॑ रू॒पाणा᳚म् । \newline
41. त्वष्टा॑ रू॒पाणाꣳ॑ रू॒पाणा॒म् त्वष्टा॒ त्वष्टा॑ रू॒पाणा॒म् ॅविष्णु॒र् विष्णू॑ रू॒पाणा॒म् त्वष्टा॒ त्वष्टा॑ 
रू॒पाणा॒म् ॅविष्णुः॑ । \newline
42. रू॒पाणा॒म् ॅविष्णु॒र् विष्णू॑ रू॒पाणाꣳ॑ रू॒पाणा॒म् ॅविष्णुः॒ पर्व॑ताना॒म् पर्व॑ताना॒म् ॅविष्णू॑ रू॒पाणाꣳ॑ रू॒पाणा॒म् ॅविष्णुः॒ पर्व॑तानाम् । \newline
43. विष्णुः॒ पर्व॑ताना॒म् पर्व॑ताना॒म् ॅविष्णु॒र् विष्णुः॒ पर्व॑तानाम् म॒रुतो॑ म॒रुतः॒ पर्व॑ताना॒म् ॅविष्णु॒र् विष्णुः॒ पर्व॑तानाम् म॒रुतः॑ । \newline
44. पर्व॑तानाम् म॒रुतो॑ म॒रुतः॒ पर्व॑ताना॒म् पर्व॑तानाम् म॒रुतो॑ ग॒णाना᳚म् ग॒णाना᳚म् म॒रुतः॒ पर्व॑ताना॒म् 
पर्व॑तानाम् म॒रुतो॑ ग॒णाना᳚म् । \newline
45. म॒रुतो॑ ग॒णाना᳚म् ग॒णाना᳚म् म॒रुतो॑ म॒रुतो॑ ग॒णाना॒ मधि॑पत॒यो ऽधि॑पतयो ग॒णाना᳚म् म॒रुतो॑ 
म॒रुतो॑ ग॒णाना॒ मधि॑पतयः । \newline
46. ग॒णाना॒ मधि॑पत॒यो ऽधि॑पतयो ग॒णाना᳚म् ग॒णाना॒ मधि॑पतय॒ स्ते ते ऽधि॑पतयो ग॒णाना᳚म् ग॒णाना॒ मधि॑पतय॒ स्ते । \newline
47. अधि॑पतय॒ स्ते ते ऽधि॑पत॒यो ऽधि॑पतय॒ स्ते मा॑ मा॒ ते ऽधि॑पत॒यो ऽधि॑पतय॒ स्ते मा᳚ । \newline
48. अधि॑पतय॒ इत्यधि॑ - प॒त॒यः॒ । \newline
49. ते मा॑ मा॒ ते ते मा॑ ऽवन्त्व वन्तु मा॒ ते ते मा॑ ऽवन्तु । \newline
50. मा॒ ऽव॒न्त्व॒ व॒न्तु॒ मा॒ मा॒ ऽव॒न्तु॒ पित॑रः॒ पित॑रो ऽवन्तु मा मा ऽवन्तु॒ पित॑रः । \newline
51. अ॒व॒न्तु॒ पित॑रः॒ पित॑रो ऽवन्त्व वन्तु॒ पित॑रः पितामहाः पितामहाः॒ पित॑रो ऽवन्त्व वन्तु॒ पित॑रः पितामहाः । \newline
52. पित॑रः पितामहाः पितामहाः॒ पित॑रः॒ पित॑रः पितामहाः परे परे पितामहाः॒ पित॑रः॒ पित॑रः पितामहाः परे । \newline
53. पि॒ता॒म॒हाः॒ प॒रे॒ प॒रे॒ पि॒ता॒म॒हाः॒ पि॒ता॒म॒हाः॒ प॒रे॒ ऽव॒रे॒ ऽव॒रे॒ प॒रे॒ पि॒ता॒म॒हाः॒ पि॒ता॒म॒हाः॒ प॒रे॒ ऽव॒रे॒ । \newline
54. प॒रे॒ ऽव॒रे॒ ऽव॒रे॒ प॒रे॒ प॒रे॒ ऽव॒रे॒ तता॒ स्तता॑ अवरे परे परे ऽवरे॒ तताः᳚ । \newline
55. अ॒व॒रे॒ तता॒ स्तता॑ अवरे ऽवरे॒ तता᳚ स्ततामहा स्ततामहा॒ स्तता॑ अवरे ऽवरे॒ तता᳚ स्ततामहाः । \newline
56. तता᳚ स्ततामहा स्ततामहा॒ स्तता॒ स्तता᳚ स्ततामहा इ॒हेह त॑तामहा॒ स्तता॒ स्तता᳚ स्ततामहा इ॒ह । \newline
57. त॒ता॒म॒हा॒ इ॒हेह त॑तामहा स्ततामहा इ॒ह मा॑ मे॒ह त॑तामहा स्ततामहा इ॒ह मा᳚ । \newline
58. इ॒ह मा॑ मे॒हे ह मा॑ ऽवता वत मे॒हे ह मा॑ ऽवत । \newline
59. मा॒ ऽव॒ता॒ व॒त॒ मा॒ मा॒ ऽव॒त॒ । \newline
60. अ॒व॒तेत्य॑वत । \newline
61. अ॒स्मिन् ब्रह्म॒न् ब्रह्म॑न्,न॒स्मिन्,न॒स्मिन् ब्रह्म॑न्,न॒स्मिन्,न॒स्मिन् ब्रह्म॑न्,न॒स्मिन्,न॒स्मिन् ब्रह्म॑न्,न॒स्मिन्न् । \newline
62. ब्रह्म॑न्,न॒स्मिन्,न॒स्मिन् ब्रह्म॒न् ब्रह्म॑न्,न॒स्मिन् क्ष॒त्रे क्ष॒त्रे᳚ ऽस्मिन् ब्रह्म॒न् ब्रह्म॑न्,न॒स्मिन् क्ष॒त्रे । \newline
63. अ॒स्मिन् क्ष॒त्रे क्ष॒त्रे᳚ ऽस्मिन्,न॒स्मिन् क्ष॒त्रे᳚ ऽस्या म॒स्याम् क्ष॒त्रे᳚ ऽस्मिन्,न॒स्मिन् क्ष॒त्रे᳚ ऽस्याम् । \newline
64. क्ष॒त्रे᳚ ऽस्या म॒स्याम् क्ष॒त्रे क्ष॒त्रे᳚ ऽस्या मा॒शि ष्या॒शि ष्य॒स्याम् क्ष॒त्रे क्ष॒त्रे᳚ ऽस्या मा॒शिषि॑ । \newline
65. अ॒स्या मा॒शि ष्या॒शि ष्य॒स्या म॒स्या मा॒शि ष्य॒स्या म॒स्या मा॒शि ष्य॒स्या म॒स्या मा॒शि ष्य॒स्याम् । \newline
66. आ॒शि ष्य॒स्या म॒स्या मा॒शि ष्या॒शि ष्य॒स्याम् पु॑रो॒धाया᳚म् पुरो॒धाया॑ म॒स्या मा॒शि ष्या॒शि ष्य॒स्याम् पु॑रो॒धाया᳚म् । \newline
67. आ॒शिषीत्या᳚ - शिषि॑ । \newline
68. अ॒स्याम् पु॑रो॒धाया᳚म् पुरो॒धाया॑ म॒स्या म॒स्याम् पु॑रो॒धाया॑ म॒स्मिन्,न॒स्मिन् पु॑रो॒धाया॑ म॒स्या म॒स्याम् पु॑रो॒धाया॑ म॒स्मिन्न् । \newline
69. पु॒रो॒धाया॑ म॒स्मिन्,न॒स्मिन् पु॑रो॒धाया᳚म् पुरो॒धाया॑ म॒स्मिन् कर्म॒न् कर्म॑न्,न॒स्मिन् पु॑रो॒धाया᳚म् पुरो॒धाया॑ म॒स्मिन् कर्मन्न्॑ । \newline
70. पु॒रो॒धाया॒मिति॑ पुरः - धाया᳚म् । \newline
71. अ॒स्मिन् कर्म॒न् कर्म॑न्,न॒स्मिन्,न॒स्मिन् कर्म॑न्,न॒स्या म॒स्याम् कर्म॑न्,न॒स्मिन्,न॒स्मिन् कर्म॑न्,न॒स्याम् । \newline
72. कर्म॑न्,न॒स्या म॒स्याम् कर्म॒न् कर्म॑न्,न॒स्याम् दे॒वहू᳚त्याम् दे॒वहू᳚त्या म॒स्याम् कर्म॒न् कर्म॑न्,न॒स्याम् दे॒वहू᳚त्याम् । \newline
73. अ॒स्याम् दे॒वहू᳚त्याम् दे॒वहू᳚त्या म॒स्या म॒स्याम् दे॒वहू᳚त्याम् । \newline
74. दे॒वहू᳚त्या॒मिति॑ दे॒व - हू॒त्या॒म् । \newline
\pagebreak
\markright{ TS 3.4.6.1  \hfill https://www.vedavms.in \hfill}

\section{ TS 3.4.6.1 }

\textbf{TS 3.4.6.1 } \newline
\textbf{Samhita Paata} \newline

दे॒वा वै यद्य॒ज्ञेऽकु॑र्वत॒ तदसु॑रा अकुर्वत॒ ते दे॒वा ए॒तान॑भ्याता॒नान॑पश्य॒न्- तान॒भ्यात॑न्वत॒ यद्दे॒वानां॒ कर्माऽऽ*सी॒दार्द्ध्य॑त॒ तद्यदसु॑राणां॒ न तदा᳚र्द्ध्यत॒ येन॒ कर्म॒णेर्थ्से॒त् तत्र॑ होत॒व्या॑ ऋ॒द्ध्नोत्ये॒व तेन॒ कर्म॑णा॒ यद्विश्वे॑ दे॒वाः स॒मभ॑र॒न् तस्मा॑-दभ्याता॒ना वै᳚श्वदे॒वायत्-प्र॒जाप॑ति॒र्जया॒न् प्राय॑च्छ॒त् तस्मा॒ज्जयाः᳚ प्राजाप॒त्या - [  ] \newline

\textbf{Pada Paata} \newline

दे॒वाः । वै । यत् । य॒ज्ञे । अकु॑र्वत । तत् । असु॑राः । अ॒कु॒र्व॒त॒ । ते । दे॒वाः । ए॒तान् । अ॒भ्या॒ता॒नानित्य॑भि - आ॒ता॒नान् । अ॒प॒श्य॒न्न् । तान् । अ॒भ्यात॑न्व॒तेत्य॑भि - आत॑न्वत । यत् । दे॒वाना᳚म् । कर्म॑ । आसी᳚त् । आर्द्ध्य॑त । तत् । यत् । असु॑राणाम् । न । तत् । आ॒र्द्ध्य॒त॒ । येन॑ । कर्म॑णा । ईर्थ्से᳚त् । तत्र॑ । हो॒त॒व्याः᳚ । ऋ॒द्ध्नोति॑ । ए॒व । तेन॑ । कर्म॑णा । यत् । विश्वे᳚ । दे॒वाः । स॒मभ॑र॒न्निति॑ सं - अभ॑रन्न् । तस्मा᳚त् । अ॒भ्या॒ता॒ना इत्य॑भि - आ॒ता॒नाः । वै॒श्व॒दे॒वा इति॑ वैश्व - दे॒वाः । यत् । प्र॒जाप॑ति॒रिति॑ प्र॒जा - प॒तिः॒ । जयान्॑ । प्रेति॑ । अय॑च्छत् । तस्मा᳚त् । जयाः᳚ । प्रा॒जा॒प॒त्या इति॑ प्राजा-प॒त्याः ।  \newline


\textbf{Krama Paata} \newline

दे॒वा वै । वै यत् । यद् य॒ज्ञे । य॒ज्ञे ऽकु॑र्वत । अकु॑र्वत॒ तत् । तदसु॑राः । असु॑रा अकुर्वत । अ॒कु॒र्व॒त॒ ते । ते दे॒वाः । दे॒वा ए॒तान् । ए॒तान॑भ्याता॒नान् । अ॒भ्या॒ता॒नान॑पश्यन्न् । अ॒भ्या॒ता॒नानित्य॑भि - आ॒ता॒नान् । अ॒प॒श्य॒न् तान् । तान॒भ्यात॑न्वत । अ॒भ्यात॑न्वत॒ यत् । अ॒भ्यात॑न्व॒तेत्य॑भि - आत॑न्वत । यद् दे॒वाना᳚म् । दे॒वाना॒म् कर्म॑ । कर्मासी᳚त् । आसी॒दार्द्ध्य॑त । आर्द्ध्य॑त॒ तत् । तद् यत् । यदसु॑राणाम् । असु॑राणा॒म् न । न तत् । तदा᳚र्द्ध्यत । आ॒र्द्ध्य॒त॒ येन॑ । येन॒ कर्म॑णा । कर्म॒णेर्त्थ्से᳚त् । ईर्त्थ्से॒त् तत्र॑ । तत्र॑ होत॒व्याः᳚ । हो॒त॒व्या॑ ऋ॒द्ध्नोति॑ । ऋ॒द्ध्नोत्ये॒व । ए॒व तेन॑ । तेन॒ कर्म॑णा । कर्म॑णा॒ यत् । यद् विश्वे᳚ । विश्वे॑ दे॒वाः । दे॒वाः स॒मभ॑रन्न् । स॒मभ॑र॒न् तस्मा᳚त् । स॒मभ॑र॒न्नि॑ति सं - अभ॑रन्न् । तस्मा॑दभ्याता॒नाः । अ॒भ्या॒ता॒ना वै᳚श्वदे॒वाः । अ॒भ्या॒ता॒ना इत्य॑भि - आ॒ता॒नाः । वै॒श्व॒दे॒वा यत् । वै॒श्व॒दे॒वा इति॑ वैश्व - दे॒वाः । यत् प्र॒जाप॑तिः । प्र॒जाप॑ति॒र् जयान्॑ । प्र॒जाप॑ति॒रिति॑प्र॒जा - प॒तिः॒ । जया॒न् प्र । प्राय॑च्छत् । अय॑च्छ॒त् तस्मा᳚त् । तस्मा॒ज् जयाः᳚ । 
जयाः᳚ प्राजाप॒त्याः । प्रा॒जा॒प॒त्या यत् । प्रा॒जा॒प॒त्या इति॑ प्राजा - प॒त्याः \newline

\textbf{Jatai Paata} \newline

1. दे॒वा वै वै दे॒वा दे॒वा वै । \newline
2. वै यद् यद् वै वै यत् । \newline
3. यद् य॒ज्ञे य॒ज्ञे यद् यद् य॒ज्ञे । \newline
4. य॒ज्ञे ऽकु॑र्व॒ता कु॑र्वत य॒ज्ञे य॒ज्ञे ऽकु॑र्वत । \newline
5. अकु॑र्वत॒ तत् तदकु॑र्व॒ता कु॑र्वत॒ तत् । \newline
6. तदसु॑रा॒ असु॑रा॒ स्तत् तदसु॑राः । \newline
7. असु॑रा अकुर्वता कुर्व॒ता सु॑रा॒ असु॑रा अकुर्वत । \newline
8. अ॒कु॒र्व॒त॒ ते ते॑ ऽकुर्वता कुर्वत॒ ते । \newline
9. ते दे॒वा दे॒वा स्ते ते दे॒वाः । \newline
10. दे॒वा ए॒ता-ने॒तान् दे॒वा दे॒वा ए॒तान् । \newline
11. ए॒ता-न॑भ्याता॒ना-न॑भ्याता॒ना-ने॒ता-ने॒ता-न॑भ्याता॒नान् । \newline
12. अ॒भ्या॒ता॒ना-न॑पश्यन्-नपश्यन्-नभ्याता॒ना-न॑भ्याता॒ना-न॑पश्यन्न् । \newline
13. अ॒भ्या॒ता॒नानित्य॑भि - आ॒ता॒नान् । \newline
14. अ॒प॒श्य॒न् ताꣳ स्ता-न॑पश्यन्-नपश्य॒न् तान् । \newline
15. तान॒भ्यात॑न्वता॒ भ्यात॑न्वत॒ ताꣳ स्ता-न॒भ्यात॑न्वत । \newline
16. अ॒भ्यात॑न्वत॒ यद् यद॒भ्यात॑न्वता॒ भ्यात॑न्वत॒ यत् । \newline
17. अ॒भ्यात॑न्व॒तेत्य॑भि - आत॑न्वत । \newline
18. यद् दे॒वाना᳚म् दे॒वानां॒ ॅयद् यद् दे॒वाना᳚म् । \newline
19. दे॒वाना॒म् कर्म॒ कर्म॑ दे॒वाना᳚म् दे॒वाना॒म् कर्म॑ । \newline
20. कर्मासी॒ दासी॒त् कर्म॒ कर्मासी᳚त् । \newline
21. आसी॒ दार्द्ध्य॒ता र्द्ध्य॒ता सी॒दा सी॒दा र्द्ध्य॑त । \newline
22. आर्द्ध्य॑त॒ तत् त दार्द्ध्य॒ता र्द्ध्य॑त॒ तत् । \newline
23. तद् यद् यत् तत् तद् यत् । \newline
24. यदसु॑राणा॒ मसु॑राणां॒ ॅयद् यदसु॑राणाम् । \newline
25. असु॑राणा॒म् न नासु॑राणा॒ मसु॑राणा॒म् न । \newline
26. न तत् तन् न न तत् । \newline
27. तदा᳚र्द्ध्यता र्द्ध्यत॒ तत् तदा᳚र्द्ध्यत । \newline
28. आ॒र्द्ध्य॒त॒ येन॒ येना᳚ र्द्ध्यता र्द्ध्यत॒ येन॑ । \newline
29. येन॒ कर्म॑णा॒ कर्म॑णा॒ येन॒ येन॒ कर्म॑णा । \newline
30. कर्म॒ णेर्थ्से॒ दीर्थ्से॒त् कर्म॑णा॒ कर्म॒ णेर्थ्से᳚त् । \newline
31. ईर्थ्से॒त् तत्र॒ तत्रेर्थ्से॒ दीर्थ्से॒त् तत्र॑ । \newline
32. तत्र॑ होत॒व्या॑ होत॒व्या᳚ स्तत्र॒ तत्र॑ होत॒व्याः᳚ । \newline
33. हो॒त॒व्या॑ ऋ॒द्ध्नो त्यृ॒द्ध्नोति॑ होत॒व्या॑ होत॒व्या॑ ऋ॒द्ध्नोति॑ । \newline
34. ऋ॒द्ध्नोत् ये॒वैव र्द्ध्नो त्यृ॒द्ध्नो त्ये॒व । \newline
35. ए॒व तेन॒ तेनै॒वैव तेन॑ । \newline
36. तेन॒ कर्म॑णा॒ कर्म॑णा॒ तेन॒ तेन॒ कर्म॑णा । \newline
37. कर्म॑णा॒ यद् यत् कर्म॑णा॒ कर्म॑णा॒ यत् । \newline
38. यद् विश्वे॒ विश्वे॒ यद् यद् विश्वे᳚ । \newline
39. विश्वे॑ दे॒वा दे॒वा विश्वे॒ विश्वे॑ दे॒वाः । \newline
40. दे॒वाः स॒मभ॑रन् थ्स॒मभ॑रन् दे॒वा दे॒वाः स॒मभ॑रन्न् । \newline
41. स॒मभ॑र॒न् तस्मा॒त् तस्मा᳚थ् स॒मभ॑रन् थ्स॒मभ॑र॒न् तस्मा᳚त् । \newline
42. स॒मभ॑र॒न्निति॑ सं - अभ॑रन्न् । \newline
43. तस्मा॑ दभ्याता॒ना अ॑भ्याता॒ना स्तस्मा॒त् तस्मा॑ दभ्याता॒नाः । \newline
44. अ॒भ्या॒ता॒ना वै᳚श्वदे॒वा वै᳚श्वदे॒वा अ॑भ्याता॒ना अ॑भ्याता॒ना वै᳚श्वदे॒वाः । \newline
45. अ॒भ्या॒ता॒ना इत्य॑भि - आ॒ता॒नाः । \newline
46. वै॒श्व॒दे॒वा यद् यद् वै᳚श्वदे॒वा वै᳚श्वदे॒वा यत् । \newline
47. वै॒श्व॒दे॒वा इति॑ वैश्व - दे॒वाः । \newline
48. यत् प्र॒जाप॑तिः प्र॒जाप॑ति॒र् यद् यत् प्र॒जाप॑तिः । \newline
49. प्र॒जाप॑ति॒र् जया॒न् जया᳚न् प्र॒जाप॑तिः प्र॒जाप॑ति॒र् जयान्॑ । \newline
50. प्र॒जाप॑ति॒रिति॑ प्र॒जा - प॒तिः॒ । \newline
51. जया॒न् प्र प्र जया॒न् जया॒न् प्र । \newline
52. प्रा य॑च्छ॒ दय॑च्छ॒त् प्र प्राय॑च्छत् । \newline
53. अय॑च्छ॒त् तस्मा॒त् तस्मा॒ दय॑च्छ॒ दय॑च्छ॒त् तस्मा᳚त् । \newline
54. तस्मा॒ज् जया॒ जया॒ स्तस्मा॒त् तस्मा॒ज् जयाः᳚ । \newline
55. जयाः᳚ प्राजाप॒त्याः प्रा॑जाप॒त्या जया॒ जयाः᳚ प्राजाप॒त्याः । \newline
56. प्रा॒जा॒प॒त्या यद् यत् प्रा॑जाप॒त्याः प्रा॑जाप॒त्या यत् । \newline
57. प्रा॒जा॒प॒त्या इति॑ प्राजा - प॒त्याः । \newline

\textbf{Ghana Paata } \newline

1. दे॒वा वै वै दे॒वा दे॒वा वै यद् यद् वै दे॒वा दे॒वा वै यत् । \newline
2. वै यद् यद् वै वै यद् य॒ज्ञे य॒ज्ञे यद् वै वै यद् य॒ज्ञे । \newline
3. यद् य॒ज्ञे य॒ज्ञे यद् यद् य॒ज्ञे ऽकु॑र्व॒ता कु॑र्वत य॒ज्ञे यद् यद् य॒ज्ञे ऽकु॑र्वत । \newline
4. य॒ज्ञे ऽकु॑र्व॒ता कु॑र्वत य॒ज्ञे य॒ज्ञे ऽकु॑र्वत॒ तत् तदकु॑र्वत य॒ज्ञे य॒ज्ञे ऽकु॑र्वत॒ तत् । \newline
5. अकु॑र्वत॒ तत् तदकु॑र्व॒ता कु॑र्वत॒ तदसु॑रा॒ असु॑रा॒ स्तदकु॑र्व॒ता कु॑र्वत॒ तदसु॑राः । \newline
6. तदसु॑रा॒ असु॑रा॒ स्तत् तदसु॑रा अकुर्वता कुर्व॒ता सु॑रा॒ स्तत् तदसु॑रा अकुर्वत । \newline
7. असु॑रा अकुर्वता कुर्व॒ता सु॑रा॒ असु॑रा अकुर्वत॒ ते ते॑ ऽकुर्व॒ता सु॑रा॒ असु॑रा अकुर्वत॒ ते । \newline
8. अ॒कु॒र्व॒त॒ ते ते॑ ऽकुर्वता कुर्वत॒ ते दे॒वा दे॒वा स्ते॑ ऽकुर्वता कुर्वत॒ ते दे॒वाः । \newline
9. ते दे॒वा दे॒वा स्ते ते दे॒वा ए॒ता ने॒तान् दे॒वा स्ते ते दे॒वा ए॒तान् । \newline
10. दे॒वा ए॒ताने॒तान् दे॒वा दे॒वा ए॒ता,न॑भ्याता॒ना,न॑भ्याता॒ना,ने॒तान् दे॒वा दे॒वा ए॒ता न॑भ्याता॒नान् । \newline
11. ए॒ता न॑भ्याता॒ना न॑भ्याता॒ना,ने॒ता,ने॒ता,न॑भ्याता॒ना न॑पश्यन्,नपश्यन्,नभ्याता॒ना,ने॒ता,ने॒ता,न॑भ्याता॒ना,न॑पश्यन्न् । \newline
12. अ॒भ्या॒ता॒ना,न॑पश्यन्,नपश्यन्,नभ्याता॒ना,न॑भ्याता॒ना,न॑पश्य॒न् ताꣳ स्तान॑पश्यन्,नभ्याता॒ना,न॑भ्याता॒ना,न॑पश्य॒न् तान् । \newline
13. अ॒भ्या॒ता॒नानित्य॑भि - आ॒ता॒नान् । \newline
14. अ॒प॒श्य॒न् ताꣳ स्तान॑पश्यन्,नपश्य॒न् तान॒भ्यात॑न्वता॒ भ्यात॑न्वत॒ तान॑पश्यन्,नपश्य॒न् तान॒भ्यात॑न्वत । \newline
15. तान॒भ्यात॑न्वता॒ भ्यात॑न्वत॒ ताꣳ स्ता न॒भ्यात॑न्वत॒ यद् यद॒भ्यात॑न्वत॒ ताꣳ स्तान् अ॒भ्यात॑न्वत॒ यत् । \newline
16. अ॒भ्यात॑न्वत॒ यद् यद॒भ्यात॑न्वता॒ भ्यात॑न्वत॒ यद् दे॒वाना᳚म् दे॒वाना॒म् ॅयद॒भ्यात॑न्वता॒ भ्यात॑न्वत॒ यद् दे॒वाना᳚म् । \newline
17. अ॒भ्यात॑न्व॒तेत्य॑भि - आत॑न्वत । \newline
18. यद् दे॒वाना᳚म् दे॒वाना॒म् ॅयद् यद् दे॒वाना॒म् कर्म॒ कर्म॑ दे॒वाना॒म् ॅयद् यद् दे॒वाना॒म् कर्म॑ । \newline
19. दे॒वाना॒म् कर्म॒ कर्म॑ दे॒वाना᳚म् दे॒वाना॒म् कर्मासी॒ दासी॒त् कर्म॑ दे॒वाना᳚म् दे॒वाना॒म् कर्मासी᳚त् । \newline
20. कर्मासी॒ दासी॒त् कर्म॒ कर्मासी॒ दार्द्ध्य॒ता र्द्ध्य॒ तासी॒त् कर्म॒ कर्मासी॒ दार्द्ध्य॑त । \newline
21. आसी॒ दार्द्ध्य॒ता र्द्ध्य॒तासी॒ दासी॒ दार्द्ध्य॑त॒ तत् तदार्द्ध्य॒ता सी॒दासी॒ दार्द्ध्य॑त॒ तत् । \newline
22. आर्द्ध्य॑त॒ तत् तदार्द्ध्य॒ता र्द्ध्य॑त॒ तद् यद् यत् तदार्द्ध्य॒ता र्द्ध्य॑त॒ तद् यत् । \newline
23. तद् यद् यत् तत् तद् यदसु॑राणा॒ मसु॑राणा॒म् ॅयत् तत् तद् यदसु॑राणाम् । \newline
24. यदसु॑राणा॒ मसु॑राणा॒म् ॅयद् यदसु॑राणा॒म् न नासु॑राणा॒म् ॅयद् यदसु॑राणा॒म् न । \newline
25. असु॑राणा॒म् न नासु॑राणा॒ मसु॑राणा॒म् न तत् तन् नासु॑राणा॒ मसु॑राणा॒म् न तत् । \newline
26. न तत् तन् न न तदा᳚र्द्ध्यता र्द्ध्यत॒ तन् न न तदा᳚र्द्ध्यत । \newline
27. तदा᳚र्द्ध्यता र्द्ध्यत॒ तत् तदा᳚र्द्ध्यत॒ येन॒ येना᳚र्द्ध्यत॒ तत् तदा᳚र्द्ध्यत॒ येन॑ । \newline
28. आ॒र्द्ध्य॒त॒ येन॒ येना᳚ र्द्ध्यता र्द्ध्यत॒ येन॒ कर्म॑णा॒ कर्म॑णा॒ येना᳚ र्द्ध्यता र्द्ध्यत॒ येन॒ कर्म॑णा । \newline
29. येन॒ कर्म॑णा॒ कर्म॑णा॒ येन॒ येन॒ कर्म॒ णेर्‌थ्से॒दीर् थ्से॒त् कर्म॑णा॒ येन॒ येन॒ कर्म॒ णेर्‌थ्से᳚त् । \newline
30. कर्म॒ णेर्‌थ्से॒दीर् थ्से॒त् कर्म॑णा॒ कर्म॒ णेर्‌थ्से॒त् तत्र॒ तत्रेर्‌थ्से॒त् कर्म॑णा॒ कर्म॒ णेर्‌थ्से॒त् तत्र॑ । \newline
31. ईर्‌थ्से॒त् तत्र॒ तत्रेर्‌थ्से॒ दीर्‌थ्से॒त् तत्र॑ होत॒व्या॑ होत॒व्या᳚ स्तत्रेर्‌थ्से॒ दीर्‌थ्से॒त् तत्र॑ होत॒व्याः᳚ । \newline
32. तत्र॑ होत॒व्या॑ होत॒व्या᳚ स्तत्र॒ तत्र॑ होत॒व्या॑ ऋ॒द्ध्नो त्यृ॒द्ध्नोति॑ होत॒व्या᳚ स्तत्र॒ तत्र॑ होत॒व्या॑ ऋ॒द्ध्नोति॑ । \newline
33. हो॒त॒व्या॑ ऋ॒द्ध्नो त्यृ॒द्ध्नोति॑ होत॒व्या॑ होत॒व्या॑ ऋ॒द्ध्नो त्ये॒वैव र्द्ध्नोति॑ होत॒व्या॑ होत॒व्या॑ ऋ॒द्ध्नो त्ये॒व । \newline
34. ऋ॒द्ध्नो त्ये॒वैव र्द्ध्नो त्यृ॒द्ध्नो त्ये॒व तेन॒ तेनै॒व र्द्ध्नो त्यृ॒द्ध्नो त्ये॒व तेन॑ । \newline
35. ए॒व तेन॒ तेनै॒वैव तेन॒ कर्म॑णा॒ कर्म॑णा॒ तेनै॒वैव तेन॒ कर्म॑णा । \newline
36. तेन॒ कर्म॑णा॒ कर्म॑णा॒ तेन॒ तेन॒ कर्म॑णा॒ यद् यत् कर्म॑णा॒ तेन॒ तेन॒ कर्म॑णा॒ यत् । \newline
37. कर्म॑णा॒ यद् यत् कर्म॑णा॒ कर्म॑णा॒ यद् विश्वे॒ विश्वे॒ यत् कर्म॑णा॒ कर्म॑णा॒ यद् विश्वे᳚ । \newline
38. यद् विश्वे॒ विश्वे॒ यद् यद् विश्वे॑ दे॒वा दे॒वा विश्वे॒ यद् यद् विश्वे॑ दे॒वाः । \newline
39. विश्वे॑ दे॒वा दे॒वा विश्वे॒ विश्वे॑ दे॒वाः स॒मभ॑रन् थ्स॒मभ॑रन् दे॒वा विश्वे॒ विश्वे॑ दे॒वाः स॒मभ॑रन्न् । \newline
40. दे॒वाः स॒मभ॑रन् थ्स॒मभ॑रन् दे॒वा दे॒वाः स॒मभ॑र॒न् तस्मा॒त् तस्मा᳚थ् स॒मभ॑रन् दे॒वा दे॒वाः स॒मभ॑र॒न् तस्मा᳚त् । \newline
41. स॒मभ॑र॒न् तस्मा॒त् तस्मा᳚थ् स॒मभ॑रन् थ्स॒मभ॑र॒न् तस्मा॑ दभ्याता॒ना अ॑भ्याता॒ना स्तस्मा᳚थ् स॒मभ॑रन् 
थ्स॒मभ॑र॒न् तस्मा॑ दभ्याता॒नाः । \newline
42. स॒मभ॑र॒न्निति॑ सम् - अभ॑रन्न् । \newline
43. तस्मा॑ दभ्याता॒ना अ॑भ्याता॒ना स्तस्मा॒त् तस्मा॑ दभ्याता॒ना वै᳚श्वदे॒वा वै᳚श्वदे॒वा अ॑भ्याता॒ना स्तस्मा॒त् तस्मा॑ दभ्याता॒ना वै᳚श्वदे॒वाः । \newline
44. अ॒भ्या॒ता॒ना वै᳚श्वदे॒वा वै᳚श्वदे॒वा अ॑भ्याता॒ना अ॑भ्याता॒ना वै᳚श्वदे॒वा यद् यद् वै᳚श्वदे॒वा अ॑भ्याता॒ना 
अ॑भ्याता॒ना वै᳚श्वदे॒वा यत् । \newline
45. अ॒भ्या॒ता॒ना इत्य॑भि - आ॒ता॒नाः । \newline
46. वै॒श्व॒दे॒वा यद् यद् वै᳚श्वदे॒वा वै᳚श्वदे॒वा यत् प्र॒जाप॑तिः प्र॒जाप॑ति॒र् यद् वै᳚श्वदे॒वा वै᳚श्वदे॒वा यत् प्र॒जाप॑तिः । \newline
47. वै॒श्व॒दे॒वा इति॑ वैश्व - दे॒वाः । \newline
48. यत् प्र॒जाप॑तिः प्र॒जाप॑ति॒र् यद् यत् प्र॒जाप॑ति॒र् जया॒न् जया᳚न् प्र॒जाप॑ति॒र् यद् यत् प्र॒जाप॑ति॒र् जयान्॑ । \newline
49. प्र॒जाप॑ति॒र् जया॒न् जया᳚न् प्र॒जाप॑तिः प्र॒जाप॑ति॒र् जया॒न् प्र प्र जया᳚न् प्र॒जाप॑तिः प्र॒जाप॑ति॒र् जया॒न् प्र । \newline
50. प्र॒जाप॑ति॒रिति॑ प्र॒जा - प॒तिः॒ । \newline
51. जया॒न् प्र प्र जया॒न् जया॒न् प्राय॑च्छ॒ दय॑च्छ॒त् प्र जया॒न् जया॒न् प्राय॑च्छत् । \newline
52. प्राय॑च्छ॒ दय॑च्छ॒त् प्र प्राय॑च्छ॒त् तस्मा॒त् तस्मा॒ दय॑च्छ॒त् प्र प्राय॑च्छ॒त् तस्मा᳚त् । \newline
53. अय॑च्छ॒त् तस्मा॒त् तस्मा॒ दय॑च्छ॒ दय॑च्छ॒त् तस्मा॒ज् जया॒ जया॒ स्तस्मा॒ दय॑च्छ॒ दय॑च्छ॒त् तस्मा॒ज् जयाः᳚ । \newline
54. तस्मा॒ज् जया॒ जया॒ स्तस्मा॒त् तस्मा॒ज् जयाः᳚ प्राजाप॒त्याः प्रा॑जाप॒त्या जया॒ स्तस्मा॒त् तस्मा॒ज् जयाः᳚ प्राजाप॒त्याः । \newline
55. जयाः᳚ प्राजाप॒त्याः प्रा॑जाप॒त्या जया॒ जयाः᳚ प्राजाप॒त्या यद् यत् प्रा॑जाप॒त्या जया॒ जयाः᳚ प्राजाप॒त्या यत् । \newline
56. प्रा॒जा॒प॒त्या यद् यत् प्रा॑जाप॒त्याः प्रा॑जाप॒त्या यद् रा᳚ष्ट्र॒भृद्भी॑ राष्ट्र॒भृद्भि॒र् यत् प्रा॑जाप॒त्याः प्रा॑जाप॒त्या यद् रा᳚ष्ट्र॒भृद्भिः॑ । \newline
57. प्रा॒जा॒प॒त्या इति॑ प्राजा - प॒त्याः । \newline
\pagebreak
\markright{ TS 3.4.6.2  \hfill https://www.vedavms.in \hfill}

\section{ TS 3.4.6.2 }

\textbf{TS 3.4.6.2 } \newline
\textbf{Samhita Paata} \newline

यद्-रा᳚ष्ट्र॒भृद्भी॑ रा॒ष्ट्रमाऽद॑दत॒ तद्-रा᳚ष्ट्र॒भृताꣳ॑ राष्ट्रभृ॒त्त्वं ते दे॒वा अ॑भ्याता॒नैरसु॑रान॒भ्यात॑न्वत॒ जयै॑रजयन्-राष्ट्र॒भृद्भी॑ रा॒ष्ट्रमाऽद॑दत॒ यद्दे॒वा अ॑भ्याता॒नैरसु॑रान॒भ्यात॑न्वत॒ तद॑भ्याता॒नाना॑मभ्यातान॒त्वं ॅयज्जयै॒रज॑य॒न् तज्जया॑नां जय॒त्वं ॅयद्-रा᳚ष्ट्र॒भृद्भी॑ रा॒ष्ट्रमाऽद॑दत॒ तद्-रा᳚ष्ट्र॒भृताꣳ॑ राष्ट्रभृ॒त्त्वं ततो॑ दे॒वा अभ॑व॒न् पराऽसु॑रा॒ यो भ्रातृ॑व्यवा॒न्थ् स्याथ् स ( ) ए॒तान् जु॑हुयादभ्याता॒नैरे॒व भ्रातृ॑व्यान॒भ्यात॑नुते॒ जयै᳚र्जयति राष्ट्र॒भृद्भी॑ रा॒ष्ट्रमा द॑त्ते॒ भव॑त्या॒त्मना॒ परा᳚ऽस्य॒ भ्रातृ॑व्यो भवति ॥ \newline

\textbf{Pada Paata} \newline

यत् । रा॒ष्ट्र॒भृद्भि॒रिति॑ राष्ट्र॒भृत् - भिः॒ । रा॒ष्ट्रम् । एति॑ । अद॑दत । तत् । रा॒ष्ट्र॒भृता॒मिति॑ राष्ट्र - भृता᳚म् । रा॒ष्ट्र॒भृ॒त्त्वमिति॑ राष्ट्रभृत् - त्वम् । ते । दे॒वाः । अ॒भ्या॒ता॒नैरित्य॑भि - आ॒ता॒नैः । असु॑रान् । अ॒भ्यात॑न्व॒तेत्य॑भि - आत॑न्वत । जयैः᳚ । अ॒ज॒य॒न् । रा॒ष्ट्र॒भृद्भि॒रिति॑ राष्ट्र॒भृत् - भिः॒ । रा॒ष्ट्रम् । एति॑ । अ॒द॒द॒त॒ । यत् । दे॒वाः । अ॒भ्या॒ता॒नैरित्य॑भि - आ॒ता॒नैः । असु॑रान् । अ॒भ्यात॑न्व॒तेत्य॑भि - आत॑न्वत । तत् । अ॒भ्या॒ता॒नाना॒मित्य॑भि - आ॒ता॒नाना᳚म् । अ॒भ्या॒ता॒न॒त्वमित्य॑भ्यातान - त्वम् । यत् । जयैः᳚ । अज॑यन्न् । तत् । जया॑नाम् । ज॒य॒त्वमिति॑ जय - त्वम् । यत् । रा॒ष्ट्र॒भृद्भि॒रिति॑ राष्ट्र॒भृत् - भिः॒ । रा॒ष्ट्रम् । एति॑ । अद॑दत । तत् । रा॒ष्ट्र॒भृता॒मिति॑ राष्ट्र - भृता᳚म् । रा॒ष्ट्र॒भृ॒त्त्वमिति॑ राष्ट्रभृत् - त्वम् । ततः॑ । दे॒वाः । अभ॑वन्न् । परेति॑ । असु॑राः । यः । भ्रातृ॑व्यवा॒निति॒ भ्रातृ॑व्य - वा॒न् । स्यात् । सः ( ) । ए॒तान् । जु॒हु॒या॒त् । अ॒भ्या॒ता॒नैरित्य॑भि-आ॒ता॒नैः । ए॒व । भ्रातृ॑व्यान् । अ॒भ्यात॑नुत॒ इत्य॑भि - आत॑नुते । जयैः᳚ । ज॒य॒ति॒ । रा॒ष्ट्र॒भृद्भि॒रिति॑ राष्ट्र॒भृत् - भिः॒ । रा॒ष्ट्रम् । एति॑ । द॒त्ते॒ । भव॑ति । आ॒त्मना᳚ । परेति॑ । अ॒स्य॒ । भ्रातृ॑व्यः । भ॒व॒ति॒ ॥  \newline


\textbf{Krama Paata} \newline

यद् रा᳚ष्ट्र॒भृद्भिः॑ । रा॒ष्ट्र॒भृद्भी॑ रा॒ष्ट्रम् । रा॒ष्ट्र॒भृद्भि॒रिति॑ राष्ट्र॒भृत् - भिः॒ । रा॒ष्टमा । आ ऽद॑दत । अद॑दत॒ तत् । तद् रा᳚ष्ट्र॒भृता᳚म् । रा॒ष्ट्र॒भृताꣳ॑ राष्ट्रभृ॒त्त्वम् । रा॒ष्ट॒भृता॒मिति॑ राष्ट्र - भृता᳚म् । रा॒ष्ट्र॒भृ॒त्त्वम् ते । रा॒ष्ट॒भृ॒त्त्वमिति॑ राष्ट्रभृत् - त्वम् । ते दे॒वाः । दे॒वा अ॑भ्याता॒नैः । अ॒भ्या॒ता॒नैरसु॑रान् । अ॒भ्या॒ता॒नैरित्य॑भि - आ॒ता॒नैः । असु॑रान॒भ्यात॑न्वत । अ॒भ्यात॑न्वत॒ जयैः᳚ । अ॒भ्यात॑न्व॒तेत्य॑भि - आत॑न्वत । 
जयै॑रजयन्न् । अ॒ज॒य॒न् रा॒ष्ट्र॒भृद्भिः॑ । रा॒ष्ट्र॒भृद्भी॑ रा॒ष्ट्रम् । रा॒ष्ट॒भृद्भि॒रिति॑ राष्ट्र॒भृत् - भिः॒ । रा॒ष्ट्रमा । आ ऽद॑दत । अ॒द॒द॒त॒ यत् । यद् दे॒वाः । दे॒वा अ॑भ्याता॒नैः । अ॒भ्या॒ता॒नैरसु॑रान् । अ॒भ्या॒ता॒नैरित्य॑भि - आ॒ता॒नैः । असु॑रान॒भ्यात॑न्वत । अ॒भ्यात॑न्वत॒ तत् । अ॒भ्यात॑न्व॒तेत्य॑भि - आत॑न्वत । तद॑भ्याता॒नाना᳚म् । अ॒भ्या॒ता॒नाना॑,मभ्यातान॒त्वम् । अ॒भ्या॒ता॒नाना॒मित्य॑भि - आ॒ता॒नाना᳚म् । अ॒भ्या॒ता॒न॒त्वं ॅयत् । अ॒भ्या॒ता॒न॒त्वमित्य॑भ्यातान - त्वम् । यज् जयैः᳚ । जयै॒रज॑यन्न् । अज॑य॒न् तत् । तज् जया॑नाम् । जया॑नाम् जय॒त्वम् । ज॒य॒त्वं ॅयत् । ज॒य॒त्वमिति॑ जय - त्वम् । यद् रा᳚ष्ट्र॒भृद्भिः॑ । रा॒ष्ट्र॒भृद्भी॑ रा॒ष्ट्रम् । रा॒ष्ट्र॒भृद्भि॒रिति॑ राष्ट्र॒भृत् - भिः॒ । रा॒ष्ट्रमा । आ ऽद॑दत । अद॑दत॒ तत् । तद् रा᳚ष्ट्र॒भृता᳚म् । रा॒ष्ट्र॒भृताꣳ॑ राष्ट्रभृ॒त्वम् । रा॒ष्ट्र॒भृता॒मिति॑ राष्ट्र - भृता᳚म् । रा॒ष्ट॒भृ॒त्वम् ततः॑ । रा॒ष्ट्र॒भृ॒त्वमिति॑ राष्ट्रभृत् - त्वम् । ततो॑ दे॒वाः । दे॒वा अभ॑वन्न् । अभ॑व॒न् परा᳚ । परा ऽसु॑राः । असु॑रा॒ यः । यो भ्रातृ॑व्यवान् । भ्रातृ॑व्यवा॒न्थ् स्यात् । भ्रातृ॑व्यवा॒निति॒ भ्रातृ॑व्य - वा॒न्॒ । स्याथ् सः ( ) । स ए॒तान् । ए॒तान् जु॑हुयात् । जु॒हु॒या॒द॒भ्या॒ता॒नैः । अ॒भ्या॒ता॒नैरे॒व । अ॒भ्या॒ता॒नैरित्य॑भि - आ॒ता॒नैः । ए॒व भ्रातृ॑व्यान् । भ्रातृ॑व्यान॒भ्यात॑नुते । अ॒भ्यात॑नुते॒ जयैः᳚ । अ॒भ्यात॑नुत॒ इत्य॑भि - आत॑नुते । जयै᳚र् जयति । ज॒य॒ति॒ रा॒ष्ट्र॒भृद्भिः॑ । रा॒ष्ट्र॒भृद्भी॑ रा॒ष्ट्रम् । रा॒ष्ट्र॒भृद्भि॒रिति॑ राष्ट्र॒भृत् - भिः॒ । रा॒ष्टमा । आ द॑त्ते । द॒त्ते॒ भव॑ति । भव॑त्या॒त्मना᳚ । आ॒त्मना॒ परा᳚ । परा᳚ ऽस्य । अ॒स्य॒ भ्रातृ॑व्यः । भ्रातृ॑व्यो भवति । भ॒व॒तीति॑ भवति । \newline

\textbf{Jatai Paata} \newline

1. यद् रा᳚ष्ट्र॒भृद्भी॑ राष्ट्र॒भृद्भि॒र् यद् यद् रा᳚ष्ट्र॒भृद्भिः॑ । \newline
2. रा॒ष्ट्र॒भृद्भी॑ रा॒ष्ट्रꣳ रा॒ष्ट्रꣳ रा᳚ष्ट्र॒भृद्भी॑ राष्ट्र॒भृद्भी॑ रा॒ष्ट्रम् । \newline
3. रा॒ष्ट्र॒भृद्भि॒रिति॑ राष्ट्र॒भृत् - भिः॒ । \newline
4. रा॒ष्ट्र मा रा॒ष्ट्रꣳ रा॒ष्ट्र मा । \newline
5. आ ऽद॑द॒ता द॑द॒ता ऽद॑दत । \newline
6. अद॑दत॒ तत् त दद॑द॒ता द॑दत॒ तत् । \newline
7. तद् रा᳚ष्ट्र॒भृताꣳ॑ राष्ट्र॒भृता॒म् तत् तद् रा᳚ष्ट्र॒भृता᳚म् । \newline
8. रा॒ष्ट्र॒भृताꣳ॑ राष्ट्रभृ॒त्त्वꣳ रा᳚ष्ट्रभृ॒त्त्वꣳ रा᳚ष्ट्र॒भृताꣳ॑ राष्ट्र॒भृताꣳ॑ राष्ट्रभृ॒त्त्वम् । \newline
9. रा॒ष्ट्र॒भृता॒मिति॑ राष्ट्र - भृता᳚म् । \newline
10. रा॒ष्ट्र॒भृ॒त्त्वम् ते ते रा᳚ष्ट्रभृ॒त्त्वꣳ रा᳚ष्ट्रभृ॒त्त्वम् ते । \newline
11. रा॒ष्ट्र॒भृ॒त्त्वमिति॑ राष्ट्रभृत् - त्वम् । \newline
12. ते दे॒वा दे॒वा स्ते ते दे॒वाः । \newline
13. दे॒वा अ॑भ्याता॒नै र॑भ्याता॒नैर् दे॒वा दे॒वा अ॑भ्याता॒नैः । \newline
14. अ॒भ्या॒ता॒नै रसु॑रा॒-नसु॑रा-नभ्याता॒नै र॑भ्याता॒नै रसु॑रान् । \newline
15. अ॒भ्या॒ता॒नैरित्य॑भि - आ॒ता॒नैः । \newline
16. असु॑रा-न॒भ्यात॑न्वता॒ भ्यात॑न्व॒ता सु॑रा॒-नसु॑रा-न॒भ्यात॑न्वत । \newline
17. अ॒भ्यात॑न्वत॒ जयै॒र् जयै॑ र॒भ्यात॑न्वता॒ भ्यात॑न्वत॒ जयैः᳚ । \newline
18. अ॒भ्यात॑न्व॒तेत्य॑भि - आत॑न्वत । \newline
19. जयै॑ रजय-नजय॒न् जयै॒र् जयै॑ रजयन् । \newline
20. अ॒ज॒य॒न् रा॒ष्ट्र॒भृद्भी॑ राष्ट्र॒भृद्भि॑ रजय-नजयन् राष्ट्र॒भृद्भिः॑ । \newline
21. रा॒ष्ट्र॒भृद्भी॑ रा॒ष्ट्रꣳ रा॒ष्ट्रꣳ रा᳚ष्ट्र॒भृद्भी॑ राष्ट्र॒भृद्भी॑ रा॒ष्ट्रम् । \newline
22. रा॒ष्ट्र॒भृद्भि॒रिति॑ राष्ट्र॒भृत् - भिः॒ । \newline
23. रा॒ष्ट्र मा रा॒ष्ट्रꣳ रा॒ष्ट्र मा । \newline
24. आ ऽद॑दता दद॒ता ऽद॑दत । \newline
25. अ॒द॒द॒त॒ यद् यद॑ददता ददत॒ यत् । \newline
26. यद् दे॒वा दे॒वा यद् यद् दे॒वाः । \newline
27. दे॒वा अ॑भ्याता॒नै र॑भ्याता॒नैर् दे॒वा दे॒वा अ॑भ्याता॒नैः । \newline
28. अ॒भ्या॒ता॒नै रसु॑रा॒-नसु॑रा-नभ्याता॒नै र॑भ्याता॒नै रसु॑रान् । \newline
29. अ॒भ्या॒ता॒नैरित्य॑भि - आ॒ता॒नैः । \newline
30. असु॑रा-न॒भ्यात॑न्वता॒ भ्यात॑न्व॒ता सु॑रा॒-नसु॑रा-न॒भ्यात॑न्वत । \newline
31. अ॒भ्यात॑न्वत॒ तत् तद् अ॒भ्यात॑न्वता॒ भ्यात॑न्वत॒ तत् । \newline
32. अ॒भ्यात॑न्व॒तेत्य॑भि - आत॑न्वत । \newline
33. तद॑भ्याता॒नाना॑ मभ्याता॒नाना॒म् तत् तद॑भ्याता॒नाना᳚म् । \newline
34. अ॒भ्या॒ता॒नाना॑ मभ्यातान॒त्व म॑भ्यातान॒त्व म॑भ्याता॒नाना॑ मभ्याता॒नाना॑ मभ्यातान॒त्वम् । \newline
35. अ॒भ्या॒ता॒नाना॒मित्य॑भि - आ॒ता॒नाना᳚म् । \newline
36. अ॒भ्या॒ता॒न॒त्वं ॅयद् यद॑भ्यातान॒त्व म॑भ्यातान॒त्वं ॅयत् । \newline
37. अ॒भ्या॒ता॒न॒त्वमित्य॑भ्यातान - त्वम् । \newline
38. यज् जयै॒र् जयै॒र् यद् यज् जयैः᳚ । \newline
39. जयै॒ रज॑य॒न्-नज॑य॒न् जयै॒र् जयै॒ रज॑यन्न् । \newline
40. अज॑य॒न् तत् तदज॑य॒न्-नज॑य॒न् तत् । \newline
41. तज् जया॑ना॒म् जया॑ना॒म् तत् तज् जया॑नाम् । \newline
42. जया॑नाम् जय॒त्वम् ज॑य॒त्वम् जया॑ना॒म् जया॑नाम् जय॒त्वम् । \newline
43. ज॒य॒त्वं ॅयद् यज् ज॑य॒त्वम् ज॑य॒त्वं ॅयत् । \newline
44. ज॒य॒त्वमिति॑ जय - त्वम् । \newline
45. यद् रा᳚ष्ट्र॒भृद्भी॑ राष्ट्र॒भृद्भि॒र् यद् यद् रा᳚ष्ट्र॒भृद्भिः॑ । \newline
46. रा॒ष्ट्र॒भृद्भी॑ रा॒ष्ट्रꣳ रा॒ष्ट्रꣳ रा᳚ष्ट्र॒भृद्भी॑ राष्ट्र॒भृद्भी॑ रा॒ष्ट्रम् । \newline
47. रा॒ष्ट्र॒भृद्भि॒रिति॑ राष्ट्र॒भृत् - भिः॒ । \newline
48. रा॒ष्ट्र मा रा॒ष्ट्रꣳ रा॒ष्ट्र मा । \newline
49. आ ऽद॑द॒ता द॑द॒ता ऽद॑दत । \newline
50. अद॑दत॒ तत् तदद॑द॒ता द॑दत॒ तत् । \newline
51. तद् रा᳚ष्ट्र॒भृताꣳ॑ राष्ट्र॒भृता॒म् तत् तद् रा᳚ष्ट्र॒भृता᳚म् । \newline
52. रा॒ष्ट्र॒भृताꣳ॑ राष्ट्रभृ॒त्त्वꣳ रा᳚ष्ट्रभृ॒त्त्वꣳ रा᳚ष्ट्र॒भृताꣳ॑ राष्ट्र॒भृताꣳ॑ राष्ट्रभृ॒त्त्वम् । \newline
53. रा॒ष्ट्र॒भृता॒मिति॑ राष्ट्र - भृता᳚म् । \newline
54. रा॒ष्ट्र॒भृ॒त्त्वम् तत॒ स्ततो॑ राष्ट्रभृ॒त्त्वꣳ रा᳚ष्ट्रभृ॒त्त्वम् ततः॑ । \newline
55. रा॒ष्ट्र॒भृ॒त्त्वमिति॑ राष्ट्रभृत् - त्वम् । \newline
56. ततो॑ दे॒वा दे॒वा स्तत॒ स्ततो॑ दे॒वाः । \newline
57. दे॒वा अभ॑व॒न्,नभ॑वन् दे॒वा दे॒वा अभ॑वन्न् । \newline
58. अभ॑व॒न् परा॒ परा ऽभ॑व॒न्-नभ॑व॒न् परा᳚ । \newline
59. परा ऽसु॑रा॒ असु॑राः॒ परा॒ परा ऽसु॑राः । \newline
60. असु॑रा॒ यो यो ऽसु॑रा॒ असु॑रा॒ यः । \newline
61. यो भ्रातृ॑व्यवा॒न् भ्रातृ॑व्यवा॒न्॒. यो यो भ्रातृ॑व्यवान् । \newline
62. भ्रातृ॑व्यवा॒न् थ्स्याथ् स्याद् भ्रातृ॑व्यवा॒न् भ्रातृ॑व्यवा॒न् थ्स्यात् । \newline
63. भ्रातृ॑व्यवा॒निति॒ भ्रातृ॑व्य - वा॒न् । \newline
64. स्याथ् स स स्याथ् स्याथ् सः । \newline
65. स ए॒ता-ने॒तान् थ्स स ए॒तान् । \newline
66. ए॒तान् जु॑हुयाज् जुहुया दे॒ता-ने॒तान् जु॑हुयात् । \newline
67. जु॒हु॒या॒ द॒भ्या॒ता॒नै र॑भ्याता॒नैर् जु॑हुयाज् जुहुया दभ्याता॒नैः । \newline
68. अ॒भ्या॒ता॒नै रे॒वैवा भ्या॑ता॒नै र॑भ्याता॒नै रे॒व । \newline
69. अ॒भ्या॒ता॒नैरित्य॑भि - आ॒ता॒नैः । \newline
70. ए॒व भ्रातृ॑व्या॒न् भ्रातृ॑व्या-ने॒वैव भ्रातृ॑व्यान् । \newline
71. भ्रातृ॑व्या-न॒भ्यात॑नुते॒ ऽभ्यात॑नुते॒ भ्रातृ॑व्या॒न् भ्रातृ॑व्या-न॒भ्यात॑नुते । \newline
72. अ॒भ्यात॑नुते॒ जयै॒र् जयै॑ र॒भ्यात॑नुते॒ ऽभ्यात॑नुते॒ जयैः᳚ । \newline
73. अ॒भ्यात॑नुत॒ इत्य॑भि - आत॑नुते । \newline
74. जयै᳚र् जयति जयति॒ जयै॒र् जयै᳚र् जयति । \newline
75. ज॒य॒ति॒ रा॒ष्ट्र॒भृद्भी॑ राष्ट्र॒भृद्भि॑र् जयति जयति राष्ट्र॒भृद्भिः॑ । \newline
76. रा॒ष्ट्र॒भृद्भी॑ रा॒ष्ट्रꣳ रा॒ष्ट्रꣳ रा᳚ष्ट्र॒भृद्भी॑ राष्ट्र॒भृद्भी॑ रा॒ष्ट्रम् । \newline
77. रा॒ष्ट्र॒भृद्भि॒रिति॑ राष्ट्र॒भृत् - भिः॒ । \newline
78. रा॒ष्ट्रमा रा॒ष्ट्रꣳ रा॒ष्ट्रमा । \newline
79. आ द॑त्ते दत्त॒ आ द॑त्ते । \newline
80. द॒त्ते॒ भव॑ति॒ भव॑ति दत्ते दत्ते॒ भव॑ति । \newline
81. भव॑ त्या॒त्मना॒ ऽऽत्मना॒ भव॑ति॒ भव॑ त्या॒त्मना᳚ । \newline
82. आ॒त्मना॒ परा॒ परा॒ ऽऽत्मना॒ ऽऽत्मना॒ परा᳚ । \newline
83. परा᳚ ऽस्या स्य॒ परा॒ परा᳚ ऽस्य । \newline
84. अ॒स्य॒ भ्रातृ॑व्यो॒ भ्रातृ॑व्यो ऽस्या स्य॒ भ्रातृ॑व्यः । \newline
85. भ्रातृ॑व्यो भवति भवति॒ भ्रातृ॑व्यो॒ भ्रातृ॑व्यो भवति । \newline
86. भ॒व॒तीति॑ भवति । \newline

\textbf{Ghana Paata } \newline

1. यद् रा᳚ष्ट्र॒भृद्भी॑ राष्ट्र॒भृद्भि॒र् यद् यद् रा᳚ष्ट्र॒भृद्भी॑ रा॒ष्ट्रꣳ रा॒ष्ट्रꣳ रा᳚ष्ट्र॒भृद्भि॒र् 
यद् यद् रा᳚ष्ट्र॒भृद्भी॑ रा॒ष्ट्रम् । \newline
2. रा॒ष्ट्र॒भृद्भी॑ रा॒ष्ट्रꣳ रा॒ष्ट्रꣳ रा᳚ष्ट्र॒भृद्भी॑ राष्ट्र॒भृद्भी॑ रा॒ष्ट्रमा रा॒ष्ट्रꣳ 
रा᳚ष्ट्र॒भृद्भी॑ राष्ट्र॒भृद्भी॑ रा॒ष्ट्रमा । \newline
3. रा॒ष्ट्र॒भृद्भि॒रिति॑ राष्ट्र॒भृत् - भिः॒ । \newline
4. रा॒ष्ट्रमा रा॒ष्ट्रꣳ रा॒ष्ट्रमा ऽद॑द॒ता द॑द॒ता रा॒ष्ट्रꣳ रा॒ष्ट्रमा ऽद॑दत । \newline
5. आ ऽद॑द॒ता द॑द॒ता ऽद॑दत॒ तत् तदद॑द॒ता ऽद॑दत॒ तत् । \newline
6. अद॑दत॒ तत् तदद॑द॒ता द॑दत॒ तद् रा᳚ष्ट्र॒भृताꣳ॑ राष्ट्र॒भृता॒म् तदद॑द॒ता द॑दत॒ तद् 
रा᳚ष्ट्र॒भृता᳚म् । \newline
7. तद् रा᳚ष्ट्र॒भृताꣳ॑ राष्ट्र॒भृता॒म् तत् तद् रा᳚ष्ट्र॒भृताꣳ॑ राष्ट्रभृ॒त्त्वꣳ रा᳚ष्ट्रभृ॒त्त्वꣳ रा᳚ष्ट्र॒भृता॒म् तत् तद् रा᳚ष्ट्र॒भृताꣳ॑ राष्ट्रभृ॒त्त्वम् । \newline
8. रा॒ष्ट्र॒भृताꣳ॑ राष्ट्रभृ॒त्त्वꣳ रा᳚ष्ट्रभृ॒त्त्वꣳ रा᳚ष्ट्र॒भृताꣳ॑ राष्ट्र॒भृताꣳ॑ राष्ट्रभृ॒त्त्वम् ते ते रा᳚ष्ट्रभृ॒त्त्वꣳ रा᳚ष्ट्र॒भृताꣳ॑ राष्ट्र॒भृताꣳ॑ राष्ट्रभृ॒त्त्वम् ते । \newline
9. रा॒ष्ट्र॒भृता॒मिति॑ राष्ट्र - भृता᳚म् । \newline
10. रा॒ष्ट्र॒भृ॒त्त्वम् ते ते रा᳚ष्ट्रभृ॒त्त्वꣳ रा᳚ष्ट्रभृ॒त्त्वम् ते दे॒वा दे॒वा स्ते रा᳚ष्ट्रभृ॒त्त्वꣳ 
रा᳚ष्ट्रभृ॒त्त्वम् ते दे॒वाः । \newline
11. रा॒ष्ट्र॒भृ॒त्त्वमिति॑ राष्ट्रभृत् - त्वम् । \newline
12. ते दे॒वा दे॒वा स्ते ते दे॒वा अ॑भ्याता॒नै र॑भ्याता॒नैर् दे॒वा स्ते ते दे॒वा अ॑भ्याता॒नैः । \newline
13. दे॒वा अ॑भ्याता॒नै र॑भ्याता॒नैर् दे॒वा दे॒वा अ॑भ्याता॒नै रसु॑रा॒,नसु॑रा,नभ्याता॒नैर् दे॒वा दे॒वा अ॑भ्याता॒नै रसु॑रान् । \newline
14. अ॒भ्या॒ता॒नै रसु॑रा॒,नसु॑रा,नभ्याता॒नै र॑भ्याता॒नै रसु॑रा,न॒भ्यात॑न्वता॒ भ्यात॑न्व॒ता सु॑रा,नभ्याता॒नै र॑भ्याता॒नै रसु॑रा,न॒भ्यात॑न्वत । \newline
15. अ॒भ्या॒ता॒नैरित्य॑भि - आ॒ता॒नैः । \newline
16. असु॑रा,न॒भ्यात॑न्वता॒ भ्यात॑न्व॒ता सु॑रा॒,नसु॑रा,न॒भ्यात॑न्वत॒ जयै॒र् जयै॑ र॒भ्यात॑न्व॒ता सु॑रा॒,नसु॑रा,न॒भ्यात॑न्वत॒ जयैः᳚ । \newline
17. अ॒भ्यात॑न्वत॒ जयै॒र् जयै॑ र॒भ्यात॑न्वता॒ भ्यात॑न्वत॒ जयै॑ रजय नजय॒न् जयै॑ र॒भ्यात॑न्वता॒ भ्यात॑न्वत॒ जयै॑ रजयन् । \newline
18. अ॒भ्यात॑न्व॒तेत्य॑भि - आत॑न्वत । \newline
19. जयै॑ रजय,नजय॒न् जयै॒र् जयै॑,रजयन् राष्ट्र॒भृद्भी॑ राष्ट्र॒भृद्भि॑ रजय॒न् जयै॒र् जयै॑ रजयन् राष्ट्र॒भृद्भिः॑ । \newline
20. अ॒ज॒य॒न् रा॒ष्ट्र॒भृद्भी॑ राष्ट्र॒भृद्भि॑ रजय,नजयन् राष्ट्र॒भृद्भी॑ रा॒ष्ट्रꣳ रा॒ष्ट्रꣳ 
रा᳚ष्ट्र॒भृद्भि॑ रजय,नजयन् राष्ट्र॒भृद्भी॑ रा॒ष्ट्रम् । \newline
21. रा॒ष्ट्र॒भृद्भी॑ रा॒ष्ट्रꣳ रा॒ष्ट्रꣳ रा᳚ष्ट्र॒भृद्भी॑ राष्ट्र॒भृद्भी॑ रा॒ष्ट्रमा रा॒ष्ट्रꣳ 
रा᳚ष्ट्र॒भृद्भी॑ राष्ट्र॒भृद्भी॑ रा॒ष्ट्रमा । \newline
22. रा॒ष्ट्र॒भृद्भि॒रिति॑ राष्ट्र॒भृत् - भिः॒ । \newline
23. रा॒ष्ट्रमा रा॒ष्ट्रꣳ रा॒ष्ट्रमा ऽद॑दता दद॒ता रा॒ष्ट्रꣳ रा॒ष्ट्रमा ऽद॑दत । \newline
24. आ ऽद॑दता दद॒ता ऽद॑दत॒ यद् यद॑दद॒ता ऽद॑दत॒ यत् । \newline
25. अ॒द॒द॒त॒ यद् यद॑ददता ददत॒ यद् दे॒वा दे॒वा यद॑ददता ददत॒ यद् दे॒वाः । \newline
26. यद् दे॒वा दे॒वा यद् यद् दे॒वा अ॑भ्याता॒नै र॑भ्याता॒नैर् दे॒वा यद् यद् दे॒वा अ॑भ्याता॒नैः । \newline
27. दे॒वा अ॑भ्याता॒नै र॑भ्याता॒नैर् दे॒वा दे॒वा अ॑भ्याता॒नै रसु॑रा॒ नसु॑रा,नभ्याता॒नैर् दे॒वा दे॒वा अ॑भ्याता॒नै,रसु॑रान् । \newline
28. अ॒भ्या॒ता॒नै रसु॑रा॒,नसु॑रा,नभ्याता॒नै र॑भ्याता॒नै रसु॑रा न॒भ्यात॑न्वता॒ भ्यात॑न्व॒ता सु॑रा नभ्याता॒नै र॑भ्याता॒नै रसु॑रा,न॒भ्यात॑न्वत । \newline
29. अ॒भ्या॒ता॒नैरित्य॑भि - आ॒ता॒नैः । \newline
30. असु॑रा,न॒भ्यात॑न्वता॒ भ्यात॑न्व॒ता सु॑रा॒,नसु॑रा, न॒भ्यात॑न्वत॒ तत् तद॒भ्यात॑न्व॒ता सु॑रा॒,नसु॑रा, न॒भ्यात॑न्वत॒ तत् । \newline
31. अ॒भ्यात॑न्वत॒ तत् तद॒भ्यात॑न्वता॒ भ्यात॑न्वत॒ तद॑भ्याता॒नाना॑ मभ्याता॒नाना॒म् तद॒भ्यात॑न्वता॒ भ्यात॑न्वत॒ तद॑भ्याता॒नाना᳚म् । \newline
32. अ॒भ्यात॑न्व॒तेत्य॑भि - आत॑न्वत । \newline
33. तद॑भ्याता॒नाना॑ मभ्याता॒नाना॒म् तत् तद॑भ्याता॒नाना॑ मभ्यातान॒त्व म॑भ्यातान॒त्व म॑भ्याता॒नाना॒म् तत् 
तद॑भ्याता॒नाना॑ मभ्यातान॒त्वम् । \newline
34. अ॒भ्या॒ता॒नाना॑ मभ्यातान॒त्व म॑भ्यातान॒त्व म॑भ्याता॒नाना॑ मभ्याता॒नाना॑ मभ्यातान॒त्वम् ॅयद् यद॑भ्यातान॒त्व म॑भ्याता॒नाना॑ मभ्याता॒नाना॑ मभ्यातान॒त्वम् ॅयत् । \newline
35. अ॒भ्या॒ता॒नाना॒मित्य॑भि - आ॒ता॒नाना᳚म् । \newline
36. अ॒भ्या॒ता॒न॒त्वम् ॅयद् यद॑भ्यातान॒त्व म॑भ्यातान॒त्वम् ॅयज् जयै॒र् जयै॒र् यद॑भ्यातान॒त्व म॑भ्यातान॒त्वम् ॅयज् जयैः᳚ । \newline
37. अ॒भ्या॒ता॒न॒त्वमित्य॑भ्यातान - त्वम् । \newline
38. यज् जयै॒र् जयै॒र् यद् यज् जयै॒ रज॑य॒न्,नज॑य॒न् जयै॒र् यद् यज् जयै॒ रज॑यन्न् । \newline
39. जयै॒ रज॑य॒न्,नज॑य॒न् जयै॒र् जयै॒ रज॑य॒न् तत् तदज॑य॒न् जयै॒र् जयै॒ रज॑य॒न् तत् । \newline
40. अज॑य॒न् तत् तदज॑य॒न्,नज॑य॒न् तज् जया॑ना॒म् जया॑ना॒म् तदज॑य॒न्,नज॑य॒न् तज् जया॑नाम् । \newline
41. तज् जया॑ना॒म् जया॑ना॒म् तत् तज् जया॑नाम् जय॒त्वम् ज॑य॒त्वम् जया॑ना॒म् तत् तज् जया॑नाम् जय॒त्वम् । \newline
42. जया॑नाम् जय॒त्वम् ज॑य॒त्वम् जया॑ना॒म् जया॑नाम् जय॒त्वम् ॅयद् यज् ज॑य॒त्वम् जया॑ना॒म् जया॑नाम् जय॒त्वम् ॅयत् । \newline
43. ज॒य॒त्वम् ॅयद् यज् ज॑य॒त्वम् ज॑य॒त्वम् ॅयद् रा᳚ष्ट्र॒भृद्भी॑ राष्ट्र॒भृद्भि॒र् यज् ज॑य॒त्वम् ज॑य॒त्वम् ॅयद् रा᳚ष्ट्र॒भृद्भिः॑ । \newline
44. ज॒य॒त्वमिति॑ जय - त्वम् । \newline
45. यद् रा᳚ष्ट्र॒भृद्भी॑ राष्ट्र॒भृद्भि॒र् यद् यद् रा᳚ष्ट्र॒भृद्भी॑ रा॒ष्ट्रꣳ रा॒ष्ट्रꣳ रा᳚ष्ट्र॒भृद्भि॒र् यद् यद् रा᳚ष्ट्र॒भृद्भी॑ रा॒ष्ट्रम् । \newline
46. रा॒ष्ट्र॒भृद्भी॑ रा॒ष्ट्रꣳ रा॒ष्ट्रꣳ रा᳚ष्ट्र॒भृद्भी॑ राष्ट्र॒भृद्भी॑ रा॒ष्ट्रमा रा॒ष्ट्रꣳ 
रा᳚ष्ट्र॒भृद्भी॑ राष्ट्र॒भृद्भी॑ रा॒ष्ट्र मा । \newline
47. रा॒ष्ट्र॒भृद्भि॒रिति॑ राष्ट्र॒भृत् - भिः॒ । \newline
48. रा॒ष्ट्रमा रा॒ष्ट्रꣳ रा॒ष्ट्रमा ऽद॑द॒ता द॑द॒ता रा॒ष्ट्रꣳ रा॒ष्ट्रमा ऽद॑दत । \newline
49. आ ऽद॑द॒ता द॑द॒ता ऽद॑दत॒ तत् तदद॑द॒ता ऽद॑दत॒ तत् । \newline
50. अद॑दत॒ तत् तदद॑द॒ता द॑दत॒ तद् रा᳚ष्ट्र॒भृताꣳ॑ राष्ट्र॒भृता॒म् तदद॑द॒ता द॑दत॒ तद् 
रा᳚ष्ट्र॒भृता᳚म् । \newline
51. तद् रा᳚ष्ट्र॒भृताꣳ॑ राष्ट्र॒भृता॒म् तत् तद् रा᳚ष्ट्र॒भृताꣳ॑ राष्ट्रभृ॒त्त्वꣳ रा᳚ष्ट्रभृ॒त्त्वꣳ 
रा᳚ष्ट्र॒भृता॒म् तत् तद् रा᳚ष्ट्र॒भृताꣳ॑ राष्ट्रभृ॒त्त्वम् । \newline
52. रा॒ष्ट्र॒भृताꣳ॑ राष्ट्रभृ॒त्त्वꣳ रा᳚ष्ट्रभृ॒त्त्वꣳ रा᳚ष्ट्र॒भृताꣳ॑ राष्ट्र॒भृताꣳ॑ राष्ट्रभृ॒त्त्वम् तत॒ स्ततो॑ राष्ट्रभृ॒त्त्वꣳ रा᳚ष्ट्र॒भृताꣳ॑ राष्ट्र॒भृताꣳ॑ राष्ट्रभृ॒त्त्वम् ततः॑ । \newline
53. रा॒ष्ट्र॒भृता॒मिति॑ राष्ट्र - भृता᳚म् । \newline
54. रा॒ष्ट्र॒भृ॒त्त्वम् तत॒ स्ततो॑ राष्ट्रभृ॒त्त्वꣳ रा᳚ष्ट्रभृ॒त्त्वम् ततो॑ दे॒वा दे॒वा स्ततो॑ राष्ट्रभृ॒त्त्वꣳ रा᳚ष्ट्रभृ॒त्त्वम् ततो॑ दे॒वाः । \newline
55. रा॒ष्ट्र॒भृ॒त्त्वमिति॑ राष्ट्रभृत् - त्वम् । \newline
56. ततो॑ दे॒वा दे॒वा स्तत॒ स्ततो॑ दे॒वा अभ॑व॒न्,नभ॑वन् दे॒वा स्तत॒ स्ततो॑ दे॒वा अभ॑वन्न् । \newline
57. दे॒वा अभ॑व॒न्,नभ॑वन् दे॒वा दे॒वा अभ॑व॒न् परा॒ परा ऽभ॑वन् दे॒वा दे॒वा अभ॑व॒न् परा᳚ । \newline
58. अभ॑व॒न् परा॒ परा ऽभ॑व॒न्,नभ॑व॒न् परा ऽसु॑रा॒ असु॑राः॒ परा ऽभ॑व॒न्,नभ॑व॒न् परा ऽसु॑राः । \newline
59. परा ऽसु॑रा॒ असु॑राः॒ परा॒ परा ऽसु॑रा॒ यो यो ऽसु॑राः॒ परा॒ परा ऽसु॑रा॒ यः । \newline
60. असु॑रा॒ यो यो ऽसु॑रा॒ असु॑रा॒ यो भ्रातृ॑व्यवा॒न् भ्रातृ॑व्यवा॒न्॒. यो ऽसु॑रा॒ असु॑रा॒ यो भ्रातृ॑व्यवान् । \newline
61. यो भ्रातृ॑व्यवा॒न् भ्रातृ॑व्यवा॒न्॒. यो यो भ्रातृ॑व्यवा॒न् थ्स्या थ्स्याद् भ्रातृ॑व्यवा॒न्॒. यो यो भ्रातृ॑व्यवा॒न् थ्स्यात् । \newline
62. भ्रातृ॑व्यवा॒न् थ्स्या थ्स्याद् भ्रातृ॑व्यवा॒न् भ्रातृ॑व्यवा॒न् थ्स्याथ् स स स्याद् भ्रातृ॑व्यवा॒न् भ्रातृ॑व्यवा॒न् 
थ्स्याथ् सः । \newline
63. भ्रातृ॑व्यवा॒निति॒ भ्रातृ॑व्य - वा॒न् । \newline
64. स्याथ् स स स्याथ् स्याथ् स ए॒ता,ने॒तान् थ्स स्याथ् स्याथ् स ए॒तान् । \newline
65. स ए॒ता,ने॒तान् थ्स स ए॒तान् जु॑हुयाज् जुहुया दे॒तान् थ्स स ए॒तान् जु॑हुयात् । \newline
66. ए॒तान् जु॑हुयाज् जुहुया दे॒ता,ने॒तान् जु॑हुया दभ्याता॒नै र॑भ्याता॒नैर् जु॑हुया दे॒ता,ने॒तान् जु॑हुया दभ्याता॒नैः । \newline
67. जु॒हु॒या॒ द॒भ्या॒ता॒नै र॑भ्याता॒नैर् जु॑हुयाज् जुहुया दभ्याता॒नै रे॒वैवा भ्या॑ता॒नैर् जु॑हुयाज् जुहुया दभ्याता॒नै रे॒व । \newline
68. अ॒भ्या॒ता॒नै रे॒वैवा भ्या॑ता॒नै र॑भ्याता॒नै रे॒व भ्रातृ॑व्या॒न् भ्रातृ॑व्या,ने॒वाभ्या॑ता॒नै र॑भ्याता॒नै 
रे॒व भ्रातृ॑व्यान् । \newline
69. अ॒भ्या॒ता॒नैरित्य॑भि - आ॒ता॒नैः । \newline
70. ए॒व भ्रातृ॑व्या॒न् भ्रातृ॑व्या,ने॒वैव भ्रातृ॑व्या,न॒भ्यात॑नुते॒ ऽभ्यात॑नुते॒ भ्रातृ॑व्या,ने॒वैव भ्रातृ॑व्या, न॒भ्यात॑नुते । \newline
71. भ्रातृ॑व्या,न॒भ्यात॑नुते॒ ऽभ्यात॑नुते॒ भ्रातृ॑व्या॒न् भ्रातृ॑व्या, न॒भ्यात॑नुते॒ जयै॒र् जयै॑ र॒भ्यात॑नुते॒ भ्रातृ॑व्या॒न् भ्रातृ॑व्या न॒भ्यात॑नुते॒ जयैः᳚ । \newline
72. अ॒भ्यात॑नुते॒ जयै॒र् जयै॑ र॒भ्यात॑नुते॒ ऽभ्यात॑नुते॒ जयै᳚र् जयति जयति॒ जयै॑ र॒भ्यात॑नुते॒ ऽभ्यात॑नुते॒ जयै᳚र् जयति । \newline
73. अ॒भ्यात॑नुत॒ इत्य॑भि - आत॑नुते । \newline
74. जयै᳚र् जयति जयति॒ जयै॒र् जयै᳚र् जयति राष्ट्र॒भृद्भी॑ राष्ट्र॒भृद्भि॑र् जयति॒ जयै॒र् जयै᳚र् जयति राष्ट्र॒भृद्भिः॑ । \newline
75. ज॒य॒ति॒ रा॒ष्ट्र॒भृद्भी॑ राष्ट्र॒भृद्भि॑र् जयति जयति राष्ट्र॒भृद्भी॑ रा॒ष्ट्रꣳ रा॒ष्ट्रꣳ 
रा᳚ष्ट्र॒भृद्भि॑र् जयति जयति राष्ट्र॒भृद्भी॑ रा॒ष्ट्रम् । \newline
76. रा॒ष्ट्र॒भृद्भी॑ रा॒ष्ट्रꣳ रा॒ष्ट्रꣳ रा᳚ष्ट्र॒भृद्भी॑ राष्ट्र॒भृद्भी॑ रा॒ष्ट्रमा रा॒ष्ट्रꣳ 
रा᳚ष्ट्र॒भृद्भी॑ राष्ट्र॒भृद्भी॑ रा॒ष्ट्रमा । \newline
77. रा॒ष्ट्र॒भृद्भि॒रिति॑ राष्ट्र॒भृत् - भिः॒ । \newline
78. रा॒ष्ट्रमा रा॒ष्ट्रꣳ रा॒ष्ट्रमा द॑त्ते दत्त॒ आ रा॒ष्ट्रꣳ रा॒ष्ट्रमा द॑त्ते । \newline
79. आ द॑त्ते दत्त॒ आ द॑त्ते॒ भव॑ति॒ भव॑ति दत्त॒ आ द॑त्ते॒ भव॑ति । \newline
80. द॒त्ते॒ भव॑ति॒ भव॑ति दत्ते दत्ते॒ भव॑ त्या॒त्मना॒ ऽऽत्मना॒ भव॑ति दत्ते दत्ते॒ भव॑ त्या॒त्मना᳚ । \newline
81. भव॑त्या॒त्मना॒ ऽऽत्मना॒ भव॑ति॒ भव॑ त्या॒त्मना॒ परा॒ परा॒ ऽऽत्मना॒ भव॑ति॒ भव॑ त्या॒त्मना॒ परा᳚ । \newline
82. आ॒त्मना॒ परा॒ परा॒ ऽऽत्मना॒ ऽऽत्मना॒ परा᳚ ऽस्यास्य॒ परा॒ ऽऽत्मना॒ ऽऽत्मना॒ परा᳚ ऽस्य । \newline
83. परा᳚ ऽस्यास्य॒ परा॒ परा᳚ ऽस्य॒ भ्रातृ॑व्यो॒ भ्रातृ॑व्यो ऽस्य॒ परा॒ परा᳚ ऽस्य॒ भ्रातृ॑व्यः । \newline
84. अ॒स्य॒ भ्रातृ॑व्यो॒ भ्रातृ॑व्यो ऽस्यास्य॒ भ्रातृ॑व्यो भवति भवति॒ भ्रातृ॑व्यो ऽस्यास्य॒ भ्रातृ॑व्यो भवति । \newline
85. भ्रातृ॑व्यो भवति भवति॒ भ्रातृ॑व्यो॒ भ्रातृ॑व्यो भवति । \newline
86. भ॒व॒तीति॑ भवति । \newline
\pagebreak
\markright{ TS 3.4.7.1  \hfill https://www.vedavms.in \hfill}

\section{ TS 3.4.7.1 }

\textbf{TS 3.4.7.1 } \newline
\textbf{Samhita Paata} \newline

ऋ॒ता॒षाड् ऋ॒तधा॑मा॒ऽग्नि-र्ग॑न्ध॒र्वस्त-स्यौष॑धयोऽफ्स॒रस॒ ऊर्जो॒ नाम॒ स इ॒दं ब्रह्म॑ क्ष॒त्रं पा॑तु॒ ता इ॒दं ब्रह्म॑ क्ष॒त्रं पा᳚न्तु॒ तस्मै॒ स्वाहा॒ ताभ्यः॒ स्वाहा॑ सꣳहि॒तो वि॒श्वसा॑मा॒ सूर्यो॑ गन्ध॒र्व-स्तस्य॒ मरी॑चयोऽफ्स॒रस॑ आ॒युवः॑ सुषु॒म्नः सूर्य॑ रश्मि-श्च॒न्द्रमा॑ गन्ध॒र्व-स्तस्य॒ नक्ष॑त्राण्य-फ्स॒रसो॑ बे॒कुर॑योभु॒ज्युः सु॑प॒र्णो य॒ज्ञो ग॑न्ध॒र्व-स्तस्य॒ दक्षि॑णा अप्स॒रस॑ स्त॒वाः प्र॒जाप॑ति-र्वि॒श्वक॑र्मा॒ मनो॑ - [  ] \newline

\textbf{Pada Paata} \newline

ऋ॒ता॒षाट् । ऋ॒तधा॒मेत्यृ॒त - धा॒मा॒ । अ॒ग्निः । ग॒न्ध॒र्वः । तस्य॑ । ओष॑धयः । अ॒फ्स॒रसः॑ । ऊर्जः॑ । नाम॑ । सः । इ॒दम् । ब्रह्म॑ । क्ष॒त्रम् । पा॒तु॒ । ताः । इ॒दम् । ब्रह्म॑ । क्ष॒त्रम् । पा॒न्तु॒ । तस्मै᳚ । स्वाहा᳚ । ताभ्यः॑ । स्वाहा᳚ । सꣳ॒॒हि॒त इति॑ सं - हि॒तः । वि॒श्वसा॒मेति॑ वि॒श्व - सा॒मा॒ । सूर्यः॑ । ग॒न्ध॒र्वः । तस्य॑ । मरी॑चयः । अ॒फ्स॒रसः॑ । आ॒युव॒ इत्या᳚ - युवः॑ । सु॒षु॒म्न इति॑ सु - सु॒म्नः । सूर्य॑रश्मि॒रिति॒ सूर्य॑-र॒श्मिः॒ । च॒न्द्रमाः᳚ । ग॒न्ध॒र्वः । तस्य॑ । नक्ष॑त्राणि । अ॒फ्स॒रसः॑ । बे॒कुर॑यः । भु॒ज्युः । सु॒प॒र्ण इति॑ सु - प॒र्णः । य॒ज्ञ्ः । ग॒न्ध॒र्वः । तस्य॑ । दक्षि॑णाः । अ॒फ्स॒रसः॑ । स्त॒वाः । प्र॒जाप॑ति॒रिति॑ प्र॒जा-प॒तिः॒ । वि॒श्वक॒र्मेति॑ वि॒श्व - क॒र्मा॒ । मनः॑ ।  \newline


\textbf{Krama Paata} \newline

ऋ॒ता॒षाड्,ऋ॒तधा॑मा । ऋ॒तधा॑मा॒ ऽग्निः । ऋ॒तधा॒मेत्यृ॒त - धा॒मा॒ । अ॒ग्निर् ग॑न्ध॒र्वः । ग॒न्ध॒र्व,स्तस्य॑ । तस्यौष॑धयः । ओष॑धयो ऽफ्स॒रसः॑ । अ॒फ्स॒रस॒ ऊर्जः॑ । ऊर्जो॒ नाम॑ । नाम॒ सः । स इ॒दम् । इ॒दम् ब्रह्म॑ । ब्रह्म॑ क्ष॒त्रम् । क्ष॒त्रम् पा॑तु । पा॒तु॒ ताः । ता इ॒दम् । इ॒दम् ब्रह्म॑ । ब्रह्म॑ क्ष॒त्रम् । क्ष॒त्रम् पा᳚न्तु । पा॒न्तु॒ तस्मै᳚ । तस्मै॒ स्वाहा᳚ । स्वाहा॒ ताभ्यः॑ । ताभ्यः॒ स्वाहा᳚ । स्वाहा॑ सꣳहि॒तः । सꣳ॒॒हि॒तो वि॒श्वसा॑मा । सꣳ॒॒हि॒त इति॑ सं - हि॒तः । वि॒श्वसा॑मा॒ सूर्यः॑ । वि॒श्वसा॒मेति॑ वि॒श्व - सा॒मा॒ । सूर्यो॑ गन्ध॒र्वः । ग॒न्ध॒र्वस्तस्य॑ । तस्य॒ मरी॑चयः । मरी॑चयो ऽफ्स॒रसः॑ । अ॒फ्स॒रस॑ आ॒युवः॑ । आ॒युवः॑ सुषु॒म्नः । आ॒युव॒ इत्या᳚ - युवः॑ । सु॒षु॒म्नः 
सूर्य॑रश्मिः । सु॒षु॒म्न इति॑ सु - सु॒म्नः । सूर्य॑रश्मि श्च॒न्द्रमाः᳚ । सूर्य॑रश्मि॒रिति॒ सूर्य॑ - र॒श्मिः॒ । च॒न्द्रमा॑ गन्ध॒र्वः । ग॒न्ध॒र्वस्तस्य॑ । तस्य॒ नक्ष॑त्राणि । नक्ष॑त्राण्यफ्स॒रसः॑ । अ॒फ्स॒रसो॑ बे॒कुर॑यः । बे॒कुर॑यो भु॒ज्युः । भु॒ज्युः सु॑प॒र्णः । सु॒प॒र्णो य॒ज्ञ्ः । सु॒प॒र्ण इति॑ सु - प॒र्णः । य॒ज्ञो ग॑न्ध॒र्वः । ग॒न्ध॒र्वस्तस्य॑ । तस्य॒ दक्षि॑णाः । दक्षि॑णा अफ्स॒रसः॑ । अ॒फ्स॒रसः॑ स्त॒वाः । स्त॒वाः प्र॒जाप॑तिः । प्र॒जाप॑तिर् वि॒श्वक॑र्मा । प्र॒जाप॑ति॒रिति॑ प्र॒जा - प॒तिः॒ । वि॒श्वक॑र्मा॒ मनः॑ । वि॒श्वक॒र्मेति॑ वि॒श्व - क॒र्मा॒ । मनो॑ गन्ध॒र्वः \newline

\textbf{Jatai Paata} \newline

1. ऋ॒ता॒षा डृ॒तधा॑म॒ र्‌तधा॑म र्‌ता॒षा डृ॑ता॒षा डृ॒तधा॑मा । \newline
2. ऋ॒तधा॑मा॒ ऽग्नि र॒ग्निर्. ऋ॒तधा॑म॒ र्‌तधा॑मा॒ ऽग्निः । \newline
3. ऋ॒तधा॒मेत्यृ॒त - धा॒मा॒ । \newline
4. अ॒ग्निर् ग॑न्ध॒र्वो ग॑न्ध॒र्वो᳚ ऽग्नि र॒ग्निर् ग॑न्ध॒र्वः । \newline
5. ग॒न्ध॒र्व स्तस्य॒ तस्य॑ गन्ध॒र्वो ग॑न्ध॒र्व स्तस्य॑ । \newline
6. तस्यौष॑धय॒ ओष॑धय॒ स्तस्य॒ तस्यौष॑धयः । \newline
7. ओष॑धयो ऽफ्स॒रसो᳚ ऽफ्स॒रस॒ ओष॑धय॒ ओष॑धयो ऽफ्स॒रसः॑ । \newline
8. अ॒फ्स॒रस॒ ऊर्ज॒ ऊर्जो᳚ ऽफ्स॒रसो᳚ ऽफ्स॒रस॒ ऊर्जः॑ । \newline
9. ऊर्जो॒ नाम॒ नामोर्ज॒ ऊर्जो॒ नाम॑ । \newline
10. नाम॒ स स नाम॒ नाम॒ सः । \newline
11. स इ॒द मि॒दꣳ स स इ॒दम् । \newline
12. इ॒दम् ब्रह्म॒ ब्रह्मे॒द मि॒दम् ब्रह्म॑ । \newline
13. ब्रह्म॑ क्ष॒त्रम् क्ष॒त्रम् ब्रह्म॒ ब्रह्म॑ क्ष॒त्रम् । \newline
14. क्ष॒त्रम् पा॑तु पातु क्ष॒त्रम् क्ष॒त्रम् पा॑तु । \newline
15. पा॒तु॒ ता स्ताः पा॑तु पातु॒ ताः । \newline
16. ता इ॒द मि॒दम् ता स्ता इ॒दम् । \newline
17. इ॒दम् ब्रह्म॒ ब्रह्मे॒द मि॒दम् ब्रह्म॑ । \newline
18. ब्रह्म॑ क्ष॒त्रम् क्ष॒त्रम् ब्रह्म॒ ब्रह्म॑ क्ष॒त्रम् । \newline
19. क्ष॒त्रम् पा᳚न्तु पान्तु क्ष॒त्रम् क्ष॒त्रम् पा᳚न्तु । \newline
20. पा॒न्तु॒ तस्मै॒ तस्मै॑ पान्तु पान्तु॒ तस्मै᳚ । \newline
21. तस्मै॒ स्वाहा॒ स्वाहा॒ तस्मै॒ तस्मै॒ स्वाहा᳚ । \newline
22. स्वाहा॒ ताभ्य॒ स्ताभ्यः॒ स्वाहा॒ स्वाहा॒ ताभ्यः॑ । \newline
23. ताभ्यः॒ स्वाहा॒ स्वाहा॒ ताभ्य॒ स्ताभ्यः॒ स्वाहा᳚ । \newline
24. स्वाहा॑ सꣳहि॒तः सꣳ॑हि॒तः स्वाहा॒ स्वाहा॑ सꣳहि॒तः । \newline
25. सꣳ॒॒हि॒तो वि॒श्वसा॑मा वि॒श्वसा॑मा सꣳहि॒तः सꣳ॑हि॒तो वि॒श्वसा॑मा । \newline
26. सꣳ॒॒हि॒त इति॑ सं - हि॒तः । \newline
27. वि॒श्वसा॑मा॒ सूर्यः॒ सूर्यो॑ वि॒श्वसा॑मा वि॒श्वसा॑मा॒ सूर्यः॑ । \newline
28. वि॒श्वसा॒मेति॑ वि॒श्व - सा॒मा॒ । \newline
29. सूर्यो॑ गन्ध॒र्वो ग॑न्ध॒र्वः सूर्यः॒ सूर्यो॑ गन्ध॒र्वः । \newline
30. ग॒न्ध॒र्व स्तस्य॒ तस्य॑ गन्ध॒र्वो ग॑न्ध॒र्व स्तस्य॑ । \newline
31. तस्य॒ मरी॑चयो॒ मरी॑चय॒ स्तस्य॒ तस्य॒ मरी॑चयः । \newline
32. मरी॑चयो ऽफ्स॒रसो᳚ ऽफ्स॒रसो॒ मरी॑चयो॒ मरी॑चयो ऽफ्स॒रसः॑ । \newline
33. अ॒फ्स॒रस॑ आ॒युव॑ आ॒युवो᳚ ऽफ्स॒रसो᳚ ऽफ्स॒रस॑ आ॒युवः॑ । \newline
34. आ॒युवः॑ सुषु॒म्नः सु॑षु॒म्न आ॒युव॑ आ॒युवः॑ सुषु॒म्नः । \newline
35. आ॒युव॒ इत्या᳚ - युवः॑ । \newline
36. सु॒षु॒म्नः सूर्य॑रश्मिः॒ सूर्य॑रश्मिः सुषु॒म्नः सु॑षु॒म्नः सूर्य॑रश्मिः । \newline
37. सु॒षु॒म्न इति॑ सु - सु॒म्नः । \newline
38. सूर्य॑रश्मि श्च॒न्द्रमा᳚ श्च॒न्द्रमाः॒ सूर्य॑रश्मिः॒ सूर्य॑रश्मि श्च॒न्द्रमाः᳚ । \newline
39. सूर्य॑रश्मि॒रिति॒ सूर्य॑ - र॒श्मिः॒ । \newline
40. च॒न्द्रमा॑ गन्ध॒र्वो ग॑न्ध॒र्व श्च॒न्द्रमा᳚ श्च॒न्द्रमा॑ गन्ध॒र्वः । \newline
41. ग॒न्ध॒र्व स्तस्य॒ तस्य॑ गन्ध॒र्वो ग॑न्ध॒र्व स्तस्य॑ । \newline
42. तस्य॒ नक्ष॑त्राणि॒ नक्ष॑त्राणि॒ तस्य॒ तस्य॒ नक्ष॑त्राणि । \newline
43. नक्ष॑त्रा ण्यफ्स॒रसो᳚ ऽफ्स॒रसो॒ नक्ष॑त्राणि॒ नक्ष॑त्रा ण्यफ्स॒रसः॑ । \newline
44. अ॒फ्स॒रसो॑ बे॒कुर॑यो बे॒कुर॑यो ऽफ्स॒रसो᳚ ऽफ्स॒रसो॑ बे॒कुर॑यः । \newline
45. बे॒कुर॑यो भु॒ज्युर् भु॒ज्युर् बे॒कुर॑यो बे॒कुर॑यो भु॒ज्युः । \newline
46. भु॒ज्युः सु॑प॒र्णः सु॑प॒र्णो भु॒ज्युर् भु॒ज्युः सु॑प॒र्णः । \newline
47. सु॒प॒र्णो य॒ज्ञो य॒ज्ञ्ः सु॑प॒र्णः सु॑प॒र्णो य॒ज्ञ्ः । \newline
48. सु॒प॒र्ण इति॑ सु - प॒र्णः । \newline
49. य॒ज्ञो ग॑न्ध॒र्वो ग॑न्ध॒र्वो य॒ज्ञो य॒ज्ञो ग॑न्ध॒र्वः । \newline
50. ग॒न्ध॒र्व स्तस्य॒ तस्य॑ गन्ध॒र्वो ग॑न्ध॒र्व स्तस्य॑ । \newline
51. तस्य॒ दक्षि॑णा॒ दक्षि॑णा॒ स्तस्य॒ तस्य॒ दक्षि॑णाः । \newline
52. दक्षि॑णा अफ्स॒रसो᳚ ऽफ्स॒रसो॒ दक्षि॑णा॒ दक्षि॑णा अफ्स॒रसः॑ । \newline
53. अ॒फ्स॒रसः॑ स्त॒वाः स्त॒वा अ॑फ्स॒रसो᳚ ऽफ्स॒रसः॑ स्त॒वाः । \newline
54. स्त॒वाः प्र॒जाप॑तिः प्र॒जाप॑तिः स्त॒वाः स्त॒वाः प्र॒जाप॑तिः । \newline
55. प्र॒जाप॑तिर् वि॒श्वक॑र्मा वि॒श्वक॑र्मा प्र॒जाप॑तिः प्र॒जाप॑तिर् वि॒श्वक॑र्मा । \newline
56. प्र॒जाप॑ति॒रिति॑ प्र॒जा - प॒तिः॒ । \newline
57. वि॒श्वक॑र्मा॒ मनो॒ मनो॑ वि॒श्वक॑र्मा वि॒श्वक॑र्मा॒ मनः॑ । \newline
58. वि॒श्वक॒र्मेति॑ वि॒श्व - क॒र्मा॒ । \newline
59. मनो॑ गन्ध॒र्वो ग॑न्ध॒र्वो मनो॒ मनो॑ गन्ध॒र्वः । \newline

\textbf{Ghana Paata } \newline

1. ऋ॒ता॒षा डृ॒तधा॑म॒ र्‌तधा॑म र्ता॒षाडृ॑ता॒षा डृ॒तधा॑मा॒ ऽग्नि र॒ग्निर्. 
ऋ॒तधा॑म र्‌ता॒षा डृ॑ता॒षा डृ॒तधा॑मा॒ ऽग्निः । \newline
2. ऋ॒तधा॑मा॒ ऽग्नि र॒ग्निर्. ऋ॒तधा॑म॒ र्तधा॑मा॒ ऽग्निर् ग॑न्ध॒र्वो ग॑न्ध॒र्वो᳚ ऽग्निर्. ऋ॒तधा॑म॒ र्तधा॑मा॒ ऽग्निर् ग॑न्ध॒र्वः । \newline
3. ऋ॒तधा॒मेत्यृ॒त - धा॒मा॒ । \newline
4. अ॒ग्निर् ग॑न्ध॒र्वो ग॑न्ध॒र्वो᳚ ऽग्नि र॒ग्निर् ग॑न्ध॒र्व स्तस्य॒ तस्य॑ गन्ध॒र्वो᳚ ऽग्नि र॒ग्निर् ग॑न्ध॒र्व स्तस्य॑ । \newline
5. ग॒न्ध॒र्व स्तस्य॒ तस्य॑ गन्ध॒र्वो ग॑न्ध॒र्व स्तस्यौष॑धय॒ ओष॑धय॒ स्तस्य॑ गन्ध॒र्वो 
ग॑न्ध॒र्व स्तस्यौष॑धयः । \newline
6. तस्यौष॑धय॒ ओष॑धय॒ स्तस्य॒ तस्यौष॑धयो ऽफ्स॒रसो᳚ ऽफ्स॒रस॒ ओष॑धय॒ स्तस्य॒ 
तस्यौष॑धयो ऽफ्स॒रसः॑ । \newline
7. ओष॑धयो ऽफ्स॒रसो᳚ ऽफ्स॒रस॒ ओष॑धय॒ ओष॑धयो ऽफ्स॒रस॒ ऊर्ज॒ ऊर्जो᳚ ऽफ्स॒रस॒ ओष॑धय॒ ओष॑धयो ऽफ्स॒रस॒ ऊर्जः॑ । \newline
8. अ॒फ्स॒रस॒ ऊर्ज॒ ऊर्जो᳚ ऽफ्स॒रसो᳚ ऽफ्स॒रस॒ ऊर्जो॒ नाम॒ नामोर्जो᳚ ऽफ्स॒रसो᳚ ऽफ्स॒रस॒ ऊर्जो॒ नाम॑ । \newline
9. ऊर्जो॒ नाम॒ नामोर्ज॒ ऊर्जो॒ नाम॒ स स नामोर्ज॒ ऊर्जो॒ नाम॒ सः । \newline
10. नाम॒ स स नाम॒ नाम॒ स इ॒द मि॒दꣳ स नाम॒ नाम॒ स इ॒दम् । \newline
11. स इ॒द मि॒दꣳ स स इ॒दम् ब्रह्म॒ ब्रह्मे॒दꣳ स स इ॒दम् ब्रह्म॑ । \newline
12. इ॒दम् ब्रह्म॒ ब्रह्मे॒द मि॒दम् ब्रह्म॑ क्ष॒त्रम् क्ष॒त्रम् ब्रह्मे॒दमि॒दम् ब्रह्म॑ क्ष॒त्रम् । \newline
13. ब्रह्म॑ क्ष॒त्रम् क्ष॒त्रम् ब्रह्म॒ ब्रह्म॑ क्ष॒त्रम् पा॑तु पातु क्ष॒त्रम् ब्रह्म॒ ब्रह्म॑ क्ष॒त्रम् पा॑तु । \newline
14. क्ष॒त्रम् पा॑तु पातु क्ष॒त्रम् क्ष॒त्रम् पा॑तु॒ ता स्ताः पा॑तु क्ष॒त्रम् क्ष॒त्रम् पा॑तु॒ ताः । \newline
15. पा॒तु॒ ता स्ताः पा॑तु पातु॒ ता इ॒द मि॒दम् ताः पा॑तु पातु॒ ता इ॒दम् । \newline
16. ता इ॒द मि॒दम् ता स्ता इ॒दम् ब्रह्म॒ ब्रह्मे॒दम् ता स्ता इ॒दम् ब्रह्म॑ । \newline
17. इ॒दम् ब्रह्म॒ ब्रह्मे॒द मि॒दम् ब्रह्म॑ क्ष॒त्रम् क्ष॒त्रम् ब्रह्मे॒द मि॒दम् ब्रह्म॑ क्ष॒त्रम् । \newline
18. ब्रह्म॑ क्ष॒त्रम् क्ष॒त्रम् ब्रह्म॒ ब्रह्म॑ क्ष॒त्रम् पा᳚न्तु पान्तु क्ष॒त्रम् ब्रह्म॒ ब्रह्म॑ क्ष॒त्रम् पा᳚न्तु । \newline
19. क्ष॒त्रम् पा᳚न्तु पान्तु क्ष॒त्रम् क्ष॒त्रम् पा᳚न्तु॒ तस्मै॒ तस्मै॑ पान्तु क्ष॒त्रम् क्ष॒त्रम् पा᳚न्तु॒ तस्मै᳚ । \newline
20. पा॒न्तु॒ तस्मै॒ तस्मै॑ पान्तु पान्तु॒ तस्मै॒ स्वाहा॒ स्वाहा॒ तस्मै॑ पान्तु पान्तु॒ तस्मै॒ स्वाहा᳚ । \newline
21. तस्मै॒ स्वाहा॒ स्वाहा॒ तस्मै॒ तस्मै॒ स्वाहा॒ ताभ्य॒ स्ताभ्यः॒ स्वाहा॒ तस्मै॒ तस्मै॒ स्वाहा॒ ताभ्यः॑ । \newline
22. स्वाहा॒ ताभ्य॒ स्ताभ्यः॒ स्वाहा॒ स्वाहा॒ ताभ्यः॒ स्वाहा॒ स्वाहा॒ ताभ्यः॒ स्वाहा॒ स्वाहा॒ ताभ्यः॒ स्वाहा᳚ । \newline
23. ताभ्यः॒ स्वाहा॒ स्वाहा॒ ताभ्य॒ स्ताभ्यः॒ स्वाहा॑ सꣳहि॒तः सꣳ॑हि॒तः स्वाहा॒ ताभ्य॒ स्ताभ्यः॒ स्वाहा॑ 
सꣳहि॒तः । \newline
24. स्वाहा॑ सꣳहि॒तः सꣳ॑हि॒तः स्वाहा॒ स्वाहा॑ सꣳहि॒तो वि॒श्वसा॑मा वि॒श्वसा॑मा सꣳहि॒तः स्वाहा॒ स्वाहा॑ 
सꣳहि॒तो वि॒श्वसा॑मा । \newline
25. सꣳ॒॒हि॒तो वि॒श्वसा॑मा वि॒श्वसा॑मा सꣳहि॒तः सꣳ॑हि॒तो वि॒श्वसा॑मा॒ सूर्यः॒ सूर्यो॑ वि॒श्वसा॑मा सꣳहि॒तः सꣳ॑हि॒तो वि॒श्वसा॑मा॒ सूर्यः॑ । \newline
26. सꣳ॒॒हि॒त इति॑ सम् - हि॒तः । \newline
27. वि॒श्वसा॑मा॒ सूर्यः॒ सूर्यो॑ वि॒श्वसा॑मा वि॒श्वसा॑मा॒ सूर्यो॑ गन्ध॒र्वो ग॑न्ध॒र्वः सूर्यो॑ वि॒श्वसा॑मा वि॒श्वसा॑मा॒ सूर्यो॑ गन्ध॒र्वः । \newline
28. वि॒श्वसा॒मेति॑ वि॒श्व - सा॒मा॒ । \newline
29. सूर्यो॑ गन्ध॒र्वो ग॑न्ध॒र्वः सूर्यः॒ सूर्यो॑ गन्ध॒र्व स्तस्य॒ तस्य॑ गन्ध॒र्वः सूर्यः॒ सूर्यो॑ गन्ध॒र्व स्तस्य॑ । \newline
30. ग॒न्ध॒र्व स्तस्य॒ तस्य॑ गन्ध॒र्वो ग॑न्ध॒र्व स्तस्य॒ मरी॑चयो॒ मरी॑चय॒ स्तस्य॑ गन्ध॒र्वो ग॑न्ध॒र्व स्तस्य॒ मरी॑चयः । \newline
31. तस्य॒ मरी॑चयो॒ मरी॑चय॒ स्तस्य॒ तस्य॒ मरी॑चयो ऽफ्स॒रसो᳚ ऽफ्स॒रसो॒ मरी॑चय॒ स्तस्य॒ तस्य॒ मरी॑चयो ऽफ्स॒रसः॑ । \newline
32. मरी॑चयो ऽफ्स॒रसो᳚ ऽफ्स॒रसो॒ मरी॑चयो॒ मरी॑चयो ऽफ्स॒रस॑ आ॒युव॑ आ॒युवो᳚ ऽफ्स॒रसो॒ मरी॑चयो॒ मरी॑चयो ऽफ्स॒रस॑ आ॒युवः॑ । \newline
33. अ॒फ्स॒रस॑ आ॒युव॑ आ॒युवो᳚ ऽफ्स॒रसो᳚ ऽफ्स॒रस॑ आ॒युवः॑ सुषु॒म्नः सु॑षु॒म्न आ॒युवो᳚ ऽफ्स॒रसो᳚ ऽफ्स॒रस॑ आ॒युवः॑ सुषु॒म्नः । \newline
34. आ॒युवः॑ सुषु॒म्नः सु॑षु॒म्न आ॒युव॑ आ॒युवः॑ सुषु॒म्नः सूर्य॑रश्मिः॒ सूर्य॑रश्मिः सुषु॒म्न आ॒युव॑ आ॒युवः॑ सुषु॒म्नः सूर्य॑रश्मिः । \newline
35. आ॒युव॒ इत्या᳚ - युवः॑ । \newline
36. सु॒षु॒म्नः सूर्य॑रश्मिः॒ सूर्य॑रश्मिः सुषु॒म्नः सु॑षु॒म्नः सूर्य॑रश्मि श्च॒न्द्रमा᳚ श्च॒न्द्रमाः॒ सूर्य॑रश्मिः सुषु॒म्नः सु॑षु॒म्नः सूर्य॑रश्मि श्च॒न्द्रमाः᳚ । \newline
37. सु॒षु॒म्न इति॑ सु - सु॒म्नः । \newline
38. सूर्य॑रश्मि श्च॒न्द्रमा᳚ श्च॒न्द्रमाः॒ सूर्य॑रश्मिः॒ सूर्य॑रश्मि श्च॒न्द्रमा॑ गन्ध॒र्वो ग॑न्ध॒र्व श्च॒न्द्रमाः॒ सूर्य॑रश्मिः॒ सूर्य॑रश्मि श्च॒न्द्रमा॑ गन्ध॒र्वः । \newline
39. सूर्य॑रश्मि॒रिति॒ सूर्य॑ - र॒श्मिः॒ । \newline
40. च॒न्द्रमा॑ गन्ध॒र्वो ग॑न्ध॒र्व श्च॒न्द्रमा᳚ श्च॒न्द्रमा॑ गन्ध॒र्व स्तस्य॒ तस्य॑ गन्ध॒र्व श्च॒न्द्रमा᳚ श्च॒न्द्रमा॑ गन्ध॒र्व स्तस्य॑ । \newline
41. ग॒न्ध॒र्व स्तस्य॒ तस्य॑ गन्ध॒र्वो ग॑न्ध॒र्व स्तस्य॒ नक्ष॑त्राणि॒ नक्ष॑त्राणि॒ तस्य॑ गन्ध॒र्वो ग॑न्ध॒र्व स्तस्य॒ नक्ष॑त्राणि । \newline
42. तस्य॒ नक्ष॑त्राणि॒ नक्ष॑त्राणि॒ तस्य॒ तस्य॒ नक्ष॑त्रा ण्यफ्स॒रसो᳚ ऽफ्स॒रसो॒ नक्ष॑त्राणि॒ तस्य॒ तस्य॒ नक्ष॑त्रा ण्यफ्स॒रसः॑ । \newline
43. नक्ष॑त्रा ण्यफ्स॒रसो᳚ ऽफ्स॒रसो॒ नक्ष॑त्राणि॒ नक्ष॑त्रा ण्यफ्स॒रसो॑ बे॒कुर॑यो बे॒कुर॑यो ऽफ्स॒रसो॒ नक्ष॑त्राणि॒ नक्ष॑त्रा ण्यफ्स॒रसो॑ बे॒कुर॑यः । \newline
44. अ॒फ्स॒रसो॑ बे॒कुर॑यो बे॒कुर॑यो ऽफ्स॒रसो᳚ ऽफ्स॒रसो॑ बे॒कुर॑यो भु॒ज्युर् भु॒ज्युर् बे॒कुर॑यो ऽफ्स॒रसो᳚ ऽफ्स॒रसो॑ बे॒कुर॑यो भु॒ज्युः । \newline
45. बे॒कुर॑यो भु॒ज्युर् भु॒ज्युर् बे॒कुर॑यो बे॒कुर॑यो भु॒ज्युः सु॑प॒र्णः सु॑प॒र्णो भु॒ज्युर् बे॒कुर॑यो बे॒कुर॑यो भु॒ज्युः सु॑प॒र्णः । \newline
46. भु॒ज्युः सु॑प॒र्णः सु॑प॒र्णो भु॒ज्युर् भु॒ज्युः सु॑प॒र्णो य॒ज्ञो य॒ज्ञ्ः सु॑प॒र्णो भु॒ज्युर् भु॒ज्युः सु॑प॒र्णो य॒ज्ञ्ः । \newline
47. सु॒प॒र्णो य॒ज्ञो य॒ज्ञ्ः सु॑प॒र्णः सु॑प॒र्णो य॒ज्ञो ग॑न्ध॒र्वो ग॑न्ध॒र्वो य॒ज्ञ्ः सु॑प॒र्णः सु॑प॒र्णो य॒ज्ञो ग॑न्ध॒र्वः । \newline
48. सु॒प॒र्ण इति॑ सु - प॒र्णः । \newline
49. य॒ज्ञो ग॑न्ध॒र्वो ग॑न्ध॒र्वो य॒ज्ञो य॒ज्ञो ग॑न्ध॒र्व स्तस्य॒ तस्य॑ गन्ध॒र्वो य॒ज्ञो य॒ज्ञो ग॑न्ध॒र्व स्तस्य॑ । \newline
50. ग॒न्ध॒र्व स्तस्य॒ तस्य॑ गन्ध॒र्वो ग॑न्ध॒र्व स्तस्य॒ दक्षि॑णा॒ दक्षि॑णा॒ स्तस्य॑ गन्ध॒र्वो ग॑न्ध॒र्व स्तस्य॒ दक्षि॑णाः । \newline
51. तस्य॒ दक्षि॑णा॒ दक्षि॑णा॒ स्तस्य॒ तस्य॒ दक्षि॑णा अफ्स॒रसो᳚ ऽफ्स॒रसो॒ दक्षि॑णा॒ स्तस्य॒ तस्य॒ दक्षि॑णा अफ्स॒रसः॑ । \newline
52. दक्षि॑णा अफ्स॒रसो᳚ ऽफ्स॒रसो॒ दक्षि॑णा॒ दक्षि॑णा अफ्स॒रसः॑ स्त॒वाः स्त॒वा अ॑फ्स॒रसो॒ दक्षि॑णा॒ दक्षि॑णा अफ्स॒रसः॑ स्त॒वाः । \newline
53. अ॒फ्स॒रसः॑ स्त॒वाः स्त॒वा अ॑फ्स॒रसो᳚ ऽफ्स॒रसः॑ स्त॒वाः प्र॒जाप॑तिः प्र॒जाप॑तिः स्त॒वा अ॑फ्स॒रसो᳚ ऽफ्स॒रसः॑ स्त॒वाः प्र॒जाप॑तिः । \newline
54. स्त॒वाः प्र॒जाप॑तिः प्र॒जाप॑तिः स्त॒वाः स्त॒वाः प्र॒जाप॑तिर् वि॒श्वक॑र्मा वि॒श्वक॑र्मा प्र॒जाप॑तिः स्त॒वाः स्त॒वाः 
प्र॒जाप॑तिर् वि॒श्वक॑र्मा । \newline
55. प्र॒जाप॑तिर् वि॒श्वक॑र्मा वि॒श्वक॑र्मा प्र॒जाप॑तिः प्र॒जाप॑तिर् वि॒श्वक॑र्मा॒ मनो॒ मनो॑ वि॒श्वक॑र्मा 
प्र॒जाप॑तिः प्र॒जाप॑तिर् वि॒श्वक॑र्मा॒ मनः॑ । \newline
56. प्र॒जाप॑ति॒रिति॑ प्र॒जा - प॒तिः॒ । \newline
57. वि॒श्वक॑र्मा॒ मनो॒ मनो॑ वि॒श्वक॑र्मा वि॒श्वक॑र्मा॒ मनो॑ गन्ध॒र्वो ग॑न्ध॒र्वो मनो॑ वि॒श्वक॑र्मा वि॒श्वक॑र्मा॒ मनो॑ गन्ध॒र्वः । \newline
58. वि॒श्वक॒र्मेति॑ वि॒श्व - क॒र्मा॒ । \newline
59. मनो॑ गन्ध॒र्वो ग॑न्ध॒र्वो मनो॒ मनो॑ गन्ध॒र्व स्तस्य॒ तस्य॑ गन्ध॒र्वो मनो॒ मनो॑ गन्ध॒र्व स्तस्य॑ । \newline
\pagebreak
\markright{ TS 3.4.7.2  \hfill https://www.vedavms.in \hfill}

\section{ TS 3.4.7.2 }

\textbf{TS 3.4.7.2 } \newline
\textbf{Samhita Paata} \newline

गन्ध॒र्वस्तस्य॑र्ख् सा॒मान्य॑-फ्स॒रसो॒ वह्न॑यैषि॒रो वि॒श्वव्य॑चा॒ वातो॑ गन्ध॒र्व-स्तस्याऽऽ*पो᳚ ऽफ्स॒रसो॑ मु॒दाभुव॑नस्य पते॒ यस्य॑त उ॒परि॑ गृ॒हा इ॒ह च॑ । स नो॑ रा॒स्वाज्या॑निꣳ रा॒यस्पोषꣳ॑ सु॒वीर्यꣳ॑ संवॅथ्स॒रीणाꣳ॑ स्व॒स्तिं ॥ प॒र॒मे॒ष्ठ्यधि॑पति-र्मृ॒त्युर् ग॑न्ध॒र्व-स्तस्य॒ विश्व॑मप्स॒रसो॒ भुवः॑ सुक्षि॒तिः- सुभू॑ति-र्भद्र॒कृथ् सुव॑र्वान् प॒र्जन्यो॑ गन्ध॒र्व-स्तस्य॑ वि॒द्युतो᳚ ऽफ्स॒रसो॒ रुचो॑ दू॒रे हे॑तिर-मृड॒यो - [  ] \newline

\textbf{Pada Paata} \newline

ग॒न्ध॒र्वः । तस्य॑ । ऋ॒ख्सा॒मानीत्यृ॑क्-सा॒मानि॑ । अ॒फ्स॒रसः॑ । वह्न॑यः । इ॒षि॒रः । वि॒श्वव्य॑चा॒ इति॑ वि॒श्व - व्य॒चाः॒ । वातः॑ । ग॒न्ध॒र्वः । तस्य॑ । आपः॑ । अ॒फ्स॒रसः॑ । मु॒दाः । भुव॑नस्य । प॒ते॒ । यस्य॑ । ते॒ । उ॒परि॑ । गृ॒हाः । इ॒ह । च॒ ॥ सः । नः॒ । रा॒स्व॒ । अज्या॑निम् । रा॒यः । पोष᳚म् । सु॒वीर्य॒मिति॑ सु - वीर्य᳚म् । सं॒ॅव॒थ्स॒रीणा॒मिति॑ सं - व॒थ्स॒रीणा᳚म् । स्व॒स्तिम् ॥ प॒र॒मे॒ष्ठी । अधि॑प॒तिरित्यधि॑ - प॒तिः॒ । मृ॒त्युः । ग॒न्ध॒र्वः । तस्य॑ । विश्व᳚म् । अ॒फ्स॒रसः॑ । भुवः॑ । सु॒क्षि॒तिरिति॑ सु - क्षि॒तिः । सुभू॑ति॒रिति॒ सु - भू॒तिः॒ । भ॒द्र॒कृदिति॑ भद्र - कृत् । सुव॑र्वा॒निति॒ सुवः॑ - वा॒न् । प॒र्जन्यः॑ । ग॒न्ध॒र्वः । तस्य॑ । वि॒द्युत॒ इति॑ वि-द्युतः॑ । अ॒फ्स॒रसः॑ । रुचः॑ । दू॒रेहे॑ति॒रिति॑ दू॒रे - हे॒तिः॒ । अ॒मृ॒ड॒यः ।  \newline


\textbf{Krama Paata} \newline

ग॒न्ध॒र्वस्तस्य॑ । तस्य॑र्ख्सा॒मानि॑ । ऋ॒ख्सा॒मान्य॑फ्स॒रसः॑ । ऋ॒ख्सा॒मानीत्यृ॑क् - सा॒मानि॑ । अ॒फ्स॒रसो॒ वह्न॑यः । वह्न॑य इषि॒रः । इ॒षि॒रो वि॒श्वव्य॑चाः । वि॒श्वव्य॑चा॒ वातः॑ । वि॒श्वव्य॑चा॒ इति॑ वि॒श्व - व्य॒चाः॒ । वातो॑ गन्ध॒र्वः । ग॒न्ध॒र्वस्तस्य॑ । तस्यापः॑ । आपो᳚ऽफ्स॒रसः॑ । अ॒फ्स॒रसो॑ मु॒दाः । मु॒दा भुव॑नस्य । भुव॑नस्य पते । प॒ते॒ यस्य॑ । यस्य॑ ते । त॒ उ॒परि॑ । उ॒परि॑ गृ॒हाः । गृ॒हा इ॒ह । इ॒ह च॑ । चेति॑ च ॥ स नः॑ । नो॒ रा॒स्व॒ । रा॒स्वाज्या॑निम् । अज्या॑निꣳ रा॒यः । रा॒यस्पोष᳚म् । पोषꣳ॑ सु॒वीर्य᳚म् । सु॒वीर्यꣳ॑ सम्ॅवथ्स॒रीणा᳚म् । सु॒वीर्य॒मिति॑ सु - वीर्य᳚म् । स॒म्ॅव॒थ्स॒रीणाꣳ॑ स्व॒स्तिम् । स॒म्ॅव॒थ्स॒रीणा॒मिति॑ सम् - व॒थ्स॒रीणा᳚म् । स्व॒स्तिमिति॑ स्व॒स्तिम् ॥ प॒र॒मे॒ष्ठ्यधि॑पतिः । अधि॑पतिर् मृ॒त्युः । अधि॑पति॒रित्यधि॑ - प॒तिः॒ । मृ॒त्युर् ग॑न्ध॒र्वः । ग॒न्ध॒र्वस्तस्य॑ । तस्य॒ विश्व᳚म् । विश्व॑मफ्स॒रसः॑ । अ॒फ्स॒रसो॒ भुवः॑ । भुवः॑ सुक्षि॒तिः । सु॒क्षि॒तिः सुभू॑तिः । सु॒क्षि॒तिरिति॑ सु - क्षि॒तिः । सुभू॑तिर् भद्र॒कृत् । सुभू॑ति॒रिति॒ सु - भू॒तिः॒ । भ॒द्र॒कृथ् सुव॑र्वान् । भ॒द्र॒कृदिति॑ भद्र - कृत् । सुव॑र्वान् प॒र्जन्यः॑ । सुव॑र्वा॒निति॒ सुवः॑ - वा॒न्॒ । प॒र्जन्यो॑ गन्ध॒र्वः । ग॒न्ध॒र्वस्तस्य॑ । तस्य॑ वि॒द्युतः॑ । वि॒द्युतो᳚ ऽफ्स॒रसः॑ । वि॒द्युत॒ इति॑ वि - द्युतः॑ । अ॒फ्स॒रसो॒ रुचः॑ । रुचो॑ दू॒रेहे॑तिः । दू॒रेहे॑तिरमृड॒यः ( ) । दू॒रेहे॑ति॒रिति॑ दू॒रे - हे॒तिः॒ । अ॒मृ॒ड॒यो मृ॒त्युः \newline

\textbf{Jatai Paata} \newline

1. ग॒न्ध॒र्व स्तस्य॒ तस्य॑ गन्ध॒र्वो ग॑न्ध॒र्व स्तस्य॑ । \newline
2. तस्य॑ र्‌ख्सा॒मा न्यृ॑ख्सा॒मानि॒ तस्य॒ तस्य॑ र्‌ख्सा॒मानि॑ । \newline
3. ऋ॒ख्सा॒मा न्य॑फ्स॒रसो᳚ ऽफ्स॒रस॑ ऋख्सा॒मा न्यृ॑ख्सा॒मा न्य॑फ्स॒रसः॑ । \newline
4. ऋ॒ख्सा॒मानीत्यृ॑क् - सा॒मानि॑ । \newline
5. अ॒फ्स॒रसो॒ वह्न॑यो॒ वह्न॑यो ऽफ्स॒रसो᳚ ऽफ्स॒रसो॒ वह्न॑यः । \newline
6. वह्न॑य इषि॒र इ॑षि॒रो वह्न॑यो॒ वह्न॑य इषि॒रः । \newline
7. इ॒षि॒रो वि॒श्वव्य॑चा वि॒श्वव्य॑चा इषि॒र इ॑षि॒रो वि॒श्वव्य॑चाः । \newline
8. वि॒श्वव्य॑चा॒ वातो॒ वातो॑ वि॒श्वव्य॑चा वि॒श्वव्य॑चा॒ वातः॑ । \newline
9. वि॒श्वव्य॑चा॒ इति॑ वि॒श्व - व्य॒चाः॒ । \newline
10. वातो॑ गन्ध॒र्वो ग॑न्ध॒र्वो वातो॒ वातो॑ गन्ध॒र्वः । \newline
11. ग॒न्ध॒र्व स्तस्य॒ तस्य॑ गन्ध॒र्वो ग॑न्ध॒र्व स्तस्य॑ । \newline
12. तस्याप॒ आप॒ स्तस्य॒ तस्यापः॑ । \newline
13. आपो᳚ ऽफ्स॒रसो᳚ ऽफ्स॒रस॒ आप॒ आपो᳚ ऽफ्स॒रसः॑ । \newline
14. अ॒फ्स॒रसो॑ मु॒दा मु॒दा अ॑फ्स॒रसो᳚ ऽफ्स॒रसो॑ मु॒दाः । \newline
15. मु॒दा भुव॑नस्य॒ भुव॑नस्य मु॒दा मु॒दा भुव॑नस्य । \newline
16. भुव॑नस्य पते पते॒ भुव॑नस्य॒ भुव॑नस्य पते । \newline
17. प॒ते॒ यस्य॒ यस्य॑ पते पते॒ यस्य॑ । \newline
18. यस्य॑ ते ते॒ यस्य॒ यस्य॑ ते । \newline
19. त॒ उ॒पर्यु॒परि॑ ते त उ॒परि॑ । \newline
20. उ॒परि॑ गृ॒हा गृ॒हा उ॒पर्यु॒परि॑ गृ॒हाः । \newline
21. गृ॒हा इ॒हेह गृ॒हा गृ॒हा इ॒ह । \newline
22. इ॒ह च॑ चे॒हेह च॑ । \newline
23. चेति॑ च । \newline
24. स नो॑ नः॒ स स नः॑ । \newline
25. नो॒ रा॒स्व॒ रा॒स्व॒ नो॒ नो॒ रा॒स्व॒ । \newline
26. रा॒स्वा ज्या॑नि॒ मज्या॑निꣳ रास्व रा॒स्वा ज्या॑निम् । \newline
27. अज्या॑निꣳ रा॒यो रा॒यो अज्या॑नि॒ मज्या॑निꣳ रा॒यः । \newline
28. रा॒य स्पोष॒म् पोषꣳ॑ रा॒यो रा॒य स्पोष᳚म् । \newline
29. पोषꣳ॑ सु॒वीर्यꣳ॑ सु॒वीर्य॒म् पोष॒म् पोषꣳ॑ सु॒वीर्य᳚म् । \newline
30. सु॒वीर्यꣳ॑ संॅवथ्स॒रीणाꣳ॑ संॅवथ्स॒रीणाꣳ॑ सु॒वीर्यꣳ॑ सु॒वीर्यꣳ॑ संॅवथ्स॒रीणा᳚म् । \newline
31. सु॒वीर्य॒मिति॑ सु - वीर्य᳚म् । \newline
32. सं॒ॅव॒थ्स॒रीणाꣳ॑ स्व॒स्तिꣳ स्व॒स्तिꣳ सं॑ॅवथ्स॒रीणाꣳ॑ संॅवथ्स॒रीणाꣳ॑ स्व॒स्तिम् । \newline
33. सं॒ॅव॒थ्स॒रीणा॒मिति॑ सं - व॒थ्स॒रीणा᳚म् । \newline
34. स्व॒स्तिमिति॑ स्व॒स्तिम् । \newline
35. प॒र॒मे॒ ष्ठ्‌यधि॑पति॒ रधि॑पतिः परमे॒ष्ठी प॑रमे॒ ष्ठ्‌यधि॑पतिः । \newline
36. अधि॑पतिर् मृ॒त्युर् मृ॒त्यु रधि॑पति॒ रधि॑पतिर् मृ॒त्युः । \newline
37. अधि॑पति॒रित्यधि॑ - प॒तिः॒ । \newline
38. मृ॒त्युर् ग॑न्ध॒र्वो ग॑न्ध॒र्वो मृ॒त्युर् मृ॒त्युर् ग॑न्ध॒र्वः । \newline
39. ग॒न्ध॒र्व स्तस्य॒ तस्य॑ गन्ध॒र्वो ग॑न्ध॒र्व स्तस्य॑ । \newline
40. तस्य॒ विश्वं॒ ॅविश्व॒म् तस्य॒ तस्य॒ विश्व᳚म् । \newline
41. विश्व॑ मफ्स॒रसो᳚ ऽफ्स॒रसो॒ विश्वं॒ ॅविश्व॑ मफ्स॒रसः॑ । \newline
42. अ॒फ्स॒रसो॒ भुवो॒ भुवो᳚ ऽफ्स॒रसो᳚ ऽफ्स॒रसो॒ भुवः॑ । \newline
43. भुवः॑ सुक्षि॒तिः सु॑क्षि॒तिर् भुवो॒ भुवः॑ सुक्षि॒तिः । \newline
44. सु॒क्षि॒तिः सुभू॑तिः॒ सुभू॑तिः सुक्षि॒तिः सु॑क्षि॒तिः सुभू॑तिः । \newline
45. सु॒क्षि॒तिरिति॑ सु - क्षि॒तिः । \newline
46. सुभू॑तिर् भद्र॒कृद् भ॑द्र॒कृथ् सुभू॑तिः॒ सुभू॑तिर् भद्र॒कृत् । \newline
47. सुभू॑ति॒रिति॒ सु - भू॒तिः॒ । \newline
48. भ॒द्र॒कृथ् सुव॑र्वा॒न् थ्सुव॑र्वान् भद्र॒कृद् भ॑द्र॒कृथ् सुव॑र्वान् । \newline
49. भ॒द्र॒कृदिति॑ भद्र - कृत् । \newline
50. सुव॑र्वान् प॒र्जन्यः॑ प॒र्जन्यः॒ सुव॑र्वा॒न् थ्सुव॑र्वान् प॒र्जन्यः॑ । \newline
51. सुव॑र्वा॒निति॒ सुवः॑ - वा॒न् । \newline
52. प॒र्जन्यो॑ गन्ध॒र्वो ग॑न्ध॒र्वः प॒र्जन्यः॑ प॒र्जन्यो॑ गन्ध॒र्वः । \newline
53. ग॒न्ध॒र्व स्तस्य॒ तस्य॑ गन्ध॒र्वो ग॑न्ध॒र्व स्तस्य॑ । \newline
54. तस्य॑ वि॒द्युतो॑ वि॒द्युत॒ स्तस्य॒ तस्य॑ वि॒द्युतः॑ । \newline
55. वि॒द्युतो᳚ ऽफ्स॒रसो᳚ ऽफ्स॒रसो॑ वि॒द्युतो॑ वि॒द्युतो᳚ ऽफ्स॒रसः॑ । \newline
56. वि॒द्युत॒ इति॑ वि - द्युतः॑ । \newline
57. अ॒फ्स॒रसो॒ रुचो॒ रुचो᳚ ऽफ्स॒रसो᳚ ऽफ्स॒रसो॒ रुचः॑ । \newline
58. रुचो॑ दू॒रेहे॑तिर् दू॒रे हे॑ती॒ रुचो॒ रुचो॑ दू॒रेहे॑तिः । \newline
59. दू॒रेहे॑ति रमृड॒यो॑ ऽमृड॒यो दू॒रेहे॑तिर् दू॒रेहे॑ति रमृड॒यः । \newline
60. दू॒रेहे॑ति॒रिति॑ दू॒रे - हे॒तिः॒ । \newline
61. अ॒मृ॒ड॒यो मृ॒त्युर् मृ॒त्यु र॑मृड॒यो॑ ऽमृड॒यो मृ॒त्युः । \newline

\textbf{Ghana Paata } \newline

1. ग॒न्ध॒र्व स्तस्य॒ तस्य॑ गन्ध॒र्वो ग॑न्ध॒र्व स्तस्य॑ र्‌ख्सा॒मा न्यृ॑ख्सा॒मानि॒ तस्य॑ गन्ध॒र्वो ग॑न्ध॒र्व स्तस्य॑ र्‌ख्सा॒मानि॑ । \newline
2. तस्य॑ र्‌ख्सा॒मा न्यृ॑ख्सा॒मानि॒ तस्य॒ तस्य॑ र्‌ख्सा॒मा न्य॑फ्स॒रसो᳚ ऽफ्स॒रस॑ ऋख्सा॒मानि॒ तस्य॒ तस्य॑ र्‌ख्सा॒मा न्य॑फ्स॒रसः॑ । \newline
3. ऋ॒ख्सा॒मा न्य॑फ्स॒रसो᳚ ऽफ्स॒रस॑ ऋख्सा॒मा न्यृ॑ख्सा॒मा न्य॑फ्स॒रसो॒ वह्न॑यो॒ वह्न॑यो ऽफ्स॒रस॑ ऋख्सा॒मा न्यृ॑ख्सा॒मा न्य॑फ्स॒रसो॒ वह्न॑यः । \newline
4. ऋ॒ख्सा॒मानीत्यृ॑क् - सा॒मानि॑ । \newline
5. अ॒फ्स॒रसो॒ वह्न॑यो॒ वह्न॑यो ऽफ्स॒रसो᳚ ऽफ्स॒रसो॒ वह्न॑य इषि॒र इ॑षि॒रो वह्न॑यो ऽफ्स॒रसो᳚ ऽफ्स॒रसो॒ वह्न॑य इषि॒रः । \newline
6. वह्न॑य इषि॒र इ॑षि॒रो वह्न॑यो॒ वह्न॑य इषि॒रो वि॒श्वव्य॑चा वि॒श्वव्य॑चा इषि॒रो वह्न॑यो॒ वह्न॑य इषि॒रो वि॒श्वव्य॑चाः । \newline
7. इ॒षि॒रो वि॒श्वव्य॑चा वि॒श्वव्य॑चा इषि॒र इ॑षि॒रो वि॒श्वव्य॑चा॒ वातो॒ वातो॑ वि॒श्वव्य॑चा इषि॒र इ॑षि॒रो वि॒श्वव्य॑चा॒ वातः॑ । \newline
8. वि॒श्वव्य॑चा॒ वातो॒ वातो॑ वि॒श्वव्य॑चा वि॒श्वव्य॑चा॒ वातो॑ गन्ध॒र्वो ग॑न्ध॒र्वो वातो॑ वि॒श्वव्य॑चा वि॒श्वव्य॑चा॒ वातो॑ गन्ध॒र्वः । \newline
9. वि॒श्वव्य॑चा॒ इति॑ वि॒श्व - व्य॒चाः॒ । \newline
10. वातो॑ गन्ध॒र्वो ग॑न्ध॒र्वो वातो॒ वातो॑ गन्ध॒र्व स्तस्य॒ तस्य॑ गन्ध॒र्वो वातो॒ वातो॑ गन्ध॒र्व स्तस्य॑ । \newline
11. ग॒न्ध॒र्व स्तस्य॒ तस्य॑ गन्ध॒र्वो ग॑न्ध॒र्व स्तस्याप॒ आप॒ स्तस्य॑ गन्ध॒र्वो ग॑न्ध॒र्व स्तस्यापः॑ । \newline
12. तस्याप॒ आप॒ स्तस्य॒ तस्यापो᳚ ऽफ्स॒रसो᳚ ऽफ्स॒रस॒ आप॒ स्तस्य॒ तस्यापो᳚ ऽफ्स॒रसः॑ । \newline
13. आपो᳚ ऽफ्स॒रसो᳚ ऽफ्स॒रस॒ आप॒ आपो᳚ ऽफ्स॒रसो॑ मु॒दा मु॒दा अ॑फ्स॒रस॒ आप॒ आपो᳚ ऽफ्स॒रसो॑ मु॒दाः । \newline
14. अ॒फ्स॒रसो॑ मु॒दा मु॒दा अ॑फ्स॒रसो᳚ ऽफ्स॒रसो॑ मु॒दा भुव॑नस्य॒ भुव॑नस्य मु॒दा अ॑फ्स॒रसो᳚ ऽफ्स॒रसो॑ मु॒दा भुव॑नस्य । \newline
15. मु॒दा भुव॑नस्य॒ भुव॑नस्य मु॒दा मु॒दा भुव॑नस्य पते पते॒ भुव॑नस्य मु॒दा मु॒दा भुव॑नस्य पते । \newline
16. भुव॑नस्य पते पते॒ भुव॑नस्य॒ भुव॑नस्य पते॒ यस्य॒ यस्य॑ पते॒ भुव॑नस्य॒ भुव॑नस्य पते॒ यस्य॑ । \newline
17. प॒ते॒ यस्य॒ यस्य॑ पते पते॒ यस्य॑ ते ते॒ यस्य॑ पते पते॒ यस्य॑ ते । \newline
18. यस्य॑ ते ते॒ यस्य॒ यस्य॑ त उ॒पर्यु॒परि॑ ते॒ यस्य॒ यस्य॑ त उ॒परि॑ । \newline
19. त॒ उ॒पर्यु॒ परि॑ ते त उ॒परि॑ गृ॒हा गृ॒हा उ॒परि॑ ते त उ॒परि॑ गृ॒हाः । \newline
20. उ॒परि॑ गृ॒हा गृ॒हा उ॒पर्यु॒ परि॑ गृ॒हा इ॒हेह गृ॒हा उ॒पर्यु॒ परि॑ गृ॒हा इ॒ह । \newline
21. गृ॒हा इ॒हेह गृ॒हा गृ॒हा इ॒ह च॑ चे॒ह गृ॒हा गृ॒हा इ॒ह च॑ । \newline
22. इ॒ह च॑ चे॒ हेह च॑ । \newline
23. चेति॑ च । \newline
24. स नो॑ नः॒ स स नो॑ रास्व रास्व नः॒ स स नो॑ रास्व । \newline
25. नो॒ रा॒स्व॒ रा॒स्व॒ नो॒ नो॒ रा॒स्वाज्या॑नि॒ मज्या॑निꣳ रास्व नो नो रा॒स्वाज्या॑निम् । \newline
26. रा॒स्वाज्या॑नि॒ मज्या॑निꣳ रास्व रा॒स्वाज्या॑निꣳ रा॒यो रा॒यो अज्या॑निꣳ रास्व रा॒स्वाज्या॑निꣳ रा॒यः । \newline
27. अज्या॑निꣳ रा॒यो रा॒यो अज्या॑नि॒ मज्या॑निꣳ रा॒य स्पोष॒म् पोषꣳ॑ रा॒यो अज्या॑नि॒ मज्या॑निꣳ रा॒य स्पोष᳚म् । \newline
28. रा॒य स्पोष॒म् पोषꣳ॑ रा॒यो रा॒य स्पोषꣳ॑ सु॒वीर्यꣳ॑ सु॒वीर्य॒म् पोषꣳ॑ रा॒यो रा॒य स्पोषꣳ॑ सु॒वीर्य᳚म् । \newline
29. पोषꣳ॑ सु॒वीर्यꣳ॑ सु॒वीर्य॒म् पोष॒म् पोषꣳ॑ सु॒वीर्यꣳ॑ सम्ॅवथ्स॒रीणाꣳ॑ सम्ॅवथ्स॒रीणाꣳ॑ सु॒वीर्य॒म् पोष॒म् पोषꣳ॑ सु॒वीर्यꣳ॑ सम्ॅवथ्स॒रीणा᳚म् । \newline
30. सु॒वीर्यꣳ॑ सम्ॅवथ्स॒रीणाꣳ॑ सम्ॅवथ्स॒रीणाꣳ॑ सु॒वीर्यꣳ॑ सु॒वीर्यꣳ॑ सम्ॅवथ्स॒रीणाꣳ॑ स्व॒स्तिꣳ स्व॒स्तिꣳ स॑म्ॅवथ्स॒रीणाꣳ॑ सु॒वीर्यꣳ॑ सु॒वीर्यꣳ॑ सम्ॅवथ्स॒रीणाꣳ॑ स्व॒स्तिम् । \newline
31. सु॒वीर्य॒मिति॑ सु - वीर्य᳚म् । \newline
32. स॒म्ॅव॒थ्स॒रीणाꣳ॑ स्व॒स्तिꣳ स्व॒स्तिꣳ स॑म्ॅवथ्स॒रीणाꣳ॑ सम्ॅवथ्स॒रीणाꣳ॑ स्व॒स्तिम् । \newline
33. स॒म्ॅव॒थ्स॒रीणा॒मिति॑ सम् - व॒थ्स॒रीणा᳚म् । \newline
34. स्व॒स्तिमिति॑ स्व॒स्तिम् । \newline
35. प॒र॒मे॒ ष्ठ्‌यधि॑पति॒ रधि॑पतिः परमे॒ष्ठी प॑रमे॒ ष्ठ्‌यधि॑पतिर् मृ॒त्युर् मृ॒त्यु रधि॑पतिः परमे॒ष्ठी प॑रमे॒ ष्ठ्‌यधि॑पतिर् मृ॒त्युः । \newline
36. अधि॑पतिर् मृ॒त्युर् मृ॒त्यु रधि॑पति॒ रधि॑पतिर् मृ॒त्युर् ग॑न्ध॒र्वो ग॑न्ध॒र्वो मृ॒त्यु रधि॑पति॒ रधि॑पतिर् मृ॒त्युर् ग॑न्ध॒र्वः । \newline
37. अधि॑पति॒रित्यधि॑ - प॒तिः॒ । \newline
38. मृ॒त्युर् ग॑न्ध॒र्वो ग॑न्ध॒र्वो मृ॒त्युर् मृ॒त्युर् ग॑न्ध॒र्व स्तस्य॒ तस्य॑ गन्ध॒र्वो मृ॒त्युर् मृ॒त्युर् ग॑न्ध॒र्व स्तस्य॑ । \newline
39. ग॒न्ध॒र्व स्तस्य॒ तस्य॑ गन्ध॒र्वो ग॑न्ध॒र् वस्तस्य॒ विश्व॒म् ॅविश्व॒म् तस्य॑ गन्ध॒र्वो ग॑न्ध॒र्व स्तस्य॒ विश्व᳚म् । \newline
40. तस्य॒ विश्व॒म् ॅविश्व॒म् तस्य॒ तस्य॒ विश्व॑ मफ्स॒रसो᳚ ऽफ्स॒रसो॒ विश्व॒म् तस्य॒ तस्य॒ विश्व॑ मफ्स॒रसः॑ । \newline
41. विश्व॑ मफ्स॒रसो᳚ ऽफ्स॒रसो॒ विश्व॒म् ॅविश्व॑ मफ्स॒रसो॒ भुवो॒ भुवो᳚ ऽफ्स॒रसो॒ विश्व॒म् ॅविश्व॑ मफ्स॒रसो॒ भुवः॑ । \newline
42. अ॒फ्स॒रसो॒ भुवो॒ भुवो᳚ ऽफ्स॒रसो᳚ ऽफ्स॒रसो॒ भुवः॑ सुक्षि॒तिः सु॑क्षि॒तिर् भुवो᳚ ऽफ्स॒रसो᳚ ऽफ्स॒रसो॒ भुवः॑ सुक्षि॒तिः । \newline
43. भुवः॑ सुक्षि॒तिः सु॑क्षि॒तिर् भुवो॒ भुवः॑ सुक्षि॒तिः सुभू॑तिः॒ सुभू॑तिः सुक्षि॒तिर् भुवो॒ भुवः॑ सुक्षि॒तिः सुभू॑तिः । \newline
44. सु॒क्षि॒तिः सुभू॑तिः॒ सुभू॑तिः सुक्षि॒तिः सु॑क्षि॒तिः सुभू॑तिर् भद्र॒कृद् भ॑द्र॒कृथ् सुभू॑तिः सुक्षि॒तिः सु॑क्षि॒तिः सुभू॑तिर् भद्र॒कृत् । \newline
45. सु॒क्षि॒तिरिति॑ सु - क्षि॒तिः । \newline
46. सुभू॑तिर् भद्र॒कृद् भ॑द्र॒कृथ् सुभू॑तिः॒ सुभू॑तिर् भद्र॒कृथ् सुव॑र्वा॒न् थ्सुव॑र्वान् भद्र॒कृथ् सुभू॑तिः॒ सुभू॑तिर् भद्र॒कृथ् सुव॑र्वान् । \newline
47. सुभू॑ति॒रिति॒ सु - भू॒तिः॒ । \newline
48. भ॒द्र॒कृथ् सुव॑र्वा॒न् थ्सुव॑र्वान् भद्र॒कृद् भ॑द्र॒कृथ् सुव॑र्वान् प॒र्जन्यः॑ प॒र्जन्यः॒ सुव॑र्वान् भद्र॒कृद् भ॑द्र॒कृथ् सुव॑र्वान् प॒र्जन्यः॑ । \newline
49. भ॒द्र॒कृदिति॑ भद्र - कृत् । \newline
50. सुव॑र्वान् प॒र्जन्यः॑ प॒र्जन्यः॒ सुव॑र्वा॒न् थ्सुव॑र्वान् प॒र्जन्यो॑ गन्ध॒र्वो ग॑न्ध॒र्वः प॒र्जन्यः॒ सुव॑र्वा॒न् थ्सुव॑र्वान् प॒र्जन्यो॑ गन्ध॒र्वः । \newline
51. सुव॑र्वा॒निति॒ सुवः॑ - वा॒न् । \newline
52. प॒र्जन्यो॑ गन्ध॒र्वो ग॑न्ध॒र्वः प॒र्जन्यः॑ प॒र्जन्यो॑ गन्ध॒र्व स्तस्य॒ तस्य॑ गन्ध॒र्वः प॒र्जन्यः॑ प॒र्जन्यो॑ गन्ध॒र्व स्तस्य॑ । \newline
53. ग॒न्ध॒र्व स्तस्य॒ तस्य॑ गन्ध॒र्वो ग॑न्ध॒र्व स्तस्य॑ वि॒द्युतो॑ वि॒द्युत॒ स्तस्य॑ गन्ध॒र्वो ग॑न्ध॒र्व स्तस्य॑ वि॒द्युतः॑ । \newline
54. तस्य॑ वि॒द्युतो॑ वि॒द्युत॒ स्तस्य॒ तस्य॑ वि॒द्युतो᳚ ऽफ्स॒रसो᳚ ऽफ्स॒रसो॑ वि॒द्युत॒ स्तस्य॒ तस्य॑ वि॒द्युतो᳚ ऽफ्स॒रसः॑ । \newline
55. वि॒द्युतो᳚ ऽफ्स॒रसो᳚ ऽफ्स॒रसो॑ वि॒द्युतो॑ वि॒द्युतो᳚ ऽफ्स॒रसो॒ रुचो॒ रुचो᳚ ऽफ्स॒रसो॑ वि॒द्युतो॑ वि॒द्युतो᳚ ऽफ्स॒रसो॒ रुचः॑ । \newline
56. वि॒द्युत॒ इति॑ वि - द्युतः॑ । \newline
57. अ॒फ्स॒रसो॒ रुचो॒ रुचो᳚ ऽफ्स॒रसो᳚ ऽफ्स॒रसो॒ रुचो॑ दू॒रेहे॑तिर् दू॒रेहे॑ती॒ रुचो᳚ ऽफ्स॒रसो᳚ ऽफ्स॒रसो॒ रुचो॑ दू॒रेहे॑तिः । \newline
58. रुचो॑ दू॒रेहे॑तिर् दू॒रेहे॑ती॒ रुचो॒ रुचो॑ दू॒रेहे॑ति रमृड॒यो॑ ऽमृड॒यो दू॒रेहे॑ती॒ रुचो॒ रुचो॑ दू॒रेहे॑ति रमृड॒यः । \newline
59. दू॒रेहे॑ति रमृड॒यो॑ ऽमृड॒यो दू॒रेहे॑तिर् दू॒रेहे॑ति रमृड॒यो मृ॒त्युर् मृ॒त्यु र॑मृड॒यो दू॒रेहे॑तिर् दू॒रेहे॑ति रमृड॒यो मृ॒त्युः । \newline
60. दू॒रेहे॑ति॒रिति॑ दू॒रे - हे॒तिः॒ । \newline
61. अ॒मृ॒ड॒यो मृ॒त्युर् मृ॒त्यु र॑मृड॒यो॑ ऽमृड॒यो मृ॒त्युर् ग॑न्ध॒र्वो ग॑न्ध॒र्वो मृ॒त्यु र॑मृड॒यो॑ 
ऽमृड॒यो मृ॒त्युर् ग॑न्ध॒र्वः । \newline
\pagebreak
\markright{ TS 3.4.7.3  \hfill https://www.vedavms.in \hfill}

\section{ TS 3.4.7.3 }

\textbf{TS 3.4.7.3 } \newline
\textbf{Samhita Paata} \newline

मृ॒त्युर्ग॑न्ध॒र्व-स्तस्य॑ प्र॒जा अ॑फ्स॒रसो॑ भी॒रुव॒श्चरुः॑ कृपण का॒शी कामो॑ गन्ध॒र्व-स्तस्या॒धयो᳚ ऽफ्स॒रसः॑ शो॒चय॑न्ती॒र्नाम॒ स इ॒दं ब्रह्म॑ क्ष॒त्रं पा॑त॒ ता इ॒दं ब्रह्म॑ क्ष॒त्रं पा᳚न्तु॒ तस्मै॒ स्वाहा॒ ताभ्यः॒ स्वाहा॒ स नो॑ भुवनस्य पते॒ यस्य॑त उ॒परि॑ गृ॒हा इ॒ह च॑ । उ॒रु ब्र॒ह्म॑णे॒ऽस्मै क्ष॒त्राय॒ महि॒ शर्म॑ यच्छ ॥ \newline

\textbf{Pada Paata} \newline

मृ॒त्युः । ग॒न्ध॒र्वः । तस्य॑ । प्र॒जा इति॑ प्र - जाः । अ॒फ्स॒रसः॑ । भी॒रुवः॑ । चारुः॑ । कृ॒प॒ण॒का॒शीति॑ कृपण-का॒शी । कामः॑ । ग॒न्ध॒र्वः । तस्य॑ । आ॒धय॒ इत्या᳚ - धयः॑ । अ॒फ्स॒रसः॑ । शो॒चय॑न्तीः । नाम॑ । सः । इ॒दम् । ब्रह्म॑ । क्ष॒त्रम् । पा॒तु॒ । ताः । इ॒दम् । ब्रह्म॑ । क्ष॒त्रम् । पा॒न्तु॒ । तस्मै᳚ । स्वाहा᳚ । ताभ्यः॑ । स्वाहा᳚ । सः । नः॒ । भु॒व॒न॒स्य॒ । प॒ते॒ । यस्य॑ । ते॒ । उ॒परि॑ । गृ॒हाः । इ॒ह । च॒ ॥ उ॒रु । ब्रह्म॑णे । अ॒स्मै । क्ष॒त्राय॑ । महि॑ । शर्म॑ । य॒च्छ॒ ॥  \newline


\textbf{Krama Paata} \newline

मृ॒त्युर् ग॑न्ध॒र्वः । ग॒न्ध॒र्वस्तस्य॑ । तस्य॑ प्र॒जाः । प्र॒जा अ॑फ्स॒रसः॑ । प्र॒जा इति॑ प्र - जाः । अ॒फ्स॒रसो॑ भी॒रुवः॑ । भी॒रुव॒श्चारुः॑ । चारुः॑ कृपणका॒शी । कृ॒प॒ण॒का॒शी कामः॑ । कृ॒प॒ण॒का॒शीति॑ कृपण - का॒शी । कामो॑ गन्ध॒र्वः । ग॒न्ध॒र्वस्तस्य॑ । तस्या॒धयः॑ । आ॒धयो᳚ ऽफ्स॒रसः॑ । आ॒धय॒ इत्या᳚ - धयः॑ । अ॒फ्स॒रसः॑ शो॒चय॑न्तीः । शो॒चय॑न्ती॒र् नाम॑ । नाम॒ सः । स इ॒दम् । इ॒दम् ब्रह्म॑ । ब्रह्म॑ क्ष॒त्रम् । क्ष॒त्रम् पा॑तु । पा॒तु॒ ताः । ता इ॒दम् । इ॒दम् ब्रह्म॑ । ब्रह्म॑ क्ष॒त्रम् । क्ष॒त्रम् पा᳚न्तु । पा॒न्तु॒ तस्मै᳚ । तस्मै॒ स्वाहा᳚ । स्वाहा॒ ताभ्यः॑ । ताभ्यः॒ स्वाहा᳚ । स्वाहा॒ सः । स नः॑ । नो॒ भु॒व॒न॒स्य॒ । भु॒व॒न॒स्य॒ प॒ते॒ । प॒ते॒ यस्य॑ । यस्य॑ ते । त॒ उ॒परि॑ । उ॒परि॑ गृ॒हाः । गृ॒हा इ॒ह । इ॒ह च॑ । चेति॑ च ॥ उ॒रु ब्रह्म॑णे । ब्रह्म॑णे॒ ऽस्मै । अ॒स्मै क्ष॒त्राय॑ । क्ष॒त्राय॒ महि॑ । महि॒ शर्म॑ । शर्म॑ यच्छ । य॒च्छे॒ति॑ यच्छ । \newline

\textbf{Jatai Paata} \newline

1. मृ॒त्युर् ग॑न्ध॒र्वो ग॑न्ध॒र्वो मृ॒त्युर् मृ॒त्युर् ग॑न्ध॒र्वः । \newline
2. ग॒न्ध॒र्व स्तस्य॒ तस्य॑ गन्ध॒र्वो ग॑न्ध॒र्व स्तस्य॑ । \newline
3. तस्य॑ प्र॒जाः प्र॒जा स्तस्य॒ तस्य॑ प्र॒जाः । \newline
4. प्र॒जा अ॑फ्स॒रसो᳚ ऽफ्स॒रसः॑ प्र॒जाः प्र॒जा अ॑फ्स॒रसः॑ । \newline
5. प्र॒जा इति॑ प्र - जाः । \newline
6. अ॒फ्स॒रसो॑ भी॒रुवो॑ भी॒रुवो᳚ ऽफ्स॒रसो᳚ ऽफ्स॒रसो॑ भी॒रुवः॑ । \newline
7. भि॒रुव॒ श्चारु॒ श्चारु॑र् भी॒रुवो॑ भि॒रुव॒ श्चारुः॑ । \newline
8. चारुः॑ कृपणका॒शी कृ॑पणका॒शी चारु॒ श्चारुः॑ कृपणका॒शी । \newline
9. कृ॒प॒ण॒का॒शी कामः॒ कामः॑ कृपणका॒शी कृ॑पणका॒शी कामः॑ । \newline
10. कृ॒प॒ण॒का॒शीति॑ कृपण - का॒शी । \newline
11. कामो॑ गन्ध॒र्वो ग॑न्ध॒र्वः कामः॒ कामो॑ गन्ध॒र्वः । \newline
12. ग॒न्ध॒र्व स्तस्य॒ तस्य॑ गन्ध॒र्वो ग॑न्ध॒र्व स्तस्य॑ । \newline
13. तस्या॒ धय॑ आ॒धय॒ स्तस्य॒ तस्या॒ धयः॑ । \newline
14. आ॒धयो᳚ ऽफ्स॒रसो᳚ ऽफ्स॒रस॑ आ॒धय॑ आ॒धयो᳚ ऽफ्स॒रसः॑ । \newline
15. आ॒धय॒ इत्या᳚ - धयः॑ । \newline
16. अ॒फ्स॒रसः॑ शो॒चय॑न्तीः शो॒चय॑न्ती रफ्स॒रसो᳚ ऽफ्स॒रसः॑ शो॒चय॑न्तीः । \newline
17. शो॒चय॑न्ती॒र् नाम॒ नाम॑ शो॒चय॑न्तीः शो॒चय॑न्ती॒र् नाम॑ । \newline
18. नाम॒ स स नाम॒ नाम॒ सः । \newline
19. स इ॒द मि॒दꣳ स स इ॒दम् । \newline
20. इ॒दम् ब्रह्म॒ ब्रह्मे॒द मि॒दम् ब्रह्म॑ । \newline
21. ब्रह्म॑ क्ष॒त्रम् क्ष॒त्रम् ब्रह्म॒ ब्रह्म॑ क्ष॒त्रम् । \newline
22. क्ष॒त्रम् पा॑तु पातु क्ष॒त्रम् क्ष॒त्रम् पा॑तु । \newline
23. पा॒तु॒ ता स्ताः पा॑तु पातु॒ ताः । \newline
24. ता इ॒द मि॒दम् ता स्ता इ॒दम् । \newline
25. इ॒दम् ब्रह्म॒ ब्रह्मे॒द मि॒दम् ब्रह्म॑ । \newline
26. ब्रह्म॑ क्ष॒त्रम् क्ष॒त्रम् ब्रह्म॒ ब्रह्म॑ क्ष॒त्रम् । \newline
27. क्ष॒त्रम् पा᳚न्तु पान्तु क्ष॒त्रम् क्ष॒त्रम् पा᳚न्तु । \newline
28. पा॒न्तु॒ तस्मै॒ तस्मै॑ पान्तु पान्तु॒ तस्मै᳚ । \newline
29. तस्मै॒ स्वाहा॒ स्वाहा॒ तस्मै॒ तस्मै॒ स्वाहा᳚ । \newline
30. स्वाहा॒ ताभ्य॒ स्ताभ्यः॒ स्वाहा॒ स्वाहा॒ ताभ्यः॑ । \newline
31. ताभ्यः॒ स्वाहा॒ स्वाहा॒ ताभ्य॒ स्ताभ्यः॒ स्वाहा᳚ । \newline
32. स्वाहा॒ स स स्वाहा॒ स्वाहा॒ सः । \newline
33. स नो॑ नः॒ स स नः॑ । \newline
34. नो॒ भु॒व॒न॒स्य॒ भु॒व॒न॒स्य॒ नो॒ नो॒ भु॒व॒न॒स्य॒ । \newline
35. भु॒व॒न॒स्य॒ प॒ते॒ प॒ते॒ भु॒व॒न॒स्य॒ भु॒व॒न॒स्य॒ प॒ते॒ । \newline
36. प॒ते॒ यस्य॒ यस्य॑ पते पते॒ यस्य॑ । \newline
37. यस्य॑ ते ते॒ यस्य॒ यस्य॑ ते । \newline
38. त॒ उ॒पर्यु॒परि॑ ते त उ॒परि॑ । \newline
39. उ॒परि॑ गृ॒हा गृ॒हा उ॒पर्यु॒परि॑ गृ॒हाः । \newline
40. गृ॒हा इ॒हेह गृ॒हा गृ॒हा इ॒ह । \newline
41. इ॒ह च॑ चे॒हेह च॑ । \newline
42. चेति॑ च । \newline
43. उ॒रु ब्रह्म॑णे॒ ब्रह्म॑ण उ॒रू॑रु ब्रह्म॑णे । \newline
44. ब्रह्म॑णे॒ ऽस्मा अ॒स्मै ब्रह्म॑णे॒ ब्रह्म॑णे॒ ऽस्मै । \newline
45. अ॒स्मै क्ष॒त्राय॑ क्ष॒त्राया॒ स्मा अ॒स्मै क्ष॒त्राय॑ । \newline
46. क्ष॒त्राय॒ महि॒ महि॑ क्ष॒त्राय॑ क्ष॒त्राय॒ महि॑ । \newline
47. महि॒ शर्म॒ शर्म॒ महि॒ महि॒ शर्म॑ । \newline
48. शर्म॑ यच्छ यच्छ॒ शर्म॒ शर्म॑ यच्छ । \newline
49. य॒च्छेति॑ यच्छ । \newline

\textbf{Ghana Paata } \newline

1. मृ॒त्युर् ग॑न्ध॒र्वो ग॑न्ध॒र्वो मृ॒त्युर् मृ॒त्युर् ग॑न्ध॒र्व स्तस्य॒ तस्य॑ गन्ध॒र्वो मृ॒त्युर् मृ॒त्युर् ग॑न्ध॒र्व स्तस्य॑ । \newline
2. ग॒न्ध॒र्व स्तस्य॒ तस्य॑ गन्ध॒र्वो ग॑न्ध॒र्व स्तस्य॑ प्र॒जाः प्र॒जा स्तस्य॑ गन्ध॒र्वो ग॑न्ध॒र्व स्तस्य॑ प्र॒जाः । \newline
3. तस्य॑ प्र॒जाः प्र॒जा स्तस्य॒ तस्य॑ प्र॒जा अ॑फ्स॒रसो᳚ ऽफ्स॒रसः॑ प्र॒जा स्तस्य॒ तस्य॑ प्र॒जा अ॑फ्स॒रसः॑ । \newline
4. प्र॒जा अ॑फ्स॒रसो᳚ ऽफ्स॒रसः॑ प्र॒जाः प्र॒जा अ॑फ्स॒रसो॑ भी॒रुवो॑ भी॒रुवो᳚ ऽफ्स॒रसः॑ प्र॒जाः प्र॒जा अ॑फ्स॒रसो॑ भी॒रुवः॑ । \newline
5. प्र॒जा इति॑ प्र - जाः । \newline
6. अ॒फ्स॒रसो॑ भी॒रुवो॑ भी॒रुवो᳚ ऽफ्स॒रसो᳚ ऽफ्स॒रसो॑ भी॒रुव॒ श्चारु॒ श्चारु॑र् भी॒रुवो᳚ ऽफ्स॒रसो᳚ ऽफ्स॒रसो॑ भी॒रुव॒ श्चारुः॑ । \newline
7. भी॒रुव॒ श्चारु॒ श्चारु॑र् भी॒रुवो॑ भी॒रुव॒ श्चारुः॑ कृपणका॒शी कृ॑पणका॒शी चारु॑र् भी॒रुवो॑ भी॒रुव॒ श्चारुः॑ कृपणका॒शी । \newline
8. चारुः॑ कृपणका॒शी कृ॑पणका॒शी चारु॒ श्चारुः॑ कृपणका॒शी कामः॒ कामः॑ कृपणका॒शी चारु॒ श्चारुः॑ कृपणका॒शी कामः॑ । \newline
9. कृ॒प॒ण॒का॒शी कामः॒ कामः॑ कृपणका॒शी कृ॑पणका॒शी कामो॑ गन्ध॒र्वो ग॑न्ध॒र्वः कामः॑ कृपणका॒शी कृ॑पणका॒शी कामो॑ गन्ध॒र्वः । \newline
10. कृ॒प॒ण॒का॒शीति॑ कृपण - का॒शी । \newline
11. कामो॑ गन्ध॒र्वो ग॑न्ध॒र्वः कामः॒ कामो॑ गन्ध॒र्व स्तस्य॒ तस्य॑ गन्ध॒र्वः कामः॒ कामो॑ गन्ध॒र्व स्तस्य॑ । \newline
12. ग॒न्ध॒र्व स्तस्य॒ तस्य॑ गन्ध॒र्वो ग॑न्ध॒र्व स्तस्या॒धय॑ आ॒धय॒ स्तस्य॑ गन्ध॒र्वो ग॑न्ध॒र्व 
स्तस्या॒धयः॑ । \newline
13. तस्या॒धय॑ आ॒धय॒ स्तस्य॒ तस्या॒धयो᳚ ऽफ्स॒रसो᳚ ऽफ्स॒रस॑ आ॒धय॒ स्तस्य॒ तस्या॒धयो᳚ ऽफ्स॒रसः॑ । \newline
14. आ॒धयो᳚ ऽफ्स॒रसो᳚ ऽफ्स॒रस॑ आ॒धय॑ आ॒धयो᳚ ऽफ्स॒रसः॑ शो॒चय॑न्तीः शो॒चय॑न्ती रफ्स॒रस॑ आ॒धय॑ 
आ॒धयो᳚ ऽफ्स॒रसः॑ शो॒चय॑न्तीः । \newline
15. आ॒धय॒ इत्या᳚ - धयः॑ । \newline
16. अ॒फ्स॒रसः॑ शो॒चय॑न्तीः शो॒चय॑न्ती रफ्स॒रसो᳚ ऽफ्स॒रसः॑ शो॒चय॑न्ती॒र् नाम॒ नाम॑ शो॒चय॑न्ती 
रफ्स॒रसो᳚ ऽफ्स॒रसः॑ शो॒चय॑न्ती॒र् नाम॑ । \newline
17. शो॒चय॑न्ती॒र् नाम॒ नाम॑ शो॒चय॑न्तीः शो॒चय॑न्ती॒र् नाम॒ स स नाम॑ शो॒चय॑न्तीः शो॒चय॑न्ती॒र् नाम॒ सः । \newline
18. नाम॒ स स नाम॒ नाम॒ स इ॒द मि॒दꣳ स नाम॒ नाम॒ स इ॒दम् । \newline
19. स इ॒द मि॒दꣳ स स इ॒दम् ब्रह्म॒ ब्रह्मे॒दꣳ स स इ॒दम् ब्रह्म॑ । \newline
20. इ॒दम् ब्रह्म॒ ब्रह्मे॒द मि॒दम् ब्रह्म॑ क्ष॒त्रम् क्ष॒त्रम् ब्रह्मे॒द मि॒दम् ब्रह्म॑ क्ष॒त्रम् । \newline
21. ब्रह्म॑ क्ष॒त्रम् क्ष॒त्रम् ब्रह्म॒ ब्रह्म॑ क्ष॒त्रम् पा॑तु पातु क्ष॒त्रम् ब्रह्म॒ ब्रह्म॑ क्ष॒त्रम् पा॑तु । \newline
22. क्ष॒त्रम् पा॑तु पातु क्ष॒त्रम् क्ष॒त्रम् पा॑तु॒ ता स्ताः पा॑तु क्ष॒त्रम् क्ष॒त्रम् पा॑तु॒ ताः । \newline
23. पा॒तु॒ ता स्ताः पा॑तु पातु॒ ता इ॒द मि॒दम् ताः पा॑तु पातु॒ ता इ॒दम् । \newline
24. ता इ॒द मि॒दम् ता स्ता इ॒दम् ब्रह्म॒ ब्रह्मे॒दम् ता स्ता इ॒दम् ब्रह्म॑ । \newline
25. इ॒दम् ब्रह्म॒ ब्रह्मे॒द मि॒दम् ब्रह्म॑ क्ष॒त्रम् क्ष॒त्रम् ब्रह्मे॒द मि॒दम् ब्रह्म॑ क्ष॒त्रम् । \newline
26. ब्रह्म॑ क्ष॒त्रम् क्ष॒त्रम् ब्रह्म॒ ब्रह्म॑ क्ष॒त्रम् पा᳚न्तु पान्तु क्ष॒त्रम् ब्रह्म॒ ब्रह्म॑ क्ष॒त्रम् पा᳚न्तु । \newline
27. क्ष॒त्रम् पा᳚न्तु पान्तु क्ष॒त्रम् क्ष॒त्रम् पा᳚न्तु॒ तस्मै॒ तस्मै॑ पान्तु क्ष॒त्रम् क्ष॒त्रम् पा᳚न्तु॒ तस्मै᳚ । \newline
28. पा॒न्तु॒ तस्मै॒ तस्मै॑ पान्तु पान्तु॒ तस्मै॒ स्वाहा॒ स्वाहा॒ तस्मै॑ पान्तु पान्तु॒ तस्मै॒ स्वाहा᳚ । \newline
29. तस्मै॒ स्वाहा॒ स्वाहा॒ तस्मै॒ तस्मै॒ स्वाहा॒ ताभ्य॒ स्ताभ्यः॒ स्वाहा॒ तस्मै॒ तस्मै॒ स्वाहा॒ ताभ्यः॑ । \newline
30. स्वाहा॒ ताभ्य॒ स्ताभ्यः॒ स्वाहा॒ स्वाहा॒ ताभ्यः॒ स्वाहा॒ स्वाहा॒ ताभ्यः॒ स्वाहा॒ स्वाहा॒ ताभ्यः॒ स्वाहा᳚ । \newline
31. ताभ्यः॒ स्वाहा॒ स्वाहा॒ ताभ्य॒ स्ताभ्यः॒ स्वाहा॒ स स स्वाहा॒ ताभ्य॒ स्ताभ्यः॒ स्वाहा॒ सः । \newline
32. स्वाहा॒ स स स्वाहा॒ स्वाहा॒ स नो॑ नः॒ स स्वाहा॒ स्वाहा॒ स नः॑ । \newline
33. स नो॑ नः॒ स स नो॑ भुवनस्य भुवनस्य नः॒ स स नो॑ भुवनस्य । \newline
34. नो॒ भु॒व॒न॒स्य॒ भु॒व॒न॒स्य॒ नो॒ नो॒ भु॒व॒न॒स्य॒ प॒ते॒ प॒ते॒ भु॒व॒न॒स्य॒ नो॒ नो॒ भु॒व॒न॒स्य॒ प॒ते॒ । \newline
35. भु॒व॒न॒स्य॒ प॒ते॒ प॒ते॒ भु॒व॒न॒स्य॒ भु॒व॒न॒स्य॒ प॒ते॒ यस्य॒ यस्य॑ पते भुवनस्य भुवनस्य पते॒ यस्य॑ । \newline
36. प॒ते॒ यस्य॒ यस्य॑ पते पते॒ यस्य॑ ते ते॒ यस्य॑ पते पते॒ यस्य॑ ते । \newline
37. यस्य॑ ते ते॒ यस्य॒ यस्य॑ त उ॒पर्यु॒परि॑ ते॒ यस्य॒ यस्य॑ त उ॒परि॑ । \newline
38. त॒ उ॒पर्यु॒परि॑ ते त उ॒परि॑ गृ॒हा गृ॒हा उ॒परि॑ ते त उ॒परि॑ गृ॒हाः । \newline
39. उ॒परि॑ गृ॒हा गृ॒हा उ॒पर्यु॒परि॑ गृ॒हा इ॒हेह गृ॒हा उ॒पर्यु॒परि॑ गृ॒हा इ॒ह । \newline
40. गृ॒हा इ॒हेह गृ॒हा गृ॒हा इ॒ह च॑ चे॒ह गृ॒हा गृ॒हा इ॒ह च॑ । \newline
41. इ॒ह च॑ चे॒ हेह च॑ । \newline
42. चेति॑ च । \newline
43. उ॒रु ब्रह्म॑णे॒ ब्रह्म॑ण उ॒रू॑रु ब्रह्म॑णे॒ ऽस्मा अ॒स्मै ब्रह्म॑ण उ॒रू॑रु ब्रह्म॑णे॒ ऽस्मै । \newline
44. ब्रह्म॑णे॒ ऽस्मा अ॒स्मै ब्रह्म॑णे॒ ब्रह्म॑णे॒ ऽस्मै क्ष॒त्राय॑ क्ष॒त्राया॒स्मै ब्रह्म॑णे॒ ब्रह्म॑णे॒ ऽस्मै क्ष॒त्राय॑ । \newline
45. अ॒स्मै क्ष॒त्राय॑ क्ष॒त्राया॒स्मा अ॒स्मै क्ष॒त्राय॒ महि॒ महि॑ क्ष॒त्राया॒स्मा अ॒स्मै क्ष॒त्राय॒ महि॑ । \newline
46. क्ष॒त्राय॒ महि॒ महि॑ क्ष॒त्राय॑ क्ष॒त्राय॒ महि॒ शर्म॒ शर्म॒ महि॑ क्ष॒त्राय॑ क्ष॒त्राय॒ महि॒ शर्म॑ । \newline
47. महि॒ शर्म॒ शर्म॒ महि॒ महि॒ शर्म॑ यच्छ यच्छ॒ शर्म॒ महि॒ महि॒ शर्म॑ यच्छ । \newline
48. शर्म॑ यच्छ यच्छ॒ शर्म॒ शर्म॑ यच्छ । \newline
49. य॒च्छेति॑ यच्छ । \newline
\pagebreak
\markright{ TS 3.4.8.1  \hfill https://www.vedavms.in \hfill}

\section{ TS 3.4.8.1 }

\textbf{TS 3.4.8.1 } \newline
\textbf{Samhita Paata} \newline

रा॒ष्ट्रका॑माय होत॒व्या॑ रा॒ष्ट्रं ॅवै रा᳚ष्ट्र॒भृतो॑ रा॒ष्ट्रेणै॒वास्मै॑ रा॒ष्ट्रमव॑ रुन्धे रा॒ष्ट्रमे॒व भ॑वत्या॒त्मने॑ होत॒व्या॑ रा॒ष्ट्रं ॅवै रा᳚ष्ट्र॒भृतो॑ रा॒ष्ट्रं प्र॒जा रा॒ष्ट्रं प॒शवो॑ रा॒ष्ट्रं ॅयच्छ्रेष्ठो॒ भव॑ति रा॒ष्ट्रेणै॒व रा॒ष्ट्रमव॑ रुन्धे॒ वसि॑ष्ठः समा॒नानां᳚ भवति॒ ग्राम॑कामाय होत॒व्या॑ रा॒ष्ट्रं ॅवै रा᳚ष्ट्र॒भृतो॑ रा॒ष्ट्रꣳ स॑जा॒ता रा॒ष्ट्रेणै॒वास्मै॑ रा॒ष्ट्रꣳ स॑जा॒तानव॑ रुन्धे ग्रा॒म्ये॑ - [  ] \newline

\textbf{Pada Paata} \newline

रा॒ष्ट्रका॑मा॒येति॑ रा॒ष्ट्र - का॒मा॒य॒ । हो॒त॒व्याः᳚ । रा॒ष्ट्रम् । वै । रा॒ष्ट्र॒भृत॒ इति॑ राष्ट्र - भृतः॑ । रा॒ष्ट्रेण॑ । ए॒व । अ॒स्मै॒ । रा॒ष्ट्रम् । अवेति॑ । रु॒न्धे॒ । रा॒ष्ट्रम् । ए॒व । भ॒व॒ति॒ । आ॒त्मने᳚ । हो॒त॒व्याः᳚ । रा॒ष्ट्रम् । वै । रा॒ष्ट्र॒भृत॒ इति॑ राष्ट्र - भृतः॑ । रा॒ष्ट्रम् । प्र॒जेति॑ प्र - जा । रा॒ष्ट्रम् । प॒शवः॑ । रा॒ष्ट्रम् । यत् । श्रेष्ठः॑ । भव॑ति । रा॒ष्ट्रेण॑ । ए॒व । रा॒ष्ट्रम् । अवेति॑ । रु॒न्धे॒ । वसि॑ष्ठः । स॒मा॒नाना᳚म् । भ॒व॒ति॒ । ग्राम॑कामा॒येति॒ ग्राम॑ - का॒मा॒य॒ । हो॒त॒व्याः᳚ । रा॒ष्ट्रम् । वै । रा॒ष्ट्र॒भृत॒ इति॑ राष्ट्र-भृतः॑ । रा॒ष्ट्रम् । स॒जा॒ता इति॑ स - जा॒ताः । रा॒ष्ट्रेण॑ । ए॒व । अ॒स्मै॒ । रा॒ष्ट्रम् । स॒जा॒तानिति॑ स - जा॒तान् । अवेति॑ । रु॒न्धे॒ । ग्रा॒मी ।  \newline


\textbf{Krama Paata} \newline

रा॒ष्ट्रका॑माय होत॒व्याः᳚ । रा॒ष्ट्रका॑मा॒येति॑ रा॒ष्ट्र - का॒मा॒य॒ । हो॒त॒व्या॑ रा॒ष्ट्रम् । रा॒ष्ट्रं ॅवै । वै रा᳚ष्ट्र॒भृतः॑ । रा॒ष्ट्र॒भृतो॑ रा॒ष्ट्रेण॑ । रा॒ष्ट्र॒भृत॒ इति॑ राष्ट्र - भृतः॑ । रा॒ष्ट्रेणै॒व । ए॒वास्मै᳚ । अ॒स्मै॒ रा॒ष्ट्रम् । रा॒ष्ट्रमव॑ । अव॑ रुन्धे । रु॒न्धे॒ रा॒ष्ट्रम् । रा॒ष्ट्रमे॒व । ए॒व भ॑वति । भ॒व॒त्या॒त्मने᳚ । आ॒त्मने॑ होत॒व्याः᳚ । हो॒त॒व्या॑ रा॒ष्ट्रम् । रा॒ष्ट्रं ॅवै । वै रा᳚ष्ट्र॒भृतः॑ । रा॒ष्ट्र॒भृतो॑ रा॒ष्ट्रम् । रा॒ष्ट्र॒भृत॒ इति॑ राष्ट्र - भृतः॑ । रा॒ष्ट्रम् प्र॒जा । प्र॒जा रा॒ष्ट्रम् । प्र॒जेति॑ प्र - जा । रा॒ष्ट्रम् प॒शवः॑ । प॒शवो॑ रा॒ष्ट्रम् । रा॒ष्ट्रं ॅयत् । यच्छ्रेष्ठः॑ । श्रेष्ठो॒ भव॑ति । भव॑ति रा॒ष्ट्रेण॑ । रा॒ष्ट्रेणै॒व । ए॒व रा॒ष्ट्रम् । रा॒ष्ट्रमव॑ । अव॑ रुन्धे । रु॒न्धे॒ वसि॑ष्ठः । वसि॑ष्ठः समा॒नाना᳚म् । स॒मा॒नाना᳚म् भवति । भ॒व॒ति॒ ग्राम॑कामाय । ग्राम॑कामाय होत॒व्याः᳚ । ग्राम॑कामा॒येति॒ ग्राम॑ - का॒मा॒य॒ । हो॒त॒व्या॑ रा॒ष्ट्रम् । रा॒ष्ट्रं ॅवै । वै रा᳚ष्ट्र॒भृतः॑ । रा॒ष्ट्र॒भृतो॑ रा॒ष्ट्रम् । रा॒ष्ट्र॒भृत॒ इति॑ राष्ट्र - भृतः॑ । रा॒ष्ट्रꣳ स॑जा॒ताः । स॒जा॒ता रा॒ष्ट्रेण॑ । स॒जा॒ता इति॑ स - जा॒ताः । रा॒ष्ट्रेणै॒व । ए॒वास्मै᳚ । अ॒स्मै॒ रा॒ष्ट्रम् । रा॒ष्ट्रꣳ स॑जा॒तान् । स॒जा॒तानव॑ । स॒जा॒तानिति॑ स - जा॒तान् । अव॑ रुन्धे । रु॒न्धे॒ ग्रा॒मी । ग्रा॒म्ये॑व \newline

\textbf{Jatai Paata} \newline

1. रा॒ष्ट्रका॑माय होत॒व्या॑ होत॒व्या॑ रा॒ष्ट्रका॑माय रा॒ष्ट्रका॑माय होत॒व्याः᳚ । \newline
2. रा॒ष्ट्रका॑मा॒येति॑ रा॒ष्ट्र - का॒मा॒य॒ । \newline
3. हो॒त॒व्या॑ रा॒ष्ट्रꣳ रा॒ष्ट्रꣳ हो॑त॒व्या॑ होत॒व्या॑ रा॒ष्ट्रम् । \newline
4. रा॒ष्ट्रं ॅवै वै रा॒ष्ट्रꣳ रा॒ष्ट्रं ॅवै । \newline
5. वै रा᳚ष्ट्र॒भृतो॑ राष्ट्र॒भृतो॒ वै वै रा᳚ष्ट्र॒भृतः॑ । \newline
6. रा॒ष्ट्र॒भृतो॑ रा॒ष्ट्रेण॑ रा॒ष्ट्रेण॑ राष्ट्र॒भृतो॑ राष्ट्र॒भृतो॑ रा॒ष्ट्रेण॑ । \newline
7. रा॒ष्ट्र॒भृत॒ इति॑ राष्ट्र - भृतः॑ । \newline
8. रा॒ष्ट्रेणै॒वैव रा॒ष्ट्रेण॑ रा॒ष्ट्रेणै॒व । \newline
9. ए॒वास्मा॑ अस्मा ए॒वैवास्मै᳚ । \newline
10. अ॒स्मै॒ रा॒ष्ट्रꣳ रा॒ष्ट्र म॑स्मा अस्मै रा॒ष्ट्रम् । \newline
11. रा॒ष्ट्र मवाव॑ रा॒ष्ट्रꣳ रा॒ष्ट्र मव॑ । \newline
12. अव॑ रुन्धे रु॒न्धे ऽवाव॑ रुन्धे । \newline
13. रु॒न्धे॒ रा॒ष्ट्रꣳ रा॒ष्ट्रꣳ रु॑न्धे रुन्धे रा॒ष्ट्रम् । \newline
14. रा॒ष्ट्र मे॒वैव रा॒ष्ट्रꣳ रा॒ष्ट्र मे॒व । \newline
15. ए॒व भ॑वति भव त्ये॒वैव भ॑वति । \newline
16. भ॒व॒ त्या॒त्मन॑ आ॒त्मने॑ भवति भव त्या॒त्मने᳚ । \newline
17. आ॒त्मने॑ होत॒व्या॑ होत॒व्या॑ आ॒त्मन॑ आ॒त्मने॑ होत॒व्याः᳚ । \newline
18. हो॒त॒व्या॑ रा॒ष्ट्रꣳ रा॒ष्ट्रꣳ हो॑त॒व्या॑ होत॒व्या॑ रा॒ष्ट्रम् । \newline
19. रा॒ष्ट्रं ॅवै वै रा॒ष्ट्रꣳ रा॒ष्ट्रं ॅवै । \newline
20. वै रा᳚ष्ट्र॒भृतो॑ राष्ट्र॒भृतो॒ वै वै रा᳚ष्ट्र॒भृतः॑ । \newline
21. रा॒ष्ट्र॒भृतो॑ रा॒ष्ट्रꣳ रा॒ष्ट्रꣳ रा᳚ष्ट्र॒भृतो॑ राष्ट्र॒भृतो॑ रा॒ष्ट्रम् । \newline
22. रा॒ष्ट्र॒भृत॒ इति॑ राष्ट्र - भृतः॑ । \newline
23. रा॒ष्ट्रम् प्र॒जा प्र॒जा रा॒ष्ट्रꣳ रा॒ष्ट्रम् प्र॒जा । \newline
24. प्र॒जा रा॒ष्ट्रꣳ रा॒ष्ट्रम् प्र॒जा प्र॒जा रा॒ष्ट्रम् । \newline
25. प्र॒जेति॑ प्र - जा । \newline
26. रा॒ष्ट्रम् प॒शवः॑ प॒शवो॑ रा॒ष्ट्रꣳ रा॒ष्ट्रम् प॒शवः॑ । \newline
27. प॒शवो॑ रा॒ष्ट्रꣳ रा॒ष्ट्रम् प॒शवः॑ प॒शवो॑ रा॒ष्ट्रम् । \newline
28. रा॒ष्ट्रं ॅयद् यद् रा॒ष्ट्रꣳ रा॒ष्ट्रं ॅयत् । \newline
29. यच्छ्रेष्ठः॒ श्रेष्ठो॒ यद् यच्छ्रेष्ठः॑ । \newline
30. श्रेष्ठो॒ भव॑ति॒ भव॑ति॒ श्रेष्ठः॒ श्रेष्ठो॒ भव॑ति । \newline
31. भव॑ति रा॒ष्ट्रेण॑ रा॒ष्ट्रेण॒ भव॑ति॒ भव॑ति रा॒ष्ट्रेण॑ । \newline
32. रा॒ष्ट्रेणै॒वैव रा॒ष्ट्रेण॑ रा॒ष्ट्रेणै॒व । \newline
33. ए॒व रा॒ष्ट्रꣳ रा॒ष्ट्र मे॒वैव रा॒ष्ट्रम् । \newline
34. रा॒ष्ट्र मवाव॑ रा॒ष्ट्रꣳ रा॒ष्ट्र मव॑ । \newline
35. अव॑ रुन्धे रु॒न्धे ऽवाव॑ रुन्धे । \newline
36. रु॒न्धे॒ वसि॑ष्ठो॒ वसि॑ष्ठो रुन्धे रुन्धे॒ वसि॑ष्ठः । \newline
37. वसि॑ष्ठः समा॒नानाꣳ॑ समा॒नानां॒ ॅवसि॑ष्ठो॒ वसि॑ष्ठः समा॒नाना᳚म् । \newline
38. स॒मा॒नाना᳚म् भवति भवति समा॒नानाꣳ॑ समा॒नाना᳚म् भवति । \newline
39. भ॒व॒ति॒ ग्राम॑कामाय॒ ग्राम॑कामाय भवति भवति॒ ग्राम॑कामाय । \newline
40. ग्राम॑कामाय होत॒व्या॑ होत॒व्या᳚ ग्राम॑कामाय॒ ग्राम॑कामाय होत॒व्याः᳚ । \newline
41. ग्राम॑कामा॒येति॒ ग्राम॑ - का॒मा॒य॒ । \newline
42. हो॒त॒व्या॑ रा॒ष्ट्रꣳ रा॒ष्ट्रꣳ हो॑त॒व्या॑ होत॒व्या॑ रा॒ष्ट्रम् । \newline
43. रा॒ष्ट्रं ॅवै वै रा॒ष्ट्रꣳ रा॒ष्ट्रं ॅवै । \newline
44. वै रा᳚ष्ट्र॒भृतो॑ राष्ट्र॒भृतो॒ वै वै रा᳚ष्ट्र॒भृतः॑ । \newline
45. रा॒ष्ट्र॒भृतो॑ रा॒ष्ट्रꣳ रा॒ष्ट्रꣳ रा᳚ष्ट्र॒भृतो॑ राष्ट्र॒भृतो॑ रा॒ष्ट्रम् । \newline
46. रा॒ष्ट्र॒भृत॒ इति॑ राष्ट्र - भृतः॑ । \newline
47. रा॒ष्ट्रꣳ स॑जा॒ताः स॑जा॒ता रा॒ष्ट्रꣳ रा॒ष्ट्रꣳ स॑जा॒ताः । \newline
48. स॒जा॒ता रा॒ष्ट्रेण॑ रा॒ष्ट्रेण॑ सजा॒ताः स॑जा॒ता रा॒ष्ट्रेण॑ । \newline
49. स॒जा॒ता इति॑ स - जा॒ताः । \newline
50. रा॒ष्ट्रेणै॒वैव रा॒ष्ट्रेण॑ रा॒ष्ट्रेणै॒व । \newline
51. ए॒वास्मा॑ अस्मा ए॒वैवास्मै᳚ । \newline
52. अ॒स्मै॒ रा॒ष्ट्रꣳ रा॒ष्ट्र म॑स्मा अस्मै रा॒ष्ट्रम् । \newline
53. रा॒ष्ट्रꣳ स॑जा॒तान् थ्स॑जा॒तान् रा॒ष्ट्रꣳ रा॒ष्ट्रꣳ स॑जा॒तान् । \newline
54. स॒जा॒ता-नवाव॑ सजा॒तान् थ्स॑जा॒ता-नव॑ । \newline
55. स॒जा॒तानिति॑ स - जा॒तान् । \newline
56. अव॑ रुन्धे रु॒न्धे ऽवाव॑ रुन्धे । \newline
57. रु॒न्धे॒ ग्रा॒मी ग्रा॒मी रु॑न्धे रुन्धे ग्रा॒मी । \newline
58. ग्रा॒म्ये॑वैव ग्रा॒मी ग्रा॒म्ये॑व । \newline

\textbf{Ghana Paata } \newline

1. रा॒ष्ट्रका॑माय होत॒व्या॑ होत॒व्या॑ रा॒ष्ट्रका॑माय रा॒ष्ट्रका॑माय होत॒व्या॑ रा॒ष्ट्रꣳ रा॒ष्ट्रꣳ हो॑त॒व्या॑ रा॒ष्ट्रका॑माय रा॒ष्ट्रका॑माय होत॒व्या॑ रा॒ष्ट्रम् । \newline
2. रा॒ष्ट्रका॑मा॒येति॑ रा॒ष्ट्र - का॒मा॒य॒ । \newline
3. हो॒त॒व्या॑ रा॒ष्ट्रꣳ रा॒ष्ट्रꣳ हो॑त॒व्या॑ होत॒व्या॑ रा॒ष्ट्रम् ॅवै वै रा॒ष्ट्रꣳ हो॑त॒व्या॑ होत॒व्या॑ रा॒ष्ट्रम् ॅवै । \newline
4. रा॒ष्ट्रम् ॅवै वै रा॒ष्ट्रꣳ रा॒ष्ट्रम् ॅवै रा᳚ष्ट्र॒भृतो॑ राष्ट्र॒भृतो॒ वै रा॒ष्ट्रꣳ रा॒ष्ट्रम् ॅवै रा᳚ष्ट्र॒भृतः॑ । \newline
5. वै रा᳚ष्ट्र॒भृतो॑ राष्ट्र॒भृतो॒ वै वै रा᳚ष्ट्र॒भृतो॑ रा॒ष्ट्रेण॑ रा॒ष्ट्रेण॑ राष्ट्र॒भृतो॒ वै वै 
रा᳚ष्ट्र॒भृतो॑ रा॒ष्ट्रेण॑ । \newline
6. रा॒ष्ट्र॒भृतो॑ रा॒ष्ट्रेण॑ रा॒ष्ट्रेण॑ राष्ट्र॒भृतो॑ राष्ट्र॒भृतो॑ रा॒ष्ट्रेणै॒वैव रा॒ष्ट्रेण॑ राष्ट्र॒भृतो॑ राष्ट्र॒भृतो॑ रा॒ष्ट्रेणै॒व । \newline
7. रा॒ष्ट्र॒भृत॒ इति॑ राष्ट्र - भृतः॑ । \newline
8. रा॒ष्ट्रे णै॒वैव रा॒ष्ट्रेण॑ रा॒ष्ट्रे णै॒वास्मा॑ अस्मा ए॒व रा॒ष्ट्रेण॑ रा॒ष्ट्रे णै॒वा स्मै᳚ । \newline
9. ए॒वास्मा॑ अस्मा ए॒वैवास्मै॑ रा॒ष्ट्रꣳ रा॒ष्ट्र म॑स्मा ए॒वैवास्मै॑ रा॒ष्ट्रम् । \newline
10. अ॒स्मै॒ रा॒ष्ट्रꣳ रा॒ष्ट्र म॑स्मा अस्मै रा॒ष्ट्र मवाव॑ रा॒ष्ट्र म॑स्मा अस्मै रा॒ष्ट्र मव॑ । \newline
11. रा॒ष्ट्र मवाव॑ रा॒ष्ट्रꣳ रा॒ष्ट्र मव॑ रुन्धे रु॒न्धे ऽव॑ रा॒ष्ट्रꣳ रा॒ष्ट्र मव॑ रुन्धे । \newline
12. अव॑ रुन्धे रु॒न्धे ऽवाव॑ रुन्धे रा॒ष्ट्रꣳ रा॒ष्ट्रꣳ रु॒न्धे ऽवाव॑ रुन्धे रा॒ष्ट्रम् । \newline
13. रु॒न्धे॒ रा॒ष्ट्रꣳ रा॒ष्ट्रꣳ रु॑न्धे रुन्धे रा॒ष्ट्र मे॒वैव रा॒ष्ट्रꣳ रु॑न्धे रुन्धे रा॒ष्ट्र मे॒व । \newline
14. रा॒ष्ट्र मे॒वैव रा॒ष्ट्रꣳ रा॒ष्ट्र मे॒व भ॑वति भव त्ये॒व रा॒ष्ट्रꣳ रा॒ष्ट्र मे॒व भ॑वति । \newline
15. ए॒व भ॑वति भव त्ये॒वैव भ॑व त्या॒त्मन॑ आ॒त्मने॑ भव त्ये॒वैव भ॑व त्या॒त्मने᳚ । \newline
16. भ॒व॒ त्या॒त्मन॑ आ॒त्मने॑ भवति भव त्या॒त्मने॑ होत॒व्या॑ होत॒व्या॑ आ॒त्मने॑ भवति भव त्या॒त्मने॑ होत॒व्याः᳚ । \newline
17. आ॒त्मने॑ होत॒व्या॑ होत॒व्या॑ आ॒त्मन॑ आ॒त्मने॑ होत॒व्या॑ रा॒ष्ट्रꣳ रा॒ष्ट्रꣳ हो॑त॒व्या॑ आ॒त्मन॑ आ॒त्मने॑ होत॒व्या॑ रा॒ष्ट्रम् । \newline
18. हो॒त॒व्या॑ रा॒ष्ट्रꣳ रा॒ष्ट्रꣳ हो॑त॒व्या॑ होत॒व्या॑ रा॒ष्ट्रम् ॅवै वै रा॒ष्ट्रꣳ हो॑त॒व्या॑ होत॒व्या॑ रा॒ष्ट्रम् ॅवै । \newline
19. रा॒ष्ट्रम् ॅवै वै रा॒ष्ट्रꣳ रा॒ष्ट्रम् ॅवै रा᳚ष्ट्र॒भृतो॑ राष्ट्र॒भृतो॒ वै रा॒ष्ट्रꣳ रा॒ष्ट्रम् ॅवै रा᳚ष्ट्र॒भृतः॑ । \newline
20. वै रा᳚ष्ट्र॒भृतो॑ राष्ट्र॒भृतो॒ वै वै रा᳚ष्ट्र॒भृतो॑ रा॒ष्ट्रꣳ रा॒ष्ट्रꣳ रा᳚ष्ट्र॒भृतो॒ वै वै 
रा᳚ष्ट्र॒भृतो॑ रा॒ष्ट्रम् । \newline
21. रा॒ष्ट्र॒भृतो॑ रा॒ष्ट्रꣳ रा॒ष्ट्रꣳ रा᳚ष्ट्र॒भृतो॑ राष्ट्र॒भृतो॑ रा॒ष्ट्रम् प्र॒जा प्र॒जा रा॒ष्ट्रꣳ 
रा᳚ष्ट्र॒भृतो॑ राष्ट्र॒भृतो॑ रा॒ष्ट्रम् प्र॒जा । \newline
22. रा॒ष्ट्र॒भृत॒ इति॑ राष्ट्र - भृतः॑ । \newline
23. रा॒ष्ट्रम् प्र॒जा प्र॒जा रा॒ष्ट्रꣳ रा॒ष्ट्रम् प्र॒जा रा॒ष्ट्रꣳ रा॒ष्ट्रम् प्र॒जा रा॒ष्ट्रꣳ रा॒ष्ट्रम् प्र॒जा रा॒ष्ट्रम् । \newline
24. प्र॒जा रा॒ष्ट्रꣳ रा॒ष्ट्रम् प्र॒जा प्र॒जा रा॒ष्ट्रम् प॒शवः॑ प॒शवो॑ रा॒ष्ट्रम् प्र॒जा प्र॒जा रा॒ष्ट्रम् प॒शवः॑ । \newline
25. प्र॒जेति॑ प्र - जा । \newline
26. रा॒ष्ट्रम् प॒शवः॑ प॒शवो॑ रा॒ष्ट्रꣳ रा॒ष्ट्रम् प॒शवो॑ रा॒ष्ट्रꣳ रा॒ष्ट्रम् प॒शवो॑ रा॒ष्ट्रꣳ रा॒ष्ट्रम् प॒शवो॑ रा॒ष्ट्रम् । \newline
27. प॒शवो॑ रा॒ष्ट्रꣳ रा॒ष्ट्रम् प॒शवः॑ प॒शवो॑ रा॒ष्ट्रम् ॅयद् यद् रा॒ष्ट्रम् प॒शवः॑ प॒शवो॑ रा॒ष्ट्रम् ॅयत् । \newline
28. रा॒ष्ट्रम् ॅयद् यद् रा॒ष्ट्रꣳ रा॒ष्ट्रम् ॅय च्छ्रेष्ठः॒ श्रेष्ठो॒ यद् रा॒ष्ट्रꣳ रा॒ष्ट्रम् ॅय च्छ्रेष्ठः॑ । \newline
29. य च्छ्रेष्ठः॒ श्रेष्ठो॒ यद् य च्छ्रेष्ठो॒ भव॑ति॒ भव॑ति॒ श्रेष्ठो॒ यद् य च्छ्रेष्ठो॒ भव॑ति । \newline
30. श्रेष्ठो॒ भव॑ति॒ भव॑ति॒ श्रेष्ठः॒ श्रेष्ठो॒ भव॑ति रा॒ष्ट्रेण॑ रा॒ष्ट्रेण॒ भव॑ति॒ श्रेष्ठः॒ श्रेष्ठो॒ भव॑ति रा॒ष्ट्रेण॑ । \newline
31. भव॑ति रा॒ष्ट्रेण॑ रा॒ष्ट्रेण॒ भव॑ति॒ भव॑ति रा॒ष्ट्रे णै॒वैव रा॒ष्ट्रेण॒ भव॑ति॒ भव॑ति रा॒ष्ट्रेणै॒व । \newline
32. रा॒ष्ट्रे णै॒वैव रा॒ष्ट्रेण॑ रा॒ष्ट्रेणै॒व रा॒ष्ट्रꣳ रा॒ष्ट्र मे॒व रा॒ष्ट्रेण॑ रा॒ष्ट्रे णै॒व रा॒ष्ट्रम् । \newline
33. ए॒व रा॒ष्ट्रꣳ रा॒ष्ट्र मे॒वैव रा॒ष्ट्र मवाव॑ रा॒ष्ट्र मे॒वैव रा॒ष्ट्र मव॑ । \newline
34. रा॒ष्ट्र मवाव॑ रा॒ष्ट्रꣳ रा॒ष्ट्र मव॑ रुन्धे रु॒न्धे ऽव॑ रा॒ष्ट्रꣳ रा॒ष्ट्र मव॑ रुन्धे । \newline
35. अव॑ रुन्धे रु॒न्धे ऽवाव॑ रुन्धे॒ वसि॑ष्ठो॒ वसि॑ष्ठो रु॒न्धे ऽवाव॑ रुन्धे॒ वसि॑ष्ठः । \newline
36. रु॒न्धे॒ वसि॑ष्ठो॒ वसि॑ष्ठो रुन्धे रुन्धे॒ वसि॑ष्ठः समा॒नानाꣳ॑ समा॒नाना॒म् ॅवसि॑ष्ठो रुन्धे रुन्धे॒ वसि॑ष्ठः समा॒नाना᳚म् । \newline
37. वसि॑ष्ठः समा॒नानाꣳ॑ समा॒नाना॒म् ॅवसि॑ष्ठो॒ वसि॑ष्ठः समा॒नाना᳚म् भवति भवति समा॒नाना॒म् ॅवसि॑ष्ठो॒ वसि॑ष्ठः समा॒नाना᳚म् भवति । \newline
38. स॒मा॒नाना᳚म् भवति भवति समा॒नानाꣳ॑ समा॒नाना᳚म् भवति॒ ग्राम॑कामाय॒ ग्राम॑कामाय भवति समा॒नानाꣳ॑ समा॒नाना᳚म् भवति॒ ग्राम॑कामाय । \newline
39. भ॒व॒ति॒ ग्राम॑कामाय॒ ग्राम॑कामाय भवति भवति॒ ग्राम॑कामाय होत॒व्या॑ होत॒व्या᳚ ग्राम॑कामाय भवति भवति॒ ग्राम॑कामाय होत॒व्याः᳚ । \newline
40. ग्राम॑कामाय होत॒व्या॑ होत॒व्या᳚ ग्राम॑कामाय॒ ग्राम॑कामाय होत॒व्या॑ रा॒ष्ट्रꣳ रा॒ष्ट्रꣳ हो॑त॒व्या᳚ ग्राम॑कामाय॒ ग्राम॑कामाय होत॒व्या॑ रा॒ष्ट्रम् । \newline
41. ग्राम॑कामा॒येति॒ ग्राम॑ - का॒मा॒य॒ । \newline
42. हो॒त॒व्या॑ रा॒ष्ट्रꣳ रा॒ष्ट्रꣳ हो॑त॒व्या॑ होत॒व्या॑ रा॒ष्ट्रम् ॅवै वै रा॒ष्ट्रꣳ हो॑त॒व्या॑ होत॒व्या॑ रा॒ष्ट्रम् ॅवै । \newline
43. रा॒ष्ट्रम् ॅवै वै रा॒ष्ट्रꣳ रा॒ष्ट्रम् ॅवै रा᳚ष्ट्र॒भृतो॑ राष्ट्र॒भृतो॒ वै रा॒ष्ट्रꣳ रा॒ष्ट्रम् ॅवै रा᳚ष्ट्र॒भृतः॑ । \newline
44. वै रा᳚ष्ट्र॒भृतो॑ राष्ट्र॒भृतो॒ वै वै रा᳚ष्ट्र॒भृतो॑ रा॒ष्ट्रꣳ रा॒ष्ट्रꣳ रा᳚ष्ट्र॒भृतो॒ वै वै 
रा᳚ष्ट्र॒भृतो॑ रा॒ष्ट्रम् । \newline
45. रा॒ष्ट्र॒भृतो॑ रा॒ष्ट्रꣳ रा॒ष्ट्रꣳ रा᳚ष्ट्र॒भृतो॑ राष्ट्र॒भृतो॑ रा॒ष्ट्रꣳ स॑जा॒ताः स॑जा॒ता रा॒ष्ट्रꣳ रा᳚ष्ट्र॒भृतो॑ राष्ट्र॒भृतो॑ रा॒ष्ट्रꣳ स॑जा॒ताः । \newline
46. रा॒ष्ट्र॒भृत॒ इति॑ राष्ट्र - भृतः॑ । \newline
47. रा॒ष्ट्रꣳ स॑जा॒ताः स॑जा॒ता रा॒ष्ट्रꣳ रा॒ष्ट्रꣳ स॑जा॒ता रा॒ष्ट्रेण॑ रा॒ष्ट्रेण॑ सजा॒ता रा॒ष्ट्रꣳ रा॒ष्ट्रꣳ स॑जा॒ता रा॒ष्ट्रेण॑ । \newline
48. स॒जा॒ता रा॒ष्ट्रेण॑ रा॒ष्ट्रेण॑ सजा॒ताः स॑जा॒ता रा॒ष्ट्रे णै॒वैव रा॒ष्ट्रेण॑ सजा॒ताः स॑जा॒ता रा॒ष्ट्रेणै॒व । \newline
49. स॒जा॒ता इति॑ स - जा॒ताः । \newline
50. रा॒ष्ट्रे णै॒वैव रा॒ष्ट्रेण॑ रा॒ष्ट्रे णै॒वास्मा॑ अस्मा ए॒व रा॒ष्ट्रेण॑ रा॒ष्ट्रे णै॒वास्मै᳚ । \newline
51. ए॒वास्मा॑ अस्मा ए॒वै वास्मै॑ रा॒ष्ट्रꣳ रा॒ष्ट्र म॑स्मा ए॒वै वास्मै॑ रा॒ष्ट्रम् । \newline
52. अ॒स्मै॒ रा॒ष्ट्रꣳ रा॒ष्ट्र म॑स्मा अस्मै रा॒ष्ट्रꣳ स॑जा॒तान् थ्स॑जा॒तान् रा॒ष्ट्र म॑स्मा अस्मै रा॒ष्ट्रꣳ स॑जा॒तान् । \newline
53. रा॒ष्ट्रꣳ स॑जा॒तान् थ्स॑जा॒तान् रा॒ष्ट्रꣳ रा॒ष्ट्रꣳ स॑जा॒ता नवाव॑ सजा॒तान् रा॒ष्ट्रꣳ रा॒ष्ट्रꣳ स॑जा॒ता,नव॑ । \newline
54. स॒जा॒ता,नवाव॑ सजा॒तान् थ्स॑जा॒ता,नव॑ रुन्धे रु॒न्धे ऽव॑ सजा॒तान् थ्स॑जा॒ता,नव॑ रुन्धे । \newline
55. स॒जा॒तानिति॑ स - जा॒तान् । \newline
56. अव॑ रुन्धे रु॒न्धे ऽवाव॑ रुन्धे ग्रा॒मी ग्रा॒मी रु॒न्धे ऽवाव॑ रुन्धे ग्रा॒मी । \newline
57. रु॒न्धे॒ ग्रा॒मी ग्रा॒मी रु॑न्धे रुन्धे ग्रा॒म्ये॑वैव ग्रा॒मी रु॑न्धे रुन्धे ग्रा॒म्ये॑व । \newline
58. ग्रा॒म्ये॑वैव ग्रा॒मी ग्रा॒म्ये॑व भ॑वति भवत्ये॒व ग्रा॒मी ग्रा॒म्ये॑व भ॑वति । \newline
\pagebreak
\markright{ TS 3.4.8.2  \hfill https://www.vedavms.in \hfill}

\section{ TS 3.4.8.2 }

\textbf{TS 3.4.8.2 } \newline
\textbf{Samhita Paata} \newline

-व भ॑वत्यधि॒देव॑ने जुहोत्यधि॒देव॑न ए॒वास्मै॑ सजा॒तानव॑ रुन्धे॒ त ए॑न॒मव॑रुद्धा॒ उप॑ तिष्ठन्ते रथमु॒ख ओज॑स्कामस्य होत॒व्या॑ ओजो॒ वै रा᳚ष्ट्र॒भृत॒ ओजो॒ रथ॒ ओज॑सै॒वास्मा॒ ओजोऽव॑ रुन्ध ओज॒स्व्ये॑व भ॑वति॒ यो रा॒ष्ट्रादप॑भूतः॒ स्यात् तस्मै॑ होत॒व्या॑ याव॑न्तोऽस्य॒ रथाः॒ स्युस्तान् ब्रू॑याद् यु॒ङ्ध्वमिति॑ रा॒ष्ट्रमे॒वास्मै॑ युन॒क्त्या - [  ] \newline

\textbf{Pada Paata} \newline

ए॒व । भ॒व॒ति॒ । अ॒धि॒देव॑न॒ इत्य॑धि - देव॑ने । जु॒हो॒ति॒ । अ॒धि॒देव॑न॒ इत्य॑धि - देव॑ने । ए॒व । अ॒स्मै॒ । स॒जा॒तानिति॑ स-जा॒तान् । अवेति॑ । रु॒न्धे॒ । ते । ए॒न॒म् । अव॑रुद्धा॒ इत्यव॑ - रु॒द्धाः॒ । उपेति॑ । ति॒ष्ठ॒न्ते॒ । र॒थ॒मु॒ख इति॑ रथ - मु॒खे । ओज॑स्काम॒स्येत्योजः॑ - का॒म॒स्य॒ । हो॒त॒व्याः᳚ । ओजः॑ । वै । रा॒ष्ट्र॒भृत॒ इति॑ राष्ट्र-भृतः॑ । ओजः॑ । रथः॑ । ओज॑सा । ए॒व । अ॒स्मै॒ । ओजः॑ । अवेति॑ । रु॒न्धे॒ । ओ॒ज॒स्वी । ए॒व । भ॒व॒ति॒ । यः । रा॒ष्ट्रात् । अप॑भूत॒ इत्यप॑ - भू॒तः॒ । स्यात् । तस्मै᳚ । हो॒त॒व्याः᳚ । याव॑न्तः । अ॒स्य॒ । रथाः᳚ । स्युः । तान् । ब्रू॒या॒त् । यु॒ङ्ध्वम् । इति॑ । रा॒ष्ट्रम् । ए॒व । अ॒स्मै॒ । यु॒न॒क्ति॒ ।  \newline


\textbf{Krama Paata} \newline

ए॒व भ॑वति । भ॒व॒त्य॒धि॒देव॑ने । अ॒धि॒देव॑ने जुहोति । अ॒धि॒देव॑न॒ इत्य॑धि - देव॑ने । जु॒हो॒त्य॒धि॒देव॑ने । अ॒धि॒देव॑न ए॒व । अ॒धि॒देव॑न॒ इत्य॑धि - देव॑ने । ए॒वास्मै᳚ । अ॒स्मै॒ स॒जा॒तान् । स॒जा॒तानव॑ । स॒जा॒तानिति॑ स - जा॒तान् । अव॑ रुन्धे । रु॒न्धे॒ ते । त ए॑नम् । ए॒न॒मव॑रुद्धाः । अव॑रुद्धा॒ उप॑ । अव॑रुद्धा॒ इत्यव॑ - रु॒द्धाः॒ । उप॑ तिष्ठन्ते । ति॒ष्ठ॒न्ते॒ र॒थ॒मु॒खे । र॒थ॒मु॒ख ओज॑स्कामस्य । र॒थ॒मु॒ख इति॑रथ - मु॒खे । ओज॑स्कामस्य होत॒व्याः᳚ । ओज॑स्काम॒स्येत्योजः॑ - का॒म॒स्य॒ । हो॒त॒व्या॑ ओजः॑ । ओजो॒ वै । वै रा᳚ष्ट्र॒भृतः॑ । रा॒ष्ट्र॒भृत॒ ओजः॑ । रा॒ष्ट्र॒भृत॒ इति॑ राष्ट्र - भृतः॑ । ओजो॒ रथः॑ । रथ॒ ओज॑सा । ओज॑सै॒व । ए॒वास्मै᳚ । अ॒स्मा॒ ओजः॑ । ओजो ऽव॑ । अव॑ रुन्धे । रु॒न्ध॒ ओ॒ज॒स्वी । ओ॒ज॒स्व्ये॑व । ए॒व भ॑वति । भ॒व॒ति॒ यः । यो रा॒ष्ट्रात् । रा॒ष्ट्रादप॑भूतः । अप॑भूतः॒ स्यात् । अप॑भूत॒ इत्यप॑ - भू॒तः॒ । स्यात् तस्मै᳚ । तस्मै॑ होत॒व्याः᳚ । हो॒त॒व्या॑ याव॑न्तः । याव॑न्तो ऽस्य । अ॒स्य॒ रथाः᳚ । रथाः॒ स्युः । स्युस्तान् । तान् ब्रू॑यात् । ब्रू॒या॒द् यु॒ङ्ध्वम् । यु॒ङ्ध्वमिति॑ । इति॑ रा॒ष्ट्रम् । रा॒ष्ट्रमे॒व । ए॒वास्मै᳚ । अ॒स्मै॒ यु॒न॒क्ति॒ । यु॒न॒क्त्याहु॑तयः \newline

\textbf{Jatai Paata} \newline

1. ए॒व भ॑वति भव त्ये॒वैव भ॑वति । \newline
2. भ॒व॒ त्य॒धि॒देव॑ने ऽधि॒देव॑ने भवति भव त्यधि॒देव॑ने । \newline
3. अ॒धि॒देव॑ने जुहोति जुहो त्यधि॒देव॑ने ऽधि॒देव॑ने जुहोति । \newline
4. अ॒धि॒देव॑न॒ इत्य॑धि - देव॑ने । \newline
5. जु॒हो॒ त्य॒धि॒देव॑ने ऽधि॒देव॑ने जुहोति जुहो त्यधि॒देव॑ने । \newline
6. अ॒धि॒देव॑न ए॒वैवा धि॒देव॑ने ऽधि॒देव॑न ए॒व । \newline
7. अ॒धि॒देव॑न॒ इत्य॑धि - देव॑ने । \newline
8. ए॒वास्मा॑ अस्मा ए॒वैवास्मै᳚ । \newline
9. अ॒स्मै॒ स॒जा॒तान् थ्स॑जा॒ता-न॑स्मा अस्मै सजा॒तान् । \newline
10. स॒जा॒ता-नवाव॑ सजा॒तान् थ्स॑जा॒ता-नव॑ । \newline
11. स॒जा॒तानिति॑ स - जा॒तान् । \newline
12. अव॑ रुन्धे रु॒न्धे ऽवाव॑ रुन्धे । \newline
13. रु॒न्धे॒ ते ते रु॑न्धे रुन्धे॒ ते । \newline
14. त ए॑न मेन॒म् ते त ए॑नम् । \newline
15. ए॒न॒ मव॑रुद्धा॒ अव॑रुद्धा एन मेन॒ मव॑रुद्धाः । \newline
16. अव॑रुद्धा॒ उपोपा व॑रुद्धा॒ अव॑रुद्धा॒ उप॑ । \newline
17. अव॑रुद्धा॒ इत्यव॑ - रु॒द्धाः॒ । \newline
18. उप॑ तिष्ठन्ते तिष्ठन्त॒ उपोप॑ तिष्ठन्ते । \newline
19. ति॒ष्ठ॒न्ते॒ र॒थ॒मु॒खे र॑थमु॒खे ति॑ष्ठन्ते तिष्ठन्ते रथमु॒खे । \newline
20. र॒थ॒मु॒ख ओज॑स्काम॒स्यौ ज॑स्कामस्य रथमु॒खे र॑थमु॒ख ओज॑स्कामस्य । \newline
21. र॒थ॒मु॒ख इति॑ रथ - मु॒खे । \newline
22. ओज॑स्कामस्य होत॒व्या॑ होत॒व्या॑ ओज॑स्काम॒ स्यौज॑स्कामस्य होत॒व्याः᳚ । \newline
23. ओज॑स्काम॒स्येत्योजः॑ - का॒म॒स्य॒ । \newline
24. हो॒त॒व्या॑ ओज॒ ओजो॑ होत॒व्या॑ होत॒व्या॑ ओजः॑ । \newline
25. ओजो॒ वै वा ओज॒ ओजो॒ वै । \newline
26. वै रा᳚ष्ट्र॒भृतो॑ राष्ट्र॒भृतो॒ वै वै रा᳚ष्ट्र॒भृतः॑ । \newline
27. रा॒ष्ट्र॒भृत॒ ओज॒ ओजो॑ राष्ट्र॒भृतो॑ राष्ट्र॒भृत॒ ओजः॑ । \newline
28. रा॒ष्ट्र॒भृत॒ इति॑ राष्ट्र - भृतः॑ । \newline
29. ओजो॒ रथो॒ रथ॒ ओज॒ ओजो॒ रथः॑ । \newline
30. रथ॒ ओज॒ सौज॑सा॒ रथो॒ रथ॒ ओज॑सा । \newline
31. ओज॑सै॒वै वौज॒सौ ज॑सै॒व । \newline
32. ए॒वास्मा॑ अस्मा ए॒वै वास्मै᳚ । \newline
33. अ॒स्मा॒ ओज॒ ओजो᳚ ऽस्मा अस्मा॒ ओजः॑ । \newline
34. ओजो ऽवा वौज॒ ओजो ऽव॑ । \newline
35. अव॑ रुन्धे रु॒न्धे ऽवाव॑ रुन्धे । \newline
36. रु॒न्ध॒ ओ॒ज॒स्व्यो॑ज॒स्वी रु॑न्धे रुन्ध ओज॒स्वी । \newline
37. ओ॒ज॒ स्व्ये॑वैवौज॒ स्व्यो॑ज॒ स्व्ये॑व । \newline
38. ए॒व भ॑वति भव त्ये॒वैव भ॑वति । \newline
39. भ॒व॒ति॒ यो यो भ॑वति भवति॒ यः । \newline
40. यो रा॒ष्ट्राद् रा॒ष्ट्राद् यो यो रा॒ष्ट्रात् । \newline
41. रा॒ष्ट्रा दप॑भू॒तो ऽप॑भूतो रा॒ष्ट्राद् रा॒ष्ट्रा दप॑भूतः । \newline
42. अप॑भूतः॒ स्याथ् स्यादप॑भू॒तो ऽप॑भूतः॒ स्यात् । \newline
43. अप॑भूत॒ इत्यप॑ - भू॒तः॒ । \newline
44. स्यात् तस्मै॒ तस्मै॒ स्याथ् स्यात् तस्मै᳚ । \newline
45. तस्मै॑ होत॒व्या॑ होत॒व्या᳚ स्तस्मै॒ तस्मै॑ होत॒व्याः᳚ । \newline
46. हो॒त॒व्या॑ याव॑न्तो॒ याव॑न्तो होत॒व्या॑ होत॒व्या॑ याव॑न्तः । \newline
47. याव॑न्तो ऽस्यास्य॒ याव॑न्तो॒ याव॑न्तो ऽस्य । \newline
48. अ॒स्य॒ रथा॒ रथा॑ अस्यास्य॒ रथाः᳚ । \newline
49. रथाः॒ स्युः स्यू रथा॒ रथाः॒ स्युः । \newline
50. स्यु स्ताꣳ स्तान् थ्स्युः स्यु स्तान् । \newline
51. तान् ब्रू॑याद् ब्रूया॒त् ताꣳ स्तान् ब्रू॑यात् । \newline
52. ब्रू॒या॒द् यु॒ङ्ध्वं ॅयु॒ङ्ध्वम् ब्रू॑याद् ब्रूयाद् यु॒ङ्ध्वम् । \newline
53. यु॒ङ्ध्व मितीति॑ यु॒ङ्ध्वं ॅयु॒ङ्ध्व मिति॑ । \newline
54. इति॑ रा॒ष्ट्रꣳ रा॒ष्ट्र मितीति॑ रा॒ष्ट्रम् । \newline
55. रा॒ष्ट्र मे॒वैव रा॒ष्ट्रꣳ रा॒ष्ट्र मे॒व । \newline
56. ए॒वास्मा॑ अस्मा ए॒वै वास्मै᳚ । \newline
57. अ॒स्मै॒ यु॒न॒क्ति॒ यु॒न॒क्त्य॒ स्मा॒ अ॒स्मै॒ यु॒न॒क्ति॒ । \newline
58. यु॒न॒क्त्या हु॑तय॒ आहु॑तयो युनक्ति युन॒क्त्या हु॑तयः । \newline

\textbf{Ghana Paata } \newline

1. ए॒व भ॑वति भव त्ये॒वैव भ॑व त्यधि॒देव॑ने ऽधि॒देव॑ने भव त्ये॒वैव भ॑व त्यधि॒देव॑ने । \newline
2. भ॒व॒ त्य॒धि॒देव॑ने ऽधि॒देव॑ने भवति भव त्यधि॒देव॑ने जुहोति जुहो त्यधि॒देव॑ने भवति भव त्यधि॒देव॑ने जुहोति । \newline
3. अ॒धि॒देव॑ने जुहोति जुहो त्यधि॒देव॑ने ऽधि॒देव॑ने जुहो त्यधि॒देव॑ने ऽधि॒देव॑ने जुहो त्यधि॒देव॑ने ऽधि॒देव॑ने जुहो त्यधि॒देव॑ने । \newline
4. अ॒धि॒देव॑न॒ इत्य॑धि - देव॑ने । \newline
5. जु॒हो॒ त्य॒धि॒देव॑ने ऽधि॒देव॑ने जुहोति जुहो त्यधि॒देव॑न ए॒वैवा धि॒देव॑ने जुहोति जुहो त्यधि॒देव॑न ए॒व । \newline
6. अ॒धि॒देव॑न ए॒वैवा धि॒देव॑ने ऽधि॒देव॑न ए॒वास्मा॑ अस्मा ए॒वा धि॒देव॑ने ऽधि॒देव॑न ए॒वास्मै᳚ । \newline
7. अ॒धि॒देव॑न॒ इत्य॑धि - देव॑ने । \newline
8. ए॒वास्मा॑ अस्मा ए॒वै वास्मै॑ सजा॒तान् थ्स॑जा॒ता,न॑स्मा ए॒वैवास्मै॑ सजा॒तान् । \newline
9. अ॒स्मै॒ स॒जा॒तान् थ्स॑जा॒ता,न॑स्मा अस्मै सजा॒ता,नवाव॑ सजा॒ता,न॑स्मा अस्मै सजा॒ता नव॑ । \newline
10. स॒जा॒ता,नवाव॑ सजा॒तान् थ्स॑जा॒ता,नव॑ रुन्धे रु॒न्धे ऽव॑ सजा॒तान् थ्स॑जा॒ता,नव॑ रुन्धे । \newline
11. स॒जा॒तानिति॑ स - जा॒तान् । \newline
12. अव॑ रुन्धे रु॒न्धे ऽवाव॑ रुन्धे॒ ते ते रु॒न्धे ऽवाव॑ रुन्धे॒ ते । \newline
13. रु॒न्धे॒ ते ते रु॑न्धे रुन्धे॒ त ए॑न मेन॒म् ते रु॑न्धे रुन्धे॒ त ए॑नम् । \newline
14. त ए॑न मेन॒म् ते त ए॑न॒ मव॑रुद्धा॒ अव॑रुद्धा एन॒म् ते त ए॑न॒ मव॑रुद्धाः । \newline
15. ए॒न॒ मव॑रुद्धा॒ अव॑रुद्धा एन मेन॒ मव॑रुद्धा॒ उपोपा व॑रुद्धा एन मेन॒ मव॑रुद्धा॒ उप॑ । \newline
16. अव॑रुद्धा॒ उपोपा व॑रुद्धा॒ अव॑रुद्धा॒ उप॑ तिष्ठन्ते तिष्ठन्त॒ उपा व॑रुद्धा॒ अव॑रुद्धा॒ उप॑ तिष्ठन्ते । \newline
17. अव॑रुद्धा॒ इत्यव॑ - रु॒द्धाः॒ । \newline
18. उप॑ तिष्ठन्ते तिष्ठन्त॒ उपोप॑ तिष्ठन्ते रथमु॒खे र॑थमु॒खे ति॑ष्ठन्त॒ उपोप॑ तिष्ठन्ते रथमु॒खे । \newline
19. ति॒ष्ठ॒न्ते॒ र॒थ॒मु॒खे र॑थमु॒खे ति॑ष्ठन्ते तिष्ठन्ते रथमु॒ख ओज॑स्काम॒ स्यौज॑स्कामस्य रथमु॒खे ति॑ष्ठन्ते तिष्ठन्ते रथमु॒ख ओज॑स्कामस्य । \newline
20. र॒थ॒मु॒ख ओज॑स्काम॒ स्यौज॑स्कामस्य रथमु॒खे र॑थमु॒ख ओज॑स्कामस्य होत॒व्या॑ होत॒व्या॑ ओज॑स्कामस्य रथमु॒खे र॑थमु॒ख ओज॑स्कामस्य होत॒व्याः᳚ । \newline
21. र॒थ॒मु॒ख इति॑ रथ - मु॒खे । \newline
22. ओज॑स्कामस्य होत॒व्या॑ होत॒व्या॑ ओज॑स्काम॒ स्यौज॑स्कामस्य होत॒व्या॑ ओज॒ ओजो॑ होत॒व्या॑ ओज॑स्काम॒
स्यौज॑स्कामस्य होत॒व्या॑ ओजः॑ । \newline
23. ओज॑स्काम॒स्येत्योजः॑ - का॒म॒स्य॒ । \newline
24. हो॒त॒व्या॑ ओज॒ ओजो॑ होत॒व्या॑ होत॒व्या॑ ओजो॒ वै वा ओजो॑ होत॒व्या॑ होत॒व्या॑ ओजो॒ वै । \newline
25. ओजो॒ वै वा ओज॒ ओजो॒ वै रा᳚ष्ट्र॒भृतो॑ राष्ट्र॒भृतो॒ वा ओज॒ ओजो॒ वै रा᳚ष्ट्र॒भृतः॑ । \newline
26. वै रा᳚ष्ट्र॒भृतो॑ राष्ट्र॒भृतो॒ वै वै रा᳚ष्ट्र॒भृत॒ ओज॒ ओजो॑ राष्ट्र॒भृतो॒ वै वै रा᳚ष्ट्र॒भृत॒ ओजः॑ । \newline
27. रा॒ष्ट्र॒भृत॒ ओज॒ ओजो॑ राष्ट्र॒भृतो॑ राष्ट्र॒भृत॒ ओजो॒ रथो॒ रथ॒ ओजो॑ राष्ट्र॒भृतो॑ राष्ट्र॒भृत॒ ओजो॒ रथः॑ । \newline
28. रा॒ष्ट्र॒भृत॒ इति॑ राष्ट्र - भृतः॑ । \newline
29. ओजो॒ रथो॒ रथ॒ ओज॒ ओजो॒ रथ॒ ओज॒सौज॑सा॒ रथ॒ ओज॒ ओजो॒ रथ॒ ओज॑सा । \newline
30. रथ॒ ओज॒ सौज॑सा॒ रथो॒ रथ॒ ओज॑सै॒ वैवौज॑सा॒ रथो॒ रथ॒ ओज॑सै॒व । \newline
31. ओज॑ सै॒वै वौज॒ सौज॑ सै॒वास्मा॑ अस्मा ए॒वौज॒ सौज॑ सै॒वास्मै᳚ । \newline
32. ए॒वास्मा॑ अस्मा ए॒वै वास्मा॒ ओज॒ ओजो᳚ ऽस्मा ए॒वै वास्मा॒ ओजः॑ । \newline
33. अ॒स्मा॒ ओज॒ ओजो᳚ ऽस्मा अस्मा॒ ओजो ऽवा वौजो᳚ ऽस्मा अस्मा॒ ओजो ऽव॑ । \newline
34. ओजो ऽवा वौज॒ ओजो ऽव॑ रुन्धे रु॒न्धे ऽवौज॒ ओजो ऽव॑ रुन्धे । \newline
35. अव॑ रुन्धे रु॒न्धे ऽवाव॑ रुन्ध ओज॒ स्व्यो॑ज॒स्वी रु॒न्धे ऽवाव॑ रुन्ध ओज॒स्वी । \newline
36. रु॒न्ध॒ ओ॒ज॒ स्व्यो॑ज॒स्वी रु॑न्धे रुन्ध ओज॒ स्व्ये॑ वैवौ ज॒स्वी रु॑न्धे रुन्ध ओज॒स्व्ये॑व । \newline
37. ओ॒ज॒ स्व्ये॑ वैवौज॒ स्व्यो॑ज॒ स्व्ये॑व भ॑वति भव त्ये॒वौज॒ स्व्यो॑ज॒ स्व्ये॑व भ॑वति । \newline
38. ए॒व भ॑वति भव त्ये॒वैव भ॑वति॒ यो यो भ॑व त्ये॒वैव भ॑वति॒ यः । \newline
39. भ॒व॒ति॒ यो यो भ॑वति भवति॒ यो रा॒ष्ट्राद् रा॒ष्ट्राद् यो भ॑वति भवति॒ यो रा॒ष्ट्रात् । \newline
40. यो रा॒ष्ट्राद् रा॒ष्ट्राद् यो यो रा॒ष्ट्रा दप॑भू॒तो ऽप॑भूतो रा॒ष्ट्राद् यो यो रा॒ष्ट्रा दप॑भूतः । \newline
41. रा॒ष्ट्रा दप॑भू॒तो ऽप॑भूतो रा॒ष्ट्राद् रा॒ष्ट्रा दप॑भूतः॒ स्याथ् स्या दप॑भूतो रा॒ष्ट्राद् रा॒ष्ट्रा दप॑भूतः॒ स्यात् । \newline
42. अप॑भूतः॒ स्याथ् स्या दप॑भू॒तो ऽप॑भूतः॒ स्यात् तस्मै॒ तस्मै॒ स्या दप॑भू॒तो ऽप॑भूतः॒ स्यात् तस्मै᳚ । \newline
43. अप॑भूत॒ इत्यप॑ - भू॒तः॒ । \newline
44. स्यात् तस्मै॒ तस्मै॒ स्याथ् स्यात् तस्मै॑ होत॒व्या॑ होत॒व्या᳚ स्तस्मै॒ स्याथ् स्यात् तस्मै॑ होत॒व्याः᳚ । \newline
45. तस्मै॑ होत॒व्या॑ होत॒व्या᳚ स्तस्मै॒ तस्मै॑ होत॒व्या॑ याव॑न्तो॒ याव॑न्तो होत॒व्या᳚ स्तस्मै॒ तस्मै॑ होत॒व्या॑ याव॑न्तः । \newline
46. हो॒त॒व्या॑ याव॑न्तो॒ याव॑न्तो होत॒व्या॑ होत॒व्या॑ याव॑न्तो ऽस्यास्य॒ याव॑न्तो होत॒व्या॑ होत॒व्या॑ याव॑न्तो ऽस्य । \newline
47. याव॑न्तो ऽस्यास्य॒ याव॑न्तो॒ याव॑न्तो ऽस्य॒ रथा॒ रथा॑ अस्य॒ याव॑न्तो॒ याव॑न्तो ऽस्य॒ रथाः᳚ । \newline
48. अ॒स्य॒ रथा॒ रथा॑ अस्यास्य॒ रथाः॒ स्युः स्यू रथा॑ अस्यास्य॒ रथाः॒ स्युः । \newline
49. रथाः॒ स्युः स्यू रथा॒ रथाः॒ स्यु स्ताꣳ स्तान् थ्स्यू रथा॒ रथाः॒ स्यु स्तान् । \newline
50. स्यु स्ताꣳ स्तान् थ्स्युः स्यु स्तान् ब्रू॑याद् ब्रूया॒त् तान् थ्स्युः स्यु स्तान् ब्रू॑यात् । \newline
51. तान् ब्रू॑याद् ब्रूया॒त् ताꣳ स्तान् ब्रू॑याद् यु॒ङ्ध्वम् ॅयु॒ङ्ध्वम् ब्रू॑या॒त् ताꣳ स्तान् ब्रू॑याद् यु॒ङ्ध्वम् । \newline
52. ब्रू॒या॒द् यु॒ङ्ध्वम् ॅयु॒ङ्ध्वम् ब्रू॑याद् ब्रूयाद् यु॒ङ्ध्व मितीति॑ यु॒ङ्ध्वम् ब्रू॑याद् ब्रूयाद् यु॒ङ्ध्व मिति॑ । \newline
53. यु॒ङ्ध्व मितीति॑ यु॒ङ्ध्वम् ॅयु॒ङ्ध्व मिति॑ रा॒ष्ट्रꣳ रा॒ष्ट्र मिति॑ यु॒ङ्ध्वम् ॅयु॒ङ्ध्व मिति॑ रा॒ष्ट्रम् । \newline
54. इति॑ रा॒ष्ट्रꣳ रा॒ष्ट्र मितीति॑ रा॒ष्ट्र मे॒वैव रा॒ष्ट्र मितीति॑ रा॒ष्ट्र मे॒व । \newline
55. रा॒ष्ट्र मे॒वैव रा॒ष्ट्रꣳ रा॒ष्ट्र मे॒वास्मा॑ अस्मा ए॒व रा॒ष्ट्रꣳ रा॒ष्ट्र मे॒वास्मै᳚ । \newline
56. ए॒वास्मा॑ अस्मा ए॒वैवास्मै॑ युनक्ति युनक्त्यस्मा ए॒वैवास्मै॑ युनक्ति । \newline
57. अ॒स्मै॒ यु॒न॒क्ति॒ यु॒न॒क्त्य॒स्मा॒ अ॒स्मै॒ यु॒न॒ क्त्याहु॑तय॒ आहु॑तयो युनक्त्यस्मा अस्मै युन॒ क्त्याहु॑तयः । \newline
58. यु॒न॒ क्त्याहु॑तय॒ आहु॑तयो युनक्ति युन॒ क्त्याहु॑तयो॒ वै वा आहु॑तयो युनक्ति युन॒ क्त्याहु॑तयो॒ वै । \newline
\pagebreak
\markright{ TS 3.4.8.3  \hfill https://www.vedavms.in \hfill}

\section{ TS 3.4.8.3 }

\textbf{TS 3.4.8.3 } \newline
\textbf{Samhita Paata} \newline

हु॑तयो॒ वा ए॒तस्याक्लृ॑प्ता॒ यस्य॑ रा॒ष्ट्रं न कल्प॑ते स्वर॒थस्य॒ दक्षि॑णं च॒क्रं प्र॒वृह्य॑ ना॒डीम॒भि जु॑हुया॒दाहु॑तीरे॒वास्य॑ कल्पयति॒ ता अ॑स्य॒ कल्प॑माना रा॒ष्ट्रमनु॑ कल्पते संग्रा॒मे संॅय॑त्ते होत॒व्या॑ रा॒ष्ट्रं ॅवै रा᳚ष्ट्र॒भृतो॑ रा॒ष्ट्रे खलु॒ वा ए॒ते व्याय॑च्छन्ते॒ ये स॑ङ्ग्रा॒मꣳ सं॒ ॅयन्ति॒ यस्य॒ पूर्व॑स्य॒ जुह्व॑ति॒ स ए॒व भ॑वति॒ जय॑ति॒ तꣳ स॑ग्रां॒मं मा᳚न्धु॒क इ॒द्ध्मो- [  ] \newline

\textbf{Pada Paata} \newline

आहु॑तय॒ इत्या - हु॒त॒यः॒ । वै । ए॒तस्य॑ । अक्लृ॑प्ताः । यस्य॑ । रा॒ष्ट्रम् । न । कल्प॑ते । स्व॒र॒थस्येति॑ स्व - र॒थस्य॑ । दक्षि॑णम् । च॒क्रम् । प्र॒वृह्येति॑ प्र - वृह्य॑ । ना॒डीम् । अ॒भीति॑ । जु॒हु॒या॒त् । आहु॑ती॒रित्या - हु॒तीः॒ । ए॒व । अ॒स्य॒ । क॒ल्प॒य॒ति॒ । ताः । अ॒स्य॒ । कल्प॑मानाः । रा॒ष्ट्रम् । अन्विति॑ । क॒ल्प॒ते॒ । स॒ग्रां॒म इति॑ सं - ग्रा॒मे । संॅय॑त्त॒ इति॒ सं - य॒त्ते॒ । हो॒त॒व्याः᳚ । रा॒ष्ट्रम् । वै । रा॒ष्ट्र॒भृत॒ इति॑ राष्ट्र - भृतः॑ । रा॒ष्ट्रे । खलु॑ । वै । ए॒ते । व्याय॑च्छन्त॒ इति॑ वि - आय॑च्छन्ते । ये । स॒ग्रां॒ममिति॑ सं - ग्रा॒मम् । सं॒ॅयन्तीति॑ सं-यन्ति॑ । यस्य॑ । पूर्व॑स्य । जुह्व॑ति । सः । ए॒व । भ॒व॒ति॒ । जय॑ति । तम् । स॒ग्रां॒ममिति॑ सं - ग्रा॒मम् । मा॒न्धु॒कः । इ॒द्ध्मः ।  \newline


\textbf{Krama Paata} \newline

आहु॑तयो॒ वै । आहु॑तय॒ इत्या - हु॒त॒यः॒ । वा ए॒तस्य॑ । ए॒तस्याक्लृ॑प्ताः । अक्लृ॑प्ता॒ यस्य॑ । यस्य॑ रा॒ष्ट्रम् । रा॒ष्ट्रम् न । न कल्प॑ते । कल्प॑ते स्वर॒थस्य॑ । स्व॒र॒थस्य॒ दक्षि॑णम् । स्व॒र॒थस्येति॑ स्व - र॒थस्य॑ । दक्षि॑णम् च॒क्रम् । च॒क्रम् प्र॒वृह्य॑ । प्र॒वृह्य॑ ना॒डीम् । प्र॒वृह्येति॑ प्र - वृह्य॑ । ना॒डीम॒भि । अ॒भि जु॑हुयात् । जु॒हु॒या॒दाहु॑तीः । आहु॑तीरे॒व । आहु॑ती॒रित्या - हु॒तीः॒ । ए॒वास्य॑ । अ॒स्य॒ क॒ल्प॒य॒ति॒ । क॒ल्प॒य॒ति॒ ताः । ता अ॑स्य । अ॒स्य॒ कल्प॑मानाः । कल्प॑माना रा॒ष्ट्रम् । रा॒ष्ट्रमनु॑ । अनु॑ कल्पते । क॒ल्प॒ते॒ स॒ङ्ग्रा॒मे । स॒ङ्ग्रा॒मे सम्ॅय॑त्ते । स॒ङ्ग्रा॒म इति॑ सं - ग्रा॒मे । सम्ॅय॑त्ते होत॒व्याः᳚ । सम्ॅय॑त्त॒ इति॒ सं - य॒त्ते॒ । हो॒त॒व्या॑ रा॒ष्ट्रम् । रा॒ष्ट्रं ॅवै । वै रा᳚ष्ट्र॒भृतः॑ । रा॒ष्ट्र॒भृतो॑ रा॒ष्ट्रे । रा॒ष्ट्र॒भृत॒ इति॑ राष्ट्र - भृतः॑ । रा॒ष्टे खलु॑ । खलु॒ वै । वा ए॒ते । ए॒ते व्याय॑च्छन्ते । व्याय॑च्छन्ते॒ ये । व्याय॑च्छन्त॒ इति॑ वि - आय॑च्छन्ते । ये स॑ङ्ग्रा॒मम् । स॒ङ्ग्रा॒मꣳ स॒म्ॅयन्ति॑ । स॒ङ्ग्रा॒ममिति॑ सम् - ग्रा॒मम् । स॒म्ॅयन्ति॒ यस्य॑ । स॒म्ॅयन्तीति॑ सं - यन्ति॑ । यस्य॒ पूर्व॑स्य । पूर्व॑स्य॒ जुह्व॑ति । जुह्व॑ति॒ सः । स ए॒व । ए॒व भ॑वति । भ॒व॒ति॒ जय॑ति । जय॑ति॒ तम् । तꣳ स॑ङ्ग्रा॒मम् । स॒ङ्ग्रा॒मम् मा᳚न्धु॒कः । स॒ङ्ग्रा॒ममिति॑ सम् - ग्रा॒मम् । मा॒न्धु॒क इ॒ध्मः । इ॒ध्मो भ॑वति \newline

\textbf{Jatai Paata} \newline

1. आहु॑तयो॒ वै वा आहु॑तय॒ आहु॑तयो॒ वै । \newline
2. आहु॑तय॒ इत्या - हु॒त॒यः॒ । \newline
3. वा ए॒त स्यै॒तस्य॒ वै वा ए॒तस्य॑ । \newline
4. ए॒तस्या क्लृ॑प्ता॒ अक्लृ॑प्ता ए॒त स्यै॒तस्या क्लृ॑प्ताः । \newline
5. अक्लृ॑प्ता॒ यस्य॒ यस्या क्लृ॑प्ता॒ अक्लृ॑प्ता॒ यस्य॑ । \newline
6. यस्य॑ रा॒ष्ट्रꣳ रा॒ष्ट्रं ॅयस्य॒ यस्य॑ रा॒ष्ट्रम् । \newline
7. रा॒ष्ट्रम् न न रा॒ष्ट्रꣳ रा॒ष्ट्रम् न । \newline
8. न कल्प॑ते॒ कल्प॑ते॒ न न कल्प॑ते । \newline
9. कल्प॑ते स्वर॒थस्य॑ स्वर॒थस्य॒ कल्प॑ते॒ कल्प॑ते स्वर॒थस्य॑ । \newline
10. स्व॒र॒थस्य॒ दक्षि॑ण॒म् दक्षि॑णꣳ स्वर॒थस्य॑ स्वर॒थस्य॒ दक्षि॑णम् । \newline
11. स्व॒र॒थस्येति॑ स्व - र॒थस्य॑ । \newline
12. दक्षि॑णम् च॒क्रम् च॒क्रम् दक्षि॑ण॒म् दक्षि॑णम् च॒क्रम् । \newline
13. च॒क्रम् प्र॒वृह्य॑ प्र॒वृह्य॑ च॒क्रम् च॒क्रम् प्र॒वृह्य॑ । \newline
14. प्र॒वृह्य॑ ना॒डीम् ना॒डीम् प्र॒वृह्य॑ प्र॒वृह्य॑ ना॒डीम् । \newline
15. प्र॒वृह्येति॑ प्र - वृह्य॑ । \newline
16. ना॒डी म॒भ्य॑भि ना॒डीम् ना॒डी म॒भि । \newline
17. अ॒भि जु॑हुयाज् जुहुया द॒भ्य॑भि जु॑हुयात् । \newline
18. जु॒हु॒या॒ दाहु॑ती॒ राहु॑तीर् जुहुयाज् जुहुया॒ दाहु॑तीः । \newline
19. आहु॑ती रे॒वै वाहु॑ती॒ राहु॑ती रे॒व । \newline
20. आहु॑ती॒रित्या - हु॒तीः॒ । \newline
21. ए॒वास्या᳚ स्यै॒वैवास्य॑ । \newline
22. अ॒स्य॒ क॒ल्प॒य॒ति॒ क॒ल्प॒य॒ त्य॒स्या॒स्य॒ क॒ल्प॒य॒ति॒ । \newline
23. क॒ल्प॒य॒ति॒ ता स्ताः क॑ल्पयति कल्पयति॒ ताः । \newline
24. ता अ॑स्या स्य॒ ता स्ता अ॑स्य । \newline
25. अ॒स्य॒ कल्प॑मानाः॒ कल्प॑माना अस्यास्य॒ कल्प॑मानाः । \newline
26. कल्प॑माना रा॒ष्ट्रꣳ रा॒ष्ट्रम् कल्प॑मानाः॒ कल्प॑माना रा॒ष्ट्रम् । \newline
27. रा॒ष्ट्र मन्वनु॑ रा॒ष्ट्रꣳ रा॒ष्ट्र मनु॑ । \newline
28. अनु॑ कल्पते कल्पते॒ ऽन्वनु॑ कल्पते । \newline
29. क॒ल्प॒ते॒ स॒ङ्ग्रा॒मे स॑ङ्ग्रा॒मे क॑ल्पते कल्पते सङ्ग्रा॒मे । \newline
30. स॒ङ्ग्रा॒मे संॅय॑त्ते॒ संॅय॑त्ते सङ्ग्रा॒मे स॑ङ्ग्रा॒मे संॅय॑त्ते । \newline
31. स॒ङ्ग्रा॒म इति॑ सं - ग्रा॒मे । \newline
32. संॅय॑त्ते होत॒व्या॑ होत॒व्याः᳚ संॅय॑त्ते॒ संॅय॑त्ते होत॒व्याः᳚ । \newline
33. संॅय॑त्त॒ इति॒ सं - य॒त्ते॒ । \newline
34. हो॒त॒व्या॑ रा॒ष्ट्रꣳ रा॒ष्ट्रꣳ हो॑त॒व्या॑ होत॒व्या॑ रा॒ष्ट्रम् । \newline
35. रा॒ष्ट्रं ॅवै वै रा॒ष्ट्रꣳ रा॒ष्ट्रं ॅवै । \newline
36. वै रा᳚ष्ट्र॒भृतो॑ राष्ट्र॒भृतो॒ वै वै रा᳚ष्ट्र॒भृतः॑ । \newline
37. रा॒ष्ट्र॒भृतो॑ रा॒ष्ट्रे रा॒ष्ट्रे रा᳚ष्ट्र॒भृतो॑ राष्ट्र॒भृतो॑ रा॒ष्ट्रे । \newline
38. रा॒ष्ट्र॒भृत॒ इति॑ राष्ट्र - भृतः॑ । \newline
39. रा॒ष्ट्रे खलु॒ खलु॑ रा॒ष्ट्रे रा॒ष्ट्रे खलु॑ । \newline
40. खलु॒ वै वै खलु॒ खलु॒ वै । \newline
41. वा ए॒त ए॒ते वै वा ए॒ते । \newline
42. ए॒ते व्याय॑च्छन्ते॒ व्याय॑च्छन्त ए॒त ए॒ते व्याय॑च्छन्ते । \newline
43. व्याय॑च्छन्ते॒ ये ये व्याय॑च्छन्ते॒ व्याय॑च्छन्ते॒ ये । \newline
44. व्याय॑च्छन्त॒ इति॑ वि - आय॑च्छन्ते । \newline
45. ये स॑ङ्ग्रा॒मꣳ स॑ङ्ग्रा॒मं ॅये ये स॑ङ्ग्रा॒मम् । \newline
46. स॒ङ्ग्रा॒मꣳ सं॒ॅयन्ति॑ सं॒ॅयन्ति॑ सङ्ग्रा॒मꣳ स॑ङ्ग्रा॒मꣳ सं॒ॅयन्ति॑ । \newline
47. स॒ङ्ग्रा॒ममिति॑ सं - ग्रा॒मम् । \newline
48. सं॒ॅयन्ति॒ यस्य॒ यस्य॑ सं॒ॅयन्ति॑ सं॒ॅयन्ति॒ यस्य॑ । \newline
49. सं॒ॅयन्तीति॑ सं - यन्ति॑ । \newline
50. यस्य॒ पूर्व॑स्य॒ पूर्व॑स्य॒ यस्य॒ यस्य॒ पूर्व॑स्य । \newline
51. पूर्व॑स्य॒ जुह्व॑ति॒ जुह्व॑ति॒ पूर्व॑स्य॒ पूर्व॑स्य॒ जुह्व॑ति । \newline
52. जुह्व॑ति॒ स स जुह्व॑ति॒ जुह्व॑ति॒ सः । \newline
53. स ए॒वैव स स ए॒व । \newline
54. ए॒व भ॑वति भव त्ये॒वैव भ॑वति । \newline
55. भ॒व॒ति॒ जय॑ति॒ जय॑ति भवति भवति॒ जय॑ति । \newline
56. जय॑ति॒ तम् तम् जय॑ति॒ जय॑ति॒ तम् । \newline
57. तꣳ स॑ङ्ग्रा॒मꣳ स॑ङ्ग्रा॒मम् तम् तꣳ स॑ङ्ग्रा॒मम् । \newline
58. स॒ङ्ग्रा॒मम् मा᳚न्धु॒को मा᳚न्धु॒कः स॑ङ्ग्रा॒मꣳ स॑ङ्ग्रा॒मम् मा᳚न्धु॒कः । \newline
59. स॒ङ्ग्रा॒ममिति॑ सं - ग्रा॒मम् । \newline
60. मा॒न्धु॒क इ॒द्ध्म इ॒द्ध्मो मा᳚न्धु॒को मा᳚न्धु॒क इ॒द्ध्मः । \newline
61. इ॒द्ध्मो भ॑वति भवती॒ द्ध्म इ॒द्ध्मो भ॑वति । \newline

\textbf{Ghana Paata } \newline

1. आहु॑तयो॒ वै वा आहु॑तय॒ आहु॑तयो॒ वा ए॒त स्यै॒तस्य॒ वा आहु॑तय॒ आहु॑तयो॒ वा ए॒तस्य॑ । \newline
2. आहु॑तय॒ इत्या - हु॒त॒यः॒ । \newline
3. वा ए॒त स्यै॒तस्य॒ वै वा ए॒तस्या क्लृ॑प्ता॒ अक्लृ॑प्ता ए॒तस्य॒ वै वा ए॒तस्या क्लृ॑प्ताः । \newline
4. ए॒तस्या क्लृ॑प्ता॒ अक्लृ॑प्ता ए॒त स्यै॒तस्या क्लृ॑प्ता॒ यस्य॒ यस्या क्लृ॑प्ता ए॒त स्यै॒तस्या क्लृ॑प्ता॒ यस्य॑ । \newline
5. अक्लृ॑प्ता॒ यस्य॒ यस्या क्लृ॑प्ता॒ अक्लृ॑प्ता॒ यस्य॑ रा॒ष्ट्रꣳ रा॒ष्ट्रम् ॅयस्या क्लृ॑प्ता॒ अक्लृ॑प्ता॒ यस्य॑ रा॒ष्ट्रम् । \newline
6. यस्य॑ रा॒ष्ट्रꣳ रा॒ष्ट्रम् ॅयस्य॒ यस्य॑ रा॒ष्ट्रम् न न रा॒ष्ट्रम् ॅयस्य॒ यस्य॑ रा॒ष्ट्रम् न । \newline
7. रा॒ष्ट्रम् न न रा॒ष्ट्रꣳ रा॒ष्ट्रम् न कल्प॑ते॒ कल्प॑ते॒ न रा॒ष्ट्रꣳ रा॒ष्ट्रम् न कल्प॑ते । \newline
8. न कल्प॑ते॒ कल्प॑ते॒ न न कल्प॑ते स्वर॒थस्य॑ स्वर॒थस्य॒ कल्प॑ते॒ न न कल्प॑ते स्वर॒थस्य॑ । \newline
9. कल्प॑ते स्वर॒थस्य॑ स्वर॒थस्य॒ कल्प॑ते॒ कल्प॑ते स्वर॒थस्य॒ दक्षि॑ण॒म् दक्षि॑णꣳ स्वर॒थस्य॒ कल्प॑ते॒ कल्प॑ते स्वर॒थस्य॒ दक्षि॑णम् । \newline
10. स्व॒र॒थस्य॒ दक्षि॑ण॒म् दक्षि॑णꣳ स्वर॒थस्य॑ स्वर॒थस्य॒ दक्षि॑णम् च॒क्रम् च॒क्रम् दक्षि॑णꣳ स्वर॒थस्य॑ स्वर॒थस्य॒ दक्षि॑णम् च॒क्रम् । \newline
11. स्व॒र॒थस्येति॑ स्व - र॒थस्य॑ । \newline
12. दक्षि॑णम् च॒क्रम् च॒क्रम् दक्षि॑ण॒म् दक्षि॑णम् च॒क्रम् प्र॒वृह्य॑ प्र॒वृह्य॑ च॒क्रम् दक्षि॑ण॒म् दक्षि॑णम् च॒क्रम् प्र॒वृह्य॑ । \newline
13. च॒क्रम् प्र॒वृह्य॑ प्र॒वृह्य॑ च॒क्रम् च॒क्रम् प्र॒वृह्य॑ ना॒डीम्,ना॒डीम् प्र॒वृह्य॑ च॒क्रम् च॒क्रम् प्र॒वृह्य॑ ना॒डीम् । \newline
14. प्र॒वृह्य॑ ना॒डीम्,ना॒डीम् प्र॒वृह्य॑ प्र॒वृह्य॑ ना॒डी म॒भ्य॑भि ना॒डीम् प्र॒वृह्य॑ प्र॒वृह्य॑ ना॒डी म॒भि । \newline
15. प्र॒वृह्येति॑ प्र - वृह्य॑ । \newline
16. ना॒डी म॒भ्य॑भि ना॒डीम्,ना॒डी म॒भि जु॑हुयाज् जुहुया द॒भि ना॒डीम्,ना॒डी म॒भि जु॑हुयात् । \newline
17. अ॒भि जु॑हुयाज् जुहुया द॒भ्य॑भि जु॑हुया॒ दाहु॑ती॒ राहु॑तीर् जुहुया द॒भ्य॑भि जु॑हुया॒ दाहु॑तीः । \newline
18. जु॒हु॒या॒ दाहु॑ती॒ राहु॑तीर् जुहुयाज् जुहुया॒ दाहु॑ती रे॒वैवाहु॑तीर् जुहुयाज् जुहुया॒ दाहु॑ती रे॒व । \newline
19. आहु॑ती रे॒वै वाहु॑ती॒ राहु॑ती रे॒वास्या᳚ स्यै॒वा हु॑ती॒रा हु॑ती रे॒वास्य॑ । \newline
20. आहु॑ती॒रित्या - हु॒तीः॒ । \newline
21. ए॒वास्या᳚ स्यै॒वैवास्य॑ कल्पयति कल्पय त्यस्यै॒वै वास्य॑ कल्पयति । \newline
22. अ॒स्य॒ क॒ल्प॒य॒ति॒ क॒ल्प॒य॒ त्य॒स्या॒स्य॒ क॒ल्प॒य॒ति॒ ता स्ताः क॑ल्पय त्यस्यास्य कल्पयति॒ ताः । \newline
23. क॒ल्प॒य॒ति॒ ता स्ताः क॑ल्पयति कल्पयति॒ ता अ॑स्यास्य॒ ताः क॑ल्पयति कल्पयति॒ ता अ॑स्य । \newline
24. ता अ॑स्यास्य॒ ता स्ता अ॑स्य॒ कल्प॑मानाः॒ कल्प॑माना अस्य॒ ता स्ता अ॑स्य॒ कल्प॑मानाः । \newline
25. अ॒स्य॒ कल्प॑मानाः॒ कल्प॑माना अस्यास्य॒ कल्प॑माना रा॒ष्ट्रꣳ रा॒ष्ट्रम् कल्प॑माना अस्यास्य॒ कल्प॑माना रा॒ष्ट्रम् । \newline
26. कल्प॑माना रा॒ष्ट्रꣳ रा॒ष्ट्रम् कल्प॑मानाः॒ कल्प॑माना रा॒ष्ट्र मन्वनु॑ रा॒ष्ट्रम् कल्प॑मानाः॒ कल्प॑माना रा॒ष्ट्र मनु॑ । \newline
27. रा॒ष्ट्र मन्वनु॑ रा॒ष्ट्रꣳ रा॒ष्ट्र मनु॑ कल्पते कल्पते ऽनु रा॒ष्ट्रꣳ रा॒ष्ट्र मनु॑ कल्पते । \newline
28. अनु॑ कल्पते कल्पते॒ ऽन्वनु॑ कल्पते सङ्ग्रा॒मे स॑ङ्ग्रा॒मे क॑ल्पते॒ ऽन्वनु॑ कल्पते सङ्ग्रा॒मे । \newline
29. क॒ल्प॒ते॒ स॒ङ्ग्रा॒मे स॑ङ्ग्रा॒मे क॑ल्पते कल्पते सङ्ग्रा॒मे सम्ॅय॑त्ते॒ सम्ॅय॑त्ते सङ्ग्रा॒मे क॑ल्पते कल्पते सङ्ग्रा॒मे सम्ॅय॑त्ते । \newline
30. स॒ङ्ग्रा॒मे सम्ॅय॑त्ते॒ सम्ॅय॑त्ते सङ्ग्रा॒मे स॑ङ्ग्रा॒मे सम्ॅय॑त्ते होत॒व्या॑ होत॒व्याः᳚ सम्ॅय॑त्ते सङ्ग्रा॒मे स॑ङ्ग्रा॒मे सम्ॅय॑त्ते होत॒व्याः᳚ । \newline
31. स॒ङ्ग्रा॒म इति॑ सम् - ग्रा॒मे । \newline
32. सम्ॅय॑त्ते होत॒व्या॑ होत॒व्याः᳚ सम्ॅय॑त्ते॒ सम्ॅय॑त्ते होत॒व्या॑ रा॒ष्ट्रꣳ रा॒ष्ट्रꣳ हो॑त॒व्याः᳚ सम्ॅय॑त्ते॒ सम्ॅय॑त्ते होत॒व्या॑ रा॒ष्ट्रम् । \newline
33. सम्ॅय॑त्त॒ इति॒ सम् - य॒त्ते॒ । \newline
34. हो॒त॒व्या॑ रा॒ष्ट्रꣳ रा॒ष्ट्रꣳ हो॑त॒व्या॑ होत॒व्या॑ रा॒ष्ट्रम् ॅवै वै रा॒ष्ट्रꣳ हो॑त॒व्या॑ होत॒व्या॑ रा॒ष्ट्रम् ॅवै । \newline
35. रा॒ष्ट्रम् ॅवै वै रा॒ष्ट्रꣳ रा॒ष्ट्रम् ॅवै रा᳚ष्ट्र॒भृतो॑ राष्ट्र॒भृतो॒ वै रा॒ष्ट्रꣳ रा॒ष्ट्रम् ॅवै रा᳚ष्ट्र॒भृतः॑ । \newline
36. वै रा᳚ष्ट्र॒भृतो॑ राष्ट्र॒भृतो॒ वै वै रा᳚ष्ट्र॒भृतो॑ रा॒ष्ट्रे रा॒ष्ट्रे रा᳚ष्ट्र॒भृतो॒ वै वै 
रा᳚ष्ट्र॒भृतो॑ रा॒ष्ट्रे । \newline
37. रा॒ष्ट्र॒भृतो॑ रा॒ष्ट्रे रा॒ष्ट्रे रा᳚ष्ट्र॒भृतो॑ राष्ट्र॒भृतो॑ रा॒ष्ट्रे खलु॒ खलु॑ रा॒ष्ट्रे रा᳚ष्ट्र॒भृतो॑ राष्ट्र॒भृतो॑ रा॒ष्ट्रे खलु॑ । \newline
38. रा॒ष्ट्र॒भृत॒ इति॑ राष्ट्र - भृतः॑ । \newline
39. रा॒ष्ट्रे खलु॒ खलु॑ रा॒ष्ट्रे रा॒ष्ट्रे खलु॒ वै वै खलु॑ रा॒ष्ट्रे रा॒ष्ट्रे खलु॒ वै । \newline
40. खलु॒ वै वै खलु॒ खलु॒ वा ए॒त ए॒ते वै खलु॒ खलु॒ वा ए॒ते । \newline
41. वा ए॒त ए॒ते वै वा ए॒ते व्याय॑च्छन्ते॒ व्याय॑च्छन्त ए॒ते वै वा ए॒ते व्याय॑च्छन्ते । \newline
42. ए॒ते व्याय॑च्छन्ते॒ व्याय॑च्छन्त ए॒त ए॒ते व्याय॑च्छन्ते॒ ये ये व्याय॑च्छन्त ए॒त ए॒ते व्याय॑च्छन्ते॒ ये । \newline
43. व्याय॑च्छन्ते॒ ये ये व्याय॑च्छन्ते॒ व्याय॑च्छन्ते॒ ये स॑ङ्ग्रा॒मꣳ स॑ङ्ग्रा॒मम् ॅये व्याय॑च्छन्ते॒ व्याय॑च्छन्ते॒ ये स॑ङ्ग्रा॒मम् । \newline
44. व्याय॑च्छन्त॒ इति॑ वि - आय॑च्छन्ते । \newline
45. ये स॑ङ्ग्रा॒मꣳ स॑ङ्ग्रा॒मम् ॅये ये स॑ङ्ग्रा॒मꣳ स॒म्ॅयन्ति॑ स॒म्ॅयन्ति॑ सङ्ग्रा॒मम् ॅये ये स॑ङ्ग्रा॒मꣳ स॒म्ॅयन्ति॑ । \newline
46. स॒ङ्ग्रा॒मꣳ स॒म्ॅयन्ति॑ स॒म्ॅयन्ति॑ सङ्ग्रा॒मꣳ स॑ङ्ग्रा॒मꣳ स॒म्ॅयन्ति॒ यस्य॒ यस्य॑ स॒म्ॅयन्ति॑ सङ्ग्रा॒मꣳ स॑ङ्ग्रा॒मꣳ स॒म्ॅयन्ति॒ यस्य॑ । \newline
47. स॒ङ्ग्रा॒ममिति॑ सम् - ग्रा॒मम् । \newline
48. स॒म्ॅयन्ति॒ यस्य॒ यस्य॑ स॒म्ॅयन्ति॑ स॒म्ॅयन्ति॒ यस्य॒ पूर्व॑स्य॒ पूर्व॑स्य॒ यस्य॑ स॒म्ॅयन्ति॑ स॒म्ॅयन्ति॒ यस्य॒ पूर्व॑स्य । \newline
49. स॒म्ॅयन्तीति॑ सम् - यन्ति॑ । \newline
50. यस्य॒ पूर्व॑स्य॒ पूर्व॑स्य॒ यस्य॒ यस्य॒ पूर्व॑स्य॒ जुह्व॑ति॒ जुह्व॑ति॒ पूर्व॑स्य॒ यस्य॒ यस्य॒ पूर्व॑स्य॒ जुह्व॑ति । \newline
51. पूर्व॑स्य॒ जुह्व॑ति॒ जुह्व॑ति॒ पूर्व॑स्य॒ पूर्व॑स्य॒ जुह्व॑ति॒ स स जुह्व॑ति॒ पूर्व॑स्य॒ पूर्व॑स्य॒ जुह्व॑ति॒ सः । \newline
52. जुह्व॑ति॒ स स जुह्व॑ति॒ जुह्व॑ति॒ स ए॒वैव स जुह्व॑ति॒ जुह्व॑ति॒ स ए॒व । \newline
53. स ए॒वैव स स ए॒व भ॑वति भव त्ये॒व स स ए॒व भ॑वति । \newline
54. ए॒व भ॑वति भव त्ये॒वैव भ॑वति॒ जय॑ति॒ जय॑ति भव त्ये॒वैव भ॑वति॒ जय॑ति । \newline
55. भ॒व॒ति॒ जय॑ति॒ जय॑ति भवति भवति॒ जय॑ति॒ तम् तम् जय॑ति भवति भवति॒ जय॑ति॒ तम् । \newline
56. जय॑ति॒ तम् तम् जय॑ति॒ जय॑ति॒ तꣳ स॑ङ्ग्रा॒मꣳ स॑ङ्ग्रा॒मम् तम् जय॑ति॒ जय॑ति॒ तꣳ स॑ङ्ग्रा॒मम् । \newline
57. तꣳ स॑ङ्ग्रा॒मꣳ स॑ङ्ग्रा॒मम् तम् तꣳ स॑ङ्ग्रा॒मम् मा᳚न्धु॒को मा᳚न्धु॒कः स॑ङ्ग्रा॒मम् तम् तꣳ स॑ङ्ग्रा॒मम् मा᳚न्धु॒कः । \newline
58. स॒ङ्ग्रा॒मम् मा᳚न्धु॒को मा᳚न्धु॒कः स॑ङ्ग्रा॒मꣳ स॑ङ्ग्रा॒मम् मा᳚न्धु॒क इ॒द्ध्म इ॒द्ध्मो मा᳚न्धु॒कः स॑ङ्ग्रा॒मꣳ स॑ङ्ग्रा॒मम् मा᳚न्धु॒क इ॒द्ध्मः । \newline
59. स॒ङ्ग्रा॒ममिति॑ सम् - ग्रा॒मम् । \newline
60. मा॒न्धु॒क इ॒द्ध्म इ॒द्ध्मो मा᳚न्धु॒को मा᳚न्धु॒क इ॒द्ध्मो भ॑वति भवती॒द्ध्मो मा᳚न्धु॒को मा᳚न्धु॒क इ॒द्ध्मो भ॑वति । \newline
61. इ॒द्ध्मो भ॑वति भवती॒द्ध्म इ॒द्ध्मो भ॑व॒ त्यङ्गा॑रा॒ अङ्गा॑रा भवती॒ द्ध्म इ॒द्ध्मो भ॑व॒ त्यङ्गा॑राः । \newline
\pagebreak
\markright{ TS 3.4.8.4  \hfill https://www.vedavms.in \hfill}

\section{ TS 3.4.8.4 }

\textbf{TS 3.4.8.4 } \newline
\textbf{Samhita Paata} \newline

भ॑व॒त्यङ्गा॑रा ए॒व प्र॑ति॒वेष्ट॑माना अ॒मित्रा॑णामस्य॒ सेनां॒ प्रति॑वेष्टयन्ति॒ य उ॒न्माद्ये॒त् तस्मै॑ होत॒व्या॑ गन्धर्वाफ्स॒रसो॒ वा ए॒तमुन्मा॑दयन्ति॒ य उ॒न्माद्य॑त्ये॒ते खलु॒ वै ग॑न्धर्वाफ्स॒रसो॒ यद्रा᳚ष्ट्र॒भृत॒स्तस्मै॒ स्वाहा॒ ताभ्यः॒ स्वाहेति॑ जुहोति॒ तेनै॒वैना᳚ञ्छमयति॒ नैय॑ग्रोध॒ औदु॑म्बर॒ आश्व॑त्थः॒ प्लाक्ष॒ इती॒द्ध्मो भ॑वत्ये॒ते वै ग॑न्धर्वाफ्स॒रसां᳚ गृ॒हाः स्व ए॒वैना॑ - [  ] \newline

\textbf{Pada Paata} \newline

भ॒व॒ति॒ । अङ्गा॑राः । ए॒व । प्र॒ति॒वेष्ट॑माना॒ इति॑ प्रति - वेष्ट॑मानाः । अ॒मित्रा॑णाम् । अ॒स्य॒ । सेना᳚म् । प्रतीति॑ । वे॒ष्ट॒य॒न्ति॒ । यः । उ॒न्माद्ये॒दित्यु॑त् - माद्ये᳚त् । तस्मै᳚ । हो॒त॒व्याः᳚ । ग॒न्ध॒र्वा॒फ्स॒रस॒ इति॑ गन्धर्व - अ॒फ्स॒रसः॑ । वै । ए॒तम् । उदिति॑ । मा॒द॒य॒न्ति॒ । यः । उ॒न्माद्य॒तीत्यु॑त् - माद्य॑ति । ए॒ते । खलु॑ । वै । ग॒न्ध॒र्वा॒फ्स॒रस॒ इति॑ गन्धर्व - अ॒फ्स॒रसः॑ । यत् । रा॒ष्ट्र॒भृत॒ इति॑ राष्ट्र - भृतः॑ । तस्मै᳚ । स्वाहा᳚ । ताभ्यः॑ । स्वाहा᳚ । इति॑ । जु॒हो॒ति॒ । तेन॑ । ए॒व । ए॒ना॒न् । श॒म॒य॒ति॒ । नैय॑ग्रोधः । औदु॑बंरः । आश्व॑त्थः । प्लाक्षः॑ । इति॑ । इ॒द्ध्मः । भ॒व॒ति॒ । ए॒ते । वै । ग॒न्ध॒र्वा॒फ्स॒रसा॒मिति॑ गन्धर्व - अ॒फ्स॒रसा᳚म् । गृ॒हाः । स्वे । ए॒व । ए॒ना॒न् ।  \newline


\textbf{Krama Paata} \newline

भ॒व॒त्यङ्गा॑राः । अङ्गा॑रा ए॒व । ए॒व प्र॑ति॒वेष्ट॑मानाः । प्र॒ति॒वेष्ट॑माना अ॒मित्रा॑णाम् । प्र॒ति॒वेष्ट॑माना॒ इति॑ प्रति - वेष्ट॑मानाः । अ॒मित्रा॑णामस्य । अ॒स्य॒ सेना᳚म् । सेना॒म् प्रति॑ । प्रति॑ वेष्टयन्ति । वे॒ष्ट॒य॒न्ति॒ यः । य उ॒न्माद्ये᳚त् । उ॒न्माद्ये॒त् तस्मै᳚ । उ॒न्माद्ये॒दित्यु॑त् - माद्ये᳚त् । तस्मै॑ होत॒व्याः᳚ । हो॒त॒व्या॑ गन्धर्वाफ्स॒रसः॑ । ग॒न्ध॒र्वा॒फ्स॒रसो॒ वै । ग॒न्ध॒र्वा॒फ्स॒रस॒ इति॑ गन्धर्व - अ॒फ्स॒रसः॑ । वा ए॒तम् । ए॒तमुत् । उन् मा॑दयन्ति । मा॒द॒य॒न्ति॒ यः । य उ॒न्माद्य॑ति । उ॒न्माद्य॑त्ये॒ते । उ॒न्माद्य॒तीत्यु॑त् - माद्य॑ति । ए॒ते खलु॑ । खलु॒ वै । वै ग॑न्धर्वाफ्स॒रसः॑ । ग॒न्ध॒र्वा॒फ्स॒रसो॒ यत् । ग॒न्ध॒र्वा॒फ्स॒रस॒ इति॑ गन्धर्व - अ॒फ्स॒रसः॑ । यद् रा᳚ष्ट्र॒भृतः॑ । रा॒ष्ट्र॒भृत॒स्तस्मै᳚ । रा॒ष्ट्र॒भृत॒ इति॑ राष्ट्र - भृतः॑ । तस्मै॒ स्वाहा᳚ । स्वाहा॒ ताभ्यः॑ । ताभ्यः॒ स्वाहा᳚ । स्वाहेति॑ । इति॑ जुहोति । जु॒हो॒ति॒ तेन॑ । तेनै॒व । ए॒वैनान्॑ । ए॒ना॒ञ्छ॒म॒य॒ति॒ । श॒म॒य॒ति॒ नैय॑ग्रोधः । नैय॑ग्रोध॒ औदु॑म्बरः । औदु॑म्बर॒ आश्व॑त्थः । आश्व॑त्थः॒ प्लाक्षः॑ । प्लाक्ष॒ इति॑ । इती॒द्ध्मः । इ॒द्ध्मो भ॑वति । भ॒व॒त्ये॒ते । ए॒ते वै । वै ग॑न्धर्वाफ्स॒रसा᳚म् । ग॒न्ध॒र्वा॒फ्स॒रसा᳚म् गृ॒हाः । ग॒न्ध॒र्वा॒फ्स॒रसा॒मिति॑ गन्धर्व - अ॒फ्स॒रसा᳚म् । गृ॒हाः स्वे । स्व ए॒व । ए॒वैनान्॑ । ए॒ना॒ना॒यत॑ने \newline

\textbf{Jatai Paata} \newline

1. भ॒व॒ त्यङ्गा॑रा॒ अङ्गा॑रा भवति भव॒ त्यङ्गा॑राः । \newline
2. अङ्गा॑रा ए॒वैवाङ्गा॑रा॒ अङ्गा॑रा ए॒व । \newline
3. ए॒व प्र॑ति॒वेष्ट॑मानाः प्रति॒वेष्ट॑माना ए॒वैव प्र॑ति॒वेष्ट॑मानाः । \newline
4. प्र॒ति॒वेष्ट॑माना अ॒मित्रा॑णा म॒मित्रा॑णाम् प्रति॒वेष्ट॑मानाः प्रति॒वेष्ट॑माना अ॒मित्रा॑णाम् । \newline
5. प्र॒ति॒वेष्ट॑माना॒ इति॑ प्रति - वेष्ट॑मानाः । \newline
6. अ॒मित्रा॑णा मस्यास्या॒ मित्रा॑णा म॒मित्रा॑णा मस्य । \newline
7. अ॒स्य॒ सेनाꣳ॒॒ सेना॑ मस्यास्य॒ सेना᳚म् । \newline
8. सेना॒म् प्रति॒ प्रति॒ सेनाꣳ॒॒ सेना॒म् प्रति॑ । \newline
9. प्रति॑ वेष्टयन्ति वेष्टयन्ति॒ प्रति॒ प्रति॑ वेष्टयन्ति । \newline
10. वे॒ष्ट॒य॒न्ति॒ यो यो वे᳚ष्टयन्ति वेष्टयन्ति॒ यः । \newline
11. य उ॒न्माद्ये॑ दु॒न्माद्ये॒द् यो य उ॒न्माद्ये᳚त् । \newline
12. उ॒न्माद्ये॒त् तस्मै॒ तस्मा॑ उ॒न्माद्ये॑ दु॒न्माद्ये॒त् तस्मै᳚ । \newline
13. उ॒न्माद्ये॒दित्यु॑त् - माद्ये᳚त् । \newline
14. तस्मै॑ होत॒व्या॑ होत॒व्या᳚ स्तस्मै॒ तस्मै॑ होत॒व्याः᳚ । \newline
15. हो॒त॒व्या॑ गन्धर्वाफ्स॒रसो॑ गन्धर्वाफ्स॒रसो॑ होत॒व्या॑ होत॒व्या॑ गन्धर्वाफ्स॒रसः॑ । \newline
16. ग॒न्ध॒र्वा॒फ्स॒रसो॒ वै वै ग॑न्धर्वाफ्स॒रसो॑ गन्धर्वाफ्स॒रसो॒ वै । \newline
17. ग॒न्ध॒र्वा॒फ्स॒रस॒ इति॑ गन्धर्व - अ॒फ्स॒रसः॑ । \newline
18. वा ए॒त मे॒तं ॅवै वा ए॒तम् । \newline
19. ए॒त मुदु दे॒त मे॒त मुत् । \newline
20. उन् मा॑दयन्ति मादय॒ न्त्युदुन् मा॑दयन्ति । \newline
21. मा॒द॒य॒न्ति॒ यो यो मा॑दयन्ति मादयन्ति॒ यः । \newline
22. य उ॒न्माद्य॑ त्यु॒न्माद्य॑ति॒ यो य उ॒न्माद्य॑ति । \newline
23. उ॒न्माद्य॑ त्ये॒त ए॒त उ॒न्माद्य॑ त्यु॒न्माद्य॑ त्ये॒ते । \newline
24. उ॒न्माद्य॒तीत्यु॑त् - माद्य॑ति । \newline
25. ए॒ते खलु॒ खल्वे॒त ए॒ते खलु॑ । \newline
26. खलु॒ वै वै खलु॒ खलु॒ वै । \newline
27. वै ग॑न्धर्वाफ्स॒रसो॑ गन्धर्वाफ्स॒रसो॒ वै वै ग॑न्धर्वाफ्स॒रसः॑ । \newline
28. ग॒न्ध॒र्वा॒ फ्स॒रसो॒ यद् यद् ग॑न्धर्वा फ्स॒रसो॑ गन्धर्वा फ्स॒रसो॒ यत् । \newline
29. ग॒न्ध॒र्वा॒फ्स॒रस॒ इति॑ गन्धर्व - अ॒फ्स॒रसः॑ । \newline
30. यद् रा᳚ष्ट्र॒भृतो॑ राष्ट्र॒भृतो॒ यद् यद् रा᳚ष्ट्र॒भृतः॑ । \newline
31. रा॒ष्ट्र॒भृत॒ स्तस्मै॒ तस्मै॑ राष्ट्र॒भृतो॑ राष्ट्र॒भृत॒ स्तस्मै᳚ । \newline
32. रा॒ष्ट्र॒भृत॒ इति॑ राष्ट्र - भृतः॑ । \newline
33. तस्मै॒ स्वाहा॒ स्वाहा॒ तस्मै॒ तस्मै॒ स्वाहा᳚ । \newline
34. स्वाहा॒ ताभ्य॒ स्ताभ्यः॒ स्वाहा॒ स्वाहा॒ ताभ्यः॑ । \newline
35. ताभ्यः॒ स्वाहा॒ स्वाहा॒ ताभ्य॒ स्ताभ्यः॒ स्वाहा᳚ । \newline
36. स्वाहे तीति॒ स्वाहा॒ स्वाहेति॑ । \newline
37. इति॑ जुहोति जुहो॒ती तीति॑ जुहोति । \newline
38. जु॒हो॒ति॒ तेन॒ तेन॑ जुहोति जुहोति॒ तेन॑ । \newline
39. तेनै॒वैव तेन॒ तेनै॒व । \newline
40. ए॒वैना॑-नेना-ने॒वैवैनान्॑ । \newline
41. ए॒ना॒ञ् छ॒म॒य॒ति॒ श॒म॒य॒ त्ये॒ना॒-ने॒ना॒ञ् छ॒म॒य॒ति॒ । \newline
42. श॒म॒य॒ति॒ नैय॑ग्रोधो॒ नैय॑ग्रोधः शमयति शमयति॒ नैय॑ग्रोधः । \newline
43. नैय॑ग्रोध॒ औदु॑म्बर॒ औदु॑म्बरो॒ नैय॑ग्रोधो॒ नैय॑ग्रोध॒ औदु॑म्बरः । \newline
44. औदु॑म्बर॒ आश्व॑त्थ॒ आश्व॑त्थ॒ औदु॑म्बर॒ औदु॑म्बर॒ आश्व॑त्थः । \newline
45. आश्व॑त्थः॒ प्लाक्षः॒ प्लाक्ष॒ आश्व॑त्थ॒ आश्व॑त्थः॒ प्लाक्षः॑ । \newline
46. प्लाक्ष॒ इतीति॒ प्लाक्षः॒ प्लाक्ष॒ इति॑ । \newline
47. इती॒ द्ध्म इ॒द्ध्म इतीती॒ द्ध्मः । \newline
48. इ॒द्ध्मो भ॑वति भवती॒ द्ध्म इ॒द्ध्मो भ॑वति । \newline
49. भ॒व॒ त्ये॒त ए॒ते भ॑वति भव त्ये॒ते । \newline
50. ए॒ते वै वा ए॒त ए॒ते वै । \newline
51. वै ग॑न्धर्वाफ्स॒रसा᳚म् गन्धर्वाफ्स॒रसां॒ ॅवै वै ग॑न्धर्वाफ्स॒रसा᳚म् । \newline
52. ग॒न्ध॒र्वा॒फ्स॒रसा᳚म् गृ॒हा गृ॒हा ग॑न्धर्वाफ्स॒रसा᳚म् गन्धर्वाफ्स॒रसा᳚म् गृ॒हाः । \newline
53. ग॒न्ध॒र्वा॒फ्स॒रसा॒मिति॑ गन्धर्व - अ॒फ्स॒रसा᳚म् । \newline
54. गृ॒हाः स्वे स्वे गृ॒हा गृ॒हाः स्वे । \newline
55. स्व ए॒वैव स्वे स्व ए॒व । \newline
56. ए॒वैना॑-नेना-ने॒वैवैनान्॑ । \newline
57. ए॒ना॒-ना॒यत॑न आ॒यत॑न एना-नेना-ना॒यत॑ने । \newline

\textbf{Ghana Paata } \newline

1. भ॒व॒ त्यङ्गा॑रा॒ अङ्गा॑रा भवति भव॒ त्यङ्गा॑रा ए॒वैवाङ्गा॑रा भवति भव॒ त्यङ्गा॑रा ए॒व । \newline
2. अङ्गा॑रा ए॒वैवाङ्गा॑रा॒ अङ्गा॑रा ए॒व प्र॑ति॒वेष्ट॑मानाः प्रति॒वेष्ट॑माना ए॒वाङ्गा॑रा॒ अङ्गा॑रा ए॒व प्र॑ति॒वेष्ट॑मानाः । \newline
3. ए॒व प्र॑ति॒वेष्ट॑मानाः प्रति॒वेष्ट॑माना ए॒वैव प्र॑ति॒वेष्ट॑माना अ॒मित्रा॑णा म॒मित्रा॑णाम् प्रति॒वेष्ट॑माना ए॒वैव प्र॑ति॒वेष्ट॑माना अ॒मित्रा॑णाम् । \newline
4. प्र॒ति॒वेष्ट॑माना अ॒मित्रा॑णा म॒मित्रा॑णाम् प्रति॒वेष्ट॑मानाः प्रति॒वेष्ट॑माना अ॒मित्रा॑णा मस्या स्या॒मित्रा॑णाम् प्रति॒वेष्ट॑मानाः प्रति॒वेष्ट॑माना अ॒मित्रा॑णा मस्य । \newline
5. प्र॒ति॒वेष्ट॑माना॒ इति॑ प्रति - वेष्ट॑मानाः । \newline
6. अ॒मित्रा॑णा मस्या स्या॒मित्रा॑णा म॒मित्रा॑णा मस्य॒ सेनाꣳ॒॒ सेना॑ मस्या॒मित्रा॑णा म॒मित्रा॑णा मस्य॒ सेना᳚म् । \newline
7. अ॒स्य॒ सेनाꣳ॒॒ सेना॑ मस्यास्य॒ सेना॒म् प्रति॒ प्रति॒ सेना॑ मस्यास्य॒ सेना॒म् प्रति॑ । \newline
8. सेना॒म् प्रति॒ प्रति॒ सेनाꣳ॒॒ सेना॒म् प्रति॑ वेष्टयन्ति वेष्टयन्ति॒ प्रति॒ सेनाꣳ॒॒ सेना॒म् प्रति॑ वेष्टयन्ति । \newline
9. प्रति॑ वेष्टयन्ति वेष्टयन्ति॒ प्रति॒ प्रति॑ वेष्टयन्ति॒ यो यो वे᳚ष्टयन्ति॒ प्रति॒ प्रति॑ वेष्टयन्ति॒ यः । \newline
10. वे॒ष्ट॒य॒न्ति॒ यो यो वे᳚ष्टयन्ति वेष्टयन्ति॒ य उ॒न्माद्ये॑ दु॒न्माद्ये॒द् यो वे᳚ष्टयन्ति वेष्टयन्ति॒ य उ॒न्माद्ये᳚त् । \newline
11. य उ॒न्माद्ये॑ दु॒न्माद्ये॒द् यो य उ॒न्माद्ये॒त् तस्मै॒ तस्मा॑ उ॒न्माद्ये॒द् यो य उ॒न्माद्ये॒त् तस्मै᳚ । \newline
12. उ॒न्माद्ये॒त् तस्मै॒ तस्मा॑ उ॒न्माद्ये॑ दु॒न्माद्ये॒त् तस्मै॑ होत॒व्या॑ होत॒व्या᳚ स्तस्मा॑ उ॒न्माद्ये॑ दु॒न्माद्ये॒त् तस्मै॑ होत॒व्याः᳚ । \newline
13. उ॒न्माद्ये॒दित्यु॑त् - माद्ये᳚त् । \newline
14. तस्मै॑ होत॒व्या॑ होत॒व्या᳚ स्तस्मै॒ तस्मै॑ होत॒व्या॑ गन्धर्वाफ्स॒रसो॑ गन्धर्वाफ्स॒रसो॑ होत॒व्या᳚ स्तस्मै॒ तस्मै॑ होत॒व्या॑ गन्धर्वाफ्स॒रसः॑ । \newline
15. हो॒त॒व्या॑ गन्धर्वाफ्स॒रसो॑ गन्धर्वाफ्स॒रसो॑ होत॒व्या॑ होत॒व्या॑ गन्धर्वाफ्स॒रसो॒ वै वै ग॑न्धर्वाफ्स॒रसो॑ होत॒व्या॑ होत॒व्या॑ गन्धर्वाफ्स॒रसो॒ वै । \newline
16. ग॒न्ध॒र्वा॒फ्स॒रसो॒ वै वै ग॑न्धर्वाफ्स॒रसो॑ गन्धर्वाफ्स॒रसो॒ वा ए॒त मे॒तम् ॅवै ग॑न्धर्वाफ्स॒रसो॑ गन्धर्वाफ्स॒रसो॒ वा ए॒तम् । \newline
17. ग॒न्ध॒र्वा॒फ्स॒रस॒ इति॑ गन्धर्व - अ॒फ्स॒रसः॑ । \newline
18. वा ए॒त मे॒तम् ॅवै वा ए॒त मुदु दे॒तम् ॅवै वा ए॒त मुत् । \newline
19. ए॒त मुदु दे॒त मे॒त मुन् मा॑दयन्ति मादय॒ न्त्युदे॒त मे॒त मुन् मा॑दयन्ति । \newline
20. उन् मा॑दयन्ति मादय॒ न्त्युदुन् मा॑दयन्ति॒ यो यो मा॑दय॒ न्त्युदुन् मा॑दयन्ति॒ यः । \newline
21. मा॒द॒य॒न्ति॒ यो यो मा॑दयन्ति मादयन्ति॒ य उ॒न्माद्य॑ त्यु॒न्माद्य॑ति॒ यो मा॑दयन्ति मादयन्ति॒ य उ॒न्माद्य॑ति । \newline
22. य उ॒न्माद्य॑ त्यु॒न्माद्य॑ति॒ यो य उ॒न्माद्य॑ त्ये॒त ए॒त उ॒न्माद्य॑ति॒ यो य उ॒न्माद्य॑ त्ये॒ते । \newline
23. उ॒न्माद्य॑ त्ये॒त ए॒त उ॒न्माद्य॑ त्यु॒न्माद्य॑ त्ये॒ते खलु॒ खल्वे॒त उ॒न्माद्य॑ त्यु॒न् आद्य॑त्ये॒ते खलु॑ । \newline
24. उ॒न्माद्य॒तीत्यु॑त् - माद्य॑ति । \newline
25. ए॒ते खलु॒ खल्वे॒त ए॒ते खलु॒ वै वै खल्वे॒त ए॒ते खलु॒ वै । \newline
26. खलु॒ वै वै खलु॒ खलु॒ वै ग॑न्धर्वाफ्स॒रसो॑ गन्धर्वाफ्स॒रसो॒ वै खलु॒ खलु॒ वै ग॑न्धर्वाफ्स॒रसः॑ । \newline
27. वै ग॑न्धर्वाफ्स॒रसो॑ गन्धर्वाफ्स॒रसो॒ वै वै ग॑न्धर्वाफ्स॒रसो॒ यद् यद् ग॑न्धर्वाफ्स॒रसो॒ वै वै ग॑न्धर्वाफ्स॒रसो॒ यत् । \newline
28. ग॒न्ध॒र्वा॒फ्स॒रसो॒ यद् यद् ग॑न्धर्वाफ्स॒रसो॑ गन्धर्वाफ्स॒रसो॒ यद् रा᳚ष्ट्र॒भृतो॑ राष्ट्र॒भृतो॒ यद् ग॑न्धर्वाफ्स॒रसो॑ गन्धर्वाफ्स॒रसो॒ यद् रा᳚ष्ट्र॒भृतः॑ । \newline
29. ग॒न्ध॒र्वा॒फ्स॒रस॒ इति॑ गन्धर्व - अ॒फ्स॒रसः॑ । \newline
30. यद् रा᳚ष्ट्र॒भृतो॑ राष्ट्र॒भृतो॒ यद् यद् रा᳚ष्ट्र॒भृत॒ स्तस्मै॒ तस्मै॑ राष्ट्र॒भृतो॒ यद् यद् 
रा᳚ष्ट्र॒भृत॒ स्तस्मै᳚ । \newline
31. रा॒ष्ट्र॒भृत॒ स्तस्मै॒ तस्मै॑ राष्ट्र॒भृतो॑ राष्ट्र॒भृत॒ स्तस्मै॒ स्वाहा॒ स्वाहा॒ तस्मै॑ राष्ट्र॒भृतो॑ राष्ट्र॒भृत॒ स्तस्मै॒ स्वाहा᳚ । \newline
32. रा॒ष्ट्र॒भृत॒ इति॑ राष्ट्र - भृतः॑ । \newline
33. तस्मै॒ स्वाहा॒ स्वाहा॒ तस्मै॒ तस्मै॒ स्वाहा॒ ताभ्य॒ स्ताभ्यः॒ स्वाहा॒ तस्मै॒ तस्मै॒ स्वाहा॒ ताभ्यः॑ । \newline
34. स्वाहा॒ ताभ्य॒ स्ताभ्यः॒ स्वाहा॒ स्वाहा॒ ताभ्यः॒ स्वाहा॒ स्वाहा॒ ताभ्यः॒ स्वाहा॒ स्वाहा॒ ताभ्यः॒ स्वाहा᳚ । \newline
35. ताभ्यः॒ स्वाहा॒ स्वाहा॒ ताभ्य॒ स्ताभ्यः॒ स्वाहेतीति॒ स्वाहा॒ ताभ्य॒ स्ताभ्यः॒ स्वाहेति॑ । \newline
36. स्वाहेतीति॒ स्वाहा॒ स्वाहेति॑ जुहोति जुहो॒तीति॒ स्वाहा॒ स्वाहेति॑ जुहोति । \newline
37. इति॑ जुहोति जुहो॒तीतीति॑ जुहोति॒ तेन॒ तेन॑ जुहो॒तीतीति॑ जुहोति॒ तेन॑ । \newline
38. जु॒हो॒ति॒ तेन॒ तेन॑ जुहोति जुहोति॒ तेनै॒वैव तेन॑ जुहोति जुहोति॒ तेनै॒व । \newline
39. तेनै॒वैव तेन॒ तेनै॒वैना॑ नेनाने॒व तेन॒ तेनै॒वैनान्॑ । \newline
40. ए॒वैना॑,नेना,ने॒वैवैना᳚ञ् छमयति शमय त्येना, ने॒वैवैना᳚ञ् छमयति । \newline
41. ए॒ना॒ञ् छ॒म॒य॒ति॒ श॒म॒य॒ त्ये॒ना॒, ने॒ना॒ञ् छ॒म॒य॒ति॒ नैय॑ग्रोधो॒ नैय॑ग्रोधः शमय त्येना नेनाञ् छमयति॒ नैय॑ग्रोधः । \newline
42. श॒म॒य॒ति॒ नैय॑ग्रोधो॒ नैय॑ग्रोधः शमयति शमयति॒ नैय॑ग्रोध॒ औदु॑म्बर॒ औदु॑म्बरो॒ नैय॑ग्रोधः शमयति शमयति॒ नैय॑ग्रोध॒ औदु॑म्बरः । \newline
43. नैय॑ग्रोध॒ औदु॑म्बर॒ औदु॑म्बरो॒ नैय॑ग्रोधो॒ नैय॑ग्रोध॒ औदु॑म्बर॒ आश्व॑त्थ॒ आश्व॑त्थ॒ औदु॑म्बरो॒ नैय॑ग्रोधो॒ नैय॑ग्रोध॒ औदु॑म्बर॒ आश्व॑त्थः । \newline
44. औदु॑म्बर॒ आश्व॑त्थ॒ आश्व॑त्थ॒ औदु॑म्बर॒ औदु॑म्बर॒ आश्व॑त्थः॒ प्लाक्षः॒ प्लाक्ष॒ आश्व॑त्थ॒ औदु॑म्बर॒ औदु॑म्बर॒ आश्व॑त्थः॒ प्लाक्षः॑ । \newline
45. आश्व॑त्थः॒ प्लाक्षः॒ प्लाक्ष॒ आश्व॑त्थ॒ आश्व॑त्थः॒ प्लाक्ष॒ इतीति॒ प्लाक्ष॒ आश्व॑त्थ॒ आश्व॑त्थः॒ प्लाक्ष॒ इति॑ । \newline
46. प्लाक्ष॒ इतीति॒ प्लाक्षः॒ प्लाक्ष॒ इती॒ द्ध्म इ॒द्ध्म इति॒ प्लाक्षः॒ प्लाक्ष॒ इती॒ द्ध्मः । \newline
47. इती॒ द्ध्म इ॒द्ध्म इतीती॒ द्ध्मो भ॑वति भवती॒ द्ध्म इतीती॒ द्ध्मो भ॑वति । \newline
48. इ॒द्ध्मो भ॑वति भवती॒ द्ध्म इ॒द्ध्मो भ॑व त्ये॒त ए॒ते भ॑वती॒ द्ध्म इ॒द्ध्मो भ॑व त्ये॒ते । \newline
49. भ॒व॒ त्ये॒त ए॒ते भ॑वति भव त्ये॒ते वै वा ए॒ते भ॑वति भव त्ये॒ते वै । \newline
50. ए॒ते वै वा ए॒त ए॒ते वै ग॑न्धर्वाफ्स॒रसा᳚म् गन्धर्वाफ्स॒रसा॒म् ॅवा ए॒त ए॒ते वै ग॑न्धर्वाफ्स॒रसा᳚म् । \newline
51. वै ग॑न्धर्वाफ्स॒रसा᳚म् गन्धर्वाफ्स॒रसा॒म् ॅवै वै ग॑न्धर्वाफ्स॒रसा᳚म् गृ॒हा गृ॒हा ग॑न्धर्वाफ्स॒रसा॒म् ॅवै वै ग॑न्धर्वाफ्स॒रसा᳚म् गृ॒हाः । \newline
52. ग॒न्ध॒र्वा॒फ्स॒रसा᳚म् गृ॒हा गृ॒हा ग॑न्धर्वाफ्स॒रसा᳚म् गन्धर्वाफ्स॒रसा᳚म् गृ॒हाः स्वे स्वे गृ॒हा ग॑न्धर्वाफ्स॒रसा᳚म् गन्धर्वाफ्स॒रसा᳚म् गृ॒हाः स्वे । \newline
53. ग॒न्ध॒र्वा॒फ्स॒रसा॒मिति॑ गन्धर्व - अ॒फ्स॒रसा᳚म् । \newline
54. गृ॒हाः स्वे स्वे गृ॒हा गृ॒हाः स्व ए॒वैव स्वे गृ॒हा गृ॒हाः स्व ए॒व । \newline
55. स्व ए॒वैव स्वे स्व ए॒वैना॑,नेना,ने॒व स्वे स्व ए॒वैनान्॑ । \newline
56. ए॒वैना॑,नेना,ने॒वैवैना॑,ना॒यत॑न आ॒यत॑न एना,ने॒वैवैना॑ ना॒यत॑ने । \newline
57. ए॒ना॒,ना॒यत॑न आ॒यत॑न एना,नेना,ना॒यत॑ने शमयति शमय त्या॒यत॑न एनानेना,ना॒यत॑ने शमयति । \newline
\pagebreak
\markright{ TS 3.4.8.5  \hfill https://www.vedavms.in \hfill}

\section{ TS 3.4.8.5 }

\textbf{TS 3.4.8.5 } \newline
\textbf{Samhita Paata} \newline

ना॒यत॑ने शमयत्यभि॒चर॑ता प्रतिलो॒मꣳ हो॑त॒व्याः᳚ प्रा॒णाने॒वास्य॑ प्र॒तीचः॒ प्रति॑ यौति॒ तं ततो॒ येन॒ केन॑ च स्तृणुते॒ स्वकृ॑त॒ इरि॑णे जुहोति प्रद॒रे वै॒तद्वा अ॒स्यै निर्.ऋ॑तिगृहीतं॒ निर्.ऋ॑तिगृहीत ए॒वैनं॒ निर्.ऋ॑त्या ग्राहयति॒ यद्वा॒चः क्रू॒रं तेन॒ वष॑ट् करोति वा॒च ए॒वैनं॑ क्रू॒रेण॒ प्रवृ॑श्चति ता॒जगार्ति॒मार्च्छ॑ति॒ यस्य॑ का॒मये॑ता॒न्नाद्य॒ - [  ] \newline

\textbf{Pada Paata} \newline

आ॒यत॑न॒ इत्या᳚ - यत॑ने । श॒म॒य॒ति॒ । अ॒भि॒चर॒तेत्य॑भि - चर॑ता । प्र॒ति॒लो॒ममिति॑ प्रति - लो॒मम् । हो॒त॒व्याः᳚ । प्रा॒णानिति॑ प्र - अ॒नान् । ए॒व । अ॒स्य॒ । प्र॒तीचः॑ । प्रतीति॑ । यौ॒ति॒ । तम् । ततः॑ । येन॑ । केन॑ । च॒ । स्तृ॒णु॒ते॒ । स्वकृ॑त॒ इति॒ स्व - कृ॒ते॒ । इरि॑णे । जु॒हो॒ति॒ । प्र॒द॒र इति॑ प्र - द॒रे । वा॒ । ए॒तत् । वै । अ॒स्यै । निर्.ऋ॑तिगृहीत॒मिति॒ निर्.ऋ॑ति - गृ॒ही॒त॒म् । निर्.ऋ॑तिगृहीत॒ इति॒ निर्.ऋ॑ति-गृ॒ही॒ते॒ । ए॒व । ए॒न॒म् । निर्.ऋ॒त्येति॒ निः-ऋ॒त्या॒ । ग्रा॒ह॒य॒ति॒ । यत् । वा॒चः । क्रू॒रम् । तेन॑ । वष॑ट् । क॒रो॒ति॒ । वा॒चः । ए॒व । ए॒न॒म् । क्रू॒रेण॑ । प्रेति॑ । वृ॒श्च॒ति॒ । ता॒जक् । आर्ति᳚म् । एति॑ । ऋ॒च्छ॒ति॒ । यस्य॑ । का॒मये॑त । अ॒न्नाद्य॒मित्य॑न्न - अद्य᳚म् ।  \newline


\textbf{Krama Paata} \newline

आ॒यत॑ने शमयति । आ॒यत॑न॒ इत्या᳚ - यत॑ने । श॒म॒य॒त्य॒भि॒चर॑ता । अ॒भि॒चर॑ता प्रतिलो॒मम् । अ॒भि॒चर॒तेत्य॑भि - चर॑ता । प्र॒ति॒लो॒मꣳ हो॑त॒व्याः᳚ । प्र॒ति॒लो॒ममिति॑ प्रति - लो॒मम् । हो॒त॒व्याः᳚ प्रा॒णान् । प्रा॒णाने॒व । प्रा॒णानिति॑ प्र - अ॒नान् । ए॒वास्य॑ । अ॒स्य॒ प्र॒तीचः॑ । प्र॒तीचः॒ प्रति॑ । प्रति॑ यौति । यौ॒ति॒ तम् । तम् ततः॑ । ततो॒ येन॑ । येन॒ केन॑ । केन॑ च । च॒ स्तृ॒णु॒ते॒ । स्तृ॒णु॒ते॒ स्वकृ॑ते । स्वकृ॑त॒ इरि॑णे । स्वकृ॑त॒ इति॒ स्व - कृ॒ते॒ । इरि॑णे जुहोति । जु॒हो॒ति॒ प्र॒द॒रे । प्र॒द॒रे वा᳚ । प्र॒द॒र इति॑ प्र - द॒रे । वै॒तत् । ए॒तद् वै । वा अ॒स्यै । अ॒स्यै निर्.ऋ॑तिगृहीतम् । निर्.ऋ॑तिगृहीत॒म् निर्.ऋ॑तिगृहीते । निर्.ऋ॑तिगृहीत॒मिति॒ निर्.ऋ॑ति - गृ॒ही॒त॒म् । निर्.ऋ॑तिगृहीत ए॒व । निर्.ऋ॑तिगृहीत॒ इति॒ निर्.ऋ॑ति - गृ॒ही॒ते॒ । ए॒वैन᳚म् । ए॒न॒म् निर्.ऋ॑त्या । निर्.ऋ॑त्या ग्राहयति । निर्.ऋ॒त्येति॒ निः - ऋ॒त्या॒ । ग्रा॒ह॒य॒ति॒ यत् । यद् वा॒चः । वा॒चः क्रू॒रम् । क्रू॒रम् तेन॑ । तेन॒ वष॑ट् । वष॑ट् करोति । क॒रो॒ति॒ वा॒चः । वा॒च ए॒व । ए॒वैन᳚म् । ए॒न॒म् क्रू॒रेण॑ । क्रू॒रेण॒ प्र । प्र वृ॑श्चति । वृ॒श्च॒ति॒ ता॒जक् । ता॒जगार्ति᳚म् । आर्ति॒मा । आर्च्छ॑ति । ऋ॒च्छ॒ति॒ यस्य॑ । यस्य॑ का॒मये॑त । का॒मये॑ता॒न्नाद्य᳚म् । अ॒न्नाद्य॒मा । अ॒न्नाद्य॒मित्य॑न्न - अद्य᳚म् \newline

\textbf{Jatai Paata} \newline

1. आ॒यत॑ने शमयति शमयत्या॒ यत॑न आ॒यत॑ने शमयति । \newline
2. आ॒यत॑न॒ इत्या᳚ - यत॑ने । \newline
3. श॒म॒य॒ त्य॒भि॒चर॑ता ऽभि॒चर॑ता शमयति शमय त्यभि॒चर॑ता । \newline
4. अ॒भि॒चर॑ता प्रतिलो॒मम् प्र॑तिलो॒म म॑भि॒चर॑ता ऽभि॒चर॑ता प्रतिलो॒मम् । \newline
5. अ॒भि॒चर॒तेत्य॑भि - चर॑ता । \newline
6. प्र॒ति॒लो॒मꣳ हो॑त॒व्या॑ होत॒व्याः᳚ प्रतिलो॒मम् प्र॑तिलो॒मꣳ हो॑त॒व्याः᳚ । \newline
7. प्र॒ति॒लो॒ममिति॑ प्रति - लो॒मम् । \newline
8. हो॒त॒व्याः᳚ प्रा॒णान् प्रा॒णान्. हो॑त॒व्या॑ होत॒व्याः᳚ प्रा॒णान् । \newline
9. प्रा॒णा-ने॒वैव प्रा॒णान् प्रा॒णा-ने॒व । \newline
10. प्रा॒णानिति॑ प्र - अ॒नान् । \newline
11. ए॒वा स्या᳚ स्यै॒वैवास्य॑ । \newline
12. अ॒स्य॒ प्र॒तीचः॑ प्र॒तीचो᳚ ऽस्यास्य प्र॒तीचः॑ । \newline
13. प्र॒तीचः॒ प्रति॒ प्रति॑ प्र॒तीचः॑ प्र॒तीचः॒ प्रति॑ । \newline
14. प्रति॑ यौति यौति॒ प्रति॒ प्रति॑ यौति । \newline
15. यौ॒ति॒ तम् तं ॅयौ॑ति यौति॒ तम् । \newline
16. तम् तत॒ स्तत॒ स्तम् तम् ततः॑ । \newline
17. ततो॒ येन॒ येन॒ तत॒ स्ततो॒ येन॑ । \newline
18. येन॒ केन॒ केन॒ येन॒ येन॒ केन॑ । \newline
19. केन॑ च च॒ केन॒ केन॑ च । \newline
20. च॒ स्तृ॒णु॒ते॒ स्तृ॒णु॒ते॒ च॒ च॒ स्तृ॒णु॒ते॒ । \newline
21. स्तृ॒णु॒ते॒ स्वकृ॑ते॒ स्वकृ॑ते स्तृणुते स्तृणुते॒ स्वकृ॑ते । \newline
22. स्वकृ॑त॒ इरि॑ण॒ इरि॑णे॒ स्वकृ॑ते॒ स्वकृ॑त॒ इरि॑णे । \newline
23. स्वकृ॑त॒ इति॒ स्व - कृ॒ते॒ । \newline
24. इरि॑णे जुहोति जुहो॒ती रि॑ण॒ इरि॑णे जुहोति । \newline
25. जु॒हो॒ति॒ प्र॒द॒रे प्र॑द॒रे जु॑होति जुहोति प्रद॒रे । \newline
26. प्र॒द॒रे वा॑ वा प्रद॒रे प्र॑द॒रे वा᳚ । \newline
27. प्र॒द॒र इति॑ प्र - द॒रे । \newline
28. वै॒त दे॒तद् वा॑ वै॒तत् । \newline
29. ए॒तद् वै वा ए॒त दे॒तद् वै । \newline
30. वा अ॒स्या अ॒स्यै वै वा अ॒स्यै । \newline
31. अ॒स्यै निर्.ऋ॑तिगृहीत॒न्-निर्.ऋ॑तिगृहीत म॒स्या अ॒स्यै निर्.ऋ॑तिगृहीतम् । \newline
32. निर्.ऋ॑तिगृहीत॒न्-निर्.ऋ॑तिगृहीते॒ निर्.ऋ॑तिगृहीते॒ निर्.ऋ॑तिगृहीत॒न्-निर्.ऋ॑तिगृहीत॒न्-निर्.ऋ॑तिगृहीते । \newline
33. निर्.ऋ॑तिगृहीत॒मिति॒ निर्.ऋ॑ति - गृ॒ही॒त॒म् । \newline
34. निर्.ऋ॑तिगृहीत ए॒वैव निर्.ऋ॑तिगृहीते॒ निर्.ऋ॑तिगृहीत ए॒व । \newline
35. निर्.ऋ॑तिगृहीत॒ इति॒ निर्.ऋ॑ति - गृ॒ही॒ते॒ । \newline
36. ए॒वैन॑ मेन मे॒वैवैन᳚म् । \newline
37. ए॒न॒म् निर्.ऋ॑त्या॒ निर्.ऋ॑ त्यैन मेन॒म् निर्.ऋ॑त्या । \newline
38. निर्.ऋ॑त्या ग्राहयति ग्राहयति॒ निर्.ऋ॑त्या॒ निर्.ऋ॑त्या ग्राहयति । \newline
39. निर्.ऋ॒त्येति॒ निः - ऋ॒त्या॒ । \newline
40. ग्रा॒ह॒य॒ति॒ यद् यद् ग्रा॑हयति ग्राहयति॒ यत् । \newline
41. यद् वा॒चो वा॒चो यद् यद् वा॒चः । \newline
42. वा॒चः क्रू॒रम् क्रू॒रं ॅवा॒चो वा॒चः क्रू॒रम् । \newline
43. क्रू॒रम् तेन॒ तेन॑ क्रू॒रम् क्रू॒रम् तेन॑ । \newline
44. तेन॒ वष॒ड् वष॒ट् तेन॒ तेन॒ वष॑ट् । \newline
45. वष॑ट् करोति करोति॒ वष॒ड् वष॑ट् करोति । \newline
46. क॒रो॒ति॒ वा॒चो वा॒चः क॑रोति करोति वा॒चः । \newline
47. वा॒च ए॒वैव वा॒चो वा॒च ए॒व । \newline
48. ए॒वैन॑ मेन मे॒वै वैन᳚म् । \newline
49. ए॒न॒म् क्रू॒रेण॑ क्रू॒रेणै॑न मेनम् क्रू॒रेण॑ । \newline
50. क्रू॒रेण॒ प्र प्र क्रू॒रेण॑ क्रू॒रेण॒ प्र । \newline
51. प्र वृ॑श्चति वृश्चति॒ प्र प्र वृ॑श्चति । \newline
52. वृ॒श्च॒ति॒ ता॒जक् ता॒जग् वृ॑श्चति वृश्चति ता॒जक् । \newline
53. ता॒ज गार्ति॒ मार्ति॑म् ता॒जक् ता॒ज गार्ति᳚म् । \newline
54. आर्ति॒ मा ऽऽर्ति॒ मार्ति॒ मा । \newline
55. आर्च्छ॑ त्यृच्छ॒ त्यार्च्छ॑ति । \newline
56. ऋ॒च्छ॒ति॒ यस्य॒ यस्य॑ र्‌च्छ त्यृच्छति॒ यस्य॑ । \newline
57. यस्य॑ का॒मये॑त का॒मये॑त॒ यस्य॒ यस्य॑ का॒मये॑त । \newline
58. का॒मये॑ता॒ न्नाद्य॑ म॒न्नाद्य॑म् का॒मये॑त का॒मये॑ता॒ न्नाद्य᳚म् । \newline
59. अ॒न्नाद्य॒ मा ऽन्नाद्य॑ म॒न्नाद्य॒ मा । \newline
60. अ॒न्नाद्य॒मित्य॑न्न - अद्य᳚म् । \newline

\textbf{Ghana Paata } \newline

1. आ॒यत॑ने शमयति शमय त्या॒यत॑न आ॒यत॑ने शमय त्यभि॒चर॑ता ऽभि॒चर॑ता शमय त्या॒यत॑न आ॒यत॑ने शमय त्यभि॒चर॑ता । \newline
2. आ॒यत॑न॒ इत्या᳚ - यत॑ने । \newline
3. श॒म॒य॒ त्य॒भि॒चर॑ता ऽभि॒चर॑ता शमयति शमय त्यभि॒चर॑ता प्रतिलो॒मम् प्र॑तिलो॒म म॑भि॒चर॑ता शमयति शमय त्यभि॒चर॑ता प्रतिलो॒मम् । \newline
4. अ॒भि॒चर॑ता प्रतिलो॒मम् प्र॑तिलो॒म म॑भि॒चर॑ता ऽभि॒चर॑ता प्रतिलो॒मꣳ हो॑त॒व्या॑ होत॒व्याः᳚ प्रतिलो॒म म॑भि॒चर॑ता ऽभि॒चर॑ता प्रतिलो॒मꣳ हो॑त॒व्याः᳚ । \newline
5. अ॒भि॒चर॒तेत्य॑भि - चर॑ता । \newline
6. प्र॒ति॒लो॒मꣳ हो॑त॒व्या॑ होत॒व्याः᳚ प्रतिलो॒मम् प्र॑तिलो॒मꣳ हो॑त॒व्याः᳚ प्रा॒णान् प्रा॒णान्. हो॑त॒व्याः᳚ प्रतिलो॒मम् प्र॑तिलो॒मꣳ हो॑त॒व्याः᳚ प्रा॒णान् । \newline
7. प्र॒ति॒लो॒ममिति॑ प्रति - लो॒मम् । \newline
8. हो॒त॒व्याः᳚ प्रा॒णान् प्रा॒णान्. हो॑त॒व्या॑ होत॒व्याः᳚ प्रा॒णा,ने॒वैव प्रा॒णान्. हो॑त॒व्या॑ होत॒व्याः᳚ प्रा॒णा,ने॒व । \newline
9. प्रा॒णा,ने॒वैव प्रा॒णान् प्रा॒णा,ने॒वास्या᳚ स्यै॒व प्रा॒णान् प्रा॒णा,ने॒वास्य॑ । \newline
10. प्रा॒णानिति॑ प्र - अ॒नान् । \newline
11. ए॒वास्या᳚ स्यै॒वैवास्य॑ प्र॒तीचः॑ प्र॒तीचो᳚ ऽस्यै॒वैवास्य॑ प्र॒तीचः॑ । \newline
12. अ॒स्य॒ प्र॒तीचः॑ प्र॒तीचो᳚ ऽस्यास्य प्र॒तीचः॒ प्रति॒ प्रति॑ प्र॒तीचो᳚ ऽस्यास्य प्र॒तीचः॒ प्रति॑ । \newline
13. प्र॒तीचः॒ प्रति॒ प्रति॑ प्र॒तीचः॑ प्र॒तीचः॒ प्रति॑ यौति यौति॒ प्रति॑ प्र॒तीचः॑ प्र॒तीचः॒ प्रति॑ यौति । \newline
14. प्रति॑ यौति यौति॒ प्रति॒ प्रति॑ यौति॒ तम् तम् ॅयौ॑ति॒ प्रति॒ प्रति॑ यौति॒ तम् । \newline
15. यौ॒ति॒ तम् तम् ॅयौ॑ति यौति॒ तम् तत॒ स्तत॒ स्तम् ॅयौ॑ति यौति॒ तम् ततः॑ । \newline
16. तम् तत॒ स्तत॒ स्तम् तम् ततो॒ येन॒ येन॒ तत॒ स्तम् तम् ततो॒ येन॑ । \newline
17. ततो॒ येन॒ येन॒ तत॒ स्ततो॒ येन॒ केन॒ केन॒ येन॒ तत॒ स्ततो॒ येन॒ केन॑ । \newline
18. येन॒ केन॒ केन॒ येन॒ येन॒ केन॑ च च॒ केन॒ येन॒ येन॒ केन॑ च । \newline
19. केन॑ च च॒ केन॒ केन॑ च स्तृणुते स्तृणुते च॒ केन॒ केन॑ च स्तृणुते । \newline
20. च॒ स्तृ॒णु॒ते॒ स्तृ॒णु॒ते॒ च॒ च॒ स्तृ॒णु॒ते॒ स्वकृ॑ते॒ स्वकृ॑ते स्तृणुते च च स्तृणुते॒ स्वकृ॑ते । \newline
21. स्तृ॒णु॒ते॒ स्वकृ॑ते॒ स्वकृ॑ते स्तृणुते स्तृणुते॒ स्वकृ॑त॒ इरि॑ण॒ इरि॑णे॒ स्वकृ॑ते स्तृणुते स्तृणुते॒ स्वकृ॑त॒ इरि॑णे । \newline
22. स्वकृ॑त॒ इरि॑ण॒ इरि॑णे॒ स्वकृ॑ते॒ स्वकृ॑त॒ इरि॑णे जुहोति जुहो॒ती रि॑णे॒ स्वकृ॑ते॒ स्वकृ॑त॒ इरि॑णे जुहोति । \newline
23. स्वकृ॑त॒ इति॒ स्व - कृ॒ते॒ । \newline
24. इरि॑णे जुहोति जुहो॒तीरि॑ण॒ इरि॑णे जुहोति प्रद॒रे प्र॑द॒रे जु॑हो॒तीरि॑ण॒ इरि॑णे जुहोति प्रद॒रे । \newline
25. जु॒हो॒ति॒ प्र॒द॒रे प्र॑द॒रे जु॑होति जुहोति प्रद॒रे वा॑ वा प्रद॒रे जु॑होति जुहोति प्रद॒रे वा᳚ । \newline
26. प्र॒द॒रे वा॑ वा प्रद॒रे प्र॑द॒रे वै॒त दे॒तद् वा᳚ प्रद॒रे प्र॑द॒रे वै॒तत् । \newline
27. प्र॒द॒र इति॑ प्र - द॒रे । \newline
28. वै॒त दे॒तद् वा॑ वै॒तद् वै वा ए॒तद् वा॑ वै॒तद् वै । \newline
29. ए॒तद् वै वा ए॒त दे॒तद् वा अ॒स्या अ॒स्यै वा ए॒त दे॒तद् वा अ॒स्यै । \newline
30. वा अ॒स्या अ॒स्यै वै वा अ॒स्यै निर्.ऋ॑तिगृहीत॒म् निर्.ऋ॑तिगृहीत म॒स्यै वै वा अ॒स्यै निर्.ऋ॑तिगृहीतम् । \newline
31. अ॒स्यै निर्.ऋ॑तिगृहीत॒म् निर्.ऋ॑तिगृहीत म॒स्या अ॒स्यै निर्.ऋ॑तिगृहीत॒म् निर्.ऋ॑तिगृहीते॒ निर्.ऋ॑तिगृहीते॒ निर्.ऋ॑तिगृहीत म॒स्या अ॒स्यै निर्.ऋ॑तिगृहीत॒म् निर्.ऋ॑तिगृहीते । \newline
32. निर्.ऋ॑तिगृहीत॒म्, निर्.ऋ॑तिगृहीते॒ निर्.ऋ॑तिगृहीते॒ निर्.ऋ॑तिगृहीत॒म्, निर्.ऋ॑तिगृहीत॒म्, निर्.ऋ॑तिगृहीत ए॒वैव निर्.ऋ॑तिगृहीते॒ निर्.ऋ॑तिगृहीत॒म्, निर्.ऋ॑तिगृहीत॒म्, निर्.ऋ॑तिगृहीत ए॒व । \newline
33. निर्.ऋ॑तिगृहीत॒मिति॒ निर्.ऋ॑ति - गृ॒ही॒त॒म् । \newline
34. निर्.ऋ॑तिगृहीत ए॒वैव निर्.ऋ॑तिगृहीते॒ निर्.ऋ॑तिगृहीत ए॒वैन॑ मेन मे॒व निर्.ऋ॑तिगृहीते॒ निर्.ऋ॑तिगृहीत ए॒वैन᳚म् । \newline
35. निर्.ऋ॑तिगृहीत॒ इति॒ निर्.ऋ॑ति - गृ॒ही॒ते॒ । \newline
36. ए॒वै न॑ मेन मे॒वैवैन॒म्,निर्.ऋ॑त्या॒ निर्.ऋ॑ त्यैन मे॒वैवैन॒म्,निर्.ऋ॑त्या । \newline
37. ए॒न॒म्,निर्.ऋ॑त्या॒ निर्.ऋ॑त्यैन मेन॒म् निर्.ऋ॑त्या ग्राहयति ग्राहयति॒ निर्.ऋ॑त्यैन मेन॒म् निर्.ऋ॑त्या ग्राहयति । \newline
38. निर्.ऋ॑त्या ग्राहयति ग्राहयति॒ निर्.ऋ॑त्या॒ निर्.ऋ॑त्या ग्राहयति॒ यद् यद् ग्रा॑हयति॒ निर्.ऋ॑त्या॒ निर्.ऋ॑त्या ग्राहयति॒ यत् । \newline
39. निर्.ऋ॒त्येति॒ निः - ऋ॒त्या॒ । \newline
40. ग्रा॒ह॒य॒ति॒ यद् यद् ग्रा॑हयति ग्राहयति॒ यद् वा॒चो वा॒चो यद् ग्रा॑हयति ग्राहयति॒ यद् वा॒चः । \newline
41. यद् वा॒चो वा॒चो यद् यद् वा॒चः क्रू॒रम् क्रू॒रम् ॅवा॒चो यद् यद् वा॒चः क्रू॒रम् । \newline
42. वा॒चः क्रू॒रम् क्रू॒रम् ॅवा॒चो वा॒चः क्रू॒रम् तेन॒ तेन॑ क्रू॒रम् ॅवा॒चो वा॒चः क्रू॒रम् तेन॑ । \newline
43. क्रू॒रम् तेन॒ तेन॑ क्रू॒रम् क्रू॒रम् तेन॒ वष॒ड् वष॒ट् तेन॑ क्रू॒रम् क्रू॒रम् तेन॒ वष॑ट् । \newline
44. तेन॒ वष॒ड् वष॒ट् तेन॒ तेन॒ वष॑ट् करोति करोति॒ वष॒ट् तेन॒ तेन॒ वष॑ट् करोति । \newline
45. वष॑ट् करोति करोति॒ वष॒ड् वष॑ट् करोति वा॒चो वा॒चः क॑रोति॒ वष॒ड् वष॑ट् करोति वा॒चः । \newline
46. क॒रो॒ति॒ वा॒चो वा॒चः क॑रोति करोति वा॒च ए॒वैव वा॒चः क॑रोति करोति वा॒च ए॒व । \newline
47. वा॒च ए॒वैव वा॒चो वा॒च ए॒वैन॑ मेन मे॒व वा॒चो वा॒च ए॒वैन᳚म् । \newline
48. ए॒वैन॑ मेन मे॒वैवैन॑म् क्रू॒रेण॑ क्रू॒रेणै॑न मे॒वैवैन॑म् क्रू॒रेण॑ । \newline
49. ए॒न॒म् क्रू॒रेण॑ क्रू॒रे णै॑न मेनम् क्रू॒रेण॒ प्र प्र क्रू॒रेणै॑न मेनम् क्रू॒रेण॒ प्र । \newline
50. क्रू॒रेण॒ प्र प्र क्रू॒रेण॑ क्रू॒रेण॒ प्र वृ॑श्चति वृश्चति॒ प्र क्रू॒रेण॑ क्रू॒रेण॒ प्र वृ॑श्चति । \newline
51. प्र वृ॑श्चति वृश्चति॒ प्र प्र वृ॑श्चति ता॒जक् ता॒जग् वृ॑श्चति॒ प्र प्र वृ॑श्चति ता॒जक् । \newline
52. वृ॒श्च॒ति॒ ता॒जक् ता॒जग् वृ॑श्चति वृश्चति ता॒ज गार्ति॒ मार्ति॑म् ता॒जग् वृ॑श्चति वृश्चति ता॒ज गार्ति᳚म् । \newline
53. ता॒जगार्ति॒ मार्ति॑म् ता॒जक् ता॒जगार्ति॒ माऽऽर्ति॑म् ता॒जक् ता॒जगार्ति॒मा । \newline
54. आर्ति॒ मा ऽऽर्ति॒ मार्ति॒ मार्च्छ॑ त्यृच्छ॒त्या ऽऽर्ति॒ मार्ति॒ मार्च्छ॑ति । \newline
55. आर्च्छ॑ त्यृच्छ॒त्या र्च्छ॑ति॒ यस्य॒ यस्य॑ र्च्छ॒ त्यार्च्छ॑ति॒ यस्य॑ । \newline
56. ऋ॒च्छ॒ति॒ यस्य॒ यस्य॑ र्‌च्छ त्यृच्छति॒ यस्य॑ का॒मये॑त का॒मये॑त॒ यस्य॑ र्‌च्छ त्यृच्छति॒ यस्य॑ का॒मये॑त । \newline
57. यस्य॑ का॒मये॑त का॒मये॑त॒ यस्य॒ यस्य॑ का॒मये॑ता॒ न्नाद्य॑ म॒न्नाद्य॑म् का॒मये॑त॒ यस्य॒ यस्य॑ का॒मये॑ता॒ न्नाद्य᳚म् । \newline
58. का॒मये॑ता॒ न्नाद्य॑ म॒न्नाद्य॑म् का॒मये॑त का॒मये॑ता॒ न्नाद्य॒मा ऽन्नाद्य॑म् का॒मये॑त का॒मये॑ता॒ न्नाद्य॒ मा । \newline
59. अ॒न्नाद्य॒मा ऽन्नाद्य॑ म॒न्नाद्य॒मा द॑दीय ददी॒या ऽन्नाद्य॑ म॒न्नाद्य॒मा द॑दीय । \newline
60. अ॒न्नाद्य॒मित्य॑न्न - अद्य᳚म् । \newline
\pagebreak
\markright{ TS 3.4.8.6  \hfill https://www.vedavms.in \hfill}

\section{ TS 3.4.8.6 }

\textbf{TS 3.4.8.6 } \newline
\textbf{Samhita Paata} \newline

मा द॑दी॒येति॒ तस्य॑ स॒भाया॑मुत्ता॒नो नि॒पद्य॒ भुव॑नस्य पत॒ इति॒ तृणा॑नि॒ सं गृ॑ह्णीयात् प्र॒जाप॑ति॒र्वै भुव॑नस्य॒ पतिः॑ प्र॒जाप॑तिनै॒वास्या॒न्नाद्य॒मा द॑त्त इ॒दम॒हम॒मुष्या॑ ऽऽ*मुष्याय॒णस्या॒न्नाद्यꣳ॑ हरा॒मीत्या॑हा॒न्नाद्य॑मे॒वास्य॑ हरति ष॒ड्भिर्.ह॑रति॒ षड्वा ऋ॒तवः॑ प्र॒जाप॑तिनै॒वास्या॒-न्नाद्य॑मा॒दाय॒र्तवो᳚ ऽस्मा॒ अनु॒ प्रय॑च्छन्ति॒ - [  ] \newline

\textbf{Pada Paata} \newline

एति॑ । द॒दी॒य॒ । इति॑ । तस्य॑ । स॒भाया᳚म् । उ॒त्ता॒न इत्यु॑त् - ता॒नः । नि॒पद्येति॑ नि - पद्य॑ । भुव॑नस्य । प॒ते॒ । इति॑ । तृणा॑नि । समिति॑ । गृ॒ह्णी॒या॒त् । प्र॒जाप॑ति॒रिति॑ प्र॒जा - प॒तिः॒ । वै । भुव॑नस्य । पतिः॑ । प्र॒जाप॑ति॒नेति॑ प्र॒जा - प॒ति॒ना॒ । ए॒व । अ॒स्य॒ । अ॒न्नाद्य॒मित्य॑न्न - अद्य᳚म् । एति॑ । द॒त्ते॒ । इ॒दम् । अ॒हम् । अ॒मुष्य॑ । आ॒मु॒ष्या॒य॒णस्य॑ । अ॒न्नाद्य॒मित्य॑न्न - अद्य᳚म् । ह॒रा॒मि॒ । इति॑ । आ॒ह॒ । अ॒न्नाद्य॒मित्य॑न्न - अद्य᳚म् । ए॒व । अ॒स्य॒ । ह॒र॒ति॒ । ष॒ड्भिरिति॑ षट् - भिः । ह॒र॒ति॒ । षट् । वै । ऋ॒तवः॑ । प्र॒जाप॑ति॒नेति॑ प्र॒जा-प॒ति॒ना॒ । ए॒व । अ॒स्य॒ । अ॒न्नाद्य॒मित्य॑न्न -अद्य᳚म् । आ॒दायेत्या᳚-दाय॑ । ऋ॒तवः॑ । अ॒स्मै॒ । अनु॑ । प्रेति॑ । य॒च्छ॒न्ति॒ ।  \newline


\textbf{Krama Paata} \newline

आ द॑दीय । द॒दी॒येति॑ । इति॒ तस्य॑ । तस्य॑ स॒भाया᳚म् । स॒भाया॑मुत्ता॒नः । उ॒त्ता॒नो नि॒पद्य॑ । उ॒त्ता॒न इत्यु॑त् - ता॒नः । नि॒पद्य॒ भुव॑नस्य । नि॒पद्येति॑ नि - पद्य॑ । भुव॑नस्य पते । प॒त॒ इति॑ । इति॒ तृणा॑नि । तृणा॑नि॒ सम् । सम् गृ॑ह्णीयात् । गृ॒ह्णी॒या॒त् प्र॒जाप॑तिः । प्र॒जाप॑ति॒र् वै । प्र॒जाप॑ति॒रिति॑ प्र॒जा - प॒तिः॒ । वै भुव॑नस्य । भुव॑नस्य॒ पतिः॑ । पतिः॑ प्र॒जाप॑तिना । प्र॒जाप॑तिनै॒व । प्र॒जाप॑ति॒नेति॑ प्र॒जा - प॒ति॒ना॒ । ए॒वास्य॑ । अ॒स्या॒न्नाद्य᳚म् । अ॒न्नाद्य॒मा । अ॒न्नाद्य॒मित्य॑न्न - अद्य᳚म् । आ द॑त्ते । द॒त्त॒ इ॒दम् । इ॒दम॒हम् । अ॒हम॒मुष्य॑ । अ॒मुष्या॑मुष्याय॒णस्य॑ । आ॒मु॒ष्या॒य॒णस्या॒न्नाद्य᳚म् । अ॒न्नाद्यꣳ॑ हरामि । अ॒न्नाद्य॒मित्य॑न्न - अद्य᳚म् । ह॒रा॒मीति॑ । इत्या॑ह । आ॒हा॒न्नाद्य᳚म् । अ॒न्नाद्य॑मे॒व । अ॒न्नाद्य॒मित्य॑न्न - अद्य᳚म् । ए॒वास्य॑ । अ॒स्य॒ ह॒र॒ति॒ । ह॒र॒ति॒ ष॒ड्भिः । ष॒ड्भिर्. ह॑रति । ष॒ड्भिरिति॑ षट् - भिः । ह॒र॒ति॒ षट् । षड् वै । वा ऋ॒तवः॑ । ऋ॒तवः॑ प्र॒जाप॑तिना । प्र॒जाप॑तिनै॒व । प्र॒जाप॑ति॒नेति॑ प्र॒जा - प॒ति॒ना॒ । ए॒वास्य॑ । अ॒स्या॒न्नाद्य᳚म् । अ॒न्नाद्य॑मा॒दाय॑ । अ॒न्नाद्य॒मित्य॑न्न - अद्य᳚म् । आ॒दाय॒र्तवः॑ । आ॒दायेत्या᳚ - दाय॑ । ऋ॒तवो᳚ ऽस्मै । अ॒स्मा॒ अनु॑ । अनु॒ प्र । प्र य॑च्छन्ति । य॒च्छ॒न्ति॒ यः ( ) \newline

\textbf{Jatai Paata} \newline

1. आ द॑दीय ददी॒या द॑दीय । \newline
2. द॒दी॒येतीति॑ ददीय ददी॒येति॑ । \newline
3. इति॒ तस्य॒ तस्ये तीति॒ तस्य॑ । \newline
4. तस्य॑ स॒भायाꣳ॑ स॒भाया॒म् तस्य॒ तस्य॑ स॒भाया᳚म् । \newline
5. स॒भाया॑ मुत्ता॒न उ॑त्ता॒नः स॒भायाꣳ॑ स॒भाया॑ मुत्ता॒नः । \newline
6. उ॒त्ता॒नो नि॒पद्य॑ नि॒पद्यो᳚ त्ता॒न उ॑त्ता॒नो नि॒पद्य॑ । \newline
7. उ॒त्ता॒न इत्यु॑त् - ता॒नः । \newline
8. नि॒पद्य॒ भुव॑नस्य॒ भुव॑नस्य नि॒पद्य॑ नि॒पद्य॒ भुव॑नस्य । \newline
9. नि॒पद्येति॑ नि - पद्य॑ । \newline
10. भुव॑नस्य पते पते॒ भुव॑नस्य॒ भुव॑नस्य पते । \newline
11. प॒त॒ इतीति॑ पते पत॒ इति॑ । \newline
12. इति॒ तृणा॑नि॒ तृणा॒नी तीति॒ तृणा॑नि । \newline
13. तृणा॑नि॒ सꣳ सम् तृणा॑नि॒ तृणा॑नि॒ सम् । \newline
14. सम् गृ॑ह्णीयाद् गृह्णीया॒थ् सꣳ सम् गृ॑ह्णीयात् । \newline
15. गृ॒ह्णी॒या॒त् प्र॒जाप॑तिः प्र॒जाप॑तिर् गृह्णीयाद् गृह्णीयात् प्र॒जाप॑तिः । \newline
16. प्र॒जाप॑ति॒र् वै वै प्र॒जाप॑तिः प्र॒जाप॑ति॒र् वै । \newline
17. प्र॒जाप॑ति॒रिति॑ प्र॒जा - प॒तिः॒ । \newline
18. वै भुव॑नस्य॒ भुव॑नस्य॒ वै वै भुव॑नस्य । \newline
19. भुव॑नस्य॒ पति॒ष् पति॒र् भुव॑नस्य॒ भुव॑नस्य॒ पतिः॑ । \newline
20. पतिः॑ प्र॒जाप॑तिना प्र॒जाप॑तिना॒ पति॒ष् पतिः॑ प्र॒जाप॑तिना । \newline
21. प्र॒जाप॑तिनै॒वैव प्र॒जाप॑तिना प्र॒जाप॑तिनै॒व । \newline
22. प्र॒जाप॑ति॒नेति॑ प्र॒जा - प॒ति॒ना॒ । \newline
23. ए॒वा स्या᳚ स्यै॒वैवास्य॑ । \newline
24. अ॒स्या॒न्नाद्य॑ म॒न्नाद्य॑ मस्या स्या॒-न्नाद्य᳚म् । \newline
25. अ॒न्नाद्य॒ मा ऽन्नाद्य॑ म॒न्नाद्य॒ मा । \newline
26. अ॒न्नाद्य॒मित्य॑न्न - अद्य᳚म् । \newline
27. आ द॑त्ते दत्त॒ आ द॑त्ते । \newline
28. द॒त्त॒ इ॒द मि॒दम् द॑त्ते दत्त इ॒दम् । \newline
29. इ॒द म॒ह म॒ह मि॒द मि॒द म॒हम् । \newline
30. अ॒ह म॒मुष्या॒ मुष्या॒ह म॒ह म॒मुष्य॑ । \newline
31. अ॒मुष्या॑ मुष्याय॒णस्या॑ मुष्याय॒णस्या॒ मुष्या॒ मुष्या॑ मुष्याय॒णस्य॑ । \newline
32. आ॒मु॒ष्या॒य॒णस्या॒ न्नाद्य॑ म॒न्नाद्य॑ मामुष्याय॒णस्या॑ मुष्याय॒णस्या॒-न्नाद्य᳚म् । \newline
33. अ॒न्नाद्यꣳ॑ हरामि हरा म्य॒-न्नाद्य॑ म॒न्नाद्यꣳ॑ हरामि । \newline
34. अ॒न्नाद्य॒मित्य॑न्न - अद्य᳚म् । \newline
35. ह॒रा॒मीतीति॑ हरामि हरा॒मीति॑ । \newline
36. इत्या॑ हा॒हे तीत्या॑ह । \newline
37. आ॒हा॒ न्नाद्य॑ म॒न्नाद्य॑ माहाहा॒ न्नाद्य᳚म् । \newline
38. अ॒न्नाद्य॑ मे॒वै वान्नाद्य॑ म॒न्नाद्य॑ मे॒व । \newline
39. अ॒न्नाद्य॒मित्य॑न्न - अद्य᳚म् । \newline
40. ए॒वास्या᳚ स्यै॒वै वास्य॑ । \newline
41. अ॒स्य॒ ह॒र॒ति॒ ह॒र॒ त्य॒स्या॒स्य॒ ह॒र॒ति॒ । \newline
42. ह॒र॒ति॒ ष॒ड्भि ष्ष॒ड्भिर्. ह॑रति हरति ष॒ड्भिः । \newline
43. ष॒ड्भिर्. ह॑रति हरति ष॒ड्भि ष्ष॒ड्भिर्. ह॑रति । \newline
44. ष॒ड्भिरिति॑ षट् - भिः । \newline
45. ह॒र॒ति॒ षट्थ् षड्ढ॑रति हरति॒ षट् । \newline
46. षड् वै वै षट्थ् षड् वै । \newline
47. वा ऋ॒तव॑ ऋ॒तवो॒ वै वा ऋ॒तवः॑ । \newline
48. ऋ॒तवः॑ प्र॒जाप॑तिना प्र॒जाप॑तिन॒ र्‌तव॑ ऋ॒तवः॑ प्र॒जाप॑तिना । \newline
49. प्र॒जाप॑तिनै॒वैव प्र॒जाप॑तिना प्र॒जाप॑तिनै॒व । \newline
50. प्र॒जाप॑ति॒नेति॑ प्र॒जा - प॒ति॒ना॒ । \newline
51. ए॒वा स्या᳚ स्यै॒वै वास्य॑ । \newline
52. अ॒स्या॒ न्नाद्य॑ म॒न्नाद्य॑ मस्या स्या॒ न्नाद्य᳚म् । \newline
53. अ॒न्नाद्य॑ मा॒दाया॒ दाया॒ न्नाद्य॑ म॒न्नाद्य॑ मा॒दाय॑ । \newline
54. अ॒न्नाद्य॒मित्य॑न्न - अद्य᳚म् । \newline
55. आ॒दाय॒ र्‌तव॑ ऋ॒तव॑ आ॒दाया॒ दाय॒ र्‌तवः॑ । \newline
56. आ॒दायेत्या᳚ - दाय॑ । \newline
57. ऋ॒तवो᳚ ऽस्मा अस्मा ऋ॒तव॑ ऋ॒तवो᳚ ऽस्मै । \newline
58. अ॒स्मा॒ अन्वन्व॑स्मा अस्मा॒ अनु॑ । \newline
59. अनु॒ प्र प्राण्वनु॒ प्र । \newline
60. प्र य॑च्छन्ति यच्छन्ति॒ प्र प्र य॑च्छन्ति । \newline
61. य॒च्छ॒न्ति॒ यो यो य॑च्छन्ति यच्छन्ति॒ यः । \newline

\textbf{Ghana Paata } \newline

1. आ द॑दीय ददी॒या द॑दी॒येतीति॑ ददी॒या द॑दी॒येति॑ । \newline
2. द॒दी॒येतीति॑ ददीय ददी॒येति॒ तस्य॒ तस्येति॑ ददीय ददी॒येति॒ तस्य॑ । \newline
3. इति॒ तस्य॒ तस्येतीति॒ तस्य॑ स॒भायाꣳ॑ स॒भाया॒म् तस्येतीति॒ तस्य॑ स॒भाया᳚म् । \newline
4. तस्य॑ स॒भायाꣳ॑ स॒भाया॒म् तस्य॒ तस्य॑ स॒भाया॑ मुत्ता॒न उ॑त्ता॒नः स॒भाया॒म् तस्य॒ तस्य॑ स॒भाया॑ मुत्ता॒नः । \newline
5. स॒भाया॑ मुत्ता॒न उ॑त्ता॒नः स॒भायाꣳ॑ स॒भाया॑ मुत्ता॒नो नि॒पद्य॑ नि॒पद्यो᳚त्ता॒नः स॒भायाꣳ॑ स॒भाया॑ मुत्ता॒नो नि॒पद्य॑ । \newline
6. उ॒त्ता॒नो नि॒पद्य॑ नि॒पद्यो᳚ त्ता॒न उ॑त्ता॒नो नि॒पद्य॒ भुव॑नस्य॒ भुव॑नस्य नि॒पद्यो᳚ त्ता॒न उ॑त्ता॒नो नि॒पद्य॒ भुव॑नस्य । \newline
7. उ॒त्ता॒न इत्यु॑त् - ता॒नः । \newline
8. नि॒पद्य॒ भुव॑नस्य॒ भुव॑नस्य नि॒पद्य॑ नि॒पद्य॒ भुव॑नस्य पते पते॒ भुव॑नस्य नि॒पद्य॑ नि॒पद्य॒ भुव॑नस्य पते । \newline
9. नि॒पद्येति॑ नि - पद्य॑ । \newline
10. भुव॑नस्य पते पते॒ भुव॑नस्य॒ भुव॑नस्य पत॒ इतीति॑ पते॒ भुव॑नस्य॒ भुव॑नस्य पत॒ इति॑ । \newline
11. प॒त॒ इतीति॑ पते पत॒ इति॒ तृणा॑नि॒ तृणा॒नीति॑ पते पत॒ इति॒ तृणा॑नि । \newline
12. इति॒ तृणा॑नि॒ तृणा॒नीतीति॒ तृणा॑नि॒ सꣳ सम् तृणा॒नीतीति॒ तृणा॑नि॒ सम् । \newline
13. तृणा॑नि॒ सꣳ सम् तृणा॑नि॒ तृणा॑नि॒ सम् गृ॑ह्णीयाद् गृह्णीया॒थ् सम् तृणा॑नि॒ तृणा॑नि॒ सम् गृ॑ह्णीयात् । \newline
14. सम् गृ॑ह्णीयाद् गृह्णीया॒थ् सꣳ सम् गृ॑ह्णीयात् प्र॒जाप॑तिः प्र॒जाप॑तिर् गृह्णीया॒थ् सꣳ सम् गृ॑ह्णीयात् प्र॒जाप॑तिः । \newline
15. गृ॒ह्णी॒या॒त् प्र॒जाप॑तिः प्र॒जाप॑तिर् गृह्णीयाद् गृह्णीयात् प्र॒जाप॑ति॒र् वै वै प्र॒जाप॑तिर् गृह्णीयाद् गृह्णीयात् प्र॒जाप॑ति॒र् वै । \newline
16. प्र॒जाप॑ति॒र् वै वै प्र॒जाप॑तिः प्र॒जाप॑ति॒र् वै भुव॑नस्य॒ भुव॑नस्य॒ वै प्र॒जाप॑तिः प्र॒जाप॑ति॒र् वै भुव॑नस्य । \newline
17. प्र॒जाप॑ति॒रिति॑ प्र॒जा - प॒तिः॒ । \newline
18. वै भुव॑नस्य॒ भुव॑नस्य॒ वै वै भुव॑नस्य॒ पति॒ष् पति॒र् भुव॑नस्य॒ वै वै भुव॑नस्य॒ पतिः॑ । \newline
19. भुव॑नस्य॒ पति॒ष् पति॒र् भुव॑नस्य॒ भुव॑नस्य॒ पतिः॑ प्र॒जाप॑तिना प्र॒जाप॑तिना॒ पति॒र् भुव॑नस्य॒ भुव॑नस्य॒ पतिः॑ प्र॒जाप॑तिना । \newline
20. पतिः॑ प्र॒जाप॑तिना प्र॒जाप॑तिना॒ पति॒ष् पतिः॑ प्र॒जाप॑ति नै॒वैव प्र॒जाप॑तिना॒ पति॒ष् पतिः॑ प्र॒जाप॑ति नै॒व । \newline
21. प्र॒जाप॑ति नै॒वैव प्र॒जाप॑तिना प्र॒जाप॑ति नै॒वास्या᳚स्यै॒व प्र॒जाप॑तिना प्र॒जाप॑ति नै॒वास्य॑ । \newline
22. प्र॒जाप॑ति॒नेति॑ प्र॒जा - प॒ति॒ना॒ । \newline
23. ए॒वास्या᳚ स्यै॒वैवास्या॒ न्नाद्य॑ म॒न्नाद्य॑ मस्यै॒ वैवास्या॒न्नाद्य᳚म् । \newline
24. अ॒स्या॒न्नाद्य॑ म॒न्नाद्य॑ मस्यास्या॒ न्नाद्य॒मा ऽन्नाद्य॑ मस्यास्या॒ न्नाद्य॒मा । \newline
25. अ॒न्नाद्य॒मा ऽन्नाद्य॑ म॒न्नाद्य॒मा द॑त्ते दत्त॒ आ ऽन्नाद्य॑ म॒न्नाद्य॒मा द॑त्ते । \newline
26. अ॒न्नाद्य॒मित्य॑न्न - अद्य᳚म् । \newline
27. आ द॑त्ते दत्त॒ आ द॑त्त इ॒द मि॒दम् द॑त्त॒ आ द॑त्त इ॒दम् । \newline
28. द॒त्त॒ इ॒द मि॒दम् द॑त्ते दत्त इ॒द म॒ह म॒ह मि॒दम् द॑त्ते दत्त इ॒द म॒हम् । \newline
29. इ॒द म॒ह म॒ह मि॒द मि॒द म॒ह म॒मुष्या॒ मुष्या॒ह मि॒द मि॒द म॒ह म॒मुष्य॑ । \newline
30. अ॒ह म॒मुष्या॒ मुष्या॒ह म॒ह म॒मुष्या॑ मुष्याय॒णस्या॑ मुष्याय॒णस्या॒ मुष्या॒ह म॒ह म॒मुष्या॑ मुष्याय॒णस्य॑ । \newline
31. अ॒मुष्या॑ मुष्याय॒णस्या॑ मुष्याय॒णस्या॒ मुष्या॒ मुष्या॑ मुष्याय॒णस्या॒ न्नाद्य॑ म॒न्नाद्य॑ मामुष्याय॒णस्या॒ मुष्या॒ मुष्या॑ मुष्याय॒णस्या॒ न्नाद्य᳚म् । \newline
32. आ॒मु॒ष्या॒य॒णस्या॒ न्नाद्य॑ म॒न्नाद्य॑ मामुष्याय॒णस्या॑ मुष्याय॒णस्या॒ न्नाद्यꣳ॑ हरामि हरा
म्य॒न्नाद्य॑ मामुष्याय॒णस्या॑ मुष्याय॒णस्या॒ न्नाद्यꣳ॑ हरामि । \newline
33. अ॒न्नाद्यꣳ॑ हरामि हरा म्य॒न्नाद्य॑ म॒न्नाद्यꣳ॑ हरा॒मीतीति॑ हरा म्य॒न्नाद्य॑ म॒न्नाद्यꣳ॑ हरा॒मीति॑ । \newline
34. अ॒न्नाद्य॒मित्य॑न्न - अद्य᳚म् । \newline
35. ह॒रा॒मीतीति॑ हरामि हरा॒मी त्या॑हा॒हेति॑ हरामि हरा॒मीत्या॑ह । \newline
36. इत्या॑ हा॒हेतीत्या॑हा॒,न्नाद्य॑ म॒न्नाद्य॑ मा॒हेतीत्या॑हा॒,न्नाद्य᳚म् । \newline
37. आ॒हा॒न्नाद्य॑ म॒न्नाद्य॑ माहाहा॒,न्नाद्य॑ मे॒वैवान्नाद्य॑ माहाहा॒,न्नाद्य॑ मे॒व । \newline
38. अ॒न्नाद्य॑ मे॒वैवान्नाद्य॑ म॒न्नाद्य॑ मे॒वास्या᳚ स्यै॒वान्नाद्य॑ म॒न्नाद्य॑ मे॒वास्य॑ । \newline
39. अ॒न्नाद्य॒मित्य॑न्न - अद्य᳚म् । \newline
40. ए॒वास्या᳚ स्यै॒वैवास्य॑ हरति हर त्यस्यै॒वैवास्य॑ हरति । \newline
41. अ॒स्य॒ ह॒र॒ति॒ ह॒र॒ त्य॒स्या॒स्य॒ ह॒र॒ति॒ ष॒ड्भि ष्ष॒ड्भिर्. ह॑र त्यस्यास्य हरति ष॒ड्भिः । \newline
42. ह॒र॒ति॒ ष॒ड्भिष् ष॒ड्भिर्. ह॑रति हरति ष॒ड्भिर्. ह॑रति हरति ष॒ड्भिर्. ह॑रति हरति ष॒ड्भिर्. ह॑रति । \newline
43. ष॒ड्भिर्. ह॑रति हरति ष॒ड्भिष् ष॒ड्भिर्. ह॑रति॒ षट् थ्षड्ढ॑रति ष॒ड्भिष् ष॒ड्भिर्. ह॑रति॒ षट् । \newline
44. ष॒ड्भिरिति॑ षट् - भिः । \newline
45. ह॒र॒ति॒ षट्थ् षड्ढ॑रति हरति॒ षड् वै वै षड्ढ॑रति हरति॒ षड् वै । \newline
46. षड् वै वै षट्थ् षड् वा ऋ॒तव॑ ऋ॒तवो॒ वै षट्थ् षड् वा ऋ॒तवः॑ । \newline
47. वा ऋ॒तव॑ ऋ॒तवो॒ वै वा ऋ॒तवः॑ प्र॒जाप॑तिना प्र॒जाप॑तिन॒ र्‌तवो॒ वै वा ऋ॒तवः॑ प्र॒जाप॑तिना । \newline
48. ऋ॒तवः॑ प्र॒जाप॑तिना प्र॒जाप॑तिन॒ र्‌तव॑ ऋ॒तवः॑ प्र॒जाप॑ति नै॒वैव प्र॒जाप॑तिन॒ र्‌तव॑ ऋ॒तवः॑ प्र॒जाप॑ति नै॒व । \newline
49. प्र॒जाप॑ति नै॒वैव प्र॒जाप॑तिना प्र॒जाप॑ति नै॒वास्या᳚ स्यै॒व प्र॒जाप॑तिना प्र॒जाप॑ति नै॒वास्य॑ । \newline
50. प्र॒जाप॑ति॒नेति॑ प्र॒जा - प॒ति॒ना॒ । \newline
51. ए॒वास्या᳚ स्यै॒वैवास्या॒ न्नाद्य॑ म॒न्नाद्य॑ मस्यै॒ वैवास्या॒,न्नाद्य᳚म् । \newline
52. अ॒स्या॒ न्नाद्य॑ म॒न्नाद्य॑ मस्यास्या॒ न्नाद्य॑ मा॒दाया॒ दाया॒ न्नाद्य॑ मस्यास्या॒ न्नाद्य॑ मा॒दाय॑ । \newline
53. अ॒न्नाद्य॑ मा॒दाया॒ दाया॒ न्नाद्य॑ म॒न्नाद्य॑ मा॒दाय॒ र्‌तव॑ ऋ॒तव॑ आ॒दाया॒ न्नाद्य॑ म॒न्नाद्य॑ मा॒दाय॒ र्‌तवः॑ । \newline
54. अ॒न्नाद्य॒मित्य॑न्न - अद्य᳚म् । \newline
55. आ॒दाय॒ र्‌तव॑ ऋ॒तव॑ आ॒दाया॒ दाय॒ र्‌तवो᳚ ऽस्मा अस्मा ऋ॒तव॑ आ॒दाया॒ दाय॒ र्‌तवो᳚ ऽस्मै । \newline
56. आ॒दायेत्या᳚ - दाय॑ । \newline
57. ऋ॒तवो᳚ ऽस्मा अस्मा ऋ॒तव॑ ऋ॒तवो᳚ ऽस्मा॒ अन्वन् व॑स्मा ऋ॒तव॑ ऋ॒तवो᳚ ऽस्मा॒ अनु॑ । \newline
58. अ॒स्मा॒ अन्वन् व॑स्मा अस्मा॒ अनु॒ प्र प्राण्व॑स्मा अस्मा॒ अनु॒ प्र । \newline
59. अनु॒ प्र प्राण्वनु॒ प्र य॑च्छन्ति यच्छन्ति॒ प्राण्वनु॒ प्र य॑च्छन्ति । \newline
60. प्र य॑च्छन्ति यच्छन्ति॒ प्र प्र य॑च्छन्ति॒ यो यो य॑च्छन्ति॒ प्र प्र य॑च्छन्ति॒ यः । \newline
61. य॒च्छ॒न्ति॒ यो यो य॑च्छन्ति यच्छन्ति॒ यो ज्ये॒ष्ठब॑न्धुर् ज्ये॒ष्ठब॑न्धु॒र् यो य॑च्छन्ति यच्छन्ति॒ यो ज्ये॒ष्ठब॑न्धुः । \newline
\pagebreak
\markright{ TS 3.4.8.7  \hfill https://www.vedavms.in \hfill}

\section{ TS 3.4.8.7 }

\textbf{TS 3.4.8.7 } \newline
\textbf{Samhita Paata} \newline

यो ज्ये॒ष्ठब॑न्धु॒रप॑ भूतः॒ स्यात् तꣳस्थले॑ऽव॒साय्य॑ ब्रह्मौद॒नं चतुः॑ शरावं प॒क्त्वा तस्मै॑ होत॒व्या॑ वर्ष्म॒ वै रा᳚ष्ट्र॒भृतो॒ वष्म॒ स्थलं॒ ॅवर्ष्म॑णै॒वैनं॒ ॅवष्म॑ समा॒नानां᳚ गमयति॒ चतुः॑ शरावो भवति दि॒क्ष्वे॑व प्रति॑तिष्ठति क्षी॒रे भ॑वति॒ रुच॑मे॒वास्मि॑न् दधा॒त्युद्ध॑रति शृत॒त्वाय॑ स॒र्पिष्वा᳚न् भवति मेद्ध्य॒त्वाय॑ च॒त्वार॑ आर्.षे॒याः प्राश्न॑न्ति दि॒शामे॒व ज्योति॑षि जुहोति ॥ \newline

\textbf{Pada Paata} \newline

यः । ज्ये॒ष्ठब॑न्धु॒रिति॑ ज्ये॒ष्ठ - ब॒न्धुः॒ । अप॑भूत॒ इत्यप॑ - भू॒तः॒ । स्यात् । तम् । स्थले᳚ । अ॒व॒साय्येत्य॑व - साय्य॑ । ब्र॒ह्मौ॒द॒नमिति॑ ब्रह्म - ओ॒द॒नम् । चतुः॑ शराव॒मिति॒ चतुः॑ - श॒रा॒व॒म् । प॒क्त्वा । तस्मै᳚ । हो॒त॒व्याः᳚ । वर्ष्म॑ । वै । रा॒ष्ट्र॒भृत॒ इति॑ राष्ट्र - भृतः॑ । वर्ष्म॑ । स्थल᳚म् । वर्ष्म॑णा । ए॒व । ए॒न॒म् । वर्ष्म॑ । स॒मा॒नाना᳚म् । ग॒म॒य॒ति॒ । चतुः॑ शराव॒ इति॒ चतुः॑ - श॒रा॒वः॒ । भ॒व॒ति॒ । दि॒क्षु । ए॒व । प्रतीति॑ । ति॒ष्ठ॒ति॒ । क्षी॒रे । भ॒व॒ति॒ । रुच᳚म् । ए॒व । अ॒स्मि॒न्न् । द॒धा॒ति॒ । उदिति॑ । ह॒र॒ति॒ । शृ॒त॒त्वायेति॑ शृत - त्वाय॑ । स॒र्पिष्वान्॑ । भ॒व॒ति॒ । मे॒द्ध्य॒त्वायेति॑ मेद्ध्य-त्वाय॑ । च॒त्वारः॑ । आ॒र्॒.षे॒याः । प्रेति॑ । अ॒श्न॒न्ति॒ । दि॒शाम् । ए॒व । ज्योति॑षि । जु॒हो॒ति॒ ॥  \newline


\textbf{Krama Paata} \newline

यो ज्ये॒ष्ठब॑न्धुः । ज्ये॒ष्ठब॑न्धु॒रप॑भूतः । ज्ये॒ष्ठब॑न्धु॒रिति॑ ज्ये॒ष्ठ - ब॒न्धुः॒ । अप॑भूतः॒ स्यात् । अप॑भूत॒ इत्यप॑ - भू॒तः॒ । स्यात् तम् । तꣳ स्थले᳚ । स्थले॑ ऽव॒साय्य॑ । अ॒व॒साय्य॑ ब्रह्मौद॒नम् । अ॒व॒साय्येत्य॑व - साय्य॑ । ब्र॒ह्मौ॒द॒नम् चतु॑श्शरावम् । ब्र॒ह्मौ॒द॒नमिति॑ ब्रह्म - ओ॒द॒नम् । चतु॑श्शरावम् प॒क्त्वा । चतु॑श्शराव॒मिति॒ चतुः॑ - श॒रा॒व॒म् । प॒क्त्वा तस्मै᳚ । तस्मै॑ होत॒व्याः᳚ । हो॒त॒व्या॑ वर्ष्म॑ । वर्ष्म॒ वै । वै रा᳚ष्ट्र॒भृतः॑ । रा॒ष्ट्र॒भृतो॒ वर्ष्म॑ । रा॒ष्ट्र॒भृत॒ इति॑ राष्ट्र - भृतः॑ । वर्ष्म॒ स्थल᳚म् । स्थल॒म् ॅवर्ष्म॑णा । वर्ष्म॑णै॒व । ए॒वैन᳚म् । ए॒न॒म् ॅवर्ष्म॑ । वर्ष्म॑ समा॒नाना᳚म् । स॒मा॒नाना᳚म् गमयति । ग॒म॒य॒ति॒ चतु॑श्शरावः । चतु॑श्शरावो भवति । चतु॑श्शराव॒ इति॒ चतुः॑ - श॒रा॒वः॒ । भ॒व॒ति॒ दि॒क्षु । दि॒क्ष्वे॑व । ए॒व प्रति॑ । प्रति॑ तिष्ठति । ति॒ष्ठ॒ति॒ क्षी॒रे । क्षी॒रे भ॑वति । भ॒व॒ति॒ रुच᳚म् । रुच॑मे॒व । ए॒वास्मिन्न्॑ । अ॒स्मि॒न् द॒धा॒ति॒ । द॒धा॒त्युत् । उद्ध॑रति । ह॒र॒ति॒ शृ॒त॒त्वाय॑ । शृ॒त॒त्वाय॑ स॒र्पिष्वान्॑ । शृ॒त॒त्वायेति॑ शृत - त्वाय॑ । स॒र्पिष्वा᳚न् भवति । भ॒व॒ति॒ मे॒द्ध्य॒त्वाय॑ । मे॒द्ध्य॒त्वाय॑ च॒त्वारः॑ । मे॒द्ध्य॒त्वायेति॑ मेद्ध्य - त्वाय॑ । च॒त्वार॑ आर्.षे॒याः । आ॒र्॒.षे॒याः प्र । प्राश्ञ॑न्ति । अ॒श्ञ॒न्ति॒ दि॒शाम् । दि॒शामे॒व । ए॒व ज्योति॑षि । ज्योति॑षि जुहोति । जु॒हो॒तीति॑ जुहोति । \newline

\textbf{Jatai Paata} \newline

1. यो ज्ये॒ष्ठब॑न्धुर् ज्ये॒ष्ठब॑न्धु॒र् यो यो ज्ये॒ष्ठब॑न्धुः । \newline
2. ज्ये॒ष्ठब॑न्धु॒ रप॑भू॒तो ऽप॑भूतो ज्ये॒ष्ठब॑न्धुर् ज्ये॒ष्ठब॑न्धु॒ रप॑भूतः । \newline
3. ज्ये॒ष्ठब॑न्धु॒रिति॑ ज्ये॒ष्ठ - ब॒न्धुः॒ । \newline
4. अप॑भूतः॒ स्याथ् स्या दप॑भू॒तो ऽप॑भूतः॒ स्यात् । \newline
5. अप॑भूत॒ इत्यप॑ - भू॒तः॒ । \newline
6. स्यात् तम् तꣳ स्याथ् स्यात् तम् । \newline
7. तꣳ स्थले॒ स्थले॒ तम् तꣳ स्थले᳚ । \newline
8. स्थले॑ ऽव॒साय्या॑ व॒साय्य॒ स्थले॒ स्थले॑ ऽव॒साय्य॑ । \newline
9. अ॒व॒साय्य॑ ब्रह्मौद॒नम् ब्र॑ह्मौद॒न म॑व॒साय्या॑ व॒साय्य॑ ब्रह्मौद॒नम् । \newline
10. अ॒व॒साय्येत्य॑व - साय्य॑ । \newline
11. ब्र॒ह्मौ॒द॒नम् चतु॑श्शराव॒म् चतु॑श्शरावम् ब्रह्मौद॒नम् ब्र॑ह्मौद॒नम् चतु॑श्शरावम् । \newline
12. ब्र॒ह्मौ॒द॒नमिति॑ ब्रह्म - ओ॒द॒नम् । \newline
13. चतु॑श्शरावम् प॒क्त्वा प॒क्त्वा चतु॑श्शराव॒म् चतु॑श्शरावम् प॒क्त्वा । \newline
14. चतु॑श्शराव॒मिति॒ चतुः॑ - श॒रा॒व॒म् । \newline
15. प॒क्त्वा तस्मै॒ तस्मै॑ प॒क्त्वा प॒क्त्वा तस्मै᳚ । \newline
16. तस्मै॑ होत॒व्या॑ होत॒व्या᳚ स्तस्मै॒ तस्मै॑ होत॒व्याः᳚ । \newline
17. हो॒त॒व्या॑ वर्ष्म॒ वर्ष्म॑ होत॒व्या॑ होत॒व्या॑ वर्ष्म॑ । \newline
18. वर्ष्म॒ वै वै वर्ष्म॒ वर्ष्म॒ वै । \newline
19. वै रा᳚ष्ट्र॒भृतो॑ राष्ट्र॒भृतो॒ वै वै रा᳚ष्ट्र॒भृतः॑ । \newline
20. रा॒ष्ट्र॒भृतो॒ वर्ष्म॒ वर्ष्म॑ राष्ट्र॒भृतो॑ राष्ट्र॒भृतो॒ वर्ष्म॑ । \newline
21. रा॒ष्ट्र॒भृत॒ इति॑ राष्ट्र - भृतः॑ । \newline
22. वर्ष्म॒ स्थलꣳ॒॒ स्थलं॒ ॅवर्ष्म॒ वर्ष्म॒ स्थल᳚म् । \newline
23. स्थलं॒ ॅवर्ष्म॑णा॒ वर्ष्म॑णा॒ स्थलꣳ॒॒ स्थलं॒ ॅवर्ष्म॑णा । \newline
24. वर्ष्म॑णै॒वैव वर्ष्म॑णा॒ वर्ष्म॑णै॒व । \newline
25. ए॒वैन॑ मेन मे॒वै वैन᳚म् । \newline
26. ए॒नं॒ ॅवर्ष्म॒ वर्ष्मै॑न मेनं॒ ॅवर्ष्म॑ । \newline
27. वर्ष्म॑ समा॒नानाꣳ॑ समा॒नानां॒ ॅवर्ष्म॒ वर्ष्म॑ समा॒नाना᳚म् । \newline
28. स॒मा॒नाना᳚म् गमयति गमयति समा॒नानाꣳ॑ समा॒नाना᳚म् गमयति । \newline
29. ग॒म॒य॒ति॒ चतु॑श्शराव॒ श्चतु॑श्शरावो गमयति गमयति॒ चतु॑श्शरावः । \newline
30. चतु॑श्शरावो भवति भवति॒ चतु॑श्शराव॒ श्चतु॑श्शरावो भवति । \newline
31. चतु॑श्शराव॒ इति॒ चतुः॑ - श॒रा॒वः॒ । \newline
32. भ॒व॒ति॒ दि॒क्षु दि॒क्षु भ॑वति भवति दि॒क्षु । \newline
33. दि॒क्ष्वे॑वैव दि॒क्षु दि॒क्ष्वे॑व । \newline
34. ए॒व प्रति॒ प्रत्ये॒वैव प्रति॑ । \newline
35. प्रति॑ तिष्ठति तिष्ठति॒ प्रति॒ प्रति॑ तिष्ठति । \newline
36. ति॒ष्ठ॒ति॒ क्षी॒रे क्षी॒रे ति॑ष्ठति तिष्ठति क्षी॒रे । \newline
37. क्षी॒रे भ॑वति भवति क्षी॒रे क्षी॒रे भ॑वति । \newline
38. भ॒व॒ति॒ रुचꣳ॒॒ रुच॑म् भवति भवति॒ रुच᳚म् । \newline
39. रुच॑ मे॒वैव रुचꣳ॒॒ रुच॑ मे॒व । \newline
40. ए॒वास्मि॑न्-नस्मिन्-ने॒वैवास्मिन्न्॑ । \newline
41. अ॒स्मि॒न् द॒धा॒ति॒ द॒धा॒त्य॒ स्मि॒न्-न॒स्मि॒न् द॒धा॒ति॒ । \newline
42. द॒धा॒ त्युदुद् द॑धाति दधा॒त्युत् । \newline
43. उद्ध॑रति हर॒ त्युदु द्ध॑रति । \newline
44. ह॒र॒ति॒ शृ॒त॒त्वाय॑ शृत॒त्वाय॑ हरति हरति शृत॒त्वाय॑ । \newline
45. शृ॒त॒त्वाय॑ स॒र्पिष्वा᳚न् थ्स॒र्पिष्वा᳚ञ् छृत॒त्वाय॑ शृत॒त्वाय॑ स॒र्पिष्वान्॑ । \newline
46. शृ॒त॒त्वायेति॑ शृत - त्वाय॑ । \newline
47. स॒र्पिष्वा᳚न् भवति भवति स॒र्पिष्वा᳚न् थ्स॒र्पिष्वा᳚न् भवति । \newline
48. भ॒व॒ति॒ मे॒द्ध्य॒त्वाय॑ मेद्ध्य॒त्वाय॑ भवति भवति मेद्ध्य॒त्वाय॑ । \newline
49. मे॒द्ध्य॒त्वाय॑ च॒त्वार॑श्च॒त्वारो॑ मेद्ध्य॒त्वाय॑ मेद्ध्य॒त्वाय॑ च॒त्वारः॑ । \newline
50. मे॒द्ध्य॒त्वायेति॑ मेद्ध्य - त्वाय॑ । \newline
51. च॒त्वार॑ आर्.षे॒या आ॑र्.षे॒या श्च॒त्वार॑ श्च॒त्वार॑ आर्.षे॒याः । \newline
52. आ॒र्॒.षे॒याः प्र प्रार्.षे॒या आ॑र्.षे॒याः प्र । \newline
53. प्राश्ञ॑ न्त्यश्ञन्ति॒ प्र प्राश्ञ॑न्ति । \newline
54. अ॒श्ञ॒न्ति॒ दि॒शाम् दि॒शा म॑श्ञ न्त्यश्ञन्ति दि॒शाम् । \newline
55. दि॒शा मे॒वैव दि॒शाम् दि॒शा मे॒व । \newline
56. ए॒व ज्योति॑षि॒ ज्योति॑ष्ये॒वैव ज्योति॑षि । \newline
57. ज्योति॑षि जुहोति जुहोति॒ ज्योति॑षि॒ ज्योति॑षि जुहोति । \newline
58. जु॒हो॒तीति॑ जुहोति । \newline

\textbf{Ghana Paata } \newline

1. यो ज्ये॒ष्ठब॑न्धुर् ज्ये॒ष्ठब॑न्धु॒र् यो यो ज्ये॒ष्ठब॑न्धु॒ रप॑भू॒तो ऽप॑भूतो ज्ये॒ष्ठब॑न्धु॒र् 
यो यो ज्ये॒ष्ठब॑न्धु॒ रप॑भूतः । \newline
2. ज्ये॒ष्ठब॑न्धु॒ रप॑भू॒तो ऽप॑भूतो ज्ये॒ष्ठब॑न्धुर् ज्ये॒ष्ठब॑न्धु॒ रप॑भूतः॒ स्याथ् स्यादप॑भूतो ज्ये॒ष्ठब॑न्धुर् ज्ये॒ष्ठब॑न्धु॒ रप॑भूतः॒ स्यात् । \newline
3. ज्ये॒ष्ठब॑न्धु॒रिति॑ ज्ये॒ष्ठ - ब॒न्धुः॒ । \newline
4. अप॑भूतः॒ स्याथ् स्या दप॑भू॒तो ऽप॑भूतः॒ स्यात् तम् तꣳ स्या दप॑भू॒तो ऽप॑भूतः॒ स्यात् तम् । \newline
5. अप॑भूत॒ इत्यप॑ - भू॒तः॒ । \newline
6. स्यात् तम् तꣳ स्याथ् स्यात् तꣳ स्थले॒ स्थले॒ तꣳ स्याथ् स्यात् तꣳ स्थले᳚ । \newline
7. तꣳ स्थले॒ स्थले॒ तम् तꣳ स्थले॑ ऽव॒साय्या॑ व॒साय्य॒ स्थले॒ तम् तꣳ स्थले॑ ऽव॒साय्य॑ । \newline
8. स्थले॑ ऽव॒साय्या॑ व॒साय्य॒ स्थले॒ स्थले॑ ऽव॒साय्य॑ ब्रह्मौद॒नम् ब्र॑ह्मौद॒न म॑व॒साय्य॒ स्थले॒ स्थले॑ ऽव॒साय्य॑ ब्रह्मौद॒नम् । \newline
9. अ॒व॒साय्य॑ ब्रह्मौद॒नम् ब्र॑ह्मौद॒न म॑व॒साय्या॑ व॒साय्य॑ ब्रह्मौद॒नम् चतु॑श्शराव॒म् चतु॑श्शरावम् ब्रह्मौद॒न म॑व॒साय्या॑ व॒साय्य॑ ब्रह्मौद॒नम् चतु॑श्शरावम् । \newline
10. अ॒व॒साय्येत्य॑व - साय्य॑ । \newline
11. ब्र॒ह्मौ॒द॒नम् चतु॑श्शराव॒म् चतु॑श्शरावम् ब्रह्मौद॒नम् ब्र॑ह्मौद॒नम् चतु॑श्शरावम् प॒क्त्वा प॒क्त्वा चतु॑श्शरावम् ब्रह्मौद॒नम् ब्र॑ह्मौद॒नम् चतु॑श्शरावम् प॒क्त्वा । \newline
12. ब्र॒ह्मौ॒द॒नमिति॑ ब्रह्म - ओ॒द॒नम् । \newline
13. चतु॑श्शरावम् प॒क्त्वा प॒क्त्वा चतु॑श्शराव॒म् चतु॑श्शरावम् प॒क्त्वा तस्मै॒ तस्मै॑ प॒क्त्वा चतु॑श्शराव॒म् चतु॑श्शरावम् प॒क्त्वा तस्मै᳚ । \newline
14. चतु॑श्शराव॒मिति॒ चतुः॑ - श॒रा॒व॒म् । \newline
15. प॒क्त्वा तस्मै॒ तस्मै॑ प॒क्त्वा प॒क्त्वा तस्मै॑ होत॒व्या॑ होत॒व्या᳚ स्तस्मै॑ प॒क्त्वा प॒क्त्वा तस्मै॑ होत॒व्याः᳚ । \newline
16. तस्मै॑ होत॒व्या॑ होत॒व्या᳚ स्तस्मै॒ तस्मै॑ होत॒व्या॑ वर्ष्म॒ वर्ष्म॑ होत॒व्या᳚ स्तस्मै॒ तस्मै॑ होत॒व्या॑ वर्ष्म॑ । \newline
17. हो॒त॒व्या॑ वर्ष्म॒ वर्ष्म॑ होत॒व्या॑ होत॒व्या॑ वर्ष्म॒ वै वै वर्ष्म॑ होत॒व्या॑ होत॒व्या॑ वर्ष्म॒ वै । \newline
18. वर्ष्म॒ वै वै वर्ष्म॒ वर्ष्म॒ वै रा᳚ष्ट्र॒भृतो॑ राष्ट्र॒भृतो॒ वै वर्ष्म॒ वर्ष्म॒ वै रा᳚ष्ट्र॒भृतः॑ । \newline
19. वै रा᳚ष्ट्र॒भृतो॑ राष्ट्र॒भृतो॒ वै वै रा᳚ष्ट्र॒भृतो॒ वर्ष्म॒ वर्ष्म॑ राष्ट्र॒भृतो॒ वै वै 
रा᳚ष्ट्र॒भृतो॒ वर्ष्म॑ । \newline
20. रा॒ष्ट्र॒भृतो॒ वर्ष्म॒ वर्ष्म॑ राष्ट्र॒भृतो॑ राष्ट्र॒भृतो॒ वर्ष्म॒ स्थलꣳ॒॒ स्थल॒म् ॅवर्ष्म॑ राष्ट्र॒भृतो॑ राष्ट्र॒भृतो॒ वर्ष्म॒ स्थल᳚म् । \newline
21. रा॒ष्ट्र॒भृत॒ इति॑ राष्ट्र - भृतः॑ । \newline
22. वर्ष्म॒ स्थलꣳ॒॒ स्थल॒म् ॅवर्ष्म॒ वर्ष्म॒ स्थल॒म् ॅवर्ष्म॑णा॒ वर्ष्म॑णा॒ स्थल॒म् ॅवर्ष्म॒ वर्ष्म॒ स्थल॒म् ॅवर्ष्म॑णा । \newline
23. स्थल॒म् ॅवर्ष्म॑णा॒ वर्ष्म॑णा॒ स्थलꣳ॒॒ स्थल॒म् ॅवर्ष्म॑णै॒वैव वर्ष्म॑णा॒ स्थलꣳ॒॒ 
स्थल॒म् ॅवर्ष्म॑णै॒व । \newline
24. वर्ष्म॑णै॒वैव वर्ष्म॑णा॒ वर्ष्म॑णै॒वैन॑ मेन मे॒व वर्ष्म॑णा॒ वर्ष्म॑णै॒वैन᳚म् । \newline
25. ए॒वैन॑ मेन मे॒वैवैन॒म् ॅवर्ष्म॒ वर्ष्मै॑न मे॒वैवैन॒म् ॅवर्ष्म॑ । \newline
26. ए॒न॒म् ॅवर्ष्म॒ वर्ष्मै॑न मेन॒म् ॅवर्ष्म॑ समा॒नानाꣳ॑ समा॒नाना॒म् ॅवर्ष्मै॑न मेन॒म् ॅवर्ष्म॑ समा॒नाना᳚म् । \newline
27. वर्ष्म॑ समा॒नानाꣳ॑ समा॒नाना॒म् ॅवर्ष्म॒ वर्ष्म॑ समा॒नाना᳚म् गमयति गमयति समा॒नाना॒म् ॅवर्ष्म॒ वर्ष्म॑ समा॒नाना᳚म् गमयति । \newline
28. स॒मा॒नाना᳚म् गमयति गमयति समा॒नानाꣳ॑ समा॒नाना᳚म् गमयति॒ चतु॑श्शराव॒ श्चतु॑श्शरावो गमयति समा॒नानाꣳ॑ समा॒नाना᳚म् गमयति॒ चतु॑श्शरावः । \newline
29. ग॒म॒य॒ति॒ चतु॑श्शराव॒ श्चतु॑श्शरावो गमयति गमयति॒ चतु॑श्शरावो भवति भवति॒ चतु॑श्शरावो गमयति गमयति॒ चतु॑श्शरावो भवति । \newline
30. चतु॑श्शरावो भवति भवति॒ चतु॑श्शराव॒ श्चतु॑श्शरावो भवति दि॒क्षु दि॒क्षु भ॑वति॒ चतु॑श्शराव॒ श्चतु॑श्शरावो भवति दि॒क्षु । \newline
31. चतु॑श्शराव॒ इति॒ चतुः॑ - श॒रा॒वः॒ । \newline
32. भ॒व॒ति॒ दि॒क्षु दि॒क्षु भ॑वति भवति दि॒क्ष्वे॑वैव दि॒क्षु भ॑वति भवति दि॒क्ष्वे॑व । \newline
33. दि॒क्ष्वे॑वैव दि॒क्षु दि॒क्ष्वे॑व प्रति॒ प्रत्ये॒व दि॒क्षु दि॒क्ष्वे॑व प्रति॑ । \newline
34. ए॒व प्रति॒ प्रत्ये॒वैव प्रति॑ तिष्ठति तिष्ठति॒ प्रत्ये॒वैव प्रति॑ तिष्ठति । \newline
35. प्रति॑ तिष्ठति तिष्ठति॒ प्रति॒ प्रति॑ तिष्ठति क्षी॒रे क्षी॒रे ति॑ष्ठति॒ प्रति॒ प्रति॑ तिष्ठति क्षी॒रे । \newline
36. ति॒ष्ठ॒ति॒ क्षी॒रे क्षी॒रे ति॑ष्ठति तिष्ठति क्षी॒रे भ॑वति भवति क्षी॒रे ति॑ष्ठति तिष्ठति क्षी॒रे भ॑वति । \newline
37. क्षी॒रे भ॑वति भवति क्षी॒रे क्षी॒रे भ॑वति॒ रुचꣳ॒॒ रुच॑म् भवति क्षी॒रे क्षी॒रे भ॑वति॒ रुच᳚म् । \newline
38. भ॒व॒ति॒ रुचꣳ॒॒ रुच॑म् भवति भवति॒ रुच॑ मे॒वैव रुच॑म् भवति भवति॒ रुच॑ मे॒व । \newline
39. रुच॑ मे॒वैव रुचꣳ॒॒ रुच॑ मे॒वास्मि॑न्,नस्मिन्,ने॒व रुचꣳ॒॒ रुच॑ मे॒वास्मिन्न्॑ । \newline
40. ए॒वास्मि॑न्,नस्मिन्,ने॒वैवास्मि॑न् दधाति दधा त्यस्मिन्,ने॒वैवास्मि॑न् दधाति । \newline
41. अ॒स्मि॒न् द॒धा॒ति॒ द॒धा॒ त्य॒स्मि॒न्,न॒स्मि॒न् द॒धा॒ त्युदुद् द॑धा त्यस्मिन्,नस्मिन् दधा॒ त्युत् । \newline
42. द॒धा॒ त्युदुद् द॑धाति दधा॒ त्युद्ध॑रति हर॒ त्युद् द॑धाति दधा॒ त्युद्ध॑रति । \newline
43. उद्ध॑रति हर॒ त्युदु द्ध॑रति शृत॒त्वाय॑ शृत॒त्वाय॑ हर॒ त्युदु द्ध॑रति शृत॒त्वाय॑ । \newline
44. ह॒र॒ति॒ शृ॒त॒त्वाय॑ शृत॒त्वाय॑ हरति हरति शृत॒त्वाय॑ स॒र्पिष्वा᳚न् थ्स॒र्पिष्वा᳚ञ् छृत॒त्वाय॑ हरति हरति 
शृत॒त्वाय॑ स॒र्पिष्वान्॑ । \newline
45. शृ॒त॒त्वाय॑ स॒र्पिष्वा᳚न् थ्स॒र्पिष्वा᳚ञ् छृत॒त्वाय॑ शृत॒त्वाय॑ स॒र्पिष्वा᳚न् भवति भवति स॒र्पिष्वा᳚ञ् छृत॒त्वाय॑ शृत॒त्वाय॑ स॒र्पिष्वा᳚न् भवति । \newline
46. शृ॒त॒त्वायेति॑ शृत - त्वाय॑ । \newline
47. स॒र्पिष्वा᳚न् भवति भवति स॒र्पिष्वा᳚न् थ्स॒र्पिष्वा᳚न् भवति मेद्ध्य॒त्वाय॑ मेद्ध्य॒त्वाय॑ भवति स॒र्पिष्वा᳚न् थ्स॒र्पिष्वा᳚न् भवति मेद्ध्य॒त्वाय॑ । \newline
48. भ॒व॒ति॒ मे॒द्ध्य॒त्वाय॑ मेद्ध्य॒त्वाय॑ भवति भवति मेद्ध्य॒त्वाय॑ च॒त्वार॑ श्च॒त्वारो॑ मेद्ध्य॒त्वाय॑ भवति भवति मेद्ध्य॒त्वाय॑ च॒त्वारः॑ । \newline
49. मे॒द्ध्य॒त्वाय॑ च॒त्वार॑ श्च॒त्वारो॑ मेद्ध्य॒त्वाय॑ मेद्ध्य॒त्वाय॑ च॒त्वार॑ आर्.षे॒या आ॑र्.षे॒या श्च॒त्वारो॑ मेद्ध्य॒त्वाय॑ मेद्ध्य॒त्वाय॑ च॒त्वार॑ आर्.षे॒याः । \newline
50. मे॒द्ध्य॒त्वायेति॑ मेद्ध्य - त्वाय॑ । \newline
51. च॒त्वार॑ आर्.षे॒या आ॑र्.षे॒या श्च॒त्वार॑ श्च॒त्वार॑ आर्.षे॒याः प्र प्रार्.षे॒या श्च॒त्वार॑ श्च॒त्वार॑ आर्.षे॒याः प्र । \newline
52. आ॒र्॒.षे॒याः प्र प्रार्.षे॒या आ॑र्.षे॒याः प्राश्ञ॑ न्त्यश्ञन्ति॒ प्रार्.षे॒या आ॑र्.षे॒याः प्राश्ञ॑न्ति । \newline
53. प्राश्ञ॑ न्त्यश्ञन्ति॒ प्र प्राश्ञ॑न्ति दि॒शाम् दि॒शा म॑श्ञन्ति॒ प्र प्राश्ञ॑न्ति दि॒शाम् । \newline
54. अ॒श्ञ॒न्ति॒ दि॒शाम् दि॒शा म॑श्ञ न्त्यश्ञन्ति दि॒शा मे॒वैव दि॒शा म॑श्ञन्त्य श्ञन्ति दि॒शा मे॒व । \newline
55. दि॒शा मे॒वैव दि॒शाम् दि॒शा मे॒व ज्योति॑षि॒ ज्योति॑ष्ये॒व दि॒शाम् दि॒शा मे॒व ज्योति॑षि । \newline
56. ए॒व ज्योति॑षि॒ ज्योति॑ष्ये॒वैव ज्योति॑षि जुहोति जुहोति॒ ज्योति॑ष्ये॒वैव ज्योति॑षि जुहोति । \newline
57. ज्योति॑षि जुहोति जुहोति॒ ज्योति॑षि॒ ज्योति॑षि जुहोति । \newline
58. जु॒हो॒तीति॑ जुहोति । \newline
\pagebreak
\markright{ TS 3.4.9.1  \hfill https://www.vedavms.in \hfill}

\section{ TS 3.4.9.1 }

\textbf{TS 3.4.9.1 } \newline
\textbf{Samhita Paata} \newline

देवि॑का॒ निव॑र्पेत् प्र॒जाका॑म॒श्छन्दाꣳ॑सि॒ वै देवि॑का॒श्छन्दाꣳ॑सीव॒ खलु॒ वै प्र॒जाश्छन्दो॑भिरे॒वास्मै᳚ प्र॒जाः प्रज॑नयति प्रथ॒मं धा॒तारं॑ करोति मिथु॒नी ए॒व तेन॑ करो॒त्यन्वे॒वास्मा॒ अनु॑मतिर्मन्यते रा॒ते रा॒का प्र सि॑नीवा॒ली ज॑नयति प्र॒जास्वे॒व प्रजा॑तासु कु॒ह्वा॑ वाचं॑ दधात्ये॒ता ए॒व निव॑र्पेत् प॒शुका॑म॒श्छन्दाꣳ॑सि॒ वै देवि॑का॒श्छन्दाꣳ॑सी - [  ] \newline

\textbf{Pada Paata} \newline

देवि॑काः । निरिति॑ । व॒पे॒त् । प्र॒जाका॑म॒ इति॑ प्र॒जा - का॒मः॒ । छन्दाꣳ॑सि । वै । देवि॑काः । छन्दाꣳ॑सि । इ॒व॒ । खलु॑ । वै । प्र॒जा इति॑ प्र-जाः । छन्दो॑भि॒रिति॒ छन्दः॑ - भिः॒ । ए॒व । अ॒स्मै॒ । प्र॒जा इति॑ प्र-जाः । प्रेति॑ । ज॒न॒य॒ति॒ । प्र॒थ॒मम् । धा॒तार᳚म् । क॒रो॒ति॒ । मि॒थु॒नी । ए॒व । तेन॑ । क॒रो॒ति॒ । अन्विति॑ । ए॒व । अ॒स्मै॒ । अनु॑मति॒रित्य॑नु - म॒तिः॒ । म॒न्य॒ते॒ । रा॒ते । रा॒का । प्रेति॑ । सि॒नी॒वा॒ली । ज॒न॒य॒ति॒ । प्र॒जास्विति॑ प्र - जासु॑ । ए॒व । प्रजा॑ता॒स्विति॒ प्र - जा॒ता॒सु॒ । कु॒ह्वा᳚ । वाच᳚म् । द॒धा॒ति॒ । ए॒ताः । ए॒व । निरिति॑ । व॒पे॒त् । प॒शुका॑म॒ इति॑ प॒शु - का॒मः॒ । छन्दाꣳ॑सि । वै । देवि॑काः । छन्दाꣳ॑सि ।  \newline


\textbf{Krama Paata} \newline

देवि॑का॒ निः । निर् व॑पेत् । व॒पे॒त् प्र॒जाका॑मः । प्र॒जाका॑म॒ श्छन्दाꣳ॑सि । प्र॒जाका॑म॒ इति॑ प्र॒जा - का॒मः॒ । छन्दाꣳ॑सि॒ वै । वै देवि॑काः । देवि॑का॒ श्छन्दाꣳ॑सि । छन्दाꣳ॑सीव । इ॒व॒ खलु॑ । खलु॒ वै । वै प्र॒जाः । प्र॒जा श्छन्दो॑भिः । प्र॒जा इति॑ प्र - जाः । छन्दो॑भिरे॒व । छन्दो॑भि॒रिति॒ छन्दः॑ - भिः॒ । ए॒वास्मै᳚ । अ॒स्मै॒ प्र॒जाः । प्र॒जाः प्र । प्र॒जा इति॑ प्र - जाः । प्र ज॑नयति । ज॒न॒य॒ति॒ प्र॒थ॒मम् । प्र॒थ॒मम् धा॒तार᳚म् । धा॒तार॑म् करोति । क॒रो॒ति॒ मि॒थु॒नी । मि॒थु॒नी ए॒व । ए॒व तेन॑ । तेन॑ करोति । क॒रो॒त्यनु॑ । अन्वे॒व । ए॒वास्मै᳚ । अ॒स्मा॒ अनु॑मतिः । अनु॑मतिर् मन्यते । अनु॑मति॒रित्यनु॑ - म॒तिः॒ । म॒न्य॒ते॒ रा॒ते । रा॒ते रा॒का । रा॒का प्र । प्र सि॑नीवा॒ली । सि॒नी॒वा॒ली ज॑नयति । ज॒न॒य॒ति॒ प्र॒जासु॑ । प्र॒जास्वे॒व । प्र॒जास्विति॑ प्र - जासु॑ । ए॒व प्रजा॑तासु । प्रजा॑तासु कु॒ह्वा᳚ । प्रजा॑ता॒स्विति॒ प्र - जा॒ता॒सु॒ । कु॒ह्वा॑ वाच᳚म् । वाच॑म् दधाति । द॒धा॒त्ये॒ताः । ए॒ता ए॒व । ए॒व निः । निर् व॑पेत् । व॒पे॒त् प॒शुका॑मः । प॒शुका॑म॒ श्छन्दाꣳ॑सि । प॒शुका॑म॒ इति॑ प॒शु - का॒मः॒ । छन्दाꣳ॑सि॒ वै । वै देवि॑काः । देवि॑का॒ श्छन्दाꣳ॑सि । छन्दाꣳ॑सीव \newline

\textbf{Jatai Paata} \newline

1. देवि॑का॒ निर् णिर् देवि॑का॒ देवि॑का॒ निः । \newline
2. निर् व॑पेद् वपे॒न् निर् णिर् व॑पेत् । \newline
3. व॒पे॒त् प्र॒जाका॑मः प्र॒जाका॑मो वपेद् वपेत् प्र॒जाका॑मः । \newline
4. प्र॒जाका॑म॒ श्छन्दाꣳ॑सि॒ छन्दाꣳ॑सि प्र॒जाका॑मः प्र॒जाका॑म॒ श्छन्दाꣳ॑सि । \newline
5. प्र॒जाका॑म॒ इति॑ प्र॒जा - का॒मः॒ । \newline
6. छन्दाꣳ॑सि॒ वै वै छन्दाꣳ॑सि॒ छन्दाꣳ॑सि॒ वै । \newline
7. वै देवि॑का॒ देवि॑का॒ वै वै देवि॑काः । \newline
8. देवि॑का॒ श्छन्दाꣳ॑सि॒ छन्दाꣳ॑सि॒ देवि॑का॒ देवि॑का॒ श्छन्दाꣳ॑सि । \newline
9. छन्दाꣳ॑सीवेव॒ छन्दाꣳ॑सि॒ छन्दाꣳ॑सीव । \newline
10. इ॒व॒ खलु॒ खल्वि॑वेव॒ खलु॑ । \newline
11. खलु॒ वै वै खलु॒ खलु॒ वै । \newline
12. वै प्र॒जाः प्र॒जा वै वै प्र॒जाः । \newline
13. प्र॒जा श्छन्दो॑भि॒ श्छन्दो॑भिः प्र॒जाः प्र॒जा श्छन्दो॑भिः । \newline
14. प्र॒जा इति॑ प्र - जाः । \newline
15. छन्दो॑भि रे॒वैव छन्दो॑भि॒ श्छन्दो॑भि रे॒व । \newline
16. छन्दो॑भि॒रिति॒ छन्दः॑ - भिः॒ । \newline
17. ए॒वास्मा॑ अस्मा ए॒वै वास्मै᳚ । \newline
18. अ॒स्मै॒ प्र॒जाः प्र॒जा अ॑स्मा अस्मै प्र॒जाः । \newline
19. प्र॒जाः प्र प्र प्र॒जाः प्र॒जाः प्र । \newline
20. प्र॒जा इति॑ प्र - जाः । \newline
21. प्र ज॑नयति जनयति॒ प्र प्र ज॑नयति । \newline
22. ज॒न॒य॒ति॒ प्र॒थ॒मम् प्र॑थ॒मम् ज॑नयति जनयति प्रथ॒मम् । \newline
23. प्र॒थ॒मम् धा॒तार॑म् धा॒तार॑म् प्रथ॒मम् प्र॑थ॒मम् धा॒तार᳚म् । \newline
24. धा॒तार॑म् करोति करोति धा॒तार॑म् धा॒तार॑म् करोति । \newline
25. क॒रो॒ति॒ मि॒थु॒नी मि॑थु॒नी क॑रोति करोति मिथु॒नी । \newline
26. मि॒थु॒नी ए॒वैव मि॑थु॒नी मि॑थु॒नी ए॒व । \newline
27. ए॒व तेन॒ तेनै॒वैव तेन॑ । \newline
28. तेन॑ करोति करोति॒ तेन॒ तेन॑ करोति । \newline
29. क॒रो॒ त्यन्वनु॑ करोति करो॒ त्यनु॑ । \newline
30. अन्वे॒ वैवान् वन्वे॒व । \newline
31. ए॒वास्मा॑ अस्मा ए॒वै वास्मै᳚ । \newline
32. अ॒स्मा॒ अनु॑मति॒ रनु॑मति रस्मा अस्मा॒ अनु॑मतिः । \newline
33. अनु॑मतिर् मन्यते मन्य॒ते ऽनु॑मति॒ रनु॑मतिर् मन्यते । \newline
34. अनु॑मति॒रित्य॑नु - म॒तिः॒ । \newline
35. म॒न्य॒ते॒ रा॒ते रा॒ते म॑न्यते मन्यते रा॒ते । \newline
36. रा॒ते रा॒का रा॒का रा॒ते रा॒ते रा॒का । \newline
37. रा॒का प्र प्र रा॒का रा॒का प्र । \newline
38. प्र सि॑नीवा॒ली सि॑नीवा॒ली प्र प्र सि॑नीवा॒ली । \newline
39. सि॒नी॒वा॒ली ज॑नयति जनयति सिनीवा॒ली सि॑नीवा॒ली ज॑नयति । \newline
40. ज॒न॒य॒ति॒ प्र॒जासु॑ प्र॒जासु॑ जनयति जनयति प्र॒जासु॑ । \newline
41. प्र॒जास्वे॒वैव प्र॒जासु॑ प्र॒जा स्वे॒व । \newline
42. प्र॒जास्विति॑ प्र - जासु॑ । \newline
43. ए॒व प्रजा॑तासु॒ प्रजा॑तास्वे॒वैव प्रजा॑तासु । \newline
44. प्रजा॑तासु कु॒ह्वा॑ कु॒ह्वा᳚ प्रजा॑तासु॒ प्रजा॑तासु कु॒ह्वा᳚ । \newline
45. प्रजा॑ता॒स्विति॒ प्र - जा॒ता॒सु॒ । \newline
46. कु॒ह्वा॑ वाचं॒ ॅवाच॑म् कु॒ह्वा॑ कु॒ह्वा॑ वाच᳚म् । \newline
47. वाच॑म् दधाति दधाति॒ वाचं॒ ॅवाच॑म् दधाति । \newline
48. द॒धा॒ त्ये॒ता ए॒ता द॑धाति दधात्ये॒ताः । \newline
49. ए॒ता ए॒वैवैता ए॒ता ए॒व । \newline
50. ए॒व निर् णि रे॒वैव निः । \newline
51. निर् व॑पेद् वपे॒न् निर् णिर् व॑पेत् । \newline
52. व॒पे॒त् प॒शुका॑मः प॒शुका॑मो वपेद् वपेत् प॒शुका॑मः । \newline
53. प॒शुका॑म॒ श्छन्दाꣳ॑सि॒ छन्दाꣳ॑सि प॒शुका॑मः प॒शुका॑म॒ श्छन्दाꣳ॑सि । \newline
54. प॒शुका॑म॒ इति॑ प॒शु - का॒मः॒ । \newline
55. छन्दाꣳ॑सि॒ वै वै छन्दाꣳ॑सि॒ छन्दाꣳ॑सि॒ वै । \newline
56. वै देवि॑का॒ देवि॑का॒ वै वै देवि॑काः । \newline
57. देवि॑का॒ श्छन्दाꣳ॑सि॒ छन्दाꣳ॑सि॒ देवि॑का॒ देवि॑का॒ श्छन्दाꣳ॑सि । \newline
58. छन्दाꣳ॑सीवेव॒ छन्दाꣳ॑सि॒ छन्दाꣳ॑सीव । \newline

\textbf{Ghana Paata } \newline

1. देवि॑का॒ निर् णिर् देवि॑का॒ देवि॑का॒ निर् व॑पेद् वपे॒न् निर् देवि॑का॒ देवि॑का॒ निर् व॑पेत् । \newline
2. निर् व॑पेद् वपे॒न् निर् णिर् व॑पेत् प्र॒जाका॑मः प्र॒जाका॑मो वपे॒न् निर् णिर् व॑पेत् प्र॒जाका॑मः । \newline
3. व॒पे॒त् प्र॒जाका॑मः प्र॒जाका॑मो वपेद् वपेत् प्र॒जाका॑म॒ श्छन्दाꣳ॑सि॒ छन्दाꣳ॑सि प्र॒जाका॑मो वपेद् वपेत् प्र॒जाका॑म॒ श्छन्दाꣳ॑सि । \newline
4. प्र॒जाका॑म॒ श्छन्दाꣳ॑सि॒ छन्दाꣳ॑सि प्र॒जाका॑मः प्र॒जाका॑म॒ श्छन्दाꣳ॑सि॒ वै वै छन्दाꣳ॑सि प्र॒जाका॑मः प्र॒जाका॑म॒ श्छन्दाꣳ॑सि॒ वै । \newline
5. प्र॒जाका॑म॒ इति॑ प्र॒जा - का॒मः॒ । \newline
6. छन्दाꣳ॑सि॒ वै वै छन्दाꣳ॑सि॒ छन्दाꣳ॑सि॒ वै देवि॑का॒ देवि॑का॒ वै छन्दाꣳ॑सि॒ छन्दाꣳ॑सि॒ वै देवि॑काः । \newline
7. वै देवि॑का॒ देवि॑का॒ वै वै देवि॑का॒ श्छन्दाꣳ॑सि॒ छन्दाꣳ॑सि॒ देवि॑का॒ वै वै देवि॑का॒ श्छन्दाꣳ॑सि । \newline
8. देवि॑का॒ श्छन्दाꣳ॑सि॒ छन्दाꣳ॑सि॒ देवि॑का॒ देवि॑का॒ श्छन्दाꣳ॑सीवेव॒ छन्दाꣳ॑सि॒ देवि॑का॒ देवि॑का॒ श्छन्दाꣳ॑सीव । \newline
9. छन्दाꣳ॑सीवेव॒ छन्दाꣳ॑सि॒ छन्दाꣳ॑सीव॒ खलु॒ खल्वि॑व॒ छन्दाꣳ॑सि॒ छन्दाꣳ॑सीव॒ खलु॑ । \newline
10. इ॒व॒ खलु॒ खल्वि॑वेव॒ खलु॒ वै वै खल्वि॑वेव॒ खलु॒ वै । \newline
11. खलु॒ वै वै खलु॒ खलु॒ वै प्र॒जाः प्र॒जा वै खलु॒ खलु॒ वै प्र॒जाः । \newline
12. वै प्र॒जाः प्र॒जा वै वै प्र॒जा श्छन्दो॑भि॒ श्छन्दो॑भिः प्र॒जा वै वै प्र॒जा श्छन्दो॑भिः । \newline
13. प्र॒जा श्छन्दो॑भि॒ श्छन्दो॑भिः प्र॒जाः प्र॒जा श्छन्दो॑भि रे॒वैव छन्दो॑भिः प्र॒जाः प्र॒जा श्छन्दो॑भि रे॒व । \newline
14. प्र॒जा इति॑ प्र - जाः । \newline
15. छन्दो॑भि रे॒वैव छन्दो॑भि॒ श्छन्दो॑भि रे॒वास्मा॑ अस्मा ए॒व छन्दो॑भि॒ श्छन्दो॑भि रे॒वा स्मै᳚ । \newline
16. छन्दो॑भि॒रिति॒ छन्दः॑ - भिः॒ । \newline
17. ए॒वास्मा॑ अस्मा ए॒वैवास्मै᳚ प्र॒जाः प्र॒जा अ॑स्मा ए॒वैवास्मै᳚ प्र॒जाः । \newline
18. अ॒स्मै॒ प्र॒जाः प्र॒जा अ॑स्मा अस्मै प्र॒जाः प्र प्र प्र॒जा अ॑स्मा अस्मै प्र॒जाः प्र । \newline
19. प्र॒जाः प्र प्र प्र॒जाः प्र॒जाः प्र ज॑नयति जनयति॒ प्र प्र॒जाः प्र॒जाः प्र ज॑नयति । \newline
20. प्र॒जा इति॑ प्र - जाः । \newline
21. प्र ज॑नयति जनयति॒ प्र प्र ज॑नयति प्रथ॒मम् प्र॑थ॒मम् ज॑नयति॒ प्र प्र ज॑नयति प्रथ॒मम् । \newline
22. ज॒न॒य॒ति॒ प्र॒थ॒मम् प्र॑थ॒मम् ज॑नयति जनयति प्रथ॒मम् धा॒तार॑म् धा॒तार॑म् प्रथ॒मम् ज॑नयति जनयति प्रथ॒मम् धा॒तार᳚म् । \newline
23. प्र॒थ॒मम् धा॒तार॑म् धा॒तार॑म् प्रथ॒मम् प्र॑थ॒मम् धा॒तार॑म् करोति करोति धा॒तार॑म् प्रथ॒मम् प्र॑थ॒मम् धा॒तार॑म् करोति । \newline
24. धा॒तार॑म् करोति करोति धा॒तार॑म् धा॒तार॑म् करोति मिथु॒नी मि॑थु॒नी क॑रोति धा॒तार॑म् धा॒तार॑म् करोति मिथु॒नी । \newline
25. क॒रो॒ति॒ मि॒थु॒नी मि॑थु॒नी क॑रोति करोति मिथु॒नी ए॒वैव मि॑थु॒नी क॑रोति करोति मिथु॒नी ए॒व । \newline
26. मि॒थु॒नी ए॒वैव मि॑थु॒नी मि॑थु॒नी ए॒व तेन॒ तेनै॒व मि॑थु॒नी मि॑थु॒नी ए॒व तेन॑ । \newline
27. ए॒व तेन॒ तेनै॒वैव तेन॑ करोति करोति॒ तेनै॒वैव तेन॑ करोति । \newline
28. तेन॑ करोति करोति॒ तेन॒ तेन॑ करो॒ त्यन्वनु॑ करोति॒ तेन॒ तेन॑ करो॒ त्यनु॑ । \newline
29. क॒रो॒ त्यन्वनु॑ करोति करो॒ त्यन्वे॒वैवानु॑ करोति करो॒ त्यन्वे॒व । \newline
30. अन्वे॒ वैवान् वन्वे॒वा स्मा॑ अस्मा ए॒वान् वन्वे॒वास्मै᳚ । \newline
31. ए॒वास्मा॑ अस्मा ए॒वैवास्मा॒ अनु॑मति॒ रनु॑मति रस्मा ए॒वैवास्मा॒ अनु॑मतिः । \newline
32. अ॒स्मा॒ अनु॑मति॒ रनु॑मति रस्मा अस्मा॒ अनु॑मतिर् मन्यते मन्य॒ते ऽनु॑मति रस्मा अस्मा॒ अनु॑मतिर् मन्यते । \newline
33. अनु॑मतिर् मन्यते मन्य॒ते ऽनु॑मति॒ रनु॑मतिर् मन्यते रा॒ते रा॒ते म॑न्य॒ते ऽनु॑मति॒ रनु॑मतिर् मन्यते रा॒ते । \newline
34. अनु॑मति॒रित्य॑नु - म॒तिः॒ । \newline
35. म॒न्य॒ते॒ रा॒ते रा॒ते म॑न्यते मन्यते रा॒ते रा॒का रा॒का रा॒ते म॑न्यते मन्यते रा॒ते रा॒का । \newline
36. रा॒ते रा॒का रा॒का रा॒ते रा॒ते रा॒का प्र प्र रा॒का रा॒ते रा॒ते रा॒का प्र । \newline
37. रा॒का प्र प्र रा॒का रा॒का प्र सि॑नीवा॒ली सि॑नीवा॒ली प्र रा॒का रा॒का प्र सि॑नीवा॒ली । \newline
38. प्र सि॑नीवा॒ली सि॑नीवा॒ली प्र प्र सि॑नीवा॒ली ज॑नयति जनयति सिनीवा॒ली प्र प्र सि॑नीवा॒ली ज॑नयति । \newline
39. सि॒नी॒वा॒ली ज॑नयति जनयति सिनीवा॒ली सि॑नीवा॒ली ज॑नयति प्र॒जासु॑ प्र॒जासु॑ जनयति सिनीवा॒ली सि॑नीवा॒ली ज॑नयति प्र॒जासु॑ । \newline
40. ज॒न॒य॒ति॒ प्र॒जासु॑ प्र॒जासु॑ जनयति जनयति प्र॒जास्वे॒वैव प्र॒जासु॑ जनयति जनयति प्र॒जास्वे॒व । \newline
41. प्र॒जास्वे॒वैव प्र॒जासु॑ प्र॒जास्वे॒व प्रजा॑तासु॒ प्रजा॑तास्वे॒व प्र॒जासु॑ प्र॒जास्वे॒व प्रजा॑तासु । \newline
42. प्र॒जास्विति॑ प्र - जासु॑ । \newline
43. ए॒व प्रजा॑तासु॒ प्रजा॑तास्वे॒वैव प्रजा॑तासु कु॒ह्वा॑ कु॒ह्वा᳚ प्रजा॑तास्वे॒वैव प्रजा॑तासु कु॒ह्वा᳚ । \newline
44. प्रजा॑तासु कु॒ह्वा॑ कु॒ह्वा᳚ प्रजा॑तासु॒ प्रजा॑तासु कु॒ह्वा॑ वाच॒म् ॅवाच॑म् कु॒ह्वा᳚ प्रजा॑तासु॒ प्रजा॑तासु कु॒ह्वा॑ वाच᳚म् । \newline
45. प्रजा॑ता॒स्विति॒ प्र - जा॒ता॒सु॒ । \newline
46. कु॒ह्वा॑ वाच॒म् ॅवाच॑म् कु॒ह्वा॑ कु॒ह्वा॑ वाच॑म् दधाति दधाति॒ वाच॑म् कु॒ह्वा॑ कु॒ह्वा॑ वाच॑म् दधाति । \newline
47. वाच॑म् दधाति दधाति॒ वाच॒म् ॅवाच॑म् दधा त्ये॒ता ए॒ता द॑धाति॒ वाच॒म् ॅवाच॑म् दधा त्ये॒ताः । \newline
48. द॒धा॒ त्ये॒ता ए॒ता द॑धाति दधा त्ये॒ता ए॒वैवैता द॑धाति दधा त्ये॒ता ए॒व । \newline
49. ए॒ता ए॒वैवैता ए॒ता ए॒व निर् णिरे॒वैता ए॒ता ए॒व निः । \newline
50. ए॒व निर् णिरे॒वैव निर् व॑पेद् वपे॒न् निरे॒वैव निर् व॑पेत् । \newline
51. निर् व॑पेद् वपे॒न् निर् णिर् व॑पेत् प॒शुका॑मः प॒शुका॑मो वपे॒न् निर् णिर् व॑पेत् प॒शुका॑मः । \newline
52. व॒पे॒त् प॒शुका॑मः प॒शुका॑मो वपेद् वपेत् प॒शुका॑म॒ श्छन्दाꣳ॑सि॒ छन्दाꣳ॑सि प॒शुका॑मो वपेद् वपेत् 
प॒शुका॑म॒ श्छन्दाꣳ॑सि । \newline
53. प॒शुका॑म॒ श्छन्दाꣳ॑सि॒ छन्दाꣳ॑सि प॒शुका॑मः प॒शुका॑म॒ श्छन्दाꣳ॑सि॒ वै वै छन्दाꣳ॑सि प॒शुका॑मः प॒शुका॑म॒ श्छन्दाꣳ॑सि॒ वै । \newline
54. प॒शुका॑म॒ इति॑ प॒शु - का॒मः॒ । \newline
55. छन्दाꣳ॑सि॒ वै वै छन्दाꣳ॑सि॒ छन्दाꣳ॑सि॒ वै देवि॑का॒ देवि॑का॒ वै छन्दाꣳ॑सि॒ छन्दाꣳ॑सि॒ वै देवि॑काः । \newline
56. वै देवि॑का॒ देवि॑का॒ वै वै देवि॑का॒ श्छन्दाꣳ॑सि॒ छन्दाꣳ॑सि॒ देवि॑का॒ वै वै देवि॑का॒ श्छन्दाꣳ॑सि । \newline
57. देवि॑का॒ श्छन्दाꣳ॑सि॒ छन्दाꣳ॑सि॒ देवि॑का॒ देवि॑का॒ श्छन्दाꣳ॑सीवेव॒ छन्दाꣳ॑सि॒ देवि॑का॒ देवि॑का॒ श्छन्दाꣳ॑सीव । \newline
58. छन्दाꣳ॑सीवेव॒ छन्दाꣳ॑सि॒ छन्दाꣳ॑सीव॒ खलु॒ खल्वि॑व॒ छन्दाꣳ॑सि॒ छन्दाꣳ॑सीव॒ खलु॑ । \newline
\pagebreak
\markright{ TS 3.4.9.2  \hfill https://www.vedavms.in \hfill}

\section{ TS 3.4.9.2 }

\textbf{TS 3.4.9.2 } \newline
\textbf{Samhita Paata} \newline

व॒ खलु॒ वै प॒शव॒श्छन्दो॑भिरे॒वास्मै॑ प॒शून् प्रज॑नयति प्रथ॒मं धा॒तारं॑ करोति॒ प्रैव तेन॑ वापय॒त्यन्वे॒वास्मा॒ अनु॑मतिर्मन्यते रा॒ते रा॒का प्र सि॑नीवा॒ली ज॑नयति प॒शूने॒व प्रजा॑तान् कु॒ह्वा᳚ प्रति॑ष्ठापयत्ये॒ता ए॒व निर्व॑पे॒द्-ग्राम॑काम॒श्छन्दाꣳ॑सि॒ वै देवि॑का॒श्छन्दाꣳ॑सी व॒ खलु॒ वै ग्राम॒श्छन्दो॑भिरे॒वास्मै॒ ग्राम॒ - [  ] \newline

\textbf{Pada Paata} \newline

इ॒व॒ । खलु॑ । वै । प॒शवः॑ । छन्दो॑भि॒रिति॒ छन्दः॑ - भिः॒ । ए॒व । अ॒स्मै॒ । प॒शून् । प्रेति॑ । ज॒न॒य॒ति॒ । प्र॒थ॒मम् । धा॒तार᳚म् । क॒रो॒ति॒ । प्रेति॑ । ए॒व । तेन॑ । वा॒प॒य॒ति॒ । अन्विति॑ । ए॒व । अ॒स्मै॒ । अनु॑मति॒रित्य॑नु - म॒तिः॒ । म॒न्य॒ते॒ । रा॒ते । रा॒का । प्रेति॑ । सि॒नी॒वा॒ली । ज॒न॒य॒ति॒ । प॒शून् । ए॒व । प्रजा॑ता॒निति॒ प्र-जा॒ता॒न् । कु॒ह्वा᳚ । प्रतीति॑ । स्था॒प॒य॒ति॒ । ए॒ताः । ए॒व । निरिति॑ । व॒पे॒त् । ग्राम॑काम॒ इति॒ ग्राम॑ - का॒मः॒ । छन्दाꣳ॑सि । वै । देवि॑काः । छन्दाꣳ॑सि । इ॒व॒ । खलु॑ । वै । ग्रामः॑ । छन्दो॑भि॒रिति॒ छन्दः॑ - भिः॒ । ए॒व । अ॒स्मै॒ । ग्राम᳚म् ।  \newline


\textbf{Krama Paata} \newline

इ॒व॒ खलु॑ । खलु॒ वै । वै प॒शवः॑ । प॒शव॒ श्छन्दो॑भिः । छन्दो॑भिरे॒व । छन्दो॑भि॒रिति॒ छन्दः॑ - भिः॒ । ए॒वास्मै᳚ । अ॒स्मै॒ प॒शून् । प॒शून् प्र । प्र ज॑नयति । ज॒न॒य॒ति॒ प्र॒थ॒मम् । प्र॒थ॒मम् धा॒तार᳚म् । धा॒तार॑म् करोति । क॒रो॒ति॒ प्र । प्रैव । ए॒व तेन॑ । तेन॑ वापयति । वा॒प॒य॒त्यनु॑ । अन्वे॒व । ए॒वास्मै᳚ । अ॒स्मा॒ अनु॑मतिः । अनु॑मतिर् मन्यते । अनु॑मति॒रित्यनु॑ - म॒तिः॒ । म॒न्य॒ते॒ रा॒ते । रा॒ते रा॒का । रा॒का प्र । प्र सि॑नीवा॒ली । सि॒नी॒वा॒ली ज॑नयति । ज॒न॒य॒ति॒ प॒शून् । प॒शूने॒व । ए॒व प्रजा॑तान् । प्रजा॑तान् कु॒ह्वा᳚ । प्रजा॑ता॒निति॒ प्र - जा॒ता॒न्॒ । कु॒ह्वा᳚ प्रति॑ । प्रति॑ष्ठापयति । स्था॒प॒य॒त्ये॒ताः । ए॒ता ए॒व । ए॒व निः । निर् व॑पेत् । व॒पे॒द् ग्राम॑कामः । ग्राम॑काम॒ श्छन्दाꣳ॑सि । ग्राम॑काम॒ इति॒ ग्राम॑ - का॒मः॒ । छन्दाꣳ॑सि॒ वै । वै देवि॑काः । देवि॑का॒ श्छन्दाꣳ॑सि । छन्दाꣳ॑सीव । इ॒व॒ खलु॑ । खलु॒ वै । वै ग्रामः॑ । ग्राम॒ श्छन्दो॑भिः । छन्दो॑भिरे॒व । छन्दो॑भि॒रिति॒ छन्दः॑ - भिः॒ । ए॒वास्मै᳚ । अ॒स्मै॒ ग्राम᳚म् । ग्राम॒मव॑ \newline

\textbf{Jatai Paata} \newline

1. इ॒व॒ खलु॒ खल्वि॑वे व॒ खलु॑ । \newline
2. खलु॒ वै वै खलु॒ खलु॒ वै । \newline
3. वै प॒शवः॑ प॒शवो॒ वै वै प॒शवः॑ । \newline
4. प॒शव॒ श्छन्दो॑भि॒ श्छन्दो॑भिः प॒शवः॑ प॒शव॒ श्छन्दो॑भिः । \newline
5. छन्दो॑भि रे॒वैव छन्दो॑भि॒ श्छन्दो॑भि रे॒व । \newline
6. छन्दो॑भि॒रिति॒ छन्दः॑ - भिः॒ । \newline
7. ए॒वास्मा॑ अस्मा ए॒वै वास्मै᳚ । \newline
8. अ॒स्मै॒ प॒शून् प॒शू-न॑स्मा अस्मै प॒शून् । \newline
9. प॒शून् प्र प्र प॒शून् प॒शून् प्र । \newline
10. प्र ज॑नयति जनयति॒ प्र प्र ज॑नयति । \newline
11. ज॒न॒य॒ति॒ प्र॒थ॒मम् प्र॑थ॒मम् ज॑नयति जनयति प्रथ॒मम् । \newline
12. प्र॒थ॒मम् धा॒तार॑म् धा॒तार॑म् प्रथ॒मम् प्र॑थ॒मम् धा॒तार᳚म् । \newline
13. धा॒तार॑म् करोति करोति धा॒तार॑म् धा॒तार॑म् करोति । \newline
14. क॒रो॒ति॒ प्र प्र क॑रोति करोति॒ प्र । \newline
15. प्रैवैव प्र प्रैव । \newline
16. ए॒व तेन॒ तेनै॒वैव तेन॑ । \newline
17. तेन॑ वापयति वापयति॒ तेन॒ तेन॑ वापयति । \newline
18. वा॒प॒य॒ त्यन्वनु॑ वापयति वापय॒ त्यनु॑ । \newline
19. अन्वे॒ वैवा,न्वन्,वे॒व । \newline
20. ए॒वास्मा॑ अस्मा ए॒वै वास्मै᳚ । \newline
21. अ॒स्मा॒ अनु॑मति॒ रनु॑मति रस्मा अस्मा॒ अनु॑मतिः । \newline
22. अनु॑मतिर् मन्यते मन्य॒ते ऽनु॑मति॒ रनु॑मतिर् मन्यते । \newline
23. अनु॑मति॒रित्य॑नु - म॒तिः॒ । \newline
24. म॒न्य॒ते॒ रा॒ते रा॒ते म॑न्यते मन्यते रा॒ते । \newline
25. रा॒ते रा॒का रा॒का रा॒ते रा॒ते रा॒का । \newline
26. रा॒का प्र प्र रा॒का रा॒का प्र । \newline
27. प्र सि॑नीवा॒ली सि॑नीवा॒ली प्र प्र सि॑नीवा॒ली । \newline
28. सि॒नी॒वा॒ली ज॑नयति जनयति सिनीवा॒ली सि॑नीवा॒ली ज॑नयति । \newline
29. ज॒न॒य॒ति॒ प॒शून् प॒शून् ज॑नयति जनयति प॒शून् । \newline
30. प॒शू-ने॒वैव प॒शून् प॒शू,ने॒व । \newline
31. ए॒व प्रजा॑ता॒न् प्रजा॑ता-ने॒वैव प्रजा॑तान् । \newline
32. प्रजा॑तान् कु॒ह्वा॑ कु॒ह्वा᳚ प्रजा॑ता॒न् प्रजा॑तान् कु॒ह्वा᳚ । \newline
33. प्रजा॑ता॒निति॒ प्र - जा॒ता॒न् । \newline
34. कु॒ह्वा᳚ प्रति॒ प्रति॑ कु॒ह्वा॑ कु॒ह्वा᳚ प्रति॑ । \newline
35. प्रति॑ ष्ठापयति स्थापयति॒ प्रति॒ प्रति॑ ष्ठापयति । \newline
36. स्था॒प॒य॒ त्ये॒ता ए॒ताः स्था॑पयति स्थापय त्ये॒ताः । \newline
37. ए॒ता ए॒वैवैता ए॒ता ए॒व । \newline
38. ए॒व निर् णि रे॒वैव निः । \newline
39. निर् व॑पेद् वपे॒न् निर् णिर् व॑पेत् । \newline
40. व॒पे॒द् ग्राम॑कामो॒ ग्राम॑कामो वपेद् वपे॒द् ग्राम॑कामः । \newline
41. ग्राम॑काम॒ श्छन्दाꣳ॑सि॒ छन्दाꣳ॑सि॒ ग्राम॑कामो॒ ग्राम॑काम॒ श्छन्दाꣳ॑सि । \newline
42. ग्राम॑काम॒ इति॒ ग्राम॑ - का॒मः॒ । \newline
43. छन्दाꣳ॑सि॒ वै वै छन्दाꣳ॑सि॒ छन्दाꣳ॑सि॒ वै । \newline
44. वै देवि॑का॒ देवि॑का॒ वै वै देवि॑काः । \newline
45. देवि॑का॒ श्छन्दाꣳ॑सि॒ छन्दाꣳ॑सि॒ देवि॑का॒ देवि॑का॒ श्छन्दाꣳ॑सि । \newline
46. छन्दाꣳ॑सीवेव॒ छन्दाꣳ॑सि॒ छन्दाꣳ॑सीव । \newline
47. इ॒व॒ खलु॒ खल्वि॑वे व॒ खलु॑ । \newline
48. खलु॒ वै वै खलु॒ खलु॒ वै । \newline
49. वै ग्रामो॒ ग्रामो॒ वै वै ग्रामः॑ । \newline
50. ग्राम॒ श्छन्दो॑भि॒ श्छन्दो॑भि॒र् ग्रामो॒ ग्राम॒ श्छन्दो॑भिः । \newline
51. छन्दो॑भि रे॒वैव छन्दो॑भि॒ श्छन्दो॑भि रे॒व । \newline
52. छन्दो॑भि॒रिति॒ छन्दः॑ - भिः॒ । \newline
53. ए॒वास्मा॑ अस्मा ए॒वै वास्मै᳚ । \newline
54. अ॒स्मै॒ ग्राम॒म् ग्राम॑ मस्मा अस्मै॒ ग्राम᳚म् । \newline
55. ग्राम॒ मवाव॒ ग्राम॒म् ग्राम॒ मव॑ । \newline

\textbf{Ghana Paata } \newline

1. इ॒व॒ खलु॒ खल्वि॑वेव॒ खलु॒ वै वै खल्वि॑वेव॒ खलु॒ वै । \newline
2. खलु॒ वै वै खलु॒ खलु॒ वै प॒शवः॑ प॒शवो॒ वै खलु॒ खलु॒ वै प॒शवः॑ । \newline
3. वै प॒शवः॑ प॒शवो॒ वै वै प॒शव॒ श्छन्दो॑भि॒ श्छन्दो॑भिः प॒शवो॒ वै वै प॒शव॒ श्छन्दो॑भिः । \newline
4. प॒शव॒ श्छन्दो॑भि॒ श्छन्दो॑भिः प॒शवः॑ प॒शव॒ श्छन्दो॑भि रे॒वैव छन्दो॑भिः प॒शवः॑ प॒शव॒ श्छन्दो॑भि रे॒व । \newline
5. छन्दो॑भि रे॒वैव छन्दो॑भि॒ श्छन्दो॑भि रे॒वास्मा॑ अस्मा ए॒व छन्दो॑भि॒ श्छन्दो॑भि रे॒वास्मै᳚ । \newline
6. छन्दो॑भि॒रिति॒ छन्दः॑ - भिः॒ । \newline
7. ए॒वास्मा॑ अस्मा ए॒वैवास्मै॑ प॒शून् प॒शू,न॑स्मा ए॒वैवास्मै॑ प॒शून् । \newline
8. अ॒स्मै॒ प॒शून् प॒शू,न॑स्मा अस्मै प॒शून् प्र प्र प॒शू,न॑स्मा अस्मै प॒शून् प्र । \newline
9. प॒शून् प्र प्र प॒शून् प॒शून् प्र ज॑नयति जनयति॒ प्र प॒शून् प॒शून् प्र ज॑नयति । \newline
10. प्र ज॑नयति जनयति॒ प्र प्र ज॑नयति प्रथ॒मम् प्र॑थ॒मम् ज॑नयति॒ प्र प्र ज॑नयति प्रथ॒मम् । \newline
11. ज॒न॒य॒ति॒ प्र॒थ॒मम् प्र॑थ॒मम् ज॑नयति जनयति प्रथ॒मम् धा॒तार॑म् धा॒तार॑म् प्रथ॒मम् ज॑नयति 
जनयति प्रथ॒मम् धा॒तार᳚म् । \newline
12. प्र॒थ॒मम् धा॒तार॑म् धा॒तार॑म् प्रथ॒मम् प्र॑थ॒मम् धा॒तार॑म् करोति करोति धा॒तार॑म् प्रथ॒मम् प्र॑थ॒मम् धा॒तार॑म् करोति । \newline
13. धा॒तार॑म् करोति करोति धा॒तार॑म् धा॒तार॑म् करोति॒ प्र प्र क॑रोति धा॒तार॑म् धा॒तार॑म् करोति॒ प्र । \newline
14. क॒रो॒ति॒ प्र प्र क॑रोति करोति॒ प्रैवैव प्र क॑रोति करोति॒ प्रैव । \newline
15. प्रैवैव प्र प्रैव तेन॒ तेनै॒व प्र प्रैव तेन॑ । \newline
16. ए॒व तेन॒ तेनै॒वैव तेन॑ वापयति वापयति॒ तेनै॒वैव तेन॑ वापयति । \newline
17. तेन॑ वापयति वापयति॒ तेन॒ तेन॑ वापय॒ त्यन्वनु॑ वापयति॒ तेन॒ तेन॑ वापय॒ त्यनु॑ । \newline
18. वा॒प॒य॒ त्यन्वनु॑ वापयति वापय॒ त्यन्वे॒वै वानु॑ वापयति वापय॒ त्यन्वे॒व । \newline
19. अन्वे॒ वैवान् वन्वे॒वास्मा॑ अस्मा ए॒वान् वन्वे॒ वास्मै᳚ । \newline
20. ए॒वास्मा॑ अस्मा ए॒वैवास्मा॒ अनु॑मति॒ रनु॑मति रस्मा ए॒वैवास्मा॒ अनु॑मतिः । \newline
21. अ॒स्मा॒ अनु॑मति॒ रनु॑मति रस्मा अस्मा॒ अनु॑मतिर् मन्यते मन्य॒ते ऽनु॑मति रस्मा अस्मा॒ अनु॑मतिर् मन्यते । \newline
22. अनु॑मतिर् मन्यते मन्य॒ते ऽनु॑मति॒ रनु॑मतिर् मन्यते रा॒ते रा॒ते म॑न्य॒ते ऽनु॑मति॒ रनु॑मतिर् मन्यते रा॒ते । \newline
23. अनु॑मति॒रित्य॑नु - म॒तिः॒ । \newline
24. म॒न्य॒ते॒ रा॒ते रा॒ते म॑न्यते मन्यते रा॒ते रा॒का रा॒का रा॒ते म॑न्यते मन्यते रा॒ते रा॒का । \newline
25. रा॒ते रा॒का रा॒का रा॒ते रा॒ते रा॒का प्र प्र रा॒का रा॒ते रा॒ते रा॒का प्र । \newline
26. रा॒का प्र प्र रा॒का रा॒का प्र सि॑नीवा॒ली सि॑नीवा॒ली प्र रा॒का रा॒का प्र सि॑नीवा॒ली । \newline
27. प्र सि॑नीवा॒ली सि॑नीवा॒ली प्र प्र सि॑नीवा॒ली ज॑नयति जनयति सिनीवा॒ली प्र प्र सि॑नीवा॒ली ज॑नयति । \newline
28. सि॒नी॒वा॒ली ज॑नयति जनयति सिनीवा॒ली सि॑नीवा॒ली ज॑नयति प॒शून् प॒शून् ज॑नयति सिनीवा॒ली सि॑नीवा॒ली ज॑नयति प॒शून् । \newline
29. ज॒न॒य॒ति॒ प॒शून् प॒शून् ज॑नयति जनयति प॒शूने॒वैव प॒शून् ज॑नयति जनयति प॒शूने॒व । \newline
30. प॒शूने॒वैव प॒शून् प॒शूने॒व प्रजा॑ता॒न् प्रजा॑ताने॒व प॒शून् प॒शूने॒व प्रजा॑तान् । \newline
31. ए॒व प्रजा॑ता॒न् प्रजा॑ताने॒वैव प्रजा॑तान् कु॒ह्वा॑ कु॒ह्वा᳚ प्रजा॑ताने॒वैव प्रजा॑तान् कु॒ह्वा᳚ । \newline
32. प्रजा॑तान् कु॒ह्वा॑ कु॒ह्वा᳚ प्रजा॑ता॒न् प्रजा॑तान् कु॒ह्वा᳚ प्रति॒ प्रति॑ कु॒ह्वा᳚ प्रजा॑ता॒न् प्रजा॑तान् कु॒ह्वा᳚ प्रति॑ । \newline
33. प्रजा॑ता॒निति॒ प्र - जा॒ता॒न् । \newline
34. कु॒ह्वा᳚ प्रति॒ प्रति॑ कु॒ह्वा॑ कु॒ह्वा᳚ प्रति॑ ष्ठापयति स्थापयति॒ प्रति॑ कु॒ह्वा॑ कु॒ह्वा᳚ प्रति॑ ष्ठापयति । \newline
35. प्रति॑ ष्ठापयति स्थापयति॒ प्रति॒ प्रति॑ ष्ठापय त्ये॒ता ए॒ताः स्था॑पयति॒ प्रति॒ प्रति॑ ष्ठापय त्ये॒ताः । \newline
36. स्था॒प॒य॒ त्ये॒ता ए॒ताः स्था॑पयति स्थापय त्ये॒ता ए॒वैवैताः स्था॑पयति स्थापय त्ये॒ता ए॒व । \newline
37. ए॒ता ए॒वैवैता ए॒ता ए॒व निर् णि रे॒वैता ए॒ता ए॒व निः । \newline
38. ए॒व निर् णि रे॒वैव निर् व॑पेद् वपे॒न् नि रे॒वैव निर् व॑पेत् । \newline
39. निर् व॑पेद् वपे॒न् निर् णिर् व॑पे॒द् ग्राम॑कामो॒ ग्राम॑कामो वपे॒न् निर् णिर् व॑पे॒द् ग्राम॑कामः । \newline
40. व॒पे॒द् ग्राम॑कामो॒ ग्राम॑कामो वपेद् वपे॒द् ग्राम॑काम॒ श्छन्दाꣳ॑सि॒ छन्दाꣳ॑सि॒ ग्राम॑कामो वपेद् वपे॒द् ग्राम॑काम॒ श्छन्दाꣳ॑सि । \newline
41. ग्राम॑काम॒ श्छन्दाꣳ॑सि॒ छन्दाꣳ॑सि॒ ग्राम॑कामो॒ ग्राम॑काम॒ श्छन्दाꣳ॑सि॒ वै वै छन्दाꣳ॑सि॒ ग्राम॑कामो॒ ग्राम॑काम॒ श्छन्दाꣳ॑सि॒ वै । \newline
42. ग्राम॑काम॒ इति॒ ग्राम॑ - का॒मः॒ । \newline
43. छन्दाꣳ॑सि॒ वै वै छन्दाꣳ॑सि॒ छन्दाꣳ॑सि॒ वै देवि॑का॒ देवि॑का॒ वै छन्दाꣳ॑सि॒ छन्दाꣳ॑सि॒ वै देवि॑काः । \newline
44. वै देवि॑का॒ देवि॑का॒ वै वै देवि॑का॒ श्छन्दाꣳ॑सि॒ छन्दाꣳ॑सि॒ देवि॑का॒ वै वै देवि॑का॒ श्छन्दाꣳ॑सि । \newline
45. देवि॑का॒ श्छन्दाꣳ॑सि॒ छन्दाꣳ॑सि॒ देवि॑का॒ देवि॑का॒ श्छन्दाꣳ॑सीवेव॒ छन्दाꣳ॑सि॒ देवि॑का॒ देवि॑का॒ श्छन्दाꣳ॑सीव । \newline
46. छन्दाꣳ॑सीवेव॒ छन्दाꣳ॑सि॒ छन्दाꣳ॑सीव॒ खलु॒ खल्वि॑व॒ छन्दाꣳ॑सि॒ छन्दाꣳ॑सीव॒ खलु॑ । \newline
47. इ॒व॒ खलु॒ खल्वि॑वेव॒ खलु॒ वै वै खल्वि॑वेव॒ खलु॒ वै । \newline
48. खलु॒ वै वै खलु॒ खलु॒ वै ग्रामो॒ ग्रामो॒ वै खलु॒ खलु॒ वै ग्रामः॑ । \newline
49. वै ग्रामो॒ ग्रामो॒ वै वै ग्राम॒ श्छन्दो॑भि॒ श्छन्दो॑भि॒र् ग्रामो॒ वै वै ग्राम॒ श्छन्दो॑भिः । \newline
50. ग्राम॒ श्छन्दो॑भि॒ श्छन्दो॑भि॒र् ग्रामो॒ ग्राम॒ श्छन्दो॑भि रे॒वैव छन्दो॑भि॒र् ग्रामो॒ ग्राम॒ श्छन्दो॑भि रे॒व । \newline
51. छन्दो॑भि रे॒वैव छन्दो॑भि॒ श्छन्दो॑भि रे॒वास्मा॑ अस्मा ए॒व छन्दो॑भि॒ श्छन्दो॑भि रे॒वास्मै᳚ । \newline
52. छन्दो॑भि॒रिति॒ छन्दः॑ - भिः॒ । \newline
53. ए॒वास्मा॑ अस्मा ए॒वैवास्मै॒ ग्राम॒म् ग्राम॑ मस्मा ए॒वैवास्मै॒ ग्राम᳚म् । \newline
54. अ॒स्मै॒ ग्राम॒म् ग्राम॑ मस्मा अस्मै॒ ग्राम॒ मवाव॒ ग्राम॑ मस्मा अस्मै॒ ग्राम॒ मव॑ । \newline
55. ग्राम॒ मवाव॒ ग्राम॒म् ग्राम॒ मव॑ रुन्धे रु॒न्धे ऽव॒ ग्राम॒म् ग्राम॒ मव॑ रुन्धे । \newline
\pagebreak
\markright{ TS 3.4.9.3  \hfill https://www.vedavms.in \hfill}

\section{ TS 3.4.9.3 }

\textbf{TS 3.4.9.3 } \newline
\textbf{Samhita Paata} \newline

मव॑ रुन्धे मद्ध्य॒तो धा॒तारं॑ करोति मद्ध्य॒त ए॒वैनं॒ ग्राम॑स्य दधात्ये॒ता ए॒व निर्व॑पे॒ज्ज्योगा॑मयावी॒ छन्दाꣳ॑सि॒ वै देवि॑का॒श्छन्दाꣳ॑सि॒ खलु॒ वा ए॒तम॒भि म॑न्यन्ते॒ यस्य॒ ज्योगा॒मय॑ति॒ छन्दो॑भिरे॒वैन॑-मग॒दं क॑रोति मद्ध्य॒तो धा॒तारं॑ करोति मद्ध्य॒तो वा ए॒तस्याक्लृ॑प्तं॒ ॅयस्य॒ ज्योगा॒मय॑ति मद्ध्य॒त ए॒वास्य॒ तेन॑ कल्पयत्ये॒ता ए॒व नि - [  ] \newline

\textbf{Pada Paata} \newline

अवेति॑ । रु॒न्धे॒ । म॒द्ध्य॒तः । धा॒तार᳚म् । क॒रो॒ति॒ । म॒द्ध्य॒तः । ए॒व । ए॒न॒म् । ग्राम॑स्य । द॒धा॒ति॒ । ए॒ताः । ए॒व । निरिति॑ । व॒पे॒त् । ज्योगा॑मया॒वीति॒ ज्योक् - आ॒म॒या॒वी॒ । छन्दाꣳ॑सि । वै । देवि॑काः । छन्दाꣳ॑सि । खलु॑ । वै । ए॒तम् । अ॒भीति॑ । म॒न्य॒न्ते॒ । यस्य॑ । ज्योक् । आ॒मय॑ति । छन्दो॑भि॒रिति॒ छन्दः॑ - भिः॒ । ए॒व । ए॒न॒म् । अ॒ग॒दम् । क॒रो॒ति॒ । म॒द्ध्य॒तः । धा॒तार᳚म् । क॒रो॒ति॒ । म॒द्ध्य॒तः । वै । ए॒तस्य॑ । अक्लृ॑प्तम् । यस्य॑ । ज्योक् । आ॒मय॑ति । म॒द्ध्य॒तः । ए॒व । अ॒स्य॒ । तेन॑ । क॒ल्प॒य॒ति॒ । ए॒ताः । ए॒व । निरिति॑ ।  \newline


\textbf{Krama Paata} \newline

अव॑ रुन्धे । रु॒न्धे॒ म॒द्ध्य॒तः । म॒द्ध्य॒तो धा॒तार᳚म् । धा॒तार॑म् करोति । क॒रो॒ति॒ म॒द्ध्य॒तः । म॒द्ध्य॒त ए॒व । ए॒वैन᳚म् । ए॒न॒म् ग्राम॑स्य । ग्राम॑स्य दधाति । द॒ध॒त्ये॒ताः । ए॒ता ए॒व । ए॒व निः । निर् व॑पेत् । व॒पे॒ज् ज्योगा॑मयावी । ज्योगा॑मयावी॒ छन्दाꣳ॑सि । ज्योगा॑मया॒वीति॒ ज्योक् - आ॒म॒या॒वी॒ । छन्दाꣳ॑सि॒ वै । वै देवि॑काः । देवि॑का॒ श्छन्दाꣳ॑सि । छन्दाꣳ॑सि॒ खलु॑ । खलु॒ वै । वा ए॒तम् । ए॒तम॒भि । अ॒भि म॑न्यन्ते । म॒न्य॒न्ते॒ यस्य॑ । यस्य॒ ज्योक् । ज्योगा॒मय॑ति । आ॒मय॑ति॒ छन्दो॑भिः । छन्दो॑भिरे॒व । छन्दो॑भि॒रिति॒ छन्दः॑ - भिः॒ । ए॒वैन᳚म् । ए॒न॒म॒ग॒दम् । अ॒ग॒दम् क॑रोति । क॒रो॒ति॒ म॒द्ध्य॒तः । म॒द्ध्य॒तो धा॒तार᳚म् । धा॒तार॑म् करोति । क॒रो॒ति॒ म॒द्ध्य॒तः । म॒द्ध्य॒तो वै । वा ए॒तस्य॑ । ए॒तस्याक्लृ॑प्तम् । अक्लृ॑प्त॒म् ॅयस्य॑ । यस्य॒ ज्योक् । ज्योगा॒मय॑ति । आ॒मय॑ति मद्ध्य॒तः । म॒द्ध्य॒त ए॒व । ए॒वास्य॑ । अ॒स्य॒ तेन॑ । तेन॑ कल्पयति । क॒ल्प॒य॒त्ये॒ताः । ए॒ता ए॒व । ए॒व निः । निर् व॑पेत् \newline

\textbf{Jatai Paata} \newline

1. अव॑ रुन्धे रु॒न्धे ऽवाव॑ रुन्धे । \newline
2. रु॒न्धे॒ म॒द्ध्य॒तो म॑द्ध्य॒तो रु॑न्धे रुन्धे मद्ध्य॒तः । \newline
3. म॒द्ध्य॒तो धा॒तार॑म् धा॒तार॑म् मद्ध्य॒तो म॑द्ध्य॒तो धा॒तार᳚म् । \newline
4. धा॒तार॑म् करोति करोति धा॒तार॑म् धा॒तार॑म् करोति । \newline
5. क॒रो॒ति॒ म॒द्ध्य॒तो म॑द्ध्य॒तः क॑रोति करोति मद्ध्य॒तः । \newline
6. म॒द्ध्य॒त ए॒वैव म॑द्ध्य॒तो म॑द्ध्य॒त ए॒व । \newline
7. ए॒वैन॑ मेन मे॒वैवैन᳚म् । \newline
8. ए॒न॒म् ग्राम॑स्य॒ ग्राम॑स्यैन मेन॒म् ग्राम॑स्य । \newline
9. ग्राम॑स्य दधाति दधाति॒ ग्राम॑स्य॒ ग्राम॑स्य दधाति । \newline
10. द॒धा॒ त्ये॒ता ए॒ता द॑धाति दधा त्ये॒ताः । \newline
11. ए॒ता ए॒वैवैता ए॒ता ए॒व । \newline
12. ए॒व निर् णि रे॒वैव निः । \newline
13. निर् व॑पेद् वपे॒न् निर् णिर् व॑पेत् । \newline
14. व॒पे॒ज् ज्योगा॑मयावी॒ ज्योगा॑मयावी वपेद् वपे॒ज् ज्योगा॑मयावी । \newline
15. ज्योगा॑मयावी॒ छन्दाꣳ॑सि॒ छन्दाꣳ॑सि॒ ज्योगा॑मयावी॒ ज्योगा॑मयावी॒ छन्दाꣳ॑सि । \newline
16. ज्योगा॑मया॒वीति॒ ज्योक् - आ॒म॒या॒वी॒ । \newline
17. छन्दाꣳ॑सि॒ वै वै छन्दाꣳ॑सि॒ छन्दाꣳ॑सि॒ वै । \newline
18. वै देवि॑का॒ देवि॑का॒ वै वै देवि॑काः । \newline
19. देवि॑का॒ श्छन्दाꣳ॑सि॒ छन्दाꣳ॑सि॒ देवि॑का॒ देवि॑का॒ श्छन्दाꣳ॑सि । \newline
20. छन्दाꣳ॑सि॒ खलु॒ खलु॒ छन्दाꣳ॑सि॒ छन्दाꣳ॑सि॒ खलु॑ । \newline
21. खलु॒ वै वै खलु॒ खलु॒ वै । \newline
22. वा ए॒त मे॒तं ॅवै वा ए॒तम् । \newline
23. ए॒त म॒भ्या᳚(1॒)भ्ये॑त मे॒त म॒भि । \newline
24. अ॒भि म॑न्यन्ते मन्यन्ते॒ ऽभ्य॑भि म॑न्यन्ते । \newline
25. म॒न्य॒न्ते॒ यस्य॒ यस्य॑ मन्यन्ते मन्यन्ते॒ यस्य॑ । \newline
26. यस्य॒ ज्योग् ज्योग् यस्य॒ यस्य॒ ज्योक् । \newline
27. ज्यो गा॒मय॑ त्या॒मय॑ति॒ ज्योग् ज्यो गा॒मय॑ति । \newline
28. आ॒मय॑ति॒ छन्दो॑भि॒ श्छन्दो॑भि रा॒मय॑ त्या॒मय॑ति॒ छन्दो॑भिः । \newline
29. छन्दो॑भि रे॒वैव छन्दो॑भि॒ श्छन्दो॑भि रे॒व । \newline
30. छन्दो॑भि॒रिति॒ छन्दः॑ - भिः॒ । \newline
31. ए॒वैन॑ मेन मे॒वैवैन᳚म् । \newline
32. ए॒न॒ म॒ग॒द म॑ग॒द मे॑न मेन मग॒दम् । \newline
33. अ॒ग॒दम् क॑रोति करो त्यग॒द म॑ग॒दम् क॑रोति । \newline
34. क॒रो॒ति॒ म॒द्ध्य॒तो म॑द्ध्य॒तः क॑रोति करोति मद्ध्य॒तः । \newline
35. म॒द्ध्य॒तो धा॒तार॑म् धा॒तार॑म् मद्ध्य॒तो म॑द्ध्य॒तो धा॒तार᳚म् । \newline
36. धा॒तार॑म् करोति करोति धा॒तार॑म् धा॒तार॑म् करोति । \newline
37. क॒रो॒ति॒ म॒द्ध्य॒तो म॑द्ध्य॒तः क॑रोति करोति मद्ध्य॒तः । \newline
38. म॒द्ध्य॒तो वै वै म॑द्ध्य॒तो म॑द्ध्य॒तो वै । \newline
39. वा ए॒तस्यै॒ तस्य॒ वै वा ए॒तस्य॑ । \newline
40. ए॒तस्या क्लृ॑प्त॒ मक्लृ॑प्त मे॒त स्यै॒तस्या क्लृ॑प्तम् । \newline
41. अक्लृ॑प्तं॒ ॅयस्य॒ यस्या क्लृ॑प्त॒ मक्लृ॑प्तं॒ ॅयस्य॑ । \newline
42. यस्य॒ ज्योग् ज्योग् यस्य॒ यस्य॒ ज्योक् । \newline
43. ज्यो गा॒मय॑ त्या॒मय॑ति॒ ज्योग् ज्यो गा॒मय॑ति । \newline
44. आ॒मय॑ति मद्ध्य॒तो म॑द्ध्य॒त आ॒मय॑ त्या॒मय॑ति मद्ध्य॒तः । \newline
45. म॒द्ध्य॒त ए॒वैव म॑द्ध्य॒तो म॑द्ध्य॒त ए॒व । \newline
46. ए॒वा स्या᳚ स्यै॒वै वास्य॑ । \newline
47. अ॒स्य॒ तेन॒ तेना᳚ स्यास्य॒ तेन॑ । \newline
48. तेन॑ कल्पयति कल्पयति॒ तेन॒ तेन॑ कल्पयति । \newline
49. क॒ल्प॒य॒ त्ये॒ता ए॒ताः क॑ल्पयति कल्पय त्ये॒ताः । \newline
50. ए॒ता ए॒वैवैता ए॒ता ए॒व । \newline
51. ए॒व निर् णि रे॒वैव निः । \newline
52. निर् व॑पेद् वपे॒न् निर् णिर् व॑पेत् । \newline

\textbf{Ghana Paata } \newline

1. अव॑ रुन्धे रु॒न्धे ऽवाव॑ रुन्धे मद्ध्य॒तो म॑द्ध्य॒तो रु॒न्धे ऽवाव॑ रुन्धे मद्ध्य॒तः । \newline
2. रु॒न्धे॒ म॒द्ध्य॒तो म॑द्ध्य॒तो रु॑न्धे रुन्धे मद्ध्य॒तो धा॒तार॑म् धा॒तार॑म् मद्ध्य॒तो रु॑न्धे रुन्धे मद्ध्य॒तो धा॒तार᳚म् । \newline
3. म॒द्ध्य॒तो धा॒तार॑म् धा॒तार॑म् मद्ध्य॒तो म॑द्ध्य॒तो धा॒तार॑म् करोति करोति धा॒तार॑म् मद्ध्य॒तो म॑द्ध्य॒तो धा॒तार॑म् करोति । \newline
4. धा॒तार॑म् करोति करोति धा॒तार॑म् धा॒तार॑म् करोति मद्ध्य॒तो म॑द्ध्य॒तः क॑रोति धा॒तार॑म् धा॒तार॑म् करोति मद्ध्य॒तः । \newline
5. क॒रो॒ति॒ म॒द्ध्य॒तो म॑द्ध्य॒तः क॑रोति करोति मद्ध्य॒त ए॒वैव म॑द्ध्य॒तः क॑रोति करोति मद्ध्य॒त ए॒व । \newline
6. म॒द्ध्य॒त ए॒वैव म॑द्ध्य॒तो म॑द्ध्य॒त ए॒वैन॑ मेन मे॒व म॑द्ध्य॒तो म॑द्ध्य॒त ए॒वैन᳚म् । \newline
7. ए॒वैन॑ मेन मे॒वैवैन॒म् ग्राम॑स्य॒ ग्राम॑स्यैन मे॒वैवैन॒म् ग्राम॑स्य । \newline
8. ए॒न॒म् ग्राम॑स्य॒ ग्राम॑ स्यैनमेन॒म् ग्राम॑स्य दधाति दधाति॒ ग्राम॑ स्यैनमेन॒म् ग्राम॑स्य दधाति । \newline
9. ग्राम॑स्य दधाति दधाति॒ ग्राम॑स्य॒ ग्राम॑स्य दधा त्ये॒ता ए॒ता द॑धाति॒ ग्राम॑स्य॒ ग्राम॑स्य दधा त्ये॒ताः । \newline
10. द॒धा॒ त्ये॒ता ए॒ता द॑धाति दधा त्ये॒ता ए॒वैवैता द॑धाति दधा त्ये॒ता ए॒व । \newline
11. ए॒ता ए॒वैवैता ए॒ता ए॒व निर् णि रे॒वैता ए॒ता ए॒व निः । \newline
12. ए॒व निर् णि रे॒वैव निर् व॑पेद् वपे॒न् नि रे॒वैव निर् व॑पेत् । \newline
13. निर् व॑पेद् वपे॒न् निर् णिर् व॑पे॒ज् ज्योगा॑मयावी॒ ज्योगा॑मयावी वपे॒न् निर् णिर् व॑पे॒ज् ज्योगा॑मयावी । \newline
14. व॒पे॒ज् ज्योगा॑मयावी॒ ज्योगा॑मयावी वपेद् वपे॒ज् ज्योगा॑मयावी॒ छन्दाꣳ॑सि॒ छन्दाꣳ॑सि॒ ज्योगा॑मयावी वपेद् वपे॒ज् ज्योगा॑मयावी॒ छन्दाꣳ॑सि । \newline
15. ज्योगा॑मयावी॒ छन्दाꣳ॑सि॒ छन्दाꣳ॑सि॒ ज्योगा॑मयावी॒ ज्योगा॑मयावी॒ छन्दाꣳ॑सि॒ वै वै छन्दाꣳ॑सि॒ ज्योगा॑मयावी॒ ज्योगा॑मयावी॒ छन्दाꣳ॑सि॒ वै । \newline
16. ज्योगा॑मया॒वीति॒ ज्योक् - आ॒म॒या॒वी॒ । \newline
17. छन्दाꣳ॑सि॒ वै वै छन्दाꣳ॑सि॒ छन्दाꣳ॑सि॒ वै देवि॑का॒ देवि॑का॒ वै छन्दाꣳ॑सि॒ छन्दाꣳ॑सि॒ वै देवि॑काः । \newline
18. वै देवि॑का॒ देवि॑का॒ वै वै देवि॑का॒ श्छन्दाꣳ॑सि॒ छन्दाꣳ॑सि॒ देवि॑का॒ वै वै देवि॑का॒ श्छन्दाꣳ॑सि । \newline
19. देवि॑का॒ श्छन्दाꣳ॑सि॒ छन्दाꣳ॑सि॒ देवि॑का॒ देवि॑का॒ श्छन्दाꣳ॑सि॒ खलु॒ खलु॒ छन्दाꣳ॑सि॒ देवि॑का॒ देवि॑का॒ श्छन्दाꣳ॑सि॒ खलु॑ । \newline
20. छन्दाꣳ॑सि॒ खलु॒ खलु॒ छन्दाꣳ॑सि॒ छन्दाꣳ॑सि॒ खलु॒ वै वै खलु॒ छन्दाꣳ॑सि॒ छन्दाꣳ॑सि॒ खलु॒ वै । \newline
21. खलु॒ वै वै खलु॒ खलु॒ वा ए॒त मे॒तम् ॅवै खलु॒ खलु॒ वा ए॒तम् । \newline
22. वा ए॒त मे॒तम् ॅवै वा ए॒त म॒भ्या᳚(1॒)भ्ये॑तम् ॅवै वा ए॒त म॒भि । \newline
23. ए॒त म॒भ्या᳚(1॒)भ्ये॑त मे॒त म॒भि म॑न्यन्ते मन्यन्ते॒ ऽभ्ये॑त मे॒त म॒भि म॑न्यन्ते । \newline
24. अ॒भि म॑न्यन्ते मन्यन्ते॒ ऽभ्य॑भि म॑न्यन्ते॒ यस्य॒ यस्य॑ मन्यन्ते॒ ऽभ्य॑भि म॑न्यन्ते॒ यस्य॑ । \newline
25. म॒न्य॒न्ते॒ यस्य॒ यस्य॑ मन्यन्ते मन्यन्ते॒ यस्य॒ ज्योग् ज्योग् यस्य॑ मन्यन्ते मन्यन्ते॒ यस्य॒ ज्योक् । \newline
26. यस्य॒ ज्योग् ज्योग् यस्य॒ यस्य॒ ज्योगा॒मय॑ त्या॒मय॑ति॒ ज्योग् यस्य॒ यस्य॒ ज्योगा॒मय॑ति । \newline
27. ज्योगा॒मय॑ त्या॒मय॑ति॒ ज्योग् ज्योगा॒ मय॑ति॒ छन्दो॑भि॒ श्छन्दो॑भि रा॒मय॑ति॒ ज्योग् ज्योगा॒ मय॑ति॒ छन्दो॑भिः । \newline
28. आ॒मय॑ति॒ छन्दो॑भि॒ श्छन्दो॑भि रा॒मय॑ त्या॒मय॑ति॒ छन्दो॑भि रे॒वैव छन्दो॑भि रा॒मय॑ त्या॒मय॑ति॒ छन्दो॑भि रे॒व । \newline
29. छन्दो॑भि रे॒वैव छन्दो॑भि॒ श्छन्दो॑भि रे॒वैन॑ मेन मे॒व छन्दो॑भि॒ श्छन्दो॑भि रे॒वैन᳚म् । \newline
30. छन्दो॑भि॒रिति॒ छन्दः॑ - भिः॒ । \newline
31. ए॒वैन॑ मेन मे॒वैवैन॑ मग॒द म॑ग॒द मे॑न मे॒वैवैन॑ मग॒दम् । \newline
32. ए॒न॒ म॒ग॒द म॑ग॒द मे॑न मेन मग॒दम् क॑रोति करो त्यग॒द मे॑न मेन मग॒दम् क॑रोति । \newline
33. अ॒ग॒दम् क॑रोति करो त्यग॒द म॑ग॒दम् क॑रोति मद्ध्य॒तो म॑द्ध्य॒तः क॑रो त्यग॒द म॑ग॒दम् क॑रोति मद्ध्य॒तः । \newline
34. क॒रो॒ति॒ म॒द्ध्य॒तो म॑द्ध्य॒तः क॑रोति करोति मद्ध्य॒तो धा॒तार॑म् धा॒तार॑म् मद्ध्य॒तः क॑रोति करोति मद्ध्य॒तो धा॒तार᳚म् । \newline
35. म॒द्ध्य॒तो धा॒तार॑म् धा॒तार॑म् मद्ध्य॒तो म॑द्ध्य॒तो धा॒तार॑म् करोति करोति धा॒तार॑म् मद्ध्य॒तो म॑द्ध्य॒तो धा॒तार॑म् करोति । \newline
36. धा॒तार॑म् करोति करोति धा॒तार॑म् धा॒तार॑म् करोति मद्ध्य॒तो म॑द्ध्य॒तः क॑रोति धा॒तार॑म् धा॒तार॑म् करोति मद्ध्य॒तः । \newline
37. क॒रो॒ति॒ म॒द्ध्य॒तो म॑द्ध्य॒तः क॑रोति करोति मद्ध्य॒तो वै वै म॑द्ध्य॒तः क॑रोति करोति मद्ध्य॒तो वै । \newline
38. म॒द्ध्य॒तो वै वै म॑द्ध्य॒तो म॑द्ध्य॒तो वा ए॒त स्यै॒तस्य॒ वै म॑द्ध्य॒तो म॑द्ध्य॒तो वा ए॒तस्य॑ । \newline
39. वा ए॒तस्यै॒तस्य॒ वै वा ए॒तस्या क्लृ॑प्त॒ मक्लृ॑प्त मे॒तस्य॒ वै वा ए॒तस्या क्लृ॑प्तम् । \newline
40. ए॒तस्या क्लृ॑प्त॒ मक्लृ॑प्त मे॒त स्यै॒तस्या क्लृ॑प्त॒म् ॅयस्य॒ यस्या क्लृ॑प्त मे॒त स्यै॒तस्या क्लृ॑प्त॒म् ॅयस्य॑ । \newline
41. अक्लृ॑प्त॒म् ॅयस्य॒ यस्या क्लृ॑प्त॒ मक्लृ॑प्त॒म् ॅयस्य॒ ज्योग् ज्योग् यस्या क्लृ॑प्त॒ मक्लृ॑प्त॒म् ॅयस्य॒ ज्योक् । \newline
42. यस्य॒ ज्योग् ज्योग् यस्य॒ यस्य॒ ज्योगा॒मय॑ त्या॒मय॑ति॒ ज्योग् यस्य॒ यस्य॒ ज्योगा॒मय॑ति । \newline
43. ज्योगा॒ मय॑ त्या॒मय॑ति॒ ज्योग् ज्योगा॒मय॑ति मद्ध्य॒तो म॑द्ध्य॒त आ॒मय॑ति॒ ज्योग् ज्योगा॒मय॑ति मद्ध्य॒तः । \newline
44. आ॒मय॑ति मद्ध्य॒तो म॑द्ध्य॒त आ॒मय॑ त्या॒मय॑ति मद्ध्य॒त ए॒वैव म॑द्ध्य॒त आ॒मय॑ त्या॒मय॑ति मद्ध्य॒त ए॒व । \newline
45. म॒द्ध्य॒त ए॒वैव म॑द्ध्य॒तो म॑द्ध्य॒त ए॒वास्या᳚स्यै॒व म॑द्ध्य॒तो म॑द्ध्य॒त ए॒वास्य॑ । \newline
46. ए॒वास्या᳚ स्यै॒वैवास्य॒ तेन॒ तेना᳚ स्यै॒वैवास्य॒ तेन॑ । \newline
47. अ॒स्य॒ तेन॒ तेना᳚ स्यास्य॒ तेन॑ कल्पयति कल्पयति॒ तेना᳚ स्यास्य॒ तेन॑ कल्पयति । \newline
48. तेन॑ कल्पयति कल्पयति॒ तेन॒ तेन॑ कल्पय त्ये॒ता ए॒ताः क॑ल्पयति॒ तेन॒ तेन॑ कल्पय त्ये॒ताः । \newline
49. क॒ल्प॒य॒ त्ये॒ता ए॒ताः क॑ल्पयति कल्पय त्ये॒ता ए॒वैवैताः क॑ल्पयति कल्पय त्ये॒ता ए॒व । \newline
50. ए॒ता ए॒वैवैता ए॒ता ए॒व निर् णि रे॒वैता ए॒ता ए॒व निः । \newline
51. ए॒व निर् णि रे॒वैव निर् व॑पेद् वपे॒न् निरे॒ वैव निर् व॑पेत् । \newline
52. निर् व॑पेद् वपे॒न् निर् णिर् व॑पे॒द् यम् ॅयम् ॅव॑पे॒न् निर् णिर् व॑पे॒द् यम् । \newline
\pagebreak
\markright{ TS 3.4.9.4  \hfill https://www.vedavms.in \hfill}

\section{ TS 3.4.9.4 }

\textbf{TS 3.4.9.4 } \newline
\textbf{Samhita Paata} \newline

र्व॑पे॒द्यं ॅय॒ज्ञो नोप॒नमे॒च्छन्दाꣳ॑सि॒ वै देवि॑का॒श्छन्दाꣳ॑सि॒ खलु॒ वा ए॒तं नोप॑ नमन्ति॒ यं ॅय॒ज्ञो नोप॒नम॑ति प्रथ॒मं धा॒तारं॑ करोति मुख॒त ए॒वास्मै॒ छन्दाꣳ॑सि दधा॒त्युपै॑नं ॅय॒ज्ञो न॑मत्ये॒ता ए॒व निव॑र्पेदीजा॒नश्छन्दाꣳ॑सि॒ वै देवि॑का या॒तया॑मानीव॒ खलु॒ वा ए॒तस्य॒ छन्दाꣳ॑सि॒ य ई॑जा॒न उ॑त्त॒मं धा॒तारं॑ करो - [  ] \newline

\textbf{Pada Paata} \newline

व॒पे॒त् । यम् । य॒ज्ञ्ः । न । उ॒प॒नमे॒दित्यु॑प - नमे᳚त् । छन्दाꣳ॑सि । वै । देवि॑काः । छन्दाꣳ॑सि । खलु॑ । वै । ए॒तम् । न । उपेति॑ । न॒म॒न्ति॒ । यम् । य॒ज्ञ्ः । न । उ॒प॒नम॒तीत्यु॑प - नम॑ति । प्र॒थ॒मम् । धा॒तार᳚म् । क॒रो॒ति॒ । मु॒ख॒तः । ए॒व । अ॒स्मै॒ । छन्दाꣳ॑सि । द॒धा॒ति॒ । उपेति॑ । ए॒न॒म् । य॒ज्ञ्ः । न॒म॒ति॒ । ए॒ताः । ए॒व । निरिति॑ । व॒पे॒त् । ई॒जा॒नः । छन्दाꣳ॑सि । वै । देवि॑काः । या॒तया॑मा॒नीति॑ या॒त - या॒मा॒नि॒ । इ॒व॒ । खलु॑ । वै । ए॒तस्य॑ । छन्दाꣳ॑सि । यः । ई॒जा॒नः । उ॒त्त॒ममित्यु॑त् - त॒मम् । धा॒तार᳚म् । क॒रो॒ति॒ ।  \newline


\textbf{Krama Paata} \newline

व॒पे॒द् यम् । यं ॅय॒ज्ञ्ः । य॒ज्ञो न । नोप॒नमे᳚त् । उ॒प॒नमे॒च्छन्दाꣳ॑सि । उ॒प॒नमे॒दित्यु॑प - नमे᳚त् । छन्दाꣳ॑सि॒ वै । वै देवि॑काः । देवि॑का॒ श्छन्दाꣳ॑सि । छन्दाꣳ॑सि॒ खलु॑ । खलु॒ वै । वा ए॒तम् । ए॒तम् न । नोप॑ । उप॑ नमन्ति । न॒म॒न्ति॒ यम् । यं ॅय॒ज्ञ्ः । य॒ज्ञो न । नोप॒नम॑ति । उ॒प॒नम॑ति प्रथ॒मम् । उ॒प॒नम॒तीत्यु॑प - नम॑ति । प्र॒थ॒मम् धा॒तार᳚म् । धा॒तार॑म् करोति । क॒रो॒ति॒ मु॒ख॒तः । मु॒ख॒त ए॒व । ए॒वास्मै᳚ । अ॒स्मै॒ छन्दाꣳ॑सि । छन्दाꣳ॑सि दधाति । द॒धा॒त्युप॑ । उपै॑नम् । ए॒नं॒ ॅय॒ज्ञ्ः । य॒ज्ञो न॑मति । न॒म॒त्ये॒ताः । ए॒ता ए॒व । ए॒व निः । निर् व॑पेत् । व॒पे॒दी॒जा॒नः । ई॒जा॒न श्छन्दाꣳ॑सि । छन्दाꣳ॑सि॒ वै । वै देवि॑काः । देवि॑का या॒तया॑मानि । या॒तया॑मानीव । या॒तया॑मा॒नीति॑ या॒त - या॒मा॒नि॒ । इ॒व॒ खलु॑ । खलु॒ वै । वा ए॒तस्य॑ । ए॒तस्य॒ छन्दाꣳ॑सि । छन्दाꣳ॑सि॒ यः । य ई॑जा॒नः । ई॒जा॒न उ॑त्त॒मम् । उ॒त्त॒मम् धा॒तार᳚म् । उ॒त्त॒ममित्यु॑त् - त॒मम् । धा॒तार॑म् करोति । क॒रो॒त्यु॒परि॑ष्टात् \newline

\textbf{Jatai Paata} \newline

1. व॒पे॒द् यं ॅयं ॅव॑पेद् वपे॒द् यम् । \newline
2. यं ॅय॒ज्ञो य॒ज्ञो यं ॅयं ॅय॒ज्ञ्ः । \newline
3. य॒ज्ञो न न य॒ज्ञो य॒ज्ञो न । \newline
4. नो प॒नमे॑ दुप॒नमे॒न् न नो प॒नमे᳚त् । \newline
5. उ॒प॒नमे॒च् छन्दाꣳ॑सि॒ छन्दाꣳ॑स्यु प॒नमे॑ दुप॒नमे॒च् छन्दाꣳ॑सि । \newline
6. उ॒प॒नमे॒दित्यु॑प - नमे᳚त् । \newline
7. छन्दाꣳ॑सि॒ वै वै छन्दाꣳ॑सि॒ छन्दाꣳ॑सि॒ वै । \newline
8. वै देवि॑का॒ देवि॑का॒ वै वै देवि॑काः । \newline
9. देवि॑का॒ श्छन्दाꣳ॑सि॒ छन्दाꣳ॑सि॒ देवि॑का॒ देवि॑का॒ श्छन्दाꣳ॑सि । \newline
10. छन्दाꣳ॑सि॒ खलु॒ खलु॒ छन्दाꣳ॑सि॒ छन्दाꣳ॑सि॒ खलु॑ । \newline
11. खलु॒ वै वै खलु॒ खलु॒ वै । \newline
12. वा ए॒त मे॒तं ॅवै वा ए॒तम् । \newline
13. ए॒तन् न नैत मे॒तन् न । \newline
14. नो पोप॒ न नोप॑ । \newline
15. उप॑ नमन्ति नम॒ न्त्युपोप॑ नमन्ति । \newline
16. न॒म॒न्ति॒ यं ॅयन्-न॑मन्ति नमन्ति॒ यम् । \newline
17. यं ॅय॒ज्ञो य॒ज्ञो यं ॅयं ॅय॒ज्ञ्ः । \newline
18. य॒ज्ञो न न य॒ज्ञो य॒ज्ञो न । \newline
19. नो प॒नम॑ त्युप॒नम॑ति॒ न नो प॒नम॑ति । \newline
20. उ॒प॒नम॑ति प्रथ॒मम् प्र॑थ॒म मु॑प॒नम॑ त्युप॒नम॑ति प्रथ॒मम् । \newline
21. उ॒प॒नम॒तीत्यु॑प - नम॑ति । \newline
22. प्र॒थ॒मम् धा॒तार॑म् धा॒तार॑म् प्रथ॒मम् प्र॑थ॒मम् धा॒तार᳚म् । \newline
23. धा॒तार॑म् करोति करोति धा॒तार॑म् धा॒तार॑म् करोति । \newline
24. क॒रो॒ति॒ मु॒ख॒तो मु॑ख॒तः क॑रोति करोति मुख॒तः । \newline
25. मु॒ख॒त ए॒वैव मु॑ख॒तो मु॑ख॒त ए॒व । \newline
26. ए॒वास्मा॑ अस्मा ए॒वै वास्मै᳚ । \newline
27. अ॒स्मै॒ छन्दाꣳ॑सि॒ छन्दाꣳ॑स्यस्मा अस्मै॒ छन्दाꣳ॑सि । \newline
28. छन्दाꣳ॑सि दधाति दधाति॒ छन्दाꣳ॑सि॒ छन्दाꣳ॑सि दधाति । \newline
29. द॒धा॒ त्युपोप॑ दधाति दधा॒ त्युप॑ । \newline
30. उपै॑न मेन॒ मुपो पै॑नम् । \newline
31. ए॒नं॒ ॅय॒ज्ञो य॒ज्ञ् ए॑न मेनं ॅय॒ज्ञ्ः । \newline
32. य॒ज्ञो न॑मति नमति य॒ज्ञो य॒ज्ञो न॑मति । \newline
33. न॒म॒ त्ये॒ता ए॒ता न॑मति नम त्ये॒ताः । \newline
34. ए॒ता ए॒वै वैता ए॒ता ए॒व । \newline
35. ए॒व निर् णि रे॒वैव निः । \newline
36. निर् व॑पेद् वपे॒न् निर् णिर् व॑पेत् । \newline
37. व॒पे॒ दी॒जा॒न ई॑जा॒नो व॑पेद् वपे दीजा॒नः । \newline
38. ई॒जा॒न श्छन्दाꣳ॑सि॒ छन्दाꣳ॑सी जा॒न ई॑जा॒न श्छन्दाꣳ॑सि । \newline
39. छन्दाꣳ॑सि॒ वै वै छन्दाꣳ॑सि॒ छन्दाꣳ॑सि॒ वै । \newline
40. वै देवि॑का॒ देवि॑का॒ वै वै देवि॑काः । \newline
41. देवि॑का या॒तया॑मानि या॒तया॑मानि॒ देवि॑का॒ देवि॑का या॒तया॑मानि । \newline
42. या॒तया॑मा नीवेव या॒तया॑मानि या॒तया॑मा नीव । \newline
43. या॒तया॑मा॒नीति॑ या॒त - या॒मा॒नि॒ । \newline
44. इ॒व॒ खलु॒ खल्वि॑वेव॒ खलु॑ । \newline
45. खलु॒ वै वै खलु॒ खलु॒ वै । \newline
46. वा ए॒तस्यै॒तस्य॒ वै वा ए॒तस्य॑ । \newline
47. ए॒तस्य॒ छन्दाꣳ॑सि॒ छन्दाꣳ॑ स्ये॒त स्यै॒तस्य॒ छन्दाꣳ॑सि । \newline
48. छन्दाꣳ॑सि॒ यो य श्छन्दाꣳ॑सि॒ छन्दाꣳ॑सि॒ यः । \newline
49. य ई॑जा॒न ई॑जा॒नो यो य ई॑जा॒नः । \newline
50. ई॒जा॒न उ॑त्त॒म मु॑त्त॒म मी॑जा॒न ई॑जा॒न उ॑त्त॒मम् । \newline
51. उ॒त्त॒मम् धा॒तार॑म् धा॒तार॑ मुत्त॒म मु॑त्त॒मम् धा॒तार᳚म् । \newline
52. उ॒त्त॒ममित्यु॑त् - त॒मम् । \newline
53. धा॒तार॑म् करोति करोति धा॒तार॑म् धा॒तार॑म् करोति । \newline
54. क॒रो॒ त्यु॒परि॑ष्टा दु॒परि॑ष्टात् करोति करो त्यु॒परि॑ष्टात् । \newline

\textbf{Ghana Paata } \newline

1. व॒पे॒द् यम् ॅयम् ॅव॑पेद् वपे॒द् यम् ॅय॒ज्ञो य॒ज्ञो यम् ॅव॑पेद् वपे॒द् यम् ॅय॒ज्ञ्ः । \newline
2. यम् ॅय॒ज्ञो य॒ज्ञो यम् ॅयम् ॅय॒ज्ञो न न य॒ज्ञो यम् ॅयम् ॅय॒ज्ञो न । \newline
3. य॒ज्ञो न न य॒ज्ञो य॒ज्ञो नोप॒नमे॑ दुप॒नमे॒न् न य॒ज्ञो य॒ज्ञो नोप॒नमे᳚त् । \newline
4. नो प॒नमे॑ दुप॒नमे॒न् न नोप॒नमे॒च् छन्दाꣳ॑सि॒ छन्दाꣳ॑ स्युप॒नमे॒न् न नोप॒नमे॒च् छन्दाꣳ॑सि । \newline
5. उ॒प॒नमे॒च् छन्दाꣳ॑सि॒ छन्दाꣳ॑ स्युप॒नमे॑ दुप॒नमे॒च् छन्दाꣳ॑सि॒ वै वै छन्दाꣳ॑ स्युप॒नमे॑ दुप॒नमे॒च् छन्दाꣳ॑सि॒ वै । \newline
6. उ॒प॒नमे॒दित्यु॑प - नमे᳚त् । \newline
7. छन्दाꣳ॑सि॒ वै वै छन्दाꣳ॑सि॒ छन्दाꣳ॑सि॒ वै देवि॑का॒ देवि॑का॒ वै छन्दाꣳ॑सि॒ छन्दाꣳ॑सि॒ वै देवि॑काः । \newline
8. वै देवि॑का॒ देवि॑का॒ वै वै देवि॑का॒ श्छन्दाꣳ॑सि॒ छन्दाꣳ॑सि॒ देवि॑का॒ वै वै देवि॑का॒ श्छन्दाꣳ॑सि । \newline
9. देवि॑का॒ श्छन्दाꣳ॑सि॒ छन्दाꣳ॑सि॒ देवि॑का॒ देवि॑का॒ श्छन्दाꣳ॑सि॒ खलु॒ खलु॒ छन्दाꣳ॑सि॒ देवि॑का॒ देवि॑का॒ श्छन्दाꣳ॑सि॒ खलु॑ । \newline
10. छन्दाꣳ॑सि॒ खलु॒ खलु॒ छन्दाꣳ॑सि॒ छन्दाꣳ॑सि॒ खलु॒ वै वै खलु॒ छन्दाꣳ॑सि॒ छन्दाꣳ॑सि॒ खलु॒ वै । \newline
11. खलु॒ वै वै खलु॒ खलु॒ वा ए॒त मे॒तम् ॅवै खलु॒ खलु॒ वा ए॒तम् । \newline
12. वा ए॒त मे॒तम् ॅवै वा ए॒तम् न नैतम् ॅवै वा ए॒तम् न । \newline
13. ए॒तम् न नैत मे॒तम् नोपोप॒ नैत मे॒तम् नोप॑ । \newline
14. नो पोप॒ ननोप॑ नमन्ति नम॒ न्त्युप॒ ननोप॑ नमन्ति । \newline
15. उप॑ नमन्ति नम॒ न्त्युपोप॑ नमन्ति॒ यम् ॅयम्,न॑म॒ न्त्युपोप॑ नमन्ति॒ यम् । \newline
16. न॒म॒न्ति॒ यम् ॅयम्,न॑मन्ति नमन्ति॒ यम् ॅय॒ज्ञो य॒ज्ञो यम् न॑मन्ति नमन्ति॒ यम् ॅय॒ज्ञ्ः । \newline
17. यम् ॅय॒ज्ञो य॒ज्ञो यम् ॅयम् ॅय॒ज्ञो न न य॒ज्ञो यम् ॅयम् ॅय॒ज्ञो न । \newline
18. य॒ज्ञो न न य॒ज्ञो य॒ज्ञो नोप॒नम॑ त्युप॒नम॑ति॒ न य॒ज्ञो य॒ज्ञो नोप॒नम॑ति । \newline
19. नोप॒नम॑ त्युप॒नम॑ति॒ न नोप॒नम॑ति प्रथ॒मम् प्र॑थ॒म मु॑प॒नम॑ति॒ न नोप॒नम॑ति प्रथ॒मम् । \newline
20. उ॒प॒नम॑ति प्रथ॒मम् प्र॑थ॒म मु॑प॒नम॑ त्युप॒नम॑ति प्रथ॒मम् धा॒तार॑म् धा॒तार॑म् प्रथ॒म मु॑प॒नम॑ त्युप॒नम॑ति प्रथ॒मम् धा॒तार᳚म् । \newline
21. उ॒प॒नम॒तीत्यु॑प - नम॑ति । \newline
22. प्र॒थ॒मम् धा॒तार॑म् धा॒तार॑म् प्रथ॒मम् प्र॑थ॒मम् धा॒तार॑म् करोति करोति धा॒तार॑म् प्रथ॒मम् प्र॑थ॒मम् धा॒तार॑म् करोति । \newline
23. धा॒तार॑म् करोति करोति धा॒तार॑म् धा॒तार॑म् करोति मुख॒तो मु॑ख॒तः क॑रोति धा॒तार॑म् धा॒तार॑म् करोति मुख॒तः । \newline
24. क॒रो॒ति॒ मु॒ख॒तो मु॑ख॒तः क॑रोति करोति मुख॒त ए॒वैव मु॑ख॒तः क॑रोति करोति मुख॒त ए॒व । \newline
25. मु॒ख॒त ए॒वैव मु॑ख॒तो मु॑ख॒त ए॒वास्मा॑ अस्मा ए॒व मु॑ख॒तो मु॑ख॒त ए॒वास्मै᳚ । \newline
26. ए॒वास्मा॑ अस्मा ए॒वैवास्मै॒ छन्दाꣳ॑सि॒ छन्दाꣳ॑ स्यस्मा ए॒वैवास्मै॒ छन्दाꣳ॑सि । \newline
27. अ॒स्मै॒ छन्दाꣳ॑सि॒ छन्दाꣳ॑ स्यस्मा अस्मै॒ छन्दाꣳ॑सि दधाति दधाति॒ छन्दाꣳ॑ स्यस्मा अस्मै॒ छन्दाꣳ॑सि दधाति । \newline
28. छन्दाꣳ॑सि दधाति दधाति॒ छन्दाꣳ॑सि॒ छन्दाꣳ॑सि दधा॒ त्युपोप॑ दधाति॒ छन्दाꣳ॑सि॒ छन्दाꣳ॑सि दधा॒त्युप॑ । \newline
29. द॒धा॒ त्युपोप॑ दधाति दधा॒ त्युपै॑न मेन॒ मुप॑ दधाति दधा॒ त्युपै॑नम् । \newline
30. उपै॑न मेन॒ मुपोपै॑नम् ॅय॒ज्ञो य॒ज्ञ् ए॑न॒ मुपोपै॑नम् ॅय॒ज्ञ्ः । \newline
31. ए॒न॒म् ॅय॒ज्ञो य॒ज्ञ् ए॑न मेनम् ॅय॒ज्ञो न॑मति नमति य॒ज्ञ् ए॑न मेनम् ॅय॒ज्ञो न॑मति । \newline
32. य॒ज्ञो न॑मति नमति य॒ज्ञो य॒ज्ञो न॑म त्ये॒ता ए॒ता न॑मति य॒ज्ञो य॒ज्ञो न॑म त्ये॒ताः । \newline
33. न॒म॒ त्ये॒ता ए॒ता न॑मति नम त्ये॒ता ए॒वैवैता न॑मति नम त्ये॒ता ए॒व । \newline
34. ए॒ता ए॒वैवैता ए॒ता ए॒व निर् णिरे॒वैता ए॒ता ए॒व निः । \newline
35. ए॒व निर् णि रे॒वैव निर् व॑पेद् वपे॒न् नि रे॒वैव निर् व॑पेत् । \newline
36. निर् व॑पेद् वपे॒न् निर् णिर् व॑पे दीजा॒न ई॑जा॒नो व॑पे॒न् निर् णिर् व॑पे दीजा॒नः । \newline
37. व॒पे॒दी॒जा॒न ई॑जा॒नो व॑पेद् वपेद् ईजा॒न श्छन्दाꣳ॑सि॒ छन्दाꣳ॑सीजा॒नो व॑पेद् वपेदीजा॒न श्छन्दाꣳ॑सि । \newline
38. ई॒जा॒न श्छन्दाꣳ॑सि॒ छन्दाꣳ॑सीजा॒न ई॑जा॒न श्छन्दाꣳ॑सि॒ वै वै छन्दाꣳ॑ सीजा॒न ई॑जा॒न श्छन्दाꣳ॑सि॒ वै । \newline
39. छन्दाꣳ॑सि॒ वै वै छन्दाꣳ॑सि॒ छन्दाꣳ॑सि॒ वै देवि॑का॒ देवि॑का॒ वै छन्दाꣳ॑सि॒ छन्दाꣳ॑सि॒ वै देवि॑काः । \newline
40. वै देवि॑का॒ देवि॑का॒ वै वै देवि॑का या॒तया॑मानि या॒तया॑मानि॒ देवि॑का॒ वै वै देवि॑का या॒तया॑मानि । \newline
41. देवि॑का या॒तया॑मानि या॒तया॑मानि॒ देवि॑का॒ देवि॑का या॒तया॑मानीवेव या॒तया॑मानि॒ देवि॑का॒ देवि॑का या॒तया॑मानीव । \newline
42. या॒तया॑मानीवेव या॒तया॑मानि या॒तया॑मानीव॒ खलु॒ खल्वि॑व या॒तया॑मानि या॒तया॑मानीव॒ खलु॑ । \newline
43. या॒तया॑मा॒नीति॑ या॒त - या॒मा॒नि॒ । \newline
44. इ॒व॒ खलु॒ खल्वि॑वेव॒ खलु॒ वै वै खल्वि॑वेव॒ खलु॒ वै । \newline
45. खलु॒ वै वै खलु॒ खलु॒ वा ए॒त स्यै॒तस्य॒ वै खलु॒ खलु॒ वा ए॒तस्य॑ । \newline
46. वा ए॒तस्यै॒तस्य॒ वै वा ए॒तस्य॒ छन्दाꣳ॑सि॒ छन्दाꣳ॑ स्ये॒तस्य॒ वै वा ए॒तस्य॒ छन्दाꣳ॑सि । \newline
47. ए॒तस्य॒ छन्दाꣳ॑सि॒ छन्दाꣳ॑ स्ये॒त स्यै॒तस्य॒ छन्दाꣳ॑सि॒ यो य श्छन्दाꣳ॑ स्ये॒त स्यै॒तस्य॒ छन्दाꣳ॑सि॒ यः । \newline
48. छन्दाꣳ॑सि॒ यो यश्छन्दाꣳ॑सि॒ छन्दाꣳ॑सि॒ य ई॑जा॒न ई॑जा॒नो यश्छन्दाꣳ॑सि॒ छन्दाꣳ॑सि॒ य ई॑जा॒नः । \newline
49. य ई॑जा॒न ई॑जा॒नो यो य ई॑जा॒न उ॑त्त॒म मु॑त्त॒म मी॑जा॒नो यो य ई॑जा॒न उ॑त्त॒मम् । \newline
50. ई॒जा॒न उ॑त्त॒म मु॑त्त॒म मी॑जा॒न ई॑जा॒न उ॑त्त॒मम् धा॒तार॑म् धा॒तार॑ मुत्त॒म मी॑जा॒न ई॑जा॒न उ॑त्त॒मम् धा॒तार᳚म् । \newline
51. उ॒त्त॒मम् धा॒तार॑म् धा॒तार॑ मुत्त॒म मु॑त्त॒मम् धा॒तार॑म् करोति करोति धा॒तार॑ मुत्त॒म मु॑त्त॒मम् धा॒तार॑म् करोति । \newline
52. उ॒त्त॒ममित्यु॑त् - त॒मम् । \newline
53. धा॒तार॑म् करोति करोति धा॒तार॑म् धा॒तार॑म् करो त्यु॒परि॑ष्टा दु॒परि॑ष्टात् करोति धा॒तार॑म् धा॒तार॑म् 
करो त्यु॒परि॑ष्टात् । \newline
54. क॒रो॒ त्यु॒परि॑ष्टा दु॒परि॑ष्टात् करोति करो त्यु॒परि॑ष्टा दे॒वैवोपरि॑ष्टात् करोति करो त्यु॒परि॑ष्टा दे॒व । \newline
\pagebreak
\markright{ TS 3.4.9.5  \hfill https://www.vedavms.in \hfill}

\section{ TS 3.4.9.5 }

\textbf{TS 3.4.9.5 } \newline
\textbf{Samhita Paata} \newline

-त्यु॒परि॑ष्टादे॒वास्मै॒ छन्दाꣳ॒॒स्यया॑तयामा॒न्यव॑ रुन्ध॒ उपै॑न॒मुत्त॑रो य॒ज्ञो न॑मत्ये॒ता ए॒व निव॑र्पे॒द्यं मे॒धा नोप॒नमे॒च्छन्दाꣳ॑सि॒ वै देवि॑का॒श्छन्दाꣳ॑सि॒ खलु॒ वा ए॒तं नोप॑ नमन्ति॒ यं मे॒धा नोप॒नम॑ति प्रथ॒मं धा॒तारं॑ करोति मुख॒त ए॒वास्मै॒ छन्दाꣳ॑सि दधा॒त्युपै॑नं मे॒धा न॑मत्ये॒ता ए॒व निव॑र्पे॒ - [  ] \newline

\textbf{Pada Paata} \newline

उ॒परि॑ष्टात् । ए॒व । अ॒स्मै॒ । छन्दाꣳ॑सि । अया॑तयामा॒नीत्यया॑त - या॒मा॒नि॒ । अवेति॑ । रु॒न्धे॒ । उपेति॑ । ए॒न॒म् । उत्त॑र॒ इत्युत् - त॒रः॒ । य॒ज्ञ्ः । न॒म॒ति॒ । ए॒ताः । ए॒व । निरिति॑ । व॒पे॒त् । यम् । मे॒धा । न । उ॒प॒नमे॒दित्यु॑प - नमे᳚त् । छन्दाꣳ॑सि । वै । देवि॑काः । छन्दाꣳ॑सि । खलु॑ । वै । ए॒तम् । न । उपेति॑ । न॒म॒न्ति॒ । यम् । मे॒धा । न । उ॒प॒नम॒तीत्यु॑प - नम॑ति । प्र॒थ॒मम् । धा॒तार᳚म् । क॒रो॒ति॒ । मु॒ख॒तः । ए॒व । अ॒स्मै॒ । छन्दाꣳ॑सि । द॒धा॒ति॒ । उपेति॑ । ए॒न॒म् । मे॒धा । न॒म॒ति॒ । ए॒ताः । ए॒व । निरिति॑ । व॒पे॒त् ।  \newline


\textbf{Krama Paata} \newline

उ॒परि॑ष्टादे॒व । ए॒वास्मै᳚ । अ॒स्मै॒ छन्दाꣳ॑सि । छन्दाꣳ॒॒स्यया॑तयामानि । अया॑तयामा॒न्यव॑ । अया॑तयामा॒नीत्यया॑त - या॒मा॒नि॒ । अव॑ रुन्धे । रु॒न्ध॒ उप॑ । उपै॑नम् । ए॒न॒मुत्त॑रः । उत्त॑रो य॒ज्ञ्ः । उत्त॑र॒ इत्युत् - त॒रः॒ । य॒ज्ञो न॑मति । न॒म॒त्ये॒ताः । ए॒ता ए॒व । ए॒व निः । निर् व॑पेत् । व॒पे॒द् यम् । यम् मे॒धा । मे॒धा न । नोप॒नमे᳚त् । उ॒प॒नमे॒च्छन्दाꣳ॑सि । उ॒प॒नमे॒दित्यु॑प - नमे᳚त् । छन्दाꣳ॑सि॒ वै । वै देवि॑काः । देवि॑का॒ श्छन्दाꣳ॑सि । छन्दाꣳ॑सि॒ खलु॑ । खलु॒ वै । वा ए॒तम् । ए॒तम् न । नोप॑ । उप॑ नमन्ति । न॒म॒न्ति॒ यम् । यम् मे॒धा । मे॒धा न । नोप॒नम॑ति । उ॒प॒नम॑ति प्रथ॒मम् । उ॒प॒नम॒तीत्यु॑प - नम॑ति । प्र॒थ॒मम् धा॒तार᳚म् । धा॒तार॑म् करोति । क॒रो॒ति॒ मु॒ख॒तः । मु॒ख॒त ए॒व । ए॒वास्मै᳚ । अ॒स्मै॒ छन्दाꣳ॑सि । छन्दाꣳ॑सि दधाति । द॒धा॒त्युप॑ । उपै॑नम् । ए॒न॒म् मे॒धा । मे॒धा न॑मति । न॒म॒त्ये॒ताः । ए॒ता ए॒व । ए॒व निः । निर् व॑पेत् । व॒पे॒द् रुक्का॑मः \newline

\textbf{Jatai Paata} \newline

1. उ॒परि॑ष्टा दे॒वै वोपरि॑ष्टा दु॒परि॑ष्टा दे॒व । \newline
2. ए॒वास्मा॑ अस्मा ए॒वै वास्मै᳚ । \newline
3. अ॒स्मै॒ छन्दाꣳ॑सि॒ छन्दाꣳ॑स्य स्मा अस्मै॒ छन्दाꣳ॑सि । \newline
4. छन्दाꣳ॒॒ स्यया॑तयामा॒न्य या॑तयामानि॒ छन्दाꣳ॑सि॒ छन्दाꣳ॒॒ स्यया॑तयामानि । \newline
5. अया॑तयामा॒ न्यवावा या॑तयामा॒ न्यया॑तयामा॒ न्यव॑ । \newline
6. अया॑तयामा॒नीत्यया॑त - या॒मा॒नि॒ । \newline
7. अव॑ रुन्धे रु॒न्धे ऽवाव॑ रुन्धे । \newline
8. रु॒न्ध॒ उपोप॑ रुन्धे रुन्ध॒ उप॑ । \newline
9. उपै॑न मेन॒ मुपो पै॑नम् । \newline
10. ए॒न॒ मुत्त॑र॒ उत्त॑र एन मेन॒ मुत्त॑रः । \newline
11. उत्त॑रो य॒ज्ञो य॒ज्ञ् उत्त॑र॒ उत्त॑रो य॒ज्ञ्ः । \newline
12. उत्त॑र॒ इत्युत् - त॒रः॒ । \newline
13. य॒ज्ञो न॑मति नमति य॒ज्ञो य॒ज्ञो न॑मति । \newline
14. न॒म॒ त्ये॒ता ए॒ता न॑मति नम त्ये॒ताः । \newline
15. ए॒ता ए॒वै वैता ए॒ता ए॒व । \newline
16. ए॒व निर् णि रे॒वैव निः । \newline
17. निर् व॑पेद् वपे॒न् निर् णिर् व॑पेत् । \newline
18. व॒पे॒द् यं ॅयं ॅव॑पेद् वपे॒द् यम् । \newline
19. यम् मे॒धा मे॒धा यं ॅयम् मे॒धा । \newline
20. मे॒धा न न मे॒धा मे॒धा न । \newline
21. नो प॒नमे॑ दुप॒नमे॒न् न नो प॒नमे᳚त् । \newline
22. उ॒प॒नमे॒च् छन्दाꣳ॑सि॒ छन्दाꣳ॑ स्युप॒नमे॑ दुप॒नमे॒च् छन्दाꣳ॑सि । \newline
23. उ॒प॒नमे॒दित्यु॑प - नमे᳚त् । \newline
24. छन्दाꣳ॑सि॒ वै वै छन्दाꣳ॑सि॒ छन्दाꣳ॑सि॒ वै । \newline
25. वै देवि॑का॒ देवि॑का॒ वै वै देवि॑काः । \newline
26. देवि॑का॒ श्छन्दाꣳ॑सि॒ छन्दाꣳ॑सि॒ देवि॑का॒ देवि॑का॒ श्छन्दाꣳ॑सि । \newline
27. छन्दाꣳ॑सि॒ खलु॒ खलु॒ छन्दाꣳ॑सि॒ छन्दाꣳ॑सि॒ खलु॑ । \newline
28. खलु॒ वै वै खलु॒ खलु॒ वै । \newline
29. वा ए॒त मे॒तं ॅवै वा ए॒तम् । \newline
30. ए॒तन् न नैत मे॒तन् न । \newline
31. नो पोप॒ न नोप॑ । \newline
32. उप॑ नमन्ति नम॒ न्त्युपोप॑ नमन्ति । \newline
33. न॒म॒न्ति॒ यं ॅयम् न॑मन्ति नमन्ति॒ यम् । \newline
34. यम् मे॒धा मे॒धा यं ॅयम् मे॒धा । \newline
35. मे॒धा न न मे॒धा मे॒धा न । \newline
36. नो प॒नम॑ त्युप॒नम॑ति॒ न नो प॒नम॑ति । \newline
37. उ॒प॒नम॑ति प्रथ॒मम् प्र॑थ॒म मु॑प॒नम॑ त्युप॒नम॑ति प्रथ॒मम् । \newline
38. उ॒प॒नम॒तीत्यु॑प - नम॑ति । \newline
39. प्र॒थ॒मम् धा॒तार॑म् धा॒तार॑म् प्रथ॒मम् प्र॑थ॒मम् धा॒तार᳚म् । \newline
40. धा॒तार॑म् करोति करोति धा॒तार॑म् धा॒तार॑म् करोति । \newline
41. क॒रो॒ति॒ मु॒ख॒तो मु॑ख॒तः क॑रोति करोति मुख॒तः । \newline
42. मु॒ख॒त ए॒वैव मु॑ख॒तो मु॑ख॒त ए॒व । \newline
43. ए॒वास्मा॑ अस्मा ए॒वै वास्मै᳚ । \newline
44. अ॒स्मै॒ छन्दाꣳ॑सि॒ छन्दाꣳ॑ स्यस्मा अस्मै॒ छन्दाꣳ॑सि । \newline
45. छन्दाꣳ॑सि दधाति दधाति॒ छन्दाꣳ॑सि॒ छन्दाꣳ॑सि दधाति । \newline
46. द॒धा॒ त्युपोप॑ दधाति दधा॒ त्युप॑ । \newline
47. उपै॑न मेन॒ मुपो पै॑नम् । \newline
48. ए॒न॒म् मे॒धा मे॒धै न॑ मेनम् मे॒धा । \newline
49. मे॒धा न॑मति नमति मे॒धा मे॒धा न॑मति । \newline
50. न॒म॒ त्ये॒ता ए॒ता न॑मति नम त्ये॒ताः । \newline
51. ए॒ता ए॒वै वैता ए॒ता ए॒व । \newline
52. ए॒व निर् णि रे॒वैव निः । \newline
53. निर् व॑पेद् वपे॒न् निर् णिर् व॑पेत् । \newline
54. व॒पे॒द् रुक्का॑मो॒ रुक्का॑मो वपेद् वपे॒द् रुक्का॑मः । \newline

\textbf{Ghana Paata } \newline

1. उ॒परि॑ष्टा दे॒वैवोपरि॑ष्टा दु॒परि॑ष्टा दे॒वास्मा॑ अस्मा ए॒वोपरि॑ष्टा दु॒परि॑ष्टा दे॒वास्मै᳚ । \newline
2. ए॒वास्मा॑ अस्मा ए॒वैवास्मै॒ छन्दाꣳ॑सि॒ छन्दाꣳ॑ स्यस्मा ए॒वैवास्मै॒ छन्दाꣳ॑सि । \newline
3. अ॒स्मै॒ छन्दाꣳ॑सि॒ छन्दाꣳ॑ स्यस्मा अस्मै॒ छन्दाꣳ॒॒ स्यया॑तयामा॒ न्यया॑तयामानि॒ छन्दाꣳ॑
स्यस्मा अस्मै॒ छन्दाꣳ॒॒ स्यया॑तयामानि । \newline
4. छन्दाꣳ॒॒ स्यया॑तयामा॒ न्यया॑तयामानि॒ छन्दाꣳ॑सि॒ छन्दाꣳ॒॒ स्यया॑तयामा॒ न्यवावा या॑तयामानि॒ छन्दाꣳ॑सि॒ छन्दाꣳ॒॒ स्यया॑तयामा॒ न्यव॑ । \newline
5. अया॑तयामा॒ न्यवावा या॑तयामा॒ न्यया॑तयामा॒ न्यव॑ रुन्धे रु॒न्धे ऽवा या॑तयामा॒ न्यया॑तयामा॒ न्यव॑ रुन्धे । \newline
6. अया॑तयामा॒नीत्यया॑त - या॒मा॒नि॒ । \newline
7. अव॑ रुन्धे रु॒न्धे ऽवाव॑ रुन्ध॒ उपोप॑ रु॒न्धे ऽवाव॑ रुन्ध॒ उप॑ । \newline
8. रु॒न्ध॒ उपोप॑ रुन्धे रुन्ध॒ उपै॑न मेन॒ मुप॑ रुन्धे रुन्ध॒ उपै॑नम् । \newline
9. उपै॑न मेन॒ मुपोपै॑न॒ मुत्त॑र॒ उत्त॑र एन॒ मुपोपै॑न॒ मुत्त॑रः । \newline
10. ए॒न॒ मुत्त॑र॒ उत्त॑र एन मेन॒ मुत्त॑रो य॒ज्ञो य॒ज्ञ् उत्त॑र एन मेन॒ मुत्त॑रो य॒ज्ञ्ः । \newline
11. उत्त॑रो य॒ज्ञो य॒ज्ञ् उत्त॑र॒ उत्त॑रो य॒ज्ञो न॑मति नमति य॒ज्ञ् उत्त॑र॒ उत्त॑रो य॒ज्ञो न॑मति । \newline
12. उत्त॑र॒ इत्युत् - त॒रः॒ । \newline
13. य॒ज्ञो न॑मति नमति य॒ज्ञो य॒ज्ञो न॑म त्ये॒ता ए॒ता न॑मति य॒ज्ञो य॒ज्ञो न॑म त्ये॒ताः । \newline
14. न॒म॒ त्ये॒ता ए॒ता न॑मति नम त्ये॒ता ए॒वैवैता न॑मति नम त्ये॒ता ए॒व । \newline
15. ए॒ता ए॒वैवैता ए॒ता ए॒व निर् णि रे॒वैता ए॒ता ए॒व निः । \newline
16. ए॒व निर् णि रे॒वैव निर् व॑पेद् वपे॒न् नि रे॒वैव निर् व॑पेत् । \newline
17. निर् व॑पेद् वपे॒न् निर् णिर् व॑पे॒द् यम् ॅयम् ॅव॑पे॒न् निर् णिर् व॑पे॒द् यम् । \newline
18. व॒पे॒द् यम् ॅयम् ॅव॑पेद् वपे॒द् यम् मे॒धा मे॒धा यम् ॅव॑पेद् वपे॒द् यम् मे॒धा । \newline
19. यम् मे॒धा मे॒धा यम् ॅयम् मे॒धा न न मे॒धा यम् ॅयम् मे॒धा न । \newline
20. मे॒धा न न मे॒धा मे॒धा नोप॒नमे॑ दुप॒नमे॒न् न मे॒धा मे॒धा नोप॒नमे᳚त् । \newline
21. नोप॒नमे॑ दुप॒नमे॒न् न नोप॒नमे॒च् छन्दाꣳ॑सि॒ छन्दाꣳ॑ स्युप॒नमे॒न् न नोप॒नमे॒च् छन्दाꣳ॑सि । \newline
22. उ॒प॒नमे॒च् छन्दाꣳ॑सि॒ छन्दाꣳ॑ स्युप॒नमे॑ दुप॒नमे॒च् छन्दाꣳ॑सि॒ वै वै छन्दाꣳ॑ स्युप॒नमे॑ दुप॒नमे॒च् छन्दाꣳ॑सि॒ वै । \newline
23. उ॒प॒नमे॒दित्यु॑प - नमे᳚त् । \newline
24. छन्दाꣳ॑सि॒ वै वै छन्दाꣳ॑सि॒ छन्दाꣳ॑सि॒ वै देवि॑का॒ देवि॑का॒ वै छन्दाꣳ॑सि॒ छन्दाꣳ॑सि॒ वै देवि॑काः । \newline
25. वै देवि॑का॒ देवि॑का॒ वै वै देवि॑का॒ श्छन्दाꣳ॑सि॒ छन्दाꣳ॑सि॒ देवि॑का॒ वै वै देवि॑का॒ श्छन्दाꣳ॑सि । \newline
26. देवि॑का॒ श्छन्दाꣳ॑सि॒ छन्दाꣳ॑सि॒ देवि॑का॒ देवि॑का॒ श्छन्दाꣳ॑सि॒ खलु॒ खलु॒ छन्दाꣳ॑सि॒ देवि॑का॒ देवि॑का॒ श्छन्दाꣳ॑सि॒ खलु॑ । \newline
27. छन्दाꣳ॑सि॒ खलु॒ खलु॒ छन्दाꣳ॑सि॒ छन्दाꣳ॑सि॒ खलु॒ वै वै खलु॒ छन्दाꣳ॑सि॒ छन्दाꣳ॑सि॒ खलु॒ वै । \newline
28. खलु॒ वै वै खलु॒ खलु॒ वा ए॒त मे॒तम् ॅवै खलु॒ खलु॒ वा ए॒तम् । \newline
29. वा ए॒त मे॒तम् ॅवै वा ए॒तम् न नैतम् ॅवै वा ए॒तम् न । \newline
30. ए॒तम् न नैत मे॒तम्,नोपोप॒ नैत मे॒तम्,नोप॑ । \newline
31. नोपोप॒ न नोप॑ नमन्ति नम॒ न्त्युप॒ न नोप॑ नमन्ति । \newline
32. उप॑ नमन्ति नम॒ न्त्युपोप॑ नमन्ति॒ यम् ॅयम्,न॑म॒ न्त्युपोप॑ नमन्ति॒ यम् । \newline
33. न॒म॒न्ति॒ यम् ॅयम् न॑मन्ति नमन्ति॒ यम् मे॒धा मे॒धा यम् न॑मन्ति नमन्ति॒ यम् मे॒धा । \newline
34. यम् मे॒धा मे॒धा यम् ॅयम् मे॒धा न न मे॒धा यम् ॅयम् मे॒धा न । \newline
35. मे॒धा न न मे॒धा मे॒धा नोप॒नम॑ त्युप॒नम॑ति॒ न मे॒धा मे॒धा नोप॒नम॑ति । \newline
36. नोप॒नम॑ त्युप॒नम॑ति॒ न नोप॒नम॑ति प्रथ॒मम् प्र॑थ॒म मु॑प॒नम॑ति॒ न नोप॒नम॑ति प्रथ॒मम् । \newline
37. उ॒प॒नम॑ति प्रथ॒मम् प्र॑थ॒म मु॑प॒नम॑ त्युप॒नम॑ति प्रथ॒मम् धा॒तार॑म् धा॒तार॑म् प्रथ॒म मु॑प॒नम॑ त्युप॒नम॑ति प्रथ॒मम् धा॒तार᳚म् । \newline
38. उ॒प॒नम॒तीत्यु॑प - नम॑ति । \newline
39. प्र॒थ॒मम् धा॒तार॑म् धा॒तार॑म् प्रथ॒मम् प्र॑थ॒मम् धा॒तार॑म् करोति करोति धा॒तार॑म् 
प्रथ॒मम् प्र॑थ॒मम् धा॒तार॑म् करोति । \newline
40. धा॒तार॑म् करोति करोति धा॒तार॑म् धा॒तार॑म् करोति मुख॒तो मु॑ख॒तः क॑रोति धा॒तार॑म् धा॒तार॑म् करोति मुख॒तः । \newline
41. क॒रो॒ति॒ मु॒ख॒तो मु॑ख॒तः क॑रोति करोति मुख॒त ए॒वैव मु॑ख॒तः क॑रोति करोति मुख॒त ए॒व । \newline
42. मु॒ख॒त ए॒वैव मु॑ख॒तो मु॑ख॒त ए॒वास्मा॑ अस्मा ए॒व मु॑ख॒तो मु॑ख॒त ए॒वास्मै᳚ । \newline
43. ए॒वास्मा॑ अस्मा ए॒वैवास्मै॒ छन्दाꣳ॑सि॒ छन्दाꣳ॑ स्यस्मा ए॒वैवास्मै॒ छन्दाꣳ॑सि । \newline
44. अ॒स्मै॒ छन्दाꣳ॑सि॒ छन्दाꣳ॑ स्यस्मा अस्मै॒ छन्दाꣳ॑सि दधाति दधाति॒ छन्दाꣳ॑ स्यस्मा अस्मै॒ छन्दाꣳ॑सि दधाति । \newline
45. छन्दाꣳ॑सि दधाति दधाति॒ छन्दाꣳ॑सि॒ छन्दाꣳ॑सि दधा॒ त्युपोप॑ दधाति॒ छन्दाꣳ॑सि॒ छन्दाꣳ॑सि दधा॒त्युप॑ । \newline
46. द॒धा॒ त्युपोप॑ दधाति दधा॒ त्युपै॑न मेन॒ मुप॑ दधाति दधा॒ त्युपै॑नम् । \newline
47. उपै॑न मेन॒ मुपो पै॑नम् मे॒धा मे॒धैन॒ मुपो पै॑नम् मे॒धा । \newline
48. ए॒न॒म् मे॒धा मे॒धैन॑ मेनम् मे॒धा न॑मति नमति मे॒धैन॑ मेनम् मे॒धा न॑मति । \newline
49. मे॒धा न॑मति नमति मे॒धा मे॒धा न॑म त्ये॒ता ए॒ता न॑मति मे॒धा मे॒धा न॑म त्ये॒ताः । \newline
50. न॒म॒ त्ये॒ता ए॒ता न॑मति नम त्ये॒ता ए॒वैवैता न॑मति नम त्ये॒ता ए॒व । \newline
51. ए॒ता ए॒वैवैता ए॒ता ए॒व निर् णिरे॒वैता ए॒ता ए॒व निः । \newline
52. ए॒व निर् णि रे॒वैव निर् व॑पेद् वपे॒न् नि रे॒वैव निर् व॑पेत् । \newline
53. निर् व॑पेद् वपे॒न् निर् णिर् व॑पे॒द् रुक्का॑मो॒ रुक्का॑मो वपे॒न् निर् णिर् व॑पे॒द् रुक्का॑मः । \newline
54. व॒पे॒द् रुक्का॑मो॒ रुक्का॑मो वपेद् वपे॒द् रुक्का॑म॒ श्छन्दाꣳ॑सि॒ छन्दाꣳ॑सि॒ रुक्का॑मो वपेद् वपे॒द् रुक्का॑म॒ श्छन्दाꣳ॑सि । \newline
\pagebreak
\markright{ TS 3.4.9.6  \hfill https://www.vedavms.in \hfill}

\section{ TS 3.4.9.6 }

\textbf{TS 3.4.9.6 } \newline
\textbf{Samhita Paata} \newline

द्रुक्का॑म॒श्छन्दाꣳ॑सि॒ वै देवि॑का॒श्छन्दाꣳ॑सीव॒ खलु॒ वै रुक् छन्दो॑भिरे॒वास्मि॒न् रुचं॑ दधातिक्षी॒रे भ॑वन्ति॒ रुच॑मे॒वास्मि॑न् दधति मद्ध्य॒तो धा॒तारं॑ करोति मद्ध्य॒त ए॒वैनꣳ॑ रु॒चो द॑धातिगाय॒त्री वा अनु॑मतिस्त्रि॒ष्टुग्रा॒का जग॑ती सिनीवा॒ल्य॑नु॒ष्टुप् कु॒हूर्द्धा॒ता व॑षट्का॒रः पू᳚र्वप॒क्षो रा॒काऽप॑रप॒क्षः कु॒हूर॑मावा॒स्या॑ सिनीवा॒ली पौ᳚र्णमा॒स्यनु॑मतिश्च॒न्द्रमा॑ धा॒ताऽष्टौ - [  ] \newline

\textbf{Pada Paata} \newline

रुक्का॑म॒ इति॒ रुक्-का॒मः॒ । छन्दाꣳ॑सि । वै । देवि॑काः । छन्दाꣳ॑सि । इ॒व॒ । खलु॑ । वै । रुक् । छन्दो॑भि॒रिति॒ छन्दः॑-भिः॒ । ए॒व । अ॒स्मि॒न्न् । रुच᳚म् । द॒धा॒ति॒ । क्षी॒रे । भ॒व॒न्ति॒ । रुच᳚म् । ए॒व । अ॒स्मि॒न्न् । द॒ध॒ति॒ । म॒द्ध्य॒तः । धा॒तार᳚म् । क॒रो॒ति॒ । म॒द्ध्य॒तः । ए॒व । ए॒न॒म् । रु॒चः । द॒धा॒ति॒ । गा॒य॒त्री । वै । अनु॑मति॒रित्य॑नु - म॒तिः॒ । त्रि॒ष्टुक् । रा॒का । जग॑ती । सि॒नी॒वा॒ली । अ॒नु॒ष्टुबित्य॑नु - स्तुप् । कु॒हूः । धा॒ता । व॒ष॒ट्का॒र इति॑ वषट् - का॒रः । पू॒र्व॒प॒क्ष इति॑ पूर्व - प॒क्षः । रा॒का । अ॒प॒र॒प॒क्ष इत्य॑पर - प॒क्षः । कु॒हूः । अ॒मा॒वा॒स्येत्य॑मा - वा॒स्या᳚ । सि॒नी॒वा॒ली । पौ॒र्ण॒मा॒सीति॑ पौर्ण - मा॒सी । अनु॑मति॒रित्य॑नु - म॒तिः॒ । च॒न्द्रमाः᳚ । धा॒ता । अ॒ष्टौ ।  \newline


\textbf{Krama Paata} \newline

रुक्का॑म॒ श्छन्दाꣳ॑सि । रुक्का॑म॒ इति॒ रुक् - का॒मः॒ । छन्दाꣳ॑सि॒ वै । वै देवि॑काः । देवि॑का॒ श्छन्दाꣳ॑सि । छन्दाꣳ॑सीव । इ॒व॒ खलु॑ । खलु॒ वै । वै रुक् । रुक् छन्दो॑भिः । छन्दो॑भि॒रेव । छन्दो॑भि॒रिति॒ छन्दः॑ - भिः॒ । ए॒वास्मिन्न्॑ । अ॒स्मि॒न् रुच᳚म् । रुच॑म् दधाति । द॒धा॒ति॒ क्षी॒रे । क्षी॒रे भ॑वन्ति । भ॒व॒न्ति॒ रुच᳚म् । रुच॑मे॒व । ए॒वास्मिन्न्॑ । अ॒स्मि॒न् द॒ध॒ति॒ । द॒ध॒ति॒ म॒द्ध्य॒तः । म॒द्ध्य॒तो धा॒तार᳚म् । धा॒तार॑म् करोति । क॒रो॒ति॒ म॒द्ध्य॒तः । म॒द्ध्य॒त ए॒व । ए॒वैन᳚म् । ए॒नꣳ॒॒ रु॒चः । रु॒चो द॑धाति । द॒धा॒ति॒ गा॒य॒त्री । गा॒य॒त्री वै । वा अनु॑मतिः । अनु॑मतिस्त्रि॒ष्टुक् । अनु॑मति॒रित्यनु॑ - म॒तिः॒ । त्रि॒ष्टुग् रा॒का । रा॒का जग॑ती । जग॑ती सिनीवा॒ली । सि॒नी॒वा॒ल्य॑नु॒ष्टुप् । अ॒नु॒ष्टुप् कु॒हूः । अ॒नु॒ष्टुबित्य॑नु - स्तुप् । कु॒हूर् धा॒ता । धा॒ता व॑षट्का॒रः । व॒ष॒ट्का॒रः पू᳚र्वप॒क्षः । व॒ष॒ट्का॒र इति॑ वषट् - का॒रः । पू॒र्व॒प॒क्षो रा॒का । पू॒र्व॒प॒क्ष इति॑ पूर्व - प॒क्षः । रा॒का ऽप॑रप॒क्षः । अ॒प॒र॒प॒क्षः कु॒हूः । अ॒प॒र॒प॒क्ष इत्य॑पर - प॒क्षः । कु॒हूर॑मावा॒स्या᳚ । अ॒मा॒वा॒स्या॑ सिनीवा॒ली । अ॒मा॒वा॒स्येत्य॑मा - वा॒स्या᳚ । सि॒नी॒वा॒ली पौ᳚र्णमा॒सी । पौ॒र्ण॒मा॒स्यनु॑मतिः । पौ॒र्ण॒मा॒सीति॑ पौर्ण - मा॒सी । अनु॑मति श्च॒न्द्रमाः᳚ । अनु॑मति॒रित्यनु॑ - म॒तिः॒ । च॒न्द्रमा॑ धा॒ता । धा॒ता ऽष्टौ । अ॒ष्टौ वस॑वः \newline

\textbf{Jatai Paata} \newline

1. रुक्का॑म॒ श्छन्दाꣳ॑सि॒ छन्दाꣳ॑सि॒ रुक्का॑मो॒ रुक्का॑म॒ श्छन्दाꣳ॑सि । \newline
2. रुक्का॑म॒ इति॒ रुक् - का॒मः॒ । \newline
3. छन्दाꣳ॑सि॒ वै वै छन्दाꣳ॑सि॒ छन्दाꣳ॑सि॒ वै । \newline
4. वै देवि॑का॒ देवि॑का॒ वै वै देवि॑काः । \newline
5. देवि॑का॒ श्छन्दाꣳ॑सि॒ छन्दाꣳ॑सि॒ देवि॑का॒ देवि॑का॒ श्छन्दाꣳ॑सि । \newline
6. छन्दाꣳ॑सीवेव॒ छन्दाꣳ॑सि॒ छन्दाꣳ॑सीव । \newline
7. इ॒व॒ खलु॒ खल्वि॑वेव॒ खलु॑ । \newline
8. खलु॒ वै वै खलु॒ खलु॒ वै । \newline
9. वै रुग् रुग् वै वै रुक् । \newline
10. रुक् छन्दो॑भि॒ श्छन्दो॑भी॒ रुग् रुक् छन्दो॑भिः । \newline
11. छन्दो॑भि रे॒वैव छन्दो॑भि॒ श्छन्दो॑भि रे॒व । \newline
12. छन्दो॑भि॒रिति॒ छन्दः॑ - भिः॒ । \newline
13. ए॒वास्मि॑न्-नस्मिन्-ने॒वै वास्मिन्न्॑ । \newline
14. अ॒स्मि॒न् रुचꣳ॒॒ रुच॑ मस्मिन्-नस्मि॒न् रुच᳚म् । \newline
15. रुच॑म् दधाति दधाति॒ रुचꣳ॒॒ रुच॑म् दधाति । \newline
16. द॒धा॒ति॒ क्षी॒रे क्षी॒रे द॑धाति दधाति क्षी॒रे । \newline
17. क्षी॒रे भ॑वन्ति भवन्ति क्षी॒रे क्षी॒रे भ॑वन्ति । \newline
18. भ॒व॒न्ति॒ रुचꣳ॒॒ रुच॑म् भवन्ति भवन्ति॒ रुच᳚म् । \newline
19. रुच॑ मे॒वैव रुचꣳ॒॒ रुच॑ मे॒व । \newline
20. ए॒वास्मि॑न्-नस्मिन्-ने॒वै वास्मिन्न्॑ । \newline
21. अ॒स्मि॒न् द॒ध॒ति॒ द॒ध॒ त्य॒स्मि॒न्-न॒स्मि॒न् द॒ध॒ति॒ । \newline
22. द॒ध॒ति॒ म॒द्ध्य॒तो म॑द्ध्य॒तो द॑धति दधति मद्ध्य॒तः । \newline
23. म॒द्ध्य॒तो धा॒तार॑म् धा॒तार॑म् मद्ध्य॒तो म॑द्ध्य॒तो धा॒तार᳚म् । \newline
24. धा॒तार॑म् करोति करोति धा॒तार॑म् धा॒तार॑म् करोति । \newline
25. क॒रो॒ति॒ म॒द्ध्य॒तो म॑द्ध्य॒तः क॑रोति करोति मद्ध्य॒तः । \newline
26. म॒द्ध्य॒त ए॒वैव म॑द्ध्य॒तो म॑द्ध्य॒त ए॒व । \newline
27. ए॒वैन॑ मेन मे॒वै वैन᳚म् । \newline
28. ए॒नꣳ॒॒ रु॒चो रु॒च ए॑न मेनꣳ रु॒चः । \newline
29. रु॒चो द॑धाति दधाति रु॒चो रु॒चो द॑धाति । \newline
30. द॒धा॒ति॒ गा॒य॒त्री गा॑य॒त्री द॑धाति दधाति गाय॒त्री । \newline
31. गा॒य॒त्री वै वै गा॑य॒त्री गा॑य॒त्री वै । \newline
32. वा अनु॑मति॒ रनु॑मति॒र् वै वा अनु॑मतिः । \newline
33. अनु॑मति स्त्रि॒ष्टुक् त्रि॒ष्टु गनु॑मति॒ रनु॑मति स्त्रि॒ष्टुक् । \newline
34. अनु॑मति॒रित्य॑नु - म॒तिः॒ । \newline
35. त्रि॒ष्टुग् रा॒का रा॒का त्रि॒ष्टुक् त्रि॒ष्टुग् रा॒का । \newline
36. रा॒का जग॑ती॒ जग॑ती रा॒का रा॒का जग॑ती । \newline
37. जग॑ती सिनीवा॒ली सि॑नीवा॒ली जग॑ती॒ जग॑ती सिनीवा॒ली । \newline
38. सि॒नी॒वा॒ ल्य॑नु॒ष्टु ब॑नु॒ष्टुफ् सि॑नीवा॒ली सि॑नीवा॒ ल्य॑नु॒ष्टुप् । \newline
39. अ॒नु॒ष्टुप् कु॒हूः कु॒हू र॑नु॒ष्टु ब॑नु॒ष्टुप् कु॒हूः । \newline
40. अ॒नु॒ष्टुबित्य॑नु - स्तुप् । \newline
41. कु॒हूर् धा॒ता धा॒ता कु॒हूः कु॒हूर् धा॒ता । \newline
42. धा॒ता व॑षट्का॒रो व॑षट्का॒रो धा॒ता धा॒ता व॑षट्का॒रः । \newline
43. व॒ष॒ट्का॒रः पू᳚र्वप॒क्षः पू᳚र्वप॒क्षो व॑षट्का॒रो व॑षट्का॒रः पू᳚र्वप॒क्षः । \newline
44. व॒ष॒ट्का॒र इति॑ वषट् - का॒रः । \newline
45. पू॒र्व॒प॒क्षो रा॒का रा॒का पू᳚र्वप॒क्षः पू᳚र्वप॒क्षो रा॒का । \newline
46. पू॒र्व॒प॒क्ष इति॑ पूर्व - प॒क्षः । \newline
47. रा॒का ऽप॑रप॒क्षो॑ ऽपरप॒क्षो रा॒का रा॒का ऽप॑रप॒क्षः । \newline
48. अ॒प॒र॒प॒क्षः कु॒हूः कु॒हू र॑परप॒क्षो॑ ऽपरप॒क्षः कु॒हूः । \newline
49. अ॒प॒र॒प॒क्ष इत्य॑पर - प॒क्षः । \newline
50. कु॒हू र॑मावा॒स्या॑ ऽमावा॒स्या॑ कु॒हूः कु॒हू र॑मावा॒स्या᳚ । \newline
51. अ॒मा॒वा॒स्या॑ सिनीवा॒ली सि॑नीवा॒ल्य॑मावा॒स्या॑ ऽमावा॒स्या॑ सिनीवा॒ली । \newline
52. अ॒मा॒वा॒स्येत्य॑मा - वा॒स्या᳚ । \newline
53. सि॒नी॒वा॒ली पौ᳚र्णमा॒सी पौ᳚र्णमा॒सी सि॑नीवा॒ली सि॑नीवा॒ली पौ᳚र्णमा॒सी । \newline
54. पौ॒र्ण॒मा॒स्य नु॑मति॒ रनु॑मतिः पौर्णमा॒सी पौ᳚र्णमा॒स्य नु॑मतिः । \newline
55. पौ॒र्ण॒मा॒सीति॑ पौर्ण - मा॒सी । \newline
56. अनु॑मति श्च॒न्द्रमा᳚ श्च॒न्द्रमा॒ अनु॑मति॒ रनु॑मति श्च॒न्द्रमाः᳚ । \newline
57. अनु॑मति॒रित्य॑नु - म॒तिः॒ । \newline
58. च॒न्द्रमा॑ धा॒ता धा॒ता च॒न्द्रमा᳚ श्च॒न्द्रमा॑ धा॒ता । \newline
59. धा॒ता ऽष्टा व॒ष्टौ धा॒ता धा॒ता ऽष्टौ । \newline
60. अ॒ष्टौ वस॑वो॒ वस॑वो॒ ऽष्टा व॒ष्टौ वस॑वः । \newline

\textbf{Ghana Paata } \newline

1. रुक्का॑म॒ श्छन्दाꣳ॑सि॒ छन्दाꣳ॑सि॒ रुक्का॑मो॒ रुक्का॑म॒ श्छन्दाꣳ॑सि॒ वै वै छन्दाꣳ॑सि॒ रुक्का॑मो॒ रुक्का॑म॒ श्छन्दाꣳ॑सि॒ वै । \newline
2. रुक्का॑म॒ इति॒ रुक् - का॒मः॒ । \newline
3. छन्दाꣳ॑सि॒ वै वै छन्दाꣳ॑सि॒ छन्दाꣳ॑सि॒ वै देवि॑का॒ देवि॑का॒ वै छन्दाꣳ॑सि॒ छन्दाꣳ॑सि॒ वै देवि॑काः । \newline
4. वै देवि॑का॒ देवि॑का॒ वै वै देवि॑का॒ श्छन्दाꣳ॑सि॒ छन्दाꣳ॑सि॒ देवि॑का॒ वै वै देवि॑का॒ श्छन्दाꣳ॑सि । \newline
5. देवि॑का॒ श्छन्दाꣳ॑सि॒ छन्दाꣳ॑सि॒ देवि॑का॒ देवि॑का॒ श्छन्दाꣳ॑सीवेव॒ छन्दाꣳ॑सि॒ देवि॑का॒ देवि॑का॒ श्छन्दाꣳ॑सीव । \newline
6. छन्दाꣳ॑सीवेव॒ छन्दाꣳ॑सि॒ छन्दाꣳ॑सीव॒ खलु॒ खल्वि॑व॒ छन्दाꣳ॑सि॒ छन्दाꣳ॑सीव॒ खलु॑ । \newline
7. इ॒व॒ खलु॒ खल्वि॑वेव॒ खलु॒ वै वै खल्वि॑वेव॒ खलु॒ वै । \newline
8. खलु॒ वै वै खलु॒ खलु॒ वै रुग् रुग् वै खलु॒ खलु॒ वै रुक् । \newline
9. वै रुग् रुग् वै वै रुक् छन्दो॑भि॒ श्छन्दो॑भी॒ रुग् वै वै रुक् छन्दो॑भिः । \newline
10. रुक् छन्दो॑भि॒ श्छन्दो॑भी॒ रुग् रुक् छन्दो॑भि रे॒वैव छन्दो॑भी॒ रुग् रुक् छन्दो॑भि रे॒व । \newline
11. छन्दो॑भि रे॒वैव छन्दो॑भि॒ श्छन्दो॑भि रे॒वास्मि॑न्,नस्मिन्,ने॒व छन्दो॑भि॒ श्छन्दो॑भि रे॒वास्मिन्न्॑ । \newline
12. छन्दो॑भि॒रिति॒ छन्दः॑ - भिः॒ । \newline
13. ए॒वास्मि॑न्,नस्मिन्,ने॒वैवास्मि॒न् रुचꣳ॒॒ रुच॑ मस्मिन्,ने॒वैवास्मि॒न् रुच᳚म् । \newline
14. अ॒स्मि॒न् रुचꣳ॒॒ रुच॑ मस्मिन्,नस्मि॒न् रुच॑म् दधाति दधाति॒ रुच॑ मस्मिन्,नस्मि॒न् रुच॑म् दधाति । \newline
15. रुच॑म् दधाति दधाति॒ रुचꣳ॒॒ रुच॑म् दधाति क्षी॒रे क्षी॒रे द॑धाति॒ रुचꣳ॒॒ रुच॑म् दधाति क्षी॒रे । \newline
16. द॒धा॒ति॒ क्षी॒रे क्षी॒रे द॑धाति दधाति क्षी॒रे भ॑वन्ति भवन्ति क्षी॒रे द॑धाति दधाति क्षी॒रे भ॑वन्ति । \newline
17. क्षी॒रे भ॑वन्ति भवन्ति क्षी॒रे क्षी॒रे भ॑वन्ति॒ रुचꣳ॒॒ रुच॑म् भवन्ति क्षी॒रे क्षी॒रे भ॑वन्ति॒ रुच᳚म् । \newline
18. भ॒व॒न्ति॒ रुचꣳ॒॒ रुच॑म् भवन्ति भवन्ति॒ रुच॑ मे॒वैव रुच॑म् भवन्ति भवन्ति॒ रुच॑ मे॒व । \newline
19. रुच॑ मे॒वैव रुचꣳ॒॒ रुच॑ मे॒वास्मि॑न्,नस्मिन्,ने॒व रुचꣳ॒॒ रुच॑ मे॒वास्मिन्न्॑ । \newline
20. ए॒वास्मि॑न्,नस्मिन्,ने॒वैवास्मि॑न् दधति दध त्यस्मिन्,ने॒वैवास्मि॑न् दधति । \newline
21. अ॒स्मि॒न् द॒ध॒ति॒ द॒ध॒ त्य॒स्मि॒न्,न॒स्मि॒न् द॒ध॒ति॒ म॒द्ध्य॒तो म॑द्ध्य॒तो द॑ध त्यस्मिन्,नस्मिन् दधति मद्ध्य॒तः । \newline
22. द॒ध॒ति॒ म॒द्ध्य॒तो म॑द्ध्य॒तो द॑धति दधति मद्ध्य॒तो धा॒तार॑म् धा॒तार॑म् मद्ध्य॒तो द॑धति 
दधति मद्ध्य॒तो धा॒तार᳚म् । \newline
23. म॒द्ध्य॒तो धा॒तार॑म् धा॒तार॑म् मद्ध्य॒तो म॑द्ध्य॒तो धा॒तार॑म् करोति करोति धा॒तार॑म् मद्ध्य॒तो म॑द्ध्य॒तो धा॒तार॑म् करोति । \newline
24. धा॒तार॑म् करोति करोति धा॒तार॑म् धा॒तार॑म् करोति मद्ध्य॒तो म॑द्ध्य॒तः क॑रोति धा॒तार॑म् धा॒तार॑म् करोति मद्ध्य॒तः । \newline
25. क॒रो॒ति॒ म॒द्ध्य॒तो म॑द्ध्य॒तः क॑रोति करोति मद्ध्य॒त ए॒वैव म॑द्ध्य॒तः क॑रोति करोति मद्ध्य॒त ए॒व । \newline
26. म॒द्ध्य॒त ए॒वैव म॑द्ध्य॒तो म॑द्ध्य॒त ए॒वैन॑ मेन मे॒व म॑द्ध्य॒तो म॑द्ध्य॒त ए॒वैन᳚म् । \newline
27. ए॒वैन॑ मेन मे॒वैवैनꣳ॑ रु॒चो रु॒च ए॑न मे॒वैवैनꣳ॑ रु॒चः । \newline
28. ए॒नꣳ॒॒ रु॒चो रु॒च ए॑न मेनꣳ रु॒चो द॑धाति दधाति रु॒च ए॑न मेनꣳ रु॒चो द॑धाति । \newline
29. रु॒चो द॑धाति दधाति रु॒चो रु॒चो द॑धाति गाय॒त्री गा॑य॒त्री द॑धाति रु॒चो रु॒चो द॑धाति गाय॒त्री । \newline
30. द॒धा॒ति॒ गा॒य॒त्री गा॑य॒त्री द॑धाति दधाति गाय॒त्री वै वै गा॑य॒त्री द॑धाति दधाति गाय॒त्री वै । \newline
31. गा॒य॒त्री वै वै गा॑य॒त्री गा॑य॒त्री वा अनु॑मति॒ रनु॑मति॒र् वै गा॑य॒त्री गा॑य॒त्री वा अनु॑मतिः । \newline
32. वा अनु॑मति॒र् अनु॑मति॒र् वै वा अनु॑मति स्त्रि॒ष्टुक् त्रि॒ष्टु गनु॑मति॒र् वै वा अनु॑मति स्त्रि॒ष्टुक् । \newline
33. अनु॑मति स्त्रि॒ष्टुक् त्रि॒ष्टु गनु॑मति॒ रनु॑मति स्त्रि॒ष्टुग् रा॒का रा॒का त्रि॒ष्टु गनु॑मति॒ रनु॑मति स्त्रि॒ष्टुग् रा॒का । \newline
34. अनु॑मति॒रित्य॑नु - म॒तिः॒ । \newline
35. त्रि॒ष्टुग् रा॒का रा॒का त्रि॒ष्टुक् त्रि॒ष्टुग् रा॒का जग॑ती॒ जग॑ती रा॒का त्रि॒ष्टुक् त्रि॒ष्टुग् रा॒का जग॑ती । \newline
36. रा॒का जग॑ती॒ जग॑ती रा॒का रा॒का जग॑ती सिनीवा॒ली सि॑नीवा॒ली जग॑ती रा॒का रा॒का जग॑ती सिनीवा॒ली । \newline
37. जग॑ती सिनीवा॒ली सि॑नीवा॒ली जग॑ती॒ जग॑ती सिनीवा॒ ल्य॑नु॒ष्टु ब॑नु॒ष्टुफ् सि॑नीवा॒ली जग॑ती॒ जग॑ती सिनीवा॒ ल्य॑नु॒ष्टुप् । \newline
38. सि॒नी॒वा॒ ल्य॑नु॒ष्टु ब॑नु॒ष्टुफ् सि॑नीवा॒ली सि॑नीवा॒ ल्य॑नु॒ष्टुप् कु॒हूः कु॒हू र॑नु॒ष्टुफ् सि॑नीवा॒ली सि॑नीवा॒
ल्य॑नु॒ष्टुप् कु॒हूः । \newline
39. अ॒नु॒ष्टुप् कु॒हूः कु॒हू र॑नु॒ष्टु ब॑नु॒ष्टुप् कु॒हूर् धा॒ता धा॒ता कु॒हू र॑नु॒ष्टु ब॑नु॒ष्टुप् कु॒हूर् धा॒ता । \newline
40. अ॒नु॒ष्टुबित्य॑नु - स्तुप् । \newline
41. कु॒हूर् धा॒ता धा॒ता कु॒हूः कु॒हूर् धा॒ता व॑षट्का॒रो व॑षट्का॒रो धा॒ता कु॒हूः कु॒हूर् धा॒ता व॑षट्का॒रः । \newline
42. धा॒ता व॑षट्का॒रो व॑षट्का॒रो धा॒ता धा॒ता व॑षट्का॒रः पू᳚र्वप॒क्षः पू᳚र्वप॒क्षो व॑षट्का॒रो धा॒ता धा॒ता व॑षट्का॒रः पू᳚र्वप॒क्षः । \newline
43. व॒ष॒ट्का॒रः पू᳚र्वप॒क्षः पू᳚र्वप॒क्षो व॑षट्का॒रो व॑षट्का॒रः पू᳚र्वप॒क्षो रा॒का रा॒का 
पू᳚र्वप॒क्षो व॑षट्का॒रो व॑षट्का॒रः पू᳚र्वप॒क्षो रा॒का । \newline
44. व॒ष॒ट्का॒र इति॑ वषट् - का॒रः । \newline
45. पू॒र्व॒प॒क्षो रा॒का रा॒का पू᳚र्वप॒क्षः पू᳚र्वप॒क्षो रा॒का ऽप॑रप॒क्षो॑ ऽपरप॒क्षो रा॒का 
पू᳚र्वप॒क्षः पू᳚र्वप॒क्षो रा॒का ऽप॑रप॒क्षः । \newline
46. पू॒र्व॒प॒क्ष इति॑ पूर्व - प॒क्षः । \newline
47. रा॒का ऽप॑रप॒क्षो॑ ऽपरप॒क्षो रा॒का रा॒का ऽप॑रप॒क्षः कु॒हूः कु॒हू र॑परप॒क्षो रा॒का रा॒का ऽप॑रप॒क्षः कु॒हूः । \newline
48. अ॒प॒र॒प॒क्षः कु॒हूः कु॒हू र॑परप॒क्षो॑ ऽपरप॒क्षः कु॒हू र॑मावा॒स्या॑ ऽमावा॒स्या॑ कु॒हू र॑परप॒क्षो॑ ऽपरप॒क्षः कु॒हू र॑मावा॒स्या᳚ । \newline
49. अ॒प॒र॒प॒क्ष इत्य॑पर - प॒क्षः । \newline
50. कु॒हू र॑मावा॒स्या॑ ऽमावा॒स्या॑ कु॒हूः कु॒हू र॑मावा॒स्या॑ सिनीवा॒ली सि॑नीवा॒ ल्य॑मावा॒स्या॑ कु॒हूः कु॒हूर॑ मावा॒स्या॑ सिनीवा॒ली । \newline
51. अ॒मा॒वा॒स्या॑ सिनीवा॒ली सि॑नीवा॒ ल्य॑मावा॒स्या॑ ऽमावा॒स्या॑ सिनीवा॒ली पौ᳚र्णमा॒सी पौ᳚र्णमा॒सी सि॑नीवा॒
ल्य॑मावा॒स्या॑ ऽमावा॒स्या॑ सिनीवा॒ली पौ᳚र्णमा॒सी । \newline
52. अ॒मा॒वा॒स्येत्य॑मा - वा॒स्या᳚ । \newline
53. सि॒नी॒वा॒ली पौ᳚र्णमा॒सी पौ᳚र्णमा॒सी सि॑नीवा॒ली सि॑नीवा॒ली पौ᳚र्णमा॒ स्यनु॑मति॒ रनु॑मतिः पौर्णमा॒सी सि॑नीवा॒ली 
सि॑नीवा॒ली पौ᳚र्णमा॒ स्यनु॑मतिः । \newline
54. पौ॒र्ण॒मा॒ स्यनु॑मति॒ रनु॑मतिः पौर्णमा॒सी पौ᳚र्णमा॒ स्यनु॑मति श्च॒न्द्रमा᳚ श्च॒न्द्रमा॒ अनु॑मतिः पौर्णमा॒सी पौ᳚र्णमा॒ स्यनु॑मति श्च॒न्द्रमाः᳚ । \newline
55. पौ॒र्ण॒मा॒सीति॑ पौर्ण - मा॒सी । \newline
56. अनु॑मति श्च॒न्द्रमा᳚ श्च॒न्द्रमा॒ अनु॑मति॒ रनु॑मति श्च॒न्द्रमा॑ धा॒ता धा॒ता च॒न्द्रमा॒ अनु॑मति॒ रनु॑मति श्च॒न्द्रमा॑ धा॒ता । \newline
57. अनु॑मति॒रित्य॑नु - म॒तिः॒ । \newline
58. च॒न्द्रमा॑ धा॒ता धा॒ता च॒न्द्रमा᳚ श्च॒न्द्रमा॑ धा॒ता ऽष्टा व॒ष्टौ धा॒ता च॒न्द्रमा᳚ श्च॒न्द्रमा॑ धा॒ता ऽष्टौ । \newline
59. धा॒ता ऽष्टा व॒ष्टौ धा॒ता धा॒ता ऽष्टौ वस॑वो॒ वस॑वो॒ ऽष्टौ धा॒ता धा॒ता ऽष्टौ वस॑वः । \newline
60. अ॒ष्टौ वस॑वो॒ वस॑वो॒ ऽष्टा व॒ष्टौ वस॑वो॒ ऽष्टाक्ष॑रा॒ ऽष्टाक्ष॑रा॒ वस॑वो॒ ऽष्टा व॒ष्टौ वस॑वो॒ ऽष्टाक्ष॑रा । \newline
\pagebreak
\markright{ TS 3.4.9.7  \hfill https://www.vedavms.in \hfill}

\section{ TS 3.4.9.7 }

\textbf{TS 3.4.9.7 } \newline
\textbf{Samhita Paata} \newline

वस॑वो॒ऽष्टाक्ष॑रा गाय॒त्र्येका॑दश रु॒द्रा एका॑दशाक्षरा त्रि॒ष्टुब् द्वाद॑शाऽऽ*दि॒त्या द्वाद॑शाक्षरा॒ जग॑ती प्र॒जाप॑तिरनु॒ष्टुब् धा॒ता व॑षट्का॒र ए॒तद्वै देवि॑काः॒ सर्वा॑णि च॒ छन्दाꣳ॑सि॒ सर्वा᳚श्च दे॒वता॑ वषट्का॒रस्ता यथ् स॒ह सर्वा॑ नि॒र्वपे॑दीश्व॒रा ए॑नं प्र॒दहो॒ द्वे प्र॑थ॒मे नि॒रुप्य॑ धा॒तुस्तृ॒तीयं॒ निव॑र्पे॒त् तथो॑ ए॒वोत्त॑रे॒ निव॑र्पे॒त् तथै॑नं॒ न प्रद॑ह॒न्त्य ( ) थो॒ यस्मै॒ कामा॑य निरु॒प्यन्ते॒ तमे॒वाऽऽ*भि॒रुपा᳚ऽऽ*प्नोति ॥ \newline

\textbf{Pada Paata} \newline

वस॑वः । अ॒ष्टाक्ष॒रेत्य॒ष्टा - अ॒क्ष॒रा॒ । गा॒य॒त्री । एका॑दश । रु॒द्राः । एका॑दशाक्ष॒रेत्येका॑दश - अ॒क्ष॒रा॒ । त्रि॒ष्टुप् । द्वाद॑श । आ॒दि॒त्याः । द्वाद॑शाक्ष॒रेति॒ द्वाद॑श-अ॒क्ष॒रा॒ । जग॑ती । प्र॒जाप॑ति॒रिति॑ प्र॒जा - प॒तिः॒ । अ॒नु॒ष्टुबित्य॑नु-स्तुप् । धा॒ता । व॒ष॒ट्का॒र इति॑ वषट् - का॒रः । ए॒तत् । वै । देवि॑काः । सर्वा॑णि । च॒ । छन्दाꣳ॑सि । सर्वाः᳚ । च॒ । दे॒वताः᳚ । व॒ष॒ट्का॒र इति॑ वषट् - का॒रः । ताः । यत् । स॒ह । सर्वाः᳚ । नि॒र्वपे॒दिति॑ निः - वपे᳚त् । ई॒श्व॒राः । ए॒न॒म् । प्र॒दह॒ इति॑ प्र - दहः॑ । द्वे इति॑ । प्र॒थ॒मे इति॑ । नि॒रुप्येति॑ निः - उप्य॑ । धा॒तुः । तृ॒तीय᳚म् । निरिति॑ । व॒पे॒त् । तथो॒ इति॑ । ए॒व । उत्त॑रे॒ इत्युत् - त॒रे॒ । निरिति॑ । व॒पे॒त् । तथा᳚ । ए॒न॒म् । न । प्रेति॑ । द॒ह॒न्ति॒ ( ) । अथो॒ इति॑ । यस्मै᳚ । कामा॑य । नि॒रु॒प्यन्त॒ इति॑ निः - उ॒प्यन्ते᳚ । तम् । ए॒व । आ॒भिः॒ । उपेति॑ । आ॒प्नो॒ति॒ ॥  \newline


\textbf{Krama Paata} \newline

वस॑वो॒ ऽष्टाक्ष॑रा । अ॒ष्टाक्ष॑रा गाय॒त्री । अ॒ष्टाक्ष॒रेत्य॒ष्टा - अ॒क्ष॒रा॒ । गा॒य॒त्र्येका॑दश । एका॑दश रु॒द्राः । रु॒द्रा एका॑दशाक्षरा । एका॑दशाक्षरा त्रि॒ष्टुप् । एका॑दशाक्ष॒रेत्येका॑दश - अ॒क्ष॒रा॒ । त्रि॒ष्टुब्,द्वाद॑श । द्वाद॑शादि॒त्याः । आ॒दि॒त्या द्वाद॑शाक्षरा । द्वाद॑शाक्षरा॒ जग॑ती । द्वाद॑शाक्ष॒रेति॒ द्वाद॑श - अ॒क्ष॒रा॒ । जग॑ती प्र॒जाप॑तिः । प्र॒जाप॑तिरनु॒ष्टुप् । प्र॒जाप॑ति॒रिति॑ प्र॒जा - प॒तिः॒ । अ॒नु॒ष्टुब्,धा॒ता । अ॒नु॒ष्टुबित्य॑नु - स्तुप् । धा॒ता व॑षट्का॒रः । व॒ष॒ट्का॒र ए॒तत् । व॒ष॒ट्का॒र इति॑ वषट् - का॒रः । ए॒तद् वै । वै देवि॑काः । देवि॑काः॒ सर्वा॑णि । सर्वा॑णि च । च॒ छन्दाꣳ॑सि । छन्दाꣳ॑सि॒ सर्वाः᳚ । सर्वा᳚श्च । च॒ दे॒वताः᳚ । दे॒वता॑ वषट्का॒रः । व॒ष॒ट्का॒रस्ताः । व॒ष॒ट्का॒र इति॑ वषट् - का॒रः । ता यत् । यथ् स॒ह । स॒ह सर्वाः᳚ । सर्वा॑ नि॒र्वपे᳚त् । नि॒र्वपे॑दीश्व॒राः । नि॒र्वपे॒दिति॑ निः - वपे᳚त् । ई॒श्व॒रा ए॑नम् । ए॒न॒म् प्र॒दहः॑ । प्र॒दहो॒ द्वे । प्र॒दह॒ इति॑ प्र - दहः॑ । द्वे प्र॑थ॒मे । द्वे इति॒ द्वे । प्र॒थ॒मे नि॒रुप्य॑ । प्र॒थ॒मे इति॑ प्रथ॒मे । नि॒रुप्य॑ धा॒तुः । नि॒रुप्येति॑ निः - उप्य॑ । धा॒तु स्तृ॒तीय᳚म् । तृ॒तीय॒म् निः । निर् व॑पेत् । व॒पे॒त् तथो᳚ । तथो॑ ए॒व । तथो॒ इति॒ तथो᳚ । ए॒वोत्त॑रे । उत्त॑रे॒ निः । उत्त॑रे॒ इत्युत् - त॒रे॒ । निर् व॑पेत् । व॒पे॒त् तथा᳚ । तथै॑नम् । ए॒न॒म् न । न प्र । प्र द॑हन्ति ( ) । द॒ह॒न्त्यथो᳚ । अथो॒ यस्मै᳚ । अथो॒ इत्यथो᳚ । यस्मै॒ कामा॑य । कामा॑य निरु॒प्यन्ते᳚ । नि॒रु॒प्यन्ते॒ तम् । नि॒रु॒प्यन्त॒ इति॑ निः - उ॒प्यन्ते᳚ । तमे॒व । ए॒वाभिः॑ । आ॒भि॒रुप॑ । उपा᳚प्नोति । आ॒प्नो॒तीत्या᳚प्नोति । \newline

\textbf{Jatai Paata} \newline

1. वस॑वो॒ ऽष्टाक्ष॑रा॒ ऽष्टाक्ष॑रा॒ वस॑वो॒ वस॑वो॒ ऽष्टाक्ष॑रा । \newline
2. अ॒ष्टाक्ष॑रा गाय॒त्री गा॑य॒ त्र्य॑ष्टाक्ष॑रा॒ ऽष्टाक्ष॑रा गाय॒त्री । \newline
3. अ॒ष्टाक्ष॒रेत्य॒ष्टा - अ॒क्ष॒रा॒ । \newline
4. गा॒य॒ त्र्येका॑द॒शै का॑दश गाय॒त्री गा॑य॒ त्र्येका॑दश । \newline
5. एका॑दश रु॒द्रा रु॒द्रा एका॑द॒शै का॑दश रु॒द्राः । \newline
6. रु॒द्रा एका॑दशाक्ष॒ रैका॑दशाक्षरा रु॒द्रा रु॒द्रा एका॑दशाक्षरा । \newline
7. एका॑दशाक्षरा त्रि॒ष्टुप् त्रि॒ष्टु बेका॑दशाक्ष॒ रैका॑दशाक्षरा त्रि॒ष्टुप् । \newline
8. एका॑दशाक्ष॒रेत्येका॑दश - अ॒क्ष॒रा॒ । \newline
9. त्रि॒ष्टुब् द्वाद॑श॒ द्वाद॑श त्रि॒ष्टुप् त्रि॒ष्टुब् द्वाद॑श । \newline
10. द्वाद॑शादि॒त्या आ॑दि॒त्या द्वाद॑श॒ द्वाद॑शादि॒त्याः । \newline
11. आ॒दि॒त्या द्वाद॑शाक्षरा॒ द्वाद॑शाक्षरा ऽऽदि॒त्या आ॑दि॒त्या द्वाद॑शाक्षरा । \newline
12. द्वाद॑शाक्षरा॒ जग॑ती॒ जग॑ती॒ द्वाद॑शाक्षरा॒ द्वाद॑शाक्षरा॒ जग॑ती । \newline
13. द्वाद॑शाक्ष॒रेति॒ द्वाद॑श - अ॒क्ष॒रा॒ । \newline
14. जग॑ती प्र॒जाप॑तिः प्र॒जाप॑ति॒र् जग॑ती॒ जग॑ती प्र॒जाप॑तिः । \newline
15. प्र॒जाप॑ति रनु॒ष्टु ब॑नु॒ष्टुप् प्र॒जाप॑तिः प्र॒जाप॑ति रनु॒ष्टुप् । \newline
16. प्र॒जाप॑ति॒रिति॑ प्र॒जा - प॒तिः॒ । \newline
17. अ॒नु॒ष्टुब् धा॒ता धा॒ता ऽनु॒ष्टु ब॑नु॒ष्टुब् धा॒ता । \newline
18. अ॒नु॒ष्टुबित्य॑नु - स्तुप् । \newline
19. धा॒ता व॑षट्का॒रो व॑षट्का॒रो धा॒ता धा॒ता व॑षट्का॒रः । \newline
20. व॒ष॒ट्का॒र ए॒त दे॒तद् व॑षट्का॒रो व॑षट्का॒र ए॒तत् । \newline
21. व॒ष॒ट्का॒र इति॑ वषट् - का॒रः । \newline
22. ए॒तद् वै वा ए॒त दे॒तद् वै । \newline
23. वै देवि॑का॒ देवि॑का॒ वै वै देवि॑काः । \newline
24. देवि॑काः॒ सर्वा॑णि॒ सर्वा॑णि॒ देवि॑का॒ देवि॑काः॒ सर्वा॑णि । \newline
25. सर्वा॑णि च च॒ सर्वा॑णि॒ सर्वा॑णि च । \newline
26. च॒ छन्दाꣳ॑सि॒ छन्दाꣳ॑सि च च॒ छन्दाꣳ॑सि । \newline
27. छन्दाꣳ॑सि॒ सर्वाः॒ सर्वा॒ श्छन्दाꣳ॑सि॒ छन्दाꣳ॑सि॒ सर्वाः᳚ । \newline
28. सर्वा᳚श्च च॒ सर्वाः॒ सर्वा᳚श्च । \newline
29. च॒ दे॒वता॑ दे॒वता᳚श्च च दे॒वताः᳚ । \newline
30. दे॒वता॑ वषट्का॒रो व॑षट्का॒रो दे॒वता॑ दे॒वता॑ वषट्का॒रः । \newline
31. व॒ष॒ट्का॒र स्ता स्ता व॑षट्का॒रो व॑षट्का॒र स्ताः । \newline
32. व॒ष॒ट्का॒र इति॑ वषट् - का॒रः । \newline
33. ता यद् यत् ता स्ता यत् । \newline
34. यथ् स॒ह स॒ह यद् यथ् स॒ह । \newline
35. स॒ह सर्वाः॒ सर्वाः᳚ स॒ह स॒ह सर्वाः᳚ । \newline
36. सर्वा॑ नि॒र्वपे᳚न् नि॒र्वपे॒थ् सर्वाः॒ सर्वा॑ नि॒र्वपे᳚त् । \newline
37. नि॒र्वपे॑ दीश्व॒रा ई᳚श्व॒रा नि॒र्वपे᳚न् नि॒र्वपे॑ दीश्व॒राः । \newline
38. नि॒र्वपे॒दिति॑ निः - वपे᳚त् । \newline
39. ई॒श्व॒रा ए॑न मेन मीश्व॒रा ई᳚श्व॒रा ए॑नम् । \newline
40. ए॒न॒म् प्र॒दहः॑ प्र॒दह॑ एन मेनम् प्र॒दहः॑ । \newline
41. प्र॒दहो॒ द्वे द्वे प्र॒दहः॑ प्र॒दहो॒ द्वे । \newline
42. प्र॒दह॒ इति॑ प्र - दहः॑ । \newline
43. द्वे प्र॑थ॒मे प्र॑थ॒मे द्वे द्वे प्र॑थ॒मे । \newline
44. द्वे इति॒ द्वे । \newline
45. प्र॒थ॒मे नि॒रुप्य॑ नि॒रुप्य॑ प्रथ॒मे प्र॑थ॒मे नि॒रुप्य॑ । \newline
46. प्र॒थ॒मे इति॑ प्रथ॒मे । \newline
47. नि॒रुप्य॑ धा॒तुर् धा॒तुर् नि॒रुप्य॑ नि॒रुप्य॑ धा॒तुः । \newline
48. नि॒रुप्येति॑ निः - उप्य॑ । \newline
49. धा॒तु स्तृ॒तीय॑म् तृ॒तीय॑म् धा॒तुर् धा॒तु स्तृ॒तीय᳚म् । \newline
50. तृ॒तीय॒म् निर् णिष् टृ॒तीय॑म् तृ॒तीय॒म् निः । \newline
51. निर् व॑पेद् वपे॒न् निर् णिर् व॑पेत् । \newline
52. व॒पे॒त् तथो॒ तथो॑ वपेद् वपे॒त् तथो᳚ । \newline
53. तथो॑ ए॒वैव तथो॒ तथो॑ ए॒व । \newline
54. तथो॒ इति॒ तथो᳚ । \newline
55. ए॒वो त्त॑रे॒ उत्त॑रे ए॒वै वोत्त॑रे । \newline
56. उत्त॑रे॒ निर् णि रुत्त॑रे॒ उत्त॑रे॒ निः । \newline
57. उत्त॑रे॒ इत्युत् - त॒रे॒ । \newline
58. निर् व॑पेद् वपे॒न् निर् णिर् व॑पेत् । \newline
59. व॒पे॒त् तथा॒ तथा॑ वपेद् वपे॒त् तथा᳚ । \newline
60. तथै॑न मेन॒म् तथा॒ तथै॑नम् । \newline
61. ए॒न॒म् न नैन॑ मेन॒म् न । \newline
62. न प्र प्र ण न प्र । \newline
63. प्र द॑हन्ति दहन्ति॒ प्र प्र द॑हन्ति । \newline
64. द॒ह॒ न्त्यथो॒ अथो॑ दहन्ति दह॒ न्त्यथो᳚ । \newline
65. अथो॒ यस्मै॒ यस्मा॒ अथो॒ अथो॒ यस्मै᳚ । \newline
66. अथो॒ इत्यथो᳚ । \newline
67. यस्मै॒ कामा॑य॒ कामा॑य॒ यस्मै॒ यस्मै॒ कामा॑य । \newline
68. कामा॑य निरु॒प्यन्ते॑ निरु॒प्यन्ते॒ कामा॑य॒ कामा॑य निरु॒प्यन्ते᳚ । \newline
69. नि॒रु॒प्यन्ते॒ तम् तम् नि॑रु॒प्यन्ते॑ निरु॒प्यन्ते॒ तम् । \newline
70. नि॒रु॒प्यन्त॒ इति॑ निः - उ॒प्यन्ते᳚ । \newline
71. त मे॒वैव तम् त मे॒व । \newline
72. ए॒वाभि॑र् आभि रे॒वै वाभिः॑ । \newline
73. आ॒भि॒ रुपोपा॑ भिराभि॒ रुप॑ । \newline
74. उपा᳚प्नो त्याप्नो॒ त्युपो पा᳚प्नोति । \newline
75. आ॒प्नो॒तीत्या᳚प्नोति । \newline

\textbf{Ghana Paata } \newline

1. वस॑वो॒ ऽष्टाक्ष॑रा॒ ऽष्टाक्ष॑रा॒ वस॑वो॒ वस॑वो॒ ऽष्टाक्ष॑रा गाय॒त्री गा॑य॒त्र्य॑ष्टाक्ष॑रा॒ वस॑वो॒ वस॑वो॒ ऽष्टाक्ष॑रा गाय॒त्री । \newline
2. अ॒ष्टाक्ष॑रा गाय॒त्री गा॑य॒ त्र्य॑ष्टाक्ष॑रा॒ ऽष्टाक्ष॑रा गाय॒ त्र्येका॑द॒ शैका॑दश 
गाय॒ त्र्य॑ष्टाक्ष॑रा॒ ऽष्टाक्ष॑रा गाय॒ त्र्येका॑दश । \newline
3. अ॒ष्टाक्ष॒रेत्य॒ष्टा - अ॒क्ष॒रा॒ । \newline
4. गा॒य॒ त्र्येका॑द॒ शैका॑दश गाय॒त्री गा॑य॒ त्र्येका॑दश रु॒द्रा रु॒द्रा एका॑दश गाय॒त्री गा॑य॒ त्र्येका॑दश रु॒द्राः । \newline
5. एका॑दश रु॒द्रा रु॒द्रा एका॑द॒ शैका॑दश रु॒द्रा एका॑दशाक्ष॒ रैका॑दशाक्षरा रु॒द्रा एका॑द॒शै का॑दश रु॒द्रा एका॑दशाक्षरा । \newline
6. रु॒द्रा एका॑दशाक्ष॒ रैका॑दशाक्षरा रु॒द्रा रु॒द्रा एका॑दशाक्षरा त्रि॒ष्टुप् त्रि॒ष्टु बेका॑दशाक्षरा रु॒द्रा रु॒द्रा एका॑दशाक्षरा त्रि॒ष्टुप् । \newline
7. एका॑दशाक्षरा त्रि॒ष्टुप् त्रि॒ष्टु बेका॑दशाक्ष॒ रैका॑दशाक्षरा त्रि॒ष्टुब् द्वाद॑श॒ द्वाद॑श त्रि॒ष्टु 
बेका॑दशाक्ष॒ रैका॑दशाक्षरा त्रि॒ष्टुब् द्वाद॑श । \newline
8. एका॑दशाक्ष॒रेत्येका॑दश - अ॒क्ष॒रा॒ । \newline
9. त्रि॒ष्टुब् द्वाद॑श॒ द्वाद॑श त्रि॒ष्टुप् त्रि॒ष्टुब् द्वाद॑शा दि॒त्या आ॑दि॒त्या द्वाद॑श त्रि॒ष्टुप् त्रि॒ष्टुब् द्वाद॑शा दि॒त्याः । \newline
10. द्वाद॑शा दि॒त्या आ॑दि॒त्या द्वाद॑श॒ द्वाद॑शा दि॒त्या द्वाद॑शाक्षरा॒ द्वाद॑शाक्षरा ऽऽदि॒त्या द्वाद॑श॒ द्वाद॑ शादि॒त्या द्वाद॑शाक्षरा । \newline
11. आ॒दि॒त्या द्वाद॑शाक्षरा॒ द्वाद॑शाक्षरा ऽऽदि॒त्या आ॑दि॒त्या द्वाद॑शाक्षरा॒ जग॑ती॒ जग॑ती॒ द्वाद॑शाक्षरा ऽऽदि॒त्या आ॑दि॒त्या द्वाद॑शाक्षरा॒ जग॑ती । \newline
12. द्वाद॑शाक्षरा॒ जग॑ती॒ जग॑ती॒ द्वाद॑शाक्षरा॒ द्वाद॑शाक्षरा॒ जग॑ती प्र॒जाप॑तिः प्र॒जाप॑ति॒र् जग॑ती॒ द्वाद॑शाक्षरा॒ द्वाद॑शाक्षरा॒ जग॑ती प्र॒जाप॑तिः । \newline
13. द्वाद॑शाक्ष॒रेति॒ द्वाद॑श - अ॒क्ष॒रा॒ । \newline
14. जग॑ती प्र॒जाप॑तिः प्र॒जाप॑ति॒र् जग॑ती॒ जग॑ती प्र॒जाप॑ति रनु॒ष्टु ब॑नु॒ष्टुप् प्र॒जाप॑ति॒र् जग॑ती॒ जग॑ती प्र॒जाप॑ति रनु॒ष्टुप् । \newline
15. प्र॒जाप॑ति रनु॒ष्टु ब॑नु॒ष्टुप् प्र॒जाप॑तिः प्र॒जाप॑ति रनु॒ष्टुब् धा॒ता धा॒ता ऽनु॒ष्टुप् प्र॒जाप॑तिः प्र॒जाप॑ति रनु॒ष्टुब् धा॒ता । \newline
16. प्र॒जाप॑ति॒रिति॑ प्र॒जा - प॒तिः॒ । \newline
17. अ॒नु॒ष्टुब् धा॒ता धा॒ता ऽनु॒ष्टु ब॑नु॒ष्टुब् धा॒ता व॑षट्का॒रो व॑षट्का॒रो धा॒ता ऽनु॒ष्टु ब॑नु॒ष्टुब् धा॒ता व॑षट्का॒रः । \newline
18. अ॒नु॒ष्टुबित्य॑नु - स्तुप् । \newline
19. धा॒ता व॑षट्का॒रो व॑षट्का॒रो धा॒ता धा॒ता व॑षट्का॒र ए॒त दे॒तद् व॑षट्का॒रो धा॒ता धा॒ता व॑षट्का॒र ए॒तत् । \newline
20. व॒ष॒ट्का॒र ए॒त दे॒तद् व॑षट्का॒रो व॑षट्का॒र ए॒तद् वै वा ए॒तद् व॑षट्का॒रो व॑षट्का॒र ए॒तद् वै । \newline
21. व॒ष॒ट्का॒र इति॑ वषट् - का॒रः । \newline
22. ए॒तद् वै वा ए॒त दे॒तद् वै देवि॑का॒ देवि॑का॒ वा ए॒त दे॒तद् वै देवि॑काः । \newline
23. वै देवि॑का॒ देवि॑का॒ वै वै देवि॑काः॒ सर्वा॑णि॒ सर्वा॑णि॒ देवि॑का॒ वै वै देवि॑काः॒ सर्वा॑णि । \newline
24. देवि॑काः॒ सर्वा॑णि॒ सर्वा॑णि॒ देवि॑का॒ देवि॑काः॒ सर्वा॑णि च च॒ सर्वा॑णि॒ देवि॑का॒ देवि॑काः॒ सर्वा॑णि च । \newline
25. सर्वा॑णि च च॒ सर्वा॑णि॒ सर्वा॑णि च॒ छन्दाꣳ॑सि॒ छन्दाꣳ॑सि च॒ सर्वा॑णि॒ सर्वा॑णि च॒ छन्दाꣳ॑सि । \newline
26. च॒ छन्दाꣳ॑सि॒ छन्दाꣳ॑सि च च॒ छन्दाꣳ॑सि॒ सर्वाः॒ सर्वा॒ श्छन्दाꣳ॑सि च च॒ छन्दाꣳ॑सि॒ सर्वाः᳚ । \newline
27. छन्दाꣳ॑सि॒ सर्वाः॒ सर्वा॒ श्छन्दाꣳ॑सि॒ छन्दाꣳ॑सि॒ सर्वा᳚श्च च॒ सर्वा॒ श्छन्दाꣳ॑सि॒ छन्दाꣳ॑सि॒ सर्वा᳚श्च । \newline
28. सर्वा᳚श्च च॒ सर्वाः॒ सर्वा᳚श्च दे॒वता॑ दे॒वता᳚श्च॒ सर्वाः॒ सर्वा᳚श्च दे॒वताः᳚ । \newline
29. च॒ दे॒वता॑ दे॒वता᳚श्च च दे॒वता॑ वषट्का॒रो व॑षट्का॒रो दे॒वता᳚श्च च दे॒वता॑ वषट्का॒रः । \newline
30. दे॒वता॑ वषट्का॒रो व॑षट्का॒रो दे॒वता॑ दे॒वता॑ वषट्का॒र स्ता स्ता व॑षट्का॒रो दे॒वता॑ दे॒वता॑ वषट्का॒र स्ताः । \newline
31. व॒ष॒ट्का॒र स्ता स्ता व॑षट्का॒रो व॑षट्का॒र स्ता यद् यत् ता व॑षट्का॒रो व॑षट्का॒र स्ता यत् । \newline
32. व॒ष॒ट्का॒र इति॑ वषट् - का॒रः । \newline
33. ता यद् यत् ता स्ता यथ् स॒ह स॒ह यत् ता स्ता यथ् स॒ह । \newline
34. यथ् स॒ह स॒ह यद् यथ् स॒ह सर्वाः॒ सर्वाः᳚ स॒ह यद् यथ् स॒ह सर्वाः᳚ । \newline
35. स॒ह सर्वाः॒ सर्वाः᳚ स॒ह स॒ह सर्वा॑ नि॒र्वपे᳚न् नि॒र्वपे॒थ् सर्वाः᳚ स॒ह स॒ह सर्वा॑ नि॒र्वपे᳚त् । \newline
36. सर्वा॑ नि॒र्वपे᳚न् नि॒र्वपे॒थ् सर्वाः॒ सर्वा॑ नि॒र्वपे॑ दीश्व॒रा ई᳚श्व॒रा नि॒र्वपे॒थ् सर्वाः॒ सर्वा॑ 
नि॒र्वपे॑ दीश्व॒राः । \newline
37. नि॒र्वपे॑ दीश्व॒रा ई᳚श्व॒रा नि॒र्वपे᳚न् नि॒र्वपे॑ दीश्व॒रा ए॑न मेन मीश्व॒रा नि॒र्वपे᳚न् नि॒र्वपे॑ दीश्व॒रा ए॑नम् । \newline
38. नि॒र्वपे॒दिति॑ निः - वपे᳚त् । \newline
39. ई॒श्व॒रा ए॑न मेन मीश्व॒रा ई᳚श्व॒रा ए॑नम् प्र॒दहः॑ प्र॒दह॑ एन मीश्व॒रा ई᳚श्व॒रा ए॑नम् प्र॒दहः॑ । \newline
40. ए॒न॒म् प्र॒दहः॑ प्र॒दह॑ एन मेनम् प्र॒दहो॒ द्वे द्वे प्र॒दह॑ एन मेनम् प्र॒दहो॒ द्वे । \newline
41. प्र॒दहो॒ द्वे द्वे प्र॒दहः॑ प्र॒दहो॒ द्वे प्र॑थ॒मे प्र॑थ॒मे द्वे प्र॒दहः॑ प्र॒दहो॒ द्वे प्र॑थ॒मे । \newline
42. प्र॒दह॒ इति॑ प्र - दहः॑ । \newline
43. द्वे प्र॑थ॒मे प्र॑थ॒मे द्वे द्वे प्र॑थ॒मे नि॒रुप्य॑ नि॒रुप्य॑ प्रथ॒मे द्वे द्वे प्र॑थ॒मे नि॒रुप्य॑ । \newline
44. द्वे इति॒ द्वे । \newline
45. प्र॒थ॒मे नि॒रुप्य॑ नि॒रुप्य॑ प्रथ॒मे प्र॑थ॒मे नि॒रुप्य॑ धा॒तुर् धा॒तुर् नि॒रुप्य॑ प्रथ॒मे प्र॑थ॒मे नि॒रुप्य॑ धा॒तुः । \newline
46. प्र॒थ॒मे इति॑ प्रथ॒मे । \newline
47. नि॒रुप्य॑ धा॒तुर् धा॒तुर् नि॒रुप्य॑ नि॒रुप्य॑ धा॒तु स्तृ॒तीय॑म् तृ॒तीय॑म् धा॒तुर् नि॒रुप्य॑ नि॒रुप्य॑ धा॒तु स्तृ॒तीय᳚म् । \newline
48. नि॒रुप्येति॑ निः - उप्य॑ । \newline
49. धा॒तु स्तृ॒तीय॑म् तृ॒तीय॑म् धा॒तुर् धा॒तु स्तृ॒तीय॒म् निर् णिष् टृ॒तीय॑म् धा॒तुर् धा॒तु स्तृ॒तीय॒म् निः । \newline
50. तृ॒तीय॒म् निर् णिष् टृ॒तीय॑म् तृ॒तीय॒म् निर् व॑पेद् वपे॒न् निष् टृ॒तीय॑म् तृ॒तीय॒म् निर् व॑पेत् । \newline
51. निर् व॑पेद् वपे॒न् निर् णिर् व॑पे॒त् तथो॒ तथो॑ वपे॒न् निर् णिर् व॑पे॒त् तथो᳚ । \newline
52. व॒पे॒त् तथो॒ तथो॑ वपेद् वपे॒त् तथो॑ ए॒वैव तथो॑ वपेद् वपे॒त् तथो॑ ए॒व । \newline
53. तथो॑ ए॒वैव तथो॒ तथो॑ ए॒वोत्त॑रे॒ उत्त॑रे ए॒व तथो॒ तथो॑ ए॒वोत्त॑रे । \newline
54. तथो॒ इति॒ तथो᳚ । \newline
55. ए॒वोत्त॑रे॒ उत्त॑रे ए॒वैवोत्त॑रे॒ निर् णिर् उत्त॑रे ए॒वैवोत्त॑रे॒ निः । \newline
56. उत्त॑रे॒ निर् णि रुत्त॑रे॒ उत्त॑रे॒ निर् व॑पेद् वपे॒न् नि रुत्त॑रे॒ उत्त॑रे॒ निर् व॑पेत् । \newline
57. उत्त॑रे॒ इत्युत् - त॒रे॒ । \newline
58. निर् व॑पेद् वपे॒न् निर् णिर् व॑पे॒त् तथा॒ तथा॑ वपे॒न् निर् णिर् व॑पे॒त् तथा᳚ । \newline
59. व॒पे॒त् तथा॒ तथा॑ वपेद् वपे॒त् तथै॑न मेन॒म् तथा॑ वपेद् वपे॒त् तथै॑नम् । \newline
60. तथै॑न मेन॒म् तथा॒ तथै॑न॒म्,न नैन॒म् तथा॒ तथै॑न॒म्,न । \newline
61. ए॒न॒म्,न नैन॑ मेन॒म् न प्र प्र णैन॑ मेन॒म्,न प्र । \newline
62. न प्र प्र ण न प्र द॑हन्ति दहन्ति॒ प्र ण न प्र द॑हन्ति । \newline
63. प्र द॑हन्ति दहन्ति॒ प्र प्र द॑ह॒न्त्यथो॒ अथो॑ दहन्ति॒ प्र प्र द॑ह॒न्त्यथो᳚ । \newline
64. द॒ह॒न्त्यथो॒ अथो॑ दहन्ति दह॒न्त्यथो॒ यस्मै॒ यस्मा॒ अथो॑ दहन्ति दह॒न्त्यथो॒ यस्मै᳚ । \newline
65. अथो॒ यस्मै॒ यस्मा॒ अथो॒ अथो॒ यस्मै॒ कामा॑य॒ कामा॑य॒ यस्मा॒ अथो॒ अथो॒ यस्मै॒ कामा॑य । \newline
66. अथो॒ इत्यथो᳚ । \newline
67. यस्मै॒ कामा॑य॒ कामा॑य॒ यस्मै॒ यस्मै॒ कामा॑य निरु॒प्यन्ते॑ निरु॒प्यन्ते॒ कामा॑य॒ यस्मै॒ यस्मै॒ कामा॑य निरु॒प्यन्ते᳚ । \newline
68. कामा॑य निरु॒प्यन्ते॑ निरु॒प्यन्ते॒ कामा॑य॒ कामा॑य निरु॒प्यन्ते॒ तम् तम् नि॑रु॒प्यन्ते॒ कामा॑य॒ कामा॑य निरु॒प्यन्ते॒ तम् । \newline
69. नि॒रु॒प्यन्ते॒ तम् तम् नि॑रु॒प्यन्ते॑ निरु॒प्यन्ते॒ तमे॒वैव तम् नि॑रु॒प्यन्ते॑ निरु॒प्यन्ते॒ तमे॒व । \newline
70. नि॒रु॒प्यन्त॒ इति॑ निः - उ॒प्यन्ते᳚ । \newline
71. तमे॒वैव तम् त मे॒वाभि॑ राभि रे॒व तम् त मे॒वाभिः॑ । \newline
72. ए॒वाभि॑ राभि रे॒वैवाभि॒ रुपोपा॑भि रे॒वैवाभि॒ रुप॑ । \newline
73. आ॒भि॒ रुपोपा॑भि राभि॒ रुपा᳚प्नो त्याप्नो॒ त्युपा॑भि राभि॒ रुपा᳚प्नोति । \newline
74. उपा᳚प्नो त्याप्नो॒ त्युपोपा᳚प्नोति । \newline
75. आ॒प्नो॒तीत्या᳚प्नोति । \newline
\pagebreak
\markright{ TS 3.4.10.1  \hfill https://www.vedavms.in \hfill}

\section{ TS 3.4.10.1 }

\textbf{TS 3.4.10.1 } \newline
\textbf{Samhita Paata} \newline

वास्तो᳚ष्पते॒ प्रति॑ जानी ह्य॒स्मान्थ् स्वा॑वे॒शो अ॑नमी॒वो भ॑वानः । यत् त्वेम॑हे॒ प्रति॒तन्नो॑ जुषस्व॒ शन्न॑ एधि द्वि॒पदे॒ शञ्चतु॑ष्पदे ॥वास्तो᳚ष्पते श॒ग्मया॑ सꣳ॒॒ सदा॑ते सक्षी॒महि॑ र॒ण्वया॑ गातु॒मत्या᳚ । आवः॒ क्षेम॑ उ॒त योगे॒ वर॑न्नो यू॒यं पा॑त स्व॒स्तिभिः॒ सदा॑नः ॥ यथ् सा॒यं प्रा॑तरग्निहो॒त्रं जु॒होत्या॑हुतीष्ट॒का ए॒व ता उप॑ धत्ते॒ - [  ] \newline

\textbf{Pada Paata} \newline

वास्तोः᳚ । प॒ते॒ । प्रतीति॑ । जा॒नी॒हि॒ । अ॒स्मान् । स्वा॒वे॒श इति॑ सु-आ॒वे॒शः । अ॒न॒मी॒वः । भ॒व॒ । नः॒ ॥ यत् । त्वा॒ । ईम॑हे । प्रतीति॑ । तत् । नः॒ । जु॒ष॒स्व॒ । शम् । नः॒ । ए॒धि॒ । द्वि॒पद॒ इति॑ द्वि - पदे᳚ । शम् । चतु॑ष्पद॒ इति॒ चतुः॑ - प॒दे॒ ॥ वास्तोः᳚ । प॒ते॒ । श॒ग्मया᳚ । सꣳ॒॒सदेति॑ सं - सदा᳚ । ते॒ । स॒क्षी॒महि॑ । र॒ण्वया᳚ । गा॒तु॒मत्येति॑ गातु - मत्या᳚ ॥ आवः॑ । क्षेमे᳚ । उ॒त । योगे᳚ । वर᳚म् । नः॒ । यू॒यम् । पा॒त॒ । स्व॒स्तिभि॒रिति॑ स्व॒स्ति - भिः॒ । सदा᳚ । नः॒ ॥ यत् । सा॒यंप्रा॑त॒रिति॑ सा॒यं - प्रा॒तः॒ । अ॒ग्नि॒हो॒त्रमित्य॑ग्नि - हो॒त्रम् । जु॒होति॑ । आ॒हु॒ती॒ष्ट॒का इत्या॑हुति - इ॒ष्ट॒काः । ए॒व । ताः । उपेति॑ । ध॒त्ते॒ ।  \newline


\textbf{Krama Paata} \newline

वास्तो᳚ष्पते । प॒ते॒ प्रति॑ । प्रति॑ जानीहि । जा॒नी॒ह्य॒स्मान् । अ॒स्मान्थ् स्वा॑वे॒शः । स्वा॒वे॒शो अ॑नमी॒वः । स्वा॒वे॒श इति॑ सु - आ॒वे॒शः । अ॒न॒मी॒वो भ॑व । भ॒वा॒ नः॒ । न॒ इति॑ नः ॥ यत् त्वा᳚ । त्वेम॑हे । ईम॑हे॒ प्रति॑ । प्रति॒ तत् । तन् नः॑ । नो॒ जु॒ष॒स्व॒ । जु॒ष॒स्व॒ शम् । शम् नः॑ । न॒ ए॒धि॒ । ए॒धि॒ द्वि॒पदे᳚ । द्वि॒पदे॒ शम् । द्वि॒पद॒ इति॑ द्वि - पदे᳚ । शम् चतु॑ष्पदे । चतु॑ष्पद॒ इति॒ चतुः॑ - प॒दे॒ ॥ वास्तो᳚ष्पते । प॒ते॒ श॒ग्मया᳚ । श॒ग्मया॑ सꣳ॒॒सदा᳚ । सꣳ॒॒सदा॑ ते । सꣳ॒॒सदेति॑ सम् - सदा᳚ । ते॒ स॒क्षी॒महि॑ । स॒क्षी॒महि॑ र॒ण्वया᳚ । र॒ण्वया॑ गातु॒मत्या᳚ । गा॒तु॒मत्येति॑ गातु - मत्या᳚ ॥ आवः॒ क्षेमे᳚ । क्षेम॑ उ॒त । उ॒त योगे᳚ । योगे॒ वर᳚म् । वर॑म् नः । नो॒ यू॒यम् । यू॒यम् पा॑त । पा॒त॒ स्व॒स्तिभिः॑ । स्व॒स्तिभिः॒ सदा᳚ । स्व॒स्तिभि॒रिति॑ स्व॒स्ति - भिः॒ । सदा॑ नः । न॒ इति॑ नः ॥ यथ् सा॒यम्प्रा॑तः । सा॒यम्प्रा॑त,रग्निहो॒त्रम् । सा॒यम्प्रा॑त॒रिति॑ सा॒यम् - प्रा॒तः॒ । अ॒ग्नि॒हो॒त्रम् जु॒होति॑ । अ॒ग्नि॒हो॒त्रमित्य॑ग्नि - हो॒त्रम् । जु॒होत्या॑हुतीष्ट॒काः । आ॒हु॒ती॒ष्ट॒का ए॒व । आ॒हु॒ती॒ष्ट॒का इत्या॑हुति - इ॒ष्ट॒काः । ए॒व ताः । ता उप॑ । उप॑ धत्ते । ध॒त्ते॒ यज॑मानः \newline

\textbf{Jatai Paata} \newline

1. वास्तो᳚ष् पते पते॒ वास्तो॒र् वास्तो᳚ष् पते । \newline
2. प॒ते॒ प्रति॒ प्रति॑ पते पते॒ प्रति॑ । \newline
3. प्रति॑ जानीहि जानीहि॒ प्रति॒ प्रति॑ जानीहि । \newline
4. जा॒नी॒ ह्य॒स्मा-न॒स्मान् जा॑नीहि जानी ह्य॒स्मान् । \newline
5. अ॒स्मान् थ्स्वा॑वे॒शः स्वा॑वे॒शो᳚ ऽस्मा-न॒स्मान् थ्स्वा॑वे॒शः । \newline
6. स्वा॒वे॒शो अ॑नमी॒वो अ॑नमी॒वः स्वा॑वे॒शः स्वा॑वे॒शो अ॑नमी॒वः । \newline
7. स्वा॒वे॒श इति॑ सु - आ॒वे॒शः । \newline
8. अ॒न॒मी॒वो भ॑व भवा नमी॒वो अ॑नमी॒वो भ॑व । \newline
9. भ॒वा॒ नो॒ नो॒ भ॒व॒ भ॒वा॒ नः॒ । \newline
10. न॒ इति॑ नः । \newline
11. यत् त्वा᳚ त्वा॒ यद् यत् त्वा᳚ । \newline
12. त्वे म॑ह॒ ईम॑हे त्वा॒ त्वे म॑हे । \newline
13. ईम॑हे॒ प्रति॒ प्रती म॑ह॒ ईम॑हे॒ प्रति॑ । \newline
14. प्रति॒ तत् तत् प्रति॒ प्रति॒ तत् । \newline
15. तन् नो॑ न॒ स्तत् तन् नः॑ । \newline
16. नो॒ जु॒ष॒स्व॒ जु॒ष॒स्व॒ नो॒ नो॒ जु॒ष॒स्व॒ । \newline
17. जु॒ष॒स्व॒ शꣳ शम् जु॑षस्व जुषस्व॒ शम् । \newline
18. शन्नो॑ नः॒ शꣳ शन्नः॑ । \newline
19. न॒ ए॒ध्ये॒धि॒ नो॒ न॒ ए॒धि॒ । \newline
20. ए॒धि॒ द्वि॒पदे᳚ द्वि॒पद॑ एध्येधि द्वि॒पदे᳚ । \newline
21. द्वि॒पदे॒ शꣳ शम् द्वि॒पदे᳚ द्वि॒पदे॒ शम् । \newline
22. द्वि॒पद॒ इति॑ द्वि - पदे᳚ । \newline
23. शम् चतु॑ष्पदे॒ चतु॑ष्पदे॒ शꣳ शम् चतु॑ष्पदे । \newline
24. चतु॑ष्पद॒ इति॒ चतुः॑ - प॒दे॒ । \newline
25. वास्तो᳚ष् पते पते॒ वास्तो॒र् वास्तो᳚ष् पते । \newline
26. प॒ते॒ श॒ग्मया॑ श॒ग्मया॑ पते पते श॒ग्मया᳚ । \newline
27. श॒ग्मया॑ सꣳ॒॒सदा॑ सꣳ॒॒सदा॑ श॒ग्मया॑ श॒ग्मया॑ सꣳ॒॒सदा᳚ । \newline
28. सꣳ॒॒सदा॑ ते ते सꣳ॒॒सदा॑ सꣳ॒॒सदा॑ ते । \newline
29. सꣳ॒॒सदेति॑ सं - सदा᳚ । \newline
30. ते॒ स॒क्षी॒महि॑ सक्षी॒महि॑ ते ते सक्षी॒महि॑ । \newline
31. स॒क्षी॒महि॑ र॒ण्वया॑ र॒ण्वया॑ सक्षी॒महि॑ सक्षी॒महि॑ र॒ण्वया᳚ । \newline
32. र॒ण्वया॑ गातु॒मत्या॑ गातु॒मत्या॑ र॒ण्वया॑ र॒ण्वया॑ गातु॒मत्या᳚ । \newline
33. गा॒तु॒मत्येति॑ गातु - मत्या᳚ । \newline
34. आवः॒ क्षेमे॒ क्षेम॒ आव॒ आवः॒ क्षेमे᳚ । \newline
35. क्षेम॑ उ॒तोत क्षेमे॒ क्षेम॑ उ॒त । \newline
36. उ॒त योगे॒ योग॑ उ॒तोत योगे᳚ । \newline
37. योगे॒ वरं॒ ॅवरं॒ ॅयोगे॒ योगे॒ वर᳚म् । \newline
38. वर॑न्नो नो॒ वरं॒ ॅवर॑न्नः । \newline
39. नो॒ यू॒यं ॅयू॒यन्नो॑ नो यू॒यम् । \newline
40. यू॒यम् पा॑त पात यू॒यं ॅयू॒यम् पा॑त । \newline
41. पा॒त॒ स्व॒स्तिभिः॑ स्व॒स्तिभिः॑ पात पात स्व॒स्तिभिः॑ । \newline
42. स्व॒स्तिभिः॒ सदा॒ सदा᳚ स्व॒स्तिभिः॑ स्व॒स्तिभिः॒ सदा᳚ । \newline
43. स्व॒स्तिभि॒रिति॑ स्व॒स्ति - भिः॒ । \newline
44. सदा॑ नो नः॒ सदा॒ सदा॑ नः । \newline
45. न॒ इति॑ नः । \newline
46. यथ् सा॒यम्प्रा॑तः सा॒यम्प्रा॑तो॒ यद् यथ् सा॒यम्प्रा॑तः । \newline
47. सा॒यम्प्रा॑त रग्निहो॒त्र म॑ग्निहो॒त्रꣳ सा॒यम्प्रा॑तः सा॒यम्प्रा॑त रग्निहो॒त्रम् । \newline
48. सा॒यम्प्रा॑त॒रिति॑ सा॒यं - प्रा॒तः॒ । \newline
49. अ॒ग्नि॒हो॒त्रम् जु॒होति॑ जु॒होत्य॑ ग्निहो॒त्र म॑ग्निहो॒त्रम् जु॒होति॑ । \newline
50. अ॒ग्नि॒हो॒त्रमित्य॑ग्नि - हो॒त्रम् । \newline
51. जु॒हो त्या॑हुतीष्ट॒का आ॑हुतीष्ट॒का जु॒होति॑ जु॒हो त्या॑हुतीष्ट॒काः । \newline
52. आ॒हु॒ती॒ष्ट॒का ए॒वैवा हु॑तीष्ट॒का आ॑हुतीष्ट॒का ए॒व । \newline
53. आ॒हु॒ती॒ष्ट॒का इत्या॑हुति - इ॒ष्ट॒काः । \newline
54. ए॒व ता स्ता ए॒वै व ताः । \newline
55. ता उपोप॒ तास्ता उप॑ । \newline
56. उप॑ धत्ते धत्त॒ उपोप॑ धत्ते । \newline
57. ध॒त्ते॒ यज॑मानो॒ यज॑मानो धत्ते धत्ते॒ यज॑मानः । \newline

\textbf{Ghana Paata } \newline

1. वास्तो᳚ष् पते पते॒ वास्तो॒र् वास्तो᳚ष् पते॒ प्रति॒ प्रति॑ पते॒ वास्तो॒र् वास्तो᳚ष् पते॒ प्रति॑ । \newline
2. प॒ते॒ प्रति॒ प्रति॑ पते पते॒ प्रति॑ जानीहि जानीहि॒ प्रति॑ पते पते॒ प्रति॑ जानीहि । \newline
3. प्रति॑ जानीहि जानीहि॒ प्रति॒ प्रति॑ जानी ह्य॒स्मा,न॒स्मान् जा॑नीहि॒ प्रति॒ प्रति॑ जानीह्य॒स्मान् । \newline
4. जा॒नी॒ह्य॒स्मा,न॒स्मान् जा॑नीहि जानीह्य॒स्मान् थ्स्वा॑वे॒शः स्वा॑वे॒शो᳚ ऽस्मान् जा॑नीहि जानीह्य॒स्मान् थ्स्वा॑वे॒शः । \newline
5. अ॒स्मान् थ्स्वा॑वे॒शः स्वा॑वे॒शो᳚ ऽस्मान॒स्मान् थ्स्वा॑वे॒शो अ॑नमी॒वो अ॑नमी॒वः स्वा॑वे॒शो᳚ ऽस्मान॒स्मान् थ्स्वा॑वे॒शो अ॑नमी॒वः । \newline
6. स्वा॒वे॒शो अ॑नमी॒वो अ॑नमी॒वः स्वा॑वे॒शः स्वा॑वे॒शो अ॑नमी॒वो भ॑व भवा नमी॒वः स्वा॑वे॒शः स्वा॑वे॒शो अ॑नमी॒वो भ॑व । \newline
7. स्वा॒वे॒श इति॑ सु - आ॒वे॒शः । \newline
8. अ॒न॒मी॒वो भ॑व भवा नमी॒वो अ॑नमी॒वो भ॑वा नो नो भवा नमी॒वो अ॑नमी॒वो भ॑वा नः । \newline
9. भ॒वा॒ नो॒ नो॒ भ॒व॒ भ॒वा॒ नः॒ । \newline
10. न॒ इति॑ नः । \newline
11. यत् त्वा᳚ त्वा॒ यद् यत् त्वेम॑ह॒ ईम॑हे त्वा॒ यद् यत् त्वेम॑हे । \newline
12. त्वेम॑ह॒ ईम॑हे त्वा॒ त्वेम॑हे॒ प्रति॒ प्रती म॑हे त्वा॒ त्वेम॑हे॒ प्रति॑ । \newline
13. ईम॑हे॒ प्रति॒ प्रती म॑ह॒ ईम॑हे॒ प्रति॒ तत् तत् प्रती म॑ह॒ ईम॑हे॒ प्रति॒ तत् । \newline
14. प्रति॒ तत् तत् प्रति॒ प्रति॒ तन् नो॑ न॒ स्तत् प्रति॒ प्रति॒ तन् नः॑ । \newline
15. तन् नो॑ न॒ स्तत् तन् नो॑ जुषस्व जुषस्व न॒ स्तत् तन् नो॑ जुषस्व । \newline
16. नो॒ जु॒ष॒स्व॒ जु॒ष॒स्व॒ नो॒ नो॒ जु॒ष॒स्व॒ शꣳ शम् जु॑षस्व नो नो जुषस्व॒ शम् । \newline
17. जु॒ष॒स्व॒ शꣳ शम् जु॑षस्व जुषस्व॒ शम् नो॑ नः॒ शम् जु॑षस्व जुषस्व॒ शम् नः॑ । \newline
18. शम् नो॑ नः॒ शꣳ शम् न॑ एध्येधि नः॒ शꣳ शम् न॑ एधि । \newline
19. न॒ ए॒ध्ये॒धि॒ नो॒ न॒ ए॒धि॒ द्वि॒पदे᳚ द्वि॒पद॑ एधि नो न एधि द्वि॒पदे᳚ । \newline
20. ए॒धि॒ द्वि॒पदे᳚ द्वि॒पद॑ एध्येधि द्वि॒पदे॒ शꣳ शम् द्वि॒पद॑ एध्येधि द्वि॒पदे॒ शम् । \newline
21. द्वि॒पदे॒ शꣳ शम् द्वि॒पदे᳚ द्वि॒पदे॒ शम् चतु॑ष्पदे॒ चतु॑ष्पदे॒ शम् द्वि॒पदे᳚ द्वि॒पदे॒ शम् चतु॑ष्पदे । \newline
22. द्वि॒पद॒ इति॑ द्वि - पदे᳚ । \newline
23. शम् चतु॑ष्पदे॒ चतु॑ष्पदे॒ शꣳ शम् चतु॑ष्पदे । \newline
24. चतु॑ष्पद॒ इति॒ चतुः॑ - प॒दे॒ । \newline
25. वास्तो᳚ष् पते पते॒ वास्तो॒र् वास्तो᳚ष् पते श॒ग्मया॑ श॒ग्मया॑ पते॒ वास्तो॒र् वास्तो᳚ष् पते श॒ग्मया᳚ । \newline
26. प॒ते॒ श॒ग्मया॑ श॒ग्मया॑ पते पते श॒ग्मया॑ सꣳ॒॒सदा॑ सꣳ॒॒सदा॑ श॒ग्मया॑ पते पते श॒ग्मया॑ सꣳ॒॒सदा᳚ । \newline
27. श॒ग्मया॑ सꣳ॒॒सदा॑ सꣳ॒॒सदा॑ श॒ग्मया॑ श॒ग्मया॑ सꣳ॒॒सदा॑ ते ते सꣳ॒॒सदा॑ श॒ग्मया॑ 
श॒ग्मया॑ सꣳ॒॒सदा॑ ते । \newline
28. सꣳ॒॒सदा॑ ते ते सꣳ॒॒सदा॑ सꣳ॒॒सदा॑ ते सक्षी॒महि॑ सक्षी॒महि॑ ते सꣳ॒॒सदा॑ सꣳ॒॒सदा॑ ते सक्षी॒महि॑ । \newline
29. सꣳ॒॒सदेति॑ सम् - सदा᳚ । \newline
30. ते॒ स॒क्षी॒महि॑ सक्षी॒महि॑ ते ते सक्षी॒महि॑ र॒ण्वया॑ र॒ण्वया॑ सक्षी॒महि॑ ते ते सक्षी॒महि॑ र॒ण्वया᳚ । \newline
31. स॒क्षी॒महि॑ र॒ण्वया॑ र॒ण्वया॑ सक्षी॒महि॑ सक्षी॒महि॑ र॒ण्वया॑ गातु॒मत्या॑ गातु॒मत्या॑ र॒ण्वया॑ सक्षी॒महि॑ सक्षी॒महि॑ र॒ण्वया॑ गातु॒मत्या᳚ । \newline
32. र॒ण्वया॑ गातु॒मत्या॑ गातु॒मत्या॑ र॒ण्वया॑ र॒ण्वया॑ गातु॒मत्या᳚ । \newline
33. गा॒तु॒मत्येति॑ गातु - मत्या᳚ । \newline
34. आवः॒ क्षेमे॒ क्षेम॒ आव॒ आवः॒ क्षेम॑ उ॒तोत क्षेम॒ आव॒ आवः॒ क्षेम॑ उ॒त । \newline
35. क्षेम॑ उ॒तोत क्षेमे॒ क्षेम॑ उ॒त योगे॒ योग॑ उ॒त क्षेमे॒ क्षेम॑ उ॒त योगे᳚ । \newline
36. उ॒त योगे॒ योग॑ उ॒तोत योगे॒ वर॒म् ॅवर॒म् ॅयोग॑ उ॒तोत योगे॒ वर᳚म् । \newline
37. योगे॒ वर॒म् ॅवर॒म् ॅयोगे॒ योगे॒ वर॑म् नो नो॒ वर॒म् ॅयोगे॒ योगे॒ वर॑म् नः । \newline
38. वर॑म् नो नो॒ वर॒म् ॅवर॑म् नो यू॒यम् ॅयू॒यम् नो॒ वर॒म् ॅवर॑म् नो यू॒यम् । \newline
39. नो॒ यू॒यम् ॅयू॒यम् नो॑ नो यू॒यम् पा॑त पात यू॒यम् नो॑ नो यू॒यम् पा॑त । \newline
40. यू॒यम् पा॑त पात यू॒यम् ॅयू॒यम् पा॑त स्व॒स्तिभिः॑ स्व॒स्तिभिः॑ पात यू॒यम् ॅयू॒यम् पा॑त स्व॒स्तिभिः॑ । \newline
41. पा॒त॒ स्व॒स्तिभिः॑ स्व॒स्तिभिः॑ पात पात स्व॒स्तिभिः॒ सदा॒ सदा᳚ स्व॒स्तिभिः॑ पात पात स्व॒स्तिभिः॒ सदा᳚ । \newline
42. स्व॒स्तिभिः॒ सदा॒ सदा᳚ स्व॒स्तिभिः॑ स्व॒स्तिभिः॒ सदा॑ नो नः॒ सदा᳚ स्व॒स्तिभिः॑ स्व॒स्तिभिः॒ सदा॑ नः । \newline
43. स्व॒स्तिभि॒रिति॑ स्व॒स्ति - भिः॒ । \newline
44. सदा॑ नो नः॒ सदा॒ सदा॑ नः । \newline
45. न॒ इति॑ नः । \newline
46. यथ् सा॒यम्प्रा॑तः सा॒यम्प्रा॑तो॒ यद् यथ् सा॒यम्प्रा॑त रग्निहो॒त्र म॑ग्निहो॒त्रꣳ सा॒यम्प्रा॑तो॒ यद् यथ् सा॒यम्प्रा॑त रग्निहो॒त्रम् । \newline
47. सा॒यम्प्रा॑त रग्निहो॒त्र म॑ग्निहो॒त्रꣳ सा॒यम्प्रा॑तः सा॒यम्प्रा॑त रग्निहो॒त्रम् जु॒होति॑ जु॒हो त्य॑ग्निहो॒त्रꣳ 
सा॒यम्प्रा॑तः सा॒यम्प्रा॑त रग्निहो॒त्रम् जु॒होति॑ । \newline
48. सा॒यम्प्रा॑त॒रिति॑ सा॒यम् - प्रा॒तः॒ । \newline
49. अ॒ग्नि॒हो॒त्रम् जु॒होति॑ जु॒हो त्य॑ग्निहो॒त्र म॑ग्निहो॒त्रम् जु॒हो त्या॑हुतीष्ट॒का आ॑हुतीष्ट॒का जु॒हो त्य॑ग्निहो॒त्र 
म॑ग्निहो॒त्रम् जु॒हो त्या॑हुतीष्ट॒काः । \newline
50. अ॒ग्नि॒हो॒त्रमित्य॑ग्नि - हो॒त्रम् । \newline
51. जु॒हो त्या॑हुतीष्ट॒का आ॑हुतीष्ट॒का जु॒होति॑ जु॒हो त्या॑हुतीष्ट॒का ए॒वैवाहु॑तीष्ट॒का जु॒होति॑ जु॒हो त्या॑हुतीष्ट॒का ए॒व । \newline
52. आ॒हु॒ती॒ष्ट॒का ए॒वै वाहु॑तीष्ट॒का आ॑हुतीष्ट॒का ए॒व ता स्ता ए॒वाहु॑तीष्ट॒का आ॑हुतीष्ट॒का ए॒व ताः । \newline
53. आ॒हु॒ती॒ष्ट॒का इत्या॑हुति - इ॒ष्ट॒काः । \newline
54. ए॒व ता स्ता ए॒वैव ता उपोप॒ ता ए॒वैव ता उप॑ । \newline
55. ता उपोप॒ ता स्ता उप॑ धत्ते धत्त॒ उप॒ ता स्ता उप॑ धत्ते । \newline
56. उप॑ धत्ते धत्त॒ उपोप॑ धत्ते॒ यज॑मानो॒ यज॑मानो धत्त॒ उपोप॑ धत्ते॒ यज॑मानः । \newline
57. ध॒त्ते॒ यज॑मानो॒ यज॑मानो धत्ते धत्ते॒ यज॑मानो ऽहोरा॒त्रा ण्य॑होरा॒त्राणि॒ यज॑मानो धत्ते धत्ते॒ यज॑मानो ऽहोरा॒त्राणि॑ । \newline
\pagebreak
\markright{ TS 3.4.10.2  \hfill https://www.vedavms.in \hfill}

\section{ TS 3.4.10.2 }

\textbf{TS 3.4.10.2 } \newline
\textbf{Samhita Paata} \newline

यज॑मानोऽहोरा॒त्राणि॒ वा ए॒तस्येष्ट॑का॒ य आहि॑ताग्नि॒र्यथ् सा॒यं प्रा॑तर्जु॒होत्य॑होरा॒त्राण्ये॒वा ऽऽ*प्त्वेष्ट॑काः कृ॒त्वोप॑ धत्ते॒ दश॑ समा॒नत्र॑ जुहोति॒ दशा᳚क्षरा वि॒राड् वि॒राज॑मे॒वाऽऽ*प्त्वेष्ट॑कां कृ॒त्वोप॑ ध॒त्तेऽथो॑ वि॒राज्ये॒व य॒ज्ञ्मा᳚प्नोति॒ चित्य॑श्चित्योऽस्य भवति॒ तस्मा॒द्यत्र॒ दशो॑षि॒त्वा प्र॒याति॒ तद्-य॑ज्ञ्वा॒स्त्ववा᳚स्त्वे॒व तद्यत् ततो᳚ऽर्वा॒चीनꣳ॑ - [  ] \newline

\textbf{Pada Paata} \newline

यज॑मानः । अ॒हो॒रा॒त्राणीत्य॑हः-रा॒त्राणि॑ । वै । ए॒तस्य॑ । इष्ट॑काः । यः । आहि॑ताग्नि॒रित्याहि॑त - अ॒ग्निः॒ । यत् । सा॒यंप्रा॑त॒रिति॑ सा॒यं-प्रा॒तः॒ । जु॒होति॑ । अ॒हो॒रा॒त्राणीत्य॑हः - रा॒त्राणि॑ । ए॒व । आ॒प्त्वा । इष्ट॑काः । कृ॒त्वा । उपेति॑ । ध॒त्ते॒ । दश॑ । स॒मा॒नत्र॑ । जु॒हो॒ति॒ । दशा᳚क्ष॒रेति॒ दश॑ - अ॒क्ष॒रा॒ । वि॒राडिति॑ वि - राट् । वि॒राज॒मिति॑ वि - राज᳚म् । ए॒व । आ॒प्त्वा । इष्ट॑काम् । कृ॒त्वा । उपेति॑ । ध॒त्ते॒ । अथो॒ इति॑ । वि॒राजीति॑ वि - राजि॑ । ए॒व । य॒ज्ञ्म् । आ॒प्नो॒ति॒ । चित्य॑श्चित्य॒ इति॒ चित्यः॑ - चि॒त्यः॒ । अ॒स्य॒ । भ॒व॒ति॒ । तस्मा᳚त् । यत्र॑ । दश॑ । उ॒षि॒त्वा । प्र॒यातीति॑ प्र - याति॑ । तत् । य॒ज्ञ्॒वा॒स्त्विति॑ यज्ञ्-वा॒स्तु । अवा᳚स्तु । ए॒व । तत् । यत् । ततः॑ । अ॒र्वा॒चीन᳚म् ।  \newline


\textbf{Krama Paata} \newline

यज॑मानो ऽहोरा॒त्राणि॑ । अ॒हो॒रा॒त्राणि॒ वै । अ॒हो॒रा॒त्राणीत्य॑हः - रा॒त्राणि॑ । वा ए॒तस्य॑ । ए॒तस्येष्ट॑काः । इष्ट॑का॒ यः । य आहि॑ताग्निः । आहि॑ताग्नि॒र् यत् । आहि॑ताग्नि॒रित्याहि॑त - अ॒ग्निः॒ । यथ् सा॒यम्प्रा॑तः । सा॒यम्प्रा॑तर् जु॒होति॑ । सा॒यम्प्रा॑त॒रिति॑ सा॒यं - प्रा॒तः॒ । जु॒होत्य॑होरा॒त्राणि॑ । अ॒हो॒रा॒त्राण्ये॒व । अ॒हो॒रा॒त्राणीत्य॑हः - रा॒त्राणि॑ । ए॒वाप्त्वा । आ॒प्त्वेष्ट॑काः । इष्ट॑काः कृ॒त्वा । कृ॒त्वोप॑ । उप॑ धत्ते । ध॒त्ते॒ दश॑ । दश॑ समा॒नत्र॑ । स॒मा॒नत्र॑ जुहोति । जु॒हो॒ति॒ दशा᳚क्षरा । दशा᳚क्षरा वि॒राट् । दशा᳚क्ष॒रेति॒ दश॑ - अ॒क्ष॒रा॒ । वि॒राड् वि॒राज᳚म् । वि॒राडिति॑ वि - राट् । वि॒राज॑मे॒व । वि॒राज॒मिति॑ वि - राज᳚म् । ए॒वाप्त्वा । आ॒प्त्वेष्ट॑काम् । इष्ट॑काम् कृ॒त्वा । कृ॒त्वोप॑ । उप॑ धत्ते । ध॒त्ते ऽथो᳚ । अथो॑ वि॒राजि॑ । अथो॒ इत्यथो᳚ । वि॒राज्ये॒व । वि॒राजीति॑ वि - राजि॑ । ए॒व य॒ज्ञ्म् । य॒ज्ञ्मा᳚प्नोति । आ॒प्नो॒ति॒ चित्य॑श्चित्यः । चित्य॑श्चित्यो ऽस्य । चित्य॑श्चित्य॒ इति॒ चित्यः॑ - चि॒त्यः॒ । अ॒स्य॒ भ॒व॒ति॒ । भ॒व॒ति॒ तस्मा᳚त् । तस्मा॒द् यत्र॑ । यत्र॒ दश॑ । दशो॑षि॒त्वा । उ॒षि॒त्वा प्र॒याति॑ । प्र॒याति॒ तत् । प्र॒यातीति॑ प्र - याति॑ । तद् य॑ज्ञ्वा॒स्तु । य॒ज्ञ्॒वा॒स्त्ववा᳚स्तु । य॒ज्ञ्॒वा॒स्त्विति॑ यज्ञ् - वा॒स्तु । अवा᳚स्त्वे॒व । ए॒व तत् । तद् यत् । यत् ततः॑ । ततो᳚ ऽर्वा॒चीन᳚म् । अ॒र्वा॒चीनꣳ॑ रु॒द्रः \newline

\textbf{Jatai Paata} \newline

1. यज॑मानो ऽहोरा॒त्रा ण्य॑होरा॒त्राणि॒ यज॑मानो॒ यज॑मानो ऽहोरा॒त्राणि॑ । \newline
2. अ॒हो॒रा॒त्राणि॒ वै वा अ॑होरा॒त्राण्य॑ होरा॒त्राणि॒ वै । \newline
3. अ॒हो॒रा॒त्राणीत्य॑हः - रा॒त्राणि॑ । \newline
4. वा ए॒तस्यै॒तस्य॒ वै वा ए॒तस्य॑ । \newline
5. ए॒त स्येष्ट॑का॒ इष्ट॑का ए॒त स्यै॒त स्येष्ट॑काः । \newline
6. इष्ट॑का॒ यो य इष्ट॑का॒ इष्ट॑का॒ यः । \newline
7. य आहि॑ताग्नि॒ राहि॑ताग्नि॒र् यो य आहि॑ताग्निः । \newline
8. आहि॑ताग्नि॒र् यद् यदाहि॑ताग्नि॒ राहि॑ताग्नि॒र् यत् । \newline
9. आहि॑ताग्नि॒रित्याहि॑त - अ॒ग्निः॒ । \newline
10. यथ् सा॒यम्प्रा॑तः सा॒यम्प्रा॑त॒र् यद् यथ् सा॒यम्प्रा॑तः । \newline
11. सा॒यम्प्रा॑तर् जु॒होति॑ जु॒होति॑ सा॒यम्प्रा॑तः सा॒यम्प्रा॑तर् जु॒होति॑ । \newline
12. सा॒यम्प्रा॑त॒रिति॑ सा॒यं - प्रा॒तः॒ । \newline
13. जु॒हो त्य॑होरा॒त्राण्य॑ होरा॒त्राणि॑ जु॒होति॑ जु॒हो त्य॑होरा॒त्राणि॑ । \newline
14. अ॒हो॒रा॒त्रा ण्ये॒वैवा हो॑रा॒त्रा ण्य॑होरा॒त्रा ण्ये॒व । \newline
15. अ॒हो॒रा॒त्राणीत्य॑हः - रा॒त्राणि॑ । \newline
16. ए॒वाप्त्वा ऽऽप्त्वै वैवाप्त्वा । \newline
17. आ॒प्त्वेष्ट॑का॒ इष्ट॑का आ॒प्त्वा ऽऽप्त्वेष्ट॑काः । \newline
18. इष्ट॑काः कृ॒त्वा कृ॒त्वेष्ट॑का॒ इष्ट॑काः कृ॒त्वा । \newline
19. कृ॒त्वोपोप॑ कृ॒त्वा कृ॒त्वोप॑ । \newline
20. उप॑ धत्ते धत्त॒ उपोप॑ धत्ते । \newline
21. ध॒त्ते॒ दश॒ दश॑ धत्ते धत्ते॒ दश॑ । \newline
22. दश॑ समा॒नत्र॑ समा॒नत्र॒ दश॒ दश॑ समा॒नत्र॑ । \newline
23. स॒मा॒नत्र॑ जुहोति जुहोति समा॒नत्र॑ समा॒नत्र॑ जुहोति । \newline
24. जु॒हो॒ति॒ दशा᳚क्षरा॒ दशा᳚क्षरा जुहोति जुहोति॒ दशा᳚क्षरा । \newline
25. दशा᳚क्षरा वि॒राड् वि॒राड् दशा᳚क्षरा॒ दशा᳚क्षरा वि॒राट् । \newline
26. दशा᳚क्ष॒रेति॒ दश॑ - अ॒क्ष॒रा॒ । \newline
27. वि॒राड् वि॒राजं॑ ॅवि॒राजं॑ ॅवि॒राड् वि॒राड् वि॒राज᳚म् । \newline
28. वि॒राडिति॑ वि - राट् । \newline
29. वि॒राज॑ मे॒वैव वि॒राजं॑ ॅवि॒राज॑ मे॒व । \newline
30. वि॒राज॒मिति॑ वि - राज᳚म् । \newline
31. ए॒वाप्त्वा ऽऽप्त्वै वै वाप्त्वा । \newline
32. आ॒प्त्वेष्ट॑का॒ मिष्ट॑का मा॒प्त्वा ऽऽप्त्वेष्ट॑काम् । \newline
33. इष्ट॑काम् कृ॒त्वा कृ॒त्वेष्ट॑का॒ मिष्ट॑काम् कृ॒त्वा । \newline
34. कृ॒त्वो पोप॑ कृ॒त्वा कृ॒त्वोप॑ । \newline
35. उप॑ धत्ते धत्त॒ उपोप॑ धत्ते । \newline
36. ध॒त्ते ऽथो॒ अथो॑ धत्ते ध॒त्ते ऽथो᳚ । \newline
37. अथो॑ वि॒राजि॑ वि॒राज्यथो॒ अथो॑ वि॒राजि॑ । \newline
38. अथो॒ इत्यथो᳚ । \newline
39. वि॒राज्ये॒वैव वि॒राजि॑ वि॒राज्ये॒व । \newline
40. वि॒राजीति॑ वि - राजि॑ । \newline
41. ए॒व य॒ज्ञ्ं ॅय॒ज्ञ् मे॒वैव य॒ज्ञ्म् । \newline
42. य॒ज्ञ् मा᳚प्नो त्याप्नोति य॒ज्ञ्ं ॅय॒ज्ञ् मा᳚प्नोति । \newline
43. आ॒प्नो॒ति॒ चित्य॑श्चित्य॒ श्चित्य॑श्चित्य आप्नो त्याप्नोति॒ चित्य॑श्चित्यः । \newline
44. चित्य॑श्चित्यो ऽस्यास्य॒ चित्य॑श्चित्य॒ श्चित्य॑श्चित्यो ऽस्य । \newline
45. चित्य॑श्चित्य॒ इति॒ चित्यः॑ - चि॒त्यः॒ । \newline
46. अ॒स्य॒ भ॒व॒ति॒ भ॒व॒त्य॒ स्या॒स्य॒ भ॒व॒ति॒ । \newline
47. भ॒व॒ति॒ तस्मा॒त् तस्मा᳚द् भवति भवति॒ तस्मा᳚त् । \newline
48. तस्मा॒द् यत्र॒ यत्र॒ तस्मा॒त् तस्मा॒द् यत्र॑ । \newline
49. यत्र॒ दश॒ दश॒ यत्र॒ यत्र॒ दश॑ । \newline
50. दशो॑ षि॒त्वो षि॒त्वा दश॒ दशो॑ षि॒त्वा । \newline
51. उ॒षि॒त्वा प्र॒याति॑ प्र॒या त्यु॑षि॒त्वो षि॒त्वा प्र॒याति॑ । \newline
52. प्र॒याति॒ तत् तत् प्र॒याति॑ प्र॒याति॒ तत् । \newline
53. प्र॒यातीति॑ प्र - याति॑ । \newline
54. तद् य॑ज्ञ्वा॒स्तु य॑ज्ञ्वा॒स्तु तत् तद् य॑ज्ञ्वा॒स्तु । \newline
55. य॒ज्ञ्॒वा॒ स्त्ववा॒ स्त्ववा᳚स्तु यज्ञ्वा॒स्तु य॑ज्ञ्वा॒ स्त्ववा᳚स्तु । \newline
56. य॒ज्ञ्॒वा॒स्त्विति॑ यज्ञ् - वा॒स्तु । \newline
57. अवा᳚स्त्वे॒वैवा वा॒स्त्व वा᳚स्त्वे॒व । \newline
58. ए॒व तत् तदे॒वैव तत् । \newline
59. तद् यद् यत् तत् तद् यत् । \newline
60. यत् तत॒ स्ततो॒ यद् यत् ततः॑ । \newline
61. ततो᳚ ऽर्वा॒चीन॑ मर्वा॒चीन॒म् तत॒ स्ततो᳚ ऽर्वा॒चीन᳚म् । \newline
62. अ॒र्वा॒चीनꣳ॑ रु॒द्रो रु॒द्रो᳚ ऽर्वा॒चीन॑ मर्वा॒चीनꣳ॑ रु॒द्रः । \newline

\textbf{Ghana Paata } \newline

1. यज॑मानो ऽहोरा॒त्रा ण्य॑होरा॒त्राणि॒ यज॑मानो॒ यज॑मानो ऽहोरा॒त्राणि॒ वै वा अ॑होरा॒त्राणि॒ यज॑मानो॒ यज॑मानो ऽहोरा॒त्राणि॒ वै । \newline
2. अ॒हो॒रा॒त्राणि॒ वै वा अ॑होरा॒त्रा ण्य॑होरा॒त्राणि॒ वा ए॒तस्यै॒तस्य॒ वा अ॑होरा॒त्रा ण्य॑होरा॒त्राणि॒ वा ए॒तस्य॑ । \newline
3. अ॒हो॒रा॒त्राणीत्य॑हः - रा॒त्राणि॑ । \newline
4. वा ए॒त स्यै॒तस्य॒ वै वा ए॒त स्येष्ट॑का॒ इष्ट॑का ए॒तस्य॒ वै वा ए॒त स्येष्ट॑काः । \newline
5. ए॒तस्येष्ट॑का॒ इष्ट॑का ए॒त स्यै॒त स्येष्ट॑का॒ यो य इष्ट॑का ए॒त स्यै॒त स्येष्ट॑का॒ यः । \newline
6. इष्ट॑का॒ यो य इष्ट॑का॒ इष्ट॑का॒ य आहि॑ताग्नि॒ राहि॑ताग्नि॒र् य इष्ट॑का॒ इष्ट॑का॒ य आहि॑ताग्निः । \newline
7. य आहि॑ताग्नि॒ राहि॑ताग्नि॒र् यो य आहि॑ताग्नि॒र् यद् य दाहि॑ताग्नि॒र् यो य आहि॑ताग्नि॒र् यत् । \newline
8. आहि॑ताग्नि॒र् यद् यदाहि॑ताग्नि॒ राहि॑ताग्नि॒र् यथ् सा॒यम्प्रा॑तः सा॒यम्प्रा॑त॒र् य दाहि॑ताग्नि॒ राहि॑ताग्नि॒र् यथ् सा॒यम्प्रा॑तः । \newline
9. आहि॑ताग्नि॒रित्याहि॑त - अ॒ग्निः॒ । \newline
10. यथ् सा॒यम्प्रा॑तः सा॒यम्प्रा॑त॒र् यद् यथ् सा॒यम्प्रा॑तर् जु॒होति॑ जु॒होति॑ सा॒यम्प्रा॑त॒र् यद् यथ् सा॒यम्प्रा॑तर् जु॒होति॑ । \newline
11. सा॒यम्प्रा॑तर् जु॒होति॑ जु॒होति॑ सा॒यम्प्रा॑तः सा॒यम्प्रा॑तर् जु॒हो त्य॑होरा॒त्रा ण्य॑होरा॒त्राणि॑ जु॒होति॑ सा॒यम्प्रा॑तः 
सा॒यम्प्रा॑तर् जु॒हो त्य॑होरा॒त्राणि॑ । \newline
12. सा॒यम्प्रा॑त॒रिति॑ सा॒यम् - प्रा॒तः॒ । \newline
13. जु॒हो त्य॑होरा॒त्रा ण्य॑होरा॒त्राणि॑ जु॒होति॑ जु॒हो त्य॑होरा॒त्रा ण्ये॒वैवा हो॑रा॒त्राणि॑ जु॒होति॑ जु॒हो त्य॑होरा॒त्राण्ये॒व । \newline
14. अ॒हो॒रा॒त्रा ण्ये॒वैवा हो॑रा॒त्रा ण्य॑होरा॒त्रा ण्ये॒वाप्त्वा ऽऽप्त्वै वा हो॑रा॒त्रा ण्य॑होरा॒त्रा ण्ये॒वाप्त्वा । \newline
15. अ॒हो॒रा॒त्राणीत्य॑हः - रा॒त्राणि॑ । \newline
16. ए॒वाप्त्वा ऽऽप्त्वैवैवा प्त्वेष्ट॑का॒ इष्ट॑का आ॒प्त्वैवैवा प्त्वेष्ट॑काः । \newline
17. आ॒प्त्वेष्ट॑का॒ इष्ट॑का आ॒प्त्वा ऽऽप्त्वेष्ट॑काः कृ॒त्वा कृ॒त्वेष्ट॑का आ॒प्त्वा ऽऽप्त्वेष्ट॑काः कृ॒त्वा । \newline
18. इष्ट॑काः कृ॒त्वा कृ॒त्वेष्ट॑का॒ इष्ट॑काः कृ॒त्वोपोप॑ कृ॒त्वेष्ट॑का॒ इष्ट॑काः कृ॒त्वोप॑ । \newline
19. कृ॒त्वोपोप॑ कृ॒त्वा कृ॒त्वोप॑ धत्ते धत्त॒ उप॑ कृ॒त्वा कृ॒त्वोप॑ धत्ते । \newline
20. उप॑ धत्ते धत्त॒ उपोप॑ धत्ते॒ दश॒ दश॑ धत्त॒ उपोप॑ धत्ते॒ दश॑ । \newline
21. ध॒त्ते॒ दश॒ दश॑ धत्ते धत्ते॒ दश॑ समा॒नत्र॑ समा॒नत्र॒ दश॑ धत्ते धत्ते॒ दश॑ समा॒नत्र॑ । \newline
22. दश॑ समा॒नत्र॑ समा॒नत्र॒ दश॒ दश॑ समा॒नत्र॑ जुहोति जुहोति समा॒नत्र॒ दश॒ दश॑ समा॒नत्र॑ जुहोति । \newline
23. स॒मा॒नत्र॑ जुहोति जुहोति समा॒नत्र॑ समा॒नत्र॑ जुहोति॒ दशा᳚क्षरा॒ दशा᳚क्षरा जुहोति समा॒नत्र॑ समा॒नत्र॑ जुहोति॒ दशा᳚क्षरा । \newline
24. जु॒हो॒ति॒ दशा᳚क्षरा॒ दशा᳚क्षरा जुहोति जुहोति॒ दशा᳚क्षरा वि॒राड् वि॒राड् दशा᳚क्षरा जुहोति जुहोति॒ दशा᳚क्षरा वि॒राट् । \newline
25. दशा᳚क्षरा वि॒राड् वि॒राड् दशा᳚क्षरा॒ दशा᳚क्षरा वि॒राड् वि॒राज॑म् ॅवि॒राज॑म् ॅवि॒राड् दशा᳚क्षरा॒ 
दशा᳚क्षरा वि॒राड् वि॒राज᳚म् । \newline
26. दशा᳚क्ष॒रेति॒ दश॑ - अ॒क्ष॒रा॒ । \newline
27. वि॒राड् वि॒राज॑म् ॅवि॒राज॑म् ॅवि॒राड् वि॒राड् वि॒राज॑ मे॒वैव वि॒राज॑म् ॅवि॒राड् वि॒राड् वि॒राज॑ मे॒व । \newline
28. वि॒राडिति॑ वि - राट् । \newline
29. वि॒राज॑ मे॒वैव वि॒राज॑म् ॅवि॒राज॑ मे॒वाप्त्वा ऽऽप्त्वैव वि॒राज॑म् ॅवि॒राज॑ मे॒वाप्त्वा । \newline
30. वि॒राज॒मिति॑ वि - राज᳚म् । \newline
31. ए॒वाप्त्वा ऽऽप्त्वैवैवा प्त्वेष्ट॑का॒ मिष्ट॑का मा॒प्त्वैवैवा प्त्वेष्ट॑काम् । \newline
32. आ॒प्त्वेष्ट॑का॒ मिष्ट॑का मा॒प्त्वा ऽऽप्त्वेष्ट॑काम् कृ॒त्वा कृ॒त्वेष्ट॑का मा॒प्त्वा ऽऽप्त्वेष्ट॑काम् कृ॒त्वा । \newline
33. इष्ट॑काम् कृ॒त्वा कृ॒त्वेष्ट॑का॒ मिष्ट॑काम् कृ॒त्वोपोप॑ कृ॒त्वेष्ट॑का॒ मिष्ट॑काम् कृ॒त्वोप॑ । \newline
34. कृ॒त्वोपोप॑ कृ॒त्वा कृ॒त्वोप॑ धत्ते धत्त॒ उप॑ कृ॒त्वा कृ॒त्वोप॑ धत्ते । \newline
35. उप॑ धत्ते धत्त॒ उपोप॑ ध॒त्ते ऽथो॒ अथो॑ धत्त॒ उपोप॑ ध॒त्ते ऽथो᳚ । \newline
36. ध॒त्ते ऽथो॒ अथो॑ धत्ते ध॒त्ते ऽथो॑ वि॒राजि॑ वि॒राज्यथो॑ धत्ते ध॒त्ते ऽथो॑ वि॒राजि॑ । \newline
37. अथो॑ वि॒राजि॑ वि॒राज्यथो॒ अथो॑ वि॒राज्ये॒वैव वि॒राज्यथो॒ अथो॑ वि॒राज्ये॒व । \newline
38. अथो॒ इत्यथो᳚ । \newline
39. वि॒राज्ये॒वैव वि॒राजि॑ वि॒राज्ये॒व य॒ज्ञ्म् ॅय॒ज्ञ् मे॒व वि॒राजि॑ वि॒राज्ये॒व य॒ज्ञ्म् । \newline
40. वि॒राजीति॑ वि - राजि॑ । \newline
41. ए॒व य॒ज्ञ्म् ॅय॒ज्ञ् मे॒वैव य॒ज्ञ् मा᳚प्नो त्याप्नोति य॒ज्ञ् मे॒वैव य॒ज्ञ् मा᳚प्नोति । \newline
42. य॒ज्ञ् मा᳚प्नो त्याप्नोति य॒ज्ञ्म् ॅय॒ज्ञ् मा᳚प्नोति॒ चित्य॑श्चित्य॒ श्चित्य॑श्चित्य आप्नोति य॒ज्ञ्म् ॅय॒ज्ञ् मा᳚प्नोति॒ चित्य॑श्चित्यः । \newline
43. आ॒प्नो॒ति॒ चित्य॑श्चित्य॒ श्चित्य॑श्चित्य आप्नो त्याप्नोति॒ चित्य॑श्चित्यो ऽस्यास्य॒ चित्य॑श्चित्य आप्नो त्याप्नोति॒ चित्य॑श्चित्यो ऽस्य । \newline
44. चित्य॑श्चित्यो ऽस्यास्य॒ चित्य॑श्चित्य॒ श्चित्य॑श्चित्यो ऽस्य भवति भवत्यस्य॒ चित्य॑श्चित्य॒ श्चित्य॑श्चित्यो ऽस्य भवति । \newline
45. चित्य॑श्चित्य॒ इति॒ चित्यः॑ - चि॒त्यः॒ । \newline
46. अ॒स्य॒ भ॒व॒ति॒ भ॒व॒ त्य॒स्या॒स्य॒ भ॒व॒ति॒ तस्मा॒त् तस्मा᳚द् भव त्यस्यास्य भवति॒ तस्मा᳚त् । \newline
47. भ॒व॒ति॒ तस्मा॒त् तस्मा᳚द् भवति भवति॒ तस्मा॒द् यत्र॒ यत्र॒ तस्मा᳚द् भवति भवति॒ तस्मा॒द् यत्र॑ । \newline
48. तस्मा॒द् यत्र॒ यत्र॒ तस्मा॒त् तस्मा॒द् यत्र॒ दश॒ दश॒ यत्र॒ तस्मा॒त् तस्मा॒द् यत्र॒ दश॑ । \newline
49. यत्र॒ दश॒ दश॒ यत्र॒ यत्र॒ दशो॑षि॒ त्वोषि॒त्वा दश॒ यत्र॒ यत्र॒ दशो॑षि॒त्वा । \newline
50. दशो॑षि॒त्वो षि॒त्वा दश॒ दशो॑ षि॒त्वा प्र॒याति॑ प्र॒या त्यु॑षि॒त्वा दश॒ दशो॑ षि॒त्वा प्र॒याति॑ । \newline
51. उ॒षि॒त्वा प्र॒याति॑ प्र॒या त्यु॑षि॒त्वो षि॒त्वा प्र॒याति॒ तत् तत् प्र॒या त्यु॑षि॒त्वो षि॒त्वा प्र॒याति॒ तत् । \newline
52. प्र॒याति॒ तत् तत् प्र॒याति॑ प्र॒याति॒ तद् य॑ज्ञ्वा॒स्तु य॑ज्ञ्वा॒स्तु तत् प्र॒याति॑ प्र॒याति॒ तद् य॑ज्ञ्वा॒स्तु । \newline
53. प्र॒यातीति॑ प्र - याति॑ । \newline
54. तद् य॑ज्ञ्वा॒स्तु य॑ज्ञ्वा॒स्तु तत् तद् य॑ज्ञ्वा॒ स्त्ववा॒ स्त्ववा᳚स्तु यज्ञ्वा॒स्तु तत् तद् य॑ज्ञ्वा॒ स्त्ववा᳚स्तु । \newline
55. य॒ज्ञ्॒वा॒ स्त्ववा॒ स्त्ववा᳚स्तु यज्ञ्वा॒स्तु य॑ज्ञ्वा॒ स्त्ववा᳚ स्त्वे॒वैवा वा᳚स्तु यज्ञ्वा॒स्तु य॑ज्ञ्वा॒ स्त्ववा᳚ स्त्वे॒व । \newline
56. य॒ज्ञ्॒वा॒स्त्विति॑ यज्ञ् - वा॒स्तु । \newline
57. अवा᳚ स्त्वे॒वैवावा॒ स्त्ववा᳚ स्त्वे॒व तत् त दे॒वा वा॒ स्त्ववा᳚ स्त्वे॒व तत् । \newline
58. ए॒व तत् तदे॒ वैव तद् यद् यत् तदे॒ वैव तद् यत् । \newline
59. तद् यद् यत् तत् तद् यत् तत॒ स्ततो॒ यत् तत् तद् यत् ततः॑ । \newline
60. यत् तत॒ स्ततो॒ यद् यत् ततो᳚ ऽर्वा॒चीन॑ मर्वा॒चीन॒म् ततो॒ यद् यत् ततो᳚ ऽर्वा॒चीन᳚म् । \newline
61. ततो᳚ ऽर्वा॒चीन॑ मर्वा॒चीन॒म् तत॒ स्ततो᳚ ऽर्वा॒चीनꣳ॑ रु॒द्रो रु॒द्रो᳚ ऽर्वा॒चीन॒म् तत॒ स्ततो᳚ ऽर्वा॒चीनꣳ॑ रु॒द्रः । \newline
62. अ॒र्वा॒चीनꣳ॑ रु॒द्रो रु॒द्रो᳚ ऽर्वा॒चीन॑ मर्वा॒चीनꣳ॑ रु॒द्रः खलु॒ खलु॑ रु॒द्रो᳚ ऽर्वा॒चीन॑ मर्वा॒चीनꣳ॑ रु॒द्रः खलु॑ । \newline
\pagebreak
\markright{ TS 3.4.10.3  \hfill https://www.vedavms.in \hfill}

\section{ TS 3.4.10.3 }

\textbf{TS 3.4.10.3 } \newline
\textbf{Samhita Paata} \newline

रु॒द्रः खलु॒ वै वा᳚स्तोष्प॒तिर्यदहु॑त्वा वास्तोष्प॒तीयं॑ प्रया॒याद् रु॒द्र ए॑नं भू॒त्वाऽग्निर॑नू॒त्थाय॑ हन्याद्वास्तोष्प॒तीयं॑ जुहोति भाग॒धेये॑नै॒वैनꣳ॑ शमयति॒ नाऽऽ*र्ति॒मार्च्छ॑ति॒ यज॑मानो॒ यद्यु॒क्ते जु॑हु॒याद्यथा॒ प्रया॑ते॒ वास्ता॒वाहु॑तिं जु॒होति॑ ता॒दृगे॒व तद्यदयु॑क्ते जुहु॒याद्यथा॒ क्षेम॒ आहु॑तिं जु॒होति॑ ता॒दृगे॒व तदहु॑तमस्य वास्तोष्प॒तीयꣳ॑ स्या॒ - [  ] \newline

\textbf{Pada Paata} \newline

रु॒द्रः । खलु॑ । वै । वा॒स्तो॒ष्प॒तिरिति॑ वास्तोः-प॒तिः । यत् । अहु॑त्वा । वा॒स्तो॒ष्प॒तीय॒मिति॑ वास्तोः - प॒तीय᳚म् । प्र॒या॒यादिति॑ प्र - या॒यात् । रु॒द्रः । ए॒न॒म् । भू॒त्वा । अ॒ग्निः । अ॒नू॒त्थायेत्य॑नु - उ॒त्थाय॑ । ह॒न्या॒त् । वा॒स्तो॒ष्प॒तीय॒मिति॑ वास्तोः - प॒तीय᳚म् । जु॒हो॒ति॒ । भा॒ग॒धेये॒नेति॑ भाग - धेये॑न । ए॒व । ए॒न॒म् । श॒म॒य॒ति॒ । न । आर्ति᳚म् । एति॑ । ऋ॒च्छ॒ति॒ । यज॑मानः । यत् । यु॒क्ते । जु॒हु॒यात् । यथा᳚ । प्रया॑त॒ इति॒ प्र - या॒ते॒ । वास्तौ᳚ । आहु॑ति॒मित्या-हु॒ति॒म् । जु॒होति॑ । ता॒दृक् । ए॒व । तत् । यत् । अयु॑क्ते । जु॒हु॒यात् । यथा᳚ । क्षेमे᳚ । आहु॑ति॒मित्या - हु॒ति॒म् । जु॒होति॑ । ता॒दृक् । ए॒व । तत् । अहु॑तम् । अ॒स्य॒ । वा॒स्तो॒ष्प॒तीय॒मिति॑ वास्तोः - प॒तीय᳚म् । स्यात् ।  \newline


\textbf{Krama Paata} \newline

रु॒द्रः खलु॑ । खलु॒ वै । वै वा᳚स्तोष्प॒तिः । वा॒स्तो॒ष्प॒तिर् यत् । वा॒स्तो॒ष्प॒तिरिति॑ वास्तोः - प॒तिः । यदहु॑त्वा । अहु॑त्वा वास्तोष्प॒तीय᳚म् । वा॒स्तो॒ष्प॒तीय॑म् प्रया॒यात् । वा॒स्तो॒ष्प॒तीय॒मिति॑ वास्तोः - प॒तीय᳚म् । प्र॒या॒याद्,रु॒द्रः । प्र॒या॒यादिति॑ प्र - या॒यात् । रु॒द्र ए॑नम् । ए॒न॒म् भू॒त्वा । भू॒त्वा ऽग्निः । अ॒ग्निर॑नू॒त्थाय॑ । अ॒नू॒त्थाय॑ हन्यात् । अ॒नू॒त्थायेत्य॑नु - उ॒त्थाय॑ । ह॒न्या॒द् वा॒स्तो॒ष्प॒तीय᳚म् । वा॒स्तो॒ष्प॒तीय॑म् जुहोति । वा॒स्तो॒ष्प॒तीय॒मिति॑ वास्तोः - प॒तीय᳚म् । जु॒हो॒ति॒ भा॒ग॒धेये॑न । भा॒ग॒धेये॑नै॒व । भा॒ग॒धेये॒नेति॑ भाग - धेये॑न । ए॒वैन᳚म् । ए॒नꣳ॒॒ श॒म॒य॒ति॒ । श॒म॒य॒ति॒ न । नार्ति᳚म् । आर्ति॒मा । आर्च्छ॑ति । ऋ॒च्छ॒ति॒ यज॑मानः । यज॑मानो॒ यत् । यद् यु॒क्ते । यु॒क्ते जु॑हु॒यात् । जु॒हु॒याद् यथा᳚ । यथा॒ प्रया॑ते । प्रया॑ते॒ वास्तौ᳚ । प्रया॑त॒ इति॒ प्र - या॒ते॒ । वास्ता॒वाहु॑तिम् । आहु॑तिम् जु॒होति॑ । आहु॑ति॒मित्या - हु॒ति॒म् । जु॒होति॑ ता॒दृक् । ता॒दृगे॒व । ए॒व तत् । तद् यत् । यदयु॑क्ते । अयु॑क्ते जुहु॒यात् । जु॒हु॒याद् यथा᳚ । यथा॒ क्षेमे᳚ । क्षेम॒ आहु॑तिम् । आहु॑तिम् जु॒होति॑ । आहु॑ति॒मित्या - हु॒ति॒म् । जु॒होति॑ ता॒दृक् । ता॒दृगे॒व । ए॒व तत् । तदहु॑तम् । अहु॑तमस्य । अ॒स्य॒ वा॒स्तो॒ष्प॒तीय᳚म् । वा॒स्तो॒ष्प॒तीयꣳ॑ स्यात् । वा॒स्तो॒ष्प॒तीय॒मिति॑ वास्तोः - प॒तीय᳚म् । स्या॒द् दक्षि॑णः \newline

\textbf{Jatai Paata} \newline

1. रु॒द्रः खलु॒ खलु॑ रु॒द्रो रु॒द्रः खलु॑ । \newline
2. खलु॒ वै वै खलु॒ खलु॒ वै । \newline
3. वै वा᳚स्तोष्प॒तिर् वा᳚स्तोष्प॒तिर् वै वै वा᳚स्तोष्प॒तिः । \newline
4. वा॒स्तो॒ष्प॒तिर् यद् यद् वा᳚स्तोष्प॒तिर् वा᳚स्तोष्प॒तिर् यत् । \newline
5. वा॒स्तो॒ष्प॒तिरिति॑ वास्तोः - प॒तिः । \newline
6. यदहु॒त्वा ऽहु॑त्वा॒ यद् यदहु॑त्वा । \newline
7. अहु॑त्वा वास्तोष्प॒तीयं॑ ॅवास्तोष्प॒तीय॒ महु॒त्वा ऽहु॑त्वा वास्तोष्प॒तीय᳚म् । \newline
8. वा॒स्तो॒ष्प॒तीय॑म् प्रया॒यात् प्र॑या॒याद् वा᳚स्तोष्प॒तीयं॑ ॅवास्तोष्प॒तीय॑म् प्रया॒यात् । \newline
9. वा॒स्तो॒ष्प॒तीय॒मिति॑ वास्तोः - प॒तीय᳚म् । \newline
10. प्र॒या॒याद् रु॒द्रो रु॒द्रः प्र॑या॒यात् प्र॑या॒याद् रु॒द्रः । \newline
11. प्र॒या॒यादिति॑ प्र - या॒यात् । \newline
12. रु॒द्र ए॑न मेनꣳ रु॒द्रो रु॒द्र ए॑नम् । \newline
13. ए॒न॒म् भू॒त्वा भू॒त्वैन॑ मेनम् भू॒त्वा । \newline
14. भू॒त्वा ऽग्नि र॒ग्निर् भू॒त्वा भू॒त्वा ऽग्निः । \newline
15. आ॒ग्नि र॑नू॒त्थाया॑ नू॒त्थाया॒ ग्नि र॒ग्नि र॑नू॒त्थाय॑ । \newline
16. अ॒नू॒त्थाय॑ हन्या द्धन्या दनू॒त्थाया॑ नू॒त्थाय॑ हन्यात् । \newline
17. अ॒नू॒त्थायेत्य॑नु - उ॒त्थाय॑ । \newline
18. ह॒न्या॒द् वा॒स्तो॒ष्प॒तीयं॑ ॅवास्तोष्प॒तीयꣳ॑ हन्याद्धन्याद् वास्तोष्प॒तीय᳚म् । \newline
19. वा॒स्तो॒ष्प॒तीय॑म् जुहोति जुहोति वास्तोष्प॒तीयं॑ ॅवास्तोष्प॒तीय॑म् जुहोति । \newline
20. वा॒स्तो॒ष्प॒तीय॒मिति॑ वास्तोः - प॒तीय᳚म् । \newline
21. जु॒हो॒ति॒ भा॒ग॒धेये॑न भाग॒धेये॑न जुहोति जुहोति भाग॒धेये॑न । \newline
22. भा॒ग॒धेये॑ नै॒वैव भा॑ग॒धेये॑न भाग॒धेये॑नै॒व । \newline
23. भा॒ग॒धेये॒नेति॑ भाग - धेये॑न । \newline
24. ए॒वैन॑ मेन मे॒वै वैन᳚म् । \newline
25. ए॒नꣳ॒॒ श॒म॒य॒ति॒ श॒म॒य॒ त्ये॒न॒ मे॒नꣳ॒॒ श॒म॒य॒ति॒ । \newline
26. श॒म॒य॒ति॒ न न श॑मयति शमयति॒ न । \newline
27. नार्ति॒ मार्ति॒म् न नार्ति᳚म् । \newline
28. आर्ति॒ मा ऽऽर्ति॒ मार्ति॒ मा । \newline
29. आर्च्छ॑ त्यृच्छ॒ त्यार्च्छ॑ति । \newline
30. ऋ॒च्छ॒ति॒ यज॑मानो॒ यज॑मान ऋच्छ त्यृच्छति॒ यज॑मानः । \newline
31. यज॑मानो॒ यद् यद् यज॑मानो॒ यज॑मानो॒ यत् । \newline
32. यद् यु॒क्ते यु॒क्ते यद् यद् यु॒क्ते । \newline
33. यु॒क्ते जु॑हु॒याज् जु॑हु॒याद् यु॒क्ते यु॒क्ते जु॑हु॒यात् । \newline
34. जु॒हु॒याद् यथा॒ यथा॑ जुहु॒याज् जु॑हु॒याद् यथा᳚ । \newline
35. यथा॒ प्रया॑ते॒ प्रया॑ते॒ यथा॒ यथा॒ प्रया॑ते । \newline
36. प्रया॑ते॒ वास्तौ॒ वास्तौ॒ प्रया॑ते॒ प्रया॑ते॒ वास्तौ᳚ । \newline
37. प्रया॑त॒ इति॒ प्र - या॒ते॒ । \newline
38. वास्ता॒ वाहु॑ति॒ माहु॑तिं॒ ॅवास्तौ॒ वास्ता॒ वाहु॑तिम् । \newline
39. आहु॑तिम् जु॒होति॑ जु॒हो त्याहु॑ति॒ माहु॑तिम् जु॒होति॑ । \newline
40. आहु॑ति॒मित्या - हु॒ति॒म् । \newline
41. जु॒होति॑ ता॒दृक् ता॒दृग् जु॒होति॑ जु॒होति॑ ता॒दृक् । \newline
42. ता॒दृग् ए॒वैव ता॒दृक् ता॒दृगे॒व । \newline
43. ए॒व तत् तदे॒वैव तत् । \newline
44. तद् यद् यत् तत् तद् यत् । \newline
45. यदयु॒क्ते ऽयु॑क्ते॒ यद् यदयु॑क्ते । \newline
46. अयु॑क्ते जुहु॒याज् जु॑हु॒या दयु॒क्ते ऽयु॑क्ते जुहु॒यात् । \newline
47. जु॒हु॒याद् यथा॒ यथा॑ जुहु॒याज् जु॑हु॒याद् यथा᳚ । \newline
48. यथा॒ क्षेमे॒ क्षेमे॒ यथा॒ यथा॒ क्षेमे᳚ । \newline
49. क्षेम॒ आहु॑ति॒ माहु॑ति॒म् क्षेमे॒ क्षेम॒ आहु॑तिम् । \newline
50. आहु॑तिम् जु॒होति॑ जु॒हो त्याहु॑ति॒ माहु॑तिम् जु॒होति॑ । \newline
51. आहु॑ति॒मित्या - हु॒ति॒म् । \newline
52. जु॒होति॑ ता॒दृक् ता॒दृग् जु॒होति॑ जु॒होति॑ ता॒दृक् । \newline
53. ता॒दृगे॒वैव ता॒दृक् ता॒दृगे॒व । \newline
54. ए॒व तत् तदे॒वैव तत् । \newline
55. तदहु॑त॒ महु॑त॒म् तत् तदहु॑तम् । \newline
56. अहु॑त मस्या॒ स्याहु॑त॒ महु॑त मस्य । \newline
57. अ॒स्य॒ वा॒स्तो॒ष्प॒तीयं॑ ॅवास्तोष्प॒तीय॑ मस्यास्य वास्तोष्प॒तीय᳚म् । \newline
58. वा॒स्तो॒ष्प॒तीयꣳ॑ स्याथ् स्याद् वास्तोष्प॒तीयं॑ ॅवास्तोष्प॒तीयꣳ॑ स्यात् । \newline
59. वा॒स्तो॒ष्प॒तीय॒मिति॑ वास्तोः - प॒तीय᳚म् । \newline
60. स्या॒द् दक्षि॑णो॒ दक्षि॑णः स्याथ् स्या॒द् दक्षि॑णः । \newline

\textbf{Ghana Paata } \newline

1. रु॒द्रः खलु॒ खलु॑ रु॒द्रो रु॒द्रः खलु॒ वै वै खलु॑ रु॒द्रो रु॒द्रः खलु॒ वै । \newline
2. खलु॒ वै वै खलु॒ खलु॒ वै वा᳚स्तोष्प॒तिर् वा᳚स्तोष्प॒तिर् वै खलु॒ खलु॒ वै वा᳚स्तोष्प॒तिः । \newline
3. वै वा᳚स्तोष्प॒तिर् वा᳚स्तोष्प॒तिर् वै वै वा᳚स्तोष्प॒तिर् यद् यद् वा᳚स्तोष्प॒तिर् वै वै वा᳚स्तोष्प॒तिर् यत् । \newline
4. वा॒स्तो॒ष्प॒तिर् यद् यद् वा᳚स्तोष्प॒तिर् वा᳚स्तोष्प॒तिर् यदहु॒त्वा ऽहु॑त्वा॒ यद् वा᳚स्तोष्प॒तिर् वा᳚स्तोष्प॒तिर् यदहु॑त्वा । \newline
5. वा॒स्तो॒ष्प॒तिरिति॑ वास्तोः - प॒तिः । \newline
6. यदहु॒त्वा ऽहु॑त्वा॒ यद् यदहु॑त्वा वास्तोष्प॒तीय॑म् ॅवास्तोष्प॒तीय॒ महु॑त्वा॒ यद् यदहु॑त्वा वास्तोष्प॒तीय᳚म् । \newline
7. अहु॑त्वा वास्तोष्प॒तीय॑म् ॅवास्तोष्प॒तीय॒ महु॒त्वा ऽहु॑त्वा वास्तोष्प॒तीय॑म् प्रया॒यात् प्र॑या॒याद् वा᳚स्तोष्प॒तीय॒ महु॒त्वा ऽहु॑त्वा वास्तोष्प॒तीय॑म् प्रया॒यात् । \newline
8. वा॒स्तो॒ष्प॒तीय॑म् प्रया॒यात् प्र॑या॒याद् वा᳚स्तोष्प॒तीय॑म् ॅवास्तोष्प॒तीय॑म् प्रया॒याद् रु॒द्रो रु॒द्रः प्र॑या॒याद् वा᳚स्तोष्प॒तीय॑म् ॅवास्तोष्प॒तीय॑म् प्रया॒याद् रु॒द्रः । \newline
9. वा॒स्तो॒ष्प॒तीय॒मिति॑ वास्तोः - प॒तीय᳚म् । \newline
10. प्र॒या॒याद् रु॒द्रो रु॒द्रः प्र॑या॒यात् प्र॑या॒याद् रु॒द्र ए॑न मेनꣳ रु॒द्रः प्र॑या॒यात् प्र॑या॒याद् रु॒द्र ए॑नम् । \newline
11. प्र॒या॒यादिति॑ प्र - या॒यात् । \newline
12. रु॒द्र ए॑न मेनꣳ रु॒द्रो रु॒द्र ए॑नम् भू॒त्वा भू॒त्वैनꣳ॑ रु॒द्रो रु॒द्र ए॑नम् भू॒त्वा । \newline
13. ए॒न॒म् भू॒त्वा भू॒त्वैन॑मेनम् भू॒त्वा ऽग्नि र॒ग्निर् भू॒त्वैन॑मेनम् भू॒त्वा ऽग्निः । \newline
14. भू॒त्वा ऽग्नि र॒ग्निर् भू॒त्वा भू॒त्वा ऽग्नि र॑नू॒त्थाया॑ नू॒त्थाया॒ग्निर् भू॒त्वा भू॒त्वा ऽग्नि र॑नू॒त्थाय॑ । \newline
15. अ॒ग्नि र॑नू॒त्थाया॑ नू॒त्थाया॒ ग्नि र॒ग्नि र॑नू॒त्थाय॑ हन्या द्धन्या दनू॒त्थाया॒ ग्नि र॒ग्निर॑ नू॒त्थाय॑ हन्यात् । \newline
16. अ॒नू॒त्थाय॑ हन्या द्धन्या दनू॒त्थाया॑ नू॒त्थाय॑ हन्याद् वास्तोष्प॒तीय॑म् ॅवास्तोष्प॒तीयꣳ॑ हन्या दनू॒त्थाया॑ नू॒त्थाय॑ हन्याद् वास्तोष्प॒तीय᳚म् । \newline
17. अ॒नू॒त्थायेत्य॑नु - उ॒त्थाय॑ । \newline
18. ह॒न्या॒द् वा॒स्तो॒ष्प॒तीय॑म् ॅवास्तोष्प॒तीयꣳ॑ हन्या द्धन्याद् वास्तोष्प॒तीय॑म् जुहोति जुहोति वास्तोष्प॒तीयꣳ॑ हन्या द्धन्याद् वास्तोष्प॒तीय॑म् जुहोति । \newline
19. वा॒स्तो॒ष्प॒तीय॑म् जुहोति जुहोति वास्तोष्प॒तीय॑म् ॅवास्तोष्प॒तीय॑म् जुहोति भाग॒धेये॑न भाग॒धेये॑न जुहोति वास्तोष्प॒तीय॑म् ॅवास्तोष्प॒तीय॑म् जुहोति भाग॒धेये॑न । \newline
20. वा॒स्तो॒ष्प॒तीय॒मिति॑ वास्तोः - प॒तीय᳚म् । \newline
21. जु॒हो॒ति॒ भा॒ग॒धेये॑न भाग॒धेये॑न जुहोति जुहोति भाग॒धेये॑नै॒वैव भा॑ग॒धेये॑न जुहोति जुहोति भाग॒धेये॑नै॒व । \newline
22. भा॒ग॒धेये॑ नै॒वैव भा॑ग॒धेये॑न भाग॒धेये॑नै॒वैन॑ मेन मे॒व भा॑ग॒धेये॑न भाग॒धेये॑नै॒वैन᳚म् । \newline
23. भा॒ग॒धेये॒नेति॑ भाग - धेये॑न । \newline
24. ए॒वैन॑ मेन मे॒वैवैनꣳ॑ शमयति शमय त्येन मे॒वैवैनꣳ॑ शमयति । \newline
25. ए॒नꣳ॒॒ श॒म॒य॒ति॒ श॒म॒य॒ त्ये॒न॒ मे॒नꣳ॒॒ श॒म॒य॒ति॒ न न श॑मय त्येन मेनꣳ शमयति॒ न । \newline
26. श॒म॒य॒ति॒ न न श॑मयति शमयति॒ नार्ति॒ मार्ति॒म् न श॑मयति शमयति॒ नार्ति᳚म् । \newline
27. नार्ति॒ मार्ति॒म् न नार्ति॒ मा ऽऽर्ति॒म् न नार्ति॒ मा । \newline
28. आर्ति॒ मा ऽऽर्ति॒ मार्ति॒ मार्च्छ॑ त्यृच्छ॒ त्याऽऽर्ति॒ मार्ति॒ मार्च्छ॑ति । \newline
29. आर्च्छ॑ त्यृच्छ॒ त्यार्च्छ॑ति॒ यज॑मानो॒ यज॑मान ऋच्छ॒ त्यार्च्छ॑ति॒ यज॑मानः । \newline
30. ऋ॒च्छ॒ति॒ यज॑मानो॒ यज॑मान ऋच्छ त्यृच्छति॒ यज॑मानो॒ यद् यद् यज॑मान ऋच्छ त्यृच्छति॒ यज॑मानो॒ यत् । \newline
31. यज॑मानो॒ यद् यद् यज॑मानो॒ यज॑मानो॒ यद् यु॒क्ते यु॒क्ते यद् यज॑मानो॒ यज॑मानो॒ यद् यु॒क्ते । \newline
32. यद् यु॒क्ते यु॒क्ते यद् यद् यु॒क्ते जु॑हु॒याज् जु॑हु॒याद् यु॒क्ते यद् यद् यु॒क्ते जु॑हु॒यात् । \newline
33. यु॒क्ते जु॑हु॒याज् जु॑हु॒याद् यु॒क्ते यु॒क्ते जु॑हु॒याद् यथा॒ यथा॑ जुहु॒याद् यु॒क्ते यु॒क्ते जु॑हु॒याद् यथा᳚ । \newline
34. जु॒हु॒याद् यथा॒ यथा॑ जुहु॒याज् जु॑हु॒याद् यथा॒ प्रया॑ते॒ प्रया॑ते॒ यथा॑ जुहु॒याज् जु॑हु॒याद् यथा॒ प्रया॑ते । \newline
35. यथा॒ प्रया॑ते॒ प्रया॑ते॒ यथा॒ यथा॒ प्रया॑ते॒ वास्तौ॒ वास्तौ॒ प्रया॑ते॒ यथा॒ यथा॒ प्रया॑ते॒ वास्तौ᳚ । \newline
36. प्रया॑ते॒ वास्तौ॒ वास्तौ॒ प्रया॑ते॒ प्रया॑ते॒ वास्ता॒ वाहु॑ति॒ माहु॑ति॒म् ॅवास्तौ॒ प्रया॑ते॒ प्रया॑ते॒ वास्ता॒ वाहु॑तिम् । \newline
37. प्रया॑त॒ इति॒ प्र - या॒ते॒ । \newline
38. वास्ता॒ वाहु॑ति॒ माहु॑ति॒म् ॅवास्तौ॒ वास्ता॒ वाहु॑तिम् जु॒होति॑ जु॒हो त्याहु॑ति॒म् ॅवास्तौ॒ वास्ता॒ वाहु॑तिम् जु॒होति॑ । \newline
39. आहु॑तिम् जु॒होति॑ जु॒हो त्याहु॑ति॒ माहु॑तिम् जु॒होति॑ ता॒दृक् ता॒दृग् जु॒हो त्याहु॑ति॒ माहु॑तिम् जु॒होति॑ ता॒दृक् । \newline
40. आहु॑ति॒मित्या - हु॒ति॒म् । \newline
41. जु॒होति॑ ता॒दृक् ता॒दृग् जु॒होति॑ जु॒होति॑ ता॒दृगे॒वैव ता॒दृग् जु॒होति॑ जु॒होति॑ ता॒दृगे॒व । \newline
42. ता॒दृगे॒वैव ता॒दृक् ता॒दृगे॒व तत् तदे॒व ता॒दृक् ता॒दृगे॒व तत् । \newline
43. ए॒व तत् तदे॒वैव तद् यद् यत् तदे॒वैव तद् यत् । \newline
44. तद् यद् यत् तत् तद् य दयु॒क्ते ऽयु॑क्ते॒ यत् तत् तद् य दयु॑क्ते । \newline
45. य दयु॒क्ते ऽयु॑क्ते॒ यद् यदयु॑क्ते जुहु॒याज् जु॑हु॒या दयु॑क्ते॒ यद् यदयु॑क्ते जुहु॒यात् । \newline
46. अयु॑क्ते जुहु॒याज् जु॑हु॒या दयु॒क्ते ऽयु॑क्ते जुहु॒याद् यथा॒ यथा॑ जुहु॒या दयु॒क्ते ऽयु॑क्ते जुहु॒याद् यथा᳚ । \newline
47. जु॒हु॒याद् यथा॒ यथा॑ जुहु॒याज् जु॑हु॒याद् यथा॒ क्षेमे॒ क्षेमे॒ यथा॑ जुहु॒याज् जु॑हु॒याद् यथा॒ क्षेमे᳚ । \newline
48. यथा॒ क्षेमे॒ क्षेमे॒ यथा॒ यथा॒ क्षेम॒ आहु॑ति॒ माहु॑ति॒म् क्षेमे॒ यथा॒ यथा॒ क्षेम॒ आहु॑तिम् । \newline
49. क्षेम॒ आहु॑ति॒ माहु॑ति॒म् क्षेमे॒ क्षेम॒ आहु॑तिम् जु॒होति॑ जु॒हो त्याहु॑ति॒म् क्षेमे॒ क्षेम॒ आहु॑तिम् जु॒होति॑ । \newline
50. आहु॑तिम् जु॒होति॑ जु॒हो त्याहु॑ति॒ माहु॑तिम् जु॒होति॑ ता॒दृक् ता॒दृग् जु॒हो त्याहु॑ति॒ माहु॑तिम् जु॒होति॑ ता॒दृक् । \newline
51. आहु॑ति॒मित्या - हु॒ति॒म् । \newline
52. जु॒होति॑ ता॒दृक् ता॒दृग् जु॒होति॑ जु॒होति॑ ता॒दृगे॒वैव ता॒दृग् जु॒होति॑ जु॒होति॑ ता॒दृगे॒व । \newline
53. ता॒दृगे॒वैव ता॒दृक् ता॒दृगे॒व तत् तदे॒व ता॒दृक् ता॒दृगे॒व तत् । \newline
54. ए॒व तत् तदे॒वैव तदहु॑त॒ महु॑त॒म् तदे॒वैव तदहु॑तम् । \newline
55. तदहु॑त॒ महु॑त॒म् तत् तदहु॑त मस्या॒स्या हु॑त॒म् तत् तदहु॑त मस्य । \newline
56. अहु॑त मस्या॒स्या हु॑त॒ महु॑त मस्य वास्तोष्प॒तीय॑म् ॅवास्तोष्प॒तीय॑ म॒स्या हु॑त॒ महु॑त मस्य वास्तोष्प॒तीय᳚म् । \newline
57. अ॒स्य॒ वा॒स्तो॒ष्प॒तीय॑म् ॅवास्तोष्प॒तीय॑ मस्यास्य वास्तोष्प॒तीयꣳ॑ स्याथ् स्याद् वास्तोष्प॒तीय॑ मस्यास्य वास्तोष्प॒तीयꣳ॑ स्यात् । \newline
58. वा॒स्तो॒ष्प॒तीयꣳ॑ स्याथ् स्याद् वास्तोष्प॒तीय॑म् ॅवास्तोष्प॒तीयꣳ॑ स्या॒द् दक्षि॑णो॒ दक्षि॑णः स्याद् वास्तोष्प॒तीय॑म् ॅवास्तोष्प॒तीयꣳ॑ स्या॒द् दक्षि॑णः । \newline
59. वा॒स्तो॒ष्प॒तीय॒मिति॑ वास्तोः - प॒तीय᳚म् । \newline
60. स्या॒द् दक्षि॑णो॒ दक्षि॑णः स्याथ् स्या॒द् दक्षि॑णो यु॒क्तो यु॒क्तो दक्षि॑णः स्याथ् स्या॒द् दक्षि॑णो यु॒क्तः । \newline
\pagebreak
\markright{ TS 3.4.10.4  \hfill https://www.vedavms.in \hfill}

\section{ TS 3.4.10.4 }

\textbf{TS 3.4.10.4 } \newline
\textbf{Samhita Paata} \newline

द्दक्षि॑णो यु॒क्तो भव॑ति स॒व्योऽयु॒क्तोऽथ॑ वास्तोष्प॒तीयं॑ जुहोत्यु॒भय॑मे॒वाऽ क॒रप॑रिवर्गमे॒वैनꣳ॑ शमयति॒ यदेक॑या जुहु॒याद्द॑र्विहो॒मं कु॑र्यात् पुरोऽनुवा॒क्या॑ म॒नूच्य॑ या॒ज्य॑या जुहोति सदेव॒त्वाय॒ यद्धु॒त आ॑द॒द्ध्याद् रु॒द्रं गृ॒हान॒न्वारो॑हये॒द्-यद॑व॒क्षाणा॒न्यसं॑ प्रक्षाप्य प्रया॒याद्यथा॑ यज्ञ्वेश॒सं ॅवा॒ऽऽदह॑नं ॅवा ता॒दृगे॒व तद॒यन्ते॒ योनि॑र्.ऋ॒त्विय॒ इत्य॒रण्योः᳚ स॒मारो॑हय - [  ] \newline

\textbf{Pada Paata} \newline

दक्षि॑णः । यु॒क्तः । भव॑ति । स॒व्यः । अयु॑क्तः । अथ॑ । वा॒स्तो॒ष्प॒तीय॒मिति॑ वास्तोः -प॒तीय᳚म् । जु॒हो॒ति॒ । उ॒भय᳚म् । ए॒व । अ॒कः॒ । अप॑रिवर्ग॒मित्यप॑रि - व॒र्ग॒म् । ए॒व । ए॒न॒म् । श॒म॒य॒ति॒ । यत् । एक॑या । जु॒हु॒यात् । द॒र्वि॒हो॒ममिति॑ दर्वि - हो॒मम् । कु॒र्या॒त् । पु॒रो॒नु॒वा॒क्या॑मिति॑ पुरः - अ॒नु॒वा॒क्या᳚म् । अ॒नूच्येत्य॑नु - उच्य॑ । या॒ज्य॑या । जु॒हो॒ति॒ । स॒दे॒व॒त्वायेति॑ सदेव - त्वाय॑ । यत् । हु॒ते । आ॒द॒द्ध्यादित्या᳚ - द॒द्ध्यात् । रु॒द्रम् । गृ॒हान् । अ॒न्वारो॑हये॒दित्य॑नु - आरो॑हयेत् । यत् । अ॒व॒क्षाणा॒नीत्य॑व - क्षाणा॑नि । अस॑प्रंक्षा॒प्येत्यसं᳚ - प्र॒क्षा॒प्य॒ । प्र॒या॒यादिति॑ प्र - या॒यात् । यथा᳚ । य॒ज्ञ्॒वे॒श॒समिति॑ यज्ञ् - वे॒श॒सम् । वा॒ । आ॒दह॑न॒मित्या᳚ - दह॑नम् । वा॒ । ता॒दृक् । ए॒व । तत् । अ॒यम् । ते॒ । योनिः॑ । ऋ॒त्वियः॑ । इति॑ । अ॒रण्योः᳚ । स॒मारो॑हय॒तीति॑ सं - आरो॑हयति ।  \newline


\textbf{Krama Paata} \newline

दक्षि॑णो यु॒क्तः । यु॒क्तो भव॑ति । भव॑ति स॒व्यः । स॒व्यो ऽयु॑क्तः । अयु॒क्तो ऽथ॑ । अथ॑ वास्तोष्प॒तीय᳚म् । वा॒स्तो॒ष्प॒तीय॑म् जुहोति । वा॒स्तो॒ष्प॒तीय॒मिति॑ वास्तोः - प॒तीय᳚म् । जु॒हो॒त्यु॒भय᳚म् । उ॒भय॑मे॒व । ए॒वाकः॑ । अ॒क॒रप॑रिवर्गम् । अप॑रिवर्गमे॒व । अप॑रिवर्ग॒मित्यप॑रि - व॒र्ग॒म् । ए॒वैन᳚म् । ए॒नꣳ॒॒ श॒म॒य॒ति॒ । श॒म॒य॒ति॒ यत् । यदेक॑या । एक॑या जुहु॒यात् । जु॒हु॒याद् द॑र्विहो॒मम् । द॒र्वि॒हो॒मम् कु॑र्यात् । द॒र्वि॒हो॒ममिति॑ दर्वि - हो॒मम् । कु॒र्या॒त् पु॒रो॒नु॒वा॒क्या᳚म् । पु॒रो॒नु॒वा॒क्या॑म॒नूच्य॑ । पु॒रो॒नु॒वा॒क्या॑मिति॑ पुरः - अ॒नु॒वा॒क्या᳚म् । अ॒नूच्य॑ या॒ज्य॑या । अ॒नुच्येत्य॑नु - उच्य॑ । या॒ज्य॑या जुहोति । जु॒हो॒ति॒ स॒दे॒व॒त्वाय॑ । स॒दे॒व॒त्वाय॒ यत् । स॒दे॒व॒त्वायेति॑ सदेव - त्वाय॑ । यद्धु॒ते । हु॒त आ॑द॒द्ध्यात् । आ॒द॒द्ध्याद् रु॒द्रम् । आ॒द॒द्ध्यादित्या᳚ - द॒द्ध्यात् । रु॒द्रम् गृ॒हान् । गृ॒हान॒न्वारो॑हयेत् । अ॒न्वारो॑हये॒द् यत् । अ॒न्वारो॑हये॒दित्य॑नु - आरो॑हयेत् । यद॑व॒क्षाणा॑नि । अ॒व॒क्षाणा॒न्यस॑म्प्रक्षाप्य । अ॒व॒क्षाणा॒नीत्य॑व - क्षाणा॑नि । अस॑म्प्रक्षाप्य प्रया॒यात् । अस॑म्प्रक्षा॒प्येत्यस᳚म् - प्र॒क्षा॒प्य॒ । प्र॒या॒याद् यथा᳚ । प्र॒या॒यादिति॑ प्र - या॒यात् । यथा॑ यज्ञ्वेश॒सम् । य॒ज्ञ्॒वे॒श॒सं ॅवा᳚ । य॒ज्ञ्॒वे॒श॒समिति॑ यज्ञ् - वे॒श॒सम् । वा॒ ऽऽदह॑नम् । आ॒दह॑नं ॅवा । आ॒दह॑न॒मित्या᳚ - दह॑नम् । वा॒ ता॒दृक् । ता॒दृगे॒व । ए॒व तत् । तद॒यम् । अ॒यम् ते᳚ । ते॒ योनिः॑ । योनि॑र्. ऋ॒त्वियः॑ । ऋ॒त्विय॒ इति॑ । इत्य॒रण्योः᳚ । अ॒रण्योः᳚ स॒मारो॑हयति ( ) । स॒मारो॑हयत्ये॒षः । स॒मारो॑हय॒तीति॑ सम् - आरो॑हयति \newline

\textbf{Jatai Paata} \newline

1. दक्षि॑णो यु॒क्तो यु॒क्तो दक्षि॑णो॒ दक्षि॑णो यु॒क्तः । \newline
2. यु॒क्तो भव॑ति॒ भव॑ति यु॒क्तो यु॒क्तो भव॑ति । \newline
3. भव॑ति स॒व्यः स॒व्यो भव॑ति॒ भव॑ति स॒व्यः । \newline
4. स॒व्यो ऽयु॒क्तो ऽयु॑क्तः स॒व्यः स॒व्यो ऽयु॑क्तः । \newline
5. अयु॒क्तो ऽथाथा यु॒क्तो ऽयु॒क्तो ऽथ॑ । \newline
6. अथ॑ वास्तोष्प॒तीयं॑ ॅवास्तोष्प॒तीय॒ मथाथ॑ वास्तोष्प॒तीय᳚म् । \newline
7. वा॒स्तो॒ष्प॒तीय॑म् जुहोति जुहोति वास्तोष्प॒तीयं॑ ॅवास्तोष्प॒तीय॑म् जुहोति । \newline
8. वा॒स्तो॒ष्प॒तीय॒मिति॑ वास्तोः - प॒तीय᳚म् । \newline
9. जु॒हो॒ त्यु॒भय॑ मु॒भय॑म् जुहोति जुहो त्यु॒भय᳚म् । \newline
10. उ॒भय॑ मे॒वैवोभय॑ मु॒भय॑ मे॒व । \newline
11. ए॒वाक॑ रक रे॒वैवाकः॑ । \newline
12. अ॒क॒ रप॑रिवर्ग॒ मप॑रिवर्ग मक रक॒ रप॑रिवर्गम् । \newline
13. अप॑रिवर्ग मे॒वैवा प॑रिवर्ग॒ मप॑रिवर्ग मे॒व । \newline
14. अप॑रिवर्ग॒मित्यप॑रि - व॒र्ग॒म् । \newline
15. ए॒वैन॑ मेन मे॒वै वैन᳚म् । \newline
16. ए॒नꣳ॒॒ श॒म॒य॒ति॒ श॒म॒य॒ त्ये॒न॒मे॒नꣳ॒॒ श॒म॒य॒ति॒ । \newline
17. श॒म॒य॒ति॒ यद् यच्छ॑मयति शमयति॒ यत् । \newline
18. यदेक॒यैक॑या॒ यद् यदेक॑या । \newline
19. एक॑या जुहु॒याज् जु॑हु॒या देक॒यैक॑या जुहु॒यात् । \newline
20. जु॒हु॒याद् द॑र्विहो॒मम् द॑र्विहो॒मम् जु॑हु॒याज् जु॑हु॒याद् द॑र्विहो॒मम् । \newline
21. द॒र्वि॒हो॒मम् कु॑र्यात् कुर्याद् दर्विहो॒मम् द॑र्विहो॒मम् कु॑र्यात् । \newline
22. द॒र्वि॒हो॒ममिति॑ दर्वि - हो॒मम् । \newline
23. कु॒र्या॒त् पु॒रो॒नु॒वा॒क्या᳚म् पुरोनुवा॒क्या᳚म् कुर्यात् कुर्यात् पुरोनुवा॒क्या᳚म् । \newline
24. पु॒रो॒नु॒वा॒क्या॑ म॒नूच्या॒ नूच्य॑ पुरोनुवा॒क्या᳚म् पुरोनुवा॒क्या॑ म॒नूच्य॑ । \newline
25. पु॒रो॒नु॒वा॒क्या॑मिति॑ पुरः - अ॒नु॒वा॒क्या᳚म् । \newline
26. अ॒नूच्य॑ या॒ज्य॑या या॒ज्य॑या॒ ऽनूच्या॒ नूच्य॑ या॒ज्य॑या । \newline
27. अ॒नूच्येत्य॑नु - उच्य॑ । \newline
28. या॒ज्य॑या जुहोति जुहोति या॒ज्य॑या या॒ज्य॑या जुहोति । \newline
29. जु॒हो॒ति॒ स॒दे॒व॒त्वाय॑ सदेव॒त्वाय॑ जुहोति जुहोति सदेव॒त्वाय॑ । \newline
30. स॒दे॒व॒त्वाय॒ यद् यथ् स॑देव॒त्वाय॑ सदेव॒त्वाय॒ यत् । \newline
31. स॒दे॒व॒त्वायेति॑ सदेव - त्वाय॑ । \newline
32. यद्धु॒ते हु॒ते यद् यद्धु॒ते । \newline
33. हु॒त आ॑द॒द्ध्या दा॑द॒द्ध्या द्धु॒ते हु॒त आ॑द॒द्ध्यात् । \newline
34. आ॒द॒द्ध्याद् रु॒द्रꣳ रु॒द्र मा॑द॒द्ध्यादा॑ द॒द्ध्याद् रु॒द्रम् । \newline
35. आ॒द॒द्ध्यादित्या᳚ - द॒द्ध्यात् । \newline
36. रु॒द्रम् गृ॒हान् गृ॒हान् रु॒द्रꣳ रु॒द्रम् गृ॒हान् । \newline
37. गृ॒हा-न॒न्वारो॑हये द॒न्वारो॑हयेद् गृ॒हान् गृ॒हा-न॒न्वारो॑हयेत् । \newline
38. अ॒न्वारो॑हये॒द् यद् यद॒न्वारो॑हये द॒न्वारो॑हये॒द् यत् । \newline
39. अ॒न्वारो॑हये॒दित्य॑नु - आरो॑हयेत् । \newline
40. यद॑व॒क्षाणा᳚ न्यव॒क्षाणा॑नि॒ यद् यद॑व॒क्षाणा॑नि । \newline
41. अ॒व॒क्षाणा॒ न्यस॑म्प्रक्षा॒प्या स॑म्प्रक्षाप्या व॒क्षाणा᳚ न्यव॒क्षाणा॒ न्यस॑म्प्रक्षाप्य । \newline
42. अ॒व॒क्षाणा॒नीत्य॑व - क्षाणा॑नि । \newline
43. अस॑म्प्रक्षाप्य प्रया॒यात् प्र॑या॒या दस॑म्प्रक्षा॒प्या स॑म्प्रक्षाप्य प्रया॒यात् । \newline
44. अस॑म्प्रक्षा॒प्येत्यसं᳚ - प्र॒क्षा॒प्य॒ । \newline
45. प्र॒या॒याद् यथा॒ यथा᳚ प्रया॒यात् प्र॑या॒याद् यथा᳚ । \newline
46. प्र॒या॒यादिति॑ प्र - या॒यात् । \newline
47. यथा॑ यज्ञ्वेश॒सं ॅय॑ज्ञ्वेश॒सं ॅयथा॒ यथा॑ यज्ञ्वेश॒सम् । \newline
48. य॒ज्ञ्॒वे॒श॒सं ॅवा॑ वा यज्ञ्वेश॒सं ॅय॑ज्ञ्वेश॒सं ॅवा᳚ । \newline
49. य॒ज्ञ्॒वे॒श॒समिति॑ यज्ञ् - वे॒श॒सम् । \newline
50. वा॒ ऽऽदह॑न मा॒दह॑नं ॅवा वा॒ ऽऽदह॑नम् । \newline
51. आ॒दह॑नं ॅवा वा॒ ऽऽदह॑न मा॒दह॑नं ॅवा । \newline
52. आ॒दह॑न॒मित्या᳚ - दह॑नम् । \newline
53. वा॒ ता॒दृक् ता॒दृग् वा॑ वा ता॒दृक् । \newline
54. ता॒दृगे॒वैव ता॒दृक् ता॒दृगे॒व । \newline
55. ए॒व तत् तदे॒ वैव तत् । \newline
56. तद॒य म॒यम् तत् तद॒यम् । \newline
57. अ॒यम् ते॑ ते॒ ऽय म॒यम् ते᳚ । \newline
58. ते॒ योनि॒र् योनि॑ स्ते ते॒ योनिः॑ । \newline
59. योनि॑र्. ऋ॒त्विय॑ ऋ॒त्वियो॒ योनि॒र् योनि॑र्. ऋ॒त्वियः॑ । \newline
60. ऋ॒त्विय॒ इती त्यृ॒त्विय॑ ऋ॒त्विय॒ इति॑ । \newline
61. इत्य॒ रण्यो॑ र॒रण्यो॒ रिती त्य॒रण्योः᳚ । \newline
62. अ॒रण्योः᳚ स॒मारो॑हयति स॒मारो॑हय त्य॒रण्यो॑र॒ रण्योः᳚ स॒मारो॑हयति । \newline
63. स॒मारो॑हय त्ये॒ष ए॒ष स॒मारो॑हयति स॒मारो॑हय त्ये॒षः । \newline
64. स॒मारो॑हय॒तीति॑ सं - आरो॑हयति । \newline

\textbf{Ghana Paata } \newline

1. दक्षि॑णो यु॒क्तो यु॒क्तो दक्षि॑णो॒ दक्षि॑णो यु॒क्तो भव॑ति॒ भव॑ति यु॒क्तो दक्षि॑णो॒ दक्षि॑णो यु॒क्तो भव॑ति । \newline
2. यु॒क्तो भव॑ति॒ भव॑ति यु॒क्तो यु॒क्तो भव॑ति स॒व्यः स॒व्यो भव॑ति यु॒क्तो यु॒क्तो भव॑ति स॒व्यः । \newline
3. भव॑ति स॒व्यः स॒व्यो भव॑ति॒ भव॑ति स॒व्यो ऽयु॒क्तो ऽयु॑क्तः स॒व्यो भव॑ति॒ भव॑ति स॒व्यो ऽयु॑क्तः । \newline
4. स॒व्यो ऽयु॒क्तो ऽयु॑क्तः स॒व्यः स॒व्यो ऽयु॒क्तो ऽथाथा यु॑क्तः स॒व्यः स॒व्यो ऽयु॒क्तो ऽथ॑ । \newline
5. अयु॒क्तो ऽथा थायु॒क्तो ऽयु॒क्तो ऽथ॑ वास्तोष्प॒तीय॑म् ॅवास्तोष्प॒तीय॒ मथा यु॒क्तो ऽयु॒क्तो ऽथ॑ वास्तोष्प॒तीय᳚म् । \newline
6. अथ॑ वास्तोष्प॒तीय॑म् ॅवास्तोष्प॒तीय॒ मथाथ॑ वास्तोष्प॒तीय॑म् जुहोति जुहोति वास्तोष्प॒तीय॒ मथाथ॑ वास्तोष्प॒तीय॑म् जुहोति । \newline
7. वा॒स्तो॒ष्प॒तीय॑म् जुहोति जुहोति वास्तोष्प॒तीय॑म् ॅवास्तोष्प॒तीय॑म् जुहो त्यु॒भय॑ मु॒भय॑म् जुहोति 
वास्तोष्प॒तीय॑म् ॅवास्तोष्प॒तीय॑म् जुहो त्यु॒भय᳚म् । \newline
8. वा॒स्तो॒ष्प॒तीय॒मिति॑ वास्तोः - प॒तीय᳚म् । \newline
9. जु॒हो॒ त्यु॒भय॑ मु॒भय॑म् जुहोति जुहो त्यु॒भय॑ मे॒वैवोभय॑म् जुहोति जुहो त्यु॒भय॑ मे॒व । \newline
10. उ॒भय॑ मे॒वैवोभय॑ मु॒भय॑ मे॒वाक॑ रक रे॒वो भय॑ मु॒भय॑ मे॒वाकः॑ । \newline
11. ए॒वाक॑ रक रे॒वैवाक॒ रप॑रिवर्ग॒ मप॑रिवर्ग मक रे॒वैवाक॒ रप॑रिवर्गम् । \newline
12. अ॒क॒ रप॑रिवर्ग॒ मप॑रिवर्ग मक रक॒ रप॑रिवर्ग मे॒वैवा प॑रिवर्ग मक रक॒ रप॑रिवर्ग मे॒व । \newline
13. अप॑रिवर्ग मे॒वैवा प॑रिवर्ग॒ मप॑रिवर्ग मे॒वैन॑ मेन मे॒वा प॑रिवर्ग॒ मप॑रिवर्ग मे॒वैन᳚म् । \newline
14. अप॑रिवर्ग॒मित्यप॑रि - व॒र्ग॒म् । \newline
15. ए॒वैन॑ मेन मे॒वैवैनꣳ॑ शमयति शमय त्येन मे॒वैवैनꣳ॑ शमयति । \newline
16. ए॒नꣳ॒॒ श॒म॒य॒ति॒ श॒म॒य॒ त्ये॒न॒ मे॒नꣳ॒॒ श॒म॒य॒ति॒ यद् यच्छ॑मय त्येन मेनꣳ शमयति॒ यत् । \newline
17. श॒म॒य॒ति॒ यद् यच्छ॑मयति शमयति॒ यदेक॒ यैक॑या॒ यच्छ॑मयति शमयति॒ यदेक॑या । \newline
18. यदेक॒ यैक॑या॒ यद् यदेक॑या जुहु॒याज् जु॑हु॒या देक॑या॒ यद् यदेक॑या जुहु॒यात् । \newline
19. एक॑या जुहु॒याज् जु॑हु॒या देक॒ यैक॑या जुहु॒याद् द॑र्विहो॒मम् द॑र्विहो॒मम् जु॑हु॒या देक॒ यैक॑या जुहु॒याद् द॑र्विहो॒मम् । \newline
20. जु॒हु॒याद् द॑र्विहो॒मम् द॑र्विहो॒मम् जु॑हु॒याज् जु॑हु॒याद् द॑र्विहो॒मम् कु॑र्यात् कुर्याद् दर्विहो॒मम् जु॑हु॒याज् जु॑हु॒याद् द॑र्विहो॒मम् कु॑र्यात् । \newline
21. द॒र्वि॒हो॒मम् कु॑र्यात् कुर्याद् दर्विहो॒मम् द॑र्विहो॒मम् कु॑र्यात् पुरोनुवा॒क्या᳚म् पुरोनुवा॒क्या᳚म् कुर्याद् दर्विहो॒मम् द॑र्विहो॒मम् कु॑र्यात् पुरोनुवा॒क्या᳚म् । \newline
22. द॒र्वि॒हो॒ममिति॑ दर्वि - हो॒मम् । \newline
23. कु॒र्या॒त् पु॒रो॒नु॒वा॒क्या᳚म् पुरोनुवा॒क्या᳚म् कुर्यात् कुर्यात् पुरोनुवा॒क्या॑ म॒नूच्या॒ नूच्य॑ पुरोनुवा॒क्या᳚म् कुर्यात् कुर्यात् पुरोनुवा॒क्या॑ म॒नूच्य॑ । \newline
24. पु॒रो॒नु॒वा॒क्या॑ म॒नूच्या॒ नूच्य॑ पुरोनुवा॒क्या᳚म् पुरोनुवा॒क्या॑ म॒नूच्य॑ या॒ज्य॑या या॒ज्य॑या॒ ऽनूच्य॑ 
पुरोनुवा॒क्या᳚म् पुरोनुवा॒क्या॑ म॒नूच्य॑ या॒ज्य॑या । \newline
25. पु॒रो॒नु॒वा॒क्या॑मिति॑ पुरः - अ॒नु॒वा॒क्या᳚म् । \newline
26. अ॒नूच्य॑ या॒ज्य॑या या॒ज्य॑या॒ ऽनूच्या॒ नूच्य॑ या॒ज्य॑या जुहोति जुहोति या॒ज्य॑या॒ ऽनूच्या॒ नूच्य॑ या॒ज्य॑या जुहोति । \newline
27. अ॒नूच्येत्य॑नु - उच्य॑ । \newline
28. या॒ज्य॑या जुहोति जुहोति या॒ज्य॑या या॒ज्य॑या जुहोति सदेव॒त्वाय॑ सदेव॒त्वाय॑ जुहोति या॒ज्य॑या या॒ज्य॑या जुहोति सदेव॒त्वाय॑ । \newline
29. जु॒हो॒ति॒ स॒दे॒व॒त्वाय॑ सदेव॒त्वाय॑ जुहोति जुहोति सदेव॒त्वाय॒ यद् यथ् स॑देव॒त्वाय॑ जुहोति जुहोति सदेव॒त्वाय॒ यत् । \newline
30. स॒दे॒व॒त्वाय॒ यद् यथ् स॑देव॒त्वाय॑ सदेव॒त्वाय॒ यद्धु॒ते हु॒ते यथ् स॑देव॒त्वाय॑ सदेव॒त्वाय॒ यद्धु॒ते । \newline
31. स॒दे॒व॒त्वायेति॑ सदेव - त्वाय॑ । \newline
32. यद्धु॒ते हु॒ते यद् यद्धु॒त आ॑द॒द्ध्या दा॑द॒द्ध्या द्धु॒ते यद् यद्धु॒त आ॑द॒द्ध्यात् । \newline
33. हु॒त आ॑द॒द्ध्यादा॑ द॒द्ध्या द्धु॒ते हु॒त आ॑द॒द्ध्याद् रु॒द्रꣳ रु॒द्र मा॑द॒द्ध्या द्धु॒ते हु॒त आ॑द॒द्ध्याद् रु॒द्रम् । \newline
34. आ॒द॒द्ध्याद् रु॒द्रꣳ रु॒द्र मा॑द॒द्ध्या दा॑द॒द्ध्याद् रु॒द्रम् गृ॒हान् गृ॒हान् रु॒द्र मा॑द॒द्ध्या दा॑द॒द्ध्याद् रु॒द्रम् गृ॒हान् । \newline
35. आ॒द॒द्ध्यादित्या᳚ - द॒द्ध्यात् । \newline
36. रु॒द्रम् गृ॒हान् गृ॒हान् रु॒द्रꣳ रु॒द्रम् गृ॒हा,न॒न्वारो॑हये द॒न्वारो॑हयेद् गृ॒हान् रु॒द्रꣳ रु॒द्रम् गृ॒हा,न॒न्वारो॑हयेत् । \newline
37. गृ॒हा,न॒न्वारो॑हये द॒न्वारो॑हयेद् गृ॒हान् गृ॒हा,न॒न्वारो॑हये॒द् यद् यद॒न्वारो॑हयेद् गृ॒हान् गृ॒हा,न॒न्वारो॑हये॒द् यत् । \newline
38. अ॒न्वारो॑हये॒द् यद् यद॒न्वारो॑हये द॒न्वारो॑हये॒द् यद॑व॒क्षाणा᳚ न्यव॒क्षाणा॑नि॒ यद॒न्वारो॑हये 
द॒न्वारो॑हये॒द् यद॑व॒क्षाणा॑नि । \newline
39. अ॒न्वारो॑हये॒दित्य॑नु - आरो॑हयेत् । \newline
40. यद॑व॒क्षाणा᳚ न्यव॒क्षाणा॑नि॒ यद् यद॑व॒क्षाणा॒ न्यस॑म्प्रक्षा॒प्या स॑म्प्रक्षाप्या व॒क्षाणा॑नि॒ यद् यद॑व॒क्षाणा॒ न्यस॑म्प्रक्षाप्य । \newline
41. अ॒व॒क्षाणा॒ न्यस॑म्प्रक्षा॒प्या स॑म्प्रक्षाप्या व॒क्षाणा᳚ न्यव॒क्षाणा॒ न्यस॑म्प्रक्षाप्य प्रया॒यात् प्र॑या॒या दस॑म्प्रक्षाप्या व॒क्षाणा᳚ न्यव॒क्षाणा॒ न्यस॑म्प्रक्षाप्य प्रया॒यात् । \newline
42. अ॒व॒क्षाणा॒नीत्य॑व - क्षाणा॑नि । \newline
43. अस॑म्प्रक्षाप्य प्रया॒यात् प्र॑या॒या दस॑म्प्रक्षा॒प्या स॑म्प्रक्षाप्य प्रया॒याद् यथा॒ यथा᳚ 
प्रया॒या दस॑म्प्रक्षा॒प्या स॑म्प्रक्षाप्य प्रया॒याद् यथा᳚ । \newline
44. अस॑म्प्रक्षा॒प्येत्यस᳚म् - प्र॒क्षा॒प्य॒ । \newline
45. प्र॒या॒याद् यथा॒ यथा᳚ प्रया॒यात् प्र॑या॒याद् यथा॑ यज्ञ्वेश॒सम् ॅय॑ज्ञ्वेश॒सम् ॅयथा᳚ प्रया॒यात् 
प्र॑या॒याद् यथा॑ यज्ञ्वेश॒सम् । \newline
46. प्र॒या॒यादिति॑ प्र - या॒यात् । \newline
47. यथा॑ यज्ञ्वेश॒सम् ॅय॑ज्ञ्वेश॒सम् ॅयथा॒ यथा॑ यज्ञ्वेश॒सम् ॅवा॑ वा यज्ञ्वेश॒सम् ॅयथा॒ यथा॑ 
यज्ञ्वेश॒सम् ॅवा᳚ । \newline
48. य॒ज्ञ्॒वे॒श॒सम् ॅवा॑ वा यज्ञ्वेश॒सम् ॅय॑ज्ञ्वेश॒सम् ॅवा॒ ऽऽदह॑न मा॒दह॑नम् ॅवा यज्ञ्वेश॒सम् ॅय॑ज्ञ्वेश॒सम् ॅवा॒ ऽऽदह॑नम् । \newline
49. य॒ज्ञ्॒वे॒श॒समिति॑ यज्ञ् - वे॒श॒सम् । \newline
50. वा॒ ऽऽदह॑न मा॒दह॑नम् ॅवा वा॒ ऽऽदह॑नम् ॅवा वा॒ ऽऽदह॑नम् ॅवा वा॒ ऽऽदह॑नम् ॅवा । \newline
51. आ॒दह॑नम् ॅवा वा॒ ऽऽदह॑न मा॒दह॑नम् ॅवा ता॒दृक् ता॒दृग् वा॒ ऽऽदह॑न मा॒दह॑नम् ॅवा ता॒दृक् । \newline
52. आ॒दह॑न॒मित्या᳚ - दह॑नम् । \newline
53. वा॒ ता॒दृक् ता॒दृग् वा॑ वा ता॒दृ गे॒वैव ता॒दृग् वा॑ वा ता॒दृगे॒व । \newline
54. ता॒दृ गे॒वैव ता॒दृक् ता॒दृगे॒व तत् तदे॒व ता॒दृक् ता॒दृगे॒व तत् । \newline
55. ए॒व तत् तदे॒वैव तद॒य म॒यम् तदे॒वैव तद॒यम् । \newline
56. तद॒य म॒यम् तत् तद॒यम् ते॑ ते॒ ऽयम् तत् तद॒यम् ते᳚ । \newline
57. अ॒यम् ते॑ ते॒ ऽय म॒यम् ते॒ योनि॒र् योनि॑ स्ते॒ ऽय म॒यम् ते॒ योनिः॑ । \newline
58. ते॒ योनि॒र् योनि॑ स्ते ते॒ योनि॑र्. ऋ॒त्विय॑ ऋ॒त्वियो॒ योनि॑ स्ते ते॒ योनि॑र्. ऋ॒त्वियः॑ । \newline
59. योनि॑र्. ऋ॒त्विय॑ ऋ॒त्वियो॒ योनि॒र् योनि॑र्. ऋ॒त्विय॒ इती त्यृ॒त्वियो॒ योनि॒र् योनि॑र्. ऋ॒त्विय॒ इति॑ । \newline
60. ऋ॒त्विय॒ इती त्यृ॒त्विय॑ ऋ॒त्विय॒ इत्य॒रण्यो॑ र॒रण्यो॒ रित्यृ॒त्विय॑ ऋ॒त्विय॒ इत्य॒रण्योः᳚ । \newline
61. इत्य॒रण्यो॑र॒ रण्यो॒ रिती त्य॒रण्योः᳚ स॒मारो॑हयति स॒मारो॑हय त्य॒रण्यो॒ रिती त्य॒रण्योः᳚ स॒मारो॑हयति । \newline
62. अ॒रण्योः᳚ स॒मारो॑हयति स॒मारो॑हय त्य॒रण्यो॑ र॒रण्योः᳚ स॒मारो॑हय त्ये॒ष ए॒ष स॒मारो॑हय
त्य॒रण्यो॑र॒ रण्योः᳚ स॒मारो॑हय त्ये॒षः । \newline
63. स॒मारो॑हय त्ये॒ष ए॒ष स॒मारो॑हयति स॒मारो॑हय त्ये॒ष वै वा ए॒ष स॒मारो॑हयति स॒मारो॑हय
त्ये॒ष वै । \newline
64. स॒मारो॑हय॒तीति॑ सम् - आरो॑हयति । \newline
\pagebreak
\markright{ TS 3.4.10.5  \hfill https://www.vedavms.in \hfill}

\section{ TS 3.4.10.5 }

\textbf{TS 3.4.10.5 } \newline
\textbf{Samhita Paata} \newline

-त्ये॒ष वा अ॒ग्नेर्योनिः॒ स्व ए॒वैनं॒ ॅयोनौ॑ स॒मारो॑हय॒त्यथो॒ खल्वा॑हु॒र्यद॒रण्योः᳚ स॒मारू॑ढो॒ नश्ये॒दुद॑स्या॒ग्निः सी॑देत् पुनरा॒धेयः॑ स्या॒दिति॒ या ते॑ अग्ने य॒ज्ञिया॑ त॒नूस्तयेह्या रो॒हेत्या॒त्मन्थ् स॒मारो॑हयते॒ यज॑मानो॒ वा अ॒ग्नेर्योनिः॒ स्वाया॑मे॒वैनं॒ ॅयोन्याꣳ॑ स॒मारो॑हयते ॥ \newline

\textbf{Pada Paata} \newline

ए॒षः । वै । अ॒ग्नेः । योनिः॑ । स्वे । ए॒व । ए॒न॒म् । योनौ᳚ । स॒मारो॑हय॒तीति॑ सं-आरो॑हयति । अथो॒ इति॑ । खलु॑ । आ॒हुः॒ । यत् । अ॒रण्योः᳚ । स॒मारू॑ढ॒ इति॑ सं-आरू॑ढः । नश्ये᳚त् । उदिति॑ । अ॒स्य॒ । अ॒ग्निः । सी॒दे॒त् । पु॒न॒रा॒धेय॒ इति॑ पुनः - आ॒धेयः॑ । स्या॒त् । इति॑ । या । ते॒ । अ॒ग्न॒ । य॒ज्ञिया᳚ । त॒नूः । तया᳚ । एति॑ । इ॒हि॒ । एति॑ । रो॒ह॒ । इति॑ । आ॒त्मन्न् । स॒मारो॑हयत॒ इति॑ सं-आरो॑हयते । यज॑मानः । वै । अ॒ग्नेः । योनिः॑ । स्वाया᳚म् । ए॒व । ए॒न॒म् । योन्या᳚म् । स॒मारो॑हयत॒ इति॑ सं - आरो॑हयते ॥  \newline


\textbf{Krama Paata} \newline

ए॒ष वै । वा अ॒ग्नेः । अ॒ग्नेर् योनिः॑ । योनिः॒ स्वे । स्व ए॒व । ए॒वैन᳚म् । ए॒नं॒ ॅयोनौ᳚ । योनौ॑ स॒मारो॑हयति । स॒मारो॑हय॒त्यथो᳚ । स॒मारो॑हय॒तीति॑ सम् - आरो॑हयति । अथो॒ खलु॑ । अथो॒ इत्यथो᳚ । खल्वा॑हुः । आ॒हु॒र् यत् । यद॒रण्योः᳚ । अ॒रण्योः᳚ स॒मारू॑ढः । स॒मारू॑ढो॒ नश्ये᳚त् । स॒मारू॑ढ॒ इति॑ सम् - आरू॑ढः । नश्ये॒दुत् । उद॑स्य । अ॒स्या॒ग्निः । अ॒ग्निः सी॑देत् । सी॒दे॒त् पु॒न॒रा॒धेयः॑ । पु॒न॒रा॒धेयः॑ स्यात् । पु॒न॒रा॒धेय॒ इति॑ पुनः - आ॒धेयः॑ । स्या॒दिति॑ । इति॒ या । या ते᳚ । ते॒ अ॒ग्ने॒ । अ॒ग्ने॒ य॒ज्ञिया᳚ । य॒ज्ञिया॑ त॒नूः । त॒नूस्तया᳚ । तया । एहि॑ । इ॒ह्या । आ रो॑ह । रो॒हेति॑ । इत्या॒त्मन्न् । आ॒त्मन्थ् स॒मारो॑हयते । स॒मारो॑हयते॒ यज॑मानः । स॒मारो॑हयत॒ इति॑ सम् - आरो॑हयते । यज॑मानो॒ वै । वा अ॒ग्नेः । अ॒ग्नेर् योनिः॑ । योनिः॒ स्वाया᳚म् । स्वाया॑मे॒व । ए॒वैन᳚म् । ए॒नं॒ ॅयोन्या᳚म् । योन्याꣳ॑ स॒मारो॑हयते । स॒मारो॑हयत॒ इति॑ सम् - आरो॑हयते । \newline

\textbf{Jatai Paata} \newline

1. ए॒ष वै वा ए॒ष ए॒ष वै । \newline
2. वा अ॒ग्ने र॒ग्नेर् वै वा अ॒ग्नेः । \newline
3. अ॒ग्नेर् योनि॒र् योनि॑ र॒ग्ने र॒ग्नेर् योनिः॑ । \newline
4. योनिः॒ स्वे स्वे योनि॒र् योनिः॒ स्वे । \newline
5. स्व ए॒वैव स्वे स्व ए॒व । \newline
6. ए॒वैन॑ मेन मे॒वै वैन᳚म् । \newline
7. ए॒नं॒ ॅयोनौ॒ योना॑ वेन मेनं॒ ॅयोनौ᳚ । \newline
8. योनौ॑ स॒मारो॑हयति स॒मारो॑हयति॒ योनौ॒ योनौ॑ स॒मारो॑हयति । \newline
9. स॒मारो॑हय॒ त्यथो॒ अथो॑ स॒मारो॑हयति स॒मारो॑हय॒ त्यथो᳚ । \newline
10. स॒मारो॑हय॒तीति॑ सं - आरो॑हयति । \newline
11. अथो॒ खलु॒ खल् वथो॒ अथो॒ खलु॑ । \newline
12. अथो॒ इत्यथो᳚ । \newline
13. खल्वा॑हु राहुः॒ खलु॒ खल्वा॑हुः । \newline
14. आ॒हु॒र् यद् यदा॑हु राहु॒र् यत् । \newline
15. यद॒रण्यो॑ र॒रण्यो॒र् यद् यद॒रण्योः᳚ । \newline
16. अ॒रण्योः᳚ स॒मारू॑ढः स॒मारू॑ढो॒ ऽरण्यो॑ र॒रण्योः᳚ स॒मारू॑ढः । \newline
17. स॒मारू॑ढो॒ नश्ये॒न् नश्ये᳚थ् स॒मारू॑ढः स॒मारू॑ढो॒ नश्ये᳚त् । \newline
18. स॒मारू॑ढ॒ इति॑ सं - आरू॑ढः । \newline
19. नश्ये॒ दुदुन् नश्ये॒न् नश्ये॒ दुत् । \newline
20. उद॑स्या॒ स्यो दुद॑स्य । \newline
21. अ॒स्या॒ ग्नि र॒ग्नि र॑स्या स्या॒ ग्निः । \newline
22. अ॒ग्निः सी॑देथ् सीदे द॒ग्नि र॒ग्निः सी॑देत् । \newline
23. सी॒दे॒त् पु॒न॒रा॒धेयः॑ पुनरा॒धेयः॑ सीदेथ् सीदेत् पुनरा॒धेयः॑ । \newline
24. पु॒न॒रा॒धेयः॑ स्याथ् स्यात् पुनरा॒धेयः॑ पुनरा॒धेयः॑ स्यात् । \newline
25. पु॒न॒रा॒धेय॒ इति॑ पुनः - आ॒धेयः॑ । \newline
26. स्या॒दि तीति॑ स्याथ् स्या॒ दिति॑ । \newline
27. इति॒ या येतीति॒ या । \newline
28. या ते॑ ते॒ या या ते᳚ । \newline
29. ते॒ अ॒ग्ने॒ ऽग्ने॒ ते॒ ते॒ अ॒ग्ने॒ । \newline
30. अ॒ग्ने॒ य॒ज्ञिया॑ य॒ज्ञिया᳚ ऽग्ने ऽग्ने य॒ज्ञिया᳚ । \newline
31. य॒ज्ञिया॑ त॒नू स्त॒नूर् य॒ज्ञिया॑ य॒ज्ञिया॑ त॒नूः । \newline
32. त॒नू स्तया॒ तया॑ त॒नू स्त॒नू स्तया᳚ । \newline
33. तया ऽऽतया॒ तया । \newline
34. एही॒ ह्ये हि॑ । \newline
35. इ॒ह्ये ही॒ ह्या । \newline
36. आ रो॑ह रो॒हा रो॑ह । \newline
37. रो॒हेतीति॑ रोह रो॒हेति॑ । \newline
38. इत्या॒त्मन्-ना॒त्मन्-निती त्या॒त्मन्न् । \newline
39. आ॒त्मन् थ्स॒मारो॑हयते स॒मारो॑हयत आ॒त्मन्-ना॒त्मन् थ्स॒मारो॑हयते । \newline
40. स॒मारो॑हयते॒ यज॑मानो॒ यज॑मानः स॒मारो॑हयते स॒मारो॑हयते॒ यज॑मानः । \newline
41. स॒मारो॑हयत॒ इति॑ सं - आरो॑हयते । \newline
42. यज॑मानो॒ वै वै यज॑मानो॒ यज॑मानो॒ वै । \newline
43. वा अ॒ग्ने र॒ग्नेर् वै वा अ॒ग्नेः । \newline
44. अ॒ग्नेर् योनि॒र् योनि॑ र॒ग्ने र॒ग्नेर् योनिः॑ । \newline
45. योनिः॒ स्वायाꣳ॒॒ स्वायां॒ ॅयोनि॒र् योनिः॒ स्वाया᳚म् । \newline
46. स्वाया॑ मे॒वैव स्वायाꣳ॒॒ स्वाया॑ मे॒व । \newline
47. ए॒वैन॑ मेन मे॒वै वैन᳚म् । \newline
48. ए॒नं॒ ॅयोन्यां॒ ॅयोन्या॑ मेन मेनं॒ ॅयोन्या᳚म् । \newline
49. योन्याꣳ॑ स॒मारो॑हयते स॒मारो॑हयते॒ योन्यां॒ ॅयोन्याꣳ॑ स॒मारो॑हयते । \newline
50. स॒मारो॑हयत॒ इति॑ सं - आरो॑हयते । \newline

\textbf{Ghana Paata } \newline

1. ए॒ष वै वा ए॒ष ए॒ष वा अ॒ग्ने र॒ग्नेर् वा ए॒ष ए॒ष वा अ॒ग्नेः । \newline
2. वा अ॒ग्ने र॒ग्नेर् वै वा अ॒ग्नेर् योनि॒र् योनि॑ र॒ग्नेर् वै वा अ॒ग्नेर् योनिः॑ । \newline
3. अ॒ग्नेर् योनि॒र् योनि॑ र॒ग्ने र॒ग्नेर् योनिः॒ स्वे स्वे योनि॑ र॒ग्ने र॒ग्नेर् योनिः॒ स्वे । \newline
4. योनिः॒ स्वे स्वे योनि॒र् योनिः॒ स्व ए॒वैव स्वे योनि॒र् योनिः॒ स्व ए॒व । \newline
5. स्व ए॒वैव स्वे स्व ए॒वैन॑ मेन मे॒व स्वे स्व ए॒वैन᳚म् । \newline
6. ए॒वैन॑ मेन मे॒वैवैन॒म् ॅयोनौ॒ योना॑ वेन मे॒वैवैन॒म् ॅयोनौ᳚ । \newline
7. ए॒न॒म् ॅयोनौ॒ योना॑ वेन मेन॒म् ॅयोनौ॑ स॒मारो॑हयति स॒मारो॑हयति॒ योना॑ वेन मेन॒म् ॅयोनौ॑ स॒मारो॑हयति । \newline
8. योनौ॑ स॒मारो॑हयति स॒मारो॑हयति॒ योनौ॒ योनौ॑ स॒मारो॑हय॒ त्यथो॒ अथो॑ स॒मारो॑हयति॒ योनौ॒ योनौ॑ स॒मारो॑हय॒ त्यथो᳚ । \newline
9. स॒मारो॑हय॒त्यथो॒ अथो॑ स॒मारो॑हयति स॒मारो॑हय॒ त्यथो॒ खलु॒ खल् वथो॑ स॒मारो॑हयति स॒मारो॑हय॒
त्यथो॒ खलु॑ । \newline
10. स॒मारो॑हय॒तीति॑ सम् - आरो॑हयति । \newline
11. अथो॒ खलु॒ खल् वथो॒ अथो॒ खल्वा॑हु राहुः॒ खल्वथो॒ अथो॒ खल्वा॑हुः । \newline
12. अथो॒ इत्यथो᳚ । \newline
13. खल्वा॑हु राहुः॒ खलु॒ खल्वा॑ हु॒र् यद् यदा॑हुः॒ खलु॒ खल्वा॑ हु॒र् यत् । \newline
14. आ॒हु॒र् यद् यदा॑हु राहु॒र् यद॒रण्यो॑ र॒रण्यो॒र् यदा॑हु राहु॒र् यद॒रण्योः᳚ । \newline
15. यद॒रण्यो॑ र॒रण्यो॒र् यद् यद॒रण्योः᳚ स॒मारू॑ढः स॒मारू॑ढो॒ ऽरण्यो॒र् यद् यद॒रण्योः᳚ स॒मारू॑ढः । \newline
16. अ॒रण्योः᳚ स॒मारू॑ढः स॒मारू॑ढो॒ ऽरण्यो॑ र॒रण्योः᳚ स॒मारू॑ढो॒ नश्ये॒न् नश्ये᳚थ् स॒मारू॑ढो॒ ऽरण्यो॑ र॒रण्योः᳚ स॒मारू॑ढो॒ नश्ये᳚त् । \newline
17. स॒मारू॑ढो॒ नश्ये॒न् नश्ये᳚थ् स॒मारू॑ढः स॒मारू॑ढो॒ नश्ये॒ दुदुन्,नश्ये᳚थ् स॒मारू॑ढः स॒मारू॑ढो॒ नश्ये॒दुत् । \newline
18. स॒मारू॑ढ॒ इति॑ सम् - आरू॑ढः । \newline
19. नश्ये॒ दुदुन् नश्ये॒न् नश्ये॒ दुद॑स्या ॒स्योन् नश्ये॒न् नश्ये॒ दुद॑स्य । \newline
20. उद॑स्या॒ स्योदुद॑स्या॒ ग्नि र॒ग्नि र॒स्योदु द॑स्या॒ ग्निः । \newline
21. अ॒स्या॒ ग्नि र॒ग्नि र॑स्या स्या॒ग्निः सी॑देथ् सीदे द॒ग्नि र॑स्या स्या॒ग्निः सी॑देत् । \newline
22. अ॒ग्निः सी॑देथ् सीदे द॒ग्नि र॒ग्निः सी॑देत् पुनरा॒धेयः॑ पुनरा॒धेयः॑ सीदे द॒ग्नि र॒ग्निः सी॑देत् पुनरा॒धेयः॑ । \newline
23. सी॒दे॒त् पु॒न॒रा॒धेयः॑ पुनरा॒धेयः॑ सीदेथ् सीदेत् पुनरा॒धेयः॑ स्याथ् स्यात् पुनरा॒धेयः॑ सीदेथ् सीदेत् पुनरा॒धेयः॑ स्यात् । \newline
24. पु॒न॒रा॒धेयः॑ स्याथ् स्यात् पुनरा॒धेयः॑ पुनरा॒धेयः॑ स्या॒दितीति॑ स्यात् पुनरा॒धेयः॑ पुनरा॒धेयः॑ स्या॒दिति॑ । \newline
25. पु॒न॒रा॒धेय॒ इति॑ पुनः - आ॒धेयः॑ । \newline
26. स्या॒ दितीति॑ स्याथ् स्या॒दिति॒ या येति॑ स्याथ् स्या॒दिति॒ या । \newline
27. इति॒ या येतीति॒ या ते॑ ते॒ येतिति॒ या ते᳚ । \newline
28. या ते॑ ते॒ या या ते॑ अग्ने ऽग्ने ते॒ या या ते॑ अग्ने । \newline
29. ते॒ अ॒ग्ने॒ ऽग्ने॒ ते॒ ते॒ अ॒ग्ने॒ य॒ज्ञिया॑ य॒ज्ञिया᳚ ऽग्ने ते ते अग्ने य॒ज्ञिया᳚ । \newline
30. अ॒ग्ने॒ य॒ज्ञिया॑ य॒ज्ञिया᳚ ऽग्ने ऽग्ने य॒ज्ञिया॑ त॒नू स्त॒नूर् य॒ज्ञिया᳚ ऽग्ने ऽग्ने य॒ज्ञिया॑ त॒नूः । \newline
31. य॒ज्ञिया॑ त॒नू स्त॒नूर् य॒ज्ञिया॑ य॒ज्ञिया॑ त॒नू स्तया॒ तया॑ त॒नूर् य॒ज्ञिया॑ य॒ज्ञिया॑ त॒नू स्तया᳚ । \newline
32. त॒नू स्तया॒ तया॑ त॒नू स्त॒नू स्तया ऽऽतया॑ त॒नू स्त॒नू स्तया । \newline
33. तया ऽऽतया॒ तयेही॒ ह्या तया॒ तयेहि॑ । \newline
34. एही॒ ह्ये ह्ये ह्येह्या । \newline
35. इ॒ह्ये ही॒ह्या रो॑ह रो॒हे ही॒ह्या रो॑ह । \newline
36. आ रो॑ह रो॒हा रो॒हेतीति॑ रो॒हा रो॒हेति॑ । \newline
37. रो॒हेतीति॑ रोह रो॒हे त्या॒त्मन्,ना॒त्मन्,निति॑ रोह रो॒हे त्या॒त्मन्न् । \newline
38. इत्या॒त्मन्,ना॒त्मन्,नितीत्या॒त्मन् थ्स॒मारो॑हयते स॒मारो॑हयत आ॒त्मन्निती त्या॒त्मन् थ्स॒मारो॑हयते । \newline
39. आ॒त्मन् थ्स॒मारो॑हयते स॒मारो॑हयत आ॒त्मन्,ना॒त्मन् थ्स॒मारो॑हयते॒ यज॑मानो॒ यज॑मानः स॒मारो॑हयत आ॒त्मन्,ना॒त्मन् थ्स॒मारो॑हयते॒ यज॑मानः । \newline
40. स॒मारो॑हयते॒ यज॑मानो॒ यज॑मानः स॒मारो॑हयते स॒मारो॑हयते॒ यज॑मानो॒ वै वै यज॑मानः स॒मारो॑हयते स॒मारो॑हयते॒ यज॑मानो॒ वै । \newline
41. स॒मारो॑हयत॒ इति॑ सम् - आरो॑हयते । \newline
42. यज॑मानो॒ वै वै यज॑मानो॒ यज॑मानो॒ वा अ॒ग्ने र॒ग्नेर् वै यज॑मानो॒ यज॑मानो॒ वा अ॒ग्नेः । \newline
43. वा अ॒ग्ने र॒ग्नेर् वै वा अ॒ग्नेर् योनि॒र् योनि॑ र॒ग्नेर् वै वा अ॒ग्नेर् योनिः॑ । \newline
44. अ॒ग्नेर् योनि॒र् योनि॑ र॒ग्ने र॒ग्नेर् योनिः॒ स्वायाꣳ॒॒ स्वाया॒म् ॅयोनि॑ र॒ग्ने र॒ग्नेर् योनिः॒ स्वाया᳚म् । \newline
45. योनिः॒ स्वायाꣳ॒॒ स्वाया॒म् ॅयोनि॒र् योनिः॒ स्वाया॑ मे॒वैव स्वाया॒म् ॅयोनि॒र् योनिः॒ स्वाया॑ मे॒व । \newline
46. स्वाया॑ मे॒वैव स्वायाꣳ॒॒ स्वाया॑ मे॒वैन॑ मेन मे॒व स्वायाꣳ॒॒ स्वाया॑ मे॒वैन᳚म् । \newline
47. ए॒वैन॑ मेन मे॒वैवैन॒म् ॅयोन्या॒म् ॅयोन्या॑ मेन मे॒वैवैन॒म् ॅयोन्या᳚म् । \newline
48. ए॒न॒म् ॅयोन्या॒म् ॅयोन्या॑ मेन मेन॒म् ॅयोन्याꣳ॑ स॒मारो॑हयते स॒मारो॑हयते॒ योन्या॑ मेन मेन॒म् 
ॅयोन्याꣳ॑ स॒मारो॑हयते । \newline
49. योन्याꣳ॑ स॒मारो॑हयते स॒मारो॑हयते॒ योन्या॒म् ॅयोन्याꣳ॑ स॒मारो॑हयते । \newline
50. स॒मारो॑हयत॒ इति॑ सम् - आरो॑हयते । \newline
\pagebreak
\markright{ TS 3.4.11.1  \hfill https://www.vedavms.in \hfill}

\section{ TS 3.4.11.1 }

\textbf{TS 3.4.11.1 } \newline
\textbf{Samhita Paata} \newline

त्वम॑ग्ने बृ॒हद्वयो॒ दधा॑सि देव दा॒शुषे᳚ । क॒विर्गृ॒हप॑ति॒र्युवा᳚ ॥ह॒व्य॒वाड॒ग्निर॒जरः॑ पि॒ता नो॑ वि॒भुर्वि॒भावा॑ सु॒दृशी॑को अ॒स्मे । सु॒गा॒र्॒.ह॒प॒त्याः समिषो॑ दिदीह्यस्म॒द्रिय॒ख्सं मि॑मीहि॒ श्रवाꣳ॑सि ॥ त्वं च॑ सोम नो॒ वशो॑ जी॒वातुं॒ न म॑रामहे । प्रि॒यस्तो᳚त्रो॒ वन॒स्पतिः॑ ॥ ब्र॒ह्मा दे॒वानां᳚ पद॒वीः क॑वी॒नामृषि॒र्विप्रा॑णां महि॒षो मृ॒गाणां᳚ । श्ये॒नो गृद्ध्रा॑णाꣳ॒॒ स्वधि॑ति॒ र्वना॑नाꣳ॒॒ सोमः॑ - [  ] \newline

\textbf{Pada Paata} \newline

त्वम् । अ॒ग्ने॒ । बृ॒हत् । वयः॑ । दधा॑सि । दे॒व॒ । दा॒शुषे᳚ ॥ क॒विः । गृ॒हप॑ति॒रिति॑ गृ॒ह - प॒तिः॒ । युवा᳚ ॥ ह॒व्य॒वाडिति॑ हव्य - वाट् । अ॒ग्निः । अ॒जरः॑ । पि॒ता । नः॒ । वि॒भुरिति॑ वि - भुः । वि॒भावेति॑ वि - भावा᳚ । सु॒दृशी॑क॒ इति॑ सु - दृशी॑कः । अ॒स्मे इति॑ ॥ सु॒गा॒र्॒.ह॒प॒त्या इति॑ सु - गा॒र्॒.ह॒प॒त्याः । समिति॑ । इषः॑ । दि॒दी॒हि॒ । ॒स्म॒द्रिय॒गित्य॑स्म - द्रिय॑क् । समिति॑ । मि॒मी॒हि॒ । श्रवाꣳ॑सि ॥ त्वम् । च॒ । सो॒म॒ । नः॒ । वशः॑ । जी॒वातु᳚म् । न । म॒रा॒म॒हे॒ ॥ प्रि॒यस्तो᳚त्र॒ इति॑ प्रि॒य - स्तो॒त्रः॒ । वन॒स्पतिः॑ ॥ ब्र॒ह्मा । दे॒वाना᳚म् । प॒द॒वीरिति॑ पद - वीः । क॒वी॒नाम् । ऋषिः॑ । विप्रा॑णाम् । म॒हि॒षः । मृ॒गाणा᳚म् ॥ श्ये॒नः । गृद्ध्रा॑णाम् । स्वधि॑ति॒रिति॒ स्व - धि॒तिः॒ । वना॑नाम् । सोमः॑ ।  \newline


\textbf{Krama Paata} \newline

त्वम॑ग्ने । अ॒ग्ने॒ बृ॒हत् । बृ॒हद् वयः॑ । वयो॒ दधा॑सि । दधा॑सि देव । दे॒व॒ दा॒शुषे᳚ । दा॒शुष॒ इति॑ दा॒शुषे᳚ ॥ क॒विर् गृ॒हप॑तिः । गृ॒हप॑ति॒र् युवा᳚ । गृ॒हप॑ति॒रिति॑ गृ॒ह - प॒तिः॒ । युवेति॒ युवा᳚ ॥ ह॒व्य॒वाड॒ग्निः । ह॒व्य॒वाडिति॑ हव्य - वाट् । अ॒ग्निर॒जरः॑ । अ॒जरः॑ पि॒ता । पि॒ता नः॑ । नो॒ वि॒भुः । वि॒भुर् वि॒भावा᳚ । वि॒भुरिति॑ वि - भुः । वि॒भावा॑ सु॒दृशी॑कः । वि॒भावेति॑ वि - भावा᳚ । सु॒दृशी॑को अ॒स्मे । सु॒दृशी॑क॒ इति॑ सु - दृशी॑कः । अ॒स्मे इत्य॒स्मे ॥ सु॒गा॒र्॒.ह॒प॒त्याः सम् । सु॒गा॒र्॒.ह॒प॒त्या इति॑ सु - गा॒र्॒ह॒प॒त्याः । समिषः॑ । इषो॑ दिदीहि । दि॒दी॒ह्य॒स्म॒द्रिय॑क् । अ॒स्म॒द्रिय॒ख् सम् । अ॒स्म॒द्रिय॒गित्य॑स्म - द्रिय॑क् । सम् मि॑मीहि । मि॒मी॒हि॒ श्रवाꣳ॑सि । श्रवाꣳ॒॒सीति॒ श्रवाꣳ॑सि ॥ त्वम् च॑ । च॒ सो॒म॒ । सो॒म॒ नः॒ । नो॒ वशः॑ । वशो॑ जी॒वातु᳚म् । जी॒वातु॒म् न । न म॑रामहे । म॒रा॒म॒ह॒ इति॑ मरामहे ॥ प्रि॒यस्तो᳚त्रो॒ वन॒स्पतिः॑ । प्रि॒यस्तो᳚त्र॒ इति॑ प्रि॒य - स्तो॒त्रः॒ । वन॒स्पति॒रिति॒ वन॒स्पतिः॑ ॥ ब्र॒ह्मा दे॒वाना᳚म् । दे॒वाना᳚म् पद॒वीः । प॒द॒वीः क॑वी॒नाम् । प॒द॒वीरिति॑ पद - वीः । क॒वी॒नामृषिः॑ । ऋषि॒र् विप्रा॑णाम् । विप्रा॑णाम् महि॒षः । म॒हि॒षो मृ॒गाणा᳚म् । मृ॒गाणा॒मिति॑ मृ॒गाणा᳚म् ॥ श्ये॒नो गृद्ध्रा॑णाम् । गृद्ध्रा॑णाꣳ॒॒ स्वधि॑तिः । स्वधि॑ति॒र् वना॑नाम् । स्वधि॑ति॒रिति॒ स्व - धि॒तिः॒ । वना॑नाꣳ॒॒ सोमः॑ । सोमः॑ प॒वित्र᳚म् \newline

\textbf{Jatai Paata} \newline

1. त्व म॑ग्ने अग्ने॒ त्वम् त्व म॑ग्ने । \newline
2. अ॒ग्ने॒ बृ॒हद् बृ॒ह द॑ग्ने अग्ने बृ॒हत् । \newline
3. बृ॒हद् वयो॒ वयो॑ बृ॒हद् बृ॒हद् वयः॑ । \newline
4. वयो॒ दधा॑सि॒ दधा॑सि॒ वयो॒ वयो॒ दधा॑सि । \newline
5. दधा॑सि देव देव॒ दधा॑सि॒ दधा॑सि देव । \newline
6. दे॒व॒ दा॒शुषे॑ दा॒शुषे॑ देव देव दा॒शुषे᳚ । \newline
7. दा॒शुष॒ इति॑ दा॒शुषे᳚ । \newline
8. क॒विर् गृ॒हप॑तिर् गृ॒हप॑तिः क॒विः क॒विर् गृ॒हप॑तिः । \newline
9. गृ॒हप॑ति॒र् युवा॒ युवा॑ गृ॒हप॑तिर् गृ॒हप॑ति॒र् युवा᳚ । \newline
10. गृ॒हप॑ति॒रिति॑ गृ॒ह - प॒तिः॒ । \newline
11. युवेति॒ युवा᳚ । \newline
12. ह॒व्य॒वा ड॒ग्नि र॒ग्निर्. ह॑व्य॒वा ड्ढ॑व्य॒वा ड॒ग्निः । \newline
13. ह॒व्य॒वाडिति॑ हव्य - वाट् । \newline
14. अ॒ग्नि र॒जरो॑ अ॒जरो॑ अ॒ग्नि र॒ग्नि र॒जरः॑ । \newline
15. अ॒जरः॑ पि॒ता पि॒ता ऽजरो॑ अ॒जरः॑ पि॒ता । \newline
16. पि॒ता नो॑ नः पि॒ता पि॒ता नः॑ । \newline
17. नो॒ वि॒भुर् वि॒भुर् नो॑ नो वि॒भुः । \newline
18. वि॒भुर् वि॒भावा॑ वि॒भावा॑ वि॒भुर् वि॒भुर् वि॒भावा᳚ । \newline
19. वि॒भुरिति॑ वि - भुः । \newline
20. वि॒भावा॑ सु॒दृशी॑कः सु॒दृशी॑को वि॒भावा॑ वि॒भावा॑ सु॒दृशी॑कः । \newline
21. वि॒भावेति॑ वि - भावा᳚ । \newline
22. सु॒दृशी॑को अ॒स्मे अ॒स्मे सु॒दृशी॑कः सु॒दृशी॑को अ॒स्मे । \newline
23. सु॒दृशी॑क॒ इति॑ सु - दृशी॑कः । \newline
24. अ॒स्मे इत्य॒स्मे । \newline
25. सु॒गा॒र्॒.ह॒प॒त्याः सꣳ सꣳ सु॑गार्.हप॒त्याः सु॑गार्.हप॒त्याः सम् । \newline
26. सु॒गा॒र्॒.ह॒प॒त्या इति॑ सु - गा॒र्॒.ह॒प॒त्याः । \newline
27. समिष॒ इषः॒ सꣳ समिषः॑ । \newline
28. इषो॑ दिदीहि दिदी॒हीष॒ इषो॑ दिदीहि । \newline
29. दि॒दी॒ ह्य॒स्म॒द्रिय॑ गस्म॒द्रिय॑ग् दिदीहि दिदी ह्यस्म॒द्रिय॑क् । \newline
30. अ॒स्म॒द्रिय॒ख् सꣳ स म॑स्म॒द्रिय॑ गस्म॒द्रिय॒ख् सम् । \newline
31. अ॒स्म॒द्रिय॒गित्य॑स्म - द्रिय॑क् । \newline
32. सम् मि॑मीहि मिमीहि॒ सꣳ सम् मि॑मीहि । \newline
33. मि॒मी॒हि॒ श्रवाꣳ॑सि॒ श्रवाꣳ॑सि मिमीहि मिमीहि॒ श्रवाꣳ॑सि । \newline
34. श्र॒वाꣳ॒॒सीति॒ श्रवाꣳ॑सि । \newline
35. त्वम् च॑ च॒ त्वम् त्वम् च॑ । \newline
36. च॒ सो॒म॒ सो॒म॒ च॒ च॒ सो॒म॒ । \newline
37. सो॒म॒ नो॒ नः॒ सो॒म॒ सो॒म॒ नः॒ । \newline
38. नो॒ वशो॒ वशो॑ नो नो॒ वशः॑ । \newline
39. वशो॑ जी॒वातु॑म् जी॒वातुं॒ ॅवशो॒ वशो॑ जी॒वातु᳚म् । \newline
40. जी॒वातु॒म् न न जी॒वातु॑म् जी॒वातु॒म् न । \newline
41. न म॑रामहे मरामहे॒ न न म॑रामहे । \newline
42. म॒रा॒म॒ह॒ इति॑ मरामहे । \newline
43. प्रि॒यस्तो᳚त्रो॒ वन॒स्पति॒र् वन॒स्पतिः॑ प्रि॒यस्तो᳚त्रः प्रि॒यस्तो᳚त्रो॒ वन॒स्पतिः॑ । \newline
44. प्रि॒यस्तो᳚त्र॒ इति॑ प्रि॒य - स्तो॒त्रः॒ । \newline
45. वन॒स्पति॒रिति॒ वन॒स्पतिः॑ । \newline
46. ब्र॒ह्मा दे॒वाना᳚म् दे॒वाना᳚म् ब्र॒ह्मा ब्र॒ह्मा दे॒वाना᳚म् । \newline
47. दे॒वाना᳚म् पद॒वीः प॑द॒वीर् दे॒वाना᳚म् दे॒वाना᳚म् पद॒वीः । \newline
48. प॒द॒वीः क॑वी॒नाम् क॑वी॒नाम् प॑द॒वीः प॑द॒वीः क॑वी॒नाम् । \newline
49. प॒द॒वीरिति॑ पद - वीः । \newline
50. क॒वी॒ना मृषि॒र्॒. ऋषिः॑ कवी॒नाम् क॑वी॒ना मृषिः॑ । \newline
51. ऋषि॒र् विप्रा॑णां॒ ॅविप्रा॑णा॒ मृषि॒र्॒. ऋषि॒र् विप्रा॑णाम् । \newline
52. विप्रा॑णाम् महि॒षो म॑हि॒षो विप्रा॑णां॒ ॅविप्रा॑णाम् महि॒षः । \newline
53. म॒हि॒षो मृ॒गाणा᳚म् मृ॒गाणा᳚म् महि॒षो म॑हि॒षो मृ॒गाणा᳚म् । \newline
54. मृ॒गाणा॒मिति॑ मृ॒गाणा᳚म् । \newline
55. श्ये॒नो गृद्ध्रा॑णा॒म् गृद्ध्रा॑णाꣳ श्ये॒नः श्ये॒नो गृद्ध्रा॑णाम् । \newline
56. गृद्ध्रा॑णाꣳ॒॒ स्वधि॑तिः॒ स्वधि॑ति॒र् गृद्ध्रा॑णा॒म् गृद्ध्रा॑णाꣳ॒॒ स्वधि॑तिः । \newline
57. स्वधि॑ति॒र् वना॑नां॒ ॅवना॑नाꣳ॒॒ स्वधि॑तिः॒ स्वधि॑ति॒र् वना॑नाम् । \newline
58. स्वधि॑ति॒रिति॒ स्व - धि॒तिः॒ । \newline
59. वना॑नाꣳ॒॒ सोमः॒ सोमो॒ वना॑नां॒ ॅवना॑नाꣳ॒॒ सोमः॑ । \newline
60. सोमः॑ प॒वित्र॑म् प॒वित्रꣳ॒॒ सोमः॒ सोमः॑ प॒वित्र᳚म् । \newline

\textbf{Ghana Paata } \newline

1. त्व म॑ग्ने अग्ने॒ त्वम् त्व म॑ग्ने बृ॒हद् बृ॒ह द॑ग्ने॒ त्वम् त्व म॑ग्ने बृ॒हत् । \newline
2. अ॒ग्ने॒ बृ॒हद् बृ॒ह द॑ग्ने अग्ने बृ॒हद् वयो॒ वयो॑ बृ॒ह द॑ग्ने अग्ने बृ॒हद् वयः॑ । \newline
3. बृ॒हद् वयो॒ वयो॑ बृ॒हद् बृ॒हद् वयो॒ दधा॑सि॒ दधा॑सि॒ वयो॑ बृ॒हद् बृ॒हद् वयो॒ दधा॑सि । \newline
4. वयो॒ दधा॑सि॒ दधा॑सि॒ वयो॒ वयो॒ दधा॑सि देव देव॒ दधा॑सि॒ वयो॒ वयो॒ दधा॑सि देव । \newline
5. दधा॑सि देव देव॒ दधा॑सि॒ दधा॑सि देव दा॒शुषे॑ दा॒शुषे॑ देव॒ दधा॑सि॒ दधा॑सि देव दा॒शुषे᳚ । \newline
6. दे॒व॒ दा॒शुषे॑ दा॒शुषे॑ देव देव दा॒शुषे᳚ । \newline
7. दा॒शुष॒ इति॑ दा॒शुषे᳚ । \newline
8. क॒विर् गृ॒हप॑तिर् गृ॒हप॑तिः क॒विः क॒विर् गृ॒हप॑ति॒र् युवा॒ युवा॑ गृ॒हप॑तिः क॒विः क॒विर् गृ॒हप॑ति॒र् युवा᳚ । \newline
9. गृ॒हप॑ति॒र् युवा॒ युवा॑ गृ॒हप॑तिर् गृ॒हप॑ति॒र् युवा᳚ । \newline
10. गृ॒हप॑ति॒रिति॑ गृ॒ह - प॒तिः॒ । \newline
11. युवेति॒ युवा᳚ । \newline
12. ह॒व्य॒वा ड॒ग्नि र॒ग्निर्. ह॑व्य॒वा ड्ढ॑व्य॒वा ड॒ग्नि र॒जरो॑ अ॒जरो॑ अ॒ग्निर्. ह॑व्य॒वा ड्ढ॑व्य॒वा ड॒ग्नि र॒जरः॑ । \newline
13. ह॒व्य॒वाडिति॑ हव्य - वाट् । \newline
14. अ॒ग्नि र॒जरो॑ अ॒जरो॑ अ॒ग्नि र॒ग्नि र॒जरः॑ पि॒ता पि॒ता ऽजरो॑ अ॒ग्नि र॒ग्नि र॒जरः॑ पि॒ता । \newline
15. अ॒जरः॑ पि॒ता पि॒ता ऽजरो॑ अ॒जरः॑ पि॒ता नो॑ नः पि॒ता ऽजरो॑ अ॒जरः॑ पि॒ता नः॑ । \newline
16. पि॒ता नो॑ नः पि॒ता पि॒ता नो॑ वि॒भुर् वि॒भुर् नः॑ पि॒ता पि॒ता नो॑ वि॒भुः । \newline
17. नो॒ वि॒भुर् वि॒भुर् नो॑ नो वि॒भुर् वि॒भावा॑ वि॒भावा॑ वि॒भुर् नो॑ नो वि॒भुर् वि॒भावा᳚ । \newline
18. वि॒भुर् वि॒भावा॑ वि॒भावा॑ वि॒भुर् वि॒भुर् वि॒भावा॑ सु॒दृशी॑कः सु॒दृशी॑को वि॒भावा॑ वि॒भुर् वि॒भुर् वि॒भावा॑ सु॒दृशी॑कः । \newline
19. वि॒भुरिति॑ वि - भुः । \newline
20. वि॒भावा॑ सु॒दृशी॑कः सु॒दृशी॑को वि॒भावा॑ वि॒भावा॑ सु॒दृशी॑को अ॒स्मे अ॒स्मे सु॒दृशी॑को वि॒भावा॑ वि॒भावा॑ सु॒दृशी॑को अ॒स्मे । \newline
21. वि॒भावेति॑ वि - भावा᳚ । \newline
22. सु॒दृशी॑को अ॒स्मे अ॒स्मे सु॒दृशी॑कः सु॒दृशी॑को अ॒स्मे । \newline
23. सु॒दृशी॑क॒ इति॑ सु - दृशी॑कः । \newline
24. अ॒स्मे इत्य॒स्मे । \newline
25. सु॒गा॒र्॒.ह॒प॒त्याः सꣳ सꣳ सु॑गार्.हप॒त्याः सु॑गार्.हप॒त्याः स मिष॒ इषः॒ सꣳ सु॑गार्.हप॒त्याः सु॑गार्.हप॒त्याः स मिषः॑ । \newline
26. सु॒गा॒र्॒.ह॒प॒त्या इति॑ सु - गा॒र्॒.ह॒प॒त्याः । \newline
27. स मिष॒ इषः॒ सꣳ स मिषो॑ दिदीहि दिदी॒हीषः॒ सꣳ स मिषो॑ दिदीहि । \newline
28. इषो॑ दिदीहि दिदी॒ हीष॒ इषो॑ दिदीह्यस्म॒द्रिय॑ गस्म॒द्रिय॑ग् दिदी॒हीष॒ इषो॑ दिदी ह्यस्म॒द्रिय॑क् । \newline
29. दि॒दी॒ ह्य॒स्म॒द्रिय॑ गस्म॒द्रिय॑ग् दिदीहि दिदी ह्यस्म॒द्रिय॒ख् सꣳ स म॑स्म॒द्रिय॑ग् दिदीहि दिदी ह्यस्म॒द्रिय॒ख् सम् । \newline
30. अ॒स्म॒द्रिय॒ख् सꣳ स म॑स्म॒द्रिय॑ गस्म॒द्रिय॒ख् सम् मि॑मीहि मिमीहि॒ स म॑स्म॒द्रिय॑ गस्म॒द्रिय॒ख् सम् मि॑मीहि । \newline
31. अ॒स्म॒द्रिय॒गित्य॑स्म - द्रिय॑क् । \newline
32. सम् मि॑मीहि मिमीहि॒ सꣳ सम् मि॑मीहि॒ श्रवाꣳ॑सि॒ श्रवाꣳ॑सि मिमीहि॒ सꣳ सम् मि॑मीहि॒ श्रवाꣳ॑सि । \newline
33. मि॒मी॒हि॒ श्रवाꣳ॑सि॒ श्रवाꣳ॑सि मिमीहि मिमीहि॒ श्रवाꣳ॑सि । \newline
34. श्र॒वाꣳ॒॒सीति॒ श्रवाꣳ॑सि । \newline
35. त्वम् च॑ च॒ त्वम् त्वम् च॑ सोम सोम च॒ त्वम् त्वम् च॑ सोम । \newline
36. च॒ सो॒म॒ सो॒म॒ च॒ च॒ सो॒म॒ नो॒ नः॒ सो॒म॒ च॒ च॒ सो॒म॒ नः॒ । \newline
37. सो॒म॒ नो॒ नः॒ सो॒म॒ सो॒म॒ नो॒ वशो॒ वशो॑ नः सोम सोम नो॒ वशः॑ । \newline
38. नो॒ वशो॒ वशो॑ नो नो॒ वशो॑ जी॒वातु॑म् जी॒वातु॒म् ॅवशो॑ नो नो॒ वशो॑ जी॒वातु᳚म् । \newline
39. वशो॑ जी॒वातु॑म् जी॒वातु॒म् ॅवशो॒ वशो॑ जी॒वातु॒म् न न जी॒वातु॒म् ॅवशो॒ वशो॑ जी॒वातु॒म् न । \newline
40. जी॒वातु॒म् न न जी॒वातु॑म् जी॒वातु॒म् न म॑रामहे मरामहे॒ न जी॒वातु॑म् जी॒वातु॒म् न म॑रामहे । \newline
41. न म॑रामहे मरामहे॒ न न म॑रामहे । \newline
42. म॒रा॒म॒ह॒ इति॑ मरामहे । \newline
43. प्रि॒यस्तो᳚त्रो॒ वन॒स्पति॒र् वन॒स्पतिः॑ प्रि॒यस्तो᳚त्रः प्रि॒यस्तो᳚त्रो॒ वन॒स्पतिः॑ । \newline
44. प्रि॒यस्तो᳚त्र॒ इति॑ प्रि॒य - स्तो॒त्रः॒ । \newline
45. वन॒स्पति॒रिति॒ वन॒स्पतिः॑ । \newline
46. ब्र॒ह्मा दे॒वाना᳚म् दे॒वाना᳚म् ब्र॒ह्मा ब्र॒ह्मा दे॒वाना᳚म् पद॒वीः प॑द॒वीर् दे॒वाना᳚म् ब्र॒ह्मा ब्र॒ह्मा दे॒वाना᳚म् पद॒वीः । \newline
47. दे॒वाना᳚म् पद॒वीः प॑द॒वीर् दे॒वाना᳚म् दे॒वाना᳚म् पद॒वीः क॑वी॒नाम् क॑वी॒नाम् प॑द॒वीर् दे॒वाना᳚म् दे॒वाना᳚म् पद॒वीः क॑वी॒नाम् । \newline
48. प॒द॒वीः क॑वी॒नाम् क॑वी॒नाम् प॑द॒वीः प॑द॒वीः क॑वी॒ना मृषि॒र्॒. ऋषिः॑ कवी॒नाम् प॑द॒वीः प॑द॒वीः क॑वी॒ना मृषिः॑ । \newline
49. प॒द॒वीरिति॑ पद - वीः । \newline
50. क॒वी॒ना मृषि॒र्॒. ऋषिः॑ कवी॒नाम् क॑वी॒ना मृषि॒र् विप्रा॑णा॒म् ॅविप्रा॑णा॒ मृषिः॑ कवी॒नाम् क॑वी॒ना मृषि॒र् विप्रा॑णाम् । \newline
51. ऋषि॒र् विप्रा॑णा॒म् ॅविप्रा॑णा॒ मृषि॒र्॒. ऋषि॒र् विप्रा॑णाम् महि॒षो म॑हि॒षो विप्रा॑णा॒ मृषि॒र्॒. ऋषि॒र् विप्रा॑णाम् महि॒षः । \newline
52. विप्रा॑णाम् महि॒षो म॑हि॒षो विप्रा॑णा॒म् ॅविप्रा॑णाम् महि॒षो मृ॒गाणा᳚म् मृ॒गाणा᳚म् महि॒षो विप्रा॑णा॒म् 
ॅविप्रा॑णाम् महि॒षो मृ॒गाणा᳚म् । \newline
53. म॒हि॒षो मृ॒गाणा᳚म् मृ॒गाणा᳚म् महि॒षो म॑हि॒षो मृ॒गाणा᳚म् । \newline
54. मृ॒गाणा॒मिति॑ मृ॒गाणा᳚म् । \newline
55. श्ये॒नो गृद्ध्रा॑णा॒म् गृद्ध्रा॑णाꣳ श्ये॒नः श्ये॒नो गृद्ध्रा॑णाꣳ॒॒ स्वधि॑तिः॒ स्वधि॑ति॒र् गृद्ध्रा॑णाꣳ 
श्ये॒नः श्ये॒नो गृद्ध्रा॑णाꣳ॒॒ स्वधि॑तिः । \newline
56. गृद्ध्रा॑णाꣳ॒॒ स्वधि॑तिः॒ स्वधि॑ति॒र् गृद्ध्रा॑णा॒म् गृद्ध्रा॑णाꣳ॒॒ स्वधि॑ति॒र् वना॑ना॒म् ॅवना॑नाꣳ॒॒ 
स्वधि॑ति॒र् गृद्ध्रा॑णा॒म् गृद्ध्रा॑णाꣳ॒॒ स्वधि॑ति॒र् वना॑नाम् । \newline
57. स्वधि॑ति॒र् वना॑ना॒म् ॅवना॑नाꣳ॒॒ स्वधि॑तिः॒ स्वधि॑ति॒र् वना॑नाꣳ॒॒ सोमः॒ सोमो॒ वना॑नाꣳ॒॒ स्वधि॑तिः॒ स्वधि॑ति॒र् वना॑नाꣳ॒॒ सोमः॑ । \newline
58. स्वधि॑ति॒रिति॒ स्व - धि॒तिः॒ । \newline
59. वना॑नाꣳ॒॒ सोमः॒ सोमो॒ वना॑ना॒म् ॅवना॑नाꣳ॒॒ सोमः॑ प॒वित्र॑म् प॒वित्रꣳ॒॒ सोमो॒ वना॑ना॒म् 
ॅवना॑नाꣳ॒॒ सोमः॑ प॒वित्र᳚म् । \newline
60. सोमः॑ प॒वित्र॑म् प॒वित्रꣳ॒॒ सोमः॒ सोमः॑ प॒वित्र॒ मत्यति॑ प॒वित्रꣳ॒॒ सोमः॒ सोमः॑ प॒वित्र॒ मति॑ । \newline
\pagebreak
\markright{ TS 3.4.11.2  \hfill https://www.vedavms.in \hfill}

\section{ TS 3.4.11.2 }

\textbf{TS 3.4.11.2 } \newline
\textbf{Samhita Paata} \newline

प॒वित्र॒मत्ये॑ति॒ रेभन्न्॑ ॥ आ वि॒श्वदे॑वꣳ॒॒ सत्प॑तिꣳ सू॒क्तैर॒द्या वृ॑णीमहे । स॒त्यस॑वꣳ सवि॒तारं᳚ ॥ आस॒त्येन॒ रज॑सा॒ वर्त॑मानो निवे॒शय॑न्न॒मृतं॒ मर्त्य॑ञ्च । हि॒र॒ण्यये॑न सवि॒ता रथे॒नाऽऽ* दे॒वोया॑ति॒ भुव॑ना वि॒पश्यन्न्॑ ॥ यथा॑ नो॒ अदि॑तिः॒ कर॒त् पश्वे॒ नृभ्यो॒ यथा॒ गवे᳚ । यथा॑ तो॒काय॑ रु॒द्रियं᳚ ॥ मा न॑स्तो॒के तन॑ये॒ मा न॒ आयु॑षि॒ मा नो॒ गोषु॒ मा- [  ] \newline

\textbf{Pada Paata} \newline

प॒वित्र᳚म् । अतीति॑ । ए॒ति॒ । रेभन्न्॑ ॥ एति॑ । वि॒श्वदे॑व॒मिति॑ वि॒श्व - दे॒व॒म् । सत्प॑ति॒मिति॒ सत् - प॒ति॒म् । सू॒क्तैरिति॑ सु-उ॒क्तैः । अ॒द्य । वृ॒णी॒म॒हे॒ ॥ स॒त्यस॑व॒मिति॑ स॒त्य-स॒व॒म् । स॒वि॒तार᳚म् ॥ एति॑ । स॒त्येन॑ । रज॑सा । वर्त॑मानः । नि॒वे॒शय॒न्निति॑ नि-वे॒शयन्न्॑ । अ॒मृत᳚म् । मर्त्य᳚म् । च॒ ॥ हि॒र॒ण्यये॑न । स॒वि॒ता । रथे॑न । एति॑ । दे॒वः । या॒ति॒ । भुव॑ना । वि॒पश्य॒न्निति॑ वि-पश्यन्न्॑ ॥ यथा᳚ । नः॒ । अदि॑तिः । कर॑त् । पश्वे᳚ । नृभ्य॒ इति॒ नृ - भ्यः॒ । यथा᳚ । गवे᳚ ॥ यथा᳚ । तो॒काय॑ । रु॒द्रिय᳚म् ॥ मा । नः॒ । तो॒के । तन॑ये । मा । नः॒ । आयु॑षि । मा । नः॒ । गोषु॑ । मा ।  \newline


\textbf{Krama Paata} \newline

प॒वित्र॒मति॑ । अत्ये॑ति । ए॒ति॒ रेभन्न्॑ । रेभ॒न्निति॒ रेभन्न्॑ ॥ आ वि॒श्वदे॑वम् । वि॒श्वदे॑वꣳ॒॒ सत्प॑तिम् । वि॒श्वदे॑व॒मिति॑ वि॒श्व - दे॒व॒म् । सत्प॑तिꣳ सू॒क्तैः । सत्प॑ति॒मिति॒ सत् - प॒ति॒म् । सू॒क्तैर॒द्य । सू॒क्तैरिति॑ सु - उ॒क्तैः । अ॒द्या वृ॑णीमहे । वृ॒णी॒म॒ह॒ इति॑ वृणीमहे ॥ स॒त्यस॑वꣳ सवि॒तार᳚म् । स॒त्यस॑व॒मिति॑ स॒त्य - स॒व॒म् । स॒वि॒तार॒मिति॑ सवि॒तार᳚म् ॥ आ स॒त्येन॑ । स॒त्येन॒ रज॑सा । रज॑सा॒ वर्त॑मानः । वर्त॑मानो निवे॒शयन्न्॑ । नि॒वे॒शय॑न्न॒मृत᳚म् । नि॒वे॒शय॒न्निति॑ नि - वे॒शयन्न्॑ । अ॒मृत॒म् मर्त्य᳚म् । मर्त्य॑म् च । चेति॑ च ॥ हि॒र॒ण्यये॑न सवि॒ता । स॒वि॒ता रथे॑न । रथे॒ना । आ दे॒वः । दे॒वो या॑ति । या॒ति॒ भुव॑ना । भुव॑ना वि॒पश्यन्न्॑ । वि॒पश्य॒न्निति॑ वि - पश्यन्न्॑ ॥ यथा॑ नः । नो॒ अदि॑तिः । अदि॑तिः॒ कर॑त् । कर॒त् पश्वे᳚ । पश्वे॒ नृभ्यः॑ । नृभ्यो॒ यथा᳚ । नृभ्य॒ इति॒ नृ - भ्यः॒ । यथा॒ गवे᳚ । गव॒ इति॒ गवे᳚ ॥ यथा॑ तो॒काय॑ । तो॒काय॑ रु॒द्रिय᳚म् । रु॒द्रिय॒मिति॑ रु॒द्रिय᳚म् ॥ मा नः॑ । न॒ स्तो॒के । तो॒के तन॑ये । तन॑ये॒ मा । मा नः॑ । न॒ आयु॑षि । आयु॑षि॒ मा । मा नः॑ । नो॒ गोषु॑ । गोषु॒ मा । मा नः॑ \newline

\textbf{Jatai Paata} \newline

1. प॒वित्र॒ मत्यति॑ प॒वित्र॑म् प॒वित्र॒ मति॑ । \newline
2. अत्ये᳚ त्ये॒ त्यत्य त्ये॑ति । \newline
3. ए॒ति॒ रेभ॒न् रेभ॑न्-नेत्येति॒ रेभन्न्॑ । \newline
4. रेभ॒न्निति॒ रेभन्न्॑ । \newline
5. आ वि॒श्वदे॑वं ॅवि॒श्वदे॑व॒मा वि॒श्वदे॑वम् । \newline
6. वि॒श्वदे॑वꣳ॒॒ सत्प॑तिꣳ॒॒ सत्प॑तिं ॅवि॒श्वदे॑वं ॅवि॒श्वदे॑वꣳ॒॒ सत्प॑तिम् । \newline
7. वि॒श्वदे॑व॒मिति॑ वि॒श्व - दे॒व॒म् । \newline
8. सत्प॑तिꣳ सू॒क्तैः सू॒क्तैः सत्प॑तिꣳ॒॒ सत्प॑तिꣳ सू॒क्तैः । \newline
9. सत्प॑ति॒मिति॒ सत् - प॒ति॒म् । \newline
10. सू॒क्तै र॒द्याद्य सू॒क्तैः सू॒क्तै र॒द्य । \newline
11. सू॒क्तैरिति॑ सु - उ॒क्तैः । \newline
12. अ॒द्या वृ॑णीमहे वृणीमहे अ॒द्याद्या वृ॑णीमहे । \newline
13. वृ॒णी॒म॒ह॒ इति॑ वृणीमहे । \newline
14. स॒त्यस॑वꣳ सवि॒तारꣳ॑ सवि॒तारꣳ॑ स॒त्यस॑वꣳ स॒त्यस॑वꣳ सवि॒तार᳚म् । \newline
15. स॒त्यस॑व॒मिति॑ स॒त्य - स॒व॒म् । \newline
16. स॒वि॒तार॒मिति॑ सवि॒तार᳚म् । \newline
17. आ स॒त्येन॑ स॒त्येना स॒त्येन॑ । \newline
18. स॒त्येन॒ रज॑सा॒ रज॑सा स॒त्येन॑ स॒त्येन॒ रज॑सा । \newline
19. रज॑सा॒ वर्त॑मानो॒ वर्त॑मानो॒ रज॑सा॒ रज॑सा॒ वर्त॑मानः । \newline
20. वर्त॑मानो निवे॒शय॑न् निवे॒शय॒न्॒. वर्त॑मानो॒ वर्त॑मानो निवे॒शयन्न्॑ । \newline
21. नि॒वे॒शय॑न्-न॒मृत॑ म॒मृत॑म् निवे॒शय॑न् निवे॒शय॑न्-न॒मृत᳚म् । \newline
22. नि॒वे॒शय॒न्निति॑ नि - वे॒शयन्न्॑ । \newline
23. अ॒मृत॒म् मर्त्य॒म् मर्त्य॑ म॒मृत॑ म॒मृत॒म् मर्त्य᳚म् । \newline
24. मर्त्य॑म् च च॒ मर्त्य॒म् मर्त्य॑म् च । \newline
25. चेति॑ च । \newline
26. हि॒र॒ण्यये॑न सवि॒ता स॑वि॒ता हि॑र॒ण्यये॑न हिर॒ण्यये॑न सवि॒ता । \newline
27. स॒वि॒ता रथे॑न॒ रथे॑न सवि॒ता स॑वि॒ता रथे॑न । \newline
28. रथे॒ना रथे॑न॒ रथे॒ना । \newline
29. आ दे॒वो दे॒व आ दे॒वः । \newline
30. दे॒वो या॑ति याति दे॒वो दे॒वो या॑ति । \newline
31. या॒ति॒ भुव॑ना॒ भुव॑ना याति याति॒ भुव॑ना । \newline
32. भुव॑ना वि॒पश्य॑न्. वि॒पश्य॒न् भुव॑ना॒ भुव॑ना वि॒पश्यन्न्॑ । \newline
33. वि॒पश्य॒न्निति॑ वि - पश्यन्न्॑ । \newline
34. यथा॑ नो नो॒ यथा॒ यथा॑ नः । \newline
35. नो॒ अदि॑ति॒ रदि॑तिर् नो नो॒ अदि॑तिः । \newline
36. अदि॑तिः॒ कर॒त् कर॒ ददि॑ति॒ रदि॑तिः॒ कर॑त् । \newline
37. कर॒त् पश्वे॒ पश्वे॒ कर॒त् कर॒त् पश्वे᳚ । \newline
38. पश्वे॒ नृभ्यो॒ नृभ्यः॒ पश्वे॒ पश्वे॒ नृभ्यः॑ । \newline
39. नृभ्यो॒ यथा॒ यथा॒ नृभ्यो॒ नृभ्यो॒ यथा᳚ । \newline
40. नृभ्य॒ इति॒ नृ - भ्यः॒ । \newline
41. यथा॒ गवे॒ गवे॒ यथा॒ यथा॒ गवे᳚ । \newline
42. गव॒ इति॒ गवे᳚ । \newline
43. यथा॑ तो॒काय॑ तो॒काय॒ यथा॒ यथा॑ तो॒काय॑ । \newline
44. तो॒काय॑ रु॒द्रियꣳ॑ रु॒द्रिय॑म् तो॒काय॑ तो॒काय॑ रु॒द्रिय᳚म् । \newline
45. रु॒द्रिय॒म् इति॑ रु॒द्रिय᳚म् । \newline
46. मा नो॑ नो॒ मा मा नः॑ । \newline
47. न॒ स्तो॒के तो॒के नो॑ न स्तो॒के । \newline
48. तो॒के तन॑ये॒ तन॑ये तो॒के तो॒के तन॑ये । \newline
49. तन॑ये॒ मा मा तन॑ये॒ तन॑ये॒ मा । \newline
50. मा नो॑ नो॒ मा मा नः॑ । \newline
51. न॒ आयु॒ ष्यायु॑षि नो न॒ आयु॑षि । \newline
52. आयु॑षि॒ मा मा ऽऽयु॒ष्या यु॑षि॒ मा । \newline
53. मा नो॑ नो॒ मा मा नः॑ । \newline
54. नो॒ गोषु॒ गोषु॑ नो नो॒ गोषु॑ । \newline
55. गोषु॒ मा मा गोषु॒ गोषु॒ मा । \newline
56. मा नो॑ नो॒ मा मा नः॑ । \newline

\textbf{Ghana Paata } \newline

1. प॒वित्र॒ मत्यति॑ प॒वित्र॑म् प॒वित्र॒ मत्ये᳚ त्ये॒त्यति॑ प॒वित्र॑म् प॒वित्र॒ मत्ये॑ति । \newline
2. अत्ये᳚ त्ये॒त्य त्यत्ये॑ति॒ रेभ॒न् रेभ॑न्,ने॒त्य त्यत्ये॑ति॒ रेभन्न्॑ । \newline
3. ए॒ति॒ रेभ॒न् रेभ॑न्,नेत्येति॒ रेभन्न्॑ । \newline
4. रेभ॒न्निति॒ रेभन्न्॑ । \newline
5. आ वि॒श्वदे॑वम् ॅवि॒श्वदे॑व॒ मा वि॒श्वदे॑वꣳ॒॒ सत्प॑तिꣳ॒॒ सत्प॑तिम् ॅवि॒श्वदे॑व॒ मा वि॒श्वदे॑वꣳ॒॒ सत्प॑तिम् । \newline
6. वि॒श्वदे॑वꣳ॒॒ सत्प॑तिꣳ॒॒ सत्प॑तिम् ॅवि॒श्वदे॑वम् ॅवि॒श्वदे॑वꣳ॒॒ सत्प॑तिꣳ सू॒क्तैः सू॒क्तैः सत्प॑तिम् ॅवि॒श्वदे॑वम् ॅवि॒श्वदे॑वꣳ॒॒ सत्प॑तिꣳ सू॒क्तैः । \newline
7. वि॒श्वदे॑व॒मिति॑ वि॒श्व - दे॒व॒म् । \newline
8. सत्प॑तिꣳ सू॒क्तैः सू॒क्तैः सत्प॑तिꣳ॒॒ सत्प॑तिꣳ सू॒क्तै र॒द्याद्य सू॒क्तैः सत्प॑तिꣳ॒॒ सत्प॑तिꣳ सू॒क्तै र॒द्य । \newline
9. सत्प॑ति॒मिति॒ सत् - प॒ति॒म् । \newline
10. सू॒क्तै र॒द्याद्य सू॒क्तैः सू॒क्तै र॒द्या वृ॑णीमहे वृणीमहे अ॒द्य सू॒क्तैः सू॒क्तै र॒द्या वृ॑णीमहे । \newline
11. सू॒क्तैरिति॑ सु - उ॒क्तैः । \newline
12. अ॒द्या वृ॑णीमहे वृणीमहे अ॒द्याद्या वृ॑णीमहे । \newline
13. वृ॒णी॒म॒ह॒ इति॑ वृणीमहे । \newline
14. स॒त्यस॑वꣳ सवि॒तारꣳ॑ सवि॒तारꣳ॑ स॒त्यस॑वꣳ स॒त्यस॑वꣳ सवि॒तार᳚म् । \newline
15. स॒त्यस॑व॒मिति॑ स॒त्य - स॒व॒म् । \newline
16. स॒वि॒तार॒मिति॑ सवि॒तार᳚म् । \newline
17. आ स॒त्येन॑ स॒त्येना स॒त्येन॒ रज॑सा॒ रज॑सा स॒त्येना स॒त्येन॒ रज॑सा । \newline
18. स॒त्येन॒ रज॑सा॒ रज॑सा स॒त्येन॑ स॒त्येन॒ रज॑सा॒ वर्त॑मानो॒ वर्त॑मानो॒ रज॑सा स॒त्येन॑ स॒त्येन॒ रज॑सा॒ वर्त॑मानः । \newline
19. रज॑सा॒ वर्त॑मानो॒ वर्त॑मानो॒ रज॑सा॒ रज॑सा॒ वर्त॑मानो निवे॒शय॑न् निवे॒शय॒न्॒. वर्त॑मानो॒ रज॑सा॒ 
रज॑सा॒ वर्त॑मानो निवे॒शयन्न्॑ । \newline
20. वर्त॑मानो निवे॒शय॑न् निवे॒शय॒न्॒. वर्त॑मानो॒ वर्त॑मानो निवे॒शय॑न्,न॒मृत॑ म॒मृत॑म् निवे॒शय॒न्॒. वर्त॑मानो॒ वर्त॑मानो निवे॒शय॑न्,न॒मृत᳚म् । \newline
21. नि॒वे॒शय॑न्,न॒मृत॑ म॒मृत॑म् निवे॒शय॑न् निवे॒शय॑न्,न॒मृत॒म् मर्त्य॒म् मर्त्य॑ म॒मृत॑म् निवे॒शय॑न् निवे॒शय॑न्,न॒मृत॒म् मर्त्य᳚म् । \newline
22. नि॒वे॒शय॒न्निति॑ नि - वे॒शयन्न्॑ । \newline
23. अ॒मृत॒म् मर्त्य॒म् मर्त्य॑ म॒मृत॑ म॒मृत॒म् मर्त्य॑म् च च॒ मर्त्य॑ म॒मृत॑ म॒मृत॒म् मर्त्य॑म् च । \newline
24. मर्त्य॑म् च च॒ मर्त्य॒म् मर्त्य॑म् च । \newline
25. चेति॑ च । \newline
26. हि॒र॒ण्यये॑न सवि॒ता स॑वि॒ता हि॑र॒ण्यये॑न हिर॒ण्यये॑न सवि॒ता रथे॑न॒ रथे॑न सवि॒ता हि॑र॒ण्यये॑न हिर॒ण्यये॑न सवि॒ता रथे॑न । \newline
27. स॒वि॒ता रथे॑न॒ रथे॑न सवि॒ता स॑वि॒ता रथे॒ना रथे॑न सवि॒ता स॑वि॒ता रथे॒ना । \newline
28. रथे॒ना रथे॑न॒ रथे॒ना दे॒वो दे॒व आ रथे॑न॒ रथे॒ना दे॒वः । \newline
29. आ दे॒वो दे॒व आ दे॒वो या॑ति याति दे॒व आ दे॒वो या॑ति । \newline
30. दे॒वो या॑ति याति दे॒वो दे॒वो या॑ति॒ भुव॑ना॒ भुव॑ना याति दे॒वो दे॒वो या॑ति॒ भुव॑ना । \newline
31. या॒ति॒ भुव॑ना॒ भुव॑ना याति याति॒ भुव॑ना वि॒पश्य॑न्. वि॒पश्य॒न् भुव॑ना याति याति॒ भुव॑ना वि॒पश्यन्न्॑ । \newline
32. भुव॑ना वि॒पश्य॑न्. वि॒पश्य॒न् भुव॑ना॒ भुव॑ना वि॒पश्यन्न्॑ । \newline
33. वि॒पश्य॒न्निति॑ वि - पश्यन्न्॑ । \newline
34. यथा॑ नो नो॒ यथा॒ यथा॑ नो॒ अदि॑ति॒ रदि॑तिर् नो॒ यथा॒ यथा॑ नो॒ अदि॑तिः । \newline
35. नो॒ अदि॑ति॒ रदि॑तिर् नो नो॒ अदि॑तिः॒ कर॒त् कर॒ ददि॑तिर् नो नो॒ अदि॑तिः॒ कर॑त् । \newline
36. अदि॑तिः॒ कर॒त् कर॒ ददि॑ति॒ रदि॑तिः॒ कर॒त् पश्वे॒ पश्वे॒ कर॒ ददि॑ति॒ रदि॑तिः॒ कर॒त् पश्वे᳚ । \newline
37. कर॒त् पश्वे॒ पश्वे॒ कर॒त् कर॒त् पश्वे॒ नृभ्यो॒ नृभ्यः॒ पश्वे॒ कर॒त् कर॒त् पश्वे॒ नृभ्यः॑ । \newline
38. पश्वे॒ नृभ्यो॒ नृभ्यः॒ पश्वे॒ पश्वे॒ नृभ्यो॒ यथा॒ यथा॒ नृभ्यः॒ पश्वे॒ पश्वे॒ नृभ्यो॒ यथा᳚ । \newline
39. नृभ्यो॒ यथा॒ यथा॒ नृभ्यो॒ नृभ्यो॒ यथा॒ गवे॒ गवे॒ यथा॒ नृभ्यो॒ नृभ्यो॒ यथा॒ गवे᳚ । \newline
40. नृभ्य॒ इति॒ नृ - भ्यः॒ । \newline
41. यथा॒ गवे॒ गवे॒ यथा॒ यथा॒ गवे᳚ । \newline
42. गव॒ इति॒ गवे᳚ । \newline
43. यथा॑ तो॒काय॑ तो॒काय॒ यथा॒ यथा॑ तो॒काय॑ रु॒द्रियꣳ॑ रु॒द्रिय॑म् तो॒काय॒ यथा॒ यथा॑ तो॒काय॑ रु॒द्रिय᳚म् । \newline
44. तो॒काय॑ रु॒द्रियꣳ॑ रु॒द्रिय॑म् तो॒काय॑ तो॒काय॑ रु॒द्रिय᳚म् । \newline
45. रु॒द्रिय॒म् इति॑ रु॒द्रिय᳚म् । \newline
46. मा नो॑ नो॒ मा मा न॑ स्तो॒के तो॒के नो॒ मा मा न॑ स्तो॒के । \newline
47. न॒ स्तो॒के तो॒के नो॑ न स्तो॒के तन॑ये॒ तन॑ये तो॒के नो॑ न स्तो॒के तन॑ये । \newline
48. तो॒के तन॑ये॒ तन॑ये तो॒के तो॒के तन॑ये॒ मा मा तन॑ये तो॒के तो॒के तन॑ये॒ मा । \newline
49. तन॑ये॒ मा मा तन॑ये॒ तन॑ये॒ मा नो॑ नो॒ मा तन॑ये॒ तन॑ये॒ मा नः॑ । \newline
50. मा नो॑ नो॒ मा मा न॒ आयु॒ ष्यायु॑षि नो॒ मा मा न॒ आयु॑षि । \newline
51. न॒ आयु॒ ष्यायु॑षि नो न॒ आयु॑षि॒ मा मा ऽऽयु॑षि नो न॒ आयु॑षि॒ मा । \newline
52. आयु॑षि॒ मा मा ऽऽयु॒ ष्यायु॑षि॒ मा नो॑ नो॒ मा ऽऽयु॒ ष्यायु॑षि॒ मा नः॑ । \newline
53. मा नो॑ नो॒ मा मा नो॒ गोषु॒ गोषु॑ नो॒ मा मा नो॒ गोषु॑ । \newline
54. नो॒ गोषु॒ गोषु॑ नो नो॒ गोषु॒ मा मा गोषु॑ नो नो॒ गोषु॒ मा । \newline
55. गोषु॒ मा मा गोषु॒ गोषु॒ मा नो॑ नो॒ मा गोषु॒ गोषु॒ मा नः॑ । \newline
56. मा नो॑ नो॒ मा मा नो॒ अश्वे॒ ष्वश्वे॑षु नो॒ मा मा नो॒ अश्वे॑षु । \newline
\pagebreak
\markright{ TS 3.4.11.3  \hfill https://www.vedavms.in \hfill}

\section{ TS 3.4.11.3 }

\textbf{TS 3.4.11.3 } \newline
\textbf{Samhita Paata} \newline

नो॒ अश्वे॑षु रीरिषः । वी॒रान् मानो॑ रुद्र भामि॒तो व॑धीर्.ह॒विष्म॑न्तो॒ नम॑सा विधेम ते ॥ उ॒द॒प्रुतो॒ न वयो॒ रक्ष॑माणा॒ वाव॑दतो अ॒भ्रिय॑स्येव॒ घोषाः᳚ । गि॒रि॒भ्रजो॒ नोर्मयो॒ मद॑न्तो॒ बृह॒स्पति॑म॒भ्य॑र्का अ॑नावन्न् ॥ हꣳ॒॒सैरि॑व॒ सखि॑भि॒र्वाव॑दद्भिरश्म॒न्- मया॑नि॒ नह॑ना॒ व्यस्यन्न्॑ । बृह॒स्पति॑रभि॒कनि॑क्रद॒द्गा उ॒त प्रास्तौ॒दुच्च॑ वि॒द्वाꣳ अ॑गायत् ॥एन्द्र॑ सान॒सिꣳ र॒यिꣳ - [  ] \newline

\textbf{Pada Paata} \newline

नः॒ । अश्वे॑षु । री॒रि॒षः॒ ॥ वी॒रान् । मा । नः॒ । रु॒द्र॒ । भा॒मि॒तः । व॒धीः॒ । ह॒विष्म॑न्तः । नम॑सा । वि॒धे॒म॒ । ते॒ ॥ उ॒द॒प्रुत॒ इत्यु॑द - प्रुतः॑ । न । वयः॑ । रक्ष॑माणाः । वाव॑दतः । अ॒भ्रिय॑स्य । इ॒व॒ । घोषाः᳚ ॥ गि॒रि॒भ्रज॒ इति॑ गिरि - भ्रजः॑ । न । ऊ॒र्मयः॑ । मद॑न्तः । बृह॒स्पति᳚म् । अ॒भीति॑ । अ॒र्काः । अ॒ना॒व॒न्न् ॥ हꣳ॒॒सैः । इ॒व॒ । सखि॑भि॒रिति॒ सखि॑ - भिः॒ । वाव॑दद्भि॒रिति॒ वाव॑दत्- भिः॒ । अ॒श्म॒न्मया॒नीत्य॑श्मन्न् - मया॑नि । नह॑ना । व्यस्य॒न्निति॑ वि-अस्यन्न्॑ ॥ बृह॒स्पतिः॑ । अ॒भि॒कनि॑क्रद॒दित्य॑भि - कनि॑क्रदत् । गाः । उ॒त । प्रेति॑ । अ॒स्तौ॒त् । उदिति॑ । च॒ । वि॒द्वान् । अ॒गा॒य॒त् ॥ एति॑ । इ॒न्द्र॒ । सा॒न॒सिम् । र॒यिम् ।  \newline


\textbf{Krama Paata} \newline

नो॒ अश्वे॑षु । अश्वे॑षु रीरिषः । री॒रि॒ष॒ इति॑ रीरिषः ॥ वी॒रान् मा । मा नः॑ । नो॒ रु॒द्र॒ । रु॒द्र॒ भा॒मि॒तः । भा॒मि॒तो व॑धीः । व॒धी॒र्॒. ह॒विष्म॑न्तः । ह॒विष्म॑न्तो॒ नम॑सा । नम॑सा विधेम । वि॒धे॒म॒ ते॒ । त॒ इति॑ ते ॥ उ॒द॒प्रुतो॒ न । उ॒द॒प्रुत॒ इत्यु॑द - प्रुतः॑ । न वयः॑ । वयो॒ रक्ष॑माणाः । रक्ष॑माणा॒ वाव॑दतः । वाव॑दतो अ॒भ्रिय॑स्य । अ॒भ्रिय॑स्येव । इ॒व॒ घोषाः᳚ । घोषा॒ इति॒ घोषाः᳚ ॥ गि॒रि॒भ्रजो॒ न । गि॒रि॒भ्रज॒ इति॑ गिरि - भ्रजः॑ । नोर्मयः॑ । ऊ॒र्मयो॒ मद॑न्तः । मद॑न्तो॒ बृह॒स्पति᳚म् । बृह॒स्पति॑म॒भि । अ॒भ्य॑र्काः । अ॒र्का अ॑नावन्न् । अ॒ना॒व॒न्नि॒त्य॑नावन्न् ॥ हꣳ॒॒सैरि॑व । इ॒व॒ सखि॑भिः । सखि॑भि॒र् वाव॑दद्भिः । सखि॑भि॒रिति॒ सखि॑ - भिः॒ । वाव॑दद्भिरश्म॒न्मया॑नि । वाव॑दद्भि॒रिति॒ वाव॑दत् - भिः॒ । अ॒श्म॒न्मया॑नि॒ नह॑ना । अ॒श्म॒न्मया॒नीत्य॑श्मन्न् - मया॑नि । नह॑ना॒ व्यस्यन्न्॑ । व्यस्य॒न्निति॑ वि - अस्यन्न्॑ ॥ बृह॒स्पति॑रभि॒कनि॑क्रदत् । अ॒भि॒कनि॑क्रदद् गाः । अ॒भि॒कनि॑क्रद॒दित्य॑भि - कनि॑क्रदत् । गा उ॒त । उ॒त प्र । प्रास्तौ᳚त् । अ॒स्तौ॒दुत् । उच् च॑ । च॒ वि॒द्वान् । वि॒द्वाꣳ अ॑गायत् । अ॒गा॒य॒दि॒त्य॑गायत् ॥ एन्द्र॑ । इ॒न्द्र॒ सा॒न॒सिम् । सा॒न॒सिꣳ र॒यिम् । र॒यिꣳ स॒जित्वा॑नम् \newline

\textbf{Jatai Paata} \newline

1. नो॒ अश्वे॒ ष्वश्वे॑षु नो नो॒ अश्वे॑षु । \newline
2. अश्वे॑षु रीरिषो रीरिषो॒ अश्वे॒ ष्वश्वे॑षु रीरिषः । \newline
3. री॒रि॒ष॒ इति॑ रीरिषः । \newline
4. वी॒रान् मा मा वी॒रान्. वी॒रान् मा । \newline
5. मा नो॑ नो॒ मा मा नः॑ । \newline
6. नो॒ रु॒द्र॒ रु॒द्र॒ नो॒ नो॒ रु॒द्र॒ । \newline
7. रु॒द्र॒ भा॒मि॒तो भा॑मि॒तो रु॑द्र रुद्र भामि॒तः । \newline
8. भा॒मि॒तो व॑धीर् वधीर् भामि॒तो भा॑मि॒तो व॑धीः । \newline
9. व॒धी॒र्॒. ह॒विष्म॑न्तो ह॒विष्म॑न्तो वधीर् वधीर्. ह॒विष्म॑न्तः । \newline
10. ह॒विष्म॑न्तो॒ नम॑सा॒ नम॑सा ह॒विष्म॑न्तो ह॒विष्म॑न्तो॒ नम॑सा । \newline
11. नम॑सा विधेम विधेम॒ नम॑सा॒ नम॑सा विधेम । \newline
12. वि॒धे॒म॒ ते॒ ते॒ वि॒धे॒म॒ वि॒धे॒म॒ ते॒ । \newline
13. त॒ इति॑ ते । \newline
14. उ॒द॒प्रुतो॒ न नो द॒प्रुत॑ उद॒प्रुतो॒ न । \newline
15. उ॒द॒प्रुत॒ इत्यु॑द - प्रुतः॑ । \newline
16. न वयो॒ वयो॒ न न वयः॑ । \newline
17. वयो॒ रक्ष॑माणा॒ रक्ष॑माणा॒ वयो॒ वयो॒ रक्ष॑माणाः । \newline
18. रक्ष॑माणा॒ वाव॑दतो॒ वाव॑दतो॒ रक्ष॑माणा॒ रक्ष॑माणा॒ वाव॑दतः । \newline
19. वाव॑दतो अ॒भ्रिय॑स्या॒ भ्रिय॑स्य॒ वाव॑दतो॒ वाव॑दतो अ॒भ्रिय॑स्य । \newline
20. अ॒भ्रिय॑स्ये वे वा॒भ्रिय॑स्या॒ भ्रिय॑स्ये व । \newline
21. इ॒व॒ घोषा॒ घोषा॑ इवे व॒ घोषाः᳚ । \newline
22. घोषा॒ इति॒ घोषाः᳚ । \newline
23. गि॒रि॒भ्रजो॒ न न गि॑रि॒भ्रजो॑ गिरि॒भ्रजो॒ न । \newline
24. गि॒रि॒भ्रज॒ इति॑ गिरि - भ्रजः॑ । \newline
25. नोर्मय॑ ऊ॒र्मयो॒ न नोर्मयः॑ । \newline
26. ऊ॒र्मयो॒ मद॑न्तो॒ मद॑न्त ऊ॒र्मय॑ ऊ॒र्मयो॒ मद॑न्तः । \newline
27. मद॑न्तो॒ बृह॒स्पति॒म् बृह॒स्पति॒म् मद॑न्तो॒ मद॑न्तो॒ बृह॒स्पति᳚म् । \newline
28. बृह॒स्पति॑ म॒भ्य॑भि बृह॒स्पति॒म् बृह॒स्पति॑ म॒भि । \newline
29. अ॒भ्य॑र्का अ॒र्का अ॒भ्या᳚(1॒)भ्य॑र्काः । \newline
30. अ॒र्का अ॑नावन्-ननावन्-न॒र्का अ॒र्का अ॑नावन्न् । \newline
31. अ॒ना॒व॒न्नित्य॑नावन्न् । \newline
32. हꣳ॒॒ सैरि॑वे व हꣳ॒॒सैर्. हꣳ॒॒ सैरि॑व । \newline
33. इ॒व॒ सखि॑भिः॒ सखि॑भि रिवेव॒ सखि॑भिः । \newline
34. सखि॑भि॒र् वाव॑दद्भि॒र् वाव॑दद्भिः॒ सखि॑भिः॒ सखि॑भि॒र् वाव॑दद्भिः । \newline
35. सखि॑भि॒रिति॒ सखि॑ - भिः॒ । \newline
36. वाव॑दद्भि रश्म॒न्मया᳚ न्यश्म॒न्मया॑नि॒ वाव॑दद्भि॒र् वाव॑दद्भि रश्म॒न्मया॑नि । \newline
37. वाव॑दद्भि॒रिति॒ वाव॑दत् - भिः॒ । \newline
38. अ॒श्म॒न्मया॑नि॒ नह॑ना॒ नह॑ना ऽश्म॒न्मया᳚ न्यश्म॒न्मया॑नि॒ नह॑ना । \newline
39. अ॒श्म॒न्मया॒नीत्य॑श्मन्न् - मया॑नि । \newline
40. नह॑ना॒ व्यस्य॒न् व्यस्य॒न् नह॑ना॒ नह॑ना॒ व्यस्यन्न्॑ । \newline
41. व्यस्य॒न्निति॑ वि - अस्यन्न्॑ । \newline
42. बृह॒स्पति॑ रभि॒कनि॑क्रद दभि॒कनि॑क्रद॒द् बृह॒स्पति॒र् बृह॒स्पति॑ रभि॒कनि॑क्रदत् । \newline
43. अ॒भि॒कनि॑क्रद॒द् गा गा अ॑भि॒कनि॑क्रद दभि॒कनि॑क्रद॒द् गाः । \newline
44. अ॒भि॒कनि॑क्रद॒दित्य॑भि - कनि॑क्रदत् । \newline
45. गा उ॒तोत गा गा उ॒त । \newline
46. उ॒त प्र प्रो तोत प्र । \newline
47. प्रा स्तौ॑ दस्तौ॒त् प्र प्रा स्तौ᳚त् । \newline
48. अ॒स्तौ॒ दुदु द॑स्तौ दस्तौ॒ दुत् । \newline
49. उच् च॒ चोदुच् च॑ । \newline
50. च॒ वि॒द्वान्. वि॒द्वाꣳश्च॑ च वि॒द्वान् । \newline
51. वि॒द्वाꣳ अ॑गाय दगायद् वि॒द्वान्. वि॒द्वाꣳ अ॑गायत् । \newline
52. अ॒गा॒य॒दित्य॑गायत् । \newline
53. एन्द्रे॒ न्द्रेन्द्र॑ । \newline
54. इ॒न्द्र॒ सा॒न॒सिꣳ सा॑न॒सि मि॑न्द्रेन्द्र सान॒सिम् । \newline
55. सा॒न॒सिꣳ र॒यिꣳ र॒यिꣳ सा॑न॒सिꣳ सा॑न॒सिꣳ र॒यिम् । \newline
56. र॒यिꣳ स॒जित्वा॑नꣳ स॒जित्वा॑नꣳ र॒यिꣳ र॒यिꣳ स॒जित्वा॑नम् । \newline

\textbf{Ghana Paata } \newline

1. नो॒ अश्वे॒ ष्वश्वे॑षु नो नो॒ अश्वे॑षु रीरिषो रीरिषो॒ अश्वे॑षु नो नो॒ अश्वे॑षु रीरिषः । \newline
2. अश्वे॑षु रीरिषो रीरिषो॒ अश्वे॒ ष्वश्वे॑षु रीरिषः । \newline
3. री॒रि॒ष॒ इति॑ रीरिषः । \newline
4. वी॒रान् मा मा वी॒रान्. वी॒रान् मा नो॑ नो॒ मा वी॒रान्. वी॒रान् मा नः॑ । \newline
5. मा नो॑ नो॒ मा मा नो॑ रुद्र रुद्र नो॒ मा मा नो॑ रुद्र । \newline
6. नो॒ रु॒द्र॒ रु॒द्र॒ नो॒ नो॒ रु॒द्र॒ भा॒मि॒तो भा॑मि॒तो रु॑द्र नो नो रुद्र भामि॒तः । \newline
7. रु॒द्र॒ भा॒मि॒तो भा॑मि॒तो रु॑द्र रुद्र भामि॒तो व॑धीर् वधीर् भामि॒तो रु॑द्र रुद्र भामि॒तो व॑धीः । \newline
8. भा॒मि॒तो व॑धीर् वधीर् भामि॒तो भा॑मि॒तो व॑धीर्. ह॒विष्म॑न्तो ह॒विष्म॑न्तो वधीर् भामि॒तो भा॑मि॒तो व॑धीर्. ह॒विष्म॑न्तः । \newline
9. व॒धी॒र्॒. ह॒विष्म॑न्तो ह॒विष्म॑न्तो वधीर् वधीर्. ह॒विष्म॑न्तो॒ नम॑सा॒ नम॑सा ह॒विष्म॑न्तो वधीर् वधीर्. ह॒विष्म॑न्तो॒ नम॑सा । \newline
10. ह॒विष्म॑न्तो॒ नम॑सा॒ नम॑सा ह॒विष्म॑न्तो ह॒विष्म॑न्तो॒ नम॑सा विधेम विधेम॒ नम॑सा 
ह॒विष्म॑न्तो ह॒विष्म॑न्तो॒ नम॑सा विधेम । \newline
11. नम॑सा विधेम विधेम॒ नम॑सा॒ नम॑सा विधेम ते ते विधेम॒ नम॑सा॒ नम॑सा विधेम ते । \newline
12. वि॒धे॒म॒ ते॒ ते॒ वि॒धे॒म॒ वि॒धे॒म॒ ते॒ । \newline
13. त॒ इति॑ ते । \newline
14. उ॒द॒प्रुतो॒ न नोद॒प्रुत॑ उद॒प्रुतो॒ न वयो॒ वयो॒ नोद॒प्रुत॑ उद॒प्रुतो॒ न वयः॑ । \newline
15. उ॒द॒प्रुत॒ इत्यु॑द - प्रुतः॑ । \newline
16. न वयो॒ वयो॒ न न वयो॒ रक्ष॑माणा॒ रक्ष॑माणा॒ वयो॒ न न वयो॒ रक्ष॑माणाः । \newline
17. वयो॒ रक्ष॑माणा॒ रक्ष॑माणा॒ वयो॒ वयो॒ रक्ष॑माणा॒ वाव॑दतो॒ वाव॑दतो॒ रक्ष॑माणा॒ 
वयो॒ वयो॒ रक्ष॑माणा॒ वाव॑दतः । \newline
18. रक्ष॑माणा॒ वाव॑दतो॒ वाव॑दतो॒ रक्ष॑माणा॒ रक्ष॑माणा॒ वाव॑दतो अ॒भ्रिय॑स्या॒ भ्रिय॑स्य॒ वाव॑दतो॒ रक्ष॑माणा॒ रक्ष॑माणा॒ वाव॑दतो अ॒भ्रिय॑स्य । \newline
19. वाव॑दतो अ॒भ्रिय॑स्या॒ भ्रिय॑स्य॒ वाव॑दतो॒ वाव॑दतो अ॒भ्रिय॑ स्येवे वा॒भ्रिय॑स्य॒ वाव॑दतो॒ वाव॑दतो अ॒भ्रिय॑स्येव । \newline
20. अ॒भ्रिय॑ स्येवेवा॒ भ्रिय॑स्या॒ भ्रिय॑स्येव॒ घोषा॒ घोषा॑ इवा॒भ्रिय॑स्या॒ भ्रिय॑स्येव॒ घोषाः᳚ । \newline
21. इ॒व॒ घोषा॒ घोषा॑ इवे व॒ घोषाः᳚ । \newline
22. घोषा॒ इति॒ घोषाः᳚ । \newline
23. गि॒रि॒भ्रजो॒ न न गि॑रि॒भ्रजो॑ गिरि॒भ्रजो॒ नोर्मय॑ ऊ॒र्मयो॒ न गि॑रि॒भ्रजो॑ गिरि॒भ्रजो॒ नोर्मयः॑ । \newline
24. गि॒रि॒भ्रज॒ इति॑ गिरि - भ्रजः॑ । \newline
25. नोर्मय॑ ऊ॒र्मयो॒ न नोर्मयो॒ मद॑न्तो॒ मद॑न्त ऊ॒र्मयो॒ न नोर्मयो॒ मद॑न्तः । \newline
26. ऊ॒र्मयो॒ मद॑न्तो॒ मद॑न्त ऊ॒र्मय॑ ऊ॒र्मयो॒ मद॑न्तो॒ बृह॒स्पति॒म् बृह॒स्पति॒म् मद॑न्त ऊ॒र्मय॑ ऊ॒र्मयो॒ मद॑न्तो॒ बृह॒स्पति᳚म् । \newline
27. मद॑न्तो॒ बृह॒स्पति॒म् बृह॒स्पति॒म् मद॑न्तो॒ मद॑न्तो॒ बृह॒स्पति॑ म॒भ्य॑भि बृह॒स्पति॒म् मद॑न्तो॒ 
मद॑न्तो॒ बृह॒स्पति॑ म॒भि । \newline
28. बृह॒स्पति॑ म॒भ्य॑भि बृह॒स्पति॒म् बृह॒स्पति॑ म॒भ्य॑र्का अ॒र्का अ॒भि बृह॒स्पति॒म् बृह॒स्पति॑ म॒भ्य॑र्काः । \newline
29. अ॒भ्य॑र्का अ॒र्का अ॒भ्या᳚(1॒)भ्य॑र्का अ॑नावन्,ननावन्,न॒र्का अ॒भ्या᳚(1॒)भ्य॑र्का अ॑नावन्न् । \newline
30. अ॒र्का अ॑नावन्,ननावन्,न॒र्का अ॒र्का अ॑नावन्न् । \newline
31. अ॒ना॒व॒न्नित्य॑नावन्न् । \newline
32. हꣳ॒॒सै रि॑वेव हꣳ॒॒सैर्. हꣳ॒॒सै रि॑व॒ सखि॑भिः॒ सखि॑भिरिव हꣳ॒॒सैर्. हꣳ॒॒सै रि॑व॒ सखि॑भिः । \newline
33. इ॒व॒ सखि॑भिः॒ सखि॑भि रिवेव॒ सखि॑भि॒र् वाव॑दद्भि॒र् वाव॑दद्भिः॒ सखि॑भि रिवेव॒ सखि॑भि॒र् वाव॑दद्भिः । \newline
34. सखि॑भि॒र् वाव॑दद्भि॒र् वाव॑दद्भिः॒ सखि॑भिः॒ सखि॑भि॒र् वाव॑दद्भि रश्म॒न्मया᳚ न्यश्म॒न्मया॑नि॒ वाव॑दद्भिः॒ सखि॑भिः॒ सखि॑भि॒र् वाव॑दद्भि रश्म॒न्मया॑नि । \newline
35. सखि॑भि॒रिति॒ सखि॑ - भिः॒ । \newline
36. वाव॑दद्भि रश्म॒न्मया᳚ न्यश्म॒न्मया॑नि॒ वाव॑दद्भि॒र् वाव॑दद्भि रश्म॒न्मया॑नि॒ नह॑ना॒ नह॑ना ऽश्म॒न्मया॑नि॒ वाव॑दद्भि॒र् वाव॑दद्भि रश्म॒न्मया॑नि॒ नह॑ना । \newline
37. वाव॑दद्भि॒रिति॒ वाव॑दत् - भिः॒ । \newline
38. अ॒श्म॒न्मया॑नि॒ नह॑ना॒ नह॑ना ऽश्म॒न्मया᳚न्य श्म॒न्मया॑नि॒ नह॑ना॒ व्यस्य॒न् व्यस्य॒न् नह॑ना ऽश्म॒न्मया᳚ न्यश्म॒न्मया॑नि॒ नह॑ना॒ व्यस्यन्न्॑ । \newline
39. अ॒श्म॒न्मया॒नीत्य॑श्मन्न् - मया॑नि । \newline
40. नह॑ना॒ व्यस्य॒न् व्यस्य॒न् नह॑ना॒ नह॑ना॒ व्यस्यन्न्॑ । \newline
41. व्यस्य॒न्निति॑ वि - अस्यन्न्॑ । \newline
42. बृह॒स्पति॑ रभि॒कनि॑क्रद दभि॒कनि॑क्रद॒द् बृह॒स्पति॒र् बृह॒स्पति॑ रभि॒कनि॑क्रद॒द् गा गा अ॑भि॒कनि॑क्रद॒द् 
बृह॒स्पति॒र् बृह॒स्पति॑ रभि॒कनि॑क्रद॒द् गाः । \newline
43. अ॒भि॒कनि॑क्रद॒द् गा गा अ॑भि॒कनि॑क्रद दभि॒कनि॑क्रद॒द् गा उ॒तोत गा अ॑भि॒कनि॑क्रद दभि॒कनि॑क्रद॒द् गा उ॒त । \newline
44. अ॒भि॒कनि॑क्रद॒दित्य॑भि - कनि॑क्रदत् । \newline
45. गा उ॒तोत गा गा उ॒त प्र प्रोत गा गा उ॒त प्र । \newline
46. उ॒त प्र प्रोतोत प्रास्तौ॑ दस्तौ॒त् प्रोतोत प्रास्तौ᳚त् । \newline
47. प्रास्तौ॑ दस्तौ॒त् प्र प्रास्तौ॒ दुदु द॑स्तौ॒त् प्र प्रास्तौ॒ दुत् । \newline
48. अ॒स्तौ॒ दुदु द॑स्तौ दस्तौ॒ दुच्च॒ चोद॑स्तौ दस्तौ॒ दुच् च॑ । \newline
49. उच् च॒ चोदुच् च॑ वि॒द्वान्. वि॒द्वाꣳ श्चोदुच् च॑ वि॒द्वान् । \newline
50. च॒ वि॒द्वान्. वि॒द्वाꣳ श्च॑ च वि॒द्वाꣳ अ॑गाय दगायद् वि॒द्वाꣳ श्च॑ च वि॒द्वाꣳ अ॑गायत् । \newline
51. वि॒द्वाꣳ अ॑गाय दगायद् वि॒द्वान्. वि॒द्वाꣳ अ॑गायत् । \newline
52. अ॒गा॒य॒दित्य॑गायत् । \newline
53. एन्द्रे॒ न्द्रेन्द्र॑ सान॒सिꣳ सा॑न॒सि मि॒न्द्रे न्द्र॑ सान॒सिम् । \newline
54. इ॒न्द्र॒ सा॒न॒सिꣳ सा॑न॒सि मि॑न्द्रे न्द्र सान॒सिꣳ र॒यिꣳ र॒यिꣳ सा॑न॒सि मि॑न्द्रे न्द्र सान॒सिꣳ र॒यिम् । \newline
55. सा॒न॒सिꣳ र॒यिꣳ र॒यिꣳ सा॑न॒सिꣳ सा॑न॒सिꣳ र॒यिꣳ स॒जित्वा॑नꣳ स॒जित्वा॑नꣳ र॒यिꣳ सा॑न॒सिꣳ सा॑न॒सिꣳ र॒यिꣳ स॒जित्वा॑नम् । \newline
56. र॒यिꣳ स॒जित्वा॑नꣳ स॒जित्वा॑नꣳ र॒यिꣳ र॒यिꣳ स॒जित्वा॑नꣳ सदा॒सहꣳ॑ सदा॒सहꣳ॑ स॒जित्वा॑नꣳ र॒यिꣳ र॒यिꣳ स॒जित्वा॑नꣳ सदा॒सह᳚म् । \newline
\pagebreak
\markright{ TS 3.4.11.4  \hfill https://www.vedavms.in \hfill}

\section{ TS 3.4.11.4 }

\textbf{TS 3.4.11.4 } \newline
\textbf{Samhita Paata} \newline

स॒जित्वा॑नꣳ सदा॒सहं᳚ । वर्.षि॑ष्ठमू॒तये॑ भर ॥प्र स॑साहिषे पुरुहूत॒ शत्रू॒न् ज्येष्ठ॑स्ते॒ शुष्म॑ इ॒ह रा॒तिर॑स्तु । इन्द्राऽऽ* भ॑र॒ दक्षि॑णेना॒ वसू॑नि॒ पतिः॒ सिन्धू॑नामसि रे॒वती॑नां ॥ त्वꣳ सु॒तस्य॑ पी॒तये॑ स॒द्यो वृ॒द्धो अ॑जायथाः । इन्द्र॒ ज्यैष्ठ्या॑य सुक्रतो ॥ भुव॒स्त्वमि॑न्द्र॒ ब्रह्म॑णा म॒हान् भुवो॒ विश्वे॑षु॒ सव॑नेषु य॒ज्ञियः॑ । भुवो॒ नॄꣳश्च्यौ॒त्नो विश्व॑स्मि॒न् भरे॒ ज्येष्ठ॑श्च॒ मन्त्रो॑ - [  ] \newline

\textbf{Pada Paata} \newline

स॒जित्वा॑न॒मिति॑ स - जित्वा॑नम् । स॒दा॒सह॒मिति॑ सदा - सह᳚म् ॥ वर्.षि॑ष्ठम् । ऊ॒तये᳚ । भ॒र॒ ॥ प्रेति॑ । स॒सा॒हि॒षे॒ । पु॒रु॒हू॒तेति॑ पुरु-हू॒त॒ । शत्रून्॑ । ज्येष्ठः॑ । ते॒ । शुष्मः॑ । इ॒ह । रा॒तिः । अ॒स्तु॒ ॥ इन्द्र॑ । एति॑ । भ॒र॒ । दक्षि॑णेन । वसू॑नि । पतिः॑ । सिन्धू॑नाम् । अ॒सि॒ । रे॒वती॑नाम् ॥ त्वम् । सु॒तस्य॑ । पी॒तये᳚ । स॒द्यः । वृ॒द्धः । अ॒जा॒य॒थाः॒ । इन्द्र॑ । ज्यैष्ठ्या॑य । सु॒क्र॒तो॒ इति॑ सु-क्र॒तो॒ ॥ भुवः॑ । त्वम् । इ॒न्द्र॒ । ब्रह्म॑णा । म॒हान् । भुवः॑ । विश्वे॑षु । सव॑नेषु । य॒ज्ञियः॑ ॥ भुवः॑ । नॄन् । च्यौ॒त्नः । विश्व॑स्मिन्न् । भरे᳚ । ज्येष्ठः॑ । च॒ । मन्त्रः॑ ।  \newline


\textbf{Krama Paata} \newline

स॒जित्वा॑नꣳ सदा॒सह᳚म् । स॒जित्वा॑न॒मिति॑ स - जित्वा॑नम् । स॒दा॒सह॒मिति॑ सदा - सह᳚म् ॥ वर्.षि॑ष्ठमू॒तये᳚ । ऊ॒तये॑ भर । भ॒रेति॑ भर ॥ प्र स॑साहिषे । स॒सा॒हि॒षे॒ पु॒रु॒हू॒त॒ । पु॒रु॒हू॒त॒ शत्रून्॑ । पु॒रु॒हू॒तेति॑ पुरु - हू॒त॒ । शत्रू॒न् ज्येष्ठः॑ । ज्येष्ठ॑स्ते । ते॒ शुष्मः॑ । शुष्म॑ इ॒ह । इ॒ह रा॒तिः । रा॒तिर॑स्तु । अ॒स्त्वित्य॑स्तु ॥ इन्द्रा । आ भ॑र । भ॒र॒ दक्षि॑णेन । दक्षि॑णेना॒ वसू॑नि । वसू॑नि॒ पतिः॑ । पतिः॒ सिन्धू॑नाम् । सिन्धू॑नामसि । अ॒सि॒ रे॒वती॑नाम् । रे॒वती॑ना॒मिति॑ रे॒वती॑नाम् ॥ त्वꣳ सु॒तस्य॑ । सु॒तस्य॑ पी॒तये᳚ । पी॒तये॑ स॒द्यः । स॒द्यो वृ॒द्धः । वृ॒द्धो अ॑जायथाः । अ॒जा॒य॒था॒ इत्य॑जायथाः ॥  इन्द्र॒ ज्यैष्ठ्या॑य । ज्यैष्ठ्या॑य सुक्रतो । सु॒क्र॒तो॒ इति॑ सु - क्र॒तो॒ ॥ भुव॒स्त्वम् । त्वमि॑न्द्र । इ॒न्द्र॒ ब्रह्म॑णा । ब्रह्म॑णा म॒हान् । म॒हान् भुवः॑ । भुवो॒ विश्वे॑षु । विश्वे॑षु॒ सव॑नेषु । सव॑नेषु य॒ज्ञियः॑ । य॒ज्ञिय॒ इति॑ य॒ज्ञियः॑ ॥ भुवो॒ नॄन् । नॄꣳश्च्यौ॒त्नः । च्यौ॒त्नो विश्व॑स्मिन्न् । विश्व॑स्मि॒न् भरे᳚ । भरे॒ ज्येष्ठः॑ । ज्येष्ठ॑श्च । च॒ मन्त्रः॑ । मन्त्रो॑ विश्वचर्.षणे \newline

\textbf{Jatai Paata} \newline

1. स॒जित्वा॑नꣳ सदा॒सहꣳ॑ सदा॒सहꣳ॑ स॒जित्वा॑नꣳ स॒जित्वा॑नꣳ सदा॒सह᳚म् । \newline
2. स॒जित्वा॑न॒मिति॑ स - जित्वा॑नम् । \newline
3. स॒दा॒सह॒मिति॑ सदा - सह᳚म् । \newline
4. वर्.षि॑ष्ठ मू॒तय॑ ऊ॒तये॒ वर्.षि॑ष्ठं॒ ॅवर्.षि॑ष्ठ मू॒तये᳚ । \newline
5. ऊ॒तये॑ भर भरो॒तय॑ ऊ॒तये॑ भर । \newline
6. भ॒रेति॑ भर । \newline
7. प्र स॑साहिषे ससाहिषे॒ प्र प्र स॑साहिषे । \newline
8. स॒सा॒हि॒षे॒ पु॒रु॒हू॒त॒ पु॒रु॒हू॒त॒ स॒सा॒हि॒षे॒ स॒सा॒हि॒षे॒ पु॒रु॒हू॒त॒ । \newline
9. पु॒रु॒हू॒त॒ शत्रू॒ञ् छत्रू᳚न् पुरुहूत पुरुहूत॒ शत्रून्॑ । \newline
10. पु॒रु॒हू॒तेति॑ पुरु - हू॒त॒ । \newline
11. शत्रू॒न् ज्येष्ठो॒ ज्येष्ठः॒ शत्रू॒ञ् छत्रू॒न् ज्येष्ठः॑ । \newline
12. ज्येष्ठ॑ स्ते ते॒ ज्येष्ठो॒ ज्येष्ठ॑ स्ते । \newline
13. ते॒ शुष्मः॒ शुष्म॑ स्ते ते॒ शुष्मः॑ । \newline
14. शुष्म॑ इ॒हे ह शुष्मः॒ शुष्म॑ इ॒ह । \newline
15. इ॒ह रा॒ती रा॒ति रि॒हे ह रा॒तिः । \newline
16. रा॒ति र॑स्त्वस्तु रा॒ती रा॒तिर॑स्तु । \newline
17. अ॒स्त्वित्य॑स्तु । \newline
18. इन्द्रेन्द्रे न्द्रा । \newline
19. आ भ॑र भ॒रा भ॑र । \newline
20. भ॒र॒ दक्षि॑णेन॒ दक्षि॑णेन भर भर॒ दक्षि॑णेन । \newline
21. दक्षि॑णेना॒ वसू॑नि॒ वसू॑नि॒ दक्षि॑णेन॒ दक्षि॑णेना॒ वसू॑नि । \newline
22. वसू॑नि॒ पति॒ष् पति॒र् वसू॑नि॒ वसू॑नि॒ पतिः॑ । \newline
23. पतिः॒ सिन्धू॑नाꣳ॒॒ सिन्धू॑ना॒म् पति॒ष् पतिः॒ सिन्धू॑नाम् । \newline
24. सिन्धू॑ना मस्यसि॒ सिन्धू॑नाꣳ॒॒ सिन्धू॑ना मसि । \newline
25. अ॒सि॒ रे॒वती॑नाꣳ रे॒वती॑ना मस्यसि रे॒वती॑नाम् । \newline
26. रे॒वती॑ना॒मिति॑ रे॒वती॑नाम् । \newline
27. त्वꣳ सु॒तस्य॑ सु॒तस्य॒ त्वम् त्वꣳ सु॒तस्य॑ । \newline
28. सु॒तस्य॑ पी॒तये॑ पी॒तये॑ सु॒तस्य॑ सु॒तस्य॑ पी॒तये᳚ । \newline
29. पी॒तये॑ स॒द्यः स॒द्यः पी॒तये॑ पी॒तये॑ स॒द्यः । \newline
30. स॒द्यो वृ॒द्धो वृ॒द्धः स॒द्यः स॒द्यो वृ॒द्धः । \newline
31. वृ॒द्धो अ॑जायथा अजायथा वृ॒द्धो वृ॒द्धो अ॑जायथाः । \newline
32. अ॒जा॒य॒था॒ इत्य॑जायथाः । \newline
33. इन्द्र॒ ज्यैष्ठ्‌या॑य॒ ज्यैष्ठ्या॒ येन्द्रेन्द्र॒ ज्यैष्ठ्‌या॑य । \newline
34. ज्यैष्ठ्‌या॑य सुक्रतो सुक्रतो॒ ज्यैष्ठ्‌या॑य॒ ज्यैष्ठ्‌या॑य सुक्रतो । \newline
35. सु॒क्र॒तो॒ इति॑ सु - क्र॒तो॒ । \newline
36. भुव॒ स्त्वम् त्वम् भुवो॒ भुव॒ स्त्वम् । \newline
37. त्व मि॑न्द्रेन्द्र॒ त्वम् त्व मि॑न्द्र । \newline
38. इ॒न्द्र॒ ब्रह्म॑णा॒ ब्रह्म॑ णेन्द्रेन्द्र॒ ब्रह्म॑णा । \newline
39. ब्रह्म॑णा म॒हान् म॒हान् ब्रह्म॑णा॒ ब्रह्म॑णा म॒हान् । \newline
40. म॒हान् भुवो॒ भुवो॑ म॒हान् म॒हान् भुवः॑ । \newline
41. भुवो॒ विश्वे॑षु॒ विश्वे॑षु॒ भुवो॒ भुवो॒ विश्वे॑षु । \newline
42. विश्वे॑षु॒ सव॑नेषु॒ सव॑नेषु॒ विश्वे॑षु॒ विश्वे॑षु॒ सव॑नेषु । \newline
43. सव॑नेषु य॒ज्ञियो॑ य॒ज्ञियः॒ सव॑नेषु॒ सव॑नेषु य॒ज्ञियः॑ । \newline
44. य॒ज्ञिय॒ इति॑ य॒ज्ञियः॑ । \newline
45. भुवो॒ नॄन् नॄन् भुवो॒ भुवो॒ नॄन् । \newline
46. नॄꣳ श्च्यौ॒त्न श्च्यौ॒त्नो नॄन् नॄꣳ श्च्यौ॒त्नः । \newline
47. च्यौ॒त्नो विश्व॑स्मि॒न्॒. विश्व॑स्मिꣳ श्च्यौ॒त्न श्च्यौ॒त्नो विश्व॑स्मिन्न् । \newline
48. विश्व॑स्मि॒न् भरे॒ भरे॒ विश्व॑स्मि॒न्॒. विश्व॑स्मि॒न् भरे᳚ । \newline
49. भरे॒ ज्येष्ठो॒ ज्येष्ठो॒ भरे॒ भरे॒ ज्येष्ठः॑ । \newline
50. ज्येष्ठ॑श्च च॒ ज्येष्ठो॒ ज्येष्ठ॑श्च । \newline
51. च॒ मन्त्रो॒ मन्त्र॑श्च च॒ मन्त्रः॑ । \newline
52. मन्त्रो॑ विश्वचर्.षणे विश्वचर्.षणे॒ मन्त्रो॒ मन्त्रो॑ विश्वचर्.षणे । \newline

\textbf{Ghana Paata } \newline

1. स॒जित्वा॑नꣳ सदा॒सहꣳ॑ सदा॒सहꣳ॑ स॒जित्वा॑नꣳ स॒जित्वा॑नꣳ सदा॒सह᳚म् । \newline
2. स॒जित्वा॑न॒मिति॑ स - जित्वा॑नम् । \newline
3. स॒दा॒सह॒मिति॑ सदा - सह᳚म् । \newline
4. वर्.षि॑ष्ठ मू॒तय॑ ऊ॒तये॒ वर्.षि॑ष्ठ॒म् ॅवर्.षि॑ष्ठ मू॒तये॑ भर भरो॒ तये॒ वर्.षि॑ष्ठ॒म् ॅवर्.षि॑ष्ठ मू॒तये॑ भर । \newline
5. ऊ॒तये॑ भर भरो॒तय॑ ऊ॒तये॑ भर । \newline
6. भ॒रेति॑ भर । \newline
7. प्र स॑साहिषे ससाहिषे॒ प्र प्र स॑साहिषे पुरुहूत पुरुहूत ससाहिषे॒ प्र प्र स॑साहिषे पुरुहूत । \newline
8. स॒सा॒हि॒षे॒ पु॒रु॒हू॒त॒ पु॒रु॒हू॒त॒ स॒सा॒हि॒षे॒ स॒सा॒हि॒षे॒ पु॒रु॒हू॒त॒ शत्रू॒ञ् छत्रू᳚न् पुरुहूत ससाहिषे ससाहिषे पुरुहूत॒ शत्रून्॑ । \newline
9. पु॒रु॒हू॒त॒ शत्रू॒ञ् छत्रू᳚न् पुरुहूत पुरुहूत॒ शत्रू॒न् ज्येष्ठो॒ ज्येष्ठः॒ शत्रू᳚न् पुरुहूत पुरुहूत॒ शत्रू॒न् ज्येष्ठः॑ । \newline
10. पु॒रु॒हू॒तेति॑ पुरु - हू॒त॒ । \newline
11. शत्रू॒न् ज्येष्ठो॒ ज्येष्ठः॒ शत्रू॒ञ् छत्रू॒न् ज्येष्ठ॑ स्ते ते॒ ज्येष्ठः॒ शत्रू॒ञ् छत्रू॒न् ज्येष्ठ॑ स्ते । \newline
12. ज्येष्ठ॑ स्ते ते॒ ज्येष्ठो॒ ज्येष्ठ॑ स्ते॒ शुष्मः॒ शुष्म॑ स्ते॒ ज्येष्ठो॒ ज्येष्ठ॑ स्ते॒ शुष्मः॑ । \newline
13. ते॒ शुष्मः॒ शुष्म॑ स्ते ते॒ शुष्म॑ इ॒हेह शुष्म॑ स्ते ते॒ शुष्म॑ इ॒ह । \newline
14. शुष्म॑ इ॒हेह शुष्मः॒ शुष्म॑ इ॒ह रा॒ती रा॒ति रि॒ह शुष्मः॒ शुष्म॑ इ॒ह रा॒तिः । \newline
15. इ॒ह रा॒ती रा॒ति रि॒हेह रा॒ति र॑स्त्वस्तु रा॒ति रि॒हेह रा॒ति र॑स्तु । \newline
16. रा॒ति र॑स्त्वस्तु रा॒ती रा॒ति र॑स्तु । \newline
17. अ॒स्त्वित्य॑स्तु । \newline
18. इन्द्रेन्द्रे न्द्रा भ॑र भ॒रेन्द्रे न्द्रा भ॑र । \newline
19. आ भ॑र भ॒रा भ॑र॒ दक्षि॑णेन॒ दक्षि॑णेन भ॒रा भ॑र॒ दक्षि॑णेन । \newline
20. भ॒र॒ दक्षि॑णेन॒ दक्षि॑णेन भर भर॒ दक्षि॑णेना॒ वसू॑नि॒ वसू॑नि॒ दक्षि॑णेन भर भर॒ दक्षि॑णेना॒ वसू॑नि । \newline
21. दक्षि॑णेना॒ वसू॑नि॒ वसू॑नि॒ दक्षि॑णेन॒ दक्षि॑णेना॒ वसू॑नि॒ पति॒ष् पति॒र् वसू॑नि॒ दक्षि॑णेन॒ दक्षि॑णेना॒ वसू॑नि॒ पतिः॑ । \newline
22. वसू॑नि॒ पति॒ष् पति॒र् वसू॑नि॒ वसू॑नि॒ पतिः॒ सिन्धू॑नाꣳ॒॒ सिन्धू॑ना॒म् पति॒र् वसू॑नि॒ वसू॑नि॒ पतिः॒ सिन्धू॑नाम् । \newline
23. पतिः॒ सिन्धू॑नाꣳ॒॒ सिन्धू॑ना॒म् पति॒ष् पतिः॒ सिन्धू॑ना मस्यसि॒ सिन्धू॑ना॒म् पति॒ष् पतिः॒ सिन्धू॑ना मसि । \newline
24. सिन्धू॑ना मस्यसि॒ सिन्धू॑नाꣳ॒॒ सिन्धू॑ना मसि रे॒वती॑नाꣳ रे॒वती॑ना मसि॒ सिन्धू॑नाꣳ॒॒ सिन्धू॑ना मसि रे॒वती॑नाम् । \newline
25. अ॒सि॒ रे॒वती॑नाꣳ रे॒वती॑ना मस्यसि रे॒वती॑नाम् । \newline
26. रे॒वती॑ना॒मिति॑ रे॒वती॑नाम् । \newline
27. त्वꣳ सु॒तस्य॑ सु॒तस्य॒ त्वम् त्वꣳ सु॒तस्य॑ पी॒तये॑ पी॒तये॑ सु॒तस्य॒ त्वम् त्वꣳ सु॒तस्य॑ पी॒तये᳚ । \newline
28. सु॒तस्य॑ पी॒तये॑ पी॒तये॑ सु॒तस्य॑ सु॒तस्य॑ पी॒तये॑ स॒द्यः स॒द्यः पी॒तये॑ सु॒तस्य॑ सु॒तस्य॑ पी॒तये॑ स॒द्यः । \newline
29. पी॒तये॑ स॒द्यः स॒द्यः पी॒तये॑ पी॒तये॑ स॒द्यो वृ॒द्धो वृ॒द्धः स॒द्यः पी॒तये॑ पी॒तये॑ स॒द्यो वृ॒द्धः । \newline
30. स॒द्यो वृ॒द्धो वृ॒द्धः स॒द्यः स॒द्यो वृ॒द्धो अ॑जायथा अजायथा वृ॒द्धः स॒द्यः स॒द्यो वृ॒द्धो अ॑जायथाः । \newline
31. वृ॒द्धो अ॑जायथा अजायथा वृ॒द्धो वृ॒द्धो अ॑जायथाः । \newline
32. अ॒जा॒य॒था॒ इत्य॑जायथाः । \newline
33. इन्द्र॒ ज्यैष्ठ्‍या॑य॒ ज्यैष्ठ्‍या॒ येन्द्रेन्द्र॒ ज्यैष्ठ्‍या॑य सुक्रतो सुक्रतो॒ ज्यैष्ठ्‍या॒ येन्द्रेन्द्र॒ ज्यैष्ठ्या॑य सुक्रतो । \newline
34. ज्यैष्ठ्‍या॑य सुक्रतो सुक्रतो॒ ज्यैष्ठ्‍या॑य॒ ज्यैष्ठ्‍या॑य सुक्रतो । \newline
35. सु॒क्र॒तो॒ इति॑ सु - क्र॒तो॒ । \newline
36. भुव॒स्त्वम् त्वम् भुवो॒ भुव॒स्त्व मि॑न्द्रेन्द्र॒ त्वम् भुवो॒ भुव॒स्त्व मि॑न्द्र । \newline
37. त्व मि॑न्द्रेन्द्र॒ त्वम् त्व मि॑न्द्र॒ ब्रह्म॑णा॒ ब्रह्म॑णेन्द्र॒ त्वम् त्व मि॑न्द्र॒ ब्रह्म॑णा । \newline
38. इ॒न्द्र॒ ब्रह्म॑णा॒ ब्रह्म॑णेन्द्रेन्द्र॒ ब्रह्म॑णा म॒हान् म॒हान् ब्रह्म॑णेन्द्रेन्द्र॒ ब्रह्म॑णा म॒हान् । \newline
39. ब्रह्म॑णा म॒हान् म॒हान् ब्रह्म॑णा॒ ब्रह्म॑णा म॒हान् भुवो॒ भुवो॑ म॒हान् ब्रह्म॑णा॒ ब्रह्म॑णा म॒हान् भुवः॑ । \newline
40. म॒हान् भुवो॒ भुवो॑ म॒हान् म॒हान् भुवो॒ विश्वे॑षु॒ विश्वे॑षु॒ भुवो॑ म॒हान् म॒हान् भुवो॒ विश्वे॑षु । \newline
41. भुवो॒ विश्वे॑षु॒ विश्वे॑षु॒ भुवो॒ भुवो॒ विश्वे॑षु॒ सव॑नेषु॒ सव॑नेषु॒ विश्वे॑षु॒ भुवो॒ भुवो॒ विश्वे॑षु॒ सव॑नेषु । \newline
42. विश्वे॑षु॒ सव॑नेषु॒ सव॑नेषु॒ विश्वे॑षु॒ विश्वे॑षु॒ सव॑नेषु य॒ज्ञियो॑ य॒ज्ञियः॒ सव॑नेषु॒ विश्वे॑षु॒ विश्वे॑षु॒ सव॑नेषु य॒ज्ञियः॑ । \newline
43. सव॑नेषु य॒ज्ञियो॑ य॒ज्ञियः॒ सव॑नेषु॒ सव॑नेषु य॒ज्ञियः॑ । \newline
44. य॒ज्ञिय॒ इति॑ य॒ज्ञियः॑ । \newline
45. भुवो॒ नॄन् नॄन् भुवो॒ भुवो॒ नॄꣳ श्च्यौ॒त्न श्च्यौ॒त्नो नॄन् भुवो॒ भुवो॒ नॄꣳ श्च्यौ॒त्नः । \newline
46. नॄꣳ श्च्यौ॒त्न श्च्यौ॒त्नो नॄन् नॄꣳ श्च्यौ॒त्नो विश्व॑स्मि॒न्॒. विश्व॑स्मिꣳ श्च्यौ॒त्नो नॄन् नॄꣳ श्च्यौ॒त्नो विश्व॑स्मिन्न् । \newline
47. च्यौ॒त्नो विश्व॑स्मि॒न्॒. विश्व॑स्मिꣳ श्च्यौ॒त्न श्च्यौ॒त्नो विश्व॑स्मि॒न् भरे॒ भरे॒ विश्व॑स्मिꣳ श्च्यौ॒त्न श्च्यौ॒त्नो विश्व॑स्मि॒न् भरे᳚ । \newline
48. विश्व॑स्मि॒न् भरे॒ भरे॒ विश्व॑स्मि॒न्॒. विश्व॑स्मि॒न् भरे॒ ज्येष्ठो॒ ज्येष्ठो॒ भरे॒ विश्व॑स्मि॒न्॒. विश्व॑स्मि॒न् भरे॒ ज्येष्ठः॑ । \newline
49. भरे॒ ज्येष्ठो॒ ज्येष्ठो॒ भरे॒ भरे॒ ज्येष्ठ॑श्च च॒ ज्येष्ठो॒ भरे॒ भरे॒ ज्येष्ठ॑श्च । \newline
50. ज्येष्ठ॑श्च च॒ ज्येष्ठो॒ ज्येष्ठ॑श्च॒ मन्त्रो॒ मन्त्र॑श्च॒ ज्येष्ठो॒ ज्येष्ठ॑श्च॒ मन्त्रः॑ । \newline
51. च॒ मन्त्रो॒ मन्त्र॑श्च च॒ मन्त्रो॑ विश्वचर्.षणे विश्वचर्.षणे॒ मन्त्र॑श्च च॒ मन्त्रो॑ विश्वचर्.षणे । \newline
52. मन्त्रो॑ विश्वचर्.षणे विश्वचर्.षणे॒ मन्त्रो॒ मन्त्रो॑ विश्वचर्.षणे । \newline
\pagebreak
\markright{ TS 3.4.11.5  \hfill https://www.vedavms.in \hfill}

\section{ TS 3.4.11.5 }

\textbf{TS 3.4.11.5 } \newline
\textbf{Samhita Paata} \newline

विश्वचर्.षणे ॥ मि॒त्रस्य॑ चर्.षणी॒धृतः॒ श्रवो॑ दे॒वस्य॑ सान॒सिं । स॒त्यं चि॒त्र श्र॑वस्तमं ॥मि॒त्रो जनान्॑ यातयति प्रजा॒नन् मि॒त्रो दा॑धार पृथि॒वीमु॒त द्यां । मि॒त्रः कृ॒ष्टीरनि॑मिषा॒ऽभि च॑ष्टे स॒त्याय॑ ह॒व्यं घृ॒तव॑द्-विधेम ॥ प्रसमि॑त्र॒ मर्तो॑ अस्तु॒ प्रय॑स्वा॒न्॒. यस्त॑ आदित्य॒ शिक्ष॑ति व्र॒तेन॑ । न ह॑न्यते॒ न जी॑यते॒ त्वोतो॒ नैन॒मꣳहो॑ अश्नो॒त्यन्ति॑तो॒ न दू॒रात् ॥ य- [  ] \newline

\textbf{Pada Paata} \newline

वि॒श्व॒च॒र्॒.ष॒ण॒ इति॑ विश्व - च॒र्॒.ष॒णे॒ ॥ मि॒त्रस्य॑ । च॒र्॒.ष॒णी॒धृत॒ इति॑ चर्.षणी - धृतः॑ । श्रवः॑ । दे॒वस्य॑ । सा॒न॒सिम् ॥ स॒त्यम् । चि॒त्रश्र॑वस्तम॒मिति॑ चि॒त्रश्र॑वः - त॒म॒म् ॥ मि॒त्रः । जनान्॑ । या॒त॒य॒ति॒ । प्र॒जा॒नन्निति॑ प्र - जा॒नन् । मि॒त्रः । दा॒धा॒र॒ । पृ॒थि॒वीम् । उ॒त । द्याम् ॥ मि॒त्रः । कृ॒ष्टीः । अनि॑मि॒षेत्यनि॑ - मि॒षा॒ । अ॒भीति॑ । च॒ष्टे॒ । स॒त्याय॑ । ह॒व्यम् । घृ॒तव॒दिति॑ घृ॒त-व॒त् । वि॒धे॒म॒ ॥ प्रेति॑ । सः । मि॒त्र॒ । मर्तः॑ । अ॒स्तु॒ । प्रय॑स्वान् । यः । ते॒ । आ॒दि॒त्य॒ । शिक्ष॑ति । व्र॒तेन॑ ॥ न । ह॒न्य॒ते॒ । न । जी॒य॒ते॒ । त्वोतः॑ । न । ए॒न॒म् । अꣳहः॑ । अ॒श्नो॒ति॒ । अन्ति॑तः । न । दू॒रात् ॥ यत् ।  \newline


\textbf{Krama Paata} \newline

वि॒श्व॒च॒र्॒.ष॒ण॒ इति॑ विश्व - च॒र्॒.ष॒णे॒ ॥  मि॒त्रस्य॑ चर्.षणी॒धृतः॑ । च॒र्॒.ष॒णी॒धृतः॒ श्रवः॑ । च॒र्॒.ष॒णी॒धृत॒ इति॑ चर्.षणि - धृतः॑ । श्रवो॑ दे॒वस्य॑ । दे॒वस्य॑ सान॒सिम् । सा॒न॒सिमिति॑ सान॒सिम् ॥ स॒त्यम् चि॒त्रश्र॑वस्तमम् । चि॒त्रश्र॑वस्तम॒मिति॑ चि॒त्रश्र॑वः - त॒म॒म् । मि॒त्रो जनान्॑ । जनान्॑. यातयति । या॒त॒य॒ति॒ प्र॒जा॒नन्न् । प्र॒जा॒नन् मि॒त्रः । प्र॒जा॒नन्निति॑ प्र - जा॒नन्न् । मि॒त्रो दा॑धार । दा॒धा॒र॒ पृ॒थि॒वीम् । पृ॒थि॒वीमु॒त । उ॒त द्याम् । द्यामिति॒ द्याम् ॥ मि॒त्रः कृ॒ष्टीः । कृ॒ष्टीरनि॑मिषा । अनि॑मिषा॒ ऽभि । अनि॑मि॒षेत्यनि॑ - मि॒षा॒ । 
अ॒भि च॑ष्टे । च॒ष्टे॒ स॒त्याय॑ । स॒त्याय॑ ह॒व्यम् । ह॒व्यम् घृ॒तव॑त् । घृ॒तव॑द् विधेम । घृ॒तव॒दिति॑ घृत - व॒त्॒ । वि॒धे॒मेति॑ विधेम ॥ प्र सः । स मि॑त्र । मि॒त्र॒ मर्तः॑ । मर्तो॑ अस्तु । अ॒स्तु॒ प्रय॑स्वान् । प्रय॑स्वा॒न्॒. यः । यस्ते᳚ । त॒ आ॒दि॒त्य॒ । आ॒दि॒त्य॒ शिक्ष॑ति । शिक्ष॑ति व्र॒तेन॑ । व्र॒तेनेति॑ व्र॒तेन॑ ॥ न ह॑न्यते । ह॒न्य॒ते॒ न । न जी॑यते । जी॒य॒ते॒ त्वोतः॑ । त्वोतो॒ न । नैन᳚म् । ए॒न॒मꣳहः॑ । अꣳहो॑ अश्ञोति । अ॒श्ञो॒त्यन्ति॑तः । अन्ति॑तो॒ न । न दू॒रात् । दू॒रादिति॑ दू॒रात् ॥ यच्चि॑त् \newline

\textbf{Jatai Paata} \newline

1. वि॒श्व॒च॒र्॒.ष॒ण॒ इति॑ विश्व - च॒र्॒.ष॒णे॒ । \newline
2. मि॒त्रस्य॑ चर्.षणी॒धृत॑ श्चर्.षणी॒धृतो॑ मि॒त्रस्य॑ मि॒त्रस्य॑ चर्.षणी॒धृतः॑ । \newline
3. च॒र्॒.ष॒णी॒धृतः॒ श्रवः॒ श्रव॑श्चर्.षणी॒धृत॑ श्चर्.षणी॒धृतः॒ श्रवः॑ । \newline
4. च॒र्॒.ष॒णी॒धृत॒ इति॑ चर्.षणि - धृतः॑ । \newline
5. श्रवो॑ दे॒वस्य॑ दे॒वस्य॒ श्रवः॒ श्रवो॑ दे॒वस्य॑ । \newline
6. दे॒वस्य॑ सान॒सिꣳ सा॑न॒सिम् दे॒वस्य॑ दे॒वस्य॑ सान॒सिम् । \newline
7. सा॒न॒सिमिति॑ सान॒सिम् । \newline
8. स॒त्यम् चि॒त्रश्र॑वस्तमम् चि॒त्रश्र॑वस्तमꣳ स॒त्यꣳ स॒त्यम् चि॒त्रश्र॑वस्तमम् । \newline
9. चि॒त्रश्र॑वस्तम॒मिति॑ चि॒त्रश्र॑वः - त॒म॒म् । \newline
10. मि॒त्रो जना॒न् जना᳚न् मि॒त्रो मि॒त्रो जनान्॑ । \newline
11. जनान्॑. यातयति यातयति॒ जना॒न् जनान्॑. यातयति । \newline
12. या॒त॒य॒ति॒ प्र॒जा॒नन् प्र॑जा॒नन्. या॑तयति यातयति प्रजा॒नन्न् । \newline
13. प्र॒जा॒नन् मि॒त्रो मि॒त्रः प्र॑जा॒नन् प्र॑जा॒नन् मि॒त्रः । \newline
14. प्र॒जा॒नन्निति॑ प्र - जा॒नन् । \newline
15. मि॒त्रो दा॑धार दाधार मि॒त्रो मि॒त्रो दा॑धार । \newline
16. दा॒धा॒र॒ पृ॒थि॒वीम् पृ॑थि॒वीम् दा॑धार दाधार पृथि॒वीम् । \newline
17. पृ॒थि॒वी मु॒तोत पृ॑थि॒वीम् पृ॑थि॒वी मु॒त । \newline
18. उ॒त द्याम् द्या मु॒तोत द्याम् । \newline
19. द्यामिति॒ द्याम् । \newline
20. मि॒त्रः कृ॒ष्टीः कृ॒ष्टीर् मि॒त्रो मि॒त्रः कृ॒ष्टीः । \newline
21. कृ॒ष्टी रनि॑मि॒षा ऽनि॑मिषा कृ॒ष्टीः कृ॒ष्टी रनि॑मिषा । \newline
22. अनि॑मिषा॒ ऽभ्य॑भ्य नि॑मि॒षा ऽनि॑मिषा॒ ऽभि । \newline
23. अनि॑मि॒षेत्यनि॑ - मि॒षा॒ । \newline
24. अ॒भि च॑ष्टे चष्टे अ॒भ्य॑भि च॑ष्टे । \newline
25. च॒ष्टे॒ स॒त्याय॑ स॒त्याय॑ चष्टे चष्टे स॒त्याय॑ । \newline
26. स॒त्याय॑ ह॒व्यꣳ ह॒व्यꣳ स॒त्याय॑ स॒त्याय॑ ह॒व्यम् । \newline
27. ह॒व्यम् घृ॒तव॑द् घृ॒तव॑द्ध॒ व्यꣳ ह॒व्यम् घृ॒तव॑त् । \newline
28. घृ॒तव॑द् विधेम विधेम घृ॒तव॑द् घृ॒तव॑द् विधेम । \newline
29. घृ॒तव॒दिति॑ घृ॒त - व॒त् । \newline
30. वि॒धे॒मेति॑ विधेम । \newline
31. प्र स स प्र प्र सः । \newline
32. स मि॑त्र मित्र॒ स स मि॑त्र । \newline
33. मि॒त्र॒ मर्तो॒ मर्तो॑ मित्र मित्र॒ मर्तः॑ । \newline
34. मर्तो॑ अस्त्वस्तु॒ मर्तो॒ मर्तो॑ अस्तु । \newline
35. अ॒स्तु॒ प्रय॑स्वा॒न् प्रय॑स्वा-नस्त्वस्तु॒ प्रय॑स्वान् । \newline
36. प्रय॑स्वा॒न्॒. यो यः प्रय॑स्वा॒न् प्रय॑स्वा॒न्॒. यः । \newline
37. य स्ते॑ ते॒ यो य स्ते᳚ । \newline
38. त॒ आ॒दि॒त्या॒ दि॒त्य॒ ते॒ त॒ आ॒दि॒त्य॒ । \newline
39. आ॒दि॒त्य॒ शिक्ष॑ति॒ शिक्ष॑ त्यादित्या दित्य॒ शिक्ष॑ति । \newline
40. शिक्ष॑ति व्र॒तेन॑ व्र॒तेन॒ शिक्ष॑ति॒ शिक्ष॑ति व्र॒तेन॑ । \newline
41. व्र॒तेनेति॑ व्र॒तेन॑ । \newline
42. न ह॑न्यते हन्यते॒ न न ह॑न्यते । \newline
43. ह॒न्य॒ते॒ न न ह॑न्यते हन्यते॒ न । \newline
44. न जी॑यते जीयते॒ न न जी॑यते । \newline
45. जी॒य॒ते॒ त्वोत॒ स्त्वोतो॑ जीयते जीयते॒ त्वोतः॑ । \newline
46. त्वोतो॒ न न त्वोत॒ स्त्वोतो॒ न । \newline
47. नैन॑ मेन॒म् न नैन᳚म् । \newline
48. ए॒न॒ मꣳहो ऽꣳह॑ एन मेन॒ मꣳहः॑ । \newline
49. अꣳहो॑ अश्ञो त्यश्ञो॒ त्यꣳहो ऽꣳहो॑ अश्ञोति । \newline
50. अ॒श्ञो॒ त्यन्ति॑तो॒ अन्ति॑तो अश्ञो त्यश्ञो॒ त्यन्ति॑तः । \newline
51. अन्ति॑तो॒ न नान्ति॑तो॒ अन्ति॑तो॒ न । \newline
52. न दू॒राद् दू॒रान् न न दू॒रात् । \newline
53. दू॒रादिति॑ दू॒रात् । \newline
54. यच् चि॑च् चि॒द् यद् यच् चि॑त् । \newline

\textbf{Ghana Paata } \newline

1. वि॒श्व॒च॒र्॒.ष॒ण॒ इति॑ विश्व - च॒र्॒.ष॒णे॒ । \newline
2. मि॒त्रस्य॑ चर्.षणी॒धृत॑ श्चर्.षणी॒धृतो॑ मि॒त्रस्य॑ मि॒त्रस्य॑ चर्.षणी॒धृतः॒ श्रवः॒ श्रव॑ श्चर्.षणी॒धृतो॑ मि॒त्रस्य॑ मि॒त्रस्य॑ चर्.षणी॒धृतः॒ श्रवः॑ । \newline
3. च॒र्॒.ष॒णी॒धृतः॒ श्रवः॒ श्रव॑ श्चर्.षणी॒धृत॑ श्चर्.षणी॒धृतः॒ श्रवो॑ दे॒वस्य॑ दे॒वस्य॒ श्रव॑ श्चर्.षणी॒धृत॑ श्चर्.षणी॒धृतः॒ श्रवो॑ दे॒वस्य॑ । \newline
4. च॒र्॒.ष॒णी॒धृत॒ इति॑ चर्.षणि - धृतः॑ । \newline
5. श्रवो॑ दे॒वस्य॑ दे॒वस्य॒ श्रवः॒ श्रवो॑ दे॒वस्य॑ सान॒सिꣳ सा॑न॒सिम् दे॒वस्य॒ श्रवः॒ श्रवो॑ दे॒वस्य॑ सान॒सिम् । \newline
6. दे॒वस्य॑ सान॒सिꣳ सा॑न॒सिम् दे॒वस्य॑ दे॒वस्य॑ सान॒सिम् । \newline
7. सा॒न॒सिमिति॑ सान॒सिम् । \newline
8. स॒त्यम् चि॒त्रश्र॑वस्तमम् चि॒त्रश्र॑वस्तमꣳ स॒त्यꣳ स॒त्यम् चि॒त्रश्र॑वस्तमम् । \newline
9. चि॒त्रश्र॑वस्तम॒मिति॑ चि॒त्रश्र॑वः - त॒म॒म् । \newline
10. मि॒त्रो जना॒न् जना᳚न् मि॒त्रो मि॒त्रो जनान्॑. यातयति यातयति॒ जना᳚न् मि॒त्रो मि॒त्रो जनान्॑. यातयति । \newline
11. जनान्॑. यातयति यातयति॒ जना॒न् जनान्॑. यातयति प्रजा॒नन् प्र॑जा॒नन्. या॑तयति॒ जना॒न् जनान्॑. यातयति प्रजा॒नन्न् । \newline
12. या॒त॒य॒ति॒ प्र॒जा॒नन् प्र॑जा॒नन्. या॑तयति यातयति प्रजा॒नन् मि॒त्रो मि॒त्रः प्र॑जा॒नन्. या॑तयति यातयति प्रजा॒नन् मि॒त्रः । \newline
13. प्र॒जा॒नन् मि॒त्रो मि॒त्रः प्र॑जा॒नन् प्र॑जा॒नन् मि॒त्रो दा॑धार दाधार मि॒त्रः प्र॑जा॒नन् प्र॑जा॒नन् मि॒त्रो दा॑धार । \newline
14. प्र॒जा॒नन्निति॑ प्र - जा॒नन् । \newline
15. मि॒त्रो दा॑धार दाधार मि॒त्रो मि॒त्रो दा॑धार पृथि॒वीम् पृ॑थि॒वीम् दा॑धार मि॒त्रो मि॒त्रो दा॑धार पृथि॒वीम् । \newline
16. दा॒धा॒र॒ पृ॒थि॒वीम् पृ॑थि॒वीम् दा॑धार दाधार पृथि॒वी मु॒तोत पृ॑थि॒वीम् दा॑धार दाधार पृथि॒वी मु॒त । \newline
17. पृ॒थि॒वी मु॒तोत पृ॑थि॒वीम् पृ॑थि॒वी मु॒त द्याम् द्या मु॒त पृ॑थि॒वीम् पृ॑थि॒वी मु॒त द्याम् । \newline
18. उ॒त द्याम् द्या मु॒तोत द्याम् । \newline
19. द्यामिति॒ द्याम् । \newline
20. मि॒त्रः कृ॒ष्टीः कृ॒ष्टीर् मि॒त्रो मि॒त्रः कृ॒ष्टी रनि॑मि॒षा ऽनि॑मिषा कृ॒ष्टीर् मि॒त्रो मि॒त्रः कृ॒ष्टी रनि॑मिषा । \newline
21. कृ॒ष्टी रनि॑मि॒षा ऽनि॑मिषा कृ॒ष्टीः कृ॒ष्टी रनि॑मिषा॒ ऽभ्य॑भ्य नि॑मिषा कृ॒ष्टीः कृ॒ष्टी 
रनि॑मिषा॒ ऽभि । \newline
22. अनि॑मिषा॒ ऽभ्य॑भ्य नि॑मि॒षा ऽनि॑मिषा॒ ऽभि च॑ष्टे चष्टे अ॒भ्यनि॑मि॒षा ऽनि॑मिषा॒ ऽभिच॑ष्टे । \newline
23. अनि॑मि॒षेत्यनि॑ - मि॒षा॒ । \newline
24. अ॒भि च॑ष्टे चष्टे अ॒भ्य॑भि च॑ष्टे स॒त्याय॑ स॒त्याय॑ चष्टे अ॒भ्य॑भि च॑ष्टे स॒त्याय॑ । \newline
25. च॒ष्टे॒ स॒त्याय॑ स॒त्याय॑ चष्टे चष्टे स॒त्याय॑ ह॒व्यꣳ ह॒व्यꣳ स॒त्याय॑ चष्टे चष्टे स॒त्याय॑ ह॒व्यम् । \newline
26. स॒त्याय॑ ह॒व्यꣳ ह॒व्यꣳ स॒त्याय॑ स॒त्याय॑ ह॒व्यम् घृ॒तव॑द् घृ॒तव॑ द्ध॒व्यꣳ स॒त्याय॑ स॒त्याय॑ ह॒व्यम् घृ॒तव॑त् । \newline
27. ह॒व्यम् घृ॒तव॑द् घृ॒तव॑ द्ध॒व्यꣳ ह॒व्यम् घृ॒तव॑द् विधेम विधेम घृ॒तव॑ द्ध॒व्यꣳ ह॒व्यम् घृ॒तव॑द् विधेम । \newline
28. घृ॒तव॑द् विधेम विधेम घृ॒तव॑द् घृ॒तव॑द् विधेम । \newline
29. घृ॒तव॒दिति॑ घृ॒त - व॒त् । \newline
30. वि॒धे॒मेति॑ विधेम । \newline
31. प्र स स प्र प्र स मि॑त्र मित्र॒ स प्र प्र स मि॑त्र । \newline
32. स मि॑त्र मित्र॒ स स मि॑त्र॒ मर्तो॒ मर्तो॑ मित्र॒ स स मि॑त्र॒ मर्तः॑ । \newline
33. मि॒त्र॒ मर्तो॒ मर्तो॑ मित्र मित्र॒ मर्तो॑ अस्त्वस्तु॒ मर्तो॑ मित्र मित्र॒ मर्तो॑ अस्तु । \newline
34. मर्तो॑ अस्त्वस्तु॒ मर्तो॒ मर्तो॑ अस्तु॒ प्रय॑स्वा॒न् प्रय॑स्वा,नस्तु॒ मर्तो॒ मर्तो॑ अस्तु॒ प्रय॑स्वान् । \newline
35. अ॒स्तु॒ प्रय॑स्वा॒न् प्रय॑स्वा,नस्त्वस्तु॒ प्रय॑स्वा॒न्॒. यो यः प्रय॑स्वा,नस्त्वस्तु॒ प्रय॑स्वा॒न्॒. यः । \newline
36. प्रय॑स्वा॒न्॒. यो यः प्रय॑स्वा॒न् प्रय॑स्वा॒न्॒. य स्ते॑ ते॒ यः प्रय॑स्वा॒न् प्रय॑स्वा॒न्॒. य स्ते᳚ । \newline
37. यस्ते॑ ते॒ यो यस्त॑ आदित्या दित्य ते॒ यो यस्त॑ आदित्य । \newline
38. त॒ आ॒दि॒ त्या॒दि॒त्य॒ ते॒ त॒ आ॒दि॒त्य॒ शिक्ष॑ति॒ शिक्ष॑ त्यादित्य ते त आदित्य॒ शिक्ष॑ति । \newline
39. आ॒दि॒त्य॒ शिक्ष॑ति॒ शिक्ष॑ त्यादित्या दित्य॒ शिक्ष॑ति व्र॒तेन॑ व्र॒तेन॒ शिक्ष॑ त्यादित्या दित्य॒ शिक्ष॑ति व्र॒तेन॑ । \newline
40. शिक्ष॑ति व्र॒तेन॑ व्र॒तेन॒ शिक्ष॑ति॒ शिक्ष॑ति व्र॒तेन॑ । \newline
41. व्र॒तेनेति॑ व्र॒तेन॑ । \newline
42. न ह॑न्यते हन्यते॒ न न ह॑न्यते॒ न न ह॑न्यते॒ न न ह॑न्यते॒ न । \newline
43. ह॒न्य॒ते॒ न न ह॑न्यते हन्यते॒ न जी॑यते जीयते॒ न ह॑न्यते हन्यते॒ न जी॑यते । \newline
44. न जी॑यते जीयते॒ न न जी॑यते॒ त्वोत॒ स्त्वोतो॑ जीयते॒ न न जी॑यते॒ त्वोतः॑ । \newline
45. जी॒य॒ते॒ त्वोत॒ स्त्वोतो॑ जीयते जीयते॒ त्वोतो॒ न न त्वोतो॑ जीयते जीयते॒ त्वोतो॒ न । \newline
46. त्वोतो॒ न न त्वोत॒ स्त्वोतो॒ नैन॑ मेन॒म् न त्वोत॒ स्त्वोतो॒ नैन᳚म् । \newline
47. नैन॑ मेन॒म् न नैन॒ मꣳहो ऽꣳह॑ एन॒म् न नैन॒ मꣳहः॑ । \newline
48. ए॒न॒ मꣳहो ऽꣳह॑ एन मेन॒ मꣳहो॑ अश्ञो त्यश्ञो॒ त्यꣳह॑ एन मेन॒ मꣳहो॑ अश्ञोति । \newline
49. अꣳहो॑ अश्ञो त्यश्ञो॒ त्यꣳहो ऽꣳहो॑ अश्ञो॒ त्यन्ति॑तो॒ अन्ति॑तो अश्ञो॒ त्यꣳहो ऽꣳहो॑ अश्ञो॒ त्यन्ति॑तः । \newline
50. अ॒श्ञो॒ त्यन्ति॑तो॒ अन्ति॑तो अश्ञो त्यश्ञो॒ त्यन्ति॑तो॒ न नान्ति॑तो अश्ञो त्यश्ञो॒ त्यन्ति॑तो॒ न । \newline
51. अन्ति॑तो॒ न नान्ति॑तो॒ अन्ति॑तो॒ न दू॒राद् दू॒रान् नान्ति॑तो॒ अन्ति॑तो॒ न दू॒रात् । \newline
52. न दू॒राद् दू॒रान् न न दू॒रात् । \newline
53. दू॒रादिति॑ दू॒रात् । \newline
54. यच् चि॑च् चि॒द् यद् यच् चि॒द्धि हि चि॒द् यद् यच् चि॒द्धि । \newline
\pagebreak
\markright{ TS 3.4.11.6  \hfill https://www.vedavms.in \hfill}

\section{ TS 3.4.11.6 }

\textbf{TS 3.4.11.6 } \newline
\textbf{Samhita Paata} \newline

च्चि॒द्धि ते॒ विशो॑ यथा॒ प्रदे॑व वरुण व्र॒तं । मि॒नी॒मसि॒ द्यवि॑द्यवि ॥यत् किञ्चे॒दं ॅव॑रुण॒ दैव्ये॒ जने॑ऽभिद्रो॒हं म॑नु॒ष्या᳚श्चरा॑मसि । अचि॑त्ती॒यत् तव॒ धर्मा॑ युयोपि॒ममा न॒स्तस्मा॒ देन॑सो देव रीरिषः ॥ कि॒त॒वासो॒ यद्रि॑ रि॒पुर्न दी॒वि यद्वा॑ घा स॒त्य मु॒तयन्न वि॒द्म । सर्वा॒ ता विष्य॑ शिथि॒रे ( ) व॑ दे॒वाथा॑ ते स्याम वरुण प्रि॒यासः॑ ॥ \newline

\textbf{Pada Paata} \newline

चि॒त् । हि । ते॒ । विशः॑ । य॒था॒ । प्रेति॑ । दे॒व॒ । व॒रु॒ण॒ । व्र॒तम् ॥ मि॒नी॒मसि॑ । द्यवि॑द्य॒वीति॒ द्यवि॑ - द्य॒वि॒ ॥ यत् । किम् । च॒ । इ॒दम् । व॒रु॒ण॒ । दैव्ये᳚ । जने᳚ । अ॒भि॒द्रो॒हमित्य॑भि - द्रो॒हम् । म॒नु॒ष्याः᳚ । चरा॑मसि ॥ अचि॑त्ती । यत् । तव॑ । धर्मा᳚ । यु॒यो॒पि॒म । मा । नः॒ । तस्मा᳚त् । एन॑सः । दे॒व॒ । री॒रि॒षः॒ ॥ कि॒त॒वासः॑ । यत् । रि॒रि॒पुः । न । दी॒वि । यत् । वा॒ । घ॒ । स॒त्यम् । उ॒त । यत् । न । वि॒द्म ॥ सर्वा᳚ । ता । वीति॑ । स्य॒ । शि॒थि॒रा ( ) । इ॒व॒ । दे॒व॒ । अथ॑ । ते॒ । स्या॒म॒ । व॒रु॒ण॒ । प्रि॒यासः॑ ॥  \newline


\textbf{Krama Paata} \newline

चि॒द्धि । हि ते᳚ । ते॒ विशः॑ । विशो॑ यथा । य॒था॒ प्र । प्र दे॑व । दे॒व॒ व॒रु॒ण॒ । व॒रु॒ण॒ व्र॒तम् । व्र॒तमिति॑ व्र॒तम् ॥ मी॒नी॒मसि॒ द्यवि॑द्यवि । द्यवि॑द्य॒वीति॒ द्यवि॑ - द्य॒वि॒ ॥ यत् किम् । किम् च॑ । चे॒दम् । इ॒दं ॅव॑रुण । व॒रु॒ण॒ दैव्ये᳚ । दैव्ये॒ जने᳚ । जने॑ ऽभिद्रो॒हम् । अ॒भि॒द्रो॒हम् म॑नु॒ष्याः᳚ । अ॒भि॒द्रो॒हमित्य॑भि - द्रो॒हम् । म॒नु॒ष्या᳚श्चरा॑मसि । चरा॑म॒सीति॒ चरा॑मसि ॥ अचि॑त्ती॒ यत् । यत् तव॑ । तव॒ धर्मा᳚ । धर्मा॑ युयोपि॒म । यु॒यो॒पि॒म मा । मा नः॑ । न॒ स्त॒स्मा᳚त् । तस्मा॒देन॑सः । एन॑सो देव । दे॒व॒ री॒रि॒षः॒ । री॒रि॒ष॒ इति॑ रीरिषः ॥ कि॒त॒वासो॒ यत् । यद् रि॑रि॒पुः । रि॒रि॒पुर् न । न दी॒वि । दी॒वि यत् । यद् वा᳚ । वा॒ घ॒ । घा॒ स॒त्यम् । स॒त्यमु॒त । उ॒त यत् । यन् न । न वि॒द्म । वि॒द्मेति॑ वि॒द्म ॥ सर्वा॒ ता । ता वि । वि ष्य॑ । स्य॒ शि॒थि॒रा ( ) । शि॒थि॒रेव॑ । इ॒व॒ दे॒व॒ । दे॒वाथ॑ । अथा॑ ते । ते॒ स्या॒म॒ । स्या॒म॒ व॒रु॒ण॒ । व॒रु॒ण॒ प्रि॒यासः॑ । प्रि॒यास॒ इति॑ प्रि॒यासः॑ । \newline

\textbf{Jatai Paata} \newline

1. चि॒द्धि हि चि॑च् चि॒द्धि । \newline
2. हि ते॑ ते॒ हि हि ते᳚ । \newline
3. ते॒ विशो॒ विश॑ स्ते ते॒ विशः॑ । \newline
4. विशो॑ यथा यथा॒ विशो॒ विशो॑ यथा । \newline
5. य॒था॒ प्र प्र य॑था यथा॒ प्र । \newline
6. प्र दे॑व देव॒ प्र प्र दे॑व । \newline
7. दे॒व॒ व॒रु॒ण॒ व॒रु॒ण॒ दे॒व॒ दे॒व॒ व॒रु॒ण॒ । \newline
8. व॒रु॒ण॒ व्र॒तं ॅव्र॒तं ॅव॑रुण वरुण व्र॒तम् । \newline
9. व्र॒तमिति॑ व्र॒तम् । \newline
10. मि॒नी॒मसि॒ द्यवि॑द्यवि॒ द्यवि॑द्यवि मिनी॒मसि॑ मिनी॒मसि॒ द्यवि॑द्यवि । \newline
11. द्यवि॑द्य॒वीति॒ द्यवि॑ - द्य॒वि॒ । \newline
12. यत् किम् किं ॅयद् यत् किम् । \newline
13. किम् च॑ च॒ किम् किम् च॑ । \newline
14. चे॒द मि॒दम् च॑ चे॒दम् । \newline
15. इ॒दं ॅव॑रुण वरुणे॒द मि॒दं ॅव॑रुण । \newline
16. व॒रु॒ण॒ दैव्ये॒ दैव्ये॑ वरुण वरुण॒ दैव्ये᳚ । \newline
17. दैव्ये॒ जने॒ जने॒ दैव्ये॒ दैव्ये॒ जने᳚ । \newline
18. जने॑ ऽभिद्रो॒ह म॑भिद्रो॒हम् जने॒ जने॑ ऽभिद्रो॒हम् । \newline
19. अ॒भि॒द्रो॒हम् म॑नु॒ष्या॑ मनु॒ष्या॑ अभिद्रो॒ह म॑भिद्रो॒हम् म॑नु॒ष्याः᳚ । \newline
20. अ॒भि॒द्रो॒हमित्य॑भि - द्रो॒हम् । \newline
21. म॒नु॒ष्या᳚श्चरा॑मसि॒ चरा॑मसि मनु॒ष्या॑ मनु॒ष्या᳚ श्चरा॑मसि । \newline
22. चरा॑म॒सीति॒ चरा॑मसि । \newline
23. अचि॑त्ती॒ यद् यदचि॒ त्त्यचि॑त्ती॒ यत् । \newline
24. यत् तव॒ तव॒ यद् यत् तव॑ । \newline
25. तव॒ धर्मा॒ धर्मा॒ तव॒ तव॒ धर्मा᳚ । \newline
26. धर्मा॑ युयोपि॒म यु॑योपि॒म धर्मा॒ धर्मा॑ युयोपि॒म । \newline
27. यु॒यो॒पि॒म मा मा यु॑योपि॒म यु॑योपि॒म मा । \newline
28. मा नो॑ नो॒ मा मा नः॑ । \newline
29. न॒ स्तस्मा॒त् तस्मा᳚न् नो न॒ स्तस्मा᳚त् । \newline
30. तस्मा॒ देन॑स॒ एन॑स॒ स्तस्मा॒त् तस्मा॒ देन॑सः । \newline
31. एन॑सो देव दे॒वैन॑स॒ एन॑सो देव । \newline
32. दे॒व॒ री॒रि॒षो॒ री॒रि॒षो॒ दे॒व॒ दे॒व॒ री॒रि॒षः॒ । \newline
33. री॒रि॒ष॒ इति॑ रीरिषः । \newline
34. कि॒त॒वासो॒ यद् यत् कि॑त॒वासः॑ कित॒वासो॒ यत् । \newline
35. यद् रि॑रि॒पू रि॑रि॒पुर् यद् यद् रि॑रि॒पुः । \newline
36. रि॒रि॒पुर् न न रि॑रि॒पू रि॑रि॒पुर् न । \newline
37. न दी॒वि दी॒वि न न दी॒वि । \newline
38. दी॒वि यद् यद् दी॒वि दी॒वि यत् । \newline
39. यद् वा॑ वा॒ यद् यद् वा᳚ । \newline
40. वा॒ घ॒ घ॒ वा॒ वा॒ घ॒ । \newline
41. घा॒ स॒त्यꣳ स॒त्यम् घ॑ घा स॒त्यम् । \newline
42. स॒त्य मु॒तोत स॒त्यꣳ स॒त्य मु॒त । \newline
43. उ॒त यद् यदु॒ तोत यत् । \newline
44. यन् न न यद् यन् न । \newline
45. न वि॒द्म वि॒द्म न न वि॒द्म । \newline
46. वि॒द्मेति॑ वि॒द्म । \newline
47. सर्वा॒ ता ता सर्वा॒ सर्वा॒ ता । \newline
48. ता वि वि ता ता वि । \newline
49. वि ष्य॑ स्य॒ वि वि ष्य॑ । \newline
50. स्य॒ शि॒थि॒रा शि॑थि॒रा स्य॑ स्य शिथि॒रा । \newline
51. शि॒थि॒ रेवे॑व शिथि॒रा शि॑थि॒ रेव॑ । \newline
52. इ॒व॒ दे॒व॒ दे॒वे॒ वे॒व॒ दे॒व॒ । \newline
53. दे॒वा थाथ॑ देव दे॒वाथ॑ । \newline
54. अथा॑ ते ते॒ अथाथा॑ ते । \newline
55. ते॒ स्या॒म॒ स्या॒म॒ ते॒ ते॒ स्या॒म॒ । \newline
56. स्या॒म॒ व॒रु॒ण॒ व॒रु॒ण॒ स्या॒म॒ स्या॒म॒ व॒रु॒ण॒ । \newline
57. व॒रु॒ण॒ प्रि॒यासः॑ प्रि॒यासो॑ वरुण वरुण प्रि॒यासः॑ । \newline
58. प्रि॒यास॒ इति॑ प्रि॒यासः॑ । \newline

\textbf{Ghana Paata } \newline

1. चि॒द्धि हि चि॑च् चि॒द्धि ते॑ ते॒ हि चि॑च् चि॒द्धि ते᳚ । \newline
2. हि ते॑ ते॒ हि हि ते॒ विशो॒ विश॑ स्ते॒ हि हि ते॒ विशः॑ । \newline
3. ते॒ विशो॒ विश॑ स्ते ते॒ विशो॑ यथा यथा॒ विश॑ स्ते ते॒ विशो॑ यथा । \newline
4. विशो॑ यथा यथा॒ विशो॒ विशो॑ यथा॒ प्र प्र य॑था॒ विशो॒ विशो॑ यथा॒ प्र । \newline
5. य॒था॒ प्र प्र य॑था यथा॒ प्र दे॑व देव॒ प्र य॑था यथा॒ प्र दे॑व । \newline
6. प्र दे॑व देव॒ प्र प्र दे॑व वरुण वरुण देव॒ प्र प्र दे॑व वरुण । \newline
7. दे॒व॒ व॒रु॒ण॒ व॒रु॒ण॒ दे॒व॒ दे॒व॒ व॒रु॒ण॒ व्र॒तम् ॅव्र॒तम् ॅव॑रुण देव देव वरुण व्र॒तम् । \newline
8. व॒रु॒ण॒ व्र॒तम् ॅव्र॒तम् ॅव॑रुण वरुण व्र॒तम् । \newline
9. व्र॒तमिति॑ व्र॒तम् । \newline
10. मि॒नी॒मसि॒ द्यवि॑द्यवि॒ द्यवि॑द्यवि मिनी॒मसि॑ मिनी॒मसि॒ द्यवि॑द्यवि । \newline
11. द्यवि॑द्य॒वीति॒ द्यवि॑ - द्य॒वि॒ । \newline
12. यत् किम् किम् ॅयद् यत् किम् च॑ च॒ किम् ॅयद् यत् किम् च॑ । \newline
13. किम् च॑ च॒ किम् किम् चे॒द मि॒दम् च॒ किम् किम् चे॒दम् । \newline
14. चे॒द मि॒दम् च॑ चे॒दम् ॅव॑रुण वरुणे॒दम् च॑ चे॒दम् ॅव॑रुण । \newline
15. इ॒दम् ॅव॑रुण वरुणे॒द मि॒दम् ॅव॑रुण॒ दैव्ये॒ दैव्ये॑ वरुणे॒द मि॒दम् ॅव॑रुण॒ दैव्ये᳚ । \newline
16. व॒रु॒ण॒ दैव्ये॒ दैव्ये॑ वरुण वरुण॒ दैव्ये॒ जने॒ जने॒ दैव्ये॑ वरुण वरुण॒ दैव्ये॒ जने᳚ । \newline
17. दैव्ये॒ जने॒ जने॒ दैव्ये॒ दैव्ये॒ जने॑ ऽभिद्रो॒ह म॑भिद्रो॒हम् जने॒ दैव्ये॒ दैव्ये॒ जने॑ ऽभिद्रो॒हम् । \newline
18. जने॑ ऽभिद्रो॒ह म॑भिद्रो॒हम् जने॒ जने॑ ऽभिद्रो॒हम् म॑नु॒ष्या॑ मनु॒ष्या॑ अभिद्रो॒हम् जने॒ जने॑ ऽभिद्रो॒हम् म॑नु॒ष्याः᳚ । \newline
19. अ॒भि॒द्रो॒हम् म॑नु॒ष्या॑ मनु॒ष्या॑ अभिद्रो॒ह म॑भिद्रो॒हम् म॑नु॒ष्या᳚ श्चरा॑मसि॒ चरा॑मसि मनु॒ष्या॑ अभिद्रो॒ह म॑भिद्रो॒हम् म॑नु॒ष्या᳚ श्चरा॑मसि । \newline
20. अ॒भि॒द्रो॒हमित्य॑भि - द्रो॒हम् । \newline
21. म॒नु॒ष्या᳚ श्चरा॑मसि॒ चरा॑मसि मनु॒ष्या॑ मनु॒ष्या᳚ श्चरा॑मसि । \newline
22. चरा॑म॒सीति॒ चरा॑मसि । \newline
23. अचि॑त्ती॒ यद् य दचि॒ त्त्यचि॑त्ती॒ यत् तव॒ तव॒ य दचि॒ त्त्यचि॑त्ती॒ यत् तव॑ । \newline
24. यत् तव॒ तव॒ यद् यत् तव॒ धर्मा॒ धर्मा॒ तव॒ यद् यत् तव॒ धर्मा᳚ । \newline
25. तव॒ धर्मा॒ धर्मा॒ तव॒ तव॒ धर्मा॑ युयोपि॒म यु॑योपि॒म धर्मा॒ तव॒ तव॒ धर्मा॑ युयोपि॒म । \newline
26. धर्मा॑ युयोपि॒म यु॑योपि॒म धर्मा॒ धर्मा॑ युयोपि॒म मा मा यु॑योपि॒म धर्मा॒ धर्मा॑ युयोपि॒म मा । \newline
27. यु॒यो॒पि॒म मा मा यु॑योपि॒म यु॑योपि॒म मा नो॑ नो॒ मा यु॑योपि॒म यु॑योपि॒म मा नः॑ । \newline
28. मा नो॑ नो॒ मा मा न॒ स्तस्मा॒त् तस्मा᳚न् नो॒ मा मा न॒ स्तस्मा᳚त् । \newline
29. न॒ स्तस्मा॒त् तस्मा᳚न् नो न॒ स्तस्मा॒ देन॑स॒ एन॑स॒ स्तस्मा᳚न् नो न॒ स्तस्मा॒ देन॑सः । \newline
30. तस्मा॒ देन॑स॒ एन॑स॒ स्तस्मा॒त् तस्मा॒ देन॑सो देव दे॒वैन॑स॒ स्तस्मा॒त् तस्मा॒ देन॑सो देव । \newline
31. एन॑सो देव दे॒वैन॑स॒ एन॑सो देव रीरिषो रीरिषो दे॒वैन॑स॒ एन॑सो देव रीरिषः । \newline
32. दे॒व॒ री॒रि॒षो॒ री॒रि॒षो॒ दे॒व॒ दे॒व॒ री॒रि॒षः॒ । \newline
33. री॒रि॒ष॒ इति॑ रीरिषः । \newline
34. कि॒त॒वासो॒ यद् यत् कि॑त॒वासः॑ कित॒वासो॒ यद् रि॑रि॒पू रि॑रि॒पुर् यत् कि॑त॒वासः॑ कित॒वासो॒ यद् रि॑रि॒पुः । \newline
35. यद् रि॑रि॒पू रि॑रि॒पुर् यद् यद् रि॑रि॒पुर् न न रि॑रि॒पुर् यद् यद् रि॑रि॒पुर् न । \newline
36. रि॒रि॒पुर् न न रि॑रि॒पू रि॑रि॒पुर् न दी॒वि दी॒वि न रि॑रि॒पू रि॑रि॒पुर् न दी॒वि । \newline
37. न दी॒वि दी॒वि न न दी॒वि यद् यद् दी॒वि न न दी॒वि यत् । \newline
38. दी॒वि यद् यद् दी॒वि दी॒वि यद् वा॑ वा॒ यद् दी॒वि दी॒वि यद् वा᳚ । \newline
39. यद् वा॑ वा॒ यद् यद् वा॑ घ घ वा॒ यद् यद् वा॑ घ । \newline
40. वा॒ घ॒ घ॒ वा॒ वा॒ घा॒ स॒त्यꣳ स॒त्यम् घ॑ वा वा घा स॒त्यम् । \newline
41. घा॒ स॒त्यꣳ स॒त्यम् घ॑ घा स॒त्य मु॒तोत स॒त्यम् घ॑ घा स॒त्य मु॒त । \newline
42. स॒त्य मु॒तोत स॒त्यꣳ स॒त्य मु॒त यद् यदु॒त स॒त्यꣳ स॒त्य मु॒त यत् । \newline
43. उ॒त यद् य दु॒तोत यन् न न य दु॒तोत यन् न । \newline
44. यन् न न यद् यन् न वि॒द्म वि॒द्म न यद् यन् न वि॒द्म । \newline
45. न वि॒द्म वि॒द्म न न वि॒द्म । \newline
46. वि॒द्मेति॑ वि॒द्म । \newline
47. सर्वा॒ ता ता सर्वा॒ सर्वा॒ ता वि वि ता सर्वा॒ सर्वा॒ ता वि । \newline
48. ता वि वि ता ता वि ष्य॑ स्य॒ वि ता ता वि ष्य॑ । \newline
49. वि ष्य॑ स्य॒ वि वि ष्य॑ शिथि॒रा शि॑थि॒रा स्य॒ वि वि ष्य॑ शिथि॒रा । \newline
50. स्य॒ शि॒थि॒रा शि॑थि॒रा स्य॑ स्य शिथि॒रेवे॑व शिथि॒रा स्य॑ स्य शिथि॒रेव॑ । \newline
51. शि॒थि॒ रेवे॑व शिथि॒रा शि॑थि॒ रेव॑ देव देवेव शिथि॒रा शि॑थि॒ रेव॑ देव । \newline
52. इ॒व॒ दे॒व॒ दे॒वे॒ वे॒व॒ दे॒वा थाथ॑ देवे वेव दे॒वाथ॑ । \newline
53. दे॒वा थाथ॑ देव दे॒वा था॑ ते ते॒ अथ॑ देव दे॒वा था॑ ते । \newline
54. अथा॑ ते ते॒ अथाथा॑ ते स्याम स्याम ते॒ अथाथा॑ ते स्याम । \newline
55. ते॒ स्या॒म॒ स्या॒म॒ ते॒ ते॒ स्या॒म॒ व॒रु॒ण॒ व॒रु॒ण॒ स्या॒म॒ ते॒ ते॒ स्या॒म॒ व॒रु॒ण॒ । \newline
56. स्या॒म॒ व॒रु॒ण॒ व॒रु॒ण॒ स्या॒म॒ स्या॒म॒ व॒रु॒ण॒ प्रि॒यासः॑ प्रि॒यासो॑ वरुण स्याम स्याम वरुण प्रि॒यासः॑ । \newline
57. व॒रु॒ण॒ प्रि॒यासः॑ प्रि॒यासो॑ वरुण वरुण प्रि॒यासः॑ । \newline
58. प्रि॒यास॒ इति॑ प्रि॒यासः॑ । \newline
\pagebreak


\end{document}