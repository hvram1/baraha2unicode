\documentclass[17pt]{extarticle}
\usepackage{babel}
\usepackage{fontspec}
\usepackage{polyglossia}
\usepackage{extsizes}

\usepackage{color}   %May be necessary if you want to color links
\usepackage{hyperref}
\hypersetup{
    colorlinks=true, %set true if you want colored links
    linktoc=all,     %set to all if you want both sections and subsections linked
    linkcolor=black,  %choose some color if you want links to stand out
}

\setmainlanguage{sanskrit}
\setotherlanguages{english} %% or other languages
\setlength{\parindent}{0pt}
\pagestyle{myheadings}
\newfontfamily\devanagarifont[Script=Devanagari]{AdishilaVedic}
\renewcommand{\theHsection}{\thepart.section.\thesection}

\newcommand{\VAR}[1]{}
\newcommand{\BLOCK}[1]{}




\begin{document}
\begin{titlepage}
    \begin{center}
 
\begin{sanskrit}
    { \Large
    कृष्ण यजुर्वेदीय तैत्तिरीय संहिता,पद,जटा,घन पाठः 
    }
    \\
    \vspace{2.5cm}
    \mbox{ \Large
    7.5      सप्तमकाण्डे पञ्चमः प्रश्नः - सत्रविशेषाभिधानं   }
\end{sanskrit}
\end{center}

\end{titlepage}
\tableofcontents
\phantomsection
\pagebreak

\markright{ TS 7.5.1.1  \hfill https://www.vedavms.in \hfill}

\section{ TS 7.5.1.1 }

\textbf{TS 7.5.1.1 } \newline
\textbf{Samhita Paata} \newline

गावो॒ वा ए॒तथ् स॒त्र-मा॑सताशृ॒ङ्गाः स॒तीः शृङ्गा॑णि नो जायन्ता॒ इति॒ कामे॑न॒ तासां॒ दश॒मासा॒ निष॑ण्णा॒ आस॒न्नथ॒ शृङ्गा᳚ण्यजायन्त॒ ता उद॑तिष्ठ॒न्नरा॒थ्स्मेत्यथ॒ यासां॒ नाजा॑यन्त॒ ताः सं॑ॅवथ्स॒र-मा॒प्त्वोद॑तिष्ठ॒ -न्नरा॒थ्स्मेति॒ यासां॒ चाजा॑यन्त॒ यासां᳚ च॒ न ता उ॒भयी॒रु-द॑तिष्ठ॒-न्नरा॒थ्स्मेति॑ गोस॒त्रं ॅवै - [  ] \newline

\textbf{Pada Paata} \newline

गावः॑ । वै । ए॒तत् । स॒त्रम् । आ॒स॒त॒ । अ॒शृ॒ङ्गाः । स॒तीः । शृङ्गा॑णि । नः॒ । जा॒य॒न्तै॒ । इति॑ । कामे॑न । तासा᳚म् । दश॑ । मासाः᳚ । निष॑ण्णा॒ इति॒ नि - स॒न्नाः॒ । आसन्न्॑ । अथ॑ । शृङ्गा॑णि । अ॒जा॒य॒न्त॒ । ताः । उदिति॑ । अ॒ति॒ष्ठ॒न्न् । अरा᳚थ्स्म । इति॑ । अथ॑ । यासा᳚म् । न । अजा॑यन्त । ताः । सं॒ॅव॒थ्स॒रमिति॑ सं -  व॒थ्स॒रम् । आ॒प्त्वा । उदिति॑ । अ॒ति॒ष्ठ॒न्न् । अरा᳚थ्स्म । इति॑ । यासा᳚म् । च॒ । अजा॑यन्त । यासा᳚म् । च॒ । न । ताः । उ॒भयीः᳚ । उदिति॑ । अ॒ति॒ष्ठ॒न्न् । अरा᳚थ्स्म । इति॑ । गो॒स॒त्रमिति॑ गो - स॒त्रम् । वै ।  \newline




\markright{ TS 7.5.1.2  \hfill https://www.vedavms.in \hfill}

\section{ TS 7.5.1.2 }

\textbf{TS 7.5.1.2 } \newline
\textbf{Samhita Paata} \newline

सं॑ॅवथ्स॒रो य ए॒वं ॅवि॒द्वाꣳसः॑ संॅवथ्स॒र-मु॑प॒यन्त्यृ॑द्ध्नु॒वन्त्ये॒व तस्मा᳚त् तूप॒रा वार्.षि॑कौ॒ मासौ॒ पर्त्वा॑ चरति स॒त्राभि॑जितꣳ॒॒ह्य॑स्यै॒ तस्मा᳚थ् संॅवथ्सर॒सदो॒ यत् किं च॑ गृ॒हे क्रि॒यते॒ तदा॒प्त-मव॑रुद्ध-म॒भिजि॑तं क्रियते समु॒द्रं ॅवा ए॒ते प्र प्ल॑वन्ते॒ ये सं॑ॅवथ्स॒रमु॑प॒यन्ति॒ यो वै स॑मु॒द्रस्य॑ पा॒रं न पश्य॑ति॒ न वै स तत॒ उदे॑ति संॅवथ्स॒रो - [  ] \newline

\textbf{Pada Paata} \newline

सं॒ॅव॒थ्स॒र इति॑ सं - व॒थ्स॒रः । ये । ए॒वम् । वि॒द्वाꣳसः॑ । सं॒ॅव॒थ्स॒रमिति॑ सं - व॒थ्स॒रम् । उ॒प॒यन्तीत्यु॑प - यन्ति॑ । ऋ॒द्ध्नु॒वन्ति॑ । ए॒व । तस्मा᳚त् । तू॒प॒रा । वार्.षि॑कौ । मासौ᳚ । पर्त्वा᳚ । च॒र॒ति॒ । स॒त्राभि॑जित॒मिति॑ स॒त्र - अ॒भि॒जि॒त॒म् । हि । अ॒स्यै॒ । तस्मा᳚त् । सं॒ॅव॒थ्स॒र॒सद॒ इति॑ संॅवथ्सर - सदः॑ । यत् । किम् । च॒ । गृ॒हे । क्रि॒यते᳚ । तत् । आ॒प्तम् । अव॑रुद्ध॒मित्यव॑-रु॒द्ध॒म् । अ॒भिजि॑त॒मित्य॒भि - जि॒त॒म् । क्रि॒य॒ते॒ । स॒मु॒द्रम् । वै । ए॒ते । प्रेति॑ । प्ल॒व॒न्ते॒ । ये । सं॒ॅव॒थ्स॒रमिति॑ सं - व॒थ्स॒रम् । उ॒प॒यन्तीत्यु॑प-यन्ति॑ । यः । वै । स॒मु॒द्रस्य॑ । पा॒रम् । न । पश्य॑ति । न । वै । सः । ततः॑ । उदिति॑ । ए॒ति॒ । सं॒ॅव॒थ्स॒र इति॑ सं - व॒थ्स॒रः ।  \newline




\markright{ TS 7.5.1.3  \hfill https://www.vedavms.in \hfill}

\section{ TS 7.5.1.3 }

\textbf{TS 7.5.1.3 } \newline
\textbf{Samhita Paata} \newline

वै स॑मु॒द्रस्तस्यै॒तत् पा॒रं ॅयद॑तिरा॒त्रौ य ए॒वं ॅवि॒द्वासः॑ संॅवथ्स॒र-मु॑प॒यन्त्यना᳚र्ता ए॒वोदृचं॑ गच्छन्ती॒यं ॅवै पूर्वो॑ऽतिरा॒त्रो॑ ऽसावुत्त॑रो॒ मनः॒ पूर्वो॒ वागुत्त॑रः प्रा॒णः पूर्वो॑ऽपा॒न उत्त॑रः प्र॒रोध॑नं॒ पूर्व॑ उ॒दय॑न॒मुत्त॑रो॒ ज्योति॑ष्टोमो वैश्वान॒रो॑ ऽतिरा॒त्रो भ॑वति॒ ज्योति॑रे॒व पु॒रस्ता᳚द्दधते सुव॒र्गस्य॑ लो॒कस्या-नु॑ख्यात्यै चतुर्विꣳ॒॒शः प्रा॑य॒णीयो॑ भवति॒ चतु॑र्विꣳशति-रर्द्धमा॒साः - [  ] \newline

\textbf{Pada Paata} \newline

वै । स॒मु॒द्रः । तस्य॑ । ए॒तत् । पा॒रम् । यत् । अ॒ति॒रा॒त्रावित्य॑ति-रा॒त्रौ । ये । ए॒वम् । वि॒द्वाꣳसः॑ । सं॒ॅव॒थ्स॒रमिति॑ सं - व॒थ्स॒रम् । उ॒प॒यन्तीत्यु॑प - यन्ति॑ । अना᳚र्ताः । ए॒व । उ॒दृच॒मित्यु॑त् - ऋच᳚म् । ग॒च्छ॒न्ति॒ । इ॒यम् । वै । पूर्वः॑ । अ॒ति॒रा॒त्र इत्य॑ति - रा॒त्रः । अ॒सौ । उत्त॑र॒ इत्युत् - त॒रः॒ । मनः॑ । पूर्वः॑ । वाक् । उत्त॑र॒ इत्युत् - त॒रः॒ । प्रा॒ण इति॑ प्र -अ॒नः । पूर्वः॑ । अ॒पा॒न इत्य॑प -   अ॒नः । उत्त॑र॒ इत्युत् -त॒रः॒ । प्र॒रोध॑न॒मिति॑ प्र - रोध॑नम् । पूर्वः॑ । उ॒दय॑न॒मित्यु॑त्-अय॑नम् । उत्त॑र॒ इत्युत् - त॒रः॒ । ज्योति॑ष्टोम॒ इति॒ ज्योतिः॑-स्तो॒मः॒ । वै॒श्वा॒न॒रः । अ॒ति॒रा॒त्र इत्य॑ति - रा॒त्रः । भ॒व॒ति॒ । ज्योतिः॑ । ए॒व । पु॒रस्ता᳚त् । द॒ध॒ते॒ । सु॒व॒र्गस्येति॑ सुवः - गस्य॑ । लो॒कस्य॑ । अनु॑ख्यात्या॒ इत्यनु॑ - ख्या॒त्यै॒ । च॒तु॒र्विꣳ॒॒श इति॑ चतुः - विꣳ॒॒शः । प्रा॒य॒णीय॒ इति॑ प्र - अ॒य॒नीयः॑ । भ॒व॒ति॒ । चतु॑र्विꣳशति॒रिति॒ चतुः॑ - विꣳ॒॒श॒तिः॒ । अ॒द्‌र्ध॒मा॒सा इत्य॑द्‌र्ध - मा॒साः ।  \newline




\markright{ TS 7.5.1.4  \hfill https://www.vedavms.in \hfill}

\section{ TS 7.5.1.4 }

\textbf{TS 7.5.1.4 } \newline
\textbf{Samhita Paata} \newline

सं॑ॅवथ्स॒रः प्र॒यन्त॑ ए॒व सं॑ॅवथ्स॒रे प्रति॑ तिष्ठन्ति॒ तस्य॒ त्रीणि॑ च श॒तानि॑ ष॒ष्टिश्च॑ स्तो॒त्रीया॒स्ताव॑तीः संॅवथ्स॒रस्य॒ रात्र॑य उ॒भे ए॒व सं॑ॅवथ्स॒रस्य॑ रू॒पे आ᳚प्नुवन्ति॒ ते सꣳस्थि॑त्या॒ अरि॑ष्ट्या॒ उत्त॑रै॒रहो॑भिश्चरन्ति षड॒हा भ॑वन्ति॒ षड् वा ऋ॒तवः॑ संॅवथ्स॒र ऋ॒तुष्वे॒व सं॑ॅवथ्स॒रे प्रति॑ तिष्ठन्ति॒ गौश्चाऽऽ*यु॑श्च मद्ध्य॒तः स्तोमौ॑ भवतः संॅवथ्स॒रस्यै॒व तन्मि॑थु॒नं म॑द्ध्य॒तो -[  ] \newline

\textbf{Pada Paata} \newline

सं॒ॅव॒थ्स॒र इति॑ सं - व॒थ्स॒रः । प्र॒यन्त॒ इति॑ प्र - यन्तः॑ । ए॒व । सं॒ॅव॒थ्स॒र इति॑ सं - व॒थ्स॒रे । प्रतीति॑ । ति॒ष्ठ॒न्ति॒ । तस्य॑ । त्रीणि॑ । च॒ । श॒तानि॑ । ष॒ष्टिः । च॒ । स्तो॒त्रीयाः᳚ । ताव॑तीः । सं॒ॅव॒थ्स॒रस्येति॑ सं - व॒थ्स॒रस्य॑ । रात्र॑यः । उ॒भे इति॑ । ए॒व । सं॒ॅव॒थ्स॒रस्येति॑ सं - व॒थ्स॒रस्य॑ । रू॒पे इति॑ । आ॒प्नु॒व॒न्ति॒ । ते । सꣳस्थि॑त्या॒ इति॒ सं-स्थि॒त्यै॒ । अरि॑ष्ट्यै । उत्त॑रै॒रित्युत् - त॒रैः॒ । अहो॑भि॒रित्यहः॑-भिः॒ । च॒र॒न्ति॒ । ष॒ड॒हा इति॑ षट् - अ॒हाः । भ॒व॒न्ति॒ । षट् । वै । ऋ॒तवः॑ । सं॒ॅव॒थ्स॒र इति॑ सं - व॒थ्स॒रः । ऋ॒तुषु॑ । ए॒व । सं॒ॅव॒थ्स॒र इति॑ सं-व॒थ्स॒रे । प्रतीति॑ । ति॒ष्ठ॒न्ति॒ । गौः । च॒ । आयुः॑ । च॒ । म॒द्ध्य॒तः । स्तोमौ᳚ । भ॒व॒तः॒ । सं॒ॅव॒थ्स॒रस्येति॑ सं - व॒थ्स॒रस्य॑ । ए॒व । तत् । मि॒थु॒नम् । म॒द्ध्य॒तः ।  \newline




\markright{ TS 7.5.1.5  \hfill https://www.vedavms.in \hfill}

\section{ TS 7.5.1.5 }

\textbf{TS 7.5.1.5 } \newline
\textbf{Samhita Paata} \newline

द॑धति प्र॒जन॑नाय॒ ज्योति॑र॒भितो॑ भवति वि॒मोच॑नमे॒व तच्छन्दाꣳ॑स्ये॒व तद्-वि॒मोकं॑ ॅय॒न्त्यथो॑ उभ॒यतो᳚ज्योतिषै॒व ष॑ड॒हेन॑ सुव॒र्गं ॅलो॒कं ॅय॑न्ति ब्रह्मवा॒दिनो॑ वद॒न्त्यास॑ते॒ केन॑ य॒न्तीति॑ देव॒याने॑न प॒थेति॑ ब्रूया॒च्छन्दाꣳ॑सि॒ वै दे॑व॒यानः॒ पन्था॑ गाय॒त्री त्रि॒ष्टुब्-जग॑ती॒ज्योति॒र्वै गा॑य॒त्री गौस्त्रि॒ष्टुगायु॒र्जग॑ती॒ यदे॒ते स्तोमा॒ भव॑न्ति देव॒याने॑नै॒व - [  ] \newline

\textbf{Pada Paata} \newline

द॒ध॒ति॒ । प्र॒जन॑ना॒येति॑ प्र - जन॑नाय । ज्योतिः॑ । अ॒भितः॑ । भ॒व॒ति॒ । वि॒मोच॑न॒मिति॑ वि - मोच॑नम् । ए॒व । तत् । छन्दाꣳ॑सि । ए॒व । तत् । वि॒मोक॒मिति॑ वि - मोक᳚म् । य॒न्ति॒ । अथो॒ इति॑ । उ॒भ॒यतो᳚ज्योति॒षेत्यु॑भ॒यतः॑-ज्यो॒ति॒षा॒ । ए॒व । ष॒ड॒हेनेति॑ षट्-अ॒हेन॑ । सु॒व॒र्गमिति॑ सुवः - गम् । लो॒कम् । य॒न्ति॒ । ब्र॒ह्म॒वा॒दिन॒ इति॑ ब्रह्म - वा॒दिनः॑ । व॒द॒न्ति॒ । आस॑ते । केन॑ । य॒न्ति॒ । इति॑ । दे॒व॒याने॒नेति॑ देव - याने॑न । प॒था । इति॑ । ब्रू॒या॒त् । छन्दाꣳ॑सि । वै । दे॒व॒यान॒ इति॑ देव - यानः॑ । पन्थाः᳚ । गा॒य॒त्री । त्रि॒ष्टुप् । जग॑ती । ज्योतिः॑ । वै । गा॒य॒त्री । गौः । त्रि॒ष्टुक् । आयुः॑ । जग॑ती । यत् । ए॒ते । स्तोमाः᳚ । भव॑न्ति । द॒व॒याने॒नेति॑ देव-याने॑न । ए॒व ।  \newline




\markright{ TS 7.5.1.6  \hfill https://www.vedavms.in \hfill}

\section{ TS 7.5.1.6 }

\textbf{TS 7.5.1.6 } \newline
\textbf{Samhita Paata} \newline

तत् प॒था य॑न्ति समा॒नꣳ साम॑ भवति देवलो॒को वै साम॑ देवलो॒कादे॒व नय॑न्त्य॒न्याअ॑न्या॒ ऋचो॑ भवन्ति मनुष्यलो॒को वा ऋचो॑ मनुष्यलो॒कादे॒वान्यम॑न्यं देवलो॒कम॑भ्या॒रोह॑न्तो यन्त्यभिव॒र्तो ब्र॑ह्मसा॒मं भ॑वति सुव॒र्गस्य॑ लो॒कस्या॒भिवृ॑त्या अभि॒जिद्-भ॑वति सुव॒र्गस्य॑ लो॒कस्या॒भिजि॑त्यै विश्व॒जिद्-भ॑वति॒ विश्व॑स्य॒ जित्यै॑ मा॒सिमा॑सि पृ॒ष्ठान्युप॑ यन्ति मा॒सिमा᳚स्यतिग्रा॒ह्या॑ गृह्यन्ते मा॒सिमा᳚स्ये॒व वी॒र्यं॑ ( ) दधति मा॒सां प्रति॑ष्ठित्या उ॒परि॑ष्टान्मा॒सां पृ॒ष्ठान्युप॑ यन्ति॒ तस्मा॑दु॒परि॑ष्टा॒दोष॑धयः॒ फलं॑ गृह्णन्ति ॥ \newline

\textbf{Pada Paata} \newline

तत् । प॒था । य॒न्ति॒ । स॒मा॒नम् । साम॑ । भ॒व॒ति॒ । दे॒व॒लो॒क इति॑ देव - लो॒कः । वै । साम॑ । दे॒व॒लो॒कादिति॑ देव - लो॒कात् । ए॒व । न । य॒न्ति॒ । अ॒न्या‌अ॑न्या॒ इत्य॒न्याः - अ॒न्याः॒ । ऋचः॑ । भ॒व॒न्ति॒ । म॒नु॒ष्य॒लो॒क इति॑ मनुष्य - लो॒कः । वै । ऋचः॑ । म॒नु॒ष्य॒लो॒कादिति॑ मनुष्य - लो॒कात् । ए॒व । अ॒न्यम॑न्य॒मित्य॒न्यम् - अ॒न्य॒म् । दे॒व॒लो॒कमिति॑ देव - लो॒कम् । अ॒भ्या॒रोह॑न्त॒ इत्य॑भि-आ॒रोह॑न्तः । य॒न्ति॒ । अ॒भि॒व॒र्त इत्य॑भि  -  व॒र्तः । ब्र॒ह्म॒सा॒ममिति॑ ब्रह्म - सा॒मम् । भ॒व॒ति॒ । सु॒व॒र्गस्येति॑ सुवः - गस्य॑ । लो॒कस्य॑ । अ॒भिवृ॑त्या॒ इत्य॒भि - वृ॒त्यै॒ । अ॒भि॒जिदित्य॑भि - जित् । भ॒व॒ति॒ । सु॒व॒र्गस्येति॑ सुवः - गस्य॑ । लो॒कस्य॑ । अ॒भिजि॑त्या॒ इत्य॒भि - जि॒त्यै॒ । वि॒श्व॒जिदिति॑ विश्व - जित् । भ॒व॒ति॒ । विश्व॑स्य । जित्यै᳚ । मा॒सिमा॒सीति॑ मा॒सि - मा॒सि॒ । पृ॒ष्ठानि॑ । उपेति॑ । य॒न्ति॒ । मा॒सिमा॒सीति॑ मा॒सि - मा॒सि॒ । अ॒ति॒ग्रा॒ह्या॑ इत्य॑ति - ग्रा॒ह्याः᳚ । गृ॒ह्य॒न्ते॒ । मा॒सिमा॒सीति॑ मा॒सि - मा॒सि॒ । ए॒व । वी॒र्य᳚म् ( ) । द॒ध॒ति॒ । मा॒साम् । प्रति॑ष्ठित्या॒ इति॒ प्रति॑ - स्थि॒त्यै॒ । उ॒परि॑ष्टात् । मा॒साम् । पृ॒ष्ठानि॑ । उपेति॑ । य॒न्ति॒ । तस्मा᳚त् । उ॒परि॑ष्टात् । ओष॑धयः । फल᳚म् । गृ॒ह्ण॒न्ति॒ ॥  \newline




\markright{ TS 7.5.2.1  \hfill https://www.vedavms.in \hfill}

\section{ TS 7.5.2.1 }

\textbf{TS 7.5.2.1 } \newline
\textbf{Samhita Paata} \newline

गावो॒ वा ए॒तथ् स॒त्रमा॑सताशृ॒ङ्गाः स॒तीः शृङ्गा॑णि॒ सिषा॑सन्ती॒स्तासां॒ दश॒ मासा॒ निष॑ण्णा॒ आस॒न्नथ॒ शृङ्गा᳚ण्यजायन्त॒ ता अ॑ब्रुव॒न्नरा॒थ्स्मोत् ति॑ष्ठा॒माव॒ तं काम॑मरुथ्स्महि॒ येन॒ कामे॑न॒ न्यष॑दा॒मेति॒ तासा॑मु॒ त्वा अ॑ब्रुवन्न॒र्द्धावा॒ याव॑ती॒र्वाऽऽसा॑महा ए॒वेमौद्वा॑द॒शौ मासौ॑ संॅवथ्स॒रꣳ स॒पांद्योत् ति॑ष्ठा॒मेति॒ तासां᳚ - [  ] \newline

\textbf{Pada Paata} \newline

गावः॑ । वै । ए॒तत् । स॒त्रम् । आ॒स॒त॒ । अ॒शृ॒ङ्गाः । स॒तीः । शृङ्गा॑णि । सिषा॑सन्तीः । तासा᳚म् । दश॑ । मासाः᳚ । निष॑ण्णा॒ इति॒ नि - स॒न्नाः॒ । आसन्न्॑ । अथ॑ । शृङ्गा॑णि । अ॒जा॒य॒न्त॒ । ताः । अ॒ब्रु॒व॒न्न् । अरा᳚थ्स्म । उदिति॑ । ति॒ष्ठा॒म॒ । अवेति॑ । तम् । काम᳚म् । अ॒रु॒थ्स्म॒हि॒ । येन॑ । कामे॑न । न्यष॑दा॒मेति॑ नि - अस॑दाम । इति॑ । तासा᳚म् । उ॒ । तु । वै । अ॒ब्रु॒व॒न्न् । अ॒द्‌र्धाः । वा॒ । याव॑तीः । वा॒ । आसा॑महै । ए॒व । इ॒मौ । द्वा॒द॒शौ । मासौ᳚ । सं॒ॅव॒थ्स॒रमिति॑ सं - व॒थ्स॒रम् । स॒पांद्येति॑ सं - पाद्य॑ । उदिति॑ । ति॒ष्ठा॒म॒ । इति॑ । तासा᳚म् ।  \newline




\markright{ TS 7.5.2.2  \hfill https://www.vedavms.in \hfill}

\section{ TS 7.5.2.2 }

\textbf{TS 7.5.2.2 } \newline
\textbf{Samhita Paata} \newline

द्वाद॒शे मा॒सि शृङ्गा॑णि॒ प्राव॑र्तन्त श्र॒द्धया॒ वाऽश्र॑द्धया वा॒ ता इ॒मा यास्तू॑प॒रा उ॒भय्यो॒ वाव ता आ᳚र्द्ध्नुव॒न्॒. याश्च॒ शृङ्गा॒ण्यस॑न्व॒न्॒. याश्चोर्ज॑म॒वारु॑न्धत॒र्द्ध्नोति॑ द॒शसु॑ मा॒सू᳚त्तिष्ठ॑न्नृ॒द्ध्नोति॑ द्वाद॒शसु॒ य ए॒वं ॅवेद॑ प॒देन॒ खलु॒ वा ए॒ते य॑न्ति वि॒न्दति॒ खलु॒ वै प॒देन॒ यन् तद्वा ए॒तदृ॒द्धमय॑नं॒ तस्मा॑ ( ) दे॒तद्-गो॒सनि॑ ॥ \newline

\textbf{Pada Paata} \newline

द्वा॒द॒शे । मा॒सि । शृङ्गा॑णि । प्रेति॑ । अ॒व॒र्त॒न्त॒ । श्र॒द्धयेति॑ श्रत्-धया᳚ । वा॒ । अश्र॑द्ध॒येत्यश्र॑त् - ध॒या॒ । वा॒ । ताः । इ॒माः । याः । तू॒प॒राः । उ॒भय्यः॑ । वाव । ताः । आ॒द्‌र्ध्नु॒व॒न्न् । याः । च॒ । शृङ्गा॑णि । अस॑न्वन्न् । याः । च॒ । ऊर्ज᳚म् । अ॒वारु॑न्ध॒तेत्य॑व - अरु॑न्धत । ऋ॒द्ध्नोति॑ । द॒शस्विति॑ द॒श - सु॒ । मा॒सु । उ॒त्तिष्ठ॒न्नित्यु॑त्-तिष्ठन्न्॑ । ऋ॒द्ध्नोति॑ । द्वा॒द॒शस्विति॑ द्वाद॒श - सु॒ । यः । ए॒वम् । वेद॑ । प॒देन॑ । खलु॑ । वै । ए॒ते । य॒न्ति॒ । वि॒न्दति॑ । खलु॑ । वै । प॒देन॑ । यन्न् । तत् । वै । ए॒तत् । ऋ॒द्धम् । अय॑नम् । तस्मा᳚त् ( ) । ए॒तत् । गो॒सनीति॑ गो - सनि॑ ॥  \newline




\markright{ TS 7.5.3.1  \hfill https://www.vedavms.in \hfill}

\section{ TS 7.5.3.1 }

\textbf{TS 7.5.3.1 } \newline
\textbf{Samhita Paata} \newline

प्र॒थ॒मे मा॒सि पृ॒ष्ठान्युप॑ यन्ति मद्ध्य॒म उप॑ यन्त्युत्त॒म उप॑ यन्ति॒ तदा॑हु॒र्यां ॅवै त्रिरेक॒स्याह्न॑ उप॒सीद॑न्ति द॒ह्रं ॅवै साऽप॑राभ्यां॒ दोहा᳚भ्यां दु॒हेऽथ॒ कुतः॒ सा धो᳚क्ष्यते॒ यां द्वाद॑श॒ कृत्व॑ उप॒सीद॒न्तीति॑ संॅवथ्स॒रꣳ स॒पांद्यो᳚त्त॒मे मा॒सि स॒कृत् पृ॒ष्ठान्युपे॑यु॒स्तद्-यज॑माना य॒ज्ञ्ं प॒शूनव॑ रुन्धते समु॒द्रं ॅवा - [  ] \newline

\textbf{Pada Paata} \newline

प्र॒थ॒मे । मा॒सि । पृ॒ष्ठानि॑ । उपेति॑ । य॒न्ति॒ । म॒द्ध्य॒मे । उपेति॑ । य॒न्ति॒ । उ॒त्त॒म इत्यु॑त् - त॒मे । उपेति॑ । य॒न्ति॒ । तत् । आ॒हुः॒ । याम् । वै । त्रिः । एक॑स्य । अह्नः॑ । उ॒प॒सीद॒न्तीत्यु॑प - सीद॑न्ति । द॒ह्रम् । वै । सा । अप॑राभ्याम् । दोहा᳚भ्याम् । दु॒हे॒ । अथ॑ । कुतः॑ । सा । धो॒क्ष्य॒ते॒ । याम् । द्वाद॑श । कृत्वः॑ । उ॒प॒सीद॒न्तीत्यु॑प - सीद॑न्ति । इति॑ । सं॒ॅव॒थ्स॒रमिति॑ सं - व॒थ्स॒रम् । स॒पांद्येति॑ सं - पाद्य॑ । उ॒त्त॒म इत्यु॑त् - त॒मे । मा॒सि । स॒कृत् । पृ॒ष्ठानि॑ । उपेति॑ । इ॒युः॒ । तत् । यज॑मानाः । य॒ज्ञ्म् । प॒शून् । अवेति॑ । रु॒न्ध॒ते॒ । स॒मु॒द्रम् । वै ।  \newline




\markright{ TS 7.5.3.2  \hfill https://www.vedavms.in \hfill}

\section{ TS 7.5.3.2 }

\textbf{TS 7.5.3.2 } \newline
\textbf{Samhita Paata} \newline

ए॒ते॑नवा॒रम॑पा॒रं प्र प्ल॑वन्ते॒ ये सं॑ॅवथ्स॒रमु॑प॒यन्ति॒ यद् बृ॑हद्-रथन्त॒रे अ॒न्वर्जे॑यु॒र्यथा॒ मद्ध्ये॑ समु॒द्रस्य॑ प्ल॒वम॒न्वर्जे॑युस्ता॒दृक् तदनु॑थ्सर्गं बृहद्-रथन्त॒राभ्या॑मि॒त्वा प्र॑ति॒ष्ठां ग॑च्छन्ति॒ सर्वे᳚भ्यो॒ वै कामे᳚भ्यः स॒न्धिर्दु॑हे॒ तद्-यज॑मानाः॒ सर्वा॒न् कामा॒नव॑ रुन्धते ॥ \newline

\textbf{Pada Paata} \newline

ए॒ते । अ॒न॒वा॒रम् । अ॒पा॒रम् । प्रेति॑ । प्ल॒व॒न्ते॒ । ये । सं॒ॅव॒थ्स॒रमिति॑ सं - व॒थ्स॒रम् । उ॒प॒यन्तीत्यु॑प - यन्ति॑ । यत् । बृ॒ह॒द्र॒थ॒न्त॒रे इति॑ बृहत् - र॒थ॒न्त॒रे । अ॒न्वर्जे॑यु॒रित्य॑नु - अर्जे॑युः । यथा᳚ । मद्ध्ये᳚ । स॒मु॒द्रस्य॑ । प्ल॒वम् । अ॒न्वर्जे॑यु॒रित्य॑नु - अर्जे॑युः । ता॒दृक् । तत् । अनु॑थ्सर्ग॒मित्यनु॑त् - स॒र्ग॒म् । बृ॒ह॒द्र॒थ॒न्त॒राभ्या॒मिति॑ बृहत् - र॒थ॒न्त॒राभ्या᳚म् । इ॒त्वा । प्र॒ति॒ष्ठामिति॑ प्रति -  स्थाम् । ग॒च्छ॒न्ति॒ । सर्वे᳚भ्यः । वै । कामे᳚भ्यः । स॒न्धिरिति॑ सं - धिः । दु॒हे॒ । तत् । यज॑मानाः ।  सर्वान्॑ । कामान्॑ । अवेति॑ । रु॒न्ध॒ते॒ ॥  \newline




\markright{ TS 7.5.4.1  \hfill https://www.vedavms.in \hfill}

\section{ TS 7.5.4.1 }

\textbf{TS 7.5.4.1 } \newline
\textbf{Samhita Paata} \newline

स॒मा॒न्य॑ ऋचो॑ भवन्ति मनुष्यलो॒को वा ऋचो॑ मनुष्यलो॒कादे॒व न य॑न्त्य॒न्यद॑न्य॒थ् साम॑ भवति देवलो॒को वै साम॑ देवलो॒कादे॒वान्यम॑न्यं मनुष्यलो॒कं प्र॑त्यव॒रोह॑न्तो यन्ति॒ जग॑ती॒मग्र॒ उप॑ यन्ति॒ जग॑तीं॒ ॅवै छन्दाꣳ॑सि प्र॒त्यव॑रोहन्त्या-ग्रय॒णं ग्रहा॑ बृ॒हत् पृ॒ष्ठानि॑ त्रयस्त्रिꣳ॒॒शꣳस्तोमा॒-स्तस्मा॒-ज्ज्यायाꣳ॑सं॒ कनी॑यान् प्र॒त्यव॑रोहति वैश्वकर्म॒णो गृ॑ह्यते॒विश्वा᳚न्ये॒व तेन॒ कर्मा॑णि॒ यज॑माना॒ अव॑ रुन्धत आदि॒त्यो-[  ] \newline

\textbf{Pada Paata} \newline

स॒मा॒न्यः॑ । ऋचः॑ । भ॒व॒न्ति॒ । म॒नु॒ष्य॒लो॒क इति॑ मनुष्य - लो॒कः । वै । ऋचः॑ । म॒नु॒ष्य॒लो॒कादिति॑ मनुष्य - लो॒कात् । ए॒व । न । य॒न्ति॒ । अ॒न्यद॑न्य॒दित्य॒न्यत् - अ॒न्य॒त् । साम॑ । भ॒व॒ति॒ । दे॒व॒लो॒क इति॑ देव - लो॒कः । वै । साम॑ । दे॒व॒लो॒कादिति॑ देव - लो॒कात् । ए॒व । अ॒न्यम॑न्य॒मित्य॒न्यं - अ॒न्य॒म् । म॒नु॒ष्य॒लो॒कमिति॑ मनुष्य - लो॒कम् । प्र॒त्य॒व॒रोह॑न्त॒ इति॑ प्रति - अ॒व॒रोह॑न्तः । य॒न्ति॒ । जग॑तीम् । अग्रे᳚ । उपेति॑ । य॒न्ति॒ । जग॑तीम् । वै । छन्दाꣳ॑सि । प्र॒त्यव॑रोह॒न्तीति॑ प्रति - अव॑रोहन्ति । आ॒ग्र॒य॒णम् । ग्रहाः᳚ । बृ॒हत् । पृ॒ष्ठानि॑ । त्र॒य॒स्त्रिꣳ॒॒शमिति॑ त्रयः - त्रिꣳ॒॒शम् । स्तोमाः᳚ । तस्मा᳚त् । ज्यायाꣳ॑सम् । कनी॑यान् । प्र॒त्यव॑रोह॒तीति॑ प्रति - अव॑रोहति । वै॒श्व॒क॒र्म॒ण इति॑ वैश्व - क॒र्म॒णः । गृ॒ह्य॒ते॒ । विश्वा॑नि । ए॒व । तेन॑ । कर्मा॑णि । यज॑मानाः । अवेति॑ । रु॒न्ध॒ते॒ । आ॒दि॒त्यः ।  \newline




\markright{ TS 7.5.4.2  \hfill https://www.vedavms.in \hfill}

\section{ TS 7.5.4.2 }

\textbf{TS 7.5.4.2 } \newline
\textbf{Samhita Paata} \newline

गृ॑ह्यत इ॒यं ॅवा अदि॑तिर॒स्यामे॒व प्रति॑ तिष्ठन्त्य॒न्यो᳚ऽन्यो गृह्येते मिथुन॒त्वाय॒ प्रजा᳚त्या अवान्त॒रं ॅवै द॑शरा॒त्रेण॑ प्र॒जाप॑तिः प्र॒जा अ॑सृजत॒ यद्-द॑शरा॒त्रो भव॑ति प्र॒जा ए॒व तद्-यज॑मानाः सृजन्त ए॒ताꣳ ह॒ वा उ॑द॒ङ्कः शौ᳚ल्बाय॒नः स॒त्रस्यर्द्धि॑मुवाच॒ यद्-द॑शरा॒त्रोयद्-द॑शरा॒त्रो भव॑ति स॒त्रस्यर्द्ध्या॒ अथो॒ यदे॒व पूर्वे॒ष्वह॑स्सु॒ विलो॑म क्रि॒यते॒ तस्यै॒वै ( )-षा शान्तिः॑ ॥ \newline

\textbf{Pada Paata} \newline

गृ॒ह्य॒ते॒ । इ॒यम् । वै । अदि॑तिः । अ॒स्याम् । ए॒व । प्रतीति॑ । ति॒ष्ठ॒न्ति॒ । अ॒न्यो᳚न्य॒ इत्य॒न्यः - अ॒न्यः॒ । गृ॒ह्ये॒ते॒ इति॑ । मि॒थु॒न॒त्वायेति॑ मिथुन - त्वाय॑ । प्रजा᳚त्या॒ इति॒ प्र - जा॒त्यै॒ । अ॒वा॒न्त॒रमित्य॑व - अ॒न्त॒रम् । वै । द॒श॒रा॒त्रेणेति॑ दश - रा॒त्रेण॑ । प्र॒जाप॑ति॒रिति॑ प्र॒जा - प॒तिः॒ । प्र॒जा इति॑ प्र - जाः । अ॒सृ॒ज॒त॒ । यत् । द॒श॒रा॒त्र इति॑ दश - रा॒त्रः । भव॑ति । प्र॒जा इति॑ प्र - जाः । ए॒व । तत् । यज॑मानाः । सृ॒ज॒न्ते॒ । ए॒ताम् । ह॒ । वै । उ॒द॒ङ्कः । शौ॒ल्बा॒य॒नः । स॒त्रस्य॑ । ऋद्धि᳚म् । उ॒वा॒च॒ । यत् । द॒श॒रा॒त्र इति॑ दश - रा॒त्रः । यत् । द॒श॒रा॒त्र इति॑ दश - रा॒त्रः । भव॑ति । स॒त्रस्य॑ । ऋद्ध्यै᳚ । अथो॒ इति॑ । यत् । ए॒व । पूर्वे॑षु । अह॒स्स्वित्यहः॑ - सु॒ । विलो॒मेति॒ वि - लो॒म॒ । क्रि॒यते᳚ । तस्य॑ । ए॒व ( ) । ए॒षा । शान्तिः॑ ॥  \newline




\markright{ TS 7.5.5.1  \hfill https://www.vedavms.in \hfill}

\section{ TS 7.5.5.1 }

\textbf{TS 7.5.5.1 } \newline
\textbf{Samhita Paata} \newline

यदि॒ सोमौ॒ सꣳसु॑तौ॒ स्यातां᳚ मह॒ति रात्रि॑यै प्रातरनुवा॒क-मु॒पाकु॑र्या॒त् पूर्वो॒ वाचं॒ पूर्वो॑ दे॒वताः॒ पूर्वः॒ छन्दाꣳ॑सि वृङ्क्ते॒ वृष॑ण्वतीं प्रति॒पदं॑ कुर्यात् प्रातस्सव॒नादे॒वैषा॒मिन्द्रं॑ ॅवृ॒ङ्क्ते ऽथो॒ खल्वा॑हुःसवनमु॒खे-स॑वनमुखे का॒र्येति॑ सवनमु॒खाथ् स॑वनमुखा-दे॒वैषा॒मिन्द्रं॑ ॅवृङ्क्ते संॅवे॒शायो॑पवे॒शाय॑ गायत्रि॒यास्त्रि॒ष्टुभो॒ जग॑त्या अनु॒ष्टुभः॑ प॒ङ्क्त्या अ॒भिभू᳚त्यै॒ स्वाहा॒ छन्दाꣳ॑सि॒ वै सं॑ॅवे॒श उ॑पवे॒शः छन्दो॑भिरे॒वैषां॒- [  ] \newline

\textbf{Pada Paata} \newline

यदि॑ । सोमौ᳚ । सꣳसु॑ता॒विति॒ सं - सु॒तौ॒ । स्याता᳚म् । म॒ह॒ति । रात्रि॑यै । प्रा॒त॒र॒नु॒वा॒कमिति॑ प्रातः - अ॒नु॒वा॒कम् । उ॒पाकु॑र्या॒दित्यु॑प - आकु॑र्यात् । पूर्वः॑ । वाच᳚म् । पूर्वः॑ । दे॒वताः᳚ । पूर्वः॑ । छन्दाꣳ॑सि । वृ॒ङ्क्ते॒ । वृष॑ण्वती॒मिति॒ वृषण्॑ - व॒ती॒म् । प्र॒ति॒पद॒मिति॑ प्रति - पद᳚म् । कु॒र्या॒त् । प्रा॒त॒स्स॒व॒नादिति॑ प्रातः-स॒व॒नात् । ए॒व । ए॒षा॒म् । इन्द्र᳚म् । वृ॒ङ्क्ते॒ । अथो॒ इति॑ । खलु॑ । आ॒हुः॒ । स॒व॒न॒मु॒खेस॑वनमुख॒ इति॑ सवनमु॒खे - स॒व॒न॒मु॒खे॒ । का॒र्या᳚ । इति॑ । स॒व॒न॒मु॒खाथ्स॑वनमुखा॒दिति॑ सवनमु॒खात् - स॒व॒न॒मु॒खा॒त् । ए॒व । ए॒षा॒म् । इन्द्र᳚म् । वृ॒ङ्क्ते॒ । सं॒ॅवे॒शायेति॑ सं - वे॒शाय॑ । उ॒प॒वे॒शायेत्यु॑प - वे॒शाय॑ । गा॒य॒त्रि॒याः । त्रि॒ष्टुभः॑ । जग॑त्याः । अ॒नु॒ष्टुभ॒ इत्य॑नु - स्तुभः॑ । प॒ङ्क्त्याः । अ॒भिभू᳚त्या॒ इत्य॒भि - भू॒त्यै॒ । स्वाहा᳚ । छन्दाꣳ॑सि । वै । सं॒ॅवे॒श इति॑ सं - वे॒शः । उ॒प॒वे॒श इत्यु॑प - वे॒शः । छन्दो॑भि॒रिति॒ छन्दः॑ - भिः॒ । ए॒व । ए॒षा॒म् ।  \newline




\markright{ TS 7.5.5.2  \hfill https://www.vedavms.in \hfill}

\section{ TS 7.5.5.2 }

\textbf{TS 7.5.5.2 } \newline
\textbf{Samhita Paata} \newline

छन्दाꣳ॑सि वृङ्क्ते सज॒नीयꣳ॒॒ शस्यं॑ ॅविह॒व्यꣳ॑ शस्य॑म॒गस्त्य॑स्य कयाशु॒भीयꣳ॒॒ शस्य॑मे॒ताव॒द्वा अ॑स्ति॒ याव॑दे॒तद्-याव॑दे॒वास्ति॒ तदे॑षां ॅवृङ्क्ते॒ यदि॑ प्रातस्सव॒ने क॒लशो॒ दीर्ये॑त वैष्ण॒वीषु॑ शिपिवि॒ष्टव॑तीषु स्तुवीर॒न्॒.यद्वै य॒ज्ञ्स्या॑-ति॒रिच्य॑ते॒ विष्णुं॒ तच्छि॑पिवि॒ष्टम॒भ्यति॑ रिच्यते॒ तद्विष्णुः॑ शिपिवि॒ष्टोऽति॑रिक्त ए॒वाति॑रिक्तं दधा॒त्यथो॒ अति॑रिक्तेनै॒वा-ति॑रिक्तमा॒प्त्वाऽव॑ ( ) रुन्धते॒ यदि॑ म॒द्ध्यन्दि॑ने॒ दीर्ये॑त वषट्का॒रनि॑धनꣳ॒॒ साम॑ कुर्युर्वषट्का॒रो वै य॒ज्ञ्स्य॑ प्रति॒ष्ठा प्र॑ति॒ष्ठामे॒वैन॑द्-गमयन्ति॒ यदि॑ तृतीयसव॒न ए॒तदे॒व ॥ \newline

\textbf{Pada Paata} \newline

छन्दाꣳ॑सि । वृ॒ङ्क्ते॒ । स॒ज॒नीय॒मिति॑ स - ज॒नीय᳚म् । शस्य᳚म् । वि॒ह॒व्य॑मिति॑ वि - ह॒व्य᳚म् । शस्य᳚म् । अ॒गस्त्य॑स्य । क॒या॒शु॒भीय॒मिति॑ कया-शु॒भीय᳚म् । शस्य᳚म् । ए॒ताव॑त् । वै । अ॒स्ति॒ । याव॑त् । ए॒तत् । याव॑त् । ए॒व । अस्ति॑ । तत् । ए॒षा॒म् । वृ॒ङ्क्ते॒ । यदि॑ । प्रा॒त॒स्स॒व॒न इति॑ प्रातः - स॒व॒ने । क॒लशः॑ । दीर्ये॑त । वै॒ष्ण॒वीषु॑ । शि॒पि॒वि॒ष्टव॑ती॒ष्विति॑ शिपिवि॒ष्ट - व॒ती॒षु॒ । स्तु॒वी॒र॒न्न् । यत् । वै । य॒ज्ञ्स्य॑ । अ॒ति॒रिच्य॑त॒ इत्य॑ति - रिच्य॑ते । विष्णु᳚म् । तत् । शि॒पि॒वि॒ष्टमिति॑ शिपि - वि॒ष्टम् । अ॒भि । अतीति॑ । रि॒च्य॒ते॒ । तत् । विष्णुः॑ । शि॒पि॒वि॒ष्ट इति॑ शिपि - वि॒ष्टः । अति॑रिक्त॒ इत्यति॑- रि॒क्ते॒ । ए॒व । अति॑रिक्त॒मित्यति॑ - रि॒क्त॒म् । द॒धा॒ति॒ । अथो॒ इति॑ । अति॑रिक्ते॒नेत्यति॑ - रि॒क्ते॒न॒ । ए॒व । अति॑रिक्त॒मित्यति॑ - रि॒क्त॒म् । आ॒प्त्वा । अवेति॑ ( ) । रु॒न्ध॒ते॒ । यदि॑ । म॒द्ध्यन्दि॑ने । दीर्ये॑त । व॒ष॒ट्का॒रनि॑धन॒मिति॑ वषट्का॒र - नि॒ध॒न॒म् । साम॑ । कु॒र्युः॒ । व॒ष॒ट्का॒र इति॑ वषट् - का॒रः । वै । य॒ज्ञ्स्य॑ । प्र॒ति॒ष्ठेति॑ प्रति - स्था । प्र॒ति॒ष्ठामिति॑ प्रति - स्थाम् । ए॒व । ए॒न॒त् । ग॒म॒य॒न्ति॒ । यदि॑ । तृ॒ती॒य॒स॒व॒न इति॑ तृतीय - स॒व॒ने । ए॒तत् । ए॒व ॥  \newline




\markright{ TS 7.5.6.1  \hfill https://www.vedavms.in \hfill}

\section{ TS 7.5.6.1 }

\textbf{TS 7.5.6.1 } \newline
\textbf{Samhita Paata} \newline

ष॒ड॒हैर्मासा᳚न्थ् स॒पांद्याह॒रुथ् सृ॑जन्ति षड॒हैर्.हि मासा᳚न्थ् स॒पंश्य॑न्त्य-र्द्धमा॒सैर्मासा᳚न्थ् स॒पांद्याह॒रुथ् सृ॑जन्त्य-र्द्धमा॒सैर्.हि मासा᳚न्थ् स॒पंश्य॑न्त्यमावा॒स्य॑या॒ मासा᳚न्थ् स॒पांद्याह॒रुथ् सृ॑जन्त्यमावा॒स्य॑या॒ हि मासा᳚न्थ् स॒पंश्य॑न्ति पौर्णमा॒स्या मासा᳚न्थ् स॒पांद्याऽहरुथ् सृ॑जन्ति पौर्णमा॒स्या हि मासा᳚न्थ् स॒पंश्य॑न्ति॒ यो वै पू॒र्ण आ॑सि॒ञ्चति॒ परा॒ स सि॑ञ्चति॒ यः पू॒र्णादु॒दच॑ति - [  ] \newline

\textbf{Pada Paata} \newline

ष॒ड॒हैरिति॑ षट् - अ॒हैः । मासान्॑ । स॒पांद्येति॑ सं - पाद्य॑ । अहः॑ । उदिति॑ । सृ॒ज॒न्ति॒ । ष॒ड॒हैरिति॑ षट् - अ॒हैः । हि । मासान्॑ । स॒पंश्य॒न्तीति॑ सं - पश्य॑न्ति । अ॒द्‌र्ध॒मा॒सैरित्य॑द्‌र्ध - मा॒सैः । मासान्॑ । स॒पांद्येति॑ सं - पाद्य॑ । अहः॑ । उदिति॑ । सृ॒ज॒न्ति॒ । अ॒द्‌र्ध॒मा॒सैरित्य॑द्‌र्ध - मा॒सैः । हि । मासान्॑ । स॒पंश्य॒न्तीति॑ सं - पश्य॑न्ति । अ॒मा॒वा॒स्य॑येत्य॑मा - वा॒स्य॑या । मासान्॑ । स॒पांद्येति॑ सं - पाद्य॑ । अहः॑ । उदिति॑ । सृ॒ज॒न्ति॒ । अ॒मा॒वा॒स्य॑येत्य॑मा - वा॒स्य॑या । हि । मासान्॑ । स॒पंश्य॒न्तीति॑ सं-पश्य॑न्ति । पौ॒र्ण॒मा॒स्येति॑ पौर्ण-मा॒स्या । मासान्॑ । स॒पांद्येति॑ सं - पाद्य॑ । अहः॑ । उदिति॑ । सृ॒ज॒न्ति॒ । पौ॒र्ण॒मा॒स्येति॑ पौर्ण - मा॒स्या । हि । मासान्॑ । स॒पंश्य॒न्तीति॑ सं - पश्य॑न्ति । यः । वै । पू॒र्णे । आ॒सि॒ञ्चतीत्या᳚ - सि॒ञ्चति॑ ।परेति॑ । सः । सि॒ञ्च॒ति॒ । यः । पू॒र्णात् । उ॒दच॒तीत्यु॑त् - अच॑ति ।  \newline




\markright{ TS 7.5.6.2  \hfill https://www.vedavms.in \hfill}

\section{ TS 7.5.6.2 }

\textbf{TS 7.5.6.2 } \newline
\textbf{Samhita Paata} \newline

प्रा॒णम॑स्मि॒न्थ्स द॑धाति॒ यत् पौ᳚र्णमा॒स्या मासा᳚न्थ् स॒पांद्याह॑रुथ् सृ॒जन्ति॑ संॅवथ्स॒रायै॒व तत् प्रा॒णं द॑धति॒ तदनु॑ स॒त्रिणः॒ प्राण॑न्ति॒ यदह॒र्नोथ्-सृ॒जेयु॒र्यथा॒ दृति॒रुप॑नद्धो वि॒पत॑त्ये॒वꣳ सं॑ॅवथ्स॒रो वि प॑ते॒दार्ति॒-मार्च्छे॑यु॒र्यत् पौ᳚र्णमा॒स्या मासा᳚न्थ्-स॒पांद्याह॑रुथ् सृ॒जन्ति॑ संॅवथ्स॒रायै॒व तदु॑दा॒नं द॑धति॒ तदनु॑ स॒त्रिण॒ उ - [  ] \newline

\textbf{Pada Paata} \newline

प्रा॒णमिति॑ प्र - अ॒नम् । अ॒स्मि॒न्न् । सः । द॒धा॒ति॒ । यत् । पौ॒र्ण॒मा॒स्येति॑ पौर्ण - मा॒स्या । मासान्॑ । स॒पांद्येति॑ सं - पाद्य॑ । अहः॑ । उ॒थ्सृ॒जन्तीत्यु॑त् - सृ॒जन्ति॑ । सं॒ॅव॒थ्स॒रायेति॑ सं-व॒थ्स॒राय॑ । ए॒व । तत् । प्रा॒णमिति॑ प्र - अ॒नम् । द॒ध॒ति॒ । तत् । अन्विति॑ । स॒त्रिणः॑ । प्रेति॑ । अ॒न॒न्ति॒ । यत् । अहः॑ । न । उ॒थ्सृ॒जेयु॒रित्यु॑त्-सृ॒जेयुः॑ । यथा᳚ । दृतिः॑ । उप॑नद्ध॒ इत्युप॑ - न॒द्धः॒ । वि॒पत॒तीति॑ वि - पत॑ति । ए॒वम् । सं॒ॅव॒थ्स॒र इति॑ सं - व॒थ्स॒रः । वीति॑ । प॒ते॒त् । आर्ति᳚म् । एति॑ । ऋ॒च्छे॒युः॒ । यत् । पौ॒र्ण॒मा॒स्येति॑ पौर्ण - मा॒स्या । मासान्॑ । स॒पांद्येति॑ सं - पाद्य॑ । अहः॑ । उ॒थ्सृ॒जन्तीत्यु॑त् - सृ॒जन्ति॑ । सं॒ॅव॒थ्स॒रायेति॑ सं - व॒थ्स॒राय॑ । ए॒व । तत् । उ॒दा॒नमित्यु॑त् - अ॒नम् । द॒ध॒ति॒ । तत् । अन्विति॑ । स॒त्रिणः॑ । उदिति॑ ।  \newline




\markright{ TS 7.5.6.3  \hfill https://www.vedavms.in \hfill}

\section{ TS 7.5.6.3 }

\textbf{TS 7.5.6.3 } \newline
\textbf{Samhita Paata} \newline

द॑नन्ति॒ नाऽऽ*र्ति॒मार्च्छ॑न्ति पू॒र्णमा॑से॒ वै दे॒वानाꣳ॑ सु॒तो यत् पौ᳚र्णमा॒स्या मासा᳚न्थ्-स॒पांद्याह॑रुथ् सृ॒जन्ति॑ दे॒वाना॑मे॒व तद्-य॒ज्ञेन॑ य॒ज्ञ्ं प्र॒त्यव॑रोहन्ति॒ वि वा ए॒तद्-य॒ज्ञ्ं छि॑न्दन्ति॒ यथ् ष॑ड॒हस॑तंतꣳ॒॒ संत॒मथाह॑रुथ् सृ॒जन्ति॑ प्राजाप॒त्यं प॒शुमा ल॑भन्ते प्र॒जाप॑तिः॒ सर्वा॑ दे॒वता॑ दे॒वता॑भिरे॒व य॒ज्ञ्ꣳ सं त॑न्वन्ति॒ यन्ति॒ वा ए॒ते सव॑ना॒द्येऽह॑ -[  ] \newline

\textbf{Pada Paata} \newline

अ॒न॒न्ति॒ । न । आर्ति᳚म् । एति॑ । ऋ॒च्छ॒न्ति॒ । पू॒र्णमा॑स॒ इति॑ पू॒र्ण - मा॒से॒ । वै । दे॒वाना᳚म् । सु॒तः । यत् । पौ॒र्ण॒मा॒स्येति॑ पौर्ण - मा॒स्या । मासान्॑ । स॒पांद्येति॑ सं - पाद्य॑ । अहः॑ । उ॒थ्सृ॒जन्तीत्य॑त् - सृ॒जन्ति॑ । दे॒वाना᳚म् । ए॒व । तत् । य॒ज्ञेन॑ । य॒ज्ञ्म् । प्र॒त्यव॑रोह॒न्तीति॑ प्रति - अव॑रोहन्ति । वीति॑ । वै । ए॒तत् । य॒ज्ञ्म् । छि॒न्द॒न्ति॒ । यत् । ष॒ड॒हस॑न्तत॒मिति॑ षड॒ह - स॒न्त॒त॒म् । सन्त᳚म् । अथ॑ । अहः॑ । उ॒थ्सृ॒जन्तीत्यु॑त् - सृ॒जन्ति॑ । प्रा॒जा॒प॒त्यमिति॑ प्राजा-प॒त्यम् । प॒शुम् । एति॑ । ल॒भ॒न्ते॒ । प्र॒जाप॑ति॒रिति॑ प्र॒जा-प॒तिः॒ । सर्वाः᳚ । दे॒वताः᳚ । दे॒वता॑भिः । ए॒व । य॒ज्ञ्म् । समिति॑ । त॒न्व॒न्ति॒ । यन्ति॑ । वै । ए॒ते । सव॑नात् । ये । अहः॑ ।  \newline




\markright{ TS 7.5.6.4  \hfill https://www.vedavms.in \hfill}

\section{ TS 7.5.6.4 }

\textbf{TS 7.5.6.4 } \newline
\textbf{Samhita Paata} \newline

-रुथ् सृ॒जन्ति॑ तु॒रीयं॒ खलु॒ वा ए॒तथ् सव॑नं॒ ॅयथ् सा᳚नां॒य्यं ॅयथ् सा᳚नां॒य्यं भव॑ति॒ तेनै॒व सव॑ना॒न्न य॑न्ति समुप॒हूय॑ भक्षयन्त्ये॒तथ्- सो॑मपीथा॒ ह्ये॑तर्.हि॑ यथायत॒नं ॅवा ए॒तेषाꣳ॑ सवन॒भाजो॑ दे॒वता॑ गच्छन्ति॒ येऽह॑रुथ् सृ॒जन्त्य॑नुसव॒नं पु॑रो॒डाशा॒न् निर्व॑पन्ति यथायत॒नादे॒व स॑वन॒भाजो॑ दे॒वता॒ अव॑ रुन्धते॒ ऽष्टाक॑पालान् प्रातस्सव॒न एका॑दशकपाला॒न् माद्ध्य॑न्दिने॒ सव॑ने॒ द्वाद॑शकपालाꣳ-स्तृतीयसव॒ने छन्दाꣳ॑स्ये॒वाऽऽ*प्त्वा ( ) -ऽव॑ रुन्धते वैश्वदे॒वं च॒रुं तृ॑तीयसव॒ने निर्व॑पन्ति वैश्वदे॒वं ॅवै तृ॑तीयसव॒नं तेनै॒व तृ॑तीयसव॒नान्न य॑न्ति ॥ \newline

\textbf{Pada Paata} \newline

उ॒थ्सृ॒जन्तीत्यु॑त् - सृ॒जन्ति॑ । तु॒रीय᳚म् । खलु॑ । वै । ए॒तत् । सव॑नम् । यत् । सा॒न्ना॒य्यमिति॑ सां-ना॒य्यम् । यत् । सा॒न्ना॒य्यमिति॑ सां-ना॒य्यम् । भव॑ति । तेन॑ । ए॒व । सव॑नात् । न । य॒न्ति॒ । स॒मु॒प॒हूयेति॑ सं - उ॒प॒हूय॑ । भ॒क्ष॒य॒न्ति॒ । ए॒तथ्सो॑मपीथा॒ इत्ये॒तत् - सो॒म॒पी॒थाः॒ । हि । ए॒तर्.हि॑ । य॒था॒य॒त॒नमिति॑ यथा - अ॒य॒त॒नम् । वै । ए॒तेषा᳚म् । स॒व॒न॒भाज॒ इति॑ सवन - भाजः॑ । दे॒वताः᳚ । ग॒च्छ॒न्ति॒ । ये । अहः॑ । उ॒थ्सृ॒जन्तीत्यु॑त् - सृ॒जन्ति॑ । अ॒नु॒स॒व॒नमित्य॑नु - स॒व॒नम् । पु॒रो॒डाशान्॑ । निरिति॑ । व॒प॒न्ति॒ । य॒था॒य॒त॒नादिति॑ यथा - आ॒य॒त॒नात् । ए॒व । स॒व॒न॒भाज॒ इति॑ सवन - भाजः॑ । दे॒वताः᳚ । अवेति॑ । रु॒न्ध॒ते॒ । अ॒ष्टाक॑पाला॒नित्य॒ष्टा - क॒पा॒ला॒न् । प्रा॒त॒स्स॒व॒न इति॑ प्रातः - स॒व॒ने । एका॑दशकपाला॒नित्येका॑दश - क॒पा॒ला॒न्न् । माद्ध्य॑न्दिने । सव॑ने । द्वाद॑शकपाला॒निति॒ द्वाद॑श - क॒पा॒ला॒न्न् । तृ॒ती॒य॒स॒व॒न इति॑ तृतीय - स॒व॒ने । छन्दाꣳ॑सि । ए॒व । आ॒प्त्वा ( ) । अवेति॑ । रु॒न्ध॒ते॒ । वै॒श्व॒दे॒वमिति॑ वैश्व-दे॒वम् । च॒रुम् । तृ॒ती॒य॒स॒व॒न इति॑ तृतीय-स॒व॒ने । निरिति॑ । व॒प॒न्ति॒ । वै॒श्व॒दे॒वमिति॑ वैश्व - दे॒वम् । वै । तृ॒ती॒य॒स॒व॒नमिति॑ तृतीय - स॒व॒नम् । तेन॑ । ए॒व । तृ॒ती॒य॒स॒व॒नादिति॑ तृतीय - स॒व॒नात् । न । य॒न्ति॒ ॥  \newline




\markright{ TS 7.5.7.1  \hfill https://www.vedavms.in \hfill}

\section{ TS 7.5.7.1 }

\textbf{TS 7.5.7.1 } \newline
\textbf{Samhita Paata} \newline

उ॒थ्सृज्यां(3)नोथ्सृज्या(3)मिति॑ मीमाꣳसन्ते ब्रह्मवा॒दिन॒-स्तद्वा॑हुरु॒थ् सृज्य॑मे॒वेत्य॑-मावा॒स्या॑यां च पौर्णमा॒स्यां चो॒थ्-सृज्य॒मित्या॑हुरे॒ते हि य॒ज्ञ्ं ॅवह॑त॒ इति॒ ते त्वाव नोथ्सृज्ये॒ इत्या॑हु॒र्ये अ॑वान्त॒रं ॅय॒ज्ञ्ं भे॒जाते॒ इति॒ या प्र॑थ॒मा व्य॑ष्टका॒ तस्या॑मु॒थ्-सृज्य॒मित्या॑हुरे॒ष वै मा॒सो वि॑श॒र इति॒ नाऽऽ*दि॑ष्ट॒ - [  ] \newline

\textbf{Pada Paata} \newline

उ॒थ्सृज्या(3)मित्यु॑त् - सृज्या(3)म् । न । उ॒थ्सृज्या(3)मित्यु॑त्- सृज्या(3)म् । इति॑ । मी॒माꣳ॒॒स॒न्ते॒ । ब्र॒ह्म॒वा॒दिन॒ इति॑ ब्रह्म - वा॒दिनः॑ । तत् । उ॒ । आ॒हुः॒ । उ॒थ्सृज्य॒मित्यु॑त् - सृज्य᳚म् । ए॒व । इति॑ । अ॒मा॒वा॒स्या॑या॒मित्य॑मा - वा॒स्या॑याम् । च॒ । पौ॒र्ण॒मा॒स्यामिति॑ पौर्ण - मा॒स्याम् । च॒ । उ॒थ्सृज्य॒मित्यु॑त् - सृज्य᳚म् । इति॑ । आ॒हुः॒ । ए॒ते इति॑ । हि । य॒ज्ञ्म् । वह॑तः । इति॑ । ते इति॑ । तु । वाव । न । उ॒थ्सृज्ये॒ इत्यु॑त् - सृज्ये᳚ । इति॑ । आ॒हुः॒ । ये इति॑ । अ॒वा॒न्त॒रमित्य॑व - अ॒न्त॒रम् । य॒ज्ञ्म् । भे॒जाते॒ इति॑ । इति॑ । या । प्र॒थ॒मा । व्य॑ष्ट॒केति॒ वि - अ॒ष्ट॒का॒ । तस्या᳚म् । उ॒थ्सृज्य॒मित्यु॑त् - सृज्य᳚म् । इति॑ । आ॒हुः॒ । ए॒षः । वै । मा॒सः । वि॒श॒र इति॑ वि - श॒रः । इति॑ । न । आदि॑ष्ट॒मित्या - दि॒ष्ट॒म् ।  \newline




\markright{ TS 7.5.7.2  \hfill https://www.vedavms.in \hfill}

\section{ TS 7.5.7.2 }

\textbf{TS 7.5.7.2 } \newline
\textbf{Samhita Paata} \newline

-मुथ्सृ॑जेयु॒-र्यदादि॑ष्ट-मुथ्सृ॒जेयु॑र्या॒दृशे॒ पुनः॑ पर्याप्ला॒वे मद्ध्ये॑ षड॒हस्य॑ स॒पंद्ये॑त षड॒हैर्मासा᳚न्थ् स॒पांद्य॒ यथ् स॑प्त॒म- मह॒स्तस्मि॒न्नुथ् सृ॑जेयु॒-स्तद॒ग्नये॒ वसु॑मते पुरो॒डाश॑म॒ष्टाक॑पालं॒ निर्व॑पेयुरै॒न्द्रं दधीन्द्रा॑य म॒रुत्व॑ते पुरो॒डाश॒मेका॑दशकपालं ॅवैश्वदे॒वं द्वाद॑शकपालम॒ग्नेर्वै वसु॑मतः प्रातस्सव॒नं ॅयद॒ग्नये॒ वसु॑मते पुरो॒डाश॑म॒ष्टाक॑पालं नि॒र्वप॑न्ति दे॒वता॑मे॒व तद्-भा॒गिनीं᳚ कु॒र्वन्ति॒ - [  ] \newline

\textbf{Pada Paata} \newline

उदिति॑ । सृ॒जे॒युः॒ । यत् । आदि॑ष्ट॒मित्या - दि॒ष्ट॒म् । उ॒थ्सृ॒जेयु॒रित्यु॑त् - सृ॒जेयुः॑ । या॒दृशे᳚ । पुनः॑ । प॒र्या॒प्ला॒व इति॑ परि - आ॒प्ला॒वे । मद्ध्ये᳚ । ष॒ड॒हस्येति॑ षट् - अ॒हस्य॑ । स॒पंद्ये॒तेति॑ सं - पद्ये॑त । ष॒ड॒हैरिति॑ षट् - अ॒हैः । मासान्॑ । स॒पांद्येति॑ सं - पाद्य॑ । यत् । स॒प्त॒मम् । अहः॑ । तस्मिन्न्॑ । उदिति॑ । सृ॒जे॒युः॒ । तत् । अ॒ग्नये᳚ । वसु॑मत॒ इति॒ वसु॑ - म॒ते॒ । पु॒रो॒डाश᳚म् । अ॒ष्टाक॑पाल॒मित्य॒ष्टा - क॒पा॒ल॒म् । निरितिः॑ । व॒पे॒युः॒ । ऐ॒न्द्रम् । दधि॑ । इन्द्रा॑य । म॒रुत्व॑ते । पु॒रो॒डाश᳚म् । एका॑दशकपाल॒मित्येका॑दश - क॒पा॒ल॒म् । वै॒श्व॒दे॒वमिति॑ वैश्व-दे॒वम् । द्वाद॑शकपाल॒मिति॒ द्वाद॑श - क॒पा॒ल॒म् । अ॒ग्नेः । वै । वसु॑मत॒ इति॒ वसु॑ - म॒तः॒ । प्रा॒त॒स्स॒व॒नमिति॑ प्रातः - स॒व॒नम् । यत् । अ॒ग्नये᳚ । वसु॑मत॒ इति॒ वसु॑-म॒ते॒ । पु॒रो॒डाश᳚म् । अ॒ष्टाक॑पाल॒मित्य॒ष्टा - क॒पा॒ल॒म् । नि॒र्वप॒न्तीति॑ निः - वप॑न्ति । दे॒वता᳚म् । ए॒व । तत् । भा॒गिनी᳚म् । कु॒र्वन्ति॑ ।  \newline




\markright{ TS 7.5.7.3  \hfill https://www.vedavms.in \hfill}

\section{ TS 7.5.7.3 }

\textbf{TS 7.5.7.3 } \newline
\textbf{Samhita Paata} \newline

सव॑नमष्टा॒भिरुप॑ यन्ति॒ यदै॒न्द्रं दधि॒ भव॒तीन्द्र॑मे॒व तद्-भा॑ग॒धेया॒न्न च्या॑वय॒न्तीन्द्र॑स्य॒ वै म॒रुत्व॑तो॒ माद्ध्य॑न्दिनꣳ॒॒ सव॑नं॒ ॅयदिन्द्रा॑य म॒रुत्व॑ते पुरो॒डाश॒मेका॑दशकपालं नि॒र्वप॑न्ति दे॒वता॑मे॒व तद्-भा॒गिनीं᳚ कु॒र्वन्ति॒ सव॑नमेकाद॒शभि॒रुप॑ यन्ति॒ विश्वे॑षां॒ ॅवै दे॒वाना॑मृभु॒मतां᳚ तृतीयसव॒नंॅयद्-वै᳚श्वदे॒वं द्वाद॑शकपालं नि॒र्वप॑न्ति दे॒वता॑ ए॒व तद्-भा॒गिनीः᳚ कु॒र्वन्ति॒ सव॑नं द्वाद॒शभि॒ - [  ] \newline

\textbf{Pada Paata} \newline

सव॑नम् । अ॒ष्टा॒भिः । उपेति॑ । य॒न्ति॒ । यत् । ऐ॒न्द्रम् । दधि॑ । भव॑ति । इन्द्र᳚म् । ए॒व । तत् । भा॒ग॒धेया॒दिति॑ भाग-धेया᳚त् । न । च्या॒व॒य॒न्ति॒ । इन्द्र॑स्य । वै । म॒रुत्व॑तः । माद्ध्य॑न्दिनम् । सव॑नम् । यत् । इन्द्रा॑य । म॒रुत्व॑ते । पु॒रो॒डाश᳚म् । एका॑दशकपाल॒मित्येका॑दश - क॒पा॒ल॒म् । नि॒र्वप॒न्तीति॑ निः-वप॑न्ति । दे॒वता᳚म् । ए॒व । तत् । भा॒गिनी᳚म् । कु॒र्वन्ति॑ । सव॑नम् । ए॒का॒द॒शभि॒रित्ये॑काद॒श - भिः॒ । उपेति॑ । य॒न्ति॒ । विश्वे॑षाम् । वै । दे॒वाना᳚म् । ऋ॒भु॒मता॒मित्यृ॑भु - मता᳚म् । तृ॒ती॒य॒स॒व॒नमिति॑ तृतीय-स॒व॒नम् । यत् । वै॒श्व॒दे॒वमिति॑ वैश्व-दे॒वम् । द्वाद॑शकपाल॒मिति॒ द्वाद॑श - क॒पा॒ल॒म् । नि॒र्वप॒न्तीति॑ निः - वप॑न्ति । दे॒वताः᳚ । ए॒व । तत् । भा॒गिनीः᳚ । कु॒र्वन्ति॑ । सव॑नम् । द्वा॒द॒शभि॒रिति॑ द्वाद॒श - भिः॒ ।  \newline




\markright{ TS 7.5.7.4  \hfill https://www.vedavms.in \hfill}

\section{ TS 7.5.7.4 }

\textbf{TS 7.5.7.4 } \newline
\textbf{Samhita Paata} \newline

-रुप॑ यन्ति प्राजाप॒त्यं प॒शुमा ल॑भन्ते य॒ज्ञो वै प्र॒जाप॑ति-र्य॒ज्ञ्स्या-न॑नुसर्गायाभिव॒र्त इ॒तः षण्मा॒सो ब्र॑ह्मसा॒मं भ॑वति॒ ब्रह्म॒ वा अ॑भिव॒र्तो ब्रह्म॑णै॒व तथ् सु॑व॒र्गं ॅलो॒क-म॑भिव॒र्तय॑न्तो यन्ति प्रतिकू॒लमि॑व॒ हीतः सु॑व॒र्गो लो॒क इन्द्र॒ क्रतुं॑ न॒ आ भ॑र पि॒ता पु॒त्रेभ्यो॒ यथा᳚ । शिक्षा॑ नो अ॒स्मिन् पु॑रुहूत॒ याम॑नि जी॒वा ज्योति॑रशीम॒हीत्य॒ ( )-मुत॑ आय॒ताꣳ षण्मा॒सो ब्र॑ह्मसा॒मं भ॑वत्य॒यं ॅवै लो॒को ज्योतिः॑ प्र॒जा ज्योति॑रि॒ममे॒व तल्लो॒कं पश्य॑न्तोऽभि॒वद॑न्त॒ आ य॑न्ति ॥ \newline

\textbf{Pada Paata} \newline

उपेति॑ । य॒न्ति॒ । प्रा॒जा॒प॒त्यमिति॑ प्राजा - प॒त्यम् । प॒शुम् । एति॑ । ल॒भ॒न्ते॒ । य॒ज्ञ्ः । वै । प्र॒जाप॑ति॒रिति॑ प्र॒जा - प॒तिः॒ । य॒ज्ञ्स्य॑ । अन॑नुसर्गा॒येत्यन॑नु - स॒र्गा॒य॒ । अ॒भि॒व॒र्त इत्य॑भि - व॒र्तः । इ॒तः । षट् । मा॒सः । ब्र॒ह्म॒सा॒ममिति॑ ब्रह्म - सा॒मम् । भ॒व॒ति॒ । ब्रह्म॑ । वै । अ॒भि॒व॒र्त इत्य॑भि - व॒र्तः । ब्रह्म॑णा । ए॒व । तत् । सु॒व॒र्गमिति॑ सुवः - गम् । लो॒कम् । अ॒भि॒व॒र्तय॑न्त॒ इत्य॑भि - व॒र्तय॑न्तः । य॒न्ति॒ । प्र॒ति॒कू॒लमिति॑ प्रति - कू॒लम् । इ॒व॒ । हि । इ॒तः । सु॒व॒र्ग इति॑ सुवः - गः । लो॒कः । इन्द्र॑ । क्रतु᳚म् । नः॒ । एति॑ । भ॒र॒ । पि॒ता । पु॒त्रेभ्यः॑ । यथा᳚ ॥ शिक्ष॑ । नः॒ । अ॒स्मिन्न् । पु॒रु॒हू॒तेति॑ पुरु - हू॒त॒ । याम॑नि । जी॒वाः । ज्योतिः॑ । अ॒शी॒म॒हि॒ । इति॑ ( ) । अ॒मुतः॑ । आ॒य॒तामित्या᳚ - य॒ताम् । षट् । मा॒सः । ब्र॒ह्म॒सा॒ममिति॑ ब्रह्म-सा॒मम् । भ॒व॒ति॒ । अ॒यम् । वै । लो॒कः । ज्योतिः॑ । प्र॒जेति॑ प्र-जा । ज्योतिः॑ । इ॒मम् । ए॒व । तत् । लो॒कम् । पश्य॑न्तः । अ॒भि॒वद॑न्त॒इत्य॑भि - वद॑न्तः । एति॑ । य॒न्ति॒ ॥  \newline




\markright{ TS 7.5.8.1  \hfill https://www.vedavms.in \hfill}

\section{ TS 7.5.8.1 }

\textbf{TS 7.5.8.1 } \newline
\textbf{Samhita Paata} \newline

दे॒वानां॒ ॅवा अन्तं॑ ज॒ग्मुषा॑मिन्द्रि॒यं ॅवी॒र्य॑-मपा᳚क्राम॒त् तत् क्रो॒शेनावा॑रुन्धत॒ तत् क्रो॒शस्य॑ क्रोश॒त्वं ॅयत् क्रो॒शेन॒ चात्वा॑ल॒स्यान्ते᳚ स्तु॒वन्ति॑ य॒ज्ञ्स्यै॒वान्तं॑ ग॒त्वेन्द्रि॒यं ॅवी॒र्य॑मव॑ रुन्धते स॒त्रस्यर्द्ध्या॑ ऽऽहव॒नीय॒स्यान्ते᳚ स्तुवन्त्य॒ग्नि-मे॒वोप॑द्-र॒ष्टारं॑ कृ॒त्वर्द्धि॒मुप॑ यन्ति प्र॒जाप॑ते॒र्॒.हृद॑येन हवि॒र्द्धाने॒ऽन्तः स्तु॑वन्ति प्रे॒माण॑मे॒वास्य॑ गच्छन्ति श्लो॒केन॑ पु॒रस्ता॒थ् सद॑सः - [  ] \newline

\textbf{Pada Paata} \newline

दे॒वाना᳚म् । वै । अन्त᳚म् । ज॒ग्मुषा᳚म् । इ॒न्द्रि॒यम् । वी॒र्य᳚म् । अपेति॑ । अ॒क्रा॒म॒त् । तत् । क्रो॒शेन॑ । अवेति॑ । अ॒रु॒न्ध॒त॒ । तत् । क्रो॒शस्य॑ । क्रो॒श॒त्वमिति॑ क्रोश - त्वम् । यत् । क्रो॒शेन॑ । चात्वा॑लस्य । अन्ते᳚ । स्तु॒वन्ति॑ । य॒ज्ञ्स्य॑ । ए॒व । अन्त᳚म् । ग॒त्वा । इ॒न्द्रि॒यम् । वी॒र्य᳚म् । अवेति॑ । रु॒न्ध॒ते॒ । स॒त्रस्यद्‌र्ध्या᳚ । आ॒ह॒व॒नीय॒स्येत्या᳚ - ह॒व॒नीय॑स्य । अन्ते᳚ । स्तु॒व॒न्ति॒ । अ॒ग्निम् । ए॒व । उ॒प॒द्र॒ष्टार॒मित्यू॑प - द्र॒ष्टार᳚म् । कृ॒त्वा । ऋद्धि᳚म् । उपेति॑ । य॒न्ति॒ । प॒जाप॑त॒र्॒.हृद॑येन । ह॒वि॒द्‌र्धान॒ इति॑ हविः - धाने᳚ । अ॒न्तः । स्तु॒व॒न्ति॒ । प्रे॒माण᳚म् । ए॒व । अ॒स्य॒ । ग॒च्छ॒न्ति॒ । श्लो॒केन॑ । पु॒रस्ता᳚त् । सद॑सः ।  \newline




\markright{ TS 7.5.8.2  \hfill https://www.vedavms.in \hfill}

\section{ TS 7.5.8.2 }

\textbf{TS 7.5.8.2 } \newline
\textbf{Samhita Paata} \newline

स्तुव॒न्त्यनु॑श्लोकेन प॒श्चाद्-य॒ज्ञ्स्यै॒वान्तं॑ ग॒त्वा श्लो॑क॒भाजो॑ भवन्ति न॒वभि॑-रद्ध्व॒र्युरुद्-गा॑यति॒ नव॒ वै पुरु॑षे प्रा॒णाः प्रा॒णाने॒व यज॑मानेषु दधाति॒ सर्वा॑ ऐ॒न्द्रियो॑ भवन्ति प्रा॒णेष्वे॒वेन्द्रि॒यं द॑ध॒-त्यप्र॑तिहृताभि॒रुद्-गा॑यति॒ तस्मा॒त् पुरु॑षः॒ सर्वा᳚ण्य॒न्यानि॑ शी॒र्.ष्णोऽङ्गा॑नि॒ प्रत्य॑चति॒ शिर॑ ए॒व न प॑ञ्चद॒शꣳर॑थन्त॒रं भ॑वतीन्द्रि॒यमे॒वाव॑ रुन्धते सप्तद॒शं-[  ] \newline

\textbf{Pada Paata} \newline

स्तु॒व॒न्ति॒ । अनु॑श्लोके॒नेत्यनु॑ - श्लो॒के॒न॒ । प॒श्चात् । य॒ज्ञ्स्य॑ । ए॒व । अन्त᳚म् । ग॒त्वा । श्लो॒क॒भाज॒ इति॑ श्लोक - भाजः॑ । भ॒व॒न्ति॒ । न॒वभि॒रिति॑ न॒व - भिः॒ । अ॒द्ध्व॒र्युः । उदिति॑ । गा॒य॒ति॒ । नव॑ । वै । पुरु॑षे । प्रा॒णा इति॑ प्र - अ॒नाः । प्रा॒णानिति॑ प्र - अ॒नान् । ए॒व । यज॑मानेषु । द॒धा॒ति॒ । सर्वाः᳚ । ऐ॒न्द्रियः॑ । भ॒व॒न्ति॒ । प्रा॒णेष्विति॑ प्र - अ॒नेषु॑ । ए॒व । इ॒न्द्रि॒यम् । द॒ध॒ति॒ । अप्र॑तिहृताभि॒रित्यप्र॑ति - हृ॒ता॒भिः॒ । उदिति॑ । गा॒य॒ति॒ । तस्मा᳚त् । पुरु॑षः । सर्वा॑णि । अ॒न्यानि॑ । शी॒र्ष्णः । अङ्गा॑नि । प्रतीति॑ । अ॒च॒ति॒ । शिरः॑ । ए॒व । न । प॒ञ्च॒द॒शमिति॑ पञ्च - द॒शम् । र॒थ॒न्त॒रमिति॑ रथं-त॒रम् । भ॒व॒ति॒ । इ॒न्द्रि॒यम् । ए॒व । अवेति॑ । रु॒न्ध॒ते॒ । स॒प्त॒द॒शमिति॑ सप्त - द॒शम् ।  \newline




\markright{ TS 7.5.8.3  \hfill https://www.vedavms.in \hfill}

\section{ TS 7.5.8.3 }

\textbf{TS 7.5.8.3 } \newline
\textbf{Samhita Paata} \newline

बृ॒ह-द॒न्नाद्य॒स्यावरुद्ध्या॒ अथो॒ प्रैव तेन॑ जायन्त एकविꣳ॒॒शं भ॒द्रं द्वि॒पदा॑सु॒ प्रति॑ष्ठित्यै॒ पत्न॑य॒ उप॑ गायन्ति मिथुन॒त्वाय॒ प्रजा᳚त्यै प्र॒जा॑पतिः प्र॒जा अ॑सृजत॒ सो॑ऽकामयता॒ऽऽ*साम॒हꣳ रा॒ज्यं परी॑या॒मिति॒ तासाꣳ॑ राज॒नेनै॒व रा॒ज्यं पर्यै॒त् तद्-रा॑ज॒नस्य॑ राजन॒त्वं ॅयद्-रा॑ज॒नं भव॑ति प्र॒जाना॑मे॒व तद्-यज॑माना रा॒ज्यं परि॑ यन्ति पञ्चविꣳ॒॒शं भ॑वति प्र॒जाप॑ते॒ - [  ] \newline

\textbf{Pada Paata} \newline

बृ॒हत् । अ॒न्नाद्य॒स्येत्य॑न्न - अद्य॑स्य । अव॑रुद्ध्या॒ इत्यव॑ - रु॒द्ध्यै॒ । अथो॒ इति॑ । प्रेति॑ । ए॒व । तेन॑ । जा॒य॒न्ते॒ । ए॒क॒विꣳ॒॒शमित्ये॑क - विꣳ॒॒शम् । भ॒द्रम् । द्वि॒पदा॒स्विति॑ द्वि - पदा॑सु । प्रति॑ष्ठित्या॒ इति॒ प्रति॑ - स्थि॒त्यै॒ । पत्न॑यः । उपेति॑ । गा॒य॒न्ति॒ । मि॒थु॒न॒त्वायेति॑ मिथुन - त्वाय॑ । प्रजा᳚त्या॒ इति॒ प्र - जा॒त्यै॒ । प्र॒जाप॑ति॒रिति॑ प्र॒जा - प॒तिः॒ । प्र॒जा इति॑ प्र-जाः । अ॒सृ॒ज॒त॒ । सः । अ॒का॒म॒य॒त॒ । आ॒साम् । अ॒हम् । रा॒ज्यम् । परीति॑ । इ॒या॒म् । इति॑ । तासा᳚म् । रा॒ज॒नेन॑ । ए॒व । रा॒ज्यम् । परीति॑ । ऐ॒त् । तत् । रा॒ज॒नस्य॑ । रा॒ज॒न॒त्वमिति॑ राजन - त्वम् । यत् । रा॒ज॒नम् । भव॑ति । प्र॒जाना॒मिति॑ प्र - जाना᳚म् । ए॒व । तत् । यज॑मानाः । रा॒ज्यम् । परीति॑ । य॒न्ति॒ । प॒ञ्च॒विꣳ॒॒शमिति॑ पञ्च - विꣳ॒॒शम् । भ॒व॒ति॒ । प्र॒जाप॑ते॒रिति॑ प्र॒जा - प॒तेः॒ ।  \newline




\markright{ TS 7.5.8.4  \hfill https://www.vedavms.in \hfill}

\section{ TS 7.5.8.4 }

\textbf{TS 7.5.8.4 } \newline
\textbf{Samhita Paata} \newline

-राप्त्यै॑ प॒ञ्चभि॒-स्तिष्ठ॑न्तः स्तुवन्ति देवलो॒कमे॒वाभि ज॑यन्ति प॒ञ्चभि॒रासी॑ना मनुष्यलो॒कमे॒वाभि ज॑यन्ति॒ दश॒ संप॑द्यन्ते॒ दशा᳚क्षरा वि॒राडन्नं॑ ॅवि॒राड् वि॒राजै॒वा-न्नाद्य॒मव॑ रुन्धते पञ्च॒धा वि॑नि॒षद्य॑ स्तुवन्ति॒ पञ्च॒ दिशो॑ दि॒क्षवे॑व प्रति॑तिष्ठ॒न्त्येकै॑क॒याऽस्तु॑तया स॒माय॑न्ति दि॒ग्भ्य ए॒वान्नाद्यꣳ॒॒ सं भ॑रन्ति॒ ताभि॑-रुद्गा॒तोद्-गा॑यति दि॒ग्भ्य ए॒वान्नाद्यꣳ॑-[  ] \newline

\textbf{Pada Paata} \newline

आप्त्यै᳚ । प॒ञ्चभि॒रिति॑ प॒ञ्च - भिः॒ । तिष्ठ॑न्तः । स्तु॒व॒न्ति॒ । दे॒व॒लो॒कमिति॑ देव - लो॒कम् । ए॒व । अ॒भीति॑ । ज॒य॒न्ति॒ । प॒ञ्चभि॒रिति॑ प॒ञ्च - भिः॒ । आसी॑नाः । म॒नु॒ष्य॒लो॒कमिति॑ मनुष्य - लो॒कम् । ए॒व । अ॒भीति॑ । ज॒य॒न्ति॒ । दश॑ । समिति॑ । प॒द्य॒न्ते॒ । दशा᳚क्ष॒रेति॒ दश॑ - अ॒क्ष॒रा॒ । वि॒राडिति॑ वि - राट् । अन्न᳚म् । वि॒राडिति॑ वि - राट् । वि॒राजेति॑ वि - राजा᳚ । ए॒व । अ॒न्नाद्य॒मित्य॑न्न -अद्य᳚म् । अवेति॑ । रु॒न्ध॒ते॒ । प॒ञ्च॒धेति॑ पञ्च - धा । वि॒नि॒षद्येति॑ वि - नि॒षद्य॑ । स्तु॒व॒न्ति॒ । पञ्च॑ । दिशः॑ । दि॒क्षु । ए॒व । प्रतीति॑ । ति॒ष्ठ॒न्ति॒ । एकै॑क॒येत्येक॑या - ए॒क॒या॒ । अस्तु॑तया । स॒माय॒न्तीति॑ सं - आय॑न्ति । दि॒ग्भ्य इति॑ दिक् - भ्यः । ए॒व । अ॒न्नाद्य॒मित्य॑न्न - अद्य᳚म् । समिति॑ । भ॒र॒न्ति॒ । ताभिः॑ । उ॒द्गा॒तेत्यु॑त् - गा॒ता । उदिति॑ । गा॒य॒ति॒ । दि॒ग्भ्य इति॑ दिक् - भ्यः । ए॒व । अ॒न्नाद्य॒मित्य॑न्न - अद्य᳚म् ।  \newline




\markright{ TS 7.5.8.5  \hfill https://www.vedavms.in \hfill}

\section{ TS 7.5.8.5 }

\textbf{TS 7.5.8.5 } \newline
\textbf{Samhita Paata} \newline

सं॒भृत्य॒ तेज॑ आ॒त्मन् द॑धते॒ तस्मा॒देकः॑ प्रा॒णः सर्वा॒ण्यङ्गा᳚न्यव॒त्यथो॒ यथा॑ सुप॒र्ण उ॑त्पति॒ष्यञ्छिर॑ उत्त॒मं कु॑रु॒त ए॒वमे॒व तद्-यज॑मानाः प्र॒जाना॑मुत्त॒मा भ॑वन्त्यास॒न्दी-मु॑द्गा॒ता ऽऽरो॑हति॒ साम्रा᳚ज्यमे॒व ग॑च्छन्ति प्ले॒ङ्खꣳ होता॒ नाक॑स्यै॒व पृ॒ष्ठꣳ रो॑हन्ति कू॒र्चाव॑द्ध्व॒र्यु-र्ब्र॒द्ध्नस्यै॒व वि॒ष्टपं॑ गच्छन्त्ये॒ताव॑न्तो॒ वै दे॑वलो॒कास्तेष्वे॒व य॑थापू॒र्वं प्रति॑ ( ) तिष्ठ॒न्त्यथो॑ आ॒क्रम॑णमे॒व तथ् सेतुं॒ ॅयज॑मानाः कुर्वते सुव॒र्गस्य॑ लो॒कस्य॒ सम॑ष्ट्यै ॥ \newline

\textbf{Pada Paata} \newline

स॒भृंत्येति॑ सं - भृत्य॑ । तेजः॑ । आ॒त्मन्न् । द॒ध॒ते॒ । तस्मा᳚त् । एकः॑ । प्रा॒ण इति॑ प्र - अ॒नः । सर्वा॑णि । अङ्गा॑नि । अ॒व॒ति॒ । अथो॒ इति॑ । यथा᳚ । सु॒प॒र्ण इति॑ सु - प॒र्णः । उ॒त्प॒ति॒ष्यन्नित्यु॑त् - प॒ति॒ष्यन्न् । शिरः॑ । उ॒त्त॒ममित्यु॑त् - त॒मम् । कु॒रु॒ते । ए॒वम् । ए॒व । तत् । यज॑मानाः । प्र॒जाना॒मिति॑ प्र - जाना᳚म् । उ॒त्त॒मा इत्यु॑त् - त॒माः । भ॒व॒न्ति॒ । आ॒स॒न्दीमित्या᳚ - स॒न्दीम् । उ॒द्गा॒तेत्यु॑त् - गा॒ता । एति॑ । रो॒ह॒ति॒ । साम्रा᳚ज्य॒मिति॒ सां - रा॒ज्य॒म् । ए॒व । ग॒च्छ॒न्ति॒ । प्ले॒ङ्खम् । होता᳚ । नाक॑स्य । ए॒व । पृ॒ष्ठम् । रो॒ह॒न्ति॒ । कू॒र्चौ । अ॒द्ध्व॒र्युः । ब्र॒द्ध्नस्य॑ । ए॒व । वि॒ष्टप᳚म् । ग॒च्छ॒न्ति॒ । ए॒ताव॑न्तः । वै । दे॒व॒लो॒का इति॑ देव - लो॒काः । तेषु॑ । ए॒व । य॒था॒पू॒र्वमिति॑ यथा - पू॒र्वम् । प्रतीति॑ ( ) । ति॒ष्ठ॒न्ति॒ । अथो॒ इति॑ । आ॒क्रम॑ण॒मित्या᳚ - क्रम॑णम् । ए॒व । तत् । सेतु᳚म् । यज॑मानाः । कु॒र्व॒ते॒ । सु॒व॒र्गस्येति॑ सुवः-गस्य॑ । लो॒कस्य॑ । सम॑ष्ट्या॒ इति॒ सं - अ॒ष्ट्यै॒ ॥  \newline




\markright{ TS 7.5.9.1  \hfill https://www.vedavms.in \hfill}

\section{ TS 7.5.9.1 }

\textbf{TS 7.5.9.1 } \newline
\textbf{Samhita Paata} \newline

अ॒र्क्ये॑ण॒ वै स॑हस्र॒शः प्र॒जाप॑तिः प्र॒जा अ॑सृजत॒ ताभ्य॒ इला᳚दें॒नेरां॒ ॅलूता॒मवा॑रुन्ध॒ यद॒र्क्यं॑ भव॑ति प्र॒जा ए॒व तद्-यज॑मानाः सृजन्त॒ इला᳚दं भवति प्र॒जाभ्य॑ ए॒व सृ॒ष्टाभ्य॒ इरां॒ ॅलूता॒मव॑ रुन्धते॒ तस्मा॒द्याꣳ समाꣳ॑ स॒त्रꣳ समृ॑द्धं॒ क्षोधु॑का॒स्ताꣳ समां᳚ प्र॒जा इषꣳ॒॒ ह्या॑सा॒मूर्ज॑मा॒दद॑ते॒ याꣳ समां॒ ॅव्यृ॑द्ध॒-मक्षो॑धुका॒स्ताꣳ समां᳚ प्र॒जा - [  ] \newline

\textbf{Pada Paata} \newline

अ॒र्क्ये॑ण । वै । स॒ह॒स्र॒श इति॑ सहस्र - शः । प्र॒जाप॑ति॒रिति॑ प्र॒जा - प॒तिः॒ । प्र॒जा इति॑ प्र - जाः । अ॒सृ॒ज॒त॒ । ताभ्यः॑ । इला᳚न्देन । इरा᳚म् । लूता᳚म् । अवेति॑ । अ॒रु॒न्ध॒ । यत् । अ॒र्क्य᳚म् । भव॑ति । प्र॒जा इति॑ प्र - जाः । ए॒व । तत् । यज॑मानाः । सृ॒ज॒न्ते॒ । इला᳚न्दम् । भ॒व॒ति॒ । प्र॒जाभ्य॒ इति॑ प्र - जाभ्यः॑ । ए॒व । सृ॒ष्टाभ्यः॑ । इरा᳚म् । लूता᳚म् । अवेति॑ । रु॒न्ध॒ते॒ । तस्मा᳚त् । याम् । समा᳚म् । स॒त्रम् । समृ॑द्ध॒मिति॒ सं - ऋ॒द्ध॒म् । क्षोधु॑काः । ताम् । समा᳚म् । प्र॒जा इति॑ प्र - जाः । इष᳚म् । हि । आ॒सा॒म् । ऊर्ज᳚म् । आ॒दद॑त॒ इत्या᳚ - दद॑ते । याम् । समा᳚म् । व्यृ॑द्ध॒मिति॒ वि - ऋ॒द्ध॒म् । अक्षो॑धुकाः । ताम् । समा᳚म् । प्र॒जा इति॑ प्र - जाः ।  \newline




\markright{ TS 7.5.9.2  \hfill https://www.vedavms.in \hfill}

\section{ TS 7.5.9.2 }

\textbf{TS 7.5.9.2 } \newline
\textbf{Samhita Paata} \newline

न ह्या॑सा॒मिष॒मूर्ज॑मा॒दद॑त उत्क्रो॒दं कु॑र्वते॒ यथा॑ ब॒न्धान्-मु॑मुचा॒ना उ॑त्क्रो॒दं कु॒र्वत॑ ए॒वमे॒व तद्-यज॑माना देवब॒न्धान्-मु॑मुचा॒ना उ॑त्क्रो॒दं कु॑र्वत॒ इष॒मूर्ज॑मा॒त्मन् दधा॑ना वा॒णः श॒तत॑न्तुर्भवति श॒तायुः॒ पुरु॑षः श॒तेन्द्रि॑य॒ आयु॑ष्ये॒वेन्द्रि॒ये प्रति॑ तिष्ठन्त्या॒जिं धा॑व॒न्त्यन॑भिजितस्या॒-भिजि॑त्यै दुन्दु॒भीन्थ् स॒माघ्न॑न्ति पर॒मा वा ए॒षा वाग्या दु॑न्दु॒भौ प॑र॒मामे॒व - [  ] \newline

\textbf{Pada Paata} \newline

न । हि । आ॒सा॒म् । इष᳚म् । ऊर्ज᳚म् । आ॒दद॑त॒ इत्या᳚ - दद॑ते । उ॒त्क्रो॒दमित्यु॑त् - क्रो॒दम् । कु॒र्व॒ते॒ । यथा᳚ । ब॒न्धात् । मु॒मु॒चा॒नाः । उ॒त्क्रो॒दमित्यु॑त् - क्रो॒दम् । कु॒र्वते᳚ । ए॒वम् । ए॒व । तत् । यज॑मानाः । दे॒व॒ब॒न्धादिति॑ देव-ब॒न्धात् । मु॒मु॒चा॒नाः । उ॒त्क्रो॒दमित्यु॑त् - क्रो॒दम् । कु॒र्व॒ते॒ । इष᳚म् । ऊर्ज᳚म् । आ॒त्मन्न् । दधा॑नाः । वा॒णः । श॒तत॑न्तु॒रिति॑ श॒त - त॒न्तुः॒ । भ॒व॒ति॒ । श॒तायु॒रिति॑ श॒त - आ॒युः॒ । पुरु॑षः । श॒तेन्द्रि॑य॒ इति॑ श॒त - इ॒न्द्रि॒यः॒ । आयु॑षि । ए॒व । इ॒न्द्रि॒ये । प्रतीति॑ । ति॒ष्ठ॒न्ति॒ । आ॒जिम् । धा॒व॒न्ति॒ । अन॑भिजित॒स्येत्यन॑भि - जि॒त॒स्य॒ । अ॒भिजि॑त्या॒ इत्य॒भि - जि॒त्यै॒ । दु॒न्दु॒भीन् । स॒माघ्न॒न्तीति॑ सं - आघ्न॑न्ति । प॒र॒मा । वै । ए॒षा । वाक् । या । दु॒न्दु॒भौ । प॒र॒माम् । ए॒व ।  \newline




\markright{ TS 7.5.9.3  \hfill https://www.vedavms.in \hfill}

\section{ TS 7.5.9.3 }

\textbf{TS 7.5.9.3 } \newline
\textbf{Samhita Paata} \newline

वाच॒मव॑ रुन्धते भूमिदुन्दु॒भिमा घ्न॑न्ति॒ यैवेमां ॅवाक् प्रवि॑ष्टा॒ तामे॒वाव॑ रुन्ध॒ते ऽथो॑ इ॒मामे॒व ज॑यन्ति॒ सर्वा॒ वाचो॑ वदन्ति॒ सर्वा॑सां ॅवा॒चामव॑रुद्ध्या आ॒र्द्रेचर्म॒न् व्याय॑च्छेते इन्द्रि॒यस्या व॑रुद्ध्या॒ आऽन्यः क्रोश॑ति॒ प्रान्यः शꣳ॑सति॒ य आ॒क्रोश॑ति पु॒नात्ये॒वैना॒न्थ्स यः प्र॒शꣳस॑ति पू॒तेष्वे॒वान्नाद्यं॑ दधा॒त्यृषि॑कृतं च॒ - [  ] \newline

\textbf{Pada Paata} \newline

वाच᳚म् । अवेति॑ । रु॒न्ध॒ते॒ । भू॒मि॒दु॒न्दु॒भिमिति॑ भूमि - दु॒न्दु॒भिम् । एति॑ । घ्न॒न्ति॒ । या । ए॒व । इ॒माम् । वाक् । प्रवि॒ष्टेति॒ प्र - वि॒ष्टा॒ । ताम् । ए॒व । अवेति॑ । रु॒न्ध॒ते॒ । अथो॒ इति॑ । इ॒माम् । ए॒व । ज॒य॒न्ति॒ । सर्वाः᳚ । वाचः॑ । व॒द॒न्ति॒ । सर्वा॑साम् । वा॒चाम् । अव॑रुद्ध्या॒ इत्यव॑ - रु॒द्ध्यै॒ । आ॒र्द्रे । चर्मन्न्॑ । व्याय॑च्छेते॒ इति॑ वि-आय॑च्छेते । इ॒न्द्रि॒यस्य॑ । अव॑रुद्ध्या॒ इत्यव॑ - रु॒द्ध्यै॒ । एति॑ । अ॒न्यः । क्रोश॑ति । प्रेति॑ । अ॒न्यः । शꣳ॒॒स॒ति॒ । यः । आ॒क्रोश॒तीत्या᳚-क्रोश॑ति । पु॒नाति॑ । ए॒व । ए॒ना॒न् । सः । यः । प्र॒शꣳस॒तीति॑ प्र - शꣳस॑ति । पू॒तेषु॑ । ए॒व । अ॒न्नाद्य॒मित्य॑न्न - अद्य᳚म् । द॒धा॒ति॒ । ऋषि॑कृत॒मित्यृषि॑-कृ॒त॒म् । च॒ ।  \newline




\markright{ TS 7.5.9.4  \hfill https://www.vedavms.in \hfill}

\section{ TS 7.5.9.4 }

\textbf{TS 7.5.9.4 } \newline
\textbf{Samhita Paata} \newline

वा ए॒ते दे॒वकृ॑तं च॒ पूर्वै॒र्मासै॒रव॑ रुन्धते॒ यद्-भू॑ते॒च्छदाꣳ॒॒ सामा॑नि॒ भव॑न्त्यु॒भय॒स्याव॑रुद्ध्यै॒ यन्ति॒ वा ए॒ते मि॑थु॒नाद्ये सं॑ॅवथ्स॒र-मु॑प॒यन्त्य॑न्तर्वे॒दि मि॑थु॒नौ सं भ॑वत॒स्तेनै॒व मि॑थु॒नान्न य॑न्ति ॥ \newline

\textbf{Pada Paata} \newline

वै । ए॒ते । दे॒वकृ॑त॒मिति॑ दे॒व-कृ॒त॒म् । च॒ । पूर्वैः᳚ । मासैः᳚ । अवेति॑ । रु॒न्ध॒ते॒ । यत् । भू॒ते॒च्छदा॒मिति॑ भूते - छदा᳚म् । सामा॑नि । भव॑न्ति । उ॒भय॑स्य । अव॑रुद्ध्या॒ इत्यव॑ - रु॒द्ध्यै॒ । यन्ति॑ । वै । ए॒ते । मि॒थु॒नात् । ये । सं॒ॅव॒थ्स॒रमिति॑ सं-व॒थ्स॒रम् । उ॒प॒यन्तीत्यु॑प-यन्ति॑ । अ॒न्त॒र्वे॒दीत्य॑न्तः - वे॒दि । मि॒थु॒नौ । समिति॑ । भ॒व॒तः॒ । तेन॑ । ए॒व । मि॒थु॒नात् । न । य॒न्ति॒ ॥  \newline




\markright{ TS 7.5.10.1  \hfill https://www.vedavms.in \hfill}

\section{ TS 7.5.10.1 }

\textbf{TS 7.5.10.1 } \newline
\textbf{Samhita Paata} \newline

चर्माव॑ भिन्दन्ति पा॒प्मान॑मे॒वैषा॒मव॑ भिन्दन्ति॒ माऽप॑ राथ्सी॒र्माऽति॑ व्याथ्सी॒रित्या॑ह संप्र॒त्ये॑वैषां᳚ पा॒प्मान॒मव॑ भिन्दन्त्युदकु॒म्भान॑धिनि॒धाय॑ दा॒स्यो॑ मार्जा॒लीयं॒ परि॑ नृत्यन्ति प॒दो नि॑घ्न॒तीरि॒दंम॑धुं॒ गाय॑न्त्यो॒ मधु॒ वै दे॒वानां᳚ पर॒म-म॒न्नाद्यं॑ पर॒ममे॒वा-न्नाद्य॒मव॑ रुन्धते प॒दो नि घ्न॑न्ति मही॒यामे॒वैषु॑ दधति ॥ \newline

\textbf{Pada Paata} \newline

चर्म॑ । अवेति॑ । भि॒न्द॒न्ति॒ । पा॒प्मान᳚म् । ए॒व । ए॒षा॒म् । अवेति॑ । भि॒न्द॒न्ति॒ । मा । अपेति॑ । रा॒थ्सीः॒ । मा । अतीति॑ । व्या॒थ्सीः॒ । इति॑ । आ॒ह॒ । स॒प्रं॒तीति॑ सं - प्र॒ति । ए॒व । ए॒षा॒म् । पा॒प्मान᳚म् । अवेति॑ । भि॒न्द॒न्ति॒ । उ॒द॒कु॒म्भानित्यु॑द-कु॒म्भान् । अ॒धि॒नि॒धायेत्य॑धि - नि॒धाय॑ । दा॒स्यः॑ । मा॒र्जा॒लीय᳚म् । परीति॑ । नृ॒त्य॒न्ति॒ । प॒दः । नि॒घ्न॒तीरिति॑ नि - घ्न॒तीः । इ॒दंम॑धु॒मिती॒दं - म॒धु॒म् । गाय॑न्त्यः । मधु॑ । वै । दे॒वाना᳚म् । प॒र॒मम् । अ॒न्नाद्य॒मित्य॑न्न - अद्य᳚म् । प॒र॒मम् । ए॒व । अ॒न्नाद्य॒मित्य॑न्न - अद्य᳚म् । अवेति॑ । रु॒न्ध॒ते॒ । प॒दः । नीति॑ । घ्न॒न्ति॒ । म॒ही॒याम् । ए॒व । ए॒षु॒ । द॒ध॒ति॒ ॥  \newline




\markright{ TS 7.5.11.1  \hfill https://www.vedavms.in \hfill}

\section{ TS 7.5.11.1 }

\textbf{TS 7.5.11.1 } \newline
\textbf{Samhita Paata} \newline

पृ॒थि॒व्यै स्वाहा॒ ऽन्तरि॑क्षाय॒ स्वाहा॑ दि॒वे स्वाहा॑ संप्लोष्य॒ते स्वाहा॑ स॒प्लंव॑मानाय॒ स्वाहा॒ संप्लु॑ताय॒ स्वाहा॑ मेघायिष्य॒ते स्वाहा॑ मेघाय॒ते स्वाहा॑ मेघि॒ताय॒ स्वाहा॑ मे॒घाय॒ स्वाहा॑ नीहा॒राय॒ स्वाहा॑ नि॒हाका॑यै॒ स्वाहा᳚ प्रास॒चाय॒ स्वाहा᳚ प्रच॒लाका॑यै॒ स्वाहा॑ विद्योतिष्य॒ते स्वाहा॑ वि॒द्योत॑मानाय॒ स्वाहा॑ संॅवि॒द्योत॑मानाय॒ स्वाहा᳚ स्तनयिष्य॒ते स्वाहा᳚ स्त॒नय॑ते॒ स्वाहो॒ -ग्रꣳ स्त॒नय॑ते॒ स्वाहा॑ वर्.षिष्य॒ते स्वाहा॒ वर्.ष॑ते॒ स्वाहा॑ ऽभि॒वर्.ष॑ते॒ स्वाहा॑ परि॒वर्.ष॑ते॒ स्वाहा॑ सं॒ॅवर्.ष॑ते॒ - [  ] \newline

\textbf{Pada Paata} \newline

पृ॒थि॒व्यै । स्वाहा᳚ । अ॒न्तरि॑क्षाय । स्वाहा᳚ । दि॒वे । स्वाहा᳚ । स॒प्ल्ॐ॒ष्य॒त इति॑ सं - प्लो॒ष्य॒ते । स्वाहा᳚ । स॒प्लंव॑माना॒येति॑ सं - प्लव॑मानाय । स्वाहा᳚ । संप्लु॑ता॒येति॒ सं-प्लु॒ता॒य॒ । स्वाहा᳚ । मे॒घा॒यि॒ष्य॒ते । स्वाहा᳚ । मे॒घा॒य॒त इति॑ मेघ - य॒ते । स्वाहा᳚ । मे॒घि॒ताय॑ । स्वाहा᳚ । मे॒घाय॑ । स्वाहा᳚ । नी॒हा॒राय॑ । स्वाहा᳚ । नि॒हाका॑या॒ इति॑ नि-हाका॑यै । स्वाहा᳚ । प्रा॒स॒चाय॑ । स्वाहा᳚ । प्र॒च॒लाका॑या॒ इति॑ प्र - च॒लाका॑यै । स्वाहा᳚ । वि॒द्यो॒ति॒ष्य॒त इति॑ वि - द्यो॒ति॒ष्य॒ते । स्वाहा᳚ । वि॒द्योत॑माना॒येति॑ वि - द्योत॑मानाय । स्वाहा᳚ । सं॒ॅवि॒द्योत॑माना॒येति॑ सं-वि॒द्योत॑मानाय । स्वाहा᳚ । स्त॒न॒यि॒ष्य॒ते । स्वाहा᳚ । स्त॒नय॑ते । स्वाहा᳚ । उ॒ग्रम् । स्त॒नय॑ते । स्वाहा᳚ । व॒र्.॒षि॒ष्य॒ते । स्वाहा᳚ । वर्.ष॑ते । स्वाहा᳚ । अ॒भि॒वर्.ष॑त॒ इत्य॑भि - वर्.ष॑ते । स्वाहा᳚ । प॒रि॒वर्.ष॑त॒ इति॑ परि - वर्.ष॑ते । स्वाहा᳚ । सं॒ॅवर्.ष॑त॒ इति॑ सं - वर्.ष॑ते ।  \newline




\markright{ TS 7.5.11.2  \hfill https://www.vedavms.in \hfill}

\section{ TS 7.5.11.2 }

\textbf{TS 7.5.11.2 } \newline
\textbf{Samhita Paata} \newline

स्वाहा॑ ऽनु॒वर्.ष॑ते॒ स्वाहा॑ शीकायिष्य॒ते स्वाहा॑ शीकाय॒ते स्वाहा॑ शीकि॒ताय॒ स्वाहा᳚प्रोषिष्य॒ते स्वाहा᳚ प्रुष्ण॒ते स्वाहा॑ परिप्रुष्ण॒ते स्वाहो᳚-द्ग्रहीष्य॒ते स्वाहो᳚ द्गृह्ण॒ते स्वाहो-द्गृ॑हीताय॒ स्वाहा॑ विप्लोष्य॒ते स्वाहा॑ वि॒प्लव॑मानाय॒ स्वाहा॒ विप्लु॑ताय॒ स्वाहा॑ ऽऽतफ्स्य॒ते स्वाहा॒ ऽऽतप॑ते ॒स्वाहो॒-ग्रमा॒तप॑ते॒ स्वाह॒ -र्ग्भ्यः स्वाहा॒ यजु॑र्भ्यः॒ स्वाहा॒ साम॑भ्यः॒ स्वाहा ऽङ्गि॑रोभ्यः॒ स्वाहा॒ वेदे᳚भ्यः॒ स्वाहा॒ गाथा᳚भ्यः॒ स्वाहा॑ नाराशꣳ॒॒सीभ्यः॒ स्वाहा॒ रैभी᳚भ्यः स्वाहा॒ ( ) सर्व॑स्मै॒ स्वाहा᳚ ॥ \newline

\textbf{Pada Paata} \newline

स्वाहा᳚ । अ॒नु॒वर्.ष॑त॒ इत्य॑नु - वर्.ष॑ते । स्वाहा᳚ । शी॒का॒यि॒ष्य॒ते । स्वाहा᳚ । शी॒का॒य॒त इति॑ शीक - य॒ते । स्वाहा᳚ । शी॒कि॒ताय॑ । स्वाहा᳚ । प्रो॒षि॒ष्य॒ते । स्वाहा᳚ । प्रु॒ष्ण॒ते । स्वाहा᳚ । प॒रि॒प्रु॒ष्ण॒त इति॑ परि - प्रु॒ष्ण॒ते । स्वाहा᳚ । उ॒द्ग्र॒ही॒ष्य॒त इत्यु॑त् - ग्र॒ही॒ष्य॒ते । स्वाहा᳚ । उ॒द्गृ॒ह्ण॒त इत्यु॑त् - गृ॒ह्ण॒ते । स्वाहा᳚ । उद्गृ॑हीता॒येत्युत् -   गृ॒ही॒ता॒य॒ । स्वाहा᳚ । वि॒प्लो॒ष्य॒त इति॑ वि - प्लो॒ष्य॒ते । स्वाहा᳚ । वि॒प्लव॑माना॒येति॑ वि - प्लव॑मानाय । स्वाहा᳚ । विप्लु॑ता॒येति॒ वि - प्लु॒ता॒य॒ । स्वाहा᳚ । आ॒त॒फ्स्य॒त इत्या᳚ - त॒फ्स्य॒ते । स्वाहा᳚ । आ॒तप॑त॒ इत्या᳚ - तप॑ते । स्वाहा᳚ । उ॒ग्रम् । आ॒तप॑त॒ इत्या᳚ - तप॑ते । स्वाहा᳚ । ऋ॒ग्भ्य इत्यृ॑क् - भ्यः । स्वाहा᳚ । यजु॑र्भ्य॒ इति॒ यजुः॑ - भ्यः॒ । स्वाहा᳚ । साम॑भ्य॒ इति॒ साम॑ - भ्यः॒ । स्वाहा᳚ । अङ्गि॑रोभ्य॒ इत्यङ्गि॑रः -भ्यः॒ । स्वाहा᳚ । वेदे᳚भ्यः । स्वाहा᳚ । गाथा᳚भ्यः । स्वाहा᳚ । ना॒रा॒शꣳ॒॒सीभ्यः॑ । स्वाहा᳚ । रैभी᳚भ्यः । स्वाहा॑ ( ) । सर्व॑स्मै । स्वाहा᳚ ॥  \newline




\markright{ TS 7.5.12.1  \hfill https://www.vedavms.in \hfill}

\section{ TS 7.5.12.1 }

\textbf{TS 7.5.12.1 } \newline
\textbf{Samhita Paata} \newline

द॒त्वते॒ स्वाहा॑ ऽद॒न्तका॑य॒ स्वाहा᳚ प्रा॒णिने॒ स्वाहा᳚ ऽप्रा॒णाय॒ स्वाहा॒ मुख॑वते॒ स्वाहा॑ऽमु॒खाय॒ स्वाहा॒ नासि॑कवते॒ स्वाहा॑ ऽनासि॒काय॒ स्वाहा᳚ ऽक्ष॒ण्वते॒ स्वाहा॑ऽन॒क्षिका॑य॒ स्वाहा॑ क॒र्णिने॒ स्वाहा॑ ऽक॒र्णका॑य॒ स्वाहा॑ शीर्.ष॒ण्वते॒ स्वाहा॑ऽशी॒र्॒.षका॑य॒ स्वाहा॑ प॒द्वते॒ स्वाहा॑ ऽपा॒दका॑य॒ स्वाहा᳚ प्राण॒ते स्वाहा ऽप्रा॑णते॒ स्वाहा॒ वद॑ते॒ स्वाहा ऽव॑दते॒ स्वाहा॒ पश्य॑ते॒ स्वाहा ऽप॑श्यते॒ स्वाहा॑ शृण्व॒ते स्वाहा ऽशृ॑ण्वते॒ स्वाहा॑ मन॒स्विने॒ स्वाहा॑ - [  ] \newline

\textbf{Pada Paata} \newline

द॒त्वते᳚ । स्वाहा᳚ । अ॒द॒न्तका॑य । स्वाहा᳚ । प्रा॒णिने᳚ । स्वाहा᳚ । अ॒प्रा॒णाय॑ । स्वाहा᳚ । मुख॑वत॒ इति॒ मुख॑-व॒ते॒ । स्वाहा᳚ । अ॒मु॒खाय॑ । स्वाहा᳚ । नासि॑कवत॒ इति॒ नासि॑क - व॒ते॒ । स्वाहा᳚ । अ॒ना॒सि॒काय॑ । स्वाहा᳚ । अ॒क्ष॒ण्वत॒ इत्य॑क्षण्-वते᳚ । स्वाहा᳚ । अ॒न॒क्षिका॑य । स्वाहा᳚ । क॒र्णिने᳚ । स्वाहा᳚ । अ॒क॒र्णका॑य । स्वाहा᳚ । शी॒र्.॒ष॒ण्वत॒ इति॑ शीर्.षण् - वते᳚ । स्वाहा᳚ । अ॒शी॒र्॒.षका॑य । स्वाहा᳚ । प॒द्वत॒ इति॑ पत् - वते᳚ । स्वाहा᳚ । अ॒पा॒दका॑य । स्वाहा᳚ । प्रा॒ण॒त इति॑ प्र-अ॒न॒ते । स्वाहा᳚ । अप्रा॑णत॒ इत्यप्र॑ - अ॒न॒ते॒ । स्वाहा᳚ । वद॑ते । स्वाहा᳚ । अव॑दते । स्वाहा᳚ । पश्य॑ते । स्वाहा᳚ । अप॑श्यते । स्वाहा᳚ । शृ॒ण्व॒ते । स्वाहा᳚ । अशृ॑ण्वते । स्वाहा᳚ । म॒न॒स्विने᳚ । स्वाहा᳚ ।  \newline




\markright{ TS 7.5.12.2  \hfill https://www.vedavms.in \hfill}

\section{ TS 7.5.12.2 }

\textbf{TS 7.5.12.2 } \newline
\textbf{Samhita Paata} \newline

ऽम॒नसे॒ स्वाहा॑ रेत॒स्विने॒ स्वाहा॑ ऽरे॒तस्का॑य॒ स्वाहा᳚ प्र॒जाभ्यः॒ स्वाहा᳚ प्र॒जन॑नाय॒ स्वाहा॒ लोम॑वते॒ स्वाहा॑ ऽलो॒मका॑य॒ स्वाहा᳚ त्व॒चे स्वाहा॒ ऽत्वक्का॑य॒ स्वाहा॒ चर्म॑ण्वते॒ स्वाहा॑ ऽच॒र्मका॑य॒ स्वाहा॒ लोहि॑तवते॒ स्वाहा॑ऽलोहि॒ताय॒ स्वाहा॑ माꣳस॒न्वते॒ स्वाहा॑ ऽमाꣳ॒॒सका॑य॒ स्वाहा॒ स्नाव॑भ्यः॒ स्वाहा᳚ ऽस्ना॒वका॑य॒ स्वाहा᳚ स्थ॒न्वते॒ स्वाहा॑ऽन॒स्थिका॑य॒ स्वाहा॑ मज्ज॒न्वते॒ स्वाहा॑ ऽम॒ज्जका॑य॒ स्वाहा॒ ऽङ्गिने॒ स्वाहा॑ऽन॒ङ्गाय॒ स्वाहा॒ ऽऽत्मने॒ स्वाहा ऽना᳚त्मने॒ स्वाहा॒ ( ) सर्व॑स्मै॒ स्वाहा᳚ ॥ \newline

\textbf{Pada Paata} \newline

अ॒म॒नसे᳚ । स्वाहा᳚ । रे॒त॒स्विने᳚ । स्वाहा᳚ । अ॒रे॒तस्का॒येत्य॑रे॒तः - का॒य॒ । स्वाहा᳚ । प्र॒जाभ्य॒ इति॑ प्र - जाभ्यः॑ । स्वाहा᳚ । प्र॒जन॑ना॒येति॑ प्र - जन॑नाय । स्वाहा᳚ । लोम॑वत॒ इति॒ लोम॑ - व॒ते॒ । स्वाहा᳚ । अ॒लो॒मका॑य । स्वाहा᳚ । त्व॒चे । स्वाहा᳚ । अ॒त्वक्का॑य । स्वाहा᳚ । चर्म॑ण्वत॒ इति॒ चर्मण्॑ -   व॒ते॒ । स्वाहा᳚ । अ॒च॒र्मका॑य । स्वाहा᳚ । लोहि॑तवत॒ इति॒ लोहि॑त - व॒ते॒ । स्वाहा᳚ । अ॒लो॒हि॒ताय॑ । स्वाहा᳚ । माꣳ॒॒स॒न्वत॒ इति॑ माꣳसन्न्-वते᳚ । स्वाहा᳚ । अ॒माꣳ॒॒सका॑य । स्वाहा᳚ । स्नाव॑भ्य॒ इति॒ स्नाव॑ - भ्यः॒ । स्वाहा᳚ । अ॒स्ना॒वका॑य । स्वाहा᳚ । अ॒स्थ॒न्वत॒ इत्य॑स्थन्न् - वते᳚ । स्वाहा᳚ । अ॒न॒स्थिका॑य । स्वाहा᳚ । म॒ज्ज॒न्वत॒ इति॑ मज्जन्न् - वते᳚ । स्वाहा᳚ । अ॒म॒ज्जका॑य । स्वाहा᳚ । अ॒ङ्गिने᳚ । स्वाहा᳚ । अ॒न॒ङ्गाय॑ । स्वाहा᳚ । आ॒त्मने᳚ । स्वाहा᳚ । अना᳚त्मने । स्वाहा᳚ ( ) । सर्व॑स्मै । स्वाहा᳚ ॥  \newline




\markright{ TS 7.5.13.1  \hfill https://www.vedavms.in \hfill}

\section{ TS 7.5.13.1 }

\textbf{TS 7.5.13.1 } \newline
\textbf{Samhita Paata} \newline

कस्त्वा॑ युनक्ति॒ स त्वा॑ युनक्तु॒ विष्णु॑स्त्वा युनक्त्व॒स्य य॒ज्ञ्स्यर्द्ध्यै॒ मह्यꣳ॒॒ सन्न॑त्या अ॒मुष्मै॒ कामा॒याऽऽ*यु॑षे त्वा प्रा॒णाय॑ त्वा ऽपा॒नाय॑ त्वा व्या॒नाय॑ त्वा॒ व्यु॑ष्ट्यै त्वा र॒य्यै त्वा॒ राध॑से त्वा॒ घोषा॑य त्वा॒ पोषा॑य त्वा ऽऽराद्घो॒षाय॑ त्वा॒ प्रच्यु॑त्यै त्वा ॥ \newline

\textbf{Pada Paata} \newline

कः । त्वा॒ । यु॒न॒क्ति॒ । सः । त्वा॒ । यु॒न॒क्तु॒ । विष्णुः॑ । त्वा॒ । यु॒न॒क्तु॒ । अ॒स्य । य॒ज्ञ्स्य॑ । ऋद्ध्यै᳚ । मह्य᳚म् । सन्न॑त्या॒ इति॒ सं - न॒त्यै॒ । अ॒मुष्मै᳚ । कामा॑य । आयु॑षे । त्वा॒ । प्रा॒णायेति॑ प्र - अ॒नाय॑ । त्वा॒ । अ॒पा॒नायेत्य॑प - अ॒नाय॑ । त्वा॒ । व्या॒नायेति॑ वि - अ॒नाय॑ । त्वा॒ । व्यु॑ष्ट्या॒ इति॒ वि - उ॒ष्ट्यै॒ । त्वा॒ । र॒य्यै । त्वा॒ । राध॑से । त्वा॒ । घोषा॑य । त्वा॒ । पोषा॑य । त्वा॒ । आ॒रा॒द्घो॒षायेत्या॑रात् - घो॒षाय॑ । त्वा॒ । प्रच्यु॑त्या॒ इति॒ प्र - च्यु॒त्यै॒ । त्वा॒ ॥  \newline




\markright{ TS 7.5.14.1  \hfill https://www.vedavms.in \hfill}

\section{ TS 7.5.14.1 }

\textbf{TS 7.5.14.1 } \newline
\textbf{Samhita Paata} \newline

अ॒ग्नये॑ गाय॒त्राय॑ त्रि॒वृते॒ राथ॑न्तराय वास॒न्ताया॒-ष्टाक॑पाल॒ इन्द्रा॑य॒ त्रैष्टु॑भाय पञ्चद॒शाय॒ बार्.ह॑ताय॒ ग्रैष्मा॒यैका॑दशकपालो॒ विश्वे᳚भ्यो दे॒वेभ्यो॒ जाग॑तेभ्यः सप्तद॒शेभ्यो॑ वैरू॒पेभ्यो॒ वार्.षि॑केभ्यो॒ द्वाद॑शकपालो मि॒त्रावरु॑णाभ्या॒-मानु॑ष्टुभाभ्या-मेकविꣳ॒॒शाभ्यां᳚ ॅवैरा॒जाभ्याꣳ॑ शार॒दाभ्यां᳚ पय॒स्या॑ बृह॒स्पत॑ये॒ पाङ्क्ता॑य त्रिण॒वाय॑ शाक्व॒राय॒ हैम॑न्तिकाय च॒रुः स॑वि॒त्र आ॑तिच्छन्द॒साय॑ त्रयस्त्रिꣳ॒॒शाय॑ रैव॒ताय॑ शैशि॒राय॒ द्वाद॑शकपा॒लो ऽदि॑त्यै॒ विष्णु॑पत्न्यै च॒रुर॒ग्नये॑ वैश्वान॒राय॒ द्वाद॑शकपा॒लो ऽनु॑मत्यै च॒रुः का॒य एक॑कपालः ॥ \newline

\textbf{Pada Paata} \newline

अ॒ग्नये᳚ । गा॒य॒त्राय॑ । त्रि॒वृत॒ इति॑ त्रि-वृते᳚ । राथ॑न्तरा॒येति॒ राथं᳚-त॒रा॒य॒ । वा॒स॒न्ताय॑ । अ॒ष्टाक॑पाल॒ इत्य॒ष्टा - क॒पा॒लः॒ । इन्द्रा॑य । त्रैष्टु॑भाय । प॒ञ्च॒द॒शायेति॑ पञ्च - द॒शाय॑ । बार्.ह॑ताय । ग्रैष्मा॑य । एका॑दशकपाल॒ इत्येका॑दश - क॒पा॒लः॒ । विश्वे᳚भ्यः । दे॒वेभ्यः॑ । जाग॑तेभ्यः । स॒प्त॒द॒शेभ्य॒ इति॑ सप्त - द॒शेभ्यः॑ । वै॒रू॒पेभ्यः॑ । वार्.षि॑केभ्यः । द्वाद॑शकपाल॒ इति॒ द्वाद॑श - क॒पा॒लः॒ । मि॒त्रावरु॑णाभ्या॒मिति॑ मि॒त्रा - वरु॑णाभ्याम् । आनु॑ष्टुभाभ्या॒मित्यानु॑- स्तु॒भा॒भ्या॒म् । ए॒क॒विꣳ॒॒शाभ्या॒मित्ये॑क-विꣳ॒॒शाभ्या᳚म् । वै॒रा॒जाभ्या᳚म् । शा॒र॒दाभ्या᳚म् । प॒य॒स्या᳚ । बृह॒स्पत॑ये । पाङ्क्ता॑य । त्रि॒ण॒वायेति॑ त्रि - न॒वाय॑ । शा॒क्व॒राय॑ । हैम॑न्तिकाय । च॒रुः । स॒वि॒त्रे । आ॒ति॒च्छ॒न्द॒सायेत्या॑ति - छ॒न्द॒साय॑ । त्र॒य॒स्त्रिꣳ॒॒शायेति॑ त्रयः - त्रिꣳ॒॒शाय॑ । रै॒व॒ताय॑ । शै॒शि॒राय॑ । द्वाद॑शकपाल॒ इति॒ द्वाद॑श -क॒पा॒लः॒ । अदि॑त्यै । विष्णु॑पत्न्या॒ इति॒ विष्णु॑ - प॒त्न्यै॒ । च॒रुः । अ॒ग्नये᳚ । वै॒श्वा॒न॒राय॑ । द्वाद॑शकपाल॒ इति॒ द्वाद॑श - क॒पा॒लः॒ । अनु॑मत्या॒ इत्यनु॑ - म॒त्यै॒ । च॒रुः । का॒यः । एक॑कपाल॒ इत्येक॑-क॒पा॒लः॒ ॥  \newline




\markright{ TS 7.5.15.1  \hfill https://www.vedavms.in \hfill}

\section{ TS 7.5.15.1 }

\textbf{TS 7.5.15.1 } \newline
\textbf{Samhita Paata} \newline

यो वा अ॒ग्नाव॒ग्निः प्र॑ह्रि॒यते॒ यश्च॒ सोमो॒ राजा॒ तयो॑रे॒ष आ॑ति॒थ्यं ॅयद॑ग्नीषो॒मीयोऽथै॒ष रु॒द्रो यश्ची॒यते॒ यथ् संचि॑ते॒ऽग्नावे॒तानि॑ ह॒वीꣳषि॒ न नि॒र्वपे॑दे॒ष ए॒व रु॒द्रोऽशा᳚न्त उपो॒त्थाय॑ प्र॒जां प॒शून् यज॑मानस्या॒भि म॑न्येत॒ यथ् संचि॑ते॒ऽग्नावे॒तानि॑ ह॒वीꣳषि॑ नि॒र्वप॑ति भाग॒धेये॑नै॒वैनꣳ॑ शमयति॒ नास्य॑ रु॒द्रोऽशा᳚न्त - [  ] \newline

\textbf{Pada Paata} \newline

यः । वै । अ॒ग्नौ । अ॒ग्निः । प्र॒ह्रि॒यत॒ इति॑ प्र - ह्रि॒यते᳚ । यः । च॒ । सोमः॑ । राजा᳚ । तयोः᳚ । ए॒षः । आ॒ति॒थ्यम् । यत् । अ॒ग्नी॒षो॒मीय॒ इत्य॑ग्नी - सो॒मीयः॑ । अथ॑ । ए॒षः । रु॒द्रः । यः । ची॒यते᳚ । यत् । सञ्चि॑त॒ इति॒ सं - चि॒ते॒ । अ॒ग्नौ । ए॒तानि॑ । ह॒वीꣳषि॑ । न । नि॒र्वपे॒दिति॑ निः - वपे᳚त् । ए॒षः । ए॒व । रु॒द्रः । अशा᳚न्तः । उ॒पो॒त्थायेत्यु॑प - उ॒त्थाय॑ । प्र॒जामिति॑ प्र - जाम् । प॒शून् । यज॑मानस्य । अ॒भीति॑ । म॒न्ये॒त॒ । यत् । सञ्चि॑त॒ इति॒ सं-चि॒ते॒ । अ॒ग्नौ । ए॒तानि॑ । ह॒वीꣳषि॑ । नि॒र्वप॒तीति॑ निः - वप॑ति । भा॒ग॒धेये॒नेति॑ भाग - धेये॑न । ए॒व । ए॒न॒म् । श॒म॒य॒ति॒ । न । अ॒स्य॒ । रु॒द्रः । अशा᳚न्तः ।  \newline




\markright{ TS 7.5.15.2  \hfill https://www.vedavms.in \hfill}

\section{ TS 7.5.15.2 }

\textbf{TS 7.5.15.2 } \newline
\textbf{Samhita Paata} \newline

उपो॒त्थाय॑ प्र॒जां प॒शून॒भि म॑न्यते॒ दश॑ ह॒वीꣳषि॑ भवन्ति॒ नव॒ वै पुरु॑षे प्रा॒णा नाभि॑र्दश॒मी प्रा॒णाने॒व यज॑माने दधा॒त्यथो॒ दशा᳚क्षरा वि॒राडन्नं॑ ॅवि॒राड् वि॒राज्ये॒वान्नाद्ये॒ प्रति॑ तिष्ठत्यृ॒तुभि॒र्वा ए॒ष छन्दो॑भिः॒ स्तोमैः᳚ पृ॒ष्ठैश्चे॑त॒व्य॑ इत्या॑हु॒र्यदे॒तानि॑ ह॒वीꣳषि॑ नि॒र्वप॑त्यृ॒तुभि॑रे॒वैनं॒ छन्दो॑भिः॒ स्तोमैः᳚ पृ॒ष्ठैश्चि॑नुते॒ दिशः॑ सुषुवा॒णेना॑ - [  ] \newline

\textbf{Pada Paata} \newline

उ॒पो॒त्थायेत्यु॑प - उ॒त्थाय॑ । प्र॒जामिति॑ प्र-जाम् । प॒शून् । अ॒भीति॑ । म॒न्य॒ते॒ । दश॑ । ह॒वीꣳषि॑ । भ॒व॒न्ति॒ । नव॑ । वै । पुरु॑षे । प्रा॒णा इति॑ प्र - अ॒नाः । नाभिः॑ । द॒श॒मी । प्रा॒णानिति॑ प्र - अ॒नान् । ए॒व । यज॑माने । द॒धा॒ति॒ । अथो॒ इति॑ । दशा᳚क्ष॒रेति॒ दश॑ - अ॒क्ष॒रा॒ । वि॒राडिति॑ वि - राट् । अन्न᳚म् । वि॒राडिति॑ वि - राट् । वि॒राजीति॑ वि - राजि॑ । ए॒व । अ॒न्नाद्य॒ इत्य॑न्न - अद्ये᳚ । प्रतीति॑ । ति॒ष्ठ॒ति॒ । ऋ॒तुभि॒रित्यृ॒तु - भिः॒ । वै । ए॒षः । छन्दो॑भि॒रिति॒ छन्दः॑- भिः॒ । स्तोमैः᳚ । पृ॒ष्ठैः । चे॒त॒व्यः॑ । इति॑ । आ॒हुः॒ । यत् । ए॒तानि॑ । ह॒वीꣳषि॑ । नि॒र्वप॒तीति॑ निः - वप॑ति । ऋ॒तुभि॒रित्यृ॒तु - भिः॒ । ए॒व । ए॒न॒म् । छन्दो॑भि॒रिति॒ छन्दः॑ - भिः॒ । स्तोमैः᳚ । पृ॒ष्ठैः । चि॒नु॒ते॒ । दिशः॑ । सु॒षु॒वा॒णेन॑ ।  \newline




\markright{ TS 7.5.15.3  \hfill https://www.vedavms.in \hfill}

\section{ TS 7.5.15.3 }

\textbf{TS 7.5.15.3 } \newline
\textbf{Samhita Paata} \newline

-भि॒जित्या॒ इत्या॑हु॒र्यदे॒तानि॑ ह॒वीꣳषि॑ नि॒र्वप॑ति दि॒शाम॒भिजि॑त्या ए॒तया॒ वा इन्द्रं॑ दे॒वा अ॑याजय॒न् तस्मा॑दिन्द्रस॒व ए॒तया॒ मनुं॑ मनु॒ष्या᳚स्तस्मा᳚न्-मनुस॒वो यथेन्द्रो॑ दे॒वानां॒ ॅयथा॒ मनु॑र्मनु॒ष्या॑णामे॒वं भ॑वति॒ य ए॒वं ॅवि॒द्वाने॒तयेष्ट्या॒ यज॑ते॒ दिग्व॑तीः पुरोऽनुवा॒क्या॑ भवन्ति॒ सर्वा॑सां दि॒शाम॒भिजि॑त्यै ॥ \newline

\textbf{Pada Paata} \newline

अ॒भि॒जित्या॒ इत्य॑भि - जित्याः᳚ । इति॑ । आ॒हुः॒ । यत् । ए॒तानि॑ । ह॒वीꣳषि॑ । नि॒र्वप॒तीति॑ निः - वप॑ति । दि॒शाम् । अ॒भिजि॑त्या॒ इत्य॒भि - जि॒त्यै॒ । ए॒तया᳚ । वै । इन्द्र᳚म् । दे॒वाः । अ॒या॒ज॒य॒न्न् । तस्मा᳚त् । इ॒न्द्र॒स॒व इती᳚न्द्र - स॒वः । ए॒तया᳚ । मनु᳚म् । म॒नु॒ष्याः᳚ । तस्मा᳚त् । म॒नु॒स॒व इति॑ मनु - स॒वः । यथा᳚ । इन्द्रः॑ । दे॒वाना᳚म् । यथा᳚ । मनुः॑ । म॒नु॒ष्या॑णाम् । ए॒वम् । भ॒व॒ति॒ । यः । ए॒वम् । वि॒द्वान् । ए॒तया᳚ । इष्ट्या᳚ । यज॑ते । दिग्व॑ती॒रिति॒ दिक्-व॒तीः॒ । पु॒रो॒ऽनु॒वा॒क्या॑ इति॑ पुरः-अ॒नु॒वा॒क्याः᳚ । भ॒व॒न्ति॒ । सर्वा॑साम् । दि॒शाम् । अ॒भिजि॑त्या॒ इत्य॒भि - जि॒त्यै॒ ॥  \newline




\markright{ TS 7.5.16.1  \hfill https://www.vedavms.in \hfill}

\section{ TS 7.5.16.1 }

\textbf{TS 7.5.16.1 } \newline
\textbf{Samhita Paata} \newline

यः प्रा॑ण॒तो नि॑मिष॒तो म॑हि॒त्वैक॒ इद्राजा॒ जग॑तो ब॒भूव॑ । य ईशे॑ अ॒स्य द्वि॒पद॒श्चतु॑ष्पदः॒ कस्मै॑ दे॒वाय॑ ह॒विषा॑ विधेम ॥उ॒प॒या॒मगृ॑हीतोऽसि प्र॒जाप॑तये त्वा॒ जुष्टं॑ गृह्णामि॒ तस्य॑ ते॒ द्यौर्म॑हि॒मा नक्ष॑त्राणि रू॒पमा॑दि॒त्यस्ते॒ तेज॒स्तस्मै᳚ त्वा महि॒म्ने प्र॒जाप॑तये॒ स्वाहा᳚ ॥ \newline

\textbf{Pada Paata} \newline

यः । प्रा॒ण॒त इति॑ प्र - अ॒न॒तः । नि॒मि॒ष॒त इति॑ नि - मि॒ष॒तः । म॒हि॒त्वेति॑ महि - त्वा । एकः॑ । इत् । राजा᳚ । जग॑तः । ब॒भूव॑ ॥ यः । ईशे᳚ । अ॒स्य । द्वि॒पद॒ इति॑ द्वि - पदः॑ । चतु॑ष्पद॒ इति॒ चतुः॑ - प॒दः॒ । कस्मै᳚ । दे॒वाय॑ । ह॒विषा᳚ । वि॒धे॒म॒ ॥ उ॒प॒या॒मगृ॑हीत॒ इत्यु॑पया॒म-गृ॒ही॒तः॒ । अ॒सि॒ । प्र॒जाप॑तय॒ इति॑ प्र॒जा - प॒त॒ये॒ । त्वा॒ । जुष्ट᳚म् । गृ॒ह्णा॒मि॒ । तस्य॑ । ते॒ । द्यौः । म॒हि॒मा । नक्ष॑त्राणि । रू॒पम् । आ॒दि॒त्यः । ते॒ । तेजः॑ । तस्मै᳚ । त्वा॒ । म॒हि॒म्ने । प्र॒जाप॑तय॒ इति॑ प्र॒जा - प॒त॒ये॒ । स्वाहा᳚ ॥  \newline




\markright{ TS 7.5.17.1  \hfill https://www.vedavms.in \hfill}

\section{ TS 7.5.17.1 }

\textbf{TS 7.5.17.1 } \newline
\textbf{Samhita Paata} \newline

य आ᳚त्म॒दा ब॑ल॒दा यस्य॒ विश्व॑ उ॒पास॑ते प्र॒शिषं॒ ॅयस्य॑ दे॒वाः । यस्य॑ छा॒याऽमृतं॒ ॅयस्य॑ मृ॒त्युः कस्मै॑ दे॒वाय॑ ह॒विषा॑ विधेम ॥उ॒प॒या॒मगृ॑हीतोऽसि प्र॒जाप॑तये त्वा॒ जुष्टं॑ गृह्णामि॒ तस्य॑ ते पृथि॒वी म॑हि॒मौष॑धयो॒ वन॒स्पत॑यो रू॒पम॒ग्निस्ते॒ तेज॒स्तस्मै᳚ त्वा महि॒म्ने प्र॒जाप॑तये॒ स्वाहा᳚ ॥ \newline

\textbf{Pada Paata} \newline

यः । आ॒त्म॒दा इत्या᳚त्म - दाः । ब॒ल॒दा इति॑ बल - दाः । यस्य॑ । विश्वे᳚ । उ॒पास॑त॒ इत्यु॑प - आस॑ते । प्र॒शिष॒मिति॑ प्र-शिष᳚म् । यस्य॑ । दे॒वाः ॥ यस्य॑ । छा॒या । अ॒मृत᳚म् । यस्य॑ । मृ॒त्युः । कस्मै᳚ । दे॒वाय॑ । ह॒विषा᳚ । वि॒धे॒म॒ ॥ उ॒प॒या॒मगृ॑हीत॒ इत्यु॑पया॒म - गृ॒ही॒तः॒ । अ॒सि॒ । प्र॒जाप॑तय॒ इति॑ प्र॒जा - प॒त॒ये॒ । त्वा॒ । जुष्ट᳚म् । गृ॒ह्णा॒मि॒ । तस्य॑ । ते॒ । पृ॒थि॒वी । म॒हि॒मा । ओष॑धयः । वन॒स्पत॑यः । रू॒पम् । अ॒ग्निः । ते॒ । तेजः॑ । तस्मै᳚ । त्वा॒ । म॒हि॒म्ने । प्र॒जाप॑तय॒ इति॑ प्र॒जा - प॒त॒ये॒ । स्वाहा᳚ ॥  \newline




\markright{ TS 7.5.18.1  \hfill https://www.vedavms.in \hfill}

\section{ TS 7.5.18.1 }

\textbf{TS 7.5.18.1 } \newline
\textbf{Samhita Paata} \newline

आ ब्रह्म॑न् ब्राह्म॒णो ब्र॑ह्मवर्च॒सी जा॑यता॒मा ऽस्मिन् रा॒ष्ट्रे रा॑ज॒न्य॑ इष॒व्यः॑ शूरो॑ महार॒थो जा॑यतां॒ दोग्ध्री॑धे॒नुर्वोढा॑ ऽन॒ड्वाना॒शुः सप्तिः॒ पुर॑न्धि॒र्योषा॑ जि॒ष्णू र॑थे॒ष्ठाः स॒भेयो॒ युवा ऽऽस्य यज॑मानस्य वी॒रो जा॑यतां निका॒मेनि॑कामे नः प॒र्जन्यो॑ वर्.षतु फ॒लिन्यो॑ न॒ ओष॑धयः पच्यन्तां ॅयोगक्षे॒मोनः॑ कल्पतां ॥ \newline

\textbf{Pada Paata} \newline

एति॑ । ब्रह्मन्न्॑ । ब्रा॒ह्म॒णः । ब्र॒ह्म॒व॒र्च॒सीति॑ ब्रह्म-व॒र्च॒सी । जा॒य॒ता॒म् । एति॑ । अ॒स्मिन्न् । रा॒ष्ट्रे । रा॒ज॒न्यः॑ । इ॒ष॒व्यः॑ । शूरः॑ । म॒हा॒र॒थ इति॑ महा - र॒थः । जा॒य॒ता॒म् । दोग्ध्री᳚ । धे॒नुः । वोढा᳚ । अ॒न॒ड्वान् । आ॒शुः । सप्तिः॑ । पुर॑न्धिः । योषा᳚ । जि॒ष्णूः । र॒थे॒ष्ठा इति॑ रथे - स्थाः । स॒भेयः॑ । युवा᳚ । एति॑ । अ॒स्य । यज॑मानस्य । वी॒रः । जा॒य॒ता॒म् । नि॒का॒मेनि॑काम॒ इति॑ निका॒मे - नि॒का॒मे॒ । नः॒ । प॒र्जन्यः॑ । व॒र्.॒ष॒तु॒ । फ॒लिन्यः॑ । नः॒ । ओष॑धयः । प॒च्य॒न्ता॒म् । यो॒ग॒क्षे॒म इति॑ योग - क्षे॒मः । नः॒ । क॒ल्प॒ता॒म् ॥  \newline




\markright{ TS 7.5.19.1  \hfill https://www.vedavms.in \hfill}

\section{ TS 7.5.19.1 }

\textbf{TS 7.5.19.1 } \newline
\textbf{Samhita Paata} \newline

आऽक्रान्॑ वा॒जी पृ॑थि॒वीम॒ग्निं ॅयुज॑मकृत वा॒ज्यर्वा ऽऽक्रान्॑ वा॒ज्य॑न्तरि॑क्षं ॅवा॒युं ॅयुज॑मकृत वा॒ज्यर्वा॒ द्यां ॅवा॒ज्याऽक्रꣳ॑स्त॒ सूर्यं॒ ॅयुज॑मकृत वा॒ज्यर्वा॒ ऽग्निस्ते॑ वाजि॒न्॒ युङ्ङनु॒ त्वा ऽऽ र॑भे स्व॒स्ति मा॒ सं पा॑रय वा॒युस्ते॑ वाजि॒न्॒ युङ्ङनु॒ त्वा ऽऽ र॑भे स्व॒स्ति मा॒ सं - [  ] \newline

\textbf{Pada Paata} \newline

एति॑ । अ॒क्रा॒न् । वा॒जी । पृ॒थि॒वीम् । अ॒ग्निम् । युज᳚म् । अ॒कृ॒त॒ । वा॒जी । अर्वा᳚ । एति॑ । अ॒क्रा॒न् । वा॒जी । अ॒न्तरि॑क्षम् । वा॒युम् । युज᳚म् । अ॒कृ॒त॒ । वा॒जी । अर्वा᳚ । द्याम् । वा॒जी । एति॑ । अ॒क्रꣳ॒॒स्त॒ । सूर्य᳚म् । युज᳚म् । अ॒कृ॒त॒ । वा॒जी । अर्वा᳚ । अ॒ग्निः । ते॒ । वा॒जि॒न्न् । युङ् । अन्विति॑ । त्वा॒ । एति॑ । र॒भे॒ । स्व॒स्ति । मा॒ । समिति॑ । पा॒र॒य॒ । वा॒युः । ते॒ । वा॒जि॒न्न् । युङ् । अन्विति॑ । त्वा॒ । एति॑ । र॒भे॒ । स्व॒स्ति । मा॒ । समिति॑ ।  \newline




\markright{ TS 7.5.19.2  \hfill https://www.vedavms.in \hfill}

\section{ TS 7.5.19.2 }

\textbf{TS 7.5.19.2 } \newline
\textbf{Samhita Paata} \newline

पा॑रया ऽऽदि॒त्यस्ते॑ वाजि॒न्॒ युङ्ङनु॒ त्वा ऽऽ र॑भे स्व॒स्ति मा॒ सं पा॑रय प्राण॒धृग॑सि प्रा॒णं मे॑ दृꣳह व्यान॒धृग॑सि व्या॒नं मे॑ दृꣳहा ऽपान॒धृग॑स्यपा॒नं म॑ दृꣳह॒ चक्षु॑रसि॒ चक्षु॒र्मयि॑ धेहि॒ श्रोत्र॑मसि॒ श्रोत्रं॒ मयि॑ धे॒ह्यायु॑र॒स्यायु॒र्मयि॑ धेहि ॥ \newline

\textbf{Pada Paata} \newline

पा॒र॒य॒ । आ॒दि॒त्यः । ते॒ । वा॒जि॒न्न् । युङ् । अन्विति॑ । त्वा॒ । एति॑ । र॒भे॒ । स्व॒स्ति । मा॒ । समिति॑ । पा॒र॒य॒ । प्रा॒ण॒धृगिति॑ प्राण - धृक् । अ॒सि॒ । प्रा॒णमिति॑ प्र - अ॒नम् । मे॒ । दृꣳ॒॒ह॒ । व्या॒न॒धृगिति॑ व्यान - धृक् । अ॒सि॒ । व्या॒नमिति॑ वि - अ॒नम् । मे॒ । दृꣳ॒॒ह॒ । अ॒पा॒न॒धृगित्य॑पान - धृक् । अ॒सि॒ । अ॒पा॒नमित्य॑प - अ॒नम् । मे॒ । दृꣳ॒॒ह॒ । चक्षुः॑ । अ॒सि॒ । चक्षुः॑ । मयि॑ । धे॒हि॒ । श्रोत्र᳚म् । अ॒सि॒ । श्रोत्र᳚म् । मयि॑ । धे॒हि॒ । आयुः॑ । अ॒सि॒ । आयुः॑ । मयि॑ । धे॒हि॒ ॥  \newline




\markright{ TS 7.5.20.1  \hfill https://www.vedavms.in \hfill}

\section{ TS 7.5.20.1 }

\textbf{TS 7.5.20.1 } \newline
\textbf{Samhita Paata} \newline

जज्ञि॒ बीजं॒ ॅवर्ष्टा॑ प॒र्जन्यः॒ पक्ता॑ स॒स्यꣳ सु॑पिप्प॒ला ओष॑धयः स्वधिचर॒णेयꣳ सू॑पसद॒नो᳚ऽग्निः स्व॑द्ध्य॒क्षम॒न्तरि॑क्षꣳसुपा॒वः पव॑मानः सूपस्था॒ना द्यौः शि॒वम॒सौ तप॑न् यथापू॒र्वम॑होरा॒त्रे प॑ञ्चद॒शिनो᳚ ऽर्द्धमा॒सा-स्त्रिꣳ॒॒शिनो॒ मासाः᳚ क्लृ॒प्ता ऋ॒तवः॑ शा॒न्तः सं॑ॅवथ्स॒रः ॥ \newline

\textbf{Pada Paata} \newline

जज्ञि॑ । बीज᳚म् । वर्ष्टा᳚ । प॒र्जन्यः॑ । पक्ता᳚ । स॒स्यम् । सु॒पि॒प्प॒ला इति॑ सु - पि॒प्प॒लाः । ओष॑धयः । स्व॒धि॒च॒र॒णेति॑ सु - अ॒धि॒च॒र॒णा । इ॒यम् । सू॒प॒स॒द॒न इति॑ सु - उ॒प॒स॒द॒नः । अ॒ग्निः । स्व॒द्ध्य॒क्षमिति॑ सु - अ॒द्ध्य॒क्षम् । अ॒न्तरि॑क्षम् । सु॒पा॒व इति॑ सु-पा॒वः । पव॑मानः । सू॒प॒स्था॒नेति॑ सु - उ॒प॒स्था॒ना । द्यौः । शि॒वम् । अ॒सौ । तपन्न्॑ । य॒था॒पू॒र्वमिति॑ यथा - पू॒र्वम् । अ॒हो॒रा॒त्रे इत्य॑हः - रा॒त्रे । प॒ञ्च॒द॒शिन॒ इति॑ पञ्च - द॒शिनः॑ । अ॒द्‌र्ध॒मा॒सा इत्य॑द्‌र्ध - मा॒साः । त्रिꣳ॒॒शिनः॑ । मासाः᳚ । क्लृ॒प्ताः । ऋ॒तवः॑ । शा॒न्तः । सं॒ॅव॒थ्स॒र इति॑ सं - व॒थ्स॒रः ॥  \newline




\markright{ TS 7.5.21.1  \hfill https://www.vedavms.in \hfill}

\section{ TS 7.5.21.1 }

\textbf{TS 7.5.21.1 } \newline
\textbf{Samhita Paata} \newline

आ॒ग्ने॒यो᳚ऽष्टाक॑पालः सौ॒म्यश्च॒रुः सा॑वि॒त्रो᳚ऽष्टाक॑पालः पौ॒ष्णश्च॒रू रौ॒द्रश्च॒रुर॒ग्नये॑ वैश्वान॒राय॒ द्वाद॑शकपालो मृगाख॒रे यदि॒ नाऽऽ*गच्छे॑-द॒ग्नये-ऽꣳ॑हो॒मुचे॒-ऽष्टाक॑पालः सौ॒र्यं पयो॑ वाय॒व्य॑ आज्य॑भागः ॥ \newline

\textbf{Pada Paata} \newline

आ॒ग्ने॒यः । अ॒ष्टाक॑पाल॒ इत्य॒ष्टा - क॒पा॒लः॒ । सौ॒म्यः । च॒रुः । सा॒वि॒त्रः । अ॒ष्टाक॑पाल॒ इत्य॒ष्टा - क॒पा॒लः॒ । पौ॒ष्णः । च॒रुः । रौ॒द्रः । च॒रुः । अ॒ग्नये᳚ । वै॒श्वा॒न॒राय॑ । द्वाद॑शकपाल॒ इति॒ द्वाद॑श - क॒पा॒लः॒ । मृ॒गा॒ख॒र इति॑ मृग - आ॒ख॒रे । यदि॑ । न । आ॒गच्छे॒दित्या᳚-गच्छे᳚त् । अ॒ग्नये᳚ । अꣳ॒॒हो॒मुच॒ इत्यꣳ॑हः-मुचे᳚ । अ॒ष्टाक॑पाल॒ इत्य॒ष्टा-क॒पा॒लः॒ । सौ॒र्यम् । पयः॑ । वा॒य॒व्यः॑ । आज्य॑भाग॒ इत्याज्य॑ - भा॒गः॒ ॥  \newline




\markright{ TS 7.5.22.1  \hfill https://www.vedavms.in \hfill}

\section{ TS 7.5.22.1 }

\textbf{TS 7.5.22.1 } \newline
\textbf{Samhita Paata} \newline

अ॒ग्नये-ऽꣳ॑हो॒मुचे॒-ऽष्टाक॑पाल॒ इन्द्रा॑याऽꣳहो॒मुच॒ एका॑दशकपालो मि॒त्रावरु॑णाभ्या-मागो॒मुग्भ्यां᳚ पय॒स्या॑ वायोसावि॒त्र आ॑गो॒मुग्भ्यां᳚ च॒रुर॒श्विभ्या॑-मागो॒मुग्भ्यां᳚ धा॒ना म॒रुद्भ्य॑ एनो॒मुग्भ्यः॑ स॒प्तक॑पालो॒ विश्वे᳚भ्यो दे॒वेभ्य॑ एनो॒मुग्भ्यो॒ द्वाद॑शकपा॒लो ऽनु॑मत्यै च॒रुर॒ग्नये॑ वैश्वान॒राय॒ द्वाद॑शकपालो॒ द्यावा॑पृथि॒वीभ्या॑-मꣳहो॒मुग्भ्यां᳚ द्विकपा॒लः ॥ \newline

\textbf{Pada Paata} \newline

अ॒ग्नये᳚ । अꣳ॒॒हो॒मुच॒ इत्यꣳ॑हः-मुचे᳚ । अष्टाक॑पाल॒ इत्य॒ष्टा-क॒पा॒लः॒ । इन्द्रा॑य । अꣳ॒॒हो॒मुच॒ इत्यꣳ॑हः - मुचे᳚ । एका॑दशकपाल॒ इत्येका॑दश - क॒पा॒लः॒ । मि॒त्रावरु॑णाभ्या॒मिति॑ मि॒त्रा - वरु॑णाभ्याम् । आ॒गो॒मुग्भ्या॒मित्या॑गो॒मुक् - भ्या॒म् । प॒य॒स्या᳚ । वा॒यो॒सा॒वि॒त्र इति॑ वायो - सा॒वि॒त्रः । आ॒गो॒मुग्भ्या॒मित्या॑गो॒मुक् - भ्या॒म् । च॒रुः । अ॒श्विभ्या॒मित्य॒श्वि - भ्या॒म् । अ॒गो॒मुग्भ्या॒मित्या॑गो॒मुक् - भ्या॒म् । धा॒नाः । म॒रुद्भ्य॒ इति॑ म॒रुत् - भ्यः॒ । ए॒नो॒मुग्भ्य॒ इत्ये॑नो॒मुक्-भ्यः॒ । स॒प्तक॑पाल॒ इति॑ स॒प्त - क॒पा॒लः॒ । विश्वे᳚भ्यः । दे॒वेभ्यः॑ । ए॒नो॒मुग्भ्य॒ इत्ये॑नो॒मुक् - भ्यः॒ । द्वाद॑शकपाल॒ इति॒ द्वाद॑श - क॒पा॒लः॒ । अनु॑मत्या॒ इत्यनु॑ - म॒त्यै॒ । च॒रुः । अ॒ग्नये᳚ । वै॒श्वा॒न॒राय॑ । द्वाद॑शकपाल॒ इति॒ द्वाद॑श - क॒पा॒लः॒ । द्यावा॑पृथि॒वीभ्या॒मिति॒ द्यावा᳚ - पृ॒थि॒वीभ्या᳚म् । अꣳ॒॒हो॒मुग्भ्या॒मित्यꣳ॑हो॒मुक् - भ्या॒म् । द्वि॒क॒पा॒ल इति॑ द्वि - क॒पा॒लः ॥  \newline




\markright{ TS 7.5.23.1  \hfill https://www.vedavms.in \hfill}

\section{ TS 7.5.23.1 }

\textbf{TS 7.5.23.1 } \newline
\textbf{Samhita Paata} \newline

अ॒ग्नये॒ सम॑नमत् पृथि॒व्यै सम॑नम॒द्यथा॒ऽग्निः पृ॑थि॒व्या स॒मन॑मदे॒वं मह्यं॑ भ॒द्राः संन॑तयः॒ सं न॑मन्तु वा॒यवे॒ सम॑नमद॒न्तरि॑क्षाय॒ सम॑नम॒द्यथा॑ वा॒युर॒न्तरि॑क्षेण॒ सूर्या॑य॒ सम॑नमद्दि॒वे सम॑नम॒द्यथा॒ सूर्यो॑ दि॒वा च॒न्द्रम॑से॒ सम॑नम॒न्नक्ष॑त्रेभ्यः॒ सम॑नम॒द्यथा॑ च॒न्द्रमा॒ नक्ष॑त्रै॒र्वरु॑णाय॒ सम॑नमद॒द्भ्यः सम॑नम॒द्यथा॒ - [  ] \newline

\textbf{Pada Paata} \newline

अ॒ग्नये᳚ । समिति॑ । अ॒न॒म॒त् । पृ॒थि॒व्यै । समिति॑ । अ॒न॒म॒त् । यथा᳚ । अ॒ग्निः । पृ॒थि॒व्या । स॒मन॑म॒दिति॑ सं - अन॑मत् । ए॒वम् । मह्य᳚म् । भ॒द्राः । संन॑तय॒ इति॒ सं - न॒त॒यः॒ । समिति॑ । न॒म॒न्तु॒ । वा॒यवे᳚ । समिति॑ । अ॒न॒म॒त् । अ॒न्तरि॑क्षाय । समिति॑ । अ॒न॒म॒त् । यथा᳚ । वा॒युः । अ॒न्तरि॑क्षेण । सूर्या॑य । समिति॑ । अ॒न॒म॒त् । दि॒वे । समिति॑ । अ॒न॒म॒त् । यथा᳚ । सूर्यः॑ । दि॒वा । च॒न्द्रम॑से । समिति॑ । अ॒न॒म॒त् । नक्ष॑त्रेभ्यः । समिति॑ । अ॒न॒म॒त् । यथा᳚ । च॒न्द्रमाः᳚ । नक्ष॑त्रैः । वरु॑णाय । समिति॑ । अ॒न॒म॒त् । अ॒द्भ्य इत्य॑त् - भ्यः । समिति॑ । अ॒न॒म॒त्॒ । यथा᳚ ।  \newline




\markright{ TS 7.5.23.2  \hfill https://www.vedavms.in \hfill}

\section{ TS 7.5.23.2 }

\textbf{TS 7.5.23.2 } \newline
\textbf{Samhita Paata} \newline

वरु॑णो॒ऽद्भिः साम्ने॒ सम॑नमदृ॒चे सम॑नम॒द्यथा॒ साम॒र्चा ब्रह्म॑णे॒ सम॑नमत् क्ष॒त्राय॒ सम॑नम॒द्यथा॒ ब्रह्म॑ क्ष॒त्रेण॒ राज्ञे॒ सम॑नमद्-वि॒शे सम॑नम॒द्यथा॒ राजा॑ वि॒शा रथा॑यः॒ सम॑नम॒दश्वे᳚भ्यः॒ सम॑नम॒द्यथा॒ रथोऽश्वैः᳚ प्र॒जाप॑तये॒ सम॑नमद्-भू॒तेभ्यः॒ सम॑नम॒द्यथा᳚ प्र॒जाप॑तिर्भू॒तैः स॒मन॑मदे॒वं मह्यं॑ ( ) भ॒द्राः संन॑तयः॒ सं न॑मन्तु ॥ \newline

\textbf{Pada Paata} \newline

वरु॑णः । अ॒द्भिरित्य॑त् - भिः । साम्ने᳚ । समिति॑ । अ॒न॒म॒त् । ऋ॒चे । समिति॑ । अ॒न॒म॒त् । यथा᳚ । साम॑ । ऋ॒चा । ब्रह्म॑णे । समिति॑ । अ॒न॒म॒त् । क्ष॒त्राय॑ । समिति॑ । अ॒न॒म॒त् । यथा᳚ । ब्रह्म॑ । क्ष॒त्रेण॑ । राज्ञे᳚ । समिति॑ । अ॒न॒म॒त् । वि॒शे । समिति॑ । अ॒न॒म॒त् । यथा᳚ । राजा᳚ । वि॒शा । रथा॑य । समिति॑ । अ॒न॒म॒त् । अश्वे᳚भ्यः । समिति॑ । अ॒न॒म॒त् । यथा᳚ । रथः॑ । अश्वैः᳚ । प्र॒जाप॑तय॒ इति॑ प्र॒जा - प॒त॒ये॒ । समिति॑ । अ॒न॒म॒त् । भू॒तेभ्यः॑ । समिति॑ । अ॒न॒म॒त् । यथा᳚ । प्र॒जाप॑ति॒रिति॑ प्र॒जा - प॒तिः॒ । भू॒तैः । स॒मन॑म॒दिति॑ सं-अन॑मत् । ए॒वम् । मह्य᳚म् ( ) । भ॒द्राः । संन॑तय॒ इति॒ सं - न॒त॒यः॒ । समिति॑ । न॒म॒न्तु॒ ॥  \newline




\markright{ TS 7.5.24.1  \hfill https://www.vedavms.in \hfill}

\section{ TS 7.5.24.1 }

\textbf{TS 7.5.24.1 } \newline
\textbf{Samhita Paata} \newline

ये ते॒ पन्था॑नः सवितः पू॒र्व्यासो॑ऽरे॒णवो॒ वित॑ता अ॒न्तरि॑क्षे । तेभि॑र्नो अ॒द्य प॒थिभिः॑ सु॒गेभी॒ रक्षा॑ च नो॒ अधि॑ च देव ब्रूहि ॥नमो॒ऽग्नये॑ पृथिवि॒क्षिते॑ लोक॒स्पृते॑ लो॒कम॒स्मै यज॑मानाय देहि॒ नमो॑ वा॒यवे᳚ऽन्तरिक्ष॒क्षिते॑ लोक॒स्पृते॑ लो॒कम॒स्मै यज॑मानाय देहि॒ नमः॒ सूर्या॑य दिवि॒क्षिते॑ लोक॒स्पृते॑ लो॒कम॒स्मै यज॑मानाय देहि ॥ \newline

\textbf{Pada Paata} \newline

ये । त॒ । पन्था॑नः । स॒वि॒तः॒ । पू॒र्व्यासः॑ । अ॒रे॒णवः॑ । वित॑ता॒ इति॒ वि - त॒ताः॒ । अ॒न्तरि॑क्षे ॥ तेभिः॑ । नः॒ । अ॒द्य । प॒थिभि॒रिति॑ प॒थि - भिः॒ । सु॒गेभि॒रिति॑ सु - गेभिः॑ । रक्ष॑ । च॒ । नः॒ । अधीति॑ । च॒ । दे॒व॒ । ब्रू॒हि॒ ॥ नमः॑ । अ॒ग्नये᳚ । पृ॒थि॒वि॒क्षित॒ इति॑ पृथिवि-क्षिते᳚ । लो॒क॒स्पृत॒ इति॑ लोक - स्पृते᳚ । लो॒कम् । अ॒स्मै । यज॑मानाय । दे॒हि॒ । नमः॑ । वा॒यवे᳚ । अ॒न्त॒रि॒क्ष॒क्षित॒ इत्य॑न्तरिक्ष - क्षिते᳚ । लो॒क॒स्पृत॒ इति॑ लोक - स्पृते᳚ । लो॒कम् । अ॒स्मै । यज॑मानाय । दे॒हि॒ । नमः॑ । सूर्या॑य । दि॒वि॒क्षित॒ इति॑ दिवि - क्षिते᳚ । लो॒क॒स्पृत॒ इति॑ लोक - स्पृते᳚ । लो॒कम् । अ॒स्मै । यज॑मानाय । दे॒हि॒ ॥  \newline




\markright{ TS 7.5.25.1  \hfill https://www.vedavms.in \hfill}

\section{ TS 7.5.25.1 }

\textbf{TS 7.5.25.1 } \newline
\textbf{Samhita Paata} \newline

यो वा अश्व॑स्य॒ मेद्ध्य॑स्य॒ शिरो॒ वेद॑ शीर्.ष॒ण्वान् मेद्ध्यो॑ भवत्यु॒षा वा अश्व॑स्य॒ मेद्ध्य॑स्य॒ शिरः॒ सूर्य॒श्चक्षु॒र्वातः॑ प्रा॒णश्च॒न्द्रमाः॒ श्रोत्रं॒ दिशः॒ पादा॑ अवान्तरदि॒शाः पर्.श॑वोऽहोरा॒त्रे नि॑मे॒षो᳚ऽर्द्धमा॒साः पर्वा॑णि॒ मासाः᳚ स॒धांना᳚न्यृ॒तवोऽङ्गा॑नि संॅवथ्स॒र आ॒त्मा र॒श्मयः॒ केशा॒ नक्ष॑त्राणि रू॒पं तार॑का अ॒स्थानि॒ नभो॑ माꣳ॒॒सान्योष॑धयो॒ लोमा॑नि॒ वन॒स्पत॑यो॒ वाला॑ अ॒ग्निर्मुखं॑ ॅवैश्वान॒रो व्यात्तꣳ॑ - [  ] \newline

\textbf{Pada Paata} \newline

यः । वै । अश्व॑स्य । मेद्ध्य॑स्य । शिरः॑ । वेद॑ । शी॒र्.॒ष॒ण्वानिति॑ शीर्.षण्-वान् । मेद्ध्यः॑ । भ॒व॒ति॒ । उ॒षाः । वै । अश्व॑स्य । मेद्ध्य॑स्य । शिरः॑ । सूर्यः॑ । चक्षुः॑ । वातः॑ । प्रा॒ण इति॑ प्र - अ॒नः । च॒न्द्रमाः᳚ । श्रोत्र᳚म् । दिशः॑ । पादाः᳚ । अ॒वा॒न्त॒र॒दि॒शा इत्य॑वान्तर - दि॒शाः । पर्.श॑वः । अ॒हो॒रा॒त्रे इत्य॑हः - रा॒त्रे । नि॒मे॒ष इति॑ नि - मे॒षः । अ॒द्‌र्ध॒मा॒सा इत्य॑द्‌र्ध - मा॒साः । पर्वा॑णि । मासाः᳚ । स॒धांना॒नीति॑ सं - धाना॑नि । ऋ॒तवः॑ । अङ्गा॑नि । सं॒ॅव॒थ्स॒र इति॑ सं - व॒थ्स॒रः । आ॒त्मा । र॒श्मयः॑ । केशाः᳚ । नक्ष॑त्राणि । रू॒पम् । तार॑काः । अ॒स्थानि॑ । नभः॑ । माꣳ॒॒सानि॑ । ओष॑धयः । लोमा॑नि । वन॒स्पत॑यः । वालाः᳚ । अ॒ग्निः । मुख᳚म् । वै॒श्वा॒न॒रः । व्यात्त॒मिति॑ वि - आत्त᳚म् ।  \newline




\markright{ TS 7.5.25.2  \hfill https://www.vedavms.in \hfill}

\section{ TS 7.5.25.2 }

\textbf{TS 7.5.25.2 } \newline
\textbf{Samhita Paata} \newline

समु॒द्र उ॒दर॑म॒न्तरि॑क्षं पा॒यु-र्द्यावा॑पृथि॒वी आ॒ण्डौ ग्रावा॒ शेपः॒ सोमो॒ रेतो॒ यज्ज॑ञ्ज॒भ्यते॒ तद्वि द्यो॑तते॒ यद्वि॑धूनु॒ते तथ् स्त॑नयति॒ यन्मेह॑ति॒ तद्व॑र्.षति॒ वागे॒वास्य॒ वागह॒र्वा अश्व॑स्य॒ जाय॑मानस्य महि॒मा पु॒रस्ता᳚ज्जायते॒ रात्रि॑रेनं महि॒मा प॒श्चादनु॑ जायत ए॒तौ वै म॑हि॒माना॒- वश्व॑म॒भितः॒ सं ब॑भूवतु॒र्॒.हयो॑ दे॒वान॑वह॒ () दर्वाऽसु॑रान् वा॒जी ग॑न्ध॒र्वानश्वो॑मनु॒ष्या᳚न्थ् समु॒द्रो वा अश्व॑स्य॒ योनिः॑ समु॒द्रो बन्धुः॑ ॥ \newline

\textbf{Pada Paata} \newline

स॒मु॒द्रः । उ॒दर᳚म् । अ॒न्तरि॑क्षम् । पा॒युः । द्यावा॑पृथि॒वी इति॒ द्यावा᳚ - पृ॒थि॒वी । आ॒ण्डौ । ग्रावा᳚ । शेपः॑ । सोमः॑ । रेतः॑ । यत् । ज॒ञ्ज॒भ्यते᳚ । तत् । वीति॑ । द्यो॒त॒ते॒ । यत् । वि॒धू॒नु॒त इति॑ वि-धू॒नु॒ते । तत् । स्त॒न॒य॒ति॒ । यत् । मेह॑ति । तत् । व॒र्.॒ष॒ति॒ । वाक् । ए॒व । अ॒स्य॒ । वाक् । अहः॑ । वै । अश्व॑स्य । जाय॑मानस्य । म॒हि॒मा । पु॒रस्ता᳚त् । जा॒य॒ते॒ । रात्रिः॑ । ए॒न॒म् । म॒हि॒मा । प॒श्चात् । अन्विति॑ । जा॒य॒ते॒ । ए॒तौ । वै । म॒हि॒मानौ᳚ । अश्व᳚म् । अ॒भितः॑ । समिति॑ । ब॒भू॒व॒तुः॒ । हयः॑ । दे॒वान् । अ॒व॒ह॒त् ( ) । अर्वा᳚ । असु॑रान् । वा॒जी । ग॒न्ध॒र्वान् । अश्वः॑ । म॒नु॒ष्यान्॑ । स॒मु॒द्रः । वै । अश्व॑स्य । योनिः॑ । स॒मु॒द्रः । बन्धुः॑ ॥  \newline






\end{document}