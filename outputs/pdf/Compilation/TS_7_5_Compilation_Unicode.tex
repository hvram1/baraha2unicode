\documentclass[17pt]{extarticle}
\usepackage{babel}
\usepackage{fontspec}
\usepackage{polyglossia}
\usepackage{extsizes}

\usepackage{color}   %May be necessary if you want to color links
\usepackage{hyperref}
\hypersetup{
    colorlinks=true, %set true if you want colored links
    linktoc=all,     %set to all if you want both sections and subsections linked
    linkcolor=black,  %choose some color if you want links to stand out
}

\setmainlanguage{sanskrit}
\setotherlanguages{english} %% or other languages
\setlength{\parindent}{0pt}
\pagestyle{myheadings}
\newfontfamily\devanagarifont[Script=Devanagari]{AdishilaVedic}
\renewcommand{\theHsection}{\thepart.section.\thesection}

\newcommand{\VAR}[1]{}
\newcommand{\BLOCK}[1]{}




\begin{document}
\begin{titlepage}
    \begin{center}
 
\begin{sanskrit}
    { \Large
    कृष्ण यजुर्वेदीय तैत्तिरीय संहिता,पद,जटा,घन पाठः 
    }
    \\
    \vspace{2.5cm}
    \mbox{ \Large
    7.5      सप्तमकाण्डे पञ्चमः प्रश्नः - सत्रविशेषाभिधानं   }
\end{sanskrit}
\end{center}

\end{titlepage}
\tableofcontents
\phantomsection
\pagebreak

\markright{ TS 7.5.1.1  \hfill https://www.vedavms.in \hfill}

\section{ TS 7.5.1.1 }

\textbf{TS 7.5.1.1 } \newline
\textbf{Samhita Paata} \newline

गावो॒ वा ए॒तथ् स॒त्र-मा॑सताशृ॒ङ्गाः स॒तीः शृङ्गा॑णि नो जायन्ता॒ इति॒ कामे॑न॒ तासां॒ दश॒मासा॒ निष॑ण्णा॒ आस॒न्नथ॒ शृङ्गा᳚ण्यजायन्त॒ ता उद॑तिष्ठ॒न्नरा॒थ्स्मेत्यथ॒ यासां॒ नाजा॑यन्त॒ ताः सं॑ॅवथ्स॒र-मा॒प्त्वोद॑तिष्ठ॒ -न्नरा॒थ्स्मेति॒ यासां॒ चाजा॑यन्त॒ यासां᳚ च॒ न ता उ॒भयी॒रु-द॑तिष्ठ॒-न्नरा॒थ्स्मेति॑ गोस॒त्रं ॅवै - [  ] \newline

\textbf{Pada Paata} \newline

गावः॑ । वै । ए॒तत् । स॒त्रम् । आ॒स॒त॒ । अ॒शृ॒ङ्गाः । स॒तीः । शृङ्गा॑णि । नः॒ । जा॒य॒न्तै॒ । इति॑ । कामे॑न । तासा᳚म् । दश॑ । मासाः᳚ । निष॑ण्णा॒ इति॒ नि - स॒न्नाः॒ । आसन्न्॑ । अथ॑ । शृङ्गा॑णि । अ॒जा॒य॒न्त॒ । ताः । उदिति॑ । अ॒ति॒ष्ठ॒न्न् । अरा᳚थ्स्म । इति॑ । अथ॑ । यासा᳚म् । न । अजा॑यन्त । ताः । सं॒ॅव॒थ्स॒रमिति॑ सं -  व॒थ्स॒रम् । आ॒प्त्वा । उदिति॑ । अ॒ति॒ष्ठ॒न्न् । अरा᳚थ्स्म । इति॑ । यासा᳚म् । च॒ । अजा॑यन्त । यासा᳚म् । च॒ । न । ताः । उ॒भयीः᳚ । उदिति॑ । अ॒ति॒ष्ठ॒न्न् । अरा᳚थ्स्म । इति॑ । गो॒स॒त्रमिति॑ गो - स॒त्रम् । वै ।  \newline


\textbf{Krama Paata} \newline

गावो॒ वै । वा ए॒तत् । ए॒तथ् स॒त्रम् । स॒त्रमा॑सत । आ॒स॒ता॒शृ॒ङ्‍गाः । अ॒शृ॒ङ्‍गाः स॒तीः । स॒तीः शृङ्‍गा॑णि । शृङ्‍गा॑णि नः । नो॒ जा॒य॒न्तै॒ । जा॒य॒न्ता॒ इति॑ । इति॒ कामे॑न । कामे॑न॒ तासा᳚म् । तासा॒म् दश॑ । दश॒ मासाः᳚ । मासा॒ निष॑ण्णाः । निष॑ण्णा॒ आसन्न्॑ । निष॑ण्णा॒ इति॒ नि - स॒न्नाः॒ । आस॒न्नथ॑ । अथ॒ शृङ्‍गा॑णि । शृङ्गा᳚ण्यजायन्त । अ॒जा॒य॒न्त॒ ताः । ता उत् । उद॑तिष्ठन्न् । अ॒ति॒ष्ठ॒न्नरा᳚थ्स्म । अरा॒थ्स्मेति॑ । इत्यथ॑ । अथ॒ यासा᳚म् । यासा॒म् न । नाजा॑यन्त । अजा॑यन्त॒ ताः । ताः स॑म्ॅवथ्स॒रम् । स॒म्ॅव॒थ्स॒रमा॒प्त्वा । स॒म्ॅव॒थ्स॒रमिति॑ सम् - व॒थ्स॒रम् । आ॒प्त्वोत् । उद॑तिष्ठन्न् । अ॒ति॒ष्ठ॒न्नरा᳚थ्स्म । अरा॒थ्स्मेति॑ । इति॒ यासा᳚म् । यासा᳚म् च । चाजा॑यन्त । अजा॑यन्त॒ यासा᳚म् । यासा᳚म् च । च॒ न । न ताः । ता उ॒भयीः᳚ । उ॒भयी॒रुत् । उद॑तिष्ठन्न् । अ॒ति॒ष्ठ॒न्नरा᳚थ्स्म । अरा॒थ्स्मेति॑ । इति॑ गोस॒त्रम् । गो॒स॒त्रम् ॅवै । गो॒स॒त्रमिति॑ गो - स॒त्रम् । वै स॑म्ॅवथ्स॒रः \newline

\textbf{Jatai Paata} \newline

1. गावो॒ वै वै गावो॒ गावो॒ वै । \newline
2. वा ए॒त दे॒तद् वै वा ए॒तत् । \newline
3. ए॒तथ् स॒त्रꣳ स॒त्र मे॒त दे॒तथ् स॒त्रम् । \newline
4. स॒त्र मा॑सता सत स॒त्रꣳ स॒त्र मा॑सत । \newline
5. आ॒स॒ता॒ शृ॒ङ्गा अ॑शृ॒ङ्गा आ॑सता सता शृ॒ङ्गाः । \newline
6. अ॒शृ॒ङ्गाः स॒तीः स॒ती र॑शृ॒ङ्गा अ॑शृ॒ङ्गाः स॒तीः । \newline
7. स॒तीः शृङ्गा॑णि॒ शृङ्गा॑णि स॒तीः स॒तीः शृङ्गा॑णि । \newline
8. शृङ्गा॑णि नो नः॒ शृङ्गा॑णि॒ शृङ्गा॑णि नः । \newline
9. नो॒ जा॒य॒न्तै॒ जा॒य॒न्तै॒ नो॒ नो॒ जा॒य॒न्तै॒ । \newline
10. जा॒य॒न्ता॒ इतीति॑ जायन्तै जायन्ता॒ इति॑ । \newline
11. इति॒ कामे॑न॒ कामे॒नेतीति॒ कामे॑न । \newline
12. कामे॑न॒ तासा॒म् तासा॒म् कामे॑न॒ कामे॑न॒ तासा᳚म् । \newline
13. तासा॒म् दश॒ दश॒ तासा॒म् तासा॒म् दश॑ । \newline
14. दश॒ मासा॒ मासा॒ दश॒ दश॒ मासाः᳚ । \newline
15. मासा॒ निष॑ण्णा॒ निष॑ण्णा॒ मासा॒ मासा॒ निष॑ण्णाः । \newline
16. निष॑ण्णा॒ आस॒न् नास॒न् निष॑ण्णा॒ निष॑ण्णा॒ आसन्न्॑ । \newline
17. निष॑ण्णा॒ इति॒ नि - स॒न्नाः॒ । \newline
18. आस॒न् नथा थास॒न् नास॒न् नथ॑ । \newline
19. अथ॒ शृङ्गा॑णि॒ शृङ्गा॒ ण्यथाथ॒ शृङ्गा॑णि । \newline
20. शृङ्गा᳚ ण्यजायन्ता जायन्त॒ शृङ्गा॑णि॒ शृङ्गा᳚ ण्यजायन्त । \newline
21. अ॒जा॒य॒न्त॒ ता स्ता अ॑जायन्ता जायन्त॒ ताः । \newline
22. ता उदुत् ता स्ता उत् । \newline
23. उद॑तिष्ठन् नतिष्ठ॒न् नुदु द॑तिष्ठन्न् । \newline
24. अ॒ति॒ष्ठ॒न् नरा॒थ्स्मा रा᳚थ्स्मा तिष्ठन् नतिष्ठ॒न् नरा᳚थ्स्म । \newline
25. अरा॒थ्स्मेतीत्य रा॒थ्स्मा रा॒थ्स्मेति॑ । \newline
26. इत्यथा थेती त्यथ॑ । \newline
27. अथ॒ यासां॒ ॅयासा॒ मथाथ॒ यासा᳚म् । \newline
28. यासा॒न् न न यासां॒ ॅयासा॒न् न । \newline
29. ना जा॑य॒न्ता जा॑यन्त॒ न ना जा॑यन्त । \newline
30. अजा॑यन्त॒ ता स्ता अजा॑य॒न्ता जा॑यन्त॒ ताः । \newline
31. ताः सं॑ॅवथ्स॒रꣳ सं॑ॅवथ्स॒रम् ता स्ताः सं॑ॅवथ्स॒रम् । \newline
32. सं॒ॅव॒थ्स॒र मा॒प्त्वा ऽऽप्त्वा सं॑ॅवथ्स॒रꣳ सं॑ॅवथ्स॒र मा॒प्त्वा । \newline
33. सं॒ॅव॒थ्स॒रमिति॑ सं - व॒थ्स॒रम् । \newline
34. आ॒प्त्वोदु दा॒प्त्वा ऽऽप्त्वोत् । \newline
35. उद॑तिष्ठन् नतिष्ठ॒न् नुदु द॑तिष्ठन्न् । \newline
36. अ॒ति॒ष्ठ॒न् नरा॒थ्स्मा रा᳚थ्स्मा तिष्ठन् नतिष्ठ॒न् नरा᳚थ्स्म । \newline
37. अरा॒थ्स्मे तीत्य रा॒थ्स्मा रा॒थ्स्मेति॑ । \newline
38. इति॒ यासां॒ ॅयासा॒ मितीति॒ यासा᳚म् । \newline
39. यासा᳚म् च च॒ यासां॒ ॅयासा᳚म् च । \newline
40. चा जा॑य॒न्ता जा॑यन्त च॒ चा जा॑यन्त । \newline
41. अजा॑यन्त॒ यासां॒ ॅयासा॒ मजा॑य॒न्ता जा॑यन्त॒ यासा᳚म् । \newline
42. यासा᳚म् च च॒ यासां॒ ॅयासा᳚म् च । \newline
43. च॒ न न च॑ च॒ न । \newline
44. न ता स्ता न न ताः । \newline
45. ता उ॒भयी॑ रु॒भयी॒ स्ता स्ता उ॒भयीः᳚ । \newline
46. उ॒भयी॒ रुदु दु॒भयी॑ रु॒भयी॒ रुत् । \newline
47. उद॑तिष्ठन् नतिष्ठ॒न् नुदु द॑तिष्ठन्न् । \newline
48. अ॒ति॒ष्ठ॒न् नरा॒थ्स्मा रा᳚थ्स्मा तिष्ठन् नतिष्ठ॒न् नरा᳚थ्स्म । \newline
49. अरा॒थ्स्मे तीत्य रा॒थ्स्मा रा॒थ्स्मेति॑ । \newline
50. इति॑ गोस॒त्रम् गो॑स॒त्र मितीति॑ गोस॒त्रम् । \newline
51. गो॒स॒त्रं ॅवै वै गो॑स॒त्रम् गो॑स॒त्रं ॅवै । \newline
52. गो॒स॒त्रमिति॑ गो - स॒त्रम् । \newline
53. वै सं॑ॅवथ्स॒रः सं॑ॅवथ्स॒रो वै वै सं॑ॅवथ्स॒रः । \newline

\textbf{Ghana Paata } \newline

1. गावो॒ वै वै गावो॒ गावो॒ वा ए॒त दे॒तद् वै गावो॒ गावो॒ वा ए॒तत् । \newline
2. वा ए॒त दे॒तद् वै वा ए॒तथ् स॒त्रꣳ स॒त्र मे॒तद् वै वा ए॒तथ् स॒त्रम् । \newline
3. ए॒तथ् स॒त्रꣳ स॒त्र मे॒त दे॒तथ् स॒त्र मा॑सता सत स॒त्र मे॒त दे॒तथ् स॒त्र मा॑सत । \newline
4. स॒त्र मा॑सता सत स॒त्रꣳ स॒त्र मा॑सता शृ॒ङ्गा अ॑शृ॒ङ्गा आ॑सत स॒त्रꣳ स॒त्र मा॑सता शृ॒ङ्गाः । \newline
5. आ॒स॒ता॒ शृ॒ङ्गा अ॑शृ॒ङ्गा आ॑सता सता शृ॒ङ्गाः स॒तीः स॒ती र॑शृ॒ङ्गा आ॑सता सता शृ॒ङ्गाः स॒तीः । \newline
6. अ॒शृ॒ङ्गाः स॒तीः स॒ती र॑शृ॒ङ्गा अ॑शृ॒ङ्गाः स॒तीः शृङ्गा॑णि॒ शृङ्गा॑णि स॒ती र॑शृ॒ङ्गा अ॑शृ॒ङ्गाः स॒तीः शृङ्गा॑णि । \newline
7. स॒तीः शृङ्गा॑णि॒ शृङ्गा॑णि स॒तीः स॒तीः शृङ्गा॑णि नो नः॒ शृङ्गा॑णि स॒तीः स॒तीः शृङ्गा॑णि नः । \newline
8. शृङ्गा॑णि नो नः॒ शृङ्गा॑णि॒ शृङ्गा॑णि नो जायन्तै जायन्तै नः॒ शृङ्गा॑णि॒ शृङ्गा॑णि नो जायन्तै । \newline
9. नो॒ जा॒य॒न्तै॒ जा॒य॒न्तै॒ नो॒ नो॒ जा॒य॒न्ता॒ इतीति॑ जायन्तै नो नो जायन्ता॒ इति॑ । \newline
10. जा॒य॒न्ता॒ इतीति॑ जायन्तै जायन्ता॒ इति॒ कामे॑न॒ कामे॒नेति॑ जायन्तै जायन्ता॒ इति॒ कामे॑न । \newline
11. इति॒ कामे॑न॒ कामे॒नेतीति॒ कामे॑न॒ तासा॒म् तासा॒म् कामे॒नेतीति॒ कामे॑न॒ तासा᳚म् । \newline
12. कामे॑न॒ तासा॒म् तासा॒म् कामे॑न॒ कामे॑न॒ तासा॒म् दश॒ दश॒ तासा॒म् कामे॑न॒ कामे॑न॒ तासा॒म् दश॑ । \newline
13. तासा॒म् दश॒ दश॒ तासा॒म् तासा॒म् दश॒ मासा॒ मासा॒ दश॒ तासा॒म् तासा॒म् दश॒ मासाः᳚ । \newline
14. दश॒ मासा॒ मासा॒ दश॒ दश॒ मासा॒ निष॑ण्णा॒ निष॑ण्णा॒ मासा॒ दश॒ दश॒ मासा॒ निष॑ण्णाः । \newline
15. मासा॒ निष॑ण्णा॒ निष॑ण्णा॒ मासा॒ मासा॒ निष॑ण्णा॒ आस॒न् नास॒न् निष॑ण्णा॒ मासा॒ मासा॒ निष॑ण्णा॒ आसन्न्॑ । \newline
16. निष॑ण्णा॒ आस॒न् नास॒न् निष॑ण्णा॒ निष॑ण्णा॒ आस॒न् नथा थास॒न् निष॑ण्णा॒ निष॑ण्णा॒ आस॒न् नथ॑ । \newline
17. निष॑ण्णा॒ इति॒ नि - स॒न्नाः॒ । \newline
18. आस॒न् नथा थास॒न् नास॒न् नथ॒ शृङ्गा॑णि॒ शृङ्गा॒ ण्यथास॒न् नास॒न् नथ॒ शृङ्गा॑णि । \newline
19. अथ॒ शृङ्गा॑णि॒ शृङ्गा॒ ण्यथाथ॒ शृङ्गा᳚ ण्यजायन्ता जायन्त॒ शृङ्गा॒ ण्यथाथ॒ शृङ्गा᳚ ण्यजायन्त । \newline
20. शृङ्गा᳚ ण्यजायन्ता जायन्त॒ शृङ्गा॑णि॒ शृङ्गा᳚ ण्यजायन्त॒ ता स्ता अ॑जायन्त॒ शृङ्गा॑णि॒ शृङ्गा᳚ 
ण्यजायन्त॒ ताः । \newline
21. अ॒जा॒य॒न्त॒ ता स्ता अ॑जायन्ता जायन्त॒ ता उदुत् ता अ॑जायन्ता जायन्त॒ ता उत् । \newline
22. ता उदुत् ता स्ता उद॑तिष्ठन् नतिष्ठ॒न् नुत् ता स्ता उद॑तिष्ठन्न् । \newline
23. उद॑तिष्ठन् नतिष्ठ॒न् नुदुद॑तिष्ठ॒न् नरा॒थ्स्मा रा᳚थ्स्मा तिष्ठ॒न् नुदु द॑तिष्ठ॒न् नरा᳚थ्स्म । \newline
24. अ॒ति॒ष्ठ॒न् नरा॒थ्स्मा रा᳚थ्स्मा तिष्ठन् नतिष्ठ॒न् नरा॒थ्स्मे तीत्यरा᳚थ्स्मा तिष्ठन् नतिष्ठ॒न् नरा॒थ्स्मेति॑ । \newline
25. अरा॒थ्स्मे तीत्यरा॒थ्स्मा रा॒थ्स्मे त्यथाथे त्यरा॒थ्स्मा रा॒थ्स्मे त्यथ॑ । \newline
26. इत्यथाथे तीत्यथ॒ यासां॒ ॅयासा॒ मथे तीत्यथ॒ यासा᳚म् । \newline
27. अथ॒ यासां॒ ॅयासा॒ मथाथ॒ यासा॒न् न न यासा॒ मथाथ॒ यासा॒न् न । \newline
28. यासा॒न् न न यासां॒ ॅयासा॒न् नाजा॑य॒न्ता जा॑यन्त॒ न यासां॒ ॅयासा॒न् ना जा॑यन्त । \newline
29. नाजा॑य॒न्ता जा॑यन्त॒ न नाजा॑यन्त॒ ता स्ता अजा॑यन्त॒ न नाजा॑यन्त॒ ताः । \newline
30. अजा॑यन्त॒ ता स्ता अजा॑य॒न्ता जा॑यन्त॒ ताः सं॑ॅवथ्स॒रꣳ सं॑ॅवथ्स॒रम् ता अजा॑य॒न्ता जा॑यन्त॒ ताः सं॑ॅवथ्स॒रम् । \newline
31. ताः सं॑ॅवथ्स॒रꣳ सं॑ॅवथ्स॒रम् ता स्ताः सं॑ॅवथ्स॒र मा॒प्त्वा ऽऽप्त्वा सं॑ॅवथ्स॒रम् ता स्ताः सं॑ॅवथ्स॒र मा॒प्त्वा । \newline
32. सं॒ॅव॒थ्स॒र मा॒प्त्वा ऽऽप्त्वा सं॑ॅवथ्स॒रꣳ सं॑ॅवथ्स॒र मा॒प्त्वोदु दा॒प्त्वा सं॑ॅवथ्स॒रꣳ सं॑ॅवथ्स॒र मा॒प्त्वोत् । \newline
33. सं॒ॅव॒थ्स॒रमिति॑ सं - व॒थ्स॒रम् । \newline
34. आ॒प्त्वोदु दा॒प्त्वा ऽऽप्त्वो द॑तिष्ठन् नतिष्ठ॒न् नुदा॒प्त्वा ऽऽप्त्वो द॑तिष्ठन्न् । \newline
35. उद॑तिष्ठन् नतिष्ठ॒न् नुदु द॑तिष्ठ॒न् नरा॒थ्स्मा रा᳚थ्स्मा तिष्ठ॒न् नुदु द॑तिष्ठ॒न् नरा᳚थ्स्म । \newline
36. अ॒ति॒ष्ठ॒न् नरा॒थ्स्मा रा᳚थ्स्मा तिष्ठन् नतिष्ठ॒न् नरा॒थ्स्मे तीत्यरा᳚थ्स्मा तिष्ठन् नतिष्ठ॒न् नरा॒थ्स्मेति॑ । \newline
37. अरा॒थ्स्मे तीत्य रा॒थ्स्मा रा॒थ्स्मेति॒ यासां॒ ॅयासा॒ मित्य रा॒थ्स्मा रा॒थ्स्मेति॒ यासा᳚म् । \newline
38. इति॒ यासां॒ ॅयासा॒ मितीति॒ यासा᳚म् च च॒ यासा॒ मितीति॒ यासा᳚म् च । \newline
39. यासा᳚म् च च॒ यासां॒ ॅयासा॒म् चा जा॑य॒न्ता जा॑यन्त च॒ यासां॒ ॅयासा॒म् चा जा॑यन्त । \newline
40. चा जा॑य॒न्ता जा॑यन्त च॒ चा जा॑यन्त॒ यासां॒ ॅयासा॒ मजा॑यन्त च॒ चा जा॑यन्त॒ यासा᳚म् । \newline
41. अजा॑यन्त॒ यासां॒ ॅयासा॒ मजा॑य॒न्ता जा॑यन्त॒ यासा᳚म् च च॒ यासा॒ मजा॑य॒न्ता जा॑यन्त॒ यासा᳚म् च । \newline
42. यासा᳚म् च च॒ यासां॒ ॅयासा᳚म् च॒ न न च॒ यासां॒ ॅयासा᳚म् च॒ न । \newline
43. च॒ न न च॑ च॒ न ता स्ता न च॑ च॒ न ताः । \newline
44. न ता स्ता न न ता उ॒भयी॑ रु॒भयी॒ स्ता न न ता उ॒भयीः᳚ । \newline
45. ता उ॒भयी॑ रु॒भयी॒ स्ता स्ता उ॒भयी॒ रुदु दु॒भयी॒ स्ता स्ता उ॒भयी॒ रुत् । \newline
46. उ॒भयी॒ रुदु दु॒भयी॑ रु॒भयी॒ रुद॑तिष्ठन् नतिष्ठ॒न् नुदु॒भयी॑ रु॒भयी॒ रुद॑तिष्ठन्न् । \newline
47. उद॑तिष्ठन् नतिष्ठ॒न् नुदु द॑तिष्ठ॒न् नरा॒थ्स्मा रा᳚थ्स्मा तिष्ठ॒न् नुदु द॑तिष्ठ॒न् नरा᳚थ्स्म । \newline
48. अ॒ति॒ष्ठ॒न् नरा॒थ्स्मा रा᳚थ्स्मा तिष्ठन् नतिष्ठ॒न् नरा॒थ्स्मे तीत्य रा᳚थ्स्मा तिष्ठन् नतिष्ठ॒न् नरा॒थ्स्मेति॑ । \newline
49. अरा॒थ्स्मे तीत्यरा॒थ्स्मा रा॒थ्स्मेति॑ गोस॒त्रम् गो॑स॒त्र मित्यरा॒थ्स्मा रा॒थ्स्मेति॑ गोस॒त्रम् । \newline
50. इति॑ गोस॒त्रम् गो॑स॒त्र मितीति॑ गोस॒त्रं ॅवै वै गो॑स॒त्र मितीति॑ गोस॒त्रं ॅवै । \newline
51. गो॒स॒त्रं ॅवै वै गो॑स॒त्रम् गो॑स॒त्रं ॅवै सं॑ॅवथ्स॒रः सं॑ॅवथ्स॒रो वै गो॑स॒त्रम् गो॑स॒त्रं ॅवै सं॑ॅवथ्स॒रः । \newline
52. गो॒स॒त्रमिति॑ गो - स॒त्रम् । \newline
53. वै सं॑ॅवथ्स॒रः सं॑ॅवथ्स॒रो वै वै सं॑ॅवथ्स॒रो ये ये सं॑ॅवथ्स॒रो वै वै सं॑ॅवथ्स॒रो ये । \newline
\pagebreak
\markright{ TS 7.5.1.2  \hfill https://www.vedavms.in \hfill}

\section{ TS 7.5.1.2 }

\textbf{TS 7.5.1.2 } \newline
\textbf{Samhita Paata} \newline

सं॑ॅवथ्स॒रो य ए॒वं ॅवि॒द्वाꣳसः॑ संॅवथ्स॒र-मु॑प॒यन्त्यृ॑द्ध्नु॒वन्त्ये॒व तस्मा᳚त् तूप॒रा वार्.षि॑कौ॒ मासौ॒ पर्त्वा॑ चरति स॒त्राभि॑जितꣳ॒॒ह्य॑स्यै॒ तस्मा᳚थ् संॅवथ्सर॒सदो॒ यत् किं च॑ गृ॒हे क्रि॒यते॒ तदा॒प्त-मव॑रुद्ध-म॒भिजि॑तं क्रियते समु॒द्रं ॅवा ए॒ते प्र प्ल॑वन्ते॒ ये सं॑ॅवथ्स॒रमु॑प॒यन्ति॒ यो वै स॑मु॒द्रस्य॑ पा॒रं न पश्य॑ति॒ न वै स तत॒ उदे॑ति संॅवथ्स॒रो - [  ] \newline

\textbf{Pada Paata} \newline

सं॒ॅव॒थ्स॒र इति॑ सं - व॒थ्स॒रः । ये । ए॒वम् । वि॒द्वाꣳसः॑ । सं॒ॅव॒थ्स॒रमिति॑ सं - व॒थ्स॒रम् । उ॒प॒यन्तीत्यु॑प - यन्ति॑ । ऋ॒द्ध्नु॒वन्ति॑ । ए॒व । तस्मा᳚त् । तू॒प॒रा । वार्.षि॑कौ । मासौ᳚ । पर्त्वा᳚ । च॒र॒ति॒ । स॒त्राभि॑जित॒मिति॑ स॒त्र - अ॒भि॒जि॒त॒म् । हि । अ॒स्यै॒ । तस्मा᳚त् । सं॒ॅव॒थ्स॒र॒सद॒ इति॑ संॅवथ्सर - सदः॑ । यत् । किम् । च॒ । गृ॒हे । क्रि॒यते᳚ । तत् । आ॒प्तम् । अव॑रुद्ध॒मित्यव॑-रु॒द्ध॒म् । अ॒भिजि॑त॒मित्य॒भि - जि॒त॒म् । क्रि॒य॒ते॒ । स॒मु॒द्रम् । वै । ए॒ते । प्रेति॑ । प्ल॒व॒न्ते॒ । ये । सं॒ॅव॒थ्स॒रमिति॑ सं - व॒थ्स॒रम् । उ॒प॒यन्तीत्यु॑प-यन्ति॑ । यः । वै । स॒मु॒द्रस्य॑ । पा॒रम् । न । पश्य॑ति । न । वै । सः । ततः॑ । उदिति॑ । ए॒ति॒ । सं॒ॅव॒थ्स॒र इति॑ सं - व॒थ्स॒रः ।  \newline


\textbf{Krama Paata} \newline

स॒म्ॅव॒थ्स॒रो ये । स॒म्ॅव॒थ्स॒र इति॑ सम् - व॒थ्स॒रः । य ए॒वम् । ए॒वम् ॅवि॒द्वाꣳसः॑ । वि॒द्वाꣳसः॑ सम्ॅवथ्स॒रम् । स॒म्ॅव॒थ्स॒रमु॑प॒यन्ति॑ । स॒म्ॅव॒थ्स॒रमिति॑ सम् - व॒थ्स॒रम् । उ॒प॒यन्त्यृ॑द्ध्नु॒वन्ति॑ । उ॒प॒यन्तीत्यु॑प - यन्ति॑ । ऋ॒द्ध्नु॒वन्त्ये॒व । ए॒व तस्मा᳚त् । तस्मा᳚त् तूप॒रा । तू॒प॒रा वार्.षि॑कौ । वार्.षि॑कौ॒ मासौ᳚ । मासौ॒ पर्त्वा᳚ । पर्त्वा॑ चरति । च॒र॒ति॒ स॒त्राभि॑जितम् । स॒त्राभि॑जितꣳ॒॒ हि । स॒त्राभि॑जित॒मिति॑ स॒त्र - अ॒भि॒जि॒त॒म् । ह्य॑स्यै । अ॒स्यै॒ तस्मा᳚त् । तस्मा᳚थ् सम्ॅवथ्सर॒सदः॑ । स॒म्ॅव॒थ्स॒र॒सदो॒ यत् । स॒म्ॅव॒थ्स॒र॒सद॒ इति॑ सम्ॅवथ्सर - सदः॑ । यत् किम् । किम् च॑ । च॒ गृ॒हे । गृ॒हे क्रि॒यते᳚ ।? क्रि॒यते॒ तत् । तदा॒प्तम् । आ॒प्तमव॑रुद्धम् । अव॑रुद्धम॒भिजि॑तम् । अव॑रुद्ध॒मित्यव॑ - रु॒द्ध॒म् । अ॒भिजि॑तम् क्रियते । अ॒भिजि॑त॒मित्य॒भि - जि॒त॒म् । क्रि॒य॒ते॒ स॒मु॒द्रम् । स॒मु॒द्रम् ॅवै । वा ए॒ते । ए॒ते प्र । प्र प्ल॑वन्ते । प्ल॒व॒न्ते॒ ये । ये स॑म्ॅवथ्स॒रम् । स॒म्ॅव॒थ्स॒रमु॑प॒यन्ति॑ । स॒म्ॅव॒थ्स॒रमिति॑ सम् - व॒थ्स॒रम् । उ॒प॒यन्ति॒ यः । उ॒प॒यन्तीत्यु॑प - यन्ति॑ । यो वै । वै स॑मु॒द्रस्य॑ । स॒मु॒द्रस्य॑ पा॒रम् । पा॒रम् न । न पश्य॑ति । पश्य॑ति॒ न । न वै । वै सः । स ततः॑ । तत॒ उत् । उदे॑ति । ए॒ति॒ स॒म्ॅव॒थ्स॒रः । स॒म्ॅव॒थ्स॒रो वै । स॒म्ॅव॒थ्स॒र इति॑ सम् - व॒थ्स॒रः \newline

\textbf{Jatai Paata} \newline

1. सं॒ॅव॒थ्स॒रो ये ये सं॑ॅवथ्स॒रः सं॑ॅवथ्स॒रो ये । \newline
2. सं॒ॅव॒थ्स॒र इति॑ सं - व॒थ्स॒रः । \newline
3. य ए॒व मे॒वं ॅये य ए॒वम् । \newline
4. ए॒वं ॅवि॒द्वाꣳसो॑ वि॒द्वाꣳस॑ ए॒व मे॒वं ॅवि॒द्वाꣳसः॑ । \newline
5. वि॒द्वाꣳसः॑ संॅवथ्स॒रꣳ सं॑ॅवथ्स॒रं ॅवि॒द्वाꣳसो॑ वि॒द्वाꣳसः॑ संॅवथ्स॒रम् । \newline
6. सं॒ॅव॒थ्स॒र मु॑प॒य न्त्यु॑प॒यन्ति॑ संॅवथ्स॒रꣳ सं॑ॅवथ्स॒र मु॑प॒यन्ति॑ । \newline
7. सं॒ॅव॒थ्स॒रमिति॑ सं - व॒थ्स॒रम् । \newline
8. उ॒प॒य न्त्यृ॑द्ध्नु॒व न्त्यृ॑द्ध्नु॒व न्त्यु॑प॒य न्त्यु॑प॒य न्त्यृ॑द्ध्नु॒वन्ति॑ । \newline
9. उ॒प॒यन्तीत्यु॑प - यन्ति॑ । \newline
10. ऋ॒द्ध्नु॒व न्त्ये॒वैव र्द्ध्नु॒व न्त्यृ॑द्ध्नु॒व न्त्ये॒व । \newline
11. ए॒व तस्मा॒त् तस्मा॑ दे॒वैव तस्मा᳚त् । \newline
12. तस्मा᳚त् तूप॒रा तू॑प॒रा तस्मा॒त् तस्मा᳚त् तूप॒रा । \newline
13. तू॒प॒रा वार्.षि॑कौ॒ वार्.षि॑कौ तूप॒रा तू॑प॒रा वार्.षि॑कौ । \newline
14. वार्.षि॑कौ॒ मासौ॒ मासौ॒ वार्.षि॑कौ॒ वार्.षि॑कौ॒ मासौ᳚ । \newline
15. मासौ॒ पर्त्वा॒ पर्त्वा॒ मासौ॒ मासौ॒ पर्त्वा᳚ । \newline
16. पर्त्वा॑ चरति चरति॒ पर्त्वा॒ पर्त्वा॑ चरति । \newline
17. च॒र॒ति॒ स॒त्राभि॑जितꣳ स॒त्राभि॑जितम् चरति चरति स॒त्राभि॑जितम् । \newline
18. स॒त्राभि॑जितꣳ॒॒ हि हि स॒त्राभि॑जितꣳ स॒त्राभि॑जितꣳ॒॒ हि । \newline
19. स॒त्राभि॑जित॒मिति॑ स॒त्र - अ॒भि॒जि॒त॒म् । \newline
20. ह्य॑स्या अस्यै॒ हि ह्य॑स्यै । \newline
21. अ॒स्यै॒ तस्मा॒त् तस्मा॑ दस्या अस्यै॒ तस्मा᳚त् । \newline
22. तस्मा᳚थ् संॅवथ्सर॒सदः॑ संॅवथ्सर॒सद॒ स्तस्मा॒त् तस्मा᳚थ् संॅवथ्सर॒सदः॑ । \newline
23. सं॒ॅव॒थ्स॒र॒सदो॒ यद् यथ् सं॑ॅवथ्सर॒सदः॑ संॅवथ्सर॒सदो॒ यत् । \newline
24. सं॒ॅव॒थ्स॒र॒सद॒ इति॑ संॅवथ्सर - सदः॑ । \newline
25. यत् किम् किं ॅयद् यत् किम् । \newline
26. किम् च॑ च॒ किम् किम् च॑ । \newline
27. च॒ गृ॒हे गृ॒हे च॑ च गृ॒हे । \newline
28. गृ॒हे क्रि॒यते᳚ क्रि॒यते॑ गृ॒हे गृ॒हे क्रि॒यते᳚ । \newline
29. क्रि॒यते॒ तत् तत् क्रि॒यते᳚ क्रि॒यते॒ तत् । \newline
30. तदा॒प्त मा॒प्तम् तत् तदा॒प्तम् । \newline
31. आ॒प्त मव॑रुद्ध॒ मव॑रुद्ध मा॒प्त मा॒प्त मव॑रुद्धम् । \newline
32. अव॑रुद्ध म॒भिजि॑त म॒भिजि॑त॒ मव॑रुद्ध॒ मव॑रुद्ध म॒भिजि॑तम् । \newline
33. अव॑रुद्ध॒मित्यव॑ - रु॒द्ध॒म् । \newline
34. अ॒भिजि॑तम् क्रियते क्रियते॒ ऽभिजि॑त म॒भिजि॑तम् क्रियते । \newline
35. अ॒भिजि॑त॒मित्य॒भि - जि॒त॒म् । \newline
36. क्रि॒य॒ते॒ स॒मु॒द्रꣳ स॑मु॒द्रम् क्रि॑यते क्रियते समु॒द्रम् । \newline
37. स॒मु॒द्रं ॅवै वै स॑मु॒द्रꣳ स॑मु॒द्रं ॅवै । \newline
38. वा ए॒त ए॒ते वै वा ए॒ते । \newline
39. ए॒ते प्र प्रैत ए॒ते प्र । \newline
40. प्र प्ल॑वन्ते प्लवन्ते॒ प्र प्र प्ल॑वन्ते । \newline
41. प्ल॒व॒न्ते॒ ये ये प्ल॑वन्ते प्लवन्ते॒ ये । \newline
42. ये सं॑ॅवथ्स॒रꣳ सं॑ॅवथ्स॒रं ॅये ये सं॑ॅवथ्स॒रम् । \newline
43. सं॒ॅव॒थ्स॒र मु॑प॒य न्त्यु॑प॒यन्ति॑ संॅवथ्स॒रꣳ सं॑ॅवथ्स॒र मु॑प॒यन्ति॑ । \newline
44. सं॒ॅव॒थ्स॒रमिति॑ सं - व॒थ्स॒रम् । \newline
45. उ॒प॒यन्ति॒ यो य उ॑प॒य न्त्यु॑प॒यन्ति॒ यः । \newline
46. उ॒प॒यन्तीत्यु॑प - यन्ति॑ । \newline
47. यो वै वै यो यो वै । \newline
48. वै स॑मु॒द्रस्य॑ समु॒द्रस्य॒ वै वै स॑मु॒द्रस्य॑ । \newline
49. स॒मु॒द्रस्य॑ पा॒रम् पा॒रꣳ स॑मु॒द्रस्य॑ समु॒द्रस्य॑ पा॒रम् । \newline
50. पा॒रन् न न पा॒रम् पा॒रन् न । \newline
51. न पश्य॑ति॒ पश्य॑ति॒ न न पश्य॑ति । \newline
52. पश्य॑ति॒ न न पश्य॑ति॒ पश्य॑ति॒ न । \newline
53. न वै वै न न वै । \newline
54. वै स स वै वै सः । \newline
55. स तत॒ स्ततः॒ स स ततः॑ । \newline
56. तत॒ उदुत् तत॒ स्तत॒ उत् । \newline
57. उदे᳚ त्ये॒त्युदु दे॑ति । \newline
58. ए॒ति॒ सं॒ॅव॒थ्स॒रः सं॑ॅवथ्स॒र ए᳚त्येति संॅवथ्स॒रः । \newline
59. सं॒ॅव॒थ्स॒रो वै वै सं॑ॅवथ्स॒रः सं॑ॅवथ्स॒रो वै । \newline
60. सं॒ॅव॒थ्स॒र इति॑ सं - व॒थ्स॒रः । \newline

\textbf{Ghana Paata } \newline

1. सं॒ॅव॒थ्स॒रो ये ये सं॑ॅवथ्स॒रः सं॑ॅवथ्स॒रो य ए॒व मे॒वं ॅये सं॑ॅवथ्स॒रः सं॑ॅवथ्स॒रो य ए॒वम् । \newline
2. सं॒ॅव॒थ्स॒र इति॑ सं - व॒थ्स॒रः । \newline
3. य ए॒व मे॒वं ॅये य ए॒वं ॅवि॒द्वाꣳसो॑ वि॒द्वाꣳस॑ ए॒वं ॅये य ए॒वं ॅवि॒द्वाꣳसः॑ । \newline
4. ए॒वं ॅवि॒द्वाꣳसो॑ वि॒द्वाꣳस॑ ए॒व मे॒वं ॅवि॒द्वाꣳसः॑ संॅवथ्स॒रꣳ सं॑ॅवथ्स॒रं ॅवि॒द्वाꣳस॑ ए॒व मे॒वं ॅवि॒द्वाꣳसः॑ संॅवथ्स॒रम् । \newline
5. वि॒द्वाꣳसः॑ संॅवथ्स॒रꣳ सं॑ॅवथ्स॒रं ॅवि॒द्वाꣳसो॑ वि॒द्वाꣳसः॑ संॅवथ्स॒र मु॑प॒य न्त्यु॑प॒यन्ति॑ संॅवथ्स॒रं ॅवि॒द्वाꣳसो॑ वि॒द्वाꣳसः॑ संॅवथ्स॒र मु॑प॒यन्ति॑ । \newline
6. सं॒ॅव॒थ्स॒र मु॑प॒य न्त्यु॑प॒यन्ति॑ संॅवथ्स॒रꣳ सं॑ॅवथ्स॒र मु॑प॒य न्त्यृ॑द्ध्नु॒व
न्त्यृ॑द्ध्नु॒व न्त्यु॑प॒यन्ति॑ संॅवथ्स॒रꣳ सं॑ॅवथ्स॒र मु॑प॒य न्त्यृ॑द्ध्नु॒वन्ति॑ । \newline
7. सं॒ॅव॒थ्स॒रमिति॑ सं - व॒थ्स॒रम् । \newline
8. उ॒प॒य न्त्यृ॑द्ध्नु॒व न्त्यृ॑द्ध्नु॒व न्त्यु॑प॒य न्त्यु॑प॒य न्त्यृ॑द्ध्नु॒व न्त्ये॒वैव र्‌द्ध्नु॒व न्त्यु॑प॒य न्त्यु॑प॒य न्त्यृ॑द्ध्नु॒व न्त्ये॒व । \newline
9. उ॒प॒यन्तीत्यु॑प - यन्ति॑ । \newline
10. ऋ॒द्ध्नु॒व न्त्ये॒वैव र्‌द्ध्नु॒व न्त्यृ॑द्ध्नु॒व न्त्ये॒व तस्मा॒त् तस्मा॑ दे॒व र्‌द्ध्नु॒व न्त्यृ॑द्ध्नु॒व न्त्ये॒व तस्मा᳚त् । \newline
11. ए॒व तस्मा॒त् तस्मा॑ दे॒वैव तस्मा᳚त् तूप॒रा तू॑प॒रा तस्मा॑ दे॒वैव तस्मा᳚त् तूप॒रा । \newline
12. तस्मा᳚त् तूप॒रा तू॑प॒रा तस्मा॒त् तस्मा᳚त् तूप॒रा वार्.षि॑कौ॒ वार्.षि॑कौ तूप॒रा तस्मा॒त् तस्मा᳚त् तूप॒रा वार्.षि॑कौ । \newline
13. तू॒प॒रा वार्.षि॑कौ॒ वार्.षि॑कौ तूप॒रा तू॑प॒रा वार्.षि॑कौ॒ मासौ॒ मासौ॒ वार्.षि॑कौ तूप॒रा तू॑प॒रा वार्.षि॑कौ॒ मासौ᳚ । \newline
14. वार्.षि॑कौ॒ मासौ॒ मासौ॒ वार्.षि॑कौ॒ वार्.षि॑कौ॒ मासौ॒ पर्त्वा॒ पर्त्वा॒ मासौ॒ वार्.षि॑कौ॒ वार्.षि॑कौ॒ मासौ॒ पर्त्वा᳚ । \newline
15. मासौ॒ पर्त्वा॒ पर्त्वा॒ मासौ॒ मासौ॒ पर्त्वा॑ चरति चरति॒ पर्त्वा॒ मासौ॒ मासौ॒ पर्त्वा॑ चरति । \newline
16. पर्त्वा॑ चरति चरति॒ पर्त्वा॒ पर्त्वा॑ चरति स॒त्राभि॑जितꣳ स॒त्राभि॑जितम् चरति॒ पर्त्वा॒ पर्त्वा॑ चरति स॒त्राभि॑जितम् । \newline
17. च॒र॒ति॒ स॒त्राभि॑जितꣳ स॒त्राभि॑जितम् चरति चरति स॒त्राभि॑जितꣳ॒॒ हि हि स॒त्राभि॑जितम् चरति चरति स॒त्राभि॑जितꣳ॒॒ हि । \newline
18. स॒त्राभि॑जितꣳ॒॒ हि हि स॒त्राभि॑जितꣳ स॒त्राभि॑जितꣳ॒॒ ह्य॑स्या अस्यै॒ हि स॒त्राभि॑जितꣳ स॒त्राभि॑जितꣳ॒॒ ह्य॑स्यै । \newline
19. स॒त्राभि॑जित॒मिति॑ स॒त्र - अ॒भि॒जि॒त॒म् । \newline
20. ह्य॑स्या अस्यै॒ हि ह्य॑स्यै॒ तस्मा॒त् तस्मा॑ दस्यै॒ हि ह्य॑स्यै॒ तस्मा᳚त् । \newline
21. अ॒स्यै॒ तस्मा॒त् तस्मा॑ दस्या अस्यै॒ तस्मा᳚थ् संॅवथ्सर॒सदः॑ संॅवथ्सर॒सद॒ स्तस्मा॑ दस्या अस्यै॒ तस्मा᳚थ् संॅवथ्सर॒सदः॑ । \newline
22. तस्मा᳚थ् संॅवथ्सर॒सदः॑ संॅवथ्सर॒सद॒ स्तस्मा॒त् तस्मा᳚थ् संॅवथ्सर॒सदो॒ यद् यथ् सं॑ॅवथ्सर॒सद॒ स्तस्मा॒त् तस्मा᳚थ् संॅवथ्सर॒सदो॒ यत् । \newline
23. सं॒ॅव॒थ्स॒र॒सदो॒ यद् यथ् सं॑ॅवथ्सर॒सदः॑ संॅवथ्सर॒सदो॒ यत् किम् किं ॅयथ् सं॑ॅवथ्सर॒सदः॑ संॅवथ्सर॒सदो॒ यत् किम् । \newline
24. सं॒ॅव॒थ्स॒र॒सद॒ इति॑ संॅवथ्सर - सदः॑ । \newline
25. यत् किम् किं ॅयद् यत् किम् च॑ च॒ किं ॅयद् यत् किम् च॑ । \newline
26. किम् च॑ च॒ किम् किम् च॑ गृ॒हे गृ॒हे च॒ किम् किम् च॑ गृ॒हे । \newline
27. च॒ गृ॒हे गृ॒हे च॑ च गृ॒हे क्रि॒यते᳚ क्रि॒यते॑ गृ॒हे च॑ च गृ॒हे क्रि॒यते᳚ । \newline
28. गृ॒हे क्रि॒यते᳚ क्रि॒यते॑ गृ॒हे गृ॒हे क्रि॒यते॒ तत् तत् क्रि॒यते॑ गृ॒हे गृ॒हे क्रि॒यते॒ तत् । \newline
29. क्रि॒यते॒ तत् तत् क्रि॒यते᳚ क्रि॒यते॒ तदा॒प्त मा॒प्तम् तत् क्रि॒यते᳚ क्रि॒यते॒ तदा॒प्तम् । \newline
30. तदा॒प्त मा॒प्तम् तत् तदा॒प्त मव॑रुद्ध॒ मव॑रुद्ध मा॒प्तम् तत् तदा॒प्त मव॑रुद्धम् । \newline
31. आ॒प्त मव॑रुद्ध॒ मव॑रुद्ध मा॒प्त मा॒प्त मव॑रुद्ध म॒भिजि॑त म॒भिजि॑त॒ मव॑रुद्ध मा॒प्त मा॒प्त मव॑रुद्ध म॒भिजि॑तम् । \newline
32. अव॑रुद्ध म॒भिजि॑त म॒भिजि॑त॒ मव॑रुद्ध॒ मव॑रुद्ध म॒भिजि॑तम् क्रियते क्रियते॒ ऽभिजि॑त॒ मव॑रुद्ध॒ मव॑रुद्ध म॒भिजि॑तम् क्रियते । \newline
33. अव॑रुद्ध॒मित्यव॑ - रु॒द्ध॒म् । \newline
34. अ॒भिजि॑तम् क्रियते क्रियते॒ ऽभिजि॑त म॒भिजि॑तम् क्रियते समु॒द्रꣳ स॑मु॒द्रम् क्रि॑यते॒ ऽभिजि॑त म॒भिजि॑तम् क्रियते समु॒द्रम् । \newline
35. अ॒भिजि॑त॒मित्य॒भि - जि॒त॒म् । \newline
36. क्रि॒य॒ते॒ स॒मु॒द्रꣳ स॑मु॒द्रम् क्रि॑यते क्रियते समु॒द्रं ॅवै वै स॑मु॒द्रम् क्रि॑यते क्रियते समु॒द्रं ॅवै । \newline
37. स॒मु॒द्रं ॅवै वै स॑मु॒द्रꣳ स॑मु॒द्रं ॅवा ए॒त ए॒ते वै स॑मु॒द्रꣳ स॑मु॒द्रं ॅवा ए॒ते । \newline
38. वा ए॒त ए॒ते वै वा ए॒ते प्र प्रैते वै वा ए॒ते प्र । \newline
39. ए॒ते प्र प्रैत ए॒ते प्र प्ल॑वन्ते प्लवन्ते॒ प्रैत ए॒ते प्र प्ल॑वन्ते । \newline
40. प्र प्ल॑वन्ते प्लवन्ते॒ प्र प्र प्ल॑वन्ते॒ ये ये प्ल॑वन्ते॒ प्र प्र प्ल॑वन्ते॒ ये । \newline
41. प्ल॒व॒न्ते॒ ये ये प्ल॑वन्ते प्लवन्ते॒ ये सं॑ॅवथ्स॒रꣳ सं॑ॅवथ्स॒रं ॅये प्ल॑वन्ते प्लवन्ते॒ ये सं॑ॅवथ्स॒रम् । \newline
42. ये सं॑ॅवथ्स॒रꣳ सं॑ॅवथ्स॒रं ॅये ये सं॑ॅवथ्स॒र मु॑प॒य न्त्यु॑प॒यन्ति॑ संॅवथ्स॒रं ॅये ये सं॑ॅवथ्स॒र मु॑प॒यन्ति॑ । \newline
43. सं॒ॅव॒थ्स॒र मु॑प॒य न्त्यु॑प॒यन्ति॑ संॅवथ्स॒रꣳ सं॑ॅवथ्स॒र मु॑प॒यन्ति॒ यो य उ॑प॒यन्ति॑ संॅवथ्स॒रꣳ सं॑ॅवथ्स॒र मु॑प॒यन्ति॒ यः । \newline
44. सं॒ॅव॒थ्स॒रमिति॑ सं - व॒थ्स॒रम् । \newline
45. उ॒प॒यन्ति॒ यो य उ॑प॒य न्त्यु॑प॒यन्ति॒ यो वै वै य उ॑प॒य न्त्यु॑प॒यन्ति॒ यो वै । \newline
46. उ॒प॒यन्तीत्यु॑प - यन्ति॑ । \newline
47. यो वै वै यो यो वै स॑मु॒द्रस्य॑ समु॒द्रस्य॒ वै यो यो वै स॑मु॒द्रस्य॑ । \newline
48. वै स॑मु॒द्रस्य॑ समु॒द्रस्य॒ वै वै स॑मु॒द्रस्य॑ पा॒रम् पा॒रꣳ स॑मु॒द्रस्य॒ वै वै स॑मु॒द्रस्य॑ पा॒रम् । \newline
49. स॒मु॒द्रस्य॑ पा॒रम् पा॒रꣳ स॑मु॒द्रस्य॑ समु॒द्रस्य॑ पा॒रन् न न पा॒रꣳ स॑मु॒द्रस्य॑ समु॒द्रस्य॑ पा॒रन् न । \newline
50. पा॒रन् न न पा॒रम् पा॒रन् न पश्य॑ति॒ पश्य॑ति॒ न पा॒रम् पा॒रन् न पश्य॑ति । \newline
51. न पश्य॑ति॒ पश्य॑ति॒ न न पश्य॑ति॒ न न पश्य॑ति॒ न न पश्य॑ति॒ न । \newline
52. पश्य॑ति॒ न न पश्य॑ति॒ पश्य॑ति॒ न वै वै न पश्य॑ति॒ पश्य॑ति॒ न वै । \newline
53. न वै वै न न वै स स वै न न वै सः । \newline
54. वै स स वै वै स तत॒ स्ततः॒ स वै वै स ततः॑ । \newline
55. स तत॒ स्ततः॒ स स तत॒ उदुत् ततः॒ स स तत॒ उत् । \newline
56. तत॒ उदुत् तत॒ स्तत॒ उदे᳚ त्ये॒त्युत् तत॒ स्तत॒ उदे॑ति । \newline
57. उदे᳚ त्ये॒त्यु दुदे॑ति संॅवथ्स॒रः सं॑ॅवथ्स॒र ए॒त्यु दुदे॑ति संॅवथ्स॒रः । \newline
58. ए॒ति॒ सं॒ॅव॒थ्स॒रः सं॑ॅवथ्स॒र ए᳚त्येति संॅवथ्स॒रो वै वै सं॑ॅवथ्स॒र ए᳚त्येति संॅवथ्स॒रो वै । \newline
59. सं॒ॅव॒थ्स॒रो वै वै सं॑ॅवथ्स॒रः सं॑ॅवथ्स॒रो वै स॑मु॒द्रः स॑मु॒द्रो वै सं॑ॅवथ्स॒रः सं॑ॅवथ्स॒रो वै स॑मु॒द्रः । \newline
60. सं॒ॅव॒थ्स॒र इति॑ सं - व॒थ्स॒रः । \newline
\pagebreak
\markright{ TS 7.5.1.3  \hfill https://www.vedavms.in \hfill}

\section{ TS 7.5.1.3 }

\textbf{TS 7.5.1.3 } \newline
\textbf{Samhita Paata} \newline

वै स॑मु॒द्रस्तस्यै॒तत् पा॒रं ॅयद॑तिरा॒त्रौ य ए॒वं ॅवि॒द्वासः॑ संॅवथ्स॒र-मु॑प॒यन्त्यना᳚र्ता ए॒वोदृचं॑ गच्छन्ती॒यं ॅवै पूर्वो॑ऽतिरा॒त्रो॑ ऽसावुत्त॑रो॒ मनः॒ पूर्वो॒ वागुत्त॑रः प्रा॒णः पूर्वो॑ऽपा॒न उत्त॑रः प्र॒रोध॑नं॒ पूर्व॑ उ॒दय॑न॒मुत्त॑रो॒ ज्योति॑ष्टोमो वैश्वान॒रो॑ ऽतिरा॒त्रो भ॑वति॒ ज्योति॑रे॒व पु॒रस्ता᳚द्दधते सुव॒र्गस्य॑ लो॒कस्या-नु॑ख्यात्यै चतुर्विꣳ॒॒शः प्रा॑य॒णीयो॑ भवति॒ चतु॑र्विꣳशति-रर्द्धमा॒साः - [  ] \newline

\textbf{Pada Paata} \newline

वै । स॒मु॒द्रः । तस्य॑ । ए॒तत् । पा॒रम् । यत् । अ॒ति॒रा॒त्रावित्य॑ति-रा॒त्रौ । ये । ए॒वम् । वि॒द्वाꣳसः॑ । सं॒ॅव॒थ्स॒रमिति॑ सं - व॒थ्स॒रम् । उ॒प॒यन्तीत्यु॑प - यन्ति॑ । अना᳚र्ताः । ए॒व । उ॒दृच॒मित्यु॑त् - ऋच᳚म् । ग॒च्छ॒न्ति॒ । इ॒यम् । वै । पूर्वः॑ । अ॒ति॒रा॒त्र इत्य॑ति - रा॒त्रः । अ॒सौ । उत्त॑र॒ इत्युत् - त॒रः॒ । मनः॑ । पूर्वः॑ । वाक् । उत्त॑र॒ इत्युत् - त॒रः॒ । प्रा॒ण इति॑ प्र -अ॒नः । पूर्वः॑ । अ॒पा॒न इत्य॑प -   अ॒नः । उत्त॑र॒ इत्युत् -त॒रः॒ । प्र॒रोध॑न॒मिति॑ प्र - रोध॑नम् । पूर्वः॑ । उ॒दय॑न॒मित्यु॑त्-अय॑नम् । उत्त॑र॒ इत्युत् - त॒रः॒ । ज्योति॑ष्टोम॒ इति॒ ज्योतिः॑-स्तो॒मः॒ । वै॒श्वा॒न॒रः । अ॒ति॒रा॒त्र इत्य॑ति - रा॒त्रः । भ॒व॒ति॒ । ज्योतिः॑ । ए॒व । पु॒रस्ता᳚त् । द॒ध॒ते॒ । सु॒व॒र्गस्येति॑ सुवः - गस्य॑ । लो॒कस्य॑ । अनु॑ख्यात्या॒ इत्यनु॑ - ख्या॒त्यै॒ । च॒तु॒र्विꣳ॒॒श इति॑ चतुः - विꣳ॒॒शः । प्रा॒य॒णीय॒ इति॑ प्र - अ॒य॒नीयः॑ । भ॒व॒ति॒ । चतु॑र्विꣳशति॒रिति॒ चतुः॑ - विꣳ॒॒श॒तिः॒ । अ॒द्‌र्ध॒मा॒सा इत्य॑द्‌र्ध - मा॒साः ।  \newline


\textbf{Krama Paata} \newline

वै स॑मु॒द्रः । स॒मु॒द्रस्तस्य॑ । तस्यै॒तत् । ए॒तत् पा॒रम् । पा॒रम् ॅयत् । यद॑तिरा॒त्रौ । अ॒ति॒रा॒त्रौ ये । अ॒ति॒रा॒त्रावित्य॑ति - रा॒त्रौ । य ए॒वम् । ए॒वम् ॅवि॒द्वाꣳसः॑ । वि॒द्वाꣳसः॑ सम्ॅवथ्स॒रम् । स॒म्ॅव॒थ्स॒रमु॑प॒यन्ति॑ । स॒म्ॅव॒थ्स॒रमिति॑ सम् - व॒थ्स॒रम् । उ॒प॒यन्त्यना᳚र्ताः । उ॒प॒यन्तीत्यु॑प - यन्ति॑ । अना᳚र्ता ए॒व । ए॒वोदृच᳚म् । उ॒दृच॑म् गच्छन्ति । उ॒दृच॒मित्यु॑त् - ऋच᳚म् । ग॒च्छ॒न्ती॒यम् । इ॒यम् ॅवै । वै पूर्वः॑ । पूर्वो॑ऽतिरा॒त्रः । अ॒ति॒रा॒त्रो॑ऽसौ । अ॒ति॒रा॒त्र इत्य॑ति - रा॒त्रः । अ॒सावुत्त॑रः । उत्त॑रो॒ मनः॑ । उत्त॑र॒ इत्युत् - त॒रः॒ । मनः॒ पूर्वः॑ । पूर्वो॒ वाक् । वागुत्त॑रः । उत्त॑रः प्रा॒णः । उत्त॑र॒ इत्युत् - त॒रः॒ । प्रा॒णः पूर्वः॑ । प्रा॒ण इति॑ प्र - अ॒नः । पूर्वो॑ऽपा॒नः । अ॒पा॒न उत्त॑रः । अ॒पा॒न इत्य॑प - अ॒नः । उत्त॑रः प्र॒रोध॑नम् । उत्त॑र॒ इत्युत् - त॒रः॒ । प्र॒रोध॑न॒म् पूर्वः॑ । प्र॒रोध॑न॒मिति॑ प्र - रोध॑नम् । पूर्व॑ उ॒दय॑नम् । उ॒दय॑न॒मुत्त॑रः । उ॒दय॑न॒मित्यु॑त् - अय॑नम् । उत्त॑रो॒ ज्योति॑ष्टोमः । उत्त॑र॒ इत्युत् - त॒रः॒ । ज्योति॑ष्टोमो वैश्वान॒रः । ज्योति॑ष्टोम॒ इति॒ ज्योतिः॑ - स्तो॒मः॒ । वै॒श्वा॒न॒रो॑ऽतिरा॒त्रः । अ॒ति॒रा॒त्रो भ॑वति । अ॒ति॒रा॒त्र इत्य॑ति - रा॒त्रः । भ॒व॒ति॒ ज्योतिः॑ । ज्योति॑रे॒व । ए॒व पु॒रस्ता᳚त् । पु॒रस्ता᳚द् दधते । द॒ध॒ते॒ सु॒व॒र्गस्य॑ । सु॒व॒र्गस्य॑ लो॒कस्य॑ । सु॒व॒र्गस्येति॑ सुवः - गस्य॑ । लो॒कस्यानु॑ख्यात्यै । अनु॑ख्यात्यै चतुर्विꣳ॒॒शः । अनु॑ख्यात्या॒ इत्यनु॑ - ख्या॒त्यै॒ । च॒तु॒र्विꣳ॒॒शः प्रा॑य॒णीयः॑ । च॒तु॒र्विꣳ॒॒श इति॑ चतुः - विꣳ॒॒शः । प्रा॒य॒णीयो॑ भवति । प्रा॒य॒णीय॒ इति॑ प्र - अ॒य॒नीयः॑ । भ॒व॒ति॒ चतु॑र्विꣳशतिः । चतु॑र्विꣳशतिरर्द्धमा॒साः । चतु॑र्विꣳशति॒रिति॒ चतुः॑ - विꣳ॒॒श॒तिः॒ । अ॒र्द्ध॒मा॒साः स॑म्ॅवथ्स॒रः । अ॒र्द्ध॒मा॒सा इत्य॑र्द्ध - मा॒साः \newline

\textbf{Jatai Paata} \newline

1. वै स॑मु॒द्रः स॑मु॒द्रो वै वै स॑मु॒द्रः । \newline
2. स॒मु॒द्र स्तस्य॒ तस्य॑ समु॒द्रः स॑मु॒द्र स्तस्य॑ । \newline
3. तस्यै॒त दे॒तत् तस्य॒ तस्यै॒तत् । \newline
4. ए॒तत् पा॒रम् पा॒र मे॒त दे॒तत् पा॒रम् । \newline
5. पा॒रं ॅयद् यत् पा॒रम् पा॒रं ॅयत् । \newline
6. यद॑तिरा॒त्रा व॑तिरा॒त्रौ यद् यद॑तिरा॒त्रौ । \newline
7. अ॒ति॒रा॒त्रौ ये ये॑ ऽतिरा॒त्रा व॑तिरा॒त्रौ ये । \newline
8. अ॒ति॒रा॒त्रावित्य॑ति - रा॒त्रौ । \newline
9. य ए॒व मे॒वं ॅये य ए॒वम् । \newline
10. ए॒वं ॅवि॒द्वाꣳसो॑ वि॒द्वाꣳस॑ ए॒व मे॒वं ॅवि॒द्वाꣳसः॑ । \newline
11. वि॒द्वाꣳसः॑ संॅवथ्स॒रꣳ सं॑ॅवथ्स॒रं ॅवि॒द्वाꣳसो॑ वि॒द्वाꣳसः॑ संॅवथ्स॒रम् । \newline
12. सं॒ॅव॒थ्स॒र मु॑प॒य न्त्यु॑प॒यन्ति॑ संॅवथ्स॒रꣳ सं॑ॅवथ्स॒र मु॑प॒यन्ति॑ । \newline
13. सं॒ॅव॒थ्स॒रमिति॑ सं - व॒थ्स॒रम् । \newline
14. उ॒प॒य न्त्यना᳚र्ता॒ अना᳚र्ता उप॒य न्त्यु॑प॒य न्त्यना᳚र्ताः । \newline
15. उ॒प॒यन्तीत्यु॑प - यन्ति॑ । \newline
16. अना᳚र्ता ए॒वैवा ना᳚र्ता॒ अना᳚र्ता ए॒व । \newline
17. ए॒वोदृच॑ मु॒दृच॑ मे॒वै वोदृच᳚म् । \newline
18. उ॒दृच॑म् गच्छन्ति गच्छ न्त्यु॒दृच॑ मु॒दृच॑म् गच्छन्ति । \newline
19. उ॒दृच॒मित्यु॑त् - ऋच᳚म् । \newline
20. ग॒च्छ॒न्ती॒य मि॒यम् ग॑च्छन्ति गच्छन्ती॒यम् । \newline
21. इ॒यं ॅवै वा इ॒य मि॒यं ॅवै । \newline
22. वै पूर्वः॒ पूर्वो॒ वै वै पूर्वः॑ । \newline
23. पूर्वो॑ ऽतिरा॒त्रो॑ ऽतिरा॒त्रः पूर्वः॒ पूर्वो॑ ऽतिरा॒त्रः । \newline
24. अ॒ति॒रा॒त्रो॑ ऽसा व॒सा व॑तिरा॒त्रो॑ ऽतिरा॒त्रो॑ ऽसौ । \newline
25. अ॒ति॒रा॒त्र इत्य॑ति - रा॒त्रः । \newline
26. अ॒सा वुत्त॑र॒ उत्त॑रो॒ ऽसा व॒सा वुत्त॑रः । \newline
27. उत्त॑रो॒ मनो॒ मन॒ उत्त॑र॒ उत्त॑रो॒ मनः॑ । \newline
28. उत्त॑र॒ इत्युत् - त॒रः॒ । \newline
29. मनः॒ पूर्वः॒ पूर्वो॒ मनो॒ मनः॒ पूर्वः॑ । \newline
30. पूर्वो॒ वाग् वाक् पूर्वः॒ पूर्वो॒ वाक् । \newline
31. वागुत्त॑र॒ उत्त॑रो॒ वाग् वागुत्त॑रः । \newline
32. उत्त॑रः प्रा॒णः प्रा॒ण उत्त॑र॒ उत्त॑रः प्रा॒णः । \newline
33. उत्त॑र॒ इत्युत् - त॒रः॒ । \newline
34. प्रा॒णः पूर्वः॒ पूर्वः॑ प्रा॒णः प्रा॒णः पूर्वः॑ । \newline
35. प्रा॒ण इति॑ प्र - अ॒नः । \newline
36. पूर्वो॑ ऽपा॒नो॑ ऽपा॒नः पूर्वः॒ पूर्वो॑ ऽपा॒नः । \newline
37. अ॒पा॒न उत्त॑र॒ उत्त॑रो ऽपा॒नो॑ ऽपा॒न उत्त॑रः । \newline
38. अ॒पा॒न इत्य॑प - अ॒नः । \newline
39. उत्त॑रः प्र॒रोध॑नम् प्र॒रोध॑न॒ मुत्त॑र॒ उत्त॑रः प्र॒रोध॑नम् । \newline
40. उत्त॑र॒ इत्युत् - त॒रः॒ । \newline
41. प्र॒रोध॑न॒म् पूर्वः॒ पूर्वः॑ प्र॒रोध॑नम् प्र॒रोध॑न॒म् पूर्वः॑ । \newline
42. प्र॒रोध॑न॒मिति॑ प्र - रोध॑नम् । \newline
43. पूर्व॑ उ॒दय॑न मु॒दय॑न॒म् पूर्वः॒ पूर्व॑ उ॒दय॑नम् । \newline
44. उ॒दय॑न॒ मुत्त॑र॒ उत्त॑र उ॒दय॑न मु॒दय॑न॒ मुत्त॑रः । \newline
45. उ॒दय॑न॒मित्यु॑त् - अय॑नम् । \newline
46. उत्त॑रो॒ ज्योति॑ष्टोमो॒ ज्योति॑ष्टोम॒ उत्त॑र॒ उत्त॑रो॒ ज्योति॑ष्टोमः । \newline
47. उत्त॑र॒ इत्युत् - त॒रः॒ । \newline
48. ज्योति॑ष्टोमो वैश्वान॒रो वै᳚श्वान॒रो ज्योति॑ष्टोमो॒ ज्योति॑ष्टोमो वैश्वान॒रः । \newline
49. ज्योति॑ष्टोम॒ इति॒ ज्योतिः॑ - स्तो॒मः॒ । \newline
50. वै॒श्वा॒न॒रो॑ ऽतिरा॒त्रो॑ ऽतिरा॒त्रो वै᳚श्वान॒रो वै᳚श्वान॒रो॑ ऽतिरा॒त्रः । \newline
51. अ॒ति॒रा॒त्रो भ॑वति भव त्यतिरा॒त्रो॑ ऽतिरा॒त्रो भ॑वति । \newline
52. अ॒ति॒रा॒त्र इत्य॑ति - रा॒त्रः । \newline
53. भ॒व॒ति॒ ज्योति॒र् ज्योति॑र् भवति भवति॒ ज्योतिः॑ । \newline
54. ज्योति॑ रे॒वैव ज्योति॒र् ज्योति॑ रे॒व । \newline
55. ए॒व पु॒रस्ता᳚त् पु॒रस्ता॑ दे॒वैव पु॒रस्ता᳚त् । \newline
56. पु॒रस्ता᳚द् दधते दधते पु॒रस्ता᳚त् पु॒रस्ता᳚द् दधते । \newline
57. द॒ध॒ते॒ सु॒व॒र्गस्य॑ सुव॒र्गस्य॑ दधते दधते सुव॒र्गस्य॑ । \newline
58. सु॒व॒र्गस्य॑ लो॒कस्य॑ लो॒कस्य॑ सुव॒र्गस्य॑ सुव॒र्गस्य॑ लो॒कस्य॑ । \newline
59. सु॒व॒र्गस्येति॑ सुवः - गस्य॑ । \newline
60. लो॒कस्या नु॑ख्यात्या॒ अनु॑ख्यात्यै लो॒कस्य॑ लो॒कस्या नु॑ख्यात्यै । \newline
61. अनु॑ख्यात्यै चतुर्विꣳ॒॒श श्च॑तुर्विꣳ॒॒शो ऽनु॑ख्यात्या॒ अनु॑ख्यात्यै चतुर्विꣳ॒॒शः । \newline
62. अनु॑ख्यात्या॒ इत्यनु॑ - ख्या॒त्यै॒ । \newline
63. च॒तु॒र्विꣳ॒॒शः प्रा॑य॒णीयः॑ प्राय॒णीय॑ श्चतुर्विꣳ॒॒श श्च॑तुर्विꣳ॒॒शः प्रा॑य॒णीयः॑ । \newline
64. च॒तु॒र्विꣳ॒॒श इति॑ चतुः - विꣳ॒॒शः । \newline
65. प्रा॒य॒णीयो॑ भवति भवति प्राय॒णीयः॑ प्राय॒णीयो॑ भवति । \newline
66. प्रा॒य॒णीय॒ इति॑ प्र - अ॒य॒नीयः॑ । \newline
67. भ॒व॒ति॒ चतु॑र्विꣳशति॒ श्चतु॑र्विꣳशतिर् भवति भवति॒ चतु॑र्विꣳशतिः । \newline
68. चतु॑र्विꣳशति रर्द्धमा॒सा अ॑र्द्धमा॒सा श्चतु॑र्विꣳशति॒ श्चतु॑र्विꣳशति रर्द्धमा॒साः । \newline
69. चतु॑र्विꣳशति॒रिति॒ चतुः॑ - विꣳ॒॒श॒तिः॒ । \newline
70. अ॒र्द्ध॒मा॒साः सं॑ॅवथ्स॒रः सं॑ॅवथ्स॒रो᳚ ऽर्द्धमा॒सा अ॑र्द्धमा॒साः सं॑ॅवथ्स॒रः । \newline
71. अ॒र्द्ध॒मा॒सा इत्य॑र्द्ध - मा॒साः । \newline

\textbf{Ghana Paata } \newline

1. वै स॑मु॒द्रः स॑मु॒द्रो वै वै स॑मु॒द्र स्तस्य॒ तस्य॑ समु॒द्रो वै वै स॑मु॒द्र स्तस्य॑ । \newline
2. स॒मु॒द्र स्तस्य॒ तस्य॑ समु॒द्रः स॑मु॒द्र स्तस्यै॒त दे॒तत् तस्य॑ समु॒द्रः स॑मु॒द्र स्तस्यै॒तत् । \newline
3. तस्यै॒त दे॒तत् तस्य॒ तस्यै॒तत् पा॒रम् पा॒र मे॒तत् तस्य॒ तस्यै॒तत् पा॒रम् । \newline
4. ए॒तत् पा॒रम् पा॒र मे॒त दे॒तत् पा॒रं ॅयद् यत् पा॒र मे॒त दे॒तत् पा॒रं ॅयत् । \newline
5. पा॒रं ॅयद् यत् पा॒रम् पा॒रं ॅयद॑तिरा॒त्रा व॑तिरा॒त्रौ यत् पा॒रम् पा॒रं ॅयद॑तिरा॒त्रौ । \newline
6. यद॑तिरा॒त्रा व॑तिरा॒त्रौ यद् यद॑तिरा॒त्रौ ये ये॑ ऽतिरा॒त्रौ यद् यद॑तिरा॒त्रौ ये । \newline
7. अ॒ति॒रा॒त्रौ ये ये॑ ऽतिरा॒त्रा व॑तिरा॒त्रौ य ए॒व मे॒वं ॅये॑ ऽतिरा॒त्रा व॑तिरा॒त्रौ य ए॒वम् । \newline
8. अ॒ति॒रा॒त्रावित्य॑ति - रा॒त्रौ । \newline
9. य ए॒व मे॒वं ॅये य ए॒वं ॅवि॒द्वाꣳसो॑ वि॒द्वाꣳस॑ ए॒वं ॅये य ए॒वं ॅवि॒द्वाꣳसः॑ । \newline
10. ए॒वं ॅवि॒द्वाꣳसो॑ वि॒द्वाꣳस॑ ए॒व मे॒वं ॅवि॒द्वाꣳसः॑ संॅवथ्स॒रꣳ सं॑ॅवथ्स॒रं ॅवि॒द्वाꣳस॑ ए॒व मे॒वं ॅवि॒द्वाꣳसः॑ संॅवथ्स॒रम् । \newline
11. वि॒द्वाꣳसः॑ संॅवथ्स॒रꣳ सं॑ॅवथ्स॒रं ॅवि॒द्वाꣳसो॑ वि॒द्वाꣳसः॑ संॅवथ्स॒र मु॑प॒य न्त्यु॑प॒यन्ति॑ संॅवथ्स॒रं ॅवि॒द्वाꣳसो॑ वि॒द्वाꣳसः॑ संॅवथ्स॒र मु॑प॒यन्ति॑ । \newline
12. सं॒ॅव॒थ्स॒र मु॑प॒य न्त्यु॑प॒यन्ति॑ संॅवथ्स॒रꣳ सं॑ॅवथ्स॒र मु॑प॒य न्त्यना᳚र्ता॒ अना᳚र्ता उप॒यन्ति॑ संॅवथ्स॒रꣳ सं॑ॅवथ्स॒र मु॑प॒य न्त्यना᳚र्ताः । \newline
13. सं॒ॅव॒थ्स॒रमिति॑ सं - व॒थ्स॒रम् । \newline
14. उ॒प॒य न्त्यना᳚र्ता॒ अना᳚र्ता उप॒य न्त्यु॑प॒य न्त्यना᳚र्ता ए॒वैवाना᳚र्ता उप॒य न्त्यु॑प॒य न्त्यना᳚र्ता ए॒व । \newline
15. उ॒प॒यन्तीत्यु॑प - यन्ति॑ । \newline
16. अना᳚र्ता ए॒वै वाना᳚र्ता॒ अना᳚र्ता ए॒वो दृच॑ मु॒दृच॑ मे॒वा ना᳚र्ता॒ अना᳚र्ता ए॒वो दृच᳚म् । \newline
17. ए॒वोदृच॑ मु॒दृच॑ मे॒वैवोदृच॑म् गच्छन्ति गच्छ न्त्यु॒दृच॑ मे॒वैवोदृच॑म् गच्छन्ति । \newline
18. उ॒दृच॑म् गच्छन्ति गच्छ न्त्यु॒दृच॑ मु॒दृच॑म् गच्छन्ती॒य मि॒यम् ग॑च्छ न्त्यु॒दृच॑ मु॒दृच॑म् 
गच्छन्ती॒यम् । \newline
19. उ॒दृच॒मित्यु॑त् - ऋच᳚म् । \newline
20. ग॒च्छ॒न्ती॒य मि॒यम् ग॑च्छन्ति गच्छन्ती॒यं ॅवै वा इ॒यम् ग॑च्छन्ति गच्छन्ती॒यं ॅवै । \newline
21. इ॒यं ॅवै वा इ॒य मि॒यं ॅवै पूर्वः॒ पूर्वो॒ वा इ॒य मि॒यं ॅवै पूर्वः॑ । \newline
22. वै पूर्वः॒ पूर्वो॒ वै वै पूर्वो॑ ऽतिरा॒त्रो॑ ऽतिरा॒त्रः पूर्वो॒ वै वै पूर्वो॑ ऽतिरा॒त्रः । \newline
23. पूर्वो॑ ऽतिरा॒त्रो॑ ऽतिरा॒त्रः पूर्वः॒ पूर्वो॑ ऽतिरा॒त्रो॑ ऽसा व॒सा व॑तिरा॒त्रः पूर्वः॒ पूर्वो॑ ऽतिरा॒त्रो॑ ऽसौ । \newline
24. अ॒ति॒रा॒त्रो॑ ऽसा व॒सा व॑तिरा॒त्रो॑ ऽतिरा॒त्रो॑ ऽसा वुत्त॑र॒ उत्त॑रो॒ ऽसा व॑तिरा॒त्रो॑ ऽतिरा॒त्रो॑ ऽसा वुत्त॑रः । \newline
25. अ॒ति॒रा॒त्र इत्य॑ति - रा॒त्रः । \newline
26. अ॒सा वुत्त॑र॒ उत्त॑रो॒ ऽसा व॒सा वुत्त॑रो॒ मनो॒ मन॒ उत्त॑रो॒ ऽसा व॒सा वुत्त॑रो॒ मनः॑ । \newline
27. उत्त॑रो॒ मनो॒ मन॒ उत्त॑र॒ उत्त॑रो॒ मनः॒ पूर्वः॒ पूर्वो॒ मन॒ उत्त॑र॒ उत्त॑रो॒ मनः॒ पूर्वः॑ । \newline
28. उत्त॑र॒ इत्युत् - त॒रः॒ । \newline
29. मनः॒ पूर्वः॒ पूर्वो॒ मनो॒ मनः॒ पूर्वो॒ वाग् वाक् पूर्वो॒ मनो॒ मनः॒ पूर्वो॒ वाक् । \newline
30. पूर्वो॒ वाग् वाक् पूर्वः॒ पूर्वो॒ वागुत्त॑र॒ उत्त॑रो॒ वाक् पूर्वः॒ पूर्वो॒ वागुत्त॑रः । \newline
31. वागुत्त॑र॒ उत्त॑रो॒ वाग् वागुत्त॑रः प्रा॒णः प्रा॒ण उत्त॑रो॒ वाग् वागुत्त॑रः प्रा॒णः । \newline
32. उत्त॑रः प्रा॒णः प्रा॒ण उत्त॑र॒ उत्त॑रः प्रा॒णः पूर्वः॒ पूर्वः॑ प्रा॒ण उत्त॑र॒ उत्त॑रः प्रा॒णः पूर्वः॑ । \newline
33. उत्त॑र॒ इत्युत् - त॒रः॒ । \newline
34. प्रा॒णः पूर्वः॒ पूर्वः॑ प्रा॒णः प्रा॒णः पूर्वो॑ ऽपा॒नो॑ ऽपा॒नः पूर्वः॑ प्रा॒णः प्रा॒णः पूर्वो॑ ऽपा॒नः । \newline
35. प्रा॒ण इति॑ प्र - अ॒नः । \newline
36. पूर्वो॑ ऽपा॒नो॑ ऽपा॒नः पूर्वः॒ पूर्वो॑ ऽपा॒न उत्त॑र॒ उत्त॑रो ऽपा॒नः पूर्वः॒ पूर्वो॑ ऽपा॒न उत्त॑रः । \newline
37. अ॒पा॒न उत्त॑र॒ उत्त॑रो ऽपा॒नो॑ ऽपा॒न उत्त॑रः प्र॒रोध॑नम् प्र॒रोध॑न॒ मुत्त॑रो ऽपा॒नो॑ ऽपा॒न उत्त॑रः प्र॒रोध॑नम् । \newline
38. अ॒पा॒न इत्य॑प - अ॒नः । \newline
39. उत्त॑रः प्र॒रोध॑नम् प्र॒रोध॑न॒ मुत्त॑र॒ उत्त॑रः प्र॒रोध॑न॒म् पूर्वः॒ पूर्वः॑ प्र॒रोध॑न॒ मुत्त॑र॒ उत्त॑रः प्र॒रोध॑न॒म् पूर्वः॑ । \newline
40. उत्त॑र॒ इत्युत् - त॒रः॒ । \newline
41. प्र॒रोध॑न॒म् पूर्वः॒ पूर्वः॑ प्र॒रोध॑नम् प्र॒रोध॑न॒म् पूर्व॑ उ॒दय॑न मु॒दय॑न॒म् पूर्वः॑ प्र॒रोध॑नम् प्र॒रोध॑न॒म् पूर्व॑ उ॒दय॑नम् । \newline
42. प्र॒रोध॑न॒मिति॑ प्र - रोध॑नम् । \newline
43. पूर्व॑ उ॒दय॑न मु॒दय॑न॒म् पूर्वः॒ पूर्व॑ उ॒दय॑न॒ मुत्त॑र॒ उत्त॑र उ॒दय॑न॒म् पूर्वः॒ पूर्व॑ उ॒दय॑न॒ मुत्त॑रः । \newline
44. उ॒दय॑न॒ मुत्त॑र॒ उत्त॑र उ॒दय॑न मु॒दय॑न॒ मुत्त॑रो॒ ज्योति॑ष्टोमो॒ ज्योति॑ष्टोम॒ उत्त॑र उ॒दय॑न मु॒दय॑न॒ मुत्त॑रो॒ ज्योति॑ष्टोमः । \newline
45. उ॒दय॑न॒मित्यु॑त् - अय॑नम् । \newline
46. उत्त॑रो॒ ज्योति॑ष्टोमो॒ ज्योति॑ष्टोम॒ उत्त॑र॒ उत्त॑रो॒ ज्योति॑ष्टोमो वैश्वान॒रो वै᳚श्वान॒रो ज्योति॑ष्टोम॒ उत्त॑र॒ उत्त॑रो॒ ज्योति॑ष्टोमो वैश्वान॒रः । \newline
47. उत्त॑र॒ इत्युत् - त॒रः॒ । \newline
48. ज्योति॑ष्टोमो वैश्वान॒रो वै᳚श्वान॒रो ज्योति॑ष्टोमो॒ ज्योति॑ष्टोमो वैश्वान॒रो॑ ऽतिरा॒त्रो॑ ऽतिरा॒त्रो वै᳚श्वान॒रो ज्योति॑ष्टोमो॒ ज्योति॑ष्टोमो वैश्वान॒रो॑ ऽतिरा॒त्रः । \newline
49. ज्योति॑ष्टोम॒ इति॒ ज्योतिः॑ - स्तो॒मः॒ । \newline
50. वै॒श्वा॒न॒रो॑ ऽतिरा॒त्रो॑ ऽतिरा॒त्रो वै᳚श्वान॒रो वै᳚श्वान॒रो॑ ऽतिरा॒त्रो भ॑वति भव त्यतिरा॒त्रो वै᳚श्वान॒रो वै᳚श्वान॒रो॑ ऽतिरा॒त्रो भ॑वति । \newline
51. अ॒ति॒रा॒त्रो भ॑वति भव त्यतिरा॒त्रो॑ ऽतिरा॒त्रो भ॑वति॒ ज्योति॒र् ज्योति॑र् भव त्यतिरा॒त्रो॑ ऽतिरा॒त्रो भ॑वति॒ ज्योतिः॑ । \newline
52. अ॒ति॒रा॒त्र इत्य॑ति - रा॒त्रः । \newline
53. भ॒व॒ति॒ ज्योति॒र् ज्योति॑र् भवति भवति॒ ज्योति॑ रे॒वैव ज्योति॑र् भवति भवति॒ ज्योति॑ रे॒व । \newline
54. ज्योति॑ रे॒वैव ज्योति॒र् ज्योति॑ रे॒व पु॒रस्ता᳚त् पु॒रस्ता॑ दे॒व ज्योति॒र् ज्योति॑ रे॒व पु॒रस्ता᳚त् । \newline
55. ए॒व पु॒रस्ता᳚त् पु॒रस्ता॑ दे॒वैव पु॒रस्ता᳚द् दधते दधते पु॒रस्ता॑ दे॒वैव पु॒रस्ता᳚द् दधते । \newline
56. पु॒रस्ता᳚द् दधते दधते पु॒रस्ता᳚त् पु॒रस्ता᳚द् दधते सुव॒र्गस्य॑ सुव॒र्गस्य॑ दधते पु॒रस्ता᳚त् पु॒रस्ता᳚द् दधते सुव॒र्गस्य॑ । \newline
57. द॒ध॒ते॒ सु॒व॒र्गस्य॑ सुव॒र्गस्य॑ दधते दधते सुव॒र्गस्य॑ लो॒कस्य॑ लो॒कस्य॑ सुव॒र्गस्य॑ दधते दधते सुव॒र्गस्य॑ लो॒कस्य॑ । \newline
58. सु॒व॒र्गस्य॑ लो॒कस्य॑ लो॒कस्य॑ सुव॒र्गस्य॑ सुव॒र्गस्य॑ लो॒कस्या नु॑ख्यात्या॒ अनु॑ख्यात्यै लो॒कस्य॑ सुव॒र्गस्य॑ सुव॒र्गस्य॑ लो॒कस्या नु॑ख्यात्यै । \newline
59. सु॒व॒र्गस्येति॑ सुवः - गस्य॑ । \newline
60. लो॒कस्या नु॑ख्यात्या॒ अनु॑ख्यात्यै लो॒कस्य॑ लो॒कस्या नु॑ख्यात्यै चतुर्विꣳ॒॒श श्च॑तुर्विꣳ॒॒शो ऽनु॑ख्यात्यै लो॒कस्य॑ लो॒कस्या नु॑ख्यात्यै चतुर्विꣳ॒॒शः । \newline
61. अनु॑ख्यात्यै चतुर्विꣳ॒॒श श्च॑तुर्विꣳ॒॒शो ऽनु॑ख्यात्या॒ अनु॑ख्यात्यै चतुर्विꣳ॒॒शः प्रा॑य॒णीयः॑ प्राय॒णीय॑ श्चतुर्विꣳ॒॒शो ऽनु॑ख्यात्या॒ अनु॑ख्यात्यै चतुर्विꣳ॒॒शः प्रा॑य॒णीयः॑ । \newline
62. अनु॑ख्यात्या॒ इत्यनु॑ - ख्या॒त्यै॒ । \newline
63. च॒तु॒र्विꣳ॒॒शः प्रा॑य॒णीयः॑ प्राय॒णीय॑ श्चतुर्विꣳ॒॒श श्च॑तुर्विꣳ॒॒शः प्रा॑य॒णीयो॑ भवति भवति प्राय॒णीय॑ श्चतुर्विꣳ॒॒श श्च॑तुर्विꣳ॒॒शः प्रा॑य॒णीयो॑ भवति । \newline
64. च॒तु॒र्विꣳ॒॒श इति॑ चतुः - विꣳ॒॒शः । \newline
65. प्रा॒य॒णीयो॑ भवति भवति प्राय॒णीयः॑ प्राय॒णीयो॑ भवति॒ चतु॑र्विꣳशति॒ श्चतु॑र्विꣳशतिर् भवति प्राय॒णीयः॑ प्राय॒णीयो॑ भवति॒ चतु॑र्विꣳशतिः । \newline
66. प्रा॒य॒णीय॒ इति॑ प्र - अ॒य॒नीयः॑ । \newline
67. भ॒व॒ति॒ चतु॑र्विꣳशति॒ श्चतु॑र्विꣳशतिर् भवति भवति॒ चतु॑र्विꣳशति रर्द्धमा॒सा अ॑र्द्धमा॒सा 
श्चतु॑र्विꣳशतिर् भवति भवति॒ चतु॑र्विꣳशति रर्द्धमा॒साः । \newline
68. चतु॑र्विꣳशति रर्द्धमा॒सा अ॑र्द्धमा॒सा श्चतु॑र्विꣳशति॒ श्चतु॑र्विꣳशति रर्द्धमा॒साः सं॑ॅवथ्स॒रः सं॑ॅवथ्स॒रो᳚ ऽर्द्धमा॒सा श्चतु॑र्विꣳशति॒ श्चतु॑र्विꣳशति रर्द्धमा॒साः सं॑ॅवथ्स॒रः । \newline
69. चतु॑र्विꣳशति॒रिति॒ चतुः॑ - विꣳ॒॒श॒तिः॒ । \newline
70. अ॒र्द्ध॒मा॒साः सं॑ॅवथ्स॒रः सं॑ॅवथ्स॒रो᳚ ऽर्द्धमा॒सा अ॑र्द्धमा॒साः सं॑ॅवथ्स॒रः प्र॒यन्तः॑ प्र॒यन्तः॑ संॅवथ्स॒रो᳚ ऽर्द्धमा॒सा अ॑र्द्धमा॒साः सं॑ॅवथ्स॒रः प्र॒यन्तः॑ । \newline
71. अ॒र्द्ध॒मा॒सा इत्य॑र्द्ध - मा॒साः । \newline
\pagebreak
\markright{ TS 7.5.1.4  \hfill https://www.vedavms.in \hfill}

\section{ TS 7.5.1.4 }

\textbf{TS 7.5.1.4 } \newline
\textbf{Samhita Paata} \newline

सं॑ॅवथ्स॒रः प्र॒यन्त॑ ए॒व सं॑ॅवथ्स॒रे प्रति॑ तिष्ठन्ति॒ तस्य॒ त्रीणि॑ च श॒तानि॑ ष॒ष्टिश्च॑ स्तो॒त्रीया॒स्ताव॑तीः संॅवथ्स॒रस्य॒ रात्र॑य उ॒भे ए॒व सं॑ॅवथ्स॒रस्य॑ रू॒पे आ᳚प्नुवन्ति॒ ते सꣳस्थि॑त्या॒ अरि॑ष्ट्या॒ उत्त॑रै॒रहो॑भिश्चरन्ति षड॒हा भ॑वन्ति॒ षड् वा ऋ॒तवः॑ संॅवथ्स॒र ऋ॒तुष्वे॒व सं॑ॅवथ्स॒रे प्रति॑ तिष्ठन्ति॒ गौश्चाऽऽ*यु॑श्च मद्ध्य॒तः स्तोमौ॑ भवतः संॅवथ्स॒रस्यै॒व तन्मि॑थु॒नं म॑द्ध्य॒तो -[  ] \newline

\textbf{Pada Paata} \newline

सं॒ॅव॒थ्स॒र इति॑ सं - व॒थ्स॒रः । प्र॒यन्त॒ इति॑ प्र - यन्तः॑ । ए॒व । सं॒ॅव॒थ्स॒र इति॑ सं - व॒थ्स॒रे । प्रतीति॑ । ति॒ष्ठ॒न्ति॒ । तस्य॑ । त्रीणि॑ । च॒ । श॒तानि॑ । ष॒ष्टिः । च॒ । स्तो॒त्रीयाः᳚ । ताव॑तीः । सं॒ॅव॒थ्स॒रस्येति॑ सं - व॒थ्स॒रस्य॑ । रात्र॑यः । उ॒भे इति॑ । ए॒व । सं॒ॅव॒थ्स॒रस्येति॑ सं - व॒थ्स॒रस्य॑ । रू॒पे इति॑ । आ॒प्नु॒व॒न्ति॒ । ते । सꣳस्थि॑त्या॒ इति॒ सं-स्थि॒त्यै॒ । अरि॑ष्ट्यै । उत्त॑रै॒रित्युत् - त॒रैः॒ । अहो॑भि॒रित्यहः॑-भिः॒ । च॒र॒न्ति॒ । ष॒ड॒हा इति॑ षट् - अ॒हाः । भ॒व॒न्ति॒ । षट् । वै । ऋ॒तवः॑ । सं॒ॅव॒थ्स॒र इति॑ सं - व॒थ्स॒रः । ऋ॒तुषु॑ । ए॒व । सं॒ॅव॒थ्स॒र इति॑ सं-व॒थ्स॒रे । प्रतीति॑ । ति॒ष्ठ॒न्ति॒ । गौः । च॒ । आयुः॑ । च॒ । म॒द्ध्य॒तः । स्तोमौ᳚ । भ॒व॒तः॒ । सं॒ॅव॒थ्स॒रस्येति॑ सं - व॒थ्स॒रस्य॑ । ए॒व । तत् । मि॒थु॒नम् । म॒द्ध्य॒तः ।  \newline


\textbf{Krama Paata} \newline

स॒म्ॅव॒थ्स॒रः प्र॒यन्तः॑ । स॒म्ॅव॒थ्स॒र इति॑ सम् - व॒थ्स॒रः । प्र॒यन्त॑ ए॒व । प्र॒यन्त॒ इति॑ प्र - यन्तः॑ । ए॒व स॑म्ॅवथ्स॒रे । स॒म्ॅव॒थ्स॒रे प्रति॑ । स॒म्ॅव॒थ्स॒र इति॑ सम् - व॒थ्स॒रे । प्रति॑ तिष्ठन्ति । ति॒ष्ठ॒न्ति॒ तस्य॑ । तस्य॒ त्रीणि॑ । त्रीणि॑ च । च॒ श॒तानि॑ । श॒तानि॑ ष॒ष्टिः । ष॒ष्टिश्च॑ । च॒ स्तो॒त्रीयाः᳚ । स्तो॒त्रीया॒स्ताव॑तीः । ताव॑तीः सम्ॅवथ्स॒रस्य॑ । स॒म्ॅव॒थ्स॒रस्य॒ रात्र॑यः । स॒म्ॅव॒थ्स॒रस्येति॑ सम् - व॒थ्स॒रस्य॑ । रात्र॑य उ॒भे । उ॒भे ए॒व । उ॒भे इत्यु॒भे । ए॒व स॑म्ॅवथ्स॒रस्य॑ । स॒म्ॅव॒थ्स॒रस्य॑ रू॒पे । स॒म्ॅव॒थ्स॒रस्येति॑ सम् - व॒थ्स॒रस्य॑ । रू॒पे आ᳚प्नुवन्ति । रू॒पे इति॑ रू॒पे । आ॒प्नु॒व॒न्ति॒ ते । ते सꣳस्थि॑त्यै । सꣳस्थि॑त्या॒ अरि॑ष्ट्‍यै । सꣳस्थि॑त्या॒ इति॒ सम् - स्थि॒त्यै॒ । अरि॑ष्ट्‍या॒ उत्त॑रैः । उत्त॑रै॒रहो॑भिः । उत्त॑रै॒रित्युत् - त॒रैः॒ । अहो॑भिश्चरन्ति । अहो॑भि॒रित्यहः॑ - भिः॒ । च॒र॒न्ति॒ ष॒ड॒हाः । ष॒ड॒हा भ॑वन्ति । ष॒ड॒हा इति॑ षट् - अ॒हाः । भ॒व॒न्ति॒ षट् । षड् वै । वा ऋ॒तवः॑ । ऋ॒तवः॑ सम्ॅवथ्स॒रः । स॒म्ॅव॒थ्स॒र ऋ॒तुषु॑ । स॒म्ॅव॒थ्स॒र इति॑ सम् - व॒थ्स॒रः । ऋ॒तुष्वे॒व । ए॒व स॑म्ॅवथ्स॒रे । स॒म्ॅव॒थ्स॒रे प्रति॑ । स॒म्ॅव॒थ्स॒र इति॑ सम् - व॒थ्स॒रे । प्रति॑ तिष्ठन्ति । ति॒ष्ठ॒न्ति॒ गौः । गौश्च॑ । चायुः॑ । आयु॑श्च । च॒ म॒द्ध्य॒तः । म॒द्ध्य॒तः स्तोमौ᳚ । स्तोमौ॑ भवतः । भ॒व॒तः॒ स॒म्ॅव॒थ्स॒रस्य॑ । स॒म्ॅव॒थ्स॒रस्यै॒व । स॒म्ॅव॒थ्स॒रस्येति॑ सम् - व॒थ्स॒रस्य॑ । ए॒व तत् । तन् मि॑थु॒नम् । मि॒थु॒नम् म॑द्ध्य॒तः । म॒द्ध्य॒तो द॑धति \newline

\textbf{Jatai Paata} \newline

1. सं॒ॅव॒थ्स॒रः प्र॒यन्तः॑ प्र॒यन्तः॑ संॅवथ्स॒रः सं॑ॅवथ्स॒रः प्र॒यन्तः॑ । \newline
2. सं॒ॅव॒थ्स॒र इति॑ सं - व॒थ्स॒रः । \newline
3. प्र॒यन्त॑ ए॒वैव प्र॒यन्तः॑ प्र॒यन्त॑ ए॒व । \newline
4. प्र॒यन्त॒ इति॑ प्र - यन्तः॑ । \newline
5. ए॒व सं॑ॅवथ्स॒रे सं॑ॅवथ्स॒र ए॒वैव सं॑ॅवथ्स॒रे । \newline
6. सं॒ॅव॒थ्स॒रे प्रति॒ प्रति॑ संॅवथ्स॒रे सं॑ॅवथ्स॒रे प्रति॑ । \newline
7. सं॒ॅव॒थ्स॒र इति॑ सं - व॒थ्स॒रे । \newline
8. प्रति॑ तिष्ठन्ति तिष्ठन्ति॒ प्रति॒ प्रति॑ तिष्ठन्ति । \newline
9. ति॒ष्ठ॒न्ति॒ तस्य॒ तस्य॑ तिष्ठन्ति तिष्ठन्ति॒ तस्य॑ । \newline
10. तस्य॒ त्रीणि॒ त्रीणि॒ तस्य॒ तस्य॒ त्रीणि॑ । \newline
11. त्रीणि॑ च च॒ त्रीणि॒ त्रीणि॑ च । \newline
12. च॒ श॒तानि॑ श॒तानि॑ च च श॒तानि॑ । \newline
13. श॒तानि॑ ष॒ष्टि ष्ष॒ष्टिः श॒तानि॑ श॒तानि॑ ष॒ष्टिः । \newline
14. ष॒ष्टिश्च॑ च ष॒ष्टि ष्ष॒ष्टिश्च॑ । \newline
15. च॒ स्तो॒त्रीयाः᳚ स्तो॒त्रीया᳚ श्च च स्तो॒त्रीयाः᳚ । \newline
16. स्तो॒त्रीया॒ स्ताव॑ती॒ स्ताव॑तीः स्तो॒त्रीयाः᳚ स्तो॒त्रीया॒ स्ताव॑तीः । \newline
17. ताव॑तीः संॅवथ्स॒रस्य॑ संॅवथ्स॒रस्य॒ ताव॑ती॒ स्ताव॑तीः संॅवथ्स॒रस्य॑ । \newline
18. सं॒ॅव॒थ्स॒रस्य॒ रात्र॑यो॒ रात्र॑यः संॅवथ्स॒रस्य॑ संॅवथ्स॒रस्य॒ रात्र॑यः । \newline
19. सं॒ॅव॒थ्स॒रस्येति॑ सं - व॒थ्स॒रस्य॑ । \newline
20. रात्र॑य उ॒भे उ॒भे रात्र॑यो॒ रात्र॑य उ॒भे । \newline
21. उ॒भे ए॒वैवोभे उ॒भे ए॒व । \newline
22. उ॒भे इत्यु॒भे । \newline
23. ए॒व सं॑ॅवथ्स॒रस्य॑ संॅवथ्स॒रस्यै॒ वैव सं॑ॅवथ्स॒रस्य॑ । \newline
24. सं॒ॅव॒थ्स॒रस्य॑ रू॒पे रू॒पे सं॑ॅवथ्स॒रस्य॑ संॅवथ्स॒रस्य॑ रू॒पे । \newline
25. सं॒ॅव॒थ्स॒रस्येति॑ सं - व॒थ्स॒रस्य॑ । \newline
26. रू॒पे आ᳚प्नुव न्त्याप्नुवन्ति रू॒पे रू॒पे आ᳚प्नुवन्ति । \newline
27. रू॒पे इति॑ रू॒पे । \newline
28. आ॒प्नु॒व॒न्ति॒ ते त आ᳚प्नुव न्त्याप्नुवन्ति॒ ते । \newline
29. ते सꣳस्थि॑त्यै॒ सꣳस्थि॑त्यै॒ ते ते सꣳस्थि॑त्यै । \newline
30. सꣳस्थि॑त्या॒ अरि॑ष्ट्या॒ अरि॑ष्ट्यै॒ सꣳस्थि॑त्यै॒ सꣳस्थि॑त्या॒ अरि॑ष्ट्यै । \newline
31. सꣳस्थि॑त्या॒ इति॒ सं - स्थि॒त्यै॒ । \newline
32. अरि॑ष्ट्या॒ उत्त॑रै॒ रुत्त॑रै॒ ररि॑ष्ट्या॒ अरि॑ष्ट्या॒ उत्त॑रैः । \newline
33. उत्त॑रै॒ रहो॑भि॒ रहो॑भि॒ रुत्त॑रै॒ रुत्त॑रै॒ रहो॑भिः । \newline
34. उत्त॑रै॒रित्युत् - त॒रैः॒ । \newline
35. अहो॑भि श्चरन्ति चर॒ न्त्यहो॑भि॒ रहो॑भि श्चरन्ति । \newline
36. अहो॑भि॒रित्यहः॑ - भिः॒ । \newline
37. च॒र॒न्ति॒ ष॒ड॒हा ष्ष॑ड॒हा श्च॑रन्ति चरन्ति षड॒हाः । \newline
38. ष॒ड॒हा भ॑वन्ति भवन्ति षड॒हा ष्ष॑ड॒हा भ॑वन्ति । \newline
39. ष॒ड॒हा इति॑ षट् - अ॒हाः । \newline
40. भ॒व॒न्ति॒ षट् थ्षड् भ॑वन्ति भवन्ति॒ षट् । \newline
41. षड् वै वै षट् थ्षड् वै । \newline
42. वा ऋ॒तव॑ ऋ॒तवो॒ वै वा ऋ॒तवः॑ । \newline
43. ऋ॒तवः॑ संॅवथ्स॒रः सं॑ॅवथ्स॒र ऋ॒तव॑ ऋ॒तवः॑ संॅवथ्स॒रः । \newline
44. सं॒ॅव॒थ्स॒र ऋ॒तुष् वृ॒तुषु॑ संॅवथ्स॒रः सं॑ॅवथ्स॒र ऋ॒तुषु॑ । \newline
45. सं॒ॅव॒थ्स॒र इति॑ सं - व॒थ्स॒रः । \newline
46. ऋ॒तु ष्वे॒वैव र्‌तुष् वृ॒तु ष्वे॒व । \newline
47. ए॒व सं॑ॅवथ्स॒रे सं॑ॅवथ्स॒र ए॒वैव सं॑ॅवथ्स॒रे । \newline
48. सं॒ॅव॒थ्स॒रे प्रति॒ प्रति॑ संॅवथ्स॒रे सं॑ॅवथ्स॒रे प्रति॑ । \newline
49. सं॒ॅव॒थ्स॒र इति॑ सं - व॒थ्स॒रे । \newline
50. प्रति॑ तिष्ठन्ति तिष्ठन्ति॒ प्रति॒ प्रति॑ तिष्ठन्ति । \newline
51. ति॒ष्ठ॒न्ति॒ गौर् गौ स्ति॑ष्ठन्ति तिष्ठन्ति॒ गौः । \newline
52. गौ श्च॑ च॒ गौर् गौ श्च॑ । \newline
53. चायु॒ रायु॑ श्च॒ चायुः॑ । \newline
54. आयु॑ श्च॒ चायु॒ रायु॑ श्च । \newline
55. च॒ म॒द्ध्य॒तो म॑द्ध्य॒त श्च॑ च मद्ध्य॒तः । \newline
56. म॒द्ध्य॒तः स्तोमौ॒ स्तोमौ॑ मद्ध्य॒तो म॑द्ध्य॒तः स्तोमौ᳚ । \newline
57. स्तोमौ॑ भवतो भवतः॒ स्तोमौ॒ स्तोमौ॑ भवतः । \newline
58. भ॒व॒तः॒ सं॒ॅव॒थ्स॒रस्य॑ संॅवथ्स॒रस्य॑ भवतो भवतः संॅवथ्स॒रस्य॑ । \newline
59. सं॒ॅव॒थ्स॒र स्यै॒वैव सं॑ॅवथ्स॒रस्य॑ संॅवथ्स॒र स्यै॒व । \newline
60. सं॒ॅव॒थ्स॒रस्येति॑ सं - व॒थ्स॒रस्य॑ । \newline
61. ए॒व तत् तदे॒ वैव तत् । \newline
62. तन् मि॑थु॒नम् मि॑थु॒नम् तत् तन् मि॑थु॒नम् । \newline
63. मि॒थु॒नम् म॑द्ध्य॒तो म॑द्ध्य॒तो मि॑थु॒नम् मि॑थु॒नम् म॑द्ध्य॒तः । \newline
64. म॒द्ध्य॒तो द॑धति दधति मद्ध्य॒तो म॑द्ध्य॒तो द॑धति । \newline

\textbf{Ghana Paata } \newline

1. सं॒ॅव॒थ्स॒रः प्र॒यन्तः॑ प्र॒यन्तः॑ संॅवथ्स॒रः सं॑ॅवथ्स॒रः प्र॒यन्त॑ ए॒वैव प्र॒यन्तः॑ संॅवथ्स॒रः सं॑ॅवथ्स॒रः प्र॒यन्त॑ ए॒व । \newline
2. सं॒ॅव॒थ्स॒र इति॑ सं - व॒थ्स॒रः । \newline
3. प्र॒यन्त॑ ए॒वैव प्र॒यन्तः॑ प्र॒यन्त॑ ए॒व सं॑ॅवथ्स॒रे सं॑ॅवथ्स॒र ए॒व प्र॒यन्तः॑ प्र॒यन्त॑ ए॒व सं॑ॅवथ्स॒रे । \newline
4. प्र॒यन्त॒ इति॑ प्र - यन्तः॑ । \newline
5. ए॒व सं॑ॅवथ्स॒रे सं॑ॅवथ्स॒र ए॒वैव सं॑ॅवथ्स॒रे प्रति॒ प्रति॑ संॅवथ्स॒र ए॒वैव सं॑ॅवथ्स॒रे प्रति॑ । \newline
6. सं॒ॅव॒थ्स॒रे प्रति॒ प्रति॑ संॅवथ्स॒रे सं॑ॅवथ्स॒रे प्रति॑ तिष्ठन्ति तिष्ठन्ति॒ प्रति॑ संॅवथ्स॒रे सं॑ॅवथ्स॒रे प्रति॑ तिष्ठन्ति । \newline
7. सं॒ॅव॒थ्स॒र इति॑ सं - व॒थ्स॒रे । \newline
8. प्रति॑ तिष्ठन्ति तिष्ठन्ति॒ प्रति॒ प्रति॑ तिष्ठन्ति॒ तस्य॒ तस्य॑ तिष्ठन्ति॒ प्रति॒ प्रति॑ तिष्ठन्ति॒ तस्य॑ । \newline
9. ति॒ष्ठ॒न्ति॒ तस्य॒ तस्य॑ तिष्ठन्ति तिष्ठन्ति॒ तस्य॒ त्रीणि॒ त्रीणि॒ तस्य॑ तिष्ठन्ति तिष्ठन्ति॒ तस्य॒ त्रीणि॑ । \newline
10. तस्य॒ त्रीणि॒ त्रीणि॒ तस्य॒ तस्य॒ त्रीणि॑ च च॒ त्रीणि॒ तस्य॒ तस्य॒ त्रीणि॑ च । \newline
11. त्रीणि॑ च च॒ त्रीणि॒ त्रीणि॑ च श॒तानि॑ श॒तानि॑ च॒ त्रीणि॒ त्रीणि॑ च श॒तानि॑ । \newline
12. च॒ श॒तानि॑ श॒तानि॑ च च श॒तानि॑ ष॒ष्टि ष्ष॒ष्टिः श॒तानि॑ च च श॒तानि॑ ष॒ष्टिः । \newline
13. श॒तानि॑ ष॒ष्टि ष्ष॒ष्टिः श॒तानि॑ श॒तानि॑ ष॒ष्टि श्च॑ च ष॒ष्टिः श॒तानि॑ श॒तानि॑ ष॒ष्टि श्च॑ । \newline
14. ष॒ष्टि श्च॑ च ष॒ष्टि ष्ष॒ष्टि श्च॑ स्तो॒त्रीयाः᳚ स्तो॒त्रीया᳚ श्च ष॒ष्टि ष्ष॒ष्टि श्च॑ स्तो॒त्रीयाः᳚ । \newline
15. च॒ स्तो॒त्रीयाः᳚ स्तो॒त्रीया᳚ श्च च स्तो॒त्रीया॒ स्ताव॑ती॒ स्ताव॑तीः स्तो॒त्रीया᳚ श्च च स्तो॒त्रीया॒ स्ताव॑तीः । \newline
16. स्तो॒त्रीया॒ स्ताव॑ती॒ स्ताव॑तीः स्तो॒त्रीयाः᳚ स्तो॒त्रीया॒ स्ताव॑तीः संॅवथ्स॒रस्य॑ संॅवथ्स॒रस्य॒ ताव॑तीः स्तो॒त्रीयाः᳚ स्तो॒त्रीया॒ स्ताव॑तीः संॅवथ्स॒रस्य॑ । \newline
17. ताव॑तीः संॅवथ्स॒रस्य॑ संॅवथ्स॒रस्य॒ ताव॑ती॒ स्ताव॑तीः संॅवथ्स॒रस्य॒ रात्र॑यो॒ रात्र॑यः संॅवथ्स॒रस्य॒ ताव॑ती॒ स्ताव॑तीः संॅवथ्स॒रस्य॒ रात्र॑यः । \newline
18. सं॒ॅव॒थ्स॒रस्य॒ रात्र॑यो॒ रात्र॑यः संॅवथ्स॒रस्य॑ संॅवथ्स॒रस्य॒ रात्र॑य उ॒भे उ॒भे रात्र॑यः संॅवथ्स॒रस्य॑ संॅवथ्स॒रस्य॒ रात्र॑य उ॒भे । \newline
19. सं॒ॅव॒थ्स॒रस्येति॑ सं - व॒थ्स॒रस्य॑ । \newline
20. रात्र॑य उ॒भे उ॒भे रात्र॑यो॒ रात्र॑य उ॒भे ए॒वैवोभे रात्र॑यो॒ रात्र॑य उ॒भे ए॒व । \newline
21. उ॒भे ए॒वैवोभे उ॒भे ए॒व सं॑ॅवथ्स॒रस्य॑ संॅवथ्स॒र स्यै॒वोभे उ॒भे ए॒व सं॑ॅवथ्स॒रस्य॑ । \newline
22. उ॒भे इत्यु॒भे । \newline
23. ए॒व सं॑ॅवथ्स॒रस्य॑ संॅवथ्स॒र स्यै॒वैव सं॑ॅवथ्स॒रस्य॑ रू॒पे रू॒पे सं॑ॅवथ्स॒र स्यै॒वैव सं॑ॅवथ्स॒रस्य॑ रू॒पे । \newline
24. सं॒ॅव॒थ्स॒रस्य॑ रू॒पे रू॒पे सं॑ॅवथ्स॒रस्य॑ संॅवथ्स॒रस्य॑ रू॒पे आ᳚प्नुव न्त्याप्नुवन्ति रू॒पे सं॑ॅवथ्स॒रस्य॑ संॅवथ्स॒रस्य॑ रू॒पे आ᳚प्नुवन्ति । \newline
25. सं॒ॅव॒थ्स॒रस्येति॑ सं - व॒थ्स॒रस्य॑ । \newline
26. रू॒पे आ᳚प्नुव न्त्याप्नुवन्ति रू॒पे रू॒पे आ᳚प्नुवन्ति॒ ते त आ᳚प्नुवन्ति रू॒पे रू॒पे आ᳚प्नुवन्ति॒ ते । \newline
27. रू॒पे इति॑ रू॒पे । \newline
28. आ॒प्नु॒व॒न्ति॒ ते त आ᳚प्नुव न्त्याप्नुवन्ति॒ ते सꣳस्थि॑त्यै॒ सꣳस्थि॑त्यै॒ त आ᳚प्नुव न्त्याप्नुवन्ति॒ ते सꣳस्थि॑त्यै । \newline
29. ते सꣳस्थि॑त्यै॒ सꣳस्थि॑त्यै॒ ते ते सꣳस्थि॑त्या॒ अरि॑ष्ट्या॒ अरि॑ष्ट्यै॒ सꣳस्थि॑त्यै॒ ते ते सꣳस्थि॑त्या॒ अरि॑ष्ट्यै । \newline
30. सꣳस्थि॑त्या॒ अरि॑ष्ट्या॒ अरि॑ष्ट्यै॒ सꣳस्थि॑त्यै॒ सꣳस्थि॑त्या॒ अरि॑ष्ट्या॒ उत्त॑रै॒ रुत्त॑रै॒ ररि॑ष्ट्यै॒ सꣳस्थि॑त्यै॒ सꣳस्थि॑त्या॒ अरि॑ष्ट्या॒ उत्त॑रैः । \newline
31. सꣳस्थि॑त्या॒ इति॒ सं - स्थि॒त्यै॒ । \newline
32. अरि॑ष्ट्या॒ उत्त॑रै॒ रुत्त॑रै॒ ररि॑ष्ट्या॒ अरि॑ष्ट्या॒ उत्त॑रै॒ रहो॑भि॒ रहो॑भि॒ रुत्त॑रै॒ ररि॑ष्ट्या॒ अरि॑ष्ट्या॒ उत्त॑रै॒ रहो॑भिः । \newline
33. उत्त॑रै॒ रहो॑भि॒ रहो॑भि॒ रुत्त॑रै॒ रुत्त॑रै॒ रहो॑भि श्चरन्ति चर॒ न्त्यहो॑भि॒ रुत्त॑रै॒ रुत्त॑रै॒ रहो॑भि श्चरन्ति । \newline
34. उत्त॑रै॒रित्युत् - त॒रैः॒ । \newline
35. अहो॑भि श्चरन्ति चर॒ न्त्यहो॑भि॒ रहो॑भि श्चरन्ति षड॒हा ष्ष॑ड॒हा श्च॑र॒ न्त्यहो॑भि॒ रहो॑भि श्चरन्ति षड॒हाः । \newline
36. अहो॑भि॒रित्यहः॑ - भिः॒ । \newline
37. च॒र॒न्ति॒ ष॒ड॒हा ष्ष॑ड॒हा श्च॑रन्ति चरन्ति षड॒हा भ॑वन्ति भवन्ति षड॒हा श्च॑रन्ति चरन्ति षड॒हा भ॑वन्ति । \newline
38. ष॒ड॒हा भ॑वन्ति भवन्ति षड॒हा ष्ष॑ड॒हा भ॑वन्ति॒ षट् थ्षड् भ॑वन्ति षड॒हा ष्ष॑ड॒हा भ॑वन्ति॒ षट् । \newline
39. ष॒ड॒हा इति॑ षट् - अ॒हाः । \newline
40. भ॒व॒न्ति॒ षट् थ्षड् भ॑वन्ति भवन्ति॒ षड् वै वै षड् भ॑वन्ति भवन्ति॒ षड् वै । \newline
41. षड् वै वै षट् थ्षड् वा ऋ॒तव॑ ऋ॒तवो॒ वै षट् थ्षड् वा ऋ॒तवः॑ । \newline
42. वा ऋ॒तव॑ ऋ॒तवो॒ वै वा ऋ॒तवः॑ संॅवथ्स॒रः सं॑ॅवथ्स॒र ऋ॒तवो॒ वै वा ऋ॒तवः॑ संॅवथ्स॒रः । \newline
43. ऋ॒तवः॑ संॅवथ्स॒रः सं॑ॅवथ्स॒र ऋ॒तव॑ ऋ॒तवः॑ संॅवथ्स॒र ऋ॒तुष् वृ॒तुषु॑ संॅवथ्स॒र ऋ॒तव॑ ऋ॒तवः॑ संॅवथ्स॒र ऋ॒तुषु॑ । \newline
44. सं॒ॅव॒थ्स॒र ऋ॒तुष् वृ॒तुषु॑ संॅवथ्स॒रः सं॑ॅवथ्स॒र ऋ॒तु ष्वे॒वैव र्‌तुषु॑ संॅवथ्स॒रः सं॑ॅवथ्स॒र ऋ॒तुष्वे॒व । \newline
45. सं॒ॅव॒थ्स॒र इति॑ सं - व॒थ्स॒रः । \newline
46. ऋ॒तु ष्वे॒वैव र्‌तुष् वृ॒तुष्वे॒व सं॑ॅवथ्स॒रे सं॑ॅवथ्स॒र ए॒व र्‌तुष् वृ॒तुष्वे॒व सं॑ॅवथ्स॒रे । \newline
47. ए॒व सं॑ॅवथ्स॒रे सं॑ॅवथ्स॒र ए॒वैव सं॑ॅवथ्स॒रे प्रति॒ प्रति॑ संॅवथ्स॒र ए॒वैव सं॑ॅवथ्स॒रे प्रति॑ । \newline
48. सं॒ॅव॒थ्स॒रे प्रति॒ प्रति॑ संॅवथ्स॒रे सं॑ॅवथ्स॒रे प्रति॑ तिष्ठन्ति तिष्ठन्ति॒ प्रति॑ संॅवथ्स॒रे सं॑ॅवथ्स॒रे प्रति॑ तिष्ठन्ति । \newline
49. सं॒ॅव॒थ्स॒र इति॑ सं - व॒थ्स॒रे । \newline
50. प्रति॑ तिष्ठन्ति तिष्ठन्ति॒ प्रति॒ प्रति॑ तिष्ठन्ति॒ गौर् गौ स्ति॑ष्ठन्ति॒ प्रति॒ प्रति॑ तिष्ठन्ति॒ गौः । \newline
51. ति॒ष्ठ॒न्ति॒ गौर् गौ स्ति॑ष्ठन्ति तिष्ठन्ति॒ गौश्च॑ च॒ गौ स्ति॑ष्ठन्ति तिष्ठन्ति॒ गौश्च॑ । \newline
52. गौ श्च॑ च॒ गौर् गौ श्चायु॒ रायु॑ श्च॒ गौर् गौ श्चायुः॑ । \newline
53. चायु॒ रायु॑ श्च॒ चायु॑ श्च॒ चायु॑ श्च॒ चायु॑ श्च । \newline
54. आयु॑ श्च॒ चायु॒ रायु॑ श्च मद्ध्य॒तो म॑द्ध्य॒त श्चायु॒ रायु॑ श्च मद्ध्य॒तः । \newline
55. च॒ म॒द्ध्य॒तो म॑द्ध्य॒त श्च॑ च मद्ध्य॒तः स्तोमौ॒ स्तोमौ॑ मद्ध्य॒त श्च॑ च मद्ध्य॒तः स्तोमौ᳚ । \newline
56. म॒द्ध्य॒तः स्तोमौ॒ स्तोमौ॑ मद्ध्य॒तो म॑द्ध्य॒तः स्तोमौ॑ भवतो भवतः॒ स्तोमौ॑ मद्ध्य॒तो म॑द्ध्य॒तः स्तोमौ॑ भवतः । \newline
57. स्तोमौ॑ भवतो भवतः॒ स्तोमौ॒ स्तोमौ॑ भवतः संॅवथ्स॒रस्य॑ संॅवथ्स॒रस्य॑ भवतः॒ स्तोमौ॒ स्तोमौ॑ भवतः संॅवथ्स॒रस्य॑ । \newline
58. भ॒व॒तः॒ सं॒ॅव॒थ्स॒रस्य॑ संॅवथ्स॒रस्य॑ भवतो भवतः संॅवथ्स॒र स्यै॒वैव सं॑ॅवथ्स॒रस्य॑ भवतो भवतः संॅवथ्स॒र स्यै॒व । \newline
59. सं॒ॅव॒थ्स॒र स्यै॒वैव सं॑ॅवथ्स॒रस्य॑ संॅवथ्स॒र स्यै॒व तत् तदे॒व सं॑ॅवथ्स॒रस्य॑ संॅवथ्स॒र स्यै॒व तत् । \newline
60. सं॒ॅव॒थ्स॒रस्येति॑ सं - व॒थ्स॒रस्य॑ । \newline
61. ए॒व तत् तदे॒वैव तन् मि॑थु॒नम् मि॑थु॒नम् तदे॒वैव तन् मि॑थु॒नम् । \newline
62. तन् मि॑थु॒नम् मि॑थु॒नम् तत् तन् मि॑थु॒नम् म॑द्ध्य॒तो म॑द्ध्य॒तो मि॑थु॒नम् तत् तन् मि॑थु॒नम् म॑द्ध्य॒तः । \newline
63. मि॒थु॒नम् म॑द्ध्य॒तो म॑द्ध्य॒तो मि॑थु॒नम् मि॑थु॒नम् म॑द्ध्य॒तो द॑धति दधति मद्ध्य॒तो मि॑थु॒नम् मि॑थु॒नम् म॑द्ध्य॒तो द॑धति । \newline
64. म॒द्ध्य॒तो द॑धति दधति मद्ध्य॒तो म॑द्ध्य॒तो द॑धति प्र॒जन॑नाय प्र॒जन॑नाय दधति मद्ध्य॒तो म॑द्ध्य॒तो द॑धति प्र॒जन॑नाय । \newline
\pagebreak
\markright{ TS 7.5.1.5  \hfill https://www.vedavms.in \hfill}

\section{ TS 7.5.1.5 }

\textbf{TS 7.5.1.5 } \newline
\textbf{Samhita Paata} \newline

द॑धति प्र॒जन॑नाय॒ ज्योति॑र॒भितो॑ भवति वि॒मोच॑नमे॒व तच्छन्दाꣳ॑स्ये॒व तद्-वि॒मोकं॑ ॅय॒न्त्यथो॑ उभ॒यतो᳚ज्योतिषै॒व ष॑ड॒हेन॑ सुव॒र्गं ॅलो॒कं ॅय॑न्ति ब्रह्मवा॒दिनो॑ वद॒न्त्यास॑ते॒ केन॑ य॒न्तीति॑ देव॒याने॑न प॒थेति॑ ब्रूया॒च्छन्दाꣳ॑सि॒ वै दे॑व॒यानः॒ पन्था॑ गाय॒त्री त्रि॒ष्टुब्-जग॑ती॒ज्योति॒र्वै गा॑य॒त्री गौस्त्रि॒ष्टुगायु॒र्जग॑ती॒ यदे॒ते स्तोमा॒ भव॑न्ति देव॒याने॑नै॒व - [  ] \newline

\textbf{Pada Paata} \newline

द॒ध॒ति॒ । प्र॒जन॑ना॒येति॑ प्र - जन॑नाय । ज्योतिः॑ । अ॒भितः॑ । भ॒व॒ति॒ । वि॒मोच॑न॒मिति॑ वि - मोच॑नम् । ए॒व । तत् । छन्दाꣳ॑सि । ए॒व । तत् । वि॒मोक॒मिति॑ वि - मोक᳚म् । य॒न्ति॒ । अथो॒ इति॑ । उ॒भ॒यतो᳚ज्योति॒षेत्यु॑भ॒यतः॑-ज्यो॒ति॒षा॒ । ए॒व । ष॒ड॒हेनेति॑ षट्-अ॒हेन॑ । सु॒व॒र्गमिति॑ सुवः - गम् । लो॒कम् । य॒न्ति॒ । ब्र॒ह्म॒वा॒दिन॒ इति॑ ब्रह्म - वा॒दिनः॑ । व॒द॒न्ति॒ । आस॑ते । केन॑ । य॒न्ति॒ । इति॑ । दे॒व॒याने॒नेति॑ देव - याने॑न । प॒था । इति॑ । ब्रू॒या॒त् । छन्दाꣳ॑सि । वै । दे॒व॒यान॒ इति॑ देव - यानः॑ । पन्थाः᳚ । गा॒य॒त्री । त्रि॒ष्टुप् । जग॑ती । ज्योतिः॑ । वै । गा॒य॒त्री । गौः । त्रि॒ष्टुक् । आयुः॑ । जग॑ती । यत् । ए॒ते । स्तोमाः᳚ । भव॑न्ति । द॒व॒याने॒नेति॑ देव-याने॑न । ए॒व ।  \newline


\textbf{Krama Paata} \newline

द॒ध॒ति॒ प्र॒जन॑नाय । प्र॒जन॑नाय॒ ज्योतिः॑ । प्र॒जन॑ना॒येति॑ प्र - जन॑नाय । ज्योति॑र॒भितः॑ । अ॒भितो॑ भवति । भ॒व॒ति॒ वि॒मोच॑नम् । वि॒मोच॑नमे॒व । वि॒मोच॑न॒मिति॑ वि - मोच॑नम् । ए॒व तत् । तच् छन्दाꣳ॑सि । छन्दाꣳ॑स्ये॒व । ए॒व तत् । तद् वि॒मोक᳚म् । वि॒मोक॑म् ॅयन्ति । वि॒मोक॒मिति॑ वि - मोक᳚म् । य॒न्त्यथो᳚ । अथो॑ उभ॒यतो᳚ज्योतिषा । अथो॒ इत्यथो᳚ । उ॒भ॒यतो᳚ज्योतिषै॒व । उ॒भ॒यतो᳚ज्योति॒षेत्यु॑भ॒यतः॑ - ज्यो॒ति॒षा॒ । ए॒व ष॑ड॒हेन॑ । ष॒ड॒हेन॑ सुव॒र्गम् । ष॒ड॒हेनेति॑ षट् - अ॒हेन॑ । सु॒व॒र्गम् ॅलो॒कम् । सु॒व॒र्गमिति॑ सुवः - गम् । लो॒कम् ॅय॑न्ति । य॒न्ति॒ ब्र॒ह्म॒वा॒दिनः॑ । ब्र॒ह्म॒वा॒दिनो॑ वदन्ति । ब्र॒ह्म॒वा॒दिन॒ इति॑ ब्रह्म - वा॒दिनः॑ । व॒द॒न्त्यास॑ते । आस॑ते॒ केन॑ । केन॑ यन्ति । य॒न्तीति॑ । इति॑ देव॒याने॑न । दे॒व॒याने॑न प॒था । दे॒व॒याने॒नेति॑ देव - याने॑न । प॒थेति॑ । इति॑ ब्रूयात् । ब्रू॒या॒च् छन्दाꣳ॑सि । छन्दाꣳ॑सि॒ वै । वै दे॑व॒यानः॑ । दे॒व॒यानः॒ पन्थाः᳚ । दे॒व॒यान॒ इति॑ देव - यानः॑ । पन्था॑ गाय॒त्री । गा॒य॒त्री त्रि॒ष्टुप् । त्रि॒ष्टुब् जग॑ती । जग॑ती॒ ज्योतिः॑ । ज्योति॒र् वै । वै गा॑य॒त्री । गा॒य॒त्री गौः । गौस्त्रि॒ष्टुक् । त्रि॒ष्टुगायुः॑ । आयु॒र् जग॑ती । जग॑ती॒ यत् । यदे॒ते । ए॒ते स्तोमाः᳚ । स्तोमा॒ भव॑न्ति । भव॑न्ति देव॒याने॑न । दे॒व॒याने॑नै॒व । दे॒व॒याने॒नेति॑ देव - याने॑न । ए॒व तत् \newline

\textbf{Jatai Paata} \newline

1. द॒ध॒ति॒ प्र॒जन॑नाय प्र॒जन॑नाय दधति दधति प्र॒जन॑नाय । \newline
2. प्र॒जन॑नाय॒ ज्योति॒र् ज्योतिः॑ प्र॒जन॑नाय प्र॒जन॑नाय॒ ज्योतिः॑ । \newline
3. प्र॒जन॑ना॒येति॑ प्र - जन॑नाय । \newline
4. ज्योति॑ र॒भितो॒ ऽभितो॒ ज्योति॒र् ज्योति॑ र॒भितः॑ । \newline
5. अ॒भितो॑ भवति भव त्य॒भितो॒ ऽभितो॑ भवति । \newline
6. भ॒व॒ति॒ वि॒मोच॑नं ॅवि॒मोच॑नम् भवति भवति वि॒मोच॑नम् । \newline
7. वि॒मोच॑न मे॒वैव वि॒मोच॑नं ॅवि॒मोच॑न मे॒व । \newline
8. वि॒मोच॑न॒मिति॑ वि - मोच॑नम् । \newline
9. ए॒व तत् तदे॒वैव तत् । \newline
10. तच् छन्दाꣳ॑सि॒ छन्दाꣳ॑सि॒ तत् तच् छन्दाꣳ॑सि । \newline
11. छन्दाꣳ॑ स्ये॒वैव छन्दाꣳ॑सि॒ छन्दाꣳ॑ स्ये॒व । \newline
12. ए॒व तत् तदे॒वैव तत् । \newline
13. तद् वि॒मोकं॑ ॅवि॒मोक॒म् तत् तद् वि॒मोक᳚म् । \newline
14. वि॒मोकं॑ ॅयन्ति यन्ति वि॒मोकं॑ ॅवि॒मोकं॑ ॅयन्ति । \newline
15. वि॒मोक॒मिति॑ वि - मोक᳚म् । \newline
16. य॒न्त्यथो॒ अथो॑ यन्ति य॒न्त्यथो᳚ । \newline
17. अथो॑ उभ॒यतो᳚ज्योति षोभ॒यतो᳚ज्योति॒षा ऽथो॒ अथो॑ उभ॒यतो᳚ज्योतिषा । \newline
18. अथो॒ इत्यथो᳚ । \newline
19. उ॒भ॒यतो᳚ज्योति षै॒वै वोभ॒यतो᳚ज्योति षोभ॒यतो᳚ज्योति षै॒व । \newline
20. उ॒भ॒यतो᳚ज्योति॒षेत्यु॑भ॒यतः॑ - ज्यो॒ति॒षा॒ । \newline
21. ए॒व ष॑ड॒हेन॑ षड॒हे नै॒वैव ष॑ड॒हेन॑ । \newline
22. ष॒ड॒हेन॑ सुव॒र्गꣳ सु॑व॒र्गꣳ ष॑ड॒हेन॑ षड॒हेन॑ सुव॒र्गम् । \newline
23. ष॒ड॒हेनेति॑ षट् - अ॒हेन॑ । \newline
24. सु॒व॒र्गम् ॅलो॒कम् ॅलो॒कꣳ सु॑व॒र्गꣳ सु॑व॒र्गम् ॅलो॒कम् । \newline
25. सु॒व॒र्गमिति॑ सुवः - गम् । \newline
26. लो॒कं ॅय॑न्ति यन्ति लो॒कम् ॅलो॒कं ॅय॑न्ति । \newline
27. य॒न्ति॒ ब्र॒ह्म॒वा॒दिनो᳚ ब्रह्मवा॒दिनो॑ यन्ति यन्ति ब्रह्मवा॒दिनः॑ । \newline
28. ब्र॒ह्म॒वा॒दिनो॑ वदन्ति वदन्ति ब्रह्मवा॒दिनो᳚ ब्रह्मवा॒दिनो॑ वदन्ति । \newline
29. ब्र॒ह्म॒वा॒दिन॒ इति॑ ब्रह्म - वा॒दिनः॑ । \newline
30. व॒द॒ न्त्यास॑त॒ आस॑ते वदन्ति वद॒ न्त्यास॑ते । \newline
31. आस॑ते॒ केन॒ केनास॑त॒ आस॑ते॒ केन॑ । \newline
32. केन॑ यन्ति यन्ति॒ केन॒ केन॑ यन्ति । \newline
33. य॒न्तीतीति॑ यन्ति य॒न्तीति॑ । \newline
34. इति॑ देव॒याने॑न देव॒याने॒ नेतीति॑ देव॒याने॑न । \newline
35. दे॒व॒याने॑न प॒था प॒था दे॑व॒याने॑न देव॒याने॑न प॒था । \newline
36. दे॒व॒याने॒नेति॑ देव - याने॑न । \newline
37. प॒थेतीति॑ प॒था प॒थेति॑ । \newline
38. इति॑ ब्रूयाद् ब्रूया॒दितीति॑ ब्रूयात् । \newline
39. ब्रू॒या॒च् छन्दाꣳ॑सि॒ छन्दाꣳ॑सि ब्रूयाद् ब्रूया॒च् छन्दाꣳ॑सि । \newline
40. छन्दाꣳ॑सि॒ वै वै छन्दाꣳ॑सि॒ छन्दाꣳ॑सि॒ वै । \newline
41. वै दे॑व॒यानो॑ देव॒यानो॒ वै वै दे॑व॒यानः॑ । \newline
42. दे॒व॒यानः॒ पन्थाः॒ पन्था॑ देव॒यानो॑ देव॒यानः॒ पन्थाः᳚ । \newline
43. दे॒व॒यान॒ इति॑ देव - यानः॑ । \newline
44. पन्था॑ गाय॒त्री गा॑य॒त्री पन्थाः॒ पन्था॑ गाय॒त्री । \newline
45. गा॒य॒त्री त्रि॒ष्टुप् त्रि॒ष्टुब् गा॑य॒त्री गा॑य॒त्री त्रि॒ष्टुप् । \newline
46. त्रि॒ष्टुब् जग॑ती॒ जग॑ती त्रि॒ष्टुप् त्रि॒ष्टुब् जग॑ती । \newline
47. जग॑ती॒ ज्योति॒र् ज्योति॒र् जग॑ती॒ जग॑ती॒ ज्योतिः॑ । \newline
48. ज्योति॒र् वै वै ज्योति॒र् ज्योति॒र् वै । \newline
49. वै गा॑य॒त्री गा॑य॒त्री वै वै गा॑य॒त्री । \newline
50. गा॒य॒त्री गौर् गौर् गा॑य॒त्री गा॑य॒त्री गौः । \newline
51. गौ स्त्रि॒ष्टुक् त्रि॒ष्टुग् गौर् गौ स्त्रि॒ष्टुक् । \newline
52. त्रि॒ष्टु गायु॒ रायु॑ष् ट्रि॒ष्टुक् त्रि॒ष्टु गायुः॑ । \newline
53. आयु॒र् जग॑ती॒ जग॒ त्यायु॒ रायु॒र् जग॑ती । \newline
54. जग॑ती॒ यद् यज् जग॑ती॒ जग॑ती॒ यत् । \newline
55. यदे॒त ए॒ते यद् यदे॒ते । \newline
56. ए॒ते स्तोमाः॒ स्तोमा॑ ए॒त ए॒ते स्तोमाः᳚ । \newline
57. स्तोमा॒ भव॑न्ति॒ भव॑न्ति॒ स्तोमाः॒ स्तोमा॒ भव॑न्ति । \newline
58. भव॑न्ति दव॒याने॑न दव॒याने॑न॒ भव॑न्ति॒ भव॑न्ति दव॒याने॑न । \newline
59. द॒व॒याने॑ नै॒वैव द॑व॒याने॑न दव॒याने॑ नै॒व । \newline
60. द॒व॒याने॒नेति॑ देव - याने॑न । \newline
61. ए॒व तत् तदे॒वैव तत् । \newline

\textbf{Ghana Paata } \newline

1. द॒ध॒ति॒ प्र॒जन॑नाय प्र॒जन॑नाय दधति दधति प्र॒जन॑नाय॒ ज्योति॒र् ज्योतिः॑ प्र॒जन॑नाय दधति दधति प्र॒जन॑नाय॒ ज्योतिः॑ । \newline
2. प्र॒जन॑नाय॒ ज्योति॒र् ज्योतिः॑ प्र॒जन॑नाय प्र॒जन॑नाय॒ ज्योति॑ र॒भितो॒ ऽभितो॒ ज्योतिः॑ प्र॒जन॑नाय प्र॒जन॑नाय॒ ज्योति॑ र॒भितः॑ । \newline
3. प्र॒जन॑ना॒येति॑ प्र - जन॑नाय । \newline
4. ज्योति॑ र॒भितो॒ ऽभितो॒ ज्योति॒र् ज्योति॑ र॒भितो॑ भवति भव त्य॒भितो॒ ज्योति॒र् ज्योति॑ र॒भितो॑ भवति । \newline
5. अ॒भितो॑ भवति भव त्य॒भितो॒ ऽभितो॑ भवति वि॒मोच॑नं ॅवि॒मोच॑नम् भव त्य॒भितो॒ ऽभितो॑ भवति वि॒मोच॑नम् । \newline
6. भ॒व॒ति॒ वि॒मोच॑नं ॅवि॒मोच॑नम् भवति भवति वि॒मोच॑न मे॒वैव वि॒मोच॑नम् भवति भवति वि॒मोच॑न मे॒व । \newline
7. वि॒मोच॑न मे॒वैव वि॒मोच॑नं ॅवि॒मोच॑न मे॒व तत् तदे॒व वि॒मोच॑नं ॅवि॒मोच॑न मे॒व तत् । \newline
8. वि॒मोच॑न॒मिति॑ वि - मोच॑नम् । \newline
9. ए॒व तत् तदे॒वैव तच् छन्दाꣳ॑सि॒ छन्दाꣳ॑सि॒ तदे॒वैव तच् छन्दाꣳ॑सि । \newline
10. तच् छन्दाꣳ॑सि॒ छन्दाꣳ॑सि॒ तत् तच् छन्दाꣳ॑ स्ये॒वैव छन्दाꣳ॑सि॒ तत् तच् छन्दाꣳ॑ स्ये॒व । \newline
11. छन्दाꣳ॑ स्ये॒वैव छन्दाꣳ॑सि॒ छन्दाꣳ॑ स्ये॒व तत् तदे॒व छन्दाꣳ॑सि॒ छन्दाꣳ॑ स्ये॒व तत् । \newline
12. ए॒व तत् तदे॒वैव तद् वि॒मोकं॑ ॅवि॒मोक॒म् तदे॒वैव तद् वि॒मोक᳚म् । \newline
13. तद् वि॒मोकं॑ ॅवि॒मोक॒म् तत् तद् वि॒मोकं॑ ॅयन्ति यन्ति वि॒मोक॒म् तत् तद् वि॒मोकं॑ ॅयन्ति । \newline
14. वि॒मोकं॑ ॅयन्ति यन्ति वि॒मोकं॑ ॅवि॒मोकं॑ ॅय॒न्त्यथो॒ अथो॑ यन्ति वि॒मोकं॑ ॅवि॒मोकं॑ ॅय॒न्त्यथो᳚ । \newline
15. वि॒मोक॒मिति॑ वि - मोक᳚म् । \newline
16. य॒न्त्यथो॒ अथो॑ यन्ति य॒न्त्यथो॑ उभ॒यतो᳚ज्योति षोभ॒यतो᳚ज्योति॒षा ऽथो॑ यन्ति य॒न्त्यथो॑ उभ॒यतो᳚ज्योतिषा । \newline
17. अथो॑ उभ॒यतो᳚ज्योति षोभ॒यतो᳚ज्योति॒षा ऽथो॒ अथो॑ उभ॒यतो᳚ज्योति षै॒वै वोभ॒यतो᳚ज्योति॒षा ऽथो॒ अथो॑ उभ॒यतो᳚ज्योतिषै॒व । \newline
18. अथो॒ इत्यथो᳚ । \newline
19. उ॒भ॒यतो᳚ज्योति षै॒वै वोभ॒यतो᳚ज्योति षोभ॒यतो᳚ज्योति षै॒व ष॑ड॒हेन॑ षड॒हे नै॒वोभ॒यतो᳚ज्योति
षोभ॒यतो᳚ज्योति षै॒व ष॑ड॒हेन॑ । \newline
20. उ॒भ॒यतो᳚ज्योति॒षेत्यु॑भ॒यतः॑ - ज्यो॒ति॒षा॒ । \newline
21. ए॒व ष॑ड॒हेन॑ षड॒हे नै॒वैव ष॑ड॒हेन॑ सुव॒र्गꣳ सु॑व॒र्गꣳ ष॑ड॒हे नै॒वैव ष॑ड॒हेन॑ सुव॒र्गम् । \newline
22. ष॒ड॒हेन॑ सुव॒र्गꣳ सु॑व॒र्गꣳ ष॑ड॒हेन॑ षड॒हेन॑ सुव॒र्गम् ॅलो॒कम् ॅलो॒कꣳ सु॑व॒र्गꣳ ष॑ड॒हेन॑ षड॒हेन॑ सुव॒र्गम् ॅलो॒कम् । \newline
23. ष॒ड॒हेनेति॑ षट् - अ॒हेन॑ । \newline
24. सु॒व॒र्गम् ॅलो॒कम् ॅलो॒कꣳ सु॑व॒र्गꣳ सु॑व॒र्गम् ॅलो॒कं ॅय॑न्ति यन्ति लो॒कꣳ सु॑व॒र्गꣳ सु॑व॒र्गम् ॅलो॒कं ॅय॑न्ति । \newline
25. सु॒व॒र्गमिति॑ सुवः - गम् । \newline
26. लो॒कं ॅय॑न्ति यन्ति लो॒कम् ॅलो॒कं ॅय॑न्ति ब्रह्मवा॒दिनो᳚ ब्रह्मवा॒दिनो॑ यन्ति लो॒कम् ॅलो॒कं ॅय॑न्ति ब्रह्मवा॒दिनः॑ । \newline
27. य॒न्ति॒ ब्र॒ह्म॒वा॒दिनो᳚ ब्रह्मवा॒दिनो॑ यन्ति यन्ति ब्रह्मवा॒दिनो॑ वदन्ति वदन्ति ब्रह्मवा॒दिनो॑ यन्ति यन्ति ब्रह्मवा॒दिनो॑ वदन्ति । \newline
28. ब्र॒ह्म॒वा॒दिनो॑ वदन्ति वदन्ति ब्रह्मवा॒दिनो᳚ ब्रह्मवा॒दिनो॑ वद॒ न्त्यास॑त॒ आस॑ते वदन्ति ब्रह्मवा॒दिनो᳚ ब्रह्मवा॒दिनो॑ वद॒ न्त्यास॑ते । \newline
29. ब्र॒ह्म॒वा॒दिन॒ इति॑ ब्रह्म - वा॒दिनः॑ । \newline
30. व॒द॒ न्त्यास॑त॒ आस॑ते वदन्ति वद॒ न्त्यास॑ते॒ केन॒ केनास॑ते वदन्ति वद॒ न्त्यास॑ते॒ केन॑ । \newline
31. आस॑ते॒ केन॒ केना स॑त॒ आस॑ते॒ केन॑ यन्ति यन्ति॒ केना स॑त॒ आस॑ते॒ केन॑ यन्ति । \newline
32. केन॑ यन्ति यन्ति॒ केन॒ केन॑ य॒न्तीतीति॑ यन्ति॒ केन॒ केन॑ य॒न्तीति॑ । \newline
33. य॒न्तीतीति॑ यन्ति य॒न्तीति॑ देव॒याने॑न देव॒याने॒ नेति॑ यन्ति य॒न्तीति॑ देव॒याने॑न । \newline
34. इति॑ देव॒याने॑न देव॒याने॒ नेतीति॑ देव॒याने॑न प॒था प॒था दे॑व॒याने॒ नेतीति॑ देव॒याने॑न प॒था । \newline
35. दे॒व॒याने॑न प॒था प॒था दे॑व॒याने॑न देव॒याने॑न प॒थेतीति॑ प॒था दे॑व॒याने॑न देव॒याने॑न प॒थेति॑ । \newline
36. दे॒व॒याने॒नेति॑ देव - याने॑न । \newline
37. प॒थेतीति॑ प॒था प॒थेति॑ ब्रूयाद् ब्रूया॒ दिति॑ प॒था प॒थेति॑ ब्रूयात् । \newline
38. इति॑ ब्रूयाद् ब्रूया॒दितीति॑ ब्रूया॒च् छन्दाꣳ॑सि॒ छन्दाꣳ॑सि ब्रूया॒दितीति॑ ब्रूया॒च् छन्दाꣳ॑सि । \newline
39. ब्रू॒या॒च् छन्दाꣳ॑सि॒ छन्दाꣳ॑सि ब्रूयाद् ब्रूया॒च् छन्दाꣳ॑सि॒ वै वै छन्दाꣳ॑सि ब्रूयाद् ब्रूया॒च् छन्दाꣳ॑सि॒ वै । \newline
40. छन्दाꣳ॑सि॒ वै वै छन्दाꣳ॑सि॒ छन्दाꣳ॑सि॒ वै दे॑व॒यानो॑ देव॒यानो॒ वै छन्दाꣳ॑सि॒ छन्दाꣳ॑सि॒ वै दे॑व॒यानः॑ । \newline
41. वै दे॑व॒यानो॑ देव॒यानो॒ वै वै दे॑व॒यानः॒ पन्थाः॒ पन्था॑ देव॒यानो॒ वै वै दे॑व॒यानः॒ पन्थाः᳚ । \newline
42. दे॒व॒यानः॒ पन्थाः॒ पन्था॑ देव॒यानो॑ देव॒यानः॒ पन्था॑ गाय॒त्री गा॑य॒त्री पन्था॑ देव॒यानो॑ देव॒यानः॒ पन्था॑ गाय॒त्री । \newline
43. दे॒व॒यान॒ इति॑ देव - यानः॑ । \newline
44. पन्था॑ गाय॒त्री गा॑य॒त्री पन्थाः॒ पन्था॑ गाय॒त्री त्रि॒ष्टुप् त्रि॒ष्टुब् गा॑य॒त्री पन्थाः॒ पन्था॑ गाय॒त्री त्रि॒ष्टुप् । \newline
45. गा॒य॒त्री त्रि॒ष्टुप् त्रि॒ष्टुब् गा॑य॒त्री गा॑य॒त्री त्रि॒ष्टुब् जग॑ती॒ जग॑ती त्रि॒ष्टुब् गा॑य॒त्री गा॑य॒त्री त्रि॒ष्टुब् जग॑ती । \newline
46. त्रि॒ष्टुब् जग॑ती॒ जग॑ती त्रि॒ष्टुप् त्रि॒ष्टुब् जग॑ती॒ ज्योति॒र् ज्योति॒र् जग॑ती त्रि॒ष्टुप् त्रि॒ष्टुब् जग॑ती॒ ज्योतिः॑ । \newline
47. जग॑ती॒ ज्योति॒र् ज्योति॒र् जग॑ती॒ जग॑ती॒ ज्योति॒र् वै वै ज्योति॒र् जग॑ती॒ जग॑ती॒ ज्योति॒र् वै । \newline
48. ज्योति॒र् वै वै ज्योति॒र् ज्योति॒र् वै गा॑य॒त्री गा॑य॒त्री वै ज्योति॒र् ज्योति॒र् वै गा॑य॒त्री । \newline
49. वै गा॑य॒त्री गा॑य॒त्री वै वै गा॑य॒त्री गौर् गौर् गा॑य॒त्री वै वै गा॑य॒त्री गौः । \newline
50. गा॒य॒त्री गौर् गौर् गा॑य॒त्री गा॑य॒त्री गौ स्त्रि॒ष्टुक् त्रि॒ष्टुग् गौर् गा॑य॒त्री गा॑य॒त्री गौ स्त्रि॒ष्टुक् । \newline
51. गौ स्त्रि॒ष्टुक् त्रि॒ष्टुग् गौर् गौ स्त्रि॒ष्टु गायु॒ रायु॑ष् ट्रि॒ष्टुग् गौर् गौ स्त्रि॒ष्टु गायुः॑ । \newline
52. त्रि॒ष्टु गायु॒ रायु॑ष् ट्रि॒ष्टुक् त्रि॒ष्टु गायु॒र् जग॑ती॒ जग॒ त्यायु॑ष् ट्रि॒ष्टुक् त्रि॒ष्टु गायु॒र् जग॑ती । \newline
53. आयु॒र् जग॑ती॒ जग॒ त्यायु॒ रायु॒र् जग॑ती॒ यद् यज् जग॒ त्यायु॒ रायु॒र् जग॑ती॒ यत् । \newline
54. जग॑ती॒ यद् यज् जग॑ती॒ जग॑ती॒ यदे॒त ए॒ते यज् जग॑ती॒ जग॑ती॒ यदे॒ते । \newline
55. यदे॒त ए॒ते यद् यदे॒ते स्तोमाः॒ स्तोमा॑ ए॒ते यद् यदे॒ते स्तोमाः᳚ । \newline
56. ए॒ते स्तोमाः॒ स्तोमा॑ ए॒त ए॒ते स्तोमा॒ भव॑न्ति॒ भव॑न्ति॒ स्तोमा॑ ए॒त ए॒ते स्तोमा॒ भव॑न्ति । \newline
57. स्तोमा॒ भव॑न्ति॒ भव॑न्ति॒ स्तोमाः॒ स्तोमा॒ भव॑न्ति दव॒याने॑न दव॒याने॑न॒ भव॑न्ति॒ स्तोमाः॒ स्तोमा॒ भव॑न्ति दव॒याने॑न । \newline
58. भव॑न्ति दव॒याने॑न दव॒याने॑न॒ भव॑न्ति॒ भव॑न्ति दव॒याने॑ नै॒वैव द॑व॒याने॑न॒ भव॑न्ति॒ भव॑न्ति दव॒याने॑ नै॒व । \newline
59. द॒व॒याने॑ नै॒वैव द॑व॒याने॑न दव॒याने॑ नै॒व तत् तदे॒व द॑व॒याने॑न दव॒याने॑ नै॒व तत् । \newline
60. द॒व॒याने॒नेति॑ देव - याने॑न । \newline
61. ए॒व तत् तदे॒वैव तत् प॒था प॒था तदे॒वैव तत् प॒था । \newline
\pagebreak
\markright{ TS 7.5.1.6  \hfill https://www.vedavms.in \hfill}

\section{ TS 7.5.1.6 }

\textbf{TS 7.5.1.6 } \newline
\textbf{Samhita Paata} \newline

तत् प॒था य॑न्ति समा॒नꣳ साम॑ भवति देवलो॒को वै साम॑ देवलो॒कादे॒व नय॑न्त्य॒न्याअ॑न्या॒ ऋचो॑ भवन्ति मनुष्यलो॒को वा ऋचो॑ मनुष्यलो॒कादे॒वान्यम॑न्यं देवलो॒कम॑भ्या॒रोह॑न्तो यन्त्यभिव॒र्तो ब्र॑ह्मसा॒मं भ॑वति सुव॒र्गस्य॑ लो॒कस्या॒भिवृ॑त्या अभि॒जिद्-भ॑वति सुव॒र्गस्य॑ लो॒कस्या॒भिजि॑त्यै विश्व॒जिद्-भ॑वति॒ विश्व॑स्य॒ जित्यै॑ मा॒सिमा॑सि पृ॒ष्ठान्युप॑ यन्ति मा॒सिमा᳚स्यतिग्रा॒ह्या॑ गृह्यन्ते मा॒सिमा᳚स्ये॒व वी॒र्यं॑ ( ) दधति मा॒सां प्रति॑ष्ठित्या उ॒परि॑ष्टान्मा॒सां पृ॒ष्ठान्युप॑ यन्ति॒ तस्मा॑दु॒परि॑ष्टा॒दोष॑धयः॒ फलं॑ गृह्णन्ति ॥ \newline

\textbf{Pada Paata} \newline

तत् । प॒था । य॒न्ति॒ । स॒मा॒नम् । साम॑ । भ॒व॒ति॒ । दे॒व॒लो॒क इति॑ देव - लो॒कः । वै । साम॑ । दे॒व॒लो॒कादिति॑ देव - लो॒कात् । ए॒व । न । य॒न्ति॒ । अ॒न्या‌अ॑न्या॒ इत्य॒न्याः - अ॒न्याः॒ । ऋचः॑ । भ॒व॒न्ति॒ । म॒नु॒ष्य॒लो॒क इति॑ मनुष्य - लो॒कः । वै । ऋचः॑ । म॒नु॒ष्य॒लो॒कादिति॑ मनुष्य - लो॒कात् । ए॒व । अ॒न्यम॑न्य॒मित्य॒न्यम् - अ॒न्य॒म् । दे॒व॒लो॒कमिति॑ देव - लो॒कम् । अ॒भ्या॒रोह॑न्त॒ इत्य॑भि-आ॒रोह॑न्तः । य॒न्ति॒ । अ॒भि॒व॒र्त इत्य॑भि  -  व॒र्तः । ब्र॒ह्म॒सा॒ममिति॑ ब्रह्म - सा॒मम् । भ॒व॒ति॒ । सु॒व॒र्गस्येति॑ सुवः - गस्य॑ । लो॒कस्य॑ । अ॒भिवृ॑त्या॒ इत्य॒भि - वृ॒त्यै॒ । अ॒भि॒जिदित्य॑भि - जित् । भ॒व॒ति॒ । सु॒व॒र्गस्येति॑ सुवः - गस्य॑ । लो॒कस्य॑ । अ॒भिजि॑त्या॒ इत्य॒भि - जि॒त्यै॒ । वि॒श्व॒जिदिति॑ विश्व - जित् । भ॒व॒ति॒ । विश्व॑स्य । जित्यै᳚ । मा॒सिमा॒सीति॑ मा॒सि - मा॒सि॒ । पृ॒ष्ठानि॑ । उपेति॑ । य॒न्ति॒ । मा॒सिमा॒सीति॑ मा॒सि - मा॒सि॒ । अ॒ति॒ग्रा॒ह्या॑ इत्य॑ति - ग्रा॒ह्याः᳚ । गृ॒ह्य॒न्ते॒ । मा॒सिमा॒सीति॑ मा॒सि - मा॒सि॒ । ए॒व । वी॒र्य᳚म् ( ) । द॒ध॒ति॒ । मा॒साम् । प्रति॑ष्ठित्या॒ इति॒ प्रति॑ - स्थि॒त्यै॒ । उ॒परि॑ष्टात् । मा॒साम् । पृ॒ष्ठानि॑ । उपेति॑ । य॒न्ति॒ । तस्मा᳚त् । उ॒परि॑ष्टात् । ओष॑धयः । फल᳚म् । गृ॒ह्ण॒न्ति॒ ॥  \newline


\textbf{Krama Paata} \newline

तत् प॒था । प॒था य॑न्ति । य॒न्ति॒ स॒मा॒नम् । स॒मा॒नꣳ साम॑ । साम॑ भवति । भ॒व॒ति॒ दे॒व॒लो॒कः । दे॒व॒लो॒को वै । दे॒व॒लो॒क इति॑ देव - लो॒कः । वै साम॑ । साम॑ देवलो॒कात् । दे॒व॒लो॒कादे॒व । दे॒व॒लो॒कादिति॑ देव - लो॒कात् । ए॒व न । न य॑न्ति । य॒न्त्य॒न्याअ॑न्याः । अ॒न्याअ॑न्या॒ ऋचः॑ । अ॒न्याअ॑न्या॒ इत्य॒न्याः - अ॒न्याः॒ । ऋचो॑ भवन्ति । भ॒व॒न्ति॒ म॒नु॒ष्य॒लो॒कः । म॒नु॒ष्य॒लो॒को वै । म॒नु॒ष्य॒लो॒क इति॑ मनुष्य - लो॒कः । वा ऋचः॑ । ऋचो॑ मनुष्यलो॒कात् । म॒नु॒ष्य॒लो॒कादे॒व । म॒नु॒ष्य॒लो॒कादिति॑ मनुष्य - लो॒कात् । ए॒वान्यम॑न्यम् । अ॒न्यम॑न्यम् देवलो॒कम् । अ॒न्यम॑न्य॒मित्य॒न्यम् - अ॒न्य॒म् । दे॒व॒लो॒कम॑भ्या॒रोह॑न्तः । दे॒व॒लो॒कमिति॑ देव - लो॒कम् । अ॒भ्या॒रोह॑न्तो यन्ति । अ॒भ्या॒रोह॑न्त॒ इत्य॑भि - आ॒रोह॑न्तः । य॒न्त्य॒भि॒व॒र्तः । अ॒भि॒व॒र्तो ब्र॑ह्मसा॒मम् । अ॒भि॒व॒र्त इत्य॑भि - व॒र्तः । ब्र॒ह्म॒सा॒मम् भ॑वति । ब्र॒ह्म॒सा॒ममिति॑ ब्रह्म - सा॒मम् । भ॒व॒ति॒ सु॒व॒र्गस्य॑ । सु॒व॒र्गस्य॑ लो॒कस्य॑ । सु॒व॒र्गस्येति॑ सुवः - गस्य॑ । लो॒कस्या॒भिवृ॑त्यै । अ॒भिवृ॑त्या अभि॒जित् । अ॒भिवृ॑त्या॒ इत्य॒भि - वृ॒त्यै॒ । अ॒भि॒जिद् भ॑वति । अ॒भि॒जिदित्य॑भि - जित् । भ॒व॒ति॒ सु॒व॒र्गस्य॑ । सु॒व॒र्गस्य॑ लो॒कस्य॑ । सु॒व॒र्गस्येति॑ सुवः - गस्य॑ । लो॒कस्या॒भिजि॑त्यै । अ॒भिजि॑त्यै विश्व॒जित् । अ॒भिजि॑त्या॒ इत्य॒भि - जि॒त्यै॒ । वि॒श्व॒जिद् भ॑वति । वि॒श्व॒जिदिति॑ विश्व - जित् । भ॒व॒ति॒ विश्व॑स्य । विश्व॑स्य॒ जित्यै᳚ । जित्यै॑ मा॒सिमा॑सि । मा॒सिमा॑सि पृ॒ष्ठानि॑ । मा॒सिमा॒सीति॑ मा॒सि - मा॒सि॒ । पृ॒ष्ठान्युप॑ । उप॑ यन्ति । य॒न्ति॒ मा॒सिमा॑सि । मा॒सिमा᳚स्यतिग्रा॒ह्याः᳚ । मा॒सिमा॒सीति॑ मा॒सि - मा॒सि॒ । अ॒ति॒ग्रा॒ह्या॑ गृह्यन्ते । अ॒ति॒ग्रा॒ह्या॑ इत्य॑ति - ग्रा॒ह्याः᳚ । गृ॒ह्य॒न्ते॒ मा॒सिमा॑सि । मा॒सिमा᳚स्ये॒व । मा॒सिमा॒सीति॑ मा॒सि - मा॒सि॒ । ए॒व वी॒र्य᳚म् ( ) । वी॒र्य॑म् दधति । द॒ध॒ति॒ मा॒साम् । मा॒साम् प्रति॑ष्ठित्यै । प्रति॑ष्ठित्या उ॒परि॑ष्टात् । प्रति॑ष्ठित्या॒ इति॒ प्रति॑ - स्थि॒त्यै॒ । उ॒परि॑ष्टान् मा॒साम् । मा॒साम् पृ॒ष्ठानि॑ । पृ॒ष्ठान्युप॑ । उप॑ यन्ति । य॒न्ति॒ तस्मा᳚त् । तस्मा॑दु॒परि॑ष्टात् । उ॒परि॑ष्टा॒दोष॑धयः । ओष॑धयः॒ फल᳚म् । फल॑म् गृह्णन्ति । गृ॒ह्ण॒न्तीति॑ गृह्णन्ति । \newline

\textbf{Jatai Paata} \newline

1. तत् प॒था प॒था तत् तत् प॒था । \newline
2. प॒था य॑न्ति यन्ति प॒था प॒था य॑न्ति । \newline
3. य॒न्ति॒ स॒मा॒नꣳ स॑मा॒नं ॅय॑न्ति यन्ति समा॒नम् । \newline
4. स॒मा॒नꣳ साम॒ साम॑ समा॒नꣳ स॑मा॒नꣳ साम॑ । \newline
5. साम॑ भवति भवति॒ साम॒ साम॑ भवति । \newline
6. भ॒व॒ति॒ दे॒व॒लो॒को दे॑वलो॒को भ॑वति भवति देवलो॒कः । \newline
7. दे॒व॒लो॒को वै वै दे॑वलो॒को दे॑वलो॒को वै । \newline
8. दे॒व॒लो॒क इति॑ देव - लो॒कः । \newline
9. वै साम॒ साम॒ वै वै साम॑ । \newline
10. साम॑ देवलो॒काद् दे॑वलो॒काथ् साम॒ साम॑ देवलो॒कात् । \newline
11. दे॒व॒लो॒का दे॒वैव दे॑वलो॒काद् दे॑वलो॒का दे॒व । \newline
12. दे॒व॒लो॒कादिति॑ देव - लो॒कात् । \newline
13. ए॒व न नैवैव न । \newline
14. न य॑न्ति यन्ति॒ न न य॑न्ति । \newline
15. य॒न्त्य॒न्याअ॑न्या अ॒न्याअ॑न्या यन्ति यन्त्य॒न्याअ॑न्याः । \newline
16. अ॒न्याअ॑न्या॒ ऋच॒ ऋचो॒ ऽन्याअ॑न्या अ॒न्याअ॑न्या॒ ऋचः॑ । \newline
17. अ॒न्याअ॑न्या॒ इत्य॒न्याः - अ॒न्याः॒ । \newline
18. ऋचो॑ भवन्ति भव॒ न्त्यृच॒ ऋचो॑ भवन्ति । \newline
19. भ॒व॒न्ति॒ म॒नु॒ष्य॒लो॒को म॑नुष्यलो॒को भ॑वन्ति भवन्ति मनुष्यलो॒कः । \newline
20. म॒नु॒ष्य॒लो॒को वै वै म॑नुष्यलो॒को म॑नुष्यलो॒को वै । \newline
21. म॒नु॒ष्य॒लो॒क इति॑ मनुष्य - लो॒कः । \newline
22. वा ऋच॒ ऋचो॒ वै वा ऋचः॑ । \newline
23. ऋचो॑ मनुष्यलो॒कान् म॑नुष्यलो॒का दृच॒ ऋचो॑ मनुष्यलो॒कात् । \newline
24. म॒नु॒ष्य॒लो॒का दे॒वैव म॑नुष्यलो॒कान् म॑नुष्यलो॒का दे॒व । \newline
25. म॒नु॒ष्य॒लो॒कादिति॑ मनुष्य - लो॒कात् । \newline
26. ए॒वा न्यम॑न्य म॒न्यम॑न्य मे॒वै वान्यम॑न्यम् । \newline
27. अ॒न्यम॑न्यम् देवलो॒कम् दे॑वलो॒क म॒न्यम॑न्य म॒न्यम॑न्यम् देवलो॒कम् । \newline
28. अ॒न्यम॑न्य॒मित्य॒न्यम् - अ॒न्य॒म् । \newline
29. दे॒व॒लो॒क म॑भ्या॒रोह॑न्तो ऽभ्या॒रोह॑न्तो देवलो॒कम् दे॑वलो॒क म॑भ्या॒रोह॑न्तः । \newline
30. दे॒व॒लो॒कमिति॑ देव - लो॒कम् । \newline
31. अ॒भ्या॒रोह॑न्तो यन्ति यन्त्यभ्या॒रोह॑न्तो ऽभ्या॒रोह॑न्तो यन्ति । \newline
32. अ॒भ्या॒रोह॑न्त॒ इत्य॑भि - आ॒रोह॑न्तः । \newline
33. य॒न्त्य॒भि॒व॒र्तो॑ ऽभिव॒र्तो य॑न्ति यन्त्यभिव॒र्तः । \newline
34. अ॒भि॒व॒र्तो ब्र॑ह्मसा॒मम् ब्र॑ह्मसा॒म म॑भिव॒र्तो॑ ऽभिव॒र्तो ब्र॑ह्मसा॒मम् । \newline
35. अ॒भि॒व॒र्त इत्य॑भि - व॒र्तः । \newline
36. ब्र॒ह्म॒सा॒मम् भ॑वति भवति ब्रह्मसा॒मम् ब्र॑ह्मसा॒मम् भ॑वति । \newline
37. ब्र॒ह्म॒सा॒ममिति॑ ब्रह्म - सा॒मम् । \newline
38. भ॒व॒ति॒ सु॒व॒र्गस्य॑ सुव॒र्गस्य॑ भवति भवति सुव॒र्गस्य॑ । \newline
39. सु॒व॒र्गस्य॑ लो॒कस्य॑ लो॒कस्य॑ सुव॒र्गस्य॑ सुव॒र्गस्य॑ लो॒कस्य॑ । \newline
40. सु॒व॒र्गस्येति॑ सुवः - गस्य॑ । \newline
41. लो॒कस्या॒ भिवृ॑त्या अ॒भिवृ॑त्यै लो॒कस्य॑ लो॒कस्या॒ भिवृ॑त्यै । \newline
42. अ॒भिवृ॑त्या अभि॒जि द॑भि॒जि द॒भिवृ॑त्या अ॒भिवृ॑त्या अभि॒जित् । \newline
43. अ॒भिवृ॑त्या॒ इत्य॒भि - वृ॒त्यै॒ । \newline
44. अ॒भि॒जिद् भ॑वति भव त्यभि॒जि द॑भि॒जिद् भ॑वति । \newline
45. अ॒भि॒जिदित्य॑भि - जित् । \newline
46. भ॒व॒ति॒ सु॒व॒र्गस्य॑ सुव॒र्गस्य॑ भवति भवति सुव॒र्गस्य॑ । \newline
47. सु॒व॒र्गस्य॑ लो॒कस्य॑ लो॒कस्य॑ सुव॒र्गस्य॑ सुव॒र्गस्य॑ लो॒कस्य॑ । \newline
48. सु॒व॒र्गस्येति॑ सुवः - गस्य॑ । \newline
49. लो॒कस्या॒ भिजि॑त्या अ॒भिजि॑त्यै लो॒कस्य॑ लो॒कस्या॒ भिजि॑त्यै । \newline
50. अ॒भिजि॑त्यै विश्व॒जिद् वि॑श्व॒जि द॒भिजि॑त्या अ॒भिजि॑त्यै विश्व॒जित् । \newline
51. अ॒भिजि॑त्या॒ इत्य॒भि - जि॒त्यै॒ । \newline
52. वि॒श्व॒जिद् भ॑वति भवति विश्व॒जिद् वि॑श्व॒जिद् भ॑वति । \newline
53. वि॒श्व॒जिदिति॑ विश्व - जित् । \newline
54. भ॒व॒ति॒ विश्व॑स्य॒ विश्व॑स्य भवति भवति॒ विश्व॑स्य । \newline
55. विश्व॑स्य॒ जित्यै॒ जित्यै॒ विश्व॑स्य॒ विश्व॑स्य॒ जित्यै᳚ । \newline
56. जित्यै॑ मा॒सिमा॑सि मा॒सिमा॑सि॒ जित्यै॒ जित्यै॑ मा॒सिमा॑सि । \newline
57. मा॒सिमा॑सि पृ॒ष्ठानि॑ पृ॒ष्ठानि॑ मा॒सिमा॑सि मा॒सिमा॑सि पृ॒ष्ठानि॑ । \newline
58. मा॒सिमा॒सीति॑ मा॒सि - मा॒सि॒ । \newline
59. पृ॒ष्ठा न्युपोप॑ पृ॒ष्ठानि॑ पृ॒ष्ठा न्युप॑ । \newline
60. उप॑ यन्ति य॒न्त्युपोप॑ यन्ति । \newline
61. य॒न्ति॒ मा॒सिमा॑सि मा॒सिमा॑सि यन्ति यन्ति मा॒सिमा॑सि । \newline
62. मा॒सिमा᳚ स्यतिग्रा॒ह्या॑ अतिग्रा॒ह्या॑ मा॒सिमा॑सि मा॒सिमा᳚ स्यतिग्रा॒ह्याः᳚ । \newline
63. मा॒सिमा॒सीति॑ मा॒सि - मा॒सि॒ । \newline
64. अ॒ति॒ग्रा॒ह्या॑ गृह्यन्ते गृह्यन्ते ऽतिग्रा॒ह्या॑ अतिग्रा॒ह्या॑ गृह्यन्ते । \newline
65. अ॒ति॒ग्रा॒ह्या॑ इत्य॑ति - ग्रा॒ह्याः᳚ । \newline
66. गृ॒ह्य॒न्ते॒ मा॒सिमा॑सि मा॒सिमा॑सि गृह्यन्ते गृह्यन्ते मा॒सिमा॑सि । \newline
67. मा॒सिमा᳚ स्ये॒वैव मा॒सिमा॑सि मा॒सिमा᳚ स्ये॒व । \newline
68. मा॒सिमा॒सीति॑ मा॒सि - मा॒सि॒ । \newline
69. ए॒व वी॒र्यं॑ ॅवी॒र्य॑ मे॒वैव वी॒र्य᳚म् । \newline
70. वी॒र्य॑म् दधति दधति वी॒र्यं॑ ॅवी॒र्य॑म् दधति । \newline
71. द॒ध॒ति॒ मा॒साम् मा॒साम् द॑धति दधति मा॒साम् । \newline
72. मा॒साम् प्रति॑ष्ठित्यै॒ प्रति॑ष्ठित्यै मा॒साम् मा॒साम् प्रति॑ष्ठित्यै । \newline
73. प्रति॑ष्ठित्या उ॒परि॑ष्टा दु॒परि॑ष्टा॒त् प्रति॑ष्ठित्यै॒ प्रति॑ष्ठित्या उ॒परि॑ष्टात् । \newline
74. प्रति॑ष्ठित्या॒ इति॒ प्रति॑ - स्थि॒त्यै॒ । \newline
75. उ॒परि॑ष्टान् मा॒साम् मा॒सा मु॒परि॑ष्टा दु॒परि॑ष्टान् मा॒साम् । \newline
76. मा॒साम् पृ॒ष्ठानि॑ पृ॒ष्ठानि॑ मा॒साम् मा॒साम् पृ॒ष्ठानि॑ । \newline
77. पृ॒ष्ठा न्युपोप॑ पृ॒ष्ठानि॑ पृ॒ष्ठा न्युप॑ । \newline
78. उप॑ यन्ति य॒न्त्युपोप॑ यन्ति । \newline
79. य॒न्ति॒ तस्मा॒त् तस्मा᳚द् यन्ति यन्ति॒ तस्मा᳚त् । \newline
80. तस्मा॑ दु॒परि॑ष्टा दु॒परि॑ष्टा॒त् तस्मा॒त् तस्मा॑ दु॒परि॑ष्टात् । \newline
81. उ॒परि॑ष्टा॒ दोष॑धय॒ ओष॑धय उ॒परि॑ष्टा दु॒परि॑ष्टा॒ दोष॑धयः । \newline
82. ओष॑धयः॒ फल॒म् फल॒ मोष॑धय॒ ओष॑धयः॒ फल᳚म् । \newline
83. फल॑म् गृह्णन्ति गृह्णन्ति॒ फल॒म् फल॑म् गृह्णन्ति । \newline
84. गृ॒ह्ण॒न्तीति॑ गृह्णन्ति । \newline

\textbf{Ghana Paata } \newline

1. तत् प॒था प॒था तत् तत् प॒था य॑न्ति यन्ति प॒था तत् तत् प॒था य॑न्ति । \newline
2. प॒था य॑न्ति यन्ति प॒था प॒था य॑न्ति समा॒नꣳ स॑मा॒नं ॅय॑न्ति प॒था प॒था य॑न्ति समा॒नम् । \newline
3. य॒न्ति॒ स॒मा॒नꣳ स॑मा॒नं ॅय॑न्ति यन्ति समा॒नꣳ साम॒ साम॑ समा॒नं ॅय॑न्ति यन्ति समा॒नꣳ साम॑ । \newline
4. स॒मा॒नꣳ साम॒ साम॑ समा॒नꣳ स॑मा॒नꣳ साम॑ भवति भवति॒ साम॑ समा॒नꣳ स॑मा॒नꣳ साम॑ भवति । \newline
5. साम॑ भवति भवति॒ साम॒ साम॑ भवति देवलो॒को दे॑वलो॒को भ॑वति॒ साम॒ साम॑ भवति देवलो॒कः । \newline
6. भ॒व॒ति॒ दे॒व॒लो॒को दे॑वलो॒को भ॑वति भवति देवलो॒को वै वै दे॑वलो॒को भ॑वति भवति देवलो॒को वै । \newline
7. दे॒व॒लो॒को वै वै दे॑वलो॒को दे॑वलो॒को वै साम॒ साम॒ वै दे॑वलो॒को दे॑वलो॒को वै साम॑ । \newline
8. दे॒व॒लो॒क इति॑ देव - लो॒कः । \newline
9. वै साम॒ साम॒ वै वै साम॑ देवलो॒काद् दे॑वलो॒काथ् साम॒ वै वै साम॑ देवलो॒कात् । \newline
10. साम॑ देवलो॒काद् दे॑वलो॒काथ् साम॒ साम॑ देवलो॒का दे॒वैव दे॑वलो॒काथ् साम॒ साम॑ देवलो॒का दे॒व । \newline
11. दे॒व॒लो॒का दे॒वैव दे॑वलो॒काद् दे॑वलो॒का दे॒व न नैव दे॑वलो॒काद् दे॑वलो॒का दे॒व न । \newline
12. दे॒व॒लो॒कादिति॑ देव - लो॒कात् । \newline
13. ए॒व न नैवैव न य॑न्ति यन्ति॒ नैवैव न य॑न्ति । \newline
14. न य॑न्ति यन्ति॒ न न य॑न्त्य॒न्याअ॑न्या अ॒न्याअ॑न्या यन्ति॒ न न य॑न्त्य॒न्याअ॑न्याः । \newline
15. य॒न्त्य॒न्याअ॑न्या अ॒न्याअ॑न्या यन्ति यन्त्य॒न्याअ॑न्या॒ ऋच॒ ऋचो॒ ऽन्याअ॑न्या यन्ति यन्त्य॒न्याअ॑न्या॒ ऋचः॑ । \newline
16. अ॒न्याअ॑न्या॒ ऋच॒ ऋचो॒ ऽन्याअ॑न्या अ॒न्याअ॑न्या॒ ऋचो॑ भवन्ति भव॒ न्त्यृचो॒ ऽन्याअ॑न्या अ॒न्याअ॑न्या॒ ऋचो॑ भवन्ति । \newline
17. अ॒न्याअ॑न्या॒ इत्य॒न्याः - अ॒न्याः॒ । \newline
18. ऋचो॑ भवन्ति भव॒ न्त्यृच॒ ऋचो॑ भवन्ति मनुष्यलो॒को म॑नुष्यलो॒को भ॑व॒ न्त्यृच॒ ऋचो॑ भवन्ति मनुष्यलो॒कः । \newline
19. भ॒व॒न्ति॒ म॒नु॒ष्य॒लो॒को म॑नुष्यलो॒को भ॑वन्ति भवन्ति मनुष्यलो॒को वै वै म॑नुष्यलो॒को भ॑वन्ति भवन्ति मनुष्यलो॒को वै । \newline
20. म॒नु॒ष्य॒लो॒को वै वै म॑नुष्यलो॒को म॑नुष्यलो॒को वा ऋच॒ ऋचो॒ वै म॑नुष्यलो॒को म॑नुष्यलो॒को वा ऋचः॑ । \newline
21. म॒नु॒ष्य॒लो॒क इति॑ मनुष्य - लो॒कः । \newline
22. वा ऋच॒ ऋचो॒ वै वा ऋचो॑ मनुष्यलो॒कान् म॑नुष्यलो॒का दृचो॒ वै वा ऋचो॑ मनुष्यलो॒कात् । \newline
23. ऋचो॑ मनुष्यलो॒कान् म॑नुष्यलो॒का दृच॒ ऋचो॑ मनुष्यलो॒का दे॒वैव म॑नुष्यलो॒का दृच॒ ऋचो॑ मनुष्यलो॒का दे॒व । \newline
24. म॒नु॒ष्य॒लो॒का दे॒वैव म॑नुष्यलो॒कान् म॑नुष्यलो॒का दे॒वा न्यम॑न्य म॒न्यम॑न्य मे॒व म॑नुष्यलो॒कान् म॑नुष्यलो॒का दे॒वा न्यम॑न्यम् । \newline
25. म॒नु॒ष्य॒लो॒कादिति॑ मनुष्य - लो॒कात् । \newline
26. ए॒वा न्यम॑न्य म॒न्यम॑न्य मे॒वै वान्यम॑न्यम् देवलो॒कम् दे॑वलो॒क म॒न्यम॑न्य मे॒वैवा न्यम॑न्यम् देवलो॒कम् । \newline
27. अ॒न्यम॑न्यम् देवलो॒कम् दे॑वलो॒क म॒न्यम॑न्य म॒न्यम॑न्यम् देवलो॒क म॑भ्या॒रोह॑न्तो ऽभ्या॒रोह॑न्तो देवलो॒क म॒न्यम॑न्य म॒न्यम॑न्यम् देवलो॒क म॑भ्या॒रोह॑न्तः । \newline
28. अ॒न्यम॑न्य॒मित्य॒न्यम् - अ॒न्य॒म् । \newline
29. दे॒व॒लो॒क म॑भ्या॒रोह॑न्तो ऽभ्या॒रोह॑न्तो देवलो॒कम् दे॑वलो॒क म॑भ्या॒रोह॑न्तो यन्ति यन्त्यभ्या॒रोह॑न्तो देवलो॒कम् दे॑वलो॒क म॑भ्या॒रोह॑न्तो यन्ति । \newline
30. दे॒व॒लो॒कमिति॑ देव - लो॒कम् । \newline
31. अ॒भ्या॒रोह॑न्तो यन्ति यन्त्यभ्या॒रोह॑न्तो ऽभ्या॒रोह॑न्तो यन्त्यभिव॒र्तो॑ ऽभिव॒र्तो य॑न्त्यभ्या॒रोह॑न्तो ऽभ्या॒रोह॑न्तो यन्त्यभिव॒र्तः । \newline
32. अ॒भ्या॒रोह॑न्त॒ इत्य॑भि - आ॒रोह॑न्तः । \newline
33. य॒न्त्य॒भि॒व॒र्तो॑ ऽभिव॒र्तो य॑न्ति यन्त्यभिव॒र्तो ब्र॑ह्मसा॒मम् ब्र॑ह्मसा॒म म॑भिव॒र्तो य॑न्ति यन्त्यभिव॒र्तो ब्र॑ह्मसा॒मम् । \newline
34. अ॒भि॒व॒र्तो ब्र॑ह्मसा॒मम् ब्र॑ह्मसा॒म म॑भिव॒र्तो॑ ऽभिव॒र्तो ब्र॑ह्मसा॒मम् भ॑वति भवति ब्रह्मसा॒म म॑भिव॒र्तो॑ ऽभिव॒र्तो ब्र॑ह्मसा॒मम् भ॑वति । \newline
35. अ॒भि॒व॒र्त इत्य॑भि - व॒र्तः । \newline
36. ब्र॒ह्म॒सा॒मम् भ॑वति भवति ब्रह्मसा॒मम् ब्र॑ह्मसा॒मम् भ॑वति सुव॒र्गस्य॑ सुव॒र्गस्य॑ भवति ब्रह्मसा॒मम् ब्र॑ह्मसा॒मम् भ॑वति सुव॒र्गस्य॑ । \newline
37. ब्र॒ह्म॒सा॒ममिति॑ ब्रह्म - सा॒मम् । \newline
38. भ॒व॒ति॒ सु॒व॒र्गस्य॑ सुव॒र्गस्य॑ भवति भवति सुव॒र्गस्य॑ लो॒कस्य॑ लो॒कस्य॑ सुव॒र्गस्य॑ भवति भवति सुव॒र्गस्य॑ लो॒कस्य॑ । \newline
39. सु॒व॒र्गस्य॑ लो॒कस्य॑ लो॒कस्य॑ सुव॒र्गस्य॑ सुव॒र्गस्य॑ लो॒कस्या॒ भिवृ॑त्या अ॒भिवृ॑त्यै लो॒कस्य॑ सुव॒र्गस्य॑ सुव॒र्गस्य॑ लो॒कस्या॒ भिवृ॑त्यै । \newline
40. सु॒व॒र्गस्येति॑ सुवः - गस्य॑ । \newline
41. लो॒कस्या॒ भिवृ॑त्या अ॒भिवृ॑त्यै लो॒कस्य॑ लो॒कस्या॒ भिवृ॑त्या अभि॒जि द॑भि॒जि द॒भिवृ॑त्यै लो॒कस्य॑ लो॒कस्या॒ भिवृ॑त्या अभि॒जित् । \newline
42. अ॒भिवृ॑त्या अभि॒जि द॑भि॒जि द॒भिवृ॑त्या अ॒भिवृ॑त्या अभि॒जिद् भ॑वति भव त्यभि॒जि द॒भिवृ॑त्या अ॒भिवृ॑त्या अभि॒जिद् भ॑वति । \newline
43. अ॒भिवृ॑त्या॒ इत्य॒भि - वृ॒त्यै॒ । \newline
44. अ॒भि॒जिद् भ॑वति भव त्यभि॒जि द॑भि॒जिद् भ॑वति सुव॒र्गस्य॑ सुव॒र्गस्य॑ भव त्यभि॒जि द॑भि॒जिद् भ॑वति सुव॒र्गस्य॑ । \newline
45. अ॒भि॒जिदित्य॑भि - जित् । \newline
46. भ॒व॒ति॒ सु॒व॒र्गस्य॑ सुव॒र्गस्य॑ भवति भवति सुव॒र्गस्य॑ लो॒कस्य॑ लो॒कस्य॑ सुव॒र्गस्य॑ भवति भवति सुव॒र्गस्य॑ लो॒कस्य॑ । \newline
47. सु॒व॒र्गस्य॑ लो॒कस्य॑ लो॒कस्य॑ सुव॒र्गस्य॑ सुव॒र्गस्य॑ लो॒कस्या॒ भिजि॑त्या अ॒भिजि॑त्यै लो॒कस्य॑ सुव॒र्गस्य॑ सुव॒र्गस्य॑ लो॒कस्या॒ भिजि॑त्यै । \newline
48. सु॒व॒र्गस्येति॑ सुवः - गस्य॑ । \newline
49. लो॒कस्या॒ भिजि॑त्या अ॒भिजि॑त्यै लो॒कस्य॑ लो॒कस्या॒ भिजि॑त्यै विश्व॒जिद् वि॑श्व॒जि द॒भिजि॑त्यै लो॒कस्य॑ लो॒कस्या॒ भिजि॑त्यै विश्व॒जित् । \newline
50. अ॒भिजि॑त्यै विश्व॒जिद् वि॑श्व॒जि द॒भिजि॑त्या अ॒भिजि॑त्यै विश्व॒जिद् भ॑वति भवति विश्व॒जि द॒भिजि॑त्या अ॒भिजि॑त्यै विश्व॒जिद् भ॑वति । \newline
51. अ॒भिजि॑त्या॒ इत्य॒भि - जि॒त्यै॒ । \newline
52. वि॒श्व॒जिद् भ॑वति भवति विश्व॒जिद् वि॑श्व॒जिद् भ॑वति॒ विश्व॑स्य॒ विश्व॑स्य भवति विश्व॒जिद् वि॑श्व॒जिद् भ॑वति॒ विश्व॑स्य । \newline
53. वि॒श्व॒जिदिति॑ विश्व - जित् । \newline
54. भ॒व॒ति॒ विश्व॑स्य॒ विश्व॑स्य भवति भवति॒ विश्व॑स्य॒ जित्यै॒ जित्यै॒ विश्व॑स्य भवति भवति॒ विश्व॑स्य॒ जित्यै᳚ । \newline
55. विश्व॑स्य॒ जित्यै॒ जित्यै॒ विश्व॑स्य॒ विश्व॑स्य॒ जित्यै॑ मा॒सिमा॑सि मा॒सिमा॑सि॒ जित्यै॒ विश्व॑स्य॒ विश्व॑स्य॒ जित्यै॑ मा॒सिमा॑सि । \newline
56. जित्यै॑ मा॒सिमा॑सि मा॒सिमा॑सि॒ जित्यै॒ जित्यै॑ मा॒सिमा॑सि पृ॒ष्ठानि॑ पृ॒ष्ठानि॑ मा॒सिमा॑सि॒ जित्यै॒ जित्यै॑ मा॒सिमा॑सि पृ॒ष्ठानि॑ । \newline
57. मा॒सिमा॑सि पृ॒ष्ठानि॑ पृ॒ष्ठानि॑ मा॒सिमा॑सि मा॒सिमा॑सि पृ॒ष्ठा न्युपोप॑ पृ॒ष्ठानि॑ मा॒सिमा॑सि मा॒सिमा॑सि पृ॒ष्ठा न्युप॑ । \newline
58. मा॒सिमा॒सीति॑ मा॒सि - मा॒सि॒ । \newline
59. पृ॒ष्ठा न्युपोप॑ पृ॒ष्ठानि॑ पृ॒ष्ठा न्युप॑ यन्ति य॒न्त्युप॑ पृ॒ष्ठानि॑ पृ॒ष्ठा न्युप॑ यन्ति । \newline
60. उप॑ यन्ति य॒न्त्युपोप॑ यन्ति मा॒सिमा॑सि मा॒सिमा॑सि य॒न्त्युपोप॑ यन्ति मा॒सिमा॑सि । \newline
61. य॒न्ति॒ मा॒सिमा॑सि मा॒सिमा॑सि यन्ति यन्ति मा॒सिमा᳚ स्यतिग्रा॒ह्या॑ अतिग्रा॒ह्या॑ मा॒सिमा॑सि यन्ति यन्ति मा॒सिमा᳚
स्यतिग्रा॒ह्याः᳚ । \newline
62. मा॒सिमा᳚ स्यतिग्रा॒ह्या॑ अतिग्रा॒ह्या॑ मा॒सिमा॑सि मा॒सिमा᳚ स्यतिग्रा॒ह्या॑ गृह्यन्ते गृह्यन्ते ऽतिग्रा॒ह्या॑ मा॒सिमा॑सि मा॒सिमा᳚ स्यतिग्रा॒ह्या॑ गृह्यन्ते । \newline
63. मा॒सिमा॒सीति॑ मा॒सि - मा॒सि॒ । \newline
64. अ॒ति॒ग्रा॒ह्या॑ गृह्यन्ते गृह्यन्ते ऽतिग्रा॒ह्या॑ अतिग्रा॒ह्या॑ गृह्यन्ते मा॒सिमा॑सि मा॒सिमा॑सि गृह्यन्ते ऽतिग्रा॒ह्या॑ अतिग्रा॒ह्या॑ गृह्यन्ते मा॒सिमा॑सि । \newline
65. अ॒ति॒ग्रा॒ह्या॑ इत्य॑ति - ग्रा॒ह्याः᳚ । \newline
66. गृ॒ह्य॒न्ते॒ मा॒सिमा॑सि मा॒सिमा॑सि गृह्यन्ते गृह्यन्ते मा॒सिमा᳚ स्ये॒वैव मा॒सिमा॑सि गृह्यन्ते गृह्यन्ते मा॒सिमा᳚ स्ये॒व । \newline
67. मा॒सिमा᳚ स्ये॒वैव मा॒सिमा॑सि मा॒सिमा᳚ स्ये॒व वी॒र्यं॑ ॅवी॒र्य॑ मे॒व मा॒सिमा॑सि मा॒सिमा᳚स्ये॒व वी॒र्य᳚म् । \newline
68. मा॒सिमा॒सीति॑ मा॒सि - मा॒सि॒ । \newline
69. ए॒व वी॒र्यं॑ ॅवी॒र्य॑ मे॒वैव वी॒र्य॑म् दधति दधति वी॒र्य॑ मे॒वैव वी॒र्य॑म् दधति । \newline
70. वी॒र्य॑म् दधति दधति वी॒र्यं॑ ॅवी॒र्य॑म् दधति मा॒साम् मा॒साम् द॑धति वी॒र्यं॑ ॅवी॒र्य॑म् दधति मा॒साम् । \newline
71. द॒ध॒ति॒ मा॒साम् मा॒साम् द॑धति दधति मा॒साम् प्रति॑ष्ठित्यै॒ प्रति॑ष्ठित्यै मा॒साम् द॑धति दधति मा॒साम् प्रति॑ष्ठित्यै । \newline
72. मा॒साम् प्रति॑ष्ठित्यै॒ प्रति॑ष्ठित्यै मा॒साम् मा॒साम् प्रति॑ष्ठित्या उ॒परि॑ष्टा दु॒परि॑ष्टा॒त् प्रति॑ष्ठित्यै मा॒साम् मा॒साम् प्रति॑ष्ठित्या उ॒परि॑ष्टात् । \newline
73. प्रति॑ष्ठित्या उ॒परि॑ष्टा दु॒परि॑ष्टा॒त् प्रति॑ष्ठित्यै॒ प्रति॑ष्ठित्या उ॒परि॑ष्टान् मा॒साम् मा॒सा मु॒परि॑ष्टा॒त् प्रति॑ष्ठित्यै॒ प्रति॑ष्ठित्या उ॒परि॑ष्टान् मा॒साम् । \newline
74. प्रति॑ष्ठित्या॒ इति॒ प्रति॑ - स्थि॒त्यै॒ । \newline
75. उ॒परि॑ष्टान् मा॒साम् मा॒सा मु॒परि॑ष्टा दु॒परि॑ष्टान् मा॒साम् पृ॒ष्ठानि॑ पृ॒ष्ठानि॑ मा॒सा मु॒परि॑ष्टा दु॒परि॑ष्टान् मा॒साम् पृ॒ष्ठानि॑ । \newline
76. मा॒साम् पृ॒ष्ठानि॑ पृ॒ष्ठानि॑ मा॒साम् मा॒साम् पृ॒ष्ठा न्युपोप॑ पृ॒ष्ठानि॑ मा॒साम् मा॒साम् पृ॒ष्ठा न्युप॑ । \newline
77. पृ॒ष्ठा न्युपोप॑ पृ॒ष्ठानि॑ पृ॒ष्ठा न्युप॑ यन्ति य॒न्त्युप॑ पृ॒ष्ठानि॑ पृ॒ष्ठा न्युप॑ यन्ति । \newline
78. उप॑ यन्ति य॒न्त्युपोप॑ यन्ति॒ तस्मा॒त् तस्मा᳚द् य॒न्त्युपोप॑ यन्ति॒ तस्मा᳚त् । \newline
79. य॒न्ति॒ तस्मा॒त् तस्मा᳚द् यन्ति यन्ति॒ तस्मा॑ दु॒परि॑ष्टा दु॒परि॑ष्टा॒त् तस्मा᳚द् यन्ति यन्ति॒ तस्मा॑ दु॒परि॑ष्टात् । \newline
80. तस्मा॑ दु॒परि॑ष्टा दु॒परि॑ष्टा॒त् तस्मा॒त् तस्मा॑ दु॒परि॑ष्टा॒ दोष॑धय॒ ओष॑धय उ॒परि॑ष्टा॒त् तस्मा॒त् तस्मा॑ दु॒परि॑ष्टा॒ दोष॑धयः । \newline
81. उ॒परि॑ष्टा॒ दोष॑धय॒ ओष॑धय उ॒परि॑ष्टा दु॒परि॑ष्टा॒ दोष॑धयः॒ फल॒म् फल॒ मोष॑धय उ॒परि॑ष्टा दु॒परि॑ष्टा॒ दोष॑धयः॒ फल᳚म् । \newline
82. ओष॑धयः॒ फल॒म् फल॒ मोष॑धय॒ ओष॑धयः॒ फल॑म् गृह्णन्ति गृह्णन्ति॒ फल॒ मोष॑धय॒ ओष॑धयः॒ फल॑म् गृह्णन्ति । \newline
83. फल॑म् गृह्णन्ति गृह्णन्ति॒ फल॒म् फल॑म् गृह्णन्ति । \newline
84. गृ॒ह्ण॒न्तीति॑ गृह्णन्ति । \newline
\pagebreak
\markright{ TS 7.5.2.1  \hfill https://www.vedavms.in \hfill}

\section{ TS 7.5.2.1 }

\textbf{TS 7.5.2.1 } \newline
\textbf{Samhita Paata} \newline

गावो॒ वा ए॒तथ् स॒त्रमा॑सताशृ॒ङ्गाः स॒तीः शृङ्गा॑णि॒ सिषा॑सन्ती॒स्तासां॒ दश॒ मासा॒ निष॑ण्णा॒ आस॒न्नथ॒ शृङ्गा᳚ण्यजायन्त॒ ता अ॑ब्रुव॒न्नरा॒थ्स्मोत् ति॑ष्ठा॒माव॒ तं काम॑मरुथ्स्महि॒ येन॒ कामे॑न॒ न्यष॑दा॒मेति॒ तासा॑मु॒ त्वा अ॑ब्रुवन्न॒र्द्धावा॒ याव॑ती॒र्वाऽऽसा॑महा ए॒वेमौद्वा॑द॒शौ मासौ॑ संॅवथ्स॒रꣳ स॒पांद्योत् ति॑ष्ठा॒मेति॒ तासां᳚ - [  ] \newline

\textbf{Pada Paata} \newline

गावः॑ । वै । ए॒तत् । स॒त्रम् । आ॒स॒त॒ । अ॒शृ॒ङ्गाः । स॒तीः । शृङ्गा॑णि । सिषा॑सन्तीः । तासा᳚म् । दश॑ । मासाः᳚ । निष॑ण्णा॒ इति॒ नि - स॒न्नाः॒ । आसन्न्॑ । अथ॑ । शृङ्गा॑णि । अ॒जा॒य॒न्त॒ । ताः । अ॒ब्रु॒व॒न्न् । अरा᳚थ्स्म । उदिति॑ । ति॒ष्ठा॒म॒ । अवेति॑ । तम् । काम᳚म् । अ॒रु॒थ्स्म॒हि॒ । येन॑ । कामे॑न । न्यष॑दा॒मेति॑ नि - अस॑दाम । इति॑ । तासा᳚म् । उ॒ । तु । वै । अ॒ब्रु॒व॒न्न् । अ॒द्‌र्धाः । वा॒ । याव॑तीः । वा॒ । आसा॑महै । ए॒व । इ॒मौ । द्वा॒द॒शौ । मासौ᳚ । सं॒ॅव॒थ्स॒रमिति॑ सं - व॒थ्स॒रम् । स॒पांद्येति॑ सं - पाद्य॑ । उदिति॑ । ति॒ष्ठा॒म॒ । इति॑ । तासा᳚म् ।  \newline


\textbf{Krama Paata} \newline

गावो॒ वै । वा ए॒तत् । ए॒तथ् स॒त्रम् । स॒त्रमा॑सत । आ॒स॒ता॒शृ॒ङ्‍गाः । अ॒शृ॒ङ्‍गाः स॒तीः । स॒तीः शृङ्‍गा॑णि । शृङ्‍गा॑णि॒ सिषा॑सन्तीः । सिषा॑सन्ती॒स्तासा᳚म् । तासा॒म् दश॑ । दश॒ मासाः᳚ । मासा॒ निष॑ण्णाः । निष॑ण्णा॒ आसन्न्॑ । निष॑ण्णा॒ इति॒ नि - स॒न्नाः॒ । आस॒न्नथ॑ । अथ॒ शृङ्‍गा॑णि । शृङ्‍गा᳚ण्यजायन्त । अ॒जा॒य॒न्त॒ ताः । ता अ॑ब्रुवन्न् । अ॒ब्रु॒व॒न्नरा᳚थ्स्म । अरा॒थ्स्मोत् । उत् ति॑ष्ठाम । ति॒ष्ठा॒माव॑ । अव॑ तम् । तम् काम᳚म् । काम॑मरुथ्स्महि । अ॒रु॒थ्स्म॒हि॒ येन॑ । येन॒ कामे॑न । कामे॑न॒ न्यष॑दाम । न्यष॑दा॒मेति॑ । न्यष॑दा॒मेति॑ नि - अस॑दाम । इति॒ तासा᳚म् । तासा॑मु । उ॒ तु । त्वै । वा अ॑ब्रुवन्न् । अ॒ब्रु॒व॒न्न॒र्द्धाः । अ॒र्द्धा वा᳚ । वा॒ याव॑तीः । याव॑तीर् वा । वाऽऽसा॑महै । आसा॑महा ए॒व । ए॒वेमौ । इ॒मौ द्वा॑द॒शौ । द्वा॒द॒शौ मासौ᳚ । मासौ॑ सम्ॅवथ्स॒रम् । स॒म्ॅव॒थ्स॒रꣳ स॒म्पाद्य॑ । स॒म्ॅव॒थ्स॒रमिति॑ सम् - व॒थ्स॒रम् । स॒म्पाद्योत् । स॒म्पाद्येति॑ सम् - पाद्य॑ । उत् ति॑ष्ठाम । ति॒ष्ठा॒मेति॑ । इति॒ तासा᳚म् । तासा᳚म् द्वाद॒शे \newline

\textbf{Jatai Paata} \newline

1. गावो॒ वै वै गावो॒ गावो॒ वै । \newline
2. वा ए॒त दे॒तद् वै वा ए॒तत् । \newline
3. ए॒तथ् स॒त्रꣳ स॒त्र मे॒त दे॒तथ् स॒त्रम् । \newline
4. स॒त्र मा॑सता सत स॒त्रꣳ स॒त्र मा॑सत । \newline
5. आ॒स॒ता॒ शृ॒ङ्गा अ॑शृ॒ङ्गा आ॑सता सता शृ॒ङ्गाः । \newline
6. अ॒शृ॒ङ्गाः स॒तीः स॒ती र॑शृ॒ङ्गा अ॑शृ॒ङ्गाः स॒तीः । \newline
7. स॒तीः शृङ्गा॑णि॒ शृङ्गा॑णि स॒तीः स॒तीः शृङ्गा॑णि । \newline
8. शृङ्गा॑णि॒ सिषा॑सन्तीः॒ सिषा॑सन्तीः॒ शृङ्गा॑णि॒ शृङ्गा॑णि॒ सिषा॑सन्तीः । \newline
9. सिषा॑सन्ती॒ स्तासा॒म् तासाꣳ॒॒ सिषा॑सन्तीः॒ सिषा॑सन्ती॒ स्तासा᳚म् । \newline
10. तासा॒म् दश॒ दश॒ तासा॒म् तासा॒म् दश॑ । \newline
11. दश॒ मासा॒ मासा॒ दश॒ दश॒ मासाः᳚ । \newline
12. मासा॒ निष॑ण्णा॒ निष॑ण्णा॒ मासा॒ मासा॒ निष॑ण्णाः । \newline
13. निष॑ण्णा॒ आस॒न् नास॒न् निष॑ण्णा॒ निष॑ण्णा॒ आसन्न्॑ । \newline
14. निष॑ण्णा॒ इति॒ नि - स॒न्नाः॒ । \newline
15. आस॒न् नथा थास॒न् नास॒न् नथ॑ । \newline
16. अथ॒ शृङ्गा॑णि॒ शृङ्गा॒ ण्यथाथ॒ शृङ्गा॑णि । \newline
17. शृङ्गा᳚ ण्यजायन्ता जायन्त॒ शृङ्गा॑णि॒ शृङ्गा᳚ ण्यजायन्त । \newline
18. अ॒जा॒य॒न्त॒ ता स्ता अ॑जायन्ता जायन्त॒ ताः । \newline
19. ता अ॑ब्रुवन् नब्रुव॒न् ता स्ता अ॑ब्रुवन्न् । \newline
20. अ॒ब्रु॒व॒न् नरा॒थ्स्मा रा᳚थ्स्मा ब्रुवन् नब्रुव॒न् नरा᳚थ्स्म । \newline
21. अरा॒थ्स् मोदु दरा॒थ्स्मा रा॒थ्स्मोत् । \newline
22. उत् ति॑ष्ठाम तिष्ठा॒ मोदुत् ति॑ष्ठाम । \newline
23. ति॒ष्ठा॒मा वाव॑ तिष्ठाम तिष्ठा॒ माव॑ । \newline
24. अव॒ तम् त मवाव॒ तम् । \newline
25. तम् काम॒म् काम॒म् तम् तम् काम᳚म् । \newline
26. काम॑ मरुथ्स्म ह्यरुथ्स्महि॒ काम॒म् काम॑ मरुथ्स्महि । \newline
27. अ॒रु॒थ्स्म॒हि॒ येन॒ येना॑ रुथ्स्म ह्यरुथ्स्महि॒ येन॑ । \newline
28. येन॒ कामे॑न॒ कामे॑न॒ येन॒ येन॒ कामे॑न । \newline
29. कामे॑न॒ न्यष॑दाम॒ न्यष॑दाम॒ कामे॑न॒ कामे॑न॒ न्यष॑दाम । \newline
30. न्यष॑दा॒मेतीति॒ न्यष॑दाम॒ न्यष॑दा॒मेति॑ । \newline
31. न्यष॑दा॒मेति॑ नि - अस॑दाम । \newline
32. इति॒ तासा॒म् तासा॒ मितीति॒ तासा᳚म् । \newline
33. तासा॑ मु वु॒ तासा॒म् तासा॑ मु । \newline
34. उ॒ तु तू॑ तु । \newline
35. त्वै वै तुत्वै । \newline
36. वा अ॑ब्रुवन् नब्रुव॒न्॒. वै वा अ॑ब्रुवन्न् । \newline
37. अ॒ब्रु॒व॒न् न॒र्द्धा अ॒र्द्धा अ॑ब्रुवन् नब्रुवन् न॒र्द्धाः । \newline
38. अ॒र्द्धा वा॑ वा॒ ऽर्द्धा अ॒र्द्धा वा᳚ । \newline
39. वा॒ याव॑ती॒र् याव॑तीर् वा वा॒ याव॑तीः । \newline
40. याव॑तीर् वा वा॒ याव॑ती॒र् याव॑तीर् वा । \newline
41. वा ऽऽसा॑महा॒ आसा॑महै वा॒ वा ऽऽसा॑महै । \newline
42. आसा॑महा ए॒वैवा सा॑महा॒ आसा॑महा ए॒व । \newline
43. ए॒वेमा वि॒मा वे॒वैवेमौ । \newline
44. इ॒मौ द्वा॑द॒शौ द्वा॑द॒शा वि॒मा वि॒मौ द्वा॑द॒शौ । \newline
45. द्वा॒द॒शौ मासौ॒ मासौ᳚ द्वाद॒शौ द्वा॑द॒शौ मासौ᳚ । \newline
46. मासौ॑ संॅवथ्स॒रꣳ सं॑ॅवथ्स॒रम् मासौ॒ मासौ॑ संॅवथ्स॒रम् । \newline
47. सं॒ॅव॒थ्स॒रꣳ सं॒पाद्य॑ सं॒पाद्य॑ संॅवथ्स॒रꣳ सं॑ॅवथ्स॒रꣳ सं॒पाद्य॑ । \newline
48. सं॒ॅव॒थ्स॒रमिति॑ सं - व॒थ्स॒रम् । \newline
49. सं॒पा द्योदुथ् सं॒पाद्य॑ सं॒पा द्योत् । \newline
50. सं॒पाद्येति॑ सं - पाद्य॑ । \newline
51. उत् ति॑ष्ठाम तिष्ठा॒ मोदुत् ति॑ष्ठाम । \newline
52. ति॒ष्ठा॒ मेतीति॑ तिष्ठाम तिष्ठा॒ मेति॑ । \newline
53. इति॒ तासा॒म् तासा॒ मितीति॒ तासा᳚म् । \newline
54. तासा᳚म् द्वाद॒शे द्वा॑द॒शे तासा॒म् तासा᳚म् द्वाद॒शे । \newline

\textbf{Ghana Paata } \newline

1. गावो॒ वै वै गावो॒ गावो॒ वा ए॒त दे॒तद् वै गावो॒ गावो॒ वा ए॒तत् । \newline
2. वा ए॒त दे॒तद् वै वा ए॒तथ् स॒त्रꣳ स॒त्र मे॒तद् वै वा ए॒तथ् स॒त्रम् । \newline
3. ए॒तथ् स॒त्रꣳ स॒त्र मे॒त दे॒तथ् स॒त्र मा॑सता सत स॒त्र मे॒त दे॒तथ् स॒त्र मा॑सत । \newline
4. स॒त्र मा॑सता सत स॒त्रꣳ स॒त्र मा॑सता शृ॒ङ्गा अ॑शृ॒ङ्गा आ॑सत स॒त्रꣳ स॒त्र मा॑सता शृ॒ङ्गाः । \newline
5. आ॒स॒ता॒ शृ॒ङ्गा अ॑शृ॒ङ्गा आ॑सता सता शृ॒ङ्गाः स॒तीः स॒ती र॑शृ॒ङ्गा आ॑सता सता शृ॒ङ्गाः स॒तीः । \newline
6. अ॒शृ॒ङ्गाः स॒तीः स॒ती र॑शृ॒ङ्गा अ॑शृ॒ङ्गाः स॒तीः शृङ्गा॑णि॒ शृङ्गा॑णि स॒ती र॑शृ॒ङ्गा अ॑शृ॒ङ्गाः स॒तीः शृङ्गा॑णि । \newline
7. स॒तीः शृङ्गा॑णि॒ शृङ्गा॑णि स॒तीः स॒तीः शृङ्गा॑णि॒ सिषा॑सन्तीः॒ सिषा॑सन्तीः॒ शृङ्गा॑णि स॒तीः स॒तीः शृङ्गा॑णि॒ सिषा॑सन्तीः । \newline
8. शृङ्गा॑णि॒ सिषा॑सन्तीः॒ सिषा॑सन्तीः॒ शृङ्गा॑णि॒ शृङ्गा॑णि॒ सिषा॑सन्ती॒ स्तासा॒म् तासाꣳ॒॒ सिषा॑सन्तीः॒ शृङ्गा॑णि॒ शृङ्गा॑णि॒ सिषा॑सन्ती॒ स्तासा᳚म् । \newline
9. सिषा॑सन्ती॒ स्तासा॒म् तासाꣳ॒॒ सिषा॑सन्तीः॒ सिषा॑सन्ती॒ स्तासा॒म् दश॒ दश॒ तासाꣳ॒॒ सिषा॑सन्तीः॒ सिषा॑सन्ती॒ स्तासा॒म् दश॑ । \newline
10. तासा॒म् दश॒ दश॒ तासा॒म् तासा॒म् दश॒ मासा॒ मासा॒ दश॒ तासा॒म् तासा॒म् दश॒ मासाः᳚ । \newline
11. दश॒ मासा॒ मासा॒ दश॒ दश॒ मासा॒ निष॑ण्णा॒ निष॑ण्णा॒ मासा॒ दश॒ दश॒ मासा॒ निष॑ण्णाः । \newline
12. मासा॒ निष॑ण्णा॒ निष॑ण्णा॒ मासा॒ मासा॒ निष॑ण्णा॒ आस॒न् नास॒न् निष॑ण्णा॒ मासा॒ मासा॒ निष॑ण्णा॒ आसन्न्॑ । \newline
13. निष॑ण्णा॒ आस॒न् नास॒न् निष॑ण्णा॒ निष॑ण्णा॒ आस॒न् नथा थास॒न् निष॑ण्णा॒ निष॑ण्णा॒ आस॒न् नथ॑ । \newline
14. निष॑ण्णा॒ इति॒ नि - स॒न्नाः॒ । \newline
15. आस॒न् नथा थास॒न् नास॒न् नथ॒ शृङ्गा॑णि॒ शृङ्गा॒ ण्यथास॒न् नास॒न् नथ॒ शृङ्गा॑णि । \newline
16. अथ॒ शृङ्गा॑णि॒ शृङ्गा॒ ण्यथाथ॒ शृङ्गा᳚ ण्यजायन्ता जायन्त॒ शृङ्गा॒ ण्यथाथ॒ शृङ्गा᳚ ण्यजायन्त । \newline
17. शृङ्गा᳚ ण्यजायन्ता जायन्त॒ शृङ्गा॑णि॒ शृङ्गा᳚ ण्यजायन्त॒ ता स्ता अ॑जायन्त॒ शृङ्गा॑णि॒ शृङ्गा᳚ ण्यजायन्त॒ ताः । \newline
18. अ॒जा॒य॒न्त॒ ता स्ता अ॑जायन्ता जायन्त॒ ता अ॑ब्रुवन् नब्रुव॒न् ता अ॑जायन्ता जायन्त॒ ता अ॑ब्रुवन्न् । \newline
19. ता अ॑ब्रुवन् नब्रुव॒न् ता स्ता अ॑ब्रुव॒न् नरा॒थ्स्मा रा᳚थ्स्मा ब्रुव॒न् ता स्ता अ॑ब्रुव॒न् नरा᳚थ्स्म । \newline
20. अ॒ब्रु॒व॒न् नरा॒थ्स्मा रा᳚थ्स्मा ब्रुवन् नब्रुव॒न् नरा॒थ्स्मो दुद रा᳚थ्स्मा ब्रुवन् नब्रुव॒न् नरा॒थ्स्मोत् । \newline
21. अरा॒थ्स् मोदु दरा॒थ्स्मा रा॒थ्स्मोत् ति॑ष्ठाम तिष्ठा॒मो दरा॒थ्स्मा रा॒थ्स्मोत् ति॑ष्ठाम । \newline
22. उत् ति॑ष्ठाम तिष्ठा॒ मोदुत् ति॑ष्ठा॒मा वाव॑ तिष्ठा॒ मोदुत् ति॑ष्ठा॒ माव॑ । \newline
23. ति॒ष्ठा॒मा वाव॑ तिष्ठाम तिष्ठा॒ माव॒ तम् त मव॑ तिष्ठाम तिष्ठा॒ माव॒ तम् । \newline
24. अव॒ तम् त मवाव॒ तम् काम॒म् काम॒म् त मवाव॒ तम् काम᳚म् । \newline
25. तम् काम॒म् काम॒म् तम् तम् काम॑ मरुथ्स्मह्य रुथ्स्महि॒ काम॒म् तम् तम् काम॑ मरुथ्स्महि । \newline
26. काम॑ मरुथ्स्मह्य रुथ्स्महि॒ काम॒म् काम॑ मरुथ्स्महि॒ येन॒ येना॑ रुथ्स्महि॒ काम॒म् काम॑ मरुथ्स्महि॒ येन॑ । \newline
27. अ॒रु॒थ्स्म॒हि॒ येन॒ येना॑ रुथ्स्मह्य रुथ्स्महि॒ येन॒ कामे॑न॒ कामे॑न॒ येना॑ रुथ्स्मह्य रुथ्स्महि॒ येन॒ कामे॑न । \newline
28. येन॒ कामे॑न॒ कामे॑न॒ येन॒ येन॒ कामे॑न॒ न्यष॑दाम॒ न्यष॑दाम॒ कामे॑न॒ येन॒ येन॒ कामे॑न॒ न्यष॑दाम । \newline
29. कामे॑न॒ न्यष॑दाम॒ न्यष॑दाम॒ कामे॑न॒ कामे॑न॒ न्यष॑दा॒मेतीति॒ न्यष॑दाम॒ कामे॑न॒ कामे॑न॒ न्यष॑दा॒मेति॑ । \newline
30. न्यष॑दा॒मेतीति॒ न्यष॑दाम॒ न्यष॑दा॒मेति॒ तासा॒म् तासा॒ मिति॒ न्यष॑दाम॒ न्यष॑दा॒मेति॒ तासा᳚म् । \newline
31. न्यष॑दा॒मेति॑ नि - अस॑दाम । \newline
32. इति॒ तासा॒म् तासा॒ मितीति॒ तासा॑ मु वु॒ तासा॒ मितीति॒ तासा॑ मु । \newline
33. तासा॑ मु वु॒ तासा॒म् तासा॑ मु॒ तु तू॑ तासा॒म् तासा॑ मु॒ तु । \newline
34. उ॒ तु तू᳚ त्वै वै तू᳚ त्वै । \newline
35. त्वै वै तु त्वा अ॑ब्रुवन् नब्रुव॒न्॒. वै तु त्वा अ॑ब्रुवन्न् । \newline
36. वा अ॑ब्रुवन् नब्रुव॒न्॒. वै वा अ॑ब्रुवन् न॒र्द्धा अ॒र्द्धा अ॑ब्रुव॒न्॒. वै वा अ॑ब्रुवन् न॒र्द्धाः । \newline
37. अ॒ब्रु॒व॒न् न॒र्द्धा अ॒र्द्धा अ॑ब्रुवन् नब्रुवन् न॒र्द्धा वा॑ वा॒ ऽर्द्धा अ॑ब्रुवन् नब्रुवन् न॒र्द्धा वा᳚ । \newline
38. अ॒र्द्धा वा॑ वा॒ ऽर्द्धा अ॒र्द्धा वा॒ याव॑ती॒र् याव॑तीर् वा॒ ऽर्द्धा अ॒र्द्धा वा॒ याव॑तीः । \newline
39. वा॒ याव॑ती॒र् याव॑तीर् वा वा॒ याव॑तीर् वा वा॒ याव॑तीर् वा वा॒ याव॑तीर् वा । \newline
40. याव॑तीर् वा वा॒ याव॑ती॒र् याव॑ती॒र् वा ऽऽसा॑महा॒ आसा॑महै वा॒ याव॑ती॒र् याव॑ती॒र् वा ऽऽसा॑महै । \newline
41. वा ऽऽसा॑महा॒ आसा॑महै वा॒ वा ऽऽसा॑महा ए॒वैवा सा॑महै वा॒ वा ऽऽसा॑महा ए॒व । \newline
42. आसा॑महा ए॒वैवा सा॑महा॒ आसा॑महा ए॒वेमा वि॒मा वे॒वा सा॑महा॒ आसा॑महा ए॒वेमौ । \newline
43. ए॒वेमा वि॒मा वे॒वैवेमौ द्वा॑द॒शौ द्वा॑द॒शा वि॒मा वे॒वैवेमौ द्वा॑द॒शौ । \newline
44. इ॒मौ द्वा॑द॒शौ द्वा॑द॒शा वि॒मा वि॒मौ द्वा॑द॒शौ मासौ॒ मासौ᳚ द्वाद॒शा वि॒मा वि॒मौ द्वा॑द॒शौ मासौ᳚ । \newline
45. द्वा॒द॒शौ मासौ॒ मासौ᳚ द्वाद॒शौ द्वा॑द॒शौ मासौ॑ संॅवथ्स॒रꣳ सं॑ॅवथ्स॒रम् मासौ᳚ द्वाद॒शौ द्वा॑द॒शौ मासौ॑ संॅवथ्स॒रम् । \newline
46. मासौ॑ संॅवथ्स॒रꣳ सं॑ॅवथ्स॒रम् मासौ॒ मासौ॑ संॅवथ्स॒रꣳ सं॒पाद्य॑ सं॒पाद्य॑ संॅवथ्स॒रम् मासौ॒ मासौ॑ संॅवथ्स॒रꣳ सं॒पाद्य॑ । \newline
47. सं॒ॅव॒थ्स॒रꣳ सं॒पाद्य॑ सं॒पाद्य॑ संॅवथ्स॒रꣳ सं॑ॅवथ्स॒रꣳ सं॒पा द्योदुथ् सं॒पाद्य॑ संॅवथ्स॒रꣳ सं॑ॅवथ्स॒रꣳ सं॒पा द्योत् । \newline
48. सं॒ॅव॒थ्स॒रमिति॑ सं - व॒थ्स॒रम् । \newline
49. सं॒पा द्योदुथ् सं॒पाद्य॑ सं॒पा द्योत् ति॑ष्ठाम तिष्ठा॒ मोथ् सं॒पाद्य॑ सं॒पा द्योत् ति॑ष्ठाम । \newline
50. सं॒पाद्येति॑ सं - पाद्य॑ । \newline
51. उत् ति॑ष्ठाम तिष्ठा॒मो दुत् ति॑ष्ठा॒ मेतीति॑ तिष्ठा॒मो दुत् ति॑ष्ठा॒ मेति॑ । \newline
52. ति॒ष्ठा॒ मेतीति॑ तिष्ठाम तिष्ठा॒मेति॒ तासा॒म् तासा॒ मिति॑ तिष्ठाम तिष्ठा॒मेति॒ तासा᳚म् । \newline
53. इति॒ तासा॒म् तासा॒ मितीति॒ तासा᳚म् द्वाद॒शे द्वा॑द॒शे तासा॒ मितीति॒ तासा᳚म् द्वाद॒शे । \newline
54. तासा᳚म् द्वाद॒शे द्वा॑द॒शे तासा॒म् तासा᳚म् द्वाद॒शे मा॒सि मा॒सि द्वा॑द॒शे तासा॒म् तासा᳚म् द्वाद॒शे मा॒सि । \newline
\pagebreak
\markright{ TS 7.5.2.2  \hfill https://www.vedavms.in \hfill}

\section{ TS 7.5.2.2 }

\textbf{TS 7.5.2.2 } \newline
\textbf{Samhita Paata} \newline

द्वाद॒शे मा॒सि शृङ्गा॑णि॒ प्राव॑र्तन्त श्र॒द्धया॒ वाऽश्र॑द्धया वा॒ ता इ॒मा यास्तू॑प॒रा उ॒भय्यो॒ वाव ता आ᳚र्द्ध्नुव॒न्॒. याश्च॒ शृङ्गा॒ण्यस॑न्व॒न्॒. याश्चोर्ज॑म॒वारु॑न्धत॒र्द्ध्नोति॑ द॒शसु॑ मा॒सू᳚त्तिष्ठ॑न्नृ॒द्ध्नोति॑ द्वाद॒शसु॒ य ए॒वं ॅवेद॑ प॒देन॒ खलु॒ वा ए॒ते य॑न्ति वि॒न्दति॒ खलु॒ वै प॒देन॒ यन् तद्वा ए॒तदृ॒द्धमय॑नं॒ तस्मा॑ ( ) दे॒तद्-गो॒सनि॑ ॥ \newline

\textbf{Pada Paata} \newline

द्वा॒द॒शे । मा॒सि । शृङ्गा॑णि । प्रेति॑ । अ॒व॒र्त॒न्त॒ । श्र॒द्धयेति॑ श्रत्-धया᳚ । वा॒ । अश्र॑द्ध॒येत्यश्र॑त् - ध॒या॒ । वा॒ । ताः । इ॒माः । याः । तू॒प॒राः । उ॒भय्यः॑ । वाव । ताः । आ॒द्‌र्ध्नु॒व॒न्न् । याः । च॒ । शृङ्गा॑णि । अस॑न्वन्न् । याः । च॒ । ऊर्ज᳚म् । अ॒वारु॑न्ध॒तेत्य॑व - अरु॑न्धत । ऋ॒द्ध्नोति॑ । द॒शस्विति॑ द॒श - सु॒ । मा॒सु । उ॒त्तिष्ठ॒न्नित्यु॑त्-तिष्ठन्न्॑ । ऋ॒द्ध्नोति॑ । द्वा॒द॒शस्विति॑ द्वाद॒श - सु॒ । यः । ए॒वम् । वेद॑ । प॒देन॑ । खलु॑ । वै । ए॒ते । य॒न्ति॒ । वि॒न्दति॑ । खलु॑ । वै । प॒देन॑ । यन्न् । तत् । वै । ए॒तत् । ऋ॒द्धम् । अय॑नम् । तस्मा᳚त् ( ) । ए॒तत् । गो॒सनीति॑ गो - सनि॑ ॥  \newline


\textbf{Krama Paata} \newline

द्वा॒द॒शे मा॒सि । मा॒सि शृङ्‍गा॑णि । शृङ्‍गा॑णि॒ प्र । प्राव॑र्तन्त । अ॒व॒र्त॒न्त॒ श्र॒द्धया᳚ । श्र॒द्धया॑ वा । श्र॒द्धयेति॑ श्रत् - धया᳚ । वाऽश्र॑द्धया । अश्र॑द्धया वा । अश्र॑द्ध॒येत्यश्र॑त् - ध॒या॒ । वा॒ ताः । ता इ॒माः । इ॒मा याः । यास्तू॑प॒राः । तू॒प॒रा उ॒भय्यः॑ । उ॒भय्यो॒ वाव । वाव ताः । ता आ᳚र्द्ध्नुवन्न् । आ॒र्द्धु॒व॒न्॒. याः । याश्च॑ । 
च॒ शृङ्‍गा॑णि । शृङ्‍गा॒ण्यस॑न्वन्न् । अस॑न्व॒न्॒. याः । याश्च॑ । चोर्ज᳚म् । ऊर्ज॑म॒वारु॑न्धत । अ॒वारु॑न्धत॒र्द्ध्नोति॑ । अ॒वारु॑न्ध॒तेत्य॑व - अरु॑न्धत । ऋ॒द्ध्नोति॑ द॒शसु॑ । द॒शसु॑ मा॒सु । द॒शस्विति॑ द॒श - सु॒ । मा॒सू᳚त्तिष्ठन्न्॑ । उ॒त्तिष्ठ॑न्‌नृ॒द्ध्नोति॑ । उ॒त्तिष्ठ॒न्नित्यु॑त् - तिष्ठन्न्॑ । ऋ॒द्ध्नोति॑ द्वाद॒शसु॑ । द्वा॒द॒शसु॒ यः । द्वा॒द॒शस्विति॑ द्वाद॒श - सु॒ । य ए॒वम् । ए॒वम् ॅवेद॑ । वेद॑ प॒देन॑ । प॒देन॒ खलु॑ । खलु॒ वै । वा ए॒ते । ए॒ते य॑न्ति । य॒न्ति॒ वि॒न्दति॑ । वि॒न्दति॒ खलु॑ । खलु॒ वै । वै प॒देन॑ । प॒देन॒ यन्न् । यन् तत् । तद् वै । वा ए॒तत् । ए॒तदृ॒द्धम् । ऋ॒द्धमय॑नम् । अय॑न॒म् तस्मा᳚त् ( ) । तस्मा॑दे॒तत् । ए॒तद् गो॒सनि॑ । गो॒सनीति॑ गो - सनि॑ । \newline

\textbf{Jatai Paata} \newline

1. द्वा॒द॒शे मा॒सि मा॒सि द्वा॑द॒शे द्वा॑द॒शे मा॒सि । \newline
2. मा॒सि शृङ्गा॑णि॒ शृङ्गा॑णि मा॒सि मा॒सि शृङ्गा॑णि । \newline
3. शृङ्गा॑णि॒ प्र प्र शृङ्गा॑णि॒ शृङ्गा॑णि॒ प्र । \newline
4. प्रा व॑र्तन्ता वर्तन्त॒ प्र प्रा व॑र्तन्त । \newline
5. अ॒व॒र्त॒न्त॒ श्र॒द्धया᳚ श्र॒द्धया॑ ऽवर्तन्ता वर्तन्त श्र॒द्धया᳚ । \newline
6. श्र॒द्धया॑ वा वा श्र॒द्धया᳚ श्र॒द्धया॑ वा । \newline
7. श्र॒द्धयेति॑ श्रत् - धया᳚ । \newline
8. वा ऽश्र॑द्ध॒या ऽश्र॑द्धया वा॒ वा ऽश्र॑द्धया । \newline
9. अश्र॑द्धया वा॒ वा ऽश्र॑द्ध॒या ऽश्र॑द्धया वा । \newline
10. अश्र॑द्ध॒येत्यश्र॑त् - ध॒या॒ । \newline
11. वा॒ ता स्ता वा॑ वा॒ ताः । \newline
12. ता इ॒मा इ॒मा स्ता स्ता इ॒माः । \newline
13. इ॒मा या या इ॒मा इ॒मा याः । \newline
14. या स्तू॑प॒रा स्तू॑प॒रा या या स्तू॑प॒राः । \newline
15. तू॒प॒रा उ॒भय्य॑ उ॒भय्य॑ स्तूप॒रा स्तू॑प॒रा उ॒भय्यः॑ । \newline
16. उ॒भय्यो॒ वाव वावोभय्य॑ उ॒भय्यो॒ वाव । \newline
17. वाव ता स्ता वाव वाव ताः । \newline
18. ता आ᳚र्द्ध्नुवन् नार्द्ध्नुव॒न् ता स्ता आ᳚र्द्ध्नुवन्न् । \newline
19. आ॒र्द्ध्नु॒व॒न्॒. या या आ᳚र्द्ध्नुवन् नार्द्ध्नुव॒न्॒. याः । \newline
20. या श्च॑ च॒ या या श्च॑ । \newline
21. च॒ शृङ्गा॑णि॒ शृङ्गा॑णि च च॒ शृङ्गा॑णि । \newline
22. शृङ्गा॒ ण्यस॑न्व॒न् नस॑न्व॒ञ् छृङ्गा॑णि॒ शृङ्गा॒ ण्यस॑न्वन्न् । \newline
23. अस॑न्व॒न्॒. या या अस॑न्व॒न् नस॑न्व॒न्॒. याः । \newline
24. या श्च॑ च॒ या या श्च॑ । \newline
25. चोर्ज॒ मूर्ज॑म् च॒ चोर्ज᳚म् । \newline
26. ऊर्ज॑ म॒वारु॑न्धता॒ वारु॑न्ध॒ तोर्ज॒ मूर्ज॑ म॒वारु॑न्धत । \newline
27. अ॒वारु॑न्धत॒ र्‌द्ध्नो त्यृ॒द्ध्नो त्य॒वारु॑न्धता॒ वारु॑न्धत॒ र्‌द्ध्नोति॑ । \newline
28. अ॒वारु॑न्ध॒तेत्य॑व - अरु॑न्धत । \newline
29. ऋ॒द्ध्नोति॑ द॒शसु॑ द॒श स्वृ॒द्ध्नो त्यृ॒द्ध्नोति॑ द॒शसु॑ । \newline
30. द॒शसु॑ मा॒सु मा॒सु द॒शसु॑ द॒शसु॑ मा॒सु । \newline
31. द॒शस्विति॑ द॒श - सु॒ । \newline
32. मा॒सू᳚त्तिष्ठ॑न् नु॒त्तिष्ठ॑न् मा॒सु मा॒सू᳚त्तिष्ठन्न्॑ । \newline
33. उ॒त्तिष्ठ॑न् नृ॒द्ध्नो त्यृ॒द्ध्नो त्यु॒त्तिष्ठ॑न् नु॒त्तिष्ठ॑न् नृ॒द्ध्नोति॑ । \newline
34. उ॒त्तिष्ठ॒न्नित्यु॑त् - तिष्ठन्न्॑ । \newline
35. ऋ॒द्ध्नोति॑ द्वाद॒शसु॑ द्वाद॒श स्वृ॒द्ध्नो त्यृ॒द्ध्नोति॑ द्वाद॒शसु॑ । \newline
36. द्वा॒द॒शसु॒ यो यो द्वा॑द॒शसु॑ द्वाद॒शसु॒ यः । \newline
37. द्वा॒द॒शस्विति॑ द्वाद॒श - सु॒ । \newline
38. य ए॒व मे॒वं ॅयो य ए॒वम् । \newline
39. ए॒वं ॅवेद॒ वेदै॒व मे॒वं ॅवेद॑ । \newline
40. वेद॑ प॒देन॑ प॒देन॒ वेद॒ वेद॑ प॒देन॑ । \newline
41. प॒देन॒ खलु॒ खलु॑ प॒देन॑ प॒देन॒ खलु॑ । \newline
42. खलु॒ वै वै खलु॒ खलु॒ वै । \newline
43. वा ए॒त ए॒ते वै वा ए॒ते । \newline
44. ए॒ते य॑न्ति यन्त्ये॒त ए॒ते य॑न्ति । \newline
45. य॒न्ति॒ वि॒न्दति॑ वि॒न्दति॑ यन्ति यन्ति वि॒न्दति॑ । \newline
46. वि॒न्दति॒ खलु॒ खलु॑ वि॒न्दति॑ वि॒न्दति॒ खलु॑ । \newline
47. खलु॒ वै वै खलु॒ खलु॒ वै । \newline
48. वै प॒देन॑ प॒देन॒ वै वै प॒देन॑ । \newline
49. प॒देन॒ यन्. यन् प॒देन॑ प॒देन॒ यन्न् । \newline
50. यन् तत् तद् यन्. यन् तत् । \newline
51. तद् वै वै तत् तद् वै । \newline
52. वा ए॒त दे॒तद् वै वा ए॒तत् । \newline
53. ए॒त दृ॒द्ध मृ॒द्ध मे॒त दे॒त दृ॒द्धम् । \newline
54. ऋ॒द्ध मय॑न॒ मय॑न मृ॒द्ध मृ॒द्ध मय॑नम् । \newline
55. अय॑न॒म् तस्मा॒त् तस्मा॒ दय॑न॒ मय॑न॒म् तस्मा᳚त् । \newline
56. तस्मा॑ दे॒त दे॒तत् तस्मा॒त् तस्मा॑ दे॒तत् । \newline
57. ए॒तद् गो॒सनि॑ गो॒स न्ये॒त दे॒तद् गो॒सनि॑ । \newline
58. गो॒सनीति॑ गो - सनि॑ । \newline

\textbf{Ghana Paata } \newline

1. द्वा॒द॒शे मा॒सि मा॒सि द्वा॑द॒शे द्वा॑द॒शे मा॒सि शृङ्गा॑णि॒ शृङ्गा॑णि मा॒सि द्वा॑द॒शे द्वा॑द॒शे मा॒सि शृङ्गा॑णि । \newline
2. मा॒सि शृङ्गा॑णि॒ शृङ्गा॑णि मा॒सि मा॒सि शृङ्गा॑णि॒ प्र प्र शृङ्गा॑णि मा॒सि मा॒सि शृङ्गा॑णि॒ प्र । \newline
3. शृङ्गा॑णि॒ प्र प्र शृङ्गा॑णि॒ शृङ्गा॑णि॒ प्राव॑र्तन्ता वर्तन्त॒ प्र शृङ्गा॑णि॒ शृङ्गा॑णि॒ प्राव॑र्तन्त । \newline
4. प्राव॑र्तन्ता वर्तन्त॒ प्र प्राव॑र्तन्त श्र॒द्धया᳚ श्र॒द्धया॑ ऽवर्तन्त॒ प्र प्राव॑र्तन्त श्र॒द्धया᳚ । \newline
5. अ॒व॒र्त॒न्त॒ श्र॒द्धया᳚ श्र॒द्धया॑ ऽवर्तन्ता वर्तन्त श्र॒द्धया॑ वा वा श्र॒द्धया॑ ऽवर्तन्ता वर्तन्त श्र॒द्धया॑ वा । \newline
6. श्र॒द्धया॑ वा वा श्र॒द्धया᳚ श्र॒द्धया॒ वा ऽश्र॑द्ध॒या ऽश्र॑द्धया वा श्र॒द्धया᳚ श्र॒द्धया॒ वा ऽश्र॑द्धया । \newline
7. श्र॒द्धयेति॑ श्रत् - धया᳚ । \newline
8. वा ऽश्र॑द्ध॒या ऽश्र॑द्धया वा॒ वा ऽश्र॑द्धया वा॒ वा ऽश्र॑द्धया वा॒ वा ऽश्र॑द्धया वा । \newline
9. अश्र॑द्धया वा॒ वा ऽश्र॑द्ध॒या ऽश्र॑द्धया वा॒ ता स्ता वा ऽश्र॑द्ध॒या ऽश्र॑द्धया वा॒ ताः । \newline
10. अश्र॑द्ध॒येत्यश्र॑त् - ध॒या॒ । \newline
11. वा॒ ता स्ता वा॑ वा॒ ता इ॒मा इ॒मा स्ता वा॑ वा॒ ता इ॒माः । \newline
12. ता इ॒मा इ॒मा स्ता स्ता इ॒मा या या इ॒मा स्ता स्ता इ॒मा याः । \newline
13. इ॒मा या या इ॒मा इ॒मा या स्तू॑प॒रा स्तू॑प॒रा या इ॒मा इ॒मा या स्तू॑प॒राः । \newline
14. या स्तू॑प॒रा स्तू॑प॒रा या या स्तू॑प॒रा उ॒भय्य॑ उ॒भय्य॑ स्तूप॒रा या या स्तू॑प॒रा उ॒भय्यः॑ । \newline
15. तू॒प॒रा उ॒भय्य॑ उ॒भय्य॑ स्तूप॒रा स्तू॑प॒रा उ॒भय्यो॒ वाव वावो भय्य॑ स्तूप॒रा स्तू॑प॒रा उ॒भय्यो॒ वाव । \newline
16. उ॒भय्यो॒ वाव वावोभय्य॑ उ॒भय्यो॒ वाव ता स्ता वावोभय्य॑ उ॒भय्यो॒ वाव ताः । \newline
17. वाव ता स्ता वाव वाव ता आ᳚र्द्ध्नुवन् नार्द्ध्नुव॒न् ता वाव वाव ता आ᳚र्द्ध्नुवन्न् । \newline
18. ता आ᳚र्द्ध्नुवन् नार्द्ध्नुव॒न् ता स्ता आ᳚र्द्ध्नुव॒न्॒. या या आ᳚र्द्ध्नुव॒न् ता स्ता आ᳚र्द्ध्नुव॒न्॒. याः । \newline
19. आ॒र्द्ध्नु॒व॒न्॒. या या आ᳚र्द्ध्नुवन् नार्द्ध्नुव॒न्॒. याश्च॑ च॒ या आ᳚र्द्ध्नुवन् नार्द्ध्नुव॒न्॒. याश्च॑ । \newline
20. याश्च॑ च॒ या याश्च॒ शृङ्गा॑णि॒ शृङ्गा॑णि च॒ या याश्च॒ शृङ्गा॑णि । \newline
21. च॒ शृङ्गा॑णि॒ शृङ्गा॑णि च च॒ शृङ्गा॒ ण्यस॑न्व॒न् नस॑न्व॒ञ् छृङ्गा॑णि च च॒ शृङ्गा॒ ण्यस॑न्वन्न् । \newline
22. शृङ्गा॒ ण्यस॑न्व॒न् नस॑न्व॒ञ् छृङ्गा॑णि॒ शृङ्गा॒ ण्यस॑न्व॒न्॒. या या अस॑न्व॒ञ् छृङ्गा॑णि॒ शृङ्गा॒ ण्यस॑न्व॒न्॒. याः । \newline
23. अस॑न्व॒न्॒. या या अस॑न्व॒न् नस॑न्व॒न्॒. या श्च॑ च॒ या अस॑न्व॒न् नस॑न्व॒न्॒. याश्च॑ । \newline
24. या श्च॑ च॒ या या श्चोर्ज॒ मूर्ज॑म् च॒ या या श्चोर्ज᳚म् । \newline
25. चोर्ज॒ मूर्ज॑म् च॒ चोर्ज॑ म॒वारु॑न्धता॒ वारु॑न्ध॒ तोर्ज॑म् च॒ चोर्ज॑ म॒वारु॑न्धत । \newline
26. ऊर्ज॑ म॒वारु॑न्धता॒ वारु॑न्ध॒ तोर्ज॒ मूर्ज॑ म॒वारु॑न्धत॒ र्‌द्ध्नो त्यृ॒द्ध्नो त्य॒वारु॑न्ध॒ तोर्ज॒ मूर्ज॑ म॒वारु॑न्धत॒ र्‌द्ध्नोति॑ । \newline
27. अ॒वारु॑न्धत॒ र्‌द्ध्नो त्यृ॒द्ध्नो त्य॒वारु॑न्धता॒ वारु॑न्धत॒ र्‌द्ध्नोति॑ द॒शसु॑ द॒श स्वृ॒द्ध्नो त्य॒वारु॑न्धता॒ वारु॑न्धत॒ र्‌द्ध्नोति॑ द॒शसु॑ । \newline
28. अ॒वारु॑न्ध॒तेत्य॑व - अरु॑न्धत । \newline
29. ऋ॒द्ध्नोति॑ द॒शसु॑ द॒श स्वृ॒द्ध्नो त्यृ॒द्ध्नोति॑ द॒शसु॑ मा॒सु मा॒सु द॒श स्वृ॒द्ध्नो त्यृ॒द्ध्नोति॑ द॒शसु॑ मा॒सु । \newline
30. द॒शसु॑ मा॒सु मा॒सु द॒शसु॑ द॒शसु॑ मा॒सू᳚त्तिष्ठ॑न् नु॒त्तिष्ठ॑न् मा॒सु द॒शसु॑ द॒शसु॑ मा॒सू᳚त्तिष्ठन्न्॑ । \newline
31. द॒शस्विति॑ द॒श - सु॒ । \newline
32. मा॒सू᳚ त्तिष्ठ॑न् नु॒त्तिष्ठ॑न् मा॒सु मा॒सू᳚ त्तिष्ठ॑न् नृ॒द्ध्नो त्यृ॒द्ध्नो त्यु॒त्तिष्ठ॑न् मा॒सु मा॒सू᳚ त्तिष्ठ॑न् नृ॒द्ध्नोति॑ । \newline
33. उ॒त्तिष्ठ॑न् नृ॒द्ध्नो त्यृ॒द्ध्नो त्यु॒त्तिष्ठ॑न् नु॒त्तिष्ठ॑न् नृ॒द्ध्नोति॑ द्वाद॒शसु॑ द्वाद॒श स्वृ॒द्ध्नो त्यु॒त्तिष्ठ॑न् नु॒त्तिष्ठ॑न् नृ॒द्ध्नोति॑ द्वाद॒शसु॑ । \newline
34. उ॒त्तिष्ठ॒न्नित्यु॑त् - तिष्ठन्न्॑ । \newline
35. ऋ॒द्ध्नोति॑ द्वाद॒शसु॑ द्वाद॒श स्वृ॒द्ध्नो त्यृ॒द्ध्नोति॑ द्वाद॒शसु॒ यो यो द्वा॑द॒श स्वृ॒द्ध्नो त्यृ॒द्ध्नोति॑ द्वाद॒शसु॒ यः । \newline
36. द्वा॒द॒शसु॒ यो यो द्वा॑द॒शसु॑ द्वाद॒शसु॒ य ए॒व मे॒वं ॅयो द्वा॑द॒शसु॑ द्वाद॒शसु॒ य ए॒वम् । \newline
37. द्वा॒द॒शस्विति॑ द्वाद॒श - सु॒ । \newline
38. य ए॒व मे॒वं ॅयो य ए॒वं ॅवेद॒ वेदै॒वं ॅयो य ए॒वं ॅवेद॑ । \newline
39. ए॒वं ॅवेद॒ वेदै॒व मे॒वं ॅवेद॑ प॒देन॑ प॒देन॒ वेदै॒व मे॒वं ॅवेद॑ प॒देन॑ । \newline
40. वेद॑ प॒देन॑ प॒देन॒ वेद॒ वेद॑ प॒देन॒ खलु॒ खलु॑ प॒देन॒ वेद॒ वेद॑ प॒देन॒ खलु॑ । \newline
41. प॒देन॒ खलु॒ खलु॑ प॒देन॑ प॒देन॒ खलु॒ वै वै खलु॑ प॒देन॑ प॒देन॒ खलु॒ वै । \newline
42. खलु॒ वै वै खलु॒ खलु॒ वा ए॒त ए॒ते वै खलु॒ खलु॒ वा ए॒ते । \newline
43. वा ए॒त ए॒ते वै वा ए॒ते य॑न्ति यन्त्ये॒ते वै वा ए॒ते य॑न्ति । \newline
44. ए॒ते य॑न्ति यन्त्ये॒त ए॒ते य॑न्ति वि॒न्दति॑ वि॒न्दति॑ यन्त्ये॒त ए॒ते य॑न्ति वि॒न्दति॑ । \newline
45. य॒न्ति॒ वि॒न्दति॑ वि॒न्दति॑ यन्ति यन्ति वि॒न्दति॒ खलु॒ खलु॑ वि॒न्दति॑ यन्ति यन्ति वि॒न्दति॒ खलु॑ । \newline
46. वि॒न्दति॒ खलु॒ खलु॑ वि॒न्दति॑ वि॒न्दति॒ खलु॒ वै वै खलु॑ वि॒न्दति॑ वि॒न्दति॒ खलु॒ वै । \newline
47. खलु॒ वै वै खलु॒ खलु॒ वै प॒देन॑ प॒देन॒ वै खलु॒ खलु॒ वै प॒देन॑ । \newline
48. वै प॒देन॑ प॒देन॒ वै वै प॒देन॒ यन्. यन् प॒देन॒ वै वै प॒देन॒ यन्न् । \newline
49. प॒देन॒ यन्. यन् प॒देन॑ प॒देन॒ यन् तत् तद् यन् प॒देन॑ प॒देन॒ यन् तत् । \newline
50. यन् तत् तद् यन्. यन् तद् वै वै तद् यन्. यन् तद् वै । \newline
51. तद् वै वै तत् तद् वा ए॒त दे॒तद् वै तत् तद् वा ए॒तत् । \newline
52. वा ए॒त दे॒तद् वै वा ए॒त दृ॒द्ध मृ॒द्ध मे॒तद् वै वा ए॒त दृ॒द्धम् । \newline
53. ए॒त दृ॒द्ध मृ॒द्ध मे॒त दे॒त दृ॒द्ध मय॑न॒ मय॑न मृ॒द्ध मे॒त दे॒त दृ॒द्ध मय॑नम् । \newline
54. ऋ॒द्ध मय॑न॒ मय॑न मृ॒द्ध मृ॒द्ध मय॑न॒म् तस्मा॒त् तस्मा॒ दय॑न मृ॒द्ध मृ॒द्ध मय॑न॒म् तस्मा᳚त् । \newline
55. अय॑न॒म् तस्मा॒त् तस्मा॒ दय॑न॒ मय॑न॒म् तस्मा॑ दे॒त दे॒तत् तस्मा॒ दय॑न॒ मय॑न॒म् तस्मा॑ दे॒तत् । \newline
56. तस्मा॑ दे॒त दे॒तत् तस्मा॒त् तस्मा॑ दे॒तद् गो॒सनि॑ गो॒स न्ये॒तत् तस्मा॒त् तस्मा॑ दे॒तद् गो॒सनि॑ । \newline
57. ए॒तद् गो॒सनि॑ गो॒स न्ये॒त दे॒तद् गो॒सनि॑ । \newline
58. गो॒सनीति॑ गो - सनि॑ । \newline
\pagebreak
\markright{ TS 7.5.3.1  \hfill https://www.vedavms.in \hfill}

\section{ TS 7.5.3.1 }

\textbf{TS 7.5.3.1 } \newline
\textbf{Samhita Paata} \newline

प्र॒थ॒मे मा॒सि पृ॒ष्ठान्युप॑ यन्ति मद्ध्य॒म उप॑ यन्त्युत्त॒म उप॑ यन्ति॒ तदा॑हु॒र्यां ॅवै त्रिरेक॒स्याह्न॑ उप॒सीद॑न्ति द॒ह्रं ॅवै साऽप॑राभ्यां॒ दोहा᳚भ्यां दु॒हेऽथ॒ कुतः॒ सा धो᳚क्ष्यते॒ यां द्वाद॑श॒ कृत्व॑ उप॒सीद॒न्तीति॑ संॅवथ्स॒रꣳ स॒पांद्यो᳚त्त॒मे मा॒सि स॒कृत् पृ॒ष्ठान्युपे॑यु॒स्तद्-यज॑माना य॒ज्ञ्ं प॒शूनव॑ रुन्धते समु॒द्रं ॅवा - [  ] \newline

\textbf{Pada Paata} \newline

प्र॒थ॒मे । मा॒सि । पृ॒ष्ठानि॑ । उपेति॑ । य॒न्ति॒ । म॒द्ध्य॒मे । उपेति॑ । य॒न्ति॒ । उ॒त्त॒म इत्यु॑त् - त॒मे । उपेति॑ । य॒न्ति॒ । तत् । आ॒हुः॒ । याम् । वै । त्रिः । एक॑स्य । अह्नः॑ । उ॒प॒सीद॒न्तीत्यु॑प - सीद॑न्ति । द॒ह्रम् । वै । सा । अप॑राभ्याम् । दोहा᳚भ्याम् । दु॒हे॒ । अथ॑ । कुतः॑ । सा । धो॒क्ष्य॒ते॒ । याम् । द्वाद॑श । कृत्वः॑ । उ॒प॒सीद॒न्तीत्यु॑प - सीद॑न्ति । इति॑ । सं॒ॅव॒थ्स॒रमिति॑ सं - व॒थ्स॒रम् । स॒पांद्येति॑ सं - पाद्य॑ । उ॒त्त॒म इत्यु॑त् - त॒मे । मा॒सि । स॒कृत् । पृ॒ष्ठानि॑ । उपेति॑ । इ॒युः॒ । तत् । यज॑मानाः । य॒ज्ञ्म् । प॒शून् । अवेति॑ । रु॒न्ध॒ते॒ । स॒मु॒द्रम् । वै ।  \newline


\textbf{Krama Paata} \newline

प्र॒थ॒मे मा॒सि । मा॒सि पृ॒ष्ठानि॑ । पृ॒ष्ठान्युप॑ । उप॑ यन्ति । य॒न्ति॒ म॒द्ध्य॒मे । म॒द्ध्य॒म उप॑ । उप॑ यन्ति । य॒न्त्यु॒त्त॒मे । उ॒त्त॒म उप॑ । उ॒त्त॒म इत्यु॑त् - त॒मे । उप॑ यन्ति । य॒न्ति॒ तत् । तदा॑हुः । आ॒हु॒र् याम् । याम् ॅवै । वै त्रिः । त्रिरेक॑स्य । एक॒स्याह्नः॑ । अह्न॑ उप॒सीद॑न्ति । उ॒प॒सीद॑न्ति द॒ह्रम् । उ॒प॒सीद॒न्तीत्यु॑प - सीद॑न्ति । द॒ह्रम् ॅवै । वै सा । साऽप॑राभ्याम् । अप॑राभ्या॒म् दोहा᳚भ्याम् । दोहा᳚भ्याम् दुहे । दु॒हेऽथ॑ । अथ॒ कुतः॑ । कुतः॒ सा । सा धो᳚क्ष्यते । धो॒क्ष्य॒ते॒ याम् । याम् द्वाद॑श । द्वाद॑श॒ कृत्वः॑ । कृत्व॑ उप॒सीद॑न्ति । उ॒प॒सीद॒न्तीति॑ । उ॒प॒सीद॒न्तीत्यु॑प - सीद॑न्ति । इति॑ सम्ॅवथ्स॒रम् । स॒म्ॅव॒थ्स॒रꣳ स॒म्पाद्य॑ । स॒म्ॅव॒थ्स॒रमिति॑ सम् - व॒थ्स॒रम् । स॒म्पाद्यो᳚त्त॒मे । स॒म्पाद्येति॑ सम् - पाद्य॑ । उ॒त्त॒मे मा॒सि । उ॒त्त॒म इत्यु॑त् - त॒मे । मा॒सि स॒कृत् । स॒कृत् पृ॒ष्ठानि॑ । पृ॒ष्ठान्युप॑ । उपे॑युः । इ॒यु॒स्तत् । तद् यज॑मानाः । यज॑माना य॒ज्ञ्म् । य॒ज्ञ्म् प॒शून् । प॒शूनव॑ । अव॑ रुन्धते । रु॒न्ध॒ते॒ स॒मु॒द्रम् । स॒मु॒द्रम् ॅवै ( ) । वा ए॒ते \newline

\textbf{Jatai Paata} \newline

1. प्र॒थ॒मे मा॒सि मा॒सि प्र॑थ॒मे प्र॑थ॒मे मा॒सि । \newline
2. मा॒सि पृ॒ष्ठानि॑ पृ॒ष्ठानि॑ मा॒सि मा॒सि पृ॒ष्ठानि॑ । \newline
3. पृ॒ष्ठा न्युपोप॑ पृ॒ष्ठानि॑ पृ॒ष्ठा न्युप॑ । \newline
4. उप॑ यन्ति य॒न्त्युपोप॑ यन्ति । \newline
5. य॒न्ति॒ म॒द्ध्य॒मे म॑द्ध्य॒मे य॑न्ति यन्ति मद्ध्य॒मे । \newline
6. म॒द्ध्य॒म उपोप॑ मद्ध्य॒मे म॑द्ध्य॒म उप॑ । \newline
7. उप॑ यन्ति य॒न्त्युपोप॑ यन्ति । \newline
8. य॒न्त्यु॒त्त॒म उ॑त्त॒मे य॑न्ति यन्त्युत्त॒मे । \newline
9. उ॒त्त॒म उपोपो᳚त्त॒म उ॑त्त॒म उप॑ । \newline
10. उ॒त्त॒म इत्यु॑त् - त॒मे । \newline
11. उप॑ यन्ति य॒न्त्युपोप॑ यन्ति । \newline
12. य॒न्ति॒ तत् तद् य॑न्ति यन्ति॒ तत् । \newline
13. तदा॑हु राहु॒ स्तत् तदा॑हुः । \newline
14. आ॒हु॒र् यां ॅया मा॑हु राहु॒र् याम् । \newline
15. यां ॅवै वै यां ॅयां ॅवै । \newline
16. वै त्रि स्त्रिर् वै वै त्रिः । \newline
17. त्रि रेक॒ स्यैक॑स्य॒ त्रि स्त्रि रेक॑स्य । \newline
18. एक॒ स्याह्नो ऽह्न॒ एक॒ स्यैक॒ स्याह्नः॑ । \newline
19. अह्न॑ उप॒सीद॑ न्त्युप॒सीद॒ न्त्यह्नो ऽह्न॑ उप॒सीद॑न्ति । \newline
20. उ॒प॒सीद॑न्ति द॒ह्रम् द॒ह्र मु॑प॒सीद॑ न्त्युप॒सीद॑न्ति द॒ह्रम् । \newline
21. उ॒प॒सीद॒न्तीत्यु॑प - सीद॑न्ति । \newline
22. द॒ह्रं ॅवै वै द॒ह्रम् द॒ह्रं ॅवै । \newline
23. वै सा सा वै वै सा । \newline
24. सा ऽप॑राभ्या॒ मप॑राभ्याꣳ॒॒ सा सा ऽप॑राभ्याम् । \newline
25. अप॑राभ्या॒म् दोहा᳚भ्या॒म् दोहा᳚भ्या॒ मप॑राभ्या॒ मप॑राभ्या॒म् दोहा᳚भ्याम् । \newline
26. दोहा᳚भ्याम् दुहे दुहे॒ दोहा᳚भ्या॒म् दोहा᳚भ्याम् दुहे । \newline
27. दु॒हे ऽथाथ॑ दुहे दु॒हे ऽथ॑ । \newline
28. अथ॒ कुतः॒ कुतो ऽथाथ॒ कुतः॑ । \newline
29. कुतः॒ सा सा कुतः॒ कुतः॒ सा । \newline
30. सा धो᳚क्ष्यते धोक्ष्यते॒ सा सा धो᳚क्ष्यते । \newline
31. धो॒क्ष्य॒ते॒ यां ॅयाम् धो᳚क्ष्यते धोक्ष्यते॒ याम् । \newline
32. याम् द्वाद॑श॒ द्वाद॑श॒ यां ॅयाम् द्वाद॑श । \newline
33. द्वाद॑श॒ कृत्वः॒ कृत्वो॒ द्वाद॑श॒ द्वाद॑श॒ कृत्वः॑ । \newline
34. कृत्व॑ उप॒सीद॑ न्त्युप॒सीद॑न्ति॒ कृत्वः॒ कृत्व॑ उप॒सीद॑न्ति । \newline
35. उ॒प॒सीद॒न्तीती त्यु॑प॒सीद॑ न्त्युप॒सीद॒न्तीति॑ । \newline
36. उ॒प॒सीद॒न्तीत्यु॑प - सीद॑न्ति । \newline
37. इति॑ संॅवथ्स॒रꣳ सं॑ॅवथ्स॒र मितीति॑ संॅवथ्स॒रम् । \newline
38. सं॒ॅव॒थ्स॒रꣳ सं॒पाद्य॑ सं॒पाद्य॑ संॅवथ्स॒रꣳ सं॑ॅवथ्स॒रꣳ सं॒पाद्य॑ । \newline
39. सं॒ॅव॒थ्स॒रमिति॑ सं - व॒थ्स॒रम् । \newline
40. सं॒पाद्यो᳚त्त॒म उ॑त्त॒मे सं॒पाद्य॑ सं॒पाद्यो᳚त्त॒मे । \newline
41. सं॒पाद्येति॑ सं - पाद्य॑ । \newline
42. उ॒त्त॒मे मा॒सि मा॒स्यु॑त्त॒म उ॑त्त॒मे मा॒सि । \newline
43. उ॒त्त॒म इत्यु॑त् - त॒मे । \newline
44. मा॒सि स॒कृथ् स॒कृन् मा॒सि मा॒सि स॒कृत् । \newline
45. स॒कृत् पृ॒ष्ठानि॑ पृ॒ष्ठानि॑ स॒कृथ् स॒कृत् पृ॒ष्ठानि॑ । \newline
46. पृ॒ष्ठा न्युपोप॑ पृ॒ष्ठानि॑ पृ॒ष्ठा न्युप॑ । \newline
47. उपे॑यु रियु॒ रुपोपे॑युः । \newline
48. इ॒यु॒ स्तत् तदि॑यु रियु॒ स्तत् । \newline
49. तद् यज॑माना॒ यज॑माना॒ स्तत् तद् यज॑मानाः । \newline
50. यज॑माना य॒ज्ञ्ं ॅय॒ज्ञ्ं ॅयज॑माना॒ यज॑माना य॒ज्ञ्म् । \newline
51. य॒ज्ञ्म् प॒शून् प॒शून्. य॒ज्ञ्ं ॅय॒ज्ञ्म् प॒शून् । \newline
52. प॒शूनवाव॑ प॒शून् प॒शूनव॑ । \newline
53. अव॑ रुन्धते रुन्ध॒ते ऽवाव॑ रुन्धते । \newline
54. रु॒न्ध॒ते॒ स॒मु॒द्रꣳ स॑मु॒द्रꣳ रु॑न्धते रुन्धते समु॒द्रम् । \newline
55. स॒मु॒द्रं ॅवै वै स॑मु॒द्रꣳ स॑मु॒द्रं ॅवै । \newline
56. वा ए॒त ए॒ते वै वा ए॒ते । \newline

\textbf{Ghana Paata } \newline

1. प्र॒थ॒मे मा॒सि मा॒सि प्र॑थ॒मे प्र॑थ॒मे मा॒सि पृ॒ष्ठानि॑ पृ॒ष्ठानि॑ मा॒सि प्र॑थ॒मे प्र॑थ॒मे मा॒सि पृ॒ष्ठानि॑ । \newline
2. मा॒सि पृ॒ष्ठानि॑ पृ॒ष्ठानि॑ मा॒सि मा॒सि पृ॒ष्ठा न्युपोप॑ पृ॒ष्ठानि॑ मा॒सि मा॒सि पृ॒ष्ठा न्युप॑ । \newline
3. पृ॒ष्ठा न्युपोप॑ पृ॒ष्ठानि॑ पृ॒ष्ठा न्युप॑ यन्ति य॒न्त्युप॑ पृ॒ष्ठानि॑ पृ॒ष्ठा न्युप॑ यन्ति । \newline
4. उप॑ यन्ति य॒न्त्युपोप॑ यन्ति मद्ध्य॒मे म॑द्ध्य॒मे य॒न्त्युपोप॑ यन्ति मद्ध्य॒मे । \newline
5. य॒न्ति॒ म॒द्ध्य॒मे म॑द्ध्य॒मे य॑न्ति यन्ति मद्ध्य॒म उपोप॑ मद्ध्य॒मे य॑न्ति यन्ति मद्ध्य॒म उप॑ । \newline
6. म॒द्ध्य॒म उपोप॑ मद्ध्य॒मे म॑द्ध्य॒म उप॑ यन्ति य॒न्त्युप॑ मद्ध्य॒मे म॑द्ध्य॒म उप॑ यन्ति । \newline
7. उप॑ यन्ति य॒न्त्युपोप॑ यन्त्युत्त॒म उ॑त्त॒मे य॒न्त्युपोप॑ यन्त्युत्त॒मे । \newline
8. य॒न्त्यु॒त्त॒म उ॑त्त॒मे य॑न्ति यन्त्युत्त॒म उपोपो᳚त्त॒मे य॑न्ति यन्त्युत्त॒म उप॑ । \newline
9. उ॒त्त॒म उपोपो᳚ त्त॒म उ॑त्त॒म उप॑ यन्ति य॒न्त्युपो᳚ त्त॒म उ॑त्त॒म उप॑ यन्ति । \newline
10. उ॒त्त॒म इत्यु॑त् - त॒मे । \newline
11. उप॑ यन्ति य॒न्त्युपोप॑ यन्ति॒ तत् तद् य॒न्त्युपोप॑ यन्ति॒ तत् । \newline
12. य॒न्ति॒ तत् तद् य॑न्ति यन्ति॒ तदा॑हु राहु॒ स्तद् य॑न्ति यन्ति॒ तदा॑हुः । \newline
13. तदा॑हु राहु॒ स्तत् तदा॑हु॒र् यां ॅया मा॑हु॒ स्तत् तदा॑हु॒र् याम् । \newline
14. आ॒हु॒र् यां ॅया मा॑हु राहु॒र् यां ॅवै वै या मा॑हु राहु॒र् यां ॅवै । \newline
15. यां ॅवै वै यां ॅयां ॅवै त्रि स्त्रिर् वै यां ॅयां ॅवै त्रिः । \newline
16. वै त्रि स्त्रिर् वै वै त्रि रेक॒ स्यैक॑स्य॒ त्रिर् वै वै त्रि रेक॑स्य । \newline
17. त्रि रेक॒ स्यैक॑स्य॒ त्रि स्त्रि रेक॒ स्याह्नो ऽह्न॒ एक॑स्य॒ त्रि स्त्रि रेक॒ स्याह्नः॑ । \newline
18. एक॒ स्याह्नो ऽह्न॒ एक॒ स्यैक॒ स्याह्न॑ उप॒सीद॑ न्त्युप॒सीद॒ न्त्यह्न॒ एक॒ स्यैक॒ स्याह्न॑ उप॒सीद॑न्ति । \newline
19. अह्न॑ उप॒सीद॑ न्त्युप॒सीद॒ न्त्यह्नो ऽह्न॑ उप॒सीद॑न्ति द॒ह्रम् द॒ह्र मु॑प॒सीद॒ न्त्यह्नो ऽह्न॑ उप॒सीद॑न्ति द॒ह्रम् । \newline
20. उ॒प॒सीद॑न्ति द॒ह्रम् द॒ह्र मु॑प॒सीद॑ न्त्युप॒सीद॑न्ति द॒ह्रं ॅवै वै द॒ह्र मु॑प॒सीद॑ न्त्युप॒सीद॑न्ति द॒ह्रं ॅवै । \newline
21. उ॒प॒सीद॒न्तीत्यु॑प - सीद॑न्ति । \newline
22. द॒ह्रं ॅवै वै द॒ह्रम् द॒ह्रं ॅवै सा सा वै द॒ह्रम् द॒ह्रं ॅवै सा । \newline
23. वै सा सा वै वै सा ऽप॑राभ्या॒ मप॑राभ्याꣳ॒॒ सा वै वै सा ऽप॑राभ्याम् । \newline
24. सा ऽप॑राभ्या॒ मप॑राभ्याꣳ॒॒ सा सा ऽप॑राभ्या॒म् दोहा᳚भ्या॒म् दोहा᳚भ्या॒ मप॑राभ्याꣳ॒॒ सा सा ऽप॑राभ्या॒म् दोहा᳚भ्याम् । \newline
25. अप॑राभ्या॒म् दोहा᳚भ्या॒म् दोहा᳚भ्या॒ मप॑राभ्या॒ मप॑राभ्या॒म् दोहा᳚भ्याम् दुहे दुहे॒ दोहा᳚भ्या॒ मप॑राभ्या॒ मप॑राभ्या॒म् दोहा᳚भ्याम् दुहे । \newline
26. दोहा᳚भ्याम् दुहे दुहे॒ दोहा᳚भ्या॒म् दोहा᳚भ्याम् दु॒हे ऽथाथ॑ दुहे॒ दोहा᳚भ्या॒म् दोहा᳚भ्याम् दु॒हे ऽथ॑ । \newline
27. दु॒हे ऽथाथ॑ दुहे दु॒हे ऽथ॒ कुतः॒ कुतो ऽथ॑ दुहे दु॒हे ऽथ॒ कुतः॑ । \newline
28. अथ॒ कुतः॒ कुतो ऽथाथ॒ कुतः॒ सा सा कुतो ऽथाथ॒ कुतः॒ सा । \newline
29. कुतः॒ सा सा कुतः॒ कुतः॒ सा धो᳚क्ष्यते धोक्ष्यते॒ सा कुतः॒ कुतः॒ सा धो᳚क्ष्यते । \newline
30. सा धो᳚क्ष्यते धोक्ष्यते॒ सा सा धो᳚क्ष्यते॒ यां ॅयाम् धो᳚क्ष्यते॒ सा सा धो᳚क्ष्यते॒ याम् । \newline
31. धो॒क्ष्य॒ते॒ यां ॅयाम् धो᳚क्ष्यते धोक्ष्यते॒ याम् द्वाद॑श॒ द्वाद॑श॒ याम् धो᳚क्ष्यते धोक्ष्यते॒ याम् द्वाद॑श । \newline
32. याम् द्वाद॑श॒ द्वाद॑श॒ यां ॅयाम् द्वाद॑श॒ कृत्वः॒ कृत्वो॒ द्वाद॑श॒ यां ॅयाम् द्वाद॑श॒ कृत्वः॑ । \newline
33. द्वाद॑श॒ कृत्वः॒ कृत्वो॒ द्वाद॑श॒ द्वाद॑श॒ कृत्व॑ उप॒सीद॑ न्त्युप॒सीद॑न्ति॒ कृत्वो॒ द्वाद॑श॒ द्वाद॑श॒ कृत्व॑ उप॒सीद॑न्ति । \newline
34. कृत्व॑ उप॒सीद॑ न्त्युप॒सीद॑न्ति॒ कृत्वः॒ कृत्व॑ उप॒सीद॒न्तीती त्यु॑प॒सीद॑न्ति॒ कृत्वः॒ कृत्व॑ उप॒सीद॒न्तीति॑ । \newline
35. उ॒प॒सीद॒न्तीती त्यु॑प॒सीद॑ न्त्युप॒सीद॒न्तीति॑ संॅवथ्स॒रꣳ सं॑ॅवथ्स॒र मित्यु॑प॒सीद॑ न्त्युप॒सीद॒न्तीति॑ संॅवथ्स॒रम् । \newline
36. उ॒प॒सीद॒न्तीत्यु॑प - सीद॑न्ति । \newline
37. इति॑ संॅवथ्स॒रꣳ सं॑ॅवथ्स॒र मितीति॑ संॅवथ्स॒रꣳ सं॒पाद्य॑ सं॒पाद्य॑ संॅवथ्स॒र मितीति॑ संॅवथ्स॒रꣳ सं॒पाद्य॑ । \newline
38. सं॒ॅव॒थ्स॒रꣳ सं॒पाद्य॑ सं॒पाद्य॑ संॅवथ्स॒रꣳ सं॑ॅवथ्स॒रꣳ सं॒पाद्यो᳚त्त॒म उ॑त्त॒मे सं॒पाद्य॑ संॅवथ्स॒रꣳ सं॑ॅवथ्स॒रꣳ सं॒पाद्यो᳚त्त॒मे । \newline
39. सं॒ॅव॒थ्स॒रमिति॑ सं - व॒थ्स॒रम् । \newline
40. सं॒पाद्यो᳚त्त॒म उ॑त्त॒मे सं॒पाद्य॑ सं॒पाद्यो᳚त्त॒मे मा॒सि मा॒स्यु॑त्त॒मे सं॒पाद्य॑ सं॒पाद्यो᳚त्त॒मे मा॒सि । \newline
41. सं॒पाद्येति॑ सं - पाद्य॑ । \newline
42. उ॒त्त॒मे मा॒सि मा॒स्यु॑त्त॒म उ॑त्त॒मे मा॒सि स॒कृथ् स॒कृन् मा॒स्यु॑त्त॒म उ॑त्त॒मे मा॒सि स॒कृत् । \newline
43. उ॒त्त॒म इत्यु॑त् - त॒मे । \newline
44. मा॒सि स॒कृथ् स॒कृन् मा॒सि मा॒सि स॒कृत् पृ॒ष्ठानि॑ पृ॒ष्ठानि॑ स॒कृन् मा॒सि मा॒सि स॒कृत् पृ॒ष्ठानि॑ । \newline
45. स॒कृत् पृ॒ष्ठानि॑ पृ॒ष्ठानि॑ स॒कृथ् स॒कृत् पृ॒ष्ठा न्युपोप॑ पृ॒ष्ठानि॑ स॒कृथ् स॒कृत् पृ॒ष्ठा न्युप॑ । \newline
46. पृ॒ष्ठा न्युपोप॑ पृ॒ष्ठानि॑ पृ॒ष्ठा न्युपे॑यु रियु॒ रुप॑ पृ॒ष्ठानि॑ पृ॒ष्ठा न्युपे॑युः । \newline
47. उपे॑यु रियु॒ रुपो पे॑यु॒ स्तत् तदि॑यु॒ रुपो पे॑यु॒ स्तत् । \newline
48. इ॒यु॒ स्तत् तदि॑यु रियु॒ स्तद् यज॑माना॒ यज॑माना॒ स्तदि॑यु रियु॒ स्तद् यज॑मानाः । \newline
49. तद् यज॑माना॒ यज॑माना॒ स्तत् तद् यज॑माना य॒ज्ञ्ं ॅय॒ज्ञ्ं ॅयज॑माना॒ स्तत् तद् यज॑माना य॒ज्ञ्म् । \newline
50. यज॑माना य॒ज्ञ्ं ॅय॒ज्ञ्ं ॅयज॑माना॒ यज॑माना य॒ज्ञ्म् प॒शून् प॒शून्. य॒ज्ञ्ं ॅयज॑माना॒ यज॑माना य॒ज्ञ्म् प॒शून् । \newline
51. य॒ज्ञ्म् प॒शून् प॒शून्. य॒ज्ञ्ं ॅय॒ज्ञ्म् प॒शूनवाव॑ प॒शून्. य॒ज्ञ्ं ॅय॒ज्ञ्म् प॒शूनव॑ । \newline
52. प॒शूनवाव॑ प॒शून् प॒शूनव॑ रुन्धते रुन्ध॒ते ऽव॑ प॒शून् प॒शूनव॑ रुन्धते । \newline
53. अव॑ रुन्धते रुन्ध॒ते ऽवाव॑ रुन्धते समु॒द्रꣳ स॑मु॒द्रꣳ रु॑न्ध॒ते ऽवाव॑ रुन्धते समु॒द्रम् । \newline
54. रु॒न्ध॒ते॒ स॒मु॒द्रꣳ स॑मु॒द्रꣳ रु॑न्धते रुन्धते समु॒द्रं ॅवै वै स॑मु॒द्रꣳ रु॑न्धते रुन्धते समु॒द्रं ॅवै । \newline
55. स॒मु॒द्रं ॅवै वै स॑मु॒द्रꣳ स॑मु॒द्रं ॅवा ए॒त ए॒ते वै स॑मु॒द्रꣳ स॑मु॒द्रं ॅवा ए॒ते । \newline
56. वा ए॒त ए॒ते वै वा ए॒ते॑ ऽनवा॒र म॑नवा॒र मे॒ते वै वा ए॒ते॑ ऽनवा॒रम् । \newline
\pagebreak
\markright{ TS 7.5.3.2  \hfill https://www.vedavms.in \hfill}

\section{ TS 7.5.3.2 }

\textbf{TS 7.5.3.2 } \newline
\textbf{Samhita Paata} \newline

ए॒ते॑नवा॒रम॑पा॒रं प्र प्ल॑वन्ते॒ ये सं॑ॅवथ्स॒रमु॑प॒यन्ति॒ यद् बृ॑हद्-रथन्त॒रे अ॒न्वर्जे॑यु॒र्यथा॒ मद्ध्ये॑ समु॒द्रस्य॑ प्ल॒वम॒न्वर्जे॑युस्ता॒दृक् तदनु॑थ्सर्गं बृहद्-रथन्त॒राभ्या॑मि॒त्वा प्र॑ति॒ष्ठां ग॑च्छन्ति॒ सर्वे᳚भ्यो॒ वै कामे᳚भ्यः स॒न्धिर्दु॑हे॒ तद्-यज॑मानाः॒ सर्वा॒न् कामा॒नव॑ रुन्धते ॥ \newline

\textbf{Pada Paata} \newline

ए॒ते । अ॒न॒वा॒रम् । अ॒पा॒रम् । प्रेति॑ । प्ल॒व॒न्ते॒ । ये । सं॒ॅव॒थ्स॒रमिति॑ सं - व॒थ्स॒रम् । उ॒प॒यन्तीत्यु॑प - यन्ति॑ । यत् । बृ॒ह॒द्र॒थ॒न्त॒रे इति॑ बृहत् - र॒थ॒न्त॒रे । अ॒न्वर्जे॑यु॒रित्य॑नु - अर्जे॑युः । यथा᳚ । मद्ध्ये᳚ । स॒मु॒द्रस्य॑ । प्ल॒वम् । अ॒न्वर्जे॑यु॒रित्य॑नु - अर्जे॑युः । ता॒दृक् । तत् । अनु॑थ्सर्ग॒मित्यनु॑त् - स॒र्ग॒म् । बृ॒ह॒द्र॒थ॒न्त॒राभ्या॒मिति॑ बृहत् - र॒थ॒न्त॒राभ्या᳚म् । इ॒त्वा । प्र॒ति॒ष्ठामिति॑ प्रति -  स्थाम् । ग॒च्छ॒न्ति॒ । सर्वे᳚भ्यः । वै । कामे᳚भ्यः । स॒न्धिरिति॑ सं - धिः । दु॒हे॒ । तत् । यज॑मानाः ।  सर्वान्॑ । कामान्॑ । अवेति॑ । रु॒न्ध॒ते॒ ॥  \newline


\textbf{Krama Paata} \newline

ए॒ते॑ऽनवा॒रम् । अ॒न॒वा॒रम॑पा॒रम् । अ॒पा॒रम् प्र । प्र प्ल॑वन्ते । प्ल॒व॒न्ते॒ ये । ये स॑म्ॅवथ्स॒रम् । स॒म्ॅव॒थ्स॒रमु॑प॒यन्ति॑ । स॒म्ॅव॒थ्स॒रमिति॑ सम् - व॒थ्स॒रम् । उ॒प॒यन्ति॒ यत् । उ॒प॒यन्तीत्यु॑प - यन्ति॑ । यद् बृ॑हद्‍रथन्त॒रे । बृ॒ह॒द्‍र॒थ॒न्त॒रे अ॒न्वर्जे॑युः । बृ॒ह॒द्‍र॒थ॒न्त॒रे इति॑ बृहत् - र॒थ॒न्त॒रे । अ॒न्वर्जे॑यु॒र् यथा᳚ । अ॒न्वर्जे॑यु॒रित्य॑नु - अर्जे॑युः । यथा॒ मद्ध्ये᳚ । मद्ध्ये॑ समु॒द्रस्य॑ । स॒मु॒द्रस्य॑ प्ल॒वम् । प्ल॒वम॒न्वर्जे॑युः । अ॒न्वर्जे॑युस्ता॒दृक् । अ॒न्वर्जे॑यु॒रित्य॑नु - अर्जे॑युः । ता॒दृक् तत् । तदनु॑थ्सर्गम् । अनु॑थ्सर्गम् बृहद्‍रथन्त॒राभ्या᳚म् । अनु॑थ्सर्ग॒मित्यनु॑ - स॒र्ग॒म् । बृ॒ह॒द्‍र॒थ॒न्त॒राभ्या॑मि॒त्वा । बृ॒ह॒द्‍र॒थ॒न्त॒राभ्या॒मिति॑ बृहत् - र॒थ॒न्त॒राभ्या᳚म् । इ॒त्वा प्र॑ति॒ष्ठाम् । प्र॒ति॒ष्ठाम् ग॑च्छन्ति । प्र॒ति॒ष्ठामिति॑ प्रति - स्थाम् । ग॒च्छ॒न्ति॒ सर्वे᳚भ्यः । सर्वे᳚भ्यो॒ वै । वै कामे᳚भ्यः । कामे᳚भ्यः स॒न्धिः । स॒न्धिर् दु॑हे । स॒न्धिरिति॑ सम् - धिः । दु॒हे॒ तत् । तद् यज॑मानाः । यज॑मानाः॒ सर्वान्॑ । सर्वा॒न् कामान्॑ । कामा॒नव॑ । अव॑ रुन्धते । रु॒न्ध॒त॒ इति॑ रुन्धते । \newline

\textbf{Jatai Paata} \newline

1. ए॒ते॑ ऽनवा॒र म॑नवा॒र मे॒त ए॒ते॑ ऽनवा॒रम् । \newline
2. अ॒न॒वा॒र म॑पा॒र म॑पा॒र म॑नवा॒र म॑नवा॒र म॑पा॒रम् । \newline
3. अ॒पा॒रम् प्र प्रा पा॒र म॑पा॒रम् प्र । \newline
4. प्र प्ल॑वन्ते प्लवन्ते॒ प्र प्र प्ल॑वन्ते । \newline
5. प्ल॒व॒न्ते॒ ये ये प्ल॑वन्ते प्लवन्ते॒ ये । \newline
6. ये सं॑ॅवथ्स॒रꣳ सं॑ॅवथ्स॒रं ॅये ये सं॑ॅवथ्स॒रम् । \newline
7. सं॒ॅव॒थ्स॒र मु॑प॒य न्त्यु॑प॒यन्ति॑ संॅवथ्स॒रꣳ सं॑ॅवथ्स॒र मु॑प॒यन्ति॑ । \newline
8. सं॒ॅव॒थ्स॒रमिति॑ सं - व॒थ्स॒रम् । \newline
9. उ॒प॒यन्ति॒ यद् यदु॑प॒य न्त्यु॑प॒यन्ति॒ यत् । \newline
10. उ॒प॒यन्तीत्यु॑प - यन्ति॑ । \newline
11. यद् बृ॑हद्रथन्त॒रे बृ॑हद्रथन्त॒रे यद् यद् बृ॑हद्रथन्त॒रे । \newline
12. बृ॒ह॒द्र॒थ॒न्त॒रे अ॒न्वर्जे॑यु र॒न्वर्जे॑युर् बृहद्रथन्त॒रे बृ॑हद्रथन्त॒रे अ॒न्वर्जे॑युः । \newline
13. बृ॒ह॒द्र॒थ॒न्त॒रे इति॑ बृहत् - र॒थ॒न्त॒रे । \newline
14. अ॒न्वर्जे॑यु॒र् यथा॒ यथा॒ ऽन्वर्जे॑यु र॒न्वर्जे॑यु॒र् यथा᳚ । \newline
15. अ॒न्वर्जे॑यु॒रित्य॑नु - अर्जे॑युः । \newline
16. यथा॒ मद्ध्ये॒ मद्ध्ये॒ यथा॒ यथा॒ मद्ध्ये᳚ । \newline
17. मद्ध्ये॑ समु॒द्रस्य॑ समु॒द्रस्य॒ मद्ध्ये॒ मद्ध्ये॑ समु॒द्रस्य॑ । \newline
18. स॒मु॒द्रस्य॑ प्ल॒वम् प्ल॒वꣳ स॑मु॒द्रस्य॑ समु॒द्रस्य॑ प्ल॒वम् । \newline
19. प्ल॒व म॒न्वर्जे॑यु र॒न्वर्जे॑युः प्ल॒वम् प्ल॒व म॒न्वर्जे॑युः । \newline
20. अ॒न्वर्जे॑यु स्ता॒दृक् ता॒दृ ग॒न्वर्जे॑यु र॒न्वर्जे॑यु स्ता॒दृक् । \newline
21. अ॒न्वर्जे॑यु॒रित्य॑नु - अर्जे॑युः । \newline
22. ता॒दृक् तत् तत् ता॒दृक् ता॒दृक् तत् । \newline
23. तदनु॑थ्सर्ग॒ मनु॑थ्सर्ग॒म् तत् तदनु॑थ्सर्गम् । \newline
24. अनु॑थ्सर्गम् बृहद्रथन्त॒राभ्या᳚म् बृहद्रथन्त॒राभ्या॒ मनु॑थ्सर्ग॒ मनु॑थ्सर्गम् बृहद्रथन्त॒राभ्या᳚म् । \newline
25. अनु॑थ्सर्ग॒मित्यनु॑त् - स॒र्ग॒म् । \newline
26. बृ॒ह॒द्र॒थ॒न्त॒राभ्या॑ मि॒त्वेत्वा बृ॑हद्रथन्त॒राभ्या᳚म् बृहद्रथन्त॒राभ्या॑ मि॒त्वा । \newline
27. बृ॒ह॒द्र॒थ॒न्त॒राभ्या॒मिति॑ बृहत् - र॒थ॒न्त॒राभ्या᳚म् । \newline
28. इ॒त्वा प्र॑ति॒ष्ठाम् प्र॑ति॒ष्ठा मि॒त्वेत्वा प्र॑ति॒ष्ठाम् । \newline
29. प्र॒ति॒ष्ठाम् ग॑च्छन्ति गच्छन्ति प्रति॒ष्ठाम् प्र॑ति॒ष्ठाम् ग॑च्छन्ति । \newline
30. प्र॒ति॒ष्ठामिति॑ प्रति - स्थाम् । \newline
31. ग॒च्छ॒न्ति॒ सर्वे᳚भ्यः॒ सर्वे᳚भ्यो गच्छन्ति गच्छन्ति॒ सर्वे᳚भ्यः । \newline
32. सर्वे᳚भ्यो॒ वै वै सर्वे᳚भ्यः॒ सर्वे᳚भ्यो॒ वै । \newline
33. वै कामे᳚भ्यः॒ कामे᳚भ्यो॒ वै वै कामे᳚भ्यः । \newline
34. कामे᳚भ्यः स॒न्धिः स॒न्धिः कामे᳚भ्यः॒ कामे᳚भ्यः स॒न्धिः । \newline
35. स॒न्धिर् दु॑हे दुहे स॒न्धिः स॒न्धिर् दु॑हे । \newline
36. स॒न्धिरिति॑ सं - धिः । \newline
37. दु॒हे॒ तत् तद् दु॑हे दुहे॒ तत् । \newline
38. तद् यज॑माना॒ यज॑माना॒ स्तत् तद् यज॑मानाः । \newline
39. यज॑मानाः॒ सर्वा॒न् थ्सर्वा॒न्॒. यज॑माना॒ यज॑मानाः॒ सर्वान्॑ । \newline
40. सर्वा॒न् कामा॒न् कामा॒न् थ्सर्वा॒न् थ्सर्वा॒न् कामान्॑ । \newline
41. कामा॒ नवाव॒ कामा॒न् कामा॒ नव॑ । \newline
42. अव॑ रुन्धते रुन्ध॒ते ऽवाव॑ रुन्धते । \newline
43. रु॒न्ध॒त॒ इति॑ रुन्धते । \newline

\textbf{Ghana Paata } \newline

1. ए॒ते॑ ऽनवा॒र म॑नवा॒र मे॒त ए॒ते॑ ऽनवा॒र म॑पा॒र म॑पा॒र म॑नवा॒र मे॒त ए॒ते॑ ऽनवा॒र म॑पा॒रम् । \newline
2. अ॒न॒वा॒र म॑पा॒र म॑पा॒र म॑नवा॒र म॑नवा॒र म॑पा॒रम् प्र प्रापा॒र म॑नवा॒र म॑नवा॒र म॑पा॒रम् प्र । \newline
3. अ॒पा॒रम् प्र प्रापा॒र म॑पा॒रम् प्र प्ल॑वन्ते प्लवन्ते॒ प्रापा॒र म॑पा॒रम् प्र प्ल॑वन्ते । \newline
4. प्र प्ल॑वन्ते प्लवन्ते॒ प्र प्र प्ल॑वन्ते॒ ये ये प्ल॑वन्ते॒ प्र प्र प्ल॑वन्ते॒ ये । \newline
5. प्ल॒व॒न्ते॒ ये ये प्ल॑वन्ते प्लवन्ते॒ ये सं॑ॅवथ्स॒रꣳ सं॑ॅवथ्स॒रं ॅये प्ल॑वन्ते प्लवन्ते॒ ये सं॑ॅवथ्स॒रम् । \newline
6. ये सं॑ॅवथ्स॒रꣳ सं॑ॅवथ्स॒रं ॅये ये सं॑ॅवथ्स॒र मु॑प॒य न्त्यु॑प॒यन्ति॑ संॅवथ्स॒रं ॅये ये सं॑ॅवथ्स॒र मु॑प॒यन्ति॑ । \newline
7. सं॒ॅव॒थ्स॒र मु॑प॒य न्त्यु॑प॒यन्ति॑ संॅवथ्स॒रꣳ सं॑ॅवथ्स॒र मु॑प॒यन्ति॒ यद् यदु॑प॒यन्ति॑ संॅवथ्स॒रꣳ सं॑ॅवथ्स॒र मु॑प॒यन्ति॒ यत् । \newline
8. सं॒ॅव॒थ्स॒रमिति॑ सं - व॒थ्स॒रम् । \newline
9. उ॒प॒यन्ति॒ यद् यदु॑प॒य न्त्यु॑प॒यन्ति॒ यद् बृ॑हद्रथन्त॒रे बृ॑हद्रथन्त॒रे यदु॑प॒य न्त्यु॑प॒यन्ति॒ यद् बृ॑हद्रथन्त॒रे । \newline
10. उ॒प॒यन्तीत्यु॑प - यन्ति॑ । \newline
11. यद् बृ॑हद्रथन्त॒रे बृ॑हद्रथन्त॒रे यद् यद् बृ॑हद्रथन्त॒रे अ॒न्वर्जे॑यु र॒न्वर्जे॑युर् बृहद्रथन्त॒रे यद् यद् बृ॑हद्रथन्त॒रे अ॒न्वर्जे॑युः । \newline
12. बृ॒ह॒द्र॒थ॒न्त॒रे अ॒न्वर्जे॑यु र॒न्वर्जे॑युर् बृहद्रथन्त॒रे बृ॑हद्रथन्त॒रे अ॒न्वर्जे॑यु॒र् यथा॒ यथा॒ ऽन्वर्जे॑युर् बृहद्रथन्त॒रे बृ॑हद्रथन्त॒रे अ॒न्वर्जे॑यु॒र् यथा᳚ । \newline
13. बृ॒ह॒द्र॒थ॒न्त॒रे इति॑ बृहत् - र॒थ॒न्त॒रे । \newline
14. अ॒न्वर्जे॑यु॒र् यथा॒ यथा॒ ऽन्वर्जे॑यु र॒न्वर्जे॑यु॒र् यथा॒ मद्ध्ये॒ मद्ध्ये॒ यथा॒ ऽन्वर्जे॑यु र॒न्वर्जे॑यु॒र् यथा॒ मद्ध्ये᳚ । \newline
15. अ॒न्वर्जे॑यु॒रित्य॑नु - अर्जे॑युः । \newline
16. यथा॒ मद्ध्ये॒ मद्ध्ये॒ यथा॒ यथा॒ मद्ध्ये॑ समु॒द्रस्य॑ समु॒द्रस्य॒ मद्ध्ये॒ यथा॒ यथा॒ मद्ध्ये॑ समु॒द्रस्य॑ । \newline
17. मद्ध्ये॑ समु॒द्रस्य॑ समु॒द्रस्य॒ मद्ध्ये॒ मद्ध्ये॑ समु॒द्रस्य॑ प्ल॒वम् प्ल॒वꣳ स॑मु॒द्रस्य॒ मद्ध्ये॒ मद्ध्ये॑ समु॒द्रस्य॑ प्ल॒वम् । \newline
18. स॒मु॒द्रस्य॑ प्ल॒वम् प्ल॒वꣳ स॑मु॒द्रस्य॑ समु॒द्रस्य॑ प्ल॒व म॒न्वर्जे॑यु र॒न्वर्जे॑युः प्ल॒वꣳ स॑मु॒द्रस्य॑ समु॒द्रस्य॑ प्ल॒व म॒न्वर्जे॑युः । \newline
19. प्ल॒व म॒न्वर्जे॑यु र॒न्वर्जे॑युः प्ल॒वम् प्ल॒व म॒न्वर्जे॑यु स्ता॒दृक् ता॒दृ ग॒न्वर्जे॑युः प्ल॒वम् प्ल॒व म॒न्वर्जे॑यु स्ता॒दृक् । \newline
20. अ॒न्वर्जे॑यु स्ता॒दृक् ता॒दृ ग॒न्वर्जे॑यु र॒न्वर्जे॑यु स्ता॒दृक् तत् तत् ता॒दृ ग॒न्वर्जे॑यु र॒न्वर्जे॑यु स्ता॒दृक् तत् । \newline
21. अ॒न्वर्जे॑यु॒रित्य॑नु - अर्जे॑युः । \newline
22. ता॒दृक् तत् तत् ता॒दृक् ता॒दृक् तदनु॑थ्सर्ग॒ मनु॑थ्सर्ग॒म् तत् ता॒दृक् ता॒दृक् तदनु॑थ्सर्गम् । \newline
23. तदनु॑थ्सर्ग॒ मनु॑थ्सर्ग॒म् तत् तदनु॑थ्सर्गम् बृहद्रथन्त॒राभ्या᳚म् बृहद्रथन्त॒राभ्या॒ मनु॑थ्सर्ग॒म् तत् तदनु॑थ्सर्गम् बृहद्रथन्त॒राभ्या᳚म् । \newline
24. अनु॑थ्सर्गम् बृहद्रथन्त॒राभ्या᳚म् बृहद्रथन्त॒राभ्या॒ मनु॑थ्सर्ग॒ मनु॑थ्सर्गम् बृहद्रथन्त॒राभ्या॑ मि॒त्वेत्वा बृ॑हद्रथन्त॒राभ्या॒ मनु॑थ्सर्ग॒ मनु॑थ्सर्गम् बृहद्रथन्त॒राभ्या॑ मि॒त्वा । \newline
25. अनु॑थ्सर्ग॒मित्यनु॑त् - स॒र्ग॒म् । \newline
26. बृ॒ह॒द्र॒थ॒न्त॒राभ्या॑ मि॒त्वेत्वा बृ॑हद्रथन्त॒राभ्या᳚म् बृहद्रथन्त॒राभ्या॑ मि॒त्वा प्र॑ति॒ष्ठाम् प्र॑ति॒ष्ठा मि॒त्वा बृ॑हद्रथन्त॒राभ्या᳚म् बृहद्रथन्त॒राभ्या॑ मि॒त्वा प्र॑ति॒ष्ठाम् । \newline
27. बृ॒ह॒द्र॒थ॒न्त॒राभ्या॒मिति॑ बृहत् - र॒थ॒न्त॒राभ्या᳚म् । \newline
28. इ॒त्वा प्र॑ति॒ष्ठाम् प्र॑ति॒ष्ठा मि॒त्वेत्वा प्र॑ति॒ष्ठाम् ग॑च्छन्ति गच्छन्ति प्रति॒ष्ठा मि॒त्वेत्वा प्र॑ति॒ष्ठाम् ग॑च्छन्ति । \newline
29. प्र॒ति॒ष्ठाम् ग॑च्छन्ति गच्छन्ति प्रति॒ष्ठाम् प्र॑ति॒ष्ठाम् ग॑च्छन्ति॒ सर्वे᳚भ्यः॒ सर्वे᳚भ्यो गच्छन्ति प्रति॒ष्ठाम् प्र॑ति॒ष्ठाम् ग॑च्छन्ति॒ सर्वे᳚भ्यः । \newline
30. प्र॒ति॒ष्ठामिति॑ प्रति - स्थाम् । \newline
31. ग॒च्छ॒न्ति॒ सर्वे᳚भ्यः॒ सर्वे᳚भ्यो गच्छन्ति गच्छन्ति॒ सर्वे᳚भ्यो॒ वै वै सर्वे᳚भ्यो गच्छन्ति गच्छन्ति॒ सर्वे᳚भ्यो॒ वै । \newline
32. सर्वे᳚भ्यो॒ वै वै सर्वे᳚भ्यः॒ सर्वे᳚भ्यो॒ वै कामे᳚भ्यः॒ कामे᳚भ्यो॒ वै सर्वे᳚भ्यः॒ सर्वे᳚भ्यो॒ वै कामे᳚भ्यः । \newline
33. वै कामे᳚भ्यः॒ कामे᳚भ्यो॒ वै वै कामे᳚भ्यः स॒न्धिः स॒न्धिः कामे᳚भ्यो॒ वै वै कामे᳚भ्यः स॒न्धिः । \newline
34. कामे᳚भ्यः स॒न्धिः स॒न्धिः कामे᳚भ्यः॒ कामे᳚भ्यः स॒न्धिर् दु॑हे दुहे स॒न्धिः कामे᳚भ्यः॒ कामे᳚भ्यः स॒न्धिर् दु॑हे । \newline
35. स॒न्धिर् दु॑हे दुहे स॒न्धिः स॒न्धिर् दु॑हे॒ तत् तद् दु॑हे स॒न्धिः स॒न्धिर् दु॑हे॒ तत् । \newline
36. स॒न्धिरिति॑ सं - धिः । \newline
37. दु॒हे॒ तत् तद् दु॑हे दुहे॒ तद् यज॑माना॒ यज॑माना॒ स्तद् दु॑हे दुहे॒ तद् यज॑मानाः । \newline
38. तद् यज॑माना॒ यज॑माना॒ स्तत् तद् यज॑मानाः॒ सर्वा॒न् थ्सर्वा॒न्॒. यज॑माना॒ स्तत् तद् यज॑मानाः॒ सर्वान्॑ । \newline
39. यज॑मानाः॒ सर्वा॒न् थ्सर्वा॒न्॒. यज॑माना॒ यज॑मानाः॒ सर्वा॒न् कामा॒न् कामा॒न् थ्सर्वा॒न्॒. यज॑माना॒ यज॑मानाः॒ सर्वा॒न् कामान्॑ । \newline
40. सर्वा॒न् कामा॒न् कामा॒न् थ्सर्वा॒न् थ्सर्वा॒न् कामा॒ नवाव॒ कामा॒न् थ्सर्वा॒न् थ्सर्वा॒न् कामा॒ नव॑ । \newline
41. कामा॒ नवाव॒ कामा॒न् कामा॒ नव॑ रुन्धते रुन्ध॒ते ऽव॒ कामा॒न् कामा॒ नव॑ रुन्धते । \newline
42. अव॑ रुन्धते रुन्ध॒ते ऽवाव॑ रुन्धते । \newline
43. रु॒न्ध॒त॒ इति॑ रुन्धते । \newline
\pagebreak
\markright{ TS 7.5.4.1  \hfill https://www.vedavms.in \hfill}

\section{ TS 7.5.4.1 }

\textbf{TS 7.5.4.1 } \newline
\textbf{Samhita Paata} \newline

स॒मा॒न्य॑ ऋचो॑ भवन्ति मनुष्यलो॒को वा ऋचो॑ मनुष्यलो॒कादे॒व न य॑न्त्य॒न्यद॑न्य॒थ् साम॑ भवति देवलो॒को वै साम॑ देवलो॒कादे॒वान्यम॑न्यं मनुष्यलो॒कं प्र॑त्यव॒रोह॑न्तो यन्ति॒ जग॑ती॒मग्र॒ उप॑ यन्ति॒ जग॑तीं॒ ॅवै छन्दाꣳ॑सि प्र॒त्यव॑रोहन्त्या-ग्रय॒णं ग्रहा॑ बृ॒हत् पृ॒ष्ठानि॑ त्रयस्त्रिꣳ॒॒शꣳस्तोमा॒-स्तस्मा॒-ज्ज्यायाꣳ॑सं॒ कनी॑यान् प्र॒त्यव॑रोहति वैश्वकर्म॒णो गृ॑ह्यते॒विश्वा᳚न्ये॒व तेन॒ कर्मा॑णि॒ यज॑माना॒ अव॑ रुन्धत आदि॒त्यो-[  ] \newline

\textbf{Pada Paata} \newline

स॒मा॒न्यः॑ । ऋचः॑ । भ॒व॒न्ति॒ । म॒नु॒ष्य॒लो॒क इति॑ मनुष्य - लो॒कः । वै । ऋचः॑ । म॒नु॒ष्य॒लो॒कादिति॑ मनुष्य - लो॒कात् । ए॒व । न । य॒न्ति॒ । अ॒न्यद॑न्य॒दित्य॒न्यत् - अ॒न्य॒त् । साम॑ । भ॒व॒ति॒ । दे॒व॒लो॒क इति॑ देव - लो॒कः । वै । साम॑ । दे॒व॒लो॒कादिति॑ देव - लो॒कात् । ए॒व । अ॒न्यम॑न्य॒मित्य॒न्यं - अ॒न्य॒म् । म॒नु॒ष्य॒लो॒कमिति॑ मनुष्य - लो॒कम् । प्र॒त्य॒व॒रोह॑न्त॒ इति॑ प्रति - अ॒व॒रोह॑न्तः । य॒न्ति॒ । जग॑तीम् । अग्रे᳚ । उपेति॑ । य॒न्ति॒ । जग॑तीम् । वै । छन्दाꣳ॑सि । प्र॒त्यव॑रोह॒न्तीति॑ प्रति - अव॑रोहन्ति । आ॒ग्र॒य॒णम् । ग्रहाः᳚ । बृ॒हत् । पृ॒ष्ठानि॑ । त्र॒य॒स्त्रिꣳ॒॒शमिति॑ त्रयः - त्रिꣳ॒॒शम् । स्तोमाः᳚ । तस्मा᳚त् । ज्यायाꣳ॑सम् । कनी॑यान् । प्र॒त्यव॑रोह॒तीति॑ प्रति - अव॑रोहति । वै॒श्व॒क॒र्म॒ण इति॑ वैश्व - क॒र्म॒णः । गृ॒ह्य॒ते॒ । विश्वा॑नि । ए॒व । तेन॑ । कर्मा॑णि । यज॑मानाः । अवेति॑ । रु॒न्ध॒ते॒ । आ॒दि॒त्यः ।  \newline


\textbf{Krama Paata} \newline

स॒मा॒न्य॑ ऋचः॑ । ऋचो॑ भवन्ति । भ॒व॒न्ति॒ म॒नु॒ष्य॒लो॒कः । म॒नु॒ष्य॒लो॒को वै । म॒नु॒ष्य॒लो॒क इति॑ मनुष्य - लो॒कः । वा ऋचः॑ । ऋचो॑ मनुष्यलो॒कात् । म॒नु॒ष्य॒लो॒कादे॒व । म॒नु॒ष्य॒लो॒कादिति॑ मनुष्य - लो॒कात् । ए॒व न । न य॑न्ति । य॒न्त्य॒न्यद॑न्यत् । अ॒न्यद॑न्य॒थ् साम॑ । अ॒न्यद॑न्य॒दित्य॒न्यत् - अ॒न्य॒त्॒ । साम॑ भवति । भ॒व॒ति॒ दे॒व॒लो॒कः । दे॒व॒लो॒को वै । दे॒व॒लो॒क इति॑ देव - लो॒कः । वै साम॑ । साम॑ देवलो॒कात् । दे॒व॒लो॒कादे॒व । दे॒व॒लो॒कादिति॑ देव - लो॒कात् । ए॒वान्यम॑न्यम् । अ॒न्यम॑न्यम् मनुष्यलो॒कम् । अ॒न्यम॑न्य॒मित्य॒न्यम् - अ॒न्य॒म् । म॒नु॒ष्य॒लो॒कम् प्र॑त्यव॒रोह॑न्तः । म॒नु॒ष्य॒लो॒कमिति॑ मनुष्य - लो॒कम् । प्र॒त्य॒व॒रोह॑न्तो यन्ति । प्र॒त्य॒व॒रोह॑न्त॒ इति॑ प्रति - अ॒व॒रोह॑न्तः । य॒न्ति॒ जग॑तीम् । जग॑ती॒मग्रे᳚ । अग्र॒ उप॑ । उप॑ यन्ति । य॒न्ति॒ जग॑तीम् । जग॑ती॒म् ॅवै । वै छन्दाꣳ॑सि । छन्दाꣳ॑सि प्र॒त्यव॑रोहन्ति । प्र॒त्यव॑रोहन्त्याग्रय॒णम् । प्र॒त्यव॑रोह॒न्तीति॑ प्रति - अव॑रोहन्ति । आ॒ग्र॒य॒णम् ग्रहाः᳚ । ग्रहा॑ बृ॒हत् । बृ॒हत् पृ॒ष्ठानि॑ । पृ॒ष्ठानि॑ त्रयस्त्रिꣳ॒॒शम् । त्र॒य॒स्त्रिꣳ॒॒शꣳ स्तोमाः᳚ । त्र॒य॒स्त्रिꣳ॒॒शमिति॑ त्रयः - त्रिꣳ॒॒शम् । स्तोमा॒स्तस्मा᳚त् । तस्मा॒ज् ज्यायाꣳ॑सम् । ज्यायाꣳ॑स॒म् कनी॑यान् । कनी॑यान् प्र॒त्यव॑रोहति । प्र॒त्यव॑रोहति वैश्वकर्म॒णः । प्र॒त्यव॑रोह॒तीति॑ प्रति - अव॑रोहति । वै॒श्व॒क॒र्म॒णो गृ॑ह्यते । वै॒श्व॒क॒र्म॒ण इति॑ वैश्व - क॒र्म॒णः । गृ॒ह्य॒ते॒ विश्वा॑नि । विश्वा᳚न्ये॒व । ए॒व तेन॑ । तेन॒ कर्मा॑णि । कर्मा॑णि॒ यज॑मानाः । यज॑माना॒ अव॑ । अव॑ रुन्धते । रु॒न्ध॒त॒ आ॒दि॒त्यः । आ॒दि॒त्यो गृ॑ह्यते \newline

\textbf{Jatai Paata} \newline

1. स॒मा॒न्य॑ ऋच॒ ऋचः॑ समा॒न्यः॑ समा॒न्य॑ ऋचः॑ । \newline
2. ऋचो॑ भवन्ति भव॒न्त्यृच॒ ऋचो॑ भवन्ति । \newline
3. भ॒व॒न्ति॒ म॒नु॒ष्य॒लो॒को म॑नुष्यलो॒को भ॑वन्ति भवन्ति मनुष्यलो॒कः । \newline
4. म॒नु॒ष्य॒लो॒को वै वै म॑नुष्यलो॒को म॑नुष्यलो॒को वै । \newline
5. म॒नु॒ष्य॒लो॒क इति॑ मनुष्य - लो॒कः । \newline
6. वा ऋच॒ ऋचो॒ वै वा ऋचः॑ । \newline
7. ऋचो॑ मनुष्यलो॒कान् म॑नुष्यलो॒का दृच॒ ऋचो॑ मनुष्यलो॒कात् । \newline
8. म॒नु॒ष्य॒लो॒का दे॒वैव म॑नुष्यलो॒कान् म॑नुष्यलो॒का दे॒व । \newline
9. म॒नु॒ष्य॒लो॒कादिति॑ मनुष्य - लो॒कात् । \newline
10. ए॒व न नैवैव न । \newline
11. न य॑न्ति यन्ति॒ न न य॑न्ति । \newline
12. य॒न्त्य॒न्यद॑ न्यद॒न्यद॑न्यद् यन्ति यन्त्य॒न्यद॑न्यत् । \newline
13. अ॒न्यद॑न्य॒थ् साम॒ सामा॒ न्यद॑न्यद॒ न्यद॑न्य॒थ् साम॑ । \newline
14. अ॒न्यद॑न्य॒दित्य॒न्यत् - अ॒न्य॒त् । \newline
15. साम॑ भवति भवति॒ साम॒ साम॑ भवति । \newline
16. भ॒व॒ति॒ दे॒व॒लो॒को दे॑वलो॒को भ॑वति भवति देवलो॒कः । \newline
17. दे॒व॒लो॒को वै वै दे॑वलो॒को दे॑वलो॒को वै । \newline
18. दे॒व॒लो॒क इति॑ देव - लो॒कः । \newline
19. वै साम॒ साम॒ वै वै साम॑ । \newline
20. साम॑ देवलो॒काद् दे॑वलो॒काथ् साम॒ साम॑ देवलो॒कात् । \newline
21. दे॒व॒लो॒का दे॒वैव दे॑वलो॒काद् दे॑वलो॒का दे॒व । \newline
22. दे॒व॒लो॒कादिति॑ देव - लो॒कात् । \newline
23. ए॒वा न्यम॑न्य म॒न्यम॑न्य मे॒वै वान्यम॑न्यम् । \newline
24. अ॒न्यम॑न्यम् मनुष्यलो॒कम् म॑नुष्यलो॒क म॒न्यम॑न्य म॒न्यम॑न्यम् मनुष्यलो॒कम् । \newline
25. अ॒न्यम॑न्य॒मित्य॒न्यं - अ॒न्य॒म् । \newline
26. म॒नु॒ष्य॒लो॒कम् प्र॑त्यव॒रोह॑न्तः प्रत्यव॒रोह॑न्तो मनुष्यलो॒कम् म॑नुष्यलो॒कम् प्र॑त्यव॒रोह॑न्तः । \newline
27. म॒नु॒ष्य॒लो॒कमिति॑ मनुष्य - लो॒कम् । \newline
28. प्र॒त्य॒व॒रोह॑न्तो यन्ति यन्ति प्रत्यव॒रोह॑न्तः प्रत्यव॒रोह॑न्तो यन्ति । \newline
29. प्र॒त्य॒व॒रोह॑न्त॒ इति॑ प्रति - अ॒व॒रोह॑न्तः । \newline
30. य॒न्ति॒ जग॑ती॒म् जग॑तीं ॅयन्ति यन्ति॒ जग॑तीम् । \newline
31. जग॑ती॒ मग्रे ऽग्रे॒ जग॑ती॒म् जग॑ती॒ मग्रे᳚ । \newline
32. अग्र॒ उपोपाग्रे ऽग्र॒ उप॑ । \newline
33. उप॑ यन्ति य॒न्त्युपोप॑ यन्ति । \newline
34. य॒न्ति॒ जग॑ती॒म् जग॑तीं ॅयन्ति यन्ति॒ जग॑तीम् । \newline
35. जग॑तीं॒ ॅवै वै जग॑ती॒म् जग॑तीं॒ ॅवै । \newline
36. वै छन्दाꣳ॑सि॒ छन्दाꣳ॑सि॒ वै वै छन्दाꣳ॑सि । \newline
37. छन्दाꣳ॑सि प्र॒त्यव॑रोहन्ति प्र॒त्यव॑रोहन्ति॒ छन्दाꣳ॑सि॒ छन्दाꣳ॑सि प्र॒त्यव॑रोहन्ति । \newline
38. प्र॒त्यव॑रोह न्त्याग्रय॒ण मा᳚ग्रय॒णम् प्र॒त्यव॑रोहन्ति प्र॒त्यव॑रोह न्त्याग्रय॒णम् । \newline
39. प्र॒त्यव॑रोह॒न्तीति॑ प्रति - अव॑रोहन्ति । \newline
40. आ॒ग्र॒य॒णम् ग्रहा॒ ग्रहा॑ आग्रय॒ण मा᳚ग्रय॒णम् ग्रहाः᳚ । \newline
41. ग्रहा॑ बृ॒हद् बृ॒हद् ग्रहा॒ ग्रहा॑ बृ॒हत् । \newline
42. बृ॒हत् पृ॒ष्ठानि॑ पृ॒ष्ठानि॑ बृ॒हद् बृ॒हत् पृ॒ष्ठानि॑ । \newline
43. पृ॒ष्ठानि॑ त्रयस्त्रिꣳ॒॒शम् त्र॑यस्त्रिꣳ॒॒शम् पृ॒ष्ठानि॑ पृ॒ष्ठानि॑ त्रयस्त्रिꣳ॒॒शम् । \newline
44. त्र॒य॒स्त्रिꣳ॒॒शꣳ स्तोमाः॒ स्तोमा᳚ स्त्रयस्त्रिꣳ॒॒शम् त्र॑यस्त्रिꣳ॒॒शꣳ स्तोमाः᳚ । \newline
45. त्र॒य॒स्त्रिꣳ॒॒शमिति॑ त्रयः - त्रिꣳ॒॒शम् । \newline
46. स्तोमा॒ स्तस्मा॒त् तस्मा॒थ् स्तोमाः॒ स्तोमा॒ स्तस्मा᳚त् । \newline
47. तस्मा॒ज् ज्यायाꣳ॑स॒म् ज्यायाꣳ॑स॒म् तस्मा॒त् तस्मा॒ज् ज्यायाꣳ॑सम् । \newline
48. ज्यायाꣳ॑स॒म् कनी॑या॒न् कनी॑या॒न् ज्यायाꣳ॑स॒म् ज्यायाꣳ॑स॒म् कनी॑यान् । \newline
49. कनी॑यान् प्र॒त्यव॑रोहति प्र॒त्यव॑रोहति॒ कनी॑या॒न् कनी॑यान् प्र॒त्यव॑रोहति । \newline
50. प्र॒त्यव॑रोहति वैश्वकर्म॒णो वै᳚श्वकर्म॒णः प्र॒त्यव॑रोहति प्र॒त्यव॑रोहति वैश्वकर्म॒णः । \newline
51. प्र॒त्यव॑रोह॒तीति॑ प्रति - अव॑रोहति । \newline
52. वै॒श्व॒क॒र्म॒णो गृ॑ह्यते गृह्यते वैश्वकर्म॒णो वै᳚श्वकर्म॒णो गृ॑ह्यते । \newline
53. वै॒श्व॒क॒र्म॒ण इति॑ वैश्व - क॒र्म॒णः । \newline
54. गृ॒ह्य॒ते॒ विश्वा॑नि॒ विश्वा॑नि गृह्यते गृह्यते॒ विश्वा॑नि । \newline
55. विश्वा᳚न्ये॒वैव विश्वा॑नि॒ विश्वा᳚ न्ये॒व । \newline
56. ए॒व तेन॒ तेनै॒वैव तेन॑ । \newline
57. तेन॒ कर्मा॑णि॒ कर्मा॑णि॒ तेन॒ तेन॒ कर्मा॑णि । \newline
58. कर्मा॑णि॒ यज॑माना॒ यज॑मानाः॒ कर्मा॑णि॒ कर्मा॑णि॒ यज॑मानाः । \newline
59. यज॑माना॒ अवाव॒ यज॑माना॒ यज॑माना॒ अव॑ । \newline
60. अव॑ रुन्धते रुन्ध॒ते ऽवाव॑ रुन्धते । \newline
61. रु॒न्ध॒त॒ आ॒दि॒त्य आ॑दि॒त्यो रु॑न्धते रुन्धत आदि॒त्यः । \newline
62. आ॒दि॒त्यो गृ॑ह्यते गृह्यत आदि॒त्य आ॑दि॒त्यो गृ॑ह्यते । \newline

\textbf{Ghana Paata } \newline

1. स॒मा॒न्य॑ ऋच॒ ऋचः॑ समा॒न्यः॑ समा॒न्य॑ ऋचो॑ भवन्ति भव॒ न्त्यृचः॑ समा॒न्यः॑ समा॒न्य॑ ऋचो॑ भवन्ति । \newline
2. ऋचो॑ भवन्ति भव॒ न्त्यृच॒ ऋचो॑ भवन्ति मनुष्यलो॒को म॑नुष्यलो॒को भ॑व॒ न्त्यृच॒ ऋचो॑ भवन्ति मनुष्यलो॒कः । \newline
3. भ॒व॒न्ति॒ म॒नु॒ष्य॒लो॒को म॑नुष्यलो॒को भ॑वन्ति भवन्ति मनुष्यलो॒को वै वै म॑नुष्यलो॒को भ॑वन्ति भवन्ति मनुष्यलो॒को वै । \newline
4. म॒नु॒ष्य॒लो॒को वै वै म॑नुष्यलो॒को म॑नुष्यलो॒को वा ऋच॒ ऋचो॒ वै म॑नुष्यलो॒को म॑नुष्यलो॒को वा ऋचः॑ । \newline
5. म॒नु॒ष्य॒लो॒क इति॑ मनुष्य - लो॒कः । \newline
6. वा ऋच॒ ऋचो॒ वै वा ऋचो॑ मनुष्यलो॒कान् म॑नुष्यलो॒का दृचो॒ वै वा ऋचो॑ मनुष्यलो॒कात् । \newline
7. ऋचो॑ मनुष्यलो॒कान् म॑नुष्यलो॒का दृच॒ ऋचो॑ मनुष्यलो॒का दे॒वैव म॑नुष्यलो॒का दृच॒ ऋचो॑ मनुष्यलो॒का दे॒व । \newline
8. म॒नु॒ष्य॒लो॒का दे॒वैव म॑नुष्यलो॒कान् म॑नुष्यलो॒का दे॒व न नैव म॑नुष्यलो॒कान् म॑नुष्यलो॒का दे॒व न । \newline
9. म॒नु॒ष्य॒लो॒कादिति॑ मनुष्य - लो॒कात् । \newline
10. ए॒व न नैवैव न य॑न्ति यन्ति॒ नैवैव न य॑न्ति । \newline
11. न य॑न्ति यन्ति॒ न न य॑न्त्य॒न्यद॑न्य द॒न्यद॑न्यद् यन्ति॒ न न य॑न्त्य॒न्यद॑न्यत् । \newline
12. य॒न्त्य॒न्यद॑न्य द॒न्यद॑न्यद् यन्ति यन्त्य॒न्यद॑न्य॒थ् साम॒ सामा॒ न्यद॑न्यद् यन्ति यन्त्य॒न्यद॑न्य॒थ् साम॑ । \newline
13. अ॒न्यद॑न्य॒थ् साम॒ सामा॒ न्यद॑न्यद॒ न्यद॑न्य॒थ् साम॑ भवति भवति॒ सामा॒ न्यद॑न्यद॒ न्यद॑न्य॒थ् साम॑ भवति । \newline
14. अ॒न्यद॑न्य॒दित्य॒न्यत् - अ॒न्य॒त् । \newline
15. साम॑ भवति भवति॒ साम॒ साम॑ भवति देवलो॒को दे॑वलो॒को भ॑वति॒ साम॒ साम॑ भवति देवलो॒कः । \newline
16. भ॒व॒ति॒ दे॒व॒लो॒को दे॑वलो॒को भ॑वति भवति देवलो॒को वै वै दे॑वलो॒को भ॑वति भवति देवलो॒को वै । \newline
17. दे॒व॒लो॒को वै वै दे॑वलो॒को दे॑वलो॒को वै साम॒ साम॒ वै दे॑वलो॒को दे॑वलो॒को वै साम॑ । \newline
18. दे॒व॒लो॒क इति॑ देव - लो॒कः । \newline
19. वै साम॒ साम॒ वै वै साम॑ देवलो॒काद् दे॑वलो॒काथ् साम॒ वै वै साम॑ देवलो॒कात् । \newline
20. साम॑ देवलो॒काद् दे॑वलो॒काथ् साम॒ साम॑ देवलो॒का दे॒वैव दे॑वलो॒काथ् साम॒ साम॑ देवलो॒का दे॒व । \newline
21. दे॒व॒लो॒का दे॒वैव दे॑वलो॒काद् दे॑वलो॒का दे॒वा न्यम॑न्य म॒न्यम॑न्य मे॒व दे॑वलो॒काद् दे॑वलो॒का दे॒वा न्यम॑न्यम् । \newline
22. दे॒व॒लो॒कादिति॑ देव - लो॒कात् । \newline
23. ए॒वा न्यम॑न्य म॒न्यम॑न्य मे॒वै वान्यम॑न्यम् मनुष्यलो॒कम् म॑नुष्यलो॒क म॒न्यम॑न्य मे॒वैवा न्यम॑न्यम् मनुष्यलो॒कम् । \newline
24. अ॒न्यम॑न्यम् मनुष्यलो॒कम् म॑नुष्यलो॒क म॒न्यम॑न्य म॒न्यम॑न्यम् मनुष्यलो॒कम् प्र॑त्यव॒रोह॑न्तः प्रत्यव॒रोह॑न्तो मनुष्यलो॒क म॒न्यम॑न्य म॒न्यम॑न्यम् मनुष्यलो॒कम् प्र॑त्यव॒रोह॑न्तः । \newline
25. अ॒न्यम॑न्य॒मित्य॒न्यं - अ॒न्य॒म् । \newline
26. म॒नु॒ष्य॒लो॒कम् प्र॑त्यव॒रोह॑न्तः प्रत्यव॒रोह॑न्तो मनुष्यलो॒कम् म॑नुष्यलो॒कम् प्र॑त्यव॒रोह॑न्तो यन्ति यन्ति प्रत्यव॒रोह॑न्तो मनुष्यलो॒कम् म॑नुष्यलो॒कम् प्र॑त्यव॒रोह॑न्तो यन्ति । \newline
27. म॒नु॒ष्य॒लो॒कमिति॑ मनुष्य - लो॒कम् । \newline
28. प्र॒त्य॒व॒रोह॑न्तो यन्ति यन्ति प्रत्यव॒रोह॑न्तः प्रत्यव॒रोह॑न्तो यन्ति॒ जग॑ती॒म् जग॑तीं ॅयन्ति प्रत्यव॒रोह॑न्तः प्रत्यव॒रोह॑न्तो यन्ति॒ जग॑तीम् । \newline
29. प्र॒त्य॒व॒रोह॑न्त॒ इति॑ प्रति - अ॒व॒रोह॑न्तः । \newline
30. य॒न्ति॒ जग॑ती॒म् जग॑तीं ॅयन्ति यन्ति॒ जग॑ती॒ मग्रे ऽग्रे॒ जग॑तीं ॅयन्ति यन्ति॒ जग॑ती॒ मग्रे᳚ । \newline
31. जग॑ती॒ मग्रे ऽग्रे॒ जग॑ती॒म् जग॑ती॒ मग्र॒ उपो पाग्रे॒ जग॑ती॒म् जग॑ती॒ मग्र॒ उप॑ । \newline
32. अग्र॒ उपो पाग्रे ऽग्र॒ उप॑ यन्ति य॒न्त्युपाग्रे ऽग्र॒ उप॑ यन्ति । \newline
33. उप॑ यन्ति य॒न्त्युपोप॑ यन्ति॒ जग॑ती॒म् जग॑तीं ॅय॒न्त्युपोप॑ यन्ति॒ जग॑तीम् । \newline
34. य॒न्ति॒ जग॑ती॒म् जग॑तीं ॅयन्ति यन्ति॒ जग॑तीं॒ ॅवै वै जग॑तीं ॅयन्ति यन्ति॒ जग॑तीं॒ ॅवै । \newline
35. जग॑तीं॒ ॅवै वै जग॑ती॒म् जग॑तीं॒ ॅवै छन्दाꣳ॑सि॒ छन्दाꣳ॑सि॒ वै जग॑ती॒म् जग॑तीं॒ ॅवै छन्दाꣳ॑सि । \newline
36. वै छन्दाꣳ॑सि॒ छन्दाꣳ॑सि॒ वै वै छन्दाꣳ॑सि प्र॒त्यव॑रोहन्ति प्र॒त्यव॑रोहन्ति॒ छन्दाꣳ॑सि॒ वै वै छन्दाꣳ॑सि प्र॒त्यव॑रोहन्ति । \newline
37. छन्दाꣳ॑सि प्र॒त्यव॑रोहन्ति प्र॒त्यव॑रोहन्ति॒ छन्दाꣳ॑सि॒ छन्दाꣳ॑सि प्र॒त्यव॑रोह न्त्याग्रय॒ण मा᳚ग्रय॒णम् प्र॒त्यव॑रोहन्ति॒ छन्दाꣳ॑सि॒ छन्दाꣳ॑सि प्र॒त्यव॑रोह न्त्याग्रय॒णम् । \newline
38. प्र॒त्यव॑रोह न्त्याग्रय॒ण मा᳚ग्रय॒णम् प्र॒त्यव॑रोहन्ति प्र॒त्यव॑रोह न्त्याग्रय॒णम् ग्रहा॒ ग्रहा॑ आग्रय॒णम् प्र॒त्यव॑रोहन्ति प्र॒त्यव॑रोह न्त्याग्रय॒णम् ग्रहाः᳚ । \newline
39. प्र॒त्यव॑रोह॒न्तीति॑ प्रति - अव॑रोहन्ति । \newline
40. आ॒ग्र॒य॒णम् ग्रहा॒ ग्रहा॑ आग्रय॒ण मा᳚ग्रय॒णम् ग्रहा॑ बृ॒हद् बृ॒हद् ग्रहा॑ आग्रय॒ण मा᳚ग्रय॒णम् ग्रहा॑ बृ॒हत् । \newline
41. ग्रहा॑ बृ॒हद् बृ॒हद् ग्रहा॒ ग्रहा॑ बृ॒हत् पृ॒ष्ठानि॑ पृ॒ष्ठानि॑ बृ॒हद् ग्रहा॒ ग्रहा॑ बृ॒हत् पृ॒ष्ठानि॑ । \newline
42. बृ॒हत् पृ॒ष्ठानि॑ पृ॒ष्ठानि॑ बृ॒हद् बृ॒हत् पृ॒ष्ठानि॑ त्रयस्त्रिꣳ॒॒शम् त्र॑यस्त्रिꣳ॒॒शम् पृ॒ष्ठानि॑ बृ॒हद् बृ॒हत् पृ॒ष्ठानि॑ त्रयस्त्रिꣳ॒॒शम् । \newline
43. पृ॒ष्ठानि॑ त्रयस्त्रिꣳ॒॒शम् त्र॑यस्त्रिꣳ॒॒शम् पृ॒ष्ठानि॑ पृ॒ष्ठानि॑ त्रयस्त्रिꣳ॒॒शꣳ स्तोमाः॒ स्तोमा᳚ स्त्रयस्त्रिꣳ॒॒शम् पृ॒ष्ठानि॑ पृ॒ष्ठानि॑ त्रयस्त्रिꣳ॒॒शꣳ स्तोमाः᳚ । \newline
44. त्र॒य॒स्त्रिꣳ॒॒शꣳ स्तोमाः॒ स्तोमा᳚ स्त्रयस्त्रिꣳ॒॒शम् त्र॑यस्त्रिꣳ॒॒शꣳ स्तोमा॒ स्तस्मा॒त् तस्मा॒थ् स्तोमा᳚ स्त्रयस्त्रिꣳ॒॒शम् त्र॑यस्त्रिꣳ॒॒शꣳ स्तोमा॒ स्तस्मा᳚त् । \newline
45. त्र॒य॒स्त्रिꣳ॒॒शमिति॑ त्रयः - त्रिꣳ॒॒शम् । \newline
46. स्तोमा॒ स्तस्मा॒त् तस्मा॒थ् स्तोमाः॒ स्तोमा॒ स्तस्मा॒ज् ज्यायाꣳ॑स॒म् ज्यायाꣳ॑स॒म् तस्मा॒थ् स्तोमाः॒ स्तोमा॒ स्तस्मा॒ज् ज्यायाꣳ॑सम् । \newline
47. तस्मा॒ज् ज्यायाꣳ॑स॒म् ज्यायाꣳ॑स॒म् तस्मा॒त् तस्मा॒ज् ज्यायाꣳ॑स॒म् कनी॑या॒न् कनी॑या॒न् ज्यायाꣳ॑स॒म् तस्मा॒त् तस्मा॒ज् ज्यायाꣳ॑स॒म् कनी॑यान् । \newline
48. ज्यायाꣳ॑स॒म् कनी॑या॒न् कनी॑या॒न् ज्यायाꣳ॑स॒म् ज्यायाꣳ॑स॒म् कनी॑यान् प्र॒त्यव॑रोहति प्र॒त्यव॑रोहति॒ कनी॑या॒न् ज्यायाꣳ॑स॒म् ज्यायाꣳ॑स॒म् कनी॑यान् प्र॒त्यव॑रोहति । \newline
49. कनी॑यान् प्र॒त्यव॑रोहति प्र॒त्यव॑रोहति॒ कनी॑या॒न् कनी॑यान् प्र॒त्यव॑रोहति वैश्वकर्म॒णो वै᳚श्वकर्म॒णः प्र॒त्यव॑रोहति॒ कनी॑या॒न् कनी॑यान् प्र॒त्यव॑रोहति वैश्वकर्म॒णः । \newline
50. प्र॒त्यव॑रोहति वैश्वकर्म॒णो वै᳚श्वकर्म॒णः प्र॒त्यव॑रोहति प्र॒त्यव॑रोहति वैश्वकर्म॒णो गृ॑ह्यते गृह्यते वैश्वकर्म॒णः प्र॒त्यव॑रोहति प्र॒त्यव॑रोहति वैश्वकर्म॒णो गृ॑ह्यते । \newline
51. प्र॒त्यव॑रोह॒तीति॑ प्रति - अव॑रोहति । \newline
52. वै॒श्व॒क॒र्म॒णो गृ॑ह्यते गृह्यते वैश्वकर्म॒णो वै᳚श्वकर्म॒णो गृ॑ह्यते॒ विश्वा॑नि॒ विश्वा॑नि गृह्यते वैश्वकर्म॒णो वै᳚श्वकर्म॒णो गृ॑ह्यते॒ विश्वा॑नि । \newline
53. वै॒श्व॒क॒र्म॒ण इति॑ वैश्व - क॒र्म॒णः । \newline
54. गृ॒ह्य॒ते॒ विश्वा॑नि॒ विश्वा॑नि गृह्यते गृह्यते॒ विश्वा᳚ न्ये॒वैव विश्वा॑नि गृह्यते गृह्यते॒ विश्वा᳚ न्ये॒व । \newline
55. विश्वा᳚ न्ये॒वैव विश्वा॑नि॒ विश्वा᳚ न्ये॒व तेन॒ तेनै॒व विश्वा॑नि॒ विश्वा᳚ न्ये॒व तेन॑ । \newline
56. ए॒व तेन॒ तेनै॒वैव तेन॒ कर्मा॑णि॒ कर्मा॑णि॒ तेनै॒वैव तेन॒ कर्मा॑णि । \newline
57. तेन॒ कर्मा॑णि॒ कर्मा॑णि॒ तेन॒ तेन॒ कर्मा॑णि॒ यज॑माना॒ यज॑मानाः॒ कर्मा॑णि॒ तेन॒ तेन॒ कर्मा॑णि॒ यज॑मानाः । \newline
58. कर्मा॑णि॒ यज॑माना॒ यज॑मानाः॒ कर्मा॑णि॒ कर्मा॑णि॒ यज॑माना॒ अवाव॒ यज॑मानाः॒ कर्मा॑णि॒ कर्मा॑णि॒ यज॑माना॒ अव॑ । \newline
59. यज॑माना॒ अवाव॒ यज॑माना॒ यज॑माना॒ अव॑ रुन्धते रुन्ध॒ते ऽव॒ यज॑माना॒ यज॑माना॒ अव॑ रुन्धते । \newline
60. अव॑ रुन्धते रुन्ध॒ते ऽवाव॑ रुन्धत आदि॒त्य आ॑दि॒त्यो रु॑न्ध॒ते ऽवाव॑ रुन्धत आदि॒त्यः । \newline
61. रु॒न्ध॒त॒ आ॒दि॒त्य आ॑दि॒त्यो रु॑न्धते रुन्धत आदि॒त्यो गृ॑ह्यते गृह्यत आदि॒त्यो रु॑न्धते रुन्धत आदि॒त्यो गृ॑ह्यते । \newline
62. आ॒दि॒त्यो गृ॑ह्यते गृह्यत आदि॒त्य आ॑दि॒त्यो गृ॑ह्यत इ॒य मि॒यम् गृ॑ह्यत आदि॒त्य आ॑दि॒त्यो गृ॑ह्यत इ॒यम् । \newline
\pagebreak
\markright{ TS 7.5.4.2  \hfill https://www.vedavms.in \hfill}

\section{ TS 7.5.4.2 }

\textbf{TS 7.5.4.2 } \newline
\textbf{Samhita Paata} \newline

गृ॑ह्यत इ॒यं ॅवा अदि॑तिर॒स्यामे॒व प्रति॑ तिष्ठन्त्य॒न्यो᳚ऽन्यो गृह्येते मिथुन॒त्वाय॒ प्रजा᳚त्या अवान्त॒रं ॅवै द॑शरा॒त्रेण॑ प्र॒जाप॑तिः प्र॒जा अ॑सृजत॒ यद्-द॑शरा॒त्रो भव॑ति प्र॒जा ए॒व तद्-यज॑मानाः सृजन्त ए॒ताꣳ ह॒ वा उ॑द॒ङ्कः शौ᳚ल्बाय॒नः स॒त्रस्यर्द्धि॑मुवाच॒ यद्-द॑शरा॒त्रोयद्-द॑शरा॒त्रो भव॑ति स॒त्रस्यर्द्ध्या॒ अथो॒ यदे॒व पूर्वे॒ष्वह॑स्सु॒ विलो॑म क्रि॒यते॒ तस्यै॒वै ( )-षा शान्तिः॑ ॥ \newline

\textbf{Pada Paata} \newline

गृ॒ह्य॒ते॒ । इ॒यम् । वै । अदि॑तिः । अ॒स्याम् । ए॒व । प्रतीति॑ । ति॒ष्ठ॒न्ति॒ । अ॒न्यो᳚न्य॒ इत्य॒न्यः - अ॒न्यः॒ । गृ॒ह्ये॒ते॒ इति॑ । मि॒थु॒न॒त्वायेति॑ मिथुन - त्वाय॑ । प्रजा᳚त्या॒ इति॒ प्र - जा॒त्यै॒ । अ॒वा॒न्त॒रमित्य॑व - अ॒न्त॒रम् । वै । द॒श॒रा॒त्रेणेति॑ दश - रा॒त्रेण॑ । प्र॒जाप॑ति॒रिति॑ प्र॒जा - प॒तिः॒ । प्र॒जा इति॑ प्र - जाः । अ॒सृ॒ज॒त॒ । यत् । द॒श॒रा॒त्र इति॑ दश - रा॒त्रः । भव॑ति । प्र॒जा इति॑ प्र - जाः । ए॒व । तत् । यज॑मानाः । सृ॒ज॒न्ते॒ । ए॒ताम् । ह॒ । वै । उ॒द॒ङ्कः । शौ॒ल्बा॒य॒नः । स॒त्रस्य॑ । ऋद्धि᳚म् । उ॒वा॒च॒ । यत् । द॒श॒रा॒त्र इति॑ दश - रा॒त्रः । यत् । द॒श॒रा॒त्र इति॑ दश - रा॒त्रः । भव॑ति । स॒त्रस्य॑ । ऋद्ध्यै᳚ । अथो॒ इति॑ । यत् । ए॒व । पूर्वे॑षु । अह॒स्स्वित्यहः॑ - सु॒ । विलो॒मेति॒ वि - लो॒म॒ । क्रि॒यते᳚ । तस्य॑ । ए॒व ( ) । ए॒षा । शान्तिः॑ ॥  \newline


\textbf{Krama Paata} \newline

गृ॒ह्य॒त॒ इ॒यम् । इ॒यम् ॅवै । वा अदि॑तिः । अदि॑तिर॒स्याम् । अ॒स्यामे॒व । ए॒व प्रति॑ । प्रति॑ तिष्ठन्ति । ति॒ष्ठ॒न्त्य॒न्यो᳚न्यः । अ॒न्यो᳚न्यो गृह्येते । अ॒न्यो᳚न्य॒ इत्य॒न्यः - अ॒न्यः॒ । गृ॒ह्ये॒ते॒ मि॒थु॒न॒त्वाय॑ । गृ॒ह्ये॒ते॒ इति॑ गृह्येते । मि॒थु॒न॒त्वाय॒ प्रजा᳚त्यै । मि॒थु॒न॒त्वायेति॑ मिथुन - त्वाय॑ । प्रजा᳚त्या अवान्त॒रम् । प्रजा᳚त्या॒ इति॒ प्र - जा॒त्यै॒ । अ॒वा॒न्त॒रम् ॅवै । अ॒वा॒न्त॒रमित्य॑व - अ॒न्त॒रम् । वै द॑शरा॒त्रेण॑ । द॒श॒रा॒त्रेण॑ प्र॒जाप॑तिः । द॒श॒रा॒त्रेणेति॑ दश - रा॒त्रेण॑ । प्र॒जाप॑तिः प्र॒जाः । प्र॒जाप॑ति॒रिति॑ प्र॒जा - प॒तिः॒ । प्र॒जा अ॑सृजत । प्र॒जा इति॑ प्र - जाः । अ॒सृ॒ज॒त॒ यत् । यद् द॑शरा॒त्रः । द॒श॒रा॒त्रो भव॑ति । द॒श॒रा॒त्र इति॑ दश - रा॒त्रः । भव॑ति प्र॒जाः । प्र॒जा ए॒व । प्र॒जा इति॑ प्र - जाः । ए॒व तत् । तद् यज॑मानाः । यज॑मानाः सृजन्ते । सृ॒ज॒न्त॒ ए॒ताम् । ए॒ताꣳ ह॑ । ह॒ वै । वा उ॑द॒ङ्‍कः । उ॒द॒ङ्‍कः शौ᳚ल्बाय॒नः । शौ॒ल्बा॒य॒नः स॒त्रस्य॑ । स॒त्रस्यर्द्धि᳚म् । ऋद्धि॑मुवाच । उ॒वा॒च॒ यत् । यद् द॑शरा॒त्रः । द॒श॒रा॒त्रो यत् । द॒श॒रा॒त्र इति॑ दश - रा॒त्रः । यद् द॑शरा॒त्रः । द॒श॒रा॒त्रो भव॑ति । द॒श॒रा॒त्र इति॑ दश - रा॒त्रः । भव॑ति स॒त्रस्य॑ । स॒त्रस्यर्द्ध्यै᳚ । ऋद्ध्या॒ अथो᳚ । अथो॒ यत् । अथो॒ इत्यथो᳚ । यदे॒व । ए॒व पूर्वे॑षु । पूर्वे॒ष्वह॑स्सु । अह॑स्सु॒ विलो॑म । अह॒स्स्वित्यहः॑ - सु॒ । विलो॑म क्रि॒यते᳚ । विलो॒मेति॒ वि - लो॒म॒ । क्रि॒यते॒ तस्य॑ । तस्यै॒व । ए॒वैषा ( ) । ए॒षा शान्तिः॑ । शान्ति॒रिति॒ शान्तिः॑ । \newline

\textbf{Jatai Paata} \newline

1. गृ॒ह्य॒त॒ इ॒य मि॒यम् गृ॑ह्यते गृह्यत इ॒यम् । \newline
2. इ॒यं ॅवै वा इ॒य मि॒यं ॅवै । \newline
3. वा अदि॑ति॒ रदि॑ति॒र् वै वा अदि॑तिः । \newline
4. अदि॑ति र॒स्या म॒स्या मदि॑ति॒ रदि॑ति र॒स्याम् । \newline
5. अ॒स्या मे॒वै वास्या म॒स्या मे॒व । \newline
6. ए॒व प्रति॒ प्रत्ये॒वैव प्रति॑ । \newline
7. प्रति॑ तिष्ठन्ति तिष्ठन्ति॒ प्रति॒ प्रति॑ तिष्ठन्ति । \newline
8. ति॒ष्ठ॒ न्त्य॒न्यो᳚न्यो॒ ऽन्यो᳚न्य स्तिष्ठन्ति तिष्ठ न्त्य॒न्यो᳚न्यः । \newline
9. अ॒न्यो᳚न्यो गृह्येते गृह्येते अ॒न्यो᳚न्यो॒ ऽन्यो᳚न्यो गृह्येते । \newline
10. अ॒न्यो᳚न्य॒ इत्य॒न्यः - अ॒न्यः॒ । \newline
11. गृ॒ह्ये॒ते॒ मि॒थु॒न॒त्वाय॑ मिथुन॒त्वाय॑ गृह्येते गृह्येते मिथुन॒त्वाय॑ । \newline
12. गृ॒ह्ये॒ते॒ इति॑ गृह्येते । \newline
13. मि॒थु॒न॒त्वाय॒ प्रजा᳚त्यै॒ प्रजा᳚त्यै मिथुन॒त्वाय॑ मिथुन॒त्वाय॒ प्रजा᳚त्यै । \newline
14. मि॒थु॒न॒त्वायेति॑ मिथुन - त्वाय॑ । \newline
15. प्रजा᳚त्या अवान्त॒र म॑वान्त॒रम् प्रजा᳚त्यै॒ प्रजा᳚त्या अवान्त॒रम् । \newline
16. प्रजा᳚त्या॒ इति॒ प्र - जा॒त्यै॒ । \newline
17. अ॒वा॒न्त॒रं ॅवै वा अ॑वान्त॒र म॑वान्त॒रं ॅवै । \newline
18. अ॒वा॒न्त॒रमित्य॑व - अ॒न्त॒रम् । \newline
19. वै द॑शरा॒त्रेण॑ दशरा॒त्रेण॒ वै वै द॑शरा॒त्रेण॑ । \newline
20. द॒श॒रा॒त्रेण॑ प्र॒जाप॑तिः प्र॒जाप॑तिर् दशरा॒त्रेण॑ दशरा॒त्रेण॑ प्र॒जाप॑तिः । \newline
21. द॒श॒रा॒त्रेणेति॑ दश - रा॒त्रेण॑ । \newline
22. प्र॒जाप॑तिः प्र॒जाः प्र॒जाः प्र॒जाप॑तिः प्र॒जाप॑तिः प्र॒जाः । \newline
23. प्र॒जाप॑ति॒रिति॑ प्र॒जा - प॒तिः॒ । \newline
24. प्र॒जा अ॑सृजता सृजत प्र॒जाः प्र॒जा अ॑सृजत । \newline
25. प्र॒जा इति॑ प्र - जाः । \newline
26. अ॒सृ॒ज॒त॒ यद् यद॑सृजता सृजत॒ यत् । \newline
27. यद् द॑शरा॒त्रो द॑शरा॒त्रो यद् यद् द॑शरा॒त्रः । \newline
28. द॒श॒रा॒त्रो भव॑ति॒ भव॑ति दशरा॒त्रो द॑शरा॒त्रो भव॑ति । \newline
29. द॒श॒रा॒त्र इति॑ दश - रा॒त्रः । \newline
30. भव॑ति प्र॒जाः प्र॒जा भव॑ति॒ भव॑ति प्र॒जाः । \newline
31. प्र॒जा ए॒वैव प्र॒जाः प्र॒जा ए॒व । \newline
32. प्र॒जा इति॑ प्र - जाः । \newline
33. ए॒व तत् तदे॒वैव तत् । \newline
34. तद् यज॑माना॒ यज॑माना॒ स्तत् तद् यज॑मानाः । \newline
35. यज॑मानाः सृजन्ते सृजन्ते॒ यज॑माना॒ यज॑मानाः सृजन्ते । \newline
36. सृ॒ज॒न्त॒ ए॒ता मे॒ताꣳ सृ॑जन्ते सृजन्त ए॒ताम् । \newline
37. ए॒ताꣳ ह॑ है॒ता मे॒ताꣳ ह॑ । \newline
38. ह॒ वै वै ह॑ ह॒ वै । \newline
39. वा उ॑द॒ङ्क उ॑द॒ङ्को वै वा उ॑द॒ङ्कः । \newline
40. उ॒द॒ङ्कः शौ᳚ल्बाय॒नः शौ᳚ल्बाय॒न उ॑द॒ङ्क उ॑द॒ङ्कः शौ᳚ल्बाय॒नः । \newline
41. शौ॒ल्बा॒य॒नः स॒त्रस्य॑ स॒त्रस्य॑ शौल्बाय॒नः शौ᳚ल्बाय॒नः स॒त्रस्य॑ । \newline
42. स॒त्रस्य र्‌द्धि॒ मृद्धिꣳ॑ स॒त्रस्य॑ स॒त्रस्य र्‌द्धि᳚म् । \newline
43. ऋद्धि॑ मुवाचो वा॒च र्‌द्धि॒ मृद्धि॑ मुवाच । \newline
44. उ॒वा॒च॒ यद् यदु॑वाचो वाच॒ यत् । \newline
45. यद् द॑शरा॒त्रो द॑शरा॒त्रो यद् यद् द॑शरा॒त्रः । \newline
46. द॒श॒रा॒त्रो यद् यद् द॑शरा॒त्रो द॑शरा॒त्रो यत् । \newline
47. द॒श॒रा॒त्र इति॑ दश - रा॒त्रः । \newline
48. यद् द॑शरा॒त्रो द॑शरा॒त्रो यद् यद् द॑शरा॒त्रः । \newline
49. द॒श॒रा॒त्रो भव॑ति॒ भव॑ति दशरा॒त्रो द॑शरा॒त्रो भव॑ति । \newline
50. द॒श॒रा॒त्र इति॑ दश - रा॒त्रः । \newline
51. भव॑ति स॒त्रस्य॑ स॒त्रस्य॒ भव॑ति॒ भव॑ति स॒त्रस्य॑ । \newline
52. स॒त्रस्य र्‌द्ध्या॒ ऋद्ध्यै॑ स॒त्रस्य॑ स॒त्रस्य र्‌द्ध्यै᳚ । \newline
53. ऋद्ध्या॒ अथो॒ अथो॒ ऋद्ध्या॒ ऋद्ध्या॒ अथो᳚ । \newline
54. अथो॒ यद् यदथो॒ अथो॒ यत् । \newline
55. अथो॒ इत्यथो᳚ । \newline
56. यदे॒ वैव यद् यदे॒व । \newline
57. ए॒व पूर्वे॑षु॒ पूर्वे᳚ ष्वे॒वैव पूर्वे॑षु । \newline
58. पूर्वे॒ष्वह॒ स्स्वह॑स्सु॒ पूर्वे॑षु॒ पूर्वे॒ ष्वह॑स्सु । \newline
59. अह॑स्सु॒ विलो॑म॒ विलो॒मा ह॒ स्स्वह॑स्सु॒ विलो॑म । \newline
60. अह॒स्स्वित्यहः॑ - सु॒ । \newline
61. विलो॑म क्रि॒यते᳚ क्रि॒यते॒ विलो॑म॒ विलो॑म क्रि॒यते᳚ । \newline
62. विलो॒मेति॒ वि - लो॒म॒ । \newline
63. क्रि॒यते॒ तस्य॒ तस्य॑ क्रि॒यते᳚ क्रि॒यते॒ तस्य॑ । \newline
64. तस्यै॒वैव तस्य॒ तस्यै॒व । \newline
65. ए॒वैषैषै वैवैषा । \newline
66. ए॒षा शान्तिः॒ शान्ति॑ रे॒षैषा शान्तिः॑ । \newline
67. शान्ति॒रिति॒ शान्तिः॑ । \newline

\textbf{Ghana Paata } \newline

1. गृ॒ह्य॒त॒ इ॒य मि॒यम् गृ॑ह्यते गृह्यत इ॒यं ॅवै वा इ॒यम् गृ॑ह्यते गृह्यत इ॒यं ॅवै । \newline
2. इ॒यं ॅवै वा इ॒य मि॒यं ॅवा अदि॑ति॒ रदि॑ति॒र् वा इ॒य मि॒यं ॅवा अदि॑तिः । \newline
3. वा अदि॑ति॒ रदि॑ति॒र् वै वा अदि॑ति र॒स्या म॒स्या मदि॑ति॒र् वै वा अदि॑ति र॒स्याम् । \newline
4. अदि॑ति र॒स्या म॒स्या मदि॑ति॒ रदि॑ति र॒स्या मे॒वै वास्या मदि॑ति॒ रदि॑ति र॒स्या मे॒व । \newline
5. अ॒स्या मे॒वै वास्या म॒स्या मे॒व प्रति॒ प्रत्ये॒ वास्या म॒स्या मे॒व प्रति॑ । \newline
6. ए॒व प्रति॒ प्रत्ये॒ वैव प्रति॑ तिष्ठन्ति तिष्ठन्ति॒ प्रत्ये॒ वैव प्रति॑ तिष्ठन्ति । \newline
7. प्रति॑ तिष्ठन्ति तिष्ठन्ति॒ प्रति॒ प्रति॑ तिष्ठ न्त्य॒न्यो᳚न्यो॒ ऽन्यो᳚न्य स्तिष्ठन्ति॒ प्रति॒ प्रति॑ तिष्ठ न्त्य॒न्यो᳚न्यः । \newline
8. ति॒ष्ठ॒ न्त्य॒न्यो᳚न्यो॒ ऽन्यो᳚न्य स्तिष्ठन्ति तिष्ठ न्त्य॒न्यो᳚न्यो गृह्येते गृह्येते अ॒न्यो᳚न्य स्तिष्ठन्ति तिष्ठ न्त्य॒न्यो᳚न्यो गृह्येते । \newline
9. अ॒न्यो᳚न्यो गृह्येते गृह्येते अ॒न्यो᳚न्यो॒ ऽन्यो᳚न्यो गृह्येते मिथुन॒त्वाय॑ मिथुन॒त्वाय॑ गृह्येते अ॒न्यो᳚न्यो॒ ऽन्यो᳚न्यो गृह्येते मिथुन॒त्वाय॑ । \newline
10. अ॒न्यो᳚न्य॒ इत्य॒न्यः - अ॒न्यः॒ । \newline
11. गृ॒ह्ये॒ते॒ मि॒थु॒न॒त्वाय॑ मिथुन॒त्वाय॑ गृह्येते गृह्येते मिथुन॒त्वाय॒ प्रजा᳚त्यै॒ प्रजा᳚त्यै मिथुन॒त्वाय॑ गृह्येते गृह्येते मिथुन॒त्वाय॒ प्रजा᳚त्यै । \newline
12. गृ॒ह्ये॒ते॒ इति॑ गृह्येते । \newline
13. मि॒थु॒न॒त्वाय॒ प्रजा᳚त्यै॒ प्रजा᳚त्यै मिथुन॒त्वाय॑ मिथुन॒त्वाय॒ प्रजा᳚त्या अवान्त॒र म॑वान्त॒रम् प्रजा᳚त्यै मिथुन॒त्वाय॑ मिथुन॒त्वाय॒ प्रजा᳚त्या अवान्त॒रम् । \newline
14. मि॒थु॒न॒त्वायेति॑ मिथुन - त्वाय॑ । \newline
15. प्रजा᳚त्या अवान्त॒र म॑वान्त॒रम् प्रजा᳚त्यै॒ प्रजा᳚त्या अवान्त॒रं ॅवै वा अ॑वान्त॒रम् प्रजा᳚त्यै॒ प्रजा᳚त्या अवान्त॒रं ॅवै । \newline
16. प्रजा᳚त्या॒ इति॒ प्र - जा॒त्यै॒ । \newline
17. अ॒वा॒न्त॒रं ॅवै वा अ॑वान्त॒र म॑वान्त॒रं ॅवै द॑शरा॒त्रेण॑ दशरा॒त्रेण॒ वा अ॑वान्त॒र म॑वान्त॒रं ॅवै द॑शरा॒त्रेण॑ । \newline
18. अ॒वा॒न्त॒रमित्य॑व - अ॒न्त॒रम् । \newline
19. वै द॑शरा॒त्रेण॑ दशरा॒त्रेण॒ वै वै द॑शरा॒त्रेण॑ प्र॒जाप॑तिः प्र॒जाप॑तिर् दशरा॒त्रेण॒ वै वै द॑शरा॒त्रेण॑ प्र॒जाप॑तिः । \newline
20. द॒श॒रा॒त्रेण॑ प्र॒जाप॑तिः प्र॒जाप॑तिर् दशरा॒त्रेण॑ दशरा॒त्रेण॑ प्र॒जाप॑तिः प्र॒जाः प्र॒जाः प्र॒जाप॑तिर् दशरा॒त्रेण॑ दशरा॒त्रेण॑ प्र॒जाप॑तिः प्र॒जाः । \newline
21. द॒श॒रा॒त्रेणेति॑ दश - रा॒त्रेण॑ । \newline
22. प्र॒जाप॑तिः प्र॒जाः प्र॒जाः प्र॒जाप॑तिः प्र॒जाप॑तिः प्र॒जा अ॑सृजता सृजत प्र॒जाः प्र॒जाप॑तिः प्र॒जाप॑तिः प्र॒जा अ॑सृजत । \newline
23. प्र॒जाप॑ति॒रिति॑ प्र॒जा - प॒तिः॒ । \newline
24. प्र॒जा अ॑सृजता सृजत प्र॒जाः प्र॒जा अ॑सृजत॒ यद् यद॑सृजत प्र॒जाः प्र॒जा अ॑सृजत॒ यत् । \newline
25. प्र॒जा इति॑ प्र - जाः । \newline
26. अ॒सृ॒ज॒त॒ यद् यद॑सृजता सृजत॒ यद् द॑शरा॒त्रो द॑शरा॒त्रो यद॑सृजता सृजत॒ यद् द॑शरा॒त्रः । \newline
27. यद् द॑शरा॒त्रो द॑शरा॒त्रो यद् यद् द॑शरा॒त्रो भव॑ति॒ भव॑ति दशरा॒त्रो यद् यद् द॑शरा॒त्रो भव॑ति । \newline
28. द॒श॒रा॒त्रो भव॑ति॒ भव॑ति दशरा॒त्रो द॑शरा॒त्रो भव॑ति प्र॒जाः प्र॒जा भव॑ति दशरा॒त्रो द॑शरा॒त्रो भव॑ति प्र॒जाः । \newline
29. द॒श॒रा॒त्र इति॑ दश - रा॒त्रः । \newline
30. भव॑ति प्र॒जाः प्र॒जा भव॑ति॒ भव॑ति प्र॒जा ए॒वैव प्र॒जा भव॑ति॒ भव॑ति प्र॒जा ए॒व । \newline
31. प्र॒जा ए॒वैव प्र॒जाः प्र॒जा ए॒व तत् तदे॒व प्र॒जाः प्र॒जा ए॒व तत् । \newline
32. प्र॒जा इति॑ प्र - जाः । \newline
33. ए॒व तत् तदे॒वैव तद् यज॑माना॒ यज॑माना॒ स्तदे॒वैव तद् यज॑मानाः । \newline
34. तद् यज॑माना॒ यज॑माना॒ स्तत् तद् यज॑मानाः सृजन्ते सृजन्ते॒ यज॑माना॒ स्तत् तद् यज॑मानाः सृजन्ते । \newline
35. यज॑मानाः सृजन्ते सृजन्ते॒ यज॑माना॒ यज॑मानाः सृजन्त ए॒ता मे॒ताꣳ सृ॑जन्ते॒ यज॑माना॒ यज॑मानाः सृजन्त ए॒ताम् । \newline
36. सृ॒ज॒न्त॒ ए॒ता मे॒ताꣳ सृ॑जन्ते सृजन्त ए॒ताꣳ ह॑ है॒ताꣳ सृ॑जन्ते सृजन्त ए॒ताꣳ ह॑ । \newline
37. ए॒ताꣳ ह॑ है॒ता मे॒ताꣳ ह॒ वै वै है॒ता मे॒ताꣳ ह॒ वै । \newline
38. ह॒ वै वै ह॑ ह॒ वा उ॑द॒ङ्क उ॑द॒ङ्को वै ह॑ ह॒ वा उ॑द॒ङ्कः । \newline
39. वा उ॑द॒ङ्क उ॑द॒ङ्को वै वा उ॑द॒ङ्कः शौ᳚ल्बाय॒नः शौ᳚ल्बाय॒न उ॑द॒ङ्को वै वा उ॑द॒ङ्कः शौ᳚ल्बाय॒नः । \newline
40. उ॒द॒ङ्कः शौ᳚ल्बाय॒नः शौ᳚ल्बाय॒न उ॑द॒ङ्क उ॑द॒ङ्कः शौ᳚ल्बाय॒नः स॒त्रस्य॑ स॒त्रस्य॑ शौल्बाय॒न उ॑द॒ङ्क उ॑द॒ङ्कः शौ᳚ल्बाय॒नः स॒त्रस्य॑ । \newline
41. शौ॒ल्बा॒य॒नः स॒त्रस्य॑ स॒त्रस्य॑ शौल्बाय॒नः शौ᳚ल्बाय॒नः स॒त्रस्य र्‌द्धि॒ मृद्धिꣳ॑ स॒त्रस्य॑ शौल्बाय॒नः शौ᳚ल्बाय॒नः स॒त्रस्य र्‌द्धि᳚म् । \newline
42. स॒त्रस्य र्‌द्धि॒ मृद्धिꣳ॑ स॒त्रस्य॑ स॒त्रस्य र्‌द्धि॑ मुवाचो वा॒च र्‌द्धिꣳ॑ स॒त्रस्य॑ स॒त्रस्य र्‌द्धि॑ मुवाच । \newline
43. ऋद्धि॑ मुवाचो वा॒च र्‌द्धि॒ मृद्धि॑ मुवाच॒ यद् यदु॑वा॒च र्‌द्धि॒ मृद्धि॑ मुवाच॒ यत् । \newline
44. उ॒वा॒च॒ यद् यदु॑वा चोवाच॒ यद् द॑शरा॒त्रो द॑शरा॒त्रो यदु॑वा चोवाच॒ यद् द॑शरा॒त्रः । \newline
45. यद् द॑शरा॒त्रो द॑शरा॒त्रो यद् यद् द॑शरा॒त्रो यद् यद् द॑शरा॒त्रो यद् यद् द॑शरा॒त्रो यत् । \newline
46. द॒श॒रा॒त्रो यद् यद् द॑शरा॒त्रो द॑शरा॒त्रो यद् द॑शरा॒त्रो द॑शरा॒त्रो यद् द॑शरा॒त्रो द॑शरा॒त्रो यद् द॑शरा॒त्रः । \newline
47. द॒श॒रा॒त्र इति॑ दश - रा॒त्रः । \newline
48. यद् द॑शरा॒त्रो द॑शरा॒त्रो यद् यद् द॑शरा॒त्रो भव॑ति॒ भव॑ति दशरा॒त्रो यद् यद् द॑शरा॒त्रो भव॑ति । \newline
49. द॒श॒रा॒त्रो भव॑ति॒ भव॑ति दशरा॒त्रो द॑शरा॒त्रो भव॑ति स॒त्रस्य॑ स॒त्रस्य॒ भव॑ति दशरा॒त्रो द॑शरा॒त्रो भव॑ति स॒त्रस्य॑ । \newline
50. द॒श॒रा॒त्र इति॑ दश - रा॒त्रः । \newline
51. भव॑ति स॒त्रस्य॑ स॒त्रस्य॒ भव॑ति॒ भव॑ति स॒त्रस्य र्‌द्ध्या॒ ऋद्ध्यै॑ स॒त्रस्य॒ भव॑ति॒ भव॑ति स॒त्रस्य र्‌द्ध्यै᳚ । \newline
52. स॒त्रस्य र्‌द्ध्या॒ ऋद्ध्यै॑ स॒त्रस्य॑ स॒त्रस्य र्‌द्ध्या॒ अथो॒ अथो॒ ऋद्ध्यै॑ स॒त्रस्य॑ स॒त्रस्य र्‌द्ध्या॒ अथो᳚ । \newline
53. ऋद्ध्या॒ अथो॒ अथो॒ ऋद्ध्या॒ ऋद्ध्या॒ अथो॒ यद् यदथो॒ ऋद्ध्या॒ ऋद्ध्या॒ अथो॒ यत् । \newline
54. अथो॒ यद् यदथो॒ अथो॒ यदे॒वैव यदथो॒ अथो॒ यदे॒व । \newline
55. अथो॒ इत्यथो᳚ । \newline
56. यदे॒ वैव यद् यदे॒व पूर्वे॑षु॒ पूर्वे᳚ष्वे॒व यद् यदे॒व पूर्वे॑षु । \newline
57. ए॒व पूर्वे॑षु॒ पूर्वे᳚ ष्वे॒वैव पूर्वे॒ ष्वह॒ स्स्वह॑स्सु॒ पूर्वे᳚ष्वे॒वैव पूर्वे॒ ष्वह॑स्सु । \newline
58. पूर्वे॒ष्वह॒ स्स्वह॑स्सु॒ पूर्वे॑षु॒ पूर्वे॒ ष्वह॑स्सु॒ विलो॑म॒ विलो॒मा ह॑स्सु॒ पूर्वे॑षु॒ पूर्वे॒ ष्वह॑स्सु॒ विलो॑म । \newline
59. अह॑स्सु॒ विलो॑म॒ विलो॒मा ह॒ स्स्वह॑स्सु॒ विलो॑म क्रि॒यते᳚ क्रि॒यते॒ विलो॒मा ह॒ स्स्वह॑स्सु॒ विलो॑म क्रि॒यते᳚ । \newline
60. अह॒स्स्वित्यहः॑ - सु॒ । \newline
61. विलो॑म क्रि॒यते᳚ क्रि॒यते॒ विलो॑म॒ विलो॑म क्रि॒यते॒ तस्य॒ तस्य॑ क्रि॒यते॒ विलो॑म॒ विलो॑म क्रि॒यते॒ तस्य॑ । \newline
62. विलो॒मेति॒ वि - लो॒म॒ । \newline
63. क्रि॒यते॒ तस्य॒ तस्य॑ क्रि॒यते᳚ क्रि॒यते॒ तस्यै॒वैव तस्य॑ क्रि॒यते᳚ क्रि॒यते॒ तस्यै॒व । \newline
64. तस्यै॒वैव तस्य॒ तस्यै॒ वैषै षैव तस्य॒ तस्यै॒ वैषा । \newline
65. ए॒वैषै षैवै वैषा शान्तिः॒ शान्ति॑ रे॒षैवै वैषा शान्तिः॑ । \newline
66. ए॒षा शान्तिः॒ शान्ति॑ रे॒षैषा शान्तिः॑ । \newline
67. शान्ति॒रिति॒ शान्तिः॑ । \newline
\pagebreak
\markright{ TS 7.5.5.1  \hfill https://www.vedavms.in \hfill}

\section{ TS 7.5.5.1 }

\textbf{TS 7.5.5.1 } \newline
\textbf{Samhita Paata} \newline

यदि॒ सोमौ॒ सꣳसु॑तौ॒ स्यातां᳚ मह॒ति रात्रि॑यै प्रातरनुवा॒क-मु॒पाकु॑र्या॒त् पूर्वो॒ वाचं॒ पूर्वो॑ दे॒वताः॒ पूर्वः॒ छन्दाꣳ॑सि वृङ्क्ते॒ वृष॑ण्वतीं प्रति॒पदं॑ कुर्यात् प्रातस्सव॒नादे॒वैषा॒मिन्द्रं॑ ॅवृ॒ङ्क्ते ऽथो॒ खल्वा॑हुःसवनमु॒खे-स॑वनमुखे का॒र्येति॑ सवनमु॒खाथ् स॑वनमुखा-दे॒वैषा॒मिन्द्रं॑ ॅवृङ्क्ते संॅवे॒शायो॑पवे॒शाय॑ गायत्रि॒यास्त्रि॒ष्टुभो॒ जग॑त्या अनु॒ष्टुभः॑ प॒ङ्क्त्या अ॒भिभू᳚त्यै॒ स्वाहा॒ छन्दाꣳ॑सि॒ वै सं॑ॅवे॒श उ॑पवे॒शः छन्दो॑भिरे॒वैषां॒- [  ] \newline

\textbf{Pada Paata} \newline

यदि॑ । सोमौ᳚ । सꣳसु॑ता॒विति॒ सं - सु॒तौ॒ । स्याता᳚म् । म॒ह॒ति । रात्रि॑यै । प्रा॒त॒र॒नु॒वा॒कमिति॑ प्रातः - अ॒नु॒वा॒कम् । उ॒पाकु॑र्या॒दित्यु॑प - आकु॑र्यात् । पूर्वः॑ । वाच᳚म् । पूर्वः॑ । दे॒वताः᳚ । पूर्वः॑ । छन्दाꣳ॑सि । वृ॒ङ्क्ते॒ । वृष॑ण्वती॒मिति॒ वृषण्॑ - व॒ती॒म् । प्र॒ति॒पद॒मिति॑ प्रति - पद᳚म् । कु॒र्या॒त् । प्रा॒त॒स्स॒व॒नादिति॑ प्रातः-स॒व॒नात् । ए॒व । ए॒षा॒म् । इन्द्र᳚म् । वृ॒ङ्क्ते॒ । अथो॒ इति॑ । खलु॑ । आ॒हुः॒ । स॒व॒न॒मु॒खेस॑वनमुख॒ इति॑ सवनमु॒खे - स॒व॒न॒मु॒खे॒ । का॒र्या᳚ । इति॑ । स॒व॒न॒मु॒खाथ्स॑वनमुखा॒दिति॑ सवनमु॒खात् - स॒व॒न॒मु॒खा॒त् । ए॒व । ए॒षा॒म् । इन्द्र᳚म् । वृ॒ङ्क्ते॒ । सं॒ॅवे॒शायेति॑ सं - वे॒शाय॑ । उ॒प॒वे॒शायेत्यु॑प - वे॒शाय॑ । गा॒य॒त्रि॒याः । त्रि॒ष्टुभः॑ । जग॑त्याः । अ॒नु॒ष्टुभ॒ इत्य॑नु - स्तुभः॑ । प॒ङ्क्त्याः । अ॒भिभू᳚त्या॒ इत्य॒भि - भू॒त्यै॒ । स्वाहा᳚ । छन्दाꣳ॑सि । वै । सं॒ॅवे॒श इति॑ सं - वे॒शः । उ॒प॒वे॒श इत्यु॑प - वे॒शः । छन्दो॑भि॒रिति॒ छन्दः॑ - भिः॒ । ए॒व । ए॒षा॒म् ।  \newline


\textbf{Krama Paata} \newline

यदि॒ सोमौ᳚ । सोमौ॒ सꣳसु॑तौ । सꣳसु॑तौ॒ स्याता᳚म् । सꣳसु॑ता॒विति॒ सम् - सु॒तौ॒ । स्याता᳚म् मह॒ति । म॒ह॒ति रात्रि॑यै । रात्रि॑यै प्रातरनुवा॒कम् । प्रा॒त॒र॒नु॒वा॒कमु॒पाकु॑र्यात् । प्रा॒त॒र॒नु॒वा॒कमिति॑ प्रातः - अ॒नु॒वा॒कम् । उ॒पाकु॑र्या॒त् पूर्वः॑ । उ॒पाकु॑र्या॒दित्यु॑प - आकु॑र्यात् । पूर्वो॒ वाच᳚म् । वाच॒म् पूर्वः॑ । पूर्वो॑ दे॒वताः᳚ । दे॒वताः॒ पूर्वः॑ । पूर्व॒श्छन्दाꣳ॑सि । छन्दाꣳ॑सि वृङ्‍क्ते । वृ॒ङ्‍क्ते॒ वृष॑ण्वतीम् । वृष॑ण्वतीम् प्रति॒पद᳚म् । वृष॑ण्वती॒मिति॒ वृषण्ण्॑ - व॒ती॒म् । प्र॒ति॒पद॑म् कुर्यात् । प्र॒ति॒पद॒मिति॑ प्रति - पद᳚म् । कु॒र्या॒त् प्रा॒त॒स्स॒व॒नात् । प्रा॒त॒स्स॒व॒नादे॒व । प्रा॒त॒स्स॒व॒नादिति॑ प्रातः - स॒व॒नात् । ए॒वैषा᳚म् । ए॒षा॒मिन्द्र᳚म् । इन्द्र॑म् ॅवृङ्‍क्ते । वृ॒ङ्‍क्तेऽथो᳚ । अथो॒ खलु॑ । अथो॒ इत्यथो᳚ । खल्वा॑हुः । आ॒हुः॒ स॒व॒न॒मु॒खेस॑वनमुखे । स॒व॒न॒मु॒खेस॑वनमुखे का॒र्या᳚ । स॒व॒न॒मु॒खेस॑वनमुख॒ इति॑ सवनमु॒खे - स॒व॒न॒मु॒खे॒ । का॒र्येति॑ । इति॑ सवनमु॒खाथ्‌स॑वनमुखात् । स॒व॒न॒मु॒खाथ्‌स॑वनमुखादे॒व । स॒व॒न॒मु॒खाथ्‌स॑वनमुखा॒दिति॑ सवनमु॒खात् - स॒व॒न॒मु॒खा॒त्॒ । ए॒वैषा᳚म् । ए॒षा॒मिन्द्र᳚म् । इन्द्र॑म् ॅवृङ्‍क्ते । वृ॒ङ्‍क्ते॒ स॒म्ॅवे॒शाय॑ । स॒म्ॅवे॒शायो॑पवे॒शाय॑ । स॒म्ॅवे॒शायेति॑ सम् - वे॒शाय॑ । उ॒प॒वे॒शाय॑ गायत्रि॒याः । उ॒प॒वे॒शायेत्यु॑प - वे॒शाय॑ । गा॒य॒त्रि॒यास्त्रि॒ष्टुभः॑ । त्रि॒ष्टुभो॒ जग॑त्याः । जग॑त्या अनु॒ष्टुभः॑ । अ॒नु॒ष्टुभः॑ प॒ङ्‍क्त्याः । अ॒नु॒ष्टुभ॒ इत्य॑नु - स्तुभः॑ । प॒ङ्‍क्त्या अ॒भिभू᳚त्यै । अ॒भिभू᳚त्यै॒ स्वाहा᳚ । अ॒भिभू᳚त्या॒ इत्य॒भि - भू॒त्यै॒ । स्वाहा॒ छन्दाꣳ॑सि । छन्दाꣳ॑सि॒ वै । वै स॑म्ॅवे॒शः । स॒म्ॅवे॒श उ॑पवे॒शः । स॒म्ॅवे॒श इति॑ सम् - वे॒शः । उ॒प॒वे॒शश्छन्दो॑भिः । उ॒प॒वे॒श इत्यु॑प - वे॒शः । छन्दो॑भिरे॒व । छन्दो॑भि॒रिति॒ छन्दः॑ - भिः॒ । ए॒वैषा᳚म् । ए॒षा॒म् छन्दाꣳ॑सि \newline

\textbf{Jatai Paata} \newline

1. यदि॒ सोमौ॒ सोमौ॒ यदि॒ यदि॒ सोमौ᳚ । \newline
2. सोमौ॒ सꣳसु॑तौ॒ सꣳसु॑तौ॒ सोमौ॒ सोमौ॒ सꣳसु॑तौ । \newline
3. सꣳसु॑तौ॒ स्याताꣳ॒॒ स्याताꣳ॒॒ सꣳसु॑तौ॒ सꣳसु॑तौ॒ स्याता᳚म् । \newline
4. सꣳसु॑ता॒विति॒ सं - सु॒तौ॒ । \newline
5. स्याता᳚म् मह॒ति म॑ह॒ति स्याताꣳ॒॒ स्याता᳚म् मह॒ति । \newline
6. म॒ह॒ति रात्रि॑यै॒ रात्रि॑यै मह॒ति म॑ह॒ति रात्रि॑यै । \newline
7. रात्रि॑यै प्रातरनुवा॒कम् प्रा॑तरनुवा॒कꣳ रात्रि॑यै॒ रात्रि॑यै प्रातरनुवा॒कम् । \newline
8. प्रा॒त॒र॒नु॒वा॒क मु॒पाकु॑र्या दु॒पाकु॑र्यात् प्रातरनुवा॒कम् प्रा॑तरनुवा॒क मु॒पाकु॑र्यात् । \newline
9. प्रा॒त॒र॒नु॒वा॒कमिति॑ प्रातः - अ॒नु॒वा॒कम् । \newline
10. उ॒पाकु॑र्या॒त् पूर्वः॒ पूर्व॑ उ॒पाकु॑र्या दु॒पाकु॑र्या॒त् पूर्वः॑ । \newline
11. उ॒पाकु॑र्या॒दित्यु॑प - आकु॑र्यात् । \newline
12. पूर्वो॒ वाचं॒ ॅवाच॒म् पूर्वः॒ पूर्वो॒ वाच᳚म् । \newline
13. वाच॒म् पूर्वः॒ पूर्वो॒ वाचं॒ ॅवाच॒म् पूर्वः॑ । \newline
14. पूर्वो॑ दे॒वता॑ दे॒वताः॒ पूर्वः॒ पूर्वो॑ दे॒वताः᳚ । \newline
15. दे॒वताः॒ पूर्वः॒ पूर्वो॑ दे॒वता॑ दे॒वताः॒ पूर्वः॑ । \newline
16. पूर्व॒ श्छन्दाꣳ॑सि॒ छन्दाꣳ॑सि॒ पूर्वः॒ पूर्व॒ श्छन्दाꣳ॑सि । \newline
17. छन्दाꣳ॑सि वृङ्क्ते वृङ्क्ते॒ छन्दाꣳ॑सि॒ छन्दाꣳ॑सि वृङ्क्ते । \newline
18. वृ॒ङ्क्ते॒ वृष॑ण्वतीं॒ ॅवृष॑ण्वतीं ॅवृङ्क्ते वृङ्क्ते॒ वृष॑ण्वतीम् । \newline
19. वृष॑ण्वतीम् प्रति॒पद॑म् प्रति॒पदं॒ ॅवृष॑ण्वतीं॒ ॅवृष॑ण्वतीम् प्रति॒पद᳚म् । \newline
20. वृष॑ण्वती॒मिति॒ वृषण्ण्॑ - व॒ती॒म् । \newline
21. प्र॒ति॒पद॑म् कुर्यात् कुर्यात् प्रति॒पद॑म् प्रति॒पद॑म् कुर्यात् । \newline
22. प्र॒ति॒पद॒मिति॑ प्रति - पद᳚म् । \newline
23. कु॒र्या॒त् प्रा॒त॒स्स॒व॒नात् प्रा॑तस्सव॒नात् कु॑र्यात् कुर्यात् प्रातस्सव॒नात् । \newline
24. प्रा॒त॒स्स॒व॒ना दे॒वैव प्रा॑तस्सव॒नात् प्रा॑तस्सव॒ना दे॒व । \newline
25. प्रा॒त॒स्स॒व॒नादिति॑ प्रातः - स॒व॒नात् । \newline
26. ए॒वैषा॑ मेषा मे॒वै वैषा᳚म् । \newline
27. ए॒षा॒ मिन्द्र॒ मिन्द्र॑ मेषा मेषा॒ मिन्द्र᳚म् । \newline
28. इन्द्रं॑ ॅवृङ्क्ते वृङ्क्त॒ इन्द्र॒ मिन्द्रं॑ ॅवृङ्क्ते । \newline
29. वृ॒ङ्क्ते ऽथो॒ अथो॑ वृङ्क्ते वृ॒ङ्क्ते ऽथो᳚ । \newline
30. अथो॒ खलु॒ खल्वथो॒ अथो॒ खलु॑ । \newline
31. अथो॒ इत्यथो᳚ । \newline
32. खल्वा॑हु राहुः॒ खलु॒ खल्वा॑हुः । \newline
33. आ॒हुः॒ स॒व॒न॒मु॒खेस॑वनमुखे सवनमु॒खेस॑वनमुख आहु राहुः सवनमु॒खेस॑वनमुखे । \newline
34. स॒व॒न॒मु॒खेस॑वनमुखे का॒र्या॑ का॒र्या॑ सवनमु॒खेस॑वनमुखे सवनमु॒खेस॑वनमुखे का॒र्या᳚ । \newline
35. स॒व॒न॒मु॒खेस॑वनमुख॒ इति॑ सवनमु॒खे - स॒व॒न॒मु॒खे॒ । \newline
36. का॒र्येतीति॑ का॒र्या॑ का॒र्येति॑ । \newline
37. इति॑ सवनमु॒खाथ्स॑वनमुखाथ् सवनमु॒खाथ्स॑वनमुखा॒ दितीति॑ सवनमु॒खाथ्स॑वनमुखात् । \newline
38. स॒व॒न॒मु॒खाथ्स॑वनमुखा दे॒वैव स॑वनमु॒खाथ्स॑वनमुखाथ् सवनमु॒खाथ्स॑वनमुखा दे॒व । \newline
39. स॒व॒न॒मु॒खाथ्स॑वनमुखा॒दिति॑ सवनमु॒खात् - स॒व॒न॒मु॒खा॒त् । \newline
40. ए॒वैषा॑ मेषा मे॒वै वैषा᳚म् । \newline
41. ए॒षा॒ मिन्द्र॒ मिन्द्र॑ मेषा मेषा॒ मिन्द्र᳚म् । \newline
42. इन्द्रं॑ ॅवृङ्क्ते वृङ्क्त॒ इन्द्र॒ मिन्द्रं॑ ॅवृङ्क्ते । \newline
43. वृ॒ङ्क्ते॒ सं॒ॅवे॒शाय॑ संॅवे॒शाय॑ वृङ्क्ते वृङ्क्ते संॅवे॒शाय॑ । \newline
44. सं॒ॅवे॒शा यो॑पवे॒शा यो॑पवे॒शाय॑ संॅवे॒शाय॑ संॅवे॒शा यो॑पवे॒शाय॑ । \newline
45. सं॒ॅवे॒शायेति॑ सं - वे॒शाय॑ । \newline
46. उ॒प॒वे॒शाय॑ गायत्रि॒या गा॑यत्रि॒या उ॑पवे॒शा यो॑पवे॒शाय॑ गायत्रि॒याः । \newline
47. उ॒प॒वे॒शायेत्यु॑प - वे॒शाय॑ । \newline
48. गा॒य॒त्रि॒या स्त्रि॒ष्टुभ॑ स्त्रि॒ष्टुभो॑ गायत्रि॒या गा॑यत्रि॒या स्त्रि॒ष्टुभः॑ । \newline
49. त्रि॒ष्टुभो॒ जग॑त्या॒ जग॑त्या स्त्रि॒ष्टुभ॑ स्त्रि॒ष्टुभो॒ जग॑त्याः । \newline
50. जग॑त्या अनु॒ष्टुभो॑ ऽनु॒ष्टुभो॒ जग॑त्या॒ जग॑त्या अनु॒ष्टुभः॑ । \newline
51. अ॒नु॒ष्टुभः॑ प॒ङ्क्त्याः प॒ङ्क्त्या अ॑नु॒ष्टुभो॑ ऽनु॒ष्टुभः॑ प॒ङ्क्त्याः । \newline
52. अ॒नु॒ष्टुभ॒ इत्य॑नु - स्तुभः॑ । \newline
53. प॒ङ्क्त्या अ॒भिभू᳚त्या अ॒भिभू᳚त्यै प॒ङ्क्त्याः प॒ङ्क्त्या अ॒भिभू᳚त्यै । \newline
54. अ॒भिभू᳚त्यै॒ स्वाहा॒ स्वाहा॒ ऽभिभू᳚त्या अ॒भिभू᳚त्यै॒ स्वाहा᳚ । \newline
55. अ॒भिभू᳚त्या॒ इत्य॒भि - भू॒त्यै॒ । \newline
56. स्वाहा॒ छन्दाꣳ॑सि॒ छन्दाꣳ॑सि॒ स्वाहा॒ स्वाहा॒ छन्दाꣳ॑सि । \newline
57. छन्दाꣳ॑सि॒ वै वै छन्दाꣳ॑सि॒ छन्दाꣳ॑सि॒ वै । \newline
58. वै सं॑ॅवे॒शः सं॑ॅवे॒शो वै वै सं॑ॅवे॒शः । \newline
59. सं॒ॅवे॒श उ॑पवे॒श उ॑पवे॒शः सं॑ॅवे॒शः सं॑ॅवे॒श उ॑पवे॒शः । \newline
60. सं॒ॅवे॒श इति॑ सं - वे॒शः । \newline
61. उ॒प॒वे॒श श्छन्दो॑भि॒ श्छन्दो॑भि रुपवे॒श उ॑पवे॒श श्छन्दो॑भिः । \newline
62. उ॒प॒वे॒श इत्यु॑प - वे॒शः । \newline
63. छन्दो॑भि रे॒वैव छन्दो॑भि॒ श्छन्दो॑भि रे॒व । \newline
64. छन्दो॑भि॒रिति॒ छन्दः॑ - भिः॒ । \newline
65. ए॒वैषा॑ मेषा मे॒वै वैषा᳚म् । \newline
66. ए॒षा॒म् छन्दाꣳ॑सि॒ छन्दाꣳ॑ स्येषा मेषा॒म् छन्दाꣳ॑सि । \newline

\textbf{Ghana Paata } \newline

1. यदि॒ सोमौ॒ सोमौ॒ यदि॒ यदि॒ सोमौ॒ सꣳसु॑तौ॒ सꣳसु॑तौ॒ सोमौ॒ यदि॒ यदि॒ सोमौ॒ सꣳसु॑तौ । \newline
2. सोमौ॒ सꣳसु॑तौ॒ सꣳसु॑तौ॒ सोमौ॒ सोमौ॒ सꣳसु॑तौ॒ स्याताꣳ॒॒ स्याताꣳ॒॒ सꣳसु॑तौ॒ सोमौ॒ सोमौ॒ सꣳसु॑तौ॒ स्याता᳚म् । \newline
3. सꣳसु॑तौ॒ स्याताꣳ॒॒ स्याताꣳ॒॒ सꣳसु॑तौ॒ सꣳसु॑तौ॒ स्याता᳚म् मह॒ति म॑ह॒ति स्याताꣳ॒॒ सꣳसु॑तौ॒ सꣳसु॑तौ॒ स्याता᳚म् मह॒ति । \newline
4. सꣳसु॑ता॒विति॒ सं - सु॒तौ॒ । \newline
5. स्याता᳚म् मह॒ति म॑ह॒ति स्याताꣳ॒॒ स्याता᳚म् मह॒ति रात्रि॑यै॒ रात्रि॑यै मह॒ति स्याताꣳ॒॒ स्याता᳚म् मह॒ति रात्रि॑यै । \newline
6. म॒ह॒ति रात्रि॑यै॒ रात्रि॑यै मह॒ति म॑ह॒ति रात्रि॑यै प्रातरनुवा॒कम् प्रा॑तरनुवा॒कꣳ रात्रि॑यै मह॒ति म॑ह॒ति रात्रि॑यै प्रातरनुवा॒कम् । \newline
7. रात्रि॑यै प्रातरनुवा॒कम् प्रा॑तरनुवा॒कꣳ रात्रि॑यै॒ रात्रि॑यै प्रातरनुवा॒क मु॒पाकु॑र्या दु॒पाकु॑र्यात् प्रातरनुवा॒कꣳ रात्रि॑यै॒ रात्रि॑यै प्रातरनुवा॒क मु॒पाकु॑र्यात् । \newline
8. प्रा॒त॒र॒नु॒वा॒क मु॒पाकु॑र्या दु॒पाकु॑र्यात् प्रातरनुवा॒कम् प्रा॑तरनुवा॒क मु॒पाकु॑र्या॒त् पूर्वः॒ पूर्व॑ उ॒पाकु॑र्यात् प्रातरनुवा॒कम् प्रा॑तरनुवा॒क मु॒पाकु॑र्या॒त् पूर्वः॑ । \newline
9. प्रा॒त॒र॒नु॒वा॒कमिति॑ प्रातः - अ॒नु॒वा॒कम् । \newline
10. उ॒पाकु॑र्या॒त् पूर्वः॒ पूर्व॑ उ॒पाकु॑र्या दु॒पाकु॑र्या॒त् पूर्वो॒ वाचं॒ ॅवाच॒म् पूर्व॑ उ॒पाकु॑र्या दु॒पाकु॑र्या॒त् पूर्वो॒ वाच᳚म् । \newline
11. उ॒पाकु॑र्या॒दित्यु॑प - आकु॑र्यात् । \newline
12. पूर्वो॒ वाचं॒ ॅवाच॒म् पूर्वः॒ पूर्वो॒ वाच॒म् पूर्वः॒ पूर्वो॒ वाच॒म् पूर्वः॒ पूर्वो॒ वाच॒म् पूर्वः॑ । \newline
13. वाच॒म् पूर्वः॒ पूर्वो॒ वाचं॒ ॅवाच॒म् पूर्वो॑ दे॒वता॑ दे॒वताः॒ पूर्वो॒ वाचं॒ ॅवाच॒म् पूर्वो॑ दे॒वताः᳚ । \newline
14. पूर्वो॑ दे॒वता॑ दे॒वताः॒ पूर्वः॒ पूर्वो॑ दे॒वताः॒ पूर्वः॒ पूर्वो॑ दे॒वताः॒ पूर्वः॒ पूर्वो॑ दे॒वताः॒ पूर्वः॑ । \newline
15. दे॒वताः॒ पूर्वः॒ पूर्वो॑ दे॒वता॑ दे॒वताः॒ पूर्व॒ श्छन्दाꣳ॑सि॒ छन्दाꣳ॑सि॒ पूर्वो॑ दे॒वता॑ दे॒वताः॒ पूर्व॒ श्छन्दाꣳ॑सि । \newline
16. पूर्व॒ श्छन्दाꣳ॑सि॒ छन्दाꣳ॑सि॒ पूर्वः॒ पूर्व॒ श्छन्दाꣳ॑सि वृङ्क्ते वृङ्क्ते॒ छन्दाꣳ॑सि॒ पूर्वः॒ पूर्व॒ श्छन्दाꣳ॑सि वृङ्क्ते । \newline
17. छन्दाꣳ॑सि वृङ्क्ते वृङ्क्ते॒ छन्दाꣳ॑सि॒ छन्दाꣳ॑सि वृङ्क्ते॒ वृष॑ण्वतीं॒ ॅवृष॑ण्वतीं ॅवृङ्क्ते॒ छन्दाꣳ॑सि॒ छन्दाꣳ॑सि वृङ्क्ते॒ वृष॑ण्वतीम् । \newline
18. वृ॒ङ्क्ते॒ वृष॑ण्वतीं॒ ॅवृष॑ण्वतीं ॅवृङ्क्ते वृङ्क्ते॒ वृष॑ण्वतीम् प्रति॒पद॑म् प्रति॒पदं॒ ॅवृष॑ण्वतीं ॅवृङ्क्ते वृङ्क्ते॒ वृष॑ण्वतीम् प्रति॒पद᳚म् । \newline
19. वृष॑ण्वतीम् प्रति॒पद॑म् प्रति॒पदं॒ ॅवृष॑ण्वतीं॒ ॅवृष॑ण्वतीम् प्रति॒पद॑म् कुर्यात् कुर्यात् प्रति॒पदं॒ ॅवृष॑ण्वतीं॒ ॅवृष॑ण्वतीम् प्रति॒पद॑म् कुर्यात् । \newline
20. वृष॑ण्वती॒मिति॒ वृषण्ण्॑ - व॒ती॒म् । \newline
21. प्र॒ति॒पद॑म् कुर्यात् कुर्यात् प्रति॒पद॑म् प्रति॒पद॑म् कुर्यात् प्रातस्सव॒नात् प्रा॑तस्सव॒नात् कु॑र्यात् प्रति॒पद॑म् प्रति॒पद॑म् कुर्यात् प्रातस्सव॒नात् । \newline
22. प्र॒ति॒पद॒मिति॑ प्रति - पद᳚म् । \newline
23. कु॒र्या॒त् प्रा॒त॒स्स॒व॒नात् प्रा॑तस्सव॒नात् कु॑र्यात् कुर्यात् प्रातस्सव॒ना दे॒वैव प्रा॑तस्सव॒नात् कु॑र्यात् कुर्यात् प्रातस्सव॒ना दे॒व । \newline
24. प्रा॒त॒स्स॒व॒ना दे॒वैव प्रा॑तस्सव॒नात् प्रा॑तस्सव॒ना दे॒वैषा॑ मेषा मे॒व प्रा॑तस्सव॒नात् प्रा॑तस्सव॒ना दे॒वैषा᳚म् । \newline
25. प्रा॒त॒स्स॒व॒नादिति॑ प्रातः - स॒व॒नात् । \newline
26. ए॒वैषा॑ मेषा मे॒वै वैषा॒ मिन्द्र॒ मिन्द्र॑ मेषा मे॒वै वैषा॒ मिन्द्र᳚म् । \newline
27. ए॒षा॒ मिन्द्र॒ मिन्द्र॑ मेषा मेषा॒ मिन्द्रं॑ ॅवृङ्क्ते वृङ्क्त॒ इन्द्र॑ मेषा मेषा॒ मिन्द्रं॑ ॅवृङ्क्ते । \newline
28. इन्द्रं॑ ॅवृङ्क्ते वृङ्क्त॒ इन्द्र॒ मिन्द्रं॑ ॅवृ॒ङ्क्ते ऽथो॒ अथो॑ वृङ्क्त॒ इन्द्र॒ मिन्द्रं॑ ॅवृ॒ङ्क्ते ऽथो᳚ । \newline
29. वृ॒ङ्क्ते ऽथो॒ अथो॑ वृङ्क्ते वृ॒ङ्क्ते ऽथो॒ खलु॒ खल्वथो॑ वृङ्क्ते वृ॒ङ्क्ते ऽथो॒ खलु॑ । \newline
30. अथो॒ खलु॒ खल्वथो॒ अथो॒ खल्वा॑हु राहुः॒ खल्वथो॒ अथो॒ खल्वा॑हुः । \newline
31. अथो॒ इत्यथो᳚ । \newline
32. खल्वा॑हु राहुः॒ खलु॒ खल्वा॑हुः सवनमु॒खेस॑वनमुखे सवनमु॒खेस॑वनमुख आहुः॒ खलु॒ खल्वा॑हुः सवनमु॒खेस॑वनमुखे । \newline
33. आ॒हुः॒ स॒व॒न॒मु॒खेस॑वनमुखे सवनमु॒खेस॑वनमुख आहु राहुः सवनमु॒खेस॑वनमुखे का॒र्या॑ का॒र्या॑ सवनमु॒खेस॑वनमुख आहु राहुः सवनमु॒खेस॑वनमुखे का॒र्या᳚ । \newline
34. स॒व॒न॒मु॒खेस॑वनमुखे का॒र्या॑ का॒र्या॑ सवनमु॒खेस॑वनमुखे सवनमु॒खेस॑वनमुखे का॒र्येतीति॑ का॒र्या॑ सवनमु॒खेस॑वनमुखे सवनमु॒खेस॑वनमुखे का॒र्येति॑ । \newline
35. स॒व॒न॒मु॒खेस॑वनमुख॒ इति॑ सवनमु॒खे - स॒व॒न॒मु॒खे॒ । \newline
36. का॒र्येतीति॑ का॒र्या॑ का॒र्येति॑ सवनमु॒खाथ्स॑वनमुखाथ् सवनमु॒खाथ्स॑वनमुखा॒ दिति॑ का॒र्या॑ का॒र्येति॑ सवनमु॒खाथ्स॑वनमुखात् । \newline
37. इति॑ सवनमु॒खाथ्स॑वनमुखाथ् सवनमु॒खाथ्स॑वनमुखा॒ दितीति॑ सवनमु॒खाथ्स॑वनमुखा दे॒वैव स॑वनमु॒खाथ्स॑वनमुखा॒ दितीति॑ सवनमु॒खाथ्स॑वनमुखा दे॒व । \newline
38. स॒व॒न॒मु॒खाथ्स॑वनमुखा दे॒वैव स॑वनमु॒खाथ्स॑वनमुखाथ् सवनमु॒खाथ्स॑वनमुखा दे॒वैषा॑ मेषा मे॒व स॑वनमु॒खाथ्स॑वनमुखाथ् सवनमु॒खाथ्स॑वनमुखा दे॒वैषा᳚म् । \newline
39. स॒व॒न॒मु॒खाथ्स॑वनमुखा॒दिति॑ सवनमु॒खात् - स॒व॒न॒मु॒खा॒त् । \newline
40. ए॒वैषा॑ मेषा मे॒वै वैषा॒ मिन्द्र॒ मिन्द्र॑ मेषा मे॒वै वैषा॒ मिन्द्र᳚म् । \newline
41. ए॒षा॒ मिन्द्र॒ मिन्द्र॑ मेषा मेषा॒ मिन्द्रं॑ ॅवृङ्क्ते वृङ्क्त॒ इन्द्र॑ मेषा मेषा॒ मिन्द्रं॑ ॅवृङ्क्ते । \newline
42. इन्द्रं॑ ॅवृङ्क्ते वृङ्क्त॒ इन्द्र॒ मिन्द्रं॑ ॅवृङ्क्ते संॅवे॒शाय॑ संॅवे॒शाय॑ वृङ्क्त॒ इन्द्र॒ मिन्द्रं॑ ॅवृङ्क्ते संॅवे॒शाय॑ । \newline
43. वृ॒ङ्क्ते॒ सं॒ॅवे॒शाय॑ संॅवे॒शाय॑ वृङ्क्ते वृङ्क्ते संॅवे॒शा यो॑पवे॒शा यो॑पवे॒शाय॑ संॅवे॒शाय॑ वृङ्क्ते वृङ्क्ते संॅवे॒शा यो॑पवे॒शाय॑ । \newline
44. सं॒ॅवे॒शा यो॑पवे॒शा यो॑पवे॒शाय॑ संॅवे॒शाय॑ संॅवे॒शा यो॑पवे॒शाय॑ गायत्रि॒या गा॑यत्रि॒या उ॑पवे॒शाय॑ संॅवे॒शाय॑ संॅवे॒शा यो॑पवे॒शाय॑ गायत्रि॒याः । \newline
45. सं॒ॅवे॒शायेति॑ सं - वे॒शाय॑ । \newline
46. उ॒प॒वे॒शाय॑ गायत्रि॒या गा॑यत्रि॒या उ॑पवे॒शा यो॑पवे॒शाय॑ गायत्रि॒या स्त्रि॒ष्टुभ॑ स्त्रि॒ष्टुभो॑ गायत्रि॒या उ॑पवे॒शा यो॑पवे॒शाय॑ गायत्रि॒या स्त्रि॒ष्टुभः॑ । \newline
47. उ॒प॒वे॒शायेत्यु॑प - वे॒शाय॑ । \newline
48. गा॒य॒त्रि॒या स्त्रि॒ष्टुभ॑ स्त्रि॒ष्टुभो॑ गायत्रि॒या गा॑यत्रि॒या स्त्रि॒ष्टुभो॒ जग॑त्या॒ जग॑त्या स्त्रि॒ष्टुभो॑ गायत्रि॒या गा॑यत्रि॒या स्त्रि॒ष्टुभो॒ जग॑त्याः । \newline
49. त्रि॒ष्टुभो॒ जग॑त्या॒ जग॑त्या स्त्रि॒ष्टुभ॑ स्त्रि॒ष्टुभो॒ जग॑त्या अनु॒ष्टुभो॑ ऽनु॒ष्टुभो॒ जग॑त्या स्त्रि॒ष्टुभ॑ स्त्रि॒ष्टुभो॒ जग॑त्या अनु॒ष्टुभः॑ । \newline
50. जग॑त्या अनु॒ष्टुभो॑ ऽनु॒ष्टुभो॒ जग॑त्या॒ जग॑त्या अनु॒ष्टुभः॑ प॒ङ्क्त्याः प॒ङ्क्त्या अ॑नु॒ष्टुभो॒ जग॑त्या॒ जग॑त्या अनु॒ष्टुभः॑ प॒ङ्क्त्याः । \newline
51. अ॒नु॒ष्टुभः॑ प॒ङ्क्त्याः प॒ङ्क्त्या अ॑नु॒ष्टुभो॑ ऽनु॒ष्टुभः॑ प॒ङ्क्त्या अ॒भिभू᳚त्या अ॒भिभू᳚त्यै प॒ङ्क्त्या अ॑नु॒ष्टुभो॑ ऽनु॒ष्टुभः॑ प॒ङ्क्त्या अ॒भिभू᳚त्यै । \newline
52. अ॒नु॒ष्टुभ॒ इत्य॑नु - स्तुभः॑ । \newline
53. प॒ङ्क्त्या अ॒भिभू᳚त्या अ॒भिभू᳚त्यै प॒ङ्क्त्याः प॒ङ्क्त्या अ॒भिभू᳚त्यै॒ स्वाहा॒ स्वाहा॒ ऽभिभू᳚त्यै प॒ङ्क्त्याः प॒ङ्क्त्या अ॒भिभू᳚त्यै॒ स्वाहा᳚ । \newline
54. अ॒भिभू᳚त्यै॒ स्वाहा॒ स्वाहा॒ ऽभिभू᳚त्या अ॒भिभू᳚त्यै॒ स्वाहा॒ छन्दाꣳ॑सि॒ छन्दाꣳ॑सि॒ स्वाहा॒ ऽभिभू᳚त्या अ॒भिभू᳚त्यै॒ स्वाहा॒ छन्दाꣳ॑सि । \newline
55. अ॒भिभू᳚त्या॒ इत्य॒भि - भू॒त्यै॒ । \newline
56. स्वाहा॒ छन्दाꣳ॑सि॒ छन्दाꣳ॑सि॒ स्वाहा॒ स्वाहा॒ छन्दाꣳ॑सि॒ वै वै छन्दाꣳ॑सि॒ स्वाहा॒ स्वाहा॒ छन्दाꣳ॑सि॒ वै । \newline
57. छन्दाꣳ॑सि॒ वै वै छन्दाꣳ॑सि॒ छन्दाꣳ॑सि॒ वै सं॑ॅवे॒शः सं॑ॅवे॒शो वै छन्दाꣳ॑सि॒ छन्दाꣳ॑सि॒ वै सं॑ॅवे॒शः । \newline
58. वै सं॑ॅवे॒शः सं॑ॅवे॒शो वै वै सं॑ॅवे॒श उ॑पवे॒श उ॑पवे॒शः सं॑ॅवे॒शो वै वै सं॑ॅवे॒श उ॑पवे॒शः । \newline
59. सं॒ॅवे॒श उ॑पवे॒श उ॑पवे॒शः सं॑ॅवे॒शः सं॑ॅवे॒श उ॑पवे॒श श्छन्दो॑भि॒ श्छन्दो॑भि रुपवे॒शः सं॑ॅवे॒शः सं॑ॅवे॒श उ॑पवे॒श श्छन्दो॑भिः । \newline
60. सं॒ॅवे॒श इति॑ सं - वे॒शः । \newline
61. उ॒प॒वे॒श श्छन्दो॑भि॒ श्छन्दो॑भि रुपवे॒श उ॑पवे॒श श्छन्दो॑भि रे॒वैव छन्दो॑भि रुपवे॒श उ॑पवे॒श श्छन्दो॑भि रे॒व । \newline
62. उ॒प॒वे॒श इत्यु॑प - वे॒शः । \newline
63. छन्दो॑भि रे॒वैव छन्दो॑भि॒ श्छन्दो॑भि रे॒वैषा॑ मेषा मे॒व छन्दो॑भि॒ श्छन्दो॑भि रे॒वैषा᳚म् । \newline
64. छन्दो॑भि॒रिति॒ छन्दः॑ - भिः॒ । \newline
65. ए॒वैषा॑ मेषा मे॒वै वैषा॒म् छन्दाꣳ॑सि॒ छन्दाꣳ॑स्येषा मे॒वै वैषा॒म् छन्दाꣳ॑सि । \newline
66. ए॒षा॒म् छन्दाꣳ॑सि॒ छन्दाꣳ॑स्येषा मेषा॒म् छन्दाꣳ॑सि वृङ्क्ते वृङ्क्ते॒ छन्दाꣳ॑स्येषा मेषा॒म् छन्दाꣳ॑सि वृङ्क्ते । \newline
\pagebreak
\markright{ TS 7.5.5.2  \hfill https://www.vedavms.in \hfill}

\section{ TS 7.5.5.2 }

\textbf{TS 7.5.5.2 } \newline
\textbf{Samhita Paata} \newline

छन्दाꣳ॑सि वृङ्क्ते सज॒नीयꣳ॒॒ शस्यं॑ ॅविह॒व्यꣳ॑ शस्य॑म॒गस्त्य॑स्य कयाशु॒भीयꣳ॒॒ शस्य॑मे॒ताव॒द्वा अ॑स्ति॒ याव॑दे॒तद्-याव॑दे॒वास्ति॒ तदे॑षां ॅवृङ्क्ते॒ यदि॑ प्रातस्सव॒ने क॒लशो॒ दीर्ये॑त वैष्ण॒वीषु॑ शिपिवि॒ष्टव॑तीषु स्तुवीर॒न्॒.यद्वै य॒ज्ञ्स्या॑-ति॒रिच्य॑ते॒ विष्णुं॒ तच्छि॑पिवि॒ष्टम॒भ्यति॑ रिच्यते॒ तद्विष्णुः॑ शिपिवि॒ष्टोऽति॑रिक्त ए॒वाति॑रिक्तं दधा॒त्यथो॒ अति॑रिक्तेनै॒वा-ति॑रिक्तमा॒प्त्वाऽव॑ ( ) रुन्धते॒ यदि॑ म॒द्ध्यन्दि॑ने॒ दीर्ये॑त वषट्का॒रनि॑धनꣳ॒॒ साम॑ कुर्युर्वषट्का॒रो वै य॒ज्ञ्स्य॑ प्रति॒ष्ठा प्र॑ति॒ष्ठामे॒वैन॑द्-गमयन्ति॒ यदि॑ तृतीयसव॒न ए॒तदे॒व ॥ \newline

\textbf{Pada Paata} \newline

छन्दाꣳ॑सि । वृ॒ङ्क्ते॒ । स॒ज॒नीय॒मिति॑ स - ज॒नीय᳚म् । शस्य᳚म् । वि॒ह॒व्य॑मिति॑ वि - ह॒व्य᳚म् । शस्य᳚म् । अ॒गस्त्य॑स्य । क॒या॒शु॒भीय॒मिति॑ कया-शु॒भीय᳚म् । शस्य᳚म् । ए॒ताव॑त् । वै । अ॒स्ति॒ । याव॑त् । ए॒तत् । याव॑त् । ए॒व । अस्ति॑ । तत् । ए॒षा॒म् । वृ॒ङ्क्ते॒ । यदि॑ । प्रा॒त॒स्स॒व॒न इति॑ प्रातः - स॒व॒ने । क॒लशः॑ । दीर्ये॑त । वै॒ष्ण॒वीषु॑ । शि॒पि॒वि॒ष्टव॑ती॒ष्विति॑ शिपिवि॒ष्ट - व॒ती॒षु॒ । स्तु॒वी॒र॒न्न् । यत् । वै । य॒ज्ञ्स्य॑ । अ॒ति॒रिच्य॑त॒ इत्य॑ति - रिच्य॑ते । विष्णु᳚म् । तत् । शि॒पि॒वि॒ष्टमिति॑ शिपि - वि॒ष्टम् । अ॒भि । अतीति॑ । रि॒च्य॒ते॒ । तत् । विष्णुः॑ । शि॒पि॒वि॒ष्ट इति॑ शिपि - वि॒ष्टः । अति॑रिक्त॒ इत्यति॑- रि॒क्ते॒ । ए॒व । अति॑रिक्त॒मित्यति॑ - रि॒क्त॒म् । द॒धा॒ति॒ । अथो॒ इति॑ । अति॑रिक्ते॒नेत्यति॑ - रि॒क्ते॒न॒ । ए॒व । अति॑रिक्त॒मित्यति॑ - रि॒क्त॒म् । आ॒प्त्वा । अवेति॑ ( ) । रु॒न्ध॒ते॒ । यदि॑ । म॒द्ध्यन्दि॑ने । दीर्ये॑त । व॒ष॒ट्का॒रनि॑धन॒मिति॑ वषट्का॒र - नि॒ध॒न॒म् । साम॑ । कु॒र्युः॒ । व॒ष॒ट्का॒र इति॑ वषट् - का॒रः । वै । य॒ज्ञ्स्य॑ । प्र॒ति॒ष्ठेति॑ प्रति - स्था । प्र॒ति॒ष्ठामिति॑ प्रति - स्थाम् । ए॒व । ए॒न॒त् । ग॒म॒य॒न्ति॒ । यदि॑ । तृ॒ती॒य॒स॒व॒न इति॑ तृतीय - स॒व॒ने । ए॒तत् । ए॒व ॥  \newline


\textbf{Krama Paata} \newline

छन्दाꣳ॑सि वृङ्‍क्ते । वृ॒ङ्‍क्ते॒ स॒ज॒नीय᳚म् । स॒ज॒नीयꣳ॒॒ शस्य᳚म् । स॒ज॒नीय॒मिति॑ स - ज॒नीय᳚म् । शस्य॑म् ॅविह॒व्य᳚म् । वि॒ह॒व्यꣳ॑ शस्य᳚म् । वि॒ह॒व्य॑मिति॑ वि - ह॒व्य᳚म् । शस्य॑म॒गस्त्य॑स्य । अ॒गस्त्य॑स्य कयाशु॒भीय᳚म् । क॒या॒शु॒भीयꣳ॒॒ शस्य᳚म् । क॒या॒शु॒भीय॒मिति॑ कया - शु॒भीय᳚म् । शस्य॑मे॒ताव॑त् । ए॒ताव॒द् वै । वा अ॑स्ति । अ॒स्ति॒ याव॑त् । याव॑दे॒तत् । ए॒तद् याव॑त् । याव॑दे॒व । ए॒वास्ति॑ । अस्ति॒ तत् । तदे॑षाम् । ए॒षा॒म् ॅवृ॒ङ्‍क्ते॒ । वृ॒ङ्‍क्ते॒ यदि॑ । यदि॑ प्रातस्सव॒ने । प्रा॒त॒स्स॒व॒ने क॒लशः॑ । प्रा॒त॒स्स॒व॒न इति॑ प्रातः - स॒व॒ने । क॒लशो॒ दीर्ये॑त । दीर्ये॑त वैष्ण॒वीषु॑ । वै॒ष्ण॒वीषु॑ शिपिवि॒ष्टव॑तीषु । शि॒पि॒वि॒ष्टव॑तीषु स्तुवीरन्न् । शि॒पि॒वि॒ष्टव॑ती॒ष्विति॑ शिपिवि॒ष्ट - व॒ती॒षु॒ । स्तु॒वी॒र॒न्॒. यत् । यद् वै । वै य॒ज्ञ्स्य॑ । य॒ज्ञ्स्या॑ति॒रिच्य॑ते । अ॒ति॒रिच्य॑ते॒ विष्णु᳚म् । अ॒ति॒रिच्य॑त॒ इत्य॑ति - रिच्य॑ते । विष्णु॒म् तत् । तच्छि॑पिवि॒ष्टम् । शि॒पि॒वि॒ष्टम॒भि । शि॒पि॒वि॒ष्टमिति॑ शिपि - वि॒ष्टम् । अ॒भ्यति॑ । अति॑ रिच्यते । रि॒च्य॒ते॒ तत् । तद् विष्णुः॑ । विष्णुः॑ शिपिवि॒ष्टः । शि॒पि॒वि॒ष्टोऽति॑रिक्ते । शि॒पि॒वि॒ष्ट इति॑ शिपि - वि॒ष्टः । अति॑रिक्त ए॒व । अति॑रिक्त॒ इत्यति॑ - रि॒क्ते॒ । ए॒वाति॑रिक्तम् । अति॑रिक्तम् दधाति । अति॑रिक्त॒मित्यति॑ - रि॒क्त॒म् । द॒धा॒त्यथो᳚ । अथो॒ अति॑रिक्तेन । अथो॒ इत्यथो᳚ । अति॑रिक्तेनै॒व । अति॑रिक्ते॒नेत्यति॑ - रि॒क्ते॒न॒ । ए॒वाति॑रिक्तम् । अति॑रिक्तमा॒प्त्वा । अति॑रिक्त॒मित्यति॑ - रि॒क्त॒म् । आ॒प्त्वाऽव॑ ( ) । अव॑ रुन्धते । रु॒न्ध॒ते॒ यदि॑ । यदि॑ म॒द्ध्यन्दि॑ने । म॒द्ध्यन्दि॑ने॒ दीर्ये॑त । दीर्ये॑त वषट्का॒रनि॑धनम् । व॒ष॒ट्का॒रनि॑धनꣳ॒॒ साम॑ । व॒ष॒ट्का॒रनि॑धन॒मिति॑ वषट्का॒र - नि॒ध॒न॒म् । साम॑ कुर्युः । कु॒र्यु॒र् व॒ष॒ट्का॒रः । व॒ष॒ट्का॒रो वै । व॒ष॒ट्का॒र इति॑ वषट् - का॒रः । वै य॒ज्ञ्स्य॑ । य॒ज्ञ्स्य॑ प्रति॒ष्ठा । प्र॒ति॒ष्ठा प्र॑ति॒ष्ठाम् । प्र॒ति॒ष्ठेति॑ प्रति - स्था । प्र॒ति॒ष्ठामे॒व । प्र॒ति॒ष्ठामिति॑ प्रति - स्थाम् । ए॒वैन॑त् । ए॒न॒द् ग॒म॒य॒न्ति॒ । ग॒म॒य॒न्ति॒ यदि॑ । यदि॑ तृतीयसव॒ने । तृ॒ती॒य॒स॒व॒न ए॒तत् । तृ॒ती॒य॒स॒व॒न इति॑ तृतीय - स॒व॒ने । ए॒तदे॒व । ए॒वेत्ये॒व । \newline

\textbf{Jatai Paata} \newline

1. छन्दाꣳ॑सि वृङ्क्ते वृङ्क्ते॒ छन्दाꣳ॑सि॒ छन्दाꣳ॑सि वृङ्क्ते । \newline
2. वृ॒ङ्क्ते॒ स॒ज॒नीयꣳ॑ सज॒नीयं॑ ॅवृङ्क्ते वृङ्क्ते सज॒नीय᳚म् । \newline
3. स॒ज॒नीयꣳ॒॒ शस्यꣳ॒॒ शस्यꣳ॑ सज॒नीयꣳ॑ सज॒नीयꣳ॒॒ शस्य᳚म् । \newline
4. स॒ज॒नीय॒मिति॑ स - ज॒नीय᳚म् । \newline
5. शस्यं॑ ॅविह॒व्यं॑ ॅविह॒व्यꣳ॑ शस्यꣳ॒॒ शस्यं॑ ॅविह॒व्य᳚म् । \newline
6. वि॒ह॒व्यꣳ॑ शस्यꣳ॒॒ शस्यं॑ ॅविह॒व्यं॑ ॅविह॒व्यꣳ॑ शस्य᳚म् । \newline
7. वि॒ह॒व्य॑मिति॑ वि - ह॒व्य᳚म् । \newline
8. शस्य॑ म॒गस्त्य॑स्या॒ गस्त्य॑स्य॒ शस्यꣳ॒॒ शस्य॑ म॒गस्त्य॑स्य । \newline
9. अ॒गस्त्य॑स्य कयाशु॒भीय॑म् कयाशु॒भीय॑ म॒गस्त्य॑स्या॒ गस्त्य॑स्य कयाशु॒भीय᳚म् । \newline
10. क॒या॒शु॒भीयꣳ॒॒ शस्यꣳ॒॒ शस्य॑म् कयाशु॒भीय॑म् कयाशु॒भीयꣳ॒॒ शस्य᳚म् । \newline
11. क॒या॒शु॒भीय॒मिति॑ कया - शु॒भीय᳚म् । \newline
12. शस्य॑ मे॒ताव॑ दे॒ताव॒च् छस्यꣳ॒॒ शस्य॑ मे॒ताव॑त् । \newline
13. ए॒ताव॒द् वै वा ए॒ताव॑ दे॒ताव॒द् वै । \newline
14. वा अ॑स्त्यस्ति॒ वै वा अ॑स्ति । \newline
15. अ॒स्ति॒ याव॒द् याव॑ दस्त्यस्ति॒ याव॑त् । \newline
16. याव॑ दे॒त दे॒तद् याव॒द् याव॑ दे॒तत् । \newline
17. ए॒तद् याव॒द् याव॑ दे॒त दे॒तद् याव॑त् । \newline
18. याव॑ दे॒वैव याव॒द् याव॑ दे॒व । \newline
19. ए॒वास्त्य स्त्ये॒वै वास्ति॑ । \newline
20. अस्ति॒ तत् तद स्त्यस्ति॒ तत् । \newline
21. तदे॑षा मेषा॒म् तत् तदे॑षाम् । \newline
22. ए॒षां॒ ॅवृ॒ङ्क्ते॒ वृ॒ङ्क्त॒ ए॒षा॒ मे॒षां॒ ॅवृ॒ङ्क्ते॒ । \newline
23. वृ॒ङ्क्ते॒ यदि॒ यदि॑ वृङ्क्ते वृङ्क्ते॒ यदि॑ । \newline
24. यदि॑ प्रातस्सव॒ने प्रा॑तस्सव॒ने यदि॒ यदि॑ प्रातस्सव॒ने । \newline
25. प्रा॒त॒स्स॒व॒ने क॒लशः॑ क॒लशः॑ प्रातस्सव॒ने प्रा॑तस्सव॒ने क॒लशः॑ । \newline
26. प्रा॒त॒स्स॒व॒न इति॑ प्रातः - स॒व॒ने । \newline
27. क॒लशो॒ दीर्ये॑त॒ दीर्ये॑त क॒लशः॑ क॒लशो॒ दीर्ये॑त । \newline
28. दीर्ये॑त वैष्ण॒वीषु॑ वैष्ण॒वीषु॒ दीर्ये॑त॒ दीर्ये॑त वैष्ण॒वीषु॑ । \newline
29. वै॒ष्ण॒वीषु॑ शिपिवि॒ष्टव॑तीषु शिपिवि॒ष्टव॑तीषु वैष्ण॒वीषु॑ वैष्ण॒वीषु॑ शिपिवि॒ष्टव॑तीषु । \newline
30. शि॒पि॒वि॒ष्टव॑तीषु स्तुवीरन् थ्स्तुवीरन् च्छिपिवि॒ष्टव॑तीषु शिपिवि॒ष्टव॑तीषु स्तुवीरन्न् । \newline
31. शि॒पि॒वि॒ष्टव॑ती॒ष्विति॑ शिपिवि॒ष्ट - व॒ती॒षु॒ । \newline
32. स्तु॒वी॒र॒न्॒. यद् यथ् स्तु॑वीरन् थ्स्तुवीर॒न्॒. यत् । \newline
33. यद् वै वै यद् यद् वै । \newline
34. वै य॒ज्ञ्स्य॑ य॒ज्ञ्स्य॒ वै वै य॒ज्ञ्स्य॑ । \newline
35. य॒ज्ञ्स्या॑ ति॒रिच्य॑ते ऽति॒रिच्य॑ते य॒ज्ञ्स्य॑ य॒ज्ञ्स्या॑ ति॒रिच्य॑ते । \newline
36. अ॒ति॒रिच्य॑ते॒ विष्णुं॒ ॅविष्णु॑ मति॒रिच्य॑ते ऽति॒रिच्य॑ते॒ विष्णु᳚म् । \newline
37. अ॒ति॒रिच्य॑त॒ इत्य॑ति - रिच्य॑ते । \newline
38. विष्णु॒म् तत् तद् विष्णुं॒ ॅविष्णु॒म् तत् । \newline
39. तच्छि॑पिवि॒ष्टꣳ शि॑पिवि॒ष्टम् तत् तच्छि॑पिवि॒ष्टम् । \newline
40. शि॒पि॒वि॒ष्ट म॒भ्य॑भि शि॑पिवि॒ष्टꣳ शि॑पिवि॒ष्ट म॒भि । \newline
41. शि॒पि॒वि॒ष्टमिति॑ शिपि - वि॒ष्टम् । \newline
42. अ॒भ्य त्य त्य॒भ्य॑ भ्यति॑ । \newline
43. अति॑ रिच्यते रिच्य॒ते ऽत्यति॑ रिच्यते । \newline
44. रि॒च्य॒ते॒ तत् तद् रि॑च्यते रिच्यते॒ तत् । \newline
45. तद् विष्णु॒र् विष्णु॒ स्तत् तद् विष्णुः॑ । \newline
46. विष्णुः॑ शिपिवि॒ष्टः शि॑पिवि॒ष्टो विष्णु॒र् विष्णुः॑ शिपिवि॒ष्टः । \newline
47. शि॒पि॒वि॒ष्टो ऽति॑रि॒क्ते ऽति॑रिक्ते शिपिवि॒ष्टः शि॑पिवि॒ष्टो ऽति॑रिक्ते । \newline
48. शि॒पि॒वि॒ष्ट इति॑ शिपि - वि॒ष्टः । \newline
49. अति॑रिक्त ए॒वै वाति॑रि॒क्ते ऽति॑रिक्त ए॒व । \newline
50. अति॑रिक्त॒ इत्यति॑ - रि॒क्ते॒ । \newline
51. ए॒वा ति॑रिक्त॒ मति॑रिक्त मे॒वै वाति॑रिक्तम् । \newline
52. अति॑रिक्तम् दधाति दधा॒ त्यति॑रिक्त॒ मति॑रिक्तम् दधाति । \newline
53. अति॑रिक्त॒मित्यति॑ - रि॒क्त॒म् । \newline
54. द॒धा॒ त्यथो॒ अथो॑ दधाति दधा॒ त्यथो᳚ । \newline
55. अथो॒ अति॑रिक्ते॒ना ति॑रिक्ते॒नाथो॒ अथो॒ अति॑रिक्तेन । \newline
56. अथो॒ इत्यथो᳚ । \newline
57. अति॑रिक्ते नै॒वै वाति॑रिक्ते॒ना ति॑रिक्तेनै॒व । \newline
58. अति॑रिक्ते॒नेत्यति॑ - रि॒क्ते॒न॒ । \newline
59. ए॒वा ति॑रिक्त॒ मति॑रिक्त मे॒वै वाति॑रिक्तम् । \newline
60. अति॑रिक्त मा॒प्त्वा ऽऽप्त्वा ऽति॑रिक्त॒ मति॑रिक्त मा॒प्त्वा । \newline
61. अति॑रिक्त॒मित्यति॑ - रि॒क्त॒म् । \newline
62. आ॒प्त्वा ऽवावा॒ प्त्वा ऽऽप्त्वा ऽव॑ । \newline
63. अव॑ रुन्धते रुन्ध॒ते ऽवाव॑ रुन्धते । \newline
64. रु॒न्ध॒ते॒ यदि॒ यदि॑ रुन्धते रुन्धते॒ यदि॑ । \newline
65. यदि॑ म॒द्ध्यन्दि॑ने म॒द्ध्यन्दि॑ने॒ यदि॒ यदि॑ म॒द्ध्यन्दि॑ने । \newline
66. म॒द्ध्यन्दि॑ने॒ दीर्ये॑त॒ दीर्ये॑त म॒द्ध्यन्दि॑ने म॒द्ध्यन्दि॑ने॒ दीर्ये॑त । \newline
67. दीर्ये॑त वषट्का॒रनि॑धनं ॅवषट्का॒रनि॑धन॒म् दीर्ये॑त॒ दीर्ये॑त वषट्का॒रनि॑धनम् । \newline
68. व॒ष॒ट्का॒रनि॑धनꣳ॒॒ साम॒ साम॑ वषट्का॒रनि॑धनं ॅवषट्का॒रनि॑धनꣳ॒॒ साम॑ । \newline
69. व॒ष॒ट्का॒रनि॑धन॒मिति॑ वषट्का॒र - नि॒ध॒न॒म् । \newline
70. साम॑ कुर्युः कुर्युः॒ साम॒ साम॑ कुर्युः । \newline
71. कु॒र्यु॒र् व॒ष॒ट्का॒रो व॑षट्का॒रः कु॑र्युः कुर्युर् वषट्का॒रः । \newline
72. व॒ष॒ट्का॒रो वै वै व॑षट्का॒रो व॑षट्का॒रो वै । \newline
73. व॒ष॒ट्का॒र इति॑ वषट् - का॒रः । \newline
74. वै य॒ज्ञ्स्य॑ य॒ज्ञ्स्य॒ वै वै य॒ज्ञ्स्य॑ । \newline
75. य॒ज्ञ्स्य॑ प्रति॒ष्ठा प्र॑ति॒ष्ठा य॒ज्ञ्स्य॑ य॒ज्ञ्स्य॑ प्रति॒ष्ठा । \newline
76. प्र॒ति॒ष्ठा प्र॑ति॒ष्ठाम् प्र॑ति॒ष्ठाम् प्र॑ति॒ष्ठा प्र॑ति॒ष्ठा प्र॑ति॒ष्ठाम् । \newline
77. प्र॒ति॒ष्ठेति॑ प्रति - स्था । \newline
78. प्र॒ति॒ष्ठा मे॒वैव प्र॑ति॒ष्ठाम् प्र॑ति॒ष्ठा मे॒व । \newline
79. प्र॒ति॒ष्ठामिति॑ प्रति - स्थाम् । \newline
80. ए॒वैन॑ देन दे॒वै वैन॑त् । \newline
81. ए॒न॒द् ग॒म॒य॒न्ति॒ ग॒म॒य॒ न्त्ये॒न॒ दे॒न॒द् ग॒म॒य॒न्ति॒ । \newline
82. ग॒म॒य॒न्ति॒ यदि॒ यदि॑ गमयन्ति गमयन्ति॒ यदि॑ । \newline
83. यदि॑ तृतीयसव॒ने तृ॑तीयसव॒ने यदि॒ यदि॑ तृतीयसव॒ने । \newline
84. तृ॒ती॒य॒स॒व॒न ए॒त दे॒तत् तृ॑तीयसव॒ने तृ॑तीयसव॒न ए॒तत् । \newline
85. तृ॒ती॒य॒स॒व॒न इति॑ तृतीय - स॒व॒ने । \newline
86. ए॒त दे॒वै वैत दे॒त दे॒व । \newline
87. ए॒वेत्ये॒व । \newline

\textbf{Ghana Paata } \newline

1. छन्दाꣳ॑सि वृङ्क्ते वृङ्क्ते॒ छन्दाꣳ॑सि॒ छन्दाꣳ॑सि वृङ्क्ते सज॒नीयꣳ॑ सज॒नीयं॑ ॅवृङ्क्ते॒ छन्दाꣳ॑सि॒ छन्दाꣳ॑सि वृङ्क्ते सज॒नीय᳚म् । \newline
2. वृ॒ङ्क्ते॒ स॒ज॒नीयꣳ॑ सज॒नीयं॑ ॅवृङ्क्ते वृङ्क्ते सज॒नीयꣳ॒॒ शस्यꣳ॒॒ शस्यꣳ॑ सज॒नीयं॑ ॅवृङ्क्ते वृङ्क्ते सज॒नीयꣳ॒॒ शस्य᳚म् । \newline
3. स॒ज॒नीयꣳ॒॒ शस्यꣳ॒॒ शस्यꣳ॑ सज॒नीयꣳ॑ सज॒नीयꣳ॒॒ शस्यं॑ ॅविह॒व्यं॑ ॅविह॒व्यꣳ॑ शस्यꣳ॑ सज॒नीयꣳ॑ सज॒नीयꣳ॒॒ शस्यं॑ ॅविह॒व्य᳚म् । \newline
4. स॒ज॒नीय॒मिति॑ स - ज॒नीय᳚म् । \newline
5. शस्यं॑ ॅविह॒व्यं॑ ॅविह॒व्यꣳ॑ शस्यꣳ॒॒ शस्यं॑ ॅविह॒व्यꣳ॑ शस्यꣳ॒॒ शस्यं॑ ॅविह॒व्यꣳ॑ शस्यꣳ॒॒ शस्यं॑ ॅविह॒व्यꣳ॑ शस्य᳚म् । \newline
6. वि॒ह॒व्यꣳ॑ शस्यꣳ॒॒ शस्यं॑ ॅविह॒व्यं॑ ॅविह॒व्यꣳ॑ शस्य॑ म॒गस्त्य॑स्या॒ गस्त्य॑स्य॒ शस्यं॑ ॅविह॒व्यं॑ ॅविह॒व्यꣳ॑ शस्य॑ म॒गस्त्य॑स्य । \newline
7. वि॒ह॒व्य॑मिति॑ वि - ह॒व्य᳚म् । \newline
8. शस्य॑ म॒गस्त्य॑स्या॒ गस्त्य॑स्य॒ शस्यꣳ॒॒ शस्य॑ म॒गस्त्य॑स्य कयाशु॒भीय॑म् कयाशु॒भीय॑ म॒गस्त्य॑स्य॒ शस्यꣳ॒॒ शस्य॑ म॒गस्त्य॑स्य कयाशु॒भीय᳚म् । \newline
9. अ॒गस्त्य॑स्य कयाशु॒भीय॑म् कयाशु॒भीय॑ म॒गस्त्य॑स्या॒ गस्त्य॑स्य कयाशु॒भीयꣳ॒॒ शस्यꣳ॒॒ शस्य॑म् कयाशु॒भीय॑ म॒गस्त्य॑स्या॒ गस्त्य॑स्य कयाशु॒भीयꣳ॒॒ शस्य᳚म् । \newline
10. क॒या॒शु॒भीयꣳ॒॒ शस्यꣳ॒॒ शस्य॑म् कयाशु॒भीय॑म् कयाशु॒भीयꣳ॒॒ शस्य॑ मे॒ताव॑ दे॒ताव॒च् छस्य॑म् कयाशु॒भीय॑म् कयाशु॒भीयꣳ॒॒ शस्य॑ मे॒ताव॑त् । \newline
11. क॒या॒शु॒भीय॒मिति॑ कया - शु॒भीय᳚म् । \newline
12. शस्य॑ मे॒ताव॑ दे॒ताव॒च् छस्यꣳ॒॒ शस्य॑ मे॒ताव॒द् वै वा ए॒ताव॒च् छस्यꣳ॒॒ शस्य॑ मे॒ताव॒द् वै । \newline
13. ए॒ताव॒द् वै वा ए॒ताव॑ दे॒ताव॒द् वा अ॑स्त्यस्ति॒ वा ए॒ताव॑ दे॒ताव॒द् वा अ॑स्ति । \newline
14. वा अ॑स्त्यस्ति॒ वै वा अ॑स्ति॒ याव॒द् याव॑ दस्ति॒ वै वा अ॑स्ति॒ याव॑त् । \newline
15. अ॒स्ति॒ याव॒द् याव॑द स्त्यस्ति॒ याव॑ दे॒त दे॒तद् याव॑ दस्त्यस्ति॒ याव॑ दे॒तत् । \newline
16. याव॑ दे॒त दे॒तद् याव॒द् याव॑ दे॒तद् याव॒द् याव॑ दे॒तद् याव॒द् याव॑ दे॒तद् याव॑त् । \newline
17. ए॒तद् याव॒द् याव॑ दे॒त दे॒तद् याव॑ दे॒वैव याव॑ दे॒त दे॒तद् याव॑ दे॒व । \newline
18. याव॑ दे॒वैव याव॒द् याव॑ दे॒वा स्त्य स्त्ये॒व याव॒द् याव॑ दे॒वास्ति॑ । \newline
19. ए॒वास्त्य स्त्ये॒वै वास्ति॒ तत् तदस्त्ये॒वै वास्ति॒ तत् । \newline
20. अस्ति॒ तत् तदस्त्यस्ति॒ तदे॑षा मेषा॒म् तदस्त्यस्ति॒ तदे॑षाम् । \newline
21. तदे॑षा मेषा॒म् तत् तदे॑षां ॅवृङ्क्ते वृङ्क्त एषा॒म् तत् तदे॑षां ॅवृङ्क्ते । \newline
22. ए॒षां॒ ॅवृ॒ङ्क्ते॒ वृ॒ङ्क्त॒ ए॒षा॒ मे॒षां॒ ॅवृ॒ङ्क्ते॒ यदि॒ यदि॑ वृङ्क्त एषा मेषां ॅवृङ्क्ते॒ यदि॑ । \newline
23. वृ॒ङ्क्ते॒ यदि॒ यदि॑ वृङ्क्ते वृङ्क्ते॒ यदि॑ प्रातस्सव॒ने प्रा॑तस्सव॒ने यदि॑ वृङ्क्ते वृङ्क्ते॒ यदि॑ प्रातस्सव॒ने । \newline
24. यदि॑ प्रातस्सव॒ने प्रा॑तस्सव॒ने यदि॒ यदि॑ प्रातस्सव॒ने क॒लशः॑ क॒लशः॑ प्रातस्सव॒ने यदि॒ यदि॑ प्रातस्सव॒ने क॒लशः॑ । \newline
25. प्रा॒त॒स्स॒व॒ने क॒लशः॑ क॒लशः॑ प्रातस्सव॒ने प्रा॑तस्सव॒ने क॒लशो॒ दीर्ये॑त॒ दीर्ये॑त क॒लशः॑ प्रातस्सव॒ने प्रा॑तस्सव॒ने क॒लशो॒ दीर्ये॑त । \newline
26. प्रा॒त॒स्स॒व॒न इति॑ प्रातः - स॒व॒ने । \newline
27. क॒लशो॒ दीर्ये॑त॒ दीर्ये॑त क॒लशः॑ क॒लशो॒ दीर्ये॑त वैष्ण॒वीषु॑ वैष्ण॒वीषु॒ दीर्ये॑त क॒लशः॑ क॒लशो॒ दीर्ये॑त वैष्ण॒वीषु॑ । \newline
28. दीर्ये॑त वैष्ण॒वीषु॑ वैष्ण॒वीषु॒ दीर्ये॑त॒ दीर्ये॑त वैष्ण॒वीषु॑ शिपिवि॒ष्टव॑तीषु शिपिवि॒ष्टव॑तीषु वैष्ण॒वीषु॒ दीर्ये॑त॒ दीर्ये॑त वैष्ण॒वीषु॑ शिपिवि॒ष्टव॑तीषु । \newline
29. वै॒ष्ण॒वीषु॑ शिपिवि॒ष्टव॑तीषु शिपिवि॒ष्टव॑तीषु वैष्ण॒वीषु॑ वैष्ण॒वीषु॑ शिपिवि॒ष्टव॑तीषु स्तुवीरन् थ्स्तुवीरन् च्छिपिवि॒ष्टव॑तीषु वैष्ण॒वीषु॑ वैष्ण॒वीषु॑ शिपिवि॒ष्टव॑तीषु स्तुवीरन्न् । \newline
30. शि॒पि॒वि॒ष्टव॑तीषु स्तुवीरन् थ्स्तुवीरन् च्छिपिवि॒ष्टव॑तीषु शिपिवि॒ष्टव॑तीषु स्तुवीर॒न्॒. यद् यथ् स्तु॑वीरन् च्छिपिवि॒ष्टव॑तीषु शिपिवि॒ष्टव॑तीषु स्तुवीर॒न्॒. यत् । \newline
31. शि॒पि॒वि॒ष्टव॑ती॒ष्विति॑ शिपिवि॒ष्ट - व॒ती॒षु॒ । \newline
32. स्तु॒वी॒र॒न्॒. यद् यथ् स्तु॑वीरन् थ्स्तुवीर॒न्॒. यद् वै वै यथ् स्तु॑वीरन् थ्स्तुवीर॒न्॒. यद् वै । \newline
33. यद् वै वै यद् यद् वै य॒ज्ञ्स्य॑ य॒ज्ञ्स्य॒ वै यद् यद् वै य॒ज्ञ्स्य॑ । \newline
34. वै य॒ज्ञ्स्य॑ य॒ज्ञ्स्य॒ वै वै य॒ज्ञ्स्या॑ ति॒रिच्य॑ते ऽति॒रिच्य॑ते य॒ज्ञ्स्य॒ वै वै य॒ज्ञ्स्या॑ ति॒रिच्य॑ते । \newline
35. य॒ज्ञ्स्या॑ ति॒रिच्य॑ते ऽति॒रिच्य॑ते य॒ज्ञ्स्य॑ य॒ज्ञ्स्या॑ ति॒रिच्य॑ते॒ विष्णुं॒ ॅविष्णु॑ मति॒रिच्य॑ते य॒ज्ञ्स्य॑ य॒ज्ञ्स्या॑ ति॒रिच्य॑ते॒ विष्णु᳚म् । \newline
36. अ॒ति॒रिच्य॑ते॒ विष्णुं॒ ॅविष्णु॑ मति॒रिच्य॑ते ऽति॒रिच्य॑ते॒ विष्णु॒म् तत् तद् विष्णु॑ मति॒रिच्य॑ते ऽति॒रिच्य॑ते॒ विष्णु॒म् तत् । \newline
37. अ॒ति॒रिच्य॑त॒ इत्य॑ति - रिच्य॑ते । \newline
38. विष्णु॒म् तत् तद् विष्णुं॒ ॅविष्णु॒म् तच्छि॑पिवि॒ष्टꣳ शि॑पिवि॒ष्टम् तद् विष्णुं॒ ॅविष्णु॒म् तच्छि॑पिवि॒ष्टम् । \newline
39. तच्छि॑पिवि॒ष्टꣳ शि॑पिवि॒ष्टम् तत् तच्छि॑पिवि॒ष्ट म॒भ्य॑भि शि॑पिवि॒ष्टम् तत् तच्छि॑पिवि॒ष्ट म॒भि । \newline
40. शि॒पि॒वि॒ष्ट म॒भ्य॑भि शि॑पिवि॒ष्टꣳ शि॑पिवि॒ष्ट म॒भ्य त्यत्य॒भि शि॑पिवि॒ष्टꣳ शि॑पिवि॒ष्ट म॒भ्यति॑ । \newline
41. शि॒पि॒वि॒ष्टमिति॑ शिपि - वि॒ष्टम् । \newline
42. अ॒भ्यत्य त्य॒भ्य॑ भ्यति॑ रिच्यते रिच्य॒ते ऽत्य॒भ्य॑ भ्यति॑ रिच्यते । \newline
43. अति॑ रिच्यते रिच्य॒ते ऽत्यति॑ रिच्यते॒ तत् तद् रि॑च्य॒ते ऽत्यति॑ रिच्यते॒ तत् । \newline
44. रि॒च्य॒ते॒ तत् तद् रि॑च्यते रिच्यते॒ तद् विष्णु॒र् विष्णु॒ स्तद् रि॑च्यते रिच्यते॒ तद् विष्णुः॑ । \newline
45. तद् विष्णु॒र् विष्णु॒ स्तत् तद् विष्णुः॑ शिपिवि॒ष्टः शि॑पिवि॒ष्टो विष्णु॒ स्तत् तद् विष्णुः॑ शिपिवि॒ष्टः । \newline
46. विष्णुः॑ शिपिवि॒ष्टः शि॑पिवि॒ष्टो विष्णु॒र् विष्णुः॑ शिपिवि॒ष्टो ऽति॑रि॒क्ते ऽति॑रिक्ते शिपिवि॒ष्टो विष्णु॒र् विष्णुः॑ शिपिवि॒ष्टो ऽति॑रिक्ते । \newline
47. शि॒पि॒वि॒ष्टो ऽति॑रि॒क्ते ऽति॑रिक्ते शिपिवि॒ष्टः शि॑पिवि॒ष्टो ऽति॑रिक्त ए॒वै वाति॑रिक्ते शिपिवि॒ष्टः शि॑पिवि॒ष्टो ऽति॑रिक्त ए॒व । \newline
48. शि॒पि॒वि॒ष्ट इति॑ शिपि - वि॒ष्टः । \newline
49. अति॑रिक्त ए॒वै वाति॑रि॒क्ते ऽति॑रिक्त ए॒वा ति॑रिक्त॒ मति॑रिक्त मे॒वा ति॑रि॒क्ते ऽति॑रिक्त ए॒वा ति॑रिक्तम् । \newline
50. अति॑रिक्त॒ इत्यति॑ - रि॒क्ते॒ । \newline
51. ए॒वा ति॑रिक्त॒ मति॑रिक्त मे॒वै वाति॑रिक्तम् दधाति दधा॒ त्यति॑रिक्त मे॒वै वाति॑रिक्तम् दधाति । \newline
52. अति॑रिक्तम् दधाति दधा॒ त्यति॑रिक्त॒ मति॑रिक्तम् दधा॒ त्यथो॒ अथो॑ दधा॒ त्यति॑रिक्त॒ मति॑रिक्तम् दधा॒ त्यथो᳚ । \newline
53. अति॑रिक्त॒मित्यति॑ - रि॒क्त॒म् । \newline
54. द॒धा॒ त्यथो॒ अथो॑ दधाति दधा॒ त्यथो॒ अति॑रिक्ते॒ना ति॑रिक्ते॒ नाथो॑ दधाति दधा॒ त्यथो॒ अति॑रिक्तेन । \newline
55. अथो॒ अति॑रिक्ते॒ना ति॑रिक्ते॒नाथो॒ अथो॒ अति॑रिक्तेनै॒वै वाति॑रिक्ते॒ नाथो॒ अथो॒ अति॑रिक्ते नै॒व । \newline
56. अथो॒ इत्यथो᳚ । \newline
57. अति॑रिक्ते नै॒वै वाति॑रिक्ते॒ना ति॑रिक्ते नै॒वाति॑रिक्त॒ मति॑रिक्त मे॒वाति॑रिक्ते॒ नाति॑रिक्ते नै॒वाति॑रिक्तम् । \newline
58. अति॑रिक्ते॒नेत्यति॑ - रि॒क्ते॒न॒ । \newline
59. ए॒वाति॑रिक्त॒ मति॑रिक्त मे॒वै वाति॑रिक्त मा॒प्त्वा ऽऽप्त्वा ऽति॑रिक्त मे॒वै वाति॑रिक्त मा॒प्त्वा । \newline
60. अति॑रिक्त मा॒प्त्वा ऽऽप्त्वा ऽति॑रिक्त॒ मति॑रिक्त मा॒प्त्वा ऽवावा॒प्त्वा ऽति॑रिक्त॒ मति॑रिक्त मा॒प्त्वा ऽव॑ । \newline
61. अति॑रिक्त॒मित्यति॑ - रि॒क्त॒म् । \newline
62. आ॒प्त्वा ऽवावा॒प्त्वा ऽऽप्त्वा ऽव॑ रुन्धते रुन्ध॒ते ऽवा॒प्त्वा ऽऽप्त्वा ऽव॑ रुन्धते । \newline
63. अव॑ रुन्धते रुन्ध॒ते ऽवाव॑ रुन्धते॒ यदि॒ यदि॑ रुन्ध॒ते ऽवाव॑ रुन्धते॒ यदि॑ । \newline
64. रु॒न्ध॒ते॒ यदि॒ यदि॑ रुन्धते रुन्धते॒ यदि॑ म॒द्ध्यन्दि॑ने म॒द्ध्यन्दि॑ने॒ यदि॑ रुन्धते रुन्धते॒ यदि॑ म॒द्ध्यन्दि॑ने । \newline
65. यदि॑ म॒द्ध्यन्दि॑ने म॒द्ध्यन्दि॑ने॒ यदि॒ यदि॑ म॒द्ध्यन्दि॑ने॒ दीर्ये॑त॒ दीर्ये॑त म॒द्ध्यन्दि॑ने॒ यदि॒ यदि॑ म॒द्ध्यन्दि॑ने॒ दीर्ये॑त । \newline
66. म॒द्ध्यन्दि॑ने॒ दीर्ये॑त॒ दीर्ये॑त म॒द्ध्यन्दि॑ने म॒द्ध्यन्दि॑ने॒ दीर्ये॑त वषट्का॒रनि॑धनं ॅवषट्का॒रनि॑धन॒म् दीर्ये॑त म॒द्ध्यन्दि॑ने म॒द्ध्यन्दि॑ने॒ दीर्ये॑त वषट्का॒रनि॑धनम् । \newline
67. दीर्ये॑त वषट्का॒रनि॑धनं ॅवषट्का॒रनि॑धन॒म् दीर्ये॑त॒ दीर्ये॑त वषट्का॒रनि॑धनꣳ॒॒ साम॒ साम॑ वषट्का॒रनि॑धन॒म् दीर्ये॑त॒ दीर्ये॑त वषट्का॒रनि॑धनꣳ॒॒ साम॑ । \newline
68. व॒ष॒ट्का॒रनि॑धनꣳ॒॒ साम॒ साम॑ वषट्का॒रनि॑धनं ॅवषट्का॒रनि॑धनꣳ॒॒ साम॑ कुर्युः कुर्युः॒ साम॑ वषट्का॒रनि॑धनं ॅवषट्का॒रनि॑धनꣳ॒॒ साम॑ कुर्युः । \newline
69. व॒ष॒ट्का॒रनि॑धन॒मिति॑ वषट्का॒र - नि॒ध॒न॒म् । \newline
70. साम॑ कुर्युः कुर्युः॒ साम॒ साम॑ कुर्युर् वषट्का॒रो व॑षट्का॒रः कु॑र्युः॒ साम॒ साम॑ कुर्युर् वषट्का॒रः । \newline
71. कु॒र्यु॒र् व॒ष॒ट्का॒रो व॑षट्का॒रः कु॑र्युः कुर्युर् वषट्का॒रो वै वै व॑षट्का॒रः कु॑र्युः कुर्युर् वषट्का॒रो वै । \newline
72. व॒ष॒ट्का॒रो वै वै व॑षट्का॒रो व॑षट्का॒रो वै य॒ज्ञ्स्य॑ य॒ज्ञ्स्य॒ वै व॑षट्का॒रो व॑षट्का॒रो वै य॒ज्ञ्स्य॑ । \newline
73. व॒ष॒ट्का॒र इति॑ वषट् - का॒रः । \newline
74. वै य॒ज्ञ्स्य॑ य॒ज्ञ्स्य॒ वै वै य॒ज्ञ्स्य॑ प्रति॒ष्ठा प्र॑ति॒ष्ठा य॒ज्ञ्स्य॒ वै वै य॒ज्ञ्स्य॑ प्रति॒ष्ठा । \newline
75. य॒ज्ञ्स्य॑ प्रति॒ष्ठा प्र॑ति॒ष्ठा य॒ज्ञ्स्य॑ य॒ज्ञ्स्य॑ प्रति॒ष्ठा प्र॑ति॒ष्ठाम् प्र॑ति॒ष्ठाम् प्र॑ति॒ष्ठा य॒ज्ञ्स्य॑ य॒ज्ञ्स्य॑ प्रति॒ष्ठा प्र॑ति॒ष्ठाम् । \newline
76. प्र॒ति॒ष्ठा प्र॑ति॒ष्ठाम् प्र॑ति॒ष्ठाम् प्र॑ति॒ष्ठा प्र॑ति॒ष्ठा प्र॑ति॒ष्ठा मे॒वैव प्र॑ति॒ष्ठाम् प्र॑ति॒ष्ठा प्र॑ति॒ष्ठा प्र॑ति॒ष्ठा मे॒व । \newline
77. प्र॒ति॒ष्ठेति॑ प्रति - स्था । \newline
78. प्र॒ति॒ष्ठा मे॒वैव प्र॑ति॒ष्ठाम् प्र॑ति॒ष्ठा मे॒वैन॑ देन दे॒व प्र॑ति॒ष्ठाम् प्र॑ति॒ष्ठा मे॒वैन॑त् । \newline
79. प्र॒ति॒ष्ठामिति॑ प्रति - स्थाम् । \newline
80. ए॒वैन॑ देन दे॒वै वैन॑द् गमयन्ति गमय न्त्येन दे॒वै वैन॑द् गमयन्ति । \newline
81. ए॒न॒द् ग॒म॒य॒न्ति॒ ग॒म॒य॒ न्त्ये॒न॒ दे॒न॒द् ग॒म॒य॒न्ति॒ यदि॒ यदि॑ गमय न्त्येन देनद् गमयन्ति॒ यदि॑ । \newline
82. ग॒म॒य॒न्ति॒ यदि॒ यदि॑ गमयन्ति गमयन्ति॒ यदि॑ तृतीयसव॒ने तृ॑तीयसव॒ने यदि॑ गमयन्ति गमयन्ति॒ यदि॑ तृतीयसव॒ने । \newline
83. यदि॑ तृतीयसव॒ने तृ॑तीयसव॒ने यदि॒ यदि॑ तृतीयसव॒न ए॒त दे॒तत् तृ॑तीयसव॒ने यदि॒ यदि॑ तृतीयसव॒न ए॒तत् । \newline
84. तृ॒ती॒य॒स॒व॒न ए॒त दे॒तत् तृ॑तीयसव॒ने तृ॑तीयसव॒न ए॒त दे॒वै वैतत् तृ॑तीयसव॒ने तृ॑तीयसव॒न ए॒त दे॒व । \newline
85. तृ॒ती॒य॒स॒व॒न इति॑ तृतीय - स॒व॒ने । \newline
86. ए॒त दे॒वै वैत दे॒त दे॒व । \newline
87. ए॒वेत्ये॒व । \newline
\pagebreak
\markright{ TS 7.5.6.1  \hfill https://www.vedavms.in \hfill}

\section{ TS 7.5.6.1 }

\textbf{TS 7.5.6.1 } \newline
\textbf{Samhita Paata} \newline

ष॒ड॒हैर्मासा᳚न्थ् स॒पांद्याह॒रुथ् सृ॑जन्ति षड॒हैर्.हि मासा᳚न्थ् स॒पंश्य॑न्त्य-र्द्धमा॒सैर्मासा᳚न्थ् स॒पांद्याह॒रुथ् सृ॑जन्त्य-र्द्धमा॒सैर्.हि मासा᳚न्थ् स॒पंश्य॑न्त्यमावा॒स्य॑या॒ मासा᳚न्थ् स॒पांद्याह॒रुथ् सृ॑जन्त्यमावा॒स्य॑या॒ हि मासा᳚न्थ् स॒पंश्य॑न्ति पौर्णमा॒स्या मासा᳚न्थ् स॒पांद्याऽहरुथ् सृ॑जन्ति पौर्णमा॒स्या हि मासा᳚न्थ् स॒पंश्य॑न्ति॒ यो वै पू॒र्ण आ॑सि॒ञ्चति॒ परा॒ स सि॑ञ्चति॒ यः पू॒र्णादु॒दच॑ति - [  ] \newline

\textbf{Pada Paata} \newline

ष॒ड॒हैरिति॑ षट् - अ॒हैः । मासान्॑ । स॒पांद्येति॑ सं - पाद्य॑ । अहः॑ । उदिति॑ । सृ॒ज॒न्ति॒ । ष॒ड॒हैरिति॑ षट् - अ॒हैः । हि । मासान्॑ । स॒पंश्य॒न्तीति॑ सं - पश्य॑न्ति । अ॒द्‌र्ध॒मा॒सैरित्य॑द्‌र्ध - मा॒सैः । मासान्॑ । स॒पांद्येति॑ सं - पाद्य॑ । अहः॑ । उदिति॑ । सृ॒ज॒न्ति॒ । अ॒द्‌र्ध॒मा॒सैरित्य॑द्‌र्ध - मा॒सैः । हि । मासान्॑ । स॒पंश्य॒न्तीति॑ सं - पश्य॑न्ति । अ॒मा॒वा॒स्य॑येत्य॑मा - वा॒स्य॑या । मासान्॑ । स॒पांद्येति॑ सं - पाद्य॑ । अहः॑ । उदिति॑ । सृ॒ज॒न्ति॒ । अ॒मा॒वा॒स्य॑येत्य॑मा - वा॒स्य॑या । हि । मासान्॑ । स॒पंश्य॒न्तीति॑ सं-पश्य॑न्ति । पौ॒र्ण॒मा॒स्येति॑ पौर्ण-मा॒स्या । मासान्॑ । स॒पांद्येति॑ सं - पाद्य॑ । अहः॑ । उदिति॑ । सृ॒ज॒न्ति॒ । पौ॒र्ण॒मा॒स्येति॑ पौर्ण - मा॒स्या । हि । मासान्॑ । स॒पंश्य॒न्तीति॑ सं - पश्य॑न्ति । यः । वै । पू॒र्णे । आ॒सि॒ञ्चतीत्या᳚ - सि॒ञ्चति॑ ।परेति॑ । सः । सि॒ञ्च॒ति॒ । यः । पू॒र्णात् । उ॒दच॒तीत्यु॑त् - अच॑ति ।  \newline


\textbf{Krama Paata} \newline

ष॒ड॒हैर् मासान्॑ । ष॒ड॒हैरिति॑ षट् - अ॒हैः । मासा᳚न्थ् स॒म्पाद्य॑ । स॒म्पाद्याहः॑ । स॒म्पाद्येति॑ सम् - पाद्य॑ । अह॒रुत् । उथ् सृ॑जन्ति । सृ॒ज॒न्ति॒ ष॒ड॒हैः । ष॒ड॒हैर्. हि । ष॒ड॒हैरिति॑ षट् - अ॒हैः । हि मासान्॑ । मासा᳚न्थ् स॒म्पश्य॑न्ति । स॒म्पश्य॑न्त्यर्द्धमा॒सैः । स॒म्पश्य॒न्तीति॑ सम् - पश्य॑न्ति । अ॒र्द्ध॒मा॒सैर् मासान्॑ । अ॒र्द्ध॒मा॒सैरित्य॑र्द्ध - मा॒सैः । मासा᳚न्थ् स॒म्पाद्य॑ । स॒म्पाद्याहः॑ । स॒म्पाद्येति॑ सम् - पाद्य॑ । अह॒रुत् । उथ् सृ॑जन्ति । सृ॒ज॒न्त्य॒र्द्ध॒मा॒सैः । अ॒र्द्ध॒मा॒सैर्. हि । अ॒र्द्ध॒मा॒सैरित्य॑र्द्ध - मा॒सैः । हि मासान्॑ । मासा᳚न्थ् स॒म्पश्य॑न्ति । स॒म्पश्य॑न्त्यमावा॒स्य॑या । स॒म्पश्य॒न्तीति॑ सम् - पश्य॑न्ति । अ॒मा॒वा॒स्य॑या॒ मासान्॑ । अ॒मा॒वा॒स्य॑येत्य॑मा - वा॒स्य॑या । मासा᳚न्थ् स॒म्पाद्य॑ । स॒म्पाद्याहः॑ । स॒म्पाद्येति॑ सम् - पाद्य॑ । अह॒रुत् । उथ् सृ॑जन्ति । सृ॒ज॒न्त्य॒मा॒वा॒स्य॑या । अ॒मा॒वा॒स्य॑या॒ हि । अ॒मा॒वा॒स्य॑येत्य॑मा - वा॒स्य॑या । हि मासान्॑ । मासा᳚न्थ् स॒म्पश्य॑न्ति । स॒म्पश्य॑न्ति पौर्णमा॒स्या । स॒म्पश्य॒न्तीति॑ सम् - पश्य॑न्ति । पौ॒र्ण॒मा॒स्या मासान्॑ । पौ॒र्ण॒मा॒स्येति॑ पौर्ण - मा॒स्या । मासा᳚न्थ् स॒म्पाद्य॑ । स॒म्पाद्याहः॑ । स॒म्पाद्येति॑ सम् - पाद्य॑ । अह॒रुत् । उथ् सृ॑जन्ति । सृ॒ज॒न्ति॒ पौ॒र्ण॒मा॒स्या । पौ॒र्ण॒मा॒स्या हि । पौ॒र्ण॒मा॒स्येति॑ पौर्ण - मा॒स्या । हि मासान्॑ । मासा᳚न्थ् स॒म्पश्य॑न्ति । स॒म्पश्य॑न्ति॒ यः । स॒म्पश्य॒न्तीति॑ सम् - पश्य॑न्ति । यो वै । वै पू॒र्णे । पू॒र्ण आ॑सि॒ञ्चति॑ । आ॒सि॒ञ्चति॒ परा᳚ । आ॒सि॒ञ्चतीत्या᳚ - सि॒ञ्चति॑ । परा॒ सः । स सि॑ञ्चति । सि॒ञ्च॒ति॒ यः । यः पू॒र्णात् । पू॒र्णादु॒दच॑ति । उ॒दच॑ति प्रा॒णम् । उ॒दच॒तीत्यु॑त् - अच॑ति \newline

\textbf{Jatai Paata} \newline

1. ष॒ड॒हैर् मासा॒न् मासा᳚न् षड॒है ष्ष॑ड॒हैर् मासान्॑ । \newline
2. ष॒ड॒हैरिति॑ षट् - अ॒हैः । \newline
3. मासा᳚न् थ्सं॒पाद्य॑ सं॒पाद्य॒ मासा॒न् मासा᳚न् थ्सं॒पाद्य॑ । \newline
4. सं॒पाद्या ह॒ रहः॑ सं॒पाद्य॑ सं॒पाद्या हः॑ । \newline
5. सं॒पाद्येति॑ सं - पाद्य॑ । \newline
6. अह॒ रुदु दह॒ रह॒ रुत् । \newline
7. उथ् सृ॑जन्ति सृज॒ न्त्युदुथ् सृ॑जन्ति । \newline
8. सृ॒ज॒न्ति॒ ष॒ड॒है ष्ष॑ड॒हैः सृ॑जन्ति सृजन्ति षड॒हैः । \newline
9. ष॒ड॒हैर्. हि हि ष॑ड॒है ष्ष॑ड॒हैर्. हि । \newline
10. ष॒ड॒हैरिति॑ षट् - अ॒हैः । \newline
11. हि मासा॒न् मासा॒न्॒. हि हि मासान्॑ । \newline
12. मासा᳚न् थ्सं॒पश्य॑न्ति सं॒पश्य॑न्ति॒ मासा॒न् मासा᳚न् थ्सं॒पश्य॑न्ति । \newline
13. सं॒पश्य॑ न्त्यर्द्धमा॒सै र॑र्द्धमा॒सैः सं॒पश्य॑न्ति सं॒पश्य॑ न्त्यर्द्धमा॒सैः । \newline
14. सं॒पश्य॒न्तीति॑ सं - पश्य॑न्ति । \newline
15. अ॒र्द्ध॒मा॒सैर् मासा॒न् मासा॑न-र्द्धमा॒सै र॑र्द्धमा॒सैर् मासान्॑ । \newline
16. अ॒र्द्ध॒मा॒सैरित्य॑र्द्ध - मा॒सैः । \newline
17. मासा᳚न् थ्सं॒पाद्य॑ सं॒पाद्य॒ मासा॒न् मासा᳚न् थ्सं॒पाद्य॑ । \newline
18. सं॒पाद्या ह॒ रहः॑ सं॒पाद्य॑ सं॒पाद्या हः॑ । \newline
19. सं॒पाद्येति॑ सं - पाद्य॑ । \newline
20. अह॒ रुदु दह॒ रह॒ रुत् । \newline
21. उथ् सृ॑जन्ति सृज॒ न्त्युदुथ् सृ॑जन्ति । \newline
22. सृ॒ज॒ न्त्य॒र्द्ध॒मा॒सै र॑र्द्धमा॒सैः सृ॑जन्ति सृज न्त्यर्द्धमा॒सैः । \newline
23. अ॒र्द्ध॒मा॒सैर्. हि ह्य॑र्द्धमा॒सै र॑र्द्धमा॒सैर्. हि । \newline
24. अ॒र्द्ध॒मा॒सैरित्य॑र्द्ध - मा॒सैः । \newline
25. हि मासा॒न् मासा॒न्॒. हि हि मासान्॑ । \newline
26. मासा᳚न् थ्सं॒पश्य॑न्ति सं॒पश्य॑न्ति॒ मासा॒न् मासा᳚न् थ्सं॒पश्य॑न्ति । \newline
27. सं॒पश्य॑ न्त्यमावा॒स्य॑या ऽमावा॒स्य॑या सं॒पश्य॑न्ति सं॒पश्य॑ न्त्यमावा॒स्य॑या । \newline
28. सं॒पश्य॒न्तीति॑ सं - पश्य॑न्ति । \newline
29. अ॒मा॒वा॒स्य॑या॒ मासा॒न् मासा॑ नमावा॒स्य॑या ऽमावा॒स्य॑या॒ मासान्॑ । \newline
30. अ॒मा॒वा॒स्य॑येत्य॑मा - वा॒स्य॑या । \newline
31. मासा᳚न् थ्सं॒पाद्य॑ सं॒पाद्य॒ मासा॒न् मासा᳚न् थ्सं॒पाद्य॑ । \newline
32. सं॒पाद्या ह॒ रहः॑ सं॒पाद्य॑ सं॒पाद्या हः॑ । \newline
33. सं॒पाद्येति॑ सं - पाद्य॑ । \newline
34. अह॒ रुदु दह॒ रह॒ रुत् । \newline
35. उथ् सृ॑जन्ति सृज॒ न्त्युदुथ् सृ॑जन्ति । \newline
36. सृ॒ज॒ न्त्य॒मा॒वा॒स्य॑या ऽमावा॒स्य॑या सृजन्ति सृज न्त्यमावा॒स्य॑या । \newline
37. अ॒मा॒वा॒स्य॑या॒ हि ह्य॑मावा॒स्य॑या ऽमावा॒स्य॑या॒ हि । \newline
38. अ॒मा॒वा॒स्य॑येत्य॑मा - वा॒स्य॑या । \newline
39. हि मासा॒न् मासा॒न्॒. हि हि मासान्॑ । \newline
40. मासा᳚न् थ्सं॒पश्य॑न्ति सं॒पश्य॑न्ति॒ मासा॒न् मासा᳚न् थ्सं॒पश्य॑न्ति । \newline
41. सं॒पश्य॑न्ति पौर्णमा॒स्या पौ᳚र्णमा॒स्या सं॒पश्य॑न्ति सं॒पश्य॑न्ति पौर्णमा॒स्या । \newline
42. सं॒पश्य॒न्तीति॑ सं - पश्य॑न्ति । \newline
43. पौ॒र्ण॒मा॒स्या मासा॒न् मासा᳚न् पौर्णमा॒स्या पौ᳚र्णमा॒स्या मासान्॑ । \newline
44. पौ॒र्ण॒मा॒स्येति॑ पौर्ण - मा॒स्या । \newline
45. मासा᳚न् थ्सं॒पाद्य॑ सं॒पाद्य॒ मासा॒न् मासा᳚न् थ्सं॒पाद्य॑ । \newline
46. सं॒पाद्या ह॒ रहः॑ सं॒पाद्य॑ सं॒पाद्याहः॑ । \newline
47. सं॒पाद्येति॑ सं - पाद्य॑ । \newline
48. अह॒ रुदु दह॒ रह॒ रुत् । \newline
49. उथ् सृ॑जन्ति सृज॒ न्त्युदुथ् सृ॑जन्ति । \newline
50. सृ॒ज॒न्ति॒ पौ॒र्ण॒मा॒स्या पौ᳚र्णमा॒स्या सृ॑जन्ति सृजन्ति पौर्णमा॒स्या । \newline
51. पौ॒र्ण॒मा॒स्या हि हि पौ᳚र्णमा॒स्या पौ᳚र्णमा॒स्या हि । \newline
52. पौ॒र्ण॒मा॒स्येति॑ पौर्ण - मा॒स्या । \newline
53. हि मासा॒न् मासा॒न्॒. हि हि मासान्॑ । \newline
54. मासा᳚न् थ्सं॒पश्य॑न्ति सं॒पश्य॑न्ति॒ मासा॒न् मासा᳚न् थ्सं॒पश्य॑न्ति । \newline
55. सं॒पश्य॑न्ति॒ यो यः सं॒पश्य॑न्ति सं॒पश्य॑न्ति॒ यः । \newline
56. सं॒पश्य॒न्तीति॑ सं - पश्य॑न्ति । \newline
57. यो वै वै यो यो वै । \newline
58. वै पू॒र्णे पू॒र्णे वै वै पू॒र्णे । \newline
59. पू॒र्ण आ॑सि॒ञ्च त्या॑सि॒ञ्चति॑ पू॒र्णे पू॒र्ण आ॑सि॒ञ्चति॑ । \newline
60. आ॒सि॒ञ्चति॒ परा॒ परा॑ ऽऽसि॒ञ्च त्या॑सि॒ञ्चति॒ परा᳚ । \newline
61. आ॒सि॒ञ्चतीत्या᳚ - सि॒ञ्चति॑ । \newline
62. परा॒ स स परा॒ परा॒ सः । \newline
63. स सि॑ञ्चति सिञ्चति॒ स स सि॑ञ्चति । \newline
64. सि॒ञ्च॒ति॒ यो यः सि॑ञ्चति सिञ्चति॒ यः । \newline
65. यः पू॒र्णात् पू॒र्णाद् यो यः पू॒र्णात् । \newline
66. पू॒र्णा दु॒दच॑ त्यु॒दच॑ति पू॒र्णात् पू॒र्णा दु॒दच॑ति । \newline
67. उ॒दच॑ति प्रा॒णम् प्रा॒ण मु॒दच॑ त्यु॒दच॑ति प्रा॒णम् । \newline
68. उ॒दच॒तीत्यु॑त् - अच॑ति । \newline

\textbf{Ghana Paata } \newline

1. ष॒ड॒हैर् मासा॒न् मासा᳚न् षड॒है ष्ष॑ड॒हैर् मासा᳚न् थ्सं॒पाद्य॑ सं॒पाद्य॒ मासा᳚न् षड॒है ष्ष॑ड॒हैर् मासा᳚न् थ्सं॒पाद्य॑ । \newline
2. ष॒ड॒हैरिति॑ षट् - अ॒हैः । \newline
3. मासा᳚न् थ्सं॒पाद्य॑ सं॒पाद्य॒ मासा॒न् मासा᳚न् थ्सं॒पाद्या ह॒ रहः॑ सं॒पाद्य॒ मासा॒न् मासा᳚न् थ्सं॒पाद्याहः॑ । \newline
4. सं॒पाद्या ह॒ रहः॑ सं॒पाद्य॑ सं॒पाद्याह॒ रुदु दहः॑ सं॒पाद्य॑ सं॒पाद्याह॒ रुत् । \newline
5. सं॒पाद्येति॑ सं - पाद्य॑ । \newline
6. अह॒ रुदु दह॒ रह॒ रुथ् सृ॑जन्ति सृज॒ न्त्युदह॒ रह॒ रुथ् सृ॑जन्ति । \newline
7. उथ् सृ॑जन्ति सृज॒ न्त्युदुथ् सृ॑जन्ति षड॒है ष्ष॑ड॒हैः सृ॑ज॒ न्त्युदुथ् सृ॑जन्ति षड॒हैः । \newline
8. सृ॒ज॒न्ति॒ ष॒ड॒है ष्ष॑ड॒हैः सृ॑जन्ति सृजन्ति षड॒हैर्. हि हि ष॑ड॒हैः सृ॑जन्ति सृजन्ति षड॒हैर्. हि । \newline
9. ष॒ड॒हैर्. हि हि ष॑ड॒है ष्ष॑ड॒हैर्. हि मासा॒न् मासा॒न्॒. हि ष॑ड॒है ष्ष॑ड॒हैर्. हि मासान्॑ । \newline
10. ष॒ड॒हैरिति॑ षट् - अ॒हैः । \newline
11. हि मासा॒न् मासा॒न्॒. हि हि मासा᳚न् थ्सं॒पश्य॑न्ति सं॒पश्य॑न्ति॒ मासा॒न्॒. हि हि मासा᳚न् थ्सं॒पश्य॑न्ति । \newline
12. मासा᳚न् थ्सं॒पश्य॑न्ति सं॒पश्य॑न्ति॒ मासा॒न् मासा᳚न् थ्सं॒पश्य॑ न्त्यर्द्धमा॒सै र॑र्द्धमा॒सैः सं॒पश्य॑न्ति॒ मासा॒न् मासा᳚न् थ्सं॒पश्य॑ न्त्यर्द्धमा॒सैः । \newline
13. सं॒पश्य॑ न्त्यर्द्धमा॒सै र॑र्द्धमा॒सैः सं॒पश्य॑न्ति सं॒पश्य॑ न्त्यर्द्धमा॒सैर् मासा॒न् मासा॑ नर्द्धमा॒सैः सं॒पश्य॑न्ति सं॒पश्य॑ न्त्यर्द्धमा॒सैर् मासान्॑ । \newline
14. सं॒पश्य॒न्तीति॑ सं - पश्य॑न्ति । \newline
15. अ॒र्द्ध॒मा॒सैर् मासा॒न् मासा॑ नर्द्धमा॒सै र॑र्द्धमा॒सैर् मासा᳚न् थ्सं॒पाद्य॑ सं॒पाद्य॒ मासा॑ नर्द्धमा॒सै र॑र्द्धमा॒सैर् मासा᳚न् थ्सं॒पाद्य॑ । \newline
16. अ॒र्द्ध॒मा॒सैरित्य॑र्द्ध - मा॒सैः । \newline
17. मासा᳚न् थ्सं॒पाद्य॑ सं॒पाद्य॒ मासा॒न् मासा᳚न् थ्सं॒पाद्या ह॒ रहः॑ सं॒पाद्य॒ मासा॒न् मासा᳚न् थ्सं॒पाद्याहः॑ । \newline
18. सं॒पाद्या ह॒ रहः॑ सं॒पाद्य॑ सं॒पाद्याह॒ रुदु दहः॑ सं॒पाद्य॑ सं॒पाद्याह॒ रुत् । \newline
19. सं॒पाद्येति॑ सं - पाद्य॑ । \newline
20. अह॒ रुदु दह॒ रह॒ रुथ् सृ॑जन्ति सृज॒ न्त्युदह॒ रह॒ रुथ् सृ॑जन्ति । \newline
21. उथ् सृ॑जन्ति सृज॒ न्त्युदुथ् सृ॑ज न्त्यर्द्धमा॒सै र॑र्द्धमा॒सैः सृ॑ज॒ न्त्युदुथ् सृ॑ज न्त्यर्द्धमा॒सैः । \newline
22. सृ॒ज॒ न्त्य॒र्द्ध॒मा॒सै र॑र्द्धमा॒सैः सृ॑जन्ति सृज न्त्यर्द्धमा॒सैर्. हि ह्य॑र्द्धमा॒सैः सृ॑जन्ति सृज न्त्यर्द्धमा॒सैर्. हि । \newline
23. अ॒र्द्ध॒मा॒सैर्. हि ह्य॑र्द्धमा॒सै र॑र्द्धमा॒सैर्. हि मासा॒न् मासा॒न् ह्य॑र्द्धमा॒सै र॑र्द्धमा॒सैर्. हि मासान्॑ । \newline
24. अ॒र्द्ध॒मा॒सैरित्य॑र्द्ध - मा॒सैः । \newline
25. हि मासा॒न् मासा॒न्॒. हि हि मासा᳚न् थ्सं॒पश्य॑न्ति सं॒पश्य॑न्ति॒ मासा॒न्॒. हि हि मासा᳚न् थ्सं॒पश्य॑न्ति । \newline
26. मासा᳚न् थ्सं॒पश्य॑न्ति सं॒पश्य॑न्ति॒ मासा॒न् मासा᳚न् थ्सं॒पश्य॑ न्त्यमावा॒स्य॑या ऽमावा॒स्य॑या सं॒पश्य॑न्ति॒ मासा॒न् मासा᳚न् थ्सं॒पश्य॑ न्त्यमावा॒स्य॑या । \newline
27. सं॒पश्य॑ न्त्यमावा॒स्य॑या ऽमावा॒स्य॑या सं॒पश्य॑न्ति सं॒पश्य॑ न्त्यमावा॒स्य॑या॒ मासा॒न् मासा॑ नमावा॒स्य॑या सं॒पश्य॑न्ति सं॒पश्य॑ न्त्यमावा॒स्य॑या॒ मासान्॑ । \newline
28. सं॒पश्य॒न्तीति॑ सं - पश्य॑न्ति । \newline
29. अ॒मा॒वा॒स्य॑या॒ मासा॒न् मासा॑ नमावा॒स्य॑या ऽमावा॒स्य॑या॒ मासा᳚न् थ्सं॒पाद्य॑ सं॒पाद्य॒ मासा॑ नमावा॒स्य॑या ऽमावा॒स्य॑या॒ मासा᳚न् थ्सं॒पाद्य॑ । \newline
30. अ॒मा॒वा॒स्य॑येत्य॑मा - वा॒स्य॑या । \newline
31. मासा᳚न् थ्सं॒पाद्य॑ सं॒पाद्य॒ मासा॒न् मासा᳚न् थ्सं॒पाद्या ह॒ रहः॑ सं॒पाद्य॒ मासा॒न् मासा᳚न् थ्सं॒पाद्याहः॑ । \newline
32. सं॒पाद्या ह॒ रहः॑ सं॒पाद्य॑ सं॒पाद्याह॒ रुदु दहः॑ सं॒पाद्य॑ सं॒पाद्याह॒ रुत् । \newline
33. सं॒पाद्येति॑ सं - पाद्य॑ । \newline
34. अह॒ रुदु दह॒ रह॒ रुथ् सृ॑जन्ति सृज॒ न्त्युदह॒ रह॒ रुथ् सृ॑जन्ति । \newline
35. उथ् सृ॑जन्ति सृज॒ न्त्युदुथ् सृ॑ज न्त्यमावा॒स्य॑या ऽमावा॒स्य॑या सृज॒न्त्युदुथ् सृ॑ज न्त्यमावा॒स्य॑या । \newline
36. सृ॒ज॒ न्त्य॒मा॒वा॒स्य॑या ऽमावा॒स्य॑या सृजन्ति सृज न्त्यमावा॒स्य॑या॒ हि ह्य॑मावा॒स्य॑या सृजन्ति सृज न्त्यमावा॒स्य॑या॒ हि । \newline
37. अ॒मा॒वा॒स्य॑या॒ हि ह्य॑मावा॒स्य॑या ऽमावा॒स्य॑या॒ हि मासा॒न् मासा॒न् ह्य॑मावा॒स्य॑या ऽमावा॒स्य॑या॒ हि मासान्॑ । \newline
38. अ॒मा॒वा॒स्य॑येत्य॑मा - वा॒स्य॑या । \newline
39. हि मासा॒न् मासा॒न्॒. हि हि मासा᳚न् थ्सं॒पश्य॑न्ति सं॒पश्य॑न्ति॒ मासा॒न्॒. हि हि मासा᳚न् थ्सं॒पश्य॑न्ति । \newline
40. मासा᳚न् थ्सं॒पश्य॑न्ति सं॒पश्य॑न्ति॒ मासा॒न् मासा᳚न् थ्सं॒पश्य॑न्ति पौर्णमा॒स्या पौ᳚र्णमा॒स्या सं॒पश्य॑न्ति॒ मासा॒न् मासा᳚न् थ्सं॒पश्य॑न्ति पौर्णमा॒स्या । \newline
41. सं॒पश्य॑न्ति पौर्णमा॒स्या पौ᳚र्णमा॒स्या सं॒पश्य॑न्ति सं॒पश्य॑न्ति पौर्णमा॒स्या मासा॒न् मासा᳚न् पौर्णमा॒स्या सं॒पश्य॑न्ति सं॒पश्य॑न्ति पौर्णमा॒स्या मासान्॑ । \newline
42. सं॒पश्य॒न्तीति॑ सं - पश्य॑न्ति । \newline
43. पौ॒र्ण॒मा॒स्या मासा॒न् मासा᳚न् पौर्णमा॒स्या पौ᳚र्णमा॒स्या मासा᳚न् थ्सं॒पाद्य॑ सं॒पाद्य॒ मासा᳚न् पौर्णमा॒स्या पौ᳚र्णमा॒स्या मासा᳚न् थ्सं॒पाद्य॑ । \newline
44. पौ॒र्ण॒मा॒स्येति॑ पौर्ण - मा॒स्या । \newline
45. मासा᳚न् थ्सं॒पाद्य॑ सं॒पाद्य॒ मासा॒न् मासा᳚न् थ्सं॒पा द्याह॒ रहः॑ सं॒पाद्य॒ मासा॒न् मासा᳚न् थ्सं॒पा द्याहः॑ । \newline
46. सं॒पा द्याह॒ रहः॑ सं॒पाद्य॑ सं॒पाद्याह॒ रुदु दहः॑ सं॒पाद्य॑ सं॒पाद्याह॒ रुत् । \newline
47. सं॒पाद्येति॑ सं - पाद्य॑ । \newline
48. अह॒ रुदु दह॒ रह॒ रुथ् सृ॑जन्ति सृज॒ न्त्युदह॒ रह॒ रुथ् सृ॑जन्ति । \newline
49. उथ् सृ॑जन्ति सृज॒ न्त्युदुथ् सृ॑जन्ति पौर्णमा॒स्या पौ᳚र्णमा॒स्या सृ॑ज॒ न्त्युदुथ् सृ॑जन्ति पौर्णमा॒स्या । \newline
50. सृ॒ज॒न्ति॒ पौ॒र्ण॒मा॒स्या पौ᳚र्णमा॒स्या सृ॑जन्ति सृजन्ति पौर्णमा॒स्या हि हि पौ᳚र्णमा॒स्या सृ॑जन्ति सृजन्ति पौर्णमा॒स्या हि । \newline
51. पौ॒र्ण॒मा॒स्या हि हि पौ᳚र्णमा॒स्या पौ᳚र्णमा॒स्या हि मासा॒न् मासा॒न्॒. हि पौ᳚र्णमा॒स्या पौ᳚र्णमा॒स्या हि मासान्॑ । \newline
52. पौ॒र्ण॒मा॒स्येति॑ पौर्ण - मा॒स्या । \newline
53. हि मासा॒न् मासा॒न्॒. हि हि मासा᳚न् थ्सं॒पश्य॑न्ति सं॒पश्य॑न्ति॒ मासा॒न्॒. हि हि मासा᳚न् थ्सं॒पश्य॑न्ति । \newline
54. मासा᳚न् थ्सं॒पश्य॑न्ति सं॒पश्य॑न्ति॒ मासा॒न् मासा᳚न् थ्सं॒पश्य॑न्ति॒ यो यः सं॒पश्य॑न्ति॒ मासा॒न् मासा᳚न् थ्सं॒पश्य॑न्ति॒ यः । \newline
55. सं॒पश्य॑न्ति॒ यो यः सं॒पश्य॑न्ति सं॒पश्य॑न्ति॒ यो वै वै यः सं॒पश्य॑न्ति सं॒पश्य॑न्ति॒ यो वै । \newline
56. सं॒पश्य॒न्तीति॑ सं - पश्य॑न्ति । \newline
57. यो वै वै यो यो वै पू॒र्णे पू॒र्णे वै यो यो वै पू॒र्णे । \newline
58. वै पू॒र्णे पू॒र्णे वै वै पू॒र्ण आ॑सि॒ञ्च त्या॑सि॒ञ्चति॑ पू॒र्णे वै वै पू॒र्ण आ॑सि॒ञ्चति॑ । \newline
59. पू॒र्ण आ॑सि॒ञ्च त्या॑सि॒ञ्चति॑ पू॒र्णे पू॒र्ण आ॑सि॒ञ्चति॒ परा॒ परा॑ ऽऽसि॒ञ्चति॑ पू॒र्णे पू॒र्ण आ॑सि॒ञ्चति॒ परा᳚ । \newline
60. आ॒सि॒ञ्चति॒ परा॒ परा॑ ऽऽसि॒ञ्च त्या॑सि॒ञ्चति॒ परा॒ स स परा॑ ऽऽसि॒ञ्च त्या॑सि॒ञ्चति॒ परा॒ सः । \newline
61. आ॒सि॒ञ्चतीत्या᳚ - सि॒ञ्चति॑ । \newline
62. परा॒ स स परा॒ परा॒ स सि॑ञ्चति सिञ्चति॒ स परा॒ परा॒ स सि॑ञ्चति । \newline
63. स सि॑ञ्चति सिञ्चति॒ स स सि॑ञ्चति॒ यो यः सि॑ञ्चति॒ स स सि॑ञ्चति॒ यः । \newline
64. सि॒ञ्च॒ति॒ यो यः सि॑ञ्चति सिञ्चति॒ यः पू॒र्णात् पू॒र्णाद् यः सि॑ञ्चति सिञ्चति॒ यः पू॒र्णात् । \newline
65. यः पू॒र्णात् पू॒र्णाद् यो यः पू॒र्णा दु॒दच॑ त्यु॒दच॑ति पू॒र्णाद् यो यः पू॒र्णा दु॒दच॑ति । \newline
66. पू॒र्णा दु॒दच॑ त्यु॒दच॑ति पू॒र्णात् पू॒र्णा दु॒दच॑ति प्रा॒णम् प्रा॒ण मु॒दच॑ति पू॒र्णात् पू॒र्णा दु॒दच॑ति प्रा॒णम् । \newline
67. उ॒दच॑ति प्रा॒णम् प्रा॒ण मु॒दच॑ त्यु॒दच॑ति प्रा॒ण म॑स्मिन् नस्मिन् प्रा॒ण मु॒दच॑ त्यु॒दच॑ति प्रा॒ण म॑स्मिन्न् । \newline
68. उ॒दच॒तीत्यु॑त् - अच॑ति । \newline
\pagebreak
\markright{ TS 7.5.6.2  \hfill https://www.vedavms.in \hfill}

\section{ TS 7.5.6.2 }

\textbf{TS 7.5.6.2 } \newline
\textbf{Samhita Paata} \newline

प्रा॒णम॑स्मि॒न्थ्स द॑धाति॒ यत् पौ᳚र्णमा॒स्या मासा᳚न्थ् स॒पांद्याह॑रुथ् सृ॒जन्ति॑ संॅवथ्स॒रायै॒व तत् प्रा॒णं द॑धति॒ तदनु॑ स॒त्रिणः॒ प्राण॑न्ति॒ यदह॒र्नोथ्-सृ॒जेयु॒र्यथा॒ दृति॒रुप॑नद्धो वि॒पत॑त्ये॒वꣳ सं॑ॅवथ्स॒रो वि प॑ते॒दार्ति॒-मार्च्छे॑यु॒र्यत् पौ᳚र्णमा॒स्या मासा᳚न्थ्-स॒पांद्याह॑रुथ् सृ॒जन्ति॑ संॅवथ्स॒रायै॒व तदु॑दा॒नं द॑धति॒ तदनु॑ स॒त्रिण॒ उ - [  ] \newline

\textbf{Pada Paata} \newline

प्रा॒णमिति॑ प्र - अ॒नम् । अ॒स्मि॒न्न् । सः । द॒धा॒ति॒ । यत् । पौ॒र्ण॒मा॒स्येति॑ पौर्ण - मा॒स्या । मासान्॑ । स॒पांद्येति॑ सं - पाद्य॑ । अहः॑ । उ॒थ्सृ॒जन्तीत्यु॑त् - सृ॒जन्ति॑ । सं॒ॅव॒थ्स॒रायेति॑ सं-व॒थ्स॒राय॑ । ए॒व । तत् । प्रा॒णमिति॑ प्र - अ॒नम् । द॒ध॒ति॒ । तत् । अन्विति॑ । स॒त्रिणः॑ । प्रेति॑ । अ॒न॒न्ति॒ । यत् । अहः॑ । न । उ॒थ्सृ॒जेयु॒रित्यु॑त्-सृ॒जेयुः॑ । यथा᳚ । दृतिः॑ । उप॑नद्ध॒ इत्युप॑ - न॒द्धः॒ । वि॒पत॒तीति॑ वि - पत॑ति । ए॒वम् । सं॒ॅव॒थ्स॒र इति॑ सं - व॒थ्स॒रः । वीति॑ । प॒ते॒त् । आर्ति᳚म् । एति॑ । ऋ॒च्छे॒युः॒ । यत् । पौ॒र्ण॒मा॒स्येति॑ पौर्ण - मा॒स्या । मासान्॑ । स॒पांद्येति॑ सं - पाद्य॑ । अहः॑ । उ॒थ्सृ॒जन्तीत्यु॑त् - सृ॒जन्ति॑ । सं॒ॅव॒थ्स॒रायेति॑ सं - व॒थ्स॒राय॑ । ए॒व । तत् । उ॒दा॒नमित्यु॑त् - अ॒नम् । द॒ध॒ति॒ । तत् । अन्विति॑ । स॒त्रिणः॑ । उदिति॑ ।  \newline


\textbf{Krama Paata} \newline

प्रा॒णम॑स्मिन्न् । प्रा॒णमिति॑ प्र - अ॒नम् । अ॒स्मि॒न्थ् सः । स द॑धाति । द॒धा॒ति॒ यत् । यत् पौ᳚र्णमा॒स्या । पौ॒र्ण॒मा॒स्या मासान्॑ । पौ॒र्ण॒मा॒स्येति॑ पौर्ण - मा॒स्या । मासा᳚न्थ् स॒म्पाद्य॑ । स॒म्पाद्याहः॑ । स॒म्पाद्येति॑ सम् - पाद्य॑ । अह॑रुथ्सृ॒जन्ति॑ । उ॒थ्सृ॒जन्ति॑ सम्ॅवथ्स॒राय॑ । उ॒थ्सृ॒जन्तीत्यु॑त् - सृ॒जन्ति॑ । स॒म्ॅव॒थ्स॒रायै॒व । स॒म्ॅव॒थ्स॒रायेति॑ सम् - व॒थ्स॒राय॑ । ए॒व तत् । तत् प्रा॒णम् । प्रा॒णम् द॑धति । प्रा॒णमिति॑ प्र - अ॒नम् । द॒ध॒ति॒ तत् । तदनु॑ । अनु॑ स॒त्रिणः॑ । स॒त्रिणः॒ प्र । प्राण॑न्ति । अ॒न॒न्ति॒ यत् । यदहः॑ । अह॒र् न । नोथ्सृ॒जेयुः॑ । उ॒थ्सृ॒जेयु॒र् यथा᳚ । उ॒थ्सृ॒जेयु॒रित्यु॑त् - सृ॒जेयुः॑ । यथा॒ दृतिः॑ । दृति॒रुप॑नद्धः । उप॑नद्धो वि॒पत॑ति । उप॑नद्ध॒ इत्युप॑ - न॒द्धः॒ । वि॒पत॑त्ये॒वम् । वि॒पत॒तीति॑ वि - पत॑ति । ए॒वꣳ स॑म्ॅवथ्स॒रः । स॒म्ॅव॒थ्स॒रो वि । स॒म्ॅव॒थ्स॒र इति॑ सम् - व॒थ्स॒रः । वि प॑तेत् । प॒ते॒दार्ति᳚म् । आर्ति॒मा । आर्छे॑युः । ऋ॒च्छे॒यु॒र् यत् । यत् पौ᳚र्णमा॒स्या । पौ॒र्ण॒मा॒स्या मासान्॑ । पौ॒र्ण॒मा॒स्येति॑ पौर्ण - मा॒स्या । मासा᳚न्थ् स॒म्पाद्य॑ । स॒म्पाद्याहः॑ । स॒म्पाद्येति॑ सम् - पाद्य॑ । अह॑रुथ्सृ॒जन्ति॑ । उ॒थ्सृ॒जन्ति॑ सम्ॅवथ्स॒राय॑ । उ॒थ्सृ॒जन्तीत्यु॑त् - सृ॒जन्ति॑ । स॒म्ॅव॒थ्स॒रायै॒व । स॒म्ॅव॒थ्स॒रायेति॑ सम् - व॒थ्स॒राय॑ । ए॒व तत् । तदु॑दा॒नम् । उ॒दा॒नम् द॑धति । उ॒दा॒नमित्यु॑त् - अ॒नम् । द॒ध॒ति॒ तत् । तदनु॑ । अनु॑ स॒त्रिणः॑ । स॒त्रिण॒ उत् । उद॑नन्ति \newline

\textbf{Jatai Paata} \newline

1. प्रा॒ण म॑स्मिन् नस्मिन् प्रा॒णम् प्रा॒ण म॑स्मिन्न् । \newline
2. प्रा॒णमिति॑ प्र - अ॒नम् । \newline
3. अ॒स्मि॒न् थ्स सो᳚ ऽस्मिन् नस्मि॒न् थ्सः । \newline
4. स द॑धाति दधाति॒ स स द॑धाति । \newline
5. द॒धा॒ति॒ यद् यद् द॑धाति दधाति॒ यत् । \newline
6. यत् पौ᳚र्णमा॒स्या पौ᳚र्णमा॒स्या यद् यत् पौ᳚र्णमा॒स्या । \newline
7. पौ॒र्ण॒मा॒स्या मासा॒न् मासा᳚न् पौर्णमा॒स्या पौ᳚र्णमा॒स्या मासान्॑ । \newline
8. पौ॒र्ण॒मा॒स्येति॑ पौर्ण - मा॒स्या । \newline
9. मासा᳚न् थ्सं॒पाद्य॑ सं॒पाद्य॒ मासा॒न् मासा᳚न् थ्सं॒पाद्य॑ । \newline
10. सं॒पाद्या ह॒ रहः॑ सं॒पाद्य॑ सं॒पाद्याहः॑ । \newline
11. सं॒पाद्येति॑ सं - पाद्य॑ । \newline
12. अह॑ रुथ्सृ॒ज न्त्यु॑थ्सृ॒ज न्त्यह॒ रह॑ रुथ्सृ॒जन्ति॑ । \newline
13. उ॒थ्सृ॒जन्ति॑ संॅवथ्स॒राय॑ संॅवथ्स॒रा यो᳚थ्सृ॒ज न्त्यु॑थ्सृ॒जन्ति॑ संॅवथ्स॒राय॑ । \newline
14. उ॒थ्सृ॒जन्तीत्यु॑त् - सृ॒जन्ति॑ । \newline
15. सं॒ॅव॒थ्स॒रा यै॒वैव सं॑ॅवथ्स॒राय॑ संॅवथ्स॒रा यै॒व । \newline
16. सं॒ॅव॒थ्स॒रायेति॑ सं - व॒थ्स॒राय॑ । \newline
17. ए॒व तत् तदे॒वैव तत् । \newline
18. तत् प्रा॒णम् प्रा॒णम् तत् तत् प्रा॒णम् । \newline
19. प्रा॒णम् द॑धति दधति प्रा॒णम् प्रा॒णम् द॑धति । \newline
20. प्रा॒णमिति॑ प्र - अ॒नम् । \newline
21. द॒ध॒ति॒ तत् तद् द॑धति दधति॒ तत् । \newline
22. तदन् वनु॒ तत् तदनु॑ । \newline
23. अनु॑ स॒त्रिणः॑ स॒त्रिणो ऽन्वनु॑ स॒त्रिणः॑ । \newline
24. स॒त्रिणः॒ प्र प्र स॒त्रिणः॑ स॒त्रिणः॒ प्र । \newline
25. प्राण॑ न्त्यनन्ति॒ प्र प्राण॑न्ति । \newline
26. अ॒न॒न्ति॒ यद् यद॑न न्त्यनन्ति॒ यत् । \newline
27. यदह॒ रह॒र् यद् यदहः॑ । \newline
28. अह॒र् न नाह॒ रह॒र् न । \newline
29. नोथ्सृ॒जेयु॑ रुथ्सृ॒जेयु॒र् न नोथ्सृ॒जेयुः॑ । \newline
30. उ॒थ्सृ॒जेयु॒र् यथा॒ यथो᳚थ्सृ॒जेयु॑ रुथ्सृ॒जेयु॒र् यथा᳚ । \newline
31. उ॒थ्सृ॒जेयु॒रित्यु॑त् - सृ॒जेयुः॑ । \newline
32. यथा॒ दृति॒र् दृति॒र् यथा॒ यथा॒ दृतिः॑ । \newline
33. दृति॒ रुप॑नद्ध॒ उप॑नद्धो॒ दृति॒र् दृति॒ रुप॑नद्धः । \newline
34. उप॑नद्धो वि॒पत॑ति वि॒पत॒ त्युप॑नद्ध॒ उप॑नद्धो वि॒पत॑ति । \newline
35. उप॑नद्ध॒ इत्युप॑ - न॒द्धः॒ । \newline
36. वि॒पत॑ त्ये॒व मे॒वं ॅवि॒पत॑ति वि॒पत॑ त्ये॒वम् । \newline
37. वि॒पत॒तीति॑ वि - पत॑ति । \newline
38. ए॒वꣳ सं॑ॅवथ्स॒रः सं॑ॅवथ्स॒र ए॒व मे॒वꣳ सं॑ॅवथ्स॒रः । \newline
39. सं॒ॅव॒थ्स॒रो वि वि सं॑ॅवथ्स॒रः सं॑ॅवथ्स॒रो वि । \newline
40. सं॒ॅव॒थ्स॒र इति॑ सं - व॒थ्स॒रः । \newline
41. वि प॑तेत् पते॒द् वि वि प॑तेत् । \newline
42. प॒ते॒ दार्ति॒ मार्ति॑म् पतेत् पते॒ दार्ति᳚म् । \newline
43. आर्ति॒ मा ऽऽर्ति॒ मार्ति॒ मा । \newline
44. आर्च्छे॑युर्. ऋच्छेयु॒ रार्च्छे॑युः । \newline
45. ऋ॒च्छे॒यु॒र् यद् यदृ॑च्छेयुर्. ऋच्छेयु॒र् यत् । \newline
46. यत् पौ᳚र्णमा॒स्या पौ᳚र्णमा॒स्या यद् यत् पौ᳚र्णमा॒स्या । \newline
47. पौ॒र्ण॒मा॒स्या मासा॒न् मासा᳚न् पौर्णमा॒स्या पौ᳚र्णमा॒स्या मासान्॑ । \newline
48. पौ॒र्ण॒मा॒स्येति॑ पौर्ण - मा॒स्या । \newline
49. मासा᳚न् थ्सं॒पाद्य॑ सं॒पाद्य॒ मासा॒न् मासा᳚न् थ्सं॒पाद्य॑ । \newline
50. सं॒पाद्या ह॒ रहः॑ सं॒पाद्य॑ सं॒पाद्याहः॑ । \newline
51. सं॒पाद्येति॑ सं - पाद्य॑ । \newline
52. अह॑ रुथ्सृ॒ज न्त्यु॑थ्सृ॒ज न्त्यह॒ रह॑ रुथ्सृ॒जन्ति॑ । \newline
53. उ॒थ्सृ॒जन्ति॑ संॅवथ्स॒राय॑ संॅवथ्स॒रा यो᳚थ्सृ॒ज न्त्यु॑थ्सृ॒जन्ति॑ संॅवथ्स॒राय॑ । \newline
54. उ॒थ्सृ॒जन्तीत्यु॑त् - सृ॒जन्ति॑ । \newline
55. सं॒ॅव॒थ्स॒रा यै॒वैव सं॑ॅवथ्स॒राय॑ संॅवथ्स॒रा यै॒व । \newline
56. सं॒ॅव॒थ्स॒रायेति॑ सं - व॒थ्स॒राय॑ । \newline
57. ए॒व तत् तदे॒वैव तत् । \newline
58. तदु॑दा॒न मु॑दा॒नम् तत् तदु॑दा॒नम् । \newline
59. उ॒दा॒नम् द॑धति दध त्युदा॒न मु॑दा॒नम् द॑धति । \newline
60. उ॒दा॒नमित्यु॑त् - अ॒नम् । \newline
61. द॒ध॒ति॒ तत् तद् द॑धति दधति॒ तत् । \newline
62. तदन् वनु॒ तत् तदनु॑ । \newline
63. अनु॑ स॒त्रिणः॑ स॒त्रिणो ऽन्वनु॑ स॒त्रिणः॑ । \newline
64. स॒त्रिण॒ उदुथ् स॒त्रिणः॑ स॒त्रिण॒ उत् । \newline
65. उद॑न न्त्यन॒ न्त्युदु द॑नन्ति । \newline

\textbf{Ghana Paata } \newline

1. प्रा॒ण म॑स्मिन् नस्मिन् प्रा॒णम् प्रा॒ण म॑स्मि॒न् थ्स सो᳚ ऽस्मिन् प्रा॒णम् प्रा॒ण म॑स्मि॒न् थ्सः । \newline
2. प्रा॒णमिति॑ प्र - अ॒नम् । \newline
3. अ॒स्मि॒न् थ्स सो᳚ ऽस्मिन् नस्मि॒न् थ्स द॑धाति दधाति॒ सो᳚ ऽस्मिन् नस्मि॒न् थ्स द॑धाति । \newline
4. स द॑धाति दधाति॒ स स द॑धाति॒ यद् यद् द॑धाति॒ स स द॑धाति॒ यत् । \newline
5. द॒धा॒ति॒ यद् यद् द॑धाति दधाति॒ यत् पौ᳚र्णमा॒स्या पौ᳚र्णमा॒स्या यद् द॑धाति दधाति॒ यत् पौ᳚र्णमा॒स्या । \newline
6. यत् पौ᳚र्णमा॒स्या पौ᳚र्णमा॒स्या यद् यत् पौ᳚र्णमा॒स्या मासा॒न् मासा᳚न् पौर्णमा॒स्या यद् यत् पौ᳚र्णमा॒स्या मासान्॑ । \newline
7. पौ॒र्ण॒मा॒स्या मासा॒न् मासा᳚न् पौर्णमा॒स्या पौ᳚र्णमा॒स्या मासा᳚न् थ्सं॒पाद्य॑ सं॒पाद्य॒ मासा᳚न् पौर्णमा॒स्या पौ᳚र्णमा॒स्या मासा᳚न् थ्सं॒पाद्य॑ । \newline
8. पौ॒र्ण॒मा॒स्येति॑ पौर्ण - मा॒स्या । \newline
9. मासा᳚न् थ्सं॒पाद्य॑ सं॒पाद्य॒ मासा॒न् मासा᳚न् थ्सं॒पाद्याह॒ रहः॑ सं॒पाद्य॒ मासा॒न् मासा᳚न् थ्सं॒पाद्याहः॑ । \newline
10. सं॒पाद्याह॒ रहः॑ सं॒पाद्य॑ सं॒पाद्याह॑ रुथ्सृ॒ज न्त्यु॑थ्सृ॒ज न्त्यहः॑ सं॒पाद्य॑ सं॒पाद्याह॑ रुथ्सृ॒जन्ति॑ । \newline
11. सं॒पाद्येति॑ सं - पाद्य॑ । \newline
12. अह॑ रुथ्सृ॒ज न्त्यु॑थ्सृ॒ज न्त्यह॒ रह॑ रुथ्सृ॒जन्ति॑ संॅवथ्स॒राय॑ संॅवथ्स॒रा यो᳚थ्सृ॒ज न्त्यह॒ रह॑ रुथ्सृ॒जन्ति॑ संॅवथ्स॒राय॑ । \newline
13. उ॒थ्सृ॒जन्ति॑ संॅवथ्स॒राय॑ संॅवथ्स॒रा यो᳚थ्सृ॒ज न्त्यु॑थ्सृ॒जन्ति॑ संॅवथ्स॒रा यै॒वैव सं॑ॅवथ्स॒रा यो᳚थ्सृ॒ज न्त्यु॑थ्सृ॒जन्ति॑ संॅवथ्स॒रा यै॒व । \newline
14. उ॒थ्सृ॒जन्तीत्यु॑त् - सृ॒जन्ति॑ । \newline
15. सं॒ॅव॒थ्स॒रा यै॒वैव सं॑ॅवथ्स॒राय॑ संॅवथ्स॒रा यै॒व तत् तदे॒व सं॑ॅवथ्स॒राय॑ संॅवथ्स॒रा यै॒व तत् । \newline
16. सं॒ॅव॒थ्स॒रायेति॑ सं - व॒थ्स॒राय॑ । \newline
17. ए॒व तत् तदे॒ वैव तत् प्रा॒णम् प्रा॒णम् तदे॒ वैव तत् प्रा॒णम् । \newline
18. तत् प्रा॒णम् प्रा॒णम् तत् तत् प्रा॒णम् द॑धति दधति प्रा॒णम् तत् तत् प्रा॒णम् द॑धति । \newline
19. प्रा॒णम् द॑धति दधति प्रा॒णम् प्रा॒णम् द॑धति॒ तत् तद् द॑धति प्रा॒णम् प्रा॒णम् द॑धति॒ तत् । \newline
20. प्रा॒णमिति॑ प्र - अ॒नम् । \newline
21. द॒ध॒ति॒ तत् तद् द॑धति दधति॒ तदन् वनु॒ तद् द॑धति दधति॒ तदनु॑ । \newline
22. तदन् वनु॒ तत् तदनु॑ स॒त्रिणः॑ स॒त्रिणो ऽनु॒ तत् तदनु॑ स॒त्रिणः॑ । \newline
23. अनु॑ स॒त्रिणः॑ स॒त्रिणो ऽन्वनु॑ स॒त्रिणः॒ प्र प्र स॒त्रिणो ऽन्वनु॑ स॒त्रिणः॒ प्र । \newline
24. स॒त्रिणः॒ प्र प्र स॒त्रिणः॑ स॒त्रिणः॒ प्राण॑ न्त्यनन्ति॒ प्र स॒त्रिणः॑ स॒त्रिणः॒ प्राण॑न्ति । \newline
25. प्राण॑ न्त्यनन्ति॒ प्र प्राण॑न्ति॒ यद् यद॑नन्ति॒ प्र प्राण॑न्ति॒ यत् । \newline
26. अ॒न॒न्ति॒ यद् यद॑न न्त्यनन्ति॒ यदह॒ रह॒र् यद॑न न्त्यनन्ति॒ यदहः॑ । \newline
27. यदह॒ रह॒र् यद् यदह॒र् न नाह॒र् यद् यदह॒र् न । \newline
28. अह॒र् न नाह॒ रह॒र् नोथ्सृ॒जेयु॑ रुथ्सृ॒जेयु॒र् नाह॒ रह॒र् नोथ्सृ॒जेयुः॑ । \newline
29. नोथ्सृ॒जेयु॑ रुथ्सृ॒जेयु॒र् न नोथ्सृ॒जेयु॒र् यथा॒ यथो᳚ थ्सृ॒जेयु॒र् न नोथ् सृ॒जेयु॒र् यथा᳚ । \newline
30. उ॒थ्सृ॒जेयु॒र् यथा॒ यथो᳚थ्सृ॒जेयु॑ रुथ्सृ॒जेयु॒र् यथा॒ दृति॒र् दृति॒र् यथो᳚थ्सृ॒जेयु॑ रुथ्सृ॒जेयु॒र् यथा॒ दृतिः॑ । \newline
31. उ॒थ्सृ॒जेयु॒रित्यु॑त् - सृ॒जेयुः॑ । \newline
32. यथा॒ दृति॒र् दृति॒र् यथा॒ यथा॒ दृति॒ रुप॑नद्ध॒ उप॑नद्धो॒ दृति॒र् यथा॒ यथा॒ दृति॒ रुप॑नद्धः । \newline
33. दृति॒ रुप॑नद्ध॒ उप॑नद्धो॒ दृति॒र् दृति॒ रुप॑नद्धो वि॒पत॑ति वि॒पत॒ त्युप॑नद्धो॒ दृति॒र् दृति॒ रुप॑नद्धो वि॒पत॑ति । \newline
34. उप॑नद्धो वि॒पत॑ति वि॒पत॒ त्युप॑नद्ध॒ उप॑नद्धो वि॒पत॑ त्ये॒व मे॒वं ॅवि॒पत॒ त्युप॑नद्ध॒ उप॑नद्धो वि॒पत॑ त्ये॒वम् । \newline
35. उप॑नद्ध॒ इत्युप॑ - न॒द्धः॒ । \newline
36. वि॒पत॑ त्ये॒व मे॒वं ॅवि॒पत॑ति वि॒पत॑ त्ये॒वꣳ सं॑ॅवथ्स॒रः सं॑ॅवथ्स॒र ए॒वं ॅवि॒पत॑ति वि॒पत॑ त्ये॒वꣳ सं॑ॅवथ्स॒रः । \newline
37. वि॒पत॒तीति॑ वि - पत॑ति । \newline
38. ए॒वꣳ सं॑ॅवथ्स॒रः सं॑ॅवथ्स॒र ए॒व मे॒वꣳ सं॑ॅवथ्स॒रो वि वि सं॑ॅवथ्स॒र ए॒व मे॒वꣳ सं॑ॅवथ्स॒रो वि । \newline
39. सं॒ॅव॒थ्स॒रो वि वि सं॑ॅवथ्स॒रः सं॑ॅवथ्स॒रो वि प॑तेत् पते॒द् वि सं॑ॅवथ्स॒रः सं॑ॅवथ्स॒रो वि प॑तेत् । \newline
40. सं॒ॅव॒थ्स॒र इति॑ सं - व॒थ्स॒रः । \newline
41. वि प॑तेत् पते॒द् वि वि प॑ते॒ दार्ति॒ मार्ति॑म् पते॒द् वि वि प॑ते॒ दार्ति᳚म् । \newline
42. प॒ते॒ दार्ति॒ मार्ति॑म् पतेत् पते॒ दार्ति॒ मा ऽऽर्ति॑म् पतेत् पते॒ दार्ति॒ मा । \newline
43. आर्ति॒ मा ऽऽर्ति॒ मार्ति॒ मा र्‌च्छे॑युर्. ऋच्छेयु॒रा ऽऽर्ति॒ मार्ति॒ मा र्‌च्छे॑युः । \newline
44. आ र्‌च्छे॑युर्. ऋच्छेयु॒ रा र्‌च्छे॑यु॒र् यद् यदृ॑च्छेयु॒ रा र्‌च्छे॑यु॒र् यत् । \newline
45. ऋ॒च्छे॒यु॒र् यद् यदृ॑च्छेयुर्. ऋच्छेयु॒र् यत् पौ᳚र्णमा॒स्या पौ᳚र्णमा॒स्या यदृ॑च्छेयुर्. ऋच्छेयु॒र् यत् पौ᳚र्णमा॒स्या । \newline
46. यत् पौ᳚र्णमा॒स्या पौ᳚र्णमा॒स्या यद् यत् पौ᳚र्णमा॒स्या मासा॒न् मासा᳚न् पौर्णमा॒स्या यद् यत् पौ᳚र्णमा॒स्या मासान्॑ । \newline
47. पौ॒र्ण॒मा॒स्या मासा॒न् मासा᳚न् पौर्णमा॒स्या पौ᳚र्णमा॒स्या मासा᳚न् थ्सं॒पाद्य॑ सं॒पाद्य॒ मासा᳚न् पौर्णमा॒स्या पौ᳚र्णमा॒स्या मासा᳚न् थ्सं॒पाद्य॑ । \newline
48. पौ॒र्ण॒मा॒स्येति॑ पौर्ण - मा॒स्या । \newline
49. मासा᳚न् थ्सं॒पाद्य॑ सं॒पाद्य॒ मासा॒न् मासा᳚न् थ्सं॒पाद्याह॒ रहः॑ सं॒पाद्य॒ मासा॒न् मासा᳚न् थ्सं॒पाद्याहः॑ । \newline
50. सं॒पाद्याह॒ रहः॑ सं॒पाद्य॑ सं॒पाद्याह॑ रुथ्सृ॒ज न्त्यु॑थ्सृ॒ज न्त्यहः॑ सं॒पाद्य॑ सं॒पाद्याह॑ रुथ्सृ॒जन्ति॑ । \newline
51. सं॒पाद्येति॑ सं - पाद्य॑ । \newline
52. अह॑ रुथ्सृ॒ज न्त्यु॑थ्सृ॒ज न्त्यह॒ रह॑ रुथ्सृ॒जन्ति॑ संॅवथ्स॒राय॑ संॅवथ्स॒रा यो᳚थ्सृ॒ज न्त्यह॒ रह॑ रुथ्सृ॒जन्ति॑ संॅवथ्स॒राय॑ । \newline
53. उ॒थ्सृ॒जन्ति॑ संॅवथ्स॒राय॑ संॅवथ्स॒रा यो᳚थ्सृ॒ज न्त्यु॑थ्सृ॒जन्ति॑ संॅवथ्स॒रा यै॒वैव सं॑ॅवथ्स॒रा यो᳚थ्सृ॒ज न्त्यु॑थ्सृ॒जन्ति॑ संॅवथ्स॒रा यै॒व । \newline
54. उ॒थ्सृ॒जन्तीत्यु॑त् - सृ॒जन्ति॑ । \newline
55. सं॒ॅव॒थ्स॒रायै॒ वैव सं॑ॅवथ्स॒राय॑ संॅवथ्स॒रा यै॒व तत् तदे॒व सं॑ॅवथ्स॒राय॑ संॅवथ्स॒रा यै॒व तत् । \newline
56. सं॒ॅव॒थ्स॒रायेति॑ सं - व॒थ्स॒राय॑ । \newline
57. ए॒व तत् तदे॒ वैव तदु॑दा॒न मु॑दा॒नम् तदे॒ वैव तदु॑दा॒नम् । \newline
58. तदु॑दा॒न मु॑दा॒नम् तत् तदु॑दा॒नम् द॑धति दध त्युदा॒नम् तत् तदु॑दा॒नम् द॑धति । \newline
59. उ॒दा॒नम् द॑धति दध त्युदा॒न मु॑दा॒नम् द॑धति॒ तत् तद् द॑ध त्युदा॒न मु॑दा॒नम् द॑धति॒ तत् । \newline
60. उ॒दा॒नमित्यु॑त् - अ॒नम् । \newline
61. द॒ध॒ति॒ तत् तद् द॑धति दधति॒ तदन् वनु॒ तद् द॑धति दधति॒ तदनु॑ । \newline
62. तदन् वनु॒ तत् तदनु॑ स॒त्रिणः॑ स॒त्रिणो ऽनु॒ तत् तदनु॑ स॒त्रिणः॑ । \newline
63. अनु॑ स॒त्रिणः॑ स॒त्रिणो ऽन्वनु॑ स॒त्रिण॒ उदुथ् स॒त्रिणो ऽन्वनु॑ स॒त्रिण॒ उत् । \newline
64. स॒त्रिण॒ उदुथ् स॒त्रिणः॑ स॒त्रिण॒ उद॑न न्त्यन॒ न्त्युथ् स॒त्रिणः॑ स॒त्रिण॒ उद॑नन्ति । \newline
65. उद॑न न्त्यन॒ न्त्युदु द॑नन्ति॒ न नान॒ न्त्युदु द॑नन्ति॒ न । \newline
\pagebreak
\markright{ TS 7.5.6.3  \hfill https://www.vedavms.in \hfill}

\section{ TS 7.5.6.3 }

\textbf{TS 7.5.6.3 } \newline
\textbf{Samhita Paata} \newline

द॑नन्ति॒ नाऽऽ*र्ति॒मार्च्छ॑न्ति पू॒र्णमा॑से॒ वै दे॒वानाꣳ॑ सु॒तो यत् पौ᳚र्णमा॒स्या मासा᳚न्थ्-स॒पांद्याह॑रुथ् सृ॒जन्ति॑ दे॒वाना॑मे॒व तद्-य॒ज्ञेन॑ य॒ज्ञ्ं प्र॒त्यव॑रोहन्ति॒ वि वा ए॒तद्-य॒ज्ञ्ं छि॑न्दन्ति॒ यथ् ष॑ड॒हस॑तंतꣳ॒॒ संत॒मथाह॑रुथ् सृ॒जन्ति॑ प्राजाप॒त्यं प॒शुमा ल॑भन्ते प्र॒जाप॑तिः॒ सर्वा॑ दे॒वता॑ दे॒वता॑भिरे॒व य॒ज्ञ्ꣳ सं त॑न्वन्ति॒ यन्ति॒ वा ए॒ते सव॑ना॒द्येऽह॑ -[  ] \newline

\textbf{Pada Paata} \newline

अ॒न॒न्ति॒ । न । आर्ति᳚म् । एति॑ । ऋ॒च्छ॒न्ति॒ । पू॒र्णमा॑स॒ इति॑ पू॒र्ण - मा॒से॒ । वै । दे॒वाना᳚म् । सु॒तः । यत् । पौ॒र्ण॒मा॒स्येति॑ पौर्ण - मा॒स्या । मासान्॑ । स॒पांद्येति॑ सं - पाद्य॑ । अहः॑ । उ॒थ्सृ॒जन्तीत्य॑त् - सृ॒जन्ति॑ । दे॒वाना᳚म् । ए॒व । तत् । य॒ज्ञेन॑ । य॒ज्ञ्म् । प्र॒त्यव॑रोह॒न्तीति॑ प्रति - अव॑रोहन्ति । वीति॑ । वै । ए॒तत् । य॒ज्ञ्म् । छि॒न्द॒न्ति॒ । यत् । ष॒ड॒हस॑न्तत॒मिति॑ षड॒ह - स॒न्त॒त॒म् । सन्त᳚म् । अथ॑ । अहः॑ । उ॒थ्सृ॒जन्तीत्यु॑त् - सृ॒जन्ति॑ । प्रा॒जा॒प॒त्यमिति॑ प्राजा-प॒त्यम् । प॒शुम् । एति॑ । ल॒भ॒न्ते॒ । प्र॒जाप॑ति॒रिति॑ प्र॒जा-प॒तिः॒ । सर्वाः᳚ । दे॒वताः᳚ । दे॒वता॑भिः । ए॒व । य॒ज्ञ्म् । समिति॑ । त॒न्व॒न्ति॒ । यन्ति॑ । वै । ए॒ते । सव॑नात् । ये । अहः॑ ।  \newline


\textbf{Krama Paata} \newline

अ॒न॒न्ति॒ न । नार्ति᳚म् । आर्ति॒मा । आर्च्छ॑न्ति । ऋ॒च्छ॒न्ति॒ पू॒र्णमा॑से । पू॒र्णमा॑से॒ वै । पू॒र्णमा॑स॒ इति॑ पू॒र्ण - मा॒से॒ । वै दे॒वाना᳚म् । दे॒वानाꣳ॑ सु॒तः । सु॒तो यत् । यत् पौ᳚र्णमा॒स्या । पौ॒र्ण॒मा॒स्या मासान्॑ । पौ॒र्ण॒मा॒स्येति॑ पौर्ण - मा॒स्या । मासा᳚न्थ् स॒म्पाद्य॑ । स॒म्पाद्याहः॑ । स॒म्पाद्येति॑ सम् - पाद्य॑ । अह॑रुथ्सृ॒जन्ति॑ । उ॒थ्सृ॒जन्ति॑ दे॒वाना᳚म् । उ॒थ्सृ॒जन्तीत्यु॑त् - सृ॒जन्ति॑ । दे॒वाना॑मे॒व । ए॒व तत् । तद् य॒ज्ञेन॑ । य॒ज्ञेन॑ य॒ज्ञ्म् । य॒ज्ञ्म् प्र॒त्यव॑रोहन्ति । प्र॒त्यव॑रोहन्ति॒ वि । प्र॒त्यव॑रोह॒न्तीति॑ प्रति - अव॑रोहन्ति । वि वै । वा ए॒तत् । ए॒तद् य॒ज्ञ्म् । य॒ज्ञ्म् छि॑न्दन्ति । छि॒न्द॒न्ति॒ यत् । यथ् ष॑ड॒हस॑न्ततम् । ष॒ड॒हस॑न्ततꣳ॒॒ सन्त᳚म् । ष॒ड॒हस॑न्तत॒मिति॑ षड॒ह - स॒न्त॒त॒म् । सन्त॒मथ॑ । अथाहः॑ । अह॑रुथ्सृ॒जन्ति॑ । उ॒थ्सृ॒जन्ति॑ प्राजाप॒त्यम् । उ॒थ्सृ॒जन्तीत्यु॑त् - सृ॒जन्ति॑ । प्रा॒जा॒प॒त्यम् प॒शुम् । प्रा॒जा॒प॒त्यमिति॑ प्राजा - प॒त्यम् । प॒शुमा । आ ल॑भन्ते । ल॒भ॒न्ते॒ प्र॒जाप॑तिः । प्र॒जाप॑तिः॒ सर्वाः᳚ । प्र॒जाप॑ति॒रिति॑ प्र॒जा - प॒तिः॒ । सर्वा॑ दे॒वताः᳚ । दे॒वता॑ दे॒वता॑भिः । दे॒वता॑भिरे॒व । ए॒व य॒ज्ञ्म् । य॒ज्ञ्ꣳ सम् । सम् त॑न्वन्ति । त॒न्व॒न्ति॒ यन्ति॑ । यन्ति॒ वै । वा ए॒ते । ए॒ते सव॑नात् । सव॑ना॒द् ये । येऽहः॑ । अह॑रुथ्सृ॒जन्ति॑ \newline

\textbf{Jatai Paata} \newline

1. अ॒न॒न्ति॒ न नान॑ न्त्यनन्ति॒ न । \newline
2. नार्ति॒ मार्ति॒न् न नार्ति᳚म् । \newline
3. आर्ति॒ मा ऽऽर्ति॒ मार्ति॒ मा । \newline
4. आर्च्छ॑ न्त्यृच्छ॒ न्त्यार्च्छ॑न्ति । \newline
5. ऋ॒च्छ॒न्ति॒ पू॒र्णमा॑से पू॒र्णमा॑स ऋच्छ न्त्यृच्छन्ति पू॒र्णमा॑से । \newline
6. पू॒र्णमा॑से॒ वै वै पू॒र्णमा॑से पू॒र्णमा॑से॒ वै । \newline
7. पू॒र्णमा॑स॒ इति॑ पू॒र्ण - मा॒से॒ । \newline
8. वै दे॒वाना᳚म् दे॒वानां॒ ॅवै वै दे॒वाना᳚म् । \newline
9. दे॒वानाꣳ॑ सु॒तः सु॒तो दे॒वाना᳚म् दे॒वानाꣳ॑ सु॒तः । \newline
10. सु॒तो यद् यथ् सु॒तः सु॒तो यत् । \newline
11. यत् पौ᳚र्णमा॒स्या पौ᳚र्णमा॒स्या यद् यत् पौ᳚र्णमा॒स्या । \newline
12. पौ॒र्ण॒मा॒स्या मासा॒न् मासा᳚न् पौर्णमा॒स्या पौ᳚र्णमा॒स्या मासान्॑ । \newline
13. पौ॒र्ण॒मा॒स्येति॑ पौर्ण - मा॒स्या । \newline
14. मासा᳚न् थ्सं॒पाद्य॑ सं॒पाद्य॒ मासा॒न् मासा᳚न् थ्सं॒पाद्य॑ । \newline
15. सं॒पाद्या ह॒ रहः॑ सं॒पाद्य॑ सं॒पाद्याहः॑ । \newline
16. सं॒पाद्येति॑ सं - पाद्य॑ । \newline
17. अह॑ रुथ्सृ॒ज न्त्यु॑थ्सृ॒ज न्त्यह॒ रह॑ रुथ्सृ॒जन्ति॑ । \newline
18. उ॒थ्सृ॒जन्ति॑ दे॒वाना᳚म् दे॒वाना॑ मुथ्सृ॒ज न्त्यु॑थ्सृ॒जन्ति॑ दे॒वाना᳚म् । \newline
19. उ॒थ्सृ॒जन्तीत्यु॑त् - सृ॒जन्ति॑ । \newline
20. दे॒वाना॑ मे॒वैव दे॒वाना᳚म् दे॒वाना॑ मे॒व । \newline
21. ए॒व तत् तदे॒वैव तत् । \newline
22. तद् य॒ज्ञेन॑ य॒ज्ञेन॒ तत् तद् य॒ज्ञेन॑ । \newline
23. य॒ज्ञेन॑ य॒ज्ञ्ं ॅय॒ज्ञ्ं ॅय॒ज्ञेन॑ य॒ज्ञेन॑ य॒ज्ञ्म् । \newline
24. य॒ज्ञ्म् प्र॒त्यव॑रोहन्ति प्र॒त्यव॑रोहन्ति य॒ज्ञ्ं ॅय॒ज्ञ्म् प्र॒त्यव॑रोहन्ति । \newline
25. प्र॒त्यव॑रोहन्ति॒ वि वि प्र॒त्यव॑रोहन्ति प्र॒त्यव॑रोहन्ति॒ वि । \newline
26. प्र॒त्यव॑रोह॒न्तीति॑ प्रति - अव॑रोहन्ति । \newline
27. वि वै वै वि वि वै । \newline
28. वा ए॒त दे॒तद् वै वा ए॒तत् । \newline
29. ए॒तद् य॒ज्ञ्ं ॅय॒ज्ञ् मे॒त दे॒तद् य॒ज्ञ्म् । \newline
30. य॒ज्ञ्म् छि॑न्दन्ति छिन्दन्ति य॒ज्ञ्ं ॅय॒ज्ञ्म् छि॑न्दन्ति । \newline
31. छि॒न्द॒न्ति॒ यद् यच् छि॑न्दन्ति छिन्दन्ति॒ यत् । \newline
32. यथ्ष॑ड॒हस॑न्ततꣳ षड॒हस॑न्ततं॒ ॅयद् यथ्ष॑ड॒हस॑न्ततम् । \newline
33. ष॒ड॒हस॑न्ततꣳ॒॒ सन्तꣳ॒॒ सन्तꣳ॑ षड॒हस॑न्ततꣳ षड॒हस॑न्ततꣳ॒॒ सन्त᳚म् । \newline
34. ष॒ड॒हस॑न्तत॒मिति॑ षड॒ह - स॒न्त॒त॒म् । \newline
35. सन्त॒ मथाथ॒ सन्तꣳ॒॒ सन्त॒ मथ॑ । \newline
36. अथाह॒ रह॒ रथा थाहः॑ । \newline
37. अह॑ रुथ्सृ॒ज न्त्यु॑थ्सृ॒ज न्त्यह॒ रह॑ रुथ्सृ॒जन्ति॑ । \newline
38. उ॒थ्सृ॒जन्ति॑ प्राजाप॒त्यम् प्रा॑जाप॒त्य मु॑थ्सृ॒ज न्त्यु॑थ्सृ॒जन्ति॑ प्राजाप॒त्यम् । \newline
39. उ॒थ्सृ॒जन्तीत्यु॑त् - सृ॒जन्ति॑ । \newline
40. प्रा॒जा॒प॒त्यम् प॒शुम् प॒शुम् प्रा॑जाप॒त्यम् प्रा॑जाप॒त्यम् प॒शुम् । \newline
41. प्रा॒जा॒प॒त्यमिति॑ प्राजा - प॒त्यम् । \newline
42. प॒शु मा प॒शुम् प॒शु मा । \newline
43. आ ल॑भन्ते लभन्त॒ आ ल॑भन्ते । \newline
44. ल॒भ॒न्ते॒ प्र॒जाप॑तिः प्र॒जाप॑तिर् लभन्ते लभन्ते प्र॒जाप॑तिः । \newline
45. प्र॒जाप॑तिः॒ सर्वाः॒ सर्वाः᳚ प्र॒जाप॑तिः प्र॒जाप॑तिः॒ सर्वाः᳚ । \newline
46. प्र॒जाप॑ति॒रिति॑ प्र॒जा - प॒तिः॒ । \newline
47. सर्वा॑ दे॒वता॑ दे॒वताः॒ सर्वाः॒ सर्वा॑ दे॒वताः᳚ । \newline
48. दे॒वता॑ दे॒वता॑भिर् दे॒वता॑भिर् दे॒वता॑ दे॒वता॑ दे॒वता॑भिः । \newline
49. दे॒वता॑भि रे॒वैव दे॒वता॑भिर् दे॒वता॑भि रे॒व । \newline
50. ए॒व य॒ज्ञ्ं ॅय॒ज्ञ् मे॒वैव य॒ज्ञ्म् । \newline
51. य॒ज्ञ्ꣳ सꣳ सं ॅय॒ज्ञ्ं ॅय॒ज्ञ्ꣳ सम् । \newline
52. सम् त॑न्वन्ति तन्वन्ति॒ सꣳ सम् त॑न्वन्ति । \newline
53. त॒न्व॒न्ति॒ यन्ति॒ यन्ति॑ तन्वन्ति तन्वन्ति॒ यन्ति॑ । \newline
54. यन्ति॒ वै वै यन्ति॒ यन्ति॒ वै । \newline
55. वा ए॒त ए॒ते वै वा ए॒ते । \newline
56. ए॒ते सव॑ना॒थ् सव॑ना दे॒त ए॒ते सव॑नात् । \newline
57. सव॑ना॒द् ये ये सव॑ना॒थ् सव॑ना॒द् ये । \newline
58. ये ऽह॒ रह॒र् ये ये ऽहः॑ । \newline
59. अह॑ रुथ्सृ॒ज न्त्यु॑थ्सृ॒ज न्त्यह॒ रह॑ रुथ्सृ॒जन्ति॑ । \newline

\textbf{Ghana Paata } \newline

1. अ॒न॒न्ति॒ न नान॑ न्त्यनन्ति॒ नार्ति॒ मार्ति॒न् नान॑ न्त्यनन्ति॒ नार्ति᳚म् । \newline
2. नार्ति॒ मार्ति॒न् न नार्ति॒ मा ऽऽर्ति॒न् न नार्ति॒ मा । \newline
3. आर्ति॒ मा ऽऽर्ति॒ मार्ति॒ मार्च्छ॑ न्त्यृच्छ॒ न्त्याऽऽर्ति॒ मार्ति॒ मार्च्छ॑न्ति । \newline
4. आर्च्छ॑ न्त्यृच्छ॒ न्त्यार्च्छ॑न्ति पू॒र्णमा॑से पू॒र्णमा॑स ऋच्छ॒ न्त्यार्च्छ॑न्ति पू॒र्णमा॑से । \newline
5. ऋ॒च्छ॒न्ति॒ पू॒र्णमा॑से पू॒र्णमा॑स ऋच्छ न्त्यृच्छन्ति पू॒र्णमा॑से॒ वै वै पू॒र्णमा॑स ऋच्छ न्त्यृच्छन्ति पू॒र्णमा॑से॒ वै । \newline
6. पू॒र्णमा॑से॒ वै वै पू॒र्णमा॑से पू॒र्णमा॑से॒ वै दे॒वाना᳚म् दे॒वानां॒ ॅवै पू॒र्णमा॑से पू॒र्णमा॑से॒ वै दे॒वाना᳚म् । \newline
7. पू॒र्णमा॑स॒ इति॑ पू॒र्ण - मा॒से॒ । \newline
8. वै दे॒वाना᳚म् दे॒वानां॒ ॅवै वै दे॒वानाꣳ॑ सु॒तः सु॒तो दे॒वानां॒ ॅवै वै दे॒वानाꣳ॑ सु॒तः । \newline
9. दे॒वानाꣳ॑ सु॒तः सु॒तो दे॒वाना᳚म् दे॒वानाꣳ॑ सु॒तो यद् यथ् सु॒तो दे॒वाना᳚म् दे॒वानाꣳ॑ सु॒तो यत् । \newline
10. सु॒तो यद् यथ् सु॒तः सु॒तो यत् पौ᳚र्णमा॒स्या पौ᳚र्णमा॒स्या यथ् सु॒तः सु॒तो यत् पौ᳚र्णमा॒स्या । \newline
11. यत् पौ᳚र्णमा॒स्या पौ᳚र्णमा॒स्या यद् यत् पौ᳚र्णमा॒स्या मासा॒न् मासा᳚न् पौर्णमा॒स्या यद् यत् पौ᳚र्णमा॒स्या मासान्॑ । \newline
12. पौ॒र्ण॒मा॒स्या मासा॒न् मासा᳚न् पौर्णमा॒स्या पौ᳚र्णमा॒स्या मासा᳚न् थ्सं॒पाद्य॑ सं॒पाद्य॒ मासा᳚न् पौर्णमा॒स्या पौ᳚र्णमा॒स्या मासा᳚न् थ्सं॒पाद्य॑ । \newline
13. पौ॒र्ण॒मा॒स्येति॑ पौर्ण - मा॒स्या । \newline
14. मासा᳚न् थ्सं॒पाद्य॑ सं॒पाद्य॒ मासा॒न् मासा᳚न् थ्सं॒पाद्याह॒ रहः॑ सं॒पाद्य॒ मासा॒न् मासा᳚न् थ्सं॒पाद्याहः॑ । \newline
15. सं॒पाद्याह॒ रहः॑ सं॒पाद्य॑ सं॒पाद्याह॑ रुथ्सृ॒ज न्त्यु॑थ्सृ॒ज न्त्यहः॑ सं॒पाद्य॑ सं॒पाद्याह॑ रुथ्सृ॒जन्ति॑ । \newline
16. सं॒पाद्येति॑ सं - पाद्य॑ । \newline
17. अह॑ रुथ्सृ॒ज न्त्यु॑थ्सृ॒ज न्त्यह॒ रह॑ रुथ्सृ॒जन्ति॑ दे॒वाना᳚म् दे॒वाना॑ मुथ्सृ॒ज न्त्यह॒ रह॑ रुथ्सृ॒जन्ति॑ दे॒वाना᳚म् । \newline
18. उ॒थ्सृ॒जन्ति॑ दे॒वाना᳚म् दे॒वाना॑ मुथ्सृ॒ज न्त्यु॑थ्सृ॒जन्ति॑ दे॒वाना॑ मे॒वैव दे॒वाना॑ मुथ्सृ॒ज न्त्यु॑थ्सृ॒जन्ति॑ दे॒वाना॑ मे॒व । \newline
19. उ॒थ्सृ॒जन्तीत्यु॑त् - सृ॒जन्ति॑ । \newline
20. दे॒वाना॑ मे॒वैव दे॒वाना᳚म् दे॒वाना॑ मे॒व तत् तदे॒व दे॒वाना᳚म् दे॒वाना॑ मे॒व तत् । \newline
21. ए॒व तत् तदे॒वैव तद् य॒ज्ञेन॑ य॒ज्ञेन॒ तदे॒वैव तद् य॒ज्ञेन॑ । \newline
22. तद् य॒ज्ञेन॑ य॒ज्ञेन॒ तत् तद् य॒ज्ञेन॑ य॒ज्ञ्ं ॅय॒ज्ञ्ं ॅय॒ज्ञेन॒ तत् तद् य॒ज्ञेन॑ य॒ज्ञ्म् । \newline
23. य॒ज्ञेन॑ य॒ज्ञ्ं ॅय॒ज्ञ्ं ॅय॒ज्ञेन॑ य॒ज्ञेन॑ य॒ज्ञ्म् प्र॒त्यव॑रोहन्ति प्र॒त्यव॑रोहन्ति य॒ज्ञ्ं ॅय॒ज्ञेन॑ य॒ज्ञेन॑ य॒ज्ञ्म् प्र॒त्यव॑रोहन्ति । \newline
24. य॒ज्ञ्म् प्र॒त्यव॑रोहन्ति प्र॒त्यव॑रोहन्ति य॒ज्ञ्ं ॅय॒ज्ञ्म् प्र॒त्यव॑रोहन्ति॒ वि वि प्र॒त्यव॑रोहन्ति य॒ज्ञ्ं ॅय॒ज्ञ्म् प्र॒त्यव॑रोहन्ति॒ वि । \newline
25. प्र॒त्यव॑रोहन्ति॒ वि वि प्र॒त्यव॑रोहन्ति प्र॒त्यव॑रोहन्ति॒ वि वै वै वि प्र॒त्यव॑रोहन्ति प्र॒त्यव॑रोहन्ति॒ वि वै । \newline
26. प्र॒त्यव॑रोह॒न्तीति॑ प्रति - अव॑रोहन्ति । \newline
27. वि वै वै वि वि वा ए॒त दे॒तद् वै वि वि वा ए॒तत् । \newline
28. वा ए॒त दे॒तद् वै वा ए॒तद् य॒ज्ञ्ं ॅय॒ज्ञ् मे॒तद् वै वा ए॒तद् य॒ज्ञ्म् । \newline
29. ए॒तद् य॒ज्ञ्ं ॅय॒ज्ञ् मे॒त दे॒तद् य॒ज्ञ्म् छि॑न्दन्ति छिन्दन्ति य॒ज्ञ् मे॒त दे॒तद् य॒ज्ञ्म् छि॑न्दन्ति । \newline
30. य॒ज्ञ्म् छि॑न्दन्ति छिन्दन्ति य॒ज्ञ्ं ॅय॒ज्ञ्म् छि॑न्दन्ति॒ यद् यच् छि॑न्दन्ति य॒ज्ञ्ं ॅय॒ज्ञ्म् छि॑न्दन्ति॒ यत् । \newline
31. छि॒न्द॒न्ति॒ यद् यच् छि॑न्दन्ति छिन्दन्ति॒ यथ् ष॑ड॒हस॑न्ततꣳ षड॒हस॑न्ततं॒ ॅयच् छि॑न्दन्ति छिन्दन्ति॒ यथ् ष॑ड॒हस॑न्ततम् । \newline
32. यथ् ष॑ड॒हस॑न्ततꣳ षड॒हस॑न्ततं॒ ॅयद् यथ् ष॑ड॒हस॑न्ततꣳ॒॒ सन्तꣳ॒॒ सन्तꣳ॑ षड॒हस॑न्ततं॒ ॅयद् यथ् ष॑ड॒हस॑न्ततꣳ॒॒ सन्त᳚म् । \newline
33. ष॒ड॒हस॑न्ततꣳ॒॒ सन्तꣳ॒॒ सन्तꣳ॑ षड॒हस॑न्ततꣳ षड॒हस॑न्ततꣳ॒॒ सन्त॒ मथाथ॒ सन्तꣳ॑ षड॒हस॑न्ततꣳ षड॒हस॑न्ततꣳ॒॒ सन्त॒ मथ॑ । \newline
34. ष॒ड॒हस॑न्तत॒मिति॑ षड॒ह - स॒न्त॒त॒म् । \newline
35. सन्त॒ मथाथ॒ सन्तꣳ॒॒ सन्त॒ मथाह॒ रह॒ रथ॒ सन्तꣳ॒॒ सन्त॒ मथाहः॑ । \newline
36. अथाह॒ रह॒ रथा थाह॑ रुथ्सृ॒ज न्त्यु॑थ्सृ॒ज न्त्यह॒ रथा थाह॑ रुथ्सृ॒जन्ति॑ । \newline
37. अह॑ रुथ्सृ॒ज न्त्यु॑थ्सृ॒ज न्त्यह॒ रह॑ रुथ्सृ॒जन्ति॑ प्राजाप॒त्यम् प्रा॑जाप॒त्य मु॑थ्सृ॒ज न्त्यह॒ रह॑ रुथ्सृ॒जन्ति॑ प्राजाप॒त्यम् । \newline
38. उ॒थ्सृ॒जन्ति॑ प्राजाप॒त्यम् प्रा॑जाप॒त्य मु॑थ्सृ॒ज न्त्यु॑थ्सृ॒जन्ति॑ प्राजाप॒त्यम् प॒शुम् प॒शुम् प्रा॑जाप॒त्य मु॑थ्सृ॒ज न्त्यु॑थ्सृ॒जन्ति॑ प्राजाप॒त्यम् प॒शुम् । \newline
39. उ॒थ्सृ॒जन्तीत्यु॑त् - सृ॒जन्ति॑ । \newline
40. प्रा॒जा॒प॒त्यम् प॒शुम् प॒शुम् प्रा॑जाप॒त्यम् प्रा॑जाप॒त्यम् प॒शु मा प॒शुम् प्रा॑जाप॒त्यम् प्रा॑जाप॒त्यम् प॒शु मा । \newline
41. प्रा॒जा॒प॒त्यमिति॑ प्राजा - प॒त्यम् । \newline
42. प॒शु मा प॒शुम् प॒शु मा ल॑भन्ते लभन्त॒ आ प॒शुम् प॒शु मा ल॑भन्ते । \newline
43. आ ल॑भन्ते लभन्त॒ आ ल॑भन्ते प्र॒जाप॑तिः प्र॒जाप॑तिर् लभन्त॒ आ ल॑भन्ते प्र॒जाप॑तिः । \newline
44. ल॒भ॒न्ते॒ प्र॒जाप॑तिः प्र॒जाप॑तिर् लभन्ते लभन्ते प्र॒जाप॑तिः॒ सर्वाः॒ सर्वाः᳚ प्र॒जाप॑तिर् लभन्ते लभन्ते प्र॒जाप॑तिः॒ सर्वाः᳚ । \newline
45. प्र॒जाप॑तिः॒ सर्वाः॒ सर्वाः᳚ प्र॒जाप॑तिः प्र॒जाप॑तिः॒ सर्वा॑ दे॒वता॑ दे॒वताः॒ सर्वाः᳚ प्र॒जाप॑तिः प्र॒जाप॑तिः॒ सर्वा॑ दे॒वताः᳚ । \newline
46. प्र॒जाप॑ति॒रिति॑ प्र॒जा - प॒तिः॒ । \newline
47. सर्वा॑ दे॒वता॑ दे॒वताः॒ सर्वाः॒ सर्वा॑ दे॒वता॑ दे॒वता॑भिर् दे॒वता॑भिर् दे॒वताः॒ सर्वाः॒ सर्वा॑ दे॒वता॑ दे॒वता॑भिः । \newline
48. दे॒वता॑ दे॒वता॑भिर् दे॒वता॑भिर् दे॒वता॑ दे॒वता॑ दे॒वता॑भि रे॒वैव दे॒वता॑भिर् दे॒वता॑ दे॒वता॑ दे॒वता॑भि रे॒व । \newline
49. दे॒वता॑भि रे॒वैव दे॒वता॑भिर् दे॒वता॑भि रे॒व य॒ज्ञ्ं ॅय॒ज्ञ् मे॒व दे॒वता॑भिर् दे॒वता॑भि रे॒व य॒ज्ञ्म् । \newline
50. ए॒व य॒ज्ञ्ं ॅय॒ज्ञ् मे॒वैव य॒ज्ञ्ꣳ सꣳ सं ॅय॒ज्ञ् मे॒वैव य॒ज्ञ्ꣳ सम् । \newline
51. य॒ज्ञ्ꣳ सꣳ सं ॅय॒ज्ञ्ं ॅय॒ज्ञ्ꣳ सम् त॑न्वन्ति तन्वन्ति॒ सं ॅय॒ज्ञ्ं ॅय॒ज्ञ्ꣳ सम् त॑न्वन्ति । \newline
52. सम् त॑न्वन्ति तन्वन्ति॒ सꣳ सम् त॑न्वन्ति॒ यन्ति॒ यन्ति॑ तन्वन्ति॒ सꣳ सम् त॑न्वन्ति॒ यन्ति॑ । \newline
53. त॒न्व॒न्ति॒ यन्ति॒ यन्ति॑ तन्वन्ति तन्वन्ति॒ यन्ति॒ वै वै यन्ति॑ तन्वन्ति तन्वन्ति॒ यन्ति॒ वै । \newline
54. यन्ति॒ वै वै यन्ति॒ यन्ति॒ वा ए॒त ए॒ते वै यन्ति॒ यन्ति॒ वा ए॒ते । \newline
55. वा ए॒त ए॒ते वै वा ए॒ते सव॑ना॒थ् सव॑ना दे॒ते वै वा ए॒ते सव॑नात् । \newline
56. ए॒ते सव॑ना॒थ् सव॑ना दे॒त ए॒ते सव॑ना॒द् ये ये सव॑ना दे॒त ए॒ते सव॑ना॒द् ये । \newline
57. सव॑ना॒द् ये ये सव॑ना॒थ् सव॑ना॒द् ये ऽह॒ रह॒र् ये सव॑ना॒थ् सव॑ना॒द् ये ऽहः॑ । \newline
58. ये ऽह॒ रह॒र् ये ये ऽह॑ रुथ्सृ॒ज न्त्यु॑थ्सृ॒ज न्त्यह॒र् ये ये ऽह॑ रुथ्सृ॒जन्ति॑ । \newline
59. अह॑ रुथ्सृ॒ज न्त्यु॑थ्सृ॒ज न्त्यह॒ रह॑ रुथ्सृ॒जन्ति॑ तु॒रीय॑म् तु॒रीय॑ मुथ्सृ॒ज न्त्यह॒ रह॑ रुथ्सृ॒जन्ति॑ तु॒रीय᳚म् । \newline
\pagebreak
\markright{ TS 7.5.6.4  \hfill https://www.vedavms.in \hfill}

\section{ TS 7.5.6.4 }

\textbf{TS 7.5.6.4 } \newline
\textbf{Samhita Paata} \newline

-रुथ् सृ॒जन्ति॑ तु॒रीयं॒ खलु॒ वा ए॒तथ् सव॑नं॒ ॅयथ् सा᳚नां॒य्यं ॅयथ् सा᳚नां॒य्यं भव॑ति॒ तेनै॒व सव॑ना॒न्न य॑न्ति समुप॒हूय॑ भक्षयन्त्ये॒तथ्- सो॑मपीथा॒ ह्ये॑तर्.हि॑ यथायत॒नं ॅवा ए॒तेषाꣳ॑ सवन॒भाजो॑ दे॒वता॑ गच्छन्ति॒ येऽह॑रुथ् सृ॒जन्त्य॑नुसव॒नं पु॑रो॒डाशा॒न् निर्व॑पन्ति यथायत॒नादे॒व स॑वन॒भाजो॑ दे॒वता॒ अव॑ रुन्धते॒ ऽष्टाक॑पालान् प्रातस्सव॒न एका॑दशकपाला॒न् माद्ध्य॑न्दिने॒ सव॑ने॒ द्वाद॑शकपालाꣳ-स्तृतीयसव॒ने छन्दाꣳ॑स्ये॒वाऽऽ*प्त्वा ( ) -ऽव॑ रुन्धते वैश्वदे॒वं च॒रुं तृ॑तीयसव॒ने निर्व॑पन्ति वैश्वदे॒वं ॅवै तृ॑तीयसव॒नं तेनै॒व तृ॑तीयसव॒नान्न य॑न्ति ॥ \newline

\textbf{Pada Paata} \newline

उ॒थ्सृ॒जन्तीत्यु॑त् - सृ॒जन्ति॑ । तु॒रीय᳚म् । खलु॑ । वै । ए॒तत् । सव॑नम् । यत् । सा॒न्ना॒य्यमिति॑ सां-ना॒य्यम् । यत् । सा॒न्ना॒य्यमिति॑ सां-ना॒य्यम् । भव॑ति । तेन॑ । ए॒व । सव॑नात् । न । य॒न्ति॒ । स॒मु॒प॒हूयेति॑ सं - उ॒प॒हूय॑ । भ॒क्ष॒य॒न्ति॒ । ए॒तथ्सो॑मपीथा॒ इत्ये॒तत् - सो॒म॒पी॒थाः॒ । हि । ए॒तर्.हि॑ । य॒था॒य॒त॒नमिति॑ यथा - अ॒य॒त॒नम् । वै । ए॒तेषा᳚म् । स॒व॒न॒भाज॒ इति॑ सवन - भाजः॑ । दे॒वताः᳚ । ग॒च्छ॒न्ति॒ । ये । अहः॑ । उ॒थ्सृ॒जन्तीत्यु॑त् - सृ॒जन्ति॑ । अ॒नु॒स॒व॒नमित्य॑नु - स॒व॒नम् । पु॒रो॒डाशान्॑ । निरिति॑ । व॒प॒न्ति॒ । य॒था॒य॒त॒नादिति॑ यथा - आ॒य॒त॒नात् । ए॒व । स॒व॒न॒भाज॒ इति॑ सवन - भाजः॑ । दे॒वताः᳚ । अवेति॑ । रु॒न्ध॒ते॒ । अ॒ष्टाक॑पाला॒नित्य॒ष्टा - क॒पा॒ला॒न् । प्रा॒त॒स्स॒व॒न इति॑ प्रातः - स॒व॒ने । एका॑दशकपाला॒नित्येका॑दश - क॒पा॒ला॒न्न् । माद्ध्य॑न्दिने । सव॑ने । द्वाद॑शकपाला॒निति॒ द्वाद॑श - क॒पा॒ला॒न्न् । तृ॒ती॒य॒स॒व॒न इति॑ तृतीय - स॒व॒ने । छन्दाꣳ॑सि । ए॒व । आ॒प्त्वा ( ) । अवेति॑ । रु॒न्ध॒ते॒ । वै॒श्व॒दे॒वमिति॑ वैश्व-दे॒वम् । च॒रुम् । तृ॒ती॒य॒स॒व॒न इति॑ तृतीय-स॒व॒ने । निरिति॑ । व॒प॒न्ति॒ । वै॒श्व॒दे॒वमिति॑ वैश्व - दे॒वम् । वै । तृ॒ती॒य॒स॒व॒नमिति॑ तृतीय - स॒व॒नम् । तेन॑ । ए॒व । तृ॒ती॒य॒स॒व॒नादिति॑ तृतीय - स॒व॒नात् । न । य॒न्ति॒ ॥  \newline


\textbf{Krama Paata} \newline

उ॒थ्सृ॒जन्ति॑ तु॒रीय᳚म् । उ॒थ्सृ॒जन्तीत्यु॑त् - सृ॒जन्ति॑ । तु॒रीय॒म् खलु॑ । खलु॒ वै । वा ए॒तत् । ए॒तथ् सव॑नम् । सव॑न॒म् ॅयत् । यथ् सा᳚न्ना॒य्यम् । सा॒न्ना॒य्यम् ॅयत् । सा॒न्ना॒य्यमिति॑ साम् - ना॒य्यम् । यथ् सा᳚न्ना॒यम् । सा॒न्ना॒य्यम् भव॑ति । सा॒न्ना॒य्यमिति॑ साम् - ना॒य्यम् । भव॑ति॒ तेन॑ । तेनै॒व । ए॒व सव॑नात् । सव॑ना॒न् न । न य॑न्ति । य॒न्ति॒ स॒मु॒प॒हूय॑ । स॒मु॒प॒हूय॑ भक्षयन्ति । स॒मु॒प॒हूयेति॑ सम् - उ॒प॒हूय॑ । भ॒क्ष॒य॒न्त्ये॒तथ्सो॑मपीथाः । ए॒तथ्सो॑मपीथा॒ हि । ए॒तथ्सो॑मपीथा॒ इत्ये॒तत् - सो॒म॒पी॒थाः॒ । ह्ये॑तर्.हि॑ । ए॒तर्.हि॑ यथायत॒नम् । य॒था॒य॒त॒नम् ॅवै । य॒था॒य॒त॒नमिति॑ यथा - आ॒य॒त॒नम् । वा ए॒तेषा᳚म् । 
ए॒तेषाꣳ॑ सवन॒भाजः॑ । स॒व॒न॒भाजो॑ दे॒वताः᳚ । स॒व॒न॒भाज॒ इति॑ सवन - भाजः॑ । दे॒वता॑ गच्छन्ति । ग॒च्छ॒न्ति॒ ये । येऽहः॑ । अह॑रुथ्सृ॒जन्ति॑ । उ॒थ्सृ॒जन्त्य॑नुसव॒नम् । उ॒थ्सृ॒जन्तीत्यु॑त् - सृ॒जन्ति॑ । अ॒नु॒स॒व॒नम् पु॑रो॒डाशान्॑ । अ॒नु॒स॒व॒नमित्य॑नु - स॒व॒नम् । पु॒रो॒डाशा॒न् निः । निर् व॑पन्ति । व॒प॒न्ति॒ य॒था॒य॒त॒नात् । य॒था॒य॒त॒नादे॒व । य॒था॒य॒त॒नादिति॑ यथा - आ॒य॒त॒नात् । ए॒व स॑वन॒भाजः॑ । स॒व॒न॒भाजो॑ दे॒वताः᳚ । स॒व॒न॒भाज॒ इति॑ सवन - भाजः॑ । दे॒वता॒ अव॑ । अव॑ रुन्धते । रु॒न्ध॒ते॒ऽष्टाक॑पालान् । अ॒ष्टाक॑पालान् प्रातस्सव॒ने । अ॒ष्टाक॑पाला॒नित्य॒ष्टा - क॒पा॒ला॒न्॒ । प्रा॒त॒स्स॒व॒न एका॑दशकपालान् । प्रा॒त॒स्स॒व॒न इति॑ प्रातः - स॒व॒ने । एका॑दशकपाला॒न् माद्ध्य॑न्दिने । एका॑दशकपाला॒नित्येका॑दश - क॒पा॒ला॒न्॒ । माद्ध्य॑न्दिने॒ सव॑ने । सव॑ने॒ द्वाद॑शकपालान् । द्वाद॑शकपालाꣳ स्तृतीयसव॒ने । द्वाद॑शकपाला॒निति॒ द्वाद॑श - क॒पा॒ला॒न्॒ । तृ॒ती॒य॒स॒व॒ने छन्दाꣳ॑सि । तृ॒ती॒य॒स॒व॒न इति॑ तृतीय - स॒व॒ने । छन्दाꣳ॑स्ये॒व । ए॒वाप्त्वा ( ) । आ॒प्त्वाऽव॑ । अव॑ रुन्धते । रु॒न्ध॒ते॒ वै॒श्व॒दे॒वम् । वै॒श्व॒दे॒वम् च॒रुम् । वै॒श्व॒दे॒वमिति॑ वैश्व - दे॒वम् । च॒रुम् तृ॑तीयसव॒ने । तृ॒ती॒य॒स॒व॒ने निः । तृ॒ती॒य॒स॒व॒न इति॑ तृतीय - स॒व॒ने । निर् व॑पन्ति । व॒प॒न्ति॒ वै॒श्व॒दे॒वम् । वै॒श्व॒दे॒वम् ॅवै । वै॒श्व॒दे॒व॒मिति॑ वैश्व - दे॒वम् । वै तृ॑तीयसव॒नम् । तृ॒ती॒य॒स॒व॒नम् तेन॑ । तृ॒ती॒य॒स॒व॒नमिति॑ तृतीय - स॒व॒नम् । तेनै॒व । ए॒व तृ॑तीयसव॒नात् । तृ॒ती॒य॒स॒व॒नान् न । तृ॒ती॒य॒स॒व॒नादिति॑ तृतीय - स॒व॒नात् । न य॑न्ति । य॒न्तीति॑ यन्ति । \newline

\textbf{Jatai Paata} \newline

1. उ॒थ्सृ॒जन्ति॑ तु॒रीय॑म् तु॒रीय॑ मुथ्सृ॒ज न्त्यु॑थ्सृ॒जन्ति॑ तु॒रीय᳚म् । \newline
2. उ॒थ्सृ॒जन्तीत्यु॑त् - सृ॒जन्ति॑ । \newline
3. तु॒रीय॒म् खलु॒ खलु॑ तु॒रीय॑म् तु॒रीय॒म् खलु॑ । \newline
4. खलु॒ वै वै खलु॒ खलु॒ वै । \newline
5. वा ए॒त दे॒तद् वै वा ए॒तत् । \newline
6. ए॒तथ् सव॑नꣳ॒॒ सव॑न मे॒त दे॒तथ् सव॑नम् । \newline
7. सव॑नं॒ ॅयद् यथ् सव॑नꣳ॒॒ सव॑नं॒ ॅयत् । \newline
8. यथ् सा᳚न्ना॒य्यꣳ सा᳚न्ना॒य्यं ॅयद् यथ् सा᳚न्ना॒य्यम् । \newline
9. सा॒न्ना॒य्यं ॅयद् यथ् सा᳚न्ना॒य्यꣳ सा᳚न्ना॒य्यं ॅयत् । \newline
10. सा॒न्ना॒य्यमिति॑ सां - ना॒य्यम् । \newline
11. यथ् सा᳚न्ना॒य्यꣳ सा᳚न्ना॒य्यं ॅयद् यथ् सा᳚न्ना॒य्यम् । \newline
12. सा॒न्ना॒य्यम् भव॑ति॒ भव॑ति सान्ना॒य्यꣳ सा᳚न्ना॒य्यम् भव॑ति । \newline
13. सा॒न्ना॒य्यमिति॑ सां - ना॒य्यम् । \newline
14. भव॑ति॒ तेन॒ तेन॒ भव॑ति॒ भव॑ति॒ तेन॑ । \newline
15. तेनै॒वैव तेन॒ तेनै॒व । \newline
16. ए॒व सव॑ना॒थ् सव॑ना दे॒वैव सव॑नात् । \newline
17. सव॑ना॒न् न न सव॑ना॒थ् सव॑ना॒न् न । \newline
18. न य॑न्ति यन्ति॒ न न य॑न्ति । \newline
19. य॒न्ति॒ स॒मु॒प॒हूय॑ समुप॒हूय॑ यन्ति यन्ति समुप॒हूय॑ । \newline
20. स॒मु॒प॒हूय॑ भक्षयन्ति भक्षयन्ति समुप॒हूय॑ समुप॒हूय॑ भक्षयन्ति । \newline
21. स॒मु॒प॒हूयेति॑ सं - उ॒प॒हूय॑ । \newline
22. भ॒क्ष॒य॒ न्त्ये॒तथ्सो॑मपीथा ए॒तथ्सो॑मपीथा भक्षयन्ति भक्षय न्त्ये॒तथ्सो॑मपीथाः । \newline
23. ए॒तथ्सो॑मपीथा॒ हि ह्ये॑तथ्सो॑मपीथा ए॒तथ्सो॑मपीथा॒ हि । \newline
24. ए॒तथ्सो॑मपीथा॒ इत्ये॒तत् - सो॒म॒पी॒थाः॒ । \newline
25. ह्ये॑तर् ह्ये॒तर्.हि॒ हि ह्ये॑तर्.हि॑ । \newline
26. ए॒तर्.हि॑ यथायत॒नं ॅय॑थायत॒न मे॒तर् ह्ये॒तर्.हि॑ यथायत॒नम् । \newline
27. य॒था॒य॒त॒नं ॅवै वै य॑थायत॒नं ॅय॑थायत॒नं ॅवै । \newline
28. य॒था॒य॒त॒नमिति॑ यथा - अ॒य॒त॒नम् । \newline
29. वा ए॒तेषा॑ मे॒तेषां॒ ॅवै वा ए॒तेषा᳚म् । \newline
30. ए॒तेषाꣳ॑ सवन॒भाजः॑ सवन॒भाज॑ ए॒तेषा॑ मे॒तेषाꣳ॑ सवन॒भाजः॑ । \newline
31. स॒व॒न॒भाजो॑ दे॒वता॑ दे॒वताः᳚ सवन॒भाजः॑ सवन॒भाजो॑ दे॒वताः᳚ । \newline
32. स॒व॒न॒भाज॒ इति॑ सवन - भाजः॑ । \newline
33. दे॒वता॑ गच्छन्ति गच्छन्ति दे॒वता॑ दे॒वता॑ गच्छन्ति । \newline
34. ग॒च्छ॒न्ति॒ ये ये ग॑च्छन्ति गच्छन्ति॒ ये । \newline
35. ये ऽह॒ रह॒र् ये ये ऽहः॑ । \newline
36. अह॑ रुथ्सृ॒ज न्त्यु॑थ्सृ॒ज न्त्यह॒ रह॑ रुथ्सृ॒जन्ति॑ । \newline
37. उ॒थ्सृ॒ज न्त्य॑नुसव॒न म॑नुसव॒न मु॑थ्सृ॒ज न्त्यु॑थ्सृ॒ज न्त्य॑नुसव॒नम् । \newline
38. उ॒थ्सृ॒जन्तीत्यु॑त् - सृ॒जन्ति॑ । \newline
39. अ॒नु॒स॒व॒नम् पु॑रो॒डाशा᳚न् पुरो॒डाशा॑ ननुसव॒न म॑नुसव॒नम् पु॑रो॒डाशान्॑ । \newline
40. अ॒नु॒स॒व॒नमित्य॑नु - स॒व॒नम् । \newline
41. पु॒रो॒डाशा॒न् निर् णिष् पु॑रो॒डाशा᳚न् पुरो॒डाशा॒न् निः । \newline
42. निर् व॑पन्ति वपन्ति॒ निर् णिर् व॑पन्ति । \newline
43. व॒प॒न्ति॒ य॒था॒य॒त॒नाद् य॑थायत॒नाद् व॑पन्ति वपन्ति यथायत॒नात् । \newline
44. य॒था॒य॒त॒ना दे॒वैव य॑थायत॒नाद् य॑थायत॒ना दे॒व । \newline
45. य॒था॒य॒त॒नादिति॑ यथा - आ॒य॒त॒नात् । \newline
46. ए॒व स॑वन॒भाजः॑ सवन॒भाज॑ ए॒वैव स॑वन॒भाजः॑ । \newline
47. स॒व॒न॒भाजो॑ दे॒वता॑ दे॒वताः᳚ सवन॒भाजः॑ सवन॒भाजो॑ दे॒वताः᳚ । \newline
48. स॒व॒न॒भाज॒ इति॑ सवन - भाजः॑ । \newline
49. दे॒वता॒ अवाव॑ दे॒वता॑ दे॒वता॒ अव॑ । \newline
50. अव॑ रुन्धते रुन्ध॒ते ऽवाव॑ रुन्धते । \newline
51. रु॒न्ध॒ते॒ ऽष्टाक॑पाला न॒ष्टाक॑पालान् रुन्धते रुन्धते॒ ऽष्टाक॑पालान् । \newline
52. अ॒ष्टाक॑पालान् प्रातस्सव॒ने प्रा॑तस्सव॒ने᳚ ऽष्टाक॑पाला न॒ष्टाक॑पालान् प्रातस्सव॒ने । \newline
53. अ॒ष्टाक॑पाला॒नित्य॒ष्टा - क॒पा॒ला॒न् । \newline
54. प्रा॒त॒स्स॒व॒न एका॑दशकपाला॒ नेका॑दशकपालान् प्रातस्सव॒ने प्रा॑तस्सव॒न एका॑दशकपालान् । \newline
55. प्रा॒त॒स्स॒व॒न इति॑ प्रातः - स॒व॒ने । \newline
56. एका॑दशकपाला॒न् माद्ध्य॑न्दिने॒ माद्ध्य॑न्दिन॒ एका॑दशकपाला॒ नेका॑दशकपाला॒न् माद्ध्य॑न्दिने । \newline
57. एका॑दशकपाला॒नित्येका॑दश - क॒पा॒ला॒न् । \newline
58. माद्ध्य॑न्दिने॒ सव॑ने॒ सव॑ने॒ माद्ध्य॑न्दिने॒ माद्ध्य॑न्दिने॒ सव॑ने । \newline
59. सव॑ने॒ द्वाद॑शकपाला॒न् द्वाद॑शकपाला॒न् थ्सव॑ने॒ सव॑ने॒ द्वाद॑शकपालान् । \newline
60. द्वाद॑शकपालाꣳ स्तृतीयसव॒ने तृ॑तीयसव॒ने द्वाद॑शकपाला॒न् द्वाद॑शकपालाꣳ स्तृतीयसव॒ने । \newline
61. द्वाद॑शकपाला॒निति॒ द्वाद॑श - क॒पा॒ला॒न् । \newline
62. तृ॒ती॒य॒स॒व॒ने छन्दाꣳ॑सि॒ छन्दाꣳ॑सि तृतीयसव॒ने तृ॑तीयसव॒ने छन्दाꣳ॑सि । \newline
63. तृ॒ती॒य॒स॒व॒न इति॑ तृतीय - स॒व॒ने । \newline
64. छन्दाꣳ॑ स्ये॒वैव छन्दाꣳ॑सि॒ छन्दाꣳ॑ स्ये॒व । \newline
65. ए॒वाप्त्वा ऽऽप्त्वै वैवाप्त्वा । \newline
66. आ॒प्त्वा ऽवावा॒ प्त्वा ऽऽप्त्वा ऽव॑ । \newline
67. अव॑ रुन्धते रुन्ध॒ते ऽवाव॑ रुन्धते । \newline
68. रु॒न्ध॒ते॒ वै॒श्व॒दे॒वं ॅवै᳚श्वदे॒वꣳ रु॑न्धते रुन्धते वैश्वदे॒वम् । \newline
69. वै॒श्व॒दे॒वम् च॒रुम् च॒रुं ॅवै᳚श्वदे॒वं ॅवै᳚श्वदे॒वम् च॒रुम् । \newline
70. वै॒श्व॒दे॒वमिति॑ वैश्व - दे॒वम् । \newline
71. च॒रुम् तृ॑तीयसव॒ने तृ॑तीयसव॒ने च॒रुम् च॒रुम् तृ॑तीयसव॒ने । \newline
72. तृ॒ती॒य॒स॒व॒ने निर् णिष् टृ॑तीयसव॒ने तृ॑तीयसव॒ने निः । \newline
73. तृ॒ती॒य॒स॒व॒न इति॑ तृतीय - स॒व॒ने । \newline
74. निर् व॑पन्ति वपन्ति॒ निर् णिर् व॑पन्ति । \newline
75. व॒प॒न्ति॒ वै॒श्व॒दे॒वं ॅवै᳚श्वदे॒वं ॅव॑पन्ति वपन्ति वैश्वदे॒वम् । \newline
76. वै॒श्व॒दे॒वं ॅवै वै वै᳚श्वदे॒वं ॅवै᳚श्वदे॒वं ॅवै । \newline
77. वै॒श्व॒दे॒वमिति॑ वैश्व - दे॒वम् । \newline
78. वै तृ॑तीयसव॒नम् तृ॑तीयसव॒नं ॅवै वै तृ॑तीयसव॒नम् । \newline
79. तृ॒ती॒य॒स॒व॒नम् तेन॒ तेन॑ तृतीयसव॒नम् तृ॑तीयसव॒नम् तेन॑ । \newline
80. तृ॒ती॒य॒स॒व॒नमिति॑ तृतीय - स॒व॒नम् । \newline
81. तेनै॒वैव तेन॒ तेनै॒व । \newline
82. ए॒व तृ॑तीयसव॒नात् तृ॑तीयसव॒ना दे॒वैव तृ॑तीयसव॒नात् । \newline
83. तृ॒ती॒य॒स॒व॒नान् न न तृ॑तीयसव॒नात् तृ॑तीयसव॒नान् न । \newline
84. तृ॒ती॒य॒स॒व॒नादिति॑ तृतीय - स॒व॒नात् । \newline
85. न य॑न्ति यन्ति॒ न न य॑न्ति । \newline
86. य॒न्तीति॑ यन्ति । \newline

\textbf{Ghana Paata } \newline

1. उ॒थ्सृ॒जन्ति॑ तु॒रीय॑म् तु॒रीय॑ मुथ्सृ॒ज न्त्यु॑थ्सृ॒जन्ति॑ तु॒रीय॒म् खलु॒ खलु॑ तु॒रीय॑ मुथ्सृ॒ज न्त्यु॑थ्सृ॒जन्ति॑ तु॒रीय॒म् खलु॑ । \newline
2. उ॒थ्सृ॒जन्तीत्यु॑त् - सृ॒जन्ति॑ । \newline
3. तु॒रीय॒म् खलु॒ खलु॑ तु॒रीय॑म् तु॒रीय॒म् खलु॒ वै वै खलु॑ तु॒रीय॑म् तु॒रीय॒म् खलु॒ वै । \newline
4. खलु॒ वै वै खलु॒ खलु॒ वा ए॒त दे॒तद् वै खलु॒ खलु॒ वा ए॒तत् । \newline
5. वा ए॒त दे॒तद् वै वा ए॒तथ् सव॑नꣳ॒॒ सव॑न मे॒तद् वै वा ए॒तथ् सव॑नम् । \newline
6. ए॒तथ् सव॑नꣳ॒॒ सव॑न मे॒त दे॒तथ् सव॑नं॒ ॅयद् यथ् सव॑न मे॒त दे॒तथ् सव॑नं॒ ॅयत् । \newline
7. सव॑नं॒ ॅयद् यथ् सव॑नꣳ॒॒ सव॑नं॒ ॅयथ् सा᳚न्ना॒य्यꣳ सा᳚न्ना॒य्यं ॅयथ् सव॑नꣳ॒॒ सव॑नं॒ ॅयथ् सा᳚न्ना॒य्यम् । \newline
8. यथ् सा᳚न्ना॒य्यꣳ सा᳚न्ना॒य्यं ॅयद् यथ् सा᳚न्ना॒य्यं ॅयद् यथ् सा᳚न्ना॒य्यं ॅयद् यथ् सा᳚न्ना॒य्यं ॅयत् । \newline
9. सा॒न्ना॒य्यं ॅयद् यथ् सा᳚न्ना॒य्यꣳ सा᳚न्ना॒य्यं ॅयथ् सा᳚न्ना॒य्यꣳ सा᳚न्ना॒य्यं ॅयथ् सा᳚न्ना॒य्यꣳ सा᳚न्ना॒य्यं ॅयथ् सा᳚न्ना॒य्यम् । \newline
10. सा॒न्ना॒य्यमिति॑ सां - ना॒य्यम् । \newline
11. यथ् सा᳚न्ना॒य्यꣳ सा᳚न्ना॒य्यं ॅयद् यथ् सा᳚न्ना॒य्यम् भव॑ति॒ भव॑ति सान्ना॒य्यं ॅयद् यथ् सा᳚न्ना॒य्यम् भव॑ति । \newline
12. सा॒न्ना॒य्यम् भव॑ति॒ भव॑ति सान्ना॒य्यꣳ सा᳚न्ना॒य्यम् भव॑ति॒ तेन॒ तेन॒ भव॑ति सान्ना॒य्यꣳ सा᳚न्ना॒य्यम् भव॑ति॒ तेन॑ । \newline
13. सा॒न्ना॒य्यमिति॑ सां - ना॒य्यम् । \newline
14. भव॑ति॒ तेन॒ तेन॒ भव॑ति॒ भव॑ति॒ तेनै॒वैव तेन॒ भव॑ति॒ भव॑ति॒ तेनै॒व । \newline
15. तेनै॒ वैव तेन॒ तेनै॒व सव॑ना॒थ् सव॑ना दे॒व तेन॒ तेनै॒व सव॑नात् । \newline
16. ए॒व सव॑ना॒थ् सव॑ना दे॒वैव सव॑ना॒न् न न सव॑ना दे॒वैव सव॑ना॒न् न । \newline
17. सव॑ना॒न् न न सव॑ना॒थ् सव॑ना॒न् न य॑न्ति यन्ति॒ न सव॑ना॒थ् सव॑ना॒न् न य॑न्ति । \newline
18. न य॑न्ति यन्ति॒ न न य॑न्ति समुप॒हूय॑ समुप॒हूय॑ यन्ति॒ न न य॑न्ति समुप॒हूय॑ । \newline
19. य॒न्ति॒ स॒मु॒प॒हूय॑ समुप॒हूय॑ यन्ति यन्ति समुप॒हूय॑ भक्षयन्ति भक्षयन्ति समुप॒हूय॑ यन्ति यन्ति समुप॒हूय॑ भक्षयन्ति । \newline
20. स॒मु॒प॒हूय॑ भक्षयन्ति भक्षयन्ति समुप॒हूय॑ समुप॒हूय॑ भक्षय न्त्ये॒तथ्सो॑मपीथा ए॒तथ्सो॑मपीथा भक्षयन्ति समुप॒हूय॑ समुप॒हूय॑ भक्षय न्त्ये॒तथ्सो॑मपीथाः । \newline
21. स॒मु॒प॒हूयेति॑ सं - उ॒प॒हूय॑ । \newline
22. भ॒क्ष॒य॒ न्त्ये॒तथ्सो॑मपीथा ए॒तथ्सो॑मपीथा भक्षयन्ति भक्षय न्त्ये॒तथ्सो॑मपीथा॒ हि ह्ये॑तथ्सो॑मपीथा भक्षयन्ति भक्षय न्त्ये॒तथ्सो॑मपीथा॒ हि । \newline
23. ए॒तथ्सो॑मपीथा॒ हि ह्ये॑तथ्सो॑मपीथा ए॒तथ्सो॑मपीथा॒ ह्ये॑तर् ह्ये॒तर्. हि॒ ह्ये॑तथ्सो॑मपीथा ए॒तथ्सो॑मपीथा॒ ह्ये॑तर्.हि॑ । \newline
24. ए॒तथ्सो॑मपीथा॒ इत्ये॒तत् - सो॒म॒पी॒थाः॒ । \newline
25. ह्ये॑तर् ह्ये॒तर्.हि॒ हि ह्ये॑तर्.हि॑ यथायत॒नं ॅय॑थायत॒न मे॒तर्.हि॒ हि ह्ये॑तर्.हि॑ यथायत॒नम् । \newline
26. ए॒तर्.हि॑ यथायत॒नं ॅय॑थायत॒न मे॒तर् ह्ये॒तर्.हि॑ यथायत॒नं ॅवै वै य॑थायत॒न मे॒तर् ह्ये॒तर्.हि॑ यथायत॒नं ॅवै । \newline
27. य॒था॒य॒त॒नं ॅवै वै य॑थायत॒नं ॅय॑थायत॒नं ॅवा ए॒तेषा॑ मे॒तेषां॒ ॅवै य॑थायत॒नं ॅय॑थायत॒नं ॅवा ए॒तेषा᳚म् । \newline
28. य॒था॒य॒त॒नमिति॑ यथा - अ॒य॒त॒नम् । \newline
29. वा ए॒तेषा॑ मे॒तेषां॒ ॅवै वा ए॒तेषाꣳ॑ सवन॒भाजः॑ सवन॒भाज॑ ए॒तेषां॒ ॅवै वा ए॒तेषाꣳ॑ सवन॒भाजः॑ । \newline
30. ए॒तेषाꣳ॑ सवन॒भाजः॑ सवन॒भाज॑ ए॒तेषा॑ मे॒तेषाꣳ॑ सवन॒भाजो॑ दे॒वता॑ दे॒वताः᳚ सवन॒भाज॑ ए॒तेषा॑ मे॒तेषाꣳ॑ सवन॒भाजो॑ दे॒वताः᳚ । \newline
31. स॒व॒न॒भाजो॑ दे॒वता॑ दे॒वताः᳚ सवन॒भाजः॑ सवन॒भाजो॑ दे॒वता॑ गच्छन्ति गच्छन्ति दे॒वताः᳚ सवन॒भाजः॑ सवन॒भाजो॑ दे॒वता॑ गच्छन्ति । \newline
32. स॒व॒न॒भाज॒ इति॑ सवन - भाजः॑ । \newline
33. दे॒वता॑ गच्छन्ति गच्छन्ति दे॒वता॑ दे॒वता॑ गच्छन्ति॒ ये ये ग॑च्छन्ति दे॒वता॑ दे॒वता॑ गच्छन्ति॒ ये । \newline
34. ग॒च्छ॒न्ति॒ ये ये ग॑च्छन्ति गच्छन्ति॒ ये ऽह॒ रह॒र् ये ग॑च्छन्ति गच्छन्ति॒ ये ऽहः॑ । \newline
35. ये ऽह॒ रह॒र् ये ये ऽह॑ रुथ्सृ॒ज न्त्यु॑थ्सृ॒ज न्त्यह॒र् ये ये ऽह॑ रुथ्सृ॒जन्ति॑ । \newline
36. अह॑ रुथ्सृ॒ज न्त्यु॑थ्सृ॒ज न्त्यह॒ रह॑ रुथ्सृ॒ज न्त्य॑नुसव॒न म॑नुसव॒न मु॑थ्सृ॒ज न्त्यह॒ रह॑ रुथ्सृ॒ज न्त्य॑नुसव॒नम् । \newline
37. उ॒थ्सृ॒ज न्त्य॑नुसव॒न म॑नुसव॒न मु॑थ्सृ॒ज न्त्यु॑थ्सृ॒ज न्त्य॑नुसव॒नम् पु॑रो॒डाशा᳚न् पुरो॒डाशा॑ ननुसव॒न मु॑थ्सृ॒ज न्त्यु॑थ्सृ॒ज न्त्य॑नुसव॒नम् पु॑रो॒डाशान्॑ । \newline
38. उ॒थ्सृ॒जन्तीत्यु॑त् - सृ॒जन्ति॑ । \newline
39. अ॒नु॒स॒व॒नम् पु॑रो॒डाशा᳚न् पुरो॒डाशा॑ ननुसव॒न म॑नुसव॒नम् पु॑रो॒डाशा॒न् निर् णिष् पु॑रो॒डाशा॑ ननुसव॒न म॑नुसव॒नम् पु॑रो॒डाशा॒न् निः । \newline
40. अ॒नु॒स॒व॒नमित्य॑नु - स॒व॒नम् । \newline
41. पु॒रो॒डाशा॒न् निर् णिष् पु॑रो॒डाशा᳚न् पुरो॒डाशा॒न् निर् व॑पन्ति वपन्ति॒ निष् पु॑रो॒डाशा᳚न् पुरो॒डाशा॒न् निर् व॑पन्ति । \newline
42. निर् व॑पन्ति वपन्ति॒ निर् णिर् व॑पन्ति यथायत॒नाद् य॑थायत॒नाद् व॑पन्ति॒ निर् णिर् व॑पन्ति यथायत॒नात् । \newline
43. व॒प॒न्ति॒ य॒था॒य॒त॒नाद् य॑थायत॒नाद् व॑पन्ति वपन्ति यथायत॒ना दे॒वैव य॑थायत॒नाद् व॑पन्ति वपन्ति यथायत॒ना दे॒व । \newline
44. य॒था॒य॒त॒ना दे॒वैव य॑थायत॒नाद् य॑थायत॒ना दे॒व स॑वन॒भाजः॑ सवन॒भाज॑ ए॒व य॑थायत॒नाद् य॑थायत॒ना दे॒व स॑वन॒भाजः॑ । \newline
45. य॒था॒य॒त॒नादिति॑ यथा - आ॒य॒त॒नात् । \newline
46. ए॒व स॑वन॒भाजः॑ सवन॒भाज॑ ए॒वैव स॑वन॒भाजो॑ दे॒वता॑ दे॒वताः᳚ सवन॒भाज॑ ए॒वैव स॑वन॒भाजो॑ दे॒वताः᳚ । \newline
47. स॒व॒न॒भाजो॑ दे॒वता॑ दे॒वताः᳚ सवन॒भाजः॑ सवन॒भाजो॑ दे॒वता॒ अवाव॑ दे॒वताः᳚ सवन॒भाजः॑ सवन॒भाजो॑ दे॒वता॒ अव॑ । \newline
48. स॒व॒न॒भाज॒ इति॑ सवन - भाजः॑ । \newline
49. दे॒वता॒ अवाव॑ दे॒वता॑ दे॒वता॒ अव॑ रुन्धते रुन्ध॒ते ऽव॑ दे॒वता॑ दे॒वता॒ अव॑ रुन्धते । \newline
50. अव॑ रुन्धते रुन्ध॒ते ऽवाव॑ रुन्धते॒ ऽष्टाक॑पाला न॒ष्टाक॑पालान् रुन्ध॒ते ऽवाव॑ रुन्धते॒ ऽष्टाक॑पालान् । \newline
51. रु॒न्ध॒ते॒ ऽष्टाक॑पाला न॒ष्टाक॑पालान् रुन्धते रुन्धते॒ ऽष्टाक॑पालान् प्रातस्सव॒ने प्रा॑तस्सव॒ने᳚ ऽष्टाक॑पालान् रुन्धते रुन्धते॒ ऽष्टाक॑पालान् प्रातस्सव॒ने । \newline
52. अ॒ष्टाक॑पालान् प्रातस्सव॒ने प्रा॑तस्सव॒ने᳚ ऽष्टाक॑पाला न॒ष्टाक॑पालान् प्रातस्सव॒न एका॑दशकपाला॒ नेका॑दशकपालान् प्रातस्सव॒ने᳚ ऽष्टाक॑पाला न॒ष्टाक॑पालान् प्रातस्सव॒न एका॑दशकपालान् । \newline
53. अ॒ष्टाक॑पाला॒नित्य॒ष्टा - क॒पा॒ला॒न् । \newline
54. प्रा॒त॒स्स॒व॒न एका॑दशकपाला॒ नेका॑दशकपालान् प्रातस्सव॒ने प्रा॑तस्सव॒न एका॑दशकपाला॒न् माद्ध्य॑न्दिने॒ माद्ध्य॑न्दिन॒ एका॑दशकपालान् प्रातस्सव॒ने प्रा॑तस्सव॒न एका॑दशकपाला॒न् माद्ध्य॑न्दिने । \newline
55. प्रा॒त॒स्स॒व॒न इति॑ प्रातः - स॒व॒ने । \newline
56. एका॑दशकपाला॒न् माद्ध्य॑न्दिने॒ माद्ध्य॑न्दिन॒ एका॑दशकपाला॒ नेका॑दशकपाला॒न् माद्ध्य॑न्दिने॒ सव॑ने॒ सव॑ने॒ माद्ध्य॑न्दिन॒ एका॑दशकपाला॒ नेका॑दशकपाला॒न् माद्ध्य॑न्दिने॒ सव॑ने । \newline
57. एका॑दशकपाला॒नित्येका॑दश - क॒पा॒ला॒न् । \newline
58. माद्ध्य॑न्दिने॒ सव॑ने॒ सव॑ने॒ माद्ध्य॑न्दिने॒ माद्ध्य॑न्दिने॒ सव॑ने॒ द्वाद॑शकपाला॒न् द्वाद॑शकपाला॒न् थ्सव॑ने॒ माद्ध्य॑न्दिने॒ माद्ध्य॑न्दिने॒ सव॑ने॒ द्वाद॑शकपालान् । \newline
59. सव॑ने॒ द्वाद॑शकपाला॒न् द्वाद॑शकपाला॒न् थ्सव॑ने॒ सव॑ने॒ द्वाद॑शकपालाꣳ स्तृतीयसव॒ने तृ॑तीयसव॒ने द्वाद॑शकपाला॒न् थ्सव॑ने॒ सव॑ने॒ द्वाद॑शकपालाꣳ स्तृतीयसव॒ने । \newline
60. द्वाद॑शकपालाꣳ स्तृतीयसव॒ने तृ॑तीयसव॒ने द्वाद॑शकपाला॒न् द्वाद॑शकपालाꣳ स्तृतीयसव॒ने छन्दाꣳ॑सि॒ छन्दाꣳ॑सि तृतीयसव॒ने द्वाद॑शकपाला॒न् द्वाद॑शकपालाꣳ स्तृतीयसव॒ने छन्दाꣳ॑सि । \newline
61. द्वाद॑शकपाला॒निति॒ द्वाद॑श - क॒पा॒ला॒न् । \newline
62. तृ॒ती॒य॒स॒व॒ने छन्दाꣳ॑सि॒ छन्दाꣳ॑सि तृतीयसव॒ने तृ॑तीयसव॒ने छन्दाꣳ॑ स्ये॒वैव छन्दाꣳ॑सि तृतीयसव॒ने तृ॑तीयसव॒ने छन्दाꣳ॑ स्ये॒व । \newline
63. तृ॒ती॒य॒स॒व॒न इति॑ तृतीय - स॒व॒ने । \newline
64. छन्दाꣳ॑ स्ये॒वैव छन्दाꣳ॑सि॒ छन्दाꣳ॑ स्ये॒वाप्त्वा ऽऽप्त्वैव छन्दाꣳ॑सि॒ छन्दाꣳ॑ स्ये॒वाप्त्वा । \newline
65. ए॒वाप्त्वा ऽऽप्त्वैवै वाप्त्वा ऽवावा॒ प्त्वैवै वाप्त्वा ऽव॑ । \newline
66. आ॒प्त्वा ऽवावा॒प्त्वा ऽऽप्त्वा ऽव॑ रुन्धते रुन्ध॒ते ऽवा॒प्त्वा ऽऽप्त्वा ऽव॑ रुन्धते । \newline
67. अव॑ रुन्धते रुन्ध॒ते ऽवाव॑ रुन्धते वैश्वदे॒वं ॅवै᳚श्वदे॒वꣳ रु॑न्ध॒ते ऽवाव॑ रुन्धते वैश्वदे॒वम् । \newline
68. रु॒न्ध॒ते॒ वै॒श्व॒दे॒वं ॅवै᳚श्वदे॒वꣳ रु॑न्धते रुन्धते वैश्वदे॒वम् च॒रुम् च॒रुं ॅवै᳚श्वदे॒वꣳ रु॑न्धते रुन्धते वैश्वदे॒वम् च॒रुम् । \newline
69. वै॒श्व॒दे॒वम् च॒रुम् च॒रुं ॅवै᳚श्वदे॒वं ॅवै᳚श्वदे॒वम् च॒रुम् तृ॑तीयसव॒ने तृ॑तीयसव॒ने च॒रुं ॅवै᳚श्वदे॒वं ॅवै᳚श्वदे॒वम् च॒रुम् तृ॑तीयसव॒ने । \newline
70. वै॒श्व॒दे॒वमिति॑ वैश्व - दे॒वम् । \newline
71. च॒रुम् तृ॑तीयसव॒ने तृ॑तीयसव॒ने च॒रुम् च॒रुम् तृ॑तीयसव॒ने निर् णिष् टृ॑तीयसव॒ने च॒रुम् च॒रुम् तृ॑तीयसव॒ने निः । \newline
72. तृ॒ती॒य॒स॒व॒ने निर् णिष् टृ॑तीयसव॒ने तृ॑तीयसव॒ने निर् व॑पन्ति वपन्ति॒ निष् टृ॑तीयसव॒ने तृ॑तीयसव॒ने निर् व॑पन्ति । \newline
73. तृ॒ती॒य॒स॒व॒न इति॑ तृतीय - स॒व॒ने । \newline
74. निर् व॑पन्ति वपन्ति॒ निर् णिर् व॑पन्ति वैश्वदे॒वं ॅवै᳚श्वदे॒वं ॅव॑पन्ति॒ निर् णिर् व॑पन्ति वैश्वदे॒वम् । \newline
75. व॒प॒न्ति॒ वै॒श्व॒दे॒वं ॅवै᳚श्वदे॒वं ॅव॑पन्ति वपन्ति वैश्वदे॒वं ॅवै वै वै᳚श्वदे॒वं ॅव॑पन्ति वपन्ति वैश्वदे॒वं ॅवै । \newline
76. वै॒श्व॒दे॒वं ॅवै वै वै᳚श्वदे॒वं ॅवै᳚श्वदे॒वं ॅवै तृ॑तीयसव॒नम् तृ॑तीयसव॒नं ॅवै वै᳚श्वदे॒वं ॅवै᳚श्वदे॒वं ॅवै तृ॑तीयसव॒नम् । \newline
77. वै॒श्व॒दे॒वमिति॑ वैश्व - दे॒वम् । \newline
78. वै तृ॑तीयसव॒नम् तृ॑तीयसव॒नं ॅवै वै तृ॑तीयसव॒नम् तेन॒ तेन॑ तृतीयसव॒नं ॅवै वै तृ॑तीयसव॒नम् तेन॑ । \newline
79. तृ॒ती॒य॒स॒व॒नम् तेन॒ तेन॑ तृतीयसव॒नम् तृ॑तीयसव॒नम् तेनै॒ वैव तेन॑ तृतीयसव॒नम् तृ॑तीयसव॒नम् तेनै॒व । \newline
80. तृ॒ती॒य॒स॒व॒नमिति॑ तृतीय - स॒व॒नम् । \newline
81. तेनै॒ वैव तेन॒ तेनै॒व तृ॑तीयसव॒नात् तृ॑तीयसव॒ना दे॒व तेन॒ तेनै॒व तृ॑तीयसव॒नात् । \newline
82. ए॒व तृ॑तीयसव॒नात् तृ॑तीयसव॒ना दे॒वैव तृ॑तीयसव॒नान् न न तृ॑तीयसव॒ना दे॒वैव तृ॑तीयसव॒नान् न । \newline
83. तृ॒ती॒य॒स॒व॒नान् न न तृ॑तीयसव॒नात् तृ॑तीयसव॒नान् न य॑न्ति यन्ति॒ न तृ॑तीयसव॒नात् तृ॑तीयसव॒नान् न य॑न्ति । \newline
84. तृ॒ती॒य॒स॒व॒नादिति॑ तृतीय - स॒व॒नात् । \newline
85. न य॑न्ति यन्ति॒ न न य॑न्ति । \newline
86. य॒न्तीति॑ यन्ति । \newline
\pagebreak
\markright{ TS 7.5.7.1  \hfill https://www.vedavms.in \hfill}

\section{ TS 7.5.7.1 }

\textbf{TS 7.5.7.1 } \newline
\textbf{Samhita Paata} \newline

उ॒थ्सृज्यां(3)नोथ्सृज्या(3)मिति॑ मीमाꣳसन्ते ब्रह्मवा॒दिन॒-स्तद्वा॑हुरु॒थ् सृज्य॑मे॒वेत्य॑-मावा॒स्या॑यां च पौर्णमा॒स्यां चो॒थ्-सृज्य॒मित्या॑हुरे॒ते हि य॒ज्ञ्ं ॅवह॑त॒ इति॒ ते त्वाव नोथ्सृज्ये॒ इत्या॑हु॒र्ये अ॑वान्त॒रं ॅय॒ज्ञ्ं भे॒जाते॒ इति॒ या प्र॑थ॒मा व्य॑ष्टका॒ तस्या॑मु॒थ्-सृज्य॒मित्या॑हुरे॒ष वै मा॒सो वि॑श॒र इति॒ नाऽऽ*दि॑ष्ट॒ - [  ] \newline

\textbf{Pada Paata} \newline

उ॒थ्सृज्या(3)मित्यु॑त् - सृज्या(3)म् । न । उ॒थ्सृज्या(3)मित्यु॑त्- सृज्या(3)म् । इति॑ । मी॒माꣳ॒॒स॒न्ते॒ । ब्र॒ह्म॒वा॒दिन॒ इति॑ ब्रह्म - वा॒दिनः॑ । तत् । उ॒ । आ॒हुः॒ । उ॒थ्सृज्य॒मित्यु॑त् - सृज्य᳚म् । ए॒व । इति॑ । अ॒मा॒वा॒स्या॑या॒मित्य॑मा - वा॒स्या॑याम् । च॒ । पौ॒र्ण॒मा॒स्यामिति॑ पौर्ण - मा॒स्याम् । च॒ । उ॒थ्सृज्य॒मित्यु॑त् - सृज्य᳚म् । इति॑ । आ॒हुः॒ । ए॒ते इति॑ । हि । य॒ज्ञ्म् । वह॑तः । इति॑ । ते इति॑ । तु । वाव । न । उ॒थ्सृज्ये॒ इत्यु॑त् - सृज्ये᳚ । इति॑ । आ॒हुः॒ । ये इति॑ । अ॒वा॒न्त॒रमित्य॑व - अ॒न्त॒रम् । य॒ज्ञ्म् । भे॒जाते॒ इति॑ । इति॑ । या । प्र॒थ॒मा । व्य॑ष्ट॒केति॒ वि - अ॒ष्ट॒का॒ । तस्या᳚म् । उ॒थ्सृज्य॒मित्यु॑त् - सृज्य᳚म् । इति॑ । आ॒हुः॒ । ए॒षः । वै । मा॒सः । वि॒श॒र इति॑ वि - श॒रः । इति॑ । न । आदि॑ष्ट॒मित्या - दि॒ष्ट॒म् ।  \newline


\textbf{Krama Paata} \newline

उ॒थ्सृज्या(3)म् न । उ॒थ्सृज्या(3)मित्यु॑त् - सृज्या(3)म् । नोथ्सृज्या(3)म् । उ॒थ्सृज्या(3)मिति॑ । उ॒थ्सृज्या(3)मित्यु॑त् - सृज्या(3)म् । इति॑ मीमाꣳसन्ते । मी॒माꣳ॒॒स॒न्ते॒ ब्र॒ह्म॒वा॒दिनः॑ । ब्र॒ह्म॒वा॒दिन॒स्तत् । ब्र॒ह्म॒वा॒दिन॒ इति॑ ब्रह्म - वा॒दिनः॑ । तदु॑ । वा॒हुः॒ । आ॒हु॒रु॒थ्सृज्य᳚म् । उ॒थ्सृज्य॑मे॒व । उ॒थ्सृज्य॒मित्यु॑त् - सृज्य᳚म् । ए॒वेति॑ । इत्य॑मावा॒स्या॑याम् । अ॒मा॒वा॒स्या॑याम् च । अ॒मा॒वा॒स्या॑या॒मित्य॑मा - वा॒स्या॑याम् । च॒ पौ॒र्ण॒मा॒स्याम् । पौ॒र्ण॒मा॒स्याम् च॑ । पौ॒र्ण॒मा॒स्यामिति॑ पौर्ण - मा॒स्याम् । चो॒थ्सृज्य᳚म् । उ॒थ्सृज्य॒मिति॑ । उ॒थ्सृज्य॒मित्यु॑त् - सृज्य᳚म् । इत्या॑हुः । आ॒हु॒रे॒ते । ए॒ते हि । ए॒ते इत्ये॒ते । हि य॒ज्ञ्म् । य॒ज्ञ्म् ॅवह॑तः । वह॑त॒ इति॑ । इति॒ ते । ते तु । ते इति॒ ते । त्वाव । वाव न । नोथ्सृज्ये᳚ । उ॒थ्सृज्ये॒ इति॑ । उ॒थ्सृज्ये॒ इत्यु॑त् - सृज्ये᳚ । इत्या॑हुः । आ॒हु॒र् ये । ये अ॑वान्त॒रम् । ये इति॒ ये । अ॒वा॒न्त॒रम् ॅय॒ज्ञ्म् । अ॒वा॒न्त॒रमित्य॑व - अ॒न्त॒रम् । य॒ज्ञ्म् भे॒जाते᳚ । भे॒जाते॒ इति॑ । भे॒जाते॒ इति॑ भे॒जाते᳚ । इति॒ या । या प्र॑थ॒मा । प्र॒थ॒मा व्य॑ष्टका । व्य॑ष्टका॒ तस्या᳚म् । व्य॑ष्ट॒केति॒ वि - अ॒ष्ट॒का॒ । तस्या॑मु॒थ्सृज्य᳚म् । उ॒थ्सृज्य॒मिति॑ । उ॒थ्सृज्य॒मित्यु॑त् - सृज्य᳚म् । इत्या॑हुः । आ॒हु॒रे॒षः । ए॒ष वै । वै मा॒सः । मा॒सो वि॑श॒रः । वि॒श॒र इति॑ । वि॒श॒र इति॑ वि - श॒रः । इति॒ न । नादि॑ष्टम् । आदि॑ष्ट॒मुत् । आदि॑ष्ट॒मित्या - दि॒ष्ट॒म् \newline

\textbf{Jatai Paata} \newline

1. उ॒थ्सृज्या(3)न् न नोथ्सृज्या(3) मु॒थ्सृज्या(3)न् न । \newline
2. उ॒थ्सृज्या(3)मित्यु॑त् - सृज्या(3)म् । \newline
3. नोथ्सृज्या(3) मु॒थ्सृज्या(3)न् न नोथ्सृज्या(3)म् । \newline
4. उ॒थ्सृज्या(3) मिती त्यु॒थ्सृज्या(3) मु॒थ्सृज्या(3) मिति॑ । \newline
5. उ॒थ्सृज्या(3)मित्यु॑त् - सृज्या(3)म् । \newline
6. इति॑ मीमाꣳसन्ते मीमाꣳसन्त॒ इतीति॑ मीमाꣳसन्ते । \newline
7. मी॒माꣳ॒॒स॒न्ते॒ ब्र॒ह्म॒वा॒दिनो᳚ ब्रह्मवा॒दिनो॑ मीमाꣳसन्ते मीमाꣳसन्ते ब्रह्मवा॒दिनः॑ । \newline
8. ब्र॒ह्म॒वा॒दिन॒ स्तत् तद् ब्र॑ह्मवा॒दिनो᳚ ब्रह्मवा॒दिन॒ स्तत् । \newline
9. ब्र॒ह्म॒वा॒दिन॒ इति॑ ब्रह्म - वा॒दिनः॑ । \newline
10. तदू॒ तत् तदु॑ । \newline
11. उ॒ वा॒हु॒ रा॒हु॒ रु॒ वु॒ वा॒हुः॒ । \newline
12. आ॒हु॒ रु॒थ्सृज्य॑ मु॒थ्सृज्य॑ माहु राहु रु॒थ्सृज्य᳚म् । \newline
13. उ॒थ्सृज्य॑ मे॒वै वोथ्सृज्य॑ मु॒थ्सृज्य॑ मे॒व । \newline
14. उ॒थ्सृज्य॒मित्यु॑त् - सृज्य᳚म् । \newline
15. ए॒वेती त्ये॒वैवेति॑ । \newline
16. इत्य॑ मावा॒स्या॑या ममावा॒स्या॑या॒ मिती त्य॑मावा॒स्या॑याम् । \newline
17. अ॒मा॒वा॒स्या॑याम् च चामावा॒स्या॑या ममावा॒स्या॑याम् च । \newline
18. अ॒मा॒वा॒स्या॑या॒मित्य॑मा - वा॒स्या॑याम् । \newline
19. च॒ पौ॒र्ण॒मा॒स्याम् पौ᳚र्णमा॒स्याम् च॑ च पौर्णमा॒स्याम् । \newline
20. पौ॒र्ण॒मा॒स्याम् च॑ च पौर्णमा॒स्याम् पौ᳚र्णमा॒स्याम् च॑ । \newline
21. पौ॒र्ण॒मा॒स्यामिति॑ पौर्ण - मा॒स्याम् । \newline
22. चो॒थ्सृज्य॑ मु॒थ्सृज्य॑म् च चो॒थ्सृज्य᳚म् । \newline
23. उ॒थ्सृज्य॒ मिती त्यु॒थ्सृज्य॑ मु॒थ्सृज्य॒ मिति॑ । \newline
24. उ॒थ्सृज्य॒मित्यु॑त् - सृज्य᳚म् । \newline
25. इत्या॑हु राहु॒ रिती त्या॑हुः । \newline
26. आ॒हु॒ रे॒ते ए॒ते आ॑हु राहु रे॒ते । \newline
27. ए॒ते हि ह्ये॑ते ए॒ते हि । \newline
28. ए॒ते इत्ये॒ते । \newline
29. हि य॒ज्ञ्ं ॅय॒ज्ञ्ꣳ हि हि य॒ज्ञ्म् । \newline
30. य॒ज्ञ्ं ॅवह॑तो॒ वह॑तो य॒ज्ञ्ं ॅय॒ज्ञ्ं ॅवह॑तः । \newline
31. वह॑त॒ इतीति॒ वह॑तो॒ वह॑त॒ इति॑ । \newline
32. इति॒ ते ते इतीति॒ ते । \newline
33. ते तु तु ते ते तु । \newline
34. ते इति॒ ते । \newline
35. त्वाव वाव तु त्वाव । \newline
36. वाव न न वाव वाव न । \newline
37. नोथ्सृज्ये॑ उ॒थ्सृज्ये॒ न नोथ्सृज्ये᳚ । \newline
38. उ॒थ्सृज्ये॒ इतीत्यु॒थ्सृज्ये॑ उ॒थ्सृज्ये॒ इति॑ । \newline
39. उ॒थ्सृज्ये॒ इत्यु॑त् - सृज्ये᳚ । \newline
40. इत्या॑हु राहु॒ रिती त्या॑हुः । \newline
41. आ॒हु॒र् ये ये आ॑हु राहु॒र् ये । \newline
42. ये अ॑वान्त॒र म॑वान्त॒रं ॅये ये अ॑वान्त॒रम् । \newline
43. ये इति॒ ये । \newline
44. अ॒वा॒न्त॒रं ॅय॒ज्ञ्ं ॅय॒ज्ञ् म॑वान्त॒र म॑वान्त॒रं ॅय॒ज्ञ्म् । \newline
45. अ॒वा॒न्त॒रमित्य॑व - अ॒न्त॒रम् । \newline
46. य॒ज्ञ्म् भे॒जाते॑ भे॒जाते॑ य॒ज्ञ्ं ॅय॒ज्ञ्म् भे॒जाते᳚ । \newline
47. भे॒जाते॒ इतीति॑ भे॒जाते॑ भे॒जाते॒ इति॑ । \newline
48. भे॒जाते॒ इति॑ भे॒जाते᳚ । \newline
49. इति॒ या येतीति॒ या । \newline
50. या प्र॑थ॒मा प्र॑थ॒मा या या प्र॑थ॒मा । \newline
51. प्र॒थ॒मा व्य॑ष्टका॒ व्य॑ष्टका प्रथ॒मा प्र॑थ॒मा व्य॑ष्टका । \newline
52. व्य॑ष्टका॒ तस्या॒म् तस्यां॒ ॅव्य॑ष्टका॒ व्य॑ष्टका॒ तस्या᳚म् । \newline
53. व्य॑ष्ट॒केति॒ वि - अ॒ष्ट॒का॒ । \newline
54. तस्या॑ मु॒थ्सृज्य॑ मु॒थ्सृज्य॒म् तस्या॒म् तस्या॑ मु॒थ्सृज्य᳚म् । \newline
55. उ॒थ्सृज्य॒ मिती त्यु॒थ्सृज्य॑ मु॒थ्सृज्य॒ मिति॑ । \newline
56. उ॒थ्सृज्य॒मित्यु॑त् - सृज्य᳚म् । \newline
57. इत्या॑हु राहु॒ रिती त्या॑हुः । \newline
58. आ॒हु॒ रे॒ष ए॒ष आ॑हु राहु रे॒षः । \newline
59. ए॒ष वै वा ए॒ष ए॒ष वै । \newline
60. वै मा॒सो मा॒सो वै वै मा॒सः । \newline
61. मा॒सो वि॑श॒रो वि॑श॒रो मा॒सो मा॒सो वि॑श॒रः । \newline
62. वि॒श॒र इतीति॑ विश॒रो वि॑श॒र इति॑ । \newline
63. वि॒श॒र इति॑ वि - श॒रः । \newline
64. इति॒ न नेतीति॒ न । \newline
65. नादि॑ष्ट॒ मादि॑ष्ट॒न् न नादि॑ष्टम् । \newline
66. आदि॑ष्ट॒ मुदु दादि॑ष्ट॒ मादि॑ष्ट॒ मुत् । \newline
67. आदि॑ष्ट॒मित्या - दि॒ष्ट॒म् । \newline

\textbf{Ghana Paata } \newline

1. उ॒थ्सृज्या(3)न् न नोथ्सृज्या(3) मु॒थ्सृज्या(3)न् नोथ्सृज्या(3) मु॒थ्सृज्या(3)न् नोथ्सृज्या(3) मु॒थ्सृज्या(3)न् नोथ्सृज्या(3)म् । \newline
2. उ॒थ्सृज्या(3)मित्यु॑त् - सृज्या(3)म् । \newline
3. नोथ्सृज्या(3) मु॒थ्सृज्या(3)न् न नोथ्सृज्या(3) मिती त्यु॒थ्सृज्या(3)न् न नोथ्सृज्या(3) मिति॑ । \newline
4. उ॒थ्सृज्या(3) मिती त्यु॒थ्सृज्या(3) मु॒थ्सृज्या(3) मिति॑ मीमाꣳसन्ते मीमाꣳसन्त॒ इत्यु॒थ्सृज्या(3) मु॒थ्सृज्या(3) मिति॑ मीमाꣳसन्ते । \newline
5. उ॒थ्सृज्या(3)मित्यु॑त् - सृज्या(3)म् । \newline
6. इति॑ मीमाꣳसन्ते मीमाꣳसन्त॒ इतीति॑ मीमाꣳसन्ते ब्रह्मवा॒दिनो᳚ ब्रह्मवा॒दिनो॑ मीमाꣳसन्त॒ इतीति॑ मीमाꣳसन्ते ब्रह्मवा॒दिनः॑ । \newline
7. मी॒माꣳ॒॒स॒न्ते॒ ब्र॒ह्म॒वा॒दिनो᳚ ब्रह्मवा॒दिनो॑ मीमाꣳसन्ते मीमाꣳसन्ते ब्रह्मवा॒दिन॒ स्तत् तद् ब्र॑ह्मवा॒दिनो॑ मीमाꣳसन्ते मीमाꣳसन्ते ब्रह्मवा॒दिन॒ स्तत् । \newline
8. ब्र॒ह्म॒वा॒दिन॒ स्तत् तद् ब्र॑ह्मवा॒दिनो᳚ ब्रह्मवा॒दिन॒ स्तदू॒ तद् ब्र॑ह्मवा॒दिनो᳚ ब्रह्मवा॒दिन॒ स्तदु॑ । \newline
9. ब्र॒ह्म॒वा॒दिन॒ इति॑ ब्रह्म - वा॒दिनः॑ । \newline
10. तदू॒ तत् तद् वा॑हु राहु रु॒तत् तद् वा॑हुः । \newline
11. उ॒ वा॒हु॒ रा॒हु॒ रु॒ वु॒ वा॒हु॒ रु॒थ्सृज्य॑ मु॒थ्सृज्य॑ माहु रु वु वाहु रु॒थ्सृज्य᳚म् । \newline
12. आ॒हु॒ रु॒थ्सृज्य॑ मु॒थ्सृज्य॑ माहु राहु रु॒थ्सृज्य॑ मे॒वै वोथ्सृज्य॑ माहु राहु रु॒थ्सृज्य॑ मे॒व । \newline
13. उ॒थ्सृज्य॑ मे॒वै वोथ्सृज्य॑ मु॒थ्सृज्य॑ मे॒वे तीत्ये॒वो थ्सृज्य॑ मु॒थ्सृज्य॑ मे॒वेति॑ । \newline
14. उ॒थ्सृज्य॒मित्यु॑त् - सृज्य᳚म् । \newline
15. ए॒वेती त्ये॒वैवे त्य॑मावा॒स्या॑या ममावा॒स्या॑या॒ मित्ये॒ वैवे त्य॑मावा॒स्या॑याम् । \newline
16. इत्य॑मावा॒स्या॑या ममावा॒स्या॑या॒ मिती त्य॑मावा॒स्या॑याम् च चामावा॒स्या॑या॒ मिती त्य॑मावा॒स्या॑याम् च । \newline
17. अ॒मा॒वा॒स्या॑याम् च चामावा॒स्या॑या ममावा॒स्या॑याम् च पौर्णमा॒स्याम् पौ᳚र्णमा॒स्याम् चा॑मावा॒स्या॑या ममावा॒स्या॑याम् च पौर्णमा॒स्याम् । \newline
18. अ॒मा॒वा॒स्या॑या॒मित्य॑मा - वा॒स्या॑याम् । \newline
19. च॒ पौ॒र्ण॒मा॒स्याम् पौ᳚र्णमा॒स्याम् च॑ च पौर्णमा॒स्याम् च॑ च पौर्णमा॒स्याम् च॑ च पौर्णमा॒स्याम् च॑ । \newline
20. पौ॒र्ण॒मा॒स्याम् च॑ च पौर्णमा॒स्याम् पौ᳚र्णमा॒स्याम् चो॒थ्सृज्य॑ मु॒थ्सृज्य॑म् च पौर्णमा॒स्याम् पौ᳚र्णमा॒स्याम् चो॒थ्सृज्य᳚म् । \newline
21. पौ॒र्ण॒मा॒स्यामिति॑ पौर्ण - मा॒स्याम् । \newline
22. चो॒थ्सृज्य॑ मु॒थ्सृज्य॑म् च चो॒थ्सृज्य॒ मिती त्यु॒थ्सृज्य॑म् च चो॒थ्सृज्य॒ मिति॑ । \newline
23. उ॒थ्सृज्य॒ मिती त्यु॒थ्सृज्य॑ मु॒थ्सृज्य॒ मित्या॑हु राहु॒ रित्यु॒थ्सृज्य॑ मु॒थ्सृज्य॒ मित्या॑हुः । \newline
24. उ॒थ्सृज्य॒मित्यु॑त् - सृज्य᳚म् । \newline
25. इत्या॑हु राहु॒ रिती त्या॑हु रे॒ते ए॒ते आ॑हु॒ रिती त्या॑हु रे॒ते । \newline
26. आ॒हु॒ रे॒ते ए॒ते आ॑हु राहु रे॒ते हि ह्ये॑ते आ॑हु राहु रे॒ते हि । \newline
27. ए॒ते हि ह्ये॑ते ए॒ते हि य॒ज्ञ्ं ॅय॒ज्ञ्ꣳ ह्ये॑ते ए॒ते हि य॒ज्ञ्म् । \newline
28. ए॒ते इत्ये॒ते । \newline
29. हि य॒ज्ञ्ं ॅय॒ज्ञ्ꣳ हि हि य॒ज्ञ्ं ॅवह॑तो॒ वह॑तो य॒ज्ञ्ꣳ हि हि य॒ज्ञ्ं ॅवह॑तः । \newline
30. य॒ज्ञ्ं ॅवह॑तो॒ वह॑तो य॒ज्ञ्ं ॅय॒ज्ञ्ं ॅवह॑त॒ इतीति॒ वह॑तो य॒ज्ञ्ं ॅय॒ज्ञ्ं ॅवह॑त॒ इति॑ । \newline
31. वह॑त॒ इतीति॒ वह॑तो॒ वह॑त॒ इति॒ ते ते इति॒ वह॑तो॒ वह॑त॒ इति॒ ते । \newline
32. इति॒ ते ते इतीति॒ ते तु तु ते इतीति॒ ते तु । \newline
33. ते तु तु ते ते त्वाव वाव तु ते ते त्वाव । \newline
34. ते इति॒ ते । \newline
35. त्वाव वाव तु त्वाव न न वाव तु त्वाव न । \newline
36. वाव न न वाव वाव नोथ्सृज्ये॑ उ॒थ्सृज्ये॒ न वाव वाव नोथ्सृज्ये᳚ । \newline
37. नोथ्सृज्ये॑ उ॒थ्सृज्ये॒ न नोथ्सृज्ये॒ इती त्यु॒थ्सृज्ये॒ न नोथ्सृज्ये॒ इति॑ । \newline
38. उ॒थ्सृज्ये॒ इती त्यु॒थ्सृज्ये॑ उ॒थ्सृज्ये॒ इत्या॑हु राहु॒रि त्यु॒थ्सृज्ये॑ उ॒थ्सृज्ये॒ इत्या॑हुः । \newline
39. उ॒थ्सृज्ये॒ इत्यु॑त् - सृज्ये᳚ । \newline
40. इत्या॑हु राहु॒ रिती त्या॑हु॒र् ये ये आ॑हु॒ रिती त्या॑हु॒र् ये । \newline
41. आ॒हु॒र् ये ये आ॑हु राहु॒र् ये अ॑वान्त॒र म॑वान्त॒रं ॅये आ॑हु राहु॒र् ये अ॑वान्त॒रम् । \newline
42. ये अ॑वान्त॒र म॑वान्त॒रं ॅये ये अ॑वान्त॒रं ॅय॒ज्ञ्ं ॅय॒ज्ञ् म॑वान्त॒रं ॅये ये अ॑वान्त॒रं ॅय॒ज्ञ्म् । \newline
43. ये इति॒ ये । \newline
44. अ॒वा॒न्त॒रं ॅय॒ज्ञ्ं ॅय॒ज्ञ् म॑वान्त॒र म॑वान्त॒रं ॅय॒ज्ञ्म् भे॒जाते॑ भे॒जाते॑ य॒ज्ञ् म॑वान्त॒र म॑वान्त॒रं ॅय॒ज्ञ्म् भे॒जाते᳚ । \newline
45. अ॒वा॒न्त॒रमित्य॑व - अ॒न्त॒रम् । \newline
46. य॒ज्ञ्म् भे॒जाते॑ भे॒जाते॑ य॒ज्ञ्ं ॅय॒ज्ञ्म् भे॒जाते॒ इतीति॑ भे॒जाते॑ य॒ज्ञ्ं ॅय॒ज्ञ्म् भे॒जाते॒ इति॑ । \newline
47. भे॒जाते॒ इतीति॑ भे॒जाते॑ भे॒जाते॒ इति॒ या येति॑ भे॒जाते॑ भे॒जाते॒ इति॒ या । \newline
48. भे॒जाते॒ इति॑ भे॒जाते᳚ । \newline
49. इति॒ या येतीति॒ या प्र॑थ॒मा प्र॑थ॒मा येतीति॒ या प्र॑थ॒मा । \newline
50. या प्र॑थ॒मा प्र॑थ॒मा या या प्र॑थ॒मा व्य॑ष्टका॒ व्य॑ष्टका प्रथ॒मा या या प्र॑थ॒मा व्य॑ष्टका । \newline
51. प्र॒थ॒मा व्य॑ष्टका॒ व्य॑ष्टका प्रथ॒मा प्र॑थ॒मा व्य॑ष्टका॒ तस्या॒म् तस्यां॒ ॅव्य॑ष्टका प्रथ॒मा प्र॑थ॒मा व्य॑ष्टका॒ तस्या᳚म् । \newline
52. व्य॑ष्टका॒ तस्या॒म् तस्यां॒ ॅव्य॑ष्टका॒ व्य॑ष्टका॒ तस्या॑ मु॒थ्सृज्य॑ मु॒थ्सृज्य॒म् तस्यां॒ ॅव्य॑ष्टका॒ व्य॑ष्टका॒ तस्या॑ मु॒थ्सृज्य᳚म् । \newline
53. व्य॑ष्ट॒केति॒ वि - अ॒ष्ट॒का॒ । \newline
54. तस्या॑ मु॒थ्सृज्य॑ मु॒थ्सृज्य॒म् तस्या॒म् तस्या॑ मु॒थ्सृज्य॒ मिती त्यु॒थ्सृज्य॒म् तस्या॒म् तस्या॑ मु॒थ्सृज्य॒ मिति॑ । \newline
55. उ॒थ्सृज्य॒ मिती त्यु॒थ्सृज्य॑ मु॒थ्सृज्य॒ मित्या॑हु राहु॒ रित्यु॒थ्सृज्य॑ मु॒थ्सृज्य॒ मित्या॑हुः । \newline
56. उ॒थ्सृज्य॒मित्यु॑त् - सृज्य᳚म् । \newline
57. इत्या॑हु राहु॒ रिती त्या॑हु रे॒ष ए॒ष आ॑हु॒ रिती त्या॑हु रे॒षः । \newline
58. आ॒हु॒ रे॒ष ए॒ष आ॑हु राहु रे॒ष वै वा ए॒ष आ॑हु राहु रे॒ष वै । \newline
59. ए॒ष वै वा ए॒ष ए॒ष वै मा॒सो मा॒सो वा ए॒ष ए॒ष वै मा॒सः । \newline
60. वै मा॒सो मा॒सो वै वै मा॒सो वि॑श॒रो वि॑श॒रो मा॒सो वै वै मा॒सो वि॑श॒रः । \newline
61. मा॒सो वि॑श॒रो वि॑श॒रो मा॒सो मा॒सो वि॑श॒र इतीति॑ विश॒रो मा॒सो मा॒सो वि॑श॒र इति॑ । \newline
62. वि॒श॒र इतीति॑ विश॒रो वि॑श॒र इति॒ न नेति॑ विश॒रो वि॑श॒र इति॒ न । \newline
63. वि॒श॒र इति॑ वि - श॒रः । \newline
64. इति॒ न नेतीति॒ नादि॑ष्ट॒ मादि॑ष्ट॒न् नेतीति॒ नादि॑ष्टम् । \newline
65. नादि॑ष्ट॒ मादि॑ष्ट॒न् न नादि॑ष्ट॒ मुदु दादि॑ष्ट॒न् न नादि॑ष्ट॒ मुत् । \newline
66. आदि॑ष्ट॒ मुदु दादि॑ष्ट॒ मादि॑ष्ट॒ मुथ् सृ॑जेयुः सृजेयु॒ रुदादि॑ष्ट॒ मादि॑ष्ट॒ मुथ् सृ॑जेयुः । \newline
67. आदि॑ष्ट॒मित्या - दि॒ष्ट॒म् । \newline
\pagebreak
\markright{ TS 7.5.7.2  \hfill https://www.vedavms.in \hfill}

\section{ TS 7.5.7.2 }

\textbf{TS 7.5.7.2 } \newline
\textbf{Samhita Paata} \newline

-मुथ्सृ॑जेयु॒-र्यदादि॑ष्ट-मुथ्सृ॒जेयु॑र्या॒दृशे॒ पुनः॑ पर्याप्ला॒वे मद्ध्ये॑ षड॒हस्य॑ स॒पंद्ये॑त षड॒हैर्मासा᳚न्थ् स॒पांद्य॒ यथ् स॑प्त॒म- मह॒स्तस्मि॒न्नुथ् सृ॑जेयु॒-स्तद॒ग्नये॒ वसु॑मते पुरो॒डाश॑म॒ष्टाक॑पालं॒ निर्व॑पेयुरै॒न्द्रं दधीन्द्रा॑य म॒रुत्व॑ते पुरो॒डाश॒मेका॑दशकपालं ॅवैश्वदे॒वं द्वाद॑शकपालम॒ग्नेर्वै वसु॑मतः प्रातस्सव॒नं ॅयद॒ग्नये॒ वसु॑मते पुरो॒डाश॑म॒ष्टाक॑पालं नि॒र्वप॑न्ति दे॒वता॑मे॒व तद्-भा॒गिनीं᳚ कु॒र्वन्ति॒ - [  ] \newline

\textbf{Pada Paata} \newline

उदिति॑ । सृ॒जे॒युः॒ । यत् । आदि॑ष्ट॒मित्या - दि॒ष्ट॒म् । उ॒थ्सृ॒जेयु॒रित्यु॑त् - सृ॒जेयुः॑ । या॒दृशे᳚ । पुनः॑ । प॒र्या॒प्ला॒व इति॑ परि - आ॒प्ला॒वे । मद्ध्ये᳚ । ष॒ड॒हस्येति॑ षट् - अ॒हस्य॑ । स॒पंद्ये॒तेति॑ सं - पद्ये॑त । ष॒ड॒हैरिति॑ षट् - अ॒हैः । मासान्॑ । स॒पांद्येति॑ सं - पाद्य॑ । यत् । स॒प्त॒मम् । अहः॑ । तस्मिन्न्॑ । उदिति॑ । सृ॒जे॒युः॒ । तत् । अ॒ग्नये᳚ । वसु॑मत॒ इति॒ वसु॑ - म॒ते॒ । पु॒रो॒डाश᳚म् । अ॒ष्टाक॑पाल॒मित्य॒ष्टा - क॒पा॒ल॒म् । निरितिः॑ । व॒पे॒युः॒ । ऐ॒न्द्रम् । दधि॑ । इन्द्रा॑य । म॒रुत्व॑ते । पु॒रो॒डाश᳚म् । एका॑दशकपाल॒मित्येका॑दश - क॒पा॒ल॒म् । वै॒श्व॒दे॒वमिति॑ वैश्व-दे॒वम् । द्वाद॑शकपाल॒मिति॒ द्वाद॑श - क॒पा॒ल॒म् । अ॒ग्नेः । वै । वसु॑मत॒ इति॒ वसु॑ - म॒तः॒ । प्रा॒त॒स्स॒व॒नमिति॑ प्रातः - स॒व॒नम् । यत् । अ॒ग्नये᳚ । वसु॑मत॒ इति॒ वसु॑-म॒ते॒ । पु॒रो॒डाश᳚म् । अ॒ष्टाक॑पाल॒मित्य॒ष्टा - क॒पा॒ल॒म् । नि॒र्वप॒न्तीति॑ निः - वप॑न्ति । दे॒वता᳚म् । ए॒व । तत् । भा॒गिनी᳚म् । कु॒र्वन्ति॑ ।  \newline


\textbf{Krama Paata} \newline

उथ् सृ॑जेयुः । सृ॒जे॒यु॒र् यत् । यदादि॑ष्टम् । आदि॑ष्टमुथ्सृ॒जेयुः॑ । आदि॑ष्ट॒मित्या - दि॒ष्ट॒म् । 
उ॒थ्सृ॒जेयु॑र् या॒दृशे᳚ । उ॒थ्सृ॒जेयु॒रित्यु॑त् - सृ॒जेयुः॑ । या॒दृशे॒ पुनः॑ । पुनः॑ पर्याप्ला॒वे । प॒र्या॒प्ला॒वे मद्ध्ये᳚ । प॒र्या॒प्ला॒व इति॑ परि - आ॒प्ला॒वे । मद्ध्ये॑ षड॒हस्य॑ । ष॒ड॒हस्य॑ स॒म्पद्ये॑त । ष॒ड॒हस्येति॑ षट् - अ॒हस्य॑ । स॒म्पद्ये॑त षड॒हैः । स॒म्पद्ये॒तेति॑ सम् - पद्ये॑त । ष॒ड॒हैर् मासान्॑ । ष॒ड॒हैरिति॑ षट् - अ॒हैः । मासा᳚न्थ् स॒म्पाद्य॑ । स॒म्पाद्य॒ यत् । स॒म्पाद्येति॑ सम् - पाद्य॑ । यथ् स॑प्त॒मम् । स॒प्त॒ममहः॑ । अह॒स्तस्मिन्न्॑ । तस्मि॒न्नुत् । उथ् सृ॑जेयुः । सृ॒जे॒यु॒स्तत् । तद॒ग्नये᳚ । अ॒ग्नये॒ वसु॑मते । वसु॑मते पुरो॒डाश᳚म् । वसु॑मत॒ इति॒ वसु॑ - म॒ते॒ । पु॒रो॒डाश॑म॒ष्टाक॑पालम् । अ॒ष्टाक॑पाल॒म् निः । अ॒ष्टाक॑पाल॒मित्य॒ष्टा - क॒पा॒ल॒म् । निर् व॑पेयुः । व॒पे॒यु॒रै॒न्द्रम् । ऐ॒न्द्रम् दधि॑ । दधीन्द्रा॑य । इन्द्रा॑य म॒रुत्व॑ते । म॒रुत्व॑ते पुरो॒डाश᳚म् । पु॒रो॒डाश॒मेका॑दशकपालम् । एका॑दशकपालम् ॅवैश्वदे॒वम् । एका॑दशकपाल॒मित्येका॑दश - क॒पा॒ल॒म् । वै॒श्व॒दे॒वम् द्वाद॑शकपालम् । वै॒श्व॒दे॒वमिति॑ वैश्व - दे॒वम् । द्वाद॑शकपालम॒ग्नेः । द्वाद॑शकपाल॒मिति॒ द्वाद॑श - क॒पा॒ल॒म् । अ॒ग्नेर् वै । वै वसु॑मतः । वसु॑मतः प्रातस्सव॒नम् । वसु॑मत॒ इति॒ वसु॑ - म॒तः॒ । प्रा॒त॒स्स॒व॒नम् ॅयत् । प्रा॒त॒स्स॒व॒नमिति॑ प्रातः - स॒व॒नम् । यद॒ग्नये᳚ । अ॒ग्नये॒ वसु॑मते । वसु॑मते पुरो॒डाश᳚म् । वसु॑मत॒ इति॒ वसु॑ - म॒ते॒ । पु॒रो॒डाश॑म॒ष्टाक॑पालम् । अ॒ष्टाक॑पालम् नि॒र्वप॑न्ति । अ॒ष्टाक॑पाल॒मित्य॒ष्टा - क॒पा॒ल॒म् । नि॒र्वप॑न्ति दे॒वता᳚म् । नि॒र्वप॒न्तीति॑ निः - वप॑न्ति । दे॒वता॑मे॒व । ए॒व तत् । तद् भा॒गिनी᳚म् । भा॒गिनी᳚म् कु॒र्वन्ति॑ । कु॒र्वन्ति॒ सव॑नम् \newline

\textbf{Jatai Paata} \newline

1. उथ् सृ॑जेयुः सृजेयु॒ रुदुथ् सृ॑जेयुः । \newline
2. सृ॒जे॒यु॒र् यद् यथ् सृ॑जेयुः सृजेयु॒र् यत् । \newline
3. यदादि॑ष्ट॒ मादि॑ष्टं॒ ॅयद् यदादि॑ष्टम् । \newline
4. आदि॑ष्ट मुथ्सृ॒जेयु॑ रुथ्सृ॒जेयु॒ रादि॑ष्ट॒ मादि॑ष्ट मुथ्सृ॒जेयुः॑ । \newline
5. आदि॑ष्ट॒मित्या - दि॒ष्ट॒म् । \newline
6. उ॒थ्सृ॒जेयु॑र् या॒दृशे॑ या॒दृश॑ उथ्सृ॒जेयु॑ रुथ्सृ॒जेयु॑र् या॒दृशे᳚ । \newline
7. उ॒थ्सृ॒जेयु॒रित्यु॑त् - सृ॒जेयुः॑ । \newline
8. या॒दृशे॒ पुनः॒ पुन॑र् या॒दृशे॑ या॒दृशे॒ पुनः॑ । \newline
9. पुनः॑ पर्याप्ला॒वे प॑र्याप्ला॒वे पुनः॒ पुनः॑ पर्याप्ला॒वे । \newline
10. प॒र्या॒प्ला॒वे मद्ध्ये॒ मद्ध्ये॑ पर्याप्ला॒वे प॑र्याप्ला॒वे मद्ध्ये᳚ । \newline
11. प॒र्या॒प्ला॒व इति॑ परि - आ॒प्ला॒वे । \newline
12. मद्ध्ये॑ षड॒हस्य॑ षड॒हस्य॒ मद्ध्ये॒ मद्ध्ये॑ षड॒हस्य॑ । \newline
13. ष॒ड॒हस्य॑ सं॒पद्ये॑त सं॒पद्ये॑त षड॒हस्य॑ षड॒हस्य॑ सं॒पद्ये॑त । \newline
14. ष॒ड॒हस्येति॑ षट् - अ॒हस्य॑ । \newline
15. सं॒पद्ये॑त षड॒है ष्ष॑ड॒हैः सं॒पद्ये॑त सं॒पद्ये॑त षड॒हैः । \newline
16. सं॒पद्ये॒तेति॑ सं - पद्ये॑त । \newline
17. ष॒ड॒हैर् मासा॒न् मासा᳚न् षड॒है ष्ष॑ड॒हैर् मासान्॑ । \newline
18. ष॒ड॒हैरिति॑ षट् - अ॒हैः । \newline
19. मासा᳚न् थ्सं॒पाद्य॑ सं॒पाद्य॒ मासा॒न् मासा᳚न् थ्सं॒पाद्य॑ । \newline
20. सं॒पाद्य॒ यद् यथ् सं॒पाद्य॑ सं॒पाद्य॒ यत् । \newline
21. सं॒पाद्येति॑ सं - पाद्य॑ । \newline
22. यथ् स॑प्त॒मꣳ स॑प्त॒मं ॅयद् यथ् स॑प्त॒मम् । \newline
23. स॒प्त॒म मह॒ रहः॑ सप्त॒मꣳ स॑प्त॒म महः॑ । \newline
24. अह॒ स्तस्मिꣳ॒॒ स्तस्मि॒न् नह॒ रह॒ स्तस्मिन्न्॑ । \newline
25. तस्मि॒न् नुदुत् तस्मिꣳ॒॒ स्तस्मि॒न् नुत् । \newline
26. उथ् सृ॑जेयुः सृजेयु॒ रुदुथ् सृ॑जेयुः । \newline
27. सृ॒जे॒यु॒ स्तत् तथ् सृ॑जेयुः सृजेयु॒ स्तत् । \newline
28. तद॒ग्नये॒ ऽग्नये॒ तत् तद॒ग्नये᳚ । \newline
29. अ॒ग्नये॒ वसु॑मते॒ वसु॑मते॒ ऽग्नये॒ ऽग्नये॒ वसु॑मते । \newline
30. वसु॑मते पुरो॒डाश॑म् पुरो॒डाशं॒ ॅवसु॑मते॒ वसु॑मते पुरो॒डाश᳚म् । \newline
31. वसु॑मत॒ इति॒ वसु॑ - म॒ते॒ । \newline
32. पु॒रो॒डाश॑ म॒ष्टाक॑पाल म॒ष्टाक॑पालम् पुरो॒डाश॑म् पुरो॒डाश॑ म॒ष्टाक॑पालम् । \newline
33. अ॒ष्टाक॑पाल॒न् निर् णिर॒ष्टाक॑पाल म॒ष्टाक॑पाल॒न् निः । \newline
34. अ॒ष्टाक॑पाल॒मित्य॒ष्टा - क॒पा॒ल॒म् । \newline
35. निर् व॑पेयुर् वपेयु॒र् निर् णिर् व॑पेयुः । \newline
36. व॒पे॒यु॒ रै॒न्द्र मै॒न्द्रं ॅव॑पेयुर् वपेयु रै॒न्द्रम् । \newline
37. ऐ॒न्द्रम् दधि॒ दध्यै॒न्द्र मै॒न्द्रम् दधि॑ । \newline
38. दधीन्द्रा॒ येन्द्रा॑य॒ दधि॒ दधीन्द्रा॑य । \newline
39. इन्द्रा॑य म॒रुत्व॑ते म॒रुत्व॑त॒ इन्द्रा॒ येन्द्रा॑य म॒रुत्व॑ते । \newline
40. म॒रुत्व॑ते पुरो॒डाश॑म् पुरो॒डाश॑म् म॒रुत्व॑ते म॒रुत्व॑ते पुरो॒डाश᳚म् । \newline
41. पु॒रो॒डाश॒ मेका॑दशकपाल॒ मेका॑दशकपालम् पुरो॒डाश॑म् पुरो॒डाश॒ मेका॑दशकपालम् । \newline
42. एका॑दशकपालं ॅवैश्वदे॒वं ॅवै᳚श्वदे॒व मेका॑दशकपाल॒ मेका॑दशकपालं ॅवैश्वदे॒वम् । \newline
43. एका॑दशकपाल॒मित्येका॑दश - क॒पा॒ल॒म् । \newline
44. वै॒श्व॒दे॒वम् द्वाद॑शकपाल॒म् द्वाद॑शकपालं ॅवैश्वदे॒वं ॅवै᳚श्वदे॒वम् द्वाद॑शकपालम् । \newline
45. वै॒श्व॒दे॒वमिति॑ वैश्व - दे॒वम् । \newline
46. द्वाद॑शकपाल म॒ग्ने र॒ग्नेर् द्वाद॑शकपाल॒म् द्वाद॑शकपाल म॒ग्नेः । \newline
47. द्वाद॑शकपाल॒मिति॒ द्वाद॑श - क॒पा॒ल॒म् । \newline
48. अ॒ग्नेर् वै वा अ॒ग्ने र॒ग्नेर् वै । \newline
49. वै वसु॑मतो॒ वसु॑मतो॒ वै वै वसु॑मतः । \newline
50. वसु॑मतः प्रातस्सव॒नम् प्रा॑तस्सव॒नं ॅवसु॑मतो॒ वसु॑मतः प्रातस्सव॒नम् । \newline
51. वसु॑मत॒ इति॒ वसु॑ - म॒तः॒ । \newline
52. प्रा॒त॒स्स॒व॒नं ॅयद् यत् प्रा॑तस्सव॒नम् प्रा॑तस्सव॒नं ॅयत् । \newline
53. प्रा॒त॒स्स॒व॒नमिति॑ प्रातः - स॒व॒नम् । \newline
54. यद॒ग्नये॒ ऽग्नये॒ यद् यद॒ग्नये᳚ । \newline
55. अ॒ग्नये॒ वसु॑मते॒ वसु॑मते॒ ऽग्नये॒ ऽग्नये॒ वसु॑मते । \newline
56. वसु॑मते पुरो॒डाश॑म् पुरो॒डाशं॒ ॅवसु॑मते॒ वसु॑मते पुरो॒डाश᳚म् । \newline
57. वसु॑मत॒ इति॒ वसु॑ - म॒ते॒ । \newline
58. पु॒रो॒डाश॑ म॒ष्टाक॑पाल म॒ष्टाक॑पालम् पुरो॒डाश॑म् पुरो॒डाश॑ म॒ष्टाक॑पालम् । \newline
59. अ॒ष्टाक॑पालन् नि॒र्वप॑न्ति नि॒र्वप॑ न्त्य॒ष्टाक॑पाल म॒ष्टाक॑पालन् नि॒र्वप॑न्ति । \newline
60. अ॒ष्टाक॑पाल॒मित्य॒ष्टा - क॒पा॒ल॒म् । \newline
61. नि॒र्वप॑न्ति दे॒वता᳚म् दे॒वता᳚न् नि॒र्वप॑न्ति नि॒र्वप॑न्ति दे॒वता᳚म् । \newline
62. नि॒र्वप॒न्तीति॑ निः - वप॑न्ति । \newline
63. दे॒वता॑ मे॒वैव दे॒वता᳚म् दे॒वता॑ मे॒व । \newline
64. ए॒व तत् तदे॒ वैव तत् । \newline
65. तद् भा॒गिनी᳚म् भा॒गिनी॒म् तत् तद् भा॒गिनी᳚म् । \newline
66. भा॒गिनी᳚म् कु॒र्वन्ति॑ कु॒र्वन्ति॑ भा॒गिनी᳚म् भा॒गिनी᳚म् कु॒र्वन्ति॑ । \newline
67. कु॒र्वन्ति॒ सव॑नꣳ॒॒ सव॑नम् कु॒र्वन्ति॑ कु॒र्वन्ति॒ सव॑नम् । \newline

\textbf{Ghana Paata } \newline

1. उथ् सृ॑जेयुः सृजेयु॒ रुदुथ् सृ॑जेयु॒र् यद् यथ् सृ॑जेयु॒ रुदुथ् सृ॑जेयु॒र् यत् । \newline
2. सृ॒जे॒यु॒र् यद् यथ् सृ॑जेयुः सृजेयु॒र् यदादि॑ष्ट॒ मादि॑ष्टं॒ ॅयथ् सृ॑जेयुः सृजेयु॒र् यदादि॑ष्टम् । \newline
3. यदादि॑ष्ट॒ मादि॑ष्टं॒ ॅयद् यदादि॑ष्ट मुथ्सृ॒जेयु॑ रुथ्सृ॒जेयु॒ रादि॑ष्टं॒ ॅयद् यदादि॑ष्ट मुथ्सृ॒जेयुः॑ । \newline
4. आदि॑ष्ट मुथ्सृ॒जेयु॑ रुथ्सृ॒जेयु॒ रादि॑ष्ट॒ मादि॑ष्ट मुथ्सृ॒जेयु॑र् या॒दृशे॑ या॒दृश॑ उथ्सृ॒जेयु॒ रादि॑ष्ट॒ मादि॑ष्ट मुथ्सृ॒जेयु॑र् या॒दृशे᳚ । \newline
5. आदि॑ष्ट॒मित्या - दि॒ष्ट॒म् । \newline
6. उ॒थ्सृ॒जेयु॑र् या॒दृशे॑ या॒दृश॑ उथ्सृ॒जेयु॑ रुथ्सृ॒जेयु॑र् या॒दृशे॒ पुनः॒ पुन॑र् या॒दृश॑ उथ्सृ॒जेयु॑ रुथ्सृ॒जेयु॑र् या॒दृशे॒ पुनः॑ । \newline
7. उ॒थ्सृ॒जेयु॒रित्यु॑त् - सृ॒जेयुः॑ । \newline
8. या॒दृशे॒ पुनः॒ पुन॑र् या॒दृशे॑ या॒दृशे॒ पुनः॑ पर्याप्ला॒वे प॑र्याप्ला॒वे पुन॑र् या॒दृशे॑ या॒दृशे॒ पुनः॑ पर्याप्ला॒वे । \newline
9. पुनः॑ पर्याप्ला॒वे प॑र्याप्ला॒वे पुनः॒ पुनः॑ पर्याप्ला॒वे मद्ध्ये॒ मद्ध्ये॑ पर्याप्ला॒वे पुनः॒ पुनः॑ पर्याप्ला॒वे मद्ध्ये᳚ । \newline
10. प॒र्या॒प्ला॒वे मद्ध्ये॒ मद्ध्ये॑ पर्याप्ला॒वे प॑र्याप्ला॒वे मद्ध्ये॑ षड॒हस्य॑ षड॒हस्य॒ मद्ध्ये॑ पर्याप्ला॒वे प॑र्याप्ला॒वे मद्ध्ये॑ षड॒हस्य॑ । \newline
11. प॒र्या॒प्ला॒व इति॑ परि - आ॒प्ला॒वे । \newline
12. मद्ध्ये॑ षड॒हस्य॑ षड॒हस्य॒ मद्ध्ये॒ मद्ध्ये॑ षड॒हस्य॑ सं॒पद्ये॑त सं॒पद्ये॑त षड॒हस्य॒ मद्ध्ये॒ मद्ध्ये॑ षड॒हस्य॑ सं॒पद्ये॑त । \newline
13. ष॒ड॒हस्य॑ सं॒पद्ये॑त सं॒पद्ये॑त षड॒हस्य॑ षड॒हस्य॑ सं॒पद्ये॑त षड॒है ष्ष॑ड॒हैः सं॒पद्ये॑त षड॒हस्य॑ षड॒हस्य॑ सं॒पद्ये॑त षड॒हैः । \newline
14. ष॒ड॒हस्येति॑ षट् - अ॒हस्य॑ । \newline
15. सं॒पद्ये॑त षड॒है ष्ष॑ड॒हैः सं॒पद्ये॑त सं॒पद्ये॑त षड॒हैर् मासा॒न् मासा᳚न् षड॒हैः सं॒पद्ये॑त सं॒पद्ये॑त षड॒हैर् मासान्॑ । \newline
16. सं॒पद्ये॒तेति॑ सं - पद्ये॑त । \newline
17. ष॒ड॒हैर् मासा॒न् मासा᳚न् षड॒है ष्ष॑ड॒हैर् मासा᳚न् थ्सं॒पाद्य॑ सं॒पाद्य॒ मासा᳚न् षड॒है ष्ष॑ड॒हैर् मासा᳚न् थ्सं॒पाद्य॑ । \newline
18. ष॒ड॒हैरिति॑ षट् - अ॒हैः । \newline
19. मासा᳚न् थ्सं॒पाद्य॑ सं॒पाद्य॒ मासा॒न् मासा᳚न् थ्सं॒पाद्य॒ यद् यथ् सं॒पाद्य॒ मासा॒न् मासा᳚न् थ्सं॒पाद्य॒ यत् । \newline
20. सं॒पाद्य॒ यद् यथ् सं॒पाद्य॑ सं॒पाद्य॒ यथ् स॑प्त॒मꣳ स॑प्त॒मं ॅयथ् सं॒पाद्य॑ सं॒पाद्य॒ यथ् स॑प्त॒मम् । \newline
21. सं॒पाद्येति॑ सं - पाद्य॑ । \newline
22. यथ् स॑प्त॒मꣳ स॑प्त॒मं ॅयद् यथ् स॑प्त॒म मह॒ रहः॑ सप्त॒मं ॅयद् यथ् स॑प्त॒म महः॑ । \newline
23. स॒प्त॒म मह॒ रहः॑ सप्त॒मꣳ स॑प्त॒म मह॒ स्तस्मिꣳ॒॒ स्तस्मि॒न् नहः॑ सप्त॒मꣳ स॑प्त॒म मह॒ स्तस्मिन्न्॑ । \newline
24. अह॒ स्तस्मिꣳ॒॒ स्तस्मि॒न् नह॒ रह॒ स्तस्मि॒न् नुदुत् तस्मि॒न् नह॒ रह॒ स्तस्मि॒न् नुत् । \newline
25. तस्मि॒न् नुदुत् तस्मिꣳ॒॒ स्तस्मि॒न् नुथ् सृ॑जेयुः सृजेयु॒ रुत् तस्मिꣳ॒॒ स्तस्मि॒न् नुथ् सृ॑जेयुः । \newline
26. उथ् सृ॑जेयुः सृजेयु॒ रुदुथ् सृ॑जेयु॒ स्तत् तथ् सृ॑जेयु॒ रुदुथ् सृ॑जेयु॒ स्तत् । \newline
27. सृ॒जे॒यु॒स्तत् तथ् सृ॑जेयुः सृजेयु॒ स्त द॒ग्नये॒ ऽग्नये॒ तथ् सृ॑जेयुः सृजेयु॒ स्त द॒ग्नये᳚ । \newline
28. तद॒ग्नये॒ ऽग्नये॒ तत् तद॒ग्नये॒ वसु॑मते॒ वसु॑मते॒ ऽग्नये॒ तत् तद॒ग्नये॒ वसु॑मते । \newline
29. अ॒ग्नये॒ वसु॑मते॒ वसु॑मते॒ ऽग्नये॒ ऽग्नये॒ वसु॑मते पुरो॒डाश॑म् पुरो॒डाशं॒ ॅवसु॑मते॒ ऽग्नये॒ ऽग्नये॒ वसु॑मते पुरो॒डाश᳚म् । \newline
30. वसु॑मते पुरो॒डाश॑म् पुरो॒डाशं॒ ॅवसु॑मते॒ वसु॑मते पुरो॒डाश॑ म॒ष्टाक॑पाल म॒ष्टाक॑पालम् पुरो॒डाशं॒ ॅवसु॑मते॒ वसु॑मते पुरो॒डाश॑ म॒ष्टाक॑पालम् । \newline
31. वसु॑मत॒ इति॒ वसु॑ - म॒ते॒ । \newline
32. पु॒रो॒डाश॑ म॒ष्टाक॑पाल म॒ष्टाक॑पालम् पुरो॒डाश॑म् पुरो॒डाश॑ म॒ष्टाक॑पाल॒न् निर् णिर॒ष्टाक॑पालम् पुरो॒डाश॑म् पुरो॒डाश॑ म॒ष्टाक॑पाल॒न् निः । \newline
33. अ॒ष्टाक॑पाल॒न् निर् णिर॒ष्टाक॑पाल म॒ष्टाक॑पाल॒न् निर् व॑पेयुर् वपेयु॒र् निर॒ष्टाक॑पाल म॒ष्टाक॑पाल॒न् निर् व॑पेयुः । \newline
34. अ॒ष्टाक॑पाल॒मित्य॒ष्टा - क॒पा॒ल॒म् । \newline
35. निर् व॑पेयुर् वपेयु॒र् निर् णिर् व॑पेयु रै॒न्द्र मै॒न्द्रं ॅव॑पेयु॒र् निर् णिर् व॑पेयु रै॒न्द्रम् । \newline
36. व॒पे॒यु॒ रै॒न्द्र मै॒न्द्रं ॅव॑पेयुर् वपेयु रै॒न्द्रम् दधि॒ दध्यै॒न्द्रं ॅव॑पेयुर् वपेयु रै॒न्द्रम् दधि॑ । \newline
37. ऐ॒न्द्रम् दधि॒ दध्यै॒न्द्र मै॒न्द्रम् दधीन्द्रा॒ येन्द्रा॑य॒ दध्यै॒न्द्र मै॒न्द्रम् दधीन्द्रा॑य । \newline
38. दधीन्द्रा॒ येन्द्रा॑य॒ दधि॒ दधीन्द्रा॑य म॒रुत्व॑ते म॒रुत्व॑त॒ इन्द्रा॑य॒ दधि॒ दधीन्द्रा॑य म॒रुत्व॑ते । \newline
39. इन्द्रा॑य म॒रुत्व॑ते म॒रुत्व॑त॒ इन्द्रा॒ येन्द्रा॑य म॒रुत्व॑ते पुरो॒डाश॑म् पुरो॒डाश॑म् म॒रुत्व॑त॒ इन्द्रा॒
येन्द्रा॑य म॒रुत्व॑ते पुरो॒डाश᳚म् । \newline
40. म॒रुत्व॑ते पुरो॒डाश॑म् पुरो॒डाश॑म् म॒रुत्व॑ते म॒रुत्व॑ते पुरो॒डाश॒ मेका॑दशकपाल॒ मेका॑दशकपालम् पुरो॒डाश॑म् म॒रुत्व॑ते म॒रुत्व॑ते पुरो॒डाश॒ मेका॑दशकपालम् । \newline
41. पु॒रो॒डाश॒ मेका॑दशकपाल॒ मेका॑दशकपालम् पुरो॒डाश॑म् पुरो॒डाश॒ मेका॑दशकपालं ॅवैश्वदे॒वं ॅवै᳚श्वदे॒व मेका॑दशकपालम् पुरो॒डाश॑म् पुरो॒डाश॒ मेका॑दशकपालं ॅवैश्वदे॒वम् । \newline
42. एका॑दशकपालं ॅवैश्वदे॒वं ॅवै᳚श्वदे॒व मेका॑दशकपाल॒ मेका॑दशकपालं ॅवैश्वदे॒वम् द्वाद॑शकपाल॒म् द्वाद॑शकपालं ॅवैश्वदे॒व मेका॑दशकपाल॒ मेका॑दशकपालं ॅवैश्वदे॒वम् द्वाद॑शकपालम् । \newline
43. एका॑दशकपाल॒मित्येका॑दश - क॒पा॒ल॒म् । \newline
44. वै॒श्व॒दे॒वम् द्वाद॑शकपाल॒म् द्वाद॑शकपालं ॅवैश्वदे॒वं ॅवै᳚श्वदे॒वम् द्वाद॑शकपाल म॒ग्ने र॒ग्नेर् द्वाद॑शकपालं ॅवैश्वदे॒वं ॅवै᳚श्वदे॒वम् द्वाद॑शकपाल म॒ग्नेः । \newline
45. वै॒श्व॒दे॒वमिति॑ वैश्व - दे॒वम् । \newline
46. द्वाद॑शकपाल म॒ग्ने र॒ग्नेर् द्वाद॑शकपाल॒म् द्वाद॑शकपाल म॒ग्नेर् वै वा अ॒ग्नेर् द्वाद॑शकपाल॒म् द्वाद॑शकपाल म॒ग्नेर् वै । \newline
47. द्वाद॑शकपाल॒मिति॒ द्वाद॑श - क॒पा॒ल॒म् । \newline
48. अ॒ग्नेर् वै वा अ॒ग्ने र॒ग्नेर् वै वसु॑मतो॒ वसु॑मतो॒ वा अ॒ग्ने र॒ग्नेर् वै वसु॑मतः । \newline
49. वै वसु॑मतो॒ वसु॑मतो॒ वै वै वसु॑मतः प्रातस्सव॒नम् प्रा॑तस्सव॒नं ॅवसु॑मतो॒ वै वै वसु॑मतः प्रातस्सव॒नम् । \newline
50. वसु॑मतः प्रातस्सव॒नम् प्रा॑तस्सव॒नं ॅवसु॑मतो॒ वसु॑मतः प्रातस्सव॒नं ॅयद् यत् प्रा॑तस्सव॒नं ॅवसु॑मतो॒ वसु॑मतः प्रातस्सव॒नं ॅयत् । \newline
51. वसु॑मत॒ इति॒ वसु॑ - म॒तः॒ । \newline
52. प्रा॒त॒स्स॒व॒नं ॅयद् यत् प्रा॑तस्सव॒नम् प्रा॑तस्सव॒नं ॅयद॒ग्नये॒ ऽग्नये॒ यत् प्रा॑तस्सव॒नम् प्रा॑तस्सव॒नं ॅयद॒ग्नये᳚ । \newline
53. प्रा॒त॒स्स॒व॒नमिति॑ प्रातः - स॒व॒नम् । \newline
54. यद॒ग्नये॒ ऽग्नये॒ यद् यद॒ग्नये॒ वसु॑मते॒ वसु॑मते॒ ऽग्नये॒ यद् यद॒ग्नये॒ वसु॑मते । \newline
55. अ॒ग्नये॒ वसु॑मते॒ वसु॑मते॒ ऽग्नये॒ ऽग्नये॒ वसु॑मते पुरो॒डाश॑म् पुरो॒डाशं॒ ॅवसु॑मते॒ ऽग्नये॒ ऽग्नये॒ वसु॑मते पुरो॒डाश᳚म् । \newline
56. वसु॑मते पुरो॒डाश॑म् पुरो॒डाशं॒ ॅवसु॑मते॒ वसु॑मते पुरो॒डाश॑ म॒ष्टाक॑पाल म॒ष्टाक॑पालम् पुरो॒डाशं॒ ॅवसु॑मते॒ वसु॑मते पुरो॒डाश॑ म॒ष्टाक॑पालम् । \newline
57. वसु॑मत॒ इति॒ वसु॑ - म॒ते॒ । \newline
58. पु॒रो॒डाश॑ म॒ष्टाक॑पाल म॒ष्टाक॑पालम् पुरो॒डाश॑म् पुरो॒डाश॑ म॒ष्टाक॑पालन् नि॒र्वप॑न्ति नि॒र्वप॑ न्त्य॒ष्टाक॑पालम् पुरो॒डाश॑म् पुरो॒डाश॑ म॒ष्टाक॑पालन् नि॒र्वप॑न्ति । \newline
59. अ॒ष्टाक॑पालन् नि॒र्वप॑न्ति नि॒र्वप॑ न्त्य॒ष्टाक॑पाल म॒ष्टाक॑पालन् नि॒र्वप॑न्ति दे॒वता᳚म् दे॒वता᳚न् नि॒र्वप॑ न्त्य॒ष्टाक॑पाल म॒ष्टाक॑पालन् नि॒र्वप॑न्ति दे॒वता᳚म् । \newline
60. अ॒ष्टाक॑पाल॒मित्य॒ष्टा - क॒पा॒ल॒म् । \newline
61. नि॒र्वप॑न्ति दे॒वता᳚म् दे॒वता᳚न् नि॒र्वप॑न्ति नि॒र्वप॑न्ति दे॒वता॑ मे॒वैव दे॒वता᳚न् नि॒र्वप॑न्ति नि॒र्वप॑न्ति दे॒वता॑ मे॒व । \newline
62. नि॒र्वप॒न्तीति॑ निः - वप॑न्ति । \newline
63. दे॒वता॑ मे॒वैव दे॒वता᳚म् दे॒वता॑ मे॒व तत् तदे॒व दे॒वता᳚म् दे॒वता॑ मे॒व तत् । \newline
64. ए॒व तत् तदे॒ वैव तद् भा॒गिनी᳚म् भा॒गिनी॒म् तदे॒ वैव तद् भा॒गिनी᳚म् । \newline
65. तद् भा॒गिनी᳚म् भा॒गिनी॒म् तत् तद् भा॒गिनी᳚म् कु॒र्वन्ति॑ कु॒र्वन्ति॑ भा॒गिनी॒म् तत् तद् भा॒गिनी᳚म् कु॒र्वन्ति॑ । \newline
66. भा॒गिनी᳚म् कु॒र्वन्ति॑ कु॒र्वन्ति॑ भा॒गिनी᳚म् भा॒गिनी᳚म् कु॒र्वन्ति॒ सव॑नꣳ॒॒ सव॑नम् कु॒र्वन्ति॑ भा॒गिनी᳚म् भा॒गिनी᳚म् कु॒र्वन्ति॒ सव॑नम् । \newline
67. कु॒र्वन्ति॒ सव॑नꣳ॒॒ सव॑नम् कु॒र्वन्ति॑ कु॒र्वन्ति॒ सव॑न मष्टा॒भि र॑ष्टा॒भिः सव॑नम् कु॒र्वन्ति॑ कु॒र्वन्ति॒ सव॑न मष्टा॒भिः । \newline
\pagebreak
\markright{ TS 7.5.7.3  \hfill https://www.vedavms.in \hfill}

\section{ TS 7.5.7.3 }

\textbf{TS 7.5.7.3 } \newline
\textbf{Samhita Paata} \newline

सव॑नमष्टा॒भिरुप॑ यन्ति॒ यदै॒न्द्रं दधि॒ भव॒तीन्द्र॑मे॒व तद्-भा॑ग॒धेया॒न्न च्या॑वय॒न्तीन्द्र॑स्य॒ वै म॒रुत्व॑तो॒ माद्ध्य॑न्दिनꣳ॒॒ सव॑नं॒ ॅयदिन्द्रा॑य म॒रुत्व॑ते पुरो॒डाश॒मेका॑दशकपालं नि॒र्वप॑न्ति दे॒वता॑मे॒व तद्-भा॒गिनीं᳚ कु॒र्वन्ति॒ सव॑नमेकाद॒शभि॒रुप॑ यन्ति॒ विश्वे॑षां॒ ॅवै दे॒वाना॑मृभु॒मतां᳚ तृतीयसव॒नंॅयद्-वै᳚श्वदे॒वं द्वाद॑शकपालं नि॒र्वप॑न्ति दे॒वता॑ ए॒व तद्-भा॒गिनीः᳚ कु॒र्वन्ति॒ सव॑नं द्वाद॒शभि॒ - [  ] \newline

\textbf{Pada Paata} \newline

सव॑नम् । अ॒ष्टा॒भिः । उपेति॑ । य॒न्ति॒ । यत् । ऐ॒न्द्रम् । दधि॑ । भव॑ति । इन्द्र᳚म् । ए॒व । तत् । भा॒ग॒धेया॒दिति॑ भाग-धेया᳚त् । न । च्या॒व॒य॒न्ति॒ । इन्द्र॑स्य । वै । म॒रुत्व॑तः । माद्ध्य॑न्दिनम् । सव॑नम् । यत् । इन्द्रा॑य । म॒रुत्व॑ते । पु॒रो॒डाश᳚म् । एका॑दशकपाल॒मित्येका॑दश - क॒पा॒ल॒म् । नि॒र्वप॒न्तीति॑ निः-वप॑न्ति । दे॒वता᳚म् । ए॒व । तत् । भा॒गिनी᳚म् । कु॒र्वन्ति॑ । सव॑नम् । ए॒का॒द॒शभि॒रित्ये॑काद॒श - भिः॒ । उपेति॑ । य॒न्ति॒ । विश्वे॑षाम् । वै । दे॒वाना᳚म् । ऋ॒भु॒मता॒मित्यृ॑भु - मता᳚म् । तृ॒ती॒य॒स॒व॒नमिति॑ तृतीय-स॒व॒नम् । यत् । वै॒श्व॒दे॒वमिति॑ वैश्व-दे॒वम् । द्वाद॑शकपाल॒मिति॒ द्वाद॑श - क॒पा॒ल॒म् । नि॒र्वप॒न्तीति॑ निः - वप॑न्ति । दे॒वताः᳚ । ए॒व । तत् । भा॒गिनीः᳚ । कु॒र्वन्ति॑ । सव॑नम् । द्वा॒द॒शभि॒रिति॑ द्वाद॒श - भिः॒ ।  \newline


\textbf{Krama Paata} \newline

सव॑नमष्टा॒भिः । अ॒ष्टा॒भिरुप॑ । उप॑ यन्ति । य॒न्ति॒ यत् । यदै॒न्द्रम् । ऐ॒न्द्रम् दधि॑ । दधि॒ भव॑ति । भव॒तीन्द्र᳚म् । इन्द्र॑मे॒व । ए॒व तत् । तद् भा॑ग॒धेया᳚त् । भा॒ग॒धेया॒न् न । भा॒ग॒धेया॒दिति॑ भाग - धेया᳚त् । न च्या॑वयन्ति । च्या॒व॒य॒न्तीन्द्र॑स्य । इन्द्र॑स्य॒ वै । वै म॒रुत्व॑तः । म॒रुत्व॑तो॒ माद्ध्य॑न्दिनम् । माद्ध्य॑न्दिनꣳ॒॒ सव॑नम् । सव॑न॒म् ॅयत् । यदिन्द्रा॑य । इन्द्रा॑य म॒रुत्व॑ते । म॒रुत्व॑ते पुरो॒डाश᳚म् । पु॒रो॒डाश॒मेका॑दशकपालम् । एका॑दशकपालम् नि॒र्वप॑न्ति । एका॑दशकपाल॒मित्येका॑दश - क॒पा॒ल॒म् । नि॒र्वप॑न्ति दे॒वता᳚म् । नि॒र्वप॒न्तीति॑ निः - वप॑न्ति । दे॒वता॑मे॒व । ए॒व तत् । तद् भा॒गिनी᳚म् । भा॒गिनी᳚म् कु॒र्वन्ति॑ । कु॒र्वन्ति॒ सव॑नम् । सव॑नमेकाद॒शभिः॑ । ए॒का॒द॒शभि॒रुप॑ । ए॒का॒द॒शभि॒रित्ये॑काद॒श - भिः॒ । उप॑ यन्ति । य॒न्ति॒ विश्वे॑षाम् । विश्वे॑षा॒म् ॅवै । वै दे॒वाना᳚म् । दे॒वाना॑मृभु॒मता᳚म् । ऋ॒भु॒मता᳚म् तृतीयसव॒नम् । ऋ॒भु॒मता॒मित्यृ॑भु - मता᳚म् । तृ॒ती॒य॒स॒व॒नम् ॅयत् । तृ॒ती॒य॒स॒व॒नमिति॑ तृतीय - स॒व॒नम् । यद् वै᳚श्वदे॒वम् । वै॒श्व॒दे॒वम् द्वाद॑शकपालम् । वै॒श्व॒दे॒वमिति॑ वैश्व - दे॒वम् । द्वाद॑शकपालम् नि॒र्वप॑न्ति । द्वाद॑शकपाल॒मिति॒ द्वाद॑श - क॒पा॒ल॒म् । नि॒र्वप॑न्ति दे॒वताः᳚ । नि॒र्वप॒न्तीति॑ निः - वप॑न्ति । दे॒वता॑ ए॒व । ए॒व तत् । तद् भा॒गिनीः᳚ । भा॒गिनीः᳚ कु॒र्वन्ति॑ । कु॒र्वन्ति॒ सव॑नम् । सव॑नम् द्वाद॒शभिः॑ । द्वा॒द॒शभि॒रुप॑ । द्वा॒द॒शभि॒रिति॑ द्वाद॒श - भिः॒ \newline

\textbf{Jatai Paata} \newline

1. सव॑न मष्टा॒भि र॑ष्टा॒भिः सव॑नꣳ॒॒ सव॑न मष्टा॒भिः । \newline
2. अ॒ष्टा॒भि रुपोपा᳚ ष्टा॒भि र॑ष्टा॒भि रुप॑ । \newline
3. उप॑ यन्ति य॒न्त्युपोप॑ यन्ति । \newline
4. य॒न्ति॒ यद् यद् य॑न्ति यन्ति॒ यत् । \newline
5. यदै॒न्द्र मै॒न्द्रं ॅयद् यदै॒न्द्रम् । \newline
6. ऐ॒न्द्रम् दधि॒ दध्यै॒न्द्र मै॒न्द्रम् दधि॑ । \newline
7. दधि॒ भव॑ति॒ भव॑ति॒ दधि॒ दधि॒ भव॑ति । \newline
8. भव॒तीन्द्र॒ मिन्द्र॒म् भव॑ति॒ भव॒तीन्द्र᳚म् । \newline
9. इन्द्र॑ मे॒वैवेन्द्र॒ मिन्द्र॑ मे॒व । \newline
10. ए॒व तत् तदे॒वैव तत् । \newline
11. तद् भा॑ग॒धेया᳚द् भाग॒धेया॒त् तत् तद् भा॑ग॒धेया᳚त् । \newline
12. भा॒ग॒धेया॒न् न न भा॑ग॒धेया᳚द् भाग॒धेया॒न् न । \newline
13. भा॒ग॒धेया॒दिति॑ भाग - धेया᳚त् । \newline
14. न च्या॑वयन्ति च्यावयन्ति॒ न न च्या॑वयन्ति । \newline
15. च्या॒व॒य॒न् तीन्द्र॒ स्येन्द्र॑स्य च्यावयन्ति च्यावय॒ न्तीन्द्र॑स्य । \newline
16. इन्द्र॑स्य॒ वै वा इन्द्र॒ स्येन्द्र॑स्य॒ वै । \newline
17. वै म॒रुत्व॑तो म॒रुत्व॑तो॒ वै वै म॒रुत्व॑तः । \newline
18. म॒रुत्व॑तो॒ माद्ध्य॑न्दिन॒म् माद्ध्य॑न्दिनम् म॒रुत्व॑तो म॒रुत्व॑तो॒ माद्ध्य॑न्दिनम् । \newline
19. माद्ध्य॑न्दिनꣳ॒॒ सव॑नꣳ॒॒ सव॑न॒म् माद्ध्य॑न्दिन॒म् माद्ध्य॑न्दिनꣳ॒॒ सव॑नम् । \newline
20. सव॑नं॒ ॅयद् यथ् सव॑नꣳ॒॒ सव॑नं॒ ॅयत् । \newline
21. यदिन्द्रा॒ येन्द्रा॑य॒ यद् यदिन्द्रा॑य । \newline
22. इन्द्रा॑य म॒रुत्व॑ते म॒रुत्व॑त॒ इन्द्रा॒ येन्द्रा॑य म॒रुत्व॑ते । \newline
23. म॒रुत्व॑ते पुरो॒डाश॑म् पुरो॒डाश॑म् म॒रुत्व॑ते म॒रुत्व॑ते पुरो॒डाश᳚म् । \newline
24. पु॒रो॒डाश॒ मेका॑दशकपाल॒ मेका॑दशकपालम् पुरो॒डाश॑म् पुरो॒डाश॒ मेका॑दशकपालम् । \newline
25. एका॑दशकपालन् नि॒र्वप॑न्ति नि॒र्वप॒ न्त्येका॑दशकपाल॒ मेका॑दशकपालन् नि॒र्वप॑न्ति । \newline
26. एका॑दशकपाल॒मित्येका॑दश - क॒पा॒ल॒म् । \newline
27. नि॒र्वप॑न्ति दे॒वता᳚म् दे॒वता᳚न् नि॒र्वप॑न्ति नि॒र्वप॑न्ति दे॒वता᳚म् । \newline
28. नि॒र्वप॒न्तीति॑ निः - वप॑न्ति । \newline
29. दे॒वता॑ मे॒वैव दे॒वता᳚म् दे॒वता॑ मे॒व । \newline
30. ए॒व तत् तदे॒ वैव तत् । \newline
31. तद् भा॒गिनी᳚म् भा॒गिनी॒म् तत् तद् भा॒गिनी᳚म् । \newline
32. भा॒गिनी᳚म् कु॒र्वन्ति॑ कु॒र्वन्ति॑ भा॒गिनी᳚म् भा॒गिनी᳚म् कु॒र्वन्ति॑ । \newline
33. कु॒र्वन्ति॒ सव॑नꣳ॒॒ सव॑नम् कु॒र्वन्ति॑ कु॒र्वन्ति॒ सव॑नम् । \newline
34. सव॑न मेकाद॒शभि॑ रेकाद॒शभिः॒ सव॑नꣳ॒॒ सव॑न मेकाद॒शभिः॑ । \newline
35. ए॒का॒द॒शभि॒ रुपो पै॑काद॒शभि॑ रेकाद॒शभि॒ रुप॑ । \newline
36. ए॒का॒द॒शभि॒रित्ये॑काद॒श - भिः॒ । \newline
37. उप॑ यन्ति य॒न्त्युपोप॑ यन्ति । \newline
38. य॒न्ति॒ विश्वे॑षां॒ ॅविश्वे॑षां ॅयन्ति यन्ति॒ विश्वे॑षाम् । \newline
39. विश्वे॑षां॒ ॅवै वै विश्वे॑षां॒ ॅविश्वे॑षां॒ ॅवै । \newline
40. वै दे॒वाना᳚म् दे॒वानां॒ ॅवै वै दे॒वाना᳚म् । \newline
41. दे॒वाना॑ मृभु॒मता॑ मृभु॒मता᳚म् दे॒वाना᳚म् दे॒वाना॑ मृभु॒मता᳚म् । \newline
42. ऋ॒भु॒मता᳚म् तृतीयसव॒नम् तृ॑तीयसव॒न मृ॑भु॒मता॑ मृभु॒मता᳚म् तृतीयसव॒नम् । \newline
43. ऋ॒भु॒मता॒मित्यृ॑भु - मता᳚म् । \newline
44. तृ॒ती॒य॒स॒व॒नं ॅयद् यत् तृ॑तीयसव॒नम् तृ॑तीयसव॒नं ॅयत् । \newline
45. तृ॒ती॒य॒स॒व॒नमिति॑ तृतीय - स॒व॒नम् । \newline
46. यद् वै᳚श्वदे॒वं ॅवै᳚श्वदे॒वं ॅयद् यद् वै᳚श्वदे॒वम् । \newline
47. वै॒श्व॒दे॒वम् द्वाद॑शकपाल॒म् द्वाद॑शकपालं ॅवैश्वदे॒वं ॅवै᳚श्वदे॒वम् द्वाद॑शकपालम् । \newline
48. वै॒श्व॒दे॒वमिति॑ वैश्व - दे॒वम् । \newline
49. द्वाद॑शकपालन् नि॒र्वप॑न्ति नि॒र्वप॑न्ति॒ द्वाद॑शकपाल॒म् द्वाद॑शकपालन् नि॒र्वप॑न्ति । \newline
50. द्वाद॑शकपाल॒मिति॒ द्वाद॑श - क॒पा॒ल॒म् । \newline
51. नि॒र्वप॑न्ति दे॒वता॑ दे॒वता॑ नि॒र्वप॑न्ति नि॒र्वप॑न्ति दे॒वताः᳚ । \newline
52. नि॒र्वप॒न्तीति॑ निः - वप॑न्ति । \newline
53. दे॒वता॑ ए॒वैव दे॒वता॑ दे॒वता॑ ए॒व । \newline
54. ए॒व तत् तदे॒वैव तत् । \newline
55. तद् भा॒गिनी᳚र् भा॒गिनी॒ स्तत् तद् भा॒गिनीः᳚ । \newline
56. भा॒गिनीः᳚ कु॒र्वन्ति॑ कु॒र्वन्ति॑ भा॒गिनी᳚र् भा॒गिनीः᳚ कु॒र्वन्ति॑ । \newline
57. कु॒र्वन्ति॒ सव॑नꣳ॒॒ सव॑नम् कु॒र्वन्ति॑ कु॒र्वन्ति॒ सव॑नम् । \newline
58. सव॑नम् द्वाद॒शभि॑र् द्वाद॒शभिः॒ सव॑नꣳ॒॒ सव॑नम् द्वाद॒शभिः॑ । \newline
59. द्वा॒द॒शभि॒ रुपोप॑ द्वाद॒शभि॑र् द्वाद॒शभि॒ रुप॑ । \newline
60. द्वा॒द॒शभि॒रिति॑ द्वाद॒श - भिः॒ । \newline

\textbf{Ghana Paata } \newline

1. सव॑न मष्टा॒भि र॑ष्टा॒भिः सव॑नꣳ॒॒ सव॑न मष्टा॒भि रुपो पा᳚ष्टा॒भिः सव॑नꣳ॒॒ सव॑न मष्टा॒भि रुप॑ । \newline
2. अ॒ष्टा॒भि रुपो पा᳚ष्टा॒भि र॑ष्टा॒भि रुप॑ यन्ति य॒न्त्युपा᳚ष्टा॒भि र॑ष्टा॒भि रुप॑ यन्ति । \newline
3. उप॑ यन्ति य॒न्त्युपोप॑ यन्ति॒ यद् यद् य॒न्त्युपोप॑ यन्ति॒ यत् । \newline
4. य॒न्ति॒ यद् यद् य॑न्ति यन्ति॒ यदै॒न्द्र मै॒न्द्रं ॅयद् य॑न्ति यन्ति॒ यदै॒न्द्रम् । \newline
5. यदै॒न्द्र मै॒न्द्रं ॅयद् यदै॒न्द्रम् दधि॒ दध्यै॒न्द्रं ॅयद् यदै॒न्द्रम् दधि॑ । \newline
6. ऐ॒न्द्रम् दधि॒ दध्यै॒न्द्र मै॒न्द्रम् दधि॒ भव॑ति॒ भव॑ति॒ दध्यै॒न्द्र मै॒न्द्रम् दधि॒ भव॑ति । \newline
7. दधि॒ भव॑ति॒ भव॑ति॒ दधि॒ दधि॒ भव॒तीन्द्र॒ मिन्द्र॒म् भव॑ति॒ दधि॒ दधि॒ भव॒तीन्द्र᳚म् । \newline
8. भव॒तीन्द्र॒ मिन्द्र॒म् भव॑ति॒ भव॒तीन्द्र॑ मे॒वैवेन्द्र॒म् भव॑ति॒ भव॒तीन्द्र॑ मे॒व । \newline
9. इन्द्र॑ मे॒वैवेन्द्र॒ मिन्द्र॑ मे॒व तत् तदे॒वेन्द्र॒ मिन्द्र॑ मे॒व तत् । \newline
10. ए॒व तत् तदे॒वैव तद् भा॑ग॒धेया᳚द् भाग॒धेया॒त् तदे॒वैव तद् भा॑ग॒धेया᳚त् । \newline
11. तद् भा॑ग॒धेया᳚द् भाग॒धेया॒त् तत् तद् भा॑ग॒धेया॒न् न न भा॑ग॒धेया॒त् तत् तद् भा॑ग॒धेया॒न् न । \newline
12. भा॒ग॒धेया॒न् न न भा॑ग॒धेया᳚द् भाग॒धेया॒न् न च्या॑वयन्ति च्यावयन्ति॒ न भा॑ग॒धेया᳚द् भाग॒धेया॒न् न च्या॑वयन्ति । \newline
13. भा॒ग॒धेया॒दिति॑ भाग - धेया᳚त् । \newline
14. न च्या॑वयन्ति च्यावयन्ति॒ न न च्या॑वय॒ न्तीन्द्र॒ स्येन्द्र॑स्य च्यावयन्ति॒ न न च्या॑वय॒ न्तीन्द्र॑स्य । \newline
15. च्या॒व॒य॒ न्तीन्द्र॒ स्येन्द्र॑स्य च्यावयन्ति च्यावय॒ न्तीन्द्र॑स्य॒ वै वा इन्द्र॑स्य च्यावयन्ति च्यावय॒ न्तीन्द्र॑स्य॒ वै । \newline
16. इन्द्र॑स्य॒ वै वा इन्द्र॒ स्येन्द्र॑स्य॒ वै म॒रुत्व॑तो म॒रुत्व॑तो॒ वा इन्द्र॒ स्येन्द्र॑स्य॒ वै म॒रुत्व॑तः । \newline
17. वै म॒रुत्व॑तो म॒रुत्व॑तो॒ वै वै म॒रुत्व॑तो॒ माद्ध्य॑न्दिन॒म् माद्ध्य॑न्दिनम् म॒रुत्व॑तो॒ वै वै म॒रुत्व॑तो॒ माद्ध्य॑न्दिनम् । \newline
18. म॒रुत्व॑तो॒ माद्ध्य॑न्दिन॒म् माद्ध्य॑न्दिनम् म॒रुत्व॑तो म॒रुत्व॑तो॒ माद्ध्य॑न्दिनꣳ॒॒ सव॑नꣳ॒॒ सव॑न॒म् माद्ध्य॑न्दिनम् म॒रुत्व॑तो म॒रुत्व॑तो॒ माद्ध्य॑न्दिनꣳ॒॒ सव॑नम् । \newline
19. माद्ध्य॑न्दिनꣳ॒॒ सव॑नꣳ॒॒ सव॑न॒म् माद्ध्य॑न्दिन॒म् माद्ध्य॑न्दिनꣳ॒॒ सव॑नं॒ ॅयद् यथ् सव॑न॒म् माद्ध्य॑न्दिन॒म् माद्ध्य॑न्दिनꣳ॒॒ सव॑नं॒ ॅयत् । \newline
20. सव॑नं॒ ॅयद् यथ् सव॑नꣳ॒॒ सव॑नं॒ ॅयदिन्द्रा॒ येन्द्रा॑य॒ यथ् सव॑नꣳ॒॒ सव॑नं॒ ॅयदिन्द्रा॑य । \newline
21. यदिन्द्रा॒ येन्द्रा॑य॒ यद् यदिन्द्रा॑य म॒रुत्व॑ते म॒रुत्व॑त॒ इन्द्रा॑य॒ यद् यदिन्द्रा॑य म॒रुत्व॑ते । \newline
22. इन्द्रा॑य म॒रुत्व॑ते म॒रुत्व॑त॒ इन्द्रा॒ येन्द्रा॑य म॒रुत्व॑ते पुरो॒डाश॑म् पुरो॒डाश॑म् म॒रुत्व॑त॒ इन्द्रा॒
येन्द्रा॑य म॒रुत्व॑ते पुरो॒डाश᳚म् । \newline
23. म॒रुत्व॑ते पुरो॒डाश॑म् पुरो॒डाश॑म् म॒रुत्व॑ते म॒रुत्व॑ते पुरो॒डाश॒ मेका॑दशकपाल॒ मेका॑दशकपालम् पुरो॒डाश॑म् म॒रुत्व॑ते म॒रुत्व॑ते पुरो॒डाश॒ मेका॑दशकपालम् । \newline
24. पु॒रो॒डाश॒ मेका॑दशकपाल॒ मेका॑दशकपालम् पुरो॒डाश॑म् पुरो॒डाश॒ मेका॑दशकपालन् नि॒र्वप॑न्ति नि॒र्वप॒ न्त्येका॑दशकपालम् पुरो॒डाश॑म् पुरो॒डाश॒ मेका॑दशकपालन् नि॒र्वप॑न्ति । \newline
25. एका॑दशकपालन् नि॒र्वप॑न्ति नि॒र्वप॒ न्त्येका॑दशकपाल॒ मेका॑दशकपालन् नि॒र्वप॑न्ति दे॒वता᳚म् दे॒वता᳚न् नि॒र्वप॒ न्त्येका॑दशकपाल॒ मेका॑दशकपालन् नि॒र्वप॑न्ति दे॒वता᳚म् । \newline
26. एका॑दशकपाल॒मित्येका॑दश - क॒पा॒ल॒म् । \newline
27. नि॒र्वप॑न्ति दे॒वता᳚म् दे॒वता᳚न् नि॒र्वप॑न्ति नि॒र्वप॑न्ति दे॒वता॑ मे॒वैव दे॒वता᳚न् नि॒र्वप॑न्ति नि॒र्वप॑न्ति दे॒वता॑ मे॒व । \newline
28. नि॒र्वप॒न्तीति॑ निः - वप॑न्ति । \newline
29. दे॒वता॑ मे॒वैव दे॒वता᳚म् दे॒वता॑ मे॒व तत् तदे॒व दे॒वता᳚म् दे॒वता॑ मे॒व तत् । \newline
30. ए॒व तत् तदे॒वैव तद् भा॒गिनी᳚म् भा॒गिनी॒म् तदे॒वैव तद् भा॒गिनी᳚म् । \newline
31. तद् भा॒गिनी᳚म् भा॒गिनी॒म् तत् तद् भा॒गिनी᳚म् कु॒र्वन्ति॑ कु॒र्वन्ति॑ भा॒गिनी॒म् तत् तद् भा॒गिनी᳚म् कु॒र्वन्ति॑ । \newline
32. भा॒गिनी᳚म् कु॒र्वन्ति॑ कु॒र्वन्ति॑ भा॒गिनी᳚म् भा॒गिनी᳚म् कु॒र्वन्ति॒ सव॑नꣳ॒॒ सव॑नम् कु॒र्वन्ति॑ भा॒गिनी᳚म् भा॒गिनी᳚म् कु॒र्वन्ति॒ सव॑नम् । \newline
33. कु॒र्वन्ति॒ सव॑नꣳ॒॒ सव॑नम् कु॒र्वन्ति॑ कु॒र्वन्ति॒ सव॑न मेकाद॒शभि॑ रेकाद॒शभिः॒ सव॑नम् कु॒र्वन्ति॑ कु॒र्वन्ति॒ सव॑न मेकाद॒शभिः॑ । \newline
34. सव॑न मेकाद॒शभि॑ रेकाद॒शभिः॒ सव॑नꣳ॒॒ सव॑न मेकाद॒शभि॒ रुपो पै॑काद॒शभिः॒ सव॑नꣳ॒॒ सव॑न मेकाद॒शभि॒ रुप॑ । \newline
35. ए॒का॒द॒शभि॒ रुपो पै॑काद॒शभि॑ रेकाद॒शभि॒ रुप॑ यन्ति य॒न्त्यु पै॑काद॒शभि॑ रेकाद॒शभि॒ रुप॑ यन्ति । \newline
36. ए॒का॒द॒शभि॒रित्ये॑काद॒श - भिः॒ । \newline
37. उप॑ यन्ति य॒न्त्युपोप॑ यन्ति॒ विश्वे॑षां॒ ॅविश्वे॑षां ॅय॒न्त्युपोप॑ यन्ति॒ विश्वे॑षाम् । \newline
38. य॒न्ति॒ विश्वे॑षां॒ ॅविश्वे॑षां ॅयन्ति यन्ति॒ विश्वे॑षां॒ ॅवै वै विश्वे॑षां ॅयन्ति यन्ति॒ विश्वे॑षां॒ ॅवै । \newline
39. विश्वे॑षां॒ ॅवै वै विश्वे॑षां॒ ॅविश्वे॑षां॒ ॅवै दे॒वाना᳚म् दे॒वानां॒ ॅवै विश्वे॑षां॒ ॅविश्वे॑षां॒ ॅवै दे॒वाना᳚म् । \newline
40. वै दे॒वाना᳚म् दे॒वानां॒ ॅवै वै दे॒वाना॑ मृभु॒मता॑ मृभु॒मता᳚म् दे॒वानां॒ ॅवै वै दे॒वाना॑ मृभु॒मता᳚म् । \newline
41. दे॒वाना॑ मृभु॒मता॑ मृभु॒मता᳚म् दे॒वाना᳚म् दे॒वाना॑ मृभु॒मता᳚म् तृतीयसव॒नम् तृ॑तीयसव॒न मृ॑भु॒मता᳚म् दे॒वाना᳚म् दे॒वाना॑ मृभु॒मता᳚म् तृतीयसव॒नम् । \newline
42. ऋ॒भु॒मता᳚म् तृतीयसव॒नम् तृ॑तीयसव॒न मृ॑भु॒मता॑ मृभु॒मता᳚म् तृतीयसव॒नं ॅयद् यत् तृ॑तीयसव॒न मृ॑भु॒मता॑ मृभु॒मता᳚म् तृतीयसव॒नं ॅयत् । \newline
43. ऋ॒भु॒मता॒मित्यृ॑भु - मता᳚म् । \newline
44. तृ॒ती॒य॒स॒व॒नं ॅयद् यत् तृ॑तीयसव॒नम् तृ॑तीयसव॒नं ॅयद् वै᳚श्वदे॒वं ॅवै᳚श्वदे॒वं ॅयत् तृ॑तीयसव॒नम् तृ॑तीयसव॒नं ॅयद् वै᳚श्वदे॒वम् । \newline
45. तृ॒ती॒य॒स॒व॒नमिति॑ तृतीय - स॒व॒नम् । \newline
46. यद् वै᳚श्वदे॒वं ॅवै᳚श्वदे॒वं ॅयद् यद् वै᳚श्वदे॒वम् द्वाद॑शकपाल॒म् द्वाद॑शकपालं ॅवैश्वदे॒वं ॅयद् यद् वै᳚श्वदे॒वम् द्वाद॑शकपालम् । \newline
47. वै॒श्व॒दे॒वम् द्वाद॑शकपाल॒म् द्वाद॑शकपालं ॅवैश्वदे॒वं ॅवै᳚श्वदे॒वम् द्वाद॑शकपालन् नि॒र्वप॑न्ति नि॒र्वप॑न्ति॒ द्वाद॑शकपालं ॅवैश्वदे॒वं ॅवै᳚श्वदे॒वम् द्वाद॑शकपालन् नि॒र्वप॑न्ति । \newline
48. वै॒श्व॒दे॒वमिति॑ वैश्व - दे॒वम् । \newline
49. द्वाद॑शकपालन् नि॒र्वप॑न्ति नि॒र्वप॑न्ति॒ द्वाद॑शकपाल॒म् द्वाद॑शकपालन् नि॒र्वप॑न्ति दे॒वता॑ दे॒वता॑ नि॒र्वप॑न्ति॒ द्वाद॑शकपाल॒म् द्वाद॑शकपालन् नि॒र्वप॑न्ति दे॒वताः᳚ । \newline
50. द्वाद॑शकपाल॒मिति॒ द्वाद॑श - क॒पा॒ल॒म् । \newline
51. नि॒र्वप॑न्ति दे॒वता॑ दे॒वता॑ नि॒र्वप॑न्ति नि॒र्वप॑न्ति दे॒वता॑ ए॒वैव दे॒वता॑ नि॒र्वप॑न्ति नि॒र्वप॑न्ति दे॒वता॑ ए॒व । \newline
52. नि॒र्वप॒न्तीति॑ निः - वप॑न्ति । \newline
53. दे॒वता॑ ए॒वैव दे॒वता॑ दे॒वता॑ ए॒व तत् तदे॒व दे॒वता॑ दे॒वता॑ ए॒व तत् । \newline
54. ए॒व तत् तदे॒वैव तद् भा॒गिनी᳚र् भा॒गिनी॒ स्तदे॒वैव तद् भा॒गिनीः᳚ । \newline
55. तद् भा॒गिनी᳚र् भा॒गिनी॒ स्तत् तद् भा॒गिनीः᳚ कु॒र्वन्ति॑ कु॒र्वन्ति॑ भा॒गिनी॒ स्तत् तद् भा॒गिनीः᳚ कु॒र्वन्ति॑ । \newline
56. भा॒गिनीः᳚ कु॒र्वन्ति॑ कु॒र्वन्ति॑ भा॒गिनी᳚र् भा॒गिनीः᳚ कु॒र्वन्ति॒ सव॑नꣳ॒॒ सव॑नम् कु॒र्वन्ति॑ भा॒गिनी᳚र् भा॒गिनीः᳚ कु॒र्वन्ति॒ सव॑नम् । \newline
57. कु॒र्वन्ति॒ सव॑नꣳ॒॒ सव॑नम् कु॒र्वन्ति॑ कु॒र्वन्ति॒ सव॑नम् द्वाद॒शभि॑र् द्वाद॒शभिः॒ सव॑नम् कु॒र्वन्ति॑ कु॒र्वन्ति॒ सव॑नम् द्वाद॒शभिः॑ । \newline
58. सव॑नम् द्वाद॒शभि॑र् द्वाद॒शभिः॒ सव॑नꣳ॒॒ सव॑नम् द्वाद॒शभि॒ रुपोप॑ द्वाद॒शभिः॒ सव॑नꣳ॒॒ सव॑नम् द्वाद॒शभि॒ रुप॑ । \newline
59. द्वा॒द॒शभि॒ रुपोप॑ द्वाद॒शभि॑र् द्वाद॒शभि॒ रुप॑ यन्ति य॒न्त्युप॑ द्वाद॒शभि॑र् द्वाद॒शभि॒ रुप॑ यन्ति । \newline
60. द्वा॒द॒शभि॒रिति॑ द्वाद॒श - भिः॒ । \newline
\pagebreak
\markright{ TS 7.5.7.4  \hfill https://www.vedavms.in \hfill}

\section{ TS 7.5.7.4 }

\textbf{TS 7.5.7.4 } \newline
\textbf{Samhita Paata} \newline

-रुप॑ यन्ति प्राजाप॒त्यं प॒शुमा ल॑भन्ते य॒ज्ञो वै प्र॒जाप॑ति-र्य॒ज्ञ्स्या-न॑नुसर्गायाभिव॒र्त इ॒तः षण्मा॒सो ब्र॑ह्मसा॒मं भ॑वति॒ ब्रह्म॒ वा अ॑भिव॒र्तो ब्रह्म॑णै॒व तथ् सु॑व॒र्गं ॅलो॒क-म॑भिव॒र्तय॑न्तो यन्ति प्रतिकू॒लमि॑व॒ हीतः सु॑व॒र्गो लो॒क इन्द्र॒ क्रतुं॑ न॒ आ भ॑र पि॒ता पु॒त्रेभ्यो॒ यथा᳚ । शिक्षा॑ नो अ॒स्मिन् पु॑रुहूत॒ याम॑नि जी॒वा ज्योति॑रशीम॒हीत्य॒ ( )-मुत॑ आय॒ताꣳ षण्मा॒सो ब्र॑ह्मसा॒मं भ॑वत्य॒यं ॅवै लो॒को ज्योतिः॑ प्र॒जा ज्योति॑रि॒ममे॒व तल्लो॒कं पश्य॑न्तोऽभि॒वद॑न्त॒ आ य॑न्ति ॥ \newline

\textbf{Pada Paata} \newline

उपेति॑ । य॒न्ति॒ । प्रा॒जा॒प॒त्यमिति॑ प्राजा - प॒त्यम् । प॒शुम् । एति॑ । ल॒भ॒न्ते॒ । य॒ज्ञ्ः । वै । प्र॒जाप॑ति॒रिति॑ प्र॒जा - प॒तिः॒ । य॒ज्ञ्स्य॑ । अन॑नुसर्गा॒येत्यन॑नु - स॒र्गा॒य॒ । अ॒भि॒व॒र्त इत्य॑भि - व॒र्तः । इ॒तः । षट् । मा॒सः । ब्र॒ह्म॒सा॒ममिति॑ ब्रह्म - सा॒मम् । भ॒व॒ति॒ । ब्रह्म॑ । वै । अ॒भि॒व॒र्त इत्य॑भि - व॒र्तः । ब्रह्म॑णा । ए॒व । तत् । सु॒व॒र्गमिति॑ सुवः - गम् । लो॒कम् । अ॒भि॒व॒र्तय॑न्त॒ इत्य॑भि - व॒र्तय॑न्तः । य॒न्ति॒ । प्र॒ति॒कू॒लमिति॑ प्रति - कू॒लम् । इ॒व॒ । हि । इ॒तः । सु॒व॒र्ग इति॑ सुवः - गः । लो॒कः । इन्द्र॑ । क्रतु᳚म् । नः॒ । एति॑ । भ॒र॒ । पि॒ता । पु॒त्रेभ्यः॑ । यथा᳚ ॥ शिक्ष॑ । नः॒ । अ॒स्मिन्न् । पु॒रु॒हू॒तेति॑ पुरु - हू॒त॒ । याम॑नि । जी॒वाः । ज्योतिः॑ । अ॒शी॒म॒हि॒ । इति॑ ( ) । अ॒मुतः॑ । आ॒य॒तामित्या᳚ - य॒ताम् । षट् । मा॒सः । ब्र॒ह्म॒सा॒ममिति॑ ब्रह्म-सा॒मम् । भ॒व॒ति॒ । अ॒यम् । वै । लो॒कः । ज्योतिः॑ । प्र॒जेति॑ प्र-जा । ज्योतिः॑ । इ॒मम् । ए॒व । तत् । लो॒कम् । पश्य॑न्तः । अ॒भि॒वद॑न्त॒इत्य॑भि - वद॑न्तः । एति॑ । य॒न्ति॒ ॥  \newline


\textbf{Krama Paata} \newline

उप॑ यन्ति । य॒न्ति॒ प्रा॒जा॒प॒त्यम् । प्रा॒जा॒प॒त्यम् प॒शुम् । प्रा॒जा॒प॒त्यमिति॑ प्राजा - प॒त्यम् । प॒शुमा । आ ल॑भन्ते । ल॒भ॒न्ते॒ य॒ज्ञ्ः । य॒ज्ञो वै । वै प्र॒जाप॑तिः । प्र॒जाप॑तिर् य॒ज्ञ्स्य॑ । प्र॒जाप॑ति॒रिति॑ प्र॒जा - प॒तिः॒ । य॒ज्ञ्स्यान॑नुसर्गाय । अन॑नुसर्गायाभिव॒र्तः । अन॑नुसर्गा॒येत्यन॑नु - स॒र्गा॒य॒ । अ॒भि॒व॒र्त इ॒तः । अ॒भि॒व॒र्त इत्य॑भि - व॒र्तः । इ॒तः षट् । षण्मा॒सः । मा॒सो ब्र॑ह्मसा॒मम् । ब्र॒ह्म॒सा॒मम् भ॑वति । ब्र॒ह्म॒सा॒ममिति॑ ब्रह्म - सा॒मम् । भ॒व॒ति॒ ब्रह्म॑ । ब्रह्म॒ वै । वा अ॑भिव॒र्तः । अ॒भि॒व॒र्तो ब्रह्म॑णा । अ॒भि॒व॒र्त इत्य॑भि - व॒र्तः । ब्रह्म॑णै॒व । ए॒व तत् । तथ् सु॑व॒र्गम् । सु॒व॒र्गम् ॅलो॒कम् । सु॒व॒र्गमिति॑ सुवः - गम् । लो॒कम॑भिव॒र्तय॑न्तः । अ॒भि॒व॒र्तय॑न्तो यन्ति । अ॒भि॒व॒र्तय॑न्त॒ इत्य॑भि - व॒र्तय॑न्तः । य॒न्ति॒ प्र॒ति॒कू॒लम् । प्र॒ति॒कू॒लमि॑व । प्र॒ति॒कू॒लमिति॑ प्रति - कू॒लम् । इ॒व॒ हि । हीतः । इ॒तः सु॑व॒र्गः । सु॒व॒र्गो लो॒कः । सु॒व॒र्ग इति॑ सुवः - गः । लो॒क इन्द्र॑ । इन्द्र॒ क्रतु᳚म् । क्रतु॑म् नः । न॒ आ । आ भ॑र । भ॒र॒ पि॒ता । पि॒ता पु॒त्रेभ्यः॑ । पु॒त्रेभ्यो॒ यथा᳚ । यथेति॒ यथा᳚ ॥ शिक्षा॑ नः । नो॒ अ॒स्मिन्न् । अ॒स्मिन् पु॑रुहूत । पु॒रु॒हू॒त॒ याम॑नि । पु॒रु॒हू॒तेति॑ पुरु - हू॒त॒ । याम॑नि जी॒वाः । जी॒वा ज्योतिः॑ । ज्योति॑रशीमहि । अ॒शी॒म॒हीति॑ ( ) । इत्य॒मुतः॑ । अ॒मुत॑ आय॒ताम् । आ॒य॒ताꣳ षट् । आ॒य॒तामित्या᳚ - य॒ताम् । षण्मा॒सः । मा॒सो ब्र॑ह्मसा॒मम् । ब्र॒ह्म॒सा॒मम् भ॑वति । ब्र॒ह्म॒सा॒ममिति॑ ब्रह्म - सा॒मम् । भ॒व॒त्य॒यम् । अ॒यम् ॅवै । वै लो॒कः । लो॒को ज्योतिः॑ । ज्योतिः॑ प्र॒जा । प्र॒जा ज्योतिः॑ । प्र॒जेति॑ प्र - जा । ज्योति॑रि॒मम् । इ॒ममे॒व । ए॒व तत् । तल्लो॒कम् । लो॒कम् पश्य॑न्तः । पश्य॑न्तोऽभि॒वद॑न्तः । अ॒भि॒वद॑न्त॒ आ । अ॒भि॒वद॑न्त॒ इत्य॑भि - वद॑न्तः । आ य॑न्ति । य॒न्तीति॑ यन्ति । \newline

\textbf{Jatai Paata} \newline

1. उप॑ यन्ति य॒न्त्युपोप॑ यन्ति । \newline
2. य॒न्ति॒ प्रा॒जा॒प॒त्यम् प्रा॑जाप॒त्यं ॅय॑न्ति यन्ति प्राजाप॒त्यम् । \newline
3. प्रा॒जा॒प॒त्यम् प॒शुम् प॒शुम् प्रा॑जाप॒त्यम् प्रा॑जाप॒त्यम् प॒शुम् । \newline
4. प्रा॒जा॒प॒त्यमिति॑ प्राजा - प॒त्यम् । \newline
5. प॒शु मा प॒शुम् प॒शु मा । \newline
6. आ ल॑भन्ते लभन्त॒ आ ल॑भन्ते । \newline
7. ल॒भ॒न्ते॒ य॒ज्ञो य॒ज्ञो ल॑भन्ते लभन्ते य॒ज्ञ्ः । \newline
8. य॒ज्ञो वै वै य॒ज्ञो य॒ज्ञो वै । \newline
9. वै प्र॒जाप॑तिः प्र॒जाप॑ति॒र् वै वै प्र॒जाप॑तिः । \newline
10. प्र॒जाप॑तिर् य॒ज्ञ्स्य॑ य॒ज्ञ्स्य॑ प्र॒जाप॑तिः प्र॒जाप॑तिर् य॒ज्ञ्स्य॑ । \newline
11. प्र॒जाप॑ति॒रिति॑ प्र॒जा - प॒तिः॒ । \newline
12. य॒ज्ञ्स्या न॑नुसर्गा॒या न॑नुसर्गाय य॒ज्ञ्स्य॑ य॒ज्ञ्स्या न॑नुसर्गाय । \newline
13. अन॑नुसर्गाया भिव॒र्तो॑ ऽभिव॒र्तो ऽन॑नुसर्गा॒या न॑नुसर्गाया भिव॒र्तः । \newline
14. अन॑नुसर्गा॒येत्यन॑नु - स॒र्गा॒य॒ । \newline
15. अ॒भि॒व॒र्त इ॒त इ॒तो॑ ऽभिव॒र्तो॑ ऽभिव॒र्त इ॒तः । \newline
16. अ॒भि॒व॒र्त इत्य॑भि - व॒र्तः । \newline
17. इ॒त ष्षट् थ्षडि॒त इ॒त ष्षट् । \newline
18. षण् मा॒सो मा॒स ष्षट् थ्षण् मा॒सः । \newline
19. मा॒सो ब्र॑ह्मसा॒मम् ब्र॑ह्मसा॒मम् मा॒सो मा॒सो ब्र॑ह्मसा॒मम् । \newline
20. ब्र॒ह्म॒सा॒मम् भ॑वति भवति ब्रह्मसा॒मम् ब्र॑ह्मसा॒मम् भ॑वति । \newline
21. ब्र॒ह्म॒सा॒ममिति॑ ब्रह्म - सा॒मम् । \newline
22. भ॒व॒ति॒ ब्रह्म॒ ब्रह्म॑ भवति भवति॒ ब्रह्म॑ । \newline
23. ब्रह्म॒ वै वै ब्रह्म॒ ब्रह्म॒ वै । \newline
24. वा अ॑भिव॒र्तो॑ ऽभिव॒र्तो वै वा अ॑भिव॒र्तः । \newline
25. अ॒भि॒व॒र्तो ब्रह्म॑णा॒ ब्रह्म॑णा ऽभिव॒र्तो॑ ऽभिव॒र्तो ब्रह्म॑णा । \newline
26. अ॒भि॒व॒र्त इत्य॑भि - व॒र्तः । \newline
27. ब्रह्म॑णै॒वैव ब्रह्म॑णा॒ ब्रह्म॑णै॒व । \newline
28. ए॒व तत् तदे॒वैव तत् । \newline
29. तथ् सु॑व॒र्गꣳ सु॑व॒र्गम् तत् तथ् सु॑व॒र्गम् । \newline
30. सु॒व॒र्गम् ॅलो॒कम् ॅलो॒कꣳ सु॑व॒र्गꣳ सु॑व॒र्गम् ॅलो॒कम् । \newline
31. सु॒व॒र्गमिति॑ सुवः - गम् । \newline
32. लो॒क म॑भिव॒र्तय॑न्तो ऽभिव॒र्तय॑न्तो लो॒कम् ॅलो॒क म॑भिव॒र्तय॑न्तः । \newline
33. अ॒भि॒व॒र्तय॑न्तो यन्ति यन्त्यभिव॒र्तय॑न्तो ऽभिव॒र्तय॑न्तो यन्ति । \newline
34. अ॒भि॒व॒र्तय॑न्त॒ इत्य॑भि - व॒र्तय॑न्तः । \newline
35. य॒न्ति॒ प्र॒ति॒कू॒लम् प्र॑तिकू॒लं ॅय॑न्ति यन्ति प्रतिकू॒लम् । \newline
36. प्र॒ति॒कू॒ल मि॑वेव प्रतिकू॒लम् प्र॑तिकू॒ल मि॑व । \newline
37. प्र॒ति॒कू॒लमिति॑ प्रति - कू॒लम् । \newline
38. इ॒व॒ हि हीवे॑व॒ हि । \newline
39. हीत इ॒तो हि हीतः । \newline
40. इ॒तः सु॑व॒र्गः सु॑व॒र्ग इ॒त इ॒तः सु॑व॒र्गः । \newline
41. सु॒व॒र्गो लो॒को लो॒कः सु॑व॒र्गः सु॑व॒र्गो लो॒कः । \newline
42. सु॒व॒र्ग इति॑ सुवः - गः । \newline
43. लो॒क इन्द्रेन्द्र॑ लो॒को लो॒क इन्द्र॑ । \newline
44. इन्द्र॒ क्रतु॒म् क्रतु॒ मिन्द्रेन्द्र॒ क्रतु᳚म् । \newline
45. क्रतु॑न् नो नः॒ क्रतु॒म् क्रतु॑न् नः । \newline
46. न॒ आ नो॑ न॒ आ । \newline
47. आ भ॑र भ॒रा भ॑र । \newline
48. भ॒र॒ पि॒ता पि॒ता भ॑र भर पि॒ता । \newline
49. पि॒ता पु॒त्रेभ्यः॑ पु॒त्रेभ्यः॑ पि॒ता पि॒ता पु॒त्रेभ्यः॑ । \newline
50. पु॒त्रेभ्यो॒ यथा॒ यथा॑ पु॒त्रेभ्यः॑ पु॒त्रेभ्यो॒ यथा᳚ । \newline
51. यथेति॒ यथा᳚ । \newline
52. शिक्षा॑ नो नः॒ शिक्ष॒ शिक्षा॑ नः । \newline
53. नो॒ अ॒स्मिन् न॒स्मिन् नो॑ नो अ॒स्मिन्न् । \newline
54. अ॒स्मिन् पु॑रुहूत पुरुहूता॒स्मिन् न॒स्मिन् पु॑रुहूत । \newline
55. पु॒रु॒हू॒त॒ याम॑नि॒ याम॑नि पुरुहूत पुरुहूत॒ याम॑नि । \newline
56. पु॒रु॒हू॒तेति॑ पुरु - हू॒त॒ । \newline
57. याम॑नि जी॒वा जी॒वा याम॑नि॒ याम॑नि जी॒वाः । \newline
58. जी॒वा ज्योति॒र् ज्योति॑र् जी॒वा जी॒वा ज्योतिः॑ । \newline
59. ज्योति॑ रशीमह्य शीमहि॒ ज्योति॒र् ज्योति॑ रशीमहि । \newline
60. अ॒शी॒म॒हीती त्य॑शीमह्य शीम॒हीति॑ । \newline
61. इत्य॒मुतो॒ ऽमुत॒ इती त्य॒मुतः॑ । \newline
62. अ॒मुत॑ आय॒ता मा॑य॒ता म॒मुतो॒ ऽमुत॑ आय॒ताम् । \newline
63. आ॒य॒ताꣳ षट् थ्षडा॑य॒ता मा॑य॒ताꣳ षट् । \newline
64. आ॒य॒तामित्या᳚ - य॒ताम् । \newline
65. षण् मा॒सो मा॒स ष्षट् थ्षण् मा॒सः । \newline
66. मा॒सो ब्र॑ह्मसा॒मम् ब्र॑ह्मसा॒मम् मा॒सो मा॒सो ब्र॑ह्मसा॒मम् । \newline
67. ब्र॒ह्म॒सा॒मम् भ॑वति भवति ब्रह्मसा॒मम् ब्र॑ह्मसा॒मम् भ॑वति । \newline
68. ब्र॒ह्म॒सा॒ममिति॑ ब्रह्म - सा॒मम् । \newline
69. भ॒व॒ त्य॒य म॒यम् भ॑वति भव त्य॒यम् । \newline
70. अ॒यं ॅवै वा अ॒य म॒यं ॅवै । \newline
71. वै लो॒को लो॒को वै वै लो॒कः । \newline
72. लो॒को ज्योति॒र् ज्योति॑र् लो॒को लो॒को ज्योतिः॑ । \newline
73. ज्योतिः॑ प्र॒जा प्र॒जा ज्योति॒र् ज्योतिः॑ प्र॒जा । \newline
74. प्र॒जा ज्योति॒र् ज्योतिः॑ प्र॒जा प्र॒जा ज्योतिः॑ । \newline
75. प्र॒जेति॑ प्र - जा । \newline
76. ज्योति॑ रि॒म मि॒मम् ज्योति॒र् ज्योति॑ रि॒मम् । \newline
77. इ॒म मे॒वैवेम मि॒म मे॒व । \newline
78. ए॒व तत् तदे॒वैव तत् । \newline
79. तल्लो॒कम् ॅलो॒कम् तत् तल्लो॒कम् । \newline
80. लो॒कम् पश्य॑न्तः॒ पश्य॑न्तो लो॒कम् ॅलो॒कम् पश्य॑न्तः । \newline
81. पश्य॑न्तो ऽभि॒वद॑न्तो ऽभि॒वद॑न्तः॒ पश्य॑न्तः॒ पश्य॑न्तो ऽभि॒वद॑न्तः । \newline
82. अ॒भि॒वद॑न्त॒ आ ऽभि॒वद॑न्तो ऽभि॒वद॑न्त॒ आ । \newline
83. अ॒भि॒वद॑न्त॒ इत्य॑भि - वद॑न्तः । \newline
84. आ य॑न्ति य॒न्त्या य॑न्ति । \newline
85. य॒न्तीति॑ यन्ति । \newline

\textbf{Ghana Paata } \newline

1. उप॑ यन्ति य॒न्त्युपोप॑ यन्ति प्राजाप॒त्यम् प्रा॑जाप॒त्यं ॅय॒न्त्युपोप॑ यन्ति प्राजाप॒त्यम् । \newline
2. य॒न्ति॒ प्रा॒जा॒प॒त्यम् प्रा॑जाप॒त्यं ॅय॑न्ति यन्ति प्राजाप॒त्यम् प॒शुम् प॒शुम् प्रा॑जाप॒त्यं ॅय॑न्ति यन्ति प्राजाप॒त्यम् प॒शुम् । \newline
3. प्रा॒जा॒प॒त्यम् प॒शुम् प॒शुम् प्रा॑जाप॒त्यम् प्रा॑जाप॒त्यम् प॒शु मा प॒शुम् प्रा॑जाप॒त्यम् प्रा॑जाप॒त्यम् प॒शु मा । \newline
4. प्रा॒जा॒प॒त्यमिति॑ प्राजा - प॒त्यम् । \newline
5. प॒शु मा प॒शुम् प॒शु मा ल॑भन्ते लभन्त॒ आ प॒शुम् प॒शु मा ल॑भन्ते । \newline
6. आ ल॑भन्ते लभन्त॒ आ ल॑भन्ते य॒ज्ञो य॒ज्ञो ल॑भन्त॒ आ ल॑भन्ते य॒ज्ञ्ः । \newline
7. ल॒भ॒न्ते॒ य॒ज्ञो य॒ज्ञो ल॑भन्ते लभन्ते य॒ज्ञो वै वै य॒ज्ञो ल॑भन्ते लभन्ते य॒ज्ञो वै । \newline
8. य॒ज्ञो वै वै य॒ज्ञो य॒ज्ञो वै प्र॒जाप॑तिः प्र॒जाप॑ति॒र् वै य॒ज्ञो य॒ज्ञो वै प्र॒जाप॑तिः । \newline
9. वै प्र॒जाप॑तिः प्र॒जाप॑ति॒र् वै वै प्र॒जाप॑तिर् य॒ज्ञ्स्य॑ य॒ज्ञ्स्य॑ प्र॒जाप॑ति॒र् वै वै प्र॒जाप॑तिर् य॒ज्ञ्स्य॑ । \newline
10. प्र॒जाप॑तिर् य॒ज्ञ्स्य॑ य॒ज्ञ्स्य॑ प्र॒जाप॑तिः प्र॒जाप॑तिर् य॒ज्ञ्स्या न॑नुसर्गा॒या न॑नुसर्गाय य॒ज्ञ्स्य॑ प्र॒जाप॑तिः प्र॒जाप॑तिर् य॒ज्ञ्स्या न॑नुसर्गाय । \newline
11. प्र॒जाप॑ति॒रिति॑ प्र॒जा - प॒तिः॒ । \newline
12. य॒ज्ञ्स्या न॑नुसर्गा॒या न॑नुसर्गाय य॒ज्ञ्स्य॑ य॒ज्ञ्स्या न॑नुसर्गाया भिव॒र्तो॑ ऽभिव॒र्तो ऽन॑नुसर्गाय य॒ज्ञ्स्य॑ य॒ज्ञ्स्या न॑नुसर्गाया भिव॒र्तः । \newline
13. अन॑नुसर्गाया भिव॒र्तो॑ ऽभिव॒र्तो ऽन॑नुसर्गा॒या न॑नुसर्गाया भिव॒र्त इ॒त इ॒तो॑ ऽभिव॒र्तो ऽन॑नुसर्गा॒या न॑नुसर्गाया भिव॒र्त इ॒तः । \newline
14. अन॑नुसर्गा॒येत्यन॑नु - स॒र्गा॒य॒ । \newline
15. अ॒भि॒व॒र्त इ॒त इ॒तो॑ ऽभिव॒र्तो॑ ऽभिव॒र्त इ॒त ष्षट् थ्षडि॒तो॑ ऽभिव॒र्तो॑ ऽभिव॒र्त इ॒त ष्षट् । \newline
16. अ॒भि॒व॒र्त इत्य॑भि - व॒र्तः । \newline
17. इ॒त ष्षट् थ्षडि॒त इ॒त ष्षण् मा॒सो मा॒स ष्षडि॒त इ॒त ष्षण् मा॒सः । \newline
18. षण् मा॒सो मा॒स ष्षट् थ्षण् मा॒सो ब्र॑ह्मसा॒मम् ब्र॑ह्मसा॒मम् मा॒स ष्षट् थ्षण् मा॒सो ब्र॑ह्मसा॒मम् । \newline
19. मा॒सो ब्र॑ह्मसा॒मम् ब्र॑ह्मसा॒मम् मा॒सो मा॒सो ब्र॑ह्मसा॒मम् भ॑वति भवति ब्रह्मसा॒मम् मा॒सो मा॒सो ब्र॑ह्मसा॒मम् भ॑वति । \newline
20. ब्र॒ह्म॒सा॒मम् भ॑वति भवति ब्रह्मसा॒मम् ब्र॑ह्मसा॒मम् भ॑वति॒ ब्रह्म॒ ब्रह्म॑ भवति ब्रह्मसा॒मम् ब्र॑ह्मसा॒मम् भ॑वति॒ ब्रह्म॑ । \newline
21. ब्र॒ह्म॒सा॒ममिति॑ ब्रह्म - सा॒मम् । \newline
22. भ॒व॒ति॒ ब्रह्म॒ ब्रह्म॑ भवति भवति॒ ब्रह्म॒ वै वै ब्रह्म॑ भवति भवति॒ ब्रह्म॒ वै । \newline
23. ब्रह्म॒ वै वै ब्रह्म॒ ब्रह्म॒ वा अ॑भिव॒र्तो॑ ऽभिव॒र्तो वै ब्रह्म॒ ब्रह्म॒ वा अ॑भिव॒र्तः । \newline
24. वा अ॑भिव॒र्तो॑ ऽभिव॒र्तो वै वा अ॑भिव॒र्तो ब्रह्म॑णा॒ ब्रह्म॑णा ऽभिव॒र्तो वै वा अ॑भिव॒र्तो ब्रह्म॑णा । \newline
25. अ॒भि॒व॒र्तो ब्रह्म॑णा॒ ब्रह्म॑णा ऽभिव॒र्तो॑ ऽभिव॒र्तो ब्रह्म॑णै॒वैव ब्रह्म॑णा ऽभिव॒र्तो॑ ऽभिव॒र्तो 
ब्रह्म॑णै॒व । \newline
26. अ॒भि॒व॒र्त इत्य॑भि - व॒र्तः । \newline
27. ब्रह्म॑णै॒वैव ब्रह्म॑णा॒ ब्रह्म॑णै॒व तत् तदे॒व ब्रह्म॑णा॒ ब्रह्म॑णै॒व तत् । \newline
28. ए॒व तत् तदे॒वैव तथ् सु॑व॒र्गꣳ सु॑व॒र्गम् तदे॒वैव तथ् सु॑व॒र्गम् । \newline
29. तथ् सु॑व॒र्गꣳ सु॑व॒र्गम् तत् तथ् सु॑व॒र्गम् ॅलो॒कम् ॅलो॒कꣳ सु॑व॒र्गम् तत् तथ् सु॑व॒र्गम् ॅलो॒कम् । \newline
30. सु॒व॒र्गम् ॅलो॒कम् ॅलो॒कꣳ सु॑व॒र्गꣳ सु॑व॒र्गम् ॅलो॒क म॑भिव॒र्तय॑न्तो ऽभिव॒र्तय॑न्तो लो॒कꣳ सु॑व॒र्गꣳ सु॑व॒र्गम् ॅलो॒क म॑भिव॒र्तय॑न्तः । \newline
31. सु॒व॒र्गमिति॑ सुवः - गम् । \newline
32. लो॒क म॑भिव॒र्तय॑न्तो ऽभिव॒र्तय॑न्तो लो॒कम् ॅलो॒क म॑भिव॒र्तय॑न्तो यन्ति यन्त्यभिव॒र्तय॑न्तो लो॒कम् ॅलो॒क म॑भिव॒र्तय॑न्तो यन्ति । \newline
33. अ॒भि॒व॒र्तय॑न्तो यन्ति यन्त्यभिव॒र्तय॑न्तो ऽभिव॒र्तय॑न्तो यन्ति प्रतिकू॒लम् प्र॑तिकू॒लं ॅय॑न्त्यभिव॒र्तय॑न्तो ऽभिव॒र्तय॑न्तो यन्ति प्रतिकू॒लम् । \newline
34. अ॒भि॒व॒र्तय॑न्त॒ इत्य॑भि - व॒र्तय॑न्तः । \newline
35. य॒न्ति॒ प्र॒ति॒कू॒लम् प्र॑तिकू॒लं ॅय॑न्ति यन्ति प्रतिकू॒ल मि॑वेव प्रतिकू॒लं ॅय॑न्ति यन्ति प्रतिकू॒ल मि॑व । \newline
36. प्र॒ति॒कू॒ल मि॑वेव प्रतिकू॒लम् प्र॑तिकू॒ल मि॑व॒ हि हीव॑ प्रतिकू॒लम् प्र॑तिकू॒ल मि॑व॒ हि । \newline
37. प्र॒ति॒कू॒लमिति॑ प्रति - कू॒लम् । \newline
38. इ॒व॒ हि हीवे॑व॒ हीत इ॒तो हीवे॑व॒ हीतः । \newline
39. हीत इ॒तो हि हीतः सु॑व॒र्गः सु॑व॒र्ग इ॒तो हि हीतः सु॑व॒र्गः । \newline
40. इ॒तः सु॑व॒र्गः सु॑व॒र्ग इ॒त इ॒तः सु॑व॒र्गो लो॒को लो॒कः सु॑व॒र्ग इ॒त इ॒तः सु॑व॒र्गो लो॒कः । \newline
41. सु॒व॒र्गो लो॒को लो॒कः सु॑व॒र्गः सु॑व॒र्गो लो॒क इन्द्रेन्द्र॑ लो॒कः सु॑व॒र्गः सु॑व॒र्गो लो॒क इन्द्र॑ । \newline
42. सु॒व॒र्ग इति॑ सुवः - गः । \newline
43. लो॒क इन्द्रेन्द्र॑ लो॒को लो॒क इन्द्र॒ क्रतु॒म् क्रतु॒ मिन्द्र॑ लो॒को लो॒क इन्द्र॒ क्रतु᳚म् । \newline
44. इन्द्र॒ क्रतु॒म् क्रतु॒ मिन्द्रेन्द्र॒ क्रतु॑न् नो नः॒ क्रतु॒ मिन्द्रेन्द्र॒ क्रतु॑न् नः । \newline
45. क्रतु॑न् नो नः॒ क्रतु॒म् क्रतु॑न् न॒ आ नः॒ क्रतु॒म् क्रतु॑न् न॒ आ । \newline
46. न॒ आ नो॑ न॒ आ भ॑र भ॒रा नो॑ न॒ आ भ॑र । \newline
47. आ भ॑र भ॒रा भ॑र पि॒ता पि॒ता भ॒रा भ॑र पि॒ता । \newline
48. भ॒र॒ पि॒ता पि॒ता भ॑र भर पि॒ता पु॒त्रेभ्यः॑ पु॒त्रेभ्यः॑ पि॒ता भ॑र भर पि॒ता पु॒त्रेभ्यः॑ । \newline
49. पि॒ता पु॒त्रेभ्यः॑ पु॒त्रेभ्यः॑ पि॒ता पि॒ता पु॒त्रेभ्यो॒ यथा॒ यथा॑ पु॒त्रेभ्यः॑ पि॒ता पि॒ता पु॒त्रेभ्यो॒ यथा᳚ । \newline
50. पु॒त्रेभ्यो॒ यथा॒ यथा॑ पु॒त्रेभ्यः॑ पु॒त्रेभ्यो॒ यथा᳚ । \newline
51. यथेति॒ यथा᳚ । \newline
52. शिक्षा॑ नो नः॒ शिक्ष॒ शिक्षा॑ नो अ॒स्मिन् न॒स्मिन् नः॒ शिक्ष॒ शिक्षा॑ नो अ॒स्मिन्न् । \newline
53. नो॒ अ॒स्मिन् न॒स्मिन् नो॑ नो अ॒स्मिन् पु॑रुहूत पुरुहूता॒स्मिन् नो॑ नो अ॒स्मिन् पु॑रुहूत । \newline
54. अ॒स्मिन् पु॑रुहूत पुरुहूता॒स्मिन् न॒स्मिन् पु॑रुहूत॒ याम॑नि॒ याम॑नि पुरुहूता॒स्मिन् न॒स्मिन् पु॑रुहूत॒ याम॑नि । \newline
55. पु॒रु॒हू॒त॒ याम॑नि॒ याम॑नि पुरुहूत पुरुहूत॒ याम॑नि जी॒वा जी॒वा याम॑नि पुरुहूत पुरुहूत॒ याम॑नि जी॒वाः । \newline
56. पु॒रु॒हू॒तेति॑ पुरु - हू॒त॒ । \newline
57. याम॑नि जी॒वा जी॒वा याम॑नि॒ याम॑नि जी॒वा ज्योति॒र् ज्योति॑र् जी॒वा याम॑नि॒ याम॑नि जी॒वा ज्योतिः॑ । \newline
58. जी॒वा ज्योति॒र् ज्योति॑र् जी॒वा जी॒वा ज्योति॑ रशीम ह्यशीमहि॒ ज्योति॑र् जी॒वा जी॒वा ज्योति॑ रशीमहि । \newline
59. ज्योति॑ रशीम ह्यशीमहि॒ ज्योति॒र् ज्योति॑ रशीम॒ही तीत्य॑शीमहि॒ ज्योति॒र् ज्योति॑ रशीम॒हीति॑ । \newline
60. अ॒शी॒म॒ हीती त्य॑शीम ह्यशीम॒ही त्य॒मुतो॒ ऽमुत॒ इत्य॑ शीम ह्यशीम॒ही त्य॒मुतः॑ । \newline
61. इत्य॒ मुतो॒ ऽमुत॒ इती त्य॒मुत॑ आय॒ता मा॑य॒ता म॒मुत॒ इती त्य॒मुत॑ आय॒ताम् । \newline
62. अ॒मुत॑ आय॒ता मा॑य॒ता म॒मुतो॒ ऽमुत॑ आय॒ताꣳ षट् थ्षडा॑य॒ता म॒मुतो॒ ऽमुत॑ आय॒ताꣳ षट् । \newline
63. आ॒य॒ताꣳ षट् थ्षडा॑य॒ता मा॑य॒ताꣳ षण् मा॒सो मा॒स ष्षडा॑य॒ता मा॑य॒ताꣳ षण् मा॒सः । \newline
64. आ॒य॒तामित्या᳚ - य॒ताम् । \newline
65. षण् मा॒सो मा॒स ष्षट् थ्षण् मा॒सो ब्र॑ह्मसा॒मम् ब्र॑ह्मसा॒मम् मा॒स ष्षट् थ्षण् मा॒सो ब्र॑ह्मसा॒मम् । \newline
66. मा॒सो ब्र॑ह्मसा॒मम् ब्र॑ह्मसा॒मम् मा॒सो मा॒सो ब्र॑ह्मसा॒मम् भ॑वति भवति ब्रह्मसा॒मम् मा॒सो मा॒सो ब्र॑ह्मसा॒मम् भ॑वति । \newline
67. ब्र॒ह्म॒सा॒मम् भ॑वति भवति ब्रह्मसा॒मम् ब्र॑ह्मसा॒मम् भ॑व त्य॒य म॒यम् भ॑वति ब्रह्मसा॒मम् ब्र॑ह्मसा॒मम् भ॑व त्य॒यम् । \newline
68. ब्र॒ह्म॒सा॒ममिति॑ ब्रह्म - सा॒मम् । \newline
69. भ॒व॒ त्य॒य म॒यम् भ॑वति भव त्य॒यं ॅवै वा अ॒यम् भ॑वति भव त्य॒यं ॅवै । \newline
70. अ॒यं ॅवै वा अ॒य म॒यं ॅवै लो॒को लो॒को वा अ॒य म॒यं ॅवै लो॒कः । \newline
71. वै लो॒को लो॒को वै वै लो॒को ज्योति॒र् ज्योति॑र् लो॒को वै वै लो॒को ज्योतिः॑ । \newline
72. लो॒को ज्योति॒र् ज्योति॑र् लो॒को लो॒को ज्योतिः॑ प्र॒जा प्र॒जा ज्योति॑र् लो॒को लो॒को ज्योतिः॑ प्र॒जा । \newline
73. ज्योतिः॑ प्र॒जा प्र॒जा ज्योति॒र् ज्योतिः॑ प्र॒जा ज्योति॒र् ज्योतिः॑ प्र॒जा ज्योति॒र् ज्योतिः॑ प्र॒जा ज्योतिः॑ । \newline
74. प्र॒जा ज्योति॒र् ज्योतिः॑ प्र॒जा प्र॒जा ज्योति॑ रि॒म मि॒मम् ज्योतिः॑ प्र॒जा प्र॒जा ज्योति॑ रि॒मम् । \newline
75. प्र॒जेति॑ प्र - जा । \newline
76. ज्योति॑ रि॒म मि॒मम् ज्योति॒र् ज्योति॑ रि॒म मे॒वैवेमम् ज्योति॒र् ज्योति॑ रि॒म मे॒व । \newline
77. इ॒म मे॒वैवेम मि॒म मे॒व तत् तदे॒वेम मि॒म मे॒व तत् । \newline
78. ए॒व तत् तदे॒ वैव तल्लो॒कम् ॅलो॒कम् तदे॒ वैव तल्लो॒कम् । \newline
79. तल्लो॒कम् ॅलो॒कम् तत् तल्लो॒कम् पश्य॑न्तः॒ पश्य॑न्तो लो॒कम् तत् तल्लो॒कम् पश्य॑न्तः । \newline
80. लो॒कम् पश्य॑न्तः॒ पश्य॑न्तो लो॒कम् ॅलो॒कम् पश्य॑न्तो ऽभि॒वद॑न्तो ऽभि॒वद॑न्तः॒ पश्य॑न्तो लो॒कम् ॅलो॒कम् पश्य॑न्तो ऽभि॒वद॑न्तः । \newline
81. पश्य॑न्तो ऽभि॒वद॑न्तो ऽभि॒वद॑न्तः॒ पश्य॑न्तः॒ पश्य॑न्तो ऽभि॒वद॑न्त॒ आ ऽभि॒वद॑न्तः॒ पश्य॑न्तः॒ पश्य॑न्तो ऽभि॒वद॑न्त॒ आ । \newline
82. अ॒भि॒वद॑न्त॒ आ ऽभि॒वद॑न्तो ऽभि॒वद॑न्त॒ आ य॑न्ति य॒न्त्या ऽभि॒वद॑न्तो ऽभि॒वद॑न्त॒ आ य॑न्ति । \newline
83. अ॒भि॒वद॑न्त॒ इत्य॑भि - वद॑न्तः । \newline
84. आ य॑न्ति य॒न्त्या य॑न्ति । \newline
85. य॒न्तीति॑ यन्ति । \newline
\pagebreak
\markright{ TS 7.5.8.1  \hfill https://www.vedavms.in \hfill}

\section{ TS 7.5.8.1 }

\textbf{TS 7.5.8.1 } \newline
\textbf{Samhita Paata} \newline

दे॒वानां॒ ॅवा अन्तं॑ ज॒ग्मुषा॑मिन्द्रि॒यं ॅवी॒र्य॑-मपा᳚क्राम॒त् तत् क्रो॒शेनावा॑रुन्धत॒ तत् क्रो॒शस्य॑ क्रोश॒त्वं ॅयत् क्रो॒शेन॒ चात्वा॑ल॒स्यान्ते᳚ स्तु॒वन्ति॑ य॒ज्ञ्स्यै॒वान्तं॑ ग॒त्वेन्द्रि॒यं ॅवी॒र्य॑मव॑ रुन्धते स॒त्रस्यर्द्ध्या॑ ऽऽहव॒नीय॒स्यान्ते᳚ स्तुवन्त्य॒ग्नि-मे॒वोप॑द्-र॒ष्टारं॑ कृ॒त्वर्द्धि॒मुप॑ यन्ति प्र॒जाप॑ते॒र्॒.हृद॑येन हवि॒र्द्धाने॒ऽन्तः स्तु॑वन्ति प्रे॒माण॑मे॒वास्य॑ गच्छन्ति श्लो॒केन॑ पु॒रस्ता॒थ् सद॑सः - [  ] \newline

\textbf{Pada Paata} \newline

दे॒वाना᳚म् । वै । अन्त᳚म् । ज॒ग्मुषा᳚म् । इ॒न्द्रि॒यम् । वी॒र्य᳚म् । अपेति॑ । अ॒क्रा॒म॒त् । तत् । क्रो॒शेन॑ । अवेति॑ । अ॒रु॒न्ध॒त॒ । तत् । क्रो॒शस्य॑ । क्रो॒श॒त्वमिति॑ क्रोश - त्वम् । यत् । क्रो॒शेन॑ । चात्वा॑लस्य । अन्ते᳚ । स्तु॒वन्ति॑ । य॒ज्ञ्स्य॑ । ए॒व । अन्त᳚म् । ग॒त्वा । इ॒न्द्रि॒यम् । वी॒र्य᳚म् । अवेति॑ । रु॒न्ध॒ते॒ । स॒त्रस्यद्‌र्ध्या᳚ । आ॒ह॒व॒नीय॒स्येत्या᳚ - ह॒व॒नीय॑स्य । अन्ते᳚ । स्तु॒व॒न्ति॒ । अ॒ग्निम् । ए॒व । उ॒प॒द्र॒ष्टार॒मित्यू॑प - द्र॒ष्टार᳚म् । कृ॒त्वा । ऋद्धि᳚म् । उपेति॑ । य॒न्ति॒ । प॒जाप॑त॒र्॒.हृद॑येन । ह॒वि॒द्‌र्धान॒ इति॑ हविः - धाने᳚ । अ॒न्तः । स्तु॒व॒न्ति॒ । प्रे॒माण᳚म् । ए॒व । अ॒स्य॒ । ग॒च्छ॒न्ति॒ । श्लो॒केन॑ । पु॒रस्ता᳚त् । सद॑सः ।  \newline


\textbf{Krama Paata} \newline

दे॒वाना॒म् ॅवै । वा अन्त᳚म् । अन्त॑म् ज॒ग्मुषा᳚म् । ज॒ग्मुषा॑मिन्द्रि॒यम् । इ॒न्द्रि॒यम् ॅवी॒र्य᳚म् । वी॒र्य॑मप॑ । अपा᳚क्रामत् । अ॒क्रा॒म॒त् तत् । तत् क्रो॒शेन॑ । क्रो॒शेनाव॑ । अवा॑रुन्धत । अ॒रु॒न्ध॒त॒ तत् । तत् क्रो॒शस्य॑ । क्रो॒शस्य॑ क्रोश॒त्वम् । क्रो॒श॒त्वम् ॅयत् । क्रो॒श॒त्वमिति॑ क्रोश - त्वम् । यत् क्रो॒शेन॑ । क्रो॒शेन॒ चात्वा॑लस्य । चात्वा॑ल॒स्यान्ते᳚ । अन्ते᳚ स्तु॒वन्ति॑ । स्तु॒वन्ति॑ य॒ज्ञ्स्य॑ । य॒ज्ञ्स्यै॒व । ए॒वान्त᳚म् । अन्त॑म् ग॒त्वा । ग॒त्वेन्द्रि॒यम् । इ॒न्द्रि॒यम् ॅवी॒र्य᳚म् । वी॒र्य॑मव॑ । अव॑ रुन्धते । रु॒न्ध॒ते॒ स॒त्रस्यर्द्ध्या᳚ । स॒त्रस्यर्द्ध्या॑ऽऽहव॒नीय॑स्य । आ॒ह॒व॒नीय॒स्यान्ते᳚ । आ॒ह॒व॒नीय॒स्येत्या᳚ - ह॒व॒नीय॑स्य । अन्ते᳚ स्तुवन्ति । स्तु॒व॒न्त्य॒ग्निम् । अ॒ग्निमे॒व । ए॒वोप॑द्र॒ष्टार᳚म् । उ॒प॒द्र॒ष्टार॑म् कृ॒त्वा । उ॒प॒द्र॒ष्टार॒मित्यु॑प - द्र॒ष्टार᳚म् । कृ॒त्वर्‌द्धि᳚म् । ऋद्धि॒मुप॑ । उप॑ यन्ति । य॒न्ति॒ प्र॒जाप॑ते॒र्.॒हृद॑येन । प्र॒जाप॑ते॒र्.॒हृद॑येन हवि॒र्द्धाने᳚ । ह॒वि॒र्द्धाने॒ऽन्तः । ह॒वि॒र्द्धान॒ इति॑ हविः - धाने᳚ । अ॒न्तः स्तु॑वन्ति । स्तु॒व॒न्ति॒ प्रे॒माण᳚म् । प्रे॒माण॑मे॒व । ए॒वास्य॑ । अ॒स्य॒ ग॒च्छ॒न्ति॒ । ग॒च्छ॒न्ति॒ श्लो॒केन॑ । श्लो॒केन॑ पु॒रस्ता᳚त् । पु॒रस्ता॒थ् सद॑सः । सद॑सः स्तुवन्ति \newline

\textbf{Jatai Paata} \newline

1. दे॒वानां॒ ॅवै वै दे॒वाना᳚म् दे॒वानां॒ ॅवै । \newline
2. वा अन्त॒ मन्तं॒ ॅवै वा अन्त᳚म् । \newline
3. अन्त॑म् ज॒ग्मुषा᳚म् ज॒ग्मुषा॒ मन्त॒ मन्त॑म् ज॒ग्मुषा᳚म् । \newline
4. ज॒ग्मुषा॑ मिन्द्रि॒य मि॑न्द्रि॒यम् ज॒ग्मुषा᳚म् ज॒ग्मुषा॑ मिन्द्रि॒यम् । \newline
5. इ॒न्द्रि॒यं ॅवी॒र्यं॑ ॅवी॒र्य॑ मिन्द्रि॒य मि॑न्द्रि॒यं ॅवी॒र्य᳚म् । \newline
6. वी॒र्य॑ मपाप॑ वी॒र्यं॑ ॅवी॒र्य॑ मप॑ । \newline
7. अपा᳚क्राम दक्राम॒ दपा पा᳚क्रामत् । \newline
8. अ॒क्रा॒म॒त् तत् तद॑क्राम दक्राम॒त् तत् । \newline
9. तत् क्रो॒शेन॑ क्रो॒शेन॒ तत् तत् क्रो॒शेन॑ । \newline
10. क्रो॒शेना वाव॑ क्रो॒शेन॑ क्रो॒शे नाव॑ । \newline
11. अवा॑ रुन्धता रुन्ध॒ता वावा॑ रुन्धत । \newline
12. अ॒रु॒न्ध॒त॒ तत् तद॑रुन्धता रुन्धत॒ तत् । \newline
13. तत् क्रो॒शस्य॑ क्रो॒शस्य॒ तत् तत् क्रो॒शस्य॑ । \newline
14. क्रो॒शस्य॑ क्रोश॒त्वम् क्रो॑श॒त्वम् क्रो॒शस्य॑ क्रो॒शस्य॑ क्रोश॒त्वम् । \newline
15. क्रो॒श॒त्वं ॅयद् यत् क्रो॑श॒त्वम् क्रो॑श॒त्वं ॅयत् । \newline
16. क्रो॒श॒त्वमिति॑ क्रोश - त्वम् । \newline
17. यत् क्रो॒शेन॑ क्रो॒शेन॒ यद् यत् क्रो॒शेन॑ । \newline
18. क्रो॒शेन॒ चात्वा॑लस्य॒ चात्वा॑लस्य क्रो॒शेन॑ क्रो॒शेन॒ चात्वा॑लस्य । \newline
19. चात्वा॑ल॒ स्यान्ते ऽन्ते॒ चात्वा॑लस्य॒ चात्वा॑ल॒ स्यान्ते᳚ । \newline
20. अन्ते᳚ स्तु॒वन्ति॑ स्तु॒व न्त्यन्ते ऽन्ते᳚ स्तु॒वन्ति॑ । \newline
21. स्तु॒वन्ति॑ य॒ज्ञ्स्य॑ य॒ज्ञ्स्य॑ स्तु॒वन्ति॑ स्तु॒वन्ति॑ य॒ज्ञ्स्य॑ । \newline
22. य॒ज्ञ् स्यै॒वैव य॒ज्ञ्स्य॑ य॒ज्ञ् स्यै॒व । \newline
23. ए॒वान्त॒ मन्त॑ मे॒वै वान्त᳚म् । \newline
24. अन्त॑म् ग॒त्वा ग॒त्वा ऽन्त॒ मन्त॑म् ग॒त्वा । \newline
25. ग॒त्वेन्द्रि॒य मि॑न्द्रि॒यम् ग॒त्वा ग॒त्वेन्द्रि॒यम् । \newline
26. इ॒न्द्रि॒यं ॅवी॒र्यं॑ ॅवी॒र्य॑ मिन्द्रि॒य मि॑न्द्रि॒यं ॅवी॒र्य᳚म् । \newline
27. वी॒र्य॑ मवाव॑ वी॒र्यं॑ ॅवी॒र्य॑ मव॑ । \newline
28. अव॑ रुन्धते रुन्ध॒ते ऽवाव॑ रुन्धते । \newline
29. रु॒न्ध॒ते॒ स॒त्रस्यर्द्ध्या॑ स॒त्रस्यर्द्ध्या॑ रुन्धते रुन्धते स॒त्रस्यर्द्ध्या᳚ । \newline
30. स॒त्रस्यर्द्ध्या॑ ऽऽहव॒नीय॑स्या हव॒नीय॑स्य स॒त्रस्यर्द्ध्या॑ स॒त्रस्यर्द्ध्या॑ ऽऽहव॒नीय॑स्य । \newline
31. आ॒ह॒व॒नीय॒ स्यान्ते ऽन्त॑ आहव॒नीय॑स्या हव॒नीय॒ स्यान्ते᳚ । \newline
32. आ॒ह॒व॒नीय॒स्येत्या᳚ - ह॒व॒नीय॑स्य । \newline
33. अन्ते᳚ स्तुवन्ति स्तुव॒ न्त्यन्ते ऽन्ते᳚ स्तुवन्ति । \newline
34. स्तु॒व॒ न्त्य॒ग्नि म॒ग्निꣳ स्तु॑वन्ति स्तुव न्त्य॒ग्निम् । \newline
35. अ॒ग्नि मे॒वै वाग्नि म॒ग्नि मे॒व । \newline
36. ए॒वोप॑द्र॒ष्टार॑ मुपद्र॒ष्टार॑ मे॒वैवोप॑द्र॒ष्टार᳚म् । \newline
37. उ॒प॒द्र॒ष्टार॑म् कृ॒त्वा कृ॒त्वोप॑द्र॒ष्टार॑ मुपद्र॒ष्टार॑म् कृ॒त्वा । \newline
38. उ॒प॒द्र॒ष्टार॒मित्यु॑प - द्र॒ष्टार᳚म् । \newline
39. कृ॒त्व र्‌द्धि॒ मृद्धि॑म् कृ॒त्वा कृ॒त्व र्‌द्धि᳚म् । \newline
40. ऋद्धि॒ मुपोपा र्‌द्धि॒ मृद्धि॒ मुप॑ । \newline
41. उप॑ यन्ति य॒न्त्युपोप॑ यन्ति । \newline
42. य॒न्ति॒ प्र॒जाप॑ते॒र्॒.हृद॑येन प्र॒जाप॑ते॒र्॒.हृद॑येन यन्ति यन्ति प्र॒जाप॑ते॒र्॒.हृद॑येन । \newline
43. प्र॒जाप॑ते॒र्॒.हृद॑येन हवि॒र्द्धाने॑ हवि॒र्द्धाने᳚ प्र॒जाप॑ते॒र्॒.हृद॑येन प्र॒जाप॑ते॒र्॒.हृद॑येन हवि॒र्द्धाने᳚ । \newline
44. ह॒वि॒र्द्धाने॒ ऽन्त र॒न्तर्. ह॑वि॒र्द्धाने॑ हवि॒र्द्धाने॒ ऽन्तः । \newline
45. ह॒वि॒र्द्धान॒ इति॑ हविः - धाने᳚ । \newline
46. अ॒न्तः स्तु॑वन्ति स्तुव न्त्य॒न्त र॒न्तः स्तु॑वन्ति । \newline
47. स्तु॒व॒न्ति॒ प्रे॒माण॑म् प्रे॒माणꣳ॑ स्तुवन्ति स्तुवन्ति प्रे॒माण᳚म् । \newline
48. प्रे॒माण॑ मे॒वैव प्रे॒माण॑म् प्रे॒माण॑ मे॒व । \newline
49. ए॒वास्या᳚ स्यै॒वै वास्य॑ । \newline
50. अ॒स्य॒ ग॒च्छ॒न्ति॒ ग॒च्छ॒ न्त्य॒स्या॒स्य॒ ग॒च्छ॒न्ति॒ । \newline
51. ग॒च्छ॒न्ति॒ श्लो॒केन॑ श्लो॒केन॑ गच्छन्ति गच्छन्ति श्लो॒केन॑ । \newline
52. श्लो॒केन॑ पु॒रस्ता᳚त् पु॒रस्ता᳚च् छ्लो॒केन॑ श्लो॒केन॑ पु॒रस्ता᳚त् । \newline
53. पु॒रस्ता॒थ् सद॑सः॒ सद॑सः पु॒रस्ता᳚त् पु॒रस्ता॒थ् सद॑सः । \newline
54. सद॑सः स्तुवन्ति स्तुवन्ति॒ सद॑सः॒ सद॑सः स्तुवन्ति । \newline

\textbf{Ghana Paata } \newline

1. दे॒वानां॒ ॅवै वै दे॒वाना᳚म् दे॒वानां॒ ॅवा अन्त॒ मन्तं॒ ॅवै दे॒वाना᳚म् दे॒वानां॒ ॅवा अन्त᳚म् । \newline
2. वा अन्त॒ मन्तं॒ ॅवै वा अन्त॑म् ज॒ग्मुषा᳚म् ज॒ग्मुषा॒ मन्तं॒ ॅवै वा अन्त॑म् ज॒ग्मुषा᳚म् । \newline
3. अन्त॑म् ज॒ग्मुषा᳚म् ज॒ग्मुषा॒ मन्त॒ मन्त॑म् ज॒ग्मुषा॑ मिन्द्रि॒य मि॑न्द्रि॒यम् ज॒ग्मुषा॒ मन्त॒ मन्त॑म् ज॒ग्मुषा॑ मिन्द्रि॒यम् । \newline
4. ज॒ग्मुषा॑ मिन्द्रि॒य मि॑न्द्रि॒यम् ज॒ग्मुषा᳚म् ज॒ग्मुषा॑ मिन्द्रि॒यं ॅवी॒र्यं॑ ॅवी॒र्य॑ मिन्द्रि॒यम् ज॒ग्मुषा᳚म् ज॒ग्मुषा॑ मिन्द्रि॒यं ॅवी॒र्य᳚म् । \newline
5. इ॒न्द्रि॒यं ॅवी॒र्यं॑ ॅवी॒र्य॑ मिन्द्रि॒य मि॑न्द्रि॒यं ॅवी॒र्य॑ मपाप॑ वी॒र्य॑ मिन्द्रि॒य मि॑न्द्रि॒यं ॅवी॒र्य॑ मप॑ । \newline
6. वी॒र्य॑ मपाप॑ वी॒र्यं॑ ॅवी॒र्य॑ मपा᳚क्राम दक्राम॒ दप॑ वी॒र्यं॑ ॅवी॒र्य॑ मपा᳚क्रामत् । \newline
7. अपा᳚क्राम दक्राम॒ दपापा᳚क्राम॒त् तत् तद॑क्राम॒ दपापा᳚क्राम॒त् तत् । \newline
8. अ॒क्रा॒म॒त् तत् तद॑क्राम दक्राम॒त् तत् क्रो॒शेन॑ क्रो॒शेन॒ तद॑क्राम दक्राम॒त् तत् क्रो॒शेन॑ । \newline
9. तत् क्रो॒शेन॑ क्रो॒शेन॒ तत् तत् क्रो॒शेना वाव॑ क्रो॒शेन॒ तत् तत् क्रो॒शे नाव॑ । \newline
10. क्रो॒शेना वाव॑ क्रो॒शेन॑ क्रो॒शे नावा॑ रुन्धता रुन्ध॒ ताव॑ क्रो॒शेन॑ क्रो॒शे नावा॑ रुन्धत । \newline
11. अवा॑रुन्धता रुन्ध॒ता वावा॑ रुन्धत॒ तत् तद॑रुन्ध॒ता वावा॑ रुन्धत॒ तत् । \newline
12. अ॒रु॒न्ध॒त॒ तत् तद॑रुन्धता रुन्धत॒ तत् क्रो॒शस्य॑ क्रो॒शस्य॒ तद॑रुन्धता रुन्धत॒ तत् क्रो॒शस्य॑ । \newline
13. तत् क्रो॒शस्य॑ क्रो॒शस्य॒ तत् तत् क्रो॒शस्य॑ क्रोश॒त्वम् क्रो॑श॒त्वम् क्रो॒शस्य॒ तत् तत् क्रो॒शस्य॑ क्रोश॒त्वम् । \newline
14. क्रो॒शस्य॑ क्रोश॒त्वम् क्रो॑श॒त्वम् क्रो॒शस्य॑ क्रो॒शस्य॑ क्रोश॒त्वं ॅयद् यत् क्रो॑श॒त्वम् क्रो॒शस्य॑ क्रो॒शस्य॑ क्रोश॒त्वं ॅयत् । \newline
15. क्रो॒श॒त्वं ॅयद् यत् क्रो॑श॒त्वम् क्रो॑श॒त्वं ॅयत् क्रो॒शेन॑ क्रो॒शेन॒ यत् क्रो॑श॒त्वम् क्रो॑श॒त्वं ॅयत् क्रो॒शेन॑ । \newline
16. क्रो॒श॒त्वमिति॑ क्रोश - त्वम् । \newline
17. यत् क्रो॒शेन॑ क्रो॒शेन॒ यद् यत् क्रो॒शेन॒ चात्वा॑लस्य॒ चात्वा॑लस्य क्रो॒शेन॒ यद् यत् क्रो॒शेन॒ चात्वा॑लस्य । \newline
18. क्रो॒शेन॒ चात्वा॑लस्य॒ चात्वा॑लस्य क्रो॒शेन॑ क्रो॒शेन॒ चात्वा॑ल॒स्या न्ते ऽन्ते॒ चात्वा॑लस्य क्रो॒शेन॑ क्रो॒शेन॒ चात्वा॑ल॒स्यान्ते᳚ । \newline
19. चात्वा॑ल॒स्यान्ते ऽन्ते॒ चात्वा॑लस्य॒ चात्वा॑ल॒स्यान्ते᳚ स्तु॒वन्ति॑ स्तु॒व न्त्यन्ते॒ चात्वा॑लस्य॒ चात्वा॑ल॒स्यान्ते᳚ स्तु॒वन्ति॑ । \newline
20. अन्ते᳚ स्तु॒वन्ति॑ स्तु॒व न्त्यन्ते ऽन्ते᳚ स्तु॒वन्ति॑ य॒ज्ञ्स्य॑ य॒ज्ञ्स्य॑ स्तु॒व न्त्यन्ते ऽन्ते᳚ स्तु॒वन्ति॑ य॒ज्ञ्स्य॑ । \newline
21. स्तु॒वन्ति॑ य॒ज्ञ्स्य॑ य॒ज्ञ्स्य॑ स्तु॒वन्ति॑ स्तु॒वन्ति॑ य॒ज्ञ्स्यै॒वैव य॒ज्ञ्स्य॑ स्तु॒वन्ति॑ स्तु॒वन्ति॑ य॒ज्ञ् स्यै॒व । \newline
22. य॒ज्ञ्स्यै॒वैव य॒ज्ञ्स्य॑ य॒ज्ञ्स्यै॒वान्त॒ मन्त॑ मे॒व य॒ज्ञ्स्य॑ य॒ज्ञ्स्यै॒वान्त᳚म् । \newline
23. ए॒वान्त॒ मन्त॑ मे॒वै वान्त॑म् ग॒त्वा ग॒त्वा ऽन्त॑ मे॒वै वान्त॑म् ग॒त्वा । \newline
24. अन्त॑म् ग॒त्वा ग॒त्वा ऽन्त॒ मन्त॑म् ग॒त्वेन्द्रि॒य मि॑न्द्रि॒यम् ग॒त्वा ऽन्त॒ मन्त॑म् ग॒त्वेन्द्रि॒यम् । \newline
25. ग॒त्वेन्द्रि॒य मि॑न्द्रि॒यम् ग॒त्वा ग॒त्वेन्द्रि॒यं ॅवी॒र्यं॑ ॅवी॒र्य॑ मिन्द्रि॒यम् ग॒त्वा ग॒त्वेन्द्रि॒यं ॅवी॒र्य᳚म् । \newline
26. इ॒न्द्रि॒यं ॅवी॒र्यं॑ ॅवी॒र्य॑ मिन्द्रि॒य मि॑न्द्रि॒यं ॅवी॒र्य॑ मवाव॑ वी॒र्य॑ मिन्द्रि॒य मि॑न्द्रि॒यं ॅवी॒र्य॑ मव॑ । \newline
27. वी॒र्य॑ मवाव॑ वी॒र्यं॑ ॅवी॒र्य॑ मव॑ रुन्धते रुन्ध॒ते ऽव॑ वी॒र्यं॑ ॅवी॒र्य॑ मव॑ रुन्धते । \newline
28. अव॑ रुन्धते रुन्ध॒ते ऽवाव॑ रुन्धते स॒त्रस्यर्द्ध्या॑ स॒त्रस्यर्द्ध्या॑ रुन्ध॒ते ऽवाव॑ रुन्धते स॒त्रस्यर्द्ध्या᳚ । \newline
29. रु॒न्ध॒ते॒ स॒त्रस्यर्द्ध्या॑ स॒त्रस्यर्द्ध्या॑ रुन्धते रुन्धते स॒त्रस्यर्द्ध्या॑ ऽऽहव॒नीय॑स्या हव॒नीय॑स्य स॒त्रस्यर्द्ध्या॑ रुन्धते रुन्धते स॒त्रस्यर्द्ध्या॑ ऽऽहव॒नीय॑स्य । \newline
30. स॒त्रस्यर्द्ध्या॑ ऽऽहव॒नीय॑स्या हव॒नीय॑स्य स॒त्रस्यर्द्ध्या॑ स॒त्रस्यर्द्ध्या॑ ऽऽहव॒नीय॒स्यान्ते ऽन्त॑ आहव॒नीय॑स्य स॒त्रस्यर्द्ध्या॑ स॒त्रस्यर्द्ध्या॑ ऽऽहव॒नीय॒स्यान्ते᳚ । \newline
31. आ॒ह॒व॒नीय॒स्यान्ते ऽन्त॑ आहव॒नीय॑स्या हव॒नीय॒स्यान्ते᳚ स्तुवन्ति स्तुव॒न्त्यन्त॑ आहव॒नीय॑स्या हव॒नीय॒स्यान्ते᳚ स्तुवन्ति । \newline
32. आ॒ह॒व॒नीय॒स्येत्या᳚ - ह॒व॒नीय॑स्य । \newline
33. अन्ते᳚ स्तुवन्ति स्तुव॒न्त्यन्ते ऽन्ते᳚ स्तुव न्त्य॒ग्नि म॒ग्निꣳ स्तु॑व॒ न्त्यन्ते ऽन्ते᳚ स्तुव न्त्य॒ग्निम् । \newline
34. स्तु॒व॒ न्त्य॒ग्नि म॒ग्निꣳ स्तु॑वन्ति स्तुव न्त्य॒ग्नि मे॒वैवाग्निꣳ स्तु॑वन्ति स्तुव न्त्य॒ग्नि मे॒व । \newline
35. अ॒ग्नि मे॒वै वाग्नि म॒ग्नि मे॒वोप॑द्र॒ष्टार॑ मुपद्र॒ष्टार॑ मे॒वाग्नि म॒ग्नि मे॒वोप॑द्र॒ष्टार᳚म् । \newline
36. ए॒वोप॑द्र॒ष्टार॑ मुपद्र॒ष्टार॑ मे॒वैवोप॑द्र॒ष्टार॑म् कृ॒त्वा कृ॒त्वोप॑द्र॒ष्टार॑ 
मे॒वैवोप॑द्र॒ष्टार॑म् कृ॒त्वा । \newline
37. उ॒प॒द्र॒ष्टार॑म् कृ॒त्वा कृ॒त्वोप॑द्र॒ष्टार॑ मुपद्र॒ष्टार॑म् कृ॒त्व र्‌द्धि॒ मृद्धि॑म् 
कृ॒त्वोप॑द्र॒ष्टार॑ मुपद्र॒ष्टार॑म् कृ॒त्व र्‌द्धि᳚म् । \newline
38. उ॒प॒द्र॒ष्टार॒मित्यु॑प - द्र॒ष्टार᳚म् । \newline
39. कृ॒त्व र्‌द्धि॒ मृद्धि॑म् कृ॒त्वा कृ॒त्व र्‌द्धि॒ मुपोपा र्‌द्धि॑म् कृ॒त्वा कृ॒त्व र्‌द्धि॒ मुप॑ । \newline
40. ऋद्धि॒ मुपोपा र्‌द्धि॒ मृद्धि॒ मुप॑ यन्ति य॒न्त्युपा र्‌द्धि॒ मृद्धि॒ मुप॑ यन्ति । \newline
41. उप॑ यन्ति य॒न्त्युपोप॑ यन्ति प्र॒जाप॑ते॒र्॒.हृद॑येन प्र॒जाप॑ते॒र्॒.हृद॑येन य॒न्त्युपोप॑ यन्ति प्र॒जाप॑ते॒र्॒.हृद॑येन । \newline
42. य॒न्ति॒ प्र॒जाप॑ते॒र्॒.हृद॑येन प्र॒जाप॑ते॒र्॒.हृद॑येन यन्ति यन्ति प्र॒जाप॑ते॒र्॒.हृद॑येन हवि॒र्द्धाने॑ हवि॒र्द्धाने᳚ प्र॒जाप॑ते॒र्॒.हृद॑येन यन्ति यन्ति प्र॒जाप॑ते॒र्॒.हृद॑येन हवि॒र्द्धाने᳚ । \newline
43. प्र॒जाप॑ते॒र्॒.हृद॑येन हवि॒र्द्धाने॑ हवि॒र्द्धाने᳚ प्र॒जाप॑ते॒र्॒.हृद॑येन प्र॒जाप॑ते॒र्॒.हृद॑येन हवि॒र्द्धाने॒ ऽन्त र॒न्तर्. ह॑वि॒र्द्धाने᳚ प्र॒जाप॑ते॒र्॒.हृद॑येन प्र॒जाप॑ते॒र्॒.हृद॑येन हवि॒र्द्धाने॒ ऽन्तः । \newline
44. ह॒वि॒र्द्धाने॒ ऽन्त र॒न्तर्. ह॑वि॒र्द्धाने॑ हवि॒र्द्धाने॒ ऽन्तः स्तु॑वन्ति स्तुव न्त्य॒न्तर्. ह॑वि॒र्द्धाने॑ हवि॒र्द्धाने॒ ऽन्तः स्तु॑वन्ति । \newline
45. ह॒वि॒र्द्धान॒ इति॑ हविः - धाने᳚ । \newline
46. अ॒न्तः स्तु॑वन्ति स्तुव न्त्य॒न्त र॒न्तः स्तु॑वन्ति प्रे॒माण॑म् प्रे॒माणꣳ॑ स्तुव न्त्य॒न्त र॒न्तः स्तु॑वन्ति प्रे॒माण᳚म् । \newline
47. स्तु॒व॒न्ति॒ प्रे॒माण॑म् प्रे॒माणꣳ॑ स्तुवन्ति स्तुवन्ति प्रे॒माण॑ मे॒वैव प्रे॒माणꣳ॑ स्तुवन्ति स्तुवन्ति प्रे॒माण॑ मे॒व । \newline
48. प्रे॒माण॑ मे॒वैव प्रे॒माण॑म् प्रे॒माण॑ मे॒वास्या᳚ स्यै॒व प्रे॒माण॑म् प्रे॒माण॑ मे॒वास्य॑ । \newline
49. ए॒वास्या᳚ स्यै॒वै वास्य॑ गच्छन्ति गच्छ न्त्यस्यै॒वै वास्य॑ गच्छन्ति । \newline
50. अ॒स्य॒ ग॒च्छ॒न्ति॒ ग॒च्छ॒ न्त्य॒स्या॒स्य॒ ग॒च्छ॒न्ति॒ श्लो॒केन॑ श्लो॒केन॑ गच्छ न्त्यस्यास्य गच्छन्ति श्लो॒केन॑ । \newline
51. ग॒च्छ॒न्ति॒ श्लो॒केन॑ श्लो॒केन॑ गच्छन्ति गच्छन्ति श्लो॒केन॑ पु॒रस्ता᳚त् पु॒रस्ता᳚च् छ्लो॒केन॑ गच्छन्ति गच्छन्ति श्लो॒केन॑ पु॒रस्ता᳚त् । \newline
52. श्लो॒केन॑ पु॒रस्ता᳚त् पु॒रस्ता᳚च् छ्लो॒केन॑ श्लो॒केन॑ पु॒रस्ता॒थ् सद॑सः॒ सद॑सः पु॒रस्ता᳚च् छ्लो॒केन॑ श्लो॒केन॑ पु॒रस्ता॒थ् सद॑सः । \newline
53. पु॒रस्ता॒थ् सद॑सः॒ सद॑सः पु॒रस्ता᳚त् पु॒रस्ता॒थ् सद॑सः स्तुवन्ति स्तुवन्ति॒ सद॑सः पु॒रस्ता᳚त् पु॒रस्ता॒थ् सद॑सः स्तुवन्ति । \newline
54. सद॑सः स्तुवन्ति स्तुवन्ति॒ सद॑सः॒ सद॑सः स्तुव॒ न्त्यनु॑श्लोके॒ना नु॑श्लोकेन स्तुवन्ति॒ सद॑सः॒ सद॑सः स्तुव॒ न्त्यनु॑श्लोकेन । \newline
\pagebreak
\markright{ TS 7.5.8.2  \hfill https://www.vedavms.in \hfill}

\section{ TS 7.5.8.2 }

\textbf{TS 7.5.8.2 } \newline
\textbf{Samhita Paata} \newline

स्तुव॒न्त्यनु॑श्लोकेन प॒श्चाद्-य॒ज्ञ्स्यै॒वान्तं॑ ग॒त्वा श्लो॑क॒भाजो॑ भवन्ति न॒वभि॑-रद्ध्व॒र्युरुद्-गा॑यति॒ नव॒ वै पुरु॑षे प्रा॒णाः प्रा॒णाने॒व यज॑मानेषु दधाति॒ सर्वा॑ ऐ॒न्द्रियो॑ भवन्ति प्रा॒णेष्वे॒वेन्द्रि॒यं द॑ध॒-त्यप्र॑तिहृताभि॒रुद्-गा॑यति॒ तस्मा॒त् पुरु॑षः॒ सर्वा᳚ण्य॒न्यानि॑ शी॒र्.ष्णोऽङ्गा॑नि॒ प्रत्य॑चति॒ शिर॑ ए॒व न प॑ञ्चद॒शꣳर॑थन्त॒रं भ॑वतीन्द्रि॒यमे॒वाव॑ रुन्धते सप्तद॒शं-[  ] \newline

\textbf{Pada Paata} \newline

स्तु॒व॒न्ति॒ । अनु॑श्लोके॒नेत्यनु॑ - श्लो॒के॒न॒ । प॒श्चात् । य॒ज्ञ्स्य॑ । ए॒व । अन्त᳚म् । ग॒त्वा । श्लो॒क॒भाज॒ इति॑ श्लोक - भाजः॑ । भ॒व॒न्ति॒ । न॒वभि॒रिति॑ न॒व - भिः॒ । अ॒द्ध्व॒र्युः । उदिति॑ । गा॒य॒ति॒ । नव॑ । वै । पुरु॑षे । प्रा॒णा इति॑ प्र - अ॒नाः । प्रा॒णानिति॑ प्र - अ॒नान् । ए॒व । यज॑मानेषु । द॒धा॒ति॒ । सर्वाः᳚ । ऐ॒न्द्रियः॑ । भ॒व॒न्ति॒ । प्रा॒णेष्विति॑ प्र - अ॒नेषु॑ । ए॒व । इ॒न्द्रि॒यम् । द॒ध॒ति॒ । अप्र॑तिहृताभि॒रित्यप्र॑ति - हृ॒ता॒भिः॒ । उदिति॑ । गा॒य॒ति॒ । तस्मा᳚त् । पुरु॑षः । सर्वा॑णि । अ॒न्यानि॑ । शी॒र्ष्णः । अङ्गा॑नि । प्रतीति॑ । अ॒च॒ति॒ । शिरः॑ । ए॒व । न । प॒ञ्च॒द॒शमिति॑ पञ्च - द॒शम् । र॒थ॒न्त॒रमिति॑ रथं-त॒रम् । भ॒व॒ति॒ । इ॒न्द्रि॒यम् । ए॒व । अवेति॑ । रु॒न्ध॒ते॒ । स॒प्त॒द॒शमिति॑ सप्त - द॒शम् ।  \newline


\textbf{Krama Paata} \newline

स्तु॒व॒न्त्यनु॑श्लोकेन । अनु॑श्लोकेन प॒श्चात् । अनु॑श्लोके॒नेत्यनु॑ - श्लो॒के॒न॒ । प॒श्चाद् य॒ज्ञ्स्य॑ । य॒ज्ञ्स्यै॒व । ए॒वान्त᳚म् । अन्त॑म् ग॒त्वा । ग॒त्वा श्लो॑क॒भाजः॑ । श्लो॒क॒भाजो॑ भवन्ति । श्लो॒क॒भाज॒ इति॑ श्लोक - भाजः॑ । भ॒व॒न्ति॒ न॒वभिः॑ । न॒वभि॑रद्ध्व॒र्युः । न॒वभि॒रिति॑ न॒व - भिः॒ । अ॒द्ध्व॒र्युरुत् । उद् गा॑यति । गा॒य॒ति॒ नव॑ । नव॒ वै । वै पुरु॑षे । पुरु॑षे प्रा॒णाः । प्रा॒णाः प्रा॒णान् । प्रा॒णा इति॑ प्र - अ॒नाः । प्रा॒णाने॒व । प्रा॒णानिति॑ प्र - अ॒नान् । ए॒व यज॑मानेषु । यज॑मानेषु दधाति । द॒धा॒ति॒ सर्वाः᳚ । सर्वा॑ ऐ॒न्द्रियः॑ । ऐ॒न्द्रियो॑ भवन्ति । भ॒व॒न्ति॒ प्रा॒णेषु॑ । प्रा॒णेष्वे॒व । प्रा॒णेष्विति॑ प्र - अ॒नेषु॑ । ए॒वेन्द्रि॒यम् । इ॒न्द्रि॒यम् द॑धति । द॒ध॒त्यप्र॑तिहृताभिः । अप्र॑तिहृताभि॒रुत् । अप्र॑तिहृताभि॒रित्यप्र॑ति - हृ॒ता॒भिः॒ । उद् गा॑यति । गा॒य॒ति॒ तस्मा᳚त् । तस्मा॒त् पुरु॑षः । पुरु॑षः॒ सर्वा॑णि । सर्वा᳚ण्य॒न्यानि॑ । अ॒न्यानि॑ शी॒र्ष्णः । शी॒र्ष्णोऽङ्‍गा॑नि । अङ्‍गा॑नि॒ प्रति॑ । प्रत्य॑चति । अ॒च॒ति॒ शिरः॑ । शिर॑ ए॒व । ए॒व न । न प॑ञ्चद॒शम् । प॒ञ्च॒द॒शꣳ र॑थन्त॒रम् । प॒ञ्च॒द॒शमिति॑ पञ्च - द॒शम् । र॒थ॒न्त॒रम् भ॑वति । र॒थ॒न्त॒रमिति॑ रथम् - त॒रम् । भ॒व॒ती॒न्द्रि॒यम् । इ॒न्द्रि॒यमे॒व । ए॒वाव॑ । अव॑ रुन्धते । रु॒न्ध॒ते॒ स॒प्त॒द॒शम् । स॒प्त॒द॒शम् बृ॒हत् । स॒प्त॒द॒शमिति॑ सप्त - द॒शम् \newline

\textbf{Jatai Paata} \newline

1. स्तु॒व॒ न्त्यनु॑श्लोके॒ना नु॑श्लोकेन स्तुवन्ति स्तुव॒ न्त्यनु॑श्लोकेन । \newline
2. अनु॑श्लोकेन प॒श्चात् प॒श्चा दनु॑श्लोके॒ना नु॑श्लोकेन प॒श्चात् । \newline
3. अनु॑श्लोके॒नेत्यनु॑ - श्लो॒के॒न॒ । \newline
4. प॒श्चाद् य॒ज्ञ्स्य॑ य॒ज्ञ्स्य॑ प॒श्चात् प॒श्चाद् य॒ज्ञ्स्य॑ । \newline
5. य॒ज्ञ्स्यै॒वैव य॒ज्ञ्स्य॑ य॒ज्ञ् स्यै॒व । \newline
6. ए॒वान्त॒ मन्त॑ मे॒वै वान्त᳚म् । \newline
7. अन्त॑म् ग॒त्वा ग॒त्वा ऽन्त॒ मन्त॑म् ग॒त्वा । \newline
8. ग॒त्वा श्लो॑क॒भाजः॑ श्लोक॒भाजो॑ ग॒त्वा ग॒त्वा श्लो॑क॒भाजः॑ । \newline
9. श्लो॒क॒भाजो॑ भवन्ति भवन्ति श्लोक॒भाजः॑ श्लोक॒भाजो॑ भवन्ति । \newline
10. श्लो॒क॒भाज॒ इति॑ श्लोक - भाजः॑ । \newline
11. भ॒व॒न्ति॒ न॒वभि॑र् न॒वभि॑र् भवन्ति भवन्ति न॒वभिः॑ । \newline
12. न॒वभि॑ रद्ध्व॒र्यु र॑द्ध्व॒र्युर् न॒वभि॑र् न॒वभि॑ रद्ध्व॒र्युः । \newline
13. न॒वभि॒रिति॑ न॒व - भिः॒ । \newline
14. अ॒द्ध्व॒र्य् उरुदु द॑द्ध्व॒र्यु र॑द्ध्व॒र्यु रुत् । \newline
15. उद् गा॑यति गाय॒ त्युदुद् गा॑यति । \newline
16. गा॒य॒ति॒ नव॒ नव॑ गायति गायति॒ नव॑ । \newline
17. नव॒ वै वै नव॒ नव॒ वै । \newline
18. वै पुरु॑षे॒ पुरु॑षे॒ वै वै पुरु॑षे । \newline
19. पुरु॑षे प्रा॒णाः प्रा॒णाः पुरु॑षे॒ पुरु॑षे प्रा॒णाः । \newline
20. प्रा॒णाः प्रा॒णान् प्रा॒णान् प्रा॒णाः प्रा॒णाः प्रा॒णान् । \newline
21. प्रा॒णा इति॑ प्र - अ॒नाः । \newline
22. प्रा॒णा ने॒वैव प्रा॒णान् प्रा॒णाने॒व । \newline
23. प्रा॒णानिति॑ प्र - अ॒नान् । \newline
24. ए॒व यज॑मानेषु॒ यज॑माने ष्वे॒वैव यज॑मानेषु । \newline
25. यज॑मानेषु दधाति दधाति॒ यज॑मानेषु॒ यज॑मानेषु दधाति । \newline
26. द॒धा॒ति॒ सर्वाः॒ सर्वा॑ दधाति दधाति॒ सर्वाः᳚ । \newline
27. सर्वा॑ ऐ॒न्द्रिय॑ ऐ॒न्द्रियः॒ सर्वाः॒ सर्वा॑ ऐ॒न्द्रियः॑ । \newline
28. ऐ॒न्द्रियो॑ भवन्ति भव न्त्यै॒न्द्रिय॑ ऐ॒न्द्रियो॑ भवन्ति । \newline
29. भ॒व॒न्ति॒ प्रा॒णेषु॑ प्रा॒णेषु॑ भवन्ति भवन्ति प्रा॒णेषु॑ । \newline
30. प्रा॒णे ष्वे॒वैव प्रा॒णेषु॑ प्रा॒णेष्वे॒व । \newline
31. प्रा॒णेष्विति॑ प्र - अ॒नेषु॑ । \newline
32. ए॒वेन्द्रि॒य मि॑न्द्रि॒य मे॒वैवेन्द्रि॒यम् । \newline
33. इ॒न्द्रि॒यम् द॑धति दधतीन्द्रि॒य मि॑न्द्रि॒यम् द॑धति । \newline
34. द॒ध॒ त्यप्र॑तिहृताभि॒ रप्र॑तिहृताभिर् दधति दध॒ त्यप्र॑तिहृताभिः । \newline
35. अप्र॑तिहृताभि॒ रुदु दप्र॑तिहृताभि॒ रप्र॑तिहृताभि॒ रुत् । \newline
36. अप्र॑तिहृताभि॒रित्यप्र॑ति - हृ॒ता॒भिः॒ । \newline
37. उद् गा॑यति गाय॒ त्युदुद् गा॑यति । \newline
38. गा॒य॒ति॒ तस्मा॒त् तस्मा᳚द् गायति गायति॒ तस्मा᳚त् । \newline
39. तस्मा॒त् पुरु॑षः॒ पुरु॑ष॒ स्तस्मा॒त् तस्मा॒त् पुरु॑षः । \newline
40. पुरु॑षः॒ सर्वा॑णि॒ सर्वा॑णि॒ पुरु॑षः॒ पुरु॑षः॒ सर्वा॑णि । \newline
41. सर्वा᳚ ण्य॒न्या न्य॒न्यानि॒ सर्वा॑णि॒ सर्वा᳚ ण्य॒न्यानि॑ । \newline
42. अ॒न्यानि॑ शी॒र्ष्णः शी॒र्ष्णो᳚ ऽन्यान्य॒ न्यानि॑ शी॒र्ष्णः । \newline
43. शी॒र्ष्णो ऽङ्गा॒ न्यङ्गा॑नि शी॒र्ष्णः शी॒र्ष्णो ऽङ्गा॑नि । \newline
44. अङ्गा॑नि॒ प्रति॒ प्रत्यङ्गा॒ न्यङ्गा॑नि॒ प्रति॑ । \newline
45. प्रत्य॑च त्यचति॒ प्रति॒ प्रत्य॑चति । \newline
46. अ॒च॒ति॒ शिरः॒ शिरो॑ ऽच त्यचति॒ शिरः॑ । \newline
47. शिर॑ ए॒वैव शिरः॒ शिर॑ ए॒व । \newline
48. ए॒व न नैवैव न । \newline
49. न प॑ञ्चद॒शम् प॑ञ्चद॒शन् न न प॑ञ्चद॒शम् । \newline
50. प॒ञ्च॒द॒शꣳ र॑थन्त॒रꣳ र॑थन्त॒रम् प॑ञ्चद॒शम् प॑ञ्चद॒शꣳ र॑थन्त॒रम् । \newline
51. प॒ञ्च॒द॒शमिति॑ पञ्च - द॒शम् । \newline
52. र॒थ॒न्त॒रम् भ॑वति भवति रथन्त॒रꣳ र॑थन्त॒रम् भ॑वति । \newline
53. र॒थ॒न्त॒रमिति॑ रथं - त॒रम् । \newline
54. भ॒व॒ती॒न्द्रि॒य मि॑न्द्रि॒यम् भ॑वति भवतीन्द्रि॒यम् । \newline
55. इ॒न्द्रि॒य मे॒वैवेन्द्रि॒य मि॑न्द्रि॒य मे॒व । \newline
56. ए॒वावा वै॒वै वाव॑ । \newline
57. अव॑ रुन्धते रुन्ध॒ते ऽवाव॑ रुन्धते । \newline
58. रु॒न्ध॒ते॒ स॒प्त॒द॒शꣳ स॑प्तद॒शꣳ रु॑न्धते रुन्धते सप्तद॒शम् । \newline
59. स॒प्त॒द॒शम् बृ॒हद् बृ॒हथ् स॑प्तद॒शꣳ स॑प्तद॒शम् बृ॒हत् । \newline
60. स॒प्त॒द॒शमिति॑ सप्त - द॒शम् । \newline

\textbf{Ghana Paata } \newline

1. स्तु॒व॒ न्त्यनु॑श्लोके॒ना नु॑श्लोकेन स्तुवन्ति स्तुव॒ न्त्यनु॑श्लोकेन प॒श्चात् प॒श्चा दनु॑श्लोकेन स्तुवन्ति स्तुव॒ न्त्यनु॑श्लोकेन प॒श्चात् । \newline
2. अनु॑श्लोकेन प॒श्चात् प॒श्चा दनु॑श्लोके॒ना नु॑श्लोकेन प॒श्चाद् य॒ज्ञ्स्य॑ य॒ज्ञ्स्य॑ प॒श्चा दनु॑श्लोके॒ना नु॑श्लोकेन प॒श्चाद् य॒ज्ञ्स्य॑ । \newline
3. अनु॑श्लोके॒नेत्यनु॑ - श्लो॒के॒न॒ । \newline
4. प॒श्चाद् य॒ज्ञ्स्य॑ य॒ज्ञ्स्य॑ प॒श्चात् प॒श्चाद् य॒ज्ञ्स्यै॒वैव य॒ज्ञ्स्य॑ प॒श्चात् प॒श्चाद् य॒ज्ञ् स्यै॒व । \newline
5. य॒ज्ञ्स्यै॒वैव य॒ज्ञ्स्य॑ य॒ज्ञ् स्यै॒वान्त॒ मन्त॑ मे॒व य॒ज्ञ्स्य॑ य॒ज्ञ् स्यै॒वान्त᳚म् । \newline
6. ए॒वान्त॒ मन्त॑ मे॒वैवान्त॑म् ग॒त्वा ग॒त्वा ऽन्त॑ मे॒वैवान्त॑म् ग॒त्वा । \newline
7. अन्त॑म् ग॒त्वा ग॒त्वा ऽन्त॒ मन्त॑म् ग॒त्वा श्लो॑क॒भाजः॑ श्लोक॒भाजो॑ ग॒त्वा ऽन्त॒ मन्त॑म् ग॒त्वा श्लो॑क॒भाजः॑ । \newline
8. ग॒त्वा श्लो॑क॒भाजः॑ श्लोक॒भाजो॑ ग॒त्वा ग॒त्वा श्लो॑क॒भाजो॑ भवन्ति भवन्ति श्लोक॒भाजो॑ ग॒त्वा ग॒त्वा श्लो॑क॒भाजो॑ भवन्ति । \newline
9. श्लो॒क॒भाजो॑ भवन्ति भवन्ति श्लोक॒भाजः॑ श्लोक॒भाजो॑ भवन्ति न॒वभि॑र् न॒वभि॑र् भवन्ति श्लोक॒भाजः॑ श्लोक॒भाजो॑ भवन्ति न॒वभिः॑ । \newline
10. श्लो॒क॒भाज॒ इति॑ श्लोक - भाजः॑ । \newline
11. भ॒व॒न्ति॒ न॒वभि॑र् न॒वभि॑र् भवन्ति भवन्ति न॒वभि॑ रद्ध्व॒र्यु र॑द्ध्व॒र्युर् न॒वभि॑र् भवन्ति भवन्ति न॒वभि॑ रद्ध्व॒र्युः । \newline
12. न॒वभि॑ रद्ध्व॒र्यु र॑द्ध्व॒र्युर् न॒वभि॑र् न॒वभि॑ रद्ध्व॒र्यु रुदु द॑द्ध्व॒र्युर् न॒वभि॑र् न॒वभि॑ रद्ध्व॒र्यु रुत् । \newline
13. न॒वभि॒रिति॑ न॒व - भिः॒ । \newline
14. अ॒द्ध्व॒र्यु रुदु द॑द्ध्व॒र्यु र॑द्ध्व॒र्यु रुद् गा॑यति गाय॒ त्युद॑द्ध्व॒र्यु र॑द्ध्व॒र्यु रुद् गा॑यति । \newline
15. उद् गा॑यति गाय॒ त्युदुद् गा॑यति॒ नव॒ नव॑ गाय॒ त्युदुद् गा॑यति॒ नव॑ । \newline
16. गा॒य॒ति॒ नव॒ नव॑ गायति गायति॒ नव॒ वै वै नव॑ गायति गायति॒ नव॒ वै । \newline
17. नव॒ वै वै नव॒ नव॒ वै पुरु॑षे॒ पुरु॑षे॒ वै नव॒ नव॒ वै पुरु॑षे । \newline
18. वै पुरु॑षे॒ पुरु॑षे॒ वै वै पुरु॑षे प्रा॒णाः प्रा॒णाः पुरु॑षे॒ वै वै पुरु॑षे प्रा॒णाः । \newline
19. पुरु॑षे प्रा॒णाः प्रा॒णाः पुरु॑षे॒ पुरु॑षे प्रा॒णाः प्रा॒णान् प्रा॒णान् प्रा॒णाः पुरु॑षे॒ पुरु॑षे प्रा॒णाः प्रा॒णान् । \newline
20. प्रा॒णाः प्रा॒णान् प्रा॒णान् प्रा॒णाः प्रा॒णाः प्रा॒णाने॒वैव प्रा॒णान् प्रा॒णाः प्रा॒णाः प्रा॒णा ने॒व । \newline
21. प्रा॒णा इति॑ प्र - अ॒नाः । \newline
22. प्रा॒णाने॒वैव प्रा॒णान् प्रा॒णाने॒व यज॑मानेषु॒ यज॑मा नेष्वे॒व प्रा॒णान् प्रा॒णाने॒व यज॑मानेषु । \newline
23. प्रा॒णानिति॑ प्र - अ॒नान् । \newline
24. ए॒व यज॑मानेषु॒ यज॑मा नेष्वे॒वैव यज॑मानेषु दधाति दधाति॒ यज॑मा नेष्वे॒वैव यज॑मानेषु दधाति । \newline
25. यज॑मानेषु दधाति दधाति॒ यज॑मानेषु॒ यज॑मानेषु दधाति॒ सर्वाः॒ सर्वा॑ दधाति॒ यज॑मानेषु॒ यज॑मानेषु दधाति॒ सर्वाः᳚ । \newline
26. द॒धा॒ति॒ सर्वाः॒ सर्वा॑ दधाति दधाति॒ सर्वा॑ ऐ॒न्द्रिय॑ ऐ॒न्द्रियः॒ सर्वा॑ दधाति दधाति॒ सर्वा॑ ऐ॒न्द्रियः॑ । \newline
27. सर्वा॑ ऐ॒न्द्रिय॑ ऐ॒न्द्रियः॒ सर्वाः॒ सर्वा॑ ऐ॒न्द्रियो॑ भवन्ति भव न्त्यै॒न्द्रियः॒ सर्वाः॒ सर्वा॑ ऐ॒न्द्रियो॑ भवन्ति । \newline
28. ऐ॒न्द्रियो॑ भवन्ति भव न्त्यै॒न्द्रिय॑ ऐ॒न्द्रियो॑ भवन्ति प्रा॒णेषु॑ प्रा॒णेषु॑ भव न्त्यै॒न्द्रिय॑ ऐ॒न्द्रियो॑ भवन्ति प्रा॒णेषु॑ । \newline
29. भ॒व॒न्ति॒ प्रा॒णेषु॑ प्रा॒णेषु॑ भवन्ति भवन्ति प्रा॒णे ष्वे॒वैव प्रा॒णेषु॑ भवन्ति भवन्ति प्रा॒णे ष्वे॒व । \newline
30. प्रा॒णे ष्वे॒वैव प्रा॒णेषु॑ प्रा॒णे ष्वे॒वेन्द्रि॒य मि॑न्द्रि॒य मे॒व प्रा॒णेषु॑ प्रा॒णे ष्वे॒वेन्द्रि॒यम् । \newline
31. प्रा॒णेष्विति॑ प्र - अ॒नेषु॑ । \newline
32. ए॒वेन्द्रि॒य मि॑न्द्रि॒य मे॒वैवेन्द्रि॒यम् द॑धति दधतीन्द्रि॒य मे॒वैवेन्द्रि॒यम् द॑धति । \newline
33. इ॒न्द्रि॒यम् द॑धति दधतीन्द्रि॒य मि॑न्द्रि॒यम् द॑ध॒ त्यप्र॑तिहृताभि॒ रप्र॑तिहृताभिर् दधतीन्द्रि॒य मि॑न्द्रि॒यम् द॑ध॒ त्यप्र॑तिहृताभिः । \newline
34. द॒ध॒ त्यप्र॑तिहृताभि॒ रप्र॑तिहृताभिर् दधति दध॒ त्यप्र॑तिहृताभि॒ रुदु दप्र॑तिहृताभिर् दधति दध॒ त्यप्र॑तिहृताभि॒ रुत् । \newline
35. अप्र॑तिहृताभि॒ रुदु दप्र॑तिहृताभि॒ रप्र॑तिहृताभि॒ रुद् गा॑यति गाय॒ त्युदप्र॑तिहृताभि॒ रप्र॑तिहृताभि॒ रुद् गा॑यति । \newline
36. अप्र॑तिहृताभि॒रित्यप्र॑ति - हृ॒ता॒भिः॒ । \newline
37. उद् गा॑यति गाय॒ त्युदुद् गा॑यति॒ तस्मा॒त् तस्मा᳚द् गाय॒ त्युदुद् गा॑यति॒ तस्मा᳚त् । \newline
38. गा॒य॒ति॒ तस्मा॒त् तस्मा᳚द् गायति गायति॒ तस्मा॒त् पुरु॑षः॒ पुरु॑ष॒ स्तस्मा᳚द् गायति गायति॒ तस्मा॒त् पुरु॑षः । \newline
39. तस्मा॒त् पुरु॑षः॒ पुरु॑ष॒ स्तस्मा॒त् तस्मा॒त् पुरु॑षः॒ सर्वा॑णि॒ सर्वा॑णि॒ पुरु॑ष॒ स्तस्मा॒त् तस्मा॒त् पुरु॑षः॒ सर्वा॑णि । \newline
40. पुरु॑षः॒ सर्वा॑णि॒ सर्वा॑णि॒ पुरु॑षः॒ पुरु॑षः॒ सर्वा᳚ ण्य॒न्या न्य॒न्यानि॒ सर्वा॑णि॒ पुरु॑षः॒ पुरु॑षः॒ सर्वा᳚ ण्य॒न्यानि॑ । \newline
41. सर्वा᳚ ण्य॒न्या न्य॒न्यानि॒ सर्वा॑णि॒ सर्वा᳚ ण्य॒न्यानि॑ शी॒र्ष्णः शी॒र्ष्णो᳚ ऽन्यानि॒ सर्वा॑णि॒ सर्वा᳚ ण्य॒न्यानि॑ शी॒र्ष्णः । \newline
42. अ॒न्यानि॑ शी॒र्ष्णः शी॒र्ष्णो᳚ ऽन्या न्य॒न्यानि॑ शी॒र्ष्णो ऽङ्गा॒ न्यङ्गा॑नि शी॒र्ष्णो᳚ ऽन्या न्य॒न्यानि॑ शी॒र्ष्णो ऽङ्गा॑नि । \newline
43. शी॒र्ष्णो ऽङ्गा॒ न्यङ्गा॑नि शी॒र्ष्णः शी॒र्ष्णो ऽङ्गा॑नि॒ प्रति॒ प्रत्यङ्गा॑नि शी॒र्ष्णः शी॒र्ष्णो ऽङ्गा॑नि॒ प्रति॑ । \newline
44. अङ्गा॑नि॒ प्रति॒ प्रत्यङ्गा॒ न्यङ्गा॑नि॒ प्रत्य॑च त्यचति॒ प्रत्यङ्गा॒ न्यङ्गा॑नि॒ प्रत्य॑चति । \newline
45. प्रत्य॑च त्यचति॒ प्रति॒ प्रत्य॑चति॒ शिरः॒ शिरो॑ ऽचति॒ प्रति॒ प्रत्य॑चति॒ शिरः॑ । \newline
46. अ॒च॒ति॒ शिरः॒ शिरो॑ ऽच त्यचति॒ शिर॑ ए॒वैव शिरो॑ ऽच त्यचति॒ शिर॑ ए॒व । \newline
47. शिर॑ ए॒वैव शिरः॒ शिर॑ ए॒व न नैव शिरः॒ शिर॑ ए॒व न । \newline
48. ए॒व न नैवैव न प॑ञ्चद॒शम् प॑ञ्चद॒शन् नैवैव न प॑ञ्चद॒शम् । \newline
49. न प॑ञ्चद॒शम् प॑ञ्चद॒शन् न न प॑ञ्चद॒शꣳ र॑थन्त॒रꣳ र॑थन्त॒रम् प॑ञ्चद॒शन् न न प॑ञ्चद॒शꣳ र॑थन्त॒रम् । \newline
50. प॒ञ्च॒द॒शꣳ र॑थन्त॒रꣳ र॑थन्त॒रम् प॑ञ्चद॒शम् प॑ञ्चद॒शꣳ र॑थन्त॒रम् भ॑वति भवति रथन्त॒रम् प॑ञ्चद॒शम् प॑ञ्चद॒शꣳ र॑थन्त॒रम् भ॑वति । \newline
51. प॒ञ्च॒द॒शमिति॑ पञ्च - द॒शम् । \newline
52. र॒थ॒न्त॒रम् भ॑वति भवति रथन्त॒रꣳ र॑थन्त॒रम् भ॑वतीन्द्रि॒य मि॑न्द्रि॒यम् भ॑वति रथन्त॒रꣳ र॑थन्त॒रम् भ॑वतीन्द्रि॒यम् । \newline
53. र॒थ॒न्त॒रमिति॑ रथं - त॒रम् । \newline
54. भ॒व॒ती॒न्द्रि॒य मि॑न्द्रि॒यम् भ॑वति भवतीन्द्रि॒य मे॒वैवेन्द्रि॒यम् भ॑वति भवतीन्द्रि॒य मे॒व । \newline
55. इ॒न्द्रि॒य मे॒वैवेन्द्रि॒य मि॑न्द्रि॒य मे॒वा वावै॒वेन्द्रि॒य मि॑न्द्रि॒य मे॒वाव॑ । \newline
56. ए॒वावा वै॒वै वाव॑ रुन्धते रुन्ध॒ते ऽवै॒वै वाव॑ रुन्धते । \newline
57. अव॑ रुन्धते रुन्ध॒ते ऽवाव॑ रुन्धते सप्तद॒शꣳ स॑प्तद॒शꣳ रु॑न्ध॒ते ऽवाव॑ रुन्धते सप्तद॒शम् । \newline
58. रु॒न्ध॒ते॒ स॒प्त॒द॒शꣳ स॑प्तद॒शꣳ रु॑न्धते रुन्धते सप्तद॒शम् बृ॒हद् बृ॒हथ् स॑प्तद॒शꣳ रु॑न्धते रुन्धते सप्तद॒शम् बृ॒हत् । \newline
59. स॒प्त॒द॒शम् बृ॒हद् बृ॒हथ् स॑प्तद॒शꣳ स॑प्तद॒शम् बृ॒ह द॒न्नाद्य॑स्या॒ न्नाद्य॑स्य बृ॒हथ् स॑प्तद॒शꣳ स॑प्तद॒शम् बृ॒ह द॒न्नाद्य॑स्य । \newline
60. स॒प्त॒द॒शमिति॑ सप्त - द॒शम् । \newline
\pagebreak
\markright{ TS 7.5.8.3  \hfill https://www.vedavms.in \hfill}

\section{ TS 7.5.8.3 }

\textbf{TS 7.5.8.3 } \newline
\textbf{Samhita Paata} \newline

बृ॒ह-द॒न्नाद्य॒स्यावरुद्ध्या॒ अथो॒ प्रैव तेन॑ जायन्त एकविꣳ॒॒शं भ॒द्रं द्वि॒पदा॑सु॒ प्रति॑ष्ठित्यै॒ पत्न॑य॒ उप॑ गायन्ति मिथुन॒त्वाय॒ प्रजा᳚त्यै प्र॒जा॑पतिः प्र॒जा अ॑सृजत॒ सो॑ऽकामयता॒ऽऽ*साम॒हꣳ रा॒ज्यं परी॑या॒मिति॒ तासाꣳ॑ राज॒नेनै॒व रा॒ज्यं पर्यै॒त् तद्-रा॑ज॒नस्य॑ राजन॒त्वं ॅयद्-रा॑ज॒नं भव॑ति प्र॒जाना॑मे॒व तद्-यज॑माना रा॒ज्यं परि॑ यन्ति पञ्चविꣳ॒॒शं भ॑वति प्र॒जाप॑ते॒ - [  ] \newline

\textbf{Pada Paata} \newline

बृ॒हत् । अ॒न्नाद्य॒स्येत्य॑न्न - अद्य॑स्य । अव॑रुद्ध्या॒ इत्यव॑ - रु॒द्ध्यै॒ । अथो॒ इति॑ । प्रेति॑ । ए॒व । तेन॑ । जा॒य॒न्ते॒ । ए॒क॒विꣳ॒॒शमित्ये॑क - विꣳ॒॒शम् । भ॒द्रम् । द्वि॒पदा॒स्विति॑ द्वि - पदा॑सु । प्रति॑ष्ठित्या॒ इति॒ प्रति॑ - स्थि॒त्यै॒ । पत्न॑यः । उपेति॑ । गा॒य॒न्ति॒ । मि॒थु॒न॒त्वायेति॑ मिथुन - त्वाय॑ । प्रजा᳚त्या॒ इति॒ प्र - जा॒त्यै॒ । प्र॒जाप॑ति॒रिति॑ प्र॒जा - प॒तिः॒ । प्र॒जा इति॑ प्र-जाः । अ॒सृ॒ज॒त॒ । सः । अ॒का॒म॒य॒त॒ । आ॒साम् । अ॒हम् । रा॒ज्यम् । परीति॑ । इ॒या॒म् । इति॑ । तासा᳚म् । रा॒ज॒नेन॑ । ए॒व । रा॒ज्यम् । परीति॑ । ऐ॒त् । तत् । रा॒ज॒नस्य॑ । रा॒ज॒न॒त्वमिति॑ राजन - त्वम् । यत् । रा॒ज॒नम् । भव॑ति । प्र॒जाना॒मिति॑ प्र - जाना᳚म् । ए॒व । तत् । यज॑मानाः । रा॒ज्यम् । परीति॑ । य॒न्ति॒ । प॒ञ्च॒विꣳ॒॒शमिति॑ पञ्च - विꣳ॒॒शम् । भ॒व॒ति॒ । प्र॒जाप॑ते॒रिति॑ प्र॒जा - प॒तेः॒ ।  \newline


\textbf{Krama Paata} \newline

बृ॒हद॒न्नाद्य॑स्य । अ॒न्नाद्य॒स्याव॑रुद्ध्यै । अ॒न्नाद्य॒स्येत्य॑न्न - अद्य॑स्य । अव॑रुद्ध्या॒ अथो᳚ । अव॑रुद्ध्या॒ इत्यव॑ - रु॒द्ध्यै॒ । अथो॒ प्र । अथो॒ इत्यथो᳚ । प्रैव । ए॒व तेन॑ । तेन॑ जायन्ते । जा॒य॒न्त॒ ए॒क॒विꣳ॒॒शम् । ए॒क॒विꣳ॒॒शम् भ॒द्रम् । ए॒क॒विꣳ॒॒शमित्ये॑क - विꣳ॒॒शम् । भ॒द्रम् द्वि॒पदा॑सु । द्वि॒पदा॑सु॒ प्रति॑ष्ठित्यै । द्वि॒पदा॒स्विति॑ द्वि - पदा॑सु । प्रति॑ष्ठित्यै॒ पत्न॑यः । प्रति॑ष्ठित्या॒ इति॒ प्रति॑ - स्थि॒त्यै॒ । पत्न॑य॒ उप॑ । उप॑ गायन्ति । गा॒य॒न्ति॒ मि॒थु॒न॒त्वाय॑ । मि॒थु॒न॒त्वाय॒ प्रजा᳚त्यै । मि॒थु॒न॒त्वायेति॑ मिथुन - त्वाय॑ । प्रजा᳚त्यै प्र॒जाप॑तिः । प्रजा᳚त्या॒ इति॒ प्र - जा॒त्यै॒ । प्र॒जाप॑तिः प्र॒जाः । प्र॒जाप॑ति॒रिति॑ प्र॒जाः - प॒तिः॒ । प्र॒जा अ॑सृजत । प्र॒जा इति॑ प्र - जाः । अ॒सृ॒ज॒त॒ सः । सो॑ऽकामयत । अ॒का॒म॒य॒ता॒साम् । आ॒साम॒हम् । अ॒हꣳ रा॒ज्यम् । रा॒ज्यम् परि॑ । परी॑याम् । इ॒या॒मिति॑ । इति॒ तासा᳚म् । तासाꣳ॑ राज॒नेन॑ । रा॒ज॒नेनै॒व । ए॒व रा॒ज्यम् । रा॒ज्यम् परि॑ । पर्यै᳚त् । ऐ॒त् तत् । तद् रा॑ज॒नस्य॑ । रा॒ज॒नस्य॑ राजन॒त्वम् । रा॒ज॒न॒त्वम् ॅयत् । रा॒ज॒न॒त्वमिति॑ राजन - त्वम् । यद् रा॑ज॒नम् । रा॒ज॒नम् भव॑ति । भव॑ति प्र॒जाना᳚म् । प्र॒जाना॑मे॒व । प्र॒जाना॒मिति॑ प्र - जाना᳚म् । ए॒व तत् । तद् यज॑मानाः । यज॑माना रा॒ज्यम् । रा॒ज्यम् परि॑ । परि॑ यन्ति । य॒न्ति॒ प॒ञ्च॒विꣳ॒॒शम् । प॒ञ्च॒विꣳ॒॒शम् भ॑वति । प॒ञ्च॒विꣳ॒॒शमिति॑ पञ्च - विꣳ॒॒शम् । भ॒व॒ति॒ प्र॒जाप॑तेः । प्र॒जाप॑ते॒राप्त्यै᳚ । प्र॒जाप॑ते॒रिति॑ प्र॒जा - प॒तेः॒ \newline

\textbf{Jatai Paata} \newline

1. बृ॒ह द॒न्नाद्य॑स्या॒ न्नाद्य॑स्य बृ॒हद् बृ॒ह द॒न्नाद्य॑स्य । \newline
2. अ॒न्नाद्य॒स्या व॑रुद्ध्या॒ अव॑रुद्ध्या अ॒न्नाद्य॑स्या॒ न्नाद्य॒स्या व॑रुद्ध्यै । \newline
3. अ॒न्नाद्य॒स्येत्य॑न्न - अद्य॑स्य । \newline
4. अव॑रुद्ध्या॒ अथो॒ अथो॒ अव॑रुद्ध्या॒ अव॑रुद्ध्या॒ अथो᳚ । \newline
5. अव॑रुद्ध्या॒ इत्यव॑ - रु॒द्ध्यै॒ । \newline
6. अथो॒ प्र प्राथो॒ अथो॒ प्र । \newline
7. अथो॒ इत्यथो᳚ । \newline
8. प्रैवैव प्र प्रैव । \newline
9. ए॒व तेन॒ तेनै॒वैव तेन॑ । \newline
10. तेन॑ जायन्ते जायन्ते॒ तेन॒ तेन॑ जायन्ते । \newline
11. जा॒य॒न्त॒ ए॒क॒विꣳ॒॒श मे॑कविꣳ॒॒शम् जा॑यन्ते जायन्त एकविꣳ॒॒शम् । \newline
12. ए॒क॒विꣳ॒॒शम् भ॒द्रम् भ॒द्र मे॑कविꣳ॒॒श मे॑कविꣳ॒॒शम् भ॒द्रम् । \newline
13. ए॒क॒विꣳ॒॒शमित्ये॑क - विꣳ॒॒शम् । \newline
14. भ॒द्रम् द्वि॒पदा॑सु द्वि॒पदा॑सु भ॒द्रम् भ॒द्रम् द्वि॒पदा॑सु । \newline
15. द्वि॒पदा॑सु॒ प्रति॑ष्ठित्यै॒ प्रति॑ष्ठित्यै द्वि॒पदा॑सु द्वि॒पदा॑सु॒ प्रति॑ष्ठित्यै । \newline
16. द्वि॒पदा॒स्विति॑ द्वि - पदा॑सु । \newline
17. प्रति॑ष्ठित्यै॒ पत्न॑यः॒ पत्न॑यः॒ प्रति॑ष्ठित्यै॒ प्रति॑ष्ठित्यै॒ पत्न॑यः । \newline
18. प्रति॑ष्ठित्या॒ इति॒ प्रति॑ - स्थि॒त्यै॒ । \newline
19. पत्न॑य॒ उपोप॒ पत्न॑यः॒ पत्न॑य॒ उप॑ । \newline
20. उप॑ गायन्ति गाय॒ न्त्युपोप॑ गायन्ति । \newline
21. गा॒य॒न्ति॒ मि॒थु॒न॒त्वाय॑ मिथुन॒त्वाय॑ गायन्ति गायन्ति मिथुन॒त्वाय॑ । \newline
22. मि॒थु॒न॒त्वाय॒ प्रजा᳚त्यै॒ प्रजा᳚त्यै मिथुन॒त्वाय॑ मिथुन॒त्वाय॒ प्रजा᳚त्यै । \newline
23. मि॒थु॒न॒त्वायेति॑ मिथुन - त्वाय॑ । \newline
24. प्रजा᳚त्यै प्र॒जाप॑तिः प्र॒जाप॑तिः॒ प्रजा᳚त्यै॒ प्रजा᳚त्यै प्र॒जाप॑तिः । \newline
25. प्रजा᳚त्या॒ इति॒ प्र - जा॒त्यै॒ । \newline
26. प्र॒जाप॑तिः प्र॒जाः प्र॒जाः प्र॒जाप॑तिः प्र॒जाप॑तिः प्र॒जाः । \newline
27. प्र॒जाप॑ति॒रिति॑ प्र॒जा - प॒तिः॒ । \newline
28. प्र॒जा अ॑सृजता सृजत प्र॒जाः प्र॒जा अ॑सृजत । \newline
29. प्र॒जा इति॑ प्र - जाः । \newline
30. अ॒सृ॒ज॒त॒ स सो॑ ऽसृजता सृजत॒ सः । \newline
31. सो॑ ऽकामयता कामयत॒ स सो॑ ऽकामयत । \newline
32. अ॒का॒म॒य॒ ता॒सा मा॒सा म॑कामयता कामय ता॒साम् । \newline
33. आ॒सा म॒ह म॒ह मा॒सा मा॒सा म॒हम् । \newline
34. अ॒हꣳ रा॒ज्यꣳ रा॒ज्य म॒ह म॒हꣳ रा॒ज्यम् । \newline
35. रा॒ज्यम् परि॒ परि॑ रा॒ज्यꣳ रा॒ज्यम् परि॑ । \newline
36. परी॑या मिया॒म् परि॒ परी॑याम् । \newline
37. इ॒या॒ मितीती॑या मिया॒ मिति॑ । \newline
38. इति॒ तासा॒म् तासा॒ मितीति॒ तासा᳚म् । \newline
39. तासाꣳ॑ राज॒नेन॑ राज॒नेन॒ तासा॒म् तासाꣳ॑ राज॒नेन॑ । \newline
40. रा॒ज॒ने नै॒वैव रा॑ज॒नेन॑ राज॒ने नै॒व । \newline
41. ए॒व रा॒ज्यꣳ रा॒ज्य मे॒वैव रा॒ज्यम् । \newline
42. रा॒ज्यम् परि॒ परि॑ रा॒ज्यꣳ रा॒ज्यम् परि॑ । \newline
43. पर्यै॑दै॒त् परि॒ पर्यै᳚त् । \newline
44. ऐ॒त् तत् तदै॑ दै॒त् तत् । \newline
45. तद् रा॑ज॒नस्य॑ राज॒नस्य॒ तत् तद् रा॑ज॒नस्य॑ । \newline
46. रा॒ज॒नस्य॑ राजन॒त्वꣳ रा॑जन॒त्वꣳ रा॑ज॒नस्य॑ राज॒नस्य॑ राजन॒त्वम् । \newline
47. रा॒ज॒न॒त्वं ॅयद् यद् रा॑जन॒त्वꣳ रा॑जन॒त्वं ॅयत् । \newline
48. रा॒ज॒न॒त्वमिति॑ राजन - त्वम् । \newline
49. यद् रा॑ज॒नꣳ रा॑ज॒नं ॅयद् यद् रा॑ज॒नम् । \newline
50. रा॒ज॒नम् भव॑ति॒ भव॑ति राज॒नꣳ रा॑ज॒नम् भव॑ति । \newline
51. भव॑ति प्र॒जाना᳚म् प्र॒जाना॒म् भव॑ति॒ भव॑ति प्र॒जाना᳚म् । \newline
52. प्र॒जाना॑ मे॒वैव प्र॒जाना᳚म् प्र॒जाना॑ मे॒व । \newline
53. प्र॒जाना॒मिति॑ प्र - जाना᳚म् । \newline
54. ए॒व तत् तदे॒वैव तत् । \newline
55. तद् यज॑माना॒ यज॑माना॒ स्तत् तद् यज॑मानाः । \newline
56. यज॑माना रा॒ज्यꣳ रा॒ज्यं ॅयज॑माना॒ यज॑माना रा॒ज्यम् । \newline
57. रा॒ज्यम् परि॒ परि॑ रा॒ज्यꣳ रा॒ज्यम् परि॑ । \newline
58. परि॑ यन्ति यन्ति॒ परि॒ परि॑ यन्ति । \newline
59. य॒न्ति॒ प॒ञ्च॒विꣳ॒॒शम् प॑ञ्चविꣳ॒॒शं ॅय॑न्ति यन्ति पञ्चविꣳ॒॒शम् । \newline
60. प॒ञ्च॒विꣳ॒॒शम् भ॑वति भवति पञ्चविꣳ॒॒शम् प॑ञ्चविꣳ॒॒शम् भ॑वति । \newline
61. प॒ञ्च॒विꣳ॒॒शमिति॑ पञ्च - विꣳ॒॒शम् । \newline
62. भ॒व॒ति॒ प्र॒जाप॑तेः प्र॒जाप॑तेर् भवति भवति प्र॒जाप॑तेः । \newline
63. प्र॒जाप॑ते॒ राप्त्या॒ आप्त्यै᳚ प्र॒जाप॑तेः प्र॒जाप॑ते॒ राप्त्यै᳚ । \newline
64. प्र॒जाप॑ते॒रिति॑ प्र॒जा - प॒तेः॒ । \newline

\textbf{Ghana Paata } \newline

1. बृ॒ह द॒न्नाद्य॑स्या॒ न्नाद्य॑स्य बृ॒हद् बृ॒ह द॒न्नाद्य॒स्या व॑रुद्ध्या॒ अव॑रुद्ध्या अ॒न्नाद्य॑स्य बृ॒हद् बृ॒ह द॒न्नाद्य॒स्या व॑रुद्ध्यै । \newline
2. अ॒न्नाद्य॒स्या व॑रुद्ध्या॒ अव॑रुद्ध्या अ॒न्नाद्य॑स्या॒ न्नाद्य॒स्या व॑रुद्ध्या॒ अथो॒ अथो॒ अव॑रुद्ध्या 
अ॒न्नाद्य॑स्या॒ न्नाद्य॒स्या व॑रुद्ध्या॒ अथो᳚ । \newline
3. अ॒न्नाद्य॒स्येत्य॑न्न - अद्य॑स्य । \newline
4. अव॑रुद्ध्या॒ अथो॒ अथो॒ अव॑रुद्ध्या॒ अव॑रुद्ध्या॒ अथो॒ प्र प्राथो॒ अव॑रुद्ध्या॒ अव॑रुद्ध्या॒ अथो॒ प्र । \newline
5. अव॑रुद्ध्या॒ इत्यव॑ - रु॒द्ध्यै॒ । \newline
6. अथो॒ प्र प्राथो॒ अथो॒ प्रैवैव प्राथो॒ अथो॒ प्रैव । \newline
7. अथो॒ इत्यथो᳚ । \newline
8. प्रै वैव प्र प्रैव तेन॒ तेनै॒व प्र प्रैव तेन॑ । \newline
9. ए॒व तेन॒ तेनै॒ वैव तेन॑ जायन्ते जायन्ते॒ तेनै॒ वैव तेन॑ जायन्ते । \newline
10. तेन॑ जायन्ते जायन्ते॒ तेन॒ तेन॑ जायन्त एकविꣳ॒॒श मे॑कविꣳ॒॒शम् जा॑यन्ते॒ तेन॒ तेन॑ जायन्त एकविꣳ॒॒शम् । \newline
11. जा॒य॒न्त॒ ए॒क॒विꣳ॒॒श मे॑कविꣳ॒॒शम् जा॑यन्ते जायन्त एकविꣳ॒॒शम् भ॒द्रम् भ॒द्र मे॑कविꣳ॒॒शम् जा॑यन्ते जायन्त एकविꣳ॒॒शम् भ॒द्रम् । \newline
12. ए॒क॒विꣳ॒॒शम् भ॒द्रम् भ॒द्र मे॑कविꣳ॒॒श मे॑कविꣳ॒॒शम् भ॒द्रम् द्वि॒पदा॑सु द्वि॒पदा॑सु भ॒द्र मे॑कविꣳ॒॒श मे॑कविꣳ॒॒शम् भ॒द्रम् द्वि॒पदा॑सु । \newline
13. ए॒क॒विꣳ॒॒शमित्ये॑क - विꣳ॒॒शम् । \newline
14. भ॒द्रम् द्वि॒पदा॑सु द्वि॒पदा॑सु भ॒द्रम् भ॒द्रम् द्वि॒पदा॑सु॒ प्रति॑ष्ठित्यै॒ प्रति॑ष्ठित्यै द्वि॒पदा॑सु भ॒द्रम् भ॒द्रम् द्वि॒पदा॑सु॒ प्रति॑ष्ठित्यै । \newline
15. द्वि॒पदा॑सु॒ प्रति॑ष्ठित्यै॒ प्रति॑ष्ठित्यै द्वि॒पदा॑सु द्वि॒पदा॑सु॒ प्रति॑ष्ठित्यै॒ पत्न॑यः॒ पत्न॑यः॒ प्रति॑ष्ठित्यै द्वि॒पदा॑सु द्वि॒पदा॑सु॒ प्रति॑ष्ठित्यै॒ पत्न॑यः । \newline
16. द्वि॒पदा॒स्विति॑ द्वि - पदा॑सु । \newline
17. प्रति॑ष्ठित्यै॒ पत्न॑यः॒ पत्न॑यः॒ प्रति॑ष्ठित्यै॒ प्रति॑ष्ठित्यै॒ पत्न॑य॒ उपोप॒ पत्न॑यः॒ प्रति॑ष्ठित्यै॒ प्रति॑ष्ठित्यै॒ पत्न॑य॒ उप॑ । \newline
18. प्रति॑ष्ठित्या॒ इति॒ प्रति॑ - स्थि॒त्यै॒ । \newline
19. पत्न॑य॒ उपोप॒ पत्न॑यः॒ पत्न॑य॒ उप॑ गायन्ति गाय॒ न्त्युप॒ पत्न॑यः॒ पत्न॑य॒ उप॑ गायन्ति । \newline
20. उप॑ गायन्ति गाय॒ न्त्युपोप॑ गायन्ति मिथुन॒त्वाय॑ मिथुन॒त्वाय॑ गाय॒ न्त्युपोप॑ गायन्ति मिथुन॒त्वाय॑ । \newline
21. गा॒य॒न्ति॒ मि॒थु॒न॒त्वाय॑ मिथुन॒त्वाय॑ गायन्ति गायन्ति मिथुन॒त्वाय॒ प्रजा᳚त्यै॒ प्रजा᳚त्यै मिथुन॒त्वाय॑ गायन्ति गायन्ति मिथुन॒त्वाय॒ प्रजा᳚त्यै । \newline
22. मि॒थु॒न॒त्वाय॒ प्रजा᳚त्यै॒ प्रजा᳚त्यै मिथुन॒त्वाय॑ मिथुन॒त्वाय॒ प्रजा᳚त्यै प्र॒जाप॑तिः प्र॒जाप॑तिः॒ प्रजा᳚त्यै मिथुन॒त्वाय॑ मिथुन॒त्वाय॒ प्रजा᳚त्यै प्र॒जाप॑तिः । \newline
23. मि॒थु॒न॒त्वायेति॑ मिथुन - त्वाय॑ । \newline
24. प्रजा᳚त्यै प्र॒जाप॑तिः प्र॒जाप॑तिः॒ प्रजा᳚त्यै॒ प्रजा᳚त्यै प्र॒जाप॑तिः प्र॒जाः प्र॒जाः प्र॒जाप॑तिः॒ प्रजा᳚त्यै॒ प्रजा᳚त्यै प्र॒जाप॑तिः प्र॒जाः । \newline
25. प्रजा᳚त्या॒ इति॒ प्र - जा॒त्यै॒ । \newline
26. प्र॒जाप॑तिः प्र॒जाः प्र॒जाः प्र॒जाप॑तिः प्र॒जाप॑तिः प्र॒जा अ॑सृजता सृजत प्र॒जाः प्र॒जाप॑तिः प्र॒जाप॑तिः प्र॒जा अ॑सृजत । \newline
27. प्र॒जाप॑ति॒रिति॑ प्र॒जा - प॒तिः॒ । \newline
28. प्र॒जा अ॑सृजता सृजत प्र॒जाः प्र॒जा अ॑सृजत॒ स सो॑ ऽसृजत प्र॒जाः प्र॒जा अ॑सृजत॒ सः । \newline
29. प्र॒जा इति॑ प्र - जाः । \newline
30. अ॒सृ॒ज॒त॒ स सो॑ ऽसृजता सृजत॒ सो॑ ऽकामयता कामयत॒ सो॑ ऽसृजता सृजत॒ सो॑ ऽकामयत । \newline
31. सो॑ ऽकामयता कामयत॒ स सो॑ ऽकामयता॒ सा मा॒सा म॑कामयत॒ स सो॑ ऽकामयता॒ साम् । \newline
32. अ॒का॒म॒य॒ता॒ सा मा॒सा म॑कामयता कामयता॒सा म॒ह म॒ह मा॒सा म॑कामयता कामयता॒सा म॒हम् । \newline
33. आ॒सा म॒ह म॒ह मा॒सा मा॒सा म॒हꣳ रा॒ज्यꣳ रा॒ज्य म॒ह मा॒सा मा॒सा म॒हꣳ रा॒ज्यम् । \newline
34. अ॒हꣳ रा॒ज्यꣳ रा॒ज्य म॒ह म॒हꣳ रा॒ज्यम् परि॒ परि॑ रा॒ज्य म॒ह म॒हꣳ रा॒ज्यम् परि॑ । \newline
35. रा॒ज्यम् परि॒ परि॑ रा॒ज्यꣳ रा॒ज्यम् परी॑या मिया॒म् परि॑ रा॒ज्यꣳ रा॒ज्यम् परी॑याम् । \newline
36. परी॑या मिया॒म् परि॒ परी॑या॒ मिती ती॑या॒म् परि॒ परी॑या॒ मिति॑ । \newline
37. इ॒या॒ मिती ती॑या मिया॒ मिति॒ तासा॒म् तासा॒ मिती॑या मिया॒ मिति॒ तासा᳚म् । \newline
38. इति॒ तासा॒म् तासा॒ मितीति॒ तासाꣳ॑ राज॒नेन॑ राज॒नेन॒ तासा॒ मितीति॒ तासाꣳ॑ राज॒नेन॑ । \newline
39. तासाꣳ॑ राज॒नेन॑ राज॒नेन॒ तासा॒म् तासाꣳ॑ राज॒ने नै॒वैव रा॑ज॒नेन॒ तासा॒म् तासाꣳ॑ राज॒ने नै॒व । \newline
40. रा॒ज॒ने नै॒वैव रा॑ज॒नेन॑ राज॒ने नै॒व रा॒ज्यꣳ रा॒ज्य मे॒व रा॑ज॒नेन॑ राज॒ने नै॒व रा॒ज्यम् । \newline
41. ए॒व रा॒ज्यꣳ रा॒ज्य मे॒वैव रा॒ज्यम् परि॒ परि॑ रा॒ज्य मे॒वैव रा॒ज्यम् परि॑ । \newline
42. रा॒ज्यम् परि॒ परि॑ रा॒ज्यꣳ रा॒ज्यम् पर्यै॑दै॒त् परि॑ रा॒ज्यꣳ रा॒ज्यम् पर्यै᳚त् । \newline
43. पर्यै॑दै॒त् परि॒ पर्यै॒त् तत् तदै॒त् परि॒ पर्यै॒त् तत् । \newline
44. ऐ॒त् तत् तदै॑ दै॒त् तद् रा॑ज॒नस्य॑ राज॒नस्य॒ तदै॑ दै॒त् तद् रा॑ज॒नस्य॑ । \newline
45. तद् रा॑ज॒नस्य॑ राज॒नस्य॒ तत् तद् रा॑ज॒नस्य॑ राजन॒त्वꣳ रा॑जन॒त्वꣳ रा॑ज॒नस्य॒ तत् तद् रा॑ज॒नस्य॑ राजन॒त्वम् । \newline
46. रा॒ज॒नस्य॑ राजन॒त्वꣳ रा॑जन॒त्वꣳ रा॑ज॒नस्य॑ राज॒नस्य॑ राजन॒त्वं ॅयद् यद् रा॑जन॒त्वꣳ रा॑ज॒नस्य॑ राज॒नस्य॑ राजन॒त्वं ॅयत् । \newline
47. रा॒ज॒न॒त्वं ॅयद् यद् रा॑जन॒त्वꣳ रा॑जन॒त्वं ॅयद् रा॑ज॒नꣳ रा॑ज॒नं ॅयद् रा॑जन॒त्वꣳ रा॑जन॒त्वं ॅयद् रा॑ज॒नम् । \newline
48. रा॒ज॒न॒त्वमिति॑ राजन - त्वम् । \newline
49. यद् रा॑ज॒नꣳ रा॑ज॒नं ॅयद् यद् रा॑ज॒नम् भव॑ति॒ भव॑ति राज॒नं ॅयद् यद् रा॑ज॒नम् भव॑ति । \newline
50. रा॒ज॒नम् भव॑ति॒ भव॑ति राज॒नꣳ रा॑ज॒नम् भव॑ति प्र॒जाना᳚म् प्र॒जाना॒म् भव॑ति राज॒नꣳ रा॑ज॒नम् भव॑ति प्र॒जाना᳚म् । \newline
51. भव॑ति प्र॒जाना᳚म् प्र॒जाना॒म् भव॑ति॒ भव॑ति प्र॒जाना॑ मे॒वैव प्र॒जाना॒म् भव॑ति॒ भव॑ति प्र॒जाना॑ मे॒व । \newline
52. प्र॒जाना॑ मे॒वैव प्र॒जाना᳚म् प्र॒जाना॑ मे॒व तत् तदे॒व प्र॒जाना᳚म् प्र॒जाना॑ मे॒व तत् । \newline
53. प्र॒जाना॒मिति॑ प्र - जाना᳚म् । \newline
54. ए॒व तत् तदे॒वैव तद् यज॑माना॒ यज॑माना॒ स्तदे॒वैव तद् यज॑मानाः । \newline
55. तद् यज॑माना॒ यज॑माना॒ स्तत् तद् यज॑माना रा॒ज्यꣳ रा॒ज्यं ॅयज॑माना॒ स्तत् तद् यज॑माना रा॒ज्यम् । \newline
56. यज॑माना रा॒ज्यꣳ रा॒ज्यं ॅयज॑माना॒ यज॑माना रा॒ज्यम् परि॒ परि॑ रा॒ज्यं ॅयज॑माना॒ यज॑माना रा॒ज्यम् परि॑ । \newline
57. रा॒ज्यम् परि॒ परि॑ रा॒ज्यꣳ रा॒ज्यम् परि॑ यन्ति यन्ति॒ परि॑ रा॒ज्यꣳ रा॒ज्यम् परि॑ यन्ति । \newline
58. परि॑ यन्ति यन्ति॒ परि॒ परि॑ यन्ति पञ्चविꣳ॒॒शम् प॑ञ्चविꣳ॒॒शं ॅय॑न्ति॒ परि॒ परि॑ यन्ति पञ्चविꣳ॒॒शम् । \newline
59. य॒न्ति॒ प॒ञ्च॒विꣳ॒॒शम् प॑ञ्चविꣳ॒॒शं ॅय॑न्ति यन्ति पञ्चविꣳ॒॒शम् भ॑वति भवति पञ्चविꣳ॒॒शं ॅय॑न्ति यन्ति पञ्चविꣳ॒॒शम् भ॑वति । \newline
60. प॒ञ्च॒विꣳ॒॒शम् भ॑वति भवति पञ्चविꣳ॒॒शम् प॑ञ्चविꣳ॒॒शम् भ॑वति प्र॒जाप॑तेः प्र॒जाप॑तेर् भवति पञ्चविꣳ॒॒शम् प॑ञ्चविꣳ॒॒शम् भ॑वति प्र॒जाप॑तेः । \newline
61. प॒ञ्च॒विꣳ॒॒शमिति॑ पञ्च - विꣳ॒॒शम् । \newline
62. भ॒व॒ति॒ प्र॒जाप॑तेः प्र॒जाप॑तेर् भवति भवति प्र॒जाप॑ते॒ राप्त्या॒ आप्त्यै᳚ प्र॒जाप॑तेर् भवति भवति प्र॒जाप॑ते॒ राप्त्यै᳚ । \newline
63. प्र॒जाप॑ते॒ राप्त्या॒ आप्त्यै᳚ प्र॒जाप॑तेः प्र॒जाप॑ते॒ राप्त्यै॑ प॒ञ्चभिः॑ प॒ञ्चभि॒ राप्त्यै᳚ प्र॒जाप॑तेः प्र॒जाप॑ते॒ राप्त्यै॑ प॒ञ्चभिः॑ । \newline
64. प्र॒जाप॑ते॒रिति॑ प्र॒जा - प॒तेः॒ । \newline
\pagebreak
\markright{ TS 7.5.8.4  \hfill https://www.vedavms.in \hfill}

\section{ TS 7.5.8.4 }

\textbf{TS 7.5.8.4 } \newline
\textbf{Samhita Paata} \newline

-राप्त्यै॑ प॒ञ्चभि॒-स्तिष्ठ॑न्तः स्तुवन्ति देवलो॒कमे॒वाभि ज॑यन्ति प॒ञ्चभि॒रासी॑ना मनुष्यलो॒कमे॒वाभि ज॑यन्ति॒ दश॒ संप॑द्यन्ते॒ दशा᳚क्षरा वि॒राडन्नं॑ ॅवि॒राड् वि॒राजै॒वा-न्नाद्य॒मव॑ रुन्धते पञ्च॒धा वि॑नि॒षद्य॑ स्तुवन्ति॒ पञ्च॒ दिशो॑ दि॒क्षवे॑व प्रति॑तिष्ठ॒न्त्येकै॑क॒याऽस्तु॑तया स॒माय॑न्ति दि॒ग्भ्य ए॒वान्नाद्यꣳ॒॒ सं भ॑रन्ति॒ ताभि॑-रुद्गा॒तोद्-गा॑यति दि॒ग्भ्य ए॒वान्नाद्यꣳ॑-[  ] \newline

\textbf{Pada Paata} \newline

आप्त्यै᳚ । प॒ञ्चभि॒रिति॑ प॒ञ्च - भिः॒ । तिष्ठ॑न्तः । स्तु॒व॒न्ति॒ । दे॒व॒लो॒कमिति॑ देव - लो॒कम् । ए॒व । अ॒भीति॑ । ज॒य॒न्ति॒ । प॒ञ्चभि॒रिति॑ प॒ञ्च - भिः॒ । आसी॑नाः । म॒नु॒ष्य॒लो॒कमिति॑ मनुष्य - लो॒कम् । ए॒व । अ॒भीति॑ । ज॒य॒न्ति॒ । दश॑ । समिति॑ । प॒द्य॒न्ते॒ । दशा᳚क्ष॒रेति॒ दश॑ - अ॒क्ष॒रा॒ । वि॒राडिति॑ वि - राट् । अन्न᳚म् । वि॒राडिति॑ वि - राट् । वि॒राजेति॑ वि - राजा᳚ । ए॒व । अ॒न्नाद्य॒मित्य॑न्न -अद्य᳚म् । अवेति॑ । रु॒न्ध॒ते॒ । प॒ञ्च॒धेति॑ पञ्च - धा । वि॒नि॒षद्येति॑ वि - नि॒षद्य॑ । स्तु॒व॒न्ति॒ । पञ्च॑ । दिशः॑ । दि॒क्षु । ए॒व । प्रतीति॑ । ति॒ष्ठ॒न्ति॒ । एकै॑क॒येत्येक॑या - ए॒क॒या॒ । अस्तु॑तया । स॒माय॒न्तीति॑ सं - आय॑न्ति । दि॒ग्भ्य इति॑ दिक् - भ्यः । ए॒व । अ॒न्नाद्य॒मित्य॑न्न - अद्य᳚म् । समिति॑ । भ॒र॒न्ति॒ । ताभिः॑ । उ॒द्गा॒तेत्यु॑त् - गा॒ता । उदिति॑ । गा॒य॒ति॒ । दि॒ग्भ्य इति॑ दिक् - भ्यः । ए॒व । अ॒न्नाद्य॒मित्य॑न्न - अद्य᳚म् ।  \newline


\textbf{Krama Paata} \newline

आप्त्यै॑ प॒ञ्चभिः॑ । प॒ञ्चभि॒स्तिष्ठ॑न्तः । प॒ञ्चभि॒रिति॑ प॒ञ्च - भिः॒ । तिष्ठ॑न्तः स्तुवन्ति । स्तु॒व॒न्ति॒ दे॒व॒लो॒कम् । दे॒व॒लो॒कमे॒व । दे॒व॒लो॒कमिति॑ देव - लो॒कम् । ए॒वाभि । अ॒भि ज॑यन्ति । ज॒य॒न्ति॒ प॒ञ्चभिः॑ । प॒ञ्चभि॒रासी॑नाः । प॒ञ्चभि॒रिति॑ प॒ञ्च - भिः॒ । आसी॑ना मनुष्यलो॒कम् । म॒नु॒ष्य॒लो॒कमे॒व । म॒नु॒ष्य॒लो॒कमिति॑ मनुष्य - लो॒कम् । ए॒वाभि । अ॒भि ज॑यन्ति । ज॒य॒न्ति॒ दश॑ । दश॒ सम् । सम् प॑द्यन्ते । प॒द्य॒न्ते॒ दशा᳚क्षरा । दशा᳚क्षरा वि॒राट् । दशा᳚क्ष॒रेति॒ दश॑ - अ॒क्ष॒रा॒ । वि॒राडन्न᳚म् । वि॒राडिति॑ वि - राट् । अन्न॑म् ॅवि॒राट् । वि॒राड् वि॒राजा᳚ । वि॒राडिति॑ वि - राट् । वि॒राजै॒व । वि॒राजेति॑ वि - राजा᳚ । ए॒वान्नाद्य᳚म् । अ॒न्नाद्य॒मव॑ । अ॒न्नाद्य॒मित्य॑न्न - अद्य᳚म् । अव॑ रुन्धते । रु॒न्ध॒ते॒ प॒ञ्च॒धा । प॒ञ्च॒धा वि॑नि॒षद्य॑ । प॒ञ्च॒धेति॑ पञ्च - धा । वि॒नि॒षद्य॑ स्तुवन्ति । वि॒नि॒षद्येति॑ वि - नि॒षद्य॑ । स्तु॒व॒न्ति॒ पञ्च॑ । पञ्च॒ दिशः॑ । दिशो॑ दि॒क्षु । दि॒क्ष्वे॑व । ए॒व प्रति॑ । प्रति॑ तिष्ठन्ति । ति॒ष्ठ॒न्त्येकै॑कया । एकै॑क॒याऽस्तु॑तया । एकै॑क॒येत्येक॑या - ए॒क॒या॒ । अस्तु॑तया स॒माय॑न्ति । स॒माय॑न्ति दि॒ग्भ्यः । स॒माय॒न्तीति॑ सम् - आय॑न्ति । दि॒ग्भ्य ए॒व । दि॒ग्भ्य इति॑ दिक् - भ्यः । ए॒वान्नाद्य᳚म् । अ॒न्नाद्यꣳ॒॒ सम् । अ॒न्नाद्या॒मित्य॑न्न - अद्य᳚म् । सम् भ॑रन्ति । भ॒र॒न्ति॒ ताभिः॑ । ताभि॑रुद्‍गा॒ता । उ॒द्‍गा॒तोत् । उ॒द्‍गा॒तेत्यु॑त् - गा॒ता । उद् गा॑यति । गा॒य॒ति॒ दि॒ग्भ्यः । दि॒ग्भ्य ए॒व । दि॒ग्भ्य इति॑ दिक् - भ्यः । ए॒वान्नाद्य᳚म् । अ॒न्नाद्यꣳ॑ स॒म्भृत्य॑ । अ॒न्नाद्य॒मित्य॑न्न - अद्य᳚म् \newline

\textbf{Jatai Paata} \newline

1. आप्त्यै॑ प॒ञ्चभिः॑ प॒ञ्चभि॒ राप्त्या॒ आप्त्यै॑ प॒ञ्चभिः॑ । \newline
2. प॒ञ्चभि॒ स्तिष्ठ॑न्त॒ स्तिष्ठ॑न्तः प॒ञ्चभिः॑ प॒ञ्चभि॒ स्तिष्ठ॑न्तः । \newline
3. प॒ञ्चभि॒रिति॑ प॒ञ्च - भिः॒ । \newline
4. तिष्ठ॑न्तः स्तुवन्ति स्तुवन्ति॒ तिष्ठ॑न्त॒ स्तिष्ठ॑न्तः स्तुवन्ति । \newline
5. स्तु॒व॒न्ति॒ दे॒व॒लो॒कम् दे॑वलो॒कꣳ स्तु॑वन्ति स्तुवन्ति देवलो॒कम् । \newline
6. दे॒व॒लो॒क मे॒वैव दे॑वलो॒कम् दे॑वलो॒क मे॒व । \newline
7. दे॒व॒लो॒कमिति॑ देव - लो॒कम् । \newline
8. ए॒वा भ्या᳚(1॒)भ्ये॑ वैवाभि । \newline
9. अ॒भि ज॑यन्ति जय न्त्य॒भ्य॑भि ज॑यन्ति । \newline
10. ज॒य॒न्ति॒ प॒ञ्चभिः॑ प॒ञ्चभि॑र् जयन्ति जयन्ति प॒ञ्चभिः॑ । \newline
11. प॒ञ्चभि॒ रासी॑ना॒ आसी॑नाः प॒ञ्चभिः॑ प॒ञ्चभि॒ रासी॑नाः । \newline
12. प॒ञ्चभि॒रिति॑ प॒ञ्च - भिः॒ । \newline
13. आसी॑ना मनुष्यलो॒कम् म॑नुष्यलो॒क मासी॑ना॒ आसी॑ना मनुष्यलो॒कम् । \newline
14. म॒नु॒ष्य॒लो॒क मे॒वैव म॑नुष्यलो॒कम् म॑नुष्यलो॒क मे॒व । \newline
15. म॒नु॒ष्य॒लो॒कमिति॑ मनुष्य - लो॒कम् । \newline
16. ए॒वा भ्या᳚(1॒)भ्ये॑ वैवाभि । \newline
17. अ॒भि ज॑यन्ति जय न्त्य॒भ्य॑भि ज॑यन्ति । \newline
18. ज॒य॒न्ति॒ दश॒ दश॑ जयन्ति जयन्ति॒ दश॑ । \newline
19. दश॒ सꣳ सम् दश॒ दश॒ सम् । \newline
20. सम् प॑द्यन्ते पद्यन्ते॒ सꣳ सम् प॑द्यन्ते । \newline
21. प॒द्य॒न्ते॒ दशा᳚क्षरा॒ दशा᳚क्षरा पद्यन्ते पद्यन्ते॒ दशा᳚क्षरा । \newline
22. दशा᳚क्षरा वि॒राड् वि॒राड् दशा᳚क्षरा॒ दशा᳚क्षरा वि॒राट् । \newline
23. दशा᳚क्ष॒रेति॒ दश॑ - अ॒क्ष॒रा॒ । \newline
24. वि॒रा डन्न॒ मन्नं॑ ॅवि॒राड् वि॒रा डन्न᳚म् । \newline
25. वि॒राडिति॑ वि - राट् । \newline
26. अन्नं॑ ॅवि॒राड् वि॒रा डन्न॒ मन्नं॑ ॅवि॒राट् । \newline
27. वि॒राड् वि॒राजा॑ वि॒राजा॑ वि॒राड् वि॒राड् वि॒राजा᳚ । \newline
28. वि॒राडिति॑ वि - राट् । \newline
29. वि॒रा जै॒वैव वि॒राजा॑ वि॒रा जै॒व । \newline
30. वि॒राजेति॑ वि - राजा᳚ । \newline
31. ए॒वान्नाद्य॑ म॒न्नाद्य॑ मे॒वै वान्नाद्य᳚म् । \newline
32. अ॒न्नाद्य॒ मवा वा॒न्नाद्य॑ म॒न्नाद्य॒ मव॑ । \newline
33. अ॒न्नाद्य॒मित्य॑न्न - अद्य᳚म् । \newline
34. अव॑ रुन्धते रुन्ध॒ते ऽवाव॑ रुन्धते । \newline
35. रु॒न्ध॒ते॒ प॒ञ्च॒धा प॑ञ्च॒धा रु॑न्धते रुन्धते पञ्च॒धा । \newline
36. प॒ञ्च॒धा वि॑नि॒षद्य॑ विनि॒षद्य॑ पञ्च॒धा प॑ञ्च॒धा वि॑नि॒षद्य॑ । \newline
37. प॒ञ्च॒धेति॑ पञ्च - धा । \newline
38. वि॒नि॒षद्य॑ स्तुवन्ति स्तुवन्ति विनि॒षद्य॑ विनि॒षद्य॑ स्तुवन्ति । \newline
39. वि॒नि॒षद्येति॑ वि - नि॒षद्य॑ । \newline
40. स्तु॒व॒न्ति॒ पञ्च॒ पञ्च॑ स्तुवन्ति स्तुवन्ति॒ पञ्च॑ । \newline
41. पञ्च॒ दिशो॒ दिशः॒ पञ्च॒ पञ्च॒ दिशः॑ । \newline
42. दिशो॑ दि॒क्षु दि॒क्षु दिशो॒ दिशो॑ दि॒क्षु । \newline
43. दि॒क्ष्वे॑वैव दि॒क्षु दि॒क्ष्वे॑व । \newline
44. ए॒व प्रति॒ प्रत्ये॒वैव प्रति॑ । \newline
45. प्रति॑ तिष्ठन्ति तिष्ठन्ति॒ प्रति॒ प्रति॑ तिष्ठन्ति । \newline
46. ति॒ष्ठ॒ न्त्येकै॑क॒ यैकै॑कया तिष्ठन्ति तिष्ठ॒ न्त्येकै॑कया । \newline
47. एकै॑क॒या ऽस्तु॑त॒या ऽस्तु॑त॒ यैकै॑क॒ यैकै॑क॒या ऽस्तु॑तया । \newline
48. एकै॑क॒येत्येक॑या - ए॒क॒या॒ । \newline
49. अस्तु॑तया स॒माय॑न्ति स॒माय॒ न्त्यस्तु॑त॒या ऽस्तु॑तया स॒माय॑न्ति । \newline
50. स॒माय॑न्ति दि॒ग्भ्यो दि॒ग्भ्यः स॒माय॑न्ति स॒माय॑न्ति दि॒ग्भ्यः । \newline
51. स॒माय॒न्तीति॑ सं - आय॑न्ति । \newline
52. दि॒ग्भ्य ए॒वैव दि॒ग्भ्यो दि॒ग्भ्य ए॒व । \newline
53. दि॒ग्भ्य इति॑ दिक् - भ्यः । \newline
54. ए॒वान्नाद्य॑ म॒न्नाद्य॑ मे॒वै वान्नाद्य᳚म् । \newline
55. अ॒न्नाद्यꣳ॒॒ सꣳ स म॒न्नाद्य॑ म॒न्नाद्यꣳ॒॒ सम् । \newline
56. अ॒न्नाद्य॒मित्य॑न्न - अद्य᳚म् । \newline
57. सम् भ॑रन्ति भरन्ति॒ सꣳ सम् भ॑रन्ति । \newline
58. भ॒र॒न्ति॒ ताभि॒ स्ताभि॑र् भरन्ति भरन्ति॒ ताभिः॑ । \newline
59. ताभि॑ रुद्‍गा॒ तोद्‍गा॒ता ताभि॒ स्ताभि॑ रुद्‍गा॒ता । \newline
60. उ॒द्‍गा॒ तोदु दु॑द्‍गा॒ तोद्‍गा॒ तोत् । \newline
61. उ॒द्‍गा॒तेत्यु॑त् - गा॒ता । \newline
62. उद् गा॑यति गाय॒ त्युदुद् गा॑यति । \newline
63. गा॒य॒ति॒ दि॒ग्भ्यो दि॒ग्भ्यो गा॑यति गायति दि॒ग्भ्यः । \newline
64. दि॒ग्भ्य ए॒वैव दि॒ग्भ्यो दि॒ग्भ्य ए॒व । \newline
65. दि॒ग्भ्य इति॑ दिक् - भ्यः । \newline
66. ए॒वान्नाद्य॑ म॒न्नाद्य॑ मे॒वै वान्नाद्य᳚म् । \newline
67. अ॒न्नाद्यꣳ॑ सं॒भृत्य॑ सं॒भृ त्या॒न्नाद्य॑ म॒न्नाद्यꣳ॑ सं॒भृत्य॑ । \newline
68. अ॒न्नाद्य॒मित्य॑न्न - अद्य᳚म् । \newline

\textbf{Ghana Paata } \newline

1. आप्त्यै॑ प॒ञ्चभिः॑ प॒ञ्चभि॒ राप्त्या॒ आप्त्यै॑ प॒ञ्चभि॒ स्तिष्ठ॑न्त॒ स्तिष्ठ॑न्तः प॒ञ्चभि॒ राप्त्या॒ आप्त्यै॑ प॒ञ्चभि॒ स्तिष्ठ॑न्तः । \newline
2. प॒ञ्चभि॒ स्तिष्ठ॑न्त॒ स्तिष्ठ॑न्तः प॒ञ्चभिः॑ प॒ञ्चभि॒ स्तिष्ठ॑न्तः स्तुवन्ति स्तुवन्ति॒ तिष्ठ॑न्तः प॒ञ्चभिः॑ प॒ञ्चभि॒ स्तिष्ठ॑न्तः स्तुवन्ति । \newline
3. प॒ञ्चभि॒रिति॑ प॒ञ्च - भिः॒ । \newline
4. तिष्ठ॑न्तः स्तुवन्ति स्तुवन्ति॒ तिष्ठ॑न्त॒ स्तिष्ठ॑न्तः स्तुवन्ति देवलो॒कम् दे॑वलो॒कꣳ स्तु॑वन्ति॒ तिष्ठ॑न्त॒ स्तिष्ठ॑न्तः स्तुवन्ति देवलो॒कम् । \newline
5. स्तु॒व॒न्ति॒ दे॒व॒लो॒कम् दे॑वलो॒कꣳ स्तु॑वन्ति स्तुवन्ति देवलो॒क मे॒वैव दे॑वलो॒कꣳ स्तु॑वन्ति स्तुवन्ति देवलो॒क मे॒व । \newline
6. दे॒व॒लो॒क मे॒वैव दे॑वलो॒कम् दे॑वलो॒क मे॒वाभ्या᳚(1॒)भ्ये॑व दे॑वलो॒कम् दे॑वलो॒क मे॒वाभि । \newline
7. दे॒व॒लो॒कमिति॑ देव - लो॒कम् । \newline
8. ए॒वाभ्या᳚(1॒)भ्ये॑ वैवाभि ज॑यन्ति जय न्त्य॒भ्ये॑ वैवाभि ज॑यन्ति । \newline
9. अ॒भि ज॑यन्ति जय न्त्य॒भ्य॑भि ज॑यन्ति प॒ञ्चभिः॑ प॒ञ्चभि॑र् जय न्त्य॒भ्य॑भि ज॑यन्ति प॒ञ्चभिः॑ । \newline
10. ज॒य॒न्ति॒ प॒ञ्चभिः॑ प॒ञ्चभि॑र् जयन्ति जयन्ति प॒ञ्चभि॒ रासी॑ना॒ आसी॑नाः प॒ञ्चभि॑र् जयन्ति जयन्ति प॒ञ्चभि॒ रासी॑नाः । \newline
11. प॒ञ्चभि॒ रासी॑ना॒ आसी॑नाः प॒ञ्चभिः॑ प॒ञ्चभि॒ रासी॑ना मनुष्यलो॒कम् म॑नुष्यलो॒क मासी॑नाः प॒ञ्चभिः॑ प॒ञ्चभि॒ रासी॑ना मनुष्यलो॒कम् । \newline
12. प॒ञ्चभि॒रिति॑ प॒ञ्च - भिः॒ । \newline
13. आसी॑ना मनुष्यलो॒कम् म॑नुष्यलो॒क मासी॑ना॒ आसी॑ना मनुष्यलो॒क मे॒वैव म॑नुष्यलो॒क मासी॑ना॒ आसी॑ना मनुष्यलो॒क मे॒व । \newline
14. म॒नु॒ष्य॒लो॒क मे॒वैव म॑नुष्यलो॒कम् म॑नुष्यलो॒क मे॒वाभ्या᳚(1॒)भ्ये॑व म॑नुष्यलो॒कम् म॑नुष्यलो॒क मे॒वाभि । \newline
15. म॒नु॒ष्य॒लो॒कमिति॑ मनुष्य - लो॒कम् । \newline
16. ए॒वाभ्या᳚(1॒)भ्ये॑ वैवाभि ज॑यन्ति जय न्त्य॒भ्ये॑ वैवाभि ज॑यन्ति । \newline
17. अ॒भि ज॑यन्ति जय न्त्य॒भ्य॑भि ज॑यन्ति॒ दश॒ दश॑ जय न्त्य॒भ्य॑भि ज॑यन्ति॒ दश॑ । \newline
18. ज॒य॒न्ति॒ दश॒ दश॑ जयन्ति जयन्ति॒ दश॒ सꣳ सम् दश॑ जयन्ति जयन्ति॒ दश॒ सम् । \newline
19. दश॒ सꣳ सम् दश॒ दश॒ सम् प॑द्यन्ते पद्यन्ते॒ सम् दश॒ दश॒ सम् प॑द्यन्ते । \newline
20. सम् प॑द्यन्ते पद्यन्ते॒ सꣳ सम् प॑द्यन्ते॒ दशा᳚क्षरा॒ दशा᳚क्षरा पद्यन्ते॒ सꣳ सम् प॑द्यन्ते॒ दशा᳚क्षरा । \newline
21. प॒द्य॒न्ते॒ दशा᳚क्षरा॒ दशा᳚क्षरा पद्यन्ते पद्यन्ते॒ दशा᳚क्षरा वि॒राड् वि॒राड् दशा᳚क्षरा पद्यन्ते पद्यन्ते॒ दशा᳚क्षरा वि॒राट् । \newline
22. दशा᳚क्षरा वि॒राड् वि॒राड् दशा᳚क्षरा॒ दशा᳚क्षरा वि॒रा डन्न॒ मन्नं॑ ॅवि॒राड् दशा᳚क्षरा॒ दशा᳚क्षरा वि॒रा डन्न᳚म् । \newline
23. दशा᳚क्ष॒रेति॒ दश॑ - अ॒क्ष॒रा॒ । \newline
24. वि॒रा डन्न॒ मन्नं॑ ॅवि॒राड् वि॒रा डन्नं॑ ॅवि॒राड् वि॒रा डन्नं॑ ॅवि॒राड् वि॒रा डन्नं॑ ॅवि॒राट् । \newline
25. वि॒राडिति॑ वि - राट् । \newline
26. अन्नं॑ ॅवि॒राड् वि॒रा डन्न॒ मन्नं॑ ॅवि॒राड् वि॒राजा॑ वि॒राजा॑ वि॒रा डन्न॒ मन्नं॑ ॅवि॒राड् वि॒राजा᳚ । \newline
27. वि॒राड् वि॒राजा॑ वि॒राजा॑ वि॒राड् वि॒राड् वि॒राजै॒वैव वि॒राजा॑ वि॒राड् वि॒राड् वि॒रा जै॒व । \newline
28. वि॒राडिति॑ वि - राट् । \newline
29. वि॒राजै॒वैव वि॒राजा॑ वि॒राजै॒ वान्नाद्य॑ म॒न्नाद्य॑ मे॒व वि॒राजा॑ वि॒रा जै॒वान्नाद्य᳚म् । \newline
30. वि॒राजेति॑ वि - राजा᳚ । \newline
31. ए॒वान्नाद्य॑ म॒न्नाद्य॑ मे॒वै वान्नाद्य॒ मवा वा॒न्नाद्य॑ मे॒वै वान्नाद्य॒ मव॑ । \newline
32. अ॒न्नाद्य॒ मवा वा॒न्नाद्य॑ म॒न्नाद्य॒ मव॑ रुन्धते रुन्ध॒ते ऽवा॒न्नाद्य॑ म॒न्नाद्य॒ मव॑ रुन्धते । \newline
33. अ॒न्नाद्य॒मित्य॑न्न - अद्य᳚म् । \newline
34. अव॑ रुन्धते रुन्ध॒ते ऽवाव॑ रुन्धते पञ्च॒धा प॑ञ्च॒धा रु॑न्ध॒ते ऽवाव॑ रुन्धते पञ्च॒धा । \newline
35. रु॒न्ध॒ते॒ प॒ञ्च॒धा प॑ञ्च॒धा रु॑न्धते रुन्धते पञ्च॒धा वि॑नि॒षद्य॑ विनि॒षद्य॑ पञ्च॒धा रु॑न्धते रुन्धते पञ्च॒धा वि॑नि॒षद्य॑ । \newline
36. प॒ञ्च॒धा वि॑नि॒षद्य॑ विनि॒षद्य॑ पञ्च॒धा प॑ञ्च॒धा वि॑नि॒षद्य॑ स्तुवन्ति स्तुवन्ति विनि॒षद्य॑ पञ्च॒धा प॑ञ्च॒धा वि॑नि॒षद्य॑ स्तुवन्ति । \newline
37. प॒ञ्च॒धेति॑ पञ्च - धा । \newline
38. वि॒नि॒षद्य॑ स्तुवन्ति स्तुवन्ति विनि॒षद्य॑ विनि॒षद्य॑ स्तुवन्ति॒ पञ्च॒ पञ्च॑ स्तुवन्ति विनि॒षद्य॑ विनि॒षद्य॑ स्तुवन्ति॒ पञ्च॑ । \newline
39. वि॒नि॒षद्येति॑ वि - नि॒षद्य॑ । \newline
40. स्तु॒व॒न्ति॒ पञ्च॒ पञ्च॑ स्तुवन्ति स्तुवन्ति॒ पञ्च॒ दिशो॒ दिशः॒ पञ्च॑ स्तुवन्ति स्तुवन्ति॒ पञ्च॒ दिशः॑ । \newline
41. पञ्च॒ दिशो॒ दिशः॒ पञ्च॒ पञ्च॒ दिशो॑ दि॒क्षु दि॒क्षु दिशः॒ पञ्च॒ पञ्च॒ दिशो॑ दि॒क्षु । \newline
42. दिशो॑ दि॒क्षु दि॒क्षु दिशो॒ दिशो॑ दि॒क्ष्वे॑वैव दि॒क्षु दिशो॒ दिशो॑ दि॒क्ष्वे॑व । \newline
43. दि॒क्ष्वे॑वैव दि॒क्षु दि॒क्ष्वे॑व प्रति॒ प्रत्ये॒व दि॒क्षु दि॒क्ष्वे॑व प्रति॑ । \newline
44. ए॒व प्रति॒ प्रत्ये॒वैव प्रति॑ तिष्ठन्ति तिष्ठन्ति॒ प्रत्ये॒वैव प्रति॑ तिष्ठन्ति । \newline
45. प्रति॑ तिष्ठन्ति तिष्ठन्ति॒ प्रति॒ प्रति॑ तिष्ठ॒ न्त्येकै॑क॒ यैकै॑कया तिष्ठन्ति॒ प्रति॒ प्रति॑ तिष्ठ॒ न्त्येकै॑कया । \newline
46. ति॒ष्ठ॒ न्त्येकै॑क॒ यैकै॑कया तिष्ठन्ति तिष्ठ॒ न्त्येकै॑क॒या ऽस्तु॑त॒या ऽस्तु॑त॒ यैकै॑कया तिष्ठन्ति तिष्ठ॒ न्त्येकै॑क॒या ऽस्तु॑तया । \newline
47. एकै॑क॒या ऽस्तु॑त॒या ऽस्तु॑त॒ यैकै॑क॒ यैकै॑क॒या ऽस्तु॑तया स॒माय॑न्ति स॒माय॒ न्त्यस्तु॑त॒ यैकै॑क॒
यैकै॑क॒या ऽस्तु॑तया स॒माय॑न्ति । \newline
48. एकै॑क॒येत्येक॑या - ए॒क॒या॒ । \newline
49. अस्तु॑तया स॒माय॑न्ति स॒माय॒ न्त्यस्तु॑त॒या ऽस्तु॑तया स॒माय॑न्ति दि॒ग्भ्यो दि॒ग्भ्यः स॒माय॒ न्त्यस्तु॑त॒या ऽस्तु॑तया स॒माय॑न्ति दि॒ग्भ्यः । \newline
50. स॒माय॑न्ति दि॒ग्भ्यो दि॒ग्भ्यः स॒माय॑न्ति स॒माय॑न्ति दि॒ग्भ्य ए॒वैव दि॒ग्भ्यः स॒माय॑न्ति स॒माय॑न्ति दि॒ग्भ्य ए॒व । \newline
51. स॒माय॒न्तीति॑ सं - आय॑न्ति । \newline
52. दि॒ग्भ्य ए॒वैव दि॒ग्भ्यो दि॒ग्भ्य ए॒वान्नाद्य॑ म॒न्नाद्य॑ मे॒व दि॒ग्भ्यो दि॒ग्भ्य ए॒वान्नाद्य᳚म् । \newline
53. दि॒ग्भ्य इति॑ दिक् - भ्यः । \newline
54. ए॒वान्नाद्य॑ म॒न्नाद्य॑ मे॒वै वान्नाद्यꣳ॒॒ सꣳ स म॒न्नाद्य॑ मे॒वै वान्नाद्यꣳ॒॒ सम् । \newline
55. अ॒न्नाद्यꣳ॒॒ सꣳ स म॒न्नाद्य॑ म॒न्नाद्यꣳ॒॒ सम् भ॑रन्ति भरन्ति॒ स म॒न्नाद्य॑ म॒न्नाद्यꣳ॒॒ सम् भ॑रन्ति । \newline
56. अ॒न्नाद्य॒मित्य॑न्न - अद्य᳚म् । \newline
57. सम् भ॑रन्ति भरन्ति॒ सꣳ सम् भ॑रन्ति॒ ताभि॒ स्ताभि॑र् भरन्ति॒ सꣳ सम् भ॑रन्ति॒ ताभिः॑ । \newline
58. भ॒र॒न्ति॒ ताभि॒ स्ताभि॑र् भरन्ति भरन्ति॒ ताभि॑ रुद्‍गा॒ तोद्‍गा॒ता ताभि॑र् भरन्ति भरन्ति॒ ताभि॑ रुद्‍गा॒ता । \newline
59. ताभि॑ रुद्‍गा॒ तोद्‍गा॒ता ताभि॒ स्ताभि॑ रुद्‍गा॒तो दुदु॑द्‍गा॒ता ताभि॒ स्ताभि॑ रुद्‍गा॒तोत् । \newline
60. उ॒द्‍गा॒तो दुदु॑द्‍गा॒ तोद्‍गा॒ तोद्‍गा॑यति गाय॒ त्युदु॑द्‍गा॒ तोद्‍गा॒ तोद्‍गा॑यति । \newline
61. उ॒द्‍गा॒तेत्यु॑त् - गा॒ता । \newline
62. उद् गा॑यति गाय॒ त्युदुद् गा॑यति दि॒ग्भ्यो दि॒ग्भ्यो गा॑य॒ त्युदुद् गा॑यति दि॒ग्भ्यः । \newline
63. गा॒य॒ति॒ दि॒ग्भ्यो दि॒ग्भ्यो गा॑यति गायति दि॒ग्भ्य ए॒वैव दि॒ग्भ्यो गा॑यति गायति दि॒ग्भ्य ए॒व । \newline
64. दि॒ग्भ्य ए॒वैव दि॒ग्भ्यो दि॒ग्भ्य ए॒वान्नाद्य॑ म॒न्नाद्य॑ मे॒व दि॒ग्भ्यो दि॒ग्भ्य ए॒वान्नाद्य᳚म् । \newline
65. दि॒ग्भ्य इति॑ दिक् - भ्यः । \newline
66. ए॒वान्नाद्य॑ म॒न्नाद्य॑ मे॒वै वान्नाद्यꣳ॑ सं॒भृत्य॑ सं॒भृ त्या॒न्नाद्य॑ मे॒वै वान्नाद्यꣳ॑ सं॒भृत्य॑ । \newline
67. अ॒न्नाद्यꣳ॑ सं॒भृत्य॑ सं॒भृ त्या॒न्नाद्य॑ म॒न्नाद्यꣳ॑ सं॒भृत्य॒ तेज॒ स्तेजः॑ सं॒भृ त्या॒न्नाद्य॑ म॒न्नाद्यꣳ॑ सं॒भृत्य॒ तेजः॑ । \newline
68. अ॒न्नाद्य॒मित्य॑न्न - अद्य᳚म् । \newline
\pagebreak
\markright{ TS 7.5.8.5  \hfill https://www.vedavms.in \hfill}

\section{ TS 7.5.8.5 }

\textbf{TS 7.5.8.5 } \newline
\textbf{Samhita Paata} \newline

सं॒भृत्य॒ तेज॑ आ॒त्मन् द॑धते॒ तस्मा॒देकः॑ प्रा॒णः सर्वा॒ण्यङ्गा᳚न्यव॒त्यथो॒ यथा॑ सुप॒र्ण उ॑त्पति॒ष्यञ्छिर॑ उत्त॒मं कु॑रु॒त ए॒वमे॒व तद्-यज॑मानाः प्र॒जाना॑मुत्त॒मा भ॑वन्त्यास॒न्दी-मु॑द्गा॒ता ऽऽरो॑हति॒ साम्रा᳚ज्यमे॒व ग॑च्छन्ति प्ले॒ङ्खꣳ होता॒ नाक॑स्यै॒व पृ॒ष्ठꣳ रो॑हन्ति कू॒र्चाव॑द्ध्व॒र्यु-र्ब्र॒द्ध्नस्यै॒व वि॒ष्टपं॑ गच्छन्त्ये॒ताव॑न्तो॒ वै दे॑वलो॒कास्तेष्वे॒व य॑थापू॒र्वं प्रति॑ ( ) तिष्ठ॒न्त्यथो॑ आ॒क्रम॑णमे॒व तथ् सेतुं॒ ॅयज॑मानाः कुर्वते सुव॒र्गस्य॑ लो॒कस्य॒ सम॑ष्ट्यै ॥ \newline

\textbf{Pada Paata} \newline

स॒भृंत्येति॑ सं - भृत्य॑ । तेजः॑ । आ॒त्मन्न् । द॒ध॒ते॒ । तस्मा᳚त् । एकः॑ । प्रा॒ण इति॑ प्र - अ॒नः । सर्वा॑णि । अङ्गा॑नि । अ॒व॒ति॒ । अथो॒ इति॑ । यथा᳚ । सु॒प॒र्ण इति॑ सु - प॒र्णः । उ॒त्प॒ति॒ष्यन्नित्यु॑त् - प॒ति॒ष्यन्न् । शिरः॑ । उ॒त्त॒ममित्यु॑त् - त॒मम् । कु॒रु॒ते । ए॒वम् । ए॒व । तत् । यज॑मानाः । प्र॒जाना॒मिति॑ प्र - जाना᳚म् । उ॒त्त॒मा इत्यु॑त् - त॒माः । भ॒व॒न्ति॒ । आ॒स॒न्दीमित्या᳚ - स॒न्दीम् । उ॒द्गा॒तेत्यु॑त् - गा॒ता । एति॑ । रो॒ह॒ति॒ । साम्रा᳚ज्य॒मिति॒ सां - रा॒ज्य॒म् । ए॒व । ग॒च्छ॒न्ति॒ । प्ले॒ङ्खम् । होता᳚ । नाक॑स्य । ए॒व । पृ॒ष्ठम् । रो॒ह॒न्ति॒ । कू॒र्चौ । अ॒द्ध्व॒र्युः । ब्र॒द्ध्नस्य॑ । ए॒व । वि॒ष्टप᳚म् । ग॒च्छ॒न्ति॒ । ए॒ताव॑न्तः । वै । दे॒व॒लो॒का इति॑ देव - लो॒काः । तेषु॑ । ए॒व । य॒था॒पू॒र्वमिति॑ यथा - पू॒र्वम् । प्रतीति॑ ( ) । ति॒ष्ठ॒न्ति॒ । अथो॒ इति॑ । आ॒क्रम॑ण॒मित्या᳚ - क्रम॑णम् । ए॒व । तत् । सेतु᳚म् । यज॑मानाः । कु॒र्व॒ते॒ । सु॒व॒र्गस्येति॑ सुवः-गस्य॑ । लो॒कस्य॑ । सम॑ष्ट्या॒ इति॒ सं - अ॒ष्ट्यै॒ ॥  \newline


\textbf{Krama Paata} \newline

स॒म्भृत्य॒ तेजः॑ । स॒म्भृत्येति॑ सम् - भृत्य॑ । तेज॑ आ॒त्मन्न् । आ॒त्मन् द॑धते । द॒ध॒ते॒ तस्मा᳚त् । तस्मा॒देकः॑ । एकः॑ प्रा॒णः । प्रा॒णः सर्वा॑णि । प्रा॒ण इति॑ प्र - अ॒नः । सर्वा॒ण्यङ्‍गा॑नि । अङ्‍गा᳚न्यवति । अ॒व॒त्यथो᳚ । अथो॒ यथा᳚ । अथो॒ इत्यथो᳚ । यथा॑ सुप॒र्णः । सु॒प॒र्ण उ॑त्पति॒ष्यन्न् । सु॒प॒र्ण इति॑ सु - प॒र्णः । उ॒त्प॒ति॒ष्यञ्छिरः॑ । उ॒त्प॒ति॒ष्यन्नित्यु॑त् - प॒ति॒ष्यन्न् । शिर॑ उत्त॒मम् । उ॒त्त॒मम् कु॑रु॒ते । उ॒त्त॒ममित्यु॑त् - त॒मम् । कु॒रु॒त ए॒वम् । ए॒वमे॒व । ए॒व तत् । तद् यज॑मानाः । यज॑मानाः प्र॒जाना᳚म् । प्र॒जाना॑मुत्त॒माः । प्र॒जाना॒मिति॑ प्र - जाना᳚म् । उ॒त्त॒मा भ॑वन्ति । उ॒त्त॒मा इत्यु॑त् - त॒माः । भ॒व॒न्त्या॒स॒न्दीम् । आ॒स॒न्दीमु॑द्‍गा॒ता । आ॒स॒न्दीमित्या᳚ - स॒न्दीम् । उ॒द्‍गा॒ता ऽऽ रो॑हति । उ॒द्‍गा॒तेत्यु॑त् - गा॒ता । आ रो॑हति । रो॒ह॒ति॒ साम्रा᳚ज्यम् । साम्रा᳚ज्यमे॒व । साम्रा᳚ज्य॒मिति॒ साम् - रा॒ज्य॒म् । ए॒व ग॑च्छन्ति । ग॒च्छ॒न्ति॒ प्ले॒ङ्‍खम् । प्ले॒ङ्‍खꣳ होता᳚ । होता॒ नाक॑स्य । नाक॑स्यै॒व । ए॒व पृ॒ष्ठम् । पृ॒ष्ठꣳ रो॑हन्ति । रो॒ह॒न्ति॒ कू॒र्चौ । कू॒र्चाव॑द्ध्व॒र्युः । अ॒द्ध्व॒र्युर् ब्र॒द्ध्नस्य॑ । ब्र॒द्ध्नस्यै॒व । ए॒व वि॒ष्टप᳚म् । वि॒ष्टप॑म् गच्छन्ति । ग॒च्छ॒न्त्ये॒ताव॑न्तः । ए॒ताव॑न्तो॒ वै । वै दे॑वलो॒काः । दे॒व॒लो॒कास्तेषु॑ । दे॒व॒लो॒का इति॑ देव - लो॒काः । तेष्वे॒व । ए॒व य॑थापू॒र्वम् । य॒था॒पू॒र्वम् प्रति॑ ( ) । य॒था॒पू॒र्वमिति॑ यथा - पू॒र्वम् । प्रति॑ तिष्ठन्ति । ति॒ष्ठ॒न्त्यथो᳚ । अथो॑ आ॒क्रम॑णम् । अथो॒ इत्यथो᳚ । आ॒क्रम॑णमे॒व । आ॒क्रम॑ण॒मित्या᳚ - क्रम॑णम् । ए॒व तत् । तथ् सेतु᳚म् । सेतु॒म् ॅयज॑मानाः । यज॑मानाः कुर्वते । कु॒र्व॒ते॒ सु॒व॒र्गस्य॑ । सु॒व॒र्गस्य॑ लो॒कस्य॑ । सु॒व॒र्गस्येति॑ सुवः - गस्य॑ । लो॒कस्य॒ सम॑ष्ट्‍यै । सम॑ष्ट्‍या॒ इति॒ सम् - अ॒ष्ट्‍यै॒ । \newline

\textbf{Jatai Paata} \newline

1. सं॒भृत्य॒ तेज॒ स्तेजः॑ सं॒भृत्य॑ सं॒भृत्य॒ तेजः॑ । \newline
2. सं॒भृत्येति॑ सं - भृत्य॑ । \newline
3. तेज॑ आ॒त्मन् ना॒त्मन् तेज॒ स्तेज॑ आ॒त्मन्न् । \newline
4. आ॒त्मन् द॑धते दधत आ॒त्मन् ना॒त्मन् द॑धते । \newline
5. द॒ध॒ते॒ तस्मा॒त् तस्मा᳚द् दधते दधते॒ तस्मा᳚त् । \newline
6. तस्मा॒ देक॒ एक॒ स्तस्मा॒त् तस्मा॒ देकः॑ । \newline
7. एकः॑ प्रा॒णः प्रा॒ण एक॒ एकः॑ प्रा॒णः । \newline
8. प्रा॒णः सर्वा॑णि॒ सर्वा॑णि प्रा॒णः प्रा॒णः सर्वा॑णि । \newline
9. प्रा॒ण इति॑ प्र - अ॒नः । \newline
10. सर्वा॒ ण्यङ्गा॒ न्यङ्गा॑नि॒ सर्वा॑णि॒ सर्वा॒ ण्यङ्गा॑नि । \newline
11. अङ्गा᳚ न्यव त्यव॒ त्यङ्गा॒ न्यङ्गा᳚ न्यवति । \newline
12. अ॒व॒ त्यथो॒ अथो॑ अव त्यव॒ त्यथो᳚ । \newline
13. अथो॒ यथा॒ यथा ऽथो॒ अथो॒ यथा᳚ । \newline
14. अथो॒ इत्यथो᳚ । \newline
15. यथा॑ सुप॒र्णः सु॑प॒र्णो यथा॒ यथा॑ सुप॒र्णः । \newline
16. सु॒प॒र्ण उ॑त्पति॒ष्यन् नु॑त्पति॒ष्यन् थ्सु॑प॒र्णः सु॑प॒र्ण उ॑त्पति॒ष्यन्न् । \newline
17. सु॒प॒र्ण इति॑ सु - प॒र्णः । \newline
18. उ॒त्प॒ति॒ष्यञ् छिरः॒ शिर॑ उत्पति॒ष्यन् नु॑त्पति॒ष्यञ् छिरः॑ । \newline
19. उ॒त्प॒ति॒ष्यन्नित्यु॑त् - प॒ति॒ष्यन्न् । \newline
20. शिर॑ उत्त॒म मु॑त्त॒मꣳ शिरः॒ शिर॑ उत्त॒मम् । \newline
21. उ॒त्त॒मम् कु॑रु॒ते कु॑रु॒त उ॑त्त॒म मु॑त्त॒मम् कु॑रु॒ते । \newline
22. उ॒त्त॒ममित्यु॑त् - त॒मम् । \newline
23. कु॒रु॒त ए॒व मे॒वम् कु॑रु॒ते कु॑रु॒त ए॒वम् । \newline
24. ए॒व मे॒वै वैव मे॒व मे॒व । \newline
25. ए॒व तत् तदे॒वैव तत् । \newline
26. तद् यज॑माना॒ यज॑माना॒ स्तत् तद् यज॑मानाः । \newline
27. यज॑मानाः प्र॒जाना᳚म् प्र॒जानां॒ ॅयज॑माना॒ यज॑मानाः प्र॒जाना᳚म् । \newline
28. प्र॒जाना॑ मुत्त॒मा उ॑त्त॒माः प्र॒जाना᳚म् प्र॒जाना॑ मुत्त॒माः । \newline
29. प्र॒जाना॒मिति॑ प्र - जाना᳚म् । \newline
30. उ॒त्त॒मा भ॑वन्ति भव न्त्युत्त॒मा उ॑त्त॒मा भ॑वन्ति । \newline
31. उ॒त्त॒मा इत्यु॑त् - त॒माः । \newline
32. भ॒व॒ न्त्या॒स॒न्दी मा॑स॒न्दीम् भ॑वन्ति भव न्त्यास॒न्दीम् । \newline
33. आ॒स॒न्दी मु॑द्‍गा॒ तोद्‍गा॒ता ऽऽस॒न्दी मा॑स॒न्दी मु॑द्‍गा॒ता । \newline
34. आ॒स॒न्दीमित्या᳚ - स॒न्दीम् । \newline
35. उ॒द्‍गा॒ ऽऽरो॑हति रोह त्योद्‍गा॒ता तोद्‍गा॒ता ऽऽरो॑हति । \newline
36. उ॒द्‍गा॒तेत्यु॑त् - गा॒ता । \newline
37. आ रो॑हति रोह॒ त्यारो॑हति । \newline
38. रो॒ह॒ति॒ साम्रा᳚ज्यꣳ॒॒ साम्रा᳚ज्यꣳ रोहति रोहति॒ साम्रा᳚ज्यम् । \newline
39. साम्रा᳚ज्य मे॒वैव साम्रा᳚ज्यꣳ॒॒ साम्रा᳚ज्य मे॒व । \newline
40. साम्रा᳚ज्य॒मिति॒ सां - रा॒ज्य॒म् । \newline
41. ए॒व ग॑च्छन्ति गच्छ न्त्ये॒वैव ग॑च्छन्ति । \newline
42. ग॒च्छ॒न्ति॒ प्ले॒ङ्‍खम् प्ले॒ङ्‍खम् ग॑च्छन्ति गच्छन्ति प्ले॒ङ्‍खम् । \newline
43. प्ले॒ङ्‍खꣳ होता॒ होता᳚ प्ले॒ङ्‍खम् प्ले॒ङ्‍खꣳ होता᳚ । \newline
44. होता॒ नाक॑स्य॒ नाक॑स्य॒ होता॒ होता॒ नाक॑स्य । \newline
45. नाक॑ स्यै॒वैव नाक॑स्य॒ नाक॑ स्यै॒व । \newline
46. ए॒व पृ॒ष्ठम् पृ॒ष्ठ मे॒वैव पृ॒ष्ठम् । \newline
47. पृ॒ष्ठꣳ रो॑हन्ति रोहन्ति पृ॒ष्ठम् पृ॒ष्ठꣳ रो॑हन्ति । \newline
48. रो॒ह॒न्ति॒ कू॒र्चौ कू॒र्चौ रो॑हन्ति रोहन्ति कू॒र्चौ । \newline
49. कू॒र्चा व॑द्ध्व॒र्यु र॑द्ध्व॒र्युः कू॒र्चौ कू॒र्चा व॑द्ध्व॒र्युः । \newline
50. अ॒द्ध्व॒र्युर् ब्र॒द्ध्नस्य॑ ब्र॒द्ध्न स्या᳚द्ध्व॒र्यु र॑द्ध्व॒र्युर् ब्र॒द्ध्नस्य॑ । \newline
51. ब्र॒द्ध्न स्यै॒वैव ब्र॒द्ध्नस्य॑ ब्र॒द्ध्न स्यै॒व । \newline
52. ए॒व वि॒ष्टपं॑ ॅवि॒ष्टप॑ मे॒वैव वि॒ष्टप᳚म् । \newline
53. वि॒ष्टप॑म् गच्छन्ति गच्छन्ति वि॒ष्टपं॑ ॅवि॒ष्टप॑म् गच्छन्ति । \newline
54. ग॒च्छ॒ न्त्ये॒ताव॑न्त ए॒ताव॑न्तो गच्छन्ति गच्छ न्त्ये॒ताव॑न्तः । \newline
55. ए॒ताव॑न्तो॒ वै वा ए॒ताव॑न्त ए॒ताव॑न्तो॒ वै । \newline
56. वै दे॑वलो॒का दे॑वलो॒का वै वै दे॑वलो॒काः । \newline
57. दे॒व॒लो॒का स्तेषु॒ तेषु॑ देवलो॒का दे॑वलो॒का स्तेषु॑ । \newline
58. दे॒व॒लो॒का इति॑ देव - लो॒काः । \newline
59. तेष्वे॒वैव तेषु॒ तेष्वे॒व । \newline
60. ए॒व य॑थापू॒र्वं ॅय॑थापू॒र्व मे॒वैव य॑थापू॒र्वम् । \newline
61. य॒था॒पू॒र्वम् प्रति॒ प्रति॑ यथापू॒र्वं ॅय॑थापू॒र्वम् प्रति॑ । \newline
62. य॒था॒पू॒र्वमिति॑ यथा - पू॒र्वम् । \newline
63. प्रति॑ तिष्ठन्ति तिष्ठन्ति॒ प्रति॒ प्रति॑ तिष्ठन्ति । \newline
64. ति॒ष्ठ॒ न्त्यथो॒ अथो॑ तिष्ठन्ति तिष्ठ॒ न्त्यथो᳚ । \newline
65. अथो॑ आ॒क्रम॑ण मा॒क्रम॑ण॒ मथो॒ अथो॑ आ॒क्रम॑णम् । \newline
66. अथो॒ इत्यथो᳚ । \newline
67. आ॒क्रम॑ण मे॒वै वाक्रम॑ण मा॒क्रम॑ण मे॒व । \newline
68. आ॒क्रम॑ण॒मित्या᳚ - क्रम॑णम् । \newline
69. ए॒व तत् तदे॒वैव तत् । \newline
70. तथ् सेतुꣳ॒॒ सेतु॒म् तत् तथ् सेतु᳚म् । \newline
71. सेतुं॒ ॅयज॑माना॒ यज॑मानाः॒ सेतुꣳ॒॒ सेतुं॒ ॅयज॑मानाः । \newline
72. यज॑मानाः कुर्वते कुर्वते॒ यज॑माना॒ यज॑मानाः कुर्वते । \newline
73. कु॒र्व॒ते॒ सु॒व॒र्गस्य॑ सुव॒र्गस्य॑ कुर्वते कुर्वते सुव॒र्गस्य॑ । \newline
74. सु॒व॒र्गस्य॑ लो॒कस्य॑ लो॒कस्य॑ सुव॒र्गस्य॑ सुव॒र्गस्य॑ लो॒कस्य॑ । \newline
75. सु॒व॒र्गस्येति॑ सुवः - गस्य॑ । \newline
76. लो॒कस्य॒ सम॑ष्ट्यै॒ सम॑ष्ट्यै लो॒कस्य॑ लो॒कस्य॒ सम॑ष्ट्यै । \newline
77. सम॑ष्ट्या॒ इति॒ सं - अ॒ष्ट्यै॒ । \newline

\textbf{Ghana Paata } \newline

1. सं॒भृत्य॒ तेज॒ स्तेजः॑ सं॒भृत्य॑ सं॒भृत्य॒ तेज॑ आ॒त्मन् ना॒त्मन् तेजः॑ सं॒भृत्य॑ सं॒भृत्य॒ तेज॑ आ॒त्मन्न् । \newline
2. सं॒भृत्येति॑ सं - भृत्य॑ । \newline
3. तेज॑ आ॒त्मन् ना॒त्मन् तेज॒ स्तेज॑ आ॒त्मन् द॑धते दधत आ॒त्मन् तेज॒ स्तेज॑ आ॒त्मन् द॑धते । \newline
4. आ॒त्मन् द॑धते दधत आ॒त्मन् ना॒त्मन् द॑धते॒ तस्मा॒त् तस्मा᳚द् दधत आ॒त्मन् ना॒त्मन् द॑धते॒ तस्मा᳚त् । \newline
5. द॒ध॒ते॒ तस्मा॒त् तस्मा᳚द् दधते दधते॒ तस्मा॒ देक॒ एक॒ स्तस्मा᳚द् दधते दधते॒ तस्मा॒ देकः॑ । \newline
6. तस्मा॒ देक॒ एक॒ स्तस्मा॒त् तस्मा॒ देकः॑ प्रा॒णः प्रा॒ण एक॒ स्तस्मा॒त् तस्मा॒ देकः॑ प्रा॒णः । \newline
7. एकः॑ प्रा॒णः प्रा॒ण एक॒ एकः॑ प्रा॒णः सर्वा॑णि॒ सर्वा॑णि प्रा॒ण एक॒ एकः॑ प्रा॒णः सर्वा॑णि । \newline
8. प्रा॒णः सर्वा॑णि॒ सर्वा॑णि प्रा॒णः प्रा॒णः सर्वा॒ ण्यङ्गा॒ न्यङ्गा॑नि॒ सर्वा॑णि प्रा॒णः प्रा॒णः सर्वा॒ ण्यङ्गा॑नि । \newline
9. प्रा॒ण इति॑ प्र - अ॒नः । \newline
10. सर्वा॒ ण्यङ्गा॒ न्यङ्गा॑नि॒ सर्वा॑णि॒ सर्वा॒ ण्यङ्गा᳚ न्यव त्यव॒ त्यङ्गा॑नि॒ सर्वा॑णि॒ सर्वा॒ ण्यङ्गा᳚ न्यवति । \newline
11. अङ्गा᳚ न्यव त्यव॒ त्यङ्गा॒ न्यङ्गा᳚ न्यव॒ त्यथो॒ अथो॑ अव॒ त्यङ्गा॒ न्यङ्गा᳚ न्यव॒ त्यथो᳚ । \newline
12. अ॒व॒ त्यथो॒ अथो॑ अव त्यव॒ त्यथो॒ यथा॒ यथा ऽथो॑ अव त्यव॒ त्यथो॒ यथा᳚ । \newline
13. अथो॒ यथा॒ यथा ऽथो॒ अथो॒ यथा॑ सुप॒र्णः सु॑प॒र्णो यथा ऽथो॒ अथो॒ यथा॑ सुप॒र्णः । \newline
14. अथो॒ इत्यथो᳚ । \newline
15. यथा॑ सुप॒र्णः सु॑प॒र्णो यथा॒ यथा॑ सुप॒र्ण उ॑त्पति॒ष्यन् नु॑त्पति॒ष्यन् थ्सु॑प॒र्णो यथा॒ यथा॑ सुप॒र्ण उ॑त्पति॒ष्यन्न् । \newline
16. सु॒प॒र्ण उ॑त्पति॒ष्यन् नु॑त्पति॒ष्यन् थ्सु॑प॒र्णः सु॑प॒र्ण उ॑त्पति॒ष्यञ् छिरः॒ शिर॑ उत्पति॒ष्यन् थ्सु॑प॒र्णः सु॑प॒र्ण उ॑त्पति॒ष्यञ् छिरः॑ । \newline
17. सु॒प॒र्ण इति॑ सु - प॒र्णः । \newline
18. उ॒त्प॒ति॒ष्यञ् छिरः॒ शिर॑ उत्पति॒ष्यन् नु॑त्पति॒ष्यञ् छिर॑ उत्त॒म मु॑त्त॒मꣳ शिर॑ उत्पति॒ष्यन् नु॑त्पति॒ष्यञ् छिर॑ उत्त॒मम् । \newline
19. उ॒त्प॒ति॒ष्यन्नित्यु॑त् - प॒ति॒ष्यन्न् । \newline
20. शिर॑ उत्त॒म मु॑त्त॒मꣳ शिरः॒ शिर॑ उत्त॒मम् कु॑रु॒ते कु॑रु॒त उ॑त्त॒मꣳ शिरः॒ शिर॑ उत्त॒मम् कु॑रु॒ते । \newline
21. उ॒त्त॒मम् कु॑रु॒ते कु॑रु॒त उ॑त्त॒म मु॑त्त॒मम् कु॑रु॒त ए॒व मे॒वम् कु॑रु॒त उ॑त्त॒म मु॑त्त॒मम् कु॑रु॒त ए॒वम् । \newline
22. उ॒त्त॒ममित्यु॑त् - त॒मम् । \newline
23. कु॒रु॒त ए॒व मे॒वम् कु॑रु॒ते कु॑रु॒त ए॒व मे॒वै वैवम् कु॑रु॒ते कु॑रु॒त ए॒व मे॒व । \newline
24. ए॒व मे॒वै वैव मे॒व मे॒व तत् तदे॒वैव मे॒व मे॒व तत् । \newline
25. ए॒व तत् तदे॒वैव तद् यज॑माना॒ यज॑माना॒ स्तदे॒वैव तद् यज॑मानाः । \newline
26. तद् यज॑माना॒ यज॑माना॒ स्तत् तद् यज॑मानाः प्र॒जाना᳚म् प्र॒जानां॒ ॅयज॑माना॒ स्तत् तद् यज॑मानाः प्र॒जाना᳚म् । \newline
27. यज॑मानाः प्र॒जाना᳚म् प्र॒जानां॒ ॅयज॑माना॒ यज॑मानाः प्र॒जाना॑ मुत्त॒मा उ॑त्त॒माः प्र॒जानां॒ ॅयज॑माना॒ यज॑मानाः प्र॒जाना॑ मुत्त॒माः । \newline
28. प्र॒जाना॑ मुत्त॒मा उ॑त्त॒माः प्र॒जाना᳚म् प्र॒जाना॑ मुत्त॒मा भ॑वन्ति भव न्त्युत्त॒माः प्र॒जाना᳚म् प्र॒जाना॑ मुत्त॒मा भ॑वन्ति । \newline
29. प्र॒जाना॒मिति॑ प्र - जाना᳚म् । \newline
30. उ॒त्त॒मा भ॑वन्ति भव न्त्युत्त॒मा उ॑त्त॒मा भ॑व न्त्यास॒न्दी मा॑स॒न्दीम् भ॑व न्त्युत्त॒मा उ॑त्त॒मा भ॑व न्त्यास॒न्दीम् । \newline
31. उ॒त्त॒मा इत्यु॑त् - त॒माः । \newline
32. भ॒व॒ न्त्या॒स॒न्दी मा॑स॒न्दीम् भ॑वन्ति भव न्त्यास॒न्दी मु॑द्‍गा॒ तोद्‍गा॒ता ऽऽस॒न्दीम् भ॑वन्ति भव न्त्यास॒न्दी मु॑द्‍गा॒ता । \newline
33. आ॒स॒न्दी मु॑द्‍गा॒ तोद्‍गा॒ता ऽऽस॒न्दी मा॑स॒न्दी मु॑द्‍गा॒ ऽऽरो॑हति रोह॒ त्योद्‍गा॒ता ऽऽस॒न्दी मा॑स॒न्दी मु॑द्‍गा॒ता ऽऽरो॑हति । \newline
34. आ॒स॒न्दीमित्या᳚ - स॒न्दीम् । \newline
35. उ॒द्‍गा॒ ऽऽरो॑हति रोह त्योद्‍गा॒ता तोद्‍गा॒ता ऽऽरो॑हति॒ साम्रा᳚ज्यꣳ॒॒ साम्रा᳚ज्यꣳ रोह॒ त्योद्‍गा॒ तोद्‍गा॒ता ऽऽरो॑हति॒ साम्रा᳚ज्यम् । \newline
36. उ॒द्‍गा॒तेत्यु॑त् - गा॒ता । \newline
37. आ रो॑हति रोह॒ त्यारो॑हति॒ साम्रा᳚ज्यꣳ॒॒ साम्रा᳚ज्यꣳ रोह॒ त्यारो॑हति॒ साम्रा᳚ज्यम् । \newline
38. रो॒ह॒ति॒ साम्रा᳚ज्यꣳ॒॒ साम्रा᳚ज्यꣳ रोहति रोहति॒ साम्रा᳚ज्य मे॒वैव साम्रा᳚ज्यꣳ रोहति रोहति॒ साम्रा᳚ज्य मे॒व । \newline
39. साम्रा᳚ज्य मे॒वैव साम्रा᳚ज्यꣳ॒॒ साम्रा᳚ज्य मे॒व ग॑च्छन्ति गच्छ न्त्ये॒व साम्रा᳚ज्यꣳ॒॒ साम्रा᳚ज्य मे॒व ग॑च्छन्ति । \newline
40. साम्रा᳚ज्य॒मिति॒ सां - रा॒ज्य॒म् । \newline
41. ए॒व ग॑च्छन्ति गच्छ न्त्ये॒वैव ग॑च्छन्ति प्ले॒ङ्‍खम् प्ले॒ङ्‍खम् ग॑च्छ न्त्ये॒वैव ग॑च्छन्ति प्ले॒ङ्‍खम् । \newline
42. ग॒च्छ॒न्ति॒ प्ले॒ङ्‍खम् प्ले॒ङ्‍खम् ग॑च्छन्ति गच्छन्ति प्ले॒ङ्‍खꣳ होता॒ होता᳚ प्ले॒ङ्‍खम् ग॑च्छन्ति गच्छन्ति प्ले॒ङ्‍खꣳ होता᳚ । \newline
43. प्ले॒ङ्‍खꣳ होता॒ होता᳚ प्ले॒ङ्‍खम् प्ले॒ङ्‍खꣳ होता॒ नाक॑स्य॒ नाक॑स्य॒ होता᳚ प्ले॒ङ्‍खम् प्ले॒ङ्‍खꣳ होता॒ नाक॑स्य । \newline
44. होता॒ नाक॑स्य॒ नाक॑स्य॒ होता॒ होता॒ नाक॑स्यै॒वैव नाक॑स्य॒ होता॒ होता॒ नाक॑ स्यै॒व । \newline
45. नाक॑स्यै॒वैव नाक॑स्य॒ नाक॑स्यै॒व पृ॒ष्ठम् पृ॒ष्ठ मे॒व नाक॑स्य॒ नाक॑स्यै॒व पृ॒ष्ठम् । \newline
46. ए॒व पृ॒ष्ठम् पृ॒ष्ठ मे॒वैव पृ॒ष्ठꣳ रो॑हन्ति रोहन्ति पृ॒ष्ठ मे॒वैव पृ॒ष्ठꣳ रो॑हन्ति । \newline
47. पृ॒ष्ठꣳ रो॑हन्ति रोहन्ति पृ॒ष्ठम् पृ॒ष्ठꣳ रो॑हन्ति कू॒र्चौ कू॒र्चौ रो॑हन्ति पृ॒ष्ठम् पृ॒ष्ठꣳ रो॑हन्ति कू॒र्चौ । \newline
48. रो॒ह॒न्ति॒ कू॒र्चौ कू॒र्चौ रो॑हन्ति रोहन्ति कू॒र्चा व॑द्ध्व॒र्यु र॑द्ध्व॒र्युः कू॒र्चौ रो॑हन्ति रोहन्ति कू॒र्चा व॑द्ध्व॒र्युः । \newline
49. कू॒र्चा व॑द्ध्व॒र्यु र॑द्ध्व॒र्युः कू॒र्चौ कू॒र्चा व॑द्ध्व॒र्युर् ब्र॒द्ध्नस्य॑ ब्र॒द्ध्नस्या᳚ द्ध्व॒र्युः कू॒र्चौ कू॒र्चा व॑द्ध्व॒र्युर् ब्र॒द्ध्नस्य॑ । \newline
50. अ॒द्ध्व॒र्युर् ब्र॒द्ध्नस्य॑ ब्र॒द्ध्नस्या᳚ द्ध्व॒र्यु र॑द्ध्व॒र्युर् ब्र॒द्ध्न स्यै॒वैव ब्र॒द्ध्नस्या᳚ द्ध्व॒र्यु र॑द्ध्व॒र्युर् ब्र॒द्ध्न स्यै॒व । \newline
51. ब्र॒द्ध्न स्यै॒वैव ब्र॒द्ध्नस्य॑ ब्र॒द्ध्न स्यै॒व वि॒ष्टपं॑ ॅवि॒ष्टप॑ मे॒व ब्र॒द्ध्नस्य॑ ब्र॒द्ध्न स्यै॒व वि॒ष्टप᳚म् । \newline
52. ए॒व वि॒ष्टपं॑ ॅवि॒ष्टप॑ मे॒वैव वि॒ष्टप॑म् गच्छन्ति गच्छन्ति वि॒ष्टप॑ मे॒वैव वि॒ष्टप॑म् गच्छन्ति । \newline
53. वि॒ष्टप॑म् गच्छन्ति गच्छन्ति वि॒ष्टपं॑ ॅवि॒ष्टप॑म् गच्छ न्त्ये॒ताव॑न्त ए॒ताव॑न्तो गच्छन्ति वि॒ष्टपं॑ ॅवि॒ष्टप॑म् गच्छ न्त्ये॒ताव॑न्तः । \newline
54. ग॒च्छ॒ न्त्ये॒ताव॑न्त ए॒ताव॑न्तो गच्छन्ति गच्छ न्त्ये॒ताव॑न्तो॒ वै वा ए॒ताव॑न्तो गच्छन्ति गच्छ न्त्ये॒ताव॑न्तो॒ वै । \newline
55. ए॒ताव॑न्तो॒ वै वा ए॒ताव॑न्त ए॒ताव॑न्तो॒ वै दे॑वलो॒का दे॑वलो॒का वा ए॒ताव॑न्त ए॒ताव॑न्तो॒ वै दे॑वलो॒काः । \newline
56. वै दे॑वलो॒का दे॑वलो॒का वै वै दे॑वलो॒का स्तेषु॒ तेषु॑ देवलो॒का वै वै दे॑वलो॒का स्तेषु॑ । \newline
57. दे॒व॒लो॒का स्तेषु॒ तेषु॑ देवलो॒का दे॑वलो॒का स्तेष्वे॒वैव तेषु॑ देवलो॒का दे॑वलो॒का स्तेष्वे॒व । \newline
58. दे॒व॒लो॒का इति॑ देव - लो॒काः । \newline
59. तेष्वे॒वैव तेषु॒ तेष्वे॒व य॑थापू॒र्वं ॅय॑थापू॒र्व मे॒व तेषु॒ तेष्वे॒व य॑थापू॒र्वम् । \newline
60. ए॒व य॑थापू॒र्वं ॅय॑थापू॒र्व मे॒वैव य॑थापू॒र्वम् प्रति॒ प्रति॑ यथापू॒र्व मे॒वैव य॑थापू॒र्वम् प्रति॑ । \newline
61. य॒था॒पू॒र्वम् प्रति॒ प्रति॑ यथापू॒र्वं ॅय॑थापू॒र्वम् प्रति॑ तिष्ठन्ति तिष्ठन्ति॒ प्रति॑ यथापू॒र्वं ॅय॑थापू॒र्वम् प्रति॑ तिष्ठन्ति । \newline
62. य॒था॒पू॒र्वमिति॑ यथा - पू॒र्वम् । \newline
63. प्रति॑ तिष्ठन्ति तिष्ठन्ति॒ प्रति॒ प्रति॑ तिष्ठ॒ न्त्यथो॒ अथो॑ तिष्ठन्ति॒ प्रति॒ प्रति॑ तिष्ठ॒ न्त्यथो᳚ । \newline
64. ति॒ष्ठ॒ न्त्यथो॒ अथो॑ तिष्ठन्ति तिष्ठ॒ न्त्यथो॑ आ॒क्रम॑ण मा॒क्रम॑ण॒ मथो॑ तिष्ठन्ति तिष्ठ॒ न्त्यथो॑ आ॒क्रम॑णम् । \newline
65. अथो॑ आ॒क्रम॑ण मा॒क्रम॑ण॒ मथो॒ अथो॑ आ॒क्रम॑ण मे॒वै वाक्रम॑ण॒ मथो॒ अथो॑ आ॒क्रम॑ण मे॒व । \newline
66. अथो॒ इत्यथो᳚ । \newline
67. आ॒क्रम॑ण मे॒वै वाक्रम॑ण मा॒क्रम॑ण मे॒व तत् तदे॒वाक्रम॑ण मा॒क्रम॑ण मे॒व तत् । \newline
68. आ॒क्रम॑ण॒मित्या᳚ - क्रम॑णम् । \newline
69. ए॒व तत् तदे॒वैव तथ् सेतुꣳ॒॒ सेतु॒म् तदे॒वैव तथ् सेतु᳚म् । \newline
70. तथ् सेतुꣳ॒॒ सेतु॒म् तत् तथ् सेतुं॒ ॅयज॑माना॒ यज॑मानाः॒ सेतु॒म् तत् तथ् सेतुं॒ ॅयज॑मानाः । \newline
71. सेतुं॒ ॅयज॑माना॒ यज॑मानाः॒ सेतुꣳ॒॒ सेतुं॒ ॅयज॑मानाः कुर्वते कुर्वते॒ यज॑मानाः॒ सेतुꣳ॒॒ सेतुं॒ ॅयज॑मानाः कुर्वते । \newline
72. यज॑मानाः कुर्वते कुर्वते॒ यज॑माना॒ यज॑मानाः कुर्वते सुव॒र्गस्य॑ सुव॒र्गस्य॑ कुर्वते॒ यज॑माना॒ यज॑मानाः कुर्वते सुव॒र्गस्य॑ । \newline
73. कु॒र्व॒ते॒ सु॒व॒र्गस्य॑ सुव॒र्गस्य॑ कुर्वते कुर्वते सुव॒र्गस्य॑ लो॒कस्य॑ लो॒कस्य॑ सुव॒र्गस्य॑ कुर्वते कुर्वते सुव॒र्गस्य॑ लो॒कस्य॑ । \newline
74. सु॒व॒र्गस्य॑ लो॒कस्य॑ लो॒कस्य॑ सुव॒र्गस्य॑ सुव॒र्गस्य॑ लो॒कस्य॒ सम॑ष्ट्यै॒ सम॑ष्ट्यै लो॒कस्य॑ सुव॒र्गस्य॑ सुव॒र्गस्य॑ लो॒कस्य॒ सम॑ष्ट्यै । \newline
75. सु॒व॒र्गस्येति॑ सुवः - गस्य॑ । \newline
76. लो॒कस्य॒ सम॑ष्ट्यै॒ सम॑ष्ट्यै लो॒कस्य॑ लो॒कस्य॒ सम॑ष्ट्यै । \newline
77. सम॑ष्ट्या॒ इति॒ सं - अ॒ष्ट्यै॒ । \newline
\pagebreak
\markright{ TS 7.5.9.1  \hfill https://www.vedavms.in \hfill}

\section{ TS 7.5.9.1 }

\textbf{TS 7.5.9.1 } \newline
\textbf{Samhita Paata} \newline

अ॒र्क्ये॑ण॒ वै स॑हस्र॒शः प्र॒जाप॑तिः प्र॒जा अ॑सृजत॒ ताभ्य॒ इला᳚दें॒नेरां॒ ॅलूता॒मवा॑रुन्ध॒ यद॒र्क्यं॑ भव॑ति प्र॒जा ए॒व तद्-यज॑मानाः सृजन्त॒ इला᳚दं भवति प्र॒जाभ्य॑ ए॒व सृ॒ष्टाभ्य॒ इरां॒ ॅलूता॒मव॑ रुन्धते॒ तस्मा॒द्याꣳ समाꣳ॑ स॒त्रꣳ समृ॑द्धं॒ क्षोधु॑का॒स्ताꣳ समां᳚ प्र॒जा इषꣳ॒॒ ह्या॑सा॒मूर्ज॑मा॒दद॑ते॒ याꣳ समां॒ ॅव्यृ॑द्ध॒-मक्षो॑धुका॒स्ताꣳ समां᳚ प्र॒जा - [  ] \newline

\textbf{Pada Paata} \newline

अ॒र्क्ये॑ण । वै । स॒ह॒स्र॒श इति॑ सहस्र - शः । प्र॒जाप॑ति॒रिति॑ प्र॒जा - प॒तिः॒ । प्र॒जा इति॑ प्र - जाः । अ॒सृ॒ज॒त॒ । ताभ्यः॑ । इला᳚न्देन । इरा᳚म् । लूता᳚म् । अवेति॑ । अ॒रु॒न्ध॒ । यत् । अ॒र्क्य᳚म् । भव॑ति । प्र॒जा इति॑ प्र - जाः । ए॒व । तत् । यज॑मानाः । सृ॒ज॒न्ते॒ । इला᳚न्दम् । भ॒व॒ति॒ । प्र॒जाभ्य॒ इति॑ प्र - जाभ्यः॑ । ए॒व । सृ॒ष्टाभ्यः॑ । इरा᳚म् । लूता᳚म् । अवेति॑ । रु॒न्ध॒ते॒ । तस्मा᳚त् । याम् । समा᳚म् । स॒त्रम् । समृ॑द्ध॒मिति॒ सं - ऋ॒द्ध॒म् । क्षोधु॑काः । ताम् । समा᳚म् । प्र॒जा इति॑ प्र - जाः । इष᳚म् । हि । आ॒सा॒म् । ऊर्ज᳚म् । आ॒दद॑त॒ इत्या᳚ - दद॑ते । याम् । समा᳚म् । व्यृ॑द्ध॒मिति॒ वि - ऋ॒द्ध॒म् । अक्षो॑धुकाः । ताम् । समा᳚म् । प्र॒जा इति॑ प्र - जाः ।  \newline


\textbf{Krama Paata} \newline

अ॒र्क्ये॑ण॒ वै । वै स॑हस्र॒शः । स॒ह॒स्र॒शः प्र॒जाप॑तिः । स॒ह॒स्र॒श इति॑ सहस्र - शः । प्र॒जाप॑तिः प्र॒जाः । प्र॒जाप॑ति॒रिति॑ प्र॒जा - प॒तिः॒ । प्र॒जा अ॑सृजत । प्र॒जा इति॑ प्र - जाः । अ॒सृ॒ज॒त॒ ताभ्यः॑ । ताभ्य॒ इला᳚न्देन । इला᳚न्दे॒नेरा᳚म् । इरा॒म् ॅलूता᳚म् । लूता॒मव॑ । अवा॑रुन्ध । अ॒रु॒न्ध॒ यत् । यद॒र्क्य᳚म् । अ॒र्क्य॑म् भव॑ति । भव॑ति प्र॒जाः । प्र॒जा ए॒व । प्र॒जा इति॑ प्र - जाः । ए॒व तत् । तद् यज॑मानाः । यज॑मानाः सृजन्ते । सृ॒ज॒न्त॒ इला᳚न्दम् । इला᳚न्दम् भवति । भ॒व॒ति॒ प्र॒जाभ्यः॑ । प्र॒जाभ्य॑ ए॒व । प्र॒जाभ्य॒ इति॑ प्र - जाभ्यः॑ । ए॒व सृ॒ष्टाभ्यः॑ । सृ॒ष्टाभ्य॒ इरा᳚म् । इरा॒म् ॅलूता᳚म् । लूता॒मव॑ । अव॑ रुन्धते । रु॒न्ध॒ते॒ तस्मा᳚त् । तस्मा॒द् याम् । याꣳ समा᳚म् । समाꣳ॑ स॒त्रम् । स॒त्रꣳ समृ॑द्धम् । समृ॑द्ध॒म् क्षोधु॑काः । समृ॑द्ध॒मिति॒ सम् - ऋ॒द्ध॒म् । क्षोधु॑का॒स्ताम् । ताꣳ समा᳚म् । समा᳚म् प्र॒जाः । प्र॒जा इष᳚म् । प्र॒जा इति॑ प्र - जाः । इषꣳ॒॒ हि । ह्या॑साम् । आ॒सा॒मूर्ज᳚म् । ऊर्ज॑मा॒दद॑ते । आ॒दद॑ते॒ याम् । आ॒दद॑त॒ इत्या᳚ - दद॑ते । याꣳ समा᳚म् । समा॒म् ॅव्यृ॑द्धम् । व्यृ॑द्ध॒मक्षो॑धुकाः । व्यृ॑द्ध॒मिति॒ वि - ऋ॒द्ध॒म् । अक्षो॑धुका॒स्ताम् । ताꣳ समा᳚म् । समा᳚म् प्र॒जाः । प्र॒जा न । प्र॒जा इति॑ प्र - जाः \newline

\textbf{Jatai Paata} \newline

1. अ॒र्क्ये॑ण॒ वै वा अ॒र्क्ये॑णा॒ र्क्ये॑ण॒ वै । \newline
2. वै स॑हस्र॒शः स॑हस्र॒शो वै वै स॑हस्र॒शः । \newline
3. स॒ह॒स्र॒शः प्र॒जाप॑तिः प्र॒जाप॑तिः सहस्र॒शः स॑हस्र॒शः प्र॒जाप॑तिः । \newline
4. स॒ह॒स्र॒श इति॑ सहस्र - शः । \newline
5. प्र॒जाप॑तिः प्र॒जाः प्र॒जाः प्र॒जाप॑तिः प्र॒जाप॑तिः प्र॒जाः । \newline
6. प्र॒जाप॑ति॒रिति॑ प्र॒जा - प॒तिः॒ । \newline
7. प्र॒जा अ॑सृजता सृजत प्र॒जाः प्र॒जा अ॑सृजत । \newline
8. प्र॒जा इति॑ प्र - जाः । \newline
9. अ॒सृ॒ज॒त॒ ताभ्य॒ स्ताभ्यो॑ ऽसृजता सृजत॒ ताभ्यः॑ । \newline
10. ताभ्य॒ इला᳚न्दे॒ नेला᳚न्देन॒ ताभ्य॒ स्ताभ्य॒ इला᳚न्देन । \newline
11. इला᳚न्दे॒ नेरा॒ मिरा॒ मिला᳚न्दे॒ नेला᳚न्दे॒ नेरा᳚म् । \newline
12. इरा॒म् ॅलूता॒म् ॅलूता॒ मिरा॒ मिरा॒म् ॅलूता᳚म् । \newline
13. लूता॒ मवाव॒ लूता॒म् ॅलूता॒ मव॑ । \newline
14. अवा॑ रुन्धा रु॒न्धा वावा॑ रुन्ध । \newline
15. अ॒रु॒न्ध॒ यद् यद॑रुन्धा रुन्ध॒ यत् । \newline
16. यद॒र्क्य॑ म॒र्क्यं॑ ॅयद् यद॒र्क्य᳚म् । \newline
17. अ॒र्क्य॑म् भव॑ति॒ भव॑ त्य॒र्क्य॑ म॒र्क्य॑म् भव॑ति । \newline
18. भव॑ति प्र॒जाः प्र॒जा भव॑ति॒ भव॑ति प्र॒जाः । \newline
19. प्र॒जा ए॒वैव प्र॒जाः प्र॒जा ए॒व । \newline
20. प्र॒जा इति॑ प्र - जाः । \newline
21. ए॒व तत् तदे॒वैव तत् । \newline
22. तद् यज॑माना॒ यज॑माना॒ स्तत् तद् यज॑मानाः । \newline
23. यज॑मानाः सृजन्ते सृजन्ते॒ यज॑माना॒ यज॑मानाः सृजन्ते । \newline
24. सृ॒ज॒न्त॒ इला᳚न्द॒ मिला᳚न्दꣳ सृजन्ते सृजन्त॒ इला᳚न्दम् । \newline
25. इला᳚न्दम् भवति भव॒तीला᳚न्द॒ मिला᳚न्दम् भवति । \newline
26. भ॒व॒ति॒ प्र॒जाभ्यः॑ प्र॒जाभ्यो॑ भवति भवति प्र॒जाभ्यः॑ । \newline
27. प्र॒जाभ्य॑ ए॒वैव प्र॒जाभ्यः॑ प्र॒जाभ्य॑ ए॒व । \newline
28. प्र॒जाभ्य॒ इति॑ प्र - जाभ्यः॑ । \newline
29. ए॒व सृ॒ष्टाभ्यः॑ सृ॒ष्टाभ्य॑ ए॒वैव सृ॒ष्टाभ्यः॑ । \newline
30. सृ॒ष्टाभ्य॒ इरा॒ मिराꣳ॑ सृ॒ष्टाभ्यः॑ सृ॒ष्टाभ्य॒ इरा᳚म् । \newline
31. इरा॒म् ॅलूता॒म् ॅलूता॒ मिरा॒ मिरा॒म् ॅलूता᳚म् । \newline
32. लूता॒ मवाव॒ लूता॒म् ॅलूता॒ मव॑ । \newline
33. अव॑ रुन्धते रुन्ध॒ते ऽवाव॑ रुन्धते । \newline
34. रु॒न्ध॒ते॒ तस्मा॒त् तस्मा᳚द् रुन्धते रुन्धते॒ तस्मा᳚त् । \newline
35. तस्मा॒द् यां ॅयाम् तस्मा॒त् तस्मा॒द् याम् । \newline
36. याꣳ समाꣳ॒॒ समां॒ ॅयां ॅयाꣳ समा᳚म् । \newline
37. समाꣳ॑ स॒त्रꣳ स॒त्रꣳ समाꣳ॒॒ समाꣳ॑ स॒त्रम् । \newline
38. स॒त्रꣳ समृ॑द्धꣳ॒॒ समृ॑द्धꣳ स॒त्रꣳ स॒त्रꣳ समृ॑द्धम् । \newline
39. समृ॑द्ध॒म् क्षोधु॑काः॒ क्षोधु॑काः॒ समृ॑द्धꣳ॒॒ समृ॑द्ध॒म् क्षोधु॑काः । \newline
40. समृ॑द्ध॒मिति॒ सं - ऋ॒द्ध॒म् । \newline
41. क्षोधु॑का॒ स्ताम् ताम् क्षोधु॑काः॒ क्षोधु॑का॒ स्ताम् । \newline
42. ताꣳ समाꣳ॒॒ समा॒म् ताम् ताꣳ समा᳚म् । \newline
43. समा᳚म् प्र॒जाः प्र॒जाः समाꣳ॒॒ समा᳚म् प्र॒जाः । \newline
44. प्र॒जा इष॒ मिष॑म् प्र॒जाः प्र॒जा इष᳚म् । \newline
45. प्र॒जा इति॑ प्र - जाः । \newline
46. इषꣳ॒॒ हि हीष॒ मिषꣳ॒॒ हि । \newline
47. ह्या॑सा मासाꣳ॒॒ हि ह्या॑साम् । \newline
48. आ॒सा॒ मूर्ज॒ मूर्ज॑ मासा मासा॒ मूर्ज᳚म् । \newline
49. ऊर्ज॑ मा॒दद॑त आ॒दद॑त॒ ऊर्ज॒ मूर्ज॑ मा॒दद॑ते । \newline
50. आ॒दद॑ते॒ यां ॅया मा॒दद॑त आ॒दद॑ते॒ याम् । \newline
51. आ॒दद॑त॒ इत्या᳚ - दद॑ते । \newline
52. याꣳ समाꣳ॒॒ समां॒ ॅयां ॅयाꣳ समा᳚म् । \newline
53. समां॒ ॅव्यृ॑द्धं॒ ॅव्यृ॑द्धꣳ॒॒ समाꣳ॒॒ समां॒ ॅव्यृ॑द्धम् । \newline
54. व्यृ॑द्ध॒ मक्षो॑धुका॒ अक्षो॑धुका॒ व्यृ॑द्धं॒ ॅव्यृ॑द्ध॒ मक्षो॑धुकाः । \newline
55. व्यृ॑द्ध॒मिति॒ वि - ऋ॒द्ध॒म् । \newline
56. अक्षो॑धुका॒ स्ताम् ता मक्षो॑धुका॒ अक्षो॑धुका॒ स्ताम् । \newline
57. ताꣳ समाꣳ॒॒ समा॒म् ताम् ताꣳ समा᳚म् । \newline
58. समा᳚म् प्र॒जाः प्र॒जाः समाꣳ॒॒ समा᳚म् प्र॒जाः । \newline
59. प्र॒जा न न प्र॒जाः प्र॒जा न । \newline
60. प्र॒जा इति॑ प्र - जाः । \newline

\textbf{Ghana Paata } \newline

1. अ॒र्क्ये॑ण॒ वै वा अ॒र्क्ये॑ णा॒र्क्ये॑ण॒ वै स॑हस्र॒शः स॑हस्र॒शो वा अ॒र्क्ये॑ णा॒र्क्ये॑ण॒ वै स॑हस्र॒शः । \newline
2. वै स॑हस्र॒शः स॑हस्र॒शो वै वै स॑हस्र॒शः प्र॒जाप॑तिः प्र॒जाप॑तिः सहस्र॒शो वै वै स॑हस्र॒शः प्र॒जाप॑तिः । \newline
3. स॒ह॒स्र॒शः प्र॒जाप॑तिः प्र॒जाप॑तिः सहस्र॒शः स॑हस्र॒शः प्र॒जाप॑तिः प्र॒जाः प्र॒जाः प्र॒जाप॑तिः सहस्र॒शः स॑हस्र॒शः प्र॒जाप॑तिः प्र॒जाः । \newline
4. स॒ह॒स्र॒श इति॑ सहस्र - शः । \newline
5. प्र॒जाप॑तिः प्र॒जाः प्र॒जाः प्र॒जाप॑तिः प्र॒जाप॑तिः प्र॒जा अ॑सृजता सृजत प्र॒जाः प्र॒जाप॑तिः प्र॒जाप॑तिः प्र॒जा अ॑सृजत । \newline
6. प्र॒जाप॑ति॒रिति॑ प्र॒जा - प॒तिः॒ । \newline
7. प्र॒जा अ॑सृजता सृजत प्र॒जाः प्र॒जा अ॑सृजत॒ ताभ्य॒ स्ताभ्यो॑ ऽसृजत प्र॒जाः प्र॒जा अ॑सृजत॒ ताभ्यः॑ । \newline
8. प्र॒जा इति॑ प्र - जाः । \newline
9. अ॒सृ॒ज॒त॒ ताभ्य॒ स्ताभ्यो॑ ऽसृजता सृजत॒ ताभ्य॒ इला᳚न्दे॒ नेला᳚न्देन॒ ताभ्यो॑ ऽसृजता सृजत॒ ताभ्य॒ इला᳚न्देन । \newline
10. ताभ्य॒ इला᳚न्दे॒ नेला᳚न्देन॒ ताभ्य॒ स्ताभ्य॒ इला᳚न्दे॒ नेरा॒ मिरा॒ मिला᳚न्देन॒ ताभ्य॒ स्ताभ्य॒ इला᳚न्दे॒ नेरा᳚म् । \newline
11. इला᳚न्दे॒ नेरा॒ मिरा॒ मिला᳚न्दे॒ नेला᳚न्दे॒ नेरा॒म् ॅलूता॒म् ॅलूता॒ मिरा॒ मिला᳚न्दे॒ नेला᳚न्दे॒ नेरा॒म् ॅलूता᳚म् । \newline
12. इरा॒म् ॅलूता॒म् ॅलूता॒ मिरा॒ मिरा॒म् ॅलूता॒ मवाव॒ लूता॒ मिरा॒ मिरा॒म् ॅलूता॒ मव॑ । \newline
13. लूता॒ मवाव॒ लूता॒म् ॅलूता॒ मवा॑रुन्धा रु॒न्धाव॒ लूता॒म् ॅलूता॒ मवा॑रुन्ध । \newline
14. अवा॑ रुन्धा रु॒न्धा वावा॑ रुन्ध॒ यद् यद॑ रु॒न्धा वावा॑ रुन्ध॒ यत् । \newline
15. अ॒रु॒न्ध॒ यद् यद॑ रुन्धा रुन्ध॒ यद॒र्क्य॑ म॒र्क्यं॑ ॅयद॑ रुन्धा रुन्ध॒ यद॒र्क्य᳚म् । \newline
16. यद॒र्क्य॑ म॒र्क्यं॑ ॅयद् यद॒र्क्य॑म् भव॑ति॒ भव॑त्य॒र्क्यं॑ ॅयद् यद॒र्क्य॑म् भव॑ति । \newline
17. अ॒र्क्य॑म् भव॑ति॒ भव॑ त्य॒र्क्य॑ म॒र्क्य॑म् भव॑ति प्र॒जाः प्र॒जा भव॑ त्य॒र्क्य॑ म॒र्क्य॑म् भव॑ति प्र॒जाः । \newline
18. भव॑ति प्र॒जाः प्र॒जा भव॑ति॒ भव॑ति प्र॒जा ए॒वैव प्र॒जा भव॑ति॒ भव॑ति प्र॒जा ए॒व । \newline
19. प्र॒जा ए॒वैव प्र॒जाः प्र॒जा ए॒व तत् तदे॒व प्र॒जाः प्र॒जा ए॒व तत् । \newline
20. प्र॒जा इति॑ प्र - जाः । \newline
21. ए॒व तत् तदे॒वैव तद् यज॑माना॒ यज॑माना॒ स्तदे॒वैव तद् यज॑मानाः । \newline
22. तद् यज॑माना॒ यज॑माना॒ स्तत् तद् यज॑मानाः सृजन्ते सृजन्ते॒ यज॑माना॒ स्तत् तद् यज॑मानाः सृजन्ते । \newline
23. यज॑मानाः सृजन्ते सृजन्ते॒ यज॑माना॒ यज॑मानाः सृजन्त॒ इला᳚न्द॒ मिला᳚न्दꣳ सृजन्ते॒ यज॑माना॒ यज॑मानाः सृजन्त॒ इला᳚न्दम् । \newline
24. सृ॒ज॒न्त॒ इला᳚न्द॒ मिला᳚न्दꣳ सृजन्ते सृजन्त॒ इला᳚न्दम् भवति भव॒ती ला᳚न्दꣳ सृजन्ते सृजन्त॒ इला᳚न्दम् भवति । \newline
25. इला᳚न्दम् भवति भव॒ती ला᳚न्द॒ मिला᳚न्दम् भवति प्र॒जाभ्यः॑ प्र॒जाभ्यो॑ भव॒ती ला᳚न्द॒ मिला᳚न्दम् भवति प्र॒जाभ्यः॑ । \newline
26. भ॒व॒ति॒ प्र॒जाभ्यः॑ प्र॒जाभ्यो॑ भवति भवति प्र॒जाभ्य॑ ए॒वैव प्र॒जाभ्यो॑ भवति भवति प्र॒जाभ्य॑ ए॒व । \newline
27. प्र॒जाभ्य॑ ए॒वैव प्र॒जाभ्यः॑ प्र॒जाभ्य॑ ए॒व सृ॒ष्टाभ्यः॑ सृ॒ष्टाभ्य॑ ए॒व प्र॒जाभ्यः॑ प्र॒जाभ्य॑ ए॒व सृ॒ष्टाभ्यः॑ । \newline
28. प्र॒जाभ्य॒ इति॑ प्र - जाभ्यः॑ । \newline
29. ए॒व सृ॒ष्टाभ्यः॑ सृ॒ष्टाभ्य॑ ए॒वैव सृ॒ष्टाभ्य॒ इरा॒ मिराꣳ॑ सृ॒ष्टाभ्य॑ ए॒वैव सृ॒ष्टाभ्य॒ इरा᳚म् । \newline
30. सृ॒ष्टाभ्य॒ इरा॒ मिराꣳ॑ सृ॒ष्टाभ्यः॑ सृ॒ष्टाभ्य॒ इरा॒म् ॅलूता॒म् ॅलूता॒ मिराꣳ॑ सृ॒ष्टाभ्यः॑ सृ॒ष्टाभ्य॒ इरा॒म् ॅलूता᳚म् । \newline
31. इरा॒म् ॅलूता॒म् ॅलूता॒ मिरा॒ मिरा॒म् ॅलूता॒ मवाव॒ लूता॒ मिरा॒ मिरा॒म् ॅलूता॒ मव॑ । \newline
32. लूता॒ मवाव॒ लूता॒म् ॅलूता॒ मव॑ रुन्धते रुन्ध॒ते ऽव॒ लूता॒म् ॅलूता॒ मव॑ रुन्धते । \newline
33. अव॑ रुन्धते रुन्ध॒ते ऽवाव॑ रुन्धते॒ तस्मा॒त् तस्मा᳚द् रुन्ध॒ते ऽवाव॑ रुन्धते॒ तस्मा᳚त् । \newline
34. रु॒न्ध॒ते॒ तस्मा॒त् तस्मा᳚द् रुन्धते रुन्धते॒ तस्मा॒द् यां ॅयाम् तस्मा᳚द् रुन्धते रुन्धते॒ तस्मा॒द् याम् । \newline
35. तस्मा॒द् यां ॅयाम् तस्मा॒त् तस्मा॒द् याꣳ समाꣳ॒॒ समां॒ ॅयाम् तस्मा॒त् तस्मा॒द् याꣳ समा᳚म् । \newline
36. याꣳ समाꣳ॒॒ समां॒ ॅयां ॅयाꣳ समाꣳ॑ स॒त्रꣳ स॒त्रꣳ समां॒ ॅयां ॅयाꣳ समाꣳ॑ स॒त्रम् । \newline
37. समाꣳ॑ स॒त्रꣳ स॒त्रꣳ समाꣳ॒॒ समाꣳ॑ स॒त्रꣳ समृ॑द्धꣳ॒॒ समृ॑द्धꣳ स॒त्रꣳ समाꣳ॒॒ समाꣳ॑ स॒त्रꣳ समृ॑द्धम् । \newline
38. स॒त्रꣳ समृ॑द्धꣳ॒॒ समृ॑द्धꣳ स॒त्रꣳ स॒त्रꣳ समृ॑द्ध॒म् क्षोधु॑काः॒ क्षोधु॑काः॒ समृ॑द्धꣳ स॒त्रꣳ स॒त्रꣳ समृ॑द्ध॒म् क्षोधु॑काः । \newline
39. समृ॑द्ध॒म् क्षोधु॑काः॒ क्षोधु॑काः॒ समृ॑द्धꣳ॒॒ समृ॑द्ध॒म् क्षोधु॑का॒ स्ताम् ताम् क्षोधु॑काः॒ समृ॑द्धꣳ॒॒ समृ॑द्ध॒म् क्षोधु॑का॒ स्ताम् । \newline
40. समृ॑द्ध॒मिति॒ सं - ऋ॒द्ध॒म् । \newline
41. क्षोधु॑का॒ स्ताम् ताम् क्षोधु॑काः॒ क्षोधु॑का॒ स्ताꣳ समाꣳ॒॒ समा॒म् ताम् क्षोधु॑काः॒ क्षोधु॑का॒ स्ताꣳ समा᳚म् । \newline
42. ताꣳ समाꣳ॒॒ समा॒म् ताम् ताꣳ समा᳚म् प्र॒जाः प्र॒जाः समा॒म् ताम् ताꣳ समा᳚म् प्र॒जाः । \newline
43. समा᳚म् प्र॒जाः प्र॒जाः समाꣳ॒॒ समा᳚म् प्र॒जा इष॒ मिष॑म् प्र॒जाः समाꣳ॒॒ समा᳚म् प्र॒जा इष᳚म् । \newline
44. प्र॒जा इष॒ मिष॑म् प्र॒जाः प्र॒जा इषꣳ॒॒ हि हीष॑म् प्र॒जाः प्र॒जा इषꣳ॒॒ हि । \newline
45. प्र॒जा इति॑ प्र - जाः । \newline
46. इषꣳ॒॒ हि हीष॒ मिषꣳ॒॒ ह्या॑सा मासाꣳ॒॒ हीष॒ मिषꣳ॒॒ ह्या॑साम् । \newline
47. ह्या॑सा मासाꣳ॒॒ हि ह्या॑सा॒ मूर्ज॒ मूर्ज॑ मासाꣳ॒॒ हि ह्या॑सा॒ मूर्ज᳚म् । \newline
48. आ॒सा॒ मूर्ज॒ मूर्ज॑ मासा मासा॒ मूर्ज॑ मा॒दद॑त आ॒दद॑त॒ ऊर्ज॑ मासा मासा॒ मूर्ज॑ मा॒दद॑ते । \newline
49. ऊर्ज॑ मा॒दद॑त आ॒दद॑त॒ ऊर्ज॒ मूर्ज॑ मा॒दद॑ते॒ यां ॅया मा॒दद॑त॒ ऊर्ज॒ मूर्ज॑ मा॒दद॑ते॒ याम् । \newline
50. आ॒दद॑ते॒ यां ॅया मा॒दद॑त आ॒दद॑ते॒ याꣳ समाꣳ॒॒ समां॒ ॅया मा॒दद॑त आ॒दद॑ते॒ याꣳ समा᳚म् । \newline
51. आ॒दद॑त॒ इत्या᳚ - दद॑ते । \newline
52. याꣳ समाꣳ॒॒ समां॒ ॅयां ॅयाꣳ समां॒ ॅव्यृ॑द्धं॒ ॅव्यृ॑द्धꣳ॒॒ समां॒ ॅयां ॅयाꣳ समां॒ ॅव्यृ॑द्धम् । \newline
53. समां॒ ॅव्यृ॑द्धं॒ ॅव्यृ॑द्धꣳ॒॒ समाꣳ॒॒ समां॒ ॅव्यृ॑द्ध॒ मक्षो॑धुका॒ अक्षो॑धुका॒ व्यृ॑द्धꣳ॒॒ समाꣳ॒॒ समां॒ ॅव्यृ॑द्ध॒ मक्षो॑धुकाः । \newline
54. व्यृ॑द्ध॒ मक्षो॑धुका॒ अक्षो॑धुका॒ व्यृ॑द्धं॒ ॅव्यृ॑द्ध॒ मक्षो॑धुका॒ स्ताम् ता मक्षो॑धुका॒ व्यृ॑द्धं॒ ॅव्यृ॑द्ध॒ मक्षो॑धुका॒ स्ताम् । \newline
55. व्यृ॑द्ध॒मिति॒ वि - ऋ॒द्ध॒म् । \newline
56. अक्षो॑धुका॒ स्ताम् ता मक्षो॑धुका॒ अक्षो॑धुका॒ स्ताꣳ समाꣳ॒॒ समा॒म् ता मक्षो॑धुका॒ अक्षो॑धुका॒ स्ताꣳ समा᳚म् । \newline
57. ताꣳ समाꣳ॒॒ समा॒म् ताम् ताꣳ समा᳚म् प्र॒जाः प्र॒जाः समा॒म् ताम् ताꣳ समा᳚म् प्र॒जाः । \newline
58. समा᳚म् प्र॒जाः प्र॒जाः समाꣳ॒॒ समा᳚म् प्र॒जा न न प्र॒जाः समाꣳ॒॒ समा᳚म् प्र॒जा न । \newline
59. प्र॒जा न न प्र॒जाः प्र॒जा न हि हि न प्र॒जाः प्र॒जा न हि । \newline
60. प्र॒जा इति॑ प्र - जाः । \newline
\pagebreak
\markright{ TS 7.5.9.2  \hfill https://www.vedavms.in \hfill}

\section{ TS 7.5.9.2 }

\textbf{TS 7.5.9.2 } \newline
\textbf{Samhita Paata} \newline

न ह्या॑सा॒मिष॒मूर्ज॑मा॒दद॑त उत्क्रो॒दं कु॑र्वते॒ यथा॑ ब॒न्धान्-मु॑मुचा॒ना उ॑त्क्रो॒दं कु॒र्वत॑ ए॒वमे॒व तद्-यज॑माना देवब॒न्धान्-मु॑मुचा॒ना उ॑त्क्रो॒दं कु॑र्वत॒ इष॒मूर्ज॑मा॒त्मन् दधा॑ना वा॒णः श॒तत॑न्तुर्भवति श॒तायुः॒ पुरु॑षः श॒तेन्द्रि॑य॒ आयु॑ष्ये॒वेन्द्रि॒ये प्रति॑ तिष्ठन्त्या॒जिं धा॑व॒न्त्यन॑भिजितस्या॒-भिजि॑त्यै दुन्दु॒भीन्थ् स॒माघ्न॑न्ति पर॒मा वा ए॒षा वाग्या दु॑न्दु॒भौ प॑र॒मामे॒व - [  ] \newline

\textbf{Pada Paata} \newline

न । हि । आ॒सा॒म् । इष᳚म् । ऊर्ज᳚म् । आ॒दद॑त॒ इत्या᳚ - दद॑ते । उ॒त्क्रो॒दमित्यु॑त् - क्रो॒दम् । कु॒र्व॒ते॒ । यथा᳚ । ब॒न्धात् । मु॒मु॒चा॒नाः । उ॒त्क्रो॒दमित्यु॑त् - क्रो॒दम् । कु॒र्वते᳚ । ए॒वम् । ए॒व । तत् । यज॑मानाः । दे॒व॒ब॒न्धादिति॑ देव-ब॒न्धात् । मु॒मु॒चा॒नाः । उ॒त्क्रो॒दमित्यु॑त् - क्रो॒दम् । कु॒र्व॒ते॒ । इष᳚म् । ऊर्ज᳚म् । आ॒त्मन्न् । दधा॑नाः । वा॒णः । श॒तत॑न्तु॒रिति॑ श॒त - त॒न्तुः॒ । भ॒व॒ति॒ । श॒तायु॒रिति॑ श॒त - आ॒युः॒ । पुरु॑षः । श॒तेन्द्रि॑य॒ इति॑ श॒त - इ॒न्द्रि॒यः॒ । आयु॑षि । ए॒व । इ॒न्द्रि॒ये । प्रतीति॑ । ति॒ष्ठ॒न्ति॒ । आ॒जिम् । धा॒व॒न्ति॒ । अन॑भिजित॒स्येत्यन॑भि - जि॒त॒स्य॒ । अ॒भिजि॑त्या॒ इत्य॒भि - जि॒त्यै॒ । दु॒न्दु॒भीन् । स॒माघ्न॒न्तीति॑ सं - आघ्न॑न्ति । प॒र॒मा । वै । ए॒षा । वाक् । या । दु॒न्दु॒भौ । प॒र॒माम् । ए॒व ।  \newline


\textbf{Krama Paata} \newline

न हि । ह्या॑साम् । आ॒सा॒मिष᳚म् । इष॒मूर्ज᳚म् । ऊर्ज॑मा॒दद॑ते । आ॒दद॑त उत्क्रो॒दम् । आ॒दद॑त॒ इत्या᳚ - दद॑ते । उ॒त्क्रो॒दम् कु॑र्वते । उ॒त्क्रो॒दमित्यु॑त् - क्रो॒दम् । कु॒र्व॒ते॒ यथा᳚ । यथा॑ ब॒न्धात् । ब॒न्धान् मु॑मुचा॒नाः । मु॒मु॒चा॒ना उ॑त्क्रो॒दम् । उ॒त्क्रो॒दम् कु॒र्वते᳚ । उ॒त्क्रो॒दमित्यु॑त् - क्रो॒दम् । कु॒र्वत॑ ए॒वम् । ए॒वमे॒व । ए॒व तत् । 
तद् यज॑मानाः । यज॑माना देवब॒न्धात् । दे॒व॒ब॒न्धान् मु॑मुचा॒नाः । दे॒व॒ब॒न्धादिति॑ देव - ब॒न्धात् । मु॒मु॒चा॒ना उ॑त्क्रो॒दम् । उ॒त्क्रो॒दम् कु॑र्वते । उ॒त्क्रो॒दमित्यु॑त् - क्रो॒दम् । कु॒र्व॒त॒ इष᳚म् । इष॒मूर्ज᳚म् । उर्ज॑मा॒त्मन्न् । आ॒त्मन् दधा॑नाः । दधा॑ना वा॒णः । वा॒णः श॒तत॑न्तुः । श॒तत॑न्तुर् भवति । श॒तत॑न्तु॒रिति॑ श॒त - त॒न्तुः॒ । भ॒व॒ति॒ श॒तायुः॑ । श॒तायुः॒ पुरु॑षः । श॒तायु॒रिति॑ श॒त - आ॒युः॒ । पुरु॑षः श॒तेन्द्रि॑यः । श॒तेन्द्रि॑य॒ आयु॑षि । श॒तेन्द्रि॑य॒ इति॑ श॒त - इ॒न्द्रि॒यः॒ । आयु॑ष्ये॒व । ए॒वेन्द्रि॒ये । इ॒न्द्रि॒ये प्रति॑ । प्रति॑ तिष्ठन्ति । ति॒ष्ठ॒न्त्या॒जिम् । आ॒जिम् धा॑वन्ति । धा॒व॒न्त्यन॑भिजितस्य । अन॑भिजितस्या॒भिजि॑त्यै । अन॑भिजित॒स्येत्यन॑भि - जि॒त॒स्य॒ । अ॒भिजि॑त्यै दुन्दु॒भीन् । अ॒भिजि॑त्या॒ इत्य॒भि - जि॒त्यै॒ । दु॒न्दु॒भीन्थ् स॒माघ्न॑न्ति । स॒माघ्न॑न्ति पर॒मा । स॒माघ्न॒न्तीति॑ सम् - आघ्न॑न्ति । प॒र॒मा वै । वा ए॒षा । ए॒षा वाक् । वाग् या । या दु॑न्दु॒भौ । दु॒न्दु॒भौ प॑र॒माम् । प॒र॒मामे॒व । ए॒व वाच᳚म् \newline

\textbf{Jatai Paata} \newline

1. न हि हि न न हि । \newline
2. ह्या॑सा मासाꣳ॒॒ हि ह्या॑साम् । \newline
3. आ॒सा॒ मिष॒ मिष॑ मासा मासा॒ मिष᳚म् । \newline
4. इष॒ मूर्ज॒ मूर्ज॒ मिष॒ मिष॒ मूर्ज᳚म् । \newline
5. ऊर्ज॑ मा॒दद॑त आ॒दद॑त॒ ऊर्ज॒ मूर्ज॑ मा॒दद॑ते । \newline
6. आ॒दद॑त उत्क्रो॒द मु॑त्क्रो॒द मा॒दद॑त आ॒दद॑त उत्क्रो॒दम् । \newline
7. आ॒दद॑त॒ इत्या᳚ - दद॑ते । \newline
8. उ॒त्क्रो॒दम् कु॑र्वते कुर्वत उत्क्रो॒द मु॑त्क्रो॒दम् कु॑र्वते । \newline
9. उ॒त्क्रो॒दमित्यु॑त् - क्रो॒दम् । \newline
10. कु॒र्व॒ते॒ यथा॒ यथा॑ कुर्वते कुर्वते॒ यथा᳚ । \newline
11. यथा॑ ब॒न्धाद् ब॒न्धाद् यथा॒ यथा॑ ब॒न्धात् । \newline
12. ब॒न्धान् मु॑मुचा॒ना मु॑मुचा॒ना ब॒न्धाद् ब॒न्धान् मु॑मुचा॒नाः । \newline
13. मु॒मु॒चा॒ना उ॑त्क्रो॒द मु॑त्क्रो॒दम् मु॑मुचा॒ना मु॑मुचा॒ना उ॑त्क्रो॒दम् । \newline
14. उ॒त्क्रो॒दम् कु॒र्वते॑ कु॒र्वत॑ उत्क्रो॒द मु॑त्क्रो॒दम् कु॒र्वते᳚ । \newline
15. उ॒त्क्रो॒दमित्यु॑त् - क्रो॒दम् । \newline
16. कु॒र्वत॑ ए॒व मे॒वम् कु॒र्वते॑ कु॒र्वत॑ ए॒वम् । \newline
17. ए॒व मे॒वै वैव मे॒व मे॒व । \newline
18. ए॒व तत् तदे॒वैव तत् । \newline
19. तद् यज॑माना॒ यज॑माना॒ स्तत् तद् यज॑मानाः । \newline
20. यज॑माना देवब॒न्धाद् दे॑वब॒न्धाद् यज॑माना॒ यज॑माना देवब॒न्धात् । \newline
21. दे॒व॒ब॒न्धान् मु॑मुचा॒ना मु॑मुचा॒ना दे॑वब॒न्धाद् दे॑वब॒न्धान् मु॑मुचा॒नाः । \newline
22. दे॒व॒ब॒न्धादिति॑ देव - ब॒न्धात् । \newline
23. मु॒मु॒चा॒ना उ॑त्क्रो॒द मु॑त्क्रो॒दम् मु॑मुचा॒ना मु॑मुचा॒ना उ॑त्क्रो॒दम् । \newline
24. उ॒त्क्रो॒दम् कु॑र्वते कुर्वत उत्क्रो॒द मु॑त्क्रो॒दम् कु॑र्वते । \newline
25. उ॒त्क्रो॒दमित्यु॑त् - क्रो॒दम् । \newline
26. कु॒र्व॒त॒ इष॒ मिष॑म् कुर्वते कुर्वत॒ इष᳚म् । \newline
27. इष॒ मूर्ज॒ मूर्ज॒ मिष॒ मिष॒ मूर्ज᳚म् । \newline
28. ऊर्ज॑ मा॒त्मन् ना॒त्मन् नूर्ज॒ मूर्ज॑ मा॒त्मन्न् । \newline
29. आ॒त्मन् दधा॑ना॒ दधा॑ना आ॒त्मन् ना॒त्मन् दधा॑नाः । \newline
30. दधा॑ना वा॒णो वा॒णो दधा॑ना॒ दधा॑ना वा॒णः । \newline
31. वा॒णः श॒तत॑न्तुः श॒तत॑न्तुर् वा॒णो वा॒णः श॒तत॑न्तुः । \newline
32. श॒तत॑न्तुर् भवति भवति श॒तत॑न्तुः श॒तत॑न्तुर् भवति । \newline
33. श॒तत॑न्तु॒रिति॑ श॒त - त॒न्तुः॒ । \newline
34. भ॒व॒ति॒ श॒तायुः॑ श॒तायु॑र् भवति भवति श॒तायुः॑ । \newline
35. श॒तायुः॒ पुरु॑षः॒ पुरु॑षः श॒तायुः॑ श॒तायुः॒ पुरु॑षः । \newline
36. श॒तायु॒रिति॑ श॒त - आ॒युः॒ । \newline
37. पुरु॑षः श॒तेन्द्रि॑यः श॒तेन्द्रि॑यः॒ पुरु॑षः॒ पुरु॑षः श॒तेन्द्रि॑यः । \newline
38. श॒तेन्द्रि॑य॒ आयु॒ ष्यायु॑षि श॒तेन्द्रि॑यः श॒तेन्द्रि॑य॒ आयु॑षि । \newline
39. श॒तेन्द्रि॑य॒ इति॑ श॒त - इ॒न्द्रि॒यः॒ । \newline
40. आयु॑ ष्ये॒वै वायु॒ ष्यायु॑ ष्ये॒व । \newline
41. ए॒वेन्द्रि॒य इ॑न्द्रि॒य ए॒वैवेन्द्रि॒ये । \newline
42. इ॒न्द्रि॒ये प्रति॒ प्रती᳚न्द्रि॒य इ॑न्द्रि॒ये प्रति॑ । \newline
43. प्रति॑ तिष्ठन्ति तिष्ठन्ति॒ प्रति॒ प्रति॑ तिष्ठन्ति । \newline
44. ति॒ष्ठ॒ न्त्या॒जि मा॒जिम् ति॑ष्ठन्ति तिष्ठ न्त्या॒जिम् । \newline
45. आ॒जिम् धा॑वन्ति धाव न्त्या॒जि मा॒जिम् धा॑वन्ति । \newline
46. धा॒व॒ न्त्यन॑भिजित॒स्या न॑भिजितस्य धावन्ति धाव॒ न्त्यन॑भिजितस्य । \newline
47. अन॑भिजितस्या॒ भिजि॑त्या अ॒भिजि॑त्या॒ अन॑भिजित॒स्या न॑भिजितस्या॒ भिजि॑त्यै । \newline
48. अन॑भिजित॒स्येत्यन॑भि - जि॒त॒स्य॒ । \newline
49. अ॒भिजि॑त्यै दुन्दु॒भीन् दु॑न्दु॒भी न॒भिजि॑त्या अ॒भिजि॑त्यै दुन्दु॒भीन् । \newline
50. अ॒भिजि॑त्या॒ इत्य॒भि - जि॒त्यै॒ । \newline
51. दु॒न्दु॒भीन् थ्स॒माघ्न॑न्ति स॒माघ्न॑न्ति दुन्दु॒भीन् दु॑न्दु॒भीन् थ्स॒माघ्न॑न्ति । \newline
52. स॒माघ्न॑न्ति पर॒मा प॑र॒मा स॒माघ्न॑न्ति स॒माघ्न॑न्ति पर॒मा । \newline
53. स॒माघ्न॒न्तीति॑ सं - आघ्न॑न्ति । \newline
54. प॒र॒मा वै वै प॑र॒मा प॑र॒मा वै । \newline
55. वा ए॒षैषा वै वा ए॒षा । \newline
56. ए॒षा वाग् वागे॒षैषा वाक् । \newline
57. वाग् या या वाग् वाग् या । \newline
58. या दु॑न्दु॒भौ दु॑न्दु॒भौ या या दु॑न्दु॒भौ । \newline
59. दु॒न्दु॒भौ प॑र॒माम् प॑र॒माम् दु॑न्दु॒भौ दु॑न्दु॒भौ प॑र॒माम् । \newline
60. प॒र॒मा मे॒वैव प॑र॒माम् प॑र॒मा मे॒व । \newline
61. ए॒व वाचं॒ ॅवाच॑ मे॒वैव वाच᳚म् । \newline

\textbf{Ghana Paata } \newline

1. न हि हि न न ह्या॑सा मासाꣳ॒॒ हि न न ह्या॑साम् । \newline
2. ह्या॑सा मासाꣳ॒॒ हि ह्या॑सा॒ मिष॒ मिष॑ मासाꣳ॒॒ हि ह्या॑सा॒ मिष᳚म् । \newline
3. आ॒सा॒ मिष॒ मिष॑ मासा मासा॒ मिष॒ मूर्ज॒ मूर्ज॒ मिष॑ मासा मासा॒ मिष॒ मूर्ज᳚म् । \newline
4. इष॒ मूर्ज॒ मूर्ज॒ मिष॒ मिष॒ मूर्ज॑ मा॒दद॑त आ॒दद॑त॒ ऊर्ज॒ मिष॒ मिष॒ मूर्ज॑ मा॒दद॑ते । \newline
5. ऊर्ज॑ मा॒दद॑त आ॒दद॑त॒ ऊर्ज॒ मूर्ज॑ मा॒दद॑त उत्क्रो॒द मु॑त्क्रो॒द मा॒दद॑त॒ ऊर्ज॒ मूर्ज॑ मा॒दद॑त उत्क्रो॒दम् । \newline
6. आ॒दद॑त उत्क्रो॒द मु॑त्क्रो॒द मा॒दद॑त आ॒दद॑त उत्क्रो॒दम् कु॑र्वते कुर्वत उत्क्रो॒द मा॒दद॑त आ॒दद॑त उत्क्रो॒दम् कु॑र्वते । \newline
7. आ॒दद॑त॒ इत्या᳚ - दद॑ते । \newline
8. उ॒त्क्रो॒दम् कु॑र्वते कुर्वत उत्क्रो॒द मु॑त्क्रो॒दम् कु॑र्वते॒ यथा॒ यथा॑ कुर्वत उत्क्रो॒द मु॑त्क्रो॒दम् कु॑र्वते॒ यथा᳚ । \newline
9. उ॒त्क्रो॒दमित्यु॑त् - क्रो॒दम् । \newline
10. कु॒र्व॒ते॒ यथा॒ यथा॑ कुर्वते कुर्वते॒ यथा॑ ब॒न्धाद् ब॒न्धाद् यथा॑ कुर्वते कुर्वते॒ यथा॑ ब॒न्धात् । \newline
11. यथा॑ ब॒न्धाद् ब॒न्धाद् यथा॒ यथा॑ ब॒न्धान् मु॑मुचा॒ना मु॑मुचा॒ना ब॒न्धाद् यथा॒ यथा॑ ब॒न्धान् मु॑मुचा॒नाः । \newline
12. ब॒न्धान् मु॑मुचा॒ना मु॑मुचा॒ना ब॒न्धाद् ब॒न्धान् मु॑मुचा॒ना उ॑त्क्रो॒द मु॑त्क्रो॒दम् मु॑मुचा॒ना ब॒न्धाद् ब॒न्धान् मु॑मुचा॒ना उ॑त्क्रो॒दम् । \newline
13. मु॒मु॒चा॒ना उ॑त्क्रो॒द मु॑त्क्रो॒दम् मु॑मुचा॒ना मु॑मुचा॒ना उ॑त्क्रो॒दम् कु॒र्वते॑ कु॒र्वत॑ उत्क्रो॒दम् मु॑मुचा॒ना मु॑मुचा॒ना उ॑त्क्रो॒दम् कु॒र्वते᳚ । \newline
14. उ॒त्क्रो॒दम् कु॒र्वते॑ कु॒र्वत॑ उत्क्रो॒द मु॑त्क्रो॒दम् कु॒र्वत॑ ए॒व मे॒वम् कु॒र्वत॑ उत्क्रो॒द मु॑त्क्रो॒दम् कु॒र्वत॑ ए॒वम् । \newline
15. उ॒त्क्रो॒दमित्यु॑त् - क्रो॒दम् । \newline
16. कु॒र्वत॑ ए॒व मे॒वम् कु॒र्वते॑ कु॒र्वत॑ ए॒व मे॒वै वैवम् कु॒र्वते॑ कु॒र्वत॑ ए॒व मे॒व । \newline
17. ए॒व मे॒वै वैव मे॒व मे॒व तत् तदे॒वैव मे॒व मे॒व तत् । \newline
18. ए॒व तत् तदे॒वैव तद् यज॑माना॒ यज॑माना॒ स्तदे॒वैव तद् यज॑मानाः । \newline
19. तद् यज॑माना॒ यज॑माना॒ स्तत् तद् यज॑माना देवब॒न्धाद् दे॑वब॒न्धाद् यज॑माना॒ स्तत् तद् यज॑माना देवब॒न्धात् । \newline
20. यज॑माना देवब॒न्धाद् दे॑वब॒न्धाद् यज॑माना॒ यज॑माना देवब॒न्धान् मु॑मुचा॒ना मु॑मुचा॒ना दे॑वब॒न्धाद् यज॑माना॒ यज॑माना देवब॒न्धान् मु॑मुचा॒नाः । \newline
21. दे॒व॒ब॒न्धान् मु॑मुचा॒ना मु॑मुचा॒ना दे॑वब॒न्धाद् दे॑वब॒न्धान् मु॑मुचा॒ना उ॑त्क्रो॒द मु॑त्क्रो॒दम् मु॑मुचा॒ना दे॑वब॒न्धाद् दे॑वब॒न्धान् मु॑मुचा॒ना उ॑त्क्रो॒दम् । \newline
22. दे॒व॒ब॒न्धादिति॑ देव - ब॒न्धात् । \newline
23. मु॒मु॒चा॒ना उ॑त्क्रो॒द मु॑त्क्रो॒दम् मु॑मुचा॒ना मु॑मुचा॒ना उ॑त्क्रो॒दम् कु॑र्वते कुर्वत उत्क्रो॒दम् मु॑मुचा॒ना मु॑मुचा॒ना उ॑त्क्रो॒दम् कु॑र्वते । \newline
24. उ॒त्क्रो॒दम् कु॑र्वते कुर्वत उत्क्रो॒द मु॑त्क्रो॒दम् कु॑र्वत॒ इष॒ मिष॑म् कुर्वत उत्क्रो॒द मु॑त्क्रो॒दम् कु॑र्वत॒ इष᳚म् । \newline
25. उ॒त्क्रो॒दमित्यु॑त् - क्रो॒दम् । \newline
26. कु॒र्व॒त॒ इष॒ मिष॑म् कुर्वते कुर्वत॒ इष॒ मूर्ज॒ मूर्ज॒ मिष॑म् कुर्वते कुर्वत॒ इष॒ मूर्ज᳚म् । \newline
27. इष॒ मूर्ज॒ मूर्ज॒ मिष॒ मिष॒ मूर्ज॑ मा॒त्मन् ना॒त्मन् नूर्ज॒ मिष॒ मिष॒ मूर्ज॑ मा॒त्मन्न् । \newline
28. ऊर्ज॑ मा॒त्मन् ना॒त्मन् नूर्ज॒ मूर्ज॑ मा॒त्मन् दधा॑ना॒ दधा॑ना आ॒त्मन् नूर्ज॒ मूर्ज॑ मा॒त्मन् दधा॑नाः । \newline
29. आ॒त्मन् दधा॑ना॒ दधा॑ना आ॒त्मन् ना॒त्मन् दधा॑ना वा॒णो वा॒णो दधा॑ना आ॒त्मन् ना॒त्मन् दधा॑ना वा॒णः । \newline
30. दधा॑ना वा॒णो वा॒णो दधा॑ना॒ दधा॑ना वा॒णः श॒तत॑न्तुः श॒तत॑न्तुर् वा॒णो दधा॑ना॒ दधा॑ना वा॒णः श॒तत॑न्तुः । \newline
31. वा॒णः श॒तत॑न्तुः श॒तत॑न्तुर् वा॒णो वा॒णः श॒तत॑न्तुर् भवति भवति श॒तत॑न्तुर् वा॒णो वा॒णः श॒तत॑न्तुर् भवति । \newline
32. श॒तत॑न्तुर् भवति भवति श॒तत॑न्तुः श॒तत॑न्तुर् भवति श॒तायुः॑ श॒तायु॑र् भवति श॒तत॑न्तुः श॒तत॑न्तुर् भवति श॒तायुः॑ । \newline
33. श॒तत॑न्तु॒रिति॑ श॒त - त॒न्तुः॒ । \newline
34. भ॒व॒ति॒ श॒तायुः॑ श॒तायु॑र् भवति भवति श॒तायुः॒ पुरु॑षः॒ पुरु॑षः श॒तायु॑र् भवति भवति श॒तायुः॒ पुरु॑षः । \newline
35. श॒तायुः॒ पुरु॑षः॒ पुरु॑षः श॒तायुः॑ श॒तायुः॒ पुरु॑षः श॒तेन्द्रि॑यः श॒तेन्द्रि॑यः॒ पुरु॑षः श॒तायुः॑ श॒तायुः॒ पुरु॑षः श॒तेन्द्रि॑यः । \newline
36. श॒तायु॒रिति॑ श॒त - आ॒युः॒ । \newline
37. पुरु॑षः श॒तेन्द्रि॑यः श॒तेन्द्रि॑यः॒ पुरु॑षः॒ पुरु॑षः श॒तेन्द्रि॑य॒ आयु॒ ष्यायु॑षि श॒तेन्द्रि॑यः॒ पुरु॑षः॒ पुरु॑षः श॒तेन्द्रि॑य॒ आयु॑षि । \newline
38. श॒तेन्द्रि॑य॒ आयु॒ ष्यायु॑षि श॒तेन्द्रि॑यः श॒तेन्द्रि॑य॒ आयु॑ष्ये॒वै वायु॑षि श॒तेन्द्रि॑यः श॒तेन्द्रि॑य॒ आयु॑ष्ये॒व । \newline
39. श॒तेन्द्रि॑य॒ इति॑ श॒त - इ॒न्द्रि॒यः॒ । \newline
40. आयु॑ष्ये॒वै वायु॒ ष्यायु॑ ष्ये॒वेन्द्रि॒य इ॑न्द्रि॒य ए॒वायु॒ ष्यायु॑ ष्ये॒वेन्द्रि॒ये । \newline
41. ए॒वेन्द्रि॒य इ॑न्द्रि॒य ए॒वैवेन्द्रि॒ये प्रति॒ प्रती᳚न्द्रि॒य ए॒वैवेन्द्रि॒ये प्रति॑ । \newline
42. इ॒न्द्रि॒ये प्रति॒ प्रती᳚न्द्रि॒य इ॑न्द्रि॒ये प्रति॑ तिष्ठन्ति तिष्ठन्ति॒ प्रती᳚न्द्रि॒य इ॑न्द्रि॒ये प्रति॑ तिष्ठन्ति । \newline
43. प्रति॑ तिष्ठन्ति तिष्ठन्ति॒ प्रति॒ प्रति॑ तिष्ठ न्त्या॒जि मा॒जिम् ति॑ष्ठन्ति॒ प्रति॒ प्रति॑ तिष्ठ न्त्या॒जिम् । \newline
44. ति॒ष्ठ॒ न्त्या॒जि मा॒जिम् ति॑ष्ठन्ति तिष्ठ न्त्या॒जिम् धा॑वन्ति धाव न्त्या॒जिम् ति॑ष्ठन्ति तिष्ठ न्त्या॒जिम् धा॑वन्ति । \newline
45. आ॒जिम् धा॑वन्ति धाव न्त्या॒जि मा॒जिम् धा॑व॒ न्त्यन॑भिजित॒स्या न॑भिजितस्य धाव न्त्या॒जि मा॒जिम् धा॑व॒ न्त्यन॑भिजितस्य । \newline
46. धा॒व॒ न्त्यन॑भिजित॒स्या न॑भिजितस्य धावन्ति धाव॒ न्त्यन॑भिजितस्या॒ भिजि॑त्या अ॒भिजि॑त्या॒ अन॑भिजितस्य धावन्ति धाव॒ न्त्यन॑भिजितस्या॒ भिजि॑त्यै । \newline
47. अन॑भिजितस्या॒ भिजि॑त्या अ॒भिजि॑त्या॒ अन॑भिजित॒स्या न॑भिजितस्या॒ भिजि॑त्यै दुन्दु॒भीन् दु॑न्दु॒भी न॒भिजि॑त्या॒ अन॑भिजित॒स्या न॑भिजितस्या॒ भिजि॑त्यै दुन्दु॒भीन् । \newline
48. अन॑भिजित॒स्येत्यन॑भि - जि॒त॒स्य॒ । \newline
49. अ॒भिजि॑त्यै दुन्दु॒भीन् दु॑न्दु॒भी न॒भिजि॑त्या अ॒भिजि॑त्यै दुन्दु॒भीन् थ्स॒माघ्न॑न्ति स॒माघ्न॑न्ति दुन्दु॒भी न॒भिजि॑त्या अ॒भिजि॑त्यै दुन्दु॒भीन् थ्स॒माघ्न॑न्ति । \newline
50. अ॒भिजि॑त्या॒ इत्य॒भि - जि॒त्यै॒ । \newline
51. दु॒न्दु॒भीन् थ्स॒माघ्न॑न्ति स॒माघ्न॑न्ति दुन्दु॒भीन् दु॑न्दु॒भीन् थ्स॒माघ्न॑न्ति पर॒मा प॑र॒मा स॒माघ्न॑न्ति दुन्दु॒भीन् दु॑न्दु॒भीन् थ्स॒माघ्न॑न्ति पर॒मा । \newline
52. स॒माघ्न॑न्ति पर॒मा प॑र॒मा स॒माघ्न॑न्ति स॒माघ्न॑न्ति पर॒मा वै वै प॑र॒मा स॒माघ्न॑न्ति स॒माघ्न॑न्ति पर॒मा वै । \newline
53. स॒माघ्न॒न्तीति॑ सं - आघ्न॑न्ति । \newline
54. प॒र॒मा वै वै प॑र॒मा प॑र॒मा वा ए॒षैषा वै प॑र॒मा प॑र॒मा वा ए॒षा । \newline
55. वा ए॒षैषा वै वा ए॒षा वाग् वागे॒षा वै वा ए॒षा वाक् । \newline
56. ए॒षा वाग् वागे॒षैषा वाग् या या वागे॒षैषा वाग् या । \newline
57. वाग् या या वाग् वाग् या दु॑न्दु॒भौ दु॑न्दु॒भौ या वाग् वाग् या दु॑न्दु॒भौ । \newline
58. या दु॑न्दु॒भौ दु॑न्दु॒भौ या या दु॑न्दु॒भौ प॑र॒माम् प॑र॒माम् दु॑न्दु॒भौ या या दु॑न्दु॒भौ प॑र॒माम् । \newline
59. दु॒न्दु॒भौ प॑र॒माम् प॑र॒माम् दु॑न्दु॒भौ दु॑न्दु॒भौ प॑र॒मा मे॒वैव प॑र॒माम् दु॑न्दु॒भौ दु॑न्दु॒भौ प॑र॒मा मे॒व । \newline
60. प॒र॒मा मे॒वैव प॑र॒माम् प॑र॒मा मे॒व वाचं॒ ॅवाच॑ मे॒व प॑र॒माम् प॑र॒मा मे॒व वाच᳚म् । \newline
61. ए॒व वाचं॒ ॅवाच॑ मे॒वैव वाच॒ मवाव॒ वाच॑ मे॒वैव वाच॒ मव॑ । \newline
\pagebreak
\markright{ TS 7.5.9.3  \hfill https://www.vedavms.in \hfill}

\section{ TS 7.5.9.3 }

\textbf{TS 7.5.9.3 } \newline
\textbf{Samhita Paata} \newline

वाच॒मव॑ रुन्धते भूमिदुन्दु॒भिमा घ्न॑न्ति॒ यैवेमां ॅवाक् प्रवि॑ष्टा॒ तामे॒वाव॑ रुन्ध॒ते ऽथो॑ इ॒मामे॒व ज॑यन्ति॒ सर्वा॒ वाचो॑ वदन्ति॒ सर्वा॑सां ॅवा॒चामव॑रुद्ध्या आ॒र्द्रेचर्म॒न् व्याय॑च्छेते इन्द्रि॒यस्या व॑रुद्ध्या॒ आऽन्यः क्रोश॑ति॒ प्रान्यः शꣳ॑सति॒ य आ॒क्रोश॑ति पु॒नात्ये॒वैना॒न्थ्स यः प्र॒शꣳस॑ति पू॒तेष्वे॒वान्नाद्यं॑ दधा॒त्यृषि॑कृतं च॒ - [  ] \newline

\textbf{Pada Paata} \newline

वाच᳚म् । अवेति॑ । रु॒न्ध॒ते॒ । भू॒मि॒दु॒न्दु॒भिमिति॑ भूमि - दु॒न्दु॒भिम् । एति॑ । घ्न॒न्ति॒ । या । ए॒व । इ॒माम् । वाक् । प्रवि॒ष्टेति॒ प्र - वि॒ष्टा॒ । ताम् । ए॒व । अवेति॑ । रु॒न्ध॒ते॒ । अथो॒ इति॑ । इ॒माम् । ए॒व । ज॒य॒न्ति॒ । सर्वाः᳚ । वाचः॑ । व॒द॒न्ति॒ । सर्वा॑साम् । वा॒चाम् । अव॑रुद्ध्या॒ इत्यव॑ - रु॒द्ध्यै॒ । आ॒र्द्रे । चर्मन्न्॑ । व्याय॑च्छेते॒ इति॑ वि-आय॑च्छेते । इ॒न्द्रि॒यस्य॑ । अव॑रुद्ध्या॒ इत्यव॑ - रु॒द्ध्यै॒ । एति॑ । अ॒न्यः । क्रोश॑ति । प्रेति॑ । अ॒न्यः । शꣳ॒॒स॒ति॒ । यः । आ॒क्रोश॒तीत्या᳚-क्रोश॑ति । पु॒नाति॑ । ए॒व । ए॒ना॒न् । सः । यः । प्र॒शꣳस॒तीति॑ प्र - शꣳस॑ति । पू॒तेषु॑ । ए॒व । अ॒न्नाद्य॒मित्य॑न्न - अद्य᳚म् । द॒धा॒ति॒ । ऋषि॑कृत॒मित्यृषि॑-कृ॒त॒म् । च॒ ।  \newline


\textbf{Krama Paata} \newline

वाच॒मव॑ । अव॑ रुन्धते । रु॒न्ध॒ते॒ भू॒मि॒दु॒न्दु॒भिम् । भू॒मि॒दु॒न्दु॒भिमा । भू॒मि॒दु॒न्दु॒भिमिति॑ भूमि - दु॒न्दु॒भिम् । 
आ घ्न॑न्ति । घ्न॒न्ति॒ या । यैव । ए॒वेमाम् । इ॒माम् ॅवाक् । वाक् प्रवि॑ष्टा । प्रवि॑ष्टा॒ ताम् । प्रवि॒ष्टेति॒ प्र - वि॒ष्टा॒ । तामे॒व । ए॒वाव॑ । अव॑ रुन्धते । रु॒न्ध॒तेऽथो᳚ । अथो॑ इ॒माम् । अथो॒ इत्यथो᳚ । इ॒मामे॒व । ए॒व ज॑यन्ति । ज॒य॒न्ति॒ सर्वाः᳚ । सर्वा॒ वाचः॑ । वाचो॑ वदन्ति । व॒द॒न्ति॒ सर्वा॑साम् । सर्वा॑साम् ॅवा॒चाम् । वा॒चामव॑रुद्ध्यै । अव॑रुद्ध्या आ॒र्द्रे । अव॑रुद्ध्या॒ इत्यव॑ - रु॒द्ध्यै॒ । आ॒र्द्रे चर्मन्न्॑ । चर्म॒न् व्याय॑च्छेते । व्याय॑च्छेते इन्द्रि॒यस्य॑ । व्याय॑च्छेते॒ इति॑ वि - आय॑च्छेते । इ॒न्द्रि॒यस्याव॑रुद्ध्यै । अव॑रुद्ध्या॒ आ । अव॑रुद्ध्या॒ इत्यव॑ - रु॒द्ध्यै॒ । आऽन्यः । अ॒न्यः क्रोश॑ति । क्रोश॑ति॒ प्र । प्रान्यः । अ॒न्यः शꣳ॑सति । शꣳ॒॒स॒ति॒ यः । य आ॒क्रोश॑ति । आ॒क्रोश॑ति पु॒नाति॑ । आ॒क्रोश॒तीत्या᳚ - क्रोश॑ति । पु॒नात्ये॒व । ए॒वैनान्॑ । ए॒ना॒न्थ् सः । स यः । यः प्र॒शꣳस॑ति । प्र॒शꣳस॑ति पू॒तेषु॑ । प्र॒शꣳस॒तीति॑ प्र - शꣳस॑ति । पू॒तेष्वे॒व । ए॒वान्नाद्य᳚म् । अ॒न्नाद्य॑म् दधाति । अ॒न्नाद्य॒मित्य॑न्न - अद्य᳚म् । द॒धा॒त्यृषि॑कृतम् । ऋषि॑कृतम् च ( ) । ऋषि॑कृत॒मित्यृषि॑ - कृ॒त॒म् । च॒ वै \newline

\textbf{Jatai Paata} \newline

1. वाच॒ मवाव॒ वाचं॒ ॅवाच॒ मव॑ । \newline
2. अव॑ रुन्धते रुन्ध॒ते ऽवाव॑ रुन्धते । \newline
3. रु॒न्ध॒ते॒ भू॒मि॒दु॒न्दु॒भिम् भू॑मिदुन्दु॒भिꣳ रु॑न्धते रुन्धते भूमिदुन्दु॒भिम् । \newline
4. भू॒मि॒दु॒न्दु॒भि मा भू॑मिदुन्दु॒भिम् भू॑मिदुन्दु॒भि मा । \newline
5. भू॒मि॒दु॒न्दु॒भिमिति॑ भूमि - दु॒न्दु॒भिम् । \newline
6. आ घ्न॑न्ति घ्न॒न्त्या घ्न॑न्ति । \newline
7. घ्न॒न्ति॒ या या घ्न॑न्ति घ्नन्ति॒ या । \newline
8. यैवैव या यैव । \newline
9. ए॒वे मा मि॒मा मे॒वैवे माम् । \newline
10. इ॒मां ॅवाग् वागि॒मा मि॒मां ॅवाक् । \newline
11. वाक् प्रवि॑ष्टा॒ प्रवि॑ष्टा॒ वाग् वाक् प्रवि॑ष्टा । \newline
12. प्रवि॑ष्टा॒ ताम् ताम् प्रवि॑ष्टा॒ प्रवि॑ष्टा॒ ताम् । \newline
13. प्रवि॒ष्टेति॒ प्र - वि॒ष्टा॒ । \newline
14. ता मे॒वैव ताम् ता मे॒व । \newline
15. ए॒वावा वै॒वै वाव॑ । \newline
16. अव॑ रुन्धते रुन्ध॒ते ऽवाव॑ रुन्धते । \newline
17. रु॒न्ध॒ते ऽथो॒ अथो॑ रुन्धते रुन्ध॒ते ऽथो᳚ । \newline
18. अथो॑ इ॒मा मि॒मा मथो॒ अथो॑ इ॒माम् । \newline
19. अथो॒ इत्यथो᳚ । \newline
20. इ॒मा मे॒वैवे मा मि॒मा मे॒व । \newline
21. ए॒व ज॑यन्ति जय न्त्ये॒वैव ज॑यन्ति । \newline
22. ज॒य॒न्ति॒ सर्वाः॒ सर्वा॑ जयन्ति जयन्ति॒ सर्वाः᳚ । \newline
23. सर्वा॒ वाचो॒ वाचः॒ सर्वाः॒ सर्वा॒ वाचः॑ । \newline
24. वाचो॑ वदन्ति वदन्ति॒ वाचो॒ वाचो॑ वदन्ति । \newline
25. व॒द॒न्ति॒ सर्वा॑साꣳ॒॒ सर्वा॑सां ॅवदन्ति वदन्ति॒ सर्वा॑साम् । \newline
26. सर्वा॑सां ॅवा॒चां ॅवा॒चाꣳ सर्वा॑साꣳ॒॒ सर्वा॑सां ॅवा॒चाम् । \newline
27. वा॒चा मव॑रुद्ध्या॒ अव॑रुद्ध्यै वा॒चां ॅवा॒चा मव॑रुद्ध्यै । \newline
28. अव॑रुद्ध्या आ॒र्द्र आ॒र्द्रे ऽव॑रुद्ध्या॒ अव॑रुद्ध्या आ॒र्द्रे । \newline
29. अव॑रुद्ध्या॒ इत्यव॑ - रु॒द्ध्यै॒ । \newline
30. आ॒र्द्रे चर्मꣳ॒॒ श्चर्म॑न् ना॒र्द्र आ॒र्द्रे चर्मन्न्॑ । \newline
31. चर्म॒न् व्याय॑च्छेते॒ व्याय॑च्छेते॒ चर्मꣳ॒॒ श्चर्म॒न् व्याय॑च्छेते । \newline
32. व्याय॑च्छेते इन्द्रि॒य स्ये᳚न्द्रि॒यस्य॒ व्याय॑च्छेते॒ व्याय॑च्छेते इन्द्रि॒यस्य॑ । \newline
33. व्याय॑च्छेते॒ इति॑ वि - आय॑च्छेते । \newline
34. इ॒न्द्रि॒यस्या व॑रुद्ध्या॒ अव॑रुद्ध्या इन्द्रि॒य स्ये᳚न्द्रि॒य स्याव॑रुद्ध्यै । \newline
35. अव॑रुद्ध्या॒ आ ऽव॑रुद्ध्या॒ अव॑रुद्ध्या॒ आ । \newline
36. अव॑रुद्ध्या॒ इत्यव॑ - रु॒द्ध्यै॒ । \newline
37. आ ऽन्यो᳚ ऽन्य आ ऽन्यः । \newline
38. अ॒न्यः क्रोश॑ति॒ क्रोश॑ त्य॒न्यो᳚ ऽन्यः क्रोश॑ति । \newline
39. क्रोश॑ति॒ प्र प्र क्रोश॑ति॒ क्रोश॑ति॒ प्र । \newline
40. प्रान्यो᳚ ऽन्यः प्र प्रान्यः । \newline
41. अ॒न्यः शꣳ॑सति शꣳस त्य॒न्यो᳚ ऽन्यः शꣳ॑सति । \newline
42. शꣳ॒॒स॒ति॒ यो यः शꣳ॑सति शꣳसति॒ यः । \newline
43. य आ॒क्रोश॑ त्या॒क्रोश॑ति॒ यो य आ॒क्रोश॑ति । \newline
44. आ॒क्रोश॑ति पु॒नाति॑ पु॒ना त्या॒क्रोश॑ त्या॒क्रोश॑ति पु॒नाति॑ । \newline
45. आ॒क्रोश॒तीत्या᳚ - क्रोश॑ति । \newline
46. पु॒ना त्ये॒वैव पु॒नाति॑ पु॒ना त्ये॒व । \newline
47. ए॒वैना॑ नेना ने॒वै वैनान्॑ । \newline
48. ए॒ना॒न् थ्स स ए॑ना नेना॒न् थ्सः । \newline
49. स यो यः स स यः । \newline
50. यः प्र॒शꣳस॑ति प्र॒शꣳस॑ति॒ यो यः प्र॒शꣳस॑ति । \newline
51. प्र॒शꣳस॑ति पू॒तेषु॑ पू॒तेषु॑ प्र॒शꣳस॑ति प्र॒शꣳस॑ति पू॒तेषु॑ । \newline
52. प्र॒शꣳस॒तीति॑ प्र - शꣳस॑ति । \newline
53. पू॒ते ष्वे॒वैव पू॒तेषु॑ पू॒तेष्वे॒व । \newline
54. ए॒वान्नाद्य॑ म॒न्नाद्य॑ मे॒वै वान्नाद्य᳚म् । \newline
55. अ॒न्नाद्य॑म् दधाति दधा त्य॒न्नाद्य॑ म॒न्नाद्य॑म् दधाति । \newline
56. अ॒न्नाद्य॒मित्य॑न्न - अद्य᳚म् । \newline
57. द॒धा॒ त्यृषि॑कृत॒ मृषि॑कृतम् दधाति दधा॒ त्यृषि॑कृतम् । \newline
58. ऋषि॑कृतम् च॒ चर्.षि॑कृत॒ मृषि॑कृतम् च । \newline
59. ऋषि॑कृत॒मित्यृषि॑ - कृ॒त॒म् । \newline
60. च॒ वै वै च॑ च॒ वै । \newline

\textbf{Ghana Paata } \newline

1. वाच॒ मवाव॒ वाचं॒ ॅवाच॒ मव॑ रुन्धते रुन्ध॒ते ऽव॒ वाचं॒ ॅवाच॒ मव॑ रुन्धते । \newline
2. अव॑ रुन्धते रुन्ध॒ते ऽवाव॑ रुन्धते भूमिदुन्दु॒भिम् भू॑मिदुन्दु॒भिꣳ रु॑न्ध॒ते ऽवाव॑ रुन्धते भूमिदुन्दु॒भिम् । \newline
3. रु॒न्ध॒ते॒ भू॒मि॒दु॒न्दु॒भिम् भू॑मिदुन्दु॒भिꣳ रु॑न्धते रुन्धते भूमिदुन्दु॒भि मा भू॑मिदुन्दु॒भिꣳ रु॑न्धते रुन्धते भूमिदुन्दु॒भि मा । \newline
4. भू॒मि॒दु॒न्दु॒भि मा भू॑मिदुन्दु॒भिम् भू॑मिदुन्दु॒भि मा घ्न॑न्ति घ्न॒न्त्या भू॑मिदुन्दु॒भिम् भू॑मिदुन्दु॒भि मा घ्न॑न्ति । \newline
5. भू॒मि॒दु॒न्दु॒भिमिति॑ भूमि - दु॒न्दु॒भिम् । \newline
6. आ घ्न॑न्ति घ्न॒न्त्या घ्न॑न्ति॒ या या घ्न॒न्त्या घ्न॑न्ति॒ या । \newline
7. घ्न॒न्ति॒ या या घ्न॑न्ति घ्नन्ति॒ यैवैव या घ्न॑न्ति घ्नन्ति॒ यैव । \newline
8. यैवैव या यैवेमा मि॒मा मे॒व या यैवेमाम् । \newline
9. ए॒वेमा मि॒मा मे॒वैवेमां ॅवाग् वागि॒मा मे॒वैवेमां ॅवाक् । \newline
10. इ॒मां ॅवाग् वागि॒मा मि॒मां ॅवाक् प्रवि॑ष्टा॒ प्रवि॑ष्टा॒ वागि॒मा मि॒मां ॅवाक् प्रवि॑ष्टा । \newline
11. वाक् प्रवि॑ष्टा॒ प्रवि॑ष्टा॒ वाग् वाक् प्रवि॑ष्टा॒ ताम् ताम् प्रवि॑ष्टा॒ वाग् वाक् प्रवि॑ष्टा॒ ताम् । \newline
12. प्रवि॑ष्टा॒ ताम् ताम् प्रवि॑ष्टा॒ प्रवि॑ष्टा॒ ता मे॒वैव ताम् प्रवि॑ष्टा॒ प्रवि॑ष्टा॒ ता मे॒व । \newline
13. प्रवि॒ष्टेति॒ प्र - वि॒ष्टा॒ । \newline
14. ता मे॒वैव ताम् ता मे॒वावा वै॒व ताम् ता मे॒वाव॑ । \newline
15. ए॒वावा वै॒वै वाव॑ रुन्धते रुन्ध॒ते ऽवै॒वै वाव॑ रुन्धते । \newline
16. अव॑ रुन्धते रुन्ध॒ते ऽवाव॑ रुन्ध॒ते ऽथो॒ अथो॑ रुन्ध॒ते ऽवाव॑ रुन्ध॒ते ऽथो᳚ । \newline
17. रु॒न्ध॒ते ऽथो॒ अथो॑ रुन्धते रुन्ध॒ते ऽथो॑ इ॒मा मि॒मा मथो॑ रुन्धते रुन्ध॒ते ऽथो॑ इ॒माम् । \newline
18. अथो॑ इ॒मा मि॒मा मथो॒ अथो॑ इ॒मा मे॒वैवेमा मथो॒ अथो॑ इ॒मा मे॒व । \newline
19. अथो॒ इत्यथो᳚ । \newline
20. इ॒मा मे॒वैवेमा मि॒मा मे॒व ज॑यन्ति जय न्त्ये॒वेमा मि॒मा मे॒व ज॑यन्ति । \newline
21. ए॒व ज॑यन्ति जय न्त्ये॒वैव ज॑यन्ति॒ सर्वाः॒ सर्वा॑ जय न्त्ये॒वैव ज॑यन्ति॒ सर्वाः᳚ । \newline
22. ज॒य॒न्ति॒ सर्वाः॒ सर्वा॑ जयन्ति जयन्ति॒ सर्वा॒ वाचो॒ वाचः॒ सर्वा॑ जयन्ति जयन्ति॒ सर्वा॒ वाचः॑ । \newline
23. सर्वा॒ वाचो॒ वाचः॒ सर्वाः॒ सर्वा॒ वाचो॑ वदन्ति वदन्ति॒ वाचः॒ सर्वाः॒ सर्वा॒ वाचो॑ वदन्ति । \newline
24. वाचो॑ वदन्ति वदन्ति॒ वाचो॒ वाचो॑ वदन्ति॒ सर्वा॑साꣳ॒॒ सर्वा॑सां ॅवदन्ति॒ वाचो॒ वाचो॑ वदन्ति॒ सर्वा॑साम् । \newline
25. व॒द॒न्ति॒ सर्वा॑साꣳ॒॒ सर्वा॑सां ॅवदन्ति वदन्ति॒ सर्वा॑सां ॅवा॒चां ॅवा॒चाꣳ सर्वा॑सां ॅवदन्ति वदन्ति॒ सर्वा॑सां ॅवा॒चाम् । \newline
26. सर्वा॑सां ॅवा॒चां ॅवा॒चाꣳ सर्वा॑साꣳ॒॒ सर्वा॑सां ॅवा॒चा मव॑रुद्ध्या॒ अव॑रुद्ध्यै वा॒चाꣳ सर्वा॑साꣳ॒॒ सर्वा॑सां ॅवा॒चा मव॑रुद्ध्यै । \newline
27. वा॒चा मव॑रुद्ध्या॒ अव॑रुद्ध्यै वा॒चां ॅवा॒चा मव॑रुद्ध्या आ॒र्द्र आ॒र्द्रे ऽव॑रुद्ध्यै वा॒चां ॅवा॒चा मव॑रुद्ध्या आ॒र्द्रे । \newline
28. अव॑रुद्ध्या आ॒र्द्र आ॒र्द्रे ऽव॑रुद्ध्या॒ अव॑रुद्ध्या आ॒र्द्रे चर्मꣳ॒॒ श्चर्म॑न् ना॒र्द्रे ऽव॑रुद्ध्या॒ अव॑रुद्ध्या आ॒र्द्रे चर्मन्न्॑ । \newline
29. अव॑रुद्ध्या॒ इत्यव॑ - रु॒द्ध्यै॒ । \newline
30. आ॒र्द्रे चर्मꣳ॒॒ श्चर्म॑न् ना॒र्द्र आ॒र्द्रे चर्म॒न् व्याय॑च्छेते॒ व्याय॑च्छेते॒ चर्म॑न् ना॒र्द्र आ॒र्द्रे चर्म॒न् व्याय॑च्छेते । \newline
31. चर्म॒न् व्याय॑च्छेते॒ व्याय॑च्छेते॒ चर्मꣳ॒॒ श्चर्म॒न् व्याय॑च्छेते इन्द्रि॒य स्ये᳚न्द्रि॒यस्य॒ व्याय॑च्छेते॒ चर्मꣳ॒॒ श्चर्म॒न् व्याय॑च्छेते इन्द्रि॒यस्य॑ । \newline
32. व्याय॑च्छेते इन्द्रि॒य स्ये᳚न्द्रि॒यस्य॒ व्याय॑च्छेते॒ व्याय॑च्छेते इन्द्रि॒यस्या व॑रुद्ध्या॒ अव॑रुद्ध्या इन्द्रि॒यस्य॒ व्याय॑च्छेते॒ व्याय॑च्छेते इन्द्रि॒यस्या व॑रुद्ध्यै । \newline
33. व्याय॑च्छेते॒ इति॑ वि - आय॑च्छेते । \newline
34. इ॒न्द्रि॒यस्या व॑रुद्ध्या॒ अव॑रुद्ध्या इन्द्रि॒य स्ये᳚न्द्रि॒यस्या व॑रुद्ध्या॒ आ ऽव॑रुद्ध्या इन्द्रि॒य स्ये᳚न्द्रि॒यस्या व॑रुद्ध्या॒ आ । \newline
35. अव॑रुद्ध्या॒ आ ऽव॑रुद्ध्या॒ अव॑रुद्ध्या॒ आ ऽन्यो᳚ ऽन्य आ ऽव॑रुद्ध्या॒ अव॑रुद्ध्या॒ आ ऽन्यः । \newline
36. अव॑रुद्ध्या॒ इत्यव॑ - रु॒द्ध्यै॒ । \newline
37. आ ऽन्यो᳚ ऽन्य आ ऽन्यः क्रोश॑ति॒ क्रोश॑ त्य॒न्य आ ऽन्यः क्रोश॑ति । \newline
38. अ॒न्यः क्रोश॑ति॒ क्रोश॑ त्य॒न्यो᳚ ऽन्यः क्रोश॑ति॒ प्र प्र क्रोश॑ त्य॒न्यो᳚ ऽन्यः क्रोश॑ति॒ प्र । \newline
39. क्रोश॑ति॒ प्र प्र क्रोश॑ति॒ क्रोश॑ति॒ प्रान्यो᳚ ऽन्यः प्र क्रोश॑ति॒ क्रोश॑ति॒ प्रान्यः । \newline
40. प्रान्यो᳚ ऽन्यः प्र प्रान्यः शꣳ॑सति शꣳस त्य॒न्यः प्र प्रान्यः शꣳ॑सति । \newline
41. अ॒न्यः शꣳ॑सति शꣳस त्य॒न्यो᳚ ऽन्यः शꣳ॑सति॒ यो यः शꣳ॑स त्य॒न्यो᳚ ऽन्यः शꣳ॑सति॒ यः । \newline
42. शꣳ॒॒स॒ति॒ यो यः शꣳ॑सति शꣳसति॒ य आ॒क्रोश॑ त्या॒क्रोश॑ति॒ यः शꣳ॑सति शꣳसति॒ य आ॒क्रोश॑ति । \newline
43. य आ॒क्रोश॑ त्या॒क्रोश॑ति॒ यो य आ॒क्रोश॑ति पु॒नाति॑ पु॒ना त्या॒क्रोश॑ति॒ यो य आ॒क्रोश॑ति पु॒नाति॑ । \newline
44. आ॒क्रोश॑ति पु॒नाति॑ पु॒ना त्या॒क्रोश॑ त्या॒क्रोश॑ति पु॒ना त्ये॒वैव पु॒ना त्या॒क्रोश॑ त्या॒क्रोश॑ति पु॒ना त्ये॒व । \newline
45. आ॒क्रोश॒तीत्या᳚ - क्रोश॑ति । \newline
46. पु॒ना त्ये॒वैव पु॒नाति॑ पु॒ना त्ये॒वैना॑ नेना ने॒व पु॒नाति॑ पु॒ना त्ये॒वैनान्॑ । \newline
47. ए॒वैना॑ नेनाने॒ वैवैना॒न् थ्स स ए॑ना ने॒वैवैना॒न् थ्सः । \newline
48. ए॒ना॒न् थ्स स ए॑ना नेना॒न् थ्स यो यः स ए॑ना नेना॒न् थ्स यः । \newline
49. स यो यः स स यः प्र॒शꣳस॑ति प्र॒शꣳस॑ति॒ यः स स यः प्र॒शꣳस॑ति । \newline
50. यः प्र॒शꣳस॑ति प्र॒शꣳस॑ति॒ यो यः प्र॒शꣳस॑ति पू॒तेषु॑ पू॒तेषु॑ प्र॒शꣳस॑ति॒ यो यः प्र॒शꣳस॑ति पू॒तेषु॑ । \newline
51. प्र॒शꣳस॑ति पू॒तेषु॑ पू॒तेषु॑ प्र॒शꣳस॑ति प्र॒शꣳस॑ति पू॒ते ष्वे॒वैव पू॒तेषु॑ प्र॒शꣳस॑ति प्र॒शꣳस॑ति पू॒ते ष्वे॒व । \newline
52. प्र॒शꣳस॒तीति॑ प्र - शꣳस॑ति । \newline
53. पू॒ते ष्वे॒वैव पू॒तेषु॑ पू॒ते ष्वे॒वान्नाद्य॑ म॒न्नाद्य॑ मे॒व पू॒तेषु॑ पू॒ते ष्वे॒वान्नाद्य᳚म् । \newline
54. ए॒वान्नाद्य॑ म॒न्नाद्य॑ मे॒वै वान्नाद्य॑म् दधाति दधा त्य॒न्नाद्य॑ मे॒वै वान्नाद्य॑म् दधाति । \newline
55. अ॒न्नाद्य॑म् दधाति दधा त्य॒न्नाद्य॑ म॒न्नाद्य॑म् दधा॒ त्यृषि॑कृत॒ मृषि॑कृतम् दधा त्य॒न्नाद्य॑ म॒न्नाद्य॑म् दधा॒ त्यृषि॑कृतम् । \newline
56. अ॒न्नाद्य॒मित्य॑न्न - अद्य᳚म् । \newline
57. द॒धा॒ त्यृषि॑कृत॒ मृषि॑कृतम् दधाति दधा॒ त्यृषि॑कृतम् च॒ च र्.षि॑कृतम् दधाति दधा॒ त्यृषि॑कृतम् च । \newline
58. ऋषि॑कृतम् च॒ च र्.षि॑कृत॒ मृषि॑कृतम् च॒ वै वै च र्.षि॑कृत॒ मृषि॑कृतम् च॒ वै । \newline
59. ऋषि॑कृत॒मित्यृषि॑ - कृ॒त॒म् । \newline
60. च॒ वै वै च॑ च॒ वा ए॒त ए॒ते वै च॑ च॒ वा ए॒ते । \newline
\pagebreak
\markright{ TS 7.5.9.4  \hfill https://www.vedavms.in \hfill}

\section{ TS 7.5.9.4 }

\textbf{TS 7.5.9.4 } \newline
\textbf{Samhita Paata} \newline

वा ए॒ते दे॒वकृ॑तं च॒ पूर्वै॒र्मासै॒रव॑ रुन्धते॒ यद्-भू॑ते॒च्छदाꣳ॒॒ सामा॑नि॒ भव॑न्त्यु॒भय॒स्याव॑रुद्ध्यै॒ यन्ति॒ वा ए॒ते मि॑थु॒नाद्ये सं॑ॅवथ्स॒र-मु॑प॒यन्त्य॑न्तर्वे॒दि मि॑थु॒नौ सं भ॑वत॒स्तेनै॒व मि॑थु॒नान्न य॑न्ति ॥ \newline

\textbf{Pada Paata} \newline

वै । ए॒ते । दे॒वकृ॑त॒मिति॑ दे॒व-कृ॒त॒म् । च॒ । पूर्वैः᳚ । मासैः᳚ । अवेति॑ । रु॒न्ध॒ते॒ । यत् । भू॒ते॒च्छदा॒मिति॑ भूते - छदा᳚म् । सामा॑नि । भव॑न्ति । उ॒भय॑स्य । अव॑रुद्ध्या॒ इत्यव॑ - रु॒द्ध्यै॒ । यन्ति॑ । वै । ए॒ते । मि॒थु॒नात् । ये । सं॒ॅव॒थ्स॒रमिति॑ सं-व॒थ्स॒रम् । उ॒प॒यन्तीत्यु॑प-यन्ति॑ । अ॒न्त॒र्वे॒दीत्य॑न्तः - वे॒दि । मि॒थु॒नौ । समिति॑ । भ॒व॒तः॒ । तेन॑ । ए॒व । मि॒थु॒नात् । न । य॒न्ति॒ ॥  \newline


\textbf{Krama Paata} \newline

वा ए॒ते । ए॒ते दे॒वकृ॑तम् । दे॒वकृ॑तम् च । दे॒वकृ॑त॒मिति॑ दे॒व - कृ॒त॒म् । च॒ पूर्वैः᳚ । पूर्वै॒र् मासैः᳚ । मासै॒रव॑ । अव॑ रुन्धते । रु॒न्ध॒ते॒ यत् । यद् भू॑ते॒च्छदा᳚म् । भू॒ते॒च्छदाꣳ॒॒ सामा॑नि । भू॒ते॒च्छदा॒मिति॑ भूते - छदा᳚म् । सामा॑नि॒ भव॑न्ति । भव॑न्त्यु॒भय॑स्य । उ॒भय॒स्याव॑रुद्ध्यै । अव॑रुद्ध्यै॒ यन्ति॑ । अव॑रुद्ध्या॒ इत्यव॑ - रु॒द्ध्यै॒ । यन्ति॒ वै । वा ए॒ते । ए॒ते मि॑थु॒नात् । मि॒थु॒नाद् ये । ये स॑म्ॅवथ्स॒रम् । स॒म्ॅव॒थ्स॒रमु॑प॒यन्ति॑ । स॒म्ॅव॒थ्स॒रमिति॑ सम् - व॒थ्स॒रम् । उ॒प॒यन्त्य॑न्तर्वे॒दि । उ॒प॒यन्तीत्यु॑प - यन्ति॑ । अ॒न्त॒र्वे॒दि मि॑थु॒नौ । अ॒न्त॒र्वे॒दीत्य॑न्तः - वे॒दि । मि॒थु॒नौ सम् । सम् भ॑वतः । भ॒व॒त॒स्तेन॑ । तेनै॒व । ए॒व मि॑थु॒नात् । मि॒थु॒नान् न । न य॑न्ति । य॒न्तीति॑ यन्ति । \newline

\textbf{Jatai Paata} \newline

1. वा ए॒त ए॒ते वै वा ए॒ते । \newline
2. ए॒ते दे॒वकृ॑तम् दे॒वकृ॑त मे॒त ए॒ते दे॒वकृ॑तम् । \newline
3. दे॒वकृ॑तम् च च दे॒वकृ॑तम् दे॒वकृ॑तम् च । \newline
4. दे॒वकृ॑त॒मिति॑ दे॒व - कृ॒त॒म् । \newline
5. च॒ पूर्वैः॒ पूर्वै᳚ श्च च॒ पूर्वैः᳚ । \newline
6. पूर्वै॒र् मासै॒र् मासैः॒ पूर्वैः॒ पूर्वै॒र् मासैः᳚ । \newline
7. मासै॒ रवाव॒ मासै॒र् मासै॒ रव॑ । \newline
8. अव॑ रुन्धते रुन्ध॒ते ऽवाव॑ रुन्धते । \newline
9. रु॒न्ध॒ते॒ यद् यद् रु॑न्धते रुन्धते॒ यत् । \newline
10. यद् भू॑ते॒च्छदा᳚म् भूते॒च्छदां॒ ॅयद् यद् भू॑ते॒च्छदा᳚म् । \newline
11. भू॒ते॒च्छदाꣳ॒॒ सामा॑नि॒ सामा॑नि भूते॒च्छदा᳚म् भूते॒च्छदाꣳ॒॒ सामा॑नि । \newline
12. भू॒ते॒च्छदा॒मिति॑ भूते - छदा᳚म् । \newline
13. सामा॑नि॒ भव॑न्ति॒ भव॑न्ति॒ सामा॑नि॒ सामा॑नि॒ भव॑न्ति । \newline
14. भव॑ न्त्यु॒भय॑ स्यो॒भय॑स्य॒ भव॑न्ति॒ भव॑ न्त्यु॒भय॑स्य । \newline
15. उ॒भय॒स्या व॑रुद्ध्या॒ अव॑रुद्ध्या उ॒भय॑ स्यो॒भय॒स्या व॑रुद्ध्यै । \newline
16. अव॑रुद्ध्यै॒ यन्ति॒ यन्त्यव॑रुद्ध्या॒ अव॑रुद्ध्यै॒ यन्ति॑ । \newline
17. अव॑रुद्ध्या॒ इत्यव॑ - रु॒द्ध्यै॒ । \newline
18. यन्ति॒ वै वै यन्ति॒ यन्ति॒ वै । \newline
19. वा ए॒त ए॒ते वै वा ए॒ते । \newline
20. ए॒ते मि॑थु॒नान् मि॑थु॒ना दे॒त ए॒ते मि॑थु॒नात् । \newline
21. मि॒थु॒नाद् ये ये मि॑थु॒नान् मि॑थु॒नाद् ये । \newline
22. ये सं॑ॅवथ्स॒रꣳ सं॑ॅवथ्स॒रं ॅये ये सं॑ॅवथ्स॒रम् । \newline
23. सं॒ॅव॒थ्स॒र मु॑प॒य न्त्यु॑प॒यन्ति॑ संॅवथ्स॒रꣳ सं॑ॅवथ्स॒र मु॑प॒यन्ति॑ । \newline
24. सं॒ॅव॒थ्स॒रमिति॑ सं - व॒थ्स॒रम् । \newline
25. उ॒प॒य न्त्य॑न्तर्वे॒ द्य॑न्तर्वे॒ द्यु॑प॒य न्त्यु॑प॒य न्त्य॑न्तर्वे॒दि । \newline
26. उ॒प॒यन्तीत्यु॑प - यन्ति॑ । \newline
27. अ॒न्त॒र्वे॒दि मि॑थु॒नौ मि॑थु॒ना व॑न्तर्वे॒ द्य॑न्तर्वे॒दि मि॑थु॒नौ । \newline
28. अ॒न्त॒र्वे॒दीत्य॑न्तः - वे॒दि । \newline
29. मि॒थु॒नौ सꣳ सम् मि॑थु॒नौ मि॑थु॒नौ सम् । \newline
30. सम् भ॑वतो भवतः॒ सꣳ सम् भ॑वतः । \newline
31. भ॒व॒त॒ स्तेन॒ तेन॑ भवतो भवत॒ स्तेन॑ । \newline
32. तेनै॒वैव तेन॒ तेनै॒व । \newline
33. ए॒व मि॑थु॒नान् मि॑थु॒ना दे॒वैव मि॑थु॒नात् । \newline
34. मि॒थु॒नान् न न मि॑थु॒नान् मि॑थु॒नान् न । \newline
35. न य॑न्ति यन्ति॒ न न य॑न्ति । \newline
36. य॒न्तीति॑ यन्ति । \newline

\textbf{Ghana Paata } \newline

1. वा ए॒त ए॒ते वै वा ए॒ते दे॒वकृ॑तम् दे॒वकृ॑त मे॒ते वै वा ए॒ते दे॒वकृ॑तम् । \newline
2. ए॒ते दे॒वकृ॑तम् दे॒वकृ॑त मे॒त ए॒ते दे॒वकृ॑तम् च च दे॒वकृ॑त मे॒त ए॒ते दे॒वकृ॑तम् च । \newline
3. दे॒वकृ॑तम् च च दे॒वकृ॑तम् दे॒वकृ॑तम् च॒ पूर्वैः॒ पूर्वै᳚ श्च दे॒वकृ॑तम् दे॒वकृ॑तम् च॒ पूर्वैः᳚ । \newline
4. दे॒वकृ॑त॒मिति॑ दे॒व - कृ॒त॒म् । \newline
5. च॒ पूर्वैः॒ पूर्वै᳚ श्च च॒ पूर्वै॒र् मासै॒र् मासैः॒ पूर्वै᳚ श्च च॒ पूर्वै॒र् मासैः᳚ । \newline
6. पूर्वै॒र् मासै॒र् मासैः॒ पूर्वैः॒ पूर्वै॒र् मासै॒ रवाव॒ मासैः॒ पूर्वैः॒ पूर्वै॒र् मासै॒ रव॑ । \newline
7. मासै॒ रवाव॒ मासै॒र् मासै॒ रव॑ रुन्धते रुन्ध॒ते ऽव॒ मासै॒र् मासै॒ रव॑ रुन्धते । \newline
8. अव॑ रुन्धते रुन्ध॒ते ऽवाव॑ रुन्धते॒ यद् यद् रु॑न्ध॒ते ऽवाव॑ रुन्धते॒ यत् । \newline
9. रु॒न्ध॒ते॒ यद् यद् रु॑न्धते रुन्धते॒ यद् भू॑ते॒च्छदा᳚म् भूते॒च्छदां॒ ॅयद् रु॑न्धते रुन्धते॒ यद् भू॑ते॒च्छदा᳚म् । \newline
10. यद् भू॑ते॒च्छदा᳚म् भूते॒च्छदां॒ ॅयद् यद् भू॑ते॒च्छदाꣳ॒॒ सामा॑नि॒ सामा॑नि भूते॒च्छदां॒ ॅयद् यद् भू॑ते॒च्छदाꣳ॒॒ सामा॑नि । \newline
11. भू॒ते॒च्छदाꣳ॒॒ सामा॑नि॒ सामा॑नि भूते॒च्छदा᳚म् भूते॒च्छदाꣳ॒॒ सामा॑नि॒ भव॑न्ति॒ भव॑न्ति॒ सामा॑नि भूते॒च्छदा᳚म् भूते॒च्छदाꣳ॒॒ सामा॑नि॒ भव॑न्ति । \newline
12. भू॒ते॒च्छदा॒मिति॑ भूते - छदा᳚म् । \newline
13. सामा॑नि॒ भव॑न्ति॒ भव॑न्ति॒ सामा॑नि॒ सामा॑नि॒ भव॑ न्त्यु॒भय॑ स्यो॒भय॑स्य॒ भव॑न्ति॒ सामा॑नि॒ सामा॑नि॒ भव॑ न्त्यु॒भय॑स्य । \newline
14. भव॑ न्त्यु॒भय॑ स्यो॒भय॑स्य॒ भव॑न्ति॒ भव॑ न्त्यु॒भय॒स्या व॑रुद्ध्या॒ अव॑रुद्ध्या उ॒भय॑स्य॒ भव॑न्ति॒ भव॑ न्त्यु॒भय॒स्या व॑रुद्ध्यै । \newline
15. उ॒भय॒स्या व॑रुद्ध्या॒ अव॑रुद्ध्या उ॒भय॑ स्यो॒भय॒स्या व॑रुद्ध्यै॒ यन्ति॒ यन्त्यव॑रुद्ध्या उ॒भय॑ 
स्यो॒भय॒स्या व॑रुद्ध्यै॒ यन्ति॑ । \newline
16. अव॑रुद्ध्यै॒ यन्ति॒ यन्त्यव॑रुद्ध्या॒ अव॑रुद्ध्यै॒ यन्ति॒ वै वै यन्त्यव॑रुद्ध्या॒ अव॑रुद्ध्यै॒ यन्ति॒ वै । \newline
17. अव॑रुद्ध्या॒ इत्यव॑ - रु॒द्ध्यै॒ । \newline
18. यन्ति॒ वै वै यन्ति॒ यन्ति॒ वा ए॒त ए॒ते वै यन्ति॒ यन्ति॒ वा ए॒ते । \newline
19. वा ए॒त ए॒ते वै वा ए॒ते मि॑थु॒नान् मि॑थु॒ना दे॒ते वै वा ए॒ते मि॑थु॒नात् । \newline
20. ए॒ते मि॑थु॒नान् मि॑थु॒ना दे॒त ए॒ते मि॑थु॒नाद् ये ये मि॑थु॒ना दे॒त ए॒ते मि॑थु॒नाद् ये । \newline
21. मि॒थु॒नाद् ये ये मि॑थु॒नान् मि॑थु॒नाद् ये सं॑ॅवथ्स॒रꣳ सं॑ॅवथ्स॒रं ॅये मि॑थु॒नान् मि॑थु॒नाद् ये सं॑ॅवथ्स॒रम् । \newline
22. ये सं॑ॅवथ्स॒रꣳ सं॑ॅवथ्स॒रं ॅये ये सं॑ॅवथ्स॒र मु॑प॒य न्त्यु॑प॒यन्ति॑ संॅवथ्स॒रं ॅये ये सं॑ॅवथ्स॒र मु॑प॒यन्ति॑ । \newline
23. सं॒ॅव॒थ्स॒र मु॑प॒य न्त्यु॑प॒यन्ति॑ संॅवथ्स॒रꣳ सं॑ॅवथ्स॒र मु॑प॒य न्त्य॑न्तर्वे॒ द्य॑न्तर्वे॒ द्यु॑प॒यन्ति॑ संॅवथ्स॒रꣳ सं॑ॅवथ्स॒र मु॑प॒य न्त्य॑न्तर्वे॒दि । \newline
24. सं॒ॅव॒थ्स॒रमिति॑ सं - व॒थ्स॒रम् । \newline
25. उ॒प॒य न्त्य॑न्तर्वे॒ द्य॑न्तर्वे॒ द्यु॑प॒य न्त्यु॑प॒य न्त्य॑न्तर्वे॒दि मि॑थु॒नौ मि॑थु॒ना व॑न्तर्वे॒ द्यु॑प॒य न्त्यु॑प॒य न्त्य॑न्तर्वे॒दि मि॑थु॒नौ । \newline
26. उ॒प॒यन्तीत्यु॑प - यन्ति॑ । \newline
27. अ॒न्त॒र्वे॒दि मि॑थु॒नौ मि॑थु॒ना व॑न्तर्वे॒ द्य॑न्तर्वे॒दि मि॑थु॒नौ सꣳ सम् मि॑थु॒ना व॑न्तर्वे॒ द्य॑न्तर्वे॒दि मि॑थु॒नौ सम् । \newline
28. अ॒न्त॒र्वे॒दीत्य॑न्तः - वे॒दि । \newline
29. मि॒थु॒नौ सꣳ सम् मि॑थु॒नौ मि॑थु॒नौ सम् भ॑वतो भवतः॒ सम् मि॑थु॒नौ मि॑थु॒नौ सम् भ॑वतः । \newline
30. सम् भ॑वतो भवतः॒ सꣳ सम् भ॑वत॒ स्तेन॒ तेन॑ भवतः॒ सꣳ सम् भ॑वत॒ स्तेन॑ । \newline
31. भ॒व॒त॒ स्तेन॒ तेन॑ भवतो भवत॒ स्तेनै॒वैव तेन॑ भवतो भवत॒ स्तेनै॒व । \newline
32. तेनै॒वैव तेन॒ तेनै॒व मि॑थु॒नान् मि॑थु॒ना दे॒व तेन॒ तेनै॒व मि॑थु॒नात् । \newline
33. ए॒व मि॑थु॒नान् मि॑थु॒ना दे॒वैव मि॑थु॒नान् न न मि॑थु॒ना दे॒वैव मि॑थु॒नान् न । \newline
34. मि॒थु॒नान् न न मि॑थु॒नान् मि॑थु॒नान् न य॑न्ति यन्ति॒ न मि॑थु॒नान् मि॑थु॒नान् न य॑न्ति । \newline
35. न य॑न्ति यन्ति॒ न न य॑न्ति । \newline
36. य॒न्तीति॑ यन्ति । \newline
\pagebreak
\markright{ TS 7.5.10.1  \hfill https://www.vedavms.in \hfill}

\section{ TS 7.5.10.1 }

\textbf{TS 7.5.10.1 } \newline
\textbf{Samhita Paata} \newline

चर्माव॑ भिन्दन्ति पा॒प्मान॑मे॒वैषा॒मव॑ भिन्दन्ति॒ माऽप॑ राथ्सी॒र्माऽति॑ व्याथ्सी॒रित्या॑ह संप्र॒त्ये॑वैषां᳚ पा॒प्मान॒मव॑ भिन्दन्त्युदकु॒म्भान॑धिनि॒धाय॑ दा॒स्यो॑ मार्जा॒लीयं॒ परि॑ नृत्यन्ति प॒दो नि॑घ्न॒तीरि॒दंम॑धुं॒ गाय॑न्त्यो॒ मधु॒ वै दे॒वानां᳚ पर॒म-म॒न्नाद्यं॑ पर॒ममे॒वा-न्नाद्य॒मव॑ रुन्धते प॒दो नि घ्न॑न्ति मही॒यामे॒वैषु॑ दधति ॥ \newline

\textbf{Pada Paata} \newline

चर्म॑ । अवेति॑ । भि॒न्द॒न्ति॒ । पा॒प्मान᳚म् । ए॒व । ए॒षा॒म् । अवेति॑ । भि॒न्द॒न्ति॒ । मा । अपेति॑ । रा॒थ्सीः॒ । मा । अतीति॑ । व्या॒थ्सीः॒ । इति॑ । आ॒ह॒ । स॒प्रं॒तीति॑ सं - प्र॒ति । ए॒व । ए॒षा॒म् । पा॒प्मान᳚म् । अवेति॑ । भि॒न्द॒न्ति॒ । उ॒द॒कु॒म्भानित्यु॑द-कु॒म्भान् । अ॒धि॒नि॒धायेत्य॑धि - नि॒धाय॑ । दा॒स्यः॑ । मा॒र्जा॒लीय᳚म् । परीति॑ । नृ॒त्य॒न्ति॒ । प॒दः । नि॒घ्न॒तीरिति॑ नि - घ्न॒तीः । इ॒दंम॑धु॒मिती॒दं - म॒धु॒म् । गाय॑न्त्यः । मधु॑ । वै । दे॒वाना᳚म् । प॒र॒मम् । अ॒न्नाद्य॒मित्य॑न्न - अद्य᳚म् । प॒र॒मम् । ए॒व । अ॒न्नाद्य॒मित्य॑न्न - अद्य᳚म् । अवेति॑ । रु॒न्ध॒ते॒ । प॒दः । नीति॑ । घ्न॒न्ति॒ । म॒ही॒याम् । ए॒व । ए॒षु॒ । द॒ध॒ति॒ ॥  \newline


\textbf{Krama Paata} \newline

चर्माव॑ । अव॑ भिन्दन्ति । भि॒न्द॒न्ति॒ पा॒प्मान᳚म् । पा॒प्मान॑मे॒व । ए॒वैषा᳚म् । ए॒षा॒मव॑ । अव॑ भिन्दन्ति । भि॒न्द॒न्ति॒ मा । माऽप॑ । अप॑ राथ्सीः । रा॒थ्सी॒र् मा । माऽति॑ । अति॑ व्याथ्सीः । व्या॒थ्सी॒रिति॑ । इत्या॑ह । आ॒ह॒ स॒म्प्र॒ति । स॒म्प्र॒त्ये॑व । स॒म्प्र॒तीति॑ सम् - प्र॒ति । ए॒वैषा᳚म् । ए॒षा॒म् पा॒प्मान᳚म् । पा॒प्मान॒मव॑ । अव॑ भिन्दन्ति । भि॒न्द॒न्त्यु॒द॒कु॒म्भान् । उ॒द॒कु॒म्भान॑धिनि॒धाय॑ । उ॒द॒कु॒म्भानित्यु॑द - कु॒म्भान् । अ॒धि॒नि॒धाय॑ दा॒स्यः॑ । अ॒दि॒नि॒धायेत्य॑धि - नि॒धाय॑ । दा॒स्यो॑ मार्जा॒लीय᳚म् । मा॒र्जा॒लीय॒म् परि॑ । परि॑ नृत्यन्ति । नृ॒त्य॒न्ति॒ प॒दः । प॒दो नि॑घ्न॒तीः । नि॒घ्न॒तीरि॒दम्म॑धुम् । नि॒घ्न॒तीरिति॑ नि - घ्न॒तीः । इ॒दम्म॑धु॒म् गाय॑न्त्यः । इ॒दम्म॑धु॒मिती॒दम् - म॒धु॒म् । गाय॑न्त्यो॒ मधु॑ । मधु॒ वै । वै दे॒वाना᳚म् । दे॒वाना᳚म् पर॒मम् । प॒र॒मम॒न्नाद्य᳚म् । अ॒न्नाद्य॑म् पर॒मम् । अ॒न्नाद्य॒मित्य॑न्न - अद्य᳚म् । प॒र॒ममे॒व । ए॒वान्नाद्य᳚म् । अ॒न्नाद्य॒मव॑ । अ॒न्नाद्य॒मित्य॑न्न - अद्य᳚म् । अव॑ रुन्धते । रु॒न्ध॒ते॒ प॒दः । प॒दो निः । निर् घ्न॑न्ति । घ्न॒न्ति॒ म॒ही॒याम् । म॒ही॒यामे॒व । ए॒वैषु॑ । ए॒षु॒ द॒ध॒ति॒ । द॒ध॒तीति॑ दधति । \newline

\textbf{Jatai Paata} \newline

1. चर्मा वाव॒ चर्म॒ चर्माव॑ । \newline
2. अव॑ भिन्दन्ति भिन्द॒ न्त्यवाव॑ भिन्दन्ति । \newline
3. भि॒न्द॒न्ति॒ पा॒प्मान॑म् पा॒प्मान॑म् भिन्दन्ति भिन्दन्ति पा॒प्मान᳚म् । \newline
4. पा॒प्मान॑ मे॒वैव पा॒प्मान॑म् पा॒प्मान॑ मे॒व । \newline
5. ए॒वैषा॑ मेषा मे॒वै वैषा᳚म् । \newline
6. ए॒षा॒ मवा वै॑षा मेषा॒ मव॑ । \newline
7. अव॑ भिन्दन्ति भिन्द॒ न्त्यवाव॑ भिन्दन्ति । \newline
8. भि॒न्द॒न्ति॒ मा मा भि॑न्दन्ति भिन्दन्ति॒ मा । \newline
9. मा ऽपाप॒ मा मा ऽप॑ । \newline
10. अप॑ राथ्सी राथ्सी॒ रपाप॑ राथ्सीः । \newline
11. रा॒थ्सी॒र् मा मा रा᳚थ्सी राथ्सी॒र् मा । \newline
12. मा ऽत्यति॒ मा मा ऽति॑ । \newline
13. अति॑ व्याथ्सीर् व्याथ्सी॒ रत्यति॑ व्याथ्सीः । \newline
14. व्या॒थ्सी॒ रितीति॑ व्याथ्सीर् व्याथ्सी॒ रिति॑ । \newline
15. इत्या॑हा॒हे तीत्या॑ह । \newline
16. आ॒ह॒ सं॒प्र॒ति सं॑प्र॒ त्या॑हाह संप्र॒ति । \newline
17. सं॒प्र॒ त्ये॑वैव सं॑प्र॒ति सं॑प्र॒ त्ये॑व । \newline
18. सं॒प्र॒तीति॑ सं - प्र॒ति । \newline
19. ए॒वैषा॑ मेषा मे॒वै वैषा᳚म् । \newline
20. ए॒षा॒म् पा॒प्मान॑म् पा॒प्मान॑ मेषा मेषाम् पा॒प्मान᳚म् । \newline
21. पा॒प्मान॒ मवाव॑ पा॒प्मान॑म् पा॒प्मान॒ मव॑ । \newline
22. अव॑ भिन्दन्ति भिन्द॒ न्त्यवाव॑ भिन्दन्ति । \newline
23. भि॒न्द॒ न्त्यु॒द॒कु॒म्भा नु॑दकु॒म्भान् भि॑न्दन्ति भिन्द न्त्युदकु॒म्भान् । \newline
24. उ॒द॒कु॒म्भा न॑धिनि॒धाया॑ धिनि॒धा यो॑दकु॒म्भा नु॑दकु॒म्भा न॑धिनि॒धाय॑ । \newline
25. उ॒द॒कु॒म्भानित्यु॑द - कु॒म्भान् । \newline
26. अ॒धि॒नि॒धाय॑ दा॒स्यो॑ दा॒स्यो॑ ऽधिनि॒धाया॑ धिनि॒धाय॑ दा॒स्यः॑ । \newline
27. अ॒धि॒नि॒धायेत्य॑धि - नि॒धाय॑ । \newline
28. दा॒स्यो॑ मार्जा॒लीय॑म् मार्जा॒लीय॑म् दा॒स्यो॑ दा॒स्यो॑ मार्जा॒लीय᳚म् । \newline
29. मा॒र्जा॒लीय॒म् परि॒ परि॑ मार्जा॒लीय॑म् मार्जा॒लीय॒म् परि॑ । \newline
30. परि॑ नृत्यन्ति नृत्यन्ति॒ परि॒ परि॑ नृत्यन्ति । \newline
31. नृ॒त्य॒न्ति॒ प॒दः प॒दो नृ॑त्यन्ति नृत्यन्ति प॒दः । \newline
32. प॒दो नि॑घ्न॒तीर् नि॑घ्न॒तीः प॒दः प॒दो नि॑घ्न॒तीः । \newline
33. नि॒घ्न॒ती रि॒दम्म॑धु मि॒दम्म॑धुन् निघ्न॒तीर् नि॑घ्न॒ती रि॒दम्म॑धुम् । \newline
34. नि॒घ्न॒तीरिति॑ नि - घ्न॒तीः । \newline
35. इ॒दम्म॑धु॒म् गाय॑न्त्यो॒ गाय॑न्त्य इ॒दम्म॑धु मि॒दम्म॑धु॒म् गाय॑न्त्यः । \newline
36. इ॒दम्म॑धु॒मिती॒दं - म॒धु॒म् । \newline
37. गाय॑न्त्यो॒ मधु॒ मधु॒ गाय॑न्त्यो॒ गाय॑न्त्यो॒ मधु॑ । \newline
38. मधु॒ वै वै मधु॒ मधु॒ वै । \newline
39. वै दे॒वाना᳚म् दे॒वानां॒ ॅवै वै दे॒वाना᳚म् । \newline
40. दे॒वाना᳚म् पर॒मम् प॑र॒मम् दे॒वाना᳚म् दे॒वाना᳚म् पर॒मम् । \newline
41. प॒र॒म म॒न्नाद्य॑ म॒न्नाद्य॑म् पर॒मम् प॑र॒म म॒न्नाद्य᳚म् । \newline
42. अ॒न्नाद्य॑म् पर॒मम् प॑र॒म म॒न्नाद्य॑ म॒न्नाद्य॑म् पर॒मम् । \newline
43. अ॒न्नाद्य॒मित्य॑न्न - अद्य᳚म् । \newline
44. प॒र॒म मे॒वैव प॑र॒मम् प॑र॒म मे॒व । \newline
45. ए॒वान्नाद्य॑ म॒न्नाद्य॑ मे॒वै वान्नाद्य᳚म् । \newline
46. अ॒न्नाद्य॒ मवा वा॒न्नाद्य॑ म॒न्नाद्य॒ मव॑ । \newline
47. अ॒न्नाद्य॒मित्य॑न्न - अद्य᳚म् । \newline
48. अव॑ रुन्धते रुन्ध॒ते ऽवाव॑ रुन्धते । \newline
49. रु॒न्ध॒ते॒ प॒दः प॒दो रु॑न्धते रुन्धते प॒दः । \newline
50. प॒दो नि नि प॒दः प॒दो नि । \newline
51. नि घ्न॑न्ति घ्नन्ति॒ नि नि घ्न॑न्ति । \newline
52. घ्न॒न्ति॒ म॒ही॒याम् म॑ही॒याम् घ्न॑न्ति घ्नन्ति मही॒याम् । \newline
53. म॒ही॒या मे॒वैव म॑ही॒याम् म॑ही॒या मे॒व । \newline
54. ए॒वैष्वे᳚ ष्वे॒वै वैषु॑ । \newline
55. ए॒षु॒ द॒ध॒ति॒ द॒ध॒ त्ये॒ष्वे॒षु॒ द॒ध॒ति॒ । \newline
56. द॒ध॒तीति॑ दधति । \newline

\textbf{Ghana Paata } \newline

1. चर्मा वाव॒ चर्म॒ चर्माव॑ भिन्दन्ति भिन्द॒ न्त्यव॒ चर्म॒ चर्माव॑ भिन्दन्ति । \newline
2. अव॑ भिन्दन्ति भिन्द॒ न्त्यवाव॑ भिन्दन्ति पा॒प्मान॑म् पा॒प्मान॑म् भिन्द॒ न्त्यवाव॑ भिन्दन्ति पा॒प्मान᳚म् । \newline
3. भि॒न्द॒न्ति॒ पा॒प्मान॑म् पा॒प्मान॑म् भिन्दन्ति भिन्दन्ति पा॒प्मान॑ मे॒वैव पा॒प्मान॑म् भिन्दन्ति भिन्दन्ति पा॒प्मान॑ मे॒व । \newline
4. पा॒प्मान॑ मे॒वैव पा॒प्मान॑म् पा॒प्मान॑ मे॒वैषा॑ मेषा मे॒व पा॒प्मान॑म् पा॒प्मान॑ मे॒वैषा᳚म् । \newline
5. ए॒वैषा॑ मेषा मे॒वै वैषा॒ मवा वै॑षा मे॒वै वैषा॒ मव॑ । \newline
6. ए॒षा॒ मवा वै॑षा मेषा॒ मव॑ भिन्दन्ति भिन्द॒ न्त्यवै॑षा मेषा॒ मव॑ भिन्दन्ति । \newline
7. अव॑ भिन्दन्ति भिन्द॒ न्त्यवाव॑ भिन्दन्ति॒ मा मा भि॑न्द॒ न्त्यवाव॑ भिन्दन्ति॒ मा । \newline
8. भि॒न्द॒न्ति॒ मा मा भि॑न्दन्ति भिन्दन्ति॒ मा ऽपाप॒ मा भि॑न्दन्ति भिन्दन्ति॒ मा ऽप॑ । \newline
9. मा ऽपाप॒ मा मा ऽप॑ राथ्सी राथ्सी॒ रप॒ मा मा ऽप॑ राथ्सीः । \newline
10. अप॑ राथ्सी राथ्सी॒ रपाप॑ राथ्सी॒र् मा मा रा᳚थ्सी॒ रपाप॑ राथ्सी॒र् मा । \newline
11. रा॒थ्सी॒र् मा मा रा᳚थ्सी राथ्सी॒र् मा ऽत्यति॒ मा रा᳚थ्सी राथ्सी॒र् मा ऽति॑ । \newline
12. मा ऽत्यति॒ मा मा ऽति॑ व्याथ्सीर् व्याथ्सी॒रति॒ मा मा ऽति॑ व्याथ्सीः । \newline
13. अति॑ व्याथ्सीर् व्याथ्सी॒ रत्यति॑ व्याथ्सी॒ रितीति॑ व्याथ्सी॒ रत्यति॑ व्याथ्सी॒ रिति॑ । \newline
14. व्या॒थ्सी॒ रितीति॑ व्याथ्सीर् व्याथ्सी॒ रित्या॑हा॒ हेति॑ व्याथ्सीर् व्याथ्सी॒ रित्या॑ह । \newline
15. इत्या॑हा॒हे तीत्या॑ह संप्र॒ति सं॑प्र॒ त्या॑हे तीत्या॑ह संप्र॒ति । \newline
16. आ॒ह॒ सं॒प्र॒ति सं॑प्र॒त्या॑हाह संप्र॒त्ये॑वैव सं॑प्र॒त्या॑हाह संप्र॒त्ये॑व । \newline
17. सं॒प्र॒त्ये॑ वैव सं॑प्र॒ति सं॑प्र॒ त्ये॑वैषा॑ मेषा मे॒व सं॑प्र॒ति सं॑प्र॒ त्ये॑वैषा᳚म् । \newline
18. सं॒प्र॒तीति॑ सं - प्र॒ति । \newline
19. ए॒वैषा॑ मेषा मे॒वै वैषा᳚म् पा॒प्मान॑म् पा॒प्मान॑ मेषा मे॒वै वैषा᳚म् पा॒प्मान᳚म् । \newline
20. ए॒षा॒म् पा॒प्मान॑म् पा॒प्मान॑ मेषा मेषाम् पा॒प्मान॒ मवाव॑ पा॒प्मान॑ मेषा मेषाम् पा॒प्मान॒ मव॑ । \newline
21. पा॒प्मान॒ मवाव॑ पा॒प्मान॑म् पा॒प्मान॒ मव॑ भिन्दन्ति भिन्द॒ न्त्यव॑ पा॒प्मान॑म् पा॒प्मान॒ मव॑ भिन्दन्ति । \newline
22. अव॑ भिन्दन्ति भिन्द॒ न्त्यवाव॑ भिन्द न्त्युदकु॒म्भा नु॑दकु॒म्भान् भि॑न्द॒ न्त्यवाव॑ भिन्द न्त्युदकु॒म्भान् । \newline
23. भि॒न्द॒ न्त्यु॒द॒कु॒म्भा नु॑दकु॒म्भान् भि॑न्दन्ति भिन्द न्त्युदकु॒म्भा न॑धिनि॒धाया॑ धिनि॒धायो॑ दकु॒म्भान् भि॑न्दन्ति भिन्द न्त्युदकु॒म्भा न॑धिनि॒धाय॑ । \newline
24. उ॒द॒कु॒म्भा न॑धिनि॒धाया॑ धिनि॒धायो॑ दकु॒म्भा नु॑दकु॒म्भा न॑धिनि॒धाय॑ दा॒स्यो॑ दा॒स्यो॑ ऽधिनि॒धायो॑ दकु॒म्भा नु॑दकु॒म्भा न॑धिनि॒धाय॑ दा॒स्यः॑ । \newline
25. उ॒द॒कु॒म्भानित्यु॑द - कु॒म्भान् । \newline
26. अ॒धि॒नि॒धाय॑ दा॒स्यो॑ दा॒स्यो॑ ऽधिनि॒धाया॑ धिनि॒धाय॑ दा॒स्यो॑ मार्जा॒लीय॑म् मार्जा॒लीय॑म् दा॒स्यो॑ ऽधिनि॒धाया॑ धिनि॒धाय॑ दा॒स्यो॑ मार्जा॒लीय᳚म् । \newline
27. अ॒धि॒नि॒धायेत्य॑धि - नि॒धाय॑ । \newline
28. दा॒स्यो॑ मार्जा॒लीय॑म् मार्जा॒लीय॑म् दा॒स्यो॑ दा॒स्यो॑ मार्जा॒लीय॒म् परि॒ परि॑ मार्जा॒लीय॑म् दा॒स्यो॑ दा॒स्यो॑ मार्जा॒लीय॒म् परि॑ । \newline
29. मा॒र्जा॒लीय॒म् परि॒ परि॑ मार्जा॒लीय॑म् मार्जा॒लीय॒म् परि॑ नृत्यन्ति नृत्यन्ति॒ परि॑ मार्जा॒लीय॑म् मार्जा॒लीय॒म् परि॑ नृत्यन्ति । \newline
30. परि॑ नृत्यन्ति नृत्यन्ति॒ परि॒ परि॑ नृत्यन्ति प॒दः प॒दो नृ॑त्यन्ति॒ परि॒ परि॑ नृत्यन्ति प॒दः । \newline
31. नृ॒त्य॒न्ति॒ प॒दः प॒दो नृ॑त्यन्ति नृत्यन्ति प॒दो नि॑घ्न॒तीर् नि॑घ्न॒तीः प॒दो नृ॑त्यन्ति नृत्यन्ति प॒दो नि॑घ्न॒तीः । \newline
32. प॒दो नि॑घ्न॒तीर् नि॑घ्न॒तीः प॒दः प॒दो नि॑घ्न॒ती रि॒दम्म॑धु मि॒दम्म॑धुन् निघ्न॒तीः प॒दः प॒दो नि॑घ्न॒ती रि॒दम्म॑धुम् । \newline
33. नि॒घ्न॒ती रि॒दम्म॑धु मि॒दम्म॑धुन् निघ्न॒तीर् नि॑घ्न॒ती रि॒दम्म॑धु॒म् गाय॑न्त्यो॒ गाय॑न्त्य इ॒दम्म॑धुन् निघ्न॒तीर् नि॑घ्न॒ती रि॒दम्म॑धु॒म् गाय॑न्त्यः । \newline
34. नि॒घ्न॒तीरिति॑ नि - घ्न॒तीः । \newline
35. इ॒दम्म॑धु॒म् गाय॑न्त्यो॒ गाय॑न्त्य इ॒दम्म॑धु मि॒दम्म॑धु॒म् गाय॑न्त्यो॒ मधु॒ मधु॒ गाय॑न्त्य इ॒दम्म॑धु मि॒दम्म॑धु॒म् गाय॑न्त्यो॒ मधु॑ । \newline
36. इ॒दम्म॑धु॒मिती॒दं - म॒धु॒म् । \newline
37. गाय॑न्त्यो॒ मधु॒ मधु॒ गाय॑न्त्यो॒ गाय॑न्त्यो॒ मधु॒ वै वै मधु॒ गाय॑न्त्यो॒ गाय॑न्त्यो॒ मधु॒ वै । \newline
38. मधु॒ वै वै मधु॒ मधु॒ वै दे॒वाना᳚म् दे॒वानां॒ ॅवै मधु॒ मधु॒ वै दे॒वाना᳚म् । \newline
39. वै दे॒वाना᳚म् दे॒वानां॒ ॅवै वै दे॒वाना᳚म् पर॒मम् प॑र॒मम् दे॒वानां॒ ॅवै वै दे॒वाना᳚म् पर॒मम् । \newline
40. दे॒वाना᳚म् पर॒मम् प॑र॒मम् दे॒वाना᳚म् दे॒वाना᳚म् पर॒म म॒न्नाद्य॑ म॒न्नाद्य॑म् पर॒मम् दे॒वाना᳚म् दे॒वाना᳚म् पर॒म म॒न्नाद्य᳚म् । \newline
41. प॒र॒म म॒न्नाद्य॑ म॒न्नाद्य॑म् पर॒मम् प॑र॒म म॒न्नाद्य॑म् पर॒मम् प॑र॒म म॒न्नाद्य॑म् पर॒मम् प॑र॒म म॒न्नाद्य॑म् पर॒मम् । \newline
42. अ॒न्नाद्य॑म् पर॒मम् प॑र॒म म॒न्नाद्य॑ म॒न्नाद्य॑म् पर॒म मे॒वैव प॑र॒म म॒न्नाद्य॑ म॒न्नाद्य॑म् पर॒म मे॒व । \newline
43. अ॒न्नाद्य॒मित्य॑न्न - अद्य᳚म् । \newline
44. प॒र॒म मे॒वैव प॑र॒मम् प॑र॒म मे॒वान्नाद्य॑ म॒न्नाद्य॑ मे॒व प॑र॒मम् प॑र॒म मे॒वान्नाद्य᳚म् । \newline
45. ए॒वान्नाद्य॑ म॒न्नाद्य॑ मे॒वै वान्नाद्य॒ मवावा॒न्नाद्य॑ मे॒वै वान्नाद्य॒ मव॑ । \newline
46. अ॒न्नाद्य॒ मवा वा॒न्नाद्य॑ म॒न्नाद्य॒ मव॑ रुन्धते रुन्ध॒ते ऽवा॒न्नाद्य॑ म॒न्नाद्य॒ मव॑ रुन्धते । \newline
47. अ॒न्नाद्य॒मित्य॑न्न - अद्य᳚म् । \newline
48. अव॑ रुन्धते रुन्ध॒ते ऽवाव॑ रुन्धते प॒दः प॒दो रु॑न्ध॒ते ऽवाव॑ रुन्धते प॒दः । \newline
49. रु॒न्ध॒ते॒ प॒दः प॒दो रु॑न्धते रुन्धते प॒दो नि नि प॒दो रु॑न्धते रुन्धते प॒दो नि । \newline
50. प॒दो नि नि प॒दः प॒दो नि घ्न॑न्ति घ्नन्ति॒ नि प॒दः प॒दो नि घ्न॑न्ति । \newline
51. नि घ्न॑न्ति घ्नन्ति॒ नि नि घ्न॑न्ति मही॒याम् म॑ही॒याम् घ्न॑न्ति॒ नि नि घ्न॑न्ति मही॒याम् । \newline
52. घ्न॒न्ति॒ म॒ही॒याम् म॑ही॒याम् घ्न॑न्ति घ्नन्ति मही॒या मे॒वैव म॑ही॒याम् घ्न॑न्ति घ्नन्ति मही॒या मे॒व । \newline
53. म॒ही॒या मे॒वैव म॑ही॒याम् म॑ही॒या मे॒वै ष्वे᳚ष्वे॒व म॑ही॒याम् म॑ही॒या मे॒वैषु॑ । \newline
54. ए॒वैष्वे᳚ष्वे॒ वैवैषु॑ दधति दध त्येष्वे॒ वैवैषु॑ दधति । \newline
55. ए॒षु॒ द॒ध॒ति॒ द॒ध॒ त्ये॒ष्वे॒षु॒ द॒ध॒ति॒ । \newline
56. द॒ध॒तीति॑ दधति । \newline
\pagebreak
\markright{ TS 7.5.11.1  \hfill https://www.vedavms.in \hfill}

\section{ TS 7.5.11.1 }

\textbf{TS 7.5.11.1 } \newline
\textbf{Samhita Paata} \newline

पृ॒थि॒व्यै स्वाहा॒ ऽन्तरि॑क्षाय॒ स्वाहा॑ दि॒वे स्वाहा॑ संप्लोष्य॒ते स्वाहा॑ स॒प्लंव॑मानाय॒ स्वाहा॒ संप्लु॑ताय॒ स्वाहा॑ मेघायिष्य॒ते स्वाहा॑ मेघाय॒ते स्वाहा॑ मेघि॒ताय॒ स्वाहा॑ मे॒घाय॒ स्वाहा॑ नीहा॒राय॒ स्वाहा॑ नि॒हाका॑यै॒ स्वाहा᳚ प्रास॒चाय॒ स्वाहा᳚ प्रच॒लाका॑यै॒ स्वाहा॑ विद्योतिष्य॒ते स्वाहा॑ वि॒द्योत॑मानाय॒ स्वाहा॑ संॅवि॒द्योत॑मानाय॒ स्वाहा᳚ स्तनयिष्य॒ते स्वाहा᳚ स्त॒नय॑ते॒ स्वाहो॒ -ग्रꣳ स्त॒नय॑ते॒ स्वाहा॑ वर्.षिष्य॒ते स्वाहा॒ वर्.ष॑ते॒ स्वाहा॑ ऽभि॒वर्.ष॑ते॒ स्वाहा॑ परि॒वर्.ष॑ते॒ स्वाहा॑ सं॒ॅवर्.ष॑ते॒ - [  ] \newline

\textbf{Pada Paata} \newline

पृ॒थि॒व्यै । स्वाहा᳚ । अ॒न्तरि॑क्षाय । स्वाहा᳚ । दि॒वे । स्वाहा᳚ । स॒प्ल्ॐ॒ष्य॒त इति॑ सं - प्लो॒ष्य॒ते । स्वाहा᳚ । स॒प्लंव॑माना॒येति॑ सं - प्लव॑मानाय । स्वाहा᳚ । संप्लु॑ता॒येति॒ सं-प्लु॒ता॒य॒ । स्वाहा᳚ । मे॒घा॒यि॒ष्य॒ते । स्वाहा᳚ । मे॒घा॒य॒त इति॑ मेघ - य॒ते । स्वाहा᳚ । मे॒घि॒ताय॑ । स्वाहा᳚ । मे॒घाय॑ । स्वाहा᳚ । नी॒हा॒राय॑ । स्वाहा᳚ । नि॒हाका॑या॒ इति॑ नि-हाका॑यै । स्वाहा᳚ । प्रा॒स॒चाय॑ । स्वाहा᳚ । प्र॒च॒लाका॑या॒ इति॑ प्र - च॒लाका॑यै । स्वाहा᳚ । वि॒द्यो॒ति॒ष्य॒त इति॑ वि - द्यो॒ति॒ष्य॒ते । स्वाहा᳚ । वि॒द्योत॑माना॒येति॑ वि - द्योत॑मानाय । स्वाहा᳚ । सं॒ॅवि॒द्योत॑माना॒येति॑ सं-वि॒द्योत॑मानाय । स्वाहा᳚ । स्त॒न॒यि॒ष्य॒ते । स्वाहा᳚ । स्त॒नय॑ते । स्वाहा᳚ । उ॒ग्रम् । स्त॒नय॑ते । स्वाहा᳚ । व॒र्.॒षि॒ष्य॒ते । स्वाहा᳚ । वर्.ष॑ते । स्वाहा᳚ । अ॒भि॒वर्.ष॑त॒ इत्य॑भि - वर्.ष॑ते । स्वाहा᳚ । प॒रि॒वर्.ष॑त॒ इति॑ परि - वर्.ष॑ते । स्वाहा᳚ । सं॒ॅवर्.ष॑त॒ इति॑ सं - वर्.ष॑ते ।  \newline


\textbf{Krama Paata} \newline

पृ॒थि॒व्यै स्वाहा᳚ । स्वाहा॒ऽन्तरि॑क्षाय । अ॒न्तरि॑क्षाय॒ स्वाहा᳚ । स्वाहा॑ दि॒वे । दि॒वे स्वाहा᳚ । स्वाहा॑ सम्प्लोष्य॒ते । स॒म्प्लो॒ष्य॒ते स्वाहा᳚ । स॒म्प्लो॒ष्य॒त इति॑ सम् - प्लो॒ष्य॒ते । स्वाहा॑ स॒म्प्लव॑मानाय । स॒म्प्लव॑मानाय॒ स्वाहा᳚ । स॒म्प्लव॑माना॒येति॑ सम् - प्लव॑मानाय । स्वाहा॒ सम्प्लु॑ताय । सम्प्लु॑ताय॒ स्वाहा᳚ । सम्प्लु॑ता॒येति॒ सम् - प्लु॒ता॒य॒ । स्वाहा॑ मेघायिष्य॒ते । मे॒घा॒यि॒ष्य॒ते स्वाहा᳚ । स्वाहा॑ मेघाय॒ते । मे॒घा॒य॒ते स्वाहा᳚ । मे॒घा॒य॒त इति॑ मेघ - य॒ते । स्वाहा॑ मेघि॒ताय॑ । मे॒घि॒ताय॒ स्वाहा᳚ । स्वाहा॑ मे॒घाय॑ । मे॒घाय॒ स्वाहा᳚ । स्वाहा॑ नीहा॒राय॑ । नी॒हा॒राय॒ स्वाहा᳚ । स्वाहा॑ नि॒हाका॑यै । नि॒हाका॑यै॒ स्वाहा᳚ । नि॒हाका॑या॒ इति॑ नि - हाका॑यै । स्वाहा᳚ प्रास॒चाय॑ । प्रा॒स॒चाय॒ स्वाहा᳚ । स्वाहा᳚ प्रच॒लाका॑यै । प्र॒च॒लाका॑यै॒ स्वाहा᳚ । प्र॒च॒लाका॑या॒ इति॑ प्र - च॒लाका॑यै । स्वाहा॑ विद्योतिष्य॒ते । वि॒द्यो॒ति॒ष्य॒ते स्वाहा᳚ । वि॒द्यो॒ति॒ष्य॒त इति॑ वि - द्यो॒ति॒ष्य॒ते । स्वाहा॑ वि॒द्योत॑मानाय । वि॒द्योत॑मानाय॒ स्वाहा᳚ । वि॒द्योत॑माना॒येति॑ वि - द्योत॑मानाय । स्वाहा॑ सम्ॅवि॒द्योत॑मानाय । स॒म्ॅवि॒द्योत॑मानाय॒ स्वाहा᳚ । स॒म्ॅवि॒द्योत॑मान॒येति॑ सम् - वि॒द्योत॑मानाय । स्वाहा᳚ स्तनयिष्य॒ते । स्त॒न॒यि॒ष्य॒ते स्वाहा᳚ । स्वाहा᳚ स्त॒नय॑ते । स्त॒नय॑ते॒ स्वाहा᳚ । स्वाहो॒ग्रम् । उ॒ग्रꣳ स्त॒नय॑ते । स्त॒नय॑ते॒ स्वाहा᳚ । स्वाहा॑ वर्.षिष्य॒ते । व॒र्॒.षि॒ष्य॒ते स्वाहा᳚ । स्वाहा॒ वर्.ष॑ते । वर्.ष॑ते॒ स्वाहा᳚ । स्वाहा॑ऽभि॒वर्.ष॑ते । अ॒भि॒वर्.ष॑ते॒ स्वाहा᳚ । अ॒भि॒वर्.ष॑त॒ इत्य॑भि - वर्.ष॑ते । स्वाहा॑ परि॒वर्.ष॑ते । प॒रि॒वर्.ष॑ते॒ स्वाहा᳚ । प॒रि॒वर्.ष॑त॒ इति॑ परि - वर्.ष॑ते । स्वाहा॑ स॒म्ॅवर्.ष॑ते । स॒म्ॅवर्.ष॑ते॒ स्वाहा᳚ । स॒म्ॅवर्.ष॑त॒ इति॑ सम् - वर्.ष॑ते \newline

\textbf{Jatai Paata} \newline

1. पृ॒थि॒व्यै स्वाहा॒ स्वाहा॑ पृथि॒व्यै पृ॑थि॒व्यै स्वाहा᳚ । \newline
2. स्वाहा॒ ऽन्तरि॑क्षाया॒ न्तरि॑क्षाय॒ स्वाहा॒ स्वाहा॒ ऽन्तरि॑क्षाय । \newline
3. अ॒न्तरि॑क्षाय॒ स्वाहा॒ स्वाहा॒ ऽन्तरि॑क्षाया॒ न्तरि॑क्षाय॒ स्वाहा᳚ । \newline
4. स्वाहा॑ दि॒वे दि॒वे स्वाहा॒ स्वाहा॑ दि॒वे । \newline
5. दि॒वे स्वाहा॒ स्वाहा॑ दि॒वे दि॒वे स्वाहा᳚ । \newline
6. स्वाहा॑ संप्लोष्य॒ते सं॑प्लोष्य॒ते स्वाहा॒ स्वाहा॑ संप्लोष्य॒ते । \newline
7. सं॒प्लो॒ष्य॒ते स्वाहा॒ स्वाहा॑ संप्लोष्य॒ते सं॑प्लोष्य॒ते स्वाहा᳚ । \newline
8. सं॒प्लो॒ष्य॒त इति॑ सं - प्लो॒ष्य॒ते । \newline
9. स्वाहा॑ सं॒प्लव॑मानाय सं॒प्लव॑मानाय॒ स्वाहा॒ स्वाहा॑ सं॒प्लव॑मानाय । \newline
10. सं॒प्लव॑मानाय॒ स्वाहा॒ स्वाहा॑ सं॒प्लव॑मानाय सं॒प्लव॑मानाय॒ स्वाहा᳚ । \newline
11. सं॒प्लव॑माना॒येति॑ सं - प्लव॑मानाय । \newline
12. स्वाहा॒ संप्लु॑ताय॒ संप्लु॑ताय॒ स्वाहा॒ स्वाहा॒ संप्लु॑ताय । \newline
13. संप्लु॑ताय॒ स्वाहा॒ स्वाहा॒ संप्लु॑ताय॒ संप्लु॑ताय॒ स्वाहा᳚ । \newline
14. संप्लु॑ता॒येति॒ सं - प्लु॒ता॒य॒ । \newline
15. स्वाहा॑ मेघायिष्य॒ते मे॑घायिष्य॒ते स्वाहा॒ स्वाहा॑ मेघायिष्य॒ते । \newline
16. मे॒घा॒यि॒ष्य॒ते स्वाहा॒ स्वाहा॑ मेघायिष्य॒ते मे॑घायिष्य॒ते स्वाहा᳚ । \newline
17. स्वाहा॑ मेघाय॒ते मे॑घाय॒ते स्वाहा॒ स्वाहा॑ मेघाय॒ते । \newline
18. मे॒घा॒य॒ते स्वाहा॒ स्वाहा॑ मेघाय॒ते मे॑घाय॒ते स्वाहा᳚ । \newline
19. मे॒घा॒य॒त इति॑ मेघ - य॒ते । \newline
20. स्वाहा॑ मेघि॒ताय॑ मेघि॒ताय॒ स्वाहा॒ स्वाहा॑ मेघि॒ताय॑ । \newline
21. मे॒घि॒ताय॒ स्वाहा॒ स्वाहा॑ मेघि॒ताय॑ मेघि॒ताय॒ स्वाहा᳚ । \newline
22. स्वाहा॑ मे॒घाय॑ मे॒घाय॒ स्वाहा॒ स्वाहा॑ मे॒घाय॑ । \newline
23. मे॒घाय॒ स्वाहा॒ स्वाहा॑ मे॒घाय॑ मे॒घाय॒ स्वाहा᳚ । \newline
24. स्वाहा॑ नीहा॒राय॑ नीहा॒राय॒ स्वाहा॒ स्वाहा॑ नीहा॒राय॑ । \newline
25. नी॒हा॒राय॒ स्वाहा॒ स्वाहा॑ नीहा॒राय॑ नीहा॒राय॒ स्वाहा᳚ । \newline
26. स्वाहा॑ नि॒हाका॑यै नि॒हाका॑यै॒ स्वाहा॒ स्वाहा॑ नि॒हाका॑यै । \newline
27. नि॒हाका॑यै॒ स्वाहा॒ स्वाहा॑ नि॒हाका॑यै नि॒हाका॑यै॒ स्वाहा᳚ । \newline
28. नि॒हाका॑या॒ इति॑ नि - हाका॑यै । \newline
29. स्वाहा᳚ प्रास॒चाय॑ प्रास॒चाय॒ स्वाहा॒ स्वाहा᳚ प्रास॒चाय॑ । \newline
30. प्रा॒स॒चाय॒ स्वाहा॒ स्वाहा᳚ प्रास॒चाय॑ प्रास॒चाय॒ स्वाहा᳚ । \newline
31. स्वाहा᳚ प्रच॒लाका॑यै प्रच॒लाका॑यै॒ स्वाहा॒ स्वाहा᳚ प्रच॒लाका॑यै । \newline
32. प्र॒च॒लाका॑यै॒ स्वाहा॒ स्वाहा᳚ प्रच॒लाका॑यै प्रच॒लाका॑यै॒ स्वाहा᳚ । \newline
33. प्र॒च॒लाका॑या॒ इति॑ प्र - च॒लाका॑यै । \newline
34. स्वाहा॑ विद्योतिष्य॒ते वि॑द्योतिष्य॒ते स्वाहा॒ स्वाहा॑ विद्योतिष्य॒ते । \newline
35. वि॒द्यो॒ति॒ष्य॒ते स्वाहा॒ स्वाहा॑ विद्योतिष्य॒ते वि॑द्योतिष्य॒ते स्वाहा᳚ । \newline
36. वि॒द्यो॒ति॒ष्य॒त इति॑ वि - द्यो॒ति॒ष्य॒ते । \newline
37. स्वाहा॑ वि॒द्योत॑मानाय वि॒द्योत॑मानाय॒ स्वाहा॒ स्वाहा॑ वि॒द्योत॑मानाय । \newline
38. वि॒द्योत॑मानाय॒ स्वाहा॒ स्वाहा॑ वि॒द्योत॑मानाय वि॒द्योत॑मानाय॒ स्वाहा᳚ । \newline
39. वि॒द्योत॑माना॒येति॑ वि - द्योत॑मानाय । \newline
40. स्वाहा॑ संॅवि॒द्योत॑मानाय संॅवि॒द्योत॑मानाय॒ स्वाहा॒ स्वाहा॑ संॅवि॒द्योत॑मानाय । \newline
41. सं॒ॅवि॒द्योत॑मानाय॒ स्वाहा॒ स्वाहा॑ संॅवि॒द्योत॑मानाय संॅवि॒द्योत॑मानाय॒ स्वाहा᳚ । \newline
42. सं॒ॅवि॒द्योत॑माना॒येति॑ सं - वि॒द्योत॑मानाय । \newline
43. स्वाहा᳚ स्तनयिष्य॒ते स्त॑नयिष्य॒ते स्वाहा॒ स्वाहा᳚ स्तनयिष्य॒ते । \newline
44. स्त॒न॒यि॒ष्य॒ते स्वाहा॒ स्वाहा᳚ स्तनयिष्य॒ते स्त॑नयिष्य॒ते स्वाहा᳚ । \newline
45. स्वाहा᳚ स्त॒नय॑ते स्त॒नय॑ते॒ स्वाहा॒ स्वाहा᳚ स्त॒नय॑ते । \newline
46. स्त॒नय॑ते॒ स्वाहा॒ स्वाहा᳚ स्त॒नय॑ते स्त॒नय॑ते॒ स्वाहा᳚ । \newline
47. स्वाहो॒ग्र मु॒ग्रꣳ स्वाहा॒ स्वाहो॒ग्रम् । \newline
48. उ॒ग्रꣳ स्त॒नय॑ते स्त॒नय॑त उ॒ग्र मु॒ग्रꣳ स्त॒नय॑ते । \newline
49. स्त॒नय॑ते॒ स्वाहा॒ स्वाहा᳚ स्त॒नय॑ते स्त॒नय॑ते॒ स्वाहा᳚ । \newline
50. स्वाहा॑ वर्.षिष्य॒ते व॑र्.षिष्य॒ते स्वाहा॒ स्वाहा॑ वर्.षिष्य॒ते । \newline
51. व॒र्॒.षि॒ष्य॒ते स्वाहा॒ स्वाहा॑ वर्.षिष्य॒ते व॑र्.षिष्य॒ते स्वाहा᳚ । \newline
52. स्वाहा॒ वर्.ष॑ते॒ वर्.ष॑ते॒ स्वाहा॒ स्वाहा॒ वर्.ष॑ते । \newline
53. वर्.ष॑ते॒ स्वाहा॒ स्वाहा॒ वर्.ष॑ते॒ वर्.ष॑ते॒ स्वाहा᳚ । \newline
54. स्वाहा॑ ऽभि॒वर्.ष॑ते ऽभि॒वर्.ष॑ते॒ स्वाहा॒ स्वाहा॑ ऽभि॒वर्.ष॑ते । \newline
55. अ॒भि॒वर्.ष॑ते॒ स्वाहा॒ स्वाहा॑ ऽभि॒वर्.ष॑ते ऽभि॒वर्.ष॑ते॒ स्वाहा᳚ । \newline
56. अ॒भि॒वर्.ष॑त॒ इत्य॑भि - वर्.ष॑ते । \newline
57. स्वाहा॑ परि॒वर्.ष॑ते परि॒वर्.ष॑ते॒ स्वाहा॒ स्वाहा॑ परि॒वर्.ष॑ते । \newline
58. प॒रि॒वर्.ष॑ते॒ स्वाहा॒ स्वाहा॑ परि॒वर्.ष॑ते परि॒वर्.ष॑ते॒ स्वाहा᳚ । \newline
59. प॒रि॒वर्.ष॑त॒ इति॑ परि - वर्.ष॑ते । \newline
60. स्वाहा॑ सं॒ॅवर्.ष॑ते सं॒ॅवर्.ष॑ते॒ स्वाहा॒ स्वाहा॑ सं॒ॅवर्.ष॑ते । \newline
61. सं॒ॅवर्.ष॑ते॒ स्वाहा॒ स्वाहा॑ सं॒ॅवर्.ष॑ते सं॒ॅवर्.ष॑ते॒ स्वाहा᳚ । \newline
62. सं॒ॅवर्.ष॑त॒ इति॑ सं - वर्.ष॑ते । \newline

\textbf{Ghana Paata } \newline

1. पृ॒थि॒व्यै स्वाहा॒ स्वाहा॑ पृथि॒व्यै पृ॑थि॒व्यै स्वाहा॒ ऽन्तरि॑क्षाया॒ न्तरि॑क्षाय॒ स्वाहा॑ पृथि॒व्यै पृ॑थि॒व्यै स्वाहा॒ ऽन्तरि॑क्षाय । \newline
2. स्वाहा॒ ऽन्तरि॑क्षाया॒ न्तरि॑क्षाय॒ स्वाहा॒ स्वाहा॒ ऽन्तरि॑क्षाय॒ स्वाहा॒ स्वाहा॒ ऽन्तरि॑क्षाय॒ स्वाहा॒ स्वाहा॒ ऽन्तरि॑क्षाय॒ स्वाहा᳚ । \newline
3. अ॒न्तरि॑क्षाय॒ स्वाहा॒ स्वाहा॒ ऽन्तरि॑क्षाया॒ न्तरि॑क्षाय॒ स्वाहा॑ दि॒वे दि॒वे स्वाहा॒ ऽन्तरि॑क्षाया॒ न्तरि॑क्षाय॒ स्वाहा॑ दि॒वे । \newline
4. स्वाहा॑ दि॒वे दि॒वे स्वाहा॒ स्वाहा॑ दि॒वे स्वाहा॒ स्वाहा॑ दि॒वे स्वाहा॒ स्वाहा॑ दि॒वे स्वाहा᳚ । \newline
5. दि॒वे स्वाहा॒ स्वाहा॑ दि॒वे दि॒वे स्वाहा॑ संप्लोष्य॒ते सं॑प्लोष्य॒ते स्वाहा॑ दि॒वे दि॒वे स्वाहा॑ संप्लोष्य॒ते । \newline
6. स्वाहा॑ संप्लोष्य॒ते सं॑प्लोष्य॒ते स्वाहा॒ स्वाहा॑ संप्लोष्य॒ते स्वाहा॒ स्वाहा॑ संप्लोष्य॒ते स्वाहा॒ स्वाहा॑ संप्लोष्य॒ते स्वाहा᳚ । \newline
7. सं॒प्लो॒ष्य॒ते स्वाहा॒ स्वाहा॑ संप्लोष्य॒ते सं॑प्लोष्य॒ते स्वाहा॑ सं॒प्लव॑मानाय सं॒प्लव॑मानाय॒ स्वाहा॑ संप्लोष्य॒ते सं॑प्लोष्य॒ते स्वाहा॑ सं॒प्लव॑मानाय । \newline
8. सं॒प्लो॒ष्य॒त इति॑ सं - प्लो॒ष्य॒ते । \newline
9. स्वाहा॑ सं॒प्लव॑मानाय सं॒प्लव॑मानाय॒ स्वाहा॒ स्वाहा॑ सं॒प्लव॑मानाय॒ स्वाहा॒ स्वाहा॑ सं॒प्लव॑मानाय॒ स्वाहा॒ स्वाहा॑ सं॒प्लव॑मानाय॒ स्वाहा᳚ । \newline
10. सं॒प्लव॑मानाय॒ स्वाहा॒ स्वाहा॑ सं॒प्लव॑मानाय सं॒प्लव॑मानाय॒ स्वाहा॒ संप्लु॑ताय॒ संप्लु॑ताय॒ स्वाहा॑ सं॒प्लव॑मानाय सं॒प्लव॑मानाय॒ स्वाहा॒ संप्लु॑ताय । \newline
11. सं॒प्लव॑माना॒येति॑ सं - प्लव॑मानाय । \newline
12. स्वाहा॒ संप्लु॑ताय॒ संप्लु॑ताय॒ स्वाहा॒ स्वाहा॒ संप्लु॑ताय॒ स्वाहा॒ स्वाहा॒ संप्लु॑ताय॒ स्वाहा॒ स्वाहा॒ संप्लु॑ताय॒ स्वाहा᳚ । \newline
13. संप्लु॑ताय॒ स्वाहा॒ स्वाहा॒ संप्लु॑ताय॒ संप्लु॑ताय॒ स्वाहा॑ मेघायिष्य॒ते मे॑घायिष्य॒ते स्वाहा॒ संप्लु॑ताय॒ संप्लु॑ताय॒ स्वाहा॑ मेघायिष्य॒ते । \newline
14. संप्लु॑ता॒येति॒ सं - प्लु॒ता॒य॒ । \newline
15. स्वाहा॑ मेघायिष्य॒ते मे॑घायिष्य॒ते स्वाहा॒ स्वाहा॑ मेघायिष्य॒ते स्वाहा॒ स्वाहा॑ मेघायिष्य॒ते स्वाहा॒ स्वाहा॑ मेघायिष्य॒ते स्वाहा᳚ । \newline
16. मे॒घा॒यि॒ष्य॒ते स्वाहा॒ स्वाहा॑ मेघायिष्य॒ते मे॑घायिष्य॒ते स्वाहा॑ मेघाय॒ते मे॑घाय॒ते स्वाहा॑ मेघायिष्य॒ते मे॑घायिष्य॒ते स्वाहा॑ मेघाय॒ते । \newline
17. स्वाहा॑ मेघाय॒ते मे॑घाय॒ते स्वाहा॒ स्वाहा॑ मेघाय॒ते स्वाहा॒ स्वाहा॑ मेघाय॒ते स्वाहा॒ स्वाहा॑ मेघाय॒ते स्वाहा᳚ । \newline
18. मे॒घा॒य॒ते स्वाहा॒ स्वाहा॑ मेघाय॒ते मे॑घाय॒ते स्वाहा॑ मेघि॒ताय॑ मेघि॒ताय॒ स्वाहा॑ मेघाय॒ते मे॑घाय॒ते स्वाहा॑ मेघि॒ताय॑ । \newline
19. मे॒घा॒य॒त इति॑ मेघ - य॒ते । \newline
20. स्वाहा॑ मेघि॒ताय॑ मेघि॒ताय॒ स्वाहा॒ स्वाहा॑ मेघि॒ताय॒ स्वाहा॒ स्वाहा॑ मेघि॒ताय॒ स्वाहा॒ स्वाहा॑ मेघि॒ताय॒ स्वाहा᳚ । \newline
21. मे॒घि॒ताय॒ स्वाहा॒ स्वाहा॑ मेघि॒ताय॑ मेघि॒ताय॒ स्वाहा॑ मे॒घाय॑ मे॒घाय॒ स्वाहा॑ मेघि॒ताय॑ मेघि॒ताय॒ स्वाहा॑ मे॒घाय॑ । \newline
22. स्वाहा॑ मे॒घाय॑ मे॒घाय॒ स्वाहा॒ स्वाहा॑ मे॒घाय॒ स्वाहा॒ स्वाहा॑ मे॒घाय॒ स्वाहा॒ स्वाहा॑ मे॒घाय॒ स्वाहा᳚ । \newline
23. मे॒घाय॒ स्वाहा॒ स्वाहा॑ मे॒घाय॑ मे॒घाय॒ स्वाहा॑ नीहा॒राय॑ नीहा॒राय॒ स्वाहा॑ मे॒घाय॑ मे॒घाय॒ स्वाहा॑ नीहा॒राय॑ । \newline
24. स्वाहा॑ नीहा॒राय॑ नीहा॒राय॒ स्वाहा॒ स्वाहा॑ नीहा॒राय॒ स्वाहा॒ स्वाहा॑ नीहा॒राय॒ स्वाहा॒ स्वाहा॑ नीहा॒राय॒ स्वाहा᳚ । \newline
25. नी॒हा॒राय॒ स्वाहा॒ स्वाहा॑ नीहा॒राय॑ नीहा॒राय॒ स्वाहा॑ नि॒हाका॑यै नि॒हाका॑यै॒ स्वाहा॑ नीहा॒राय॑ नीहा॒राय॒ स्वाहा॑ नि॒हाका॑यै । \newline
26. स्वाहा॑ नि॒हाका॑यै नि॒हाका॑यै॒ स्वाहा॒ स्वाहा॑ नि॒हाका॑यै॒ स्वाहा॒ स्वाहा॑ नि॒हाका॑यै॒ स्वाहा॒ स्वाहा॑ नि॒हाका॑यै॒ स्वाहा᳚ । \newline
27. नि॒हाका॑यै॒ स्वाहा॒ स्वाहा॑ नि॒हाका॑यै नि॒हाका॑यै॒ स्वाहा᳚ प्रास॒चाय॑ प्रास॒चाय॒ स्वाहा॑ नि॒हाका॑यै नि॒हाका॑यै॒ स्वाहा᳚ प्रास॒चाय॑ । \newline
28. नि॒हाका॑या॒ इति॑ नि - हाका॑यै । \newline
29. स्वाहा᳚ प्रास॒चाय॑ प्रास॒चाय॒ स्वाहा॒ स्वाहा᳚ प्रास॒चाय॒ स्वाहा॒ स्वाहा᳚ प्रास॒चाय॒ स्वाहा॒ स्वाहा᳚ प्रास॒चाय॒ स्वाहा᳚ । \newline
30. प्रा॒स॒चाय॒ स्वाहा॒ स्वाहा᳚ प्रास॒चाय॑ प्रास॒चाय॒ स्वाहा᳚ प्रच॒लाका॑यै प्रच॒लाका॑यै॒ स्वाहा᳚ प्रास॒चाय॑ प्रास॒चाय॒ स्वाहा᳚ प्रच॒लाका॑यै । \newline
31. स्वाहा᳚ प्रच॒लाका॑यै प्रच॒लाका॑यै॒ स्वाहा॒ स्वाहा᳚ प्रच॒लाका॑यै॒ स्वाहा॒ स्वाहा᳚ प्रच॒लाका॑यै॒ स्वाहा॒ स्वाहा᳚ प्रच॒लाका॑यै॒ स्वाहा᳚ । \newline
32. प्र॒च॒लाका॑यै॒ स्वाहा॒ स्वाहा᳚ प्रच॒लाका॑यै प्रच॒लाका॑यै॒ स्वाहा॑ विद्योतिष्य॒ते वि॑द्योतिष्य॒ते स्वाहा᳚ प्रच॒लाका॑यै प्रच॒लाका॑यै॒ स्वाहा॑ विद्योतिष्य॒ते । \newline
33. प्र॒च॒लाका॑या॒ इति॑ प्र - च॒लाका॑यै । \newline
34. स्वाहा॑ विद्योतिष्य॒ते वि॑द्योतिष्य॒ते स्वाहा॒ स्वाहा॑ विद्योतिष्य॒ते स्वाहा॒ स्वाहा॑ विद्योतिष्य॒ते स्वाहा॒ स्वाहा॑ विद्योतिष्य॒ते स्वाहा᳚ । \newline
35. वि॒द्यो॒ति॒ष्य॒ते स्वाहा॒ स्वाहा॑ विद्योतिष्य॒ते वि॑द्योतिष्य॒ते स्वाहा॑ वि॒द्योत॑मानाय वि॒द्योत॑मानाय॒ स्वाहा॑ विद्योतिष्य॒ते वि॑द्योतिष्य॒ते स्वाहा॑ वि॒द्योत॑मानाय । \newline
36. वि॒द्यो॒ति॒ष्य॒त इति॑ वि - द्यो॒ति॒ष्य॒ते । \newline
37. स्वाहा॑ वि॒द्योत॑मानाय वि॒द्योत॑मानाय॒ स्वाहा॒ स्वाहा॑ वि॒द्योत॑मानाय॒ स्वाहा॒ स्वाहा॑ वि॒द्योत॑मानाय॒ स्वाहा॒ स्वाहा॑ वि॒द्योत॑मानाय॒ स्वाहा᳚ । \newline
38. वि॒द्योत॑मानाय॒ स्वाहा॒ स्वाहा॑ वि॒द्योत॑मानाय वि॒द्योत॑मानाय॒ स्वाहा॑ संॅवि॒द्योत॑मानाय संॅवि॒द्योत॑मानाय॒ स्वाहा॑ वि॒द्योत॑मानाय वि॒द्योत॑मानाय॒ स्वाहा॑ संॅवि॒द्योत॑मानाय । \newline
39. वि॒द्योत॑माना॒येति॑ वि - द्योत॑मानाय । \newline
40. स्वाहा॑ संॅवि॒द्योत॑मानाय संॅवि॒द्योत॑मानाय॒ स्वाहा॒ स्वाहा॑ संॅवि॒द्योत॑मानाय॒ स्वाहा॒ स्वाहा॑ संॅवि॒द्योत॑मानाय॒ स्वाहा॒ स्वाहा॑ संॅवि॒द्योत॑मानाय॒ स्वाहा᳚ । \newline
41. सं॒ॅवि॒द्योत॑मानाय॒ स्वाहा॒ स्वाहा॑ संॅवि॒द्योत॑मानाय संॅवि॒द्योत॑मानाय॒ स्वाहा᳚ स्तनयिष्य॒ते स्त॑नयिष्य॒ते स्वाहा॑ संॅवि॒द्योत॑मानाय संॅवि॒द्योत॑मानाय॒ स्वाहा᳚ स्तनयिष्य॒ते । \newline
42. सं॒ॅवि॒द्योत॑माना॒येति॑ सं - वि॒द्योत॑मानाय । \newline
43. स्वाहा᳚ स्तनयिष्य॒ते स्त॑नयिष्य॒ते स्वाहा॒ स्वाहा᳚ स्तनयिष्य॒ते स्वाहा॒ स्वाहा᳚ स्तनयिष्य॒ते स्वाहा॒ स्वाहा᳚ स्तनयिष्य॒ते स्वाहा᳚ । \newline
44. स्त॒न॒यि॒ष्य॒ते स्वाहा॒ स्वाहा᳚ स्तनयिष्य॒ते स्त॑नयिष्य॒ते स्वाहा᳚ स्त॒नय॑ते स्त॒नय॑ते॒ स्वाहा᳚ स्तनयिष्य॒ते स्त॑नयिष्य॒ते स्वाहा᳚ स्त॒नय॑ते । \newline
45. स्वाहा᳚ स्त॒नय॑ते स्त॒नय॑ते॒ स्वाहा॒ स्वाहा᳚ स्त॒नय॑ते॒ स्वाहा॒ स्वाहा᳚ स्त॒नय॑ते॒ स्वाहा॒ स्वाहा᳚ स्त॒नय॑ते॒ स्वाहा᳚ । \newline
46. स्त॒नय॑ते॒ स्वाहा॒ स्वाहा᳚ स्त॒नय॑ते स्त॒नय॑ते॒ स्वाहो॒ग्र मु॒ग्रꣳ स्वाहा᳚ स्त॒नय॑ते स्त॒नय॑ते॒ स्वाहो॒ग्रम् । \newline
47. स्वाहो॒ग्र मु॒ग्रꣳ स्वाहा॒ स्वाहो॒ग्रꣳ स्त॒नय॑ते स्त॒नय॑त उ॒ग्रꣳ स्वाहा॒ स्वाहो॒ग्रꣳ स्त॒नय॑ते । \newline
48. उ॒ग्रꣳ स्त॒नय॑ते स्त॒नय॑त उ॒ग्र मु॒ग्रꣳ स्त॒नय॑ते॒ स्वाहा॒ स्वाहा᳚ स्त॒नय॑त उ॒ग्र मु॒ग्रꣳ स्त॒नय॑ते॒ स्वाहा᳚ । \newline
49. स्त॒नय॑ते॒ स्वाहा॒ स्वाहा᳚ स्त॒नय॑ते स्त॒नय॑ते॒ स्वाहा॑ वर्.षिष्य॒ते व॑र्.षिष्य॒ते स्वाहा᳚ स्त॒नय॑ते स्त॒नय॑ते॒ स्वाहा॑ वर्.षिष्य॒ते । \newline
50. स्वाहा॑ वर्.षिष्य॒ते व॑र्.षिष्य॒ते स्वाहा॒ स्वाहा॑ वर्.षिष्य॒ते स्वाहा॒ स्वाहा॑ वर्.षिष्य॒ते स्वाहा॒ स्वाहा॑ वर्.षिष्य॒ते स्वाहा᳚ । \newline
51. व॒र्॒.षि॒ष्य॒ते स्वाहा॒ स्वाहा॑ वर्.षिष्य॒ते व॑र्.षिष्य॒ते स्वाहा॒ वर्.ष॑ते॒ वर्.ष॑ते॒ स्वाहा॑ वर्.षिष्य॒ते व॑र्.षिष्य॒ते स्वाहा॒ वर्.ष॑ते । \newline
52. स्वाहा॒ वर्.ष॑ते॒ वर्.ष॑ते॒ स्वाहा॒ स्वाहा॒ वर्.ष॑ते॒ स्वाहा॒ स्वाहा॒ वर्.ष॑ते॒ स्वाहा॒ स्वाहा॒ वर्.ष॑ते॒ स्वाहा᳚ । \newline
53. वर्.ष॑ते॒ स्वाहा॒ स्वाहा॒ वर्.ष॑ते॒ वर्.ष॑ते॒ स्वाहा॑ ऽभि॒वर्.ष॑ते ऽभि॒वर्.ष॑ते॒ स्वाहा॒ वर्.ष॑ते॒ वर्.ष॑ते॒ स्वाहा॑ ऽभि॒वर्.ष॑ते । \newline
54. स्वाहा॑ ऽभि॒वर्.ष॑ते ऽभि॒वर्.ष॑ते॒ स्वाहा॒ स्वाहा॑ ऽभि॒वर्.ष॑ते॒ स्वाहा॒ स्वाहा॑ ऽभि॒वर्.ष॑ते॒ स्वाहा॒ स्वाहा॑ ऽभि॒वर्.ष॑ते॒ स्वाहा᳚ । \newline
55. अ॒भि॒वर्.ष॑ते॒ स्वाहा॒ स्वाहा॑ ऽभि॒वर्.ष॑ते ऽभि॒वर्.ष॑ते॒ स्वाहा॑ परि॒वर्.ष॑ते परि॒वर्.ष॑ते॒ स्वाहा॑ ऽभि॒वर्.ष॑ते ऽभि॒वर्.ष॑ते॒ स्वाहा॑ परि॒वर्.ष॑ते । \newline
56. अ॒भि॒वर्.ष॑त॒ इत्य॑भि - वर्.ष॑ते । \newline
57. स्वाहा॑ परि॒वर्.ष॑ते परि॒वर्.ष॑ते॒ स्वाहा॒ स्वाहा॑ परि॒वर्.ष॑ते॒ स्वाहा॒ स्वाहा॑ परि॒वर्.ष॑ते॒ स्वाहा॒ स्वाहा॑ परि॒वर्.ष॑ते॒ स्वाहा᳚ । \newline
58. प॒रि॒वर्.ष॑ते॒ स्वाहा॒ स्वाहा॑ परि॒वर्.ष॑ते परि॒वर्.ष॑ते॒ स्वाहा॑ सं॒ॅवर्.ष॑ते सं॒ॅवर्.ष॑ते॒ स्वाहा॑ परि॒वर्.ष॑ते परि॒वर्.ष॑ते॒ स्वाहा॑ सं॒ॅवर्.ष॑ते । \newline
59. प॒रि॒वर्.ष॑त॒ इति॑ परि - वर्.ष॑ते । \newline
60. स्वाहा॑ सं॒ॅवर्.ष॑ते सं॒ॅवर्.ष॑ते॒ स्वाहा॒ स्वाहा॑ सं॒ॅवर्.ष॑ते॒ स्वाहा॒ स्वाहा॑ सं॒ॅवर्.ष॑ते॒ स्वाहा॒ स्वाहा॑ सं॒ॅवर्.ष॑ते॒ स्वाहा᳚ । \newline
61. सं॒ॅवर्.ष॑ते॒ स्वाहा॒ स्वाहा॑ सं॒ॅवर्.ष॑ते सं॒ॅवर्.ष॑ते॒ स्वाहा॑ ऽनु॒वर्.ष॑ते ऽनु॒वर्.ष॑ते॒ स्वाहा॑ सं॒ॅवर्.ष॑ते सं॒ॅवर्.ष॑ते॒ स्वाहा॑ ऽनु॒वर्.ष॑ते । \newline
62. सं॒ॅवर्.ष॑त॒ इति॑ सं - वर्.ष॑ते । \newline
\pagebreak
\markright{ TS 7.5.11.2  \hfill https://www.vedavms.in \hfill}

\section{ TS 7.5.11.2 }

\textbf{TS 7.5.11.2 } \newline
\textbf{Samhita Paata} \newline

स्वाहा॑ ऽनु॒वर्.ष॑ते॒ स्वाहा॑ शीकायिष्य॒ते स्वाहा॑ शीकाय॒ते स्वाहा॑ शीकि॒ताय॒ स्वाहा᳚प्रोषिष्य॒ते स्वाहा᳚ प्रुष्ण॒ते स्वाहा॑ परिप्रुष्ण॒ते स्वाहो᳚-द्ग्रहीष्य॒ते स्वाहो᳚ द्गृह्ण॒ते स्वाहो-द्गृ॑हीताय॒ स्वाहा॑ विप्लोष्य॒ते स्वाहा॑ वि॒प्लव॑मानाय॒ स्वाहा॒ विप्लु॑ताय॒ स्वाहा॑ ऽऽतफ्स्य॒ते स्वाहा॒ ऽऽतप॑ते ॒स्वाहो॒-ग्रमा॒तप॑ते॒ स्वाह॒ -र्ग्भ्यः स्वाहा॒ यजु॑र्भ्यः॒ स्वाहा॒ साम॑भ्यः॒ स्वाहा ऽङ्गि॑रोभ्यः॒ स्वाहा॒ वेदे᳚भ्यः॒ स्वाहा॒ गाथा᳚भ्यः॒ स्वाहा॑ नाराशꣳ॒॒सीभ्यः॒ स्वाहा॒ रैभी᳚भ्यः स्वाहा॒ ( ) सर्व॑स्मै॒ स्वाहा᳚ ॥ \newline

\textbf{Pada Paata} \newline

स्वाहा᳚ । अ॒नु॒वर्.ष॑त॒ इत्य॑नु - वर्.ष॑ते । स्वाहा᳚ । शी॒का॒यि॒ष्य॒ते । स्वाहा᳚ । शी॒का॒य॒त इति॑ शीक - य॒ते । स्वाहा᳚ । शी॒कि॒ताय॑ । स्वाहा᳚ । प्रो॒षि॒ष्य॒ते । स्वाहा᳚ । प्रु॒ष्ण॒ते । स्वाहा᳚ । प॒रि॒प्रु॒ष्ण॒त इति॑ परि - प्रु॒ष्ण॒ते । स्वाहा᳚ । उ॒द्ग्र॒ही॒ष्य॒त इत्यु॑त् - ग्र॒ही॒ष्य॒ते । स्वाहा᳚ । उ॒द्गृ॒ह्ण॒त इत्यु॑त् - गृ॒ह्ण॒ते । स्वाहा᳚ । उद्गृ॑हीता॒येत्युत् -   गृ॒ही॒ता॒य॒ । स्वाहा᳚ । वि॒प्लो॒ष्य॒त इति॑ वि - प्लो॒ष्य॒ते । स्वाहा᳚ । वि॒प्लव॑माना॒येति॑ वि - प्लव॑मानाय । स्वाहा᳚ । विप्लु॑ता॒येति॒ वि - प्लु॒ता॒य॒ । स्वाहा᳚ । आ॒त॒फ्स्य॒त इत्या᳚ - त॒फ्स्य॒ते । स्वाहा᳚ । आ॒तप॑त॒ इत्या᳚ - तप॑ते । स्वाहा᳚ । उ॒ग्रम् । आ॒तप॑त॒ इत्या᳚ - तप॑ते । स्वाहा᳚ । ऋ॒ग्भ्य इत्यृ॑क् - भ्यः । स्वाहा᳚ । यजु॑र्भ्य॒ इति॒ यजुः॑ - भ्यः॒ । स्वाहा᳚ । साम॑भ्य॒ इति॒ साम॑ - भ्यः॒ । स्वाहा᳚ । अङ्गि॑रोभ्य॒ इत्यङ्गि॑रः -भ्यः॒ । स्वाहा᳚ । वेदे᳚भ्यः । स्वाहा᳚ । गाथा᳚भ्यः । स्वाहा᳚ । ना॒रा॒शꣳ॒॒सीभ्यः॑ । स्वाहा᳚ । रैभी᳚भ्यः । स्वाहा॑ ( ) । सर्व॑स्मै । स्वाहा᳚ ॥  \newline


\textbf{Krama Paata} \newline

स्वाहा॑ऽनु॒वर्.ष॑ते । अ॒नु॒वर्.ष॑ते॒ स्वाहा᳚ । अ॒नु॒वर्.॑षत॒ इत्य॑नु - वर्.ष॑ते । स्वाहा॑ शीकायिष्य॒ते । शी॒का॒यि॒ष्य॒ते स्वाहा᳚ । स्वाहा॑ शीकाय॒ते । शी॒का॒य॒ते स्वाहा᳚ । शी॒का॒य॒त इति॑ शीक - य॒ते । स्वाहा॑ शीकि॒ताय॑ । शी॒कि॒ताय॒ स्वाहा᳚ । स्वाहा᳚ प्रोषिष्य॒ते । प्रो॒षि॒ष्य॒ते स्वाहा᳚ । स्वाहा᳚ प्रुष्ण॒ते । प्रु॒ष्ण॒ते स्वाहा᳚ । स्वाहा॑ परिप्रुष्ण॒ते । प॒रि॒प्रु॒ष्ण॒ते स्वाहा᳚ । प॒रि॒प्रु॒ष्ण॒त इति॑ परि - प्रु॒ष्ण॒ते । स्वाहो᳚द्ग्रहीष्य॒ते । उ॒द्ग्र॒ही॒ष्य॒ते स्वाहा᳚ । उ॒द्‍ग॒ही॒ष्य॒त इत्यु॑त् - ग्र॒ही॒ष्य॒ते । स्वाहो᳚द्‍गृह्ण॒ते । उ॒द्‍गृ॒ह्ण॒ते स्वाहा᳚ । उ॒द्‍गृ॒ह्ण॒त इत्यु॑त् - गृ॒ह्ण॒ते । स्वाहोद्‍गृ॑हीताय । उद्‍गृ॑हीताय॒ स्वाहा᳚ । उद्‍गृ॑हीता॒येत्युत् - गृ॒ही॒ता॒य॒ । स्वाहा॑ विप्लोष्य॒ते । वि॒प्लो॒ष्य॒ते स्वाहा᳚ । वि॒प्लो॒ष्य॒त इति॑ वि - प्लो॒ष्य॒ते । स्वाहा॑ वि॒प्लव॑मानाय । वि॒प्लव॑मानाय॒ स्वाहा᳚ । वि॒प्लव॑माना॒येति॑ वि - प्लव॑मानाय । स्वाहा॒ विप्लु॑ताय । विप्लु॑ताय॒ स्वाहा᳚ । विप्लु॑ता॒येति॒ वि - प्लु॒ता॒य॒ । स्वाहा॑ऽऽतफ्स्य॒ते । आ॒त॒फ्स्य॒ते स्वाहा᳚ । आ॒त॒फ्स्य॒त इत्या᳚ - त॒फ्स्य॒ते । स्वाहा॒ऽऽतप॑ते । आ॒तप॑ते॒ स्वाहा᳚ । आ॒तप॑त॒ इत्या᳚ - तप॑ते । स्वाहो॒ग्रम् । उ॒ग्रमा॒तप॑ते । आ॒तप॑ते॒ स्वाहा᳚ । आ॒तप॑त॒ इत्या᳚ - तप॑ते । स्वाह॒र्ग्भ्यः । ऋ॒ग्भ्यः स्वाहा᳚ । ऋ॒ग्भ्य इत्यृ॑क् - भ्यः । स्वाहा॒ यजु॑र्भ्यः । यजु॑र्भ्यः॒ स्वाहा᳚ । यजु॑र्भ्य॒ इति॒ यजुः॑ - भ्यः॒ । स्वाहा॒ साम॑भ्यः । साम॑भ्यः॒ स्वाहा᳚ । साम॑भ्य॒ इति॒ साम॑ - भ्यः॒ । स्वाहाऽङ्‍गि॑रोभ्यः । अङ्‍गि॑रोभ्यः॒ स्वाहा᳚ । अङ्‍गि॑रोभ्य॒ इत्यङ्‍गि॑रः - भ्यः॒ । स्वाहा॒ वेदे᳚भ्यः । वेदे᳚भ्यः॒ स्वाहा᳚ । स्वाहा॒ गाथा᳚भ्यः । गाथा᳚भ्यः॒ स्वाहा᳚ । स्वाहा॑ नाराशꣳ॒॒सीभ्यः॑ । ना॒रा॒शꣳ॒॒सीभ्यः॒ स्वाहा᳚ । स्वाहा॒ रैभी᳚भ्यः । रैभी᳚भ्यः॒ स्वाहा᳚ ( ) । स्वाहा॒ सर्व॑स्मै । सर्व॑स्मै॒ स्वाहा᳚ । स्वाहेति॒ स्वाहा᳚ । \newline

\textbf{Jatai Paata} \newline

1. स्वाहा॑ ऽनु॒वर्.ष॑ते ऽनु॒वर्.ष॑ते॒ स्वाहा॒ स्वाहा॑ ऽनु॒वर्.ष॑ते । \newline
2. अ॒नु॒वर्.ष॑ते॒ स्वाहा॒ स्वाहा॑ ऽनु॒वर्.ष॑ते ऽनु॒वर्.ष॑ते॒ स्वाहा᳚ । \newline
3. अ॒नु॒वर्.ष॑त॒ इत्य॑नु - वर्.ष॑ते । \newline
4. स्वाहा॑ शीकायिष्य॒ते शी॑कायिष्य॒ते स्वाहा॒ स्वाहा॑ शीकायिष्य॒ते । \newline
5. शी॒का॒यि॒ष्य॒ते स्वाहा॒ स्वाहा॑ शीकायिष्य॒ते शी॑कायिष्य॒ते स्वाहा᳚ । \newline
6. स्वाहा॑ शीकाय॒ते शी॑काय॒ते स्वाहा॒ स्वाहा॑ शीकाय॒ते । \newline
7. शी॒का॒य॒ते स्वाहा॒ स्वाहा॑ शीकाय॒ते शी॑काय॒ते स्वाहा᳚ । \newline
8. शी॒का॒य॒त इति॑ शीक - य॒ते । \newline
9. स्वाहा॑ शीकि॒ताय॑ शीकि॒ताय॒ स्वाहा॒ स्वाहा॑ शीकि॒ताय॑ । \newline
10. शी॒कि॒ताय॒ स्वाहा॒ स्वाहा॑ शीकि॒ताय॑ शीकि॒ताय॒ स्वाहा᳚ । \newline
11. स्वाहा᳚ प्रोषिष्य॒ते प्रो॑षिष्य॒ते स्वाहा॒ स्वाहा᳚ प्रोषिष्य॒ते । \newline
12. प्रो॒षि॒ष्य॒ते स्वाहा॒ स्वाहा᳚ प्रोषिष्य॒ते प्रो॑षिष्य॒ते स्वाहा᳚ । \newline
13. स्वाहा᳚ प्रुष्ण॒ते प्रु॑ष्ण॒ते स्वाहा॒ स्वाहा᳚ प्रुष्ण॒ते । \newline
14. प्रु॒ष्ण॒ते स्वाहा॒ स्वाहा᳚ प्रुष्ण॒ते प्रु॑ष्ण॒ते स्वाहा᳚ । \newline
15. स्वाहा॑ परिप्रुष्ण॒ते प॑रिप्रुष्ण॒ते स्वाहा॒ स्वाहा॑ परिप्रुष्ण॒ते । \newline
16. प॒रि॒प्रु॒ष्ण॒ते स्वाहा॒ स्वाहा॑ परिप्रुष्ण॒ते प॑रिप्रुष्ण॒ते स्वाहा᳚ । \newline
17. प॒रि॒प्रु॒ष्ण॒त इति॑ परि - प्रु॒ष्ण॒ते । \newline
18. स्वाहो᳚ द्‍ग्रहीष्य॒त उ॑द्‍ग्रहीष्य॒ते स्वाहा॒ स्वाहो᳚ द्‍ग्रहीष्य॒ते । \newline
19. उ॒द्‍ग्र॒ही॒ष्य॒ते स्वाहा॒ स्वाहो᳚ द्‍ग्रहीष्य॒त उ॑द्‍ग्रहीष्य॒ते स्वाहा᳚ । \newline
20. उ॒द्‍ग्र॒ही॒ष्य॒त इत्यु॑त् - ग्र॒ही॒ष्य॒ते । \newline
21. स्वाहो᳚द्‍गृह्ण॒त उ॑द्‍गृह्ण॒ते स्वाहा॒ स्वाहो᳚द्‍गृह्ण॒ते । \newline
22. उ॒द्‍गृ॒ह्ण॒ते स्वाहा॒ स्वाहो᳚द्‍गृह्ण॒त उ॑द्‍गृह्ण॒ते स्वाहा᳚ । \newline
23. उ॒द्‍गृ॒ह्ण॒त इत्यु॑त् - गृ॒ह्ण॒ते । \newline
24. स्वाहोद्‍गृ॑हीता॒ योद्‍गृ॑हीताय॒ स्वाहा॒ स्वाहोद्‍गृ॑हीताय । \newline
25. उद्‍गृ॑हीताय॒ स्वाहा॒ स्वाहोद्‍गृ॑हीता॒ योद्‍गृ॑हीताय॒ स्वाहा᳚ । \newline
26. उद्‍गृ॑हीता॒येत्युत् - गृ॒ही॒ता॒य॒ । \newline
27. स्वाहा॑ विप्लोष्य॒ते वि॑प्लोष्य॒ते स्वाहा॒ स्वाहा॑ विप्लोष्य॒ते । \newline
28. वि॒प्लो॒ष्य॒ते स्वाहा॒ स्वाहा॑ विप्लोष्य॒ते वि॑प्लोष्य॒ते स्वाहा᳚ । \newline
29. वि॒प्लो॒ष्य॒त इति॑ वि - प्लो॒ष्य॒ते । \newline
30. स्वाहा॑ वि॒प्लव॑मानाय वि॒प्लव॑मानाय॒ स्वाहा॒ स्वाहा॑ वि॒प्लव॑मानाय । \newline
31. वि॒प्लव॑मानाय॒ स्वाहा॒ स्वाहा॑ वि॒प्लव॑मानाय वि॒प्लव॑मानाय॒ स्वाहा᳚ । \newline
32. वि॒प्लव॑माना॒येति॑ वि - प्लव॑मानाय । \newline
33. स्वाहा॒ विप्लु॑ताय॒ विप्लु॑ताय॒ स्वाहा॒ स्वाहा॒ विप्लु॑ताय । \newline
34. विप्लु॑ताय॒ स्वाहा॒ स्वाहा॒ विप्लु॑ताय॒ विप्लु॑ताय॒ स्वाहा᳚ । \newline
35. विप्लु॑ता॒येति॒ वि - प्लु॒ता॒य॒ । \newline
36. स्वाहा॑ ऽऽतफ्स्य॒त आ॑तफ्स्य॒ते स्वाहा॒ स्वाहा॑ ऽऽतफ्स्य॒ते । \newline
37. आ॒त॒फ्स्य॒ते स्वाहा॒ स्वाहा॑ ऽऽतफ्स्य॒त आ॑तफ्स्य॒ते स्वाहा᳚ । \newline
38. आ॒त॒फ्स्य॒त इत्या᳚ - त॒फ्स्य॒ते । \newline
39. स्वाहा॒ ऽऽतप॑त आ॒तप॑ते॒ स्वाहा॒ स्वाहा॒ ऽऽतप॑ते । \newline
40. आ॒तप॑ते॒ स्वाहा॒ स्वाहा॒ ऽऽतप॑त आ॒तप॑ते॒ स्वाहा᳚ । \newline
41. आ॒तप॑त॒ इत्या᳚ - तप॑ते । \newline
42. स्वाहो॒ग्र मु॒ग्रꣳ स्वाहा॒ स्वाहो॒ग्रम् । \newline
43. उ॒ग्र मा॒तप॑त आ॒तप॑त उ॒ग्र मु॒ग्र मा॒तप॑ते । \newline
44. आ॒तप॑ते॒ स्वाहा॒ स्वाहा॒ ऽऽतप॑त आ॒तप॑ते॒ स्वाहा᳚ । \newline
45. आ॒तप॑त॒ इत्या᳚ - तप॑ते । \newline
46. स्वाह॒ र्‌ग्भ्य ऋ॒ग्भ्यः स्वाहा॒ स्वाह॒ र्‌ग्भ्यः । \newline
47. ऋ॒ग्भ्यः स्वाहा॒ स्वाह॒ र्‌ग्भ्य ऋ॒ग्भ्यः स्वाहा᳚ । \newline
48. ऋ॒ग्भ्य इत्यृ॑क् - भ्यः । \newline
49. स्वाहा॒ यजु॑र्भ्यो॒ यजु॑र्भ्यः॒ स्वाहा॒ स्वाहा॒ यजु॑र्भ्यः । \newline
50. यजु॑र्भ्यः॒ स्वाहा॒ स्वाहा॒ यजु॑र्भ्यो॒ यजु॑र्भ्यः॒ स्वाहा᳚ । \newline
51. यजु॑र्भ्य॒ इति॒ यजुः॑ - भ्यः॒ । \newline
52. स्वाहा॒ साम॑भ्यः॒ साम॑भ्यः॒ स्वाहा॒ स्वाहा॒ साम॑भ्यः । \newline
53. साम॑भ्यः॒ स्वाहा॒ स्वाहा॒ साम॑भ्यः॒ साम॑भ्यः॒ स्वाहा᳚ । \newline
54. साम॑भ्य॒ इति॒ साम॑ - भ्यः॒ । \newline
55. स्वाहा ऽङ्गि॑रोभ्यो॒ अङ्गि॑रोभ्यः॒ स्वाहा॒ स्वाहा ऽङ्गि॑रोभ्यः । \newline
56. अङ्गि॑रोभ्यः॒ स्वाहा॒ स्वाहा ऽङ्गि॑रोभ्यो॒ अङ्गि॑रोभ्यः॒ स्वाहा᳚ । \newline
57. अङ्गि॑रोभ्य॒ इत्यङ्गि॑रः - भ्यः॒ । \newline
58. स्वाहा॒ वेदे᳚भ्यो॒ वेदे᳚भ्यः॒ स्वाहा॒ स्वाहा॒ वेदे᳚भ्यः । \newline
59. वेदे᳚भ्यः॒ स्वाहा॒ स्वाहा॒ वेदे᳚भ्यो॒ वेदे᳚भ्यः॒ स्वाहा᳚ । \newline
60. स्वाहा॒ गाथा᳚भ्यो॒ गाथा᳚भ्यः॒ स्वाहा॒ स्वाहा॒ गाथा᳚भ्यः । \newline
61. गाथा᳚भ्यः॒ स्वाहा॒ स्वाहा॒ गाथा᳚भ्यो॒ गाथा᳚भ्यः॒ स्वाहा᳚ । \newline
62. स्वाहा॑ नाराशꣳ॒॒सीभ्यो॑ नाराशꣳ॒॒सीभ्यः॒ स्वाहा॒ स्वाहा॑ नाराशꣳ॒॒सीभ्यः॑ । \newline
63. ना॒रा॒शꣳ॒॒सीभ्यः॒ स्वाहा॒ स्वाहा॑ नाराशꣳ॒॒सीभ्यो॑ नाराशꣳ॒॒सीभ्यः॒ स्वाहा᳚ । \newline
64. स्वाहा॒ रैभी᳚भ्यो॒ रैभी᳚भ्यः॒ स्वाहा॒ स्वाहा॒ रैभी᳚भ्यः । \newline
65. रैभी᳚भ्यः॒ स्वाहा॒ स्वाहा॒ रैभी᳚भ्यो॒ रैभी᳚भ्यः॒ स्वाहा᳚ । \newline
66. स्वाहा॒ सर्व॑स्मै॒ सर्व॑स्मै॒ स्वाहा॒ स्वाहा॒ सर्व॑स्मै । \newline
67. सर्व॑स्मै॒ स्वाहा॒ स्वाहा॒ सर्व॑स्मै॒ सर्व॑स्मै॒ स्वाहा᳚ । \newline
68. स्वाहेति॒ स्वाहा᳚ । \newline

\textbf{Ghana Paata } \newline

1. स्वाहा॑ ऽनु॒वर्.ष॑ते ऽनु॒वर्.ष॑ते॒ स्वाहा॒ स्वाहा॑ ऽनु॒वर्.ष॑ते॒ स्वाहा॒ स्वाहा॑ ऽनु॒वर्.ष॑ते॒ स्वाहा॒ स्वाहा॑ ऽनु॒वर्.ष॑ते॒ स्वाहा᳚ । \newline
2. अ॒नु॒वर्.ष॑ते॒ स्वाहा॒ स्वाहा॑ ऽनु॒वर्.ष॑ते ऽनु॒वर्.ष॑ते॒ स्वाहा॑ शीकायिष्य॒ते शी॑कायिष्य॒ते स्वाहा॑ ऽनु॒वर्.ष॑ते ऽनु॒वर्.ष॑ते॒ स्वाहा॑ शीकायिष्य॒ते । \newline
3. अ॒नु॒वर्.ष॑त॒ इत्य॑नु - वर्.ष॑ते । \newline
4. स्वाहा॑ शीकायिष्य॒ते शी॑कायिष्य॒ते स्वाहा॒ स्वाहा॑ शीकायिष्य॒ते स्वाहा॒ स्वाहा॑ शीकायिष्य॒ते स्वाहा॒ स्वाहा॑ शीकायिष्य॒ते स्वाहा᳚ । \newline
5. शी॒का॒यि॒ष्य॒ते स्वाहा॒ स्वाहा॑ शीकायिष्य॒ते शी॑कायिष्य॒ते स्वाहा॑ शीकाय॒ते शी॑काय॒ते स्वाहा॑ शीकायिष्य॒ते शी॑कायिष्य॒ते स्वाहा॑ शीकाय॒ते । \newline
6. स्वाहा॑ शीकाय॒ते शी॑काय॒ते स्वाहा॒ स्वाहा॑ शीकाय॒ते स्वाहा॒ स्वाहा॑ शीकाय॒ते स्वाहा॒ स्वाहा॑ शीकाय॒ते स्वाहा᳚ । \newline
7. शी॒का॒य॒ते स्वाहा॒ स्वाहा॑ शीकाय॒ते शी॑काय॒ते स्वाहा॑ शीकि॒ताय॑ शीकि॒ताय॒ स्वाहा॑ शीकाय॒ते शी॑काय॒ते स्वाहा॑ शीकि॒ताय॑ । \newline
8. शी॒का॒य॒त इति॑ शीक - य॒ते । \newline
9. स्वाहा॑ शीकि॒ताय॑ शीकि॒ताय॒ स्वाहा॒ स्वाहा॑ शीकि॒ताय॒ स्वाहा॒ स्वाहा॑ शीकि॒ताय॒ स्वाहा॒ स्वाहा॑ शीकि॒ताय॒ स्वाहा᳚ । \newline
10. शी॒कि॒ताय॒ स्वाहा॒ स्वाहा॑ शीकि॒ताय॑ शीकि॒ताय॒ स्वाहा᳚ प्रोषिष्य॒ते प्रो॑षिष्य॒ते स्वाहा॑ शीकि॒ताय॑ शीकि॒ताय॒ स्वाहा᳚ प्रोषिष्य॒ते । \newline
11. स्वाहा᳚ प्रोषिष्य॒ते प्रो॑षिष्य॒ते स्वाहा॒ स्वाहा᳚ प्रोषिष्य॒ते स्वाहा॒ स्वाहा᳚ प्रोषिष्य॒ते स्वाहा॒ स्वाहा᳚ प्रोषिष्य॒ते स्वाहा᳚ । \newline
12. प्रो॒षि॒ष्य॒ते स्वाहा॒ स्वाहा᳚ प्रोषिष्य॒ते प्रो॑षिष्य॒ते स्वाहा᳚ प्रुष्ण॒ते प्रु॑ष्ण॒ते स्वाहा᳚ प्रोषिष्य॒ते प्रो॑षिष्य॒ते स्वाहा᳚ प्रुष्ण॒ते । \newline
13. स्वाहा᳚ प्रुष्ण॒ते प्रु॑ष्ण॒ते स्वाहा॒ स्वाहा᳚ प्रुष्ण॒ते स्वाहा॒ स्वाहा᳚ प्रुष्ण॒ते स्वाहा॒ स्वाहा᳚ प्रुष्ण॒ते स्वाहा᳚ । \newline
14. प्रु॒ष्ण॒ते स्वाहा॒ स्वाहा᳚ प्रुष्ण॒ते प्रु॑ष्ण॒ते स्वाहा॑ परिप्रुष्ण॒ते प॑रिप्रुष्ण॒ते स्वाहा᳚ प्रुष्ण॒ते प्रु॑ष्ण॒ते स्वाहा॑ परिप्रुष्ण॒ते । \newline
15. स्वाहा॑ परिप्रुष्ण॒ते प॑रिप्रुष्ण॒ते स्वाहा॒ स्वाहा॑ परिप्रुष्ण॒ते स्वाहा॒ स्वाहा॑ परिप्रुष्ण॒ते स्वाहा॒ स्वाहा॑ परिप्रुष्ण॒ते स्वाहा᳚ । \newline
16. प॒रि॒प्रु॒ष्ण॒ते स्वाहा॒ स्वाहा॑ परिप्रुष्ण॒ते प॑रिप्रुष्ण॒ते स्वाहो᳚द्ग्रहीष्य॒त उ॑द्ग्रहीष्य॒ते स्वाहा॑ परिप्रुष्ण॒ते प॑रिप्रुष्ण॒ते स्वाहो᳚द्ग्रहीष्य॒ते । \newline
17. प॒रि॒प्रु॒ष्ण॒त इति॑ परि - प्रु॒ष्ण॒ते । \newline
18. स्वाहो᳚द्ग्रहीष्य॒त उ॑द्ग्रहीष्य॒ते स्वाहा॒ स्वाहो᳚द्ग्रहीष्य॒ते स्वाहा॒ स्वाहो᳚द्ग्रहीष्य॒ते स्वाहा॒ 
स्वाहो᳚द्ग्रहीष्य॒ते स्वाहा᳚ । \newline
19. उ॒द्ग्र॒ही॒ष्य॒ते स्वाहा॒ स्वाहो᳚द्ग्रहीष्य॒त उ॑द्ग्रहीष्य॒ते स्वाहो᳚द्‌गृह्ण॒त उ॑द्‌गृह्ण॒ते 
स्वाहो᳚द्‌ग्रहीष्य॒त उ॑द्ग्रहीष्य॒ते स्वाहो᳚द्‌गृह्ण॒ते । \newline
20. उ॒द्ग्र॒ही॒ष्य॒त इत्यु॑त् - ग्र॒ही॒ष्य॒ते । \newline
21. स्वाहो᳚द्‌गृह्ण॒त उ॑द्‌गृह्ण॒ते स्वाहा॒ स्वाहो᳚द्‌गृह्ण॒ते स्वाहा॒ स्वाहो᳚द्‌गृह्ण॒ते स्वाहा॒ 
स्वाहो᳚द्‌गृह्ण॒ते स्वाहा᳚ । \newline
22. उ॒द्‍गृ॒ह्ण॒ते स्वाहा॒ स्वाहो᳚द्‍गृह्ण॒त उ॑द्‍गृह्ण॒ते स्वाहोद्‍गृ॑हीता॒ योद्‍गृ॑हीताय॒ स्वाहो᳚द्‍गृह्ण॒त उ॑द्‍गृह्ण॒ते स्वाहोद्‍गृ॑हीताय । \newline
23. उ॒द्‍गृ॒ह्ण॒त इत्यु॑त् - गृ॒ह्ण॒ते । \newline
24. स्वाहोद्‍गृ॑हीता॒ योद्‍गृ॑हीताय॒ स्वाहा॒ स्वाहोद्‍गृ॑हीताय॒ स्वाहा॒ स्वाहोद्‍गृ॑हीताय॒ स्वाहा॒ स्वाहोद्‍गृ॑हीताय॒ स्वाहा᳚ । \newline
25. उद्‍गृ॑हीताय॒ स्वाहा॒ स्वाहोद्‍गृ॑हीता॒ योद्‍गृ॑हीताय॒ स्वाहा॑ विप्लोष्य॒ते वि॑प्लोष्य॒ते स्वाहोद्‍गृ॑हीता॒
योद्‍गृ॑हीताय॒ स्वाहा॑ विप्लोष्य॒ते । \newline
26. उद्‍गृ॑हीता॒येत्युत् - गृ॒ही॒ता॒य॒ । \newline
27. स्वाहा॑ विप्लोष्य॒ते वि॑प्लोष्य॒ते स्वाहा॒ स्वाहा॑ विप्लोष्य॒ते स्वाहा॒ स्वाहा॑ विप्लोष्य॒ते स्वाहा॒ स्वाहा॑ विप्लोष्य॒ते स्वाहा᳚ । \newline
28. वि॒प्लो॒ष्य॒ते स्वाहा॒ स्वाहा॑ विप्लोष्य॒ते वि॑प्लोष्य॒ते स्वाहा॑ वि॒प्लव॑मानाय वि॒प्लव॑मानाय॒ स्वाहा॑ विप्लोष्य॒ते वि॑प्लोष्य॒ते स्वाहा॑ वि॒प्लव॑मानाय । \newline
29. वि॒प्लो॒ष्य॒त इति॑ वि - प्लो॒ष्य॒ते । \newline
30. स्वाहा॑ वि॒प्लव॑मानाय वि॒प्लव॑मानाय॒ स्वाहा॒ स्वाहा॑ वि॒प्लव॑मानाय॒ स्वाहा॒ स्वाहा॑ वि॒प्लव॑मानाय॒ स्वाहा॒ स्वाहा॑ वि॒प्लव॑मानाय॒ स्वाहा᳚ । \newline
31. वि॒प्लव॑मानाय॒ स्वाहा॒ स्वाहा॑ वि॒प्लव॑मानाय वि॒प्लव॑मानाय॒ स्वाहा॒ विप्लु॑ताय॒ विप्लु॑ताय॒ स्वाहा॑ वि॒प्लव॑मानाय वि॒प्लव॑मानाय॒ स्वाहा॒ विप्लु॑ताय । \newline
32. वि॒प्लव॑माना॒येति॑ वि - प्लव॑मानाय । \newline
33. स्वाहा॒ विप्लु॑ताय॒ विप्लु॑ताय॒ स्वाहा॒ स्वाहा॒ विप्लु॑ताय॒ स्वाहा॒ स्वाहा॒ विप्लु॑ताय॒ स्वाहा॒ स्वाहा॒ विप्लु॑ताय॒ स्वाहा᳚ । \newline
34. विप्लु॑ताय॒ स्वाहा॒ स्वाहा॒ विप्लु॑ताय॒ विप्लु॑ताय॒ स्वाहा॑ ऽऽतफ्स्य॒त आ॑तफ्स्य॒ते स्वाहा॒ विप्लु॑ताय॒ विप्लु॑ताय॒ स्वाहा॑ ऽऽतफ्स्य॒ते । \newline
35. विप्लु॑ता॒येति॒ वि - प्लु॒ता॒य॒ । \newline
36. स्वाहा॑ ऽऽतफ्स्य॒त आ॑तफ्स्य॒ते स्वाहा॒ स्वाहा॑ ऽऽतफ्स्य॒ते स्वाहा॒ स्वाहा॑ ऽऽतफ्स्य॒ते स्वाहा॒ स्वाहा॑ ऽऽतफ्स्य॒ते स्वाहा᳚ । \newline
37. आ॒त॒फ्स्य॒ते स्वाहा॒ स्वाहा॑ ऽऽतफ्स्य॒त आ॑तफ्स्य॒ते स्वाहा॒ ऽऽतप॑त आ॒तप॑ते॒ स्वाहा॑ ऽऽतफ्स्य॒त आ॑तफ्स्य॒ते स्वाहा॒ ऽऽतप॑ते । \newline
38. आ॒त॒फ्स्य॒त इत्या᳚ - त॒फ्स्य॒ते । \newline
39. स्वाहा॒ ऽऽतप॑त आ॒तप॑ते॒ स्वाहा॒ स्वाहा॒ ऽऽतप॑ते॒ स्वाहा॒ स्वाहा॒ ऽऽतप॑ते॒ स्वाहा॒ स्वाहा॒ ऽऽतप॑ते॒ स्वाहा᳚ । \newline
40. आ॒तप॑ते॒ स्वाहा॒ स्वाहा॒ ऽऽतप॑त आ॒तप॑ते॒ स्वाहो॒ग्र मु॒ग्रꣳ स्वाहा॒ ऽऽतप॑त आ॒तप॑ते॒ स्वाहो॒ग्रम् । \newline
41. आ॒तप॑त॒ इत्या᳚ - तप॑ते । \newline
42. स्वाहो॒ग्र मु॒ग्रꣳ स्वाहा॒ स्वाहो॒ग्र मा॒तप॑त आ॒तप॑त उ॒ग्रꣳ स्वाहा॒ स्वाहो॒ग्र मा॒तप॑ते । \newline
43. उ॒ग्र मा॒तप॑त आ॒तप॑त उ॒ग्र मु॒ग्र मा॒तप॑ते॒ स्वाहा॒ स्वाहा॒ ऽऽतप॑त उ॒ग्र मु॒ग्र मा॒तप॑ते॒ स्वाहा᳚ । \newline
44. आ॒तप॑ते॒ स्वाहा॒ स्वाहा॒ ऽऽतप॑त आ॒तप॑ते॒ स्वाह॒ र्‌ग्भ्य ऋ॒ग्भ्यः स्वाहा॒ ऽऽतप॑त आ॒तप॑ते॒ स्वाह॒ र्‌ग्भ्यः । \newline
45. आ॒तप॑त॒ इत्या᳚ - तप॑ते । \newline
46. स्वाह॒ र्‌ग्भ्य ऋ॒ग्भ्यः स्वाहा॒ स्वाह॒ र्‌ग्भ्यः स्वाहा॒ स्वाह॒ र्‌ग्भ्यः स्वाहा॒ स्वाह॒ र्‌ग्भ्यः स्वाहा᳚ । \newline
47. ऋ॒ग्भ्यः स्वाहा॒ स्वाह॒ र्‌ग्भ्य ऋ॒ग्भ्यः स्वाहा॒ यजु॑र्भ्यो॒ यजु॑र्भ्यः॒ स्वाह॒ र्‌ग्भ्य ऋ॒ग्भ्यः स्वाहा॒ यजु॑र्भ्यः । \newline
48. ऋ॒ग्भ्य इत्यृ॑क् - भ्यः । \newline
49. स्वाहा॒ यजु॑र्भ्यो॒ यजु॑र्भ्यः॒ स्वाहा॒ स्वाहा॒ यजु॑र्भ्यः॒ स्वाहा॒ स्वाहा॒ यजु॑र्भ्यः॒ स्वाहा॒ स्वाहा॒ यजु॑र्भ्यः॒ स्वाहा᳚ । \newline
50. यजु॑र्भ्यः॒ स्वाहा॒ स्वाहा॒ यजु॑र्भ्यो॒ यजु॑र्भ्यः॒ स्वाहा॒ साम॑भ्यः॒ साम॑भ्यः॒ स्वाहा॒ यजु॑र्भ्यो॒ यजु॑र्भ्यः॒ स्वाहा॒ साम॑भ्यः । \newline
51. यजु॑र्भ्य॒ इति॒ यजुः॑ - भ्यः॒ । \newline
52. स्वाहा॒ साम॑भ्यः॒ साम॑भ्यः॒ स्वाहा॒ स्वाहा॒ साम॑भ्यः॒ स्वाहा॒ स्वाहा॒ साम॑भ्यः॒ स्वाहा॒ स्वाहा॒ साम॑भ्यः॒ स्वाहा᳚ । \newline
53. साम॑भ्यः॒ स्वाहा॒ स्वाहा॒ साम॑भ्यः॒ साम॑भ्यः॒ स्वाहा ऽङ्गि॑रोभ्यो॒ अङ्गि॑रोभ्यः॒ स्वाहा॒ साम॑भ्यः॒ साम॑भ्यः॒ स्वाहा ऽङ्गि॑रोभ्यः । \newline
54. साम॑भ्य॒ इति॒ साम॑ - भ्यः॒ । \newline
55. स्वाहा ऽङ्गि॑रोभ्यो॒ अङ्गि॑रोभ्यः॒ स्वाहा॒ स्वाहा ऽङ्गि॑रोभ्यः॒ स्वाहा॒ स्वाहा ऽङ्गि॑रोभ्यः॒ स्वाहा॒ स्वाहा ऽङ्गि॑रोभ्यः॒ स्वाहा᳚ । \newline
56. अङ्गि॑रोभ्यः॒ स्वाहा॒ स्वाहा ऽङ्गि॑रोभ्यो॒ अङ्गि॑रोभ्यः॒ स्वाहा॒ वेदे᳚भ्यो॒ वेदे᳚भ्यः॒ स्वाहा ऽङ्गि॑रोभ्यो॒ अङ्गि॑रोभ्यः॒ स्वाहा॒ वेदे᳚भ्यः । \newline
57. अङ्गि॑रोभ्य॒ इत्यङ्गि॑रः - भ्यः॒ । \newline
58. स्वाहा॒ वेदे᳚भ्यो॒ वेदे᳚भ्यः॒ स्वाहा॒ स्वाहा॒ वेदे᳚भ्यः॒ स्वाहा॒ स्वाहा॒ वेदे᳚भ्यः॒ स्वाहा॒ स्वाहा॒ वेदे᳚भ्यः॒ स्वाहा᳚ । \newline
59. वेदे᳚भ्यः॒ स्वाहा॒ स्वाहा॒ वेदे᳚भ्यो॒ वेदे᳚भ्यः॒ स्वाहा॒ गाथा᳚भ्यो॒ गाथा᳚भ्यः॒ स्वाहा॒ वेदे᳚भ्यो॒ वेदे᳚भ्यः॒ स्वाहा॒ गाथा᳚भ्यः । \newline
60. स्वाहा॒ गाथा᳚भ्यो॒ गाथा᳚भ्यः॒ स्वाहा॒ स्वाहा॒ गाथा᳚भ्यः॒ स्वाहा॒ स्वाहा॒ गाथा᳚भ्यः॒ स्वाहा॒ स्वाहा॒ गाथा᳚भ्यः॒ स्वाहा᳚ । \newline
61. गाथा᳚भ्यः॒ स्वाहा॒ स्वाहा॒ गाथा᳚भ्यो॒ गाथा᳚भ्यः॒ स्वाहा॑ नाराशꣳ॒॒सीभ्यो॑ नाराशꣳ॒॒सीभ्यः॒ स्वाहा॒ गाथा᳚भ्यो॒ गाथा᳚भ्यः॒ स्वाहा॑ नाराशꣳ॒॒सीभ्यः॑ । \newline
62. स्वाहा॑ नाराशꣳ॒॒सीभ्यो॑ नाराशꣳ॒॒सीभ्यः॒ स्वाहा॒ स्वाहा॑ नाराशꣳ॒॒सीभ्यः॒ स्वाहा॒ स्वाहा॑ नाराशꣳ॒॒सीभ्यः॒ स्वाहा॒ स्वाहा॑ नाराशꣳ॒॒सीभ्यः॒ स्वाहा᳚ । \newline
63. ना॒रा॒शꣳ॒॒सीभ्यः॒ स्वाहा॒ स्वाहा॑ नाराशꣳ॒॒सीभ्यो॑ नाराशꣳ॒॒सीभ्यः॒ स्वाहा॒ रैभी᳚भ्यो॒ रैभी᳚भ्यः॒ स्वाहा॑ नाराशꣳ॒॒सीभ्यो॑ नाराशꣳ॒॒सीभ्यः॒ स्वाहा॒ रैभी᳚भ्यः । \newline
64. स्वाहा॒ रैभी᳚भ्यो॒ रैभी᳚भ्यः॒ स्वाहा॒ स्वाहा॒ रैभी᳚भ्यः॒ स्वाहा॒ स्वाहा॒ रैभी᳚भ्यः॒ स्वाहा॒ स्वाहा॒ रैभी᳚भ्यः॒ स्वाहा᳚ । \newline
65. रैभी᳚भ्यः॒ स्वाहा॒ स्वाहा॒ रैभी᳚भ्यो॒ रैभी᳚भ्यः॒ स्वाहा॒ सर्व॑स्मै॒ सर्व॑स्मै॒ स्वाहा॒ रैभी᳚भ्यो॒ रैभी᳚भ्यः॒ स्वाहा॒ सर्व॑स्मै । \newline
66. स्वाहा॒ सर्व॑स्मै॒ सर्व॑स्मै॒ स्वाहा॒ स्वाहा॒ सर्व॑स्मै॒ स्वाहा॒ स्वाहा॒ सर्व॑स्मै॒ स्वाहा॒ स्वाहा॒ सर्व॑स्मै॒ स्वाहा᳚ । \newline
67. सर्व॑स्मै॒ स्वाहा॒ स्वाहा॒ सर्व॑स्मै॒ सर्व॑स्मै॒ स्वाहा᳚ । \newline
68. स्वाहेति॒ स्वाहा᳚ । \newline
\pagebreak
\markright{ TS 7.5.12.1  \hfill https://www.vedavms.in \hfill}

\section{ TS 7.5.12.1 }

\textbf{TS 7.5.12.1 } \newline
\textbf{Samhita Paata} \newline

द॒त्वते॒ स्वाहा॑ ऽद॒न्तका॑य॒ स्वाहा᳚ प्रा॒णिने॒ स्वाहा᳚ ऽप्रा॒णाय॒ स्वाहा॒ मुख॑वते॒ स्वाहा॑ऽमु॒खाय॒ स्वाहा॒ नासि॑कवते॒ स्वाहा॑ ऽनासि॒काय॒ स्वाहा᳚ ऽक्ष॒ण्वते॒ स्वाहा॑ऽन॒क्षिका॑य॒ स्वाहा॑ क॒र्णिने॒ स्वाहा॑ ऽक॒र्णका॑य॒ स्वाहा॑ शीर्.ष॒ण्वते॒ स्वाहा॑ऽशी॒र्॒.षका॑य॒ स्वाहा॑ प॒द्वते॒ स्वाहा॑ ऽपा॒दका॑य॒ स्वाहा᳚ प्राण॒ते स्वाहा ऽप्रा॑णते॒ स्वाहा॒ वद॑ते॒ स्वाहा ऽव॑दते॒ स्वाहा॒ पश्य॑ते॒ स्वाहा ऽप॑श्यते॒ स्वाहा॑ शृण्व॒ते स्वाहा ऽशृ॑ण्वते॒ स्वाहा॑ मन॒स्विने॒ स्वाहा॑ - [  ] \newline

\textbf{Pada Paata} \newline

द॒त्वते᳚ । स्वाहा᳚ । अ॒द॒न्तका॑य । स्वाहा᳚ । प्रा॒णिने᳚ । स्वाहा᳚ । अ॒प्रा॒णाय॑ । स्वाहा᳚ । मुख॑वत॒ इति॒ मुख॑-व॒ते॒ । स्वाहा᳚ । अ॒मु॒खाय॑ । स्वाहा᳚ । नासि॑कवत॒ इति॒ नासि॑क - व॒ते॒ । स्वाहा᳚ । अ॒ना॒सि॒काय॑ । स्वाहा᳚ । अ॒क्ष॒ण्वत॒ इत्य॑क्षण्-वते᳚ । स्वाहा᳚ । अ॒न॒क्षिका॑य । स्वाहा᳚ । क॒र्णिने᳚ । स्वाहा᳚ । अ॒क॒र्णका॑य । स्वाहा᳚ । शी॒र्.॒ष॒ण्वत॒ इति॑ शीर्.षण् - वते᳚ । स्वाहा᳚ । अ॒शी॒र्॒.षका॑य । स्वाहा᳚ । प॒द्वत॒ इति॑ पत् - वते᳚ । स्वाहा᳚ । अ॒पा॒दका॑य । स्वाहा᳚ । प्रा॒ण॒त इति॑ प्र-अ॒न॒ते । स्वाहा᳚ । अप्रा॑णत॒ इत्यप्र॑ - अ॒न॒ते॒ । स्वाहा᳚ । वद॑ते । स्वाहा᳚ । अव॑दते । स्वाहा᳚ । पश्य॑ते । स्वाहा᳚ । अप॑श्यते । स्वाहा᳚ । शृ॒ण्व॒ते । स्वाहा᳚ । अशृ॑ण्वते । स्वाहा᳚ । म॒न॒स्विने᳚ । स्वाहा᳚ ।  \newline


\textbf{Krama Paata} \newline

द॒त्वते॒ स्वाहा᳚ । स्वाहा॑ऽद॒न्तका॑य । अ॒द॒न्तका॑य॒ स्वाहा᳚ । स्वाहा᳚ प्रा॒णिने᳚ । प्रा॒णिने॒ स्वाहा᳚ । स्वाहा᳚ऽप्रा॒णाय॑ । अ॒प्रा॒णाय॒ स्वाहा᳚ । स्वाहा॒ मुख॑वते । मुख॑वते॒ स्वाहा᳚ । मुख॑वत॒ इति॒ मुख॑ - व॒ते॒ । स्वाहा॑ऽमु॒खाय॑ । अ॒मु॒खाय॒ स्वाहा᳚ । स्वाहा॒ नासि॑कवते । नासि॑कवते॒ स्वाहा᳚ । नासि॑कवत॒ इति॒ नासि॑क - व॒ते॒ । स्वाहा॑ऽनासि॒काय॑ । अ॒ना॒सि॒काय॒ स्वाहा᳚ । स्वाहा᳚ऽक्ष॒ण्वते᳚ । अ॒क्ष॒ण्वते॒ स्वाहा᳚ । अ॒क्ष॒ण्वत॒ इत्य॑क्षण् - वते᳚ । स्वाहा॑ऽन॒क्षिका॑य । अ॒न॒क्षिका॑य॒ स्वाहा᳚ । स्वाहा॑ क॒र्णिने᳚ । क॒र्णिने॒ स्वाहा᳚ । स्वाहा॑ऽक॒र्णका॑य । अ॒क॒र्णका॑य॒ स्वाहा᳚ । स्वाहा॑ शीर्.ष॒ण्वते᳚ । शी॒र्॒.ष॒ण्वते॒ स्वाहा᳚ । शी॒र्॒.ष॒ण्वत॒ इति॑ शीर्.षण् - वते᳚ । स्वाहा॑ऽशी॒र्॒.षका॑य । अ॒शी॒र्.॒षका॑य॒ स्वाहा᳚ । स्वाहा॑ प॒द्‍वते᳚ । प॒द्‍वते॒ स्वाहा᳚ । प॒द्‍वत॒ इति॑ पत् - वते᳚ । स्वाहा॑ऽपा॒दका॑य । अ॒पा॒दका॑य॒ स्वाहा᳚ । स्वाहा᳚ प्राण॒ते । प्रा॒ण॒ते स्वाहा᳚ । प्रा॒ण॒त इति॑ प्र - अ॒न॒ते । स्वाहाऽप्रा॑णते । अप्रा॑णते॒ स्वाहा᳚ । अप्रा॑णत॒ इत्यप्र॑ - अ॒न॒ते॒ । स्वाहा॒ वद॑ते । वद॑ते॒ स्वाहा᳚ । स्वाहाऽव॑दते । अव॑दते॒ स्वाहा᳚ । स्वाहा॒ पश्य॑ते । पश्य॑ते॒ स्वाहा᳚ । स्वाहाऽप॑श्यते । अप॑श्यते॒ स्वाहा᳚ । स्वाहा॑ शृण्व॒ते । शृ॒ण्व॒ते स्वाहा᳚ । स्वाहाऽशृ॑ण्वते । अशृ॑ण्वते॒ स्वाहा᳚ । स्वाहा॑ मन॒स्विने᳚ । म॒न॒स्विने॒ स्वाहा᳚ । स्वाहा॑ऽम॒नसे᳚ \newline

\textbf{Jatai Paata} \newline

1. द॒त्वते॒ स्वाहा॒ स्वाहा॑ द॒त्वते॑ द॒त्वते॒ स्वाहा᳚ । \newline
2. स्वाहा॑ ऽद॒न्तका॑या द॒न्तका॑य॒ स्वाहा॒ स्वाहा॑ ऽद॒न्तका॑य । \newline
3. अ॒द॒न्तका॑य॒ स्वाहा॒ स्वाहा॑ ऽद॒न्तका॑या द॒न्तका॑य॒ स्वाहा᳚ । \newline
4. स्वाहा᳚ प्रा॒णिने᳚ प्रा॒णिने॒ स्वाहा॒ स्वाहा᳚ प्रा॒णिने᳚ । \newline
5. प्रा॒णिने॒ स्वाहा॒ स्वाहा᳚ प्रा॒णिने᳚ प्रा॒णिने॒ स्वाहा᳚ । \newline
6. स्वाहा᳚ ऽप्रा॒णाया᳚ प्रा॒णाय॒ स्वाहा॒ स्वाहा᳚ ऽप्रा॒णाय॑ । \newline
7. अ॒प्रा॒णाय॒ स्वाहा॒ स्वाहा᳚ ऽप्रा॒णाया᳚ प्रा॒णाय॒ स्वाहा᳚ । \newline
8. स्वाहा॒ मुख॑वते॒ मुख॑वते॒ स्वाहा॒ स्वाहा॒ मुख॑वते । \newline
9. मुख॑वते॒ स्वाहा॒ स्वाहा॒ मुख॑वते॒ मुख॑वते॒ स्वाहा᳚ । \newline
10. मुख॑वत॒ इति॒ मुख॑ - व॒ते॒ । \newline
11. स्वाहा॑ ऽमु॒खाया॑ मु॒खाय॒ स्वाहा॒ स्वाहा॑ ऽमु॒खाय॑ । \newline
12. अ॒मु॒खाय॒ स्वाहा॒ स्वाहा॑ ऽमु॒खाया॑ मु॒खाय॒ स्वाहा᳚ । \newline
13. स्वाहा॒ नासि॑कवते॒ नासि॑कवते॒ स्वाहा॒ स्वाहा॒ नासि॑कवते । \newline
14. नासि॑कवते॒ स्वाहा॒ स्वाहा॒ नासि॑कवते॒ नासि॑कवते॒ स्वाहा᳚ । \newline
15. नासि॑कवत॒ इति॒ नासि॑क - व॒ते॒ । \newline
16. स्वाहा॑ ऽनासि॒काया॑ नासि॒काय॒ स्वाहा॒ स्वाहा॑ ऽनासि॒काय॑ । \newline
17. अ॒ना॒सि॒काय॒ स्वाहा॒ स्वाहा॑ ऽनासि॒काया॑ नासि॒काय॒ स्वाहा᳚ । \newline
18. स्वाहा᳚ ऽक्ष॒ण्वते᳚ ऽक्ष॒ण्वते॒ स्वाहा॒ स्वाहा᳚ ऽक्ष॒ण्वते᳚ । \newline
19. अ॒क्ष॒ण्वते॒ स्वाहा॒ स्वाहा᳚ ऽक्ष॒ण्वते᳚ ऽक्ष॒ण्वते॒ स्वाहा᳚ । \newline
20. अ॒क्ष॒ण्वत॒ इत्य॑क्षण् - वते᳚ । \newline
21. स्वाहा॑ ऽन॒क्षिका॑या न॒क्षिका॑य॒ स्वाहा॒ स्वाहा॑ ऽन॒क्षिका॑य । \newline
22. अ॒न॒क्षिका॑य॒ स्वाहा॒ स्वाहा॑ ऽन॒क्षिका॑या न॒क्षिका॑य॒ स्वाहा᳚ । \newline
23. स्वाहा॑ क॒र्णिने॑ क॒र्णिने॒ स्वाहा॒ स्वाहा॑ क॒र्णिने᳚ । \newline
24. क॒र्णिने॒ स्वाहा॒ स्वाहा॑ क॒र्णिने॑ क॒र्णिने॒ स्वाहा᳚ । \newline
25. स्वाहा॑ ऽक॒र्णका॑या क॒र्णका॑य॒ स्वाहा॒ स्वाहा॑ ऽक॒र्णका॑य । \newline
26. अ॒क॒र्णका॑य॒ स्वाहा॒ स्वाहा॑ ऽक॒र्णका॑या क॒र्णका॑य॒ स्वाहा᳚ । \newline
27. स्वाहा॑ शीर्.ष॒ण्वते॑ शीर्.ष॒ण्वते॒ स्वाहा॒ स्वाहा॑ शीर्.ष॒ण्वते᳚ । \newline
28. शी॒र्॒.ष॒ण्वते॒ स्वाहा॒ स्वाहा॑ शीर्.ष॒ण्वते॑ शीर्.ष॒ण्वते॒ स्वाहा᳚ । \newline
29. शी॒र्.॒ष॒ण्वत॒ इति॑ शीर्.षण् - वते᳚ । \newline
30. स्वाहा॑ ऽशी॒र्॒.षका॑या शी॒र्॒.षका॑य॒ स्वाहा॒ स्वाहा॑ ऽशी॒र्॒.षका॑य । \newline
31. अ॒शी॒र्॒.षका॑य॒ स्वाहा॒ स्वाहा॑ ऽशी॒र्॒.षका॑या शी॒र्॒.षका॑य॒ स्वाहा᳚ । \newline
32. स्वाहा॑ प॒द्वते॑ प॒द्वते॒ स्वाहा॒ स्वाहा॑ प॒द्वते᳚ । \newline
33. प॒द्वते॒ स्वाहा॒ स्वाहा॑ प॒द्वते॑ प॒द्वते॒ स्वाहा᳚ । \newline
34. प॒द्वत॒ इति॑ पत् - वते᳚ । \newline
35. स्वाहा॑ ऽपा॒दका॑या पा॒दका॑य॒ स्वाहा॒ स्वाहा॑ ऽपा॒दका॑य । \newline
36. अ॒पा॒दका॑य॒ स्वाहा॒ स्वाहा॑ ऽपा॒दका॑या पा॒दका॑य॒ स्वाहा᳚ । \newline
37. स्वाहा᳚ प्राण॒ते प्रा॑ण॒ते स्वाहा॒ स्वाहा᳚ प्राण॒ते । \newline
38. प्रा॒ण॒ते स्वाहा॒ स्वाहा᳚ प्राण॒ते प्रा॑ण॒ते स्वाहा᳚ । \newline
39. प्रा॒ण॒त इति॑ प्र - अ॒न॒ते । \newline
40. स्वाहा ऽप्रा॑ण॒ते ऽप्रा॑णते॒ स्वाहा॒ स्वाहा ऽप्रा॑णते । \newline
41. अप्रा॑णते॒ स्वाहा॒ स्वाहा ऽप्रा॑ण॒ते ऽप्रा॑णते॒ स्वाहा᳚ । \newline
42. अप्रा॑णत॒ इत्यप्र॑ - अ॒न॒ते॒ । \newline
43. स्वाहा॒ वद॑ते॒ वद॑ते॒ स्वाहा॒ स्वाहा॒ वद॑ते । \newline
44. वद॑ते॒ स्वाहा॒ स्वाहा॒ वद॑ते॒ वद॑ते॒ स्वाहा᳚ । \newline
45. स्वाहा ऽव॑द॒ते ऽव॑दते॒ स्वाहा॒ स्वाहा ऽव॑दते । \newline
46. अव॑दते॒ स्वाहा॒ स्वाहा ऽव॑द॒ते ऽव॑दते॒ स्वाहा᳚ । \newline
47. स्वाहा॒ पश्य॑ते॒ पश्य॑ते॒ स्वाहा॒ स्वाहा॒ पश्य॑ते । \newline
48. पश्य॑ते॒ स्वाहा॒ स्वाहा॒ पश्य॑ते॒ पश्य॑ते॒ स्वाहा᳚ । \newline
49. स्वाहा ऽप॑श्य॒ते ऽप॑श्यते॒ स्वाहा॒ स्वाहा ऽप॑श्यते । \newline
50. अप॑श्यते॒ स्वाहा॒ स्वाहा ऽप॑श्य॒ते ऽप॑श्यते॒ स्वाहा᳚ । \newline
51. स्वाहा॑ शृण्व॒ते शृ॑ण्व॒ते स्वाहा॒ स्वाहा॑ शृण्व॒ते । \newline
52. शृ॒ण्व॒ते स्वाहा॒ स्वाहा॑ शृण्व॒ते शृ॑ण्व॒ते स्वाहा᳚ । \newline
53. स्वाहा ऽशृ॑ण्व॒ते ऽशृ॑ण्वते॒ स्वाहा॒ स्वाहा ऽशृ॑ण्वते । \newline
54. अशृ॑ण्वते॒ स्वाहा॒ स्वाहा ऽशृ॑ण्व॒ते ऽशृ॑ण्वते॒ स्वाहा᳚ । \newline
55. स्वाहा॑ मन॒स्विने॑ मन॒स्विने॒ स्वाहा॒ स्वाहा॑ मन॒स्विने᳚ । \newline
56. म॒न॒स्विने॒ स्वाहा॒ स्वाहा॑ मन॒स्विने॑ मन॒स्विने॒ स्वाहा᳚ । \newline
57. स्वाहा॑ ऽम॒नसे॑ ऽम॒नसे॒ स्वाहा॒ स्वाहा॑ ऽम॒नसे᳚ । \newline

\textbf{Ghana Paata } \newline

1. द॒त्वते॒ स्वाहा॒ स्वाहा॑ द॒त्वते॑ द॒त्वते॒ स्वाहा॑ ऽद॒न्तका॑या द॒न्तका॑य॒ स्वाहा॑ द॒त्वते॑ द॒त्वते॒ स्वाहा॑ ऽद॒न्तका॑य । \newline
2. स्वाहा॑ ऽद॒न्तका॑या द॒न्तका॑य॒ स्वाहा॒ स्वाहा॑ ऽद॒न्तका॑य॒ स्वाहा॒ स्वाहा॑ ऽद॒न्तका॑य॒ स्वाहा॒ स्वाहा॑ ऽद॒न्तका॑य॒ स्वाहा᳚ । \newline
3. अ॒द॒न्तका॑य॒ स्वाहा॒ स्वाहा॑ ऽद॒न्तका॑या द॒न्तका॑य॒ स्वाहा᳚ प्रा॒णिने᳚ प्रा॒णिने॒ स्वाहा॑ ऽद॒न्तका॑या द॒न्तका॑य॒ स्वाहा᳚ प्रा॒णिने᳚ । \newline
4. स्वाहा᳚ प्रा॒णिने᳚ प्रा॒णिने॒ स्वाहा॒ स्वाहा᳚ प्रा॒णिने॒ स्वाहा॒ स्वाहा᳚ प्रा॒णिने॒ स्वाहा॒ स्वाहा᳚ प्रा॒णिने॒ स्वाहा᳚ । \newline
5. प्रा॒णिने॒ स्वाहा॒ स्वाहा᳚ प्रा॒णिने᳚ प्रा॒णिने॒ स्वाहा᳚ ऽप्रा॒णाया᳚ प्रा॒णाय॒ स्वाहा᳚ प्रा॒णिने᳚ प्रा॒णिने॒ स्वाहा᳚ ऽप्रा॒णाय॑ । \newline
6. स्वाहा᳚ ऽप्रा॒णाया᳚ प्रा॒णाय॒ स्वाहा॒ स्वाहा᳚ ऽप्रा॒णाय॒ स्वाहा॒ स्वाहा᳚ ऽप्रा॒णाय॒ स्वाहा॒ स्वाहा᳚ ऽप्रा॒णाय॒ स्वाहा᳚ । \newline
7. अ॒प्रा॒णाय॒ स्वाहा॒ स्वाहा᳚ ऽप्रा॒णाया᳚ प्रा॒णाय॒ स्वाहा॒ मुख॑वते॒ मुख॑वते॒ स्वाहा᳚ ऽप्रा॒णाया᳚ प्रा॒णाय॒ स्वाहा॒ मुख॑वते । \newline
8. स्वाहा॒ मुख॑वते॒ मुख॑वते॒ स्वाहा॒ स्वाहा॒ मुख॑वते॒ स्वाहा॒ स्वाहा॒ मुख॑वते॒ स्वाहा॒ स्वाहा॒ मुख॑वते॒ स्वाहा᳚ । \newline
9. मुख॑वते॒ स्वाहा॒ स्वाहा॒ मुख॑वते॒ मुख॑वते॒ स्वाहा॑ ऽमु॒खाया॑ मु॒खाय॒ स्वाहा॒ मुख॑वते॒ मुख॑वते॒ स्वाहा॑ ऽमु॒खाय॑ । \newline
10. मुख॑वत॒ इति॒ मुख॑ - व॒ते॒ । \newline
11. स्वाहा॑ ऽमु॒खाया॑ मु॒खाय॒ स्वाहा॒ स्वाहा॑ ऽमु॒खाय॒ स्वाहा॒ स्वाहा॑ ऽमु॒खाय॒ स्वाहा॒ स्वाहा॑ ऽमु॒खाय॒ स्वाहा᳚ । \newline
12. अ॒मु॒खाय॒ स्वाहा॒ स्वाहा॑ ऽमु॒खाया॑ मु॒खाय॒ स्वाहा॒ नासि॑कवते॒ नासि॑कवते॒ स्वाहा॑ ऽमु॒खाया॑ मु॒खाय॒ स्वाहा॒ नासि॑कवते । \newline
13. स्वाहा॒ नासि॑कवते॒ नासि॑कवते॒ स्वाहा॒ स्वाहा॒ नासि॑कवते॒ स्वाहा॒ स्वाहा॒ नासि॑कवते॒ स्वाहा॒ स्वाहा॒ नासि॑कवते॒ स्वाहा᳚ । \newline
14. नासि॑कवते॒ स्वाहा॒ स्वाहा॒ नासि॑कवते॒ नासि॑कवते॒ स्वाहा॑ ऽनासि॒काया॑ नासि॒काय॒ स्वाहा॒ नासि॑कवते॒ नासि॑कवते॒ स्वाहा॑ ऽनासि॒काय॑ । \newline
15. नासि॑कवत॒ इति॒ नासि॑क - व॒ते॒ । \newline
16. स्वाहा॑ ऽनासि॒काया॑ नासि॒काय॒ स्वाहा॒ स्वाहा॑ ऽनासि॒काय॒ स्वाहा॒ स्वाहा॑ ऽनासि॒काय॒ स्वाहा॒ स्वाहा॑ ऽनासि॒काय॒ स्वाहा᳚ । \newline
17. अ॒ना॒सि॒काय॒ स्वाहा॒ स्वाहा॑ ऽनासि॒काया॑ नासि॒काय॒ स्वाहा᳚ ऽक्ष॒ण्वते᳚ ऽक्ष॒ण्वते॒ स्वाहा॑ ऽनासि॒काया॑ नासि॒काय॒ स्वाहा᳚ ऽक्ष॒ण्वते᳚ । \newline
18. स्वाहा᳚ ऽक्ष॒ण्वते᳚ ऽक्ष॒ण्वते॒ स्वाहा॒ स्वाहा᳚ ऽक्ष॒ण्वते॒ स्वाहा॒ स्वाहा᳚ ऽक्ष॒ण्वते॒ स्वाहा॒ स्वाहा᳚ ऽक्ष॒ण्वते॒ स्वाहा᳚ । \newline
19. अ॒क्ष॒ण्वते॒ स्वाहा॒ स्वाहा᳚ ऽक्ष॒ण्वते᳚ ऽक्ष॒ण्वते॒ स्वाहा॑ ऽन॒क्षिका॑या न॒क्षिका॑य॒ स्वाहा᳚ ऽक्ष॒ण्वते᳚ ऽक्ष॒ण्वते॒ स्वाहा॑ ऽन॒क्षिका॑य । \newline
20. अ॒क्ष॒ण्वत॒ इत्य॑क्षण् - वते᳚ । \newline
21. स्वाहा॑ ऽन॒क्षिका॑या न॒क्षिका॑य॒ स्वाहा॒ स्वाहा॑ ऽन॒क्षिका॑य॒ स्वाहा॒ स्वाहा॑ ऽन॒क्षिका॑य॒ स्वाहा॒ स्वाहा॑ ऽन॒क्षिका॑य॒ स्वाहा᳚ । \newline
22. अ॒न॒क्षिका॑य॒ स्वाहा॒ स्वाहा॑ ऽन॒क्षिका॑या न॒क्षिका॑य॒ स्वाहा॑ क॒र्णिने॑ क॒र्णिने॒ स्वाहा॑ ऽन॒क्षिका॑या न॒क्षिका॑य॒ स्वाहा॑ क॒र्णिने᳚ । \newline
23. स्वाहा॑ क॒र्णिने॑ क॒र्णिने॒ स्वाहा॒ स्वाहा॑ क॒र्णिने॒ स्वाहा॒ स्वाहा॑ क॒र्णिने॒ स्वाहा॒ स्वाहा॑ क॒र्णिने॒ स्वाहा᳚ । \newline
24. क॒र्णिने॒ स्वाहा॒ स्वाहा॑ क॒र्णिने॑ क॒र्णिने॒ स्वाहा॑ ऽक॒र्णका॑या क॒र्णका॑य॒ स्वाहा॑ क॒र्णिने॑ क॒र्णिने॒ स्वाहा॑ ऽक॒र्णका॑य । \newline
25. स्वाहा॑ ऽक॒र्णका॑या क॒र्णका॑य॒ स्वाहा॒ स्वाहा॑ ऽक॒र्णका॑य॒ स्वाहा॒ स्वाहा॑ ऽक॒र्णका॑य॒ स्वाहा॒ स्वाहा॑ ऽक॒र्णका॑य॒ स्वाहा᳚ । \newline
26. अ॒क॒र्णका॑य॒ स्वाहा॒ स्वाहा॑ ऽक॒र्णका॑या क॒र्णका॑य॒ स्वाहा॑ शीर्.ष॒ण्वते॑ शीर्.ष॒ण्वते॒ स्वाहा॑ ऽक॒र्णका॑या क॒र्णका॑य॒ स्वाहा॑ शीर्.ष॒ण्वते᳚ । \newline
27. स्वाहा॑ शीर्.ष॒ण्वते॑ शीर्.ष॒ण्वते॒ स्वाहा॒ स्वाहा॑ शीर्.ष॒ण्वते॒ स्वाहा॒ स्वाहा॑ शीर्.ष॒ण्वते॒ स्वाहा॒ स्वाहा॑ शीर्.ष॒ण्वते॒ स्वाहा᳚ । \newline
28. शी॒र्॒.ष॒ण्वते॒ स्वाहा॒ स्वाहा॑ शीर्.ष॒ण्वते॑ शीर्.ष॒ण्वते॒ स्वाहा॑ ऽशी॒र्॒.षका॑या शी॒र्॒.षका॑य॒ स्वाहा॑ शीर्.ष॒ण्वते॑ शीर्.ष॒ण्वते॒ स्वाहा॑ ऽशी॒र्॒.षका॑य । \newline
29. शी॒र्.॒ष॒ण्वत॒ इति॑ शीर्.षण् - वते᳚ । \newline
30. स्वाहा॑ ऽशी॒र्॒.षका॑या शी॒र्॒.षका॑य॒ स्वाहा॒ स्वाहा॑ ऽशी॒र्॒.षका॑य॒ स्वाहा॒ स्वाहा॑ ऽशी॒र्॒.षका॑य॒ स्वाहा॒ स्वाहा॑ ऽशी॒र्॒.षका॑य॒ स्वाहा᳚ । \newline
31. अ॒शी॒र्॒.षका॑य॒ स्वाहा॒ स्वाहा॑ ऽशी॒र्॒.षका॑या शी॒र्॒.षका॑य॒ स्वाहा॑ प॒द्वते॑ प॒द्वते॒ स्वाहा॑ ऽशी॒र्॒.षका॑या शी॒र्॒.षका॑य॒ स्वाहा॑ प॒द्वते᳚ । \newline
32. स्वाहा॑ प॒द्वते॑ प॒द्वते॒ स्वाहा॒ स्वाहा॑ प॒द्वते॒ स्वाहा॒ स्वाहा॑ प॒द्वते॒ स्वाहा॒ स्वाहा॑ प॒द्वते॒ स्वाहा᳚ । \newline
33. प॒द्वते॒ स्वाहा॒ स्वाहा॑ प॒द्वते॑ प॒द्वते॒ स्वाहा॑ ऽपा॒दका॑या पा॒दका॑य॒ स्वाहा॑ प॒द्वते॑ प॒द्वते॒ स्वाहा॑ ऽपा॒दका॑य । \newline
34. प॒द्वत॒ इति॑ पत् - वते᳚ । \newline
35. स्वाहा॑ ऽपा॒दका॑या पा॒दका॑य॒ स्वाहा॒ स्वाहा॑ ऽपा॒दका॑य॒ स्वाहा॒ स्वाहा॑ ऽपा॒दका॑य॒ स्वाहा॒ स्वाहा॑ ऽपा॒दका॑य॒ स्वाहा᳚ । \newline
36. अ॒पा॒दका॑य॒ स्वाहा॒ स्वाहा॑ ऽपा॒दका॑या पा॒दका॑य॒ स्वाहा᳚ प्राण॒ते प्रा॑ण॒ते स्वाहा॑ ऽपा॒दका॑या पा॒दका॑य॒ स्वाहा᳚ प्राण॒ते । \newline
37. स्वाहा᳚ प्राण॒ते प्रा॑ण॒ते स्वाहा॒ स्वाहा᳚ प्राण॒ते स्वाहा॒ स्वाहा᳚ प्राण॒ते स्वाहा॒ स्वाहा᳚ प्राण॒ते स्वाहा᳚ । \newline
38. प्रा॒ण॒ते स्वाहा॒ स्वाहा᳚ प्राण॒ते प्रा॑ण॒ते स्वाहा ऽप्रा॑ण॒ते ऽप्रा॑णते॒ स्वाहा᳚ प्राण॒ते प्रा॑ण॒ते स्वाहा ऽप्रा॑णते । \newline
39. प्रा॒ण॒त इति॑ प्र - अ॒न॒ते । \newline
40. स्वाहा ऽप्रा॑ण॒ते ऽप्रा॑णते॒ स्वाहा॒ स्वाहा ऽप्रा॑णते॒ स्वाहा॒ स्वाहा ऽप्रा॑णते॒ स्वाहा॒ स्वाहा ऽप्रा॑णते॒ स्वाहा᳚ । \newline
41. अप्रा॑णते॒ स्वाहा॒ स्वाहा ऽप्रा॑ण॒ते ऽप्रा॑णते॒ स्वाहा॒ वद॑ते॒ वद॑ते॒ स्वाहा ऽप्रा॑ण॒ते ऽप्रा॑णते॒ स्वाहा॒ वद॑ते । \newline
42. अप्रा॑णत॒ इत्यप्र॑ - अ॒न॒ते॒ । \newline
43. स्वाहा॒ वद॑ते॒ वद॑ते॒ स्वाहा॒ स्वाहा॒ वद॑ते॒ स्वाहा॒ स्वाहा॒ वद॑ते॒ स्वाहा॒ स्वाहा॒ वद॑ते॒ स्वाहा᳚ । \newline
44. वद॑ते॒ स्वाहा॒ स्वाहा॒ वद॑ते॒ वद॑ते॒ स्वाहा ऽव॑द॒ते ऽव॑दते॒ स्वाहा॒ वद॑ते॒ वद॑ते॒ स्वाहा ऽव॑दते । \newline
45. स्वाहा ऽव॑द॒ते ऽव॑दते॒ स्वाहा॒ स्वाहा ऽव॑दते॒ स्वाहा॒ स्वाहा ऽव॑दते॒ स्वाहा॒ स्वाहा ऽव॑दते॒ स्वाहा᳚ । \newline
46. अव॑दते॒ स्वाहा॒ स्वाहा ऽव॑द॒ते ऽव॑दते॒ स्वाहा॒ पश्य॑ते॒ पश्य॑ते॒ स्वाहा ऽव॑द॒ते ऽव॑दते॒ स्वाहा॒ पश्य॑ते । \newline
47. स्वाहा॒ पश्य॑ते॒ पश्य॑ते॒ स्वाहा॒ स्वाहा॒ पश्य॑ते॒ स्वाहा॒ स्वाहा॒ पश्य॑ते॒ स्वाहा॒ स्वाहा॒ पश्य॑ते॒ स्वाहा᳚ । \newline
48. पश्य॑ते॒ स्वाहा॒ स्वाहा॒ पश्य॑ते॒ पश्य॑ते॒ स्वाहा ऽप॑श्य॒ते ऽप॑श्यते॒ स्वाहा॒ पश्य॑ते॒ पश्य॑ते॒ स्वाहा ऽप॑श्यते । \newline
49. स्वाहा ऽप॑श्य॒ते ऽप॑श्यते॒ स्वाहा॒ स्वाहा ऽप॑श्यते॒ स्वाहा॒ स्वाहा ऽप॑श्यते॒ स्वाहा॒ स्वाहा ऽप॑श्यते॒ स्वाहा᳚ । \newline
50. अप॑श्यते॒ स्वाहा॒ स्वाहा ऽप॑श्य॒ते ऽप॑श्यते॒ स्वाहा॑ शृण्व॒ते शृ॑ण्व॒ते स्वाहा ऽप॑श्य॒ते ऽप॑श्यते॒ स्वाहा॑ शृण्व॒ते । \newline
51. स्वाहा॑ शृण्व॒ते शृ॑ण्व॒ते स्वाहा॒ स्वाहा॑ शृण्व॒ते स्वाहा॒ स्वाहा॑ शृण्व॒ते स्वाहा॒ स्वाहा॑ शृण्व॒ते स्वाहा᳚ । \newline
52. शृ॒ण्व॒ते स्वाहा॒ स्वाहा॑ शृण्व॒ते शृ॑ण्व॒ते स्वाहा ऽशृ॑ण्व॒ते ऽशृ॑ण्वते॒ स्वाहा॑ शृण्व॒ते शृ॑ण्व॒ते स्वाहा ऽशृ॑ण्वते । \newline
53. स्वाहा ऽशृ॑ण्व॒ते ऽशृ॑ण्वते॒ स्वाहा॒ स्वाहा ऽशृ॑ण्वते॒ स्वाहा॒ स्वाहा ऽशृ॑ण्वते॒ स्वाहा॒ स्वाहा ऽशृ॑ण्वते॒ स्वाहा᳚ । \newline
54. अशृ॑ण्वते॒ स्वाहा॒ स्वाहा ऽशृ॑ण्व॒ते ऽशृ॑ण्वते॒ स्वाहा॑ मन॒स्विने॑ मन॒स्विने॒ स्वाहा ऽशृ॑ण्व॒ते ऽशृ॑ण्वते॒ स्वाहा॑ मन॒स्विने᳚ । \newline
55. स्वाहा॑ मन॒स्विने॑ मन॒स्विने॒ स्वाहा॒ स्वाहा॑ मन॒स्विने॒ स्वाहा॒ स्वाहा॑ मन॒स्विने॒ स्वाहा॒ स्वाहा॑ मन॒स्विने॒ स्वाहा᳚ । \newline
56. म॒न॒स्विने॒ स्वाहा॒ स्वाहा॑ मन॒स्विने॑ मन॒स्विने॒ स्वाहा॑ ऽम॒नसे॑ ऽम॒नसे॒ स्वाहा॑ मन॒स्विने॑ मन॒स्विने॒ स्वाहा॑ ऽम॒नसे᳚ । \newline
57. स्वाहा॑ ऽम॒नसे॑ ऽम॒नसे॒ स्वाहा॒ स्वाहा॑ ऽम॒नसे॒ स्वाहा॒ स्वाहा॑ ऽम॒नसे॒ स्वाहा॒ स्वाहा॑ ऽम॒नसे॒ स्वाहा᳚ । \newline
\pagebreak
\markright{ TS 7.5.12.2  \hfill https://www.vedavms.in \hfill}

\section{ TS 7.5.12.2 }

\textbf{TS 7.5.12.2 } \newline
\textbf{Samhita Paata} \newline

ऽम॒नसे॒ स्वाहा॑ रेत॒स्विने॒ स्वाहा॑ ऽरे॒तस्का॑य॒ स्वाहा᳚ प्र॒जाभ्यः॒ स्वाहा᳚ प्र॒जन॑नाय॒ स्वाहा॒ लोम॑वते॒ स्वाहा॑ ऽलो॒मका॑य॒ स्वाहा᳚ त्व॒चे स्वाहा॒ ऽत्वक्का॑य॒ स्वाहा॒ चर्म॑ण्वते॒ स्वाहा॑ ऽच॒र्मका॑य॒ स्वाहा॒ लोहि॑तवते॒ स्वाहा॑ऽलोहि॒ताय॒ स्वाहा॑ माꣳस॒न्वते॒ स्वाहा॑ ऽमाꣳ॒॒सका॑य॒ स्वाहा॒ स्नाव॑भ्यः॒ स्वाहा᳚ ऽस्ना॒वका॑य॒ स्वाहा᳚ स्थ॒न्वते॒ स्वाहा॑ऽन॒स्थिका॑य॒ स्वाहा॑ मज्ज॒न्वते॒ स्वाहा॑ ऽम॒ज्जका॑य॒ स्वाहा॒ ऽङ्गिने॒ स्वाहा॑ऽन॒ङ्गाय॒ स्वाहा॒ ऽऽत्मने॒ स्वाहा ऽना᳚त्मने॒ स्वाहा॒ ( ) सर्व॑स्मै॒ स्वाहा᳚ ॥ \newline

\textbf{Pada Paata} \newline

अ॒म॒नसे᳚ । स्वाहा᳚ । रे॒त॒स्विने᳚ । स्वाहा᳚ । अ॒रे॒तस्का॒येत्य॑रे॒तः - का॒य॒ । स्वाहा᳚ । प्र॒जाभ्य॒ इति॑ प्र - जाभ्यः॑ । स्वाहा᳚ । प्र॒जन॑ना॒येति॑ प्र - जन॑नाय । स्वाहा᳚ । लोम॑वत॒ इति॒ लोम॑ - व॒ते॒ । स्वाहा᳚ । अ॒लो॒मका॑य । स्वाहा᳚ । त्व॒चे । स्वाहा᳚ । अ॒त्वक्का॑य । स्वाहा᳚ । चर्म॑ण्वत॒ इति॒ चर्मण्॑ -   व॒ते॒ । स्वाहा᳚ । अ॒च॒र्मका॑य । स्वाहा᳚ । लोहि॑तवत॒ इति॒ लोहि॑त - व॒ते॒ । स्वाहा᳚ । अ॒लो॒हि॒ताय॑ । स्वाहा᳚ । माꣳ॒॒स॒न्वत॒ इति॑ माꣳसन्न्-वते᳚ । स्वाहा᳚ । अ॒माꣳ॒॒सका॑य । स्वाहा᳚ । स्नाव॑भ्य॒ इति॒ स्नाव॑ - भ्यः॒ । स्वाहा᳚ । अ॒स्ना॒वका॑य । स्वाहा᳚ । अ॒स्थ॒न्वत॒ इत्य॑स्थन्न् - वते᳚ । स्वाहा᳚ । अ॒न॒स्थिका॑य । स्वाहा᳚ । म॒ज्ज॒न्वत॒ इति॑ मज्जन्न् - वते᳚ । स्वाहा᳚ । अ॒म॒ज्जका॑य । स्वाहा᳚ । अ॒ङ्गिने᳚ । स्वाहा᳚ । अ॒न॒ङ्गाय॑ । स्वाहा᳚ । आ॒त्मने᳚ । स्वाहा᳚ । अना᳚त्मने । स्वाहा᳚ ( ) । सर्व॑स्मै । स्वाहा᳚ ॥  \newline


\textbf{Krama Paata} \newline

अ॒म॒नसे॒ स्वाहा᳚ । स्वाहा॑ रेत॒स्विने᳚ । रे॒त॒स्विने॒ स्वाहा᳚ । स्वाहा॑ऽरे॒तस्का॑य । अ॒रे॒तस्का॑य॒ स्वाहा᳚ । अ॒रे॒तस्का॒येत्य॑रे॒तः - का॒य॒ । स्वाहा᳚ प्र॒जाभ्यः॑ । प्र॒जाभ्यः॒ स्वाहा᳚ । प्र॒जाभ्य॒ इति॑ प्र - जाभ्यः॑ । स्वाहा᳚ प्र॒जन॑नाय । प्र॒जन॑नाय॒ स्वाहा᳚ । प्र॒जन॑ना॒येति॑ प्र - जन॑नाय । स्वाहा॒ लोम॑वते । लोम॑वते॒ स्वाहा᳚ । लोम॑वत॒ इति॒ लोम॑ - व॒ते॒ । स्वाहा॑ऽलो॒मका॑य । अ॒लो॒मका॑य॒ स्वाहा᳚ । स्वाहा᳚ त्व॒चे । त्व॒चे स्वाहा᳚ । स्वाहा॒ऽत्वक्का॑य । अ॒त्वक्का॑य॒ स्वाहा᳚ । स्वाहा॒ चर्म॑ण्वते । चर्म॑ण्वते॒ स्वाहा᳚ । चर्म॑ण्वत॒ इति॒ चर्मण्ण्॑ - व॒ते॒ । स्वाहा॑ऽच॒र्मका॑य । अ॒च॒र्मका॑य॒ स्वाहा᳚ । स्वाहा॒ लोहि॑तवते । लोहि॑तवते॒ स्वाहा᳚ । लोहि॑तवत॒ इति॒ लोहि॑त - व॒ते॒ । स्वाहा॑ऽलोहि॒ताय॑ । अ॒लो॒हि॒ताय॒ स्वाहा᳚ । स्वाहा॑ माꣳस॒न्वते᳚ । माꣳ॒॒स॒न्वते॒ स्वाहा᳚ । माꣳ॒॒स॒न्वत॒ इति॑ माꣳसन्न् - वते᳚ । स्वाहा॑ऽमाꣳ॒॒सका॑य । अ॒माꣳ॒॒सका॑य॒ स्वाहा᳚ । स्वाहा॒ स्नाव॑भ्यः । स्नाव॑भ्यः॒ स्वाहा᳚ । स्नाव॑भ्य॒ इति॒ स्नाव॑ - भ्यः॒ । स्वाहा᳚ऽस्ना॒वका॑य । अ॒स्ना॒वका॑य॒ स्वाहा᳚ । स्वाहा᳚ऽस्थ॒न्वते᳚ । अ॒स्थ॒न्वते॒ स्वाहा᳚ । अ॒स्थ॒न्वत॒ इत्य॑स्थन्न् - वते᳚ । स्वाहा॑ऽन॒स्थिका॑य । अ॒न॒स्थिका॑य॒ स्वाहा᳚ । स्वाहा॑ मज्ज॒न्वते᳚ । म॒ज्ज॒न्वते॒ स्वाहा᳚ । म॒ज्ज॒न्वत॒ इति॑ मज्जन्न् - वते᳚ । स्वाहा॑ऽम॒ज्जका॑य । अ॒म॒ज्जका॑य॒ स्वाहा᳚ । स्वाहा॒ऽङ्‍गिने᳚ । अ॒ङ्‍गिने॒ स्वाहा᳚ । स्वाहा॑ऽन॒ङ्‍गाय॑ । अ॒न॒ङ्‍गाय॒ स्वाहा᳚ । स्वाहा॒ऽऽत्मने᳚ । आ॒त्मने॒ स्वाहा᳚ । स्वाहाऽना᳚त्मने । अना᳚त्मने॒ स्वाहा᳚ ( ) । स्वाहा॒ सर्व॑स्मै । सर्व॑स्मै॒ स्वाहा᳚ । स्वाहेति॒ स्वाहा᳚ । \newline

\textbf{Jatai Paata} \newline

1. अ॒म॒नसे॒ स्वाहा॒ स्वाहा॑ ऽम॒नसे॑ ऽम॒नसे॒ स्वाहा᳚ । \newline
2. स्वाहा॑ रेत॒स्विने॑ रेत॒स्विने॒ स्वाहा॒ स्वाहा॑ रेत॒स्विने᳚ । \newline
3. रे॒त॒स्विने॒ स्वाहा॒ स्वाहा॑ रेत॒स्विने॑ रेत॒स्विने॒ स्वाहा᳚ । \newline
4. स्वाहा॑ ऽरे॒तस्का॑या रे॒तस्का॑य॒ स्वाहा॒ स्वाहा॑ ऽरे॒तस्का॑य । \newline
5. अ॒रे॒तस्का॑य॒ स्वाहा॒ स्वाहा॑ ऽरे॒तस्का॑या रे॒तस्का॑य॒ स्वाहा᳚ । \newline
6. अ॒रे॒तस्का॒येत्य॑रे॒तः - का॒य॒ । \newline
7. स्वाहा᳚ प्र॒जाभ्यः॑ प्र॒जाभ्यः॒ स्वाहा॒ स्वाहा᳚ प्र॒जाभ्यः॑ । \newline
8. प्र॒जाभ्यः॒ स्वाहा॒ स्वाहा᳚ प्र॒जाभ्यः॑ प्र॒जाभ्यः॒ स्वाहा᳚ । \newline
9. प्र॒जाभ्य॒ इति॑ प्र - जाभ्यः॑ । \newline
10. स्वाहा᳚ प्र॒जन॑नाय प्र॒जन॑नाय॒ स्वाहा॒ स्वाहा᳚ प्र॒जन॑नाय । \newline
11. प्र॒जन॑नाय॒ स्वाहा॒ स्वाहा᳚ प्र॒जन॑नाय प्र॒जन॑नाय॒ स्वाहा᳚ । \newline
12. प्र॒जन॑ना॒येति॑ प्र - जन॑नाय । \newline
13. स्वाहा॒ लोम॑वते॒ लोम॑वते॒ स्वाहा॒ स्वाहा॒ लोम॑वते । \newline
14. लोम॑वते॒ स्वाहा॒ स्वाहा॒ लोम॑वते॒ लोम॑वते॒ स्वाहा᳚ । \newline
15. लोम॑वत॒ इति॒ लोम॑ - व॒ते॒ । \newline
16. स्वाहा॑ ऽलो॒मका॑या लो॒मका॑य॒ स्वाहा॒ स्वाहा॑ ऽलो॒मका॑य । \newline
17. अ॒लो॒मका॑य॒ स्वाहा॒ स्वाहा॑ ऽलो॒मका॑या लो॒मका॑य॒ स्वाहा᳚ । \newline
18. स्वाहा᳚ त्व॒चे त्व॒चे स्वाहा॒ स्वाहा᳚ त्व॒चे । \newline
19. त्व॒चे स्वाहा॒ स्वाहा᳚ त्व॒चे त्व॒चे स्वाहा᳚ । \newline
20. स्वाहा॒ ऽत्वक्का॑या॒ त्वक्का॑य॒ स्वाहा॒ स्वाहा॒ ऽत्वक्का॑य । \newline
21. अ॒त्वक्का॑य॒ स्वाहा॒ स्वाहा॒ ऽत्वक्का॑या॒ त्वक्का॑य॒ स्वाहा᳚ । \newline
22. स्वाहा॒ चर्म॑ण्वते॒ चर्म॑ण्वते॒ स्वाहा॒ स्वाहा॒ चर्म॑ण्वते । \newline
23. चर्म॑ण्वते॒ स्वाहा॒ स्वाहा॒ चर्म॑ण्वते॒ चर्म॑ण्वते॒ स्वाहा᳚ । \newline
24. चर्म॑ण्वत॒ इति॒ चर्मण्॑ - व॒ते॒ । \newline
25. स्वाहा॑ ऽच॒र्मका॑या च॒र्मका॑य॒ स्वाहा॒ स्वाहा॑ ऽच॒र्मका॑य । \newline
26. अ॒च॒र्मका॑य॒ स्वाहा॒ स्वाहा॑ ऽच॒र्मका॑या च॒र्मका॑य॒ स्वाहा᳚ । \newline
27. स्वाहा॒ लोहि॑तवते॒ लोहि॑तवते॒ स्वाहा॒ स्वाहा॒ लोहि॑तवते । \newline
28. लोहि॑तवते॒ स्वाहा॒ स्वाहा॒ लोहि॑तवते॒ लोहि॑तवते॒ स्वाहा᳚ । \newline
29. लोहि॑तवत॒ इति॒ लोहि॑त - व॒ते॒ । \newline
30. स्वाहा॑ ऽलोहि॒ताया॑ लोहि॒ताय॒ स्वाहा॒ स्वाहा॑ ऽलोहि॒ताय॑ । \newline
31. अ॒लो॒हि॒ताय॒ स्वाहा॒ स्वाहा॑ ऽलोहि॒ताया॑ लोहि॒ताय॒ स्वाहा᳚ । \newline
32. स्वाहा॑ माꣳस॒न्वते॑ माꣳस॒न्वते॒ स्वाहा॒ स्वाहा॑ माꣳस॒न्वते᳚ । \newline
33. माꣳ॒॒स॒न्वते॒ स्वाहा॒ स्वाहा॑ माꣳस॒न्वते॑ माꣳस॒न्वते॒ स्वाहा᳚ । \newline
34. माꣳ॒॒स॒न्वत॒ इति॑ माꣳसन्न् - वते᳚ । \newline
35. स्वाहा॑ ऽमाꣳ॒॒सका॑या माꣳ॒॒सका॑य॒ स्वाहा॒ स्वाहा॑ ऽमाꣳ॒॒सका॑य । \newline
36. अ॒माꣳ॒॒सका॑य॒ स्वाहा॒ स्वाहा॑ ऽमाꣳ॒॒सका॑या माꣳ॒॒सका॑य॒ स्वाहा᳚ । \newline
37. स्वाहा॒ स्नाव॑भ्यः॒ स्नाव॑भ्यः॒ स्वाहा॒ स्वाहा॒ स्नाव॑भ्यः । \newline
38. स्नाव॑भ्यः॒ स्वाहा॒ स्वाहा॒ स्नाव॑भ्यः॒ स्नाव॑भ्यः॒ स्वाहा᳚ । \newline
39. स्नाव॑भ्य॒ इति॒ स्नाव॑ - भ्यः॒ । \newline
40. स्वाहा᳚ ऽस्ना॒वका॑या स्ना॒वका॑य॒ स्वाहा॒ स्वाहा᳚ ऽस्ना॒वका॑य । \newline
41. अ॒स्ना॒वका॑य॒ स्वाहा॒ स्वाहा᳚ ऽस्ना॒वका॑या स्ना॒वका॑य॒ स्वाहा᳚ । \newline
42. स्वाहा᳚ ऽस्थ॒न्वते᳚ ऽस्थ॒न्वते॒ स्वाहा॒ स्वाहा᳚ ऽस्थ॒न्वते᳚ । \newline
43. अ॒स्थ॒न्वते॒ स्वाहा॒ स्वाहा᳚ ऽस्थ॒न्वते᳚ ऽस्थ॒न्वते॒ स्वाहा᳚ । \newline
44. अ॒स्थ॒न्वत॒ इत्य॑स्थन्न् - वते᳚ । \newline
45. स्वाहा॑ ऽन॒स्थिका॑या न॒स्थिका॑य॒ स्वाहा॒ स्वाहा॑ ऽन॒स्थिका॑य । \newline
46. अ॒न॒स्थिका॑य॒ स्वाहा॒ स्वाहा॑ ऽन॒स्थिका॑या न॒स्थिका॑य॒ स्वाहा᳚ । \newline
47. स्वाहा॑ मज्ज॒न्वते॑ मज्ज॒न्वते॒ स्वाहा॒ स्वाहा॑ मज्ज॒न्वते᳚ । \newline
48. म॒ज्ज॒न्वते॒ स्वाहा॒ स्वाहा॑ मज्ज॒न्वते॑ मज्ज॒न्वते॒ स्वाहा᳚ । \newline
49. म॒ज्ज॒न्वत॒ इति॑ मज्जन्न् - वते᳚ । \newline
50. स्वाहा॑ ऽम॒ज्जका॑या म॒ज्जका॑य॒ स्वाहा॒ स्वाहा॑ ऽम॒ज्जका॑य । \newline
51. अ॒म॒ज्जका॑य॒ स्वाहा॒ स्वाहा॑ ऽम॒ज्जका॑या म॒ज्जका॑य॒ स्वाहा᳚ । \newline
52. स्वाहा॒ ऽङ्गिने॒ ऽङ्गिने॒ स्वाहा॒ स्वाहा॒ ऽङ्गिने᳚ । \newline
53. अ॒ङ्गिने॒ स्वाहा॒ स्वाहा॒ ऽङ्गिने॒ ऽङ्गिने॒ स्वाहा᳚ । \newline
54. स्वाहा॑ ऽन॒ङ्गाया॑ न॒ङ्गाय॒ स्वाहा॒ स्वाहा॑ ऽन॒ङ्गाय॑ । \newline
55. अ॒न॒ङ्गाय॒ स्वाहा॒ स्वाहा॑ ऽन॒ङ्गाया॑ न॒ङ्गाय॒ स्वाहा᳚ । \newline
56. स्वाहा॒ ऽऽत्मन॑ आ॒त्मने॒ स्वाहा॒ स्वाहा॒ ऽऽत्मने᳚ । \newline
57. आ॒त्मने॒ स्वाहा॒ स्वाहा॒ ऽऽत्मन॑ आ॒त्मने॒ स्वाहा᳚ । \newline
58. स्वाहा ऽना᳚त्म॒ने ऽना᳚त्मने॒ स्वाहा॒ स्वाहा ऽना᳚त्मने । \newline
59. अना᳚त्मने॒ स्वाहा॒ स्वाहा ऽना᳚त्म॒ने ऽना᳚त्मने॒ स्वाहा᳚ । \newline
60. स्वाहा॒ सर्व॑स्मै॒ सर्व॑स्मै॒ स्वाहा॒ स्वाहा॒ सर्व॑स्मै । \newline
61. सर्व॑स्मै॒ स्वाहा॒ स्वाहा॒ सर्व॑स्मै॒ सर्व॑स्मै॒ स्वाहा᳚ । \newline
62. स्वाहेति॒ स्वाहा᳚ । \newline

\textbf{Ghana Paata } \newline

1. अ॒म॒नसे॒ स्वाहा॒ स्वाहा॑ ऽम॒नसे॑ ऽम॒नसे॒ स्वाहा॑ रेत॒स्विने॑ रेत॒स्विने॒ स्वाहा॑ ऽम॒नसे॑ ऽम॒नसे॒ स्वाहा॑ रेत॒स्विने᳚ । \newline
2. स्वाहा॑ रेत॒स्विने॑ रेत॒स्विने॒ स्वाहा॒ स्वाहा॑ रेत॒स्विने॒ स्वाहा॒ स्वाहा॑ रेत॒स्विने॒ स्वाहा॒ स्वाहा॑ रेत॒स्विने॒ स्वाहा᳚ । \newline
3. रे॒त॒स्विने॒ स्वाहा॒ स्वाहा॑ रेत॒स्विने॑ रेत॒स्विने॒ स्वाहा॑ ऽरे॒तस्का॑या रे॒तस्का॑य॒ स्वाहा॑ रेत॒स्विने॑ रेत॒स्विने॒ स्वाहा॑ ऽरे॒तस्का॑य । \newline
4. स्वाहा॑ ऽरे॒तस्का॑या रे॒तस्का॑य॒ स्वाहा॒ स्वाहा॑ ऽरे॒तस्का॑य॒ स्वाहा॒ स्वाहा॑ ऽरे॒तस्का॑य॒ स्वाहा॒ स्वाहा॑ ऽरे॒तस्का॑य॒ स्वाहा᳚ । \newline
5. अ॒रे॒तस्का॑य॒ स्वाहा॒ स्वाहा॑ ऽरे॒तस्का॑या रे॒तस्का॑य॒ स्वाहा᳚ प्र॒जाभ्यः॑ प्र॒जाभ्यः॒ स्वाहा॑ ऽरे॒तस्का॑या रे॒तस्का॑य॒ स्वाहा᳚ प्र॒जाभ्यः॑ । \newline
6. अ॒रे॒तस्का॒येत्य॑रे॒तः - का॒य॒ । \newline
7. स्वाहा᳚ प्र॒जाभ्यः॑ प्र॒जाभ्यः॒ स्वाहा॒ स्वाहा᳚ प्र॒जाभ्यः॒ स्वाहा॒ स्वाहा᳚ प्र॒जाभ्यः॒ स्वाहा॒ स्वाहा᳚ प्र॒जाभ्यः॒ स्वाहा᳚ । \newline
8. प्र॒जाभ्यः॒ स्वाहा॒ स्वाहा᳚ प्र॒जाभ्यः॑ प्र॒जाभ्यः॒ स्वाहा᳚ प्र॒जन॑नाय प्र॒जन॑नाय॒ स्वाहा᳚ प्र॒जाभ्यः॑ प्र॒जाभ्यः॒ स्वाहा᳚ प्र॒जन॑नाय । \newline
9. प्र॒जाभ्य॒ इति॑ प्र - जाभ्यः॑ । \newline
10. स्वाहा᳚ प्र॒जन॑नाय प्र॒जन॑नाय॒ स्वाहा॒ स्वाहा᳚ प्र॒जन॑नाय॒ स्वाहा॒ स्वाहा᳚ प्र॒जन॑नाय॒ स्वाहा॒ स्वाहा᳚ प्र॒जन॑नाय॒ स्वाहा᳚ । \newline
11. प्र॒जन॑नाय॒ स्वाहा॒ स्वाहा᳚ प्र॒जन॑नाय प्र॒जन॑नाय॒ स्वाहा॒ लोम॑वते॒ लोम॑वते॒ स्वाहा᳚ प्र॒जन॑नाय प्र॒जन॑नाय॒ स्वाहा॒ लोम॑वते । \newline
12. प्र॒जन॑ना॒येति॑ प्र - जन॑नाय । \newline
13. स्वाहा॒ लोम॑वते॒ लोम॑वते॒ स्वाहा॒ स्वाहा॒ लोम॑वते॒ स्वाहा॒ स्वाहा॒ लोम॑वते॒ स्वाहा॒ स्वाहा॒ लोम॑वते॒ स्वाहा᳚ । \newline
14. लोम॑वते॒ स्वाहा॒ स्वाहा॒ लोम॑वते॒ लोम॑वते॒ स्वाहा॑ ऽलो॒मका॑या लो॒मका॑य॒ स्वाहा॒ लोम॑वते॒ लोम॑वते॒ स्वाहा॑ ऽलो॒मका॑य । \newline
15. लोम॑वत॒ इति॒ लोम॑ - व॒ते॒ । \newline
16. स्वाहा॑ ऽलो॒मका॑या लो॒मका॑य॒ स्वाहा॒ स्वाहा॑ ऽलो॒मका॑य॒ स्वाहा॒ स्वाहा॑ ऽलो॒मका॑य॒ स्वाहा॒ स्वाहा॑ ऽलो॒मका॑य॒ स्वाहा᳚ । \newline
17. अ॒लो॒मका॑य॒ स्वाहा॒ स्वाहा॑ ऽलो॒मका॑या लो॒मका॑य॒ स्वाहा᳚ त्व॒चे त्व॒चे स्वाहा॑ ऽलो॒मका॑या लो॒मका॑य॒ स्वाहा᳚ त्व॒चे । \newline
18. स्वाहा᳚ त्व॒चे त्व॒चे स्वाहा॒ स्वाहा᳚ त्व॒चे स्वाहा॒ स्वाहा᳚ त्व॒चे स्वाहा॒ स्वाहा᳚ त्व॒चे स्वाहा᳚ । \newline
19. त्व॒चे स्वाहा॒ स्वाहा᳚ त्व॒चे त्व॒चे स्वाहा॒ ऽत्वक्का॑या॒ त्वक्का॑य॒ स्वाहा᳚ त्व॒चे त्व॒चे स्वाहा॒ ऽत्वक्का॑य । \newline
20. स्वाहा॒ ऽत्वक्का॑या॒ त्वक्का॑य॒ स्वाहा॒ स्वाहा॒ ऽत्वक्का॑य॒ स्वाहा॒ स्वाहा॒ ऽत्वक्का॑य॒ स्वाहा॒ स्वाहा॒ ऽत्वक्का॑य॒ स्वाहा᳚ । \newline
21. अ॒त्वक्का॑य॒ स्वाहा॒ स्वाहा॒ ऽत्वक्का॑या॒ त्वक्का॑य॒ स्वाहा॒ चर्म॑ण्वते॒ चर्म॑ण्वते॒ स्वाहा॒ ऽत्वक्का॑या॒ त्वक्का॑य॒ स्वाहा॒ चर्म॑ण्वते । \newline
22. स्वाहा॒ चर्म॑ण्वते॒ चर्म॑ण्वते॒ स्वाहा॒ स्वाहा॒ चर्म॑ण्वते॒ स्वाहा॒ स्वाहा॒ चर्म॑ण्वते॒ स्वाहा॒ स्वाहा॒ चर्म॑ण्वते॒ स्वाहा᳚ । \newline
23. चर्म॑ण्वते॒ स्वाहा॒ स्वाहा॒ चर्म॑ण्वते॒ चर्म॑ण्वते॒ स्वाहा॑ ऽच॒र्मका॑या च॒र्मका॑य॒ स्वाहा॒ चर्म॑ण्वते॒ चर्म॑ण्वते॒ स्वाहा॑ ऽच॒र्मका॑य । \newline
24. चर्म॑ण्वत॒ इति॒ चर्मण्॑ - व॒ते॒ । \newline
25. स्वाहा॑ ऽच॒र्मका॑या च॒र्मका॑य॒ स्वाहा॒ स्वाहा॑ ऽच॒र्मका॑य॒ स्वाहा॒ स्वाहा॑ ऽच॒र्मका॑य॒ स्वाहा॒ स्वाहा॑ ऽच॒र्मका॑य॒ स्वाहा᳚ । \newline
26. अ॒च॒र्मका॑य॒ स्वाहा॒ स्वाहा॑ ऽच॒र्मका॑या च॒र्मका॑य॒ स्वाहा॒ लोहि॑तवते॒ लोहि॑तवते॒ स्वाहा॑ ऽच॒र्मका॑या च॒र्मका॑य॒ स्वाहा॒ लोहि॑तवते । \newline
27. स्वाहा॒ लोहि॑तवते॒ लोहि॑तवते॒ स्वाहा॒ स्वाहा॒ लोहि॑तवते॒ स्वाहा॒ स्वाहा॒ लोहि॑तवते॒ स्वाहा॒ स्वाहा॒ लोहि॑तवते॒ स्वाहा᳚ । \newline
28. लोहि॑तवते॒ स्वाहा॒ स्वाहा॒ लोहि॑तवते॒ लोहि॑तवते॒ स्वाहा॑ ऽलोहि॒ताया॑ लोहि॒ताय॒ स्वाहा॒ लोहि॑तवते॒ लोहि॑तवते॒ स्वाहा॑ ऽलोहि॒ताय॑ । \newline
29. लोहि॑तवत॒ इति॒ लोहि॑त - व॒ते॒ । \newline
30. स्वाहा॑ ऽलोहि॒ताया॑ लोहि॒ताय॒ स्वाहा॒ स्वाहा॑ ऽलोहि॒ताय॒ स्वाहा॒ स्वाहा॑ ऽलोहि॒ताय॒ स्वाहा॒ स्वाहा॑ ऽलोहि॒ताय॒ स्वाहा᳚ । \newline
31. अ॒लो॒हि॒ताय॒ स्वाहा॒ स्वाहा॑ ऽलोहि॒ताया॑ लोहि॒ताय॒ स्वाहा॑ माꣳस॒न्वते॑ माꣳस॒न्वते॒ स्वाहा॑ ऽलोहि॒ताया॑ लोहि॒ताय॒ स्वाहा॑ माꣳस॒न्वते᳚ । \newline
32. स्वाहा॑ माꣳस॒न्वते॑ माꣳस॒न्वते॒ स्वाहा॒ स्वाहा॑ माꣳस॒न्वते॒ स्वाहा॒ स्वाहा॑ माꣳस॒न्वते॒ स्वाहा॒ स्वाहा॑ माꣳस॒न्वते॒ स्वाहा᳚ । \newline
33. माꣳ॒॒स॒न्वते॒ स्वाहा॒ स्वाहा॑ माꣳस॒न्वते॑ माꣳस॒न्वते॒ स्वाहा॑ ऽमाꣳ॒॒सका॑या माꣳ॒॒सका॑य॒ स्वाहा॑ माꣳस॒न्वते॑ माꣳस॒न्वते॒ स्वाहा॑ ऽमाꣳ॒॒सका॑य । \newline
34. माꣳ॒॒स॒न्वत॒ इति॑ माꣳसन्न् - वते᳚ । \newline
35. स्वाहा॑ ऽमाꣳ॒॒सका॑या माꣳ॒॒सका॑य॒ स्वाहा॒ स्वाहा॑ ऽमाꣳ॒॒सका॑य॒ स्वाहा॒ स्वाहा॑ ऽमाꣳ॒॒सका॑य॒ स्वाहा॒ स्वाहा॑ ऽमाꣳ॒॒सका॑य॒ स्वाहा᳚ । \newline
36. अ॒माꣳ॒॒सका॑य॒ स्वाहा॒ स्वाहा॑ ऽमाꣳ॒॒सका॑या माꣳ॒॒सका॑य॒ स्वाहा॒ स्नाव॑भ्यः॒ स्नाव॑भ्यः॒ स्वाहा॑ ऽमाꣳ॒॒सका॑या माꣳ॒॒सका॑य॒ स्वाहा॒ स्नाव॑भ्यः । \newline
37. स्वाहा॒ स्नाव॑भ्यः॒ स्नाव॑भ्यः॒ स्वाहा॒ स्वाहा॒ स्नाव॑भ्यः॒ स्वाहा॒ स्वाहा॒ स्नाव॑भ्यः॒ स्वाहा॒ स्वाहा॒ स्नाव॑भ्यः॒ स्वाहा᳚ । \newline
38. स्नाव॑भ्यः॒ स्वाहा॒ स्वाहा॒ स्नाव॑भ्यः॒ स्नाव॑भ्यः॒ स्वाहा᳚ ऽस्ना॒वका॑या स्ना॒वका॑य॒ स्वाहा॒ स्नाव॑भ्यः॒ स्नाव॑भ्यः॒ स्वाहा᳚ ऽस्ना॒वका॑य । \newline
39. स्नाव॑भ्य॒ इति॒ स्नाव॑ - भ्यः॒ । \newline
40. स्वाहा᳚ ऽस्ना॒वका॑या स्ना॒वका॑य॒ स्वाहा॒ स्वाहा᳚ ऽस्ना॒वका॑य॒ स्वाहा॒ स्वाहा᳚ ऽस्ना॒वका॑य॒ स्वाहा॒ स्वाहा᳚ ऽस्ना॒वका॑य॒ स्वाहा᳚ । \newline
41. अ॒स्ना॒वका॑य॒ स्वाहा॒ स्वाहा᳚ ऽस्ना॒वका॑या स्ना॒वका॑य॒ स्वाहा᳚ ऽस्थ॒न्वते᳚ ऽस्थ॒न्वते॒ स्वाहा᳚ ऽस्ना॒वका॑या स्ना॒वका॑य॒ स्वाहा᳚ ऽस्थ॒न्वते᳚ । \newline
42. स्वाहा᳚ ऽस्थ॒न्वते᳚ ऽस्थ॒न्वते॒ स्वाहा॒ स्वाहा᳚ ऽस्थ॒न्वते॒ स्वाहा॒ स्वाहा᳚ ऽस्थ॒न्वते॒ स्वाहा॒ स्वाहा᳚ ऽस्थ॒न्वते॒ स्वाहा᳚ । \newline
43. अ॒स्थ॒न्वते॒ स्वाहा॒ स्वाहा᳚ ऽस्थ॒न्वते᳚ ऽस्थ॒न्वते॒ स्वाहा॑ ऽन॒स्थिका॑या न॒स्थिका॑य॒ स्वाहा᳚ ऽस्थ॒न्वते᳚ ऽस्थ॒न्वते॒ स्वाहा॑ ऽन॒स्थिका॑य । \newline
44. अ॒स्थ॒न्वत॒ इत्य॑स्थन्न् - वते᳚ । \newline
45. स्वाहा॑ ऽन॒स्थिका॑या न॒स्थिका॑य॒ स्वाहा॒ स्वाहा॑ ऽन॒स्थिका॑य॒ स्वाहा॒ स्वाहा॑ ऽन॒स्थिका॑य॒ स्वाहा॒ स्वाहा॑ ऽन॒स्थिका॑य॒ स्वाहा᳚ । \newline
46. अ॒न॒स्थिका॑य॒ स्वाहा॒ स्वाहा॑ ऽन॒स्थिका॑या न॒स्थिका॑य॒ स्वाहा॑ मज्ज॒न्वते॑ मज्ज॒न्वते॒ स्वाहा॑ ऽन॒स्थिका॑या न॒स्थिका॑य॒ स्वाहा॑ मज्ज॒न्वते᳚ । \newline
47. स्वाहा॑ मज्ज॒न्वते॑ मज्ज॒न्वते॒ स्वाहा॒ स्वाहा॑ मज्ज॒न्वते॒ स्वाहा॒ स्वाहा॑ मज्ज॒न्वते॒ स्वाहा॒ स्वाहा॑ मज्ज॒न्वते॒ स्वाहा᳚ । \newline
48. म॒ज्ज॒न्वते॒ स्वाहा॒ स्वाहा॑ मज्ज॒न्वते॑ मज्ज॒न्वते॒ स्वाहा॑ ऽम॒ज्जका॑या म॒ज्जका॑य॒ स्वाहा॑ मज्ज॒न्वते॑ मज्ज॒न्वते॒ स्वाहा॑ ऽम॒ज्जका॑य । \newline
49. म॒ज्ज॒न्वत॒ इति॑ मज्जन्न् - वते᳚ । \newline
50. स्वाहा॑ ऽम॒ज्जका॑या म॒ज्जका॑य॒ स्वाहा॒ स्वाहा॑ ऽम॒ज्जका॑य॒ स्वाहा॒ स्वाहा॑ ऽम॒ज्जका॑य॒ स्वाहा॒ स्वाहा॑ ऽम॒ज्जका॑य॒ स्वाहा᳚ । \newline
51. अ॒म॒ज्जका॑य॒ स्वाहा॒ स्वाहा॑ ऽम॒ज्जका॑या म॒ज्जका॑य॒ स्वाहा॒ ऽङ्गिने॒ ऽङ्गिने॒ स्वाहा॑ ऽम॒ज्जका॑या म॒ज्जका॑य॒ स्वाहा॒ ऽङ्गिने᳚ । \newline
52. स्वाहा॒ ऽङ्गिने॒ ऽङ्गिने॒ स्वाहा॒ स्वाहा॒ ऽङ्गिने॒ स्वाहा॒ स्वाहा॒ ऽङ्गिने॒ स्वाहा॒ स्वाहा॒ ऽङ्गिने॒ स्वाहा᳚ । \newline
53. अ॒ङ्गिने॒ स्वाहा॒ स्वाहा॒ ऽङ्गिने॒ ऽङ्गिने॒ स्वाहा॑ ऽन॒ङ्गाया॑ न॒ङ्गाय॒ स्वाहा॒ ऽङ्गिने॒ ऽङ्गिने॒ स्वाहा॑ ऽन॒ङ्गाय॑ । \newline
54. स्वाहा॑ ऽन॒ङ्गाया॑ न॒ङ्गाय॒ स्वाहा॒ स्वाहा॑ ऽन॒ङ्गाय॒ स्वाहा॒ स्वाहा॑ ऽन॒ङ्गाय॒ स्वाहा॒ स्वाहा॑ ऽन॒ङ्गाय॒ स्वाहा᳚ । \newline
55. अ॒न॒ङ्गाय॒ स्वाहा॒ स्वाहा॑ ऽन॒ङ्गाया॑ न॒ङ्गाय॒ स्वाहा॒ ऽऽत्मन॑ आ॒त्मने॒ स्वाहा॑ ऽन॒ङ्गाया॑ न॒ङ्गाय॒ स्वाहा॒ ऽऽत्मने᳚ । \newline
56. स्वाहा॒ ऽऽत्मन॑ आ॒त्मने॒ स्वाहा॒ स्वाहा॒ ऽऽत्मने॒ स्वाहा॒ स्वाहा॒ ऽऽत्मने॒ स्वाहा॒ स्वाहा॒ ऽऽत्मने॒ स्वाहा᳚ । \newline
57. आ॒त्मने॒ स्वाहा॒ स्वाहा॒ ऽऽत्मन॑ आ॒त्मने॒ स्वाहा ऽना᳚त्म॒ने ऽना᳚त्मने॒ स्वाहा॒ ऽऽत्मन॑ आ॒त्मने॒ स्वाहा ऽना᳚त्मने । \newline
58. स्वाहा ऽना᳚त्म॒ने ऽना᳚त्मने॒ स्वाहा॒ स्वाहा ऽना᳚त्मने॒ स्वाहा॒ स्वाहा ऽना᳚त्मने॒ स्वाहा॒ स्वाहा ऽना᳚त्मने॒ स्वाहा᳚ । \newline
59. अना᳚त्मने॒ स्वाहा॒ स्वाहा ऽना᳚त्म॒ने ऽना᳚त्मने॒ स्वाहा॒ सर्व॑स्मै॒ सर्व॑स्मै॒ स्वाहा ऽना᳚त्म॒ने ऽना᳚त्मने॒ स्वाहा॒ सर्व॑स्मै । \newline
60. स्वाहा॒ सर्व॑स्मै॒ सर्व॑स्मै॒ स्वाहा॒ स्वाहा॒ सर्व॑स्मै॒ स्वाहा॒ स्वाहा॒ सर्व॑स्मै॒ स्वाहा॒ स्वाहा॒ सर्व॑स्मै॒ स्वाहा᳚ । \newline
61. सर्व॑स्मै॒ स्वाहा॒ स्वाहा॒ सर्व॑स्मै॒ सर्व॑स्मै॒ स्वाहा᳚ । \newline
62. स्वाहेति॒ स्वाहा᳚ । \newline
\pagebreak
\markright{ TS 7.5.13.1  \hfill https://www.vedavms.in \hfill}

\section{ TS 7.5.13.1 }

\textbf{TS 7.5.13.1 } \newline
\textbf{Samhita Paata} \newline

कस्त्वा॑ युनक्ति॒ स त्वा॑ युनक्तु॒ विष्णु॑स्त्वा युनक्त्व॒स्य य॒ज्ञ्स्यर्द्ध्यै॒ मह्यꣳ॒॒ सन्न॑त्या अ॒मुष्मै॒ कामा॒याऽऽ*यु॑षे त्वा प्रा॒णाय॑ त्वा ऽपा॒नाय॑ त्वा व्या॒नाय॑ त्वा॒ व्यु॑ष्ट्यै त्वा र॒य्यै त्वा॒ राध॑से त्वा॒ घोषा॑य त्वा॒ पोषा॑य त्वा ऽऽराद्घो॒षाय॑ त्वा॒ प्रच्यु॑त्यै त्वा ॥ \newline

\textbf{Pada Paata} \newline

कः । त्वा॒ । यु॒न॒क्ति॒ । सः । त्वा॒ । यु॒न॒क्तु॒ । विष्णुः॑ । त्वा॒ । यु॒न॒क्तु॒ । अ॒स्य । य॒ज्ञ्स्य॑ । ऋद्ध्यै᳚ । मह्य᳚म् । सन्न॑त्या॒ इति॒ सं - न॒त्यै॒ । अ॒मुष्मै᳚ । कामा॑य । आयु॑षे । त्वा॒ । प्रा॒णायेति॑ प्र - अ॒नाय॑ । त्वा॒ । अ॒पा॒नायेत्य॑प - अ॒नाय॑ । त्वा॒ । व्या॒नायेति॑ वि - अ॒नाय॑ । त्वा॒ । व्यु॑ष्ट्या॒ इति॒ वि - उ॒ष्ट्यै॒ । त्वा॒ । र॒य्यै । त्वा॒ । राध॑से । त्वा॒ । घोषा॑य । त्वा॒ । पोषा॑य । त्वा॒ । आ॒रा॒द्घो॒षायेत्या॑रात् - घो॒षाय॑ । त्वा॒ । प्रच्यु॑त्या॒ इति॒ प्र - च्यु॒त्यै॒ । त्वा॒ ॥  \newline


\textbf{Krama Paata} \newline

कस्त्वा᳚ । त्वा॒ यु॒न॒क्ति॒ । यु॒न॒क्ति॒ सः । स त्वा᳚ । त्वा॒ यु॒न॒क्तु॒ । यु॒न॒क्तु॒ विष्णुः॑ । विष्णु॑स्त्वा । त्वा॒ यु॒न॒क्तु॒ । यु॒न॒क्त्व॒स्य । अ॒स्य य॒ज्ञ्स्य॑ । य॒ज्ञ्स्यर्द्ध्यै᳚ । ऋद्ध्यै॒ मह्य᳚म् । मह्यꣳ॒॒ सन्न॑त्यै । सन्न॑त्या अ॒मुष्मै᳚ । सन्न॑त्या॒ इति॒ सम् - न॒त्यै॒ । अ॒मुष्मै॒ कामा॑य । कामा॒यायु॑षे । आयु॑षे त्वा । त्वा॒ प्रा॒णाय॑ । प्रा॒णाय॑ त्वा । प्रा॒णायेति॑ प्र - अ॒नाय॑ । त्वा॒ऽपा॒नाय॑ । अ॒पा॒नाय॑ त्वा । अ॒पा॒नायेत्य॑प - अ॒नाय॑ । त्वा॒ व्या॒नाय॑ । व्या॒नाय॑ त्वा । व्या॒नायेति॑ वि - अ॒नाय॑ । त्वा॒ व्यु॑ष्ट्‍यै । व्यु॑ष्ट्‍यै त्वा । व्यु॑ष्ट्‍या॒ इति॒ वि - उ॒ष्ट्‍यै॒ । त्वा॒ र॒य्यै । र॒य्यै त्वा᳚ । त्वा॒ राध॑से । राध॑से त्वा । त्वा॒ घोषा॑य । घोषा॑य त्वा । त्वा॒ पोषा॑य । पोषा॑य त्वा । त्वा॒ऽऽरा॒द्‍घो॒षाय॑ । आ॒रा॒द्‍घो॒षाय॑ त्वा । आ॒रा॒द्‍घो॒षायेत्या॑रात् - घो॒षाय॑ । त्वा॒ प्रच्यु॑त्यै । प्रच्यु॑त्यै त्वा । प्रच्यु॑त्या॒ इति॒ प्र - च्यु॒त्यै॒ । त्वेति॑ त्वा । \newline

\textbf{Jatai Paata} \newline

1. कस्त्वा᳚ त्वा॒ कः कस्त्वा᳚ । \newline
2. त्वा॒ यु॒न॒क्ति॒ यु॒न॒क्ति॒ त्वा॒ त्वा॒ यु॒न॒क्ति॒ । \newline
3. यु॒न॒क्ति॒ स स यु॑नक्ति युनक्ति॒ सः । \newline
4. स त्वा᳚ त्वा॒ स स त्वा᳚ । \newline
5. त्वा॒ यु॒न॒क्तु॒ यु॒न॒क्तु॒ त्वा॒ त्वा॒ यु॒न॒क्तु॒ । \newline
6. यु॒न॒क्तु॒ विष्णु॒र् विष्णु॑र् युनक्तु युनक्तु॒ विष्णुः॑ । \newline
7. विष्णु॑ स्त्वा त्वा॒ विष्णु॒र् विष्णु॑ स्त्वा । \newline
8. त्वा॒ यु॒न॒क्तु॒ यु॒न॒क्तु॒ त्वा॒ त्वा॒ यु॒न॒क्तु॒ । \newline
9. यु॒न॒क् त्व॒स्यास्य यु॑नक्तु युनक् त्व॒स्य । \newline
10. अ॒स्य य॒ज्ञ्स्य॑ य॒ज्ञ्स्या॒ स्यास्य य॒ज्ञ्स्य॑ । \newline
11. य॒ज्ञ्स्य र्‌द्ध्या॒ ऋद्ध्यै॑ य॒ज्ञ्स्य॑ य॒ज्ञ्स्य र्‌द्ध्यै᳚ । \newline
12. ऋद्ध्यै॒ मह्य॒म् मह्य॒ मृद्ध्या॒ ऋद्ध्यै॒ मह्य᳚म् । \newline
13. मह्यꣳ॒॒ सन्न॑त्यै॒ सन्न॑त्यै॒ मह्य॒म् मह्यꣳ॒॒ सन्न॑त्यै । \newline
14. सन्न॑त्या अ॒मुष्मा॑ अ॒मुष्मै॒ सन्न॑त्यै॒ सन्न॑त्या अ॒मुष्मै᳚ । \newline
15. सन्न॑त्या॒ इति॒ सं - न॒त्यै॒ । \newline
16. अ॒मुष्मै॒ कामा॑य॒ कामा॑या॒ मुष्मा॑ अ॒मुष्मै॒ कामा॑य । \newline
17. कामा॒या यु॑ष॒ आयु॑षे॒ कामा॑य॒ कामा॒या यु॑षे । \newline
18. आयु॑षे त्वा॒ त्वा ऽऽयु॑ष॒ आयु॑षे त्वा । \newline
19. त्वा॒ प्रा॒णाय॑ प्रा॒णाय॑ त्वा त्वा प्रा॒णाय॑ । \newline
20. प्रा॒णाय॑ त्वा त्वा प्रा॒णाय॑ प्रा॒णाय॑ त्वा । \newline
21. प्रा॒णायेति॑ प्र - अ॒नाय॑ । \newline
22. त्वा॒ ऽपा॒नाया॑ पा॒नाय॑ त्वा त्वा ऽपा॒नाय॑ । \newline
23. अ॒पा॒नाय॑ त्वा त्वा ऽपा॒नाया॑ पा॒नाय॑ त्वा । \newline
24. अ॒पा॒नायेत्य॑प - अ॒नाय॑ । \newline
25. त्वा॒ व्या॒नाय॑ व्या॒नाय॑ त्वा त्वा व्या॒नाय॑ । \newline
26. व्या॒नाय॑ त्वा त्वा व्या॒नाय॑ व्या॒नाय॑ त्वा । \newline
27. व्या॒नायेति॑ वि - अ॒नाय॑ । \newline
28. त्वा॒ व्यु॑ष्ट्यै॒ व्यु॑ष्ट्यै त्वा त्वा॒ व्यु॑ष्ट्यै । \newline
29. व्यु॑ष्ट्यै त्वा त्वा॒ व्यु॑ष्ट्यै॒ व्यु॑ष्ट्यै त्वा । \newline
30. व्यु॑ष्ट्या॒ इति॒ वि - उ॒ष्ट्यै॒ । \newline
31. त्वा॒ र॒य्यै र॒य्यै त्वा᳚ त्वा र॒य्यै । \newline
32. र॒य्यै त्वा᳚ त्वा र॒य्यै र॒य्यै त्वा᳚ । \newline
33. त्वा॒ राध॑से॒ राध॑से त्वा त्वा॒ राध॑से । \newline
34. राध॑से त्वा त्वा॒ राध॑से॒ राध॑से त्वा । \newline
35. त्वा॒ घोषा॑य॒ घोषा॑य त्वा त्वा॒ घोषा॑य । \newline
36. घोषा॑य त्वा त्वा॒ घोषा॑य॒ घोषा॑य त्वा । \newline
37. त्वा॒ पोषा॑य॒ पोषा॑य त्वा त्वा॒ पोषा॑य । \newline
38. पोषा॑य त्वा त्वा॒ पोषा॑य॒ पोषा॑य त्वा । \newline
39. त्वा॒ ऽऽरा॒द्घ्‍ओ॒षाया॑ राद्घ्‍ओ॒षाय॑ त्वा त्वा ऽऽराद्घ्‍ओ॒षाय॑ । \newline
40. आ॒रा॒द्‍घो॒षाय॑ त्वा त्वा ऽऽराद्‍घो॒षाया॑ राद्‍घो॒षाय॑ त्वा । \newline
41. आ॒रा॒द्‍घो॒षायेत्या॑रात् - घो॒षाय॑ । \newline
42. त्वा॒ प्रच्यु॑त्यै॒ प्रच्यु॑त्यै त्वा त्वा॒ प्रच्यु॑त्यै । \newline
43. प्रच्यु॑त्यै त्वा त्वा॒ प्रच्यु॑त्यै॒ प्रच्यु॑त्यै त्वा । \newline
44. प्रच्यु॑त्या॒ इति॒ प्र - च्यु॒त्यै॒ । \newline
45. त्वेति॑ त्वा । \newline

\textbf{Ghana Paata } \newline

1. क स्त्वा᳚ त्वा॒ कः क स्त्वा॑ युनक्ति युनक्ति त्वा॒ कः क स्त्वा॑ युनक्ति । \newline
2. त्वा॒ यु॒न॒क्ति॒ यु॒न॒क्ति॒ त्वा॒ त्वा॒ यु॒न॒क्ति॒ स स यु॑नक्ति त्वा त्वा युनक्ति॒ सः । \newline
3. यु॒न॒क्ति॒ स स यु॑नक्ति युनक्ति॒ स त्वा᳚ त्वा॒ स यु॑नक्ति युनक्ति॒ स त्वा᳚ । \newline
4. स त्वा᳚ त्वा॒ स स त्वा॑ युनक्तु युनक्तु त्वा॒ स स त्वा॑ युनक्तु । \newline
5. त्वा॒ यु॒न॒क्तु॒ यु॒न॒क्तु॒ त्वा॒ त्वा॒ यु॒न॒क्तु॒ विष्णु॒र् विष्णु॑र् युनक्तु त्वा त्वा युनक्तु॒ विष्णुः॑ । \newline
6. यु॒न॒क्तु॒ विष्णु॒र् विष्णु॑र् युनक्तु युनक्तु॒ विष्णु॑ स्त्वा त्वा॒ विष्णु॑र् युनक्तु युनक्तु॒ विष्णु॑ स्त्वा । \newline
7. विष्णु॑ स्त्वा त्वा॒ विष्णु॒र् विष्णु॑ स्त्वा युनक्तु युनक्तु त्वा॒ विष्णु॒र् विष्णु॑ स्त्वा युनक्तु । \newline
8. त्वा॒ यु॒न॒क्तु॒ यु॒न॒क्तु॒ त्वा॒ त्वा॒ यु॒न॒क् त्व॒स्यास्य यु॑नक्तु त्वा त्वा युनक् त्व॒स्य । \newline
9. यु॒न॒क् त्व॒स्यास्य यु॑नक्तु युनक् त्व॒स्य य॒ज्ञ्स्य॑ य॒ज्ञ् स्या॒स्य यु॑नक्तु युनक् त्व॒स्य य॒ज्ञ्स्य॑ । \newline
10. अ॒स्य य॒ज्ञ्स्य॑ य॒ज्ञ्स्या॒ स्यास्य य॒ज्ञ्स्य र्‌द्ध्या॒ ऋद्ध्यै॑ य॒ज्ञ्स्या॒ स्यास्य य॒ज्ञ्स्य र्‌द्ध्यै᳚ । \newline
11. य॒ज्ञ्स्य र्‌द्ध्या॒ ऋद्ध्यै॑ य॒ज्ञ्स्य॑ य॒ज्ञ्स्य र्‌द्ध्यै॒ मह्य॒म् मह्य॒ मृद्ध्यै॑ य॒ज्ञ्स्य॑ य॒ज्ञ्स्य र्‌द्ध्यै॒ मह्य᳚म् । \newline
12. ऋद्ध्यै॒ मह्य॒म् मह्य॒ मृद्ध्या॒ ऋद्ध्यै॒ मह्यꣳ॒॒ सन्न॑त्यै॒ सन्न॑त्यै॒ मह्य॒ मृद्ध्या॒ ऋद्ध्यै॒ मह्यꣳ॒॒ सन्न॑त्यै । \newline
13. मह्यꣳ॒॒ सन्न॑त्यै॒ सन्न॑त्यै॒ मह्य॒म् मह्यꣳ॒॒ सन्न॑त्या अ॒मुष्मा॑ अ॒मुष्मै॒ सन्न॑त्यै॒ मह्य॒म् मह्यꣳ॒॒ सन्न॑त्या अ॒मुष्मै᳚ । \newline
14. सन्न॑त्या अ॒मुष्मा॑ अ॒मुष्मै॒ सन्न॑त्यै॒ सन्न॑त्या अ॒मुष्मै॒ कामा॑य॒ कामा॑या॒ मुष्मै॒ सन्न॑त्यै॒ सन्न॑त्या अ॒मुष्मै॒ कामा॑य । \newline
15. सन्न॑त्या॒ इति॒ सं - न॒त्यै॒ । \newline
16. अ॒मुष्मै॒ कामा॑य॒ कामा॑या॒ मुष्मा॑ अ॒मुष्मै॒ कामा॒या यु॑ष॒ आयु॑षे॒ कामा॑या॒ मुष्मा॑ अ॒मुष्मै॒ कामा॒या यु॑षे । \newline
17. कामा॒या यु॑ष॒ आयु॑षे॒ कामा॑य॒ कामा॒या यु॑षे त्वा॒ त्वा ऽऽयु॑षे॒ कामा॑य॒ कामा॒या यु॑षे त्वा । \newline
18. आयु॑षे त्वा॒ त्वा ऽऽयु॑ष॒ आयु॑षे त्वा प्रा॒णाय॑ प्रा॒णाय॒ त्वा ऽऽयु॑ष॒ आयु॑षे त्वा प्रा॒णाय॑ । \newline
19. त्वा॒ प्रा॒णाय॑ प्रा॒णाय॑ त्वा त्वा प्रा॒णाय॑ त्वा त्वा प्रा॒णाय॑ त्वा त्वा प्रा॒णाय॑ त्वा । \newline
20. प्रा॒णाय॑ त्वा त्वा प्रा॒णाय॑ प्रा॒णाय॑ त्वा ऽपा॒नाया॑ पा॒नाय॑ त्वा प्रा॒णाय॑ प्रा॒णाय॑ त्वा ऽपा॒नाय॑ । \newline
21. प्रा॒णायेति॑ प्र - अ॒नाय॑ । \newline
22. त्वा॒ ऽपा॒नाया॑ पा॒नाय॑ त्वा त्वा ऽपा॒नाय॑ त्वा त्वा ऽपा॒नाय॑ त्वा त्वा ऽपा॒नाय॑ त्वा । \newline
23. अ॒पा॒नाय॑ त्वा त्वा ऽपा॒नाया॑ पा॒नाय॑ त्वा व्या॒नाय॑ व्या॒नाय॑ त्वा ऽपा॒नाया॑ पा॒नाय॑ त्वा व्या॒नाय॑ । \newline
24. अ॒पा॒नायेत्य॑प - अ॒नाय॑ । \newline
25. त्वा॒ व्या॒नाय॑ व्या॒नाय॑ त्वा त्वा व्या॒नाय॑ त्वा त्वा व्या॒नाय॑ त्वा त्वा व्या॒नाय॑ त्वा । \newline
26. व्या॒नाय॑ त्वा त्वा व्या॒नाय॑ व्या॒नाय॑ त्वा॒ व्यु॑ष्ट्यै॒ व्यु॑ष्ट्यै त्वा व्या॒नाय॑ व्या॒नाय॑ त्वा॒ व्यु॑ष्ट्यै । \newline
27. व्या॒नायेति॑ वि - अ॒नाय॑ । \newline
28. त्वा॒ व्यु॑ष्ट्यै॒ व्यु॑ष्ट्यै त्वा त्वा॒ व्यु॑ष्ट्यै त्वा त्वा॒ व्यु॑ष्ट्यै त्वा त्वा॒ व्यु॑ष्ट्यै त्वा । \newline
29. व्यु॑ष्ट्यै त्वा त्वा॒ व्यु॑ष्ट्यै॒ व्यु॑ष्ट्यै त्वा र॒य्यै र॒य्यै त्वा॒ व्यु॑ष्ट्यै॒ व्यु॑ष्ट्यै त्वा र॒य्यै । \newline
30. व्यु॑ष्ट्या॒ इति॒ वि - उ॒ष्ट्यै॒ । \newline
31. त्वा॒ र॒य्यै र॒य्यै त्वा᳚ त्वा र॒य्यै त्वा᳚ त्वा र॒य्यै त्वा᳚ त्वा र॒य्यै त्वा᳚ । \newline
32. र॒य्यै त्वा᳚ त्वा र॒य्यै र॒य्यै त्वा॒ राध॑से॒ राध॑से त्वा र॒य्यै र॒य्यै त्वा॒ राध॑से । \newline
33. त्वा॒ राध॑से॒ राध॑से त्वा त्वा॒ राध॑से त्वा त्वा॒ राध॑से त्वा त्वा॒ राध॑से त्वा । \newline
34. राध॑से त्वा त्वा॒ राध॑से॒ राध॑से त्वा॒ घोषा॑य॒ घोषा॑य त्वा॒ राध॑से॒ राध॑से त्वा॒ घोषा॑य । \newline
35. त्वा॒ घोषा॑य॒ घोषा॑य त्वा त्वा॒ घोषा॑य त्वा त्वा॒ घोषा॑य त्वा त्वा॒ घोषा॑य त्वा । \newline
36. घोषा॑य त्वा त्वा॒ घोषा॑य॒ घोषा॑य त्वा॒ पोषा॑य॒ पोषा॑य त्वा॒ घोषा॑य॒ घोषा॑य त्वा॒ पोषा॑य । \newline
37. त्वा॒ पोषा॑य॒ पोषा॑य त्वा त्वा॒ पोषा॑य त्वा त्वा॒ पोषा॑य त्वा त्वा॒ पोषा॑य त्वा । \newline
38. पोषा॑य त्वा त्वा॒ पोषा॑य॒ पोषा॑य त्वा ऽऽराद्‍घो॒षाया॑ राद्‍घो॒षाय॑ त्वा॒ पोषा॑य॒ पोषा॑य त्वा ऽऽराद्‍घो॒षाय॑ । \newline
39. त्वा॒ ऽऽरा॒द्‍घो॒षाया॑ राद्‍घो॒षाय॑ त्वा त्वा ऽऽराद्‍घो॒षाय॑ त्वा त्वा ऽऽराद्‍घो॒षाय॑ त्वा त्वा ऽऽराद्‍घो॒षाय॑ त्वा । \newline
40. आ॒रा॒द्‍घो॒षाय॑ त्वा त्वा ऽऽराद्‍घो॒षाया॑ राद्‍घो॒षाय॑ त्वा॒ प्रच्यु॑त्यै॒ प्रच्यु॑त्यै त्वा ऽऽराद्‍घो॒षाया॑ राद्‍घो॒षाय॑ त्वा॒ प्रच्यु॑त्यै । \newline
41. आ॒रा॒द्‍घो॒षायेत्या॑रात् - घो॒षाय॑ । \newline
42. त्वा॒ प्रच्यु॑त्यै॒ प्रच्यु॑त्यै त्वा त्वा॒ प्रच्यु॑त्यै त्वा त्वा॒ प्रच्यु॑त्यै त्वा त्वा॒ प्रच्यु॑त्यै त्वा । \newline
43. प्रच्यु॑त्यै त्वा त्वा॒ प्रच्यु॑त्यै॒ प्रच्यु॑त्यै त्वा । \newline
44. प्रच्यु॑त्या॒ इति॒ प्र - च्यु॒त्यै॒ । \newline
45. त्वेति॑ त्वा । \newline
\pagebreak
\markright{ TS 7.5.14.1  \hfill https://www.vedavms.in \hfill}

\section{ TS 7.5.14.1 }

\textbf{TS 7.5.14.1 } \newline
\textbf{Samhita Paata} \newline

अ॒ग्नये॑ गाय॒त्राय॑ त्रि॒वृते॒ राथ॑न्तराय वास॒न्ताया॒-ष्टाक॑पाल॒ इन्द्रा॑य॒ त्रैष्टु॑भाय पञ्चद॒शाय॒ बार्.ह॑ताय॒ ग्रैष्मा॒यैका॑दशकपालो॒ विश्वे᳚भ्यो दे॒वेभ्यो॒ जाग॑तेभ्यः सप्तद॒शेभ्यो॑ वैरू॒पेभ्यो॒ वार्.षि॑केभ्यो॒ द्वाद॑शकपालो मि॒त्रावरु॑णाभ्या॒-मानु॑ष्टुभाभ्या-मेकविꣳ॒॒शाभ्यां᳚ ॅवैरा॒जाभ्याꣳ॑ शार॒दाभ्यां᳚ पय॒स्या॑ बृह॒स्पत॑ये॒ पाङ्क्ता॑य त्रिण॒वाय॑ शाक्व॒राय॒ हैम॑न्तिकाय च॒रुः स॑वि॒त्र आ॑तिच्छन्द॒साय॑ त्रयस्त्रिꣳ॒॒शाय॑ रैव॒ताय॑ शैशि॒राय॒ द्वाद॑शकपा॒लो ऽदि॑त्यै॒ विष्णु॑पत्न्यै च॒रुर॒ग्नये॑ वैश्वान॒राय॒ द्वाद॑शकपा॒लो ऽनु॑मत्यै च॒रुः का॒य एक॑कपालः ॥ \newline

\textbf{Pada Paata} \newline

अ॒ग्नये᳚ । गा॒य॒त्राय॑ । त्रि॒वृत॒ इति॑ त्रि-वृते᳚ । राथ॑न्तरा॒येति॒ राथं᳚-त॒रा॒य॒ । वा॒स॒न्ताय॑ । अ॒ष्टाक॑पाल॒ इत्य॒ष्टा - क॒पा॒लः॒ । इन्द्रा॑य । त्रैष्टु॑भाय । प॒ञ्च॒द॒शायेति॑ पञ्च - द॒शाय॑ । बार्.ह॑ताय । ग्रैष्मा॑य । एका॑दशकपाल॒ इत्येका॑दश - क॒पा॒लः॒ । विश्वे᳚भ्यः । दे॒वेभ्यः॑ । जाग॑तेभ्यः । स॒प्त॒द॒शेभ्य॒ इति॑ सप्त - द॒शेभ्यः॑ । वै॒रू॒पेभ्यः॑ । वार्.षि॑केभ्यः । द्वाद॑शकपाल॒ इति॒ द्वाद॑श - क॒पा॒लः॒ । मि॒त्रावरु॑णाभ्या॒मिति॑ मि॒त्रा - वरु॑णाभ्याम् । आनु॑ष्टुभाभ्या॒मित्यानु॑- स्तु॒भा॒भ्या॒म् । ए॒क॒विꣳ॒॒शाभ्या॒मित्ये॑क-विꣳ॒॒शाभ्या᳚म् । वै॒रा॒जाभ्या᳚म् । शा॒र॒दाभ्या᳚म् । प॒य॒स्या᳚ । बृह॒स्पत॑ये । पाङ्क्ता॑य । त्रि॒ण॒वायेति॑ त्रि - न॒वाय॑ । शा॒क्व॒राय॑ । हैम॑न्तिकाय । च॒रुः । स॒वि॒त्रे । आ॒ति॒च्छ॒न्द॒सायेत्या॑ति - छ॒न्द॒साय॑ । त्र॒य॒स्त्रिꣳ॒॒शायेति॑ त्रयः - त्रिꣳ॒॒शाय॑ । रै॒व॒ताय॑ । शै॒शि॒राय॑ । द्वाद॑शकपाल॒ इति॒ द्वाद॑श -क॒पा॒लः॒ । अदि॑त्यै । विष्णु॑पत्न्या॒ इति॒ विष्णु॑ - प॒त्न्यै॒ । च॒रुः । अ॒ग्नये᳚ । वै॒श्वा॒न॒राय॑ । द्वाद॑शकपाल॒ इति॒ द्वाद॑श - क॒पा॒लः॒ । अनु॑मत्या॒ इत्यनु॑ - म॒त्यै॒ । च॒रुः । का॒यः । एक॑कपाल॒ इत्येक॑-क॒पा॒लः॒ ॥  \newline


\textbf{Krama Paata} \newline

अ॒ग्नये॑ गाय॒त्राय॑ । गा॒य॒त्राय॑ त्रि॒वृते᳚ । त्रि॒वृते॒ राथ॑न्तराय । त्रि॒वृत॒ इति॑ त्रि - वृते᳚ । राथ॑न्तराय वास॒न्ताय॑ । राथ॑न्ततरा॒येति॒ राथ᳚म् - त॒रा॒य॒ । वा॒स॒न्ताया॒ष्टाक॑पालः । अ॒ष्टाक॑पाल॒ इन्द्रा॑य । अ॒ष्टाक॑पाल॒ इत्य॒ष्टा - क॒पा॒लः॒ । इन्द्रा॑य॒ त्रैष्टु॑भाय । त्रैष्टु॑भाय पञ्चद॒शाय॑ । प॒ञ्च॒द॒शाय॒ बार्.ह॑ताय । प॒ञ्च॒द॒शायेति॑ पञ्च - द॒शाय॑ । बार्.ह॑ताय॒ ग्रैष्मा॑य । ग्रैष्मा॒यैका॑दशकपालः । एका॑दशकपालो॒ विश्वे᳚भ्यः । एका॑दशकपाल॒ इत्येका॑दश - क॒पा॒लः॒ । विश्वे᳚भ्यो दे॒वेभ्यः॑ । दे॒वेभ्यो॒ जाग॑तेभ्यः । जाग॑तेभ्यः सप्तद॒शेभ्यः॑ । स॒प्त॒द॒शेभ्यो॑ वैरू॒पेभ्यः॑ । स॒प्त॒द॒शेभ्य॒ इति॑ सप्त - द॒शेभ्यः॑ । वै॒रू॒पेभ्यो॒ वार्.षि॑केभ्यः । वार्.षि॑केभ्यो॒ द्वाद॑शकपालः । द्वाद॑शकपालो मि॒त्रावरु॑णाभ्याम् । द्वाद॑शकपाल॒ इति॒ द्वाद॑श - क॒पा॒लः॒ । मि॒त्रावरु॑णाभ्या॒मानु॑ष्टुभाभ्याम् । मि॒त्रावरु॑णाभ्या॒मिति॑ मि॒त्रा - वरु॑णाभ्याम् । आनु॑ष्टुभाभ्यामेकविꣳ॒॒शाभ्या᳚म् । आनु॑ष्टुभाभ्या॒मित्यानु॑ - स्तु॒भा॒भ्या॒म् । ए॒क॒विꣳ॒॒शाभ्या᳚म् ॅवैरा॒जाभ्या᳚म् । ए॒क॒विꣳ॒॒शाभ्या॒मित्ये॑क - विꣳ॒॒शाभ्या᳚म् । वै॒रा॒जाभ्याꣳ॑ शार॒दाभ्या᳚म् । शा॒र॒दाभ्या᳚म् पय॒स्या᳚ । प॒य॒स्या॑ बृह॒स्पत॑ये । बृह॒स्पत॑ये॒ पाङ्‍क्ता॑य । पाङ्‍क्ता॑य त्रिण॒वाय॑ । त्रि॒ण॒वाय॑ शाक्व॒राय॑ । त्रि॒ण॒वायेति॑ त्रि - न॒वाय॑ । शा॒क्व॒राय॒ हैम॑न्तिकाय । हैम॑न्तिकाय च॒रुः । च॒रुः स॑वि॒त्रे । स॒वि॒त्र आ॑तिच्छन्द॒साय॑ । आ॒ति॒च्छ॒न्द॒साय॑ त्रयस्त्रिꣳ॒॒शाय॑ । आ॒ति॒च्छ॒न्द॒सायेत्या॑ति - छ॒न्द॒साय॑ । त्र॒य॒स्त्रिꣳ॒॒शाय॑ रैव॒ताय॑ । त्र॒य॒स्त्रिꣳ॒॒शायेति॑ त्रयः - त्रिꣳ॒॒शाय॑ । रै॒व॒ताय॑ शैशि॒राय॑ । शै॒शि॒राय॒ द्वाद॑शकपालः । द्वाद॑शकपा॒लोऽदि॑त्यै । द्वाद॑शकपाल॒ इति॒ द्वाद॑श - क॒पा॒लः॒ । अदि॑त्यै॒ विष्णु॑पत्न्यै । विष्णु॑पत्न्यै च॒रुः । विष्णु॑पत्न्या॒ इति॒ विष्णु॑ - प॒त्न्यै॒ । च॒रुर॒ग्नये᳚ । अ॒ग्नये॑ वैश्वान॒राय॑ । वै॒श्वा॒न॒राय॒ द्वाद॑शकपालः । द्वाद॑शकपा॒लोऽनु॑मत्यै । द्वाद॑शकपाल॒ इति॒ द्वाद॑श - क॒पा॒लः॒ । अनु॑मत्यै च॒रुः । अनु॑मत्या॒ इत्यनु॑ - म॒त्यै॒ । च॒रुः का॒यः । का॒य एक॑कपालः । एक॑कपाल॒ इत्येक॑ - क॒पा॒लः॒ । \newline

\textbf{Jatai Paata} \newline

1. अ॒ग्नये॑ गाय॒त्राय॑ गाय॒त्राया॒ ग्नये॒ ऽग्नये॑ गाय॒त्राय॑ । \newline
2. गा॒य॒त्राय॑ त्रि॒वृते᳚ त्रि॒वृते॑ गाय॒त्राय॑ गाय॒त्राय॑ त्रि॒वृते᳚ । \newline
3. त्रि॒वृते॒ राथ॑न्तराय॒ राथ॑न्तराय त्रि॒वृते᳚ त्रि॒वृते॒ राथ॑न्तराय । \newline
4. त्रि॒वृत॒ इति॑ त्रि - वृते᳚ । \newline
5. राथ॑न्तराय वास॒न्ताय॑ वास॒न्ताय॒ राथ॑न्तराय॒ राथ॑न्तराय वास॒न्ताय॑ । \newline
6. राथ॑न्तरा॒येति॒ राथं᳚ - त॒रा॒य॒ । \newline
7. वा॒स॒न्ताया॒ ष्टाक॑पालो॒ ऽष्टाक॑पालो वास॒न्ताय॑ वास॒न्ताया॒ ष्टाक॑पालः । \newline
8. अ॒ष्टाक॑पाल॒ इन्द्रा॒ येन्द्रा॑या॒ष्टा क॑पालो॒ ऽष्टाक॑पाल॒ इन्द्रा॑य । \newline
9. अ॒ष्टाक॑पाल॒ इत्य॒ष्टा - क॒पा॒लः॒ । \newline
10. इन्द्रा॑य॒ त्रैष्टु॑भाय॒ त्रैष्टु॑भा॒ येन्द्रा॒ येन्द्रा॑य॒ त्रैष्टु॑भाय । \newline
11. त्रैष्टु॑भाय पञ्चद॒शाय॑ पञ्चद॒शाय॒ त्रैष्टु॑भाय॒ त्रैष्टु॑भाय पञ्चद॒शाय॑ । \newline
12. प॒ञ्च॒द॒शाय॒ बार्.ह॑ताय॒ बार्.ह॑ताय पञ्चद॒शाय॑ पञ्चद॒शाय॒ बार्.ह॑ताय । \newline
13. प॒ञ्च॒द॒शायेति॑ पञ्च - द॒शाय॑ । \newline
14. बार्.ह॑ताय॒ ग्रैष्मा॑य॒ ग्रैष्मा॑य॒ बार्.ह॑ताय॒ बार्.ह॑ताय॒ ग्रैष्मा॑य । \newline
15. ग्रैष्मा॒ यैका॑दशकपाल॒ एका॑दशकपालो॒ ग्रैष्मा॑य॒ ग्रैष्मा॒ यैका॑दशकपालः । \newline
16. एका॑दशकपालो॒ विश्वे᳚भ्यो॒ विश्वे᳚भ्य॒ एका॑दशकपाल॒ एका॑दशकपालो॒ विश्वे᳚भ्यः । \newline
17. एका॑दशकपाल॒ इत्येका॑दश - क॒पा॒लः॒ । \newline
18. विश्वे᳚भ्यो दे॒वेभ्यो॑ दे॒वेभ्यो॒ विश्वे᳚भ्यो॒ विश्वे᳚भ्यो दे॒वेभ्यः॑ । \newline
19. दे॒वेभ्यो॒ जाग॑तेभ्यो॒ जाग॑तेभ्यो दे॒वेभ्यो॑ दे॒वेभ्यो॒ जाग॑तेभ्यः । \newline
20. जाग॑तेभ्यः सप्तद॒शेभ्यः॑ सप्तद॒शेभ्यो॒ जाग॑तेभ्यो॒ जाग॑तेभ्यः सप्तद॒शेभ्यः॑ । \newline
21. स॒प्त॒द॒शेभ्यो॑ वैरू॒पेभ्यो॑ वैरू॒पेभ्यः॑ सप्तद॒शेभ्यः॑ सप्तद॒शेभ्यो॑ वैरू॒पेभ्यः॑ । \newline
22. स॒प्त॒द॒शेभ्य॒ इति॑ सप्त - द॒शेभ्यः॑ । \newline
23. वै॒रू॒पेभ्यो॒ वार्.षि॑केभ्यो॒ वार्.षि॑केभ्यो वैरू॒पेभ्यो॑ वैरू॒पेभ्यो॒ वार्.षि॑केभ्यः । \newline
24. वार्.षि॑केभ्यो॒ द्वाद॑शकपालो॒ द्वाद॑शकपालो॒ वार्.षि॑केभ्यो॒ वार्.षि॑केभ्यो॒ द्वाद॑शकपालः । \newline
25. द्वाद॑शकपालो मि॒त्रावरु॑णाभ्याम् मि॒त्रावरु॑णाभ्या॒म् द्वाद॑शकपालो॒ द्वाद॑शकपालो मि॒त्रावरु॑णाभ्याम् । \newline
26. द्वाद॑शकपाल॒ इति॒ द्वाद॑श - क॒पा॒लः॒ । \newline
27. मि॒त्रावरु॑णाभ्या॒ मानु॑ष्टुभाभ्या॒ मानु॑ष्टुभाभ्याम् मि॒त्रावरु॑णाभ्याम् मि॒त्रावरु॑णाभ्या॒ मानु॑ष्टुभाभ्याम् । \newline
28. मि॒त्रावरु॑णाभ्या॒मिति॑ मि॒त्रा - वरु॑णाभ्याम् । \newline
29. आनु॑ष्टुभाभ्या मेकविꣳ॒॒शाभ्या॑ मेकविꣳ॒॒शाभ्या॒ मानु॑ष्टुभाभ्या॒ मानु॑ष्टुभाभ्या मेकविꣳ॒॒शाभ्या᳚म् । \newline
30. आनु॑ष्टुभाभ्या॒मित्यानु॑ - स्तु॒भा॒भ्या॒म् । \newline
31. ए॒क॒विꣳ॒॒शाभ्यां᳚ ॅवैरा॒जाभ्यां᳚ ॅवैरा॒जाभ्या॑ मेकविꣳ॒॒शाभ्या॑ मेकविꣳ॒॒शाभ्यां᳚ ॅवैरा॒जाभ्या᳚म् । \newline
32. ए॒क॒विꣳ॒॒शाभ्या॒मित्ये॑क - विꣳ॒॒शाभ्या᳚म् । \newline
33. वै॒रा॒जाभ्याꣳ॑ शार॒दाभ्याꣳ॑ शार॒दाभ्यां᳚ ॅवैरा॒जाभ्यां᳚ ॅवैरा॒जाभ्याꣳ॑ शार॒दाभ्या᳚म् । \newline
34. शा॒र॒दाभ्या᳚म् पय॒स्या॑ पय॒स्या॑ शार॒दाभ्याꣳ॑ शार॒दाभ्या᳚म् पय॒स्या᳚ । \newline
35. प॒य॒स्या॑ बृह॒स्पत॑ये॒ बृह॒स्पत॑ये पय॒स्या॑ पय॒स्या॑ बृह॒स्पत॑ये । \newline
36. बृह॒स्पत॑ये॒ पाङ्क्ता॑य॒ पाङ्क्ता॑य॒ बृह॒स्पत॑ये॒ बृह॒स्पत॑ये॒ पाङ्क्ता॑य । \newline
37. पाङ्क्ता॑य त्रिण॒वाय॑ त्रिण॒वाय॒ पाङ्क्ता॑य॒ पाङ्क्ता॑य त्रिण॒वाय॑ । \newline
38. त्रि॒ण॒वाय॑ शाक्व॒राय॑ शाक्व॒राय॑ त्रिण॒वाय॑ त्रिण॒वाय॑ शाक्व॒राय॑ । \newline
39. त्रि॒ण॒वायेति॑ त्रि - न॒वाय॑ । \newline
40. शा॒क्व॒राय॒ हैम॑न्तिकाय॒ हैम॑न्तिकाय शाक्व॒राय॑ शाक्व॒राय॒ हैम॑न्तिकाय । \newline
41. हैम॑न्तिकाय च॒रु श्च॒रुर्. हैम॑न्तिकाय॒ हैम॑न्तिकाय च॒रुः । \newline
42. च॒रुः स॑वि॒त्रे स॑वि॒त्रे च॒रु श्च॒रुः स॑वि॒त्रे । \newline
43. स॒वि॒त्र आ॑तिच्छन्द॒साया॑ तिच्छन्द॒साय॑ सवि॒त्रे स॑वि॒त्र आ॑तिच्छन्द॒साय॑ । \newline
44. आ॒ति॒च्छ॒न्द॒साय॑ त्रयस्त्रिꣳ॒॒शाय॑ त्रयस्त्रिꣳ॒॒शाया॑ तिच्छन्द॒साया॑ तिच्छन्द॒साय॑ त्रयस्त्रिꣳ॒॒शाय॑ । \newline
45. आ॒ति॒च्छ॒न्द॒सायेत्या॑ति - छ॒न्द॒साय॑ । \newline
46. त्र॒य॒स्त्रिꣳ॒॒शाय॑ रैव॒ताय॑ रैव॒ताय॑ त्रयस्त्रिꣳ॒॒शाय॑ त्रयस्त्रिꣳ॒॒शाय॑ रैव॒ताय॑ । \newline
47. त्र॒य॒स्त्रिꣳ॒॒शायेति॑ त्रयः - त्रिꣳ॒॒शाय॑ । \newline
48. रै॒व॒ताय॑ शैशि॒राय॑ शैशि॒राय॑ रैव॒ताय॑ रैव॒ताय॑ शैशि॒राय॑ । \newline
49. शै॒शि॒राय॒ द्वाद॑शकपालो॒ द्वाद॑शकपालः शैशि॒राय॑ शैशि॒राय॒ द्वाद॑शकपालः । \newline
50. द्वाद॑शकपा॒लो ऽदि॑त्या॒ अदि॑त्यै॒ द्वाद॑शकपालो॒ द्वाद॑शकपा॒लो ऽदि॑त्यै । \newline
51. द्वाद॑शकपाल॒ इति॒ द्वाद॑श - क॒पा॒लः॒ । \newline
52. अदि॑त्यै॒ विष्णु॑पत्न्यै॒ विष्णु॑पत्न्या॒ अदि॑त्या॒ अदि॑त्यै॒ विष्णु॑पत्न्यै । \newline
53. विष्णु॑पत्न्यै च॒रु श्च॒रुर् विष्णु॑पत्न्यै॒ विष्णु॑पत्न्यै च॒रुः । \newline
54. विष्णु॑पत्न्या॒ इति॒ विष्णु॑ - प॒त्न्यै॒ । \newline
55. च॒रु र॒ग्नये॒ ऽग्नये॑ च॒रु श्च॒रु र॒ग्नये᳚ । \newline
56. अ॒ग्नये॑ वैश्वान॒राय॑ वैश्वान॒राया॒ ग्नये॒ ऽग्नये॑ वैश्वान॒राय॑ । \newline
57. वै॒श्वा॒न॒राय॒ द्वाद॑शकपालो॒ द्वाद॑शकपालो वैश्वान॒राय॑ वैश्वान॒राय॒ द्वाद॑शकपालः । \newline
58. द्वाद॑शकपा॒लो ऽनु॑मत्या॒ अनु॑मत्यै॒ द्वाद॑शकपालो॒ द्वाद॑शकपा॒लो ऽनु॑मत्यै । \newline
59. द्वाद॑शकपाल॒ इति॒ द्वाद॑श - क॒पा॒लः॒ । \newline
60. अनु॑मत्यै च॒रु श्च॒रु रनु॑मत्या॒ अनु॑मत्यै च॒रुः । \newline
61. अनु॑मत्या॒ इत्यनु॑ - म॒त्यै॒ । \newline
62. च॒रुः का॒यः का॒य श्च॒रु श्च॒रुः का॒यः । \newline
63. का॒य एक॑कपाल॒ एक॑कपालः का॒यः का॒य एक॑कपालः । \newline
64. एक॑कपाल॒ इत्येक॑ - क॒पा॒लः॒ । \newline

\textbf{Ghana Paata } \newline

1. अ॒ग्नये॑ गाय॒त्राय॑ गाय॒त्राया॒ ग्नये॒ ऽग्नये॑ गाय॒त्राय॑ त्रि॒वृते᳚ त्रि॒वृते॑ गाय॒त्राया॒ ग्नये॒ ऽग्नये॑ गाय॒त्राय॑ त्रि॒वृते᳚ । \newline
2. गा॒य॒त्राय॑ त्रि॒वृते᳚ त्रि॒वृते॑ गाय॒त्राय॑ गाय॒त्राय॑ त्रि॒वृते॒ राथ॑न्तराय॒ राथ॑न्तराय त्रि॒वृते॑ गाय॒त्राय॑ गाय॒त्राय॑ त्रि॒वृते॒ राथ॑न्तराय । \newline
3. त्रि॒वृते॒ राथ॑न्तराय॒ राथ॑न्तराय त्रि॒वृते᳚ त्रि॒वृते॒ राथ॑न्तराय वास॒न्ताय॑ वास॒न्ताय॒ राथ॑न्तराय त्रि॒वृते᳚ त्रि॒वृते॒ राथ॑न्तराय वास॒न्ताय॑ । \newline
4. त्रि॒वृत॒ इति॑ त्रि - वृते᳚ । \newline
5. राथ॑न्तराय वास॒न्ताय॑ वास॒न्ताय॒ राथ॑न्तराय॒ राथ॑न्तराय वास॒न्ताया॒ ष्टाक॑पालो॒ ऽष्टाक॑पालो वास॒न्ताय॒ राथ॑न्तराय॒ राथ॑न्तराय वास॒न्ताया॒ ष्टाक॑पालः । \newline
6. राथ॑न्तरा॒येति॒ राथं᳚ - त॒रा॒य॒ । \newline
7. वा॒स॒न्ताया॒ ष्टाक॑पालो॒ ऽष्टाक॑पालो वास॒न्ताय॑ वास॒न्ताया॒ ष्टाक॑पाल॒ इन्द्रा॒ येन्द्रा॑या॒ ष्टाक॑पालो वास॒न्ताय॑ वास॒न्ताया॒ ष्टाक॑पाल॒ इन्द्रा॑य । \newline
8. अ॒ष्टाक॑पाल॒ इन्द्रा॒ येन्द्रा॑या॒ ष्टाक॑पालो॒ ऽष्टाक॑पाल॒ इन्द्रा॑य॒ त्रैष्टु॑भाय॒ त्रैष्टु॑भा॒ येन्द्रा॑या॒ ष्टाक॑पालो॒ ऽष्टाक॑पाल॒ इन्द्रा॑य॒ त्रैष्टु॑भाय । \newline
9. अ॒ष्टाक॑पाल॒ इत्य॒ष्टा - क॒पा॒लः॒ । \newline
10. इन्द्रा॑य॒ त्रैष्टु॑भाय॒ त्रैष्टु॑भा॒ येन्द्रा॒ येन्द्रा॑य॒ त्रैष्टु॑भाय पञ्चद॒शाय॑ पञ्चद॒शाय॒ त्रैष्टु॑भा॒ येन्द्रा॒ येन्द्रा॑य॒ त्रैष्टु॑भाय पञ्चद॒शाय॑ । \newline
11. त्रैष्टु॑भाय पञ्चद॒शाय॑ पञ्चद॒शाय॒ त्रैष्टु॑भाय॒ त्रैष्टु॑भाय पञ्चद॒शाय॒ बार्.ह॑ताय॒ बार्.ह॑ताय पञ्चद॒शाय॒ त्रैष्टु॑भाय॒ त्रैष्टु॑भाय पञ्चद॒शाय॒ बार्.ह॑ताय । \newline
12. प॒ञ्च॒द॒शाय॒ बार्.ह॑ताय॒ बार्.ह॑ताय पञ्चद॒शाय॑ पञ्चद॒शाय॒ बार्.ह॑ताय॒ ग्रैष्मा॑य॒ ग्रैष्मा॑य॒ बार्.ह॑ताय पञ्चद॒शाय॑ पञ्चद॒शाय॒ बार्.ह॑ताय॒ ग्रैष्मा॑य । \newline
13. प॒ञ्च॒द॒शायेति॑ पञ्च - द॒शाय॑ । \newline
14. बार्.ह॑ताय॒ ग्रैष्मा॑य॒ ग्रैष्मा॑य॒ बार्.ह॑ताय॒ बार्.ह॑ताय॒ ग्रैष्मा॒ यैका॑दशकपाल॒ एका॑दशकपालो॒ ग्रैष्मा॑य॒ बार्.ह॑ताय॒ बार्.ह॑ताय॒ ग्रैष्मा॒ यैका॑दशकपालः । \newline
15. ग्रैष्मा॒ यैका॑दशकपाल॒ एका॑दशकपालो॒ ग्रैष्मा॑य॒ ग्रैष्मा॒ यैका॑दशकपालो॒ विश्वे᳚भ्यो॒ विश्वे᳚भ्य॒ एका॑दशकपालो॒ ग्रैष्मा॑य॒ ग्रैष्मा॒ यैका॑दशकपालो॒ विश्वे᳚भ्यः । \newline
16. एका॑दशकपालो॒ विश्वे᳚भ्यो॒ विश्वे᳚भ्य॒ एका॑दशकपाल॒ एका॑दशकपालो॒ विश्वे᳚भ्यो दे॒वेभ्यो॑ दे॒वेभ्यो॒ विश्वे᳚भ्य॒ एका॑दशकपाल॒ एका॑दशकपालो॒ विश्वे᳚भ्यो दे॒वेभ्यः॑ । \newline
17. एका॑दशकपाल॒ इत्येका॑दश - क॒पा॒लः॒ । \newline
18. विश्वे᳚भ्यो दे॒वेभ्यो॑ दे॒वेभ्यो॒ विश्वे᳚भ्यो॒ विश्वे᳚भ्यो दे॒वेभ्यो॒ जाग॑तेभ्यो॒ जाग॑तेभ्यो दे॒वेभ्यो॒ विश्वे᳚भ्यो॒ विश्वे᳚भ्यो दे॒वेभ्यो॒ जाग॑तेभ्यः । \newline
19. दे॒वेभ्यो॒ जाग॑तेभ्यो॒ जाग॑तेभ्यो दे॒वेभ्यो॑ दे॒वेभ्यो॒ जाग॑तेभ्यः सप्तद॒शेभ्यः॑ सप्तद॒शेभ्यो॒ जाग॑तेभ्यो दे॒वेभ्यो॑ दे॒वेभ्यो॒ जाग॑तेभ्यः सप्तद॒शेभ्यः॑ । \newline
20. जाग॑तेभ्यः सप्तद॒शेभ्यः॑ सप्तद॒शेभ्यो॒ जाग॑तेभ्यो॒ जाग॑तेभ्यः सप्तद॒शेभ्यो॑ वैरू॒पेभ्यो॑ वैरू॒पेभ्यः॑ सप्तद॒शेभ्यो॒ जाग॑तेभ्यो॒ जाग॑तेभ्यः सप्तद॒शेभ्यो॑ वैरू॒पेभ्यः॑ । \newline
21. स॒प्त॒द॒शेभ्यो॑ वैरू॒पेभ्यो॑ वैरू॒पेभ्यः॑ सप्तद॒शेभ्यः॑ सप्तद॒शेभ्यो॑ वैरू॒पेभ्यो॒ वार्.षि॑केभ्यो॒ वार्.षि॑केभ्यो वैरू॒पेभ्यः॑ सप्तद॒शेभ्यः॑ सप्तद॒शेभ्यो॑ वैरू॒पेभ्यो॒ वार्.षि॑केभ्यः । \newline
22. स॒प्त॒द॒शेभ्य॒ इति॑ सप्त - द॒शेभ्यः॑ । \newline
23. वै॒रू॒पेभ्यो॒ वार्.षि॑केभ्यो॒ वार्.षि॑केभ्यो वैरू॒पेभ्यो॑ वैरू॒पेभ्यो॒ वार्.षि॑केभ्यो॒ द्वाद॑शकपालो॒ द्वाद॑शकपालो॒ वार्.षि॑केभ्यो वैरू॒पेभ्यो॑ वैरू॒पेभ्यो॒ वार्.षि॑केभ्यो॒ द्वाद॑शकपालः । \newline
24. वार्.षि॑केभ्यो॒ द्वाद॑शकपालो॒ द्वाद॑शकपालो॒ वार्.षि॑केभ्यो॒ वार्.षि॑केभ्यो॒ द्वाद॑शकपालो मि॒त्रावरु॑णाभ्याम् मि॒त्रावरु॑णाभ्या॒म् द्वाद॑शकपालो॒ वार्.षि॑केभ्यो॒ वार्.षि॑केभ्यो॒ द्वाद॑शकपालो मि॒त्रावरु॑णाभ्याम् । \newline
25. द्वाद॑शकपालो मि॒त्रावरु॑णाभ्याम् मि॒त्रावरु॑णाभ्या॒म् द्वाद॑शकपालो॒ द्वाद॑शकपालो मि॒त्रावरु॑णाभ्या॒ मानु॑ष्टुभाभ्या॒ मानु॑ष्टुभाभ्याम् मि॒त्रावरु॑णाभ्या॒म् द्वाद॑शकपालो॒ द्वाद॑शकपालो मि॒त्रावरु॑णाभ्या॒ मानु॑ष्टुभाभ्याम् । \newline
26. द्वाद॑शकपाल॒ इति॒ द्वाद॑श - क॒पा॒लः॒ । \newline
27. मि॒त्रावरु॑णाभ्या॒ मानु॑ष्टुभाभ्या॒ मानु॑ष्टुभाभ्याम् मि॒त्रावरु॑णाभ्याम् मि॒त्रावरु॑णाभ्या॒ मानु॑ष्टुभाभ्या मेकविꣳ॒॒शाभ्या॑ मेकविꣳ॒॒शाभ्या॒ मानु॑ष्टुभाभ्याम् मि॒त्रावरु॑णाभ्याम् मि॒त्रावरु॑णाभ्या॒ मानु॑ष्टुभाभ्या मेकविꣳ॒॒शाभ्या᳚म् । \newline
28. मि॒त्रावरु॑णाभ्या॒मिति॑ मि॒त्रा - वरु॑णाभ्याम् । \newline
29. आनु॑ष्टुभाभ्या मेकविꣳ॒॒शाभ्या॑ मेकविꣳ॒॒शाभ्या॒ मानु॑ष्टुभाभ्या॒ मानु॑ष्टुभाभ्या मेकविꣳ॒॒शाभ्यां᳚ ॅवैरा॒जाभ्यां᳚ ॅवैरा॒जाभ्या॑ मेकविꣳ॒॒शाभ्या॒ मानु॑ष्टुभाभ्या॒ मानु॑ष्टुभाभ्या मेकविꣳ॒॒शाभ्यां᳚ ॅवैरा॒जाभ्या᳚म् । \newline
30. आनु॑ष्टुभाभ्या॒मित्यानु॑ - स्तु॒भा॒भ्या॒म् । \newline
31. ए॒क॒विꣳ॒॒शाभ्यां᳚ ॅवैरा॒जाभ्यां᳚ ॅवैरा॒जाभ्या॑ मेकविꣳ॒॒शाभ्या॑ मेकविꣳ॒॒शाभ्यां᳚ ॅवैरा॒जाभ्याꣳ॑ शार॒दाभ्याꣳ॑ शार॒दाभ्यां᳚ ॅवैरा॒जाभ्या॑ मेकविꣳ॒॒शाभ्या॑ मेकविꣳ॒॒शाभ्यां᳚ ॅवैरा॒जाभ्याꣳ॑ शार॒दाभ्या᳚म् । \newline
32. ए॒क॒विꣳ॒॒शाभ्या॒मित्ये॑क - विꣳ॒॒शाभ्या᳚म् । \newline
33. वै॒रा॒जाभ्याꣳ॑ शार॒दाभ्याꣳ॑ शार॒दाभ्यां᳚ ॅवैरा॒जाभ्यां᳚ ॅवैरा॒जाभ्याꣳ॑ शार॒दाभ्या᳚म् पय॒स्या॑ पय॒स्या॑ शार॒दाभ्यां᳚ ॅवैरा॒जाभ्यां᳚ ॅवैरा॒जाभ्याꣳ॑ शार॒दाभ्या᳚म् पय॒स्या᳚ । \newline
34. शा॒र॒दाभ्या᳚म् पय॒स्या॑ पय॒स्या॑ शार॒दाभ्याꣳ॑ शार॒दाभ्या᳚म् पय॒स्या॑ बृह॒स्पत॑ये॒ बृह॒स्पत॑ये पय॒स्या॑ शार॒दाभ्याꣳ॑ शार॒दाभ्या᳚म् पय॒स्या॑ बृह॒स्पत॑ये । \newline
35. प॒य॒स्या॑ बृह॒स्पत॑ये॒ बृह॒स्पत॑ये पय॒स्या॑ पय॒स्या॑ बृह॒स्पत॑ये॒ पाङ्क्ता॑य॒ पाङ्क्ता॑य॒ बृह॒स्पत॑ये पय॒स्या॑ पय॒स्या॑ बृह॒स्पत॑ये॒ पाङ्क्ता॑य । \newline
36. बृह॒स्पत॑ये॒ पाङ्क्ता॑य॒ पाङ्क्ता॑य॒ बृह॒स्पत॑ये॒ बृह॒स्पत॑ये॒ पाङ्क्ता॑य त्रिण॒वाय॑ त्रिण॒वाय॒ पाङ्क्ता॑य॒ बृह॒स्पत॑ये॒ बृह॒स्पत॑ये॒ पाङ्क्ता॑य त्रिण॒वाय॑ । \newline
37. पाङ्क्ता॑य त्रिण॒वाय॑ त्रिण॒वाय॒ पाङ्क्ता॑य॒ पाङ्क्ता॑य त्रिण॒वाय॑ शाक्व॒राय॑ शाक्व॒राय॑ त्रिण॒वाय॒ पाङ्क्ता॑य॒ पाङ्क्ता॑य त्रिण॒वाय॑ शाक्व॒राय॑ । \newline
38. त्रि॒ण॒वाय॑ शाक्व॒राय॑ शाक्व॒राय॑ त्रिण॒वाय॑ त्रिण॒वाय॑ शाक्व॒राय॒ हैम॑न्तिकाय॒ हैम॑न्तिकाय शाक्व॒राय॑ त्रिण॒वाय॑ त्रिण॒वाय॑ शाक्व॒राय॒ हैम॑न्तिकाय । \newline
39. त्रि॒ण॒वायेति॑ त्रि - न॒वाय॑ । \newline
40. शा॒क्व॒राय॒ हैम॑न्तिकाय॒ हैम॑न्तिकाय शाक्व॒राय॑ शाक्व॒राय॒ हैम॑न्तिकाय च॒रु श्च॒रुर्. हैम॑न्तिकाय शाक्व॒राय॑ शाक्व॒राय॒ हैम॑न्तिकाय च॒रुः । \newline
41. हैम॑न्तिकाय च॒रु श्च॒रुर्. हैम॑न्तिकाय॒ हैम॑न्तिकाय च॒रुः स॑वि॒त्रे स॑वि॒त्रे च॒रुर्. हैम॑न्तिकाय॒ हैम॑न्तिकाय च॒रुः स॑वि॒त्रे । \newline
42. च॒रुः स॑वि॒त्रे स॑वि॒त्रे च॒रु श्च॒रुः स॑वि॒त्र आ॑तिच्छन्द॒साया॑ तिच्छन्द॒साय॑ सवि॒त्रे च॒रु श्च॒रुः स॑वि॒त्र आ॑तिच्छन्द॒साय॑ । \newline
43. स॒वि॒त्र आ॑तिच्छन्द॒साया॑ तिच्छन्द॒साय॑ सवि॒त्रे स॑वि॒त्र आ॑तिच्छन्द॒साय॑ त्रयस्त्रिꣳ॒॒शाय॑ त्रयस्त्रिꣳ॒॒शाया॑ तिच्छन्द॒साय॑ सवि॒त्रे स॑वि॒त्र आ॑तिच्छन्द॒साय॑ त्रयस्त्रिꣳ॒॒शाय॑ । \newline
44. आ॒ति॒च्छ॒न्द॒साय॑ त्रयस्त्रिꣳ॒॒शाय॑ त्रयस्त्रिꣳ॒॒शाया॑ तिच्छन्द॒साया॑ तिच्छन्द॒साय॑ त्रयस्त्रिꣳ॒॒शाय॑ रैव॒ताय॑ रैव॒ताय॑ त्रयस्त्रिꣳ॒॒शाया॑ तिच्छन्द॒साया॑ तिच्छन्द॒साय॑ त्रयस्त्रिꣳ॒॒शाय॑ रैव॒ताय॑ । \newline
45. आ॒ति॒च्छ॒न्द॒सायेत्या॑ति - छ॒न्द॒साय॑ । \newline
46. त्र॒य॒स्त्रिꣳ॒॒शाय॑ रैव॒ताय॑ रैव॒ताय॑ त्रयस्त्रिꣳ॒॒शाय॑ त्रयस्त्रिꣳ॒॒शाय॑ रैव॒ताय॑ शैशि॒राय॑ शैशि॒राय॑ रैव॒ताय॑ त्रयस्त्रिꣳ॒॒शाय॑ त्रयस्त्रिꣳ॒॒शाय॑ रैव॒ताय॑ शैशि॒राय॑ । \newline
47. त्र॒य॒स्त्रिꣳ॒॒शायेति॑ त्रयः - त्रिꣳ॒॒शाय॑ । \newline
48. रै॒व॒ताय॑ शैशि॒राय॑ शैशि॒राय॑ रैव॒ताय॑ रैव॒ताय॑ शैशि॒राय॒ द्वाद॑शकपालो॒ द्वाद॑शकपालः शैशि॒राय॑ रैव॒ताय॑ रैव॒ताय॑ शैशि॒राय॒ द्वाद॑शकपालः । \newline
49. शै॒शि॒राय॒ द्वाद॑शकपालो॒ द्वाद॑शकपालः शैशि॒राय॑ शैशि॒राय॒ द्वाद॑शकपा॒लो ऽदि॑त्या॒ अदि॑त्यै॒ द्वाद॑शकपालः शैशि॒राय॑ शैशि॒राय॒ द्वाद॑शकपा॒लो ऽदि॑त्यै । \newline
50. द्वाद॑शकपा॒लो ऽदि॑त्या॒ अदि॑त्यै॒ द्वाद॑शकपालो॒ द्वाद॑शकपा॒लो ऽदि॑त्यै॒ विष्णु॑पत्न्यै॒ विष्णु॑पत्न्या॒ अदि॑त्यै॒ द्वाद॑शकपालो॒ द्वाद॑शकपा॒लो ऽदि॑त्यै॒ विष्णु॑पत्न्यै । \newline
51. द्वाद॑शकपाल॒ इति॒ द्वाद॑श - क॒पा॒लः॒ । \newline
52. अदि॑त्यै॒ विष्णु॑पत्न्यै॒ विष्णु॑पत्न्या॒ अदि॑त्या॒ अदि॑त्यै॒ विष्णु॑पत्न्यै च॒रु श्च॒रुर् विष्णु॑पत्न्या॒ अदि॑त्या॒ अदि॑त्यै॒ विष्णु॑पत्न्यै च॒रुः । \newline
53. विष्णु॑पत्न्यै च॒रु श्च॒रुर् विष्णु॑पत्न्यै॒ विष्णु॑पत्न्यै च॒रु र॒ग्नये॒ ऽग्नये॑ च॒रुर् विष्णु॑पत्न्यै॒ विष्णु॑पत्न्यै च॒रु र॒ग्नये᳚ । \newline
54. विष्णु॑पत्न्या॒ इति॒ विष्णु॑ - प॒त्न्यै॒ । \newline
55. च॒रु र॒ग्नये॒ ऽग्नये॑ च॒रु श्च॒रु र॒ग्नये॑ वैश्वान॒राय॑ वैश्वान॒राया॒ ग्नये॑ च॒रु श्च॒रु र॒ग्नये॑ वैश्वान॒राय॑ । \newline
56. अ॒ग्नये॑ वैश्वान॒राय॑ वैश्वान॒राया॒ ग्नये॒ ऽग्नये॑ वैश्वान॒राय॒ द्वाद॑शकपालो॒ द्वाद॑शकपालो वैश्वान॒राया॒ ग्नये॒ ऽग्नये॑ वैश्वान॒राय॒ द्वाद॑शकपालः । \newline
57. वै॒श्वा॒न॒राय॒ द्वाद॑शकपालो॒ द्वाद॑शकपालो वैश्वान॒राय॑ वैश्वान॒राय॒ द्वाद॑शकपा॒लो ऽनु॑मत्या॒ अनु॑मत्यै॒ द्वाद॑शकपालो वैश्वान॒राय॑ वैश्वान॒राय॒ द्वाद॑शकपा॒लो ऽनु॑मत्यै । \newline
58. द्वाद॑शकपा॒लो ऽनु॑मत्या॒ अनु॑मत्यै॒ द्वाद॑शकपालो॒ द्वाद॑शकपा॒लो ऽनु॑मत्यै च॒रु श्च॒रु रनु॑मत्यै॒ द्वाद॑शकपालो॒ द्वाद॑शकपा॒लो ऽनु॑मत्यै च॒रुः । \newline
59. द्वाद॑शकपाल॒ इति॒ द्वाद॑श - क॒पा॒लः॒ । \newline
60. अनु॑मत्यै च॒रु श्च॒रु रनु॑मत्या॒ अनु॑मत्यै च॒रुः का॒यः का॒य श्च॒रु रनु॑मत्या॒ अनु॑मत्यै च॒रुः का॒यः । \newline
61. अनु॑मत्या॒ इत्यनु॑ - म॒त्यै॒ । \newline
62. च॒रुः का॒यः का॒य श्च॒रु श्च॒रुः का॒य एक॑कपाल॒ एक॑कपालः का॒य श्च॒रु श्च॒रुः का॒य एक॑कपालः । \newline
63. का॒य एक॑कपाल॒ एक॑कपालः का॒यः का॒य एक॑कपालः । \newline
64. एक॑कपाल॒ इत्येक॑ - क॒पा॒लः॒ । \newline
\pagebreak
\markright{ TS 7.5.15.1  \hfill https://www.vedavms.in \hfill}

\section{ TS 7.5.15.1 }

\textbf{TS 7.5.15.1 } \newline
\textbf{Samhita Paata} \newline

यो वा अ॒ग्नाव॒ग्निः प्र॑ह्रि॒यते॒ यश्च॒ सोमो॒ राजा॒ तयो॑रे॒ष आ॑ति॒थ्यं ॅयद॑ग्नीषो॒मीयोऽथै॒ष रु॒द्रो यश्ची॒यते॒ यथ् संचि॑ते॒ऽग्नावे॒तानि॑ ह॒वीꣳषि॒ न नि॒र्वपे॑दे॒ष ए॒व रु॒द्रोऽशा᳚न्त उपो॒त्थाय॑ प्र॒जां प॒शून् यज॑मानस्या॒भि म॑न्येत॒ यथ् संचि॑ते॒ऽग्नावे॒तानि॑ ह॒वीꣳषि॑ नि॒र्वप॑ति भाग॒धेये॑नै॒वैनꣳ॑ शमयति॒ नास्य॑ रु॒द्रोऽशा᳚न्त - [  ] \newline

\textbf{Pada Paata} \newline

यः । वै । अ॒ग्नौ । अ॒ग्निः । प्र॒ह्रि॒यत॒ इति॑ प्र - ह्रि॒यते᳚ । यः । च॒ । सोमः॑ । राजा᳚ । तयोः᳚ । ए॒षः । आ॒ति॒थ्यम् । यत् । अ॒ग्नी॒षो॒मीय॒ इत्य॑ग्नी - सो॒मीयः॑ । अथ॑ । ए॒षः । रु॒द्रः । यः । ची॒यते᳚ । यत् । सञ्चि॑त॒ इति॒ सं - चि॒ते॒ । अ॒ग्नौ । ए॒तानि॑ । ह॒वीꣳषि॑ । न । नि॒र्वपे॒दिति॑ निः - वपे᳚त् । ए॒षः । ए॒व । रु॒द्रः । अशा᳚न्तः । उ॒पो॒त्थायेत्यु॑प - उ॒त्थाय॑ । प्र॒जामिति॑ प्र - जाम् । प॒शून् । यज॑मानस्य । अ॒भीति॑ । म॒न्ये॒त॒ । यत् । सञ्चि॑त॒ इति॒ सं-चि॒ते॒ । अ॒ग्नौ । ए॒तानि॑ । ह॒वीꣳषि॑ । नि॒र्वप॒तीति॑ निः - वप॑ति । भा॒ग॒धेये॒नेति॑ भाग - धेये॑न । ए॒व । ए॒न॒म् । श॒म॒य॒ति॒ । न । अ॒स्य॒ । रु॒द्रः । अशा᳚न्तः ।  \newline


\textbf{Krama Paata} \newline

यो वै । वा अ॒ग्नौ । अ॒ग्नाव॒ग्निः । अ॒ग्निः प्र॑ह्रि॒यते᳚ । प्र॒ह्रि॒यते॒ यः । प्र॒ह्रि॒यत॒ इति॑ प्र - ह्रि॒यते᳚ । यश्च॑ । च॒ सोमः॑ । सोमो॒ राजा᳚ । राजा॒ तयोः᳚ । तयो॑रे॒षः । ए॒ष आ॑ति॒थ्यम् । आ॒ति॒थ्यम् ॅयत् । यद॑ग्नीषो॒मीयः॑ । अ॒ग्नी॒षो॒मीयोऽथ॑ । अ॒ग्नी॒षो॒मीय॒ इत्य॑ग्नी - सो॒मीयः॑ । अथै॒षः । ए॒ष रु॒द्रः । रु॒द्रो यः । यश्ची॒यते᳚ । ची॒यते॒ यत् । यथ् सञ्चि॑ते । सञ्चि॑ते॒ऽग्नौ । सञ्चि॑त॒ इति॒ सम् - चि॒ते॒ । अ॒ग्नावे॒तानि॑ । ए॒तानि॑ ह॒वीꣳषि॑ । ह॒वीꣳषि॒ न । न नि॒र्वपे᳚त् । नि॒र्वपे॑दे॒षः । नि॒र्वपे॒दिति॑ निः - वपे᳚त् । ए॒ष ए॒व । ए॒व रु॒द्रः । रु॒द्रोऽशा᳚न्तः । अशा᳚न्त उपो॒त्थाय॑ । उ॒पो॒त्थाय॑ प्र॒जाम् । उ॒पो॒त्थायेत्यु॑प - उ॒त्थाय॑ । प्र॒जाम् प॒शून् । प्र॒जामिति॑ प्र - जाम् । प॒शून्. यज॑मानस्य । यज॑मानस्या॒भि । अ॒भि म॑न्येत । म॒न्ये॒त॒ यत् । यथ् सञ्चि॑ते । सञ्चि॑ते॒ऽग्नौ । सञ्चि॑त॒ इति॒ सम् - चि॒ते॒ । अ॒ग्नावे॒तानि॑ । ए॒तानि॑ ह॒वीꣳषि॑ । ह॒वीꣳषि॑ नि॒र्वप॑ति । नि॒र्वप॑ति भाग॒धेये॑न । नि॒र्वप॒तीति॑ निः - वप॑ति । भा॒ग॒धेये॑नै॒व । भा॒ग॒धेये॒नेति॑ भाग - धेये॑न । ए॒वैन᳚म् । ए॒नꣳ॒॒ श॒म॒य॒ति॒ । श॒म॒य॒ति॒ न । नास्य॑ । अ॒स्य॒ रु॒द्रः । रु॒द्रोऽशा᳚न्तः । अशा᳚न्त उपो॒त्थाय॑ \newline

\textbf{Jatai Paata} \newline

1. यो वै वै यो यो वै । \newline
2. वा अ॒ग्ना व॒ग्नौ वै वा अ॒ग्नौ । \newline
3. अ॒ग्ना व॒ग्नि र॒ग्नि र॒ग्ना व॒ग्ना व॒ग्निः । \newline
4. अ॒ग्निः प्र॑ह्रि॒यते᳚ प्रह्रि॒यते॒ ऽग्नि र॒ग्निः प्र॑ह्रि॒यते᳚ । \newline
5. प्र॒ह्रि॒यते॒ यो यः प्र॑ह्रि॒यते᳚ प्रह्रि॒यते॒ यः । \newline
6. प्र॒ह्रि॒यत॒ इति॑ प्र - ह्रि॒यते᳚ । \newline
7. यश्च॑ च॒ यो यश्च॑ । \newline
8. च॒ सोमः॒ सोम॑श्च च॒ सोमः॑ । \newline
9. सोमो॒ राजा॒ राजा॒ सोमः॒ सोमो॒ राजा᳚ । \newline
10. राजा॒ तयो॒ स्तयो॒ राजा॒ राजा॒ तयोः᳚ । \newline
11. तयो॑ रे॒ष ए॒ष तयो॒ स्तयो॑ रे॒षः । \newline
12. ए॒ष आ॑ति॒थ्य मा॑ति॒थ्य मे॒ष ए॒ष आ॑ति॒थ्यम् । \newline
13. आ॒ति॒थ्यं ॅयद् यदा॑ति॒थ्य मा॑ति॒थ्यं ॅयत् । \newline
14. यद॑ग्नीषो॒मीयो᳚ ऽग्नीषो॒मीयो॒ यद् यद॑ग्नीषो॒मीयः॑ । \newline
15. अ॒ग्नी॒षो॒मीयो ऽथाथा᳚ ग्नीषो॒मीयो᳚ ऽग्नीषो॒मीयो ऽथ॑ । \newline
16. अ॒ग्नी॒षो॒मीय॒ इत्य॑ग्नी - सो॒मीयः॑ । \newline
17. अथै॒ष ए॒षो ऽथा थै॒षः । \newline
18. ए॒ष रु॒द्रो रु॒द्र ए॒ष ए॒ष रु॒द्रः । \newline
19. रु॒द्रो यो यो रु॒द्रो रु॒द्रो यः । \newline
20. य श्ची॒यते॑ ची॒यते॒ यो य श्ची॒यते᳚ । \newline
21. ची॒यते॒ यद् यच् ची॒यते॑ ची॒यते॒ यत् । \newline
22. यथ् सञ्चि॑ते॒ सञ्चि॑ते॒ यद् यथ् सञ्चि॑ते । \newline
23. सञ्चि॑ते॒ ऽग्ना व॒ग्नौ सञ्चि॑ते॒ सञ्चि॑ते॒ ऽग्नौ । \newline
24. सञ्चि॑त॒ इति॒ सं - चि॒ते॒ । \newline
25. अ॒ग्ना वे॒ता न्ये॒ता न्य॒ग्ना व॒ग्ना वे॒तानि॑ । \newline
26. ए॒तानि॑ ह॒वीꣳषि॑ ह॒वीꣳ ष्ये॒ता न्ये॒तानि॑ ह॒वीꣳषि॑ । \newline
27. ह॒वीꣳषि॒ न न ह॒वीꣳषि॑ ह॒वीꣳषि॒ न । \newline
28. न नि॒र्वपे᳚न् नि॒र्वपे॒न् न न नि॒र्वपे᳚त् । \newline
29. नि॒र्वपे॑ दे॒ष ए॒ष नि॒र्वपे᳚न् नि॒र्वपे॑ दे॒षः । \newline
30. नि॒र्वपे॒दिति॑ निः - वपे᳚त् । \newline
31. ए॒ष ए॒वै वैष ए॒ष ए॒व । \newline
32. ए॒व रु॒द्रो रु॒द्र ए॒वैव रु॒द्रः । \newline
33. रु॒द्रो ऽशा॒न्तो ऽशा᳚न्तो रु॒द्रो रु॒द्रो ऽशा᳚न्तः । \newline
34. अशा᳚न्त उपो॒त्था यो॑पो॒त्थाया शा॒न्तो ऽशा᳚न्त उपो॒त्थाय॑ । \newline
35. उ॒पो॒त्थाय॑ प्र॒जाम् प्र॒जा मु॑पो॒त्था यो॑पो॒त्थाय॑ प्र॒जाम् । \newline
36. उ॒पो॒त्थायेत्यु॑प - उ॒त्थाय॑ । \newline
37. प्र॒जाम् प॒शून् प॒शून् प्र॒जाम् प्र॒जाम् प॒शून् । \newline
38. प्र॒जामिति॑ प्र - जाम् । \newline
39. प॒शून्. यज॑मानस्य॒ यज॑मानस्य प॒शून् प॒शून्. यज॑मानस्य । \newline
40. यज॑मान स्या॒भ्य॑भि यज॑मानस्य॒ यज॑मान स्या॒भि । \newline
41. अ॒भि म॑न्येत मन्ये ता॒भ्य॑भि म॑न्येत । \newline
42. म॒न्ये॒त॒ यद् यन् म॑न्येत मन्येत॒ यत् । \newline
43. यथ् सञ्चि॑ते॒ सञ्चि॑ते॒ यद् यथ् सञ्चि॑ते । \newline
44. सञ्चि॑ते॒ ऽग्ना व॒ग्नौ सञ्चि॑ते॒ सञ्चि॑ते॒ ऽग्नौ । \newline
45. सञ्चि॑त॒ इति॒ सं - चि॒ते॒ । \newline
46. अ॒ग्ना वे॒ता न्ये॒ता न्य॒ग्ना व॒ग्ना वे॒तानि॑ । \newline
47. ए॒तानि॑ ह॒वीꣳषि॑ ह॒वीꣳ ष्ये॒ता न्ये॒तानि॑ ह॒वीꣳषि॑ । \newline
48. ह॒वीꣳषि॑ नि॒र्वप॑ति नि॒र्वप॑ति ह॒वीꣳषि॑ ह॒वीꣳषि॑ नि॒र्वप॑ति । \newline
49. नि॒र्वप॑ति भाग॒धेये॑न भाग॒धेये॑न नि॒र्वप॑ति नि॒र्वप॑ति भाग॒धेये॑न । \newline
50. नि॒र्वप॒तीति॑ निः - वप॑ति । \newline
51. भा॒ग॒धेये॑ नै॒वैव भा॑ग॒धेये॑न भाग॒धेये॑ नै॒व । \newline
52. भा॒ग॒धेये॒नेति॑ भाग - धेये॑न । \newline
53. ए॒वैन॑ मेन मे॒वै वैन᳚म् । \newline
54. ए॒नꣳ॒॒ श॒म॒य॒ति॒ श॒म॒य॒ त्ये॒न॒ मे॒नꣳ॒॒ श॒म॒य॒ति॒ । \newline
55. श॒म॒य॒ति॒ न न श॑मयति शमयति॒ न । \newline
56. नास्या᳚स्य॒ न नास्य॑ । \newline
57. अ॒स्य॒ रु॒द्रो रु॒द्रो᳚ ऽस्यास्य रु॒द्रः । \newline
58. रु॒द्रो ऽशा॒न्तो ऽशा᳚न्तो रु॒द्रो रु॒द्रो ऽशा᳚न्तः । \newline
59. अशा᳚न्त उपो॒त्था यो॑पो॒त्थाया शा॒न्तो ऽशा᳚न्त उपो॒त्थाय॑ । \newline

\textbf{Ghana Paata } \newline

1. यो वै वै यो यो वा अ॒ग्ना व॒ग्नौ वै यो यो वा अ॒ग्नौ । \newline
2. वा अ॒ग्ना व॒ग्नौ वै वा अ॒ग्ना व॒ग्नि र॒ग्नि र॒ग्नौ वै वा अ॒ग्ना व॒ग्निः । \newline
3. अ॒ग्ना व॒ग्नि र॒ग्नि र॒ग्ना व॒ग्ना व॒ग्निः प्र॑ह्रि॒यते᳚ प्रह्रि॒यते॒ ऽग्नि र॒ग्ना व॒ग्ना व॒ग्निः प्र॑ह्रि॒यते᳚ । \newline
4. अ॒ग्निः प्र॑ह्रि॒यते᳚ प्रह्रि॒यते॒ ऽग्नि र॒ग्निः प्र॑ह्रि॒यते॒ यो यः प्र॑ह्रि॒यते॒ ऽग्नि र॒ग्निः प्र॑ह्रि॒यते॒ यः । \newline
5. प्र॒ह्रि॒यते॒ यो यः प्र॑ह्रि॒यते᳚ प्रह्रि॒यते॒ य श्च॑ च॒ यः प्र॑ह्रि॒यते᳚ प्रह्रि॒यते॒ य श्च॑ । \newline
6. प्र॒ह्रि॒यत॒ इति॑ प्र - ह्रि॒यते᳚ । \newline
7. य श्च॑ च॒ यो य श्च॒ सोमः॒ सोम॑ श्च॒ यो य श्च॒ सोमः॑ । \newline
8. च॒ सोमः॒ सोम॑ श्च च॒ सोमो॒ राजा॒ राजा॒ सोम॑ श्च च॒ सोमो॒ राजा᳚ । \newline
9. सोमो॒ राजा॒ राजा॒ सोमः॒ सोमो॒ राजा॒ तयो॒ स्तयो॒ राजा॒ सोमः॒ सोमो॒ राजा॒ तयोः᳚ । \newline
10. राजा॒ तयो॒ स्तयो॒ राजा॒ राजा॒ तयो॑ रे॒ष ए॒ष तयो॒ राजा॒ राजा॒ तयो॑ रे॒षः । \newline
11. तयो॑ रे॒ष ए॒ष तयो॒ स्तयो॑ रे॒ष आ॑ति॒थ्य मा॑ति॒थ्य मे॒ष तयो॒ स्तयो॑ रे॒ष आ॑ति॒थ्यम् । \newline
12. ए॒ष आ॑ति॒थ्य मा॑ति॒थ्य मे॒ष ए॒ष आ॑ति॒थ्यं ॅयद् यदा॑ति॒थ्य मे॒ष ए॒ष आ॑ति॒थ्यं ॅयत् । \newline
13. आ॒ति॒थ्यं ॅयद् यदा॑ति॒थ्य मा॑ति॒थ्यं ॅयद॑ग्नीषो॒मीयो᳚ ऽग्नीषो॒मीयो॒ यदा॑ति॒थ्य मा॑ति॒थ्यं ॅयद॑ग्नीषो॒मीयः॑ । \newline
14. यद॑ग्नीषो॒मीयो᳚ ऽग्नीषो॒मीयो॒ यद् यद॑ग्नीषो॒मीयो ऽथाथा᳚ ग्नीषो॒मीयो॒ यद् यद॑ग्नीषो॒मीयो ऽथ॑ । \newline
15. अ॒ग्नी॒षो॒मीयो ऽथाथा᳚ ग्नीषो॒मीयो᳚ ऽग्नीषो॒मीयो ऽथै॒ष ए॒षो ऽथा᳚ग्नीषो॒मीयो᳚ ऽग्नीषो॒मीयो ऽथै॒षः । \newline
16. अ॒ग्नी॒षो॒मीय॒ इत्य॑ग्नी - सो॒मीयः॑ । \newline
17. अथै॒ष ए॒षो ऽथा थै॒ष रु॒द्रो रु॒द्र ए॒षो ऽथा थै॒ष रु॒द्रः । \newline
18. ए॒ष रु॒द्रो रु॒द्र ए॒ष ए॒ष रु॒द्रो यो यो रु॒द्र ए॒ष ए॒ष रु॒द्रो यः । \newline
19. रु॒द्रो यो यो रु॒द्रो रु॒द्रो य श्ची॒यते॑ ची॒यते॒ यो रु॒द्रो रु॒द्रो य श्ची॒यते᳚ । \newline
20. य श्ची॒यते॑ ची॒यते॒ यो य श्ची॒यते॒ यद् यच् ची॒यते॒ यो य श्ची॒यते॒ यत् । \newline
21. ची॒यते॒ यद् यच् ची॒यते॑ ची॒यते॒ यथ् सञ्चि॑ते॒ सञ्चि॑ते॒ यच् ची॒यते॑ ची॒यते॒ यथ् सञ्चि॑ते । \newline
22. यथ् सञ्चि॑ते॒ सञ्चि॑ते॒ यद् यथ् सञ्चि॑ते॒ ऽग्ना व॒ग्नौ सञ्चि॑ते॒ यद् यथ् सञ्चि॑ते॒ ऽग्नौ । \newline
23. सञ्चि॑ते॒ ऽग्ना व॒ग्नौ सञ्चि॑ते॒ सञ्चि॑ते॒ ऽग्ना वे॒ता न्ये॒ता न्य॒ग्नौ सञ्चि॑ते॒ सञ्चि॑ते॒ ऽग्ना वे॒तानि॑ । \newline
24. सञ्चि॑त॒ इति॒ सं - चि॒ते॒ । \newline
25. अ॒ग्ना वे॒ता न्ये॒ता न्य॒ग्ना व॒ग्ना वे॒तानि॑ ह॒वीꣳषि॑ ह॒वीꣳ ष्ये॒ता न्य॒ग्ना व॒ग्ना वे॒तानि॑ ह॒वीꣳषि॑ । \newline
26. ए॒तानि॑ ह॒वीꣳषि॑ ह॒वीꣳ ष्ये॒ता न्ये॒तानि॑ ह॒वीꣳषि॒ न न ह॒वीꣳ ष्ये॒ता न्ये॒तानि॑ ह॒वीꣳषि॒ न । \newline
27. ह॒वीꣳषि॒ न न ह॒वीꣳषि॑ ह॒वीꣳषि॒ न नि॒र्वपे᳚न् नि॒र्वपे॒न् न ह॒वीꣳषि॑ ह॒वीꣳषि॒ न नि॒र्वपे᳚त् । \newline
28. न नि॒र्वपे᳚न् नि॒र्वपे॒न् न न नि॒र्वपे॑ दे॒ष ए॒ष नि॒र्वपे॒न् न न नि॒र्वपे॑ दे॒षः । \newline
29. नि॒र्वपे॑ दे॒ष ए॒ष नि॒र्वपे᳚न् नि॒र्वपे॑ दे॒ष ए॒वैवैष नि॒र्वपे᳚न् नि॒र्वपे॑ दे॒ष ए॒व । \newline
30. नि॒र्वपे॒दिति॑ निः - वपे᳚त् । \newline
31. ए॒ष ए॒वैवैष ए॒ष ए॒व रु॒द्रो रु॒द्र ए॒वैष ए॒ष ए॒व रु॒द्रः । \newline
32. ए॒व रु॒द्रो रु॒द्र ए॒वैव रु॒द्रो ऽशा॒न्तो ऽशा᳚न्तो रु॒द्र ए॒वैव रु॒द्रो ऽशा᳚न्तः । \newline
33. रु॒द्रो ऽशा॒न्तो ऽशा᳚न्तो रु॒द्रो रु॒द्रो ऽशा᳚न्त उपो॒त्था यो॑पो॒त्थाया शा᳚न्तो रु॒द्रो रु॒द्रो ऽशा᳚न्त उपो॒त्थाय॑ । \newline
34. अशा᳚न्त उपो॒त्था यो॑पो॒त्थाया शा॒न्तो ऽशा᳚न्त उपो॒त्थाय॑ प्र॒जाम् प्र॒जा मु॑पो॒त्थाया शा॒न्तो ऽशा᳚न्त उपो॒त्थाय॑ प्र॒जाम् । \newline
35. उ॒पो॒त्थाय॑ प्र॒जाम् प्र॒जा मु॑पो॒त्था यो॑पो॒त्थाय॑ प्र॒जाम् प॒शून् प॒शून् प्र॒जा मु॑पो॒त्था यो॑पो॒त्थाय॑ प्र॒जाम् प॒शून् । \newline
36. उ॒पो॒त्थायेत्यु॑प - उ॒त्थाय॑ । \newline
37. प्र॒जाम् प॒शून् प॒शून् प्र॒जाम् प्र॒जाम् प॒शून्. यज॑मानस्य॒ यज॑मानस्य प॒शून् प्र॒जाम् प्र॒जाम् प॒शून्. यज॑मानस्य । \newline
38. प्र॒जामिति॑ प्र - जाम् । \newline
39. प॒शून्. यज॑मानस्य॒ यज॑मानस्य प॒शून् प॒शून्. यज॑मानस्या॒ भ्य॑भि यज॑मानस्य प॒शून् प॒शून्. 
यज॑मानस्या॒भि । \newline
40. यज॑मानस्या॒ भ्य॑भि यज॑मानस्य॒ यज॑मानस्या॒भि म॑न्येत मन्येता॒भि यज॑मानस्य॒ यज॑मानस्या॒भि म॑न्येत । \newline
41. अ॒भि म॑न्येत मन्येता॒ भ्य॑भि म॑न्येत॒ यद् यन् म॑न्येता॒ भ्य॑भि म॑न्येत॒ यत् । \newline
42. म॒न्ये॒त॒ यद् यन् म॑न्येत मन्येत॒ यथ् सञ्चि॑ते॒ सञ्चि॑ते॒ यन् म॑न्येत मन्येत॒ यथ् सञ्चि॑ते । \newline
43. यथ् सञ्चि॑ते॒ सञ्चि॑ते॒ यद् यथ् सञ्चि॑ते॒ ऽग्ना व॒ग्नौ सञ्चि॑ते॒ यद् यथ् सञ्चि॑ते॒ ऽग्नौ । \newline
44. सञ्चि॑ते॒ ऽग्ना व॒ग्नौ सञ्चि॑ते॒ सञ्चि॑ते॒ ऽग्ना वे॒ता न्ये॒ता न्य॒ग्नौ सञ्चि॑ते॒ सञ्चि॑ते॒ ऽग्ना वे॒तानि॑ । \newline
45. सञ्चि॑त॒ इति॒ सं - चि॒ते॒ । \newline
46. अ॒ग्ना वे॒ता न्ये॒ता न्य॒ग्ना व॒ग्ना वे॒तानि॑ ह॒वीꣳषि॑ ह॒वीꣳ ष्ये॒ता न्य॒ग्ना व॒ग्ना वे॒तानि॑ ह॒वीꣳषि॑ । \newline
47. ए॒तानि॑ ह॒वीꣳषि॑ ह॒वीꣳ ष्ये॒ता न्ये॒तानि॑ ह॒वीꣳषि॑ नि॒र्वप॑ति नि॒र्वप॑ति ह॒वीꣳ ष्ये॒ता न्ये॒तानि॑ ह॒वीꣳषि॑ नि॒र्वप॑ति । \newline
48. ह॒वीꣳषि॑ नि॒र्वप॑ति नि॒र्वप॑ति ह॒वीꣳषि॑ ह॒वीꣳषि॑ नि॒र्वप॑ति भाग॒धेये॑न भाग॒धेये॑न नि॒र्वप॑ति ह॒वीꣳषि॑ ह॒वीꣳषि॑ नि॒र्वप॑ति भाग॒धेये॑न । \newline
49. नि॒र्वप॑ति भाग॒धेये॑न भाग॒धेये॑न नि॒र्वप॑ति नि॒र्वप॑ति भाग॒धेये॑ नै॒वैव भा॑ग॒धेये॑न नि॒र्वप॑ति नि॒र्वप॑ति भाग॒धेये॑ नै॒व । \newline
50. नि॒र्वप॒तीति॑ निः - वप॑ति । \newline
51. भा॒ग॒धेये॑ नै॒वैव भा॑ग॒धेये॑न भाग॒धेये॑ नै॒वैन॑ मेन मे॒व भा॑ग॒धेये॑न भाग॒धेये॑ नै॒वैन᳚म् । \newline
52. भा॒ग॒धेये॒नेति॑ भाग - धेये॑न । \newline
53. ए॒वैन॑ मेन मे॒वै वैनꣳ॑ शमयति शमय त्येन मे॒वै वैनꣳ॑ शमयति । \newline
54. ए॒नꣳ॒॒ श॒म॒य॒ति॒ श॒म॒य॒ त्ये॒न॒ मे॒नꣳ॒॒ श॒म॒य॒ति॒ न न श॑मय त्येन मेनꣳ शमयति॒ न । \newline
55. श॒म॒य॒ति॒ न न श॑मयति शमयति॒ नास्या᳚स्य॒ न श॑मयति शमयति॒ नास्य॑ । \newline
56. नास्या᳚स्य॒ न नास्य॑ रु॒द्रो रु॒द्रो᳚ ऽस्य॒ न नास्य॑ रु॒द्रः । \newline
57. अ॒स्य॒ रु॒द्रो रु॒द्रो᳚ ऽस्यास्य रु॒द्रो ऽशा॒न्तो ऽशा᳚न्तो रु॒द्रो᳚ ऽस्यास्य रु॒द्रो ऽशा᳚न्तः । \newline
58. रु॒द्रो ऽशा॒न्तो ऽशा᳚न्तो रु॒द्रो रु॒द्रो ऽशा᳚न्त उपो॒त्था यो॑पो॒त्थाया शा᳚न्तो रु॒द्रो रु॒द्रो ऽशा᳚न्त उपो॒त्थाय॑ । \newline
59. अशा᳚न्त उपो॒त्था यो॑पो॒त्थाया शा॒न्तो ऽशा᳚न्त उपो॒त्थाय॑ प्र॒जाम् प्र॒जा मु॑पो॒त्थाया शा॒न्तो ऽशा᳚न्त उपो॒त्थाय॑ प्र॒जाम् । \newline
\pagebreak
\markright{ TS 7.5.15.2  \hfill https://www.vedavms.in \hfill}

\section{ TS 7.5.15.2 }

\textbf{TS 7.5.15.2 } \newline
\textbf{Samhita Paata} \newline

उपो॒त्थाय॑ प्र॒जां प॒शून॒भि म॑न्यते॒ दश॑ ह॒वीꣳषि॑ भवन्ति॒ नव॒ वै पुरु॑षे प्रा॒णा नाभि॑र्दश॒मी प्रा॒णाने॒व यज॑माने दधा॒त्यथो॒ दशा᳚क्षरा वि॒राडन्नं॑ ॅवि॒राड् वि॒राज्ये॒वान्नाद्ये॒ प्रति॑ तिष्ठत्यृ॒तुभि॒र्वा ए॒ष छन्दो॑भिः॒ स्तोमैः᳚ पृ॒ष्ठैश्चे॑त॒व्य॑ इत्या॑हु॒र्यदे॒तानि॑ ह॒वीꣳषि॑ नि॒र्वप॑त्यृ॒तुभि॑रे॒वैनं॒ छन्दो॑भिः॒ स्तोमैः᳚ पृ॒ष्ठैश्चि॑नुते॒ दिशः॑ सुषुवा॒णेना॑ - [  ] \newline

\textbf{Pada Paata} \newline

उ॒पो॒त्थायेत्यु॑प - उ॒त्थाय॑ । प्र॒जामिति॑ प्र-जाम् । प॒शून् । अ॒भीति॑ । म॒न्य॒ते॒ । दश॑ । ह॒वीꣳषि॑ । भ॒व॒न्ति॒ । नव॑ । वै । पुरु॑षे । प्रा॒णा इति॑ प्र - अ॒नाः । नाभिः॑ । द॒श॒मी । प्रा॒णानिति॑ प्र - अ॒नान् । ए॒व । यज॑माने । द॒धा॒ति॒ । अथो॒ इति॑ । दशा᳚क्ष॒रेति॒ दश॑ - अ॒क्ष॒रा॒ । वि॒राडिति॑ वि - राट् । अन्न᳚म् । वि॒राडिति॑ वि - राट् । वि॒राजीति॑ वि - राजि॑ । ए॒व । अ॒न्नाद्य॒ इत्य॑न्न - अद्ये᳚ । प्रतीति॑ । ति॒ष्ठ॒ति॒ । ऋ॒तुभि॒रित्यृ॒तु - भिः॒ । वै । ए॒षः । छन्दो॑भि॒रिति॒ छन्दः॑- भिः॒ । स्तोमैः᳚ । पृ॒ष्ठैः । चे॒त॒व्यः॑ । इति॑ । आ॒हुः॒ । यत् । ए॒तानि॑ । ह॒वीꣳषि॑ । नि॒र्वप॒तीति॑ निः - वप॑ति । ऋ॒तुभि॒रित्यृ॒तु - भिः॒ । ए॒व । ए॒न॒म् । छन्दो॑भि॒रिति॒ छन्दः॑ - भिः॒ । स्तोमैः᳚ । पृ॒ष्ठैः । चि॒नु॒ते॒ । दिशः॑ । सु॒षु॒वा॒णेन॑ ।  \newline


\textbf{Krama Paata} \newline

उ॒पो॒त्थाय॑ प्र॒जाम् । उ॒पो॒त्थायेत्यु॑प - उ॒त्थाय॑ । प्र॒जाम् प॒शून् । प्र॒जामिति॑ प्र - जाम् । प॒शून॒भि । अ॒भि म॑न्यते । म॒न्य॒ते॒ दश॑ । दश॑ ह॒वीꣳषि॑ । ह॒वीꣳषि॑ भवन्ति । भ॒व॒न्ति॒ नव॑ । नव॒ वै । वै पुरु॑षे । पुरु॑षे प्रा॒णाः । प्रा॒णा नाभिः॑ । प्रा॒णा इति॑ प्र - अ॒नाः । नाभि॑र् दश॒मी । द॒श॒मी प्रा॒णान् । प्रा॒णाने॒व । प्रा॒णानिति॑ प्र - अ॒नान् । ए॒व यज॑माने । यज॑माने दधाति । द॒धा॒त्यथो᳚ । अथो॒ दशा᳚क्षरा । अथो॒ इत्यथो᳚ । दशा᳚क्षरा वि॒राट् । दशा᳚क्ष॒रेति॒ दश॑ - अ॒क्ष॒रा॒ । वि॒राडन्न᳚म् । वि॒राडिति॑ वि - राट् । अन्न॑म् ॅवि॒राट् । वि॒राड् वि॒राजि॑ । वि॒राडिति॑ वि - राट् । वि॒राज्ये॒व । वि॒राजीति॑ वि - राजि॑ । ए॒वान्नाद्ये᳚ । अ॒न्नाद्ये॒ प्रति॑ । अ॒न्नाद्य॒ इत्य॑न्न - अद्ये᳚ । प्रति॑ तिष्ठति । ति॒ष्ठ॒त्यृ॒तुभिः॑ । ऋ॒तुभि॒र् वै । ऋ॒तुभि॒रित्यृ॒तु - भिः॒ । वा ए॒षः । ए॒ष छन्दो॑भिः । छन्दो॑भिः॒ स्तोमैः᳚ । छन्दो॑भि॒रिति॒ छन्दः॑ - भिः॒ । स्तोमैः᳚ पृ॒ष्ठैः । पृ॒ष्ठैश्चे॑त॒व्यः॑ । चे॒त॒व्य॑ इति॑ । इत्या॑हुः । आ॒हु॒र्. यत् । यदे॒तानि॑ । ए॒तानि॑ ह॒वीꣳषि॑ । ह॒वीꣳषि॑ नि॒र्वप॑ति । नि॒र्वप॑त्यृ॒तुभिः॑ । नि॒र्वप॒तीति॑ निः - वप॑ति । ऋ॒तुभि॑रे॒व । ऋ॒तुभि॒रित्यृ॒तु - भिः॒ । ए॒वैन᳚म् । ए॒न॒म् छन्दो॑भिः । छन्दो॑भिः॒ स्तोमैः᳚ । छन्दो॑भि॒रिति॒ छन्दः॑ - भिः॒ । स्तोमैः᳚ पृ॒ष्ठैः । पृ॒ष्ठैश्चि॑नुते । चि॒नु॒ते॒ दिशः॑ । दिशः॑ सुषुवा॒णेन॑ ( ) । सु॒षु॒वा॒णेना॑भि॒जित्याः᳚ \newline

\textbf{Jatai Paata} \newline

1. उ॒पो॒त्थाय॑ प्र॒जाम् प्र॒जा मु॑पो॒त्था यो॑पो॒त्थाय॑ प्र॒जाम् । \newline
2. उ॒पो॒त्थायेत्यु॑प - उ॒त्थाय॑ । \newline
3. प्र॒जाम् प॒शून् प॒शून् प्र॒जाम् प्र॒जाम् प॒शून् । \newline
4. प्र॒जामिति॑ प्र - जाम् । \newline
5. प॒शू न॒भ्य॑भि प॒शून् प॒शू न॒भि । \newline
6. अ॒भि म॑न्यते मन्यते॒ ऽभ्य॑भि म॑न्यते । \newline
7. म॒न्य॒ते॒ दश॒ दश॑ मन्यते मन्यते॒ दश॑ । \newline
8. दश॑ ह॒वीꣳषि॑ ह॒वीꣳषि॒ दश॒ दश॑ ह॒वीꣳषि॑ । \newline
9. ह॒वीꣳषि॑ भवन्ति भवन्ति ह॒वीꣳषि॑ ह॒वीꣳषि॑ भवन्ति । \newline
10. भ॒व॒न्ति॒ नव॒ नव॑ भवन्ति भवन्ति॒ नव॑ । \newline
11. नव॒ वै वै नव॒ नव॒ वै । \newline
12. वै पुरु॑षे॒ पुरु॑षे॒ वै वै पुरु॑षे । \newline
13. पुरु॑षे प्रा॒णाः प्रा॒णाः पुरु॑षे॒ पुरु॑षे प्रा॒णाः । \newline
14. प्रा॒णा नाभि॒र् नाभिः॑ प्रा॒णाः प्रा॒णा नाभिः॑ । \newline
15. प्रा॒णा इति॑ प्र - अ॒नाः । \newline
16. नाभि॑र् दश॒मी द॑श॒मी नाभि॒र् नाभि॑र् दश॒मी । \newline
17. द॒श॒मी प्रा॒णान् प्रा॒णान् द॑श॒मी द॑श॒मी प्रा॒णान् । \newline
18. प्रा॒णा ने॒वैव प्रा॒णान् प्रा॒णा ने॒व । \newline
19. प्रा॒णानिति॑ प्र - अ॒नान् । \newline
20. ए॒व यज॑माने॒ यज॑मान ए॒वैव यज॑माने । \newline
21. यज॑माने दधाति दधाति॒ यज॑माने॒ यज॑माने दधाति । \newline
22. द॒धा॒ त्यथो॒ अथो॑ दधाति दधा॒ त्यथो᳚ । \newline
23. अथो॒ दशा᳚क्षरा॒ दशा᳚क्ष॒रा ऽथो॒ अथो॒ दशा᳚क्षरा । \newline
24. अथो॒ इत्यथो᳚ । \newline
25. दशा᳚क्षरा वि॒राड् वि॒राड् दशा᳚क्षरा॒ दशा᳚क्षरा वि॒राट् । \newline
26. दशा᳚क्ष॒रेति॒ दश॑ - अ॒क्ष॒रा॒ । \newline
27. वि॒रा डन्न॒ मन्नं॑ ॅवि॒राड् वि॒रा डन्न᳚म् । \newline
28. वि॒राडिति॑ वि - राट् । \newline
29. अन्नं॑ ॅवि॒राड् वि॒रा डन्न॒ मन्नं॑ ॅवि॒राट् । \newline
30. वि॒राड् वि॒राजि॑ वि॒राजि॑ वि॒राड् वि॒राड् वि॒राजि॑ । \newline
31. वि॒राडिति॑ वि - राट् । \newline
32. वि॒राज्ये॒वैव वि॒राजि॑ वि॒राज्ये॒व । \newline
33. वि॒राजीति॑ वि - राजि॑ । \newline
34. ए॒वान्नाद्ये॒ ऽन्नाद्य॑ ए॒वै वान्नाद्ये᳚ । \newline
35. अ॒न्नाद्ये॒ प्रति॒ प्रत्य॒न्नाद्ये॒ ऽन्नाद्ये॒ प्रति॑ । \newline
36. अ॒न्नाद्य॒ इत्य॑न्न - अद्ये᳚ । \newline
37. प्रति॑ तिष्ठति तिष्ठति॒ प्रति॒ प्रति॑ तिष्ठति । \newline
38. ति॒ष्ठ॒ त्यृ॒तुभिर्॑. ऋ॒तुभि॑ स्तिष्ठति तिष्ठ त्यृ॒तुभिः॑ । \newline
39. ऋ॒तुभि॒र् वै वा ऋ॒तुभिर्॑. ऋ॒तुभि॒र् वै । \newline
40. ऋ॒तुभि॒रित्यृ॒तु - भिः॒ । \newline
41. वा ए॒ष ए॒ष वै वा ए॒षः । \newline
42. ए॒ष छन्दो॑भि॒ श्छन्दो॑भि रे॒ष ए॒ष छन्दो॑भिः । \newline
43. छन्दो॑भिः॒ स्तोमैः॒ स्तोमै॒ श्छन्दो॑भि॒ श्छन्दो॑भिः॒ स्तोमैः᳚ । \newline
44. छन्दो॑भि॒रिति॒ छन्दः॑ - भिः॒ । \newline
45. स्तोमैः᳚ पृ॒ष्ठैः पृ॒ष्ठैः स्तोमैः॒ स्तोमैः᳚ पृ॒ष्ठैः । \newline
46. पृ॒ष्ठै श्चे॑त॒व्य॑ श्चेत॒व्यः॑ पृ॒ष्ठैः पृ॒ष्ठै श्चे॑त॒व्यः॑ । \newline
47. चे॒त॒व्य॑ इतीति॑ चेत॒व्य॑ श्चेत॒व्य॑ इति॑ । \newline
48. इत्या॑हु राहु॒ रिती त्या॑हुः । \newline
49. आ॒हु॒र् यद् यदा॑हु राहु॒र् यत् । \newline
50. यदे॒ता न्ये॒तानि॒ यद् यदे॒तानि॑ । \newline
51. ए॒तानि॑ ह॒वीꣳषि॑ ह॒वीꣳ ष्ये॒ता न्ये॒तानि॑ ह॒वीꣳषि॑ । \newline
52. ह॒वीꣳषि॑ नि॒र्वप॑ति नि॒र्वप॑ति ह॒वीꣳषि॑ ह॒वीꣳषि॑ नि॒र्वप॑ति । \newline
53. नि॒र्वप॑ त्यृ॒तुभिर्॑. ऋ॒तुभि॑र् नि॒र्वप॑ति नि॒र्वप॑ त्यृ॒तुभिः॑ । \newline
54. नि॒र्वप॒तीति॑ निः - वप॑ति । \newline
55. ऋ॒तुभि॑ रे॒वैव र्‌तुभिर्॑. ऋ॒तुभि॑ रे॒व । \newline
56. ऋ॒तुभि॒रित्यृ॒तु - भिः॒ । \newline
57. ए॒वैन॑ मेन मे॒वै वैन᳚म् । \newline
58. ए॒न॒म् छन्दो॑भि॒ श्छन्दो॑भि रेन मेन॒म् छन्दो॑भिः । \newline
59. छन्दो॑भिः॒ स्तोमैः॒ स्तोमै॒ श्छन्दो॑भि॒ श्छन्दो॑भिः॒ स्तोमैः᳚ । \newline
60. छन्दो॑भि॒रिति॒ छन्दः॑ - भिः॒ । \newline
61. स्तोमैः᳚ पृ॒ष्ठैः पृ॒ष्ठैः स्तोमैः॒ स्तोमैः᳚ पृ॒ष्ठैः । \newline
62. पृ॒ष्ठै श्चि॑नुते चिनुते पृ॒ष्ठैः पृ॒ष्ठै श्चि॑नुते । \newline
63. चि॒नु॒ते॒ दिशो॒ दिश॑ श्चिनुते चिनुते॒ दिशः॑ । \newline
64. दिशः॑ सुषुवा॒णेन॑ सुषुवा॒णेन॒ दिशो॒ दिशः॑ सुषुवा॒णेन॑ । \newline
65. सु॒षु॒वा॒णेना॑ भि॒जित्या॑ अभि॒जित्याः᳚ सुषुवा॒णेन॑ सुषुवा॒णेना॑ भि॒जित्याः᳚ । \newline

\textbf{Ghana Paata } \newline

1. उ॒पो॒त्थाय॑ प्र॒जाम् प्र॒जा मु॑पो॒त्था यो॑पो॒त्थाय॑ प्र॒जाम् प॒शून् प॒शून् प्र॒जा मु॑पो॒त्था यो॑पो॒त्थाय॑ प्र॒जाम् प॒शून् । \newline
2. उ॒पो॒त्थायेत्यु॑प - उ॒त्थाय॑ । \newline
3. प्र॒जाम् प॒शून् प॒शून् प्र॒जाम् प्र॒जाम् प॒शू न॒भ्य॑भि प॒शून् प्र॒जाम् प्र॒जाम् प॒शून॒भि । \newline
4. प्र॒जामिति॑ प्र - जाम् । \newline
5. प॒शू न॒भ्य॑भि प॒शून् प॒शून॒भि म॑न्यते मन्यते॒ ऽभि प॒शून् प॒शून॒भि म॑न्यते । \newline
6. अ॒भि म॑न्यते मन्यते॒ ऽभ्य॑भि म॑न्यते॒ दश॒ दश॑ मन्यते॒ ऽभ्य॑भि म॑न्यते॒ दश॑ । \newline
7. म॒न्य॒ते॒ दश॒ दश॑ मन्यते मन्यते॒ दश॑ ह॒वीꣳषि॑ ह॒वीꣳषि॒ दश॑ मन्यते मन्यते॒ दश॑ ह॒वीꣳषि॑ । \newline
8. दश॑ ह॒वीꣳषि॑ ह॒वीꣳषि॒ दश॒ दश॑ ह॒वीꣳषि॑ भवन्ति भवन्ति ह॒वीꣳषि॒ दश॒ दश॑ ह॒वीꣳषि॑ भवन्ति । \newline
9. ह॒वीꣳषि॑ भवन्ति भवन्ति ह॒वीꣳषि॑ ह॒वीꣳषि॑ भवन्ति॒ नव॒ नव॑ भवन्ति ह॒वीꣳषि॑ ह॒वीꣳषि॑ भवन्ति॒ नव॑ । \newline
10. भ॒व॒न्ति॒ नव॒ नव॑ भवन्ति भवन्ति॒ नव॒ वै वै नव॑ भवन्ति भवन्ति॒ नव॒ वै । \newline
11. नव॒ वै वै नव॒ नव॒ वै पुरु॑षे॒ पुरु॑षे॒ वै नव॒ नव॒ वै पुरु॑षे । \newline
12. वै पुरु॑षे॒ पुरु॑षे॒ वै वै पुरु॑षे प्रा॒णाः प्रा॒णाः पुरु॑षे॒ वै वै पुरु॑षे प्रा॒णाः । \newline
13. पुरु॑षे प्रा॒णाः प्रा॒णाः पुरु॑षे॒ पुरु॑षे प्रा॒णा नाभि॒र् नाभिः॑ प्रा॒णाः पुरु॑षे॒ पुरु॑षे प्रा॒णा नाभिः॑ । \newline
14. प्रा॒णा नाभि॒र् नाभिः॑ प्रा॒णाः प्रा॒णा नाभि॑र् दश॒मी द॑श॒मी नाभिः॑ प्रा॒णाः प्रा॒णा नाभि॑र् दश॒मी । \newline
15. प्रा॒णा इति॑ प्र - अ॒नाः । \newline
16. नाभि॑र् दश॒मी द॑श॒मी नाभि॒र् नाभि॑र् दश॒मी प्रा॒णान् प्रा॒णान् द॑श॒मी नाभि॒र् नाभि॑र् दश॒मी प्रा॒णान् । \newline
17. द॒श॒मी प्रा॒णान् प्रा॒णान् द॑श॒मी द॑श॒मी प्रा॒णा ने॒वैव प्रा॒णान् द॑श॒मी द॑श॒मी प्रा॒णाने॒व । \newline
18. प्रा॒णा ने॒वैव प्रा॒णान् प्रा॒णाने॒व यज॑माने॒ यज॑मान ए॒व प्रा॒णान् प्रा॒णाने॒व यज॑माने । \newline
19. प्रा॒णानिति॑ प्र - अ॒नान् । \newline
20. ए॒व यज॑माने॒ यज॑मान ए॒वैव यज॑माने दधाति दधाति॒ यज॑मान ए॒वैव यज॑माने दधाति । \newline
21. यज॑माने दधाति दधाति॒ यज॑माने॒ यज॑माने दधा॒ त्यथो॒ अथो॑ दधाति॒ यज॑माने॒ यज॑माने दधा॒ त्यथो᳚ । \newline
22. द॒धा॒ त्यथो॒ अथो॑ दधाति दधा॒ त्यथो॒ दशा᳚क्षरा॒ दशा᳚क्ष॒रा ऽथो॑ दधाति दधा॒ त्यथो॒ दशा᳚क्षरा । \newline
23. अथो॒ दशा᳚क्षरा॒ दशा᳚क्ष॒रा ऽथो॒ अथो॒ दशा᳚क्षरा वि॒राड् वि॒राड् दशा᳚क्ष॒रा ऽथो॒ अथो॒ दशा᳚क्षरा वि॒राट् । \newline
24. अथो॒ इत्यथो᳚ । \newline
25. दशा᳚क्षरा वि॒राड् वि॒राड् दशा᳚क्षरा॒ दशा᳚क्षरा वि॒रा डन्न॒ मन्नं॑ ॅवि॒राड् दशा᳚क्षरा॒ दशा᳚क्षरा वि॒रा डन्न᳚म् । \newline
26. दशा᳚क्ष॒रेति॒ दश॑ - अ॒क्ष॒रा॒ । \newline
27. वि॒रा डन्न॒ मन्नं॑ ॅवि॒राड् वि॒रा डन्नं॑ ॅवि॒राड् वि॒रा डन्नं॑ ॅवि॒राड् वि॒रा डन्नं॑ ॅवि॒राट् । \newline
28. वि॒राडिति॑ वि - राट् । \newline
29. अन्नं॑ ॅवि॒राड् वि॒रा डन्न॒ मन्नं॑ ॅवि॒राड् वि॒राजि॑ वि॒राजि॑ वि॒रा डन्न॒ मन्नं॑ ॅवि॒राड् वि॒राजि॑ । \newline
30. वि॒राड् वि॒राजि॑ वि॒राजि॑ वि॒राड् वि॒राड् वि॒रा ज्ये॒वैव वि॒राजि॑ वि॒राड् वि॒राड् वि॒रा ज्ये॒व । \newline
31. वि॒राडिति॑ वि - राट् । \newline
32. वि॒रा ज्ये॒वैव वि॒राजि॑ वि॒राज्ये॒ वान्नाद्ये॒ ऽन्नाद्य॑ ए॒व वि॒राजि॑ वि॒रा ज्ये॒वान्नाद्ये᳚ । \newline
33. वि॒राजीति॑ वि - राजि॑ । \newline
34. ए॒वान्नाद्ये॒ ऽन्नाद्य॑ ए॒वै वान्नाद्ये॒ प्रति॒ प्रत्य॒न्नाद्य॑ ए॒वै वान्नाद्ये॒ प्रति॑ । \newline
35. अ॒न्नाद्ये॒ प्रति॒ प्रत्य॒न्नाद्ये॒ ऽन्नाद्ये॒ प्रति॑ तिष्ठति तिष्ठति॒ प्रत्य॒न्नाद्ये॒ ऽन्नाद्ये॒ प्रति॑ तिष्ठति । \newline
36. अ॒न्नाद्य॒ इत्य॑न्न - अद्ये᳚ । \newline
37. प्रति॑ तिष्ठति तिष्ठति॒ प्रति॒ प्रति॑ तिष्ठ त्यृ॒तुभिर्॑. ऋ॒तुभि॑ स्तिष्ठति॒ प्रति॒ प्रति॑ तिष्ठ त्यृ॒तुभिः॑ । \newline
38. ति॒ष्ठ॒ त्यृ॒तुभिर्॑. ऋ॒तुभि॑ स्तिष्ठति तिष्ठ त्यृ॒तुभि॒र् वै वा ऋ॒तुभि॑ स्तिष्ठति तिष्ठ त्यृ॒तुभि॒र् वै । \newline
39. ऋ॒तुभि॒र् वै वा ऋ॒तुभिर्॑. ऋ॒तुभि॒र् वा ए॒ष ए॒ष वा ऋ॒तुभिर्॑. ऋ॒तुभि॒र् वा ए॒षः । \newline
40. ऋ॒तुभि॒रित्यृ॒तु - भिः॒ । \newline
41. वा ए॒ष ए॒ष वै वा ए॒ष छन्दो॑भि॒ श्छन्दो॑भि रे॒ष वै वा ए॒ष छन्दो॑भिः । \newline
42. ए॒ष छन्दो॑भि॒ श्छन्दो॑भि रे॒ष ए॒ष छन्दो॑भिः॒ स्तोमैः॒ स्तोमै॒ श्छन्दो॑भि रे॒ष ए॒ष छन्दो॑भिः॒ स्तोमैः᳚ । \newline
43. छन्दो॑भिः॒ स्तोमैः॒ स्तोमै॒ श्छन्दो॑भि॒ श्छन्दो॑भिः॒ स्तोमैः᳚ पृ॒ष्ठैः पृ॒ष्ठैः स्तोमै॒ श्छन्दो॑भि॒ श्छन्दो॑भिः॒ स्तोमैः᳚ पृ॒ष्ठैः । \newline
44. छन्दो॑भि॒रिति॒ छन्दः॑ - भिः॒ । \newline
45. स्तोमैः᳚ पृ॒ष्ठैः पृ॒ष्ठैः स्तोमैः॒ स्तोमैः᳚ पृ॒ष्ठै श्चे॑त॒व्य॑ श्चेत॒व्यः॑ पृ॒ष्ठैः स्तोमैः॒ स्तोमैः᳚ पृ॒ष्ठै श्चे॑त॒व्यः॑ । \newline
46. पृ॒ष्ठै श्चे॑त॒व्य॑ श्चेत॒व्यः॑ पृ॒ष्ठैः पृ॒ष्ठै श्चे॑त॒व्य॑ इतीति॑ चेत॒व्यः॑ पृ॒ष्ठैः पृ॒ष्ठै श्चे॑त॒व्य॑ इति॑ । \newline
47. चे॒त॒व्य॑ इतीति॑ चेत॒व्य॑ श्चेत॒व्य॑ इत्या॑हु राहु॒ रिति॑ चेत॒व्य॑ श्चेत॒व्य॑ इत्या॑हुः । \newline
48. इत्या॑हु राहु॒ रिती त्या॑हु॒र् यद् यदा॑हु॒ रिती त्या॑हु॒र् यत् । \newline
49. आ॒हु॒र् यद् यदा॑हु राहु॒र् यदे॒ता न्ये॒तानि॒ यदा॑हु राहु॒र् यदे॒तानि॑ । \newline
50. यदे॒ता न्ये॒तानि॒ यद् यदे॒तानि॑ ह॒वीꣳषि॑ ह॒वीꣳ ष्ये॒तानि॒ यद् यदे॒तानि॑ ह॒वीꣳषि॑ । \newline
51. ए॒तानि॑ ह॒वीꣳषि॑ ह॒वीꣳ ष्ये॒ता न्ये॒तानि॑ ह॒वीꣳषि॑ नि॒र्वप॑ति नि॒र्वप॑ति ह॒वीꣳ ष्ये॒ता न्ये॒तानि॑ ह॒वीꣳषि॑ नि॒र्वप॑ति । \newline
52. ह॒वीꣳषि॑ नि॒र्वप॑ति नि॒र्वप॑ति ह॒वीꣳषि॑ ह॒वीꣳषि॑ नि॒र्वप॑ त्यृ॒तुभिर्॑. ऋ॒तुभि॑र् नि॒र्वप॑ति ह॒वीꣳषि॑ ह॒वीꣳषि॑ नि॒र्वप॑ त्यृ॒तुभिः॑ । \newline
53. नि॒र्वप॑ त्यृ॒तुभिर्॑. ऋ॒तुभि॑र् नि॒र्वप॑ति नि॒र्वप॑ त्यृ॒तुभि॑ रे॒वैव र्‌तुभि॑र् नि॒र्वप॑ति नि॒र्वप॑ त्यृ॒तुभि॑ रे॒व । \newline
54. नि॒र्वप॒तीति॑ निः - वप॑ति । \newline
55. ऋ॒तुभि॑ रे॒वैव र्‌तुभिर्॑. ऋ॒तुभि॑ रे॒वैन॑ मेन मे॒व र्‌तुभिर्॑. ऋ॒तुभि॑ रे॒वैन᳚म् । \newline
56. ऋ॒तुभि॒रित्यृ॒तु - भिः॒ । \newline
57. ए॒वैन॑ मेन मे॒वै वैन॒म् छन्दो॑भि॒ श्छन्दो॑भि रेन मे॒वै वैन॒म् छन्दो॑भिः । \newline
58. ए॒न॒म् छन्दो॑भि॒ श्छन्दो॑भि रेन मेन॒म् छन्दो॑भिः॒ स्तोमैः॒ स्तोमै॒ श्छन्दो॑भि रेन मेन॒म् छन्दो॑भिः॒ स्तोमैः᳚ । \newline
59. छन्दो॑भिः॒ स्तोमैः॒ स्तोमै॒ श्छन्दो॑भि॒ श्छन्दो॑भिः॒ स्तोमैः᳚ पृ॒ष्ठैः पृ॒ष्ठैः स्तोमै॒ श्छन्दो॑भि॒ श्छन्दो॑भिः॒ स्तोमैः᳚ पृ॒ष्ठैः । \newline
60. छन्दो॑भि॒रिति॒ छन्दः॑ - भिः॒ । \newline
61. स्तोमैः᳚ पृ॒ष्ठैः पृ॒ष्ठैः स्तोमैः॒ स्तोमैः᳚ पृ॒ष्ठै श्चि॑नुते चिनुते पृ॒ष्ठैः स्तोमैः॒ स्तोमैः᳚ पृ॒ष्ठै श्चि॑नुते । \newline
62. पृ॒ष्ठै श्चि॑नुते चिनुते पृ॒ष्ठैः पृ॒ष्ठै श्चि॑नुते॒ दिशो॒ दिश॑ श्चिनुते पृ॒ष्ठैः पृ॒ष्ठै श्चि॑नुते॒ दिशः॑ । \newline
63. चि॒नु॒ते॒ दिशो॒ दिश॑ श्चिनुते चिनुते॒ दिशः॑ सुषुवा॒णेन॑ सुषुवा॒णेन॒ दिश॑ श्चिनुते चिनुते॒ दिशः॑ सुषुवा॒णेन॑ । \newline
64. दिशः॑ सुषुवा॒णेन॑ सुषुवा॒णेन॒ दिशो॒ दिशः॑ सुषुवा॒णेना॑ भि॒जित्या॑ अभि॒जित्याः᳚ सुषुवा॒णेन॒ दिशो॒ दिशः॑ सुषुवा॒णेना॑ भि॒जित्याः᳚ । \newline
65. सु॒षु॒वा॒णेना॑ भि॒जित्या॑ अभि॒जित्याः᳚ सुषुवा॒णेन॑ सुषुवा॒णेना॑ भि॒जित्या॒ इती त्य॑भि॒जित्याः᳚ सुषुवा॒णेन॑ सुषुवा॒णेना॑ भि॒जित्या॒ इति॑ । \newline
\pagebreak
\markright{ TS 7.5.15.3  \hfill https://www.vedavms.in \hfill}

\section{ TS 7.5.15.3 }

\textbf{TS 7.5.15.3 } \newline
\textbf{Samhita Paata} \newline

-भि॒जित्या॒ इत्या॑हु॒र्यदे॒तानि॑ ह॒वीꣳषि॑ नि॒र्वप॑ति दि॒शाम॒भिजि॑त्या ए॒तया॒ वा इन्द्रं॑ दे॒वा अ॑याजय॒न् तस्मा॑दिन्द्रस॒व ए॒तया॒ मनुं॑ मनु॒ष्या᳚स्तस्मा᳚न्-मनुस॒वो यथेन्द्रो॑ दे॒वानां॒ ॅयथा॒ मनु॑र्मनु॒ष्या॑णामे॒वं भ॑वति॒ य ए॒वं ॅवि॒द्वाने॒तयेष्ट्या॒ यज॑ते॒ दिग्व॑तीः पुरोऽनुवा॒क्या॑ भवन्ति॒ सर्वा॑सां दि॒शाम॒भिजि॑त्यै ॥ \newline

\textbf{Pada Paata} \newline

अ॒भि॒जित्या॒ इत्य॑भि - जित्याः᳚ । इति॑ । आ॒हुः॒ । यत् । ए॒तानि॑ । ह॒वीꣳषि॑ । नि॒र्वप॒तीति॑ निः - वप॑ति । दि॒शाम् । अ॒भिजि॑त्या॒ इत्य॒भि - जि॒त्यै॒ । ए॒तया᳚ । वै । इन्द्र᳚म् । दे॒वाः । अ॒या॒ज॒य॒न्न् । तस्मा᳚त् । इ॒न्द्र॒स॒व इती᳚न्द्र - स॒वः । ए॒तया᳚ । मनु᳚म् । म॒नु॒ष्याः᳚ । तस्मा᳚त् । म॒नु॒स॒व इति॑ मनु - स॒वः । यथा᳚ । इन्द्रः॑ । दे॒वाना᳚म् । यथा᳚ । मनुः॑ । म॒नु॒ष्या॑णाम् । ए॒वम् । भ॒व॒ति॒ । यः । ए॒वम् । वि॒द्वान् । ए॒तया᳚ । इष्ट्या᳚ । यज॑ते । दिग्व॑ती॒रिति॒ दिक्-व॒तीः॒ । पु॒रो॒ऽनु॒वा॒क्या॑ इति॑ पुरः-अ॒नु॒वा॒क्याः᳚ । भ॒व॒न्ति॒ । सर्वा॑साम् । दि॒शाम् । अ॒भिजि॑त्या॒ इत्य॒भि - जि॒त्यै॒ ॥  \newline


\textbf{Krama Paata} \newline

अ॒भि॒जित्या॒ इति॑ । अ॒भि॒जित्या॒ इत्य॑भि - जित्याः᳚ । इत्या॑हुः । आ॒हु॒र् यत् । यदे॒तानि॑ । ए॒तानि॑ ह॒वीꣳषि॑ । ह॒वीꣳषि॑ नि॒र्वप॑ति । नि॒र्वप॑ति दि॒शाम् । नि॒र्वप॒तीति॑ निः - वप॑ति । दि॒शाम॒भिजि॑त्यै । अ॒भिजि॑त्या ए॒तया᳚ । अ॒भिजि॑त्या॒ इत्य॒भि - जि॒त्यै॒ । ए॒तया॒ वै । वा इन्द्र᳚म् । इन्द्र॑म् दे॒वाः । दे॒वा अ॑याजयन्न् । अ॒या॒ज॒य॒न् तस्मा᳚त् । तस्मा॑दिन्द्रस॒वः । इ॒न्द्र॒स॒व ए॒तया᳚ । इ॒न्द्र॒स॒व इती᳚न्द्र - स॒वः । ए॒तया॒ मनु᳚म् । मनु॑म् मनु॒ष्याः᳚ । म॒नु॒ष्या᳚स्तस्मा᳚त् । तस्मा᳚न् मनुस॒वः । म॒नु॒स॒वो यथा᳚ । म॒नु॒स॒व इति॑ मनु - स॒वः । यथेन्द्रः॑ । इन्द्रो॑ दे॒वाना᳚म् । दे॒वाना॒म् ॅयथा᳚ । यथा॒ मनुः॑ । मनु॑र् मनु॒ष्या॑णाम् । म॒नु॒ष्या॑णामे॒वम् । ए॒वम् भ॑वति । भ॒व॒ति॒ यः । य ए॒वम् । ए॒वम् ॅवि॒द्वान् । वि॒द्वाने॒तया᳚ । ए॒तयेष्ट्‍या᳚ । इष्ट्‍या॒ यज॑ते । यज॑ते॒ दिग्व॑तीः । दिग्व॑तीः पुरोनुवा॒क्याः᳚ । दिग्व॑ती॒रिति॒ दिक् - व॒तीः॒ । पु॒रो॒नु॒वा॒क्या॑ भवन्ति । पु॒रो॒नु॒वा॒क्या॑ इति॑ पुरः - अ॒नु॒वा॒क्याः᳚ । भ॒व॒न्ति॒ सर्वा॑साम् । सर्वा॑साम् दि॒शाम् । दि॒शाम॒भिजि॑त्यै । अ॒भिजि॑त्या॒ इत्य॒भि - जि॒त्यै॒ । \newline

\textbf{Jatai Paata} \newline

1. अ॒भि॒जित्या॒ इती त्य॑भि॒जित्या॑ अभि॒जित्या॒ इति॑ । \newline
2. अ॒भि॒जित्या॒ इत्य॑भि - जित्याः᳚ । \newline
3. इत्या॑हु राहु॒ रिती त्या॑हुः । \newline
4. आ॒हु॒र् यद् यदा॑हु राहु॒र् यत् । \newline
5. यदे॒ता न्ये॒तानि॒ यद् यदे॒तानि॑ । \newline
6. ए॒तानि॑ ह॒वीꣳषि॑ ह॒वीꣳ ष्ये॒ता न्ये॒तानि॑ ह॒वीꣳषि॑ । \newline
7. ह॒वीꣳषि॑ नि॒र्वप॑ति नि॒र्वप॑ति ह॒वीꣳषि॑ ह॒वीꣳषि॑ नि॒र्वप॑ति । \newline
8. नि॒र्वप॑ति दि॒शाम् दि॒शान् नि॒र्वप॑ति नि॒र्वप॑ति दि॒शाम् । \newline
9. नि॒र्वप॒तीति॑ निः - वप॑ति । \newline
10. दि॒शा म॒भिजि॑त्या अ॒भिजि॑त्यै दि॒शाम् दि॒शा म॒भिजि॑त्यै । \newline
11. अ॒भिजि॑त्या ए॒तयै॒तया॒ ऽभिजि॑त्या अ॒भिजि॑त्या ए॒तया᳚ । \newline
12. अ॒भिजि॑त्या॒ इत्य॒भि - जि॒त्यै॒ । \newline
13. ए॒तया॒ वै वा ए॒तयै॒तया॒ वै । \newline
14. वा इन्द्र॒ मिन्द्रं॒ ॅवै वा इन्द्र᳚म् । \newline
15. इन्द्र॑म् दे॒वा दे॒वा इन्द्र॒ मिन्द्र॑म् दे॒वाः । \newline
16. दे॒वा अ॑याजयन् नयाजयन् दे॒वा दे॒वा अ॑याजयन्न् । \newline
17. अ॒या॒ज॒य॒न् तस्मा॒त् तस्मा॑ दयाजयन् नयाजय॒न् तस्मा᳚त् । \newline
18. तस्मा॑ दिन्द्रस॒व इ॑न्द्रस॒व स्तस्मा॒त् तस्मा॑ दिन्द्रस॒वः । \newline
19. इ॒न्द्र॒स॒व ए॒तयै॒ तये᳚न्द्रस॒व इ॑न्द्रस॒व ए॒तया᳚ । \newline
20. इ॒न्द्र॒स॒व इती᳚न्द्र - स॒वः । \newline
21. ए॒तया॒ मनु॒म् मनु॑ मे॒त यै॒तया॒ मनु᳚म् । \newline
22. मनु॑म् मनु॒ष्या॑ मनु॒ष्या॑ मनु॒म् मनु॑म् मनु॒ष्याः᳚ । \newline
23. म॒नु॒ष्या᳚ स्तस्मा॒त् तस्मा᳚न् मनु॒ष्या॑ मनु॒ष्या᳚ स्तस्मा᳚त् । \newline
24. तस्मा᳚न् मनुस॒वो म॑नुस॒व स्तस्मा॒त् तस्मा᳚न् मनुस॒वः । \newline
25. म॒नु॒स॒वो यथा॒ यथा॑ मनुस॒वो म॑नुस॒वो यथा᳚ । \newline
26. म॒नु॒स॒व इति॑ मनु - स॒वः । \newline
27. यथेन्द्र॒ इन्द्रो॒ यथा॒ यथेन्द्रः॑ । \newline
28. इन्द्रो॑ दे॒वाना᳚म् दे॒वाना॒ मिन्द्र॒ इन्द्रो॑ दे॒वाना᳚म् । \newline
29. दे॒वानां॒ ॅयथा॒ यथा॑ दे॒वाना᳚म् दे॒वानां॒ ॅयथा᳚ । \newline
30. यथा॒ मनु॒र् मनु॒र् यथा॒ यथा॒ मनुः॑ । \newline
31. मनु॑र् मनु॒ष्या॑णाम् मनु॒ष्या॑णा॒म् मनु॒र् मनु॑र् मनु॒ष्या॑णाम् । \newline
32. म॒नु॒ष्या॑णा मे॒व मे॒वम् म॑नु॒ष्या॑णाम् मनु॒ष्या॑णा मे॒वम् । \newline
33. ए॒वम् भ॑वति भव त्ये॒व मे॒वम् भ॑वति । \newline
34. भ॒व॒ति॒ यो यो भ॑वति भवति॒ यः । \newline
35. य ए॒व मे॒वं ॅयो य ए॒वम् । \newline
36. ए॒वं ॅवि॒द्वान्. वि॒द्वा ने॒व मे॒वं ॅवि॒द्वान् । \newline
37. वि॒द्वा ने॒त यै॒तया॑ वि॒द्वान्. वि॒द्वा ने॒तया᳚ । \newline
38. ए॒त येष्ट्ये ष्ट्यै॒त यै॒त येष्ट्या᳚ । \newline
39. इष्ट्या॒ यज॑ते॒ यज॑त॒ इष्ट्ये ष्ट्या॒ यज॑ते । \newline
40. यज॑ते॒ दिग्व॑ती॒र् दिग्व॑ती॒र् यज॑ते॒ यज॑ते॒ दिग्व॑तीः । \newline
41. दिग्व॑तीः पुरोऽनुवा॒क्याः᳚ पुरोऽनुवा॒क्या॑ दिग्व॑ती॒र् दिग्व॑तीः पुरोऽनुवा॒क्याः᳚ । \newline
42. दिग्व॑ती॒रिति॒ दिक् - व॒तीः॒ । \newline
43. पु॒रो॒ऽनु॒वा॒क्या॑ भवन्ति भवन्ति पुरोऽनुवा॒क्याः᳚ पुरोऽनुवा॒क्या॑ भवन्ति । \newline
44. पु॒रो॒ऽनु॒वा॒क्या॑ इति॑ पुरः - अ॒नु॒वा॒क्याः᳚ । \newline
45. भ॒व॒न्ति॒ सर्वा॑साꣳ॒॒ सर्वा॑साम् भवन्ति भवन्ति॒ सर्वा॑साम् । \newline
46. सर्वा॑साम् दि॒शाम् दि॒शाꣳ सर्वा॑साꣳ॒॒ सर्वा॑साम् दि॒शाम् । \newline
47. दि॒शा म॒भिजि॑त्या अ॒भिजि॑त्यै दि॒शाम् दि॒शा म॒भिजि॑त्यै । \newline
48. अ॒भिजि॑त्या॒ इत्य॒भि - जि॒त्यै॒ । \newline

\textbf{Ghana Paata } \newline

1. अ॒भि॒जित्या॒ इती त्य॑भि॒जित्या॑ अभि॒जित्या॒ इत्या॑हु राहु॒ रित्य॑भि॒जित्या॑ अभि॒जित्या॒ इत्या॑हुः । \newline
2. अ॒भि॒जित्या॒ इत्य॑भि - जित्याः᳚ । \newline
3. इत्या॑हु राहु॒ रिती त्या॑हु॒र् यद् यदा॑हु॒ रिती त्या॑हु॒र् यत् । \newline
4. आ॒हु॒र् यद् यदा॑हु राहु॒र् यदे॒ता न्ये॒तानि॒ यदा॑हु राहु॒र् यदे॒तानि॑ । \newline
5. यदे॒ता न्ये॒तानि॒ यद् यदे॒तानि॑ ह॒वीꣳषि॑ ह॒वीꣳ ष्ये॒तानि॒ यद् यदे॒तानि॑ ह॒वीꣳषि॑ । \newline
6. ए॒तानि॑ ह॒वीꣳषि॑ ह॒वीꣳ ष्ये॒ता न्ये॒तानि॑ ह॒वीꣳषि॑ नि॒र्वप॑ति नि॒र्वप॑ति ह॒वीꣳ ष्ये॒ता न्ये॒तानि॑ ह॒वीꣳषि॑ नि॒र्वप॑ति । \newline
7. ह॒वीꣳषि॑ नि॒र्वप॑ति नि॒र्वप॑ति ह॒वीꣳषि॑ ह॒वीꣳषि॑ नि॒र्वप॑ति दि॒शाम् दि॒शान् नि॒र्वप॑ति ह॒वीꣳषि॑ ह॒वीꣳषि॑ नि॒र्वप॑ति दि॒शाम् । \newline
8. नि॒र्वप॑ति दि॒शाम् दि॒शान् नि॒र्वप॑ति नि॒र्वप॑ति दि॒शा म॒भिजि॑त्या अ॒भिजि॑त्यै दि॒शान् नि॒र्वप॑ति नि॒र्वप॑ति दि॒शा म॒भिजि॑त्यै । \newline
9. नि॒र्वप॒तीति॑ निः - वप॑ति । \newline
10. दि॒शा म॒भिजि॑त्या अ॒भिजि॑त्यै दि॒शाम् दि॒शा म॒भिजि॑त्या ए॒त यै॒तया॒ ऽभिजि॑त्यै दि॒शाम् दि॒शा म॒भिजि॑त्या ए॒तया᳚ । \newline
11. अ॒भिजि॑त्या ए॒त यै॒तया॒ ऽभिजि॑त्या अ॒भिजि॑त्या ए॒तया॒ वै वा ए॒तया॒ ऽभिजि॑त्या अ॒भिजि॑त्या ए॒तया॒ वै । \newline
12. अ॒भिजि॑त्या॒ इत्य॒भि - जि॒त्यै॒ । \newline
13. ए॒तया॒ वै वा ए॒तयै॒तया॒ वा इन्द्र॒ मिन्द्रं॒ ॅवा ए॒त यै॒तया॒ वा इन्द्र᳚म् । \newline
14. वा इन्द्र॒ मिन्द्रं॒ ॅवै वा इन्द्र॑म् दे॒वा दे॒वा इन्द्रं॒ ॅवै वा इन्द्र॑म् दे॒वाः । \newline
15. इन्द्र॑म् दे॒वा दे॒वा इन्द्र॒ मिन्द्र॑म् दे॒वा अ॑याजयन् नयाजयन् दे॒वा इन्द्र॒ मिन्द्र॑म् दे॒वा अ॑याजयन्न् । \newline
16. दे॒वा अ॑याजयन् नयाजयन् दे॒वा दे॒वा अ॑याजय॒न् तस्मा॒त् तस्मा॑ दयाजयन् दे॒वा दे॒वा अ॑याजय॒न् तस्मा᳚त् । \newline
17. अ॒या॒ज॒य॒न् तस्मा॒त् तस्मा॑ दयाजयन् नयाजय॒न् तस्मा॑ दिन्द्रस॒व इ॑न्द्रस॒व स्तस्मा॑ दयाजयन् नयाजय॒न् तस्मा॑ दिन्द्रस॒वः । \newline
18. तस्मा॑ दिन्द्रस॒व इ॑न्द्रस॒व स्तस्मा॒त् तस्मा॑ दिन्द्रस॒व ए॒त यै॒त ये᳚न्द्रस॒व स्तस्मा॒त् तस्मा॑ दिन्द्रस॒व ए॒तया᳚ । \newline
19. इ॒न्द्र॒स॒व ए॒त यै॒त ये᳚न्द्रस॒व इ॑न्द्रस॒व ए॒तया॒ मनु॒म् मनु॑ मे॒त ये᳚न्द्रस॒व इ॑न्द्रस॒व ए॒तया॒ मनु᳚म् । \newline
20. इ॒न्द्र॒स॒व इती᳚न्द्र - स॒वः । \newline
21. ए॒तया॒ मनु॒म् मनु॑ मे॒त यै॒तया॒ मनु॑म् मनु॒ष्या॑ मनु॒ष्या॑ मनु॑ मे॒त यै॒तया॒ मनु॑म् मनु॒ष्याः᳚ । \newline
22. मनु॑म् मनु॒ष्या॑ मनु॒ष्या॑ मनु॒म् मनु॑म् मनु॒ष्या᳚ स्तस्मा॒त् तस्मा᳚न् मनु॒ष्या॑ मनु॒म् मनु॑म् मनु॒ष्या᳚ स्तस्मा᳚त् । \newline
23. म॒नु॒ष्या᳚ स्तस्मा॒त् तस्मा᳚न् मनु॒ष्या॑ मनु॒ष्या᳚ स्तस्मा᳚न् मनुस॒वो म॑नुस॒व स्तस्मा᳚न् मनु॒ष्या॑ मनु॒ष्या᳚ स्तस्मा᳚न् मनुस॒वः । \newline
24. तस्मा᳚न् मनुस॒वो म॑नुस॒व स्तस्मा॒त् तस्मा᳚न् मनुस॒वो यथा॒ यथा॑ मनुस॒व स्तस्मा॒त् तस्मा᳚न् मनुस॒वो यथा᳚ । \newline
25. म॒नु॒स॒वो यथा॒ यथा॑ मनुस॒वो म॑नुस॒वो यथेन्द्र॒ इन्द्रो॒ यथा॑ मनुस॒वो म॑नुस॒वो यथेन्द्रः॑ । \newline
26. म॒नु॒स॒व इति॑ मनु - स॒वः । \newline
27. यथेन्द्र॒ इन्द्रो॒ यथा॒ यथेन्द्रो॑ दे॒वाना᳚म् दे॒वाना॒ मिन्द्रो॒ यथा॒ यथेन्द्रो॑ दे॒वाना᳚म् । \newline
28. इन्द्रो॑ दे॒वाना᳚म् दे॒वाना॒ मिन्द्र॒ इन्द्रो॑ दे॒वानां॒ ॅयथा॒ यथा॑ दे॒वाना॒ मिन्द्र॒ इन्द्रो॑ दे॒वानां॒ ॅयथा᳚ । \newline
29. दे॒वानां॒ ॅयथा॒ यथा॑ दे॒वाना᳚म् दे॒वानां॒ ॅयथा॒ मनु॒र् मनु॒र् यथा॑ दे॒वाना᳚म् दे॒वानां॒ ॅयथा॒ मनुः॑ । \newline
30. यथा॒ मनु॒र् मनु॒र् यथा॒ यथा॒ मनु॑र् मनु॒ष्या॑णाम् मनु॒ष्या॑णा॒म् मनु॒र् यथा॒ यथा॒ मनु॑र् मनु॒ष्या॑णाम् । \newline
31. मनु॑र् मनु॒ष्या॑णाम् मनु॒ष्या॑णा॒म् मनु॒र् मनु॑र् मनु॒ष्या॑णा मे॒व मे॒वम् म॑नु॒ष्या॑णा॒म् मनु॒र् मनु॑र् मनु॒ष्या॑णा मे॒वम् । \newline
32. म॒नु॒ष्या॑णा मे॒व मे॒वम् म॑नु॒ष्या॑णाम् मनु॒ष्या॑णा मे॒वम् भ॑वति भव त्ये॒वम् म॑नु॒ष्या॑णाम् मनु॒ष्या॑णा मे॒वम् भ॑वति । \newline
33. ए॒वम् भ॑वति भव त्ये॒व मे॒वम् भ॑वति॒ यो यो भ॑व त्ये॒व मे॒वम् भ॑वति॒ यः । \newline
34. भ॒व॒ति॒ यो यो भ॑वति भवति॒ य ए॒व मे॒वं ॅयो भ॑वति भवति॒ य ए॒वम् । \newline
35. य ए॒व मे॒वं ॅयो य ए॒वं ॅवि॒द्वान्. वि॒द्वा ने॒वं ॅयो य ए॒वं ॅवि॒द्वान् । \newline
36. ए॒वं ॅवि॒द्वान्. वि॒द्वा ने॒व मे॒वं ॅवि॒द्वा ने॒त यै॒तया॑ वि॒द्वा ने॒व मे॒वं ॅवि॒द्वा ने॒तया᳚ । \newline
37. वि॒द्वा ने॒त यै॒तया॑ वि॒द्वान्. वि॒द्वा ने॒तयेष्ट्ये ष्ट्यै॒तया॑ वि॒द्वान्. वि॒द्वा ने॒तयेष्ट्या᳚ । \newline
38. ए॒तयेष्ट्ये ष्ट्यै॒त यै॒त येष्ट्या॒ यज॑ते॒ यज॑त॒ इष्ट्यै॒त यै॒त येष्ट्या॒ यज॑ते । \newline
39. इष्ट्या॒ यज॑ते॒ यज॑त॒ इष्ट्येष्ट्या॒ यज॑ते॒ दिग्व॑ती॒र् दिग्व॑ती॒र् यज॑त॒ इष्ट्ये ष्ट्या॒ यज॑ते॒ दिग्व॑तीः । \newline
40. यज॑ते॒ दिग्व॑ती॒र् दिग्व॑ती॒र् यज॑ते॒ यज॑ते॒ दिग्व॑तीः पुरोऽनुवा॒क्याः᳚ पुरोऽनुवा॒क्या॑ दिग्व॑ती॒र् यज॑ते॒ यज॑ते॒ दिग्व॑तीः पुरोऽनुवा॒क्याः᳚ । \newline
41. दिग्व॑तीः पुरोऽनुवा॒क्याः᳚ पुरोऽनुवा॒क्या॑ दिग्व॑ती॒र् दिग्व॑तीः पुरोऽनुवा॒क्या॑ भवन्ति भवन्ति पुरोऽनुवा॒क्या॑ दिग्व॑ती॒र् दिग्व॑तीः पुरोऽनुवा॒क्या॑ भवन्ति । \newline
42. दिग्व॑ती॒रिति॒ दिक् - व॒तीः॒ । \newline
43. पु॒रो॒ऽनु॒वा॒क्या॑ भवन्ति भवन्ति पुरोऽनुवा॒क्याः᳚ पुरोऽनुवा॒क्या॑ भवन्ति॒ सर्वा॑साꣳ॒॒ सर्वा॑साम् भवन्ति पुरोऽनुवा॒क्याः᳚ पुरोऽनुवा॒क्या॑ भवन्ति॒ सर्वा॑साम् । \newline
44. पु॒रो॒ऽनु॒वा॒क्या॑ इति॑ पुरः - अ॒नु॒वा॒क्याः᳚ । \newline
45. भ॒व॒न्ति॒ सर्वा॑साꣳ॒॒ सर्वा॑साम् भवन्ति भवन्ति॒ सर्वा॑साम् दि॒शाम् दि॒शाꣳ सर्वा॑साम् भवन्ति भवन्ति॒ सर्वा॑साम् दि॒शाम् । \newline
46. सर्वा॑साम् दि॒शाम् दि॒शाꣳ सर्वा॑साꣳ॒॒ सर्वा॑साम् दि॒शा म॒भिजि॑त्या अ॒भिजि॑त्यै दि॒शाꣳ सर्वा॑साꣳ॒॒ सर्वा॑साम् दि॒शा म॒भिजि॑त्यै । \newline
47. दि॒शा म॒भिजि॑त्या अ॒भिजि॑त्यै दि॒शाम् दि॒शा म॒भिजि॑त्यै । \newline
48. अ॒भिजि॑त्या॒ इत्य॒भि - जि॒त्यै॒ । \newline
\pagebreak
\markright{ TS 7.5.16.1  \hfill https://www.vedavms.in \hfill}

\section{ TS 7.5.16.1 }

\textbf{TS 7.5.16.1 } \newline
\textbf{Samhita Paata} \newline

यः प्रा॑ण॒तो नि॑मिष॒तो म॑हि॒त्वैक॒ इद्राजा॒ जग॑तो ब॒भूव॑ । य ईशे॑ अ॒स्य द्वि॒पद॒श्चतु॑ष्पदः॒ कस्मै॑ दे॒वाय॑ ह॒विषा॑ विधेम ॥उ॒प॒या॒मगृ॑हीतोऽसि प्र॒जाप॑तये त्वा॒ जुष्टं॑ गृह्णामि॒ तस्य॑ ते॒ द्यौर्म॑हि॒मा नक्ष॑त्राणि रू॒पमा॑दि॒त्यस्ते॒ तेज॒स्तस्मै᳚ त्वा महि॒म्ने प्र॒जाप॑तये॒ स्वाहा᳚ ॥ \newline

\textbf{Pada Paata} \newline

यः । प्रा॒ण॒त इति॑ प्र - अ॒न॒तः । नि॒मि॒ष॒त इति॑ नि - मि॒ष॒तः । म॒हि॒त्वेति॑ महि - त्वा । एकः॑ । इत् । राजा᳚ । जग॑तः । ब॒भूव॑ ॥ यः । ईशे᳚ । अ॒स्य । द्वि॒पद॒ इति॑ द्वि - पदः॑ । चतु॑ष्पद॒ इति॒ चतुः॑ - प॒दः॒ । कस्मै᳚ । दे॒वाय॑ । ह॒विषा᳚ । वि॒धे॒म॒ ॥ उ॒प॒या॒मगृ॑हीत॒ इत्यु॑पया॒म-गृ॒ही॒तः॒ । अ॒सि॒ । प्र॒जाप॑तय॒ इति॑ प्र॒जा - प॒त॒ये॒ । त्वा॒ । जुष्ट᳚म् । गृ॒ह्णा॒मि॒ । तस्य॑ । ते॒ । द्यौः । म॒हि॒मा । नक्ष॑त्राणि । रू॒पम् । आ॒दि॒त्यः । ते॒ । तेजः॑ । तस्मै᳚ । त्वा॒ । म॒हि॒म्ने । प्र॒जाप॑तय॒ इति॑ प्र॒जा - प॒त॒ये॒ । स्वाहा᳚ ॥  \newline


\textbf{Krama Paata} \newline

यः प्रा॑ण॒तः । प्रा॒ण॒तो नि॑मिष॒तः । प्रा॒ण॒त इति॑ प्र - अ॒न॒तः । नि॒मि॒ष॒तो म॑हि॒त्वा । नि॒मि॒ष॒त इति॑ नि - मि॒ष॒तः । म॒हि॒त्वैकः॑ । म॒हि॒त्वेति॑ महि - त्वा । एक॒ इत् । इद् राजा᳚ । राजा॒ जग॑तः । जग॑तो ब॒भूव॑ । ब॒भूवेति॑ ब॒भूव॑ ॥ य ईशे᳚ । ईशे॑ अ॒स्य । अ॒स्य द्वि॒पदः॑ । द्वि॒पद॒श्चतु॑ष्पदः । द्वि॒पद॒ इति॑ द्वि - पदः॑ । चतु॑ष्पदः॒ कस्मै᳚ । चतु॑ष्पद॒ इति॒ चतुः॑ - प॒दः॒ । कस्मै॑ दे॒वाय॑ । दे॒वाय॑ ह॒विषा᳚ । ह॒विषा॑ विधेम । वि॒धे॒मेति॑ विधेम ॥ उ॒प॒या॒मगृ॑हीतोऽसि । उ॒प॒या॒मगृ॑हीत॒ इत्यु॑पया॒म - गृ॒ही॒तः॒ । अ॒सि॒ प्र॒जाप॑तये । प्र॒जाप॑तये त्वा । प्र॒जाप॑तय॒ इति॑ प्र॒जा - प॒त॒ये॒ । त्वा॒ जुष्ट᳚म् । जुष्ट॑म् गृह्णामि । गृ॒ह्णा॒मि॒ तस्य॑ । तस्य॑ ते । ते॒ द्यौः । द्यौर् म॑हि॒मा । म॒हि॒मा नक्ष॑त्राणि । नक्ष॑त्राणि रू॒पम् । रू॒पमा॑दि॒त्यः । आ॒दि॒त्यस्ते᳚ । ते॒ तेजः॑ । तेज॒स्तस्मै᳚ । तस्मै᳚ त्वा । त्वा॒ म॒हि॒म्ने । म॒हि॒म्ने प्र॒जाप॑तये । प्र॒जाप॑तये॒ स्वाहा᳚ । प्र॒जाप॑तय॒ इति॑ प्र॒जा - प॒त॒ये॒ । स्वाहेति॒ स्वाहा᳚ । \newline

\textbf{Jatai Paata} \newline

1. यः प्रा॑ण॒तः प्रा॑ण॒तो यो यः प्रा॑ण॒तः । \newline
2. प्रा॒ण॒तो नि॑मिष॒तो नि॑मिष॒तः प्रा॑ण॒तः प्रा॑ण॒तो नि॑मिष॒तः । \newline
3. प्रा॒ण॒त इति॑ प्र - अ॒न॒तः । \newline
4. नि॒मि॒ष॒तो म॑हि॒त्वा म॑हि॒त्वा नि॑मिष॒तो नि॑मिष॒तो म॑हि॒त्वा । \newline
5. नि॒मि॒ष॒त इति॑ नि - मि॒ष॒तः । \newline
6. म॒हि॒ त्वैक॒ एको॑ महि॒त्वा म॑हि॒ त्वैकः॑ । \newline
7. म॒हि॒त्वेति॑ महि - त्वा । \newline
8. एक॒ इदिदेक॒ एक॒ इत् । \newline
9. इद् राजा॒ राजेदिद् राजा᳚ । \newline
10. राजा॒ जग॑तो॒ जग॑तो॒ राजा॒ राजा॒ जग॑तः । \newline
11. जग॑तो ब॒भूव॑ ब॒भूव॒ जग॑तो॒ जग॑तो ब॒भूव॑ । \newline
12. ब॒भूवेति॑ ब॒भूव॑ । \newline
13. य ईश॒ ईशे॒ यो य ईशे᳚ । \newline
14. ईशे॑ अ॒स्या स्येश॒ ईशे॑ अ॒स्य । \newline
15. अ॒स्य द्वि॒पदो᳚ द्वि॒पदो॑ अ॒स्यास्य द्वि॒पदः॑ । \newline
16. द्वि॒पद॒ श्चतु॑ष्पद॒ श्चतु॑ष्पदो द्वि॒पदो᳚ द्वि॒पद॒ श्चतु॑ष्पदः । \newline
17. द्वि॒पद॒ इति॑ द्वि - पदः॑ । \newline
18. चतु॑ष्पदः॒ कस्मै॒ कस्मै॒ चतु॑ष्पद॒ श्चतु॑ष्पदः॒ कस्मै᳚ । \newline
19. चतु॑ष्पद॒ इति॒ चतुः॑ - प॒दः॒ । \newline
20. कस्मै॑ दे॒वाय॑ दे॒वाय॒ कस्मै॒ कस्मै॑ दे॒वाय॑ । \newline
21. दे॒वाय॑ ह॒विषा॑ ह॒विषा॑ दे॒वाय॑ दे॒वाय॑ ह॒विषा᳚ । \newline
22. ह॒विषा॑ विधेम विधेम ह॒विषा॑ ह॒विषा॑ विधेम । \newline
23. वि॒धे॒मेति॑ विधेम । \newline
24. उ॒प॒या॒मगृ॑हीतो ऽस्य स्युपया॒मगृ॑हीत उपया॒मगृ॑हीतो ऽसि । \newline
25. उ॒प॒या॒मगृ॑हीत॒ इत्यु॑पया॒म - गृ॒ही॒तः॒ । \newline
26. अ॒सि॒ प्र॒जाप॑तये प्र॒जाप॑तये ऽस्यसि प्र॒जाप॑तये । \newline
27. प्र॒जाप॑तये त्वा त्वा प्र॒जाप॑तये प्र॒जाप॑तये त्वा । \newline
28. प्र॒जाप॑तय॒ इति॑ प्र॒जा - प॒त॒ये॒ । \newline
29. त्वा॒ जुष्ट॒म् जुष्ट॑म् त्वा त्वा॒ जुष्ट᳚म् । \newline
30. जुष्ट॑म् गृह्णामि गृह्णामि॒ जुष्ट॒म् जुष्ट॑म् गृह्णामि । \newline
31. गृ॒ह्णा॒मि॒ तस्य॒ तस्य॑ गृह्णामि गृह्णामि॒ तस्य॑ । \newline
32. तस्य॑ ते ते॒ तस्य॒ तस्य॑ ते । \newline
33. ते॒ द्यौर् द्यौ स्ते॑ ते॒ द्यौः । \newline
34. द्यौर् म॑हि॒मा म॑हि॒मा द्यौर् द्यौर् म॑हि॒मा । \newline
35. म॒हि॒मा नक्ष॑त्राणि॒ नक्ष॑त्राणि महि॒मा म॑हि॒मा नक्ष॑त्राणि । \newline
36. नक्ष॑त्राणि रू॒पꣳ रू॒पन् नक्ष॑त्राणि॒ नक्ष॑त्राणि रू॒पम् । \newline
37. रू॒प मा॑दि॒त्य आ॑दि॒त्यो रू॒पꣳ रू॒प मा॑दि॒त्यः । \newline
38. आ॒दि॒त्य स्ते॑ त आदि॒त्य आ॑दि॒त्य स्ते᳚ । \newline
39. ते॒ तेज॒ स्तेज॑ स्ते ते॒ तेजः॑ । \newline
40. तेज॒ स्तस्मै॒ तस्मै॒ तेज॒ स्तेज॒ स्तस्मै᳚ । \newline
41. तस्मै᳚ त्वा त्वा॒ तस्मै॒ तस्मै᳚ त्वा । \newline
42. त्वा॒ म॒हि॒म्ने म॑हि॒म्ने त्वा᳚ त्वा महि॒म्ने । \newline
43. म॒हि॒म्ने प्र॒जाप॑तये प्र॒जाप॑तये महि॒म्ने म॑हि॒म्ने प्र॒जाप॑तये । \newline
44. प्र॒जाप॑तये॒ स्वाहा॒ स्वाहा᳚ प्र॒जाप॑तये प्र॒जाप॑तये॒ स्वाहा᳚ । \newline
45. प्र॒जाप॑तय॒ इति॑ प्र॒जा - प॒त॒ये॒ । \newline
46. स्वाहेति॒ स्वाहा᳚ । \newline

\textbf{Ghana Paata } \newline

1. यः प्रा॑ण॒तः प्रा॑ण॒तो यो यः प्रा॑ण॒तो नि॑मिष॒तो नि॑मिष॒तः प्रा॑ण॒तो यो यः प्रा॑ण॒तो नि॑मिष॒तः । \newline
2. प्रा॒ण॒तो नि॑मिष॒तो नि॑मिष॒तः प्रा॑ण॒तः प्रा॑ण॒तो नि॑मिष॒तो म॑हि॒त्वा म॑हि॒त्वा नि॑मिष॒तः प्रा॑ण॒तः प्रा॑ण॒तो नि॑मिष॒तो म॑हि॒त्वा । \newline
3. प्रा॒ण॒त इति॑ प्र - अ॒न॒तः । \newline
4. नि॒मि॒ष॒तो म॑हि॒त्वा म॑हि॒त्वा नि॑मिष॒तो नि॑मिष॒तो म॑हि॒त्वैक॒ एको॑ महि॒त्वा नि॑मिष॒तो नि॑मिष॒तो 
म॑हि॒त्वैकः॑ । \newline
5. नि॒मि॒ष॒त इति॑ नि - मि॒ष॒तः । \newline
6. म॒हि॒त्वैक॒ एको॑ महि॒त्वा म॑हि॒त्वैक॒ इदिदेको॑ महि॒त्वा म॑हि॒त्वैक॒ इत् । \newline
7. म॒हि॒त्वेति॑ महि - त्वा । \newline
8. एक॒ इदिदेक॒ एक॒ इद् राजा॒ राजे देक॒ एक॒ इद् राजा᳚ । \newline
9. इद् राजा॒ राजेदिद् राजा॒ जग॑तो॒ जग॑तो॒ राजेदिद् राजा॒ जग॑तः । \newline
10. राजा॒ जग॑तो॒ जग॑तो॒ राजा॒ राजा॒ जग॑तो ब॒भूव॑ ब॒भूव॒ जग॑तो॒ राजा॒ राजा॒ जग॑तो ब॒भूव॑ । \newline
11. जग॑तो ब॒भूव॑ ब॒भूव॒ जग॑तो॒ जग॑तो ब॒भूव॑ । \newline
12. ब॒भूवेति॑ ब॒भूव॑ । \newline
13. य ईश॒ ईशे॒ यो य ईशे॑ अ॒स्या स्येशे॒ यो य ईशे॑ अ॒स्य । \newline
14. ईशे॑ अ॒स्यास्येश॒ ईशे॑ अ॒स्य द्वि॒पदो᳚ द्वि॒पदो॑ अ॒स्येश॒ ईशे॑ अ॒स्य द्वि॒पदः॑ । \newline
15. अ॒स्य द्वि॒पदो᳚ द्वि॒पदो॑ अ॒स्यास्य द्वि॒पद॒ श्चतु॑ष्पद॒ श्चतु॑ष्पदो द्वि॒पदो॑ अ॒स्या स्य द्वि॒पद॒ श्चतु॑ष्पदः । \newline
16. द्वि॒पद॒ श्चतु॑ष्पद॒ श्चतु॑ष्पदो द्वि॒पदो᳚ द्वि॒पद॒ श्चतु॑ष्पदः॒ कस्मै॒ कस्मै॒ चतु॑ष्पदो द्वि॒पदो᳚ द्वि॒पद॒ श्चतु॑ष्पदः॒ कस्मै᳚ । \newline
17. द्वि॒पद॒ इति॑ द्वि - पदः॑ । \newline
18. चतु॑ष्पदः॒ कस्मै॒ कस्मै॒ चतु॑ष्पद॒ श्चतु॑ष्पदः॒ कस्मै॑ दे॒वाय॑ दे॒वाय॒ कस्मै॒ चतु॑ष्पद॒ श्चतु॑ष्पदः॒ कस्मै॑ दे॒वाय॑ । \newline
19. चतु॑ष्पद॒ इति॒ चतुः॑ - प॒दः॒ । \newline
20. कस्मै॑ दे॒वाय॑ दे॒वाय॒ कस्मै॒ कस्मै॑ दे॒वाय॑ ह॒विषा॑ ह॒विषा॑ दे॒वाय॒ कस्मै॒ कस्मै॑ दे॒वाय॑ ह॒विषा᳚ । \newline
21. दे॒वाय॑ ह॒विषा॑ ह॒विषा॑ दे॒वाय॑ दे॒वाय॑ ह॒विषा॑ विधेम विधेम ह॒विषा॑ दे॒वाय॑ दे॒वाय॑ ह॒विषा॑ विधेम । \newline
22. ह॒विषा॑ विधेम विधेम ह॒विषा॑ ह॒विषा॑ विधेम । \newline
23. वि॒धे॒मेति॑ विधेम । \newline
24. उ॒प॒या॒मगृ॑हीतो ऽस्य स्युपया॒मगृ॑हीत उपया॒मगृ॑हीतो ऽसि प्र॒जाप॑तये प्र॒जाप॑तये 
ऽस्युपया॒मगृ॑हीत उपया॒मगृ॑हीतो ऽसि प्र॒जाप॑तये । \newline
25. उ॒प॒या॒मगृ॑हीत॒ इत्यु॑पया॒म - गृ॒ही॒तः॒ । \newline
26. अ॒सि॒ प्र॒जाप॑तये प्र॒जाप॑तये ऽस्यसि प्र॒जाप॑तये त्वा त्वा प्र॒जाप॑तये ऽस्यसि प्र॒जाप॑तये त्वा । \newline
27. प्र॒जाप॑तये त्वा त्वा प्र॒जाप॑तये प्र॒जाप॑तये त्वा॒ जुष्ट॒म् जुष्ट॑म् त्वा प्र॒जाप॑तये प्र॒जाप॑तये त्वा॒ जुष्ट᳚म् । \newline
28. प्र॒जाप॑तय॒ इति॑ प्र॒जा - प॒त॒ये॒ । \newline
29. त्वा॒ जुष्ट॒म् जुष्ट॑म् त्वा त्वा॒ जुष्ट॑म् गृह्णामि गृह्णामि॒ जुष्ट॑म् त्वा त्वा॒ जुष्ट॑म् गृह्णामि । \newline
30. जुष्ट॑म् गृह्णामि गृह्णामि॒ जुष्ट॒म् जुष्ट॑म् गृह्णामि॒ तस्य॒ तस्य॑ गृह्णामि॒ जुष्ट॒म् जुष्ट॑म् गृह्णामि॒ तस्य॑ । \newline
31. गृ॒ह्णा॒मि॒ तस्य॒ तस्य॑ गृह्णामि गृह्णामि॒ तस्य॑ ते ते॒ तस्य॑ गृह्णामि गृह्णामि॒ तस्य॑ ते । \newline
32. तस्य॑ ते ते॒ तस्य॒ तस्य॑ ते॒ द्यौर् द्यौ स्ते॒ तस्य॒ तस्य॑ ते॒ द्यौः । \newline
33. ते॒ द्यौर् द्यौ स्ते॑ ते॒ द्यौर् म॑हि॒मा म॑हि॒मा द्यौ स्ते॑ ते॒ द्यौर् म॑हि॒मा । \newline
34. द्यौर् म॑हि॒मा म॑हि॒मा द्यौर् द्यौर् म॑हि॒मा नक्ष॑त्राणि॒ नक्ष॑त्राणि महि॒मा द्यौर् द्यौर् म॑हि॒मा नक्ष॑त्राणि । \newline
35. म॒हि॒मा नक्ष॑त्राणि॒ नक्ष॑त्राणि महि॒मा म॑हि॒मा नक्ष॑त्राणि रू॒पꣳ रू॒पन् नक्ष॑त्राणि महि॒मा म॑हि॒मा नक्ष॑त्राणि रू॒पम् । \newline
36. नक्ष॑त्राणि रू॒पꣳ रू॒पन् नक्ष॑त्राणि॒ नक्ष॑त्राणि रू॒प मा॑दि॒त्य आ॑दि॒त्यो रू॒पन् नक्ष॑त्राणि॒ नक्ष॑त्राणि रू॒प मा॑दि॒त्यः । \newline
37. रू॒प मा॑दि॒त्य आ॑दि॒त्यो रू॒पꣳ रू॒प मा॑दि॒त्य स्ते॑ त आदि॒त्यो रू॒पꣳ रू॒प मा॑दि॒त्य स्ते᳚ । \newline
38. आ॒दि॒त्य स्ते॑ त आदि॒त्य आ॑दि॒त्य स्ते॒ तेज॒ स्तेज॑ स्त आदि॒त्य आ॑दि॒त्य स्ते॒ तेजः॑ । \newline
39. ते॒ तेज॒ स्तेज॑ स्ते ते॒ तेज॒ स्तस्मै॒ तस्मै॒ तेज॑ स्ते ते॒ तेज॒ स्तस्मै᳚ । \newline
40. तेज॒ स्तस्मै॒ तस्मै॒ तेज॒ स्तेज॒ स्तस्मै᳚ त्वा त्वा॒ तस्मै॒ तेज॒ स्तेज॒ स्तस्मै᳚ त्वा । \newline
41. तस्मै᳚ त्वा त्वा॒ तस्मै॒ तस्मै᳚ त्वा महि॒म्ने म॑हि॒म्ने त्वा॒ तस्मै॒ तस्मै᳚ त्वा महि॒म्ने । \newline
42. त्वा॒ म॒हि॒म्ने म॑हि॒म्ने त्वा᳚ त्वा महि॒म्ने प्र॒जाप॑तये प्र॒जाप॑तये महि॒म्ने त्वा᳚ त्वा महि॒म्ने प्र॒जाप॑तये । \newline
43. म॒हि॒म्ने प्र॒जाप॑तये प्र॒जाप॑तये महि॒म्ने म॑हि॒म्ने प्र॒जाप॑तये॒ स्वाहा॒ स्वाहा᳚ प्र॒जाप॑तये महि॒म्ने म॑हि॒म्ने प्र॒जाप॑तये॒ स्वाहा᳚ । \newline
44. प्र॒जाप॑तये॒ स्वाहा॒ स्वाहा᳚ प्र॒जाप॑तये प्र॒जाप॑तये॒ स्वाहा᳚ । \newline
45. प्र॒जाप॑तय॒ इति॑ प्र॒जा - प॒त॒ये॒ । \newline
46. स्वाहेति॒ स्वाहा᳚ । \newline
\pagebreak
\markright{ TS 7.5.17.1  \hfill https://www.vedavms.in \hfill}

\section{ TS 7.5.17.1 }

\textbf{TS 7.5.17.1 } \newline
\textbf{Samhita Paata} \newline

य आ᳚त्म॒दा ब॑ल॒दा यस्य॒ विश्व॑ उ॒पास॑ते प्र॒शिषं॒ ॅयस्य॑ दे॒वाः । यस्य॑ छा॒याऽमृतं॒ ॅयस्य॑ मृ॒त्युः कस्मै॑ दे॒वाय॑ ह॒विषा॑ विधेम ॥उ॒प॒या॒मगृ॑हीतोऽसि प्र॒जाप॑तये त्वा॒ जुष्टं॑ गृह्णामि॒ तस्य॑ ते पृथि॒वी म॑हि॒मौष॑धयो॒ वन॒स्पत॑यो रू॒पम॒ग्निस्ते॒ तेज॒स्तस्मै᳚ त्वा महि॒म्ने प्र॒जाप॑तये॒ स्वाहा᳚ ॥ \newline

\textbf{Pada Paata} \newline

यः । आ॒त्म॒दा इत्या᳚त्म - दाः । ब॒ल॒दा इति॑ बल - दाः । यस्य॑ । विश्वे᳚ । उ॒पास॑त॒ इत्यु॑प - आस॑ते । प्र॒शिष॒मिति॑ प्र-शिष᳚म् । यस्य॑ । दे॒वाः ॥ यस्य॑ । छा॒या । अ॒मृत᳚म् । यस्य॑ । मृ॒त्युः । कस्मै᳚ । दे॒वाय॑ । ह॒विषा᳚ । वि॒धे॒म॒ ॥ उ॒प॒या॒मगृ॑हीत॒ इत्यु॑पया॒म - गृ॒ही॒तः॒ । अ॒सि॒ । प्र॒जाप॑तय॒ इति॑ प्र॒जा - प॒त॒ये॒ । त्वा॒ । जुष्ट᳚म् । गृ॒ह्णा॒मि॒ । तस्य॑ । ते॒ । पृ॒थि॒वी । म॒हि॒मा । ओष॑धयः । वन॒स्पत॑यः । रू॒पम् । अ॒ग्निः । ते॒ । तेजः॑ । तस्मै᳚ । त्वा॒ । म॒हि॒म्ने । प्र॒जाप॑तय॒ इति॑ प्र॒जा - प॒त॒ये॒ । स्वाहा᳚ ॥  \newline


\textbf{Krama Paata} \newline

य आ᳚त्म॒दाः । आ॒त्म॒दा ब॑ल॒दाः । आ॒त्म॒दा इत्या᳚त्म - दाः । ब॒ल॒दा यस्य॑ । ब॒ल॒दा इति॑ बल - दाः । यस्य॒ विश्वे᳚ । विश्व॑ उ॒पास॑ते । उ॒पास॑ते प्र॒शिष᳚म् । उ॒पास॑त॒ इत्यु॑प - आस॑ते । प्र॒शिष॒म् ॅयस्य॑ । प्र॒शिष॒मिति॑ प्र - शिष᳚म् । यस्य॑ दे॒वाः । दे॒वा इति॑ दे॒वाः ॥ यस्य॑ छा॒या । छा॒याऽमृत᳚म् । अ॒मृत॒म् ॅयस्य॑ । यस्य॑ मृ॒त्युः । मृ॒त्युः कस्मै᳚ । कस्मै॑ दे॒वाय॑ । दे॒वाय॑ ह॒विषा᳚ । ह॒विषा॑ विधेम । वि॒धे॒मेति॑ विधेम ॥ उ॒प॒या॒मगृ॑हीतोऽसि । उ॒प॒या॒मगृ॑हीत॒ इत्यु॑पया॒म - गृ॒ही॒तः॒ । अ॒सि॒ प्र॒जाप॑तये । प्र॒जाप॑तये त्वा । प्र॒जाप॑तय॒ इति॑ प्र॒जा - प॒त॒ये॒ । त्वा॒ जुष्ट᳚म् । जुष्ट॑म् गृह्णामि । गृ॒ह्णा॒मि॒ तस्य॑ । तस्य॑ ते । ते॒ पृ॒थि॒वी । पृ॒थि॒वी म॑हि॒मा । म॒हि॒मौष॑धयः । ओष॑धयो॒ वन॒स्पत॑यः । वन॒स्पत॑यो रू॒पम् । रू॒पम॒ग्निः । अ॒ग्निस्ते᳚ । ते॒ तेजः॑ । तेज॒स्तस्मै᳚ । तस्मै᳚ त्वा । त्वा॒ म॒हि॒म्ने । म॒हि॒म्ने प्र॒जाप॑तये । प्र॒जाप॑तये॒ स्वाहा᳚ । प्र॒जाप॑तय॒ इति॑ प्र॒जा - प॒त॒ये॒ । स्वाहेति॒ स्वाहा᳚ । \newline

\textbf{Jatai Paata} \newline

1. य आ᳚त्म॒दा आ᳚त्म॒दा यो य आ᳚त्म॒दाः । \newline
2. आ॒त्म॒दा ब॑ल॒दा ब॑ल॒दा आ᳚त्म॒दा आ᳚त्म॒दा ब॑ल॒दाः । \newline
3. आ॒त्म॒दा इत्या᳚त्म - दाः । \newline
4. ब॒ल॒दा यस्य॒ यस्य॑ बल॒दा ब॑ल॒दा यस्य॑ । \newline
5. ब॒ल॒दा इति॑ बल - दाः । \newline
6. यस्य॒ विश्वे॒ विश्वे॒ यस्य॒ यस्य॒ विश्वे᳚ । \newline
7. विश्व॑ उ॒पास॑त उ॒पास॑ते॒ विश्वे॒ विश्व॑ उ॒पास॑ते । \newline
8. उ॒पास॑ते प्र॒शिष॑म् प्र॒शिष॑ मु॒पास॑त उ॒पास॑ते प्र॒शिष᳚म् । \newline
9. उ॒पास॑त॒ इत्यु॑प - आस॑ते । \newline
10. प्र॒शिषं॒ ॅयस्य॒ यस्य॑ प्र॒शिष॑म् प्र॒शिषं॒ ॅयस्य॑ । \newline
11. प्र॒शिष॒मिति॑ प्र - शिष᳚म् । \newline
12. यस्य॑ दे॒वा दे॒वा यस्य॒ यस्य॑ दे॒वाः । \newline
13. दे॒वा इति॑ दे॒वाः । \newline
14. यस्य॑ छा॒या छा॒या यस्य॒ यस्य॑ छा॒या । \newline
15. छा॒या ऽमृत॑ म॒मृत॑म् छा॒या छा॒या ऽमृत᳚म् । \newline
16. अ॒मृतं॒ ॅयस्य॒ यस्या॒मृत॑ म॒मृतं॒ ॅयस्य॑ । \newline
17. यस्य॑ मृ॒त्युर् मृ॒त्युर् यस्य॒ यस्य॑ मृ॒त्युः । \newline
18. मृ॒त्युः कस्मै॒ कस्मै॑ मृ॒त्युर् मृ॒त्युः कस्मै᳚ । \newline
19. कस्मै॑ दे॒वाय॑ दे॒वाय॒ कस्मै॒ कस्मै॑ दे॒वाय॑ । \newline
20. दे॒वाय॑ ह॒विषा॑ ह॒विषा॑ दे॒वाय॑ दे॒वाय॑ ह॒विषा᳚ । \newline
21. ह॒विषा॑ विधेम विधेम ह॒विषा॑ ह॒विषा॑ विधेम । \newline
22. वि॒धे॒मेति॑ विधेम । \newline
23. उ॒प॒या॒मगृ॑हीतो ऽस्य स्युपया॒मगृ॑हीत उपया॒मगृ॑हीतो ऽसि । \newline
24. उ॒प॒या॒मगृ॑हीत॒ इत्यु॑पया॒म - गृ॒ही॒तः॒ । \newline
25. अ॒सि॒ प्र॒जाप॑तये प्र॒जाप॑तये ऽस्यसि प्र॒जाप॑तये । \newline
26. प्र॒जाप॑तये त्वा त्वा प्र॒जाप॑तये प्र॒जाप॑तये त्वा । \newline
27. प्र॒जाप॑तय॒ इति॑ प्र॒जा - प॒त॒ये॒ । \newline
28. त्वा॒ जुष्ट॒म् जुष्ट॑म् त्वा त्वा॒ जुष्ट᳚म् । \newline
29. जुष्ट॑म् गृह्णामि गृह्णामि॒ जुष्ट॒म् जुष्ट॑म् गृह्णामि । \newline
30. गृ॒ह्णा॒मि॒ तस्य॒ तस्य॑ गृह्णामि गृह्णामि॒ तस्य॑ । \newline
31. तस्य॑ ते ते॒ तस्य॒ तस्य॑ ते । \newline
32. ते॒ पृ॒थि॒वी पृ॑थि॒वी ते॑ ते पृथि॒वी । \newline
33. पृ॒थि॒वी म॑हि॒मा म॑हि॒मा पृ॑थि॒वी पृ॑थि॒वी म॑हि॒मा । \newline
34. म॒हि॒मौष॑धय॒ ओष॑धयो महि॒मा म॑हि॒मौष॑धयः । \newline
35. ओष॑धयो॒ वन॒स्पत॑यो॒ वन॒स्पत॑य॒ ओष॑धय॒ ओष॑धयो॒ वन॒स्पत॑यः । \newline
36. वन॒स्पत॑यो रू॒पꣳ रू॒पं ॅवन॒स्पत॑यो॒ वन॒स्पत॑यो रू॒पम् । \newline
37. रू॒प म॒ग्नि र॒ग्नी रू॒पꣳ रू॒प म॒ग्निः । \newline
38. अ॒ग्नि स्ते॑ ते॒ ऽग्नि र॒ग्नि स्ते᳚ । \newline
39. ते॒ तेज॒ स्तेज॑ स्ते ते॒ तेजः॑ । \newline
40. तेज॒ स्तस्मै॒ तस्मै॒ तेज॒ स्तेज॒ स्तस्मै᳚ । \newline
41. तस्मै᳚ त्वा त्वा॒ तस्मै॒ तस्मै᳚ त्वा । \newline
42. त्वा॒ म॒हि॒म्ने म॑हि॒म्ने त्वा᳚ त्वा महि॒म्ने । \newline
43. म॒हि॒म्ने प्र॒जाप॑तये प्र॒जाप॑तये महि॒म्ने म॑हि॒म्ने प्र॒जाप॑तये । \newline
44. प्र॒जाप॑तये॒ स्वाहा॒ स्वाहा᳚ प्र॒जाप॑तये प्र॒जाप॑तये॒ स्वाहा᳚ । \newline
45. प्र॒जाप॑तय॒ इति॑ प्र॒जा - प॒त॒ये॒ । \newline
46. स्वाहेति॒ स्वाहा᳚ । \newline

\textbf{Ghana Paata } \newline

1. य आ᳚त्म॒दा आ᳚त्म॒दा यो य आ᳚त्म॒दा ब॑ल॒दा ब॑ल॒दा आ᳚त्म॒दा यो य आ᳚त्म॒दा ब॑ल॒दाः । \newline
2. आ॒त्म॒दा ब॑ल॒दा ब॑ल॒दा आ᳚त्म॒दा आ᳚त्म॒दा ब॑ल॒दा यस्य॒ यस्य॑ बल॒दा आ᳚त्म॒दा आ᳚त्म॒दा ब॑ल॒दा यस्य॑ । \newline
3. आ॒त्म॒दा इत्या᳚त्म - दाः । \newline
4. ब॒ल॒दा यस्य॒ यस्य॑ बल॒दा ब॑ल॒दा यस्य॒ विश्वे॒ विश्वे॒ यस्य॑ बल॒दा ब॑ल॒दा यस्य॒ विश्वे᳚ । \newline
5. ब॒ल॒दा इति॑ बल - दाः । \newline
6. यस्य॒ विश्वे॒ विश्वे॒ यस्य॒ यस्य॒ विश्व॑ उ॒पास॑त उ॒पास॑ते॒ विश्वे॒ यस्य॒ यस्य॒ विश्व॑ उ॒पास॑ते । \newline
7. विश्व॑ उ॒पास॑त उ॒पास॑ते॒ विश्वे॒ विश्व॑ उ॒पास॑ते प्र॒शिष॑म् प्र॒शिष॑ मु॒पास॑ते॒ विश्वे॒ विश्व॑ उ॒पास॑ते प्र॒शिष᳚म् । \newline
8. उ॒पास॑ते प्र॒शिष॑म् प्र॒शिष॑ मु॒पास॑त उ॒पास॑ते प्र॒शिषं॒ ॅयस्य॒ यस्य॑ प्र॒शिष॑ मु॒पास॑त उ॒पास॑ते प्र॒शिषं॒ ॅयस्य॑ । \newline
9. उ॒पास॑त॒ इत्यु॑प - आस॑ते । \newline
10. प्र॒शिषं॒ ॅयस्य॒ यस्य॑ प्र॒शिष॑म् प्र॒शिषं॒ ॅयस्य॑ दे॒वा दे॒वा यस्य॑ प्र॒शिष॑म् प्र॒शिषं॒ ॅयस्य॑ दे॒वाः । \newline
11. प्र॒शिष॒मिति॑ प्र - शिष᳚म् । \newline
12. यस्य॑ दे॒वा दे॒वा यस्य॒ यस्य॑ दे॒वाः । \newline
13. दे॒वा इति॑ दे॒वाः । \newline
14. यस्य॑ छा॒या छा॒या यस्य॒ यस्य॑ छा॒या ऽमृत॑ म॒मृत॑म् छा॒या यस्य॒ यस्य॑ छा॒या ऽमृत᳚म् । \newline
15. छा॒या ऽमृत॑ म॒मृत॑म् छा॒या छा॒या ऽमृतं॒ ॅयस्य॒ यस्या॒ मृत॑म् छा॒या छा॒या ऽमृतं॒ ॅयस्य॑ । \newline
16. अ॒मृतं॒ ॅयस्य॒ यस्या॒ मृत॑ म॒मृतं॒ ॅयस्य॑ मृ॒त्युर् मृ॒त्युर् यस्या॒ मृत॑ म॒मृतं॒ ॅयस्य॑ मृ॒त्युः । \newline
17. यस्य॑ मृ॒त्युर् मृ॒त्युर् यस्य॒ यस्य॑ मृ॒त्युः कस्मै॒ कस्मै॑ मृ॒त्युर् यस्य॒ यस्य॑ मृ॒त्युः कस्मै᳚ । \newline
18. मृ॒त्युः कस्मै॒ कस्मै॑ मृ॒त्युर् मृ॒त्युः कस्मै॑ दे॒वाय॑ दे॒वाय॒ कस्मै॑ मृ॒त्युर् मृ॒त्युः कस्मै॑ दे॒वाय॑ । \newline
19. कस्मै॑ दे॒वाय॑ दे॒वाय॒ कस्मै॒ कस्मै॑ दे॒वाय॑ ह॒विषा॑ ह॒विषा॑ दे॒वाय॒ कस्मै॒ कस्मै॑ दे॒वाय॑ ह॒विषा᳚ । \newline
20. दे॒वाय॑ ह॒विषा॑ ह॒विषा॑ दे॒वाय॑ दे॒वाय॑ ह॒विषा॑ विधेम विधेम ह॒विषा॑ दे॒वाय॑ दे॒वाय॑ ह॒विषा॑ विधेम । \newline
21. ह॒विषा॑ विधेम विधेम ह॒विषा॑ ह॒विषा॑ विधेम । \newline
22. वि॒धे॒मेति॑ विधेम । \newline
23. उ॒प॒या॒मगृ॑हीतो ऽस्य स्युपया॒मगृ॑हीत उपया॒मगृ॑हीतो ऽसि प्र॒जाप॑तये प्र॒जाप॑तये ऽस्युपया॒मगृ॑हीत उपया॒मगृ॑हीतो ऽसि प्र॒जाप॑तये । \newline
24. उ॒प॒या॒मगृ॑हीत॒ इत्यु॑पया॒म - गृ॒ही॒तः॒ । \newline
25. अ॒सि॒ प्र॒जाप॑तये प्र॒जाप॑तये ऽस्यसि प्र॒जाप॑तये त्वा त्वा प्र॒जाप॑तये ऽस्यसि प्र॒जाप॑तये त्वा । \newline
26. प्र॒जाप॑तये त्वा त्वा प्र॒जाप॑तये प्र॒जाप॑तये त्वा॒ जुष्ट॒म् जुष्ट॑म् त्वा प्र॒जाप॑तये प्र॒जाप॑तये त्वा॒ जुष्ट᳚म् । \newline
27. प्र॒जाप॑तय॒ इति॑ प्र॒जा - प॒त॒ये॒ । \newline
28. त्वा॒ जुष्ट॒म् जुष्ट॑म् त्वा त्वा॒ जुष्ट॑म् गृह्णामि गृह्णामि॒ जुष्ट॑म् त्वा त्वा॒ जुष्ट॑म् गृह्णामि । \newline
29. जुष्ट॑म् गृह्णामि गृह्णामि॒ जुष्ट॒म् जुष्ट॑म् गृह्णामि॒ तस्य॒ तस्य॑ गृह्णामि॒ जुष्ट॒म् जुष्ट॑म् गृह्णामि॒ तस्य॑ । \newline
30. गृ॒ह्णा॒मि॒ तस्य॒ तस्य॑ गृह्णामि गृह्णामि॒ तस्य॑ ते ते॒ तस्य॑ गृह्णामि गृह्णामि॒ तस्य॑ ते । \newline
31. तस्य॑ ते ते॒ तस्य॒ तस्य॑ ते पृथि॒वी पृ॑थि॒वी ते॒ तस्य॒ तस्य॑ ते पृथि॒वी । \newline
32. ते॒ पृ॒थि॒वी पृ॑थि॒वी ते॑ ते पृथि॒वी म॑हि॒मा म॑हि॒मा पृ॑थि॒वी ते॑ ते पृथि॒वी म॑हि॒मा । \newline
33. पृ॒थि॒वी म॑हि॒मा म॑हि॒मा पृ॑थि॒वी पृ॑थि॒वी म॑हि॒ मौष॑धय॒ ओष॑धयो महि॒मा पृ॑थि॒वी पृ॑थि॒वी म॑हि॒ मौष॑धयः । \newline
34. म॒हि॒ मौष॑धय॒ ओष॑धयो महि॒मा म॑हि॒ मौष॑धयो॒ वन॒स्पत॑यो॒ वन॒स्पत॑य॒ ओष॑धयो महि॒मा म॑हि॒ मौष॑धयो॒ वन॒स्पत॑यः । \newline
35. ओष॑धयो॒ वन॒स्पत॑यो॒ वन॒स्पत॑य॒ ओष॑धय॒ ओष॑धयो॒ वन॒स्पत॑यो रू॒पꣳ रू॒पं ॅवन॒स्पत॑य॒ ओष॑धय॒ ओष॑धयो॒ वन॒स्पत॑यो रू॒पम् । \newline
36. वन॒स्पत॑यो रू॒पꣳ रू॒पं ॅवन॒स्पत॑यो॒ वन॒स्पत॑यो रू॒प म॒ग्नि र॒ग्नी रू॒पं ॅवन॒स्पत॑यो॒ वन॒स्पत॑यो रू॒प म॒ग्निः । \newline
37. रू॒प म॒ग्नि र॒ग्नी रू॒पꣳ रू॒प म॒ग्नि स्ते॑ ते॒ ऽग्नी रू॒पꣳ रू॒प म॒ग्नि स्ते᳚ । \newline
38. अ॒ग्नि स्ते॑ ते॒ ऽग्नि र॒ग्नि स्ते॒ तेज॒ स्तेज॑ स्ते॒ ऽग्नि र॒ग्नि स्ते॒ तेजः॑ । \newline
39. ते॒ तेज॒ स्तेज॑ स्ते ते॒ तेज॒ स्तस्मै॒ तस्मै॒ तेज॑ स्ते ते॒ तेज॒ स्तस्मै᳚ । \newline
40. तेज॒ स्तस्मै॒ तस्मै॒ तेज॒ स्तेज॒ स्तस्मै᳚ त्वा त्वा॒ तस्मै॒ तेज॒ स्तेज॒ स्तस्मै᳚ त्वा । \newline
41. तस्मै᳚ त्वा त्वा॒ तस्मै॒ तस्मै᳚ त्वा महि॒म्ने म॑हि॒म्ने त्वा॒ तस्मै॒ तस्मै᳚ त्वा महि॒म्ने । \newline
42. त्वा॒ म॒हि॒म्ने म॑हि॒म्ने त्वा᳚ त्वा महि॒म्ने प्र॒जाप॑तये प्र॒जाप॑तये महि॒म्ने त्वा᳚ त्वा महि॒म्ने प्र॒जाप॑तये । \newline
43. म॒हि॒म्ने प्र॒जाप॑तये प्र॒जाप॑तये महि॒म्ने म॑हि॒म्ने प्र॒जाप॑तये॒ स्वाहा॒ स्वाहा᳚ प्र॒जाप॑तये महि॒म्ने म॑हि॒म्ने प्र॒जाप॑तये॒ स्वाहा᳚ । \newline
44. प्र॒जाप॑तये॒ स्वाहा॒ स्वाहा᳚ प्र॒जाप॑तये प्र॒जाप॑तये॒ स्वाहा᳚ । \newline
45. प्र॒जाप॑तय॒ इति॑ प्र॒जा - प॒त॒ये॒ । \newline
46. स्वाहेति॒ स्वाहा᳚ । \newline
\pagebreak
\markright{ TS 7.5.18.1  \hfill https://www.vedavms.in \hfill}

\section{ TS 7.5.18.1 }

\textbf{TS 7.5.18.1 } \newline
\textbf{Samhita Paata} \newline

आ ब्रह्म॑न् ब्राह्म॒णो ब्र॑ह्मवर्च॒सी जा॑यता॒मा ऽस्मिन् रा॒ष्ट्रे रा॑ज॒न्य॑ इष॒व्यः॑ शूरो॑ महार॒थो जा॑यतां॒ दोग्ध्री॑धे॒नुर्वोढा॑ ऽन॒ड्वाना॒शुः सप्तिः॒ पुर॑न्धि॒र्योषा॑ जि॒ष्णू र॑थे॒ष्ठाः स॒भेयो॒ युवा ऽऽस्य यज॑मानस्य वी॒रो जा॑यतां निका॒मेनि॑कामे नः प॒र्जन्यो॑ वर्.षतु फ॒लिन्यो॑ न॒ ओष॑धयः पच्यन्तां ॅयोगक्षे॒मोनः॑ कल्पतां ॥ \newline

\textbf{Pada Paata} \newline

एति॑ । ब्रह्मन्न्॑ । ब्रा॒ह्म॒णः । ब्र॒ह्म॒व॒र्च॒सीति॑ ब्रह्म-व॒र्च॒सी । जा॒य॒ता॒म् । एति॑ । अ॒स्मिन्न् । रा॒ष्ट्रे । रा॒ज॒न्यः॑ । इ॒ष॒व्यः॑ । शूरः॑ । म॒हा॒र॒थ इति॑ महा - र॒थः । जा॒य॒ता॒म् । दोग्ध्री᳚ । धे॒नुः । वोढा᳚ । अ॒न॒ड्वान् । आ॒शुः । सप्तिः॑ । पुर॑न्धिः । योषा᳚ । जि॒ष्णूः । र॒थे॒ष्ठा इति॑ रथे - स्थाः । स॒भेयः॑ । युवा᳚ । एति॑ । अ॒स्य । यज॑मानस्य । वी॒रः । जा॒य॒ता॒म् । नि॒का॒मेनि॑काम॒ इति॑ निका॒मे - नि॒का॒मे॒ । नः॒ । प॒र्जन्यः॑ । व॒र्.॒ष॒तु॒ । फ॒लिन्यः॑ । नः॒ । ओष॑धयः । प॒च्य॒न्ता॒म् । यो॒ग॒क्षे॒म इति॑ योग - क्षे॒मः । नः॒ । क॒ल्प॒ता॒म् ॥  \newline


\textbf{Krama Paata} \newline

आ ब्रह्मन्न्॑ । ब्रह्म॑न् ब्राह्म॒णः । ब्रा॒ह्म॒णो ब्र॑ह्मवर्च॒सी । ब्र॒ह्म॒व॒र्च॒सी जा॑यताम् । ब्र॒ह्म॒व॒र्च॒सीति॑ ब्रह्म - व॒र्च॒सी । जा॒य॒ता॒मा । आऽस्मिन्न् । अ॒स्मिन् रा॒ष्ट्रे । रा॒ष्ट्रे रा॑ज॒न्यः॑ । रा॒ज॒न्य॑ इष॒व्यः॑ । इ॒ष॒व्यः॑ शूरः॑ । शूरो॑ महार॒थः । म॒हा॒र॒थो जा॑यताम् । म॒हा॒र॒थ इति॑ महा - र॒थः । जा॒य॒ता॒म् दोग्ध्री᳚ । दोग्ध्री॑ धे॒नुः । धे॒नुर् वोढा᳚ । वोढा॑ऽन॒ड्वान् । अ॒न॒ड्वाना॒शुः । आ॒शुः सप्तिः॑ । सप्तिः॒ पुर॑न्धिः । पुर॑न्धि॒र् योषा᳚ । योषा॑ जि॒ष्णुः । जि॒ष्णू र॑थे॒ष्ठाः । र॒थे॒ष्ठाः स॒भेयः॑ । र॒थे॒ष्ठा इति॑ रथे - स्थाः । स॒भेयो॒ युवा᳚ । युवा । आऽस्य । अ॒स्य यज॑मानस्य । यज॑मानस्य वी॒रः । वी॒रो जा॑यताम् । जा॒य॒ता॒म् नि॒का॒मेनि॑कामे । नि॒का॒मेनि॑कामे नः । नि॒का॒मेनि॑काम॒ इति॑ निका॒मे - नि॒का॒मे॒ । नः॒ प॒र्जन्यः॑ । प॒र्जन्यो॑ वर्.षतु । व॒र्॒.ष॒तु॒ फ॒लिन्यः॑ । फ॒लिन्यो॑ नः । न॒ ओष॑धयः । ओष॑धयः पच्यन्ताम् । प॒च्य॒न्ता॒म् ॅयो॒ग॒क्षे॒मः । यो॒ग॒क्षे॒मो नः॑ । यो॒ग॒क्षे॒म इति॑ योग - क्षे॒म । नः॒ क॒ल्प॒ता॒म् । क॒ल्प॒ता॒मिति॑ कल्पताम् । \newline

\textbf{Jatai Paata} \newline

1. आ ब्रह्म॒न् ब्रह्म॒न् ना ब्रह्मन्न्॑ । \newline
2. ब्रह्म॑न् ब्राह्म॒णो ब्रा᳚ह्म॒णो ब्रह्म॒न् ब्रह्म॑न् ब्राह्म॒णः । \newline
3. ब्रा॒ह्म॒णो ब्र॑ह्मवर्च॒सी ब्र॑ह्मवर्च॒सी ब्रा᳚ह्म॒णो ब्रा᳚ह्म॒णो ब्र॑ह्मवर्च॒सी । \newline
4. ब्र॒ह्म॒व॒र्च॒सी जा॑यताम् जायताम् ब्रह्मवर्च॒सी ब्र॑ह्मवर्च॒सी जा॑यताम् । \newline
5. ब्र॒ह्म॒व॒र्च॒सीति॑ ब्रह्म - व॒र्च॒सी । \newline
6. जा॒य॒ता॒ मा जा॑यताम् जायता॒ मा । \newline
7. आ ऽस्मिन् न॒स्मिन् ना ऽस्मिन्न् । \newline
8. अ॒स्मिन् रा॒ष्ट्रे रा॒ष्ट्रे᳚ ऽस्मिन् न॒स्मिन् रा॒ष्ट्रे । \newline
9. रा॒ष्ट्रे रा॑ज॒न्यो॑ राज॒न्यो॑ रा॒ष्ट्रे रा॒ष्ट्रे रा॑ज॒न्यः॑ । \newline
10. रा॒ज॒न्य॑ इष॒व्य॑ इष॒व्यो॑ राज॒न्यो॑ राज॒न्य॑ इष॒व्यः॑ । \newline
11. इ॒ष॒व्यः॑ शूरः॒ शूर॑ इष॒व्य॑ इष॒व्यः॑ शूरः॑ । \newline
12. शूरो॑ महार॒थो म॑हार॒थः शूरः॒ शूरो॑ महार॒थः । \newline
13. म॒हा॒र॒थो जा॑यताम् जायताम् महार॒थो म॑हार॒थो जा॑यताम् । \newline
14. म॒हा॒र॒थ इति॑ महा - र॒थः । \newline
15. जा॒य॒ता॒म् दोग्ध्री॒ दोग्ध्री॑ जायताम् जायता॒म् दोग्ध्री᳚ । \newline
16. दोग्ध्री॑ धे॒नुर् धे॒नुर् दोग्ध्री॒ दोग्ध्री॑ धे॒नुः । \newline
17. धे॒नुर् वोढा॒ वोढा॑ धे॒नुर् धे॒नुर् वोढा᳚ । \newline
18. वोढा॑ ऽन॒ड्वा न॑न॒ड्वान्. वोढा॒ वोढा॑ ऽन॒ड्वान् । \newline
19. अ॒न॒ड्वाना॒ शुरा॒ शुर॑ न॒ड्वान॑ न॒ड्वा ना॒शुः । \newline
20. आ॒शुः सप्तिः॒ सप्ति॑ रा॒शु रा॒शुः सप्तिः॑ । \newline
21. सप्तिः॒ पुर॑न्धिः॒ पुर॑न्धिः॒ सप्तिः॒ सप्तिः॒ पुर॑न्धिः । \newline
22. पुर॑न्धि॒र् योषा॒ योषा॒ पुर॑न्धिः॒ पुर॑न्धि॒र् योषा᳚ । \newline
23. योषा॑ जि॒ष्णुर् जि॒ष्णुर् योषा॒ योषा॑ जि॒ष्णुः । \newline
24. जि॒ष्णू र॑थे॒ष्ठा र॑थे॒ष्ठा जि॒ष्णुर् जि॒ष्णू र॑थे॒ष्ठाः । \newline
25. र॒थे॒ष्ठाः स॒भेयः॑ स॒भेयो॑ रथे॒ष्ठा र॑थे॒ष्ठाः स॒भेयः॑ । \newline
26. र॒थे॒ष्ठा इति॑ रथे - स्थाः । \newline
27. स॒भेयो॒ युवा॒ युवा॑ स॒भेयः॑ स॒भेयो॒ युवा᳚ । \newline
28. युवा ऽऽयुवा॒ युवा । \newline
29. आ ऽस्यास्या ऽस्य । \newline
30. अ॒स्य यज॑मानस्य॒ यज॑मानस्या॒ स्यास्य यज॑मानस्य । \newline
31. यज॑मानस्य वी॒रो वी॒रो यज॑मानस्य॒ यज॑मानस्य वी॒रः । \newline
32. वी॒रो जा॑यताम् जायतां ॅवी॒रो वी॒रो जा॑यताम् । \newline
33. जा॒य॒ता॒न् नि॒का॒मेनि॑कामे निका॒मेनि॑कामे जायताम् जायतान् निका॒मेनि॑कामे । \newline
34. नि॒का॒मेनि॑कामे नो नो निका॒मेनि॑कामे निका॒मेनि॑कामे नः । \newline
35. नि॒का॒मेनि॑काम॒ इति॑ निका॒मे - नि॒का॒मे॒ । \newline
36. नः॒ प॒र्जन्यः॑ प॒र्जन्यो॑ नो नः प॒र्जन्यः॑ । \newline
37. प॒र्जन्यो॑ वर्.षतु वर्.षतु प॒र्जन्यः॑ प॒र्जन्यो॑ वर्.षतु । \newline
38. व॒र्॒.ष॒तु॒ फ॒लिन्यः॑ फ॒लिन्यो॑ वर्.षतु वर्.षतु फ॒लिन्यः॑ । \newline
39. फ॒लिन्यो॑ नो नः फ॒लिन्यः॑ फ॒लिन्यो॑ नः । \newline
40. न॒ ओष॑धय॒ ओष॑धयो नो न॒ ओष॑धयः । \newline
41. ओष॑धयः पच्यन्ताम् पच्यन्ता॒ मोष॑धय॒ ओष॑धयः पच्यन्ताम् । \newline
42. प॒च्य॒न्तां॒ ॅयो॒ग॒क्षे॒मो यो॑गक्षे॒मः प॑च्यन्ताम् पच्यन्तां ॅयोगक्षे॒मः । \newline
43. यो॒ग॒क्षे॒मो नो॑ नो योगक्षे॒मो यो॑गक्षे॒मो नः॑ । \newline
44. यो॒ग॒क्षे॒म इति॑ योग - क्षे॒मः । \newline
45. नः॒ क॒ल्प॒ता॒म् क॒ल्प॒ता॒न्नो॒ नः॒ क॒ल्प॒ता॒म् । \newline
46. क॒ल्प॒ता॒मिति॑ कल्पताम् । \newline

\textbf{Ghana Paata } \newline

1. आ ब्रह्म॒न् ब्रह्म॒न् ना ब्रह्म॑न् ब्राह्म॒णो ब्रा᳚ह्म॒णो ब्रह्म॒न् ना ब्रह्म॑न् ब्राह्म॒णः । \newline
2. ब्रह्म॑न् ब्राह्म॒णो ब्रा᳚ह्म॒णो ब्रह्म॒न् ब्रह्म॑न् ब्राह्म॒णो ब्र॑ह्मवर्च॒सी ब्र॑ह्मवर्च॒सी ब्रा᳚ह्म॒णो ब्रह्म॒न् ब्रह्म॑न् ब्राह्म॒णो ब्र॑ह्मवर्च॒सी । \newline
3. ब्रा॒ह्म॒णो ब्र॑ह्मवर्च॒सी ब्र॑ह्मवर्च॒सी ब्रा᳚ह्म॒णो ब्रा᳚ह्म॒णो ब्र॑ह्मवर्च॒सी जा॑यताम् जायताम् ब्रह्मवर्च॒सी ब्रा᳚ह्म॒णो ब्रा᳚ह्म॒णो ब्र॑ह्मवर्च॒सी जा॑यताम् । \newline
4. ब्र॒ह्म॒व॒र्च॒सी जा॑यताम् जायताम् ब्रह्मवर्च॒सी ब्र॑ह्मवर्च॒सी जा॑यता॒ मा जा॑यताम् ब्रह्मवर्च॒सी ब्र॑ह्मवर्च॒सी जा॑यता॒ मा । \newline
5. ब्र॒ह्म॒व॒र्च॒सीति॑ ब्रह्म - व॒र्च॒सी । \newline
6. जा॒य॒ता॒ मा जा॑यताम् जायता॒ मा ऽस्मिन् न॒स्मिन् ना जा॑यताम् जायता॒ मा ऽस्मिन्न् । \newline
7. आ ऽस्मिन् न॒स्मिन् ना ऽस्मिन् रा॒ष्ट्रे रा॒ष्ट्रे᳚ ऽस्मिन् ना ऽस्मिन् रा॒ष्ट्रे । \newline
8. अ॒स्मिन् रा॒ष्ट्रे रा॒ष्ट्रे᳚ ऽस्मिन् न॒स्मिन् रा॒ष्ट्रे रा॑ज॒न्यो॑ राज॒न्यो॑ रा॒ष्ट्रे᳚ ऽस्मिन् न॒स्मिन् रा॒ष्ट्रे रा॑ज॒न्यः॑ । \newline
9. रा॒ष्ट्रे रा॑ज॒न्यो॑ राज॒न्यो॑ रा॒ष्ट्रे रा॒ष्ट्रे रा॑ज॒न्य॑ इष॒व्य॑ इष॒व्यो॑ राज॒न्यो॑ रा॒ष्ट्रे रा॒ष्ट्रे रा॑ज॒न्य॑ इष॒व्यः॑ । \newline
10. रा॒ज॒न्य॑ इष॒व्य॑ इष॒व्यो॑ राज॒न्यो॑ राज॒न्य॑ इष॒व्यः॑ शूरः॒ शूर॑ इष॒व्यो॑ राज॒न्यो॑ राज॒न्य॑ इष॒व्यः॑ शूरः॑ । \newline
11. इ॒ष॒व्यः॑ शूरः॒ शूर॑ इष॒व्य॑ इष॒व्यः॑ शूरो॑ महार॒थो म॑हार॒थः शूर॑ इष॒व्य॑ इष॒व्यः॑ शूरो॑ महार॒थः । \newline
12. शूरो॑ महार॒थो म॑हार॒थः शूरः॒ शूरो॑ महार॒थो जा॑यताम् जायताम् महार॒थः शूरः॒ शूरो॑ महार॒थो जा॑यताम् । \newline
13. म॒हा॒र॒थो जा॑यताम् जायताम् महार॒थो म॑हार॒थो जा॑यता॒म् दोग्ध्री॒ दोग्ध्री॑ जायताम् महार॒थो म॑हार॒थो जा॑यता॒म् दोग्ध्री᳚ । \newline
14. म॒हा॒र॒थ इति॑ महा - र॒थः । \newline
15. जा॒य॒ता॒म् दोग्ध्री॒ दोग्ध्री॑ जायताम् जायता॒म् दोग्ध्री॑ धे॒नुर् धे॒नुर् दोग्ध्री॑ जायताम् जायता॒म् दोग्ध्री॑ धे॒नुः । \newline
16. दोग्ध्री॑ धे॒नुर् धे॒नुर् दोग्ध्री॒ दोग्ध्री॑ धे॒नुर् वोढा॒ वोढा॑ धे॒नुर् दोग्ध्री॒ दोग्ध्री॑ धे॒नुर् वोढा᳚ । \newline
17. धे॒नुर् वोढा॒ वोढा॑ धे॒नुर् धे॒नुर् वोढा॑ ऽन॒ड्वा न॑न॒ड्वान्. वोढा॑ धे॒नुर् धे॒नुर् वोढा॑ ऽन॒ड्वान् । \newline
18. वोढा॑ ऽन॒ड्वा न॑न॒ड्वान्. वोढा॒ वोढा॑ ऽन॒ड्वा ना॒शु रा॒शु र॑न॒ड्वान्. वोढा॒ वोढा॑ ऽन॒ड्वा ना॒शुः । \newline
19. अ॒न॒ड्वा ना॒शु रा॒शु र॑न॒ड्वा न॑न॒ड्वा ना॒शुः सप्तिः॒ सप्ति॑ रा॒शु र॑न॒ड्वा न॑न॒ड्वा ना॒शुः सप्तिः॑ । \newline
20. आ॒शुः सप्तिः॒ सप्ति॑ रा॒शु रा॒शुः सप्तिः॒ पुर॑न्धिः॒ पुर॑न्धिः॒ सप्ति॑ रा॒शु रा॒शुः सप्तिः॒ पुर॑न्धिः । \newline
21. सप्तिः॒ पुर॑न्धिः॒ पुर॑न्धिः॒ सप्तिः॒ सप्तिः॒ पुर॑न्धि॒र् योषा॒ योषा॒ पुर॑न्धिः॒ सप्तिः॒ सप्तिः॒ पुर॑न्धि॒र् योषा᳚ । \newline
22. पुर॑न्धि॒र् योषा॒ योषा॒ पुर॑न्धिः॒ पुर॑न्धि॒र् योषा॑ जि॒ष्णुर् जि॒ष्णुर् योषा॒ पुर॑न्धिः॒ पुर॑न्धि॒र् योषा॑ जि॒ष्णुः । \newline
23. योषा॑ जि॒ष्णुर् जि॒ष्णुर् योषा॒ योषा॑ जि॒ष्णू र॑थे॒ष्ठा र॑थे॒ष्ठा जि॒ष्णुर् योषा॒ योषा॑ जि॒ष्णू र॑थे॒ष्ठाः । \newline
24. जि॒ष्णू र॑थे॒ष्ठा र॑थे॒ष्ठा जि॒ष्णुर् जि॒ष्णू र॑थे॒ष्ठाः स॒भेयः॑ स॒भेयो॑ रथे॒ष्ठा जि॒ष्णुर् जि॒ष्णू र॑थे॒ष्ठाः स॒भेयः॑ । \newline
25. र॒थे॒ष्ठाः स॒भेयः॑ स॒भेयो॑ रथे॒ष्ठा र॑थे॒ष्ठाः स॒भेयो॒ युवा॒ युवा॑ स॒भेयो॑ रथे॒ष्ठा र॑थे॒ष्ठाः स॒भेयो॒ युवा᳚ । \newline
26. र॒थे॒ष्ठा इति॑ रथे - स्थाः । \newline
27. स॒भेयो॒ युवा॒ युवा॑ स॒भेयः॑ स॒भेयो॒ युवा ऽऽयुवा॑ स॒भेयः॑ स॒भेयो॒ युवा । \newline
28. युवा ऽऽयुवा॒ युवा ऽस्यास्या युवा॒ युवा ऽस्य । \newline
29. आ ऽस्यास्या ऽस्य यज॑मानस्य॒ यज॑मानस्या॒ स्या ऽस्य यज॑मानस्य । \newline
30. अ॒स्य यज॑मानस्य॒ यज॑मानस्या॒ स्यास्य यज॑मानस्य वी॒रो वी॒रो यज॑मानस्या॒ स्यास्य यज॑मानस्य वी॒रः । \newline
31. यज॑मानस्य वी॒रो वी॒रो यज॑मानस्य॒ यज॑मानस्य वी॒रो जा॑यताम् जायतां ॅवी॒रो यज॑मानस्य॒ यज॑मानस्य वी॒रो जा॑यताम् । \newline
32. वी॒रो जा॑यताम् जायतां ॅवी॒रो वी॒रो जा॑यतान् निका॒मेनि॑कामे निका॒मेनि॑कामे जायतां ॅवी॒रो वी॒रो जा॑यतान् निका॒मेनि॑कामे । \newline
33. जा॒य॒ता॒न् नि॒का॒मेनि॑कामे निका॒मेनि॑कामे जायताम् जायतान् निका॒मेनि॑कामे नो नो निका॒मेनि॑कामे जायताम् जायतान् निका॒मेनि॑कामे नः । \newline
34. नि॒का॒मेनि॑कामे नो नो निका॒मेनि॑कामे निका॒मेनि॑कामे नः प॒र्जन्यः॑ प॒र्जन्यो॑ नो निका॒मेनि॑कामे निका॒मेनि॑कामे नः प॒र्जन्यः॑ । \newline
35. नि॒का॒मेनि॑काम॒ इति॑ निका॒मे - नि॒का॒मे॒ । \newline
36. नः॒ प॒र्जन्यः॑ प॒र्जन्यो॑ नो नः प॒र्जन्यो॑ वर्.षतु वर्.षतु प॒र्जन्यो॑ नो नः प॒र्जन्यो॑ वर्.षतु । \newline
37. प॒र्जन्यो॑ वर्.षतु वर्.षतु प॒र्जन्यः॑ प॒र्जन्यो॑ वर्.षतु फ॒लिन्यः॑ फ॒लिन्यो॑ वर्.षतु प॒र्जन्यः॑ प॒र्जन्यो॑ वर्.षतु फ॒लिन्यः॑ । \newline
38. व॒र्॒.ष॒तु॒ फ॒लिन्यः॑ फ॒लिन्यो॑ वर्.षतु वर्.षतु फ॒लिन्यो॑ नो नः फ॒लिन्यो॑ वर्.षतु वर्.षतु फ॒लिन्यो॑ नः । \newline
39. फ॒लिन्यो॑ नो नः फ॒लिन्यः॑ फ॒लिन्यो॑ न॒ ओष॑धय॒ ओष॑धयो नः फ॒लिन्यः॑ फ॒लिन्यो॑ न॒ ओष॑धयः । \newline
40. न॒ ओष॑धय॒ ओष॑धयो नो न॒ ओष॑धयः पच्यन्ताम् पच्यन्ता॒ मोष॑धयो नो न॒ ओष॑धयः पच्यन्ताम् । \newline
41. ओष॑धयः पच्यन्ताम् पच्यन्ता॒ मोष॑धय॒ ओष॑धयः पच्यन्तां ॅयोगक्षे॒मो यो॑गक्षे॒मः प॑च्यन्ता॒ मोष॑धय॒ ओष॑धयः पच्यन्तां ॅयोगक्षे॒मः । \newline
42. प॒च्य॒न्तां॒ ॅयो॒ग॒क्षे॒मो यो॑गक्षे॒मः प॑च्यन्ताम् पच्यन्तां ॅयोगक्षे॒मो नो॑ नो योगक्षे॒मः प॑च्यन्ताम् पच्यन्तां ॅयोगक्षे॒मो नः॑ । \newline
43. यो॒ग॒क्षे॒मो नो॑ नो योगक्षे॒मो यो॑गक्षे॒मो नः॑ कल्पताम् कल्पतान् नो योगक्षे॒मो यो॑गक्षे॒मो नः॑ कल्पताम् । \newline
44. यो॒ग॒क्षे॒म इति॑ योग - क्षे॒मः । \newline
45. नः॒ क॒ल्प॒ता॒म् क॒ल्प॒ता॒न् नो॒ नः॒ क॒ल्प॒ता॒म् । \newline
46. क॒ल्प॒ता॒मिति॑ कल्पताम् । \newline
\pagebreak
\markright{ TS 7.5.19.1  \hfill https://www.vedavms.in \hfill}

\section{ TS 7.5.19.1 }

\textbf{TS 7.5.19.1 } \newline
\textbf{Samhita Paata} \newline

आऽक्रान्॑ वा॒जी पृ॑थि॒वीम॒ग्निं ॅयुज॑मकृत वा॒ज्यर्वा ऽऽक्रान्॑ वा॒ज्य॑न्तरि॑क्षं ॅवा॒युं ॅयुज॑मकृत वा॒ज्यर्वा॒ द्यां ॅवा॒ज्याऽक्रꣳ॑स्त॒ सूर्यं॒ ॅयुज॑मकृत वा॒ज्यर्वा॒ ऽग्निस्ते॑ वाजि॒न्॒ युङ्ङनु॒ त्वा ऽऽ र॑भे स्व॒स्ति मा॒ सं पा॑रय वा॒युस्ते॑ वाजि॒न्॒ युङ्ङनु॒ त्वा ऽऽ र॑भे स्व॒स्ति मा॒ सं - [  ] \newline

\textbf{Pada Paata} \newline

एति॑ । अ॒क्रा॒न् । वा॒जी । पृ॒थि॒वीम् । अ॒ग्निम् । युज᳚म् । अ॒कृ॒त॒ । वा॒जी । अर्वा᳚ । एति॑ । अ॒क्रा॒न् । वा॒जी । अ॒न्तरि॑क्षम् । वा॒युम् । युज᳚म् । अ॒कृ॒त॒ । वा॒जी । अर्वा᳚ । द्याम् । वा॒जी । एति॑ । अ॒क्रꣳ॒॒स्त॒ । सूर्य᳚म् । युज᳚म् । अ॒कृ॒त॒ । वा॒जी । अर्वा᳚ । अ॒ग्निः । ते॒ । वा॒जि॒न्न् । युङ् । अन्विति॑ । त्वा॒ । एति॑ । र॒भे॒ । स्व॒स्ति । मा॒ । समिति॑ । पा॒र॒य॒ । वा॒युः । ते॒ । वा॒जि॒न्न् । युङ् । अन्विति॑ । त्वा॒ । एति॑ । र॒भे॒ । स्व॒स्ति । मा॒ । समिति॑ ।  \newline


\textbf{Krama Paata} \newline

आऽक्रान्॑ । अ॒क्रा॒न्॒. वा॒जी । वा॒जी पृ॑थि॒वीम् । पृ॒थि॒वीम॒ग्निम् । अ॒ग्निम् ॅयुज᳚म् । युज॑मकृत । अ॒कृ॒त॒ वा॒जी । वा॒ज्यर्वा᳚ । अर्वा । आऽक्रान्॑ । अ॒क्रा॒न्॒. वा॒जी । वा॒ज्य॑न्तरि॑क्षम् । अ॒न्तरि॑क्षम् ॅवा॒युम् । वा॒युम् ॅयुज᳚म् । युज॑मकृत । अ॒कृ॒त॒ वा॒जी । वा॒ज्यर्वा᳚ । अर्वा॒ द्याम् । द्याम् ॅवा॒जी । वा॒ज्या । आऽक्रꣳ॑स्त । अ॒क्रꣳ॒॒स्त॒ सूर्य᳚म् । सूर्य॒म् ॅयुज᳚म् । युज॑मकृत । अ॒कृ॒त॒ वा॒जी । वा॒ज्यर्वा᳚ । अर्वा॒ऽग्निः । अ॒ग्निस्ते᳚ । ते॒ वा॒जि॒न्न्॒ । वा॒जि॒न्॒. युङ्‍ङ् । युङ्‍ङनु॑ । अनु॑ त्वा । त्वा । आ र॑भे । र॒भे॒ स्व॒स्ति । स्व॒स्ति मा᳚ । मा॒ सम् । सम् पा॑रय । पा॒र॒य॒ वा॒युः । वा॒युस्ते᳚ । ते॒ वा॒जि॒न्न्॒ । वा॒जि॒न्न्॒. युङ्‍ङ् । युङ्‍ङनु॑ । अनु॑ त्वा । त्वा । आ र॑भे । र॒भे॒ स्व॒स्ति । स्व॒स्ति मा᳚ । मा॒ सम् । सम् पा॑रय \newline

\textbf{Jatai Paata} \newline

1. आ ऽक्रा॑न क्रा॒ना ऽक्रान्॑ । \newline
2. अ॒क्रा॒न्॒. वा॒जी वा॒ज्य॑क्रा नक्रान्. वा॒जी । \newline
3. वा॒जी पृ॑थि॒वीम् पृ॑थि॒वीं ॅवा॒जी वा॒जी पृ॑थि॒वीम् । \newline
4. पृ॒थि॒वी म॒ग्नि म॒ग्निम् पृ॑थि॒वीम् पृ॑थि॒वी म॒ग्निम् । \newline
5. अ॒ग्निं ॅयुजं॒ ॅयुज॑ म॒ग्नि म॒ग्निं ॅयुज᳚म् । \newline
6. युज॑ मकृता कृत॒ युजं॒ ॅयुज॑ मकृत । \newline
7. अ॒कृ॒त॒ वा॒जी वा॒ज्य॑कृता कृत वा॒जी । \newline
8. वा॒ज्यर्वा ऽर्वा॑ वा॒जी वा॒ज्यर्वा᳚ । \newline
9. अर्वा ऽर्वा ऽर्वा । \newline
10. आ ऽक्रा॑ नक्रा॒ ना ऽक्रान्॑ । \newline
11. अ॒क्रा॒न्॒. वा॒जी वा॒ज्य॑क्रा नक्रान्. वा॒जी । \newline
12. वा॒ज्य॑न्तरि॑क्ष म॒न्तरि॑क्षं ॅवा॒जी वा॒ज्य॑न्तरि॑क्षम् । \newline
13. अ॒न्तरि॑क्षं ॅवा॒युं ॅवा॒यु म॒न्तरि॑क्ष म॒न्तरि॑क्षं ॅवा॒युम् । \newline
14. वा॒युं ॅयुजं॒ ॅयुजं॑ ॅवा॒युं ॅवा॒युं ॅयुज᳚म् । \newline
15. युज॑ मकृता कृत॒ युजं॒ ॅयुज॑ मकृत । \newline
16. अ॒कृ॒त॒ वा॒जी वा॒ज्य॑कृता कृत वा॒जी । \newline
17. वा॒ज्यर्वा ऽर्वा॑ वा॒जी वा॒ज्यर्वा᳚ । \newline
18. अर्वा॒ द्याम् द्या मर्वा ऽर्वा॒ द्याम् । \newline
19. द्यां ॅवा॒जी वा॒जी द्याम् द्यां ॅवा॒जी । \newline
20. वा॒ज्या वा॒जी वा॒ज्या । \newline
21. आ ऽक्रꣳ॑स्ता क्रꣳ॒॒स्ता ऽक्रꣳ॑स्त । \newline
22. अ॒क्रꣳ॒॒स्त॒ सूर्यꣳ॒॒ सूर्य॑ मक्रꣳस्ता क्रꣳस्त॒ सूर्य᳚म् । \newline
23. सूर्यं॒ ॅयुजं॒ ॅयुजꣳ॒॒ सूर्यꣳ॒॒ सूर्यं॒ ॅयुज᳚म् । \newline
24. युज॑ मकृता कृत॒ युजं॒ ॅयुज॑ मकृत । \newline
25. अ॒कृ॒त॒ वा॒जी वा॒ज्य॑कृता कृत वा॒जी । \newline
26. वा॒ज्यर्वा ऽर्वा॑ वा॒जी वा॒ज्यर्वा᳚ । \newline
27. अर्वा॒ ऽग्नि र॒ग्नि रर्वा ऽर्वा॒ ऽग्निः । \newline
28. अ॒ग्नि स्ते॑ ते॒ ऽग्नि र॒ग्नि स्ते᳚ । \newline
29. ते॒ वा॒जि॒न्॒. वा॒जि॒न् ते॒ ते॒ वा॒जि॒न्न् । \newline
30. वा॒जि॒न्॒. युङ् युङ् वा॑जिन्. वाजि॒न्॒. युङ् । \newline
31. युङ् ङन्वनु॒ युङ् युङ् ङनु॑ । \newline
32. अनु॑ त्वा॒ त्वा ऽन्वनु॑ त्वा । \newline
33. त्वा ऽऽत्वा॒ त्वा । \newline
34. आ र॑भे रभ॒ आ र॑भे । \newline
35. र॒भे॒ स्व॒स्ति स्व॒स्ति र॑भे रभे स्व॒स्ति । \newline
36. स्व॒स्ति मा॑ मा स्व॒स्ति स्व॒स्ति मा᳚ । \newline
37. मा॒ सꣳ सम् मा॑ मा॒ सम् । \newline
38. सम् पा॑रय पारय॒ सꣳ सम् पा॑रय । \newline
39. पा॒र॒य॒ वा॒युर् वा॒युः पा॑रय पारय वा॒युः । \newline
40. वा॒यु स्ते॑ ते वा॒युर् वा॒यु स्ते᳚ । \newline
41. ते॒ वा॒जि॒न्॒. वा॒जि॒न् ते॒ ते॒ वा॒जि॒न्न् । \newline
42. वा॒जि॒न्॒. युङ् युङ् वा॑जिन्. वाजि॒न्॒. युङ् । \newline
43. युङ् ङन्वनु॒ युङ् युङ् ङनु॑ । \newline
44. अनु॑ त्वा॒ त्वा ऽन्वनु॑ त्वा । \newline
45. त्वा ऽऽत्वा॒ त्वा । \newline
46. आ र॑भे रभ॒ आ र॑भे । \newline
47. र॒भे॒ स्व॒स्ति स्व॒स्ति र॑भे रभे स्व॒स्ति । \newline
48. स्व॒स्ति मा॑ मा स्व॒स्ति स्व॒स्ति मा᳚ । \newline
49. मा॒ सꣳ सम् मा॑ मा॒ सम् । \newline
50. सम् पा॑रय पारय॒ सꣳ सम् पा॑रय । \newline

\textbf{Ghana Paata } \newline

1. आ ऽक्रा॑ नक्रा॒ना ऽक्रान्॑. वा॒जी वा॒ज्य॑क्रा॒ नाऽक्रान्॑. वा॒जी । \newline
2. अ॒क्रा॒न्॒. वा॒जी वा॒ज्य॑क्रा नक्रान्. वा॒जी पृ॑थि॒वीम् पृ॑थि॒वीं ॅवा॒ज्य॑क्रा नक्रान्. वा॒जी पृ॑थि॒वीम् । \newline
3. वा॒जी पृ॑थि॒वीम् पृ॑थि॒वीं ॅवा॒जी वा॒जी पृ॑थि॒वी म॒ग्नि म॒ग्निम् पृ॑थि॒वीं ॅवा॒जी वा॒जी पृ॑थि॒वी म॒ग्निम् । \newline
4. पृ॒थि॒वी म॒ग्नि म॒ग्निम् पृ॑थि॒वीम् पृ॑थि॒वी म॒ग्निं ॅयुजं॒ ॅयुज॑ म॒ग्निम् पृ॑थि॒वीम् पृ॑थि॒वी म॒ग्निं ॅयुज᳚म् । \newline
5. अ॒ग्निं ॅयुजं॒ ॅयुज॑ म॒ग्नि म॒ग्निं ॅयुज॑ मकृता कृत॒ युज॑ म॒ग्नि म॒ग्निं ॅयुज॑ मकृत । \newline
6. युज॑ मकृता कृत॒ युजं॒ ॅयुज॑ मकृत वा॒जी वा॒ज्य॑कृत॒ युजं॒ ॅयुज॑ मकृत वा॒जी । \newline
7. अ॒कृ॒त॒ वा॒जी वा॒ज्य॑कृता कृत वा॒ज्यर्वा ऽर्वा॑ वा॒ज्य॑कृता कृत वा॒ज्यर्वा᳚ । \newline
8. वा॒ज्यर्वा ऽर्वा॑ वा॒जी वा॒ज्यर्वा ऽर्वा॑ वा॒जी वा॒ज्यर्वा । \newline
9. अर्वा ऽर्वा ऽर्वा ऽक्रा॑ नक्रा॒ ना ऽर्वा ऽर्वा ऽक्रान्॑ । \newline
10. आ ऽक्रा॑ नक्रा॒ नाऽक्रान्॑. वा॒जी वा॒ज्य॑क्रा॒ नाऽक्रान्॑. वा॒जी । \newline
11. अ॒क्रा॒न्॒. वा॒जी वा॒ज्य॑क्रा नक्रान्. वा॒ज्य॑न्तरि॑क्ष म॒न्तरि॑क्षं ॅवा॒ज्य॑क्रा नक्रान्. वा॒ज्य॑न्तरि॑क्षम् । \newline
12. वा॒ज्य॑न्तरि॑क्ष म॒न्तरि॑क्षं ॅवा॒जी वा॒ज्य॑न्तरि॑क्षं ॅवा॒युं ॅवा॒यु म॒न्तरि॑क्षं ॅवा॒जी वा॒ज्य॑न्तरि॑क्षं ॅवा॒युम् । \newline
13. अ॒न्तरि॑क्षं ॅवा॒युं ॅवा॒यु म॒न्तरि॑क्ष म॒न्तरि॑क्षं ॅवा॒युं ॅयुजं॒ ॅयुजं॑ ॅवा॒यु म॒न्तरि॑क्ष म॒न्तरि॑क्षं ॅवा॒युं ॅयुज᳚म् । \newline
14. वा॒युं ॅयुजं॒ ॅयुजं॑ ॅवा॒युं ॅवा॒युं ॅयुज॑ मकृता कृत॒ युजं॑ ॅवा॒युं ॅवा॒युं ॅयुज॑ मकृत । \newline
15. युज॑ मकृता कृत॒ युजं॒ ॅयुज॑ मकृत वा॒जी वा॒ज्य॑कृत॒ युजं॒ ॅयुज॑ मकृत वा॒जी । \newline
16. अ॒कृ॒त॒ वा॒जी वा॒ज्य॑कृता कृत वा॒ज्यर्वा ऽर्वा॑ वा॒ज्य॑कृता कृत वा॒ज्यर्वा᳚ । \newline
17. वा॒ज्यर्वा ऽर्वा॑ वा॒जी वा॒ज्यर्वा॒ द्याम् द्या मर्वा॑ वा॒जी वा॒ज्यर्वा॒ द्याम् । \newline
18. अर्वा॒ द्याम् द्या मर्वा ऽर्वा॒ द्यां ॅवा॒जी वा॒जी द्या मर्वा ऽर्वा॒ द्यां ॅवा॒जी । \newline
19. द्यां ॅवा॒जी वा॒जी द्याम् द्यां ॅवा॒ज्या वा॒जी द्याम् द्यां ॅवा॒ज्या । \newline
20. वा॒ज्या वा॒जी वा॒ज्या ऽक्रꣳ॑स्ता क्रꣳ॒॒स्ता वा॒जी वा॒ज्या ऽक्रꣳ॑स्त । \newline
21. आ ऽक्रꣳ॑स्ता क्रꣳ॒॒स्ता ऽक्रꣳ॑स्त॒ सूर्यꣳ॒॒ सूर्य॑ मक्रꣳ॒॒स्ता ऽक्रꣳ॑स्त॒ सूर्य᳚म् । \newline
22. अ॒क्रꣳ॒॒स्त॒ सूर्यꣳ॒॒ सूर्य॑ मक्रꣳस्ता क्रꣳस्त॒ सूर्यं॒ ॅयुजं॒ ॅयुजꣳ॒॒ सूर्य॑ मक्रꣳस्ता क्रꣳस्त॒ सूर्यं॒ ॅयुज᳚म् । \newline
23. सूर्यं॒ ॅयुजं॒ ॅयुजꣳ॒॒ सूर्यꣳ॒॒ सूर्यं॒ ॅयुज॑ मकृता कृत॒ युजꣳ॒॒ सूर्यꣳ॒॒ सूर्यं॒ ॅयुज॑ मकृत । \newline
24. युज॑ मकृता कृत॒ युजं॒ ॅयुज॑ मकृत वा॒जी वा॒ज्य॑ कृत॒ युजं॒ ॅयुज॑ मकृत वा॒जी । \newline
25. अ॒कृ॒त॒ वा॒जी वा॒ज्य॑ कृता कृत वा॒ज्यर्वा ऽर्वा॑ वा॒ज्य॑ कृता कृत वा॒ज्यर्वा᳚ । \newline
26. वा॒ज्यर्वा ऽर्वा॑ वा॒जी वा॒ज्यर्वा॒ ऽग्नि र॒ग्नि रर्वा॑ वा॒जी वा॒ज्यर्वा॒ ऽग्निः । \newline
27. अर्वा॒ ऽग्नि र॒ग्नि रर्वा ऽर्वा॒ ऽग्नि स्ते॑ ते॒ ऽग्नि रर्वा ऽर्वा॒ ऽग्नि स्ते᳚ । \newline
28. अ॒ग्नि स्ते॑ ते॒ ऽग्नि र॒ग्नि स्ते॑ वाजिन्. वाजिन् ते॒ ऽग्नि र॒ग्नि स्ते॑ वाजिन्न् । \newline
29. ते॒ वा॒जि॒न्॒. वा॒जि॒न् ते॒ ते॒ वा॒जि॒न्॒. युङ् युङ् वा॑जिन् ते ते वाजि॒न्॒. युङ् । \newline
30. वा॒जि॒न्॒. युङ् युङ् वा॑जिन्. वाजि॒न्॒. युङ् ङन्वनु॒ युङ् वा॑जिन्. वाजि॒न्॒. युङ् ङनु॑ । \newline
31. युङ् ङन्वनु॒ युङ् युङ् ङनु॑ त्वा॒ त्वा ऽनु॒ युङ् युङ् ङनु॑ त्वा । \newline
32. अनु॑ त्वा॒ त्वा ऽन्वनु॒ त्वा ऽऽत्वा ऽन्वनु॒ त्वा । \newline
33. त्वा ऽऽत्वा॒ त्वा ऽऽर॑भे रभ॒ आ त्वा॒ त्वा ऽऽर॑भे । \newline
34. आ र॑भे रभ॒ आ र॑भे स्व॒स्ति स्व॒स्ति र॑भ॒ आ र॑भे स्व॒स्ति । \newline
35. र॒भे॒ स्व॒स्ति स्व॒स्ति र॑भे रभे स्व॒स्ति मा॑ मा स्व॒स्ति र॑भे रभे स्व॒स्ति मा᳚ । \newline
36. स्व॒स्ति मा॑ मा स्व॒स्ति स्व॒स्ति मा॒ सꣳ सम् मा᳚ स्व॒स्ति स्व॒स्ति मा॒ सम् । \newline
37. मा॒ सꣳ सम् मा॑ मा॒ सम् पा॑रय पारय॒ सम् मा॑ मा॒ सम् पा॑रय । \newline
38. सम् पा॑रय पारय॒ सꣳ सम् पा॑रय वा॒युर् वा॒युः पा॑रय॒ सꣳ सम् पा॑रय वा॒युः । \newline
39. पा॒र॒य॒ वा॒युर् वा॒युः पा॑रय पारय वा॒यु स्ते॑ ते वा॒युः पा॑रय पारय वा॒यु स्ते᳚ । \newline
40. वा॒यु स्ते॑ ते वा॒युर् वा॒यु स्ते॑ वाजिन्. वाजिन् ते वा॒युर् वा॒यु स्ते॑ वाजिन्न् । \newline
41. ते॒ वा॒जि॒न्॒. वा॒जि॒न् ते॒ ते॒ वा॒जि॒न्॒. युङ् युङ् वा॑जिन् ते ते वाजि॒न्॒. युङ् । \newline
42. वा॒जि॒न्॒. युङ् युङ् वा॑जिन्. वाजि॒न्॒. युङ् ङन्वनु॒ युङ् वा॑जिन्. वाजि॒न्॒. युङ् ङनु॑ । \newline
43. युङ् ङन्वनु॒ युङ् युङ् ङनु॑ त्वा॒ त्वा ऽनु॒ युङ् युङ् ङनु॑ त्वा । \newline
44. अनु॑ त्वा॒ त्वा ऽन्वनु॒ त्वा ऽऽत्वा ऽन्वनु॒ त्वा । \newline
45. त्वा ऽऽत्वा॒ त्वा ऽऽर॑भे रभ॒ आ त्वा॒ त्वा ऽऽर॑भे । \newline
46. आ र॑भे रभ॒ आ र॑भे स्व॒स्ति स्व॒स्ति र॑भ॒ आ र॑भे स्व॒स्ति । \newline
47. र॒भे॒ स्व॒स्ति स्व॒स्ति र॑भे रभे स्व॒स्ति मा॑ मा स्व॒स्ति र॑भे रभे स्व॒स्ति मा᳚ । \newline
48. स्व॒स्ति मा॑ मा स्व॒स्ति स्व॒स्ति मा॒ सꣳ सम् मा᳚ स्व॒स्ति स्व॒स्ति मा॒ सम् । \newline
49. मा॒ सꣳ सम् मा॑ मा॒ सम् पा॑रय पारय॒ सम् मा॑ मा॒ सम् पा॑रय । \newline
50. सम् पा॑रय पारय॒ सꣳ सम् पा॑रयादि॒त्य आ॑दि॒त्यः पा॑रय॒ सꣳ सम् पा॑रयादि॒त्यः । \newline
\pagebreak
\markright{ TS 7.5.19.2  \hfill https://www.vedavms.in \hfill}

\section{ TS 7.5.19.2 }

\textbf{TS 7.5.19.2 } \newline
\textbf{Samhita Paata} \newline

पा॑रया ऽऽदि॒त्यस्ते॑ वाजि॒न्॒ युङ्ङनु॒ त्वा ऽऽ र॑भे स्व॒स्ति मा॒ सं पा॑रय प्राण॒धृग॑सि प्रा॒णं मे॑ दृꣳह व्यान॒धृग॑सि व्या॒नं मे॑ दृꣳहा ऽपान॒धृग॑स्यपा॒नं म॑ दृꣳह॒ चक्षु॑रसि॒ चक्षु॒र्मयि॑ धेहि॒ श्रोत्र॑मसि॒ श्रोत्रं॒ मयि॑ धे॒ह्यायु॑र॒स्यायु॒र्मयि॑ धेहि ॥ \newline

\textbf{Pada Paata} \newline

पा॒र॒य॒ । आ॒दि॒त्यः । ते॒ । वा॒जि॒न्न् । युङ् । अन्विति॑ । त्वा॒ । एति॑ । र॒भे॒ । स्व॒स्ति । मा॒ । समिति॑ । पा॒र॒य॒ । प्रा॒ण॒धृगिति॑ प्राण - धृक् । अ॒सि॒ । प्रा॒णमिति॑ प्र - अ॒नम् । मे॒ । दृꣳ॒॒ह॒ । व्या॒न॒धृगिति॑ व्यान - धृक् । अ॒सि॒ । व्या॒नमिति॑ वि - अ॒नम् । मे॒ । दृꣳ॒॒ह॒ । अ॒पा॒न॒धृगित्य॑पान - धृक् । अ॒सि॒ । अ॒पा॒नमित्य॑प - अ॒नम् । मे॒ । दृꣳ॒॒ह॒ । चक्षुः॑ । अ॒सि॒ । चक्षुः॑ । मयि॑ । धे॒हि॒ । श्रोत्र᳚म् । अ॒सि॒ । श्रोत्र᳚म् । मयि॑ । धे॒हि॒ । आयुः॑ । अ॒सि॒ । आयुः॑ । मयि॑ । धे॒हि॒ ॥  \newline


\textbf{Krama Paata} \newline

पा॒र॒या॒दि॒त्यः । आ॒दि॒त्यस्ते᳚ । ते॒ वा॒जि॒न्न्॒ । वा॒जि॒न्॒. युङ्‍ङ् । युङ्‍ङनु॑ । अनु॑ त्वा । त्वा । आ र॑भे । र॒भे॒ स्व॒स्ति । स्व॒स्ति मा᳚ । मा॒ सम् । सम् पा॑रय । पा॒र॒य॒ प्रा॒ण॒धृक् । प्रा॒ण॒धृग॑सि । प्रा॒ण॒धृगिति॑ प्राण - धृक् । अ॒सि॒ प्रा॒णम् । प्रा॒णम् मे᳚ । प्रा॒णमिति॑ प्र - अ॒नम् । मे॒ दृꣳ॒॒ह॒ । दृꣳ॒॒ह॒ व्या॒न॒धृ॒क् । व्या॒न॒धृग॑सि । व्या॒न॒धृगिति॑ व्यान - धृक् । अ॒सि॒ व्या॒नम् । व्या॒नम् मे᳚ । व्या॒नमिति॑ वि - अ॒नम् । मे॒ दृꣳ॒॒ह॒ । दृꣳ॒॒हा॒पा॒न॒धृक् । अ॒पा॒न॒धृग॑सि । अ॒पा॒न॒धृगित्य॑पान - धृक् । अ॒स्य॒पा॒नम् । अ॒पा॒नम् मे᳚ । अ॒पा॒नमित्य॑प - अ॒नम् । मे॒ दृꣳ॒॒ह॒ । दृꣳ॒॒ह॒ चक्षुः॑ । चक्षु॑रसि । अ॒सि॒ चक्षुः॑ । चक्षु॒र् मयि॑ । मयि॑ धेहि । धे॒हि॒ श्रोत्र᳚म् । श्रोत्र॑मसि । अ॒सि॒ श्रोत्र᳚म् । श्रोत्र॒म् मयि॑ । मयि॑ धेहि । धे॒ह्यायुः॑ । आयु॑रसि । अ॒स्यायुः॑ । आयु॒र् मयि॑ । मयि॑ धेहि । धे॒हीति॑ धेहि । \newline

\textbf{Jatai Paata} \newline

1. पा॒र॒या॒ दि॒त्य आ॑दि॒त्यः पा॑रय पारया दि॒त्यः । \newline
2. आ॒दि॒त्य स्ते॑ त आदि॒त्य आ॑दि॒त्य स्ते᳚ । \newline
3. ते॒ वा॒जि॒न्॒. वा॒जि॒न् ते॒ ते॒ वा॒जि॒न्न् । \newline
4. वा॒जि॒न्॒. युङ् युङ् वा॑जिन्. वाजि॒न्॒. युङ् । \newline
5. युङ् ङन्वनु॒ युङ् युङ् ङनु॑ । \newline
6. अनु॑ त्वा॒ त्वा ऽन्वनु॑ त्वा । \newline
7. त्वा ऽऽत्वा॒ त्वा । \newline
8. आ र॑भे रभ॒ आ र॑भे । \newline
9. र॒भे॒ स्व॒स्ति स्व॒स्ति र॑भे रभे स्व॒स्ति । \newline
10. स्व॒स्ति मा॑ मा स्व॒स्ति स्व॒स्ति मा᳚ । \newline
11. मा॒ सꣳ सम् मा॑ मा॒ सम् । \newline
12. सम् पा॑रय पारय॒ सꣳ सम् पा॑रय । \newline
13. पा॒र॒य॒ प्रा॒ण॒धृक् प्रा॑ण॒धृक् पा॑रय पारय प्राण॒धृक् । \newline
14. प्रा॒ण॒धृ ग॑स्यसि प्राण॒धृक् प्रा॑ण॒धृ ग॑सि । \newline
15. प्रा॒ण॒धृगिति॑ प्राण - धृक् । \newline
16. अ॒सि॒ प्रा॒णम् प्रा॒ण म॑स्यसि प्रा॒णम् । \newline
17. प्रा॒णम् मे॑ मे प्रा॒णम् प्रा॒णम् मे᳚ । \newline
18. प्रा॒णमिति॑ प्र - अ॒नम् । \newline
19. मे॒ दृꣳ॒॒ह॒ दृꣳ॒॒ह॒ मे॒ मे॒ दृꣳ॒॒ह॒ । \newline
20. दृꣳ॒॒ह॒ व्या॒न॒धृग् व्या॑न॒धृग् दृꣳ॑ह दृꣳह व्यान॒धृक् । \newline
21. व्या॒न॒धृ ग॑स्यसि व्यान॒धृग् व्या॑न॒धृ ग॑सि । \newline
22. व्या॒न॒धृगिति॑ व्यान - धृक् । \newline
23. अ॒सि॒ व्या॒नं ॅव्या॒न म॑स्यसि व्या॒नम् । \newline
24. व्या॒नम् मे॑ मे व्या॒नं ॅव्या॒नम् मे᳚ । \newline
25. व्या॒नमिति॑ वि - अ॒नम् । \newline
26. मे॒ दृꣳ॒॒ह॒ दृꣳ॒॒ह॒ मे॒ मे॒ दृꣳ॒॒ह॒ । \newline
27. दृꣳ॒॒हा॒ पा॒न॒धृ ग॑पान॒धृग् दृꣳ॑ह दृꣳहा पान॒धृक् । \newline
28. अ॒पा॒न॒धृ ग॑स्य स्यपान॒धृ ग॑पान॒धृ ग॑सि । \newline
29. अ॒पा॒न॒धृगित्य॑पान - धृक् । \newline
30. अ॒स्य॒ पा॒न म॑पा॒न म॑स्यस्य पा॒नम् । \newline
31. अ॒पा॒नम् मे॑ मे ऽपा॒न म॑पा॒नम् मे᳚ । \newline
32. अ॒पा॒नमित्य॑प - अ॒नम् । \newline
33. मे॒ दृꣳ॒॒ह॒ दृꣳ॒॒ह॒ मे॒ मे॒ दृꣳ॒॒ह॒ । \newline
34. दृꣳ॒॒ह॒ चक्षु॒ श्चक्षु॑र् दृꣳह दृꣳह॒ चक्षुः॑ । \newline
35. चक्षु॑ रस्यसि॒ चक्षु॒ श्चक्षु॑ रसि । \newline
36. अ॒सि॒ चक्षु॒ श्चक्षु॑ रस्यसि॒ चक्षुः॑ । \newline
37. चक्षु॒र् मयि॒ मयि॒ चक्षु॒ श्चक्षु॒र् मयि॑ । \newline
38. मयि॑ धेहि धेहि॒ मयि॒ मयि॑ धेहि । \newline
39. धे॒हि॒ श्रोत्रꣳ॒॒ श्रोत्र॑म् धेहि धेहि॒ श्रोत्र᳚म् । \newline
40. श्रोत्र॑ मस्यसि॒ श्रोत्रꣳ॒॒ श्रोत्र॑ मसि । \newline
41. अ॒सि॒ श्रोत्रꣳ॒॒ श्रोत्र॑ मस्यसि॒ श्रोत्र᳚म् । \newline
42. श्रोत्र॒म् मयि॒ मयि॒ श्रोत्रꣳ॒॒ श्रोत्र॒म् मयि॑ । \newline
43. मयि॑ धेहि धेहि॒ मयि॒ मयि॑ धेहि । \newline
44. धे॒ह्यायु॒ रायु॑र् धेहि धे॒ह्यायुः॑ । \newline
45. आयु॑ रस्य॒ स्यायु॒ रायु॑ रसि । \newline
46. अ॒स्यायु॒ रायु॑ रस्य॒ स्यायुः॑ । \newline
47. आयु॒र् मयि॒ मय्यायु॒ रायु॒र् मयि॑ । \newline
48. मयि॑ धेहि धेहि॒ मयि॒ मयि॑ धेहि । \newline
49. धे॒हीति॑ धेहि । \newline

\textbf{Ghana Paata } \newline

1. पा॒र॒या॒दि॒त्य आ॑दि॒त्यः पा॑रय पारया दि॒त्य स्ते॑ त आदि॒त्यः पा॑रय पारया दि॒त्य स्ते᳚ । \newline
2. आ॒दि॒त्य स्ते॑ त आदि॒त्य आ॑दि॒त्य स्ते॑ वाजिन्. वाजिन् त आदि॒त्य आ॑दि॒त्य स्ते॑ वाजिन्न् । \newline
3. ते॒ वा॒जि॒न्॒. वा॒जि॒न् ते॒ ते॒ वा॒जि॒न्॒. युङ् युङ् वा॑जिन् ते ते वाजि॒न्॒. युङ् । \newline
4. वा॒जि॒न्॒. युङ् युङ् वा॑जिन्. वाजि॒न्॒. युङ् ङन्वनु॒ युङ् वा॑जिन्. वाजि॒न्॒. युङ् ङनु॑ । \newline
5. युङ् ङन्वनु॒ युङ् युङ् ङनु॑ त्वा॒ त्वा ऽनु॒ युङ् युङ् ङनु॑ त्वा । \newline
6. अनु॑ त्वा॒ त्वा ऽन्वनु॒ त्वा ऽऽत्वा ऽन्वनु॒ त्वा । \newline
7. त्वा ऽऽत्वा॒ त्वा ऽऽर॑भे रभ॒ आ त्वा॒ त्वा ऽऽर॑भे । \newline
8. आ र॑भे रभ॒ आ र॑भे स्व॒स्ति स्व॒स्ति र॑भ॒ आ र॑भे स्व॒स्ति । \newline
9. र॒भे॒ स्व॒स्ति स्व॒स्ति र॑भे रभे स्व॒स्ति मा॑ मा स्व॒स्ति र॑भे रभे स्व॒स्ति मा᳚ । \newline
10. स्व॒स्ति मा॑ मा स्व॒स्ति स्व॒स्ति मा॒ सꣳ सम् मा᳚ स्व॒स्ति स्व॒स्ति मा॒ सम् । \newline
11. मा॒ सꣳ सम् मा॑ मा॒ सम् पा॑रय पारय॒ सम् मा॑ मा॒ सम् पा॑रय । \newline
12. सम् पा॑रय पारय॒ सꣳ सम् पा॑रय प्राण॒धृक् प्रा॑ण॒धृक् पा॑रय॒ सꣳ सम् पा॑रय प्राण॒धृक् । \newline
13. पा॒र॒य॒ प्रा॒ण॒धृक् प्रा॑ण॒धृक् पा॑रय पारय प्राण॒धृ ग॑स्यसि प्राण॒धृक् पा॑रय पारय प्राण॒धृ ग॑सि । \newline
14. प्रा॒ण॒धृ ग॑स्यसि प्राण॒धृक् प्रा॑ण॒धृ ग॑सि प्रा॒णम् प्रा॒ण म॑सि प्राण॒धृक् प्रा॑ण॒धृ ग॑सि प्रा॒णम् । \newline
15. प्रा॒ण॒धृगिति॑ प्राण - धृक् । \newline
16. अ॒सि॒ प्रा॒णम् प्रा॒ण म॑स्यसि प्रा॒णम् मे॑ मे प्रा॒ण म॑स्यसि प्रा॒णम् मे᳚ । \newline
17. प्रा॒णम् मे॑ मे प्रा॒णम् प्रा॒णम् मे॑ दृꣳह दृꣳह मे प्रा॒णम् प्रा॒णम् मे॑ दृꣳह । \newline
18. प्रा॒णमिति॑ प्र - अ॒नम् । \newline
19. मे॒ दृꣳ॒॒ह॒ दृꣳ॒॒ह॒ मे॒ मे॒ दृꣳ॒॒ह॒ व्या॒न॒धृग् व्या॑न॒धृग् दृꣳ॑ह मे मे दृꣳह व्यान॒धृक् । \newline
20. दृꣳ॒॒ह॒ व्या॒न॒धृग् व्या॑न॒धृग् दृꣳ॑ह दृꣳह व्यान॒धृ ग॑स्यसि व्यान॒धृग् दृꣳ॑ह दृꣳह व्यान॒धृ ग॑सि । \newline
21. व्या॒न॒धृ ग॑स्यसि व्यान॒धृग् व्या॑न॒धृ ग॑सि व्या॒नं ॅव्या॒न म॑सि व्यान॒धृग् व्या॑न॒धृ ग॑सि व्या॒नम् । \newline
22. व्या॒न॒धृगिति॑ व्यान - धृक् । \newline
23. अ॒सि॒ व्या॒नं ॅव्या॒न म॑स्यसि व्या॒नम् मे॑ मे व्या॒न म॑स्यसि व्या॒नम् मे᳚ । \newline
24. व्या॒नम् मे॑ मे व्या॒नं ॅव्या॒नम् मे॑ दृꣳह दृꣳह मे व्या॒नं ॅव्या॒नम् मे॑ दृꣳह । \newline
25. व्या॒नमिति॑ वि - अ॒नम् । \newline
26. मे॒ दृꣳ॒॒ह॒ दृꣳ॒॒ह॒ मे॒ मे॒ दृꣳ॒॒हा॒ पा॒न॒धृ ग॑पान॒धृग् दृꣳ॑ह मे मे दृꣳहा पान॒धृक् । \newline
27. दृꣳ॒॒हा॒ पा॒न॒धृ ग॑पान॒धृग् दृꣳ॑ह दृꣳहा पान॒धृ ग॑स्यस्य पान॒धृग् दृꣳ॑ह दृꣳहा पान॒धृ ग॑सि । \newline
28. अ॒पा॒न॒धृ ग॑स्यस्य पान॒धृ ग॑पान॒धृ ग॑स्य पा॒न म॑पा॒न म॑स्य पान॒धृ ग॑पान॒धृ ग॑स्य पा॒नम् । \newline
29. अ॒पा॒न॒धृगित्य॑पान - धृक् । \newline
30. अ॒स्य॒ पा॒न म॑पा॒न म॑स्यस्य पा॒नम् मे॑ मे ऽपा॒न म॑स्यस्य पा॒नम् मे᳚ । \newline
31. अ॒पा॒नम् मे॑ मे ऽपा॒न म॑पा॒नम् मे॑ दृꣳह दृꣳह मे ऽपा॒न म॑पा॒नम् मे॑ दृꣳह । \newline
32. अ॒पा॒नमित्य॑प - अ॒नम् । \newline
33. मे॒ दृꣳ॒॒ह॒ दृꣳ॒॒ह॒ मे॒ मे॒ दृꣳ॒॒ह॒ चक्षु॒ श्चक्षु॑र् दृꣳह मे मे दृꣳह॒ चक्षुः॑ । \newline
34. दृꣳ॒॒ह॒ चक्षु॒ श्चक्षु॑र् दृꣳह दृꣳह॒ चक्षु॑ रस्यसि॒ चक्षु॑र् दृꣳह दृꣳह॒ चक्षु॑ रसि । \newline
35. चक्षु॑ रस्यसि॒ चक्षु॒ श्चक्षु॑रसि॒ चक्षु॒ श्चक्षु॑ रसि॒ चक्षु॒ श्चक्षु॑ रसि॒ चक्षुः॑ । \newline
36. अ॒सि॒ चक्षु॒ श्चक्षु॑ रस्यसि॒ चक्षु॒र् मयि॒ मयि॒ चक्षु॑ रस्यसि॒ चक्षु॒र् मयि॑ । \newline
37. चक्षु॒र् मयि॒ मयि॒ चक्षु॒ श्चक्षु॒र् मयि॑ धेहि धेहि॒ मयि॒ चक्षु॒ श्चक्षु॒र् मयि॑ धेहि । \newline
38. मयि॑ धेहि धेहि॒ मयि॒ मयि॑ धेहि॒ श्रोत्रꣳ॒॒ श्रोत्र॑म् धेहि॒ मयि॒ मयि॑ धेहि॒ श्रोत्र᳚म् । \newline
39. धे॒हि॒ श्रोत्रꣳ॒॒ श्रोत्र॑म् धेहि धेहि॒ श्रोत्र॑ मस्यसि॒ श्रोत्र॑म् धेहि धेहि॒ श्रोत्र॑ मसि । \newline
40. श्रोत्र॑ मस्यसि॒ श्रोत्रꣳ॒॒ श्रोत्र॑ मसि॒ श्रोत्रꣳ॒॒ श्रोत्र॑ मसि॒ श्रोत्रꣳ॒॒ श्रोत्र॑ मसि॒ श्रोत्र᳚म् । \newline
41. अ॒सि॒ श्रोत्रꣳ॒॒ श्रोत्र॑ मस्यसि॒ श्रोत्र॒म् मयि॒ मयि॒ श्रोत्र॑ मस्यसि॒ श्रोत्र॒म् मयि॑ । \newline
42. श्रोत्र॒म् मयि॒ मयि॒ श्रोत्रꣳ॒॒ श्रोत्र॒म् मयि॑ धेहि धेहि॒ मयि॒ श्रोत्रꣳ॒॒ श्रोत्र॒म् मयि॑ धेहि । \newline
43. मयि॑ धेहि धेहि॒ मयि॒ मयि॑ धे॒ह्यायु॒ रायु॑र् धेहि॒ मयि॒ मयि॑ धे॒ह्यायुः॑ । \newline
44. धे॒ह्यायु॒ रायु॑र् धेहि धे॒ह्यायु॑ रस्य॒ स्यायु॑र् धेहि धे॒ह्यायु॑ रसि । \newline
45. आयु॑ रस्य॒ स्यायु॒ रायु॑ र॒स्यायु॒ रायु॑ र॒स्यायु॒ रायु॑ र॒स्यायुः॑ । \newline
46. अ॒स्यायु॒ रायु॑ रस्य॒ स्यायु॒र् मयि॒ मय्यायु॑ रस्य॒ स्यायु॒र् मयि॑ । \newline
47. आयु॒र् मयि॒ मय्यायु॒ रायु॒र् मयि॑ धेहि धेहि॒ मय्यायु॒ रायु॒र् मयि॑ धेहि । \newline
48. मयि॑ धेहि धेहि॒ मयि॒ मयि॑ धेहि । \newline
49. धे॒हीति॑ धेहि । \newline
\pagebreak
\markright{ TS 7.5.20.1  \hfill https://www.vedavms.in \hfill}

\section{ TS 7.5.20.1 }

\textbf{TS 7.5.20.1 } \newline
\textbf{Samhita Paata} \newline

जज्ञि॒ बीजं॒ ॅवर्ष्टा॑ प॒र्जन्यः॒ पक्ता॑ स॒स्यꣳ सु॑पिप्प॒ला ओष॑धयः स्वधिचर॒णेयꣳ सू॑पसद॒नो᳚ऽग्निः स्व॑द्ध्य॒क्षम॒न्तरि॑क्षꣳसुपा॒वः पव॑मानः सूपस्था॒ना द्यौः शि॒वम॒सौ तप॑न् यथापू॒र्वम॑होरा॒त्रे प॑ञ्चद॒शिनो᳚ ऽर्द्धमा॒सा-स्त्रिꣳ॒॒शिनो॒ मासाः᳚ क्लृ॒प्ता ऋ॒तवः॑ शा॒न्तः सं॑ॅवथ्स॒रः ॥ \newline

\textbf{Pada Paata} \newline

जज्ञि॑ । बीज᳚म् । वर्ष्टा᳚ । प॒र्जन्यः॑ । पक्ता᳚ । स॒स्यम् । सु॒पि॒प्प॒ला इति॑ सु - पि॒प्प॒लाः । ओष॑धयः । स्व॒धि॒च॒र॒णेति॑ सु - अ॒धि॒च॒र॒णा । इ॒यम् । सू॒प॒स॒द॒न इति॑ सु - उ॒प॒स॒द॒नः । अ॒ग्निः । स्व॒द्ध्य॒क्षमिति॑ सु - अ॒द्ध्य॒क्षम् । अ॒न्तरि॑क्षम् । सु॒पा॒व इति॑ सु-पा॒वः । पव॑मानः । सू॒प॒स्था॒नेति॑ सु - उ॒प॒स्था॒ना । द्यौः । शि॒वम् । अ॒सौ । तपन्न्॑ । य॒था॒पू॒र्वमिति॑ यथा - पू॒र्वम् । अ॒हो॒रा॒त्रे इत्य॑हः - रा॒त्रे । प॒ञ्च॒द॒शिन॒ इति॑ पञ्च - द॒शिनः॑ । अ॒द्‌र्ध॒मा॒सा इत्य॑द्‌र्ध - मा॒साः । त्रिꣳ॒॒शिनः॑ । मासाः᳚ । क्लृ॒प्ताः । ऋ॒तवः॑ । शा॒न्तः । सं॒ॅव॒थ्स॒र इति॑ सं - व॒थ्स॒रः ॥  \newline


\textbf{Krama Paata} \newline

जज्ञि॒ बीज᳚म् । बीज॒म् ॅवर्ष्टा᳚ । वर्ष्टा॑ प॒र्जन्यः॑ । प॒र्जन्यः॒ पक्ता᳚ । पक्ता॑ स॒स्यम् । स॒स्यꣳ सु॑पिप्प॒लाः । सु॒पि॒प्प॒ला ओष॑धयः । सु॒पि॒प्प॒ला इति॑ सु - पि॒प्प॒लाः । ओष॑धयः स्वधिचर॒णा । स्व॒धि॒च॒र॒णेयम् । स्व॒धि॒च॒र॒णेति॑ सु - अ॒धि॒च॒र॒णा । इ॒यꣳ सू॑पसद॒नः । सू॒प॒स॒द॒नो᳚ऽग्निः । सू॒प॒स॒द॒न इति॑ सु - उ॒प॒स॒द॒नः । अ॒ग्निः स्व॑द्ध्य॒क्षम् । स्व॒द्ध्य॒क्षम॒न्तरि॑क्षम् । स्व॒द्ध्य॒क्षमिति॑ सु - अ॒द्ध्य॒क्षम् । अ॒न्तरि॑क्षꣳ सुपा॒वः । सु॒पा॒वः पव॑मानः । सु॒पा॒व इति॑ सु - पा॒वः । पव॑मानः सूपस्था॒ना । सू॒प॒स्था॒ना द्यौः । सू॒प॒स्था॒नेति॑ सु - उ॒प॒स्था॒ना । द्यौः शि॒वम् । शि॒वम॒सौ । अ॒सौ तपन्न्॑ । तप॑न्. यथापू॒र्वम् । य॒था॒पू॒र्वम॑होरा॒त्रे । य॒था॒पू॒र्वमिति॑ यथा - पू॒र्वम् । अ॒हो॒रा॒त्रे प॑ञ्चद॒शिनः॑ । अ॒हो॒रा॒त्रे इत्य॑हः - रा॒त्रे । प॒ञ्च॒द॒शिनो᳚ऽर्द्धमा॒साः । प॒ञ्च॒द॒शिन॒ इति॑ पञ्च - द॒शिनः॑ । अ॒र्द्ध॒मा॒सास्त्रिꣳ॒॒शिनः॑ । अ॒र्द्ध॒मा॒सा इत्य॑र्द्ध - मा॒साः । त्रिꣳ॒॒शिनो॒ मासाः᳚ । मासाः᳚ क्लृ॒प्ताः । क्लृ॒प्ता ऋ॒तवः॑ । ऋ॒तवः॑ शा॒न्तः । शा॒न्तः स॑म्ॅवथ्स॒रः । स॒म्ॅव॒थ्स॒र इति॑ सम् - व॒थ्स॒रः । \newline

\textbf{Jatai Paata} \newline

1. जज्ञि॒ बीज॒म् बीज॒म् जज्ञि॒ जज्ञि॒ बीज᳚म् । \newline
2. बीजं॒ ॅवर्ष्टा॒ वर्ष्टा॒ बीज॒म् बीजं॒ ॅवर्ष्टा᳚ । \newline
3. वर्ष्टा॑ प॒र्जन्यः॑ प॒र्जन्यो॒ वर्ष्टा॒ वर्ष्टा॑ प॒र्जन्यः॑ । \newline
4. प॒र्जन्यः॒ पक्ता॒ पक्ता॑ प॒र्जन्यः॑ प॒र्जन्यः॒ पक्ता᳚ । \newline
5. पक्ता॑ स॒स्यꣳ स॒स्यम् पक्ता॒ पक्ता॑ स॒स्यम् । \newline
6. स॒स्यꣳ सु॑पिप्प॒लाः सु॑पिप्प॒लाः स॒स्यꣳ स॒स्यꣳ सु॑पिप्प॒लाः । \newline
7. सु॒पि॒प्प॒ला ओष॑धय॒ ओष॑धयः सुपिप्प॒लाः सु॑पिप्प॒ला ओष॑धयः । \newline
8. सु॒पि॒प्प॒ला इति॑ सु - पि॒प्प॒लाः । \newline
9. ओष॑धयः स्वधिचर॒णा स्व॑धिचर॒ णौष॑धय॒ ओष॑धयः स्वधिचर॒णा । \newline
10. स्व॒धि॒च॒र॒ णेय मि॒यꣳ स्व॑धिचर॒णा स्व॑धिचर॒ णेयम् । \newline
11. स्व॒धि॒च॒र॒णेति॑ सु - अ॒धि॒च॒र॒णा । \newline
12. इ॒यꣳ सू॑पसद॒नः सू॑पसद॒न इ॒य मि॒यꣳ सू॑पसद॒नः । \newline
13. सू॒प॒स॒द॒नो᳚ ऽग्नि र॒ग्निः सू॑पसद॒नः सू॑पसद॒नो᳚ ऽग्निः । \newline
14. सू॒प॒स॒द॒न इति॑ सु - उ॒प॒स॒द॒नः । \newline
15. अ॒ग्निः स्व॑द्ध्य॒क्षꣳ स्व॑द्ध्य॒क्ष म॒ग्नि र॒ग्निः स्व॑द्ध्य॒क्षम् । \newline
16. स्व॒द्ध्य॒क्ष म॒न्तरि॑क्ष म॒न्तरि॑क्षꣳ स्वद्ध्य॒क्षꣳ स्व॑द्ध्य॒क्ष म॒न्तरि॑क्षम् । \newline
17. स्व॒द्ध्य॒क्षमिति॑ सु - अ॒द्ध्य॒क्षम् । \newline
18. अ॒न्तरि॑क्षꣳ सुपा॒वः सु॑पा॒वो᳚ ऽन्तरि॑क्ष म॒न्तरि॑क्षꣳ सुपा॒वः । \newline
19. सु॒पा॒वः पव॑मानः॒ पव॑मानः सुपा॒वः सु॑पा॒वः पव॑मानः । \newline
20. सु॒पा॒व इति॑ सु - पा॒वः । \newline
21. पव॑मानः सूपस्था॒ना सू॑पस्था॒ना पव॑मानः॒ पव॑मानः सूपस्था॒ना । \newline
22. सू॒प॒स्था॒ना द्यौर् द्यौः सू॑पस्था॒ना सू॑पस्था॒ना द्यौः । \newline
23. सू॒प॒स्था॒नेति॑ सु - उ॒प॒स्था॒ना । \newline
24. द्यौः शि॒वꣳ शि॒वम् द्यौर् द्यौः शि॒वम् । \newline
25. शि॒व म॒सा व॒सौ शि॒वꣳ शि॒व म॒सौ । \newline
26. अ॒सौ तप॒न् तप॑न् न॒सा व॒सौ तपन्न्॑ । \newline
27. तप॑न्. यथापू॒र्वं ॅय॑थापू॒र्वम् तप॒न् तप॑न्. यथापू॒र्वम् । \newline
28. य॒था॒पू॒र्व म॑होरा॒त्रे अ॑होरा॒त्रे य॑थापू॒र्वं ॅय॑थापू॒र्व म॑होरा॒त्रे । \newline
29. य॒था॒पू॒र्वमिति॑ यथा - पू॒र्वम् । \newline
30. अ॒हो॒रा॒त्रे प॑ञ्चद॒शिनः॑ पञ्चद॒शिनो॑ ऽहोरा॒त्रे अ॑होरा॒त्रे प॑ञ्चद॒शिनः॑ । \newline
31. अ॒हो॒रा॒त्रे इत्य॑हः - रा॒त्रे । \newline
32. प॒ञ्च॒द॒शिनो᳚ ऽर्द्धमा॒सा अ॑र्द्धमा॒साः प॑ञ्चद॒शिनः॑ पञ्चद॒शिनो᳚ ऽर्द्धमा॒साः । \newline
33. प॒ञ्च॒द॒शिन॒ इति॑ पञ्च - द॒शिनः॑ । \newline
34. अ॒र्द्ध॒मा॒सा स्त्रिꣳ॒॒शिन॑ स्त्रिꣳ॒॒शिनो᳚ ऽर्द्धमा॒सा अ॑र्द्धमा॒सा स्त्रिꣳ॒॒शिनः॑ । \newline
35. अ॒र्द्ध॒मा॒सा इत्य॑र्द्ध - मा॒साः । \newline
36. त्रिꣳ॒॒शिनो॒ मासा॒ मासा᳚ स्त्रिꣳ॒॒शिन॑ स्त्रिꣳ॒॒शिनो॒ मासाः᳚ । \newline
37. मासाः᳚ क्लृ॒प्ताः क्लृ॒प्ता मासा॒ मासाः᳚ क्लृ॒प्ताः । \newline
38. क्लृ॒प्ता ऋ॒तव॑ ऋ॒तवः॑ क्लृ॒प्ताः क्लृ॒प्ता ऋ॒तवः॑ । \newline
39. ऋ॒तवः॑ शा॒न्तः शा॒न्त ऋ॒तव॑ ऋ॒तवः॑ शा॒न्तः । \newline
40. शा॒न्तः सं॑ॅवथ्स॒रः सं॑ॅवथ्स॒रः शा॒न्तः शा॒न्तः सं॑ॅवथ्स॒रः । \newline
41. सं॒ॅव॒थ्स॒र इति॑ सं - व॒थ्स॒रः । \newline

\textbf{Ghana Paata } \newline

1. जज्ञि॒ बीज॒म् बीज॒म् जज्ञि॒ जज्ञि॒ बीजं॒ ॅवर्ष्टा॒ वर्ष्टा॒ बीज॒म् जज्ञि॒ जज्ञि॒ बीजं॒ ॅवर्ष्टा᳚ । \newline
2. बीजं॒ ॅवर्ष्टा॒ वर्ष्टा॒ बीज॒म् बीजं॒ ॅवर्ष्टा॑ प॒र्जन्यः॑ प॒र्जन्यो॒ वर्ष्टा॒ बीज॒म् बीजं॒ ॅवर्ष्टा॑ प॒र्जन्यः॑ । \newline
3. वर्ष्टा॑ प॒र्जन्यः॑ प॒र्जन्यो॒ वर्ष्टा॒ वर्ष्टा॑ प॒र्जन्यः॒ पक्ता॒ पक्ता॑ प॒र्जन्यो॒ वर्ष्टा॒ वर्ष्टा॑ प॒र्जन्यः॒ पक्ता᳚ । \newline
4. प॒र्जन्यः॒ पक्ता॒ पक्ता॑ प॒र्जन्यः॑ प॒र्जन्यः॒ पक्ता॑ स॒स्यꣳ स॒स्यम् पक्ता॑ प॒र्जन्यः॑ प॒र्जन्यः॒ पक्ता॑ स॒स्यम् । \newline
5. पक्ता॑ स॒स्यꣳ स॒स्यम् पक्ता॒ पक्ता॑ स॒स्यꣳ सु॑पिप्प॒लाः सु॑पिप्प॒लाः स॒स्यम् पक्ता॒ पक्ता॑ स॒स्यꣳ सु॑पिप्प॒लाः । \newline
6. स॒स्यꣳ सु॑पिप्प॒लाः सु॑पिप्प॒लाः स॒स्यꣳ स॒स्यꣳ सु॑पिप्प॒ला ओष॑धय॒ ओष॑धयः सुपिप्प॒लाः स॒स्यꣳ स॒स्यꣳ सु॑पिप्प॒ला ओष॑धयः । \newline
7. सु॒पि॒प्प॒ला ओष॑धय॒ ओष॑धयः सुपिप्प॒लाः सु॑पिप्प॒ला ओष॑धयः स्वधिचर॒णा स्व॑धिचर॒ णौष॑धयः सुपिप्प॒लाः सु॑पिप्प॒ला ओष॑धयः स्वधिचर॒णा । \newline
8. सु॒पि॒प्प॒ला इति॑ सु - पि॒प्प॒लाः । \newline
9. ओष॑धयः स्वधिचर॒णा स्व॑धिचर॒ णौष॑धय॒ ओष॑धयः स्वधिचर॒ णेय मि॒यꣳ स्व॑धिचर॒
णौष॑धय॒ ओष॑धयः स्वधिचर॒ णेयम् । \newline
10. स्व॒धि॒च॒र॒ णेय मि॒यꣳ स्व॑धिचर॒णा स्व॑धिचर॒ णेयꣳ सू॑पसद॒नः सू॑पसद॒न इ॒यꣳ स्व॑धिचर॒णा स्व॑धिचर॒ णेयꣳ सू॑पसद॒नः । \newline
11. स्व॒धि॒च॒र॒णेति॑ सु - अ॒धि॒च॒र॒णा । \newline
12. इ॒यꣳ सू॑पसद॒नः सू॑पसद॒न इ॒य मि॒यꣳ सू॑पसद॒नो᳚ ऽग्नि र॒ग्निः सू॑पसद॒न इ॒य मि॒यꣳ सू॑पसद॒नो᳚ ऽग्निः । \newline
13. सू॒प॒स॒द॒नो᳚ ऽग्नि र॒ग्निः सू॑पसद॒नः सू॑पसद॒नो᳚ ऽग्निः स्व॑द्ध्य॒क्षꣳ स्व॑द्ध्य॒क्ष म॒ग्निः सू॑पसद॒नः सू॑पसद॒नो᳚ ऽग्निः स्व॑द्ध्य॒क्षम् । \newline
14. सू॒प॒स॒द॒न इति॑ सु - उ॒प॒स॒द॒नः । \newline
15. अ॒ग्निः स्व॑द्ध्य॒क्षꣳ स्व॑द्ध्य॒क्ष म॒ग्नि र॒ग्निः स्व॑द्ध्य॒क्ष म॒न्तरि॑क्ष म॒न्तरि॑क्षꣳ स्वद्ध्य॒क्ष म॒ग्नि र॒ग्निः स्व॑द्ध्य॒क्ष म॒न्तरि॑क्षम् । \newline
16. स्व॒द्ध्य॒क्ष म॒न्तरि॑क्ष म॒न्तरि॑क्षꣳ स्वद्ध्य॒क्षꣳ स्व॑द्ध्य॒क्ष म॒न्तरि॑क्षꣳ सुपा॒वः सु॑पा॒वो᳚ ऽन्तरि॑क्षꣳ स्वद्ध्य॒क्षꣳ स्व॑द्ध्य॒क्ष म॒न्तरि॑क्षꣳ सुपा॒वः । \newline
17. स्व॒द्ध्य॒क्षमिति॑ सु - अ॒द्ध्य॒क्षम् । \newline
18. अ॒न्तरि॑क्षꣳ सुपा॒वः सु॑पा॒वो᳚ ऽन्तरि॑क्ष म॒न्तरि॑क्षꣳ सुपा॒वः पव॑मानः॒ पव॑मानः सुपा॒वो᳚ ऽन्तरि॑क्ष म॒न्तरि॑क्षꣳ सुपा॒वः पव॑मानः । \newline
19. सु॒पा॒वः पव॑मानः॒ पव॑मानः सुपा॒वः सु॑पा॒वः पव॑मानः सूपस्था॒ना सू॑पस्था॒ना पव॑मानः सुपा॒वः सु॑पा॒वः पव॑मानः सूपस्था॒ना । \newline
20. सु॒पा॒व इति॑ सु - पा॒वः । \newline
21. पव॑मानः सूपस्था॒ना सू॑पस्था॒ना पव॑मानः॒ पव॑मानः सूपस्था॒ना द्यौर् द्यौः सू॑पस्था॒ना पव॑मानः॒ पव॑मानः सूपस्था॒ना द्यौः । \newline
22. सू॒प॒स्था॒ना द्यौर् द्यौः सू॑पस्था॒ना सू॑पस्था॒ना द्यौः शि॒वꣳ शि॒वम् द्यौः सू॑पस्था॒ना सू॑पस्था॒ना द्यौः शि॒वम् । \newline
23. सू॒प॒स्था॒नेति॑ सु - उ॒प॒स्था॒ना । \newline
24. द्यौः शि॒वꣳ शि॒वम् द्यौर् द्यौः शि॒व म॒सा व॒सौ शि॒वम् द्यौर् द्यौः शि॒व म॒सौ । \newline
25. शि॒व म॒सा व॒सौ शि॒वꣳ शि॒व म॒सौ तप॒न् तप॑न् न॒सौ शि॒वꣳ शि॒व म॒सौ तपन्न्॑ । \newline
26. अ॒सौ तप॒न् तप॑न् न॒सा व॒सौ तप॑न्. यथापू॒र्वं ॅय॑थापू॒र्वम् तप॑न् न॒सा व॒सौ तप॑न्. यथापू॒र्वम् । \newline
27. तप॑न्. यथापू॒र्वं ॅय॑थापू॒र्वम् तप॒न् तप॑न्. यथापू॒र्व म॑होरा॒त्रे अ॑होरा॒त्रे य॑थापू॒र्वम् तप॒न् तप॑न्. यथापू॒र्व म॑होरा॒त्रे । \newline
28. य॒था॒पू॒र्व म॑होरा॒त्रे अ॑होरा॒त्रे य॑थापू॒र्वं ॅय॑थापू॒र्व म॑होरा॒त्रे प॑ञ्चद॒शिनः॑ पञ्चद॒शिनो॑ ऽहोरा॒त्रे य॑थापू॒र्वं ॅय॑थापू॒र्व म॑होरा॒त्रे प॑ञ्चद॒शिनः॑ । \newline
29. य॒था॒पू॒र्वमिति॑ यथा - पू॒र्वम् । \newline
30. अ॒हो॒रा॒त्रे प॑ञ्चद॒शिनः॑ पञ्चद॒शिनो॑ ऽहोरा॒त्रे अ॑होरा॒त्रे प॑ञ्चद॒शिनो᳚ ऽर्द्धमा॒सा अ॑र्द्धमा॒साः प॑ञ्चद॒शिनो॑ ऽहोरा॒त्रे अ॑होरा॒त्रे प॑ञ्चद॒शिनो᳚ ऽर्द्धमा॒साः । \newline
31. अ॒हो॒रा॒त्रे इत्य॑हः - रा॒त्रे । \newline
32. प॒ञ्च॒द॒शिनो᳚ ऽर्द्धमा॒सा अ॑र्द्धमा॒साः प॑ञ्चद॒शिनः॑ पञ्चद॒शिनो᳚ ऽर्द्धमा॒सा स्त्रिꣳ॒॒शिन॑ स्त्रिꣳ॒॒शिनो᳚ ऽर्द्धमा॒साः प॑ञ्चद॒शिनः॑ पञ्चद॒शिनो᳚ ऽर्द्धमा॒सा स्त्रिꣳ॒॒शिनः॑ । \newline
33. प॒ञ्च॒द॒शिन॒ इति॑ पञ्च - द॒शिनः॑ । \newline
34. अ॒र्द्ध॒मा॒सा स्त्रिꣳ॒॒शिन॑ स्त्रिꣳ॒॒शिनो᳚ ऽर्द्धमा॒सा अ॑र्द्धमा॒सा स्त्रिꣳ॒॒शिनो॒ मासा॒ मासा᳚ स्त्रिꣳ॒॒शिनो᳚ ऽर्द्धमा॒सा अ॑र्द्धमा॒सा स्त्रिꣳ॒॒शिनो॒ मासाः᳚ । \newline
35. अ॒र्द्ध॒मा॒सा इत्य॑र्द्ध - मा॒साः । \newline
36. त्रिꣳ॒॒शिनो॒ मासा॒ मासा᳚ स्त्रिꣳ॒॒शिन॑ स्त्रिꣳ॒॒शिनो॒ मासाः᳚ क्लृ॒प्ताः क्लृ॒प्ता मासा᳚ स्त्रिꣳ॒॒शिन॑ 
स्त्रिꣳ॒॒शिनो॒ मासाः᳚ क्लृ॒प्ताः । \newline
37. मासाः᳚ क्लृ॒प्ताः क्लृ॒प्ता मासा॒ मासाः᳚ क्लृ॒प्ता ऋ॒तव॑ ऋ॒तवः॑ क्लृ॒प्ता मासा॒ मासाः᳚ क्लृ॒प्ता ऋ॒तवः॑ । \newline
38. क्लृ॒प्ता ऋ॒तव॑ ऋ॒तवः॑ क्लृ॒प्ताः क्लृ॒प्ता ऋ॒तवः॑ शा॒न्तः शा॒न्त ऋ॒तवः॑ क्लृ॒प्ताः क्लृ॒प्ता ऋ॒तवः॑ शा॒न्तः । \newline
39. ऋ॒तवः॑ शा॒न्तः शा॒न्त ऋ॒तव॑ ऋ॒तवः॑ शा॒न्तः सं॑ॅवथ्स॒रः सं॑ॅवथ्स॒रः शा॒न्त ऋ॒तव॑ ऋ॒तवः॑ शा॒न्तः सं॑ॅवथ्स॒रः । \newline
40. शा॒न्तः सं॑ॅवथ्स॒रः सं॑ॅवथ्स॒रः शा॒न्तः शा॒न्तः सं॑ॅवथ्स॒रः । \newline
41. सं॒ॅव॒थ्स॒र इति॑ सं - व॒थ्स॒रः । \newline
\pagebreak
\markright{ TS 7.5.21.1  \hfill https://www.vedavms.in \hfill}

\section{ TS 7.5.21.1 }

\textbf{TS 7.5.21.1 } \newline
\textbf{Samhita Paata} \newline

आ॒ग्ने॒यो᳚ऽष्टाक॑पालः सौ॒म्यश्च॒रुः सा॑वि॒त्रो᳚ऽष्टाक॑पालः पौ॒ष्णश्च॒रू रौ॒द्रश्च॒रुर॒ग्नये॑ वैश्वान॒राय॒ द्वाद॑शकपालो मृगाख॒रे यदि॒ नाऽऽ*गच्छे॑-द॒ग्नये-ऽꣳ॑हो॒मुचे॒-ऽष्टाक॑पालः सौ॒र्यं पयो॑ वाय॒व्य॑ आज्य॑भागः ॥ \newline

\textbf{Pada Paata} \newline

आ॒ग्ने॒यः । अ॒ष्टाक॑पाल॒ इत्य॒ष्टा - क॒पा॒लः॒ । सौ॒म्यः । च॒रुः । सा॒वि॒त्रः । अ॒ष्टाक॑पाल॒ इत्य॒ष्टा - क॒पा॒लः॒ । पौ॒ष्णः । च॒रुः । रौ॒द्रः । च॒रुः । अ॒ग्नये᳚ । वै॒श्वा॒न॒राय॑ । द्वाद॑शकपाल॒ इति॒ द्वाद॑श - क॒पा॒लः॒ । मृ॒गा॒ख॒र इति॑ मृग - आ॒ख॒रे । यदि॑ । न । आ॒गच्छे॒दित्या᳚-गच्छे᳚त् । अ॒ग्नये᳚ । अꣳ॒॒हो॒मुच॒ इत्यꣳ॑हः-मुचे᳚ । अ॒ष्टाक॑पाल॒ इत्य॒ष्टा-क॒पा॒लः॒ । सौ॒र्यम् । पयः॑ । वा॒य॒व्यः॑ । आज्य॑भाग॒ इत्याज्य॑ - भा॒गः॒ ॥  \newline


\textbf{Krama Paata} \newline

आ॒ग्ने॒यो᳚ऽष्टाक॑पालः । अ॒ष्टाक॑पालः सौ॒म्यः । अ॒ष्टाक॑पाल॒ इत्य॒ष्टा - क॒पा॒लः॒ । सौ॒म्यश्च॒रुः । च॒रुः सा॑वि॒त्रः । सा॒वि॒त्रो᳚ऽष्टाक॑पालः । अ॒ष्टाक॑पालः पौ॒ष्णः । अ॒ष्टाक॑पाल॒ इत्य॒ष्टा - क॒पा॒लः॒ । पौ॒ष्णश्च॒रुः । च॒रू रौ॒द्रः । रौ॒द्रश्च॒रुः । च॒रुर॒ग्नये᳚ । अ॒ग्नये॑ वैश्वान॒राय॑ । वै॒श्वा॒न॒राय॒ द्वाद॑शकपालः । द्वाद॑शकपालो मृगाख॒रे । द्वाद॑शकपाल॒ इति॒ द्वाद॑श - क॒पा॒लः॒ । मृ॒गा॒ख॒रे यदि॑ । मृ॒गा॒ख॒र इति॑ मृग - आ॒ख॒रे । यदि॒ न । नागच्छे᳚त् । आ॒गच्छे॑द॒ग्नये᳚ । आ॒गच्छे॒दित्या᳚ - गच्छे᳚त् । अ॒ग्नयेऽꣳ॑हो॒मुचे᳚ । अꣳ॒॒हो॒मुचे॒ऽष्टाक॑पालः । अꣳ॒॒हो॒मुच॒ इत्यꣳ॑हः - मुचे᳚ । अ॒ष्टाक॑पालः सौ॒र्यम् । अ॒ष्टाक॑पाल॒ इत्य॒ष्टा - क॒पा॒लः॒ । सौ॒र्यम् पयः॑ । पयो॑ वाय॒व्यः॑ । वा॒य॒व्य॑ आज्य॑भागः । आज्य॑भाग॒ इत्याज्य॑ - भा॒गः॒ । \newline

\textbf{Jatai Paata} \newline

1. आ॒ग्ने॒यो᳚ ऽष्टाक॑पालो॒ ऽष्टाक॑पाल आग्ने॒य आ᳚ग्ने॒यो᳚ ऽष्टाक॑पालः । \newline
2. अ॒ष्टाक॑पालः सौ॒म्यः सौ॒म्यो᳚ ऽष्टाक॑पालो॒ ऽष्टाक॑पालः सौ॒म्यः । \newline
3. अ॒ष्टाक॑पाल॒ इत्य॒ष्टा - क॒पा॒लः॒ । \newline
4. सौ॒म्य श्च॒रु श्च॒रुः सौ॒म्यः सौ॒म्य श्च॒रुः । \newline
5. च॒रुः सा॑वि॒त्रः सा॑वि॒त्र श्च॒रु श्च॒रुः सा॑वि॒त्रः । \newline
6. सा॒वि॒त्रो᳚ ऽष्टाक॑पालो॒ ऽष्टाक॑पालः सावि॒त्रः सा॑वि॒त्रो᳚ ऽष्टाक॑पालः । \newline
7. अ॒ष्टाक॑पालः पौ॒ष्णः पौ॒ष्णो᳚ ऽष्टाक॑पालो॒ ऽष्टाक॑पालः पौ॒ष्णः । \newline
8. अ॒ष्टाक॑पाल॒ इत्य॒ष्टा - क॒पा॒लः॒ । \newline
9. पौ॒ष्ण श्च॒रु श्च॒रुः पौ॒ष्णः पौ॒ष्ण श्च॒रुः । \newline
10. च॒रू रौ॒द्रो रौ॒द्र श्च॒रु श्च॒रू रौ॒द्रः । \newline
11. रौ॒द्र श्च॒रु श्च॒रू रौ॒द्रो रौ॒द्र श्च॒रुः । \newline
12. च॒रु र॒ग्नये॒ ऽग्नये॑ च॒रु श्च॒रु र॒ग्नये᳚ । \newline
13. अ॒ग्नये॑ वैश्वान॒राय॑ वैश्वान॒राया॒ ग्नये॒ ऽग्नये॑ वैश्वान॒राय॑ । \newline
14. वै॒श्वा॒न॒राय॒ द्वाद॑शकपालो॒ द्वाद॑शकपालो वैश्वान॒राय॑ वैश्वान॒राय॒ द्वाद॑शकपालः । \newline
15. द्वाद॑शकपालो मृगाख॒रे मृ॑गाख॒रे द्वाद॑शकपालो॒ द्वाद॑शकपालो मृगाख॒रे । \newline
16. द्वाद॑शकपाल॒ इति॒ द्वाद॑श - क॒पा॒लः॒ । \newline
17. मृ॒गा॒ख॒रे यदि॒ यदि॑ मृगाख॒रे मृ॑गाख॒रे यदि॑ । \newline
18. मृ॒गा॒ख॒र इति॑ मृग - आ॒ख॒रे । \newline
19. यदि॒ न न यदि॒ यदि॒ न । \newline
20. नागच्छे॑ दा॒गच्छे॒न् न नागच्छे᳚त् । \newline
21. आ॒गच्छे॑ द॒ग्नये॒ ऽग्नय॑ आ॒गच्छे॑ दा॒गच्छे॑ द॒ग्नये᳚ । \newline
22. आ॒गच्छे॒दित्या᳚ - गच्छे᳚त् । \newline
23. अ॒ग्नये ऽꣳ॑हो॒मुचे॑ ऽꣳहो॒मुचे॒ ऽग्नये॒ ऽग्नये ऽꣳ॑हो॒मुचे᳚ । \newline
24. अꣳ॒॒हो॒मुचे॒ ऽष्टाक॑पालो॒ ऽष्टाक॑पालो ऽꣳहो॒मुचे ऽꣳ॑हो॒मुचे॒ ऽष्टाक॑पालः । \newline
25. अꣳ॒॒हो॒मुच॒ इत्यꣳ॑हः - मुचे᳚ । \newline
26. अ॒ष्टाक॑पालः सौ॒र्यꣳ सौ॒र्य म॒ष्टाक॑पालो॒ ऽष्टाक॑पालः सौ॒र्यम् । \newline
27. अ॒ष्टाक॑पाल॒ इत्य॒ष्टा - क॒पा॒लः॒ । \newline
28. सौ॒र्यम् पयः॒ पयः॑ सौ॒र्यꣳ सौ॒र्यम् पयः॑ । \newline
29. पयो॑ वाय॒व्यो॑ वाय॒व्यः॑ पयः॒ पयो॑ वाय॒व्यः॑ । \newline
30. वा॒य॒व्य॑ आज्य॑भाग॒ आज्य॑भागो वाय॒व्यो॑ वाय॒व्य॑ आज्य॑भागः । \newline
31. आज्य॑भाग॒ इत्याज्य॑ - भा॒गः॒ । \newline

\textbf{Ghana Paata } \newline

1. आ॒ग्ने॒यो᳚ ऽष्टाक॑पालो॒ ऽष्टाक॑पाल आग्ने॒य आ᳚ग्ने॒यो᳚ ऽष्टाक॑पालः सौ॒म्यः सौ॒म्यो᳚ ऽष्टाक॑पाल आग्ने॒य आ᳚ग्ने॒यो᳚ ऽष्टाक॑पालः सौ॒म्यः । \newline
2. अ॒ष्टाक॑पालः सौ॒म्यः सौ॒म्यो᳚ ऽष्टाक॑पालो॒ ऽष्टाक॑पालः सौ॒म्य श्च॒रु श्च॒रुः सौ॒म्यो᳚ ऽष्टाक॑पालो॒ ऽष्टाक॑पालः सौ॒म्य श्च॒रुः । \newline
3. अ॒ष्टाक॑पाल॒ इत्य॒ष्टा - क॒पा॒लः॒ । \newline
4. सौ॒म्य श्च॒रु श्च॒रुः सौ॒म्यः सौ॒म्य श्च॒रुः सा॑वि॒त्रः सा॑वि॒त्र श्च॒रुः सौ॒म्यः सौ॒म्य श्च॒रुः सा॑वि॒त्रः । \newline
5. च॒रुः सा॑वि॒त्रः सा॑वि॒त्र श्च॒रु श्च॒रुः सा॑वि॒त्रो᳚ ऽष्टाक॑पालो॒ ऽष्टाक॑पालः सावि॒त्र श्च॒रु श्च॒रुः सा॑वि॒त्रो᳚ ऽष्टाक॑पालः । \newline
6. सा॒वि॒त्रो᳚ ऽष्टाक॑पालो॒ ऽष्टाक॑पालः सावि॒त्रः सा॑वि॒त्रो᳚ ऽष्टाक॑पालः पौ॒ष्णः पौ॒ष्णो᳚ ऽष्टाक॑पालः सावि॒त्रः सा॑वि॒त्रो᳚ ऽष्टाक॑पालः पौ॒ष्णः । \newline
7. अ॒ष्टाक॑पालः पौ॒ष्णः पौ॒ष्णो᳚ ऽष्टाक॑पालो॒ ऽष्टाक॑पालः पौ॒ष्ण श्च॒रु श्च॒रुः पौ॒ष्णो᳚ ऽष्टाक॑पालो॒ ऽष्टाक॑पालः पौ॒ष्ण श्च॒रुः । \newline
8. अ॒ष्टाक॑पाल॒ इत्य॒ष्टा - क॒पा॒लः॒ । \newline
9. पौ॒ष्ण श्च॒रु श्च॒रुः पौ॒ष्णः पौ॒ष्ण श्च॒रू रौ॒द्रो रौ॒द्र श्च॒रुः पौ॒ष्णः पौ॒ष्ण श्च॒रू रौ॒द्रः । \newline
10. च॒रू रौ॒द्रो रौ॒द्र श्च॒रु श्च॒रू रौ॒द्र श्च॒रु श्च॒रू रौ॒द्र श्च॒रु श्च॒रू रौ॒द्र श्च॒रुः । \newline
11. रौ॒द्र श्च॒रु श्च॒रू रौ॒द्रो रौ॒द्र श्च॒रु र॒ग्नये॒ ऽग्नये॑ च॒रू रौ॒द्रो रौ॒द्र श्च॒रु र॒ग्नये᳚ । \newline
12. च॒रु र॒ग्नये॒ ऽग्नये॑ च॒रु श्च॒रु र॒ग्नये॑ वैश्वान॒राय॑ वैश्वान॒राया॒ ग्नये॑ च॒रु श्च॒रु र॒ग्नये॑ वैश्वान॒राय॑ । \newline
13. अ॒ग्नये॑ वैश्वान॒राय॑ वैश्वान॒राया॒ ग्नये॒ ऽग्नये॑ वैश्वान॒राय॒ द्वाद॑शकपालो॒ द्वाद॑शकपालो वैश्वान॒राया॒ ग्नये॒ ऽग्नये॑ वैश्वान॒राय॒ द्वाद॑शकपालः । \newline
14. वै॒श्वा॒न॒राय॒ द्वाद॑शकपालो॒ द्वाद॑शकपालो वैश्वान॒राय॑ वैश्वान॒राय॒ द्वाद॑शकपालो मृगाख॒रे मृ॑गाख॒रे द्वाद॑शकपालो वैश्वान॒राय॑ वैश्वान॒राय॒ द्वाद॑शकपालो मृगाख॒रे । \newline
15. द्वाद॑शकपालो मृगाख॒रे मृ॑गाख॒रे द्वाद॑शकपालो॒ द्वाद॑शकपालो मृगाख॒रे यदि॒ यदि॑ मृगाख॒रे द्वाद॑शकपालो॒ द्वाद॑शकपालो मृगाख॒रे यदि॑ । \newline
16. द्वाद॑शकपाल॒ इति॒ द्वाद॑श - क॒पा॒लः॒ । \newline
17. मृ॒गा॒ख॒रे यदि॒ यदि॑ मृगाख॒रे मृ॑गाख॒रे यदि॒ न न यदि॑ मृगाख॒रे मृ॑गाख॒रे यदि॒ न । \newline
18. मृ॒गा॒ख॒र इति॑ मृग - आ॒ख॒रे । \newline
19. यदि॒ न न यदि॒ यदि॒ नागच्छे॑ दा॒गच्छे॒न् न यदि॒ यदि॒ नागच्छे᳚त् । \newline
20. नागच्छे॑ दा॒गच्छे॒न् न नागच्छे॑ द॒ग्नये॒ ऽग्नय॑ आ॒गच्छे॒न् न नागच्छे॑ द॒ग्नये᳚ । \newline
21. आ॒गच्छे॑ द॒ग्नये॒ ऽग्नय॑ आ॒गच्छे॑ दा॒गच्छे॑ द॒ग्नये ऽꣳ॑हो॒मुचे ऽꣳ॑हो॒मुचे॒ ऽग्नय॑ आ॒गच्छे॑ दा॒गच्छे॑ द॒ग्नये ऽꣳ॑हो॒मुचे᳚ । \newline
22. आ॒गच्छे॒दित्या᳚ - गच्छे᳚त् । \newline
23. अ॒ग्नये ऽꣳ॑हो॒मुचे ऽꣳ॑हो॒मुचे॒ ऽग्नये॒ ऽग्नये ऽꣳ॑हो॒मुचे॒ ऽष्टाक॑पालो॒ ऽष्टाक॑पालो ऽꣳहो॒मुचे॒ ऽग्नये॒ ऽग्नये ऽꣳ॑हो॒मुचे॒ ऽष्टाक॑पालः । \newline
24. अꣳ॒॒हो॒मुचे॒ ऽष्टाक॑पालो॒ ऽष्टाक॑पालो ऽꣳहो॒मुचे ऽꣳ॑हो॒मुचे॒ ऽष्टाक॑पालः सौ॒र्यꣳ सौ॒र्य म॒ष्टाक॑पालो ऽꣳहो॒मुचे ऽꣳ॑हो॒मुचे॒ ऽष्टाक॑पालः सौ॒र्यम् । \newline
25. अꣳ॒॒हो॒मुच॒ इत्यꣳ॑हः - मुचे᳚ । \newline
26. अ॒ष्टाक॑पालः सौ॒र्यꣳ सौ॒र्य म॒ष्टाक॑पालो॒ ऽष्टाक॑पालः सौ॒र्यम् पयः॒ पयः॑ सौ॒र्य म॒ष्टाक॑पालो॒ ऽष्टाक॑पालः सौ॒र्यम् पयः॑ । \newline
27. अ॒ष्टाक॑पाल॒ इत्य॒ष्टा - क॒पा॒लः॒ । \newline
28. सौ॒र्यम् पयः॒ पयः॑ सौ॒र्यꣳ सौ॒र्यम् पयो॑ वाय॒व्यो॑ वाय॒व्यः॑ पयः॑ सौ॒र्यꣳ सौ॒र्यम् पयो॑ वाय॒व्यः॑ । \newline
29. पयो॑ वाय॒व्यो॑ वाय॒व्यः॑ पयः॒ पयो॑ वाय॒व्य॑ आज्य॑भाग॒ आज्य॑भागो वाय॒व्यः॑ पयः॒ पयो॑ वाय॒व्य॑ आज्य॑भागः । \newline
30. वा॒य॒व्य॑ आज्य॑भाग॒ आज्य॑भागो वाय॒व्यो॑ वाय॒व्य॑ आज्य॑भागः । \newline
31. आज्य॑भाग॒ इत्याज्य॑ - भा॒गः॒ । \newline
\pagebreak
\markright{ TS 7.5.22.1  \hfill https://www.vedavms.in \hfill}

\section{ TS 7.5.22.1 }

\textbf{TS 7.5.22.1 } \newline
\textbf{Samhita Paata} \newline

अ॒ग्नये-ऽꣳ॑हो॒मुचे॒-ऽष्टाक॑पाल॒ इन्द्रा॑याऽꣳहो॒मुच॒ एका॑दशकपालो मि॒त्रावरु॑णाभ्या-मागो॒मुग्भ्यां᳚ पय॒स्या॑ वायोसावि॒त्र आ॑गो॒मुग्भ्यां᳚ च॒रुर॒श्विभ्या॑-मागो॒मुग्भ्यां᳚ धा॒ना म॒रुद्भ्य॑ एनो॒मुग्भ्यः॑ स॒प्तक॑पालो॒ विश्वे᳚भ्यो दे॒वेभ्य॑ एनो॒मुग्भ्यो॒ द्वाद॑शकपा॒लो ऽनु॑मत्यै च॒रुर॒ग्नये॑ वैश्वान॒राय॒ द्वाद॑शकपालो॒ द्यावा॑पृथि॒वीभ्या॑-मꣳहो॒मुग्भ्यां᳚ द्विकपा॒लः ॥ \newline

\textbf{Pada Paata} \newline

अ॒ग्नये᳚ । अꣳ॒॒हो॒मुच॒ इत्यꣳ॑हः-मुचे᳚ । अष्टाक॑पाल॒ इत्य॒ष्टा-क॒पा॒लः॒ । इन्द्रा॑य । अꣳ॒॒हो॒मुच॒ इत्यꣳ॑हः - मुचे᳚ । एका॑दशकपाल॒ इत्येका॑दश - क॒पा॒लः॒ । मि॒त्रावरु॑णाभ्या॒मिति॑ मि॒त्रा - वरु॑णाभ्याम् । आ॒गो॒मुग्भ्या॒मित्या॑गो॒मुक् - भ्या॒म् । प॒य॒स्या᳚ । वा॒यो॒सा॒वि॒त्र इति॑ वायो - सा॒वि॒त्रः । आ॒गो॒मुग्भ्या॒मित्या॑गो॒मुक् - भ्या॒म् । च॒रुः । अ॒श्विभ्या॒मित्य॒श्वि - भ्या॒म् । अ॒गो॒मुग्भ्या॒मित्या॑गो॒मुक् - भ्या॒म् । धा॒नाः । म॒रुद्भ्य॒ इति॑ म॒रुत् - भ्यः॒ । ए॒नो॒मुग्भ्य॒ इत्ये॑नो॒मुक्-भ्यः॒ । स॒प्तक॑पाल॒ इति॑ स॒प्त - क॒पा॒लः॒ । विश्वे᳚भ्यः । दे॒वेभ्यः॑ । ए॒नो॒मुग्भ्य॒ इत्ये॑नो॒मुक् - भ्यः॒ । द्वाद॑शकपाल॒ इति॒ द्वाद॑श - क॒पा॒लः॒ । अनु॑मत्या॒ इत्यनु॑ - म॒त्यै॒ । च॒रुः । अ॒ग्नये᳚ । वै॒श्वा॒न॒राय॑ । द्वाद॑शकपाल॒ इति॒ द्वाद॑श - क॒पा॒लः॒ । द्यावा॑पृथि॒वीभ्या॒मिति॒ द्यावा᳚ - पृ॒थि॒वीभ्या᳚म् । अꣳ॒॒हो॒मुग्भ्या॒मित्यꣳ॑हो॒मुक् - भ्या॒म् । द्वि॒क॒पा॒ल इति॑ द्वि - क॒पा॒लः ॥  \newline


\textbf{Krama Paata} \newline

अ॒ग्नयेऽꣳ॑हो॒मुचे᳚ । अꣳ॒॒हो॒मुचे॒ऽष्टाक॑पालः । अꣳ॒॒हो॒मुच॒ इत्यꣳ॑हः - मुचे᳚ । अ॒ष्टाक॑पाल॒ इन्द्रा॑य । अ॒ष्टाक॑पाल॒ इत्य॒ष्टा - क॒पा॒लः॒ । इन्द्रा॑याꣳहो॒मुचे᳚ । अꣳ॒॒हो॒मुच॒ एका॑दशकपालः । अꣳ॒॒हो॒मुच॒ इत्यꣳ॑हः - मुचे᳚ । एका॑दशकपालो मि॒त्रावरु॑णाभ्याम् । एका॑दशकपाल॒ इत्येका॑दश - क॒पा॒लः॒ । मि॒त्रावरु॑णाभ्यामागो॒मुग्भ्या᳚म् । मि॒त्रावरु॑णाभ्या॒मिति॑ मि॒त्रा - वरु॑णाभ्याम् । आ॒गो॒मुग्भ्या᳚म् पय॒स्या᳚ । आ॒गो॒मुग्भ्या॒मित्या॑गो॒मुक् - भ्या॒म् । प॒य॒स्या॑ वायोसावि॒त्रः । वा॒यो॒सा॒वि॒त्र आ॑गो॒मुग्भ्या᳚म् । वा॒यो॒सा॒वि॒त्र इति॑ वायो - सा॒वि॒त्रः । आ॒गो॒मुग्भ्या᳚म् च॒रुः । आ॒गो॒मुभ्या॒मित्या॑गो॒मुक् - भ्या॒म् । च॒रुर॒श्विभ्या᳚म् । अ॒श्विभ्या॑मागो॒मुग्भ्या᳚म् । अ॒श्विभ्या॒मित्य॒श्वि - भ्या॒म् । आ॒गो॒मुग्भ्या᳚म् धा॒नाः । आ॒गो॒मुग्भ्या॒मित्या॑गो॒मुक् - भ्या॒म् । धा॒ना म॒रुद्भ्यः॑ । म॒रुद्भ्य॑ एनो॒मुग्भ्यः॑ । म॒रुद्भ्य॒ इति॑ म॒रुत् - भ्यः॒ । ए॒नो॒मुग्भ्यः॑ स॒प्तक॑पालः । ए॒नो॒मुग्भ्य॒ इत्ये॑नो॒मुक् - भ्यः॒ । स॒प्तक॑पालो॒ विश्वे᳚भ्यः । स॒प्तक॑पाल॒ इति॑ स॒प्त - क॒पा॒लः॒ । विश्वे᳚भ्यो दे॒वेभ्यः॑ । दे॒वेभ्य॑ एनो॒मुग्भ्यः॑ । ए॒नो॒मुग्भ्यो॒ द्वाद॑शकपालः । ए॒नो॒मुग्भ्य॒ इत्ये॑नो॒मुक् - भ्यः॒ । द्वाद॑शकपा॒लोऽनु॑मत्यै । द्वाद॑शकपाल॒ इति॒ द्वाद॑श - क॒पा॒लः॒ । अनु॑मत्यै च॒रुः । अनु॑मत्या॒ इत्यनु॑ - म॒त्यै॒ । च॒रुर॒ग्नये᳚ । अ॒ग्नये॑ वैश्वान॒राय॑ । वै॒श्वा॒न॒राय॒ द्वाद॑शकपालः । द्वाद॑शकपालो॒ द्यावा॑पृथि॒वीभ्या᳚म् । द्वाद॑शकपाल॒ इति॒ द्वाद॑श - क॒पा॒लः॒ । द्यावा॑पृथि॒वीभ्या॑मꣳहो॒मुग्भ्या᳚म् । द्यावा॑पृथि॒वीभ्या॒मिति॒ द्यावा᳚ - पृ॒थि॒वीभ्या᳚म् । अꣳ॒॒हो॒मुग्भ्या᳚म् द्विकपा॒लः । 
अꣳ॒॒हो॒मुग्भ्या॒मित्यꣳ॑हो॒मुक् - भ्या॒म् । द्वि॒क॒पा॒ल इति॑ द्वि - क॒पा॒लः । \newline

\textbf{Jatai Paata} \newline

1. अ॒ग्नये ऽꣳ॑हो॒मुचे ऽꣳ॑हो॒मुचे॒ ऽग्नये॒ ऽग्नये ऽꣳ॑हो॒मुचे᳚ । \newline
2. अꣳ॒॒हो॒मुचे॒ ऽष्टाक॑पालो॒ ऽष्टाक॑पालो ऽꣳहो॒मुचे ऽꣳ॑हो॒मुचे॒ ऽष्टाक॑पालः । \newline
3. अꣳ॒॒हो॒मुच॒ इत्यꣳ॑हः - मुचे᳚ । \newline
4. अ॒ष्टाक॑पाल॒ इन्द्रा॒ येन्द्रा॑या॒ ष्टाक॑पालो॒ ऽष्टाक॑पाल॒ इन्द्रा॑य । \newline
5. अ॒ष्टाक॑पाल॒ इत्य॒ष्टा - क॒पा॒लः॒ । \newline
6. इन्द्रा॑ याꣳहो॒मुचे ऽꣳ॑हो॒मुच॒ इन्द्रा॒ येन्द्रा॑या ꣳहो॒मुचे᳚ । \newline
7. अꣳ॒॒हो॒मुच॒ एका॑दशकपाल॒ एका॑दशकपालो ऽꣳहो॒मुचे ऽꣳ॑हो॒मुच॒ एका॑दशकपालः । \newline
8. अꣳ॒॒हो॒मुच॒ इत्यꣳ॑हः - मुचे᳚ । \newline
9. एका॑दशकपालो मि॒त्रावरु॑णाभ्याम् मि॒त्रावरु॑णाभ्या॒ मेका॑दशकपाल॒ एका॑दशकपालो मि॒त्रावरु॑णाभ्याम् । \newline
10. एका॑दशकपाल॒ इत्येका॑दश - क॒पा॒लः॒ । \newline
11. मि॒त्रावरु॑णाभ्या मागो॒मुग्भ्या॑ मागो॒मुग्भ्या᳚म् मि॒त्रावरु॑णाभ्याम् मि॒त्रावरु॑णाभ्या मागो॒मुग्भ्या᳚म् । \newline
12. मि॒त्रावरु॑णाभ्या॒मिति॑ मि॒त्रा - वरु॑णाभ्याम् । \newline
13. आ॒गो॒मुग्भ्या᳚म् पय॒स्या॑ पय॒स्या॑ ऽऽगो॒मुग्भ्या॑ मागो॒मुग्भ्या᳚म् पय॒स्या᳚ । \newline
14. आ॒गो॒मुग्भ्या॒मित्या॑गो॒मुक् - भ्या॒म् । \newline
15. प॒य॒स्या॑ वायोसावि॒त्रो वा॑योसावि॒त्रः प॑य॒स्या॑ पय॒स्या॑ वायोसावि॒त्रः । \newline
16. वा॒यो॒सा॒वि॒त्र आ॑गो॒मुग्भ्या॑ मागो॒मुग्भ्यां᳚ ॅवायोसावि॒त्रो वा॑योसावि॒त्र आ॑गो॒मुग्भ्या᳚म् । \newline
17. वा॒यो॒सा॒वि॒त्र इति॑ वायो - सा॒वि॒त्रः । \newline
18. आ॒गो॒मुग्भ्या᳚म् च॒रु श्च॒रु रा॑गो॒मुग्भ्या॑ मागो॒मुग्भ्या᳚म् च॒रुः । \newline
19. आ॒गो॒मुग्भ्या॒मित्या॑गो॒मुक् - भ्या॒म् । \newline
20. च॒रु र॒श्विभ्या॑ म॒श्विभ्या᳚म् च॒रु श्च॒रु र॒श्विभ्या᳚म् । \newline
21. अ॒श्विभ्या॑ मागो॒मुग्भ्या॑ मागो॒मुग्भ्या॑ म॒श्विभ्या॑ म॒श्विभ्या॑ मागो॒मुग्भ्या᳚म् । \newline
22. अ॒श्विभ्या॒मित्य॒श्वि - भ्या॒म् । \newline
23. आ॒गो॒मुग्भ्या᳚म् धा॒ना धा॒ना आ॑गो॒मुग्भ्या॑ मागो॒मुग्भ्या᳚म् धा॒नाः । \newline
24. आ॒गो॒मुग्भ्या॒मित्या॑गो॒मुक् - भ्या॒म् । \newline
25. धा॒ना म॒रुद्भ्यो॑ म॒रुद्भ्यो॑ धा॒ना धा॒ना म॒रुद्भ्यः॑ । \newline
26. म॒रुद्भ्य॑ एनो॒मुग्भ्य॑ एनो॒मुग्भ्यो॑ म॒रुद्भ्यो॑ म॒रुद्भ्य॑ एनो॒मुग्भ्यः॑ । \newline
27. म॒रुद्भ्य॒ इति॑ म॒रुत् - भ्यः॒ । \newline
28. ए॒नो॒मुग्भ्यः॑ स॒प्तक॑पालः स॒प्तक॑पाल एनो॒मुग्भ्य॑ एनो॒मुग्भ्यः॑ स॒प्तक॑पालः । \newline
29. ए॒नो॒मुग्भ्य॒ इत्ये॑नो॒मुक् - भ्यः॒ । \newline
30. स॒प्तक॑पालो॒ विश्वे᳚भ्यो॒ विश्वे᳚भ्यः स॒प्तक॑पालः स॒प्तक॑पालो॒ विश्वे᳚भ्यः । \newline
31. स॒प्तक॑पाल॒ इति॑ स॒प्त - क॒पा॒लः॒ । \newline
32. विश्वे᳚भ्यो दे॒वेभ्यो॑ दे॒वेभ्यो॒ विश्वे᳚भ्यो॒ विश्वे᳚भ्यो दे॒वेभ्यः॑ । \newline
33. दे॒वेभ्य॑ एनो॒मुग्भ्य॑ एनो॒मुग्भ्यो॑ दे॒वेभ्यो॑ दे॒वेभ्य॑ एनो॒मुग्भ्यः॑ । \newline
34. ए॒नो॒मुग्भ्यो॒ द्वाद॑शकपालो॒ द्वाद॑शकपाल एनो॒मुग्भ्य॑ एनो॒मुग्भ्यो॒ द्वाद॑शकपालः । \newline
35. ए॒नो॒मुग्भ्य॒ इत्ये॑नो॒मुक् - भ्यः॒ । \newline
36. द्वाद॑शकपा॒लो ऽनु॑मत्या॒ अनु॑मत्यै॒ द्वाद॑शकपालो॒ द्वाद॑शकपा॒लो ऽनु॑मत्यै । \newline
37. द्वाद॑शकपाल॒ इति॒ द्वाद॑श - क॒पा॒लः॒ । \newline
38. अनु॑मत्यै च॒रु श्च॒रु रनु॑मत्या॒ अनु॑मत्यै च॒रुः । \newline
39. अनु॑मत्या॒ इत्यनु॑ - म॒त्यै॒ । \newline
40. च॒रु र॒ग्नये॒ ऽग्नये॑ च॒रु श्च॒रु र॒ग्नये᳚ । \newline
41. अ॒ग्नये॑ वैश्वान॒राय॑ वैश्वान॒राया॒ ग्नये॒ ऽग्नये॑ वैश्वान॒राय॑ । \newline
42. वै॒श्वा॒न॒राय॒ द्वाद॑शकपालो॒ द्वाद॑शकपालो वैश्वान॒राय॑ वैश्वान॒राय॒ द्वाद॑शकपालः । \newline
43. द्वाद॑शकपालो॒ द्यावा॑पृथि॒वीभ्या॒म् द्यावा॑पृथि॒वीभ्या॒म् द्वाद॑शकपालो॒ द्वाद॑शकपालो॒ द्यावा॑पृथि॒वीभ्या᳚म् । \newline
44. द्वाद॑शकपाल॒ इति॒ द्वाद॑श - क॒पा॒लः॒ । \newline
45. द्यावा॑पृथि॒वीभ्या॑ मꣳहो॒मुग्भ्या॑ मꣳहो॒मुग्भ्या॒म् द्यावा॑पृथि॒वीभ्या॒म् द्यावा॑पृथि॒वीभ्या॑ मꣳहो॒मुग्भ्या᳚म् । \newline
46. द्यावा॑पृथि॒वीभ्या॒मिति॒ द्यावा᳚ - पृ॒थि॒वीभ्या᳚म् । \newline
47. अꣳ॒॒हो॒मुग्भ्या᳚म् द्विकपा॒लो द्वि॑कपा॒लो ऽꣳहो॒मुग्भ्या॑ मꣳहो॒मुग्भ्या᳚म् द्विकपा॒लः । \newline
48. अꣳ॒॒हो॒मुग्भ्या॒मित्यꣳ॑हो॒मुक् - भ्या॒म् । \newline
49. द्वि॒क॒पा॒ल इति॑ द्वि - क॒पा॒लः । \newline

\textbf{Ghana Paata } \newline

1. अ॒ग्नये ऽꣳ॑हो॒मुचे ऽꣳ॑हो॒मुचे॒ ऽग्नये॒ ऽग्नये ऽꣳ॑हो॒मुचे॒ ऽष्टाक॑पालो॒ ऽष्टाक॑पालो ऽꣳहो॒मुचे॒ ऽग्नये॒ ऽग्नये ऽꣳ॑हो॒मुचे॒ ऽष्टाक॑पालः । \newline
2. अꣳ॒॒हो॒मुचे॒ ऽष्टाक॑पालो॒ ऽष्टाक॑पालो ऽꣳहो॒मुचे ऽꣳ॑हो॒मुचे॒ ऽष्टाक॑पाल॒ 
इन्द्रा॒ येन्द्रा॑या॒ ष्टाक॑पालो ऽꣳहो॒मुचे ऽꣳ॑हो॒मुचे॒ ऽष्टाक॑पाल॒ इन्द्रा॑य । \newline
3. अꣳ॒॒हो॒मुच॒ इत्यꣳ॑हः - मुचे᳚ । \newline
4. अ॒ष्टाक॑पाल॒ इन्द्रा॒ येन्द्रा॑या॒ ष्टाक॑पालो॒ ऽष्टाक॑पाल॒ इन्द्रा॑या ꣳहो॒मुचे ऽꣳ॑हो॒मुच॒ इन्द्रा॑या॒ ष्टाक॑पालो॒ ऽष्टाक॑पाल॒ इन्द्रा॑या ꣳहो॒मुचे᳚ । \newline
5. अ॒ष्टाक॑पाल॒ इत्य॒ष्टा - क॒पा॒लः॒ । \newline
6. इन्द्रा॑या ꣳहो॒मुचे ऽꣳ॑हो॒मुच॒ इन्द्रा॒ येन्द्रा॑या ꣳहो॒मुच॒ एका॑दशकपाल॒ एका॑दशकपालो ऽꣳहो॒मुच॒ इन्द्रा॒ येन्द्रा॑या ꣳहो॒मुच॒ एका॑दशकपालः । \newline
7. अꣳ॒॒हो॒मुच॒ एका॑दशकपाल॒ एका॑दशकपालो ऽꣳहो॒मुचे ऽꣳ॑हो॒मुच॒ एका॑दशकपालो मि॒त्रावरु॑णाभ्याम् मि॒त्रावरु॑णाभ्या॒ मेका॑दशकपालो ऽꣳहो॒मुचे ऽꣳ॑हो॒मुच॒ एका॑दशकपालो मि॒त्रावरु॑णाभ्याम् । \newline
8. अꣳ॒॒हो॒मुच॒ इत्यꣳ॑हः - मुचे᳚ । \newline
9. एका॑दशकपालो मि॒त्रावरु॑णाभ्याम् मि॒त्रावरु॑णाभ्या॒ मेका॑दशकपाल॒ एका॑दशकपालो मि॒त्रावरु॑णाभ्या मागो॒मुग्भ्या॑ मागो॒मुग्भ्या᳚म् मि॒त्रावरु॑णाभ्या॒ मेका॑दशकपाल॒ एका॑दशकपालो मि॒त्रावरु॑णाभ्या मागो॒मुग्भ्या᳚म् । \newline
10. एका॑दशकपाल॒ इत्येका॑दश - क॒पा॒लः॒ । \newline
11. मि॒त्रावरु॑णाभ्या मागो॒मुग्भ्या॑ मागो॒मुग्भ्या᳚म् मि॒त्रावरु॑णाभ्याम् मि॒त्रावरु॑णाभ्या मागो॒मुग्भ्या᳚म् पय॒स्या॑ पय॒स्या॑ ऽऽगो॒मुग्भ्या᳚म् मि॒त्रावरु॑णाभ्याम् मि॒त्रावरु॑णाभ्या मागो॒मुग्भ्या᳚म् पय॒स्या᳚ । \newline
12. मि॒त्रावरु॑णाभ्या॒मिति॑ मि॒त्रा - वरु॑णाभ्याम् । \newline
13. आ॒गो॒मुग्भ्या᳚म् पय॒स्या॑ पय॒स्या॑ ऽऽगो॒मुग्भ्या॑ मागो॒मुग्भ्या᳚म् पय॒स्या॑ वायोसावि॒त्रो वा॑योसावि॒त्रः प॑य॒स्या॑ ऽऽगो॒मुग्भ्या॑ मागो॒मुग्भ्या᳚म् पय॒स्या॑ वायोसावि॒त्रः । \newline
14. आ॒गो॒मुग्भ्या॒मित्या॑गो॒मुक् - भ्या॒म् । \newline
15. प॒य॒स्या॑ वायोसावि॒त्रो वा॑योसावि॒त्रः प॑य॒स्या॑ पय॒स्या॑ वायोसावि॒त्र आ॑गो॒मुग्भ्या॑ मागो॒मुग्भ्यां᳚ ॅवायोसावि॒त्रः प॑य॒स्या॑ पय॒स्या॑ वायोसावि॒त्र आ॑गो॒मुग्भ्या᳚म् । \newline
16. वा॒यो॒सा॒वि॒त्र आ॑गो॒मुग्भ्या॑ मागो॒मुग्भ्यां᳚ ॅवायोसावि॒त्रो वा॑योसावि॒त्र आ॑गो॒मुग्भ्या᳚म् च॒रु श्च॒रु रा॑गो॒मुग्भ्यां᳚ ॅवायोसावि॒त्रो वा॑योसावि॒त्र आ॑गो॒मुग्भ्या᳚म् च॒रुः । \newline
17. वा॒यो॒सा॒वि॒त्र इति॑ वायो - सा॒वि॒त्रः । \newline
18. आ॒गो॒मुग्भ्या᳚म् च॒रु श्च॒रु रा॑गो॒मुग्भ्या॑ मागो॒मुग्भ्या᳚म् च॒रु र॒श्विभ्या॑ म॒श्विभ्या᳚म् च॒रु रा॑गो॒मुग्भ्या॑ मागो॒मुग्भ्या᳚म् च॒रु र॒श्विभ्या᳚म् । \newline
19. आ॒गो॒मुग्भ्या॒मित्या॑गो॒मुक् - भ्या॒म् । \newline
20. च॒रु र॒श्विभ्या॑ म॒श्विभ्या᳚म् च॒रु श्च॒रु र॒श्विभ्या॑ मागो॒मुग्भ्या॑ मागो॒मुग्भ्या॑ म॒श्विभ्या᳚म् च॒रु श्च॒रु र॒श्विभ्या॑ मागो॒मुग्भ्या᳚म् । \newline
21. अ॒श्विभ्या॑ मागो॒मुग्भ्या॑ मागो॒मुग्भ्या॑ म॒श्विभ्या॑ म॒श्विभ्या॑ मागो॒मुग्भ्या᳚म् धा॒ना धा॒ना आ॑गो॒मुग्भ्या॑ म॒श्विभ्या॑ म॒श्विभ्या॑ मागो॒मुग्भ्या᳚म् धा॒नाः । \newline
22. अ॒श्विभ्या॒मित्य॒श्वि - भ्या॒म् । \newline
23. आ॒गो॒मुग्भ्या᳚म् धा॒ना धा॒ना आ॑गो॒मुग्भ्या॑ मागो॒मुग्भ्या᳚म् धा॒ना म॒रुद्भ्यो॑ म॒रुद्भ्यो॑ धा॒ना आ॑गो॒मुग्भ्या॑ मागो॒मुग्भ्या᳚म् धा॒ना म॒रुद्भ्यः॑ । \newline
24. आ॒गो॒मुग्भ्या॒मित्या॑गो॒मुक् - भ्या॒म् । \newline
25. धा॒ना म॒रुद्भ्यो॑ म॒रुद्भ्यो॑ धा॒ना धा॒ना म॒रुद्भ्य॑ एनो॒मुग्भ्य॑ एनो॒मुग्भ्यो॑ म॒रुद्भ्यो॑ धा॒ना धा॒ना म॒रुद्भ्य॑ एनो॒मुग्भ्यः॑ । \newline
26. म॒रुद्भ्य॑ एनो॒मुग्भ्य॑ एनो॒मुग्भ्यो॑ म॒रुद्भ्यो॑ म॒रुद्भ्य॑ एनो॒मुग्भ्यः॑ स॒प्तक॑पालः स॒प्तक॑पाल एनो॒मुग्भ्यो॑ म॒रुद्भ्यो॑ म॒रुद्भ्य॑ एनो॒मुग्भ्यः॑ स॒प्तक॑पालः । \newline
27. म॒रुद्भ्य॒ इति॑ म॒रुत् - भ्यः॒ । \newline
28. ए॒नो॒मुग्भ्यः॑ स॒प्तक॑पालः स॒प्तक॑पाल एनो॒मुग्भ्य॑ एनो॒मुग्भ्यः॑ स॒प्तक॑पालो॒ विश्वे᳚भ्यो॒ विश्वे᳚भ्यः स॒प्तक॑पाल एनो॒मुग्भ्य॑ एनो॒मुग्भ्यः॑ स॒प्तक॑पालो॒ विश्वे᳚भ्यः । \newline
29. ए॒नो॒मुग्भ्य॒ इत्ये॑नो॒मुक् - भ्यः॒ । \newline
30. स॒प्तक॑पालो॒ विश्वे᳚भ्यो॒ विश्वे᳚भ्यः स॒प्तक॑पालः स॒प्तक॑पालो॒ विश्वे᳚भ्यो दे॒वेभ्यो॑ दे॒वेभ्यो॒ विश्वे᳚भ्यः स॒प्तक॑पालः स॒प्तक॑पालो॒ विश्वे᳚भ्यो दे॒वेभ्यः॑ । \newline
31. स॒प्तक॑पाल॒ इति॑ स॒प्त - क॒पा॒लः॒ । \newline
32. विश्वे᳚भ्यो दे॒वेभ्यो॑ दे॒वेभ्यो॒ विश्वे᳚भ्यो॒ विश्वे᳚भ्यो दे॒वेभ्य॑ एनो॒मुग्भ्य॑ एनो॒मुग्भ्यो॑ दे॒वेभ्यो॒ विश्वे᳚भ्यो॒ विश्वे᳚भ्यो दे॒वेभ्य॑ एनो॒मुग्भ्यः॑ । \newline
33. दे॒वेभ्य॑ एनो॒मुग्भ्य॑ एनो॒मुग्भ्यो॑ दे॒वेभ्यो॑ दे॒वेभ्य॑ एनो॒मुग्भ्यो॒ द्वाद॑शकपालो॒ द्वाद॑शकपाल एनो॒मुग्भ्यो॑ दे॒वेभ्यो॑ दे॒वेभ्य॑ एनो॒मुग्भ्यो॒ द्वाद॑शकपालः । \newline
34. ए॒नो॒मुग्भ्यो॒ द्वाद॑शकपालो॒ द्वाद॑शकपाल एनो॒मुग्भ्य॑ एनो॒मुग्भ्यो॒ द्वाद॑शकपा॒लो ऽनु॑मत्या॒ अनु॑मत्यै॒ द्वाद॑शकपाल एनो॒मुग्भ्य॑ एनो॒मुग्भ्यो॒ द्वाद॑शकपा॒लो ऽनु॑मत्यै । \newline
35. ए॒नो॒मुग्भ्य॒ इत्ये॑नो॒मुक् - भ्यः॒ । \newline
36. द्वाद॑शकपा॒लो ऽनु॑मत्या॒ अनु॑मत्यै॒ द्वाद॑शकपालो॒ द्वाद॑शकपा॒लो ऽनु॑मत्यै च॒रु श्च॒रु रनु॑मत्यै॒ द्वाद॑शकपालो॒ द्वाद॑शकपा॒लो ऽनु॑मत्यै च॒रुः । \newline
37. द्वाद॑शकपाल॒ इति॒ द्वाद॑श - क॒पा॒लः॒ । \newline
38. अनु॑मत्यै च॒रु श्च॒रु रनु॑मत्या॒ अनु॑मत्यै च॒रु र॒ग्नये॒ ऽग्नये॑ च॒रु रनु॑मत्या॒ अनु॑मत्यै च॒रु र॒ग्नये᳚ । \newline
39. अनु॑मत्या॒ इत्यनु॑ - म॒त्यै॒ । \newline
40. च॒रु र॒ग्नये॒ ऽग्नये॑ च॒रु श्च॒रु र॒ग्नये॑ वैश्वान॒राय॑ वैश्वान॒राया॒ ग्नये॑ च॒रु श्च॒रु र॒ग्नये॑ वैश्वान॒राय॑ । \newline
41. अ॒ग्नये॑ वैश्वान॒राय॑ वैश्वान॒राया॒ ग्नये॒ ऽग्नये॑ वैश्वान॒राय॒ द्वाद॑शकपालो॒ द्वाद॑शकपालो वैश्वान॒राया॒ ग्नये॒ ऽग्नये॑ वैश्वान॒राय॒ द्वाद॑शकपालः । \newline
42. वै॒श्वा॒न॒राय॒ द्वाद॑शकपालो॒ द्वाद॑शकपालो वैश्वान॒राय॑ वैश्वान॒राय॒ द्वाद॑शकपालो॒ द्यावा॑पृथि॒वीभ्या॒म् द्यावा॑पृथि॒वीभ्या॒म् द्वाद॑शकपालो वैश्वान॒राय॑ वैश्वान॒राय॒ द्वाद॑शकपालो॒ द्यावा॑पृथि॒वीभ्या᳚म् । \newline
43. द्वाद॑शकपालो॒ द्यावा॑पृथि॒वीभ्या॒म् द्यावा॑पृथि॒वीभ्या॒म् द्वाद॑शकपालो॒ द्वाद॑शकपालो॒ द्यावा॑पृथि॒वीभ्या॑ मꣳहो॒मुग्भ्या॑ मꣳहो॒मुग्भ्या॒म् द्यावा॑पृथि॒वीभ्या॒म् द्वाद॑शकपालो॒ द्वाद॑शकपालो॒ द्यावा॑पृथि॒वीभ्या॑ मꣳहो॒मुग्भ्या᳚म् । \newline
44. द्वाद॑शकपाल॒ इति॒ द्वाद॑श - क॒पा॒लः॒ । \newline
45. द्यावा॑पृथि॒वीभ्या॑ मꣳहो॒मुग्भ्या॑ मꣳहो॒मुग्भ्या॒म् द्यावा॑पृथि॒वीभ्या॒म् द्यावा॑पृथि॒वीभ्या॑ मꣳहो॒मुग्भ्या᳚म् द्विकपा॒लो द्वि॑कपा॒लो ऽꣳहो॒मुग्भ्या॒म् द्यावा॑पृथि॒वीभ्या॒म् द्यावा॑पृथि॒वीभ्या॑ मꣳहो॒मुग्भ्या᳚म् द्विकपा॒लः । \newline
46. द्यावा॑पृथि॒वीभ्या॒मिति॒ द्यावा᳚ - पृ॒थि॒वीभ्या᳚म् । \newline
47. अꣳ॒॒हो॒मुग्भ्या᳚म् द्विकपा॒लो द्वि॑कपा॒लो ऽꣳहो॒मुग्भ्या॑ मꣳहो॒मुग्भ्या᳚म् द्विकपा॒लः । \newline
48. अꣳ॒॒हो॒मुग्भ्या॒मित्यꣳ॑हो॒मुक् - भ्या॒म् । \newline
49. द्वि॒क॒पा॒ल इति॑ द्वि - क॒पा॒लः । \newline
\pagebreak
\markright{ TS 7.5.23.1  \hfill https://www.vedavms.in \hfill}

\section{ TS 7.5.23.1 }

\textbf{TS 7.5.23.1 } \newline
\textbf{Samhita Paata} \newline

अ॒ग्नये॒ सम॑नमत् पृथि॒व्यै सम॑नम॒द्यथा॒ऽग्निः पृ॑थि॒व्या स॒मन॑मदे॒वं मह्यं॑ भ॒द्राः संन॑तयः॒ सं न॑मन्तु वा॒यवे॒ सम॑नमद॒न्तरि॑क्षाय॒ सम॑नम॒द्यथा॑ वा॒युर॒न्तरि॑क्षेण॒ सूर्या॑य॒ सम॑नमद्दि॒वे सम॑नम॒द्यथा॒ सूर्यो॑ दि॒वा च॒न्द्रम॑से॒ सम॑नम॒न्नक्ष॑त्रेभ्यः॒ सम॑नम॒द्यथा॑ च॒न्द्रमा॒ नक्ष॑त्रै॒र्वरु॑णाय॒ सम॑नमद॒द्भ्यः सम॑नम॒द्यथा॒ - [  ] \newline

\textbf{Pada Paata} \newline

अ॒ग्नये᳚ । समिति॑ । अ॒न॒म॒त् । पृ॒थि॒व्यै । समिति॑ । अ॒न॒म॒त् । यथा᳚ । अ॒ग्निः । पृ॒थि॒व्या । स॒मन॑म॒दिति॑ सं - अन॑मत् । ए॒वम् । मह्य᳚म् । भ॒द्राः । संन॑तय॒ इति॒ सं - न॒त॒यः॒ । समिति॑ । न॒म॒न्तु॒ । वा॒यवे᳚ । समिति॑ । अ॒न॒म॒त् । अ॒न्तरि॑क्षाय । समिति॑ । अ॒न॒म॒त् । यथा᳚ । वा॒युः । अ॒न्तरि॑क्षेण । सूर्या॑य । समिति॑ । अ॒न॒म॒त् । दि॒वे । समिति॑ । अ॒न॒म॒त् । यथा᳚ । सूर्यः॑ । दि॒वा । च॒न्द्रम॑से । समिति॑ । अ॒न॒म॒त् । नक्ष॑त्रेभ्यः । समिति॑ । अ॒न॒म॒त् । यथा᳚ । च॒न्द्रमाः᳚ । नक्ष॑त्रैः । वरु॑णाय । समिति॑ । अ॒न॒म॒त् । अ॒द्भ्य इत्य॑त् - भ्यः । समिति॑ । अ॒न॒म॒त्॒ । यथा᳚ ।  \newline


\textbf{Krama Paata} \newline

अ॒ग्नये॒ सम् । सम॑नमत् । अ॒न॒म॒त् पृ॒थि॒व्यै । पृ॒थि॒व्यै सम् । सम॑नमत् । अ॒न॒म॒द् यथा᳚ । यथा॒ऽग्निः । अ॒ग्निः पृ॑थि॒व्या । पृ॒थि॒व्या स॒मन॑मत् । स॒मन॑मदे॒वम् । स॒मन॑म॒दिति॑ सम् - अन॑मत् । ए॒वम् मह्य᳚म् । मह्य॑म् भ॒द्राः । भ॒द्राः सन्न॑तयः । सन्न॑तयः॒ सम् । सन्न॑तय॒ इति॒ सम् - न॒त॒यः॒ । सम् न॑मन्तु । न॒म॒न्तु॒ वा॒यवे᳚ । वा॒यवे॒ सम् । सम॑नमत् । अ॒न॒म॒द॒न्तरि॑क्षाय । अ॒न्तरि॑क्षाय॒ सम् । सम॑नमत् । अ॒न॒म॒द् यथा᳚ । यथा॑ वा॒युः । वा॒युर॒न्तरि॑क्षेण । अ॒न्तरि॑क्षेण॒ सूर्या॑य । सूर्या॑य॒ सम् । सम॑नमत् । अ॒न॒म॒द् दि॒वे । दि॒वे सम् । सम॑नमत् । अ॒न॒म॒द् यथा᳚ । यथा॒ सूर्यः॑ । सूर्यो॑ दि॒वा । दि॒वा च॒न्द्रम॑से । च॒न्द्रम॑से॒ सम् । सम॑नमत् । अ॒न॒म॒न् नक्ष॑त्रेभ्यः । नक्ष॑त्रेभ्यः॒ सम् । सम॑नमत् । अ॒न॒म॒द् यथा᳚ । यथा॑ च॒न्द्रमाः᳚ । च॒न्द्रमा॒ नक्ष॑त्रैः । नक्ष॑त्रै॒र् वरु॑णाय । वरु॑णाय॒ सम् । सम॑नमत् । अ॒न॒म॒द॒द्भ्यः । अ॒द्भ्यः सम् । अ॒द्भ्य इत्य॑त् - भ्यः । सम॑नमत् । अ॒न॒म॒द् यथा᳚ । यथा॒ वरु॑णः \newline

\textbf{Jatai Paata} \newline

1. अ॒ग्नये॒ सꣳ स म॒ग्नये॒ ऽग्नये॒ सम् । \newline
2. स म॑नम दनम॒थ् सꣳ स म॑नमत् । \newline
3. अ॒न॒म॒त् पृ॒थि॒व्यै पृ॑थि॒व्या अ॑नम दनमत् पृथि॒व्यै । \newline
4. पृ॒थि॒व्यै सꣳ सम् पृ॑थि॒व्यै पृ॑थि॒व्यै सम् । \newline
5. स म॑नम दनम॒थ् सꣳ स म॑नमत् । \newline
6. अ॒न॒म॒द् यथा॒ यथा॑ ऽनम दनम॒द् यथा᳚ । \newline
7. यथा॒ ऽग्नि र॒ग्निर् यथा॒ यथा॒ ऽग्निः । \newline
8. अ॒ग्निः पृ॑थि॒व्या पृ॑थि॒व्या ऽग्निर॒ग्निः पृ॑थि॒व्या । \newline
9. पृ॒थि॒व्या स॒मन॑मथ् स॒मन॑मत् पृथि॒व्या पृ॑थि॒व्या स॒मन॑मत् । \newline
10. स॒मन॑म दे॒व मे॒वꣳ स॒मन॑मथ् स॒मन॑म दे॒वम् । \newline
11. स॒मन॑म॒दिति॑ सं - अन॑मत् । \newline
12. ए॒वम् मह्य॒म् मह्य॑ मे॒व मे॒वम् मह्य᳚म् । \newline
13. मह्य॑म् भ॒द्रा भ॒द्रा मह्य॒म् मह्य॑म् भ॒द्राः । \newline
14. भ॒द्राः सन्न॑तयः॒ सन्न॑तयो भ॒द्रा भ॒द्राः सन्न॑तयः । \newline
15. सन्न॑तयः॒ सꣳ सꣳ सन्न॑तयः॒ सन्न॑तयः॒ सम् । \newline
16. सन्न॑तय॒ इति॒ सं - न॒त॒यः॒ । \newline
17. सन् न॑मन्तु नमन्तु॒ सꣳ सन् न॑मन्तु । \newline
18. न॒म॒न्तु॒ वा॒यवे॑ वा॒यवे॑ नमन्तु नमन्तु वा॒यवे᳚ । \newline
19. वा॒यवे॒ सꣳ सं ॅवा॒यवे॑ वा॒यवे॒ सम् । \newline
20. स म॑नम दनम॒थ् सꣳ स म॑नमत् । \newline
21. अ॒न॒म॒ द॒न्तरि॑क्षाया॒ न्तरि॑क्षाया नम दनम द॒न्तरि॑क्षाय । \newline
22. अ॒न्तरि॑क्षाय॒ सꣳ स म॒न्तरि॑क्षाया॒ न्तरि॑क्षाय॒ सम् । \newline
23. स म॑नम दनम॒थ् सꣳ स म॑नमत् । \newline
24. अ॒न॒म॒द् यथा॒ यथा॑ ऽनम दनम॒द् यथा᳚ । \newline
25. यथा॑ वा॒युर् वा॒युर् यथा॒ यथा॑ वा॒युः । \newline
26. वा॒यु र॒न्तरि॑क्षेणा॒ न्तरि॑क्षेण वा॒युर् वा॒यु र॒न्तरि॑क्षेण । \newline
27. अ॒न्तरि॑क्षेण॒ सूर्या॑य॒ सूर्या॑या॒ न्तरि॑क्षेणा॒ न्तरि॑क्षेण॒ सूर्या॑य । \newline
28. सूर्या॑य॒ सꣳ सꣳ सूर्या॑य॒ सूर्या॑य॒ सम् । \newline
29. स म॑नम दनम॒थ् सꣳ स म॑नमत् । \newline
30. अ॒न॒म॒द् दि॒वे दि॒वे॑ ऽनम दनमद् दि॒वे । \newline
31. दि॒वे सꣳ सम् दि॒वे दि॒वे सम् । \newline
32. स म॑नम दनम॒थ् सꣳ स म॑नमत् । \newline
33. अ॒न॒म॒द् यथा॒ यथा॑ ऽनम दनम॒द् यथा᳚ । \newline
34. यथा॒ सूर्यः॒ सूर्यो॒ यथा॒ यथा॒ सूर्यः॑ । \newline
35. सूर्यो॑ दि॒वा दि॒वा सूर्यः॒ सूर्यो॑ दि॒वा । \newline
36. दि॒वा च॒न्द्रम॑से च॒न्द्रम॑से दि॒वा दि॒वा च॒न्द्रम॑से । \newline
37. च॒न्द्रम॑से॒ सꣳ सम् च॒न्द्रम॑से च॒न्द्रम॑से॒ सम् । \newline
38. स म॑नम दनम॒थ् सꣳ स म॑नमत् । \newline
39. अ॒न॒म॒न् नक्ष॑त्रेभ्यो॒ नक्ष॑त्रेभ्यो ऽनम दनम॒न् नक्ष॑त्रेभ्यः । \newline
40. नक्ष॑त्रेभ्यः॒ सꣳ सन् नक्ष॑त्रेभ्यो॒ नक्ष॑त्रेभ्यः॒ सम् । \newline
41. स म॑नम दनम॒थ् सꣳ स म॑नमत् । \newline
42. अ॒न॒म॒द् यथा॒ यथा॑ ऽनम दनम॒द् यथा᳚ । \newline
43. यथा॑ च॒न्द्रमा᳚ श्च॒न्द्रमा॒ यथा॒ यथा॑ च॒न्द्रमाः᳚ । \newline
44. च॒न्द्रमा॒ नक्ष॑त्रै॒र् नक्ष॑त्रै श्च॒न्द्रमा᳚ श्च॒न्द्रमा॒ नक्ष॑त्रैः । \newline
45. नक्ष॑त्रै॒र् वरु॑णाय॒ वरु॑णाय॒ नक्ष॑त्रै॒र् नक्ष॑त्रै॒र् वरु॑णाय । \newline
46. वरु॑णाय॒ सꣳ सं ॅवरु॑णाय॒ वरु॑णाय॒ सम् । \newline
47. स म॑नम दनम॒थ् सꣳ स म॑नमत् । \newline
48. अ॒न॒म॒ द॒द्भ्यो᳚(1॒) ऽद्भ्यो॑ ऽनम दनम द॒द्भ्यः । \newline
49. अ॒द्भ्यः सꣳ स म॒द्भ्यो᳚ ऽद्भ्यः सम् । \newline
50. अ॒द्भ्य इत्य॑त् - भ्यः । \newline
51. स म॑नम दनम॒थ् सꣳ स म॑नमत् । \newline
52. अ॒न॒म॒द् यथा॒ यथा॑ ऽनम दनम॒द् यथा᳚ । \newline
53. यथा॒ वरु॑णो॒ वरु॑णो॒ यथा॒ यथा॒ वरु॑णः । \newline

\textbf{Ghana Paata } \newline

1. अ॒ग्नये॒ सꣳ सम॒ग्नये॒ ऽग्नये॒ सम॑नम दनम॒थ् सम॒ग्नये॒ ऽग्नये॒ सम॑नमत् । \newline
2. सम॑नम दनम॒थ् सꣳ सम॑नमत् पृथि॒व्यै पृ॑थि॒व्या अ॑नम॒थ् सꣳ सम॑नमत् पृथि॒व्यै । \newline
3. अ॒न॒म॒त् पृ॒थि॒व्यै पृ॑थि॒व्या अ॑नम दनमत् पृथि॒व्यै सꣳ सम् पृ॑थि॒व्या अ॑नम दनमत् पृथि॒व्यै सम् । \newline
4. पृ॒थि॒व्यै सꣳ सम् पृ॑थि॒व्यै पृ॑थि॒व्यै सम॑नम दनम॒थ् सम् पृ॑थि॒व्यै पृ॑थि॒व्यै सम॑नमत् । \newline
5. सम॑नम दनम॒थ् सꣳ सम॑नम॒द् यथा॒ यथा॑ ऽनम॒थ् सꣳ सम॑नम॒द् यथा᳚ । \newline
6. अ॒न॒म॒द् यथा॒ यथा॑ ऽनम दनम॒द् यथा॒ ऽग्नि र॒ग्निर् यथा॑ ऽनम दनम॒द् यथा॒ ऽग्निः । \newline
7. यथा॒ ऽग्नि र॒ग्निर् यथा॒ यथा॒ ऽग्निः पृ॑थि॒व्या पृ॑थि॒व्या ऽग्निर् यथा॒ यथा॒ ऽग्निः पृ॑थि॒व्या । \newline
8. अ॒ग्निः पृ॑थि॒व्या पृ॑थि॒व्या ऽग्नि र॒ग्निः पृ॑थि॒व्या स॒मन॑मथ् स॒मन॑मत् पृथि॒व्या ऽग्नि र॒ग्निः पृ॑थि॒व्या स॒मन॑मत् । \newline
9. पृ॒थि॒व्या स॒मन॑मथ् स॒मन॑मत् पृथि॒व्या पृ॑थि॒व्या स॒मन॑म दे॒व मे॒वꣳ स॒मन॑मत् पृथि॒व्या पृ॑थि॒व्या स॒मन॑म दे॒वम् । \newline
10. स॒मन॑म दे॒व मे॒वꣳ स॒मन॑मथ् स॒मन॑म दे॒वम् मह्य॒म् मह्य॑ मे॒वꣳ स॒मन॑मथ् स॒मन॑म दे॒वम् मह्य᳚म् । \newline
11. स॒मन॑म॒दिति॑ सं - अन॑मत् । \newline
12. ए॒वम् मह्य॒म् मह्य॑ मे॒व मे॒वम् मह्य॑म् भ॒द्रा भ॒द्रा मह्य॑ मे॒व मे॒वम् मह्य॑म् भ॒द्राः । \newline
13. मह्य॑म् भ॒द्रा भ॒द्रा मह्य॒म् मह्य॑म् भ॒द्राः सन्न॑तयः॒ सन्न॑तयो भ॒द्रा मह्य॒म् मह्य॑म् भ॒द्राः सन्न॑तयः । \newline
14. भ॒द्राः सन्न॑तयः॒ सन्न॑तयो भ॒द्रा भ॒द्राः सन्न॑तयः॒ सꣳ सꣳ सन्न॑तयो भ॒द्रा भ॒द्राः सन्न॑तयः॒ सम् । \newline
15. सन्न॑तयः॒ सꣳ सꣳ सन्न॑तयः॒ सन्न॑तयः॒ सन् न॑मन्तु नमन्तु॒ सꣳ सन्न॑तयः॒ सन्न॑तयः॒ सन् न॑मन्तु । \newline
16. सन्न॑तय॒ इति॒ सं - न॒त॒यः॒ । \newline
17. सन् न॑मन्तु नमन्तु॒ सꣳ सन् न॑मन्तु वा॒यवे॑ वा॒यवे॑ नमन्तु॒ सꣳ सन् न॑मन्तु वा॒यवे᳚ । \newline
18. न॒म॒न्तु॒ वा॒यवे॑ वा॒यवे॑ नमन्तु नमन्तु वा॒यवे॒ सꣳ सं ॅवा॒यवे॑ नमन्तु नमन्तु वा॒यवे॒ सम् । \newline
19. वा॒यवे॒ सꣳ सं ॅवा॒यवे॑ वा॒यवे॒ सम॑नम दनम॒थ् सं ॅवा॒यवे॑ वा॒यवे॒ सम॑नमत् । \newline
20. सम॑नम दनम॒थ् सꣳ सम॑नम द॒न्तरि॑क्षाया॒ न्तरि॑क्षाया नम॒थ् सꣳ सम॑नम द॒न्तरि॑क्षाय । \newline
21. अ॒न॒म॒ द॒न्तरि॑क्षाया॒ न्तरि॑क्षाया नम दनम द॒न्तरि॑क्षाय॒ सꣳ सम॒न्तरि॑क्षाया नम दनम द॒न्तरि॑क्षाय॒ सम् । \newline
22. अ॒न्तरि॑क्षाय॒ सꣳ सम॒न्तरि॑क्षाया॒ न्तरि॑क्षाय॒ सम॑नम दनम॒थ् सम॒न्तरि॑क्षाया॒ न्तरि॑क्षाय॒ 
सम॑नमत् । \newline
23. सम॑नम दनम॒थ् सꣳ सम॑नम॒द् यथा॒ यथा॑ ऽनम॒थ् सꣳ सम॑नम॒द् यथा᳚ । \newline
24. अ॒न॒म॒द् यथा॒ यथा॑ ऽनम दनम॒द् यथा॑ वा॒युर् वा॒युर् यथा॑ ऽनम दनम॒द् यथा॑ वा॒युः । \newline
25. यथा॑ वा॒युर् वा॒युर् यथा॒ यथा॑ वा॒यु र॒न्तरि॑क्षेणा॒ न्तरि॑क्षेण वा॒युर् यथा॒ यथा॑ वा॒यु र॒न्तरि॑क्षेण । \newline
26. वा॒यु र॒न्तरि॑क्षेणा॒ न्तरि॑क्षेण वा॒युर् वा॒यु र॒न्तरि॑क्षेण॒ सूर्या॑य॒ सूर्या॑या॒ न्तरि॑क्षेण वा॒युर् वा॒यु र॒न्तरि॑क्षेण॒ सूर्या॑य । \newline
27. अ॒न्तरि॑क्षेण॒ सूर्या॑य॒ सूर्या॑या॒ न्तरि॑क्षेणा॒ न्तरि॑क्षेण॒ सूर्या॑य॒ सꣳ सꣳ सूर्या॑या॒ न्तरि॑क्षेणा॒ न्तरि॑क्षेण॒ सूर्या॑य॒ सम् । \newline
28. सूर्या॑य॒ सꣳ सꣳ सूर्या॑य॒ सूर्या॑य॒ सम॑नम दनम॒थ् सꣳ सूर्या॑य॒ सूर्या॑य॒ सम॑नमत् । \newline
29. स म॑नम दनम॒थ् सꣳ स म॑नमद् दि॒वे दि॒वे॑ ऽनम॒थ् सꣳ सम॑नमद् दि॒वे । \newline
30. अ॒न॒म॒द् दि॒वे दि॒वे॑ ऽनम दनमद् दि॒वे सꣳ सम् दि॒वे॑ ऽनम दनमद् दि॒वे सम् । \newline
31. दि॒वे सꣳ सम् दि॒वे दि॒वे सम॑नम दनम॒थ् सम् दि॒वे दि॒वे सम॑नमत् । \newline
32. सम॑नम दनम॒थ् सꣳ सम॑नम॒द् यथा॒ यथा॑ ऽनम॒थ् सꣳ सम॑नम॒द् यथा᳚ । \newline
33. अ॒न॒म॒द् यथा॒ यथा॑ ऽनम दनम॒द् यथा॒ सूर्यः॒ सूर्यो॒ यथा॑ ऽनम दनम॒द् यथा॒ सूर्यः॑ । \newline
34. यथा॒ सूर्यः॒ सूर्यो॒ यथा॒ यथा॒ सूर्यो॑ दि॒वा दि॒वा सूर्यो॒ यथा॒ यथा॒ सूर्यो॑ दि॒वा । \newline
35. सूर्यो॑ दि॒वा दि॒वा सूर्यः॒ सूर्यो॑ दि॒वा च॒न्द्रम॑से च॒न्द्रम॑से दि॒वा सूर्यः॒ सूर्यो॑ दि॒वा च॒न्द्रम॑से । \newline
36. दि॒वा च॒न्द्रम॑से च॒न्द्रम॑से दि॒वा दि॒वा च॒न्द्रम॑से॒ सꣳ सम् च॒न्द्रम॑से दि॒वा दि॒वा च॒न्द्रम॑से॒ सम् । \newline
37. च॒न्द्रम॑से॒ सꣳ सम् च॒न्द्रम॑से च॒न्द्रम॑से॒ सम॑नम दनम॒थ् सम् च॒न्द्रम॑से च॒न्द्रम॑से॒ 
सम॑नमत् । \newline
38. सम॑नम दनम॒थ् सꣳ सम॑नम॒न् नक्ष॑त्रेभ्यो॒ नक्ष॑त्रेभ्यो ऽनम॒थ् सꣳ सम॑नम॒न् नक्ष॑त्रेभ्यः । \newline
39. अ॒न॒म॒न् नक्ष॑त्रेभ्यो॒ नक्ष॑त्रेभ्यो ऽनम दनम॒न् नक्ष॑त्रेभ्यः॒ सꣳ सन् नक्ष॑त्रेभ्यो ऽनम दनम॒न् नक्ष॑त्रेभ्यः॒ सम् । \newline
40. नक्ष॑त्रेभ्यः॒ सꣳ सन् नक्ष॑त्रेभ्यो॒ नक्ष॑त्रेभ्यः॒ सम॑नम दनम॒थ् सन् नक्ष॑त्रेभ्यो॒ नक्ष॑त्रेभ्यः॒ सम॑नमत् । \newline
41. सम॑नम दनम॒थ् सꣳ सम॑नम॒द् यथा॒ यथा॑ ऽनम॒थ् सꣳ सम॑नम॒द् यथा᳚ । \newline
42. अ॒न॒म॒द् यथा॒ यथा॑ ऽनम दनम॒द् यथा॑ च॒न्द्रमा᳚ श्च॒न्द्रमा॒ यथा॑ ऽनम दनम॒द् यथा॑ च॒न्द्रमाः᳚ । \newline
43. यथा॑ च॒न्द्रमा᳚ श्च॒न्द्रमा॒ यथा॒ यथा॑ च॒न्द्रमा॒ नक्ष॑त्रै॒र् नक्ष॑त्रै श्च॒न्द्रमा॒ यथा॒ यथा॑ च॒न्द्रमा॒ नक्ष॑त्रैः । \newline
44. च॒न्द्रमा॒ नक्ष॑त्रै॒र् नक्ष॑त्रै श्च॒न्द्रमा᳚ श्च॒न्द्रमा॒ नक्ष॑त्रै॒र् वरु॑णाय॒ वरु॑णाय॒ नक्ष॑त्रै श्च॒न्द्रमा᳚ श्च॒न्द्रमा॒ नक्ष॑त्रै॒र् वरु॑णाय । \newline
45. नक्ष॑त्रै॒र् वरु॑णाय॒ वरु॑णाय॒ नक्ष॑त्रै॒र् नक्ष॑त्रै॒र् वरु॑णाय॒ सꣳ सं ॅवरु॑णाय॒ नक्ष॑त्रै॒र् नक्ष॑त्रै॒र् वरु॑णाय॒ सम् । \newline
46. वरु॑णाय॒ सꣳ सं ॅवरु॑णाय॒ वरु॑णाय॒ सम॑नम दनम॒थ् सं ॅवरु॑णाय॒ वरु॑णाय॒ सम॑नमत् । \newline
47. सम॑नम दनम॒थ् सꣳ सम॑नम द॒द्भ्यो᳚(1॒) ऽद्भ्यो॑ ऽनम॒थ् सꣳ सम॑नम द॒द्भ्यः । \newline
48. अ॒न॒म॒ द॒द्भ्यो᳚(1॒) ऽद्भ्यो॑ ऽनम दनम द॒द्भ्यः सꣳ सम॒द्भ्यो॑ ऽनम दनम द॒द्भ्यः सम् । \newline
49. अ॒द्भ्यः सꣳ सम॒द्भ्यो᳚ ऽद्भ्यः सम॑नम दनम॒थ् सम॒द्भ्यो᳚ ऽद्भ्यः सम॑नमत् । \newline
50. अ॒द्भ्य इत्य॑त् - भ्यः । \newline
51. सम॑नम दनम॒थ् सꣳ सम॑नम॒द् यथा॒ यथा॑ ऽनम॒थ् सꣳ सम॑नम॒द् यथा᳚ । \newline
52. अ॒न॒म॒द् यथा॒ यथा॑ ऽनम दनम॒द् यथा॒ वरु॑णो॒ वरु॑णो॒ यथा॑ ऽनम दनम॒द् यथा॒ वरु॑णः । \newline
53. यथा॒ वरु॑णो॒ वरु॑णो॒ यथा॒ यथा॒ वरु॑णो॒ ऽद्भि र॒द्भिर् वरु॑णो॒ यथा॒ यथा॒ वरु॑णो॒ ऽद्भिः । \newline
\pagebreak
\markright{ TS 7.5.23.2  \hfill https://www.vedavms.in \hfill}

\section{ TS 7.5.23.2 }

\textbf{TS 7.5.23.2 } \newline
\textbf{Samhita Paata} \newline

वरु॑णो॒ऽद्भिः साम्ने॒ सम॑नमदृ॒चे सम॑नम॒द्यथा॒ साम॒र्चा ब्रह्म॑णे॒ सम॑नमत् क्ष॒त्राय॒ सम॑नम॒द्यथा॒ ब्रह्म॑ क्ष॒त्रेण॒ राज्ञे॒ सम॑नमद्-वि॒शे सम॑नम॒द्यथा॒ राजा॑ वि॒शा रथा॑यः॒ सम॑नम॒दश्वे᳚भ्यः॒ सम॑नम॒द्यथा॒ रथोऽश्वैः᳚ प्र॒जाप॑तये॒ सम॑नमद्-भू॒तेभ्यः॒ सम॑नम॒द्यथा᳚ प्र॒जाप॑तिर्भू॒तैः स॒मन॑मदे॒वं मह्यं॑ ( ) भ॒द्राः संन॑तयः॒ सं न॑मन्तु ॥ \newline

\textbf{Pada Paata} \newline

वरु॑णः । अ॒द्भिरित्य॑त् - भिः । साम्ने᳚ । समिति॑ । अ॒न॒म॒त् । ऋ॒चे । समिति॑ । अ॒न॒म॒त् । यथा᳚ । साम॑ । ऋ॒चा । ब्रह्म॑णे । समिति॑ । अ॒न॒म॒त् । क्ष॒त्राय॑ । समिति॑ । अ॒न॒म॒त् । यथा᳚ । ब्रह्म॑ । क्ष॒त्रेण॑ । राज्ञे᳚ । समिति॑ । अ॒न॒म॒त् । वि॒शे । समिति॑ । अ॒न॒म॒त् । यथा᳚ । राजा᳚ । वि॒शा । रथा॑य । समिति॑ । अ॒न॒म॒त् । अश्वे᳚भ्यः । समिति॑ । अ॒न॒म॒त् । यथा᳚ । रथः॑ । अश्वैः᳚ । प्र॒जाप॑तय॒ इति॑ प्र॒जा - प॒त॒ये॒ । समिति॑ । अ॒न॒म॒त् । भू॒तेभ्यः॑ । समिति॑ । अ॒न॒म॒त् । यथा᳚ । प्र॒जाप॑ति॒रिति॑ प्र॒जा - प॒तिः॒ । भू॒तैः । स॒मन॑म॒दिति॑ सं-अन॑मत् । ए॒वम् । मह्य᳚म् ( ) । भ॒द्राः । संन॑तय॒ इति॒ सं - न॒त॒यः॒ । समिति॑ । न॒म॒न्तु॒ ॥  \newline


\textbf{Krama Paata} \newline

वरु॑णो॒ऽद्‌भिः । अ॒द्‌भिः साम्ने᳚ । अ॒द्‌भिरित्य॑त् - भिः । साम्ने॒ सम् । सम॑नमत् । अ॒न॒म॒दृ॒चे । ऋ॒चे सम् । सम॑नमत् । अ॒न॒म॒द् यथा᳚ । यथा॒ साम॑ । साम॒र्चा । ऋ॒चा ब्रह्म॑णे । ब्रह्म॑णे॒ सम् । सम॑नमत् । अ॒न॒म॒त् क्ष॒त्राय॑ । क्ष॒त्राय॒ सम् । सम॑नमत् । अ॒न॒म॒द् यथा᳚ । यथा॒ ब्रह्म॑ । ब्रह्म॑ क्ष॒त्रेण॑ । क्ष॒त्रेण॒ राज्ञे᳚ । राज्ञे॒ सम् । सम॑नमत् । अ॒न॒म॒द् वि॒शे । वि॒शे सम् । सम॑नमत् । अ॒न॒म॒द् यथा᳚ । यथा॒ राजा᳚ । राजा॑ वि॒शा । वि॒शा रथा॑य । रथा॑य॒ सम् । सम॑नमत् । अ॒न॒म॒दश्वे᳚भ्यः । अश्वे᳚भ्यः॒ सम् । सम॑नमत् । 
अ॒न॒म॒द् यथा᳚ । यथा॒ रथः॑ । रथोऽश्वैः᳚ । अश्वैः᳚ प्र॒जाप॑तये । प्र॒जाप॑तये॒ सम् । प्र॒जाप॑तय॒ इति॑ प्र॒जा - प॒त॒ये॒ । सम॑नमत् । अ॒न॒म॒द् भू॒तेभ्यः॑ । भू॒तेभ्यः॒ सम् । सम॑नमत् । अ॒न॒म॒द् यथा᳚ । यथा᳚ प्र॒जाप॑तिः । प्र॒जाप॑तिर् भू॒तैः । प्र॒जाप॑ति॒रिति॑ प्र॒जा - प॒तिः॒ । भू॒तैः स॒मन॑मत् । स॒मन॑मदे॒वम् । स॒मन॑म॒दिति॑ सम् - अन॑मत् । ए॒वम् मह्य᳚म् ( ) । मह्य॑म् भ॒द्राः । भ॒द्राः सन्न॑तयः । सन्न॑तयः॒ सम् । सन्न॑तय॒ इति॒ सम् - न॒त॒यः॒ । सम् न॑मन्तु । न॒म॒न्त्विति॑ नमन्तु । \newline

\textbf{Jatai Paata} \newline

1. वरु॑णो॒ ऽद्भि र॒द्भिर् वरु॑णो॒ वरु॑णो॒ ऽद्भिः । \newline
2. अ॒द्भिः साम्ने॒ साम्ने॒ ऽद्भि र॒द्भिः साम्ने᳚ । \newline
3. अ॒द्भिरित्य॑त् - भिः । \newline
4. साम्ने॒ सꣳ सꣳ साम्ने॒ साम्ने॒ सम् । \newline
5. स म॑नम दनम॒थ् सꣳ स म॑नमत् । \newline
6. अ॒न॒म॒ दृ॒च ऋ॒चे॑ ऽनम दनम दृ॒चे । \newline
7. ऋ॒चे सꣳ स मृ॒च ऋ॒चे सम् । \newline
8. स म॑नम दनम॒थ् सꣳ स म॑नमत् । \newline
9. अ॒न॒म॒द् यथा॒ यथा॑ ऽनम दनम॒द् यथा᳚ । \newline
10. यथा॒ साम॒ साम॒ यथा॒ यथा॒ साम॑ । \newline
11. साम॒ र्‌च र्‌चा साम॒ साम॒ र्‌चा । \newline
12. ऋ॒चा ब्रह्म॑णे॒ ब्रह्म॑ण ऋ॒च र्‌चा ब्रह्म॑णे । \newline
13. ब्रह्म॑णे॒ सꣳ सम् ब्रह्म॑णे॒ ब्रह्म॑णे॒ सम् । \newline
14. स म॑नम दनम॒थ् सꣳ स म॑नमत् । \newline
15. अ॒न॒म॒त् क्ष॒त्राय॑ क्ष॒त्राया॑ नम दनमत् क्ष॒त्राय॑ । \newline
16. क्ष॒त्राय॒ सꣳ सम् क्ष॒त्राय॑ क्ष॒त्राय॒ सम् । \newline
17. स म॑नम दनम॒थ् सꣳ स म॑नमत् । \newline
18. अ॒न॒म॒द् यथा॒ यथा॑ ऽनम दनम॒द् यथा᳚ । \newline
19. यथा॒ ब्रह्म॒ ब्रह्म॒ यथा॒ यथा॒ ब्रह्म॑ । \newline
20. ब्रह्म॑ क्ष॒त्रेण॑ क्ष॒त्रेण॒ ब्रह्म॒ ब्रह्म॑ क्ष॒त्रेण॑ । \newline
21. क्ष॒त्रेण॒ राज्ञे॒ राज्ञे᳚ क्ष॒त्रेण॑ क्ष॒त्रेण॒ राज्ञे᳚ । \newline
22. राज्ञे॒ सꣳ सꣳ राज्ञे॒ राज्ञे॒ सम् । \newline
23. स म॑नम दनम॒थ् सꣳ स म॑नमत् । \newline
24. अ॒न॒म॒द् वि॒शे वि॒शे॑ ऽनम दनमद् वि॒शे । \newline
25. वि॒शे सꣳ सं ॅवि॒शे वि॒शे सम् । \newline
26. स म॑नम दनम॒थ् सꣳ स म॑नमत् । \newline
27. अ॒न॒म॒द् यथा॒ यथा॑ ऽनम दनम॒द् यथा᳚ । \newline
28. यथा॒ राजा॒ राजा॒ यथा॒ यथा॒ राजा᳚ । \newline
29. राजा॑ वि॒शा वि॒शा राजा॒ राजा॑ वि॒शा । \newline
30. वि॒शा रथा॑य॒ रथा॑य वि॒शा वि॒शा रथा॑य । \newline
31. रथा॑य॒ सꣳ सꣳ रथा॑य॒ रथा॑य॒ सम् । \newline
32. स म॑नम दनम॒थ् सꣳ स म॑नमत् । \newline
33. अ॒न॒म॒ दश्वे॒भ्यो ऽश्वे᳚भ्यो ऽनम दनम॒ दश्वे᳚भ्यः । \newline
34. अश्वे᳚भ्यः॒ सꣳ स मश्वे॒भ्यो ऽश्वे᳚भ्यः॒ सम् । \newline
35. स म॑नम दनम॒थ् सꣳ स म॑नमत् । \newline
36. अ॒न॒म॒द् यथा॒ यथा॑ ऽनम दनम॒द् यथा᳚ । \newline
37. यथा॒ रथो॒ रथो॒ यथा॒ यथा॒ रथः॑ । \newline
38. रथो ऽश्वै॒ रश्वै॒ रथो॒ रथो ऽश्वैः᳚ । \newline
39. अश्वैः᳚ प्र॒जाप॑तये प्र॒जाप॑त॒ये ऽश्वै॒ रश्वैः᳚ प्र॒जाप॑तये । \newline
40. प्र॒जाप॑तये॒ सꣳ सम् प्र॒जाप॑तये प्र॒जाप॑तये॒ सम् । \newline
41. प्र॒जाप॑तय॒ इति॑ प्र॒जा - प॒त॒ये॒ । \newline
42. स म॑नम दनम॒थ् सꣳ स म॑नमत् । \newline
43. अ॒न॒म॒द् भू॒तेभ्यो॑ भू॒तेभ्यो॑ ऽनम दनमद् भू॒तेभ्यः॑ । \newline
44. भू॒तेभ्यः॒ सꣳ सम् भू॒तेभ्यो॑ भू॒तेभ्यः॒ सम् । \newline
45. स म॑नम दनम॒थ् सꣳ स म॑नमत् । \newline
46. अ॒न॒म॒द् यथा॒ यथा॑ ऽनम दनम॒द् यथा᳚ । \newline
47. यथा᳚ प्र॒जाप॑तिः प्र॒जाप॑ति॒र् यथा॒ यथा᳚ प्र॒जाप॑तिः । \newline
48. प्र॒जाप॑तिर् भू॒तैर् भू॒तैः प्र॒जाप॑तिः प्र॒जाप॑तिर् भू॒तैः । \newline
49. प्र॒जाप॑ति॒रिति॑ प्र॒जा - प॒तिः॒ । \newline
50. भू॒तैः स॒मन॑मथ् स॒मन॑मद् भू॒तैर् भू॒तैः स॒मन॑मत् । \newline
51. स॒मन॑म दे॒व मे॒वꣳ स॒मन॑मथ् स॒मन॑म दे॒वम् । \newline
52. स॒मन॑म॒दिति॑ सं - अन॑मत् । \newline
53. ए॒वम् मह्य॒म् मह्य॑ मे॒व मे॒वम् मह्य᳚म् । \newline
54. मह्य॑म् भ॒द्रा भ॒द्रा मह्य॒म् मह्य॑म् भ॒द्राः । \newline
55. भ॒द्राः सन्न॑तयः॒ सन्न॑तयो भ॒द्रा भ॒द्राः सन्न॑तयः । \newline
56. सन् न॑तयः॒ सꣳ सꣳ सन्न॑तयः॒ सन् न॑तयः॒ सम् । \newline
57. सन्न॑तय॒ इति॒ सं - न॒त॒यः॒ । \newline
58. सन् न॑मन्तु नमन्तु॒ सꣳ सन् न॑मन्तु । \newline
59. न॒म॒न्त्विति॑ नमन्तु । \newline

\textbf{Ghana Paata } \newline

1. वरु॑णो॒ ऽद्भि र॒द्भिर् वरु॑णो॒ वरु॑णो॒ ऽद्भिः साम्ने॒ साम्ने॒ ऽद्भिर् वरु॑णो॒ वरु॑णो॒ ऽद्भिः साम्ने᳚ । \newline
2. अ॒द्भिः साम्ने॒ साम्ने॒ ऽद्भिर॒द्भिः साम्ने॒ सꣳ सꣳ साम्ने॒ ऽद्भि र॒द्भिः साम्ने॒ सम् । \newline
3. अ॒द्भिरित्य॑त् - भिः । \newline
4. साम्ने॒ सꣳ सꣳ साम्ने॒ साम्ने॒ सम॑नम दनम॒थ् सꣳ साम्ने॒ साम्ने॒ सम॑नमत् । \newline
5. सम॑नम दनम॒थ् सꣳ सम॑नम दृ॒च ऋ॒चे॑ ऽनम॒थ् सꣳ सम॑नम दृ॒चे । \newline
6. अ॒न॒म॒ दृ॒च ऋ॒चे॑ ऽनम दनम दृ॒चे सꣳ समृ॒चे॑ ऽनम दनम दृ॒चे सम् । \newline
7. ऋ॒चे सꣳ समृ॒च ऋ॒चे सम॑नम दनम॒थ् समृ॒च ऋ॒चे सम॑नमत् । \newline
8. स म॑नम दनम॒थ् सꣳ स म॑नम॒द् यथा॒ यथा॑ ऽनम॒थ् सꣳ स म॑नम॒द् यथा᳚ । \newline
9. अ॒न॒म॒द् यथा॒ यथा॑ ऽनम दनम॒द् यथा॒ साम॒ साम॒ यथा॑ ऽनम दनम॒द् यथा॒ साम॑ । \newline
10. यथा॒ साम॒ साम॒ यथा॒ यथा॒ साम॒ र्‌च र्‌चा साम॒ यथा॒ यथा॒ साम॒ र्‌चा । \newline
11. साम॒ र्‌च र्‌चा साम॒ साम॒ र्‌चा ब्रह्म॑णे॒ ब्रह्म॑ण ऋ॒चा साम॒ साम॒ र्‌चा ब्रह्म॑णे । \newline
12. ऋ॒चा ब्रह्म॑णे॒ ब्रह्म॑ण ऋ॒च र्‌चा ब्रह्म॑णे॒ सꣳ सम् ब्रह्म॑ण ऋ॒च र्‌चा ब्रह्म॑णे॒ सम् । \newline
13. ब्रह्म॑णे॒ सꣳ सम् ब्रह्म॑णे॒ ब्रह्म॑णे॒ स म॑नम दनम॒थ् सम् ब्रह्म॑णे॒ ब्रह्म॑णे॒ स म॑नमत् । \newline
14. स म॑नम दनम॒थ् सꣳ स म॑नमत् क्ष॒त्राय॑ क्ष॒त्राया॑ नम॒थ् सꣳ स म॑नमत् क्ष॒त्राय॑ । \newline
15. अ॒न॒म॒त् क्ष॒त्राय॑ क्ष॒त्राया॑ नम दनमत् क्ष॒त्राय॒ सꣳ सम् क्ष॒त्राया॑ नम दनमत् क्ष॒त्राय॒ सम् । \newline
16. क्ष॒त्राय॒ सꣳ सम् क्ष॒त्राय॑ क्ष॒त्राय॒ स म॑नम दनम॒थ् सम् क्ष॒त्राय॑ क्ष॒त्राय॒ स म॑नमत् । \newline
17. स म॑नम दनम॒थ् सꣳ स म॑नम॒द् यथा॒ यथा॑ ऽनम॒थ् सꣳ स म॑नम॒द् यथा᳚ । \newline
18. अ॒न॒म॒द् यथा॒ यथा॑ ऽनम दनम॒द् यथा॒ ब्रह्म॒ ब्रह्म॒ यथा॑ ऽनम दनम॒द् यथा॒ ब्रह्म॑ । \newline
19. यथा॒ ब्रह्म॒ ब्रह्म॒ यथा॒ यथा॒ ब्रह्म॑ क्ष॒त्रेण॑ क्ष॒त्रेण॒ ब्रह्म॒ यथा॒ यथा॒ ब्रह्म॑ क्ष॒त्रेण॑ । \newline
20. ब्रह्म॑ क्ष॒त्रेण॑ क्ष॒त्रेण॒ ब्रह्म॒ ब्रह्म॑ क्ष॒त्रेण॒ राज्ञे॒ राज्ञे᳚ क्ष॒त्रेण॒ ब्रह्म॒ ब्रह्म॑ क्ष॒त्रेण॒ राज्ञे᳚ । \newline
21. क्ष॒त्रेण॒ राज्ञे॒ राज्ञे᳚ क्ष॒त्रेण॑ क्ष॒त्रेण॒ राज्ञे॒ सꣳ सꣳ राज्ञे᳚ क्ष॒त्रेण॑ क्ष॒त्रेण॒ राज्ञे॒ सम् । \newline
22. राज्ञे॒ सꣳ सꣳ राज्ञे॒ राज्ञे॒ स म॑नम दनम॒थ् सꣳ राज्ञे॒ राज्ञे॒ स म॑नमत् । \newline
23. स म॑नम दनम॒थ् सꣳ स म॑नमद् वि॒शे वि॒शे॑ ऽनम॒थ् सꣳ स म॑नमद् वि॒शे । \newline
24. अ॒न॒म॒द् वि॒शे वि॒शे॑ ऽनम दनमद् वि॒शे सꣳ सं ॅवि॒शे॑ ऽनम दनमद् वि॒शे सम् । \newline
25. वि॒शे सꣳ सं ॅवि॒शे वि॒शे स म॑नम दनम॒थ् सं ॅवि॒शे वि॒शे स म॑नमत् । \newline
26. स म॑नम दनम॒थ् सꣳ स म॑नम॒द् यथा॒ यथा॑ ऽनम॒थ् सꣳ स म॑नम॒द् यथा᳚ । \newline
27. अ॒न॒म॒द् यथा॒ यथा॑ ऽनम दनम॒द् यथा॒ राजा॒ राजा॒ यथा॑ ऽनम दनम॒द् यथा॒ राजा᳚ । \newline
28. यथा॒ राजा॒ राजा॒ यथा॒ यथा॒ राजा॑ वि॒शा वि॒शा राजा॒ यथा॒ यथा॒ राजा॑ वि॒शा । \newline
29. राजा॑ वि॒शा वि॒शा राजा॒ राजा॑ वि॒शा रथा॑य॒ रथा॑य वि॒शा राजा॒ राजा॑ वि॒शा रथा॑य । \newline
30. वि॒शा रथा॑य॒ रथा॑य वि॒शा वि॒शा रथा॑य॒ सꣳ सꣳ रथा॑य वि॒शा वि॒शा रथा॑य॒ सम् । \newline
31. रथा॑य॒ सꣳ सꣳ रथा॑य॒ रथा॑य॒ स म॑नम दनम॒थ् सꣳ रथा॑य॒ रथा॑य॒ स म॑नमत् । \newline
32. स म॑नम दनम॒थ् सꣳ स म॑नम॒ दश्वे॒भ्यो ऽश्वे᳚भ्यो ऽनम॒थ् सꣳ स म॑नम॒ दश्वे᳚भ्यः । \newline
33. अ॒न॒म॒ दश्वे॒भ्यो ऽश्वे᳚भ्यो ऽनम दनम॒ दश्वे᳚भ्यः॒ सꣳ स मश्वे᳚भ्यो ऽनम दनम॒ दश्वे᳚भ्यः॒ सम् । \newline
34. अश्वे᳚भ्यः॒ सꣳ स मश्वे॒भ्यो ऽश्वे᳚भ्यः॒ स म॑नम दनम॒थ् स मश्वे॒भ्यो ऽश्वे᳚भ्यः॒ स म॑नमत् । \newline
35. स म॑नम दनम॒थ् सꣳ स म॑नम॒द् यथा॒ यथा॑ ऽनम॒थ् सꣳ स म॑नम॒द् यथा᳚ । \newline
36. अ॒न॒म॒द् यथा॒ यथा॑ ऽनम दनम॒द् यथा॒ रथो॒ रथो॒ यथा॑ ऽनम दनम॒द् यथा॒ रथः॑ । \newline
37. यथा॒ रथो॒ रथो॒ यथा॒ यथा॒ रथो ऽश्वै॒ रश्वै॒ रथो॒ यथा॒ यथा॒ रथो ऽश्वैः᳚ । \newline
38. रथो ऽश्वै॒ रश्वै॒ रथो॒ रथो ऽश्वैः᳚ प्र॒जाप॑तये प्र॒जाप॑त॒ये ऽश्वै॒ रथो॒ रथो ऽश्वैः᳚ प्र॒जाप॑तये । \newline
39. अश्वैः᳚ प्र॒जाप॑तये प्र॒जाप॑त॒ये ऽश्वै॒ रश्वैः᳚ प्र॒जाप॑तये॒ सꣳ सम् प्र॒जाप॑त॒ये ऽश्वै॒ रश्वैः᳚ प्र॒जाप॑तये॒ सम् । \newline
40. प्र॒जाप॑तये॒ सꣳ सम् प्र॒जाप॑तये प्र॒जाप॑तये॒ स म॑नम दनम॒थ् सम् प्र॒जाप॑तये प्र॒जाप॑तये॒ स म॑नमत् । \newline
41. प्र॒जाप॑तय॒ इति॑ प्र॒जा - प॒त॒ये॒ । \newline
42. स म॑नम दनम॒थ् सꣳ स म॑नमद् भू॒तेभ्यो॑ भू॒तेभ्यो॑ ऽनम॒थ् सꣳ स म॑नमद् भू॒तेभ्यः॑ । \newline
43. अ॒न॒म॒द् भू॒तेभ्यो॑ भू॒तेभ्यो॑ ऽनम दनमद् भू॒तेभ्यः॒ सꣳ सम् भू॒तेभ्यो॑ ऽनम दनमद् भू॒तेभ्यः॒ सम् । \newline
44. भू॒तेभ्यः॒ सꣳ सम् भू॒तेभ्यो॑ भू॒तेभ्यः॒ स म॑नम दनम॒थ् सम् भू॒तेभ्यो॑ भू॒तेभ्यः॒ स म॑नमत् । \newline
45. स म॑नम दनम॒थ् सꣳ स म॑नम॒द् यथा॒ यथा॑ ऽनम॒थ् सꣳ स म॑नम॒द् यथा᳚ । \newline
46. अ॒न॒म॒द् यथा॒ यथा॑ ऽनम दनम॒द् यथा᳚ प्र॒जाप॑तिः प्र॒जाप॑ति॒र् यथा॑ ऽनम दनम॒द् यथा᳚ प्र॒जाप॑तिः । \newline
47. यथा᳚ प्र॒जाप॑तिः प्र॒जाप॑ति॒र् यथा॒ यथा᳚ प्र॒जाप॑तिर् भू॒तैर् भू॒तैः प्र॒जाप॑ति॒र् यथा॒ यथा᳚ प्र॒जाप॑तिर् भू॒तैः । \newline
48. प्र॒जाप॑तिर् भू॒तैर् भू॒तैः प्र॒जाप॑तिः प्र॒जाप॑तिर् भू॒तैः स॒मन॑मथ् स॒मन॑मद् भू॒तैः प्र॒जाप॑तिः प्र॒जाप॑तिर् भू॒तैः स॒मन॑मत् । \newline
49. प्र॒जाप॑ति॒रिति॑ प्र॒जा - प॒तिः॒ । \newline
50. भू॒तैः स॒मन॑मथ् स॒मन॑मद् भू॒तैर् भू॒तैः स॒मन॑म दे॒व मे॒वꣳ स॒मन॑मद् भू॒तैर् भू॒तैः स॒मन॑म दे॒वम् । \newline
51. स॒मन॑म दे॒व मे॒वꣳ स॒मन॑मथ् स॒मन॑म दे॒वम् मह्य॒म् मह्य॑ मे॒वꣳ स॒मन॑मथ् स॒मन॑म दे॒वम् मह्य᳚म् । \newline
52. स॒मन॑म॒दिति॑ सं - अन॑मत् । \newline
53. ए॒वम् मह्य॒म् मह्य॑ मे॒व मे॒वम् मह्य॑म् भ॒द्रा भ॒द्रा मह्य॑ मे॒व मे॒वम् मह्य॑म् भ॒द्राः । \newline
54. मह्य॑म् भ॒द्रा भ॒द्रा मह्य॒म् मह्य॑म् भ॒द्राः सन्न॑तयः॒ सन्न॑तयो भ॒द्रा मह्य॒म् मह्य॑म् भ॒द्राः सन्न॑तयः । \newline
55. भ॒द्राः सन्न॑तयः॒ सन्न॑तयो भ॒द्रा भ॒द्राः सन्न॑तयः॒ सꣳ सꣳ सन्न॑तयो भ॒द्रा भ॒द्राः सन्न॑तयः॒ सम् । \newline
56. सन्न॑तयः॒ सꣳ सꣳ सन्न॑तयः॒ सन्न॑तयः॒ सन्न॑मन्तु नमन्तु॒ सꣳ सन्न॑तयः॒ सन्न॑तयः॒ सन्न॑मन्तु । \newline
57. सन्न॑तय॒ इति॒ सं - न॒त॒यः॒ । \newline
58. सन् न॑मन्तु नमन्तु॒ सꣳ सन् न॑मन्तु । \newline
59. न॒म॒न्त्विति॑ नमन्तु । \newline
\pagebreak
\markright{ TS 7.5.24.1  \hfill https://www.vedavms.in \hfill}

\section{ TS 7.5.24.1 }

\textbf{TS 7.5.24.1 } \newline
\textbf{Samhita Paata} \newline

ये ते॒ पन्था॑नः सवितः पू॒र्व्यासो॑ऽरे॒णवो॒ वित॑ता अ॒न्तरि॑क्षे । तेभि॑र्नो अ॒द्य प॒थिभिः॑ सु॒गेभी॒ रक्षा॑ च नो॒ अधि॑ च देव ब्रूहि ॥नमो॒ऽग्नये॑ पृथिवि॒क्षिते॑ लोक॒स्पृते॑ लो॒कम॒स्मै यज॑मानाय देहि॒ नमो॑ वा॒यवे᳚ऽन्तरिक्ष॒क्षिते॑ लोक॒स्पृते॑ लो॒कम॒स्मै यज॑मानाय देहि॒ नमः॒ सूर्या॑य दिवि॒क्षिते॑ लोक॒स्पृते॑ लो॒कम॒स्मै यज॑मानाय देहि ॥ \newline

\textbf{Pada Paata} \newline

ये । त॒ । पन्था॑नः । स॒वि॒तः॒ । पू॒र्व्यासः॑ । अ॒रे॒णवः॑ । वित॑ता॒ इति॒ वि - त॒ताः॒ । अ॒न्तरि॑क्षे ॥ तेभिः॑ । नः॒ । अ॒द्य । प॒थिभि॒रिति॑ प॒थि - भिः॒ । सु॒गेभि॒रिति॑ सु - गेभिः॑ । रक्ष॑ । च॒ । नः॒ । अधीति॑ । च॒ । दे॒व॒ । ब्रू॒हि॒ ॥ नमः॑ । अ॒ग्नये᳚ । पृ॒थि॒वि॒क्षित॒ इति॑ पृथिवि-क्षिते᳚ । लो॒क॒स्पृत॒ इति॑ लोक - स्पृते᳚ । लो॒कम् । अ॒स्मै । यज॑मानाय । दे॒हि॒ । नमः॑ । वा॒यवे᳚ । अ॒न्त॒रि॒क्ष॒क्षित॒ इत्य॑न्तरिक्ष - क्षिते᳚ । लो॒क॒स्पृत॒ इति॑ लोक - स्पृते᳚ । लो॒कम् । अ॒स्मै । यज॑मानाय । दे॒हि॒ । नमः॑ । सूर्या॑य । दि॒वि॒क्षित॒ इति॑ दिवि - क्षिते᳚ । लो॒क॒स्पृत॒ इति॑ लोक - स्पृते᳚ । लो॒कम् । अ॒स्मै । यज॑मानाय । दे॒हि॒ ॥  \newline


\textbf{Krama Paata} \newline

ये ते᳚ । ते॒ पन्था॑नः । पन्था॑नः सवितः । स॒वि॒तः॒ पू॒र्व्यासः॑ । पू॒र्व्यासो॑ऽरे॒णवः॑ । अ॒रे॒णवो॒ वित॑ताः । वित॑ता अ॒न्तरि॑क्षे । वित॑ता॒ इति॒ वि - त॒ताः॒ । अ॒न्तरि॑क्ष॒ इत्य॒न्तरि॑क्षे ॥ तेभि॑र् नः । नो॒ अ॒द्य । अ॒द्य प॒थिभिः॑ । प॒थिभिः॑ सु॒गेभिः॑ । प॒थिभि॒रिति॑ प॒थि - भिः॒ । सु॒गेभी॒ रक्ष॑ । सु॒गेभि॒रिति॑ सु - गेभिः॑ । रक्षा॑ च । च॒ नः॒ । नो॒ अधि॑ । अधि॑ च । च॒ दे॒व॒ । दे॒व॒ ब्रू॒हि॒ । ब्रू॒हीति॑ ब्रूहि ॥ नमो॒ऽग्नये᳚ । अ॒ग्नये॑ पृथिवि॒क्षिते᳚ । पृ॒थि॒वि॒क्षिते॑ लोक॒स्पृते᳚ । पृ॒थि॒वि॒क्षित॒ इति॑ पृथिवि - क्षिते᳚ । लो॒क॒स्पृते॑ लो॒कम् । लो॒क॒स्पृत॒ इति॑ लोक - स्पृते᳚ । लो॒कम॒स्मै । अ॒स्मै यज॑मानाय । यज॑मानाय देहि । दे॒हि॒ नमः॑ । नमो॑ वा॒यवे᳚ । वा॒यवे᳚ऽन्तरिक्ष॒क्षिते᳚ । अ॒न्त॒रि॒क्ष॒क्षिते॑ लोक॒स्पृते᳚ । अ॒न्त॒रि॒क्ष॒क्षित॒ इत्य॑न्तरिक्ष - क्षिते᳚ । लो॒क॒स्पृते॑ लो॒कम् । लो॒क॒स्पृत॒ इति॑ लोक - स्पृते᳚ । लो॒कम॒स्मै । अ॒स्मै यज॑मानाय । यज॑मानाय देहि । दे॒हि॒ नमः॑ । नमः॒ सूर्या॑य । सूर्या॑य दिवि॒क्षिते᳚ । दि॒वि॒क्षिते॑ लोक॒स्पृते᳚ । दि॒वि॒क्षित॒ इति॑ दिवि - क्षिते᳚ । लो॒क॒स्पृते॑ लो॒कम् । लो॒क॒स्पृत॒ इति॑ लोक - स्पृते᳚ । लो॒कम॒स्मै । अ॒स्मै यज॑मानाय । यज॑मानाय देहि । दे॒हीति॑ देहि । \newline

\textbf{Jatai Paata} \newline

1. ये ते॑ ते॒ ये ये ते᳚ । \newline
2. ते॒ पन्था॑नः॒ पन्था॑नस्ते ते॒ पन्था॑नः । \newline
3. पन्था॑नः सवितः सवितः॒ पन्था॑नः॒ पन्था॑नः सवितः । \newline
4. स॒वि॒तः॒ पू॒र्व्यासः॑ पू॒र्व्यासः॑ सवितः सवितः पू॒र्व्यासः॑ । \newline
5. पू॒र्व्यासो॑ ऽरे॒णवो॑ ऽरे॒णवः॑ पू॒र्व्यासः॑ पू॒र्व्यासो॑ ऽरे॒णवः॑ । \newline
6. अ॒रे॒णवो॒ वित॑ता॒ वित॑ता अरे॒णवो॑ ऽरे॒णवो॒ वित॑ताः । \newline
7. वित॑ता अ॒न्तरि॑क्षे॒ ऽन्तरि॑क्षे॒ वित॑ता॒ वित॑ता अ॒न्तरि॑क्षे । \newline
8. वित॑ता॒ इति॒ वि - त॒ताः॒ । \newline
9. अ॒न्तरि॑क्ष॒ इत्य॒न्तरि॑क्षे । \newline
10. तेभि॑र् नो न॒ स्तेभि॒ स्तेभि॑र् नः । \newline
11. नो॒ अ॒द्याद्य नो॑ नो अ॒द्य । \newline
12. अ॒द्य प॒थिभिः॑ प॒थिभि॑ र॒द्याद्य प॒थिभिः॑ । \newline
13. प॒थिभिः॑ सु॒गेभिः॑ सु॒गेभिः॑ प॒थिभिः॑ प॒थिभिः॑ सु॒गेभिः॑ । \newline
14. प॒थिभि॒रिति॑ प॒थि - भिः॒ । \newline
15. सु॒गेभी॒ रक्ष॒ रक्ष॑ सु॒गेभिः॑ सु॒गेभी॒ रक्ष॑ । \newline
16. सु॒गेभि॒रिति॑ सु - गेभिः॑ । \newline
17. रक्षा॑ च च॒ रक्ष॒ रक्षा॑ च । \newline
18. च॒ नो॒ न॒ श्च॒ च॒ नः॒ । \newline
19. नो॒ अध्यधि॑ नो नो॒ अधि॑ । \newline
20. अधि॑ च॒ चाध्यधि॑ च । \newline
21. च॒ दे॒व॒ दे॒व॒ च॒ च॒ दे॒व॒ । \newline
22. दे॒व॒ ब्रू॒हि॒ ब्रू॒हि॒ दे॒व॒ दे॒व॒ ब्रू॒हि॒ । \newline
23. ब्रू॒हीति॑ ब्रूहि । \newline
24. नमो॒ ऽग्नये॒ ऽग्नये॒ नमो॒ नमो॒ ऽग्नये᳚ । \newline
25. अ॒ग्नये॑ पृथिवि॒क्षिते॑ पृथिवि॒क्षिते॒ ऽग्नये॒ ऽग्नये॑ पृथिवि॒क्षिते᳚ । \newline
26. पृ॒थि॒वि॒क्षिते॑ लोक॒स्पृते॑ लोक॒स्पृते॑ पृथिवि॒क्षिते॑ पृथिवि॒क्षिते॑ लोक॒स्पृते᳚ । \newline
27. पृ॒थि॒वि॒क्षित॒ इति॑ पृथिवि - क्षिते᳚ । \newline
28. लो॒क॒स्पृते॑ लो॒कम् ॅलो॒कम् ॅलो॑क॒स्पृते॑ लोक॒स्पृते॑ लो॒कम् । \newline
29. लो॒क॒स्पृत॒ इति॑ लोक - स्पृते᳚ । \newline
30. लो॒क म॒स्मा अ॒स्मै लो॒कम् ॅलो॒क म॒स्मै । \newline
31. अ॒स्मै यज॑मानाय॒ यज॑मानाया॒ स्मा अ॒स्मै यज॑मानाय । \newline
32. यज॑मानाय देहि देहि॒ यज॑मानाय॒ यज॑मानाय देहि । \newline
33. दे॒हि॒ नमो॒ नमो॑ देहि देहि॒ नमः॑ । \newline
34. नमो॑ वा॒यवे॑ वा॒यवे॒ नमो॒ नमो॑ वा॒यवे᳚ । \newline
35. वा॒यवे᳚ ऽन्तरिक्ष॒क्षिते᳚ ऽन्तरिक्ष॒क्षिते॑ वा॒यवे॑ वा॒यवे᳚ ऽन्तरिक्ष॒क्षिते᳚ । \newline
36. अ॒न्त॒रि॒क्ष॒क्षिते॑ लोक॒स्पृते॑ लोक॒स्पृते᳚ ऽन्तरिक्ष॒क्षिते᳚ ऽन्तरिक्ष॒क्षिते॑ लोक॒स्पृते᳚ । \newline
37. अ॒न्त॒रि॒क्ष॒क्षित॒ इत्य॑न्तरिक्ष - क्षिते᳚ । \newline
38. लो॒क॒स्पृते॑ लो॒कम् ॅलो॒कम् ॅलो॑क॒स्पृते॑ लोक॒स्पृते॑ लो॒कम् । \newline
39. लो॒क॒स्पृत॒ इति॑ लोक - स्पृते᳚ । \newline
40. लो॒क म॒स्मा अ॒स्मै लो॒कम् ॅलो॒क म॒स्मै । \newline
41. अ॒स्मै यज॑मानाय॒ यज॑मानाया॒ स्मा अ॒स्मै यज॑मानाय । \newline
42. यज॑मानाय देहि देहि॒ यज॑मानाय॒ यज॑मानाय देहि । \newline
43. दे॒हि॒ नमो॒ नमो॑ देहि देहि॒ नमः॑ । \newline
44. नमः॒ सूर्या॑य॒ सूर्या॑य॒ नमो॒ नमः॒ सूर्या॑य । \newline
45. सूर्या॑य दिवि॒क्षिते॑ दिवि॒क्षिते॒ सूर्या॑य॒ सूर्या॑य दिवि॒क्षिते᳚ । \newline
46. दि॒वि॒क्षिते॑ लोक॒स्पृते॑ लोक॒स्पृते॑ दिवि॒क्षिते॑ दिवि॒क्षिते॑ लोक॒स्पृते᳚ । \newline
47. दि॒वि॒क्षित॒ इति॑ दिवि - क्षिते᳚ । \newline
48. लो॒क॒स्पृते॑ लो॒कम् ॅलो॒कम् ॅलो॑क॒स्पृते॑ लोक॒स्पृते॑ लो॒कम् । \newline
49. लो॒क॒स्पृत॒ इति॑ लोक - स्पृते᳚ । \newline
50. लो॒क म॒स्मा अ॒स्मै लो॒कम् ॅलो॒क म॒स्मै । \newline
51. अ॒स्मै यज॑मानाय॒ यज॑मानाया॒ स्मा अ॒स्मै यज॑मानाय । \newline
52. यज॑मानाय देहि देहि॒ यज॑मानाय॒ यज॑मानाय देहि । \newline
53. दे॒हीति॑ देहि । \newline

\textbf{Ghana Paata } \newline

1. ये ते॑ ते॒ ये ये ते॒ पन्था॑नः॒ पन्था॑न स्ते॒ ये ये ते॒ पन्था॑नः । \newline
2. ते॒ पन्था॑नः॒ पन्था॑न स्ते ते॒ पन्था॑नः सवितः सवितः॒ पन्था॑न स्ते ते॒ पन्था॑नः सवितः । \newline
3. पन्था॑नः सवितः सवितः॒ पन्था॑नः॒ पन्था॑नः सवितः पू॒र्व्यासः॑ पू॒र्व्यासः॑ सवितः॒ पन्था॑नः॒ पन्था॑नः सवितः पू॒र्व्यासः॑ । \newline
4. स॒वि॒तः॒ पू॒र्व्यासः॑ पू॒र्व्यासः॑ सवितः सवितः पू॒र्व्यासो॑ ऽरे॒णवो॑ ऽरे॒णवः॑ पू॒र्व्यासः॑ सवितः सवितः पू॒र्व्यासो॑ ऽरे॒णवः॑ । \newline
5. पू॒र्व्यासो॑ ऽरे॒णवो॑ ऽरे॒णवः॑ पू॒र्व्यासः॑ पू॒र्व्यासो॑ ऽरे॒णवो॒ वित॑ता॒ वित॑ता अरे॒णवः॑ पू॒र्व्यासः॑ पू॒र्व्यासो॑ ऽरे॒णवो॒ वित॑ताः । \newline
6. अ॒रे॒णवो॒ वित॑ता॒ वित॑ता अरे॒णवो॑ ऽरे॒णवो॒ वित॑ता अ॒न्तरि॑क्षे॒ ऽन्तरि॑क्षे॒ वित॑ता अरे॒णवो॑ ऽरे॒णवो॒ वित॑ता अ॒न्तरि॑क्षे । \newline
7. वित॑ता अ॒न्तरि॑क्षे॒ ऽन्तरि॑क्षे॒ वित॑ता॒ वित॑ता अ॒न्तरि॑क्षे । \newline
8. वित॑ता॒ इति॒ वि - त॒ताः॒ । \newline
9. अ॒न्तरि॑क्ष॒ इत्य॒न्तरि॑क्षे । \newline
10. तेभि॑र् नो न॒ स्तेभि॒ स्तेभि॑र् नो अ॒द्याद्य न॒ स्तेभि॒ स्तेभि॑र् नो अ॒द्य । \newline
11. नो॒ अ॒द्याद्य नो॑ नो अ॒द्य प॒थिभिः॑ प॒थिभि॑ र॒द्य नो॑ नो अ॒द्य प॒थिभिः॑ । \newline
12. अ॒द्य प॒थिभिः॑ प॒थिभि॑ र॒द्याद्य प॒थिभिः॑ सु॒गेभिः॑ सु॒गेभिः॑ प॒थिभि॑ र॒द्याद्य प॒थिभिः॑ सु॒गेभिः॑ । \newline
13. प॒थिभिः॑ सु॒गेभिः॑ सु॒गेभिः॑ प॒थिभिः॑ प॒थिभिः॑ सु॒गेभी॒ रक्ष॒ रक्ष॑ सु॒गेभिः॑ प॒थिभिः॑ प॒थिभिः॑ सु॒गेभी॒ रक्ष॑ । \newline
14. प॒थिभि॒रिति॑ प॒थि - भिः॒ । \newline
15. सु॒गेभी॒ रक्ष॒ रक्ष॑ सु॒गेभिः॑ सु॒गेभी॒ रक्षा॑ च च॒ रक्ष॑ सु॒गेभिः॑ सु॒गेभी॒ रक्षा॑ च । \newline
16. सु॒गेभि॒रिति॑ सु - गेभिः॑ । \newline
17. रक्षा॑ च च॒ रक्ष॒ रक्षा॑ च नो न श्च॒ रक्ष॒ रक्षा॑ च नः । \newline
18. च॒ नो॒ न॒ श्च॒ च॒ नो॒ अध्यधि॑ न श्च च नो॒ अधि॑ । \newline
19. नो॒ अध्यधि॑ नो नो॒ अधि॑ च॒ चाधि॑ नो नो॒ अधि॑ च । \newline
20. अधि॑ च॒ चाध्यधि॑ च देव देव॒ चाध्यधि॑ च देव । \newline
21. च॒ दे॒व॒ दे॒व॒ च॒ च॒ दे॒व॒ ब्रू॒हि॒ ब्रू॒हि॒ दे॒व॒ च॒ च॒ दे॒व॒ ब्रू॒हि॒ । \newline
22. दे॒व॒ ब्रू॒हि॒ ब्रू॒हि॒ दे॒व॒ दे॒व॒ ब्रू॒हि॒ । \newline
23. ब्रू॒हीति॑ ब्रूहि । \newline
24. नमो॒ ऽग्नये॒ ऽग्नये॒ नमो॒ नमो॒ ऽग्नये॑ पृथिवि॒क्षिते॑ पृथिवि॒क्षिते॒ ऽग्नये॒ नमो॒ नमो॒ ऽग्नये॑ पृथिवि॒क्षिते᳚ । \newline
25. अ॒ग्नये॑ पृथिवि॒क्षिते॑ पृथिवि॒क्षिते॒ ऽग्नये॒ ऽग्नये॑ पृथिवि॒क्षिते॑ लोक॒स्पृते॑ लोक॒स्पृते॑ पृथिवि॒क्षिते॒ ऽग्नये॒ ऽग्नये॑ पृथिवि॒क्षिते॑ लोक॒स्पृते᳚ । \newline
26. पृ॒थि॒वि॒क्षिते॑ लोक॒स्पृते॑ लोक॒स्पृते॑ पृथिवि॒क्षिते॑ पृथिवि॒क्षिते॑ लोक॒स्पृते॑ लो॒कम् ॅलो॒कम् ॅलो॑क॒स्पृते॑ पृथिवि॒क्षिते॑ पृथिवि॒क्षिते॑ लोक॒स्पृते॑ लो॒कम् । \newline
27. पृ॒थि॒वि॒क्षित॒ इति॑ पृथिवि - क्षिते᳚ । \newline
28. लो॒क॒स्पृते॑ लो॒कम् ॅलो॒कम् ॅलो॑क॒स्पृते॑ लोक॒स्पृते॑ लो॒क म॒स्मा अ॒स्मै लो॒कम् ॅलो॑क॒स्पृते॑ लोक॒स्पृते॑ लो॒क म॒स्मै । \newline
29. लो॒क॒स्पृत॒ इति॑ लोक - स्पृते᳚ । \newline
30. लो॒क म॒स्मा अ॒स्मै लो॒कम् ॅलो॒क म॒स्मै यज॑मानाय॒ यज॑मानाया॒स्मै लो॒कम् ॅलो॒क म॒स्मै यज॑मानाय । \newline
31. अ॒स्मै यज॑मानाय॒ यज॑मानाया॒ स्मा अ॒स्मै यज॑मानाय देहि देहि॒ यज॑मानाया॒ स्मा अ॒स्मै यज॑मानाय देहि । \newline
32. यज॑मानाय देहि देहि॒ यज॑मानाय॒ यज॑मानाय देहि॒ नमो॒ नमो॑ देहि॒ यज॑मानाय॒ यज॑मानाय देहि॒ नमः॑ । \newline
33. दे॒हि॒ नमो॒ नमो॑ देहि देहि॒ नमो॑ वा॒यवे॑ वा॒यवे॒ नमो॑ देहि देहि॒ नमो॑ वा॒यवे᳚ । \newline
34. नमो॑ वा॒यवे॑ वा॒यवे॒ नमो॒ नमो॑ वा॒यवे᳚ ऽन्तरिक्ष॒क्षिते᳚ ऽन्तरिक्ष॒क्षिते॑ वा॒यवे॒ नमो॒ नमो॑ वा॒यवे᳚ ऽन्तरिक्ष॒क्षिते᳚ । \newline
35. वा॒यवे᳚ ऽन्तरिक्ष॒क्षिते᳚ ऽन्तरिक्ष॒क्षिते॑ वा॒यवे॑ वा॒यवे᳚ ऽन्तरिक्ष॒क्षिते॑ लोक॒स्पृते॑ लोक॒स्पृते᳚ ऽन्तरिक्ष॒क्षिते॑ वा॒यवे॑ वा॒यवे᳚ ऽन्तरिक्ष॒क्षिते॑ लोक॒स्पृते᳚ । \newline
36. अ॒न्त॒रि॒क्ष॒क्षिते॑ लोक॒स्पृते॑ लोक॒स्पृते᳚ ऽन्तरिक्ष॒क्षिते᳚ ऽन्तरिक्ष॒क्षिते॑ लोक॒स्पृते॑ लो॒कम् ॅलो॒कम् ॅलो॑क॒स्पृते᳚ ऽन्तरिक्ष॒क्षिते᳚ ऽन्तरिक्ष॒क्षिते॑ लोक॒स्पृते॑ लो॒कम् । \newline
37. अ॒न्त॒रि॒क्ष॒क्षित॒ इत्य॑न्तरिक्ष - क्षिते᳚ । \newline
38. लो॒क॒स्पृते॑ लो॒कम् ॅलो॒कम् ॅलो॑क॒स्पृते॑ लोक॒स्पृते॑ लो॒क म॒स्मा अ॒स्मै लो॒कम् ॅलो॑क॒स्पृते॑ लोक॒स्पृते॑ लो॒क म॒स्मै । \newline
39. लो॒क॒स्पृत॒ इति॑ लोक - स्पृते᳚ । \newline
40. लो॒क म॒स्मा अ॒स्मै लो॒कम् ॅलो॒क म॒स्मै यज॑मानाय॒ यज॑मानाया॒स्मै लो॒कम् ॅलो॒क म॒स्मै यज॑मानाय । \newline
41. अ॒स्मै यज॑मानाय॒ यज॑मानाया॒ स्मा अ॒स्मै यज॑मानाय देहि देहि॒ यज॑मानाया॒ स्मा अ॒स्मै यज॑मानाय देहि । \newline
42. यज॑मानाय देहि देहि॒ यज॑मानाय॒ यज॑मानाय देहि॒ नमो॒ नमो॑ देहि॒ यज॑मानाय॒ यज॑मानाय देहि॒ नमः॑ । \newline
43. दे॒हि॒ नमो॒ नमो॑ देहि देहि॒ नमः॒ सूर्या॑य॒ सूर्या॑य॒ नमो॑ देहि देहि॒ नमः॒ सूर्या॑य । \newline
44. नमः॒ सूर्या॑य॒ सूर्या॑य॒ नमो॒ नमः॒ सूर्या॑य दिवि॒क्षिते॑ दिवि॒क्षिते॒ सूर्या॑य॒ नमो॒ नमः॒ सूर्या॑य दिवि॒क्षिते᳚ । \newline
45. सूर्या॑य दिवि॒क्षिते॑ दिवि॒क्षिते॒ सूर्या॑य॒ सूर्या॑य दिवि॒क्षिते॑ लोक॒स्पृते॑ लोक॒स्पृते॑ दिवि॒क्षिते॒ सूर्या॑य॒ सूर्या॑य दिवि॒क्षिते॑ लोक॒स्पृते᳚ । \newline
46. दि॒वि॒क्षिते॑ लोक॒स्पृते॑ लोक॒स्पृते॑ दिवि॒क्षिते॑ दिवि॒क्षिते॑ लोक॒स्पृते॑ लो॒कम् ॅलो॒कम् ॅलो॑क॒स्पृते॑ दिवि॒क्षिते॑ दिवि॒क्षिते॑ लोक॒स्पृते॑ लो॒कम् । \newline
47. दि॒वि॒क्षित॒ इति॑ दिवि - क्षिते᳚ । \newline
48. लो॒क॒स्पृते॑ लो॒कम् ॅलो॒कम् ॅलो॑क॒स्पृते॑ लोक॒स्पृते॑ लो॒क म॒स्मा अ॒स्मै लो॒कम् ॅलो॑क॒स्पृते॑ लोक॒स्पृते॑ लो॒क म॒स्मै । \newline
49. लो॒क॒स्पृत॒ इति॑ लोक - स्पृते᳚ । \newline
50. लो॒क म॒स्मा अ॒स्मै लो॒कम् ॅलो॒क म॒स्मै यज॑मानाय॒ यज॑मानाया॒स्मै लो॒कम् ॅलो॒क म॒स्मै यज॑मानाय । \newline
51. अ॒स्मै यज॑मानाय॒ यज॑मानाया॒ स्मा अ॒स्मै यज॑मानाय देहि देहि॒ यज॑मानाया॒ स्मा अ॒स्मै यज॑मानाय देहि । \newline
52. यज॑मानाय देहि देहि॒ यज॑मानाय॒ यज॑मानाय देहि । \newline
53. दे॒हीति॑ देहि । \newline
\pagebreak
\markright{ TS 7.5.25.1  \hfill https://www.vedavms.in \hfill}

\section{ TS 7.5.25.1 }

\textbf{TS 7.5.25.1 } \newline
\textbf{Samhita Paata} \newline

यो वा अश्व॑स्य॒ मेद्ध्य॑स्य॒ शिरो॒ वेद॑ शीर्.ष॒ण्वान् मेद्ध्यो॑ भवत्यु॒षा वा अश्व॑स्य॒ मेद्ध्य॑स्य॒ शिरः॒ सूर्य॒श्चक्षु॒र्वातः॑ प्रा॒णश्च॒न्द्रमाः॒ श्रोत्रं॒ दिशः॒ पादा॑ अवान्तरदि॒शाः पर्.श॑वोऽहोरा॒त्रे नि॑मे॒षो᳚ऽर्द्धमा॒साः पर्वा॑णि॒ मासाः᳚ स॒धांना᳚न्यृ॒तवोऽङ्गा॑नि संॅवथ्स॒र आ॒त्मा र॒श्मयः॒ केशा॒ नक्ष॑त्राणि रू॒पं तार॑का अ॒स्थानि॒ नभो॑ माꣳ॒॒सान्योष॑धयो॒ लोमा॑नि॒ वन॒स्पत॑यो॒ वाला॑ अ॒ग्निर्मुखं॑ ॅवैश्वान॒रो व्यात्तꣳ॑ - [  ] \newline

\textbf{Pada Paata} \newline

यः । वै । अश्व॑स्य । मेद्ध्य॑स्य । शिरः॑ । वेद॑ । शी॒र्.॒ष॒ण्वानिति॑ शीर्.षण्-वान् । मेद्ध्यः॑ । भ॒व॒ति॒ । उ॒षाः । वै । अश्व॑स्य । मेद्ध्य॑स्य । शिरः॑ । सूर्यः॑ । चक्षुः॑ । वातः॑ । प्रा॒ण इति॑ प्र - अ॒नः । च॒न्द्रमाः᳚ । श्रोत्र᳚म् । दिशः॑ । पादाः᳚ । अ॒वा॒न्त॒र॒दि॒शा इत्य॑वान्तर - दि॒शाः । पर्.श॑वः । अ॒हो॒रा॒त्रे इत्य॑हः - रा॒त्रे । नि॒मे॒ष इति॑ नि - मे॒षः । अ॒द्‌र्ध॒मा॒सा इत्य॑द्‌र्ध - मा॒साः । पर्वा॑णि । मासाः᳚ । स॒धांना॒नीति॑ सं - धाना॑नि । ऋ॒तवः॑ । अङ्गा॑नि । सं॒ॅव॒थ्स॒र इति॑ सं - व॒थ्स॒रः । आ॒त्मा । र॒श्मयः॑ । केशाः᳚ । नक्ष॑त्राणि । रू॒पम् । तार॑काः । अ॒स्थानि॑ । नभः॑ । माꣳ॒॒सानि॑ । ओष॑धयः । लोमा॑नि । वन॒स्पत॑यः । वालाः᳚ । अ॒ग्निः । मुख᳚म् । वै॒श्वा॒न॒रः । व्यात्त॒मिति॑ वि - आत्त᳚म् ।  \newline


\textbf{Krama Paata} \newline

यो वै । वा अश्व॑स्य । अश्व॑स्य॒ मेद्ध्य॑स्य । मेद्ध्य॑स्य॒ शिरः॑ । शिरो॒ वेद॑ । वेद॑ शीर्.ष॒ण्वान् । शी॒र्.॒ष॒ण्वान् मेद्ध्यः॑ । शी॒र्॒.ष॒ण्वानिति॑ शीर्.षण् - वान् । मेद्ध्यो॑ भवति । भ॒व॒त्यु॒षाः । उ॒षा वै । वा अश्व॑स्य । अश्व॑स्य॒ मेद्ध्य॑स्य । मेद्ध्य॑स्य॒ शिरः॑ । शिरः॒ सूर्यः॑ । सूर्य॒श्चक्षुः॑ । चक्षु॒र् वातः॑ । वातः॑ प्रा॒णः । प्रा॒णश्च॒न्द्रमाः᳚ । प्रा॒ण इति॑ प्र - अ॒नः । च॒न्द्रमाः॒ श्रोत्र᳚म् । श्रोत्र॒म् दिशः॑ । दिशः॒ पादाः᳚ । पादा॑ अवान्तरदि॒शाः । अ॒वा॒न्त॒र॒दि॒शाः पर्.श॑वः । अ॒वा॒न्त॒र॒दि॒शा इत्य॑वान्तर - दि॒शाः । पर्.श॑वोऽहोरा॒त्रे । अ॒हो॒रा॒त्रे नि॑मे॒षः । अ॒हो॒रा॒त्रे इत्य॑हः - रा॒त्रे । नि॒मे॒षो᳚ऽर्द्धमा॒साः । नि॒मे॒ष इति॑ नि - मे॒षः । अ॒र्द्ध॒मा॒साः पर्वा॑णि । अ॒र्द्ध॒मा॒सा इत्य॑र्द्ध - मा॒साः । पर्वा॑णि॒ मासाः᳚ । मासाः᳚ स॒न्धाना॑नि । स॒न्धाना᳚न्यृ॒तवः॑ । स॒न्धाना॒नीति॑ सम् - धाना॑नि । ऋ॒तवोऽङ्‍गा॑नि । अङ्‍गा॑नि सम्ॅवथ्स॒रः । स॒म्ॅव॒थ्स॒र आ॒त्मा । स॒म्ॅव॒थ्स॒र इति॑ सम् - व॒थ्स॒रः । आ॒त्मा र॒श्मयः॑ । र॒श्मयः॒ केशाः᳚ । केशा॒ नक्ष॑त्राणि । नक्ष॑त्राणि रू॒पम् । रू॒पम् तार॑काः । तार॑का अ॒स्थानि॑ । अ॒स्थानि॒ नभः॑ । नभो॑ माꣳ॒॒सानि॑ । माꣳ॒॒सान्योष॑धयः । ओष॑धयो॒ लोमा॑नि । लोमा॑नि॒ वन॒स्पत॑यः । वन॒स्पत॑यो॒ वालाः᳚ । वाला॑ अ॒ग्निः । अ॒ग्निर् 
मुख᳚म् । मुख॑म् ॅवैश्वान॒रः । वै॒श्वा॒न॒रो व्यात्त᳚म् । व्यात्तꣳ॑ समु॒द्रः । व्यात्त॒मिति॑ वि - आत्त᳚म् \newline

\textbf{Jatai Paata} \newline

1. यो वै वै यो यो वै । \newline
2. वा अश्व॒स्या श्व॑स्य॒ वै वा अश्व॑स्य । \newline
3. अश्व॑स्य॒ मेद्ध्य॑स्य॒ मेद्ध्य॒स्या श्व॒स्या श्व॑स्य॒ मेद्ध्य॑स्य । \newline
4. मेद्ध्य॑स्य॒ शिरः॒ शिरो॒ मेद्ध्य॑स्य॒ मेद्ध्य॑स्य॒ शिरः॑ । \newline
5. शिरो॒ वेद॒ वेद॒ शिरः॒ शिरो॒ वेद॑ । \newline
6. वेद॑ शीर्.ष॒ण्वाञ् छी॑र्.ष॒ण्वान्. वेद॒ वेद॑ शीर्.ष॒ण्वान् । \newline
7. शी॒र्॒.ष॒ण्वान् मेद्ध्यो॒ मेद्ध्यः॑ शीर्.ष॒ण्वाञ् छी॑र्.ष॒ण्वान् मेद्ध्यः॑ । \newline
8. शी॒र्.॒ष॒ण्वानिति॑ शीर्.षण् - वान् । \newline
9. मेद्ध्यो॑ भवति भवति॒ मेद्ध्यो॒ मेद्ध्यो॑ भवति । \newline
10. भ॒व॒ त्यु॒षा उ॒षा भ॑वति भव त्यु॒षाः । \newline
11. उ॒षा वै वा उ॒षा उ॒षा वै । \newline
12. वा अश्व॒स्या श्व॑स्य॒ वै वा अश्व॑स्य । \newline
13. अश्व॑स्य॒ मेद्ध्य॑स्य॒ मेद्ध्य॒स्या श्व॒स्या श्व॑स्य॒ मेद्ध्य॑स्य । \newline
14. मेद्ध्य॑स्य॒ शिरः॒ शिरो॒ मेद्ध्य॑स्य॒ मेद्ध्य॑स्य॒ शिरः॑ । \newline
15. शिरः॒ सूर्यः॒ सूर्यः॒ शिरः॒ शिरः॒ सूर्यः॑ । \newline
16. सूर्य॒ श्चक्षु॒ श्चक्षुः॒ सूर्यः॒ सूर्य॒ श्चक्षुः॑ । \newline
17. चक्षु॒र् वातो॒ वात॒ श्चक्षु॒ श्चक्षु॒र् वातः॑ । \newline
18. वातः॑ प्रा॒णः प्रा॒णो वातो॒ वातः॑ प्रा॒णः । \newline
19. प्रा॒ण श्च॒न्द्रमा᳚ श्च॒न्द्रमाः᳚ प्रा॒णः प्रा॒ण श्च॒न्द्रमाः᳚ । \newline
20. प्रा॒ण इति॑ प्र - अ॒नः । \newline
21. च॒न्द्रमाः॒ श्रोत्रꣳ॒॒ श्रोत्र॑म् च॒न्द्रमा᳚ श्च॒न्द्रमाः॒ श्रोत्र᳚म् । \newline
22. श्रोत्र॒म् दिशो॒ दिशः॒ श्रोत्रꣳ॒॒ श्रोत्र॒म् दिशः॑ । \newline
23. दिशः॒ पादाः॒ पादा॒ दिशो॒ दिशः॒ पादाः᳚ । \newline
24. पादा॑ अवान्तरदि॒शा अ॑वान्तरदि॒शाः पादाः॒ पादा॑ अवान्तरदि॒शाः । \newline
25. अ॒वा॒न्त॒र॒दि॒शाः पर्.श॑वः॒ पर्.श॑वो ऽवान्तरदि॒शा अ॑वान्तरदि॒शाः पर्.श॑वः । \newline
26. अ॒वा॒न्त॒र॒दि॒शा इत्य॑वान्तर - दि॒शाः । \newline
27. पर्.श॑वो ऽहोरा॒त्रे अ॑होरा॒त्रे पर्.श॑वः॒ पर्.श॑वो ऽहोरा॒त्रे । \newline
28. अ॒हो॒रा॒त्रे नि॑मे॒षो नि॑मे॒षो॑ ऽहोरा॒त्रे अ॑होरा॒त्रे नि॑मे॒षः । \newline
29. अ॒हो॒रा॒त्रे इत्य॑हः - रा॒त्रे । \newline
30. नि॒मे॒षो᳚ ऽर्द्धमा॒सा अ॑र्द्धमा॒सा नि॑मे॒षो नि॑मे॒षो᳚ ऽर्द्धमा॒साः । \newline
31. नि॒मे॒ष इति॑ नि - मे॒षः । \newline
32. अ॒र्द्ध॒मा॒साः पर्वा॑णि॒ पर्वा᳚ण्यर्द्धमा॒सा अ॑र्द्धमा॒साः पर्वा॑णि । \newline
33. अ॒र्द्ध॒मा॒सा इत्य॑र्द्ध - मा॒साः । \newline
34. पर्वा॑णि॒ मासा॒ मासाः॒ पर्वा॑णि॒ पर्वा॑णि॒ मासाः᳚ । \newline
35. मासाः᳚ स॒न्धाना॑नि स॒न्धाना॑नि॒ मासा॒ मासाः᳚ स॒न्धाना॑नि । \newline
36. स॒न्धाना᳚ न्यृ॒तव॑ ऋ॒तवः॑ स॒न्धाना॑नि स॒न्धाना᳚ न्यृ॒तवः॑ । \newline
37. स॒न्धाना॒नीति॑ सं - धाना॑नि । \newline
38. ऋ॒तवो ऽङ्गा॒ न्यङ्गा᳚ न्यृ॒तव॑ ऋ॒तवो ऽङ्गा॑नि । \newline
39. अङ्गा॑नि संॅवथ्स॒रः सं॑ॅवथ्स॒रो ऽङ्गा॒ न्यङ्गा॑नि संॅवथ्स॒रः । \newline
40. सं॒ॅव॒थ्स॒र आ॒त्मा ऽऽत्मा सं॑ॅवथ्स॒रः सं॑ॅवथ्स॒र आ॒त्मा । \newline
41. सं॒ॅव॒थ्स॒र इति॑ सं - व॒थ्स॒रः । \newline
42. आ॒त्मा र॒श्मयो॑ र॒श्मय॑ आ॒त्मा ऽऽत्मा र॒श्मयः॑ । \newline
43. र॒श्मयः॒ केशाः॒ केशा॑ र॒श्मयो॑ र॒श्मयः॒ केशाः᳚ । \newline
44. केशा॒ नक्ष॑त्राणि॒ नक्ष॑त्राणि॒ केशाः॒ केशा॒ नक्ष॑त्राणि । \newline
45. नक्ष॑त्राणि रू॒पꣳ रू॒पन् नक्ष॑त्राणि॒ नक्ष॑त्राणि रू॒पम् । \newline
46. रू॒पम् तार॑का॒ स्तार॑का रू॒पꣳ रू॒पम् तार॑काः । \newline
47. तार॑का अ॒स्था न्य॒स्थानि॒ तार॑का॒ स्तार॑का अ॒स्थानि॑ । \newline
48. अ॒स्थानि॒ नभो॒ नभो॒ ऽस्था न्य॒स्थानि॒ नभः॑ । \newline
49. नभो॑ माꣳ॒॒सानि॑ माꣳ॒॒सानि॒ नभो॒ नभो॑ माꣳ॒॒सानि॑ । \newline
50. माꣳ॒॒सा न्योष॑धय॒ ओष॑धयो माꣳ॒॒सानि॑ माꣳ॒॒सा न्योष॑धयः । \newline
51. ओष॑धयो॒ लोमा॑नि॒ लोमा॒ न्योष॑धय॒ ओष॑धयो॒ लोमा॑नि । \newline
52. लोमा॑नि॒ वन॒स्पत॑यो॒ वन॒स्पत॑यो॒ लोमा॑नि॒ लोमा॑नि॒ वन॒स्पत॑यः । \newline
53. वन॒स्पत॑यो॒ वाला॒ वाला॒ वन॒स्पत॑यो॒ वन॒स्पत॑यो॒ वालाः᳚ । \newline
54. वाला॑ अ॒ग्नि र॒ग्निर् वाला॒ वाला॑ अ॒ग्निः । \newline
55. अ॒ग्निर् मुख॒म् मुख॑ म॒ग्नि र॒ग्निर् मुख᳚म् । \newline
56. मुखं॑ ॅवैश्वान॒रो वै᳚श्वान॒रो मुख॒म् मुखं॑ ॅवैश्वान॒रः । \newline
57. वै॒श्वा॒न॒रो व्यात्तं॒ ॅव्यात्तं॑ ॅवैश्वान॒रो वै᳚श्वान॒रो व्यात्त᳚म् । \newline
58. व्यात्तꣳ॑ समु॒द्रः स॑मु॒द्रो व्यात्तं॒ ॅव्यात्तꣳ॑ समु॒द्रः । \newline
59. व्यात्त॒मिति॑ वि - आत्त᳚म् । \newline

\textbf{Ghana Paata } \newline

1. यो वै वै यो यो वा अश्व॒स्या श्व॑स्य॒ वै यो यो वा अश्व॑स्य । \newline
2. वा अश्व॒स्या श्व॑स्य॒ वै वा अश्व॑स्य॒ मेद्ध्य॑स्य॒ मेद्ध्य॒स्या श्व॑स्य॒ वै वा अश्व॑स्य॒ मेद्ध्य॑स्य । \newline
3. अश्व॑स्य॒ मेद्ध्य॑स्य॒ मेद्ध्य॒स्या श्व॒स्या श्व॑स्य॒ मेद्ध्य॑स्य॒ शिरः॒ शिरो॒ मेद्ध्य॒स्या श्व॒स्या श्व॑स्य॒ मेद्ध्य॑स्य॒ शिरः॑ । \newline
4. मेद्ध्य॑स्य॒ शिरः॒ शिरो॒ मेद्ध्य॑स्य॒ मेद्ध्य॑स्य॒ शिरो॒ वेद॒ वेद॒ शिरो॒ मेद्ध्य॑स्य॒ मेद्ध्य॑स्य॒ शिरो॒ वेद॑ । \newline
5. शिरो॒ वेद॒ वेद॒ शिरः॒ शिरो॒ वेद॑ शीर्.ष॒ण्वाञ् छी॑र्.ष॒ण्वान्. वेद॒ शिरः॒ शिरो॒ वेद॑ शीर्.ष॒ण्वान् । \newline
6. वेद॑ शीर्.ष॒ण्वाञ् छी॑र्.ष॒ण्वान्. वेद॒ वेद॑ शीर्.ष॒ण्वान् मेद्ध्यो॒ मेद्ध्यः॑ शीर्.ष॒ण्वान्. वेद॒ वेद॑ शीर्.ष॒ण्वान् मेद्ध्यः॑ । \newline
7. शी॒र्॒.ष॒ण्वान् मेद्ध्यो॒ मेद्ध्यः॑ शीर्.ष॒ण्वाञ् छी॑र्.ष॒ण्वान् मेद्ध्यो॑ भवति भवति॒ मेद्ध्यः॑ शीर्.ष॒ण्वाञ् छी॑र्.ष॒ण्वान् मेद्ध्यो॑ भवति । \newline
8. शी॒र्.॒ष॒ण्वानिति॑ शीर्.षण् - वान् । \newline
9. मेद्ध्यो॑ भवति भवति॒ मेद्ध्यो॒ मेद्ध्यो॑ भव त्यु॒षा उ॒षा भ॑वति॒ मेद्ध्यो॒ मेद्ध्यो॑ भव त्यु॒षाः । \newline
10. भ॒व॒ त्यु॒षा उ॒षा भ॑वति भव त्यु॒षा वै वा उ॒षा भ॑वति भव त्यु॒षा वै । \newline
11. उ॒षा वै वा उ॒षा उ॒षा वा अश्व॒स्या श्व॑स्य॒ वा उ॒षा उ॒षा वा अश्व॑स्य । \newline
12. वा अश्व॒स्या श्व॑स्य॒ वै वा अश्व॑स्य॒ मेद्ध्य॑स्य॒ मेद्ध्य॒स्या श्व॑स्य॒ वै वा अश्व॑स्य॒ मेद्ध्य॑स्य । \newline
13. अश्व॑स्य॒ मेद्ध्य॑स्य॒ मेद्ध्य॒स्या श्व॒स्या श्व॑स्य॒ मेद्ध्य॑स्य॒ शिरः॒ शिरो॒ मेद्ध्य॒स्या श्व॒स्या श्व॑स्य॒ मेद्ध्य॑स्य॒ शिरः॑ । \newline
14. मेद्ध्य॑स्य॒ शिरः॒ शिरो॒ मेद्ध्य॑स्य॒ मेद्ध्य॑स्य॒ शिरः॒ सूर्यः॒ सूर्यः॒ शिरो॒ मेद्ध्य॑स्य॒ मेद्ध्य॑स्य॒ शिरः॒ सूर्यः॑ । \newline
15. शिरः॒ सूर्यः॒ सूर्यः॒ शिरः॒ शिरः॒ सूर्य॒ श्चक्षु॒ श्चक्षुः॒ सूर्यः॒ शिरः॒ शिरः॒ सूर्य॒ श्चक्षुः॑ । \newline
16. सूर्य॒ श्चक्षु॒ श्चक्षुः॒ सूर्यः॒ सूर्य॒ श्चक्षु॒र् वातो॒ वात॒ श्चक्षुः॒ सूर्यः॒ सूर्य॒ श्चक्षु॒र् वातः॑ । \newline
17. चक्षु॒र् वातो॒ वात॒ श्चक्षु॒ श्चक्षु॒र् वातः॑ प्रा॒णः प्रा॒णो वात॒ श्चक्षु॒ श्चक्षु॒र् वातः॑ प्रा॒णः । \newline
18. वातः॑ प्रा॒णः प्रा॒णो वातो॒ वातः॑ प्रा॒ण श्च॒न्द्रमा᳚ श्च॒न्द्रमाः᳚ प्रा॒णो वातो॒ वातः॑ प्रा॒ण श्च॒न्द्रमाः᳚ । \newline
19. प्रा॒ण श्च॒न्द्रमा᳚ श्च॒न्द्रमाः᳚ प्रा॒णः प्रा॒ण श्च॒न्द्रमाः॒ श्रोत्रꣳ॒॒ श्रोत्र॑म् च॒न्द्रमाः᳚ प्रा॒णः प्रा॒ण श्च॒न्द्रमाः॒ श्रोत्र᳚म् । \newline
20. प्रा॒ण इति॑ प्र - अ॒नः । \newline
21. च॒न्द्रमाः॒ श्रोत्रꣳ॒॒ श्रोत्र॑म् च॒न्द्रमा᳚ श्च॒न्द्रमाः॒ श्रोत्र॒म् दिशो॒ दिशः॒ श्रोत्र॑म् च॒न्द्रमा᳚ श्च॒न्द्रमाः॒ श्रोत्र॒म् दिशः॑ । \newline
22. श्रोत्र॒म् दिशो॒ दिशः॒ श्रोत्रꣳ॒॒ श्रोत्र॒म् दिशः॒ पादाः॒ पादा॒ दिशः॒ श्रोत्रꣳ॒॒ श्रोत्र॒म् दिशः॒ पादाः᳚ । \newline
23. दिशः॒ पादाः॒ पादा॒ दिशो॒ दिशः॒ पादा॑ अवान्तरदि॒शा अ॑वान्तरदि॒शाः पादा॒ दिशो॒ दिशः॒ पादा॑ अवान्तरदि॒शाः । \newline
24. पादा॑ अवान्तरदि॒शा अ॑वान्तरदि॒शाः पादाः॒ पादा॑ अवान्तरदि॒शाः पर्.श॑वः॒ पर्.श॑वो ऽवान्तरदि॒शाः पादाः॒ पादा॑ अवान्तरदि॒शाः पर्.श॑वः । \newline
25. अ॒वा॒न्त॒र॒दि॒शाः पर्.श॑वः॒ पर्.श॑वो ऽवान्तरदि॒शा अ॑वान्तरदि॒शाः पर्.श॑वो ऽहोरा॒त्रे अ॑होरा॒त्रे पर्.श॑वो ऽवान्तरदि॒शा अ॑वान्तरदि॒शाः पर्.श॑वो ऽहोरा॒त्रे । \newline
26. अ॒वा॒न्त॒र॒दि॒शा इत्य॑वान्तर - दि॒शाः । \newline
27. पर्.श॑वो ऽहोरा॒त्रे अ॑होरा॒त्रे पर्.श॑वः॒ पर्.श॑वो ऽहोरा॒त्रे नि॑मे॒षो नि॑मे॒षो॑ ऽहोरा॒त्रे पर्.श॑वः॒ पर्.श॑वो ऽहोरा॒त्रे नि॑मे॒षः । \newline
28. अ॒हो॒रा॒त्रे नि॑मे॒षो नि॑मे॒षो॑ ऽहोरा॒त्रे अ॑होरा॒त्रे नि॑मे॒षो᳚ ऽर्द्धमा॒सा अ॑र्द्धमा॒सा नि॑मे॒षो॑ ऽहोरा॒त्रे अ॑होरा॒त्रे नि॑मे॒षो᳚ ऽर्द्धमा॒साः । \newline
29. अ॒हो॒रा॒त्रे इत्य॑हः - रा॒त्रे । \newline
30. नि॒मे॒षो᳚ ऽर्द्धमा॒सा अ॑र्द्धमा॒सा नि॑मे॒षो नि॑मे॒षो᳚ ऽर्द्धमा॒साः पर्वा॑णि॒ पर्वा᳚ ण्यर्द्धमा॒सा नि॑मे॒षो नि॑मे॒षो᳚ ऽर्द्धमा॒साः पर्वा॑णि । \newline
31. नि॒मे॒ष इति॑ नि - मे॒षः । \newline
32. अ॒र्द्ध॒मा॒साः पर्वा॑णि॒ पर्वा᳚ ण्यर्द्धमा॒सा अ॑र्द्धमा॒साः पर्वा॑णि॒ मासा॒ मासाः॒ पर्वा᳚ ण्यर्द्धमा॒सा अ॑र्द्धमा॒साः पर्वा॑णि॒ मासाः᳚ । \newline
33. अ॒र्द्ध॒मा॒सा इत्य॑र्द्ध - मा॒साः । \newline
34. पर्वा॑णि॒ मासा॒ मासाः॒ पर्वा॑णि॒ पर्वा॑णि॒ मासाः᳚ स॒न्धाना॑नि स॒न्धाना॑नि॒ मासाः॒ पर्वा॑णि॒ पर्वा॑णि॒ मासाः᳚ स॒न्धाना॑नि । \newline
35. मासाः᳚ स॒न्धाना॑नि स॒न्धाना॑नि॒ मासा॒ मासाः᳚ स॒न्धाना᳚ न्यृ॒तव॑ ऋ॒तवः॑ स॒न्धाना॑नि॒ मासा॒ मासाः᳚ स॒न्धाना᳚ न्यृ॒तवः॑ । \newline
36. स॒न्धाना᳚ न्यृ॒तव॑ ऋ॒तवः॑ स॒न्धाना॑नि स॒न्धाना᳚ न्यृ॒तवो ऽङ्गा॒ न्यङ्गा᳚ न्यृ॒तवः॑ स॒न्धाना॑नि स॒न्धाना᳚ न्यृ॒तवो ऽङ्गा॑नि । \newline
37. स॒न्धाना॒नीति॑ सं - धाना॑नि । \newline
38. ऋ॒तवो ऽङ्गा॒ न्यङ्गा᳚ न्यृ॒तव॑ ऋ॒तवो ऽङ्गा॑नि संॅवथ्स॒रः सं॑ॅवथ्स॒रो ऽङ्गा᳚ न्यृ॒तव॑ ऋ॒तवो ऽङ्गा॑नि संॅवथ्स॒रः । \newline
39. अङ्गा॑नि संॅवथ्स॒रः सं॑ॅवथ्स॒रो ऽङ्गा॒ न्यङ्गा॑नि संॅवथ्स॒र आ॒त्मा ऽऽत्मा सं॑ॅवथ्स॒रो ऽङ्गा॒ न्यङ्गा॑नि संॅवथ्स॒र आ॒त्मा । \newline
40. सं॒ॅव॒थ्स॒र आ॒त्मा ऽऽत्मा सं॑ॅवथ्स॒रः सं॑ॅवथ्स॒र आ॒त्मा र॒श्मयो॑ र॒श्मय॑ आ॒त्मा सं॑ॅवथ्स॒रः सं॑ॅवथ्स॒र आ॒त्मा र॒श्मयः॑ । \newline
41. सं॒ॅव॒थ्स॒र इति॑ सं - व॒थ्स॒रः । \newline
42. आ॒त्मा र॒श्मयो॑ र॒श्मय॑ आ॒त्मा ऽऽत्मा र॒श्मयः॒ केशाः॒ केशा॑ र॒श्मय॑ आ॒त्मा ऽऽत्मा र॒श्मयः॒ केशाः᳚ । \newline
43. र॒श्मयः॒ केशाः॒ केशा॑ र॒श्मयो॑ र॒श्मयः॒ केशा॒ नक्ष॑त्राणि॒ नक्ष॑त्राणि॒ केशा॑ र॒श्मयो॑ र॒श्मयः॒ केशा॒ नक्ष॑त्राणि । \newline
44. केशा॒ नक्ष॑त्राणि॒ नक्ष॑त्राणि॒ केशाः॒ केशा॒ नक्ष॑त्राणि रू॒पꣳ रू॒पन् नक्ष॑त्राणि॒ केशाः॒ केशा॒ नक्ष॑त्राणि रू॒पम् । \newline
45. नक्ष॑त्राणि रू॒पꣳ रू॒पन् नक्ष॑त्राणि॒ नक्ष॑त्राणि रू॒पम् तार॑का॒ स्तार॑का रू॒पन् नक्ष॑त्राणि॒ नक्ष॑त्राणि रू॒पम् तार॑काः । \newline
46. रू॒पम् तार॑का॒ स्तार॑का रू॒पꣳ रू॒पम् तार॑का अ॒स्था न्य॒स्थानि॒ तार॑का रू॒पꣳ रू॒पम् तार॑का अ॒स्थानि॑ । \newline
47. तार॑का अ॒स्था न्य॒स्थानि॒ तार॑का॒ स्तार॑का अ॒स्थानि॒ नभो॒ नभो॒ ऽस्थानि॒ तार॑का॒ स्तार॑का अ॒स्थानि॒ नभः॑ । \newline
48. अ॒स्थानि॒ नभो॒ नभो॒ ऽस्था न्य॒स्थानि॒ नभो॑ माꣳ॒॒सानि॑ माꣳ॒॒सानि॒ नभो॒ ऽस्था न्य॒स्थानि॒ नभो॑ माꣳ॒॒सानि॑ । \newline
49. नभो॑ माꣳ॒॒सानि॑ माꣳ॒॒सानि॒ नभो॒ नभो॑ माꣳ॒॒सा न्योष॑धय॒ ओष॑धयो माꣳ॒॒सानि॒ नभो॒ नभो॑ माꣳ॒॒सा न्योष॑धयः । \newline
50. माꣳ॒॒सा न्योष॑धय॒ ओष॑धयो माꣳ॒॒सानि॑ माꣳ॒॒सा न्योष॑धयो॒ लोमा॑नि॒ लोमा॒ न्योष॑धयो माꣳ॒॒सानि॑ माꣳ॒॒सा न्योष॑धयो॒ लोमा॑नि । \newline
51. ओष॑धयो॒ लोमा॑नि॒ लोमा॒ न्योष॑धय॒ ओष॑धयो॒ लोमा॑नि॒ वन॒स्पत॑यो॒ वन॒स्पत॑यो॒ लोमा॒
न्योष॑धय॒ ओष॑धयो॒ लोमा॑नि॒ वन॒स्पत॑यः । \newline
52. लोमा॑नि॒ वन॒स्पत॑यो॒ वन॒स्पत॑यो॒ लोमा॑नि॒ लोमा॑नि॒ वन॒स्पत॑यो॒ वाला॒ वाला॒ वन॒स्पत॑यो॒ लोमा॑नि॒ लोमा॑नि॒ वन॒स्पत॑यो॒ वालाः᳚ । \newline
53. वन॒स्पत॑यो॒ वाला॒ वाला॒ वन॒स्पत॑यो॒ वन॒स्पत॑यो॒ वाला॑ अ॒ग्नि र॒ग्निर् वाला॒ वन॒स्पत॑यो॒ वन॒स्पत॑यो॒ वाला॑ अ॒ग्निः । \newline
54. वाला॑ अ॒ग्नि र॒ग्निर् वाला॒ वाला॑ अ॒ग्निर् मुख॒म् मुख॑ म॒ग्निर् वाला॒ वाला॑ अ॒ग्निर् मुख᳚म् । \newline
55. अ॒ग्निर् मुख॒म् मुख॑ म॒ग्नि र॒ग्निर् मुखं॑ ॅवैश्वान॒रो वै᳚श्वान॒रो मुख॑ म॒ग्नि र॒ग्निर् मुखं॑ ॅवैश्वान॒रः । \newline
56. मुखं॑ ॅवैश्वान॒रो वै᳚श्वान॒रो मुख॒म् मुखं॑ ॅवैश्वान॒रो व्यात्तं॒ ॅव्यात्तं॑ ॅवैश्वान॒रो मुख॒म् मुखं॑ ॅवैश्वान॒रो व्यात्त᳚म् । \newline
57. वै॒श्वा॒न॒रो व्यात्तं॒ ॅव्यात्तं॑ ॅवैश्वान॒रो वै᳚श्वान॒रो व्यात्तꣳ॑ समु॒द्रः स॑मु॒द्रो व्यात्तं॑ ॅवैश्वान॒रो वै᳚श्वान॒रो व्यात्तꣳ॑ समु॒द्रः । \newline
58. व्यात्तꣳ॑ समु॒द्रः स॑मु॒द्रो व्यात्तं॒ ॅव्यात्तꣳ॑ समु॒द्र उ॒दर॑ मु॒दरꣳ॑ समु॒द्रो व्यात्तं॒ ॅव्यात्तꣳ॑ समु॒द्र उ॒दर᳚म् । \newline
59. व्यात्त॒मिति॑ वि - आत्त᳚म् । \newline
\pagebreak
\markright{ TS 7.5.25.2  \hfill https://www.vedavms.in \hfill}

\section{ TS 7.5.25.2 }

\textbf{TS 7.5.25.2 } \newline
\textbf{Samhita Paata} \newline

समु॒द्र उ॒दर॑म॒न्तरि॑क्षं पा॒यु-र्द्यावा॑पृथि॒वी आ॒ण्डौ ग्रावा॒ शेपः॒ सोमो॒ रेतो॒ यज्ज॑ञ्ज॒भ्यते॒ तद्वि द्यो॑तते॒ यद्वि॑धूनु॒ते तथ् स्त॑नयति॒ यन्मेह॑ति॒ तद्व॑र्.षति॒ वागे॒वास्य॒ वागह॒र्वा अश्व॑स्य॒ जाय॑मानस्य महि॒मा पु॒रस्ता᳚ज्जायते॒ रात्रि॑रेनं महि॒मा प॒श्चादनु॑ जायत ए॒तौ वै म॑हि॒माना॒- वश्व॑म॒भितः॒ सं ब॑भूवतु॒र्॒.हयो॑ दे॒वान॑वह॒ () दर्वाऽसु॑रान् वा॒जी ग॑न्ध॒र्वानश्वो॑मनु॒ष्या᳚न्थ् समु॒द्रो वा अश्व॑स्य॒ योनिः॑ समु॒द्रो बन्धुः॑ ॥ \newline

\textbf{Pada Paata} \newline

स॒मु॒द्रः । उ॒दर᳚म् । अ॒न्तरि॑क्षम् । पा॒युः । द्यावा॑पृथि॒वी इति॒ द्यावा᳚ - पृ॒थि॒वी । आ॒ण्डौ । ग्रावा᳚ । शेपः॑ । सोमः॑ । रेतः॑ । यत् । ज॒ञ्ज॒भ्यते᳚ । तत् । वीति॑ । द्यो॒त॒ते॒ । यत् । वि॒धू॒नु॒त इति॑ वि-धू॒नु॒ते । तत् । स्त॒न॒य॒ति॒ । यत् । मेह॑ति । तत् । व॒र्.॒ष॒ति॒ । वाक् । ए॒व । अ॒स्य॒ । वाक् । अहः॑ । वै । अश्व॑स्य । जाय॑मानस्य । म॒हि॒मा । पु॒रस्ता᳚त् । जा॒य॒ते॒ । रात्रिः॑ । ए॒न॒म् । म॒हि॒मा । प॒श्चात् । अन्विति॑ । जा॒य॒ते॒ । ए॒तौ । वै । म॒हि॒मानौ᳚ । अश्व᳚म् । अ॒भितः॑ । समिति॑ । ब॒भू॒व॒तुः॒ । हयः॑ । दे॒वान् । अ॒व॒ह॒त् ( ) । अर्वा᳚ । असु॑रान् । वा॒जी । ग॒न्ध॒र्वान् । अश्वः॑ । म॒नु॒ष्यान्॑ । स॒मु॒द्रः । वै । अश्व॑स्य । योनिः॑ । स॒मु॒द्रः । बन्धुः॑ ॥  \newline


\textbf{Krama Paata} \newline

स॒मु॒द्र उ॒दर᳚म् । उ॒दर॑म॒न्तरि॑क्षम् । अ॒न्तरि॑क्षम् पा॒युः । पा॒युर् द्यावा॑पृथि॒वी । द्यावा॑पृथि॒वी आ॒ण्डौ । द्यावा॑पृथि॒वी इति॒ द्यावा᳚ - पृ॒थि॒वी । आ॒ण्डौ ग्रावा᳚ । ग्रावा॒ शेपः॑ । शेपः॒ सोमः॑ । सोमो॒ रेतः॑ । रेतो॒ यत् । 
यज् ज॑ञ्ज॒भ्यते᳚ । ज॒ञ्ज॒भ्यते॒ तत् । तद् वि । वि द्यो॑तते । द्यो॒त॒ते॒ यत् । यद् वि॑धूनु॒ते । वि॒धू॒नु॒ते तत् । वि॒धू॒नु॒त इति॑ वि - धू॒नु॒ते । तद् स्त॑नयति । स्त॒न॒य॒ति॒ यत् । यन् मेह॑ति । मेह॑ति॒ तत् । तद् व॑र्.षति । व॒र्.॒ष॒ति॒ वाक् । वागे॒व । ए॒वास्य॑ । अ॒स्य॒ वाक् । वागहः॑ । अह॒र् वै । वा अश्व॑स्य । अश्व॑स्य॒ जाय॑मानस्य । जाय॑मानस्य महि॒मा । म॒हि॒मा पु॒रस्ता᳚त् । पु॒रस्ता᳚ज् जायते । जा॒य॒ते॒ रात्रिः॑ । रात्रि॑रेनम् । ए॒न॒म् म॒हि॒मा । म॒हि॒मा प॒श्चात् । प॒श्चादनु॑ । अनु॑ जायते । जा॒य॒त॒ ए॒तौ । ए॒तौ वै । वै म॑हि॒मानौ᳚ । म॒हि॒माना॒वश्व᳚म् । अश्व॑म॒भितः॑ । अ॒भितः॒ सम् । सम् ब॑भूवतुः । ब॒भू॒व॒तु॒र्॒. हयः॑ । हयो॑ दे॒वान् । दे॒वान॑वहत् ( ) । अ॒व॒ह॒दर्वा᳚ । अर्वाऽसु॑रान् । असु॑रान्. वा॒जी । वा॒जी ग॑न्ध॒र्वान् । ग॒न्ध॒र्वानश्वः॑ । अश्वो॑ मनु॒ष्यान्॑ । म॒नु॒ष्या᳚न्थ् समु॒द्रः । स॒मु॒द्रो वै । वा अश्व॑स्य । अश्व॑स्य॒ योनिः॑ । योनिः॑ समु॒द्रः । स॒मु॒द्रो बन्धुः॑ । बन्धु॒रिति॒ बन्धुः॑ । \newline

\textbf{Jatai Paata} \newline

1. स॒मु॒द्र उ॒दर॑ मु॒दरꣳ॑ समु॒द्रः स॑मु॒द्र उ॒दर᳚म् । \newline
2. उ॒दर॑ म॒न्तरि॑क्ष म॒न्तरि॑क्ष मु॒दर॑ मु॒दर॑ म॒न्तरि॑क्षम् । \newline
3. अ॒न्तरि॑क्षम् पा॒युः पा॒यु र॒न्तरि॑क्ष म॒न्तरि॑क्षम् पा॒युः । \newline
4. पा॒युर् द्यावा॑पृथि॒वी द्यावा॑पृथि॒वी पा॒युः पा॒युर् द्यावा॑पृथि॒वी । \newline
5. द्यावा॑पृथि॒वी आ॒ण्डा वा॒ण्डौ द्यावा॑पृथि॒वी द्यावा॑पृथि॒वी आ॒ण्डौ । \newline
6. द्यावा॑पृथि॒वी इति॒ द्यावा᳚ - पृ॒थि॒वी । \newline
7. आ॒ण्डौ ग्रावा॒ ग्रावा॒ ऽऽण्डा वा॒ण्डौ ग्रावा᳚ । \newline
8. ग्रावा॒ शेपः॒ शेपो॒ ग्रावा॒ ग्रावा॒ शेपः॑ । \newline
9. शेपः॒ सोमः॒ सोमः॒ शेपः॒ शेपः॒ सोमः॑ । \newline
10. सोमो॒ रेतो॒ रेतः॒ सोमः॒ सोमो॒ रेतः॑ । \newline
11. रेतो॒ यद् यद् रेतो॒ रेतो॒ यत् । \newline
12. यज् ज॑ञ्ज॒भ्यते॑ जञ्ज॒भ्यते॒ यद् यज् ज॑ञ्ज॒भ्यते᳚ । \newline
13. ज॒ञ्ज॒भ्यते॒ तत् तज् ज॑ञ्ज॒भ्यते॑ जञ्ज॒भ्यते॒ तत् । \newline
14. तद् वि वि तत् तद् वि । \newline
15. वि द्यो॑तते द्योतते॒ वि वि द्यो॑तते । \newline
16. द्यो॒त॒ते॒ यद् यद् द्यो॑तते द्योतते॒ यत् । \newline
17. यद् वि॑धूनु॒ते वि॑धूनु॒ते यद् यद् वि॑धूनु॒ते । \newline
18. वि॒धू॒नु॒ते तत् तद् वि॑धूनु॒ते वि॑धूनु॒ते तत् । \newline
19. वि॒धू॒नु॒त इति॑ वि - धू॒नु॒ते । \newline
20. तथ् स्त॑नयति स्तनयति॒ तत् तथ् स्त॑नयति । \newline
21. स्त॒न॒य॒ति॒ यद् यथ् स्त॑नयति स्तनयति॒ यत् । \newline
22. यन् मेह॑ति॒ मेह॑ति॒ यद् यन् मेह॑ति । \newline
23. मेह॑ति॒ तत् तन् मेह॑ति॒ मेह॑ति॒ तत् । \newline
24. तद् व॑र्.षति वर्.षति॒ तत् तद् व॑र्.षति । \newline
25. व॒र्॒.ष॒ति॒ वाग् वाग् व॑र्.षति वर्.षति॒ वाक् । \newline
26. वागे॒वैव वाग् वागे॒व । \newline
27. ए॒वास्या᳚ स्यै॒वै वास्य॑ । \newline
28. अ॒स्य॒ वाग् वाग॑स्यास्य॒ वाक् । \newline
29. वागह॒ रह॒र् वाग् वागहः॑ । \newline
30. अह॒र् वै वा अह॒ रह॒र् वै । \newline
31. वा अश्व॒स्या श्व॑स्य॒ वै वा अश्व॑स्य । \newline
32. अश्व॑स्य॒ जाय॑मानस्य॒ जाय॑मान॒स्या श्व॒स्या श्व॑स्य॒ जाय॑मानस्य । \newline
33. जाय॑मानस्य महि॒मा म॑हि॒मा जाय॑मानस्य॒ जाय॑मानस्य महि॒मा । \newline
34. म॒हि॒मा पु॒रस्ता᳚त् पु॒रस्ता᳚न् महि॒मा म॑हि॒मा पु॒रस्ता᳚त् । \newline
35. पु॒रस्ता᳚ज् जायते जायते पु॒रस्ता᳚त् पु॒रस्ता᳚ज् जायते । \newline
36. जा॒य॒ते॒ रात्री॒ रात्रि॑र् जायते जायते॒ रात्रिः॑ । \newline
37. रात्रि॑ रेन मेनꣳ॒॒ रात्री॒ रात्रि॑ रेनम् । \newline
38. ए॒न॒म् म॒हि॒मा म॑हि॒ मैन॑ मेनम् महि॒मा । \newline
39. म॒हि॒मा प॒श्चात् प॒श्चान् म॑हि॒मा म॑हि॒मा प॒श्चात् । \newline
40. प॒श्चा दन्वनु॑ प॒श्चात् प॒श्चा दनु॑ । \newline
41. अनु॑ जायते जाय॒ते ऽन्वनु॑ जायते । \newline
42. जा॒य॒त॒ ए॒ता वे॒तौ जा॑यते जायत ए॒तौ । \newline
43. ए॒तौ वै वा ए॒ता वे॒तौ वै । \newline
44. वै म॑हि॒मानौ॑ महि॒मानौ॒ वै वै म॑हि॒मानौ᳚ । \newline
45. म॒हि॒माना॒ वश्व॒ मश्व॑म् महि॒मानौ॑ महि॒माना॒ वश्व᳚म् । \newline
46. अश्व॑ म॒भितो॒ ऽभितो ऽश्व॒ मश्व॑ म॒भितः॑ । \newline
47. अ॒भितः॒ सꣳ स म॒भितो॒ ऽभितः॒ सम् । \newline
48. सम् ब॑भूवतुर् बभूवतुः॒ सꣳ सम् ब॑भूवतुः । \newline
49. ब॒भू॒व॒तु॒र्॒. हयो॒ हयो॑ बभूवतुर् बभूवतु॒र्॒. हयः॑ । \newline
50. हयो॑ दे॒वान् दे॒वान्. हयो॒ हयो॑ दे॒वान् । \newline
51. दे॒वा न॑वह दवहद् दे॒वान् दे॒वा न॑वहत् । \newline
52. अ॒व॒ह॒ दर्वा ऽर्वा॑ ऽवह दवह॒ दर्वा᳚ । \newline
53. अर्वा ऽसु॑रा॒ नसु॑रा॒ नर्वा ऽर्वा ऽसु॑रान् । \newline
54. असु॑रान्. वा॒जी वा॒ज्यसु॑रा॒ नसु॑रान्. वा॒जी । \newline
55. वा॒जी ग॑न्ध॒र्वान् ग॑न्ध॒र्वान्. वा॒जी वा॒जी ग॑न्ध॒र्वान् । \newline
56. ग॒न्ध॒र्वा नश्वो ऽश्वो॑ गन्ध॒र्वान् ग॑न्ध॒र्वा नश्वः॑ । \newline
57. अश्वो॑ मनु॒ष्या᳚न् मनु॒ष्या॒ नश्वो ऽश्वो॑ मनु॒ष्यान्॑ । \newline
58. म॒नु॒ष्या᳚न् थ्समु॒द्रः स॑मु॒द्रो म॑नु॒ष्या᳚न् मनु॒ष्या᳚न् थ्समु॒द्रः । \newline
59. स॒मु॒द्रो वै वै स॑मु॒द्रः स॑मु॒द्रो वै । \newline
60. वा अश्व॒स्या श्व॑स्य॒ वै वा अश्व॑स्य । \newline
61. अश्व॑स्य॒ योनि॒र् योनि॒ रश्व॒स्या श्व॑स्य॒ योनिः॑ । \newline
62. योनिः॑ समु॒द्रः स॑मु॒द्रो योनि॒र् योनिः॑ समु॒द्रः । \newline
63. स॒मु॒द्रो बन्धु॒र् बन्धुः॑ समु॒द्रः स॑मु॒द्रो बन्धुः॑ । \newline
64. बन्धु॒रिति॒ बन्धुः॑ । \newline

\textbf{Ghana Paata } \newline

1. स॒मु॒द्र उ॒दर॑ मु॒दरꣳ॑ समु॒द्रः स॑मु॒द्र उ॒दर॑ म॒न्तरि॑क्ष म॒न्तरि॑क्ष मु॒दरꣳ॑ समु॒द्रः स॑मु॒द्र उ॒दर॑ म॒न्तरि॑क्षम् । \newline
2. उ॒दर॑ म॒न्तरि॑क्ष म॒न्तरि॑क्ष मु॒दर॑ मु॒दर॑ म॒न्तरि॑क्षम् पा॒युः पा॒यु र॒न्तरि॑क्ष मु॒दर॑ मु॒दर॑ म॒न्तरि॑क्षम् पा॒युः । \newline
3. अ॒न्तरि॑क्षम् पा॒युः पा॒यु र॒न्तरि॑क्ष म॒न्तरि॑क्षम् पा॒युर् द्यावा॑पृथि॒वी द्यावा॑पृथि॒वी पा॒यु र॒न्तरि॑क्ष म॒न्तरि॑क्षम् पा॒युर् द्यावा॑पृथि॒वी । \newline
4. पा॒युर् द्यावा॑पृथि॒वी द्यावा॑पृथि॒वी पा॒युः पा॒युर् द्यावा॑पृथि॒वी आ॒ण्डा वा॒ण्डौ द्यावा॑पृथि॒वी पा॒युः पा॒युर् द्यावा॑पृथि॒वी आ॒ण्डौ । \newline
5. द्यावा॑पृथि॒वी आ॒ण्डा वा॒ण्डौ द्यावा॑पृथि॒वी द्यावा॑पृथि॒वी आ॒ण्डौ ग्रावा॒ ग्रावा॒ ऽऽण्डौ द्यावा॑पृथि॒वी द्यावा॑पृथि॒वी आ॒ण्डौ ग्रावा᳚ । \newline
6. द्यावा॑पृथि॒वी इति॒ द्यावा᳚ - पृ॒थि॒वी । \newline
7. आ॒ण्डौ ग्रावा॒ ग्रावा॒ ऽऽण्डा वा॒ण्डौ ग्रावा॒ शेपः॒ शेपो॒ ग्रावा॒ ऽऽण्डा वा॒ण्डौ ग्रावा॒ शेपः॑ । \newline
8. ग्रावा॒ शेपः॒ शेपो॒ ग्रावा॒ ग्रावा॒ शेपः॒ सोमः॒ सोमः॒ शेपो॒ ग्रावा॒ ग्रावा॒ शेपः॒ सोमः॑ । \newline
9. शेपः॒ सोमः॒ सोमः॒ शेपः॒ शेपः॒ सोमो॒ रेतो॒ रेतः॒ सोमः॒ शेपः॒ शेपः॒ सोमो॒ रेतः॑ । \newline
10. सोमो॒ रेतो॒ रेतः॒ सोमः॒ सोमो॒ रेतो॒ यद् यद् रेतः॒ सोमः॒ सोमो॒ रेतो॒ यत् । \newline
11. रेतो॒ यद् यद् रेतो॒ रेतो॒ यज् ज॑ञ्ज॒भ्यते॑ जञ्ज॒भ्यते॒ यद् रेतो॒ रेतो॒ यज् ज॑ञ्ज॒भ्यते᳚ । \newline
12. यज् ज॑ञ्ज॒भ्यते॑ जञ्ज॒भ्यते॒ यद् यज् ज॑ञ्ज॒भ्यते॒ तत् तज् ज॑ञ्ज॒भ्यते॒ यद् यज् ज॑ञ्ज॒भ्यते॒ तत् । \newline
13. ज॒ञ्ज॒भ्यते॒ तत् तज् ज॑ञ्ज॒भ्यते॑ जञ्ज॒भ्यते॒ तद् वि वि तज् ज॑ञ्ज॒भ्यते॑ जञ्ज॒भ्यते॒ तद् वि । \newline
14. तद् वि वि तत् तद् वि द्यो॑तते द्योतते॒ वि तत् तद् वि द्यो॑तते । \newline
15. वि द्यो॑तते द्योतते॒ वि वि द्यो॑तते॒ यद् यद् द्यो॑तते॒ वि वि द्यो॑तते॒ यत् । \newline
16. द्यो॒त॒ते॒ यद् यद् द्यो॑तते द्योतते॒ यद् वि॑धूनु॒ते वि॑धूनु॒ते यद् द्यो॑तते द्योतते॒ यद् वि॑धूनु॒ते । \newline
17. यद् वि॑धूनु॒ते वि॑धूनु॒ते यद् यद् वि॑धूनु॒ते तत् तद् वि॑धूनु॒ते यद् यद् वि॑धूनु॒ते तत् । \newline
18. वि॒धू॒नु॒ते तत् तद् वि॑धूनु॒ते वि॑धूनु॒ते तथ् स्त॑नयति स्तनयति॒ तद् वि॑धूनु॒ते वि॑धूनु॒ते तथ् स्त॑नयति । \newline
19. वि॒धू॒नु॒त इति॑ वि - धू॒नु॒ते । \newline
20. तथ् स्त॑नयति स्तनयति॒ तत् तथ् स्त॑नयति॒ यद् यथ् स्त॑नयति॒ तत् तथ् स्त॑नयति॒ यत् । \newline
21. स्त॒न॒य॒ति॒ यद् यथ् स्त॑नयति स्तनयति॒ यन् मेह॑ति॒ मेह॑ति॒ यथ् स्त॑नयति स्तनयति॒ यन् मेह॑ति । \newline
22. यन् मेह॑ति॒ मेह॑ति॒ यद् यन् मेह॑ति॒ तत् तन् मेह॑ति॒ यद् यन् मेह॑ति॒ तत् । \newline
23. मेह॑ति॒ तत् तन् मेह॑ति॒ मेह॑ति॒ तद् व॑र्.षति वर्.षति॒ तन् मेह॑ति॒ मेह॑ति॒ तद् व॑र्.षति । \newline
24. तद् व॑र्.षति वर्.षति॒ तत् तद् व॑र्.षति॒ वाग् वाग् व॑र्.षति॒ तत् तद् व॑र्.षति॒ वाक् । \newline
25. व॒र्॒.ष॒ति॒ वाग् वाग् व॑र्.षति वर्.षति॒ वागे॒वैव वाग् व॑र्.षति वर्.षति॒ वागे॒व । \newline
26. वागे॒वैव वाग् वागे॒ वास्या᳚ स्यै॒व वाग् वागे॒ वास्य॑ । \newline
27. ए॒वास्या᳚ स्यै॒वैवास्य॒ वाग् वाग॑ स्यै॒वैवास्य॒ वाक् । \newline
28. अ॒स्य॒ वाग् वाग॑स्यास्य॒ वागह॒ रह॒र् वाग॑स्यास्य॒ वागहः॑ । \newline
29. वागह॒ रह॒र् वाग् वागह॒र् वै वा अह॒र् वाग् वागह॒र् वै । \newline
30. अह॒र् वै वा अह॒ रह॒र् वा अश्व॒स्या श्व॑स्य॒ वा अह॒ रह॒र् वा अश्व॑स्य । \newline
31. वा अश्व॒स्या श्व॑स्य॒ वै वा अश्व॑स्य॒ जाय॑मानस्य॒ जाय॑मान॒स्या श्व॑स्य॒ वै वा अश्व॑स्य॒ जाय॑मानस्य । \newline
32. अश्व॑स्य॒ जाय॑मानस्य॒ जाय॑मान॒स्या श्व॒स्या श्व॑स्य॒ जाय॑मानस्य महि॒मा म॑हि॒मा जाय॑मान॒स्या श्व॒स्या श्व॑स्य॒ जाय॑मानस्य महि॒मा । \newline
33. जाय॑मानस्य महि॒मा म॑हि॒मा जाय॑मानस्य॒ जाय॑मानस्य महि॒मा पु॒रस्ता᳚त् पु॒रस्ता᳚न् महि॒मा जाय॑मानस्य॒ जाय॑मानस्य महि॒मा पु॒रस्ता᳚त् । \newline
34. म॒हि॒मा पु॒रस्ता᳚त् पु॒रस्ता᳚न् महि॒मा म॑हि॒मा पु॒रस्ता᳚ज् जायते जायते पु॒रस्ता᳚न् महि॒मा म॑हि॒मा पु॒रस्ता᳚ज् जायते । \newline
35. पु॒रस्ता᳚ज् जायते जायते पु॒रस्ता᳚त् पु॒रस्ता᳚ज् जायते॒ रात्री॒ रात्रि॑र् जायते पु॒रस्ता᳚त् पु॒रस्ता᳚ज् जायते॒ रात्रिः॑ । \newline
36. जा॒य॒ते॒ रात्री॒ रात्रि॑र् जायते जायते॒ रात्रि॑ रेन मेनꣳ॒॒ रात्रि॑र् जायते जायते॒ रात्रि॑ रेनम् । \newline
37. रात्रि॑रेन मेनꣳ॒॒ रात्री॒ रात्रि॑ रेनम् महि॒मा म॑हि॒मैनꣳ॒॒ रात्री॒ रात्रि॑ रेनम् महि॒मा । \newline
38. ए॒न॒म् म॒हि॒मा म॑हि॒मैन॑ मेनम् महि॒मा प॒श्चात् प॒श्चान् म॑हि॒मैन॑ मेनम् महि॒मा प॒श्चात् । \newline
39. म॒हि॒मा प॒श्चात् प॒श्चान् म॑हि॒मा म॑हि॒मा प॒श्चा दन्वनु॑ प॒श्चान् म॑हि॒मा म॑हि॒मा प॒श्चा दनु॑ । \newline
40. प॒श्चा दन्वनु॑ प॒श्चात् प॒श्चा दनु॑ जायते जाय॒ते ऽनु॑ प॒श्चात् प॒श्चा दनु॑ जायते । \newline
41. अनु॑ जायते जाय॒ते ऽन्वनु॑ जायत ए॒ता वे॒तौ जा॑य॒ते ऽन्वनु॑ जायत ए॒तौ । \newline
42. जा॒य॒त॒ ए॒ता वे॒तौ जा॑यते जायत ए॒तौ वै वा ए॒तौ जा॑यते जायत ए॒तौ वै । \newline
43. ए॒तौ वै वा ए॒ता वे॒तौ वै म॑हि॒मानौ॑ महि॒मानौ॒ वा ए॒ता वे॒तौ वै म॑हि॒मानौ᳚ । \newline
44. वै म॑हि॒मानौ॑ महि॒मानौ॒ वै वै म॑हि॒माना॒ वश्व॒ मश्व॑म् महि॒मानौ॒ वै वै म॑हि॒माना॒ वश्व᳚म् । \newline
45. म॒हि॒माना॒ वश्व॒ मश्व॑म् महि॒मानौ॑ महि॒माना॒ वश्व॑ म॒भितो॒ ऽभितो ऽश्व॑म् महि॒मानौ॑ महि॒माना॒ वश्व॑ म॒भितः॑ । \newline
46. अश्व॑ म॒भितो॒ ऽभितो ऽश्व॒ मश्व॑ म॒भितः॒ सꣳ स म॒भितो ऽश्व॒ मश्व॑ म॒भितः॒ सम् । \newline
47. अ॒भितः॒ सꣳ स म॒भितो॒ ऽभितः॒ सम् ब॑भूवतुर् बभूवतुः॒ स म॒भितो॒ ऽभितः॒ सम् ब॑भूवतुः । \newline
48. सम् ब॑भूवतुर् बभूवतुः॒ सꣳ सम् ब॑भूवतु॒र्॒. हयो॒ हयो॑ बभूवतुः॒ सꣳ सम् ब॑भूवतु॒र्॒. हयः॑ । \newline
49. ब॒भू॒व॒तु॒र्॒. हयो॒ हयो॑ बभूवतुर् बभूवतु॒र्॒. हयो॑ दे॒वान् दे॒वान्. हयो॑ बभूवतुर् बभूवतु॒र्॒. हयो॑ दे॒वान् । \newline
50. हयो॑ दे॒वान् दे॒वान्. हयो॒ हयो॑ दे॒वा न॑वह दवहद् दे॒वान्. हयो॒ हयो॑ दे॒वा न॑वहत् । \newline
51. दे॒वा न॑वह दवहद् दे॒वान् दे॒वा न॑वह॒ दर्वा ऽर्वा॑ ऽवहद् दे॒वान् दे॒वा न॑वह॒ दर्वा᳚ । \newline
52. अ॒व॒ह॒ दर्वा ऽर्वा॑ ऽवह दवह॒ दर्वा ऽसु॑रा॒ नसु॑रा॒ नर्वा॑ ऽवह दवह॒ दर्वा ऽसु॑रान् । \newline
53. अर्वा ऽसु॑रा॒ नसु॑रा॒ नर्वा ऽर्वा ऽसु॑रान्. वा॒जी वा॒ज्यसु॑रा॒ नर्वा ऽर्वा ऽसु॑रान्. वा॒जी । \newline
54. असु॑रान्. वा॒जी वा॒ज्यसु॑रा॒ नसु॑रान्. वा॒जी ग॑न्ध॒र्वान् ग॑न्ध॒र्वान्. वा॒ज्यसु॑रा॒ नसु॑रान्. वा॒जी ग॑न्ध॒र्वान् । \newline
55. वा॒जी ग॑न्ध॒र्वान् ग॑न्ध॒र्वान्. वा॒जी वा॒जी ग॑न्ध॒र्वा नश्वो ऽश्वो॑ गन्ध॒र्वान्. वा॒जी वा॒जी ग॑न्ध॒र्वा नश्वः॑ । \newline
56. ग॒न्ध॒र्वा नश्वो ऽश्वो॑ गन्ध॒र्वान् ग॑न्ध॒र्वा नश्वो॑ मनु॒ष्या᳚न् मनु॒ष्या॒ नश्वो॑ गन्ध॒र्वान् ग॑न्ध॒र्वा नश्वो॑ मनु॒ष्यान्॑ । \newline
57. अश्वो॑ मनु॒ष्या᳚न् मनु॒ष्या॒ नश्वो ऽश्वो॑ मनु॒ष्या᳚न् थ्समु॒द्रः स॑मु॒द्रो म॑नु॒ष्या॒ नश्वो ऽश्वो॑ मनु॒ष्या᳚न् थ्समु॒द्रः । \newline
58. म॒नु॒ष्या᳚न् थ्समु॒द्रः स॑मु॒द्रो म॑नु॒ष्या᳚न् मनु॒ष्या᳚न् थ्समु॒द्रो वै वै स॑मु॒द्रो म॑नु॒ष्या᳚न् मनु॒ष्या᳚न् थ्समु॒द्रो वै । \newline
59. स॒मु॒द्रो वै वै स॑मु॒द्रः स॑मु॒द्रो वा अश्व॒स्या श्व॑स्य॒ वै स॑मु॒द्रः स॑मु॒द्रो वा अश्व॑स्य । \newline
60. वा अश्व॒स्या श्व॑स्य॒ वै वा अश्व॑स्य॒ योनि॒र् योनि॒ रश्व॑स्य॒ वै वा अश्व॑स्य॒ योनिः॑ । \newline
61. अश्व॑स्य॒ योनि॒र् योनि॒ रश्व॒स्या श्व॑स्य॒ योनिः॑ समु॒द्रः स॑मु॒द्रो योनि॒ रश्व॒ स्याश्व॑स्य॒ योनिः॑ समु॒द्रः । \newline
62. योनिः॑ समु॒द्रः स॑मु॒द्रो योनि॒र् योनिः॑ समु॒द्रो बन्धु॒र् बन्धुः॑ समु॒द्रो योनि॒र् योनिः॑ समु॒द्रो बन्धुः॑ । \newline
63. स॒मु॒द्रो बन्धु॒र् बन्धुः॑ समु॒द्रः स॑मु॒द्रो बन्धुः॑ । \newline
64. बन्धु॒रिति॒ बन्धुः॑ । \newline
\pagebreak


\end{document}