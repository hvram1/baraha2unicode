\documentclass[17pt]{extarticle}
\usepackage{babel}
\usepackage{fontspec}
\usepackage{polyglossia}
\usepackage{extsizes}



\setmainlanguage{sanskrit}
\setotherlanguages{english} %% or other languages
\setlength{\parindent}{0pt}
\pagestyle{myheadings}
\newfontfamily\devanagarifont[Script=Devanagari]{AdishilaVedic}


\newcommand{\VAR}[1]{}
\newcommand{\BLOCK}[1]{}




\begin{document}
\begin{titlepage}
    \begin{center}
 
\begin{sanskrit}
    { \Huge
    कृष्ण यजुर्वेदीय तैत्तिरीय संहिता,पद,जटा,घन पाठः 
    }
    \\
    \vspace{2.5cm}
    \mbox{ \Huge
    1.5     प्रथमकाण्डे पञ्चमः प्रश्नः - (पुनराधानं)   }
\end{sanskrit}
\end{center}

\end{titlepage}
\tableofcontents
\pagebreak

\markright{ TS 1.5.1.1  \hfill https://www.vedavms.in \hfill}
\addcontentsline{toc}{section}{ TS 1.5.1.1 }
\section*{ TS 1.5.1.1 }

\textbf{TS 1.5.1.1 } \newline
\textbf{Samhita Paata} \newline

दे॒वा॒सु॒राः संॅय॑त्ता आस॒न्ते दे॒वा वि॑ज॒यमु॑प॒यन्तो॒ ऽग्नौ वा॒मं ॅवसु॒ सं न्य॑दधते॒दमु॑ नो भविष्यति॒ यदि॑ नो जे॒ष्यन्तीति॒ तद॒ग्निर्न्य॑कामयत॒ तेनापा᳚*क्राम॒त् तद्दे॒वा वि॒जित्या॑*व॒रुरु॑थ्समाना॒ अन्वा॑य॒न् तद॑स्य॒ सह॒साऽऽ*दि॑थ्सन्त॒ सो॑ ऽरोदी॒द्यदरो॑दी॒त् तद् रु॒द्रस्य॑ रुद्र॒त्वं ॅयदश्व्रशी॑यत॒ तद्- [ ] \newline

\textbf{Pada Paata} \newline

दे॒वा॒सु॒रा इति॑ देव - अ॒सु॒राः । संॅय॑त्ता॒ इति॑ सं - य॒त्ताः॒ । आ॒स॒न्न् । ते । दे॒वाः । वि॒ज॒यमिति॑ वि - ज॒यम् । उ॒प॒यन्त॒ इत्यु॑प - यन्तः॑ । अ॒ग्नौ । वा॒मम् । वसु॑ । सम् । नीति॑ । अ॒द॒ध॒त॒ । इ॒दम् । उ॒ । नः॒ । भ॒वि॒ष्य॒ति॒ । यदि॑ । नः । जे॒ष्यन्ति॑ । इति॑ । तत् । अ॒ग्निः । नीति॑ । अ॒का॒म॒य॒त॒ । तेन॑ । अपेति॑ । अ॒क्रा॒म॒त् । तत् । दे॒वाः । वि॒जित्येति॑ वि - जित्य॑ । अ॒व॒रुरु॑थ्समाना॒ इत्य॑व - रुरु॑थ्समानाः । अन्विति॑ । आ॒य॒न्न् । तत् । अ॒स्य॒ । सह॑सा । एति॑ । अ॒दि॒थ्स॒न्त॒ । सः । अ॒रो॒दी॒त् । यत् । अरो॑दीत् । तत् । रु॒द्रस्य॑ । रु॒द्र॒त्वमिति॑ रुद्र - त्वम् । यत् । अश्रु॑ । अशी॑यत । तत् ।  \newline


\textbf{Krama Paata} \newline

दे॒वा॒सु॒राः सम्ॅय॑त्ताः । दे॒वा॒सु॒रा इति॑ देव - अ॒सु॒राः । सम्ॅय॑त्ता आसन्न् । सम्ॅय॑त्ता॒ इति॒ सम् - य॒त्ताः॒ । आ॒स॒न् ते । ते दे॒वाः । दे॒वा वि॑ज॒यम् । वि॒ज॒यमु॑प॒यन्तः॑ । वि॒ज॒यमिति॑ वि - ज॒यम् । उ॒प॒यन्तो॒ऽग्नौ । उ॒प॒यन्त॒ इत्यु॑प - यन्तः॑ । अ॒ग्नौ वा॒मम् । वा॒मं ॅवसु॑ । वसु॒ सम् । सन्नि । न्य॑दधत । अ॒द॒ध॒ते॒दम् । इ॒दमु॑ । उ॒ नः॒ । नो॒ भ॒वि॒ष्य॒ति॒ । भ॒वि॒ष्य॒ति॒ यदि॑ । यदि॑ नः । नो॒ जे॒ष्यन्ति॑ । जे॒ष्यन्तीति॑ । इति॒ तत् । तद॒ग्निः । अ॒ग्निर् नि । न्य॑कामयत । अ॒का॒म॒य॒त॒ तेन॑ । तेनाप॑ । अपा᳚क्रामत् । अ॒क्रा॒म॒त् तत् । तद् दे॒वाः । दे॒वा वि॒जित्य॑ । वि॒जित्या॑व॒रुरु॑थ्समानाः । वि॒जित्येति॑ वि - जित्य॑ । अ॒व॒रुरु॑थ्समाना॒ अनु॑ । अ॒व॒रुरु॑थ्समाना॒ इत्य॑व - रुरु॑थ्समानाः । अन्वा॑यन्न् । आ॒य॒न् तत् । तद॑स्य । अ॒स्य॒ सह॑सा । सह॒सा । आऽदि॑थ्सन्त । अ॒दि॒थ्स॒न्त॒ सः । सो॑ऽरोदीत् । अ॒रो॒दी॒द् यत् । यदरो॑दीत् । अरो॑दी॒त् तत् । तद् रु॒द्रस्य॑ । रु॒द्रस्य॑ रुद्र॒त्वम् । रु॒द्र॒त्वं ॅयत् । रु॒द्र॒त्वमिति॑ रुद्र - त्वम् । यदश्रु॑ । अश्र्वशी॑यत । अशी॑यत॒ तत् । 
तद् र॑ज॒तम् \newline

\textbf{Jatai Paata} \newline

1. दे॒वा॒सु॒राः संॅय॑त्ताः॒ संॅय॑त्ता देवासु॒रा दे॑वासु॒राः संॅय॑त्ताः । \newline
2. दे॒वा॒सु॒रा इति॑ देव - अ॒सु॒राः । \newline
3. संॅय॑त्ता आसन्नास॒न् थ्संॅय॑त्ताः॒ संॅय॑त्ता आसन्न् । \newline
4. संॅय॑त्ता॒ इति॑ सं - य॒त्ताः॒ । \newline
5. आ॒स॒न् ते त आ॑सन्नास॒न् ते । \newline
6. ते दे॒वा दे॒वास्ते ते दे॒वाः । \newline
7. दे॒वा वि॑ज॒यं ॅवि॑ज॒यम् दे॒वा दे॒वा वि॑ज॒यम् । \newline
8. वि॒ज॒य मु॑प॒यन्त॑ उप॒यन्तो॑ विज॒यं ॅवि॑ज॒य मु॑प॒यन्तः॑ । \newline
9. वि॒ज॒यमिति॑ वि - ज॒यम् । \newline
10. उ॒प॒यन्तो॒ ऽग्ना व॒ग्ना वु॑प॒यन्त॑ उप॒यन्तो॒ ऽग्नौ । \newline
11. उ॒प॒यन्त॒ इत्यु॑प - यन्तः॑ । \newline
12. अ॒ग्नौ वा॒मं ॅवा॒म म॒ग्ना व॒ग्नौ वा॒मम् । \newline
13. वा॒मं ॅवसु॒ वसु॑ वा॒मं ॅवा॒मं ॅवसु॑ । \newline
14. वसु॒ सꣳ सं ॅवसु॒ वसु॒ सम् । \newline
15. सन्नि नि सꣳ सन्नि । \newline
16. न्य॑दधतादधत॒ नि न्य॑दधत । \newline
17. अ॒द॒ध॒ते॒ द मि॒द म॑दधतादधते॒ दम् । \newline
18. इ॒द मु॑ वु वि॒द मि॒द मु॑ । \newline
19. उ॒ नो॒ न॒ उ॒ वु॒ नः॒ । \newline
20. नो॒ भ॒वि॒ष्य॒ति॒ भ॒वि॒ष्य॒ति॒ नो॒ नो॒ भ॒वि॒ष्य॒ति॒ । \newline
21. भ॒वि॒ष्य॒ति॒ यदि॒ यदि॑ भविष्यति भविष्यति॒ यदि॑ । \newline
22. यदि॑ नो नो॒ यदि॒ यदि॑ नः । \newline
23. नो॒ जे॒ष्यन्ति॑ जे॒ष्यन्ति॑ नो नो जे॒ष्यन्ति॑ । \newline
24. जे॒ष्यन्तीतीति॑ जे॒ष्यन्ति॑ जे॒ष्यन्तीति॑ । \newline
25. इति॒ तत् तदितीति॒ तत् । \newline
26. तद॒ग्नि र॒ग्नि स्तत् तद॒ग्निः । \newline
27. अ॒ग्निर् नि न्य॑ग्नि र॒ग्निर् नि । \newline
28. न्य॑कामयताकामयत॒ नि न्य॑कामयत । \newline
29. अ॒का॒म॒य॒त॒ तेन॒ तेना॑कामयताकामयत॒ तेन॑ । \newline
30. तेनापाप॒ तेन॒ तेनाप॑ । \newline
31. अपा᳚क्राम दक्राम॒ दपापा᳚क्रामत् । \newline
32. अ॒क्रा॒म॒त् तत् तद॑क्राम दक्राम॒त् तत् । \newline
33. तद् दे॒वा दे॒वास्तत् तद् दे॒वाः । \newline
34. दे॒वा वि॒जित्य॑ वि॒जित्य॑ दे॒वा दे॒वा वि॒जित्य॑ । \newline
35. वि॒जित्या॑व॒रुरु॑थ्समाना अव॒रुरु॑थ्समाना वि॒जित्य॑ वि॒जित्या॑व॒रुरु॑थ्समानाः । \newline
36. वि॒जित्येति॑ वि - जित्य॑ । \newline
37. अ॒व॒रुरु॑थ्समाना॒ अन्वन्व॑व॒रुरु॑थ्समाना अव॒रुरु॑थ्समाना॒ अनु॑ । \newline
38. अ॒व॒रुरु॑थ्समाना॒ इत्य॑व - रुरु॑थ्समानाः । \newline
39. अन्वा॑यन् नाय॒न् नन्वन्वा॑यन्न् । \newline
40. आ॒य॒न् तत् तदा॑यन्नाय॒न् तत् । \newline
41. तद॑स्यास्य॒ तत् तद॑स्य । \newline
42. अ॒स्य॒ सह॑सा॒ सह॑सा ऽस्यास्य॒ सह॑सा । \newline
43. सह॒सा ऽऽसह॑सा॒ सह॒सा । \newline
44. आ ऽदि॑थ्सन्तादिथ्स॒न्ता ऽदि॑थ्सन्त । \newline
45. अ॒दि॒थ्स॒न्त॒ स सो॑ ऽदिथ्सन्तादिथ्सन्त॒ सः । \newline
46. सो॑ ऽरोदीदरोदी॒थ् स सो॑ ऽरोदीत् । \newline
47. अ॒रो॒दी॒द् यद् यद॑रोदी दरोदी॒द् यत् । \newline
48. यदरो॑दी॒ दरो॑दी॒द् यद् यदरो॑दीत् । \newline
49. अरो॑दी॒त् तत् तदरो॑दी॒ दरो॑दी॒त् तत् । \newline
50. तद् रु॒द्रस्य॑ रु॒द्रस्य॒ तत् तद् रु॒द्रस्य॑ । \newline
51. रु॒द्रस्य॑ रुद्र॒त्वꣳ रु॑द्र॒त्वꣳ रु॒द्रस्य॑ रु॒द्रस्य॑ रुद्र॒त्वम् । \newline
52. रु॒द्र॒त्वं ॅयद् यद् रु॑द्र॒त्वꣳ रु॑द्र॒त्वं ॅयत् । \newline
53. रु॒द्र॒त्वमिति॑ रुद्र - त्वम् । \newline
54. यदश्र्वश्रु॒ यद् यदश्रु॑ । \newline
55. अश्र्वशी॑य॒ताशी॑य॒ताश्र्वश्र्वशी॑यत । \newline
56. अशी॑यत॒ तत् तदशी॑य॒ताशी॑यत॒ तत् । \newline
57. तद् र॑ज॒तꣳ र॑ज॒तम् तत् तद् र॑ज॒तम् । \newline

\textbf{Ghana Paata } \newline

1. दे॒वा॒सु॒राः संॅय॑त्ताः॒ संॅय॑त्ता देवासु॒रा दे॑वासु॒राः संॅय॑त्ता आसन् नास॒न् थ्संॅय॑त्ता देवासु॒रा दे॑वासु॒राः संॅय॑त्ता आसन्न् । \newline
2. दे॒वा॒सु॒रा इति॑ देव - अ॒सु॒राः । \newline
3. संॅय॑त्ता आसन् नास॒न् थ्संॅय॑त्ताः॒ संॅय॑त्ता आस॒न् ते त आ॑स॒न् थ्संॅय॑त्ताः॒ संॅय॑त्ता आस॒न् ते । \newline
4. संॅय॑त्ता॒ इति॑ सं - य॒त्ताः॒ । \newline
5. आ॒स॒न् ते त आ॑सन् नास॒न् ते दे॒वा दे॒वास्त आ॑सन् नास॒न् ते दे॒वाः । \newline
6. ते दे॒वा दे॒वास्ते ते दे॒वा वि॑ज॒यं ॅवि॑ज॒यम् दे॒वास्ते ते दे॒वा वि॑ज॒यम् । \newline
7. दे॒वा वि॑ज॒यं ॅवि॑ज॒यम् दे॒वा दे॒वा वि॑ज॒य मु॑प॒यन्त॑ उप॒यन्तो॑ विज॒यम् दे॒वा दे॒वा वि॑ज॒य मु॑प॒यन्तः॑ । \newline
8. वि॒ज॒य मु॑प॒यन्त॑ उप॒यन्तो॑ विज॒यं ॅवि॑ज॒य मु॑प॒यन्तो॒ ऽग्ना व॒ग्ना वु॑प॒यन्तो॑ विज॒यं ॅवि॑ज॒य मु॑प॒यन्तो॒ ऽग्नौ । \newline
9. वि॒ज॒यमिति॑ वि - ज॒यम् । \newline
10. उ॒प॒यन्तो॒ ऽग्ना व॒ग्ना वु॑प॒यन्त॑ उप॒यन्तो॒ ऽग्नौ वा॒मं ॅवा॒म म॒ग्ना वु॑प॒यन्त॑ उप॒यन्तो॒ ऽग्नौ वा॒मम् । \newline
11. उ॒प॒यन्त॒ इत्यु॑प - यन्तः॑ । \newline
12. अ॒ग्नौ वा॒मं ॅवा॒म म॒ग्ना व॒ग्नौ वा॒मं ॅवसु॒ वसु॑ वा॒म म॒ग्ना व॒ग्नौ वा॒मं ॅवसु॑ । \newline
13. वा॒मं ॅवसु॒ वसु॑ वा॒मं ॅवा॒मं ॅवसु॒ सꣳ सं ॅवसु॑ वा॒मं ॅवा॒मं ॅवसु॒ सम् । \newline
14. वसु॒ सꣳ सं ॅवसु॒ वसु॒ सन्नि नि सं ॅवसु॒ वसु॒ सन्नि । \newline
15. सम् नि नि सꣳ सम् न्य॑दधतादधत॒ नि सꣳ सम् न्य॑दधत । \newline
16. न्य॑दधतादधत॒ नि न्य॑दधते॒ द मि॒द म॑दधत॒ नि न्य॑दधते॒ दम् । \newline
17. अ॒द॒ध॒ते॒ द मि॒द म॑दधतादधते॒ द मु॑ वु वि॒द म॑दधतादधते॒ द मु॑ । \newline
18. इ॒द मु॑ वु वि॒द मि॒द मु॑ नो न उ वि॒द मि॒द मु॑ नः । \newline
19. उ॒ नो॒ न॒ उ॒ वु॒ नो॒ भ॒वि॒ष्य॒ति॒ भ॒वि॒ष्य॒ति॒ न॒ उ॒ वु॒ नो॒ भ॒वि॒ष्य॒ति॒ । \newline
20. नो॒ भ॒वि॒ष्य॒ति॒ भ॒वि॒ष्य॒ति॒ नो॒ नो॒ भ॒वि॒ष्य॒ति॒ यदि॒ यदि॑ भविष्यति नो नो भविष्यति॒ यदि॑ । \newline
21. भ॒वि॒ष्य॒ति॒ यदि॒ यदि॑ भविष्यति भविष्यति॒ यदि॑ नो नो॒ यदि॑ भविष्यति भविष्यति॒ यदि॑ नः । \newline
22. यदि॑ नो नो॒ यदि॒ यदि॑ नो जे॒ष्यन्ति॑ जे॒ष्यन्ति॑ नो॒ यदि॒ यदि॑ नो जे॒ष्यन्ति॑ । \newline
23. नो॒ जे॒ष्यन्ति॑ जे॒ष्यन्ति॑ नो नो जे॒ष्यन्तीतीति॑ जे॒ष्यन्ति॑ नो नो जे॒ष्यन्तीति॑ । \newline
24. जे॒ष्यन्तीतीति॑ जे॒ष्यन्ति॑ जे॒ष्यन्तीति॒ तत् तदिति॑ जे॒ष्यन्ति॑ जे॒ष्यन्तीति॒ तत् । \newline
25. इति॒ तत् तदितीति॒ तद॒ग्नि र॒ग्निस्तदितीति॒ तद॒ग्निः । \newline
26. तद॒ग्नि र॒ग्निस्तत् तद॒ग्निर् नि न्य॑ग्निस्तत् तद॒ग्निर् नि । \newline
27. अ॒ग्निर् नि न्य॑ग्नि र॒ग्निर् न्य॑कामयताकामयत॒ न्य॑ग्नि र॒ग्निर् न्य॑कामयत । \newline
28. न्य॑कामयताकामयत॒ नि न्य॑कामयत॒ तेन॒ तेना॑कामयत॒ नि न्य॑कामयत॒ तेन॑ । \newline
29. अ॒का॒म॒य॒त॒ तेन॒ तेना॑कामयताकामयत॒ तेनापाप॒ तेना॑कामयताकामयत॒ तेनाप॑ । \newline
30. तेनापाप॒ तेन॒ तेनापा᳚क्राम दक्राम॒दप॒ तेन॒ तेनापा᳚क्रामत् । \newline
31. अपा᳚क्राम दक्राम॒ दपापा᳚क्राम॒त् तत् तद॑क्राम॒ दपापा᳚क्राम॒त् तत् । \newline
32. अ॒क्रा॒म॒त् तत् तद॑क्राम दक्राम॒त् तद् दे॒वा दे॒वास्त द॑क्राम दक्राम॒त् तद् दे॒वाः । \newline
33. तद् दे॒वा दे॒वास्तत् तद् दे॒वा वि॒जित्य॑ वि॒जित्य॑ दे॒वास्तत् तद् दे॒वा वि॒जित्य॑ । \newline
34. दे॒वा वि॒जित्य॑ वि॒जित्य॑ दे॒वा दे॒वा वि॒जित्या॑व॒रुरु॑थ्समाना अव॒रुरु॑थ्समाना वि॒जित्य॑ दे॒वा दे॒वा वि॒जित्या॑व॒रुरु॑थ्समानाः । \newline
35. वि॒जित्या॑व॒रुरु॑थ्समाना अव॒रुरु॑थ्समाना वि॒जित्य॑ वि॒जित्या॑व॒रुरु॑थ्समाना॒ अन्वन्व॑व॒रुरु॑थ्समाना वि॒जित्य॑ वि॒जित्या॑व॒रुरु॑थ्समाना॒ अनु॑ । \newline
36. वि॒जित्येति॑ वि - जित्य॑ । \newline
37. अ॒व॒रुरु॑थ्समाना॒ अन्वन्व॑व॒रुरु॑थ्समाना अव॒रुरु॑थ्समाना॒ अन्वा॑यन् नाय॒न् नन्व॑व॒रुरु॑थ्समाना अव॒रुरु॑थ्समाना॒ अन्वा॑यन्न् । \newline
38. अ॒व॒रुरु॑थ्समाना॒ इत्य॑व - रुरु॑थ्समानाः । \newline
39. अन्वा॑यन् नाय॒न् नन्वन्वा॑य॒न् तत् तदा॑य॒न् नन्वन्वा॑य॒न् तत् । \newline
40. आ॒य॒न् तत् तदा॑यन् नाय॒न् तद॑स्यास्य॒ तदा॑यन् नाय॒न् तद॑स्य । \newline
41. तद॑स्यास्य॒ तत् तद॑स्य॒ सह॑सा॒ सह॑सा ऽस्य॒ तत् तद॑स्य॒ सह॑सा । \newline
42. अ॒स्य॒ सह॑सा॒ सह॑सा ऽस्यास्य॒ सह॒सा ऽऽसह॑सा ऽस्यास्य॒ सह॒सा । \newline
43. सह॒सा ऽऽसह॑सा॒ सह॒सा ऽदि॑थ्सन्तादिथ्स॒न्ता सह॑सा॒ सह॒सा ऽदि॑थ्सन्त । \newline
44. आ ऽदि॑थ्सन्तादिथ्स॒न्ता ऽदि॑थ्सन्त॒ स सो॑ ऽदिथ्स॒न्ता ऽदि॑थ्सन्त॒ सः । \newline
45. अ॒दि॒थ्स॒न्त॒ स सो॑ ऽदिथ्सन्ता दिथ्सन्त॒ सो॑ ऽरोदीदरोदी॒थ्सो॑ ऽदिथ्सन्तादिथ्सन्त॒ सो॑ ऽरोदीत् । \newline
46. सो॑ ऽरोदीदरोदी॒थ् स सो॑ ऽरोदी॒द् यद् यद॑रोदी॒थ् स सो॑ ऽरोदी॒द् यत् । \newline
47. अ॒रो॒दी॒द् यद् यद॑रोदी दरोदी॒द् यदरो॑दी॒ दरो॑दी॒द् यद॑रोदी दरोदी॒द् यदरो॑दीत् । \newline
48. यदरो॑दी॒ दरो॑दी॒द् यद् यदरो॑दी॒त् तत् तदरो॑दी॒द् यद् यदरो॑दी॒त् तत् । \newline
49. अरो॑दी॒त् तत् तदरो॑दी॒ दरो॑दी॒त् तद् रु॒द्रस्य॑ रु॒द्रस्य॒ तदरो॑दी॒ दरो॑दी॒त् तद् रु॒द्रस्य॑ । \newline
50. तद् रु॒द्रस्य॑ रु॒द्रस्य॒ तत् तद् रु॒द्रस्य॑ रुद्र॒त्वꣳ रु॑द्र॒त्वꣳ रु॒द्रस्य॒ तत् तद् रु॒द्रस्य॑ रुद्र॒त्वम् । \newline
51. रु॒द्रस्य॑ रुद्र॒त्वꣳ रु॑द्र॒त्वꣳ रु॒द्रस्य॑ रु॒द्रस्य॑ रुद्र॒त्वं ॅयद् यद् रु॑द्र॒त्वꣳ रु॒द्रस्य॑ रु॒द्रस्य॑ रुद्र॒त्वं ॅयत् । \newline
52. रु॒द्र॒त्वं ॅयद् यद् रु॑द्र॒त्वꣳ रु॑द्र॒त्वं ॅयदश्र्वश्रु॒ यद् रु॑द्र॒त्वꣳ रु॑द्र॒त्वं ॅयदश्रु॑ । \newline
53. रु॒द्र॒त्वमिति॑ रुद्र - त्वम् । \newline
54. यदश्र्वश्रु॒ यद् यदश्र्वशी॑य॒ताशी॑य॒ताश्रु॒ यद् यदश्र्वशी॑यत । \newline
55. अश्र्वशी॑य॒ताशी॑य॒ ताश्र्वश्र्वशी॑यत॒ तत् तदशी॑य॒ताश्र्वश्र्वशी॑यत॒ तत् । \newline
56. अशी॑यत॒ तत् तदशी॑य॒ताशी॑यत॒ तद् र॑ज॒तꣳ र॑ज॒तम् तदशी॑य॒ताशी॑यत॒ तद् र॑ज॒तम् । \newline
57. तद् र॑ज॒तꣳ र॑ज॒तम् तत् तद् र॑ज॒तꣳ हिर॑ण्य॒(ग्म्॒) हिर॑ण्यꣳ रज॒तम् तत् तद् र॑ज॒तꣳ हिर॑ण्यम् । \newline
\pagebreak
\markright{ TS 1.5.1.2  \hfill https://www.vedavms.in \hfill}
\addcontentsline{toc}{section}{ TS 1.5.1.2 }
\section*{ TS 1.5.1.2 }

\textbf{TS 1.5.1.2 } \newline
\textbf{Samhita Paata} \newline

र॑ज॒तꣳ हिर॑ण्यमभव॒त् तस्मा᳚द् रज॒तꣳ हिर॑ण्य-मदक्षि॒ण्य-म॑श्रु॒जꣳ हि यो ब॒र्॒.हिषि॒ ददा॑ति पु॒राऽस्य॑ संॅवथ्स॒राद् गृ॒हे रु॑दन्ति॒ तस्मा᳚द् ब॒र्.॒हिषि॒ न देयꣳ॒॒ सो᳚ऽग्निर॑ब्रवीद् भा॒ग्य॑सा॒न्यथ॑ व इ॒दमिति॑ पुनरा॒धेयं॑ ते॒ केव॑ल॒मित्य॑ब्रुवन् नृ॒द्ध्नव॒त् खलु॒ स इत्य॑ब्रवी॒द्यो म॑द्देव॒त्य॑-म॒ग्नि-मा॒दधा॑ता॒ इति॒ तं पू॒षाऽऽध॑त्त॒ तेन॑ - [ ] \newline

\textbf{Pada Paata} \newline

र॒ज॒तम् । हिर॑ण्यम् । अ॒भ॒व॒त् । तस्मा᳚त् । र॒ज॒तम् । हिर॑ण्यम् । अ॒द॒क्षि॒ण्यम् । अ॒श्रु॒जमित्य॑श्रु - जम् । हि । यः । ब॒र्॒.हिषि॑ । ददा॑ति । पु॒रा । अ॒स्य॒ । सं॒ॅव॒थ्स॒रादिति॑ सं - व॒थ्स॒रात् । गृ॒हे । रु॒द॒न्ति॒ । तस्मा᳚त् । ब॒र्॒.हिषि॑ । न । देय᳚म् । सः । अ॒ग्निः । अ॒ब्र॒वी॒त् । भा॒गी । अ॒सा॒नि॒ । अथ॑ । वः॒ । इ॒दम् । इति॑ । पु॒न॒रा॒धेय॒मिति॑ पुनः - आ॒धेय᳚म् । ते॒ । केव॑लम् । इति॑ । अ॒ब्रु॒व॒न्न् । ऋ॒द्ध्नव॑त् । खलु॑ । सः । इति॑ । अ॒ब्र॒वी॒त् । यः । म॒द्दे॒व॒त्य॑मिति॑ मत् - दे॒व॒त्य᳚म् । अ॒ग्निम् । आ॒दधा॑ता॒ इत्या᳚ - दधा॑तै । इति॑ । तम् । पू॒षा । एति॑ । अ॒ध॒त्त॒ । तेन॑ ।  \newline


\textbf{Krama Paata} \newline

र॒ज॒तꣳ हिर॑ण्यम् । हिर॑ण्यमभवत् । अ॒भ॒व॒त् तस्मा᳚त् । तस्मा᳚द् रज॒तम् । र॒ज॒तꣳ हिर॑ण्यम् । हिर॑ण्यमदक्षि॒ण्यम् । अ॒द॒क्षि॒ण्यम॑श्रु॒जम् । अ॒श्रु॒जꣳ हि । अ॒श्रु॒जमित्य॑श्रु - जम् । हि यः । यो ब॒र्.॒हिषि॑ । ब॒र्.॒हिषि॒ ददा॑ति । ददा॑ति पु॒रा । पु॒राऽस्य॑ । अ॒स्य॒ स॒म्ॅव॒थ्स॒रात् । स॒म्ॅव॒थ्स॒राद् गृ॒हे । स॒म्ॅव॒थ्स॒रादिति॑ सं - व॒थ्स॒रात् । गृ॒हे रु॑दन्ति । रु॒द॒न्ति॒ तस्मा᳚त् । तस्मा᳚द् ब॒र्॒.हिषि॑ । ब॒र्॒.हिषि॒ न । न देय᳚म् । देयꣳ॒॒ सः । सो᳚ऽग्निः । अ॒ग्निर॑ब्रवीत् । अ॒ब्र॒वी॒द् भा॒गी । भा॒ग्य॑सानि । अ॒सा॒न्यथ॑ । अथ॑ वः । व॒ इ॒दम् । इ॒दमिति॑ । इति॑ पुनरा॒धेय᳚म् । पु॒न॒रा॒धेय॑म् ते । पु॒न॒रा॒धेय॒मिति॑ पुनः - आ॒धेय᳚म् । ते॒ केव॑लम् । केव॑ल॒मिति॑ । इत्य॑ब्रुवन्न् । अ॒ब्रु॒व॒न्नृ॒द्ध्नव॑त् । ऋ॒द्ध्नव॒त् खलु॑ । खलु॒ सः । स इति॑ । इत्य॑ब्रवीत् । अ॒ब्र॒वी॒द् यः । यो म॑द्देव॒त्य᳚म् । म॒द्दे॒व॒त्य॑म॒ग्निम् । म॒द्दे॒व॒त्य॑मिति॑ मत् - दे॒व॒त्य᳚म् । अ॒ग्निमा॒दधा॑तै । आ॒दधा॑ता॒ इति॑ । आ॒दधा॑ता॒ इत्या᳚ - दधा॑तै । इति॒ तम् । तम् पू॒षा । पू॒षा ऽऽध॑त्त । आऽध॑त्त । अ॒ध॒त्त॒ तेन॑ । तेन॑ पू॒षा \newline

\textbf{Jatai Paata} \newline

1. र॒ज॒तꣳ हिर॑ण्य॒(ग्म्॒) हिर॑ण्यꣳ रज॒तꣳ र॑ज॒तꣳ हिर॑ण्यम् । \newline
2. हिर॑ण्य मभव दभव॒द्धिर॑ण्य॒(ग्म्॒) हिर॑ण्य मभवत् । \newline
3. अ॒भ॒व॒त् तस्मा॒त् तस्मा॑ दभव दभव॒त् तस्मा᳚त् । \newline
4. तस्मा᳚द् रज॒तꣳ र॑ज॒तम् तस्मा॒त् तस्मा᳚द् रज॒तम् । \newline
5. र॒ज॒तꣳ हिर॑ण्य॒(ग्म्॒) हिर॑ण्यꣳ रज॒तꣳ र॑ज॒तꣳ हिर॑ण्यम् । \newline
6. हिर॑ण्य मदक्षि॒ण्य म॑दक्षि॒ण्यꣳ हिर॑ण्य॒(ग्म्॒) हिर॑ण्य मदक्षि॒ण्यम् । \newline
7. अ॒द॒क्षि॒ण्य म॑श्रु॒ज म॑श्रु॒ज म॑दक्षि॒ण्य म॑दक्षि॒ण्य म॑श्रु॒जम् । \newline
8. अ॒श्रु॒जꣳ हि ह्य॑श्रु॒ज म॑श्रु॒जꣳ हि । \newline
9. अ॒श्रु॒जमित्य॑श्रु - जम् । \newline
10. हि यो यो हि हि यः । \newline
11. यो ब॒र्॒.हिषि॑ ब॒र्॒.हिषि॒ यो यो ब॒र्॒.हिषि॑ । \newline
12. ब॒र्॒.हिषि॒ ददा॑ति॒ ददा॑ति ब॒र्॒.हिषि॑ ब॒र्॒.हिषि॒ ददा॑ति । \newline
13. ददा॑ति पु॒रा पु॒रा ददा॑ति॒ ददा॑ति पु॒रा । \newline
14. पु॒रा ऽस्या᳚स्य पु॒रा पु॒रा ऽस्य॑ । \newline
15. अ॒स्य॒ सं॒ॅव॒थ्स॒राथ् सं॑ॅवथ्स॒रा द॑स्यास्य संॅवथ्स॒रात् । \newline
16. सं॒ॅव॒थ्स॒राद् गृ॒हे गृ॒हे सं॑ॅवथ्स॒राथ् सं॑ॅवथ्स॒राद् गृ॒हे । \newline
17. सं॒ॅव॒थ्स॒रादिति॑ सं - व॒थ्स॒रात् । \newline
18. गृ॒हे रु॑दन्ति रुदन्ति गृ॒हे गृ॒हे रु॑दन्ति । \newline
19. रु॒द॒न्ति॒ तस्मा॒त् तस्मा᳚द् रुदन्ति रुदन्ति॒ तस्मा᳚त् । \newline
20. तस्मा᳚द् ब॒र्॒.हिषि॑ ब॒र्॒.हिषि॒ तस्मा॒त् तस्मा᳚द् ब॒र्॒.हिषि॑ । \newline
21. ब॒र्॒.हिषि॒ न न ब॒र्॒.हिषि॑ ब॒र्॒.हिषि॒ न । \newline
22. न देय॒म् देय॒न्न न देय᳚म् । \newline
23. देय॒(ग्म्॒) स स देय॒म् देय॒(ग्म्॒) सः । \newline
24. सो᳚ ऽग्निर॒ग्निः स सो᳚ ऽग्निः । \newline
25. अ॒ग्नि र॑ब्रवी दब्रवी द॒ग्नि र॒ग्नि र॑ब्रवीत् । \newline
26. अ॒ब्र॒वी॒द् भा॒गी भा॒ग्य॑ब्रवी दब्रवीद् भा॒गी । \newline
27. भा॒ग्य॑सान्यसानि भा॒गी भा॒ग्य॑सानि । \newline
28. अ॒सा॒न्यथाथा॑सान्यसा॒न्यथ॑ । \newline
29. अथ॑ वो॒ वो ऽथाथ॑ वः । \newline
30. व॒ इ॒द मि॒दं ॅवो॑ व इ॒दम् । \newline
31. इ॒द मितीती॒द मि॒द मिति॑ । \newline
32. इति॑ पुनरा॒धेय॑म् पुनरा॒धेय॒ मितीति॑ पुनरा॒धेय᳚म् । \newline
33. पु॒न॒रा॒धेय॑म् ते ते पुनरा॒धेय॑म् पुनरा॒धेय॑म् ते । \newline
34. पु॒न॒रा॒धेय॒मिति॑ पुनः - आ॒धेय᳚म् । \newline
35. ते॒ केव॑ल॒म् केव॑लम् ते ते॒ केव॑लम् । \newline
36. केव॑ल॒ मितीति॒ केव॑ल॒म् केव॑ल॒ मिति॑ । \newline
37. इत्य॑ब्रुव न्नब्रुव॒ न्नितीत्य॑ब्रुवन्न् । \newline
38. अ॒ब्रु॒व॒न् नृ॒द्ध्नव॑ दृ॒द्ध्नव॑दब्रुवन् नब्रुवन्नृ॒द्ध्नव॑त् । \newline
39. ऋ॒द्ध्नव॒त् खलु॒ खलु॑ ऋ॒द्ध्नव॑दृ॒द्ध्नव॒त् खलु॑ । \newline
40. खलु॒ स स खलु॒ खलु॒ सः । \newline
41. स इतीति॒ स स इति॑ । \newline
42. इत्य॑ब्रवी दब्रवी॒दिती त्य॑ब्रवीत् । \newline
43. अ॒ब्र॒वी॒द् यो यो᳚ ऽब्रवीदब्रवी॒द् यः । \newline
44. यो म॑द्देव॒त्य॑म् मद्देव॒त्यं॑ ॅयो यो म॑द्देव॒त्य᳚म् । \newline
45. म॒द्दे॒व॒त्य॑ म॒ग्नि म॒ग्निम् म॑द्देव॒त्य॑म् मद्देव॒त्य॑ म॒ग्निम् । \newline
46. म॒द्दे॒व॒त्य॑मिति॑ मत् - दे॒व॒त्य᳚म् । \newline
47. अ॒ग्नि मा॒दधा॑ता आ॒दधा॑ता अ॒ग्नि म॒ग्नि मा॒दधा॑तै । \newline
48. आ॒दधा॑ता॒ इतीत्या॒दधा॑ता आ॒दधा॑ता॒ इति॑ । \newline
49. आ॒दधा॑ता॒ इत्या᳚ - दधा॑तै । \newline
50. इति॒ तम् त मितीति॒ तम् । \newline
51. तम् पू॒षा पू॒षा तम् तम् पू॒षा । \newline
52. पू॒षा ऽध॑त्ता ध॒त्ता पू॒षा पू॒षा ऽध॑त्त । \newline
53. आ ऽध॑त्ताध॒त्ता ऽध॑त्त । \newline
54. अ॒ध॒त्त॒ तेन॒ तेना॑धत्ताधत्त॒ तेन॑ । \newline
55. तेन॑ पू॒षा पू॒षा तेन॒ तेन॑ पू॒षा । \newline

\textbf{Ghana Paata } \newline

1. र॒ज॒तꣳ हिर॑ण्य॒(ग्म्॒) हिर॑ण्यꣳ रज॒तꣳ र॑ज॒तꣳ हिर॑ण्य मभव दभव॒द्धिर॑ण्यꣳ रज॒तꣳ र॑ज॒तꣳ हिर॑ण्य मभवत् । \newline
2. हिर॑ण्य मभव दभव॒द्धिर॑ण्य॒(ग्म्॒) हिर॑ण्य मभव॒त् तस्मा॒त् तस्मा॑दभव॒द्धिर॑ण्य॒(ग्म्॒) हिर॑ण्य मभव॒त् तस्मा᳚त् । \newline
3. अ॒भ॒व॒त् तस्मा॒त् तस्मा॑ दभवदभव॒त् तस्मा᳚द् रज॒तꣳ र॑ज॒तम् तस्मा॑ दभवदभव॒त् तस्मा᳚द् रज॒तम् । \newline
4. तस्मा᳚द् रज॒तꣳ र॑ज॒तम् तस्मा॒त् तस्मा᳚द् रज॒तꣳ हिर॑ण्य॒(ग्म्॒) हिर॑ण्यꣳ रज॒तम् तस्मा॒त् तस्मा᳚द् रज॒तꣳ हिर॑ण्यम् । \newline
5. र॒ज॒तꣳ हिर॑ण्य॒(ग्म्॒) हिर॑ण्यꣳ रज॒तꣳ र॑ज॒तꣳ हिर॑ण्य मदक्षि॒ण्य म॑दक्षि॒ण्यꣳ हिर॑ण्यꣳ रज॒तꣳ र॑ज॒तꣳ हिर॑ण्य मदक्षि॒ण्यम् । \newline
6. हिर॑ण्य मदक्षि॒ण्य म॑दक्षि॒ण्यꣳ हिर॑ण्य॒(ग्म्॒) हिर॑ण्य मदक्षि॒ण्य म॑श्रु॒ज म॑श्रु॒ज म॑दक्षि॒ण्यꣳ हिर॑ण्य॒(ग्म्॒) हिर॑ण्य मदक्षि॒ण्य म॑श्रु॒जम् । \newline
7. अ॒द॒क्षि॒ण्य म॑श्रु॒ज म॑श्रु॒ज म॑दक्षि॒ण्य म॑दक्षि॒ण्य म॑श्रु॒जꣳ हि ह्य॑श्रु॒ज म॑दक्षि॒ण्य म॑दक्षि॒ण्य म॑श्रु॒जꣳ हि । \newline
8. अ॒श्रु॒जꣳ हि ह्य॑श्रु॒ज म॑श्रु॒जꣳ हि यो यो ह्य॑श्रु॒ज म॑श्रु॒जꣳ हि यः । \newline
9. अ॒श्रु॒जमित्य॑श्रु - जम् । \newline
10. हि यो यो हि हि यो ब॒र्॒.हिषि॑ ब॒र्॒.हिषि॒ यो हि हि यो ब॒र्॒.हिषि॑ । \newline
11. यो ब॒र्॒.हिषि॑ ब॒र्॒.हिषि॒ यो यो ब॒र्॒.हिषि॒ ददा॑ति॒ ददा॑ति ब॒र्॒.हिषि॒ यो यो ब॒र्॒.हिषि॒ ददा॑ति । \newline
12. ब॒र्॒.हिषि॒ ददा॑ति॒ ददा॑ति ब॒र्॒.हिषि॑ ब॒र्॒.हिषि॒ ददा॑ति पु॒रा पु॒रा ददा॑ति ब॒र्॒.हिषि॑ ब॒र्॒.हिषि॒ ददा॑ति पु॒रा । \newline
13. ददा॑ति पु॒रा पु॒रा ददा॑ति॒ ददा॑ति पु॒रा ऽस्या᳚स्य पु॒रा ददा॑ति॒ ददा॑ति पु॒रा ऽस्य॑ । \newline
14. पु॒रा ऽस्या᳚स्य पु॒रा पु॒रा ऽस्य॑ संॅवथ्स॒राथ् सं॑ॅवथ्स॒राद॑स्य पु॒रा पु॒रा ऽस्य॑ संॅवथ्स॒रात् । \newline
15. अ॒स्य॒ सं॒ॅव॒थ्स॒राथ् सं॑ॅवथ्स॒राद॑स्यास्य संॅवथ्स॒राद् गृ॒हे गृ॒हे सं॑ॅवथ्स॒राद॑स्यास्य संॅवथ्स॒राद् गृ॒हे । \newline
16. सं॒ॅव॒थ्स॒राद् गृ॒हे गृ॒हे सं॑ॅवथ्स॒राथ् सं॑ॅवथ्स॒राद् गृ॒हे रु॑दन्ति रुदन्ति गृ॒हे सं॑ॅवथ्स॒राथ् सं॑ॅवथ्स॒राद् गृ॒हे रु॑दन्ति । \newline
17. सं॒ॅव॒थ्स॒रादिति॑ सं - व॒थ्स॒रात् । \newline
18. गृ॒हे रु॑दन्ति रुदन्ति गृ॒हे गृ॒हे रु॑दन्ति॒ तस्मा॒त् तस्मा᳚द् रुदन्ति गृ॒हे गृ॒हे रु॑दन्ति॒ तस्मा᳚त् । \newline
19. रु॒द॒न्ति॒ तस्मा॒त् तस्मा᳚द् रुदन्ति रुदन्ति॒ तस्मा᳚द् ब॒र्॒.हिषि॑ ब॒र्॒.हिषि॒ तस्मा᳚द् रुदन्ति रुदन्ति॒ तस्मा᳚द् ब॒र्॒.हिषि॑ । \newline
20. तस्मा᳚द् ब॒र्॒.हिषि॑ ब॒र्॒.हिषि॒ तस्मा॒त् तस्मा᳚द् ब॒र्॒.हिषि॒ न न ब॒र्॒.हिषि॒ तस्मा॒त् तस्मा᳚द् ब॒र्॒.हिषि॒ न । \newline
21. ब॒र्॒.हिषि॒ न न ब॒र्॒.हिषि॑ ब॒र्॒.हिषि॒ न देय॒म् देय॒न्न ब॒र्॒.हिषि॑ ब॒र्॒.हिषि॒ न देय᳚म् । \newline
22. न देय॒म् देय॒न्न न देय॒(ग्म्॒) स स देय॒न्न न देय॒(ग्म्॒) सः । \newline
23. देय॒(ग्म्॒) स स देय॒म् देयꣳ॒॒ सो᳚ ऽग्निर॒ग्निः स देय॒म् देयꣳ॒॒ सो᳚ ऽग्निः । \newline
24. सो᳚ ऽग्निर॒ग्निः स सो᳚ ऽग्नि र॑ब्रवी दब्रवीद॒ग्निः स सो᳚ ऽग्निर॑ब्रवीत् । \newline
25. अ॒ग्नि र॑ब्रवी दब्रवी द॒ग्नि र॒ग्नि र॑ब्रवीद् भा॒गी भा॒ग्य॑ब्रवी द॒ग्नि र॒ग्नि र॑ब्रवीद् भा॒गी । \newline
26. अ॒ब्र॒वी॒द् भा॒गी भा॒ग्य॑ब्रवी दब्रवीद् भा॒ग्य॑सान्यसानि भा॒ग्य॑ब्रवी दब्रवीद् भा॒ग्य॑सानि । \newline
27. भा॒ग्य॑सान्यसानि भा॒गी भा॒ग्य॑सा॒न्यथाथा॑सानि भा॒गी भा॒ग्य॑सा॒न्यथ॑ । \newline
28. अ॒सा॒न्यथाथा॑सान्यसा॒न्यथ॑ वो॒ वो ऽथा॑सान्यसा॒न्यथ॑ वः । \newline
29. अथ॑ वो॒ वो ऽथाथ॑ व इ॒द मि॒दं ॅवो ऽथाथ॑ व इ॒दम् । \newline
30. व॒ इ॒द मि॒दं ॅवो॑ व इ॒द मितीती॒दं ॅवो॑ व इ॒द मिति॑ । \newline
31. इ॒द मितीती॒द मि॒द मिति॑ पुनरा॒धेय॑म् पुनरा॒धेय॒ मिती॒द मि॒द मिति॑ पुनरा॒धेय᳚म् । \newline
32. इति॑ पुनरा॒धेय॑म् पुनरा॒धेय॒ मितीति॑ पुनरा॒धेय॑म् ते ते पुनरा॒धेय॒ मितीति॑ पुनरा॒धेय॑म् ते । \newline
33. पु॒न॒रा॒धेय॑म् ते ते पुनरा॒धेय॑म् पुनरा॒धेय॑म् ते॒ केव॑ल॒म् केव॑लम् ते पुनरा॒धेय॑म् पुनरा॒धेय॑म् ते॒ केव॑लम् । \newline
34. पु॒न॒रा॒धेय॒मिति॑ पुनः - आ॒धेय᳚म् । \newline
35. ते॒ केव॑ल॒म् केव॑लम् ते ते॒ केव॑ल॒ मितीति॒ केव॑लम् ते ते॒ केव॑ल॒ मिति॑ । \newline
36. केव॑ल॒ मितीति॒ केव॑ल॒म् केव॑ल॒ मित्य॑ब्रुवन् नब्रुव॒न् निति॒ केव॑ल॒म् केव॑ल॒ मित्य॑ब्रुवन्न् । \newline
37. इत्य॑ब्रुवन् नब्रुव॒न् नितीत्य॑ब्रुवन् नृ॒द्ध्नव॑दृ॒द्ध्नव॑दब्रुव॒न् नितीत्य॑ब्रुवन् नृ॒द्ध्नव॑त् । \newline
38. अ॒ब्रु॒व॒न् नृ॒द्ध्नव॑दृ॒द्ध्नव॑दब्रुवन् नब्रुवन् नृ॒द्ध्नव॒त् खलु॒ खल्वृ॒द्ध्नव॑दब्रुवन् नब्रुवन् नृ॒द्ध्नव॒त् खलु॑ । \newline
39. ऋ॒द्ध्नव॒त् खलु॒ खल्वृ॒द्ध्नव॑दृ॒द्ध्नव॒त् खलु॒ स स खल्वृ॒द्ध्नव॑दृ॒द्ध्नव॒त् खलु॒ सः । \newline
40. खलु॒ स स खलु॒ खलु॒ स इतीति॒ स खलु॒ खलु॒ स इति॑ । \newline
41. स इतीति॒ स स इत्य॑ब्रवी दब्रवी॒दिति॒ स स इत्य॑ब्रवीत् । \newline
42. इत्य॑ब्रवी दब्रवी॒दितीत्य॑ब्रवी॒द् यो यो᳚ ऽब्रवी॒दितीत्य॑ब्रवी॒द् यः । \newline
43. अ॒ब्र॒वी॒द् यो यो᳚ ऽब्रवीदब्रवी॒द् यो म॑द्देव॒त्य॑म् मद्देव॒त्य᳚(1॒)ं ॅयो᳚ ऽब्रवीदब्रवी॒द् यो म॑द्देव॒त्य᳚म् । \newline
44. यो म॑द्देव॒त्य॑म् मद्देव॒त्यं॑ ॅयो यो म॑द्देव॒त्य॑ म॒ग्नि म॒ग्निम् म॑द्देव॒त्यं॑ ॅयो यो म॑द्देव॒त्य॑ म॒ग्निम् । \newline
45. म॒द्दे॒व॒त्य॑ म॒ग्नि म॒ग्निम् म॑द्देव॒त्य॑म् मद्देव॒त्य॑ म॒ग्नि मा॒दधा॑ता आ॒दधा॑ता अ॒ग्निम् म॑द्देव॒त्य॑म् मद्देव॒त्य॑ म॒ग्नि मा॒दधा॑तै । \newline
46. म॒द्दे॒व॒त्य॑मिति॑ मत् - दे॒व॒त्य᳚म् । \newline
47. अ॒ग्नि मा॒दधा॑ता आ॒दधा॑ता अ॒ग्नि म॒ग्नि मा॒दधा॑ता॒ इतीत्या॒दधा॑ता अ॒ग्नि म॒ग्नि मा॒दधा॑ता॒ इति॑ । \newline
48. आ॒दधा॑ता॒ इतीत्या॒दधा॑ता आ॒दधा॑ता॒ इति॒ तम् त मित्या॒दधा॑ता आ॒दधा॑ता॒ इति॒ तम् । \newline
49. आ॒दधा॑ता॒ इत्या᳚ - दधा॑तै । \newline
50. इति॒ तम् त मितीति॒ तम् पू॒षा पू॒षा त मितीति॒ तम् पू॒षा । \newline
51. तम् पू॒षा पू॒षा तम् तम् पू॒षा ऽध॑त्ताध॒त्ता पू॒षा तम् तम् पू॒षा ऽध॑त्त । \newline
52. पू॒षा ऽध॑त्ताध॒त्ता पू॒षा पू॒षा ऽध॑त्त॒ तेन॒ तेना॑ध॒त्ता पू॒षा पू॒षा ऽध॑त्त॒ तेन॑ । \newline
53. आ ऽध॑त्ताध॒त्ता ऽध॑त्त॒ तेन॒ तेना॑ध॒त्ता ऽध॑त्त॒ तेन॑ । \newline
54. अ॒ध॒त्त॒ तेन॒ तेना॑धत्ताधत्त॒ तेन॑ पू॒षा पू॒षा तेना॑धत्ताधत्त॒ तेन॑ पू॒षा । \newline
55. तेन॑ पू॒षा पू॒षा तेन॒ तेन॑ पू॒षा ऽऽर्द्ध्नो॑दार्द्ध्नोत् पू॒षा तेन॒ तेन॑ पू॒षा ऽऽर्द्ध्नो᳚त् । \newline
\pagebreak
\markright{ TS 1.5.1.3  \hfill https://www.vedavms.in \hfill}
\addcontentsline{toc}{section}{ TS 1.5.1.3 }
\section*{ TS 1.5.1.3 }

\textbf{TS 1.5.1.3 } \newline
\textbf{Samhita Paata} \newline

पू॒षाऽऽर्द्ध्नो॒त् तस्मा᳚त् पौ॒ष्णाः प॒शव॑ उच्यन्ते॒ तं त्वष्टाऽऽध॑त्त॒ तेन॒ त्वष्टा᳚ऽऽर्द्ध्नो॒त् तस्मा᳚त् त्वा॒ष्ट्राः प॒शव॑ उच्यन्ते॒ तं मनु॒राऽध॑त्त॒ तेन॒ मनु॑रा॒र्द्ध्नो॒त् तस्मा᳚न्मान॒व्यः॑ प्र॒जा उ॑च्यन्ते॒ तं धा॒ताऽऽ*ध॑त्त॒ तेन॑ धा॒ताऽऽ*र्द्ध्नो᳚थ् संॅवथ्स॒रो वै धा॒ता तस्मा᳚थ् संॅवथ्स॒रं प्र॒जाः प॒शवोऽनु॒ प्र जा॑यन्ते॒ य ए॒वं पु॑नरा॒धेय॒स्यर्द्धिं॒ ॅवेद॒- [ ] \newline

\textbf{Pada Paata} \newline

पू॒षा । आ॒र्द्ध्नो॒त् । तस्मा᳚त् । पौ॒ष्णाः । प॒शवः॑ । उ॒च्य॒न्ते॒ । तम् । त्वष्टा᳚ । एति॑ । अ॒ध॒त्त॒ । तेन॑ । त्वष्टा᳚ । आ॒र्द्ध्नो॒त् । तस्मा᳚त् । त्वा॒ष्ट्राः । प॒शवः॑ । उ॒च्य॒न्ते॒ । तम् । मनुः॑ । एति॑ । अ॒ध॒त्त॒ । तेन॑ । मनुः॑ । आ॒र्द्ध्नो॒त् । तस्मा᳚त् । मा॒न॒व्यः॑ । प्र॒जा इति॑ प्र - जाः । उ॒च्य॒न्ते॒ । तम् । धा॒ता । एति॑ । अ॒ध॒त्त॒ । तेन॑ । धा॒ता । आ॒र्द्ध्नो॒त् । सं॒ॅव॒थ्स॒र इति॑ सं - व॒थ्स॒रः । वै । धा॒ता । तस्मा᳚त् । सं॒ॅव॒थ्स॒रमिति॑ सं-व॒थ्स॒रम् । प्र॒जा इति॑ प्र-जाः । प॒शवः॑ । अनु॑ । प्रेति॑ । जा॒य॒न्ते॒ । यः । ए॒वम् । पु॒न॒रा॒धेय॒स्येति॑ पुनः - आ॒धेय॑स्य । ऋद्धि᳚म् । वेद॑ ।  \newline


\textbf{Krama Paata} \newline

पू॒षा ऽऽर्द्ध्नो᳚त् । आ॒र्द्ध्नो॒त् तस्मा᳚त् । तस्मा᳚त् पौ॒ष्णाः । पौ॒ष्णाः प॒शवः॑ । प॒शव॑ उच्यन्ते । उ॒च्य॒न्ते॒ तम् । तम् त्वष्टा᳚ । त्वष्टा । आ ऽध॑त्त । अ॒ध॒त्त॒ तेन॑ । तेन॒ त्वष्टा᳚ । त्वष्टा᳚,ऽऽर्द्ध्नोत् । आ॒र्द्ध्नो॒त्,तस्मा᳚त् । तस्मा᳚त् त्वा॒ष्ट्राः । त्वा॒ष्ट्राः प॒शवः॑ । प॒शव॑ उच्यन्ते । उ॒च्य॒न्ते॒ तम् । तम् मनुः॑ । मनु॒रा । आ ऽध॑त्त । अ॒ध॒त्त॒ तेन॑ । तेन॒ मनुः॑ । मनु॑रार्द्ध्नोत् । आ॒र्द्ध्नो॒त्,तस्मा᳚त् । तस्मा᳚न्,मान॒व्यः॑ । मा॒न॒व्यः॑ प्र॒जाः । प्र॒जा उ॑च्यन्ते । प्र॒जा इति॑ प्र - जाः । उ॒च्य॒न्ते॒ तम् । तम् धा॒ता । धा॒ता ऽऽध॑त्त । आ ऽध॑त्त । अ॒ध॒त्त॒ तेन॑ । तेन॑ धा॒ता । धा॒ता,ऽऽर्द्ध्नो᳚त् । आ॒र्द्ध्नो॒थ् स॒म्ॅव॒थ्स॒रः । स॒म्ॅव॒थ्स॒रो वै । स॒म्ॅव॒थ्स॒र इति॑ सं - व॒थ्स॒रः । वै धा॒ता । धा॒ता तस्मा᳚त् । तस्मा᳚थ् सम्ॅवथ्स॒रम् । स॒म्ॅव॒थ्स॒रम् प्र॒जाः । स॒म्ॅव॒थ्स॒रमिति॑ सम् - व॒थ्स॒रम् । प्र॒जाः प॒शवः॑ । प्रजा॒ इति॑ प्र - जाः । प॒शवो ऽनु॑ । अनु॒ प्र । प्र जा॑यन्ते । जा॒य॒न्ते॒ यः । य ए॒वम् । ए॒वम् पु॑नरा॒धेय॑स्य । पु॒न॒रा॒धेय॒स्यर्द्धि᳚म् । पु॒न॒रा॒धेय॒स्येति॑ पुनः - आ॒धेय॑स्य । ऋद्धिं॒ ॅवेद॑ । वेद॒र्द्ध्नोति॑ \newline

\textbf{Jatai Paata} \newline

1. पू॒षा ऽऽर्द्ध्नो॑दार्द्ध्नोत् पू॒षा पू॒षा ऽऽर्द्ध्नो᳚त् । \newline
2. आ॒र्द्ध्नो॒त् तस्मा॒त् तस्मा॑ दार्द्ध्नो दार्द्ध्नो॒त् तस्मा᳚त् । \newline
3. तस्मा᳚त् पौ॒ष्णाः पौ॒ष्णास्तस्मा॒त् तस्मा᳚त् पौ॒ष्णाः । \newline
4. पौ॒ष्णाः प॒शवः॑ प॒शवः॑ पौ॒ष्णाः पौ॒ष्णाः प॒शवः॑ । \newline
5. प॒शव॑ उच्यन्त उच्यन्ते प॒शवः॑ प॒शव॑ उच्यन्ते । \newline
6. उ॒च्य॒न्ते॒ तम् त मु॑च्यन्त उच्यन्ते॒ तम् । \newline
7. तम् त्वष्टा॒ त्वष्टा॒ तम् तम् त्वष्टा᳚ । \newline
8. त्वष्टा ऽऽत्वष्टा॒ त्वष्टा । \newline
9. आ ऽध॑त्ताध॒त्ता ऽध॑त्त । \newline
10. अ॒ध॒त्त॒ तेन॒ तेना॑धत्ताधत्त॒ तेन॑ । \newline
11. तेन॒ त्वष्टा॒ त्वष्टा॒ तेन॒ तेन॒ त्वष्टा᳚ । \newline
12. त्वष्टा᳚ ऽऽर्द्ध्नोदार्द्ध्नो॒त् त्वष्टा॒ त्वष्टा᳚ ऽऽर्द्ध्नोत् । \newline
13. आ॒र्द्ध्नो॒त् तस्मा॒त् तस्मा॑ दार्द्ध्नो दार्द्ध्नो॒त् तस्मा᳚त् । \newline
14. तस्मा᳚त् त्वा॒ष्ट्रा स्त्वा॒ष्ट्रा स्तस्मा॒त् तस्मा᳚त् त्वा॒ष्ट्राः । \newline
15. त्वा॒ष्ट्राः प॒शवः॑ प॒शव॑ स्त्वा॒ष्ट्रा स्त्वा॒ष्ट्राः प॒शवः॑ । \newline
16. प॒शव॑ उच्यन्त उच्यन्ते प॒शवः॑ प॒शव॑ उच्यन्ते । \newline
17. उ॒च्य॒न्ते॒ तम् त मु॑च्यन्त उच्यन्ते॒ तम् । \newline
18. तम् मनु॒र् मनु॒स्तम् तम् मनुः॑ । \newline
19. मनु॒रा मनु॒र् मनु॒रा । \newline
20. आ ऽध॑त्ताध॒त्ता ऽध॑त्त । \newline
21. अ॒ध॒त्त॒ तेन॒ तेना॑धत्ताधत्त॒ तेन॑ । \newline
22. तेन॒ मनु॒र् मनु॒स्तेन॒ तेन॒ मनुः॑ । \newline
23. मनु॑ रार्द्ध्नो दार्द्ध्नो॒न् मनु॒र् मनु॑ रार्द्ध्नोत् । \newline
24. आ॒र्द्ध्नो॒त् तस्मा॒त् तस्मा॑ दार्द्ध्नो दार्द्ध्नो॒त् तस्मा᳚त् । \newline
25. तस्मा᳚न् मान॒व्यो॑ मान॒व्य॑ स्तस्मा॒त् तस्मा᳚न् मान॒व्यः॑ । \newline
26. मा॒न॒व्यः॑ प्र॒जाः प्र॒जा मा॑न॒व्यो॑ मान॒व्यः॑ प्र॒जाः । \newline
27. प्र॒जा उ॑च्यन्त उच्यन्ते प्र॒जाः प्र॒जा उ॑च्यन्ते । \newline
28. प्र॒जा इति॑ प्र - जाः । \newline
29. उ॒च्य॒न्ते॒ तम् त मु॑च्यन्त उच्यन्ते॒ तम् । \newline
30. तम् धा॒ता धा॒ता तम् तम् धा॒ता । \newline
31. धा॒ता ऽध॑त्ताध॒त्ता धा॒ता धा॒ता ऽध॑त्त । \newline
32. आ ऽध॑त्ताध॒त्ता ऽध॑त्त । \newline
33. अ॒ध॒त्त॒ तेन॒ तेना॑धत्ताधत्त॒ तेन॑ । \newline
34. तेन॑ धा॒ता धा॒ता तेन॒ तेन॑ धा॒ता । \newline
35. धा॒ता ऽऽर्द्ध्नो॑ दार्द्ध्नोद् धा॒ता धा॒ता ऽऽर्द्ध्नो᳚त् । \newline
36. आ॒र्द्ध्नो॒थ् सं॒ॅव॒थ्स॒रः सं॑ॅवथ्स॒र आ᳚र्द्ध्नो दार्द्ध्नोथ् संॅवथ्स॒रः । \newline
37. सं॒ॅव॒थ्स॒रो वै वै सं॑ॅवथ्स॒रः सं॑ॅवथ्स॒रो वै । \newline
38. सं॒ॅव॒थ्स॒र इति॑ सं - व॒थ्स॒रः । \newline
39. वै धा॒ता धा॒ता वै वै धा॒ता । \newline
40. धा॒ता तस्मा॒त् तस्मा᳚द् धा॒ता धा॒ता तस्मा᳚त् । \newline
41. तस्मा᳚थ् संॅवथ्स॒रꣳ सं॑ॅवथ्स॒रम् तस्मा॒त् तस्मा᳚थ् संॅवथ्स॒रम् । \newline
42. सं॒ॅव॒थ्स॒रम् प्र॒जाः प्र॒जाः सं॑ॅवथ्स॒रꣳ सं॑ॅवथ्स॒रम् प्र॒जाः । \newline
43. सं॒ॅव॒थ्स॒रमिति॑ सं - व॒थ्स॒रम् । \newline
44. प्र॒जाः प॒शवः॑ प॒शवः॑ प्र॒जाः प्र॒जाः प॒शवः॑ । \newline
45. प्र॒जा इति॑ प्र - जाः । \newline
46. प॒शवो ऽन्वनु॑ प॒शवः॑ प॒शवो ऽनु॑ । \newline
47. अनु॒ प्र प्राण्वनु॒ प्र । \newline
48. प्र जा॑यन्ते जायन्ते॒ प्र प्र जा॑यन्ते । \newline
49. जा॒य॒न्ते॒ यो यो जा॑यन्ते जायन्ते॒ यः । \newline
50. य ए॒व मे॒वं ॅयो य ए॒वम् । \newline
51. ए॒वम् पु॑नरा॒धेय॑स्य पुनरा॒धेय॑स्यै॒व मे॒वम् पु॑नरा॒धेय॑स्य । \newline
52. पु॒न॒रा॒धेय॒स्य र्द्धि॒ मृद्धि॑म् पुनरा॒धेय॑स्य पुनरा॒धेय॒स्य र्द्धि᳚म् । \newline
53. पु॒न॒रा॒धेय॒स्येति॑ पुनः - आ॒धेय॑स्य । \newline
54. ऋद्धिं॒ ॅवेद॒ वेद र्द्धि॒ मृद्धिं॒ ॅवेद॑ । \newline
55. वेद॒ र्द्ध्नोत्यृ॒द्ध्नोति॒ वेद॒ वेद॒ र्द्ध्नोति॑ । \newline

\textbf{Ghana Paata } \newline

1. पू॒षा ऽऽर्द्ध्नो॑दार्द्ध्नोत् पू॒षा पू॒षा ऽऽर्द्ध्नो॒त् तस्मा॒त् तस्मा॑दार्द्ध्नोत् पू॒षा पू॒षा ऽऽर्द्ध्नो॒त् तस्मा᳚त् । \newline
2. आ॒र्द्ध्नो॒त् तस्मा॒त् तस्मा॑दार्द्ध्नो दार्द्ध्नो॒त् तस्मा᳚त् पौ॒ष्णाः पौ॒ष्णा स्तस्मा॑ दार्द्ध्नो दार्द्ध्नो॒त् तस्मा᳚त् पौ॒ष्णाः । \newline
3. तस्मा᳚त् पौ॒ष्णाः पौ॒ष्णा स्तस्मा॒त् तस्मा᳚त् पौ॒ष्णाः प॒शवः॑ प॒शवः॑ पौ॒ष्णा स्तस्मा॒त् तस्मा᳚त् पौ॒ष्णाः प॒शवः॑ । \newline
4. पौ॒ष्णाः प॒शवः॑ प॒शवः॑ पौ॒ष्णाः पौ॒ष्णाः प॒शव॑ उच्यन्त उच्यन्ते प॒शवः॑ पौ॒ष्णाः पौ॒ष्णाः प॒शव॑ उच्यन्ते । \newline
5. प॒शव॑ उच्यन्त उच्यन्ते प॒शवः॑ प॒शव॑ उच्यन्ते॒ तम् त मु॑च्यन्ते प॒शवः॑ प॒शव॑ उच्यन्ते॒ तम् । \newline
6. उ॒च्य॒न्ते॒ तम् त मु॑च्यन्त उच्यन्ते॒ तम् त्वष्टा॒ त्वष्टा॒ त मु॑च्यन्त उच्यन्ते॒ तम् त्वष्टा᳚ । \newline
7. तम् त्वष्टा॒ त्वष्टा॒ तम् तम् त्वष्टा ऽऽत्वष्टा॒ तम् तम् त्वष्टा । \newline
8. त्वष्टा ऽऽत्वष्टा॒ त्वष्टा ऽध॑त्ताध॒त्ता त्वष्टा॒ त्वष्टा ऽध॑त्त । \newline
9. आ ऽध॑त्ताध॒त्ता ऽध॑त्त॒ तेन॒ तेना॑ध॒त्ता ऽध॑त्त॒ तेन॑ । \newline
10. अ॒ध॒त्त॒ तेन॒ तेना॑धत्ताधत्त॒ तेन॒ त्वष्टा॒ त्वष्टा॒ तेना॑धत्ताधत्त॒ तेन॒ त्वष्टा᳚ । \newline
11. तेन॒ त्वष्टा॒ त्वष्टा॒ तेन॒ तेन॒ त्वष्टा᳚ ऽऽर्द्ध्नोदार्द्ध्नो॒त् त्वष्टा॒ तेन॒ तेन॒ त्वष्टा᳚ ऽऽर्द्ध्नोत् । \newline
12. त्वष्टा᳚ ऽऽर्द्ध्नो दार्द्ध्नो॒त् त्वष्टा॒ त्वष्टा᳚ ऽऽर्द्ध्नो॒त् तस्मा॒त् तस्मा॑ दार्द्ध्नो॒त् त्वष्टा॒ त्वष्टा᳚ ऽऽर्द्ध्नो॒त् तस्मा᳚त् । \newline
13. आ॒र्द्ध्नो॒त् तस्मा॒त् तस्मा॑ दार्द्ध्नो दार्द्ध्नो॒त् तस्मा᳚त् त्वा॒ष्ट्रा स्त्वा॒ष्ट्रा स्तस्मा॑ दार्द्ध्नो दार्द्ध्नो॒त् तस्मा᳚त् त्वा॒ष्ट्राः । \newline
14. तस्मा᳚त् त्वा॒ष्ट्रा स्त्वा॒ष्ट्रा स्तस्मा॒त् तस्मा᳚त् त्वा॒ष्ट्राः प॒शवः॑ प॒शव॑ स्त्वा॒ष्ट्रा स्तस्मा॒त् तस्मा᳚त् त्वा॒ष्ट्राः प॒शवः॑ । \newline
15. त्वा॒ष्ट्राः प॒शवः॑ प॒शव॑ स्त्वा॒ष्ट्रा स्त्वा॒ष्ट्राः प॒शव॑ उच्यन्त उच्यन्ते प॒शव॑ स्त्वा॒ष्ट्रा स्त्वा॒ष्ट्राः प॒शव॑ उच्यन्ते । \newline
16. प॒शव॑ उच्यन्त उच्यन्ते प॒शवः॑ प॒शव॑ उच्यन्ते॒ तम् त मु॑च्यन्ते प॒शवः॑ प॒शव॑ उच्यन्ते॒ तम् । \newline
17. उ॒च्य॒न्ते॒ तम् त मु॑च्यन्त उच्यन्ते॒ तम् मनु॒र् मनु॒स्त मु॑च्यन्त उच्यन्ते॒ तम् मनुः॑ । \newline
18. तम् मनु॒र् मनु॒ स्तम् तम् मनु॒रा मनु॒स्तम् तम् मनु॒रा । \newline
19. मनु॒रा मनु॒र् मनु॒रा ऽध॑त्ताध॒त्ता मनु॒र् मनु॒रा ऽध॑त्त । \newline
20. आ ऽध॑त्ताध॒त्ता ऽध॑त्त॒ तेन॒ तेना॑ध॒त्ता ऽध॑त्त॒ तेन॑ । \newline
21. अ॒ध॒त्त॒ तेन॒ तेना॑धत्ताधत्त॒ तेन॒ मनु॒र् मनु॒ स्तेना॑धत्ताधत्त॒ तेन॒ मनुः॑ । \newline
22. तेन॒ मनु॒र् मनु॒स्तेन॒ तेन॒ मनु॑रार्द्ध्नो दार्द्ध्नो॒न् मनु॒स्तेन॒ तेन॒ मनु॑रार्द्ध्नोत् । \newline
23. मनु॑रार्द्ध्नो दार्द्ध्नो॒न् मनु॒र् मनु॑रार्द्ध्नो॒त् तस्मा॒त् तस्मा॑ दार्द्ध्नो॒न् मनु॒र् मनु॑रार्द्ध्नो॒त् तस्मा᳚त् । \newline
24. आ॒र्द्ध्नो॒त् तस्मा॒त् तस्मा॑ दार्द्ध्नो दार्द्ध्नो॒त् तस्मा᳚न् मान॒व्यो॑ मान॒व्य॑ स्तस्मा॑ दार्द्ध्नो दार्द्ध्नो॒त् तस्मा᳚न् मान॒व्यः॑ । \newline
25. तस्मा᳚न् मान॒व्यो॑ मान॒व्य॑ स्तस्मा॒त् तस्मा᳚न् मान॒व्यः॑ प्र॒जाः प्र॒जा मा॑न॒व्य॑ स्तस्मा॒त् तस्मा᳚न् मान॒व्यः॑ प्र॒जाः । \newline
26. मा॒न॒व्यः॑ प्र॒जाः प्र॒जा मा॑न॒व्यो॑ मान॒व्यः॑ प्र॒जा उ॑च्यन्त उच्यन्ते प्र॒जा मा॑न॒व्यो॑ मान॒व्यः॑ प्र॒जा उ॑च्यन्ते । \newline
27. प्र॒जा उ॑च्यन्त उच्यन्ते प्र॒जाः प्र॒जा उ॑च्यन्ते॒ तम् त मु॑च्यन्ते प्र॒जाः प्र॒जा उ॑च्यन्ते॒ तम् । \newline
28. प्र॒जा इति॑ प्र - जाः । \newline
29. उ॒च्य॒न्ते॒ तम् त मु॑च्यन्त उच्यन्ते॒ तम् धा॒ता धा॒ता त मु॑च्यन्त उच्यन्ते॒ तम् धा॒ता । \newline
30. तम् धा॒ता धा॒ता तम् तम् धा॒ता ऽध॑त्ताध॒त्ता धा॒ता तम् तम् धा॒ता ऽध॑त्त । \newline
31. धा॒ता ऽध॒त्ता ध॑तधा॒ता धा॒ता ध॑त॒ तेन॒ तेना ऽध॑त्त धा॒ता धा॒ता ऽध॑त्त॒ तेन॑ । \newline
32. आ ऽध॑त्ताध॒त्ता ऽध॑त्त॒ तेन॒ तेना॑ध॒त्ता ऽध॑त्त॒ तेन॑ । \newline
33. अ॒ध॒त्त॒ तेन॒ तेना॑धत्ताधत्त॒ तेन॑ धा॒ता धा॒ता तेना॑धत्ताधत्त॒ तेन॑ धा॒ता । \newline
34. तेन॑ धा॒ता धा॒ता तेन॒ तेन॑ धा॒ता ऽऽर्द्ध्नो॑ दार्द्ध्नोद् धा॒ता तेन॒ तेन॑ धा॒ता ऽऽर्द्ध्नो᳚त् । \newline
35. धा॒ता ऽऽर्द्ध्नो॑ दार्द्ध्नोद् धा॒ता धा॒ता ऽऽर्द्ध्नो᳚थ् संॅवथ्स॒रः सं॑ॅवथ्स॒र आ᳚र्द्ध्नोद् धा॒ता धा॒ता ऽऽर्द्ध्नो᳚थ् संॅवथ्स॒रः । \newline
36. आ॒र्द्ध्नो॒थ् सं॒ॅव॒थ्स॒रः सं॑ॅवथ्स॒र आ᳚र्द्ध्नो दार्द्ध्नोथ् संॅवथ्स॒रो वै वै सं॑ॅवथ्स॒र आ᳚र्द्ध्नो दार्द्ध्नोथ् संॅवथ्स॒रो वै । \newline
37. सं॒ॅव॒थ्स॒रो वै वै सं॑ॅवथ्स॒रः सं॑ॅवथ्स॒रो वै धा॒ता धा॒ता वै सं॑ॅवथ्स॒रः सं॑ॅवथ्स॒रो वै धा॒ता । \newline
38. सं॒ॅव॒थ्स॒र इति॑ सं - व॒थ्स॒रः । \newline
39. वै धा॒ता धा॒ता वै वै धा॒ता तस्मा॒त् तस्मा᳚द् धा॒ता वै वै धा॒ता तस्मा᳚त् । \newline
40. धा॒ता तस्मा॒त् तस्मा᳚द् धा॒ता धा॒ता तस्मा᳚थ् संॅवथ्स॒रꣳ सं॑ॅवथ्स॒रम् तस्मा᳚द् धा॒ता धा॒ता तस्मा᳚थ् संॅवथ्स॒रम् । \newline
41. तस्मा᳚थ् संॅवथ्स॒रꣳ सं॑ॅवथ्स॒रम् तस्मा॒त् तस्मा᳚थ् संॅवथ्स॒रम् प्र॒जाः प्र॒जाः सं॑ॅवथ्स॒रम् तस्मा॒त् तस्मा᳚थ् संॅवथ्स॒रम् प्र॒जाः । \newline
42. सं॒ॅव॒थ्स॒रम् प्र॒जाः प्र॒जाः सं॑ॅवथ्स॒रꣳ सं॑ॅवथ्स॒रम् प्र॒जाः प॒शवः॑ प॒शवः॑ प्र॒जाः सं॑ॅवथ्स॒रꣳ सं॑ॅवथ्स॒रम् प्र॒जाः प॒शवः॑ । \newline
43. सं॒ॅव॒थ्स॒रमिति॑ सं - व॒थ्स॒रम् । \newline
44. प्र॒जाः प॒शवः॑ प॒शवः॑ प्र॒जाः प्र॒जाः प॒शवो ऽन्वनु॑ प॒शवः॑ प्र॒जाः प्र॒जाः प॒शवो ऽनु॑ । \newline
45. प्र॒जा इति॑ प्र - जाः । \newline
46. प॒शवो ऽन्वनु॑ प॒शवः॑ प॒शवो ऽनु॒ प्र प्राणु॑ प॒शवः॑ प॒शवो ऽनु॒ प्र । \newline
47. अनु॒ प्र प्राण्वनु॒ प्र जा॑यन्ते जायन्ते॒ प्राण्वनु॒ प्र जा॑यन्ते । \newline
48. प्र जा॑यन्ते जायन्ते॒ प्र प्र जा॑यन्ते॒ यो यो जा॑यन्ते॒ प्र प्र जा॑यन्ते॒ यः । \newline
49. जा॒य॒न्ते॒ यो यो जा॑यन्ते जायन्ते॒ य ए॒व मे॒वं ॅयो जा॑यन्ते जायन्ते॒ य ए॒वम् । \newline
50. य ए॒व मे॒वं ॅयो य ए॒वम् पु॑नरा॒धेय॑स्य पुनरा॒धेय॑स्यै॒वं ॅयो य ए॒वम् पु॑नरा॒धेय॑स्य । \newline
51. ए॒वम् पु॑नरा॒धेय॑स्य पुनरा॒धेय॑स्यै॒व मे॒वम् पु॑नरा॒धेय॒स्य र्‌द्धि॒ मृद्धि॑म् पुनरा॒धेय॑स्यै॒व मे॒वम् पु॑नरा॒धेय॒स्य र्‌द्धि᳚म् । \newline
52. पु॒न॒रा॒धेय॒स्यर्‌द्धि॒ मृद्धि॑म् पुनरा॒धेय॑स्य पुनरा॒धेय॒स्यर्‌द्धिं॒ ॅवेद॒ वेदर्‌द्धि॑म् पुनरा॒धेय॑स्य पुनरा॒धेय॒स्यर्‌द्धिं॒ ॅवेद॑ । \newline
53. पु॒न॒रा॒धेय॒स्येति॑ पुनः - आ॒धेय॑स्य । \newline
54. ऋद्धिं॒ ॅवेद॒ वेदर्‌द्धि॒ मृद्धिं॒ ॅवेद॒ र्‌द्ध्नो त्यृ॒द्ध्नोति॒ वेदर्‌द्धि॒ मृद्धिं॒ ॅवेद॒ र्‌द्ध्नोति॑ । \newline
55. वेद॒ र्‌द्ध्नो त्यृ॒द्ध्नोति॒ वेद॒ वेद॒ र्‌द्ध्नोत्ये॒वैव र्‌द्ध्नोति॒ वेद॒ वेद॒ र्‌द्ध्नोत्ये॒व । \newline
\pagebreak
\markright{ TS 1.5.1.4  \hfill https://www.vedavms.in \hfill}
\addcontentsline{toc}{section}{ TS 1.5.1.4 }
\section*{ TS 1.5.1.4 }

\textbf{TS 1.5.1.4 } \newline
\textbf{Samhita Paata} \newline

र्द्ध्नोत्ये॒व यो᳚ऽस्यै॒वं ब॒न्धुतां॒ ॅवेद॒ बन्धु॑मान् भवति भाग॒धेयं॒ ॅवा अ॒ग्निराहि॑त इ॒च्छमा॑नः प्र॒जां प॒शून् यज॑मान॒स्योप॑ दोद्रावो॒द्वास्य॒ पुन॒रा द॑धीत भाग॒धेये॑नै॒वैनꣳ॒॒ सम॑र्द्धय॒त्यथो॒ शान्ति॑रे॒वास्यै॒षा पुन॑र्वस्वो॒रा द॑धीतै॒तद्वै पु॑नरा॒धेय॑स्य॒ नक्ष॑त्रं॒ ॅयत् पुन॑र्वसू॒ स्वाया॑मे॒वैनं॑ दे॒वता॑यामा॒धाय॑ ब्रह्मवर्च॒सी भ॑वति द॒र्भै ( ) रा द॑धा॒त्यया॑तयामत्वाय द॒र्भैरा द॑धात्य॒द्भ्य ए॒वैन॒मोष॑धीभ्यो ऽव॒रुद्ध्या ऽऽ*ध॑त्ते॒ पञ्च॑कपालः पुरो॒डाशो॑ भवति॒ पञ्च॒ वा ऋ॒तव॑ ऋ॒तुभ्य॑ ए॒वैन॑मव॒रुद्ध्या ऽऽ*ध॑त्ते ॥ \newline

\textbf{Pada Paata} \newline

ऋ॒द्ध्नोति॑ । ए॒व । यः । अ॒स्य॒ । ए॒वम् । ब॒न्धुता᳚म् । वेद॑ । बन्धु॑मा॒निति॒ बन्धु॑ - मा॒न् । भ॒व॒ति॒ । भा॒ग॒धेय॒मिति॑ भाग -धेय᳚म् । वै । अ॒ग्निः । आहि॑त॒ इत्या - हि॒तः॒ । इ॒च्छमा॑नः । प्र॒जामिति॑ प्र-जाम् । प॒शून् । यज॑मानस्य । उपेति॑ । दो॒द्रा॒व॒ । उ॒द्वास्येत्यु॑त् - वास्य॑ । पुनः॑ । एति॑ । द॒धी॒त॒ । भा॒ग॒धेये॒नेति॑ भाग - धेये॑न । ए॒व । ए॒न॒म् । समिति॑ । अ॒र्ध॒य॒ति॒ । अथो॒ इति॑ । शान्तिः॑ । ए॒व । अ॒स्य॒ । ए॒षा । पुन॑र्वस्वो॒रिति॒ पुनः॑ - व॒स्वोः॒ । एति॑ । द॒धी॒त॒ । ए॒तत् । वै । पु॒न॒रा॒धेय॒स्येति॑ पुनः - आ॒धेय॑स्य । नक्ष॑त्रम् । यत् । पुन॑र्वसू॒ इति॒ पुनः॑ - व॒सू॒ । स्वाया᳚म् । ए॒व । ए॒न॒म् । दे॒वता॑याम् । आ॒धायेत्या᳚ - धाय॑ । ब्र॒ह्म॒व॒र्च॒सीति॑ ब्रह्म - व॒र्च॒सी । भ॒व॒ति॒ । द॒र्भैः ( ) । एति॑ । द॒धा॒ति॒ । अया॑तयामत्वा॒येत्यया॑तयाम - त्वा॒य॒ । द॒र्भैः । एति॑ । द॒धा॒ति॒ । अ॒द्भ्य इत्य॑त् - भ्यः । ए॒व । ए॒न॒म् । ओष॑धीभ्य॒ इत्योष॑धि-भ्यः॒ । अ॒व॒रुद्ध्येत्य॑व - रुद्ध्य॑ । एति॑ । ध॒त्ते॒ । पञ्च॑कपाल॒ इति॒ पञ्च॑ - क॒पा॒लः॒ । पु॒रो॒डाशः॑ । भ॒व॒ति॒ । पञ्च॑ । वै । ऋ॒तवः॑ । ऋ॒तुभ्य॒ इत्यृ॒तु - भ्यः॒ । ए॒व । ए॒न॒म् । अ॒व॒रुद्ध्येत्य॑व - रुद्ध्य॑ । एति॑ । ध॒त्ते॒ ॥  \newline


\textbf{Krama Paata} \newline

ऋ॒द्ध्नोत्ये॒व । ए॒व यः । यो᳚ऽस्य । अ॒स्यै॒वम् । ए॒वम् ब॒न्धुता᳚म् । ब॒न्धुतां॒ ॅवेद॑ । वेद॒ बन्धु॑मान् । बन्धु॑मान् भवति । बन्धु॑मा॒निति॒ बन्धु॑ - मा॒न्॒ । भ॒व॒ति॒ भा॒ग॒धेय᳚म् । भा॒ग॒धेयं॒ ॅवै । भा॒ग॒धेय॒मिति॑ भाग - धेय᳚म् । वा अ॒ग्निः । अ॒ग्निराहि॑तः । आहि॑त इ॒च्छमा॑नः । आहि॑त॒ इत्या - हि॒तः॒ । इ॒च्छमा॑नः प्र॒जाम् । प्र॒जाम् प॒शून् । प्र॒जामिति॑ प्र - जाम् । प॒शून्. यज॑मानस्य । यज॑मान॒स्योप॑ । उप॑ दोद्राव । दो॒द्रा॒वो॒द्वास्य॑ । उ॒द्वास्य॒ पुनः॑ । उ॒द्वास्येत्यु॑त् - वास्य॑ । पुन॒रा । आ द॑धीत । द॒धी॒त॒ भा॒ग॒धेये॑न । भा॒ग॒धेये॑नै॒व । भा॒ग॒धेये॒नेति॑ भाग - धेये॑न । ए॒वैन᳚म् । ए॒नꣳ॒॒ सम् । सम॑र्द्धयति । अ॒र्द्ध॒य॒त्यथो᳚ । अथो॒ शान्तिः॑ । अथो॒ इत्यथो᳚ । शान्ति॑रे॒व । ए॒वास्य॑ । अ॒स्यै॒षा । ए॒षा पुन॑र्वस्वोः । पुन॑र्वस्वो॒रा । पुन॑र्वस्वो॒रिति॒ पुनः॑ - व॒स्वोः॒ । आ द॑धीत । द॒धी॒तै॒तत् । ए॒तद् वै । वै पु॑नरा॒धेय॑स्य । पु॒न॒रा॒धेय॑स्य॒ नक्ष॑त्रम् । पु॒न॒रा॒धेय॒स्येति॑ पुनः - आ॒धेय॑स्य । नक्ष॑त्रं॒ ॅयत् । यत् पुन॑र्वसू । पुन॑र्वसू॒ स्वाया᳚म् । पुन॑र्वसू॒ इति॒ पुनः॑ - व॒सू॒ । स्वाया॑मे॒व । ए॒वैन᳚म् । ए॒न॒म् दे॒वता॑याम् । दे॒वता॑यामा॒धाय॑ । आ॒धाय॑ ब्रह्मवर्च॒सी । आ॒धायेत्या᳚ - धाय॑ । ब्र॒ह्म॒व॒र्च॒सी भ॑वति । ब्र॒ह्म॒व॒र्च॒सीति॑ ब्रह्म - व॒र्च॒सी । भ॒व॒ति॒ द॒र्भैः । द॒र्भैरा । आ द॑धाति । द॒धा॒त्यया॑तयामत्वाय । अया॑तयामत्वाय द॒र्भैः ( ) । अया॑तयामत्वा॒येत्यया॑तयाम - त्वा॒य॒ । द॒र्भैरा । आ द॑धाति । द॒धा॒त्य॒द्भ्यः । अ॒द्भ्य ए॒व । अ॒द्भ्य इत्य॑त् - भ्यः । ए॒वैन᳚म् । ए॒न॒मोष॑धीभ्यः । ओष॑धीभ्यो ऽव॒रुद्ध्य॑ । ओष॑धीभ्य॒ इत्योष॑धि - भ्यः॒ । अ॒व॒रुद्ध्या । अ॒व॒रुद्ध्येत्य॑व - रुद्ध्य॑ । आ ध॑त्ते । ध॒त्ते॒ पञ्च॑कपालः । पञ्च॑कपालः पुरो॒डाशः॑ । पञ्च॑कपाल॒ इति॒ पञ्च॑ - क॒पा॒लः॒ । पु॒रो॒डाशो॑ भवति । भ॒व॒ति॒ पञ्च॑ । पञ्च॒ वै । वा ऋ॒तवः॑ । ऋ॒तव॑ ऋ॒तुभ्यः॑ । ऋ॒तुभ्य॑ ए॒व । ऋ॒तुभ्य॒ इत्यृ॒तु - भ्यः॒ । ए॒वैन᳚म् । ए॒न॒म॒व॒रुद्ध्य॑ । अ॒व॒रुद्ध्या । अ॒व॒रुद्ध्येत्य॑व - रुद्ध्य॑ । आ ध॑त्ते । ध॒त्त॒ इति॑ धत्ते । \newline

\textbf{Jatai Paata} \newline

1. ऋ॒द्ध्नोत्ये॒वैव र्द्ध्नोत्यृ॒द्ध्नोत्ये॒व । \newline
2. ए॒व यो य ए॒वैव यः । \newline
3. यो᳚ ऽस्यास्य॒ यो यो᳚ ऽस्य । \newline
4. अ॒स्यै॒व मे॒व म॑स्यास्यै॒वम् । \newline
5. ए॒वम् ब॒न्धुता᳚म् ब॒न्धुता॑ मे॒व मे॒वम् ब॒न्धुता᳚म् । \newline
6. ब॒न्धुतां॒ ॅवेद॒ वेद॑ ब॒न्धुता᳚म् ब॒न्धुतां॒ ॅवेद॑ । \newline
7. वेद॒ बन्धु॑मा॒न् बन्धु॑मा॒न्॒. वेद॒ वेद॒ बन्धु॑मान् । \newline
8. बन्धु॑मान् भवति भवति॒ बन्धु॑मा॒न् बन्धु॑मान् भवति । \newline
9. बन्धु॑मा॒निति॒ बन्धु॑ - मा॒न् । \newline
10. भ॒व॒ति॒ भा॒ग॒धेय॑म् भाग॒धेय॑म् भवति भवति भाग॒धेय᳚म् । \newline
11. भा॒ग॒धेयं॒ ॅवै वै भा॑ग॒धेय॑म् भाग॒धेयं॒ ॅवै । \newline
12. भा॒ग॒धेय॒मिति॑ भाग - धेय᳚म् । \newline
13. वा अ॒ग्नि र॒ग्निर् वै वा अ॒ग्निः । \newline
14. अ॒ग्निराहि॑त॒ आहि॑तो॒ ऽग्निर॒ग्निराहि॑तः । \newline
15. आहि॑त इ॒च्छमा॑न इ॒च्छमा॑न॒ आहि॑त॒ आहि॑त इ॒च्छमा॑नः । \newline
16. आहि॑त॒ इत्या - हि॒तः॒ । \newline
17. इ॒च्छमा॑नः प्र॒जाम् प्र॒जा मि॒च्छमा॑न इ॒च्छमा॑नः प्र॒जाम् । \newline
18. प्र॒जाम् प॒शून् प॒शून् प्र॒जाम् प्र॒जाम् प॒शून् । \newline
19. प्र॒जामिति॑ प्र - जाम् । \newline
20. प॒शून्. यज॑मानस्य॒ यज॑मानस्य प॒शून् प॒शून्. यज॑मानस्य । \newline
21. यज॑मान॒स्योपोप॒ यज॑मानस्य॒ यज॑मान॒स्योप॑ । \newline
22. उप॑ दोद्राव दोद्रा॒वोपोप॑ दोद्राव । \newline
23. दो॒द्रा॒वो॒द्वास्यो॒द्वास्य॑ दोद्राव दोद्रावो॒द्वास्य॑ । \newline
24. उ॒द्वास्य॒ पुनः॒ पुन॑ रु॒द्वास्यो॒द्वास्य॒ पुनः॑ । \newline
25. उ॒द्वास्येत्यु॑त् - वास्य॑ । \newline
26. पुन॒ रा पुनः॒ पुन॒ रा । \newline
27. आ द॑धीत दधी॒ता द॑धीत । \newline
28. द॒धी॒त॒ भा॒ग॒धेये॑न भाग॒धेये॑न दधीत दधीत भाग॒धेये॑न । \newline
29. भा॒ग॒धेये॑नै॒वैव भा॑ग॒धेये॑न भाग॒धेये॑नै॒व । \newline
30. भा॒ग॒धेये॒नेति॑ भाग - धेये॑न । \newline
31. ए॒वैन॑ मेन मे॒वैवैन᳚म् । \newline
32. ए॒न॒(ग्म्॒) सꣳ स मे॑न मेन॒(ग्म्॒) सम् । \newline
33. स म॑र्द्धयत्यर्द्धयति॒ सꣳ स म॑र्द्धयति । \newline
34. अ॒र्द्ध॒य॒त्यथो॒ अथो॑ अर्द्धयत्यर्द्धय॒त्यथो᳚ । \newline
35. अथो॒ शान्तिः॒ शान्ति॒रथो॒ अथो॒ शान्तिः॑ । \newline
36. अथो॒ इत्यथो᳚ । \newline
37. शान्ति॑रे॒वैव शान्तिः॒ शान्ति॑रे॒व । \newline
38. ए॒वास्या᳚स्यै॒वैवास्य॑ । \newline
39. अ॒स्यै॒षैषा ऽस्या᳚स्यै॒षा । \newline
40. ए॒षा पुन॑र्वस्वोः॒ पुन॑र्वस्वो रे॒षैषा पुन॑र्वस्वोः । \newline
41. पुन॑र्वस्वो॒रा पुन॑र्वस्वोः॒ पुन॑र्वस्वो॒रा । \newline
42. पुन॑र्वस्वो॒रिति॒ पुनः॑ - व॒स्वोः॒ । \newline
43. आ द॑धीत दधी॒ता द॑धीत । \newline
44. द॒धी॒तै॒तदे॒तद् द॑धीत दधीतै॒तत् । \newline
45. ए॒तद् वै वा ए॒तदे॒तद् वै । \newline
46. वै पु॑नरा॒धेय॑स्य पुनरा॒धेय॑स्य॒ वै वै पु॑नरा॒धेय॑स्य । \newline
47. पु॒न॒रा॒धेय॑स्य॒ नक्ष॑त्र॒न्नक्ष॑त्रम् पुनरा॒धेय॑स्य पुनरा॒धेय॑स्य॒ नक्ष॑त्रम् । \newline
48. पु॒न॒रा॒धेय॒स्येति॑ पुनः - आ॒धेय॑स्य । \newline
49. नक्ष॑त्रं॒ ॅयद् यन् नक्ष॑त्र॒न्नक्ष॑त्रं॒ ॅयत् । \newline
50. यत् पुन॑र्वसू॒ पुन॑र्वसू॒ यद् यत् पुन॑र्वसू । \newline
51. पुन॑र्वसू॒ स्वाया॒(ग्ग्॒) स्वाया॒म् पुन॑र्वसू॒ पुन॑र्वसू॒ स्वाया᳚म् । \newline
52. पुन॑र्वसू॒ इति॒ पुनः॑ - व॒सू॒ । \newline
53. स्वाया॑ मे॒वैव स्वाया॒(ग्ग्॒) स्वाया॑ मे॒व । \newline
54. ए॒वैन॑ मेन मे॒वैवैन᳚म् । \newline
55. ए॒न॒म् दे॒वता॑याम् दे॒वता॑या मेन मेनम् दे॒वता॑याम् । \newline
56. दे॒वता॑या मा॒धाया॒धाय॑ दे॒वता॑याम् दे॒वता॑या मा॒धाय॑ । \newline
57. आ॒धाय॑ ब्रह्मवर्च॒सी ब्र॑ह्मवर्च॒स्या॑धाया॒धाय॑ ब्रह्मवर्च॒सी । \newline
58. आ॒धायेत्या᳚ - धाय॑ । \newline
59. ब्र॒ह्म॒व॒र्च॒सी भ॑वति भवति ब्रह्मवर्च॒सी ब्र॑ह्मवर्च॒सी भ॑वति । \newline
60. ब्र॒ह्म॒व॒र्च॒सीति॑ ब्रह्म - व॒र्च॒सी । \newline
61. भ॒व॒ति॒ द॒र्भैर् द॒र्भैर् भ॑वति भवति द॒र्भैः । \newline
62. द॒र्भैरा द॒र्भैर् द॒र्भैरा । \newline
63. आ द॑धाति दधा॒त्या द॑धाति । \newline
64. द॒धा॒त्यया॑तयामत्वा॒याया॑तयामत्वाय दधाति दधा॒त्यया॑तयामत्वाय । \newline
65. अया॑तयामत्वाय द॒र्भैर् द॒र्भै रया॑तयामत्वा॒याया॑तयामत्वाय द॒र्भैः । \newline
66. अया॑तयामत्वा॒येत्यया॑तयाम - त्वा॒य॒ । \newline
67. द॒र्भैरा द॒र्भैर् द॒र्भैरा । \newline
68. आ द॑धाति दधा॒त्या द॑धाति । \newline
69. द॒धा॒त्य॒द्भ्यो᳚ ऽद्भ्यो द॑धाति दधात्य॒द्भ्यः । \newline
70. अ॒द्भ्य ए॒वैवाद्भ्यो᳚ ऽद्भ्य ए॒व । \newline
71. अ॒द्भ्य इत्य॑त् - भ्यः । \newline
72. ए॒वैन॑ मेन मे॒वैवैन᳚म् । \newline
73. ए॒न॒ मोष॑धीभ्य॒ ओष॑धीभ्य एन मेन॒ मोष॑धीभ्यः । \newline
74. ओष॑धीभ्यो ऽव॒रुद्ध्या॑व॒रुद्ध्यौष॑धीभ्य॒ ओष॑धीभ्यो ऽव॒रुद्ध्य॑ । \newline
75. ओष॑धीभ्य॒ इत्योष॑धि - भ्यः॒ । \newline
76. अ॒व॒रुद्ध्या ऽव॒रुद्ध्या॑व॒रुद्ध्या । \newline
77. अ॒व॒रुद्ध्येत्य॑व - रुद्ध्य॑ । \newline
78. आ ध॑त्ते धत्त॒ आ ध॑त्ते । \newline
79. ध॒त्ते॒ पञ्च॑कपालः॒ पञ्च॑कपालो धत्ते धत्ते॒ पञ्च॑कपालः । \newline
80. पञ्च॑कपालः पुरो॒डाशः॑ पुरो॒डाशः॒ पञ्च॑कपालः॒ पञ्च॑कपालः पुरो॒डाशः॑ । \newline
81. पञ्च॑कपाल॒ इति॒ पञ्च॑ - क॒पा॒लः॒ । \newline
82. पु॒रो॒डाशो॑ भवति भवति पुरो॒डाशः॑ पुरो॒डाशो॑ भवति । \newline
83. भ॒व॒ति॒ पञ्च॒ पञ्च॑ भवति भवति॒ पञ्च॑ । \newline
84. पञ्च॒ वै वै पञ्च॒ पञ्च॒ वै । \newline
85. वा ऋ॒तव॑ ऋ॒तवो॒ वै वा ऋ॒तवः॑ । \newline
86. ऋ॒तव॑ ऋ॒तुभ्य॑ ऋ॒तुभ्य॑ ऋ॒तव॑ ऋ॒तव॑ ऋ॒तुभ्यः॑ । \newline
87. ऋ॒तुभ्य॑ ए॒वैव र्तुभ्य॑ ऋ॒तुभ्य॑ ए॒व । \newline
88. ऋ॒तुभ्य॒ इत्यृ॒तु - भ्यः॒ । \newline
89. ए॒वैन॑ मेन मे॒वैवैन᳚म् । \newline
90. ए॒न॒ म॒व॒रुद्ध्या॑ व॒रुद्ध्यै॑न मेन मव॒रुद्ध्य॑ । \newline
91. अ॒व॒रुद्ध्या ऽव॒रुद्ध्या॑व॒रुद्ध्या । \newline
92. अ॒व॒रुद्ध्येत्य॑व - रुद्ध्य॑ । \newline
93. आ ध॑त्ते धत्त॒ आ ध॑त्ते । \newline
94. ध॒त्त॒ इति॑ धत्ते । \newline

\textbf{Ghana Paata } \newline

1. ऋ॒द्ध्नोत्ये॒वैव र्‌द्ध्नो त्यृ॒द्ध्नोत्ये॒व यो य ए॒व र्‌द्ध्नो त्यृ॒द्ध्नोत्ये॒व यः । \newline
2. ए॒व यो य ए॒वैव यो᳚ ऽस्यास्य॒ य ए॒वैव यो᳚ ऽस्य । \newline
3. यो᳚ ऽस्यास्य॒ यो यो᳚ ऽस्यै॒व मे॒व म॑स्य॒ यो यो᳚ ऽस्यै॒वम् । \newline
4. अ॒स्यै॒व मे॒व म॑स्यास्यै॒वम् ब॒न्धुता᳚म् ब॒न्धुता॑ मे॒व म॑स्यास्यै॒वम् ब॒न्धुता᳚म् । \newline
5. ए॒वम् ब॒न्धुता᳚म् ब॒न्धुता॑ मे॒व मे॒वम् ब॒न्धुतां॒ ॅवेद॒ वेद॑ ब॒न्धुता॑ मे॒व मे॒वम् ब॒न्धुतां॒ ॅवेद॑ । \newline
6. ब॒न्धुतां॒ ॅवेद॒ वेद॑ ब॒न्धुता᳚म् ब॒न्धुतां॒ ॅवेद॒ बन्धु॑मा॒न् बन्धु॑मा॒न्॒. वेद॑ ब॒न्धुता᳚म् ब॒न्धुतां॒ ॅवेद॒ बन्धु॑मान् । \newline
7. वेद॒ बन्धु॑मा॒न् बन्धु॑मा॒न्॒. वेद॒ वेद॒ बन्धु॑मान् भवति भवति॒ बन्धु॑मा॒न्॒. वेद॒ वेद॒ बन्धु॑मान् भवति । \newline
8. बन्धु॑मान् भवति भवति॒ बन्धु॑मा॒न् बन्धु॑मान् भवति भाग॒धेय॑म् भाग॒धेय॑म् भवति॒ बन्धु॑मा॒न् बन्धु॑मान् भवति भाग॒धेय᳚म् । \newline
9. बन्धु॑मा॒निति॒ बन्धु॑ - मा॒न् । \newline
10. भ॒व॒ति॒ भा॒ग॒धेय॑म् भाग॒धेय॑म् भवति भवति भाग॒धेयं॒ ॅवै वै भा॑ग॒धेय॑म् भवति भवति भाग॒धेयं॒ ॅवै । \newline
11. भा॒ग॒धेयं॒ ॅवै वै भा॑ग॒धेय॑म् भाग॒धेयं॒ ॅवा अ॒ग्निर॒ग्निर् वै भा॑ग॒धेय॑म् भाग॒धेयं॒ ॅवा अ॒ग्निः । \newline
12. भा॒ग॒धेय॒मिति॑ भाग - धेय᳚म् । \newline
13. वा अ॒ग्निर॒ग्निर् वै वा अ॒ग्निराहि॑त॒ आहि॑तो॒ ऽग्निर् वै वा अ॒ग्निराहि॑तः । \newline
14. अ॒ग्निराहि॑त॒ आहि॑तो॒ ऽग्निर॒ग्निराहि॑त इ॒च्छमा॑न इ॒च्छमा॑न॒ आहि॑तो॒ ऽग्निर॒ग्निराहि॑त इ॒च्छमा॑नः । \newline
15. आहि॑त इ॒च्छमा॑न इ॒च्छमा॑न॒ आहि॑त॒ आहि॑त इ॒च्छमा॑नः प्र॒जाम् प्र॒जा मि॒च्छमा॑न॒ आहि॑त॒ आहि॑त इ॒च्छमा॑नः प्र॒जाम् । \newline
16. आहि॑त॒ इत्या - हि॒तः॒ । \newline
17. इ॒च्छमा॑नः प्र॒जाम् प्र॒जा मि॒च्छमा॑न इ॒च्छमा॑नः प्र॒जाम् प॒शून् प॒शून् प्र॒जा मि॒च्छमा॑न इ॒च्छमा॑नः प्र॒जाम् प॒शून् । \newline
18. प्र॒जाम् प॒शून् प॒शून् प्र॒जाम् प्र॒जाम् प॒शून्. यज॑मानस्य॒ यज॑मानस्य प॒शून् प्र॒जाम् प्र॒जाम् प॒शून्. यज॑मानस्य । \newline
19. प्र॒जामिति॑ प्र - जाम् । \newline
20. प॒शून्. यज॑मानस्य॒ यज॑मानस्य प॒शून् प॒शून्. यज॑मान॒स्योपोप॒ यज॑मानस्य प॒शून् प॒शून्. यज॑मान॒स्योप॑ । \newline
21. यज॑मान॒स्योपोप॒ यज॑मानस्य॒ यज॑मान॒स्योप॑ दोद्राव दोद्रा॒वोप॒ यज॑मानस्य॒ यज॑मान॒स्योप॑ दोद्राव । \newline
22. उप॑ दोद्राव दोद्रा॒वोपोप॑ दोद्रावो॒द्वास्यो॒द्वास्य॑ दोद्रा॒वोपोप॑ दोद्रावो॒द्वास्य॑ । \newline
23. दो॒द्रा॒वो॒द्वास्यो॒द्वास्य॑ दोद्राव दोद्रावो॒द्वास्य॒ पुनः॒ पुन॑रु॒द्वास्य॑ दोद्राव दोद्रावो॒द्वास्य॒ पुनः॑ । \newline
24. उ॒द्वास्य॒ पुनः॒ पुन॑रु॒द्वास्यो॒द्वास्य॒ पुन॒रा पुन॑रु॒द्वास्यो॒द्वास्य॒ पुन॒रा । \newline
25. उ॒द्वास्येत्यु॑त् - वास्य॑ । \newline
26. पुन॒रा पुनः॒ पुन॒रा द॑धीत दधी॒ता पुनः॒ पुन॒रा द॑धीत । \newline
27. आ द॑धीत दधी॒ता द॑धीत भाग॒धेये॑न भाग॒धेये॑न दधी॒ता द॑धीत भाग॒धेये॑न । \newline
28. द॒धी॒त॒ भा॒ग॒धेये॑न भाग॒धेये॑न दधीत दधीत भाग॒धेये॑नै॒वैव भा॑ग॒धेये॑न दधीत दधीत भाग॒धेये॑नै॒व । \newline
29. भा॒ग॒धेये॑नै॒वैव भा॑ग॒धेये॑न भाग॒धेये॑नै॒वैन॑ मेन मे॒व भा॑ग॒धेये॑न भाग॒धेये॑नै॒वैन᳚म् । \newline
30. भा॒ग॒धेये॒नेति॑ भाग - धेये॑न । \newline
31. ए॒वैन॑ मेन मे॒वैवैन॒(ग्म्॒) सꣳ समे॑न मे॒वैवैन॒(ग्म्॒) सम् । \newline
32. ए॒न॒(ग्म्॒) सꣳ समे॑न मेन॒(ग्म्॒) स म॑र्द्धयत्यर्द्धयति॒ समे॑न मेन॒(ग्म्॒) स म॑र्द्धयति । \newline
33. स म॑र्द्धयत्यर्द्धयति॒ सꣳ स म॑र्द्धय॒त्यथो॒ अथो॑ अर्द्धयति॒ सꣳ स म॑र्द्धय॒त्यथो᳚ । \newline
34. अ॒र्द्ध॒य॒त्यथो॒ अथो॑ अर्द्धयत्यर्द्धय॒त्यथो॒ शान्तिः॒ शान्ति॒रथो॑ अर्द्धयत्यर्द्धय॒त्यथो॒ शान्तिः॑ । \newline
35. अथो॒ शान्तिः॒ शान्ति॒रथो॒ अथो॒ शान्ति॑रे॒वैव शान्ति॒रथो॒ अथो॒ शान्ति॑रे॒व । \newline
36. अथो॒ इत्यथो᳚ । \newline
37. शान्ति॑रे॒वैव शान्तिः॒ शान्ति॑ रे॒वास्या᳚स्यै॒व शान्तिः॒ शान्ति॑रे॒वास्य॑ । \newline
38. ए॒वास्या᳚ स्यै॒वैवास्यै॒षैषा ऽस्यै॒वैवास्यै॒षा । \newline
39. अ॒स्यै॒षैषा ऽस्या᳚स्यै॒षा पुन॑र्वस्वोः॒ पुन॑र्वस्वोरे॒षा ऽस्या᳚स्यै॒षा पुन॑र्वस्वोः । \newline
40. ए॒षा पुन॑र्वस्वोः॒ पुन॑र्वस्वोरे॒षैषा पुन॑र्वस्वो॒रा पुन॑र्वस्वोरे॒षैषा पुन॑र्वस्वो॒रा । \newline
41. पुन॑र्वस्वो॒रा पुन॑र्वस्वोः॒ पुन॑र्वस्वो॒रा द॑धीत दधी॒ता पुन॑र्वस्वोः॒ पुन॑र्वस्वो॒रा द॑धीत । \newline
42. पुन॑र्वस्वो॒रिति॒ पुनः॑ - व॒स्वोः॒ । \newline
43. आ द॑धीत दधी॒ता द॑धीतै॒तदे॒तद् द॑धी॒ता द॑धीतै॒तत् । \newline
44. द॒धी॒तै॒तदे॒तद् द॑धीत दधीतै॒तद् वै वा ए॒तद् द॑धीत दधीतै॒तद् वै । \newline
45. ए॒तद् वै वा ए॒तदे॒तद् वै पु॑नरा॒धेय॑स्य पुनरा॒धेय॑स्य॒ वा ए॒तदे॒तद् वै पु॑नरा॒धेय॑स्य । \newline
46. वै पु॑नरा॒धेय॑स्य पुनरा॒धेय॑स्य॒ वै वै पु॑नरा॒धेय॑स्य॒ नक्ष॑त्र॒न्नक्ष॑त्रम् पुनरा॒धेय॑स्य॒ वै वै पु॑नरा॒धेय॑स्य॒ नक्ष॑त्रम् । \newline
47. पु॒न॒रा॒धेय॑स्य॒ नक्ष॑त्र॒न्नक्ष॑त्रम् पुनरा॒धेय॑स्य पुनरा॒धेय॑स्य॒ नक्ष॑त्रं॒ ॅयद् यन् नक्ष॑त्रम् पुनरा॒धेय॑स्य पुनरा॒धेय॑स्य॒ नक्ष॑त्रं॒ ॅयत् । \newline
48. पु॒न॒रा॒धेय॒स्येति॑ पुनः - आ॒धेय॑स्य । \newline
49. नक्ष॑त्रं॒ ॅयद् यन् नक्ष॑त्र॒न्नक्ष॑त्रं॒ ॅयत् पुन॑र्वसू॒ पुन॑र्वसू॒ यन् नक्ष॑त्र॒न्नक्ष॑त्रं॒ ॅयत् पुन॑र्वसू । \newline
50. यत् पुन॑र्वसू॒ पुन॑र्वसू॒ यद् यत् पुन॑र्वसू॒ स्वाया॒(ग्ग्॒) स्वाया॒म् पुन॑र्वसू॒ यद् यत् पुन॑र्वसू॒ स्वाया᳚म् । \newline
51. पुन॑र्वसू॒ स्वाया॒(ग्ग्॒) स्वाया॒म् पुन॑र्वसू॒ पुन॑र्वसू॒ स्वाया॑ मे॒वैव स्वाया॒म् पुन॑र्वसू॒ पुन॑र्वसू॒ स्वाया॑ मे॒व । \newline
52. पुन॑र्वसू॒ इति॒ पुनः॑ - व॒सू॒ । \newline
53. स्वाया॑ मे॒वैव स्वाया॒(ग्ग्॒) स्वाया॑ मे॒वैन॑ मेन मे॒व स्वाया॒(ग्ग्॒) स्वाया॑ मे॒वैन᳚म् । \newline
54. ए॒वैन॑ मेन मे॒वैवैन॑म् दे॒वता॑याम् दे॒वता॑या मेन मे॒वैवैन॑म् दे॒वता॑याम् । \newline
55. ए॒न॒म् दे॒वता॑याम् दे॒वता॑या मेन मेनम् दे॒वता॑या मा॒धाया॒धाय॑ दे॒वता॑या मेन मेनम् दे॒वता॑या मा॒धाय॑ । \newline
56. दे॒वता॑या मा॒धाया॒धाय॑ दे॒वता॑याम् दे॒वता॑या मा॒धाय॑ ब्रह्मवर्च॒सी ब्र॑ह्मवर्च॒स्या॑धाय॑ दे॒वता॑याम् दे॒वता॑या मा॒धाय॑ ब्रह्मवर्च॒सी । \newline
57. आ॒धाय॑ ब्रह्मवर्च॒सी ब्र॑ह्मवर्च॒स्या॑धाया॒धाय॑ ब्रह्मवर्च॒सी भ॑वति भवति ब्रह्मवर्च॒स्या॑धाया॒धाय॑ ब्रह्मवर्च॒सी भ॑वति । \newline
58. आ॒धायेत्या᳚ - धाय॑ । \newline
59. ब्र॒ह्म॒व॒र्च॒सी भ॑वति भवति ब्रह्मवर्च॒सी ब्र॑ह्मवर्च॒सी भ॑वति द॒र्भैर् द॒र्भैर् भ॑वति ब्रह्मवर्च॒सी ब्र॑ह्मवर्च॒सी भ॑वति द॒र्भैः । \newline
60. ब्र॒ह्म॒व॒र्च॒सीति॑ ब्रह्म - व॒र्च॒सी । \newline
61. भ॒व॒ति॒ द॒र्भैर् द॒र्भैर् भ॑वति भवति द॒र्भैरा द॒र्भैर् भ॑वति भवति द॒र्भैरा । \newline
62. द॒र्भैरा द॒र्भैर् द॒र्भैरा द॑धाति दधा॒त्या द॒र्भैर् द॒र्भैरा द॑धाति । \newline
63. आ द॑धाति दधा॒त्या द॑धा॒त्यया॑तया मत्वा॒याया॑तयामत्वाय दधा॒त्या द॑धा॒त्यया॑तयामत्वाय । \newline
64. द॒धा॒त्यया॑तया मत्वा॒याया॑तयामत्वाय दधाति दधा॒त्यया॑तयामत्वाय द॒र्भैर् द॒र्भैरया॑तयामत्वाय दधाति दधा॒त्यया॑तयामत्वाय द॒र्भैः । \newline
65. अया॑तयामत्वाय द॒र्भैर् द॒र्भै रया॑तयामत्वा॒या या॑तयामत्वाय द॒र्भैरा द॒र्भैरया॑तयामत्वा॒ याया॑तयामत्वाय द॒र्भैरा । \newline
66. अया॑तयामत्वा॒येत्यया॑तयाम - त्वा॒य॒ । \newline
67. द॒र्भैरा द॒र्भैर् द॒र्भैरा द॑धाति दधा॒त्या द॒र्भैर् द॒र्भैरा द॑धाति । \newline
68. आ द॑धाति दधा॒त्या द॑धात्य॒द्भ्यो᳚ ऽद्भ्यो द॑धा॒त्या द॑धात्य॒द्भ्यः । \newline
69. द॒धा॒त्य॒द्भ्यो᳚ ऽद्भ्यो द॑धाति दधात्य॒द्भ्य ए॒वैवाद्भ्यो द॑धाति दधात्य॒द्भ्य ए॒व । \newline
70. अ॒द्भ्य ए॒वैवाद्भ्यो᳚ ऽद्भ्य ए॒वैन॑ मेन मे॒वाद्भ्यो᳚ ऽद्भ्य ए॒वैन᳚म् । \newline
71. अ॒द्भ्य इत्य॑त् - भ्यः । \newline
72. ए॒वैन॑ मेन मे॒वैवैन॒ मोष॑धीभ्य॒ ओष॑धीभ्य एन मे॒वैवैन॒ मोष॑धीभ्यः । \newline
73. ए॒न॒ मोष॑धीभ्य॒ ओष॑धीभ्य एन मेन॒ मोष॑धीभ्यो ऽव॒रुद्ध्या॑व॒ रुद्ध्यौष॑धीभ्य एन मेन॒ मोष॑धीभ्यो ऽव॒रुद्ध्य॑ । \newline
74. ओष॑धीभ्यो ऽव॒रुद्ध्या॑व॒ रुद्ध्यौष॑धीभ्य॒ ओष॑धीभ्यो ऽव॒रुद्ध्या ऽव॒रुद्ध्यौष॑धीभ्य॒ ओष॑धीभ्यो ऽव॒रुद्ध्या । \newline
75. ओष॑धीभ्य॒ इत्योष॑धि - भ्यः॒ । \newline
76. अ॒व॒रुद्ध्या ऽव॒रुद्ध्या॑व॒रुद्ध्या ध॑त्ते धत्त॒ आ ऽव॒रुद्ध्या॑व॒रुद्ध्या ध॑त्ते । \newline
77. अ॒व॒रुद्ध्येत्य॑व - रुद्ध्य॑ । \newline
78. आ ध॑त्ते धत्त॒ आ ध॑त्ते॒ पञ्च॑कपालः॒ पञ्च॑कपालो धत्त॒ आ ध॑त्ते॒ पञ्च॑कपालः । \newline
79. ध॒त्ते॒ पञ्च॑कपालः॒ पञ्च॑कपालो धत्ते धत्ते॒ पञ्च॑कपालः पुरो॒डाशः॑ पुरो॒डाशः॒ पञ्च॑कपालो धत्ते धत्ते॒ पञ्च॑कपालः पुरो॒डाशः॑ । \newline
80. पञ्च॑कपालः पुरो॒डाशः॑ पुरो॒डाशः॒ पञ्च॑कपालः॒ पञ्च॑कपालः पुरो॒डाशो॑ भवति भवति पुरो॒डाशः॒ पञ्च॑कपालः॒ पञ्च॑कपालः पुरो॒डाशो॑ भवति । \newline
81. पञ्च॑कपाल॒ इति॒ पञ्च॑ - क॒पा॒लः॒ । \newline
82. पु॒रो॒डाशो॑ भवति भवति पुरो॒डाशः॑ पुरो॒डाशो॑ भवति॒ पञ्च॒ पञ्च॑ भवति पुरो॒डाशः॑ पुरो॒डाशो॑ भवति॒ पञ्च॑ । \newline
83. भ॒व॒ति॒ पञ्च॒ पञ्च॑ भवति भवति॒ पञ्च॒ वै वै पञ्च॑ भवति भवति॒ पञ्च॒ वै । \newline
84. पञ्च॒ वै वै पञ्च॒ पञ्च॒ वा ऋ॒तव॑ ऋ॒तवो॒ वै पञ्च॒ पञ्च॒ वा ऋ॒तवः॑ । \newline
85. वा ऋ॒तव॑ ऋ॒तवो॒ वै वा ऋ॒तव॑ ऋ॒तुभ्य॑ ऋ॒तुभ्य॑ ऋ॒तवो॒ वै वा ऋ॒तव॑ ऋ॒तुभ्यः॑ । \newline
86. ऋ॒तव॑ ऋ॒तुभ्य॑ ऋ॒तुभ्य॑ ऋ॒तव॑ ऋ॒तव॑ ऋ॒तुभ्य॑ ए॒वैव र्‌तुभ्य॑ ऋ॒तव॑ ऋ॒तव॑ ऋ॒तुभ्य॑ ए॒व । \newline
87. ऋ॒तुभ्य॑ ए॒वैव र्‌तुभ्य॑ ऋ॒तुभ्य॑ ए॒वैन॑ मेन मे॒व र्‌तुभ्य॑ ऋ॒तुभ्य॑ ए॒वैन᳚म् । \newline
88. ऋ॒तुभ्य॒ इत्यृ॒तु - भ्यः॒ । \newline
89. ए॒वैन॑ मेन मे॒वैवैन॑ मव॒रुद्ध्या॑ व॒रुद्ध्यै॑न मे॒वैवैन॑ मव॒रुद्ध्य॑ । \newline
90. ए॒न॒ म॒व॒रुद्ध्या॑ व॒रुद्ध्यै॑न मेन मव॒रुद्ध्या ऽव॒रुद्ध्यै॑न मेन मव॒रुद्ध्या । \newline
91. अ॒व॒रुद्ध्या ऽव॒रुद्ध्या॑व॒रुद्ध्या ध॑त्ते धत्त॒ आ ऽव॒रुद्ध्या॑व॒रुद्ध्या ध॑त्ते । \newline
92. अ॒व॒रुद्ध्येत्य॑व - रुद्ध्य॑ । \newline
93. आ ध॑त्ते धत्त॒ आ ध॑त्ते । \newline
94. ध॒त्त॒ इति॑ धत्ते । \newline
\pagebreak
\markright{ TS 1.5.2.1  \hfill https://www.vedavms.in \hfill}
\addcontentsline{toc}{section}{ TS 1.5.2.1 }
\section*{ TS 1.5.2.1 }

\textbf{TS 1.5.2.1 } \newline
\textbf{Samhita Paata} \newline

परा॒ वा ए॒ष य॒ज्ञ्ं प॒शून् व॑पति॒ यो᳚ऽग्निमु॑द्वा॒सय॑ते॒ पञ्च॑कपालः पुरो॒डाशो॑ भवति॒ पाङ्क्तो॑ य॒ज्ञ्ः पाङ्क्ताः᳚ प॒शवो॑ य॒ज्ञ्मे॒व प॒शूनव॑ रुन्धे वीर॒हा वा ए॒ष दे॒वानां॒ ॅयो᳚ऽग्निमु॑द्वा॒सय॑ते॒ न वा ए॒तस्य॑ ब्राह्म॒णा ऋ॑ता॒यवः॑ पु॒राऽन्न॑मक्षन् प॒ङ्क्त्यो॑ याज्यानुवा॒क्या॑ भवन्ति॒ पाङ्क्तो॑ य॒ज्ञ्ः पाङ्क्तः॒ पुरु॑षो दे॒वाने॒व वी॒रं नि॑रव॒दाया॒ग्निं पुन॒रा - [ ] \newline

\textbf{Pada Paata} \newline

परेति॑ । वै । ए॒षः । य॒ज्ञ्म् । प॒शून् । व॒प॒ति॒ । यः । अ॒ग्निम् । उ॒द्वा॒सय॑त॒ इत्यु॑त् - वा॒सय॑ते । पञ्च॑कपाल॒ इति॒ पञ्च॑ - क॒पा॒लः॒ । पु॒रो॒डाशः॑ । भ॒व॒ति॒ । पाङ्क्तः॑ । य॒ज्ञ्ः । पाङ्क्ताः᳚ । प॒शवः॑ । य॒ज्ञ्म् । ए॒व । प॒शून् । अवेति॑ । रु॒न्धे॒ । वी॒र॒हेति॑ वीर - हा । वै । ए॒षः । दे॒वाना᳚म् । यः । अ॒ग्निम् । उ॒द्वा॒सय॑त॒ इत्यु॑त् - वा॒सय॑ते । न । वै । ए॒तस्य॑ । ब्रा॒ह्म॒णाः । ऋ॒ता॒यव॒ इत्यृ॑त-यवः॑ । पु॒रा । अन्न᳚म् । अ॒क्ष॒न्न् । प॒ङ्क्त्यः॑ । या॒ज्या॒नु॒वा॒क्या॑ इति॑ याज्या - अ॒नु॒वा॒क्याः᳚ । भ॒व॒न्ति॒ । पाङ्क्तः॑ । य॒ज्ञ्ः । पाङ्क्तः॑ । पुरु॑षः । दे॒वान् । ए॒व । वी॒रम् । नि॒र॒व॒दायेति॑ निः - अ॒व॒दाय॑ । अ॒ग्निम् । पुनः॑ । एति॑ ।  \newline


\textbf{Krama Paata} \newline

परा॒ वै । वा ए॒षः । ए॒ष य॒ज्ञ्म् । य॒ज्ञ्म् प॒शून् । प॒शून्. व॑पति । व॒प॒ति॒ यः । यो᳚ऽग्निम् । अ॒ग्निमु॑द्वा॒सय॑ते । उ॒द्वा॒सय॑ते॒ पञ्च॑कपालः । उ॒द्वा॒सय॑त॒ इत्यु॑त् - वा॒सय॑ते । पञ्च॑कपालः पुरो॒डाशः॑ । पञ्च॑कपाल॒ इति॒ पञ्च॑ - क॒पा॒लः॒ । पु॒रो॒डाशो॑ भवति । भ॒व॒ति॒ पाङ्क्तः॑ । पाङ्क्तो॑ य॒ज्ञ्ः । य॒ज्ञ्ः पाङ्क्ताः᳚ । पाङ्क्ताः᳚ प॒शवः॑ । प॒शवो॑ य॒ज्ञ्म् । य॒ज्ञ्मे॒व । ए॒व प॒शून् । प॒शूनव॑ । अव॑ रुन्धे । रु॒न्धे॒ वी॒र॒हा । वी॒र॒हा वै । वी॒र॒हेति॑ वीर - हा । वा ए॒षः । ए॒ष दे॒वाना᳚म् । दे॒वानां॒ ॅयः । यो᳚ऽग्निम् । अ॒ग्निमु॑द्वा॒सय॑ते । उ॒द्वा॒सय॑ते॒ न । उ॒द्वा॒सय॑त॒ इत्यु॑त् - वा॒सय॑ते । न वै । वा ए॒तस्य॑ । ए॒तस्य॑ ब्राह्म॒णाः । ब्रा॒ह्म॒णा ऋ॑ता॒यवः॑ । ऋ॒ता॒यवः॑ पु॒रा । ऋ॒ता॒यव॒ इत्यृ॑त - यवः॑ । पु॒रा ऽन्न᳚म् । अन्न॑मक्षन्न् । अ॒क्ष॒न् प॒ङ्क्त्यः॑ । प॒ङ्क्त्यो॑ याज्यानुवा॒क्याः᳚ । या॒ज्या॒नु॒वा॒क्या॑ भवन्ति । या॒ज्या॒नु॒वा॒क्या॑ इति॑ याज्या - अ॒नु॒वा॒क्याः᳚ । भ॒व॒न्ति॒ पाङ्क्तः॑ । पाङ्क्तो॑ य॒ज्ञ्ः । य॒ज्ञ्ः पाङ्क्तः॑ । पाङ्क्तः॒ पुरु॑षः । पुरु॑षो दे॒वान् । दे॒वाने॒व । ए॒व वी॒रम् । वी॒रम् नि॑रव॒दाय॑ । नि॒र॒व॒दाया॒ग्निम् । नि॒र॒व॒दायेति॑ निः - अ॒व॒दाय॑ । अ॒ग्निम् पुनः॑ । पुन॒रा । आ ध॑त्ते \newline

\textbf{Jatai Paata} \newline

1. परा॒ वै वै परा॒ परा॒ वै । \newline
2. वा ए॒ष ए॒ष वै वा ए॒षः । \newline
3. ए॒ष य॒ज्ञ्ं ॅय॒ज्ञ् मे॒ष ए॒ष य॒ज्ञ्म् । \newline
4. य॒ज्ञ्म् प॒शून् प॒शून्. य॒ज्ञ्ं ॅय॒ज्ञ्म् प॒शून् । \newline
5. प॒शून्. व॑पति वपति प॒शून् प॒शून्. व॑पति । \newline
6. व॒प॒ति॒ यो यो व॑पति वपति॒ यः । \newline
7. यो᳚ ऽग्नि म॒ग्निं ॅयो यो᳚ ऽग्निम् । \newline
8. अ॒ग्नि मु॑द्वा॒सय॑त उद्वा॒सय॑ते॒ ऽग्नि म॒ग्नि मु॑द्वा॒सय॑ते । \newline
9. उ॒द्वा॒सय॑ते॒ पञ्च॑कपालः॒ पञ्च॑कपाल उद्वा॒सय॑त उद्वा॒सय॑ते॒ पञ्च॑कपालः । \newline
10. उ॒द्वा॒सय॑त॒ इत्यु॑त् - वा॒सय॑ते । \newline
11. पञ्च॑कपालः पुरो॒डाशः॑ पुरो॒डाशः॒ पञ्च॑कपालः॒ पञ्च॑कपालः पुरो॒डाशः॑ । \newline
12. पञ्च॑कपाल॒ इति॒ पञ्च॑ - क॒पा॒लः॒ । \newline
13. पु॒रो॒डाशो॑ भवति भवति पुरो॒डाशः॑ पुरो॒डाशो॑ भवति । \newline
14. भ॒व॒ति॒ पाङ्क्तः॒ पाङ्क्तो॑ भवति भवति॒ पाङ्क्तः॑ । \newline
15. पाङ्क्तो॑ य॒ज्ञो य॒ज्ञ्ः पाङ्क्तः॒ पाङ्क्तो॑ य॒ज्ञ्ः । \newline
16. य॒ज्ञ्ः पाङ्क्ताः॒ पाङ्क्ता॑ य॒ज्ञो य॒ज्ञ्ः पाङ्क्ताः᳚ । \newline
17. पाङ्क्ताः᳚ प॒शवः॑ प॒शवः॒ पाङ्क्ताः॒ पाङ्क्ताः᳚ प॒शवः॑ । \newline
18. प॒शवो॑ य॒ज्ञ्ं ॅय॒ज्ञ्म् प॒शवः॑ प॒शवो॑ य॒ज्ञ्म् । \newline
19. य॒ज्ञ् मे॒वैव य॒ज्ञ्ं ॅय॒ज्ञ् मे॒व । \newline
20. ए॒व प॒शून् प॒शू ने॒वैव प॒शून् । \newline
21. प॒शू नवाव॑ प॒शून् प॒शू नव॑ । \newline
22. अव॑ रुन्धे रु॒न्धे ऽवाव॑ रुन्धे । \newline
23. रु॒न्धे॒ वी॒र॒हा वी॑र॒हा रु॑न्धे रुन्धे वीर॒हा । \newline
24. वी॒र॒हा वै वै वी॑र॒हा वी॑र॒हा वै । \newline
25. वी॒र॒हेति॑ वीर - हा । \newline
26. वा ए॒ष ए॒ष वै वा ए॒षः । \newline
27. ए॒ष दे॒वाना᳚म् दे॒वाना॑ मे॒ष ए॒ष दे॒वाना᳚म् । \newline
28. दे॒वानां॒ ॅयो यो दे॒वाना᳚म् दे॒वानां॒ ॅयः । \newline
29. यो᳚ ऽग्नि म॒ग्निं ॅयो यो᳚ ऽग्निम् । \newline
30. अ॒ग्नि मु॑द्वा॒सय॑त उद्वा॒सय॑ते॒ ऽग्नि म॒ग्नि मु॑द्वा॒सय॑ते । \newline
31. उ॒द्वा॒सय॑ते॒ न नोद्वा॒सय॑त उद्वा॒सय॑ते॒ न । \newline
32. उ॒द्वा॒सय॑त॒ इत्यु॑त् - वा॒सय॑ते । \newline
33. न वै वै न न वै । \newline
34. वा ए॒तस्यै॒तस्य॒ वै वा ए॒तस्य॑ । \newline
35. ए॒तस्य॑ ब्राह्म॒णा ब्रा᳚ह्म॒णा ए॒तस्यै॒तस्य॑ ब्राह्म॒णाः । \newline
36. ब्रा॒ह्म॒णा ऋ॑ता॒यव॑ ऋता॒यवो᳚ ब्राह्म॒णा ब्रा᳚ह्म॒णा ऋ॑ता॒यवः॑ । \newline
37. ऋ॒ता॒यवः॑ पु॒रा पु॒रर्ता॒यव॑ ऋता॒यवः॑ पु॒रा । \newline
38. ऋ॒ता॒यव॒ इत्यृ॑त - यवः॑ । \newline
39. पु॒रा ऽन्न॒ मन्न॑म् पु॒रा पु॒रा ऽन्न᳚म् । \newline
40. अन्न॑ मक्षन्नक्ष॒न्नन्न॒ मन्न॑ मक्षन्न् । \newline
41. अ॒क्ष॒न् प॒ङ्क्त्यः॑ प॒ङ्क्त्यो᳚ ऽक्षन्नक्षन् प॒ङ्क्त्यः॑ । \newline
42. प॒ङ्क्त्यो॑ याज्यानुवा॒क्या॑ याज्यानुवा॒क्याः᳚ प॒ङ्क्त्यः॑ प॒ङ्क्त्यो॑ याज्यानुवा॒क्याः᳚ । \newline
43. या॒ज्या॒नु॒वा॒क्या॑ भवन्ति भवन्ति याज्यानुवा॒क्या॑ याज्यानुवा॒क्या॑ भवन्ति । \newline
44. या॒ज्या॒नु॒वा॒क्या॑ इति॑ याज्या - अ॒नु॒वा॒क्याः᳚ । \newline
45. भ॒व॒न्ति॒ पाङ्क्तः॒ पाङ्क्तो॑ भवन्ति भवन्ति॒ पाङ्क्तः॑ । \newline
46. पाङ्क्तो॑ य॒ज्ञो य॒ज्ञ्ः पाङ्क्तः॒ पाङ्क्तो॑ य॒ज्ञ्ः । \newline
47. य॒ज्ञ्ः पाङ्क्तः॒ पाङ्क्तो॑ य॒ज्ञो य॒ज्ञ्ः पाङ्क्तः॑ । \newline
48. पाङ्क्तः॒ पुरु॑षः॒ पुरु॑षः॒ पाङ्क्तः॒ पाङ्क्तः॒ पुरु॑षः । \newline
49. पुरु॑षो दे॒वान् दे॒वान् पुरु॑षः॒ पुरु॑षो दे॒वान् । \newline
50. दे॒वा ने॒वैव दे॒वान् दे॒वा ने॒व । \newline
51. ए॒व वी॒रं ॅवी॒र मे॒वैव वी॒रम् । \newline
52. वी॒रन्नि॑रव॒दाय॑ निरव॒दाय॑ वी॒रं ॅवी॒रन्नि॑रव॒दाय॑ । \newline
53. नि॒र॒व॒दाया॒ग्नि म॒ग्निन्नि॑रव॒दाय॑ निरव॒दाया॒ग्निम् । \newline
54. नि॒र॒व॒दायेति॑ निः - अ॒व॒दाय॑ । \newline
55. अ॒ग्निम् पुनः॒ पुन॑ र॒ग्नि म॒ग्निम् पुनः॑ । \newline
56. पुन॒ रा पुनः॒ पुन॒ रा । \newline
57. आ ध॒त्ते ध॒त्त आ ध॒त्ते । \newline

\textbf{Ghana Paata } \newline

1. परा॒ वै वै परा॒ परा॒ वा ए॒ष ए॒ष वै परा॒ परा॒ वा ए॒षः । \newline
2. वा ए॒ष ए॒ष वै वा ए॒ष य॒ज्ञ्ं ॅय॒ज्ञ् मे॒ष वै वा ए॒ष य॒ज्ञ्म् । \newline
3. ए॒ष य॒ज्ञ्ं ॅय॒ज्ञ् मे॒ष ए॒ष य॒ज्ञ्म् प॒शून् प॒शून्. य॒ज्ञ् मे॒ष ए॒ष य॒ज्ञ्म् प॒शून् । \newline
4. य॒ज्ञ्म् प॒शून् प॒शून्. य॒ज्ञ्ं ॅय॒ज्ञ्म् प॒शून्. व॑पति वपति प॒शून्. य॒ज्ञ्ं ॅय॒ज्ञ्म् प॒शून्. व॑पति । \newline
5. प॒शून्. व॑पति वपति प॒शून् प॒शून्. व॑पति॒ यो यो व॑पति प॒शून् प॒शून्. व॑पति॒ यः । \newline
6. व॒प॒ति॒ यो यो व॑पति वपति॒ यो᳚ ऽग्नि म॒ग्निं ॅयो व॑पति वपति॒ यो᳚ ऽग्निम् । \newline
7. यो᳚ ऽग्नि म॒ग्निं ॅयो यो᳚ ऽग्नि मु॑द्वा॒सय॑त उद्वा॒सय॑ते॒ ऽग्निं ॅयो यो᳚ ऽग्नि मु॑द्वा॒सय॑ते । \newline
8. अ॒ग्नि मु॑द्वा॒सय॑त उद्वा॒सय॑ते॒ ऽग्नि म॒ग्नि मु॑द्वा॒सय॑ते॒ पञ्च॑कपालः॒ पञ्च॑कपाल उद्वा॒सय॑ते॒ ऽग्नि म॒ग्नि मु॑द्वा॒सय॑ते॒ पञ्च॑कपालः । \newline
9. उ॒द्वा॒सय॑ते॒ पञ्च॑कपालः॒ पञ्च॑कपाल उद्वा॒सय॑त उद्वा॒सय॑ते॒ पञ्च॑कपालः पुरो॒डाशः॑ पुरो॒डाशः॒ पञ्च॑कपाल उद्वा॒सय॑त उद्वा॒सय॑ते॒ पञ्च॑कपालः पुरो॒डाशः॑ । \newline
10. उ॒द्वा॒सय॑त॒ इत्यु॑त् - वा॒सय॑ते । \newline
11. पञ्च॑कपालः पुरो॒डाशः॑ पुरो॒डाशः॒ पञ्च॑कपालः॒ पञ्च॑कपालः पुरो॒डाशो॑ भवति भवति पुरो॒डाशः॒ पञ्च॑कपालः॒ पञ्च॑कपालः पुरो॒डाशो॑ भवति । \newline
12. पञ्च॑कपाल॒ इति॒ पञ्च॑ - क॒पा॒लः॒ । \newline
13. पु॒रो॒डाशो॑ भवति भवति पुरो॒डाशः॑ पुरो॒डाशो॑ भवति॒ पाङ्क्तः॒ पाङ्क्तो॑ भवति पुरो॒डाशः॑ पुरो॒डाशो॑ भवति॒ पाङ्क्तः॑ । \newline
14. भ॒व॒ति॒ पाङ्क्तः॒ पाङ्क्तो॑ भवति भवति॒ पाङ्क्तो॑ य॒ज्ञो य॒ज्ञ्ः पाङ्क्तो॑ भवति भवति॒ पाङ्क्तो॑ य॒ज्ञ्ः । \newline
15. पाङ्क्तो॑ य॒ज्ञो य॒ज्ञ्ः पाङ्क्तः॒ पाङ्क्तो॑ य॒ज्ञ्ः पाङ्क्ताः॒ पाङ्क्ता॑ य॒ज्ञ्ः पाङ्क्तः॒ पाङ्क्तो॑ य॒ज्ञ्ः पाङ्क्ताः᳚ । \newline
16. य॒ज्ञ्ः पाङ्क्ताः॒ पाङ्क्ता॑ य॒ज्ञो य॒ज्ञ्ः पाङ्क्ताः᳚ प॒शवः॑ प॒शवः॒ पाङ्क्ता॑ य॒ज्ञो य॒ज्ञ्ः पाङ्क्ताः᳚ प॒शवः॑ । \newline
17. पाङ्क्ताः᳚ प॒शवः॑ प॒शवः॒ पाङ्क्ताः॒ पाङ्क्ताः᳚ प॒शवो॑ य॒ज्ञ्ं ॅय॒ज्ञ्म् प॒शवः॒ पाङ्क्ताः॒ पाङ्क्ताः᳚ प॒शवो॑ य॒ज्ञ्म् । \newline
18. प॒शवो॑ य॒ज्ञ्ं ॅय॒ज्ञ्म् प॒शवः॑ प॒शवो॑ य॒ज्ञ् मे॒वैव य॒ज्ञ्म् प॒शवः॑ प॒शवो॑ य॒ज्ञ् मे॒व । \newline
19. य॒ज्ञ् मे॒वैव य॒ज्ञ्ं ॅय॒ज्ञ् मे॒व प॒शून् प॒शू ने॒व य॒ज्ञ्ं ॅय॒ज्ञ् मे॒व प॒शून् । \newline
20. ए॒व प॒शून् प॒शू ने॒वैव प॒शू नवाव॑ प॒शू ने॒वैव प॒शू नव॑ । \newline
21. प॒शू नवाव॑ प॒शून् प॒शू नव॑ रुन्धे रु॒न्धे ऽव॑ प॒शून् प॒शू नव॑ रुन्धे । \newline
22. अव॑ रुन्धे रु॒न्धे ऽवाव॑ रुन्धे वीर॒हा वी॑र॒हा रु॒न्धे ऽवाव॑ रुन्धे वीर॒हा । \newline
23. रु॒न्धे॒ वी॒र॒हा वी॑र॒हा रु॑न्धे रुन्धे वीर॒हा वै वै वी॑र॒हा रु॑न्धे रुन्धे वीर॒हा वै । \newline
24. वी॒र॒हा वै वै वी॑र॒हा वी॑र॒हा वा ए॒ष ए॒ष वै वी॑र॒हा वी॑र॒हा वा ए॒षः । \newline
25. वी॒र॒हेति॑ वीर - हा । \newline
26. वा ए॒ष ए॒ष वै वा ए॒ष दे॒वाना᳚म् दे॒वाना॑ मे॒ष वै वा ए॒ष दे॒वाना᳚म् । \newline
27. ए॒ष दे॒वाना᳚म् दे॒वाना॑ मे॒ष ए॒ष दे॒वानां॒ ॅयो यो दे॒वाना॑ मे॒ष ए॒ष दे॒वानां॒ ॅयः । \newline
28. दे॒वानां॒ ॅयो यो दे॒वाना᳚म् दे॒वानां॒ ॅयो᳚ ऽग्नि म॒ग्निं ॅयो दे॒वाना᳚म् दे॒वानां॒ ॅयो᳚ ऽग्निम् । \newline
29. यो᳚ ऽग्नि म॒ग्निं ॅयो यो᳚ ऽग्नि मु॑द्वा॒सय॑त उद्वा॒सय॑ते॒ ऽग्निं ॅयो यो᳚ ऽग्नि मु॑द्वा॒सय॑ते । \newline
30. अ॒ग्नि मु॑द्वा॒सय॑त उद्वा॒सय॑ते॒ ऽग्नि म॒ग्नि मु॑द्वा॒सय॑ते॒ न नोद्वा॒सय॑ते॒ ऽग्नि म॒ग्नि मु॑द्वा॒सय॑ते॒ न । \newline
31. उ॒द्वा॒सय॑ते॒ न नोद्वा॒सय॑त उद्वा॒सय॑ते॒ न वै वै नोद्वा॒सय॑त उद्वा॒सय॑ते॒ न वै । \newline
32. उ॒द्वा॒सय॑त॒ इत्यु॑त् - वा॒सय॑ते । \newline
33. न वै वै न न वा ए॒तस्यै॒तस्य॒ वै न न वा ए॒तस्य॑ । \newline
34. वा ए॒तस्यै॒तस्य॒ वै वा ए॒तस्य॑ ब्राह्म॒णा ब्रा᳚ह्म॒णा ए॒तस्य॒ वै वा ए॒तस्य॑ ब्राह्म॒णाः । \newline
35. ए॒तस्य॑ ब्राह्म॒णा ब्रा᳚ह्म॒णा ए॒तस्यै॒तस्य॑ ब्राह्म॒णा ऋ॑ता॒यव॑ ऋता॒यवो᳚ ब्राह्म॒णा ए॒तस्यै॒तस्य॑ ब्राह्म॒णा ऋ॑ता॒यवः॑ । \newline
36. ब्रा॒ह्म॒णा ऋ॑ता॒यव॑ ऋता॒यवो᳚ ब्राह्म॒णा ब्रा᳚ह्म॒णा ऋ॑ता॒यवः॑ पु॒रा पु॒रर्ता॒यवो᳚ ब्राह्म॒णा ब्रा᳚ह्म॒णा ऋ॑ता॒यवः॑ पु॒रा । \newline
37. ऋ॒ता॒यवः॑ पु॒रा पु॒रर्ता॒यव॑ ऋता॒यवः॑ पु॒रा ऽन्न॒ मन्न॑म् पु॒रर्ता॒यव॑ ऋता॒यवः॑ पु॒रा ऽन्न᳚म् । \newline
38. ऋ॒ता॒यव॒ इत्यृ॑त - यवः॑ । \newline
39. पु॒रा ऽन्न॒ मन्न॑म् पु॒रा पु॒रा ऽन्न॑ मक्षन् नक्ष॒न् नन्न॑म् पु॒रा पु॒रा ऽन्न॑ मक्षन्न् । \newline
40. अन्न॑ मक्षन् नक्ष॒न् नन्न॒ मन्न॑ मक्षन् प॒ङ्क्त्यः॑ प॒ङ्क्त्यो᳚ ऽक्ष॒न् नन्न॒ मन्न॑ मक्षन् प॒ङ्क्त्यः॑ । \newline
41. अ॒क्ष॒न् प॒ङ्क्त्यः॑ प॒ङ्क्त्यो᳚ ऽक्षन् नक्षन् प॒ङ्क्त्यो॑ याज्यानुवा॒क्या॑ याज्यानुवा॒क्याः᳚ प॒ङ्क्त्यो᳚ ऽक्षन् नक्षन् प॒ङ्क्त्यो॑ याज्यानुवा॒क्याः᳚ । \newline
42. प॒ङ्क्त्यो॑ याज्यानुवा॒क्या॑ याज्यानुवा॒क्याः᳚ प॒ङ्क्त्यः॑ प॒ङ्क्त्यो॑ याज्यानुवा॒क्या॑ भवन्ति भवन्ति याज्यानुवा॒क्याः᳚ प॒ङ्क्त्यः॑ प॒ङ्क्त्यो॑ याज्यानुवा॒क्या॑ भवन्ति । \newline
43. या॒ज्या॒नु॒वा॒क्या॑ भवन्ति भवन्ति याज्यानुवा॒क्या॑ याज्यानुवा॒क्या॑ भवन्ति॒ पाङ्क्तः॒ पाङ्क्तो॑ भवन्ति याज्यानुवा॒क्या॑ याज्यानुवा॒क्या॑ भवन्ति॒ पाङ्क्तः॑ । \newline
44. या॒ज्या॒नु॒वा॒क्या॑ इति॑ याज्या - अ॒नु॒वा॒क्याः᳚ । \newline
45. भ॒व॒न्ति॒ पाङ्क्तः॒ पाङ्क्तो॑ भवन्ति भवन्ति॒ पाङ्क्तो॑ य॒ज्ञो य॒ज्ञ्ः पाङ्क्तो॑ भवन्ति भवन्ति॒ पाङ्क्तो॑ य॒ज्ञ्ः । \newline
46. पाङ्क्तो॑ य॒ज्ञो य॒ज्ञ्ः पाङ्क्तः॒ पाङ्क्तो॑ य॒ज्ञ्ः पाङ्क्तः॒ पाङ्क्तो॑ य॒ज्ञ्ः पाङ्क्तः॒ पाङ्क्तो॑ य॒ज्ञ्ः पाङ्क्तः॑ । \newline
47. य॒ज्ञ्ः पाङ्क्तः॒ पाङ्क्तो॑ य॒ज्ञो य॒ज्ञ्ः पाङ्क्तः॒ पुरु॑षः॒ पुरु॑षः॒ पाङ्क्तो॑ य॒ज्ञो य॒ज्ञ्ः पाङ्क्तः॒ पुरु॑षः । \newline
48. पाङ्क्तः॒ पुरु॑षः॒ पुरु॑षः॒ पाङ्क्तः॒ पाङ्क्तः॒ पुरु॑षो दे॒वान् दे॒वान् पुरु॑षः॒ पाङ्क्तः॒ पाङ्क्तः॒ पुरु॑षो दे॒वान् । \newline
49. पुरु॑षो दे॒वान् दे॒वान् पुरु॑षः॒ पुरु॑षो दे॒वा ने॒वैव दे॒वान् पुरु॑षः॒ पुरु॑षो दे॒वा ने॒व । \newline
50. दे॒वा ने॒वैव दे॒वान् दे॒वा ने॒व वी॒रं ॅवी॒र मे॒व दे॒वान् दे॒वा ने॒व वी॒रम् । \newline
51. ए॒व वी॒रं ॅवी॒र मे॒वैव वी॒रन्नि॑रव॒दाय॑ निरव॒दाय॑ वी॒र मे॒वैव वी॒रन्नि॑रव॒दाय॑ । \newline
52. वी॒रन्नि॑रव॒दाय॑ निरव॒दाय॑ वी॒रं ॅवी॒रन्नि॑रव॒दाया॒ग्नि म॒ग्निन्नि॑रव॒दाय॑ वी॒रं ॅवी॒रन्नि॑रव॒दाया॒ग्निम् । \newline
53. नि॒र॒व॒दाया॒ग्नि म॒ग्निन्नि॑रव॒दाय॑ निरव॒दाया॒ग्निम् पुनः॒ पुन॑ र॒ग्निन्नि॑रव॒दाय॑ निरव॒दाया॒ग्निम् पुनः॑ । \newline
54. नि॒र॒व॒दायेति॑ निः - अ॒व॒दाय॑ । \newline
55. अ॒ग्निम् पुनः॒ पुन॑र॒ग्नि म॒ग्निम् पुन॒रा पुन॑र॒ग्नि म॒ग्निम् पुन॒रा । \newline
56. पुन॒रा पुनः॒ पुन॒रा ध॒त्ते ध॒त्त आ पुनः॒ पुन॒रा ध॒त्ते । \newline
57. आ ध॒त्ते ध॒त्त आ ध॒त्ते श॒ताक्ष॑राः श॒ताक्ष॑रा ध॒त्त आ ध॒त्ते श॒ताक्ष॑राः । \newline
\pagebreak
\markright{ TS 1.5.2.2  \hfill https://www.vedavms.in \hfill}
\addcontentsline{toc}{section}{ TS 1.5.2.2 }
\section*{ TS 1.5.2.2 }

\textbf{TS 1.5.2.2 } \newline
\textbf{Samhita Paata} \newline

ध॑त्ते श॒ताक्ष॑रा भवन्ति श॒तायुः॒ पुरु॑षः श॒तेन्द्रि॑य॒ आयु॑ष्ये॒वेन्द्रि॒ये प्रति॑ तिष्ठति॒ यद्वा अ॒ग्निराहि॑तो॒ नर्द्ध्यते॒ ज्यायो॑ भाग॒धेयं॑ निका॒मय॑मानो॒ यदा᳚ग्ने॒यꣳ सर्वं॒ भव॑ति॒ सैवास्यर्द्धिः॒ सं ॅवा ए॒तस्य॑ गृ॒हे वाक् सृ॑ज्यते॒ यो᳚ऽग्निमु॑द्वा॒सय॑ते॒ स वाचꣳ॒॒ सꣳसृ॑ष्टां॒ ॅयज॑मान ईश्व॒रोऽनु॒ परा॑भवितो॒र् विभ॑क्तयो भवन्ति वा॒चो विधृ॑त्यै॒ यज॑मान॒स्याप॑राभावाय॒ - [ ] \newline

\textbf{Pada Paata} \newline

ध॒त्ते॒ । श॒ताक्ष॑रा॒ इति॑ श॒त - अ॒क्ष॒राः॒ । भ॒व॒न्ति॒ । श॒तायु॒रिति॑ श॒त - आ॒युः॒ । पुरु॑षः । श॒तेन्द्रि॑य॒ इति॑ श॒त - इ॒न्द्रि॒यः॒ । आयु॑षि । ए॒व । इ॒न्द्रि॒ये । प्रतीति॑ । ति॒ष्ठ॒ति॒ । यत् । वै । अ॒ग्निः । आहि॑त॒ इत्या - हि॒तः॒ । न । ऋ॒द्ध्यते᳚ । ज्यायः॑ । भा॒ग॒धेय॒मिति॑ भाग - धेय᳚म् । नि॒का॒मय॑मान॒ इति॑ नि - का॒मय॑मानः । यत् । आ॒ग्ने॒यम् । सर्व᳚म् । भव॑ति । सा । ए॒व । अ॒स्य॒ । ऋद्धिः॑ । समिति॑ । वै । ए॒तस्य॑ । गृ॒हे । वाक् । सृ॒ज्य॒ते॒ । यः । अ॒ग्निम् । उ॒द्वा॒सय॑त॒ इत्यु॑त् - वा॒सय॑ते । सः । वाच᳚म् । सꣳसृ॑ष्टा॒मिति॒ सं - सृ॒ष्टा॒म् । यज॑मानः । ई॒श्व॒रः । अन्विति॑ । परा॑भवितो॒रिति॒ परा᳚ - भ॒वि॒तोः॒ । विभ॑क्तय॒ इति॒ वि - भ॒क्त॒यः॒ । भ॒व॒न्ति॒ । वा॒चः । विधृ॑त्या॒ इति॒ वि - धृ॒त्यै॒ । यज॑मानस्य । अप॑राभावा॒येत्यप॑रा - भा॒वा॒य॒ ।  \newline


\textbf{Krama Paata} \newline

ध॒त्ते॒ श॒ताक्ष॑राः । श॒ताक्ष॑रा भवन्ति । श॒ताक्ष॑रा॒ इति॑ श॒त - अ॒क्ष॒राः॒ । भ॒व॒न्ति॒ श॒तायुः॑ । श॒तायुः॒ पुरु॑षः । श॒तायु॒रिति॑ श॒त - आ॒युः॒ । पुरु॑षः श॒तेन्द्रि॑यः । श॒तेन्द्रि॑य॒ आयु॑षि । श॒तेन्द्रि॑य॒ इति॑ श॒त - इ॒न्द्रि॒यः॒ । आयु॑ष्ये॒व । ए॒वेन्द्रि॒ये । इ॒न्द्रि॒ये प्रति॑ । प्रति॑ तिष्ठति । ति॒ष्ठ॒ति॒ यत् । यद् वै । वा अ॒ग्निः । अ॒ग्निराहि॑तः । आहि॑तो॒ न । आहि॑त॒ इत्या - हि॒तः॒ । नर्द्ध्यते᳚ । ऋ॒द्ध्यते॒ ज्यायः॑ । ज्यायो॑ भाग॒धेय᳚म् । भा॒ग॒धेय॑म् निका॒मय॑मानः । भा॒ग॒धेय॒मिति॑ भाग - धेय᳚म् । नि॒का॒मय॑मानो॒ यत् । नि॒का॒मय॑मान॒ इति॑ नि - का॒मय॑मानः । यदा᳚ग्ने॒यम् । आ॒ग्ने॒यꣳ सर्व᳚म् । सर्व॒म् भव॑ति । भव॑ति॒ सा । सैव । ए॒वास्य॑ । अ॒स्यर्द्धिः॑ । ऋद्धिः॒ सम् । सं ॅवै । वा ए॒तस्य॑ । ए॒तस्य॑ गृ॒हे । गृ॒हे वाक् । वाख् सृ॑ज्यते । सृ॒ज्य॒ते॒ यः । यो᳚ऽग्निम् । अ॒ग्निमु॑द्वा॒सय॑ते । उ॒द्वा॒सय॑ते॒ सः । उ॒द्वा॒सय॑त॒ इत्यु॑त् - वा॒सय॑ते । स वाच᳚म् । वाचꣳ॒॒ सꣳसृ॑ष्टाम् । सꣳसृ॑ष्टां॒ ॅयज॑मानः । सꣳसृ॑ष्टा॒मिति॒ सं - सृ॒ष्टा॒म् । यज॑मान ईश्व॒रः । ई॒श्व॒रोऽनु॑ । अनु॒ परा॑भवितोः । परा॑भवितो॒र् विभ॑क्तयः । परा॑भवितो॒रिति॒ परा᳚ - भ॒वि॒तोः॒ । विभ॑क्तयो भवन्ति । विभ॑क्तय॒ इति॒ वि - भ॒क्त॒यः॒ । भ॒व॒न्ति॒ वा॒चः । वा॒चो विधृ॑त्यै । विधृ॑त्यै॒ यज॑मानस्य । विधृ॑त्या॒ इति॒ वि - धृ॒त्यै॒ । यज॑मान॒स्याप॑राभावाय । अप॑राभावाय॒ विभ॑क्तिम् । अप॑राभावा॒येत्यप॑रा - भा॒वा॒य॒ \newline

\textbf{Jatai Paata} \newline

1. ध॒त्ते श॒ताक्ष॑राः श॒ताक्ष॑रा ध॒त्ते ध॒त्ते श॒ताक्ष॑राः । \newline
2. श॒ताक्ष॑रा भवन्ति भवन्ति श॒ताक्ष॑राः श॒ताक्ष॑रा भवन्ति । \newline
3. श॒ताक्ष॑रा॒ इति॑ श॒त - अ॒क्ष॒राः॒ । \newline
4. भ॒व॒न्ति॒ श॒तायुः॑ श॒तायु॑र् भवन्ति भवन्ति श॒तायुः॑ । \newline
5. श॒तायुः॒ पुरु॑षः॒ पुरु॑षः श॒तायुः॑ श॒तायुः॒ पुरु॑षः । \newline
6. श॒तायु॒रिति॑ श॒त - आ॒युः॒ । \newline
7. पुरु॑षः श॒तेन्द्रि॑यः श॒तेन्द्रि॑यः॒ पुरु॑षः॒ पुरु॑षः श॒तेन्द्रि॑यः । \newline
8. श॒तेन्द्रि॑य॒ आयु॒ष्यायु॑षि श॒तेन्द्रि॑यः श॒तेन्द्रि॑य॒ आयु॑षि । \newline
9. श॒तेन्द्रि॑य॒ इति॑ श॒त - इ॒न्द्रि॒यः॒ । \newline
10. आयु॑ष्ये॒वैवायु॒ष्यायु॑ष्ये॒व । \newline
11. ए॒वे न्द्रि॒य इ॑न्द्रि॒य ए॒वैवे न्द्रि॒ये । \newline
12. इ॒न्द्रि॒ये प्रति॒ प्रती᳚न्द्रि॒य इ॑न्द्रि॒ये प्रति॑ । \newline
13. प्रति॑ तिष्ठति तिष्ठति॒ प्रति॒ प्रति॑ तिष्ठति । \newline
14. ति॒ष्ठ॒ति॒ यद् यत् ति॑ष्ठति तिष्ठति॒ यत् । \newline
15. यद् वै वै यद् यद् वै । \newline
16. वा अ॒ग्निर॒ग्निर् वै वा अ॒ग्निः । \newline
17. अ॒ग्निराहि॑त॒ आहि॑तो॒ ऽग्निर॒ग्निराहि॑तः । \newline
18. आहि॑तो॒ न नाहि॑त॒ आहि॑तो॒ न । \newline
19. आहि॑त॒ इत्या - हि॒तः॒ । \newline
20. न र्द्ध्यत॑ ऋ॒द्ध्यते॒ न न र्द्ध्यते᳚ । \newline
21. ऋ॒द्ध्यते॒ ज्यायो॒ ज्याय॑ ऋ॒द्ध्यत॑ ऋ॒द्ध्यते॒ ज्यायः॑ । \newline
22. ज्यायो॑ भाग॒धेय॑म् भाग॒धेय॒म् ज्यायो॒ ज्यायो॑ भाग॒धेय᳚म् । \newline
23. भा॒ग॒धेय॑न्निका॒मय॑मानो निका॒मय॑मानो भाग॒धेय॑म् भाग॒धेय॑न्निका॒मय॑मानः । \newline
24. भा॒ग॒धेय॒मिति॑ भाग - धेय᳚म् । \newline
25. नि॒का॒मय॑मानो॒ यद् यन् नि॑का॒मय॑मानो निका॒मय॑मानो॒ यत् । \newline
26. नि॒का॒मय॑मान॒ इति॑ नि - का॒मय॑मानः । \newline
27. यदा᳚ग्ने॒य मा᳚ग्ने॒यं ॅयद् यदा᳚ग्ने॒यम् । \newline
28. आ॒ग्ने॒यꣳ सर्व॒(ग्म्॒) सर्व॑ माग्ने॒य मा᳚ग्ने॒यꣳ सर्व᳚म् । \newline
29. सर्व॒म् भव॑ति॒ भव॑ति॒ सर्व॒(ग्म्॒) सर्व॒म् भव॑ति । \newline
30. भव॑ति॒ सा सा भव॑ति॒ भव॑ति॒ सा । \newline
31. सैवैव सा सैव । \newline
32. ए॒वास्या᳚स्यै॒वैवास्य॑ । \newline
33. अ॒स्य र्द्धि॒र्॒. ऋद्धि॑रस्या॒स्य र्द्धिः॑ । \newline
34. ऋद्धिः॒ सꣳ स मृद्धि॒र्॒. ऋद्धिः॒ सम् । \newline
35. सं ॅवै वै सꣳ सं ॅवै । \newline
36. वा ए॒तस्यै॒तस्य॒ वै वा ए॒तस्य॑ । \newline
37. ए॒तस्य॑ गृ॒हे गृ॒ह ए॒तस्यै॒तस्य॑ गृ॒हे । \newline
38. गृ॒हे वाग् वाग् गृ॒हे गृ॒हे वाक् । \newline
39. वाख् सृ॑ज्यते सृज्यते॒ वाग् वाख् सृ॑ज्यते । \newline
40. सृ॒ज्य॒ते॒ यो यः सृ॑ज्यते सृज्यते॒ यः । \newline
41. यो᳚ ऽग्नि म॒ग्निं ॅयो यो᳚ ऽग्निम् । \newline
42. अ॒ग्नि मु॑द्वा॒सय॑त उद्वा॒सय॑ते॒ ऽग्नि म॒ग्नि मु॑द्वा॒सय॑ते । \newline
43. उ॒द्वा॒सय॑ते॒ स स उ॑द्वा॒सय॑त उद्वा॒सय॑ते॒ सः । \newline
44. उ॒द्वा॒सय॑त॒ इत्यु॑त् - वा॒सय॑ते । \newline
45. स वाचं॒ ॅवाच॒(ग्म्॒) स स वाच᳚म् । \newline
46. वाच॒(ग्म्॒) सꣳसृ॑ष्टा॒(ग्म्॒) सꣳसृ॑ष्टां॒ ॅवाचं॒ ॅवाच॒(ग्म्॒) सꣳसृ॑ष्टाम् । \newline
47. सꣳसृ॑ष्टां॒ ॅयज॑मानो॒ यज॑मानः॒ सꣳसृ॑ष्टा॒(ग्म्॒) सꣳसृ॑ष्टां॒ ॅयज॑मानः । \newline
48. सꣳसृ॑ष्टा॒मिति॒ सं - सृ॒ष्टा॒म् । \newline
49. यज॑मान ईश्व॒र ई᳚श्व॒रो यज॑मानो॒ यज॑मान ईश्व॒रः । \newline
50. ई॒श्व॒रो ऽन्वन्वी᳚श्व॒र ई᳚श्व॒रो ऽनु॑ । \newline
51. अनु॒ परा॑भवितोः॒ परा॑भवितो॒रन्वनु॒ परा॑भवितोः । \newline
52. परा॑भवितो॒र् विभ॑क्तयो॒ विभ॑क्तयः॒ परा॑भवितोः॒ परा॑भवितो॒र् विभ॑क्तयः । \newline
53. परा॑भवितो॒रिति॒ परा᳚ - भ॒वि॒तोः॒ । \newline
54. विभ॑क्तयो भवन्ति भवन्ति॒ विभ॑क्तयो॒ विभ॑क्तयो भवन्ति । \newline
55. विभ॑क्तय॒ इति॒ वि - भ॒क्त॒यः॒ । \newline
56. भ॒व॒न्ति॒ वा॒चो वा॒चो भ॑वन्ति भवन्ति वा॒चः । \newline
57. वा॒चो विधृ॑त्यै॒ विधृ॑त्यै वा॒चो वा॒चो विधृ॑त्यै । \newline
58. विधृ॑त्यै॒ यज॑मानस्य॒ यज॑मानस्य॒ विधृ॑त्यै॒ विधृ॑त्यै॒ यज॑मानस्य । \newline
59. विधृ॑त्या॒ इति॒ वि - धृ॒त्यै॒ । \newline
60. यज॑मान॒स्याप॑राभावा॒याप॑राभावाय॒ यज॑मानस्य॒ यज॑मान॒स्याप॑राभावाय । \newline
61. अप॑राभावाय॒ विभ॑क्तिं॒ ॅविभ॑क्ति॒ मप॑राभावा॒याप॑राभावाय॒ विभ॑क्तिम् । \newline
62. अप॑राभावा॒येत्यप॑रा - भा॒वा॒य॒ । \newline

\textbf{Ghana Paata } \newline

1. ध॒त्ते श॒ताक्ष॑राः श॒ताक्ष॑रा ध॒त्ते ध॒त्ते श॒ताक्ष॑रा भवन्ति भवन्ति श॒ताक्ष॑रा ध॒त्ते ध॒त्ते श॒ताक्ष॑रा भवन्ति । \newline
2. श॒ताक्ष॑रा भवन्ति भवन्ति श॒ताक्ष॑राः श॒ताक्ष॑रा भवन्ति श॒तायुः॑ श॒तायु॑र् भवन्ति श॒ताक्ष॑राः श॒ताक्ष॑रा भवन्ति श॒तायुः॑ । \newline
3. श॒ताक्ष॑रा॒ इति॑ श॒त - अ॒क्ष॒राः॒ । \newline
4. भ॒व॒न्ति॒ श॒तायुः॑ श॒तायु॑र् भवन्ति भवन्ति श॒तायुः॒ पुरु॑षः॒ पुरु॑षः श॒तायु॑र् भवन्ति भवन्ति श॒तायुः॒ पुरु॑षः । \newline
5. श॒तायुः॒ पुरु॑षः॒ पुरु॑षः श॒तायुः॑ श॒तायुः॒ पुरु॑षः श॒तेन्द्रि॑यः श॒तेन्द्रि॑यः॒ पुरु॑षः श॒तायुः॑ श॒तायुः॒ पुरु॑षः श॒तेन्द्रि॑यः । \newline
6. श॒तायु॒रिति॑ श॒त - आ॒युः॒ । \newline
7. पुरु॑षः श॒तेन्द्रि॑यः श॒तेन्द्रि॑यः॒ पुरु॑षः॒ पुरु॑षः श॒तेन्द्रि॑य॒ आयु॒ष्यायु॑षि श॒तेन्द्रि॑यः॒ पुरु॑षः॒ पुरु॑षः श॒तेन्द्रि॑य॒ आयु॑षि । \newline
8. श॒तेन्द्रि॑य॒ आयु॒ष्यायु॑षि श॒तेन्द्रि॑यः श॒तेन्द्रि॑य॒ आयु॑ष्ये॒वैवायु॑षि श॒तेन्द्रि॑यः श॒तेन्द्रि॑य॒ आयु॑ष्ये॒व । \newline
9. श॒तेन्द्रि॑य॒ इति॑ श॒त - इ॒न्द्रि॒यः॒ । \newline
10. आयु॑ष्ये॒वै वायु॒ष्यायु॑ष्ये॒वे न्द्रि॒य इ॑न्द्रि॒य ए॒वायु॒ष्यायु॑ष्ये॒वे न्द्रि॒ये । \newline
11. ए॒वे न्द्रि॒य इ॑न्द्रि॒य ए॒वैवे न्द्रि॒ये प्रति॒ प्रती᳚न्द्रि॒य ए॒वैवे न्द्रि॒ये प्रति॑ । \newline
12. इ॒न्द्रि॒ये प्रति॒ प्रती᳚न्द्रि॒य इ॑न्द्रि॒ये प्रति॑ तिष्ठति तिष्ठति॒ प्रती᳚न्द्रि॒य इ॑न्द्रि॒ये प्रति॑ तिष्ठति । \newline
13. प्रति॑ तिष्ठति तिष्ठति॒ प्रति॒ प्रति॑ तिष्ठति॒ यद् यत् ति॑ष्ठति॒ प्रति॒ प्रति॑ तिष्ठति॒ यत् । \newline
14. ति॒ष्ठ॒ति॒ यद् यत् ति॑ष्ठति तिष्ठति॒ यद् वै वै यत् ति॑ष्ठति तिष्ठति॒ यद् वै । \newline
15. यद् वै वै यद् यद् वा अ॒ग्नि र॒ग्निर् वै यद् यद् वा अ॒ग्निः । \newline
16. वा अ॒ग्नि र॒ग्निर् वै वा अ॒ग्निराहि॑त॒ आहि॑तो॒ ऽग्निर् वै वा अ॒ग्निराहि॑तः । \newline
17. अ॒ग्निराहि॑त॒ आहि॑तो॒ ऽग्नि र॒ग्निराहि॑तो॒ न नाहि॑तो॒ ऽग्निर॒ग्निराहि॑तो॒ न । \newline
18. आहि॑तो॒ न नाहि॑त॒ आहि॑तो॒ न र्‌द्ध्यत॑ ऋ॒द्ध्यते॒ नाहि॑त॒ आहि॑तो॒ न र्‌द्ध्यते᳚ । \newline
19. आहि॑त॒ इत्या - हि॒तः॒ । \newline
20. न र्‌द्ध्यत॑ ऋ॒द्ध्यते॒ न न र्‌द्ध्यते॒ ज्यायो॒ ज्याय॑ ऋ॒द्ध्यते॒ न न र्‌द्ध्यते॒ ज्यायः॑ । \newline
21. ऋ॒द्ध्यते॒ ज्यायो॒ ज्याय॑ ऋ॒द्ध्यत॑ ऋ॒द्ध्यते॒ ज्यायो॑ भाग॒धेय॑म् भाग॒धेय॒म् ज्याय॑ ऋ॒द्ध्यत॑ ऋ॒द्ध्यते॒ ज्यायो॑ भाग॒धेय᳚म् । \newline
22. ज्यायो॑ भाग॒धेय॑म् भाग॒धेय॒म् ज्यायो॒ ज्यायो॑ भाग॒धेय॑म् निका॒मय॑मानो निका॒मय॑मानो भाग॒धेय॒म् ज्यायो॒ ज्यायो॑ भाग॒धेय॑म् निका॒मय॑मानः । \newline
23. भा॒ग॒धेय॑म् निका॒मय॑मानो निका॒मय॑मानो भाग॒धेय॑म् भाग॒धेय॑म् निका॒मय॑मानो॒ यद् यन् नि॑का॒मय॑मानो भाग॒धेय॑म् भाग॒धेय॑म् निका॒मय॑मानो॒ यत् । \newline
24. भा॒ग॒धेय॒मिति॑ भाग - धेय᳚म् । \newline
25. नि॒का॒मय॑मानो॒ यद् यन् नि॑का॒मय॑मानो निका॒मय॑मानो॒ यदा᳚ग्ने॒य मा᳚ग्ने॒यं ॅयन् नि॑का॒मय॑मानो निका॒मय॑मानो॒ यदा᳚ग्ने॒यम् । \newline
26. नि॒का॒मय॑मान॒ इति॑ नि - का॒मय॑मानः । \newline
27. यदा᳚ग्ने॒य मा᳚ग्ने॒यं ॅयद् यदा᳚ग्ने॒यꣳ सर्व॒(ग्म्॒) सर्व॑ माग्ने॒यं ॅयद् यदा᳚ग्ने॒यꣳ सर्व᳚म् । \newline
28. आ॒ग्ने॒यꣳ सर्व॒(ग्म्॒) सर्व॑ माग्ने॒य मा᳚ग्ने॒यꣳ सर्व॒म् भव॑ति॒ भव॑ति॒ सर्व॑ माग्ने॒य मा᳚ग्ने॒यꣳ सर्व॒म् भव॑ति । \newline
29. सर्व॒म् भव॑ति॒ भव॑ति॒ सर्व॒(ग्म्॒) सर्व॒म् भव॑ति॒ सा सा भव॑ति॒ सर्व॒(ग्म्॒) सर्व॒म् भव॑ति॒ सा । \newline
30. भव॑ति॒ सा सा भव॑ति॒ भव॑ति॒ सैवैव सा भव॑ति॒ भव॑ति॒ सैव । \newline
31. सैवैव सा सैवास्या᳚स्यै॒व सा सैवास्य॑ । \newline
32. ए॒वास्या᳚स्यै॒वैवास्य र्‌द्धि॒र्॒. ऋद्धि॑ रस्यै॒वैवास्य र्‌द्धिः॑ । \newline
33. अ॒स्य र्‌द्धि॒र्॒. ऋद्धि॑ रस्या॒स्यर्‌द्धिः॒ सꣳ स मृद्धि॑ रस्या॒स्यर्‌द्धिः॒ सम् । \newline
34. ऋद्धिः॒ सꣳ स मृद्धि॒र्॒. ऋद्धिः॒ सं ॅवै वै स मृद्धि॒र्॒. ऋद्धिः॒ सं ॅवै । \newline
35. सं ॅवै वै सꣳ सं ॅवा ए॒तस्यै॒तस्य॒ वै सꣳ सं ॅवा ए॒तस्य॑ । \newline
36. वा ए॒तस्यै॒तस्य॒ वै वा ए॒तस्य॑ गृ॒हे गृ॒ह ए॒तस्य॒ वै वा ए॒तस्य॑ गृ॒हे । \newline
37. ए॒तस्य॑ गृ॒हे गृ॒ह ए॒तस्यै॒तस्य॑ गृ॒हे वाग् वाग् गृ॒ह ए॒तस्यै॒तस्य॑ गृ॒हे वाक् । \newline
38. गृ॒हे वाग् वाग् गृ॒हे गृ॒हे वाख् सृ॑ज्यते सृज्यते॒ वाग् गृ॒हे गृ॒हे वाख् सृ॑ज्यते । \newline
39. वाख् सृ॑ज्यते सृज्यते॒ वाग् वाख् सृ॑ज्यते॒ यो यः सृ॑ज्यते॒ वाग् वाख् सृ॑ज्यते॒ यः । \newline
40. सृ॒ज्य॒ते॒ यो यः सृ॑ज्यते सृज्यते॒ यो᳚ ऽग्नि म॒ग्निं ॅयः सृ॑ज्यते सृज्यते॒ यो᳚ ऽग्निम् । \newline
41. यो᳚ ऽग्नि म॒ग्निं ॅयो यो᳚ ऽग्नि मु॑द्वा॒सय॑त उद्वा॒सय॑ते॒ ऽग्निं ॅयो यो᳚ ऽग्नि मु॑द्वा॒सय॑ते । \newline
42. अ॒ग्नि मु॑द्वा॒सय॑त उद्वा॒सय॑ते॒ ऽग्नि म॒ग्नि मु॑द्वा॒सय॑ते॒ स स उ॑द्वा॒सय॑ते॒ ऽग्नि म॒ग्नि मु॑द्वा॒सय॑ते॒ सः । \newline
43. उ॒द्वा॒सय॑ते॒ स स उ॑द्वा॒सय॑त उद्वा॒सय॑ते॒ स वाचं॒ ॅवाच॒(ग्म्॒) स उ॑द्वा॒सय॑त उद्वा॒सय॑ते॒ स वाच᳚म् । \newline
44. उ॒द्वा॒सय॑त॒ इत्यु॑त् - वा॒सय॑ते । \newline
45. स वाचं॒ ॅवाच॒(ग्म्॒) स स वाच॒(ग्म्॒) सꣳसृ॑ष्टा॒(ग्म्॒) सꣳसृ॑ष्टां॒ ॅवाच॒(ग्म्॒) स स वाच॒(ग्म्॒) सꣳसृ॑ष्टाम् । \newline
46. वाच॒(ग्म्॒) सꣳसृ॑ष्टा॒(ग्म्॒) सꣳसृ॑ष्टां॒ ॅवाचं॒ ॅवाच॒(ग्म्॒) सꣳसृ॑ष्टां॒ ॅयज॑मानो॒ यज॑मानः॒ सꣳसृ॑ष्टां॒ ॅवाचं॒ ॅवाच॒(ग्म्॒) सꣳसृ॑ष्टां॒ ॅयज॑मानः । \newline
47. सꣳसृ॑ष्टां॒ ॅयज॑मानो॒ यज॑मानः॒ सꣳसृ॑ष्टा॒(ग्म्॒) सꣳसृ॑ष्टां॒ ॅयज॑मान ईश्व॒र ई᳚श्व॒रो यज॑मानः॒ सꣳसृ॑ष्टा॒(ग्म्॒) सꣳसृ॑ष्टां॒ ॅयज॑मान ईश्व॒रः । \newline
48. सꣳसृ॑ष्टा॒मिति॒ सं - सृ॒ष्टा॒म् । \newline
49. यज॑मान ईश्व॒र ई᳚श्व॒रो यज॑मानो॒ यज॑मान ईश्व॒रो ऽन्वन्वी᳚श्व॒रो यज॑मानो॒ यज॑मान ईश्व॒रो ऽनु॑ । \newline
50. ई॒श्व॒रो ऽन्वन्वी᳚श्व॒र ई᳚श्व॒रो ऽनु॒ परा॑भवितोः॒ परा॑भवितो॒ रन्वी᳚श्व॒र ई᳚श्व॒रो ऽनु॒ परा॑भवितोः । \newline
51. अनु॒ परा॑भवितोः॒ परा॑भवितो॒ रन्वनु॒ परा॑भवितो॒र् विभ॑क्तयो॒ विभ॑क्तयः॒ परा॑भवितो॒ रन्वनु॒ परा॑भवितो॒र् विभ॑क्तयः । \newline
52. परा॑भवितो॒र् विभ॑क्तयो॒ विभ॑क्तयः॒ परा॑भवितोः॒ परा॑भवितो॒र् विभ॑क्तयो भवन्ति भवन्ति॒ विभ॑क्तयः॒ परा॑भवितोः॒ परा॑भवितो॒र् विभ॑क्तयो भवन्ति । \newline
53. परा॑भवितो॒रिति॒ परा᳚ - भ॒वि॒तोः॒ । \newline
54. विभ॑क्तयो भवन्ति भवन्ति॒ विभ॑क्तयो॒ विभ॑क्तयो भवन्ति वा॒चो वा॒चो भ॑वन्ति॒ विभ॑क्तयो॒ विभ॑क्तयो भवन्ति वा॒चः । \newline
55. विभ॑क्तय॒ इति॒ वि - भ॒क्त॒यः॒ । \newline
56. भ॒व॒न्ति॒ वा॒चो वा॒चो भ॑वन्ति भवन्ति वा॒चो विधृ॑त्यै॒ विधृ॑त्यै वा॒चो भ॑वन्ति भवन्ति वा॒चो विधृ॑त्यै । \newline
57. वा॒चो विधृ॑त्यै॒ विधृ॑त्यै वा॒चो वा॒चो विधृ॑त्यै॒ यज॑मानस्य॒ यज॑मानस्य॒ विधृ॑त्यै वा॒चो वा॒चो विधृ॑त्यै॒ यज॑मानस्य । \newline
58. विधृ॑त्यै॒ यज॑मानस्य॒ यज॑मानस्य॒ विधृ॑त्यै॒ विधृ॑त्यै॒ यज॑मान॒ स्याप॑राभावा॒या प॑राभावाय॒ यज॑मानस्य॒ विधृ॑त्यै॒ विधृ॑त्यै॒ यज॑मान॒स्या प॑राभावाय । \newline
59. विधृ॑त्या॒ इति॒ वि - धृ॒त्यै॒ । \newline
60. यज॑मान॒स्या प॑राभावा॒या प॑राभावाय॒ यज॑मानस्य॒ यज॑मान॒स्या प॑राभावाय॒ विभ॑क्तिं॒ ॅविभ॑क्ति॒ मप॑राभावाय॒ यज॑मानस्य॒ यज॑मान॒स्या प॑राभावाय॒ विभ॑क्तिम् । \newline
61. अप॑राभावाय॒ विभ॑क्तिं॒ ॅविभ॑क्ति॒ मप॑राभावा॒या प॑राभावाय॒ विभ॑क्तिम् करोति करोति॒ विभ॑क्ति॒ मप॑राभावा॒याप॑ राभावाय॒ विभ॑क्तिम् करोति । \newline
62. अप॑राभावा॒येत्यप॑रा - भा॒वा॒य॒ । \newline
\pagebreak
\markright{ TS 1.5.2.3  \hfill https://www.vedavms.in \hfill}
\addcontentsline{toc}{section}{ TS 1.5.2.3 }
\section*{ TS 1.5.2.3 }

\textbf{TS 1.5.2.3 } \newline
\textbf{Samhita Paata} \newline

विभ॑क्तिं करोति॒ ब्रह्मै॒व तद॑करुपाꣳ॒॒शु य॑जति॒ यथा॑ वा॒मं ॅवसु॑ विविदा॒नो गूह॑ति ता॒दृगे॒व तद॒ग्निं प्रति॑ स्विष्ट॒कृतं॒ निरा॑ह॒ यथा॑ वा॒मं ॅवसु॑ विविदा॒नः प्र॑का॒शं जिग॑मिषति ता॒दृगे॒व तद्विभ॑क्तिमु॒क्त्वा प्र॑या॒जेन॒ वष॑ट्करोत्या॒यत॑नादे॒व नैति॒ यज॑मानो॒ वै पु॑रो॒डाशः॑ प॒शव॑ ए॒ते आहु॑ती॒ यद॒भितः॑ पुरो॒डाश॑मे॒ते आहु॑ती - [ ] \newline

\textbf{Pada Paata} \newline

विभ॑क्ति॒मिति॒ वि - भ॒क्ति॒म् । क॒रो॒ति॒ । ब्रह्म॑ । ए॒व । तत् । अ॒कः॒ । उ॒पाꣳ॒॒श्वित्यु॑प-अꣳ॒॒शु । य॒ज॒ति॒ । यथा᳚ । वा॒मम् । वसु॑ । वि॒वि॒दा॒नः । गूह॑ति । ता॒दृक् । ए॒व । तत् । अ॒ग्निम् । प्रतीति॑ । स्वि॒ष्ट॒कृत॒मिति॑ स्विष्ट- कृत᳚म् । निरिति॑ । आ॒ह॒ । यथा᳚ । वा॒मम् । वसु॑ । वि॒वि॒दा॒नः । प्र॒का॒शमिति॑ प्र - का॒शम् । जिग॑मिषति । ता॒दृक् । ए॒व । तत् । विभ॑क्ति॒मिति॒ वि - भ॒क्ति॒म् । उ॒क्त्वा । प्र॒या॒जेनेति॑ प्र - या॒जेन॑ । वष॑ट् । क॒रो॒ति॒ । आ॒यत॑ना॒दित्या᳚ - यत॑नात् । ए॒व । न । ए॒ति॒ । यज॑मानः । वै । पु॒रो॒डाशः॑ । प॒शवः॑ । ए॒ते इति॑ । आहु॑ती॒ इत्या - हु॒ती॒ । यत् । अ॒भितः॑ । पु॒रो॒डाश᳚म् । ए॒ते इति॑ । आहु॑ती॒ इत्या - हु॒ती॒ ।  \newline


\textbf{Krama Paata} \newline

विभ॑क्तिम् करोति । विभ॑क्ति॒मिति॒ वि - भ॒क्ति॒म् । क॒रो॒ति॒ ब्रह्म॑ । ब्रह्मै॒व । ए॒व तत् । तद॑कः । अ॒क॒रु॒पाꣳ॒॒शु । उ॒पाꣳ॒॒शु य॑जति । उ॒पाꣳ॒॒श्वित्यु॑प - अꣳ॒॒शु । य॒ज॒ति॒ यथा᳚ । यथा॑ वा॒मम् । वा॒मं ॅवसु॑ । वसु॑ विविदा॒नः । वि॒वि॒दा॒नो गूह॑ति । गूह॑ति ता॒दृक् । ता॒दृगे॒व । ए॒व तत् । तद॒ग्निम् । अ॒ग्निम् प्रति॑ । प्रति॑ स्विष्ट॒कृत᳚म् । स्वि॒ष्ट॒कृत॒न्निः । स्वि॒ष्ट॒कृत॒मिति॑ स्विष्ट - कृत᳚म् । निरा॑ह । आ॒ह॒ यथा᳚ । यथा॑ वा॒मम् । वा॒मं ॅवसु॑ । वसु॑ विविदा॒नः । वि॒वि॒दा॒नः प्र॑का॒शम् । प्र॒का॒शम् जिग॑मिषति । प्र॒का॒शमिति॑ प्र - का॒शम् । जिग॑मिषति ता॒दृक् । ता॒दृगे॒व । ए॒व तत् । तद् विभ॑क्तिम् । विभ॑क्तिमु॒क्त्वा । विभ॑क्ति॒मिति॒ वि - भ॒क्ति॒म् । उ॒क्त्वा प्र॑या॒जेन॑ । प्र॒या॒जेन॒ वष॑ट् । प्र॒या॒जेनेति॑ प्र - या॒जेन॑ । वष॑ट् करोति । क॒रो॒त्या॒यत॑नात् । आ॒यत॑नादे॒व । आ॒यत॑ना॒दित्या᳚ - यत॑नात् । ए॒व न । नैति॑ । ए॒ति॒ यज॑मानः । यज॑मानो॒ वै । वै पु॑रो॒डाशः॑ । पु॒रो॒डाशः॑ प॒शवः॑ । प॒शव॑ ए॒ते । ए॒ते आहु॑ती । ए॒ते इत्ये॒ते । आहु॑ती॒ यत् । आहु॑ती॒ इत्या - हु॒ती॒ । यद॒भितः॑ । अ॒भितः॑ पुरो॒डाश᳚म् । पु॒रो॒डाश॑मे॒ते । ए॒ते आहु॑ती । ए॒ते इत्ये॒ते । आहु॑ती जु॒होति॑ । आहु॑ती॒ इत्या - हु॒ती॒ \newline

\textbf{Jatai Paata} \newline

1. विभ॑क्तिम् करोति करोति॒ विभ॑क्तिं॒ ॅविभ॑क्तिम् करोति । \newline
2. विभ॑क्ति॒मिति॒ वि - भ॒क्ति॒म् । \newline
3. क॒रो॒ति॒ ब्रह्म॒ ब्रह्म॑ करोति करोति॒ ब्रह्म॑ । \newline
4. ब्रह्मै॒वैव ब्रह्म॒ ब्रह्मै॒व । \newline
5. ए॒व तत् तदे॒वैव तत् । \newline
6. तद॑क रक॒स्तत् तद॑कः । \newline
7. अ॒क॒ रु॒पा॒(ग्म्॒)शू॑पा॒(ग्ग्॒)श्व॑क रक रुपा॒(ग्म्॒)शु । \newline
8. उ॒पा॒(ग्म्॒)शु य॑जति यजत्युपा॒(ग्म्॒)शू॑पा॒(ग्म्॒)शु य॑जति । \newline
9. उ॒पा॒(ग्ग्॒)श्वित्यु॑प - अ॒(ग्म्॒)शु । \newline
10. य॒ज॒ति॒ यथा॒ यथा॑ यजति यजति॒ यथा᳚ । \newline
11. यथा॑ वा॒मं ॅवा॒मं ॅयथा॒ यथा॑ वा॒मम् । \newline
12. वा॒मं ॅवसु॒ वसु॑ वा॒मं ॅवा॒मं ॅवसु॑ । \newline
13. वसु॑ विविदा॒नो वि॑विदा॒नो वसु॒ वसु॑ विविदा॒नः । \newline
14. वि॒वि॒दा॒नो गूह॑ति॒ गूह॑ति विविदा॒नो वि॑विदा॒नो गूह॑ति । \newline
15. गूह॑ति ता॒दृक् ता॒दृग् गूह॑ति॒ गूह॑ति ता॒दृक् । \newline
16. ता॒दृगे॒वैव ता॒दृक् ता॒दृगे॒व । \newline
17. ए॒व तत् तदे॒वैव तत् । \newline
18. तद॒ग्नि म॒ग्निम् तत् तद॒ग्निम् । \newline
19. अ॒ग्निम् प्रति॒ प्रत्य॒ग्नि म॒ग्निम् प्रति॑ । \newline
20. प्रति॑ स्विष्ट॒कृत(ग्ग्॑) स्विष्ट॒कृत॒म् प्रति॒ प्रति॑ स्विष्ट॒कृत᳚म् । \newline
21. स्वि॒ष्ट॒कृत॒न्निर् णिष् स्वि॑ष्ट॒कृत(ग्ग्॑) स्विष्ट॒कृत॒न्निः । \newline
22. स्वि॒ष्ट॒कृत॒मिति॑ स्विष्ट - कृत᳚म् । \newline
23. निरा॑हाह॒ निर् णिरा॑ह । \newline
24. आ॒ह॒ यथा॒ यथा॑ ऽऽहाह॒ यथा᳚ । \newline
25. यथा॑ वा॒मं ॅवा॒मं ॅयथा॒ यथा॑ वा॒मम् । \newline
26. वा॒मं ॅवसु॒ वसु॑ वा॒मं ॅवा॒मं ॅवसु॑ । \newline
27. वसु॑ विविदा॒नो वि॑विदा॒नो वसु॒ वसु॑ विविदा॒नः । \newline
28. वि॒वि॒दा॒नः प्र॑का॒शम् प्र॑का॒शं ॅवि॑विदा॒नो वि॑विदा॒नः प्र॑का॒शम् । \newline
29. प्र॒का॒शम् जिग॑मिषति॒ जिग॑मिषति प्रका॒शम् प्र॑का॒शम् जिग॑मिषति । \newline
30. प्र॒का॒शमिति॑ प्र - का॒शम् । \newline
31. जिग॑मिषति ता॒दृक् ता॒दृग् जिग॑मिषति॒ जिग॑मिषति ता॒दृक् । \newline
32. ता॒दृगे॒वैव ता॒दृक् ता॒दृगे॒व । \newline
33. ए॒व तत् तदे॒वैव तत् । \newline
34. तद् विभ॑क्तिं॒ ॅविभ॑क्ति॒म् तत् तद् विभ॑क्तिम् । \newline
35. विभ॑क्ति मु॒क्त्वोक्त्वा विभ॑क्तिं॒ ॅविभ॑क्ति मु॒क्त्वा । \newline
36. विभ॑क्ति॒मिति॒ वि - भ॒क्ति॒म् । \newline
37. उ॒क्त्वा प्र॑या॒जेन॑ प्रया॒जेनो॒क्त्वोक्त्वा प्र॑या॒जेन॑ । \newline
38. प्र॒या॒जेन॒ वष॒ड् वष॑ट् प्रया॒जेन॑ प्रया॒जेन॒ वष॑ट् । \newline
39. प्र॒या॒जेनेति॑ प्र - या॒जेन॑ । \newline
40. वष॑ट् करोति करोति॒ वष॒ड् वष॑ट् करोति । \newline
41. क॒रो॒त्या॒यत॑नादा॒यत॑नात् करोति करोत्या॒यत॑नात् । \newline
42. आ॒यत॑ना दे॒वैवायत॑ना दा॒यत॑नादे॒व । \newline
43. आ॒यत॑ना॒दित्या᳚ - यत॑नात् । \newline
44. ए॒व न नैवैव न । \newline
45. नैत्ये॑ति॒ न नैति॑ । \newline
46. ए॒ति॒ यज॑मानो॒ यज॑मान एत्येति॒ यज॑मानः । \newline
47. यज॑मानो॒ वै वै यज॑मानो॒ यज॑मानो॒ वै । \newline
48. वै पु॑रो॒डाशः॑ पुरो॒डाशो॒ वै वै पु॑रो॒डाशः॑ । \newline
49. पु॒रो॒डाशः॑ प॒शवः॑ प॒शवः॑ पुरो॒डाशः॑ पुरो॒डाशः॑ प॒शवः॑ । \newline
50. प॒शव॑ ए॒ते ए॒ते प॒शवः॑ प॒शव॑ ए॒ते । \newline
51. ए॒ते आहु॑ती॒ आहु॑ती ए॒ते ए॒ते आहु॑ती । \newline
52. ए॒ते इत्ये॒ते । \newline
53. आहु॑ती॒ यद् यदाहु॑ती॒ आहु॑ती॒ यत् । \newline
54. आहु॑ती॒ इत्या - हु॒ती॒ । \newline
55. यद॒भितो॒ ऽभितो॒ यद् यद॒भितः॑ । \newline
56. अ॒भितः॑ पुरो॒डाश॑म् पुरो॒डाश॑ म॒भितो॒ ऽभितः॑ पुरो॒डाश᳚म् । \newline
57. पु॒रो॒डाश॑ मे॒ते ए॒ते पु॑रो॒डाश॑म् पुरो॒डाश॑ मे॒ते । \newline
58. ए॒ते आहु॑ती॒ आहु॑ती ए॒ते ए॒ते आहु॑ती । \newline
59. ए॒ते इत्ये॒ते । \newline
60. आहु॑ती जु॒होति॑ जु॒होत्याहु॑ती॒ आहु॑ती जु॒होति॑ । \newline
61. आहु॑ती॒ इत्या - हु॒ती॒ । \newline

\textbf{Ghana Paata } \newline

1. विभ॑क्तिम् करोति करोति॒ विभ॑क्तिं॒ ॅविभ॑क्तिम् करोति॒ ब्रह्म॒ ब्रह्म॑ करोति॒ विभ॑क्तिं॒ ॅविभ॑क्तिम् करोति॒ ब्रह्म॑ । \newline
2. विभ॑क्ति॒मिति॒ वि - भ॒क्ति॒म् । \newline
3. क॒रो॒ति॒ ब्रह्म॒ ब्रह्म॑ करोति करोति॒ ब्रह्मै॒वैव ब्रह्म॑ करोति करोति॒ ब्रह्मै॒व । \newline
4. ब्रह्मै॒वैव ब्रह्म॒ ब्रह्मै॒व तत् तदे॒व ब्रह्म॒ ब्रह्मै॒व तत् । \newline
5. ए॒व तत् तदे॒वैव तद॑क रक॒ स्तदे॒वैव तद॑कः । \newline
6. तद॑क रक॒स्तत् तद॑क रुपा॒(ग्म्॒) शू॑पा॒(ग्ग्॒)श्व॑क॒ स्तत् तद॑क रुपा॒(ग्म्॒)शु । \newline
7. अ॒क॒ रु॒पा॒(ग्म्॒) शू॑पा॒(ग्ग्॒)श्व॑क रक रुपा॒(ग्म्॒)शु य॑जति यजत्युपा॒(ग्ग्॒)श्व॑क रक रुपा॒(ग्म्॒)शु य॑जति । \newline
8. उ॒पा॒(ग्म्॒)शु य॑जति यजत्युपा॒(ग्म्॒) शू॑पा॒(ग्म्॒)शु य॑जति॒ यथा॒ यथा॑ यजत्युपा॒(ग्म्॒) शू॑पा॒(ग्म्॒)शु य॑जति॒ यथा᳚ । \newline
9. उ॒पा॒(ग्ग्॒)श्वित्यु॑प - अ॒(ग्म्॒)शु । \newline
10. य॒ज॒ति॒ यथा॒ यथा॑ यजति यजति॒ यथा॑ वा॒मं ॅवा॒मं ॅयथा॑ यजति यजति॒ यथा॑ वा॒मम् । \newline
11. यथा॑ वा॒मं ॅवा॒मं ॅयथा॒ यथा॑ वा॒मं ॅवसु॒ वसु॑ वा॒मं ॅयथा॒ यथा॑ वा॒मं ॅवसु॑ । \newline
12. वा॒मं ॅवसु॒ वसु॑ वा॒मं ॅवा॒मं ॅवसु॑ विविदा॒नो वि॑विदा॒नो वसु॑ वा॒मं ॅवा॒मं ॅवसु॑ विविदा॒नः । \newline
13. वसु॑ विविदा॒नो वि॑विदा॒नो वसु॒ वसु॑ विविदा॒नो गूह॑ति॒ गूह॑ति विविदा॒नो वसु॒ वसु॑ विविदा॒नो गूह॑ति । \newline
14. वि॒वि॒दा॒नो गूह॑ति॒ गूह॑ति विविदा॒नो वि॑विदा॒नो गूह॑ति ता॒दृक् ता॒दृग् गूह॑ति विविदा॒नो वि॑विदा॒नो गूह॑ति ता॒दृक् । \newline
15. गूह॑ति ता॒दृक् ता॒दृग् गूह॑ति॒ गूह॑ति ता॒दृगे॒वैव ता॒दृग् गूह॑ति॒ गूह॑ति ता॒दृगे॒व । \newline
16. ता॒दृगे॒वैव ता॒दृक् ता॒दृगे॒व तत् तदे॒व ता॒दृक् ता॒दृगे॒व तत् । \newline
17. ए॒व तत् तदे॒वैव तद॒ग्नि म॒ग्निम् तदे॒वैव तद॒ग्निम् । \newline
18. तद॒ग्नि म॒ग्निम् तत् तद॒ग्निम् प्रति॒ प्रत्य॒ग्निम् तत् तद॒ग्निम् प्रति॑ । \newline
19. अ॒ग्निम् प्रति॒ प्रत्य॒ग्नि म॒ग्निम् प्रति॑ स्विष्ट॒कृत(ग्ग्॑) स्विष्ट॒कृत॒म् प्रत्य॒ग्नि म॒ग्निम् प्रति॑ स्विष्ट॒कृत᳚म् । \newline
20. प्रति॑ स्विष्ट॒कृत(ग्ग्॑) स्विष्ट॒कृत॒म् प्रति॒ प्रति॑ स्विष्ट॒कृत॒न्निर् णिष् स्वि॑ष्ट॒कृत॒म् प्रति॒ प्रति॑ स्विष्ट॒कृत॒न्निः । \newline
21. स्वि॒ष्ट॒कृत॒म् निर् णिष् स्वि॑ष्ट॒कृत(ग्ग्॑) स्विष्ट॒कृत॒म् निरा॑हाह॒ निष् स्वि॑ष्ट॒कृत(ग्ग्॑) स्विष्ट॒कृत॒म् निरा॑ह । \newline
22. स्वि॒ष्ट॒कृत॒मिति॑ स्विष्ट - कृत᳚म् । \newline
23. निरा॑हाह॒ निर् णिरा॑ह॒ यथा॒ यथा॑ ऽऽह॒ निर् णिरा॑ह॒ यथा᳚ । \newline
24. आ॒ह॒ यथा॒ यथा॑ ऽऽहाह॒ यथा॑ वा॒मं ॅवा॒मं ॅयथा॑ ऽऽहाह॒ यथा॑ वा॒मम् । \newline
25. यथा॑ वा॒मं ॅवा॒मं ॅयथा॒ यथा॑ वा॒मं ॅवसु॒ वसु॑ वा॒मं ॅयथा॒ यथा॑ वा॒मं ॅवसु॑ । \newline
26. वा॒मं ॅवसु॒ वसु॑ वा॒मं ॅवा॒मं ॅवसु॑ विविदा॒नो वि॑विदा॒नो वसु॑ वा॒मं ॅवा॒मं ॅवसु॑ विविदा॒नः । \newline
27. वसु॑ विविदा॒नो वि॑विदा॒नो वसु॒ वसु॑ विविदा॒नः प्र॑का॒शम् प्र॑का॒शं ॅवि॑विदा॒नो वसु॒ वसु॑ विविदा॒नः प्र॑का॒शम् । \newline
28. वि॒वि॒दा॒नः प्र॑का॒शम् प्र॑का॒शं ॅवि॑विदा॒नो वि॑विदा॒नः प्र॑का॒शम् जिग॑मिषति॒ जिग॑मिषति प्रका॒शं ॅवि॑विदा॒नो वि॑विदा॒नः प्र॑का॒शम् जिग॑मिषति । \newline
29. प्र॒का॒शम् जिग॑मिषति॒ जिग॑मिषति प्रका॒शम् प्र॑का॒शम् जिग॑मिषति ता॒दृक् ता॒दृग् जिग॑मिषति प्रका॒शम् प्र॑का॒शम् जिग॑मिषति ता॒दृक् । \newline
30. प्र॒का॒शमिति॑ प्र - का॒शम् । \newline
31. जिग॑मिषति ता॒दृक् ता॒दृग् जिग॑मिषति॒ जिग॑मिषति ता॒दृगे॒वैव ता॒दृग् जिग॑मिषति॒ जिग॑मिषति ता॒दृगे॒व । \newline
32. ता॒दृगे॒वैव ता॒दृक् ता॒दृगे॒व तत् तदे॒व ता॒दृक् ता॒दृगे॒व तत् । \newline
33. ए॒व तत् तदे॒वैव तद् विभ॑क्तिं॒ ॅविभ॑क्ति॒म् तदे॒वैव तद् विभ॑क्तिम् । \newline
34. तद् विभ॑क्तिं॒ ॅविभ॑क्ति॒म् तत् तद् विभ॑क्ति मु॒क्त्वोक्त्वा विभ॑क्ति॒म् तत् तद् विभ॑क्ति मु॒क्त्वा । \newline
35. विभ॑क्ति मु॒क्त्वोक्त्वा विभ॑क्तिं॒ ॅविभ॑क्ति मु॒क्त्वा प्र॑या॒जेन॑ प्रया॒जेनो॒क्त्वा विभ॑क्तिं॒ ॅविभ॑क्ति मु॒क्त्वा प्र॑या॒जेन॑ । \newline
36. विभ॑क्ति॒मिति॒ वि - भ॒क्ति॒म् । \newline
37. उ॒क्त्वा प्र॑या॒जेन॑ प्रया॒जेनो॒क्त्वोक्त्वा प्र॑या॒जेन॒ वष॒ड् वष॑ट् प्रया॒जेनो॒क्त्वोक्त्वा प्र॑या॒जेन॒ वष॑ट् । \newline
38. प्र॒या॒जेन॒ वष॒ड् वष॑ट् प्रया॒जेन॑ प्रया॒जेन॒ वष॑ट् करोति करोति॒ वष॑ट् प्रया॒जेन॑ प्रया॒जेन॒ वष॑ट् करोति । \newline
39. प्र॒या॒जेनेति॑ प्र - या॒जेन॑ । \newline
40. वष॑ट् करोति करोति॒ वष॒ड् वष॑ट् करोत्या॒यत॑ना दा॒यत॑नात् करोति॒ वष॒ड् वष॑ट् करोत्या॒यत॑नात् । \newline
41. क॒रो॒त्या॒यत॑ना दा॒यत॑नात् करोति करोत्या॒यत॑ना दे॒वैवायत॑नात् करोति करोत्या॒यत॑नादे॒व । \newline
42. आ॒यत॑ना दे॒वैवायत॑ना दा॒यत॑नादे॒व न नैवायत॑ना दा॒यत॑नादे॒व न । \newline
43. आ॒यत॑ना॒दित्या᳚ - यत॑नात् । \newline
44. ए॒व न नैवैव नैत्ये॑ति॒ नैवैव नैति॑ । \newline
45. नैत्ये॑ति॒ न नैति॒ यज॑मानो॒ यज॑मान एति॒ न नैति॒ यज॑मानः । \newline
46. ए॒ति॒ यज॑मानो॒ यज॑मान एत्येति॒ यज॑मानो॒ वै वै यज॑मान एत्येति॒ यज॑मानो॒ वै । \newline
47. यज॑मानो॒ वै वै यज॑मानो॒ यज॑मानो॒ वै पु॑रो॒डाशः॑ पुरो॒डाशो॒ वै यज॑मानो॒ यज॑मानो॒ वै पु॑रो॒डाशः॑ । \newline
48. वै पु॑रो॒डाशः॑ पुरो॒डाशो॒ वै वै पु॑रो॒डाशः॑ प॒शवः॑ प॒शवः॑ पुरो॒डाशो॒ वै वै पु॑रो॒डाशः॑ प॒शवः॑ । \newline
49. पु॒रो॒डाशः॑ प॒शवः॑ प॒शवः॑ पुरो॒डाशः॑ पुरो॒डाशः॑ प॒शव॑ ए॒ते ए॒ते प॒शवः॑ पुरो॒डाशः॑ पुरो॒डाशः॑ प॒शव॑ ए॒ते । \newline
50. प॒शव॑ ए॒ते ए॒ते प॒शवः॑ प॒शव॑ ए॒ते आहु॑ती॒ आहु॑ती ए॒ते प॒शवः॑ प॒शव॑ ए॒ते आहु॑ती । \newline
51. ए॒ते आहु॑ती॒ आहु॑ती ए॒ते ए॒ते आहु॑ती॒ यद् यदाहु॑ती ए॒ते ए॒ते आहु॑ती॒ यत् । \newline
52. ए॒ते इत्ये॒ते । \newline
53. आहु॑ती॒ यद् यदाहु॑ती॒ आहु॑ती॒ यद॒भितो॒ ऽभितो॒ यदाहु॑ती॒ आहु॑ती॒ यद॒भितः॑ । \newline
54. आहु॑ती॒ इत्या - हु॒ती॒ । \newline
55. यद॒भितो॒ ऽभितो॒ यद् यद॒भितः॑ पुरो॒डाश॑म् पुरो॒डाश॑ म॒भितो॒ यद् यद॒भितः॑ पुरो॒डाश᳚म् । \newline
56. अ॒भितः॑ पुरो॒डाश॑म् पुरो॒डाश॑ म॒भितो॒ ऽभितः॑ पुरो॒डाश॑ मे॒ते ए॒ते पु॑रो॒डाश॑ म॒भितो॒ ऽभितः॑ पुरो॒डाश॑ मे॒ते । \newline
57. पु॒रो॒डाश॑ मे॒ते ए॒ते पु॑रो॒डाश॑म् पुरो॒डाश॑ मे॒ते आहु॑ती॒ आहु॑ती ए॒ते पु॑रो॒डाश॑म् पुरो॒डाश॑ मे॒ते आहु॑ती । \newline
58. ए॒ते आहु॑ती॒ आहु॑ती ए॒ते ए॒ते आहु॑ती जु॒होति॑ जु॒होत्याहु॑ती ए॒ते ए॒ते आहु॑ती जु॒होति॑ । \newline
59. ए॒ते इत्ये॒ते । \newline
60. आहु॑ती जु॒होति॑ जु॒होत्याहु॑ती॒ आहु॑ती जु॒होति॒ यज॑मानं॒ ॅयज॑मानम् जु॒होत्याहु॑ती॒ आहु॑ती जु॒होति॒ यज॑मानम् । \newline
61. आहु॑ती॒ इत्या - हु॒ती॒ । \newline
\pagebreak
\markright{ TS 1.5.2.4  \hfill https://www.vedavms.in \hfill}
\addcontentsline{toc}{section}{ TS 1.5.2.4 }
\section*{ TS 1.5.2.4 }

\textbf{TS 1.5.2.4 } \newline
\textbf{Samhita Paata} \newline

जु॒होति॒ यज॑मानमे॒वोभ॒यतः॑ प॒शुभिः॒ परि॑ गृह्णाति कृ॒तय॑जुः॒ संभृ॑तसंभार॒ इत्या॑हु॒र्न सं॒भृत्याः᳚ संभा॒रा न यजुः॑ कर्त॒व्य॑मित्यथो॒ खलु॑ सं॒भृत्या॑ ए॒व सं॑भा॒राः क॑र्त॒व्यं॑ ॅयजु॑र् य॒ज्ञ्स्य॒ समृ॑द्ध्यै पुनर्निष्कृ॒तो रथो॒ दक्षि॑णा पुनरुथ्स्यू॒तं ॅवासः॑ पुनरुथ्सृ॒ष्टो॑ऽन॒ड्वान् पु॑नरा॒धेय॑स्य॒ समृ॑द्ध्यै स॒प्त ते॑ अग्ने स॒मिधः॑ स॒प्त जि॒ह्वा इत्य॑ग्निहो॒त्रं जु॑होति॒ यत्र॑यत्रै॒वास्य॒ न्य॑क्तं॒ तत॑ - [ ] \newline

\textbf{Pada Paata} \newline

जु॒होति॑ । यज॑मानम् । ए॒व । उ॒भ॒यतः॑ । प॒शुभि॒रिति॑ प॒शु - भिः॒ । परीति॑ । गृ॒ह्णा॒ति॒ । कृ॒तय॑जु॒रिति॑ कृ॒त - य॒जुः॒ । संभृ॑तसंभार॒ इति॒ संभृ॑त - स॒भां॒रः॒ । इति॑ । आ॒हुः॒ । न । स॒भृंत्या॒ इति॑ सं - भृत्याः᳚ । सं॒भा॒रा इति॑ सं-भा॒राः । न । यजुः॑ । क॒र्त॒व्य᳚म् । इति॑ । अथा॒ इति॑ । खलु॑ । स॒भृंत्या॒ इति॑ सं - भृत्याः᳚ । ए॒व । स॒भां॒रा इति॑ सं-भा॒राः । क॒र्त॒व्य᳚म् । यजुः॑ । य॒ज्ञ्स्य॑ । समृ॑द्ध्या॒ इति॒ सं - ऋ॒द्ध्यै॒ । पु॒न॒र्नि॒ष्कृ॒त इति ॑पुनः - नि॒ष्कृ॒तः । रथः॑ । दक्षि॑णा । पु॒न॒रु॒थ्स्यू॒तमिति॑ पुनः - उ॒थ्स्यू॒तम् । वासः॑ । पु॒न॒रु॒थ्सृ॒ष्ट इति॑ पुनः - उ॒थ्सृ॒ष्टः । अ॒न॒ड्वान् । पु॒न॒रा॒धेय॒स्येति॑ पुनः - आ॒धेय॑स्य । समृ॑द्ध्या॒ इति॒ सं - ऋ॒द्ध्यै॒ । स॒प्त । ते॒ । अ॒ग्ने॒ । स॒मिध॒ इति॑ सं - इधः॑ । स॒प्त । जि॒ह्वाः । इति॑ । अ॒ग्नि॒हो॒त्रमित्य॑ग्नि - हो॒त्रम् । जु॒हो॒ति॒ । यत्र॑य॒त्रेति॒ यत्र॑-य॒त्र॒ । ए॒व । अ॒स्य॒ । न्य॑क्त॒मिति॒ नि - अ॒क्त॒म् । ततः॑ ।  \newline


\textbf{Krama Paata} \newline

जु॒होति॒ यज॑मानम् । यज॑मानमे॒व । ए॒वोभ॒यतः॑ । उ॒भ॒यतः॑ प॒शुभिः॑ । प॒शुभिः॒ परि॑ । प॒शुभि॒रिति॑ प॒शु - भिः॒ । परि॑ गृह्णाति । गृ॒ह्णा॒ति॒ कृ॒तय॑जुः । कृ॒तय॑जुः॒ सम्भृ॑तसम्भारः । कृ॒तय॑जु॒रिति॑ कृ॒त - य॒जुः॒ । सम्भृ॑तसम्भार॒ इति॑ । सम्भृ॑तसम्भार॒ इति॒ सम्भृ॑त - स॒म्भा॒रः॒ । इत्या॑हुः । आ॒हु॒र् न । न स॒म्भृत्याः᳚ । स॒म्भृत्याः᳚ सम्भा॒राः । स॒म्भृत्या॒ इति॑ सम् - भृत्याः᳚ । स॒म्भा॒रा न । स॒म्भा॒रा इति॑ सम् - भा॒राः । न यजुः॑ । यजुः॑ कर्त॒व्य᳚म् । क॒र्त॒व्य॑मिति॑ । इत्यथो᳚ । अथो॒ खलु॑ । अथो॒ इत्यथो᳚ । खलु॑ स॒म्भृत्याः᳚ । स॒म्भृत्या॑ ए॒व । स॒म्भृत्या॒ इति॑ सम् - भृत्याः᳚ । ए॒व स॑म्भा॒राः । स॒म्भा॒राः क॑र्त॒व्य᳚म् । स॒म्भा॒रा इति॑ सम् - भा॒राः । क॒र्त॒व्यं॑ ॅयजुः॑ । यजु॑र् य॒ज्ञ्स्य॑ । य॒ज्ञ्स्य॒ समृ॑द्ध्यै । समृ॑द्ध्यै पुनर्निष्कृ॒तः । समृ॑द्ध्या॒ इति॒ सम् - ऋ॒द्ध्यै॒ । पु॒न॒र्नि॒ष्कृ॒तो रथः॑ । पु॒न॒र्नि॒ष्कृ॒त इति॑ पुनः - नि॒ष्कृ॒तः । रथो॒ दक्षि॑णा । दक्षि॑णा पुनरुथ्स्यू॒तम् । पु॒न॒रु॒थ्स्यू॒तं ॅवासः॑ । पु॒न॒रु॒थ्स्यू॒तमिति॑ पुनः - उ॒थ्स्यू॒तम् । वासः॑ पुनरुथ्सृ॒ष्टः । पु॒न॒रु॒थ्सृ॒ष्टो॑ ऽन॒ड्वान् । पु॒न॒रु॒थ्सृ॒ष्ट इति॑ पुनः - उ॒थ्सृ॒ष्टः । अ॒न॒ड्वान् पु॑नरा॒धेय॑स्य । पु॒न॒रा॒धेय॑स्य॒ समृ॑द्ध्यै । पु॒न॒रा॒धेय॒स्येति॑ पुनः - आ॒धेय॑स्य । समृ॑द्ध्यै स॒प्त । समृ॑द्ध्या॒ इति॒ सम् - ऋ॒द्ध्यै॒ । स॒प्त ते᳚ । ते॒ अ॒ग्ने॒ । अ॒ग्ने॒ स॒मिधः॑ । स॒मिधः॑ स॒प्त । स॒मिध॒ इति॑ सं - इधः॑ । स॒प्त जि॒ह्वाः । जि॒ह्वा इति॑ । इत्य॑ग्निहो॒त्रम् । अ॒ग्नि॒हो॒त्रम् जु॑होति । अ॒ग्नि॒हो॒त्रमित्य॑ग्नि - हो॒त्रम् । जु॒हो॒ति॒ यत्र॑यत्र । यत्र॑यत्रै॒व । यत्र॑य॒त्रेति॒ यत्र॑ - य॒त्र॒ । ए॒वास्य॑ । अ॒स्य॒ न्य॑क्तम् । न्य॑क्त॒म् ततः॑ ( ) । न्य॑क्त॒मिति॒ नि - अ॒क्त॒म् । तत॑ ए॒व \newline

\textbf{Jatai Paata} \newline

1. जु॒होति॒ यज॑मानं॒ ॅयज॑मानम् जु॒होति॑ जु॒होति॒ यज॑मानम् । \newline
2. यज॑मान मे॒वैव यज॑मानं॒ ॅयज॑मान मे॒व । \newline
3. ए॒वोभ॒यत॑ उभ॒यत॑ ए॒वैवोभ॒यतः॑ । \newline
4. उ॒भ॒यतः॑ प॒शुभिः॑ प॒शुभि॑ रुभ॒यत॑ उभ॒यतः॑ प॒शुभिः॑ । \newline
5. प॒शुभिः॒ परि॒ परि॑ प॒शुभिः॑ प॒शुभिः॒ परि॑ । \newline
6. प॒शुभि॒रिति॑ प॒शु - भिः॒ । \newline
7. परि॑ गृह्णाति गृह्णाति॒ परि॒ परि॑ गृह्णाति । \newline
8. गृ॒ह्णा॒ति॒ कृ॒तय॑जुः कृ॒तय॑जुर् गृह्णाति गृह्णाति कृ॒तय॑जुः । \newline
9. कृ॒तय॑जुः॒ संभृ॑तसंभारः॒ संभृ॑तसंभारः कृ॒तय॑जुः कृ॒तय॑जुः॒ संभृ॑तसंभारः । \newline
10. कृ॒तय॑जु॒रिति॑ कृ॒त - य॒जुः॒ । \newline
11. संभृ॑तसंभार॒ इतीति॒ संभृ॑तसंभारः॒ संभृ॑तसंभार॒ इति॑ । \newline
12. संभृ॑तसंभार॒ इति॒ संभृ॑त - सं॒भा॒रः॒ । \newline
13. इत्या॑हु राहु॒ रितीत्या॑हुः । \newline
14. आ॒हु॒र् न नाहु॑राहु॒र् न । \newline
15. न सं॒भृत्याः᳚ सं॒भृत्या॒ न न सं॒भृत्याः᳚ । \newline
16. सं॒भृत्याः᳚ संभा॒राः सं॑भा॒राः सं॒भृत्याः᳚ सं॒भृत्याः᳚ संभा॒राः । \newline
17. सं॒भृत्या॒ इति॑ सं - भृत्याः᳚ । \newline
18. सं॒भा॒रा न न सं॑भा॒राः सं॑भा॒रा न । \newline
19. सं॒भा॒रा इति॑ सं - भा॒राः । \newline
20. न यजु॒र् यजु॒र् न न यजुः॑ । \newline
21. यजुः॑ कर्त॒व्य॑म् कर्त॒व्यं॑ ॅयजु॒र् यजुः॑ कर्त॒व्य᳚म् । \newline
22. क॒र्त॒व्य॑ मितीति॑ कर्त॒व्य॑म् कर्त॒व्य॑ मिति॑ । \newline
23. इत्यथो॒ अथो॒ इतीत्यथो᳚ । \newline
24. अथो॒ खलु॒ खल्वथो॒ अथो॒ खलु॑ । \newline
25. अथो॒ इत्यथो᳚ । \newline
26. खलु॑ सं॒भृत्याः᳚ सं॒भृत्याः॒ खलु॒ खलु॑ सं॒भृत्याः᳚ । \newline
27. सं॒भृत्या॑ ए॒वैव सं॒भृत्याः᳚ सं॒भृत्या॑ ए॒व । \newline
28. सं॒भृत्या॒ इति॑ सं - भृत्याः᳚ । \newline
29. ए॒व स॑म्भा॒राः स॑म्भा॒रा ए॒वैव स॑म्भा॒राः । \newline
30. स॒म्भा॒राः क॑र्त॒व्य॑म् कर्त॒व्य(ग्म्॑) सम्भा॒राः स॑म्भा॒राः क॑र्त॒व्य᳚म् । \newline
31. सं॒भा॒रा इति॑ सं - भा॒राः । \newline
32. क॒र्त॒व्यं॑ ॅयजु॒र् यजुः॑ कर्त॒व्य॑म् कर्त॒व्यं॑ ॅयजुः॑ । \newline
33. यजु॑र् य॒ज्ञ्स्य॑ य॒ज्ञ्स्य॒ यजु॒र् यजु॑र् य॒ज्ञ्स्य॑ । \newline
34. य॒ज्ञ्स्य॒ समृ॑द्ध्यै॒ समृ॑द्ध्यै य॒ज्ञ्स्य॑ य॒ज्ञ्स्य॒ समृ॑द्ध्यै । \newline
35. समृ॑द्ध्यै पुनर्निष्कृ॒तः पु॑नर्निष्कृ॒तः समृ॑द्ध्यै॒ समृ॑द्ध्यै पुनर्निष्कृ॒तः । \newline
36. समृ॑द्ध्या॒ इति॒ सं - ऋ॒द्ध्यै॒ । \newline
37. पु॒न॒र्नि॒ष्कृ॒तो रथो॒ रथः॑ पुनर्निष्कृ॒तः पु॑नर्निष्कृ॒तो रथः॑ । \newline
38. पु॒न॒र्नि॒ष्कृ॒त इति॑ पुनः - नि॒ष्कृ॒तः । \newline
39. रथो॒ दक्षि॑णा॒ दक्षि॑णा॒ रथो॒ रथो॒ दक्षि॑णा । \newline
40. दक्षि॑णा पुनरुथ्स्यू॒तम् पु॑नरुथ्स्यू॒तम् दक्षि॑णा॒ दक्षि॑णा पुनरुथ्स्यू॒तम् । \newline
41. पु॒न॒रु॒थ्स्यू॒तं ॅवासो॒वासः॑ पुनरुथ्स्यू॒तम् पु॑नरुथ्स्यू॒तं ॅवासः॑ । \newline
42. पु॒न॒रु॒थ्स्यू॒तमिति॑ पुनः - उ॒थ्स्यू॒तम् । \newline
43. वासः॑ पुनरुथ्सृ॒ष्टः पु॑नरुथ्सृ॒ष्टो वासो॒वासः॑ पुनरुथ्सृ॒ष्टः । \newline
44. पु॒न॒रु॒थ्सृ॒ष्टो॑ ऽन॒ड्वा न॑न॒ड्वान् पु॑नरुथ्सृ॒ष्टः पु॑नरुथ्सृ॒ष्टो॑ ऽन॒ड्वान् । \newline
45. पु॒न॒रु॒थ्सृ॒ष्ट इति॑ पुनः - उ॒थ्सृ॒ष्टः । \newline
46. अ॒न॒ड्वान् पु॑नरा॒धेय॑स्य पुनरा॒धेय॑स्यान॒ड्वा न॑न॒ड्वान् पु॑नरा॒धेय॑स्य । \newline
47. पु॒न॒रा॒धेय॑स्य॒ समृ॑द्ध्यै॒ समृ॑द्ध्यै पुनरा॒धेय॑स्य पुनरा॒धेय॑स्य॒ समृ॑द्ध्यै । \newline
48. पु॒न॒रा॒धेय॒स्येति॑ पुनः - आ॒धेय॑स्य । \newline
49. समृ॑द्ध्यै स॒प्त स॒प्त समृ॑द्ध्यै॒ समृ॑द्ध्यै स॒प्त । \newline
50. समृ॑द्ध्या॒ इति॒ सं - ऋ॒द्ध्यै॒ । \newline
51. स॒प्त ते॑ ते स॒प्त स॒प्त ते᳚ । \newline
52. ते॒ अ॒ग्ने॒ ऽग्ने॒ ते॒ ते॒ अ॒ग्ने॒ । \newline
53. अ॒ग्ने॒ स॒मिधः॑ स॒मिधो᳚ ऽग्ने ऽग्ने स॒मिधः॑ । \newline
54. स॒मिधः॑ स॒प्त स॒प्त स॒मिधः॑ स॒मिधः॑ स॒प्त । \newline
55. स॒मिध॒ इति॑ सं - इधः॑ । \newline
56. स॒प्त जि॒ह्वा जि॒ह्वाः स॒प्त स॒प्त जि॒ह्वाः । \newline
57. जि॒ह्वा इतीति॑ जि॒ह्वा जि॒ह्वा इति॑ । \newline
58. इत्य॑ग्निहो॒त्र म॑ग्निहो॒त्र मितीत्य॑ग्निहो॒त्रम् । \newline
59. अ॒ग्नि॒हो॒त्रम् जु॑होति जुहोत्यग्निहो॒त्र म॑ग्निहो॒त्रम् जु॑होति । \newline
60. अ॒ग्नि॒हो॒त्रमित्य॑ग्नि - हो॒त्रम् । \newline
61. जु॒हो॒ति॒ यत्र॑यत्र॒ यत्र॑यत्र जुहोति जुहोति॒ यत्र॑यत्र । \newline
62. यत्र॑यत्रै॒वैव यत्र॑यत्र॒ यत्र॑यत्रै॒व । \newline
63. यत्र॑य॒त्रेति॒ यत्र॑ - य॒त्र॒ । \newline
64. ए॒वास्या᳚स्यै॒वैवास्य॑ । \newline
65. अ॒स्य॒ न्य॑क्त॒न्न्य॑क्त मस्यास्य॒ न्य॑क्तम् । \newline
66. न्य॑क्त॒म् तत॒स्ततो॒ न्य॑क्त॒न्न्य॑क्त॒म् ततः॑ । \newline
67. न्य॑क्त॒मिति॒ नि - अ॒क्त॒म् । \newline
68. तत॑ ए॒वैव तत॒स्तत॑ ए॒व । \newline

\textbf{Ghana Paata } \newline

1. जु॒होति॒ यज॑मानं॒ ॅयज॑मानम् जु॒होति॑ जु॒होति॒ यज॑मान मे॒वैव यज॑मानम् जु॒होति॑ जु॒होति॒ यज॑मान मे॒व । \newline
2. यज॑मान मे॒वैव यज॑मानं॒ ॅयज॑मान मे॒वोभ॒यत॑ उभ॒यत॑ ए॒व यज॑मानं॒ ॅयज॑मान मे॒वोभ॒यतः॑ । \newline
3. ए॒वोभ॒यत॑ उभ॒यत॑ ए॒वैवोभ॒यतः॑ प॒शुभिः॑ प॒शुभि॑रुभ॒यत॑ ए॒वैवोभ॒यतः॑ प॒शुभिः॑ । \newline
4. उ॒भ॒यतः॑ प॒शुभिः॑ प॒शुभि॑रुभ॒यत॑ उभ॒यतः॑ प॒शुभिः॒ परि॒ परि॑ प॒शुभि॑रुभ॒यत॑ उभ॒यतः॑ प॒शुभिः॒ परि॑ । \newline
5. प॒शुभिः॒ परि॒ परि॑ प॒शुभिः॑ प॒शुभिः॒ परि॑ गृह्णाति गृह्णाति॒ परि॑ प॒शुभिः॑ प॒शुभिः॒ परि॑ गृह्णाति । \newline
6. प॒शुभि॒रिति॑ प॒शु - भिः॒ । \newline
7. परि॑ गृह्णाति गृह्णाति॒ परि॒ परि॑ गृह्णाति कृ॒तय॑जुः कृ॒तय॑जुर् गृह्णाति॒ परि॒ परि॑ गृह्णाति कृ॒तय॑जुः । \newline
8. गृ॒ह्णा॒ति॒ कृ॒तय॑जुः कृ॒तय॑जुर् गृह्णाति गृह्णाति कृ॒तय॑जुः॒ संभृ॑तसंभारः॒ संभृ॑तसंभारः कृ॒तय॑जुर् गृह्णाति गृह्णाति कृ॒तय॑जुः॒ संभृ॑तसंभारः । \newline
9. कृ॒तय॑जुः॒ संभृ॑तसंभारः॒ संभृ॑तसंभारः कृ॒तय॑जुः कृ॒तय॑जुः॒ संभृ॑तसंभार॒ इतीति॒ संभृ॑तसंभारः कृ॒तय॑जुः कृ॒तय॑जुः॒ संभृ॑तसंभार॒ इति॑ । \newline
10. कृ॒तय॑जु॒रिति॑ कृ॒त - य॒जुः॒ । \newline
11. संभृ॑तसंभार॒ इतीति॒ संभृ॑तसंभारः॒ संभृ॑तसंभार॒ इत्या॑हुराहु॒रिति॒ संभृ॑तसंभारः॒ संभृ॑तसंभार॒ इत्या॑हुः । \newline
12. संभृ॑तसंभार॒ इति॒ संभृ॑त - सं॒भा॒रः॒ । \newline
13. इत्या॑हु राहु॒रितीत्या॑हु॒र् न नाहु॒ रितीत्या॑हु॒र् न । \newline
14. आ॒हु॒र् न नाहु॑राहु॒र् न सं॒भृत्याः᳚ सं॒भृत्या॒ नाहु॑राहु॒र् न सं॒भृत्याः᳚ । \newline
15. न सं॒भृत्याः᳚ सं॒भृत्या॒ न न सं॒भृत्याः᳚ संभा॒राः सं॑भा॒राः सं॒भृत्या॒ न न सं॒भृत्याः᳚ संभा॒राः । \newline
16. सं॒भृत्याः᳚ संभा॒राः सं॑भा॒राः सं॒भृत्याः᳚ सं॒भृत्याः᳚ संभा॒रा न न सं॑भा॒राः सं॒भृत्याः᳚ सं॒भृत्याः᳚ संभा॒रा न । \newline
17. सं॒भृत्या॒ इति॑ सं - भृत्याः᳚ । \newline
18. सं॒भा॒रा न न सं॑भा॒राः सं॑भा॒रा न यजु॒र् यजु॒र् न सं॑भा॒राः सं॑भा॒रा न यजुः॑ । \newline
19. सं॒भा॒रा इति॑ सं - भा॒राः । \newline
20. न यजु॒र् यजु॒र् न न यजुः॑ कर्त॒व्य॑म् कर्त॒व्यं॑ ॅयजु॒र् न न यजुः॑ कर्त॒व्य᳚म् । \newline
21. यजुः॑ कर्त॒व्य॑म् कर्त॒व्यं॑ ॅयजु॒र् यजुः॑ कर्त॒व्य॑ मितीति॑ कर्त॒व्यं॑ ॅयजु॒र् यजुः॑ कर्त॒व्य॑ मिति॑ । \newline
22. क॒र्त॒व्य॑ मितीति॑ कर्त॒व्य॑म् कर्त॒व्य॑ मित्यथो॒ अथो॒ इति॑ कर्त॒व्य॑म् कर्त॒व्य॑ मित्यथो᳚ । \newline
23. इत्यथो॒ अथो॒ इतीत्यथो॒ खलु॒ खल्वथो॒ इतीत्यथो॒ खलु॑ । \newline
24. अथो॒ खलु॒ खल्वथो॒ अथो॒ खलु॑ सं॒भृत्याः᳚ सं॒भृत्याः॒ खल्वथो॒ अथो॒ खलु॑ सं॒भृत्याः᳚ । \newline
25. अथो॒ इत्यथो᳚ । \newline
26. खलु॑ सं॒भृत्याः᳚ सं॒भृत्याः॒ खलु॒ खलु॑ सं॒भृत्या॑ ए॒वैव सं॒भृत्याः॒ खलु॒ खलु॑ सं॒भृत्या॑ ए॒व । \newline
27. सं॒भृत्या॑ ए॒वैव सं॒भृत्याः᳚ सं॒भृत्या॑ ए॒व स॑म्भा॒राः स॑म्भा॒रा ए॒व सं॒भृत्याः᳚ सं॒भृत्या॑ ए॒व स॑म्भा॒राः । \newline
28. सं॒भृत्या॒ इति॑ सं - भृत्याः᳚ । \newline
29. ए॒व स॑म्भा॒राः स॑म्भा॒रा ए॒वैव स॑म्भा॒राः क॑र्त॒व्य॑म् कर्त॒व्य(ग्म्॑) सम्भा॒रा ए॒वैव स॑म्भा॒राः क॑र्त॒व्य᳚म् । \newline
30. स॒म्भा॒राः क॑र्त॒व्य॑म् कर्त॒व्य(ग्म्॑) सम्भा॒राः स॑म्भा॒राः क॑र्त॒व्यं॑ ॅयजु॒र् यजुः॑ कर्त॒व्य(ग्म्॑) सम्भा॒राः स॑म्भा॒राः क॑र्त॒व्यं॑ ॅयजुः॑ । \newline
31. सं॒भा॒रा इति॑ सं - भा॒राः । \newline
32. क॒र्त॒व्यं॑ ॅयजु॒र् यजुः॑ कर्त॒व्य॑म् कर्त॒व्यं॑ ॅयजु॑र् य॒ज्ञ्स्य॑ य॒ज्ञ्स्य॒ यजुः॑ कर्त॒व्य॑म् कर्त॒व्यं॑ ॅयजु॑र् य॒ज्ञ्स्य॑ । \newline
33. यजु॑र् य॒ज्ञ्स्य॑ य॒ज्ञ्स्य॒ यजु॒र् यजु॑र् य॒ज्ञ्स्य॒ समृ॑द्ध्यै॒ समृ॑द्ध्यै य॒ज्ञ्स्य॒ यजु॒र् यजु॑र् य॒ज्ञ्स्य॒ समृ॑द्ध्यै । \newline
34. य॒ज्ञ्स्य॒ समृ॑द्ध्यै॒ समृ॑द्ध्यै य॒ज्ञ्स्य॑ य॒ज्ञ्स्य॒ समृ॑द्ध्यै पुनर्निष्कृ॒तः पु॑नर्निष्कृ॒तः समृ॑द्ध्यै य॒ज्ञ्स्य॑ य॒ज्ञ्स्य॒ समृ॑द्ध्यै पुनर्निष्कृ॒तः । \newline
35. समृ॑द्ध्यै पुनर्निष्कृ॒तः पु॑नर्निष्कृ॒तः समृ॑द्ध्यै॒ समृ॑द्ध्यै पुनर्निष्कृ॒तो रथो॒ रथः॑ पुनर्निष्कृ॒तः समृ॑द्ध्यै॒ समृ॑द्ध्यै पुनर्निष्कृ॒तो रथः॑ । \newline
36. समृ॑द्ध्या॒ इति॒ सं - ऋ॒द्ध्यै॒ । \newline
37. पु॒न॒र्नि॒ष्कृ॒तो रथो॒ रथः॑ पुनर्निष्कृ॒तः पु॑नर्निष्कृ॒तो रथो॒ दक्षि॑णा॒ दक्षि॑णा॒ रथः॑ पुनर्निष्कृ॒तः पु॑नर्निष्कृ॒तो रथो॒ दक्षि॑णा । \newline
38. पु॒न॒र्नि॒ष्कृ॒त इति॑ पुनः - नि॒ष्कृ॒तः । \newline
39. रथो॒ दक्षि॑णा॒ दक्षि॑णा॒ रथो॒ रथो॒ दक्षि॑णा पुनरुथ्स्यू॒तम् पु॑नरुथ्स्यू॒तम् दक्षि॑णा॒ रथो॒ रथो॒ दक्षि॑णा पुनरुथ्स्यू॒तम् । \newline
40. दक्षि॑णा पुनरुथ्स्यू॒तम् पु॑नरुथ्स्यू॒तम् दक्षि॑णा॒ दक्षि॑णा पुनरुथ्स्यू॒तं ॅवासो॒ वासः॑ पुनरुथ्स्यू॒तम् दक्षि॑णा॒ दक्षि॑णा पुनरुथ्स्यू॒तं ॅवासः॑ । \newline
41. पु॒न॒रु॒थ्स्यू॒तं ॅवासो॒ वासः॑ पुनरुथ्स्यू॒तम् पु॑नरुथ्स्यू॒तं ॅवासः॑ पुनरुथ्सृ॒ष्टः पु॑नरुथ्सृ॒ष्टो वासः॑ पुनरुथ्स्यू॒तम् पु॑नरुथ्स्यू॒तं ॅवासः॑ पुनरुथ्सृ॒ष्टः । \newline
42. पु॒न॒रु॒थ्स्यू॒तमिति॑ पुनः - उ॒थ्स्यू॒तम् । \newline
43. वासः॑ पुनरुथ्सृ॒ष्टः पु॑नरुथ्सृ॒ष्टो वासो॒ वासः॑ पुनरुथ्सृ॒ष्टो॑ ऽन॒ड्वा न॑न॒ड्वान् पु॑नरुथ्सृ॒ष्टो वासो॒ वासः॑ पुनरुथ्सृ॒ष्टो॑ ऽन॒ड्वान् । \newline
44. पु॒न॒रु॒थ्सृ॒ष्टो॑ ऽन॒ड्वा न॑न॒ड्वान् पु॑नरुथ्सृ॒ष्टः पु॑नरुथ्सृ॒ष्टो॑ ऽन॒ड्वान् पु॑नरा॒धेय॑स्य पुनरा॒धेय॑स्यान॒ड्वान् पु॑नरुथ्सृ॒ष्टः पु॑नरुथ्सृ॒ष्टो॑ ऽन॒ड्वान् पु॑नरा॒धेय॑स्य । \newline
45. पु॒न॒रु॒थ्सृ॒ष्ट इति॑ पुनः - उ॒थ्सृ॒ष्टः । \newline
46. अ॒न॒ड्वान् पु॑नरा॒धेय॑स्य पुनरा॒धेय॑स्यान॒ड्वा न॑न॒ड्वान् पु॑नरा॒धेय॑स्य॒ समृ॑द्ध्यै॒ समृ॑द्ध्यै पुनरा॒धेय॑स्यान॒ड्वा न॑न॒ड्वान् पु॑नरा॒धेय॑स्य॒ समृ॑द्ध्यै । \newline
47. पु॒न॒रा॒धेय॑स्य॒ समृ॑द्ध्यै॒ समृ॑द्ध्यै पुनरा॒धेय॑स्य पुनरा॒धेय॑स्य॒ समृ॑द्ध्यै स॒प्त स॒प्त समृ॑द्ध्यै पुनरा॒धेय॑स्य पुनरा॒धेय॑स्य॒ समृ॑द्ध्यै स॒प्त । \newline
48. पु॒न॒रा॒धेय॒स्येति॑ पुनः - आ॒धेय॑स्य । \newline
49. समृ॑द्ध्यै स॒प्त स॒प्त समृ॑द्ध्यै॒ समृ॑द्ध्यै स॒प्त ते॑ ते स॒प्त समृ॑द्ध्यै॒ समृ॑द्ध्यै स॒प्त ते᳚ । \newline
50. समृ॑द्ध्या॒ इति॒ सं - ऋ॒द्ध्यै॒ । \newline
51. स॒प्त ते॑ ते स॒प्त स॒प्त ते॑ अग्ने ऽग्ने ते स॒प्त स॒प्त ते॑ अग्ने । \newline
52. ते॒ अ॒ग्ने॒ ऽग्ने॒ ते॒ ते॒ अ॒ग्ने॒ स॒मिधः॑ स॒मिधो᳚ ऽग्ने ते ते अग्ने स॒मिधः॑ । \newline
53. अ॒ग्ने॒ स॒मिधः॑ स॒मिधो᳚ ऽग्ने ऽग्ने स॒मिधः॑ स॒प्त स॒प्त स॒मिधो᳚ ऽग्ने ऽग्ने स॒मिधः॑ स॒प्त । \newline
54. स॒मिधः॑ स॒प्त स॒प्त स॒मिधः॑ स॒मिधः॑ स॒प्त जि॒ह्वा जि॒ह्वाः स॒प्त स॒मिधः॑ स॒मिधः॑ स॒प्त जि॒ह्वाः । \newline
55. स॒मिध॒ इति॑ सं - इधः॑ । \newline
56. स॒प्त जि॒ह्वा जि॒ह्वाः स॒प्त स॒प्त जि॒ह्वा इतीति॑ जि॒ह्वाः स॒प्त स॒प्त जि॒ह्वा इति॑ । \newline
57. जि॒ह्वा इतीति॑ जि॒ह्वा जि॒ह्वा इत्य॑ग्निहो॒त्र म॑ग्निहो॒त्र मिति॑ जि॒ह्वा जि॒ह्वा इत्य॑ग्निहो॒त्रम् । \newline
58. इत्य॑ग्निहो॒त्र म॑ग्निहो॒त्र मितीत्य॑ग्निहो॒त्रम् जु॑होति जुहोत्यग्निहो॒त्र मितीत्य॑ग्निहो॒त्रम् जु॑होति । \newline
59. अ॒ग्नि॒हो॒त्रम् जु॑होति जुहोत्यग्निहो॒त्र म॑ग्निहो॒त्रम् जु॑होति॒ यत्र॑यत्र॒ यत्र॑यत्र जुहोत्यग्निहो॒त्र म॑ग्निहो॒त्रम् जु॑होति॒ यत्र॑यत्र । \newline
60. अ॒ग्नि॒हो॒त्रमित्य॑ग्नि - हो॒त्रम् । \newline
61. जु॒हो॒ति॒ यत्र॑यत्र॒ यत्र॑यत्र जुहोति जुहोति॒ यत्र॑यत्रै॒वैव यत्र॑यत्र जुहोति जुहोति॒ यत्र॑यत्रै॒व । \newline
62. यत्र॑यत्रै॒वैव यत्र॑यत्र॒ यत्र॑यत्रै॒वास्या᳚स्यै॒व यत्र॑यत्र॒ यत्र॑यत्रै॒वास्य॑ । \newline
63. यत्र॑य॒त्रेति॒ यत्र॑ - य॒त्र॒ । \newline
64. ए॒वास्या᳚स्यै॒वैवास्य॒ न्य॑क्त॒म् न्य॑क्त मस्यै॒वैवास्य॒ न्य॑क्तम् । \newline
65. अ॒स्य॒ न्य॑क्त॒म् न्य॑क्त मस्यास्य॒ न्य॑क्त॒म् तत॒स्ततो॒ न्य॑क्त मस्यास्य॒ न्य॑क्त॒म् ततः॑ । \newline
66. न्य॑क्त॒म् तत॒स्ततो॒ न्य॑क्त॒म् न्य॑क्त॒म् तत॑ ए॒वैव ततो॒ न्य॑क्त॒म् न्य॑क्त॒म् तत॑ ए॒व । \newline
67. न्य॑क्त॒मिति॒ नि - अ॒क्त॒म् । \newline
68. तत॑ ए॒वैव तत॒स्तत॑ ए॒वैन॑ मेन मे॒व तत॒स्तत॑ ए॒वैन᳚म् । \newline
\pagebreak
\markright{ TS 1.5.2.5  \hfill https://www.vedavms.in \hfill}
\addcontentsline{toc}{section}{ TS 1.5.2.5 }
\section*{ TS 1.5.2.5 }

\textbf{TS 1.5.2.5 } \newline
\textbf{Samhita Paata} \newline

ए॒वैन॒मव॑ रुन्धे वीर॒हा वा ए॒ष दे॒वानां॒ ॅयो᳚ऽग्निमु॑द्वा॒सय॑ते॒ तस्य॒ वरु॑ण ए॒वर्ण॒यादा᳚ग्निवारु॒ण-मेका॑दशकपाल॒मनु॒ निर्व॑पे॒द्यं चै॒व हन्ति॒ यश्चा᳚स्यर्ण॒यात्तौ भा॑ग॒धेये॑न प्रीणाति॒ नाऽऽर्ति॒मार्च्छ॑ति॒ यज॑मानः ॥ \newline

\textbf{Pada Paata} \newline

ए॒व । ए॒न॒म् । अवेति॑ । रु॒न्धे॒ । वी॒र॒हेति॑ वीर - हा । वै । ए॒षः । दे॒वाना᳚म् । यः । अ॒ग्निम् । उ॒द्वा॒सय॑त॒ इत्यु॑त् - वा॒सय॑ते । तस्य॑ । वरु॑णः । ए॒व । ऋ॒ण॒यादित्यृ॑ण - यात् । आ॒ग्नि॒वा॒रु॒णमित्या᳚ग्नि - वा॒रु॒णम् । एका॑दशकपाल॒मित्येका॑दश - क॒पा॒ल॒म् । अनु॑ । निरिति॑ । व॒पे॒त् । यम् । च॒ । ए॒व । हन्ति॑ । यः । च॒ । अ॒स्य॒ । ऋ॒ण॒यादित्यृ॑ण - यात् । तौ । भा॒ग॒धेये॒नेति॑ भाग - धेये॑न । प्री॒णा॒ति॒ । न । आर्ति᳚म् । एति॑ । ऋ॒च्छ॒ति॒ । यज॑मानः ॥  \newline


\textbf{Krama Paata} \newline

ए॒वैन᳚म् । ए॒न॒मव॑ । अव॑रुन्धे । रु॒न्धे॒ वी॒र॒हा । वी॒र॒हा वै । वी॒र॒हेति॑ वीर - हा । वा ए॒षः । ए॒ष दे॒वाना᳚म् । दे॒वानां॒ ॅयः । यो᳚ऽग्निम् । अ॒ग्निमु॑द्वा॒सय॑ते । उ॒द्वा॒सय॑ते॒ तस्य॑ । उ॒द्वा॒सय॑त॒ इत्यु॑त् - वा॒सय॑ते । तस्य॒ वरु॑णः । वरु॑ण ए॒व । ए॒वर्ण॒यात् । ऋ॒ण॒यादा᳚ग्निवारु॒णम् । ऋ॒ण॒यादित्यृ॑ण - यात् । आ॒ग्नि॒वा॒रु॒णमेका॑दशकपालम् । आ॒ग्नि॒वा॒रु॒णमित्या᳚ग्नि - वा॒रु॒णम् । एका॑दशकपाल॒मनु॑ । एका॑दशकपाल॒मित्येका॑दश - क॒पा॒ल॒म् । अनु॒ निः । निर् व॑पेत् । व॒पे॒द् यम् । यम् च॑ । चै॒व । ए॒व हन्ति॑ । हन्ति॒ यः । यश्च॑ । चा॒स्य॒ । अ॒स्य॒र्ण॒यात् । ऋ॒ण॒यात् तौ । ऋ॒ण॒यादित्यृ॑ण - यात् । तौ भा॑ग॒धेये॑न । भा॒ग॒धेये॑न प्रीणाति । भा॒ग॒धेये॒नेति॑ भाग - धेये॑न । प्री॒णा॒ति॒ न । नार्ति᳚म् । आर्ति॒मा । आर्च्छ॑ति । ऋ॒च्छ॒ति॒ यज॑मानः । यज॑मान॒ इति॒ यज॑मानः । \newline

\textbf{Jatai Paata} \newline

1. ए॒वैन॑ मेन मे॒वैवैन᳚म् । \newline
2. ए॒न॒ मवावै॑न मेन॒ मव॑ । \newline
3. अव॑ रुन्धे रु॒न्धे ऽवाव॑ रुन्धे । \newline
4. रु॒न्धे॒ वी॒र॒हा वी॑र॒हा रु॑न्धे रुन्धे वीर॒हा । \newline
5. वी॒र॒हा वै वै वी॑र॒हा वी॑र॒हा वै । \newline
6. वी॒र॒हेति॑ वीर - हा । \newline
7. वा ए॒ष ए॒ष वै वा ए॒षः । \newline
8. ए॒ष दे॒वाना᳚म् दे॒वाना॑ मे॒ष ए॒ष दे॒वाना᳚म् । \newline
9. दे॒वानां॒ ॅयो यो दे॒वाना᳚म् दे॒वानां॒ ॅयः । \newline
10. यो᳚ ऽग्नि म॒ग्निं ॅयो यो᳚ ऽग्निम् । \newline
11. अ॒ग्नि मु॑द्वा॒सय॑त उद्वा॒सय॑ते॒ ऽग्नि म॒ग्नि मु॑द्वा॒सय॑ते । \newline
12. उ॒द्वा॒सय॑ते॒ तस्य॒ तस्यो᳚द्वा॒सय॑त उद्वा॒सय॑ते॒ तस्य॑ । \newline
13. उ॒द्वा॒सय॑त॒ इत्यु॑त् - वा॒सय॑ते । \newline
14. तस्य॒ वरु॑णो॒ वरु॑ण॒स्तस्य॒ तस्य॒ वरु॑णः । \newline
15. वरु॑ण ए॒वैव वरु॑णो॒ वरु॑ण ए॒व । \newline
16. ए॒व र्ण॒यादृ॑ण॒यादे॒वैव र्ण॒यात् । \newline
17. ऋ॒ण॒यादा᳚ग्निवारु॒ण मा᳚ग्निवारु॒ण मृ॑ण॒यादृ॑ण॒यादा᳚ग्निवारु॒णम् । \newline
18. ऋ॒ण॒यादित्यृ॑ण - यात् । \newline
19. आ॒ग्नि॒वा॒रु॒ण मेका॑दशकपाल॒ मेका॑दशकपाल माग्निवारु॒ण मा᳚ग्निवारु॒ण मेका॑दशकपालम् । \newline
20. आ॒ग्नि॒वा॒रु॒णमित्या᳚ग्नि - वा॒रु॒णम् । \newline
21. एका॑दशकपाल॒ मन्वन्वेका॑दशकपाल॒ मेका॑दशकपाल॒ मनु॑ । \newline
22. एका॑दशकपाल॒मित्येका॑दश - क॒पा॒ल॒म् । \newline
23. अनु॒ निर् णिरन्वनु॒ निः । \newline
24. निर् व॑पेद् वपे॒न् निर् णिर् व॑पेत् । \newline
25. व॒पे॒द् यं ॅयं ॅव॑पेद् वपे॒द् यम् । \newline
26. यम् च॑ च॒ यं ॅयम् च॑ । \newline
27. चै॒वैव च॑ चै॒व । \newline
28. ए॒व हन्ति॒ हन्त्ये॒वैव हन्ति॑ । \newline
29. हन्ति॒ यो यो हन्ति॒ हन्ति॒ यः । \newline
30. यश्च॑ च॒ यो यश्च॑ । \newline
31. चा॒स्या॒स्य॒ च॒ चा॒स्य॒ । \newline
32. अ॒स्य॒ र्ण॒यादृ॑ण॒याद॑स्यास्य र्ण॒यात् । \newline
33. ऋ॒ण॒यात् तौ ता वृ॑ण॒यादृ॑ण॒यात् तौ । \newline
34. ऋ॒ण॒यादित्यृ॑ण - यात् । \newline
35. तौ भा॑ग॒धेये॑न भाग॒धेये॑न॒ तौ तौ भा॑ग॒धेये॑न । \newline
36. भा॒ग॒धेये॑न प्रीणाति प्रीणाति भाग॒धेये॑न भाग॒धेये॑न प्रीणाति । \newline
37. भा॒ग॒धेये॒नेति॑ भाग - धेये॑न । \newline
38. प्री॒णा॒ति॒ न न प्री॑णाति प्रीणाति॒ न । \newline
39. नार्ति॒ मार्ति॒न्न नार्ति᳚म् । \newline
40. आर्ति॒ मा ऽऽर्ति॒ मार्ति॒ मा । \newline
41. आर्च्छ॑त्यृच्छ॒त्यार्च्छ॑ति । \newline
42. ऋ॒च्छ॒ति॒ यज॑मानो॒ यज॑मान ऋच्छत्यृच्छति॒ यज॑मानः । \newline
43. यज॑मान॒ इति॒ यज॑मानः । \newline

\textbf{Ghana Paata } \newline

1. ए॒वैन॑ मेन मे॒वैवैन॒ मवावै॑न मे॒वैवैन॒ मव॑ । \newline
2. ए॒न॒ मवावै॑न मेन॒ मव॑ रुन्धे रु॒न्धे ऽवै॑न मेन॒ मव॑ रुन्धे । \newline
3. अव॑ रुन्धे रु॒न्धे ऽवाव॑ रुन्धे वीर॒हा वी॑र॒हा रु॒न्धे ऽवाव॑ रुन्धे वीर॒हा । \newline
4. रु॒न्धे॒ वी॒र॒हा वी॑र॒हा रु॑न्धे रुन्धे वीर॒हा वै वै वी॑र॒हा रु॑न्धे रुन्धे वीर॒हा वै । \newline
5. वी॒र॒हा वै वै वी॑र॒हा वी॑र॒हा वा ए॒ष ए॒ष वै वी॑र॒हा वी॑र॒हा वा ए॒षः । \newline
6. वी॒र॒हेति॑ वीर - हा । \newline
7. वा ए॒ष ए॒ष वै वा ए॒ष दे॒वाना᳚म् दे॒वाना॑ मे॒ष वै वा ए॒ष दे॒वाना᳚म् । \newline
8. ए॒ष दे॒वाना᳚म् दे॒वाना॑ मे॒ष ए॒ष दे॒वानां॒ ॅयो यो दे॒वाना॑ मे॒ष ए॒ष दे॒वानां॒ ॅयः । \newline
9. दे॒वानां॒ ॅयो यो दे॒वाना᳚म् दे॒वानां॒ ॅयो᳚ ऽग्नि म॒ग्निं ॅयो दे॒वाना᳚म् दे॒वानां॒ ॅयो᳚ ऽग्निम् । \newline
10. यो᳚ ऽग्नि म॒ग्निं ॅयो यो᳚ ऽग्नि मु॑द्वा॒सय॑त उद्वा॒सय॑ते॒ ऽग्निं ॅयो यो᳚ ऽग्नि मु॑द्वा॒सय॑ते । \newline
11. अ॒ग्नि मु॑द्वा॒सय॑त उद्वा॒सय॑ते॒ ऽग्नि म॒ग्नि मु॑द्वा॒सय॑ते॒ तस्य॒ तस्यो᳚द्वा॒सय॑ते॒ ऽग्नि म॒ग्नि मु॑द्वा॒सय॑ते॒ तस्य॑ । \newline
12. उ॒द्वा॒सय॑ते॒ तस्य॒ तस्यो᳚द्वा॒सय॑त उद्वा॒सय॑ते॒ तस्य॒ वरु॑णो॒ वरु॑ण॒ स्तस्यो᳚द्वा॒सय॑त उद्वा॒सय॑ते॒ तस्य॒ वरु॑णः । \newline
13. उ॒द्वा॒सय॑त॒ इत्यु॑त् - वा॒सय॑ते । \newline
14. तस्य॒ वरु॑णो॒ वरु॑ण॒स्तस्य॒ तस्य॒ वरु॑ण ए॒वैव वरु॑ण॒स्तस्य॒ तस्य॒ वरु॑ण ए॒व । \newline
15. वरु॑ण ए॒वैव वरु॑णो॒ वरु॑ण ए॒व र्‌ण॒या दृ॑ण॒यादे॒व वरु॑णो॒ वरु॑ण ए॒व र्‌ण॒यात् । \newline
16. ए॒व र्‌ण॒या दृ॑ण॒यादे॒वैव र्‌ण॒यादा᳚ग्निवारु॒ण मा᳚ग्निवारु॒ण मृ॑ण॒यादे॒वैव र्‌ण॒यादा᳚ग्निवारु॒णम् । \newline
17. ऋ॒ण॒या दा᳚ग्निवारु॒ण मा᳚ग्निवारु॒ण मृ॑ण॒या दृ॑ण॒यादा᳚ग्निवारु॒ण मेका॑दशकपाल॒ मेका॑दशकपाल माग्निवारु॒ण मृ॑ण॒यादृ॑ण॒यादा᳚ग्निवारु॒ण मेका॑दशकपालम् । \newline
18. ऋ॒ण॒यादित्यृ॑ण - यात् । \newline
19. आ॒ग्नि॒वा॒रु॒ण मेका॑दशकपाल॒ मेका॑दशकपाल माग्निवारु॒ण मा᳚ग्निवारु॒ण मेका॑दशकपाल॒ मन्वन्वेका॑दशकपाल माग्निवारु॒ण मा᳚ग्निवारु॒ण मेका॑दशकपाल॒ मनु॑ । \newline
20. आ॒ग्नि॒वा॒रु॒णमित्या᳚ग्नि - वा॒रु॒णम् । \newline
21. एका॑दशकपाल॒ मन्वन्वेका॑दशकपाल॒ मेका॑दशकपाल॒ मनु॒ निर् णिरन्वेका॑दशकपाल॒ मेका॑दशकपाल॒ मनु॒ निः । \newline
22. एका॑दशकपाल॒मित्येका॑दश - क॒पा॒ल॒म् । \newline
23. अनु॒ निर् णिरन्वनु॒ निर् व॑पेद् वपे॒न् निरन्वनु॒ निर् व॑पेत् । \newline
24. निर् व॑पेद् वपे॒न् निर् णिर् व॑पे॒द् यं ॅयं ॅव॑पे॒न् निर् णिर् व॑पे॒द् यम् । \newline
25. व॒पे॒द् यं ॅयं ॅव॑पेद् वपे॒द् यम् च॑ च॒ यं ॅव॑पेद् वपे॒द् यम् च॑ । \newline
26. यम् च॑ च॒ यं ॅयम् चै॒वैव च॒ यं ॅयम् चै॒व । \newline
27. चै॒वैव च॑ चै॒व हन्ति॒ हन्त्ये॒व च॑ चै॒व हन्ति॑ । \newline
28. ए॒व हन्ति॒ हन्त्ये॒वैव हन्ति॒ यो यो हन्त्ये॒वैव हन्ति॒ यः । \newline
29. हन्ति॒ यो यो हन्ति॒ हन्ति॒ यश्च॑ च॒ यो हन्ति॒ हन्ति॒ यश्च॑ । \newline
30. यश्च॑ च॒ यो यश्चा᳚स्यास्य च॒ यो यश्चा᳚स्य । \newline
31. चा॒स्या॒स्य॒ च॒ चा॒स्य॒ र्‌ण॒या दृ॑ण॒याद॑स्य च चास्य र्‌ण॒यात् । \newline
32. अ॒स्य॒ र्‌ण॒या दृ॑ण॒याद॑स्यास्य र्‌ण॒यात् तौ ता वृ॑ण॒याद॑स्यास्य र्‌ण॒यात् तौ । \newline
33. ऋ॒ण॒यात् तौ ता वृ॑ण॒यादृ॑ण॒यात् तौ भा॑ग॒धेये॑न भाग॒धेये॑न॒ ता वृ॑ण॒यादृ॑ण॒यात् तौ भा॑ग॒धेये॑न । \newline
34. ऋ॒ण॒यादित्यृ॑ण - यात् । \newline
35. तौ भा॑ग॒धेये॑न भाग॒धेये॑न॒ तौ तौ भा॑ग॒धेये॑न प्रीणाति प्रीणाति भाग॒धेये॑न॒ तौ तौ भा॑ग॒धेये॑न प्रीणाति । \newline
36. भा॒ग॒धेये॑न प्रीणाति प्रीणाति भाग॒धेये॑न भाग॒धेये॑न प्रीणाति॒ न न प्री॑णाति भाग॒धेये॑न भाग॒धेये॑न प्रीणाति॒ न । \newline
37. भा॒ग॒धेये॒नेति॑ भाग - धेये॑न । \newline
38. प्री॒णा॒ति॒ न न प्री॑णाति प्रीणाति॒ नार्ति॒ मार्ति॒न्न प्री॑णाति प्रीणाति॒ नार्ति᳚म् । \newline
39. नार्ति॒ मार्ति॒न्न नार्ति॒ मा ऽऽर्ति॒न्न नार्ति॒ मा । \newline
40. आर्ति॒ मा ऽऽर्ति॒ मार्ति॒ मा र्च्छ॑त्यृच्छ॒त्या ऽऽर्ति॒ मार्ति॒ मा र्‌च्छ॑ति । \newline
41. आ र्‌च्छ॑ त्यृच्छ॒ त्यार्‌च्छ॑ति॒ यज॑मानो॒ यज॑मान ऋच्छ॒त्या र्‌च्छ॑ति॒ यज॑मानः । \newline
42. ऋ॒च्छ॒ति॒ यज॑मानो॒ यज॑मान ऋच्छत्यृच्छति॒ यज॑मानः । \newline
43. यज॑मान॒ इति॒ यज॑मानः । \newline
\pagebreak
\markright{ TS 1.5.3.1  \hfill https://www.vedavms.in \hfill}
\addcontentsline{toc}{section}{ TS 1.5.3.1 }
\section*{ TS 1.5.3.1 }

\textbf{TS 1.5.3.1 } \newline
\textbf{Samhita Paata} \newline

भूमि॑र् भू॒म्ना द्यौर् व॑रि॒णाऽन्तरि॑क्षं महि॒त्वा । उ॒पस्थे॑ ते देव्यदिते॒ ऽग्निम॑न्ना॒दम॒न्नाद्या॒याऽऽद॑धे ॥आऽयं गौः पृश्ञि॑रक्रमी॒दस॑नन् मा॒तरं॒ पुनः॑ । पि॒तरं॑ च प्र॒यन्थ् सुवः॑ ॥ त्रिꣳ॒॒शद्धाम॒ वि रा॑जति॒ वाक् प॑त॒ङ्गाय॑ शिश्रिये । प्रत्य॑स्य वह॒ द्युभिः॑ ॥ अ॒स्य प्रा॒णाद॑पान॒त्य॑न्तश्च॑रति रोच॒ना । व्य॑ख्यन् महि॒षः सुवः॑ ॥ यत्त्वा᳚-[ ] \newline

\textbf{Pada Paata} \newline

भूमिः॑ । भू॒म्ना । द्यौः । व॒रि॒णा । अ॒न्तरि॑क्षम् । म॒हि॒त्वेति॑ महि-त्वा ॥ उ॒पस्थ॒ इत्यु॒प - स्थे॒ । ते॒ । दे॒वि॒ । अ॒दि॒ते॒ । अ॒ग्निम् । अ॒न्ना॒दमित्य॑न्न - अ॒दम् । अ॒न्नाद्या॒येत्य॑न्न - अद्या॑य । एति॑ । द॒धे॒ ॥ एति॑ । अ॒यम् । गौः । पृश्निः॑ । अ॒क्र॒मी॒त् । अस॑नत् । मा॒तर᳚म् । पुनः॑ ॥ पि॒तर᳚म् । च॒ । प्र॒यन्निति॑ प्र - यन्न् । सुवः॑ ॥ त्रिꣳ॒॒शत् । धाम॑ । वीति॑ । रा॒ज॒ति॒ । वाक् । प॒त॒ङ्गाय॑ । शि॒श्रि॒ये॒ ॥ प्रतीति॑ । अ॒स्य॒ । व॒ह॒ । द्युभि॒रिति॒ द्यु - भिः॒ । अ॒स्य । प्रा॒णादिति॑ प्र - अ॒नात् । अ॒पा॒न॒तीत्य॑प - अ॒न॒ती । अ॒न्तः । च॒र॒ति॒ । रो॒च॒ना ॥ वीति॑ । अ॒ख्य॒त् । म॒हि॒षः । सुवः॑ ॥ यत् । त्वा॒ ।  \newline


\textbf{Krama Paata} \newline

भूमि॑र् भू॒म्ना । भू॒म्ना द्यौः । द्यौर्,व॑रि॒णा । व॒रि॒णाऽन्तरि॑क्षम् । अ॒न्तरि॑क्षम् महि॒त्वा । म॒हि॒त्वेति॑ महि - त्वा ॥ उ॒पस्थे॑ ते । उ॒पस्थ॒ इत्यु॒प - स्थे॒ । ते॒ दे॒वि॒ । दे॒व्य॒दि॒ते॒ । अ॒दि॒ते॒ऽग्निम् । अ॒ग्निम॑न्ना॒दम् । अ॒न्ना॒दम॒न्नाद्या॑य । अ॒न्ना॒दमित्य॑न्न - अ॒दम् । अ॒न्नाद्या॒या । अ॒न्नाद्या॒येत्य॑न्न - अद्या॑य । आ द॑धे । द॒ध॒ इति॑ दधे ॥ आऽयम् । अ॒यम् गौः । गौः पृश्ञिः॑ । पृश्ञि॑रक्रमीत् । अ॒क्र॒मी॒दस॑नत् । अस॑नन्,मा॒तर᳚म् । मा॒तर॒म् पुनः॑ । पुन॒रिति॒ पुनः॑ ॥ पि॒तर॑म् च । च॒ प्र॒यन्न् । प्र॒यन्थ् सुवः॑ । प्र॒यन्निति॑ प्र - यन्न् । सुव॒रिति॒ सुवः॑ ॥ त्रिꣳ॒॒शद्धाम॑ । धाम॒ वि । वि रा॑जति । रा॒ज॒ति॒ वाक् । वाक् प॑त॒ङ्गाय॑ । प॒त॒ङ्गाय॑ शिश्रिये । शि॒श्रि॒य॒ इति॑ शिश्रिये ॥ प्रत्य॑स्य । अ॒स्य॒ व॒ह॒ । व॒ह॒ द्युभिः॑ । द्युभि॒रिति॒ द्यु - भिः॒ ॥ अ॒स्य प्रा॒णात् । प्रा॒णाद॑पान॒ती । प्रा॒णादिति॑ प्र - अ॒नात् । अ॒पा॒न॒त्य॑न्तः । अ॒पा॒न॒तीत्य॑प - अ॒न॒ती । अ॒न्तश्च॑रति । च॒र॒ति॒ रो॒च॒ना । रो॒च॒नेति॑ रोच॒ना ॥ व्य॑ख्यत् । अ॒ख्य॒न्,म॒हि॒षः । म॒हि॒षः सुवः॑ । सुव॒रिति॒ सुवः॑ ॥ यत् त्वा᳚ । त्वा॒ क्रु॒द्धः \newline

\textbf{Jatai Paata} \newline

1. भूमि॑र् भू॒म्ना भू॒म्ना भूमि॒र् भूमि॑र् भू॒म्ना । \newline
2. भू॒म्ना द्यौर् द्यौर् भू॒म्ना भू॒म्ना द्यौः । \newline
3. द्यौर् व॑रि॒णा व॑रि॒णा द्यौर् द्यौर् व॑रि॒णा । \newline
4. व॒रि॒णा ऽन्तरि॑क्ष म॒न्तरि॑क्षं ॅवरि॒णा व॑रि॒णा ऽन्तरि॑क्षम् । \newline
5. अ॒न्तरि॑क्षम् महि॒त्वा म॑हि॒त्वा ऽन्तरि॑क्ष म॒न्तरि॑क्षम् महि॒त्वा । \newline
6. म॒हि॒त्वेति॑ महि - त्वा । \newline
7. उ॒पस्थे॑ ते त उ॒पस्थ॑ उ॒पस्थे॑ ते । \newline
8. उ॒पस्थ॒ इत्यु॒प - स्थे॒ । \newline
9. ते॒ दे॒वि॒ दे॒वि॒ ते॒ ते॒ दे॒वि॒ । \newline
10. दे॒व्य॒दि॒ते॒ ऽदि॒ते॒ दे॒वि॒ दे॒व्य॒दि॒ते॒ । \newline
11. अ॒दि॒ते॒ ऽग्नि म॒ग्नि म॑दिते ऽदिते॒ ऽग्निम् । \newline
12. अ॒ग्नि म॑न्ना॒द म॑न्ना॒द म॒ग्नि म॒ग्नि म॑न्ना॒दम् । \newline
13. अ॒न्ना॒द म॒न्नाद्या॑या॒ न्नाद्या॑या न्ना॒द म॑न्ना॒द म॒न्नाद्या॑य । \newline
14. अ॒न्ना॒दमित्य॑न्न - अ॒दम् । \newline
15. अ॒न्नाद्या॒या ऽन्नाद्या॑या॒ न्नाद्या॒या । \newline
16. अ॒न्नाद्या॒येत्य॑न्न - अद्या॑य । \newline
17. आ द॑धे दध॒ आ द॑धे । \newline
18. द॒ध॒ इति॑ दधे । \newline
19. आ ऽय म॒य मा ऽयम् । \newline
20. अ॒यम् गौर् गौर॒य म॒यम् गौः । \newline
21. गौः पृश्ञिः॒ पृश्ञि॒र् गौर् गौः पृश्ञिः॑ । \newline
22. पृश्ञि॑ रक्रमी दक्रमी॒त् पृश्ञिः॒ पृश्ञि॑ रक्रमीत् । \newline
23. अ॒क्र॒मी॒ दस॑न॒ दस॑न दक्रमी दक्रमी॒ दस॑नत् । \newline
24. अस॑नन् मा॒तर॑म् मा॒तर॒ मस॑न॒ दस॑नन् मा॒तर᳚म् । \newline
25. मा॒तर॒म् पुनः॒ पुन॑र् मा॒तर॑म् मा॒तर॒म् पुनः॑ । \newline
26. पुन॒रिति॒ पुनः॑ । \newline
27. पि॒तर॑म् च च पि॒तर॑म् पि॒तर॑म् च । \newline
28. च॒ प्र॒यन् प्र॒यꣳश्च॑ च प्र॒यन्न् । \newline
29. प्र॒यन् थ्सुवः॒ सुवः॑ प्र॒यन् प्र॒यन् थ्सुवः॑ । \newline
30. प्र॒यन्निति॑ प्र - यन्न् । \newline
31. सुव॒रिति॒ सुवः॑ । \newline
32. त्रि॒(ग्म्॒)शद् धाम॒ धाम॑ त्रि॒(ग्म्॒)शत् त्रि॒(ग्म्॒)शद् धाम॑ । \newline
33. धाम॒ वि वि धाम॒ धाम॒ वि । \newline
34. वि रा॑जति राजति॒ वि वि रा॑जति । \newline
35. रा॒ज॒ति॒ वाग् वाग् रा॑जति राजति॒ वाक् । \newline
36. वाक् प॑त॒ङ्गाय॑ पत॒ङ्गाय॒ वाग् वाक् प॑त॒ङ्गाय॑ । \newline
37. प॒त॒ङ्गाय॑ शिश्रिये शिश्रिये पत॒ङ्गाय॑ पत॒ङ्गाय॑ शिश्रिये । \newline
38. शि॒श्रि॒य॒ इति॑ शिश्रिये । \newline
39. प्रत्य॑स्यास्य॒ प्रति॒ प्रत्य॑स्य । \newline
40. अ॒स्य॒ व॒ह॒ व॒हा॒स्या॒स्य॒ व॒ह॒ । \newline
41. व॒ह॒ द्युभि॒र् द्युभि॑र् वह वह॒ द्युभिः॑ । \newline
42. द्युभि॒रिति॒ द्यु - भिः॒ । \newline
43. अ॒स्य प्रा॒णात् प्रा॒णा द॒स्यास्य प्रा॒णात् । \newline
44. प्रा॒णा द॑पान॒त्य॑पान॒ती प्रा॒णात् प्रा॒णाद॑पान॒ती । \newline
45. प्रा॒णादिति॑ प्र - अ॒नात् । \newline
46. अ॒पा॒न॒त्य॑न्त र॒न्त र॑पान॒त्य॑पान॒त्य॑न्तः । \newline
47. अ॒पा॒न॒तीत्य॑प - अ॒न॒ती । \newline
48. अ॒न्तश्च॑रति चरत्य॒न्त र॒न्तश्च॑रति । \newline
49. च॒र॒ति॒ रो॒च॒ना रो॑च॒ना च॑रति चरति रोच॒ना । \newline
50. रो॒च॒नेति॑ रोच॒ना । \newline
51. व्य॑ख्यदख्य॒द् वि व्य॑ख्यत् । \newline
52. अ॒ख्य॒न् म॒हि॒षो म॑हि॒षो᳚ ऽख्यदख्यन् महि॒षः । \newline
53. म॒हि॒षः सुवः॒ सुव॑र् महि॒षो म॑हि॒षः सुवः॑ । \newline
54. सुव॒रिति॒ सुवः॑ । \newline
55. यत् त्वा᳚ त्वा॒ यद् यत् त्वा᳚ । \newline
56. त्वा॒ क्रु॒द्धः क्रु॒द्धस्त्वा᳚ त्वा क्रु॒द्धः । \newline

\textbf{Ghana Paata } \newline

1. भूमि॑र् भू॒म्ना भू॒म्ना भूमि॒र् भूमि॑र् भू॒म्ना द्यौर् द्यौर् भू॒म्ना भूमि॒र् भूमि॑र् भू॒म्ना द्यौः । \newline
2. भू॒म्ना द्यौर् द्यौर् भू॒म्ना भू॒म्ना द्यौर् व॑रि॒णा व॑रि॒णा द्यौर् भू॒म्ना भू॒म्ना द्यौर् व॑रि॒णा । \newline
3. द्यौर् व॑रि॒णा व॑रि॒णा द्यौर् द्यौर् व॑रि॒णा ऽन्तरि॑क्ष म॒न्तरि॑क्षं ॅवरि॒णा द्यौर् द्यौर् व॑रि॒णा ऽन्तरि॑क्षम् । \newline
4. व॒रि॒णा ऽन्तरि॑क्ष म॒न्तरि॑क्षं ॅवरि॒णा व॑रि॒णा ऽन्तरि॑क्षम् महि॒त्वा म॑हि॒त्वा ऽन्तरि॑क्षं ॅवरि॒णा व॑रि॒णा ऽन्तरि॑क्षम् महि॒त्वा । \newline
5. अ॒न्तरि॑क्षम् महि॒त्वा म॑हि॒त्वा ऽन्तरि॑क्ष म॒न्तरि॑क्षम् महि॒त्वा । \newline
6. म॒हि॒त्वेति॑ महि - त्वा । \newline
7. उ॒पस्थे॑ ते त उ॒पस्थ॑ उ॒पस्थे॑ ते देवि देवि त उ॒पस्थ॑ उ॒पस्थे॑ ते देवि । \newline
8. उ॒पस्थ॒ इत्यु॒प - स्थे॒ । \newline
9. ते॒ दे॒वि॒ दे॒वि॒ ते॒ ते॒ दे॒व्य॒दि॒ते॒ ऽदि॒ते॒ दे॒वि॒ ते॒ ते॒ दे॒व्य॒दि॒ते॒ । \newline
10. दे॒व्य॒दि॒ते॒ ऽदि॒ते॒ दे॒वि॒ दे॒व्य॒दि॒ते॒ ऽग्नि म॒ग्नि म॑दिते देवि देव्यदिते॒ ऽग्निम् । \newline
11. अ॒दि॒ते॒ ऽग्नि म॒ग्नि म॑दिते ऽदिते॒ ऽग्नि म॑न्ना॒द म॑न्ना॒द म॒ग्नि म॑दिते ऽदिते॒ ऽग्नि म॑न्ना॒दम् । \newline
12. अ॒ग्नि म॑न्ना॒द म॑न्ना॒द म॒ग्नि म॒ग्नि म॑न्ना॒द म॒न्नाद्या॑या॒ न्नाद्या॑या न्ना॒द म॒ग्नि म॒ग्नि म॑न्ना॒द म॒न्नाद्या॑य । \newline
13. अ॒न्ना॒द म॒न्नाद्या॑या॒ न्नाद्या॑या न्ना॒द म॑न्ना॒द म॒न्नाद्या॒या ऽन्नाद्या॑या न्ना॒द म॑न्ना॒द म॒न्नाद्या॒॒या । \newline
14. अ॒न्ना॒दमित्य॑न्न - अ॒दम् । \newline
15. अ॒न्नाद्या॒या ऽन्नाद्या॑या॒ न्नाद्या॒या द॑धे दध॒ आ ऽन्नाद्या॑या॒ न्नाद्या॒या द॑धे । \newline
16. अ॒न्नाद्या॒येत्य॑न्न - अद्या॑य । \newline
17. आ द॑धे दध॒ आ द॑धे । \newline
18. द॒ध॒ इति॑ दधे । \newline
19. आ ऽय म॒य मा ऽयम् गौर् गौर॒य मा ऽयम् गौः । \newline
20. अ॒यम् गौर् गौर॒य म॒यम् गौः पृश्ञिः॒ पृश्ञि॒र् गौर॒य म॒यम् गौः पृश्ञिः॑ । \newline
21. गौः पृश्ञिः॒ पृश्ञि॒र् गौर् गौः पृश्ञि॑ रक्रमी दक्रमी॒त् पृश्ञि॒र् गौर् गौः पृश्ञि॑रक्रमीत् । \newline
22. पृश्ञि॑ रक्रमीदक्रमी॒त् पृश्ञिः॒ पृश्ञि॑ रक्रमी॒ दस॑न॒दस॑न दक्रमी॒त् पृश्ञिः॒ पृश्ञि॑ रक्रमी॒दस॑नत् । \newline
23. अ॒क्र॒मी॒ दस॑न॒ दस॑न दक्रमीदक्रमी॒ दस॑नन् मा॒तर॑म् मा॒तर॒ मस॑न दक्रमी दक्रमी॒ दस॑नन् मा॒तर᳚म् । \newline
24. अस॑नन् मा॒तर॑म् मा॒तर॒ मस॑न॒दस॑नन् मा॒तर॒म् पुनः॒ पुन॑र् मा॒तर॒ मस॑न॒ दस॑नन् मा॒तर॒म् पुनः॑ । \newline
25. मा॒तर॒म् पुनः॒ पुन॑र् मा॒तर॑म् मा॒तर॒म् पुनः॑ । \newline
26. पुन॒रिति॒ पुनः॑ । \newline
27. पि॒तर॑म् च च पि॒तर॑म् पि॒तर॑म् च प्र॒यन् प्र॒यꣳश्च॑ पि॒तर॑म् पि॒तर॑म् च प्र॒यन्न् । \newline
28. च॒ प्र॒यन् प्र॒यꣳश्च॑ च प्र॒यन् थ्सुवः॒ सुवः॑ प्र॒यꣳश्च॑ च प्र॒यन् थ्सुवः॑ । \newline
29. प्र॒यन् थ्सुवः॒ सुवः॑ प्र॒यन् प्र॒यन् थ्सुवः॑ । \newline
30. प्र॒यन्निति॑ प्र - यन्न् । \newline
31. सुव॒रिति॒ सुवः॑ । \newline
32. त्रि॒(ग्म्॒)शद् धाम॒ धाम॑ त्रि॒(ग्म्॒)शत् त्रि॒(ग्म्॒)शद् धाम॒ वि वि धाम॑ त्रि॒(ग्म्॒)शत् त्रि॒(ग्म्॒)शद् धाम॒ वि । \newline
33. धाम॒ वि वि धाम॒ धाम॒ वि रा॑जति राजति॒ वि धाम॒ धाम॒ वि रा॑जति । \newline
34. वि रा॑जति राजति॒ वि वि रा॑जति॒ वाग् वाग् रा॑जति॒ वि वि रा॑जति॒ वाक् । \newline
35. रा॒ज॒ति॒ वाग् वाग् रा॑जति राजति॒ वाक् प॑त॒ङ्गाय॑ पत॒ङ्गाय॒ वाग् रा॑जति राजति॒ वाक् प॑त॒ङ्गाय॑ । \newline
36. वाक् प॑त॒ङ्गाय॑ पत॒ङ्गाय॒ वाग् वाक् प॑त॒ङ्गाय॑ शिश्रिये शिश्रिये पत॒ङ्गाय॒ वाग् वाक् प॑त॒ङ्गाय॑ शिश्रिये । \newline
37. प॒त॒ङ्गाय॑ शिश्रिये शिश्रिये पत॒ङ्गाय॑ पत॒ङ्गाय॑ शिश्रिये । \newline
38. शि॒श्रि॒य॒ इति॑ शिश्रिये । \newline
39. प्रत्य॑स्यास्य॒ प्रति॒ प्रत्य॑स्य वह वहास्य॒ प्रति॒ प्रत्य॑स्य वह । \newline
40. अ॒स्य॒ व॒ह॒ व॒हा॒स्या॒स्य॒ व॒ह॒ द्युभि॒र् द्युभि॑र् वहास्यास्य वह॒ द्युभिः॑ । \newline
41. व॒ह॒ द्युभि॒र् द्युभि॑र् वह वह॒ द्युभिः॑ । \newline
42. द्युभि॒रिति॒ द्यु - भिः॒ । \newline
43. अ॒स्य प्रा॒णात् प्रा॒णाद॒स्यास्य प्रा॒णा द॑पान॒त्य॑पान॒ती प्रा॒णाद॒स्यास्य प्रा॒णाद॑पान॒ती । \newline
44. प्रा॒णा द॑पान॒त्य॑पान॒ती प्रा॒णात् प्रा॒णाद॑पान॒ त्य॑न्त र॒न्त र॑पान॒ती प्रा॒णात् प्रा॒णाद॑पान॒त्य॑न्तः । \newline
45. प्रा॒णादिति॑ प्र - अ॒नात् । \newline
46. अ॒पा॒न॒त्य॑न्त र॒न्त र॑पान॒ त्य॑पान॒त्य॑न्त श्च॑रति चरत्य॒न्त र॑पान॒ त्य॑पान॒त्य॑न्त श्च॑रति । \newline
47. अ॒पा॒न॒तीत्य॑प - अ॒न॒ती । \newline
48. अ॒न्तश्च॑रति चरत्य॒न्त र॒न्तश्च॑रति रोच॒ना रो॑च॒ना च॑रत्य॒न्त र॒न्तश्च॑रति रोच॒ना । \newline
49. च॒र॒ति॒ रो॒च॒ना रो॑च॒ना च॑रति चरति रोच॒ना । \newline
50. रो॒च॒नेति॑ रोच॒ना । \newline
51. व्य॑ख्यदख्य॒द् वि व्य॑ख्यन् महि॒षो म॑हि॒षो᳚ ऽख्य॒द् वि व्य॑ख्यन् महि॒षः । \newline
52. अ॒ख्य॒न् म॒हि॒षो म॑हि॒षो᳚ ऽख्यदख्यन् महि॒षः सुवः॒ सुव॑र् महि॒षो᳚ ऽख्यदख्यन् महि॒षः सुवः॑ । \newline
53. म॒हि॒षः सुवः॒ सुव॑र् महि॒षो म॑हि॒षः सुवः॑ । \newline
54. सुव॒रिति॒ सुवः॑ । \newline
55. यत् त्वा᳚ त्वा॒ यद् यत् त्वा᳚ क्रु॒द्धः क्रु॒द्धस्त्वा॒ यद् यत् त्वा᳚ क्रु॒द्धः । \newline
56. त्वा॒ क्रु॒द्धः क्रु॒द्धस्त्वा᳚ त्वा क्रु॒द्धः प॑रो॒वप॑ परो॒वप॑ क्रु॒द्धस्त्वा᳚ त्वा क्रु॒द्धः प॑रो॒वप॑ । \newline
\pagebreak
\markright{ TS 1.5.3.2  \hfill https://www.vedavms.in \hfill}
\addcontentsline{toc}{section}{ TS 1.5.3.2 }
\section*{ TS 1.5.3.2 }

\textbf{TS 1.5.3.2 } \newline
\textbf{Samhita Paata} \newline

क्रु॒द्धः प॑रो॒वप॑ म॒न्युना॒ यदव॑र्त्या । सु॒कल्प॑मग्ने॒ तत्तव॒ पुन॒स्त्वोद्दी॑पयामसि ॥यत्ते॑ म॒न्युप॑रोप्तस्य पृथि॒वीमनु॑ दद्ध्व॒से । आ॒दि॒त्या विश्वे॒ तद्दे॒वा वस॑वश्च स॒माभ॑रन्न् ॥मनो॒ ज्योति॑र् जुषता॒माज्यं॒ ॅविच्छि॑न्नं ॅय॒ज्ञ्ꣳ समि॒मं द॑धातु । बृह॒स्पति॑स्तनुतामि॒मं नो॒ विश्वे॑ दे॒वा इ॒ह मा॑दयन्तां ॥ स॒प्त ते॑ अग्ने स॒मिधः॑ स॒प्त जि॒ह्वाः स॒प्त - [ ] \newline

\textbf{Pada Paata} \newline

क्रु॒द्धः । प॒रो॒वपेति॑ परा - उ॒पव॑ । म॒न्युना᳚ । यत् । अव॑र्त्या ॥ सु॒कल्प॒मिति॑ सु - कल्प᳚म् । अ॒ग्ने॒ । तत् । तव॑ । पुनः॑ । त्वा॒ । उदिति॑ । दी॒प॒या॒म॒सि॒ ॥ यत् । ते॒ । म॒न्युप॑रोप्त॒स्येति॑ म॒न्यु-प॒रो॒प्त॒स्य॒ । पृ॒थि॒वीम् । अन्विति॑ । द॒द्ध्व॒से ॥ आ॒दि॒त्याः । विश्वे᳚ । तत् । दे॒वाः । वस॑वः । च॒ । स॒माभ॑र॒न्निति॑ सं - आभ॑रन्न् ॥ मनः॑ । ज्योतिः॑ । जु॒ष॒ता॒म् । आज्य᳚म् । विच्छि॑न्न॒मिति॒ वि - छि॒न्न॒म् । य॒ज्ञ्म् । समिति॑ । इ॒मम् । द॒धा॒तु॒ ॥ बृह॒स्पतिः॑ । त॒नु॒ता॒म् । इ॒मम् । नः॒ । विश्वे᳚ । दे॒वाः । इ॒ह । मा॒द॒य॒न्ता॒म् ॥ स॒प्त । ते॒ । अ॒ग्ने॒ । स॒मिध॒ इति॑ सं - इधः॑ । स॒प्त । जि॒ह्वाः । स॒प्त ।  \newline


\textbf{Krama Paata} \newline

क्रु॒द्धः प॑रो॒वप॑ । प॒रो॒वप॑ म॒न्युना᳚ । प॒रो॒वपेति॑ परा - उ॒वप॑ । म॒न्युना॒ यत् । यदव॑र्त्या । अव॒र्त्येत्यव॑र्त्या ॥ सु॒कल्प॑मग्ने । सु॒कल्प॒मिति॑ सु - कल्प᳚म् । अ॒ग्ने॒ तत् । तत् तव॑ । तव॒ पुनः॑ । पुन॑स्त्वा । त्वोत् । उद् दी॑पयामसि । दी॒प॒या॒म॒सीति॑ दीपयामसि ॥ यत् ते᳚ । ते॒ म॒न्युप॑रोप्तस्य । म॒न्युप॑रोप्तस्य पृथि॒वीम् । म॒न्युप॑रोप्त॒स्येति॑ म॒न्यु - प॒रो॒प्त॒स्य॒ । पृ॒थि॒वीमनु॑ । अनु॑ दद्ध्व॒से । द॒द्ध्व॒स इति॑ दद्ध्व॒से ॥ आ॒दि॒त्या विश्वे᳚ । विश्वे॒ तत् । तद् दे॒वाः । दे॒वा वस॑वः । वस॑वश्च । च॒ स॒माभ॑रन्न् । स॒माभ॑र॒न्निति॑ सम् - आभ॑रन्न् ॥ मनो॒ ज्योतिः॑ । ज्योति॑र्,जुषताम् । जु॒ष॒ता॒माज्य᳚म् । आज्यं॒ ॅविच्छि॑न्नम् । विच्छि॑न्नं ॅय॒ज्ञ्म् । विच्छि॑न्न॒मिति॒ वि - छि॒न्न॒म् । य॒ज्ञ्ꣳ सम् । समि॒मम् । इ॒मम् द॑धातु । द॒धा॒त्विति॑ दधातु ॥ बृह॒स्पति॑स्तनुताम् । त॒नु॒ता॒मि॒मम् । इ॒मम् नः॑ । नो॒ विश्वे᳚ । विश्वे॑ दे॒वाः । दे॒वा इ॒ह । इ॒ह मा॑दयन्ताम् । मा॒द॒य॒न्ता॒मिति॑ मादयन्ताम् ॥ स॒प्त ते᳚ । ते॒ अ॒ग्ने॒ । अ॒ग्ने॒ स॒मिधः॑ । स॒मिधः॑ स॒प्त । स॒मिध॒ इति॑ सम् - इधः॑ । स॒प्त जि॒ह्वाः । जि॒ह्वाः स॒प्त । स॒प्तर्.ष॑यः \newline

\textbf{Jatai Paata} \newline

1. क्रु॒द्धः प॑रो॒वप॑ परो॒वप॑ क्रु॒द्धः क्रु॒द्धः प॑रो॒वप॑ । \newline
2. प॒रो॒वप॑ म॒न्युना॑ म॒न्युना॑ परो॒वप॑ परो॒वप॑ म॒न्युना᳚ । \newline
3. प॒रो॒वपेति॑ परा - उ॒पव॑ । \newline
4. म॒न्युना॒ यद् यन् म॒न्युना॑ म॒न्युना॒ यत् । \newline
5. यदव॒र्त्या ऽव॑र्त्या॒ यद् यदव॑र्त्या । \newline
6. अव॒र्त्येत्यव॑र्त्या । \newline
7. सु॒कल्प॑ मग्ने ऽग्ने सु॒कल्प(ग्म्॑) सु॒कल्प॑ मग्ने । \newline
8. सु॒कल्प॒मिति॑ सु - कल्प᳚म् । \newline
9. अ॒ग्ने॒ तत् तद॑ग्ने ऽग्ने॒ तत् । \newline
10. तत् तव॒ तव॒ तत् तत् तव॑ । \newline
11. तव॒ पुनः॒ पुन॒स्तव॒ तव॒ पुनः॑ । \newline
12. पुन॑स्त्वा त्वा॒ पुनः॒ पुन॑स्त्वा । \newline
13. त्वोदुत् त्वा॒ त्वोत् । \newline
14. उद् दी॑पयामसि दीपयाम॒स्युदुद् दी॑पयामसि । \newline
15. दी॒प॒या॒म॒सीति॑ दीपयामसि । \newline
16. यत् ते॑ ते॒ यद् यत् ते᳚ । \newline
17. ते॒ म॒न्युप॑रोप्तस्य म॒न्युप॑रोप्तस्य ते ते म॒न्युप॑रोप्तस्य । \newline
18. म॒न्युप॑रोप्तस्य पृथि॒वीम् पृ॑थि॒वीम् म॒न्युप॑रोप्तस्य म॒न्युप॑रोप्तस्य पृथि॒वीम् । \newline
19. म॒न्युप॑रोप्त॒स्येति॑ म॒न्यु - प॒रो॒प्त॒स्य॒ । \newline
20. पृ॒थि॒वी मन्वनु॑ पृथि॒वीम् पृ॑थि॒वी मनु॑ । \newline
21. अनु॑ दद्ध्व॒से द॑द्ध्व॒से ऽन्वनु॑ दद्ध्व॒से । \newline
22. द॒द्ध्व॒स इति॑ दद्ध्व॒से । \newline
23. आ॒दि॒त्या विश्वे॒ विश्व॑ आदि॒त्या आ॑दि॒त्या विश्वे᳚ । \newline
24. विश्वे॒ तत् तद् विश्वे॒ विश्वे॒ तत् । \newline
25. तद् दे॒वा दे॒वास्तत् तद् दे॒वाः । \newline
26. दे॒वा वस॑वो॒ वस॑वो दे॒वा दे॒वा वस॑वः । \newline
27. वस॑वश्च च॒ वस॑वो॒ वस॑वश्च । \newline
28. च॒ स॒माभ॑रन् थ्स॒माभ॑रꣳश्च च स॒माभ॑रन्न् । \newline
29. स॒माभ॑र॒न्निति॑ सं - आभ॑रन्न् । \newline
30. मनो॒ ज्योति॒र् ज्योति॒र् मनो॒ मनो॒ ज्योतिः॑ । \newline
31. ज्योति॑र् जुषताम् जुषता॒म् ज्योति॒र् ज्योति॑र् जुषताम् । \newline
32. जु॒ष॒ता॒ माज्य॒ माज्य॑म् जुषताम् जुषता॒ माज्य᳚म् । \newline
33. आज्यं॒ ॅविच्छि॑न्नं॒ ॅविच्छि॑न्न॒ माज्य॒ माज्यं॒ ॅविच्छि॑न्नम् । \newline
34. विच्छि॑न्नं ॅय॒ज्ञ्ं ॅय॒ज्ञ्ं ॅविच्छि॑न्नं॒ ॅविच्छि॑न्नं ॅय॒ज्ञ्म् । \newline
35. विच्छि॑न्न॒मिति॒ वि - छि॒न्न॒म् । \newline
36. य॒ज्ञ्ꣳ सꣳ सं ॅय॒ज्ञ्ं ॅय॒ज्ञ्ꣳ सम् । \newline
37. स मि॒म मि॒मꣳ सꣳ स मि॒मम् । \newline
38. इ॒मम् द॑धातु दधात्वि॒म मि॒मम् द॑धातु । \newline
39. द॒धा॒त्विति॑ दधातु । \newline
40. बृह॒स्पति॑ स्तनुताम् तनुता॒म् बृह॒स्पति॒र् बृह॒स्पति॑ स्तनुताम् । \newline
41. त॒नु॒ता॒ मि॒म मि॒मम् त॑नुताम् तनुता मि॒मम् । \newline
42. इ॒मन्नो॑ न इ॒म मि॒मन्नः॑ । \newline
43. नो॒ विश्वे॒ विश्वे॑ नो नो॒ विश्वे᳚ । \newline
44. विश्वे॑ दे॒वा दे॒वा विश्वे॒ विश्वे॑ दे॒वाः । \newline
45. दे॒वा इ॒हे ह दे॒वा दे॒वा इ॒ह । \newline
46. इ॒ह मा॑दयन्ताम् मादयन्ता मि॒हे ह मा॑दयन्ताम् । \newline
47. मा॒द॒य॒न्ता॒मिति॑ मादयन्ताम् । \newline
48. स॒प्त ते॑ ते स॒प्त स॒प्त ते᳚ । \newline
49. ते॒ अ॒ग्ने॒ ऽग्ने॒ ते॒ ते॒ अ॒ग्ने॒ । \newline
50. अ॒ग्ने॒ स॒मिधः॑ स॒मिधो᳚ ऽग्ने ऽग्ने स॒मिधः॑ । \newline
51. स॒मिधः॑ स॒प्त स॒प्त स॒मिधः॑ स॒मिधः॑ स॒प्त । \newline
52. स॒मिध॒ इति॑ सं - इधः॑ । \newline
53. स॒प्त जि॒ह्वा जि॒ह्वाः स॒प्त स॒प्त जि॒ह्वाः । \newline
54. जि॒ह्वाः स॒प्त स॒प्त जि॒ह्वा जि॒ह्वाः स॒प्त । \newline
55. स॒प्त र्.ष॑य॒ ऋष॑यः स॒प्त स॒प्त र्.ष॑यः । \newline

\textbf{Ghana Paata } \newline

1. क्रु॒द्धः प॑रो॒वप॑ परो॒वप॑ क्रु॒द्धः क्रु॒द्धः प॑रो॒वप॑ म॒न्युना॑ म॒न्युना॑ परो॒वप॑ क्रु॒द्धः क्रु॒द्धः प॑रो॒वप॑ म॒न्युना᳚ । \newline
2. प॒रो॒वप॑ म॒न्युना॑ म॒न्युना॑ परो॒वप॑ परो॒वप॑ म॒न्युना॒ यद् यन् म॒न्युना॑ परो॒वप॑ परो॒वप॑ म॒न्युना॒ यत् । \newline
3. प॒रो॒वपेति॑ परा - उ॒पव॑ । \newline
4. म॒न्युना॒ यद् यन् म॒न्युना॑ म॒न्युना॒ यदव॒र्त्या ऽव॑र्त्या॒ यन् म॒न्युना॑ म॒न्युना॒ यदव॑र्त्या । \newline
5. यदव॒र्त्या ऽव॑र्त्या॒ यद् यदव॑र्त्या । \newline
6. अव॒र्त्येत्यव॑र्त्या । \newline
7. सु॒कल्प॑ मग्ने ऽग्ने सु॒कल्प(ग्म्॑) सु॒कल्प॑ मग्ने॒ तत् तद॑ग्ने सु॒कल्प(ग्म्॑) सु॒कल्प॑ मग्ने॒ तत् । \newline
8. सु॒कल्प॒मिति॑ सु - कल्प᳚म् । \newline
9. अ॒ग्ने॒ तत् तद॑ग्ने ऽग्ने॒ तत् तव॒ तव॒ तद॑ग्ने ऽग्ने॒ तत् तव॑ । \newline
10. तत् तव॒ तव॒ तत् तत् तव॒ पुनः॒ पुन॒स्तव॒ तत् तत् तव॒ पुनः॑ । \newline
11. तव॒ पुनः॒ पुन॒स्तव॒ तव॒ पुन॑स्त्वा त्वा॒ पुन॒स्तव॒ तव॒ पुन॑स्त्वा । \newline
12. पुन॑स्त्वा त्वा॒ पुनः॒ पुन॒स्त्वोदुत् त्वा॒ पुनः॒ पुन॒स्त्वोत् । \newline
13. त्वोदुत् त्वा॒ त्वोद् दी॑पयामसि दीपयाम॒स्युत् त्वा॒ त्वोद् दी॑पयामसि । \newline
14. उद् दी॑पयामसि दीपयाम॒स्युदुद् दी॑पयामसि । \newline
15. दी॒प॒या॒म॒सीति॑ दीपयामसि । \newline
16. यत् ते॑ ते॒ यद् यत् ते॑ म॒न्युप॑रोप्तस्य म॒न्युप॑रोप्तस्य ते॒ यद् यत् ते॑ म॒न्युप॑रोप्तस्य । \newline
17. ते॒ म॒न्युप॑रोप्तस्य म॒न्युप॑रोप्तस्य ते ते म॒न्युप॑रोप्तस्य पृथि॒वीम् पृ॑थि॒वीम् म॒न्युप॑रोप्तस्य ते ते म॒न्युप॑रोप्तस्य पृथि॒वीम् । \newline
18. म॒न्युप॑रोप्तस्य पृथि॒वीम् पृ॑थि॒वीम् म॒न्युप॑रोप्तस्य म॒न्युप॑रोप्तस्य पृथि॒वी मन्वनु॑ पृथि॒वीम् म॒न्युप॑रोप्तस्य म॒न्युप॑रोप्तस्य पृथि॒वी मनु॑ । \newline
19. म॒न्युप॑रोप्त॒स्येति॑ म॒न्यु - प॒रो॒प्त॒स्य॒ । \newline
20. पृ॒थि॒वी मन्वनु॑ पृथि॒वीम् पृ॑थि॒वी मनु॑ दद्ध्व॒से द॑द्ध्व॒से ऽनु॑ पृथि॒वीम् पृ॑थि॒वी मनु॑ दद्ध्व॒से । \newline
21. अनु॑ दद्ध्व॒से द॑द्ध्व॒से ऽन्वनु॑ दद्ध्व॒से । \newline
22. द॒द्ध्व॒स इति॑ दद्ध्व॒से । \newline
23. आ॒दि॒त्या विश्वे॒ विश्व॑ आदि॒त्या आ॑दि॒त्या विश्वे॒ तत् तद् विश्व॑ आदि॒त्या आ॑दि॒त्या विश्वे॒ तत् । \newline
24. विश्वे॒ तत् तद् विश्वे॒ विश्वे॒ तद् दे॒वा दे॒वास्तद् विश्वे॒ विश्वे॒ तद् दे॒वाः । \newline
25. तद् दे॒वा दे॒वास्तत् तद् दे॒वा वस॑वो॒ वस॑वो दे॒वास्तत् तद् दे॒वा वस॑वः । \newline
26. दे॒वा वस॑वो॒ वस॑वो दे॒वा दे॒वा वस॑वश्च च॒ वस॑वो दे॒वा दे॒वा वस॑वश्च । \newline
27. वस॑वश्च च॒ वस॑वो॒ वस॑वश्च स॒माभ॑रन् थ्स॒माभ॑रꣳश्च॒ वस॑वो॒ वस॑वश्च स॒माभ॑रन्न् । \newline
28. च॒ स॒माभ॑रन् थ्स॒माभ॑रꣳश्च च स॒माभ॑रन्न् । \newline
29. स॒माभ॑र॒न्निति॑ सं - आभ॑रन्न् । \newline
30. मनो॒ ज्योति॒र् ज्योति॒र् मनो॒ मनो॒ ज्योति॑र् जुषताम् जुषता॒म् ज्योति॒र् मनो॒ मनो॒ ज्योति॑र् जुषताम् । \newline
31. ज्योति॑र् जुषताम् जुषता॒म् ज्योति॒र् ज्योति॑र् जुषता॒ माज्य॒ माज्य॑म् जुषता॒म् ज्योति॒र् ज्योति॑र् जुषता॒ माज्य᳚म् । \newline
32. जु॒ष॒ता॒ माज्य॒ माज्य॑म् जुषताम् जुषता॒ माज्यं॒ ॅविच्छि॑न्नं॒ ॅविच्छि॑न्न॒ माज्य॑म् जुषताम् जुषता॒ माज्यं॒ ॅविच्छि॑न्नम् । \newline
33. आज्यं॒ ॅविच्छि॑न्नं॒ ॅविच्छि॑न्न॒ माज्य॒ माज्यं॒ ॅविच्छि॑न्नं ॅय॒ज्ञ्ं ॅय॒ज्ञ्ं ॅविच्छि॑न्न॒ माज्य॒ माज्यं॒ ॅविच्छि॑न्नं ॅय॒ज्ञ्म् । \newline
34. विच्छि॑न्नं ॅय॒ज्ञ्ं ॅय॒ज्ञ्ं ॅविच्छि॑न्नं॒ ॅविच्छि॑न्नं ॅय॒ज्ञ्ꣳ सꣳ सं ॅय॒ज्ञ्ं ॅविच्छि॑न्नं॒ ॅविच्छि॑न्नं ॅय॒ज्ञ्ꣳ सम् । \newline
35. विच्छि॑न्न॒मिति॒ वि - छि॒न्न॒म् । \newline
36. य॒ज्ञ्ꣳ सꣳ सं ॅय॒ज्ञ्ं ॅय॒ज्ञ्ꣳ स मि॒म मि॒मꣳ सं ॅय॒ज्ञ्ं ॅय॒ज्ञ्ꣳ स मि॒मम् । \newline
37. स मि॒म मि॒मꣳ सꣳ स मि॒मम् द॑धातु दधात्वि॒मꣳ सꣳ स मि॒मम् द॑धातु । \newline
38. इ॒मम् द॑धातु दधात्वि॒म मि॒मम् द॑धातु । \newline
39. द॒धा॒त्विति॑ दधातु । \newline
40. बृह॒स्पति॑ स्तनुताम् तनुता॒म् बृह॒स्पति॒र् बृह॒स्पति॑ स्तनुता मि॒म मि॒मम् त॑नुता॒म् बृह॒स्पति॒र् बृह॒स्पति॑ स्तनुता मि॒मम् । \newline
41. त॒नु॒ता॒ मि॒म मि॒मम् त॑नुताम् तनुता मि॒मन्नो॑ न इ॒मम् त॑नुताम् तनुता मि॒मन्नः॑ । \newline
42. इ॒मन्नो॑ न इ॒म मि॒मन्नो॒ विश्वे॒ विश्वे॑ न इ॒म मि॒मन्नो॒ विश्वे᳚ । \newline
43. नो॒ विश्वे॒ विश्वे॑ नो नो॒ विश्वे॑ दे॒वा दे॒वा विश्वे॑ नो नो॒ विश्वे॑ दे॒वाः । \newline
44. विश्वे॑ दे॒वा दे॒वा विश्वे॒ विश्वे॑ दे॒वा इ॒हे ह दे॒वा विश्वे॒ विश्वे॑ दे॒वा इ॒ह । \newline
45. दे॒वा इ॒हे ह दे॒वा दे॒वा इ॒ह मा॑दयन्ताम् मादयन्ता मि॒ह दे॒वा दे॒वा इ॒ह मा॑दयन्ताम् । \newline
46. इ॒ह मा॑दयन्ताम् मादयन्ता मि॒हे ह मा॑दयन्ताम् । \newline
47. मा॒द॒य॒न्ता॒मिति॑ मादयन्ताम् । \newline
48. स॒प्त ते॑ ते स॒प्त स॒प्त ते॑ अग्ने ऽग्ने ते स॒प्त स॒प्त ते॑ अग्ने । \newline
49. ते॒ अ॒ग्ने॒ ऽग्ने॒ ते॒ ते॒ अ॒ग्ने॒ स॒मिधः॑ स॒मिधो᳚ ऽग्ने ते ते अग्ने स॒मिधः॑ । \newline
50. अ॒ग्ने॒ स॒मिधः॑ स॒मिधो᳚ ऽग्ने ऽग्ने स॒मिधः॑ स॒प्त स॒प्त स॒मिधो᳚ ऽग्ने ऽग्ने स॒मिधः॑ स॒प्त । \newline
51. स॒मिधः॑ स॒प्त स॒प्त स॒मिधः॑ स॒मिधः॑ स॒प्त जि॒ह्वा जि॒ह्वाः स॒प्त स॒मिधः॑ स॒मिधः॑ स॒प्त जि॒ह्वाः । \newline
52. स॒मिध॒ इति॑ सं - इधः॑ । \newline
53. स॒प्त जि॒ह्वा जि॒ह्वाः स॒प्त स॒प्त जि॒ह्वाः स॒प्त स॒प्त जि॒ह्वाः स॒प्त स॒प्त जि॒ह्वाः स॒प्त । \newline
54. जि॒ह्वाः स॒प्त स॒प्त जि॒ह्वा जि॒ह्वाः स॒प्त र्‌ष॑य॒ ऋष॑यः स॒प्त जि॒ह्वा जि॒ह्वाः स॒प्त र्‌ष॑यः । \newline
55. स॒प्त र्‌ष॑य॒ ऋष॑यः स॒प्त स॒प्त र्‌ष॑यः स॒प्त स॒प्त र्‌ष॑यः स॒प्त स॒प्त र्‌ष॑यः स॒प्त । \newline
\pagebreak
\markright{ TS 1.5.3.3  \hfill https://www.vedavms.in \hfill}
\addcontentsline{toc}{section}{ TS 1.5.3.3 }
\section*{ TS 1.5.3.3 }

\textbf{TS 1.5.3.3 } \newline
\textbf{Samhita Paata} \newline

र्.ष॑यः स॒प्त धाम॑ प्रि॒याणि॑ । स॒प्त होत्राः᳚ सप्त॒धा त्वा॑ यजन्ति स॒प्त योनी॒रा पृ॑णस्वा घृ॒तेन॑ ॥ पुन॑रू॒र्जा नि व॑र्तस्व॒ पुन॑रग्न इ॒षाऽऽ*यु॑षा । पुन॑र्नः पाहि वि॒श्वतः॑ ॥ स॒ह र॒य्या नि व॑र्त॒स्वाग्ने॒ पिन्व॑स्व॒ धार॑या । वि॒श्वफ्स्नि॑या वि॒श्वत॒स्परि॑ ॥ लेकः॒ सले॑कः सु॒लेक॒स्ते न॑ आदि॒त्या आज्यं॑ जुषा॒णा वि॑यन्तु॒ केतः॒ सके॑तः सु॒केत॒स्ते न॑ ( ) आदि॒त्या आज्यं॑ जुषा॒णा वि॑यन्तु॒ विव॑स्वाꣳ॒॒ अदि॑ति॒र् देव॑जूति॒स्ते न॑ आदि॒त्या आज्यं॑ जुषा॒णा वि॑यन्तु ॥ \newline

\textbf{Pada Paata} \newline

ऋष॑यः । स॒प्त । धाम॑ । प्रि॒याणि॑ ॥ स॒प्त । होत्राः᳚ । स॒प्त॒धेति॑ सप्त - धा । त्वा॒ । य॒ज॒न्ति॒ । स॒प्त । योनीः᳚ । एति॑ । पृ॒ण॒स्व॒ । घृ॒तेन॑ ॥ पुनः॑ । ऊ॒र्जा । नीति॑ । व॒र्त॒स्व॒ । पुनः॑ । अ॒ग्ने॒ । इ॒षा । आयु॑षा ॥ पुनः॑ । नः॒ । पा॒हि॒ । वि॒श्वतः॑ ॥ स॒ह । र॒य्या । नीति॑ । व॒र्त॒स्व॒ । अग्ने᳚ । पिन्व॑स्व । धार॑या ॥ वि॒श्वफ्‌स्नि॒येति॑ वि॒श्व - फ्‌स्नि॒या॒ । वि॒श्वतः॑ । परि॑ ॥ लेकः॑ । सले॑क॒ इति॒ स - ले॒कः॒ । सु॒लेक॒ इति॑ सु - लेकः॑ । ते । नः॒ । आ॒दि॒त्याः । आज्य᳚म् । जु॒षा॒णाः । वि॒य॒न्तु॒ । केतः॑ । सके॑त॒ इति॒ स - के॒तः॒ । सु॒केत॒ इति॑ सु - केतः॑ । ते । नः॒ ( ) । आ॒दि॒त्याः । आज्य᳚म् । जु॒षा॒णाः । वि॒य॒न्तु॒ । विव॑स्वान् । अदि॑तिः । देव॑जूति॒रिति॒ देव॑ - जू॒तिः॒ । ते । नः॒ । आ॒दि॒त्याः । आज्य᳚म् । जु॒षा॒णाः । वि॒य॒न्तु॒ ॥  \newline


\textbf{Krama Paata} \newline

ऋष॑यः स॒प्त । स॒प्त धाम॑ । धाम॑ प्रि॒याणि॑ । प्रि॒याणीति॑ प्रि॒याणि॑ ॥ स॒प्त होत्राः᳚ । होत्राः᳚ सप्त॒धा । स॒प्त॒धा त्वा᳚ । स॒प्त॒धेति॑ सप्त - धा । त्वा॒ य॒ज॒न्ति॒ । य॒ज॒न्ति॒ स॒प्त । स॒प्त योनीः᳚ । योनी॒रा । आ पृ॑णस्व । पृ॒ण॒स्वा॒ घृ॒तेन॑ । घृ॒तेनेति॑ घृ॒तेन॑ ॥ पुन॑रू॒र्जा । ऊ॒र्जा नि । नि व॑र्तस्व । व॒र्त॒स्व॒ पुनः॑ । पुन॑रग्ने । अ॒ग्न॒ इ॒षा । इ॒षा ऽऽयु॑षा । आयु॒षेत्यायु॑षा ॥ पुन॑र् नः । नः॒ पा॒हि॒ । पा॒हि॒ वि॒श्वतः॑ । वि॒श्वत॒ इति॑ वि॒श्वतः॑ ॥ स॒ह र॒य्या । र॒य्या नि । नि व॑र्तस्व । व॒र्त॒स्वाग्ने᳚ । अग्ने॒ पिन्व॑स्व । पिन्व॑स्व॒ धार॑या । धार॒येति॒ धार॑या ॥ वि॒श्वफ्स्नि॑या वि॒श्वतः॑ । वि॒श्वफ्स्नि॒येति॑ वि॒श्व - फ्स्नि॒या॒ । वि॒श्वत॒स्परि॑ । परीति॒ परि॑ ॥ लेकः॒ सले॑कः । सले॑कः सु॒लेकः॑ । सले॑क॒ इति॒ स - ले॒कः॒ । सु॒लेक॒स्ते । सु॒लेक॒ इति॑ सु - लेकः॑ । ते नः॑ । न॒ आ॒दि॒त्याः । आ॒दि॒त्या आज्य᳚म् । आज्य॑म् जुषा॒णाः । जु॒षा॒णा वि॑यन्तु । वि॒य॒न्तु॒ केतः॑ । केतः॒ सके॑तः । सके॑तः सु॒केतः॑ । सके॑त॒ इति॒ स - के॒तः॒ । सु॒केत॒स्ते । सु॒केत॒ इति॑ सु - केतः॑ । ते नः॑ ( ) । न॒ आ॒दि॒त्याः । आ॒दि॒त्या आज्य᳚म् । आज्य॑म् जुषा॒णाः । जु॒षा॒णा वि॑यन्तु । वि॒य॒न्तु॒ विव॑स्वान् । विव॑स्वाꣳ॒॒ अदि॑तिः । अदि॑ति॒र् देव॑जूतिः । देव॑जूति॒स्ते । देव॑जूति॒रिति॒ देव॑ - जू॒तिः॒ । ते नः॑ । न॒ आ॒दि॒त्याः । आ॒दि॒त्या. आज्य᳚म् । आज्य॑म् जुषा॒णाः । जु॒षा॒णा वि॑यन्तु । वि॒य॒न्त्विति॑ वियन्तु । \newline

\textbf{Jatai Paata} \newline

1. ऋष॑यः स॒प्त स॒प्त र्.ष॑य॒ ऋष॑यः स॒प्त । \newline
2. स॒प्त धाम॒ धाम॑ स॒प्त स॒प्त धाम॑ । \newline
3. धाम॑ प्रि॒याणि॑ प्रि॒याणि॒ धाम॒ धाम॑ प्रि॒याणि॑ । \newline
4. प्रि॒याणीति॑ प्रि॒याणि॑ । \newline
5. स॒प्त होत्रा॒ होत्राः᳚ स॒प्त स॒प्त होत्राः᳚ । \newline
6. होत्राः᳚ सप्त॒धा स॑प्त॒धा होत्रा॒ होत्राः᳚ सप्त॒धा । \newline
7. स॒प्त॒धा त्वा᳚ त्वा सप्त॒धा स॑प्त॒धा त्वा᳚ । \newline
8. स॒प्त॒धेति॑ सप्त - धा । \newline
9. त्वा॒ य॒ज॒न्ति॒ य॒ज॒न्ति॒ त्वा॒ त्वा॒ य॒ज॒न्ति॒ । \newline
10. य॒ज॒न्ति॒ स॒प्त स॒प्त य॑जन्ति यजन्ति स॒प्त । \newline
11. स॒प्त योनी॒र् योनीः᳚ स॒प्त स॒प्त योनीः᳚ । \newline
12. योनी॒रा योनी॒र् योनी॒रा । \newline
13. आ पृ॑णस्व पृण॒स्वा पृ॑णस्व । \newline
14. पृ॒ण॒स्वा॒ घृ॒तेन॑ घृ॒तेन॑ पृणस्व पृणस्वा घृ॒तेन॑ । \newline
15. घृ॒तेनेति॑ घृ॒तेन॑ । \newline
16. पुन॑ रू॒र्जोर्जा पुनः॒ पुन॑ रू॒र्जा । \newline
17. ऊ॒र्जा नि न्यू᳚र्जोर्जा नि । \newline
18. नि व॑र्तस्व वर्तस्व॒ नि नि व॑र्तस्व । \newline
19. व॒र्त॒स्व॒ पुनः॒ पुन॑र् वर्तस्व वर्तस्व॒ पुनः॑ । \newline
20. पुन॑ रग्ने ऽग्ने॒ पुनः॒ पुन॑ रग्ने । \newline
21. अ॒ग्न॒ इ॒षेषा ऽग्ने᳚ ऽग्न इ॒षा । \newline
22. इ॒षा ऽऽयु॒षा ऽऽयु॑षे॒षेषा ऽऽयु॑षा । \newline
23. आयु॒षेत्यायु॑षा । \newline
24. पुन॑र् नो नः॒ पुनः॒ पुन॑र् नः । \newline
25. नः॒ पा॒हि॒ पा॒हि॒ नो॒ नः॒ पा॒हि॒ । \newline
26. पा॒हि॒ वि॒श्वतो॑ वि॒श्वत॑ स्पाहि पाहि वि॒श्वतः॑ । \newline
27. वि॒श्वत॒ इति॑ वि॒श्वतः॑ । \newline
28. स॒ह र॒य्या र॒य्या स॒ह स॒ह र॒य्या । \newline
29. र॒य्या नि नि र॒य्या र॒य्या नि । \newline
30. नि व॑र्तस्व वर्तस्व॒ नि नि व॑र्तस्व । \newline
31. व॒र्त॒स्वाग्ने ऽग्ने॑ वर्तस्व वर्त॒स्वाग्ने᳚ । \newline
32. अग्ने॒ पिन्व॑स्व॒ पिन्व॒स्वाग्ने ऽग्ने॒ पिन्व॑स्व । \newline
33. पिन्व॑स्व॒ धार॑या॒ धार॑या॒ पिन्व॑स्व॒ पिन्व॑स्व॒ धार॑या । \newline
34. धार॒येति॒ धार॑या । \newline
35. वि॒श्वफ्स्नि॑या वि॒श्वतो॑ वि॒श्वतो॑ वि॒श्वफ्स्नि॑या वि॒श्वफ्स्नि॑या वि॒श्वतः॑ । \newline
36. वि॒श्वफ्स्नि॒येति॑ वि॒श्व - फ्स्नि॒या॒ । \newline
37. वि॒श्वत॒ स्परि॒ परि॑ वि॒श्वतो॑ वि॒श्वत॒ स्परि॑ । \newline
38. परीति॒ परि॑ । \newline
39. लेकः॒ सले॑कः॒ सले॑को॒ लेको॒ लेकः॒ सले॑कः । \newline
40. सले॑कः सु॒लेकः॑ सु॒लेकः॒ सले॑कः॒ सले॑कः सु॒लेकः॑ । \newline
41. सले॑क॒ इति॒ स - ले॒कः॒ । \newline
42. सु॒लेक॒स्ते ते सु॒लेकः॑ सु॒लेक॒स्ते । \newline
43. सु॒लेक॒ इति॑ सु - लेकः॑ । \newline
44. ते नो॑ न॒स्ते ते नः॑ । \newline
45. न॒ आ॒दि॒त्या आ॑दि॒त्या नो॑ न आदि॒त्याः । \newline
46. आ॒दि॒त्या आज्य॒ माज्य॑ मादि॒त्या आ॑दि॒त्या आज्य᳚म् । \newline
47. आज्य॑म् जुषा॒णा जु॑षा॒णा आज्य॒ माज्य॑म् जुषा॒णाः । \newline
48. जु॒षा॒णा वि॑यन्तु वियन्तु जुषा॒णा जु॑षा॒णा वि॑यन्तु । \newline
49. वि॒य॒न्तु॒ केतः॒ केतो॑ वियन्तु वियन्तु॒ केतः॑ । \newline
50. केतः॒ सके॑तः॒ सके॑तः॒ केतः॒ केतः॒ सके॑तः । \newline
51. सके॑तः सु॒केतः॑ सु॒केतः॒ सके॑तः॒ सके॑तः सु॒केतः॑ । \newline
52. सके॑त॒ इति॒ स - के॒तः॒ । \newline
53. सु॒केत॒स्ते ते सु॒केतः॑ सु॒केत॒स्ते । \newline
54. सु॒केत॒ इति॑ सु - केतः॑ । \newline
55. ते नो॑ न॒स्ते ते नः॑ । \newline
56. न॒ आ॒दि॒त्या आ॑दि॒त्या नो॑ न आदि॒त्याः । \newline
57. आ॒दि॒त्या आज्य॒ माज्य॑ मादि॒त्या आ॑दि॒त्या आज्य᳚म् । \newline
58. आज्य॑म् जुषा॒णा जु॑षा॒णा आज्य॒ माज्य॑म् जुषा॒णाः । \newline
59. जु॒षा॒णा वि॑यन्तु वियन्तु जुषा॒णा जु॑षा॒णा वि॑यन्तु । \newline
60. वि॒य॒न्तु॒ विव॑स्वा॒न्॒. विव॑स्वान्. वियन्तु वियन्तु॒ विव॑स्वान् । \newline
61. विव॑स्वा॒(ग्म्॒) अदि॑ति॒ रदि॑ति॒र् विव॑स्वा॒न्॒. विव॑स्वा॒(ग्म्॒) अदि॑तिः । \newline
62. अदि॑ति॒र् देव॑जूति॒र् देव॑जूति॒ रदि॑ति॒ रदि॑ति॒र् देव॑जूतिः । \newline
63. देव॑जूति॒स्ते ते देव॑जूति॒र् देव॑जूति॒स्ते । \newline
64. देव॑जूति॒रिति॒ देव॑ - जू॒तिः॒ । \newline
65. ते नो॑ न॒स्ते ते नः॑ । \newline
66. न॒ आ॒दि॒त्या आ॑दि॒त्या नो॑ न आदि॒त्याः । \newline
67. आ॒दि॒त्या आज्य॒ माज्य॑ मादि॒त्या आ॑दि॒त्या आज्य᳚म् । \newline
68. आज्य॑म् जुषा॒णा जु॑षा॒णा आज्य॒ माज्य॑म् जुषा॒णाः । \newline
69. जु॒षा॒णा वि॑यन्तु वियन्तु जुषा॒णा जु॑षा॒णा वि॑यन्तु । \newline
70. वि॒य॒न्त्विति॑ वियन्तु । \newline

\textbf{Ghana Paata } \newline

1. ऋष॑यः स॒प्त स॒प्त र्‌ष॑य॒ ऋष॑यः स॒प्त धाम॒ धाम॑ स॒प्त र्‌ष॑य॒ ऋष॑यः स॒प्त धाम॑ । \newline
2. स॒प्त धाम॒ धाम॑ स॒प्त स॒प्त धाम॑ प्रि॒याणि॑ प्रि॒याणि॒ धाम॑ स॒प्त स॒प्त धाम॑ प्रि॒याणि॑ । \newline
3. धाम॑ प्रि॒याणि॑ प्रि॒याणि॒ धाम॒ धाम॑ प्रि॒याणि॑ । \newline
4. प्रि॒याणीति॑ प्रि॒याणि॑ । \newline
5. स॒प्त होत्रा॒ होत्राः᳚ स॒प्त स॒प्त होत्राः᳚ सप्त॒धा स॑प्त॒धा होत्राः᳚ स॒प्त स॒प्त होत्राः᳚ सप्त॒धा । \newline
6. होत्राः᳚ सप्त॒धा स॑प्त॒धा होत्रा॒ होत्राः᳚ सप्त॒धा त्वा᳚ त्वा सप्त॒धा होत्रा॒ होत्राः᳚ सप्त॒धा त्वा᳚ । \newline
7. स॒प्त॒धा त्वा᳚ त्वा सप्त॒धा स॑प्त॒धा त्वा॑ यजन्ति यजन्ति त्वा सप्त॒धा स॑प्त॒धा त्वा॑ यजन्ति । \newline
8. स॒प्त॒धेति॑ सप्त - धा । \newline
9. त्वा॒ य॒ज॒न्ति॒ य॒ज॒न्ति॒ त्वा॒ त्वा॒ य॒ज॒न्ति॒ स॒प्त स॒प्त य॑जन्ति त्वा त्वा यजन्ति स॒प्त । \newline
10. य॒ज॒न्ति॒ स॒प्त स॒प्त य॑जन्ति यजन्ति स॒प्त योनी॒र् योनीः᳚ स॒प्त य॑जन्ति यजन्ति स॒प्त योनीः᳚ । \newline
11. स॒प्त योनी॒र् योनीः᳚ स॒प्त स॒प्त योनी॒रा योनीः᳚ स॒प्त स॒प्त योनी॒रा । \newline
12. योनी॒रा योनी॒र् योनी॒रा पृ॑णस्व पृण॒स्वा योनी॒र् योनी॒रा पृ॑णस्व । \newline
13. आ पृ॑णस्व पृण॒स्वा पृ॑णस्वा घृ॒तेन॑ घृ॒तेन॑ पृण॒स्वा पृ॑णस्वा घृ॒तेन॑ । \newline
14. पृ॒ण॒स्वा॒ घृ॒तेन॑ घृ॒तेन॑ पृणस्व पृणस्वा घृ॒तेन॑ । \newline
15. घृ॒तेनेति॑ घृ॒तेन॑ । \newline
16. पुन॑ रू॒र्जोर्जा पुनः॒ पुन॑ रू॒र्जा नि न्यू᳚र्जा पुनः॒ पुन॑ रू॒र्जा नि । \newline
17. ऊ॒र्जा नि न्यू᳚र्जोर्जा नि व॑र्तस्व वर्तस्व॒ न्यू᳚र्जोर्जा नि व॑र्तस्व । \newline
18. नि व॑र्तस्व वर्तस्व॒ नि नि व॑र्तस्व॒ पुनः॒ पुन॑र् वर्तस्व॒ नि नि व॑र्तस्व॒ पुनः॑ । \newline
19. व॒र्त॒स्व॒ पुनः॒ पुन॑र् वर्तस्व वर्तस्व॒ पुन॑ रग्ने ऽग्ने॒ पुन॑र् वर्तस्व वर्तस्व॒ पुन॑ रग्ने । \newline
20. पुन॑ रग्ने ऽग्ने॒ पुनः॒ पुन॑ रग्न इ॒षेषा ऽग्ने॒ पुनः॒ पुन॑ रग्न इ॒षा । \newline
21. अ॒ग्न॒ इ॒षेषा ऽग्ने᳚ ऽग्न इ॒षा ऽऽयु॒षा ऽऽयु॑षे॒षा ऽग्ने᳚ ऽग्न इ॒षा ऽऽयु॑षा । \newline
22. इ॒षा ऽऽयु॒षा ऽऽयु॑षे॒षेषा ऽऽयु॑षा । \newline
23. आयु॒षेत्यायु॑षा । \newline
24. पुन॑र् नो नः॒ पुनः॒ पुन॑र् नः पाहि पाहि नः॒ पुनः॒ पुन॑र् नः पाहि । \newline
25. नः॒ पा॒हि॒ पा॒हि॒ नो॒ नः॒ पा॒हि॒ वि॒श्वतो॑ वि॒श्वत॑ स्पाहि नो नः पाहि वि॒श्वतः॑ । \newline
26. पा॒हि॒ वि॒श्वतो॑ वि॒श्वत॑ स्पाहि पाहि वि॒श्वतः॑ । \newline
27. वि॒श्वत॒ इति॑ वि॒श्वतः॑ । \newline
28. स॒ह र॒य्या र॒य्या स॒ह स॒ह र॒य्या नि नि र॒य्या स॒ह स॒ह र॒य्या नि । \newline
29. र॒य्या नि नि र॒य्या र॒य्या नि व॑र्तस्व वर्तस्व॒ नि र॒य्या र॒य्या नि व॑र्तस्व । \newline
30. नि व॑र्तस्व वर्तस्व॒ नि नि व॑र्त॒स्वाग्ने ऽग्ने॑ वर्तस्व॒ नि नि व॑र्त॒स्वाग्ने᳚ । \newline
31. व॒र्त॒स्वाग्ने ऽग्ने॑ वर्तस्व वर्त॒स्वाग्ने॒ पिन्व॑स्व॒ पिन्व॒स्वाग्ने॑ वर्तस्व वर्त॒स्वाग्ने॒ पिन्व॑स्व । \newline
32. अग्ने॒ पिन्व॑स्व॒ पिन्व॒स्वाग्ने ऽग्ने॒ पिन्व॑स्व॒ धार॑या॒ धार॑या॒ पिन्व॒स्वाग्ने ऽग्ने॒ पिन्व॑स्व॒ धार॑या । \newline
33. पिन्व॑स्व॒ धार॑या॒ धार॑या॒ पिन्व॑स्व॒ पिन्व॑स्व॒ धार॑या । \newline
34. धार॒येति॒ धार॑या । \newline
35. वि॒श्वफ्स्नि॑या वि॒श्वतो॑ वि॒श्वतो॑ वि॒श्वफ्स्नि॑या वि॒श्वफ्स्नि॑या वि॒श्वत॒ स्परि॒ परि॑ वि॒श्वतो॑ वि॒श्वफ्स्नि॑या वि॒श्वफ्स्नि॑या वि॒श्वत॒ स्परि॑ । \newline
36. वि॒श्वफ्स्नि॒येति॑ वि॒श्व - फ्स्नि॒या॒ । \newline
37. वि॒श्वत॒ स्परि॒ परि॑ वि॒श्वतो॑ वि॒श्वत॒ स्परि॑ । \newline
38. परीति॒ परि॑ । \newline
39. लेकः॒ सले॑कः॒ सले॑को॒ लेको॒ लेकः॒ सले॑कः सु॒लेकः॑ सु॒लेकः॒ सले॑को॒ लेको॒ लेकः॒ सले॑कः सु॒लेकः॑ । \newline
40. सले॑कः सु॒लेकः॑ सु॒लेकः॒ सले॑कः॒ सले॑कः सु॒लेक॒स्ते ते सु॒लेकः॒ सले॑कः॒ सले॑कः सु॒लेक॒स्ते । \newline
41. सले॑क॒ इति॒ स - ले॒कः॒ । \newline
42. सु॒लेक॒स्ते ते सु॒लेकः॑ सु॒लेक॒स्ते नो॑ न॒स्ते सु॒लेकः॑ सु॒लेक॒स्ते नः॑ । \newline
43. सु॒लेक॒ इति॑ सु - लेकः॑ । \newline
44. ते नो॑ न॒स्ते ते न॑ आदि॒त्या आ॑दि॒त्या न॒स्ते ते न॑ आदि॒त्याः । \newline
45. न॒ आ॒दि॒त्या आ॑दि॒त्या नो॑ न आदि॒त्या आज्य॒ माज्य॑ मादि॒त्या नो॑ न आदि॒त्या आज्य᳚म् । \newline
46. आ॒दि॒त्या आज्य॒ माज्य॑ मादि॒त्या आ॑दि॒त्या आज्य॑म् जुषा॒णा जु॑षा॒णा आज्य॑ मादि॒त्या आ॑दि॒त्या आज्य॑म् जुषा॒णाः । \newline
47. आज्य॑म् जुषा॒णा जु॑षा॒णा आज्य॒ माज्य॑म् जुषा॒णा वि॑यन्तु वियन्तु जुषा॒णा आज्य॒ माज्य॑म् जुषा॒णा वि॑यन्तु । \newline
48. जु॒षा॒णा वि॑यन्तु वियन्तु जुषा॒णा जु॑षा॒णा वि॑यन्तु॒ केतः॒ केतो॑ वियन्तु जुषा॒णा जु॑षा॒णा वि॑यन्तु॒ केतः॑ । \newline
49. वि॒य॒न्तु॒ केतः॒ केतो॑ वियन्तु वियन्तु॒ केतः॒ सके॑तः॒ सके॑तः॒ केतो॑ वियन्तु वियन्तु॒ केतः॒ सके॑तः । \newline
50. केतः॒ सके॑तः॒ सके॑तः॒ केतः॒ केतः॒ सके॑तः सु॒केतः॑ सु॒केतः॒ सके॑तः॒ केतः॒ केतः॒ सके॑तः सु॒केतः॑ । \newline
51. सके॑तः सु॒केतः॑ सु॒केतः॒ सके॑तः॒ सके॑तः सु॒केत॒स्ते ते सु॒केतः॒ सके॑तः॒ सके॑तः सु॒केत॒स्ते । \newline
52. सके॑त॒ इति॒ स - के॒तः॒ । \newline
53. सु॒केत॒स्ते ते सु॒केतः॑ सु॒केत॒स्ते नो॑ न॒स्ते सु॒केतः॑ सु॒केत॒स्ते नः॑ । \newline
54. सु॒केत॒ इति॑ सु - केतः॑ । \newline
55. ते नो॑ न॒स्ते ते न॑ आदि॒त्या आ॑दि॒त्या न॒स्ते ते न॑ आदि॒त्याः । \newline
56. न॒ आ॒दि॒त्या आ॑दि॒त्या नो॑ न आदि॒त्या आज्य॒ माज्य॑ मादि॒त्या नो॑ न आदि॒त्या आज्य᳚म् । \newline
57. आ॒दि॒त्या आज्य॒ माज्य॑ मादि॒त्या आ॑दि॒त्या आज्य॑म् जुषा॒णा जु॑षा॒णा आज्य॑ मादि॒त्या आ॑दि॒त्या आज्य॑म् जुषा॒णाः । \newline
58. आज्य॑म् जुषा॒णा जु॑षा॒णा आज्य॒ माज्य॑म् जुषा॒णा वि॑यन्तु वियन्तु जुषा॒णा आज्य॒ माज्य॑म् जुषा॒णा वि॑यन्तु । \newline
59. जु॒षा॒णा वि॑यन्तु वियन्तु जुषा॒णा जु॑षा॒णा वि॑यन्तु॒ विव॑स्वा॒न्॒. विव॑स्वान्. वियन्तु जुषा॒णा जु॑षा॒णा वि॑यन्तु॒ विव॑स्वान् । \newline
60. वि॒य॒न्तु॒ विव॑स्वा॒न्॒. विव॑स्वान्. वियन्तु वियन्तु॒ विव॑स्वा॒(ग्म्॒) अदि॑ति॒ रदि॑ति॒र् विव॑स्वान्. वियन्तु वियन्तु॒ विव॑स्वा॒(ग्म्॒) अदि॑तिः । \newline
61. विव॑स्वा॒(ग्म्॒) अदि॑ति॒रदि॑ति॒र् विव॑स्वा॒न्॒. विव॑स्वा॒(ग्म्॒) अदि॑ति॒र् देव॑जूति॒र् देव॑जूति॒रदि॑ति॒र् विव॑स्वा॒न्॒. विव॑स्वा॒(ग्म्॒) अदि॑ति॒र् देव॑जूतिः । \newline
62. अदि॑ति॒र् देव॑जूति॒र् देव॑जूति॒ रदि॑ति॒ रदि॑ति॒र् देव॑जूति॒स्ते ते देव॑जूति॒ रदि॑ति॒ रदि॑ति॒र् देव॑जूति॒स्ते । \newline
63. देव॑जूति॒स्ते ते देव॑जूति॒र् देव॑जूति॒स्ते नो॑ न॒स्ते देव॑जूति॒र् देव॑जूति॒स्ते नः॑ । \newline
64. देव॑जूति॒रिति॒ देव॑ - जू॒तिः॒ । \newline
65. ते नो॑ न॒स्ते ते न॑ आदि॒त्या आ॑दि॒त्या न॒स्ते ते न॑ आदि॒त्याः । \newline
66. न॒ आ॒दि॒त्या आ॑दि॒त्या नो॑ न आदि॒त्या आज्य॒ माज्य॑ मादि॒त्या नो॑ न आदि॒त्या आज्य᳚म् । \newline
67. आ॒दि॒त्या आज्य॒ माज्य॑ मादि॒त्या आ॑दि॒त्या आज्य॑म् जुषा॒णा जु॑षा॒णा आज्य॑ मादि॒त्या आ॑दि॒त्या आज्य॑म् जुषा॒णाः । \newline
68. आज्य॑म् जुषा॒णा जु॑षा॒णा आज्य॒ माज्य॑म् जुषा॒णा वि॑यन्तु वियन्तु जुषा॒णा आज्य॒ माज्य॑म् जुषा॒णा वि॑यन्तु । \newline
69. जु॒षा॒णा वि॑यन्तु वियन्तु जुषा॒णा जु॑षा॒णा वि॑यन्तु । \newline
70. वि॒य॒न्त्विति॑ वियन्तु । \newline
\pagebreak
\markright{ TS 1.5.4.1  \hfill https://www.vedavms.in \hfill}
\addcontentsline{toc}{section}{ TS 1.5.4.1 }
\section*{ TS 1.5.4.1 }

\textbf{TS 1.5.4.1 } \newline
\textbf{Samhita Paata} \newline

भूमि॑र् भू॒म्ना द्यौर् व॑रि॒णेत्या॑हा॒ऽऽशिषै॒वैन॒मा ध॑त्ते स॒र्पा वै जीर्य॑न्तो ऽमन्यन्त॒ स ए॒तं क॑स॒र्णीरः॑ काद्रवे॒यो मन्त्र॑मपश्य॒त् ततो॒ वै ते जी॒र्णास्त॒नूरपा᳚घ्नत सर्परा॒ज्ञिया॑ ऋ॒ग्भिर् गार्.ह॑पत्य॒मा द॑धाति पुनर्न॒वमे॒वैन॑म॒जरं॑ कृ॒त्वा ऽऽ*ध॒त्तेऽथो॑ पू॒तमे॒व पृ॑थि॒वीम॒न्नाद्यं॒ नोपा॑नम॒थ्सैतं - [ ] \newline

\textbf{Pada Paata} \newline

भूमिः॑ । भू॒म्ना । द्यौः । व॒रि॒णा । इति॑ । आ॒ह॒ । आ॒शिषेत्या᳚ - शिषा᳚ । ए॒व । ए॒न॒म् । एति॑ । ध॒त्ते॒ । स॒र्पाः । वै । जीर्य॑न्तः । अ॒म॒न्य॒न्त॒ । सः । ए॒तम् । क॒स॒र्णीरः॑ । का॒द्र॒वे॒यः । मन्त्र᳚म् । अ॒प॒श्य॒त् । ततः॑ । वै । ते । जी॒र्णाः । त॒नूः । अपेति॑ । अ॒घ्न॒त॒ । स॒र्प॒रा॒ज्ञिया॒ इति॑ सर्प -रा॒ज्ञियाः᳚ । ऋ॒ग्भिरित्यृ॑क् - भिः । गार्.ह॑पत्य॒मिति॒ गार्.ह॑ - प॒त्य॒म् । एति॑ । द॒धा॒ति॒ । पु॒न॒र्न॒वमिति॑ पुनः- न॒वम् । ए॒व । ए॒न॒म् । अ॒जर᳚म् । कृ॒त्वा । एति॑ । ध॒त्ते॒ । अथो॒ इति॑ । पू॒तम् । ए॒व । पृ॒थि॒वीम् । अ॒न्नाद्य॒मित्य॑न्न - अद्य᳚म् । न । उपेति॑ । अ॒न॒म॒त् । सा । ए॒तम् ।  \newline


\textbf{Krama Paata} \newline

भूमि॑र् भू॒म्ना । भू॒म्ना द्यौः । द्यौर्,व॑रि॒णा । व॒रि॒णेति॑ । इत्या॑ह । आ॒हा॒शिषा᳚ । आ॒शिषै॒व । आ॒शिषेत्या᳚ - शिषा᳚ । ए॒वैन᳚म् । ए॒न॒मा । आ ध॑त्ते । ध॒त्ते॒ स॒र्पाः । स॒र्पा वै । वै जीर्य॑न्तः । जीर्य॑न्तोऽमन्यन्त । अ॒म॒न्य॒न्त॒ सः । स ए॒तम् । ए॒तम् क॑स॒र्णीरः॑ । क॒स॒र्णीरः॑ काद्रवे॒यः । का॒द्र॒वे॒यो मन्त्र᳚म् । मन्त्र॑मपश्यत् । अ॒प॒श्य॒त्,ततः॑ । ततो॒ वै । वै ते । ते जी॒र्णाः । जी॒र्णास्त॒नूः । त॒नूरप॑ । अपा᳚घ्नत । अ॒घ्न॒त॒ स॒र्प॒रा॒ज्ञियाः᳚ । स॒र्प॒रा॒ज्ञिया॑ ऋ॒ग्भिः । स॒र्प॒रा॒ज्ञिया॒ इति॑ सर्प - रा॒ज्ञियाः᳚ । ऋ॒ग्भिर्, गार्.ह॑पत्यम् । ऋ॒ग्भिरित्यृ॑क् - भिः । गार्.ह॑पत्य॒मा । गार्.ह॑पत्य॒मिति॒ गार्.ह॑ - प॒त्य॒म् । आ द॑धाति । द॒धा॒ति॒ पु॒न॒र्न॒वम् । पु॒न॒र्न॒वमे॒व । पु॒न॒र्न॒वमिति॑ पुनः - न॒वम् । ए॒वैन᳚म् । ए॒न॒म॒जर᳚म् । अ॒जर॑म् कृ॒त्वा । कृ॒त्वा ऽऽ ध॑त्ते । आ ध॑त्ते । ध॒त्तेऽथो᳚ । अथो॑ पू॒तम् । अथो॒ इत्यथो᳚ । पू॒तमे॒व । ए॒व पृ॑थि॒वीम् । पृ॒थि॒वीम॒न्नाद्य᳚म् । अ॒न्नाद्य॒न्न । अ॒न्नाद्य॒मित्य॑न्न - अद्य᳚म् । नोप॑ । उपा॑नमत् । अ॒न॒म॒थ् सा । सैतम् । ए॒तम् मन्त्र᳚म् \newline

\textbf{Jatai Paata} \newline

1. भूमि॑र् भू॒म्ना भू॒म्ना भूमि॒र् भूमि॑र् भू॒म्ना । \newline
2. भू॒म्ना द्यौर् द्यौर् भू॒म्ना भू॒म्ना द्यौः । \newline
3. द्यौर् व॑रि॒णा व॑रि॒णा द्यौर् द्यौर् व॑रि॒णा । \newline
4. व॒रि॒णेतीति॑ वरि॒णा व॑रि॒णेति॑ । \newline
5. इत्या॑हा॒हे तीत्या॑ह । \newline
6. आ॒हा॒शिषा॒ ऽऽशिषा॑ ऽऽहाहा॒शिषा᳚ । \newline
7. आ॒शिषै॒वैवाशिषा॒ ऽऽशिषै॒व । \newline
8. आ॒शिषेत्या᳚ - शिषा᳚ । \newline
9. ए॒वैन॑ मेन मे॒वैवैन᳚म् । \newline
10. ए॒न॒ मैन॑ मेन॒ मा । \newline
11. आ ध॑त्ते धत्त॒ आ ध॑त्ते । \newline
12. ध॒त्ते॒ स॒र्पाः स॒र्पा ध॑त्ते धत्ते स॒र्पाः । \newline
13. स॒र्पा वै वै स॒र्पाः स॒र्पा वै । \newline
14. वै जीर्य॑न्तो॒ जीर्य॑न्तो॒ वै वै जीर्य॑न्तः । \newline
15. जीर्य॑न्तो ऽमन्यन्तामन्यन्त॒ जीर्य॑न्तो॒ जीर्य॑न्तो ऽमन्यन्त । \newline
16. अ॒म॒न्य॒न्त॒ स सो॑ ऽमन्यन्तामन्यन्त॒ सः । \newline
17. स ए॒त मे॒तꣳ स स ए॒तम् । \newline
18. ए॒तम् क॑स॒र्णीरः॑ कस॒र्णीर॑ ए॒त मे॒तम् क॑स॒र्णीरः॑ । \newline
19. क॒स॒र्णीरः॑ काद्रवे॒यः का᳚द्रवे॒यः क॑स॒र्णीरः॑ कस॒र्णीरः॑ काद्रवे॒यः । \newline
20. का॒द्र॒वे॒यो मन्त्र॒म् मन्त्र॑म् काद्रवे॒यः का᳚द्रवे॒यो मन्त्र᳚म् । \newline
21. मन्त्र॑ मपश्यदपश्य॒न् मन्त्र॒म् मन्त्र॑ मपश्यत् । \newline
22. अ॒प॒श्य॒त् तत॒स्ततो॑ ऽपश्यदपश्य॒त् ततः॑ । \newline
23. ततो॒ वै वै तत॒स्ततो॒ वै । \newline
24. वै ते ते वै वै ते । \newline
25. ते जी॒र्णा जी॒र्णास्ते ते जी॒र्णाः । \newline
26. जी॒र्णा स्त॒नू स्त॒नूर् जी॒र्णा जी॒र्णास्त॒नूः । \newline
27. त॒नूरपाप॑ त॒नूस्त॒नूरप॑ । \newline
28. अपा᳚घ्नताघ्न॒तापापा᳚घ्नत । \newline
29. अ॒घ्न॒त॒ स॒र्प॒रा॒ज्ञियाः᳚ सर्परा॒ज्ञिया॑ अघ्नताघ्नत सर्परा॒ज्ञियाः᳚ । \newline
30. स॒र्प॒रा॒ज्ञिया॑ ऋ॒ग्भिर्. ऋ॒ग्भिः स॑र्परा॒ज्ञियाः᳚ सर्परा॒ज्ञिया॑ ऋ॒ग्भिः । \newline
31. स॒र्प॒रा॒ज्ञिया॒ इति॑ सर्प - रा॒ज्ञियाः᳚ । \newline
32. ऋ॒ग्भिर् गार्.ह॑पत्य॒म् गार्.ह॑पत्य मृ॒ग्भिर्. ऋ॒ग्भिर् गार्.ह॑पत्यम् । \newline
33. ऋ॒ग्भिरित्यृ॑क् - भिः । \newline
34. गार्.ह॑पत्य॒ मा गार्.ह॑पत्य॒म् गार्.ह॑पत्य॒ मा । \newline
35. गार्.ह॑पत्य॒मिति॒ गार्.ह॑ - प॒त्य॒म् । \newline
36. आ द॑धाति दधा॒त्या द॑धाति । \newline
37. द॒धा॒ति॒ पु॒न॒र्न॒वम् पु॑नर्न॒वम् द॑धाति दधाति पुनर्न॒वम् । \newline
38. पु॒न॒र्न॒व मे॒वैव पु॑नर्न॒वम् पु॑नर्न॒व मे॒व । \newline
39. पु॒न॒र्न॒वमिति॑ पुनः - न॒वम् । \newline
40. ए॒वैन॑ मेन मे॒वैवैन᳚म् । \newline
41. ए॒न॒ म॒जर॑ म॒जर॑ मेन मेन म॒जर᳚म् । \newline
42. अ॒जर॑म् कृ॒त्वा कृ॒त्वा ऽजर॑ म॒जर॑म् कृ॒त्वा । \newline
43. कृ॒त्वा ऽऽध॑त्ते धत्त॒ आ कृ॒त्वा कृ॒त्वा ऽऽध॑त्ते । \newline
44. आ ध॑त्ते धत्त॒ आ ध॑त्ते । \newline
45. ध॒त्ते ऽथो॒ अथो॑ धत्ते ध॒त्ते ऽथो᳚ । \newline
46. अथो॑ पू॒तम् पू॒त मथो॒ अथो॑ पू॒तम् । \newline
47. अथो॒ इत्यथो᳚ । \newline
48. पू॒त मे॒वैव पू॒तम् पू॒त मे॒व । \newline
49. ए॒व पृ॑थि॒वीम् पृ॑थि॒वी मे॒वैव पृ॑थि॒वीम् । \newline
50. पृ॒थि॒वी म॒न्नाद्य॑ म॒न्नाद्य॑म् पृथि॒वीम् पृ॑थि॒वी म॒न्नाद्य᳚म् । \newline
51. अ॒न्नाद्य॒न्न नान्नाद्य॑ म॒न्नाद्य॒न्न । \newline
52. अ॒न्नाद्य॒मित्य॑न्न - अद्य᳚म् । \newline
53. नोपोप॒ न नोप॑ । \newline
54. उपा॑नमदनम॒ दुपोपा॑नमत् । \newline
55. अ॒न॒म॒थ् सा सा ऽन॑मदनम॒थ् सा । \newline
56. सैत मे॒तꣳ सा सैतम् । \newline
57. ए॒तम् मन्त्र॒म् मन्त्र॑ मे॒त मे॒तम् मन्त्र᳚म् । \newline

\textbf{Ghana Paata } \newline

1. भूमि॑र् भू॒म्ना भू॒म्ना भूमि॒र् भूमि॑र् भू॒म्ना द्यौर् द्यौर् भू॒म्ना भूमि॒र् भूमि॑र् भू॒म्ना द्यौः । \newline
2. भू॒म्ना द्यौर् द्यौर् भू॒म्ना भू॒म्ना द्यौर् व॑रि॒णा व॑रि॒णा द्यौर् भू॒म्ना भू॒म्ना द्यौर् व॑रि॒णा । \newline
3. द्यौर् व॑रि॒णा व॑रि॒णा द्यौर् द्यौर् व॑रि॒णेतीति॑ वरि॒णा द्यौर् द्यौर् व॑रि॒णेति॑ । \newline
4. व॒रि॒णेतीति॑ वरि॒णा व॑रि॒णेत्या॑हा॒हेति॑ वरि॒णा व॑रि॒णेत्या॑ह । \newline
5. इत्या॑हा॒हे तीत्या॑हा॒शिषा॒ ऽऽशिषा॒ ऽऽहे तीत्या॑हा॒शिषा᳚ । \newline
6. आ॒हा॒शिषा॒ ऽऽशिषा॑ ऽऽहाहा॒शिषै॒वैवाशिषा॑ ऽऽहाहा॒शिषै॒व । \newline
7. आ॒शिषै॒वैवाशिषा॒ ऽऽशिषै॒वैन॑ मेन मे॒वाशिषा॒ ऽऽशिषै॒वैन᳚म् । \newline
8. आ॒शिषेत्या᳚ - शिषा᳚ । \newline
9. ए॒वैन॑ मेन मे॒वैवैन॒ मैन॑ मे॒वैवैन॒ मा । \newline
10. ए॒न॒ मैन॑ मेन॒ मा ध॑त्ते धत्त॒ ऐन॑ मेन॒ मा ध॑त्ते । \newline
11. आ ध॑त्ते धत्त॒ आ ध॑त्ते स॒र्पाः स॒र्पा ध॑त्त॒ आ ध॑त्ते स॒र्पाः । \newline
12. ध॒त्ते॒ स॒र्पाः स॒र्पा ध॑त्ते धत्ते स॒र्पा वै वै स॒र्पा ध॑त्ते धत्ते स॒र्पा वै । \newline
13. स॒र्पा वै वै स॒र्पाः स॒र्पा वै जीर्य॑न्तो॒ जीर्य॑न्तो॒ वै स॒र्पाः स॒र्पा वै जीर्य॑न्तः । \newline
14. वै जीर्य॑न्तो॒ जीर्य॑न्तो॒ वै वै जीर्य॑न्तो ऽमन्यन्तामन्यन्त॒ जीर्य॑न्तो॒ वै वै जीर्य॑न्तो ऽमन्यन्त । \newline
15. जीर्य॑न्तो ऽमन्यन्तामन्यन्त॒ जीर्य॑न्तो॒ जीर्य॑न्तो ऽमन्यन्त॒ स सो॑ ऽमन्यन्त॒ जीर्य॑न्तो॒ जीर्य॑न्तो ऽमन्यन्त॒ सः । \newline
16. अ॒म॒न्य॒न्त॒ स सो॑ ऽमन्यन्तामन्यन्त॒ स ए॒त मे॒तꣳ सो॑ ऽमन्यन्तामन्यन्त॒ स ए॒तम् । \newline
17. स ए॒त मे॒तꣳ स स ए॒तम् क॑स॒र्णीरः॑ कस॒र्णीर॑ ए॒तꣳ स स ए॒तम् क॑स॒र्णीरः॑ । \newline
18. ए॒तम् क॑स॒र्णीरः॑ कस॒र्णीर॑ ए॒त मे॒तम् क॑स॒र्णीरः॑ काद्रवे॒यः का᳚द्रवे॒यः क॑स॒र्णीर॑ ए॒त मे॒तम् क॑स॒र्णीरः॑ काद्रवे॒यः । \newline
19. क॒स॒र्णीरः॑ काद्रवे॒यः का᳚द्रवे॒यः क॑स॒र्णीरः॑ कस॒र्णीरः॑ काद्रवे॒यो मन्त्र॒म् मन्त्र॑म् काद्रवे॒यः क॑स॒र्णीरः॑ कस॒र्णीरः॑ काद्रवे॒यो मन्त्र᳚म् । \newline
20. का॒द्र॒वे॒यो मन्त्र॒म् मन्त्र॑म् काद्रवे॒यः का᳚द्रवे॒यो मन्त्र॑ मपश्यदपश्य॒न् मन्त्र॑म् काद्रवे॒यः का᳚द्रवे॒यो मन्त्र॑ मपश्यत् । \newline
21. मन्त्र॑ मपश्य दपश्य॒न् मन्त्र॒म् मन्त्र॑ मपश्य॒त् तत॒स्ततो॑ ऽपश्य॒न् मन्त्र॒म् मन्त्र॑ मपश्य॒त् ततः॑ । \newline
22. अ॒प॒श्य॒त् तत॒स्ततो॑ ऽपश्य दपश्य॒त् ततो॒ वै वै ततो॑ ऽपश्य दपश्य॒त् ततो॒ वै । \newline
23. ततो॒ वै वै तत॒ स्ततो॒ वै ते ते वै तत॒ स्ततो॒ वै ते । \newline
24. वै ते ते वै वै ते जी॒र्णा जी॒र्णास्ते वै वै ते जी॒र्णाः । \newline
25. ते जी॒र्णा जी॒र्णास्ते ते जी॒र्णा स्त॒नू स्त॒नूर् जी॒र्णास्ते ते जी॒र्णास्त॒नूः । \newline
26. जी॒र्णा स्त॒नू स्त॒नूर् जी॒र्णा जी॒र्णा स्त॒नूरपाप॑ त॒नूर् जी॒र्णा जी॒र्णास्त॒नूरप॑ । \newline
27. त॒नूरपाप॑ त॒नू स्त॒नू रपा᳚घ्नताघ्न॒ताप॑ त॒नू स्त॒नूरपा᳚घ्नत । \newline
28. अपा᳚घ्नताघ्न॒तापापा᳚घ्नत सर्परा॒ज्ञियाः᳚ सर्परा॒ज्ञिया॑ अघ्न॒तापापा᳚घ्नत सर्परा॒ज्ञियाः᳚ । \newline
29. अ॒घ्न॒त॒ स॒र्प॒रा॒ज्ञियाः᳚ सर्परा॒ज्ञिया॑ अघ्नताघ्नत सर्परा॒ज्ञिया॑ ऋ॒ग्भिर्. ऋ॒ग्भिः स॑र्परा॒ज्ञिया॑ अघ्नताघ्नत सर्परा॒ज्ञिया॑ ऋ॒ग्भिः । \newline
30. स॒र्प॒रा॒ज्ञिया॑ ऋ॒ग्भिर्. ऋ॒ग्भिः स॑र्परा॒ज्ञियाः᳚ सर्परा॒ज्ञिया॑ ऋ॒ग्भिर् गार्.ह॑पत्य॒म् गार्.ह॑पत्य मृ॒ग्भिः स॑र्परा॒ज्ञियाः᳚ सर्परा॒ज्ञिया॑ ऋ॒ग्भिर् गार्.ह॑पत्यम् । \newline
31. स॒र्प॒रा॒ज्ञिया॒ इति॑ सर्प - रा॒ज्ञियाः᳚ । \newline
32. ऋ॒ग्भिर् गार्.ह॑पत्य॒म् गार्.ह॑पत्य मृ॒ग्भिर्. ऋ॒ग्भिर् गार्.ह॑पत्य॒ मा गार्.ह॑पत्य मृ॒ग्भिर्. ऋ॒ग्भिर् गार्.ह॑पत्य॒ मा । \newline
33. ऋ॒ग्भिरित्यृ॑क् - भिः । \newline
34. गार्.ह॑पत्य॒ मा गार्.ह॑पत्य॒म् गार्.ह॑पत्य॒ मा द॑धाति दधा॒त्या गार्.ह॑पत्य॒म् गार्.ह॑पत्य॒ मा द॑धाति । \newline
35. गार्.ह॑पत्य॒मिति॒ गार्.ह॑ - प॒त्य॒म् । \newline
36. आ द॑धाति दधा॒त्या द॑धाति पुनर्न॒वम् पु॑नर्न॒वम् द॑धा॒त्या द॑धाति पुनर्न॒वम् । \newline
37. द॒धा॒ति॒ पु॒न॒र्न॒वम् पु॑नर्न॒वम् द॑धाति दधाति पुनर्न॒व मे॒वैव पु॑नर्न॒वम् द॑धाति दधाति पुनर्न॒व मे॒व । \newline
38. पु॒न॒र्न॒व मे॒वैव पु॑नर्न॒वम् पु॑नर्न॒व मे॒वैन॑ मेन मे॒व पु॑नर्न॒वम् पु॑नर्न॒व मे॒वैन᳚म् । \newline
39. पु॒न॒र्न॒वमिति॑ पुनः - न॒वम् । \newline
40. ए॒वैन॑ मेन मे॒वैवैन॑ म॒जर॑ म॒जर॑ मेन मे॒वैवैन॑ म॒जर᳚म् । \newline
41. ए॒न॒ म॒जर॑ म॒जर॑ मेन मेन म॒जर॑म् कृ॒त्वा कृ॒त्वा ऽजर॑ मेन मेन म॒जर॑म् कृ॒त्वा । \newline
42. अ॒जर॑म् कृ॒त्वा कृ॒त्वा ऽजर॑ म॒जर॑म् कृ॒त्वा ऽऽध॑त्ते धत्त॒ आ कृ॒त्वा ऽजर॑ म॒जर॑म् कृ॒त्वा ऽऽध॑त्ते । \newline
43. कृ॒त्वा ऽऽध॑त्ते धत्त॒ आ कृ॒त्वा कृ॒त्वा ऽऽध॒त्ते ऽथो॒ अथो॑ धत्त॒ आ कृ॒त्वा कृ॒त्वा ऽऽध॒त्ते ऽथो᳚ । \newline
44. आ ध॑त्ते धत्त॒ आ ध॒त्ते ऽथो॒ अथो॑ धत्त॒ आ ध॒त्ते ऽथो᳚ । \newline
45. ध॒त्ते ऽथो॒ अथो॑ धत्ते ध॒त्ते ऽथो॑ पू॒तम् पू॒त मथो॑ धत्ते ध॒त्ते ऽथो॑ पू॒तम् । \newline
46. अथो॑ पू॒तम् पू॒त मथो॒ अथो॑ पू॒त मे॒वैव पू॒त मथो॒ अथो॑ पू॒त मे॒व । \newline
47. अथो॒ इत्यथो᳚ । \newline
48. पू॒त मे॒वैव पू॒तम् पू॒त मे॒व पृ॑थि॒वीम् पृ॑थि॒वी मे॒व पू॒तम् पू॒त मे॒व पृ॑थि॒वीम् । \newline
49. ए॒व पृ॑थि॒वीम् पृ॑थि॒वी मे॒वैव पृ॑थि॒वी म॒न्नाद्य॑ म॒न्नाद्य॑म् पृथि॒वी मे॒वैव पृ॑थि॒वी म॒न्नाद्य᳚म् । \newline
50. पृ॒थि॒वी म॒न्नाद्य॑ म॒न्नाद्य॑म् पृथि॒वीम् पृ॑थि॒वी म॒न्नाद्य॒न्न नान्नाद्य॑म् पृथि॒वीम् पृ॑थि॒वी म॒न्नाद्य॒न्न । \newline
51. अ॒न्नाद्य॒न्न नान्नाद्य॑ म॒न्नाद्य॒न्नोपोप॒ नान्नाद्य॑ म॒न्नाद्य॒न्नोप॑ । \newline
52. अ॒न्नाद्य॒मित्य॑न्न - अद्य᳚म् । \newline
53. नोपोप॒ न नोपा॑नम दनम॒दुप॒ न नोपा॑नमत् । \newline
54. उपा॑ नमदनम॒ दुपोपा॑नम॒थ् सा सा ऽन॑म॒ दुपोपा॑नम॒थ् सा । \newline
55. अ॒न॒म॒थ् सा सा ऽन॑मदनम॒थ् सैत मे॒तꣳ सा ऽन॑मदनम॒थ् सैतम् । \newline
56. सैत मे॒तꣳ सा सैतम् मन्त्र॒म् मन्त्र॑ मे॒तꣳ सा सैतम् मन्त्र᳚म् । \newline
57. ए॒तम् मन्त्र॒म् मन्त्र॑ मे॒त मे॒तम् मन्त्र॑ मपश्य दपश्य॒न् मन्त्र॑ मे॒त मे॒तम् मन्त्र॑ मपश्यत् । \newline
\pagebreak
\markright{ TS 1.5.4.2  \hfill https://www.vedavms.in \hfill}
\addcontentsline{toc}{section}{ TS 1.5.4.2 }
\section*{ TS 1.5.4.2 }

\textbf{TS 1.5.4.2 } \newline
\textbf{Samhita Paata} \newline

मन्त्र॑मपश्य॒त् ततो॒ वै ताम॒न्नाद्य॒-मुपा॑नम॒द्यथ् -स॑र्परा॒ज्ञिया॑ ऋ॒ग्भिर् गार्.ह॑पत्य-मा॒दधा᳚त्य॒न्नाद्य॒स्याव॑रुद्ध्या॒ अथो॑ अ॒स्यामे॒वैनं॒ प्रति॑ष्ठित॒मा ध॑त्ते॒ यत्त्वा᳚ क्रु॒द्धः प॑रो॒वपेत्या॒हाप॑ह्नुत ए॒वास्मै॒ तत् पुन॒स्त्वोद्दी॑पयाम॒सीत्या॑ह॒ समि॑न्ध ए॒वैनं॒ ॅयत्ते॑ म॒न्युप॑रोप्त॒स्येत्या॑ह दे॒वता॑भिरे॒वै - [ ] \newline

\textbf{Pada Paata} \newline

मन्त्र᳚म् । अ॒प॒श्य॒त् । ततः॑ । वै । ताम् । अ॒न्नाद्य॒मित्य॑न्न - अद्य᳚म् । उपेति॑ । अ॒न॒म॒त् । यत् । स॒र्प॒रा॒ज्ञिया॒ इति॑ सर्प -रा॒ज्ञियाः᳚ । ऋ॒ग्भिरित्यृ॑क् - भिः । गार्.ह॑पत्य॒मिति॒ गार्.ह॑ - प॒त्य॒म् । आ॒दधा॒तीत्या᳚ - दधा॑ति । अ॒न्नाद्य॒स्येत्य॑न्न - अद्य॑स्य । अव॑रुद्ध्या॒ इत्यव॑ - रु॒द्ध्यै॒ । अथो॒ इति॑ । अ॒स्याम् । ए॒व । ए॒न॒म् । प्रति॑ष्ठित॒मिति॒ प्रति॑ - स्थि॒त॒म् । एति॑ । ध॒त्ते॒ । यत् । त्वा॒ । क्रु॒द्धः । प॒रो॒वपेति॑ परा - उ॒वप॑ । इति॑ । आ॒ह॒ । अपेति॑ । ह्नु॒ते॒ । ए॒व । अ॒स्मै॒ । तत् । पुनः॑ । त्वा॒ । उदिति॑ । दी॒प॒या॒म॒सि॒ । इति॑ । आ॒ह॒ । समिति॑ । इ॒न्धे॒ । ए॒व । ए॒न॒म् । यत् । ते॒ । म॒न्युप॑रोप्त॒स्येति॑ म॒न्यु - प॒रो॒प्त॒स्य॒ । इति॑ । आ॒ह॒ । दे॒वता॑भिः । ए॒व ।  \newline


\textbf{Krama Paata} \newline

मन्त्र॑मपश्यत् । अ॒प॒श्य॒त् त॒तः॑ । ततो॒ वै । वै ताम् । ताम॒न्नाद्य᳚म् । अ॒न्नाद्य॒मुप॑ । अ॒न्नाद्य॒मित्य॑न्न - अद्य᳚म् । उपा॑नमत् । अ॒न॒म॒द् यत् । यथ् स॑र्परा॒ज्ञियाः᳚ । स॒र्प॒रा॒ज्ञिया॑ ऋ॒ग्भिः । स॒र्प॒रा॒ज्ञिया॒ इति॑ सर्प - रा॒ज्ञियाः᳚ । ऋ॒ग्भिर्, गार्.ह॑पत्यम् । ऋ॒ग्भिरित्यृ॑क् - भिः । गार्.ह॑पत्यमा॒दधा॑ति । गार्.ह॑पत्य॒मिति॒ गार्.ह॑ - प॒त्य॒म् । आ॒दधा᳚त्य॒न्नाद्य॑स्य । आ॒दधा॒तीत्या᳚ - दधा॑ति । अ॒न्नाद्य॒स्याव॑रुद्ध्यै । अ॒न्नाद्य॒स्येत्य॑न्न - अद्य॑स्य । अव॑रुद्ध्या॒ अथो᳚ । अव॑रुद्ध्या॒ इत्यव॑ - रु॒द्ध्यै॒ । अथो॑ अ॒स्याम् । अथो॒ इत्यथो᳚ । अ॒स्यामे॒व । ए॒वैन᳚म् । ए॒न॒म् प्रति॑ष्ठितम् । प्रति॑ष्ठित॒मा । प्रति॑ष्ठित॒मिति॒ प्रति॑ - स्थि॒त॒म् । आ ध॑त्ते । ध॒त्ते॒ यत् । यत् त्वा᳚ । त्वा॒ क्रु॒द्धः । क्रु॒द्धः प॑रो॒वप॑ । प॒रो॒वपेति॑ । प॒रो॒वपेति॑ परा - उ॒वप॑ । इत्या॑ह । आ॒हाप॑ । अप॑ ह्नुते । ह्नु॒त॒ ए॒व । ए॒वास्मै᳚ । अ॒स्मै॒ तत् । तत्,पुनः॑ । पुन॑स्त्वा । त्वोत् । उद् दी॑पयामसि । दी॒प॒या॒म॒सीति॑ । इत्या॑ह । आ॒ह॒ सम् । समि॑न्धे । इ॒न्ध॒ ए॒व । ए॒वैन᳚म् । ए॒नं॒ ॅयत् । यत्ते᳚ । ते॒ म॒न्युप॑रोप्तस्य । म॒न्युप॑रोप्त॒स्येति॑ । म॒न्युप॑रोप्त॒स्येति॑ म॒न्यु - प॒रो॒प्त॒स्य॒ । इत्या॑ह । आ॒ह॒ दे॒वता॑भिः । दे॒वता॑भिरे॒व । ए॒वैन᳚म् \newline

\textbf{Jatai Paata} \newline

1. मन्त्र॑ मपश्य दपश्य॒न् मन्त्र॒म् मन्त्र॑ मपश्यत् । \newline
2. अ॒प॒श्य॒त् तत॒स्ततो॑ ऽपश्य दपश्य॒त् ततः॑ । \newline
3. ततो॒ वै वै तत॒स्ततो॒ वै । \newline
4. वै ताम् तां ॅवै वै ताम् । \newline
5. ता म॒न्नाद्य॑ म॒न्नाद्य॒म् ताम् ता म॒न्नाद्य᳚म् । \newline
6. अ॒न्नाद्य॒ मुपोपा॒न्नाद्य॑ म॒न्नाद्य॒ मुप॑ । \newline
7. अ॒न्नाद्य॒मित्य॑न्न - अद्य᳚म् । \newline
8. उपा॑नमदनम॒ दुपोपा॑नमत् । \newline
9. अ॒न॒म॒द् यद् यद॑नमदनम॒द् यत् । \newline
10. यथ् स॑र्परा॒ज्ञियाः᳚ सर्परा॒ज्ञिया॒ यद् यथ् स॑र्परा॒ज्ञियाः᳚ । \newline
11. स॒र्प॒रा॒ज्ञिया॑ ऋ॒ग्भिर्. ऋ॒ग्भिः स॑र्परा॒ज्ञियाः᳚ सर्परा॒ज्ञिया॑ ऋ॒ग्भिः । \newline
12. स॒र्प॒रा॒ज्ञिया॒ इति॑ सर्प - रा॒ज्ञियाः᳚ । \newline
13. ऋ॒ग्भिर् गार्.ह॑पत्य॒म् गार्.ह॑पत्य मृ॒ग्भिर्. ऋ॒ग्भिर् गार्.ह॑पत्यम् । \newline
14. ऋ॒ग्भिरित्यृ॑क् - भिः । \newline
15. गार्.ह॑पत्य मा॒दधा᳚त्या॒दधा॑ति॒ गार्.ह॑पत्य॒म् गार्.ह॑पत्य मा॒दधा॑ति । \newline
16. गार्.ह॑पत्य॒मिति॒ गार्.ह॑ - प॒त्य॒म् । \newline
17. आ॒दधा᳚ त्य॒न्नाद्य॑स्या॒ न्नाद्य॑स्या॒दधा᳚ त्या॒दधा᳚त्य॒न्नाद्य॑स्य । \newline
18. आ॒दधा॒तीत्या᳚ - दधा॑ति । \newline
19. अ॒न्नाद्य॒स्याव॑रुद्ध्या॒ अव॑रुद्ध्या अ॒न्नाद्य॑स्या॒न्नाद्य॒स्याव॑रुद्ध्यै । \newline
20. अ॒न्नाद्य॒स्येत्य॑न्न - अद्य॑स्य । \newline
21. अव॑रुद्ध्या॒ अथो॒ अथो॒ अव॑रुद्ध्या॒ अव॑रुद्ध्या॒ अथो᳚ । \newline
22. अव॑रुद्ध्या॒ इत्यव॑ - रु॒द्ध्यै॒ । \newline
23. अथो॑ अ॒स्या म॒स्या मथो॒ अथो॑ अ॒स्याम् । \newline
24. अथो॒ इत्यथो᳚ । \newline
25. अ॒स्या मे॒वैवास्या म॒स्या मे॒व । \newline
26. ए॒वैन॑ मेन मे॒वैवैन᳚म् । \newline
27. ए॒न॒म् प्रति॑ष्ठित॒म् प्रति॑ष्ठित मेन मेन॒म् प्रति॑ष्ठितम् । \newline
28. प्रति॑ष्ठित॒ मा प्रति॑ष्ठित॒म् प्रति॑ष्ठित॒ मा । \newline
29. प्रति॑ष्ठित॒मिति॒ प्रति॑ - स्थि॒त॒म् । \newline
30. आ ध॑त्ते धत्त॒ आ ध॑त्ते । \newline
31. ध॒त्ते॒ यद् यद् ध॑त्ते धत्ते॒ यत् । \newline
32. यत् त्वा᳚ त्वा॒ यद् यत् त्वा᳚ । \newline
33. त्वा॒ क्रु॒द्धः क्रु॒द्धस्त्वा᳚ त्वा क्रु॒द्धः । \newline
34. क्रु॒द्धः प॑रो॒वप॑ परो॒वप॑ क्रु॒द्धः क्रु॒द्धः प॑रो॒वप॑ । \newline
35. प॒रो॒वपे तीति॑ परो॒वप॑ परो॒वपे ति॑ । \newline
36. प॒रो॒वपेति॑ परा - उ॒वप॑ । \newline
37. इत्या॑हा॒हे तीत्या॑ह । \newline
38. आ॒हापापा॑हा॒हाप॑ । \newline
39. अप॑ ह्नुते ह्नु॒ते ऽपाप॑ ह्नुते । \newline
40. ह्नु॒त॒ ए॒वैव ह्नु॑ते ह्नुत ए॒व । \newline
41. ए॒वास्मा॑ अस्मा ए॒वैवास्मै᳚ । \newline
42. अ॒स्मै॒ तत् तद॑स्मा अस्मै॒ तत् । \newline
43. तत् पुनः॒ पुन॒स्तत् तत् पुनः॑ । \newline
44. पुन॑स्त्वा त्वा॒ पुनः॒ पुन॑स्त्वा । \newline
45. त्वोदुत् त्वा॒ त्वोत् । \newline
46. उद् दी॑पयामसि दीपयाम॒स्युदुद् दी॑पयामसि । \newline
47. दी॒प॒या॒म॒सीतीति॑ दीपयामसि दीपयाम॒सीति॑ । \newline
48. इत्या॑हा॒हे तीत्या॑ह । \newline
49. आ॒ह॒ सꣳ स मा॑हाह॒ सम् । \newline
50. स मि॑न्ध इन्धे॒ सꣳ स मि॑न्धे । \newline
51. इ॒न्ध॒ ए॒वैवे न्ध॑ इन्ध ए॒व । \newline
52. ए॒वैन॑ मेन मे॒वैवैन᳚म् । \newline
53. ए॒नं॒ ॅयद् यदे॑न मेनं॒ ॅयत् । \newline
54. यत् ते॑ ते॒ यद् यत् ते᳚ । \newline
55. ते॒ म॒न्युप॑रोप्तस्य म॒न्युप॑रोप्तस्य ते ते म॒न्युप॑रोप्तस्य । \newline
56. म॒न्युप॑रोप्त॒स्ये तीति॑ म॒न्युप॑रोप्तस्य म॒न्युप॑रोप्त॒स्ये ति॑ । \newline
57. म॒न्युप॑रोप्त॒स्येति॑ म॒न्यु - प॒रो॒प्त॒स्य॒ । \newline
58. इत्या॑हा॒हे तीत्या॑ह । \newline
59. आ॒ह॒ दे॒वता॑भिर् दे॒वता॑भिराहाह दे॒वता॑भिः । \newline
60. दे॒वता॑भि रे॒वैव दे॒वता॑भिर् दे॒वता॑भिरे॒व । \newline
61. ए॒वैन॑ मेन मे॒वैवैन᳚म् । \newline

\textbf{Ghana Paata } \newline

1. मन्त्र॑ मपश्य दपश्य॒न् मन्त्र॒म् मन्त्र॑ मपश्य॒त् तत॒ स्ततो॑ ऽपश्य॒न् मन्त्र॒म् मन्त्र॑ मपश्य॒त् ततः॑ । \newline
2. अ॒प॒श्य॒त् तत॒ स्ततो॑ ऽपश्य दपश्य॒त् ततो॒ वै वै ततो॑ ऽपश्यदपश्य॒त् ततो॒ वै । \newline
3. ततो॒ वै वै तत॒स्ततो॒ वै ताम् तां ॅवै तत॒स्ततो॒ वै ताम् । \newline
4. वै ताम् तां ॅवै वै ता म॒न्नाद्य॑ म॒न्नाद्य॒म् तां ॅवै वै ता म॒न्नाद्य᳚म् । \newline
5. ता म॒न्नाद्य॑ म॒न्नाद्य॒म् ताम् ता म॒न्नाद्य॒ मुपोपा॒न्नाद्य॒म् ताम् ता म॒न्नाद्य॒ मुप॑ । \newline
6. अ॒न्नाद्य॒ मुपोपा॒न्नाद्य॑ म॒न्नाद्य॒ मुपा॑नम दनम॒ दुपा॒न्नाद्य॑ म॒न्नाद्य॒ मुपा॑नमत् । \newline
7. अ॒न्नाद्य॒मित्य॑न्न - अद्य᳚म् । \newline
8. उपा॑नम दनम॒दुपोपा॑ नम॒द् यद् यद॑नम॒ दुपोपा॑ नम॒द् यत् । \newline
9. अ॒न॒म॒द् यद् यद॑नमदनम॒द् यथ् स॑र्परा॒ज्ञियाः᳚ सर्परा॒ज्ञिया॒ यद॑नमदनम॒द् यथ् स॑र्परा॒ज्ञियाः᳚ । \newline
10. यथ् स॑र्परा॒ज्ञियाः᳚ सर्परा॒ज्ञिया॒ यद् यथ् स॑र्परा॒ज्ञिया॑ ऋ॒ग्भिर्. ऋ॒ग्भिः स॑र्परा॒ज्ञिया॒ यद् यथ् स॑र्परा॒ज्ञिया॑ ऋ॒ग्भिः । \newline
11. स॒र्प॒रा॒ज्ञिया॑ ऋ॒ग्भिर्. ऋ॒ग्भिः स॑र्परा॒ज्ञियाः᳚ सर्परा॒ज्ञिया॑ ऋ॒ग्भिर् गार्.ह॑पत्य॒म् गार्.ह॑पत्य मृ॒ग्भिः स॑र्परा॒ज्ञियाः᳚ सर्परा॒ज्ञिया॑ ऋ॒ग्भिर् गार्.ह॑पत्यम् । \newline
12. स॒र्प॒रा॒ज्ञिया॒ इति॑ सर्प - रा॒ज्ञियाः᳚ । \newline
13. ऋ॒ग्भिर् गार्.ह॑पत्य॒म् गार्.ह॑पत्य मृ॒ग्भिर्. ऋ॒ग्भिर् गार्.ह॑पत्य मा॒दधा᳚त्या॒दधा॑ति॒ गार्.ह॑पत्य मृ॒ग्भिर्. ऋ॒ग्भिर् गार्.ह॑पत्य मा॒दधा॑ति । \newline
14. ऋ॒ग्भिरित्यृ॑क् - भिः । \newline
15. गार्.ह॑पत्य मा॒दधा᳚त्या॒दधा॑ति॒ गार्.ह॑पत्य॒म् गार्.ह॑पत्य मा॒दधा᳚ त्य॒न्नाद्य॑ स्या॒न्नाद्य॑ स्या॒दधा॑ति॒ गार्.ह॑पत्य॒म् गार्.ह॑पत्य मा॒दधा᳚त्य॒न्नाद्य॑स्य । \newline
16. गार्.ह॑पत्य॒मिति॒ गार्.ह॑ - प॒त्य॒म् । \newline
17. आ॒दधा᳚ त्य॒न्नाद्य॑ स्या॒न्नाद्य॑स्या॒ दधा᳚त्या॒ दधा᳚त्य॒न्नाद्य॒ स्याव॑रुद्ध्या॒ अव॑रुद्ध्या अ॒न्नाद्य॑स्या॒दधा᳚त्या॒दधा᳚ त्य॒न्नाद्य॒स्याव॑रुद्ध्यै । \newline
18. आ॒दधा॒तीत्या᳚ - दधा॑ति । \newline
19. अ॒न्नाद्य॒स्याव॑रुद्ध्या॒ अव॑रुद्ध्या अ॒न्नाद्य॑ स्या॒न्नाद्य॒ स्याव॑रुद्ध्या॒ अथो॒ अथो॒ अव॑रुद्ध्या अ॒न्नाद्य॑ स्या॒न्नाद्य॒ स्याव॑रुद्ध्या॒ अथो᳚ । \newline
20. अ॒न्नाद्य॒स्येत्य॑न्न - अद्य॑स्य । \newline
21. अव॑रुद्ध्या॒ अथो॒ अथो॒ अव॑रुद्ध्या॒ अव॑रुद्ध्या॒ अथो॑ अ॒स्या म॒स्या मथो॒ अव॑रुद्ध्या॒ अव॑रुद्ध्या॒ अथो॑ अ॒स्याम् । \newline
22. अव॑रुद्ध्या॒ इत्यव॑ - रु॒द्ध्यै॒ । \newline
23. अथो॑ अ॒स्या म॒स्या मथो॒ अथो॑ अ॒स्या मे॒वैवास्या मथो॒ अथो॑ अ॒स्या मे॒व । \newline
24. अथो॒ इत्यथो᳚ । \newline
25. अ॒स्या मे॒वैवास्या म॒स्या मे॒वैन॑ मेन मे॒वास्या म॒स्या मे॒वैन᳚म् । \newline
26. ए॒वैन॑ मेन मे॒वैवैन॒म् प्रति॑ष्ठित॒म् प्रति॑ष्ठित मेन मे॒वैवैन॒म् प्रति॑ष्ठितम् । \newline
27. ए॒न॒म् प्रति॑ष्ठित॒म् प्रति॑ष्ठित मेन मेन॒म् प्रति॑ष्ठित॒ मा प्रति॑ष्ठित मेन मेन॒म् प्रति॑ष्ठित॒ मा । \newline
28. प्रति॑ष्ठित॒ मा प्रति॑ष्ठित॒म् प्रति॑ष्ठित॒ मा ध॑त्ते धत्त॒ आ प्रति॑ष्ठित॒म् प्रति॑ष्ठित॒ मा ध॑त्ते । \newline
29. प्रति॑ष्ठित॒मिति॒ प्रति॑ - स्थि॒त॒म् । \newline
30. आ ध॑त्ते धत्त॒ आ ध॑त्ते॒ यद् यद् ध॑त्त॒ आ ध॑त्ते॒ यत् । \newline
31. ध॒त्ते॒ यद् यद् ध॑त्ते धत्ते॒ यत् त्वा᳚ त्वा॒ यद् ध॑त्ते धत्ते॒ यत् त्वा᳚ । \newline
32. यत् त्वा᳚ त्वा॒ यद् यत् त्वा᳚ क्रु॒द्धः क्रु॒द्ध स्त्वा॒ यद् यत् त्वा᳚ क्रु॒द्धः । \newline
33. त्वा॒ क्रु॒द्धः क्रु॒द्धस्त्वा᳚ त्वा क्रु॒द्धः प॑रो॒वप॑ परो॒वप॑ क्रु॒द्धस्त्वा᳚ त्वा क्रु॒द्धः प॑रो॒वप॑ । \newline
34. क्रु॒द्धः प॑रो॒वप॑ परो॒वप॑ क्रु॒द्धः क्रु॒द्धः प॑रो॒वपे तीति॑ परो॒वप॑ क्रु॒द्धः क्रु॒द्धः प॑रो॒वपे ति॑ । \newline
35. प॒रो॒वपे तीति॑ परो॒वप॑ परो॒वपे त्या॑हा॒हे ति॑ परो॒वप॑ परो॒वपे त्या॑ह । \newline
36. प॒रो॒वपेति॑ परा - उ॒वप॑ । \newline
37. इत्या॑हा॒हे तीत्या॒हापापा॒हे तीत्या॒हाप॑ । \newline
38. आ॒हापापा॑हा॒हाप॑ ह्नुते ह्नु॒ते ऽपा॑हा॒हाप॑ ह्नुते । \newline
39. अप॑ ह्नुते ह्नु॒ते ऽपाप॑ ह्नुत ए॒वैव ह्नु॒ते ऽपाप॑ ह्नुत ए॒व । \newline
40. ह्नु॒त॒ ए॒वैव ह्नु॑ते ह्नुत ए॒वास्मा॑ अस्मा ए॒व ह्नु॑ते ह्नुत ए॒वास्मै᳚ । \newline
41. ए॒वास्मा॑ अस्मा ए॒वैवास्मै॒ तत् तद॑स्मा ए॒वैवास्मै॒ तत् । \newline
42. अ॒स्मै॒ तत् तद॑स्मा अस्मै॒ तत् पुनः॒ पुन॒स्तद॑स्मा अस्मै॒ तत् पुनः॑ । \newline
43. तत् पुनः॒ पुन॒स्तत् तत् पुन॑स्त्वा त्वा॒ पुन॒स्तत् तत् पुन॑स्त्वा । \newline
44. पुन॑स्त्वा त्वा॒ पुनः॒ पुन॒स्त्वोदुत् त्वा॒ पुनः॒ पुन॒स्त्वोत् । \newline
45. त्वोदुत् त्वा॒ त्वोद् दी॑पयामसि दीपयाम॒स्युत् त्वा॒ त्वोद् दी॑पयामसि । \newline
46. उद् दी॑पयामसि दीपयाम॒स्युदुद् दी॑पयाम॒सीतीति॑ दीपयाम॒स्युदुद् दी॑पयाम॒सीति॑ । \newline
47. दी॒प॒या॒म॒सीतीति॑ दीपयामसि दीपयाम॒सीत्या॑हा॒हे ति॑ दीपयामसि दीपयाम॒सीत्या॑ह । \newline
48. इत्या॑हा॒हे तीत्या॑ह॒ सꣳ स मा॒हे तीत्या॑ह॒ सम् । \newline
49. आ॒ह॒ सꣳ स मा॑हाह॒ स मि॑न्ध इन्धे॒ स मा॑हाह॒ स मि॑न्धे । \newline
50. स मि॑न्ध इन्धे॒ सꣳ स मि॑न्ध ए॒वैवे न्धे॒ सꣳ स मि॑न्ध ए॒व । \newline
51. इ॒न्ध॒ ए॒वैवे न्ध॑ इन्ध ए॒वैन॑ मेन मे॒वे न्ध॑ इन्ध ए॒वैन᳚म् । \newline
52. ए॒वैन॑ मेन मे॒वैवैनं॒ ॅयद् यदे॑न मे॒वैवैनं॒ ॅयत् । \newline
53. ए॒नं॒ ॅयद् यदे॑न मेनं॒ ॅयत् ते॑ ते॒ यदे॑न मेनं॒ ॅयत् ते᳚ । \newline
54. यत् ते॑ ते॒ यद् यत् ते॑ म॒न्युप॑रोप्तस्य म॒न्युप॑रोप्तस्य ते॒ यद् यत् ते॑ म॒न्युप॑रोप्तस्य । \newline
55. ते॒ म॒न्युप॑रोप्तस्य म॒न्युप॑रोप्तस्य ते ते म॒न्युप॑रोप्त॒स्ये तीति॑ म॒न्युप॑रोप्तस्य ते ते म॒न्युप॑रोप्त॒स्ये ति॑ । \newline
56. म॒न्युप॑रोप्त॒स्ये तीति॑ म॒न्युप॑रोप्तस्य म॒न्युप॑रोप्त॒स्ये त्या॑हा॒हे ति॑ म॒न्युप॑रोप्तस्य म॒न्युप॑रोप्त॒स्ये त्या॑ह । \newline
57. म॒न्युप॑रोप्त॒स्येति॑ म॒न्यु - प॒रो॒प्त॒स्य॒ । \newline
58. इत्या॑हा॒हे तीत्या॑ह दे॒वता॑भिर् दे॒वता॑भिरा॒हे तीत्या॑ह दे॒वता॑भिः । \newline
59. आ॒ह॒ दे॒वता॑भिर् दे॒वता॑भिराहाह दे॒वता॑भिरे॒वैव दे॒वता॑भिराहाह दे॒वता॑भिरे॒व । \newline
60. दे॒वता॑भिरे॒वैव दे॒वता॑भिर् दे॒वता॑भिरे॒वैन॑ मेन मे॒व दे॒वता॑भिर् दे॒वता॑भिरे॒वैन᳚म् । \newline
61. ए॒वैन॑ मेन मे॒वैवैन॒(ग्म्॒) सꣳ समे॑न मे॒वैवैन॒(ग्म्॒) सम् । \newline
\pagebreak
\markright{ TS 1.5.4.3  \hfill https://www.vedavms.in \hfill}
\addcontentsline{toc}{section}{ TS 1.5.4.3 }
\section*{ TS 1.5.4.3 }

\textbf{TS 1.5.4.3 } \newline
\textbf{Samhita Paata} \newline

नꣳ॒॒ सं भ॑रति॒ वि वा ए॒तस्य॑ य॒ज्ञ्श्छि॑द्यते॒ यो᳚ऽग्निमु॑द्वा॒सय॑ते॒ बृह॒स्पति॑वत्य॒र्चोप॑ तिष्ठते॒ ब्रह्म॒ वै दे॒वानां॒ बृह॒स्पति॒र् ब्रह्म॑णै॒व य॒ज्ञ्ꣳ सं द॑धाति॒ विच्छि॑न्नं ॅय॒ज्ञ्ꣳ समि॒मं द॑धा॒त्वित्या॑ह॒ सन्त॑त्यै॒ विश्वे॑ दे॒वा इ॒ह मा॑दयन्ता॒मित्या॑ह स॒न्तत्यै॒व य॒ज्ञ्ं दे॒वेभ्योऽनु॑ दिशति स॒प्त ते॑ अग्ने स॒मिधः॑ स॒प्त जि॒ह्वा - [ ] \newline

\textbf{Pada Paata} \newline

ए॒न॒म् । समिति॑ । भ॒र॒ति॒ । वीति॑ । वै । ए॒तस्य॑ । य॒ज्ञ्ः । छि॒द्य॒ते॒ । यः । अ॒ग्निम् । उ॒द्वा॒सय॑त॒ इत्यु॑त् - वा॒सय॑ते । बृह॒स्पति॑व॒त्येति॒॒ बृह॒स्पति॑ - व॒त्य॒ । ऋ॒चा । उपेति॑ । ति॒ष्ठ॒ते॒ । ब्रह्म॑ । वै । दे॒वाना᳚म् । बृह॒स्पतिः॑ । ब्रह्म॑णा । ए॒व । य॒ज्ञ्म् । समिति॑ । द॒धा॒ति॒ । वच्छि॑न्न॒मिति॒ वि-छि॒न्न॒म् । य॒ज्ञ्म् । समिति॑ । इ॒मम् । द॒धा॒तु॒ । इति॑ । आ॒ह॒ । संत॑त्या॒ इति॒ सं - त॒त्यै॒ । विश्वे᳚ । दे॒वाः । इ॒ह । मा॒द॒य॒न्ता॒म् । इति॑ । आ॒ह॒ । स॒न्तत्येति॑ सं - तत्य॑ । ए॒व । य॒ज्ञ्म् । दे॒वेभ्यः॑ । अन्विति॑ । दि॒श॒ति॒ । स॒प्त । ते॒ । अ॒ग्ने॒ । स॒मिध॒ इति॑ सं - इधः॑ । स॒प्त । जि॒ह्वाः ।  \newline


\textbf{Krama Paata} \newline

ए॒नꣳ॒॒ सम् । सम् भ॑रति । भ॒र॒ति॒ वि । वि वै । वा ए॒तस्य॑ । ए॒तस्य॑ य॒ज्ञ्ः । य॒ज्ञ् श्छि॑द्यते । छि॒द्य॒ते॒ यः । यो᳚ऽग्निम् । अ॒ग्निमु॑द्वा॒सय॑ते । उ॒द्वा॒सय॑ते॒ बृह॒स्पति॑वत्या । उ॒द्वा॒सय॑त॒ इत्यु॑त् - वा॒सय॑ते । बृह॒स्पति॑वत्य॒र्चा । बृह॒स्पति॑व॒त्येति॒ बृह॒स्पति॑ - व॒त्या॒ । ऋ॒चोप॑ । उप॑ तिष्ठते । ति॒ष्ठ॒ते॒ ब्रह्म॑ । ब्रह्म॒ वै । वै दे॒वाना᳚म् । दे॒वाना॒म् बृह॒स्पतिः॑ । बृह॒स्पति॒र्,ब्रह्म॑णा । ब्रह्म॑णै॒व । ए॒व य॒ज्ञ्म् । य॒ज्ञ्ꣳ सम् । सम् द॑धाति । द॒धा॒ति॒ विच्छि॑न्नम् । विच्छि॑न्नं ॅय॒ज्ञ्म् । विच्छि॑न्न॒मिति॒ वि - छि॒न्न॒म् । य॒ज्ञ्ꣳ सम् । समि॒मम् । इ॒मम् द॑धातु । द॒धा॒त्विति॑ । इत्या॑ह । आ॒ह॒ सन्त॑त्यै । सन्त॑त्यै॒ विश्वे᳚ । सन्त॑त्या॒ इति॒ सम् - त॒त्यै॒ । विश्वे॑ दे॒वाः । दे॒वा इ॒ह । इ॒ह मा॑दयन्ताम् । मा॒द॒य॒न्ता॒मिति॑ । इत्या॑ह । आ॒ह॒ स॒न्तत्य॑ । स॒न्ततै॒व । स॒न्तत्येति॑ सम् - तत्य॑ । ए॒व य॒ज्ञ्म् । य॒ज्ञ्म् दे॒वेभ्यः॑ । दे॒वेभ्योऽनु॑ । अनु॑ दिशति । दि॒श॒ति॒ स॒प्त । स॒प्त ते᳚ । ते॒ अ॒ग्ने॒ । अ॒ग्ने॒ स॒मिधः॑ । स॒मिधः॑ स॒प्त । स॒मिध॒ इति॑ सम् - इधः॑ । स॒प्त जि॒ह्वाः । जि॒ह्वा इति॑ \newline

\textbf{Jatai Paata} \newline

1. ए॒न॒(ग्म्॒) सꣳ स मे॑न मेन॒(ग्म्॒) सम् । \newline
2. सम् भ॑रति भरति॒ सꣳ सम् भ॑रति । \newline
3. भ॒र॒ति॒ वि वि भ॑रति भरति॒ वि । \newline
4. वि वै वै वि वि वै । \newline
5. वा ए॒तस्यै॒तस्य॒ वै वा ए॒तस्य॑ । \newline
6. ए॒तस्य॑ य॒ज्ञो य॒ज्ञ् ए॒तस्यै॒तस्य॑ य॒ज्ञ्ः । \newline
7. य॒ज्ञ्श्छि॑द्यते छिद्यते य॒ज्ञो य॒ज्ञ्श्छि॑द्यते । \newline
8. छि॒द्य॒ते॒ यो यश्छि॑द्यते छिद्यते॒ यः । \newline
9. यो᳚ ऽग्नि म॒ग्निं ॅयो यो᳚ ऽग्निम् । \newline
10. अ॒ग्नि मु॑द्वा॒सय॑त उद्वा॒सय॑ते॒ ऽग्नि म॒ग्नि मु॑द्वा॒सय॑ते । \newline
11. उ॒द्वा॒सय॑ते॒ बृह॒स्पति॑वत्या॒ बृह॒स्पति॑वत्योद्वा॒सय॑त उद्वा॒सय॑ते॒ बृह॒स्पति॑वत्या । \newline
12. उ॒द्वा॒सय॑त॒ इत्यु॑त् - वा॒सय॑ते । \newline
13. बृह॒स्पति॑वत्य॒र्चर्चा बृह॒स्पति॑वत्या॒ बृह॒स्पति॑वत्य॒र्चा । \newline
14. बृह॒स्पति॑व॒त्येति॒ बृह॒स्पति॑ - व॒त्या॒ । \newline
15. ऋ॒चोपोपा॒र् चर्चोप॑ । \newline
16. उप॑ तिष्ठते तिष्ठत॒ उपोप॑ तिष्ठते । \newline
17. ति॒ष्ठ॒ते॒ ब्रह्म॒ ब्रह्म॑ तिष्ठते तिष्ठते॒ ब्रह्म॑ । \newline
18. ब्रह्म॒ वै वै ब्रह्म॒ ब्रह्म॒ वै । \newline
19. वै दे॒वाना᳚म् दे॒वानां॒ ॅवै वै दे॒वाना᳚म् । \newline
20. दे॒वाना॒म् बृह॒स्पति॒र् बृह॒स्पति॑र् दे॒वाना᳚म् दे॒वाना॒म् बृह॒स्पतिः॑ । \newline
21. बृह॒स्पति॒र् ब्रह्म॑णा॒ ब्रह्म॑णा॒ बृह॒स्पति॒र् बृह॒स्पति॒र् ब्रह्म॑णा । \newline
22. ब्रह्म॑णै॒वैव ब्रह्म॑णा॒ ब्रह्म॑णै॒व । \newline
23. ए॒व य॒ज्ञ्ं ॅय॒ज्ञ् मे॒वैव य॒ज्ञ्म् । \newline
24. य॒ज्ञ्ꣳ सꣳ सं ॅय॒ज्ञ्ं ॅय॒ज्ञ्ꣳ सम् । \newline
25. सम् द॑धाति दधाति॒ सꣳ सम् द॑धाति । \newline
26. द॒धा॒ति॒ विच्छि॑न्नं॒ ॅविच्छि॑न्नम् दधाति दधाति॒ विच्छि॑न्नम् । \newline
27. विच्छि॑न्नं ॅय॒ज्ञ्ं ॅय॒ज्ञ्ं ॅविच्छि॑न्नं॒ ॅविच्छि॑न्नं ॅय॒ज्ञ्म् । \newline
28. विच्छि॑न्न॒मिति॒ वि - छि॒न्न॒म् । \newline
29. य॒ज्ञ्ꣳ सꣳ सं ॅय॒ज्ञ्ं ॅय॒ज्ञ्ꣳ सम् । \newline
30. स मि॒म मि॒मꣳ सꣳ स मि॒मम् । \newline
31. इ॒मम् द॑धातु दधात्वि॒म मि॒मम् द॑धातु । \newline
32. द॒धा॒त्वितीति॑ दधातु दधा॒त्विति॑ । \newline
33. इत्या॑हा॒हे तीत्या॑ह । \newline
34. आ॒ह॒ सन्त॑त्यै॒ सन्त॑त्या आहाह॒ सन्त॑त्यै । \newline
35. सन्त॑त्यै॒ विश्वे॒ विश्वे॒ सन्त॑त्यै॒ सन्त॑त्यै॒ विश्वे᳚ । \newline
36. सन्त॑त्या॒ इति॒ सं - त॒त्यै॒ । \newline
37. विश्वे॑ दे॒वा दे॒वा विश्वे॒ विश्वे॑ दे॒वाः । \newline
38. दे॒वा इ॒हे ह दे॒वा दे॒वा इ॒ह । \newline
39. इ॒ह मा॑दयन्ताम् मादयन्ता मि॒हे ह मा॑दयन्ताम् । \newline
40. मा॒द॒य॒न्ता॒ मितीति॑ मादयन्ताम् मादयन्ता॒ मिति॑ । \newline
41. इत्या॑हा॒हे तीत्या॑ह । \newline
42. आ॒ह॒ स॒न्तत्य॑ स॒न्तत्या॑हाह स॒न्तत्य॑ । \newline
43. स॒न्तत्यै॒वैव स॒न्तत्य॑ स॒न्तत्यै॒व । \newline
44. स॒न्तत्येति॑ सं - तत्य॑ । \newline
45. ए॒व य॒ज्ञ्ं ॅय॒ज्ञ् मे॒वैव य॒ज्ञ्म् । \newline
46. य॒ज्ञ्म् दे॒वेभ्यो॑ दे॒वेभ्यो॑ य॒ज्ञ्ं ॅय॒ज्ञ्म् दे॒वेभ्यः॑ । \newline
47. दे॒वेभ्यो ऽन्वनु॑ दे॒वेभ्यो॑ दे॒वेभ्यो ऽनु॑ । \newline
48. अनु॑ दिशति दिश॒त्यन्वनु॑ दिशति । \newline
49. दि॒श॒ति॒ स॒प्त स॒प्त दि॑शति दिशति स॒प्त । \newline
50. स॒प्त ते॑ ते स॒प्त स॒प्त ते᳚ । \newline
51. ते॒ अ॒ग्ने॒ ऽग्ने॒ ते॒ ते॒ अ॒ग्ने॒ । \newline
52. अ॒ग्ने॒ स॒मिधः॑ स॒मिधो᳚ ऽग्ने ऽग्ने स॒मिधः॑ । \newline
53. स॒मिधः॑ स॒प्त स॒प्त स॒मिधः॑ स॒मिधः॑ स॒प्त । \newline
54. स॒मिध॒ इति॑ सं - इधः॑ । \newline
55. स॒प्त जि॒ह्वा जि॒ह्वाः स॒प्त स॒प्त जि॒ह्वाः । \newline
56. जि॒ह्वा इतीति॑ जि॒ह्वा जि॒ह्वा इति॑ । \newline

\textbf{Ghana Paata } \newline

1. ए॒न॒(ग्म्॒) सꣳ स मे॑न मेन॒(ग्म्॒) सम् भ॑रति भरति॒ समे॑न मेन॒(ग्म्॒) सम् भ॑रति । \newline
2. सम् भ॑रति भरति॒ सꣳ सम् भ॑रति॒ वि वि भ॑रति॒ सꣳ सम् भ॑रति॒ वि । \newline
3. भ॒र॒ति॒ वि वि भ॑रति भरति॒ वि वै वै वि भ॑रति भरति॒ वि वै । \newline
4. वि वै वै वि वि वा ए॒तस्यै॒तस्य॒ वै वि वि वा ए॒तस्य॑ । \newline
5. वा ए॒तस्यै॒तस्य॒ वै वा ए॒तस्य॑ य॒ज्ञो य॒ज्ञ् ए॒तस्य॒ वै वा ए॒तस्य॑ य॒ज्ञ्ः । \newline
6. ए॒तस्य॑ य॒ज्ञो य॒ज्ञ् ए॒तस्यै॒तस्य॑ य॒ज्ञ् श्छि॑द्यते छिद्यते य॒ज्ञ् ए॒तस्यै॒तस्य॑ य॒ज्ञ् श्छि॑द्यते । \newline
7. य॒ज्ञ्श्छि॑द्यते छिद्यते य॒ज्ञो य॒ज्ञ्श्छि॑द्यते॒ यो यश्छि॑द्यते य॒ज्ञो य॒ज्ञ्श्छि॑द्यते॒ यः । \newline
8. छि॒द्य॒ते॒ यो यश्छि॑द्यते छिद्यते॒ यो᳚ ऽग्नि म॒ग्निं ॅयश्छि॑द्यते छिद्यते॒ यो᳚ ऽग्निम् । \newline
9. यो᳚ ऽग्नि म॒ग्निं ॅयो यो᳚ ऽग्नि मु॑द्वा॒सय॑त उद्वा॒सय॑ते॒ ऽग्निं ॅयो यो᳚ ऽग्नि मु॑द्वा॒सय॑ते । \newline
10. अ॒ग्नि मु॑द्वा॒सय॑त उद्वा॒सय॑ते॒ ऽग्नि म॒ग्नि मु॑द्वा॒सय॑ते॒ बृह॒स्पति॑वत्या॒ बृह॒स्पति॑वत्योद्वा॒सय॑ते॒ ऽग्नि म॒ग्नि मु॑द्वा॒सय॑ते॒ बृह॒स्पति॑वत्या । \newline
11. उ॒द्वा॒सय॑ते॒ बृह॒स्पति॑वत्या॒ बृह॒स्पति॑वत्योद्वा॒सय॑त उद्वा॒सय॑ते॒ बृह॒स्पति॑वत्य॒ र्चर्चा बृह॒स्पति॑वत्योद्वा॒सय॑त उद्वा॒सय॑ते॒ बृह॒स्पति॑वत्य॒र्चा । \newline
12. उ॒द्वा॒सय॑त॒ इत्यु॑त् - वा॒सय॑ते । \newline
13. बृह॒स्पति॑वत्य॒र्चर्चा बृह॒स्पति॑वत्या॒ बृह॒स्पति॑वत्य॒र्चोपोप॒ र्चा बृह॒स्पति॑वत्या॒ बृह॒स्पति॑वत्य॒र्चोप॑ । \newline
14. बृह॒स्पति॑व॒त्येति॒ बृह॒स्पति॑ - व॒त्या॒ । \newline
15. ऋ॒चोपोपा॒ र्चर्चोप॑ तिष्ठते तिष्ठत॒ उपा॒ र्चर्चोप॑ तिष्ठते । \newline
16. उप॑ तिष्ठते तिष्ठत॒ उपोप॑ तिष्ठते॒ ब्रह्म॒ ब्रह्म॑ तिष्ठत॒ उपोप॑ तिष्ठते॒ ब्रह्म॑ । \newline
17. ति॒ष्ठ॒ते॒ ब्रह्म॒ ब्रह्म॑ तिष्ठते तिष्ठते॒ ब्रह्म॒ वै वै ब्रह्म॑ तिष्ठते तिष्ठते॒ ब्रह्म॒ वै । \newline
18. ब्रह्म॒ वै वै ब्रह्म॒ ब्रह्म॒ वै दे॒वाना᳚म् दे॒वानां॒ ॅवै ब्रह्म॒ ब्रह्म॒ वै दे॒वाना᳚म् । \newline
19. वै दे॒वाना᳚म् दे॒वानां॒ ॅवै वै दे॒वाना॒म् बृह॒स्पति॒र् बृह॒स्पति॑र् दे॒वानां॒ ॅवै वै दे॒वाना॒म् बृह॒स्पतिः॑ । \newline
20. दे॒वाना॒म् बृह॒स्पति॒र् बृह॒स्पति॑र् दे॒वाना᳚म् दे॒वाना॒म् बृह॒स्पति॒र् ब्रह्म॑णा॒ ब्रह्म॑णा॒ बृह॒स्पति॑र् दे॒वाना᳚म् दे॒वाना॒म् बृह॒स्पति॒र् ब्रह्म॑णा । \newline
21. बृह॒स्पति॒र् ब्रह्म॑णा॒ ब्रह्म॑णा॒ बृह॒स्पति॒र् बृह॒स्पति॒र् ब्रह्म॑णै॒वैव ब्रह्म॑णा॒ बृह॒स्पति॒र् बृह॒स्पति॒र् ब्रह्म॑णै॒व । \newline
22. ब्रह्म॑णै॒वैव ब्रह्म॑णा॒ ब्रह्म॑णै॒व य॒ज्ञ्ं ॅय॒ज्ञ् मे॒व ब्रह्म॑णा॒ ब्रह्म॑णै॒व य॒ज्ञ्म् । \newline
23. ए॒व य॒ज्ञ्ं ॅय॒ज्ञ् मे॒वैव य॒ज्ञ्ꣳ सꣳ सं ॅय॒ज्ञ् मे॒वैव य॒ज्ञ्ꣳ सम् । \newline
24. य॒ज्ञ्ꣳ सꣳ सं ॅय॒ज्ञ्ं ॅय॒ज्ञ्ꣳ सम् द॑धाति दधाति॒ सं ॅय॒ज्ञ्ं ॅय॒ज्ञ्ꣳ सम् द॑धाति । \newline
25. सम् द॑धाति दधाति॒ सꣳ सम् द॑धाति॒ विच्छि॑न्नं॒ ॅविच्छि॑न्नम् दधाति॒ सꣳ सम् द॑धाति॒ विच्छि॑न्नम् । \newline
26. द॒धा॒ति॒ विच्छि॑न्नं॒ ॅविच्छि॑न्नम् दधाति दधाति॒ विच्छि॑न्नं ॅय॒ज्ञ्ं ॅय॒ज्ञ्ं ॅविच्छि॑न्नम् दधाति दधाति॒ विच्छि॑न्नं ॅय॒ज्ञ्म् । \newline
27. विच्छि॑न्नं ॅय॒ज्ञ्ं ॅय॒ज्ञ्ं ॅविच्छि॑न्नं॒ ॅविच्छि॑न्नं ॅय॒ज्ञ्ꣳ सꣳ सं ॅय॒ज्ञ्ं ॅविच्छि॑न्नं॒ ॅविच्छि॑न्नं ॅय॒ज्ञ्ꣳ सम् । \newline
28. विच्छि॑न्न॒मिति॒ वि - छि॒न्न॒म् । \newline
29. य॒ज्ञ्ꣳ सꣳ सं ॅय॒ज्ञ्ं ॅय॒ज्ञ्ꣳ स मि॒म मि॒मꣳ सं ॅय॒ज्ञ्ं ॅय॒ज्ञ्ꣳ स मि॒मम् । \newline
30. स मि॒म मि॒मꣳ सꣳ स मि॒मम् द॑धातु दधात्वि॒मꣳ सꣳ समि॒मम् द॑धातु । \newline
31. इ॒मम् द॑धातु दधात्वि॒म मि॒मम् द॑धा॒त्वितीति॑ दधात्वि॒म मि॒मम् द॑धा॒त्विति॑ । \newline
32. द॒धा॒त्वितीति॑ दधातु दधा॒त्वित्या॑हा॒हे ति॑ दधातु दधा॒त्वित्या॑ह । \newline
33. इत्या॑हा॒हे तीत्या॑ह॒ सन्त॑त्यै॒ सन्त॑त्या आ॒हे तीत्या॑ह॒ सन्त॑त्यै । \newline
34. आ॒ह॒ सन्त॑त्यै॒ सन्त॑त्या आहाह॒ सन्त॑त्यै॒ विश्वे॒ विश्वे॒ सन्त॑त्या आहाह॒ सन्त॑त्यै॒ विश्वे᳚ । \newline
35. सन्त॑त्यै॒ विश्वे॒ विश्वे॒ सन्त॑त्यै॒ सन्त॑त्यै॒ विश्वे॑ दे॒वा दे॒वा विश्वे॒ सन्त॑त्यै॒ सन्त॑त्यै॒ विश्वे॑ दे॒वाः । \newline
36. सन्त॑त्या॒ इति॒ सं - त॒त्यै॒ । \newline
37. विश्वे॑ दे॒वा दे॒वा विश्वे॒ विश्वे॑ दे॒वा इ॒हे ह दे॒वा विश्वे॒ विश्वे॑ दे॒वा इ॒ह । \newline
38. दे॒वा इ॒हे ह दे॒वा दे॒वा इ॒ह मा॑दयन्ताम् मादयन्ता मि॒ह दे॒वा दे॒वा इ॒ह मा॑दयन्ताम् । \newline
39. इ॒ह मा॑दयन्ताम् मादयन्ता मि॒हे ह मा॑दयन्ता॒ मितीति॑ मादयन्ता मि॒हे ह मा॑दयन्ता॒ मिति॑ । \newline
40. मा॒द॒य॒न्ता॒ मितीति॑ मादयन्ताम् मादयन्ता॒ मित्या॑हा॒हे ति॑ मादयन्ताम् मादयन्ता॒ मित्या॑ह । \newline
41. इत्या॑हा॒हे तीत्या॑ह स॒न्तत्य॑ स॒न्तत्या॒हे तीत्या॑ह स॒न्तत्य॑ । \newline
42. आ॒ह॒ स॒न्तत्य॑ स॒न्तत्या॑हाह स॒न्तत्यै॒वैव स॒न्तत्या॑हाह स॒न्तत्यै॒व । \newline
43. स॒न्तत्यै॒वैव स॒न्तत्य॑ स॒न्तत्यै॒व य॒ज्ञ्ं ॅय॒ज्ञ् मे॒व स॒न्तत्य॑ स॒न्तत्यै॒व य॒ज्ञ्म् । \newline
44. स॒न्तत्येति॑ सं - तत्य॑ । \newline
45. ए॒व य॒ज्ञ्ं ॅय॒ज्ञ् मे॒वैव य॒ज्ञ्म् दे॒वेभ्यो॑ दे॒वेभ्यो॑ य॒ज्ञ् मे॒वैव य॒ज्ञ्म् दे॒वेभ्यः॑ । \newline
46. य॒ज्ञ्म् दे॒वेभ्यो॑ दे॒वेभ्यो॑ य॒ज्ञ्ं ॅय॒ज्ञ्म् दे॒वेभ्यो ऽन्वनु॑ दे॒वेभ्यो॑ य॒ज्ञ्ं ॅय॒ज्ञ्म् दे॒वेभ्यो ऽनु॑ । \newline
47. दे॒वेभ्यो ऽन्वनु॑ दे॒वेभ्यो॑ दे॒वेभ्यो ऽनु॑ दिशति दिश॒त्यनु॑ दे॒वेभ्यो॑ दे॒वेभ्यो ऽनु॑ दिशति । \newline
48. अनु॑ दिशति दिश॒त्यन्वनु॑ दिशति स॒प्त स॒प्त दि॑श॒त्यन्वनु॑ दिशति स॒प्त । \newline
49. दि॒श॒ति॒ स॒प्त स॒प्त दि॑शति दिशति स॒प्त ते॑ ते स॒प्त दि॑शति दिशति स॒प्त ते᳚ । \newline
50. स॒प्त ते॑ ते स॒प्त स॒प्त ते॑ अग्ने ऽग्ने ते स॒प्त स॒प्त ते॑ अग्ने । \newline
51. ते॒ अ॒ग्ने॒ ऽग्ने॒ ते॒ ते॒ अ॒ग्ने॒ स॒मिधः॑ स॒मिधो᳚ ऽग्ने ते ते अग्ने स॒मिधः॑ । \newline
52. अ॒ग्ने॒ स॒मिधः॑ स॒मिधो᳚ ऽग्ने ऽग्ने स॒मिधः॑ स॒प्त स॒प्त स॒मिधो᳚ ऽग्ने ऽग्ने स॒मिधः॑ स॒प्त । \newline
53. स॒मिधः॑ स॒प्त स॒प्त स॒मिधः॑ स॒मिधः॑ स॒प्त जि॒ह्वा जि॒ह्वाः स॒प्त स॒मिधः॑ स॒मिधः॑ स॒प्त जि॒ह्वाः । \newline
54. स॒मिध॒ इति॑ सं - इधः॑ । \newline
55. स॒प्त जि॒ह्वा जि॒ह्वाः स॒प्त स॒प्त जि॒ह्वा इतीति॑ जि॒ह्वाः स॒प्त स॒प्त जि॒ह्वा इति॑ । \newline
56. जि॒ह्वा इतीति॑ जि॒ह्वा जि॒ह्वा इत्या॑हा॒हे ति॑ जि॒ह्वा जि॒ह्वा इत्या॑ह । \newline
\pagebreak
\markright{ TS 1.5.4.4  \hfill https://www.vedavms.in \hfill}
\addcontentsline{toc}{section}{ TS 1.5.4.4 }
\section*{ TS 1.5.4.4 }

\textbf{TS 1.5.4.4 } \newline
\textbf{Samhita Paata} \newline

इत्या॑ह स॒प्तस॑प्त॒ वै स॑प्त॒धाऽग्नेः प्रि॒यास्त॒नुव॒स्ता ए॒वाव॑ रुन्धे॒ पुन॑रू॒र्जा स॒ह र॒य्येत्य॒भितः॑ पुरो॒डाश॒माहु॑ती जुहोति॒ यज॑मानमे॒वोर्जा च॑ र॒य्या चो॑भ॒यतः॒ परि॑ गृह्णात्यादि॒त्या वा अ॒स्माल्लो॒काद॒मुं ॅलो॒कमा॑य॒॒न् ते॑ऽमुष्मि॑न् ॅलो॒के व्य॑तृष्य॒न् त इ॒मं ॅलो॒कं पुन॑रभ्य॒वेत्या॒ ऽग्निमा॒धायै॒-तान्. ( ) होमा॑नजुहवु॒स्त आ᳚र्द्ध्नुव॒॒न् ते सु॑व॒र्गं ॅलो॒कमा॑य॒न्॒. यः प॑रा॒चीनं॑ पुनरा॒धेया॑द॒ग्निमा॒दधी॑त॒ स ए॒तान्. होमा᳚न् जुहुया॒द्यामे॒वाऽऽ*दि॒त्या ऋद्धि॒मार्द्ध्नु॑व॒न् तामे॒वर्द्ध्नो॑ति ॥ \newline

\textbf{Pada Paata} \newline

इति॑ । आ॒ह॒ । स॒प्तस॒प्तेति॑ स॒प्त - स॒प्त॒ । वै । स॒प्त॒धेति॑ सप्त - धा । अ॒ग्नेः । प्रि॒याः । त॒नुवः॑ । ताः । ए॒व । अवेति॑ । रु॒न्धे॒ । पुनः॑ । ऊ॒र्जा । स॒ह । र॒य्या । इति॑ । अ॒भितः॑ । पु॒रो॒डाश᳚म् । आहु॑ती॒ इत्या - हु॒ती॒ । जु॒हो॒ति॒ । यज॑मानम् । ए॒व । ऊ॒र्जा । च॒ । र॒य्या । च॒ । उ॒भ॒यतः॑ । परीति॑ । गृ॒ह्णा॒ति॒ । आ॒दि॒त्याः । वै । अ॒स्मात् । लो॒कात् । अ॒मुम् । लो॒कम् । आ॒य॒न्न् । ते । अ॒मुष्मिन्न्॑ । लो॒के । वीति॑ । अ॒तृ॒ष्य॒न्न् । ते । इ॒मम् । लो॒कम् । पुनः॑ । अ॒भ्य॒वेत्येत्य॑भि - अ॒वेत्य॑ । अ॒ग्निम् । आ॒धायेत्या᳚ - धाय॑ । ए॒तान् () । होमान्॑ । अ॒जु॒ह॒वुः॒ । ते । आ॒र्द्ध्नु॒व॒न्न् । ते । सु॒व॒र्गमिति॑ सुवः - गम् । लो॒कम् । आ॒य॒न्न् । यः । प॒रा॒चीन᳚म् । पु॒न॒रा॒धेया॒दिति॑ पुनः - आ॒धेया᳚त् । अ॒ग्निम् । आ॒दधी॒तेत्या᳚- दधी॑त । सः । ए॒तान् । होमान्॑ । जु॒हु॒या॒त् । याम् । ए॒व । आ॒दि॒त्याः । ऋद्धि᳚म् । आर्द्ध्नु॑वन्न् । ताम् । ए॒व । ऋ॒द्ध्नो॒ति॒ ॥  \newline


\textbf{Krama Paata} \newline

इत्या॑ह । आ॒ह॒ स॒प्तस॑प्त । स॒प्तस॑प्त॒ वै । स॒प्तस॒प्तेति॑ स॒प्त - स॒प्त॒ । वै स॑प्त॒धा । स॒प्त॒धाऽग्नेः । स॒प्त॒धेति॑ सप्त - धा । अ॒ग्नेः प्रि॒याः । प्रि॒यास्त॒नुवः॑ । त॒नुव॒स्ताः । ता ए॒व । ए॒वाव॑ । अव॑ रुन्धे । रु॒न्धे॒ पुनः॑ । पुन॑रू॒र्जा । ऊ॒र्जा स॒ह । स॒ह र॒य्या । र॒य्येति॑ । इत्य॒भितः॑ । अ॒भितः॑ पुरो॒डाश᳚म् । पु॒रो॒डाश॒माहु॑ती । आहु॑ती जुहोति । आहु॑ती॒ इत्या - हु॒ती॒ । जु॒हो॒ति॒ यज॑मानम् । यज॑मानमे॒व । ए॒वोर्जा । ऊ॒र्जा च॑ । च॒ र॒य्या । र॒य्या च॑ । चो॒भ॒यतः॑ । उ॒भ॒यतः॒ परि॑ । परि॑ गृह्णाति । गृ॒ह्णा॒त्या॒दि॒त्याः । आ॒दि॒त्या वै । वा अ॒स्मात् । अ॒स्माल्लो॒कात् । लो॒काद॒मुम् । अ॒मुं ॅलो॒कम् । लो॒कमा॑यन्न् । आ॒य॒न् ते । ते॑ऽमुष्मिन्न्॑ । अ॒मुष्मि॑न् ॅलो॒के । लो॒के वि । व्य॑तृष्यन्न् । अ॒तृ॒ष्य॒न् ते । त इ॒मम् । इ॒मं ॅलो॒कम् । लो॒कम् पुनः॑ । पुन॑रभ्य॒वेत्य॑ । अ॒भ्य॒वेत्या॒ग्निम् । अ॒भ्य॒वेत्येत्य॑भि - अ॒वेत्य॑ । अ॒ग्निमा॒धाय॑ । आ॒धायै॒तान् ( ) । आ॒धायेत्या᳚ - धाय॑ । ए॒तान्. होमान्॑ । होमा॑नजुहवुः । अ॒जु॒ह॒वु॒स्ते । त आ᳚र्द्ध्नुवन्न् । आ॒र्द्ध्नु॒व॒न् ते । ते सु॑व॒र्गम् । सु॒व॒र्गं ॅलो॒कम् । सु॒व॒र्गमिति॑ सुवः - गम् । लो॒कमा॑यन्न् । आ॒य॒न्॒. यः । यः प॑रा॒चीन᳚म् । प॒रा॒चीन॑म् पुनरा॒धेया᳚त् । पु॒न॒रा॒धेया॑द॒ग्निम् । पु॒न॒रा॒धेया॒दिति॑ पुनः - आ॒धेया᳚त् । अ॒ग्निमा॒दधी॑त । आ॒दधी॑त॒ सः । आ॒दधी॒तेत्या᳚ - दधी॑त । स ए॒तान् । ए॒तान्. होमान्॑ । होमा᳚ञ्जुहुयात् । जु॒हु॒या॒द् याम् । यामे॒व । ए॒वादि॒त्याः । आ॒दि॒त्या ऋद्धि᳚म् । ऋद्धि॒मार्द्ध्नु॑वन्न् । आर्द्ध्नु॑व॒न् ताम् । तामे॒व । ए॒वर्द्ध्नो॑ति । ऋ॒द्ध्नो॒तीत्यृ॑र्द्ध्नोति । \newline

\textbf{Jatai Paata} \newline

1. इत्या॑हा॒हे तीत्या॑ह । \newline
2. आ॒ह॒ स॒प्तस॑प्त स॒प्तस॑प्ताहाह स॒प्तस॑प्त । \newline
3. स॒प्तस॑प्त॒ वै वै स॒प्तस॑प्त स॒प्तस॑प्त॒ वै । \newline
4. स॒प्तस॒प्तेति॑ स॒प्त - स॒प्त॒ । \newline
5. वै स॑प्त॒धा स॑प्त॒धा वै वै स॑प्त॒धा । \newline
6. स॒प्त॒धा ऽग्नेर॒ग्नेः स॑प्त॒धा स॑प्त॒धा ऽग्नेः । \newline
7. स॒प्त॒धेति॑ सप्त - धा । \newline
8. अ॒ग्नेः प्रि॒याः प्रि॒या अ॒ग्नेर॒ग्नेः प्रि॒याः । \newline
9. प्रि॒या स्त॒नुव॑ स्त॒नुवः॑ प्रि॒याः प्रि॒यास्त॒नुवः॑ । \newline
10. त॒नुव॒ स्तास्ता स्त॒नुव॑ स्त॒नुव॒स्ताः । \newline
11. ता ए॒वैव तास्ता ए॒व । \newline
12. ए॒वावावै॒वैवाव॑ । \newline
13. अव॑ रुन्धे रु॒न्धे ऽवाव॑ रुन्धे । \newline
14. रु॒न्धे॒ पुनः॒ पुना॑ रुन्धे रुन्धे॒ पुनः॑ । \newline
15. पुन॑ रू॒र्जोर्जा पुनः॒ पुन॑ रू॒र्जा । \newline
16. ऊ॒र्जा स॒ह स॒होर्जोर्जा स॒ह । \newline
17. स॒ह र॒य्या र॒य्या स॒ह स॒ह र॒य्या । \newline
18. र॒य्येतीति॑ र॒य्या र॒य्येति॑ । \newline
19. इत्य॒भितो॒ ऽभित॒ इतीत्य॒भितः॑ । \newline
20. अ॒भितः॑ पुरो॒डाश॑म् पुरो॒डाश॑ म॒भितो॒ ऽभितः॑ पुरो॒डाश᳚म् । \newline
21. पु॒रो॒डाश॒ माहु॑ती॒ आहु॑ती पुरो॒डाश॑म् पुरो॒डाश॒ माहु॑ती । \newline
22. आहु॑ती जुहोति जुहो॒त्याहु॑ती॒ आहु॑ती जुहोति । \newline
23. आहु॑ती॒ इत्या - हु॒ती॒ । \newline
24. जु॒हो॒ति॒ यज॑मानं॒ ॅयज॑मानम् जुहोति जुहोति॒ यज॑मानम् । \newline
25. यज॑मान मे॒वैव यज॑मानं॒ ॅयज॑मान मे॒व । \newline
26. ए॒वोर्जोर्जैवैवोर्जा । \newline
27. ऊ॒र्जा च॑ चो॒र्जोर्जा च॑ । \newline
28. च॒ र॒य्या र॒य्या च॑ च र॒य्या । \newline
29. र॒य्या च॑ च र॒य्या र॒य्या च॑ । \newline
30. चो॒भ॒यत॑ उभ॒यत॑श्च चोभ॒यतः॑ । \newline
31. उ॒भ॒यतः॒ परि॒ पर्यु॑भ॒यत॑ उभ॒यतः॒ परि॑ । \newline
32. परि॑ गृह्णाति गृह्णाति॒ परि॒ परि॑ गृह्णाति । \newline
33. गृ॒ह्णा॒त्या॒दि॒त्या आ॑दि॒त्या गृ॑ह्णाति गृह्णात्यादि॒त्याः । \newline
34. आ॒दि॒त्या वै वा आ॑दि॒त्या आ॑दि॒त्या वै । \newline
35. वा अ॒स्मा द॒स्माद् वै वा अ॒स्मात् । \newline
36. अ॒स्मा ल्लो॒का ल्लो॒का द॒स्मा द॒स्मा ल्लो॒कात् । \newline
37. लो॒काद॒मु म॒मुम् ॅलो॒का ल्लो॒काद॒मुम् । \newline
38. अ॒मुम् ॅलो॒कम् ॅलो॒क म॒मु म॒मुम् ॅलो॒कम् । \newline
39. लो॒क मा॑यन्नायन्न् ॅलो॒कम् ॅलो॒क मा॑यन्न् । \newline
40. आ॒य॒न् ते त आ॑यन्नाय॒न् ते । \newline
41. ते॑ ऽमुष्मि॑न्न॒मुष्मि॒न् ते ते॑ ऽमुष्मिन्न्॑ । \newline
42. अ॒मुष्मि॑न् ॅलो॒के लो॒के॑ ऽमुष्मि॑ न्न॒मुष्मि॑न् ॅलो॒के । \newline
43. लो॒के वि वि लो॒के लो॒के वि । \newline
44. व्य॑तृष्यन्नतृष्य॒न्॒. वि व्य॑तृष्यन्न् । \newline
45. अ॒तृ॒ष्य॒न् ते ते॑ ऽतृष्यन्नतृष्य॒न् ते । \newline
46. त इ॒म मि॒मम् ते त इ॒मम् । \newline
47. इ॒मम् ॅलो॒कम् ॅलो॒क मि॒म मि॒मम् ॅलो॒कम् । \newline
48. लो॒कम् पुनः॒ पुन॑र् लो॒कम् ॅलो॒कम् पुनः॑ । \newline
49. पुन॑ रभ्य॒वेत्या᳚भ्य॒वेत्य॒ पुनः॒ पुन॑ रभ्य॒वेत्य॑ । \newline
50. अ॒भ्य॒वेत्या॒ग्नि म॒ग्नि म॑भ्य॒वेत्या᳚भ्य॒वेत्या॒ग्निम् । \newline
51. अ॒भ्य॒वेत्येत्य॑भि - अ॒वेत्य॑ । \newline
52. अ॒ग्नि मा॒धाया॒धाया॒ग्नि म॒ग्नि मा॒धाय॑ । \newline
53. आ॒धायै॒ता ने॒ता ना॒धाया॒धायै॒तान् । \newline
54. आ॒धायेत्या᳚ - धाय॑ । \newline
55. ए॒तान्. होमा॒न्॒. होमा॑ ने॒ता ने॒तान्. होमान्॑ । \newline
56. होमा॑ नजुहवु रजुहवु॒र्॒. होमा॒न्॒. होमा॑ नजुहवुः । \newline
57. अ॒जु॒ह॒वु॒स्ते ते॑ ऽजुहवु रजुहवु॒स्ते । \newline
58. त आ᳚र्द्ध्नुवन्नार्द्ध्नुव॒न् ते त आ᳚र्द्ध्नुवन्न् । \newline
59. आ॒र्द्ध्नु॒व॒न् ते त आ᳚र्द्ध्नुवन्नार्द्ध्नुव॒न् ते । \newline
60. ते सु॑व॒र्गꣳ सु॑व॒र्गम् ते ते सु॑व॒र्गम् । \newline
61. सु॒व॒र्गम् ॅलो॒कम् ॅलो॒कꣳ सु॑व॒र्गꣳ सु॑व॒र्गम् ॅलो॒कम् । \newline
62. सु॒व॒र्गमिति॑ सुवः - गम् । \newline
63. लो॒क मा॑यन्नायन्न् ॅलो॒कम् ॅलो॒क मा॑यन्न् । \newline
64. आ॒य॒न्॒. यो य आ॑यन्नाय॒न्॒. यः । \newline
65. यः प॑रा॒चीन॑म् परा॒चीनं॒ ॅयो यः प॑रा॒चीन᳚म् । \newline
66. प॒रा॒चीन॑म् पुनरा॒धेया᳚त् पुनरा॒धेया᳚त् परा॒चीन॑म् परा॒चीन॑म् पुनरा॒धेया᳚त् । \newline
67. पु॒न॒रा॒धेया॑द॒ग्नि म॒ग्निम् पु॑नरा॒धेया᳚त् पुनरा॒धेया॑द॒ग्निम् । \newline
68. पु॒न॒रा॒धेया॒दिति॑ पुनः - आ॒धेया᳚त् । \newline
69. अ॒ग्नि मा॒दधी॑ता॒दधी॑ता॒ग्नि म॒ग्नि मा॒दधी॑त । \newline
70. आ॒दधी॑त॒ स स आ॒दधी॑ता॒दधी॑त॒ सः । \newline
71. आ॒दधी॒तेत्या᳚ - दधी॑त । \newline
72. स ए॒ता ने॒तान् थ्स स ए॒तान् । \newline
73. ए॒तान्. होमा॒न्॒. होमा॑ ने॒ता ने॒तान्. होमान्॑ । \newline
74. होमा᳚न् जुहुयाज् जुहुया॒द्धोमा॒न्॒. होमा᳚न् जुहुयात् । \newline
75. जु॒हु॒या॒द् यां ॅयाम् जु॑हुयाज् जुहुया॒द् याम् । \newline
76. या मे॒वैव यां ॅया मे॒व । \newline
77. ए॒वादि॒त्या आ॑दि॒त्या ए॒वैवादि॒त्याः । \newline
78. आ॒दि॒त्या ऋद्धि॒ मृद्धि॑ मादि॒त्या आ॑दि॒त्या ऋद्धि᳚म् । \newline
79. ऋद्धि॒ मार्द्ध्नु॑व॒न्नार्द्ध्नु॑व॒न्नृद्धि॒ मृद्धि॒ मार्द्ध्नु॑वन्न् । \newline
80. आर्द्ध्नु॑व॒न् ताम् ता मार्द्ध्नु॑व॒न्नार्द्ध्नु॑व॒न् ताम् । \newline
81. ता मे॒वैव ताम् ता मे॒व । \newline
82. ए॒व र्द्ध्नो᳚त्यृद्ध्नोत्ये॒वैव र्द्ध्नो॑ति । \newline
83. ऋ॒द्ध्नो॒तीत्यृ॑द्ध्नोति । \newline

\textbf{Ghana Paata } \newline

1. इत्या॑हा॒हे तीत्या॑ह स॒प्तस॑प्त स॒प्तस॑प्ता॒हे तीत्या॑ह स॒प्तस॑प्त । \newline
2. आ॒ह॒ स॒प्तस॑प्त स॒प्तस॑प्ताहाह स॒प्तस॑प्त॒ वै वै स॒प्तस॑प्ताहाह स॒प्तस॑प्त॒ वै । \newline
3. स॒प्तस॑प्त॒ वै वै स॒प्तस॑प्त स॒प्तस॑प्त॒ वै स॑प्त॒धा स॑प्त॒धा वै स॒प्तस॑प्त स॒प्तस॑प्त॒ वै स॑प्त॒धा । \newline
4. स॒प्तस॒प्तेति॑ स॒प्त - स॒प्त॒ । \newline
5. वै स॑प्त॒धा स॑प्त॒धा वै वै स॑प्त॒धा ऽग्नेर॒ग्नेः स॑प्त॒धा वै वै स॑प्त॒धा ऽग्नेः । \newline
6. स॒प्त॒धा ऽग्नेर॒ग्नेः स॑प्त॒धा स॑प्त॒धा ऽग्नेः प्रि॒याः प्रि॒या अ॒ग्नेः स॑प्त॒धा स॑प्त॒धा ऽग्नेः प्रि॒याः । \newline
7. स॒प्त॒धेति॑ सप्त - धा । \newline
8. अ॒ग्नेः प्रि॒याः प्रि॒या अ॒ग्नेर॒ग्नेः प्रि॒या स्त॒नुव॑ स्त॒नुवः॑ प्रि॒या अ॒ग्नेर॒ग्नेः प्रि॒यास्त॒नुवः॑ । \newline
9. प्रि॒या स्त॒नुव॑ स्त॒नुवः॑ प्रि॒याः प्रि॒या स्त॒नुव॒ स्ता स्ता स्त॒नुवः॑ प्रि॒याः प्रि॒या स्त॒नुव॒स्ताः । \newline
10. त॒नुव॒ स्ता स्ता स्त॒नुव॑ स्त॒नुव॒ स्ता ए॒वैव तास्त॒नुव॑ स्त॒नुव॒स्ता ए॒व । \newline
11. ता ए॒वैव तास्ता ए॒वावावै॒व तास्ता ए॒वाव॑ । \newline
12. ए॒वावावै॒वैवाव॑ रुन्धे रु॒न्धे ऽवै॒वैवाव॑ रुन्धे । \newline
13. अव॑ रुन्धे रु॒न्धे ऽवाव॑ रुन्धे॒ पुनः॒ पुना॑ रु॒न्धे ऽवाव॑ रुन्धे॒ पुनः॑ । \newline
14. रु॒न्धे॒ पुनः॒ पुना॑ रुन्धे रुन्धे॒ पुन॑ रू॒र्जोर्जा पुना॑ रुन्धे रुन्धे॒ पुन॑ रू॒र्जा । \newline
15. पुन॑ रू॒र्जोर्जा पुनः॒ पुन॑ रू॒र्जा स॒ह स॒होर्जा पुनः॒ पुन॑ रू॒र्जा स॒ह । \newline
16. ऊ॒र्जा स॒ह स॒होर्जोर्जा स॒ह र॒य्या र॒य्या स॒होर्जोर्जा स॒ह र॒य्या । \newline
17. स॒ह र॒य्या र॒य्या स॒ह स॒ह र॒य्येतीति॑ र॒य्या स॒ह स॒ह र॒य्येति॑ । \newline
18. र॒य्येतीति॑ र॒य्या र॒य्येत्य॒भितो॒ ऽभित॒ इति॑ र॒य्या र॒य्येत्य॒भितः॑ । \newline
19. इत्य॒भितो॒ ऽभित॒ इतीत्य॒भितः॑ पुरो॒डाश॑म् पुरो॒डाश॑ म॒भित॒ इतीत्य॒भितः॑ पुरो॒डाश᳚म् । \newline
20. अ॒भितः॑ पुरो॒डाश॑म् पुरो॒डाश॑ म॒भितो॒ ऽभितः॑ पुरो॒डाश॒ माहु॑ती॒ आहु॑ती पुरो॒डाश॑ म॒भितो॒ ऽभितः॑ पुरो॒डाश॒ माहु॑ती । \newline
21. पु॒रो॒डाश॒ माहु॑ती॒ आहु॑ती पुरो॒डाश॑म् पुरो॒डाश॒ माहु॑ती जुहोति जुहो॒त्याहु॑ती पुरो॒डाश॑म् पुरो॒डाश॒ माहु॑ती जुहोति । \newline
22. आहु॑ती जुहोति जुहो॒त्याहु॑ती॒ आहु॑ती जुहोति॒ यज॑मानं॒ ॅयज॑मानम् जुहो॒त्याहु॑ती॒ आहु॑ती जुहोति॒ यज॑मानम् । \newline
23. आहु॑ती॒ इत्या - हु॒ती॒ । \newline
24. जु॒हो॒ति॒ यज॑मानं॒ ॅयज॑मानम् जुहोति जुहोति॒ यज॑मान मे॒वैव यज॑मानम् जुहोति जुहोति॒ यज॑मान मे॒व । \newline
25. यज॑मान मे॒वैव यज॑मानं॒ ॅयज॑मान मे॒वोर्जोर्जैव यज॑मानं॒ ॅयज॑मान मे॒वोर्जा । \newline
26. ए॒वोर्जोर्जैवैवोर्जा च॑ चो॒र्जैवैवोर्जा च॑ । \newline
27. ऊ॒र्जा च॑ चो॒र्जोर्जा च॑ र॒य्या र॒य्या चो॒र्जोर्जा च॑ र॒य्या । \newline
28. च॒ र॒य्या र॒य्या च॑ च र॒य्या च॑ च र॒य्या च॑ च र॒य्या च॑ । \newline
29. र॒य्या च॑ च र॒य्या र॒य्या चो॑भ॒यत॑ उभ॒यत॑श्च र॒य्या र॒य्या चो॑भ॒यतः॑ । \newline
30. चो॒भ॒यत॑ उभ॒यत॑श्च चोभ॒यतः॒ परि॒ पर्यु॑भ॒यत॑श्च चोभ॒यतः॒ परि॑ । \newline
31. उ॒भ॒यतः॒ परि॒ पर्यु॑भ॒यत॑ उभ॒यतः॒ परि॑ गृह्णाति गृह्णाति॒ पर्यु॑भ॒यत॑ उभ॒यतः॒ परि॑ गृह्णाति । \newline
32. परि॑ गृह्णाति गृह्णाति॒ परि॒ परि॑ गृह्णात्यादि॒त्या आ॑दि॒त्या गृ॑ह्णाति॒ परि॒ परि॑ गृह्णात्यादि॒त्याः । \newline
33. गृ॒ह्णा॒त्या॒दि॒त्या आ॑दि॒त्या गृ॑ह्णाति गृह्णात्यादि॒त्या वै वा आ॑दि॒त्या गृ॑ह्णाति गृह्णात्यादि॒त्या वै । \newline
34. आ॒दि॒त्या वै वा आ॑दि॒त्या आ॑दि॒त्या वा अ॒स्माद॒स्माद् वा आ॑दि॒त्या आ॑दि॒त्या वा अ॒स्मात् । \newline
35. वा अ॒स्माद॒स्माद् वै वा अ॒स्माल्लो॒का ल्लो॒का द॒स्माद् वै वा अ॒स्माल्लो॒कात् । \newline
36. अ॒स्मा ल्लो॒का ल्लो॒का द॒स्मा द॒स्मा ल्लो॒काद॒मु म॒मुम् ॅलो॒का द॒स्मा द॒स्मा ल्लो॒काद॒मुम् । \newline
37. लो॒काद॒मु म॒मुम् ॅलो॒काल्लो॒काद॒मुम् ॅलो॒कम् ॅलो॒क म॒मुम् ॅलो॒काल्लो॒काद॒मुम् ॅलो॒कम् । \newline
38. अ॒मुम् ॅलो॒कम् ॅलो॒क म॒मु म॒मुम् ॅलो॒क मा॑यन् नायन् ॅलो॒क म॒मु म॒मुम् ॅलो॒क मा॑यन्न् । \newline
39. लो॒क मा॑यन् नायन् ॅलो॒कम् ॅलो॒क मा॑य॒न् ते त आ॑यन् ॅलो॒कम् ॅलो॒क मा॑य॒न् ते । \newline
40. आ॒य॒न् ते त आ॑यन् नाय॒न् ते॑ ऽमुष्मि॑न् न॒मुष्मि॒न् त आ॑यन् नाय॒न् ते॑ ऽमुष्मिन्न्॑ । \newline
41. ते॑ ऽमुष्मि॑न् न॒मुष्मि॒न् ते ते॑ ऽमुष्मि॑न्न् ॅलो॒के लो॒के॑ ऽमुष्मि॒न् ते ते॑ ऽमुष्मि॑न्न् ॅलो॒के । \newline
42. अ॒मुष्मि॑न्न् ॅलो॒के लो॒के॑ ऽमुष्मि॑न् न॒मुष्मि॑न्न् ॅलो॒के वि वि लो॒के॑ ऽमुष्मि॑न् न॒मुष्मि॑न्न् ॅलो॒के वि । \newline
43. लो॒के वि वि लो॒के लो॒के व्य॑तृष्यन् नतृष्य॒न्॒. वि लो॒के लो॒के व्य॑तृष्यन्न् । \newline
44. व्य॑तृष्यन् नतृष्य॒न्॒. वि व्य॑तृष्य॒न् ते ते॑ ऽतृष्य॒न्॒. वि व्य॑तृष्य॒न् ते । \newline
45. अ॒तृ॒ष्य॒न् ते ते॑ ऽतृष्यन् नतृष्य॒न् त इ॒म मि॒मम् ते॑ ऽतृष्यन् नतृष्य॒न् त इ॒मम् । \newline
46. त इ॒म मि॒मम् ते त इ॒मम् ॅलो॒कम् ॅलो॒क मि॒मम् ते त इ॒मम् ॅलो॒कम् । \newline
47. इ॒मम् ॅलो॒कम् ॅलो॒क मि॒म मि॒मम् ॅलो॒कम् पुनः॒ पुन॑र् लो॒क मि॒म मि॒मम् ॅलो॒कम् पुनः॑ । \newline
48. लो॒कम् पुनः॒ पुन॑र् लो॒कम् ॅलो॒कम् पुन॑ रभ्य॒वेत्या᳚भ्य॒वेत्य॒ पुन॑र् लो॒कम् ॅलो॒कम् पुन॑ रभ्य॒वेत्य॑ । \newline
49. पुन॑ रभ्य॒वेत्या᳚भ्य॒वेत्य॒ पुनः॒ पुन॑ रभ्य॒वेत्या॒ग्नि म॒ग्नि म॑भ्य॒वेत्य॒ पुनः॒ पुन॑ रभ्य॒वेत्या॒ग्निम् । \newline
50. अ॒भ्य॒वेत्या॒ग्नि म॒ग्नि म॑भ्य॒वेत्या᳚भ्य॒वेत्या॒ग्नि मा॒धाया॒धाया॒ग्नि म॑भ्य॒वेत्या᳚भ्य॒वेत्या॒ग्नि मा॒धाय॑ । \newline
51. अ॒भ्य॒वेत्येत्य॑भि - अ॒वेत्य॑ । \newline
52. अ॒ग्नि मा॒धाया॒धाया॒ग्नि म॒ग्नि मा॒धायै॒ता ने॒ता ना॒धाया॒ग्नि म॒ग्नि मा॒धायै॒तान् । \newline
53. आ॒धायै॒ता ने॒ता ना॒धाया॒धायै॒तान्. होमा॒न्॒. होमा॑ ने॒ता ना॒धाया॒धायै॒तान्. होमान्॑ । \newline
54. आ॒धायेत्या᳚ - धाय॑ । \newline
55. ए॒तान्. होमा॒न्॒. होमा॑ ने॒ता ने॒तान्. होमा॑ नजुहवु रजुहवु॒र्॒. होमा॑ ने॒ता ने॒तान्. होमा॑ नजुहवुः । \newline
56. होमा॑ नजुहवु रजुहवु॒र्॒. होमा॒न्॒. होमा॑ नजुहवु॒स्ते ते॑ ऽजुहवु॒र्॒. होमा॒न्॒. होमा॑ नजुहवु॒स्ते । \newline
57. अ॒जु॒ह॒वु॒स्ते ते॑ ऽजुहवु रजुहवु॒स्त आ᳚र्द्ध्नुवन् नार्द्ध्नुव॒न् ते॑ ऽजुहवु रजुहवु॒स्त आ᳚र्द्ध्नुवन्न् । \newline
58. त आ᳚र्द्ध्नुवन् नार्द्ध्नुव॒न् ते त आ᳚र्द्ध्नुव॒न् ते त आ᳚र्द्ध्नुव॒न् ते त आ᳚र्द्ध्नुव॒न् ते । \newline
59. आ॒र्द्ध्नु॒व॒न् ते त आ᳚र्द्ध्नुवन् नार्द्ध्नुव॒न् ते सु॑व॒र्गꣳ सु॑व॒र्गम् त आ᳚र्द्ध्नुवन् नार्द्ध्नुव॒न् ते सु॑व॒र्गम् । \newline
60. ते सु॑व॒र्गꣳ सु॑व॒र्गम् ते ते सु॑व॒र्गम् ॅलो॒कम् ॅलो॒कꣳ सु॑व॒र्गम् ते ते सु॑व॒र्गम् ॅलो॒कम् । \newline
61. सु॒व॒र्गम् ॅलो॒कम् ॅलो॒कꣳ सु॑व॒र्गꣳ सु॑व॒र्गम् ॅलो॒क मा॑यन् नायन्न् ॅलो॒कꣳ सु॑व॒र्गꣳ सु॑व॒र्गम् ॅलो॒क मा॑यन्न् । \newline
62. सु॒व॒र्गमिति॑ सुवः - गम् । \newline
63. लो॒क मा॑यन् नायन्न् ॅलो॒कम् ॅलो॒क मा॑य॒न्॒. यो य आ॑यन्न् ॅलो॒कम् ॅलो॒क मा॑य॒न्॒. यः । \newline
64. आ॒य॒न्॒. यो य आ॑यन् नाय॒न्॒. यः प॑रा॒चीन॑म् परा॒चीनं॒ ॅय आ॑यन् नाय॒न्॒. यः प॑रा॒चीन᳚म् । \newline
65. यः प॑रा॒चीन॑म् परा॒चीनं॒ ॅयो यः प॑रा॒चीन॑म् पुनरा॒धेया᳚त् पुनरा॒धेया᳚त् परा॒चीनं॒ ॅयो यः प॑रा॒चीन॑म् पुनरा॒धेया᳚त् । \newline
66. प॒रा॒चीन॑म् पुनरा॒धेया᳚त् पुनरा॒धेया᳚त् परा॒चीन॑म् परा॒चीन॑म् पुनरा॒धेया॑द॒ग्नि म॒ग्निम् पु॑नरा॒धेया᳚त् परा॒चीन॑म् परा॒चीन॑म् पुनरा॒धेया॑द॒ग्निम् । \newline
67. पु॒न॒रा॒धेया॑द॒ग्नि म॒ग्निम् पु॑नरा॒धेया᳚त् पुनरा॒धेया॑द॒ग्नि मा॒दधी॑ता॒ दधी॑ता॒ग्निम् पु॑नरा॒धेया᳚त् पुनरा॒धेया॑द॒ग्नि मा॒दधी॑त । \newline
68. पु॒न॒रा॒धेया॒दिति॑ पुनः - आ॒धेया᳚त् । \newline
69. अ॒ग्नि मा॒दधी॑ता॒ दधी॑ता॒ग्नि म॒ग्नि मा॒दधी॑त॒ स स आ॒दधी॑ता॒ग्नि म॒ग्नि मा॒दधी॑त॒ सः । \newline
70. आ॒दधी॑त॒ स स आ॒दधी॑ता॒दधी॑त॒ स ए॒ता ने॒तान् थ्स आ॒दधी॑ता॒दधी॑त॒ स ए॒तान् । \newline
71. आ॒दधी॒तेत्या᳚ - दधी॑त । \newline
72. स ए॒ता ने॒तान् थ्स स ए॒तान्. होमा॒न्॒. होमा॑ ने॒तान् थ्स स ए॒तान्. होमान्॑ । \newline
73. ए॒तान्. होमा॒न्॒. होमा॑ ने॒ता ने॒तान्. होमा᳚न् जुहुयाज् जुहुया॒द्धोमा॑ ने॒ता ने॒तान्. होमा᳚न् जुहुयात् । \newline
74. होमा᳚न् जुहुयाज् जुहुया॒द्धोमा॒न्॒. होमा᳚न् जुहुया॒द् यां ॅयाम् जु॑हुया॒द्धोमा॒न्॒. होमा᳚न् जुहुया॒द् याम् । \newline
75. जु॒हु॒या॒द् यां ॅयाम् जु॑हुयाज् जुहुया॒द् या मे॒वैव याम् जु॑हुयाज् जुहुया॒द् या मे॒व । \newline
76. या मे॒वैव यां ॅया मे॒वादि॒त्या आ॑दि॒त्या ए॒व यां ॅया मे॒वादि॒त्याः । \newline
77. ए॒वादि॒त्या आ॑दि॒त्या ए॒वैवादि॒त्या ऋद्धि॒ मृद्धि॑ मादि॒त्या ए॒वैवादि॒त्या ऋद्धि᳚म् । \newline
78. आ॒दि॒त्या ऋद्धि॒ मृद्धि॑ मादि॒त्या आ॑दि॒त्या ऋद्धि॒ मार्द्ध्नु॑व॒न् नार्द्ध्नु॑व॒न्न् ऋद्धि॑ मादि॒त्या आ॑दि॒त्या ऋद्धि॒ मार्द्ध्नु॑वन्न् । \newline
79. ऋद्धि॒ मार्द्ध्नु॑व॒न् नार्द्ध्नु॑व॒न्न् ऋद्धि॒ मृद्धि॒ मार्द्ध्नु॑व॒न् ताम् ता मार्द्ध्नु॑व॒न्न् ऋद्धि॒ मृद्धि॒ मार्द्ध्नु॑व॒न् ताम् । \newline
80. आर्द्ध्नु॑व॒न् ताम् ता मार्द्ध्नु॑व॒न् नार्द्ध्नु॑व॒न् ता मे॒वैव ता मार्द्ध्नु॑व॒न् नार्द्ध्नु॑व॒न् ता मे॒व । \newline
81. ता मे॒वैव ताम् ता मे॒व र्‌द्ध्नो᳚ त्यृद्ध्नोत्ये॒व ताम् ता मे॒व र्‌द्ध्नो॑ति । \newline
82. ए॒व र्‌द्ध्नो᳚ त्यृद्ध्नोत्ये॒वैव र्‌द्ध्नो॑ति । \newline
83. ऋ॒द्ध्नो॒तीत्यृ॑द्ध्नोति । \newline
\pagebreak
\markright{ TS 1.5.5.1  \hfill https://www.vedavms.in \hfill}
\addcontentsline{toc}{section}{ TS 1.5.5.1 }
\section*{ TS 1.5.5.1 }

\textbf{TS 1.5.5.1 } \newline
\textbf{Samhita Paata} \newline

उ॒प॒प्र॒यन्तो॑ अद्ध्व॒रं मन्त्रं॑ ॅवोचेमा॒ग्नये᳚ । आ॒रे अ॒स्मे च॑ शृण्व॒ते ॥ अ॒स्य प्र॒त्नामनु॒ द्युतꣳ॑ शु॒क्रं दु॑दुह्रे॒ अह्र॑यः । पयः॑ सहस्र॒सामृषिं᳚ ॥ अ॒ग्निर् मू॒र्द्धा दि॒वः क॒कुत्पतिः॑ पृथि॒व्या अ॒यं । अ॒पाꣳ रेताꣳ॑सि जिन्वति ॥ अ॒यमि॒ह प्र॑थ॒मो धा॑यि धा॒तृभि॒र्॒. होता॒ यजि॑ष्ठो अध्व॒रेष्वीड्यः॑ ॥ यमप्न॑वानो॒ भृग॑वो विरुरु॒चुर्वने॑षु चि॒त्रं ॅवि॒भुवं॑ ॅवि॒शेवि॑शे ॥ उ॒भा वा॑मिन्द्राग्नी आहु॒वद्ध्या॑ - [ ] \newline

\textbf{Pada Paata} \newline

उ॒प॒प्र॒यन्त॒ इत्यु॑प - प्र॒यन्तः॑ । अ॒द्ध्व॒रम् । मन्त्र᳚म् । वो॒चे॒म॒ । अ॒ग्नये᳚ ॥ आ॒रे । अ॒स्मे इति॑ । च॒ । शृ॒ण्व॒ते ॥ अ॒स्य । प्र॒त्नाम् । अन्विति॑ । द्युत᳚म् । शु॒क्रम् । दु॒दु॒ह्रे॒ । अह्र॑यः ॥ पयः॑ । स॒ह॒स्र॒सामिति॑ सहस्र - साम् । ऋषि᳚म् ॥ अ॒ग्निः । मू॒र्धा । दि॒वः । क॒कुत् । पतिः॑ । पृ॒थि॒व्याः । अ॒यम् ॥ अ॒पाम् । रेताꣳ॑सि । जि॒न्व॒ति॒ ॥ अ॒यम् । इ॒ह । प्र॒थ॒मः । धा॒यि॒ । धा॒तृभि॒रिति॑ धा॒तृ - भिः॒ । होता᳚ । यजि॑ष्ठः । अ॒द्ध्व॒रेषु॑ । ईड्यः॑ ॥ यम् । अप्न॑वानः । भृग॑वः । वि॒रु॒रु॒चुरिति॑ वि - रु॒रु॒चुः । वने॑षु । चि॒त्रम् । वि॒भुव॒मिति॑ वि - भुव᳚म् । वि॒शेवि॑श॒ इति॑ वि॒शे - वि॒शे॒ ॥ उ॒भा । वा॒म् । इ॒न्द्रा॒ग्नी॒ इती᳚न्द्र - अ॒ग्नी॒ । आ॒हु॒वद्ध्यै᳚ ।  \newline


\textbf{Krama Paata} \newline

उ॒प॒प्र॒यन्तो॑ अद्ध्व॒रम् । उ॒प॒प्र॒यन्त॒ इत्यु॑प - प्र॒यन्तः॑ । अ॒द्ध्व॒रम् मन्त्र᳚म् । मन्त्रं॑ ॅवोचेम । वो॒चे॒मा॒ग्नये᳚ । अ॒ग्नय॒ इत्य॒ग्नये᳚ ॥ आ॒रे अ॒स्मे । अ॒स्मे च॑ । अ॒स्मे इत्य॒स्मे । च॒ शृ॒ण्व॒ते । शृ॒ण्व॒त इति॑ शृण्व॒ते ॥ अ॒स्य प्र॒त्नाम् । प्र॒त्नामनु॑ । अनु॒ द्युत᳚म् । द्युतꣳ॑ शु॒क्रम् । शु॒क्रम् दु॑दुह्रे । दु॒दु॒ह्रे॒ अह्र॑यः । अह्र॑य॒ इत्यह्र॑यः ॥ पयः॑ सहस्र॒साम् । स॒ह॒स्र॒सामृषि᳚म् । स॒ह॒स्र॒सामिति॑ सहस्र - साम् । ऋषि॒मित्यृषि᳚म् ॥ अ॒ग्निर् मू॒र्द्धा । मू॒र्द्धा दि॒वः । दि॒वः क॒कुत् । क॒कुत् पतिः॑ । पतिः॑ पृथि॒व्याः । पृ॒थि॒व्या अ॒यम् । अ॒यमित्य॒यम् ॥ अ॒पाꣳ रेताꣳ॑सि । रेताꣳ॑सि जिन्वति । जि॒न्व॒तीति॑ जिन्वति ॥ अ॒यमि॒ह । इ॒ह प्र॑थ॒मः । प्र॒थ॒मो धा॑यि । धा॒यि॒ धा॒तृभिः॑ । धा॒तृभि॒र्. होता᳚ । धा॒तृभि॒रिति॑ धा॒तृ - भिः॒ । होता॒ यजि॑ष्ठः । यजि॑ष्ठो अद्ध्व॒रेषु॑ । अ॒द्ध्व॒रेष्टीड्यः॑ । ईड्य॒ इतीड्यः॑ ॥ यमप्न॑वानः । अप्न॑वानो॒ भृग॑वः । भृग॑वो विरुरु॒चुः । वि॒रु॒रु॒चुर् वने॑षु । वि॒रु॒रु॒चुरिति॑ वि - रु॒रु॒चुः । वने॑षु चि॒त्रम् । चि॒त्रं ॅवि॒भुव᳚म् । वि॒भुवं॑ ॅवि॒शेवि॑शे । वि॒भुव॒मिति॑ वि - भुव᳚म् । वि॒शे,वि॑श॒ इति॑ वि॒शे - वि॒शे॒ ॥ उ॒भा वा᳚म् । वा॒मि॒न्द्रा॒ग्नी॒ । इ॒न्द्रा॒ग्नी॒ आ॒हु॒वद्ध्यै᳚ । इ॒न्दा॒ग्नी॒ इती᳚न्द्र - अ॒ग्नी॒ । आ॒हु॒वद्ध्या॑ उ॒भा \newline

\textbf{Jatai Paata} \newline

1. उ॒प॒प्र॒यन्तो॑ अद्ध्व॒र म॑द्ध्व॒र मु॑पप्र॒यन्त॑ उपप्र॒यन्तो॑ अद्ध्व॒रम् । \newline
2. उ॒प॒प्र॒यन्त॒ इत्यु॑प - प्र॒यन्तः॑ । \newline
3. अ॒द्ध्व॒रम् मन्त्र॒म् मन्त्र॑ मद्ध्व॒र म॑द्ध्व॒रम् मन्त्र᳚म् । \newline
4. मन्त्रं॑ ॅवोचेम वोचेम॒ मन्त्र॒म् मन्त्रं॑ ॅवोचेम । \newline
5. वो॒चे॒मा॒ग्नये॑ अ॒ग्नये॑ वोचेम वोचेमा॒ग्नये᳚ । \newline
6. अ॒ग्नय॒ इत्य॒ग्नये᳚ । \newline
7. आ॒रे अ॒स्मे अ॒स्मे आ॒र आ॒रे अ॒स्मे । \newline
8. अ॒स्मे च॑ चा॒स्मे अ॒स्मे च॑ । \newline
9. अ॒स्मे इत्य॒स्मे । \newline
10. च॒ शृ॒ण्व॒ते शृ॑ण्व॒ते च॑ च शृण्व॒ते । \newline
11. शृ॒ण्व॒त इति॑ शृण्व॒ते । \newline
12. अ॒स्य प्र॒त्नाम् प्र॒त्ना म॒स्यास्य प्र॒त्नाम् । \newline
13. प्र॒त्ना मन्वनु॑ प्र॒त्नाम् प्र॒त्ना मनु॑ । \newline
14. अनु॒ द्युत॒म् द्युत॒ मन्वनु॒ द्युत᳚म् । \newline
15. द्युत(ग्म्॑) शु॒क्रꣳ शु॒क्रम् द्युत॒म् द्युत(ग्म्॑) शु॒क्रम् । \newline
16. शु॒क्रम् दु॑दुह्रे दुदुह्रे शु॒क्रꣳ शु॒क्रम् दु॑दुह्रे । \newline
17. दु॒दु॒ह्रे॒ अह्र॑यो॒ अह्र॑यो दुदुह्रे दुदुह्रे॒ अह्र॑यः । \newline
18. अह्र॑य॒ इत्यह्र॑यः । \newline
19. पयः॑ सहस्र॒साꣳ स॑हस्र॒साम् पयः॒ पयः॑ सहस्र॒साम् । \newline
20. स॒ह॒स्र॒सा मृषि॒ मृषि(ग्म्॑) सहस्र॒साꣳ स॑हस्र॒सा मृषि᳚म् । \newline
21. स॒ह॒स्र॒सामिति॑ सहस्र - साम् । \newline
22. ऋषि॒मित्यृषि᳚म् । \newline
23. अ॒ग्निर् मू॒र्द्धा मू॒र्द्धा ऽग्निर॒ग्निर् मू॒र्द्धा । \newline
24. मू॒र्द्धा दि॒वो दि॒वो मू॒र्द्धा मू॒र्द्धा दि॒वः । \newline
25. दि॒वः क॒कुत् क॒कुद् दि॒वो दि॒वः क॒कुत् । \newline
26. क॒कुत् पति॒ष् पतिः॑ क॒कुत् क॒कुत् पतिः॑ । \newline
27. पतिः॑ पृथि॒व्याः पृ॑थि॒व्या स्पति॒ष् पतिः॑ पृथि॒व्याः । \newline
28. पृ॒थि॒व्या अ॒य म॒यम् पृ॑थि॒व्याः पृ॑थि॒व्या अ॒यम् । \newline
29. अ॒यमित्य॒यम् । \newline
30. अ॒पाꣳ रेता(ग्म्॑)सि॒ रेता(ग्ग्॑)स्य॒पा म॒पाꣳ रेता(ग्म्॑)सि । \newline
31. रेता(ग्म्॑)सि जिन्वति जिन्वति॒ रेता(ग्म्॑)सि॒ रेता(ग्म्॑)सि जिन्वति । \newline
32. जि॒न्व॒तीति॑ जिन्वति । \newline
33. अ॒य मि॒हे हाय म॒य मि॒ह । \newline
34. इ॒ह प्र॑थ॒मः प्र॑थ॒म इ॒हे ह प्र॑थ॒मः । \newline
35. प्र॒थ॒मो धा॑यि धायि प्रथ॒मः प्र॑थ॒मो धा॑यि । \newline
36. धा॒यि॒ धा॒तृभि॑र् धा॒तृभि॑र् धायि धायि धा॒तृभिः॑ । \newline
37. धा॒तृभि॒र्॒. होता॒ होता॑ धा॒तृभि॑र् धा॒तृभि॒र्॒. होता᳚ । \newline
38. धा॒तृभि॒रिति॑ धा॒तृ - भिः॒ । \newline
39. होता॒ यजि॑ष्ठो॒ यजि॑ष्ठो॒ होता॒ होता॒ यजि॑ष्ठः । \newline
40. यजि॑ष्ठो अद्ध्व॒रेष्व॑द्ध्व॒रेषु॒ यजि॑ष्ठो॒ यजि॑ष्ठो अद्ध्व॒रेषु॑ । \newline
41. अ॒द्ध्व॒रेष्वीड्य॒ ईड्यो॑ अद्ध्व॒रेष्व॑द्ध्व॒रेष्वीड्यः॑ । \newline
42. ईड्य॒ इतीड्यः॑ । \newline
43. य मप्न॑वानो॒ अप्न॑वानो॒ यं ॅय मप्न॑वानः । \newline
44. अप्न॑वानो॒ भृग॑वो॒ भृग॑वो॒ अप्न॑वानो॒ अप्न॑वानो॒ भृग॑वः । \newline
45. भृग॑वो विरुरु॒चुर् वि॑रुरु॒चुर् भृग॑वो॒ भृग॑वो विरुरु॒चुः । \newline
46. वि॒रु॒रु॒चुर् वने॑षु॒ वने॑षु विरुरु॒चुर् वि॑रुरु॒चुर् वने॑षु । \newline
47. वि॒रु॒रु॒चुरिति॑ वि - रु॒रु॒चुः । \newline
48. वने॑षु चि॒त्रम् चि॒त्रं ॅवने॑षु॒ वने॑षु चि॒त्रम् । \newline
49. चि॒त्रं ॅवि॒भुवं॑ ॅवि॒भुव॑म् चि॒त्रम् चि॒त्रं ॅवि॒भुव᳚म् । \newline
50. वि॒भुवं॑ ॅवि॒शेवि॑शे वि॒शेवि॑शे वि॒भुवं॑ ॅवि॒भुवं॑ ॅवि॒शेवि॑शे । \newline
51. वि॒भुव॒मिति॑ वि - भुव᳚म् । \newline
52. वि॒शेवि॑श॒ इति॑ वि॒शे - वि॒शे॒ । \newline
53. उ॒भा वां᳚ ॅवा मु॒भोभा वा᳚म् । \newline
54. वा॒ मि॒न्द्रा॒ग्नी॒ इ॒न्द्रा॒ग्नी॒ वां॒ ॅवा॒ मि॒न्द्रा॒ग्नी॒ । \newline
55. इ॒न्द्रा॒ग्नी॒ आ॒हु॒वद्ध्या॑ आहु॒वद्ध्या॑ इन्द्राग्नी इन्द्राग्नी आहु॒वद्ध्यै᳚ । \newline
56. इ॒न्द्रा॒ग्नी॒ इती᳚न्द्र - अ॒ग्नी॒ । \newline
57. आ॒हु॒वद्ध्या॑ उ॒भोभा ऽऽहु॒वद्ध्या॑ आहु॒वद्ध्या॑ उ॒भा । \newline

\textbf{Ghana Paata } \newline

1. उ॒प॒प्र॒यन्तो॑ अद्ध्व॒र म॑द्ध्व॒र मु॑पप्र॒यन्त॑ उपप्र॒यन्तो॑ अद्ध्व॒रम् मन्त्र॒म् मन्त्र॑ मद्ध्व॒र मु॑पप्र॒यन्त॑ उपप्र॒यन्तो॑ अद्ध्व॒रम् मन्त्र᳚म् । \newline
2. उ॒प॒प्र॒यन्त॒ इत्यु॑प - प्र॒यन्तः॑ । \newline
3. अ॒द्ध्व॒रम् मन्त्र॒म् मन्त्र॑ मद्ध्व॒र म॑द्ध्व॒रम् मन्त्रं॑ ॅवोचेम वोचेम॒ मन्त्र॑ मद्ध्व॒र म॑द्ध्व॒रम् मन्त्रं॑ ॅवोचेम । \newline
4. मन्त्रं॑ ॅवोचेम वोचेम॒ मन्त्र॒म् मन्त्रं॑ ॅवोचेमा॒ग्नये॑ अ॒ग्नये॑ वोचेम॒ मन्त्र॒म् मन्त्रं॑ ॅवोचेमा॒ग्नये᳚ । \newline
5. वो॒चे॒मा॒ग्नये॑ अ॒ग्नये॑ वोचेम वोचेमा॒ग्नये᳚ । \newline
6. अ॒ग्नय॒ इत्य॒ग्नये᳚ । \newline
7. आ॒रे अ॒स्मे अ॒स्मे आ॒र आ॒रे अ॒स्मे च॑ चा॒स्मे आ॒र आ॒रे अ॒स्मे च॑ । \newline
8. अ॒स्मे च॑ चा॒स्मे अ॒स्मे च॑ शृण्व॒ते शृ॑ण्व॒ते चा॒स्मे अ॒स्मे च॑ शृण्व॒ते । \newline
9. अ॒स्मे इत्य॒स्मे । \newline
10. च॒ शृ॒ण्व॒ते शृ॑ण्व॒ते च॑ च शृण्व॒ते । \newline
11. शृ॒ण्व॒त इति॑ शृण्व॒ते । \newline
12. अ॒स्य प्र॒त्नाम् प्र॒त्ना म॒स्यास्य प्र॒त्ना मन्वनु॑ प्र॒त्ना म॒स्यास्य प्र॒त्ना मनु॑ । \newline
13. प्र॒त्ना मन्वनु॑ प्र॒त्नाम् प्र॒त्ना मनु॒ द्युत॒म् द्युत॒ मनु॑ प्र॒त्नाम् प्र॒त्ना मनु॒ द्युत᳚म् । \newline
14. अनु॒ द्युत॒म् द्युत॒ मन्वनु॒ द्युत(ग्म्॑) शु॒क्रꣳ शु॒क्रम् द्युत॒ मन्वनु॒ द्युत(ग्म्॑) शु॒क्रम् । \newline
15. द्युत(ग्म्॑) शु॒क्रꣳ शु॒क्रम् द्युत॒म् द्युत(ग्म्॑) शु॒क्रम् दु॑दुह्रे दुदुह्रे शु॒क्रम् द्युत॒म् द्युत(ग्म्॑) शु॒क्रम् दु॑दुह्रे । \newline
16. शु॒क्रम् दु॑दुह्रे दुदुह्रे शु॒क्रꣳ शु॒क्रम् दु॑दुह्रे॒ अह्र॑यो॒ अह्र॑यो दुदुह्रे शु॒क्रꣳ शु॒क्रम् दु॑दुह्रे॒ अह्र॑यः । \newline
17. दु॒दु॒ह्रे॒ अह्र॑यो॒ अह्र॑यो दुदुह्रे दुदुह्रे॒ अह्र॑यः । \newline
18. अह्र॑य॒ इत्यह्र॑यः । \newline
19. पयः॑ सहस्र॒साꣳ स॑हस्र॒साम् पयः॒ पयः॑ सहस्र॒सा मृषि॒ मृषि(ग्म्॑) सहस्र॒साम् पयः॒ पयः॑ सहस्र॒सा मृषि᳚म् । \newline
20. स॒ह॒स्र॒सा मृषि॒ मृषि(ग्म्॑) सहस्र॒साꣳ स॑हस्र॒सा मृषि᳚म् । \newline
21. स॒ह॒स्र॒सामिति॑ सहस्र - साम् । \newline
22. ऋषि॒मित्यृषि᳚म् । \newline
23. अ॒ग्निर् मू॒र्द्धा मू॒र्द्धा ऽग्निर॒ग्निर् मू॒र्द्धा दि॒वो दि॒वो मू॒र्द्धा ऽग्निर॒ग्निर् मू॒र्द्धा दि॒वः । \newline
24. मू॒र्द्धा दि॒वो दि॒वो मू॒र्द्धा मू॒र्द्धा दि॒वः क॒कुत् क॒कुद् दि॒वो मू॒र्द्धा मू॒र्द्धा दि॒वः क॒कुत् । \newline
25. दि॒वः क॒कुत् क॒कुद् दि॒वो दि॒वः क॒कुत् पति॒ष्पतिः॑ क॒कुद् दि॒वो दि॒वः क॒कुत् पतिः॑ । \newline
26. क॒कुत् पति॒ष्पतिः॑ क॒कुत् क॒कुत् पतिः॑ पृथि॒व्याः पृ॑थि॒व्या स्पतिः॑ क॒कुत् क॒कुत् पतिः॑ पृथि॒व्याः । \newline
27. पतिः॑ पृथि॒व्याः पृ॑थि॒व्या स्पति॒ष्पतिः॑ पृथि॒व्या अ॒य म॒यम् पृ॑थि॒व्या स्पति॒ष्पतिः॑ पृथि॒व्या अ॒यम् । \newline
28. पृ॒थि॒व्या अ॒य म॒यम् पृ॑थि॒व्याः पृ॑थि॒व्या अ॒यम् । \newline
29. अ॒यमित्य॒यम् । \newline
30. अ॒पाꣳ रेता(ग्म्॑)सि॒ रेता(ग्ग्॑)स्य॒पा म॒पाꣳ रेता(ग्म्॑)सि जिन्वति जिन्वति॒ रेता(ग्ग्॑)स्य॒पा म॒पाꣳ रेता(ग्म्॑)सि जिन्वति । \newline
31. रेता(ग्म्॑)सि जिन्वति जिन्वति॒ रेता(ग्म्॑)सि॒ रेता(ग्म्॑)सि जिन्वति । \newline
32. जि॒न्व॒तीति॑ जिन्वति । \newline
33. अ॒य मि॒हे हाय म॒य मि॒ह प्र॑थ॒मः प्र॑थ॒म इ॒हाय म॒य मि॒ह प्र॑थ॒मः । \newline
34. इ॒ह प्र॑थ॒मः प्र॑थ॒म इ॒हे ह प्र॑थ॒मो धा॑यि धायि प्रथ॒म इ॒हे ह प्र॑थ॒मो धा॑यि । \newline
35. प्र॒थ॒मो धा॑यि धायि प्रथ॒मः प्र॑थ॒मो धा॑यि धा॒तृभि॑र् धा॒तृभि॑र् धायि प्रथ॒मः प्र॑थ॒मो धा॑यि धा॒तृभिः॑ । \newline
36. धा॒यि॒ धा॒तृभि॑र् धा॒तृभि॑र् धायि धायि धा॒तृभि॒र्॒. होता॒ होता॑ धा॒तृभि॑र् धायि धायि धा॒तृभि॒र्॒. होता᳚ । \newline
37. धा॒तृभि॒र्॒. होता॒ होता॑ धा॒तृभि॑र् धा॒तृभि॒र्॒. होता॒ यजि॑ष्ठो॒ यजि॑ष्ठो॒ होता॑ धा॒तृभि॑र् धा॒तृभि॒र्॒. होता॒ यजि॑ष्ठः । \newline
38. धा॒तृभि॒रिति॑ धा॒तृ - भिः॒ । \newline
39. होता॒ यजि॑ष्ठो॒ यजि॑ष्ठो॒ होता॒ होता॒ यजि॑ष्ठो अद्ध्व॒रे ष्व॑द्ध्व॒रेषु॒ यजि॑ष्ठो॒ होता॒ होता॒ यजि॑ष्ठो अद्ध्व॒रेषु॑ । \newline
40. यजि॑ष्ठो अद्ध्व॒रेष्व॑द्ध्व॒रेषु॒ यजि॑ष्ठो॒ यजि॑ष्ठो अद्ध्व॒रेष्वीड्य॒ ईड्यो॑ अद्ध्व॒रेषु॒ यजि॑ष्ठो॒ यजि॑ष्ठो अद्ध्व॒रेष्वीड्यः॑ । \newline
41. अ॒द्ध्व॒रेष्वीड्य॒ ईड्यो॑ अद्ध्व॒रे ष्व॑द्ध्व॒रेष्वीड्यः॑ । \newline
42. ईड्य॒ इतीड्यः॑ । \newline
43. य मप्न॑वानो॒ अप्न॑वानो॒ यं ॅय मप्न॑वानो॒ भृग॑वो॒ भृग॑वो॒ अप्न॑वानो॒ यं ॅय मप्न॑वानो॒ भृग॑वः । \newline
44. अप्न॑वानो॒ भृग॑वो॒ भृग॑वो॒ अप्न॑वानो॒ अप्न॑वानो॒ भृग॑वो विरुरु॒चुर् वि॑रुरु॒चुर् भृग॑वो॒ अप्न॑वानो॒ अप्न॑वानो॒ भृग॑वो विरुरु॒चुः । \newline
45. भृग॑वो विरुरु॒चुर् वि॑रुरु॒चुर् भृग॑वो॒ भृग॑वो विरुरु॒चुर् वने॑षु॒ वने॑षु विरुरु॒चुर् भृग॑वो॒ भृग॑वो विरुरु॒चुर् वने॑षु । \newline
46. वि॒रु॒रु॒चुर् वने॑षु॒ वने॑षु विरुरु॒चुर् वि॑रुरु॒चुर् वने॑षु चि॒त्रम् चि॒त्रं ॅवने॑षु विरुरु॒चुर् वि॑रुरु॒चुर् वने॑षु चि॒त्रम् । \newline
47. वि॒रु॒रु॒चुरिति॑ वि - रु॒रु॒चुः । \newline
48. वने॑षु चि॒त्रम् चि॒त्रं ॅवने॑षु॒ वने॑षु चि॒त्रं ॅवि॒भुवं॑ ॅवि॒भुव॑म् चि॒त्रं ॅवने॑षु॒ वने॑षु चि॒त्रं ॅवि॒भुव᳚म् । \newline
49. चि॒त्रं ॅवि॒भुवं॑ ॅवि॒भुव॑म् चि॒त्रम् चि॒त्रं ॅवि॒भुवं॑ ॅवि॒शेवि॑शे वि॒शेवि॑शे वि॒भुव॑म् चि॒त्रम् चि॒त्रं ॅवि॒भुवं॑ ॅवि॒शेवि॑शे । \newline
50. वि॒भुवं॑ ॅवि॒शेवि॑शे वि॒शेवि॑शे वि॒भुवं॑ ॅवि॒भुवं॑ ॅवि॒शेवि॑शे । \newline
51. वि॒भुव॒मिति॑ वि - भुव᳚म् । \newline
52. वि॒शेवि॑श॒ इति॑ वि॒शे - वि॒शे॒ । \newline
53. उ॒भा वां᳚ ॅवा मु॒भोभा वा॑ मिन्द्राग्नी इन्द्राग्नी वा मु॒भोभा वा॑ मिन्द्राग्नी । \newline
54. वा॒ मि॒न्द्रा॒ग्नी॒ इ॒न्द्रा॒ग्नी॒ वां॒ ॅवा॒ मि॒न्द्रा॒ग्नी॒ आ॒हु॒वद्ध्या॑ आहु॒वद्ध्या॑ इन्द्राग्नी वां ॅवा मिन्द्राग्नी आहु॒वद्ध्यै᳚ । \newline
55. इ॒न्द्रा॒ग्नी॒ आ॒हु॒वद्ध्या॑ आहु॒वद्ध्या॑ इन्द्राग्नी इन्द्राग्नी आहु॒वद्ध्या॑ उ॒भोभा ऽऽहु॒वद्ध्या॑ इन्द्राग्नी इन्द्राग्नी आहु॒वद्ध्या॑ उ॒भा । \newline
56. इ॒न्द्रा॒ग्नी॒ इती᳚न्द्र - अ॒ग्नी॒ । \newline
57. आ॒हु॒वद्ध्या॑ उ॒भोभा ऽऽहु॒वद्ध्या॑ आहु॒वद्ध्या॑ उ॒भा राध॑सो॒ राध॑स उ॒भा ऽऽहु॒वद्ध्या॑ आहु॒वद्ध्या॑ उ॒भा राध॑सः । \newline
\pagebreak
\markright{ TS 1.5.5.2  \hfill https://www.vedavms.in \hfill}
\addcontentsline{toc}{section}{ TS 1.5.5.2 }
\section*{ TS 1.5.5.2 }

\textbf{TS 1.5.5.2 } \newline
\textbf{Samhita Paata} \newline

उ॒भा राध॑सः स॒ह मा॑द॒यद्ध्यै᳚ । उ॒भा दा॒तारा॑वि॒षाꣳ र॑यी॒णामु॒भा वाज॑स्य सा॒तये॑ हुवे वां ॥ अ॒यं ते॒ योनि॑र्. ऋ॒त्वियो॒ यतो॑ जा॒तो अरो॑चथाः । तं जा॒नन्न॑ग्न॒ आ रो॒हाथा॑ नो वर्द्धया र॒यिं ॥ अग्न॒ आयूꣳ॑षि पवस॒ आ सु॒वोर्ज॒मिषं॑ च नः । आ॒रे बा॑धस्व दु॒च्छुनां᳚ ॥ अग्ने॒ पव॑स्व॒ स्वपा॑ अ॒स्मे वर्चः॑ सु॒वीर्यं᳚ । दध॒त्पोषꣳ॑ र॒यिं - [ ] \newline

\textbf{Pada Paata} \newline

उ॒भा । राध॑सः । स॒ह । मा॒द॒यद्ध्यै᳚ ॥ उ॒भा । दा॒तारौ᳚ । इ॒षाम् । र॒यी॒णाम् । उ॒भा । वाज॑स्य । सा॒तये᳚ । हु॒वे॒ । वा॒म् ॥ अ॒यम् । ते॒ । योनिः॑ । ऋ॒त्वियः॑ । यतः॑ । जा॒तः । अरो॑चथाः ॥ तम् । जा॒नन्न् । अ॒ग्ने॒ । एति॑ । रो॒ह॒ । अथ॑ । नः॒ । व॒र्द्ध॒य॒ । र॒यिम् ॥ अग्ने᳚ । आयूꣳ॑षि । प॒व॒से॒ । एति॑ । सु॒व॒ । ऊर्ज᳚म् । इष᳚म् । च॒ । नः॒ ॥ आ॒रे । बा॒ध॒स्व॒ । दु॒च्छुना᳚म् ॥ अग्ने᳚ । पव॑स्व । स्वपा॒ इति॑ सु - अपाः᳚ । अ॒स्मे इति॑ । वर्चः॑ । सु॒वीर्य॒मिति॑ सु - वीर्य᳚म् ॥ दध॑त् । पोष᳚म् । र॒यिम् ।  \newline


\textbf{Krama Paata} \newline

उ॒भा राध॑सः । राध॑सः स॒ह । स॒ह मा॑द॒यद्ध्यै᳚ । मा॒द॒यद्ध्या॒ इति॑ माद॒यद्ध्यै᳚ ॥ उ॒भा दा॒तारौ᳚ । दा॒तारा॑वि॒षाम् । इ॒षाꣳ र॑यी॒णाम् । र॒यी॒णामु॒भा । उ॒भा वाज॑स्य । वाज॑स्य सा॒तये᳚ । सा॒तये॑ हुवे । हु॒वे॒ वा॒म् । वा॒मिति॑ वाम् ॥ अ॒यम् ते᳚ । ते॒ योनिः॑ । योनि॑र्. ऋ॒त्वियः॑ । ऋ॒त्वियो॒ यतः॑ । यतो॑ जा॒तः । जा॒तो अरो॑चथाः । अरो॑चथा॒ इत्यरो॑चथाः ॥ तम् जा॒नन्न् । जा॒नन्न॑ग्ने । अ॒ग्न आ । आ रो॑ह । रो॒हाथ॑ । अथा॑ नः । नो॒ व॒र्द्ध॒य॒ । व॒र्द्ध॒या॒ र॒यिम् । र॒यिमिति॑ र॒यिम् ॥ अग्न॒ आयूꣳ॑षि । आयूꣳ॑षि पवसे । प॒व॒स॒ आ । आ सु॑व । सु॒वोर्ज᳚म् । ऊर्ज॒मिष᳚म् । इष॑म् च । च॒ नः॒ । न॒ इति॑ नः ॥ आ॒रे बा॑धस्व । बा॒ध॒स्व॒ दु॒च्छुना᳚म् ॥ दु॒च्छुना॒मिति॑ दु॒च्छुना᳚म् ॥ अग्ने॒ पव॑स्व । पव॑स्व॒ स्वपाः᳚ । स्वपा॑ अ॒स्मे । स्वपा॒ इति॑ सु - अपाः᳚ । अ॒स्मे वर्चः॑ । अ॒स्मे इत्य॒स्मे । वर्चः॑ सु॒वीर्य᳚म् । सु॒वीर्य॒मिति॑ सु - वीर्य᳚म् ॥ दध॒त् पोष᳚म् । पोषꣳ॑ र॒यिम् । र॒यिम् मयि॑ \newline

\textbf{Jatai Paata} \newline

1. उ॒भा राध॑सो॒ राध॑स उ॒भोभा राध॑सः । \newline
2. राध॑सः स॒ह स॒ह राध॑सो॒ राध॑सः स॒ह । \newline
3. स॒ह मा॑द॒यद्ध्यै॑ माद॒यद्ध्यै॑ स॒ह स॒ह मा॑द॒यद्ध्यै᳚ । \newline
4. मा॒द॒यद्ध्या॒ इति॑ माद॒यद्ध्यै᳚ । \newline
5. उ॒भा दा॒तारौ॑ दा॒तारा॑ वु॒भोभा दा॒तारौ᳚ । \newline
6. दा॒तारा॑ वि॒षा मि॒षाम् दा॒तारौ॑ दा॒तारा॑ वि॒षाम् । \newline
7. इ॒षाꣳ र॑यी॒णाꣳ र॑यी॒णा मि॒षा मि॒षाꣳ र॑यी॒णाम् । \newline
8. र॒यी॒णा मु॒भोभा र॑यी॒णाꣳ र॑यी॒णा मु॒भा । \newline
9. उ॒भा वाज॑स्य॒ वाज॑स्यो॒भोभा वाज॑स्य । \newline
10. वाज॑स्य सा॒तये॑ सा॒तये॒ वाज॑स्य॒ वाज॑स्य सा॒तये᳚ । \newline
11. सा॒तये॑ हुवे हुवे सा॒तये॑ सा॒तये॑ हुवे । \newline
12. हु॒वे॒ वां॒ ॅवा॒(ग्म्॒) हु॒वे॒ हु॒वे॒ वा॒म् । \newline
13. वा॒मिति॑ वाम् । \newline
14. अ॒यम् ते॑ ते॒ ऽय म॒यम् ते᳚ । \newline
15. ते॒ योनि॒र् योनि॑स्ते ते॒ योनिः॑ । \newline
16. योनि॑र्. ऋ॒त्विय॑ ऋ॒त्वियो॒ योनि॒र् योनि॑र्. ऋ॒त्वियः॑ । \newline
17. ऋ॒त्वियो॒ यतो॒ यत॑ ऋ॒त्विय॑ ऋ॒त्वियो॒ यतः॑ । \newline
18. यतो॑ जा॒तो जा॒तो यतो॒ यतो॑ जा॒तः । \newline
19. जा॒तो अरो॑चथा॒ अरो॑चथा जा॒तो जा॒तो अरो॑चथाः । \newline
20. अरो॑चथा॒ इत्यरो॑चथाः । \newline
21. तम् जा॒नन् जा॒नन् तम् तम् जा॒नन्न् । \newline
22. जा॒नन्न॑ग्ने अग्ने जा॒नन् जा॒नन्न॑ग्ने । \newline
23. अ॒ग्न॒ आ ऽग्ने॑ अग्न॒ आ । \newline
24. आ रो॑ह रो॒हा रो॑ह । \newline
25. रो॒हाथाथ॑ रोह रो॒हाथ॑ । \newline
26. अथा॑ नो नो॒ अथाथा॑ नः । \newline
27. नो॒ व॒र्द्ध॒य॒ व॒र्द्ध॒य॒ नो॒ नो॒ व॒र्द्ध॒य॒ । \newline
28. व॒र्द्ध॒या॒ र॒यिꣳ र॒यिं ॅव॑र्द्धय वर्द्धया र॒यिम् । \newline
29. र॒यिमिति॑ र॒यिम् । \newline
30. अग्न॒ आयू॒(ग्ग्॒)ष्यायू॒(ग्ग्॒)ष्यग्ने ऽग्न॒ आयू(ग्म्॑)षि । \newline
31. आयू(ग्म्॑)षि पवसे पवस॒ आयू॒(ग्ग्॒) ष्यायू(ग्म्॑)षि पवसे । \newline
32. प॒व॒स॒ आ प॑वसे पवस॒ आ । \newline
33. आ सु॑व सु॒वा सु॑व । \newline
34. सु॒वोर्ज॒ मूर्ज(ग्म्॑) सुव सु॒वोर्ज᳚म् । \newline
35. ऊर्ज॒ मिष॒ मिष॒ मूर्ज॒ मूर्ज॒ मिष᳚म् । \newline
36. इष॑म् च॒ चे ष॒ मिष॑म् च । \newline
37. च॒ नो॒ न॒श्च॒ च॒ नः॒ । \newline
38. न॒ इति॑ नः । \newline
39. आ॒रे बा॑धस्व बाधस्वा॒र आ॒रे बा॑धस्व । \newline
40. बा॒ध॒स्व॒ दु॒च्छुना᳚म् दु॒च्छुना᳚म् बाधस्व बाधस्व दु॒च्छुना᳚म् । \newline
41. दु॒च्छुना॒मिति॑ दु॒च्छुना᳚म् । \newline
42. अग्ने॒ पव॑स्व॒ पव॒स्वाग्ने ऽग्ने॒ पव॑स्व । \newline
43. पव॑स्व॒ स्वपाः॒ स्वपाः॒ पव॑स्व॒ पव॑स्व॒ स्वपाः᳚ । \newline
44. स्वपा॑ अ॒स्मे अ॒स्मे स्वपाः॒ स्वपा॑ अ॒स्मे । \newline
45. स्वपा॒ इति॑ सु - अपाः᳚ । \newline
46. अ॒स्मे वर्चो॒ वर्चो॑ अ॒स्मे अ॒स्मे वर्चः॑ । \newline
47. अ॒स्मे इत्य॒स्मे । \newline
48. वर्चः॑ सु॒वीर्य(ग्म्॑) सु॒वीर्यं॒ ॅवर्चो॒ वर्चः॑ सु॒वीर्य᳚म् । \newline
49. सु॒वीर्य॒मिति॑ सु - वीर्य᳚म् । \newline
50. दध॒त् पोष॒म् पोष॒म् दध॒द् दध॒त् पोष᳚म् । \newline
51. पोष(ग्म्॑) र॒यिꣳ र॒यिम् पोष॒म् पोष(ग्म्॑) र॒यिम् । \newline
52. र॒यिम् मयि॒ मयि॑ र॒यिꣳ र॒यिम् मयि॑ । \newline

\textbf{Ghana Paata } \newline

1. उ॒भा राध॑सो॒ राध॑स उ॒भोभा राध॑सः स॒ह स॒ह राध॑स उ॒भोभा राध॑सः स॒ह । \newline
2. राध॑सः स॒ह स॒ह राध॑सो॒ राध॑सः स॒ह मा॑द॒यद्ध्यै॑ माद॒यद्ध्यै॑ स॒ह राध॑सो॒ राध॑सः स॒ह मा॑द॒यद्ध्यै᳚ । \newline
3. स॒ह मा॑द॒यद्ध्यै॑ माद॒यद्ध्यै॑ स॒ह स॒ह मा॑द॒यद्ध्यै᳚ । \newline
4. मा॒द॒यद्ध्या॒ इति॑ माद॒यद्ध्यै᳚ । \newline
5. उ॒भा दा॒तारौ॑ दा॒तारा॑ वु॒भोभा दा॒तारा॑ वि॒षा मि॒षाम् दा॒तारा॑ वु॒भोभा दा॒तारा॑ वि॒षाम् । \newline
6. दा॒तारा॑ वि॒षा मि॒षाम् दा॒तारौ॑ दा॒तारा॑ वि॒षाꣳ र॑यी॒णाꣳ र॑यी॒णा मि॒षाम् दा॒तारौ॑ दा॒तारा॑ वि॒षाꣳ र॑यी॒णाम् । \newline
7. इ॒षाꣳ र॑यी॒णाꣳ र॑यी॒णा मि॒षा मि॒षाꣳ र॑यी॒णा मु॒भोभा र॑यी॒णा मि॒षा मि॒षाꣳ र॑यी॒णा मु॒भा । \newline
8. र॒यी॒णा मु॒भोभा र॑यी॒णाꣳ र॑यी॒णा मु॒भा वाज॑स्य॒ वाज॑स्यो॒भा र॑यी॒णाꣳ र॑यी॒णा मु॒भा वाज॑स्य । \newline
9. उ॒भा वाज॑स्य॒ वाज॑स्यो॒भोभा वाज॑स्य सा॒तये॑ सा॒तये॒ वाज॑स्यो॒भोभा वाज॑स्य सा॒तये᳚ । \newline
10. वाज॑स्य सा॒तये॑ सा॒तये॒ वाज॑स्य॒ वाज॑स्य सा॒तये॑ हुवे हुवे सा॒तये॒ वाज॑स्य॒ वाज॑स्य सा॒तये॑ हुवे । \newline
11. सा॒तये॑ हुवे हुवे सा॒तये॑ सा॒तये॑ हुवे वां ॅवाꣳ हुवे सा॒तये॑ सा॒तये॑ हुवे वाम् । \newline
12. हु॒वे॒ वां॒ ॅवा॒(ग्म्॒) हु॒वे॒ हु॒वे॒ वा॒म् । \newline
13. वा॒मिति॑ वाम् । \newline
14. अ॒यम् ते॑ ते॒ ऽय म॒यम् ते॒ योनि॒र् योनि॑स्ते॒ ऽय म॒यम् ते॒ योनिः॑ । \newline
15. ते॒ योनि॒र् योनि॑स्ते ते॒ योनि॑र्. ऋ॒त्विय॑ ऋ॒त्वियो॒ योनि॑स्ते ते॒ योनि॑र्. ऋ॒त्वियः॑ । \newline
16. योनि॑र्. ऋ॒त्विय॑ ऋ॒त्वियो॒ योनि॒र् योनि॑र्. ऋ॒त्वियो॒ यतो॒ यत॑ ऋ॒त्वियो॒ योनि॒र् योनि॑र्. ऋ॒त्वियो॒ यतः॑ । \newline
17. ऋ॒त्वियो॒ यतो॒ यत॑ ऋ॒त्विय॑ ऋ॒त्वियो॒ यतो॑ जा॒तो जा॒तो यत॑ ऋ॒त्विय॑ ऋ॒त्वियो॒ यतो॑ जा॒तः । \newline
18. यतो॑ जा॒तो जा॒तो यतो॒ यतो॑ जा॒तो अरो॑चथा॒ अरो॑चथा जा॒तो यतो॒ यतो॑ जा॒तो अरो॑चथाः । \newline
19. जा॒तो अरो॑चथा॒ अरो॑चथा जा॒तो जा॒तो अरो॑चथाः । \newline
20. अरो॑चथा॒ इत्यरो॑चथाः । \newline
21. तम् जा॒नन् जा॒नन् तम् तम् जा॒नन् न॑ग्ने अग्ने जा॒नन् तम् तम् जा॒नन् न॑ग्ने । \newline
22. जा॒नन् न॑ग्ने अग्ने जा॒नन् जा॒नन् न॑ग्न॒ आ ऽग्ने॑ जा॒नन् जा॒नन् न॑ग्न॒ आ । \newline
23. अ॒ग्न॒ आ ऽग्ने॑ अग्न॒ आ रो॑ह रो॒हा ऽग्ने॑ अग्न॒ आ रो॑ह । \newline
24. आ रो॑ह रो॒हा रो॒हाथाथ॑ रो॒हा रो॒हाथ॑ । \newline
25. रो॒हाथाथ॑ रोह रो॒हाथा॑ नो नो॒ अथ॑ रोह रो॒हाथा॑ नः । \newline
26. अथा॑ नो नो॒ अथाथा॑ नो वर्द्धय वर्द्धय नो॒ अथाथा॑ नो वर्द्धय । \newline
27. नो॒ व॒र्द्ध॒य॒ व॒र्द्ध॒य॒ नो॒ नो॒ व॒र्द्ध॒या॒ र॒यिꣳ र॒यिं ॅव॑र्द्धय नो नो वर्द्धया र॒यिम् । \newline
28. व॒र्द्ध॒या॒ र॒यिꣳ र॒यिं ॅव॑र्द्धय वर्द्धया र॒यिम् । \newline
29. र॒यिमिति॑ र॒यिम् । \newline
30. अग्न॒ आयू॒(ग्ग्॒) ष्यायू॒(ग्ग्॒)ष्यग्ने ऽग्न॒ आयू(ग्म्॑)षि पवसे पवस॒ आयू॒(ग्ग्॒)ष्यग्ने ऽग्न॒ आयू(ग्म्॑)षि पवसे । \newline
31. आयू(ग्म्॑)षि पवसे पवस॒ आयू॒(ग्ग्॒) ष्यायू(ग्म्॑)षि पवस॒ आ प॑वस॒ आयू॒(ग्ग्॒) ष्यायू(ग्म्॑)षि पवस॒ आ । \newline
32. प॒व॒स॒ आ प॑वसे पवस॒ आ सु॑व सु॒वा प॑वसे पवस॒ आ सु॑व । \newline
33. आ सु॑व सु॒वा सु॒वोर्ज॒ मूर्ज(ग्म्॑) सु॒वा सु॒वोर्ज᳚म् । \newline
34. सु॒वोर्ज॒ मूर्ज(ग्म्॑) सुव सु॒वोर्ज॒ मिष॒ मिष॒ मूर्ज(ग्म्॑) सुव सु॒वोर्ज॒ मिष᳚म् । \newline
35. ऊर्ज॒ मिष॒ मिष॒ मूर्ज॒ मूर्ज॒ मिष॑म् च॒ चे ष॒ मूर्ज॒ मूर्ज॒ मिष॑म् च । \newline
36. इष॑म् च॒ चे ष॒ मिष॑म् च नो न॒श्चे ष॒ मिष॑म् च नः । \newline
37. च॒ नो॒ न॒श्च॒ च॒ नः॒ । \newline
38. न॒ इति॑ नः । \newline
39. आ॒रे बा॑धस्व बाधस्वा॒र आ॒रे बा॑धस्व दु॒च्छुना᳚म् दु॒च्छुना᳚म् बाधस्वा॒र आ॒रे बा॑धस्व दु॒च्छुना᳚म् । \newline
40. बा॒ध॒स्व॒ दु॒च्छुना᳚म् दु॒च्छुना᳚म् बाधस्व बाधस्व दु॒च्छुना᳚म् । \newline
41. दु॒च्छुना॒मिति॑ दु॒च्छुना᳚म् । \newline
42. अग्ने॒ पव॑स्व॒ पव॒स्वाग्ने ऽग्ने॒ पव॑स्व॒ स्वपाः॒ स्वपाः॒ पव॒स्वाग्ने ऽग्ने॒ पव॑स्व॒ स्वपाः᳚ । \newline
43. पव॑स्व॒ स्वपाः॒ स्वपाः॒ पव॑स्व॒ पव॑स्व॒ स्वपा॑ अ॒स्मे अ॒स्मे स्वपाः॒ पव॑स्व॒ पव॑स्व॒ स्वपा॑ अ॒स्मे । \newline
44. स्वपा॑ अ॒स्मे अ॒स्मे स्वपाः॒ स्वपा॑ अ॒स्मे वर्चो॒ वर्चो॑ अ॒स्मे स्वपाः॒ स्वपा॑ अ॒स्मे वर्चः॑ । \newline
45. स्वपा॒ इति॑ सु - अपाः᳚ । \newline
46. अ॒स्मे वर्चो॒ वर्चो॑ अ॒स्मे अ॒स्मे वर्चः॑ सु॒वीर्य(ग्म्॑) सु॒वीर्यं॒ ॅवर्चो॑ अ॒स्मे अ॒स्मे वर्चः॑ सु॒वीर्य᳚म् । \newline
47. अ॒स्मे इत्य॒स्मे । \newline
48. वर्चः॑ सु॒वीर्य(ग्म्॑) सु॒वीर्यं॒ ॅवर्चो॒ वर्चः॑ सु॒वीर्य᳚म् । \newline
49. सु॒वीर्य॒मिति॑ सु - वीर्य᳚म् । \newline
50. दध॒त् पोष॒म् पोष॒म् दध॒द् दध॒त् पोष(ग्म्॑) र॒यिꣳ र॒यिम् पोष॒म् दध॒द् दध॒त् पोष(ग्म्॑) र॒यिम् । \newline
51. पोष(ग्म्॑) र॒यिꣳ र॒यिम् पोष॒म् पोष(ग्म्॑) र॒यिम् मयि॒ मयि॑ र॒यिम् पोष॒म् पोष(ग्म्॑) र॒यिम् मयि॑ । \newline
52. र॒यिम् मयि॒ मयि॑ र॒यिꣳ र॒यिम् मयि॑ । \newline
\pagebreak
\markright{ TS 1.5.5.3  \hfill https://www.vedavms.in \hfill}
\addcontentsline{toc}{section}{ TS 1.5.5.3 }
\section*{ TS 1.5.5.3 }

\textbf{TS 1.5.5.3 } \newline
\textbf{Samhita Paata} \newline

मयि॑ ॥ अग्ने॑ पावक रो॒चिषा॑ म॒न्द्रया॑ देव जि॒ह्वया᳚ । आ दे॒वान्. व॑क्षि॒ यक्षि॑ च ॥ स नः॑ पावक दीदि॒वोऽग्ने॑ दे॒वाꣳ इ॒हा ऽऽ*व॑ह । उप॑ य॒ज्ञ्ꣳ ह॒विश्च॑ नः ॥ अ॒ग्निः शुचि॑व्रततमः॒ शुचि॒र् विप्रः॒ शुचिः॑ क॒विः । शुची॑ रोचत॒ आहु॑तः ॥ उद॑ग्ने॒ शुच॑य॒स्तव॑ शु॒क्रा भ्राज॑न्त ईरते । तव॒ ज्योतीꣳ॑ष्य॒र्चयः॑ ॥ आ॒यु॒र्दा अ॑ग्ने॒ऽस्यायु॑र्मे - [ ] \newline

\textbf{Pada Paata} \newline

मयि॑ ॥ अग्ने᳚ । पा॒व॒क॒ । रो॒चिषा᳚ । म॒न्द्रया᳚ । दे॒व॒ । जि॒ह्वया᳚ ॥ एति॑ । दे॒वान् । व॒क्षि॒ । यक्षि॑ । च॒ ॥ सः । नः॒ । पा॒व॒क॒ । दी॒दि॒वः । अग्ने᳚ । दे॒वान् । इ॒ह । एति॑ । व॒ह॒ ॥ उपेति॑ । य॒ज्ञ्म् । ह॒विः । च॒ । नः॒ ॥ अ॒ग्निः । शुचि॑व्रततम॒ इति॒ शुचि॑व्रत - त॒मः॒ । शुचिः॑ । विप्रः॑ । शुचिः॑ । क॒विः ॥ शुचिः॑ । रो॒च॒ते॒ । आहु॑त॒ इत्या - हु॒तः॒ ॥ उदिति॑ । अ॒ग्ने॒ । शुच॑यः । तव॑ । शु॒क्राः । भ्राज॑न्तः । ई॒र॒ते॒ ॥ तव॑ । ज्योतीꣳ॑षि । अ॒र्चयः॑ ॥ आ॒यु॒र्दा इत्या॑युः - दाः । अ॒ग्ने॒ । अ॒सि॒ । आयुः॑ । म॒ ।  \newline


\textbf{Krama Paata} \newline

मयीति॒ मयि॑ ॥ अग्ने॑ पावक । पा॒व॒क॒ रो॒चिषा᳚ । रो॒चिषा॑ म॒न्द्रया᳚ । म॒न्द्रया॑ देव । दे॒व॒ जि॒ह्वया᳚ । जि॒ह्वयेति॑ जि॒ह्वया᳚ ॥ आ दे॒वान् । दे॒वान्. व॑क्षि । व॒क्षि॒ यक्षि॑ । यक्षि॑ च । चेति॑ च ॥ स नः॑ । नः॒ पा॒व॒क॒ । पा॒व॒क॒ दी॒दि॒वः॒ । दी॒दि॒वोऽग्ने᳚ । अग्ने॑ दे॒वान् । दे॒वाꣳ इ॒ह । इ॒हा । आ व॑ह । व॒हेति॑ वह ॥ उप॑ य॒ज्ञ्म् । य॒ज्ञ्ꣳ ह॒विः । ह॒विश्च॑ । च॒ नः॒ । न॒ इति॑ नः ॥ अ॒ग्निः शुचि॑व्रततमः । शुचि॑व्रततमः॒ शुचिः॑ । शुचि॑व्रततम॒ इति॒ शुचि॑व्रत - त॒मः॒ । शुचि॒र्,विप्रः॑ । विप्रः॒ शुचिः॑ । शुचिः॑ क॒विः । क॒विरिति॑ क॒विः ॥ शुची॑ रोचते । रो॒च॒त॒ आहु॑तः । आहु॑त॒ इत्या - हु॒तः॒ ॥ उद॑ग्ने । अ॒ग्ने॒ शुच॑यः । शुच॑य॒स्तव॑ । तव॑ शु॒क्राः । शु॒क्रा भ्राज॑न्तः । भ्राज॑न्त ईरते । ई॒र॒त॒ इती॑रते ॥ तव॒ ज्योतीꣳ॑षि । ज्योतीꣳ॑ष्य॒र्चयः॑ । अ॒र्चय॒ इत्य॒र्चयः॑ ॥ आ॒यु॒र्दा अ॑ग्ने । आ॒यु॒र्दा इत्या॑युः - दाः । अ॒ग्ने॒ऽसि॒ । अ॒स्यायुः॑ । आयु॑र् मे । मे॒ दे॒हि॒ \newline

\textbf{Jatai Paata} \newline

1. मयीति॒ मयि॑ । \newline
2. अग्ने॑ पावक पाव॒काग्ने ऽग्ने॑ पावक । \newline
3. पा॒व॒क॒ रो॒चिषा॑ रो॒चिषा॑ पावक पावक रो॒चिषा᳚ । \newline
4. रो॒चिषा॑ म॒न्द्रया॑ म॒न्द्रया॑ रो॒चिषा॑ रो॒चिषा॑ म॒न्द्रया᳚ । \newline
5. म॒न्द्रया॑ देव देव म॒न्द्रया॑ म॒न्द्रया॑ देव । \newline
6. दे॒व॒ जि॒ह्वया॑ जि॒ह्वया॑ देव देव जि॒ह्वया᳚ । \newline
7. जि॒ह्वयेति॑ जि॒ह्वया᳚ । \newline
8. आ दे॒वान् दे॒वा ना दे॒वान् । \newline
9. दे॒वान्. व॑क्षि वक्षि दे॒वान् दे॒वान्. व॑क्षि । \newline
10. व॒क्षि॒ यक्षि॒ यक्षि॑ वक्षि वक्षि॒ यक्षि॑ । \newline
11. यक्षि॑ च च॒ यक्षि॒ यक्षि॑ च । \newline
12. चेति॑ च । \newline
13. स नो॑ नः॒ स सो नः॑ । \newline
14. नः॒ पा॒व॒क॒ पा॒व॒क॒ नो॒ नः॒ पा॒व॒क॒ । \newline
15. पा॒व॒क॒ दी॒दि॒वो॒ दी॒दि॒वः॒ पा॒व॒क॒ पा॒व॒क॒ दी॒दि॒वः॒ । \newline
16. दी॒दि॒वो ऽग्ने ऽग्ने॑ दीदिवो दीदि॒वो ऽग्ने᳚ । \newline
17. अग्ने॑ दे॒वान् दे॒वाꣳ अग्ने ऽग्ने॑ दे॒वान् । \newline
18. दे॒वाꣳ इ॒हे ह दे॒वान् दे॒वाꣳ इ॒ह । \newline
19. इ॒हेहे हा । \newline
20. आ व॑ह व॒हा व॑ह । \newline
21. व॒हेति॑ वह । \newline
22. उप॑ य॒ज्ञ्ं ॅय॒ज्ञ् मुपोप॑ य॒ज्ञ्म् । \newline
23. य॒ज्ञ्ꣳ ह॒विर्. ह॒विर् य॒ज्ञ्ं ॅय॒ज्ञ्ꣳ ह॒विः । \newline
24. ह॒विश्च॑ च ह॒विर्. ह॒विश्च॑ । \newline
25. च॒ नो॒ न॒श्च॒ च॒ नः॒ । \newline
26. न॒ इति॑ नः । \newline
27. अ॒ग्निः शुचि॑व्रततमः॒ शुचि॑व्रततमो॒ ऽग्निर॒ग्निः शुचि॑व्रततमः । \newline
28. शुचि॑व्रततमः॒ शुचिः॒ शुचिः॒ शुचि॑व्रततमः॒ शुचि॑व्रततमः॒ शुचिः॑ । \newline
29. शुचि॑व्रततम॒ इति॒ शुचि॑व्रत - त॒मः॒ । \newline
30. शुचि॒र् विप्रो॒ विप्रः॒ शुचिः॒ शुचि॒र् विप्रः॑ । \newline
31. विप्रः॒ शुचिः॒ शुचि॒र् विप्रो॒ विप्रः॒ शुचिः॑ । \newline
32. शुचिः॑ क॒विः क॒विः शुचिः॒ शुचिः॑ क॒विः । \newline
33. क॒विरिति॑ क॒विः । \newline
34. शुची॑ रोचते रोचते॒ शुचिः॒ शुची॑ रोचते । \newline
35. रो॒च॒त॒ आहु॑त॒ आहु॑तो रोचते रोचत॒ आहु॑तः । \newline
36. आहु॑त॒ इत्या - हु॒तः॒ । \newline
37. उद॑ग्ने अग्न॒ उदुद॑ग्ने । \newline
38. अ॒ग्ने॒ शुच॑यः॒ शुच॑यो अग्ने अग्ने॒ शुच॑यः । \newline
39. शुच॑य॒स्तव॒ तव॒ शुच॑यः॒ शुच॑य॒स्तव॑ । \newline
40. तव॑ शु॒क्राः शु॒क्रास्तव॒ तव॑ शु॒क्राः । \newline
41. शु॒क्रा भ्राज॑न्तो॒ भ्राज॑न्तः शु॒क्राः शु॒क्रा भ्राज॑न्तः । \newline
42. भ्राज॑न्त ईरत ईरते॒ भ्राज॑न्तो॒ भ्राज॑न्त ईरते । \newline
43. ई॒र॒त॒ इती॑रते । \newline
44. तव॒ ज्योती(ग्म्॑)षि॒ ज्योती(ग्म्॑)षि॒ तव॒ तव॒ ज्योती(ग्म्॑)षि । \newline
45. ज्योती(ग्ग्॑)ष्य॒र्चयो॑ अ॒र्चयो॒ ज्योती(ग्म्॑)षि॒ ज्योती(ग्ग्॑)ष्य॒र्चयः॑ । \newline
46. अ॒र्चय॒ इत्य॒र्चयः॑ । \newline
47. आ॒यु॒र्दा अ॑ग्ने अग्न आयु॒र्दा आ॑यु॒र्दा अ॑ग्ने । \newline
48. आ॒यु॒र्दा इत्या॑युः - दाः । \newline
49. अ॒ग्ने॒ ऽस्य॒स्य॒ग्ने॒ अ॒ग्ने॒ ऽसि॒ । \newline
50. अ॒स्यायु॒रायु॑ रस्य॒स्यायुः॑ । \newline
51. आयु॑र् मे म॒ आयु॒रायु॑र् मे । \newline
52. मे॒ दे॒हि॒ दे॒हि॒ मे॒ मे॒ दे॒हि॒ । \newline

\textbf{Ghana Paata } \newline

1. मयीति॒ मयि॑ । \newline
2. अग्ने॑ पावक पाव॒काग्ने ऽग्ने॑ पावक रो॒चिषा॑ रो॒चिषा॑ पाव॒काग्ने ऽग्ने॑ पावक रो॒चिषा᳚ । \newline
3. पा॒व॒क॒ रो॒चिषा॑ रो॒चिषा॑ पावक पावक रो॒चिषा॑ म॒न्द्रया॑ म॒न्द्रया॑ रो॒चिषा॑ पावक पावक रो॒चिषा॑ म॒न्द्रया᳚ । \newline
4. रो॒चिषा॑ म॒न्द्रया॑ म॒न्द्रया॑ रो॒चिषा॑ रो॒चिषा॑ म॒न्द्रया॑ देव देव म॒न्द्रया॑ रो॒चिषा॑ रो॒चिषा॑ म॒न्द्रया॑ देव । \newline
5. म॒न्द्रया॑ देव देव म॒न्द्रया॑ म॒न्द्रया॑ देव जि॒ह्वया॑ जि॒ह्वया॑ देव म॒न्द्रया॑ म॒न्द्रया॑ देव जि॒ह्वया᳚ । \newline
6. दे॒व॒ जि॒ह्वया॑ जि॒ह्वया॑ देव देव जि॒ह्वया᳚ । \newline
7. जि॒ह्वयेति॑ जि॒ह्वया᳚ । \newline
8. आ दे॒वान् दे॒वा ना दे॒वान्. व॑क्षि वक्षि दे॒वा ना दे॒वान्. व॑क्षि । \newline
9. दे॒वान्. व॑क्षि वक्षि दे॒वान् दे॒वान्. व॑क्षि॒ यक्षि॒ यक्षि॑ वक्षि दे॒वान् दे॒वान्. व॑क्षि॒ यक्षि॑ । \newline
10. व॒क्षि॒ यक्षि॒ यक्षि॑ वक्षि वक्षि॒ यक्षि॑ च च॒ यक्षि॑ वक्षि वक्षि॒ यक्षि॑ च । \newline
11. यक्षि॑ च च॒ यक्षि॒ यक्षि॑ च । \newline
12. चेति॑ च । \newline
13. स नो॑ नः॒ स स नः॑ पावक पावक नः॒ स स नः॑ पावक । \newline
14. नः॒ पा॒व॒क॒ पा॒व॒क॒ नो॒ नः॒ पा॒व॒क॒ दी॒दि॒वो॒ दी॒दि॒वः॒ पा॒व॒क॒ नो॒ नः॒ पा॒व॒क॒ दी॒दि॒वः॒ । \newline
15. पा॒व॒क॒ दी॒दि॒वो॒ दी॒दि॒वः॒ पा॒व॒क॒ पा॒व॒क॒ दी॒दि॒वो ऽग्ने ऽग्ने॑ दीदिवः पावक पावक दीदि॒वो ऽग्ने᳚ । \newline
16. दी॒दि॒वो ऽग्ने ऽग्ने॑ दीदिवो दीदि॒वो ऽग्ने॑ दे॒वान् दे॒वाꣳ अग्ने॑ दीदिवो दीदि॒वो ऽग्ने॑ दे॒वान् । \newline
17. अग्ने॑ दे॒वान् दे॒वाꣳ अग्ने ऽग्ने॑ दे॒वाꣳ इ॒हे ह दे॒वाꣳ अग्ने ऽग्ने॑ दे॒वाꣳ इ॒ह । \newline
18. दे॒वाꣳ इ॒हे ह दे॒वान् दे॒वाꣳ इ॒हेह दे॒वान् दे॒वाꣳ इ॒हा । \newline
19. इ॒हेहे हा व॑ह व॒हेहे हा व॑ह । \newline
20. आ व॑ह व॒हा व॑ह । \newline
21. व॒हेति॑ वह । \newline
22. उप॑ य॒ज्ञ्ं ॅय॒ज्ञ् मुपोप॑ य॒ज्ञ्ꣳ ह॒विर्. ह॒विर् य॒ज्ञ् मुपोप॑ य॒ज्ञ्ꣳ ह॒विः । \newline
23. य॒ज्ञ्ꣳ ह॒विर्. ह॒विर् य॒ज्ञ्ं ॅय॒ज्ञ्ꣳ ह॒विश्च॑ च ह॒विर् य॒ज्ञ्ं ॅय॒ज्ञ्ꣳ ह॒विश्च॑ । \newline
24. ह॒विश्च॑ च ह॒विर्. ह॒विश्च॑ नो नश्च ह॒विर्. ह॒विश्च॑ नः । \newline
25. च॒ नो॒ न॒श्च॒ च॒ नः॒ । \newline
26. न॒ इति॑ नः । \newline
27. अ॒ग्निः शुचि॑व्रततमः॒ शुचि॑व्रततमो॒ ऽग्निर॒ग्निः शुचि॑व्रततमः॒ शुचिः॒ शुचिः॒ शुचि॑व्रततमो॒ ऽग्निर॒ग्निः शुचि॑व्रततमः॒ शुचिः॑ । \newline
28. शुचि॑व्रततमः॒ शुचिः॒ शुचिः॒ शुचि॑व्रततमः॒ शुचि॑व्रततमः॒ शुचि॒र् विप्रो॒ विप्रः॒ शुचिः॒ शुचि॑व्रततमः॒ शुचि॑व्रततमः॒ शुचि॒र् विप्रः॑ । \newline
29. शुचि॑व्रततम॒ इति॒ शुचि॑व्रत - त॒मः॒ । \newline
30. शुचि॒र् विप्रो॒ विप्रः॒ शुचिः॒ शुचि॒र् विप्रः॒ शुचिः॒ शुचि॒र् विप्रः॒ शुचिः॒ शुचि॒र् विप्रः॒ शुचिः॑ । \newline
31. विप्रः॒ शुचिः॒ शुचि॒र् विप्रो॒ विप्रः॒ शुचिः॑ क॒विः क॒विः शुचि॒र् विप्रो॒ विप्रः॒ शुचिः॑ क॒विः । \newline
32. शुचिः॑ क॒विः क॒विः शुचिः॒ शुचिः॑ क॒विः । \newline
33. क॒विरिति॑ क॒विः । \newline
34. शुची॑ रोचते रोचते॒ शुचिः॒ शुची॑ रोचत॒ आहु॑त॒ आहु॑तो रोचते॒ शुचिः॒ शुची॑ रोचत॒ आहु॑तः । \newline
35. रो॒च॒त॒ आहु॑त॒ आहु॑तो रोचते रोचत॒ आहु॑तः । \newline
36. आहु॑त॒ इत्या - हु॒तः॒ । \newline
37. उद॑ग्ने अग्न॒ उदुद॑ग्ने॒ शुच॑यः॒ शुच॑यो अग्न॒ उदुद॑ग्ने॒ शुच॑यः । \newline
38. अ॒ग्ने॒ शुच॑यः॒ शुच॑यो अग्ने अग्ने॒ शुच॑य॒स्तव॒ तव॒ शुच॑यो अग्ने अग्ने॒ शुच॑य॒स्तव॑ । \newline
39. शुच॑य॒स्तव॒ तव॒ शुच॑यः॒ शुच॑य॒स्तव॑ शु॒क्राः शु॒क्रास्तव॒ शुच॑यः॒ शुच॑य॒स्तव॑ शु॒क्राः । \newline
40. तव॑ शु॒क्राः शु॒क्रास्तव॒ तव॑ शु॒क्रा भ्राज॑न्तो॒ भ्राज॑न्तः शु॒क्रास्तव॒ तव॑ शु॒क्रा भ्राज॑न्तः । \newline
41. शु॒क्रा भ्राज॑न्तो॒ भ्राज॑न्तः शु॒क्राः शु॒क्रा भ्राज॑न्त ईरत ईरते॒ भ्राज॑न्तः शु॒क्राः शु॒क्रा भ्राज॑न्त ईरते । \newline
42. भ्राज॑न्त ईरत ईरते॒ भ्राज॑न्तो॒ भ्राज॑न्त ईरते । \newline
43. ई॒र॒त॒ इती॑रते । \newline
44. तव॒ ज्योती(ग्म्॑)षि॒ ज्योती(ग्म्॑)षि॒ तव॒ तव॒ ज्योती(ग्ग्॑)ष्य॒र्चयो॑ अ॒र्चयो॒ ज्योती(ग्म्॑)षि॒ तव॒ तव॒ ज्योती(ग्ग्॑)ष्य॒र्चयः॑ । \newline
45. ज्योती(ग्ग्॑)ष्य॒र्चयो॑ अ॒र्चयो॒ ज्योती(ग्म्॑)षि॒ ज्योती(ग्ग्॑)ष्य॒र्चयः॑ । \newline
46. अ॒र्चय॒ इत्य॒र्चयः॑ ॥ \newline
47. आ॒यु॒र्दा अ॑ग्ने अग्न आयु॒र्दा आ॑यु॒र्दा अ॑ग्ने ऽस्यस्यग्न आयु॒र्दा आ॑यु॒र्दा अ॑ग्ने ऽसि । \newline
48. आ॒यु॒र्दा इत्या॑युः - दाः । \newline
49. अ॒ग्ने॒ ऽस्य॒स्य॒ग्ने॒ अ॒ग्ने॒ ऽस्यायु॒रायु॑रस्यग्ने अग्ने॒ ऽस्यायुः॑ । \newline
50. अ॒स्यायु॒ रायु॑रस्य॒स्यायु॑र् मे म॒ आयु॑ रस्य॒स्यायु॑र् मे । \newline
51. आयु॑र् मे म॒ आयु॒रायु॑र् मे देहि देहि म॒ आयु॒रायु॑र् मे देहि । \newline
52. मे॒ दे॒हि॒ दे॒हि॒ मे॒ मे॒ दे॒हि॒ व॒र्चो॒दा व॑र्चो॒दा दे॑हि मे मे देहि वर्चो॒दाः । \newline
\pagebreak
\markright{ TS 1.5.5.4  \hfill https://www.vedavms.in \hfill}
\addcontentsline{toc}{section}{ TS 1.5.5.4 }
\section*{ TS 1.5.5.4 }

\textbf{TS 1.5.5.4 } \newline
\textbf{Samhita Paata} \newline

देहि वर्चो॒दा अ॑ग्नेऽसि॒ वर्चो॑ मे देहि तनू॒पा अ॑ग्नेऽसि त॒नुवं॑ मे पा॒ह्यग्ने॒ यन्मे॑ त॒नुवा॑ ऊ॒नं तन्म॒ आ पृ॑ण॒ चित्रा॑वसो स्व॒स्ति ते॑ पा॒रम॑शी॒येन्धा॑नास्त्वा श॒तꣳ हिमा᳚ द्यु॒मन्तः॒ समि॑धीमहि॒ वय॑स्वन्तो वय॒स्कृतं॒ ॅयश॑स्वन्तो यश॒स्कृतꣳ॑ सु॒वीरा॑सो॒ अदा᳚भ्यं । अग्ने॑ सपत्न॒दंभ॑नं॒ ॅवर्.षि॑ष्ठे॒ अधि॒ नाके᳚ ॥ सं त्वम॑ग्ने॒ सूर्य॑स्य॒ वर्च॑सा ( ) ऽगथाः॒ समृषी॑णाꣳ स्तु॒तेन॒ सं प्रि॒येण॒ धाम्ना᳚ । त्वम॑ग्ने॒ सूर्य॑वर्चा असि॒ सं मामायु॑षा॒ वर्च॑सा प्र॒जया॑ सृज ॥ \newline

\textbf{Pada Paata} \newline

दे॒हि॒ । व॒र्चो॒दा इति॑ वर्चः- दाः । अ॒ग्ने॒ । अ॒सि॒ । वर्चः॑ । मे॒ । दे॒हि॒ । त॒नू॒पा इति॑ तनू - पाः । अ॒ग्ने॒ । अ॒सि॒ । त॒नुव᳚म् । मे॒ । पा॒हि॒ । अग्ने᳚ । यत् । मे॒ । त॒नुवाः᳚ । ऊ॒नम् । तत् । मे॒ । एति॑ । पृ॒ण॒ । चित्रा॑वसो॒ इति॒ चित्र॑ - व॒सो॒ । स्व॒स्ति । ते॒ । पा॒रम् । अ॒शी॒य॒ । इन्धा॑नाः । त्वा॒ । श॒तम् । हिमाः᳚ । द्यु॒मन्त॒ इति॑ द्यु- मन्तः॑ । समिति॑ । इ॒धी॒म॒हि॒ । वय॑स्वन्तः । व॒य॒स्कृत॒मिति॑ वयः-कृत᳚म् । यश॑स्वन्तः । य॒श॒स्कृत॒मिति॑ यशः - कृत᳚म् । सु॒वीरा॑स॒ इति सु - वीरा॑सः । अदा᳚भ्यम् ॥ अग्ने᳚ । स॒प॒त्न॒दंभ॑न॒मिति॑ सपत्न - दंभ॑नम् । वर्.षि॑ष्ठे । अधीति॑ । नाके᳚ ॥ समिति॑ । त्वम् । अ॒ग्ने॒ । सूर्य॑स्य । वर्च॑सा ( ) । अ॒ग॒थाः॒ । समिति॑ । ऋषी॑णाम् । स्तु॒तेन॑ । समिति॑ । प्रि॒येण॑ । धाम्ना᳚ ॥ त्वम् । अ॒ग्ने॒ । सूर्य॑वर्चा॒ इति॒ सूर्य॑ - व॒र्चाः॒ । अ॒सि॒ । समिति॑ । माम् । आयु॑षा । वर्च॑सा । प्र॒जयेति॑ प्र - जया᳚ । सृ॒ज॒ ॥  \newline


\textbf{Krama Paata} \newline

दे॒हि॒ व॒र्चो॒दाः । व॒र्चो॒दा अ॑ग्ने । व॒र्चो॒दा इति॑ वर्चः - दाः । अ॒ग्ने॒ऽसि॒ । अ॒सि॒ वर्चः॑ । वर्चो॑ मे । मे॒ दे॒हि॒ । दे॒हि॒ त॒नू॒पाः । त॒नू॒पा अ॑ग्ने । त॒नू॒पा इति॑ तनू - पाः । अ॒ग्ने॒ऽसि॒ । अ॒सि॒ त॒नुव᳚म् । त॒नुव॑म् मे । मे॒ पा॒हि॒ । पा॒ह्यग्ने᳚ । अग्ने॒ यत् । यन्मे᳚ । मे॒ त॒नुवाः᳚ । त॒नुवा॑ ऊ॒नम् । ऊ॒नम् तत् । तन्मे᳚ । म॒ आ । आ पृ॑ण । पृ॒ण॒ चित्रा॑वसो । चित्रा॑वसो स्व॒स्ति । चित्रा॑वसो॒ इति॒ चित्र॑ - व॒सो॒ । स्व॒स्ति ते᳚ । ते॒ पा॒रम् । पा॒रम॑शीय । अ॒शी॒येन्धा॑नाः । इन्धा॑नास्त्वा । त्वा॒ श॒तम् । श॒तꣳ हिमाः᳚ । हिमा᳚ द्यु॒मन्तः॑ । द्यु॒मन्तः॒ सम् । द्यु॒मन्त॒ इति॑ द्यु - मन्तः॑ । समि॑धीमहि । इ॒धी॒म॒हि॒ वय॑स्वन्तः । वय॑स्वन्तो वय॒स्कृत᳚म् । व॒य॒स्कृतं॒ ॅयश॑स्वन्तः । व॒य॒स्कृत॒मिति॑ वयः - कृत᳚म् । यश॑स्वन्तो यश॒स्कृत᳚म् । य॒श॒स्कृतꣳ॑ सु॒वीरा॑सः । य॒श॒स्कृत॒मिति॑ यशः - कृत᳚म् । सु॒वीरा॑सो॒ अदा᳚भ्यम् । सु॒वीरा॑स॒ इति॑ सु - वीरा॑सः । 
अदा᳚भ्य॒मित्यदा᳚भ्यम् ॥ अग्ने॑ सपत्न॒दम्भ॑नम् । स॒प॒त्न॒दम्भ॑नं॒ ॅवर्.षि॑ष्ठे । स॒प॒त्न॒दम्भ॑न॒मिति॑ सपत्न - दम्भ॑नम् । वर्.षि॑ष्ठे॒ अधि॑ । अधि॒ नाके᳚ । नाक॒ इति॒ नाके᳚ ॥ सम् त्वम् । त्वम॑ग्ने । अ॒ग्ने॒ सूर्य॑स्य । सूर्य॑स्य॒ वर्च॑सा ( ) । वर्च॑साऽगथाः । अ॒ग॒थाः॒ सम् । समृषी॑णाम् । ऋषी॑णाꣳ स्तु॒तेन॑ । स्तु॒तेन॒ सम् । सम् प्रि॒येण॑ । प्रि॒येण॒ धाम्ना᳚ । धाम्नेति॒ धाम्ना᳚ ॥ त्वम॑ग्ने । अ॒ग्ने॒ सूर्य॑वर्चाः । सूर्य॑वर्चा असि । सूर्य॑वर्चा॒ इति॒ सूर्य॑ - व॒र्चाः॒ । अ॒सि॒ सम् । सम् माम् । मामायु॑षा । आयु॑षा॒ वर्च॑सा । वर्च॑सा प्र॒जया᳚ । प्र॒जया॑ सृज । प्र॒जयेति॑ प्र - जया᳚ । सृ॒जेति॑ सृज । \newline

\textbf{Jatai Paata} \newline

1. दे॒हि॒ व॒र्चो॒दा व॑र्चो॒दा दे॑हि देहि वर्चो॒दाः । \newline
2. व॒र्चो॒दा अ॑ग्ने अग्ने वर्चो॒दा व॑र्चो॒दा अ॑ग्ने । \newline
3. व॒र्चो॒दा इति॑ वर्चः - दाः । \newline
4. अ॒ग्ने॒ ऽस्य॒स्य॒ग्ने॒ अ॒ग्ने॒ ऽसि॒ । \newline
5. अ॒सि॒ वर्चो॒ वर्चो᳚ ऽस्यसि॒ वर्चः॑ । \newline
6. वर्चो॑ मे मे॒ वर्चो॒ वर्चो॑ मे । \newline
7. मे॒ दे॒हि॒ दे॒हि॒ मे॒ मे॒ दे॒हि॒ । \newline
8. दे॒हि॒ त॒नू॒पास्त॑नू॒पा दे॑हि देहि तनू॒पाः । \newline
9. त॒नू॒पा अ॑ग्ने अग्ने तनू॒पास्त॑नू॒पा अ॑ग्ने । \newline
10. त॒नू॒पा इति॑ तनू - पाः । \newline
11. अ॒ग्ने॒ ऽस्य॒स्य॒ग्ने॒ अ॒ग्ने॒ ऽसि॒ । \newline
12. अ॒सि॒ त॒नुव॑म् त॒नुव॑ मस्यसि त॒नुव᳚म् । \newline
13. त॒नुव॑म् मे मे त॒नुव॑म् त॒नुव॑म् मे । \newline
14. मे॒ पा॒हि॒ पा॒हि॒ मे॒ मे॒ पा॒हि॒ । \newline
15. पा॒ह्यग्ने ऽग्ने॑ पाहि पा॒ह्यग्ने᳚ । \newline
16. अग्ने॒ यद् यदग्ने ऽग्ने॒ यत् । \newline
17. यन् मे॑ मे॒ यद् यन् मे᳚ । \newline
18. मे॒ त॒नुवा᳚स्त॒नुवा॑ मे मे त॒नुवाः᳚ । \newline
19. त॒नुवा॑ ऊ॒न मू॒नम् त॒नुवा᳚स्त॒नुवा॑ ऊ॒नम् । \newline
20. ऊ॒नम् तत् तदू॒न मू॒नम् तत् । \newline
21. तन् मे॑ मे॒ तत् तन् मे᳚ । \newline
22. म॒ आ मे॑ म॒ आ । \newline
23. आ पृ॑ण पृ॒णा पृ॑ण । \newline
24. पृ॒ण॒ चित्रा॑वसो॒ चित्रा॑वसो पृण पृण॒ चित्रा॑वसो । \newline
25. चित्रा॑वसो स्व॒स्ति स्व॒स्ति चित्रा॑वसो॒ चित्रा॑वसो स्व॒स्ति । \newline
26. चित्रा॑वसो॒ इति॒ चित्र॑ - व॒सो॒ । \newline
27. स्व॒स्ति ते॑ ते स्व॒स्ति स्व॒स्ति ते᳚ । \newline
28. ते॒ पा॒रम् पा॒रम् ते॑ ते पा॒रम् । \newline
29. पा॒र म॑शीयाशीय पा॒रम् पा॒र म॑शीय । \newline
30. अ॒शी॒ये न्धा॑ना॒ इन्धा॑ना अशीयाशी॒ये न्धा॑नाः । \newline
31. इन्धा॑नास्त्वा॒ त्वेन्धा॑ना॒ इन्धा॑नास्त्वा । \newline
32. त्वा॒ श॒तꣳ श॒तम् त्वा᳚ त्वा श॒तम् । \newline
33. श॒तꣳ हिमा॒ हिमाः᳚ श॒तꣳ श॒तꣳ हिमाः᳚ । \newline
34. हिमा᳚ द्यु॒मन्तो᳚ द्यु॒मन्तो॒ हिमा॒ हिमा᳚ द्यु॒मन्तः॑ । \newline
35. द्यु॒मन्तः॒ सꣳ सम् द्यु॒मन्तो᳚ द्यु॒मन्तः॒ सम् । \newline
36. द्यु॒मन्त॒ इति॑ द्यु - मन्तः॑ । \newline
37. स मि॑धीमहीधीमहि॒ सꣳ स मि॑धीमहि । \newline
38. इ॒धी॒म॒हि॒ वय॑स्वन्तो॒ वय॑स्वन्त इधीमहीधीमहि॒ वय॑स्वन्तः । \newline
39. वय॑स्वन्तो वय॒स्कृतं॑ ॅवय॒स्कृतं॒ ॅवय॑स्वन्तो॒ वय॑स्वन्तो वय॒स्कृत᳚म् । \newline
40. व॒य॒स्कृतं॒ ॅयश॑स्वन्तो॒ यश॑स्वन्तो वय॒स्कृतं॑ ॅवय॒स्कृतं॒ ॅयश॑स्वन्तः । \newline
41. व॒य॒स्कृत॒मिति॑ वयः - कृत᳚म् । \newline
42. यश॑स्वन्तो यश॒स्कृतं॑ ॅयश॒स्कृतं॒ ॅयश॑स्वन्तो॒ यश॑स्वन्तो यश॒स्कृत᳚म् । \newline
43. य॒श॒स्कृत(ग्म्॑) सु॒वीरा॑सः सु॒वीरा॑सो यश॒स्कृतं॑ ॅयश॒स्कृत(ग्म्॑) सु॒वीरा॑सः । \newline
44. य॒श॒स्कृत॒मिति॑ यशः - कृत᳚म् । \newline
45. सु॒वीरा॑सो॒ अदा᳚भ्य॒ मदा᳚भ्यꣳ सु॒वीरा॑सः सु॒वीरा॑सो॒ अदा᳚भ्यम् । \newline
46. सु॒वीरा॑स॒ इति॑ सु - वीरा॑सः । \newline
47. अदा᳚भ्य॒मित्यदा᳚भ्यम् । \newline
48. अग्ने॑ सपत्न॒दंभ॑नꣳ सपत्न॒दंभ॑न॒ मग्ने ऽग्ने॑ सपत्न॒दंभ॑नम् । \newline
49. स॒प॒त्न॒दंभ॑नं॒ ॅवर्.षि॑ष्ठे॒ वर्.षि॑ष्ठे सपत्न॒दंभ॑नꣳ सपत्न॒दंभ॑नं॒ ॅवर्.षि॑ष्ठे । \newline
50. स॒प॒त्न॒दंभ॑न॒मिति॑ सपत्न - दंभ॑नम् । \newline
51. वर्.षि॑ष्ठे॒ अध्यधि॒ वर्.षि॑ष्ठे॒ वर्.षि॑ष्ठे॒ अधि॑ । \newline
52. अधि॒ नाके॒ नाके॒ अध्यधि॒ नाके᳚ । \newline
53. नाक॒ इति॒ नाके᳚ । \newline
54. सम् त्वम् त्वꣳ सꣳ सम् त्वम् । \newline
55. त्व म॑ग्ने अग्ने॒ त्वम् त्व म॑ग्ने । \newline
56. अ॒ग्ने॒ सूर्य॑स्य॒ सूर्य॑स्याग्ने अग्ने॒ सूर्य॑स्य । \newline
57. सूर्य॑स्य॒ वर्च॑सा॒ वर्च॑सा॒ सूर्य॑स्य॒ सूर्य॑स्य॒ वर्च॑सा । \newline
58. वर्च॑सा ऽगथा अगथा॒ वर्च॑सा॒ वर्च॑सा ऽगथाः । \newline
59. अ॒ग॒थाः॒ सꣳ स म॑गथा अगथाः॒ सम् । \newline
60. स मृषी॑णा॒ मृषी॑णा॒(ग्म्॒) सꣳ स मृषी॑णाम् । \newline
61. ऋषी॑णाꣳ स्तु॒तेन॑ स्तु॒तेन र्.षी॑णा॒ मृषी॑णाꣳ स्तु॒तेन॑ । \newline
62. स्तु॒तेन॒ सꣳ सꣳ स्तु॒तेन॑ स्तु॒तेन॒ सम् । \newline
63. सम् प्रि॒येण॑ प्रि॒येण॒ सꣳ सम् प्रि॒येण॑ । \newline
64. प्रि॒येण॒ धाम्ना॒ धाम्ना᳚ प्रि॒येण॑ प्रि॒येण॒ धाम्ना᳚ । \newline
65. धाम्नेति॒ धाम्ना᳚ । \newline
66. त्व म॑ग्ने अग्ने॒ त्वम् त्व म॑ग्ने । \newline
67. अ॒ग्ने॒ सूर्य॑वर्चाः॒ सूर्य॑वर्चा अग्ने अग्ने॒ सूर्य॑वर्चाः । \newline
68. सूर्य॑वर्चा अस्यसि॒ सूर्य॑वर्चाः॒ सूर्य॑वर्चा असि । \newline
69. सूर्य॑वर्चा॒ इति॒ सूर्य॑ - व॒र्चाः॒ । \newline
70. अ॒सि॒ सꣳ स म॑स्यसि॒ सम् । \newline
71. सम् माम् माꣳ सꣳ सम् माम् । \newline
72. मा मायु॒षा ऽऽयु॑षा॒ माम् मा मायु॑षा । \newline
73. आयु॑षा॒ वर्च॑सा॒ वर्च॒सा ऽऽयु॒षा ऽऽयु॑षा॒ वर्च॑सा । \newline
74. वर्च॑सा प्र॒जया᳚ प्र॒जया॒ वर्च॑सा॒ वर्च॑सा प्र॒जया᳚ । \newline
75. प्र॒जया॑ सृज सृज प्र॒जया᳚ प्र॒जया॑ सृज । \newline
76. प्र॒जयेति॑ प्र - जया᳚ । \newline
77. सृ॒जेति॑ सृज । \newline

\textbf{Ghana Paata } \newline

1. दे॒हि॒ व॒र्चो॒दा व॑र्चो॒दा दे॑हि देहि वर्चो॒दा अ॑ग्ने अग्ने वर्चो॒दा दे॑हि देहि वर्चो॒दा अ॑ग्ने । \newline
2. व॒र्चो॒दा अ॑ग्ने अग्ने वर्चो॒दा व॑र्चो॒दा अ॑ग्ने ऽस्यस्यग्ने वर्चो॒दा व॑र्चो॒दा अ॑ग्ने ऽसि । \newline
3. व॒र्चो॒दा इति॑ वर्चः - दाः । \newline
4. अ॒ग्ने॒ ऽस्य॒स्य॒ग्ने॒ अ॒ग्ने॒ ऽसि॒ वर्चो॒ वर्चो᳚ ऽस्यग्ने अग्ने ऽसि॒ वर्चः॑ । \newline
5. अ॒सि॒ वर्चो॒ वर्चो᳚ ऽस्यसि॒ वर्चो॑ मे मे॒ वर्चो᳚ ऽस्यसि॒ वर्चो॑ मे । \newline
6. वर्चो॑ मे मे॒ वर्चो॒ वर्चो॑ मे देहि देहि मे॒ वर्चो॒ वर्चो॑ मे देहि । \newline
7. मे॒ दे॒हि॒ दे॒हि॒ मे॒ मे॒ दे॒हि॒ त॒नू॒पा स्त॑नू॒पा दे॑हि मे मे देहि तनू॒पाः । \newline
8. दे॒हि॒ त॒नू॒पास्त॑नू॒पा दे॑हि देहि तनू॒पा अ॑ग्ने अग्ने तनू॒पा दे॑हि देहि तनू॒पा अ॑ग्ने । \newline
9. त॒नू॒पा अ॑ग्ने अग्ने तनू॒पा स्त॑नू॒पा अ॑ग्ने ऽस्यस्यग्ने तनू॒पा स्त॑नू॒पा अ॑ग्ने ऽसि । \newline
10. त॒नू॒पा इति॑ तनू - पाः । \newline
11. अ॒ग्ने॒ ऽस्य॒स्य॒ग्ने॒ अ॒ग्ने॒ ऽसि॒ त॒नुव॑म् त॒नुव॑ मस्यग्ने अग्ने ऽसि त॒नुव᳚म् । \newline
12. अ॒सि॒ त॒नुव॑म् त॒नुव॑ मस्यसि त॒नुव॑म् मे मे त॒नुव॑ मस्यसि त॒नुव॑म् मे । \newline
13. त॒नुव॑म् मे मे त॒नुव॑म् त॒नुव॑म् मे पाहि पाहि मे त॒नुव॑म् त॒नुव॑म् मे पाहि । \newline
14. मे॒ पा॒हि॒ पा॒हि॒ मे॒ मे॒ पा॒ह्यग्ने ऽग्ने॑ पाहि मे मे पा॒ह्यग्ने᳚ । \newline
15. पा॒ह्यग्ने ऽग्ने॑ पाहि पा॒ह्यग्ने॒ यद् यदग्ने॑ पाहि पा॒ह्यग्ने॒ यत् । \newline
16. अग्ने॒ यद् यदग्ने ऽग्ने॒ यन् मे॑ मे॒ यदग्ने ऽग्ने॒ यन् मे᳚ । \newline
17. यन् मे॑ मे॒ यद् यन् मे॑ त॒नुवा᳚ स्त॒नुवा॑ मे॒ यद् यन् मे॑ त॒नुवाः᳚ । \newline
18. मे॒ त॒नुवा᳚स्त॒नुवा॑ मे मे त॒नुवा॑ ऊ॒न मू॒नम् त॒नुवा॑ मे मे त॒नुवा॑ ऊ॒नम् । \newline
19. त॒नुवा॑ ऊ॒न मू॒नम् त॒नुवा᳚ स्त॒नुवा॑ ऊ॒नम् तत् तदू॒नम् त॒नुवा᳚ स्त॒नुवा॑ ऊ॒नम् तत् । \newline
20. ऊ॒नम् तत् तदू॒न मू॒नम् तन् मे॑ मे॒ तदू॒न मू॒नम् तन् मे᳚ । \newline
21. तन् मे॑ मे॒ तत् तन् म॒ आ मे॒ तत् तन् म॒ आ । \newline
22. म॒ आ मे॑ म॒ आ पृ॑ण पृ॒णा मे॑ म॒ आ पृ॑ण । \newline
23. आ पृ॑ण पृ॒णा पृ॑ण॒ चित्रा॑वसो॒ चित्रा॑वसो पृ॒णा पृ॑ण॒ चित्रा॑वसो । \newline
24. पृ॒ण॒ चित्रा॑वसो॒ चित्रा॑वसो पृण पृण॒ चित्रा॑वसो स्व॒स्ति स्व॒स्ति चित्रा॑वसो पृण पृण॒ चित्रा॑वसो स्व॒स्ति । \newline
25. चित्रा॑वसो स्व॒स्ति स्व॒स्ति चित्रा॑वसो॒ चित्रा॑वसो स्व॒स्ति ते॑ ते स्व॒स्ति चित्रा॑वसो॒ चित्रा॑वसो स्व॒स्ति ते᳚ । \newline
26. चित्रा॑वसो॒ इति॒ चित्र॑ - व॒सो॒ । \newline
27. स्व॒स्ति ते॑ ते स्व॒स्ति स्व॒स्ति ते॑ पा॒रम् पा॒रम् ते᳚ स्व॒स्ति स्व॒स्ति ते॑ पा॒रम् । \newline
28. ते॒ पा॒रम् पा॒रम् ते॑ ते पा॒र म॑शीयाशीय पा॒रम् ते॑ ते पा॒र म॑शीय । \newline
29. पा॒र म॑शीयाशीय पा॒रम् पा॒र म॑शी॒ये न्धा॑ना॒ इन्धा॑ना अशीय पा॒रम् पा॒र म॑शी॒ये न्धा॑नाः । \newline
30. अ॒शी॒ये न्धा॑ना॒ इन्धा॑ना अशीयाशी॒ये न्धा॑नास्त्वा॒ त्वेन्धा॑ना अशीयाशी॒ये न्धा॑नास्त्वा । \newline
31. इन्धा॑ना स्त्वा॒ त्वेन्धा॑ना॒ इन्धा॑ना स्त्वा श॒तꣳ श॒तम् त्वेन्धा॑ना॒ इन्धा॑ना स्त्वा श॒तम् । \newline
32. त्वा॒ श॒तꣳ श॒तम् त्वा᳚ त्वा श॒तꣳ हिमा॒ हिमाः᳚ श॒तम् त्वा᳚ त्वा श॒तꣳ हिमाः᳚ । \newline
33. श॒तꣳ हिमा॒ हिमाः᳚ श॒तꣳ श॒तꣳ हिमा᳚ द्यु॒मन्तो᳚ द्यु॒मन्तो॒ हिमाः᳚ श॒तꣳ श॒तꣳ हिमा᳚ द्यु॒मन्तः॑ । \newline
34. हिमा᳚ द्यु॒मन्तो᳚ द्यु॒मन्तो॒ हिमा॒ हिमा᳚ द्यु॒मन्तः॒ सꣳ सम् द्यु॒मन्तो॒ हिमा॒ हिमा᳚ द्यु॒मन्तः॒ सम् । \newline
35. द्यु॒मन्तः॒ सꣳ सम् द्यु॒मन्तो᳚ द्यु॒मन्तः॒ स मि॑धीमहीधीमहि॒ सम् द्यु॒मन्तो᳚ द्यु॒मन्तः॒ स मि॑धीमहि । \newline
36. द्यु॒मन्त॒ इति॑ द्यु - मन्तः॑ । \newline
37. स मि॑धीमहीधीमहि॒ सꣳ स मि॑धीमहि॒ वय॑स्वन्तो॒ वय॑स्वन्त इधीमहि॒ सꣳ स मि॑धीमहि॒ वय॑स्वन्तः । \newline
38. इ॒धी॒म॒हि॒ वय॑स्वन्तो॒ वय॑स्वन्त इधीमहीधीमहि॒ वय॑स्वन्तो वय॒स्कृतं॑ ॅवय॒स्कृतं॒ ॅवय॑स्वन्त इधीमहीधीमहि॒ वय॑स्वन्तो वय॒स्कृत᳚म् । \newline
39. वय॑स्वन्तो वय॒स्कृतं॑ ॅवय॒स्कृतं॒ ॅवय॑स्वन्तो॒ वय॑स्वन्तो वय॒स्कृतं॒ ॅयश॑स्वन्तो॒ यश॑स्वन्तो वय॒स्कृतं॒ ॅवय॑स्वन्तो॒ वय॑स्वन्तो वय॒स्कृतं॒ ॅयश॑स्वन्तः । \newline
40. व॒य॒स्कृतं॒ ॅयश॑स्वन्तो॒ यश॑स्वन्तो वय॒स्कृतं॑ ॅवय॒स्कृतं॒ ॅयश॑स्वन्तो यश॒स्कृतं॑ ॅयश॒स्कृतं॒ ॅयश॑स्वन्तो वय॒स्कृतं॑ ॅवय॒स्कृतं॒ ॅयश॑स्वन्तो यश॒स्कृत᳚म् । \newline
41. व॒य॒स्कृत॒मिति॑ वयः - कृत᳚म् । \newline
42. यश॑स्वन्तो यश॒स्कृतं॑ ॅयश॒स्कृतं॒ ॅयश॑स्वन्तो॒ यश॑स्वन्तो यश॒स्कृत(ग्म्॑) सु॒वीरा॑सः सु॒वीरा॑सो यश॒स्कृतं॒ ॅयश॑स्वन्तो॒ यश॑स्वन्तो यश॒स्कृत(ग्म्॑) सु॒वीरा॑सः । \newline
43. य॒श॒स्कृत(ग्म्॑) सु॒वीरा॑सः सु॒वीरा॑सो यश॒स्कृतं॑ ॅयश॒स्कृत(ग्म्॑) सु॒वीरा॑सो॒ अदा᳚भ्य॒ मदा᳚भ्यꣳ सु॒वीरा॑सो यश॒स्कृतं॑ ॅयश॒स्कृत(ग्म्॑) सु॒वीरा॑सो॒ अदा᳚भ्यम् । \newline
44. य॒श॒स्कृत॒मिति॑ यशः - कृत᳚म् । \newline
45. सु॒वीरा॑सो॒ अदा᳚भ्य॒ मदा᳚भ्यꣳ सु॒वीरा॑सः सु॒वीरा॑सो॒ अदा᳚भ्यम् । \newline
46. सु॒वीरा॑स॒ इति॑ सु - वीरा॑सः । \newline
47. अदा᳚भ्य॒मित्यदा᳚भ्यम् । \newline
48. अग्ने॑ सपत्न॒दंभ॑नꣳ सपत्न॒दंभ॑न॒ मग्ने ऽग्ने॑ सपत्न॒दंभ॑नं॒ ॅवर्.षि॑ष्ठे॒ वर्.षि॑ष्ठे सपत्न॒दंभ॑न॒ मग्ने ऽग्ने॑ सपत्न॒दंभ॑नं॒ ॅवर्.षि॑ष्ठे । \newline
49. स॒प॒त्न॒दंभ॑नं॒ ॅवर्.षि॑ष्ठे॒ वर्.षि॑ष्ठे सपत्न॒दंभ॑नꣳ सपत्न॒दंभ॑नं॒ ॅवर्.षि॑ष्ठे॒ अध्यधि॒ वर्.षि॑ष्ठे सपत्न॒दंभ॑नꣳ सपत्न॒दंभ॑नं॒ ॅवर्.षि॑ष्ठे॒ अधि॑ । \newline
50. स॒प॒त्न॒दंभ॑न॒मिति॑ सपत्न - दंभ॑नम् । \newline
51. वर्.षि॑ष्ठे॒ अध्यधि॒ वर्.षि॑ष्ठे॒ वर्.षि॑ष्ठे॒ अधि॒ नाके॒ नाके॒ अधि॒ वर्.षि॑ष्ठे॒ वर्.षि॑ष्ठे॒ अधि॒ नाके᳚ । \newline
52. अधि॒ नाके॒ नाके॒ अध्यधि॒ नाके᳚ । \newline
53. नाक॒ इति॒ नाके᳚ । \newline
54. सम् त्वम् त्वꣳ सꣳ सम् त्व म॑ग्ने अग्ने॒ त्वꣳ सꣳ सम् त्व म॑ग्ने । \newline
55. त्व म॑ग्ने अग्ने॒ त्वम् त्व म॑ग्ने॒ सूर्य॑स्य॒ सूर्य॑स्याग्ने॒ त्वम् त्व म॑ग्ने॒ सूर्य॑स्य । \newline
56. अ॒ग्ने॒ सूर्य॑स्य॒ सूर्य॑स्याग्ने अग्ने॒ सूर्य॑स्य॒ वर्च॑सा॒ वर्च॑सा॒ सूर्य॑स्याग्ने अग्ने॒ सूर्य॑स्य॒ वर्च॑सा । \newline
57. सूर्य॑स्य॒ वर्च॑सा॒ वर्च॑सा॒ सूर्य॑स्य॒ सूर्य॑स्य॒ वर्च॑सा ऽगथा अगथा॒ वर्च॑सा॒ सूर्य॑स्य॒ सूर्य॑स्य॒ वर्च॑सा ऽगथाः । \newline
58. वर्च॑सा ऽगथा अगथा॒ वर्च॑सा॒ वर्च॑सा ऽगथाः॒ सꣳ स म॑गथा॒ वर्च॑सा॒ वर्च॑सा ऽगथाः॒ सम् । \newline
59. अ॒ग॒थाः॒ सꣳ स म॑गथा अगथाः॒ स मृषी॑णा॒ मृषी॑णा॒(ग्म्॒) स म॑गथा अगथाः॒ स मृषी॑णाम् । \newline
60. स मृषी॑णा॒ मृषी॑णा॒(ग्म्॒) सꣳ स मृषी॑णाꣳ स्तु॒तेन॑ स्तु॒तेन र्.षी॑णा॒(ग्म्॒) सꣳ स मृषी॑णाꣳ स्तु॒तेन॑ । \newline
61. ऋषी॑णाꣳ स्तु॒तेन॑ स्तु॒तेन र्‌षी॑णा॒ मृषी॑णाꣳ स्तु॒तेन॒ सꣳ सꣳ स्तु॒तेन र्‌षी॑णा॒ मृषी॑णाꣳ स्तु॒तेन॒ सम् । \newline
62. स्तु॒तेन॒ सꣳ सꣳ स्तु॒तेन॑ स्तु॒तेन॒ सम् प्रि॒येण॑ प्रि॒येण॒ सꣳ स्तु॒तेन॑ स्तु॒तेन॒ सम् प्रि॒येण॑ । \newline
63. सम् प्रि॒येण॑ प्रि॒येण॒ सꣳ सम् प्रि॒येण॒ धाम्ना॒ धाम्ना᳚ प्रि॒येण॒ सꣳ सम् प्रि॒येण॒ धाम्ना᳚ । \newline
64. प्रि॒येण॒ धाम्ना॒ धाम्ना᳚ प्रि॒येण॑ प्रि॒येण॒ धाम्ना᳚ । \newline
65. धाम्नेति॒ धाम्ना᳚ । \newline
66. त्व म॑ग्ने अग्ने॒ त्वम् त्व म॑ग्ने॒ सूर्य॑वर्चाः॒ सूर्य॑वर्चा अग्ने॒ त्वम् त्व म॑ग्ने॒ सूर्य॑वर्चाः । \newline
67. अ॒ग्ने॒ सूर्य॑वर्चाः॒ सूर्य॑वर्चा अग्ने अग्ने॒ सूर्य॑वर्चा अस्यसि॒ सूर्य॑वर्चा अग्ने अग्ने॒ सूर्य॑वर्चा असि । \newline
68. सूर्य॑वर्चा अस्यसि॒ सूर्य॑वर्चाः॒ सूर्य॑वर्चा असि॒ सꣳ स म॑सि॒ सूर्य॑वर्चाः॒ सूर्य॑वर्चा असि॒ सम् । \newline
69. सूर्य॑वर्चा॒ इति॒ सूर्य॑ - व॒र्चाः॒ । \newline
70. अ॒सि॒ सꣳ स म॑स्यसि॒ सम् माम् माꣳ स म॑स्यसि॒ सम् माम् । \newline
71. सम् माम् माꣳ सꣳ सम् मा मायु॒षा ऽऽयु॑षा॒ माꣳ सꣳ सम् मा मायु॑षा । \newline
72. मा मायु॒षा ऽऽयु॑षा॒ माम् मा मायु॑षा॒ वर्च॑सा॒ वर्च॒सा ऽऽयु॑षा॒ माम् मा मायु॑षा॒ वर्च॑सा । \newline
73. आयु॑षा॒ वर्च॑सा॒ वर्च॒सा ऽऽयु॒षा ऽऽयु॑षा॒ वर्च॑सा प्र॒जया᳚ प्र॒जया॒ वर्च॒सा ऽऽयु॒षा ऽऽयु॑षा॒ वर्च॑सा प्र॒जया᳚ । \newline
74. वर्च॑सा प्र॒जया᳚ प्र॒जया॒ वर्च॑सा॒ वर्च॑सा प्र॒जया॑ सृज सृज प्र॒जया॒ वर्च॑सा॒ वर्च॑सा प्र॒जया॑ सृज । \newline
75. प्र॒जया॑ सृज सृज प्र॒जया᳚ प्र॒जया॑ सृज । \newline
76. प्र॒जयेति॑ प्र - जया᳚ । \newline
77. सृ॒जेति॑ सृज । \newline
\pagebreak
\markright{ TS 1.5.6.1  \hfill https://www.vedavms.in \hfill}
\addcontentsline{toc}{section}{ TS 1.5.6.1 }
\section*{ TS 1.5.6.1 }

\textbf{TS 1.5.6.1 } \newline
\textbf{Samhita Paata} \newline

सं प॑श्यामि प्र॒जा अ॒ह-मिड॑प्रजसो मान॒वीः । सर्वा॑ भवन्तु नो गृ॒हे । अंभः॒ स्थांभो॑ वो भक्षीय॒ महः॑ स्थ॒ महो॑ वो भक्षीय॒ सहः॑ स्थ॒ सहो॑ वो भक्षी॒योर्जः॒ स्थोर्जं॑ ॅवो भक्षीय॒ रेव॑ती॒ रम॑द्ध्व-म॒स्मिन् ॅलो॒के᳚ऽस्मिन् गो॒ष्ठे᳚ऽस्मिन् क्षये॒ऽस्मिन् योना॑वि॒हैव स्ते॒तो माऽप॑ गात ब॒ह्वीर्मे॑ भूयास्त - [ ] \newline

\textbf{Pada Paata} \newline

समिति॑ । प॒श्या॒मि॒ । प्र॒जा इति॑ प्र - जाः । अ॒हम् । इड॑प्रजस॒ इतीड॑ - प्र॒ज॒सः॒ । मा॒न॒वीः ॥ सर्वाः᳚ । भ॒व॒न्तु॒ । नः॒ । गृ॒हे ॥ अभंः॑ । स्थ॒ । अम्भः॑ । वः॒ । भ॒क्षी॒य॒ । महः॑ । स्थ॒ । महः॑ । वः॒ । भ॒क्षी॒य॒ । सहः॑ । स्थ॒ । सहः॑ । वः॒ । भ॒क्षी॒य॒ । ऊर्जः॑ । स्थ॒ । ऊर्ज᳚म् । वः॒ । भ॒क्षी॒य॒ । रेव॑तीः । रम॑द्ध्वम् । अ॒स्मिन्न् । लो॒के । अ॒स्मिन्न् । गो॒ष्ठ इति॑ गो-स्थे । अ॒स्मिन्न् । क्षये᳚ । अ॒स्मिन्न् । योनौ᳚ । इ॒ह । ए॒व । स्त॒ । इ॒तः । मा । अपेति॑ । गा॒त॒ । ब॒ह्वीः । मे॒ । भू॒या॒स्त॒ ।  \newline


\textbf{Krama Paata} \newline

सम् प॑श्यामि । प॒श्या॒मि॒ प्र॒जाः । प्र॒जा अ॒हम् । प्र॒जा इति॑ प्र - जाः । अ॒हमिड॑प्रजसः । इड॑प्रजसो मान॒वीः । इड॑प्रजस॒ इतीड॑ - प्र॒ज॒सः॒ । मा॒न॒वीरिति॑ मान॒वीः ॥ सर्वा॑ भवन्तु । भ॒व॒न्तु॒ नः॒ । नो॒ गृ॒हे । गृ॒ह इति॑ गृ॒हे ॥ अम्भः॑ स्थ । स्थाम्भः॑ । अम्भो॑ वः । वो॒ भ॒क्षी॒य॒ । भ॒क्षी॒य॒ महः॑ । महः॑ स्थ । स्थ॒ महः॑ । महो॑ वः । वो॒ भ॒क्षी॒य॒ । भ॒क्षी॒य॒ सहः॑ । सहः॑ स्थ । स्थ॒ सहः॑ । सहो॑ वः । वो॒ भ॒क्षी॒य॒ । भ॒क्षी॒योर्जः॑ । ऊर्जः॑ स्थ । स्थोर्ज᳚म् । ऊर्जं॑ ॅवः । वो॒ भ॒क्षी॒य॒ । भ॒क्षी॒य॒ रेव॑तीः । रेव॑ती॒ रम॑द्ध्वम् । रम॑द्ध्वम॒स्मिन्न् । अ॒स्मिन् ॅलो॒के । लो॒के᳚ऽस्मिन्न् । अ॒स्मिन् गो॒ष्ठे । गो॒ष्ठे᳚ऽस्मिन्न् । गो॒ष्ठ इति॑ गो - स्थे । अ॒स्मिन् क्षये᳚ । क्षये॒ऽस्मिन्न् । अ॒स्मिन्. योनौ᳚ । योना॑वि॒ह । इ॒हैव । ए॒व स्त॑ । स्ते॒तः । इ॒तो मा । माऽप॑ । अप॑ गात । गा॒त॒ ब॒ह्वीः । ब॒ह्वीर् मे᳚ । मे॒ भू॒या॒स्त॒ । भू॒या॒स्त॒ सꣳ॒॒हि॒ता \newline

\textbf{Jatai Paata} \newline

1. सम् प॑श्यामि पश्यामि॒ सꣳ सम् प॑श्यामि । \newline
2. प॒श्या॒मि॒ प्र॒जाः प्र॒जाः प॑श्यामि पश्यामि प्र॒जाः । \newline
3. प्र॒जा अ॒ह म॒हम् प्र॒जाः प्र॒जा अ॒हम् । \newline
4. प्र॒जा इति॑ प्र - जाः । \newline
5. अ॒ह मिड॑प्रजस॒ इड॑प्रजसो॒ ऽह म॒ह मिड॑प्रजसः । \newline
6. इड॑प्रजसो मान॒वीर् मा॑न॒वी रिड॑प्रजस॒ इड॑प्रजसो मान॒वीः । \newline
7. इड॑प्रजस॒ इतीड॑ - प्र॒ज॒सः॒ । \newline
8. मा॒न॒वीरिति॑ मान॒वीः । \newline
9. सर्वा॑ भवन्तु भवन्तु॒ सर्वाः॒ सर्वा॑ भवन्तु । \newline
10. भ॒व॒न्तु॒ नो॒ नो॒ भ॒व॒न्तु॒ भ॒व॒न्तु॒ नः॒ । \newline
11. नो॒ गृ॒हे गृ॒हे नो॑ नो गृ॒हे । \newline
12. गृ॒ह इति॑ गृ॒हे । \newline
13. अंभः॑ स्थ॒ स्थांभ्ॐ ऽभः॑ स्थ । \newline
14. स्थाम्भो ऽम्भः॑ स्थ॒ स्थाम्भः॑ । \newline
15. अम्भो॑ वो॒ वो ऽम्भो ऽम्भो॑ वः । \newline
16. वो॒ भ॒क्षी॒य॒ भ॒क्षी॒य॒ वो॒ वो॒ भ॒क्षी॒य॒ । \newline
17. भ॒क्षी॒य॒ महो॒ महो॑ भक्षीय भक्षीय॒ महः॑ । \newline
18. महः॑ स्थ स्थ॒ महो॒ महः॑ स्थ । \newline
19. स्थ॒ महो॒ महः॑ स्थ स्थ॒ महः॑ । \newline
20. महो॑ वो वो॒ महो॒ महो॑ वः । \newline
21. वो॒ भ॒क्षी॒य॒ भ॒क्षी॒य॒ वो॒ वो॒ भ॒क्षी॒य॒ । \newline
22. भ॒क्षी॒य॒ सहः॒ सहो॑ भक्षीय भक्षीय॒ सहः॑ । \newline
23. सहः॑ स्थ स्थ॒ सहः॒ सहः॑ स्थ । \newline
24. स्थ॒ सहः॒ सहः॑ स्थ स्थ॒ सहः॑ । \newline
25. सहो॑ वो वः॒ सहः॒ सहो॑ वः । \newline
26. वो॒ भ॒क्षी॒य॒ भ॒क्षी॒य॒ वो॒ वो॒ भ॒क्षी॒य॒ । \newline
27. भ॒क्षी॒योर्ज॒ ऊर्जो॑ भक्षीय भक्षी॒योर्जः॑ । \newline
28. ऊर्जः॑ स्थ॒ स्थोर्ज॒ ऊर्जः॑ स्थ । \newline
29. स्थोर्ज॒ मूर्ज(ग्ग्॑) स्थ॒ स्थोर्ज᳚म् । \newline
30. ऊर्जं॑ ॅवो व॒ ऊर्ज॒ मूर्जं॑ ॅवः । \newline
31. वो॒ भ॒क्षी॒य॒ भ॒क्षी॒य॒ वो॒ वो॒ भ॒क्षी॒य॒ । \newline
32. भ॒क्षी॒य॒ रेव॑ती॒ रेव॑तीर् भक्षीय भक्षीय॒ रेव॑तीः । \newline
33. रेव॑ती॒ रम॑द्ध्व॒(ग्म्॒) रम॑द्ध्व॒(ग्म्॒) रेव॑ती॒ रेव॑ती॒ रम॑द्ध्वम् । \newline
34. रम॑द्ध्व म॒स्मिन्न॒स्मिन् रम॑द्ध्व॒(ग्म्॒) रम॑द्ध्व म॒स्मिन्न् । \newline
35. अ॒स्मिन्न् ॅलो॒के लो॒के᳚ ऽस्मिन्न॒स्मिन्न् ॅलो॒के । \newline
36. लो॒के᳚ ऽस्मिन्न॒स्मिन्न् ॅलो॒के लो॒के᳚ ऽस्मिन्न् । \newline
37. अ॒स्मिन् गो॒ष्ठे गो॒ष्ठे᳚ ऽस्मिन्न॒स्मिन् गो॒ष्ठे । \newline
38. गो॒ष्ठे᳚ ऽस्मिन्न॒स्मिन् गो॒ष्ठे गो॒ष्ठे᳚ ऽस्मिन्न् । \newline
39. गो॒ष्ठ इति॑ गो - स्थे । \newline
40. अ॒स्मिन् क्षये॒ क्षये॒ ऽस्मिन्न॒स्मिन् क्षये᳚ । \newline
41. क्षये॒ ऽस्मिन्न॒स्मिन् क्षये॒ क्षये॒ ऽस्मिन्न् । \newline
42. अ॒स्मिन्. योनौ॒ योना॑ व॒स्मिन्न॒स्मिन्. योनौ᳚ । \newline
43. योना॑ वि॒हे ह योनौ॒ योना॑ वि॒ह । \newline
44. इ॒हैवैवे हे हैव । \newline
45. ए॒व स्त॑ स्तै॒वैव स्त॑ । \newline
46. स्ते॒ त इ॒तः स्त॑ स्ते॒ तः । \newline
47. इ॒तो मा मेत इ॒तो मा । \newline
48. मा ऽपाप॒ मा मा ऽप॑ । \newline
49. अप॑ गात गा॒तापाप॑ गात । \newline
50. गा॒त॒ ब॒ह्वीर् ब॒ह्वीर् गा॑त गात ब॒ह्वीः । \newline
51. ब॒ह्वीर् मे॑ मे ब॒ह्वीर् ब॒ह्वीर् मे᳚ । \newline
52. मे॒ भू॒या॒स्त॒ भू॒या॒स्त॒ मे॒ मे॒ भू॒या॒स्त॒ । \newline
53. भू॒या॒स्त॒ स॒(ग्म्॒)हि॒ता स(ग्म्॑)हि॒ता भू॑यास्त भूयास्त सꣳहि॒ता । \newline

\textbf{Ghana Paata } \newline

1. सम् प॑श्यामि पश्यामि॒ सꣳ सम् प॑श्यामि प्र॒जाः प्र॒जाः प॑श्यामि॒ सꣳ सम् प॑श्यामि प्र॒जाः । \newline
2. प॒श्या॒मि॒ प्र॒जाः प्र॒जाः प॑श्यामि पश्यामि प्र॒जा अ॒ह म॒हम् प्र॒जाः प॑श्यामि पश्यामि प्र॒जा अ॒हम् । \newline
3. प्र॒जा अ॒ह म॒हम् प्र॒जाः प्र॒जा अ॒ह मिड॑प्रजस॒ इड॑प्रजसो॒ ऽहम् प्र॒जाः प्र॒जा अ॒ह मिड॑प्रजसः । \newline
4. प्र॒जा इति॑ प्र - जाः । \newline
5. अ॒ह मिड॑प्रजस॒ इड॑प्रजसो॒ ऽह म॒ह मिड॑प्रजसो मान॒वीर् मा॑न॒वी रिड॑प्रजसो॒ ऽह म॒ह मिड॑प्रजसो मान॒वीः । \newline
6. इड॑प्रजसो मान॒वीर् मा॑न॒वी रिड॑प्रजस॒ इड॑प्रजसो मान॒वीः । \newline
7. इड॑प्रजस॒ इतीड॑ - प्र॒ज॒सः॒ । \newline
8. मा॒न॒वीरिति॑ मान॒वीः । \newline
9. सर्वा॑ भवन्तु भवन्तु॒ सर्वाः॒ सर्वा॑ भवन्तु नो नो भवन्तु॒ सर्वाः॒ सर्वा॑ भवन्तु नः । \newline
10. भ॒व॒न्तु॒ नो॒ नो॒ भ॒व॒न्तु॒ भ॒व॒न्तु॒ नो॒ गृ॒हे गृ॒हे नो॑ भवन्तु भवन्तु नो गृ॒हे । \newline
11. नो॒ गृ॒हे गृ॒हे नो॑ नो गृ॒हे । \newline
12. गृ॒ह इति॑ गृ॒हे । \newline
13. अंभः॑ स्थ॒ स्थांभ्ॐ भः॒ स्थाम्भो ऽम्भः॒ स्थांभ्ॐ भः॒ स्थाम्भः॑ । \newline
14. स्थाम्भो ऽम्भः॑ स्थ॒ स्थाम्भो॑ वो॒ वो ऽम्भः॑ स्थ॒ स्थाम्भो॑ वः । \newline
15. अम्भो॑ वो॒ वो ऽम्भो ऽम्भो॑ वो भक्षीय भक्षीय॒ वो ऽम्भो ऽम्भो॑ वो भक्षीय । \newline
16. वो॒ भ॒क्षी॒य॒ भ॒क्षी॒य॒ वो॒ वो॒ भ॒क्षी॒य॒ महो॒ महो॑ भक्षीय वो वो भक्षीय॒ महः॑ । \newline
17. भ॒क्षी॒य॒ महो॒ महो॑ भक्षीय भक्षीय॒ महः॑ स्थ स्थ॒ महो॑ भक्षीय भक्षीय॒ महः॑ स्थ । \newline
18. महः॑ स्थ स्थ॒ महो॒ महः॑ स्थ॒ महो॒ महः॑ स्थ॒ महो॒ महः॑ स्थ॒ महः॑ । \newline
19. स्थ॒ महो॒ महः॑ स्थ स्थ॒ महो॑ वो वो॒ महः॑ स्थ स्थ॒ महो॑ वः । \newline
20. महो॑ वो वो॒ महो॒ महो॑ वो भक्षीय भक्षीय वो॒ महो॒ महो॑ वो भक्षीय । \newline
21. वो॒ भ॒क्षी॒य॒ भ॒क्षी॒य॒ वो॒ वो॒ भ॒क्षी॒य॒ सहः॒ सहो॑ भक्षीय वो वो भक्षीय॒ सहः॑ । \newline
22. भ॒क्षी॒य॒ सहः॒ सहो॑ भक्षीय भक्षीय॒ सहः॑ स्थ स्थ॒ सहो॑ भक्षीय भक्षीय॒ सहः॑ स्थ । \newline
23. सहः॑ स्थ स्थ॒ सहः॒ सहः॑ स्थ॒ सहः॒ सहः॑ स्थ॒ सहः॒ सहः॑ स्थ॒ सहः॑ । \newline
24. स्थ॒ सहः॒ सहः॑ स्थ स्थ॒ सहो॑ वो वः॒ सहः॑ स्थ स्थ॒ सहो॑ वः । \newline
25. सहो॑ वो वः॒ सहः॒ सहो॑ वो भक्षीय भक्षीय वः॒ सहः॒ सहो॑ वो भक्षीय । \newline
26. वो॒ भ॒क्षी॒य॒ भ॒क्षी॒य॒ वो॒ वो॒ भ॒क्षी॒योर्ज॒ ऊर्जो॑ भक्षीय वो वो भक्षी॒योर्जः॑ । \newline
27. भ॒क्षी॒योर्ज॒ ऊर्जो॑ भक्षीय भक्षी॒योर्जः॑ स्थ॒ स्थोर्जो॑ भक्षीय भक्षी॒योर्जः॑ स्थ । \newline
28. ऊर्जः॑ स्थ॒ स्थोर्ज॒ ऊर्जः॒ स्थोर्ज॒ मूर्ज॒(ग्ग्॒) स्थोर्ज॒ ऊर्जः॒ स्थोर्ज᳚म् । \newline
29. स्थोर्ज॒ मूर्ज(ग्ग्॑) स्थ॒ स्थोर्जं॑ ॅवो व॒ ऊर्ज(ग्ग्॑) स्थ॒ स्थोर्जं॑ ॅवः । \newline
30. ऊर्जं॑ ॅवो व॒ ऊर्ज॒ मूर्जं॑ ॅवो भक्षीय भक्षीय व॒ ऊर्ज॒ मूर्जं॑ ॅवो भक्षीय । \newline
31. वो॒ भ॒क्षी॒य॒ भ॒क्षी॒य॒ वो॒ वो॒ भ॒क्षी॒य॒ रेव॑ती॒ रेव॑तीर् भक्षीय वो वो भक्षीय॒ रेव॑तीः । \newline
32. भ॒क्षी॒य॒ रेव॑ती॒ रेव॑तीर् भक्षीय भक्षीय॒ रेव॑ती॒ रम॑द्ध्व॒(ग्म्॒) रम॑द्ध्व॒(ग्म्॒) रेव॑तीर् भक्षीय भक्षीय॒ रेव॑ती॒ रम॑द्ध्वम् । \newline
33. रेव॑ती॒ रम॑द्ध्व॒(ग्म्॒) रम॑द्ध्व॒(ग्म्॒) रेव॑ती॒ रेव॑ती॒ रम॑द्ध्व म॒स्मिन् न॒स्मिन् रम॑द्ध्व॒(ग्म्॒) रेव॑ती॒ रेव॑ती॒ रम॑द्ध्व म॒स्मिन्न् । \newline
34. रम॑द्ध्व म॒स्मिन् न॒स्मिन् रम॑द्ध्व॒(ग्म्॒) रम॑द्ध्व म॒स्मिन् ॅलो॒के लो॒के᳚ ऽस्मिन् रम॑द्ध्व॒(ग्म्॒) रम॑द्ध्व म॒स्मिन् ॅलो॒के । \newline
35. अ॒स्मिन् ॅलो॒के लो॒के᳚ ऽस्मिन् न॒स्मिन्न् ॅलो॒के᳚ ऽस्मिन् न॒स्मिन् ॅलो॒के᳚ ऽस्मिन् न॒स्मिन् ॅलो॒के᳚ ऽस्मिन्न् । \newline
36. लो॒के᳚ ऽस्मिन् न॒स्मिन् ॅलो॒के लो॒के᳚ ऽस्मिन् गो॒ष्ठे गो॒ष्ठे᳚ ऽस्मिन् ॅलो॒के लो॒के᳚ ऽस्मिन् गो॒ष्ठे । \newline
37. अ॒स्मिन् गो॒ष्ठे गो॒ष्ठे᳚ ऽस्मिन् न॒स्मिन् गो॒ष्ठे᳚ ऽस्मिन् न॒स्मिन् गो॒ष्ठे᳚ ऽस्मिन् न॒स्मिन् गो॒ष्ठे᳚ ऽस्मिन्न् । \newline
38. गो॒ष्ठे᳚ ऽस्मिन् न॒स्मिन् गो॒ष्ठे गो॒ष्ठे᳚ ऽस्मिन् क्षये॒ क्षये॒ ऽस्मिन् गो॒ष्ठे गो॒ष्ठे᳚ ऽस्मिन् क्षये᳚ । \newline
39. गो॒ष्ठ इति॑ गो - स्थे । \newline
40. अ॒स्मिन् क्षये॒ क्षये॒ ऽस्मिन् न॒स्मिन् क्षये॒ ऽस्मिन् न॒स्मिन् क्षये॒ ऽस्मिन् न॒स्मिन् क्षये॒ ऽस्मिन्न् । \newline
41. क्षये॒ ऽस्मिन् न॒स्मिन् क्षये॒ क्षये॒ ऽस्मिन्. योनौ॒ योना॑ व॒स्मिन् क्षये॒ क्षये॒ ऽस्मिन्. योनौ᳚ । \newline
42. अ॒स्मिन्. योनौ॒ योना॑ व॒स्मिन् न॒स्मिन्. योना॑ वि॒हे ह योना॑ व॒स्मिन् न॒स्मिन्. योना॑ वि॒ह । \newline
43. योना॑ वि॒हे ह योनौ॒ योना॑ वि॒हैवैवे ह योनौ॒ योना॑ वि॒हैव । \newline
44. इ॒हैवैवे हे हैव स्त॑ स्तै॒वे हे हैव स्त॑ । \newline
45. ए॒व स्त॑ स्तै॒वैव स्ते॒ त इ॒तः स्तै॒वैव स्ते॒ तः । \newline
46. स्ते॒ त इ॒तः स्त॑ स्ते॒ तो मा मेतः स्त॑ स्ते॒ तो मा । \newline
47. इ॒तो मा मेत इ॒तो मा ऽपाप॒ मेत इ॒तो मा ऽप॑ । \newline
48. मा ऽपाप॒ मा मा ऽप॑ गात गा॒ताप॒ मा मा ऽप॑ गात । \newline
49. अप॑ गात गा॒तापाप॑ गात ब॒ह्वीर् ब॒ह्वीर् गा॒तापाप॑ गात ब॒ह्वीः । \newline
50. गा॒त॒ ब॒ह्वीर् ब॒ह्वीर् गा॑त गात ब॒ह्वीर् मे॑ मे ब॒ह्वीर् गा॑त गात ब॒ह्वीर् मे᳚ । \newline
51. ब॒ह्वीर् मे॑ मे ब॒ह्वीर् ब॒ह्वीर् मे॑ भूयास्त भूयास्त मे ब॒ह्वीर् ब॒ह्वीर् मे॑ भूयास्त । \newline
52. मे॒ भू॒या॒स्त॒ भू॒या॒स्त॒ मे॒ मे॒ भू॒या॒स्त॒ स॒(ग्म्॒)हि॒ता स(ग्म्॑)हि॒ता भू॑यास्त मे मे भूयास्त सꣳहि॒ता । \newline
53. भू॒या॒स्त॒ स॒(ग्म्॒)हि॒ता स(ग्म्॑)हि॒ता भू॑यास्त भूयास्त सꣳहि॒ता ऽस्य॑सि सꣳहि॒ता भू॑यास्त भूयास्त सꣳहि॒ता ऽसि॑ । \newline
\pagebreak
\markright{ TS 1.5.6.2  \hfill https://www.vedavms.in \hfill}
\addcontentsline{toc}{section}{ TS 1.5.6.2 }
\section*{ TS 1.5.6.2 }

\textbf{TS 1.5.6.2 } \newline
\textbf{Samhita Paata} \newline

सꣳहि॒ताऽसि॑ विश्वरू॒पीरा मो॒र्जा वि॒शा ऽऽ*गौ॑प॒त्येना ऽऽ*रा॒यस्पोषे॑ण सहस्रपो॒षं ॅवः॑ पुष्यासं॒ मयि॑ वो॒ रायः॑ श्रयन्तां ॥ उप॑ त्वाऽग्ने दि॒वेदि॑वे॒ दोषा॑वस्तर्द्धि॒या व॒यं । नमो॒ भर॑न्त॒ एम॑सि ॥ राज॑न्तमद्ध्व॒राणां᳚ गो॒पामृ॒तस्य॒ दीदि॑विं । वर्द्ध॑मानꣳ॒॒ स्वे दमे᳚ ॥ स नः॑ पि॒तेव॑ सू॒नवेऽग्ने॑ सूपाय॒नो भ॑व । सच॑स्वा नः स्व॒स्तये᳚ ॥ अग्ने॒- [ ] \newline

\textbf{Pada Paata} \newline

सꣳ॒॒हि॒तेति॑ सं - हि॒ता । अ॒सि॒ । वि॒श्व॒रू॒पीरिति॑ विश्व - रू॒पीः । एति॑ । मा॒ । ऊ॒र्जा । वि॒श॒ । एति॑ । गौ॒प॒त्येन॑ । एति॑ । रा॒यः । पोषे॑ण । स॒ह॒स्र॒पो॒षमिति॑ सहस्र - पो॒षम् । वः॒ । पु॒ष्या॒स॒म् । मयि॑ । वः॒ । रायः॑ । श्र॒य॒न्ता॒म् ॥ उपेति॑ । त्वा॒ । अ॒ग्ने॒ । दि॒वेदि॑व॒ इति॑ दि॒वे-दि॒वे॒ । दोषा॑वस्त॒रिति॒ दोषा᳚-व॒स्तः॒ । धि॒या । व॒यम् ॥ नमः॑ । भर॑न्तः । एति॑ । इ॒म॒सि॒ ॥ राज॑न्तम् । अ॒द्ध्व॒राणा᳚म् । गो॒पामिति॑ गो - पाम् । ऋ॒तस्य॑ । दीदि॑विम् ॥ वर्ध॑मानम् । स्वे । दमे᳚ ॥ सः । नः॒ । पि॒ता । इ॒व॒ । सू॒नवे᳚ । अग्ने᳚ । सू॒पा॒य॒न इति॑ सु - उ॒पा॒य॒नः । भ॒व॒ ॥ सच॑स्व ।  नः॒ । स्व॒स्तये᳚ ॥ अग्ने᳚ ।  \newline


\textbf{Krama Paata} \newline

सꣳ॒॒हि॒ताऽसि॑ । सꣳ॒॒हि॒तेति॑ सम् - हि॒ता । अ॒सि॒ वि॒श्व॒रू॒पीः । वि॒श्व॒रू॒पीरा । वि॒श्व॒रू॒पीरिति॑ विश्व - रू॒पीः । आ मा᳚ । मो॒र्जा । ऊ॒र्जा वि॑श । वि॒शा । आ गौ॑प॒त्येन॑ । गौ॒प॒त्येना । आ रा॒यः । रा॒यस्पोषे॑ण । पोषे॑ण सहस्रपो॒षम् । स॒ह॒स्र॒पो॒षं ॅवः॑ । स॒ह॒स्र॒पो॒षमिति॑ सहस्र - पो॒षम् । वः॒ पु॒ष्या॒स॒म् । पु॒ष्या॒स॒म् मयि॑ । मयि॑ वः । वो॒ रायः॑ । रायः॑ श्रयन्ताम् । श्र॒य॒न्ता॒मिति॑ श्रयन्ताम् ॥ उप॑ त्वा । त्वा॒ऽग्ने॒ । अ॒ग्ने॒ दि॒वेदि॑वे । दि॒वेदि॑वे॒ दोषा॑वस्तः । दि॒वेदि॑व॒ इति॑ दि॒वे - दि॒वे॒ । दोषा॑वस्तर् धि॒या । दोषा॑वस्त॒रिति॒ दोषा᳚ - व॒स्तः॒ । धि॒या व॒यम् । व॒यमिति॑ व॒यम् ॥ नमो॒ भर॑न्तः । भर॑न्त॒ आ । एम॑सि । इ॒म॒सीती॑मसि ॥ राज॑न्तमद्ध्व॒राणा᳚म् । अ॒द्ध्व॒राणा᳚म् गो॒पाम् । गो॒पामृ॒तस्य॑ । गो॒पामिति॑ गो - पाम् । ऋ॒तस्य॒ दीदि॑विम् । दीदि॑वि॒मिति॒ दीदि॑विम् । वर्द्ध॑मानꣳ॒॒ स्वे । स्वे दमे᳚ । दम॒ इति॒ दमे᳚ ॥ स नः॑ । नः॒ पि॒ता । पि॒तेव॑ । इ॒व॒ सू॒नवे᳚ । सू॒नवेऽग्ने᳚ । अग्ने॑ सूपाय॒नः । सू॒पा॒य॒नो भ॑व । सू॒पा॒य॒न इति॑ सु - उ॒पा॒य॒नः । भ॒वेति॑ भव ॥ सच॑स्वा नः । नः॒ स्व॒स्तये᳚ । स्व॒स्तय॒ इति॑ स्व॒स्तये᳚ ॥ अग्ने॒ त्वम् \newline

\textbf{Jatai Paata} \newline

1. स॒(ग्म्॒)हि॒ता ऽस्य॑सि सꣳहि॒ता स(ग्म्॑)हि॒ता ऽसि॑ । \newline
2. स॒(ग्म्॒)हि॒तेति॑ सं - हि॒ता । \newline
3. अ॒सि॒ वि॒श्व॒रू॒पीर् वि॑श्वरू॒पी र॑स्यसि विश्वरू॒पीः । \newline
4. वि॒श्व॒रू॒पीरा वि॑श्वरू॒पीर् वि॑श्वरू॒पीरा । \newline
5. वि॒श्व॒रू॒पीरिति॑ विश्व - रू॒पीः । \newline
6. आ मा॒ मा ऽऽमा᳚ । \newline
7. मो॒र्जोर्जा मा॑ मो॒र्जा । \newline
8. ऊ॒र्जा वि॑श विशो॒र्जोर्जा वि॑श । \newline
9. वि॒शा वि॑श वि॒शा । \newline
10. आ गौ॑प॒त्येन॑ गौप॒त्येना गौ॑प॒त्येन॑ । \newline
11. गौ॒प॒त्येना गौ॑प॒त्येन॑ गौप॒त्येना । \newline
12. आ रा॒यो रा॒य आ रा॒यः । \newline
13. रा॒य स्पोषे॑ण॒ पोषे॑ण रा॒यो रा॒य स्पोषे॑ण । \newline
14. पोषे॑ण सहस्रपो॒षꣳ स॑हस्रपो॒षम् पोषे॑ण॒ पोषे॑ण सहस्रपो॒षम् । \newline
15. स॒ह॒स्र॒पो॒षं ॅवो॑ वः सहस्रपो॒षꣳ स॑हस्रपो॒षं ॅवः॑ । \newline
16. स॒ह॒स्र॒पो॒षमिति॑ सहस्र - पो॒षम् । \newline
17. वः॒ पु॒ष्या॒स॒म् पु॒ष्या॒सं॒ ॅवो॒ वः॒ पु॒ष्या॒स॒म् । \newline
18. पु॒ष्या॒स॒म् मयि॒ मयि॑ पुष्यासम् पुष्यास॒म् मयि॑ । \newline
19. मयि॑ वो वो॒ मयि॒ मयि॑ वः । \newline
20. वो॒ रायो॒ रायो॑ वो वो॒ रायः॑ । \newline
21. रायः॑ श्रयन्ताꣳ श्रयन्ता॒(ग्म्॒) रायो॒ रायः॑ श्रयन्ताम् । \newline
22. श्र॒य॒न्ता॒मिति॑ श्रयन्ताम् । \newline
23. उप॑ त्वा॒ त्वोपोप॑ त्वा । \newline
24. त्वा॒ ऽग्ने॒ ऽग्ने॒ त्वा॒ त्वा॒ ऽग्ने॒ । \newline
25. अ॒ग्ने॒ दि॒वेदि॑वे दि॒वेदि॑वे ऽग्ने ऽग्ने दि॒वेदि॑वे । \newline
26. दि॒वेदि॑वे॒ दोषा॑वस्त॒र् दोषा॑वस्तर् दि॒वेदि॑वे दि॒वेदि॑वे॒ दोषा॑वस्तः । \newline
27. दि॒वेदि॑व॒ इति॑ दि॒वे - दि॒वे॒ । \newline
28. दोषा॑वस्तर् धि॒या धि॒या दोषा॑वस्त॒र् दोषा॑वस्तर् धि॒या । \newline
29. दोषा॑वस्त॒रिति॒ दोषा᳚ - व॒स्तः॒ । \newline
30. धि॒या व॒यं ॅव॒यम् धि॒या धि॒या व॒यम् । \newline
31. व॒यमिति॑ व॒यम् । \newline
32. नमो॒ भर॑न्तो॒ भर॑न्तो॒ नमो॒ नमो॒ भर॑न्तः । \newline
33. भर॑न्त॒ आ भर॑न्तो॒ भर॑न्त॒ आ । \newline
34. एम॑सीम॒स्येम॑सि । \newline
35. इ॒म॒सीती॑मसि । \newline
36. राज॑न्त मद्ध्व॒राणा॑ मद्ध्व॒राणा॒(ग्म्॒) राज॑न्त॒(ग्म्॒) राज॑न्त मद्ध्व॒राणा᳚म् । \newline
37. अ॒द्ध्व॒राणा᳚म् गो॒पाम् गो॒पा म॑द्ध्व॒राणा॑ मद्ध्व॒राणा᳚म् गो॒पाम् । \newline
38. गो॒पा मृ॒तस्य॒ र्तस्य॑ गो॒पाम् गो॒पा मृ॒तस्य॑ । \newline
39. गो॒पामिति॑ गो - पाम् । \newline
40. ऋ॒तस्य॒ दीदि॑वि॒म् दीदि॑वि मृ॒तस्य॒ र्तस्य॒ दीदि॑विम् । \newline
41. दीदि॑वि॒मिति॒ दीदि॑विम् । \newline
42. वर्द्ध॑मान॒(ग्ग्॒) स्वे स्वे वर्द्ध॑मानं॒ ॅवर्द्ध॑मान॒(ग्ग्॒) स्वे । \newline
43. स्वे दमे॒ दमे॒ स्वे स्वे दमे᳚ । \newline
44. दम॒ इति॒ दमे᳚ । \newline
45. स नो॑ नः॒ स सो नः॑ । \newline
46. नः॒ पि॒ता पि॒ता नो॑ नः पि॒ता । \newline
47. पि॒तेवे॑ व पि॒ता पि॒तेव॑ । \newline
48. इ॒व॒ सू॒नवे॑ सू॒नव॑ इवे व सू॒नवे᳚ । \newline
49. सू॒नवे ऽग्ने ऽग्ने॑ सू॒नवे॑ सू॒नवे ऽग्ने᳚ । \newline
50. अग्ने॑ सूपाय॒नः सू॑पाय॒नो ऽग्ने ऽग्ने॑ सूपाय॒नः । \newline
51. सू॒पा॒य॒नो भ॑व भव सूपाय॒नः सू॑पाय॒नो भ॑व । \newline
52. सू॒पा॒य॒न इति॑ सु - उ॒पा॒य॒नः । \newline
53. भ॒वेति॑ भव । \newline
54. सच॑स्वा नो नः॒ सच॑स्व॒ सच॑स्वा नः । \newline
55. नः॒ स्व॒स्तये᳚ स्व॒स्तये॑ नो नः स्व॒स्तये᳚ । \newline
56. स्व॒स्तय॒ इति॑ स्व॒स्तये᳚ । \newline
57. अग्ने॒ त्वम् त्व मग्ने ऽग्ने॒ त्वम् । \newline

\textbf{Ghana Paata } \newline

1. स॒(ग्म्॒)हि॒ता ऽस्य॑सि सꣳहि॒ता स(ग्म्॑)हि॒ता ऽसि॑ विश्वरू॒पीर् वि॑श्वरू॒पीर॑सि सꣳहि॒ता स(ग्म्॑)हि॒ता ऽसि॑ विश्वरू॒पीः । \newline
2. स॒(ग्म्॒)हि॒तेति॑ सं - हि॒ता । \newline
3. अ॒सि॒ वि॒श्व॒रू॒पीर् वि॑श्वरू॒पी र॑स्यसि विश्वरू॒पीरा वि॑श्वरू॒पी र॑स्यसि विश्वरू॒पीरा । \newline
4. वि॒श्व॒रू॒पीरा वि॑श्वरू॒पीर् वि॑श्वरू॒पीरा मा॒ मा ऽऽवि॑श्वरू॒पीर् वि॑श्वरू॒पीरा मा᳚ । \newline
5. वि॒श्व॒रू॒पीरिति॑ विश्व - रू॒पीः । \newline
6. आ मा॒ मा ऽऽमो॒र्जोर्जा मा ऽऽमो॒र्जा । \newline
7. मो॒र्जोर्जा मा॑ मो॒र्जा वि॑श विशो॒र्जा मा॑ मो॒र्जा वि॑श । \newline
8. ऊ॒र्जा वि॑श विशो॒र्जोर्जा वि॒शा वि॑शो॒र्जोर्जा वि॒शा । \newline
9. वि॒शा वि॑श वि॒शा गौ॑प॒त्येन॑ गौप॒त्येना वि॑श वि॒शा गौ॑प॒त्येन॑ । \newline
10. आ गौ॑प॒त्येन॑ गौप॒त्येना गौ॑प॒त्येना गौ॑प॒त्येना गौ॑प॒त्येना । \newline
11. गौ॒प॒त्येना गौ॑प॒त्येन॑ गौप॒त्येना रा॒यो रा॒य आ गौ॑प॒त्येन॑ गौप॒त्येना रा॒यः । \newline
12. आ रा॒यो रा॒य आ रा॒य स्पोषे॑ण॒ पोषे॑ण रा॒य आ रा॒य स्पोषे॑ण । \newline
13. रा॒य स्पोषे॑ण॒ पोषे॑ण रा॒यो रा॒य स्पोषे॑ण सहस्रपो॒षꣳ स॑हस्रपो॒षम् पोषे॑ण रा॒यो रा॒य स्पोषे॑ण सहस्रपो॒षम् । \newline
14. पोषे॑ण सहस्रपो॒षꣳ स॑हस्रपो॒षम् पोषे॑ण॒ पोषे॑ण सहस्रपो॒षं ॅवो॑ वः सहस्रपो॒षम् पोषे॑ण॒ पोषे॑ण सहस्रपो॒षं ॅवः॑ । \newline
15. स॒ह॒स्र॒पो॒षं ॅवो॑ वः सहस्रपो॒षꣳ स॑हस्रपो॒षं ॅवः॑ पुष्यासम् पुष्यासं ॅवः सहस्रपो॒षꣳ स॑हस्रपो॒षं ॅवः॑ पुष्यासम् । \newline
16. स॒ह॒स्र॒पो॒षमिति॑ सहस्र - पो॒षम् । \newline
17. वः॒ पु॒ष्या॒स॒म् पु॒ष्या॒सं॒ ॅवो॒ वः॒ पु॒ष्या॒स॒म् मयि॒ मयि॑ पुष्यासं ॅवो वः पुष्यास॒म् मयि॑ । \newline
18. पु॒ष्या॒स॒म् मयि॒ मयि॑ पुष्यासम् पुष्यास॒म् मयि॑ वो वो॒ मयि॑ पुष्यासम् पुष्यास॒म् मयि॑ वः । \newline
19. मयि॑ वो वो॒ मयि॒ मयि॑ वो॒ रायो॒ रायो॑ वो॒ मयि॒ मयि॑ वो॒ रायः॑ । \newline
20. वो॒ रायो॒ रायो॑ वो वो॒ रायः॑ श्रयन्ताꣳ श्रयन्ता॒(ग्म्॒) रायो॑ वो वो॒ रायः॑ श्रयन्ताम् । \newline
21. रायः॑ श्रयन्ताꣳ श्रयन्ता॒(ग्म्॒) रायो॒ रायः॑ श्रयन्ताम् । \newline
22. श्र॒य॒न्ता॒मिति॑ श्रयन्ताम् । \newline
23. उप॑ त्वा॒ त्वोपोप॑ त्वा ऽग्ने ऽग्ने॒ त्वोपोप॑ त्वा ऽग्ने । \newline
24. त्वा॒ ऽग्ने॒ ऽग्ने॒ त्वा॒ त्वा॒ ऽग्ने॒ दि॒वेदि॑वे दि॒वेदि॑वे ऽग्ने त्वा त्वा ऽग्ने दि॒वेदि॑वे । \newline
25. अ॒ग्ने॒ दि॒वेदि॑वे दि॒वेदि॑वे ऽग्ने ऽग्ने दि॒वेदि॑वे॒ दोषा॑वस्त॒र् दोषा॑वस्तर् दि॒वेदि॑वे ऽग्ने ऽग्ने दि॒वेदि॑वे॒ दोषा॑वस्तः । \newline
26. दि॒वेदि॑वे॒ दोषा॑वस्त॒र् दोषा॑वस्तर् दि॒वेदि॑वे दि॒वेदि॑वे॒ दोषा॑वस्तर् धि॒या धि॒या दोषा॑वस्तर् दि॒वेदि॑वे दि॒वेदि॑वे॒ दोषा॑वस्तर् धि॒या । \newline
27. दि॒वेदि॑व॒ इति॑ दि॒वे - दि॒वे॒ । \newline
28. दोषा॑वस्तर् धि॒या धि॒या दोषा॑वस्त॒र् दोषा॑वस्तर् धि॒या व॒यं ॅव॒यम् धि॒या दोषा॑वस्त॒र् दोषा॑वस्तर् धि॒या व॒यम् । \newline
29. दोषा॑वस्त॒रिति॒ दोषा᳚ - व॒स्तः॒ । \newline
30. धि॒या व॒यं ॅव॒यम् धि॒या धि॒या व॒यम् । \newline
31. व॒यमिति॑ व॒यम् । \newline
32. नमो॒ भर॑न्तो॒ भर॑न्तो॒ नमो॒ नमो॒ भर॑न्त॒ आ भर॑न्तो॒ नमो॒ नमो॒ भर॑न्त॒ आ । \newline
33. भर॑न्त॒ आ भर॑न्तो॒ भर॑न्त॒ एम॑सीम॒स्या भर॑न्तो॒ भर॑न्त॒ एम॑सि । \newline
34. एम॑सीम॒स्येम॑सि । \newline
35. इ॒म॒सीती॑मसि । \newline
36. राज॑न्त मद्ध्व॒राणा॑ मद्ध्व॒राणा॒(ग्म्॒) राज॑न्त॒(ग्म्॒) राज॑न्त मद्ध्व॒राणा᳚म् गो॒पाम् गो॒पा म॑द्ध्व॒राणा॒(ग्म्॒) राज॑न्त॒(ग्म्॒) राज॑न्त मद्ध्व॒राणा᳚म् गो॒पाम् । \newline
37. अ॒द्ध्व॒राणा᳚म् गो॒पाम् गो॒पा म॑द्ध्व॒राणा॑ मद्ध्व॒राणा᳚म् गो॒पा मृ॒तस्य॒ र्‌तस्य॑ गो॒पा म॑द्ध्व॒राणा॑ मद्ध्व॒राणा᳚म् गो॒पा मृ॒तस्य॑ । \newline
38. गो॒पा मृ॒तस्य॒ र्‌तस्य॑ गो॒पाम् गो॒पा मृ॒तस्य॒ दीदि॑वि॒म् दीदि॑वि मृ॒तस्य॑ गो॒पाम् गो॒पा मृ॒तस्य॒ दीदि॑विम् । \newline
39. गो॒पामिति॑ गो - पाम् । \newline
40. ऋ॒तस्य॒ दीदि॑वि॒म् दीदि॑वि मृ॒तस्य॒ र्तस्य॒ दीदि॑विम् । \newline
41. दीदि॑वि॒मिति॒ दीदि॑विम् । \newline
42. वर्द्ध॑मान॒(ग्ग्॒) स्वे स्वे वर्द्ध॑मानं॒ ॅवर्द्ध॑मान॒(ग्ग्॒) स्वे दमे॒ दमे॒ स्वे वर्द्ध॑मानं॒ ॅवर्द्ध॑मान॒(ग्ग्॒) स्वे दमे᳚ । \newline
43. स्वे दमे॒ दमे॒ स्वे स्वे दमे᳚ । \newline
44. दम॒ इति॒ दमे᳚ । \newline
45. स नो॑ नः॒ स स नः॑ पि॒ता पि॒ता नः॒ स स नः॑ पि॒ता । \newline
46. नः॒ पि॒ता पि॒ता नो॑ नः पि॒तेवे॑ व पि॒ता नो॑ नः पि॒तेव॑ । \newline
47. पि॒तेवे॑ व पि॒ता पि॒तेव॑ सू॒नवे॑ सू॒नव॑ इव पि॒ता पि॒तेव॑ सू॒नवे᳚ । \newline
48. इ॒व॒ सू॒नवे॑ सू॒नव॑ इवे व सू॒नवे ऽग्ने ऽग्ने॑ सू॒नव॑ इवे व सू॒नवे ऽग्ने᳚ । \newline
49. सू॒नवे ऽग्ने ऽग्ने॑ सू॒नवे॑ सू॒नवे ऽग्ने॑ सूपाय॒नः सू॑पाय॒नो ऽग्ने॑ सू॒नवे॑ सू॒नवे ऽग्ने॑ सूपाय॒नः । \newline
50. अग्ने॑ सूपाय॒नः सू॑पाय॒नो ऽग्ने ऽग्ने॑ सूपाय॒नो भ॑व भव सूपाय॒नो ऽग्ने ऽग्ने॑ सूपाय॒नो भ॑व । \newline
51. सू॒पा॒य॒नो भ॑व भव सूपाय॒नः सू॑पाय॒नो भ॑व । \newline
52. सू॒पा॒य॒न इति॑ सु - उ॒पा॒य॒नः । \newline
53. भ॒वेति॑ भव । \newline
54. सच॑स्वा नो नः॒ सच॑स्व॒ सच॑स्वा नः स्व॒स्तये᳚ स्व॒स्तये॑ नः॒ सच॑स्व॒ सच॑स्वा नः स्व॒स्तये᳚ । \newline
55. नः॒ स्व॒स्तये᳚ स्व॒स्तये॑ नो नः स्व॒स्तये᳚ । \newline
56. स्व॒स्तय॒ इति॑ स्व॒स्तये᳚ । \newline
57. अग्ने॒ त्वम् त्व मग्ने ऽग्ने॒ त्वन्नो॑ न॒स्त्व मग्ने ऽग्ने॒ त्वन्नः॑ । \newline
\pagebreak
\markright{ TS 1.5.6.3  \hfill https://www.vedavms.in \hfill}
\addcontentsline{toc}{section}{ TS 1.5.6.3 }
\section*{ TS 1.5.6.3 }

\textbf{TS 1.5.6.3 } \newline
\textbf{Samhita Paata} \newline

त्वं नो॒ अन्त॑मः । उ॒त त्रा॒ता शि॒वो भ॑व वरू॒त्थ्यः॑ ॥ तं त्वा॑ शोचिष्ठ दीदिवः । सु॒म्नाय॑ नू॒नमी॑महे॒ सखि॑भ्यः ॥ वसु॑र॒ग्निर् वसु॑श्रवाः । अच्छा॑ नक्षि द्यु॒मत्त॑मो र॒यिं दाः᳚ ॥ ऊ॒र्जा वः॑ पश्याम्यू॒र्जा मा॑ पश्यत रा॒यस्पोषे॑ण वः पश्यामि रा॒यस्पोषे॑ण मा पश्य॒तेडाः᳚ स्थ मधु॒कृतः॑ स्यो॒ना मा ऽऽवि॑श॒तेरा॒ मदः॑ । स॒ह॒स्र॒पो॒षं ॅवः॑ पुष्यासं॒ - [ ] \newline

\textbf{Pada Paata} \newline

त्वम् । नः॒ । अन्त॑मः ॥ उ॒त । त्रा॒ता । शि॒वः । भ॒व॒ । व॒रू॒थ्यः॑ ॥ तम् । त्वा॒ । शो॒चि॒ष्ठ॒ । दी॒दि॒वः॒ ॥ सु॒म्नाय॑ । नू॒नम् । ई॒म॒हे॒ । सखि॑भ्य॒ इति॒ सखि॑ - भ्यः॒ ॥ वसुः॑ । अ॒ग्निः । वसु॑श्रवा॒ इति॒ वसु॑ - श्र॒वाः॒ ॥ अच्छ॑ । न॒क्षि॒ । द्यु॒मत्त॑म॒ इति॑ द्यु॒मत् - त॒मः॒ । र॒यिम् । दाः॒ ॥ ऊ॒र्जा । वः॒ । प॒श्या॒मि॒ । ऊ॒र्जा । मा॒ । प॒श्य॒त॒ । रा॒यः । पोषे॑ण । वः॒ । प॒श्या॒मि॒ । रा॒यः । पोषे॑ण । मा॒ । प॒श्य॒त॒ । इडाः᳚ । स्थ॒ । म॒धु॒कृत॒ इति॑ मधु - कृतः॑ । स्यो॒नाः । मा॒ । एति॑ । वि॒श॒त॒ । इराः᳚ । मदः॑ ॥ स॒ह॒स्र॒पो॒षमिति॑ सहस्र - पो॒षम् । वः॒ । पु॒ष्या॒स॒म् ।  \newline


\textbf{Krama Paata} \newline

त्वम् नः॑ । नो॒ अन्त॑मः । अन्त॑म॒ इत्यन्त॑मः ॥ उ॒त त्रा॒ता । त्रा॒ता शि॒वः । शि॒वो भ॑व । भ॒व॒ व॒रू॒थ्यः॑ । व॒रू॒थ्य॑ इति॑ वरू॒थ्यः॑ ॥ तम् त्वा᳚ । त्वा॒ शो॒चि॒ष्ठ॒ । शो॒चि॒ष्ठ॒ दी॒दि॒वः॒ । दी॒दि॒व॒ इति॑ दीदिवः ॥ सु॒म्नाय॑ नू॒नम् । नू॒नमी॑महे । ई॒म॒हे॒ सखि॑भ्यः । सखि॑भ्य॒ इति॒ सखि॑ - भ्यः॒ ॥ वसु॑र॒ग्निः । अ॒ग्निर् वसु॑श्रवाः । वसु॑श्रवा॒ इति॒ वसु॑ - श्र॒वाः॒ ॥ अच्छा॑ नक्षि । न॒क्षि॒ द्यु॒मत्त॑मः । द्यु॒मत्त॑मो र॒यिम् । द्यु॒मत्त॑म॒ इति॑ द्यु॒मत् - त॒मः॒ । र॒यिम् दाः᳚ । दा॒ इति॑ दाः ॥ ऊ॒र्जा वः॑ । वः॒ प॒श्या॒मि॒ । प॒श्या॒म्यू॒र्जा । ऊ॒र्जा मा᳚ । मा॒ प॒श्य॒त॒ । प॒श्य॒त॒ रा॒यः । रा॒यस्पोषे॑ण । पोषे॑ण वः । वः॒ प॒श्या॒मि॒ । प॒श्या॒मि॒ रा॒यः । रा॒यस्पोषे॑ण । पोषे॑ण मा । मा॒ प॒श्य॒त॒ । प॒श्य॒तेडाः᳚ । इडाः᳚ स्थ । स्थ॒ म॒धु॒कृतः॑ । म॒धु॒कृतः॑ स्यो॒नाः । म॒धु॒कृत॒ इति॑ मधु - कृतः॑ । स्यो॒ना मा᳚ । मा । आ वि॑शत । वि॒श॒तेराः᳚ । इरा॒ मदः॑ । मद॒ इति॒ मदः॑ ॥ स॒ह॒स्र॒पो॒षं ॅवः॑ । स॒ह॒स्र॒पो॒षमिति॑ सहस्र - पो॒षम् । वः॒ पु॒ष्या॒स॒म् । पु॒ष्या॒स॒म् मयि॑ \newline

\textbf{Jatai Paata} \newline

1. त्वन्नो॑ न॒स्त्वम् त्वन्नः॑ । \newline
2. नो॒ अन्त॒मो ऽन्त॑मो नो नो॒ अन्त॑मः । \newline
3. अन्त॑म॒ इत्यन्त॑मः । \newline
4. उ॒त त्रा॒ता त्रा॒तोतोत त्रा॒ता । \newline
5. त्रा॒ता शि॒वः शि॒वस्त्रा॒ता त्रा॒ता शि॒वः । \newline
6. शि॒वो भ॑व भव शि॒वः शि॒वो भ॑व । \newline
7. भ॒व॒ व॒रू॒थ्यो॑ वरू॒थ्यो॑ भव भव वरू॒थ्यः॑ । \newline
8. व॒रू॒थ्य॑ इति॑ वरू॒थ्यः॑ । \newline
9. तम् त्वा᳚ त्वा॒ तम् तम् त्वा᳚ । \newline
10. त्वा॒ शो॒चि॒ष्ठ॒ शो॒चि॒ष्ठ॒ त्वा॒ त्वा॒ शो॒चि॒ष्ठ॒ । \newline
11. शो॒चि॒ष्ठ॒ दी॒दि॒वो॒ दी॒दि॒वः॒ शो॒चि॒ष्ठ॒ शो॒चि॒ष्ठ॒ दी॒दि॒वः॒ । \newline
12. दी॒दि॒व॒ इति॑ दीदिवः । \newline
13. सु॒म्नाय॑ नू॒नन्नू॒नꣳ सु॒म्नाय॑ सु॒म्नाय॑ नू॒नम् । \newline
14. नू॒न मी॑मह ईमहे नू॒नन्नू॒न मी॑महे । \newline
15. ई॒म॒हे॒ सखि॑भ्यः॒ सखि॑भ्य ईमह ईमहे॒ सखि॑भ्यः । \newline
16. सखि॑भ्य॒ इति॒ सखि॑ - भ्यः॒ । \newline
17. वसु॑र॒ग्निर॒ग्निर् वसु॒र् वसु॑र॒ग्निः । \newline
18. अ॒ग्निर् वसु॑श्रवा॒ वसु॑श्रवा अ॒ग्निर॒ग्निर् वसु॑श्रवाः । \newline
19. वसु॑श्रवा॒ इति॒ वसु॑ - श्र॒वाः॒ । \newline
20. अच्छा॑ नक्षि न॒क्ष्यच्छाच्छा॑ नक्षि । \newline
21. न॒क्षि॒ द्यु॒मत्त॑मो द्यु॒मत्त॑मो नक्षि नक्षि द्यु॒मत्त॑मः । \newline
22. द्यु॒मत्त॑मो र॒यिꣳ र॒यिम् द्यु॒मत्त॑मो द्यु॒मत्त॑मो र॒यिम् । \newline
23. द्यु॒मत्त॑म॒ इति॑ द्यु॒मत् - त॒मः॒ । \newline
24. र॒यिम् दा॑ दा र॒यिꣳ र॒यिम् दाः᳚ । \newline
25. दा॒ इति॑ दाः । \newline
26. ऊ॒र्जा वो॑ व ऊ॒र्जोर्जा वः॑ । \newline
27. वः॒ प॒श्या॒मि॒ प॒श्या॒मि॒ वो॒ वः॒ प॒श्या॒मि॒ । \newline
28. प॒श्या॒म्यू॒र्जोर्जा प॑श्यामि पश्याम्यू॒र्जा । \newline
29. ऊ॒र्जा मा॑ मो॒र्जोर्जा मा᳚ । \newline
30. मा॒ प॒श्य॒त॒ प॒श्य॒त॒ मा॒ मा॒ प॒श्य॒त॒ । \newline
31. प॒श्य॒त॒ रा॒यो रा॒यः प॑श्यत पश्यत रा॒यः । \newline
32. रा॒य स्पोषे॑ण॒ पोषे॑ण रा॒यो रा॒य स्पोषे॑ण । \newline
33. पोषे॑ण वो वः॒ पोषे॑ण॒ पोषे॑ण वः । \newline
34. वः॒ प॒श्या॒मि॒ प॒श्या॒मि॒ वो॒ वः॒ प॒श्या॒मि॒ । \newline
35. प॒श्या॒मि॒ रा॒यो रा॒यः प॑श्यामि पश्यामि रा॒यः । \newline
36. रा॒य स्पोषे॑ण॒ पोषे॑ण रा॒यो रा॒य स्पोषे॑ण । \newline
37. पोषे॑ण मा मा॒ पोषे॑ण॒ पोषे॑ण मा । \newline
38. मा॒ प॒श्य॒त॒ प॒श्य॒त॒ मा॒ मा॒ प॒श्य॒त॒ । \newline
39. प॒श्य॒ते डा॒ इडाः᳚ पश्यत पश्य॒ते डाः᳚ । \newline
40. इडाः᳚ स्थ॒ स्थे डा॒ इडाः᳚ स्थ । \newline
41. स्थ॒ म॒धु॒कृतो॑ मधु॒कृतः॑ स्थ स्थ मधु॒कृतः॑ । \newline
42. म॒धु॒कृतः॑ स्यो॒नाः स्यो॒ना म॑धु॒कृतो॑ मधु॒कृतः॑ स्यो॒नाः । \newline
43. म॒धु॒कृत॒ इति॑ मधु - कृतः॑ । \newline
44. स्यो॒ना मा॑ मा स्यो॒नाः स्यो॒ना मा᳚ । \newline
45. मा ऽऽमा॒ मा । \newline
46. आ वि॑शत विश॒ता वि॑शत । \newline
47. वि॒श॒ते रा॒ इरा॑ विशत विश॒ते राः᳚ । \newline
48. इरा॒ मदो॒ मद॒ इरा॒ इरा॒ मदः॑ । \newline
49. मद॒ इति॒ मदः॑ । \newline
50. स॒ह॒स्र॒पो॒षं ॅवो॑ वः सहस्रपो॒षꣳ स॑हस्रपो॒षं ॅवः॑ । \newline
51. स॒ह॒स्र॒पो॒षमिति॑ सहस्र - पो॒षम् । \newline
52. वः॒ पु॒ष्या॒स॒म् पु॒ष्या॒सं॒ ॅवो॒ वः॒ पु॒ष्या॒स॒म् । \newline
53. पु॒ष्या॒स॒म् मयि॒ मयि॑ पुष्यासम् पुष्यास॒म् मयि॑ । \newline

\textbf{Ghana Paata } \newline

1. त्वन्नो॑ न॒स्त्वम् त्वन्नो॒ अन्त॒मो ऽन्त॑मो न॒स्त्वम् त्वन्नो॒ अन्त॑मः । \newline
2. नो॒ अन्त॒मो ऽन्त॑मो नो नो॒ अन्त॑मः । \newline
3. अन्त॑म॒ इत्यन्त॑मः । \newline
4. उ॒त त्रा॒ता त्रा॒तोतोत त्रा॒ता शि॒वः शि॒व स्त्रा॒तोतोत त्रा॒ता शि॒वः । \newline
5. त्रा॒ता शि॒वः शि॒वस्त्रा॒ता त्रा॒ता शि॒वो भ॑व भव शि॒वस्त्रा॒ता त्रा॒ता शि॒वो भ॑व । \newline
6. शि॒वो भ॑व भव शि॒वः शि॒वो भ॑व वरू॒थ्यो॑ वरू॒थ्यो॑ भव शि॒वः शि॒वो भ॑व वरू॒थ्यः॑ । \newline
7. भ॒व॒ व॒रू॒थ्यो॑ वरू॒थ्यो॑ भव भव वरू॒थ्यः॑ । \newline
8. व॒रू॒थ्य॑ इति॑ वरू॒थ्यः॑ । \newline
9. तम् त्वा᳚ त्वा॒ तम् तम् त्वा॑ शोचिष्ठ शोचिष्ठ त्वा॒ तम् तम् त्वा॑ शोचिष्ठ । \newline
10. त्वा॒ शो॒चि॒ष्ठ॒ शो॒चि॒ष्ठ॒ त्वा॒ त्वा॒ शो॒चि॒ष्ठ॒ दी॒दि॒वो॒ दी॒दि॒वः॒ शो॒चि॒ष्ठ॒ त्वा॒ त्वा॒ शो॒चि॒ष्ठ॒ दी॒दि॒वः॒ । \newline
11. शो॒चि॒ष्ठ॒ दी॒दि॒वो॒ दी॒दि॒वः॒ शो॒चि॒ष्ठ॒ शो॒चि॒ष्ठ॒ दी॒दि॒वः॒ । \newline
12. दी॒दि॒व॒ इति॑ दीदिवः । \newline
13. सु॒म्नाय॑ नू॒नन्नू॒नꣳ सु॒म्नाय॑ सु॒म्नाय॑ नू॒न मी॑मह ईमहे नू॒नꣳ सु॒म्नाय॑ सु॒म्नाय॑ नू॒न मी॑महे । \newline
14. नू॒न मी॑मह ईमहे नू॒नन्नू॒न मी॑महे॒ सखि॑भ्यः॒ सखि॑भ्य ईमहे नू॒नन्नू॒न मी॑महे॒ सखि॑भ्यः । \newline
15. ई॒म॒हे॒ सखि॑भ्यः॒ सखि॑भ्य ईमह ईमहे॒ सखि॑भ्यः । \newline
16. सखि॑भ्य॒ इति॒ सखि॑ - भ्यः॒ । \newline
17. वसु॑र॒ग्निर॒ग्निर् वसु॒र् वसु॑र॒ग्निर् वसु॑श्रवा॒ वसु॑श्रवा अ॒ग्निर् वसु॒र् वसु॑र॒ग्निर् वसु॑श्रवाः । \newline
18. अ॒ग्निर् वसु॑श्रवा॒ वसु॑श्रवा अ॒ग्निर॒ग्निर् वसु॑श्रवाः । \newline
19. वसु॑श्रवा॒ इति॒ वसु॑ - श्र॒वाः॒ । \newline
20. अच्छा॑ नक्षि न॒क्ष्यच्छाच्छा॑ नक्षि द्यु॒मत्त॑मो द्यु॒मत्त॑मो न॒क्ष्यच्छाच्छा॑ नक्षि द्यु॒मत्त॑मः । \newline
21. न॒क्षि॒ द्यु॒मत्त॑मो द्यु॒मत्त॑मो नक्षि नक्षि द्यु॒मत्त॑मो र॒यिꣳ र॒यिम् द्यु॒मत्त॑मो नक्षि नक्षि द्यु॒मत्त॑मो र॒यिम् । \newline
22. द्यु॒मत्त॑मो र॒यिꣳ र॒यिम् द्यु॒मत्त॑मो द्यु॒मत्त॑मो र॒यिम् दा॑ दा र॒यिम् द्यु॒मत्त॑मो द्यु॒मत्त॑मो र॒यिम् दाः᳚ । \newline
23. द्यु॒मत्त॑म॒ इति॑ द्यु॒मत् - त॒मः॒ । \newline
24. र॒यिम् दा॑ दा र॒यिꣳ र॒यिम् दाः᳚ । \newline
25. दा॒ इति॑ दाः । \newline
26. ऊ॒र्जा वो॑ व ऊ॒र्जोर्जा वः॑ पश्यामि पश्यामि व ऊ॒र्जोर्जा वः॑ पश्यामि । \newline
27. वः॒ प॒श्या॒मि॒ प॒श्या॒मि॒ वो॒ वः॒ प॒श्या॒म्यू॒र्जोर्जा प॑श्यामि वो वः पश्याम्यू॒र्जा । \newline
28. प॒श्या॒म्यू॒र्जोर्जा प॑श्यामि पश्याम्यू॒र्जा मा॑ मो॒र्जा प॑श्यामि पश्याम्यू॒र्जा मा᳚ । \newline
29. ऊ॒र्जा मा॑ मो॒र्जोर्जा मा॑ पश्यत पश्यत मो॒र्जोर्जा मा॑ पश्यत । \newline
30. मा॒ प॒श्य॒त॒ प॒श्य॒त॒ मा॒ मा॒ प॒श्य॒त॒ रा॒यो रा॒यः प॑श्यत मा मा पश्यत रा॒यः । \newline
31. प॒श्य॒त॒ रा॒यो रा॒यः प॑श्यत पश्यत रा॒य स्पोषे॑ण॒ पोषे॑ण रा॒यः प॑श्यत पश्यत रा॒य स्पोषे॑ण । \newline
32. रा॒य स्पोषे॑ण॒ पोषे॑ण रा॒यो रा॒य स्पोषे॑ण वो वः॒ पोषे॑ण रा॒यो रा॒य स्पोषे॑ण वः । \newline
33. पोषे॑ण वो वः॒ पोषे॑ण॒ पोषे॑ण वः पश्यामि पश्यामि वः॒ पोषे॑ण॒ पोषे॑ण वः पश्यामि । \newline
34. वः॒ प॒श्या॒मि॒ प॒श्या॒मि॒ वो॒ वः॒ प॒श्या॒मि॒ रा॒यो रा॒यः प॑श्यामि वो वः पश्यामि रा॒यः । \newline
35. प॒श्या॒मि॒ रा॒यो रा॒यः प॑श्यामि पश्यामि रा॒य स्पोषे॑ण॒ पोषे॑ण रा॒यः प॑श्यामि पश्यामि रा॒य स्पोषे॑ण । \newline
36. रा॒य स्पोषे॑ण॒ पोषे॑ण रा॒यो रा॒य स्पोषे॑ण मा मा॒ पोषे॑ण रा॒यो रा॒य स्पोषे॑ण मा । \newline
37. पोषे॑ण मा मा॒ पोषे॑ण॒ पोषे॑ण मा पश्यत पश्यत मा॒ पोषे॑ण॒ पोषे॑ण मा पश्यत । \newline
38. मा॒ प॒श्य॒त॒ प॒श्य॒त॒ मा॒ मा॒ प॒श्य॒ते डा॒ इडाः᳚ पश्यत मा मा पश्य॒ते डाः᳚ । \newline
39. प॒श्य॒ते डा॒ इडाः᳚ पश्यत पश्य॒ते डाः᳚ स्थ॒ स्थे डाः᳚ पश्यत पश्य॒ते डाः᳚ स्थ । \newline
40. इडाः᳚ स्थ॒ स्थे डा॒ इडाः᳚ स्थ मधु॒कृतो॑ मधु॒कृतः॒ स्थे डा॒ इडाः᳚ स्थ मधु॒कृतः॑ । \newline
41. स्थ॒ म॒धु॒कृतो॑ मधु॒कृतः॑ स्थ स्थ मधु॒कृतः॑ स्यो॒नाः स्यो॒ना म॑धु॒कृतः॑ स्थ स्थ मधु॒कृतः॑ स्यो॒नाः । \newline
42. म॒धु॒कृतः॑ स्यो॒नाः स्यो॒ना म॑धु॒कृतो॑ मधु॒कृतः॑ स्यो॒ना मा॑ मा स्यो॒ना म॑धु॒कृतो॑ मधु॒कृतः॑ स्यो॒ना मा᳚ । \newline
43. म॒धु॒कृत॒ इति॑ मधु - कृतः॑ । \newline
44. स्यो॒ना मा॑ मा स्यो॒नाः स्यो॒ना मा ऽऽमा᳚ स्यो॒नाः स्यो॒ना मा । \newline
45. मा ऽऽमा॒ मा ऽऽवि॑शत विश॒ता मा॒ मा ऽऽवि॑शत । \newline
46. आ वि॑शत विश॒ता वि॑श॒ते रा॒ इरा॑ विश॒ता वि॑श॒ते राः᳚ । \newline
47. वि॒श॒ते रा॒ इरा॑ विशत विश॒ते रा॒ मदो॒ मद॒ इरा॑ विशत विश॒ते रा॒ मदः॑ । \newline
48. इरा॒ मदो॒ मद॒ इरा॒ इरा॒ मदः॑ । \newline
49. मद॒ इति॒ मदः॑ । \newline
50. स॒ह॒स्र॒पो॒षं ॅवो॑ वः सहस्रपो॒षꣳ स॑हस्रपो॒षं ॅवः॑ पुष्यासम् पुष्यासं ॅवः सहस्रपो॒षꣳ स॑हस्रपो॒षं ॅवः॑ पुष्यासम् । \newline
51. स॒ह॒स्र॒पो॒षमिति॑ सहस्र - पो॒षम् । \newline
52. वः॒ पु॒ष्या॒स॒म् पु॒ष्या॒सं॒ ॅवो॒ वः॒ पु॒ष्या॒स॒म् मयि॒ मयि॑ पुष्यासं ॅवो वः पुष्यास॒म् मयि॑ । \newline
53. पु॒ष्या॒स॒म् मयि॒ मयि॑ पुष्यासम् पुष्यास॒म् मयि॑ वो वो॒ मयि॑ पुष्यासम् पुष्यास॒म् मयि॑ वः । \newline
\pagebreak
\markright{ TS 1.5.6.4  \hfill https://www.vedavms.in \hfill}
\addcontentsline{toc}{section}{ TS 1.5.6.4 }
\section*{ TS 1.5.6.4 }

\textbf{TS 1.5.6.4 } \newline
\textbf{Samhita Paata} \newline

मयि॑ वो॒ रायः॑ श्रयन्तां ॥ तथ्स॑वि॒तुर् वरे᳚ण्यं॒ भर्गो॑ दे॒वस्य॑ धीमहि । धियो॒ योनः॑ प्रचो॒दया᳚त् ॥ सो॒मानꣳ॒॒ स्वर॑णं कृणु॒हि ब्र॑ह्मणस्पते । क॒क्षीव॑न्तं॒ ॅय औ॑शि॒जं ॥ क॒दा च॒न स्त॒रीर॑सि॒ नेन्द्र॑ सश्चसि दा॒शुषे᳚ ॥ उपो॒पेन्नु म॑घव॒न् भुय॒ इन्नु ते॒ दानं॑ दे॒वस्य॑ पृच्यते ॥ परि॑ त्वाऽग्ने॒ पुरं॑ ॅव॒यं ॅविप्रꣳ॑ सहस्य धीमहि ॥ धृ॒षद्व॑र्णं ( ) दि॒वेदि॑वे भे॒त्तारं॑ भङ्गु॒राव॑तः ॥ अग्ने॑ गृहपते सुगृहप॒तिर॒हं त्वया॑ गृ॒हप॑तिना भूयासꣳ सुगृहप॒तिर्मया॒ त्वं गृ॒हप॑तिना भूयाः श॒तꣳ हिमा॒स्तामा॒शिष॒मा शा॑से॒ तन्त॑वे॒ ज्योति॑ष्मतीं॒ तामा॒शिष॒मा शा॑से॒ऽमुष्मै॒ ज्योति॑ष्मतीं ॥ \newline

\textbf{Pada Paata} \newline

मयि॑ । वः॒ । रायः॑ । श्र॒य॒न्ता॒म् ॥ तत् । स॒वि॒तुः । वरे᳚ण्यम् । भर्गः॑ । दे॒वस्य॑ । धी॒म॒हि॒ ॥ धियः॑ । यः । नः॒ । प्र॒चो॒दया॒दिति॑ प्र-चो॒दया᳚त् ॥ सो॒मान᳚म् । स्वर॑णम् । कृ॒णु॒हि । ब्र॒ह्म॒णः॒ । प॒ते॒ ॥ क॒क्षीव॑न्त॒मिति॑ क॒क्षी - व॒न्त॒म् । यः । औ॒शि॒जम् । क॒दा । च॒न । स्त॒रीः । अ॒सि॒ । न । इ॒न्द्र॒ । स॒श्च॒सि॒ । दा॒शुषे᳚ ॥ उपो॒पेत्युप॑ - उ॒प॒ । इत् । नु । म॒घ॒व॒न्निति॑ मघ - व॒न्न् । भूयः॑ । इत् । नु । ते॒ । दान᳚म् । दे॒वस्य॑ । पृ॒च्य॒ते॒ ॥ परीति॑ । त्वा॒ । अ॒ग्ने॒ । पुर᳚म् । व॒यम् । विप्र᳚म् । स॒ह॒स्य॒ । धी॒म॒हि॒ ॥ धृ॒षद्व॑र्ण॒मिति॑ धृ॒षत् - व॒र्ण॒म् ( ) । दि॒वेदि॑व॒ इति॑ दि॒वे - दि॒वे॒ । भे॒त्तार᳚म् । भ॒ङ्गु॒राव॑त॒ इति॑ भङ्गु॒र-व॒तः॒ ॥ अग्ने᳚ । गृ॒ह॒प॒त॒ इति॑ गृह - प॒ते॒ । सु॒गृ॒ह॒प॒तिरिति॑ सु - गृ॒ह॒प॒तिः । अ॒हम् । त्वया᳚ । गृ॒हप॑ति॒नेति॑ गृ॒ह - प॒ति॒ना॒ । भू॒या॒स॒म् । सु॒गृ॒ह॒प॒तिरिति॑ सु - गृ॒ह॒प॒तिः । मया᳚ । त्वम् । गृ॒हप॑ति॒नेति॑ गृ॒ह - प॒ति॒ना॒ । भू॒याः॒ । श॒तम् । हिमाः᳚ । ताम् । आ॒शिष॒मित्या᳚ - शिष᳚म् । एति॑ । शा॒से॒ । तन्त॑वे । ज्योति॑ष्मतीम् । ताम् । आ॒शिष॒मित्या᳚ - शिष᳚म् । एति॑ । शा॒से॒ । अ॒मुष्मै᳚ । ज्योति॑ष्मतीम् ॥  \newline


\textbf{Krama Paata} \newline

मयि॑ वः । वो॒ रायः॑ । रायः॑ श्रयन्ताम् । श्र॒य॒न्ता॒मिति॑ श्रयन्ताम् ॥ तथ् स॑वि॒तुः । स॒वि॒तुर् वरे᳚ण्यम् । वरे᳚ण्य॒म् भर्गः॑ । भर्गो॑ दे॒वस्य॑ । दे॒वस्य॑ धीमहि । धी॒म॒हीति॑ धीमहि ॥ धियो॒ यः । यो नः॑ । नः॒ प्र॒चो॒दया᳚त् । प्र॒चो॒दया॒दिति॑ प्र - चो॒दया᳚त् ॥ सो॒मानꣳ॒॒ स्वर॑णम् । स्वर॑णम् कृणु॒हि । कृ॒णु॒हि ब्र॑ह्मणः । ब्र॒ह्म॒ण॒स्प॒ते॒ । प॒त॒ इति॑ पते ॥ क॒क्षीव॑न्तं॒ ॅयः । क॒क्षीव॑न्त॒मिति॑ क॒क्षी - व॒न्त॒म् । य औ॑शि॒जम् । औ॒शि॒जमित्यौ॑शि॒जम् ॥ क॒दाच॒न । च॒न स्त॒रीः । स्त॒रीर॑सि । अ॒सि॒ न । नेन्द्र॑ । इ॒न्द्र॒ स॒श्च॒सि॒ । स॒श्च॒सि॒ दा॒शुषे᳚ । 
दा॒शुष॒ इति॑ दा॒शुषे᳚ ॥ उपो॒पेत् । उपो॒पेत्युप॑ - उ॒प॒ । इन्नु । नु म॑घवन्न् । म॒घ॒व॒न् भूयः॑ । म॒घ॒व॒न्निति॑ मघ - व॒न्न्॒ । भूय॒ इत् । इन्नु । नु ते᳚ । ते॒ दान᳚म् । दान॑म् दे॒वस्य॑ । दे॒वस्य॑ पृच्यते । पृ॒च्य॒त॒ इति॑ पृच्यते ॥ परि॑ त्वा । त्वा॒ऽग्ने॒ । अ॒ग्ने॒ पुर᳚म् । पुरं॑ ॅव॒यम् । व॒यं ॅविप्र᳚म् । विप्रꣳ॑ सहस्य । स॒ह॒स्य॒ धी॒म॒हि॒ । धी॒म॒हीति॑ धीमहि ॥ धृ॒षद्व॑र्णम् दि॒वेदि॑वे । धृ॒षद्व॑र्ण॒मिति॑ धृ॒षत् - व॒र्ण॒म् ( ) । दि॒वेदि॑वे भे॒त्तार᳚म् । दि॒वेदि॑व॒ इति॑ दि॒वे - दि॒वे॒ । भे॒त्तार॑म् भङ्गु॒राव॑तः । भ॒ङ्गु॒राव॑त॒ इति॑ भङ्गु॒र - व॒तः॒ ॥ अग्ने॑ गृहपते । गृ॒ह॒प॒ते॒ सु॒गृ॒ह॒प॒तिः । गृ॒ह॒प॒त॒ इति॑ गृह - प॒ते॒ । सु॒गृ॒ह॒प॒तिर॒हम् । सु॒गृ॒ह॒प॒तिरिति॑ सु - गृ॒ह॒प॒तिः । अ॒हम् त्वया᳚ । त्वया॑ गृ॒हप॑तिना । गृ॒हप॑तिना भूयासम् । गृ॒हप॑ति॒नेति॑ गृ॒ह - प॒ति॒ना॒ । भू॒या॒सꣳ॒॒ सु॒गृ॒ह॒प॒तिः । सु॒गृ॒ह॒प॒तिर् मया᳚ । सु॒गृ॒ह॒प॒तिरिति॑ सु - गृ॒ह॒प॒तिः । मया॒ त्वम् । त्वम् गृ॒हप॑तिना । गृ॒हप॑तिना भूयाः । गृ॒हप॑ति॒नेति॑ गृ॒ह - प॒ति॒ना॒ । भू॒याः॒ श॒तम् । श॒तꣳ हिमाः᳚ । हिमा॒स्ताम् । तामा॒शिष᳚म् । आ॒शिष॒मा । आ॒शिष॒मित्या᳚ - शिष᳚म् । आ शा॑से । शा॒से॒ तन्त॑वे । तन्त॑वे॒ ज्योति॑ष्मतीम् । ज्योति॑ष्मती॒म् ताम् । तामा॒शिष᳚म् । आ॒शिष॒ मा । आ॒शिष॒मित्या᳚ - शिष᳚म् । आ शा॑से । शा॒से॒ऽमुष्मै᳚ । अ॒मुष्मै॒ ज्योति॑ष्मतीम् । 
ज्योति॑ष्मती॒मिति॒ ज्योति॑ष्मतीम् । \newline

\textbf{Jatai Paata} \newline

1. मयि॑ वो वो॒ मयि॒ मयि॑ वः । \newline
2. वो॒ रायो॒ रायो॑ वो वो॒ रायः॑ । \newline
3. रायः॑ श्रयन्ताꣳ श्रयन्ता॒(ग्म्॒) रायो॒ रायः॑ श्रयन्ताम् । \newline
4. श्र॒य॒न्ता॒मिति॑ श्रयन्ताम् । \newline
5. तथ् स॑वि॒तुः स॑वि॒तुस्तत् तथ् स॑वि॒तुः । \newline
6. स॒वि॒तुर् वरे᳚ण्यं॒ ॅवरे᳚ण्यꣳ सवि॒तुः स॑वि॒तुर् वरे᳚ण्यम् । \newline
7. वरे᳚ण्य॒म् भर्गो॒ भर्गो॒ वरे᳚ण्यं॒ ॅवरे᳚ण्य॒म् भर्गः॑ । \newline
8. भर्गो॑ दे॒वस्य॑ दे॒वस्य॒ भर्गो॒ भर्गो॑ दे॒वस्य॑ । \newline
9. दे॒वस्य॑ धीमहि धीमहि दे॒वस्य॑ दे॒वस्य॑ धीमहि । \newline
10. धी॒म॒हीति॑ धीमहि । \newline
11. धियो॒ यो यो धियो॒ धियो॒ यः । \newline
12. यो नो॑ नो॒ यो यो नः॑ । \newline
13. नः॒ प्र॒चो॒दया᳚त् प्रचो॒दया᳚न् नो नः प्रचो॒दया᳚त् । \newline
14. प्र॒चो॒दया॒दिति॑ प्र - चो॒दया᳚त् । \newline
15. सो॒मान॒(ग्ग्॒) स्वर॑ण॒(ग्ग्॒) स्वर॑णꣳ सो॒मान(ग्म्॑) सो॒मान॒(ग्ग्॒) स्वर॑णम् । \newline
16. स्वर॑णम् कृणु॒हि कृ॑णु॒हि स्वर॑ण॒(ग्ग्॒) स्वर॑णम् कृणु॒हि । \newline
17. कृ॒णु॒हि ब्र॑ह्मणो ब्रह्मणः कृणु॒हि कृ॑णु॒हि ब्र॑ह्मणः । \newline
18. ब्र॒ह्म॒ण॒ स्प॒ते॒ प॒ते॒ ब्र॒ह्म॒णो॒ ब्र॒ह्म॒ण॒ स्प॒ते॒ । \newline
19. प॒त॒ इति॑ पते । \newline
20. क॒क्षीव॑न्तं॒ ॅयो यः क॒क्षीव॑न्तम् क॒क्षीव॑न्तं॒ ॅयः । \newline
21. क॒क्षीव॑न्त॒मिति॑ क॒क्षी - व॒न्त॒म् । \newline
22. य औ॑शि॒ज मौ॑शि॒जं ॅयो य औ॑शि॒जम् । \newline
23. औ॒शि॒जमित्यौ॑शि॒जम् । \newline
24. क॒दा च॒न च॒न क॒दा क॒दा च॒न । \newline
25. च॒न स्त॒रीः स्त॒रीश्च॒न च॒न स्त॒रीः । \newline
26. स्त॒रीर॑स्यसि स्त॒रीः स्त॒रीर॑सि । \newline
27. अ॒सि॒ न नास्य॑सि॒ न । \newline
28. ने न्द्रे᳚ न्द्र॒ न ने न्द्र॑ । \newline
29. इ॒न्द्र॒ स॒श्च॒सि॒ स॒श्च॒सी॒न्द्रे॒ न्द्र॒ स॒श्च॒सि॒ । \newline
30. स॒श्च॒सि॒ दा॒शुषे॑ दा॒शुषे॑ सश्चसि सश्चसि दा॒शुषे᳚ । \newline
31. दा॒शुष॒ इति॑ दा॒शुषे᳚ । \newline
32. उपो॒पे दिदुपो॒पोपो॒पे त् । \newline
33. उपो॒पेत्युप॑ - उ॒प॒ । \newline
34. इन् नु न्विदिन् नु । \newline
35. नु म॑घवन् मघव॒न् नु नु म॑घवन्न् । \newline
36. म॒घ॒व॒न् भूयो॒ भूयो॑ मघवन् मघव॒न् भूयः॑ । \newline
37. म॒घ॒व॒न्निति॑ मघ - व॒न्न् । \newline
38. भूय॒ इदिद् भूयो॒ भूय॒ इत् । \newline
39. इन् नु न्विदिन् नु । \newline
40. नु ते॑ ते॒ नु नु ते᳚ । \newline
41. ते॒ दान॒म् दान॑म् ते ते॒ दान᳚म् । \newline
42. दान॑म् दे॒वस्य॑ दे॒वस्य॒ दान॒म् दान॑म् दे॒वस्य॑ । \newline
43. दे॒वस्य॑ पृच्यते पृच्यते दे॒वस्य॑ दे॒वस्य॑ पृच्यते । \newline
44. पृ॒च्य॒त॒ इति॑ पृच्यते । \newline
45. परि॑ त्वा त्वा॒ परि॒ परि॑ त्वा । \newline
46. त्वा॒ ऽग्ने॒ ऽग्ने॒ त्वा॒ त्वा॒ ऽग्ने॒ । \newline
47. अ॒ग्ने॒ पुर॒म् पुर॑ मग्ने ऽग्ने॒ पुर᳚म् । \newline
48. पुरं॑ ॅव॒यं ॅव॒यम् पुर॒म् पुरं॑ ॅव॒यम् । \newline
49. व॒यं ॅविप्रं॒ ॅविप्रं॑ ॅव॒यं ॅव॒यं ॅविप्र᳚म् । \newline
50. विप्र(ग्म्॑) सहस्य सहस्य॒ विप्रं॒ ॅविप्र(ग्म्॑) सहस्य । \newline
51. स॒ह॒स्य॒ धी॒म॒हि॒ धी॒म॒हि॒ स॒ह॒स्य॒ स॒ह॒स्य॒ धी॒म॒हि॒ । \newline
52. धी॒म॒हीति॑ धीमहि । \newline
53. धृ॒षद्व॑र्णम् दि॒वेदि॑वे दि॒वेदि॑वे धृ॒षद्व॑र्णम् धृ॒षद्व॑र्णम् दि॒वेदि॑वे । \newline
54. धृ॒षद्व॑र्ण॒मिति॑ धृ॒षत् - व॒र्ण॒म् । \newline
55. दि॒वेदि॑वे भे॒त्तार॑म् भे॒त्तार॑म् दि॒वेदि॑वे दि॒वेदि॑वे भे॒त्तार᳚म् । \newline
56. दि॒वेदि॑व॒ इति॑ दि॒वे - दि॒वे॒ । \newline
57. भे॒त्तार॑म् भङ्गु॒राव॑तो भङ्गु॒राव॑तो भे॒त्तार॑म् भे॒त्तार॑म् भङ्गु॒राव॑तः । \newline
58. भ॒ङ्गु॒राव॑त॒ इति॑ भङ्गु॒र - व॒तः॒ । \newline
59. अग्ने॑ गृहपते गृहप॒ते ऽग्ने ऽग्ने॑ गृहपते । \newline
60. गृ॒ह॒प॒ते॒ सु॒गृ॒ह॒प॒तिः सु॑गृहप॒तिर् गृ॑हपते गृहपते सुगृहप॒तिः । \newline
61. गृ॒ह॒प॒त॒ इति॑ गृह - प॒ते॒ । \newline
62. सु॒गृ॒ह॒प॒तिर॒ह म॒हꣳ सु॑गृहप॒तिः सु॑गृहप॒तिर॒हम् । \newline
63. सु॒गृ॒ह॒प॒तिरिति॑ सु - गृ॒ह॒प॒तिः । \newline
64. अ॒हम् त्वया॒ त्वया॒ ऽह म॒हम् त्वया᳚ । \newline
65. त्वया॑ गृ॒हप॑तिना गृ॒हप॑तिना॒ त्वया॒ त्वया॑ गृ॒हप॑तिना । \newline
66. गृ॒हप॑तिना भूयासम् भूयासम् गृ॒हप॑तिना गृ॒हप॑तिना भूयासम् । \newline
67. गृ॒हप॑ति॒नेति॑ गृ॒ह - प॒ति॒ना॒ । \newline
68. भू॒या॒स॒(ग्म्॒) सु॒गृ॒ह॒प॒तिः सु॑गृहप॒तिर् भू॑यासम् भूयासꣳ सुगृहप॒तिः । \newline
69. सु॒गृ॒ह॒प॒तिर् मया॒ मया॑ सुगृहप॒तिः सु॑गृहप॒तिर् मया᳚ । \newline
70. सु॒गृ॒ह॒प॒तिरिति॑ सु - गृ॒ह॒प॒तिः । \newline
71. मया॒ त्वम् त्वम् मया॒ मया॒ त्वम् । \newline
72. त्वम् गृ॒हप॑तिना गृ॒हप॑तिना॒ त्वम् त्वम् गृ॒हप॑तिना । \newline
73. गृ॒हप॑तिना भूया भूया गृ॒हप॑तिना गृ॒हप॑तिना भूयाः । \newline
74. गृ॒हप॑ति॒नेति॑ गृ॒ह - प॒ति॒ना॒ । \newline
75. भू॒याः॒ श॒तꣳ श॒तम् भू॑या भूयाः श॒तम् । \newline
76. श॒तꣳ हिमा॒ हिमाः᳚ श॒तꣳ श॒तꣳ हिमाः᳚ । \newline
77. हिमा॒स्ताम् ताꣳ हिमा॒ हिमा॒स्ताम् । \newline
78. ता मा॒शिष॑ मा॒शिष॒म् ताम् ता मा॒शिष᳚म् । \newline
79. आ॒शिष॒ मा ऽऽशिष॑ मा॒शिष॒ मा । \newline
80. आ॒शिष॒मित्या᳚ - शिष᳚म् । \newline
81. आ शा॑से शास॒ आ शा॑से । \newline
82. शा॒से॒ तन्त॑वे॒ तन्त॑वे शासे शासे॒ तन्त॑वे । \newline
83. तन्त॑वे॒ ज्योति॑ष्मती॒म् ज्योति॑ष्मती॒म् तन्त॑वे॒ तन्त॑वे॒ ज्योति॑ष्मतीम् । \newline
84. ज्योति॑ष्मती॒म् ताम् ताम् ज्योति॑ष्मती॒म् ज्योति॑ष्मती॒म् ताम् । \newline
85. ता मा॒शिष॑ मा॒शिष॒म् ताम् ता मा॒शिष᳚म् । \newline
86. आ॒शिष॒ मा ऽऽशिष॑ मा॒शिष॒ मा । \newline
87. आ॒शिष॒मित्या᳚ - शिष᳚म् । \newline
88. आ शा॑से शास॒ आ शा॑से । \newline
89. शा॒से॒ ऽमुष्मा॑ अ॒मुष्मै॑ शासे शासे॒ ऽमुष्मै᳚ । \newline
90. अ॒मुष्मै॒ ज्योति॑ष्मती॒म् ज्योति॑ष्मती म॒मुष्मा॑ अ॒मुष्मै॒ ज्योति॑ष्मतीम् । \newline
91. ज्योति॑ष्मती॒मिति॒ ज्योति॑ष्मतीम् । \newline

\textbf{Ghana Paata } \newline

1. मयि॑ वो वो॒ मयि॒ मयि॑ वो॒ रायो॒ रायो॑ वो॒ मयि॒ मयि॑ वो॒ रायः॑ । \newline
2. वो॒ रायो॒ रायो॑ वो वो॒ रायः॑ श्रयन्ताꣳ श्रयन्ता॒(ग्म्॒) रायो॑ वो वो॒ रायः॑ श्रयन्ताम् । \newline
3. रायः॑ श्रयन्ताꣳ श्रयन्ता॒(ग्म्॒) रायो॒ रायः॑ श्रयन्ताम् । \newline
4. श्र॒य॒न्ता॒मिति॑ श्रयन्ताम् । \newline
5. तथ् स॑वि॒तुः स॑वि॒तुस्तत् तथ् स॑वि॒तुर् वरे᳚ण्यं॒ ॅवरे᳚ण्यꣳ सवि॒तुस्तत् तथ् स॑वि॒तुर् वरे᳚ण्यम् । \newline
6. स॒वि॒तुर् वरे᳚ण्यं॒ ॅवरे᳚ण्यꣳ सवि॒तुः स॑वि॒तुर् वरे᳚ण्य॒म् भर्गो॒ भर्गो॒ वरे᳚ण्यꣳ सवि॒तुः स॑वि॒तुर् वरे᳚ण्य॒म् भर्गः॑ । \newline
7. वरे᳚ण्य॒म् भर्गो॒ भर्गो॒ वरे᳚ण्यं॒ ॅवरे᳚ण्य॒म् भर्गो॑ दे॒वस्य॑ दे॒वस्य॒ भर्गो॒ वरे᳚ण्यं॒ ॅवरे᳚ण्य॒म् भर्गो॑ दे॒वस्य॑ । \newline
8. भर्गो॑ दे॒वस्य॑ दे॒वस्य॒ भर्गो॒ भर्गो॑ दे॒वस्य॑ धीमहि धीमहि दे॒वस्य॒ भर्गो॒ भर्गो॑ दे॒वस्य॑ धीमहि । \newline
9. दे॒वस्य॑ धीमहि धीमहि दे॒वस्य॑ दे॒वस्य॑ धीमहि । \newline
10. धी॒म॒हीति॑ धीमहि । \newline
11. धियो॒ यो यो धियो॒ धियो॒ यो नो॑ नो॒ यो धियो॒ धियो॒ यो नः॑ । \newline
12. यो नो॑ नो॒ यो यो नः॑ प्रचो॒दया᳚त् प्रचो॒दया᳚न् नो॒ यो यो नः॑ प्रचो॒दया᳚त् । \newline
13. नः॒ प्र॒चो॒दया᳚त् प्रचो॒दया᳚न् नो नः प्रचो॒दया᳚त् । \newline
14. प्र॒चो॒दया॒दिति॑ प्र - चो॒दया᳚त् । \newline
15. सो॒मान॒(ग्ग्॒) स्वर॑ण॒(ग्ग्॒) स्वर॑णꣳ सो॒मान(ग्म्॑) सो॒मान॒(ग्ग्॒) स्वर॑णम् कृणु॒हि कृ॑णु॒हि स्वर॑णꣳ सो॒मान(ग्म्॑) सो॒मान॒(ग्ग्॒) स्वर॑णम् कृणु॒हि । \newline
16. स्वर॑णम् कृणु॒हि कृ॑णु॒हि स्वर॑ण॒(ग्ग्॒) स्वर॑णम् कृणु॒हि ब्र॑ह्मणो ब्रह्मणः कृणु॒हि स्वर॑ण॒(ग्ग्॒) स्वर॑णम् कृणु॒हि ब्र॑ह्मणः । \newline
17. कृ॒णु॒हि ब्र॑ह्मणो ब्रह्मणः कृणु॒हि कृ॑णु॒हि ब्र॑ह्मण स्पते पते ब्रह्मणः कृणु॒हि कृ॑णु॒हि ब्र॑ह्मण स्पते । \newline
18. ब्र॒ह्म॒ण॒ स्प॒ते॒ प॒ते॒ ब्र॒ह्म॒णो॒ ब्र॒ह्म॒ण॒ स्प॒ते॒ । \newline
19. प॒त॒ इति॑ पते । \newline
20. क॒क्षीव॑न्तं॒ ॅयो यः क॒क्षीव॑न्तम् क॒क्षीव॑न्तं॒ ॅय औ॑शि॒ज मौ॑शि॒जं ॅयः क॒क्षीव॑न्तम् क॒क्षीव॑न्तं॒ ॅय औ॑शि॒जम् । \newline
21. क॒क्षीव॑न्त॒मिति॑ क॒क्षी - व॒न्त॒म् । \newline
22. य औ॑शि॒ज मौ॑शि॒जं ॅयो य औ॑शि॒जम् । \newline
23. औ॒शि॒जमित्यौ॑शि॒जम् । \newline
24. क॒दा च॒न च॒न क॒दा क॒दा च॒न स्त॒रीः स्त॒रीश्च॒न क॒दा क॒दा च॒न स्त॒रीः । \newline
25. च॒न स्त॒रीः स्त॒रीश्च॒न च॒न स्त॒रीर॑स्यसि स्त॒रीश्च॒न च॒न स्त॒रीर॑सि । \newline
26. स्त॒रीर॑स्यसि स्त॒रीः स्त॒रीर॑सि॒ न नासि॑ स्त॒रीः स्त॒रीर॑सि॒ न । \newline
27. अ॒सि॒ न नास्य॑सि॒ ने न्द्रे᳚ न्द्र॒ नास्य॑सि॒ ने न्द्र॑ । \newline
28. ने न्द्रे᳚ न्द्र॒ न ने न्द्र॑ सश्चसि सश्चसीन्द्र॒ न ने न्द्र॑ सश्चसि । \newline
29. इ॒न्द्र॒ स॒श्च॒सि॒ स॒श्च॒सी॒न्द्रे॒ न्द्र॒ स॒श्च॒सि॒ दा॒शुषे॑ दा॒शुषे॑ सश्चसीन्द्रे न्द्र सश्चसि दा॒शुषे᳚ । \newline
30. स॒श्च॒सि॒ दा॒शुषे॑ दा॒शुषे॑ सश्चसि सश्चसि दा॒शुषे᳚ । \newline
31. दा॒शुष॒ इति॑ दा॒शुषे᳚ । \newline
32. उपो॒पे दिदुपो॒पोपो॒पेन् नु न्विदुपो॒पोपो॒पेन् नु । \newline
33. उपो॒पेत्युप॑ - उ॒प॒ । \newline
34. इन् नु न्विदिन् नु म॑घवन् मघव॒न् न्विदिन् नु म॑घवन्न् । \newline
35. नु म॑घवन् मघव॒न् नु नु म॑घव॒न् भूयो॒ भूयो॑ मघव॒न् नु नु म॑घव॒न् भूयः॑ । \newline
36. म॒घ॒व॒न् भूयो॒ भूयो॑ मघवन् मघव॒न् भूय॒ इदिद् भूयो॑ मघवन् मघव॒न् भूय॒ इत् । \newline
37. म॒घ॒व॒न्निति॑ मघ - व॒न्न् । \newline
38. भूय॒ इदिद् भूयो॒ भूय॒ इन् नु न्विद् भूयो॒ भूय॒ इन् नु । \newline
39. इन् नु न्विदिन् नु ते॑ ते॒ न्विदिन् नु ते᳚ । \newline
40. नु ते॑ ते॒ नु नु ते॒ दान॒म् दान॑म् ते॒ नु नु ते॒ दान᳚म् । \newline
41. ते॒ दान॒म् दान॑म् ते ते॒ दान॑म् दे॒वस्य॑ दे॒वस्य॒ दान॑म् ते ते॒ दान॑म् दे॒वस्य॑ । \newline
42. दान॑म् दे॒वस्य॑ दे॒वस्य॒ दान॒म् दान॑म् दे॒वस्य॑ पृच्यते पृच्यते दे॒वस्य॒ दान॒म् दान॑म् दे॒वस्य॑ पृच्यते । \newline
43. दे॒वस्य॑ पृच्यते पृच्यते दे॒वस्य॑ दे॒वस्य॑ पृच्यते । \newline
44. पृ॒च्य॒त॒ इति॑ पृच्यते । \newline
45. परि॑ त्वा त्वा॒ परि॒ परि॑ त्वा ऽग्ने ऽग्ने त्वा॒ परि॒ परि॑ त्वा ऽग्ने । \newline
46. त्वा॒ ऽग्ने॒ ऽग्ने॒ त्वा॒ त्वा॒ ऽग्ने॒ पुर॒म् पुर॑ मग्ने त्वा त्वा ऽग्ने॒ पुर᳚म् । \newline
47. अ॒ग्ने॒ पुर॒म् पुर॑ मग्ने ऽग्ने॒ पुरं॑ ॅव॒यं ॅव॒यम् पुर॑ मग्ने ऽग्ने॒ पुरं॑ ॅव॒यम् । \newline
48. पुरं॑ ॅव॒यं ॅव॒यम् पुर॒म् पुरं॑ ॅव॒यं ॅविप्रं॒ ॅविप्रं॑ ॅव॒यम् पुर॒म् पुरं॑ ॅव॒यं ॅविप्र᳚म् । \newline
49. व॒यं ॅविप्रं॒ ॅविप्रं॑ ॅव॒यं ॅव॒यं ॅविप्र(ग्म्॑) सहस्य सहस्य॒ विप्रं॑ ॅव॒यं ॅव॒यं ॅविप्र(ग्म्॑) सहस्य । \newline
50. विप्र(ग्म्॑) सहस्य सहस्य॒ विप्रं॒ ॅविप्र(ग्म्॑) सहस्य धीमहि धीमहि सहस्य॒ विप्रं॒ ॅविप्र(ग्म्॑) सहस्य धीमहि । \newline
51. स॒ह॒स्य॒ धी॒म॒हि॒ धी॒म॒हि॒ स॒ह॒स्य॒ स॒ह॒स्य॒ धी॒म॒हि॒ । \newline
52. धी॒म॒हीति॑ धीमहि । \newline
53. धृ॒षद्व॑र्णम् दि॒वेदि॑वे दि॒वेदि॑वे धृ॒षद्व॑र्णम् धृ॒षद्व॑र्णम् दि॒वेदि॑वे भे॒त्तार॑म् भे॒त्तार॑म् दि॒वेदि॑वे धृ॒षद्व॑र्णम् धृ॒षद्व॑र्णम् दि॒वेदि॑वे भे॒त्तार᳚म् । \newline
54. धृ॒षद्व॑र्ण॒मिति॑ धृ॒षत् - व॒र्ण॒म् । \newline
55. दि॒वेदि॑वे भे॒त्तार॑म् भे॒त्तार॑म् दि॒वेदि॑वे दि॒वेदि॑वे भे॒त्तार॑म् भङ्गु॒राव॑तो भङ्गु॒राव॑तो भे॒त्तार॑म् दि॒वेदि॑वे दि॒वेदि॑वे भे॒त्तार॑म् भङ्गु॒राव॑तः । \newline
56. दि॒वेदि॑व॒ इति॑ दि॒वे - दि॒वे॒ । \newline
57. भे॒त्तार॑म् भङ्गु॒राव॑तो भङ्गु॒राव॑तो भे॒त्तार॑म् भे॒त्तार॑म् भङ्गु॒राव॑तः । \newline
58. भ॒ङ्गु॒राव॑त॒ इति॑ भङ्गु॒र - व॒तः॒ । \newline
59. अग्ने॑ गृहपते गृहप॒ते ऽग्ने ऽग्ने॑ गृहपते सुगृहप॒तिः सु॑गृहप॒तिर् गृ॑हप॒ते ऽग्ने ऽग्ने॑ गृहपते सुगृहप॒तिः । \newline
60. गृ॒ह॒प॒ते॒ सु॒गृ॒ह॒प॒तिः सु॑गृहप॒तिर् गृ॑हपते गृहपते सुगृहप॒तिर॒ह म॒हꣳ सु॑गृहप॒तिर् गृ॑हपते गृहपते सुगृहप॒ति र॒हम् । \newline
61. गृ॒ह॒प॒त॒ इति॑ गृह - प॒ते॒ । \newline
62. सु॒गृ॒ह॒प॒तिर॒ह म॒हꣳ सु॑गृहप॒तिः सु॑गृहप॒तिर॒हम् त्वया॒ त्वया॒ ऽहꣳ सु॑गृहप॒तिः सु॑गृहप॒तिर॒हम् त्वया᳚ । \newline
63. सु॒गृ॒ह॒प॒तिरिति॑ सु - गृ॒ह॒प॒तिः । \newline
64. अ॒हम् त्वया॒ त्वया॒ ऽह म॒हम् त्वया॑ गृ॒हप॑तिना गृ॒हप॑तिना॒ त्वया॒ ऽह म॒हम् त्वया॑ गृ॒हप॑तिना । \newline
65. त्वया॑ गृ॒हप॑तिना गृ॒हप॑तिना॒ त्वया॒ त्वया॑ गृ॒हप॑तिना भूयासम् भूयासम् गृ॒हप॑तिना॒ त्वया॒ त्वया॑ गृ॒हप॑तिना भूयासम् । \newline
66. गृ॒हप॑तिना भूयासम् भूयासम् गृ॒हप॑तिना गृ॒हप॑तिना भूयासꣳ सुगृहप॒तिः सु॑गृहप॒तिर् भू॑यासम् गृ॒हप॑तिना गृ॒हप॑तिना भूयासꣳ सुगृहप॒तिः । \newline
67. गृ॒हप॑ति॒नेति॑ गृ॒ह - प॒ति॒ना॒ । \newline
68. भू॒या॒स॒(ग्म्॒) सु॒गृ॒ह॒प॒तिः सु॑गृहप॒तिर् भू॑यासम् भूयासꣳ सुगृहप॒तिर् मया॒ मया॑ सुगृहप॒तिर् भू॑यासम् भूयासꣳ सुगृहप॒तिर् मया᳚ । \newline
69. सु॒गृ॒ह॒प॒तिर् मया॒ मया॑ सुगृहप॒तिः सु॑गृहप॒तिर् मया॒ त्वम् त्वम् मया॑ सुगृहप॒तिः सु॑गृहप॒तिर् मया॒ त्वम् । \newline
70. सु॒गृ॒ह॒प॒तिरिति॑ सु - गृ॒ह॒प॒तिः । \newline
71. मया॒ त्वम् त्वम् मया॒ मया॒ त्वम् गृ॒हप॑तिना गृ॒हप॑तिना॒ त्वम् मया॒ मया॒ त्वम् गृ॒हप॑तिना । \newline
72. त्वम् गृ॒हप॑तिना गृ॒हप॑तिना॒ त्वम् त्वम् गृ॒हप॑तिना भूया भूया गृ॒हप॑तिना॒ त्वम् त्वम् गृ॒हप॑तिना भूयाः । \newline
73. गृ॒हप॑तिना भूया भूया गृ॒हप॑तिना गृ॒हप॑तिना भूयाः श॒तꣳ श॒तम् भू॑या गृ॒हप॑तिना गृ॒हप॑तिना भूयाः श॒तम् । \newline
74. गृ॒हप॑ति॒नेति॑ गृ॒ह - प॒ति॒ना॒ । \newline
75. भू॒याः॒ श॒तꣳ श॒तम् भू॑या भूयाः श॒तꣳ हिमा॒ हिमाः᳚ श॒तम् भू॑या भूयाः श॒तꣳ हिमाः᳚ । \newline
76. श॒तꣳ हिमा॒ हिमाः᳚ श॒तꣳ श॒तꣳ हिमा॒स्ताम् ताꣳ हिमाः᳚ श॒तꣳ श॒तꣳ हिमा॒स्ताम् । \newline
77. हिमा॒स्ताम् ताꣳ हिमा॒ हिमा॒स्ता मा॒शिष॑ मा॒शिष॒म् ताꣳ हिमा॒ हिमा॒स्ता मा॒शिष᳚म् । \newline
78. ता मा॒शिष॑ मा॒शिष॒म् ताम् ता मा॒शिष॒ मा ऽऽशिष॒म् ताम् ता मा॒शिष॒ मा । \newline
79. आ॒शिष॒ मा ऽऽशिष॑ मा॒शिष॒ मा शा॑से शास॒ आ ऽऽशिष॑ मा॒शिष॒ मा शा॑से । \newline
80. आ॒शिष॒मित्या᳚ - शिष᳚म् । \newline
81. आ शा॑से शास॒ आ शा॑से॒ तन्त॑वे॒ तन्त॑वे शास॒ आ शा॑से॒ तन्त॑वे । \newline
82. शा॒से॒ तन्त॑वे॒ तन्त॑वे शासे शासे॒ तन्त॑वे॒ ज्योति॑ष्मती॒म् ज्योति॑ष्मती॒म् तन्त॑वे शासे शासे॒ तन्त॑वे॒ ज्योति॑ष्मतीम् । \newline
83. तन्त॑वे॒ ज्योति॑ष्मती॒म् ज्योति॑ष्मती॒म् तन्त॑वे॒ तन्त॑वे॒ ज्योति॑ष्मती॒म् ताम् ताम् ज्योति॑ष्मती॒म् तन्त॑वे॒ तन्त॑वे॒ ज्योति॑ष्मती॒म् ताम् । \newline
84. ज्योति॑ष्मती॒म् ताम् ताम् ज्योति॑ष्मती॒म् ज्योति॑ष्मती॒म् ता मा॒शिष॑ मा॒शिष॒म् ताम् ज्योति॑ष्मती॒म् ज्योति॑ष्मती॒म् ता मा॒शिष᳚म् । \newline
85. ता मा॒शिष॑ मा॒शिष॒म् ताम् ता मा॒शिष॒ मा ऽऽशिष॒म् ताम् ता मा॒शिष॒ मा । \newline
86. आ॒शिष॒ मा ऽऽशिष॑ मा॒शिष॒ मा शा॑से शास॒ आ ऽऽशिष॑ मा॒शिष॒ मा शा॑से । \newline
87. आ॒शिष॒मित्या᳚ - शिष᳚म् । \newline
88. आ शा॑से शास॒ आ शा॑से॒ ऽमुष्मा॑ अ॒मुष्मै॑ शास॒ आ शा॑से॒ ऽमुष्मै᳚ । \newline
89. शा॒से॒ ऽमुष्मा॑ अ॒मुष्मै॑ शासे शासे॒ ऽमुष्मै॒ ज्योति॑ष्मती॒म् ज्योति॑ष्मती म॒मुष्मै॑ शासे शासे॒ ऽमुष्मै॒ ज्योति॑ष्मतीम् । \newline
90. अ॒मुष्मै॒ ज्योति॑ष्मती॒म् ज्योति॑ष्मती म॒मुष्मा॑ अ॒मुष्मै॒ ज्योति॑ष्मतीम् । \newline
91. ज्योति॑ष्मती॒मिति॒ ज्योति॑ष्मतीम् । \newline
\pagebreak
\markright{ TS 1.5.7.1  \hfill https://www.vedavms.in \hfill}
\addcontentsline{toc}{section}{ TS 1.5.7.1 }
\section*{ TS 1.5.7.1 }

\textbf{TS 1.5.7.1 } \newline
\textbf{Samhita Paata} \newline

अय॑ज्ञो॒ वा ए॒ष यो॑ऽसा॒मोप॑प्र॒यन्तो॑ अद्ध्व॒रमित्या॑ह॒ स्तोम॑मे॒वास्मै॑ युन॒क्त्युपेत्या॑ह प्र॒जा वै प॒शव॒ उपे॒मं ॅलो॒कं प्र॒जामे॒व प॒शूनि॒मं ॅलो॒कमुपै᳚त्य॒स्य प्र॒त्नामनु॒ द्युत॒मित्या॑ह सुव॒र्गो वै लो॒कः प्र॒त्नः सु॑व॒र्गमे॒व लो॒कꣳ स॒मारो॑हत्य॒ग्निर् मू॒र्द्धा दि॒वः क॒कुदित्या॑ह मू॒र्द्धान॑ - [ ] \newline

\textbf{Pada Paata} \newline

अय॑ज्ञ्ः । वै । ए॒षः । यः । अ॒सा॒मा । उ॒प॒प्र॒यन्त॒ इत्यु॑प - प्र॒यन्तः॑ । अ॒द्ध्व॒रम् । इति॑ । आ॒ह॒ । स्तोम᳚म् । ए॒व । अ॒स्मै॒ । यु॒न॒क्ति॒ । उपेति॑ । इति॑ । आ॒ह॒ । प्र॒जेति॑ प्र - जा । वै । प॒शवः॑ । उपेति॑ । इ॒मम् । लो॒कम् । प्र॒जामिति॑ प्र - जाम् । ए॒व । प॒शून् । इ॒मम् । लो॒कम् । उपेति॑ । ए॒ति॒ । अ॒स्य । प्र॒त्नाम् । अन्विति॑ । द्युत᳚म् । इति॑ । आ॒ह॒ । सु॒व॒र्ग इति॑ सुवः - गः । वै । लो॒कः । प्र॒त्नः । सु॒व॒र्गमिति॑ सुवः - गम् । ए॒व । लो॒कम् । स॒मारो॑ह॒तीति॑ सं - आरो॑हति । अ॒ग्निः । मू॒र्धा । दि॒वः । क॒कुत् । इति॑ । आ॒ह॒ । मू॒र्धान᳚म् ।  \newline


\textbf{Krama Paata} \newline

अय॑ज्ञो॒ वै । वा ए॒षः । ए॒ष यः । यो॑ऽसा॒मा । अ॒सा॒मोप॑प्र॒यन्तः॑ । उ॒प॒प्र॒यन्तो॑ अद्ध्व॒रम् । उ॒प॒प्र॒यन्त॒ इत्यु॑प - प्र॒यन्तः॑ । अ॒द्ध्व॒रमिति॑ । इत्या॑ह । आ॒ह॒ स्तोम᳚म् । स्तोम॑मे॒व । ए॒वास्मै᳚ । अ॒स्मै॒ यु॒न॒क्ति॒ । यु॒न॒क्त्युप॑ । उपेति॑ । इत्या॑ह । आ॒ह॒ प्र॒जा । प्र॒जा वै । प्र॒जेति॑ प्र - जा । वै प॒शवः॑ । प॒शव॒ उप॑ । उपे॒मम् । इ॒मं ॅलो॒कम् । लो॒कम् प्र॒जाम् । प्र॒जामे॒व । प्र॒जामिति॑ प्र - जाम् । ए॒व प॒शून् । प॒शूनि॒मम् । इ॒मं ॅलो॒कम् । लो॒कमुप॑ । उपै॑ति । ए॒त्य॒स्य । अ॒स्य प्र॒त्नाम् । प्र॒त्नामनु॑ । अनु॒ द्युत᳚म् । द्युत॒मिति॑ । इत्या॑ह । आ॒ह॒ सु॒व॒र्गः । सु॒व॒र्गो वै । सु॒व॒र्ग इति॑ सुवः - गः । वै लो॒कः । लो॒कः प्र॒त्नः । प्र॒त्नः सु॑व॒र्गम् । सु॒व॒र्गमे॒व । सु॒व॒र्गमिति॑ सुवः - गम् । ए॒व लो॒कम् । लो॒कꣳ स॒मारो॑हति । स॒मारो॑हत्य॒ग्निः । स॒मारो॑ह॒तीति॑ सम् - आरो॑हति । अ॒ग्निर् मू॒र्द्धा । मू॒र्द्धा दि॒वः । दि॒वः क॒कुत् । क॒कुदिति॑ । इत्या॑ह । आ॒ह॒ मू॒र्द्धान᳚म् । मू॒र्द्धान॑मे॒व \newline

\textbf{Jatai Paata} \newline

1. अय॑ज्ञो॒ वै वा अय॒ज्ञो ऽय॑ज्ञो॒ वै । \newline
2. वा ए॒ष ए॒ष वै वा ए॒षः । \newline
3. ए॒ष यो य ए॒ष ए॒षो यः । \newline
4. यो॑ ऽसा॒मा ऽसा॒मा यो यो॑ ऽसा॒मा । \newline
5. अ॒सा॒मोप॑प्र॒यन्त॑ उपप्र॒यन्तो॑ ऽसा॒मा ऽसा॒मोप॑प्र॒यन्तः॑ । \newline
6. उ॒प॒प्र॒यन्तो॑ अद्ध्व॒र म॑द्ध्व॒र मु॑पप्र॒यन्त॑ उपप्र॒यन्तो॑ अद्ध्व॒रम् । \newline
7. उ॒प॒प्र॒यन्त॒ इत्यु॑प - प्र॒यन्तः॑ । \newline
8. अ॒द्ध्व॒र मितीत्य॑द्ध्व॒र म॑द्ध्व॒र मिति॑ । \newline
9. इत्या॑हा॒हे तीत्या॑ह । \newline
10. आ॒ह॒ स्तोम॒(ग्ग्॒) स्तोम॑ माहाह॒ स्तोम᳚म् । \newline
11. स्तोम॑ मे॒वैव स्तोम॒(ग्ग्॒) स्तोम॑ मे॒व । \newline
12. ए॒वास्मा॑ अस्मा ए॒वैवास्मै᳚ । \newline
13. अ॒स्मै॒ यु॒न॒क्ति॒ यु॒न॒क्त्य॒स्मा॒ अ॒स्मै॒ यु॒न॒क्ति॒ । \newline
14. यु॒न॒क्त्युपोप॑ युनक्ति युन॒क्त्युप॑ । \newline
15. उपे तीत्युपोपे ति॑ । \newline
16. इत्या॑हा॒हे तीत्या॑ह । \newline
17. आ॒ह॒ प्र॒जा प्र॒जा ऽऽहा॑ह प्र॒जा । \newline
18. प्र॒जा वै वै प्र॒जा प्र॒जा वै । \newline
19. प्र॒जेति॑ प्र - जा । \newline
20. वै प॒शवः॑ प॒शवो॒ वै वै प॒शवः॑ । \newline
21. प॒शव॒ उपोप॑ प॒शवः॑ प॒शव॒ उप॑ । \newline
22. उपे॒ म मि॒म मुपोपे॒ मम् । \newline
23. इ॒मम् ॅलो॒कम् ॅलो॒क मि॒म मि॒मम् ॅलो॒कम् । \newline
24. लो॒कम् प्र॒जाम् प्र॒जाम् ॅलो॒कम् ॅलो॒कम् प्र॒जाम् । \newline
25. प्र॒जा मे॒वैव प्र॒जाम् प्र॒जा मे॒व । \newline
26. प्र॒जामिति॑ प्र - जाम् । \newline
27. ए॒व प॒शून् प॒शू ने॒वैव प॒शून् । \newline
28. प॒शू नि॒म मि॒मम् प॒शून् प॒शू नि॒मम् । \newline
29. इ॒मम् ॅलो॒कम् ॅलो॒क मि॒म मि॒मम् ॅलो॒कम् । \newline
30. लो॒क मुपोप॑ लो॒कम् ॅलो॒क मुप॑ । \newline
31. उपै᳚त्ये॒त्युपोपै॑ति । \newline
32. ए॒त्य॒स्यास्यैत्ये᳚त्य॒स्य । \newline
33. अ॒स्य प्र॒त्नाम् प्र॒त्ना म॒स्यास्य प्र॒त्नाम् । \newline
34. प्र॒त्ना मन्वनु॑ प्र॒त्नाम् प्र॒त्ना मनु॑ । \newline
35. अनु॒ द्युत॒म् द्युत॒ मन्वनु॒ द्युत᳚म् । \newline
36. द्युत॒ मितीति॒ द्युत॒म् द्युत॒ मिति॑ । \newline
37. इत्या॑हा॒हे तीत्या॑ह । \newline
38. आ॒ह॒ सु॒व॒र्गः सु॑व॒र्ग आ॑हाह सुव॒र्गः । \newline
39. सु॒व॒र्गो वै वै सु॑व॒र्गः सु॑व॒र्गो वै । \newline
40. सु॒व॒र्ग इति॑ सुवः - गः । \newline
41. वै लो॒को लो॒को वै वै लो॒कः । \newline
42. लो॒कः प्र॒त्नः प्र॒त्नो लो॒को लो॒कः प्र॒त्नः । \newline
43. प्र॒त्नः सु॑व॒र्गꣳ सु॑व॒र्गम् प्र॒त्नः प्र॒त्नः सु॑व॒र्गम् । \newline
44. सु॒व॒र्ग मे॒वैव सु॑व॒र्गꣳ सु॑व॒र्ग मे॒व । \newline
45. सु॒व॒र्गमिति॑ सुवः - गम् । \newline
46. ए॒व लो॒कम् ॅलो॒क मे॒वैव लो॒कम् । \newline
47. लो॒कꣳ स॒मारो॑हति स॒मारो॑हति लो॒कम् ॅलो॒कꣳ स॒मारो॑हति । \newline
48. स॒मारो॑हत्य॒ग्निर॒ग्निः स॒मारो॑हति स॒मारो॑हत्य॒ग्निः । \newline
49. स॒मारो॑ह॒तीति॑ सं - आरो॑हति । \newline
50. अ॒ग्निर् मू॒र्द्धा मू॒र्द्धा ऽग्निर॒ग्निर् मू॒र्द्धा । \newline
51. मू॒र्द्धा दि॒वो दि॒वो मू॒र्द्धा मू॒र्द्धा दि॒वः । \newline
52. दि॒वः क॒कुत् क॒कुद् दि॒वो दि॒वः क॒कुत् । \newline
53. क॒कुदितीति॑ क॒कुत् क॒कुदिति॑ । \newline
54. इत्या॑हा॒हे तीत्या॑ह । \newline
55. आ॒ह॒ मू॒र्द्धान॑म् मू॒र्द्धान॑ माहाह मू॒र्द्धान᳚म् । \newline
56. मू॒र्द्धान॑ मे॒वैव मू॒र्द्धान॑म् मू॒र्द्धान॑ मे॒व । \newline

\textbf{Ghana Paata } \newline

1. अय॑ज्ञो॒ वै वा अय॒ज्ञो ऽय॑ज्ञो॒ वा ए॒ष ए॒ष वा अय॒ज्ञो ऽय॑ज्ञो॒ वा ए॒षः । \newline
2. वा ए॒ष ए॒ष वै वा ए॒ष यो य ए॒ष वै वा ए॒ष यः । \newline
3. ए॒ष यो य ए॒ष ए॒ष यो॑ ऽसा॒मा ऽसा॒मा य ए॒ष ए॒ष यो॑ ऽसा॒मा । \newline
4. यो॑ ऽसा॒मा ऽसा॒मा यो यो॑ ऽसा॒मोप॑प्र॒यन्त॑ उपप्र॒यन्तो॑ ऽसा॒मा यो यो॑ ऽसा॒मोप॑प्र॒यन्तः॑ । \newline
5. अ॒सा॒मोप॑प्र॒यन्त॑ उपप्र॒यन्तो॑ ऽसा॒मा ऽसा॒मोप॑प्र॒यन्तो॑ अद्ध्व॒र म॑द्ध्व॒र मु॑पप्र॒यन्तो॑ ऽसा॒मा ऽसा॒मोप॑प्र॒यन्तो॑ अद्ध्व॒रम् । \newline
6. उ॒प॒प्र॒यन्तो॑ अद्ध्व॒र म॑द्ध्व॒र मु॑पप्र॒यन्त॑ उपप्र॒यन्तो॑ अद्ध्व॒र मितीत्य॑द्ध्व॒र मु॑पप्र॒यन्त॑ उपप्र॒यन्तो॑ अद्ध्व॒र मिति॑ । \newline
7. उ॒प॒प्र॒यन्त॒ इत्यु॑प - प्र॒यन्तः॑ । \newline
8. अ॒द्ध्व॒र मितीत्य॑द्ध्व॒र म॑द्ध्व॒र मित्या॑हा॒हे त्य॑द्ध्व॒र म॑द्ध्व॒र मित्या॑ह । \newline
9. इत्या॑हा॒हे तीत्या॑ह॒ स्तोम॒(ग्ग्॒) स्तोम॑ मा॒हे तीत्या॑ह॒ स्तोम᳚म् । \newline
10. आ॒ह॒ स्तोम॒(ग्ग्॒) स्तोम॑ माहाह॒ स्तोम॑ मे॒वैव स्तोम॑ माहाह॒ स्तोम॑ मे॒व । \newline
11. स्तोम॑ मे॒वैव स्तोम॒(ग्ग्॒) स्तोम॑ मे॒वास्मा॑ अस्मा ए॒व स्तोम॒(ग्ग्॒) स्तोम॑ मे॒वास्मै᳚ । \newline
12. ए॒वास्मा॑ अस्मा ए॒वैवास्मै॑ युनक्ति युनक्त्यस्मा ए॒वैवास्मै॑ युनक्ति । \newline
13. अ॒स्मै॒ यु॒न॒क्ति॒ यु॒न॒क्त्य॒स्मा॒ अ॒स्मै॒ यु॒न॒क्त्युपोप॑ युनक्त्यस्मा अस्मै युन॒क्त्युप॑ । \newline
14. यु॒न॒क्त्युपोप॑ युनक्ति युन॒क्त्युपे तीत्युप॑ युनक्ति युन॒क्त्युपे ति॑ । \newline
15. उपे तीत्युपोपे त्या॑हा॒हे त्युपोपे त्या॑ह । \newline
16. इत्या॑हा॒हे तीत्या॑ह प्र॒जा प्र॒जा ऽऽहे तीत्या॑ह प्र॒जा । \newline
17. आ॒ह॒ प्र॒जा प्र॒जा ऽऽहा॑ह प्र॒जा वै वै प्र॒जा ऽऽहा॑ह प्र॒जा वै । \newline
18. प्र॒जा वै वै प्र॒जा प्र॒जा वै प॒शवः॑ प॒शवो॒ वै प्र॒जा प्र॒जा वै प॒शवः॑ । \newline
19. प्र॒जेति॑ प्र - जा । \newline
20. वै प॒शवः॑ प॒शवो॒ वै वै प॒शव॒ उपोप॑ प॒शवो॒ वै वै प॒शव॒ उप॑ । \newline
21. प॒शव॒ उपोप॑ प॒शवः॑ प॒शव॒ उपे॒ म मि॒म मुप॑ प॒शवः॑ प॒शव॒ उपे॒ मम् । \newline
22. उपे॒ म मि॒म मुपोपे॒ मम् ॅलो॒कम् ॅलो॒क मि॒म मुपोपे॒ मम् ॅलो॒कम् । \newline
23. इ॒मम् ॅलो॒कम् ॅलो॒क मि॒म मि॒मम् ॅलो॒कम् प्र॒जाम् प्र॒जाम् ॅलो॒क मि॒म मि॒मम् ॅलो॒कम् प्र॒जाम् । \newline
24. लो॒कम् प्र॒जाम् प्र॒जाम् ॅलो॒कम् ॅलो॒कम् प्र॒जा मे॒वैव प्र॒जाम् ॅलो॒कम् ॅलो॒कम् प्र॒जा मे॒व । \newline
25. प्र॒जा मे॒वैव प्र॒जाम् प्र॒जा मे॒व प॒शून् प॒शू ने॒व प्र॒जाम् प्र॒जा मे॒व प॒शून् । \newline
26. प्र॒जामिति॑ प्र - जाम् । \newline
27. ए॒व प॒शून् प॒शू ने॒वैव प॒शू नि॒म मि॒मम् प॒शू ने॒वैव प॒शू नि॒मम् । \newline
28. प॒शू नि॒म मि॒मम् प॒शून् प॒शू नि॒मम् ॅलो॒कम् ॅलो॒क मि॒मम् प॒शून् प॒शू नि॒मम् ॅलो॒कम् । \newline
29. इ॒मम् ॅलो॒कम् ॅलो॒क मि॒म मि॒मम् ॅलो॒क मुपोप॑ लो॒क मि॒म मि॒मम् ॅलो॒क मुप॑ । \newline
30. लो॒क मुपोप॑ लो॒कम् ॅलो॒क मुपै᳚त्ये॒त्युप॑ लो॒कम् ॅलो॒क मुपै॑ति । \newline
31. उपै᳚त्ये॒ त्युपोपै᳚ त्य॒स्या स्यैत्यु पोपै᳚त्य॒स्य । \newline
32. ए॒त्य॒ स्यास्यैत्ये᳚त्य॒स्य प्र॒त्नाम् प्र॒त्ना म॒स्यैत्ये᳚त्य॒स्य प्र॒त्नाम् । \newline
33. अ॒स्य प्र॒त्नाम् प्र॒त्ना म॒स्यास्य प्र॒त्ना मन्वनु॑ प्र॒त्ना म॒स्यास्य प्र॒त्ना मनु॑ । \newline
34. प्र॒त्ना मन्वनु॑ प्र॒त्नाम् प्र॒त्ना मनु॒ द्युत॒म् द्युत॒ मनु॑ प्र॒त्नाम् प्र॒त्ना मनु॒ द्युत᳚म् । \newline
35. अनु॒ द्युत॒म् द्युत॒ मन्वनु॒ द्युत॒ मितीति॒ द्युत॒ मन्वनु॒ द्युत॒ मिति॑ । \newline
36. द्युत॒ मितीति॒ द्युत॒म् द्युत॒ मित्या॑हा॒हे ति॒ द्युत॒म् द्युत॒ मित्या॑ह । \newline
37. इत्या॑हा॒हे तीत्या॑ह सुव॒र्गः सु॑व॒र्ग आ॒हे तीत्या॑ह सुव॒र्गः । \newline
38. आ॒ह॒ सु॒व॒र्गः सु॑व॒र्ग आ॑हाह सुव॒र्गो वै वै सु॑व॒र्ग आ॑हाह सुव॒र्गो वै । \newline
39. सु॒व॒र्गो वै वै सु॑व॒र्गः सु॑व॒र्गो वै लो॒को लो॒को वै सु॑व॒र्गः सु॑व॒र्गो वै लो॒कः । \newline
40. सु॒व॒र्ग इति॑ सुवः - गः । \newline
41. वै लो॒को लो॒को वै वै लो॒कः प्र॒त्नः प्र॒त्नो लो॒को वै वै लो॒कः प्र॒त्नः । \newline
42. लो॒कः प्र॒त्नः प्र॒त्नो लो॒को लो॒कः प्र॒त्नः सु॑व॒र्गꣳ सु॑व॒र्गम् प्र॒त्नो लो॒को लो॒कः प्र॒त्नः सु॑व॒र्गम् । \newline
43. प्र॒त्नः सु॑व॒र्गꣳ सु॑व॒र्गम् प्र॒त्नः प्र॒त्नः सु॑व॒र्ग मे॒वैव सु॑व॒र्गम् प्र॒त्नः प्र॒त्नः सु॑व॒र्ग मे॒व । \newline
44. सु॒व॒र्ग मे॒वैव सु॑व॒र्गꣳ सु॑व॒र्ग मे॒व लो॒कम् ॅलो॒क मे॒व सु॑व॒र्गꣳ सु॑व॒र्ग मे॒व लो॒कम् । \newline
45. सु॒व॒र्गमिति॑ सुवः - गम् । \newline
46. ए॒व लो॒कम् ॅलो॒क मे॒वैव लो॒कꣳ स॒मारो॑हति स॒मारो॑हति लो॒क मे॒वैव लो॒कꣳ स॒मारो॑हति । \newline
47. लो॒कꣳ स॒मारो॑हति स॒मारो॑हति लो॒कम् ॅलो॒कꣳ स॒मारो॑हत्य॒ग्नि र॒ग्निः स॒मारो॑हति लो॒कम् ॅलो॒कꣳ स॒मारो॑हत्य॒ग्निः । \newline
48. स॒मारो॑हत्य॒ग्नि र॒ग्निः स॒मारो॑हति स॒मारो॑हत्य॒ग्निर् मू॒र्द्धा मू॒र्द्धा ऽग्निः स॒मारो॑हति स॒मारो॑हत्य॒ग्निर् मू॒र्द्धा । \newline
49. स॒मारो॑ह॒तीति॑ सं - आरो॑हति । \newline
50. अ॒ग्निर् मू॒र्द्धा मू॒र्द्धा ऽग्निर॒ग्निर् मू॒र्द्धा दि॒वो दि॒वो मू॒र्द्धा ऽग्निर॒ग्निर् मू॒र्द्धा दि॒वः । \newline
51. मू॒र्द्धा दि॒वो दि॒वो मू॒र्द्धा मू॒र्द्धा दि॒वः क॒कुत् क॒कुद् दि॒वो मू॒र्द्धा मू॒र्द्धा दि॒वः क॒कुत् । \newline
52. दि॒वः क॒कुत् क॒कुद् दि॒वो दि॒वः क॒कुदितीति॑ क॒कुद् दि॒वो दि॒वः क॒कुदिति॑ । \newline
53. क॒कुदितीति॑ क॒कुत् क॒कुदित्या॑हा॒हेति॑ क॒कुत् क॒कुदित्या॑ह । \newline
54. इत्या॑हा॒हे तीत्या॑ह मू॒र्द्धान॑म् मू॒र्द्धान॑ मा॒हे तीत्या॑ह मू॒र्द्धान᳚म् । \newline
55. आ॒ह॒ मू॒र्द्धान॑म् मू॒र्द्धान॑ माहाह मू॒र्द्धान॑ मे॒वैव मू॒र्द्धान॑ माहाह मू॒र्द्धान॑ मे॒व । \newline
56. मू॒र्द्धान॑ मे॒वैव मू॒र्द्धान॑म् मू॒र्द्धान॑ मे॒वैन॑ मेन मे॒व मू॒र्द्धान॑म् मू॒र्द्धान॑ मे॒वैन᳚म् । \newline
\pagebreak
\markright{ TS 1.5.7.2  \hfill https://www.vedavms.in \hfill}
\addcontentsline{toc}{section}{ TS 1.5.7.2 }
\section*{ TS 1.5.7.2 }

\textbf{TS 1.5.7.2 } \newline
\textbf{Samhita Paata} \newline

मे॒वैनꣳ॑ समा॒नानां᳚ करो॒त्यथो॑ देवलो॒कादे॒व म॑नुष्यलो॒के प्रति॑ तिष्ठत्य॒यमि॒ह प्र॑थ॒मो धा॑यि धा॒तृभि॒रित्या॑ह॒ मुख्य॑मे॒वैनं॑ करोत्यु॒भा वा॑मिन्द्राग्नी आहु॒वद्ध्या॒ इत्या॒हौजो॒ बल॑मे॒वाव॑ रुन्धे॒ ऽयं ते॒ योनि॑र्. ऋ॒त्विय॒ इत्या॑ह प॒शवो॒ वै र॒यिः प॒शूने॒वाव॑ रुन्धे ष॒ड्भिरुप॑ तिष्ठते॒ षड्वा - [ ] \newline

\textbf{Pada Paata} \newline

ए॒व । ए॒न॒म् । स॒मा॒नाना᳚म् । क॒रो॒ति॒ । अथो॒ इति॑ । दे॒व॒लो॒कादिति॑ देव - लो॒कात् । ए॒व । म॒नु॒ष्य॒लो॒क इति॑ मनुष्य - लो॒के । प्रतीति॑ । ति॒ष्ठ॒ति॒ । अ॒यम् । इ॒ह । प्र॒थ॒मः । धा॒यि॒ । धा॒तृभि॒रिति॑ धा॒तृ - भिः॒ । इति॑ । आ॒ह॒ । मुख्य᳚म् । ए॒व । ए॒न॒म् । क॒रो॒ति॒ । उ॒भा । वा॒म् । इ॒न्द्रा॒ग्नी॒ इती᳚न्द्र - अ॒ग्नी॒ । आ॒हु॒वद्ध्यै᳚ । इति॑ । आ॒ह॒ । ओजः॑ । बल᳚म् । ए॒व । अवेति॑ । रु॒न्धे॒ । अ॒यम् । ते॒ । योनिः॑ । ऋ॒त्वियः॑ । इति॑ । आ॒ह॒ । प॒शवः॑ । वै । र॒यिः । प॒शून् । ए॒व । अवेति॑ । रु॒न्धे॒ । ष॒ड्भिरिति॑ षट् - भिः । उपेति॑ । ति॒ष्ठ॒ते॒ । षट् । वै ।  \newline


\textbf{Krama Paata} \newline

ए॒वैन᳚म् । ए॒नꣳ॒॒ स॒मा॒नाना᳚म् । स॒मा॒नाना᳚म् करोति । क॒रो॒त्यथो᳚ । अथो॑ देवलो॒कात् । अथो॒ इत्यथो᳚ । दे॒व॒लो॒कादे॒व । दे॒व॒लो॒कादिति॑ देव - लो॒कात् । ए॒व म॑नुष्यलो॒के । म॒नु॒ष्य॒लो॒के प्रति॑ । म॒नु॒ष्य॒लो॒क इति॑ मनुष्य - लो॒के । प्रति॑ तिष्ठति । ति॒ष्ठ॒त्य॒यम् । अ॒यमि॒ह । इ॒ह प्र॑थ॒मः । प्र॒थ॒मो धा॑यि । धा॒यि॒ धा॒तृभिः॑ । धा॒तृभि॒रिति॑ । धा॒तृभि॒रिति॑ धा॒तृ - भिः॒ । इत्या॑ह । आ॒ह॒ मुख्य᳚म् । मुख्य॑मे॒व । ए॒वैन᳚म् । ए॒न॒म् क॒रो॒ति॒ । क॒रो॒त्यु॒भा । उ॒भा वा᳚म् । वा॒मि॒न्द्रा॒ग्नी॒ । इ॒न्द्रा॒ग्नी॒ आ॒हु॒वद्ध्यै᳚ । इ॒न्द्रा॒ग्नी॒ इती᳚न्द्र - अ॒ग्नी॒ । आ॒हु॒वद्ध्या॒ इति॑ । इत्या॑ह । आ॒हौजः॑ । ओजो॒ बल᳚म् । बल॑मे॒व । ए॒वाव॑ । अव॑ रुन्धे । रु॒न्धे॒ऽयम् । अ॒यम् ते᳚ । ते॒ योनिः॑ । योनि॑र्. ऋ॒त्वियः॑ । ऋ॒त्विय॒ इति॑ । इत्या॑ह । आ॒ह॒ प॒शवः॑ । प॒शवो॒ वै । वै र॒यिः । र॒यिः प॒शून् । प॒शूने॒व । ए॒वाव॑ । अव॑ रुन्धे । रु॒न्धे॒ ष॒ड्भिः । ष॒ड्भिरुप॑ । ष॒ड्भिरिति॑ षट् - भिः । उप॑ तिष्ठते । ति॒ष्ठ॒ते॒ षट् । षड् वै । वा ऋ॒तवः॑ \newline

\textbf{Jatai Paata} \newline

1. ए॒वैन॑ मेन मे॒वैवैन᳚म् । \newline
2. ए॒न॒(ग्म्॒) स॒मा॒नाना(ग्म्॑) समा॒नाना॑ मेन मेनꣳ समा॒नाना᳚म् । \newline
3. स॒मा॒नाना᳚म् करोति करोति समा॒नाना(ग्म्॑) समा॒नाना᳚म् करोति । \newline
4. क॒रो॒त्यथो॒ अथो॑ करोति करो॒त्यथो᳚ । \newline
5. अथो॑ देवलो॒काद् दे॑वलो॒कादथो॒ अथो॑ देवलो॒कात् । \newline
6. अथो॒ इत्यथो᳚ । \newline
7. दे॒व॒लो॒कादे॒वैव दे॑वलो॒काद् दे॑वलो॒कादे॒व । \newline
8. दे॒व॒लो॒कादिति॑ देव - लो॒कात् । \newline
9. ए॒व म॑नुष्यलो॒के म॑नुष्यलो॒क ए॒वैव म॑नुष्यलो॒के । \newline
10. म॒नु॒ष्य॒लो॒के प्रति॒ प्रति॑ मनुष्यलो॒के म॑नुष्यलो॒के प्रति॑ । \newline
11. म॒नु॒ष्य॒लो॒क इति॑ मनुष्य - लो॒के । \newline
12. प्रति॑ तिष्ठति तिष्ठति॒ प्रति॒ प्रति॑ तिष्ठति । \newline
13. ति॒ष्ठ॒त्य॒य म॒यम् ति॑ष्ठति तिष्ठत्य॒यम् । \newline
14. अ॒य मि॒हे हाय म॒य मि॒ह । \newline
15. इ॒ह प्र॑थ॒मः प्र॑थ॒म इ॒हे ह प्र॑थ॒मः । \newline
16. प्र॒थ॒मो धा॑यि धायि प्रथ॒मः प्र॑थ॒मो धा॑यि । \newline
17. धा॒यि॒ धा॒तृभि॑र् धा॒तृभि॑र् धायि धायि धा॒तृभिः॑ । \newline
18. धा॒तृभि॒रितीति॑ धा॒तृभि॑र् धा॒तृभि॒रिति॑ । \newline
19. धा॒तृभि॒रिति॑ धा॒तृ - भिः॒ । \newline
20. इत्या॑हा॒हे तीत्या॑ह । \newline
21. आ॒ह॒ मुख्य॒म् मुख्य॑ माहाह॒ मुख्य᳚म् । \newline
22. मुख्य॑ मे॒वैव मुख्य॒म् मुख्य॑ मे॒व । \newline
23. ए॒वैन॑ मेन मे॒वैवैन᳚म् । \newline
24. ए॒न॒म् क॒रो॒ति॒ क॒रो॒त्ये॒न॒ मे॒न॒म् क॒रो॒ति॒ । \newline
25. क॒रो॒त्यु॒भोभा क॑रोति करोत्यु॒भा । \newline
26. उ॒भा वां᳚ ॅवा मु॒भोभा वा᳚म् । \newline
27. वा॒ मि॒न्द्रा॒ग्नी॒ इ॒न्द्रा॒ग्नी॒ वां॒ ॅवा॒ मि॒न्द्रा॒ग्नी॒ । \newline
28. इ॒न्द्रा॒ग्नी॒ आ॒हु॒वद्ध्या॑ आहु॒वद्ध्या॑ इन्द्राग्नी इन्द्राग्नी आहु॒वद्ध्यै᳚ । \newline
29. इ॒न्द्रा॒ग्नी॒ इती᳚न्द्र - अ॒ग्नी॒ । \newline
30. आ॒हु॒वद्ध्या॒ इतीत्या॑हु॒वद्ध्या॑ आहु॒वद्ध्या॒ इति॑ । \newline
31. इत्या॑हा॒हे तीत्या॑ह । \newline
32. आ॒हौज॒ ओज॑ आहा॒हौजः॑ । \newline
33. ओजो॒ बल॒म् बल॒ मोज॒ ओजो॒ बल᳚म् । \newline
34. बल॑ मे॒वैव बल॒म् बल॑ मे॒व । \newline
35. ए॒वावावै॒वैवाव॑ । \newline
36. अव॑ रुन्धे रु॒न्धे ऽवाव॑ रुन्धे । \newline
37. रु॒न्धे॒ ऽय म॒यꣳ रु॑न्धे रुन्धे॒ ऽयम् । \newline
38. अ॒यम् ते॑ ते॒ ऽय म॒यम् ते᳚ । \newline
39. ते॒ योनि॒र् योनि॑स्ते ते॒ योनिः॑ । \newline
40. योनि॑र्. ऋ॒त्विय॑ ऋ॒त्वियो॒ योनि॒र् योनि॑र्. ऋ॒त्वियः॑ । \newline
41. ऋ॒त्विय॒ इतीत्यृ॒त्विय॑ ऋ॒त्विय॒ इति॑ । \newline
42. इत्या॑हा॒हे तीत्या॑ह । \newline
43. आ॒ह॒ प॒शवः॑ प॒शव॑ आहाह प॒शवः॑ । \newline
44. प॒शवो॒ वै वै प॒शवः॑ प॒शवो॒ वै । \newline
45. वै र॒यी र॒यिर् वै वै र॒यिः । \newline
46. र॒यिः प॒शून् प॒शून् र॒यी र॒यिः प॒शून् । \newline
47. प॒शू ने॒वैव प॒शून् प॒शू ने॒व । \newline
48. ए॒वावावै॒वैवाव॑ । \newline
49. अव॑ रुन्धे रु॒न्धे ऽवाव॑ रुन्धे । \newline
50. रु॒न्धे॒ ष॒ड्भि ष्ष॒ड्भी रु॑न्धे रुन्धे ष॒ड्भिः । \newline
51. ष॒ड्भिरुपोप॑ ष॒ड्भि ष्ष॒ड्भिरुप॑ । \newline
52. ष॒ड्भिरिति॑ षट् - भिः । \newline
53. उप॑ तिष्ठते तिष्ठत॒ उपोप॑ तिष्ठते । \newline
54. ति॒ष्ठ॒ते॒ षट् थ् षट् ति॑ष्ठते तिष्ठते॒ षट् । \newline
55. षड् वै वै षट् थ् षड् वै । \newline
56. वा ऋ॒तव॑ ऋ॒तवो॒ वै वा ऋ॒तवः॑ । \newline

\textbf{Ghana Paata } \newline

1. ए॒वैन॑ मेन मे॒वैवैन(ग्म्॑) समा॒नाना(ग्म्॑) समा॒नाना॑ मेन मे॒वैवैन(ग्म्॑) समा॒नाना᳚म् । \newline
2. ए॒न॒(ग्म्॒) स॒मा॒नाना(ग्म्॑) समा॒नाना॑ मेन मेनꣳ समा॒नाना᳚म् करोति करोति समा॒नाना॑ मेन मेनꣳ समा॒नाना᳚म् करोति । \newline
3. स॒मा॒नाना᳚म् करोति करोति समा॒नाना(ग्म्॑) समा॒नाना᳚म् करो॒त्यथो॒ अथो॑ करोति समा॒नाना(ग्म्॑) समा॒नाना᳚म् करो॒त्यथो᳚ । \newline
4. क॒रो॒त्यथो॒ अथो॑ करोति करो॒त्यथो॑ देवलो॒काद् दे॑वलो॒कादथो॑ करोति करो॒त्यथो॑ देवलो॒कात् । \newline
5. अथो॑ देवलो॒काद् दे॑वलो॒कादथो॒ अथो॑ देवलो॒कादे॒वैव दे॑वलो॒कादथो॒ अथो॑ देवलो॒कादे॒व । \newline
6. अथो॒ इत्यथो᳚ । \newline
7. दे॒व॒लो॒कादे॒वैव दे॑वलो॒काद् दे॑वलो॒कादे॒व म॑नुष्यलो॒के म॑नुष्यलो॒क ए॒व दे॑वलो॒काद् दे॑वलो॒कादे॒व म॑नुष्यलो॒के । \newline
8. दे॒व॒लो॒कादिति॑ देव - लो॒कात् । \newline
9. ए॒व म॑नुष्यलो॒के म॑नुष्यलो॒क ए॒वैव म॑नुष्यलो॒के प्रति॒ प्रति॑ मनुष्यलो॒क ए॒वैव म॑नुष्यलो॒के प्रति॑ । \newline
10. म॒नु॒ष्य॒लो॒के प्रति॒ प्रति॑ मनुष्यलो॒के म॑नुष्यलो॒के प्रति॑ तिष्ठति तिष्ठति॒ प्रति॑ मनुष्यलो॒के म॑नुष्यलो॒के प्रति॑ तिष्ठति । \newline
11. म॒नु॒ष्य॒लो॒क इति॑ मनुष्य - लो॒के । \newline
12. प्रति॑ तिष्ठति तिष्ठति॒ प्रति॒ प्रति॑ तिष्ठत्य॒य म॒यम् ति॑ष्ठति॒ प्रति॒ प्रति॑ तिष्ठत्य॒यम् । \newline
13. ति॒ष्ठ॒त्य॒य म॒यम् ति॑ष्ठति तिष्ठत्य॒य मि॒हे हायम् ति॑ष्ठति तिष्ठत्य॒य मि॒ह । \newline
14. अ॒य मि॒हे हाय म॒य मि॒ह प्र॑थ॒मः प्र॑थ॒म इ॒हाय म॒य मि॒ह प्र॑थ॒मः । \newline
15. इ॒ह प्र॑थ॒मः प्र॑थ॒म इ॒हे ह प्र॑थ॒मो धा॑यि धायि प्रथ॒म इ॒हे ह प्र॑थ॒मो धा॑यि । \newline
16. प्र॒थ॒मो धा॑यि धायि प्रथ॒मः प्र॑थ॒मो धा॑यि धा॒तृभि॑र् धा॒तृभि॑र् धायि प्रथ॒मः प्र॑थ॒मो धा॑यि धा॒तृभिः॑ । \newline
17. धा॒यि॒ धा॒तृभि॑र् धा॒तृभि॑र् धायि धायि धा॒तृभि॒रितीति॑ धा॒तृभि॑र् धायि धायि धा॒तृभि॒रिति॑ । \newline
18. धा॒तृभि॒रितीति॑ धा॒तृभि॑र् धा॒तृभि॒ रित्या॑हा॒हे ति॑ धा॒तृभि॑र् धा॒तृभि॒रित्या॑ह । \newline
19. धा॒तृभि॒रिति॑ धा॒तृ - भिः॒ । \newline
20. इत्या॑हा॒हे तीत्या॑ह॒ मुख्य॒म् मुख्य॑ मा॒हे तीत्या॑ह॒ मुख्य᳚म् । \newline
21. आ॒ह॒ मुख्य॒म् मुख्य॑ माहाह॒ मुख्य॑ मे॒वैव मुख्य॑ माहाह॒ मुख्य॑ मे॒व । \newline
22. मुख्य॑ मे॒वैव मुख्य॒म् मुख्य॑ मे॒वैन॑ मेन मे॒व मुख्य॒म् मुख्य॑ मे॒वैन᳚म् । \newline
23. ए॒वैन॑ मेन मे॒वैवैन॑म् करोति करोत्येन मे॒वैवैन॑म् करोति । \newline
24. ए॒न॒म् क॒रो॒ति॒ क॒रो॒त्ये॒न॒ मे॒न॒म् क॒रो॒त्यु॒भोभा क॑रोत्येन मेनम् करोत्यु॒भा । \newline
25. क॒रो॒त्यु॒भोभा क॑रोति करोत्यु॒भा वां᳚ ॅवा मु॒भा क॑रोति करोत्यु॒भा वा᳚म् । \newline
26. उ॒भा वां᳚ ॅवा मु॒भोभा वा॑ मिन्द्राग्नी इन्द्राग्नी वा मु॒भोभा वा॑ मिन्द्राग्नी । \newline
27. वा॒ मि॒न्द्रा॒ग्नी॒ इ॒न्द्रा॒ग्नी॒ वां॒ ॅवा॒ मि॒न्द्रा॒ग्नी॒ आ॒हु॒वद्ध्या॑ आहु॒वद्ध्या॑ इन्द्राग्नी वां ॅवा मिन्द्राग्नी आहु॒वद्ध्यै᳚ । \newline
28. इ॒न्द्रा॒ग्नी॒ आ॒हु॒वद्ध्या॑ आहु॒वद्ध्या॑ इन्द्राग्नी इन्द्राग्नी आहु॒वद्ध्या॒ इतीत्या॑हु॒वद्ध्या॑ इन्द्राग्नी इन्द्राग्नी आहु॒वद्ध्या॒ इति॑ । \newline
29. इ॒न्द्रा॒ग्नी॒ इती᳚न्द्र - अ॒ग्नी॒ । \newline
30. आ॒हु॒वद्ध्या॒ इतीत्या॑हु॒वद्ध्या॑ आहु॒वद्ध्या॒ इत्या॑हा॒हे त्या॑हु॒वद्ध्या॑ आहु॒वद्ध्या॒ इत्या॑ह । \newline
31. इत्या॑हा॒हे तीत्या॒हौज॒ ओज॑ आ॒हे तीत्या॒हौजः॑ । \newline
32. आ॒हौज॒ ओज॑ आहा॒हौजो॒ बल॒म् बल॒ मोज॑ आहा॒हौजो॒ बल᳚म् । \newline
33. ओजो॒ बल॒म् बल॒ मोज॒ ओजो॒ बल॑ मे॒वैव बल॒ मोज॒ ओजो॒ बल॑ मे॒व । \newline
34. बल॑ मे॒वैव बल॒म् बल॑ मे॒वावावै॒व बल॒म् बल॑ मे॒वाव॑ । \newline
35. ए॒वावावै॒वैवाव॑ रुन्धे रु॒न्धे ऽवै॒वैवाव॑ रुन्धे । \newline
36. अव॑ रुन्धे रु॒न्धे ऽवाव॑ रुन्धे॒ ऽय म॒यꣳ रु॒न्धे ऽवाव॑ रुन्धे॒ ऽयम् । \newline
37. रु॒न्धे॒ ऽय म॒यꣳ रु॑न्धे रुन्धे॒ ऽयम् ते॑ ते॒ ऽयꣳ रु॑न्धे रुन्धे॒ ऽयम् ते᳚ । \newline
38. अ॒यम् ते॑ ते॒ ऽय म॒यम् ते॒ योनि॒र् योनि॑स्ते॒ ऽय म॒यम् ते॒ योनिः॑ । \newline
39. ते॒ योनि॒र् योनि॑स्ते ते॒ योनि॑र्. ऋ॒त्विय॑ ऋ॒त्वियो॒ योनि॑स्ते ते॒ योनि॑र्. ऋ॒त्वियः॑ । \newline
40. योनि॑र्. ऋ॒त्विय॑ ऋ॒त्वियो॒ योनि॒र् योनि॑र्. ऋ॒त्विय॒ इतीत्यृ॒त्वियो॒ योनि॒र् योनि॑र्. ऋ॒त्विय॒ इति॑ । \newline
41. ऋ॒त्विय॒ इतीत्यृ॒त्विय॑ ऋ॒त्विय॒ इत्या॑हा॒हे त्यृ॒त्विय॑ ऋ॒त्विय॒ इत्या॑ह । \newline
42. इत्या॑हा॒हे तीत्या॑ह प॒शवः॑ प॒शव॑ आ॒हे तीत्या॑ह प॒शवः॑ । \newline
43. आ॒ह॒ प॒शवः॑ प॒शव॑ आहाह प॒शवो॒ वै वै प॒शव॑ आहाह प॒शवो॒ वै । \newline
44. प॒शवो॒ वै वै प॒शवः॑ प॒शवो॒ वै र॒यी र॒यिर् वै प॒शवः॑ प॒शवो॒ वै र॒यिः । \newline
45. वै र॒यी र॒यिर् वै वै र॒यिः प॒शून् प॒शून् र॒यिर् वै वै र॒यिः प॒शून् । \newline
46. र॒यिः प॒शून् प॒शून् र॒यी र॒यिः प॒शू ने॒वैव प॒शून् र॒यी र॒यिः प॒शू ने॒व । \newline
47. प॒शू ने॒वैव प॒शून् प॒शू ने॒वावावै॒व प॒शून् प॒शू ने॒वाव॑ । \newline
48. ए॒वावावै॒वैवाव॑ रुन्धे रु॒न्धे ऽवै॒वैवाव॑ रुन्धे । \newline
49. अव॑ रुन्धे रु॒न्धे ऽवाव॑ रुन्धे ष॒ड्भिष्ष॒ड्भी रु॒न्धे ऽवाव॑ रुन्धे ष॒ड्भिः । \newline
50. रु॒न्धे॒ ष॒ड्भिष्ष॒ड्भी रु॑न्धे रुन्धे ष॒ड्भिरुपोप॑ ष॒ड्भी रु॑न्धे रुन्धे ष॒ड्भिरुप॑ । \newline
51. ष॒ड्भिरुपोप॑ ष॒ड्भिष्ष॒ड्भिरुप॑ तिष्ठते तिष्ठत॒ उप॑ ष॒ड्भिष्ष॒ड्भिरुप॑ तिष्ठते । \newline
52. ष॒ड्भिरिति॑ षट् - भिः । \newline
53. उप॑ तिष्ठते तिष्ठत॒ उपोप॑ तिष्ठते॒ षट् थ्षट् ति॑ष्ठत॒ उपोप॑ तिष्ठते॒ षट् । \newline
54. ति॒ष्ठ॒ते॒ षट् थ्षट् ति॑ष्ठते तिष्ठते॒ षड् वै वै षट् ति॑ष्ठते तिष्ठते॒ षड् वै । \newline
55. षड् वै वै षट् थ्षड् वा ऋ॒तव॑ ऋ॒तवो॒ वै षट् थ्षड् वा ऋ॒तवः॑ । \newline
56. वा ऋ॒तव॑ ऋ॒तवो॒ वै वा ऋ॒तव॑ ऋ॒तुष्व॒ र्‌तुषु॑ ऋ॒तवो॒ वै वा ऋ॒तव॑ ऋ॒तुषु॑ । \newline
\pagebreak
\markright{ TS 1.5.7.3  \hfill https://www.vedavms.in \hfill}
\addcontentsline{toc}{section}{ TS 1.5.7.3 }
\section*{ TS 1.5.7.3 }

\textbf{TS 1.5.7.3 } \newline
\textbf{Samhita Paata} \newline

ऋ॒तव॑ ऋ॒तुष्वे॒व प्रति॑ तिष्ठति ष॒ड्भिरुत्त॑राभि॒रुप॑ तिष्ठते॒ द्वाद॑श॒ सं प॑द्यन्ते॒ द्वाद॑श॒ मासाः᳚ संॅवथ्स॒रः सं॑ॅवथ्स॒र ए॒व प्रति॑ तिष्ठति॒ यथा॒ वै पुरु॒षोऽश्वो॒ गौर्-जीर्य॑त्ये॒व-म॒ग्निराहि॑तो जीर्यति संॅवथ्स॒रस्य॑ प॒रस्ता॑दाग्निपावमा॒नीभि॒-रुप॑ तिष्ठते पुनर्न॒व-मे॒वैन॑-म॒जरं॑ करो॒त्यथो॑ पु॒नात्ये॒वोप॑ तिष्ठते॒ योग॑ ए॒वास्यै॒ष उप॑ तिष्ठते॒ - [ ] \newline

\textbf{Pada Paata} \newline

ऋ॒तवः॑ । ऋ॒तुषु॑ । ऐ॒व । प्रतीति॑ । ति॒ष्ठ॒ति॒ । ष॒ड्भिरिति॑ षट् - भिः । उत्त॑राभि॒रित्युत् - त॒रा॒भिः॒ । उपेति॑ । ति॒ष्ठ॒ते॒ । द्वाद॑श । समिति॑ । प॒द्य॒न्ते॒ । द्वाद॑श । मासाः᳚ । स॒म्ॅव॒थ्स॒र इति॑ सं - व॒थ्स॒रः । स॒म्ॅव॒थ्स॒र इति॑ सं - व॒थ्स॒रे । ए॒व । प्रतीति॑ । ति॒ष्ठ॒ति॒ । यथा᳚ । वै । पुरु॑षः । अश्वः॑ । गौः । जीर्य॑ति । ए॒वम् । अ॒ग्निः । आहि॑त॒ इत्या - हि॒तः॒ । जी॒र्य॒ति॒ । स॒॒म्ॅव॒थ्स॒रस्येति॑ सं - वथ्स॒रस्य॑ । प॒रस्ता᳚त् । आ॒ग्नि॒पा॒व॒मा॒नीभि॒रित्या᳚ग्नि - पा॒व॒मा॒नीभिः॑ । उपेति॑ । ति॒ष्ठ॒ते॒ । पु॒न॒र्न॒वमिति॑ पुनः-न॒वम् । ए॒व । ए॒न॒म् । अ॒जर᳚म् । क॒रो॒ति॒ । अथो॒ इति॑ । पु॒नाति॑ । ए॒व । उपेति॑ । ति॒ष्ठ॒ते॒ । योगः॑ । ए॒व । अ॒स्य॒ । ए॒षः । उपेति॑ । ति॒ष्ठ॒ते॒ ।  \newline


\textbf{Krama Paata} \newline

ऋ॒तव॑ ऋ॒तुषु॑ । ऋ॒तुष्वे॒व । ए॒व प्रति॑ । प्रति॑ तिष्ठति । ति॒ष्ठ॒ति॒ ष॒ड्भिः । ष॒ड्भिरुत्त॑राभिः । ष॒ड्भिरिति॑ षट् - भिः । उत्त॑राभि॒रुप॑ । उत्त॑राभि॒रित्यु॑त् - त॒रा॒भिः॒ । उप॑ तिष्ठते । ति॒ष्ठ॒ते॒ द्वाद॑श । द्वाद॑श॒ सम् । सम् प॑द्यन्ते । प॒द्य॒न्ते॒ द्वाद॑श । द्वाद॑श॒ मासाः᳚ । मासाः᳚ सम्ॅवथ्स॒रः । स॒म्ॅव॒थ्स॒रः स॑म्ॅवथ्स॒रे । स॒म्ॅव॒थ्स॒र इति॑ सम् - व॒थ्स॒रः । स॒म्ॅव॒थ्स॒र ए॒व । स॒म्ॅव॒थ्स॒र इति॑ सम् - व॒थ्स॒रे । ए॒व प्रति॑ । प्रति॑ तिष्ठति । ति॒ष्ठ॒ति॒ यथा᳚ । यथा॒ वै । वै पुरु॑षः । पुरु॒षोऽश्वः॑ । अश्वो॒ गौः । गौर् जीर्य॑ति । जीर्य॑त्ये॒वम् । ए॒वम॒ग्निः । अ॒ग्निराहि॑तः । आहि॑तो जीर्यति । आहि॑त॒ इत्या - हि॒तः॒ । जी॒र्य॒ति॒ स॒म्ॅव॒थ्स॒रस्य॑ । स॒म्ॅव॒थ्स॒रस्य॑ प॒रस्ता᳚त् । स॒म्ॅव॒थ्स॒रस्येति॑ सं - व॒थ्स॒रस्य॑ । प॒रस्ता॑दाग्निपावमा॒नीभिः॑ । आ॒ग्नि॒पा॒व॒मा॒नीभि॒रुप॑ । आ॒ग्नि॒पा॒व॒मा॒नीभि॒रित्या᳚ग्नि - पा॒व॒मा॒नीभिः॑ । उप॑ तिष्ठते । ति॒ष्ठ॒ते॒ पु॒न॒र्न॒वम् । पु॒न॒र्न॒वमे॒व । पु॒न॒र्न॒वमिति॑ पुनः - न॒वम् । ए॒वैन᳚म् । ए॒न॒म॒जर᳚म् । अ॒जर॑म् करोति । क॒रो॒त्यथो᳚ । अथो॑ पु॒नाति॑ । अथो॒ इत्यथो᳚ । पु॒नात्ये॒व । ए॒वोप॑ । उप॑ तिष्ठते । ति॒ष्ठ॒ते॒ योगः॑ । योग॑ ए॒व । ए॒वास्य॑ । अ॒स्यै॒षः । ए॒ष उप॑ । उप॑ तिष्ठते । ति॒ष्ठ॒ते॒ दमः॑ \newline

\textbf{Jatai Paata} \newline

1. ऋ॒तव॑ ऋ॒तुष॒ र्‌तुष॒ र्‌तव॑ ऋ॒तव॑ ऋ॒तुषु॑ । \newline
2. ऋ॒तु ष्वे॒वैव र्‌तुष॒ र्‌तुष्वे॒व । \newline
3. ए॒व प्रति॒ प्रत्ये॒वैव प्रति॑ । \newline
4. प्रति॑ तिष्ठति तिष्ठति॒ प्रति॒ प्रति॑ तिष्ठति । \newline
5. ति॒ष्ठ॒ति॒ ष॒ड्भि ष्ष॒ड्भि स्ति॑ष्ठति तिष्ठति ष॒ड्भिः । \newline
6. ष॒ड्भि रुत्त॑राभि॒ रुत्त॑राभि ष्ष॒ड्भि ष्ष॒ड्भि रुत्त॑राभिः । \newline
7. ष॒ड्भिरिति॑ षट् - भिः । \newline
8. उत्त॑राभि॒ रुपोपोत्त॑राभि॒ रुत्त॑राभि॒रुप॑ । \newline
9. उत्त॑राभि॒रित्युत् - त॒रा॒भिः॒ । \newline
10. उप॑ तिष्ठते तिष्ठत॒ उपोप॑ तिष्ठते । \newline
11. ति॒ष्ठ॒ते॒ द्वाद॑श॒ द्वाद॑श तिष्ठते तिष्ठते॒ द्वाद॑श । \newline
12. द्वाद॑श॒ सꣳ सम् द्वाद॑श॒ द्वाद॑श॒ सम् । \newline
13. सम् प॑द्यन्ते पद्यन्ते॒ सꣳ सम् प॑द्यन्ते । \newline
14. प॒द्य॒न्ते॒ द्वाद॑श॒ द्वाद॑श पद्यन्ते पद्यन्ते॒ द्वाद॑श । \newline
15. द्वाद॑श॒ मासा॒ मासा॒ द्वाद॑श॒ द्वाद॑श॒ मासाः᳚ । \newline
16. मासाः᳚ संॅवथ्स॒रः सं॑ॅवथ्स॒रो मासा॒ मासाः᳚ संॅवथ्स॒रः । \newline
17. सं॒ॅव॒थ्स॒रः सं॑ॅवथ्स॒रे सं॑ॅवथ्स॒रे सं॑ॅवथ्स॒रः सं॑ॅवथ्स॒रः सं॑ॅवथ्स॒रे । \newline
18. स॒म्ॅव॒थ्स॒र इति॑ सं - व॒थ्स॒रः । \newline
19. सं॒ॅव॒थ्स॒र ए॒वैव सं॑ॅवथ्स॒रे सं॑ॅवथ्स॒र ए॒व । \newline
20. स॒म्ॅव॒थ्स॒र इति॑ सं - व॒थ्स॒रे । \newline
21. ए॒व प्रति॒ प्रत्ये॒वैव प्रति॑ । \newline
22. प्रति॑ तिष्ठति तिष्ठति॒ प्रति॒ प्रति॑ तिष्ठति । \newline
23. ति॒ष्ठ॒ति॒ यथा॒ यथा॑ तिष्ठति तिष्ठति॒ यथा᳚ । \newline
24. यथा॒ वै वै यथा॒ यथा॒ वै । \newline
25. वै पुरु॑षः॒ पुरु॑षो॒ वै वै पुरु॑षः । \newline
26. पुरु॒षो ऽश्वो ऽश्वः॒ पुरु॑षः॒ पुरु॒षो ऽश्वः॑ । \newline
27. अश्वो॒ गौर् गौरश्वो ऽश्वो॒ गौः । \newline
28. गौर् जीर्य॑ति॒ जीर्य॑ति॒ गौर् गौर् जीर्य॑ति । \newline
29. जीर्य॑त्ये॒व मे॒वम् जीर्य॑ति॒ जीर्य॑त्ये॒वम् । \newline
30. ए॒व म॒ग्निर॒ग्निरे॒व मे॒व म॒ग्निः । \newline
31. अ॒ग्निराहि॑त॒ आहि॑तो॒ ऽग्निर॒ग्निराहि॑तः । \newline
32. आहि॑तो जीर्यति जीर्य॒त्याहि॑त॒ आहि॑तो जीर्यति । \newline
33. आहि॑त॒ इत्या - हि॒तः॒ । \newline
34. जी॒र्य॒ति॒ स॒म्ॅव॒थ्स॒रस्य॑ सम्ॅवथ्स॒रस्य॑ जीर्यति जीर्यति सम्ॅवथ्स॒रस्य॑ । \newline
35. स॒म्ॅव॒थ्स॒रस्य॑ प॒रस्ता᳚त् प॒रस्ता᳚थ् सम्ॅवथ्स॒रस्य॑ सम्ॅवथ्स॒रस्य॑ प॒रस्ता᳚त् । \newline
36. स॒म्ॅव॒थ्स॒रस्येति॑ सं - व॒थ्स॒रस्य॑ । \newline
37. प॒रस्ता॑ दाग्निपावमा॒नीभि॑ राग्निपावमा॒नीभिः॑ प॒रस्ता᳚त् प॒रस्ता॑ दाग्निपावमा॒नीभिः॑ । \newline
38. आ॒ग्नि॒पा॒व॒मा॒नीभि॒ रुपोपा᳚ग्निपावमा॒नीभि॑ राग्निपावमा॒नीभि॒रुप॑ । \newline
39. आ॒ग्नि॒पा॒व॒मा॒नीभि॒रित्या᳚ग्नि - पा॒व॒मा॒नीभिः॑ । \newline
40. उप॑ तिष्ठते तिष्ठत॒ उपोप॑ तिष्ठते । \newline
41. ति॒ष्ठ॒ते॒ पु॒न॒र्न॒वम् पु॑नर्न॒वम् ति॑ष्ठते तिष्ठते पुनर्न॒वम् । \newline
42. पु॒न॒र्न॒व मे॒वैव पु॑नर्न॒वम् पु॑नर्न॒व मे॒व । \newline
43. पु॒न॒र्न॒वमिति॑ पुनः - न॒वम् । \newline
44. ए॒वैन॑ मेन मे॒वैवैन᳚म् । \newline
45. ए॒न॒ म॒जर॑ म॒जर॑ मेन मेन म॒जर᳚म् । \newline
46. अ॒जर॑म् करोति करोत्य॒जर॑ म॒जर॑म् करोति । \newline
47. क॒रो॒त्यथो॒ अथो॑ करोति करो॒त्यथो᳚ । \newline
48. अथो॑ पु॒नाति॑ पु॒नात्यथो॒ अथो॑ पु॒नाति॑ । \newline
49. अथो॒ इत्यथो᳚ । \newline
50. पु॒नात्ये॒वैव पु॒नाति॑ पु॒नात्ये॒व । \newline
51. ए॒वोपोपै॒वैवोप॑ । \newline
52. उप॑ तिष्ठते तिष्ठत॒ उपोप॑ तिष्ठते । \newline
53. ति॒ष्ठ॒ते॒ योगो॒ योग॑स्तिष्ठते तिष्ठते॒ योगः॑ । \newline
54. योग॑ ए॒वैव योगो॒ योग॑ ए॒व । \newline
55. ए॒वास्या᳚स्यै॒वैवास्य॑ । \newline
56. अ॒स्यै॒ष ए॒षो᳚ऽस्यास्यै॒षः । \newline
57. ए॒ष उपोपै॒ष ए॒ष उप॑ । \newline
58. उप॑ तिष्ठते तिष्ठत॒ उपोप॑ तिष्ठते । \newline
59. ति॒ष्ठ॒ते॒ दमो॒ दम॑स्तिष्ठते तिष्ठते॒ दमः॑ । \newline

\textbf{Ghana Paata } \newline

1. ऋ॒तव॑ ऋ॒तुष्व॒ र्‌तुष्व॒ र्‌तव॑ ऋ॒तव॑ ऋ॒तुष्वे॒वैव र्‌तुष्व॒ र्‌तव॑ ऋ॒तव॑ ऋ॒तुष्वे॒व । \newline
2. ऋ॒तुष्वे॒वैव र्‌तुष्व॒ र्‌तुष्वे॒व प्रति॒ प्रत्ये॒व र्‌तुष्व॒ र्‌तुष्वे॒व प्रति॑ । \newline
3. ए॒व प्रति॒ प्रत्ये॒वैव प्रति॑ तिष्ठति तिष्ठति॒ प्रत्ये॒वैव प्रति॑ तिष्ठति । \newline
4. प्रति॑ तिष्ठति तिष्ठति॒ प्रति॒ प्रति॑ तिष्ठति ष॒ड्भि ष्ष॒ड्भि स्ति॑ष्ठति॒ प्रति॒ प्रति॑ तिष्ठति ष॒ड्भिः । \newline
5. ति॒ष्ठ॒ति॒ ष॒ड्भि ष्ष॒ड्भि स्ति॑ष्ठति तिष्ठति ष॒ड्भि रुत्त॑राभि॒ रुत्त॑राभि ष्ष॒ड्भि स्ति॑ष्ठति तिष्ठति ष॒ड्भिरुत्त॑राभिः । \newline
6. ष॒ड्भि रुत्त॑राभि॒ रुत्त॑राभि ष्ष॒ड्भि ष्ष॒ड्भि रुत्त॑राभि॒ रुपोपोत्त॑राभि ष्ष॒ड्भि ष्ष॒ड्भि रुत्त॑राभि॒ रुप॑ । \newline
7. ष॒ड्भिरिति॑ षट् - भिः । \newline
8. उत्त॑राभि॒ रुपोपोत्त॑ राभि॒ रुत्त॑राभि॒ रुप॑ तिष्ठते तिष्ठत॒ उपोत्त॑राभि॒ रुत्त॑राभि॒ रुप॑ तिष्ठते । \newline
9. उत्त॑राभि॒रित्युत् - त॒रा॒भिः॒ । \newline
10. उप॑ तिष्ठते तिष्ठत॒ उपोप॑ तिष्ठते॒ द्वाद॑श॒ द्वाद॑श तिष्ठत॒ उपोप॑ तिष्ठते॒ द्वाद॑श । \newline
11. ति॒ष्ठ॒ते॒ द्वाद॑श॒ द्वाद॑श तिष्ठते तिष्ठते॒ द्वाद॑श॒ सꣳ सम् द्वाद॑श तिष्ठते तिष्ठते॒ द्वाद॑श॒ सम् । \newline
12. द्वाद॑श॒ सꣳ सम् द्वाद॑श॒ द्वाद॑श॒ सम् प॑द्यन्ते पद्यन्ते॒ सम् द्वाद॑श॒ द्वाद॑श॒ सम् प॑द्यन्ते । \newline
13. सम् प॑द्यन्ते पद्यन्ते॒ सꣳ सम् प॑द्यन्ते॒ द्वाद॑श॒ द्वाद॑श पद्यन्ते॒ सꣳ सम् प॑द्यन्ते॒ द्वाद॑श । \newline
14. प॒द्य॒न्ते॒ द्वाद॑श॒ द्वाद॑श पद्यन्ते पद्यन्ते॒ द्वाद॑श॒ मासा॒ मासा॒ द्वाद॑श पद्यन्ते पद्यन्ते॒ द्वाद॑श॒ मासाः᳚ । \newline
15. द्वाद॑श॒ मासा॒ मासा॒ द्वाद॑श॒ द्वाद॑श॒ मासाः᳚ संॅवथ्स॒रः सं॑ॅवथ्स॒रो मासा॒ द्वाद॑श॒ द्वाद॑श॒ मासाः᳚ संॅवथ्स॒रः । \newline
16. मासाः᳚ संॅवथ्स॒रः सं॑ॅवथ्स॒रो मासा॒ मासाः᳚ संॅवथ्स॒रः सं॑ॅवथ्स॒रे सं॑ॅवथ्स॒रे सं॑ॅवथ्स॒रो मासा॒ मासाः᳚ संॅवथ्स॒रः सं॑ॅवथ्स॒रे । \newline
17. सं॒ॅव॒थ्स॒रः सं॑ॅवथ्स॒रे सं॑ॅवथ्स॒रे सं॑ॅवथ्स॒रः सं॑ॅवथ्स॒रः सं॑ॅवथ्स॒र ए॒वैव सं॑ॅवथ्स॒रे सं॑ॅवथ्स॒रः सं॑ॅवथ्स॒रः सं॑ॅवथ्स॒र ए॒व । \newline
18. स॒म्ॅव॒थ्स॒र इति॑ सं - व॒थ्स॒रः । \newline
19. सं॒ॅव॒थ्स॒र ए॒वैव सं॑ॅवथ्स॒रे सं॑ॅवथ्स॒र ए॒व प्रति॒ प्रत्ये॒व सं॑ॅवथ्स॒रे सं॑ॅवथ्स॒र ए॒व प्रति॑ । \newline
20. स॒म्ॅव॒थ्स॒र इति॑ सं - व॒थ्स॒रे । \newline
21. ए॒व प्रति॒ प्रत्ये॒वैव प्रति॑ तिष्ठति तिष्ठति॒ प्रत्ये॒वैव प्रति॑ तिष्ठति । \newline
22. प्रति॑ तिष्ठति तिष्ठति॒ प्रति॒ प्रति॑ तिष्ठति॒ यथा॒ यथा॑ तिष्ठति॒ प्रति॒ प्रति॑ तिष्ठति॒ यथा᳚ । \newline
23. ति॒ष्ठ॒ति॒ यथा॒ यथा॑ तिष्ठति तिष्ठति॒ यथा॒ वै वै यथा॑ तिष्ठति तिष्ठति॒ यथा॒ वै । \newline
24. यथा॒ वै वै यथा॒ यथा॒ वै पुरु॑षः॒ पुरु॑षो॒ वै यथा॒ यथा॒ वै पुरु॑षः । \newline
25. वै पुरु॑षः॒ पुरु॑षो॒ वै वै पुरु॒षो ऽश्वो ऽश्वः॒ पुरु॑षो॒ वै वै पुरु॒षो ऽश्वः॑ । \newline
26. पुरु॒षो ऽश्वो ऽश्वः॒ पुरु॑षः॒ पुरु॒षो ऽश्वो॒ गौर् गौरश्वः॒ पुरु॑षः॒ पुरु॒षो ऽश्वो॒ गौः । \newline
27. अश्वो॒ गौर् गौरश्वो ऽश्वो॒ गौर् जीर्य॑ति॒ जीर्य॑ति॒ गौरश्वो ऽश्वो॒ गौर् जीर्य॑ति । \newline
28. गौर् जीर्य॑ति॒ जीर्य॑ति॒ गौर् गौर् जीर्य॑त्ये॒व मे॒वम् जीर्य॑ति॒ गौर् गौर् जीर्य॑त्ये॒वम् । \newline
29. जीर्य॑त्ये॒व मे॒वम् जीर्य॑ति॒ जीर्य॑त्ये॒व म॒ग्नि र॒ग्निरे॒वम् जीर्य॑ति॒ जीर्य॑त्ये॒व म॒ग्निः । \newline
30. ए॒व म॒ग्निर॒ग्निरे॒व मे॒व म॒ग्निराहि॑त॒ आहि॑तो॒ ऽग्निरे॒व मे॒व म॒ग्निराहि॑तः । \newline
31. अ॒ग्निराहि॑त॒ आहि॑तो॒ ऽग्निर॒ग्निराहि॑तो जीर्यति जीर्य॒त्याहि॑तो॒ ऽग्निर॒ग्निराहि॑तो जीर्यति । \newline
32. आहि॑तो जीर्यति जीर्य॒त्याहि॑त॒ आहि॑तो जीर्यति सम्ॅवथ्स॒रस्य॑ सम्ॅवथ्स॒रस्य॑ जीर्य॒त्याहि॑त॒ आहि॑तो जीर्यति सम्ॅवथ्स॒रस्य॑ । \newline
33. आहि॑त॒ इत्या - हि॒तः॒ । \newline
34. जी॒र्य॒ति॒ स॒म्ॅव॒थ्स॒रस्य॒ स॒म्ॅव॒थ्स॒रस्य॑ जीर्यति जीर्यति सम्ॅवथ्स॒रस्य॑ प॒रस्ता᳚त् 
प॒रस्ता᳚थ् सम्ॅवथ्स॒रस्य॑ जीर्यति जीर्यति सम्ॅवथ्स॒रस्य॑ प॒रस्ता᳚त् । \newline
35. स॒म्ॅव॒थ्स॒रस्य॑ प॒रस्ता᳚त् प॒रस्ता᳚थ् सम्ॅवथ्स॒रस्य॑ सम्ॅवथ्स॒रस्य॑ प॒रस्ता॑ दाग्निपावमा॒नीभि॑ राग्निपावमा॒नीभिः॑ प॒रस्ता᳚थ् सम्ॅवथ्स॒रस्य॑ सम्ॅवथ्स॒रस्य॑ प॒रस्ता॑ दाग्निपावमा॒नीभिः॑ । \newline
36. स॒म्ॅव॒थ्स॒रस्येति॑ सं - व॒थ्स॒रस्य॑ । \newline
37. प॒रस्ता॑ दाग्निपावमा॒नीभि॑ राग्निपावमा॒नीभिः॑ प॒रस्ता᳚त् प॒रस्ता॑ दाग्निपावमा॒नीभि॒ रुपोपा᳚ग्निपावमा॒नीभिः॑ प॒रस्ता᳚त् प॒रस्ता॑ दाग्निपावमा॒नीभि॒ रुप॑ । \newline
38. आ॒ग्नि॒पा॒व॒मा॒नीभि॒ रुपोपा᳚ग्निपावमा॒नीभि॑ राग्निपावमा॒नीभि॒ रुप॑ तिष्ठते तिष्ठत॒ उपा᳚ग्निपावमा॒नीभि॑ राग्निपावमा॒नीभि॒ रुप॑ तिष्ठते । \newline
39. आ॒ग्नि॒पा॒व॒मा॒नीभि॒रित्या᳚ग्नि - पा॒व॒मा॒नीभिः॑ । \newline
40. उप॑ तिष्ठते तिष्ठत॒ उपोप॑ तिष्ठते पुनर्न॒वम् पु॑नर्न॒वम् ति॑ष्ठत॒ उपोप॑ तिष्ठते पुनर्न॒वम् । \newline
41. ति॒ष्ठ॒ते॒ पु॒न॒र्न॒वम् पु॑नर्न॒वम् ति॑ष्ठते तिष्ठते पुनर्न॒व मे॒वैव पु॑नर्न॒वम् ति॑ष्ठते तिष्ठते पुनर्न॒व मे॒व । \newline
42. पु॒न॒र्न॒व मे॒वैव पु॑नर्न॒वम् पु॑नर्न॒व मे॒वैन॑ मेन मे॒व पु॑नर्न॒वम् पु॑नर्न॒व मे॒वैन᳚म् । \newline
43. पु॒न॒र्न॒वमिति॑ पुनः - न॒वम् । \newline
44. ए॒वैन॑ मेन मे॒वैवैन॑ म॒जर॑ म॒जर॑ मेन मे॒वैवैन॑ म॒जर᳚म् । \newline
45. ए॒न॒ म॒जर॑ म॒जर॑ मेन मेन म॒जर॑म् करोति करोत्य॒जर॑ मेन मेन म॒जर॑म् करोति । \newline
46. अ॒जर॑म् करोति करोत्य॒जर॑ म॒जर॑म् करो॒त्यथो॒ अथो॑ करोत्य॒जर॑ म॒जर॑म् करो॒त्यथो᳚ । \newline
47. क॒रो॒त्यथो॒ अथो॑ करोति करो॒त्यथो॑ पु॒नाति॑ पु॒नात्यथो॑ करोति करो॒त्यथो॑ पु॒नाति॑ । \newline
48. अथो॑ पु॒नाति॑ पु॒नात्यथो॒ अथो॑ पु॒नात्ये॒वैव पु॒नात्यथो॒ अथो॑ पु॒नात्ये॒व । \newline
49. अथो॒ इत्यथो᳚ । \newline
50. पु॒नात्ये॒वैव पु॒नाति॑ पु॒नात्ये॒वोपोपै॒व पु॒नाति॑ पु॒नात्ये॒वोप॑ । \newline
51. ए॒वोपोपै॒वैवोप॑ तिष्ठते तिष्ठत॒ उपै॒वैवोप॑ तिष्ठते । \newline
52. उप॑ तिष्ठते तिष्ठत॒ उपोप॑ तिष्ठते॒ योगो॒ योग॑स्तिष्ठत॒ उपोप॑ तिष्ठते॒ योगः॑ । \newline
53. ति॒ष्ठ॒ते॒ योगो॒ योग॑स्तिष्ठते तिष्ठते॒ योग॑ ए॒वैव योग॑स्तिष्ठते तिष्ठते॒ योग॑ ए॒व । \newline
54. योग॑ ए॒वैव योगो॒ योग॑ ए॒वास्या᳚स्यै॒व योगो॒ योग॑ ए॒वास्य॑ । \newline
55. ए॒वास्या᳚स्यै॒वैवास्यै॒ष ए॒षो᳚ ऽस्यै॒वैवास्यै॒षः । \newline
56. अ॒स्यै॒ष ए॒षो᳚ ऽस्यास्यै॒ष उपोपै॒षो᳚ ऽस्यास्यै॒ष उप॑ । \newline
57. ए॒ष उपोपै॒ष ए॒ष उप॑ तिष्ठते तिष्ठत॒ उपै॒ष ए॒ष उप॑ तिष्ठते । \newline
58. उप॑ तिष्ठते तिष्ठत॒ उपोप॑ तिष्ठते॒ दमो॒ दम॑स्तिष्ठत॒ उपोप॑ तिष्ठते॒ दमः॑ । \newline
59. ति॒ष्ठ॒ते॒ दमो॒ दम॑स्तिष्ठते तिष्ठते॒ दम॑ ए॒वैव दम॑स्तिष्ठते तिष्ठते॒ दम॑ ए॒व । \newline
\pagebreak
\markright{ TS 1.5.7.4  \hfill https://www.vedavms.in \hfill}
\addcontentsline{toc}{section}{ TS 1.5.7.4 }
\section*{ TS 1.5.7.4 }

\textbf{TS 1.5.7.4 } \newline
\textbf{Samhita Paata} \newline

दम॑ ए॒वास्यै॒ष उप॑ तिष्ठते याच्ञैवास्यै॒षोप॑ तिष्ठते॒ यथा॒ पापी॑या॒ञ्छ्रेय॑स आ॒हृत्य॑ नम॒स्यति॑ ता॒दृगे॒व तदा॑यु॒र्दा अ॑ग्ने॒ऽस्यायु॑र्मे दे॒हीत्या॑हा*ऽऽयु॒र्दा ह्ये॑ष व॑र्चो॒दा अ॑ग्नेऽसि॒ वर्चो॑ मे दे॒हीत्या॑ह वर्चो॒दा ह्ये॑ष त॑नू॒पा अ॑ग्नेऽसि त॒नुवं॑ मे पा॒हीत्या॑ह - [ ] \newline

\textbf{Pada Paata} \newline

दमः॑ । ए॒व । अ॒स्य॒ । ए॒षः । उपेति॑ । ति॒ष्ठ॒ते॒ । याच्ञै । ए॒व । अ॒स्य॒ । ए॒षा । उपेति॑ । ति॒ष्ठ॒ते॒ । यथा᳚ । पापी॑यान् । श्रेय॑से । आ॒हृत्येत्या᳚ - हृत्य॑ । न॒म॒स्यति॑ । ता॒दृक् । ए॒व । तत् । आ॒यु॒र्दा इत्या॑युः - दाः । अ॒ग्ने॒ । अ॒सि॒ । आयुः॑ । मे॒ । दे॒हि॒ । इति॑ । आ॒ह॒ । आ॒यु॒र्दा इत्या॑युः - दाः । हि । ए॒षः । व॒र्चो॒दा इति॑ वर्चः - दाः । अ॒ग्ने॒ । अ॒सि॒ । वर्चः॑ । मे॒ । दे॒हि॒ । इति॑ । आ॒ह॒ । व॒र्चो॒दा इति॑ वर्चः - दाः । हि । ए॒षः । त॒नू॒पा इति॑ तनू - पाः । अ॒ग्ने॒ । अ॒सि॒ । त॒नुव᳚म् । मे॒ । पा॒हि॒ । इति॑ । आ॒ह॒ ।  \newline


\textbf{Krama Paata} \newline

दम॑ ए॒व । ए॒वास्य॑ । अ॒स्यै॒षः । ए॒ष उप॑ । उप॑ तिष्ठते । ति॒ष्ठ॒ते॒ या॒च्ञा । या॒च्ञैव । ए॒वास्य॑ । अ॒स्यै॒षा । ए॒षोप॑ । उप॑ तिष्ठते । ति॒ष्ठ॒ते॒ यथा᳚ । यथा॒ पापी॑यान् । पापी॑या॒ञ्छ्रेय॑से । श्रेय॑स आ॒हृत्य॑ । आ॒हृत्य॑ नम॒स्यति॑ । आ॒हृत्येत्या᳚ - हृत्य॑ । न॒म॒स्यति॑ ता॒दृक् । ता॒दृगे॒व । ए॒व तत् । तदा॑यु॒र्दाः । आ॒यु॒र्दा अ॑ग्ने । आ॒यु॒र्दा इत्या॑युः - दाः । अ॒ग्ने॒ऽसि॒ । अ॒स्यायुः॑ । आयु॑र्मे । मे॒ दे॒हि॒ । दे॒हीति॑ । इत्या॑ह । आ॒हा॒यु॒र्दाः । आ॒यु॒र्दा हि । आ॒यु॒र्दा इत्या॑युः - दाः । ह्ये॑षः । ए॒ष व॑र्चो॒दाः । व॒र्चो॒दा अ॑ग्ने । व॒र्चो॒दा इति॑ वर्चः - दाः । अ॒ग्ने॒ऽसि॒ । अ॒सि॒ वर्चः॑ । वर्चो॑ मे । मे॒ दे॒हि॒ । दे॒हीति॑ । इत्या॑ह । आ॒ह॒ व॒र्चो॒दाः । व॒र्चो॒दा हि । व॒र्चो॒दा इति॑ वर्चः - दाः । ह्ये॑षः । ए॒ष त॑नू॒पाः । त॒नू॒पा अ॑ग्ने । त॒नू॒पा इति॑ तनू - पाः । अ॒ग्ने॒ऽसि॒ । अ॒सि॒ त॒नुव᳚म् । त॒नुव॑म् मे । मे॒ पा॒हि॒ । पा॒हीति॑ । इत्या॑ह । आ॒ह॒ त॒नू॒पाः \newline

\textbf{Jatai Paata} \newline

1. दम॑ ए॒वैव दमो॒ दम॑ ए॒व । \newline
2. ए॒वास्या᳚स्यै॒वैवास्य॑ । \newline
3. अ॒स्यै॒ष ए॒षो᳚ऽस्यास्यै॒षः । \newline
4. ए॒ष उपोपै॒ष ए॒ष उप॑ । \newline
5. उप॑ तिष्ठते तिष्ठत॒ उपोप॑ तिष्ठते । \newline
6. ति॒ष्ठ॒ते॒ या॒च्ञा या॒च्ञा ति॑ष्ठते तिष्ठते या॒च्ञा । \newline
7. या॒च्ञैवैव या॒च्ञा या॒च्ञैव । \newline
8. ए॒वास्या᳚स्यै॒वैवास्य॑ । \newline
9. अ॒स्यै॒षैषा ऽस्या᳚स्यै॒षा । \newline
10. ए॒षोपोपै॒षैषोप॑ । \newline
11. उप॑ तिष्ठते तिष्ठत॒ उपोप॑ तिष्ठते । \newline
12. ति॒ष्ठ॒ते॒ यथा॒ यथा॑ तिष्ठते तिष्ठते॒ यथा᳚ । \newline
13. यथा॒ पापी॑या॒न् पापी॑या॒न्॒. यथा॒ यथा॒ पापी॑यान् । \newline
14. पापी॑या॒ञ् छ्रेय॑से॒ श्रेय॑से॒ पापी॑या॒न् पापी॑या॒ञ् छ्रेय॑से । \newline
15. श्रेय॑स आ॒हृत्या॒हृत्य॒ श्रेय॑से॒ श्रेय॑स आ॒हृत्य॑ । \newline
16. आ॒हृत्य॑ नम॒स्यति॑ नम॒स्यत्या॒हृत्या॒हृत्य॑ नम॒स्यति॑ । \newline
17. आ॒हृत्येत्या᳚ - हृत्य॑ । \newline
18. न॒म॒स्यति॑ ता॒दृक् ता॒दृङ् न॑म॒स्यति॑ नम॒स्यति॑ ता॒दृक् । \newline
19. ता॒दृगे॒वैव ता॒दृक् ता॒दृगे॒व । \newline
20. ए॒व तत् तदे॒वैव तत् । \newline
21. तदा॑यु॒र्दा आ॑यु॒र्दास्तत् तदा॑यु॒र्दाः । \newline
22. आ॒यु॒र्दा अ॑ग्ने अग्न आयु॒र्दा आ॑यु॒र्दा अ॑ग्ने । \newline
23. आ॒यु॒र्दा इत्या॑युः - दाः । \newline
24. अ॒ग्ने॒ ऽस्य॒स्य॒ग्ने॒ अ॒ग्ने॒ ऽसि॒ । \newline
25. अ॒स्यायु॒ रायु॑ रस्य॒स्यायुः॑ । \newline
26. आयु॑र् मे म॒ आयु॒रायु॑र् मे । \newline
27. मे॒ दे॒हि॒ दे॒हि॒ मे॒ मे॒ दे॒हि॒ । \newline
28. दे॒हीतीति॑ देहि दे॒हीति॑ । \newline
29. इत्या॑हा॒हे तीत्या॑ह । \newline
30. आ॒हा॒यु॒र्दा आ॑यु॒र्दा आ॑हाहायु॒र्दाः । \newline
31. आ॒यु॒र्दा हि ह्या॑यु॒र्दा आ॑यु॒र्दा हि । \newline
32. आ॒यु॒र्दा इत्या॑युः - दाः । \newline
33. ह्ये॑ष ए॒ष हि ह्ये॑षः । \newline
34. ए॒ष व॑र्चो॒दा व॑र्चो॒दा ए॒ष ए॒ष व॑र्चो॒दाः । \newline
35. व॒र्चो॒दा अ॑ग्ने अग्ने वर्चो॒दा व॑र्चो॒दा अ॑ग्ने । \newline
36. व॒र्चो॒दा इति॑ वर्चः - दाः । \newline
37. अ॒ग्ने॒ ऽस्य॒स्य॒ग्ने॒ अ॒ग्ने॒ ऽसि॒ । \newline
38. अ॒सि॒ वर्चो॒ वर्चो᳚ ऽस्यसि॒ वर्चः॑ । \newline
39. वर्चो॑ मे मे॒ वर्चो॒ वर्चो॑ मे । \newline
40. मे॒ दे॒हि॒ दे॒हि॒ मे॒ मे॒ दे॒हि॒ । \newline
41. दे॒हीतीति॑ देहि दे॒हीति॑ । \newline
42. इत्या॑हा॒हे तीत्या॑ह । \newline
43. आ॒ह॒ व॒र्चो॒दा व॑र्चो॒दा आ॑हाह वर्चो॒दाः । \newline
44. व॒र्चो॒दा हि हि व॑र्चो॒दा व॑र्चो॒दा हि । \newline
45. व॒र्चो॒दा इति॑ वर्चः - दाः । \newline
46. ह्ये॑ष ए॒ष हि ह्ये॑षः । \newline
47. ए॒ष त॑नू॒पा स्त॑नू॒पा ए॒ष ए॒ष त॑नू॒पाः । \newline
48. त॒नू॒पा अ॑ग्ने अग्ने तनू॒पा स्त॑नू॒पा अ॑ग्ने । \newline
49. त॒नू॒पा इति॑ तनू - पाः । \newline
50. अ॒ग्ने॒ ऽस्य॒स्य॒ग्ने॒ अ॒ग्ने॒ ऽसि॒ । \newline
51. अ॒सि॒ त॒नुव॑म् त॒नुव॑ मस्यसि त॒नुव᳚म् । \newline
52. त॒नुव॑म् मे मे त॒नुव॑म् त॒नुव॑म् मे । \newline
53. मे॒ पा॒हि॒ पा॒हि॒ मे॒ मे॒ पा॒हि॒ । \newline
54. पा॒हीतीति॑ पाहि पा॒हीति॑ । \newline
55. इत्या॑हा॒हे तीत्या॑ह । \newline
56. आ॒ह॒ त॒नू॒पा स्त॑नू॒पा आ॑हाह तनू॒पाः । \newline

\textbf{Ghana Paata } \newline

1. दम॑ ए॒वैव दमो॒ दम॑ ए॒वास्या᳚स्यै॒व दमो॒ दम॑ ए॒वास्य॑ । \newline
2. ए॒वास्या᳚स्यै॒वैवास्यै॒ष ए॒षो᳚ स्यै॒वैवास्यै॒षः । \newline
3. अ॒स्यै॒ष ए॒षो᳚ ऽयास्यै॒ष उपोपै॒षो᳚ ऽस्यास्यै॒ष उप॑ । \newline
4. ए॒ष उपोपै॒ष ए॒ष उप॑ तिष्ठते तिष्ठत॒ उपै॒ष ए॒ष उप॑ तिष्ठते । \newline
5. उप॑ तिष्ठते तिष्ठत॒ उपोप॑ तिष्ठते या॒च्ञा या॒च्ञा ति॑ष्ठत॒ उपोप॑ तिष्ठते या॒च्ञा । \newline
6. ति॒ष्ठ॒ते॒ या॒च्ञा या॒च्ञा ति॑ष्ठते तिष्ठते या॒च्ञैवैव या॒च्ञा ति॑ष्ठते तिष्ठते या॒च्ञैव । \newline
7. या॒च्ञैवैव या॒च्ञा या॒च्ञैवास्या᳚स्यै॒व या॒च्ञा या॒च्ञैवास्य॑ । \newline
8. ए॒वास्या᳚स्यै॒वैवास्यै॒षैषा ऽस्यै॒वैवास्यै॒षा । \newline
9. अ॒स्यै॒षैषा ऽस्या᳚स्यै॒षोपोपै॒षा ऽस्या᳚स्यै॒षोप॑ । \newline
10. ए॒षोपोपै॒षैषोप॑ तिष्ठते तिष्ठत॒ उपै॒षैषोप॑ तिष्ठते । \newline
11. उप॑ तिष्ठते तिष्ठत॒ उपोप॑ तिष्ठते॒ यथा॒ यथा॑ तिष्ठत॒ उपोप॑ तिष्ठते॒ यथा᳚ । \newline
12. ति॒ष्ठ॒ते॒ यथा॒ यथा॑ तिष्ठते तिष्ठते॒ यथा॒ पापी॑या॒न् पापी॑या॒न्॒. यथा॑ तिष्ठते तिष्ठते॒ यथा॒ पापी॑यान् । \newline
13. यथा॒ पापी॑या॒न् पापी॑या॒न्॒. यथा॒ यथा॒ पापी॑या॒ञ् छ्रेय॑से॒ श्रेय॑से॒ पापी॑या॒न्.॒ यथा॒ यथा॒ पापी॑या॒ञ् छेय॑से । \newline
14. पापी॑या॒~न् छ्रेय॑से॒ श्रेय॑से॒ पापी॑या॒न् पापी॑या॒ञ् छ्रेय॑स आ॒हृत्या॒हृत्य॒ श्रेय॑से॒ पापी॑या॒न् पापी॑या॒ञ् छ्रेय॑स आ॒हृत्य॑ । \newline
15. श्रेय॑स आ॒हृत्या॒हृत्य॒ श्रेय॑से॒ श्रेय॑स आ॒हृत्य॑ नम॒स्यति॑ नम॒स्यत्या॒हृत्य॒ श्रेय॑से॒ श्रेय॑स आ॒हृत्य॑ नम॒स्यति॑ । \newline
16. आ॒हृत्य॑ नम॒स्यति॑ नम॒स्यत्या॒हृत्या॒हृत्य॑ नम॒स्यति॑ ता॒दृक् ता॒दृङ् न॑म॒स्यत्या॒हृत्या॒हृत्य॑ नम॒स्यति॑ ता॒दृक् । \newline
17. आ॒हृत्येत्या᳚ - हृत्य॑ । \newline
18. न॒म॒स्यति॑ ता॒दृक् ता॒दृङ् न॑म॒स्यति॑ नम॒स्यति॑ ता॒दृगे॒वैव ता॒दृङ् न॑म॒स्यति॑ नम॒स्यति॑ ता॒दृगे॒व । \newline
19. ता॒दृगे॒वैव ता॒दृक् ता॒दृगे॒व तत् तदे॒व ता॒दृक् ता॒दृगे॒व तत् । \newline
20. ए॒व तत् तदे॒वैव तदा॑यु॒र्दा आ॑यु॒र्दा स्तदे॒वैव तदा॑यु॒र्दाः । \newline
21. तदा॑यु॒र्दा आ॑यु॒र्दास्तत् तदा॑यु॒र्दा अ॑ग्ने अग्न आयु॒र्दास्तत् तदा॑यु॒र्दा अ॑ग्ने । \newline
22. आ॒यु॒र्दा अ॑ग्ने अग्न आयु॒र्दा आ॑यु॒र्दा अ॑ग्ने ऽस्यस्यग्न आयु॒र्दा आ॑यु॒र्दा अ॑ग्ने ऽसि । \newline
23. आ॒यु॒र्दा इत्या॑युः - दाः । \newline
24. अ॒ग्ने॒ ऽस्य॒स्य॒ग्ने॒ अ॒ग्ने॒ ऽस्यायु॒ रायु॑रस्यग्ने अग्ने॒ ऽस्यायुः॑ । \newline
25. अ॒स्यायु॒ रायु॑रस्य॒स्यायु॑र् मे म॒ आयु॑ रस्य॒स्यायु॑र् मे । \newline
26. आयु॑र् मे म॒ आयु॒रायु॑र् मे देहि देहि म॒ आयु॒रायु॑र् मे देहि । \newline
27. मे॒ दे॒हि॒ दे॒हि॒ मे॒ मे॒ दे॒हीतीति॑ देहि मे मे दे॒हीति॑ । \newline
28. दे॒हीतीति॑ देहि दे॒हीत्या॑हा॒हे ति॑ देहि दे॒हीत्या॑ह । \newline
29. इत्या॑हा॒हे तीत्या॑हायु॒र्दा आ॑यु॒र्दा आ॒हे तीत्या॑हायु॒र्दाः । \newline
30. आ॒हा॒यु॒र्दा आ॑यु॒र्दा आ॑हाहायु॒र्दा हि ह्या॑यु॒र्दा आ॑हाहायु॒र्दा हि । \newline
31. आ॒यु॒र्दा हि ह्या॑यु॒र्दा आ॑यु॒र्दा ह्ये॑ष ए॒ष ह्या॑यु॒र्दा आ॑यु॒र्दा ह्ये॑षः । \newline
32. आ॒यु॒र्दा इत्या॑युः - दाः । \newline
33. ह्ये॑ष ए॒ष हि ह्ये॑ष व॑र्चो॒दा व॑र्चो॒दा ए॒ष हि ह्ये॑ष व॑र्चो॒दाः । \newline
34. ए॒ष व॑र्चो॒दा व॑र्चो॒दा ए॒ष ए॒ष व॑र्चो॒दा अ॑ग्ने अग्ने वर्चो॒दा ए॒ष ए॒ष व॑र्चो॒दा अ॑ग्ने । \newline
35. व॒र्चो॒दा अ॑ग्ने अग्ने वर्चो॒दा व॑र्चो॒दा अ॑ग्ने ऽस्यस्यग्ने वर्चो॒दा व॑र्चो॒दा अ॑ग्ने ऽसि । \newline
36. व॒र्चो॒दा इति॑ वर्चः - दाः । \newline
37. अ॒ग्ने॒ ऽस्य॒स्य॒ग्ने॒ अ॒ग्ने॒ ऽसि॒ वर्चो॒ वर्चो᳚ ऽस्यग्ने अग्ने ऽसि॒ वर्चः॑ । \newline
38. अ॒सि॒ वर्चो॒ वर्चो᳚ ऽस्यसि॒ वर्चो॑ मे मे॒ वर्चो᳚ ऽस्यसि॒ वर्चो॑ मे । \newline
39. वर्चो॑ मे मे॒ वर्चो॒ वर्चो॑ मे देहि देहि मे॒ वर्चो॒ वर्चो॑ मे देहि । \newline
40. मे॒ दे॒हि॒ दे॒हि॒ मे॒ मे॒ दे॒हीतीति॑ देहि मे मे दे॒हीति॑ । \newline
41. दे॒हीतीति॑ देहि दे॒हीत्या॑हा॒हे ति॑ देहि दे॒हीत्या॑ह । \newline
42. इत्या॑हा॒हे तीत्या॑ह वर्चो॒दा व॑र्चो॒दा आ॒हे तीत्या॑ह वर्चो॒दाः । \newline
43. आ॒ह॒ व॒र्चो॒दा व॑र्चो॒दा आ॑हाह वर्चो॒दा हि हि व॑र्चो॒दा आ॑हाह वर्चो॒दा हि । \newline
44. व॒र्चो॒दा हि हि व॑र्चो॒दा व॑र्चो॒दा ह्ये॑ष ए॒ष हि व॑र्चो॒दा व॑र्चो॒दा ह्ये॑षः । \newline
45. व॒र्चो॒दा इति॑ वर्चः - दाः । \newline
46. ह्ये॑ष ए॒ष हि ह्ये॑ष त॑नू॒पा स्त॑नू॒पा ए॒ष हि ह्ये॑ष त॑नू॒पाः । \newline
47. ए॒ष त॑नू॒पा स्त॑नू॒पा ए॒ष ए॒ष त॑नू॒पा अ॑ग्ने अग्ने तनू॒पा ए॒ष ए॒ष त॑नू॒पा अ॑ग्ने । \newline
48. त॒नू॒पा अ॑ग्ने अग्ने तनू॒पा स्त॑नू॒पा अ॑ग्ने ऽस्यस्यग्ने तनू॒पा स्त॑नू॒पा अ॑ग्ने ऽसि । \newline
49. त॒नू॒पा इति॑ तनू - पाः । \newline
50. अ॒ग्ने॒ ऽस्य॒स्य॒ग्ने॒ अ॒ग्ने॒ ऽसि॒ त॒नुव॑म् त॒नुव॑ मस्यग्ने अग्ने ऽसि त॒नुव᳚म् । \newline
51. अ॒सि॒ त॒नुव॑म् त॒नुव॑ मस्यसि त॒नुव॑म् मे मे त॒नुव॑ मस्यसि त॒नुव॑म् मे । \newline
52. त॒नुव॑म् मे मे त॒नुव॑म् त॒नुव॑म् मे पाहि पाहि मे त॒नुव॑म् त॒नुव॑म् मे पाहि । \newline
53. मे॒ पा॒हि॒ पा॒हि॒ मे॒ मे॒ पा॒हीतीति॑ पाहि मे मे पा॒हीति॑ । \newline
54. पा॒हीतीति॑ पाहि पा॒हीत्या॑हा॒हे ति॑ पाहि पा॒हीत्या॑ह । \newline
55. इत्या॑हा॒हे तीत्या॑ह तनू॒पा स्त॑नू॒पा आ॒हे तीत्या॑ह तनू॒पाः । \newline
56. आ॒ह॒ त॒नू॒पा स्त॑नू॒पा आ॑हाह तनू॒पा हि हि त॑नू॒पा आ॑हाह तनू॒पा हि । \newline
\pagebreak
\markright{ TS 1.5.7.5  \hfill https://www.vedavms.in \hfill}
\addcontentsline{toc}{section}{ TS 1.5.7.5 }
\section*{ TS 1.5.7.5 }

\textbf{TS 1.5.7.5 } \newline
\textbf{Samhita Paata} \newline

तनू॒पा ह्ये॑षोऽग्ने॒ यन्मे॑ त॒नुवा॑ ऊ॒नं तन्म॒ आ पृ॒णेत्या॑ह॒ यन्मे᳚ प्र॒जायै॑ पशू॒नामू॒नं तन्म॒ आ पू॑र॒येति॒ वावैतदा॑ह॒ चित्रा॑वसो स्व॒स्ति ते॑ पा॒रम॑शी॒येत्या॑ह॒ रात्रि॒र् वै चि॒त्राव॑सु॒रव्यु॑ष्ट्यै॒ वा ए॒तस्यै॑ पु॒रा ब्रा᳚ह्म॒णा अ॑भैषु॒र् व्यु॑ष्टिमे॒वाव॑ रुन्ध॒ इन्धा॑नास्त्वा श॒तꣳ - [ ] \newline

\textbf{Pada Paata} \newline

त॒नू॒पा इति॑ तनू - पाः । हि । ए॒षः । अग्ने᳚ । यत् । मे॒ । त॒नुवाः᳚ । ऊ॒नम् । तत् । मे॒ । एति॑ । पृ॒ण॒ । इति॑ । आ॒ह॒ । यत् । मे॒ । प्र॒जाया॒ इति॑ प्र - जायै᳚ । प॒शू॒नाम् । ऊ॒नम् । तत् । मे॒ । एति॑ । पू॒र॒य॒ । इति॑ । वाव । ए॒तत् । आ॒ह॒ । चित्रा॑वसो॒ इति॒ चित्र॑ - व॒सो॒ । स्व॒स्ति । ते॒ । पा॒रम् । अ॒शी॒य॒ । इति॑ । आ॒ह॒ । रात्रिः॑ । वै । चि॒त्राव॑सु॒रिति॑ चि॒त्र - व॒सुः॒ । अव्यु॑ष्ट्या॒ इत्यवि॑ - उ॒ष्ट्यै॒ । वै । ए॒तस्यै᳚ । पु॒रा । ब्रा॒ह्म॒णाः । अ॒भै॒षुः॒ । व्यु॑ष्टि॒मिति॒ वि - उ॒ष्टि॒म् । ए॒व । अवेति॑ । रु॒न्धे॒ । इन्धा॑नाः । त्वा॒ । श॒तम् ।  \newline


\textbf{Krama Paata} \newline

त॒नू॒पा हि । त॒नू॒पा इति॑ तनू - पाः । ह्ये॑षः । ए॒षोऽग्ने᳚ । अग्ने॒ यत् । यन्मे᳚ । मे॒ त॒नुवाः᳚ । त॒नुवा॑ ऊ॒नम् । ऊ॒नम् तत् । तन्मे᳚ । म॒ आ । आ पृ॑ण । पृ॒णेति॑ । इत्या॑ह । आ॒ह॒ यत् । यन्मे᳚ । मे॒ प्र॒जायै᳚ । प्र॒जायै॑ पशू॒नाम् । प्र॒जाया॒ इति॑ प्र - जायै᳚ । प॒शू॒नामू॒नम् । ऊ॒नम् तत् । तन्मे᳚ । म॒ आ । आ पू॑रय । पू॒र॒येति॑ । इति॒ वाव । वावैतत् । ए॒तदा॑ह । आ॒ह॒ चित्रा॑वसो । चित्रा॑वसो स्व॒स्ति । चित्रा॑वसो॒ इति॒ चित्र॑ - व॒सो॒ । स्व॒स्ति ते᳚ । ते॒ पा॒रम् । पा॒रम॑शीय । अ॒शी॒येति॑ । इत्या॑ह । आ॒ह॒ रात्रिः॑ । रात्रि॒र् वै । वै चि॒त्राव॑सुः । चि॒त्राव॑सु॒रव्यु॑ष्ट्यै । चि॒त्राव॑सु॒रिति॑ चि॒त्र - व॒सुः॒ । अव्यु॑ष्ट्यै॒ वै । अव्यु॑ष्ट्या॒ इत्यवि॑ - उ॒ष्ट्यै॒ । वा ए॒तस्यै᳚ । ए॒तस्यै॑ पु॒रा । पु॒रा ब्रा᳚ह्म॒णाः । ब्रा॒ह्म॒णा अ॑भैषुः । अ॒भै॒षु॒र् व्यु॑ष्टिम् । व्यु॑ष्टिमे॒व । व्यु॑ष्टि॒मिति॒ वि - उ॒ष्टि॒म् । ए॒वाव॑ । अव॑ रुन्धे । रु॒न्ध॒ इन्धा॑नाः । इन्धा॑नास्त्वा । त्वा॒ श॒तम् । श॒तꣳ हिमाः᳚ \newline

\textbf{Jatai Paata} \newline

1. त॒नू॒पा हि हि त॑नू॒पा स्त॑नू॒पा हि । \newline
2. त॒नू॒पा इति॑ तनू - पाः । \newline
3. ह्ये॑ष ए॒ष हि ह्ये॑षः । \newline
4. ए॒षोऽग्ने ऽग्न॑ ए॒ष ए॒षोऽग्ने᳚ । \newline
5. अग्ने॒ यद् यदग्ने ऽग्ने॒ यत् । \newline
6. यन् मे॑ मे॒ यद् यन् मे᳚ । \newline
7. मे॒ त॒नुवा᳚ स्त॒नुवा॑ मे मे त॒नुवाः᳚ । \newline
8. त॒नुवा॑ ऊ॒न मू॒नम् त॒नुवा᳚ स्त॒नुवा॑ ऊ॒नम् । \newline
9. ऊ॒नम् तत् तदू॒न मू॒नम् तत् । \newline
10. तन् मे॑ मे॒ तत् तन् मे᳚ । \newline
11. म॒ आ मे॑ म॒ आ । \newline
12. आ पृ॑ण पृ॒णा पृ॑ण । \newline
13. पृ॒णे तीति॑ पृण पृ॒णे ति॑ । \newline
14. इत्या॑हा॒हे तीत्या॑ह । \newline
15. आ॒ह॒ यद् यदा॑हाह॒ यत् । \newline
16. यन् मे॑ मे॒ यद् यन् मे᳚ । \newline
17. मे॒ प्र॒जायै᳚ प्र॒जायै॑ मे मे प्र॒जायै᳚ । \newline
18. प्र॒जायै॑ पशू॒नाम् प॑शू॒नाम् प्र॒जायै᳚ प्र॒जायै॑ पशू॒नाम् । \newline
19. प्र॒जाया॒ इति॑ प्र - जायै᳚ । \newline
20. प॒शू॒ना मू॒न मू॒नम् प॑शू॒नाम् प॑शू॒ना मू॒नम् । \newline
21. ऊ॒नम् तत् तदू॒न मू॒नम् तत् । \newline
22. तन् मे॑ मे॒ तत् तन् मे᳚ । \newline
23. म॒ आ मे॑ म॒ आ । \newline
24. आ पू॑रय पूर॒या पू॑रय । \newline
25. पू॒र॒ये तीति॑ पूरय पूर॒ये ति॑ । \newline
26. इति॒ वाव वावे तीति॒ वाव । \newline
27. वावैतदे॒तद् वाव वावैतत् । \newline
28. ए॒तदा॑हाहै॒तदे॒तदा॑ह । \newline
29. आ॒ह॒ चित्रा॑वसो॒ चित्रा॑वसो आहाह॒ चित्रा॑वसो । \newline
30. चित्रा॑वसो स्व॒स्ति स्व॒स्ति चित्रा॑वसो॒ चित्रा॑वसो स्व॒स्ति । \newline
31. चित्रा॑वसो॒ इति॒ चित्र॑ - व॒सो॒ । \newline
32. स्व॒स्ति ते॑ ते स्व॒स्ति स्व॒स्ति ते᳚ । \newline
33. ते॒ पा॒रम् पा॒रम् ते॑ ते पा॒रम् । \newline
34. पा॒र म॑शीयाशीय पा॒रम् पा॒र म॑शीय । \newline
35. अ॒शी॒ये तीत्य॑शीयाशी॒ये ति॑ । \newline
36. इत्या॑हा॒हे तीत्या॑ह । \newline
37. आ॒ह॒ रात्री॒ रात्रि॑राहाह॒ रात्रिः॑ । \newline
38. रात्रि॒र् वै वै रात्री॒ रात्रि॒र् वै । \newline
39. वै चि॒त्राव॑सु श्चि॒त्राव॑सु॒र् वै वै चि॒त्राव॑सुः । \newline
40. चि॒त्राव॑सु॒ रव्यु॑ष्ट्या॒ अव्यु॑ष्ट्यै चि॒त्राव॑सु श्चि॒त्राव॑सु॒ रव्यु॑ष्ट्यै । \newline
41. चि॒त्राव॑सु॒रिति॑ चि॒त्र - व॒सुः॒ । \newline
42. अव्यु॑ष्ट्यै॒ वै वा अव्यु॑ष्ट्या॒ अव्यु॑ष्ट्यै॒ वै । \newline
43. अव्यु॑ष्ट्या॒ इत्यवि॑ - उ॒ष्ट्यै॒ । \newline
44. वा ए॒तस्या॑ ए॒तस्यै॒ वै वा ए॒तस्यै᳚ । \newline
45. ए॒तस्यै॑ पु॒रा पु॒रैतस्या॑ ए॒तस्यै॑ पु॒रा । \newline
46. पु॒रा ब्रा᳚ह्म॒णा ब्रा᳚ह्म॒णाः पु॒रा पु॒रा ब्रा᳚ह्म॒णाः । \newline
47. ब्रा॒ह्म॒णा अ॑भैषु रभैषुर् ब्राह्म॒णा ब्रा᳚ह्म॒णा अ॑भैषुः । \newline
48. अ॒भै॒षु॒र् व्यु॑ष्टिं॒ ॅव्यु॑ष्टि मभैषु रभैषु॒र् व्यु॑ष्टिम् । \newline
49. व्यु॑ष्टि मे॒वैव व्यु॑ष्टिं॒ ॅव्यु॑ष्टि मे॒व । \newline
50. व्यु॑ष्टि॒मिति॒ वि - उ॒ष्टि॒म् । \newline
51. ए॒वावावै॒वैवाव॑ । \newline
52. अव॑ रुन्धे रु॒न्धे ऽवाव॑ रुन्धे । \newline
53. रु॒न्ध॒ इन्धा॑ना॒ इन्धा॑ना रुन्धे रुन्ध॒ इन्धा॑नाः । \newline
54. इन्धा॑ना स्त्वा॒ त्वेन्धा॑ना॒ इन्धा॑ना स्त्वा । \newline
55. त्वा॒ श॒तꣳ श॒तम् त्वा᳚ त्वा श॒तम् । \newline
56. श॒तꣳ हिमा॒ हिमाः᳚ श॒तꣳ श॒तꣳ हिमाः᳚ । \newline

\textbf{Ghana Paata } \newline

1. त॒नू॒पा हि हि त॑नू॒पा स्त॑नू॒पा ह्ये॑ष ए॒ष हि त॑नू॒पा स्त॑नू॒पा ह्ये॑षः । \newline
2. त॒नू॒पा इति॑ तनू - पाः । \newline
3. ह्ये॑ष ए॒ष हि ह्ये॑षो ऽग्ने ऽग्न॑ ए॒ष हि ह्ये॑षो ऽग्ने᳚ । \newline
4. ए॒षो ऽग्ने ऽग्न॑ ए॒ष ए॒षो ऽग्ने॒ यद् यदग्न॑ ए॒ष ए॒षो ऽग्ने॒ यत् । \newline
5. अग्ने॒ यद् यदग्ने ऽग्ने॒ यन् मे॑ मे॒ यदग्ने ऽग्ने॒ यन् मे᳚ । \newline
6. यन् मे॑ मे॒ यद् यन् मे॑ त॒नुवा᳚ स्त॒नुवा॑ मे॒ यद् यन् मे॑ त॒नुवाः᳚ । \newline
7. मे॒ त॒नुवा᳚ स्त॒नुवा॑ मे मे त॒नुवा॑ ऊ॒न मू॒नम् त॒नुवा॑ मे मे त॒नुवा॑ ऊ॒नम् । \newline
8. त॒नुवा॑ ऊ॒न मू॒नम् त॒नुवा᳚ स्त॒नुवा॑ ऊ॒नम् तत् तदू॒नम् त॒नुवा᳚ स्त॒नुवा॑ ऊ॒नम् तत् । \newline
9. ऊ॒नम् तत् तदू॒न मू॒नम् तन् मे॑ मे॒ तदू॒न मू॒नम् तन् मे᳚ । \newline
10. तन् मे॑ मे॒ तत् तन् म॒ आ मे॒ तत् तन् म॒ आ । \newline
11. म॒ आ मे॑ म॒ आ पृ॑ण पृ॒णा मे॑ म॒ आ पृ॑ण । \newline
12. आ पृ॑ण पृ॒णा पृ॒णे तीति॑ पृ॒णा पृ॒णे ति॑ । \newline
13. पृ॒णे तीति॑ पृण पृ॒णे त्या॑हा॒हे ति॑ पृण पृ॒णे त्या॑ह । \newline
14. इत्या॑हा॒हे तीत्या॑ह॒ यद् यदा॒हे तीत्या॑ह॒ यत् । \newline
15. आ॒ह॒ यद् यदा॑हाह॒ यन् मे॑ मे॒ यदा॑हाह॒ यन् मे᳚ । \newline
16. यन् मे॑ मे॒ यद् यन् मे᳚ प्र॒जायै᳚ प्र॒जायै॑ मे॒ यद् यन् मे᳚ प्र॒जायै᳚ । \newline
17. मे॒ प्र॒जायै᳚ प्र॒जायै॑ मे मे प्र॒जायै॑ पशू॒नाम् प॑शू॒नाम् प्र॒जायै॑ मे मे प्र॒जायै॑ पशू॒नाम् । \newline
18. प्र॒जायै॑ पशू॒नाम् प॑शू॒नाम् प्र॒जायै᳚ प्र॒जायै॑ पशू॒ना मू॒न मू॒नम् प॑शू॒नाम् प्र॒जायै᳚ प्र॒जायै॑ पशू॒ना मू॒नम् । \newline
19. प्र॒जाया॒ इति॑ प्र - जायै᳚ । \newline
20. प॒शू॒ना मू॒न मू॒नम् प॑शू॒नाम् प॑शू॒ना मू॒नम् तत् तदू॒नम् प॑शू॒नाम् प॑शू॒ना मू॒नम् तत् । \newline
21. ऊ॒नम् तत् तदू॒न मू॒नम् तन् मे॑ मे॒ तदू॒न मू॒नम् तन् मे᳚ । \newline
22. तन् मे॑ मे॒ तत् तन् म॒ आ मे॒ तत् तन् म॒ आ । \newline
23. म॒ आ मे॑ म॒ आ पू॑रय पूर॒या मे॑ म॒ आ पू॑रय । \newline
24. आ पू॑रय पूर॒या पू॑र॒ये तीति॑ पूर॒या पू॑र॒ये ति॑ । \newline
25. पू॒र॒ये तीति॑ पूरय पूर॒ये ति॒ वाव वावे ति॑ पूरय पूर॒ये ति॒ वाव । \newline
26. इति॒ वाव वावे तीति॒ वावैतदे॒तद् वावे तीति॒ वावैतत् । \newline
27. वावैतदे॒तद् वाव वावैतदा॑हाहै॒तद् वाव वावैतदा॑ह । \newline
28. ए॒तदा॑हाहै॒तदे॒तदा॑ह॒ चित्रा॑वसो॒ चित्रा॑वसो आहै॒तदे॒तदा॑ह॒ चित्रा॑वसो । \newline
29. आ॒ह॒ चित्रा॑वसो॒ चित्रा॑वसो आहाह॒ चित्रा॑वसो स्व॒स्ति स्व॒स्ति चित्रा॑वसो आहाह॒ चित्रा॑वसो स्व॒स्ति । \newline
30. चित्रा॑वसो स्व॒स्ति स्व॒स्ति चित्रा॑वसो॒ चित्रा॑वसो स्व॒स्ति ते॑ ते स्व॒स्ति चित्रा॑वसो॒ चित्रा॑वसो स्व॒स्ति ते᳚ । \newline
31. चित्रा॑वसो॒ इति॒ चित्र॑ - व॒सो॒ । \newline
32. स्व॒स्ति ते॑ ते स्व॒स्ति स्व॒स्ति ते॑ पा॒रम् पा॒रम् ते᳚ स्व॒स्ति स्व॒स्ति ते॑ पा॒रम् । \newline
33. ते॒ पा॒रम् पा॒रम् ते॑ ते पा॒र म॑शीयाशीय पा॒रम् ते॑ ते पा॒र म॑शीय । \newline
34. पा॒र म॑शीयाशीय पा॒रम् पा॒र म॑शी॒ये तीत्य॑शीय पा॒रम् पा॒र म॑शी॒ये ति॑ । \newline
35. अ॒शी॒ये तीत्य॑शीयाशी॒ये त्या॑हा॒हे त्य॑शीयाशी॒ये त्या॑ह । \newline
36. इत्या॑हा॒हे तीत्या॑ह॒ रात्री॒ रात्रि॑रा॒हे तीत्या॑ह॒ रात्रिः॑ । \newline
37. आ॒ह॒ रात्री॒ रात्रि॑राहाह॒ रात्रि॒र् वै वै रात्रि॑राहाह॒ रात्रि॒र् वै । \newline
38. रात्रि॒र् वै वै रात्री॒ रात्रि॒र्वै चि॒त्राव॑सु श्चि॒त्राव॑सु॒र् वै रात्री॒ रात्रि॒र्वै चि॒त्राव॑सुः । \newline
39. वै चि॒त्राव॑सु श्चि॒त्राव॑सु॒र् वै वै चि॒त्राव॑सु॒ रव्यु॑ष्ट्या॒ अव्यु॑ष्ट्यै चि॒त्राव॑सु॒र् वै वै चि॒त्राव॑सु॒ रव्यु॑ष्ट्यै । \newline
40. चि॒त्राव॑सु॒ रव्यु॑ष्ट्या॒ अव्यु॑ष्ट्यै चि॒त्राव॑सु श्चि॒त्राव॑सु॒ रव्यु॑ष्ट्यै॒ वै वा अव्यु॑ष्ट्यै चि॒त्राव॑सु श्चि॒त्राव॑सु॒ रव्यु॑ष्ट्यै॒ वै । \newline
41. चि॒त्राव॑सु॒रिति॑ चि॒त्र - व॒सुः॒ । \newline
42. अव्यु॑ष्ट्यै॒ वै वा अव्यु॑ष्ट्या॒ अव्यु॑ष्ट्यै॒ वा ए॒तस्या॑ ए॒तस्यै॒ वा अव्यु॑ष्ट्या॒ अव्यु॑ष्ट्यै॒ वा ए॒तस्यै᳚ । \newline
43. अव्यु॑ष्ट्या॒ इत्यवि॑ - उ॒ष्ट्यै॒ । \newline
44. वा ए॒तस्या॑ ए॒तस्यै॒ वै वा ए॒तस्यै॑ पु॒रा पु॒रैतस्यै॒ वै वा ए॒तस्यै॑ पु॒रा । \newline
45. ए॒तस्यै॑ पु॒रा पु॒रैतस्या॑ ए॒तस्यै॑ पु॒रा ब्रा᳚ह्म॒णा ब्रा᳚ह्म॒णाः पु॒रैतस्या॑ ए॒तस्यै॑ पु॒रा ब्रा᳚ह्म॒णाः । \newline
46. पु॒रा ब्रा᳚ह्म॒णा ब्रा᳚ह्म॒णाः पु॒रा पु॒रा ब्रा᳚ह्म॒णा अ॑भैषु रभैषुर् ब्राह्म॒णाः पु॒रा पु॒रा ब्रा᳚ह्म॒णा अ॑भैषुः । \newline
47. ब्रा॒ह्म॒णा अ॑भैषु रभैषुर् ब्राह्म॒णा ब्रा᳚ह्म॒णा अ॑भैषु॒र् व्यु॑ष्टिं॒ ॅव्यु॑ष्टि मभैषुर् ब्राह्म॒णा ब्रा᳚ह्म॒णा अ॑भैषु॒र् व्यु॑ष्टिम् । \newline
48. अ॒भै॒षु॒र् व्यु॑ष्टिं॒ ॅव्यु॑ष्टि मभैषुरभैषु॒र् व्यु॑ष्टि मे॒वैव व्यु॑ष्टि मभैषुरभैषु॒र् व्यु॑ष्टि मे॒व । \newline
49. व्यु॑ष्टि मे॒वैव व्यु॑ष्टिं॒ ॅव्यु॑ष्टि मे॒वावावै॒व व्यु॑ष्टिं॒ ॅव्यु॑ष्टि मे॒वाव॑ । \newline
50. व्यु॑ष्टि॒मिति॒ वि - उ॒ष्टि॒म् । \newline
51. ए॒वावावै॒वैवाव॑ रुन्धे रु॒न्धे ऽवै॒वैवाव॑ रुन्धे । \newline
52. अव॑ रुन्धे रु॒न्धे ऽवाव॑ रुन्ध॒ इन्धा॑ना॒ इन्धा॑ना रु॒न्धे ऽवाव॑ रुन्ध॒ इन्धा॑नाः । \newline
53. रु॒न्ध॒ इन्धा॑ना॒ इन्धा॑ना रुन्धे रुन्ध॒ इन्धा॑ना स्त्वा॒ त्वेन्धा॑ना रुन्धे रुन्ध॒ इन्धा॑ना स्त्वा । \newline
54. इन्धा॑ना स्त्वा॒ त्वेन्धा॑ना॒ इन्धा॑ना स्त्वा श॒तꣳ श॒तम् त्वेन्धा॑ना॒ इन्धा॑ना स्त्वा श॒तम् । \newline
55. त्वा॒ श॒तꣳ श॒तम् त्वा᳚ त्वा श॒तꣳ हिमा॒ हिमाः᳚ श॒तम् त्वा᳚ त्वा श॒तꣳ हिमाः᳚ । \newline
56. श॒तꣳ हिमा॒ हिमाः᳚ श॒तꣳ श॒तꣳ हिमा॒ इतीति॒ हिमाः᳚ श॒तꣳ श॒तꣳ हिमा॒ इति॑ । \newline
\pagebreak
\markright{ TS 1.5.7.6  \hfill https://www.vedavms.in \hfill}
\addcontentsline{toc}{section}{ TS 1.5.7.6 }
\section*{ TS 1.5.7.6 }

\textbf{TS 1.5.7.6 } \newline
\textbf{Samhita Paata} \newline

हिमा॒ इत्या॑ह श॒तायुः॒ पुरु॑षः श॒तेन्द्रि॑य॒ आयु॑ष्ये॒वेन्द्रि॒ये प्रति॑ तिष्ठत्ये॒षा वै सू॒र्मी कर्ण॑कावत्ये॒तया॑ ह स्म॒ वै दे॒वा असु॑राणाꣳ शतत॒र्॒.हाꣳ स्तृꣳ॑हन्ति॒ यदे॒तया॑ स॒मिध॑मा॒दधा॑ति॒ वज्र॑मे॒वैतच्छ॑त॒घ्नीं ॅयज॑मानो॒ भ्रातृ॑व्याय॒ प्र ह॑रति॒ स्तृत्या॒ अछं॑बट्कारꣳ॒॒ सं त्वम॑ग्ने॒ सूर्य॑स्य॒ वर्च॑साऽगथा॒ इत्या॑है॒तत्त्वमसी॒दम॒हं ( ) भू॑यास॒मिति॒ वावैतदा॑ह॒ त्वम॑ग्ने॒ सूर्य॑वर्चा अ॒सीत्या॑हा॒ऽऽशिष॑मे॒वैतामा शा᳚स्ते ॥ \newline

\textbf{Pada Paata} \newline

हिमाः᳚ । इति॑ । आ॒ह॒ । श॒तायु॒रिति॑ श॒त - आ॒युः॒ । पुरु॑षः । श॒तेन्द्रि॑य॒ इति॑ श॒त - इ॒न्द्रि॒यः॒ । आयु॑षि । ए॒व । इ॒न्द्रि॒ये । प्रतीति॑ । ति॒ष्ठ॒ति॒ । ए॒षा । वै । सू॒र्मी । कर्ण॑काव॒तीति॒ कर्ण॑क-व॒ती॒ । ए॒तया᳚ । ह॒ । स्म॒ । वै । दे॒वाः । असु॑राणाम् । श॒त॒त॒र्॒.हानिति॑ शत - त॒र्॒.हान् । तृꣳ॒॒ह॒न्ति॒ । यत् । ए॒तया᳚ । स॒मिध॒मिति॑ सं-इध᳚म् । आ॒दधा॒तीत्या᳚ - दधा॑ति । वज्र᳚म् । ए॒व । ए॒तत् । श॒त॒घ्नीमिति॑ शत - घ्नीम् । यज॑मानः । भ्रातृ॑व्याय । प्रेति॑ । ह॒र॒ति॒ । स्तृत्यै᳚ । अच्छ॑बंट्कार॒मित्यच्छ॑बंट् - का॒र॒म् । समिति॑ । त्वम् । अ॒ग्ने॒ । सूर्य॑स्य । वर्च॑सा । अ॒ग॒थाः॒ । इति॑ । आ॒ह॒ । ए॒तत् । त्वम् । असि॑ । इ॒दम् । अ॒हम् ( ) । भू॒या॒स॒म् । इति॑ । वाव । ए॒तत् । आ॒ह॒ । त्वम् । अ॒ग्ने॒ । सूर्य॑वर्चा॒ इति॒ सूर्य॑ - व॒र्चाः॒ । अ॒सि॒ । इति॑ । आ॒ह॒ । आ॒शिष॒मित्या᳚ - शिष᳚म् । ए॒व । ए॒ताम् । एति॑ । शा॒स्ते॒ ॥  \newline


\textbf{Krama Paata} \newline

हिमा॒ इति॑ । इत्या॑ह । आ॒ह॒ श॒तायुः॑ । श॒तायुः॒ पुरु॑षः । श॒तायु॒रिति॑ श॒त - आ॒युः॒ । पुरु॑षः श॒तेन्द्रि॑यः । श॒तेन्द्रि॑य॒ आयु॑षि । श॒तेन्द्रि॑य॒ इति॑ श॒त - इ॒न्द्रि॒यः॒ । आयु॑ष्ये॒व । ए॒वेन्द्रि॒ये । इ॒न्द्रि॒ये प्रति॑ । प्रति॑ तिष्ठति । ति॒ष्ठ॒त्ये॒षा । ए॒षा वै । वै सू॒र्मी । सू॒र्मी कर्ण॑कावती । कर्ण॑कावत्ये॒तया᳚ । कर्ण॑काव॒तीति॒ कर्ण॑क - व॒ती॒ । ए॒तया॑ ह । ह॒ स्म॒ । स्म॒ वै । वै दे॒वाः । दे॒वा असु॑राणाम् । असु॑राणाꣳ शतत॒र्.॒हान् । श॒त॒त॒र्.॒हाꣳ,स्तृꣳ॑हन्ति । श॒त॒त॒र्.॒हानिति॑ शत - त॒र्॒.हान् । तृꣳ॒॒ह॒न्ति॒ यत् । यदे॒तया᳚ । ए॒तया॑ स॒मिध᳚म् । स॒मिध॑मा॒दधा॑ति । स॒मिध॒मिति॑ सम् - इध᳚म् । आ॒दधा॑ति॒ वज्र᳚म् । आ॒दधा॒तीत्या᳚ - दधा॑ति । वज्र॑मे॒व । ए॒वैतत् । ए॒तच्छ॑त॒घ्नीम् । श॒त॒घ्नीं ॅयज॑मानः । श॒त॒घ्नीमिति॑ शत - घ्नीम् । यज॑मानो॒ भ्रातृ॑व्याय । भ्रातृ॑व्याय॒ प्र । प्र ह॑रति । ह॒र॒ति॒ स्तृत्यै᳚ । स्तृत्या॒ अछ॑म्बट्कारम् । अछ॑म्बट्कारꣳ॒॒ सम् । अछ॑म्बट्कार॒मित्यछ॑म्बट् - का॒र॒म् । सन्त्वम् । त्वम॑ग्ने । अ॒ग्ने॒ सूर्य॑स्य । सूर्य॑स्य॒ वर्च॑सा । वर्च॑साऽगथाः । अ॒ग॒था॒ इति॑ । इत्या॑ह । आ॒है॒तत् । ए॒तत् त्वम् । त्वमसि॑ । असी॒दम् । इ॒दम॒हम् ( ) । अ॒हम् भू॑यासम् । भू॒या॒स॒मिति॑ । इति॒ वाव । वावैतत् । ए॒तदा॑ह । आ॒ह॒ त्वम् । त्वम॑ग्ने । अ॒ग्ने॒ सूर्य॑वर्चाः । सूर्य॑वर्चा असि । सूर्य॑वर्चा॒ इति॒ सूर्य॑ - व॒र्चाः॒ । अ॒सीति॑ । इत्या॑ह । आ॒हा॒शिष᳚म् । आ॒शिष॑मे॒व । आ॒शिष॒मित्या᳚ - शिष᳚म् । ए॒वैताम् । ए॒ता मा । आ शा᳚स्ते । शा॒स्त॒ इति॑ शास्ते । \newline

\textbf{Jatai Paata} \newline

1. हिमा॒ इतीति॒ हिमा॒ हिमा॒ इति॑ । \newline
2. इत्या॑हा॒हे तीत्या॑ह । \newline
3. आ॒ह॒ श॒तायुः॑ श॒तायु॑राहाह श॒तायुः॑ । \newline
4. श॒तायुः॒ पुरु॑षः॒ पुरु॑षः श॒तायुः॑ श॒तायुः॒ पुरु॑षः । \newline
5. श॒तायु॒रिति॑ श॒त - आ॒युः॒ । \newline
6. पुरु॑षः श॒तेन्द्रि॑यः श॒तेन्द्रि॑यः॒ पुरु॑षः॒ पुरु॑षः श॒तेन्द्रि॑यः । \newline
7. श॒तेन्द्रि॑य॒ आयु॒ष्यायु॑षि श॒तेन्द्रि॑यः श॒तेन्द्रि॑य॒ आयु॑षि । \newline
8. श॒तेन्द्रि॑य॒ इति॑ श॒त - इ॒न्द्रि॒यः॒ । \newline
9. आयु॑ष्ये॒वैवायु॒ष्यायु॑ष्ये॒व । \newline
10. ए॒वे न्द्रि॒य इ॑न्द्रि॒य ए॒वैवे न्द्रि॒ये । \newline
11. इ॒न्द्रि॒ये प्रति॒ प्रती᳚न्द्रि॒य इ॑न्द्रि॒ये प्रति॑ । \newline
12. प्रति॑ तिष्ठति तिष्ठति॒ प्रति॒ प्रति॑ तिष्ठति । \newline
13. ति॒ष्ठ॒त्ये॒षैषा ति॑ष्ठति तिष्ठत्ये॒षा । \newline
14. ए॒षा वै वा ए॒षैषा वै । \newline
15. वै सू॒र्मी सू॒र्मी वै वै सू॒र्मी । \newline
16. सू॒र्मी कर्ण॑कावती॒ कर्ण॑कावती सू॒र्मी सू॒र्मी कर्ण॑कावती । \newline
17. कर्ण॑कावत्ये॒तयै॒तया॒ कर्ण॑कावती॒ कर्ण॑कावत्ये॒तया᳚ । \newline
18. कर्ण॑काव॒तीति॒ कर्ण॑क - व॒ती॒ । \newline
19. ए॒तया॑ ह है॒तयै॒तया॑ ह । \newline
20. ह॒ स्म॒ स्म॒ ह॒ ह॒ स्म॒ । \newline
21. स्म॒ वै वै स्म॑ स्म॒ वै । \newline
22. वै दे॒वा दे॒वा वै वै दे॒वाः । \newline
23. दे॒वा असु॑राणा॒ मसु॑राणाम् दे॒वा दे॒वा असु॑राणाम् । \newline
24. असु॑राणाꣳ शतत॒र्॒.हाञ् छ॑तत॒र्॒.हा नसु॑राणा॒ मसु॑राणाꣳ शतत॒र्॒.हान् । \newline
25. श॒त॒त॒र्॒.हाꣳस् तृ(ग्म्॑)हन्ति तृꣳहन्ति शतत॒र्॒.हाञ् छ॑तत॒र्॒.हाꣳस् तृ(ग्म्॑)हन्ति । \newline
26. श॒त॒त॒र्॒.हानिति॑ शत - त॒र्॒.हान् । \newline
27. तृ॒(ग्म्॒)ह॒न्ति॒ यद् यत् तृ(ग्म्॑)हन्ति तृꣳहन्ति॒ यत् । \newline
28. यदे॒तयै॒तया॒ यद् यदे॒तया᳚ । \newline
29. ए॒तया॑ स॒मिध(ग्म्॑) स॒मिध॑ मे॒तयै॒तया॑ स॒मिध᳚म् । \newline
30. स॒मिध॑ मा॒दधा᳚त्या॒दधा॑ति स॒मिध(ग्म्॑) स॒मिध॑ मा॒दधा॑ति । \newline
31. स॒मिध॒मिति॑ सं - इध᳚म् । \newline
32. आ॒दधा॑ति॒ वज्रं॒ ॅवज्र॑ मा॒दधा᳚त्या॒दधा॑ति॒ वज्र᳚म् । \newline
33. आ॒दधा॒तीत्या᳚ - दधा॑ति । \newline
34. वज्र॑ मे॒वैव वज्रं॒ ॅवज्र॑ मे॒व । \newline
35. ए॒वैतदे॒तदे॒वैवैतत् । \newline
36. ए॒तच्छ॑त॒घ्नीꣳ श॑त॒घ्नी मे॒तदे॒तच्छ॑त॒घ्नीम् । \newline
37. श॒त॒घ्नीं ॅयज॑मानो॒ यज॑मानः शत॒घ्नीꣳ श॑त॒घ्नीं ॅयज॑मानः । \newline
38. श॒त॒घ्नीमिति॑ शत - घ्नीम् । \newline
39. यज॑मानो॒ भ्रातृ॑व्याय॒ भ्रातृ॑व्याय॒ यज॑मानो॒ यज॑मानो॒ भ्रातृ॑व्याय । \newline
40. भ्रातृ॑व्याय॒ प्र प्र भ्रातृ॑व्याय॒ भ्रातृ॑व्याय॒ प्र । \newline
41. प्र ह॑रति हरति॒ प्र प्र ह॑रति । \newline
42. ह॒र॒ति॒ स्तृत्यै॒ स्तृत्यै॑ हरति हरति॒ स्तृत्यै᳚ । \newline
43. स्तृत्या॒ अछं॑बट्कार॒ मछं॑बट्कार॒(ग्ग्॒) स्तृत्यै॒ स्तृत्या॒ अछं॑बट्कारम् । \newline
44. अछं॑बट्कार॒(ग्म्॒) सꣳ स मछं॑बट्कार॒ मछं॑बट्कार॒(ग्म्॒) सम् । \newline
45. अछं॑बट्कार॒मित्यछं॑बट् - का॒र॒म् । \newline
46. सम् त्वम् त्वꣳ सꣳ सम् त्वम् । \newline
47. त्व म॑ग्ने अग्ने॒ त्वम् त्व म॑ग्ने । \newline
48. अ॒ग्ने॒ सूर्य॑स्य॒ सूर्य॑स्याग्ने अग्ने॒ सूर्य॑स्य । \newline
49. सूर्य॑स्य॒ वर्च॑सा॒ वर्च॑सा॒ सूर्य॑स्य॒ सूर्य॑स्य॒ वर्च॑सा । \newline
50. वर्च॑सा ऽगथा अगथा॒ वर्च॑सा॒ वर्च॑सा ऽगथाः । \newline
51. अ॒ग॒था॒ इतीत्य॑गथा अगथा॒ इति॑ । \newline
52. इत्या॑हा॒हे तीत्या॑ह । \newline
53. आ॒है॒तदे॒तदा॑हाहै॒तत् । \newline
54. ए॒तत् त्वम् त्व मे॒तदे॒तत् त्वम् । \newline
55. त्व मस्यसि॒ त्वम् त्व मसि॑ । \newline
56. असी॒द मि॒द मस्यसी॒दम् । \newline
57. इ॒द म॒ह म॒ह मि॒द मि॒द म॒हम् । \newline
58. अ॒हम् भू॑यासम् भूयास म॒ह म॒हम् भू॑यासम् । \newline
59. भू॒या॒स॒ मितीति॑ भूयासम् भूयास॒ मिति॑ । \newline
60. इति॒ वाव वावे तीति॒ वाव । \newline
61. वावैतदे॒तद् वाव वावैतत् । \newline
62. ए॒तदा॑हाहै॒तदे॒तदा॑ह । \newline
63. आ॒ह॒ त्वम् त्व मा॑हाह॒ त्वम् । \newline
64. त्व म॑ग्ने अग्ने॒ त्वम् त्व म॑ग्ने । \newline
65. अ॒ग्ने॒ सूर्य॑वर्चाः॒ सूर्य॑वर्चा अग्ने अग्ने॒ सूर्य॑वर्चाः । \newline
66. सूर्य॑वर्चा अस्यसि॒ सूर्य॑वर्चाः॒ सूर्य॑वर्चा असि । \newline
67. सूर्य॑वर्चा॒ इति॒ सूर्य॑ - व॒र्चाः॒ । \newline
68. अ॒सीतीत्य॑स्य॒सीति॑ । \newline
69. इत्या॑हा॒हे तीत्या॑ह । \newline
70. आ॒हा॒शिष॑ मा॒शिष॑ माहाहा॒शिष᳚म् । \newline
71. आ॒शिष॑ मे॒वैवाशिष॑ मा॒शिष॑ मे॒व । \newline
72. आ॒शिष॒मित्या᳚ - शिष᳚म् । \newline
73. ए॒वैता मे॒ता मे॒वैवैताम् । \newline
74. ए॒ता मैता मे॒ता मा । \newline
75. आ शा᳚स्ते शास्त॒ आ शा᳚स्ते । \newline
76. शा॒स्त॒ इति॑ शास्ते । \newline

\textbf{Ghana Paata } \newline

1. हिमा॒ इतीति॒ हिमा॒ हिमा॒ इत्या॑हा॒हे ति॒ हिमा॒ हिमा॒ इत्या॑ह । \newline
2. इत्या॑हा॒हे तीत्या॑ह श॒तायुः॑ श॒तायु॑रा॒हे तीत्या॑ह श॒तायुः॑ । \newline
3. आ॒ह॒ श॒तायुः॑ श॒तायु॑राहाह श॒तायुः॒ पुरु॑षः॒ पुरु॑षः श॒तायु॑राहाह श॒तायुः॒ पुरु॑षः । \newline
4. श॒तायुः॒ पुरु॑षः॒ पुरु॑षः श॒तायुः॑ श॒तायुः॒ पुरु॑षः श॒तेन्द्रि॑यः श॒तेन्द्रि॑यः॒ पुरु॑षः श॒तायुः॑ श॒तायुः॒ पुरु॑षः श॒तेन्द्रि॑यः । \newline
5. श॒तायु॒रिति॑ श॒त - आ॒युः॒ । \newline
6. पुरु॑षः श॒तेन्द्रि॑यः श॒तेन्द्रि॑यः॒ पुरु॑षः॒ पुरु॑षः श॒तेन्द्रि॑य॒ आयु॒ष्यायु॑षि श॒तेन्द्रि॑यः॒ पुरु॑षः॒ पुरु॑षः श॒तेन्द्रि॑य॒ आयु॑षि । \newline
7. श॒तेन्द्रि॑य॒ आयु॒ष्यायु॑षि श॒तेन्द्रि॑यः श॒तेन्द्रि॑य॒ आयु॑ष्ये॒वैवायु॑षि श॒तेन्द्रि॑यः श॒तेन्द्रि॑य॒ आयु॑ष्ये॒व । \newline
8. श॒तेन्द्रि॑य॒ इति॑ श॒त - इ॒न्द्रि॒यः॒ । \newline
9. आयु॑ष्ये॒वै वायु॒ष्यायु॑ष्ये॒वे न्द्रि॒य इ॑न्द्रि॒य ए॒वायु॒ष्यायु॑ष्ये॒वे न्द्रि॒ये । \newline
10. ए॒वे न्द्रि॒य इ॑न्द्रि॒य ए॒वैवे न्द्रि॒ये प्रति॒ प्रती᳚न्द्रि॒य ए॒वैवे न्द्रि॒ये प्रति॑ । \newline
11. इ॒न्द्रि॒ये प्रति॒ प्रती᳚न्द्रि॒य इ॑न्द्रि॒ये प्रति॑ तिष्ठति तिष्ठति॒ प्रती᳚न्द्रि॒य इ॑न्द्रि॒ये प्रति॑ तिष्ठति । \newline
12. प्रति॑ तिष्ठति तिष्ठति॒ प्रति॒ प्रति॑ तिष्ठत्ये॒षैषा ति॑ष्ठति॒ प्रति॒ प्रति॑ तिष्ठत्ये॒षा । \newline
13. ति॒ष्ठ॒त्ये॒षैषा ति॑ष्ठति तिष्ठत्ये॒षा वै वा ए॒षा ति॑ष्ठति तिष्ठत्ये॒षा वै । \newline
14. ए॒षा वै वा ए॒षैषा वै सू॒र्मी सू॒र्मी वा ए॒षैषा वै सू॒र्मी । \newline
15. वै सू॒र्मी सू॒र्मी वै वै सू॒र्मी कर्ण॑कावती॒ कर्ण॑कावती सू॒र्मी वै वै सू॒र्मी कर्ण॑कावती । \newline
16. सू॒र्मी कर्ण॑कावती॒ कर्ण॑कावती सू॒र्मी सू॒र्मी कर्ण॑कावत्ये॒तयै॒तया॒ कर्ण॑कावती सू॒र्मी सू॒र्मी कर्ण॑कावत्ये॒तया᳚ । \newline
17. कर्ण॑कावत्ये॒तयै॒तया॒ कर्ण॑कावती॒ कर्ण॑कावत्ये॒तया॑ ह है॒तया॒ कर्ण॑कावती॒ कर्ण॑कावत्ये॒तया॑ ह । \newline
18. कर्ण॑काव॒तीति॒ कर्ण॑क - व॒ती॒ । \newline
19. ए॒तया॑ ह है॒तयै॒तया॑ ह स्म स्म है॒तयै॒तया॑ ह स्म । \newline
20. ह॒ स्म॒ स्म॒ ह॒ ह॒ स्म॒ वै वै स्म॑ ह ह स्म॒ वै । \newline
21. स्म॒ वै वै स्म॑ स्म॒ वै दे॒वा दे॒वा वै स्म॑ स्म॒ वै दे॒वाः । \newline
22. वै दे॒वा दे॒वा वै वै दे॒वा असु॑राणा॒ मसु॑राणाम् दे॒वा वै वै दे॒वा असु॑राणाम् । \newline
23. दे॒वा असु॑राणा॒ मसु॑राणाम् दे॒वा दे॒वा असु॑राणाꣳ शतत॒र्॒.हाञ् छ॑तत॒र्॒.हा नसु॑राणाम् दे॒वा दे॒वा असु॑राणाꣳ शतत॒र्॒.हान् । \newline
24. असु॑राणाꣳ शतत॒र्॒.हाञ् छ॑तत॒र्॒.हा नसु॑राणा॒ मसु॑राणाꣳ शतत॒र्॒.हाꣳ स्तृ(ग्म्॑)हन्ति तृꣳहन्ति शतत॒र्॒.हा नसु॑राणा॒ मसु॑राणाꣳ शतत॒र्॒.हाꣳ स्तृ(ग्म्॑)हन्ति । \newline
25. श॒त॒त॒र्॒.हाꣳ स्तृ(ग्म्॑)हन्ति तृꣳहन्ति शतत॒र्॒.हाञ् छ॑तत॒र्॒.हाꣳ स्तृ(ग्म्॑)हन्ति॒ यद् यत् तृ(ग्म्॑)हन्ति शतत॒र्॒.हाञ् छ॑तत॒र्॒.हाꣳ स्तृ(ग्म्॑)हन्ति॒ यत् । \newline
26. श॒त॒त॒र्॒.हानिति॑ शत - त॒र्॒.हान् । \newline
27. तृ॒(ग्म्॒)ह॒न्ति॒ यद् यत् तृ(ग्म्॑)हन्ति तृꣳहन्ति॒ यदे॒तयै॒तया॒ यत् तृ(ग्म्॑)हन्ति तृꣳहन्ति॒ यदे॒तया᳚ । \newline
28. यदे॒तयै॒तया॒ यद् यदे॒तया॑ स॒मिध(ग्म्॑) स॒मिध॑ मे॒तया॒ यद् यदे॒तया॑ स॒मिध᳚म् । \newline
29. ए॒तया॑ स॒मिध(ग्म्॑) स॒मिध॑ मे॒तयै॒तया॑ स॒मिध॑ मा॒दधा᳚त्या॒दधा॑ति स॒मिध॑ मे॒तयै॒तया॑ स॒मिध॑ मा॒दधा॑ति । \newline
30. स॒मिध॑ मा॒दधा᳚त्या॒दधा॑ति स॒मिध(ग्म्॑) स॒मिध॑ मा॒दधा॑ति॒ वज्रं॒ ॅवज्र॑ मा॒दधा॑ति स॒मिध(ग्म्॑) स॒मिध॑ मा॒दधा॑ति॒ वज्र᳚म् । \newline
31. स॒मिध॒मिति॑ सं - इध᳚म् । \newline
32. आ॒दधा॑ति॒ वज्रं॒ ॅवज्र॑ मा॒दधा᳚त्या॒दधा॑ति॒ वज्र॑ मे॒वैव वज्र॑ मा॒दधा᳚त्या॒दधा॑ति॒ वज्र॑ मे॒व । \newline
33. आ॒दधा॒तीत्या᳚ - दधा॑ति । \newline
34. वज्र॑ मे॒वैव वज्रं॒ ॅवज्र॑ मे॒वैतदे॒तदे॒व वज्रं॒ ॅवज्र॑ मे॒वैतत् । \newline
35. ए॒वैत दे॒तदे॒वैवैत च्छ॑त॒घ्नीꣳ श॑त॒घ्नी मे॒तदे॒वैवैत च्छ॑त॒घ्नीम् । \newline
36. ए॒तच्छ॑त॒घ्नीꣳ श॑त॒घ्नी मे॒तदे॒तच्छ॑त॒घ्नीं ॅयज॑मानो॒ यज॑मानः शत॒घ्नी मे॒तदे॒तच्छ॑त॒घ्नीं ॅयज॑मानः । \newline
37. श॒त॒घ्नीं ॅयज॑मानो॒ यज॑मानः शत॒घ्नीꣳ श॑त॒घ्नीं ॅयज॑मानो॒ भ्रातृ॑व्याय॒ भ्रातृ॑व्याय॒ यज॑मानः शत॒घ्नीꣳ श॑त॒घ्नीं ॅयज॑मानो॒ भ्रातृ॑व्याय । \newline
38. श॒त॒घ्नीमिति॑ शत - घ्नीम् । \newline
39. यज॑मानो॒ भ्रातृ॑व्याय॒ भ्रातृ॑व्याय॒ यज॑मानो॒ यज॑मानो॒ भ्रातृ॑व्याय॒ प्र प्र भ्रातृ॑व्याय॒ यज॑मानो॒ यज॑मानो॒ भ्रातृ॑व्याय॒ प्र । \newline
40. भ्रातृ॑व्याय॒ प्र प्र भ्रातृ॑व्याय॒ भ्रातृ॑व्याय॒ प्र ह॑रति हरति॒ प्र भ्रातृ॑व्याय॒ भ्रातृ॑व्याय॒ प्र ह॑रति । \newline
41. प्र ह॑रति हरति॒ प्र प्र ह॑रति॒ स्तृत्यै॒ स्तृत्यै॑ हरति॒ प्र प्र ह॑रति॒ स्तृत्यै᳚ । \newline
42. ह॒र॒ति॒ स्तृत्यै॒ स्तृत्यै॑ हरति हरति॒ स्तृत्या॒ अछं॑बट्कार॒ मछं॑बट्कार॒(ग्ग्॒) स्तृत्यै॑ हरति हरति॒ स्तृत्या॒ अछं॑बट्कारम् । \newline
43. स्तृत्या॒ अछं॑बट्कार॒ मछं॑बट्कार॒(ग्ग्॒) स्तृत्यै॒ स्तृत्या॒ अछं॑बट्कार॒(ग्म्॒) सꣳ स मछं॑बट्कार॒(ग्ग्॒) स्तृत्यै॒ स्तृत्या॒ अछं॑बट्कार॒(ग्म्॒) सम् । \newline
44. अछं॑बट्कार॒(ग्म्॒) सꣳ स मछं॑बट्कार॒ मछं॑बट्कार॒(ग्म्॒) सम् त्वम् त्वꣳ स मछं॑बट्कार॒ मछं॑बट्कार॒(ग्म्॒) सम् त्वम् । \newline
45. अछं॑बट्कार॒मित्यछं॑बट् - का॒र॒म् । \newline
46. सम् त्वम् त्वꣳ सꣳ सम् त्व म॑ग्ने अग्ने॒ त्वꣳ सꣳ सम् त्व म॑ग्ने । \newline
47. त्व म॑ग्ने अग्ने॒ त्वम् त्व म॑ग्ने॒ सूर्य॑स्य॒ सूर्य॑स्याग्ने॒ त्वम् त्व म॑ग्ने॒ सूर्य॑स्य । \newline
48. अ॒ग्ने॒ सूर्य॑स्य॒ सूर्य॑स्याग्ने अग्ने॒ सूर्य॑स्य॒ वर्च॑सा॒ वर्च॑सा॒ सूर्य॑स्याग्ने अग्ने॒ सूर्य॑स्य॒ वर्च॑सा । \newline
49. सूर्य॑स्य॒ वर्च॑सा॒ वर्च॑सा॒ सूर्य॑स्य॒ सूर्य॑स्य॒ वर्च॑सा ऽगथा अगथा॒ वर्च॑सा॒ सूर्य॑स्य॒ सूर्य॑स्य॒ वर्च॑सा ऽगथाः । \newline
50. वर्च॑सा ऽगथा अगथा॒ वर्च॑सा॒ वर्च॑सा ऽगथा॒ इतीत्य॑गथा॒ वर्च॑सा॒ वर्च॑सा ऽगथा॒ इति॑ । \newline
51. अ॒ग॒था॒ इतीत्य॑गथा अगथा॒ इत्या॑हा॒हे त्य॑गथा अगथा॒ इत्या॑ह । \newline
52. इत्या॑हा॒हे तीत्या॑है॒तदे॒तदा॒हे तीत्या॑है॒तत् । \newline
53. आ॒है॒तदे॒तदा॑हाहै॒तत् त्वम् त्व मे॒तदा॑हाहै॒तत् त्वम् । \newline
54. ए॒तत् त्वम् त्व मे॒तदे॒तत् त्व मस्यसि॒ त्व मे॒तदे॒तत् त्व मसि॑ । \newline
55. त्व मस्यसि॒ त्वम् त्व मसी॒द मि॒द मसि॒ त्वम् त्व मसी॒दम् । \newline
56. असी॒द मि॒द मस्यसी॒द म॒ह म॒ह मि॒द मस्यसी॒द म॒हम् । \newline
57. इ॒द म॒ह म॒ह मि॒द मि॒द म॒हम् भू॑यासम् भूयास म॒ह मि॒द मि॒द म॒हम् भू॑यासम् । \newline
58. अ॒हम् भू॑यासम् भूयास म॒ह म॒हम् भू॑यास॒ मितीति॑ भूयास म॒ह म॒हम् भू॑यास॒ मिति॑ । \newline
59. भू॒या॒स॒ मितीति॑ भूयासम् भूयास॒ मिति॒ वाव वावे ति॑ भूयासम् भूयास॒ मिति॒ वाव । \newline
60. इति॒ वाव वावे तीति॒ वावैतदे॒तद् वावे तीति॒ वावैतत् । \newline
61. वावैतदे॒तद् वाव वावैतदा॑हाहै॒तद् वाव वावैतदा॑ह । \newline
62. ए॒तदा॑हाहै॒तदे॒तदा॑ह॒ त्वम् त्व मा॑है॒तदे॒तदा॑ह॒ त्वम् । \newline
63. आ॒ह॒ त्वम् त्व मा॑हाह॒ त्व म॑ग्ने अग्ने॒ त्व मा॑हाह॒ त्व म॑ग्ने । \newline
64. त्व म॑ग्ने अग्ने॒ त्वम् त्व म॑ग्ने॒ सूर्य॑वर्चाः॒ सूर्य॑वर्चा अग्ने॒ त्वम् त्व म॑ग्ने॒ सूर्य॑वर्चाः । \newline
65. अ॒ग्ने॒ सूर्य॑वर्चाः॒ सूर्य॑वर्चा अग्ने अग्ने॒ सूर्य॑वर्चा अस्यसि॒ सूर्य॑वर्चा अग्ने अग्ने॒ सूर्य॑वर्चा असि । \newline
66. सूर्य॑वर्चा अस्यसि॒ सूर्य॑वर्चाः॒ सूर्य॑वर्चा अ॒सीतीत्य॑सि॒ सूर्य॑वर्चाः॒ सूर्य॑वर्चा अ॒सीति॑ । \newline
67. सूर्य॑वर्चा॒ इति॒ सूर्य॑ - व॒र्चाः॒ । \newline
68. अ॒सीतीत्य॑स्य॒सीत्या॑हा॒हे त्य॑स्य॒सीत्या॑ह । \newline
69. इत्या॑हा॒हे तीत्या॑हा॒शिष॑ मा॒शिष॑ मा॒हे तीत्या॑हा॒शिष᳚म् । \newline
70. आ॒हा॒शिष॑ मा॒शिष॑ माहाहा॒शिष॑ मे॒वैवाशिष॑ माहाहा॒शिष॑ मे॒व । \newline
71. आ॒शिष॑ मे॒वैवाशिष॑ मा॒शिष॑ मे॒वैता मे॒ता मे॒वाशिष॑ मा॒शिष॑ मे॒वैताम् । \newline
72. आ॒शिष॒मित्या᳚ - शिष᳚म् । \newline
73. ए॒वैता मे॒ता मे॒वैवैता मैता मे॒वैवैता मा । \newline
74. ए॒ता मैता मे॒ता मा शा᳚स्ते शास्त॒ ऐता मे॒ता मा शा᳚स्ते । \newline
75. आ शा᳚स्ते शास्त॒ आ शा᳚स्ते । \newline
76. शा॒स्त॒ इति॑ शास्ते । \newline
\pagebreak
\markright{ TS 1.5.8.1  \hfill https://www.vedavms.in \hfill}
\addcontentsline{toc}{section}{ TS 1.5.8.1 }
\section*{ TS 1.5.8.1 }

\textbf{TS 1.5.8.1 } \newline
\textbf{Samhita Paata} \newline

सं प॑श्यामि प्र॒जा अ॒हमित्या॑ह॒ याव॑न्त ए॒व ग्रा॒म्याः प॒शव॒स्ताने॒वाव॑ रु॒न्धेऽंभः॒ स्थांभो॑ वो भक्षी॒येत्या॒हांभो॒ ह्ये॑ता महः॑ स्थ॒ महो॑ वो भक्षी॒येत्या॑ह॒ महो॒ ह्ये॑ताः सहः॑ स्थ॒ सहो॑ वो भक्षी॒येत्या॑ह॒ सहो॒ ह्ये॑ता ऊर्जः॒ स्थोर्जं॑ ॅवो भक्षी॒येत्या॒ - [ ] \newline

\textbf{Pada Paata} \newline

समिति॑ । प॒श्या॒मि॒ । प्र॒जा इति॑ प्र - जाः । अ॒हम् । इति॑ । आ॒ह॒ । याव॑न्तः । ए॒व । ग्रा॒म्याः । प॒शवः॑ । तान् । ए॒व । अवेति॑ । रु॒न्धे॒ । अभंः॑ । स्थ॒ । अभंः॑ । वः॒ । भ॒क्षी॒य॒ । इति॑ । आ॒ह॒ । अभंः॑ । हि । ए॒ताः । महः॑ । स्थ॒ । महः॑ । वः॒ । भ॒क्षी॒य॒ । इति॑ । आ॒ह॒ । महः॑ । हि । ए॒ताः । सहः॑ । स्थ॒ । सहः॑ । वः॒ । भ॒क्षी॒य॒ । इति॑ । आ॒ह॒ । सहः॑ । हि । ए॒ताः । ऊर्जः॑ । स्थ॒ । ऊर्ज᳚म् । वः॒ । भ॒क्षी॒य॒ । इति॑ ।  \newline


\textbf{Krama Paata} \newline

सं प॑श्यामि । प॒श्या॒मि॒ प्र॒जाः । प्र॒जा अ॒हम् । प्र॒जा इति॑ प्र - जाः । अ॒हमिति॑ । इत्या॑ह । आ॒ह॒ याव॑न्तः । याव॑न्त ए॒व । ए॒व ग्रा॒म्याः । ग्रा॒म्याः प॒शवः॑ । प॒शव॒स्तान् । ताने॒व । ए॒वाव॑ । अव॑ रुन्धे । रु॒न्धेऽम्भः॑ । अम्भः॑ स्थ । स्थाम्भः॑ । अम्भो॑ वः । वो॒ भ॒क्षी॒य॒ । भ॒क्षी॒येति॑ । इत्या॑ह । आ॒हाम्भः॑ । अम्भो॒ हि । ह्ये॑ताः । ए॒ता महः॑ । महः॑ स्थ । स्थ॒ महः॑ । महो॑ वः । वो॒ भ॒क्षी॒य॒ । भ॒क्षी॒येति॑ । इत्या॑ह । आ॒ह॒ महः॑ । महो॒ हि । ह्ये॑ताः । ए॒ताः सहः॑ । सहः॑ स्थ । स्थ॒ सहः॑ । सहो॑ वः । वो॒ भ॒क्षी॒य॒ । भ॒क्षी॒येति॑ । इत्या॑ह । आ॒ह॒ सहः॑ । सहो॒ हि । ह्ये॑ताः । ए॒ता ऊर्जः॑ । ऊर्ज॑ स्थ । स्थोर्ज᳚म् । ऊर्जं॑ ॅवः । वो॒ भ॒क्षी॒य॒ । भ॒क्षी॒येति॑ । इत्या॑ह \newline

\textbf{Jatai Paata} \newline

1. सम् प॑श्यामि पश्यामि॒ सꣳ सम् प॑श्यामि । \newline
2. प॒श्या॒मि॒ प्र॒जाः प्र॒जाः प॑श्यामि पश्यामि प्र॒जाः । \newline
3. प्र॒जा अ॒ह म॒हम् प्र॒जाः प्र॒जा अ॒हम् । \newline
4. प्र॒जा इति॑ प्र - जाः । \newline
5. अ॒ह मितीत्य॒ह म॒ह मिति॑ । \newline
6. इत्या॑हा॒हे तीत्या॑ह । \newline
7. आ॒ह॒ याव॑न्तो॒ याव॑न्त आहाह॒ याव॑न्तः । \newline
8. याव॑न्त ए॒वैव याव॑न्तो॒ याव॑न्त ए॒व । \newline
9. ए॒व ग्रा॒म्या ग्रा॒म्या ए॒वैव ग्रा॒म्याः । \newline
10. ग्रा॒म्याः प॒शवः॑ प॒शवो᳚ ग्रा॒म्या ग्रा॒म्याः प॒शवः॑ । \newline
11. प॒शव॒स्ताꣳ स्तान् प॒शवः॑ प॒शव॒स्तान् । \newline
12. ता ने॒वैव ताꣳ स्ता ने॒व । \newline
13. ए॒वावावै॒वैवाव॑ । \newline
14. अव॑ रुन्धे रु॒न्धे ऽवाव॑ रुन्धे । \newline
15. रु॒न्धें ऽभ्ॐ ऽभो॑ रुन्धे रु॒न्धें ऽभः॑ । \newline
16. अंभः॑ स्थ॒ स्थां भ्ॐ ऽभः॑ स्थ । \newline
17. स्थांभ्ॐ भः॑ स्थ॒ स्थांभः॑ । \newline
18. अंभो॑ वो॒ व्ॐ भ्ॐ भो॑ वः । \newline
19. वो॒ भ॒क्षी॒य॒ भ॒क्षी॒य॒ वो॒ वो॒ भ॒क्षी॒य॒ । \newline
20. भ॒क्षी॒ये तीति॑ भक्षीय भक्षी॒ये ति॑ । \newline
21. इत्या॑हा॒हे तीत्या॑ह । \newline
22. आ॒हांभ्ॐ ऽभ॑ आहा॒हांभः॑ । \newline
23. अंभो॒ हि ह्यंभ्ॐ भो॒ हि । \newline
24. ह्ये॑ता ए॒ता हि ह्ये॑ताः । \newline
25. ए॒ता महो॒ मह॑ ए॒ता ए॒ता महः॑ । \newline
26. महः॑ स्थ स्थ॒ महो॒ महः॑ स्थ । \newline
27. स्थ॒ महो॒ महः॑ स्थ स्थ॒ महः॑ । \newline
28. महो॑ वो वो॒ महो॒ महो॑ वः । \newline
29. वो॒ भ॒क्षी॒य॒ भ॒क्षी॒य॒ वो॒ वो॒ भ॒क्षी॒य॒ । \newline
30. भ॒क्षी॒ये तीति॑ भक्षीय भक्षी॒ये ति॑ । \newline
31. इत्या॑हा॒हे तीत्या॑ह । \newline
32. आ॒ह॒ महो॒ मह॑ आहाह॒ महः॑ । \newline
33. महो॒ हि हि महो॒ महो॒ हि । \newline
34. ह्ये॑ता ए॒ता हि ह्ये॑ताः । \newline
35. ए॒ताः सहः॒ सह॑ ए॒ता ए॒ताः सहः॑ । \newline
36. सहः॑ स्थ स्थ॒ सहः॒ सहः॑ स्थ । \newline
37. स्थ॒ सहः॒ सहः॑ स्थ स्थ॒ सहः॑ । \newline
38. सहो॑ वो वः॒ सहः॒ सहो॑ वः । \newline
39. वो॒ भ॒क्षी॒य॒ भ॒क्षी॒य॒ वो॒ वो॒ भ॒क्षी॒य॒ । \newline
40. भ॒क्षी॒ये तीति॑ भक्षीय भक्षी॒ये ति॑ । \newline
41. इत्या॑हा॒हे तीत्या॑ह । \newline
42. आ॒ह॒ सहः॒ सह॑ आहाह॒ सहः॑ । \newline
43. सहो॒ हि हि सहः॒ सहो॒ हि । \newline
44. ह्ये॑ता ए॒ता हि ह्ये॑ताः । \newline
45. ए॒ता ऊर्ज॒ ऊर्ज॑ ए॒ता ए॒ता ऊर्जः॑ । \newline
46. ऊर्जः॑ स्थ॒ स्थोर्ज॒ ऊर्जः॑ स्थ । \newline
47. स्थोर्ज॒ मूर्ज(ग्ग्॑) स्थ॒ स्थोर्ज᳚म् । \newline
48. ऊर्जं॑ ॅवो व॒ ऊर्ज॒ मूर्जं॑ ॅवः । \newline
49. वो॒ भ॒क्षी॒य॒ भ॒क्षी॒य॒ वो॒ वो॒ भ॒क्षी॒य॒ । \newline
50. भ॒क्षी॒ये तीति॑ भक्षीय भक्षी॒ये ति॑ । \newline
51. इत्या॑हा॒हे तीत्या॑ह । \newline

\textbf{Ghana Paata } \newline

1. सम् प॑श्यामि पश्यामि॒ सꣳ सम् प॑श्यामि प्र॒जाः प्र॒जाः प॑श्यामि॒ सꣳ सम् प॑श्यामि प्र॒जाः । \newline
2. प॒श्या॒मि॒ प्र॒जाः प्र॒जाः प॑श्यामि पश्यामि प्र॒जा अ॒ह म॒हम् प्र॒जाः प॑श्यामि पश्यामि प्र॒जा अ॒हम् । \newline
3. प्र॒जा अ॒ह म॒हम् प्र॒जाः प्र॒जा अ॒ह मितीत्य॒हम् प्र॒जाः प्र॒जा अ॒ह मिति॑ । \newline
4. प्र॒जा इति॑ प्र - जाः । \newline
5. अ॒ह मितीत्य॒ह म॒ह मित्या॑हा॒हे त्य॒ह म॒ह मित्या॑ह । \newline
6. इत्या॑हा॒हे तीत्या॑ह॒ याव॑न्तो॒ याव॑न्त आ॒हे तीत्या॑ह॒ याव॑न्तः । \newline
7. आ॒ह॒ याव॑न्तो॒ याव॑न्त आहाह॒ याव॑न्त ए॒वैव याव॑न्त आहाह॒ याव॑न्त ए॒व । \newline
8. याव॑न्त ए॒वैव याव॑न्तो॒ याव॑न्त ए॒व ग्रा॒म्या ग्रा॒म्या ए॒व याव॑न्तो॒ याव॑न्त ए॒व ग्रा॒म्याः । \newline
9. ए॒व ग्रा॒म्या ग्रा॒म्या ए॒वैव ग्रा॒म्याः प॒शवः॑ प॒शवो᳚ ग्रा॒म्या ए॒वैव ग्रा॒म्याः प॒शवः॑ । \newline
10. ग्रा॒म्याः प॒शवः॑ प॒शवो᳚ ग्रा॒म्या ग्रा॒म्याः प॒शव॒स्ताꣳ स्तान् प॒शवो᳚ ग्रा॒म्या ग्रा॒म्याः प॒शव॒स्तान् । \newline
11. प॒शव॒स्ताꣳ स्तान् प॒शवः॑ प॒शव॒स्ता ने॒वैव तान् प॒शवः॑ प॒शव॒स्ता ने॒व । \newline
12. ता ने॒वैव ताꣳ स्ताने॒वावावै॒व ताꣳ स्ताने॒वाव॑ । \newline
13. ए॒वावावै॒वैवाव॑ रुन्धे रु॒न्धे ऽवै॒वैवाव॑ रुन्धे । \newline
14. अव॑ रुन्धे रु॒न्धे ऽवाव॑ रु॒न्धे ऽम्भो ऽंभो॑ रु॒न्धे ऽवाव॑ रु॒न्धे ऽंभः॑ । \newline
15. रु॒न्धे ऽंभो ऽंभो॑ रुन्धे रु॒न्धे ऽंभः॑ स्थ॒ स्था ऽंभो॑ रुन्धे रु॒न्धे ऽंभः॑ स्थ । \newline
16. अंभः॑ स्थ॒ स्थांभो ऽंभः॒ स्थांभो ऽंभः॒ स्थांभो ऽंभः॒ स्थांभः॑ । \newline
17. स्थांभो ऽंभः॑ स्थ॒ स्थांभो॑ वो॒ वो ऽंभः॑ स्थ॒ स्थांभो॑ वः । \newline
18. अंभो॑ वो॒ वो ऽंभो ऽंभो॑ वो भक्षीय भक्षीय॒ वो ऽंभो ऽंभो॑ वो भक्षीय । \newline
19. वो॒ भ॒क्षी॒य॒ भ॒क्षी॒य॒ वो॒ वो॒ भ॒क्षी॒ये तीति॑ भक्षीय वो वो भक्षी॒ये ति॑ । \newline
20. भ॒क्षी॒ये तीति॑ भक्षीय भक्षी॒ये त्या॑हा॒हे ति॑ भक्षीय भक्षी॒ये त्या॑ह । \newline
21. इत्या॑हा॒हे तीत्या॒हांभो ऽंभ॑ आ॒हे तीत्या॒हांभः॑ । \newline
22. आ॒हांभो ऽंभ॑ आहा॒हांभो॒ हि ह्यंभ॑ आहा॒हांभो॒ हि । \newline
23. अंभो॒ हि ह्यंभो ऽंभो॒ ह्ये॑ता ए॒ता ह्यंभो ऽंभो॒ ह्ये॑ताः । \newline
24. ह्ये॑ता ए॒ता हि ह्ये॑ता महो॒ मह॑ ए॒ता हि ह्ये॑ता महः॑ । \newline
25. ए॒ता महो॒ मह॑ ए॒ता ए॒ता महः॑ स्थ स्थ॒ मह॑ ए॒ता ए॒ता महः॑ स्थ । \newline
26. महः॑ स्थ स्थ॒ महो॒ महः॑ स्थ॒ महो॒ महः॑ स्थ॒ महो॒ महः॑ स्थ॒ महः॑ । \newline
27. स्थ॒ महो॒ महः॑ स्थ स्थ॒ महो॑ वो वो॒ महः॑ स्थ स्थ॒ महो॑ वः । \newline
28. महो॑ वो वो॒ महो॒ महो॑ वो भक्षीय भक्षीय वो॒ महो॒ महो॑ वो भक्षीय । \newline
29. वो॒ भ॒क्षी॒य॒ भ॒क्षी॒य॒ वो॒ वो॒ भ॒क्षी॒ये तीति॑ भक्षीय वो वो भक्षी॒ये ति॑ । \newline
30. भ॒क्षी॒ये तीति॑ भक्षीय भक्षी॒ये त्या॑हा॒हे ति॑ भक्षीय भक्षी॒ये त्या॑ह । \newline
31. इत्या॑हा॒हे तीत्या॑ह॒ महो॒ मह॑ आ॒हे तीत्या॑ह॒ महः॑ । \newline
32. आ॒ह॒ महो॒ मह॑ आहाह॒ महो॒ हि हि मह॑ आहाह॒ महो॒ हि । \newline
33. महो॒ हि हि महो॒ महो॒ ह्ये॑ता ए॒ता हि महो॒ महो॒ ह्ये॑ताः । \newline
34. ह्ये॑ता ए॒ता हि ह्ये॑ताः सहः॒ सह॑ ए॒ता हि ह्ये॑ताः सहः॑ । \newline
35. ए॒ताः सहः॒ सह॑ ए॒ता ए॒ताः सहः॑ स्थ स्थ॒ सह॑ ए॒ता ए॒ताः सहः॑ स्थ । \newline
36. सहः॑ स्थ स्थ॒ सहः॒ सहः॑ स्थ॒ सहः॒ सहः॑ स्थ॒ सहः॒ सहः॑ स्थ॒ सहः॑ । \newline
37. स्थ॒ सहः॒ सहः॑ स्थ स्थ॒ सहो॑ वो वः॒ सहः॑ स्थ स्थ॒ सहो॑ वः । \newline
38. सहो॑ वो वः॒ सहः॒ सहो॑ वो भक्षीय भक्षीय वः॒ सहः॒ सहो॑ वो भक्षीय । \newline
39. वो॒ भ॒क्षी॒य॒ भ॒क्षी॒य॒ वो॒ वो॒ भ॒क्षी॒ये तीति॑ भक्षीय वो वो भक्षी॒ये ति॑ । \newline
40. भ॒क्षी॒ये तीति॑ भक्षीय भक्षी॒ये त्या॑हा॒हे ति॑ भक्षीय भक्षी॒ये त्या॑ह । \newline
41. इत्या॑हा॒हे तीत्या॑ह॒ सहः॒ सह॑ आ॒हे तीत्या॑ह॒ सहः॑ । \newline
42. आ॒ह॒ सहः॒ सह॑ आहाह॒ सहो॒ हि हि सह॑ आहाह॒ सहो॒ हि । \newline
43. सहो॒ हि हि सहः॒ सहो॒ ह्ये॑ता ए॒ता हि सहः॒ सहो॒ ह्ये॑ताः । \newline
44. ह्ये॑ता ए॒ता हि ह्ये॑ता ऊर्ज॒ ऊर्ज॑ ए॒ता हि ह्ये॑ता ऊर्जः॑ । \newline
45. ए॒ता ऊर्ज॒ ऊर्ज॑ ए॒ता ए॒ता ऊर्जः॑ स्थ॒ स्थोर्ज॑ ए॒ता ए॒ता ऊर्जः॑ स्थ । \newline
46. ऊर्जः॑ स्थ॒ स्थोर्ज॒ ऊर्जः॒ स्थोर्ज॒ मूर्ज॒(ग्ग्॒) स्थोर्ज॒ ऊर्जः॒ स्थोर्ज᳚म् । \newline
47. स्थोर्ज॒ मूर्ज(ग्ग्॑) स्थ॒ स्थोर्जं॑ ॅवो व॒ ऊर्ज(ग्ग्॑) स्थ॒ स्थोर्जं॑ ॅवः । \newline
48. ऊर्जं॑ ॅवो व॒ ऊर्ज॒ मूर्जं॑ ॅवो भक्षीय भक्षीय व॒ ऊर्ज॒ मूर्जं॑ ॅवो भक्षीय । \newline
49. वो॒ भ॒क्षी॒य॒ भ॒क्षी॒य॒ वो॒ वो॒ भ॒क्षी॒ये तीति॑ भक्षीय वो वो भक्षी॒ये ति॑ । \newline
50. भ॒क्षी॒ये तीति॑ भक्षीय भक्षी॒ये त्या॑हा॒हे ति॑ भक्षीय भक्षी॒ये त्या॑ह । \newline
51. इत्या॑हा॒हे तीत्या॒होर्ज॒ ऊर्ज॑ आ॒हे तीत्या॒होर्जः॑ । \newline
\pagebreak
\markright{ TS 1.5.8.2  \hfill https://www.vedavms.in \hfill}
\addcontentsline{toc}{section}{ TS 1.5.8.2 }
\section*{ TS 1.5.8.2 }

\textbf{TS 1.5.8.2 } \newline
\textbf{Samhita Paata} \newline

-होर्जो॒ ह्ये॑ता रेव॑ती॒ रम॑द्ध्व॒मित्या॑ह प॒शवो॒ वै रे॒वतीः᳚ प॒शूने॒वात्मन् र॑मयत इ॒हैव स्ते॒तो माऽप॑ गा॒तेत्या॑ह ध्रु॒वा ए॒वैना॒ अन॑पगाः कुरुत इष्टक॒चिद्वा अ॒न्यो᳚ऽग्निः प॑शु॒चिद॒न्यः सꣳ॑हि॒तासि॑ विश्वरू॒पीरिति॑ व॒थ्सम॒भि मृ॑श॒त्युपै॒वैनं॑ धत्ते पशु॒चित॑मेनं कुरुते॒ प्र - [ ] \newline

\textbf{Pada Paata} \newline

आ॒ह॒ । ऊर्जः॑ । हि । ए॒ताः । रेव॑तीः । रम॑द्ध्वम् । इति॑ । आ॒ह॒ । प॒शवः॑ । वै । रे॒वतीः᳚ । प॒शून् । ए॒व । आ॒त्मन्न् । र॒म॒य॒ते॒ । इ॒ह । ए॒व । स्त॒ । इ॒तः । मा । अपेति॑ । गा॒त॒ । इति॑ । आ॒ह॒ । ध्रु॒वाः । ए॒व । ए॒नाः॒ । अन॑पगा॒ इत्यन॑प-गाः॒ । कु॒रु॒ते॒ । इ॒ष्ट॒क॒चिदिती᳚ष्टक-चित् । वै । अ॒न्यः । अ॒ग्निः । प॒शु॒चिदिति॑ पशु - चित् । अ॒न्यः । सꣳ॒॒हि॒तेति॑ सं - हि॒ता । अ॒सि॒ । वि॒श्व॒रू॒पीरिति॑ विश्व - रू॒पीः । इति॑ । व॒थ्सम् । अ॒भीति॑ । मृ॒श॒ति॒ । उपेति॑ । ए॒व । ए॒न॒म् । ध॒त्ते॒ । प॒शु॒चित॒मिति॑ पशु - चित᳚म् । ए॒न॒म् । कु॒रु॒ते॒ । प्रेति॑ ।  \newline


\textbf{Krama Paata} \newline

आ॒होर्जः॑ । ऊर्जो॒ हि । ह्ये॑ताः । ए॒ता रेव॑तीः । रेव॑ती॒ रम॑द्ध्वम् । रम॑द्ध्व॒मिति॑ । इत्या॑ह । आ॒ह॒ प॒शवः॑ । प॒शवो॒ वै । वै रे॒वतीः᳚ । रे॒वतीः᳚ प॒शून् । प॒शूने॒व । ए॒वात्मन्न् । आ॒त्मन् र॑मयते । र॒म॒य॒त॒ इ॒ह । इ॒हैव । ए॒व स्त॑ । स्ते॒तः । इ॒तो मा । माऽप॑ । अप॑ गात । गा॒तेति॑ । इत्या॑ह । आ॒ह॒ ध्रु॒वाः । ध्रु॒वा ए॒व । ए॒वैनाः᳚ । ए॒ना॒ अन॑पगाः । अन॑पगाः कुरुते । अन॑पगा॒ इत्यन॑प - गाः॒ । कु॒रु॒त॒ इ॒ष्ट॒क॒चित् । इ॒ष्ट॒क॒चिद् वै । इ॒ष्ट॒क॒चिदिती᳚ष्टक - चित् । वा अ॒न्यः । अ॒न्यो᳚ऽग्निः । अ॒ग्निः प॑शु॒चित् । प॒शु॒चिद॒न्यः । प॒शु॒चिदिति॑ पशु - चित् । अ॒न्यः सꣳ॑हि॒ता । सꣳ॒॒हि॒ताऽसि॑ । सꣳ॒॒हि॒तेति॑ सम् - हि॒ता । अ॒सि॒ वि॒श्व॒रू॒पीः । वि॒श्व॒रू॒पीरिति॑ । वि॒श्व॒रू॒पीरिति॑ विश्व - रू॒पीः । इति॑ व॒थ्सम् । व॒थ्सम॒भि । अ॒भि मृ॑शति । मृ॒श॒त्युप॑ । उपै॒व । ए॒वैन᳚म् । ए॒न॒म् ध॒त्ते॒ । ध॒त्ते॒ प॒शु॒चित᳚म् । प॒शु॒चित॑मेनम् । प॒शु॒चित॒मिति॑ पशु - चित᳚म् । ए॒न॒म् कु॒रु॒ते॒ । कु॒रु॒ते॒ प्र । प्र वै \newline

\textbf{Jatai Paata} \newline

1. आ॒होर्ज॒ ऊर्ज॑ आहा॒होर्जः॑ । \newline
2. ऊर्जो॒ हि ह्यूर्ज॒ ऊर्जो॒ हि । \newline
3. ह्ये॑ता ए॒ता हि ह्ये॑ताः । \newline
4. ए॒ता रेव॑ती॒ रेव॑तीरे॒ता ए॒ता रेव॑तीः । \newline
5. रेव॑ती॒ रम॑द्ध्व॒(ग्म्॒) रम॑द्ध्व॒(ग्म्॒) रेव॑ती॒ रेव॑ती॒ रम॑द्ध्वम् । \newline
6. रम॑द्ध्व॒ मितीति॒ रम॑द्ध्व॒(ग्म्॒) रम॑द्ध्व॒ मिति॑ । \newline
7. इत्या॑हा॒हे तीत्या॑ह । \newline
8. आ॒ह॒ प॒शवः॑ प॒शव॑ आहाह प॒शवः॑ । \newline
9. प॒शवो॒ वै वै प॒शवः॑ प॒शवो॒ वै । \newline
10. वै रे॒वती॑ रे॒वती॒र् वै वै रे॒वतीः᳚ । \newline
11. रे॒वतीः᳚ प॒शून् प॒शून् रे॒वती॑ रे॒वतीः᳚ प॒शून् । \newline
12. प॒शू ने॒वैव प॒शून् प॒शू ने॒व । \newline
13. ए॒वात्म न्ना॒त्म न्ने॒वैवात्मन्न् । \newline
14. आ॒त्मन् र॑मयते रमयत आ॒त्मन्ना॒त्मन् र॑मयते । \newline
15. र॒म॒य॒त॒ इ॒हे ह र॑मयते रमयत इ॒ह । \newline
16. इ॒हैवैवे हे हैव । \newline
17. ए॒व स्त॑ स्तै॒वैव स्त॑ । \newline
18. स्ते॒ त इ॒तः स्त॑ स्ते॒ तः । \newline
19. इ॒तो मा मेत इ॒तो मा । \newline
20. मा ऽपाप॒ मा मा ऽप॑ । \newline
21. अप॑ गात गा॒तापाप॑ गात । \newline
22. गा॒ते तीति॑ गात गा॒ते ति॑ । \newline
23. इत्या॑हा॒हे तीत्या॑ह । \newline
24. आ॒ह॒ ध्रु॒वा ध्रु॒वा आ॑हाह ध्रु॒वाः । \newline
25. ध्रु॒वा ए॒वैव ध्रु॒वा ध्रु॒वा ए॒व । \newline
26. ए॒वैना॑ एना ए॒वैवैनाः᳚ । \newline
27. ए॒ना॒ अन॑पगा॒ अन॑पगा एना एना॒ अन॑पगाः । \newline
28. अन॑पगाः कुरुते कुरु॒ते ऽन॑पगा॒ अन॑पगाः कुरुते । \newline
29. अन॑पगा॒ इत्यन॑प - गाः॒ । \newline
30. कु॒रु॒त॒ इ॒ष्ट॒क॒चि दि॑ष्टक॒चित् कु॑रुते कुरुत इष्टक॒चित् । \newline
31. इ॒ष्ट॒क॒चिद् वै वा इ॑ष्टक॒चि दि॑ष्टक॒चिद् वै । \newline
32. इ॒ष्ट॒क॒चिदिती᳚ष्टक - चित् । \newline
33. वा अ॒न्यो᳚ ऽन्यो वै वा अ॒न्यः । \newline
34. अ॒न्यो᳚ ऽग्नि र॒ग्निर॒न्यो᳚(1॒) ऽन्यो᳚ ऽग्निः । \newline
35. अ॒ग्निः प॑शु॒चित् प॑शु॒चि द॒ग्निर॒ग्निः प॑शु॒चित् । \newline
36. प॒शु॒चिद॒न्यो᳚ ऽन्यः प॑शु॒चित् प॑शु॒चिद॒न्यः । \newline
37. प॒शु॒चिदिति॑ पशु - चित् । \newline
38. अ॒न्यः स(ग्म्॑)हि॒ता स(ग्म्॑)हि॒ता ऽन्यो᳚ ऽन्यः स(ग्म्॑)हि॒ता । \newline
39. स॒(ग्म्॒)हि॒ता ऽस्य॑सि सꣳहि॒ता स(ग्म्॑)हि॒ता ऽसि॑ । \newline
40. स॒(ग्म्॒)हि॒तेति॑ सं - हि॒ता । \newline
41. अ॒सि॒ वि॒श्व॒रू॒पीर् वि॑श्वरू॒पी र॑स्यसि विश्वरू॒पीः । \newline
42. वि॒श्व॒रू॒पीरितीति॑ विश्वरू॒पीर् वि॑श्वरू॒पीरिति॑ । \newline
43. वि॒श्व॒रू॒पीरिति॑ विश्व - रू॒पीः । \newline
44. इति॑ व॒थ्सं ॅव॒थ्स मितीति॑ व॒थ्सम् । \newline
45. व॒थ्स म॒भ्य॑भि व॒थ्सं ॅव॒थ्स म॒भि । \newline
46. अ॒भि मृ॑शति मृशत्य॒भ्य॑भि मृ॑शति । \newline
47. मृ॒श॒त्युपोप॑ मृशति मृश॒त्युप॑ । \newline
48. उपै॒वैवोपोपै॒व । \newline
49. ए॒वैन॑ मेन मे॒वैवैन᳚म् । \newline
50. ए॒न॒म् ध॒त्ते॒ ध॒त्त॒ ए॒न॒ मे॒न॒म् ध॒त्ते॒ । \newline
51. ध॒त्ते॒ प॒शु॒चित॑म् पशु॒चित॑म् धत्ते धत्ते पशु॒चित᳚म् । \newline
52. प॒शु॒चित॑ मेन मेनम् पशु॒चित॑म् पशु॒चित॑ मेनम् । \newline
53. प॒शु॒चित॒मिति॑ पशु - चित᳚म् । \newline
54. ए॒न॒म् कु॒रु॒ते॒ कु॒रु॒त॒ ए॒न॒ मे॒न॒म् कु॒रु॒ते॒ । \newline
55. कु॒रु॒ते॒ प्र प्र कु॑रुते कुरुते॒ प्र । \newline
56. प्र वै वै प्र प्र वै । \newline

\textbf{Ghana Paata } \newline

1. आ॒होर्ज॒ ऊर्ज॑ आहा॒होर्जो॒ हि ह्यूर्ज॑ आहा॒होर्जो॒ हि । \newline
2. ऊर्जो॒ हि ह्यूर्ज॒ ऊर्जो॒ ह्ये॑ता ए॒ता ह्यूर्ज॒ ऊर्जो॒ ह्ये॑ताः । \newline
3. ह्ये॑ता ए॒ता हि ह्ये॑ता रेव॑ती॒ रेव॑तीरे॒ता हि ह्ये॑ता रेव॑तीः । \newline
4. ए॒ता रेव॑ती॒ रेव॑तीरे॒ता ए॒ता रेव॑ती॒ रम॑द्ध्व॒(ग्म्॒) रम॑द्ध्व॒(ग्म्॒) रेव॑तीरे॒ता ए॒ता रेव॑ती॒ रम॑द्ध्वम् । \newline
5. रेव॑ती॒ रम॑द्ध्व॒(ग्म्॒) रम॑द्ध्व॒(ग्म्॒) रेव॑ती॒ रेव॑ती॒ रम॑द्ध्व॒ मितीति॒ रम॑द्ध्व॒(ग्म्॒) रेव॑ती॒ रेव॑ती॒ रम॑द्ध्व॒ मिति॑ । \newline
6. रम॑द्ध्व॒ मितीति॒ रम॑द्ध्व॒(ग्म्॒) रम॑द्ध्व॒ मित्या॑हा॒हे ति॒ रम॑द्ध्व॒(ग्म्॒) रम॑द्ध्व॒ मित्या॑ह । \newline
7. इत्या॑हा॒हे तीत्या॑ह प॒शवः॑ प॒शव॑ आ॒हे तीत्या॑ह प॒शवः॑ । \newline
8. आ॒ह॒ प॒शवः॑ प॒शव॑ आहाह प॒शवो॒ वै वै प॒शव॑ आहाह प॒शवो॒ वै । \newline
9. प॒शवो॒ वै वै प॒शवः॑ प॒शवो॒ वै रे॒वती॑ रे॒वती॒र् वै प॒शवः॑ प॒शवो॒ वै रे॒वतीः᳚ । \newline
10. वै रे॒वती॑ रे॒वती॒र् वै वै रे॒वतीः᳚ प॒शून् प॒शून् रे॒वती॒र् वै वै रे॒वतीः᳚ प॒शून् । \newline
11. रे॒वतीः᳚ प॒शून् प॒शून् रे॒वती॑ रे॒वतीः᳚ प॒शू ने॒वैव प॒शून् रे॒वती॑ रे॒वतीः᳚ प॒शू ने॒व । \newline
12. प॒शू ने॒वैव प॒शून् प॒शू ने॒वात्मन् ना॒त्मन् ने॒व प॒शून् प॒शू ने॒वात्मन्न् । \newline
13. ए॒वात्मन् ना॒त्मन् ने॒वैवात्मन् र॑मयते रमयत आ॒त्मन् ने॒वैवात्मन् र॑मयते । \newline
14. आ॒त्मन् र॑मयते रमयत आ॒त्मन् ना॒त्मन् र॑मयत इ॒हे ह र॑मयत आ॒त्मन् ना॒त्मन् र॑मयत इ॒ह । \newline
15. र॒म॒य॒त॒ इ॒हे ह र॑मयते रमयत इ॒हैवैवे ह र॑मयते रमयत इ॒हैव । \newline
16. इ॒हैवैवे हे हैव स्त॑ स्तै॒वे हे हैव स्त॑ । \newline
17. ए॒व स्त॑ स्तै॒वैव स्ते॒ त इ॒तः स्तै॒वैव स्ते॒ तः । \newline
18. स्ते॒ त इ॒तः स्त॑ स्ते॒ तो मा मेतः स्त॑ स्ते॒ तो मा । \newline
19. इ॒तो मा मेत इ॒तो मा ऽपाप॒ मेत इ॒तो मा ऽप॑ । \newline
20. मा ऽपाप॒ मा मा ऽप॑ गात गा॒ताप॒ मा मा ऽप॑ गात । \newline
21. अप॑ गात गा॒तापाप॑ गा॒ते तीति॑ गा॒तापाप॑ गा॒ते ति॑ । \newline
22. गा॒ते तीति॑ गात गा॒ते त्या॑हा॒हे ति॑ गात गा॒ते त्या॑ह । \newline
23. इत्या॑हा॒हे तीत्या॑ह ध्रु॒वा ध्रु॒वा आ॒हे तीत्या॑ह ध्रु॒वाः । \newline
24. आ॒ह॒ ध्रु॒वा ध्रु॒वा आ॑हाह ध्रु॒वा ए॒वैव ध्रु॒वा आ॑हाह ध्रु॒वा ए॒व । \newline
25. ध्रु॒वा ए॒वैव ध्रु॒वा ध्रु॒वा ए॒वैना॑ एना ए॒व ध्रु॒वा ध्रु॒वा ए॒वैनाः᳚ । \newline
26. ए॒वैना॑ एना ए॒वैवैना॒ अन॑पगा॒ अन॑पगा एना ए॒वैवैना॒ अन॑पगाः । \newline
27. ए॒ना॒ अन॑पगा॒ अन॑पगा एना एना॒ अन॑पगाः कुरुते कुरु॒ते ऽन॑पगा एना एना॒ अन॑पगाः कुरुते । \newline
28. अन॑पगाः कुरुते कुरु॒ते ऽन॑पगा॒ अन॑पगाः कुरुत इष्टक॒चि दि॑ष्टक॒चित् कु॑रु॒ते ऽन॑पगा॒ अन॑पगाः कुरुत इष्टक॒चित् । \newline
29. अन॑पगा॒ इत्यन॑प - गाः॒ । \newline
30. कु॒रु॒त॒ इ॒ष्ट॒क॒चि दि॑ष्टक॒चित् कु॑रुते कुरुत इष्टक॒चिद् वै वा इ॑ष्टक॒चित् कु॑रुते कुरुत इष्टक॒चिद् वै । \newline
31. इ॒ष्ट॒क॒चिद् वै वा इ॑ष्टक॒चि दि॑ष्टक॒चिद् वा अ॒न्यो᳚ ऽन्यो वा इ॑ष्टक॒चि दि॑ष्टक॒चिद् वा अ॒न्यः । \newline
32. इ॒ष्ट॒क॒चिदिती᳚ष्टक - चित् । \newline
33. वा अ॒न्यो᳚ ऽन्यो वै वा अ॒न्यो᳚ ऽग्निर॒ग्निर॒न्यो वै वा अ॒न्यो᳚ ऽग्निः । \newline
34. अ॒न्यो᳚ ऽग्नि र॒ग्निर॒न्यो᳚(1॒) ऽन्यो᳚ ऽग्निः प॑शु॒चित् प॑शु॒चिद॒ग्नि र॒न्यो᳚(1॒) ऽन्यो᳚ ऽग्निः प॑शु॒चित् । \newline
35. अ॒ग्निः प॑शु॒चित् प॑शु॒चि द॒ग्निर॒ग्निः प॑शु॒चिद॒न्यो᳚ ऽन्यः प॑शु॒चि द॒ग्निर॒ग्निः प॑शु॒चिद॒न्यः । \newline
36. प॒शु॒चिद॒न्यो᳚ ऽन्यः प॑शु॒चित् प॑शु॒चिद॒न्यः स(ग्म्॑)हि॒ता स(ग्म्॑)हि॒ता ऽन्यः प॑शु॒चित् प॑शु॒चिद॒न्यः स(ग्म्॑)हि॒ता । \newline
37. प॒शु॒चिदिति॑ पशु - चित् । \newline
38. अ॒न्यः स(ग्म्॑)हि॒ता स(ग्म्॑)हि॒ता ऽन्यो᳚ ऽन्यः स(ग्म्॑)हि॒ता ऽस्य॑सि सꣳहि॒ता ऽन्यो᳚ ऽन्यः स(ग्म्॑)हि॒ता ऽसि॑ । \newline
39. स॒(ग्म्॒)हि॒ता ऽस्य॑सि सꣳहि॒ता स(ग्म्॑)हि॒ता ऽसि॑ विश्वरू॒पीर् वि॑श्वरू॒पीर॑सि सꣳहि॒ता स(ग्म्॑)हि॒ता ऽसि॑ विश्वरू॒पीः । \newline
40. स॒(ग्म्॒)हि॒तेति॑ सं - हि॒ता । \newline
41. अ॒सि॒ वि॒श्व॒रू॒पीर् वि॑श्वरू॒पी र॑स्यसि विश्वरू॒पी रितीति॑ विश्वरू॒पीर॑स्यसि विश्वरू॒पीरिति॑ । \newline
42. वि॒श्व॒रू॒पीरितीति॑ विश्वरू॒पीर् वि॑श्वरू॒पीरिति॑ व॒थ्सं ॅव॒थ्स मिति॑ विश्वरू॒पीर् वि॑श्वरू॒पीरिति॑ व॒थ्सम् । \newline
43. वि॒श्व॒रू॒पीरिति॑ विश्व - रू॒पीः । \newline
44. इति॑ व॒थ्सं ॅव॒थ्स मितीति॑ व॒थ्स म॒भ्य॑भि व॒थ्स मितीति॑ व॒थ्स म॒भि । \newline
45. व॒थ्स म॒भ्य॑भि व॒थ्सं ॅव॒थ्स म॒भि मृ॑शति मृशत्य॒भि व॒थ्सं ॅव॒थ्स म॒भि मृ॑शति । \newline
46. अ॒भि मृ॑शति मृशत्य॒भ्य॑भि मृ॑श॒त्युपोप॑ मृशत्य॒भ्य॑भि मृ॑श॒त्युप॑ । \newline
47. मृ॒श॒त्युपोप॑ मृशति मृश॒त्युपै॒वैवोप॑ मृशति मृश॒त्युपै॒व । \newline
48. उपै॒वैवोपोपै॒वैन॑ मेन मे॒वोपोपै॒वैन᳚म् । \newline
49. ए॒वैन॑ मेन मे॒वैवैन॑म् धत्ते धत्त एन मे॒वैवैन॑म् धत्ते । \newline
50. ए॒न॒म् ध॒त्ते॒ ध॒त्त॒ ए॒न॒ मे॒न॒म् ध॒त्ते॒ प॒शु॒चित॑म् पशु॒चित॑म् धत्त एन मेनम् धत्ते पशु॒चित᳚म् । \newline
51. ध॒त्ते॒ प॒शु॒चित॑म् पशु॒चित॑म् धत्ते धत्ते पशु॒चित॑ मेन मेनम् पशु॒चित॑म् धत्ते धत्ते पशु॒चित॑ मेनम् । \newline
52. प॒शु॒चित॑ मेन मेनम् पशु॒चित॑म् पशु॒चित॑ मेनम् कुरुते कुरुत एनम् पशु॒चित॑म् पशु॒चित॑ मेनम् कुरुते । \newline
53. प॒शु॒चित॒मिति॑ पशु - चित᳚म् । \newline
54. ए॒न॒म् कु॒रु॒ते॒ कु॒रु॒त॒ ए॒न॒ मे॒न॒म् कु॒रु॒ते॒ प्र प्र कु॑रुत एन मेनम् कुरुते॒ प्र । \newline
55. कु॒रु॒ते॒ प्र प्र कु॑रुते कुरुते॒ प्र वै वै प्र कु॑रुते कुरुते॒ प्र वै । \newline
56. प्र वै वै प्र प्र वा ए॒ष ए॒ष वै प्र प्र वा ए॒षः । \newline
\pagebreak
\markright{ TS 1.5.8.3  \hfill https://www.vedavms.in \hfill}
\addcontentsline{toc}{section}{ TS 1.5.8.3 }
\section*{ TS 1.5.8.3 }

\textbf{TS 1.5.8.3 } \newline
\textbf{Samhita Paata} \newline

वा ए॒षो᳚ऽस्माल्लो॒काच्च्य॑वते॒ य आ॑हव॒नीय॑-मुप॒तिष्ठ॑ते॒ गार्.ह॑पत्य॒मुप॑ तिष्ठते॒ ऽस्मिन्ने॒व लो॒के प्रति॑ तिष्ठ॒त्यथो॒ गार्.ह॑पत्यायै॒व नि ह्नु॑ते गाय॒त्रीभि॒रुप॑ तिष्ठते॒ तेजो॒ वै गा॑य॒त्री तेज॑ ए॒वात्मन् ध॒त्तेऽथो॒ यदे॒तं तृ॒चम॒न्वाह॒ सन्त॑त्यै॒ गार्.ह॑पत्यं॒ ॅवा अनु॑ द्वि॒पादो॑ वी॒राः प्र जा॑यन्ते॒ य ए॒वं ॅवि॒द्वान् द्वि॒पदा॑भि॒र् गार्.ह॑पत्य-मुप॒तिष्ठ॑त॒ - [ ] \newline

\textbf{Pada Paata} \newline

वै । ए॒षः । अ॒स्मात् । लो॒कात् । च्य॒व॒ते॒ । यः । आ॒ह॒व॒नीय॒मित्या᳚ - ह॒व॒नीय᳚म् । उ॒प॒तिष्ठ॑त॒ इत्यु॑प - तिष्ठ॑ते । गार्.ह॑पत्य॒मिति॒ गार्.ह॑ - प॒त्य॒म् । उपेति॑ । ति॒ष्ठ॒ते॒ । अ॒स्मिन्न् । ए॒व । लो॒के । प्रतीति॑ । ति॒ष्ठ॒ति॒ । अथो॒ इति॑ । गार्.ह॑पत्या॒येति॒ गार्.ह॑ - प॒त्या॒य॒ । ए॒व । नीति॑ । ह्नु॒ते॒ । गा॒य॒त्रीभिः॑ । उपेति॑ । ति॒ष्ठ॒ते॒ । तेजः॑ । वै । गा॒य॒त्री । तेजः॑ । ए॒व । आ॒त्मन्न् । ध॒त्ते॒ । अथो॒ इति॑ । यत् । ए॒तम् । तृ॒चम् । अ॒न्वाहेत्य॑नु - आह॑ । संत॑त्या॒ इति॒ सं - त॒त्यै॒ । गार्.ह॑पत्य॒मिति॒ गार्.ह॑ - प॒त्य॒म् । वै । अन्विति॑ । द्वि॒पाद॒ इति॑ द्वि॒ - पादः॑ । वी॒राः । प्रेति॑ । जा॒य॒न्ते॒ । यः । ए॒वम् । वि॒द्वान् । द्वि॒पदा॑भि॒रिति॑ द्वि - पदा॑भिः । गार्.ह॑पत्य॒मिति॒ गार्.ह॑ - प॒त्य॒म् । उ॒प॒तिष्ठ॑त॒ इत्यु॑प - तिष्ठ॑ते ।  \newline


\textbf{Krama Paata} \newline

वा ए॒षः । ए॒षो᳚ऽस्मात् । अ॒स्माल्लो॒कात् । लो॒काच्च्य॑वते । च्य॒व॒ते॒ यः । य आ॑हव॒नीय᳚म् । आ॒ह॒व॒नीय॑मुप॒तिष्ठ॑ते । आ॒ह॒व॒नीय॒मित्या᳚ - ह॒व॒नीय᳚म् । उ॒प॒तिष्ठ॑ते॒ गार्.ह॑पत्यम् । उ॒प॒तिष्ठ॑त॒ इत्यु॑प - तिष्ठ॑ते । गार्.ह॑पत्य॒मुप॑ । गार्.ह॑पत्य॒मिति॒ गार्.ह॑ - प॒त्य॒म् । उप॑ तिष्ठते । ति॒ष्ठ॒ते॒ऽस्मिन्न् । अ॒स्मिन्ने॒व । ए॒वलो॒के । लो॒के प्रति॑ । प्रति॑ तिष्ठति । ति॒ष्ठ॒त्यथो᳚ । अथो॒ गार्.ह॑पत्याय । अथो॒ इत्यथो᳚ । गार्.ह॑पत्यायै॒व । गार्.ह॑पत्या॒येति॒ गार्.ह॑ - प॒त्या॒य॒ । ए॒व नि । नि,ह्नु॑ते । ह्नु॒ते॒ गा॒य॒त्रीभिः॑ । गा॒य॒त्रीभि॒रुप॑ । उप॑ तिष्ठते । ति॒ष्ठ॒ते॒ तेजः॑ । तेजो॒ वै । वै गा॑य॒त्री । गा॒य॒त्री तेजः॑ । तेज॑ ए॒व । ए॒वात्मन्न् । आ॒त्मन् ध॑त्ते । ध॒त्तेऽथो᳚ । अथो॒ यत् । अथो॒ इत्यथो᳚ । यदे॒तम् । ए॒तम् तृ॒चम् । तृ॒चम॒न्वाह॑ । अ॒न्वाह॒ सन्त॑त्यै । अ॒न्वाहेत्य॑नु - आह॑ । सन्त॑त्यै॒ गार्.ह॑पत्यम् । सन्त॑त्या॒ इति॒ सम् - त॒त्यै॒ । गार्.ह॑पत्यं॒ ॅवै । गार्.ह॑पत्य॒मिति॒ गार्.ह॑ - प॒त्य॒म् । वा अनु॑ । अनु॑ द्वि॒पादः॑ । द्वि॒पादो॑ वी॒राः । द्वि॒पाद॒ इति॑ द्वि - पादः॑ । वी॒राः प्र । प्र जा॑यन्ते । जा॒य॒न्ते॒ यः । य ए॒वम् । ए॒वं ॅवि॒द्वान् । वि॒द्वान् द्वि॒पदा॑भिः । द्वि॒पदा॑भि॒र् गार्.ह॑पत्यम् । द्वि॒पदा॑भि॒रिति॑ द्वि - पदा॑भिः । गार्.ह॑पत्यमुप॒तिष्ठ॑ते । गार्.ह॑पत्य॒मिति॒ गार्.ह॑ - प॒त्य॒म् । उ॒प॒तिष्ठ॑त॒ आ । उ॒प॒तिष्ठ॑त॒ इत्यु॑प - तिष्ठ॑ते \newline

\textbf{Jatai Paata} \newline

1. वा ए॒ष ए॒ष वै वा ए॒षः । \newline
2. ए॒षो᳚ऽस्मा द॒स्मादे॒ष ए॒षो᳚ऽस्मात् । \newline
3. अ॒स्मा ल्लो॒का ल्लो॒का द॒स्मा द॒स्मा ल्लो॒कात् । \newline
4. लो॒काच् च्य॑वते च्यवते लो॒का ल्लो॒काच् च्य॑वते । \newline
5. च्य॒व॒ते॒ यो यश्च्य॑वते च्यवते॒ यः । \newline
6. य आ॑हव॒नीय॑ माहव॒नीयं॒ ॅयो य आ॑हव॒नीय᳚म् । \newline
7. आ॒ह॒व॒नीय॑ मुप॒तिष्ठ॑त उप॒तिष्ठ॑त आहव॒नीय॑ माहव॒नीय॑ मुप॒तिष्ठ॑ते । \newline
8. आ॒ह॒व॒नीय॒मित्या᳚ - ह॒व॒नीय᳚म् । \newline
9. उ॒प॒तिष्ठ॑ते॒ गार्.ह॑पत्य॒म् गार्.ह॑पत्य मुप॒तिष्ठ॑त उप॒तिष्ठ॑ते॒ गार्.ह॑पत्यम् । \newline
10. उ॒प॒तिष्ठ॑त॒ इत्यु॑प - तिष्ठ॑ते । \newline
11. गार्.ह॑पत्य॒ मुपोप॒ गार्.ह॑पत्य॒म् गार्.ह॑पत्य॒ मुप॑ । \newline
12. गार्.ह॑पत्य॒मिति॒ गार्.ह॑ - प॒त्य॒म् । \newline
13. उप॑ तिष्ठते तिष्ठत॒ उपोप॑ तिष्ठते । \newline
14. ति॒ष्ठ॒ते॒ ऽस्मिन्न॒स्मिन् ति॑ष्ठते तिष्ठते॒ ऽस्मिन्न् । \newline
15. अ॒स्मि न्ने॒वैवास्मि न्न॒स्मिन्ने॒व । \newline
16. ए॒व लो॒के लो॒क ए॒वैव लो॒के । \newline
17. लो॒के प्रति॒ प्रति॑ लो॒के लो॒के प्रति॑ । \newline
18. प्रति॑ तिष्ठति तिष्ठति॒ प्रति॒ प्रति॑ तिष्ठति । \newline
19. ति॒ष्ठ॒त्यथो॒ अथो॑ तिष्ठति तिष्ठ॒त्यथो᳚ । \newline
20. अथो॒ गार्.ह॑पत्याय॒ गार्.ह॑पत्या॒याथो॒ अथो॒ गार्.ह॑पत्याय । \newline
21. अथो॒ इत्यथो᳚ । \newline
22. गार्.ह॑पत्यायै॒वैव गार्.ह॑पत्याय॒ गार्.ह॑पत्यायै॒व । \newline
23. गार्.ह॑पत्या॒येति॒ गार्.ह॑ - प॒त्या॒य॒ । \newline
24. ए॒व नि न्ये॑वैव नि । \newline
25. नि ह्नु॑ते ह्नुते॒ नि नि ह्नु॑ते । \newline
26. ह्नु॒ते॒ गा॒य॒त्रीभि॑र् गाय॒त्रीभि॑र् ह्नुते ह्नुते गाय॒त्रीभिः॑ । \newline
27. गा॒य॒त्रीभि॒रुपोप॑ गाय॒त्रीभि॑र् गाय॒त्रीभि॒रुप॑ । \newline
28. उप॑ तिष्ठते तिष्ठत॒ उपोप॑ तिष्ठते । \newline
29. ति॒ष्ठ॒ते॒ तेज॒ स्तेज॑ स्तिष्ठते तिष्ठते॒ तेजः॑ । \newline
30. तेजो॒ वै वै तेज॒स्तेजो॒ वै । \newline
31. वै गा॑य॒त्री गा॑य॒त्री वै वै गा॑य॒त्री । \newline
32. गा॒य॒त्री तेज॒स्तेजो॑ गाय॒त्री गा॑य॒त्री तेजः॑ । \newline
33. तेज॑ ए॒वैव तेज॒स्तेज॑ ए॒व । \newline
34. ए॒वात्म न्ना॒त्म न्ने॒वैवात्मन्न् । \newline
35. आ॒त्मन् ध॑त्ते धत्त आ॒त्मन्ना॒त्मन् ध॑त्ते । \newline
36. ध॒त्ते ऽथो॒ अथो॑ धत्ते ध॒त्ते ऽथो᳚ । \newline
37. अथो॒ यद् यदथो॒ अथो॒ यत् । \newline
38. अथो॒ इत्यथो᳚ । \newline
39. यदे॒त मे॒तं ॅयद् यदे॒तम् । \newline
40. ए॒तम् तृ॒चम् तृ॒च मे॒त मे॒तम् तृ॒चम् । \newline
41. तृ॒च म॒न्वाहा॒न्वाह॑ तृ॒चम् तृ॒च म॒न्वाह॑ । \newline
42. अ॒न्वाह॒ सन्त॑त्यै॒ सन्त॑त्या अ॒न्वाहा॒न्वाह॒ सन्त॑त्यै । \newline
43. अ॒न्वाहेत्य॑नु - आह॑ । \newline
44. सन्त॑त्यै॒ गार्.ह॑पत्य॒म् गार्.ह॑पत्य॒(ग्म्॒) सन्त॑त्यै॒ सन्त॑त्यै॒ गार्.ह॑पत्यम् । \newline
45. सन्त॑त्या॒ इति॒ सं - त॒त्यै॒ । \newline
46. गार्.ह॑पत्यं॒ ॅवै वै गार्.ह॑पत्य॒म् गार्.ह॑पत्यं॒ ॅवै । \newline
47. गार्.ह॑पत्य॒मिति॒ गार्.ह॑ - प॒त्य॒म् । \newline
48. वा अन्वनु॒ वै वा अनु॑ । \newline
49. अनु॑ द्वि॒पादो᳚ द्वि॒पादो ऽन्वनु॑ द्वि॒पादः॑ । \newline
50. द्वि॒पादो॑ वी॒रा वी॒रा द्वि॒पादो᳚ द्वि॒पादो॑ वी॒राः । \newline
51. द्वि॒पाद॒ इति॑ द्वि - पादः॑ । \newline
52. वी॒राः प्र प्र वी॒रा वी॒राः प्र । \newline
53. प्र जा॑यन्ते जायन्ते॒ प्र प्र जा॑यन्ते । \newline
54. जा॒य॒न्ते॒ यो यो जा॑यन्ते जायन्ते॒ यः । \newline
55. य ए॒व मे॒वं ॅयो य ए॒वम् । \newline
56. ए॒वं ॅवि॒द्वान्. वि॒द्वा ने॒व मे॒वं ॅवि॒द्वान् । \newline
57. वि॒द्वान् द्वि॒पदा॑भिर् द्वि॒पदा॑भिर् वि॒द्वान्. वि॒द्वान् द्वि॒पदा॑भिः । \newline
58. द्वि॒पदा॑भि॒र् गार्.ह॑पत्य॒म् गार्.ह॑पत्यम् द्वि॒पदा॑भिर् द्वि॒पदा॑भि॒र् गार्.ह॑पत्यम् । \newline
59. द्वि॒पदा॑भि॒रिति॑ द्वि - पदा॑भिः । \newline
60. गार्.ह॑पत्य मुप॒तिष्ठ॑त उप॒तिष्ठ॑ते॒ गार्.ह॑पत्य॒म् गार्.ह॑पत्य मुप॒तिष्ठ॑ते । \newline
61. गार्.ह॑पत्य॒मिति॒ गार्.ह॑ - प॒त्य॒म् । \newline
62. उ॒प॒तिष्ठ॑त॒ ओप॒तिष्ठ॑त उप॒तिष्ठ॑त॒ आ । \newline
63. उ॒प॒तिष्ठ॑त॒ इत्यु॑प - तिष्ठ॑ते । \newline

\textbf{Ghana Paata } \newline

1. वा ए॒ष ए॒ष वै वा ए॒षो᳚ऽस्मा द॒स्मादे॒ष वै वा ए॒षो᳚ऽस्मात् । \newline
2. ए॒षो᳚ऽस्मा द॒स्मादे॒ष ए॒षो᳚ऽस्मा ल्लो॒का ल्लो॒का द॒स्मादे॒ष ए॒षो᳚ऽस्माल्लो॒कात् । \newline
3. अ॒स्मा ल्लो॒का ल्लो॒का द॒स्मा द॒स्मा ल्लो॒काच् च्य॑वते च्यवते लो॒का द॒स्मा द॒स्मा ल्लो॒काच् च्य॑वते । \newline
4. लो॒काच् च्य॑वते च्यवते लो॒का ल्लो॒काच् च्य॑वते॒ यो यश्च्य॑वते लो॒का ल्लो॒काच् च्य॑वते॒ यः । \newline
5. च्य॒व॒ते॒ यो यश्च्य॑वते च्यवते॒ य आ॑हव॒नीय॑ माहव॒नीयं॒ ॅयश्च्य॑वते च्यवते॒ य आ॑हव॒नीय᳚म् । \newline
6. य आ॑हव॒नीय॑ माहव॒नीयं॒ ॅयो य आ॑हव॒नीय॑ मुप॒तिष्ठ॑त उप॒तिष्ठ॑त आहव॒नीयं॒ ॅयो य आ॑हव॒नीय॑ मुप॒तिष्ठ॑ते । \newline
7. आ॒ह॒व॒नीय॑ मुप॒तिष्ठ॑त उप॒तिष्ठ॑त आहव॒नीय॑ माहव॒नीय॑ मुप॒तिष्ठ॑ते॒ गार्.ह॑पत्य॒म् गार्.ह॑पत्य मुप॒तिष्ठ॑त आहव॒नीय॑ माहव॒नीय॑ मुप॒तिष्ठ॑ते॒ गार्.ह॑पत्यम् । \newline
8. आ॒ह॒व॒नीय॒मित्या᳚ - ह॒व॒नीय᳚म् । \newline
9. उ॒प॒तिष्ठ॑ते॒ गार्.ह॑पत्य॒म् गार्.ह॑पत्य मुप॒तिष्ठ॑त उप॒तिष्ठ॑ते॒ गार्.ह॑पत्य॒ मुपोप॒ गार्.ह॑पत्य मुप॒तिष्ठ॑त उप॒तिष्ठ॑ते॒ गार्.ह॑पत्य॒ मुप॑ । \newline
10. उ॒प॒तिष्ठ॑त॒ इत्यु॑प - तिष्ठ॑ते । \newline
11. गार्.ह॑पत्य॒ मुपोप॒ गार्.ह॑पत्य॒म् गार्.ह॑पत्य॒ मुप॑ तिष्ठते तिष्ठत॒ उप॒ गार्.ह॑पत्य॒म् गार्.ह॑पत्य॒ मुप॑ तिष्ठते । \newline
12. गार्.ह॑पत्य॒मिति॒ गार्.ह॑ - प॒त्य॒म् । \newline
13. उप॑ तिष्ठते तिष्ठत॒ उपोप॑ तिष्ठते॒ ऽस्मिन् न॒स्मिन् ति॑ष्ठत॒ उपोप॑ तिष्ठते॒ ऽस्मिन्न् । \newline
14. ति॒ष्ठ॒ते॒ ऽस्मिन् न॒स्मिन् ति॑ष्ठते तिष्ठते॒ ऽस्मिन् ने॒वैवास्मिन् ति॑ष्ठते तिष्ठते॒ ऽस्मिन् ने॒व । \newline
15. अ॒स्मिन् ने॒वैवास्मिन् न॒स्मिन् ने॒व लो॒के लो॒क ए॒वास्मिन् न॒स्मिन् ने॒व लो॒के । \newline
16. ए॒व लो॒के लो॒क ए॒वैव लो॒के प्रति॒ प्रति॑ लो॒क ए॒वैव लो॒के प्रति॑ । \newline
17. लो॒के प्रति॒ प्रति॑ लो॒के लो॒के प्रति॑ तिष्ठति तिष्ठति॒ प्रति॑ लो॒के लो॒के प्रति॑ तिष्ठति । \newline
18. प्रति॑ तिष्ठति तिष्ठति॒ प्रति॒ प्रति॑ तिष्ठ॒त्यथो॒ अथो॑ तिष्ठति॒ प्रति॒ प्रति॑ तिष्ठ॒त्यथो᳚ । \newline
19. ति॒ष्ठ॒त्यथो॒ अथो॑ तिष्ठति तिष्ठ॒त्यथो॒ गार्.ह॑पत्याय॒ गार्.ह॑पत्या॒याथो॑ तिष्ठति तिष्ठ॒त्यथो॒ गार्.ह॑पत्याय । \newline
20. अथो॒ गार्.ह॑पत्याय॒ गार्.ह॑पत्या॒याथो॒ अथो॒ गार्.ह॑पत्यायै॒वैव गार्.ह॑पत्या॒याथो॒ अथो॒ गार्.ह॑पत्यायै॒व । \newline
21. अथो॒ इत्यथो᳚ । \newline
22. गार्.ह॑पत्यायै॒वैव गार्.ह॑पत्याय॒ गार्.ह॑पत्यायै॒व नि न्ये॑व गार्.ह॑पत्याय॒ गार्.ह॑पत्यायै॒व नि । \newline
23. गार्.ह॑पत्या॒येति॒ गार्.ह॑ - प॒त्या॒य॒ । \newline
24. ए॒व नि न्ये॑वैव नि ह्नु॑ते ह्नुते॒ न्ये॑वैव नि ह्नु॑ते । \newline
25. नि ह्नु॑ते ह्नुते॒ नि नि ह्नु॑ते गाय॒त्रीभि॑र् गाय॒त्रीभि॑र् ह्नुते॒ नि नि ह्नु॑ते गाय॒त्रीभिः॑ । \newline
26. ह्नु॒ते॒ गा॒य॒त्रीभि॑र् गाय॒त्रीभि॑र् ह्नुते ह्नुते गाय॒त्रीभि॒रुपोप॑ गाय॒त्रीभि॑र् ह्नुते ह्नुते गाय॒त्रीभि॒रुप॑ । \newline
27. गा॒य॒त्रीभि॒रुपोप॑ गाय॒त्रीभि॑र् गाय॒त्रीभि॒रुप॑ तिष्ठते तिष्ठत॒ उप॑ गाय॒त्रीभि॑र् गाय॒त्रीभि॒रुप॑ तिष्ठते । \newline
28. उप॑ तिष्ठते तिष्ठत॒ उपोप॑ तिष्ठते॒ तेज॒ स्तेज॑ स्तिष्ठत॒ उपोप॑ तिष्ठते॒ तेजः॑ । \newline
29. ति॒ष्ठ॒ते॒ तेज॒ स्तेज॑ स्तिष्ठते तिष्ठते॒ तेजो॒ वै वै तेज॑ स्तिष्ठते तिष्ठते॒ तेजो॒ वै । \newline
30. तेजो॒ वै वै तेज॒स्तेजो॒ वै गा॑य॒त्री गा॑य॒त्री वै तेज॒स्तेजो॒ वै गा॑य॒त्री । \newline
31. वै गा॑य॒त्री गा॑य॒त्री वै वै गा॑य॒त्री तेज॒स्तेजो॑ गाय॒त्री वै वै गा॑य॒त्री तेजः॑ । \newline
32. गा॒य॒त्री तेज॒स्तेजो॑ गाय॒त्री गा॑य॒त्री तेज॑ ए॒वैव तेजो॑ गाय॒त्री गा॑य॒त्री तेज॑ ए॒व । \newline
33. तेज॑ ए॒वैव तेज॒स्तेज॑ ए॒वात्मन् ना॒त्मन् ने॒व तेज॒स्तेज॑ ए॒वात्मन्न् । \newline
34. ए॒वात्मन् ना॒त्मन् ने॒वैवात्मन् ध॑त्ते धत्त आ॒त्मन् ने॒वैवात्मन् ध॑त्ते । \newline
35. आ॒त्मन् ध॑त्ते धत्त आ॒त्मन् ना॒त्मन् ध॒त्ते ऽथो॒ अथो॑ धत्त आ॒त्मन् ना॒त्मन् ध॒त्ते ऽथो᳚ । \newline
36. ध॒त्ते ऽथो॒ अथो॑ धत्ते ध॒त्ते ऽथो॒ यद् यदथो॑ धत्ते ध॒त्ते ऽथो॒ यत् । \newline
37. अथो॒ यद् यदथो॒ अथो॒ यदे॒त मे॒तं ॅयदथो॒ अथो॒ यदे॒तम् । \newline
38. अथो॒ इत्यथो᳚ । \newline
39. यदे॒त मे॒तं ॅयद् यदे॒तम् तृ॒चम् तृ॒च मे॒तं ॅयद् यदे॒तम् तृ॒चम् । \newline
40. ए॒तम् तृ॒चम् तृ॒च मे॒त मे॒तम् तृ॒च म॒न्वाहा॒न्वाह॑ तृ॒च मे॒त मे॒तम् तृ॒च म॒न्वाह॑ । \newline
41. तृ॒च म॒न्वाहा॒न्वाह॑ तृ॒चम् तृ॒च म॒न्वाह॒ सन्त॑त्यै॒ सन्त॑त्या अ॒न्वाह॑ तृ॒चम् तृ॒च म॒न्वाह॒ सन्त॑त्यै । \newline
42. अ॒न्वाह॒ सन्त॑त्यै॒ सन्त॑त्या अ॒न्वाहा॒न्वाह॒ सन्त॑त्यै॒ गार्.ह॑पत्य॒म् गार्.ह॑पत्य॒(ग्म्॒) सन्त॑त्या अ॒न्वाहा॒न्वाह॒ सन्त॑त्यै॒ गार्.ह॑पत्यम् । \newline
43. अ॒न्वाहेत्य॑नु - आह॑ । \newline
44. सन्त॑त्यै॒ गार्.ह॑पत्य॒म् गार्.ह॑पत्य॒(ग्म्॒) सन्त॑त्यै॒ सन्त॑त्यै॒ गार्.ह॑पत्यं॒ ॅवै वै गार्.ह॑पत्य॒(ग्म्॒) सन्त॑त्यै॒ सन्त॑त्यै॒ गार्.ह॑पत्यं॒ ॅवै । \newline
45. सन्त॑त्या॒ इति॒ सं - त॒त्यै॒ । \newline
46. गार्.ह॑पत्यं॒ ॅवै वै गार्.ह॑पत्य॒म् गार्.ह॑पत्यं॒ ॅवा अन्वनु॒ वै गार्.ह॑पत्य॒म् गार्.ह॑पत्यं॒ ॅवा अनु॑ । \newline
47. गार्.ह॑पत्य॒मिति॒ गार्.ह॑ - प॒त्य॒म् । \newline
48. वा अन्वनु॒ वै वा अनु॑ द्वि॒पादो᳚ द्वि॒पादो ऽनु॒ वै वा अनु॑ द्वि॒पादः॑ । \newline
49. अनु॑ द्वि॒पादो᳚ द्वि॒पादो ऽन्वनु॑ द्वि॒पादो॑ वी॒रा वी॒रा द्वि॒पादो ऽन्वनु॑ द्वि॒पादो॑ वी॒राः । \newline
50. द्वि॒पादो॑ वी॒रा वी॒रा द्वि॒पादो᳚ द्वि॒पादो॑ वी॒राः प्र प्र वी॒रा द्वि॒पादो᳚ द्वि॒पादो॑ वी॒राः प्र । \newline
51. द्वि॒पाद॒ इति॑ द्वि - पादः॑ । \newline
52. वी॒राः प्र प्र वी॒रा वी॒राः प्र जा॑यन्ते जायन्ते॒ प्र वी॒रा वी॒राः प्र जा॑यन्ते । \newline
53. प्र जा॑यन्ते जायन्ते॒ प्र प्र जा॑यन्ते॒ यो यो जा॑यन्ते॒ प्र प्र जा॑यन्ते॒ यः । \newline
54. जा॒य॒न्ते॒ यो यो जा॑यन्ते जायन्ते॒ य ए॒व मे॒वं ॅयो जा॑यन्ते जायन्ते॒ य ए॒वम् । \newline
55. य ए॒व मे॒वं ॅयो य ए॒वं ॅवि॒द्वान्. वि॒द्वा ने॒वं ॅयो य ए॒वं ॅवि॒द्वान् । \newline
56. ए॒वं ॅवि॒द्वान्. वि॒द्वा ने॒व मे॒वं ॅवि॒द्वान् द्वि॒पदा॑भिर् द्वि॒पदा॑भिर् वि॒द्वा ने॒व मे॒वं ॅवि॒द्वान् द्वि॒पदा॑भिः । \newline
57. वि॒द्वान् द्वि॒पदा॑भिर् द्वि॒पदा॑भिर् वि॒द्वान्. वि॒द्वान् द्वि॒पदा॑भि॒र् गार्.ह॑पत्य॒म् गार्.ह॑पत्यम् द्वि॒पदा॑भिर् वि॒द्वान्. वि॒द्वान् द्वि॒पदा॑भि॒र् गार्.ह॑पत्यम् । \newline
58. द्वि॒पदा॑भि॒र् गार्.ह॑पत्य॒म् गार्.ह॑पत्यम् द्वि॒पदा॑भिर् द्वि॒पदा॑भि॒र् गार्.ह॑पत्य मुप॒तिष्ठ॑त उप॒तिष्ठ॑ते॒ गार्.ह॑पत्यम् द्वि॒पदा॑भिर् द्वि॒पदा॑भि॒र् गार्.ह॑पत्य मुप॒तिष्ठ॑ते । \newline
59. द्वि॒पदा॑भि॒रिति॑ द्वि - पदा॑भिः । \newline
60. गार्.ह॑पत्य मुप॒तिष्ठ॑त उप॒तिष्ठ॑ते॒ गार्.ह॑पत्य॒म् गार्.ह॑पत्य मुप॒तिष्ठ॑त॒ ओप॒तिष्ठ॑ते॒ गार्.ह॑पत्य॒म् गार्.ह॑पत्य मुप॒तिष्ठ॑त॒ आ । \newline
61. गार्.ह॑पत्य॒मिति॒ गार्.ह॑ - प॒त्य॒म् । \newline
62. उ॒प॒तिष्ठ॑त॒ ओप॒तिष्ठ॑त उप॒तिष्ठ॑त॒ आ ऽस्या॒स्योप॒तिष्ठ॑त उप॒तिष्ठ॑त॒ आ ऽस्य॑ । \newline
63. उ॒प॒तिष्ठ॑त॒ इत्यु॑प - तिष्ठ॑ते । \newline
\pagebreak
\markright{ TS 1.5.8.4  \hfill https://www.vedavms.in \hfill}
\addcontentsline{toc}{section}{ TS 1.5.8.4 }
\section*{ TS 1.5.8.4 }

\textbf{TS 1.5.8.4 } \newline
\textbf{Samhita Paata} \newline

आऽस्य॑ वी॒रो जा॑यत ऊ॒र्जा वः॑ पश्याम्यू॒र्जा मा॑ पश्य॒तेत्या॑हा॒ ऽऽशिष॑मे॒वैतामा शा᳚स्ते॒ तथ्स॑वि॒तुर् वरे᳚ण्य॒मित्या॑ह॒ प्रसू᳚त्यै सो॒मानꣳ॒॒ स्वर॑ण॒मित्या॑ह सोमपी॒थमे॒वाव॑ रुन्धे कृणु॒हि ब्र॑ह्मणस्पत॒ इत्या॑ह ब्रह्मवर्च॒समे॒वाव॑ रुन्धे क॒दा च॒न स्त॒रीर॒सीत्या॑ह॒ न स्त॒रीꣳ रात्रिं॑ ॅवसति॒ - [ ] \newline

\textbf{Pada Paata} \newline

एति॑ । अ॒स्य॒ । वी॒रः । जा॒य॒ते॒ । ऊ॒र्जा । वः॒ । प॒श्या॒मि॒ । ऊ॒र्जा । मा॒ । प॒श्य॒त॒ । इति॑ । आ॒ह॒ । आ॒शिष॒मित्या᳚ - शिष᳚म् । ए॒व । ए॒ताम् । एति॑ । शा॒स्ते॒ । तत् । स॒वि॒तुः । वरे᳚ण्यम् । इति॑ । आ॒ह॒ । प्रसू᳚त्या॒ इति॒ प्र - सू॒त्यै॒ । सो॒मान᳚म् । स्वर॑णम् । इति॑ । आ॒ह॒ । सो॒म॒पी॒थमिति॑ सोम - पी॒थम् । ए॒व । अवेति॑ । रु॒न्धे॒ । कृ॒णु॒हि । ब्र॒ह्म॒णः॒ । प॒ते॒ । इति॑ । आ॒ह॒ । ब्र॒ह्म॒व॒र्च॒समिति॑ ब्रह्म-व॒र्च॒सम् । ए॒व । अवेति॑ । रु॒न्धे॒ । क॒दा । च॒न । स्त॒रीः । अ॒सि॒ । इति॑ । आ॒ह॒ । न । स्त॒रीम् । रात्रि᳚म् । व॒स॒ति॒ ।  \newline


\textbf{Krama Paata} \newline

आऽस्य॑ । अ॒स्य॒ वी॒रः । वी॒रो जा॑यते । जा॒य॒त॒ ऊ॒र्जा । ऊ॒र्जा वः॑ । वः॒ प॒श्या॒मि॒ । प॒श्या॒म्यू॒र्जा । ऊ॒र्जा मा᳚ । मा॒ प॒श्य॒त॒ । प॒श्य॒तेति॑ । इत्या॑ह । आ॒हा॒शिष᳚म् । आ॒शिष॑मे॒व । आ॒शिष॒मित्या᳚ - शिष᳚म् । ए॒वैताम् । ए॒तामा । आ शा᳚स्ते । शा॒स्ते॒ तत् । तथ् स॑वि॒तुः । स॒वि॒तुर् वरे᳚ण्यम् । वरे᳚ण्य॒मिति॑ । इत्या॑ह । आ॒ह॒ प्रसू᳚त्यै । प्रसू᳚त्यै सो॒मान᳚म् । प्रसू᳚त्या॒ इति॒ प्र - सू॒त्यै॒ । सो॒मानꣳ॒॒ स्वर॑णम् । स्वर॑ण॒मिति॑ । इत्या॑ह । आ॒ह॒ सो॒म॒पी॒थम् । सो॒म॒पी॒थमे॒व । सो॒म॒पी॒थमिति॑ सोम - पी॒थम् । ए॒वाव॑ । अव॑ रुन्धे । रु॒न्धे॒ कृ॒णु॒हि । कृ॒णु॒हि ब्र॑ह्मणः । ब्र॒ह्म॒ण॒स्प॒ते॒ । प॒त॒ इति॑ । इत्या॑ह । आ॒ह॒ ब्र॒ह्म॒व॒र्च॒सम् । ब्र॒ह्म॒व॒र्च॒समे॒व । ब्र॒ह्म॒व॒र्च॒समिति॑ ब्रह्म - व॒र्च॒सम् । ए॒वाव॑ । अव॑ रुन्धे । रु॒न्धे॒ क॒दा । क॒दाच॒न । च॒न स्त॒रीः । स्त॒रीर॑सि । अ॒सीति॑ । इत्या॑ह । आ॒ह॒ न । न स्त॒रीम् । स्त॒रीꣳ रात्रि᳚म् । रात्रिं॑ ॅवसति । व॒स॒ति॒ यः \newline

\textbf{Jatai Paata} \newline

1. आ ऽस्या॒स्या ऽस्य॑ । \newline
2. अ॒स्य॒ वी॒रो वी॒रो᳚ ऽस्यास्य वी॒रः । \newline
3. वी॒रो जा॑यते जायते वी॒रो वी॒रो जा॑यते । \newline
4. जा॒य॒त॒ ऊ॒र्जोर्जा जा॑यते जायत ऊ॒र्जा । \newline
5. ऊ॒र्जा वो॑ व ऊ॒र्जोर्जा वः॑ । \newline
6. वः॒ प॒श्या॒मि॒ प॒श्या॒मि॒ वो॒ वः॒ प॒श्या॒मि॒ । \newline
7. प॒श्या॒म्यू॒र्जोर्जा प॑श्यामि पश्याम्यू॒र्जा । \newline
8. ऊ॒र्जा मा॑ मो॒र्जोर्जा मा᳚ । \newline
9. मा॒ प॒श्य॒त॒ प॒श्य॒त॒ मा॒ मा॒ प॒श्य॒त॒ । \newline
10. प॒श्य॒ते तीति॑ पश्यत पश्य॒ते ति॑ । \newline
11. इत्या॑हा॒हे तीत्या॑ह । \newline
12. आ॒हा॒शिष॑ मा॒शिष॑ माहाहा॒शिष᳚म् । \newline
13. आ॒शिष॑ मे॒वैवाशिष॑ मा॒शिष॑ मे॒व । \newline
14. आ॒शिष॒मित्या᳚ - शिष᳚म् । \newline
15. ए॒वैता मे॒ता मे॒वैवैताम् । \newline
16. ए॒ता मैता मे॒ता मा । \newline
17. आ शा᳚स्ते शास्त॒ आ शा᳚स्ते । \newline
18. शा॒स्ते॒ तत् तच्छा᳚स्ते शास्ते॒ तत् । \newline
19. तथ् स॑वि॒तुः स॑वि॒तु स्तत् तथ् स॑वि॒तुः । \newline
20. स॒वि॒तुर् वरे᳚ण्यं॒ ॅवरे᳚ण्यꣳ सवि॒तुः स॑वि॒तुर् वरे᳚ण्यम् । \newline
21. वरे᳚ण्य॒ मितीति॒ वरे᳚ण्यं॒ ॅवरे᳚ण्य॒ मिति॑ । \newline
22. इत्या॑हा॒हे तीत्या॑ह । \newline
23. आ॒ह॒ प्रसू᳚त्यै॒ प्रसू᳚त्या आहाह॒ प्रसू᳚त्यै । \newline
24. प्रसू᳚त्यै सो॒मान(ग्म्॑) सो॒मान॒म् प्रसू᳚त्यै॒ प्रसू᳚त्यै सो॒मान᳚म् । \newline
25. प्रसू᳚त्या॒ इति॒ प्र - सू॒त्यै॒ । \newline
26. सो॒मान॒(ग्ग्॒) स्वर॑ण॒(ग्ग्॒) स्वर॑णꣳ सो॒मान(ग्म्॑) सो॒मान॒(ग्ग्॒) स्वर॑णम् । \newline
27. स्वर॑ण॒ मितीति॒ स्वर॑ण॒(ग्ग्॒) स्वर॑ण॒ मिति॑ । \newline
28. इत्या॑हा॒हे तीत्या॑ह । \newline
29. आ॒ह॒ सो॒म॒पी॒थꣳ सो॑मपी॒थ मा॑हाह सोमपी॒थम् । \newline
30. सो॒म॒पी॒थ मे॒वैव सो॑मपी॒थꣳ सो॑मपी॒थ मे॒व । \newline
31. सो॒म॒पी॒थमिति॑ सोम - पी॒थम् । \newline
32. ए॒वावावै॒वैवाव॑ । \newline
33. अव॑ रुन्धे रु॒न्धे ऽवाव॑ रुन्धे । \newline
34. रु॒न्धे॒ कृ॒णु॒हि कृ॑णु॒हि रु॑न्धे रुन्धे कृणु॒हि । \newline
35. कृ॒णु॒हि ब्र॑ह्मणो ब्रह्मणः कृणु॒हि कृ॑णु॒हि ब्र॑ह्मणः । \newline
36. ब्र॒ह्म॒ण॒ स्प॒ते॒ प॒ते॒ ब्र॒ह्म॒णो॒ ब्र॒ह्म॒ण॒ स्प॒ते॒ । \newline
37. प॒त॒ इतीति॑ पते पत॒ इति॑ । \newline
38. इत्या॑हा॒हे तीत्या॑ह । \newline
39. आ॒ह॒ ब्र॒ह्म॒व॒र्च॒सम् ब्र॑ह्मवर्च॒स मा॑हाह ब्रह्मवर्च॒सम् । \newline
40. ब्र॒ह्म॒व॒र्च॒स मे॒वैव ब्र॑ह्मवर्च॒सम् ब्र॑ह्मवर्च॒स मे॒व । \newline
41. ब्र॒ह्म॒व॒र्च॒समिति॑ ब्रह्म - व॒र्च॒सम् । \newline
42. ए॒वावावै॒वैवाव॑ । \newline
43. अव॑ रुन्धे रु॒न्धे ऽवाव॑ रुन्धे । \newline
44. रु॒न्धे॒ क॒दा क॒दा रु॑न्धे रुन्धे क॒दा । \newline
45. क॒दा च॒न च॒न क॒दा क॒दा च॒न । \newline
46. च॒न स्त॒रीः स्त॒री श्च॒न च॒न स्त॒रीः । \newline
47. स्त॒री र॑स्यसि स्त॒रीः स्त॒री र॑सि । \newline
48. अ॒सीतीत्य॑स्य॒सीति॑ । \newline
49. इत्या॑हा॒हे तीत्या॑ह । \newline
50. आ॒ह॒ न नाहा॑ह॒ न । \newline
51. न स्त॒रीꣳ स्त॒रीन्न न स्त॒रीम् । \newline
52. स्त॒रीꣳ रात्रि॒(ग्म्॒) रात्रि(ग्ग्॑) स्त॒रीꣳ स्त॒रीꣳ रात्रि᳚म् । \newline
53. रात्रिं॑ ॅवसति वसति॒ रात्रि॒(ग्म्॒) रात्रिं॑ ॅवसति । \newline
54. व॒स॒ति॒ यो यो व॑सति वसति॒ यः । \newline

\textbf{Ghana Paata } \newline

1. आ ऽस्या॒स्या ऽस्य॑ वी॒रो वी॒रो᳚ ऽस्या ऽस्य॑ वी॒रः । \newline
2. अ॒स्य॒ वी॒रो वी॒रो᳚ ऽस्यास्य वी॒रो जा॑यते जायते वी॒रो᳚ ऽस्यास्य वी॒रो जा॑यते । \newline
3. वी॒रो जा॑यते जायते वी॒रो वी॒रो जा॑यत ऊ॒र्जोर्जा जा॑यते वी॒रो वी॒रो जा॑यत ऊ॒र्जा । \newline
4. जा॒य॒त॒ ऊ॒र्जोर्जा जा॑यते जायत ऊ॒र्जा वो॑ व ऊ॒र्जा जा॑यते जायत ऊ॒र्जा वः॑ । \newline
5. ऊ॒र्जा वो॑ व ऊ॒र्जोर्जा वः॑ पश्यामि पश्यामि व ऊ॒र्जोर्जा वः॑ पश्यामि । \newline
6. वः॒ प॒श्या॒मि॒ प॒श्या॒मि॒ वो॒ वः॒ प॒श्या॒म्यू॒र्जोर्जा प॑श्यामि वो वः पश्याम्यू॒र्जा । \newline
7. प॒श्या॒म्यू॒र्जोर्जा प॑श्यामि पश्याम्यू॒र्जा मा॑ मो॒र्जा प॑श्यामि पश्याम्यू॒र्जा मा᳚ । \newline
8. ऊ॒र्जा मा॑ मो॒र्जोर्जा मा॑ पश्यत पश्यत मो॒र्जोर्जा मा॑ पश्यत । \newline
9. मा॒ प॒श्य॒त॒ प॒श्य॒त॒ मा॒ मा॒ प॒श्य॒ते तीति॑ पश्यत मा मा पश्य॒ते ति॑ । \newline
10. प॒श्य॒ते तीति॑ पश्यत पश्य॒ते त्या॑हा॒हे ति॑ पश्यत पश्य॒ते त्या॑ह । \newline
11. इत्या॑हा॒हे तीत्या॑हा॒शिष॑ मा॒शिष॑ मा॒हे तीत्या॑हा॒शिष᳚म् । \newline
12. आ॒हा॒शिष॑ मा॒शिष॑ माहाहा॒शिष॑ मे॒वैवाशिष॑ माहाहा॒शिष॑ मे॒व । \newline
13. आ॒शिष॑ मे॒वैवाशिष॑ मा॒शिष॑ मे॒वैता मे॒ता मे॒वाशिष॑ मा॒शिष॑ मे॒वैताम् । \newline
14. आ॒शिष॒मित्या᳚ - शिष᳚म् । \newline
15. ए॒वैता मे॒ता मे॒वैवैता मैता मे॒वैवैता मा । \newline
16. ए॒ता मैता मे॒ता मा शा᳚स्ते शास्त॒ ऐता मे॒ता मा शा᳚स्ते । \newline
17. आ शा᳚स्ते शास्त॒ आ शा᳚स्ते॒ तत् तच्छा᳚स्त॒ आ शा᳚स्ते॒ तत् । \newline
18. शा॒स्ते॒ तत् तच्छा᳚स्ते शास्ते॒ तथ् स॑वि॒तुः स॑वि॒तुस्तच्छा᳚स्ते शास्ते॒ तथ् स॑वि॒तुः । \newline
19. तथ् स॑वि॒तुः स॑वि॒तुस्तत् तथ् स॑वि॒तुर् वरे᳚ण्यं॒ ॅवरे᳚ण्यꣳ सवि॒तुस्तत् तथ् स॑वि॒तुर् वरे᳚ण्यम् । \newline
20. स॒वि॒तुर् वरे᳚ण्यं॒ ॅवरे᳚ण्यꣳ सवि॒तुः स॑वि॒तुर् वरे᳚ण्य॒ मितीति॒ वरे᳚ण्यꣳ सवि॒तुः स॑वि॒तुर् वरे᳚ण्य॒ मिति॑ । \newline
21. वरे᳚ण्य॒ मितीति॒ वरे᳚ण्यं॒ ॅवरे᳚ण्य॒ मित्या॑हा॒हे ति॒ वरे᳚ण्यं॒ ॅवरे᳚ण्य॒ मित्या॑ह । \newline
22. इत्या॑हा॒हे तीत्या॑ह॒ प्रसू᳚त्यै॒ प्रसू᳚त्या आ॒हे तीत्या॑ह॒ प्रसू᳚त्यै । \newline
23. आ॒ह॒ प्रसू᳚त्यै॒ प्रसू᳚त्या आहाह॒ प्रसू᳚त्यै सो॒मान(ग्म्॑) सो॒मान॒म् प्रसू᳚त्या आहाह॒ प्रसू᳚त्यै सो॒मान᳚म् । \newline
24. प्रसू᳚त्यै सो॒मान(ग्म्॑) सो॒मान॒म् प्रसू᳚त्यै॒ प्रसू᳚त्यै सो॒मान॒(ग्ग्॒) स्वर॑ण॒(ग्ग्॒) स्वर॑णꣳ सो॒मान॒म् प्रसू᳚त्यै॒ प्रसू᳚त्यै सो॒मान॒(ग्ग्॒) स्वर॑णम् । \newline
25. प्रसू᳚त्या॒ इति॒ प्र - सू॒त्यै॒ । \newline
26. सो॒मान॒(ग्ग्॒) स्वर॑ण॒(ग्ग्॒) स्वर॑णꣳ सो॒मान(ग्म्॑) सो॒मान॒(ग्ग्॒) स्वर॑ण॒ मितीति॒ स्वर॑णꣳ सो॒मान(ग्म्॑) सो॒मान॒(ग्ग्॒) स्वर॑ण॒ मिति॑ । \newline
27. स्वर॑ण॒ मितीति॒ स्वर॑ण॒(ग्ग्॒) स्वर॑ण॒ मित्या॑हा॒हे ति॒ स्वर॑ण॒(ग्ग्॒) स्वर॑ण॒ मित्या॑ह । \newline
28. इत्या॑हा॒हे तीत्या॑ह सोमपी॒थꣳ सो॑मपी॒थ मा॒हे तीत्या॑ह सोमपी॒थम् । \newline
29. आ॒ह॒ सो॒म॒पी॒थꣳ सो॑मपी॒थ मा॑हाह सोमपी॒थ मे॒वैव सो॑मपी॒थ मा॑हाह सोमपी॒थ मे॒व । \newline
30. सो॒म॒पी॒थ मे॒वैव सो॑मपी॒थꣳ सो॑मपी॒थ मे॒वावावै॒व सो॑मपी॒थꣳ सो॑मपी॒थ मे॒वाव॑ । \newline
31. सो॒म॒पी॒थमिति॑ सोम - पी॒थम् । \newline
32. ए॒वावावै॒वैवाव॑ रुन्धे रु॒न्धे ऽवै॒वैवाव॑ रुन्धे । \newline
33. अव॑ रुन्धे रु॒न्धे ऽवाव॑ रुन्धे कृणु॒हि कृ॑णु॒हि रु॒न्धे ऽवाव॑ रुन्धे कृणु॒हि । \newline
34. रु॒न्धे॒ कृ॒णु॒हि कृ॑णु॒हि रु॑न्धे रुन्धे कृणु॒हि ब्र॑ह्मणो ब्रह्मणः कृणु॒हि रु॑न्धे रुन्धे कृणु॒हि ब्र॑ह्मणः । \newline
35. कृ॒णु॒हि ब्र॑ह्मणो ब्रह्मणः कृणु॒हि कृ॑णु॒हि ब्र॑ह्मण स्पते पते ब्रह्मणः कृणु॒हि कृ॑णु॒हि ब्र॑ह्मण स्पते । \newline
36. ब्र॒ह्म॒ण॒ स्प॒ते॒ प॒ते॒ ब्र॒ह्म॒णो॒ ब्र॒ह्म॒ण॒ स्प॒त॒ इतीति॑ पते ब्रह्मणो ब्रह्मण स्पत॒ इति॑ । \newline
37. प॒त॒ इतीति॑ पते पत॒ इत्या॑हा॒हे ति॑ पते पत॒ इत्या॑ह । \newline
38. इत्या॑हा॒हे तीत्या॑ह ब्रह्मवर्च॒सम् ब्र॑ह्मवर्च॒स मा॒हे तीत्या॑ह ब्रह्मवर्च॒सम् । \newline
39. आ॒ह॒ ब्र॒ह्म॒व॒र्च॒सम् ब्र॑ह्मवर्च॒स मा॑हाह ब्रह्मवर्च॒स मे॒वैव ब्र॑ह्मवर्च॒स मा॑हाह ब्रह्मवर्च॒स मे॒व । \newline
40. ब्र॒ह्म॒व॒र्च॒स मे॒वैव ब्र॑ह्मवर्च॒सम् ब्र॑ह्मवर्च॒स मे॒वावावै॒व ब्र॑ह्मवर्च॒सम् ब्र॑ह्मवर्च॒स मे॒वाव॑ । \newline
41. ब्र॒ह्म॒व॒र्च॒समिति॑ ब्रह्म - व॒र्च॒सम् । \newline
42. ए॒वावावै॒वैवाव॑ रुन्धे रु॒न्धे ऽवै॒वैवाव॑ रुन्धे । \newline
43. अव॑ रुन्धे रु॒न्धे ऽवाव॑ रुन्धे क॒दा क॒दा रु॒न्धे ऽवाव॑ रुन्धे क॒दा । \newline
44. रु॒न्धे॒ क॒दा क॒दा रु॑न्धे रुन्धे क॒दा च॒न च॒न क॒दा रु॑न्धे रुन्धे क॒दा च॒न । \newline
45. क॒दा च॒न च॒न क॒दा क॒दा च॒न स्त॒रीः स्त॒रीश्च॒न क॒दा क॒दा च॒न स्त॒रीः । \newline
46. च॒न स्त॒रीः स्त॒रीश्च॒न च॒न स्त॒रीर॑स्यसि स्त॒रीश्च॒न च॒न स्त॒रीर॑सि । \newline
47. स्त॒रीर॑स्यसि स्त॒रीः स्त॒रीर॒सीतीत्य॑सि स्त॒रीः स्त॒रीर॒सीति॑ । \newline
48. अ॒सीतीत्य॑स्य॒सीत्या॑हा॒हे त्य॑स्य॒सीत्या॑ह । \newline
49. इत्या॑हा॒हे तीत्या॑ह॒ न नाहे तीत्या॑ह॒ न । \newline
50. आ॒ह॒ न नाहा॑ह॒ न स्त॒रीꣳ स्त॒रीन्नाहा॑ह॒ न स्त॒रीम् । \newline
51. न स्त॒रीꣳ स्त॒रीन्न न स्त॒रीꣳ रात्रि॒(ग्म्॒) रात्रि(ग्ग्॑) स्त॒रीन्न न स्त॒रीꣳ रात्रि᳚म् । \newline
52. स्त॒रीꣳ रात्रि॒(ग्म्॒) रात्रि(ग्ग्॑) स्त॒रीꣳ स्त॒रीꣳ रात्रिं॑ ॅवसति वसति॒ रात्रि(ग्ग्॑) स्त॒रीꣳ स्त॒रीꣳ रात्रिं॑ ॅवसति । \newline
53. रात्रिं॑ ॅवसति वसति॒ रात्रि॒(ग्म्॒) रात्रिं॑ ॅवसति॒ यो यो व॑सति॒ रात्रि॒(ग्म्॒) रात्रिं॑ ॅवसति॒ यः । \newline
54. व॒स॒ति॒ यो यो व॑सति वसति॒ य ए॒व मे॒वं ॅयो व॑सति वसति॒ य ए॒वम् । \newline
\pagebreak
\markright{ TS 1.5.8.5  \hfill https://www.vedavms.in \hfill}
\addcontentsline{toc}{section}{ TS 1.5.8.5 }
\section*{ TS 1.5.8.5 }

\textbf{TS 1.5.8.5 } \newline
\textbf{Samhita Paata} \newline

य ए॒वं ॅवि॒द्वान॒ग्नि-मु॑प॒तिष्ठ॑ते॒ परि॑ त्वाऽग्ने॒ पुरं॑ ॅव॒यमित्या॑ह परि॒धिमे॒वैतं परि॑ दधा॒त्यस्क॑न्दा॒याग्ने॑ गृहपत॒ इत्या॑ह यथाय॒जुरे॒वैतच्छ॒तꣳ हिमा॒ इत्या॑ह श॒तं त्वा॑ हेम॒न्तानि॑न्धिषी॒येति॒ वावैतदा॑ह पु॒त्रस्य॒ नाम॑ गृह्णात्यन्ना॒दमे॒वैनं॑ करोति॒ तामा॒शिष॒मा शा॑से॒ तन्त॑वे॒ ज्योति॑ष्मती॒मिति॑ ब्रूया॒द्यस्य॑ पु॒त्रोऽजा॑तः॒ स्यात् ते॑ज॒स्व्ये॑वास्य॑ ब्रह्मवर्च॒सी पु॒त्रो जा॑यते॒ तामा॒शिष॒मा शा॑से॒ ऽमुष्मै॒ ज्योति॑ष्मती॒ ( ) मिति॑ ब्रूया॒द्यस्य॑ पु॒त्रो जा॒तः स्यात् तेज॑ ए॒वास्मि॑न् ब्रह्मवर्च॒सं द॑धाति ॥ \newline

\textbf{Pada Paata} \newline

यः । ए॒वम् । वि॒द्वान् । अ॒ग्निम् । उ॒प॒तिष्ठ॑त॒ इत्यु॑प - तिष्ठ॑ते । परीति॑ । त्वा॒ । अ॒ग्ने॒ । पुर᳚म् । व॒यम् । इति॑ । आ॒ह॒ । प॒रि॒धिमिति॑ परि-धिम् । ए॒व । ए॒तम् । परीति॑ । द॒धा॒ति॒ । अस्क॑न्दाय । अग्ने᳚ । गृ॒ह॒प॒त॒ इति॑ गृह - प॒ते॒ । इति॑ । आ॒ह॒ । य॒था॒य॒जुरिति॑ यथा - य॒जुः । ए॒व । ए॒तत् । श॒तम् । हिमाः᳚ । इति॑ । आ॒ह॒ । श॒तम् । त्वा॒ । हे॒म॒न्तान् । इ॒न्धि॒षी॒य॒ । इति॑ । वाव । ए॒तत् । आ॒ह॒ । पु॒त्रस्य॑ । नाम॑ । गृ॒ह्णा॒ति॒ । अ॒न्ना॒दमित्य॑न्न - अ॒दम् । ए॒व । ए॒न॒म् । क॒रो॒ति॒ । ताम् । आ॒शिष॒मित्या᳚-शिष᳚म् । एति॑ । शा॒से॒ । तन्त॑वे । ज्योति॑ष्मतीम् ( ) । इति॑ । ब्र॒या॒त् । यस्य॑ । पु॒त्रः । अजा॑तः । स्यात् । ते॒ज॒स्वी । ए॒व । अ॒स्य॒ । ब्र॒ह्म॒व॒र्च॒सीति॑ ब्रह्म - व॒र्च॒सी । पु॒त्रः । जा॒य॒ते॒ । ताम् । आ॒शिष॒मित्या᳚ - शिष᳚म् । एति॑ । शा॒से॒ । अ॒मुष्मै᳚ । ज्योति॑ष्मतीम् । इति॑ । ब्रू॒या॒त् । यस्य॑ । पु॒त्रः । जा॒तः । स्यात् । तेजः॑ । ए॒व । अ॒स्मि॒न्न् । ब्र॒ह्म॒व॒र्च॒समिति॑ ब्रह्म - व॒र्च॒सम् । द॒धा॒ति॒ ॥  \newline


\textbf{Krama Paata} \newline

य ए॒वम् । ए॒वं ॅवि॒द्वान् । वि॒द्वान॒ग्निम् । अ॒ग्निमु॑प॒तिष्ठ॑ते । उ॒प॒तिष्ठ॑ते॒ परि॑ । उ॒प॒तिष्ठ॑त॒ इत्यु॑प - तिष्ठ॑ते । परि॑ त्वा । त्वा॒ऽग्ने॒ । अ॒ग्ने॒ पुर᳚म् । पुरं॑ ॅव॒यम् । व॒यमिति॑ । इत्या॑ह । आ॒ह॒ प॒रि॒धिम् । प॒रि॒धिमे॒व । प॒रि॒धिमिति॑ परि - धिम् । ए॒वैतम् । ए॒तम् परि॑ । परि॑ दधाति । द॒धा॒त्यस्क॑न्दाय । अस्क॑न्दा॒याग्ने᳚ । अग्ने॑ गृहपते । गृ॒ह॒प॒त॒ इति॑ । गृ॒ह॒प॒त॒ इति॑ गृह - प॒ते॒ । इत्या॑ह । आ॒ह॒ य॒था॒य॒जुः । य॒था॒य॒जुरे॒व । य॒था॒य॒जुरिति॑ यथा - य॒जुः । ए॒वैतत् । ए॒तच्छ॒तम् । श॒तꣳ हिमाः᳚ । हिमा॒ इति॑ । इत्या॑ह । आ॒ह॒ श॒तम् । श॒तम् त्वा᳚ । त्वा॒ हे॒म॒न्तान् । हे॒म॒न्तानि॑न्धिषीय । इ॒न्धि॒षी॒येति॑ । इति॒ वाव । वावैतत् । ए॒तदा॑ह । आ॒ह॒ पु॒त्रस्य॑ । पु॒त्रस्य॒ नाम॑ । नाम॑ गृह्णाति । गृ॒ह्णा॒त्य॒न्ना॒दम् । अ॒न्ना॒दमे॒व । अ॒न्ना॒दमित्य॑न्न - अ॒दम् । ए॒वैन᳚म् । ए॒नं॒ क॒रो॒ति॒ । क॒रो॒ति॒ ताम् । तामा॒शिष᳚म् । आ॒शिष॒मा । आ॒शिष॒मित्या᳚ - शिष᳚म् । आ शा॑से । शा॒से॒ तन्त॑वे । तन्त॑वे॒ ज्योति॑ष्मतीम् । ज्योति॑ष्मती॒मिति॑ । इति॑ ब्रूयात् । ब्रू॒या॒द् यस्य॑ । यस्य॑ पु॒त्रः । पु॒त्रोऽजा॑तः । अजा॑तः॒ स्यात् । स्यात् ते॑ज॒स्वी । ते॒ज॒स्व्ये॑व । ए॒वास्य॑ । अ॒स्य॒ ब्र॒ह्म॒व॒र्च॒सी । ब्र॒ह्म॒व॒र्च॒सी पु॒त्रः । ब्र॒ह्म॒व॒र्च॒सीति॑ ब्रह्म - व॒र्च॒सी । पु॒त्रो जा॑यते । जा॒य॒ते॒ ताम् । तामा॒शिष᳚म् । आ॒शिष॒मा । आ॒शिष॒मित्या᳚ - शिष᳚म् । आ शा॑से । शा॒से॒ऽमुष्मै᳚ । अ॒मुष्मै॒ ज्योति॑ष्मतीम् ( ) । ज्योति॑ष्मती॒मिति॑ । इति॑ ब्रूयात् । ब्रू॒या॒द् यस्य॑ । यस्य॑ पु॒त्रः । पु॒त्रो जा॒तः । जा॒तः स्यात् । स्यात् तेजः॑ । तेज॑ ए॒व । ए॒वास्मिन्न्॑ । अ॒स्मि॒न् ब्र॒ह्म॒व॒र्च॒सम् । ब्र॒ह्म॒व॒र्च॒सम् द॑धाति । ब्र॒ह्म॒व॒र्च॒समिति॑ ब्रह्म - व॒र्च॒सम् । द॒धा॒तीति॑ दधाति । \newline

\textbf{Jatai Paata} \newline

1. य ए॒व मे॒वं ॅयो य ए॒वम् । \newline
2. ए॒वं ॅवि॒द्वान्. वि॒द्वा ने॒व मे॒वं ॅवि॒द्वान् । \newline
3. वि॒द्वा न॒ग्नि म॒ग्निं ॅवि॒द्वान्. वि॒द्वा न॒ग्निम् । \newline
4. अ॒ग्नि मु॑प॒तिष्ठ॑त उप॒तिष्ठ॑ते॒ ऽग्नि म॒ग्नि मु॑प॒तिष्ठ॑ते । \newline
5. उ॒प॒तिष्ठ॑ते॒ परि॒ पर्यु॑प॒तिष्ठ॑त उप॒तिष्ठ॑ते॒ परि॑ । \newline
6. उ॒प॒तिष्ठ॑त॒ इत्यु॑प - तिष्ठ॑ते । \newline
7. परि॑ त्वा त्वा॒ परि॒ परि॑ त्वा । \newline
8. त्वा॒ ऽग्ने॒ ऽग्ने॒ त्वा॒ त्वा॒ ऽग्ने॒ । \newline
9. अ॒ग्ने॒ पुर॒म् पुर॑ मग्ने ऽग्ने॒ पुर᳚म् । \newline
10. पुरं॑ ॅव॒यं ॅव॒यम् पुर॒म् पुरं॑ ॅव॒यम् । \newline
11. व॒य मितीति॑ व॒यं ॅव॒य मिति॑ । \newline
12. इत्या॑हा॒हे तीत्या॑ह । \newline
13. आ॒ह॒ प॒रि॒धिम् प॑रि॒धि मा॑हाह परि॒धिम् । \newline
14. प॒रि॒धि मे॒वैव प॑रि॒धिम् प॑रि॒धि मे॒व । \newline
15. प॒रि॒धिमिति॑ परि - धिम् । \newline
16. ए॒वैत मे॒त मे॒वैवैतम् । \newline
17. ए॒तम् परि॒ पर्ये॒त मे॒तम् परि॑ । \newline
18. परि॑ दधाति दधाति॒ परि॒ परि॑ दधाति । \newline
19. द॒धा॒त्यस्क॑न्दा॒यास्क॑न्दाय दधाति दधा॒त्यस्क॑न्दाय । \newline
20. अस्क॑न्दा॒याग्ने ऽग्ने ऽस्क॑न्दा॒यास्क॑न्दा॒याग्ने᳚ । \newline
21. अग्ने॑ गृहपते गृहप॒ते ऽग्ने ऽग्ने॑ गृहपते । \newline
22. गृ॒ह॒प॒त॒ इतीति॑ गृहपते गृहपत॒ इति॑ । \newline
23. गृ॒ह॒प॒त॒ इति॑ गृह - प॒ते॒ । \newline
24. इत्या॑हा॒हे तीत्या॑ह । \newline
25. आ॒ह॒ य॒था॒य॒जुर् य॑थाय॒जुरा॑हाह यथाय॒जुः । \newline
26. य॒था॒य॒जु रे॒वैव य॑थाय॒जुर् य॑थाय॒जुरे॒व । \newline
27. य॒था॒य॒जुरिति॑ यथा - य॒जुः । \newline
28. ए॒वैतदे॒तदे॒वैवैतत् । \newline
29. ए॒तच्छ॒तꣳ श॒त मे॒तदे॒तच्छ॒तम् । \newline
30. श॒तꣳ हिमा॒ हिमाः᳚ श॒तꣳ श॒तꣳ हिमाः᳚ । \newline
31. हिमा॒ इतीति॒ हिमा॒ हिमा॒ इति॑ । \newline
32. इत्या॑हा॒हे तीत्या॑ह । \newline
33. आ॒ह॒ श॒तꣳ श॒त मा॑हाह श॒तम् । \newline
34. श॒तम् त्वा᳚ त्वा श॒तꣳ श॒तम् त्वा᳚ । \newline
35. त्वा॒ हे॒म॒न्तान्. हे॑म॒न्तान् त्वा᳚ त्वा हेम॒न्तान् । \newline
36. हे॒म॒न्ता नि॑न्धिषीये न्धिषीय हेम॒न्तान्. हे॑म॒न्ता नि॑न्धिषीय । \newline
37. इ॒न्धि॒षी॒ये तीती᳚न्धिषीये न्धिषी॒ये ति॑ । \newline
38. इति॒ वाव वावे तीति॒ वाव । \newline
39. वावैतदे॒तद् वाव वावैतत् । \newline
40. ए॒तदा॑हाहै॒तदे॒तदा॑ह । \newline
41. आ॒ह॒ पु॒त्रस्य॑ पु॒त्रस्या॑हाह पु॒त्रस्य॑ । \newline
42. पु॒त्रस्य॒ नाम॒ नाम॑ पु॒त्रस्य॑ पु॒त्रस्य॒ नाम॑ । \newline
43. नाम॑ गृह्णाति गृह्णाति॒ नाम॒ नाम॑ गृह्णाति । \newline
44. गृ॒ह्णा॒त्य॒न्ना॒द म॑न्ना॒दम् गृ॑ह्णाति गृह्णात्यन्ना॒दम् । \newline
45. अ॒न्ना॒द मे॒वैवान्ना॒द म॑न्ना॒द मे॒व । \newline
46. अ॒न्ना॒दमित्य॑न्न - अ॒दम् । \newline
47. ए॒वैन॑ मेन मे॒वैवैन᳚म् । \newline
48. ए॒न॒म् क॒रो॒ति॒ क॒रो॒त्ये॒न॒ मे॒न॒म् क॒रो॒ति॒ । \newline
49. क॒रो॒ति॒ ताम् ताम् क॑रोति करोति॒ ताम् । \newline
50. ता मा॒शिष॑ मा॒शिष॒म् ताम् ता मा॒शिष᳚म् । \newline
51. आ॒शिष॒ मा ऽऽशिष॑ मा॒शिष॒ मा । \newline
52. आ॒शिष॒मित्या᳚ - शिष᳚म् । \newline
53. आ शा॑से शास॒ आ शा॑से । \newline
54. शा॒से॒ तन्त॑वे॒ तन्त॑वे शासे शासे॒ तन्त॑वे । \newline
55. तन्त॑वे॒ ज्योति॑ष्मती॒म् ज्योति॑ष्मती॒म् तन्त॑वे॒ तन्त॑वे॒ ज्योति॑ष्मतीम् । \newline
56. ज्योति॑ष्मती॒ मितीति॒ ज्योति॑ष्मती॒म् ज्योति॑ष्मती॒ मिति॑ । \newline
57. इति॑ ब्रूयाद् ब्रूया॒दितीति॑ ब्रूयात् । \newline
58. ब्रू॒या॒द् यस्य॒ यस्य॑ ब्रूयाद् ब्रूया॒द् यस्य॑ । \newline
59. यस्य॑ पु॒त्रः पु॒त्रो यस्य॒ यस्य॑ पु॒त्रः । \newline
60. पु॒त्रो ऽजा॒तो ऽजा॑तः पु॒त्रः पु॒त्रो ऽजा॑तः । \newline
61. अजा॑तः॒ स्याथ् स्यादजा॒तो ऽजा॑तः॒ स्यात् । \newline
62. स्यात् ते॑ज॒स्वी ते॑ज॒स्वी स्याथ् स्यात् ते॑ज॒स्वी । \newline
63. ते॒ज॒स्व्ये॑वैव ते॑ज॒स्वी ते॑ज॒स्व्ये॑व । \newline
64. ए॒वास्या᳚स्यै॒वैवास्य॑ । \newline
65. अ॒स्य॒ ब्र॒ह्म॒व॒र्च॒सी ब्र॑ह्मवर्च॒स्य॑स्यास्य ब्रह्मवर्च॒सी । \newline
66. ब्र॒ह्म॒व॒र्च॒सी पु॒त्रः पु॒त्रो ब्र॑ह्मवर्च॒सी ब्र॑ह्मवर्च॒सी पु॒त्रः । \newline
67. ब्र॒ह्म॒व॒र्च॒सीति॑ ब्रह्म - व॒र्च॒सी । \newline
68. पु॒त्रो जा॑यते जायते पु॒त्रः पु॒त्रो जा॑यते । \newline
69. जा॒य॒ते॒ ताम् ताम् जा॑यते जायते॒ ताम् । \newline
70. ता मा॒शिष॑ मा॒शिष॒म् ताम् ता मा॒शिष᳚म् । \newline
71. आ॒शिष॒ मा ऽऽशिष॑ मा॒शिष॒ मा । \newline
72. आ॒शिष॒मित्या᳚ - शिष᳚म् । \newline
73. आ शा॑से शास॒ आ शा॑से । \newline
74. शा॒से॒ ऽमुष्मा॑ अ॒मुष्मै॑ शासे शासे॒ ऽमुष्मै᳚ । \newline
75. अ॒मुष्मै॒ ज्योति॑ष्मती॒म् ज्योति॑ष्मती म॒मुष्मा॑ अ॒मुष्मै॒ ज्योति॑ष्मतीम् । \newline
76. ज्योति॑ष्मती॒ मितीति॒ ज्योति॑ष्मती॒म् ज्योति॑ष्मती॒ मिति॑ । \newline
77. इति॑ ब्रूयाद् ब्रूया॒दितीति॑ ब्रूयात् । \newline
78. ब्रू॒या॒द् यस्य॒ यस्य॑ ब्रूयाद् ब्रूया॒द् यस्य॑ । \newline
79. यस्य॑ पु॒त्रः पु॒त्रो यस्य॒ यस्य॑ पु॒त्रः । \newline
80. पु॒त्रो जा॒तो जा॒तः पु॒त्रः पु॒त्रो जा॒तः । \newline
81. जा॒तः स्याथ् स्याज् जा॒तो जा॒तः स्यात् । \newline
82. स्यात् तेज॒स्तेजः॒ स्याथ् स्यात् तेजः॑ । \newline
83. तेज॑ ए॒वैव तेज॒स्तेज॑ ए॒व । \newline
84. ए॒वास्मि॑ न्नस्मि न्ने॒वैवास्मिन्न्॑ । \newline
85. अ॒स्मि॒न् ब्र॒ह्म॒व॒र्च॒सम् ब्र॑ह्मवर्च॒स म॑स्मिन्नस्मिन् ब्रह्मवर्च॒सम् । \newline
86. ब्र॒ह्म॒व॒र्च॒सम् द॑धाति दधाति ब्रह्मवर्च॒सम् ब्र॑ह्मवर्च॒सम् द॑धाति । \newline
87. ब्र॒ह्म॒व॒र्च॒समिति॑ ब्रह्म - व॒र्च॒सम् । \newline
88. द॒धा॒तीति॑ दधाति । \newline

\textbf{Ghana Paata } \newline

1. य ए॒व मे॒वं ॅयो य ए॒वं ॅवि॒द्वान्. वि॒द्वा ने॒वं ॅयो य ए॒वं ॅवि॒द्वान् । \newline
2. ए॒वं ॅवि॒द्वान्. वि॒द्वा ने॒व मे॒वं ॅवि॒द्वा न॒ग्नि म॒ग्निं ॅवि॒द्वा ने॒व मे॒वं ॅवि॒द्वा न॒ग्निम् । \newline
3. वि॒द्वा न॒ग्नि म॒ग्निं ॅवि॒द्वान्. वि॒द्वा न॒ग्नि मु॑प॒तिष्ठ॑त उप॒तिष्ठ॑ते॒ ऽग्निं ॅवि॒द्वान्. वि॒द्वा न॒ग्नि मु॑प॒तिष्ठ॑ते । \newline
4. अ॒ग्नि मु॑प॒तिष्ठ॑त उप॒तिष्ठ॑ते॒ ऽग्नि म॒ग्नि मु॑प॒तिष्ठ॑ते॒ परि॒ पर्यु॑प॒तिष्ठ॑ते॒ ऽग्नि म॒ग्नि मु॑प॒तिष्ठ॑ते॒ परि॑ । \newline
5. उ॒प॒तिष्ठ॑ते॒ परि॒ पर्यु॑प॒तिष्ठ॑त उप॒तिष्ठ॑ते॒ परि॑ त्वा त्वा॒ पर्यु॑प॒तिष्ठ॑त उप॒तिष्ठ॑ते॒ परि॑ त्वा । \newline
6. उ॒प॒तिष्ठ॑त॒ इत्यु॑प - तिष्ठ॑ते । \newline
7. परि॑ त्वा त्वा॒ परि॒ परि॑ त्वा ऽग्ने ऽग्ने त्वा॒ परि॒ परि॑ त्वा ऽग्ने । \newline
8. त्वा॒ ऽग्ने॒ ऽग्ने॒ त्वा॒ त्वा॒ ऽग्ने॒ पुर॒म् पुर॑ मग्ने त्वा त्वा ऽग्ने॒ पुर᳚म् । \newline
9. अ॒ग्ने॒ पुर॒म् पुर॑ मग्ने ऽग्ने॒ पुरं॑ ॅव॒यं ॅव॒यम् पुर॑ मग्ने ऽग्ने॒ पुरं॑ ॅव॒यम् । \newline
10. पुरं॑ ॅव॒यं ॅव॒यम् पुर॒म् पुरं॑ ॅव॒य मितीति॑ व॒यम् पुर॒म् पुरं॑ ॅव॒य मिति॑ । \newline
11. व॒य मितीति॑ व॒यं ॅव॒य मित्या॑हा॒हे ति॑ व॒यं ॅव॒य मित्या॑ह । \newline
12. इत्या॑हा॒हे तीत्या॑ह परि॒धिम् प॑रि॒धि मा॒हे तीत्या॑ह परि॒धिम् । \newline
13. आ॒ह॒ प॒रि॒धिम् प॑रि॒धि मा॑हाह परि॒धि मे॒वैव प॑रि॒धि मा॑हाह परि॒धि मे॒व । \newline
14. प॒रि॒धि मे॒वैव प॑रि॒धिम् प॑रि॒धि मे॒वैत मे॒त मे॒व प॑रि॒धिम् प॑रि॒धि मे॒वैतम् । \newline
15. प॒रि॒धिमिति॑ परि - धिम् । \newline
16. ए॒वैत मे॒त मे॒वैवैतम् परि॒ पर्ये॒त मे॒वैवैतम् परि॑ । \newline
17. ए॒तम् परि॒ पर्ये॒त मे॒तम् परि॑ दधाति दधाति॒ पर्ये॒त मे॒तम् परि॑ दधाति । \newline
18. परि॑ दधाति दधाति॒ परि॒ परि॑ दधा॒ त्यस्क॑न्दा॒ यास्क॑न्दाय दधाति॒ परि॒ परि॑ दधा॒त्यस्क॑न्दाय । \newline
19. द॒धा॒त्यस्क॑न्दा॒ यास्क॑न्दाय दधाति दधा॒त्यस्क॑न्दा॒याग्ने ऽग्ने ऽस्क॑न्दाय दधाति दधा॒त्यस्क॑न्दा॒याग्ने᳚ । \newline
20. अस्क॑न्दा॒याग्ने ऽग्ने ऽस्क॑न्दा॒यास्क॑न्दा॒याग्ने॑ गृहपते गृहप॒ते ऽग्ने ऽस्क॑न्दा॒यास्क॑न्दा॒याग्ने॑ गृहपते । \newline
21. अग्ने॑ गृहपते गृहप॒ते ऽग्ने ऽग्ने॑ गृहपत॒ इतीति॑ गृहप॒ते ऽग्ने ऽग्ने॑ गृहपत॒ इति॑ । \newline
22. गृ॒ह॒प॒त॒ इतीति॑ गृहपते गृहपत॒ इत्या॑हा॒हे ति॑ गृहपते गृहपत॒ इत्या॑ह । \newline
23. गृ॒ह॒प॒त॒ इति॑ गृह - प॒ते॒ । \newline
24. इत्या॑हा॒हे तीत्या॑ह यथाय॒जुर् य॑थाय॒जुरा॒हे तीत्या॑ह यथाय॒जुः । \newline
25. आ॒ह॒ य॒था॒य॒जुर् य॑थाय॒जुरा॑हाह यथाय॒जुरे॒वैव य॑थाय॒जुरा॑हाह यथाय॒जुरे॒व । \newline
26. य॒था॒य॒जुरे॒वैव य॑थाय॒जुर् य॑थाय॒जु रे॒वैतदे॒तदे॒व य॑थाय॒जुर् य॑थाय॒जुरे॒वैतत् । \newline
27. य॒था॒य॒जुरिति॑ यथा - य॒जुः । \newline
28. ए॒वैतदे॒त दे॒वैवैतच् छ॒तꣳ श॒त मे॒तदे॒वैवैतच् छ॒तम् । \newline
29. ए॒तच्छ॒तꣳ श॒त मे॒तदे॒तच्छ॒तꣳ हिमा॒ हिमाः᳚ श॒त मे॒तदे॒तच्छ॒तꣳ हिमाः᳚ । \newline
30. श॒तꣳ हिमा॒ हिमाः᳚ श॒तꣳ श॒तꣳ हिमा॒ इतीति॒ हिमाः᳚ श॒तꣳ श॒तꣳ हिमा॒ इति॑ । \newline
31. हिमा॒ इतीति॒ हिमा॒ हिमा॒ इत्या॑हा॒हे ति॒ हिमा॒ हिमा॒ इत्या॑ह । \newline
32. इत्या॑हा॒हे तीत्या॑ह श॒तꣳ श॒त मा॒हे तीत्या॑ह श॒तम् । \newline
33. आ॒ह॒ श॒तꣳ श॒त मा॑हाह श॒तम् त्वा᳚ त्वा श॒त मा॑हाह श॒तम् त्वा᳚ । \newline
34. श॒तम् त्वा᳚ त्वा श॒तꣳ श॒तम् त्वा॑ हेम॒न्तान्. हे॑म॒न्तान् त्वा॑ श॒तꣳ श॒तम् त्वा॑ हेम॒न्तान् । \newline
35. त्वा॒ हे॒म॒न्तान्. हे॑म॒न्तान् त्वा᳚ त्वा हेम॒न्ता नि॑न्धिषीये न्धिषीय हेम॒न्तान् त्वा᳚ त्वा हेम॒न्ता नि॑न्धिषीय । \newline
36. हे॒म॒न्ता नि॑न्धिषीये न्धिषीय हेम॒न्तान्. हे॑म॒न्ता नि॑न्धिषी॒ये तीती᳚न्धिषीय हेम॒न्तान्. हे॑म॒न्ता नि॑न्धिषी॒ये ति॑ । \newline
37. इ॒न्धि॒षी॒ये तीती᳚न्धिषीये न्धिषी॒ये ति॒ वाव वावे ती᳚न्धिषीये न्धिषी॒ये ति॒ वाव । \newline
38. इति॒ वाव वावे तीति॒ वावैतदे॒तद् वावे तीति॒ वावैतत् । \newline
39. वावैतदे॒तद् वाव वावैतदा॑हाहै॒तद् वाव वावैतदा॑ह । \newline
40. ए॒तदा॑हाहै॒तदे॒तदा॑ह पु॒त्रस्य॑ पु॒त्रस्या॑है॒तदे॒तदा॑ह पु॒त्रस्य॑ । \newline
41. आ॒ह॒ पु॒त्रस्य॑ पु॒त्रस्या॑हाह पु॒त्रस्य॒ नाम॒ नाम॑ पु॒त्रस्या॑हाह पु॒त्रस्य॒ नाम॑ । \newline
42. पु॒त्रस्य॒ नाम॒ नाम॑ पु॒त्रस्य॑ पु॒त्रस्य॒ नाम॑ गृह्णाति गृह्णाति॒ नाम॑ पु॒त्रस्य॑ पु॒त्रस्य॒ नाम॑ गृह्णाति । \newline
43. नाम॑ गृह्णाति गृह्णाति॒ नाम॒ नाम॑ गृह्णात्यन्ना॒द म॑न्ना॒दम् गृ॑ह्णाति॒ नाम॒ नाम॑ गृह्णात्यन्ना॒दम् । \newline
44. गृ॒ह्णा॒त्य॒न्ना॒द म॑न्ना॒दम् गृ॑ह्णाति गृह्णात्यन्ना॒द मे॒वैवान्ना॒दम् गृ॑ह्णाति गृह्णात्यन्ना॒द मे॒व । \newline
45. अ॒न्ना॒द मे॒वैवान्ना॒द म॑न्ना॒द मे॒वैन॑ मेन मे॒वान्ना॒द म॑न्ना॒द मे॒वैन᳚म् । \newline
46. अ॒न्ना॒दमित्य॑न्न - अ॒दम् । \newline
47. ए॒वैन॑ मेन मे॒वैवैन॑म् करोति करोत्येन मे॒वैवैन॑म् करोति । \newline
48. ए॒न॒म् क॒रो॒ति॒ क॒रो॒त्ये॒न॒ मे॒न॒म् क॒रो॒ति॒ ताम् ताम् क॑रोत्येन मेनम् करोति॒ ताम् । \newline
49. क॒रो॒ति॒ ताम् ताम् क॑रोति करोति॒ ता मा॒शिष॑ मा॒शिष॒म् ताम् क॑रोति करोति॒ ता मा॒शिष᳚म् । \newline
50. ता मा॒शिष॑ मा॒शिष॒म् ताम् ता मा॒शिष॒ मा ऽऽशिष॒म् ताम् ता मा॒शिष॒ मा । \newline
51. आ॒शिष॒ मा ऽऽशिष॑ मा॒शिष॒ मा शा॑से शास॒ आ ऽऽशिष॑ मा॒शिष॒ मा शा॑से । \newline
52. आ॒शिष॒मित्या᳚ - शिष᳚म् । \newline
53. आ शा॑से शास॒ आ शा॑से॒ तन्त॑वे॒ तन्त॑वे शास॒ आ शा॑से॒ तन्त॑वे । \newline
54. शा॒से॒ तन्त॑वे॒ तन्त॑वे शासे शासे॒ तन्त॑वे॒ ज्योति॑ष्मती॒म् ज्योति॑ष्मती॒म् तन्त॑वे शासे शासे॒ तन्त॑वे॒ ज्योति॑ष्मतीम् । \newline
55. तन्त॑वे॒ ज्योति॑ष्मती॒म् ज्योति॑ष्मती॒म् तन्त॑वे॒ तन्त॑वे॒ ज्योति॑ष्मती॒ मितीति॒ ज्योति॑ष्मती॒म् तन्त॑वे॒ तन्त॑वे॒ ज्योति॑ष्मती॒ मिति॑ । \newline
56. ज्योति॑ष्मती॒ मितीति॒ ज्योति॑ष्मती॒म् ज्योति॑ष्मती॒ मिति॑ ब्रूयाद् ब्रूया॒दिति॒ ज्योति॑ष्मती॒म् ज्योति॑ष्मती॒ मिति॑ ब्रूयात् । \newline
57. इति॑ ब्रूयाद् ब्रूया॒दितीति॑ ब्रूया॒द् यस्य॒ यस्य॑ ब्रूया॒दितीति॑ ब्रूया॒द् यस्य॑ । \newline
58. ब्रू॒या॒द् यस्य॒ यस्य॑ ब्रूयाद् ब्रूया॒द् यस्य॑ पु॒त्रः पु॒त्रो यस्य॑ ब्रूयाद् ब्रूया॒द् यस्य॑ पु॒त्रः । \newline
59. यस्य॑ पु॒त्रः पु॒त्रो यस्य॒ यस्य॑ पु॒त्रो ऽजा॒तो ऽजा॑तः पु॒त्रो यस्य॒ यस्य॑ पु॒त्रो ऽजा॑तः । \newline
60. पु॒त्रो ऽजा॒तो ऽजा॑तः पु॒त्रः पु॒त्रो ऽजा॑तः॒ स्याथ् स्यादजा॑तः पु॒त्रः पु॒त्रो ऽजा॑तः॒ स्यात् । \newline
61. अजा॑तः॒ स्याथ् स्यादजा॒तो ऽजा॑तः॒ स्यात् ते॑ज॒स्वी ते॑ज॒स्वी स्यादजा॒तो ऽजा॑तः॒ स्यात् ते॑ज॒स्वी । \newline
62. स्यात् ते॑ज॒स्वी ते॑ज॒स्वी स्याथ् स्यात् ते॑ज॒स्व्ये॑वैव ते॑ज॒स्वी स्याथ् स्यात् ते॑ज॒स्व्ये॑व । \newline
63. ते॒ज॒स्व्ये॑वैव ते॑ज॒स्वी ते॑ज॒स्व्ये॑वास्या᳚स्यै॒व ते॑ज॒स्वी ते॑ज॒स्व्ये॑वास्य॑ । \newline
64. ए॒वास्या᳚स्यै॒वैवास्य॑ ब्रह्मवर्च॒सी ब्र॑ह्मवर्च॒स्य॑स्यै॒वैवास्य॑ ब्रह्मवर्च॒सी । \newline
65. अ॒स्य॒ ब्र॒ह्म॒व॒र्च॒सी ब्र॑ह्मवर्च॒स्य॑स्यास्य ब्रह्मवर्च॒सी पु॒त्रः पु॒त्रो ब्र॑ह्मवर्च॒स्य॑स्यास्य ब्रह्मवर्च॒सी पु॒त्रः । \newline
66. ब्र॒ह्म॒व॒र्च॒सी पु॒त्रः पु॒त्रो ब्र॑ह्मवर्च॒सी ब्र॑ह्मवर्च॒सी पु॒त्रो जा॑यते जायते पु॒त्रो ब्र॑ह्मवर्च॒सी ब्र॑ह्मवर्च॒सी पु॒त्रो जा॑यते । \newline
67. ब्र॒ह्म॒व॒र्च॒सीति॑ ब्रह्म - व॒र्च॒सी । \newline
68. पु॒त्रो जा॑यते जायते पु॒त्रः पु॒त्रो जा॑यते॒ ताम् ताम् जा॑यते पु॒त्रः पु॒त्रो जा॑यते॒ ताम् । \newline
69. जा॒य॒ते॒ ताम् ताम् जा॑यते जायते॒ ता मा॒शिष॑ मा॒शिष॒म् ताम् जा॑यते जायते॒ ता मा॒शिष᳚म् । \newline
70. ता मा॒शिष॑ मा॒शिष॒म् ताम् ता मा॒शिष॒ मा ऽऽशिष॒म् ताम् ता मा॒शिष॒ मा । \newline
71. आ॒शिष॒ मा ऽऽशिष॑ मा॒शिष॒ मा शा॑से शास॒ आ ऽऽशिष॑ मा॒शिष॒ मा शा॑से । \newline
72. आ॒शिष॒मित्या᳚ - शिष᳚म् । \newline
73. आ शा॑से शास॒ आ शा॑से॒ ऽमुष्मा॑ अ॒मुष्मै॑ शास॒ आ शा॑से॒ ऽमुष्मै᳚ । \newline
74. शा॒से॒ ऽमुष्मा॑ अ॒मुष्मै॑ शासे शासे॒ ऽमुष्मै॒ ज्योति॑ष्मती॒म् ज्योति॑ष्मती म॒मुष्मै॑ शासे शासे॒ ऽमुष्मै॒ ज्योति॑ष्मतीम् । \newline
75. अ॒मुष्मै॒ ज्योति॑ष्मती॒म् ज्योति॑ष्मती म॒मुष्मा॑ अ॒मुष्मै॒ ज्योति॑ष्मती॒ मितीति॒ ज्योति॑ष्मती म॒मुष्मा॑ अ॒मुष्मै॒ ज्योति॑ष्मती॒ मिति॑ । \newline
76. ज्योति॑ष्मती॒ मितीति॒ ज्योति॑ष्मती॒म् ज्योति॑ष्मती॒ मिति॑ ब्रूयाद् ब्रूया॒दिति॒ ज्योति॑ष्मती॒म् ज्योति॑ष्मती॒ मिति॑ ब्रूयात् । \newline
77. इति॑ ब्रूयाद् ब्रूया॒दितीति॑ ब्रूया॒द् यस्य॒ यस्य॑ ब्रूया॒दितीति॑ ब्रूया॒द् यस्य॑ । \newline
78. ब्रू॒या॒द् यस्य॒ यस्य॑ ब्रूयाद् ब्रूया॒द् यस्य॑ पु॒त्रः पु॒त्रो यस्य॑ ब्रूयाद् ब्रूया॒द् यस्य॑ पु॒त्रः । \newline
79. यस्य॑ पु॒त्रः पु॒त्रो यस्य॒ यस्य॑ पु॒त्रो जा॒तो जा॒तः पु॒त्रो यस्य॒ यस्य॑ पु॒त्रो जा॒तः । \newline
80. पु॒त्रो जा॒तो जा॒तः पु॒त्रः पु॒त्रो जा॒तः स्याथ् स्याज् जा॒तः पु॒त्रः पु॒त्रो जा॒तः स्यात् । \newline
81. जा॒तः स्याथ् स्याज् जा॒तो जा॒तः स्यात् तेज॒ स्तेजः॒ स्याज् जा॒तो जा॒तः स्यात् तेजः॑ । \newline
82. स्यात् तेज॒ स्तेजः॒ स्याथ् स्यात् तेज॑ ए॒वैव तेजः॒ स्याथ् स्यात् तेज॑ ए॒व । \newline
83. तेज॑ ए॒वैव तेज॒स्तेज॑ ए॒वास्मि॑न् नस्मिन् ने॒व तेज॒स्तेज॑ ए॒वास्मिन्न्॑ । \newline
84. ए॒वास्मि॑न् नस्मिन् ने॒वैवास्मि॑न् ब्रह्मवर्च॒सम् ब्र॑ह्मवर्च॒स म॑स्मिन् ने॒वैवास्मि॑न् ब्रह्मवर्च॒सम् । \newline
85. अ॒स्मि॒न् ब्र॒ह्म॒व॒र्च॒सम् ब्र॑ह्मवर्च॒स म॑स्मिन् नस्मिन् ब्रह्मवर्च॒सम् द॑धाति दधाति ब्रह्मवर्च॒स म॑स्मिन् नस्मिन् ब्रह्मवर्च॒सम् द॑धाति । \newline
86. ब्र॒ह्म॒व॒र्च॒सम् द॑धाति दधाति ब्रह्मवर्च॒सम् ब्र॑ह्मवर्च॒सम् द॑धाति । \newline
87. ब्र॒ह्म॒व॒र्च॒समिति॑ ब्रह्म - व॒र्च॒सम् । \newline
88. द॒धा॒तीति॑ दधाति । \newline
\pagebreak
\markright{ TS 1.5.9.1  \hfill https://www.vedavms.in \hfill}
\addcontentsline{toc}{section}{ TS 1.5.9.1 }
\section*{ TS 1.5.9.1 }

\textbf{TS 1.5.9.1 } \newline
\textbf{Samhita Paata} \newline

अ॒ग्नि॒हो॒त्रं जु॑होति॒ यदे॒व किं च॒ यज॑मानस्य॒ स्वं तस्यै॒व तद्रेतः॑ सिञ्चति प्र॒जन॑ने प्र॒जन॑नꣳ॒॒ हि वा अ॒ग्निरथौष॑धी॒रन्त॑गता दहति॒ तास्ततो॒ भूय॑सीः॒ प्र जा॑यन्ते॒ यथ्सा॒यं जु॒होति॒ रेत॑ ए॒व तथ्सि॑ञ्चति॒ प्रैव प्रा॑त॒स्तने॑न जनयति॒ तद्रेतः॑ सि॒क्तं न त्वष्ट्राऽवि॑कृतं॒ प्रजा॑यते याव॒च्छो वै रेत॑सः सि॒क्तस्य॒ - [ ] \newline

\textbf{Pada Paata} \newline

अ॒ग्नि॒हो॒त्रमित्य॑ग्नि - हो॒त्रम् । जु॒हो॒ति॒ । यत् । ए॒व । किम् । च॒ । यज॑मानस्य । स्वम् । तस्य॑ । ए॒व । तत् । रेतः॑ । सि॒ञ्च॒ति॒ । प्र॒जन॑न॒ इति॑ प्र - जन॑ने । प्र॒जन॑न॒मिति॑ प्र - जन॑नम् । हि । वै । अ॒ग्निः । अथ॑ । ओष॑धीः । अन्त॑गता॒ इत्यन्त॑ - ग॒ताः॒ । द॒ह॒ति॒ । ताः । ततः॑ । भूय॑सीः । प्रेति॑ । जा॒य॒न्ते॒ । यत् । सा॒यम् । जु॒होति॑ । रेतः॑ । ए॒व । तत् । सि॒ञ्च॒ति॒ । प्रेति॑ । ए॒व । प्रा॒त॒स्तने॒नेति॑ प्रातः - तने॑न । ज॒न॒य॒ति॒ । तत् । रेतः॑ । सि॒क्तम् । न । त्वष्ट्रा᳚ । अवि॑कृत॒मित्यवि॑ - कृ॒त॒म् । प्रेति॑ । जा॒य॒ते॒ । या॒व॒च्छ इति॑ यावत् - शः । वै । रेत॑सः । सि॒क्तस्य॑ ।  \newline


\textbf{Krama Paata} \newline

अ॒ग्नि॒हो॒त्रं जु॑होति । अ॒ग्नि॒हो॒त्रमित्य॑ग्नि - हो॒त्रम् । जु॒हो॒ति॒ यत् । यदे॒व । ए॒व किम् । किम् च॑ । च॒ यज॑मानस्य । यज॑मानस्य॒ स्वम् । स्वम् तस्य॑ । तस्यै॒व । ए॒व तत् । तद् रेतः॑ । रेतः॑ सिञ्चति । सि॒ञ्च॒ति॒ प्र॒जन॑ने । प्र॒जन॑ने प्र॒जन॑नम् । प्र॒जन॑न॒ इति॑ प्र - जन॑ने । प्र॒जन॑नꣳ॒॒ हि । प्र॒जन॑न॒मिति॑ प्र - जन॑नम् । हि वै । वा अ॒ग्निः । अ॒ग्निरथ॑ । अथौष॑धीः । ओष॑धी॒रन्त॑गताः । अन्त॑गता दहति । अन्त॑गता॒ इत्यन्त॑ - ग॒ताः॒ । द॒ह॒ति॒ ताः । ता स्ततः॑ । ततो॒ भूय॑सीः । भूय॑सीः॒ प्र । प्र जा॑यन्ते । जा॒य॒न्ते॒ यत् । यथ् सा॒यम् । सा॒यम् जु॒होति॑ । जु॒होति॒ रेतः॑ । रेत॑ ए॒व । ए॒व तत् । तथ् सि॑ञ्चति । सि॒ञ्च॒ति॒ प्र । प्रैव । ए॒व प्रा॑त॒स्तने॑न । प्रा॒त॒स्तने॑न जनयति । प्रा॒त॒स्तने॒नेति॑ प्रातः - तने॑न । ज॒न॒य॒ति॒ तत् । तद् रेतः॑ । रेतः॑ सि॒क्तम् । सि॒क्तम् न । न त्वष्ट्रा᳚ । त्वष्ट्राऽवि॑कृतम् । अवि॑कृत॒म् प्र । अवि॑कृत॒मित्यवि॑ - कृ॒त॒म् । प्र जा॑यते । जा॒य॒ते॒ या॒व॒च्छः । या॒व॒च्छो वै । या॒व॒च्छ इति॑ यावत् - शः । वै रेत॑सः । रेत॑सः सि॒क्तस्य॑ । सि॒क्तस्य॒ त्वष्टा᳚ \newline

\textbf{Jatai Paata} \newline

1. अ॒ग्नि॒हो॒त्रम् जु॑होति जुहोत्यग्निहो॒त्र म॑ग्निहो॒त्रम् जु॑होति । \newline
2. अ॒ग्नि॒हो॒त्रमित्य॑ग्नि - हो॒त्रम् । \newline
3. जु॒हो॒ति॒ यद् यज् जु॑होति जुहोति॒ यत् । \newline
4. यदे॒वैव यद् यदे॒व । \newline
5. ए॒व किम् कि मे॒वैव किम् । \newline
6. किम् च॑ च॒ किम् किम् च॑ । \newline
7. च॒ यज॑मानस्य॒ यज॑मानस्य च च॒ यज॑मानस्य । \newline
8. यज॑मानस्य॒ स्वꣳ स्वं ॅयज॑मानस्य॒ यज॑मानस्य॒ स्वम् । \newline
9. स्वम् तस्य॒ तस्य॒ स्वꣳ स्वम् तस्य॑ । \newline
10. तस्यै॒वैव तस्य॒ तस्यै॒व । \newline
11. ए॒व तत् तदे॒वैव तत् । \newline
12. तद् रेतो॒ रेत॒स्तत् तद् रेतः॑ । \newline
13. रेतः॑ सिञ्चति सिञ्चति॒ रेतो॒ रेतः॑ सिञ्चति । \newline
14. सि॒ञ्च॒ति॒ प्र॒जन॑ने प्र॒जन॑ने सिञ्चति सिञ्चति प्र॒जन॑ने । \newline
15. प्र॒जन॑ने प्र॒जन॑नम् प्र॒जन॑नम् प्र॒जन॑ने प्र॒जन॑ने प्र॒जन॑नम् । \newline
16. प्र॒जन॑न॒ इति॑ प्र - जन॑ने । \newline
17. प्र॒जन॑न॒(ग्म्॒) हि हि प्र॒जन॑नम् प्र॒जन॑न॒(ग्म्॒) हि । \newline
18. प्र॒जन॑न॒मिति॑ प्र - जन॑नम् । \newline
19. हि वै वै हि हि वै । \newline
20. वा अ॒ग्निर॒ग्निर् वै वा अ॒ग्निः । \newline
21. अ॒ग्नि रथाथा॒ग्नि र॒ग्निरथ॑ । \newline
22. अथौष॑धी॒ रोष॑धी॒ रथाथौष॑धीः । \newline
23. ओष॑धी॒ रन्त॑गता॒ अन्त॑गता॒ ओष॑धी॒ रोष॑धी॒ रन्त॑गताः । \newline
24. अन्त॑गता दहति दह॒त्यन्त॑गता॒ अन्त॑गता दहति । \newline
25. अन्त॑गता॒ इत्यन्त॑ - ग॒ताः॒ । \newline
26. द॒ह॒ति॒ तास्ता द॑हति दहति॒ ताः । \newline
27. ता स्तत॒ स्तत॒ स्तास्ता स्ततः॑ । \newline
28. ततो॒ भूय॑सी॒र् भूय॑सी॒ स्तत॒स्ततो॒ भूय॑सीः । \newline
29. भूय॑सीः॒ प्र प्र भूय॑सी॒र् भूय॑सीः॒ प्र । \newline
30. प्र जा॑यन्ते जायन्ते॒ प्र प्र जा॑यन्ते । \newline
31. जा॒य॒न्ते॒ यद् यज् जा॑यन्ते जायन्ते॒ यत् । \newline
32. यथ् सा॒यꣳ सा॒यं ॅयद् यथ् सा॒यम् । \newline
33. सा॒यम् जु॒होति॑ जु॒होति॑ सा॒यꣳ सा॒यम् जु॒होति॑ । \newline
34. जु॒होति॒ रेतो॒ रेतो॑ जु॒होति॑ जु॒होति॒ रेतः॑ । \newline
35. रेत॑ ए॒वैव रेतो॒ रेत॑ ए॒व । \newline
36. ए॒व तत् तदे॒वैव तत् । \newline
37. तथ् सि॑ञ्चति सिञ्चति॒ तत् तथ् सि॑ञ्चति । \newline
38. सि॒ञ्च॒ति॒ प्र प्र सि॑ञ्चति सिञ्चति॒ प्र । \newline
39. प्रैवैव प्र प्रैव । \newline
40. ए॒व प्रा॑त॒स्तने॑न प्रात॒स्तने॑नै॒वैव प्रा॑त॒स्तने॑न । \newline
41. प्रा॒त॒स्तने॑न जनयति जनयति प्रात॒स्तने॑न प्रात॒स्तने॑न जनयति । \newline
42. प्रा॒त॒स्तने॒नेति॑ प्रातः - तने॑न । \newline
43. ज॒न॒य॒ति॒ तत् तज् ज॑नयति जनयति॒ तत् । \newline
44. तद् रेतो॒ रेत॒स्तत् तद् रेतः॑ । \newline
45. रेतः॑ सि॒क्तꣳ सि॒क्तꣳ रेतो॒ रेतः॑ सि॒क्तम् । \newline
46. सि॒क्तन्न न सि॒क्तꣳ सि॒क्तन्न । \newline
47. न त्वष्ट्रा॒ त्वष्ट्रा॒ न न त्वष्ट्रा᳚ । \newline
48. त्वष्ट्रा ऽवि॑कृत॒ मवि॑कृत॒म् त्वष्ट्रा॒ त्वष्ट्रा ऽवि॑कृतम् । \newline
49. अवि॑कृत॒म् प्र प्रावि॑कृत॒ मवि॑कृत॒म् प्र । \newline
50. अवि॑कृत॒मित्यवि॑ - कृ॒त॒म् । \newline
51. प्र जा॑यते जायते॒ प्र प्र जा॑यते । \newline
52. जा॒य॒ते॒ या॒व॒च्छो या॑व॒च्छो जा॑यते जायते याव॒च्छः । \newline
53. या॒व॒च्छो वै वै या॑व॒च्छो या॑व॒च्छो वै । \newline
54. या॒व॒च्छ इति॑ यावत् - शः । \newline
55. वै रेत॑सो॒ रेत॑सो॒वै वै रेत॑सः । \newline
56. रेत॑सः सि॒क्तस्य॑ सि॒क्तस्य॒ रेत॑सो॒ रेत॑सः सि॒क्तस्य॑ । \newline
57. सि॒क्तस्य॒ त्वष्टा॒ त्वष्टा॑ सि॒क्तस्य॑ सि॒क्तस्य॒ त्वष्टा᳚ । \newline

\textbf{Ghana Paata } \newline

1. अ॒ग्नि॒हो॒त्रम् जु॑होति जुहोत्यग्निहो॒त्र म॑ग्निहो॒त्रम् जु॑होति॒ यद् यज् जु॑होत्यग्निहो॒त्र म॑ग्निहो॒त्रम् जु॑होति॒ यत् । \newline
2. अ॒ग्नि॒हो॒त्रमित्य॑ग्नि - हो॒त्रम् । \newline
3. जु॒हो॒ति॒ यद् यज् जु॑होति जुहोति॒ यदे॒वैव यज् जु॑होति जुहोति॒ यदे॒व । \newline
4. यदे॒वैव यद् यदे॒व किम् कि मे॒व यद् यदे॒व किम् । \newline
5. ए॒व किम् कि मे॒वैव किम् च॑ च॒ कि मे॒वैव किम् च॑ । \newline
6. किम् च॑ च॒ किम् किम् च॒ यज॑मानस्य॒ यज॑मानस्य च॒ किम् किम् च॒ यज॑मानस्य । \newline
7. च॒ यज॑मानस्य॒ यज॑मानस्य च च॒ यज॑मानस्य॒ स्वꣳ स्वं ॅयज॑मानस्य च च॒ यज॑मानस्य॒ स्वम् । \newline
8. यज॑मानस्य॒ स्वꣳ स्वं ॅयज॑मानस्य॒ यज॑मानस्य॒ स्वम् तस्य॒ तस्य॒ स्वं ॅयज॑मानस्य॒ यज॑मानस्य॒ स्वम् तस्य॑ । \newline
9. स्वम् तस्य॒ तस्य॒ स्वꣳ स्वम् तस्यै॒वैव तस्य॒ स्वꣳ स्वम् तस्यै॒व । \newline
10. तस्यै॒वैव तस्य॒ तस्यै॒व तत् तदे॒व तस्य॒ तस्यै॒व तत् । \newline
11. ए॒व तत् तदे॒वैव तद् रेतो॒ रेत॒ स्तदे॒वैव तद् रेतः॑ । \newline
12. तद् रेतो॒ रेत॒स्तत् तद् रेतः॑ सिञ्चति सिञ्चति॒ रेत॒स्तत् तद् रेतः॑ सिञ्चति । \newline
13. रेतः॑ सिञ्चति सिञ्चति॒ रेतो॒ रेतः॑ सिञ्चति प्र॒जन॑ने प्र॒जन॑ने सिञ्चति॒ रेतो॒ रेतः॑ सिञ्चति प्र॒जन॑ने । \newline
14. सि॒ञ्च॒ति॒ प्र॒जन॑ने प्र॒जन॑ने सिञ्चति सिञ्चति प्र॒जन॑ने प्र॒जन॑नम् प्र॒जन॑नम् प्र॒जन॑ने सिञ्चति सिञ्चति प्र॒जन॑ने प्र॒जन॑नम् । \newline
15. प्र॒जन॑ने प्र॒जन॑नम् प्र॒जन॑नम् प्र॒जन॑ने प्र॒जन॑ने प्र॒जन॑न॒(ग्म्॒) हि हि प्र॒जन॑नम् प्र॒जन॑ने प्र॒जन॑ने प्र॒जन॑न॒(ग्म्॒) हि । \newline
16. प्र॒जन॑न॒ इति॑ प्र - जन॑ने । \newline
17. प्र॒जन॑न॒(ग्म्॒) हि हि प्र॒जन॑नम् प्र॒जन॑न॒(ग्म्॒) हि वै वै हि प्र॒जन॑नम् प्र॒जन॑न॒(ग्म्॒) हि वै । \newline
18. प्र॒जन॑न॒मिति॑ प्र - जन॑नम् । \newline
19. हि वै वै हि हि वा अ॒ग्निर॒ग्निर् वै हि हि वा अ॒ग्निः । \newline
20. वा अ॒ग्निर॒ग्निर् वै वा अ॒ग्नि रथाथा॒ग्निर् वै वा अ॒ग्निरथ॑ । \newline
21. अ॒ग्नि रथाथा॒ग्नि र॒ग्नि रथौष॑धी॒रोष॑धी॒ रथा॒ग्नि र॒ग्नि रथौष॑धीः । \newline
22. अथौष॑धी॒ रोष॑धी॒ रथाथौष॑धी॒ रन्त॑गता॒ अन्त॑गता॒ ओष॑धी॒ रथाथौष॑धी॒ रन्त॑गताः । \newline
23. ओष॑धी॒रन्त॑गता॒ अन्त॑गता॒ ओष॑धी॒ रोष॑धी॒रन्त॑गता दहति दह॒त्यन्त॑गता॒ ओष॑धी॒ रोष॑धी॒रन्त॑गता दहति । \newline
24. अन्त॑गता दहति दह॒त्यन्त॑गता॒ अन्त॑गता दहति॒ तास्ता द॑ह॒त्यन्त॑गता॒ अन्त॑गता दहति॒ ताः । \newline
25. अन्त॑गता॒ इत्यन्त॑ - ग॒ताः॒ । \newline
26. द॒ह॒ति॒ तास्ता द॑हति दहति॒ तास्तत॒ स्तत॒स्ता द॑हति दहति॒ तास्ततः॑ । \newline
27. तास्तत॒ स्तत॒स्तास्ता स्ततो॒ भूय॑सी॒र् भूय॑सी॒ स्तत॒स्तास्ता स्ततो॒ भूय॑सीः । \newline
28. ततो॒ भूय॑सी॒र् भूय॑सी॒ स्तत॒ स्ततो॒ भूय॑सीः॒ प्र प्र भूय॑सी॒ स्तत॒ स्ततो॒ भूय॑सीः॒ प्र । \newline
29. भूय॑सीः॒ प्र प्र भूय॑सी॒र् भूय॑सीः॒ प्र जा॑यन्ते जायन्ते॒ प्र भूय॑सी॒र् भूय॑सीः॒ प्र जा॑यन्ते । \newline
30. प्र जा॑यन्ते जायन्ते॒ प्र प्र जा॑यन्ते॒ यद् यज् जा॑यन्ते॒ प्र प्र जा॑यन्ते॒ यत् । \newline
31. जा॒य॒न्ते॒ यद् यज् जा॑यन्ते जायन्ते॒ यथ् सा॒यꣳ सा॒यं ॅयज् जा॑यन्ते जायन्ते॒ यथ् सा॒यम् । \newline
32. यथ् सा॒यꣳ सा॒यं ॅयद् यथ् सा॒यम् जु॒होति॑ जु॒होति॑ सा॒यं ॅयद् यथ् सा॒यम् जु॒होति॑ । \newline
33. सा॒यम् जु॒होति॑ जु॒होति॑ सा॒यꣳ सा॒यम् जु॒होति॒ रेतो॒ रेतो॑ जु॒होति॑ सा॒यꣳ सा॒यम् जु॒होति॒ रेतः॑ । \newline
34. जु॒होति॒ रेतो॒ रेतो॑ जु॒होति॑ जु॒होति॒ रेत॑ ए॒वैव रेतो॑ जु॒होति॑ जु॒होति॒ रेत॑ ए॒व । \newline
35. रेत॑ ए॒वैव रेतो॒ रेत॑ ए॒व तत् तदे॒व रेतो॒ रेत॑ ए॒व तत् । \newline
36. ए॒व तत् तदे॒वैव तथ् सि॑ञ्चति सिञ्चति॒ तदे॒वैव तथ् सि॑ञ्चति । \newline
37. तथ् सि॑ञ्चति सिञ्चति॒ तत् तथ् सि॑ञ्चति॒ प्र प्र सि॑ञ्चति॒ तत् तथ् सि॑ञ्चति॒ प्र । \newline
38. सि॒ञ्च॒ति॒ प्र प्र सि॑ञ्चति सिञ्चति॒ प्रैवैव प्र सि॑ञ्चति सिञ्चति॒ प्रैव । \newline
39. प्रैवैव प्र प्रैव प्रा॑त॒स्तने॑न प्रात॒स्तने॑नै॒व प्र प्रैव प्रा॑त॒स्तने॑न । \newline
40. ए॒व प्रा॑त॒स्तने॑न प्रात॒स्तने॑नै॒वैव प्रा॑त॒स्तने॑न जनयति जनयति प्रात॒स्तने॑नै॒वैव प्रा॑त॒स्तने॑न जनयति । \newline
41. प्रा॒त॒स्तने॑न जनयति जनयति प्रात॒स्तने॑न प्रात॒स्तने॑न जनयति॒ तत् तज् ज॑नयति प्रात॒स्तने॑न प्रात॒स्तने॑न जनयति॒ तत् । \newline
42. प्रा॒त॒स्तने॒नेति॑ प्रातः - तने॑न । \newline
43. ज॒न॒य॒ति॒ तत् तज् ज॑नयति जनयति॒ तद् रेतो॒ रेत॒स्तज् ज॑नयति जनयति॒ तद् रेतः॑ । \newline
44. तद् रेतो॒ रेत॒स्तत् तद् रेतः॑ सि॒क्तꣳ सि॒क्तꣳ रेत॒स्तत् तद् रेतः॑ सि॒क्तम् । \newline
45. रेतः॑ सि॒क्तꣳ सि॒क्तꣳ रेतो॒ रेतः॑ सि॒क्तन्न न सि॒क्तꣳ रेतो॒ रेतः॑ सि॒क्तन्न । \newline
46. सि॒क्तन्न न सि॒क्तꣳ सि॒क्तन्न त्वष्ट्रा॒ त्वष्ट्रा॒ न सि॒क्तꣳ सि॒क्तन्न त्वष्ट्रा᳚ । \newline
47. न त्वष्ट्रा॒ त्वष्ट्रा॒ न न त्वष्ट्रा ऽवि॑कृत॒ मवि॑कृत॒म् त्वष्ट्रा॒ न न त्वष्ट्रा ऽवि॑कृतम् । \newline
48. त्वष्ट्रा ऽवि॑कृत॒ मवि॑कृत॒म् त्वष्ट्रा॒ त्वष्ट्रा ऽवि॑कृत॒म् प्र प्रावि॑कृत॒म् त्वष्ट्रा॒ त्वष्ट्रा ऽवि॑कृत॒म् प्र । \newline
49. अवि॑कृत॒म् प्र प्रावि॑कृत॒ मवि॑कृत॒म् प्र जा॑यते जायते॒ प्रावि॑कृत॒ मवि॑कृत॒म् प्र जा॑यते । \newline
50. अवि॑कृत॒मित्यवि॑ - कृ॒त॒म् । \newline
51. प्र जा॑यते जायते॒ प्र प्र जा॑यते याव॒च्छो या॑व॒च्छो जा॑यते॒ प्र प्र जा॑यते याव॒च्छः । \newline
52. जा॒य॒ते॒ या॒व॒च्छो या॑व॒च्छो जा॑यते जायते याव॒च्छो वै वै या॑व॒च्छो जा॑यते जायते याव॒च्छो वै । \newline
53. या॒व॒च्छो वै वै या॑व॒च्छो या॑व॒च्छो वै रेत॑सो॒ रेत॑सो॒ वै या॑व॒च्छो या॑व॒च्छो वै रेत॑सः । \newline
54. या॒व॒च्छ इति॑ यावत् - शः । \newline
55. वै रेत॑सो॒ रेत॑सो॒ वै वै रेत॑सः सि॒क्तस्य॑ सि॒क्तस्य॒ रेत॑सो॒ वै वै रेत॑सः सि॒क्तस्य॑ । \newline
56. रेत॑सः सि॒क्तस्य॑ सि॒क्तस्य॒ रेत॑सो॒ रेत॑सः सि॒क्तस्य॒ त्वष्टा॒ त्वष्टा॑ सि॒क्तस्य॒ रेत॑सो॒ रेत॑सः सि॒क्तस्य॒ त्वष्टा᳚ । \newline
57. सि॒क्तस्य॒ त्वष्टा॒ त्वष्टा॑ सि॒क्तस्य॑ सि॒क्तस्य॒ त्वष्टा॑ रू॒पाणि॑ रू॒पाणि॒ त्वष्टा॑ सि॒क्तस्य॑ सि॒क्तस्य॒ त्वष्टा॑ रू॒पाणि॑ । \newline
\pagebreak
\markright{ TS 1.5.9.2  \hfill https://www.vedavms.in \hfill}
\addcontentsline{toc}{section}{ TS 1.5.9.2 }
\section*{ TS 1.5.9.2 }

\textbf{TS 1.5.9.2 } \newline
\textbf{Samhita Paata} \newline

त्वष्टा॑ रू॒पाणि॑ विक॒रोति॑ ताव॒च्छो वै तत्प्र जा॑यत ए॒ष वै दैव्य॒स्त्वष्टा॒ यो यज॑ते ब॒ह्वीभि॒रुप॑ तिष्ठते॒ रेत॑स ए॒व सि॒क्तस्य॑ बहु॒शो रू॒पाणि॒ वि क॑रोति॒ स प्रैव जा॑यते॒ श्वःश्वो॒ भूया᳚न् भवति॒ य ए॒वं ॅवि॒द्वान॒ग्निमु॑प॒तिष्ठ॒ते ऽह॑र्दे॒वाना॒मासी॒द् - रात्रि॒रसु॑राणां॒ तेऽसु॑रा॒ यद्दे॒वानां᳚ ॅवि॒त्तं ॅवेद्य॒मासी॒त्तेन॑ स॒ह- [ ] \newline

\textbf{Pada Paata} \newline

त्वष्टा᳚ । रू॒पाणि॑ । वि॒क॒रोतीति॑ वि - क॒रोति॑ । ता॒व॒च्छ इति॑ तावत्-शः । वै । तत् । प्रेति॑ । जा॒य॒ते॒ । ए॒षः । वै । दैव्यः॑ । त्वष्टा᳚ । यः । यज॑ते । ब॒ह्वीभिः॑ । उपेति॑ । ति॒ष्ठ॒ते॒ । रेत॑सः । ए॒व । सि॒क्तस्य॑ । ब॒हु॒श इति॑ बहु - शः । रू॒पाणि॑ । वीति॑ । क॒रो॒ति॒ । सः । प्रेति॑ । ए॒व । जा॒य॒ते॒ । श्वः श्व॒ इति॒ श्वः - श्वः॒ । भूयान्॑ । भ॒व॒ति॒ । यः । ए॒वम् । वि॒द्वान् । अ॒ग्निम् । उ॒प॒तिष्ठ॑त॒ इत्यु॑प - तिष्ठ॑ते । अहः॑ । दे॒वाना᳚म् । आसी᳚त् । रात्रिः॑ । असु॑राणाम् । ते । असु॑राः । यत् । दे॒वाना᳚म् । वि॒त्तम् । वेद्य᳚म् । असी᳚त् । तेन॑ । स॒ह ।  \newline


\textbf{Krama Paata} \newline

त्वष्टा॑ रू॒पाणि॑ । रू॒पाणि॑ विक॒रोति॑ । वि॒क॒रोति॑ ताव॒च्छः । वि॒क॒रोतीति॑ वि - क॒रोति॑ । ता॒व॒च्छो वै । ता॒व॒च्छ इति॑ तावत् - शः । वै तत् । तत् प्र । प्र जा॑यते । जा॒य॒त॒ ए॒षः । ए॒ष वै । वै दैव्यः॑ । दैव्य॒स्त्वष्टा᳚ । त्वष्टा॒ यः । यो यज॑ते । यज॑ते ब॒ह्वीभिः॑ । ब॒ह्वीभि॒रुप॑ । उप॑ तिष्ठते । ति॒ष्ठ॒ते॒ रेत॑सः । रेत॑स ए॒व । ए॒व सि॒क्तस्य॑ । सि॒क्तस्य॑ बहु॒शः । ब॒हु॒शो रू॒पाणि॑ । ब॒हु॒श इति॑ बहु - शः । रू॒पाणि॒ वि । वि क॑रोति । क॒रो॒ति॒ सः । स प्र । प्रैव । ए॒व जा॑यते । जा॒य॒ते॒ श्वःश्वः॑ । श्वःश्वो भूयान्॑ । श्वःश्व॒ इति॒ श्वः - श्वः॒ । भूया᳚न् भवति । भ॒व॒ति॒ यः । य ए॒वम् । ए॒वं ॅवि॒द्वान् । वि॒द्वान॒ग्निम् । अ॒ग्निमु॑प॒तिष्ठ॑ते । उ॒प॒तिष्ठ॒तेऽहः॑ । उ॒प॒तिष्ठ॑त॒ इत्यु॑प - तिष्ठ॑ते । अह॑र् दे॒वाना᳚म् । दे॒वाना॒मासी᳚त् । आसी॒द् रात्रिः॑ । रात्रि॒रसु॑राणाम् । असु॑राणा॒म् ते । तेऽसु॑राः । असु॑रा॒ यत् । यद् दे॒वाना᳚म् । दे॒वानां᳚ ॅवि॒त्तम् । वि॒त्तं ॅवेद्य᳚म् । वेद्य॒मासी᳚त् । आसी॒त् तेन॑ । तेन॑ स॒ह । स॒ह रात्रि᳚म् \newline

\textbf{Jatai Paata} \newline

1. त्वष्टा॑ रू॒पाणि॑ रू॒पाणि॒ त्वष्टा॒ त्वष्टा॑ रू॒पाणि॑ । \newline
2. रू॒पाणि॑ विक॒रोति॑ विक॒रोति॑ रू॒पाणि॑ रू॒पाणि॑ विक॒रोति॑ । \newline
3. वि॒क॒रोति॑ ताव॒च्छ स्ता॑व॒च्छो वि॑क॒रोति॑ विक॒रोति॑ ताव॒च्छः । \newline
4. वि॒क॒रोतीति॑ वि - क॒रोति॑ । \newline
5. ता॒व॒च्छो वै वै ता॑व॒च्छ स्ता॑व॒च्छो वै । \newline
6. ता॒व॒च्छ इति॑ तावत् - शः । \newline
7. वै तत् तद् वै वै तत् । \newline
8. तत् प्र प्र तत् तत् प्र । \newline
9. प्र जा॑यते जायते॒ प्र प्र जा॑यते । \newline
10. जा॒य॒त॒ ए॒ष ए॒ष जा॑यते जायत ए॒षः । \newline
11. ए॒ष वै वा ए॒ष ए॒ष वै । \newline
12. वै दैव्यो॒ दैव्यो॒ वै वै दैव्यः॑ । \newline
13. दैव्य॒ स्त्वष्टा॒ त्वष्टा॒ दैव्यो॒ दैव्य॒ स्त्वष्टा᳚ । \newline
14. त्वष्टा॒ यो यस्त्वष्टा॒ त्वष्टा॒ यः । \newline
15. यो यज॑ते॒ यज॑ते॒ यो यो यज॑ते । \newline
16. यज॑ते ब॒ह्वीभि॑र् ब॒ह्वीभि॒र् यज॑ते॒ यज॑ते ब॒ह्वीभिः॑ । \newline
17. ब॒ह्वीभि॒ रुपोप॑ ब॒ह्वीभि॑र् ब॒ह्वीभि॒रुप॑ । \newline
18. उप॑ तिष्ठते तिष्ठत॒ उपोप॑ तिष्ठते । \newline
19. ति॒ष्ठ॒ते॒ रेत॑सो॒ रेत॑स स्तिष्ठते तिष्ठते॒ रेत॑सः । \newline
20. रेत॑स ए॒वैव रेत॑सो॒ रेत॑स ए॒व । \newline
21. ए॒व सि॒क्तस्य॑ सि॒क्तस्यै॒वैव सि॒क्तस्य॑ । \newline
22. सि॒क्तस्य॑ बहु॒शो ब॑हु॒शः सि॒क्तस्य॑ सि॒क्तस्य॑ बहु॒शः । \newline
23. ब॒हु॒शो रू॒पाणि॑ रू॒पाणि॑ बहु॒शो ब॑हु॒शो रू॒पाणि॑ । \newline
24. ब॒हु॒श इति॑ बहु - शः । \newline
25. रू॒पाणि॒ वि वि रू॒पाणि॑ रू॒पाणि॒ वि । \newline
26. वि क॑रोति करोति॒ वि वि क॑रोति । \newline
27. क॒रो॒ति॒ स स क॑रोति करोति॒ सः । \newline
28. स प्र प्र स स प्र । \newline
29. प्रैवैव प्र प्रैव । \newline
30. ए॒व जा॑यते जायत ए॒वैव जा॑यते । \newline
31. जा॒य॒ते॒ श्वःश्वः॒ श्वःश्वो॑ जायते जायते॒ श्वःश्वः॑ । \newline
32. श्वःश्वो॒ भूया॒न् भूया॒ञ् छ्‌वःश्वः॒ श्वःश्वो॒ भूयान्॑ । \newline
33. श्वःश्व॒ इति॒ श्वः - श्वः॒ । \newline
34. भूया᳚न् भवति भवति॒ भूया॒न् भूया᳚न् भवति । \newline
35. भ॒व॒ति॒ यो यो भ॑वति भवति॒ यः । \newline
36. य ए॒व मे॒वं ॅयो य ए॒वम् । \newline
37. ए॒वं ॅवि॒द्वान्. वि॒द्वा ने॒व मे॒वं ॅवि॒द्वान् । \newline
38. वि॒द्वा न॒ग्नि म॒ग्निं ॅवि॒द्वान्. वि॒द्वा न॒ग्निम् । \newline
39. अ॒ग्नि मु॑प॒तिष्ठ॑त उप॒तिष्ठ॑ते॒ ऽग्नि म॒ग्नि मु॑प॒तिष्ठ॑ते । \newline
40. उ॒प॒तिष्ठ॒ते ऽह॒ रह॑ रुप॒तिष्ठ॑त उप॒तिष्ठ॒ते ऽहः॑ । \newline
41. उ॒प॒तिष्ठ॑त॒ इत्यु॑प - तिष्ठ॑ते । \newline
42. अह॑र् दे॒वाना᳚म् दे॒वाना॒ मह॒ रह॑र् दे॒वाना᳚म् । \newline
43. दे॒वाना॒ मासी॒दासी᳚द् दे॒वाना᳚म् दे॒वाना॒ मासी᳚त् । \newline
44. आसी॒द् रात्री॒ रात्रि॒रासी॒ दासी॒द् रात्रिः॑ । \newline
45. रात्रि॒रसु॑राणा॒ मसु॑राणा॒(ग्म्॒) रात्री॒ रात्रि॒रसु॑राणाम् । \newline
46. असु॑राणा॒म् ते ते ऽसु॑राणा॒ मसु॑राणा॒म् ते । \newline
47. ते ऽसु॑रा॒ असु॑रा॒स्ते ते ऽसु॑राः । \newline
48. असु॑रा॒ यद् यदसु॑रा॒ असु॑रा॒ यत् । \newline
49. यद् दे॒वाना᳚म् दे॒वानां॒ ॅयद् यद् दे॒वाना᳚म् । \newline
50. दे॒वानां᳚ ॅवि॒त्तं ॅवि॒त्तम् दे॒वाना᳚म् दे॒वानां᳚ ॅवि॒त्तम् । \newline
51. वि॒त्तं ॅवेद्यं॒ ॅवेद्यं॑ ॅवि॒त्तं ॅवि॒त्तं ॅवेद्य᳚म् । \newline
52. वेद्य॒ मासी॒दासी॒द् वेद्यं॒ ॅवेद्य॒ मासी᳚त् । \newline
53. आसी॒त् तेन॒ तेनासी॒ दासी॒त् तेन॑ । \newline
54. तेन॑ स॒ह स॒ह तेन॒ तेन॑ स॒ह । \newline
55. स॒ह रात्रि॒(ग्म्॒) रात्रि(ग्म्॑) स॒ह स॒ह रात्रि᳚म् । \newline

\textbf{Ghana Paata } \newline

1. त्वष्टा॑ रू॒पाणि॑ रू॒पाणि॒ त्वष्टा॒ त्वष्टा॑ रू॒पाणि॑ विक॒रोति॑ विक॒रोति॑ रू॒पाणि॒ त्वष्टा॒ त्वष्टा॑ रू॒पाणि॑ विक॒रोति॑ । \newline
2. रू॒पाणि॑ विक॒रोति॑ विक॒रोति॑ रू॒पाणि॑ रू॒पाणि॑ विक॒रोति॑ ताव॒च्छ स्ता॑व॒च्छो वि॑क॒रोति॑ रू॒पाणि॑ रू॒पाणि॑ विक॒रोति॑ ताव॒च्छः । \newline
3. वि॒क॒रोति॑ ताव॒च्छ स्ता॑व॒च्छो वि॑क॒रोति॑ विक॒रोति॑ ताव॒च्छो वै वै ता॑व॒च्छो वि॑क॒रोति॑ विक॒रोति॑ ताव॒च्छो वै । \newline
4. वि॒क॒रोतीति॑ वि - क॒रोति॑ । \newline
5. ता॒व॒च्छो वै वै ता॑व॒च्छ स्ता॑व॒च्छो वै तत् तद् वै ता॑व॒च्छ स्ता॑व॒च्छो वै तत् । \newline
6. ता॒व॒च्छ इति॑ तावत् - शः । \newline
7. वै तत् तद् वै वै तत् प्र प्र तद् वै वै तत् प्र । \newline
8. तत् प्र प्र तत् तत् प्र जा॑यते जायते॒ प्र तत् तत् प्र जा॑यते । \newline
9. प्र जा॑यते जायते॒ प्र प्र जा॑यत ए॒ष ए॒ष जा॑यते॒ प्र प्र जा॑यत ए॒षः । \newline
10. जा॒य॒त॒ ए॒ष ए॒ष जा॑यते जायत ए॒ष वै वा ए॒ष जा॑यते जायत ए॒ष वै । \newline
11. ए॒ष वै वा ए॒ष ए॒ष वै दैव्यो॒ दैव्यो॒ वा ए॒ष ए॒ष वै दैव्यः॑ । \newline
12. वै दैव्यो॒ दैव्यो॒ वै वै दैव्य॒ स्त्वष्टा॒ त्वष्टा॒ दैव्यो॒ वै वै दैव्य॒ स्त्वष्टा᳚ । \newline
13. दैव्य॒स्त्वष्टा॒ त्वष्टा॒ दैव्यो॒ दैव्य॒स्त्वष्टा॒ यो यस्त्वष्टा॒ दैव्यो॒ दैव्य॒स्त्वष्टा॒ यः । \newline
14. त्वष्टा॒ यो यस्त्वष्टा॒ त्वष्टा॒ यो यज॑ते॒ यज॑ते॒ यस्त्वष्टा॒ त्वष्टा॒ यो यज॑ते । \newline
15. यो यज॑ते॒ यज॑ते॒ यो यो यज॑ते ब॒ह्वीभि॑र् ब॒ह्वीभि॒र् यज॑ते॒ यो यो यज॑ते ब॒ह्वीभिः॑ । \newline
16. यज॑ते ब॒ह्वीभि॑र् ब॒ह्वीभि॒र् यज॑ते॒ यज॑ते ब॒ह्वीभि॒रुपोप॑ ब॒ह्वीभि॒र् यज॑ते॒ यज॑ते ब॒ह्वीभि॒रुप॑ । \newline
17. ब॒ह्वीभि॒रुपोप॑ ब॒ह्वीभि॑र् ब॒ह्वीभि॒रुप॑ तिष्ठते तिष्ठत॒ उप॑ ब॒ह्वीभि॑र् ब॒ह्वीभि॒रुप॑ तिष्ठते । \newline
18. उप॑ तिष्ठते तिष्ठत॒ उपोप॑ तिष्ठते॒ रेत॑सो॒ रेत॑स स्तिष्ठत॒ उपोप॑ तिष्ठते॒ रेत॑सः । \newline
19. ति॒ष्ठ॒ते॒ रेत॑सो॒ रेत॑सस्तिष्ठते तिष्ठते॒ रेत॑स ए॒वैव रेत॑सस्तिष्ठते तिष्ठते॒ रेत॑स ए॒व । \newline
20. रेत॑स ए॒वैव रेत॑सो॒ रेत॑स ए॒व सि॒क्तस्य॑ सि॒क्तस्यै॒व रेत॑सो॒ रेत॑स ए॒व सि॒क्तस्य॑ । \newline
21. ए॒व सि॒क्तस्य॑ सि॒क्तस्यै॒वैव सि॒क्तस्य॑ बहु॒शो ब॑हु॒शः सि॒क्तस्यै॒वैव सि॒क्तस्य॑ बहु॒शः । \newline
22. सि॒क्तस्य॑ बहु॒शो ब॑हु॒शः सि॒क्तस्य॑ सि॒क्तस्य॑ बहु॒शो रू॒पाणि॑ रू॒पाणि॑ बहु॒शः सि॒क्तस्य॑ सि॒क्तस्य॑ बहु॒शो रू॒पाणि॑ । \newline
23. ब॒हु॒शो रू॒पाणि॑ रू॒पाणि॑ बहु॒शो ब॑हु॒शो रू॒पाणि॒ वि वि रू॒पाणि॑ बहु॒शो ब॑हु॒शो रू॒पाणि॒ वि । \newline
24. ब॒हु॒श इति॑ बहु - शः । \newline
25. रू॒पाणि॒ वि वि रू॒पाणि॑ रू॒पाणि॒ वि क॑रोति करोति॒ वि रू॒पाणि॑ रू॒पाणि॒ वि क॑रोति । \newline
26. वि क॑रोति करोति॒ वि वि क॑रोति॒ स स क॑रोति॒ वि वि क॑रोति॒ सः । \newline
27. क॒रो॒ति॒ स स क॑रोति करोति॒ स प्र प्र स क॑रोति करोति॒ स प्र । \newline
28. स प्र प्र स स प्रैवैव प्र स स प्रैव । \newline
29. प्रैवैव प्र प्रैव जा॑यते जायत ए॒व प्र प्रैव जा॑यते । \newline
30. ए॒व जा॑यते जायत ए॒वैव जा॑यते॒ श्वःश्वः॒ श्वःश्वो॑ जायत ए॒वैव जा॑यते॒ श्वःश्वः॑ । \newline
31. जा॒य॒ते॒ श्वःश्वः॒ श्वःश्वो॑ जायते जायते॒ श्वःश्वो॒ भूया॒न् भूया॒न् श्वःश्वो॑ जायते जायते॒ श्वःश्वो॒ भूयान्॑ । \newline
32. श्वःश्वो॒ भूया॒न् भूया॒न् श्वःश्वः॒ श्वःश्वो॒ भूया᳚न् भवति भवति॒ भूया॒न् श्वःश्वः॒ श्वःश्वो॒ भूया᳚न् भवति । \newline
33. श्वःश्व॒ इति॒ श्वः - श्वः॒ । \newline
34. भूया᳚न् भवति भवति॒ भूया॒न् भूया᳚न् भवति॒ यो यो भ॑वति॒ भूया॒न् भूया᳚न् भवति॒ यः । \newline
35. भ॒व॒ति॒ यो यो भ॑वति भवति॒ य ए॒व मे॒वं ॅयो भ॑वति भवति॒ य ए॒वम् । \newline
36. य ए॒व मे॒वं ॅयो य ए॒वं ॅवि॒द्वान्. वि॒द्वा ने॒वं ॅयो य ए॒वं ॅवि॒द्वान् । \newline
37. ए॒वं ॅवि॒द्वान्. वि॒द्वा ने॒व मे॒वं ॅवि॒द्वा न॒ग्नि म॒ग्निं ॅवि॒द्वा ने॒व मे॒वं ॅवि॒द्वा न॒ग्निम् । \newline
38. वि॒द्वा न॒ग्नि म॒ग्निं ॅवि॒द्वान्. वि॒द्वा न॒ग्नि मु॑प॒तिष्ठ॑त उप॒तिष्ठ॑ते॒ ऽग्निं ॅवि॒द्वान्. वि॒द्वा न॒ग्नि मु॑प॒तिष्ठ॑ते । \newline
39. अ॒ग्नि मु॑प॒तिष्ठ॑त उप॒तिष्ठ॑ते॒ ऽग्नि म॒ग्नि मु॑प॒तिष्ठ॒ते ऽह॒ रह॑ रुप॒तिष्ठ॑ते॒ ऽग्नि म॒ग्नि मु॑प॒तिष्ठ॒ते ऽहः॑ । \newline
40. उ॒प॒तिष्ठ॒ते ऽह॒ रह॑ रुप॒तिष्ठ॑त उप॒तिष्ठ॒ते ऽह॑र् दे॒वाना᳚म् दे॒वाना॒ मह॑ रुप॒तिष्ठ॑त उप॒तिष्ठ॒ते ऽह॑र् दे॒वाना᳚म् । \newline
41. उ॒प॒तिष्ठ॑त॒ इत्यु॑प - तिष्ठ॑ते । \newline
42. अह॑र् दे॒वाना᳚म् दे॒वाना॒ मह॒ रह॑र् दे॒वाना॒ मासी॒दासी᳚द् दे॒वाना॒ मह॒ रह॑र् दे॒वाना॒ मासी᳚त् । \newline
43. दे॒वाना॒ मासी॒दासी᳚द् दे॒वाना᳚म् दे॒वाना॒ मासी॒द् रात्री॒ रात्रि॒रासी᳚द् दे॒वाना᳚म् दे॒वाना॒ मासी॒द् रात्रिः॑ । \newline
44. आसी॒द् रात्री॒ रात्रि॒ रासी॒दासी॒द् रात्रि॒रसु॑राणा॒ मसु॑राणा॒(ग्म्॒) रात्रि॒ रासी॒दासी॒द् रात्रि॒ रसु॑राणाम् । \newline
45. रात्रि॒रसु॑राणा॒ मसु॑राणा॒(ग्म्॒) रात्री॒ रात्रि॒रसु॑राणा॒म् ते ते ऽसु॑राणा॒(ग्म्॒) रात्री॒ रात्रि॒रसु॑राणा॒म् ते । \newline
46. असु॑राणा॒म् ते ते ऽसु॑राणा॒ मसु॑राणा॒म् ते ऽसु॑रा॒ असु॑रा॒स्ते ऽसु॑राणा॒ मसु॑राणा॒म् ते ऽसु॑राः । \newline
47. ते ऽसु॑रा॒ असु॑रा॒स्ते ते ऽसु॑रा॒ यद् यदसु॑रा॒स्ते ते ऽसु॑रा॒ यत् । \newline
48. असु॑रा॒ यद् यदसु॑रा॒ असु॑रा॒ यद् दे॒वाना᳚म् दे॒वानां॒ ॅयदसु॑रा॒ असु॑रा॒ यद् दे॒वाना᳚म् । \newline
49. यद् दे॒वाना᳚म् दे॒वानां॒ ॅयद् यद् दे॒वानां᳚ ॅवि॒त्तं ॅवि॒त्तम् दे॒वानां॒ ॅयद् यद् दे॒वानां᳚ ॅवि॒त्तम् । \newline
50. दे॒वानां᳚ ॅवि॒त्तं ॅवि॒त्तम् दे॒वाना᳚म् दे॒वानां᳚ ॅवि॒त्तं ॅवेद्यं॒ ॅवेद्यं॑ ॅवि॒त्तम् दे॒वाना᳚म् दे॒वानां᳚ ॅवि॒त्तं ॅवेद्य᳚म् । \newline
51. वि॒त्तं ॅवेद्यं॒ ॅवेद्यं॑ ॅवि॒त्तं ॅवि॒त्तं ॅवेद्य॒ मासी॒दसी॒द् वेद्यं॑ ॅवि॒त्तं ॅवि॒त्तं ॅवेद्य॒ मासी᳚त् । \newline
52. वेद्य॒ मासी॒दासी॒द् वेद्यं॒ ॅवेद्य॒ मासी॒त् तेन॒ तेनासी॒द् वेद्यं॒ ॅवेद्य॒ मासी॒त् तेन॑ । \newline
53. आसी॒त् तेन॒ तेनासी॒ दासी॒त् तेन॑ स॒ह स॒ह तेनासी॒ दासी॒त् तेन॑ स॒ह । \newline
54. तेन॑ स॒ह स॒ह तेन॒ तेन॑ स॒ह रात्रि॒(ग्म्॒) रात्रि(ग्म्॑) स॒ह तेन॒ तेन॑ स॒ह रात्रि᳚म् । \newline
55. स॒ह रात्रि॒(ग्म्॒) रात्रि(ग्म्॑) स॒ह स॒ह रात्रि॒म् प्र प्र रात्रि(ग्म्॑) स॒ह स॒ह रात्रि॒म् प्र । \newline
\pagebreak
\markright{ TS 1.5.9.3  \hfill https://www.vedavms.in \hfill}
\addcontentsline{toc}{section}{ TS 1.5.9.3 }
\section*{ TS 1.5.9.3 }

\textbf{TS 1.5.9.3 } \newline
\textbf{Samhita Paata} \newline

रात्रिं॒ प्राऽ*वि॑श॒न्ते दे॒वा ही॒ना अ॑मन्यन्त॒ ते॑ऽपश्यन्नाग्ने॒यी रात्रि॑राग्ने॒याः प॒शव॑ इ॒ममे॒वाग्निꣳ स्त॑वाम॒ स नः॑ स्तु॒तः प॒शून् पुन॑र्दास्य॒तीति॒ ते᳚ऽग्निम॑स्तुव॒न्थ् स ए᳚भ्यः स्तु॒तो रात्रि॑या॒ अद्ध्यह॑र॒भि प॒शून्निरा᳚र्ज॒त्ते दे॒वाः प॒शून् वि॒त्त्वा कामाꣳ॑ अकुर्वत॒ य ए॒वं ॅवि॒द्वान॒ग्निमु॑प॒तिष्ठ॑ते पशु॒मान् भ॑व - [ ] \newline

\textbf{Pada Paata} \newline

रात्रि᳚म् । प्रेति॑ । अ॒वि॒श॒न्न् । ते । दे॒वाः । ही॒नाः । अ॒म॒न्य॒न्त॒ । ते । अ॒प॒श्य॒न्न् । आ॒ग्ने॒यी । रात्रिः॑ । आ॒ग्ने॒याः । प॒शवः॑ । इ॒मम् । ए॒व । अ॒ग्निम् । स्त॒वा॒म॒ । सः । नः॒ । स्तु॒तः । प॒शून् । पुनः॑ । दा॒स्य॒ति॒ । इति॑ । ते । अ॒ग्निम् । अ॒स्तु॒व॒न्न् । सः । ए॒भ्यः॒ । स्तु॒तः । रात्रि॑याः । अधीति॑ । अहः॑ । अ॒भीति॑ । प॒शून् । निरिति॑ । आ॒र्ज॒त् । ते । दे॒वाः । प॒शून् । वि॒त्त्वा । कामान्॑ । अ॒कु॒र्व॒त॒ । यः । ए॒वम् । वि॒द्वान् । अ॒ग्निम् । उ॒प॒तिष्ठ॑त॒ इत्यु॑प - तिष्ठ॑ते । प॒शु॒मानिति॑ पशु - मान् । भ॒व॒ति॒ ।  \newline


\textbf{Krama Paata} \newline

रात्रि॒म् प्र । प्रावि॑शन्न् । अ॒वि॒श॒न् ते । ते दे॒वाः । दे॒वा ही॒नाः । ही॒ना अ॑मन्यन्त । अ॒म॒न्य॒न्त॒ ते । ते॑ऽपश्यन्न् । अ॒प॒श्य॒न्ना॒ग्ने॒यी । आ॒ग्ने॒यी रात्रिः॑ । रात्रि॑राग्ने॒याः । आ॒ग्ने॒याः प॒शवः॑ । प॒शव॑ इ॒मम् । इ॒ममे॒व । ए॒वाग्निम् । अ॒ग्निꣳ स्त॑वाम । स्त॒वा॒म॒ सः । स नः॑ । नः॒ स्तु॒तः । स्तु॒तः प॒शून् । प॒शून्,पुनः॑ । पुन॑र्,दास्यति । दा॒स्य॒तीति॑ । इति॒ ते । ते᳚ऽग्निम् । अ॒ग्निम॑स्तुवन्न् । अ॒स्तु॒व॒न्थ् सः । स ए᳚भ्यः । ए॒भ्य॒स्तु॒तः । स्तु॒तो रात्रि॑याः । रात्रि॑या॒ अधि॑ । अद्ध्यहः॑ । अह॑र॒भि । अ॒भि प॒शून् । प॒शून्निः । निरा᳚र्जत् । आ॒र्ज॒त् ते । ते दे॒वाः । दे॒वाः प॒शून् । प॒शून्. वि॒त्वा । वि॒त्वा कामान्॑ । कामाꣳ॑ अकुर्वत । अ॒कु॒र्व॒त॒ यः । य ए॒वम् । ए॒वं ॅवि॒द्वान् । वि॒द्वान॒ग्निम् । अ॒ग्निमु॑प॒तिष्ठ॑ते । उ॒प॒तिष्ठ॑ते पशु॒मान् । उ॒प॒तिष्ठ॑त॒ इत्यु॑प - तिष्ठ॑ते । प॒शु॒मान्,भ॑वति । प॒शु॒मानिति॑ पशु - मान् । भ॒व॒त्या॒दि॒त्यः \newline

\textbf{Jatai Paata} \newline

1. रात्रि॒म् प्र प्र रात्रि॒(ग्म्॒) रात्रि॒म् प्र । \newline
2. प्रावि॑शन्नविश॒न् प्र प्रावि॑शन्न् । \newline
3. अ॒वि॒श॒न् ते ते॑ ऽविशन्नविश॒न् ते । \newline
4. ते दे॒वा दे॒वास्ते ते दे॒वाः । \newline
5. दे॒वा ही॒ना ही॒ना दे॒वा दे॒वा ही॒नाः । \newline
6. ही॒ना अ॑मन्यन्तामन्यन्त ही॒ना ही॒ना अ॑मन्यन्त । \newline
7. अ॒म॒न्य॒न्त॒ ते ते॑ ऽमन्यन्तामन्यन्त॒ ते । \newline
8. ते॑ ऽपश्यन्नपश्य॒न् ते ते॑ ऽपश्यन्न् । \newline
9. अ॒प॒श्य॒ न्ना॒ग्ने॒य्या᳚ग्ने॒य्य॑पश्य न्नपश्यन्नाग्ने॒यी । \newline
10. आ॒ग्ने॒यी रात्री॒ रात्रि॑राग्ने॒य्या᳚ग्ने॒यी रात्रिः॑ । \newline
11. रात्रि॑राग्ने॒या आ᳚ग्ने॒या रात्री॒ रात्रि॑राग्ने॒याः । \newline
12. आ॒ग्ने॒याः प॒शवः॑ प॒शव॑ आग्ने॒या आ᳚ग्ने॒याः प॒शवः॑ । \newline
13. प॒शव॑ इ॒म मि॒मम् प॒शवः॑ प॒शव॑ इ॒मम् । \newline
14. इ॒म मे॒वैवे म मि॒म मे॒व । \newline
15. ए॒वाग्नि म॒ग्नि मे॒वैवाग्निम् । \newline
16. अ॒ग्निꣳ स्त॑वाम स्तवामा॒ग्नि म॒ग्निꣳ स्त॑वाम । \newline
17. स्त॒वा॒म॒ स स स्त॑वाम स्तवाम॒ सः । \newline
18. स नो॑ नः॒ स स नः॑ । \newline
19. नः॒ स्तु॒तः स्तु॒तो नो॑ नः स्तु॒तः । \newline
20. स्तु॒तः प॒शून् प॒शून् थ्स्तु॒तः स्तु॒तः प॒शून् । \newline
21. प॒शून् पुनः॒ पुनः॑ प॒शून् प॒शून् पुनः॑ । \newline
22. पुन॑र् दास्यति दास्यति॒ पुनः॒ पुन॑र् दास्यति । \newline
23. दा॒स्य॒तीतीति॑ दास्यति दास्य॒तीति॑ । \newline
24. इति॒ ते त इतीति॒ ते । \newline
25. ते᳚ ऽग्नि म॒ग्निम् ते ते᳚ ऽग्निम् । \newline
26. अ॒ग्नि म॑स्तुव न्नस्तुवन्न॒ग्नि म॒ग्नि म॑स्तुवन्न् । \newline
27. अ॒स्तु॒व॒न् थ्स सो᳚ ऽस्तुवन्नस्तुव॒न् थ्सः । \newline
28. स ए᳚भ्य एभ्यः॒ स स ए᳚भ्यः । \newline
29. ए॒भ्यः॒ स्तु॒तः स्तु॒त ए᳚भ्य एभ्यः स्तु॒तः । \newline
30. स्तु॒तो रात्रि॑या॒ रात्रि॑याः स्तु॒तः स्तु॒तो रात्रि॑याः । \newline
31. रात्रि॑या॒ अध्यधि॒ रात्रि॑या॒ रात्रि॑या॒ अधि॑ । \newline
32. अध्यह॒ रह॒ रध्यध्यहः॑ । \newline
33. अह॑ र॒भ्य॑भ्यह॒ रह॑ र॒भि । \newline
34. अ॒भि प॒शून् प॒शू न॒भ्य॑भि प॒शून् । \newline
35. प॒शून् निर् णिष् प॒शून् प॒शून् निः । \newline
36. निरा᳚र्जदार्ज॒न् निर् णिरा᳚र्जत् । \newline
37. आ॒र्ज॒त् ते त आ᳚र्जदार्ज॒त् ते । \newline
38. ते दे॒वा दे॒वास्ते ते दे॒वाः । \newline
39. दे॒वाः प॒शून् प॒शून् दे॒वा दे॒वाः प॒शून् । \newline
40. प॒शून्. वि॒त्त्वा वि॒त्त्वा प॒शून् प॒शून्. वि॒त्त्वा । \newline
41. वि॒त्त्वा कामा॒न् कामान्॑. वि॒त्त्वा वि॒त्त्वा कामान्॑ । \newline
42. कामा(ग्म्॑) अकुर्वताकुर्वत॒ कामा॒न् कामा(ग्म्॑) अकुर्वत । \newline
43. अ॒कु॒र्व॒त॒ यो यो॑ ऽकुर्वताकुर्वत॒ यः । \newline
44. य ए॒व मे॒वं ॅयो य ए॒वम् । \newline
45. ए॒वं ॅवि॒द्वान्. वि॒द्वा ने॒व मे॒वं ॅवि॒द्वान् । \newline
46. वि॒द्वा न॒ग्नि म॒ग्निं ॅवि॒द्वान्. वि॒द्वा न॒ग्निम् । \newline
47. अ॒ग्नि मु॑प॒तिष्ठ॑त उप॒तिष्ठ॑ते॒ ऽग्नि म॒ग्नि मु॑प॒तिष्ठ॑ते । \newline
48. उ॒प॒तिष्ठ॑ते पशु॒मान् प॑शु॒मा नु॑प॒तिष्ठ॑त उप॒तिष्ठ॑ते पशु॒मान् । \newline
49. उ॒प॒तिष्ठ॑त॒ इत्यु॑प - तिष्ठ॑ते । \newline
50. प॒शु॒मान् भ॑वति भवति पशु॒मान् प॑शु॒मान् भ॑वति । \newline
51. प॒शु॒मानिति॑ पशु - मान् । \newline
52. भ॒व॒त्या॒दि॒त्य आ॑दि॒त्यो भ॑वति भवत्यादि॒त्यः । \newline

\textbf{Ghana Paata } \newline

1. रात्रि॒म् प्र प्र रात्रि॒(ग्म्॒) रात्रि॒म् प्रावि॑शन् नविश॒न् प्र रात्रि॒(ग्म्॒) रात्रि॒म् प्रावि॑शन्न् । \newline
2. प्रावि॑शन् नविश॒न् प्र प्रावि॑श॒न् ते ते॑ ऽविश॒न् प्र प्रावि॑श॒न् ते । \newline
3. अ॒वि॒श॒न् ते ते॑ ऽविशन् नविश॒न् ते दे॒वा दे॒वास्ते॑ ऽविशन् नविश॒न् ते दे॒वाः । \newline
4. ते दे॒वा दे॒वास्ते ते दे॒वा ही॒ना ही॒ना दे॒वास्ते ते दे॒वा ही॒नाः । \newline
5. दे॒वा ही॒ना ही॒ना दे॒वा दे॒वा ही॒ना अ॑मन्यन्तामन्यन्त ही॒ना दे॒वा दे॒वा ही॒ना अ॑मन्यन्त । \newline
6. ही॒ना अ॑मन्यन्तामन्यन्त ही॒ना ही॒ना अ॑मन्यन्त॒ ते ते॑ ऽमन्यन्त ही॒ना ही॒ना अ॑मन्यन्त॒ ते । \newline
7. अ॒म॒न्य॒न्त॒ ते ते॑ ऽमन्यन्तामन्यन्त॒ ते॑ ऽपश्यन् नपश्य॒न् ते॑ ऽमन्यन्तामन्यन्त॒ ते॑ ऽपश्यन्न् । \newline
8. ते॑ ऽपश्यन् नपश्य॒न् ते ते॑ ऽपश्यन् नाग्ने॒य्या᳚ग्ने॒य्य॑पश्य॒न् ते ते॑ ऽपश्यन् नाग्ने॒यी । \newline
9. अ॒प॒श्य॒न् ना॒ग्ने॒य्या᳚ग्ने॒य्य॑पश्यन् नपश्यन् नाग्ने॒यी रात्री॒ रात्रि॑ राग्ने॒य्य॑पश्यन् नपश्यन् नाग्ने॒यी रात्रिः॑ । \newline
10. आ॒ग्ने॒यी रात्री॒ रात्रि॑ राग्ने॒य्या᳚ग्ने॒यी रात्रि॑राग्ने॒या आ᳚ग्ने॒या रात्रि॑ राग्ने॒य्या᳚ग्ने॒यी रात्रि॑राग्ने॒याः । \newline
11. रात्रि॑राग्ने॒या आ᳚ग्ने॒या रात्री॒ रात्रि॑राग्ने॒याः प॒शवः॑ प॒शव॑ आग्ने॒या रात्री॒ रात्रि॑राग्ने॒याः प॒शवः॑ । \newline
12. आ॒ग्ने॒याः प॒शवः॑ प॒शव॑ आग्ने॒या आ᳚ग्ने॒याः प॒शव॑ इ॒म मि॒मम् प॒शव॑ आग्ने॒या आ᳚ग्ने॒याः प॒शव॑ इ॒मम् । \newline
13. प॒शव॑ इ॒म मि॒मम् प॒शवः॑ प॒शव॑ इ॒म मे॒वैवे मम् प॒शवः॑ प॒शव॑ इ॒म मे॒व । \newline
14. इ॒म मे॒वैवे म मि॒म मे॒वाग्नि म॒ग्नि मे॒वे म मि॒म मे॒वाग्निम् । \newline
15. ए॒वाग्नि म॒ग्नि मे॒वैवाग्निꣳ स्त॑वाम स्तवामा॒ग्नि मे॒वैवाग्निꣳ स्त॑वाम । \newline
16. अ॒ग्निꣳ स्त॑वाम स्तवामा॒ग्नि म॒ग्निꣳ स्त॑वाम॒ स स स्त॑वामा॒ग्नि म॒ग्निꣳ स्त॑वाम॒ सः । \newline
17. स्त॒वा॒म॒ स स स्त॑वाम स्तवाम॒ स नो॑ नः॒ स स्त॑वाम स्तवाम॒ स नः॑ । \newline
18. स नो॑ नः॒ स स नः॑ स्तु॒तः स्तु॒तो नः॒ स स नः॑ स्तु॒तः । \newline
19. नः॒ स्तु॒तः स्तु॒तो नो॑ नः स्तु॒तः प॒शून् प॒शून् थ्स्तु॒तो नो॑ नः स्तु॒तः प॒शून् । \newline
20. स्तु॒तः प॒शून् प॒शून् थ्स्तु॒तः स्तु॒तः प॒शून् पुनः॒ पुनः॑ प॒शून् थ्स्तु॒तः स्तु॒तः प॒शून् पुनः॑ । \newline
21. प॒शून् पुनः॒ पुनः॑ प॒शून् प॒शून् पुन॑र् दास्यति दास्यति॒ पुनः॑ प॒शून् प॒शून् पुन॑र् दास्यति । \newline
22. पुन॑र् दास्यति दास्यति॒ पुनः॒ पुन॑र् दास्य॒तीतीति॑ दास्यति॒ पुनः॒ पुन॑र् दास्य॒तीति॑ । \newline
23. दा॒स्य॒तीतीति॑ दास्यति दास्य॒तीति॒ ते त इति॑ दास्यति दास्य॒तीति॒ ते । \newline
24. इति॒ ते त इतीति॒ ते᳚ ऽग्नि म॒ग्निम् त इतीति॒ ते᳚ ऽग्निम् । \newline
25. ते᳚ ऽग्नि म॒ग्निम् ते ते᳚ ऽग्नि म॑स्तुवन् नस्तुवन् न॒ग्निम् ते ते᳚ ऽग्नि म॑स्तुवन्न् । \newline
26. अ॒ग्नि म॑स्तुवन् नस्तुवन् न॒ग्नि म॒ग्नि म॑स्तुव॒न् थ्स सो᳚ ऽस्तुवन् न॒ग्नि म॒ग्नि म॑स्तुव॒न् थ्सः । \newline
27. अ॒स्तु॒व॒न् थ्स सो᳚ ऽस्तुवन् नस्तुव॒न् थ्स ए᳚भ्य एभ्यः सो᳚ ऽस्तुवन् नस्तुव॒न् थ्स ए᳚भ्यः । \newline
28. स ए᳚भ्य एभ्यः॒ स स ए᳚भ्यः स्तु॒तः स्तु॒त ए᳚भ्यः॒ स स ए᳚भ्यः स्तु॒तः । \newline
29. ए॒भ्यः॒ स्तु॒तः स्तु॒त ए᳚भ्य एभ्यः स्तु॒तो रात्रि॑या॒ रात्रि॑याः स्तु॒त ए᳚भ्य एभ्यः स्तु॒तो रात्रि॑याः । \newline
30. स्तु॒तो रात्रि॑या॒ रात्रि॑याः स्तु॒तः स्तु॒तो रात्रि॑या॒ अध्यधि॒ रात्रि॑याः स्तु॒तः स्तु॒तो रात्रि॑या॒ अधि॑ । \newline
31. रात्रि॑या॒ अध्यधि॒ रात्रि॑या॒ रात्रि॑या॒ अध्यह॒ रह॒ रधि॒ रात्रि॑या॒ रात्रि॑या॒ अध्यहः॑ । \newline
32. अध्यह॒ रह॒ रध्यध्यह॑ र॒भ्य॑भ्यह॒ रध्यध्यह॑ र॒भि । \newline
33. अह॑ र॒भ्य॑भ्यह॒ रह॑ र॒भि प॒शून् प॒शू न॒भ्यह॒ रह॑ र॒भि प॒शून् । \newline
34. अ॒भि प॒शून् प॒शू न॒भ्य॑भि प॒शून् निर् णिष् प॒शू न॒भ्य॑भि प॒शून् निः । \newline
35. प॒शून् निर् णिष् प॒शून् प॒शून् निरा᳚र्जदार्ज॒न् निष् प॒शून् प॒शून् निरा᳚र्जत् । \newline
36. निरा᳚र्जदार्ज॒न् निर् णिरा᳚र्ज॒त् ते त आ᳚र्ज॒न् निर् णिरा᳚र्ज॒त् ते । \newline
37. आ॒र्ज॒त् ते त आ᳚र्जदार्ज॒त् ते दे॒वा दे॒वास्त आ᳚र्जदार्ज॒त् ते दे॒वाः । \newline
38. ते दे॒वा दे॒वास्ते ते दे॒वाः प॒शून् प॒शून् दे॒वास्ते ते दे॒वाः प॒शून् । \newline
39. दे॒वाः प॒शून् प॒शून् दे॒वा दे॒वाः प॒शून्. वि॒त्त्वा वि॒त्त्वा प॒शून् दे॒वा दे॒वाः प॒शून्. वि॒त्त्वा । \newline
40. प॒शून्. वि॒त्त्वा वि॒त्त्वा प॒शून् प॒शून्. वि॒त्त्वा कामा॒न् कामान्॑. वि॒त्त्वा प॒शून् प॒शून्. वि॒त्त्वा कामान्॑ । \newline
41. वि॒त्त्वा कामा॒न् कामान्॑. वि॒त्त्वा वि॒त्त्वा कामा(ग्म्॑) अकुर्वताकुर्वत॒ कामान्॑. वि॒त्त्वा वि॒त्त्वा कामा(ग्म्॑) अकुर्वत । \newline
42. कामा(ग्म्॑) अकुर्वताकुर्वत॒ कामा॒न् कामा(ग्म्॑) अकुर्वत॒ यो यो॑ ऽकुर्वत॒ कामा॒न् कामा(ग्म्॑) अकुर्वत॒ यः । \newline
43. अ॒कु॒र्व॒त॒ यो यो॑ ऽकुर्वताकुर्वत॒ य ए॒व मे॒वं ॅयो॑ ऽकुर्वताकुर्वत॒ य ए॒वम् । \newline
44. य ए॒व मे॒वं ॅयो य ए॒वं ॅवि॒द्वान्. वि॒द्वा ने॒वं ॅयो य ए॒वं ॅवि॒द्वान् । \newline
45. ए॒वं ॅवि॒द्वान्. वि॒द्वा ने॒व मे॒वं ॅवि॒द्वा न॒ग्नि म॒ग्निं ॅवि॒द्वा ने॒व मे॒वं ॅवि॒द्वा न॒ग्निम् । \newline
46. वि॒द्वा न॒ग्नि म॒ग्निं ॅवि॒द्वान्. वि॒द्वा न॒ग्नि मु॑प॒तिष्ठ॑त उप॒तिष्ठ॑ते॒ ऽग्निं ॅवि॒द्वान्. वि॒द्वा न॒ग्नि मु॑प॒तिष्ठ॑ते । \newline
47. अ॒ग्नि मु॑प॒तिष्ठ॑त उप॒तिष्ठ॑ते॒ ऽग्नि म॒ग्नि मु॑प॒तिष्ठ॑ते पशु॒मान् प॑शु॒मा नु॑प॒तिष्ठ॑ते॒ ऽग्नि म॒ग्नि मु॑प॒तिष्ठ॑ते पशु॒मान् । \newline
48. उ॒प॒तिष्ठ॑ते पशु॒मान् प॑शु॒मा नु॑प॒तिष्ठ॑त उप॒तिष्ठ॑ते पशु॒मान् भ॑वति भवति पशु॒मा नु॑प॒तिष्ठ॑त उप॒तिष्ठ॑ते पशु॒मान् भ॑वति । \newline
49. उ॒प॒तिष्ठ॑त॒ इत्यु॑प - तिष्ठ॑ते । \newline
50. प॒शु॒मान् भ॑वति भवति पशु॒मान् प॑शु॒मान् भ॑वत्यादि॒त्य आ॑दि॒त्यो भ॑वति पशु॒मान् प॑शु॒मान् भ॑वत्यादि॒त्यः । \newline
51. प॒शु॒मानिति॑ पशु - मान् । \newline
52. भ॒व॒त्या॒दि॒त्य आ॑दि॒त्यो भ॑वति भवत्यादि॒त्यो वै वा आ॑दि॒त्यो भ॑वति भवत्यादि॒त्यो वै । \newline
\pagebreak
\markright{ TS 1.5.9.4  \hfill https://www.vedavms.in \hfill}
\addcontentsline{toc}{section}{ TS 1.5.9.4 }
\section*{ TS 1.5.9.4 }

\textbf{TS 1.5.9.4 } \newline
\textbf{Samhita Paata} \newline

त्यादि॒त्यो वा अ॒स्माल्लो॒काद॒मुं ॅलो॒कमै॒थ्सो॑ऽमुं ॅलो॒कं ग॒त्वा पुन॑रि॒मं ॅलो॒कम॒भ्य॑द्ध्याय॒थ् स इ॒मं ॅलो॒कमा॒गत्य॑ मृ॒त्योर॑बिभेन्मृ॒त्युसं॑ॅयुत इव॒ ह्य॑यं ॅलो॒कः सो॑ऽमन्यते॒म-मे॒वाग्निꣳ स्त॑वानि॒ स मा᳚ स्तु॒तः सु॑व॒र्गं ॅलो॒कं ग॑मयिष्य॒तीति॒ सो᳚ऽग्निम॑स्तौ॒थ् स ए॑नꣳ स्तु॒तः सु॑व॒र्गं ॅलो॒कम॑गमय॒द्य - [ ] \newline

\textbf{Pada Paata} \newline

आ॒दि॒त्यः । वै । अ॒स्मात् । लो॒कात् । अ॒मुम् । लो॒कम् । ऐ॒त् । सः । अ॒मुम् । लो॒कम् । ग॒त्वा । पुनः॑ । इ॒मम् । लो॒कम् । अ॒भीति॑ । अ॒द्ध्या॒य॒त् । सः । इ॒मम् । लो॒कम् । आ॒गत्येत्या᳚ - गत्य॑ । मृ॒त्योः । अ॒बि॒भे॒त् । मृ॒त्युस॑म्ॅयुत॒ इति॑ मृ॒त्यु-स॒म्ॅयु॒तः॒ । इ॒व॒ । हि । अ॒यम् । लो॒कः । सः । अ॒म॒न्य॒त॒ । इ॒मम् । ए॒व । अ॒ग्निम् । स्त॒वा॒नि॒ । सः । मा॒ । स्तु॒तः । सु॒व॒र्गमिति॑ सुवः-गम् । लो॒कम् । ग॒म॒यि॒ष्य॒ति॒ । इति॑ । सः । अ॒ग्निम् । अ॒स्तौ॒त् । सः । ए॒न॒म् । स्तु॒तः । सु॒व॒र्गमिति॑ सुवः - गम् । लो॒कम् । अ॒ग॒म॒य॒त् । यः ।  \newline


\textbf{Krama Paata} \newline

आ॒दि॒त्यो वै । वा अ॒स्मात् । अ॒स्माल्लो॒कात् । लो॒काद॒मुम् । अ॒मुं ॅलो॒कम् । लो॒कमै᳚त् । ऐ॒थ् सः । सो॑ऽमुम् । अ॒मुं ॅलो॒कम् । लो॒कम् ग॒त्वा । ग॒त्वा पुनः॑ । पुन॑रि॒मम् । इ॒मं ॅलो॒कम् । लो॒कम॒भि । अ॒भ्य॑द्ध्यायत् । अ॒द्ध्या॒य॒थ् सः । स इ॒मम् । इ॒मं ॅलो॒कम् । लो॒कमा॒गत्य॑ । आ॒गत्य॑ मृ॒त्योः । आ॒गत्येत्या᳚ - गत्य॑ । मृ॒त्योर॑बिभेत् । अ॒बि॒भे॒न्,मृ॒त्युस॑म्ॅयुतः । मृ॒त्युस॑म्ॅयुत इव । मृ॒त्युस॑म्ॅयुत॒ इति॑ मृ॒त्यु - स॒म्ॅयु॒तः॒ । इ॒व॒ हि । ह्य॑यम् । अ॒यं ॅलो॒कः । लो॒कः सः । सो॑ऽमन्यत । अ॒म॒न्य॒ते॒मम् । इ॒ममे॒व । ए॒वाग्निम् । अ॒ग्निꣳ स्त॑वानि । स्त॒वा॒नि॒ सः । स मा᳚ । मा॒ स्तु॒तः । स्तु॒तः सु॑व॒र्गम् । सु॒व॒र्गं ॅलो॒कम् । सु॒व॒र्गमिति॑ सुवः - गम् । लो॒कम् ग॑मयिष्यति । ग॒म॒यि॒ष्य॒तीति॑ । इति॒ सः । सो᳚ऽग्निम् । अ॒ग्निम॑स्तौत् । अ॒स्थौ॒थ् सः । स ए॑नम् । ए॒नꣳ॒॒ स्तु॒तः । स्तु॒तः सु॑व॒र्गम् । सु॒व॒र्गं ॅलो॒कम् । सु॒व॒र्गमिति॑ सुवः - गम् । लो॒कम॑गमयत् । अ॒ग॒म॒य॒द् यः । य ए॒वम् \newline

\textbf{Jatai Paata} \newline

1. आ॒दि॒त्यो वै वा आ॑दि॒त्य आ॑दि॒त्यो वै । \newline
2. वा अ॒स्माद॒स्माद् वै वा अ॒स्मात् । \newline
3. अ॒स्मा ल्लो॒का ल्लो॒का द॒स्मा द॒स्मा ल्लो॒कात् । \newline
4. लो॒काद॒मु म॒मुम् ॅलो॒का ल्लो॒काद॒मुम् । \newline
5. अ॒मुम् ॅलो॒कम् ॅलो॒क म॒मु म॒मुम् ॅलो॒कम् । \newline
6. लो॒क मै॑दैल्लो॒कम् ॅलो॒क मै᳚त् । \newline
7. ऐ॒थ् स स ऐ॑दै॒थ् सः । \newline
8. सो॑ ऽमु म॒मुꣳ स सो॑ ऽमुम् । \newline
9. अ॒मुम् ॅलो॒कम् ॅलो॒क म॒मु म॒मुम् ॅलो॒कम् । \newline
10. लो॒कम् ग॒त्वा ग॒त्वा लो॒कम् ॅलो॒कम् ग॒त्वा । \newline
11. ग॒त्वा पुनः॒ पुन॑र् ग॒त्वा ग॒त्वा पुनः॑ । \newline
12. पुन॑ रि॒म मि॒मम् पुनः॒ पुन॑ रि॒मम् । \newline
13. इ॒मम् ॅलो॒कम् ॅलो॒क मि॒म मि॒मम् ॅलो॒कम् । \newline
14. लो॒क म॒भ्य॑भि लो॒कम् ॅलो॒क म॒भि । \newline
15. अ॒भ्य॑द्ध्याय दद्ध्यायद॒भ्या᳚(1॒)भ्य॑द्ध्यायत् । \newline
16. अ॒द्ध्या॒य॒थ् स सो᳚ ऽद्ध्याय दद्ध्याय॒थ् सः । \newline
17. स इ॒म मि॒मꣳ स स इ॒मम् । \newline
18. इ॒मम् ॅलो॒कम् ॅलो॒क मि॒म मि॒मम् ॅलो॒कम् । \newline
19. लो॒क मा॒गत्या॒गत्य॑ लो॒कम् ॅलो॒क मा॒गत्य॑ । \newline
20. आ॒गत्य॑ मृ॒त्योर् मृ॒त्योरा॒गत्या॒गत्य॑ मृ॒त्योः । \newline
21. आ॒गत्येत्या᳚ - गत्य॑ । \newline
22. मृ॒त्यो र॑बिभेदबिभेन् मृ॒त्योर् मृ॒त्योर॑बिभेत् । \newline
23. अ॒बि॒भे॒न् मृ॒त्युसं॑ॅयुतो मृ॒त्युसं॑ॅयुतो ऽबिभेदबिभेन् मृ॒त्युसं॑ॅयुतः । \newline
24. मृ॒त्युसं॑ॅयुत इवे व मृ॒त्युसं॑ॅयुतो मृ॒त्युसं॑ॅयुत इव । \newline
25. मृ॒त्युस॑म्ॅयुत॒ इति॑ मृ॒त्यु - स॒म्ॅयु॒तः॒ । \newline
26. इ॒व॒ हि हीवे॑ व॒ हि । \newline
27. ह्य॑य म॒यꣳ हि ह्य॑यम् । \newline
28. अ॒यम् ॅलो॒को लो॒को॑ ऽय म॒यम् ॅलो॒कः । \newline
29. लो॒कः स स लो॒को लो॒कः सः । \newline
30. सो॑ ऽमन्यतामन्यत॒ स सो॑ ऽमन्यत । \newline
31. अ॒म॒न्य॒ते॒ म मि॒म म॑मन्यतामन्यते॒ मम् । \newline
32. इ॒म मे॒वैवे म मि॒म मे॒व । \newline
33. ए॒वाग्नि म॒ग्नि मे॒वैवाग्निम् । \newline
34. अ॒ग्निꣳ स्त॑वानि स्तवान्य॒ग्नि म॒ग्निꣳ स्त॑वानि । \newline
35. स्त॒वा॒नि॒ स स स्त॑वानि स्तवानि॒ सः । \newline
36. स मा॑ मा॒ स स मा᳚ । \newline
37. मा॒ स्तु॒तः स्तु॒तो मा॑ मा स्तु॒तः । \newline
38. स्तु॒तः सु॑व॒र्गꣳ सु॑व॒र्गꣳ स्तु॒तः स्तु॒तः सु॑व॒र्गम् । \newline
39. सु॒व॒र्गम् ॅलो॒कम् ॅलो॒कꣳ सु॑व॒र्गꣳ सु॑व॒र्गम् ॅलो॒कम् । \newline
40. सु॒व॒र्गमिति॑ सुवः - गम् । \newline
41. लो॒कम् ग॑मयिष्यति गमयिष्यति लो॒कम् ॅलो॒कम् ग॑मयिष्यति । \newline
42. ग॒म॒यि॒ष्य॒तीतीति॑ गमयिष्यति गमयिष्य॒तीति॑ । \newline
43. इति॒ स स इतीति॒ सः । \newline
44. सो᳚ ऽग्नि म॒ग्निꣳ स सो᳚ ऽग्निम् । \newline
45. अ॒ग्नि म॑स्तौ दस्तौद॒ग्नि म॒ग्नि म॑स्तौत् । \newline
46. अ॒स्तौ॒थ् स सो᳚ ऽस्तौदस्तौ॒थ् सः । \newline
47. स ए॑न मेन॒(ग्म्॒) स स ए॑नम् । \newline
48. ए॒न॒(ग्ग्॒) स्तु॒तः स्तु॒त ए॑न मेनꣳ स्तु॒तः । \newline
49. स्तु॒तः सु॑व॒र्गꣳ सु॑व॒र्गꣳ स्तु॒तः स्तु॒तः सु॑व॒र्गम् । \newline
50. सु॒व॒र्गम् ॅलो॒कम् ॅलो॒कꣳ सु॑व॒र्गꣳ सु॑व॒र्गम् ॅलो॒कम् । \newline
51. सु॒व॒र्गमिति॑ सुवः - गम् । \newline
52. लो॒क म॑गमय दगमय ल्लो॒कम् ॅलो॒क म॑गमयत् । \newline
53. अ॒ग॒म॒य॒द् यो यो॑ ऽगमयदगमय॒द् यः । \newline
54. य ए॒व मे॒वं ॅयो य ए॒वम् । \newline

\textbf{Ghana Paata } \newline

1. आ॒दि॒त्यो वै वा आ॑दि॒त्य आ॑दि॒त्यो वा अ॒स्माद॒स्माद् वा आ॑दि॒त्य आ॑दि॒त्यो वा अ॒स्मात् । \newline
2. वा अ॒स्माद॒स्माद् वै वा अ॒स्मा ल्लो॒का ल्लो॒का द॒स्माद् वै वा अ॒स्माल्लो॒कात् । \newline
3. अ॒स्मा ल्लो॒का ल्लो॒का द॒स्मा द॒स्मा ल्लो॒काद॒मु म॒मुम् ॅलो॒का द॒स्मा द॒स्मा ल्लो॒काद॒मुम् । \newline
4. लो॒काद॒मु म॒मुम् ॅलो॒का ल्लो॒काद॒मुम् ॅलो॒कम् ॅलो॒क म॒मुम् ॅलो॒काल्लो॒काद॒मुम् ॅलो॒कम् । \newline
5. अ॒मुम् ॅलो॒कम् ॅलो॒क म॒मु म॒मुम् ॅलो॒क मै॑दैल्लो॒क म॒मु म॒मुम् ॅलो॒क मै᳚त् । \newline
6. लो॒क मै॑दैल्लो॒कम् ॅलो॒क मै॒थ् स स ऐ᳚ल्लो॒कम् ॅलो॒क मै॒थ् सः । \newline
7. ऐ॒थ् स स ऐ॑दै॒थ् सो॑ ऽमु म॒मुꣳ स ऐ॑दै॒थ् सो॑ ऽमुम् । \newline
8. सो॑ ऽमु म॒मुꣳ स सो॑ ऽमुम् ॅलो॒कम् ॅलो॒क म॒मुꣳ स सो॑ ऽमुम् ॅलो॒कम् । \newline
9. अ॒मुम् ॅलो॒कम् ॅलो॒क म॒मु म॒मुम् ॅलो॒कम् ग॒त्वा ग॒त्वा लो॒क म॒मु म॒मुम् ॅलो॒कम् ग॒त्वा । \newline
10. लो॒कम् ग॒त्वा ग॒त्वा लो॒कम् ॅलो॒कम् ग॒त्वा पुनः॒ पुन॑र् ग॒त्वा लो॒कम् ॅलो॒कम् ग॒त्वा पुनः॑ । \newline
11. ग॒त्वा पुनः॒ पुन॑र् ग॒त्वा ग॒त्वा पुन॑ रि॒म मि॒मम् पुन॑र् ग॒त्वा ग॒त्वा पुन॑ रि॒मम् । \newline
12. पुन॑ रि॒म मि॒मम् पुनः॒ पुन॑ रि॒मम् ॅलो॒कम् ॅलो॒क मि॒मम् पुनः॒ पुन॑ रि॒मम् ॅलो॒कम् । \newline
13. इ॒मम् ॅलो॒कम् ॅलो॒क मि॒म मि॒मम् ॅलो॒क म॒भ्य॑भि लो॒क मि॒म मि॒मम् ॅलो॒क म॒भि । \newline
14. लो॒क म॒भ्य॑भि लो॒कम् ॅलो॒क म॒भ्य॑द्ध्याय दद्ध्यायद॒भि लो॒कम् ॅलो॒क म॒भ्य॑द्ध्यायत् । \newline
15. अ॒भ्य॑द्ध्या यदद्ध्यायद॒भ्या᳚(1॒)भ्य॑द्ध्याय॒थ् स सो᳚ ऽद्ध्यायद॒भ्या᳚(1॒)भ्य॑द्ध्याय॒थ् सः । \newline
16. अ॒द्ध्या॒य॒थ् स सो᳚ ऽद्ध्यायदद्ध्याय॒थ् स इ॒म मि॒मम्सो᳚ ऽद्ध्यायदद्ध्याय॒थ् स इ॒मम् । \newline
17. स इ॒म मि॒मꣳ स स इ॒मम् ॅलो॒कम् ॅलो॒क मि॒मꣳ स स इ॒मम् ॅलो॒कम् । \newline
18. इ॒मम् ॅलो॒कम् ॅलो॒क मि॒म मि॒मम् ॅलो॒क मा॒गत्या॒गत्य॑ लो॒क मि॒म मि॒मम् ॅलो॒क मा॒गत्य॑ । \newline
19. लो॒क मा॒गत्या॒गत्य॑ लो॒कम् ॅलो॒क मा॒गत्य॑ मृ॒त्योर् मृ॒त्योरा॒गत्य॑ लो॒कम् ॅलो॒क मा॒गत्य॑ मृ॒त्योः । \newline
20. आ॒गत्य॑ मृ॒त्योर् मृ॒त्योरा॒गत्या॒गत्य॑ मृ॒त्यो र॑बिभे दबिभेन् मृ॒त्योरा॒गत्या॒गत्य॑ मृ॒त्यो र॑बिभेत् । \newline
21. आ॒गत्येत्या᳚ - गत्य॑ । \newline
22. मृ॒त्यो र॑बिभे दबिभेन् मृ॒त्योर् मृ॒त्यो र॑बिभेन् मृ॒त्युसं॑ॅयुतो मृ॒त्युसं॑ॅयुतो ऽबिभेन् मृ॒त्योर् मृ॒त्यो र॑बिभेन् मृ॒त्युसं॑ॅयुतः । \newline
23. अ॒बि॒भे॒न् मृ॒त्युसं॑ॅयुतो मृ॒त्युसं॑ॅयुतो ऽबिभे दबिभेन् मृ॒त्युसं॑ॅयुत इवे व मृ॒त्युसं॑ॅयुतो ऽबिभे दबिभेन् मृ॒त्युसं॑ॅयुत इव । \newline
24. मृ॒त्युसं॑ॅयुत इवे व मृ॒त्युसं॑ॅयुतो मृ॒त्युसं॑ॅयुत इव॒ हि हीव॑ मृ॒त्युसं॑ॅयुतो मृ॒त्युसं॑ॅयुत इव॒ हि । \newline
25. मृ॒त्युस॑म्ॅयुत॒ इति॑ मृ॒त्यु - स॒म्ॅयु॒तः॒ । \newline
26. इ॒व॒ हि हीवे॑ व॒ ह्य॑य म॒यꣳ हीवे॑ व॒ ह्य॑यम् । \newline
27. ह्य॑य म॒यꣳ हि ह्य॑यम् ॅलो॒को लो॒को॑ ऽयꣳ हि ह्य॑यम् ॅलो॒कः । \newline
28. अ॒यम् ॅलो॒को लो॒को॑ ऽय म॒यम् ॅलो॒कः स स लो॒को॑ ऽय म॒यम् ॅलो॒कः सः । \newline
29. लो॒कः स स लो॒को लो॒कः सो॑ ऽमन्यतामन्यत॒ स लो॒को लो॒कः सो॑ ऽमन्यत । \newline
30. सो॑ ऽमन्यतामन्यत॒ स सो॑ ऽमन्यते॒ म मि॒म म॑मन्यत॒ स सो॑ ऽमन्यते॒ मम् । \newline
31. अ॒म॒न्य॒ते॒ म मि॒म म॑मन्यतामन्यते॒ म मे॒वैवे म म॑मन्यतामन्यते॒ म मे॒व । \newline
32. इ॒म मे॒वैवे म मि॒म मे॒वाग्नि म॒ग्नि मे॒वे म मि॒म मे॒वाग्निम् । \newline
33. ए॒वाग्नि म॒ग्नि मे॒वैवाग्निꣳ स्त॑वानि स्तवान्य॒ग्नि मे॒वैवाग्निꣳ स्त॑वानि । \newline
34. अ॒ग्निꣳ स्त॑वानि स्तवान्य॒ग्नि म॒ग्निꣳ स्त॑वानि॒ स स स्त॑वान्य॒ग्नि म॒ग्निꣳ स्त॑वानि॒ सः । \newline
35. स्त॒वा॒नि॒ स स स्त॑वानि स्तवानि॒ स मा॑ मा॒ स स्त॑वानि स्तवानि॒ स मा᳚ । \newline
36. स मा॑ मा॒ स स मा᳚ स्तु॒तः स्तु॒तो मा॒ स स मा᳚ स्तु॒तः । \newline
37. मा॒ स्तु॒तः स्तु॒तो मा॑ मा स्तु॒तः सु॑व॒र्गꣳ सु॑व॒र्गꣳ स्तु॒तो मा॑ मा स्तु॒तः सु॑व॒र्गम् । \newline
38. स्तु॒तः सु॑व॒र्गꣳ सु॑व॒र्गꣳ स्तु॒तः स्तु॒तः सु॑व॒र्गम् ॅलो॒कम् ॅलो॒कꣳ सु॑व॒र्गꣳ स्तु॒तः स्तु॒तः सु॑व॒र्गम् ॅलो॒कम् । \newline
39. सु॒व॒र्गम् ॅलो॒कम् ॅलो॒कꣳ सु॑व॒र्गꣳ सु॑व॒र्गम् ॅलो॒कम् ग॑मयिष्यति गमयिष्यति लो॒कꣳ सु॑व॒र्गꣳ सु॑व॒र्गम् ॅलो॒कम् ग॑मयिष्यति । \newline
40. सु॒व॒र्गमिति॑ सुवः - गम् । \newline
41. लो॒कम् ग॑मयिष्यति गमयिष्यति लो॒कम् ॅलो॒कम् ग॑मयिष्य॒तीतीति॑ गमयिष्यति लो॒कम् ॅलो॒कम् ग॑मयिष्य॒तीति॑ । \newline
42. ग॒म॒यि॒ष्य॒तीतीति॑ गमयिष्यति गमयिष्य॒तीति॒ स स इति॑ गमयिष्यति गमयिष्य॒तीति॒ सः । \newline
43. इति॒ स स इतीति॒सो᳚ ऽग्नि म॒ग्निꣳ स इतीति॒सो᳚ ऽग्निम् । \newline
44. सो᳚ ऽग्नि म॒ग्निꣳ स सो᳚ ऽग्नि म॑स्तौ दस्तौ द॒ग्निꣳ स सो᳚ ऽग्नि म॑स्तौत् । \newline
45. अ॒ग्नि म॑स्तौ दस्तौ द॒ग्नि म॒ग्नि म॑स्तौ॒थ् स सो᳚ ऽस्तौद॒ग्नि म॒ग्नि म॑स्तौ॒थ् सः । \newline
46. अ॒स्तौ॒थ् स सो᳚ ऽस्तौदस्तौ॒थ् स ए॑न मेनꣳ सो᳚ ऽस्तौदस्तौ॒थ् स ए॑नम् । \newline
47. स ए॑न मेन॒(ग्म्॒) स स ए॑नꣳ स्तु॒तः स्तु॒त ए॑न॒(ग्म्॒) स स ए॑नꣳ स्तु॒तः । \newline
48. ए॒न॒(ग्ग्॒) स्तु॒तः स्तु॒त ए॑न मेनꣳ स्तु॒तः सु॑व॒र्गꣳ सु॑व॒र्गꣳ स्तु॒त ए॑न मेनꣳ स्तु॒तः सु॑व॒र्गम् । \newline
49. स्तु॒तः सु॑व॒र्गꣳ सु॑व॒र्गꣳ स्तु॒तः स्तु॒तः सु॑व॒र्गम् ॅलो॒कम् ॅलो॒कꣳ सु॑व॒र्गꣳ स्तु॒तः स्तु॒तः सु॑व॒र्गम् ॅलो॒कम् । \newline
50. सु॒व॒र्गम् ॅलो॒कम् ॅलो॒कꣳ सु॑व॒र्गꣳ सु॑व॒र्गम् ॅलो॒क म॑गमय दगमय ल्लो॒कꣳ सु॑व॒र्गꣳ सु॑व॒र्गम् ॅलो॒क म॑गमयत् । \newline
51. सु॒व॒र्गमिति॑ सुवः - गम् । \newline
52. लो॒क म॑गमय दगमय ल्लो॒कम् ॅलो॒क म॑गमय॒द् यो यो॑ ऽगमयल्लो॒कम् ॅलो॒क म॑गमय॒द् यः । \newline
53. अ॒ग॒म॒य॒द् यो यो॑ ऽगमयदगमय॒द् य ए॒व मे॒वं ॅयो॑ ऽगमयदगमय॒द् य ए॒वम् । \newline
54. य ए॒व मे॒वं ॅयो य ए॒वं ॅवि॒द्वान्. वि॒द्वा ने॒वं ॅयो य ए॒वं ॅवि॒द्वान् । \newline
\pagebreak
\markright{ TS 1.5.9.5  \hfill https://www.vedavms.in \hfill}
\addcontentsline{toc}{section}{ TS 1.5.9.5 }
\section*{ TS 1.5.9.5 }

\textbf{TS 1.5.9.5 } \newline
\textbf{Samhita Paata} \newline

ए॒वं ॅवि॒द्वान॒ग्निमु॑प॒तिष्ठ॑ते सुव॒र्गमे॒व लो॒कमे॑ति॒ सर्व॒मायु॑रेत्य॒भि वा ए॒षो᳚ऽग्नी आ रो॑हति॒ य ए॑नावुप॒तिष्ठ॑ते॒ यथा॒ खलु॒ वै श्रेया॑न॒भ्यारू॑ढः का॒मय॑ते॒ तथा॑ करोति॒ नक्त॒मुप॑ तिष्ठते॒ न प्रा॒तः सꣳ हि नक्तं॑ ॅव्र॒तानि॑ सृ॒ज्यन्ते॑ स॒ह श्रेयाꣳ॑श्च॒ पापी॑याꣳश्चासाते॒ ज्योति॒र्वा अ॒ग्निस्तमो॒ रात्रि॒र्य - [ ] \newline

\textbf{Pada Paata} \newline

ए॒वम् । वि॒द्वान् । अ॒ग्निम् । उ॒प॒तिष्ठ॑त॒ इत्यु॑प - तिष्ठ॑ते । सु॒व॒र्गमिति॑ सुवः - गम् । ए॒व । लो॒कम् । ए॒ति॒ । सर्व᳚म् । आयुः॑ । ए॒ति॒ । अ॒भीति॑ । वै । ए॒षः । अ॒ग्नी इति॑ । एति॑ । रो॒ह॒ति॒ । यः । ए॒नौ॒ । उ॒प॒तिष्ठ॑त॒ इत्यु॑प - तिष्ठ॑ते । यथा᳚ । खलु॑ । वै । श्रेयान्॑ । अ॒भ्यारू॑ढ॒ इत्य॑भि - आरू॑ढः । का॒मय॑ते । तथा᳚ । क॒रो॒ति॒ । नक्त᳚म् । उपेति॑ । ति॒ष्ठ॒ते॒ । न । प्रा॒तः । समिति॑ । हि । नक्त᳚म् । व्र॒तानि॑ । सृ॒ज्यन्ते᳚ । स॒ह । श्रेयान्॑ । च॒ । पापी॑यान् । च॒ । आ॒सा॒ते॒ इति॑ । ज्योतिः॑ । वै । अ॒ग्निः । तमः॑ । रात्रिः॑ । यत् ।  \newline


\textbf{Krama Paata} \newline

ए॒वं ॅवि॒द्वान् । वि॒द्वान॒ग्निम् । अ॒ग्निमु॑प॒तिष्ठ॑ते । उ॒प॒तिष्ठ॑ते सुव॒र्गम् । उ॒प॒तिष्ठ॑त॒ इत्यु॑प - तिष्ठ॑ते । सु॒व॒र्गमे॒व । सु॒व॒र्गमिति॑ सुवः - गम् । ए॒व लो॒कम् । लो॒कमे॑ति । ए॒ति॒ सर्व᳚म् । सर्व॒मायुः॑ । आयु॑रेति । ए॒त्य॒भि । अ॒भि वै । वा ए॒षः । ए॒षो᳚ऽग्नी । अ॒ग्नी आ । अ॒ग्नी इत्य॒ग्नी । आ रो॑हति । रो॒ह॒ति॒ यः । य ए॑नौ । ए॒ना॒वु॒प॒तिष्ठ॑ते । उ॒प॒तिष्ठ॑ते॒ यथा᳚ । उ॒प॒तिष्ठ॑त॒ इत्यु॑प - तिष्ठ॑ते । यथा॒ खलु॑ । खलु॒ वै । वै श्रेयान्॑ । श्रेया॑न॒भ्यारू॑ढः । अ॒भ्यारू॑ढः का॒मय॑ते । अ॒भ्यारू॑ढ॒ इत्य॑भि - आरू॑ढः । का॒मय॑ते॒ तथा᳚ । तथा॑ करोति । क॒रो॒ति॒ नक्त᳚म् । नक्त॒मुप॑ । उप॑ तिष्ठते । ति॒ष्ठ॒ते॒ न । न प्रा॒तः । प्रा॒तः सम् । सꣳ हि । हि नक्त᳚म् । नक्तं॑ ॅव्र॒तानि॑ । व्र॒तानि॑ सृ॒ज्यन्ते᳚ । सृ॒ज्यन्ते॑ स॒ह । स॒ह श्रेयान्॑ । श्रेयाꣳ॑श्च । च॒ पापी॑यान् । पापी॑याꣳश्च । चा॒सा॒ते॒ । आ॒सा॒ते॒ ज्योतिः॑ । आ॒सा॒ते॒ इत्या॑साते । ज्योति॒र् वै । वा अ॒ग्निः । अ॒ग्निस्तमः॑ । तमो॒ रात्रिः॑ । रात्रि॒र्,यत् । यन्नक्त᳚म् \newline

\textbf{Jatai Paata} \newline

1. ए॒वं ॅवि॒द्वान्. वि॒द्वा ने॒व मे॒वं ॅवि॒द्वान् । \newline
2. वि॒द्वा न॒ग्नि म॒ग्निं ॅवि॒द्वान्. वि॒द्वा न॒ग्निम् । \newline
3. अ॒ग्नि मु॑प॒तिष्ठ॑त उप॒तिष्ठ॑ते॒ ऽग्नि म॒ग्नि मु॑प॒तिष्ठ॑ते । \newline
4. उ॒प॒तिष्ठ॑ते सुव॒र्गꣳ सु॑व॒र्ग मु॑प॒तिष्ठ॑त उप॒तिष्ठ॑ते सुव॒र्गम् । \newline
5. उ॒प॒तिष्ठ॑त॒ इत्यु॑प - तिष्ठ॑ते । \newline
6. सु॒व॒र्ग मे॒वैव सु॑व॒र्गꣳ सु॑व॒र्ग मे॒व । \newline
7. सु॒व॒र्गमिति॑ सुवः - गम् । \newline
8. ए॒व लो॒कम् ॅलो॒क मे॒वैव लो॒कम् । \newline
9. लो॒क मे᳚त्येति लो॒कम् ॅलो॒क मे॑ति । \newline
10. ए॒ति॒ सर्व॒(ग्म्॒) सर्व॑ मेत्येति॒ सर्व᳚म् । \newline
11. सर्व॒ मायु॒रायुः॒ सर्व॒(ग्म्॒) सर्व॒ मायुः॑ । \newline
12. आयु॑रेत्ये॒त्यायु॒रायु॑रेति । \newline
13. ए॒त्य॒भ्या᳚(1॒)भ्ये᳚त्येत्य॒भि । \newline
14. अ॒भि वै वा अ॒भ्य॑भि वै । \newline
15. वा ए॒ष ए॒ष वै वा ए॒षः । \newline
16. ए॒षो᳚ऽग्नी अ॒ग्नी ए॒ष ए॒षो᳚ऽग्नी । \newline
17. अ॒ग्नी आ ऽग्नी अ॒ग्नी आ । \newline
18. अ॒ग्नी इत्य॒ग्नी । \newline
19. आ रो॑हति रोह॒त्या रो॑हति । \newline
20. रो॒ह॒ति॒ यो यो रो॑हति रोहति॒ यः । \newline
21. य ए॑ना वेनौ॒ यो य ए॑नौ । \newline
22. ए॒ना॒ वु॒प॒तिष्ठ॑त उप॒तिष्ठ॑त एना वेना वुप॒तिष्ठ॑ते । \newline
23. उ॒प॒तिष्ठ॑ते॒ यथा॒ यथो॑प॒तिष्ठ॑त उप॒तिष्ठ॑ते॒ यथा᳚ । \newline
24. उ॒प॒तिष्ठ॑त॒ इत्यु॑प - तिष्ठ॑ते । \newline
25. यथा॒ खलु॒ खलु॒ यथा॒ यथा॒ खलु॑ । \newline
26. खलु॒ वै वै खलु॒ खलु॒ वै । \newline
27. वै श्रेया॒ञ् छ्रेया॒न्॒. वै वै श्रेयान्॑ । \newline
28. श्रेया॑ न॒भ्यारू॑ढो॒ ऽभ्यारू॑ढः॒ श्रेया॒ञ् छ्रेया॑ न॒भ्यारू॑ढः । \newline
29. अ॒भ्यारू॑ढः का॒मय॑ते का॒मय॑ते॒ ऽभ्यारू॑ढो॒ ऽभ्यारू॑ढः का॒मय॑ते । \newline
30. अ॒भ्यारू॑ढ॒ इत्य॑भि - आरू॑ढः । \newline
31. का॒मय॑ते॒ तथा॒ तथा॑ का॒मय॑ते का॒मय॑ते॒ तथा᳚ । \newline
32. तथा॑ करोति करोति॒ तथा॒ तथा॑ करोति । \newline
33. क॒रो॒ति॒ नक्त॒न्नक्त॑म् करोति करोति॒ नक्त᳚म् । \newline
34. नक्त॒ मुपोप॒ नक्त॒न्नक्त॒ मुप॑ । \newline
35. उप॑ तिष्ठते तिष्ठत॒ उपोप॑ तिष्ठते । \newline
36. ति॒ष्ठ॒ते॒ न न ति॑ष्ठते तिष्ठते॒ न । \newline
37. न प्रा॒तः प्रा॒तर् न न प्रा॒तः । \newline
38. प्रा॒तः सꣳ सम् प्रा॒तः प्रा॒तः सम् । \newline
39. सꣳ हि हि सꣳ सꣳ हि । \newline
40. हि नक्त॒न्नक्त॒(ग्म्॒) हि हि नक्त᳚म् । \newline
41. नक्तं॑ ॅव्र॒तानि॑ व्र॒तानि॒ नक्त॒न्नक्तं॑ ॅव्र॒तानि॑ । \newline
42. व्र॒तानि॑ सृ॒ज्यन्ते॑ सृ॒ज्यन्ते᳚ व्र॒तानि॑ व्र॒तानि॑ सृ॒ज्यन्ते᳚ । \newline
43. सृ॒ज्यन्ते॑ स॒ह स॒ह सृ॒ज्यन्ते॑ सृ॒ज्यन्ते॑ स॒ह । \newline
44. स॒ह श्रेया॒ञ् छ्रेया᳚न् थ्स॒ह स॒ह श्रेयान्॑ । \newline
45. श्रेया(ग्ग्॑)श्च च॒ श्रेया॒ञ् छ्रेया(ग्ग्॑)श्च । \newline
46. च॒ पापी॑या॒न् पापी॑याꣳश्च च॒ पापी॑यान् । \newline
47. पापी॑याꣳश्च च॒ पापी॑या॒न् पापी॑याꣳश्च । \newline
48. चा॒सा॒ते॒ आ॒सा॒ते॒ च॒ चा॒सा॒ते॒ । \newline
49. आ॒सा॒ते॒ ज्योति॒र् ज्योति॑रासाते आसाते॒ ज्योतिः॑ । \newline
50. आ॒सा॒ते॒ इत्या॑साते । \newline
51. ज्योति॒र् वै वै ज्योति॒र् ज्योति॒र् वै । \newline
52. वा अ॒ग्निर॒ग्निर् वै वा अ॒ग्निः । \newline
53. अ॒ग्नि स्तम॒ स्तमो॒ ऽग्नि र॒ग्निस्तमः॑ । \newline
54. तमो॒ रात्री॒ रात्रि॒ स्तम॒स्तमो॒ रात्रिः॑ । \newline
55. रात्रि॒र् यद् यद् रात्री॒ रात्रि॒र् यत् । \newline
56. यन् नक्त॒न्नक्तं॒ ॅयद् यन् नक्त᳚म् । \newline

\textbf{Ghana Paata } \newline

1. ए॒वं ॅवि॒द्वान्. वि॒द्वा ने॒व मे॒वं ॅवि॒द्वा न॒ग्नि म॒ग्निं ॅवि॒द्वा ने॒व मे॒वं ॅवि॒द्वा न॒ग्निम् । \newline
2. वि॒द्वा न॒ग्नि म॒ग्निं ॅवि॒द्वान्. वि॒द्वा न॒ग्नि मु॑प॒तिष्ठ॑त उप॒तिष्ठ॑ते॒ ऽग्निं ॅवि॒द्वान्. वि॒द्वा न॒ग्नि मु॑प॒तिष्ठ॑ते । \newline
3. अ॒ग्नि मु॑प॒तिष्ठ॑त उप॒तिष्ठ॑ते॒ ऽग्नि म॒ग्नि मु॑प॒तिष्ठ॑ते सुव॒र्गꣳ सु॑व॒र्ग मु॑प॒तिष्ठ॑ते॒ ऽग्नि म॒ग्नि मु॑प॒तिष्ठ॑ते सुव॒र्गम् । \newline
4. उ॒प॒तिष्ठ॑ते सुव॒र्गꣳ सु॑व॒र्ग मु॑प॒तिष्ठ॑त उप॒तिष्ठ॑ते सुव॒र्ग मे॒वैव सु॑व॒र्ग मु॑प॒तिष्ठ॑त उप॒तिष्ठ॑ते सुव॒र्ग मे॒व । \newline
5. उ॒प॒तिष्ठ॑त॒ इत्यु॑प - तिष्ठ॑ते । \newline
6. सु॒व॒र्ग मे॒वैव सु॑व॒र्गꣳ सु॑व॒र्ग मे॒व लो॒कम् ॅलो॒क मे॒व सु॑व॒र्गꣳ सु॑व॒र्ग मे॒व लो॒कम् । \newline
7. सु॒व॒र्गमिति॑ सुवः - गम् । \newline
8. ए॒व लो॒कम् ॅलो॒क मे॒वैव लो॒क मे᳚त्येति लो॒क मे॒वैव लो॒क मे॑ति । \newline
9. लो॒क मे᳚त्येति लो॒कम् ॅलो॒क मे॑ति॒ सर्व॒(ग्म्॒) सर्व॑ मेति लो॒कम् ॅलो॒क मे॑ति॒ सर्व᳚म् । \newline
10. ए॒ति॒ सर्व॒(ग्म्॒) सर्व॑ मेत्येति॒ सर्व॒ मायु॒रायुः॒ सर्व॑ मेत्येति॒ सर्व॒ मायुः॑ । \newline
11. सर्व॒ मायु॒रायुः॒ सर्व॒(ग्म्॒) सर्व॒ मायु॑ रेत्ये॒त्यायुः॒ सर्व॒(ग्म्॒) सर्व॒ मायु॑रेति । \newline
12. आयु॑ रेत्ये॒त्यायु॒ रायु॑रेत्य॒भ्या᳚(1॒)भ्ये᳚ त्यायु॒ रायु॑रेत्य॒भि । \newline
13. ए॒त्य॒भ्या᳚(1॒)भ्ये᳚त्येत्य॒भि वै वा अ॒भ्ये᳚त्येत्य॒भि वै । \newline
14. अ॒भि वै वा अ॒भ्य॑भि वा ए॒ष ए॒ष वा अ॒भ्य॑भि वा ए॒षः । \newline
15. वा ए॒ष ए॒ष वै वा ए॒षो᳚ ऽग्नी अ॒ग्नी ए॒ष वै वा ए॒षो᳚ ऽग्नी । \newline
16. ए॒षो᳚ ऽग्नी अ॒ग्नी ए॒ष ए॒षो᳚ ऽग्नी आ ऽग्नी ए॒ष ए॒षो᳚ ऽग्नी आ । \newline
17. अ॒ग्नी आ ऽग्नी अग्नी आ रो॑हति रोह॒त्या ऽग्नी अ॒ग्नी आ रो॑हति । \newline
18. अ॒ग्नी इत्य॒ग्नी । \newline
19. आ रो॑हति रोह॒त्या रो॑हति॒ यो यो रो॑ह॒त्या रो॑हति॒ यः । \newline
20. रो॒ह॒ति॒ यो यो रो॑हति रोहति॒ य ए॑ना वेनौ॒ यो रो॑हति रोहति॒ य ए॑नौ । \newline
21. य ए॑ना वेनौ॒ यो य ए॑ना वुप॒तिष्ठ॑त उप॒तिष्ठ॑त एनौ॒ यो य ए॑ना वुप॒तिष्ठ॑ते । \newline
22. ए॒ना॒ वु॒प॒तिष्ठ॑त उप॒तिष्ठ॑त एना वेना वुप॒तिष्ठ॑ते॒ यथा॒ यथो॑प॒तिष्ठ॑त एना वेना वुप॒तिष्ठ॑ते॒ यथा᳚ । \newline
23. उ॒प॒तिष्ठ॑ते॒ यथा॒ यथो॑प॒तिष्ठ॑त उप॒तिष्ठ॑ते॒ यथा॒ खलु॒ खलु॒ यथो॑प॒तिष्ठ॑त उप॒तिष्ठ॑ते॒ यथा॒ खलु॑ । \newline
24. उ॒प॒तिष्ठ॑त॒ इत्यु॑प - तिष्ठ॑ते । \newline
25. यथा॒ खलु॒ खलु॒ यथा॒ यथा॒ खलु॒ वै वै खलु॒ यथा॒ यथा॒ खलु॒ वै । \newline
26. खलु॒ वै वै खलु॒ खलु॒ वै श्रेया॒न् श्रेया॒न्॒. वै खलु॒ खलु॒ वै श्रेयान्॑ । \newline
27. वै श्रेया॒न् श्रेया॒न्॒. वै वै श्रेया॑ न॒भ्यारू॑ढो॒ ऽभ्यारू॑ढः॒ श्रेया॒न्॒. वै वै श्रेया॑ न॒भ्यारू॑ढः । \newline
28. श्रेया॑ न॒भ्यारू॑ढो॒ ऽभ्यारू॑ढः॒ श्रेया॒न् श्रेया॑ न॒भ्यारू॑ढः का॒मय॑ते का॒मय॑ते॒ ऽभ्यारू॑ढः॒ श्रेया॒न् श्रेया॑ न॒भ्यारू॑ढः का॒मय॑ते । \newline
29. अ॒भ्यारू॑ढः का॒मय॑ते का॒मय॑ते॒ ऽभ्यारू॑ढो॒ ऽभ्यारू॑ढः का॒मय॑ते॒ तथा॒ तथा॑ का॒मय॑ते॒ ऽभ्यारू॑ढो॒ ऽभ्यारू॑ढः का॒मय॑ते॒ तथा᳚ । \newline
30. अ॒भ्यारू॑ढ॒ इत्य॑भि - आरू॑ढः । \newline
31. का॒मय॑ते॒ तथा॒ तथा॑ का॒मय॑ते का॒मय॑ते॒ तथा॑ करोति करोति॒ तथा॑ का॒मय॑ते का॒मय॑ते॒ तथा॑ करोति । \newline
32. तथा॑ करोति करोति॒ तथा॒ तथा॑ करोति॒ नक्त॒न्नक्त॑म् करोति॒ तथा॒ तथा॑ करोति॒ नक्त᳚म् । \newline
33. क॒रो॒ति॒ नक्त॒न्नक्त॑म् करोति करोति॒ नक्त॒ मुपोप॒ नक्त॑म् करोति करोति॒ नक्त॒ मुप॑ । \newline
34. नक्त॒ मुपोप॒ नक्त॒न्नक्त॒ मुप॑ तिष्ठते तिष्ठत॒ उप॒ नक्त॒न्नक्त॒ मुप॑ तिष्ठते । \newline
35. उप॑ तिष्ठते तिष्ठत॒ उपोप॑ तिष्ठते॒ न न ति॑ष्ठत॒ उपोप॑ तिष्ठते॒ न । \newline
36. ति॒ष्ठ॒ते॒ न न ति॑ष्ठते तिष्ठते॒ न प्रा॒तः प्रा॒तर् न ति॑ष्ठते तिष्ठते॒ न प्रा॒तः । \newline
37. न प्रा॒तः प्रा॒तर् न न प्रा॒तः सꣳ सम् प्रा॒तर् न न प्रा॒तः सम् । \newline
38. प्रा॒तः सꣳ सम् प्रा॒तः प्रा॒तः सꣳ हि हि सम् प्रा॒तः प्रा॒तः सꣳ हि । \newline
39. सꣳ हि हि सꣳ सꣳ हि नक्त॒न्नक्त॒(ग्म्॒) हि सꣳ सꣳ हि नक्त᳚म् । \newline
40. हि नक्त॒न्नक्त॒(ग्म्॒) हि हि नक्तं॑ ॅव्र॒तानि॑ व्र॒तानि॒ नक्त॒(ग्म्॒) हि हि नक्तं॑ ॅव्र॒तानि॑ । \newline
41. नक्तं॑ ॅव्र॒तानि॑ व्र॒तानि॒ नक्त॒न्नक्तं॑ ॅव्र॒तानि॑ सृ॒ज्यन्ते॑ सृ॒ज्यन्ते᳚ व्र॒तानि॒ नक्त॒न्नक्तं॑ ॅव्र॒तानि॑ सृ॒ज्यन्ते᳚ । \newline
42. व्र॒तानि॑ सृ॒ज्यन्ते॑ सृ॒ज्यन्ते᳚ व्र॒तानि॑ व्र॒तानि॑ सृ॒ज्यन्ते॑ स॒ह स॒ह सृ॒ज्यन्ते᳚ व्र॒तानि॑ व्र॒तानि॑ सृ॒ज्यन्ते॑ स॒ह । \newline
43. सृ॒ज्यन्ते॑ स॒ह स॒ह सृ॒ज्यन्ते॑ सृ॒ज्यन्ते॑ स॒ह श्रेया॒ञ् छ्रेया᳚न् थ्स॒ह सृ॒ज्यन्ते॑ सृ॒ज्यन्ते॑ स॒ह श्रेयान्॑ । \newline
44. स॒ह श्रेया॒ञ् छ्रेया᳚न् थ्स॒ह स॒ह श्रेया(ग्ग्॑)श्च च॒ श्रेया᳚न् थ्स॒ह स॒ह श्रेया(ग्ग्॑)श्च । \newline
45. श्रेया(ग्ग्॑)श्च च॒ श्रेया॒ञ् छ्रेया(ग्ग्॑)श्च॒ पापी॑या॒न् पापी॑याꣳश्च॒ श्रेया॒ञ् छ्रेया(ग्ग्॑)श्च॒ पापी॑यान् । \newline
46. च॒ पापी॑या॒न् पापी॑याꣳश्च च॒ पापी॑याꣳश्च च॒ पापी॑याꣳश्च च॒ पापी॑याꣳश्च । \newline
47. पापी॑याꣳश्च च॒ पापी॑या॒न् पापी॑याꣳ श्चासाते आसाते च॒ पापी॑या॒न् पापी॑याꣳ श्चासाते । \newline
48. चा॒सा॒ते॒ आ॒सा॒ते॒ च॒ चा॒सा॒ते॒ ज्योति॒र् ज्योति॑रासाते च चासाते॒ ज्योतिः॑ । \newline
49. आ॒सा॒ते॒ ज्योति॒र् ज्योति॑रासाते आसाते॒ ज्योति॒र् वै वै ज्योति॑रासाते आसाते॒ ज्योति॒र् वै । \newline
50. आ॒सा॒ते॒ इत्या॑साते । \newline
51. ज्योति॒र् वै वै ज्योति॒र् ज्योति॒र् वा अ॒ग्नि र॒ग्निर् वै ज्योति॒र् ज्योति॒र् वा अ॒ग्निः । \newline
52. वा अ॒ग्निर॒ग्निर् वै वा अ॒ग्नि स्तम॒ स्तमो॒ ऽग्निर् वै वा अ॒ग्निस्तमः॑ । \newline
53. अ॒ग्नि स्तम॒ स्तमो॒ ऽग्निर॒ग्नि स्तमो॒ रात्री॒ रात्रि॒स्तमो॒ ऽग्निर॒ग्नि स्तमो॒ रात्रिः॑ । \newline
54. तमो॒ रात्री॒ रात्रि॒स्तम॒ स्तमो॒ रात्रि॒र् यद् यद् रात्रि॒स्तम॒ स्तमो॒ रात्रि॒र् यत् । \newline
55. रात्रि॒र् यद् यद् रात्री॒ रात्रि॒र् यन् नक्त॒न्नक्तं॒ ॅयद् रात्री॒ रात्रि॒र् यन् नक्त᳚म् । \newline
56. यन् नक्त॒न्नक्तं॒ ॅयद् यन् नक्त॑ मुप॒तिष्ठ॑त उप॒तिष्ठ॑ते॒ नक्तं॒ ॅयद् यन् नक्त॑ मुप॒तिष्ठ॑ते । \newline
\pagebreak
\markright{ TS 1.5.9.6  \hfill https://www.vedavms.in \hfill}
\addcontentsline{toc}{section}{ TS 1.5.9.6 }
\section*{ TS 1.5.9.6 }

\textbf{TS 1.5.9.6 } \newline
\textbf{Samhita Paata} \newline

न्नक्त॑मुप॒तिष्ठ॑ते॒ ज्योति॑षै॒व तम॑स्तरत्युप॒स्थेयो॒ ऽग्नी(3)र् नोप॒स्थेया(3) इत्या॑हुर् मनु॒ष्या॑येन्न्वै योऽह॑रहरा॒हृत्याऽथै॑नं॒ ॅयाच॑ति॒ स इन्न्वै तमुपा᳚र्च्छ॒त्यथ॒ को दे॒वानह॑रहर्याचिष्य॒तीति॒ तस्मा॒न्नोप॒स्थेयो ऽथो॒ खल्वा॑हुरा॒शिषे॒ वै कं ॅयज॑मानो यजत॒ इत्ये॒षा खलु॒ वा - [ ] \newline

\textbf{Pada Paata} \newline

नक्त᳚म् । उ॒प॒तिष्ठ॑त॒ इत्यु॑प - तिष्ठ॑ते । ज्योति॑षा । ए॒व । तमः॑ । त॒र॒ति॒ । उ॒प॒स्थेय॒ इत्यु॑प - स्थेयः॑ । अ॒ग्नी(3)ः । न । उ॒प॒स्थेया(3) इत्यु॑प-स्थेया(3)ः । इति॑ । आ॒हुः॒ । म॒नु॒ष्या॑य । इत् । नु । वै । यः । अह॑रह॒रित्यहः॑ - अ॒हः॒ । आ॒हृत्येत्या᳚ - हृत्य॑ । अथ॑ । ए॒न॒म् । याच॑ति । सः । इत् । नु । वै । तम् । उपेति॑ । ऋ॒च्छ॒ति॒ । अथ॑ । कः । दे॒वान् । अह॑रह॒रित्यहः॑-अ॒हः॒ । या॒चि॒ष्य॒ति॒ । इति॑ । तस्मा᳚त् । न । उ॒प॒स्थेय॒ इत्यु॑प - स्थेयः॑ । अथो॒ इति॑ । खलु॑ । आ॒हुः॒ । आ॒शिष॒ इत्या᳚ - शिषे᳚ । वै । कम् । यज॑मानः । य॒ज॒ते॒ । इति॑ । ए॒षा । खलु॑ । वै ।  \newline


\textbf{Krama Paata} \newline

नक्त॑मुप॒तिष्ठ॑ते । उ॒प॒तिष्ठ॑ते॒ ज्योति॑षा । उ॒प॒ति॑ष्ठत॒ इत्यु॑प - तिष्ठ॑ते । ज्योति॑षै॒व । ए॒व तमः॑ । तम॑स्तरति । त॒र॒त्यु॒प॒स्थेयः॑ । उ॒प॒स्थेयो॒ऽग्नी(3)ः । उ॒प॒स्थेय॒ इत्यु॑प - स्थेयः॑ । अ॒ग्नी(3)र्,न । नोप॒स्थेया(3)ः । उ॒प॒स्थेया(3) इति॑ । उ॒प॒स्थेया(3) इत्यु॑प - स्थेया(3)ः । इत्या॑हुः । आ॒हु॒र्,म॒नु॒ष्या॑य । म॒नु॒ष्या॑येत् । इन्नु । न्वै । वै यः । योऽह॑रहः । अह॑रहरा॒हृत्य॑ । अह॑रह॒रित्यहः॑ - अ॒हः॒ । आ॒हृत्याथ॑ । आ॒हृत्येत्या᳚ - हृत्य॑ । अथै॑नम् । ए॒नं॒ ॅयाच॑ति । याच॑ति॒ सः । स इत् । इन्नु । न्वै । वै तम् । तमुप॑ । उपा᳚र्च्छति । ऋ॒च्छ॒त्यथ॑ । अथ॒ कः । को दे॒वान् । दे॒वानह॑रहः । अह॑रहर् याचिष्यति । अह॑रह॒रित्यहः॑ - अ॒हः॒ । या॒चि॒ष्य॒तीति॑ । इति॒ तस्मा᳚त् । तस्मा॒न्न । नोप॒स्थेयः॑ । उ॒प॒स्थेयोऽथो᳚ । उ॒प॒स्थेय॒ इत्यु॑प - स्थेयः॑ । अथो॒ खलु॑ । अथो॒ इत्यथो᳚ । खल्वा॑हुः । आ॒हु॒रा॒शिषे᳚ । आ॒शिषे॒ वै । आ॒शिष॒ इत्या᳚ - शिषे᳚ । वै कम् । कं ॅयज॑मानः । यज॑मानो यजते । य॒ज॒त॒ इति॑ । इत्ये॒षा । ए॒षा खलु॑ । खलु॒ वै । वा आहि॑ताग्नेः \newline

\textbf{Jatai Paata} \newline

1. नक्त॑ मुप॒तिष्ठ॑त उप॒तिष्ठ॑ते॒ नक्त॒न्नक्त॑ मुप॒तिष्ठ॑ते । \newline
2. उ॒प॒तिष्ठ॑ते॒ ज्योति॑षा॒ ज्योति॑षोप॒तिष्ठ॑त उप॒तिष्ठ॑ते॒ ज्योति॑षा । \newline
3. उ॒प॒तिष्ठ॑त॒ इत्यु॑प - तिष्ठ॑ते । \newline
4. ज्योति॑षै॒वैव ज्योति॑षा॒ ज्योति॑षै॒व । \newline
5. ए॒व तम॒स्तम॑ ए॒वैव तमः॑ । \newline
6. तम॑ स्तरति तरति॒ तम॒ स्तम॑स्तरति । \newline
7. त॒र॒त्यु॒प॒स्थेय॑ उप॒स्थेय॑स्तरति तरत्युप॒स्थेयः॑ । \newline
8. उ॒प॒स्थेयो॒ ऽग्नी(3) र॒ग्नी(3) रु॑प॒स्थेय॑ उप॒स्थेयो॒ ऽग्नी(3)ः । \newline
9. उ॒प॒स्थेय॒ इत्यु॑प - स्थेयः॑ । \newline
10. अ॒ग्नी(3)र् न नाग्नी(3) र॒ग्नी(3)र् न । \newline
11. नोप॒स्थेया(3) उ॑प॒स्थेया(3) न नोप॒स्थेया(3)ः । \newline
12. उ॒प॒स्थेया(3) इतीत्यु॑प॒स्थेया(3) उ॑प॒स्थेया(3) इति॑ । \newline
13. उ॒प॒स्थेया(3) इत्यु॑प - स्थेया(3)ः । \newline
14. इत्या॑हु राहु॒ रितीत्या॑हुः । \newline
15. आ॒हु॒र् म॒नु॒ष्या॑य मनु॒ष्या॑याहुराहुर् मनु॒ष्या॑य । \newline
16. म॒नु॒ष्या॑ये दिन् म॑नु॒ष्या॑य मनु॒ष्या॑ये त् । \newline
17. इन् नु न्विदिन् नु । \newline
18. न्वै वै नुन्वै । \newline
19. वै यो यो वै वै यः । \newline
20. यो ऽह॑रह॒ रह॑रह॒र् यो यो ऽह॑रहः । \newline
21. अह॑रह रा॒हृत्या॒हृत्याह॑रह॒ रह॑रह रा॒हृत्य॑ । \newline
22. अह॑रह॒रित्यहः॑ - अ॒हः॒ । \newline
23. आ॒हृत्या थाथा॒हृत्या॒हृत्याथ॑ । \newline
24. आ॒हृत्येत्या᳚ - हृत्य॑ । \newline
25. अथै॑न मेन॒ मथाथै॑नम् । \newline
26. ए॒नं॒ ॅयाच॑ति॒ याच॑त्येन मेनं॒ ॅयाच॑ति । \newline
27. याच॑ति॒ स स याच॑ति॒ याच॑ति॒ सः । \newline
28. स इदिथ् स स इत् । \newline
29. इन् नु न्विदिन् नु । \newline
30. न्वै वै नु न्वै । \newline
31. वै तम् तं ॅवै वै तम् । \newline
32. त मुपोप॒ तम् त मुप॑ । \newline
33. उपा᳚र् च्छत्यृच्छ॒त्युपोपा᳚र् च्छति । \newline
34. ऋ॒च्छ॒त्यथाथ॑ र्च्छत्यृच्छ॒त्यथ॑ । \newline
35. अथ॒ कः को ऽथाथ॒ कः । \newline
36. को दे॒वान् दे॒वान् कः को दे॒वान् । \newline
37. दे॒वा नह॑रह॒ रह॑रहर् दे॒वान् दे॒वा नह॑रहः । \newline
38. अह॑रहर् याचिष्यति याचिष्य॒त्यह॑रह॒ रह॑रहर् याचिष्यति । \newline
39. अह॑रह॒रित्यहः॑ - अ॒हः॒ । \newline
40. या॒चि॒ष्य॒तीतीति॑ याचिष्यति याचिष्य॒तीति॑ । \newline
41. इति॒ तस्मा॒त् तस्मा॒दितीति॒ तस्मा᳚त् । \newline
42. तस्मा॒न् न न तस्मा॒त् तस्मा॒न् न । \newline
43. नोप॒स्थेय॑ उप॒स्थेयो॒ न नोप॒स्थेयः॑ । \newline
44. उ॒प॒स्थेयो ऽथो॒ अथो॑ उप॒स्थेय॑ उप॒स्थेयो ऽथो᳚ । \newline
45. उ॒प॒स्थेय॒ इत्यु॑प - स्थेयः॑ । \newline
46. अथो॒ खलु॒ खल्वथो॒ अथो॒ खलु॑ । \newline
47. अथो॒ इत्यथो᳚ । \newline
48. खल्वा॑हुराहुः॒ खलु॒ खल्वा॑हुः । \newline
49. आ॒हु॒रा॒शिष॑ आ॒शिष॑ आहुराहुरा॒शिषे᳚ । \newline
50. आ॒शिषे॒ वै वा आ॒शिष॑ आ॒शिषे॒ वै । \newline
51. आ॒शिष॒ इत्या᳚ - शिषे᳚ । \newline
52. वै कम् कं ॅवै वै कम् । \newline
53. कं ॅयज॑मानो॒ यज॑मानः॒ कम् कं ॅयज॑मानः । \newline
54. यज॑मानो यजते यजते॒ यज॑मानो॒ यज॑मानो यजते । \newline
55. य॒ज॒त॒ इतीति॑ यजते यजत॒ इति॑ । \newline
56. इत्ये॒षैषेतीत्ये॒षा । \newline
57. ए॒षा खलु॒ खल्वे॒षैषा खलु॑ । \newline
58. खलु॒ वै वै खलु॒ खलु॒ वै । \newline
59. वा आहि॑ताग्ने॒ राहि॑ताग्ने॒र् वै वा आहि॑ताग्नेः । \newline

\textbf{Ghana Paata } \newline

1. नक्त॑ मुप॒तिष्ठ॑त उप॒तिष्ठ॑ते॒ नक्त॒न्नक्त॑ मुप॒तिष्ठ॑ते॒ ज्योति॑षा॒ ज्योति॑षोप॒तिष्ठ॑ते॒ नक्त॒न्नक्त॑ मुप॒तिष्ठ॑ते॒ ज्योति॑षा । \newline
2. उ॒प॒तिष्ठ॑ते॒ ज्योति॑षा॒ ज्योति॑षोप॒तिष्ठ॑त उप॒तिष्ठ॑ते॒ ज्योति॑षै॒वैव ज्योति॑षोप॒तिष्ठ॑त उप॒तिष्ठ॑ते॒ ज्योति॑षै॒व । \newline
3. उ॒प॒तिष्ठ॑त॒ इत्यु॑प - तिष्ठ॑ते । \newline
4. ज्योति॑षै॒वैव ज्योति॑षा॒ ज्योति॑षै॒व तम॒स्तम॑ ए॒व ज्योति॑षा॒ ज्योति॑षै॒व तमः॑ । \newline
5. ए॒व तम॒स्तम॑ ए॒वैव तम॑स्तरति तरति॒ तम॑ ए॒वैव तम॑स्तरति । \newline
6. तम॑ स्तरति तरति॒ तम॒ स्तम॑ स्तरत्युप॒स्थेय॑ उप॒स्थेय॑स्तरति॒ तम॒ स्तम॑ स्तरत्युप॒स्थेयः॑ । \newline
7. त॒र॒त्यु॒प॒स्थेय॑ उप॒स्थेय॑स्तरति तरत्युप॒स्थेयो॒ ऽग्नी(3) र॒ग्नी(3) रु॑प॒स्थेय॑स्तरति तरत्युप॒स्थेयो॒ ऽग्नी(3)ः । \newline
8. उ॒प॒स्थेयो॒ ऽग्नी(3) र॒ग्नी(3) रु॑प॒स्थेय॑ उप॒स्थेयो॒ ऽग्नी(3)र् न नाग्नी(3) रु॑प॒स्थेय॑ उप॒स्थेयो॒ ऽग्नी(3)र् न । \newline
9. उ॒प॒स्थेय॒ इत्यु॑प - स्थेयः॑ । \newline
10. अ॒ग्नी(3)र् न नाग्नी(3) र॒ग्नी(3)र् नोप॒स्थेया(3) उ॑प॒स्थेया(3) नाग्नी(3)र॒ग्नी(3)र् नोप॒स्थेया(3)ः । \newline
11. नोप॒स्थेया(3) उ॑प॒स्थेया(3) न नोप॒स्थेया(3) इतीत्यु॑प॒स्थेया(3) न नोप॒स्थेया(3) इति॑ । \newline
12. उ॒प॒स्थेया(3) इतीत्यु॑प॒स्थेया(3) उ॑प॒स्थेया(3) इत्या॑हु राहु॒रित्यु॑प॒स्थेया(3) उ॑प॒स्थेया(3) इत्या॑हुः । \newline
13. उ॒प॒स्थेया(3) इत्यु॑प - स्थेया(3)ः । \newline
14. इत्या॑हु राहु॒रितीत्या॑हुर् मनु॒ष्या॑य मनु॒ष्या॑याहु॒ रितीत्या॑हुर् मनु॒ष्या॑य । \newline
15. आ॒हु॒र् म॒नु॒ष्या॑य मनु॒ष्या॑याहुराहुर् मनु॒ष्या॑ये दिन् म॑नु॒ष्या॑याहुराहुर् मनु॒ष्या॑ये त् । \newline
16. म॒नु॒ष्या॑ये दिन् म॑नु॒ष्या॑य मनु॒ष्या॑येन् नु न्विन् म॑नु॒ष्या॑य मनु॒ष्या॑येन् नु । \newline
17. इन् नु न्विदिन् न्वै वै न्विदिन् न्वै । \newline
18. न्वै वै नु न्वै यो यो वै नु न्वै यः । \newline
19. वै यो यो वै वै यो ऽह॑रह॒ रह॑रह॒र् यो वै वै यो ऽह॑रहः । \newline
20. यो ऽह॑रह॒ रह॑रह॒र् यो यो ऽह॑रह रा॒हृत्या॒हृत्या ह॑रह॒र् यो यो ऽह॑रह रा॒हृत्य॑ । \newline
21. अह॑रह रा॒हृत्या॒हृत्या ह॑रह॒ रह॑रह रा॒हृत्याथाथा॒हृत्या ह॑रह॒ रह॑रह रा॒हृत्याथ॑ । \newline
22. अह॑रह॒रित्यहः॑ - अ॒हः॒ । \newline
23. आ॒हृत्याथाथा॒ हृत्या॒हृत्याथै॑न मेन॒ मथा॒हृत्या॒ हृत्याथै॑नम् । \newline
24. आ॒हृत्येत्या᳚ - हृत्य॑ । \newline
25. अथै॑न मेन॒ मथाथै॑नं॒ ॅयाच॑ति॒ याच॑त्येन॒ मथाथै॑नं॒ ॅयाच॑ति । \newline
26. ए॒नं॒ ॅयाच॑ति॒ याच॑त्येन मेनं॒ ॅयाच॑ति॒ स स याच॑त्येन मेनं॒ ॅयाच॑ति॒ सः । \newline
27. याच॑ति॒ स स याच॑ति॒ याच॑ति॒ स इदिथ् स याच॑ति॒ याच॑ति॒ स इत् । \newline
28. स इदिथ् स स इन् नु न्विथ् स स इन् नु । \newline
29. इन् नु न्विदिन् न्वै वै न्विदिन् न्वै । \newline
30. न्वै वै नु न्वै तम् तं ॅवै नु न्वै तम् । \newline
31. वै तम् तं ॅवै वै त मुपोप॒ तं ॅवै वै त मुप॑ । \newline
32. त मुपोप॒ तम् त मुपा᳚ र्‌च्छ त्यृच्छ॒ त्युप॒ तम् त मुपा᳚ र्‌च्छति । \newline
33. उपा᳚ र्‌च्छत्यृच्छ॒ त्युपोपा᳚ र्‌च्छ॒त्यथाथ॑ र्च्छ॒त्युपोपा᳚ र्‌च्छ॒त्यथ॑ । \newline
34. ऋ॒च्छ॒त्यथाथ॑ र्च्छत्यृच्छ॒त्यथ॒ कः को ऽथ॑ र्च्छत्यृच्छ॒त्यथ॒ कः । \newline
35. अथ॒ कः को ऽथाथ॒ को दे॒वान् दे॒वान् को ऽथाथ॒ को दे॒वान् । \newline
36. को दे॒वान् दे॒वान् कः को दे॒वा नह॑रह॒ रह॑रहर् दे॒वान् कः को दे॒वा नह॑रहः । \newline
37. दे॒वा नह॑रह॒ रह॑रहर् दे॒वान् दे॒वा नह॑रहर् याचिष्यति याचिष्य॒त्यह॑रहर् दे॒वान् दे॒वा नह॑रहर् याचिष्यति । \newline
38. अह॑रहर् याचिष्यति याचिष्य॒त्यह॑रह॒ रह॑रहर् याचिष्य॒तीतीति॑ याचिष्य॒त्यह॑रह॒ रह॑रहर् याचिष्य॒तीति॑ । \newline
39. अह॑रह॒रित्यहः॑ - अ॒हः॒ । \newline
40. या॒चि॒ष्य॒तीतीति॑ याचिष्यति याचिष्य॒तीति॒ तस्मा॒त् तस्मा॒दिति॑ याचिष्यति याचिष्य॒तीति॒ तस्मा᳚त् । \newline
41. इति॒ तस्मा॒त् तस्मा॒ दितीति॒ तस्मा॒न् न न तस्मा॒ दितीति॒ तस्मा॒न् न । \newline
42. तस्मा॒न् न न तस्मा॒त् तस्मा॒न् नोप॒स्थेय॑ उप॒स्थेयो॒ न तस्मा॒त् तस्मा॒न् नोप॒स्थेयः॑ । \newline
43. नोप॒स्थेय॑ उप॒स्थेयो॒ न नोप॒स्थेयो ऽथो॒ अथो॑ उप॒स्थेयो॒ न नोप॒स्थेयो ऽथो᳚ । \newline
44. उ॒प॒स्थेयो ऽथो॒ अथो॑ उप॒स्थेय॑ उप॒स्थेयो ऽथो॒ खलु॒ खल्वथो॑ उप॒स्थेय॑ उप॒स्थेयो ऽथो॒ खलु॑ । \newline
45. उ॒प॒स्थेय॒ इत्यु॑प - स्थेयः॑ । \newline
46. अथो॒ खलु॒ खल्वथो॒ अथो॒ खल्वा॑हुराहुः॒ खल्वथो॒ अथो॒ खल्वा॑हुः । \newline
47. अथो॒ इत्यथो᳚ । \newline
48. खल्वा॑हुराहुः॒ खलु॒ खल्वा॑हु रा॒शिष॑ आ॒शिष॑ आहुः॒ खलु॒ खल्वा॑हु रा॒शिषे᳚ । \newline
49. आ॒हु॒रा॒शिष॑ आ॒शिष॑ आहुराहुरा॒शिषे॒ वै वा आ॒शिष॑ आहुराहुरा॒शिषे॒ वै । \newline
50. आ॒शिषे॒ वै वा आ॒शिष॑ आ॒शिषे॒ वै कम् कं ॅवा आ॒शिष॑ आ॒शिषे॒ वै कम् । \newline
51. आ॒शिष॒ इत्या᳚ - शिषे᳚ । \newline
52. वै कम् कं ॅवै वै कं ॅयज॑मानो॒ यज॑मानः॒ कं ॅवै वै कं ॅयज॑मानः । \newline
53. कं ॅयज॑मानो॒ यज॑मानः॒ कम् कं ॅयज॑मानो यजते यजते॒ यज॑मानः॒ कम् कं ॅयज॑मानो यजते । \newline
54. यज॑मानो यजते यजते॒ यज॑मानो॒ यज॑मानो यजत॒ इतीति॑ यजते॒ यज॑मानो॒ यज॑मानो यजत॒ इति॑ । \newline
55. य॒ज॒त॒ इतीति॑ यजते यजत॒ इत्ये॒षैषेति॑ यजते यजत॒ इत्ये॒षा । \newline
56. इत्ये॒षैषे तीत्ये॒षा खलु॒ खल्वे॒षे तीत्ये॒षा खलु॑ । \newline
57. ए॒षा खलु॒ खल्वे॒षैषा खलु॒ वै वै खल्वे॒षैषा खलु॒ वै । \newline
58. खलु॒ वै वै खलु॒ खलु॒ वा आहि॑ताग्ने॒ राहि॑ताग्ने॒र् वै खलु॒ खलु॒ वा आहि॑ताग्नेः । \newline
59. वा आहि॑ताग्ने॒ राहि॑ताग्ने॒र् वै वा आहि॑ताग्ने रा॒शी रा॒शी राहि॑ताग्ने॒र् वै वा आहि॑ताग्ने रा॒शीः । \newline
\pagebreak
\markright{ TS 1.5.9.7  \hfill https://www.vedavms.in \hfill}
\addcontentsline{toc}{section}{ TS 1.5.9.7 }
\section*{ TS 1.5.9.7 }

\textbf{TS 1.5.9.7 } \newline
\textbf{Samhita Paata} \newline

आहि॑ताग्नेरा॒शीर् यद॒ग्निमु॑प॒तिष्ठ॑ते॒ तस्मा॑दुप॒स्थेयः॑ प्र॒जाप॑तिः प॒शून॑सृजत॒ ते सृ॒ष्टा अ॑होरा॒त्रे प्राऽ*वि॑श॒न् ताञ्छन्दो॑भि॒-रन्व॑॑विन्द॒द्-यच्छन्दो॑भि-रुप॒तिष्ठ॑ते॒ स्वमे॒व तदन्वि॑च्छति॒ न तत्र॑ जा॒म्य॑स्तीत्या॑हु॒र्यो-ऽह॑रहरुप॒ तिष्ठ॑त॒ इति॒ यो वा अ॒ग्निं प्र॒त्यङ्ङु॑प॒ तिष्ठ॑ते॒ प्रत्ये॑नमोषति॒ यः परा॒ङ् विष्व॑ङ् प्र॒जया॑ प॒शुभि॑ ( ) रेति॒ कवा॑तिर्यङ्ङि॒वोप॑ तिष्ठेत॒ नैनं॑ प्र॒त्योष॑ति॒ न विष्व॑ङ् प्र॒जया॑ प॒शुभि॑रेति ॥ \newline

\textbf{Pada Paata} \newline

आहि॑ताग्ने॒रित्याहि॑त-अ॒ग्नेः॒ । आ॒शीरित्या᳚ - शीः । यत् । अ॒ग्निम् । उ॒प॒तिष्ठ॑त॒ इत्यु॑प - तिष्ठ॑ते । तस्मा᳚त् । उ॒प॒स्थेय॒ इत्यु॑प - स्थेयः॑ । प्र॒जाप॑ति॒रिति॑ प्र॒जा - प॒तिः॒ । प॒शून् । अ॒सृ॒ज॒त॒ । ते । सृ॒ष्टाः । अ॒हो॒रा॒त्रे इत्य॑हः - रा॒त्रे । प्रेति॑ । अ॒वि॒श॒न्न् । तान् । छन्दो॑भि॒रिति॒ छन्दः॑-भिः॒ । अन्विति॑ । अ॒वि॒न्द॒त् । यत् । छन्दो॑भि॒रिति॒ छन्दः॑-भिः॒ । उ॒प॒तिष्ठ॑त॒ इत्यु॑प - तिष्ठ॑ते । स्वम् । ए॒व । तत् । अन्विति॑ । इ॒च्छ॒ति॒ । न । तत्र॑ । जा॒मि । अ॒स्ति॒ । इति॑ । आ॒हुः॒ । यः । अह॑रह॒रित्यहः॑ - अ॒हः॒ । उ॒प॒तिष्ठ॑त॒ इत्यु॑प - तिष्ठ॑ते । इति॑ । यः । वै । अ॒ग्निम् । प्र॒त्यङ्ङ् । उ॒प॒तिष्ठ॑त॒ इत्यु॑प - तिष्ठ॑ते । प्रतीति॑ । ए॒न॒म् । ओ॒ष॒ति॒ । यः । पराङ्॑ । विष्वङ्ङ्॑ । प्र॒जयेति॑ प्र - जया᳚ । प॒शुभि॒रिति॑ प॒शु - भिः॒ ( ) । ए॒ति॒ । कवा॑तिर्य॒ङ्ङिति॒ कवा᳚ - ति॒र्य॒ङ्ङ् । इ॒व॒ । उपेति॑ । ति॒ष्ठे॒त॒ । न । ए॒न॒म् । प्र॒त्योष॒तीति॑ प्रति - ओष॑ति । न । विष्वङ्ङ्॑ । प्र॒जयेति॑ प्र - जया᳚ । प॒शुभि॒रिति॑ प॒शु - भिः॒ । ए॒ति॒ ॥  \newline


\textbf{Krama Paata} \newline

आहि॑ताग्नेरा॒शीः । आहि॑ताग्ने॒रित्याहि॑त - अ॒ग्नेः॒ । आ॒शीर् यत् । आ॒शीरित्या᳚ - शीः । यद॒ग्निम् । अ॒ग्निमु॑प॒तिष्ठ॑ते । उ॒प॒तिष्ठ॑ते॒ तस्मा᳚त् । उ॒प॒तिष्ठ॑त॒ इत्यु॑प - तिष्ठ॑ते । तस्मा॑दुप॒स्थेयः॑ । उ॒प॒स्थेयः॑ प्र॒जाप॑तिः । उ॒प॒स्थेय॒ इत्यु॑प - स्थेयः॑ । प्र॒जाप॑तिः प॒शून् । प्र॒जाप॑ति॒रिति॑ प्र॒जा - प॒तिः॒ । प॒शून॑सृजत । अ॒सृ॒ज॒त॒ ते । ते सृ॒ष्टाः । सृ॒ष्टा अ॑होरा॒त्रे । अ॒हो॒रा॒त्रे प्र । अ॒हो॒रा॒त्रे इत्य॑हः - रा॒त्रे । प्रावि॑शन्न् । अ॒वि॒श॒न् तान् । तान् छन्दो॑भिः । छन्दो॑भि॒रनु॑ । छन्दो॑भि॒रिति॒ छन्दः॑ - भिः॒ । अन्व॑विन्दत् । अ॒वि॒न्द॒द् यत् । यच्छन्दो॑भिः । छन्दो॑भिरुप॒तिष्ठ॑ते । छन्दो॑भि॒रिति॒ छन्दः॑ - भिः॒ । उ॒प॒तिष्ठ॑ते॒ स्वम् । उ॒प॒तिष्ठ॑त॒ इत्यु॑प - तिष्ठ॑ते । स्वमे॒व । ए॒व तत् । तदनु॑ । अन्वि॑च्छति । इ॒च्छ॒ति॒ न । न तत्र॑ । तत्र॑ जा॒मि । जा॒म्य॑स्ति । अ॒स्तीति॑ । इत्या॑हुः । आ॒हु॒र् यः । योऽह॑रहः । 
अह॑रहरुप॒तिष्ठ॑ते । अह॑रह॒रित्यहः॑ - अ॒हः॒ । उ॒प॒तिष्ठ॑त॒ इति॑ । उ॒प॒तिष्ठ॑त॒ इत्यु॑प - तिष्ठ॑ते । इति॒ यः । यो वै । वा अ॒ग्निम् । अ॒ग्निम् प्र॒त्यङ्ङ् । प्र॒त्यङ्ङु॑प॒तिष्ठ॑ते । उ॒प॒तिष्ठ॑ते॒ प्रति॑ । उ॒प॒तिष्ठ॑त॒ इत्यु॑प - तिष्ठ॑ते । प्रत्ये॑नम् । ए॒न॒मो॒ष॒ति॒ । ओ॒ष॒ति॒ यः । यः पराङ्॑ । परा॒ङ्॒ विष्वङ्ङ्॑ । विष्व॑ङ् प्र॒जया᳚ । प्र॒जया॑ प॒शुभिः॑ ( ) । प्र॒जयेति॑ प्र - जया᳚ । प॒शुभि॑रेति । प॒शुभि॒रिति॑ प॒शु - भिः॒ । ए॒ति॒ कवा॑तिर्यङ्ङ् । कवा॑तिर्यङिव । कवा॑तिर्य॒ङिति॒ कवा᳚ - ति॒र्य॒ङ्ङ्॒ । इ॒वोप॑ । उप॑ तिष्ठेत । ति॒ष्ठे॒त॒ न । नैन᳚म् । ए॒न॒म् प्र॒त्योष॑ति । प्र॒त्योष॑ति॒ न । प्र॒त्योष॒तीति॑ प्रति - ओष॑ति । न विष्वङ्ङ्॑ । विष्व॑ङ् प्र॒जया᳚ । प्र॒जया॑ प॒शुभिः॑ । प्र॒जयेति॑ प्र - जया᳚ । प॒शुभि॑रेति । प॒शुभि॒रिति॑ प॒शु - भिः॒ । ए॒तीत्ये॑ति । \newline

\textbf{Jatai Paata} \newline

1. आहि॑ताग्ने रा॒शी रा॒शी राहि॑ताग्ने॒ राहि॑ताग्ने रा॒शीः । \newline
2. आहि॑ताग्ने॒रित्याहि॑त - अ॒ग्नेः॒ । \newline
3. आ॒शीर् यद् यदा॒शीरा॒शीर् यत् । \newline
4. आ॒शीरित्या᳚ - शीः । \newline
5. यद॒ग्नि म॒ग्निं ॅयद् यद॒ग्निम् । \newline
6. अ॒ग्नि मु॑प॒तिष्ठ॑त उप॒तिष्ठ॑ते॒ ऽग्नि म॒ग्नि मु॑प॒तिष्ठ॑ते । \newline
7. उ॒प॒तिष्ठ॑ते॒ तस्मा॒त् तस्मा॑ दुप॒तिष्ठ॑त उप॒तिष्ठ॑ते॒ तस्मा᳚त् । \newline
8. उ॒प॒तिष्ठ॑त॒ इत्यु॑प - तिष्ठ॑ते । \newline
9. तस्मा॑ दुप॒स्थेय॑ उप॒स्थेय॒ स्तस्मा॒त् तस्मा॑दुप॒स्थेयः॑ । \newline
10. उ॒प॒स्थेयः॑ प्र॒जाप॑तिः प्र॒जाप॑ति रुप॒स्थेय॑ उप॒स्थेयः॑ प्र॒जाप॑तिः । \newline
11. उ॒प॒स्थेय॒ इत्यु॑प - स्थेयः॑ । \newline
12. प्र॒जाप॑तिः प॒शून् प॒शून् प्र॒जाप॑तिः प्र॒जाप॑तिः प॒शून् । \newline
13. प्र॒जाप॑ति॒रिति॑ प्र॒जा - प॒तिः॒ । \newline
14. प॒शू न॑सृजतासृजत प॒शून् प॒शू न॑सृजत । \newline
15. अ॒सृ॒ज॒त॒ ते ते॑ ऽसृजतासृजत॒ ते । \newline
16. ते सृ॒ष्टाः सृ॒ष्टास्ते ते सृ॒ष्टाः । \newline
17. सृ॒ष्टा अ॑होरा॒त्रे अ॑होरा॒त्रे सृ॒ष्टाः सृ॒ष्टा अ॑होरा॒त्रे । \newline
18. अ॒हो॒रा॒त्रे प्र प्राहो॑रा॒त्रे अ॑होरा॒त्रे प्र । \newline
19. अ॒हो॒रा॒त्रे इत्य॑हः - रा॒त्रे । \newline
20. प्रावि॑शन्नविश॒न् प्र प्रावि॑शन्न् । \newline
21. अ॒वि॒श॒न् ताꣳस्ता न॑विशन्नविश॒न् तान् । \newline
22. तान् छन्दो॑भि॒ श्छन्दो॑भि॒ स्ताꣳ स्तान् छन्दो॑भिः । \newline
23. छन्दो॑भि॒ रन्वनु॒ छन्दो॑भि॒ श्छन्दो॑भि॒रनु॑ । \newline
24. छन्दो॑भि॒रिति॒ छन्दः॑ - भिः॒ । \newline
25. अन्व॑विन्द दविन्द॒ दन्वन्व॑विन्दत् । \newline
26. अ॒वि॒न्द॒द् यद् यद॑विन्ददविन्द॒द् यत् । \newline
27. यछ् चन्दो॑भि॒ श्छन्दो॑भि॒र् यद् यछ्‌ चन्दो॑भिः । \newline
28. छन्दो॑भि रुप॒तिष्ठ॑त उप॒तिष्ठ॑ते॒ छन्दो॑भि॒ श्छन्दो॑भि रुप॒तिष्ठ॑ते । \newline
29. छन्दो॑भि॒रिति॒ छन्दः॑ - भिः॒ । \newline
30. उ॒प॒तिष्ठ॑ते॒ स्वꣳ स्व मु॑प॒तिष्ठ॑त उप॒तिष्ठ॑ते॒ स्वम् । \newline
31. उ॒प॒तिष्ठ॑त॒ इत्यु॑प - तिष्ठ॑ते । \newline
32. स्व मे॒वैव स्वꣳ स्व मे॒व । \newline
33. ए॒व तत् तदे॒वैव तत् । \newline
34. तदन्वनु॒ तत् तदनु॑ । \newline
35. अन्वि॑च्छतीच्छ॒ त्यन्वन्वि॑च्छति । \newline
36. इ॒च्छ॒ति॒ न ने च्छ॑तीच्छति॒ न । \newline
37. न तत्र॒ तत्र॒ न न तत्र॑ । \newline
38. तत्र॑ जा॒मि जा॒मि तत्र॒ तत्र॑ जा॒मि । \newline
39. जा॒म्य॑स्त्यस्ति जा॒मि जा॒म्य॑स्ति । \newline
40. अ॒स्तीतीत्य॑स्त्य॒स्तीति॑ । \newline
41. इत्या॑हु राहु॒ रितीत्या॑हुः । \newline
42. आ॒हु॒र् यो य आ॑हुराहु॒र् यः । \newline
43. यो ऽह॑रह॒ रह॑रह॒र् यो यो ऽह॑रहः । \newline
44. अह॑रह रुप॒तिष्ठ॑त उप॒तिष्ठ॒ते ऽह॑रह॒ रह॑रह रुप॒तिष्ठ॑ते । \newline
45. अह॑रह॒रित्यहः॑ - अ॒हः॒ । \newline
46. उ॒प॒तिष्ठ॑त॒ इतीत्यु॑प॒तिष्ठ॑त उप॒तिष्ठ॑त॒ इति॑ । \newline
47. उ॒प॒तिष्ठ॑त॒ इत्यु॑प - तिष्ठ॑ते । \newline
48. इति॒ यो य इतीति॒ यः । \newline
49. यो वै वै यो यो वै । \newline
50. वा अ॒ग्नि म॒ग्निं ॅवै वा अ॒ग्निम् । \newline
51. अ॒ग्निम् प्र॒त्यङ् प्र॒त्यङ् ङ॒ग्नि म॒ग्निम् प्र॒त्यङ्ङ् । \newline
52. प्र॒त्यङ् ङु॑प॒तिष्ठ॑त उप॒तिष्ठ॑ते प्र॒त्यङ् प्र॒त्यङ् ङु॑प॒तिष्ठ॑ते । \newline
53. उ॒प॒तिष्ठ॑ते॒ प्रति॒ प्रत्यु॑प॒तिष्ठ॑त उप॒तिष्ठ॑ते॒ प्रति॑ । \newline
54. उ॒प॒तिष्ठ॑त॒ इत्यु॑प - तिष्ठ॑ते । \newline
55. प्रत्ये॑न मेन॒म् प्रति॒ प्रत्ये॑नम् । \newline
56. ए॒न॒ मो॒ष॒त्यो॒ष॒त्ये॒न॒ मे॒न॒ मो॒ष॒ति॒ । \newline
57. ओ॒ष॒ति॒ यो य ओ॑षत्योषति॒ यः । \newline
58. यः परा॒ङ् परा॒ङ् यो यः पराङ्॑ । \newline
59. परा॒ङ् विष्व॒ङ्॒. विष्व॒ङ् परा॒ङ् परा॒ङ् विष्वङ्ङ्॑ । \newline
60. विष्वङ्॑ प्र॒जया᳚ प्र॒जया॒ विष्व॒ङ्॒. विष्व॑ङ् प्र॒जया᳚ । \newline
61. प्र॒जया॑ प॒शुभिः॑ प॒शुभिः॑ प्र॒जया᳚ प्र॒जया॑ प॒शुभिः॑ । \newline
62. प्र॒जयेति॑ प्र - जया᳚ । \newline
63. प॒शुभि॑ रेत्येति प॒शुभिः॑ प॒शुभि॑रेति । \newline
64. प॒शुभि॒रिति॑ प॒शु - भिः॒ । \newline
65. ए॒ति॒ कवा॑तिर्य॒ङ् कवा॑तिर्यङ् ङेत्येति॒ कवा॑तिर्यङ्ङ् । \newline
66. कवा॑तिर्यङ् ङिवेव॒ कवा॑तिर्य॒ङ् कवा॑तिर्यङिव । \newline
67. कवा॑तिर्य॒ङ् ङिति॒ कवा᳚ - ति॒र्य॒ङ्ङ् । \newline
68. इ॒वोपोपे॑ वे॒ वोप॑ । \newline
69. उप॑ तिष्ठेत तिष्ठे॒तोपोप॑ तिष्ठेत । \newline
70. ति॒ष्ठे॒त॒ न न ति॑ष्ठेत तिष्ठेत॒ न । \newline
71. नैन॑ मेन॒न्न नैन᳚म् । \newline
72. ए॒न॒म् प्र॒त्योष॑ति प्र॒त्योष॑त्येन मेनम् प्र॒त्योष॑ति । \newline
73. प्र॒त्योष॑ति॒ न न प्र॒त्योष॑ति प्र॒त्योष॑ति॒ न । \newline
74. प्र॒त्योष॒तीति॑ प्रति - ओष॑ति । \newline
75. न विष्व॒ङ्॒. विष्व॒ङ् न न विष्वङ्ङ्॑ । \newline
76. विष्व॑ङ् प्र॒जया᳚ प्र॒जया॒ विष्व॒ङ्॒. विष्व॑ङ् प्र॒जया᳚ । \newline
77. प्र॒जया॑ प॒शुभिः॑ प॒शुभिः॑ प्र॒जया᳚ प्र॒जया॑ प॒शुभिः॑ । \newline
78. प्र॒जयेति॑ प्र - जया᳚ । \newline
79. प॒शुभि॑रेत्येति प॒शुभिः॑ प॒शुभि॑रेति । \newline
80. प॒शुभि॒रिति॑ प॒शु - भिः॒ । \newline
81. ए॒तीत्ये॑ति । \newline

\textbf{Ghana Paata } \newline

1. आहि॑ताग्ने रा॒शी रा॒शी राहि॑ताग्ने॒ राहि॑ताग्ने रा॒शीर् यद् यदा॒शी राहि॑ताग्ने॒ राहि॑ताग्ने रा॒शीर् यत् । \newline
2. आहि॑ताग्ने॒रित्याहि॑त - अ॒ग्नेः॒ । \newline
3. आ॒शीर् यद् यदा॒शी रा॒शीर् यद॒ग्नि म॒ग्निं ॅयदा॒शी रा॒शीर् यद॒ग्निम् । \newline
4. आ॒शीरित्या᳚ - शीः । \newline
5. यद॒ग्नि म॒ग्निं ॅयद् यद॒ग्नि मु॑प॒तिष्ठ॑त उप॒तिष्ठ॑ते॒ ऽग्निं ॅयद् यद॒ग्नि मु॑प॒तिष्ठ॑ते । \newline
6. अ॒ग्नि मु॑प॒तिष्ठ॑त उप॒तिष्ठ॑ते॒ ऽग्नि म॒ग्नि मु॑प॒तिष्ठ॑ते॒ तस्मा॒त् तस्मा॑दुप॒तिष्ठ॑ते॒ ऽग्नि म॒ग्नि मु॑प॒तिष्ठ॑ते॒ तस्मा᳚त् । \newline
7. उ॒प॒तिष्ठ॑ते॒ तस्मा॒त् तस्मा॑दुप॒तिष्ठ॑त उप॒तिष्ठ॑ते॒ तस्मा॑दुप॒स्थेय॑ उप॒स्थेय॒ स्तस्मा॑ दुप॒तिष्ठ॑त उप॒तिष्ठ॑ते॒ तस्मा॑दुप॒स्थेयः॑ । \newline
8. उ॒प॒तिष्ठ॑त॒ इत्यु॑प - तिष्ठ॑ते । \newline
9. तस्मा॑दुप॒स्थेय॑ उप॒स्थेय॒ स्तस्मा॒त् तस्मा॑दुप॒स्थेयः॑ प्र॒जाप॑तिः प्र॒जाप॑ति रुप॒स्थेय॒ स्तस्मा॒त् तस्मा॑दुप॒स्थेयः॑ प्र॒जाप॑तिः । \newline
10. उ॒प॒स्थेयः॑ प्र॒जाप॑तिः प्र॒जाप॑ति रुप॒स्थेय॑ उप॒स्थेयः॑ प्र॒जाप॑तिः प॒शून् प॒शून् प्र॒जाप॑ति रुप॒स्थेय॑ उप॒स्थेयः॑ प्र॒जाप॑तिः प॒शून् । \newline
11. उ॒प॒स्थेय॒ इत्यु॑प - स्थेयः॑ । \newline
12. प्र॒जाप॑तिः प॒शून् प॒शून् प्र॒जाप॑तिः प्र॒जाप॑तिः प॒शू न॑सृजतासृजत प॒शून् प्र॒जाप॑तिः प्र॒जाप॑तिः प॒शू न॑सृजत । \newline
13. प्र॒जाप॑ति॒रिति॑ प्र॒जा - प॒तिः॒ । \newline
14. प॒शू न॑सृजतासृजत प॒शून् प॒शू न॑सृजत॒ ते ते॑ ऽसृजत प॒शून् प॒शू न॑सृजत॒ ते । \newline
15. अ॒सृ॒ज॒त॒ ते ते॑ ऽसृजतासृजत॒ ते सृ॒ष्टाः सृ॒ष्टास्ते॑ ऽसृजतासृजत॒ ते सृ॒ष्टाः । \newline
16. ते सृ॒ष्टाः सृ॒ष्टास्ते ते सृ॒ष्टा अ॑होरा॒त्रे अ॑होरा॒त्रे सृ॒ष्टास्ते ते सृ॒ष्टा अ॑होरा॒त्रे । \newline
17. सृ॒ष्टा अ॑होरा॒त्रे अ॑होरा॒त्रे सृ॒ष्टाः सृ॒ष्टा अ॑होरा॒त्रे प्र प्राहो॑रा॒त्रे सृ॒ष्टाः सृ॒ष्टा अ॑होरा॒त्रे प्र । \newline
18. अ॒हो॒रा॒त्रे प्र प्राहो॑रा॒त्रे अ॑होरा॒त्रे प्रावि॑शन् नविश॒न् प्राहो॑रा॒त्रे अ॑होरा॒त्रे प्रावि॑शन्न् । \newline
19. अ॒हो॒रा॒त्रे इत्य॑हः - रा॒त्रे । \newline
20. प्रावि॑शन् नविश॒न् प्र प्रावि॑श॒न् ताꣳ स्ता न॑विश॒न् प्र प्रावि॑श॒न् तान् । \newline
21. अ॒वि॒श॒न् ताꣳ स्ता न॑विशन् नविश॒न् तान् छन्दो॑भि॒ श्छन्दो॑भि॒स्ता न॑विशन् नविश॒न् तान् छन्दो॑भिः । \newline
22. तान् छन्दो॑भि॒ श्छन्दो॑भि॒स्ताꣳ स्तान् छन्दो॑भि॒ रन्वनु॒ छन्दो॑भि॒स्ताꣳ स्तान् छन्दो॑भि॒रनु॑ । \newline
23. छन्दो॑भि॒ रन्वनु॒ छन्दो॑भि॒ श्छन्दो॑भि॒ रन्व॑विन्द दविन्द॒दनु॒ छन्दो॑भि॒ श्छन्दो॑भि॒ रन्व॑विन्दत् । \newline
24. छन्दो॑भि॒रिति॒ छन्दः॑ - भिः॒ । \newline
25. अन्व॑विन्द दविन्द॒ दन्वन्व॑विन्द॒द् यद् यद॑विन्द॒ दन्वन्व॑विन्द॒द् यत् । \newline
26. अ॒वि॒न्द॒द् यद् यद॑विन्ददविन्द॒द् यछ् चन्दो॑भि॒ श्छन्दो॑भि॒र् यद॑विन्द दविन्द॒द् यछ् चन्दो॑भिः । \newline
27. यछ् चन्दो॑भि॒ श्छन्दो॑भि॒र् यद् यछ् चन्दो॑भि रुप॒तिष्ठ॑त उप॒तिष्ठ॑ते॒ छन्दो॑भि॒र् यद् यछ् चन्दो॑भि रुप॒तिष्ठ॑ते । \newline
28. छन्दो॑भि रुप॒तिष्ठ॑त उप॒तिष्ठ॑ते॒ छन्दो॑भि॒ श्छन्दो॑भि रुप॒तिष्ठ॑ते॒ स्वꣳ स्व मु॑प॒तिष्ठ॑ते॒ छन्दो॑भि॒ श्छन्दो॑भि रुप॒तिष्ठ॑ते॒ स्वम् । \newline
29. छन्दो॑भि॒रिति॒ छन्दः॑ - भिः॒ । \newline
30. उ॒प॒तिष्ठ॑ते॒ स्वꣳ स्व मु॑प॒तिष्ठ॑त उप॒तिष्ठ॑ते॒ स्व मे॒वैव स्व मु॑प॒तिष्ठ॑त उप॒तिष्ठ॑ते॒ स्व मे॒व । \newline
31. उ॒प॒तिष्ठ॑त॒ इत्यु॑प - तिष्ठ॑ते । \newline
32. स्व मे॒वैव स्वꣳ स्व मे॒व तत् तदे॒व स्वꣳ स्व मे॒व तत् । \newline
33. ए॒व तत् तदे॒वैव तदन्वनु॒ तदे॒वैव तदनु॑ । \newline
34. तदन्वनु॒ तत् तदन्वि॑ च्छतीच्छ॒त्यनु॒ तत् तदन्वि॑च्छति । \newline
35. अन्वि॑ च्छतीच्छ॒ त्यन्वन्वि॑च्छति॒ न ने च्छ॒त्यन्वन्वि॑च्छति॒ न । \newline
36. इ॒च्छ॒ति॒ न ने च्छ॑तीच्छति॒ न तत्र॒ तत्र॒ ने च्छ॑तीच्छति॒ न तत्र॑ । \newline
37. न तत्र॒ तत्र॒ न न तत्र॑ जा॒मि जा॒मि तत्र॒ न न तत्र॑ जा॒मि । \newline
38. तत्र॑ जा॒मि जा॒मि तत्र॒ तत्र॑ जा॒म्य॑ स्त्यस्ति जा॒मि तत्र॒ तत्र॑ जा॒म्य॑स्ति । \newline
39. जा॒म्य॑ स्त्यस्ति जा॒मि जा॒म्य॑ स्तीतीत्य॑स्ति जा॒मि जा॒म्य॑ स्तीति॑ । \newline
40. अ॒स्तीतीत्य॑ स्त्य॒स्तीत्या॑हु राहु॒ रित्य॑ स्त्य॒स्तीत्या॑हुः । \newline
41. इत्या॑हु राहु॒ रितीत्या॑हु॒र् यो य आ॑हु॒ रितीत्या॑हु॒र् यः । \newline
42. आ॒हु॒र् यो य आ॑हुराहु॒र् यो ऽह॑रह॒ रह॑रह॒र् य आ॑हुराहु॒र् यो ऽह॑रहः । \newline
43. यो ऽह॑रह॒ रह॑रह॒र् यो यो ऽह॑रह रुप॒तिष्ठ॑त उप॒तिष्ठ॒ते ऽह॑रह॒र् यो यो ऽह॑रह रुप॒तिष्ठ॑ते । \newline
44. अह॑रह रुप॒तिष्ठ॑त उप॒तिष्ठ॒ते ऽह॑रह॒ रह॑रह रुप॒तिष्ठ॑त॒ इतीत्यु॑प॒तिष्ठ॒ते ऽह॑रह॒ रह॑रह रुप॒तिष्ठ॑त॒ इति॑ । \newline
45. अह॑रह॒रित्यहः॑ - अ॒हः॒ । \newline
46. उ॒प॒तिष्ठ॑त॒ इतीत्यु॑प॒तिष्ठ॑त उप॒तिष्ठ॑त॒ इति॒ यो य इत्यु॑प॒तिष्ठ॑त उप॒तिष्ठ॑त॒ इति॒ यः । \newline
47. उ॒प॒तिष्ठ॑त॒ इत्यु॑प - तिष्ठ॑ते । \newline
48. इति॒ यो य इतीति॒ यो वै वै य इतीति॒ यो वै । \newline
49. यो वै वै यो यो वा अ॒ग्नि म॒ग्निं ॅवै यो यो वा अ॒ग्निम् । \newline
50. वा अ॒ग्नि म॒ग्निं ॅवै वा अ॒ग्निम् प्र॒त्यङ् प्र॒त्यङ् ङ॒ग्निं ॅवै वा अ॒ग्निम् प्र॒त्यङ्ङ् । \newline
51. अ॒ग्निम् प्र॒त्यङ् प्र॒त्यङ् ङ॒ग्नि म॒ग्निम् प्र॒त्यङ् ङु॑प॒तिष्ठ॑त उप॒तिष्ठ॑ते प्र॒त्यङ्ङ॒ग्नि म॒ग्निम् 
प्र॒त्यङ् ङु॑प॒तिष्ठ॑ते । \newline
52. प्र॒त्यङ् ङु॑प॒तिष्ठ॑त उप॒तिष्ठ॑ते प्र॒त्यङ् प्र॒त्यङ् ङु॑प॒तिष्ठ॑ते॒ प्रति॒ प्रत्यु॑प॒तिष्ठ॑ते 
प्र॒त्यङ् प्र॒त्यङ् ङु॑प॒तिष्ठ॑ते॒ प्रति॑ । \newline
53. उ॒प॒तिष्ठ॑ते॒ प्रति॒ प्रत्यु॑प॒तिष्ठ॑त उप॒तिष्ठ॑ते॒ प्रत्ये॑न मेन॒म् प्रत्यु॑प॒तिष्ठ॑त उप॒तिष्ठ॑ते॒ प्रत्ये॑नम् । \newline
54. उ॒प॒तिष्ठ॑त॒ इत्यु॑प - तिष्ठ॑ते । \newline
55. प्रत्ये॑न मेन॒म् प्रति॒ प्रत्ये॑न मोषत्योषत्येन॒म् प्रति॒ प्रत्ये॑न मोषति । \newline
56. ए॒न॒ मो॒ष॒त्यो॒ष॒त्ये॒न॒ मे॒न॒ मो॒ष॒ति॒ यो य ओ॑षत्येन मेन मोषति॒ यः । \newline
57. ओ॒ष॒ति॒ यो य ओ॑षत्योषति॒ यः परा॒ङ् परा॒ङ् य ओ॑षत्योषति॒ यः पराङ्॑ । \newline
58. यः परा॒ङ् परा॒ङ् यो यः परा॒ङ् विष्व॒ङ्.॒ विष्व॒ङ् परा॒ङ् यो यः परा॒ङ् विष्वङ्ङ्॑ । \newline
59. परा॒ङ् विष्व॒ङ्॒. विष्व॒ङ् परा॒ङ् परा॒ङ् विष्व॑ङ् प्र॒जया᳚ प्र॒जया॒ विष्व॒ङ् परा॒ङ् परा॒ङ् विष्व॑ङ् प्र॒जया᳚ । \newline
60. विष्व॑ङ् प्र॒जया᳚ प्र॒जया॒ विष्व॒ङ्॒. विष्व॑ङ् प्र॒जया॑ प॒शुभिः॑ प॒शुभिः॑ प्र॒जया॒ विष्व॒ङ्॒. विष्व॑ङ् प्र॒जया॑ प॒शुभिः॑ । \newline
61. प्र॒जया॑ प॒शुभिः॑ प॒शुभिः॑ प्र॒जया᳚ प्र॒जया॑ प॒शुभि॑ रेत्येति प॒शुभिः॑ प्र॒जया᳚ प्र॒जया॑ प॒शुभि॑रेति । \newline
62. प्र॒जयेति॑ प्र - जया᳚ । \newline
63. प॒शुभि॑रेत्येति प॒शुभिः॑ प॒शुभि॑रेति॒ कवा॑तिर्य॒ङ् कवा॑तिर्यङ् ङेति प॒शुभिः॑ प॒शुभि॑रेति॒ कवा॑तिर्यङ्ङ् । \newline
64. प॒शुभि॒रिति॑ प॒शु - भिः॒ । \newline
65. ए॒ति॒ कवा॑तिर्य॒ङ् कवा॑तिर्यङ् ङेत्येति॒ कवा॑तिर्यङ् ङिवे व॒ कवा॑तिर्यङ् ङेत्येति॒ कवा॑तिर्यङ् ङिव । \newline
66. कवा॑तिर्यङ् ङिवे व॒ कवा॑तिर्य॒ङ् कवा॑तिर्यङ् ङि॒वोपोपे॑ व॒ कवा॑तिर्य॒ङ् कवा॑तिर्यङ् ङि॒वोप॑ । \newline
67. कवा॑तिर्य॒ङ्ङिति॒ कवा᳚ - ति॒र्य॒ङ्ङ्॒ । \newline
68. इ॒वोपोपे॑ वे॒ वोप॑ तिष्ठेत तिष्ठे॒तोपे॑ वे॒ वोप॑ तिष्ठेत । \newline
69. उप॑ तिष्ठेत तिष्ठे॒तोपोप॑ तिष्ठेत॒ न न ति॑ष्ठे॒तोपोप॑ तिष्ठेत॒ न । \newline
70. ति॒ष्ठे॒त॒ न न ति॑ष्ठेत तिष्ठेत॒ नैन॑ मेन॒न्न ति॑ष्ठेत तिष्ठेत॒ नैन᳚म् । \newline
71. नैन॑ मेन॒न्न नैन॑म् प्र॒त्योष॑ति प्र॒त्योष॑त्येन॒न्न नैन॑म् प्र॒त्योष॑ति । \newline
72. ए॒न॒म् प्र॒त्योष॑ति प्र॒त्योष॑त्येन मेनम् प्र॒त्योष॑ति॒ न न प्र॒त्योष॑त्येन मेनम् प्र॒त्योष॑ति॒ न । \newline
73. प्र॒त्योष॑ति॒ न न प्र॒त्योष॑ति प्र॒त्योष॑ति॒ न विष्व॒ङ्॒. विष्व॒ङ् न प्र॒त्योष॑ति प्र॒त्योष॑ति॒ न विष्वङ्ङ्॑ । \newline
74. प्र॒त्योष॒तीति॑ प्रति - ओष॑ति । \newline
75. न विष्व॒ङ्॒. विष्व॒ङ् न न विष्व॑ङ् प्र॒जया᳚ प्र॒जया॒ विष्व॒ङ् न न विष्व॑ङ् प्र॒जया᳚ । \newline
76. विष्व॑ङ् प्र॒जया᳚ प्र॒जया॒ विष्व॒ङ्॒. विष्व॑ङ् प्र॒जया॑ प॒शुभिः॑ प॒शुभिः॑ प्र॒जया॒ विष्व॒ङ्॒. विष्व॑ङ् प्र॒जया॑ प॒शुभिः॑ । \newline
77. प्र॒जया॑ प॒शुभिः॑ प॒शुभिः॑ प्र॒जया᳚ प्र॒जया॑ प॒शुभि॑ रेत्येति प॒शुभिः॑ प्र॒जया᳚ प्र॒जया॑ प॒शुभि॑रेति । \newline
78. प्र॒जयेति॑ प्र - जया᳚ । \newline
79. प॒शुभि॑ रेत्येति प॒शुभिः॑ प॒शुभि॑रेति । \newline
80. प॒शुभि॒रिति॑ प॒शु - भिः॒ । \newline
81. ए॒तीत्ये॑ति । \newline
\pagebreak
\markright{ TS 1.5.10.1  \hfill https://www.vedavms.in \hfill}
\addcontentsline{toc}{section}{ TS 1.5.10.1 }
\section*{ TS 1.5.10.1 }

\textbf{TS 1.5.10.1 } \newline
\textbf{Samhita Paata} \newline

मम॒ नाम॑ प्रथ॒मं जा॑तवेदः पि॒ता मा॒ता च॑ दधतु॒र्यदग्रे᳚ । तत्त्वं बि॑भृहि॒ पुन॒रा मदैतो॒स्तवा॒हं नाम॑ बिभराण्यग्ने ॥ मम॒ नाम॒ तव॑ च जातवेदो॒ वास॑सी इव वि॒वसा॑नौ॒ ये चरा॑वः । आयु॑षे॒ त्वं जी॒वसे॑ व॒यं ॅय॑थाय॒थं ॅवि परि॑ दधावहै॒ पुन॒स्ते ॥ नमो॒ऽग्नये ऽप्र॑तिविद्धाय॒ नमोऽना॑धृष्टाय॒ नमः॑ स॒म्राजे᳚ । अषा॑ढो - [ ] \newline

\textbf{Pada Paata} \newline

मम॑ । नाम॑ । प्र॒थ॒मम् । जा॒त॒वे॒द॒ इति॑ जात - वे॒दः॒ । पि॒ता । मा॒ता । च॒ । द॒ध॒तुः॒ । यत् । अग्रे᳚ ॥ तत् । त्वम् । बि॒भृ॒हि॒ । पुनः॑ । एति॑ । मत् । ऐतो॒रित्या - ए॒तोः॒ । तव॑ । अ॒हम् । नाम॑ । बि॒भ॒रा॒णि॒ । अ॒ग्ने॒ ॥ मम॑ । नाम॑ । तव॑ । च॒ । जा॒त॒वे॒द॒ इति॑ जात - वे॒दः॒ । वास॑सी॒ इति॑ । इ॒व॒ । वि॒वसा॑ना॒विति॑ वि-वसा॑नौ । ये इति॑ । चरा॑वः ॥ आयु॑षे । त्वम् । जी॒वसे᳚ । व॒यम् । य॒था॒य॒थमिति॑ यथा-य॒थम् । वि । परीति॑ । द॒धा॒व॒है॒ । पुनः॑ । ते इति॑ ॥ नमः॑ । अ॒ग्नये᳚ । अप्र॑तिविद्धा॒येत्यप्र॑ति - वि॒द्धा॒य॒ । नमः॑ । अना॑धृष्टा॒येत्यना᳚ - धृ॒ष्टा॒य॒ । नमः॑ । स॒म्राज॒ इति॑ सं - राजे᳚ ॥ अषा॑ढः ।  \newline


\textbf{Krama Paata} \newline

मम॒ नाम॑ । नाम॑ प्रथ॒मम् । प्र॒थ॒मम् जा॑तवेदः । जा॒त॒वे॒दः॒ पि॒ता । जा॒त॒वे॒द॒ इति॑ जात - वे॒दः॒ । पि॒ता मा॒ता । मा॒ता च॑ । च॒ द॒ध॒तुः॒ । द॒ध॒तु॒र् यत् । यदग्रे᳚ । अग्र॒ इत्यग्रे᳚ ॥ तत्,त्वम् । त्वम् बि॑भृहि । बि॒भृ॒हि॒ पुनः॑ । पुन॒रा । आ मत् । मदैतोः᳚ । ऐतो॒ स्तव॑ । ऐतो॒रित्या - ए॒तोः॒ । तवा॒हम् । अ॒हम् नाम॑ । नाम॑ बिभराणि । बि॒भ॒रा॒ण्य॒ग्ने॒ । अ॒ग्न॒ इत्य॑ग्ने ॥ मम॒ नाम॑ । नाम॒ तव॑ । तव॑ च । च॒ जा॒त॒वे॒दः॒ । जा॒त॒वे॒दो॒ वास॑सी । जा॒त॒वे॒द॒ इति॑ जात - वे॒दः॒ । वास॑सी इव । वास॑सी॒ इति॒ वास॑सी । इ॒व॒ वि॒वसा॑नौ । वि॒वसा॑नौ॒ ये । वि॒वसा॑ना॒विति॑ वि - वसा॑नौ । ये चरा॑वः । ये इति॒ ये । चरा॑व॒ इति॒ चरा॑वः ॥ आयु॑षे॒ त्वम् । त्वम् जी॒वसे᳚ । जी॒वसे॑ व॒यम् । व॒यं ॅय॑थाय॒थम् । य॒था॒य॒थं ॅवि । य॒था॒य॒थमिति॑ यथा - य॒थम् । वि परि॑ । परि॑ दधावहै । द॒धा॒व॒है॒ पुनः॑ । पुन॒स्ते । ते इति॒ ते ॥ नमो॒ऽग्नये᳚ । अ॒ग्नयेऽप्र॑तिविद्धाय । अप्र॑तिविद्धाय॒ नमः॑ । अप्र॑तिविद्धा॒येत्यप्र॑ति - वि॒द्धा॒य॒ । नमोऽना॑धृष्टाय । अना॑धृष्टाय॒ नमः॑ । अना॑धृष्टा॒येत्यना᳚ - धृ॒ष्टा॒य॒ । नमः॑ स॒म्राजे᳚ । स॒म्राज॒ इति॑ सं - राजे᳚ ॥ अषा॑ढो अ॒ग्निः \newline

\textbf{Jatai Paata} \newline

1. मम॒ नाम॒ नाम॒ मम॒ मम॒ नाम॑ । \newline
2. नाम॑ प्रथ॒मम् प्र॑थ॒मन्नाम॒ नाम॑ प्रथ॒मम् । \newline
3. प्र॒थ॒मम् जा॑तवेदो जातवेदः प्रथ॒मम् प्र॑थ॒मम् जा॑तवेदः । \newline
4. जा॒त॒वे॒दः॒ पि॒ता पि॒ता जा॑तवेदो जातवेदः पि॒ता । \newline
5. जा॒त॒वे॒द॒ इति॑ जात - वे॒दः॒ । \newline
6. पि॒ता मा॒ता मा॒ता पि॒ता पि॒ता मा॒ता । \newline
7. मा॒ता च॑ च मा॒ता मा॒ता च॑ । \newline
8. च॒ द॒ध॒तु॒र् द॒ध॒तु॒श्च॒ च॒ द॒ध॒तुः॒ । \newline
9. द॒ध॒तु॒र् यद् यद् द॑धतुर् दधतु॒र् यत् । \newline
10. यदग्रे ऽग्रे॒ यद् यदग्रे᳚ । \newline
11. अग्र॒ इत्यग्रे᳚ । \newline
12. तत् त्वम् त्वम् तत् तत् त्वम् । \newline
13. त्वम् बि॑भृहि बिभृहि॒ त्वम् त्वम् बि॑भृहि । \newline
14. बि॒भृ॒हि॒ पुनः॒ पुन॑र् बिभृहि बिभृहि॒ पुनः॑ । \newline
15. पुन॒ रा पुनः॒ पुन॒ रा । \newline
16. आ मन् मदा मत् । \newline
17. मदैतो॒रैतो॒र् मन् मदैतोः᳚ । \newline
18. ऐतो॒ स्तव॒ तवै॑तो॒ रैतो॒स्तव॑ । \newline
19. ऐतो॒रित्या - ए॒तोः॒ । \newline
20. तवा॒ह म॒हम् तव॒ तवा॒हम् । \newline
21. अ॒हन्नाम॒ नामा॒ह म॒हन्नाम॑ । \newline
22. नाम॑ बिभराणि बिभराणि॒ नाम॒ नाम॑ बिभराणि । \newline
23. बि॒भ॒रा॒ण्य॒ग्ने॒ ऽग्ने॒ बि॒भ॒रा॒णि॒ बि॒भ॒रा॒ण्य॒ग्ने॒ । \newline
24. अ॒ग्न॒ इत्य॑ग्ने । \newline
25. मम॒ नाम॒ नाम॒ मम॒ मम॒ नाम॑ । \newline
26. नाम॒ तव॒ तव॒ नाम॒ नाम॒ तव॑ । \newline
27. तव॑ च च॒ तव॒ तव॑ च । \newline
28. च॒ जा॒त॒वे॒दो॒ जा॒त॒वे॒द॒श्च॒ च॒ जा॒त॒वे॒दः॒ । \newline
29. जा॒त॒वे॒दो॒ वास॑सी॒ वास॑सी जातवेदो जातवेदो॒ वास॑सी । \newline
30. जा॒त॒वे॒द॒ इति॑ जात - वे॒दः॒ । \newline
31. वास॑सी इवे व॒ वास॑सी॒ वास॑सी इव । \newline
32. वास॑सी॒ इति॒ वास॑सी । \newline
33. इ॒व॒ वि॒वसा॑नौ वि॒वसा॑ना विवे व वि॒वसा॑नौ । \newline
34. वि॒वसा॑नौ॒ ये ये वि॒वसा॑नौ वि॒वसा॑नौ॒ ये । \newline
35. वि॒वसा॑ना॒विति॑ वि - वसा॑नौ । \newline
36. ये चरा॑व॒ श्चरा॑वो॒ ये ये चरा॑वः । \newline
37. ये इति॒ ये । \newline
38. चरा॑व॒ इति॒ चरा॑वः । \newline
39. आयु॑षे॒ त्वम् त्व मायु॑ष॒ आयु॑षे॒ त्वम् । \newline
40. त्वम् जी॒वसे॑ जी॒वसे॒ त्वम् त्वम् जी॒वसे᳚ । \newline
41. जी॒वसे॑ व॒यं ॅव॒यम् जी॒वसे॑ जी॒वसे॑ व॒यम् । \newline
42. व॒यं ॅय॑थाय॒थं ॅय॑थाय॒थं ॅव॒यं ॅव॒यं ॅय॑थाय॒थम् । \newline
43. य॒था॒य॒थं ॅवि वि य॑थाय॒थं ॅय॑थाय॒थं ॅवि । \newline
44. य॒था॒य॒थमिति॑ यथा - य॒थम् । \newline
45. वि परि॒ परि॒ वि वि परि॑ । \newline
46. परि॑ दधावहै दधावहै॒ परि॒ परि॑ दधावहै । \newline
47. द॒धा॒व॒है॒ पुनः॒ पुन॑र् दधावहै दधावहै॒ पुनः॑ । \newline
48. पुन॒स्ते ते पुनः॒ पुन॒स्ते । \newline
49. ते इति॒ ते । \newline
50. नमो॒ ऽग्नये॒ ऽग्नये॒ नमो॒ नमो॒ ऽग्नये᳚ । \newline
51. अ॒ग्नये ऽप्र॑तिविद्धा॒याप्र॑तिविद्धाया॒ग्नये॒ ऽग्नये ऽप्र॑तिविद्धाय । \newline
52. अप्र॑तिविद्धाय॒ नमो॒ नमो ऽप्र॑तिविद्धा॒याप्र॑तिविद्धाय॒ नमः॑ । \newline
53. अप्र॑तिविद्धा॒येत्यप्र॑ति - वि॒द्धा॒य॒ । \newline
54. नमो ऽना॑धृष्टा॒याना॑धृष्टाय॒ नमो॒ नमो ऽना॑धृष्टाय । \newline
55. अना॑धृष्टाय॒ नमो॒ नमो ऽना॑धृष्टा॒याना॑धृष्टाय॒ नमः॑ । \newline
56. अना॑धृष्टा॒येत्यना᳚ - धृ॒ष्टा॒य॒ । \newline
57. नमः॑ स॒म्राजे॑ स॒म्राजे॒ नमो॒ नमः॑ स॒म्राजे᳚ । \newline
58. स॒म्राज॒ इति॑ सं - राजे᳚ । \newline
59. अषा॑ढो अ॒ग्नि र॒ग्निरषा॑ढो॒ अषा॑ढो अ॒ग्निः । \newline

\textbf{Ghana Paata } \newline

1. मम॒ नाम॒ नाम॒ मम॒ मम॒ नाम॑ प्रथ॒मम् प्र॑थ॒मन्नाम॒ मम॒ मम॒ नाम॑ प्रथ॒मम् । \newline
2. नाम॑ प्रथ॒मम् प्र॑थ॒मन्नाम॒ नाम॑ प्रथ॒मम् जा॑तवेदो जातवेदः प्रथ॒मन्नाम॒ नाम॑ प्रथ॒मम् जा॑तवेदः । \newline
3. प्र॒थ॒मम् जा॑तवेदो जातवेदः प्रथ॒मम् प्र॑थ॒मम् जा॑तवेदः पि॒ता पि॒ता जा॑तवेदः प्रथ॒मम् प्र॑थ॒मम् जा॑तवेदः पि॒ता । \newline
4. जा॒त॒वे॒दः॒ पि॒ता पि॒ता जा॑तवेदो जातवेदः पि॒ता मा॒ता मा॒ता पि॒ता जा॑तवेदो जातवेदः पि॒ता मा॒ता । \newline
5. जा॒त॒वे॒द॒ इति॑ जात - वे॒दः॒ । \newline
6. पि॒ता मा॒ता मा॒ता पि॒ता पि॒ता मा॒ता च॑ च मा॒ता पि॒ता पि॒ता मा॒ता च॑ । \newline
7. मा॒ता च॑ च मा॒ता मा॒ता च॑ दधतुर् दधतुश्च मा॒ता मा॒ता च॑ दधतुः । \newline
8. च॒ द॒ध॒तु॒र् द॒ध॒तु॒श्च॒ च॒ द॒ध॒तु॒र् यद् यद् द॑धतुश्च च दधतु॒र् यत् । \newline
9. द॒ध॒तु॒र् यद् यद् द॑धतुर् दधतु॒र् यदग्रे ऽग्रे॒ यद् द॑धतुर् दधतु॒र् यदग्रे᳚ । \newline
10. यदग्रे ऽग्रे॒ यद् यदग्रे᳚ । \newline
11. अग्र॒ इत्यग्रे᳚ । \newline
12. तत् त्वम् त्वम् तत् तत् त्वम् बि॑भृहि बिभृहि॒ त्वम् तत् तत् त्वम् बि॑भृहि । \newline
13. त्वम् बि॑भृहि बिभृहि॒ त्वम् त्वम् बि॑भृहि॒ पुनः॒ पुन॑र् बिभृहि॒ त्वम् त्वम् बि॑भृहि॒ पुनः॑ । \newline
14. बि॒भृ॒हि॒ पुनः॒ पुन॑र् बिभृहि बिभृहि॒ पुन॒ रा पुन॑र् बिभृहि बिभृहि॒ पुन॒ रा । \newline
15. पुन॒ रा पुनः॒ पुन॒ रा मन् मदा पुनः॒ पुन॒ रा मत् । \newline
16. आ मन् मदा मदैतो॒ रैतो॒र् मदा मदैतोः᳚ । \newline
17. मदैतो॒ रैतो॒र् मन् मदैतो॒ स्तव॒ तवै॑तो॒र् मन् मदैतो॒स्तव॑ । \newline
18. ऐतो॒स्तव॒ तवै॑तो॒ रैतो॒ स्तवा॒ह म॒हम् तवै॑तो॒ रैतो॒ स्तवा॒हम् । \newline
19. ऐतो॒रित्या - ए॒तोः॒ । \newline
20. तवा॒ह म॒हम् तव॒ तवा॒हन्नाम॒ नामा॒हम् तव॒ तवा॒हन्नाम॑ । \newline
21. अ॒हन्नाम॒ नामा॒ह म॒हन्नाम॑ बिभराणि बिभराणि॒ नामा॒ह म॒हन्नाम॑ बिभराणि । \newline
22. नाम॑ बिभराणि बिभराणि॒ नाम॒ नाम॑ बिभराण्यग्ने ऽग्ने बिभराणि॒ नाम॒ नाम॑ बिभराण्यग्ने । \newline
23. बि॒भ॒रा॒ण्य॒ग्ने॒ ऽग्ने॒ बि॒भ॒रा॒णि॒ बि॒भ॒रा॒ण्य॒ग्ने॒ । \newline
24. अ॒ग्न॒ इत्य॑ग्ने । \newline
25. मम॒ नाम॒ नाम॒ मम॒ मम॒ नाम॒ तव॒ तव॒ नाम॒ मम॒ मम॒ नाम॒ तव॑ । \newline
26. नाम॒ तव॒ तव॒ नाम॒ नाम॒ तव॑ च च॒ तव॒ नाम॒ नाम॒ तव॑ च । \newline
27. तव॑ च च॒ तव॒ तव॑ च जातवेदो जातवेदश्च॒ तव॒ तव॑ च जातवेदः । \newline
28. च॒ जा॒त॒वे॒दो॒ जा॒त॒वे॒द॒श्च॒ च॒ जा॒त॒वे॒दो॒ वास॑सी॒ वास॑सी जातवेदश्च च जातवेदो॒ वास॑सी । \newline
29. जा॒त॒वे॒दो॒ वास॑सी॒ वास॑सी जातवेदो जातवेदो॒ वास॑सी इवे व॒ वास॑सी जातवेदो जातवेदो॒ वास॑सी इव । \newline
30. जा॒त॒वे॒द॒ इति॑ जात - वे॒दः॒ । \newline
31. वास॑सी इवे व॒ वास॑सी॒ वास॑सी इव वि॒वसा॑नौ वि॒वसा॑ना विव॒ वास॑सी॒ वास॑सी इव वि॒वसा॑नौ । \newline
32. वास॑सी॒ इति॒ वास॑सी । \newline
33. इ॒व॒ वि॒वसा॑नौ वि॒वसा॑ना विवे व वि॒वसा॑नौ॒ ये ये वि॒वसा॑ना विवे व वि॒वसा॑नौ॒ ये । \newline
34. वि॒वसा॑नौ॒ ये ये वि॒वसा॑नौ वि॒वसा॑नौ॒ ये चरा॑व॒ श्चरा॑वो॒ ये वि॒वसा॑नौ वि॒वसा॑नौ॒ ये चरा॑वः । \newline
35. वि॒वसा॑ना॒विति॑ वि - वसा॑नौ । \newline
36. ये चरा॑व॒ श्चरा॑वो॒ ये ये चरा॑वः । \newline
37. ये इति॒ ये । \newline
38. चरा॑व॒ इति॒ चरा॑वः । \newline
39. आयु॑षे॒ त्वम् त्व मायु॑ष॒ आयु॑षे॒ त्वम् जी॒वसे॑ जी॒वसे॒ त्व मायु॑ष॒ आयु॑षे॒ त्वम् जी॒वसे᳚ । \newline
40. त्वम् जी॒वसे॑ जी॒वसे॒ त्वम् त्वम् जी॒वसे॑ व॒यं ॅव॒यम् जी॒वसे॒ त्वम् त्वम् जी॒वसे॑ व॒यम् । \newline
41. जी॒वसे॑ व॒यं ॅव॒यम् जी॒वसे॑ जी॒वसे॑ व॒यं ॅय॑थाय॒थं ॅय॑थाय॒थं ॅव॒यम् जी॒वसे॑ जी॒वसे॑ व॒यं ॅय॑थाय॒थम् । \newline
42. व॒यं ॅय॑थाय॒थं ॅय॑थाय॒थं ॅव॒यं ॅव॒यं ॅय॑थाय॒थं ॅवि वि य॑थाय॒थं ॅव॒यं ॅव॒यं ॅय॑थाय॒थं ॅवि । \newline
43. य॒था॒य॒थं ॅवि वि य॑थाय॒थं ॅय॑थाय॒थं ॅवि परि॒ परि॒ वि य॑थाय॒थं ॅय॑थाय॒थं ॅवि परि॑ । \newline
44. य॒था॒य॒थमिति॑ यथा - य॒थम् । \newline
45. वि परि॒ परि॒ वि वि परि॑ दधावहै दधावहै॒ परि॒ वि वि परि॑ दधावहै । \newline
46. परि॑ दधावहै दधावहै॒ परि॒ परि॑ दधावहै॒ पुनः॒ पुन॑र् दधावहै॒ परि॒ परि॑ दधावहै॒ पुनः॑ । \newline
47. द॒धा॒व॒है॒ पुनः॒ पुन॑र् दधावहै दधावहै॒ पुन॒स्ते ते पुन॑र् दधावहै दधावहै॒ पुन॒स्ते । \newline
48. पुन॒स्ते ते पुनः॒ पुन॒स्ते । \newline
49. ते इति॒ ते । \newline
50. नमो॒ ऽग्नये॒ ऽग्नये॒ नमो॒ नमो॒ ऽग्नये ऽप्र॑तिविद्धा॒या प्र॑तिविद्धाया॒ग्नये॒ नमो॒ नमो॒ ऽग्नये ऽप्र॑तिविद्धाय । \newline
51. अ॒ग्नये ऽप्र॑तिविद्धा॒या प्र॑तिविद्धाया॒ग्नये॒ ऽग्नये ऽप्र॑तिविद्धाय॒ नमो॒ नमो ऽप्र॑तिविद्धाया॒ग्नये॒ ऽग्नये ऽप्र॑तिविद्धाय॒ नमः॑ । \newline
52. अप्र॑तिविद्धाय॒ नमो॒ नमो ऽप्र॑तिविद्धा॒या प्र॑तिविद्धाय॒ नमो ऽना॑धृष्टा॒याना॑धृष्टाय॒ नमो ऽप्र॑तिविद्धा॒या प्र॑तिविद्धाय॒ नमो ऽना॑धृष्टाय । \newline
53. अप्र॑तिविद्धा॒येत्यप्र॑ति - वि॒द्धा॒य॒ । \newline
54. नमो ऽना॑धृष्टा॒याना॑धृष्टाय॒ नमो॒ नमो ऽना॑धृष्टाय॒ नमो॒ नमो ऽना॑धृष्टाय॒ नमो॒ नमो ऽना॑धृष्टाय॒ नमः॑ । \newline
55. अना॑धृष्टाय॒ नमो॒ नमो ऽना॑धृष्टा॒याना॑धृष्टाय॒ नमः॑ स॒म्राजे॑ स॒म्राजे॒ नमो ऽना॑धृष्टा॒याना॑धृष्टाय॒ नमः॑ स॒म्राजे᳚ । \newline
56. अना॑धृष्टा॒येत्यना᳚ - धृ॒ष्टा॒य॒ । \newline
57. नमः॑ स॒म्राजे॑ स॒म्राजे॒ नमो॒ नमः॑ स॒म्राजे᳚ । \newline
58. स॒म्राज॒ इति॑ सं - राजे᳚ । \newline
59. अषा॑ढो अ॒ग्निर॒ग्नि रषा॑ढो॒ अषा॑ढो अ॒ग्निर् बृ॒हद्व॑या बृ॒हद्व॑या अ॒ग्निरषा॑ढो॒ अषा॑ढो अ॒ग्निर् बृ॒हद्व॑याः । \newline
\pagebreak
\markright{ TS 1.5.10.2  \hfill https://www.vedavms.in \hfill}
\addcontentsline{toc}{section}{ TS 1.5.10.2 }
\section*{ TS 1.5.10.2 }

\textbf{TS 1.5.10.2 } \newline
\textbf{Samhita Paata} \newline

अ॒ग्निर्बृ॒हद्व॑या विश्व॒जिथ् सह॑न्त्यः॒ श्रेष्ठो॑ गन्ध॒र्वः । त्वत्पि॑तारो अग्ने दे॒वा-स्त्वामा॑हुतय॒-स्त्वद्वि॑वाचनाः । सं मामायु॑षा॒ सं गौ॑प॒त्येन॒ सुहि॑ते मा धाः ॥ अ॒यम॒ग्निः श्रेष्ठ॑तमो॒ ऽयं भग॑वत्तमो॒ ऽयꣳ स॑हस्र॒सात॑मः । अ॒स्मा अ॑स्तु सु॒वीर्यं᳚ ॥ मनो॒ ज्योति॑र् जुषता॒माज्यं॒ ॅविच्छि॑न्नं ॅय॒ज्ञ्ꣳ समि॒मं द॑धातु । या इ॒ष्टा उ॒षसो॑ नि॒म्रुच॑श्च॒ ताः सं द॑धामि ह॒विषा॑ घृ॒तेन॑ ॥ पय॑स्वती॒रोष॑धयः॒ - [ ] \newline

\textbf{Pada Paata} \newline

अ॒ग्निः । बृ॒हद्व॑या॒ इति॑ बृ॒हत् - व॒याः॒ । वि॒श्व॒जिदिति॑ विश्व - जित् । सह॑न्त्यः । श्रेष्ठः॑ । ग॒न्ध॒र्वः ॥ त्वत्पि॑तार॒ इति॒ त्वत्-पि॒ता॒रः॒ । अ॒ग्ने॒ । दे॒वाः । त्वामा॑हुतय॒ इति॒ त्वाम् - आ॒हु॒त॒यः॒ । त्वद्वि॑वाचना॒ इति॒ त्वत् - वि॒वा॒च॒नाः॒ ॥ समिति॑ । माम् । आयु॑षा । समिति॑ । गौ॒प॒त्येन॑ । सुहि॑त॒ इति॒ सु - हि॒ते॒ । मा॒ । धाः॒ ॥ अ॒यम् । अ॒ग्निः । श्रेष्ठ॑तम॒ इति॒ श्रेष्ठ॑ - त॒मः॒ । अ॒यम् । भग॑वत्तम॒ इति॒ भग॑वत्-त॒मः॒ । अ॒यम् । स॒ह॒स्र॒सात॑म॒ इति॑ सहस्र - सात॑मः ॥ अ॒स्मै । अ॒स्तु॒ । सु॒वीर्य॒मिति॑ सु - वीर्य᳚म् ॥ मनः॑ । ज्योतिः॑ । जु॒ष॒ता॒म् । आज्य᳚म् । विच्छि॑न्न॒मिति॒ वि - छि॒न्न॒म् । य॒ज्ञ्म् । समिति॑ । इ॒मम् । द॒धा॒तु॒ ॥ याः । इ॒ष्टाः । उ॒षसः॑ । नि॒म्रुच॒ इति॑ नि-म्रुचः॑ । च॒ । ताः । समिति॑ । द॒धा॒मि॒ । ह॒विषा᳚ । घृ॒तेन॑ ॥ पय॑स्वतीः । ओष॑धयः ।  \newline


\textbf{Krama Paata} \newline

अ॒ग्निर्,बृ॒हद्व॑याः । बृ॒हद्व॑या विश्व॒जित् । बृ॒हद्व॑या॒ इति॑ बृ॒हत् - व॒याः॒ । वि॒श्व॒जिथ् सह॑न्त्यः । वि॒श्व॒जिदिति॑ विश्व - जित् । सह॑न्त्यः॒ श्रेष्ठः॑ । श्रेष्ठो॑ गन्ध॒र्वः । ग॒न्ध॒र्व इति॑ गन्ध॒र्वः ॥ त्वत्पि॑तारो अग्ने । त्वत्पि॑तार॒ इति॒ त्वत् - पि॒ता॒रः॒ । अ॒ग्ने॒ दे॒वाः । दे॒वास्त्वामा॑हुतयः । त्वामा॑हुतय॒,स्त्वद्वि॑वाचनाः । त्वामा॑हुतय॒ इति॒ त्वाम् - आ॒हु॒त॒यः॒ । त्वद्वि॑वाचना॒ इति॒ त्वत् - वि॒वा॒च॒नाः॒ ॥ सम् माम् । मामायु॑षा । आयु॑षा॒ सम् । सम् गौ॑प॒त्येन॑ । गौ॒प॒त्येन॒ सुहि॑ते । सुहि॑ते मा । सुहि॑त॒ इति॒ सु - हि॒ते॒ । मा॒ धाः॒ । धा॒ इति॑ धाः ॥ अ॒यम॒ग्निः । अ॒ग्निः श्रेष्ठ॑तमः । श्रेष्ठ॑तमो॒ऽयम् । श्रेष्ठ॑तम॒ इति॒ श्रेष्ठ॑ - त॒मः॒ । अ॒यम् भग॑वत्तमः । भग॑वत्तमो॒ऽयम् । भग॑वत्तम॒ इति॒ भग॑वत् - त॒मः॒ । अ॒यꣳ स॑हस्र॒सात॑मः । स॒ह॒स्र॒सात॑म॒ इति॑ सहस्र - सात॑मः ॥ अ॒स्मा अ॑स्तु । अ॒स्तु॒ सु॒वीर्य᳚म् । सु॒वीर्य॒मिति॑ सु - वीर्य᳚म् ॥ मनो॒ ज्योतिः॑ । ज्योति॑र् जुषताम् । जु॒ष॒ता॒माज्य᳚म् । आज्यं॒ ॅविच्छि॑न्नम् । विच्छि॑न्नं ॅय॒ज्ञ्म् । विच्छि॑न्न॒मिति॒ वि - छि॒न्न॒म् । य॒ज्ञ्ꣳ सम् । समि॒मम् । इ॒मम् द॑धातु । द॒धा॒त्विति॑ दधातु ॥ या इ॒ष्टाः । इ॒ष्टा उ॒षसः॑ । उ॒षसो॑ नि॒म्रुचः॑ । नि॒म्रुच॑श्च । नि॒म्रुच॒ इति॑ नि - म्रुचः॑ । च॒ ताः । ताः सम् । सम् द॑धामि । द॒धा॒मि॒ ह॒विषा᳚ । ह॒विषा॑ घृ॒तेन॑ । घृ॒तेनेति॑ घृ॒तेन॑ ॥ पय॑स्वती॒रोष॑धयः । ओष॑धयः॒ पय॑स्वत् \newline

\textbf{Jatai Paata} \newline

1. अ॒ग्निर् बृ॒हद्व॑या बृ॒हद्व॑या अ॒ग्निर॒ग्निर् बृ॒हद्व॑याः । \newline
2. बृ॒हद्व॑या विश्व॒जिद् वि॑श्व॒जिद् बृ॒हद्व॑या बृ॒हद्व॑या विश्व॒जित् । \newline
3. बृ॒हद्व॑या॒ इति॑ बृ॒हत् - व॒याः॒ । \newline
4. वि॒श्व॒जिथ् सह॑न्त्यः॒ सह॑न्त्यो विश्व॒जिद् वि॑श्व॒जिथ् सह॑न्त्यः । \newline
5. वि॒श्व॒जिदिति॑ विश्व - जित् । \newline
6. सह॑न्त्यः॒ श्रेष्ठः॒ श्रेष्ठः॒ सह॑न्त्यः॒ सह॑न्त्यः॒ श्रेष्ठः॑ । \newline
7. श्रेष्ठो॑ गन्ध॒र्वो ग॑न्ध॒र्वः श्रेष्ठः॒ श्रेष्ठो॑ गन्ध॒र्वः । \newline
8. ग॒न्ध॒र्व इति॑ गन्ध॒र्वः । \newline
9. त्वत्पि॑तारो अग्ने ऽग्ने॒ त्वत्पि॑तार॒ स्त्वत्पि॑तारो अग्ने । \newline
10. त्वत्पि॑तार॒ इति॒ त्वत् - पि॒ता॒रः॒ । \newline
11. अ॒ग्ने॒ दे॒वा दे॒वा अ॑ग्ने ऽग्ने दे॒वाः । \newline
12. दे॒वा स्त्वामा॑हुतय॒ स्त्वामा॑हुतयो दे॒वा दे॒वा स्त्वामा॑हुतयः । \newline
13. त्वामा॑हुतय॒ स्त्वद्वि॑वाचना॒ स्त्वद्वि॑वाचना॒ स्त्वामा॑हुतय॒ स्त्वामा॑हुतय॒ स्त्वद्वि॑वाचनाः । \newline
14. त्वामा॑हुतय॒ इति॒ त्वाम् - आ॒हु॒त॒यः॒ । \newline
15. त्वद्वि॑वाचना॒ इति॒ त्वत् - वि॒वा॒च॒नाः॒ । \newline
16. सम् माम् माꣳ सꣳ सम् माम् । \newline
17. मा मायु॒षा ऽऽयु॑षा॒ माम् मा मायु॑षा । \newline
18. आयु॑षा॒ सꣳ स मायु॒षा ऽऽयु॑षा॒ सम् । \newline
19. सम् गौ॑प॒त्येन॑ गौप॒त्येन॒ सꣳ सम् गौ॑प॒त्येन॑ । \newline
20. गौ॒प॒त्येन॒ सुहि॑ते॒ सुहि॑ते गौप॒त्येन॑ गौप॒त्येन॒ सुहि॑ते । \newline
21. सुहि॑ते मा मा॒ सुहि॑ते॒ सुहि॑ते मा । \newline
22. सुहि॑त॒ इति॒ सु - हि॒ते॒ । \newline
23. मा॒ धा॒ धा॒ मा॒ मा॒ धाः॒ । \newline
24. धा॒ इति॑ धाः । \newline
25. अ॒य म॒ग्नि र॒ग्निर॒य म॒य म॒ग्निः । \newline
26. अ॒ग्निः श्रेष्ठ॑तमः॒ श्रेष्ठ॑तमो॒ ऽग्निर॒ग्निः श्रेष्ठ॑तमः । \newline
27. श्रेष्ठ॑तमो॒ ऽय म॒यꣳ श्रेष्ठ॑तमः॒ श्रेष्ठ॑तमो॒ ऽयम् । \newline
28. श्रेष्ठ॑तम॒ इति॒ श्रेष्ठ॑ - त॒मः॒ । \newline
29. अ॒यम् भग॑वत्तमो॒ भग॑वत्तमो॒ ऽय म॒यम् भग॑वत्तमः । \newline
30. भग॑वत्तमो॒ ऽय म॒यम् भग॑वत्तमो॒ भग॑वत्तमो॒ ऽयम् । \newline
31. भग॑वत्तम॒ इति॒ भग॑वत् - त॒मः॒ । \newline
32. अ॒यꣳ स॑हस्र॒सात॑मः सहस्र॒सात॑मो॒ ऽय म॒यꣳ स॑हस्र॒सात॑मः । \newline
33. स॒ह॒स्र॒सात॑म॒ इति॑ सहस्र - सात॑मः । \newline
34. अ॒स्मा अ॑स्त्वस्त्व॒स्मा अ॒स्मा अ॑स्तु । \newline
35. अ॒स्तु॒ सु॒वीर्य(ग्म्॑) सु॒वीर्य॑ मस्त्वस्तु सु॒वीर्य᳚म् । \newline
36. सु॒वीर्य॒मिति॑ सु - वीर्य᳚म् । \newline
37. मनो॒ ज्योति॒र् ज्योति॒र् मनो॒ मनो॒ ज्योतिः॑ । \newline
38. ज्योति॑र् जुषताम् जुषता॒म् ज्योति॒र् ज्योति॑र् जुषताम् । \newline
39. जु॒ष॒ता॒ माज्य॒ माज्य॑म् जुषताम् जुषता॒ माज्य᳚म् । \newline
40. आज्यं॒ ॅविच्छि॑न्नं॒ ॅविच्छि॑न्न॒ माज्य॒ माज्यं॒ ॅविच्छि॑न्नम् । \newline
41. विच्छि॑न्नं ॅय॒ज्ञ्ं ॅय॒ज्ञ्ं ॅविच्छि॑न्नं॒ ॅविच्छि॑न्नं ॅय॒ज्ञ्म् । \newline
42. विच्छि॑न्न॒मिति॒ वि - छि॒न्न॒म् । \newline
43. य॒ज्ञ्ꣳ सꣳ सं ॅय॒ज्ञ्ं ॅय॒ज्ञ्ꣳ सम् । \newline
44. स मि॒म मि॒मꣳ सꣳ स मि॒मम् । \newline
45. इ॒मम् द॑धातु दधात्वि॒म मि॒मम् द॑धातु । \newline
46. द॒धा॒त्विति॑ दधातु । \newline
47. या इ॒ष्टा इ॒ष्टा या या इ॒ष्टाः । \newline
48. इ॒ष्टा उ॒षस॑ उ॒षस॑ इ॒ष्टा इ॒ष्टा उ॒षसः॑ । \newline
49. उ॒षसो॑ नि॒म्रुचो॑ नि॒म्रुच॑ उ॒षस॑ उ॒षसो॑ नि॒म्रुचः॑ । \newline
50. नि॒म्रुच॑श्च च नि॒म्रुचो॑ नि॒म्रुच॑श्च । \newline
51. नि॒म्रुच॒ इति॑ नि - म्रुचः॑ । \newline
52. च॒ तास्ताश्च॑ च॒ ताः । \newline
53. ताः सꣳ सम् तास्ताः सम् । \newline
54. सम् द॑धामि दधामि॒ सꣳ सम् द॑धामि । \newline
55. द॒धा॒मि॒ ह॒विषा॑ ह॒विषा॑ दधामि दधामि ह॒विषा᳚ । \newline
56. ह॒विषा॑ घृ॒तेन॑ घृ॒तेन॑ ह॒विषा॑ ह॒विषा॑ घृ॒तेन॑ । \newline
57. घृ॒तेनेति॑ घृ॒तेन॑ । \newline
58. पय॑स्वती॒ रोष॑धय॒ ओष॑धयः॒ पय॑स्वतीः॒ पय॑स्वती॒ रोष॑धयः । \newline
59. ओष॑धयः॒ पय॑स्व॒त् पय॑स्व॒दोष॑धय॒ ओष॑धयः॒ पय॑स्वत् । \newline

\textbf{Ghana Paata } \newline

1. अ॒ग्निर् बृ॒हद्व॑या बृ॒हद्व॑या अ॒ग्निर॒ग्निर् बृ॒हद्व॑या विश्व॒जिद् वि॑श्व॒जिद् बृ॒हद्व॑या अ॒ग्निर॒ग्निर् बृ॒हद्व॑या विश्व॒जित् । \newline
2. बृ॒हद्व॑या विश्व॒जिद् वि॑श्व॒जिद् बृ॒हद्व॑या बृ॒हद्व॑या विश्व॒जिथ् सह॑न्त्यः॒ सह॑न्त्यो विश्व॒जिद् बृ॒हद्व॑या बृ॒हद्व॑या विश्व॒जिथ् सह॑न्त्यः । \newline
3. बृ॒हद्व॑या॒ इति॑ बृ॒हत् - व॒याः॒ । \newline
4. वि॒श्व॒जिथ् सह॑न्त्यः॒ सह॑न्त्यो विश्व॒जिद् वि॑श्व॒जिथ् सह॑न्त्यः॒ श्रेष्ठः॒ श्रेष्ठः॒ सह॑न्त्यो विश्व॒जिद् वि॑श्व॒जिथ् सह॑न्त्यः॒ श्रेष्ठः॑ । \newline
5. वि॒श्व॒जिदिति॑ विश्व - जित् । \newline
6. सह॑न्त्यः॒ श्रेष्ठः॒ श्रेष्ठः॒ सह॑न्त्यः॒ सह॑न्त्यः॒ श्रेष्ठो॑ गन्ध॒र्वो ग॑न्ध॒र्वः श्रेष्ठः॒ सह॑न्त्यः॒ सह॑न्त्यः॒ श्रेष्ठो॑ गन्ध॒र्वः । \newline
7. श्रेष्ठो॑ गन्ध॒र्वो ग॑न्ध॒र्वः श्रेष्ठः॒ श्रेष्ठो॑ गन्ध॒र्वः । \newline
8. ग॒न्ध॒र्व इति॑ गन्ध॒र्वः । \newline
9. त्वत्पि॑तारो अग्ने ऽग्ने॒ त्वत्पि॑तार॒ स्त्वत्पि॑तारो अग्ने दे॒वा दे॒वा अ॑ग्ने॒ त्वत्पि॑तार॒ स्त्वत्पि॑तारो अग्ने दे॒वाः । \newline
10. त्वत्पि॑तार॒ इति॒ त्वत् - पि॒ता॒रः॒ । \newline
11. अ॒ग्ने॒ दे॒वा दे॒वा अ॑ग्ने ऽग्ने दे॒वा स्त्वामा॑हुतय॒ स्त्वामा॑हुतयो दे॒वा अ॑ग्ने ऽग्ने दे॒वास्त्वामा॑हुतयः । \newline
12. दे॒वास्त्वामा॑हुतय॒ स्त्वामा॑हुतयो दे॒वा दे॒वास्त्वामा॑हुतय॒ स्त्वद्वि॑वाचना॒ स्त्वद्वि॑वाचना॒ स्त्वामा॑हुतयो दे॒वा दे॒वास्त्वामा॑हुतय॒ स्त्वद्वि॑वाचनाः । \newline
13. त्वामा॑हुतय॒ स्त्वद्वि॑वाचना॒ स्त्वद्वि॑वाचना॒ स्त्वामा॑हुतय॒ स्त्वामा॑हुतय॒ स्त्वद्वि॑वाचनाः । \newline
14. त्वामा॑हुतय॒ इति॒ त्वाम् - आ॒हु॒त॒यः॒ । \newline
15. त्वद्वि॑वाचना॒ इति॒ त्वत् - वि॒वा॒च॒नाः॒ । \newline
16. सम् माम् माꣳ सꣳ सम् मा मायु॒षा ऽऽयु॑षा॒ माꣳ सꣳ सम् मा मायु॑षा । \newline
17. मा मायु॒षा ऽऽयु॑षा॒ माम् मा मायु॑षा॒ सꣳ स मायु॑षा॒ माम् मा मायु॑षा॒ सम् । \newline
18. आयु॑षा॒ सꣳ स मायु॒षा ऽऽयु॑षा॒ सम् गौ॑प॒त्येन॑ गौप॒त्येन॒ स मायु॒षा ऽऽयु॑षा॒ सम् गौ॑प॒त्येन॑ । \newline
19. सम् गौ॑प॒त्येन॑ गौप॒त्येन॒ सꣳ सम् गौ॑प॒त्येन॒ सुहि॑ते॒ सुहि॑ते गौप॒त्येन॒ सꣳ सम् गौ॑प॒त्येन॒ सुहि॑ते । \newline
20. गौ॒प॒त्येन॒ सुहि॑ते॒ सुहि॑ते गौप॒त्येन॑ गौप॒त्येन॒ सुहि॑ते मा मा॒ सुहि॑ते गौप॒त्येन॑ गौप॒त्येन॒ सुहि॑ते मा । \newline
21. सुहि॑ते मा मा॒ सुहि॑ते॒ सुहि॑ते मा धा धा मा॒ सुहि॑ते॒ सुहि॑ते मा धाः । \newline
22. सुहि॑त॒ इति॒ सु - हि॒ते॒ । \newline
23. मा॒ धा॒ धा॒ मा॒ मा॒ धाः॒ । \newline
24. धा॒ इति॑ धाः । \newline
25. अ॒य म॒ग्निर॒ग्निर॒य म॒य म॒ग्निः श्रेष्ठ॑तमः॒ श्रेष्ठ॑तमो॒ ऽग्निर॒य म॒य म॒ग्निः श्रेष्ठ॑तमः । \newline
26. अ॒ग्निः श्रेष्ठ॑तमः॒ श्रेष्ठ॑तमो॒ ऽग्निर॒ग्निः श्रेष्ठ॑तमो॒ ऽय म॒यꣳ श्रेष्ठ॑तमो॒ ऽग्निर॒ग्निः श्रेष्ठ॑तमो॒ ऽयम् । \newline
27. श्रेष्ठ॑तमो॒ ऽय म॒यꣳ श्रेष्ठ॑तमः॒ श्रेष्ठ॑तमो॒ ऽयम् भग॑वत्तमो॒ भग॑वत्तमो॒ ऽयꣳ श्रेष्ठ॑तमः॒ श्रेष्ठ॑तमो॒ ऽयम् भग॑वत्तमः । \newline
28. श्रेष्ठ॑तम॒ इति॒ श्रेष्ठ॑ - त॒मः॒ । \newline
29. अ॒यम् भग॑वत्तमो॒ भग॑वत्तमो॒ ऽय म॒यम् भग॑वत्तमो॒ ऽय म॒यम् भग॑वत्तमो॒ ऽय म॒यम् भग॑वत्तमो॒ ऽयम् । \newline
30. भग॑वत्तमो॒ ऽय म॒यम् भग॑वत्तमो॒ भग॑वत्तमो॒ ऽयꣳ स॑हस्र॒सात॑मः सहस्र॒सात॑मो॒ ऽयम् भग॑वत्तमो॒ भग॑वत्तमो॒ ऽयꣳ स॑हस्र॒सात॑मः । \newline
31. भग॑वत्तम॒ इति॒ भग॑वत् - त॒मः॒ । \newline
32. अ॒यꣳ स॑हस्र॒सात॑मः सहस्र॒सात॑मो॒ ऽय म॒यꣳ स॑हस्र॒सात॑मः । \newline
33. स॒ह॒स्र॒सात॑म॒ इति॑ सहस्र - सात॑मः । \newline
34. अ॒स्मा अ॑स्त्वस्त्व॒स्मा अ॒स्मा अ॑स्तु सु॒वीर्य(ग्म्॑) सु॒वीर्य॑ मस्त्व॒स्मा अ॒स्मा अ॑स्तु सु॒वीर्य᳚म् । \newline
35. अ॒स्तु॒ सु॒वीर्य(ग्म्॑) सु॒वीर्य॑ मस्त्वस्तु सु॒वीर्य᳚म् । \newline
36. सु॒वीर्य॒मिति॑ सु - वीर्य᳚म् । \newline
37. मनो॒ ज्योति॒र् ज्योति॒र् मनो॒ मनो॒ ज्योति॑र् जुषताम् जुषता॒म् ज्योति॒र् मनो॒ मनो॒ ज्योति॑र् जुषताम् । \newline
38. ज्योति॑र् जुषताम् जुषता॒म् ज्योति॒र् ज्योति॑र् जुषता॒ माज्य॒ माज्य॑म् जुषता॒म् ज्योति॒र् ज्योति॑र् जुषता॒ माज्य᳚म् । \newline
39. जु॒ष॒ता॒ माज्य॒ माज्य॑म् जुषताम् जुषता॒ माज्यं॒ ॅविच्छि॑न्नं॒ ॅविच्छि॑न्न॒ माज्य॑म् जुषताम् जुषता॒ माज्यं॒ ॅविच्छि॑न्नम् । \newline
40. आज्यं॒ ॅविच्छि॑न्नं॒ ॅविच्छि॑न्न॒ माज्य॒ माज्यं॒ ॅविच्छि॑न्नं ॅय॒ज्ञ्ं ॅय॒ज्ञ्ं ॅविच्छि॑न्न॒ माज्य॒ माज्यं॒ ॅविच्छि॑न्नं ॅय॒ज्ञ्म् । \newline
41. विच्छि॑न्नं ॅय॒ज्ञ्ं ॅय॒ज्ञ्ं ॅविच्छि॑न्नं॒ ॅविच्छि॑न्नं ॅय॒ज्ञ्ꣳ सꣳ सं ॅय॒ज्ञ्ं ॅविच्छि॑न्नं॒ ॅविच्छि॑न्नं ॅय॒ज्ञ्ꣳ सम् । \newline
42. विच्छि॑न्न॒मिति॒ वि - छि॒न्न॒म् । \newline
43. य॒ज्ञ्ꣳ सꣳ सं ॅय॒ज्ञ्ं ॅय॒ज्ञ्ꣳ स मि॒म मि॒मꣳ सं ॅय॒ज्ञ्ं ॅय॒ज्ञ्ꣳ स मि॒मम् । \newline
44. स मि॒म मि॒मꣳ सꣳ स मि॒मम् द॑धातु दधात्वि॒मꣳ सꣳ स मि॒मम् द॑धातु । \newline
45. इ॒मम् द॑धातु दधात्वि॒म मि॒मम् द॑धातु । \newline
46. द॒धा॒त्विति॑ दधातु । \newline
47. या इ॒ष्टा इ॒ष्टा या या इ॒ष्टा उ॒षस॑ उ॒षस॑ इ॒ष्टा या या इ॒ष्टा उ॒षसः॑ । \newline
48. इ॒ष्टा उ॒षस॑ उ॒षस॑ इ॒ष्टा इ॒ष्टा उ॒षसो॑ नि॒म्रुचो॑ नि॒म्रुच॑ उ॒षस॑ इ॒ष्टा इ॒ष्टा उ॒षसो॑ नि॒म्रुचः॑ । \newline
49. उ॒षसो॑ नि॒म्रुचो॑ नि॒म्रुच॑ उ॒षस॑ उ॒षसो॑ नि॒म्रुच॑श्च च नि॒म्रुच॑ उ॒षस॑ उ॒षसो॑ नि॒म्रुच॑श्च । \newline
50. नि॒म्रुच॑श्च च नि॒म्रुचो॑ नि॒म्रुच॑श्च॒ तास्ताश्च॑ नि॒म्रुचो॑ नि॒म्रुच॑श्च॒ ताः । \newline
51. नि॒म्रुच॒ इति॑ नि - म्रुचः॑ । \newline
52. च॒ तास्ताश्च॑ च॒ ताः सꣳ सम् ताश्च॑ च॒ ताः सम् । \newline
53. ताः सꣳ सम् तास्ताः सम् द॑धामि दधामि॒ सम् तास्ताः सम् द॑धामि । \newline
54. सम् द॑धामि दधामि॒ सꣳ सम् द॑धामि ह॒विषा॑ ह॒विषा॑ दधामि॒ सꣳ सम् द॑धामि ह॒विषा᳚ । \newline
55. द॒धा॒मि॒ ह॒विषा॑ ह॒विषा॑ दधामि दधामि ह॒विषा॑ घृ॒तेन॑ घृ॒तेन॑ ह॒विषा॑ दधामि दधामि ह॒विषा॑ घृ॒तेन॑ । \newline
56. ह॒विषा॑ घृ॒तेन॑ घृ॒तेन॑ ह॒विषा॑ ह॒विषा॑ घृ॒तेन॑ । \newline
57. घृ॒तेनेति॑ घृ॒तेन॑ । \newline
58. पय॑स्वती॒रोष॑धय॒ ओष॑धयः॒ पय॑स्वतीः॒ पय॑स्वती॒रोष॑धयः॒ पय॑स्व॒त् पय॑स्व॒दोष॑धयः॒ पय॑स्वतीः॒ पय॑स्वती॒रोष॑धयः॒ पय॑स्वत् । \newline
59. ओष॑धयः॒ पय॑स्व॒त् पय॑स्व॒दोष॑धय॒ ओष॑धयः॒ पय॑स्वद् वी॒रुधां᳚ ॅवी॒रुधा॒म् पय॑स्व॒दोष॑धय॒ ओष॑धयः॒ पय॑स्वद् वी॒रुधा᳚म् । \newline
\pagebreak
\markright{ TS 1.5.10.3  \hfill https://www.vedavms.in \hfill}
\addcontentsline{toc}{section}{ TS 1.5.10.3 }
\section*{ TS 1.5.10.3 }

\textbf{TS 1.5.10.3 } \newline
\textbf{Samhita Paata} \newline

पय॑स्वद्वी॒रुधां॒ पयः॑ । अ॒पां पय॑सो॒ यत्पय॒स्तेन॒ मामि॑न्द्र॒ सꣳ सृ॑ज ॥ अग्ने᳚ व्रतपते व्र॒तं च॑रिष्यामि॒ तच्छ॑केयं॒ तन्मे॑ राद्ध्यतां ॥ अ॒ग्निꣳ होता॑रमि॒ह तꣳ हु॑वे दे॒वान्. य॒ज्ञिया॑नि॒ह यान्. हवा॑महे ॥ आ य॑न्तु दे॒वाः सु॑मन॒स्यमा॑ना वि॒यन्तु॑ दे॒वा ह॒विषो॑ मे अ॒स्य ॥ कस्त्वा॑ युनक्ति॒ स त्वा॑ युनक्तु॒ यानि॑ घ॒र्मे क॒पाला᳚न्युपचि॒न्वन्ति॑ - [ ] \newline

\textbf{Pada Paata} \newline

पय॑स्वत् । वी॒रुधा᳚म् । पयः॑ । अ॒पाम् । पय॑सः । यत् । पयः॑ ॥ तेन॑ । माम् । इ॒न्द्र॒ । समिति॑ । सृ॒ज॒ ॥ अग्ने᳚ । व्र॒त॒प॒त॒ इति॑ व्रत - प॒ते॒ । व्र॒तम् । च॒रि॒ष्या॒मि॒ । तत् । श॒के॒य॒म् । तत् । मे॒ । रा॒द्ध्य॒ता॒म् ॥ अ॒ग्निम् । होता॑रम् । इ॒ह । तम् । हु॒वे॒ । दे॒वान् । य॒ज्ञियान्॑ । इ॒ह । यान् । हवा॑महे ॥ एति॑ । य॒न्तु॒ । दे॒वाः । सु॒म॒न॒स्यमा॑ना॒ इति॑ सु - म॒न॒स्यमा॑नाः । वि॒यन्तु॑ । दे॒वाः । ह॒विषः॑ । मे॒ । अ॒स्य ॥ कः । त्वा॒ । यु॒न॒क्ति॒ । सः । त्वा॒ । यु॒न॒क्तु॒ । यानि॑ । घ॒र्मे । क॒पाला॑नि । उ॒प॒चि॒न्वन्तीत्यु॑प - चि॒न्वन्ति॑ ।  \newline


\textbf{Krama Paata} \newline

पय॑स्वद् वी॒रुधा᳚म् । वी॒रुधा॒म् पयः॑ । पय॒ इति॒ पयः॑ ॥ अ॒पाम् पय॑सः । पय॑सो॒ यत् । यत्,पयः॑ । पय॒स्तेन॑ । तेन॒ माम् । मामि॑न्द्र । इ॒न्द्र॒ सम् । सꣳ सृ॑ज । सृ॒जेति॑ सृज ॥ अग्ने᳚ व्रतपते । व्र॒त॒प॒ते॒ व्र॒तम् । व्र॒त॒प॒त॒ इति॑ व्रत - प॒ते॒ । व्र॒तम् च॑रिष्यामि । च॒रि॒ष्या॒मि॒ तत् । तच्छ॑केयम् । श॒के॒य॒म् तत् । तन्मे᳚ । मे॒ रा॒द्ध्य॒ता॒म् । रा॒द्ध्य॒ता॒मिति॑ राद्ध्यताम् ॥ अ॒ग्निꣳ होता॑रम् । होता॑रमि॒ह । इ॒ह तम् । तꣳ हु॑वे । हु॒वे॒ दे॒वान् । दे॒वान्. य॒ज्ञियान्॑ । य॒ज्ञिया॑नि॒ह । इ॒ह यान् । यान्. हवा॑महे । हवा॑मह॒ इति॒ हवा॑महे ॥ आ य॑न्तु । य॒न्तु॒ दे॒वाः । दे॒वाः सु॑मन॒स्यमा॑नाः । सु॒म॒न॒स्यमा॑ना वि॒यन्तु॑ । सु॒म॒न॒स्यमा॑ना॒ इति॑ सु - म॒न॒स्यमा॑नाः । वि॒यन्तु॑ दे॒वाः । दे॒वा ह॒विषः॑ । ह॒विषो॑ मे । मे॒ अ॒स्य । अ॒स्येत्य॒स्य ॥ कस्त्वा᳚ । त्वा॒ यु॒न॒क्ति॒ । यु॒न॒क्ति॒ सः । स त्वा᳚ । त्वा॒ यु॒न॒क्तु॒ । यु॒न॒क्तु॒ यानि॑ । यानि॑ घ॒र्मे । घ॒र्मे क॒पाला॑नि । क॒पाला᳚न्युपचि॒न्वन्ति॑ ( ) । उ॒प॒चि॒न्वन्ति॑ वे॒धसः॑ । उ॒प॒चि॒न्वन्तीत्यु॑प - चि॒न्वन्ति॑ \newline

\textbf{Jatai Paata} \newline

1. पय॑स्वद् वी॒रुधां᳚ ॅवी॒रुधा॒म् पय॑स्व॒त् पय॑स्वद् वी॒रुधा᳚म् । \newline
2. वी॒रुधा॒म् पयः॒ पयो॑ वी॒रुधां᳚ ॅवी॒रुधा॒म् पयः॑ । \newline
3. पय॒ इति॒ पयः॑ । \newline
4. अ॒पाम् पय॑सः॒ पय॑सो॒ ऽपा म॒पाम् पय॑सः । \newline
5. पय॑सो॒ यद् यत् पय॑सः॒ पय॑सो॒ यत् । \newline
6. यत् पयः॒ पयो॒ यद् यत् पयः॑ । \newline
7. पय॒स्तेन॒ तेन॒ पयः॒ पय॒स्तेन॑ । \newline
8. तेन॒ माम् माम् तेन॒ तेन॒ माम् । \newline
9. मा मि॑न्द्रे न्द्र॒ माम् मा मि॑न्द्र । \newline
10. इ॒न्द्र॒ सꣳ स मि॑न्द्रे न्द्र॒ सम् । \newline
11. सꣳ सृ॑ज सृज॒ सꣳ सꣳ सृ॑ज । \newline
12. सृ॒जेति॑ सृज । \newline
13. अग्ने᳚ व्रतपते व्रतप॒ते ऽग्ने ऽग्ने᳚ व्रतपते । \newline
14. व्र॒त॒प॒ते॒ व्र॒तं ॅव्र॒तं ॅव्र॑तपते व्रतपते व्र॒तम् । \newline
15. व्र॒त॒प॒त॒ इति॑ व्रत - प॒ते॒ । \newline
16. व्र॒तम् च॑रिष्यामि चरिष्यामि व्र॒तं ॅव्र॒तम् च॑रिष्यामि । \newline
17. च॒रि॒ष्या॒मि॒ तत् तच् च॑रिष्यामि चरिष्यामि॒ तत् । \newline
18. तच्छ॑केयꣳ शकेय॒म् तत् तच्छ॑केयम् । \newline
19. श॒के॒य॒म् तत् तच्छ॑केयꣳ शकेय॒म् तत् । \newline
20. तन् मे॑ मे॒ तत् तन् मे᳚ । \newline
21. मे॒ रा॒द्ध्य॒ता॒(ग्म्॒) रा॒द्ध्य॒ता॒म् मे॒ मे॒ रा॒द्ध्य॒ता॒म् । \newline
22. रा॒द्ध्य॒ता॒मिति॑ राद्ध्यताम् । \newline
23. अ॒ग्निꣳ होता॑र॒(ग्म्॒) होता॑र म॒ग्नि म॒ग्निꣳ होता॑रम् । \newline
24. होता॑र मि॒हे ह होता॑र॒(ग्म्॒) होता॑र मि॒ह । \newline
25. इ॒ह तम् त मि॒हे ह तम् । \newline
26. तꣳ हु॑वे हुवे॒ तम् तꣳ हु॑वे । \newline
27. हु॒वे॒ दे॒वान् दे॒वान्. हु॑वे हुवे दे॒वान् । \newline
28. दे॒वान्. य॒ज्ञियान्॑. य॒ज्ञिया᳚न् दे॒वान् दे॒वान्. य॒ज्ञियान्॑ । \newline
29. य॒ज्ञिया॑ नि॒हे ह य॒ज्ञियान्॑. य॒ज्ञिया॑ नि॒ह । \newline
30. इ॒ह यान्. या नि॒हे ह यान् । \newline
31. यान्. हवा॑महे॒ हवा॑महे॒ यान्. यान्. हवा॑महे । \newline
32. हवा॑मह॒ इति॒ हवा॑महे । \newline
33. आ य॑न्तु य॒न्त्वा य॑न्तु । \newline
34. य॒न्तु॒ दे॒वा दे॒वा य॑न्तु यन्तु दे॒वाः । \newline
35. दे॒वाः सु॑मन॒स्यमा॑नाः सुमन॒स्यमा॑ना दे॒वा दे॒वाः सु॑मन॒स्यमा॑नाः । \newline
36. सु॒म॒न॒स्यमा॑ना वि॒यन्तु॑ वि॒यन्तु॑ सुमन॒स्यमा॑नाः सुमन॒स्यमा॑ना वि॒यन्तु॑ । \newline
37. सु॒म॒न॒स्यमा॑ना॒ इति॑ सु - म॒न॒स्यमा॑नाः । \newline
38. वि॒यन्तु॑ दे॒वा दे॒वा वि॒यन्तु॑ वि॒यन्तु॑ दे॒वाः । \newline
39. दे॒वा ह॒विषो॑ ह॒विषो॑ दे॒वा दे॒वा ह॒विषः॑ । \newline
40. ह॒विषो॑ मे मे ह॒विषो॑ ह॒विषो॑ मे । \newline
41. मे॒ अ॒स्यास्य मे॑ मे अ॒स्य । \newline
42. अ॒स्येत्य॒स्य । \newline
43. कस्त्वा᳚ त्वा॒ कः कस्त्वा᳚ । \newline
44. त्वा॒ यु॒न॒क्ति॒ यु॒न॒क्ति॒ त्वा॒ त्वा॒ यु॒न॒क्ति॒ । \newline
45. यु॒न॒क्ति॒ स स यु॑नक्ति युनक्ति॒ सः । \newline
46. स त्वा᳚ त्वा॒ स स त्वा᳚ । \newline
47. त्वा॒ यु॒न॒क्तु॒ यु॒न॒क्तु॒ त्वा॒ त्वा॒ यु॒न॒क्तु॒ । \newline
48. यु॒न॒क्तु॒ यानि॒ यानि॑ युनक्तु युनक्तु॒ यानि॑ । \newline
49. यानि॑ घ॒र्मे घ॒र्मे यानि॒ यानि॑ घ॒र्मे । \newline
50. घ॒र्मे क॒पाला॑नि क॒पाला॑नि घ॒र्मे घ॒र्मे क॒पाला॑नि । \newline
51. क॒पाला᳚ न्युपचि॒न्व न्त्यु॑पचि॒न्वन्ति॑ क॒पाला॑नि क॒पाला᳚ न्युपचि॒न्वन्ति॑ । \newline
52. उ॒प॒चि॒न्वन्ति॑ वे॒धसो॑वे॒धस॑ उपचि॒न्व न्त्यु॑पचि॒न्वन्ति॑ वे॒धसः॑ । \newline
53. उ॒प॒चि॒न्वन्तीत्यु॑प - चि॒न्वन्ति॑ । \newline

\textbf{Ghana Paata } \newline

1. पय॑स्वद् वी॒रुधां᳚ ॅवी॒रुधा॒म् पय॑स्व॒त् पय॑स्वद् वी॒रुधा॒म् पयः॒ पयो॑ वी॒रुधा॒म् पय॑स्व॒त् पय॑स्वद् वी॒रुधा॒म् पयः॑ । \newline
2. वी॒रुधा॒म् पयः॒ पयो॑ वी॒रुधां᳚ ॅवी॒रुधा॒म् पयः॑ । \newline
3. पय॒ इति॒ पयः॑ । \newline
4. अ॒पाम् पय॑सः॒ पय॑सो॒ ऽपा म॒पाम् पय॑सो॒ यद् यत् पय॑सो॒ ऽपा म॒पाम् पय॑सो॒ यत् । \newline
5. पय॑सो॒ यद् यत् पय॑सः॒ पय॑सो॒ यत् पयः॒ पयो॒ यत् पय॑सः॒ पय॑सो॒ यत् पयः॑ । \newline
6. यत् पयः॒ पयो॒ यद् यत् पय॒स्तेन॒ तेन॒ पयो॒ यद् यत् पय॒स्तेन॑ । \newline
7. पय॒स्तेन॒ तेन॒ पयः॒ पय॒स्तेन॒ माम् माम् तेन॒ पयः॒ पय॒स्तेन॒ माम् । \newline
8. तेन॒ माम् माम् तेन॒ तेन॒ मा मि॑न्द्रे न्द्र॒ माम् तेन॒ तेन॒ मा मि॑न्द्र । \newline
9. मा मि॑न्द्रे न्द्र॒ माम् मा मि॑न्द्र॒ सꣳ स मि॑न्द्र॒ माम् मा मि॑न्द्र॒ सम् । \newline
10. इ॒न्द्र॒ सꣳ स मि॑न्द्रे न्द्र॒ सꣳ सृ॑ज सृज॒ स मि॑न्द्रे न्द्र॒ सꣳ सृ॑ज । \newline
11. सꣳ सृ॑ज सृज॒ सꣳ सꣳ सृ॑ज । \newline
12. सृ॒जेति॑ सृज । \newline
13. अग्ने᳚ व्रतपते व्रतप॒ते ऽग्ने ऽग्ने᳚ व्रतपते व्र॒तं ॅव्र॒तं ॅव्र॑तप॒ते ऽग्ने ऽग्ने᳚ व्रतपते व्र॒तम् । \newline
14. व्र॒त॒प॒ते॒ व्र॒तं ॅव्र॒तं ॅव्र॑तपते व्रतपते व्र॒तम् च॑रिष्यामि चरिष्यामि व्र॒तं ॅव्र॑तपते व्रतपते व्र॒तम् च॑रिष्यामि । \newline
15. व्र॒त॒प॒त॒ इति॑ व्रत - प॒ते॒ । \newline
16. व्र॒तम् च॑रिष्यामि चरिष्यामि व्र॒तं ॅव्र॒तम् च॑रिष्यामि॒ तत् तच् च॑रिष्यामि व्र॒तं ॅव्र॒तम् च॑रिष्यामि॒ तत् । \newline
17. च॒रि॒ष्या॒मि॒ तत् तच् च॑रिष्यामि चरिष्यामि॒ तच्छ॑केयꣳ शकेय॒म् तच् च॑रिष्यामि चरिष्यामि॒ तच्छ॑केयम् । \newline
18. तच्छ॑केयꣳ शकेय॒म् तत् तच्छ॑केय॒म् तत् तच्छ॑केय॒म् तत् तच्छ॑केय॒म् तत् । \newline
19. श॒के॒य॒म् तत् तच्छ॑केयꣳ शकेय॒म् तन् मे॑ मे॒ तच्छ॑केयꣳ शकेय॒म् तन् मे᳚ । \newline
20. तन् मे॑ मे॒ तत् तन् मे॑ राद्ध्यताꣳ राद्ध्यताम् मे॒ तत् तन् मे॑ राद्ध्यताम् । \newline
21. मे॒ रा॒द्ध्य॒ता॒(ग्म्॒) रा॒द्ध्य॒ता॒म् मे॒ मे॒ रा॒द्ध्य॒ता॒म् । \newline
22. रा॒द्ध्य॒ता॒मिति॑ राद्ध्यताम् । \newline
23. अ॒ग्निꣳ होता॑र॒(ग्म्॒) होता॑र म॒ग्नि म॒ग्निꣳ होता॑र मि॒हे ह होता॑र म॒ग्नि म॒ग्निꣳ होता॑र मि॒ह । \newline
24. होता॑र मि॒हे ह होता॑र॒(ग्म्॒) होता॑र मि॒ह तम् त मि॒ह होता॑र॒(ग्म्॒) होता॑र मि॒ह तम् । \newline
25. इ॒ह तम् त मि॒हे ह तꣳ हु॑वे हुवे॒ त मि॒हे ह तꣳ हु॑वे । \newline
26. तꣳ हु॑वे हुवे॒ तम् तꣳ हु॑वे दे॒वान् दे॒वान्. हु॑वे॒ तम् तꣳ हु॑वे दे॒वान् । \newline
27. हु॒वे॒ दे॒वान् दे॒वान्. हु॑वे हुवे दे॒वान्. य॒ज्ञियान्॑. य॒ज्ञिया᳚न् दे॒वान्. हु॑वे हुवे दे॒वान्. य॒ज्ञियान्॑ । \newline
28. दे॒वान्. य॒ज्ञियान्॑. य॒ज्ञिया᳚न् दे॒वान् दे॒वान् य॒ज्ञिया॑ नि॒हे ह य॒ज्ञिया᳚न् दे॒वान् दे॒वान्. य॒ज्ञिया॑ नि॒ह । \newline
29. य॒ज्ञिया॑ नि॒हे ह य॒ज्ञियान्॑. य॒ज्ञिया॑ नि॒ह यान्. यानि॒ह य॒ज्ञियान्॑. य॒ज्ञिया॑ नि॒ह यान् । \newline
30. इ॒ह यान्. या नि॒हे ह यान्. हवा॑महे॒ हवा॑महे॒ या नि॒हे ह यान्. हवा॑महे । \newline
31. यान्. हवा॑महे॒ हवा॑महे॒ यान्. यान्. हवा॑महे । \newline
32. हवा॑मह॒ इति॒ हवा॑महे । \newline
33. आ य॑न्तु य॒न्त्वा य॑न्तु दे॒वा दे॒वा य॒न्त्वा य॑न्तु दे॒वाः । \newline
34. य॒न्तु॒ दे॒वा दे॒वा य॑न्तु यन्तु दे॒वाः सु॑मन॒स्यमा॑नाः सुमन॒स्यमा॑ना दे॒वा य॑न्तु यन्तु दे॒वाः सु॑मन॒स्यमा॑नाः । \newline
35. दे॒वाः सु॑मन॒स्यमा॑नाः सुमन॒स्यमा॑ना दे॒वा दे॒वाः सु॑मन॒स्यमा॑ना वि॒यन्तु॑ वि॒यन्तु॑ सुमन॒स्यमा॑ना दे॒वा दे॒वाः सु॑मन॒स्यमा॑ना वि॒यन्तु॑ । \newline
36. सु॒म॒न॒स्यमा॑ना वि॒यन्तु॑ वि॒यन्तु॑ सुमन॒स्यमा॑नाः सुमन॒स्यमा॑ना वि॒यन्तु॑ दे॒वा दे॒वा वि॒यन्तु॑ सुमन॒स्यमा॑नाः सुमन॒स्यमा॑ना वि॒यन्तु॑ दे॒वाः । \newline
37. सु॒म॒न॒स्यमा॑ना॒ इति॑ सु - म॒न॒स्यमा॑नाः । \newline
38. वि॒यन्तु॑ दे॒वा दे॒वा वि॒यन्तु॑ वि॒यन्तु॑ दे॒वा ह॒विषो॑ ह॒विषो॑ दे॒वा वि॒यन्तु॑ वि॒यन्तु॑ दे॒वा ह॒विषः॑ । \newline
39. दे॒वा ह॒विषो॑ ह॒विषो॑ दे॒वा दे॒वा ह॒विषो॑ मे मे ह॒विषो॑ दे॒वा दे॒वा ह॒विषो॑ मे । \newline
40. ह॒विषो॑ मे मे ह॒विषो॑ ह॒विषो॑ मे अ॒स्यास्य मे॑ ह॒विषो॑ ह॒विषो॑ मे अ॒स्य । \newline
41. मे॒ अ॒स्यास्य मे॑ मे अ॒स्य । \newline
42. अ॒स्येत्य॒स्य । \newline
43. कस्त्वा᳚ त्वा॒ कः कस्त्वा॑ युनक्ति युनक्ति त्वा॒ कः कस्त्वा॑ युनक्ति । \newline
44. त्वा॒ यु॒न॒क्ति॒ यु॒न॒क्ति॒ त्वा॒ त्वा॒ यु॒न॒क्ति॒ स स यु॑नक्ति त्वा त्वा युनक्ति॒ सः । \newline
45. यु॒न॒क्ति॒ स स यु॑नक्ति युनक्ति॒ स त्वा᳚ त्वा॒ स यु॑नक्ति युनक्ति॒ स त्वा᳚ । \newline
46. स त्वा᳚ त्वा॒ स स त्वा॑ युनक्तु युनक्तु त्वा॒ स स त्वा॑ युनक्तु । \newline
47. त्वा॒ यु॒न॒क्तु॒ यु॒न॒क्तु॒ त्वा॒ त्वा॒ यु॒न॒क्तु॒ यानि॒ यानि॑ युनक्तु त्वा त्वा युनक्तु॒ यानि॑ । \newline
48. यु॒न॒क्तु॒ यानि॒ यानि॑ युनक्तु युनक्तु॒ यानि॑ घ॒र्मे घ॒र्मे यानि॑ युनक्तु युनक्तु॒ यानि॑ घ॒र्मे । \newline
49. यानि॑ घ॒र्मे घ॒र्मे यानि॒ यानि॑ घ॒र्मे क॒पाला॑नि क॒पाला॑नि घ॒र्मे यानि॒ यानि॑ घ॒र्मे क॒पाला॑नि । \newline
50. घ॒र्मे क॒पाला॑नि क॒पाला॑नि घ॒र्मे घ॒र्मे क॒पाला᳚ न्युपचि॒न्व न्त्यु॑पचि॒न्वन्ति॑ क॒पाला॑नि घ॒र्मे घ॒र्मे क॒पाला᳚न्युपचि॒न्वन्ति॑ । \newline
51. क॒पाला᳚ न्युपचि॒न्व न्त्यु॑पचि॒न्वन्ति॑ क॒पाला॑नि क॒पाला᳚न्युपचि॒न्वन्ति॑ वे॒धसो॑ वे॒धस॑ उपचि॒न्वन्ति॑ क॒पाला॑नि क॒पाला᳚न्युपचि॒न्वन्ति॑ वे॒धसः॑ । \newline
52. उ॒प॒चि॒न्वन्ति॑ वे॒धसो॑ वे॒धस॑ उपचि॒न्वन् त्यु॑पचि॒न्वन्ति॑ वे॒धसः॑ । \newline
53. उ॒प॒चि॒न्वन्तीत्यु॑प - चि॒न्वन्ति॑ । \newline
\pagebreak
\markright{ TS 1.5.10.4  \hfill https://www.vedavms.in \hfill}
\addcontentsline{toc}{section}{ TS 1.5.10.4 }
\section*{ TS 1.5.10.4 }

\textbf{TS 1.5.10.4 } \newline
\textbf{Samhita Paata} \newline

वे॒धसः॑ । पू॒ष्णस्तान्यपि॑ व्र॒त इ॑न्द्रवा॒यू वि मु॑ञ्चतां ॥अभि॑न्नो घ॒र्मो जी॒रदा॑नु॒र्यत॒ आत्त॒स्तद॑ग॒न् पुनः॑ । इ॒द्ध्मो वेदिः॑ परि॒धय॑श्च॒ सर्वे॑ य॒ज्ञ्स्याऽऽयु॒रनु॒ सं च॑रन्ति ॥ त्रय॑स्त्रिꣳश॒त् तन्त॑वो॒॒ ये वि॑तत्नि॒रे य इ॒मं ॅय॒ज्ञ्ꣳ स्व॒धया॒ दद॑न्ते॒ तेषां᳚ छि॒न्नं प्रत्ये॒तद्-द॑धामि॒ स्वाहा॑ घ॒र्मो दे॒वाꣳ अप्ये॑तु ॥ \newline

\textbf{Pada Paata} \newline

वे॒धसः॑ ॥ पू॒ष्णः । तानि॑ । अपीति॑ । व्र॒ते । इ॒न्द्र॒वा॒यू इती᳚न्द्र - वा॒यू । वीति॑ । मु॒ञ्च॒ता॒म् ॥ अभि॑न्नः । घ॒र्मः । जी॒रदा॑नु॒रिति॑ जी॒र - दा॒नुः॒ । यतः॑ । आत्तः॑ । तत् । अ॒ग॒न्न् । पुनः॑ ॥ इ॒द्ध्मः । वेदिः॑ । प॒रि॒ध॒य॒ इति॑ परि - धयः॑ । च॒ । सर्वे᳚ । य॒ज्ञ्स्य॑ । आयुः॑ । अनु॑ । समिति॑ । च॒र॒न्ति॒ ॥ त्रय॑स्त्रिꣳश॒दिति॒ त्रयः॑ - त्रिꣳ॒॒श॒त् । तन्त॑वः । ये । वि॒त॒त्नि॒र इति॑ वि - त॒त्नि॒रे । ये । इ॒मम् । य॒ज्ञ्म् । स्व॒धयेति॑ स्व - धया᳚ । दद॑न्ते । तेषा᳚म् । छि॒न्नम् । प्रतीति॑ । ए॒तत् । द॒धा॒मि॒ । स्वाहा᳚ । घ॒र्मः । दे॒वान् । अपीति॑ । ए॒तु॒ ॥  \newline


\textbf{Krama Paata} \newline

वे॒धस॒ इति॑ वे॒धसः॑ ॥ पू॒ष्णस्तानि॑ । तान्यपि॑ । अपि॑ व्र॒ते । व्र॒त इ॑न्द्रवा॒यू । इ॒न्द्र॒वा॒यू वि । इ॒न्द्र॒वा॒यू इती᳚न्द्र - वा॒यू । वि मु॑ञ्चताम् । मु॒ञ्च॒ता॒मिति॑ मुञ्चताम् ॥ अभि॑न्नो घ॒र्मः । घ॒र्मो जी॒रदा॑नुः । जी॒रदा॑नु॒र्,यतः॑ । जी॒रदा॑नु॒रिति॑ जी॒र - दा॒नुः॒ । यत॒ आत्तः॑ । आत्त॒स्तत् । तद॑गन्न् । अ॒ग॒न् पुनः॑ । पुन॒रिति॒ पुनः॑ ॥ इ॒द्ध्मो वेदिः॑ । वेदिः॑ परि॒धयः॑ । प॒रि॒धय॑श्च । प॒रि॒धय॒ इति॑ परि - धयः॑ । च॒ सर्वे᳚ । सर्वे॑ य॒ज्ञ्स्य॑ । य॒ज्ञ्स्यायुः॑ । आयु॒रनु॑ । अनु॒ सम् । सञ्च॑रन्ति । च॒र॒न्तीति॑ चरन्ति ॥ त्रय॑स्त्रिꣳश॒त्,तन्त॑वः । त्रय॑स्त्रिꣳश॒दिति॒ त्रयः॑ - त्रिꣳ॒॒श॒त्॒ । तन्त॑वो॒ ये । ये वि॑तत्नि॒रे । वि॒त॒त्नि॒रे ये । वि॒त॒त्नि॒र इति॑ वि - त॒त्नि॒रे । य इ॒मम् । इ॒मं ॅय॒ज्ञ्म् । य॒ज्ञ्ꣳ स्व॒धया᳚ । स्व॒धया॒ दद॑न्ते । स्व॒धयेति॑ स्व - धया᳚ । दद॑न्ते॒ तेषा᳚म् । तेषा᳚म् छि॒न्नम् । छि॒न्नम् प्रति॑ । प्रत्ये॒तत् । ए॒तद् द॑धामि । द॒धा॒मि॒ स्वाहा᳚ । स्वाहा॑ घ॒र्मः । घ॒र्मो दे॒वान् । दे॒वाꣳ अपि॑ । अप्ये॑तु । ए॒त्वित्ये॑तु । \newline

\textbf{Jatai Paata} \newline

1. वे॒धस॒ इति॑ वे॒धसः॑ । \newline
2. पू॒ष्णस्तानि॒ तानि॑ पू॒ष्णः पू॒ष्णस्तानि॑ । \newline
3. तान्यप्यपि॒ तानि॒ तान्यपि॑ । \newline
4. अपि॑ व्र॒ते व्र॒ते ऽप्यपि॑ व्र॒ते । \newline
5. व्र॒त इ॑न्द्रवा॒यू इ॑न्द्रवा॒यू व्र॒ते व्र॒त इ॑न्द्रवा॒यू । \newline
6. इ॒न्द्र॒वा॒यू वि वीन्द्र॑वा॒यू इ॑न्द्रवा॒यू वि । \newline
7. इ॒न्द्र॒वा॒यू इती᳚न्द्र - वा॒यू । \newline
8. वि मु॑ञ्चताम् मुञ्चतां॒ ॅवि वि मु॑ञ्चताम् । \newline
9. मु॒ञ्च॒ता॒मिति॑ मुञ्चताम् । \newline
10. अभि॑न्नो घ॒र्मो घ॒र्मो ऽभि॒न्नो ऽभि॑न्नो घ॒र्मः । \newline
11. घ॒र्मो जी॒रदा॑नुर् जी॒रदा॑नुर् घ॒र्मो घ॒र्मो जी॒रदा॑नुः । \newline
12. जी॒रदा॑नु॒र् यतो॒ यतो॑ जी॒रदा॑नुर् जी॒रदा॑नु॒र् यतः॑ । \newline
13. जी॒रदा॑नु॒रिति॑ जी॒र - दा॒नुः॒ । \newline
14. यत॒ आत्त॒ आत्तो॒ यतो॒ यत॒ आत्तः॑ । \newline
15. आत्त॒स्तत् तदात्त॒ आत्त॒स्तत् । \newline
16. तद॑गन्नग॒न् तत् तद॑गन्न् । \newline
17. अ॒ग॒न् पुनः॒ पुन॑ रगन्नग॒न् पुनः॑ । \newline
18. पुन॒रिति॒ पुनः॑ । \newline
19. इ॒द्ध्मो वेदि॒र् वेदि॑रि॒द्ध्म इ॒द्ध्मो वेदिः॑ । \newline
20. वेदिः॑ परि॒धयः॑ परि॒धयो॑ वेदि॒र् वेदिः॑ परि॒धयः॑ । \newline
21. प॒रि॒धय॑श्च च परि॒धयः॑ परि॒धय॑श्च । \newline
22. प॒रि॒धय॒ इति॑ परि - धयः॑ । \newline
23. च॒ सर्वे॒ सर्वे॑ च च॒ सर्वे᳚ । \newline
24. सर्वे॑ य॒ज्ञ्स्य॑ य॒ज्ञ्स्य॒ सर्वे॒ सर्वे॑ य॒ज्ञ्स्य॑ । \newline
25. य॒ज्ञ्स्यायु॒रायु॑र् य॒ज्ञ्स्य॑ य॒ज्ञ्स्यायुः॑ । \newline
26. आयु॒ रन्वन्वायु॒ रायु॒रनु॑ । \newline
27. अनु॒ सꣳ स मन्वनु॒ सम् । \newline
28. सम् च॑रन्ति चरन्ति॒ सꣳ सम् च॑रन्ति । \newline
29. च॒र॒न्तीति॑ चरन्ति । \newline
30. त्रय॑स्त्रिꣳश॒त् तन्त॑व॒ स्तन्त॑व॒ स्त्रय॑स्त्रिꣳश॒त् त्रय॑स्त्रिꣳश॒त् तन्त॑वः । \newline
31. त्रय॑स्त्रिꣳश॒दिति॒ त्रयः॑ - त्रि॒(ग्म्॒)श॒त् । \newline
32. तन्त॑वो॒ ये ये तन्त॑व॒ स्तन्त॑वो॒ ये । \newline
33. ये वि॑तत्नि॒रे वि॑तत्नि॒रे ये ये वि॑तत्नि॒रे । \newline
34. वि॒त॒त्नि॒रे ये ये वि॑तत्नि॒रे वि॑तत्नि॒रे ये । \newline
35. वि॒त॒त्नि॒र इति॑ वि - त॒त्नि॒रे । \newline
36. य इ॒म मि॒मं ॅये य इ॒मम् । \newline
37. इ॒मं ॅय॒ज्ञ्ं ॅय॒ज्ञ् मि॒म मि॒मं ॅय॒ज्ञ्म् । \newline
38. य॒ज्ञ्ꣳ स्व॒धया᳚ स्व॒धया॑ य॒ज्ञ्ं ॅय॒ज्ञ्ꣳ स्व॒धया᳚ । \newline
39. स्व॒धया॒ दद॑न्ते॒ दद॑न्ते स्व॒धया᳚ स्व॒धया॒ दद॑न्ते । \newline
40. स्व॒धयेति॑ स्व - धया᳚ । \newline
41. दद॑न्ते॒ तेषा॒म् तेषा॒म् दद॑न्ते॒ दद॑न्ते॒ तेषा᳚म् । \newline
42. तेषा᳚म् छि॒न्नम् छि॒न्नम् तेषा॒म् तेषा᳚म् छि॒न्नम् । \newline
43. छि॒न्नम् प्रति॒ प्रति॑च् छि॒न्नम् छि॒न्नम् प्रति॑ । \newline
44. प्रत्ये॒तदे॒तत् प्रति॒ प्रत्ये॒तत् । \newline
45. ए॒तद् द॑धामि दधाम्ये॒तदे॒तद् द॑धामि । \newline
46. द॒धा॒मि॒ स्वाहा॒ स्वाहा॑ दधामि दधामि॒ स्वाहा᳚ । \newline
47. स्वाहा॑ घ॒र्मो घ॒र्मः स्वाहा॒ स्वाहा॑ घ॒र्मः । \newline
48. घ॒र्मो दे॒वान् दे॒वान् घ॒र्मो घ॒र्मो दे॒वान् । \newline
49. दे॒वाꣳ अप्यपि॑ दे॒वान् दे॒वाꣳ अपि॑ । \newline
50. अप्ये᳚त्वे॒त्वप्यप्ये॑तु । \newline
51. ए॒त्वित्ये॑तु । \newline

\textbf{Ghana Paata } \newline

1. वे॒धस॒ इति॑ वे॒धसः॑ । \newline
2. पू॒ष्णस्तानि॒ तानि॑ पू॒ष्णः पू॒ष्ण स्तान्यप्यपि॒ तानि॑ पू॒ष्णः पू॒ष्णस्तान्यपि॑ । \newline
3. तान्यप्यपि॒ तानि॒ तान्यपि॑ व्र॒ते व्र॒ते ऽपि॒ तानि॒ तान्यपि॑ व्र॒ते । \newline
4. अपि॑ व्र॒ते व्र॒ते ऽप्यपि॑ व्र॒त इ॑न्द्रवा॒यू इ॑न्द्रवा॒यू व्र॒ते ऽप्यपि॑ व्र॒त इ॑न्द्रवा॒यू । \newline
5. व्र॒त इ॑न्द्रवा॒यू इ॑न्द्रवा॒यू व्र॒ते व्र॒त इ॑न्द्रवा॒यू वि वीन्द्र॑वा॒यू व्र॒ते व्र॒त इ॑न्द्रवा॒यू वि । \newline
6. इ॒न्द्र॒वा॒यू वि वीन्द्र॑वा॒यू इ॑न्द्रवा॒यू वि मु॑ञ्चताम् मुञ्चतां॒ ॅवीन्द्र॑वा॒यू इ॑न्द्रवा॒यू वि मु॑ञ्चताम् । \newline
7. इ॒न्द्र॒वा॒यू इती᳚न्द्र - वा॒यू । \newline
8. वि मु॑ञ्चताम् मुञ्चतां॒ ॅवि वि मु॑ञ्चताम् । \newline
9. मु॒ञ्च॒ता॒मिति॑ मुञ्चताम् । \newline
10. अभि॑न्नो घ॒र्मो घ॒र्मो ऽभि॒न्नो ऽभि॑न्नो घ॒र्मो जी॒रदा॑नुर् जी॒रदा॑नुर् घ॒र्मो ऽभि॒न्नो ऽभि॑न्नो घ॒र्मो जी॒रदा॑नुः । \newline
11. घ॒र्मो जी॒रदा॑नुर् जी॒रदा॑नुर् घ॒र्मो घ॒र्मो जी॒रदा॑नु॒र् यतो॒ यतो॑ जी॒रदा॑नुर् घ॒र्मो घ॒र्मो जी॒रदा॑नु॒र् यतः॑ । \newline
12. जी॒रदा॑नु॒र् यतो॒ यतो॑ जी॒रदा॑नुर् जी॒रदा॑नु॒र् यत॒ आत्त॒ आत्तो॒ यतो॑ जी॒रदा॑नुर् जी॒रदा॑नु॒र् यत॒ आत्तः॑ । \newline
13. जी॒रदा॑नु॒रिति॑ जी॒र - दा॒नुः॒ । \newline
14. यत॒ आत्त॒ आत्तो॒ यतो॒ यत॒ आत्त॒स्तत् तदात्तो॒ यतो॒ यत॒ आत्त॒स्तत् । \newline
15. आत्त॒स्तत् तदात्त॒ आत्त॒स्तद॑गन् नग॒न् तदात्त॒ आत्त॒स्तद॑गन्न् । \newline
16. तद॑गन् नग॒न् तत् तद॑ग॒न् पुनः॒ पुन॑ रग॒न् तत् तद॑ग॒न् पुनः॑ । \newline
17. अ॒ग॒न् पुनः॒ पुन॑ रगन् नग॒न् पुनः॑ । \newline
18. पुन॒रिति॒ पुनः॑ । \newline
19. इ॒द्ध्मो वेदि॒र् वेदि॑रि॒द्ध्म इ॒द्ध्मो वेदिः॑ परि॒धयः॑ परि॒धयो॑ वेदि॑रि॒द्ध्म इ॒द्ध्मो वेदिः॑ परि॒धयः॑ । \newline
20. वेदिः॑ परि॒धयः॑ परि॒धयो॑ वेदि॒र् वेदिः॑ परि॒धय॑श्च च परि॒धयो॑ वेदि॒र् वेदिः॑ परि॒धय॑श्च । \newline
21. प॒रि॒धय॑श्च च परि॒धयः॑ परि॒धय॑श्च॒ सर्वे॒ सर्वे॑ च परि॒धयः॑ परि॒धय॑श्च॒ सर्वे᳚ । \newline
22. प॒रि॒धय॒ इति॑ परि - धयः॑ । \newline
23. च॒ सर्वे॒ सर्वे॑ च च॒ सर्वे॑ य॒ज्ञ्स्य॑ य॒ज्ञ्स्य॒ सर्वे॑ च च॒ सर्वे॑ य॒ज्ञ्स्य॑ । \newline
24. सर्वे॑ य॒ज्ञ्स्य॑ य॒ज्ञ्स्य॒ सर्वे॒ सर्वे॑ य॒ज्ञ्स्यायु॒रायु॑र् य॒ज्ञ्स्य॒ सर्वे॒ सर्वे॑ य॒ज्ञ्स्यायुः॑ । \newline
25. य॒ज्ञ्स्यायु॒रायु॑र् य॒ज्ञ्स्य॑ य॒ज्ञ्स्यायु॒ रन्वन्वायु॑र् य॒ज्ञ्स्य॑ य॒ज्ञ्स्यायु॒रनु॑ । \newline
26. आयु॒रन्वन्वायु॒ रायु॒रनु॒ सꣳ स मन्वायु॒ रायु॒रनु॒ सम् । \newline
27. अनु॒ सꣳ स मन्वनु॒ सम् च॑रन्ति चरन्ति॒ स मन्वनु॒ सम् च॑रन्ति । \newline
28. सम् च॑रन्ति चरन्ति॒ सꣳ सम् च॑रन्ति । \newline
29. च॒र॒न्तीति॑ चरन्ति । \newline
30. त्रय॑स्त्रिꣳश॒त् तन्त॑व॒ स्तन्त॑व॒ स्त्रय॑ स्त्रिꣳश॒त् त्रय॑ स्त्रिꣳश॒त् तन्त॑वो॒ ये ये तन्त॑व॒ स्त्रय॑ स्त्रिꣳश॒त् त्रय॑ स्त्रिꣳश॒त् तन्त॑वो॒ ये । \newline
31. त्रय॑स्त्रिꣳश॒दिति॒ त्रयः॑ - त्रि॒(ग्म्॒)श॒त् । \newline
32. तन्त॑वो॒ ये ये तन्त॑व॒ स्तन्त॑वो॒ ये वि॑तत्नि॒रे वि॑तत्नि॒रे ये तन्त॑व॒ स्तन्त॑वो॒ ये वि॑तत्नि॒रे । \newline
33. ये वि॑तत्नि॒रे वि॑तत्नि॒रे ये ये वि॑तत्नि॒रे ये ये वि॑तत्नि॒रे ये ये वि॑तत्नि॒रे ये । \newline
34. वि॒त॒त्नि॒रे ये ये वि॑तत्नि॒रे वि॑तत्नि॒रे य इ॒म मि॒मं ॅये वि॑तत्नि॒रे वि॑तत्नि॒रे य इ॒मम् । \newline
35. वि॒त॒त्नि॒र इति॑ वि - त॒त्नि॒रे । \newline
36. य इ॒म मि॒मं ॅये य इ॒मं ॅय॒ज्ञ्ं ॅय॒ज्ञ् मि॒मं ॅये य इ॒मं ॅय॒ज्ञ्म् । \newline
37. इ॒मं ॅय॒ज्ञ्ं ॅय॒ज्ञ् मि॒म मि॒मं ॅय॒ज्ञ्ꣳ स्व॒धया᳚ स्व॒धया॑ य॒ज्ञ् मि॒म मि॒मं ॅय॒ज्ञ्ꣳ स्व॒धया᳚ । \newline
38. य॒ज्ञ्ꣳ स्व॒धया᳚ स्व॒धया॑ य॒ज्ञ्ं ॅय॒ज्ञ्ꣳ स्व॒धया॒ दद॑न्ते॒ दद॑न्ते स्व॒धया॑ य॒ज्ञ्ं ॅय॒ज्ञ्ꣳ स्व॒धया॒ दद॑न्ते । \newline
39. स्व॒धया॒ दद॑न्ते॒ दद॑न्ते स्व॒धया᳚ स्व॒धया॒ दद॑न्ते॒ तेषा॒म् तेषा॒म् दद॑न्ते स्व॒धया᳚ स्व॒धया॒ दद॑न्ते॒ तेषा᳚म् । \newline
40. स्व॒धयेति॑ स्व - धया᳚ । \newline
41. दद॑न्ते॒ तेषा॒म् तेषा॒म् दद॑न्ते॒ दद॑न्ते॒ तेषा᳚म् छि॒न्नम् छि॒न्नम् तेषा॒म् दद॑न्ते॒ दद॑न्ते॒ तेषा᳚म् छि॒न्नम् । \newline
42. तेषा᳚म् छि॒न्नम् छि॒न्नम् तेषा॒म् तेषा᳚म् छि॒न्नम् प्रति॒ प्रति॑च् छि॒न्नम् तेषा॒म् तेषा᳚म् छि॒न्नम् प्रति॑ । \newline
43. छि॒न्नम् प्रति॒ प्रति॑च् छि॒न्नम् छि॒न्नम् प्रत्ये॒तदे॒तत् प्रति॑च् छि॒न्नम् छि॒न्नम् प्रत्ये॒तत् । \newline
44. प्रत्ये॒तदे॒तत् प्रति॒ प्रत्ये॒तद् द॑धामि दधाम्ये॒तत् प्रति॒ प्रत्ये॒तद् द॑धामि । \newline
45. ए॒तद् द॑धामि दधाम्ये॒तदे॒तद् द॑धामि॒ स्वाहा॒ स्वाहा॑ दधाम्ये॒तदे॒तद् द॑धामि॒ स्वाहा᳚ । \newline
46. द॒धा॒मि॒ स्वाहा॒ स्वाहा॑ दधामि दधामि॒ स्वाहा॑ घ॒र्मो घ॒र्मः स्वाहा॑ दधामि दधामि॒ स्वाहा॑ घ॒र्मः । \newline
47. स्वाहा॑ घ॒र्मो घ॒र्मः स्वाहा॒ स्वाहा॑ घ॒र्मो दे॒वान् दे॒वान् घ॒र्मः स्वाहा॒ स्वाहा॑ घ॒र्मो दे॒वान् । \newline
48. घ॒र्मो दे॒वान् दे॒वान् घ॒र्मो घ॒र्मो दे॒वाꣳ अप्यपि॑ दे॒वान् घ॒र्मो घ॒र्मो दे॒वाꣳ अपि॑ । \newline
49. दे॒वाꣳ अप्यपि॑ दे॒वान् दे॒वाꣳ अप्ये᳚त्वे॒त्वपि॑ दे॒वान् दे॒वाꣳ अप्ये॑तु । \newline
50. अप्ये᳚त्वे॒त्वप्यप्ये॑तु । \newline
51. ए॒त्वित्ये॑तु । \newline
\pagebreak
\markright{ TS 1.5.11.1  \hfill https://www.vedavms.in \hfill}
\addcontentsline{toc}{section}{ TS 1.5.11.1 }
\section*{ TS 1.5.11.1 }

\textbf{TS 1.5.11.1 } \newline
\textbf{Samhita Paata} \newline

वै॒श्वा॒न॒रो न॑ ऊ॒त्याऽऽ प्र या॑तु परा॒वतः॑ । अ॒ग्निरु॒क्थेन॒ वाह॑सा ॥ ऋ॒तावा॑नं ॅवैश्वान॒रमृ॒तस्य॒ ज्योति॑ष॒स्पतिं᳚ । अज॑स्रं घ॒र्ममी॑महे ॥ वै॒श्वा॒न॒रस्य॑ दꣳ॒॒सना᳚भ्यो बृ॒हदरि॑णा॒देकः॑ स्वप॒स्य॑या क॒विः । उ॒भा पि॒तरा॑ म॒हय॑न्नजायता॒ग्निर् द्यावा॑पृथि॒वी भूरि॑रेतसा ॥ पृ॒ष्टो दि॒वि पृ॒ष्टो अ॒ग्निः पृ॑थि॒व्यां पृ॒ष्टो विश्वा॒ ओष॑धी॒रा वि॑वेश । वै॒श्वा॒न॒रः सह॑सा पृ॒ष्टो अ॒ग्निः सनो॒ दिवा॒ स - [ ] \newline

\textbf{Pada Paata} \newline

वै॒श्वा॒न॒रः । नः॒ । ऊ॒त्या । आ । प्रेति॑ । या॒तु॒ । प॒रा॒वत॒ इति॑ परा - वतः॑ ॥ अ॒ग्निः । उ॒क्थेन॑ । वाह॑सा ॥ ऋ॒तावा॑न॒मित्यृ॒त-वा॒न॒म् । वै॒श्वा॒न॒रम् । ऋ॒तस्य॑ । ज्योति॑षः । पति᳚म् ॥ अज॑स्रम् । घ॒र्मम् । ई॒म॒हे॒ ॥ वै॒श्वा॒न॒रस्य॑ । दꣳ॒॒सना᳚भ्यः । बृ॒हत् । अरि॑णात् । एकः॑ । स्व॒प॒स्य॑येति॑ सु - अ॒प॒स्य॑या । क॒विः ॥ उ॒भा । पि॒तरा᳚ । म॒हयन्न्॑ । अ॒जा॒य॒त॒ । अ॒ग्निः । द्यावा॑पृथि॒वी इति॒ द्यावा᳚-पृ॒थि॒वी । भूरि॑रेत॒सेति॒ भूरि॑ - रे॒त॒सा॒ ॥ पृ॒ष्टः । दि॒वि । पृ॒ष्टः । अ॒ग्निः । पृ॒थि॒व्याम् । पृ॒ष्टः । विश्वाः᳚ । ओष॑धीः । एति॑ । वि॒वे॒श॒ ॥ वै॒श्वा॒न॒रः । सह॑सा । पृ॒ष्टः । अ॒ग्निः । सः । नः॒ । दिवा᳚ । सः ।  \newline


\textbf{Krama Paata} \newline

वै॒श्वा॒न॒रो नः॑ । न॒ ऊ॒त्या । ऊ॒त्या ऽऽ प्र । आ प्र । प्र या॑तु । या॒तु॒ प॒रा॒वतः॑ । प॒रा॒वत॒ इति॑ परा - वतः॑ ॥ अ॒ग्निरु॒क्थेन॑ । उ॒क्थेन॒ वाह॑सा । वाह॒सेति॒ वाह॑सा ॥ ऋ॒तावा॑नं ॅवैश्वान॒रम् । ऋ॒तावा॑न॒मित्यृ॒त - वा॒न॒म् । वै॒श्वा॒न॒रमृ॒तस्य॑ । ऋ॒तस्य॒ ज्योति॑षः । ज्योति॑ष॒स्पति᳚म् । पति॒मिति॒ पति᳚म् ॥ अज॑स्रम् घ॒र्मम् । घ॒र्ममी॑महे । ई॒म॒ह॒ इती॑महे ॥ वै॒श्वा॒न॒रस्य॑ दꣳ॒॒सना᳚भ्यः । दꣳ॒॒सना᳚भ्यो बृ॒हत् । बृ॒हदरि॑णात् । अरि॑णा॒देकः॑ । एकः॑ स्वप॒स्य॑या । स्व॒प॒स्य॑या क॒विः । स्व॒प॒स्य॑येति॑ सु - अ॒प॒स्य॑या । क॒विरिति॑ क॒विः ॥ उ॒भा पि॒तरा᳚ । पि॒तरा॑ म॒हयन्न्॑ । म॒हय॑न्नजायत । अ॒जा॒य॒ता॒ग्निः । अ॒ग्निर्,द्यावा॑पृथि॒वी । द्यावा॑पृथि॒वी भूरि॑रेतसा । द्यावा॑पृथि॒वी इति॒ द्यावा᳚ - पृ॒थि॒वी । भूरि॑रेत॒सेति॒ भूरि॑ - रे॒त॒सा॒ ॥ पृ॒ष्टो दि॒वि । दि॒वि पृ॒ष्टः । पृ॒ष्टो अ॒ग्निः । अ॒ग्निः पृ॑थि॒व्याम् । पृ॒थि॒व्याम् पृ॒ष्टः । पृ॒ष्टो विश्वाः᳚ । विश्वा॒ ओष॑धीः । ओष॑धी॒रा । आ वि॑वेश । वि॒वे॒शेति॑ विवेश ॥ वै॒श्वा॒न॒रः सह॑सा । सह॑सा पृ॒ष्टः । पृ॒ष्टो अ॒ग्निः । अ॒ग्निः सः । स नः॑ । नो॒ दिवा᳚ । दिवा॒ सः । स रि॒षः \newline

\textbf{Jatai Paata} \newline

1. वै॒श्वा॒न॒रो नो॑ नो वैश्वान॒रो वै᳚श्वान॒रो नः॑ । \newline
2. न॒ ऊ॒त्योत्या नो॑ न ऊ॒त्या । \newline
3. ऊ॒त्या ऽऽप्र प्रोत्योत्या ऽऽप्र । \newline
4. आ प्र प्रा प्र । \newline
5. प्र या॑तु यातु॒ प्र प्र या॑तु । \newline
6. या॒तु॒ प॒रा॒वतः॑ परा॒वतो॑ यातु यातु परा॒वतः॑ । \newline
7. प॒रा॒वत॒ इति॑ परा - वतः॑ । \newline
8. अ॒ग्नि रु॒क्थेनो॒क्थे ना॒ग्नि र॒ग्निरु॒क्थेन॑ । \newline
9. उ॒क्थेन॒ वाह॑सा॒ वाह॑सो॒क्थेनो॒क्थेन॒ वाह॑सा । \newline
10. वाह॒सेति॒ वाह॑सा । \newline
11. ऋ॒तावा॑नं ॅवैश्वान॒रं ॅवै᳚श्वान॒र मृ॒तावा॑न मृ॒तावा॑नं ॅवैश्वान॒रम् । \newline
12. ऋ॒तावा॑न॒मित्यृ॒त - वा॒न॒म् । \newline
13. वै॒श्वा॒न॒र मृ॒तस्य॒ र्तस्य॑ वैश्वान॒रं ॅवै᳚श्वान॒र मृ॒तस्य॑ । \newline
14. ऋ॒तस्य॒ ज्योति॑षो॒ ज्योति॑ष ऋ॒तस्य॒ र्तस्य॒ ज्योति॑षः । \newline
15. ज्योति॑ष॒ स्पति॒म् पति॒म् ज्योति॑षो॒ ज्योति॑ष॒ स्पति᳚म् । \newline
16. पति॒मिति॒ पति᳚म् । \newline
17. अज॑स्रम् घ॒र्मम् घ॒र्म मज॑स्र॒ मज॑स्रम् घ॒र्मम् । \newline
18. घ॒र्म मी॑मह ईमहे घ॒र्मम् घ॒र्म मी॑महे । \newline
19. ई॒म॒ह॒ इती॑महे । \newline
20. वै॒श्वा॒न॒रस्य॑ द॒(ग्म्॒)सना᳚भ्यो द॒(ग्म्॒)सना᳚भ्यो वैश्वान॒रस्य॑ वैश्वान॒रस्य॑ द॒(ग्म्॒)सना᳚भ्यः । \newline
21. द॒(ग्म्॒)सना᳚भ्यो बृ॒हद् बृ॒हद् द॒(ग्म्॒)सना᳚भ्यो द॒(ग्म्॒)सना᳚भ्यो बृ॒हत् । \newline
22. बृ॒ह दरि॑णा॒दरि॑णाद् बृ॒हद् बृ॒हदरि॑णात् । \newline
23. अरि॑णा॒देक॒ एको॒ अरि॑णा॒ दरि॑णा॒देकः॑ । \newline
24. एकः॑ स्वप॒स्य॑या स्वप॒स्य॑यैक॒ एकः॑ स्वप॒स्य॑या । \newline
25. स्व॒प॒स्य॑या क॒विः क॒विः स्व॑प॒स्य॑या स्वप॒स्य॑या क॒विः । \newline
26. स्व॒प॒स्य॑येति॑ सु - अ॒प॒स्य॑या । \newline
27. क॒विरिति॑ क॒विः । \newline
28. उ॒भा पि॒तरा॑ पि॒तरो॒भोभा पि॒तरा᳚ । \newline
29. पि॒तरा॑ म॒हय॑न् म॒हय॑न् पि॒तरा॑ पि॒तरा॑ म॒हयन्न्॑ । \newline
30. म॒हय॑ न्नजायताजायत म॒हय॑न् म॒हय॑न्नजायत । \newline
31. अ॒जा॒य॒ता॒ग्नि र॒ग्नि र॑जायताजायता॒ग्निः । \newline
32. अ॒ग्निर् द्यावा॑पृथि॒वी द्यावा॑पृथि॒वी अ॒ग्निर॒ग्निर् द्यावा॑पृथि॒वी । \newline
33. द्यावा॑पृथि॒वी भूरि॑रेतसा॒ भूरि॑रेतसा॒ द्यावा॑पृथि॒वी द्यावा॑पृथि॒वी भूरि॑रेतसा । \newline
34. द्यावा॑पृथि॒वी इति॒ द्यावा᳚ - पृ॒थि॒वी । \newline
35. भूरि॑रेत॒सेति॒ भूरि॑ - रे॒त॒सा॒ । \newline
36. पृ॒ष्टो दि॒वि दि॒वि पृ॒ष्टः पृ॒ष्टो दि॒वि । \newline
37. दि॒वि पृ॒ष्टः पृ॒ष्टो दि॒वि दि॒वि पृ॒ष्टः । \newline
38. पृ॒ष्टो अ॒ग्निर॒ग्निः पृ॒ष्टः पृ॒ष्टो अ॒ग्निः । \newline
39. अ॒ग्निः पृ॑थि॒व्याम् पृ॑थि॒व्या म॒ग्निर॒ग्निः पृ॑थि॒व्याम् । \newline
40. पृ॒थि॒व्याम् पृ॒ष्टः पृ॒ष्टः पृ॑थि॒व्याम् पृ॑थि॒व्याम् पृ॒ष्टः । \newline
41. पृ॒ष्टो विश्वा॒ विश्वाः᳚ पृ॒ष्टः पृ॒ष्टो विश्वाः᳚ । \newline
42. विश्वा॒ ओष॑धी॒ रोष॑धी॒र् विश्वा॒ विश्वा॒ ओष॑धीः । \newline
43. ओष॑धी॒ रौष॑धी॒ रोष॑धी॒रा । \newline
44. आ वि॑वेश विवे॒शा वि॑वेश । \newline
45. वि॒वे॒शेति॑ विवेश । \newline
46. वै॒श्वा॒न॒रः सह॑सा॒ सह॑सा वैश्वान॒रो वै᳚श्वान॒रः सह॑सा । \newline
47. सह॑सा पृ॒ष्टः पृ॒ष्टः सह॑सा॒ सह॑सा पृ॒ष्टः । \newline
48. पृ॒ष्टो अ॒ग्निर॒ग्निः पृ॒ष्टः पृ॒ष्टो अ॒ग्निः । \newline
49. अ॒ग्निः स सो अ॒ग्निर॒ग्निः सः । \newline
50. स नो॑ नः॒ स स नः॑ । \newline
51. नो॒ दिवा॒ दिवा॑ नो नो॒ दिवा᳚ । \newline
52. दिवा॒ स स दिवा॒ दिवा॒ सः । \newline
53. स रि॒षो रि॒षः स स रि॒षः । \newline

\textbf{Ghana Paata } \newline

1. वै॒श्वा॒न॒रो नो॑ नो वैश्वान॒रो वै᳚श्वान॒रो न॑ ऊ॒त्योत्या नो॑ वैश्वान॒रो वै᳚श्वान॒रो न॑ ऊ॒त्या । \newline
2. न॒ ऊ॒त्योत्या नो॑ न ऊ॒त्या प्रा प्रोत्या नो॑ न ऊ॒त्या प्र । \newline
3. ऊ॒त्या प्रा प्रोत्योत्याऽऽ प्र या॑तु या॒त्या प्रोत्योत्याऽऽ प्र या॑तु । \newline
4. आ प्र प्रा प्र या॑तु यातु॒ प्रा प्र या॑तु । \newline
5. प्र या॑तु यातु॒ प्र प्र या॑तु परा॒वतः॑ परा॒वतो॑ यातु॒ प्र प्र या॑तु परा॒वतः॑ । \newline
6. या॒तु॒ प॒रा॒वतः॑ परा॒वतो॑ यातु यातु परा॒वतः॑ । \newline
7. प॒रा॒वत॒ इति॑ परा - वतः॑ । \newline
8. अ॒ग्नि रु॒क्थेनो॒क्थे ना॒ग्नि र॒ग्नि रु॒क्थेन॒ वाह॑सा॒ वाह॑सो॒क्थे ना॒ग्नि र॒ग्नि रु॒क्थेन॒ वाह॑सा । \newline
9. उ॒क्थेन॒ वाह॑सा॒ वाह॑सो॒क्थे नो॒क्थेन॒ वाह॑सा । \newline
10. वाह॒सेति॒ वाह॑सा । \newline
11. ऋ॒तावा॑नं ॅवैश्वान॒रं ॅवै᳚श्वान॒र मृ॒तावा॑न मृ॒तावा॑नं ॅवैश्वान॒र मृ॒तस्य॒ र्तस्य॑ वैश्वान॒र मृ॒तावा॑न मृ॒तावा॑नं ॅवैश्वान॒र मृ॒तस्य॑ । \newline
12. ऋ॒तावा॑न॒मित्यृ॒त - वा॒न॒म् । \newline
13. वै॒श्वा॒न॒र मृ॒तस्य॒ र्तस्य॑ वैश्वान॒रं ॅवै᳚श्वान॒र मृ॒तस्य॒ ज्योति॑षो॒ ज्योति॑ष ऋ॒तस्य॑ वैश्वान॒रं ॅवै᳚श्वान॒र मृ॒तस्य॒ ज्योति॑षः । \newline
14. ऋ॒तस्य॒ ज्योति॑षो॒ ज्योति॑ष ऋ॒तस्य॒ र्तस्य॒ ज्योति॑ष॒ स्पति॒म् पति॒म् ज्योति॑ष ऋ॒तस्य॒ र्तस्य॒ ज्योति॑ष॒ स्पति᳚म् । \newline
15. ज्योति॑ष॒ स्पति॒म् पति॒म् ज्योति॑षो॒ ज्योति॑ष॒ स्पति᳚म् । \newline
16. पति॒मिति॒ पति᳚म् । \newline
17. अज॑स्रम् घ॒र्मम् घ॒र्म मज॑स्र॒ मज॑स्रम् घ॒र्म मी॑मह ईमहे घ॒र्म मज॑स्र॒ मज॑स्रम् घ॒र्म मी॑महे । \newline
18. घ॒र्म मी॑मह ईमहे घ॒र्मम् घ॒र्म मी॑महे । \newline
19. ई॒म॒ह॒ इती॑महे । \newline
20. वै॒श्वा॒न॒रस्य॑ द॒(ग्म्॒)सना᳚भ्यो द॒(ग्म्॒)सना᳚भ्यो वैश्वान॒रस्य॑ वैश्वान॒रस्य॑ द॒(ग्म्॒)सना᳚भ्यो बृ॒हद् बृ॒हद् द॒(ग्म्॒)सना᳚भ्यो वैश्वान॒रस्य॑ वैश्वान॒रस्य॑ द॒(ग्म्॒)सना᳚भ्यो बृ॒हत् । \newline
21. द॒(ग्म्॒)सना᳚भ्यो बृ॒हद् बृ॒हद् द॒(ग्म्॒)सना᳚भ्यो द॒(ग्म्॒)सना᳚भ्यो बृ॒ह दरि॑णा॒ दरि॑णाद् बृ॒हद् द॒(ग्म्॒)सना᳚भ्यो द॒(ग्म्॒)सना᳚भ्यो बृ॒हदरि॑णात् । \newline
22. बृ॒हदरि॑णा॒ दरि॑णाद् बृ॒हद् बृ॒हदरि॑णा॒देक॒ एको॒ अरि॑णाद् बृ॒हद् बृ॒हदरि॑णा॒देकः॑ । \newline
23. अरि॑णा॒देक॒ एको॒ अरि॑णा॒ दरि॑णा॒देकः॑ स्वप॒स्य॑या स्वप॒स्य॑यैको॒ अरि॑णा॒ दरि॑णा॒देकः॑ स्वप॒स्य॑या । \newline
24. एकः॑ स्वप॒स्य॑या स्वप॒स्य॑यैक॒ एकः॑ स्वप॒स्य॑या क॒विः क॒विः स्व॑प॒स्य॑यैक॒ एकः॑ स्वप॒स्य॑या क॒विः । \newline
25. स्व॒प॒स्य॑या क॒विः क॒विः स्व॑प॒स्य॑या स्वप॒स्य॑या क॒विः । \newline
26. स्व॒प॒स्य॑येति॑ सु - अ॒प॒स्य॑या । \newline
27. क॒विरिति॑ क॒विः । \newline
28. उ॒भा पि॒तरा॑ पि॒तरो॒भोभा पि॒तरा॑ म॒हय॑न् म॒हय॑न् पि॒तरो॒भोभा पि॒तरा॑ म॒हयन्न्॑ । \newline
29. पि॒तरा॑ म॒हय॑न् म॒हय॑न् पि॒तरा॑ पि॒तरा॑ म॒हय॑न् नजायताजायत म॒हय॑न् पि॒तरा॑ पि॒तरा॑ म॒हय॑न् नजायत । \newline
30. म॒हय॑न् नजायताजायत म॒हय॑न् म॒हय॑न् नजायता॒ग्नि र॒ग्निर॑जायत म॒हय॑न् म॒हय॑न् नजायता॒ग्निः । \newline
31. अ॒जा॒य॒ ता॒ग्नि र॒ग्नि र॑जायताजाय ता॒ग्निर् द्यावा॑पृथि॒वी द्यावा॑पृथि॒वी अ॒ग्नि र॑जायताजाय ता॒ग्निर् द्यावा॑पृथि॒वी । \newline
32. अ॒ग्निर् द्यावा॑पृथि॒वी द्यावा॑पृथि॒वी अ॒ग्निर॒ग्निर् द्यावा॑पृथि॒वी भूरि॑रेतसा॒ भूरि॑रेतसा॒ द्यावा॑पृथि॒वी अ॒ग्निर॒ग्निर् द्यावा॑पृथि॒वी भूरि॑रेतसा । \newline
33. द्यावा॑पृथि॒वी भूरि॑रेतसा॒ भूरि॑रेतसा॒ द्यावा॑पृथि॒वी द्यावा॑पृथि॒वी भूरि॑रेतसा । \newline
34. द्यावा॑पृथि॒वी इति॒ द्यावा᳚ - पृ॒थि॒वी । \newline
35. भूरि॑रेत॒सेति॒ भूरि॑ - रे॒त॒सा॒ । \newline
36. पृ॒ष्टो दि॒वि दि॒वि पृ॒ष्टः पृ॒ष्टो दि॒वि पृ॒ष्टः पृ॒ष्टो दि॒वि पृ॒ष्टः पृ॒ष्टो दि॒वि पृ॒ष्टः । \newline
37. दि॒वि पृ॒ष्टः पृ॒ष्टो दि॒वि दि॒वि पृ॒ष्टो अ॒ग्निर॒ग्निः पृ॒ष्टो दि॒वि दि॒वि पृ॒ष्टो अ॒ग्निः । \newline
38. पृ॒ष्टो अ॒ग्निर॒ग्निः पृ॒ष्टः पृ॒ष्टो अ॒ग्निः पृ॑थि॒व्याम् पृ॑थि॒व्या म॒ग्निः पृ॒ष्टः पृ॒ष्टो अ॒ग्निः पृ॑थि॒व्याम् । \newline
39. अ॒ग्निः पृ॑थि॒व्याम् पृ॑थि॒व्या म॒ग्निर॒ग्निः पृ॑थि॒व्याम् पृ॒ष्टः पृ॒ष्टः पृ॑थि॒व्या म॒ग्निर॒ग्निः पृ॑थि॒व्याम् पृ॒ष्टः । \newline
40. पृ॒थि॒व्याम् पृ॒ष्टः पृ॒ष्टः पृ॑थि॒व्याम् पृ॑थि॒व्याम् पृ॒ष्टो विश्वा॒ विश्वाः᳚ पृ॒ष्टः पृ॑थि॒व्याम् पृ॑थि॒व्याम् पृ॒ष्टो विश्वाः᳚ । \newline
41. पृ॒ष्टो विश्वा॒ विश्वाः᳚ पृ॒ष्टः पृ॒ष्टो विश्वा॒ ओष॑धी॒ रोष॑धी॒र् विश्वाः᳚ पृ॒ष्टः पृ॒ष्टो विश्वा॒ ओष॑धीः । \newline
42. विश्वा॒ ओष॑धी॒ रोष॑धी॒र् विश्वा॒ विश्वा॒ ओष॑धी॒ रौष॑धी॒र् विश्वा॒ विश्वा॒ ओष॑धी॒रा । \newline
43. ओष॑धी॒ रौष॑धी॒ रोष॑धी॒रा वि॑वेश विवे॒शौष॑धी॒ रोष॑धी॒रा वि॑वेश । \newline
44. आ वि॑वेश विवे॒शा वि॑वेश । \newline
45. वि॒वे॒शेति॑ विवेश । \newline
46. वै॒श्वा॒न॒रः सह॑सा॒ सह॑सा वैश्वान॒रो वै᳚श्वान॒रः सह॑सा पृ॒ष्टः पृ॒ष्टः सह॑सा वैश्वान॒रो वै᳚श्वान॒रः सह॑सा पृ॒ष्टः । \newline
47. सह॑सा पृ॒ष्टः पृ॒ष्टः सह॑सा॒ सह॑सा पृ॒ष्टो अ॒ग्निर॒ग्निः पृ॒ष्टः सह॑सा॒ सह॑सा पृ॒ष्टो अ॒ग्निः । \newline
48. पृ॒ष्टो अ॒ग्निर॒ग्निः पृ॒ष्टः पृ॒ष्टो अ॒ग्निः स सो** अ॒ग्निः पृ॒ष्टः पृ॒ष्टो अ॒ग्निः सः । \newline
49. अ॒ग्निः स सो** अ॒ग्निर॒ग्निः स नो॑ नः सो** अ॒ग्निर॒ग्निः स नः॑ । \newline
50. स नो॑ नः॒ स स नो॒ दिवा॒ दिवा॑ नः॒ स स नो॒ दिवा᳚ । \newline
51. नो॒ दिवा॒ दिवा॑ नो नो॒ दिवा॒ स स दिवा॑ नो नो॒ दिवा॒ सः । \newline
52. दिवा॒ स स दिवा॒ दिवा॒ स रि॒षो रि॒षः स दिवा॒ दिवा॒ स रि॒षः । \newline
53. स रि॒षो रि॒षः स स रि॒षः पा॑तु पातु रि॒षः स स रि॒षः पा॑तु । \newline
\pagebreak
\markright{ TS 1.5.11.2  \hfill https://www.vedavms.in \hfill}
\addcontentsline{toc}{section}{ TS 1.5.11.2 }
\section*{ TS 1.5.11.2 }

\textbf{TS 1.5.11.2 } \newline
\textbf{Samhita Paata} \newline

रि॒षः पा॑तु॒ नक्तं᳚ ॥ जा॒तो यद॑ग्ने॒ भुव॑ना॒ व्यख्यः॑ प॒शुं न गो॒पा इर्यः॒ परि॑ज्मा । वैश्वा॑नर॒ ब्रह्म॑णे विन्द गा॒तुं ॅयू॒यं पा॑त स्व॒स्तिभिः॒ सदा॑ नः ॥ त्वम॑ग्ने शो॒चिषा॒ शोशु॑चान॒ आ रोद॑सी अपृणा॒ जाय॑मानः । त्वं दे॒वाꣳ अ॒भिश॑स्तेरमुञ्चो॒ वैश्वा॑नर जातवेदो महि॒त्वा ॥ अ॒स्माक॑मग्ने म॒घव॑थ्सु धार॒याना॑मि क्ष॒त्रम॒जरꣳ॑ सु॒वीर्यं᳚ । व॒यं ज॑येम श॒तिनꣳ॑ सह॒स्रिणं॒ ॅवैश्वा॑नर॒ - [ ] \newline

\textbf{Pada Paata} \newline

रि॒षः । पा॒तु॒ । नक्त᳚म् ॥ जा॒तः । यत् । अ॒ग्ने॒ । भुव॑ना । व्यख्य॒ इति॑ वि - अख्यः॑ । प॒शुम् । न । गो॒पा इति॑ गो - पाः । इर्यः॑ । परि॒ज्मेति॒ परि॑ - ज्मा॒ ॥ वैश्वा॑नर । ब्रह्म॑णे । वि॒न्द॒ । गा॒तुम् । यू॒यम् । पा॒त॒ । स्व॒स्तिभि॒रिति॑ स्व॒स्ति - भिः॒ । सदा᳚ । नः॒ ॥ त्वम् । अ॒ग्ने॒ । शो॒चिषा᳚ । शोशु॑चानः । एति॑ । रोद॑सी॒ इति॑ । अ॒पृ॒णाः॒ । जाय॑मानः ॥ त्वम् । दे॒वान् । अ॒भिश॑स्ते॒रित्य॒भि - श॒स्तेः॒ । अ॒मु॒ञ्चः॒ । वैश्वा॑नर । जा॒त॒वे॒द॒ इति॑ जात - वे॒दः॒ । म॒हि॒त्वेति॑ महि - त्वा ॥ अ॒स्माक᳚म् । अ॒ग्ने॒ । म॒घव॒थ्स्विति॑ म॒घव॑त् - सु॒ । धा॒र॒य॒ । अना॑मि । क्ष॒त्रम् । अ॒जर᳚म् । सु॒वीर्य॒मिति॑ सु - वीर्य᳚म् ॥ व॒यम् । ज॒ये॒म॒ । श॒तिन᳚म् । स॒ह॒स्रिण᳚म् । वैश्वा॑नर ।  \newline


\textbf{Krama Paata} \newline

रि॒षः पा॑तु । पा॒तु॒ नक्त᳚म् । नक्त॒मिति॒ नक्त᳚म् ॥ जा॒तो यत् । यद॑ग्ने । अ॒ग्ने॒ भुव॑ना । भुव॑ना॒ व्यख्यः॑ । व्यख्यः॑ प॒शुम् । व्यख्य॒ इति॑ वि - अख्यः॑ । प॒शुन्न । न गो॒पाः । गो॒पा इर्यः॑ । गो॒पा इति॑ गो - पाः । इर्यः॒ परि॑ज्मा । परि॒ज्मेति॒ परि॑ - ज्मा॒ ॥ वैश्वा॑नर॒ ब्रह्म॑णे । ब्रह्म॑णे विन्द । वि॒न्द॒ गा॒तुम् । गा॒तुं ॅयू॒यम् । यू॒यम् पा॑त । पा॒त॒ स्व॒स्तिभिः॑ । स्व॒स्तिभिः॒ सदा᳚ । स्व॒स्तिभि॒रिति॑ स्व॒स्ति - भिः॒ । सदा॑ नः । न॒ इति॑ नः ॥ त्वम॑ग्ने । अ॒ग्ने॒ शो॒चिषा᳚ । शो॒चिषा॒ शोशु॑चानः । शोशु॑चान॒ आ । आ रोद॑सी । रोद॑सी अपृणाः । रोद॑सी॒ इति॒ रोद॑सी । अ॒पृ॒णा॒ जाय॑मानः । जाय॑मान॒ इति॒ जाय॑मानः ॥ त्वम् दे॒वान् । दे॒वाꣳ अ॒भिश॑स्तेः । अ॒भिश॑स्तेरमुञ्चः । अ॒भिश॑स्ते॒रित्य॒भि - श॒स्तेः॒ । अ॒मु॒ञ्चो॒ वैश्वा॑नर । वैश्वा॑नर जातवेदः । जा॒त॒वे॒दो॒ म॒हि॒त्वा । जा॒त॒वे॒द॒ इति॑ जात - वे॒दः॒ । म॒हि॒त्वेति॑ महि - त्वा । अ॒स्माक॑मग्ने । अ॒ग्ने॒ म॒घव॑थ्सु । म॒घव॑थ्सु धार॒य । म॒घव॒थ्स्विति॑ म॒घव॑त् - सु । धा॒र॒याना॑मि । अना॑मि क्ष॒त्रम् । क्ष॒त्रम॒जर᳚म् । अ॒जरꣳ॑ सु॒वीर्य᳚म् । सु॒वीर्य॒मिति॑ सु - वीर्य᳚म् ॥ व॒यम् ज॑येम । ज॒ये॒म॒ श॒तिन᳚म् । श॒तिनꣳ॑ सह॒स्रिण᳚म् । स॒ह॒स्रिणं॒ ॅवैश्वा॑नर । वैश्वा॑नर॒ वाज᳚म् \newline

\textbf{Jatai Paata} \newline

1. रि॒षः पा॑तु पातु रि॒षो रि॒षः पा॑तु । \newline
2. पा॒तु॒ नक्त॒न्नक्त॑म् पातु पातु॒ नक्त᳚म् । \newline
3. नक्त॒मिति॒ नक्त᳚म् । \newline
4. जा॒तो यद् यज् जा॒तो जा॒तो यत् । \newline
5. यद॑ग्ने अग्ने॒ यद् यद॑ग्ने । \newline
6. अ॒ग्ने॒ भुव॑ना॒ भुव॑ना ऽग्ने अग्ने॒ भुव॑ना । \newline
7. भुव॑ना॒ व्यख्यो॒ व्यख्यो॒ भुव॑ना॒ भुव॑ना॒ व्यख्यः॑ । \newline
8. व्यख्यः॑ प॒शुम् प॒शुं ॅव्यख्यो॒ व्यख्यः॑ प॒शुम् । \newline
9. व्यख्य॒ इति॑ वि - अख्यः॑ । \newline
10. प॒शुन्न न प॒शुम् प॒शुन्न । \newline
11. न गो॒पा गो॒पा न न गो॒पाः । \newline
12. गो॒पा इर्य॒ इर्यो॑ गो॒पा गो॒पा इर्यः॑ । \newline
13. गो॒पा इति॑ गो - पाः । \newline
14. इर्यः॒ परि॑ज्मा॒ परि॒ज्मेर्य॒ इर्यः॒ परि॑ज्मा । \newline
15. परि॒ज्मेति॒ परि॑ - ज्मा॒ । \newline
16. वैश्वा॑नर॒ ब्रह्म॑णे॒ ब्रह्म॑णे॒ वैश्वा॑नर॒ वैश्वा॑नर॒ ब्रह्म॑णे । \newline
17. ब्रह्म॑णे विन्द विन्द॒ ब्रह्म॑णे॒ ब्रह्म॑णे विन्द । \newline
18. वि॒न्द॒ गा॒तुम् गा॒तुं ॅवि॑न्द विन्द गा॒तुम् । \newline
19. गा॒तुं ॅयू॒यं ॅयू॒यम् गा॒तुम् गा॒तुं ॅयू॒यम् । \newline
20. यू॒यम् पा॑त पात यू॒यं ॅयू॒यम् पा॑त । \newline
21. पा॒त॒ स्व॒स्तिभिः॑ स्व॒स्तिभिः॑ पात पात स्व॒स्तिभिः॑ । \newline
22. स्व॒स्तिभिः॒ सदा॒ सदा᳚ स्व॒स्तिभिः॑ स्व॒स्तिभिः॒ सदा᳚ । \newline
23. स्व॒स्तिभि॒रिति॑ स्व॒स्ति - भिः॒ । \newline
24. सदा॑ नो नः॒ सदा॒ सदा॑ नः । \newline
25. न॒ इति॑ नः । \newline
26. त्व म॑ग्ने अग्ने॒ त्वम् त्व म॑ग्ने । \newline
27. अ॒ग्ने॒ शो॒चिषा॑ शो॒चिषा᳚ ऽग्ने अग्ने शो॒चिषा᳚ । \newline
28. शो॒चिषा॒ शोशु॑चानः॒ शोशु॑चानः शो॒चिषा॑ शो॒चिषा॒ शोशु॑चानः । \newline
29. शोशु॑चान॒ आ शोशु॑चानः॒ शोशु॑चान॒ आ । \newline
30. आ रोद॑सी॒ रोद॑सी॒ आ रोद॑सी । \newline
31. रोद॑सी अपृणा अपृणा॒ रोद॑सी॒ रोद॑सी अपृणाः । \newline
32. रोद॑सी॒ इति॒ रोद॑सी । \newline
33. अ॒पृ॒णा॒ जाय॑मानो॒ जाय॑मानो ऽपृणा अपृणा॒ जाय॑मानः । \newline
34. जाय॑मान॒ इति॒ जाय॑मानः । \newline
35. त्वम् दे॒वान् दे॒वान् त्वम् त्वम् दे॒वान् । \newline
36. दे॒वाꣳ अ॒भिश॑स्ते र॒भिश॑स्तेर् दे॒वान् दे॒वाꣳ अ॒भिश॑स्तेः । \newline
37. अ॒भिश॑स्ते रमुञ्चो अमुञ्चो अ॒भिश॑स्ते र॒भिश॑स्ते रमुञ्चः । \newline
38. अ॒भिश॑स्ते॒रित्य॒भि - श॒स्तेः॒ । \newline
39. अ॒मु॒ञ्चो॒ वैश्वा॑नर॒ वैश्वा॑नरामुञ्चो अमुञ्चो॒ वैश्वा॑नर । \newline
40. वैश्वा॑नर जातवेदो जातवेदो॒ वैश्वा॑नर॒ वैश्वा॑नर जातवेदः । \newline
41. जा॒त॒वे॒दो॒ म॒हि॒त्वा म॑हि॒त्वा जा॑तवेदो जातवेदो महि॒त्वा । \newline
42. जा॒त॒वे॒द॒ इति॑ जात - वे॒दः॒ । \newline
43. म॒हि॒त्वेति॑ महि - त्वा । \newline
44. अ॒स्माक॑ मग्ने अग्ने॒ ऽस्माक॑ म॒स्माक॑ मग्ने । \newline
45. अ॒ग्ने॒ म॒घव॑थ्सु म॒घव॑थ्स्वग्ने अग्ने म॒घव॑थ्सु । \newline
46. म॒घव॑थ्सु धारय धारय म॒घव॑थ्सु म॒घव॑थ्सु धारय । \newline
47. म॒घव॒थ्स्विति॑ म॒घव॑त् - सु॒ । \newline
48. धा॒र॒याना॒म्यना॑मि धारय धार॒याना॑मि । \newline
49. अना॑मि क्ष॒त्रम् क्ष॒त्र मना॒म्यना॑मि क्ष॒त्रम् । \newline
50. क्ष॒त्र म॒जर॑ म॒जर॑म् क्ष॒त्रम् क्ष॒त्र म॒जर᳚म् । \newline
51. अ॒जर(ग्म्॑) सु॒वीर्य(ग्म्॑) सु॒वीर्य॑ म॒जर॑ म॒जर(ग्म्॑) सु॒वीर्य᳚म् । \newline
52. सु॒वीर्य॒मिति॑ सु - वीर्य᳚म् । \newline
53. व॒यम् ज॑येम जयेम व॒यं ॅव॒यम् ज॑येम । \newline
54. ज॒ये॒म॒ श॒तिन(ग्म्॑) श॒तिन॑म् जयेम जयेम श॒तिन᳚म् । \newline
55. श॒तिन(ग्म्॑) सह॒स्रिण(ग्म्॑) सह॒स्रिण(ग्म्॑) श॒तिन(ग्म्॑) श॒तिन(ग्म्॑) सह॒स्रिण᳚म् । \newline
56. स॒ह॒स्रिणं॒ ॅवैश्वा॑नर॒ वैश्वा॑नर सह॒स्रिण(ग्म्॑) सह॒स्रिणं॒ ॅवैश्वा॑नर । \newline
57. वैश्वा॑नर॒ वाजं॒ ॅवाजं॒ ॅवैश्वा॑नर॒ वैश्वा॑नर॒ वाज᳚म् । \newline

\textbf{Ghana Paata } \newline

1. रि॒षः पा॑तु पातु रि॒षो रि॒षः पा॑तु॒ नक्त॒न्नक्त॑म् पातु रि॒षो रि॒षः पा॑तु॒ नक्त᳚म् । \newline
2. पा॒तु॒ नक्त॒न्नक्त॑म् पातु पातु॒ नक्त᳚म् । \newline
3. नक्त॒मिति॒ नक्त᳚म् । \newline
4. जा॒तो यद् यज् जा॒तो जा॒तो यद॑ग्ने अग्ने॒ यज् जा॒तो जा॒तो यद॑ग्ने । \newline
5. यद॑ग्ने अग्ने॒ यद् यद॑ग्ने॒ भुव॑ना॒ भुव॑ना ऽग्ने॒ यद् यद॑ग्ने॒ भुव॑ना । \newline
6. अ॒ग्ने॒ भुव॑ना॒ भुव॑ना ऽग्ने अग्ने॒ भुव॑ना॒ व्यख्यो॒ व्यख्यो॒ भुव॑ना ऽग्ने अग्ने॒ भुव॑ना॒ व्यख्यः॑ । \newline
7. भुव॑ना॒ व्यख्यो॒ व्यख्यो॒ भुव॑ना॒ भुव॑ना॒ व्यख्यः॑ प॒शुम् प॒शुं ॅव्यख्यो॒ भुव॑ना॒ भुव॑ना॒ व्यख्यः॑ प॒शुम् । \newline
8. व्यख्यः॑ प॒शुम् प॒शुं ॅव्यख्यो॒ व्यख्यः॑ प॒शुन्न न प॒शुं ॅव्यख्यो॒ व्यख्यः॑ प॒शुन्न । \newline
9. व्यख्य॒ इति॑ वि - अख्यः॑ । \newline
10. प॒शुन्न न प॒शुम् प॒शुन्न गो॒पा गो॒पा न प॒शुम् प॒शुन्न गो॒पाः । \newline
11. न गो॒पा गो॒पा न न गो॒पा इर्य॒ इर्यो॑ गो॒पा न न गो॒पा इर्यः॑ । \newline
12. गो॒पा इर्य॒ इर्यो॑ गो॒पा गो॒पा इर्यः॒ परि॑ज्मा॒ परि॒ज्मेर्यो॑ गो॒पा गो॒पा इर्यः॒ परि॑ज्मा । \newline
13. गो॒पा इति॑ गो - पाः । \newline
14. इर्यः॒ परि॑ज्मा॒ परि॒ज्मेर्य॒ इर्यः॒ परि॑ज्मा । \newline
15. परि॒ज्मेति॒ परि॑ - ज्मा॒ । \newline
16. वैश्वा॑नर॒ ब्रह्म॑णे॒ ब्रह्म॑णे॒ वैश्वा॑नर॒ वैश्वा॑नर॒ ब्रह्म॑णे विन्द विन्द॒ ब्रह्म॑णे॒ वैश्वा॑नर॒ वैश्वा॑नर॒ ब्रह्म॑णे विन्द । \newline
17. ब्रह्म॑णे विन्द विन्द॒ ब्रह्म॑णे॒ ब्रह्म॑णे विन्द गा॒तुम् गा॒तुं ॅवि॑न्द॒ ब्रह्म॑णे॒ ब्रह्म॑णे विन्द गा॒तुम् । \newline
18. वि॒न्द॒ गा॒तुम् गा॒तुं ॅवि॑न्द विन्द गा॒तुं ॅयू॒यं ॅयू॒यम् गा॒तुं ॅवि॑न्द विन्द गा॒तुं ॅयू॒यम् । \newline
19. गा॒तुं ॅयू॒यं ॅयू॒यम् गा॒तुम् गा॒तुं ॅयू॒यम् पा॑त पात यू॒यम् गा॒तुम् गा॒तुं ॅयू॒यम् पा॑त । \newline
20. यू॒यम् पा॑त पात यू॒यं ॅयू॒यम् पा॑त स्व॒स्तिभिः॑ स्व॒स्तिभिः॑ पात यू॒यं ॅयू॒यम् पा॑त स्व॒स्तिभिः॑ । \newline
21. पा॒त॒ स्व॒स्तिभिः॑ स्व॒स्तिभिः॑ पात पात स्व॒स्तिभिः॒ सदा॒ सदा᳚ स्व॒स्तिभिः॑ पात पात स्व॒स्तिभिः॒ सदा᳚ । \newline
22. स्व॒स्तिभिः॒ सदा॒ सदा᳚ स्व॒स्तिभिः॑ स्व॒स्तिभिः॒ सदा॑ नो नः॒ सदा᳚ स्व॒स्तिभिः॑ स्व॒स्तिभिः॒ सदा॑ नः । \newline
23. स्व॒स्तिभि॒रिति॑ स्व॒स्ति - भिः॒ । \newline
24. सदा॑ नो नः॒ सदा॒ सदा॑ नः । \newline
25. न॒ इति॑ नः । \newline
26. त्व म॑ग्ने अग्ने॒ त्वम् त्व म॑ग्ने शो॒चिषा॑ शो॒चिषा᳚ ऽग्ने॒ त्वम् त्व म॑ग्ने शो॒चिषा᳚ । \newline
27. अ॒ग्ने॒ शो॒चिषा॑ शो॒चिषा᳚ ऽग्ने अग्ने शो॒चिषा॒ शोशु॑चानः॒ शोशु॑चानः शो॒चिषा᳚ ऽग्ने अग्ने शो॒चिषा॒ शोशु॑चानः । \newline
28. शो॒चिषा॒ शोशु॑चानः॒ शोशु॑चानः शो॒चिषा॑ शो॒चिषा॒ शोशु॑चान॒ आ शोशु॑चानः शो॒चिषा॑ शो॒चिषा॒ शोशु॑चान॒ आ । \newline
29. शोशु॑चान॒ आ शोशु॑चानः॒ शोशु॑चान॒ आ रोद॑सी॒ रोद॑सी॒ आ शोशु॑चानः॒ शोशु॑चान॒ आ रोद॑सी । \newline
30. आ रोद॑सी॒ रोद॑सी॒ आ रोद॑सी अपृणा अपृणा॒ रोद॑सी॒ आ रोद॑सी अपृणाः । \newline
31. रोद॑सी अपृणा अपृणा॒ रोद॑सी॒ रोद॑सी अपृणा॒ जाय॑मानो॒ जाय॑मानो ऽपृणा॒ रोद॑सी॒ रोद॑सी अपृणा॒ जाय॑मानः । \newline
32. रोद॑सी॒ इति॒ रोद॑सी । \newline
33. अ॒पृ॒णा॒ जाय॑मानो॒ जाय॑मानो ऽपृणा अपृणा॒ जाय॑मानः । \newline
34. जाय॑मान॒ इति॒ जाय॑मानः । \newline
35. त्वम् दे॒वान् दे॒वान् त्वम् त्वम् दे॒वाꣳ अ॒भिश॑स्ते र॒भिश॑स्तेर् दे॒वान् त्वम् त्वम् दे॒वाꣳ अ॒भिश॑स्तेः । \newline
36. दे॒वाꣳ अ॒भिश॑स्ते र॒भिश॑स्तेर् दे॒वान् दे॒वाꣳ अ॒भिश॑स्ते रमुञ्चो अमुञ्चो अ॒भिश॑स्तेर् दे॒वान् दे॒वाꣳ अ॒भिश॑स्ते रमुञ्चः । \newline
37. अ॒भिश॑स्ते रमुञ्चो अमुञ्चो अ॒भिश॑स्ते र॒भिश॑स्ते रमुञ्चो॒ वैश्वा॑नर॒ वैश्वा॑नरामुञ्चो अ॒भिश॑स्ते र॒भिश॑स्ते रमुञ्चो॒ वैश्वा॑नर । \newline
38. अ॒भिश॑स्ते॒रित्य॒भि - श॒स्तेः॒ । \newline
39. अ॒मु॒ञ्चो॒ वैश्वा॑नर॒ वैश्वा॑नरामुञ्चो अमुञ्चो॒ वैश्वा॑नर जातवेदो जातवेदो॒ वैश्वा॑नरामुञ्चो अमुञ्चो॒ वैश्वा॑नर जातवेदः । \newline
40. वैश्वा॑नर जातवेदो जातवेदो॒ वैश्वा॑नर॒ वैश्वा॑नर जातवेदो महि॒त्वा म॑हि॒त्वा जा॑तवेदो॒ वैश्वा॑नर॒ वैश्वा॑नर जातवेदो महि॒त्वा । \newline
41. जा॒त॒वे॒दो॒ म॒हि॒त्वा म॑हि॒त्वा जा॑तवेदो जातवेदो महि॒त्वा । \newline
42. जा॒त॒वे॒द॒ इति॑ जात - वे॒दः॒ । \newline
43. म॒हि॒त्वेति॑ महि - त्वा । \newline
44. अ॒स्माक॑ मग्ने अग्ने॒ ऽस्माक॑ म॒स्माक॑ मग्ने म॒घव॑थ्सु म॒घव॑थ्स्वग्ने॒ ऽस्माक॑ म॒स्माक॑ मग्ने म॒घव॑थ्सु । \newline
45. अ॒ग्ने॒ म॒घव॑थ्सु म॒घव॑थ्स्वग्ने अग्ने म॒घव॑थ्सु धारय धारय म॒घव॑थ्स्वग्ने अग्ने म॒घव॑थ्सु धारय । \newline
46. म॒घव॑थ्सु धारय धारय म॒घव॑थ्सु म॒घव॑थ्सु धार॒याना॒म्यना॑मि धारय म॒घव॑थ्सु म॒घव॑थ्सु धार॒याना॑मि । \newline
47. म॒घव॒थ्स्विति॑ म॒घव॑त् - सु॒ । \newline
48. धा॒र॒याना॒म्यना॑मि धारय धार॒याना॑मि क्ष॒त्रम् क्ष॒त्र मना॑मि धारय धार॒याना॑मि क्ष॒त्रम् । \newline
49. अना॑मि क्ष॒त्रम् क्ष॒त्र मना॒म्यना॑मि क्ष॒त्र म॒जर॑ म॒जर॑म् क्ष॒त्र मना॒म्यना॑मि क्ष॒त्र म॒जर᳚म् । \newline
50. क्ष॒त्र म॒जर॑ म॒जर॑म् क्ष॒त्रम् क्ष॒त्र म॒जर(ग्म्॑) सु॒वीर्य(ग्म्॑) सु॒वीर्य॑ म॒जर॑म् क्ष॒त्रम् क्ष॒त्र म॒जर(ग्म्॑) सु॒वीर्य᳚म् । \newline
51. अ॒जर(ग्म्॑) सु॒वीर्य(ग्म्॑) सु॒वीर्य॑ म॒जर॑ म॒जर(ग्म्॑) सु॒वीर्य᳚म् । \newline
52. सु॒वीर्य॒मिति॑ सु - वीर्य᳚म् । \newline
53. व॒यम् ज॑येम जयेम व॒यं ॅव॒यम् ज॑येम श॒तिन(ग्म्॑) श॒तिन॑म् जयेम व॒यं ॅव॒यम् ज॑येम श॒तिन᳚म् । \newline
54. ज॒ये॒म॒ श॒तिन(ग्म्॑) श॒तिन॑म् जयेम जयेम श॒तिन(ग्म्॑) सह॒स्रिण(ग्म्॑) सह॒स्रिण(ग्म्॑) श॒तिन॑म् जयेम जयेम श॒तिन(ग्म्॑) सह॒स्रिण᳚म् । \newline
55. श॒तिन(ग्म्॑) सह॒स्रिण(ग्म्॑) सह॒स्रिण(ग्म्॑) श॒तिन(ग्म्॑) श॒तिन(ग्म्॑) सह॒स्रिणं॒ ॅवैश्वा॑नर॒ वैश्वा॑नर सह॒स्रिण(ग्म्॑) श॒तिन(ग्म्॑) श॒तिन(ग्म्॑) सह॒स्रिणं॒ ॅवैश्वा॑नर । \newline
56. स॒ह॒स्रिणं॒ ॅवैश्वा॑नर॒ वैश्वा॑नर सह॒स्रिण(ग्म्॑) सह॒स्रिणं॒ ॅवैश्वा॑नर॒ वाजं॒ ॅवाजं॒ ॅवैश्वा॑नर सह॒स्रिण(ग्म्॑) सह॒स्रिणं॒ ॅवैश्वा॑नर॒ वाज᳚म् । \newline
57. वैश्वा॑नर॒ वाजं॒ ॅवाजं॒ ॅवैश्वा॑नर॒ वैश्वा॑नर॒ वाज॑ मग्ने अग्ने॒ वाजं॒ ॅवैश्वा॑नर॒ वैश्वा॑नर॒ वाज॑ मग्ने । \newline
\pagebreak
\markright{ TS 1.5.11.3  \hfill https://www.vedavms.in \hfill}
\addcontentsline{toc}{section}{ TS 1.5.11.3 }
\section*{ TS 1.5.11.3 }

\textbf{TS 1.5.11.3 } \newline
\textbf{Samhita Paata} \newline

वाज॑मग्ने॒ तवो॒तिभिः॑ ॥ वै॒श्वा॒न॒रस्य॑ सुम॒तौ स्या॑म॒ राजा॒ हिकं॒ भुव॑नाना-मभि॒श्रीः । इ॒तो जा॒तो विश्व॑मि॒दं ॅवि च॑ष्टे वैश्वान॒रो य॑तते॒ सूर्ये॑ण ॥ अव॑ ते॒ हेडो॑ वरुण॒ नमो॑भि॒रव॑ य॒ज्ञेभि॑रीमहे ह॒विर्भिः॑ । क्षय॑न्न॒स्मभ्य॑मसुर प्रचेतो॒ राज॒न्नेनाꣳ॑सि शिश्रथः कृ॒तानि॑ ॥ उदु॑त्त॒मं ॅव॑रुण॒ पाश॑म॒स्मदवा॑ऽध॒मं ॅविम॑द्ध्य॒मꣳ श्र॑थाय । अथा॑ व॒यमा॑दित्य - [ ] \newline

\textbf{Pada Paata} \newline

वाज᳚म् । अ॒ग्ने॒ । तव॑ । ऊ॒तिभि॒रित्यू॒ति - भिः॒ ॥ वै॒श्वा॒न॒रस्य॑ । सु॒म॒ताविति॑ सु - म॒तौ । स्या॒म॒ । राजा᳚ । हिक᳚म् । भुव॑नानाम् । अ॒भि॒श्रीरित्य॑भि - श्रीः ॥ इ॒तः । जा॒तः । विश्व᳚म् । इ॒दम् । वीति॑ । च॒ष्टे॒ । वै॒श्वा॒न॒रः । य॒त॒ते॒ । सूर्ये॑ण ॥ अवेति॑ । ते॒ । हेडः॑ । व॒रु॒ण॒ । नमो॑भि॒रिति॒ नमः॑ - भिः॒ । अवेति॑ । य॒ज्ञेभिः॑ । ई॒म॒हे॒ । ह॒विर्भि॒रिति॑ ह॒विः - भिः॒ ॥ क्षयन्न॑ । अ॒स्मभ्य॒मित्य॒स्म - भ्य॒म् । अ॒सु॒र॒ । प्र॒चे॒त॒ इति॑ प्र - चे॒तः॒ । राजन्न्॑ । ए॒नाꣳ॑सि । शि॒श्र॒थः॒ । कृ॒तानि॑ ॥ उदिति॑ । उ॒त्त॒ममित्यु॑त् - त॒मम् । व॒रु॒ण॒ । पाश᳚म् । अ॒स्मत् । अवेति॑ । अ॒ध॒मम् । वीति॑ । म॒द्ध्य॒मम् । श्र॒था॒य॒ ॥ अथ॑ । व॒यम् । आ॒दि॒त्य॒ ।  \newline


\textbf{Krama Paata} \newline

वाज॑मग्ने । अ॒ग्ने॒ तव॑ । तवो॒तिभिः॑ । ऊ॒तिभि॒रित्यू॒ति - भिः॒ ॥ वै॒श्वा॒न॒रस्य॑ सुम॒तौ । सु॒म॒तौ स्या॑म । सु॒म॒ताविति॑ सु - म॒तौ । स्या॒म॒ राजा᳚ । राजा॒ हिक᳚म् । हिक॒म् भुव॑नानाम् । भुव॑नानामभि॒श्रीः । अ॒भि॒श्रीरित्य॑भि - श्रीः ॥ इ॒तो जा॒तः । जा॒तो विश्व᳚म् । विश्व॑मि॒दम् । इ॒दं ॅवि । वि च॑ष्टे । च॒ष्टे॒ वै॒श्वा॒न॒रः । वै॒श्वा॒न॒रो य॑तते । य॒त॒ते॒ सूर्ये॑ण । सूर्ये॒णेति॒ सूर्ये॑ण ॥ अव॑ ते । ते॒ हेडः॑ । हेडो॑ वरुण । व॒रु॒ण॒ नमो॑भिः । नमो॑भि॒रव॑ । नमो॑भि॒रिति॒ नमः॑ - भिः॒ । अव॑ य॒ज्ञेभिः॑ । य॒ज्ञेभि॑रीमहे । ई॒म॒हे॒ ह॒विर्भिः॑ । ह॒विर्भि॒रिति॑ ह॒विः - भिः॒ ॥ क्षय॑न्न॒स्मभ्य᳚म् । अ॒स्मभ्य॑मसुर । अ॒स्मभ्य॒मित्य॒स्म - भ्य॒म् । अ॒सु॒र॒ प्र॒चे॒तः॒ । प्र॒चे॒तो॒ राजन्न्॑ । प्र॒चे॒त॒ इति॑ प्र - चे॒तः॒ । राज॒न्नेनाꣳ॑सि । एनाꣳ॑सि शिश्रथः । शि॒श्र॒थः॒ कृ॒तानि॑ । कृ॒तानीति॑ कृ॒तानि॑ ॥ उदु॑त्त॒मम् । उ॒त्त॒मं ॅव॑रुण । उ॒त्त॒ममित्यु॑त् - त॒मम् । व॒रु॒ण॒ पाश᳚म् । पाश॑म॒स्मत् । अ॒स्मदव॑ । अवा॑ध॒मम् । अ॒ध॒मं ॅवि । वि म॑द्ध्य॒मम् । म॒द्ध्य॒मꣳ श्र॑थाय । श्र॒था॒येति॑ श्रथाय ॥ अथा॑ व॒यम् । व॒यमा॑दित्य । आ॒दि॒त्य॒ व्र॒ते \newline

\textbf{Jatai Paata} \newline

1. वाज॑ मग्ने अग्ने॒ वाजं॒ ॅवाज॑ मग्ने । \newline
2. अ॒ग्ने॒ तव॒ तवा᳚ग्ने अग्ने॒ तव॑ । \newline
3. तवो॒तिभि॑ रू॒तिभि॒ स्तव॒ तवो॒तिभिः॑ । \newline
4. ऊ॒तिभि॒रित्यू॒ति - भिः॒ । \newline
5. वै॒श्वा॒न॒रस्य॑ सुम॒तौ सु॑म॒तौ वै᳚श्वान॒रस्य॑ वैश्वान॒रस्य॑ सुम॒तौ । \newline
6. सु॒म॒तौ स्या॑म स्याम सुम॒तौ सु॑म॒तौ स्या॑म । \newline
7. सु॒म॒ताविति॑ सु - म॒तौ । \newline
8. स्या॒म॒ राजा॒ राजा᳚ स्याम स्याम॒ राजा᳚ । \newline
9. राजा॒ हिक॒(ग्म्॒) हिक॒(ग्म्॒) राजा॒ राजा॒ हिक᳚म् । \newline
10. हिक॒म् भुव॑नाना॒म् भुव॑नाना॒(ग्म्॒) हिक॒(ग्म्॒) हिक॒म् भुव॑नानाम् । \newline
11. भुव॑नाना मभि॒श्री र॑भि॒श्रीर् भुव॑नाना॒म् भुव॑नाना मभि॒श्रीः । \newline
12. अ॒भि॒श्रीरित्य॑भि - श्रीः । \newline
13. इ॒तो जा॒तो जा॒त इ॒त इ॒तो जा॒तः । \newline
14. जा॒तो विश्वं॒ ॅविश्व॑म् जा॒तो जा॒तो विश्व᳚म् । \newline
15. विश्व॑ मि॒द मि॒दं ॅविश्वं॒ ॅविश्व॑ मि॒दम् । \newline
16. इ॒दं ॅवि वीद मि॒दं ॅवि । \newline
17. वि च॑ष्टे चष्टे॒ वि वि च॑ष्टे । \newline
18. च॒ष्टे॒ वै॒श्वा॒न॒रो वै᳚श्वान॒रश्च॑ष्टे चष्टे वैश्वान॒रः । \newline
19. वै॒श्वा॒न॒रो य॑तते यतते वैश्वान॒रो वै᳚श्वान॒रो य॑तते । \newline
20. य॒त॒ते॒ सूर्ये॑ण॒ सूर्ये॑ण यतते यतते॒ सूर्ये॑ण । \newline
21. सूर्ये॒णेति॒ सूर्ये॑ण । \newline
22. अव॑ ते॒ ते ऽवाव॑ ते । \newline
23. ते॒ हेडो॒ हेड॑स्ते ते॒ हेडः॑ । \newline
24. हेडो॑ वरुण वरुण॒ हेडो॒ हेडो॑ वरुण । \newline
25. व॒रु॒ण॒ नमो॑भि॒र् नमो॑भिर् वरुण वरुण॒ नमो॑भिः । \newline
26. नमो॑भि॒रवाव॒ नमो॑भि॒र् नमो॑भि॒रव॑ । \newline
27. नमो॑भि॒रिति॒ नमः॑ - भिः॒ । \newline
28. अव॑ य॒ज्ञेभि॑र् य॒ज्ञेभि॒रवाव॑ य॒ज्ञेभिः॑ । \newline
29. य॒ज्ञेभि॑रीमह ईमहे य॒ज्ञेभि॑र् य॒ज्ञेभि॑रीमहे । \newline
30. ई॒म॒हे॒ ह॒विर्भि॑र् ह॒विर्भि॑रीमह ईमहे ह॒विर्भिः॑ । \newline
31. ह॒विर्भि॒रिति॑ ह॒विः - भिः॒ । \newline
32. क्षय॑न्न॒स्मभ्य॑ म॒स्मभ्य॒म् क्षय॒न् क्षय॑न्न॒स्मभ्य᳚म् । \newline
33. अ॒स्मभ्य॑ मसुरासुरा॒स्मभ्य॑ म॒स्मभ्य॑ मसुर । \newline
34. अ॒स्मभ्य॒मित्य॒स्म - भ्य॒म् । \newline
35. अ॒सु॒र॒ प्र॒चे॒तः॒ प्र॒चे॒तो॒ अ॒सु॒रा॒सु॒र॒ प्र॒चे॒तः॒ । \newline
36. प्र॒चे॒तो॒ राज॒न् राज॑न् प्रचेतः प्रचेतो॒ राजन्न्॑ । \newline
37. प्र॒चे॒त॒ इति॑ प्र - चे॒तः॒ । \newline
38. राज॒न्नेना॒(ग्ग्॒) स्येना(ग्म्॑)सि॒ राज॒न् राज॒न्नेना(ग्म्॑)सि । \newline
39. एना(ग्म्॑)सि शिश्रथः शिश्रथ॒ एना॒(ग्ग्॒) स्येना(ग्म्॑)सि शिश्रथः । \newline
40. शि॒श्र॒थः॒ कृ॒तानि॑ कृ॒तानि॑ शिश्रथः शिश्रथः कृ॒तानि॑ । \newline
41. कृ॒तानीति॑ कृ॒तानि॑ । \newline
42. उदु॑त्त॒म मु॑त्त॒म मुदुदु॑त्त॒मम् । \newline
43. उ॒त्त॒मं ॅव॑रुण वरुणोत्त॒म मु॑त्त॒मं ॅव॑रुण । \newline
44. उ॒त्त॒ममित्यु॑त् - त॒मम् । \newline
45. व॒रु॒ण॒ पाश॒म् पाशं॑ ॅवरुण वरुण॒ पाश᳚म् । \newline
46. पाश॑ म॒स्मद॒स्मत् पाश॒म् पाश॑ म॒स्मत् । \newline
47. अ॒स्म दवावा॒स्म द॒स्मदव॑ । \newline
48. अवा॑ध॒म म॑ध॒म मवावा॑ध॒मम् । \newline
49. अ॒ध॒मं ॅवि व्य॑ध॒म म॑ध॒मं ॅवि । \newline
50. वि म॑द्ध्य॒मम् म॑द्ध्य॒मं ॅवि वि म॑द्ध्य॒मम् । \newline
51. म॒द्ध्य॒मꣳ श्र॑थाय श्रथाय मद्ध्य॒मम् म॑द्ध्य॒मꣳ श्र॑थाय । \newline
52. श्र॒था॒येति॑ श्रथाय । \newline
53. अथा॑ व॒यं ॅव॒य मथाथा॑ व॒यम् । \newline
54. व॒य मा॑दित्यादित्य व॒यं ॅव॒य मा॑दित्य । \newline
55. आ॒दि॒त्य॒ व्र॒ते व्र॒त आ॑दित्यादित्य व्र॒ते । \newline

\textbf{Ghana Paata } \newline

1. वाज॑ मग्ने अग्ने॒ वाजं॒ ॅवाज॑ मग्ने॒ तव॒ तवा᳚ग्ने॒ वाजं॒ ॅवाज॑ मग्ने॒ तव॑ । \newline
2. अ॒ग्ने॒ तव॒ तवा᳚ग्ने अग्ने॒ तवो॒तिभि॑ रू॒तिभि॒ स्तवा᳚ग्ने अग्ने॒ तवो॒तिभिः॑ । \newline
3. तवो॒तिभि॑ रू॒तिभि॒ स्तव॒ तवो॒तिभिः॑ । \newline
4. ऊ॒तिभि॒रित्यू॒ति - भिः॒ । \newline
5. वै॒श्वा॒न॒रस्य॑ सुम॒तौ सु॑म॒तौ वै᳚श्वान॒रस्य॑ वैश्वान॒रस्य॑ सुम॒तौ स्या॑म स्याम सुम॒तौ वै᳚श्वान॒रस्य॑ वैश्वान॒रस्य॑ सुम॒तौ स्या॑म । \newline
6. सु॒म॒तौ स्या॑म स्याम सुम॒तौ सु॑म॒तौ स्या॑म॒ राजा॒ राजा᳚ स्याम सुम॒तौ सु॑म॒तौ स्या॑म॒ राजा᳚ । \newline
7. सु॒म॒ताविति॑ सु - म॒तौ । \newline
8. स्या॒म॒ राजा॒ राजा᳚ स्याम स्याम॒ राजा॒ हिक॒(ग्म्॒) हिक॒(ग्म्॒) राजा᳚ स्याम स्याम॒ राजा॒ हिक᳚म् । \newline
9. राजा॒ हिक॒(ग्म्॒) हिक॒(ग्म्॒) राजा॒ राजा॒ हिक॒म् भुव॑नाना॒म् भुव॑नाना॒(ग्म्॒) हिक॒(ग्म्॒) राजा॒ राजा॒ हिक॒म् भुव॑नानाम् । \newline
10. हिक॒म् भुव॑नाना॒म् भुव॑नाना॒(ग्म्॒) हिक॒(ग्म्॒) हिक॒म् भुव॑नाना मभि॒श्री र॑भि॒श्रीर् भुव॑नाना॒(ग्म्॒) हिक॒(ग्म्॒) हिक॒म् भुव॑नाना मभि॒श्रीः । \newline
11. भुव॑नाना मभि॒श्रीर॑भि॒श्रीर् भुव॑नाना॒म् भुव॑नाना मभि॒श्रीः । \newline
12. अ॒भि॒श्रीरित्य॑भि - श्रीः । \newline
13. इ॒तो जा॒तो जा॒त इ॒त इ॒तो जा॒तो विश्वं॒ ॅविश्व॑म् जा॒त इ॒त इ॒तो जा॒तो विश्व᳚म् । \newline
14. जा॒तो विश्वं॒ ॅविश्व॑म् जा॒तो जा॒तो विश्व॑ मि॒द मि॒दं ॅविश्व॑म् जा॒तो जा॒तो विश्व॑ मि॒दम् । \newline
15. विश्व॑ मि॒द मि॒दं ॅविश्वं॒ ॅविश्व॑ मि॒दं ॅवि वीदं ॅविश्वं॒ ॅविश्व॑ मि॒दं ॅवि । \newline
16. इ॒दं ॅवि वीद मि॒दं ॅवि च॑ष्टे चष्टे॒ वीद मि॒दं ॅवि च॑ष्टे । \newline
17. वि च॑ष्टे चष्टे॒ वि वि च॑ष्टे वैश्वान॒रो वै᳚श्वान॒र श्च॑ष्टे॒ वि वि च॑ष्टे वैश्वान॒रः । \newline
18. च॒ष्टे॒ वै॒श्वा॒न॒रो वै᳚श्वान॒रश्च॑ष्टे चष्टे वैश्वान॒रो य॑तते यतते वैश्वान॒रश्च॑ष्टे चष्टे वैश्वान॒रो य॑तते । \newline
19. वै॒श्वा॒न॒रो य॑तते यतते वैश्वान॒रो वै᳚श्वान॒रो य॑तते॒ सूर्ये॑ण॒ सूर्ये॑ण यतते वैश्वान॒रो वै᳚श्वान॒रो य॑तते॒ सूर्ये॑ण । \newline
20. य॒त॒ते॒ सूर्ये॑ण॒ सूर्ये॑ण यतते यतते॒ सूर्ये॑ण । \newline
21. सूर्ये॒णेति॒ सूर्ये॑ण । \newline
22. अव॑ ते॒ ते ऽवाव॑ ते॒ हेडो॒ हेड॒स्ते ऽवाव॑ ते॒ हेडः॑ । \newline
23. ते॒ हेडो॒ हेड॑स्ते ते॒ हेडो॑ वरुण वरुण॒ हेड॑स्ते ते॒ हेडो॑ वरुण । \newline
24. हेडो॑ वरुण वरुण॒ हेडो॒ हेडो॑ वरुण॒ नमो॑भि॒र् नमो॑भिर् वरुण॒ हेडो॒ हेडो॑ वरुण॒ नमो॑भिः । \newline
25. व॒रु॒ण॒ नमो॑भि॒र् नमो॑भिर् वरुण वरुण॒ नमो॑भि॒रवाव॒ नमो॑भिर् वरुण वरुण॒ नमो॑भि॒रव॑ । \newline
26. नमो॑भि॒रवाव॒ नमो॑भि॒र् नमो॑भि॒रव॑ य॒ज्ञेभि॑र् य॒ज्ञेभि॒रव॒ नमो॑भि॒र् नमो॑भि॒रव॑ य॒ज्ञेभिः॑ । \newline
27. नमो॑भि॒रिति॒ नमः॑ - भिः॒ । \newline
28. अव॑ य॒ज्ञेभि॑र् य॒ज्ञेभि॒ रवाव॑ य॒ज्ञेभि॑रीमह ईमहे य॒ज्ञेभि॒ रवाव॑ य॒ज्ञेभि॑रीमहे । \newline
29. य॒ज्ञेभि॑रीमह ईमहे य॒ज्ञेभि॑र् य॒ज्ञेभि॑रीमहे ह॒विर्भि॑र् ह॒विर्भि॑रीमहे य॒ज्ञेभि॑र् य॒ज्ञेभि॑रीमहे ह॒विर्भिः॑ । \newline
30. ई॒म॒हे॒ ह॒विर्भि॑र् ह॒विर्भि॑रीमह ईमहे ह॒विर्भिः॑ । \newline
31. ह॒विर्भि॒रिति॑ ह॒विः - भिः॒ । \newline
32. क्षय॑न् न॒स्मभ्य॑ म॒स्मभ्य॒म् क्षय॒न् क्षय॑न् न॒स्मभ्य॑ मसुरासुरा॒स्मभ्य॒म् क्षय॒न् क्षय॑न् न॒स्मभ्य॑ मसुर । \newline
33. अ॒स्मभ्य॑ मसुरासुरा॒स्मभ्य॑ म॒स्मभ्य॑ मसुर प्रचेतः प्रचेतो अ॒सुरा॒स्मभ्य॑ म॒स्मभ्य॑ मसुर प्रचेतः । \newline
34. अ॒स्मभ्य॒मित्य॒स्म - भ्य॒म् । \newline
35. अ॒सु॒र॒ प्र॒चे॒तः॒ प्र॒चे॒तो॒ अ॒सु॒रा॒सु॒र॒ प्र॒चे॒तो॒ राज॒न् राज॑न् प्रचेतो असुरासुर प्रचेतो॒ राजन्न्॑ । \newline
36. प्र॒चे॒तो॒ राज॒न् राज॑न् प्रचेतः प्रचेतो॒ राज॒न् नेना॒(ग्ग्॒) स्येना(ग्म्॑)सि॒ राज॑न् प्रचेतः प्रचेतो॒ राज॒न् नेना(ग्म्॑)सि । \newline
37. प्र॒चे॒त॒ इति॑ प्र - चे॒तः॒ । \newline
38. राज॒न् नेना॒(ग्ग्॒) स्येना(ग्म्॑)सि॒ राज॒न् राज॒न् नेना(ग्म्॑)सि शिश्रथः शिश्रथ॒ एना(ग्म्॑)सि॒ राज॒न् राज॒न् नेना(ग्म्॑)सि शिश्रथः । \newline
39. एना(ग्म्॑)सि शिश्रथः शिश्रथ॒ एना॒(ग्ग्॒) स्येना(ग्म्॑)सि शिश्रथः कृ॒तानि॑ कृ॒तानि॑ शिश्रथ॒ एना॒(ग्ग्॒) स्येना(ग्म्॑)सि शिश्रथः कृ॒तानि॑ । \newline
40. शि॒श्र॒थः॒ कृ॒तानि॑ कृ॒तानि॑ शिश्रथः शिश्रथः कृ॒तानि॑ । \newline
41. कृ॒तानीति॑ कृ॒तानि॑ । \newline
42. उदु॑त्त॒म मु॑त्त॒म मुदुदु॑त्त॒मं ॅव॑रुण वरुणोत्त॒म मुदुदु॑त्त॒मं ॅव॑रुण । \newline
43. उ॒त्त॒मं ॅव॑रुण वरुणोत्त॒म मु॑त्त॒मं ॅव॑रुण॒ पाश॒म् पाशं॑ ॅवरुणोत्त॒म मु॑त्त॒मं ॅव॑रुण॒ पाश᳚म् । \newline
44. उ॒त्त॒ममित्यु॑त् - त॒मम् । \newline
45. व॒रु॒ण॒ पाश॒म् पाशं॑ ॅवरुण वरुण॒ पाश॑ म॒स्मद॒स्मत् पाशं॑ ॅवरुण वरुण॒ पाश॑ म॒स्मत् । \newline
46. पाश॑ म॒स्मद॒स्मत् पाश॒म् पाश॑ म॒स्मदवावा॒स्मत् पाश॒म् पाश॑ म॒स्मदव॑ । \newline
47. अ॒स्म दवावा॒स्म द॒स्मदवा॑ध॒म म॑ध॒म मवा॒स्म द॒स्मदवा॑ध॒मम् । \newline
48. अवा॑ध॒म म॑ध॒म मवावा॑ध॒मं ॅवि व्य॑ध॒म मवावा॑ध॒मं ॅवि । \newline
49. अ॒ध॒मं ॅवि व्य॑ध॒म म॑ध॒मं ॅवि म॑द्ध्य॒मम् म॑द्ध्य॒मं ॅव्य॑ध॒म म॑ध॒मं ॅवि म॑द्ध्य॒मम् । \newline
50. वि म॑द्ध्य॒मम् म॑द्ध्य॒मं ॅवि वि म॑द्ध्य॒मꣳ श्र॑थाय श्रथाय मद्ध्य॒मं ॅवि वि म॑द्ध्य॒मꣳ श्र॑थाय । \newline
51. म॒द्ध्य॒मꣳ श्र॑थाय श्रथाय मद्ध्य॒मम् म॑द्ध्य॒मꣳ श्र॑थाय । \newline
52. श्र॒था॒येति॑ श्रथाय । \newline
53. अथा॑ व॒यं ॅव॒य मथाथा॑ व॒य मा॑दित्यादित्य व॒य मथाथा॑ व॒य मा॑दित्य । \newline
54. व॒य मा॑दित्यादित्य व॒यं ॅव॒य मा॑दित्य व्र॒ते व्र॒त आ॑दित्य व॒यं ॅव॒य मा॑दित्य व्र॒ते । \newline
55. आ॒दि॒त्य॒ व्र॒ते व्र॒त आ॑दित्यादित्य व्र॒ते तव॒ तव॑ व्र॒त आ॑दित्यादित्य व्र॒ते तव॑ । \newline
\pagebreak
\markright{ TS 1.5.11.4  \hfill https://www.vedavms.in \hfill}
\addcontentsline{toc}{section}{ TS 1.5.11.4 }
\section*{ TS 1.5.11.4 }

\textbf{TS 1.5.11.4 } \newline
\textbf{Samhita Paata} \newline

व्र॒ते तवाऽना॑गसो॒ अदि॑तये स्याम ॥ द॒धि॒क्राव्.ण्णो॑ अकारिषं जि॒ष्णोरश्व॑स्य वा॒जिनः॑ ॥ सु॒र॒भिनो॒ मुखा॑ कर॒त् प्रण॒ आयूꣳ॑षि तारिषत् ॥ आ द॑धि॒क्राः शव॑स॒॑0078; पञ्च॑ कृ॒ष्टीः सूर्य॑ इव॒ ज्योति॑षा॒ऽपस्त॑तान । स॒ह॒स्र॒साः श॑त॒सा वा॒ज्यर्वा॑ पृ॒णक्तु॒ मद्ध्वा॒ समि॒मा वचाꣳ॑सि । अ॒ग्निर् मू॒र्द्धा>1, भुवः॑>2 । मरु॑तो॒ यद्ध॑वो दि॒वः सु॑म्ना॒यन्तो॒ हवा॑महे । आ तू न॒ - [ ] \newline

\textbf{Pada Paata} \newline

व्र॒ते । तव॑ । अना॑गसः । अदि॑तये । स्या॒म॒ ॥ द॒धि॒क्राव्‌ण्ण॒ इति॑ दधि - क्राव्‌ण्णः॑ । अ॒का॒रि॒ष॒म् । जि॒ष्णोः । अश्व॑स्य । वा॒जिनः॑ ॥ सु॒र॒भि । नः॒ । मुखा᳚ । क॒र॒त् । प्रेति॑ । नः॒ । आयूꣳ॑षि । ता॒रि॒ष॒त् ॥ एति॑ । द॒धि॒क्रा इति॑ दधि - क्राः । शव॑सा । पञ्च॑ । कृ॒ष्टीः । सूर्य॑ । इ॒व॒ । ज्योति॑षा । अ॒पः । त॒ता॒न॒ ॥ स॒ह॒स्र॒सा इति॑ स॒ह॒स्र - साः । श॒त॒सा इति॑ शत - साः । वा॒जी । अर्वा᳚ । पृ॒णक्तु॑ । मद्ध्वा᳚ । समिति॑ । इ॒मा । वचाꣳ॑सि ॥ अ॒ग्निः । मू॒र्धा । भुवः॑ ॥ मरु॑तः । यत् । ह॒ । वः॒ । दि॒वः । सु॒म्ना॒यन्त॒ इति॑ सुम्न - यन्तः॑ । हवा॑महे ॥ एति॑ । तु । नः॒ ।  \newline


\textbf{Krama Paata} \newline

व्र॒ते तव॑ । तवाना॑गसः । अना॑गसो॒ अदि॑तये । अदि॑तये स्याम । स्या॒मेति॑ स्याम ॥ द॒धि॒क्राव्.ण्णो॑ अकारिषम् । द॒धि॒क्राव्.ण्ण॒ इति॑ दधि - क्राव्.ण्णः॑ । अ॒का॒रि॒ष॒म् जि॒ष्णोः । जि॒ष्णोरश्व॑स्य । अश्व॑स्य वा॒जिनः॑ । वा॒जिन॒ इति॑ वा॒जिनः॑ ॥ सु॒र॒भि नः॑ । नो॒ मुखा᳚ । मुखा॑ करत् । क॒र॒त्,प्र । प्र॑ णः । न॒ आयूꣳ॑षि । आयूꣳ॑षि तारिषत् । ता॒रि॒ष॒दिति॑ तारिषत् ॥ आ द॑धि॒क्राः । द॒धि॒क्राः शव॑सा । द॒धि॒क्रा इति॑ दधि - क्राः । शव॑सा॒ पञ्च॑ । पञ्च॑ कृ॒ष्टीः । कृ॒ष्टीः सूर्यः॑ । सूर्य॑ इव । इ॒व॒ ज्योति॑षा । ज्योति॑षा॒ऽपः । अ॒पस्त॑तान । त॒ता॒नेति॑ ततान ॥ स॒ह॒स्र॒साः श॑त॒साः । स॒ह॒स्र॒सा इति॑ सहस्र - साः । श॒त॒सा वा॒जी । श॒त॒सा इति॑ शत - साः । वा॒ज्यर्वा᳚ । अर्वा॑ पृ॒णक्तु॑ । पृ॒णक्तु॒ मद्ध्वा᳚ । मद्ध्वा॒ सम् । समि॒मा । इ॒मा वचाꣳ॑सि । वचाꣳ॒॒सीति॒ वचाꣳ॑सि ॥ अ॒ग्निर् मू॒र्द्धा । मू॒र्द्धा भुवः॑ । भुव॒ इति॒ भुवः॑ ॥ मरु॑तो॒ यत् । यद्ध॑ । ह॒ वः॒ । वो॒ दि॒वः । दि॒वः सु॑म्ना॒यन्तः॑ । सु॒म्ना॒यन्तो॒ हवा॑महे । सु॒म्ना॒यन्त॒ इति॑ सुम्न - यन्तः॑ । हवा॑मह॒ इति॒ हवा॑महे ॥ आ तु । तू नः॑ । न॒ उप॑ \newline

\textbf{Jatai Paata} \newline

1. व्र॒ते तव॒ तव॑ व्र॒ते व्र॒ते तव॑ । \newline
2. तवाना॑ग॒सो ऽना॑गस॒स्तव॒ तवाना॑गसः । \newline
3. अना॑गसो॒ अदि॑तये॒ अदि॑त॒ये ऽना॑ग॒सो ऽना॑गसो॒ अदि॑तये । \newline
4. अदि॑तये स्याम स्या॒मादि॑तये॒ अदि॑तये स्याम । \newline
5. स्या॒मेति॑ स्याम । \newline
6. द॒धि॒क्राव्.ण्णो॑ अकारिष मकारिषम् दधि॒क्राव्.ण्णो॑ दधि॒क्राव्.ण्णो॑ अकारिषम् । \newline
7. द॒धि॒क्राव्.ण्ण॒ इति॑ दधि - क्राव्.ण्णः॑ । \newline
8. अ॒का॒रि॒ष॒म् जि॒ष्णोर् जि॒ष्णोर॑कारिष मकारिषम् जि॒ष्णोः । \newline
9. जि॒ष्णो रश्व॒स्याश्व॑स्य जि॒ष्णोर् जि॒ष्णो रश्व॑स्य । \newline
10. अश्व॑स्य वा॒जिनो॑ वा॒जिनो॒ अश्व॒स्याश्व॑स्य वा॒जिनः॑ । \newline
11. वा॒जिन॒ इति॑ वा॒जिनः॑ । \newline
12. सु॒र॒भि नो॑ नः सुर॒भि सु॑र॒भि नः॑ । \newline
13. नो॒ मुखा॒ मुखा॑ नो नो॒ मुखा᳚ । \newline
14. मुखा॑ करत् कर॒न् मुखा॒ मुखा॑ करत् । \newline
15. क॒र॒त् प्र प्र क॑रत् कर॒त् प्र । \newline
16. प्र णो॑ नः॒ प्र प्र णः॑ । \newline
17. न॒ आयू॒(ग्ग्॒) ष्यायू(ग्म्॑)षि नो न॒ आयू(ग्म्॑)षि । \newline
18. आयू(ग्म्॑)षि तारिषत् तारिष॒दायू॒(ग्ग्॒) ष्यायू(ग्म्॑)षि तारिषत् । \newline
19. ता॒रि॒ष॒दिति॑ तारिषत् । \newline
20. आ द॑धि॒क्रा द॑धि॒क्रा आ द॑धि॒क्राः । \newline
21. द॒धि॒क्राः शव॑सा॒ शव॑सा दधि॒क्रा द॑धि॒क्राः शव॑सा । \newline
22. द॒धि॒क्रा इति॑ दधि - क्राः । \newline
23. शव॑सा॒ पञ्च॒ पञ्च॒ शव॑सा॒ शव॑सा॒ पञ्च॑ । \newline
24. पञ्च॑ कृ॒ष्टीः कृ॒ष्टीः पञ्च॒ पञ्च॑ कृ॒ष्टीः । \newline
25. कृ॒ष्टीः सूर्यः॒ सूर्यः॑ कृ॒ष्टीः कृ॒ष्टीः सूर्यः॑ । \newline
26. सूर्य॑ इवे व॒ सूर्यः॒ सूर्य॑ इव । \newline
27. इ॒व॒ ज्योति॑षा॒ ज्योति॑षेवे व॒ ज्योति॑षा । \newline
28. ज्योति॑षा॒ ऽपो अ॒पो ज्योति॑षा॒ ज्योति॑षा॒ ऽपः । \newline
29. अ॒पस्त॑तान तताना॒पो अ॒पस्त॑तान । \newline
30. त॒ता॒नेति॑ ततान । \newline
31. स॒ह॒स्र॒साः श॑त॒साः श॑त॒साः स॑हस्र॒साः स॑हस्र॒साः श॑त॒साः । \newline
32. स॒ह॒स्र॒सा इति॑ सहस्र - साः । \newline
33. श॒त॒सा वा॒जी वा॒जी श॑त॒साः श॑त॒सा वा॒जी । \newline
34. श॒त॒सा इति॑ शत - साः । \newline
35. वा॒ज्यर्वा ऽर्वा॑ वा॒जी वा॒ज्यर्वा᳚ । \newline
36. अर्वा॑ पृ॒णक्तु॑ पृ॒णक्त्वर्वा ऽर्वा॑ पृ॒णक्तु॑ । \newline
37. पृ॒णक्तु॒ मद्ध्वा॒ मद्ध्वा॑ पृ॒णक्तु॑ पृ॒णक्तु॒ मद्ध्वा᳚ । \newline
38. मद्ध्वा॒ सꣳ सम् मद्ध्वा॒ मद्ध्वा॒ सम् । \newline
39. स मि॒मेमा सꣳ स मि॒मा । \newline
40. इ॒मा वचा(ग्म्॑)सि॒ वचा(ग्म्॑)सी॒मेमा वचा(ग्म्॑)सि । \newline
41. वचा॒(ग्म्॒)सीति॒ वचा(ग्म्॑)सि । \newline
42. अ॒ग्निर् मू॒र्द्धा मू॒र्द्धा ऽग्निर॒ग्निर् मू॒र्द्धा । \newline
43. मू॒र्द्धा भुवो॒ भुवो॑ मू॒र्द्धा मू॒र्द्धा भुवः॑ । \newline
44. भुव॒ इति॒ भुवः॑ । \newline
45. मरु॑तो॒ यद् यन् मरु॑तो॒ मरु॑तो॒ यत् । \newline
46. यद्ध॑ ह॒ यद् यद्ध॑ । \newline
47. ह॒ वो॒ वो॒ ह॒ ह॒ वः॒ । \newline
48. वो॒ दि॒वो दि॒वो वो॑ वो दि॒वः । \newline
49. दि॒वः सु॑म्ना॒यन्तः॑ सुम्ना॒यन्तो॑ दि॒वो दि॒वः सु॑म्ना॒यन्तः॑ । \newline
50. सु॒म्ना॒यन्तो॒ हवा॑महे॒ हवा॑महे सुम्ना॒यन्तः॑ सुम्ना॒यन्तो॒ हवा॑महे । \newline
51. सु॒म्ना॒यन्त॒ इति॑ सुम्न - यन्तः॑ । \newline
52. हवा॑मह॒ इति॒ हवा॑महे । \newline
53. आ तु त्वा तु । \newline
54. तू नो॑ न॒स्तु तू नः॑ । \newline
55. न॒ उपोप॑ नो न॒ उप॑ । \newline

\textbf{Ghana Paata } \newline

1. व्र॒ते तव॒ तव॑ व्र॒ते व्र॒ते तवाना॑ग॒सो ऽना॑गस॒स्तव॑ व्र॒ते व्र॒ते तवाना॑गसः । \newline
2. तवाना॑ग॒सो ऽना॑गस॒स्तव॒ तवाना॑गसो॒ अदि॑तये॒ अदि॑त॒ये ऽना॑गस॒स्तव॒ तवाना॑गसो॒ अदि॑तये । \newline
3. अना॑गसो॒ अदि॑तये॒ अदि॑त॒ये ऽना॑ग॒सो ऽना॑गसो॒ अदि॑तये स्याम स्या॒मादि॑त॒ये ऽना॑ग॒सो ऽना॑गसो॒ अदि॑तये स्याम । \newline
4. अदि॑तये स्याम स्या॒मादि॑तये॒ अदि॑तये स्याम । \newline
5. स्या॒मेति॑ स्याम । \newline
6. द॒धि॒क्राव्.ण्णो॑ अकारिष मकारिषम् दधि॒क्राव्.ण्णो॑ दधि॒क्राव्.ण्णो॑ अकारिषम् जि॒ष्णोर् जि॒ष्णो र॑कारिषम् दधि॒क्राव्.ण्णो॑ दधि॒क्राव्.ण्णो॑ अकारिषम् जि॒ष्णोः । \newline
7. द॒धि॒क्राव्.ण्ण॒ इति॑ दधि - क्राव्.ण्णः॑ । \newline
8. अ॒का॒रि॒ष॒म् जि॒ष्णोर् जि॒ष्णोर॑कारिष मकारिषम् जि॒ष्णोरश्व॒स्याश्व॑स्य जि॒ष्णो र॑कारिष मकारिषम् जि॒ष्णोरश्व॑स्य । \newline
9. जि॒ष्णो रश्व॒स्याश्व॑स्य जि॒ष्णोर् जि॒ष्णो रश्व॑स्य वा॒जिनो॑ वा॒जिनो॒ अश्व॑स्य जि॒ष्णोर् जि॒ष्णो रश्व॑स्य वा॒जिनः॑ । \newline
10. अश्व॑स्य वा॒जिनो॑ वा॒जिनो॒ अश्व॒स्याश्व॑स्य वा॒जिनः॑ । \newline
11. वा॒जिन॒ इति॑ वा॒जिनः॑ । \newline
12. सु॒र॒भि नो॑ नः सुर॒भि सु॑र॒भि नो॒ मुखा॒ मुखा॑ नः सुर॒भि सु॑र॒भि नो॒ मुखा᳚ । \newline
13. नो॒ मुखा॒ मुखा॑ नो नो॒ मुखा॑ करत् कर॒न् मुखा॑ नो नो॒ मुखा॑ करत् । \newline
14. मुखा॑ करत् कर॒न् मुखा॒ मुखा॑ कर॒त् प्र प्र क॑र॒न् मुखा॒ मुखा॑ कर॒त् प्र । \newline
15. क॒र॒त् प्र प्र क॑रत् कर॒त् प्र णो॑ नः॒ प्र क॑रत् कर॒त् प्र णः॑ । \newline
16. प्र णो॑ नः॒ प्र प्र ण॒ आयू॒(ग्ग्॒) ष्यायू(ग्म्॑)षि नः॒ प्र प्र ण॒ आयू(ग्म्॑)षि । \newline
17. न॒ आयू॒(ग्ग्॒) ष्यायू(ग्म्॑)षि नो न॒ आयू(ग्म्॑)षि तारिषत् तारिष॒दायू(ग्म्॑)षि नो न॒ आयू(ग्म्॑)षि तारिषत् । \newline
18. आयू(ग्म्॑)षि तारिषत् तारिष॒दायू॒(ग्ग्॒) ष्यायू(ग्म्॑)षि तारिषत् । \newline
19. ता॒रि॒ष॒दिति॑ तारिषत् । \newline
20. आ द॑धि॒क्रा द॑धि॒क्रा आ द॑धि॒क्राः शव॑सा॒ शव॑सा दधि॒क्रा आ द॑धि॒क्राः शव॑सा । \newline
21. द॒धि॒क्राः शव॑सा॒ शव॑सा दधि॒क्रा द॑धि॒क्राः शव॑सा॒ पञ्च॒ पञ्च॒ शव॑सा दधि॒क्रा द॑धि॒क्राः शव॑सा॒ पञ्च॑ । \newline
22. द॒धि॒क्रा इति॑ दधि - क्राः । \newline
23. शव॑सा॒ पञ्च॒ पञ्च॒ शव॑सा॒ शव॑सा॒ पञ्च॑ कृ॒ष्टीः कृ॒ष्टीः पञ्च॒ शव॑सा॒ शव॑सा॒ पञ्च॑ कृ॒ष्टीः । \newline
24. पञ्च॑ कृ॒ष्टीः कृ॒ष्टीः पञ्च॒ पञ्च॑ कृ॒ष्टीः सूर्यः॒ सूर्यः॑ कृ॒ष्टीः पञ्च॒ पञ्च॑ कृ॒ष्टीः सूर्यः॑ । \newline
25. कृ॒ष्टीः सूर्यः॒ सूर्यः॑ कृ॒ष्टीः कृ॒ष्टीः सूर्य॑ इवे व॒ सूर्यः॑ कृ॒ष्टीः कृ॒ष्टीः सूर्य॑ इव । \newline
26. सूर्य॑ इवे व॒ सूर्यः॒ सूर्य॑ इव॒ ज्योति॑षा॒ ज्योति॑षेव॒ सूर्यः॒ सूर्य॑ इव॒ ज्योति॑षा । \newline
27. इ॒व॒ ज्योति॑षा॒ ज्योति॑षेवे व॒ ज्योति॑षा॒ ऽपो अ॒पो ज्योति॑षेवे व॒ ज्योति॑षा॒ ऽपः । \newline
28. ज्योति॑षा॒ ऽपो अ॒पो ज्योति॑षा॒ ज्योति॑षा॒ ऽपस्त॑तान तताना॒पो ज्योति॑षा॒ ज्योति॑षा॒ ऽपस्त॑तान । \newline
29. अ॒पस्त॑तान तताना॒पो अ॒पस्त॑तान । \newline
30. त॒ता॒नेति॑ ततान । \newline
31. स॒ह॒स्र॒साः श॑त॒साः श॑त॒साः स॑हस्र॒साः स॑हस्र॒साः श॑त॒सा वा॒जी वा॒जी श॑त॒साः स॑हस्र॒साः स॑हस्र॒साः श॑त॒सा वा॒जी । \newline
32. स॒ह॒स्र॒सा इति॑ सहस्र - साः । \newline
33. श॒त॒सा वा॒जी वा॒जी श॑त॒साः श॑त॒सा वा॒ज्यर्वा ऽर्वा॑ वा॒जी श॑त॒साः श॑त॒सा वा॒ज्यर्वा᳚ । \newline
34. श॒त॒सा इति॑ शत - साः । \newline
35. वा॒ज्यर्वा ऽर्वा॑ वा॒जी वा॒ज्यर्वा॑ पृ॒णक्तु॑ पृ॒णक्त्वर्वा॑ वा॒जी वा॒ज्यर्वा॑ पृ॒णक्तु॑ । \newline
36. अर्वा॑ पृ॒णक्तु॑ पृ॒णक्त्वर्वा ऽर्वा॑ पृ॒णक्तु॒ मद्ध्वा॒ मद्ध्वा॑ पृ॒णक्त्वर्वा ऽर्वा॑ पृ॒णक्तु॒ मद्ध्वा᳚ । \newline
37. पृ॒णक्तु॒ मद्ध्वा॒ मद्ध्वा॑ पृ॒णक्तु॑ पृ॒णक्तु॒ मद्ध्वा॒ सꣳ सम् मद्ध्वा॑ पृ॒णक्तु॑ पृ॒णक्तु॒ मद्ध्वा॒ सम् । \newline
38. मद्ध्वा॒ सꣳ सम् मद्ध्वा॒ मद्ध्वा॒ स मि॒मेमा सम् मद्ध्वा॒ मद्ध्वा॒ स मि॒मा । \newline
39. स मि॒मेमा सꣳ स मि॒मा वचा(ग्म्॑)सि॒ वचा(ग्म्॑)सी॒मा सꣳ स मि॒मा वचा(ग्म्॑)सि । \newline
40. इ॒मा वचा(ग्म्॑)सि॒ वचा(ग्म्॑)सी॒मेमा वचा(ग्म्॑)सि । \newline
41. वचा॒(ग्म्॒)सीति॒ वचा(ग्म्॑)सि । \newline
42. अ॒ग्निर् मू॒र्द्धा मू॒र्द्धा ऽग्निर॒ग्निर् मू॒र्द्धा भुवो॒ भुवो॑ मू॒र्द्धा ऽग्निर॒ग्निर् मू॒र्द्धा भुवः॑ । \newline
43. मू॒र्द्धा भुवो॒ भुवो॑ मू॒र्द्धा मू॒र्द्धा भुवः॑ । \newline
44. भुव॒ इति॒ भुवः॑ । \newline
45. मरु॑तो॒ यद् यन् मरु॑तो॒ मरु॑तो॒ यद्ध॑ ह॒ यन् मरु॑तो॒ मरु॑तो॒ यद्ध॑ । \newline
46. यद्ध॑ ह॒ यद् यद्ध॑ वो वो ह॒ यद् यद्ध॑ वः । \newline
47. ह॒ वो॒ वो॒ ह॒ ह॒ वो॒ दि॒वो दि॒वो वो॑ ह ह वो दि॒वः । \newline
48. वो॒ दि॒वो दि॒वो वो॑ वो दि॒वः सु॑म्ना॒यन्तः॑ सुम्ना॒यन्तो॑ दि॒वो वो॑ वो दि॒वः सु॑म्ना॒यन्तः॑ । \newline
49. दि॒वः सु॑म्ना॒यन्तः॑ सुम्ना॒यन्तो॑ दि॒वो दि॒वः सु॑म्ना॒यन्तो॒ हवा॑महे॒ हवा॑महे सुम्ना॒यन्तो॑ दि॒वो दि॒वः सु॑म्ना॒यन्तो॒ हवा॑महे । \newline
50. सु॒म्ना॒यन्तो॒ हवा॑महे॒ हवा॑महे सुम्ना॒यन्तः॑ सुम्ना॒यन्तो॒ हवा॑महे । \newline
51. सु॒म्ना॒यन्त॒ इति॑ सुम्न - यन्तः॑ । \newline
52. हवा॑मह॒ इति॒ हवा॑महे । \newline
53. आ तु त्वा तू नो॑ न॒स्त्वा तू नः॑ । \newline
54. तू नो॑ न॒स्तु तू न॒ उपोप॑ न॒स्तु तू न॒ उप॑ । \newline
55. न॒ उपोप॑ नो न॒ उप॑ गन्तन गन्त॒नोप॑ नो न॒ उप॑ गन्तन । \newline
\pagebreak
\markright{ TS 1.5.11.5  \hfill https://www.vedavms.in \hfill}
\addcontentsline{toc}{section}{ TS 1.5.11.5 }
\section*{ TS 1.5.11.5 }

\textbf{TS 1.5.11.5 } \newline
\textbf{Samhita Paata} \newline

उप॑ गन्तन ॥ या वः॒ शर्म॑ शशमा॒नाय॒ सन्ति॑ त्रि॒धातू॑नि दा॒शुषे॑ यच्छ॒ताधि॑ । अ॒स्मभ्यं॒ तानि॑ मरुतो॒ वि य॑न्त र॒यिं नो॑ धत्त वृषणः सु॒वीरं᳚ ॥ अदि॑तिर् न उरुष्य॒त्वदि॑तिः॒ शर्म॑ यच्छतु । अदि॑तिः पा॒त्वꣳह॑सः ॥ म॒हीमू॒षु मा॒तरꣳ॑ सुव्र॒ताना॑मृ॒तस्य॒ पत्नी॒मव॑से हुवेम । तु॒वि॒क्ष॒त्रा-म॒जर॑न्ती-मुरू॒चीꣳ सु॒शर्मा॑ण॒मदि॑तिꣳ सु॒प्रणी॑तिं ॥ सु॒त्रामा॑णं पृथि॒वीं द्याम॑ने॒हसꣳ॑ सु॒शर्मा॑ण॒ ( ) मदि॑तिꣳ सु॒प्रणी॑तिं । दैवीं॒ नावꣳ॑ स्वरि॒त्रा-मना॑गस॒-मस्र॑वन्ती॒मा रु॑हेमा स्व॒स्तये᳚ ॥ इ॒माꣳ सु नाव॒माऽरु॑हꣳ श॒तारि॑त्राꣳ श॒तस्फ्यां᳚ । अच्छि॑द्रां पारयि॒ष्णुं ॥ \newline

\textbf{Pada Paata} \newline

उपेति॑ । ग॒न्त॒न॒ ॥ या । वः॒ । शर्म॑ । श॒श॒मा॒नाय॑ । सन्ति॑ । त्रि॒धातू॒नीति॑ त्रि - धातू॑नि । दा॒शुषे᳚ । य॒च्छ॒त॒ । अधि॑ ॥ अ॒स्मभ्य॒मित्य॒स्म-भ्य॒म् । तानि॑ । म॒रु॒तः॒ । वीति॑ । य॒न्त॒ । र॒यिम् । नः॒ । ध॒त्त॒ । वृ॒ष॒णः॒ । सु॒वीर॒मिति॑ सु-वीर᳚म् ॥ अदि॑तिः । नः॒ । उ॒रु॒ष्य॒तु॒ । अदि॑तिः । शर्म॑ । य॒च्छ॒तु॒ ॥ अदि॑तिः । पा॒तु॒ । अꣳह॑सः ॥ म॒हीम् । उ॒ । स्विति॑ । मा॒तर᳚म् । सु॒व्र॒ताना॒मिति॑ सु - व्र॒ताना᳚म् । ऋ॒तस्य॑ । पत्नी᳚म् । अव॑से । हु॒वे॒म॒ ॥ तु॒वि॒क्ष॒त्रामिति॑ तुवि - क्ष॒त्राम् । अ॒जर॑न्तीम् । उ॒रू॒चीम् । सु॒शर्मा॑ण॒मिति॑ सु - शर्मा॑णम् । अदि॑तिम् । सु॒प्रणी॑ति॒मिति॑ सु - प्रणी॑तिम् ॥ सु॒त्रामा॑ण॒मिति॑ सु - त्रामा॑णम् । पृ॒थि॒वीम् । द्याम् । अ॒ने॒हस᳚म् । सु॒शर्मा॑ण॒मिति॑ सु - शर्मा॑णम् ( ) । अदि॑तिम् । सु॒प्रणी॑ति॒मिति॑ सु - प्रणी॑तिम् ॥ दैवी᳚म् । नाव᳚म् । स्व॒रि॒त्रामिति॑ सु - अ॒रि॒त्राम् । अना॑गसम् । अस्र॑वन्तीम् । एति॑ । रु॒हे॒म॒ । स्व॒स्तये᳚ ॥ इ॒माम् । स्विति॑ । नाव᳚म् । एति॑ । अ॒रु॒ह॒म् । श॒तारि॑त्रा॒मिति॑ श॒त - अ॒रि॒त्रा॒म् । श॒तस्फ्या॒मिति॑ श॒त - स्फ्या॒म् ॥ अच्छि॑द्राम् । पा॒र॒यि॒ष्णुम् ॥  \newline


\textbf{Krama Paata} \newline

उप॑ गन्तन । ग॒न्त॒नेति॑ गन्तन ॥ या वः॑ । वः॒ शर्म॑ । शर्म॑ शशमा॒नाय॑ । श॒श॒मा॒नाय॒ सन्ति॑ । सन्ति॑ त्रि॒धातू॑नि । त्रि॒धातू॑नि दा॒शुषे᳚ । त्रि॒धातू॒नीति॑ त्रि - धातू॑नि । दा॒शुषे॑ यच्छत । य॒च्छ॒ताधि॑ । अधीत्यधि॑ ॥ अ॒स्मभ्य॒म् तानि॑ । अ॒स्मभ्य॒मित्य॒स्म - भ्य॒म् । तानि॑ मरुतः । म॒रु॒तो॒ वि । वि य॑न्त । य॒न्त॒ र॒यिम् । र॒यिम् नः॑ । नो॒ ध॒त्त॒ । ध॒त्त॒ वृ॒ष॒णः॒ । वृ॒ष॒णः॒ सु॒वीर᳚म् । सु॒वीर॒मिति॑ सु - वीर᳚म् । अदि॑तिर् नः । न॒ उ॒रु॒ष्य॒तु॒ । उ॒रु॒ष्य॒त्वदि॑तिः । अदि॑तिः॒ शर्म॑ । शर्म॑ यच्छतु । य॒च्छ॒त्विति॑ यच्छतु ॥ अदि॑तिः पातु । पा॒त्वꣳह॑सः । अꣳह॑स॒ इत्यꣳह॑सः ॥ म॒हीमु॑ । ऊ॒षु । सु मा॒तर᳚म् । मा॒तरꣳ॑ सुव्र॒ताना᳚म् । सु॒व्र॒ताना॑मृ॒तस्य॑ । सु॒व्र॒ताना॒मिति॑ सु - व्र॒ताना᳚म् । ऋ॒तस्य॒ पत्नी᳚म् । पत्नी॒मव॑से । अव॑से हुवेम । हु॒वे॒मेति॑ हुवेम ॥ तु॒वि॒क्ष॒त्राम॒जर॑न्तीम् । तु॒वि॒क्ष॒त्रामिति॑ तुवि - क्ष॒त्राम् । अ॒जर॑न्तीमुरू॒चीम् । उ॒रू॒चीꣳ सु॒शर्मा॑णम् । सु॒शर्मा॑ण॒मदि॑तिम् । सु॒शर्मा॑ण॒मिति॑ सु - शर्मा॑णम् । अदि॑तिꣳ सु॒प्रणी॑तिम् । सु॒प्रणी॑ति॒मिति॑ सु - प्रणी॑तिम् ॥ सु॒त्रामा॑णम् पृथि॒वीम् । सु॒त्रामा॑ण॒मिति॑ सु - त्रामा॑णम् । पृ॒थि॒वीम् द्याम् । द्याम॑ने॒हस᳚म् । 
अ॒ने॒हसꣳ॑ सु॒शर्मा॑णम् ( ) । सु॒शर्मा॑ण॒मदि॑तिम् । सु॒शर्मा॑ण॒मिति॑ सु - शर्मा॑णम् । अदि॑तिꣳ सु॒प्रणी॑तिम् । सु॒प्रणी॑ति॒मिति॑ सु - प्रणी॑तिम् ॥ दैवी॒म् नाव᳚म् । नावꣳ॑ स्वरि॒त्राम् । स्व॒रि॒त्रामना॑गसम् । स्व॒रि॒त्रामिति॑ सु - अ॒रि॒त्राम् । अना॑गस॒मस्र॑वन्तीम् । अस्र॑वन्ती॒मा । आ रु॑हेम । रु॒हे॒मा॒ स्व॒स्तये᳚ । स्व॒स्तय॒ इति॑ स्व॒स्तये᳚ ॥ इ॒माꣳ सु । सु नाव᳚म् । नाव॒मा । आऽरु॑हम् । अ॒रु॒हꣳ॒॒ श॒तारि॑त्राम् । श॒तारि॑त्राꣳ श॒तस्फ्या᳚म् । श॒तारि॑त्रा॒मिति॑ श॒त - अ॒रि॒त्रा॒म् । श॒तस्फ्या॒मिति॑ श॒त - स्फ्या॒म् ॥ अच्छि॑द्राम् पारयि॒ष्णुम् । पा॒र॒यि॒ष्णुमिति॑ पारयि॒ष्णुम् । \newline

\textbf{Jatai Paata} \newline

1. उप॑ गन्तन गन्त॒नोपोप॑ गन्तन । \newline
2. ग॒न्त॒नेति॑ गन्तन । \newline
3. या वो॑ वो॒ या या वः॑ । \newline
4. वः॒ शर्म॒ शर्म॑ वो वः॒ शर्म॑ । \newline
5. शर्म॑ शशमा॒नाय॑ शशमा॒नाय॒ शर्म॒ शर्म॑ शशमा॒नाय॑ । \newline
6. श॒श॒मा॒नाय॒ सन्ति॒ सन्ति॑ शशमा॒नाय॑ शशमा॒नाय॒ सन्ति॑ । \newline
7. सन्ति॑ त्रि॒धातू॑नि त्रि॒धातू॑नि॒ सन्ति॒ सन्ति॑ त्रि॒धातू॑नि । \newline
8. त्रि॒धातू॑नि दा॒शुषे॑ दा॒शुषे᳚ त्रि॒धातू॑नि त्रि॒धातू॑नि दा॒शुषे᳚ । \newline
9. त्रि॒धातू॒नीति॑ त्रि - धातू॑नि । \newline
10. दा॒शुषे॑ यच्छत यच्छत दा॒शुषे॑ दा॒शुषे॑ यच्छत । \newline
11. य॒च्छ॒ताध्यधि॑ यच्छत यच्छ॒ताधि॑ । \newline
12. अधीत्यधि॑ । \newline
13. अ॒स्मभ्य॒म् तानि॒ तान्य॒स्मभ्य॑ म॒स्मभ्य॒म् तानि॑ । \newline
14. अ॒स्मभ्य॒मित्य॒स्म - भ्य॒म् । \newline
15. तानि॑ मरुतो मरुत॒स्तानि॒ तानि॑ मरुतः । \newline
16. म॒रु॒तो॒ वि वि म॑रुतो मरुतो॒ वि । \newline
17. वि य॑न्त यन्त॒ वि वि य॑न्त । \newline
18. य॒न्त॒ र॒यिꣳ र॒यिं ॅय॑न्त यन्त र॒यिम् । \newline
19. र॒यिन्नो॑ नो र॒यिꣳ र॒यिन्नः॑ । \newline
20. नो॒ ध॒त्त॒ ध॒त्त॒ नो॒ नो॒ ध॒त्त॒ । \newline
21. ध॒त्त॒ वृ॒ष॒णो॒ वृ॒ष॒णो॒ ध॒त्त॒ ध॒त्त॒ वृ॒ष॒णः॒ । \newline
22. वृ॒ष॒णः॒ सु॒वीर(ग्म्॑) सु॒वीरं॑ ॅवृषणो वृषणः सु॒वीर᳚म् । \newline
23. सु॒वीर॒मिति॑ सु - वीर᳚म् । \newline
24. अदि॑तिर् नो नो॒ अदि॑ति॒रदि॑तिर् नः । \newline
25. न॒ उ॒रु॒ष्य॒तू॒रु॒ष्य॒तु॒ नो॒ न॒ उ॒रु॒ष्य॒तु॒ । \newline
26. उ॒रु॒ष्य॒ त्वदि॑ति॒ रदि॑ति रुरुष्यतूरुष्य॒त्वदि॑तिः । \newline
27. अदि॑तिः॒ शर्म॒ शर्मादि॑ति॒रदि॑तिः॒ शर्म॑ । \newline
28. शर्म॑ यच्छतु यच्छतु॒ शर्म॒ शर्म॑ यच्छतु । \newline
29. य॒च्छ॒त्विति॑ यच्छतु । \newline
30. अदि॑तिः पातु पा॒त्वदि॑ति॒रदि॑तिः पातु । \newline
31. पा॒त्वꣳह॑सो॒ अꣳह॑स स्पातु पा॒त्वꣳह॑सः । \newline
32. अꣳह॑स॒ इत्यꣳह॑सः । \newline
33. म॒ही मु॑ वु म॒हीम् म॒ही मु॑ । \newline
34. ऊ॒ षु सू॑ षु । \newline
35. सु मा॒तर॑म् मा॒तर॒(ग्म्॒) सु सु मा॒तर᳚म् । \newline
36. मा॒तर(ग्म्॑) सुव्र॒ताना(ग्म्॑) सुव्र॒ताना᳚म् मा॒तर॑म् मा॒तर(ग्म्॑) सुव्र॒ताना᳚म् । \newline
37. सु॒व्र॒ताना॑ मृ॒तस्य॒ र्तस्य॑ सुव्र॒ताना(ग्म्॑) सुव्र॒ताना॑ मृ॒तस्य॑ । \newline
38. सु॒व्र॒ताना॒मिति॑ सु - व्र॒ताना᳚म् । \newline
39. ऋ॒तस्य॒ पत्नी॒म् पत्नी॑ मृ॒तस्य॒ र्तस्य॒ पत्नी᳚म् । \newline
40. पत्नी॒ मव॒से ऽव॑से॒ पत्नी॒म् पत्नी॒ मव॑से । \newline
41. अव॑से हुवेम हुवे॒माव॒से ऽव॑से हुवेम । \newline
42. हु॒वे॒मेति॑ हुवेम । \newline
43. तु॒वि॒क्ष॒त्रा म॒जर॑न्ती म॒जर॑न्तीम् तुविक्ष॒त्राम् तु॑विक्ष॒त्रा म॒जर॑न्तीम् । \newline
44. तु॒वि॒क्ष॒त्रामिति॑ तुवि - क्ष॒त्राम् । \newline
45. अ॒जर॑न्ती मुरू॒ची मु॑रू॒ची म॒जर॑न्ती म॒जर॑न्ती मुरू॒चीम् । \newline
46. उ॒रू॒चीꣳ सु॒शर्मा॑णꣳ सु॒शर्मा॑ण मुरू॒ची मु॑रू॒चीꣳ सु॒शर्मा॑णम् । \newline
47. सु॒शर्मा॑ण॒ मदि॑ति॒ मदि॑तिꣳ सु॒शर्मा॑णꣳ सु॒शर्मा॑ण॒ मदि॑तिम् । \newline
48. सु॒शर्मा॑ण॒मिति॑ सु - शर्मा॑णम् । \newline
49. अदि॑तिꣳ सु॒प्रणी॑तिꣳ सु॒प्रणी॑ति॒ मदि॑ति॒ मदि॑तिꣳ सु॒प्रणी॑तिम् । \newline
50. सु॒प्रणी॑ति॒मिति॑ सु - प्रणी॑तिम् । \newline
51. सु॒त्रामा॑णम् पृथि॒वीम् पृ॑थि॒वीꣳ सु॒त्रामा॑णꣳ सु॒त्रामा॑णम् पृथि॒वीम् । \newline
52. सु॒त्रामा॑ण॒मिति॑ सु - त्रामा॑णम् । \newline
53. पृ॒थि॒वीम् द्याम् द्याम् पृ॑थि॒वीम् पृ॑थि॒वीम् द्याम् । \newline
54. द्या म॑ने॒हस॑ मने॒हस॒म् द्याम् द्या म॑ने॒हस᳚म् । \newline
55. अ॒ने॒हस(ग्म्॑) सु॒शर्मा॑णꣳ सु॒शर्मा॑ण मने॒हस॑ मने॒हस(ग्म्॑) सु॒शर्मा॑णम् । \newline
56. सु॒शर्मा॑ण॒ मदि॑ति॒ मदि॑तिꣳ सु॒शर्मा॑णꣳ सु॒शर्मा॑ण॒ मदि॑तिम् । \newline
57. सु॒शर्मा॑ण॒मिति॑ सु - शर्मा॑णम् । \newline
58. अदि॑तिꣳ सु॒प्रणी॑तिꣳ सु॒प्रणी॑ति॒ मदि॑ति॒ मदि॑तिꣳ सु॒प्रणी॑तिम् । \newline
59. सु॒प्रणी॑ति॒मिति॑ सु - प्रणी॑तिम् । \newline
60. दैवी॒न्नाव॒न्नाव॒म् दैवी॒म् दैवी॒न्नाव᳚म् । \newline
61. नाव(ग्ग्॑) स्वरि॒त्राꣳ स्व॑रि॒त्रान्नाव॒न्नाव(ग्ग्॑) स्वरि॒त्राम् । \newline
62. स्व॒रि॒त्रा मना॑गस॒ मना॑गसꣳ स्वरि॒त्राꣳ स्व॑रि॒त्रा मना॑गसम् । \newline
63. स्व॒रि॒त्रामिति॑ सु - अ॒रि॒त्राम् । \newline
64. अना॑गस॒ मस्र॑वन्ती॒ मस्र॑वन्ती॒ मना॑गस॒ मना॑गस॒ मस्र॑वन्तीम् । \newline
65. अस्र॑वन्ती॒ मा ऽस्र॑वन्ती॒ मस्र॑वन्ती॒ मा । \newline
66. आ रु॑हेम रुहे॒मा रु॑हेम । \newline
67. रु॒हे॒मा॒ स्व॒स्तये᳚ स्व॒स्तये॑ रुहेम रुहेमा स्व॒स्तये᳚ । \newline
68. स्व॒स्तय॒ इति॑ स्व॒स्तये᳚ । \newline
69. इ॒माꣳ सु स्वि॑मा मि॒माꣳ सु । \newline
70. सु नाव॒न्नाव॒(ग्म्॒) सु सु नाव᳚म् । \newline
71. नाव॒ मा नाव॒न्नाव॒ मा । \newline
72. आ ऽरु॑ह मरुह॒ मा ऽरु॑हम् । \newline
73. अ॒रु॒ह॒(ग्म्॒) श॒तारि॑त्राꣳ श॒तारि॑त्रा मरुह मरुहꣳ श॒तारि॑त्राम् । \newline
74. श॒तारि॑त्राꣳ श॒तस्फ्या(ग्म्॑) श॒तस्फ्या(ग्म्॑) श॒तारि॑त्राꣳ श॒तारि॑त्राꣳ श॒तस्फ्या᳚म् । \newline
75. श॒तारि॑त्रा॒मिति॑ श॒त - अ॒रि॒त्रा॒म् । \newline
76. श॒तस्फ्या॒मिति॑ श॒त - स्फ्या॒म् । \newline
77. अच्छि॑द्राम् पारयि॒ष्णुम् पा॑रयि॒ष्णु मच्छि॑द्रा॒ मच्छि॑द्राम् पारयि॒ष्णुम् । \newline
78. पा॒र॒यि॒ष्णुमिति॑ पारयि॒ष्णुम् । \newline

\textbf{Ghana Paata } \newline

1. उप॑ गन्तन गन्त॒नोपोप॑ गन्तन । \newline
2. ग॒न्त॒नेति॑ गन्तन । \newline
3. या वो॑ वो॒ या या वः॒ शर्म॒ शर्म॑ वो॒ या या वः॒ शर्म॑ । \newline
4. वः॒ शर्म॒ शर्म॑ वो वः॒ शर्म॑ शशमा॒नाय॑ शशमा॒नाय॒ शर्म॑ वो वः॒ शर्म॑ शशमा॒नाय॑ । \newline
5. शर्म॑ शशमा॒नाय॑ शशमा॒नाय॒ शर्म॒ शर्म॑ शशमा॒नाय॒ सन्ति॒ सन्ति॑ शशमा॒नाय॒ शर्म॒ शर्म॑ शशमा॒नाय॒ सन्ति॑ । \newline
6. श॒श॒मा॒नाय॒ सन्ति॒ सन्ति॑ शशमा॒नाय॑ शशमा॒नाय॒ सन्ति॑ त्रि॒धातू॑नि त्रि॒धातू॑नि॒ सन्ति॑ शशमा॒नाय॑ शशमा॒नाय॒ सन्ति॑ त्रि॒धातू॑नि । \newline
7. सन्ति॑ त्रि॒धातू॑नि त्रि॒धातू॑नि॒ सन्ति॒ सन्ति॑ त्रि॒धातू॑नि दा॒शुषे॑ दा॒शुषे᳚ त्रि॒धातू॑नि॒ सन्ति॒ सन्ति॑ त्रि॒धातू॑नि दा॒शुषे᳚ । \newline
8. त्रि॒धातू॑नि दा॒शुषे॑ दा॒शुषे᳚ त्रि॒धातू॑नि त्रि॒धातू॑नि दा॒शुषे॑ यच्छत यच्छत दा॒शुषे᳚ त्रि॒धातू॑नि त्रि॒धातू॑नि दा॒शुषे॑ यच्छत । \newline
9. त्रि॒धातू॒नीति॑ त्रि - धातू॑नि । \newline
10. दा॒शुषे॑ यच्छत यच्छत दा॒शुषे॑ दा॒शुषे॑ यच्छ॒ताध्यधि॑ यच्छत दा॒शुषे॑ दा॒शुषे॑ यच्छ॒ताधि॑ । \newline
11. य॒च्छ॒ताध्यधि॑ यच्छत यच्छ॒ताधि॑ । \newline
12. अधीत्यधि॑ । \newline
13. अ॒स्मभ्य॒म् तानि॒ तान्य॒स्मभ्य॑ म॒स्मभ्य॒म् तानि॑ मरुतो मरुत॒ स्तान्य॒स्मभ्य॑ म॒स्मभ्य॒म् तानि॑ मरुतः । \newline
14. अ॒स्मभ्य॒मित्य॒स्म - भ्य॒म् । \newline
15. तानि॑ मरुतो मरुत॒स्तानि॒ तानि॑ मरुतो॒ वि वि म॑रुत॒स्तानि॒ तानि॑ मरुतो॒ वि । \newline
16. म॒रु॒तो॒ वि वि म॑रुतो मरुतो॒ वि य॑न्त यन्त॒ वि म॑रुतो मरुतो॒ वि य॑न्त । \newline
17. वि य॑न्त यन्त॒ वि वि य॑न्त र॒यिꣳ र॒यिं ॅय॑न्त॒ वि वि य॑न्त र॒यिम् । \newline
18. य॒न्त॒ र॒यिꣳ र॒यिं ॅय॑न्त यन्त र॒यिन्नो॑ नो र॒यिं ॅय॑न्त यन्त र॒यिन्नः॑ । \newline
19. र॒यिन्नो॑ नो र॒यिꣳ र॒यिन्नो॑ धत्त धत्त नो र॒यिꣳ र॒यिन्नो॑ धत्त । \newline
20. नो॒ ध॒त्त॒ ध॒त्त॒ नो॒ नो॒ ध॒त्त॒ वृ॒ष॒णो॒ वृ॒ष॒णो॒ ध॒त्त॒ नो॒ नो॒ ध॒त्त॒ वृ॒ष॒णः॒ । \newline
21. ध॒त्त॒ वृ॒ष॒णो॒ वृ॒ष॒णो॒ ध॒त्त॒ ध॒त्त॒ वृ॒ष॒णः॒ सु॒वीर(ग्म्॑) सु॒वीरं॑ ॅवृषणो धत्त धत्त वृषणः सु॒वीर᳚म् । \newline
22. वृ॒ष॒णः॒ सु॒वीर(ग्म्॑) सु॒वीरं॑ ॅवृषणो वृषणः सु॒वीर᳚म् । \newline
23. सु॒वीर॒मिति॑ सु - वीर᳚म् । \newline
24. अदि॑तिर् नो नो॒ अदि॑ति॒ रदि॑तिर् न उरुष्यतूरुष्यतु नो॒ अदि॑ति॒ रदि॑तिर् न उरुष्यतु । \newline
25. न॒ उ॒रु॒ष्य॒तू॒रु॒ष्य॒तु॒ नो॒ न॒ उ॒रु॒ष्य॒ त्वदि॑ति॒ रदि॑तिरुरुष्यतु नो न उरुष्य॒त्वदि॑तिः । \newline
26. उ॒रु॒ष्य॒ त्वदि॑ति॒ रदि॑ति रुरुष्यतूरुष्य॒ त्वदि॑तिः॒ शर्म॒ शर्मादि॑ति रुरुष्यतूरुष्य॒ त्वदि॑तिः॒ शर्म॑ । \newline
27. अदि॑तिः॒ शर्म॒ शर्मादि॑ति॒ रदि॑तिः॒ शर्म॑ यच्छतु यच्छतु॒ शर्मादि॑ति॒ रदि॑तिः॒ शर्म॑ यच्छतु । \newline
28. शर्म॑ यच्छतु यच्छतु॒ शर्म॒ शर्म॑ यच्छतु । \newline
29. य॒च्छ॒त्विति॑ यच्छतु । \newline
30. अदि॑तिः पातु पा॒त्वदि॑ति॒ रदि॑तिः पा॒त्वꣳह॑सो॒ अꣳह॑स स्पा॒त्वदि॑ति॒ रदि॑तिः पा॒त्वꣳह॑सः । \newline
31. पा॒त्वꣳह॑सो॒ अꣳह॑स स्पातु पा॒त्वꣳह॑सः । \newline
32. अꣳह॑स॒ इत्यꣳह॑सः । \newline
33. म॒ही मु॑ वु म॒हीम् म॒ही मू॒ षु सू॑ म॒हीम् म॒ही मू॒ षु । \newline
34. ऊ॒ षु सू॑ षु मा॒तर॑म् मा॒तरꣳ॒॒ सू॑ षु मा॒तर᳚म् । \newline
35. सु मा॒तर॑म् मा॒तर॒(ग्म्॒) सु सु मा॒तर(ग्म्॑) सुव्र॒ताना(ग्म्॑) सुव्र॒ताना᳚म् मा॒तर॒(ग्म्॒) सु सु मा॒तर(ग्म्॑) सुव्र॒ताना᳚म् । \newline
36. मा॒तर(ग्म्॑) सुव्र॒ताना(ग्म्॑) सुव्र॒ताना᳚म् मा॒तर॑म् मा॒तर(ग्म्॑) सुव्र॒ताना॑ मृ॒तस्य॒ र्तस्य॑ सुव्र॒ताना᳚म् मा॒तर॑म् मा॒तर(ग्म्॑) सुव्र॒ताना॑ मृ॒तस्य॑ । \newline
37. सु॒व्र॒ताना॑ मृ॒तस्य॒ र्तस्य॑ सुव्र॒ताना(ग्म्॑) सुव्र॒ताना॑ मृ॒तस्य॒ पत्नी॒म् पत्नी॑ मृ॒तस्य॑ सुव्र॒ताना(ग्म्॑) सुव्र॒ताना॑ मृ॒तस्य॒ पत्नी᳚म् । \newline
38. सु॒व्र॒ताना॒मिति॑ सु - व्र॒ताना᳚म् । \newline
39. ऋ॒तस्य॒ पत्नी॒म् पत्नी॑ मृ॒तस्य॒ र्तस्य॒ पत्नी॒ मव॒से ऽव॑से॒ पत्नी॑ मृ॒तस्य॒ र्तस्य॒ पत्नी॒ मव॑से । \newline
40. पत्नी॒ मव॒से ऽव॑से॒ पत्नी॒म् पत्नी॒ मव॑से हुवेम हुवे॒माव॑से॒ पत्नी॒म् पत्नी॒ मव॑से हुवेम । \newline
41. अव॑से हुवेम हुवे॒माव॒से ऽव॑से हुवेम । \newline
42. हु॒वे॒मेति॑ हुवेम । \newline
43. तु॒वि॒क्ष॒त्रा म॒जर॑न्ती म॒जर॑न्तीम् तुविक्ष॒त्राम् तु॑विक्ष॒त्रा म॒जर॑न्ती मुरू॒ची मु॑रू॒ची म॒जर॑न्तीम् तुविक्ष॒त्राम् तु॑विक्ष॒त्रा म॒जर॑न्ती मुरू॒चीम् । \newline
44. तु॒वि॒क्ष॒त्रामिति॑ तुवि - क्ष॒त्राम् । \newline
45. अ॒जर॑न्ती मुरू॒ची मु॑रू॒ची म॒जर॑न्ती म॒जर॑न्ती मुरू॒चीꣳ सु॒शर्मा॑णꣳ सु॒शर्मा॑ण मुरू॒ची म॒जर॑न्ती म॒जर॑न्ती मुरू॒चीꣳ सु॒शर्मा॑णम् । \newline
46. उ॒रू॒चीꣳ सु॒शर्मा॑णꣳ सु॒शर्मा॑ण मुरू॒ची मु॑रू॒चीꣳ सु॒शर्मा॑ण॒ मदि॑ति॒ मदि॑तिꣳ सु॒शर्मा॑ण मुरू॒ची मु॑रू॒चीꣳ सु॒शर्मा॑ण॒ मदि॑तिम् । \newline
47. सु॒शर्मा॑ण॒ मदि॑ति॒ मदि॑तिꣳ सु॒शर्मा॑णꣳ सु॒शर्मा॑ण॒ मदि॑तिꣳ सु॒प्रणी॑तिꣳ सु॒प्रणी॑ति॒ मदि॑तिꣳ सु॒शर्मा॑णꣳ सु॒शर्मा॑ण॒ मदि॑तिꣳ सु॒प्रणी॑तिम् । \newline
48. सु॒शर्मा॑ण॒मिति॑ सु - शर्मा॑णम् । \newline
49. अदि॑तिꣳ सु॒प्रणी॑तिꣳ सु॒प्रणी॑ति॒ मदि॑ति॒ मदि॑तिꣳ सु॒प्रणी॑तिम् । \newline
50. सु॒प्रणी॑ति॒मिति॑ सु - प्रणी॑तिम् । \newline
51. सु॒त्रामा॑णम् पृथि॒वीम् पृ॑थि॒वीꣳ सु॒त्रामा॑णꣳ सु॒त्रामा॑णम् पृथि॒वीम् द्याम् द्याम् पृ॑थि॒वीꣳ सु॒त्रामा॑णꣳ सु॒त्रामा॑णम् पृथि॒वीम् द्याम् । \newline
52. सु॒त्रामा॑ण॒मिति॑ सु - त्रामा॑णम् । \newline
53. पृ॒थि॒वीम् द्याम् द्याम् पृ॑थि॒वीम् पृ॑थि॒वीम् द्या म॑ने॒हस॑ मने॒हस॒म् द्याम् पृ॑थि॒वीम् पृ॑थि॒वीम् द्या म॑ने॒हस᳚म् । \newline
54. द्या म॑ने॒हस॑ मने॒हस॒म् द्याम् द्या म॑ने॒हस(ग्म्॑) सु॒शर्मा॑णꣳ सु॒शर्मा॑ण मने॒हस॒म् द्याम् द्या म॑ने॒हस(ग्म्॑) सु॒शर्मा॑णम् । \newline
55. अ॒ने॒हस(ग्म्॑) सु॒शर्मा॑णꣳ सु॒शर्मा॑ण मने॒हस॑ मने॒हस(ग्म्॑) सु॒शर्मा॑ण॒ मदि॑ति॒ मदि॑तिꣳ सु॒शर्मा॑ण मने॒हस॑ मने॒हस(ग्म्॑) सु॒शर्मा॑ण॒ मदि॑तिम् । \newline
56. सु॒शर्मा॑ण॒ मदि॑ति॒ मदि॑तिꣳ सु॒शर्मा॑णꣳ सु॒शर्मा॑ण॒ मदि॑तिꣳ सु॒प्रणी॑तिꣳ सु॒प्रणी॑ति॒ मदि॑तिꣳ सु॒शर्मा॑णꣳ सु॒शर्मा॑ण॒ मदि॑तिꣳ सु॒प्रणी॑तिम् । \newline
57. सु॒शर्मा॑ण॒मिति॑ सु - शर्मा॑णम् । \newline
58. अदि॑तिꣳ सु॒प्रणी॑तिꣳ सु॒प्रणी॑ति॒ मदि॑ति॒ मदि॑तिꣳ सु॒प्रणी॑तिम् । \newline
59. सु॒प्रणी॑ति॒मिति॑ सु - प्रणी॑तिम् । \newline
60. दैवी॒न्नाव॒म् नाव॒म् दैवी॒म् दैवी॒म् नाव(ग्ग्॑) स्वरि॒त्राꣳ स्व॑रि॒त्राम् नाव॒म् दैवी॒म् दैवी॒म् नाव(ग्ग्॑) स्वरि॒त्राम् । \newline
61. नाव(ग्ग्॑) स्वरि॒त्राꣳ स्व॑रि॒त्राम् नाव॒म् नाव(ग्ग्॑) स्वरि॒त्रा मना॑गस॒ मना॑गसꣳ स्वरि॒त्राम् नाव॒म् नाव(ग्ग्॑) स्वरि॒त्रा मना॑गसम् । \newline
62. स्व॒रि॒त्रा मना॑गस॒ मना॑गसꣳ स्वरि॒त्राꣳ स्व॑रि॒त्रा मना॑गस॒ मस्र॑वन्ती॒ मस्र॑वन्ती॒ मना॑गसꣳ स्वरि॒त्राꣳ स्व॑रि॒त्रा मना॑गस॒ मस्र॑वन्तीम् । \newline
63. स्व॒रि॒त्रामिति॑ सु - अ॒रि॒त्राम् । \newline
64. अना॑गस॒ मस्र॑वन्ती॒ मस्र॑वन्ती॒ मना॑गस॒ मना॑गस॒ मस्र॑वन्ती॒ मा ऽस्र॑वन्ती॒ मना॑गस॒ मना॑गस॒ मस्र॑वन्ती॒ मा । \newline
65. अस्र॑वन्ती॒ मा ऽस्र॑वन्ती॒ मस्र॑वन्ती॒ मा रु॑हेम रुहे॒मा ऽस्र॑वन्ती॒ मस्र॑वन्ती॒ मा रु॑हेम । \newline
66. आ रु॑हेम रुहे॒मा रु॑हेमा स्व॒स्तये᳚ स्व॒स्तये॑ रुहे॒मा रु॑हेमा स्व॒स्तये᳚ । \newline
67. रु॒हे॒मा॒ स्व॒स्तये᳚ स्व॒स्तये॑ रुहेम रुहेमा स्व॒स्तये᳚ । \newline
68. स्व॒स्तय॒ इति॑ स्व॒स्तये᳚ । \newline
69. इ॒माꣳ सु स्वि॑मा मि॒माꣳ सु नाव॒न्नाव॒(ग्ग्॒) स्वि॑मा मि॒माꣳ सु नाव᳚म् । \newline
70. सु नाव॒न्नाव॒(ग्म्॒) सु सु नाव॒ मा नाव॒(ग्म्॒) सु सु नाव॒ मा । \newline
71. नाव॒ मा नाव॒म् नाव॒ मा ऽरु॑ह मरुह॒ मा नाव॒म् नाव॒ मा ऽरु॑हम् । \newline
72. आ ऽरु॑ह मरुह॒ मा ऽरु॑हꣳ श॒तारि॑त्राꣳ श॒तारि॑त्रा मरुह॒ मा ऽरु॑हꣳ श॒तारि॑त्राम् । \newline
73. अ॒रु॒ह॒(ग्म्॒) श॒तारि॑त्राꣳ श॒तारि॑त्रा मरुह मरुहꣳ श॒तारि॑त्राꣳ श॒तस्फ्या(ग्म्॑) श॒तस्फ्या(ग्म्॑) श॒तारि॑त्रा मरुह मरुहꣳ श॒तारि॑त्राꣳ श॒तस्फ्या᳚म् । \newline
74. श॒तारि॑त्राꣳ श॒तस्फ्या(ग्म्॑) श॒तस्फ्या(ग्म्॑) श॒तारि॑त्राꣳ श॒तारि॑त्राꣳ श॒तस्फ्या᳚म् । \newline
75. श॒तारि॑त्रा॒मिति॑ श॒त - अ॒रि॒त्रा॒म् । \newline
76. श॒तस्फ्या॒मिति॑ श॒त - स्फ्या॒म् । \newline
77. अच्छि॑द्राम् पारयि॒ष्णुम् पा॑रयि॒ष्णु मच्छि॑द्रा॒ मच्छि॑द्राम् पारयि॒ष्णुम् । \newline
78. पा॒र॒यि॒ष्णुमिति॑ पारयि॒ष्णुम् । \newline
\pagebreak


\end{document}