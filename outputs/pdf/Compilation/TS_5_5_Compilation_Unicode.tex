\documentclass[17pt]{extarticle}
\usepackage{babel}
\usepackage{fontspec}
\usepackage{polyglossia}
\usepackage{extsizes}

\usepackage{color}   %May be necessary if you want to color links
\usepackage{hyperref}
\hypersetup{
    colorlinks=true, %set true if you want colored links
    linktoc=all,     %set to all if you want both sections and subsections linked
    linkcolor=black,  %choose some color if you want links to stand out
}

\setmainlanguage{sanskrit}
\setotherlanguages{english} %% or other languages
\setlength{\parindent}{0pt}
\pagestyle{myheadings}
\newfontfamily\devanagarifont[Script=Devanagari]{AdishilaVedic}
\renewcommand{\theHsection}{\thepart.section.\thesection}

\newcommand{\VAR}[1]{}
\newcommand{\BLOCK}[1]{}




\begin{document}
\begin{titlepage}
    \begin{center}
 
\begin{sanskrit}
    { \Large
    कृष्ण यजुर्वेदीय तैत्तिरीय संहिता,पद,जटा,घन पाठः 
    }
    \\
    \vspace{2.5cm}
    \mbox{ \Large
    5.5      पञ्चमकाण्डे पञ्चमः प्रश्नः - वायव्यपश्वाद्यानं निरूपणं   }
\end{sanskrit}
\end{center}

\end{titlepage}
\tableofcontents
\phantomsection
\pagebreak

\markright{ TS 5.5.1.1  \hfill https://www.vedavms.in \hfill}

\section{ TS 5.5.1.1 }

\textbf{TS 5.5.1.1 } \newline
\textbf{Samhita Paata} \newline

यदेके॑न सꣳ स्था॒पय॑ति य॒ज्ञ्स्य॒ संत॑त्या॒ अवि॑च्छेदायै॒न्द्राः प॒शवो॒ ये मु॑ष्क॒रा यदै॒न्द्राः सन्तो॒ऽग्निभ्य॑ आल॒भ्यन्ते॑ दे॒वता᳚भ्यः स॒मदं॑ दधात्याग्ने॒यीस्त्रि॒ष्टुभो॑ याज्यानुवा॒क्याः᳚ कुर्या॒द्-यदा᳚ग्ने॒यीस्तेना᳚ ऽऽ*ग्ने॒या यत् त्रि॒ष्टुभ॒स्तेनै॒न्द्राः समृ॑द्ध्यै॒ न दे॒वता᳚भ्यः स॒मदं॑ दधाति वा॒यवे॑ नि॒युत्व॑ते तूप॒रमा ल॑भते॒ तेजो॒ऽग्नेर्वा॒युस्तेज॑स ए॒ष आ ल॑भ्यते॒ तस्मा᳚द्-य॒द्रिय॑ङ् वा॒यु - [  ] \newline

\textbf{Pada Paata} \newline

यत् । एके॑न । सꣳ॒॒स्था॒पय॒तीति॑ सं - स्था॒पय॑ति । य॒ज्ञ्स्य॑ । सन्त॑त्या॒ इति॒ सं - त॒त्यै॒ । अवि॑च्छेदा॒येत्यवि॑ - छे॒दा॒य॒ । ऐ॒न्द्राः । प॒शवः॑ । ये । मु॒ष्क॒राः । यत् । ऐ॒न्द्राः । सन्तः॑ । अ॒ग्निभ्य॒ इत्य॒ग्नि - भ्यः॒ । आ॒ल॒भ्यन्त॒ इत्या᳚ - ल॒भ्यन्ते᳚ । दे॒वता᳚भ्यः । स॒मद॒मिति॑ स - मद᳚म् । द॒धा॒ति॒ । आ॒ग्ने॒यीः । त्रि॒ष्टुभः॑ । या॒ज्या॒नु॒वा॒क्या॑ इति॑ याज्या - अ॒नु॒वा॒क्याः᳚ । कु॒र्या॒त् । यत् । आ॒ग्ने॒यीः । तेन॑ । आ॒ग्ने॒याः । यत् । त्रि॒ष्टुभः॑ । तेन॑ । ऐ॒न्द्राः । समृ॑द्ध्या॒ इति॒ सं - ऋ॒द्ध्यै॒ । न । दे॒वता᳚भ्यः । स॒मद॒मिति॑ स - मद᳚म् । द॒धा॒ति॒ । वा॒यवे᳚ । नि॒युत्व॑त॒ इति॑ नि-युत्व॑ते । तू॒प॒रम् । एति॑ । ल॒भ॒ते॒ । तेजः॑ । अ॒ग्नेः । वा॒युः । तेज॑से । ए॒षः । एति॑ । ल॒भ्य॒ते॒ । तस्मा᳚त् । य॒द्रियङ्॑ । वा॒युः ।  \newline




\markright{ TS 5.5.1.2  \hfill https://www.vedavms.in \hfill}

\section{ TS 5.5.1.2 }

\textbf{TS 5.5.1.2 } \newline
\textbf{Samhita Paata} \newline

-र्वाति॑ त॒द्रिय॑ङ्ङ॒-ग्निर्द॑हति॒ स्वमे॒व तत् तेजोऽन्वे॑ति॒ यन्न नि॒युत्व॑ते॒ स्यादुन्मा᳚द्ये॒द्-यज॑मानो नि॒युत्व॑ते भवति॒ यज॑मान॒स्याऽ*नु॑न्मादाय वायु॒मती᳚ श्वे॒तव॑ती याज्यानुवा॒क्ये॑ भवतः सतेज॒स्त्वाय॑ हिरण्यग॒र्भः सम॑वर्त॒ताग्र॒ इत्या॑घा॒रमा घा॑रयति प्र॒जाप॑ति॒र्वै हि॑रण्यग॒र्भः प्र॒जाप॑तेरनुरूप॒त्वाय॒ सर्वा॑णि॒ वा ए॒ष रू॒पाणि॑ पशू॒नां प्रत्या ल॑भ्यते॒ यच्छ्म॑श्रु॒णस्तत् - [  ] \newline

\textbf{Pada Paata} \newline

वाति॑ । त॒द्रियङ्॑ । अ॒ग्निः । द॒ह॒ति॒ । स्वम् । ए॒व । तत् । तेजः॑ । अन्विति॑ । ए॒ति॒ । यत् । न । नि॒युत्व॑त॒ इति॑ नि - युत्व॑ते । स्यात् । उदिति॑ । मा॒द्ये॒त् । यज॑मानः । नि॒युत्व॑त॒ इति॑ नि - युत्व॑ते । भ॒व॒ति॒ । यज॑मानस्य । अनु॑न्मादा॒येत्यनु॑त् - मा॒दा॒य॒ । वा॒यु॒मती॒ इति॑ वायु - मती᳚ । श्वे॒तव॑ती॒ इति॑ श्वे॒त - व॒ती॒ । या॒ज्या॒नु॒वा॒क्ये॑ इति॑ याज्या - अ॒नु॒वा॒क्ये᳚ । भ॒व॒तः॒ । स॒ते॒ज॒स्त्वायेति॑ सतेजः - त्वाय॑ । हि॒र॒ण्य॒ग॒र्भ इति॑ हिरण्य - ग॒र्भः । समिति॑ । अ॒व॒र्त॒त॒ । अग्रे᳚ । इति॑ । आ॒घा॒रमित्या᳚ - घा॒रम् । एति॑ । घा॒र॒य॒ति॒ । प्र॒जाप॑ति॒रिति॑ प्र॒जा-प॒तिः॒ । वै । हि॒र॒ण्य॒ग॒र्भ इति॑ हिरण्य - ग॒र्भः । प्र॒जाप॑ते॒रिति॑ प्र॒जा - प॒तेः॒ । अ॒नु॒रू॒प॒त्वायेत्य॑नुरूप - त्वाय॑ । सर्वा॑णि । वै । ए॒षः । रू॒पाणि॑ । प॒शू॒नाम् । प्रति॑ । एति॑ । ल॒भ्य॒ते॒ । यत् । श्म॒श्रु॒णः । तत् ।  \newline




\markright{ TS 5.5.1.3  \hfill https://www.vedavms.in \hfill}

\section{ TS 5.5.1.3 }

\textbf{TS 5.5.1.3 } \newline
\textbf{Samhita Paata} \newline

पुरु॑षाणाꣳ रू॒पं ॅयत् तू॑प॒रस्तदश्वा॑नां॒ ॅयद॒न्यतो॑द॒न् तद्-गवां॒ ॅयदव्या॑ इव श॒फास्तदवी॑नां॒ ॅयद॒जस्तद॒जानां᳚ ॅवा॒युर्वै प॑शू॒नां प्रि॒यं धाम॒ यद्-वा॑य॒व्यो॑ भव॑त्ये॒तमे॒वैन॑म॒भि स॑जांना॒नाः प॒शव॒ उप॑ तिष्ठन्ते वाय॒व्यः॑ का॒र्या(3)ः प्रा॑जाप॒त्या(3) इत्या॑हु॒-र्यद्-वा॑य॒व्यं॑ कु॒र्यात् प्र॒जाप॑तेरिया॒द्यत् प्रा॑जाप॒त्यं कु॒र्याद्-वा॒यो - [  ] \newline

\textbf{Pada Paata} \newline

पुरु॑षाणाम् । रू॒पम् । यत् । तू॒प॒रः । तत् । अश्वा॑नाम् । यत् । अ॒न्यतो॑द॒न्नित्य॒न्यतः॑ - द॒न्न् । तत् । गवा᳚म् । यत् । अव्याः᳚ । इ॒व॒ । श॒फाः । तत् । अवी॑नाम् । यत् । अ॒जः । तत् । अ॒जाना᳚म् । वा॒युः । वै । प॒शू॒नाम् । प्रि॒यम् । धाम॑ । यत् । वा॒य॒व्यः॑ । भव॑ति । ए॒तम् । ए॒व । ए॒न॒म् । अ॒भीति॑ । स॒ञ्जा॒ना॒ना इति॑ सं - जा॒ना॒नाः । प॒शवः॑ । उपेति॑ । ति॒ष्ठ॒न्ते॒ । वा॒य॒व्यः॑ । का॒र्या(3)ः । प्रा॒जा॒प॒त्या(3) इति॑ प्राजा - प॒त्या(3)ः । इति॑ । आ॒हुः॒ । यत् । वा॒य॒व्य᳚म् । कु॒र्यात् । प्र॒जाप॑ते॒रिति॑ प्र॒जा - प॒तेः॒ । इ॒या॒त् । यत् । प्रा॒जा॒प॒त्यमिति॑ प्राजा - प॒त्यम् । कु॒र्यात् । वा॒योः ।  \newline




\markright{ TS 5.5.1.4  \hfill https://www.vedavms.in \hfill}

\section{ TS 5.5.1.4 }

\textbf{TS 5.5.1.4 } \newline
\textbf{Samhita Paata} \newline

-रि॑या॒द्-यद्-वा॑य॒व्यः॑ प॒शुर्भव॑ति॒ तेन॑ वा॒योर्नैति॒ यत् प्रा॑जाप॒त्यः पु॑रो॒डाशो॒ भव॑ति॒ तेन॑ प्रा॒जाप॑ते॒र्नैति॒ यद् द्वाद॑शकपाल॒स्तेन॑ वैश्वान॒रान्नैत्या᳚ग्ना वैष्ण॒वमेका॑दश-कपालं॒ निर्व॑पति दीक्षि॒ष्यमा॑णो॒ऽग्निः सर्वा॑ दे॒वता॒ विष्णु॑र्य॒ज्ञो दे॒वता᳚श्चै॒व य॒ज्ञ्ं चाऽऽ* र॑भते॒ऽग्निर॑व॒मो दे॒वता॑नां॒ ॅविष्णुः॑ पर॒मो यदा᳚ग्ना-वैष्ण॒व-मेका॑दशकपालं नि॒र्वपति दे॒वता॑ - [  ] \newline

\textbf{Pada Paata} \newline

इ॒या॒त् । यत् । वा॒य॒व्यः॑ । प॒शुः । भव॑ति । तेन॑ । वा॒योः । न । ए॒ति॒ । यत् । प्रा॒जा॒प॒त्य इति॑ प्राजा - प॒त्यः । पु॒रो॒डाशः॑ । भव॑ति । तेन॑ । प्र॒जाप॑ते॒रिति॑ प्र॒जा - प॒तेः॒ । न । ए॒ति॒ । यत् । द्वाद॑शकपाल॒ इति॒ द्वाद॑श - क॒पा॒लः॒ । तेन॑ । वै॒श्वा॒न॒रात् । न । ए॒ति॒ । आ॒ग्ना॒वै॒ष्ण॒वमित्या᳚ग्ना - वै॒ष्ण॒वम् । एका॑दशकपाल॒मित्येका॑दश- क॒पा॒ल॒म् । निरिति॑ । व॒प॒ति॒ । दी॒क्षि॒ष्यमा॑णः । अ॒ग्निः । सर्वाः᳚ । दे॒वताः᳚ । विष्णुः॑ । य॒ज्ञ्ः । दे॒वताः᳚ । च॒ । ए॒व । य॒ज्ञ्म् । च॒ । एति॑ । र॒भ॒ते॒ । अ॒ग्निः । अ॒व॒मः । दे॒वता॑नाम् । विष्णुः॑ । प॒र॒मः । यत् । आ॒ग्ना॒वै॒ष्ण॒वमित्या᳚ग्ना - वै॒ष्ण॒वम् । एका॑दशकपाल॒मित्येका॑दश-क॒पा॒ल॒म् । नि॒र्वप॒तीति॑ निः - वप॑ति । दे॒वताः᳚ ।  \newline




\markright{ TS 5.5.1.5  \hfill https://www.vedavms.in \hfill}

\section{ TS 5.5.1.5 }

\textbf{TS 5.5.1.5 } \newline
\textbf{Samhita Paata} \newline

ए॒वोभ॒यतः॑ परि॒गृह्य॒ यज॑मा॒नोऽव॑ रुन्धे पुरो॒डाशे॑न॒ वै दे॒वा अ॒मुष्मि॑न् ॅलो॒क आ᳚र्द्ध्नुवन् च॒रुणा॒ऽस्मिन्. यः का॒मये॑ता॒ऽमुष्मि॑न् ॅलो॒क ऋ॑द्ध्नुया॒मिति॒ स पु॑रो॒डाशं॑ कुर्वीता॒ऽमुष्मि॑न्ने॒व लो॒क ऋ॑द्ध्नोति॒ यद॒ष्टाक॑पाल॒स्तेना᳚ऽऽ*ग्ने॒यो यत् त्रि॑कपा॒लस्तेन॑ वैष्ण॒वः समृ॑द्ध्यै॒ यः का॒मये॑ता॒स्मिन् ॅलो॒क ऋ॑द्ध्नुया॒मिति॒ स च॒रुं कु॑र्वीता॒ग्नेर्घृ॒तं ॅविष्णो᳚स्तण्डु॒ला-स्तस्मा᳚ - [  ] \newline

\textbf{Pada Paata} \newline

ए॒व । उ॒भ॒यतः॑ । प॒रि॒गृह्येति॑ परि-गृह्य॑ । यज॑मानः । अवेति॑ । रु॒न्धे॒ । पु॒रो॒डाशे॑न । वै । दे॒वाः । अ॒मुष्मिन्न्॑ । लो॒के । आ॒द्र्ध्नु॒व॒न्न् । च॒रुणा᳚ । अ॒स्मिन्न् । यः । का॒मये॑त । अ॒मुष्मिन्न्॑ । लो॒के । ऋ॒द्ध्नु॒या॒म् । इति॑ । सः । पु॒रो॒डाश᳚म् । कु॒र्वी॒त॒ । अ॒मुष्मिन्न्॑ । ए॒व । लो॒के । ऋ॒द्ध्नो॒ति॒ । यत् । अ॒ष्टाक॑पाल॒ इत्य॒ष्टा - क॒पा॒लः॒ । तेन॑ । आ॒ग्ने॒यः । यत् । त्रि॒क॒पा॒ल इति॑ त्रि - क॒पा॒लः । तेन॑ । वै॒ष्ण॒वः । समृ॑द्ध्या॒ इति॒ सं - ऋ॒द्ध्यै॒ । यः । का॒मये॑त । अ॒स्मिन्न् । लो॒के । ऋ॒द्ध्नु॒या॒म् । इति॑ । सः । च॒रुम् । कु॒र्वी॒त॒ । अ॒ग्नेः । घृ॒तम् । विष्णोः᳚ । त॒ण्डु॒लाः । तस्मा᳚त् ।  \newline




\markright{ TS 5.5.1.6  \hfill https://www.vedavms.in \hfill}

\section{ TS 5.5.1.6 }

\textbf{TS 5.5.1.6 } \newline
\textbf{Samhita Paata} \newline

-च्च॒रुः का॒र्यो᳚ऽस्मिन्ने॒व लो॒क ऋ॑द्ध्नोत्यादि॒त्यो भ॑वती॒ यं ॅवा अदि॑तिर॒स्यामे॒व प्रति॑ तिष्ठ॒त्यथो॑ अ॒स्यामे॒वाधि॑ य॒ज्ञ्ं त॑नुते॒ यो वै सं॑ॅवथ्स॒रमुख्य॒-मभृ॑त्वा॒ऽग्निं चि॑नु॒ते यथा॑ सा॒मि गर्भो॑ऽव॒पद्य॑ते ता॒दृगे॒व तदार्ति॒मार्च्छे᳚द्-वैश्वान॒रं द्वाद॑शकपालं पु॒रस्ता॒न्निर्व॑पेथ् संॅवथ्स॒रो वा अ॒ग्निर्-वै᳚श्वान॒रो यथा॑ संवथ्स॒रमा॒प्त्वा - [  ] \newline

\textbf{Pada Paata} \newline

च॒रुः । का॒र्यः॑ । अ॒स्मिन्न् । ए॒व । लो॒के । ऋ॒द्ध्नो॒ति॒ । आ॒दि॒त्यः । भ॒व॒ति॒ । इ॒यम् । वै । अदि॑तिः । अ॒स्याम् । ए॒व । प्रतीति॑ । ति॒ष्ठ॒ति॒ । अथो॒ इति॑ । अ॒स्याम् । ए॒व । अधीति॑ । य॒ज्ञ्म् । त॒नु॒ते॒ । यः । वै । सं॒ॅव॒थ्स॒रमिति॑ सं - व॒थ्स॒रम् । उख्य᳚म् । अभृ॑त्वा । अ॒ग्निम् । चि॒नु॒ते । यथा᳚ । सा॒मि । गर्भः॑ । अ॒व॒पद्य॑त॒ इत्य॑व - पद्य॑ते । ता॒दृक् । ए॒व । तत् । आर्ति᳚म् । एति॑ । ऋ॒च्छे॒त् । वै॒श्वा॒न॒रम् । द्वाद॑शकपाल॒मिति॒ द्वाद॑श - क॒पा॒ल॒म् । पु॒रस्ता᳚त् । निरिति॑ । व॒पे॒त् । सं॒ॅव॒थ्स॒र इति॑ सं - व॒थ्स॒रः । वै । अ॒ग्निः । वै॒श्वा॒न॒रः । यथा᳚ । सं॒ॅव॒थ्स॒रमिति॑ सं - व॒थ्स॒रम् । आ॒प्त्वा ।  \newline




\markright{ TS 5.5.1.7  \hfill https://www.vedavms.in \hfill}

\section{ TS 5.5.1.7 }

\textbf{TS 5.5.1.7 } \newline
\textbf{Samhita Paata} \newline

का॒ल आग॑ते वि॒जाय॑त ए॒वमे॒व सं॑ॅवथ्स॒रमा॒प्त्वा का॒ल आग॑ते॒ऽग्निं चि॑नुते॒ नाऽऽ*र्ति॒मार्च्छ॑त्ये॒षा वा अ॒ग्नेः प्रि॒या त॒नूर्यद्-वै᳚श्वान॒रः प्रि॒यामे॒वास्य॑ त॒नुव॒मव॑ रुन्धे॒ त्रीण्ये॒तानि॑ ह॒वीꣳषि॑ भवन्ति॒ त्रय॑ इ॒मे लो॒का ए॒षां ॅलो॒कानाꣳ॒॒ रोहा॑य ॥ \newline

\textbf{Pada Paata} \newline

का॒ले । आग॑त॒ इत्या - ग॒ते॒ । वि॒जाय॑त॒ इति॑ वि - जाय॑ते । ए॒वम् । ए॒व । सं॒ॅव॒थ्स॒रमिति॑ सं - व॒थ्स॒रम् । आ॒प्त्वा । का॒ले । आग॑त॒ इत्या - ग॒ते॒ । अ॒ग्निम् । चि॒नु॒ते॒ । न । आर्ति᳚म् । एति॑ । ऋ॒च्छ॒ति॒ । ए॒षा । वै । अ॒ग्नेः । प्रि॒या । त॒नूः । यत् । वै॒श्वा॒न॒रः । प्रि॒याम् । ए॒व । अ॒स्य॒ । त॒नुव᳚म् । अवेति॑ । रु॒न्धे॒ । त्रीणि॑ । ए॒तानि॑ । ह॒वीꣳ षि॑ । भ॒व॒न्ति॒ । त्रयः॑ । इ॒मे । लो॒काः । ए॒षाम् । लो॒काना᳚म् । रोहा॑य ॥  \newline




\markright{ TS 5.5.2.1  \hfill https://www.vedavms.in \hfill}

\section{ TS 5.5.2.1 }

\textbf{TS 5.5.2.1 } \newline
\textbf{Samhita Paata} \newline

प्र॒जाप॑तिः प्र॒जाः सृ॒ष्ट्वा प्रे॒णाऽनु॒ प्रावि॑श॒त् ताभ्यः॒ पुनः॒ संभ॑वितुं॒ नाश॑क्नो॒थ् सो᳚ऽब्रवीदृ॒द्ध्नव॒दिथ् स यो मे॒तः पुनः॑ संचि॒नव॒दिति॒ तं दे॒वाः सम॑चिन्व॒न् ततो॒ वै त आ᳚र्द्ध्नुव॒न्॒ यथ् स॒मचि॑न्व॒न् तच्चित्य॑स्य चित्य॒त्वं ॅय ए॒वं ॅवि॒द्वान॒ग्निं चि॑नु॒त ऋ॒द्ध्नोत्ये॒व कस्मै॒ कम॒ग्निश्ची॑यत॒ इत्या॑हुरग्नि॒वा - [  ] \newline

\textbf{Pada Paata} \newline

प्र॒जाप॑ति॒रिति॑ प्र॒जा - प॒तिः॒ । प्र॒जा इति॑ प्र - जाः । सृ॒ष्ट्वा । प्रे॒णा । अनु॑ । प्रेति॑ । अ॒वि॒श॒त् । ताभ्यः॑ । पुनः॑ । संभ॑वितु॒मिति॒ सं - भ॒वि॒तु॒म् । न । अ॒श॒क्नो॒त् । सः । अ॒ब्र॒वी॒त् । ऋ॒द्ध्नव॑त् । इत् । सः । यः । मा॒ । इ॒तः । पुनः॑ । स॒चिं॒नव॒दिति॑ सं - चि॒नव॑त् । इति॑ । तम् । दे॒वाः । समिति॑ । अ॒चि॒न्व॒न्न् । ततः॑ । वै । ते । आ॒द्‌र्ध्नु॒व॒न्न् । यत् । स॒मचि॑न्व॒न्निति॑ सं - अचि॑न्वन्न् । तत् । चित्य॑स्य । चि॒त्य॒त्वमिति॑ चित्य - त्वम् । यः । ए॒वम् । वि॒द्वान् । अ॒ग्निम् । चि॒नु॒ते । ऋ॒द्ध्नोति॑ । ए॒व । कस्मै᳚ । कम् । अ॒ग्निः । ची॒य॒ते॒ । इति॑ । आ॒हुः॒ । अ॒ग्नि॒वानित्य॑ग्नि - वान् ।  \newline




\markright{ TS 5.5.2.2  \hfill https://www.vedavms.in \hfill}

\section{ TS 5.5.2.2 }

\textbf{TS 5.5.2.2 } \newline
\textbf{Samhita Paata} \newline

-न॑सा॒नीति॒ वा अ॒ग्निश्ची॑यते ऽग्नि॒वाने॒व भ॑वति॒ कस्मै॒ कम॒ग्निश्ची॑यत॒ इत्या॑हुर्दे॒वा मा॑ वेद॒न्निति॒ वा अ॒ग्निश्ची॑यते वि॒दुरे॑नं दे॒वाः कस्मै॒ कम॒ग्निश्ची॑यत॒ इत्या॑हुर्गृ॒ह्य॑सा॒नीति॒ वा अ॒ग्निश्ची॑यते गृ॒ह्ये॑व भ॑वति॒ कस्मै॒ कम॒ग्निश्ची॑यत॒ इत्या॑हुः पशु॒मान॑सा॒नीति॒ वा अ॒ग्नि - [  ] \newline

\textbf{Pada Paata} \newline

अ॒सा॒नि॒ । इति॑ । वै । अ॒ग्निः । ची॒य॒ते॒ । अ॒ग्नि॒वानित्य॑ग्नि - वान् । ए॒व । भ॒व॒ति॒ । कस्मै᳚ । कम् । अ॒ग्निः । ची॒य॒ते॒ । इति॑ । आ॒हुः॒ । दे॒वाः । मा॒ । वे॒द॒न्न् । इति॑ । वै । अ॒ग्निः । ची॒य॒ते॒ । वि॒दुः । ए॒न॒म् । दे॒वाः । कस्मै᳚ । कम् । अ॒ग्निः । ची॒य॒ते॒ । इति॑ । आ॒हुः॒ । गृ॒ही । अ॒सा॒नि॒ । इति॑ । वै । अ॒ग्निः । ची॒य॒ते॒ । गृ॒ही । ए॒व । भ॒व॒ति॒ । कस्मै᳚ । कम् । अ॒ग्निः । ची॒य॒ते॒ । इति॑ । आ॒हुः॒ । प॒शु॒मानिति॑ पशु - मान् । अ॒सा॒नि॒ । इति॑ । वै । अ॒ग्निः ।  \newline




\markright{ TS 5.5.2.3  \hfill https://www.vedavms.in \hfill}

\section{ TS 5.5.2.3 }

\textbf{TS 5.5.2.3 } \newline
\textbf{Samhita Paata} \newline

-श्ची॑यते पशु॒माने॒व भ॑वति॒ कस्मै॒ कम॒ग्निश्ची॑यत॒ इत्या॑हुः स॒प्त मा॒ पुरु॑षा॒ उप॑ जीवा॒निति॒ वा अ॒ग्निश्ची॑यते॒ त्रयः॒ प्राञ्च॒स्त्रयः॑ प्र॒त्यञ्च॑ आ॒त्मा स॑प्त॒म ए॒ताव॑न्त ए॒वैन॑म॒मुष्मि॑न् ॅलो॒क उप॑ जीवन्ति प्र॒जाप॑तिर॒ग्निम॑चिकीषत॒ तं पृ॑थि॒व्य॑ब्रवी॒न्न मय्य॒ग्निं चे᳚ष्य॒सेऽति॑ मा धक्ष्यति॒ सा त्वा॑ऽतिद॒ह्यमा॑ना॒ वि ध॑विष्ये॒ - [  ] \newline

\textbf{Pada Paata} \newline

ची॒य॒ते॒ । प॒शु॒मानिति॑ पशु - मान् । ए॒व । भ॒व॒ति॒ । कस्मै᳚ । कम् । अ॒ग्निः । ची॒य॒ते॒ । इति॑ । आ॒हुः॒ । स॒प्त । मा॒ । पुरु॑षाः । उपेति॑ । जी॒वा॒न् । इति॑ । वै । अ॒ग्निः । ची॒य॒ते॒ । त्रयः॑ । प्राञ्चः॑ । त्रयः॑ । प्र॒त्यञ्चः॑ । आ॒त्मा । स॒प्त॒मः । ए॒ताव॑न्तः । ए॒व । ए॒न॒म् । अ॒मुष्मिन्न्॑ । लो॒के । उपेति॑ । जी॒व॒न्ति॒ । प्र॒जाप॑ति॒रिति॑ प्र॒जा - प॒तिः॒ । अ॒ग्निम् । अ॒चि॒की॒ष॒त॒ । तम् । पृ॒थि॒वी । अ॒ब्र॒वी॒त् । न । मयि॑ । अ॒ग्निम् । चे॒ष्य॒से॒ । अतीति॑ । मा॒ । ध॒क्ष्य॒ति॒ । सा । त्वा॒ । अ॒ति॒द॒ह्यमा॒नेत्य॑ति - द॒ह्यमा॑ना । वीति॑ । ध॒वि॒ष्ये॒ ।  \newline




\markright{ TS 5.5.2.4  \hfill https://www.vedavms.in \hfill}

\section{ TS 5.5.2.4 }

\textbf{TS 5.5.2.4 } \newline
\textbf{Samhita Paata} \newline

स पापी॑यान् भविष्य॒सीति॒ सो᳚ऽब्रवी॒त् तथा॒ वा अ॒हं क॑रिष्यामि॒ यथा᳚ त्वा॒ नाति॑ध॒क्ष्यतीति॒ स इ॒माम॒भ्य॑मृशत् प्र॒जाप॑तिस्त्वा सादयतु॒ तया॑ दे॒वत॑याऽङ्गिर॒स्वद् ध्रु॒वा सी॒देती॒मामे॒वेष्ट॑कां कृ॒त्वोपा॑-ध॒त्ता-न॑तिदाहाय॒ यत् प्रत्य॒ग्निं चि॑न्वी॒त तद॒भि मृ॑शेत् प्र॒जाप॑तिस्त्वा सादयतु॒ तया॑ दे॒वत॑याऽङ्गिर॒स्वद् ध्रु॒वा सी॒दे - [  ] \newline

\textbf{Pada Paata} \newline

सः । पापी॑यान् । भ॒वि॒ष्य॒सि॒ । इति॑ । सः । अ॒ब्र॒वी॒त् । तथा᳚ । वै । अ॒हम् । क॒रि॒ष्या॒मि॒ । यथा᳚ । त्वा॒ । न । अ॒ति॒ध॒क्ष्यतीत्य॑ति-ध॒क्ष्यति॑ । इति॑ । सः । इ॒माम् । अ॒भीति॑ । अ॒मृ॒श॒त् । प्र॒जाप॑ति॒रिति॑ प्र॒जा-प॒तिः॒ । त्वा॒ । सा॒द॒य॒त॒ । तया᳚ । दे॒वत॑या । अ॒ङ्गि॒र॒स्वत् । ध्रु॒वा । सी॒द॒ । इति॑ । इ॒माम् । ए॒व । इष्ट॑काम् । कृ॒त्वा । उपेति॑ । अ॒ध॒त्त॒ । अन॑तिदाहा॒येत्यन॑ति - दा॒हा॒य॒ । यत् । प्रतीति॑ । अ॒ग्निम् । चि॒न्वी॒त । तत् । अ॒भीति॑ । मृ॒शे॒त् । प्र॒जाप॑ति॒रिति॑ प्र॒जा-प॒तिः॒ । त्वा॒ । सा॒द॒य॒तु॒ । तया᳚ । दे॒वत॑या । अ॒ङ्गि॒र॒स्वत् । ध्रु॒वा । सी॒द॒ ।  \newline




\markright{ TS 5.5.2.5  \hfill https://www.vedavms.in \hfill}

\section{ TS 5.5.2.5 }

\textbf{TS 5.5.2.5 } \newline
\textbf{Samhita Paata} \newline

-ती॒मामे॒वेष्ट॑कां कृ॒त्वोप॑ ध॒त्तेऽन॑तिदाहाय प्र॒जाप॑तिरकामयत॒ प्रजा॑ये॒येति॒ स ए॒तमुख्य॑मपश्य॒त् तꣳ सं॑ॅवथ्स॒रम॑बिभ॒स्ततो॑ वै स प्राजा॑यत॒ तस्मा᳚थ् संॅवथ्स॒रं भा॒र्यः॑ प्रैव जा॑यते॒ तं ॅवस॑वोऽब्रुव॒न् प्र त्वम॑जनिष्ठा व॒यं प्रजा॑यामहा॒ इति॒ तं ॅवसु॑भ्यः॒ प्राय॑च्छ॒त् तं त्रीण्यहा᳚न्यबिभरु॒स्तेन॒ - [  ] \newline

\textbf{Pada Paata} \newline

इति॑ । इ॒माम् । ए॒व । इष्ट॑काम् । कृ॒त्वा । उपेति॑ । ध॒त्ते॒ । अन॑तिदाहा॒येत्यन॑ति - दा॒हा॒य॒ । प्र॒जाप॑ति॒रिति॑ प्र॒जा - प॒तिः॒ । अ॒का॒म॒य॒त॒ । प्रेति॑ । जा॒ये॒य॒ । इति॑ । सः । ए॒तम् । उख्य᳚म् । अ॒प॒श्य॒त् । तम् । सं॒ॅव॒थ्स॒रमिति॑ सं - व॒थ्स॒रम् । अ॒बि॒भः॒ । ततः॑ । वै । सः । प्रेति॑ । अ॒जा॒य॒त॒ । तस्मा᳚त् । सं॒ॅव॒थ्स॒रमिति॑ सं-व॒थ्स॒रम् । भा॒र्यः॑ । प्रेति॑ । ए॒व । जा॒य॒ते॒ । तम् । वस॑वः । अ॒ब्रु॒व॒न्न् । प्रेति॑ । त्वम् । अ॒ज॒नि॒ष्ठाः॒ । व॒यम् । प्रेति॑ । जा॒या॒म॒है॒ । इति॑ । तम् । वसु॑भ्य॒ इति॒ वसु॑ - भ्यः॒ । प्रेति॑ । अ॒य॒च्छ॒त् । तम् । त्रीणि॑ । अहा॑नि । अ॒बि॒भ॒रुः॒ । तेन॑ ।  \newline




\markright{ TS 5.5.2.6  \hfill https://www.vedavms.in \hfill}

\section{ TS 5.5.2.6 }

\textbf{TS 5.5.2.6 } \newline
\textbf{Samhita Paata} \newline

त्रीणि॑ च श॒तान्यसृ॑जन्त॒ त्रय॑स्त्रिꣳशतं च॒ तस्मा᳚त् त्र्य॒हं भा॒र्यः॑ प्रैव जा॑यते॒ तान् रु॒द्रा अ॑ब्रुव॒न् प्र यू॒यम॑जनिढ्वं ॅव॒यं प्रजा॑यामहा॒ इति॒ तꣳ रु॒द्रेभ्यः॒ प्राय॑च्छ॒न् तꣳ षडहा᳚न्यबिभरु॒स्तेन॒ त्रीणि॑ च श॒तान्यसृ॑जन्त॒ त्रय॑स्त्रिꣳशतं च॒ तस्मा᳚थ् षड॒हं भा॒र्यः॑ प्रैव जा॑यते॒ ताना॑दि॒त्या अ॑ब्रुव॒न् प्र यू॒यम॑जनिढ्वं ॅव॒यं - [  ] \newline

\textbf{Pada Paata} \newline

त्रीणि॑ । च॒ । श॒ताति॑ । असृ॑जन्त । त्रय॑स्त्रिꣳशत॒मिति॒ त्रयः॑- त्रिꣳ॒॒श॒त॒म् । च॒ । तस्मा᳚त् । त्र्य॒हमिति॑ त्रि - अ॒हम् । भा॒र्यः॑ । प्रेति॑ । ए॒व । जा॒य॒ते॒ । तान् । रु॒द्राः । अ॒ब्रु॒व॒न्न् । प्रेति॑ । यू॒यम् । अ॒ज॒नि॒ढ्व॒म् । व॒यम् । प्रेति॑ । जा॒या॒म॒है॒ । इति॑ । तम् । रु॒द्रेभ्यः॑ । प्रेति॑ । अ॒य॒च्छ॒न्न् । तम् । षट् । अहा॑नि । अ॒बि॒भ॒रुः॒ । तेन॑ । त्रीणि॑ । च॒ । श॒तानि॑ । असृ॑जन्त । त्रय॑स्त्रिꣳशत॒मिति॒ त्रयः॑ - त्रिꣳ॒॒श॒त॒म् । च॒ । तस्मा᳚त् । ष॒ड॒हमिति॑ षट् - अ॒हम् । भा॒र्यः॑ । प्रेति॑ । ए॒व । जा॒य॒ते॒ । तान् । आ॒दि॒त्याः । अ॒ब्रु॒व॒न्न् । प्रेति॑ । यू॒यम् । अ॒ज॒नि॒ढ्व॒म् । व॒यम् ।  \newline




\markright{ TS 5.5.2.7  \hfill https://www.vedavms.in \hfill}

\section{ TS 5.5.2.7 }

\textbf{TS 5.5.2.7 } \newline
\textbf{Samhita Paata} \newline

प्र जा॑यामहा॒ इति॒ तमा॑दि॒त्येभ्यः॒ प्राय॑च्छ॒न् तं द्वाद॒शाहा᳚न्यबिभरु॒स्तेन॒ त्रीणि॑ च श॒तान्यसृ॑जन्त॒ त्रय॑स्त्रिꣳशतं च॒ तस्मा᳚द् द्वादशा॒हं भा॒र्यः॑ प्रैव जा॑यते॒ तेन॒ वै ते स॒हस्र॑मसृजन्तो॒खाꣳ स॑हस्रत॒मीं ॅय ए॒वमुख्यꣳ॑ साह॒स्रं ॅवेद॒ प्र स॒हस्रं॑ प॒शूना᳚प्नोति ॥ \newline

\textbf{Pada Paata} \newline

प्रेति॑ । जा॒या॒म॒है॒ । इति॑ । तम् । आ॒दि॒त्येभ्यः॑ । प्रेति॑ । अ॒य॒च्छ॒न्न् । तम् । द्वाद॑श । अहा॑नि । अ॒बि॒भ॒रुः॒ । तेन॑ । त्रीणि॑ । च॒ । श॒तानि॑ । असृ॑जन्त । त्रय॑स्त्रिꣳशत॒मिति॒ त्रयः॑ - त्रिꣳ॒॒श॒त॒म् । च॒ । तस्मा᳚त् । द्वा॒द॒शा॒हमिति॑ द्वादश - अ॒हम् । भा॒र्यः॑ । प्रेति॑ । ए॒व । जा॒य॒ते॒ । तेन॑ । वै । ते । स॒हस्र᳚म् । अ॒सृ॒ज॒न्त॒ । उ॒खाम् । स॒ह॒स्र॒त॒मीमिति॑ सहस्र - त॒मीम् । यः । ए॒वम् । उख्य᳚म् । सा॒ह॒स्रम् । वेद॑ । प्रेति॑ । स॒हस्र᳚म् । प॒शून् । आ॒प्नो॒ति॒ ॥  \newline




\markright{ TS 5.5.3.1  \hfill https://www.vedavms.in \hfill}

\section{ TS 5.5.3.1 }

\textbf{TS 5.5.3.1 } \newline
\textbf{Samhita Paata} \newline

यजु॑षा॒ वा ए॒षा क्रि॑यते॒ यजु॑षा पच्यते॒ यजु॑षा॒ वि मु॑च्यते॒ यदु॒खा सा वा ए॒षैतर्.हि॑ या॒तया᳚म्नी॒ सा न पुनः॑ प्र॒युज्येत्या॑हु॒रग्ने॑ यु॒क्ष्वा हि ये तव॑ यु॒क्ष्वा हि दे॑व॒हूत॑माꣳ॒॒ इत्यु॒खायां᳚ जुहोति॒ तेनै॒वैनां॒ पुनः॒ प्रयु॑ङ्क्ते॒ तेनाया॑तयाम्नी॒ यो वा अ॒ग्निं ॅयोग॒ आग॑ते यु॒नक्ति॑ यु॒ङ्क्ते यु॑ञ्जा॒नेष्वग्ने॑ - [  ] \newline

\textbf{Pada Paata} \newline

यजु॑षा । वै । ए॒षा । क्रि॒य॒ते॒ । यजु॑षा । प॒च्य॒ते॒ । यजु॑षा । वीति॑ । मु॒च्य॒ते॒ । यत् । उ॒खा । सा । वै । ए॒षा । ए॒तर्.हि॑ । या॒तया॒म्नीति॑ या॒त - या॒म्नी॒ । सा । न । पुनः॑ । प्र॒युज्येति॑ प्र - युज्या᳚ । इति॑ । आ॒हुः॒ । अग्ने᳚ । यु॒क्ष्व । हि । ये । तव॑ । यु॒क्ष्व । हि । दे॒व॒हूत॑मा॒निति॑ देव - हूत॑मान् । इति॑ । उ॒खाया᳚म् । जु॒हो॒ति॒ । तेन॑ । ए॒व । ए॒ना॒म् । पुनः॑ । प्रेति॑ । यु॒ङ्क्ते॒ । तेन॑ । अया॑तया॒म्नीत्यया॑त - या॒म्नी॒ । यः । वै । अ॒ग्निम् । योगे᳚ । आग॑त॒ इत्या - ग॒ते॒ । यु॒नक्ति॑ । यु॒ङ्क्ते । यु॒ञ्जा॒नेषु॑ । अग्ने᳚ ।  \newline




\markright{ TS 5.5.3.2  \hfill https://www.vedavms.in \hfill}

\section{ TS 5.5.3.2 }

\textbf{TS 5.5.3.2 } \newline
\textbf{Samhita Paata} \newline

यु॒क्ष्वा हि ये तव॑ यु॒क्ष्वा हि दे॑व॒हूत॑माꣳ॒॒ इत्या॑है॒ष वा अ॒ग्नेर्योग॒स्तेनै॒वैनं॑ ॅयुनक्ति यु॒ङ्क्ते यु॑ञ्जा॒नेषु॑ ब्रह्मवा॒दिनो॑ वदन्ति॒ न्य॑ङ्ङ॒ग्निश्चे॑त॒व्या(3) उ॑त्ता॒ना(3) इति॒ वय॑सां॒ ॅवा ए॒ष प्र॑ति॒मया॑ चीयते॒ यद॒ग्निर्यन्न्य॑ञ्चं चिनु॒यात् पृ॑ष्टि॒त ए॑न॒माहु॑तय ऋच्छेयु॒र्यदु॑त्ता॒नं न पति॑तुꣳ शक्नुया॒दसु॑वर्ग्योऽस्य स्यात् प्रा॒चीन॑मुत्ता॒नं - [  ] \newline

\textbf{Pada Paata} \newline

यु॒क्ष्व । हि । ये । तव॑ । यु॒क्ष्व । हि । दे॒व॒हूत॑मा॒निति॑ देव - हूत॑मान् । इति॑ । आ॒ह॒ । ए॒षः । वै । अ॒ग्नेः । योगः॑ । तेन॑ । ए॒व । ए॒न॒म् । यु॒न॒क्ति॒ । यु॒ङ्क्ते । यु॒ञ्जा॒नेषु॑ । ब्र॒ह्म॒वा॒दिन॒ इति॑ ब्रह्म - वा॒दिनः॑ । व॒द॒न्ति॒ । न्यङ्॑ । अ॒ग्निः । चे॒त॒व्या(3)ः । उ॒त्ता॒ना(3) इत्यु॑त् - ता॒ना(3)ः । इति॑ । वय॑साम् । वै । ए॒षः । प्र॒ति॒मयेति॑ प्रति - मया᳚ । ची॒य॒ते॒ । यत् । अ॒ग्निः । यत् । न्य॑ञ्चम् । चि॒नु॒यात् । पृ॒ष्टि॒तः । ए॒न॒म् । आहु॑तय॒ इत्या - हु॒त॒यः॒ । ऋ॒च्छे॒युः॒ । यत् । उ॒त्ता॒नमित्यु॑त्-ता॒नम् । न । पति॑तुम् । श॒क्नु॒या॒त् । असु॑वर्ग्य॒ इत्यसु॑वः-ग्यः॒ । अ॒स्य॒ । स्या॒त् । प्रा॒चीन᳚म् । उ॒त्ता॒नमित्यु॑त्-ता॒नम् ।  \newline




\markright{ TS 5.5.3.3  \hfill https://www.vedavms.in \hfill}

\section{ TS 5.5.3.3 }

\textbf{TS 5.5.3.3 } \newline
\textbf{Samhita Paata} \newline

पु॑रुषशी॒र्॒.षमुप॑ दधाति मुख॒त ए॒वैन॒माहु॑तय ऋच्छन्ति॒ नोत्ता॒नं चि॑नुते सुव॒र्ग्यो᳚ऽस्य भवति सौ॒र्या जु॑होति॒ चक्षु॑रे॒वास्मि॒न् प्रति॑ दधाति॒ द्विर्जु॑होति॒ द्वे हि चक्षु॑षी समा॒न्या जु॑होति समा॒नꣳ हि चक्षुः॒ समृ॑द्ध्यै देवासु॒राः संॅय॑त्ता आस॒न् ते वा॒मं ॅवसु॒ सं न्य॑दधत॒ तद्दे॒वा वा॑म॒भृता॑ऽवृञ्जत॒ तद्वा॑म॒भृतो॑ वामभृ॒त्त्वं ॅयद्वा॑म॒भृत॑ ( ) मुप॒दधा॑ति वा॒ममे॒व तया॒ वसु॒ यज॑मानो॒ भ्रातृ॑व्यस्य वृङ्क्ते॒ हिर॑ण्यमूर्द्ध्नी भवति॒ ज्योति॒र्वै हिर॑ण्यं॒ ज्योति॑र्वा॒मं ज्योति॑षै॒वास्य॒ ज्योति॑र्वा॒मं ॅवृ॑ङ्क्ते द्विय॒जुर्भ॑वति॒ प्रति॑ष्ठित्यै ॥ \newline

\textbf{Pada Paata} \newline

पु॒रु॒ष॒शी॒र्॒.षमिति॑ पुरुष-शी॒र्॒.षम् । उपेति॑ । द॒धा॒ति॒ । मु॒ख॒तः । ए॒व । ए॒न॒म् । आहु॑तय॒ इत्या - हु॒त॒यः॒ । ऋ॒च्छ॒न्ति॒ । न । उ॒त्ता॒नमित्यु॑त्- ता॒नम् । चि॒नु॒ते॒ । सु॒व॒र्ग्य॑ इति॑ सुवः - ग्यः॑ । अ॒स्य॒ । भ॒व॒ति॒ । सौ॒र्या । जु॒हो॒ति॒ । चक्षुः॑ । ए॒व । अ॒स्मि॒न्न् । प्रतीति॑ । द॒धा॒ति॒ । द्विः । जु॒हो॒ति॒ । द्वे इति॑ । हि । चक्षु॑षी॒ इति॑ । स॒मा॒न्या । जु॒हो॒ति॒ । स॒मा॒नम् । हि । चक्षुः॑ । समृ॑द्ध्या॒ इति॒ सं - ऋ॒द्ध्यै॒ । दे॒वा॒सु॒रा इति॑ देव - अ॒सु॒राः । संॅय॑त्ता॒ इति॒ सं - य॒त्ताः॒ । आ॒स॒न्न् । ते । वा॒मम् । वसु॑ । सम् । नीति॑ । अ॒द॒ध॒त॒ । तत् । दे॒वाः । वा॒म॒भृतेति॑ वाम - भृता᳚ । अ॒वृ॒ञ्ज॒त॒ । तत् । वा॒म॒भृत॒ इति॑ वाम - भृतः॑ । वा॒म॒भृ॒त्त्वमिति॑ वामभृत् - त्वम् । यत् । वा॒म॒भृत॒मिति॑ वाम - भृत᳚म् ( ) । उ॒प॒दधा॒तीत्यु॑प - दधा॑ति । वा॒मम् । ए॒व । तया᳚ । वसु॑ । यज॑मानः । भ्रातृ॑व्यस्य । वृ॒ङ्क्ते॒ । हिर॑ण्यमू॒द्‌र्ध्नीति॒ हिर॑ण्य - मू॒द्‌र्ध्नी॒ । भ॒व॒ति॒ । ज्योतिः॑ । वै । हिर॑ण्यम् । ज्योतिः॑ । वा॒मम् । ज्योति॑षा । ए॒व । अ॒स्य॒ । ज्योतिः॑ । वा॒मम् । वृ॒ङ्क्ते॒ । द्वि॒य॒जुरिति॑ द्वि-य॒जुः । भ॒व॒ति॒ । प्रति॑ष्ठित्या॒ इति॒ प्रति॑-स्थि॒त्यै॒ ॥  \newline




\markright{ TS 5.5.4.1  \hfill https://www.vedavms.in \hfill}

\section{ TS 5.5.4.1 }

\textbf{TS 5.5.4.1 } \newline
\textbf{Samhita Paata} \newline

आपो॒ वरु॑णस्य॒ पत्न॑य आस॒न् ता अ॒ग्निर॒भ्य॑द्ध्याय॒त् ताः सम॑भव॒त् तस्य॒ रेतः॒ परा॑ऽपत॒त् तदि॒यम॑भव॒द्यद् द्वि॒तीयं॑ प॒राऽप॑त॒त् तद॒साव॑भवदि॒यं ॅवै वि॒राड॒सौ स्व॒राड् यद्-वि॒राजा॑वुप॒दधा॑ती॒मे ए॒वोप॑ धत्ते॒ यद्वा अ॒सौ रेतः॑ सि॒ञ्चति॒ तद॒स्यां प्रति॑ तिष्ठति॒ तत् प्र जा॑यते॒ ता ओष॑धयो - [  ] \newline

\textbf{Pada Paata} \newline

आपः॑ । वरु॑णस्य । पत्न॑यः । आ॒स॒न्न् । ताः । अ॒ग्निः । अ॒भीति॑ । अ॒द्ध्या॒य॒त् । ताः । समिति॑ । अ॒भ॒व॒त् । तस्य॑ । रेतः॑ । परेति॑ । अ॒प॒त॒त् । तत् । इ॒यम् । अ॒भ॒व॒त् । यत् । द्वि॒तीय᳚म् । प॒राप॑त॒दिति॑ परा - अप॑तत् । तत् । अ॒सौ । अ॒भ॒व॒त् । इ॒यम् । वै । वि॒राडिति॑ वि - राट् । अ॒सौ । स्व॒राडिति॑ स्व - राट् । यत् । वि॒राजा॒विति॑ वि - राजौ᳚ । उ॒प॒दधा॒तीत्यु॑प-दधा॑ति । इ॒मे इति॑ । ए॒व । उपेति॑ । ध॒त्ते॒ । यत् । वै । अ॒सौ । रेतः॑ । सि॒ञ्चति॑ । तत् । अ॒स्याम् । प्रतीति॑ । ति॒ष्ठ॒ति॒ । तत् । प्रेति॑ । जा॒य॒ते॒ । ताः । ओष॑धयः ।  \newline




\markright{ TS 5.5.4.2  \hfill https://www.vedavms.in \hfill}

\section{ TS 5.5.4.2 }

\textbf{TS 5.5.4.2 } \newline
\textbf{Samhita Paata} \newline

वी॒रुधो॑ भवन्ति॒ ता अ॒ग्निर॑त्ति॒ य ए॒वं ॅवेद॒ प्रैव जा॑यतेऽन्ना॒दो भ॑वति॒ यो रे॑त॒स्वी स्यात् प्र॑थ॒मायां॒ तस्य॒ चित्या॑मु॒भे उप॑ दद्ध्यादि॒मे ए॒वास्मै॑ स॒मीची॒ रेतः॑ सिञ्चतो॒ यः सि॒क्तरे॑ताः॒ स्यात् प्र॑थ॒मायां॒ तस्य॒ चित्या॑म॒न्यामुप॑ दद्ध्यादुत्त॒माया॑म॒न्याꣳ रेत॑ ए॒वास्य॑ सि॒क्तमा॒भ्यामु॑भ॒यतः॒ परि॑ गृह्णाति संॅवथ्स॒रं न क - [  ] \newline

\textbf{Pada Paata} \newline

वी॒रुधः॑ । भ॒व॒न्ति॒ । ताः । अ॒ग्निः । अ॒त्ति॒ । यः । ए॒वम् । वेद॑ । प्रेति॑ । ए॒व । जा॒य॒ते॒ । अ॒न्ना॒द इत्य॑न्न - अ॒दः । भ॒व॒ति॒ । यः । रे॒त॒स्वी । स्यात् । प्र॒थ॒माया᳚म् । तस्य॑ । चित्या᳚म् । उ॒भे इति॑ । उपेति॑ । द॒द्ध्या॒त् । इ॒मे इति॑ । ए॒व । अ॒स्मै॒ । स॒मीची॒ इति॑ । रेतः॑ । सि॒ञ्च॒तः॒ । यः । सि॒क्तरे॑ता॒ इति॑ सि॒क्त - रे॒ताः॒ । स्यात् । प्र॒थ॒माया᳚म् । तस्य॑ । चित्या᳚म् । अ॒न्याम् । उपेति॑ । द॒द्ध्या॒त् । उ॒त्त॒माया॒मित्यु॑त्-त॒माया᳚म् । अ॒न्याम् । रेतः॑ । ए॒व । अ॒स्य॒ । सि॒क्तम् । आ॒भ्याम् । उ॒भ॒यतः॑ । परीति॑ । गृ॒ह्णा॒ति॒ । सं॒ॅव॒थ्स॒रमिति॑ सं - व॒थ्स॒रम् । न । कम् ।  \newline




\markright{ TS 5.5.4.3  \hfill https://www.vedavms.in \hfill}

\section{ TS 5.5.4.3 }

\textbf{TS 5.5.4.3 } \newline
\textbf{Samhita Paata} \newline

ञ्च॒न प्र॒त्यव॑रोहे॒न्न हीमे कञ्च॒न प्र॑त्यव॒रोह॑त॒स्तदे॑नयोर्व्र॒तं ॅयो वा अप॑ शीर्.षाणम॒ग्निं चि॑नु॒तेऽप॑शीर्.षा॒ऽमुष्मि॑न् ॅलो॒के भ॑वति॒ यः सशी॑र्.षाणं चिनु॒ते सशी॑र्.षा॒ ऽमुष्मि॑न् ॅलो॒के भ॑वति॒ चित्तिं॑ जुहोमि॒ मन॑सा घृ॒तेन॒ यथा॑ दे॒वा इ॒हाऽऽ*गम॑न् वी॒तिहो᳚त्रा ऋता॒वृधः॑ समु॒द्रस्य॑ व॒युन॑स्य॒ पत्म॑न् जु॒होमि॑ वि॒श्वक॑र्मणे॒ विश्वाऽहाऽम॑र्त्यꣳ ह॒विरिति॑ स्वयमातृ॒ण्णामु॑प॒धाय॑ जुहोत्ये॒ - [  ] \newline

\textbf{Pada Paata} \newline

च॒न । प्र॒त्यव॑रोहे॒दिति॑ प्रति - अव॑रोहेत् । न । हि । इ॒मे इति॑ । कम् । च॒न । प्र॒त्य॒व॒रोह॑त॒ इति॑ प्रति - अ॒व॒रोह॑तः । तत् । ए॒न॒योः॒ । व्र॒तम् । यः । वै । अप॑शीर्.षाण॒मित्यप॑ - शी॒र्.॒षा॒ण॒म् । अ॒ग्निम् । चि॒नु॒ते । अप॑शी॒र्॒.षेत्यप॑ - शी॒र्.॒षा॒ । अ॒मुष्मिन्न्॑ । लो॒के । भ॒व॒ति॒ । यः । सशी॑र्.षाण॒मिति॒ स - शी॒र्.॒षा॒ण॒म् । चि॒नु॒ते । सशी॒र्॒.षेति॒ स - शी॒र्.॒षा॒ । अ॒मुष्मिन्न्॑ । लो॒के । भ॒व॒ति॒ । चित्ति᳚म् । जु॒हो॒मि॒ । मन॑सा । घृ॒तेन॑ । यथा᳚ । दे॒वाः । इ॒ह । आ॒गम॒न्नित्या᳚ - गमन्न्॑ । वी॒तिहो᳚त्रा॒ इति॑ वी॒ति - हो॒त्राः॒ । ऋ॒ता॒वृध॒ इत्यृ॑त - वृधः॑ । स॒मु॒द्रस्य॑ । व॒युन॑स्य । पत्मन्न्॑ । जु॒होमि॑ । वि॒श्वक॑र्मण॒ इति॑ वि॒श्व - क॒र्म॒णे॒ । विश्वा᳚ । अहा᳚ । अम॑र्त्यम् । ह॒विः । इति॑ । स्व॒य॒मा॒तृ॒ण्णामिति॑ स्वयं - आ॒तृ॒ण्णाम् । उ॒प॒धायेत्यु॑प - धाय॑ । जु॒हो॒ति॒ ।  \newline




\markright{ TS 5.5.4.4  \hfill https://www.vedavms.in \hfill}

\section{ TS 5.5.4.4 }

\textbf{TS 5.5.4.4 } \newline
\textbf{Samhita Paata} \newline

तद्वा अ॒ग्नेः शिरः॒ सशी॑र्.षाणमे॒वाग्निं चि॑नुते॒ सशी॑र्.षा॒ऽमुष्मि॑न् ॅलो॒के भ॑वति॒ य ए॒वं ॅवेद॑ सुव॒र्गाय॒ वा ए॒ष लो॒काय॑ चीयते॒ यद॒ग्निस्तस्य॒ यदय॑थापूर्वं क्रि॒यते ऽसु॑वर्ग्यमस्य॒ तथ् सु॑व॒र्ग्यो᳚ ऽग्निश्चिति॑मुप॒धाया॒भि मृ॑शे॒च्चित्ति॒मचि॑त्तिं चिनव॒द्वि वि॒द्वान् पृ॒ष्ठेव॑ वी॒ता वृ॑जि॒ना च॒ मर्ता᳚न् रा॒ये च॑ नः स्वप॒त्याय॑ ( ) देव॒ दितिं॑ च॒ रास्वा-दि॑तिमुरु॒ष्येति॑ यथापू॒र्वमे॒वैना॒मुप॑ धत्ते॒ प्राञ्च॑मेनं चिनुते सुव॒र्ग्यो᳚ऽस्य भवति ॥ \newline

\textbf{Pada Paata} \newline

ए॒तत् । वै । अ॒ग्नेः । शिरः॑ । सशी॑र्.षाण॒मिति॒ स - शी॒र्.॒षा॒ण॒म् । ए॒व । अ॒ग्निम् । चि॒नु॒ते॒ । सशी॒र्.॒षेति॒ स - शी॒र्.॒षा॒ । अ॒मुष्मिन्न्॑ । लो॒के । भ॒व॒ति॒ । यः । ए॒वम् । वेद॑ । सु॒व॒र्गायेति॑ सुवः - गाय॑ । वै । ए॒षः । लो॒काय॑ । ची॒य॒ते॒ । यत् । अ॒ग्निः । तस्य॑ । यत् । अय॑थापूर्व॒मित्यय॑था - पू॒र्व॒म् । क्रि॒यते᳚ । असु॑वर्ग्य॒मित्यसु॑वः-ग्य॒म् । अ॒स्य॒ । तत् । सु॒व॒ग्य॑ इति॑ सुवः - ग्यः॑ । अ॒ग्निः । चिति᳚म् । उ॒प॒धायेत्यु॑प - धाय॑ । अ॒भीति॑ । मृ॒शे॒त् । चित्ति᳚म् । अचि॑त्तिम् । चि॒न॒व॒त् । वीति॑ । वि॒द्वान् । पृ॒ष्ठा । इ॒व॒ । वी॒ता । वृ॒जि॒ना । च॒ । मर्तान्॑ । रा॒ये । च॒ । नः॒ । स्व॒प॒त्यायेति॑ सु - अ॒प॒त्याय॑ ( ) । दे॒व॒ । दिति᳚म् । च॒ । रास्व॑ । अदि॑तिम् । उ॒रु॒ष्य॒ । इति॑ । य॒था॒पू॒र्वमिति॑ यथा - पू॒र्वम् । ए॒व । ए॒ना॒म् । उपेति॑ । ध॒त्ते॒ । प्राञ्च᳚म् । ए॒न॒म् । चि॒नु॒ते॒ । सु॒व॒र्ग्य॑ इति॑ सुवः - ग्यः॑ । अ॒स्य॒ । भ॒व॒ति॒ ॥  \newline




\markright{ TS 5.5.5.1  \hfill https://www.vedavms.in \hfill}

\section{ TS 5.5.5.1 }

\textbf{TS 5.5.5.1 } \newline
\textbf{Samhita Paata} \newline

वि॒श्वक॑र्मा दि॒शां पतिः॒ स नः॑ प॒शून् पा॑तु॒ सो᳚ऽस्मान् पा॑तु॒ तस्मै॒ नमः॑ प्र॒जाप॑ती रु॒द्रो वरु॑णो॒ ऽग्निर्दि॒शां पतिः॒ स नः॑ प॒शून् पा॑तु॒ सो᳚ऽस्मान् पा॑तु॒ तस्मै॒ नम॑ ए॒ता वै दे॒वता॑ ए॒तेषां᳚ पशू॒ना-मधि॑पतय॒-स्ताभ्यो॒ वा ए॒ष आ वृ॑श्च्यते॒ यः प॑शुशी॒र्॒.षाण्यु॑प॒ दधा॑ति हिरण्येष्ट॒का उप॑ दधात्ये॒ताभ्य॑ ए॒व दे॒वता᳚भ्यो॒ नम॑स्करोति ब्रह्मवा॒दिनो॑ - [  ] \newline

\textbf{Pada Paata} \newline

वि॒श्वक॒र्मेति॑ वि॒श्व - क॒र्मा॒ । दि॒शाम् । पतिः॑ । सः । नः॒ । प॒शून् । पा॒तु॒ । सः । अ॒स्मान् । पा॒तु॒ । तस्मै᳚ । नमः॑ । प्र॒जाप॑ति॒रिति॑ प्र॒जा - प॒तिः॒ । रु॒द्रः । वरु॑णः । अ॒ग्निः । दि॒शाम् । पतिः॑ । सः । नः॒ । प॒शून् । पा॒तु॒ । सः । अ॒स्मान् । पा॒तु॒ । तस्मै᳚ । नमः॑ । ए॒ताः । वै । दे॒वताः᳚ । ए॒तेषा᳚म् । प॒शू॒नाम् । अधि॑पतय॒ इत्यधि॑ - प॒त॒यः॒ । ताभ्यः॑ । वै । ए॒षः । एति॑ । वृ॒श्च्य॒ते॒ । यः । प॒शु॒शी॒र्॒.षाणीति॑ पशु - शी॒र्॒.षाणि॑ । उ॒प॒दधा॒तीत्यु॑प - दधा॑ति । हि॒र॒ण्ये॒ष्ट॒का इति॑ हिरण्य - इ॒ष्ट॒काः । उपेति॑ । द॒धा॒ति॒ । ए॒ताभ्यः॑ । ए॒व । दे॒वता᳚भ्यः । नमः॑ । क॒रो॒ति॒ । ब्र॒ह्म॒वा॒दिन॒ इति॑ ब्रह्म - वा॒दिनः॑ ।  \newline




\markright{ TS 5.5.5.2  \hfill https://www.vedavms.in \hfill}

\section{ TS 5.5.5.2 }

\textbf{TS 5.5.5.2 } \newline
\textbf{Samhita Paata} \newline

वदन्त्य॒ग्नौ ग्रा॒म्यान् प॒शून् प्र द॑धाति शु॒चाऽऽर॒ण्यान॑र्पयति॒ किं तत॒ उच्छिꣳ॑ष॒तीति॒ यद्धि॑रण्येष्ट॒का उ॑प॒दधा᳚त्य॒मृतं॒ ॅवै हिर॑ण्यम॒मृते॑नै॒व ग्रा॒म्येभ्यः॑ प॒शुभ्यो॑ भेष॒जं क॑रोति॒ नैनान्॑ हिनस्ति प्रा॒णो वै प्र॑थ॒मा स्व॑यमातृ॒ण्णा व्या॒नो द्वि॒तीया॑ऽपा॒नस्तृ॒तीयाऽनु॒ प्राऽ*ण्या᳚त् प्रथ॒माꣳ स्व॑यमातृ॒ण्णामु॑प॒धाय॑ प्रा॒णेनै॒व प्रा॒णꣳ सम॑र्द्धयति॒ व्य॑न्याद् - [  ] \newline

\textbf{Pada Paata} \newline

व॒द॒न्ति॒ । अ॒ग्नौ । ग्रा॒म्यान् । प॒शून् । प्रेति॑ । द॒धा॒ति॒ । शु॒चा । आ॒र॒ण्यान् । अ॒र्प॒य॒ति॒ । किम् । ततः॑ । उदिति॑ । शिꣳ॒॒ष॒ति॒ । इति॑ । यत् । हि॒र॒ण्ये॒ष्ट॒का इति॑ हिरण्य - इ॒ष्ट॒काः । उ॒प॒दधा॒तीत्यु॑प-दधा॑ति । अ॒मृत᳚म् । वै । हिर॑ण्यम् । अ॒मृते॑न । ए॒व । ग्रा॒म्येभ्यः॑ । प॒शुभ्य॒ इति॑ प॒शु - भ्यः॒ । भे॒ष॒जम् । क॒रो॒ति॒ । न । ए॒ना॒न् । हि॒न॒स्ति॒ । प्रा॒ण इति॑ प्र - अ॒नः । वै । प्र॒थ॒मा । स्व॒य॒मा॒तृ॒ण्णेति॑ स्वयं -  आ॒तृ॒ण्णा । व्या॒न इति॑ वि - अ॒नः । द्वि॒तीया᳚ । अ॒पा॒न इत्य॑प-अ॒नः । तृ॒तीया᳚ । अनु॑ । प्रेति॑ । अ॒न्या॒त् । प्र॒थ॒माम् । स्व॒य॒मा॒तृ॒ण्णामिति॑ स्वयं - आ॒तृ॒ण्णाम् । उ॒प॒धायेत्यु॑प - धाय॑ । प्रा॒णेनेति॑ प्र - अ॒नेन॑ । ए॒व । प्रा॒णमिति॑ प्र - अ॒नम् । समिति॑ । अ॒द्‌र्ध॒य॒ति॒ । वीति॑ । अ॒न्या॒त् ।  \newline




\markright{ TS 5.5.5.3  \hfill https://www.vedavms.in \hfill}

\section{ TS 5.5.5.3 }

\textbf{TS 5.5.5.3 } \newline
\textbf{Samhita Paata} \newline

द्वि॒तीया॑मुप॒धाय॑ व्या॒नेनै॒व व्या॒नꣳ सम॑र्द्धय॒त्यपा᳚न् यात्तृ॒तीया॑मुप॒धाया॑-पा॒नेनै॒वापा॒नꣳ सम॑र्द्धय॒त्यथो᳚ प्रा॒णैरे॒वैनꣳ॒॒ समि॑न्धे॒ भूर्भुवः॒ सुव॒रिति॑ स्वयमातृ॒ण्णा उप॑ दधाती॒मे वै लो॒काः स्व॑यमातृ॒ण्णा ए॒ताभिः॒ खलु॒वै व्याहृ॑तीभिः प्र॒जाप॑तिः॒ प्रा*ऽजा॑यत॒ यदे॒ताभि॒र्व्याहृ॑तीभिः स्वयमातृ॒ण्णा उ॑प॒दधा॑ती॒माने॒व लो॒कानु॑प॒धायै॒षु - [  ] \newline

\textbf{Pada Paata} \newline

द्वि॒तीया᳚म् । उ॒प॒धायेत्यु॑प - धाय॑ । व्या॒नेनेति॑ वि - अ॒नेन॑ । ए॒व । व्या॒नमिति॑ वि - अ॒नम् । समिति॑ । अ॒द्‌र्ध॒य॒ति॒ । अपेति॑ । अ॒न्या॒त् । तृ॒तीया᳚म् । उ॒प॒धायेत्यु॑प - धाय॑ । अ॒पा॒नेनेत्य॑प - अ॒नेन॑ । ए॒व । अ॒पा॒नमित्य॑प - अ॒नम् । समिति॑ । अ॒द्‌र्ध॒य॒ति॒ । अथो॒ इति॑ । प्रा॒णैरिति॑ प्र - अ॒नैः । ए॒व । ए॒न॒म् । समिति॑ । इ॒न्धे॒ । भूः । भुवः॑ । सुवः॑ । इति॑ । स्व॒य॒मा॒तृ॒ण्णा इति॑ स्वयं-आ॒तृ॒ण्णाः । उपेति॑ । द॒धा॒ति॒ । इ॒मे । वै । लो॒काः । स्व॒य॒मा॒तृ॒ण्णा इति॑ स्वयं-आ॒तृ॒ण्णाः । ए॒ताभिः॑ । खलु॑ । वै । व्याहृ॑तीभि॒रिति॒ व्याहृ॑ति - भिः॒ । प्र॒जाप॑ति॒रिति॑ प्र॒जा - प॒तिः॒ । प्रेति॑ । अ॒जा॒य॒त॒ । यत् । ए॒ताभिः॑ । व्याहृ॑तीभि॒रिति॒ व्याहृ॑ति - भिः । स्व॒य॒मा॒तृ॒ण्णा इति॑ स्वयं - आ॒तृ॒ण्णाः । उ॒प॒दधा॒तीत्यु॑प - दधा॑ति । इ॒मान् । ए॒व । लो॒कान् । उ॒प॒धायेत्यु॑प - धाय॑ । ए॒षु ।  \newline




\markright{ TS 5.5.5.4  \hfill https://www.vedavms.in \hfill}

\section{ TS 5.5.5.4 }

\textbf{TS 5.5.5.4 } \newline
\textbf{Samhita Paata} \newline

लो॒केष्वधि॒ प्रजा॑यते प्रा॒णाय॑ व्या॒नाया॑पा॒नाय॑ वा॒चे त्वा॒ चक्षु॑षे त्वा॒ तया॑ दे॒वत॑याऽङ्गिर॒स्वद् ध्रु॒वा सी॑दा॒ग्निना॒ वै दे॒वाः सु॑व॒र्गं ॅलो॒कम॑जिगाꣳस॒न् तेन॒ पति॑तुं॒ नाश॑क्नुव॒न् त ए॒ताश्चत॑स्रः स्वयमातृ॒ण्णा अ॑पश्य॒न् ता दि॒क्षूपा॑दधत॒ तेन॑ स॒र्वत॑श्चक्षुषा सुव॒र्गं ॅलो॒कमा॑य॒न्॒ यच्चत॑स्रः स्वयमातृ॒ण्णा दि॒क्षू॑प॒दधा॑ति स॒र्वत॑श्चक्षुषै॒व तद॒ग्निना॒ यज॑मानः ( ) सुव॒र्गं ॅलो॒कमे॑ति ॥ \newline

\textbf{Pada Paata} \newline

लो॒केषु॑ । अधि॑ । प्रेति॑ । जा॒य॒ते॒ । प्रा॒णायेति॑ प्र - अ॒नाय॑ । व्या॒नायेति॑ वि - अ॒नाय॑ । अ॒पा॒नायेत्य॑प - अ॒नाय॑ । वा॒चे । त्वा॒ । चक्षु॑षे । त्वा॒ । तया᳚ । दे॒वत॑या । अ॒ङ्गि॒र॒स्वत् । ध्रु॒वा । सी॒द॒ । अ॒ग्निना᳚ । वै । दे॒वाः । सु॒व॒र्गमिति॑ सुवः - गम् । लो॒कम् । अ॒जि॒गाꣳ॒॒स॒न्न् । तेन॑ । पति॑तुम् । न । अ॒श॒क्नु॒व॒न्न् । ते । ए॒ताः । चत॑स्रः । स्व॒य॒मा॒तृ॒ण्णा इति॑ स्वयं - अ॒तृ॒ण्णाः । अ॒प॒श्य॒न्न् । ताः । दि॒क्षु । उपेति॑ । अ॒द॒ध॒त॒ । तेन॑ । स॒र्वत॑श्चक्षु॒षेति॑ स॒र्वतः॑ - च॒क्षु॒षा॒ । सु॒व॒र्गमिति॑ सुवः - गम् । लो॒कम् । आ॒य॒न्न् । यत् । चत॑स्रः । स्व॒य॒मा॒तृ॒ण्णा इति॑ स्वयं - आ॒तृ॒ण्णाः । दि॒क्षु । उ॒प॒दधा॒तीत्यु॑प - दधा॑ति । सर्वत॑श्चक्षु॒षेति॑ स॒र्वतः॑ - च॒क्षु॒षा॒ । ए॒व । तत् । अ॒ग्निना᳚ । यज॑मानः ( ) । सु॒व॒र्गमिति॑ सुवः - गम् । लो॒कम् । ए॒ति॒ ॥  \newline




\markright{ TS 5.5.6.1  \hfill https://www.vedavms.in \hfill}

\section{ TS 5.5.6.1 }

\textbf{TS 5.5.6.1 } \newline
\textbf{Samhita Paata} \newline

अग्न॒ आ या॑हि वी॒तय॒ इत्या॒हा-ह्व॑तै॒वैन॑-म॒ग्निं दू॒तं ॅवृ॑णीमह॒ इत्या॑ह हू॒त्वैवैनं॑ ॅवृणीते॒ ऽग्निना॒ऽग्निः समि॑द्ध्यत॒ इत्या॑ह॒ समि॑न्ध ए॒वैन॑म॒ग्निर्वृ॒त्राणि॑ जङ्घन॒दित्या॑ह॒ समि॑द्ध ए॒वास्मि॑न्निन्द्रि॒यं द॑धात्य॒ग्नेः स्तोमं॑ मनामह॒ इत्या॑ह मनु॒त ए॒वैन॑मे॒तानि॒ वा अह्नाꣳ॑ रू॒पाण्य॑ - [  ] \newline

\textbf{Pada Paata} \newline

अग्ने᳚ । एति॑ । या॒हि॒ । वी॒तये᳚ । इति॑ । आ॒ह॒ । अह्व॑त । ए॒व । ए॒न॒म् । अ॒ग्निम् । दू॒तम् । वृ॒णी॒म॒हे॒ । इति॑ । आ॒ह॒ । हू॒त्वा । ए॒व । ए॒न॒म् । वृ॒णी॒ते॒ । अ॒ग्निना᳚ । अ॒ग्निः । समिति॑ । इ॒द्ध्य॒ते॒ । इति॑ । आ॒ह॒ । समिति॑ । इ॒न्धे॒ । ए॒व । ए॒न॒म् । अ॒ग्निः । वृ॒त्राणि॑ । ज॒ङ्घ॒न॒त् । इति॑ । आ॒ह॒ । समि॑द्ध॒ इति॒ सं - इ॒द्धे॒ । ए॒व । अ॒स्मि॒न्न् । इ॒न्द्रि॒यम् । द॒धा॒ति॒ । अ॒ग्नेः । स्तोम᳚म् । म॒ना॒म॒हे॒ । इति॑ । आ॒ह॒ । म॒नु॒ते । ए॒व । ए॒न॒म् । ए॒तानि॑ । वै । अह्ना᳚म् । रू॒पाणि॑ ।  \newline




\markright{ TS 5.5.6.2  \hfill https://www.vedavms.in \hfill}

\section{ TS 5.5.6.2 }

\textbf{TS 5.5.6.2 } \newline
\textbf{Samhita Paata} \newline

-न्व॒हमे॒वैनं॑ चिनु॒ते ऽवाह्नाꣳ॑ रू॒पाणि॑ रुन्धे ब्रह्मवा॒दिनो॑ वदन्ति॒ कस्मा᳚थ् स॒त्याद्या॒तया᳚म्नीर॒न्या इष्ट॑का॒ अया॑तयाम्नी लोकं पृ॒णेत्यै᳚न्द्रा॒ग्नी हि बा॑र्.हस्प॒त्येति॑ ब्रूयादिन्द्रा॒ग्नी च॒ हि दे॒वानां॒ बृह॒स्पति॒श्चा-या॑तयामानो ऽनुच॒रव॑ती भव॒त्यजा॑मित्वाया -नु॒ष्टुभाऽनु॑ चरत्या॒त्मा वै लो॑कं पृ॒णा प्रा॒णो॑ ऽनु॒ष्टुप् तस्मा᳚त् प्रा॒णः सर्वा॒ण्यङ्गा॒न्यनु॑ चरति॒ ता अ॑स्य॒ सूद॑दोहस॒ - [  ] \newline

\textbf{Pada Paata} \newline

अ॒न्व॒हमित्य॑नु - अ॒हम् । ए॒व । ए॒न॒म् । चि॒नु॒ते॒ । अवेति॑ । अह्ना᳚म् । रू॒पाणि॑ । रु॒न्धे॒ । ब्र॒ह्म॒वा॒दिन॒ इति॑ ब्रह्म-वा॒दिनः॑ । व॒द॒न्ति॒ । कस्मा᳚त् । स॒त्यात् । या॒तया᳚म्नी॒रिति॑ या॒त - या॒म्नीः॒ । अ॒न्याः । इष्ट॑काः । अया॑तया॒म्नीत्यया॑त-या॒म्नी॒ । लो॒क॒पृं॒णेति॑ लोकं - पृ॒णा । इति॑ । ऐ॒न्द्रा॒ग्नीत्यै᳚न्द्र-अ॒ग्नी । हि । बा॒र्.॒ह॒स्प॒त्या । इति॑ । ब्रू॒या॒त् । इ॒न्द्रा॒ग्नी इती᳚न्द्र - अ॒ग्नी । च॒ । हि । दे॒वाना᳚म् । बृह॒स्पतिः॑ । च॒ । अया॑तयामान॒ इत्यया॑त - या॒मा॒नः॒ । अ॒नु॒च॒रव॒तीत्य॑नुच॒र - व॒ती॒ । भ॒व॒ति॒ । अजा॑मित्वा॒येत्यजा॑मि-त्वा॒य॒ । अ॒नु॒ष्टुभेत्य॑नु - स्तुभा᳚ । अन्विति॑ । च॒र॒ति॒ । आ॒त्मा । वै । लो॒क॒पृं॒णेति॑ लोकं - पृ॒णा । प्रा॒ण इति॑ प्र - अ॒नः । अ॒नु॒ष्टुबित्य॑नु - स्तुप् । तस्मा᳚त् । प्रा॒ण इति॑ प्र - अ॒नः । सर्वा॑णि । अङ्गा॑नि । अन्विति॑ । च॒र॒ति॒ । ताः । अ॒स्य॒ । सूद॑दोहस॒ इति॒ सूद॑ - दो॒ह॒सः॒ ।  \newline




\markright{ TS 5.5.6.3  \hfill https://www.vedavms.in \hfill}

\section{ TS 5.5.6.3 }

\textbf{TS 5.5.6.3 } \newline
\textbf{Samhita Paata} \newline

इत्या॑ह॒ तस्मा॒त् परु॑षिपरुषि॒ रसः॒ सोमꣳ॑ श्रीणन्ति॒ पृश्न॑य॒ इत्या॒हान्नं॒ ॅवै पृश्न्यन्न॑मे॒वाव॑ रुन्धे॒ऽर्को वा अ॒ग्निर॒र्कोऽन्न॒मन्न॑मे॒वाव॑ रुन्धे॒ जन्म॑न् दे॒वानां॒ ॅविश॑स्त्रि॒ष्वा रो॑च॒ने दि॒व इत्या॑हे॒माने॒वास्मै॑ लो॒कान् ज्योति॑ष्मतः करोति॒ यो वा इष्ट॑कानां प्रति॒ष्ठां ॅवेद॒ प्रत्ये॒व ति॑ष्ठति॒ तया॑ ( ) दे॒वत॑याऽङ्गिर॒स्वद् ध्रु॒वा सी॒देत्या॑है॒षा वा इष्ट॑कानां प्रति॒ष्ठा य ए॒वं ॅवेद॒ प्रत्ये॒वति॑ष्ठति ॥ \newline

\textbf{Pada Paata} \newline

इति॑ । आ॒ह॒ । तस्मा᳚त् । परु॑षिपरु॒षीति॒ परु॑षि - प॒रु॒षि॒ । रसः॑ । सोम᳚म् । श्री॒ण॒न्ति॒ । पृश्न॑यः । इति॑ । आ॒ह॒ । अन्न᳚म् । वै । पृश्नि॑ । अन्न᳚म् । ए॒व । अवेति॑ । रु॒न्धे॒ । अ॒र्कः । वै । अ॒ग्निः । अ॒र्कः । अन्न᳚म् । अन्न᳚म् । ए॒व । अवेति॑ । रु॒न्धे॒ । जन्मन्न्॑ । दे॒वाना᳚म् । विशः॑ । त्रि॒षु । एति॑ । रो॒च॒ने । दि॒वः । इति॑ । आ॒ह॒ । इ॒मान् । ए॒व । अ॒स्मै॒ । लो॒कान् । ज्योति॑ष्मतः । क॒रो॒ति॒ । यः । वै । इष्ट॑कानाम् । प्र॒ति॒ष्ठामिति॑ प्रति - स्थाम् । वेद॑ । प्रतीति॑ । ए॒व । ति॒ष्ठ॒ति॒ । तया᳚ ( ) । दे॒वत॑या । अ॒ङ्गि॒र॒स्वत् । ध्रु॒वा । सी॒द॒ । इति॑ । आ॒ह॒ । ए॒षा । वै । इष्ट॑कानाम् । प्र॒ति॒ष्ठेति॑ प्रति - स्था । यः । ए॒वम् । वेद॑ । प्रतीति॑ । ए॒व । ति॒ष्ठ॒ति॒ ॥  \newline




\markright{ TS 5.5.7.1  \hfill https://www.vedavms.in \hfill}

\section{ TS 5.5.7.1 }

\textbf{TS 5.5.7.1 } \newline
\textbf{Samhita Paata} \newline

सु॒व॒र्गाय॒ वा ए॒ष लो॒काय॑ चीयते॒ यद॒ग्निर्वज्र॑ एकाद॒शिनी॒ यद॒ग्नावे॑काद॒शिनीं᳚ मिनु॒याद्-वज्रे॑णैनꣳ सुव॒र्गाल्लो॒काद॒न्तर्द॑द्ध्या॒द्यन्न मि॑नु॒याथ् स्वरु॑भिः प॒शून् व्य॑र्द्धयेदेकयू॒पं मि॑नोति॒ नैनं॒ ॅवज्रे॑ण सुव॒र्गाल्लो॒काद॑न्त॒र्दधा॑ति॒ न स्वरु॑भिः प॒शून् व्य॑र्द्धयति॒ वि वा ए॒ष इ॑न्द्रि॒येण॑ वी॒र्ये॑णर्द्ध्यते॒ यो᳚ऽग्निं चि॒न्व-न्न॑धि॒क्राम॑त्यैन्द्रि॒य - [  ] \newline

\textbf{Pada Paata} \newline

सु॒व॒र्गायेति॑ सुवः - गाय॑ । वै । ए॒षः । लो॒काय॑ । ची॒य॒ते॒ । यत् । अ॒ग्निः । वज्रः॑ । ए॒का॒द॒शिनी᳚ । यत् । अ॒ग्नौ । ए॒का॒द॒शिनी᳚म् । मि॒नु॒यात् । वज्रे॑ण । ए॒न॒म् । सु॒व॒र्गादिति॑ सुवः - गात् । लो॒कात् । अ॒न्तः । द॒द्ध्या॒त् । यत् । न । मि॒नु॒यात् । स्वरु॑भि॒रिति॒ स्वरु॑-भिः॒ । प॒शून् । वीति॑ । अ॒द्‌र्ध॒ये॒त् । ए॒क॒यू॒पमित्ये॑क - यू॒पम् । मि॒नो॒ति॒ । न । ए॒न॒म् । वज्रे॑ण । सु॒व॒र्गादिति॑ सुवः - गात् । लो॒कात् । अ॒न्त॒र्दधा॒तीत्य॑न्तः - दधा॑ति । न । स्वरु॑भि॒रिति॒ स्वरु॑-भिः॒ । प॒शून् । वीति॑ । अ॒द्‌र्ध॒य॒ति॒ । वीति॑ । वै । ए॒षः । इ॒न्द्रि॒येण॑ । वी॒र्ये॑ण । ऋ॒द्ध्य॒ते॒ । यः । अ॒ग्निम् । चि॒न्वन्न् । अ॒धि॒क्राम॒तीत्य॑धि - क्राम॑ति । ऐ॒न्द्रि॒या ।  \newline




\markright{ TS 5.5.7.2  \hfill https://www.vedavms.in \hfill}

\section{ TS 5.5.7.2 }

\textbf{TS 5.5.7.2 } \newline
\textbf{Samhita Paata} \newline

-र्चा ऽऽक्रम॑णं॒ प्रतीष्ट॑का॒मुप॑ दद्ध्या॒न्नेन्द्रि॒येण॑ वी॒र्ये॑ण॒ व्यृ॑द्ध्यते रु॒द्रो वा ए॒ष यद॒ग्निस्तस्य॑ ति॒स्रः श॑र॒व्याः᳚ प्र॒तीची॑ ति॒रश्च्य॒नूची॒ ताभ्यो॒ वा ए॒ष आ वृ॑श्च्यते॒ यो᳚ऽग्निं चि॑नु॒ते᳚ ऽग्निं चि॒त्वा ति॑सृध॒न्वमया॑चितं ब्राह्म॒णाय॑ दद्या॒त् ताभ्य॑ ए॒व नम॑स्करो॒त्यथो॒ ताभ्य॑ ए॒वाऽऽत्मानं॒ निष्क्री॑णीते॒ यत्ते॑ रुद्र पु॒रो - [  ] \newline

\textbf{Pada Paata} \newline

ऋ॒चा । आ॒क्रम॑ण॒मित्या᳚ - क्रम॑णम् । प्रतीति॑ । इष्ट॑काम् । उपेति॑ । द॒द्ध्या॒त् । न । इ॒न्द्रि॒येण॑ । वी॒र्ये॑ण । वीति॑ । ऋ॒द्ध्य॒ते॒ । रु॒द्रः । वै । ए॒षः । यत् । अ॒ग्निः । तस्य॑ । ति॒स्रः । श॒र॒व्याः᳚ । प्र॒तीची᳚ । ति॒रश्ची᳚ । अ॒नूची᳚ । ताभ्यः॑ । वै । ए॒षः । एति॑ । वृ॒श्च्य॒ते॒ । यः । अ॒ग्निम् । चि॒नु॒ते । अ॒ग्निम् । चि॒त्वा । ति॒सृ॒ध॒न्वमिति॑ तिसृ - ध॒न्वम् । अया॑चितम् । ब्रा॒ह्म॒णाय॑ । द॒द्या॒त् । ताभ्यः॑ । ए॒व । नमः॑ । क॒रो॒ति॒ । अथो॒ इति॑ । ताभ्यः॑ । ए॒व । आ॒त्मान᳚म् । निरिति॑ । क्री॒णी॒ते॒ । यत् । ते॒ । रु॒द्र॒ । पु॒रः ।  \newline




\markright{ TS 5.5.7.3  \hfill https://www.vedavms.in \hfill}

\section{ TS 5.5.7.3 }

\textbf{TS 5.5.7.3 } \newline
\textbf{Samhita Paata} \newline

धनु॒स्तद्-वातो॒ अनु॑ वातु ते॒ तस्मै॑ ते॒ रुद्र संॅवथ्स॒रेण॒ नम॑स्करोमि॒ यत्ते॑ रुद्र दक्षि॒णा धनु॒स्तद्-वातो॒ अनु॑ वातु ते॒ तस्मै॑ ते रुद्र परिवथ्स॒रेण॒ नम॑स्करोमि॒ यत्ते॑ रुद्र प॒श्चाद्धनु॒स्तद्-वातो॒ अनु॑ वातु ते॒ तस्मै॑ ते रुद्रेदावथ्स॒रेण॒ नम॑स्करोमि॒ यत्ते॑ रुद्रोत्त॒राद्धनु॒स्त - [  ] \newline

\textbf{Pada Paata} \newline

धनुः॑ । तत् । वातः॑ । अन्विति॑ । वा॒तु॒ । ते॒ । तस्मै᳚ । ते॒ । रु॒द्र॒ । सं॒ॅव॒थ्स॒रेणेति॑ सं - व॒थ्स॒रेण॑ । नमः॑ । क॒रो॒मि॒ । यत् । ते॒ । रु॒द्र॒ । द॒क्षि॒णा । धनुः॑ । तत् । वातः॑ । अन्विति॑ । वा॒तु॒ । ते॒ । तस्मै᳚ । ते॒ । रु॒द्र॒ । प॒रि॒व॒थ्स॒रेणेति॑ परि - व॒थ्स॒रेण॑ । नमः॑ । क॒रो॒मि॒ । यत् । ते॒ । रु॒द्र॒ । प॒श्चात् । धनुः॑ । तत् । वातः॑ । अन्विति॑ । वा॒तु॒ । ते॒ । तस्मै᳚ । ते॒ । रु॒द्र॒ । इ॒दा॒व॒थ्स॒रेण॑ । नमः॑ । क॒रो॒मि॒ । यत् । ते॒ । रु॒द्र॒ । उ॒त्त॒रादित्यु॑त् - त॒रात् । धनुः॑ । तत् ।  \newline




\markright{ TS 5.5.7.4  \hfill https://www.vedavms.in \hfill}

\section{ TS 5.5.7.4 }

\textbf{TS 5.5.7.4 } \newline
\textbf{Samhita Paata} \newline

द्वातो॒ अनु॑ वातु ते॒ तस्मै॑ ते रुद्रेदुवथ्स॒रेण॒ नम॑स्करोमि॒ यत्ते॑ रुद्रो॒परि॒ धनु॒स्तद्-वातो॒ अनु॑ वातु ते॒ तस्मै॑ ते रुद्र वथ्स॒रेण॒ नम॑स्करोमि रु॒द्रो वा ए॒ष यद॒ग्निः स यथा᳚ व्या॒घ्रः क्रु॒द्धस्तिष्ठ॑त्ये॒वं ॅवा ए॒ष ए॒तर्.हि॒ सञ्चि॑तमे॒तैरुप॑ तिष्ठते नमस्का॒रैरे॒वैनꣳ॑ शमयति॒ ये᳚ऽग्नयः॑ - [  ] \newline

\textbf{Pada Paata} \newline

वातः॑ । अन्विति॑ । वा॒तु॒ । ते॒ । तस्मै᳚ । ते॒ । रु॒द्र॒ । इ॒दु॒व॒थ्स॒रेणेती॑दु- व॒थ्स॒रेण॑ । नमः॑ । क॒रो॒मि॒ । यत् । ते॒ । रु॒द्र॒ । उ॒परि॑ । धनुः॑ । तत् । वातः॑ । अन्विति॑ । वा॒तु॒ । ते॒ । तस्मै᳚ । ते॒ । रु॒द्र॒ । व॒थ्स॒रेण॑ । नमः॑ । क॒रो॒मि॒ । रु॒द्रः । वै । ए॒षः । यत् । अ॒ग्निः । सः । यथा᳚ । व्या॒घ्रः । क्रु॒द्धः । तिष्ठ॑ति । ए॒वम् । वै । ए॒षः । ए॒तर्.हि॑ । सञ्चि॑त॒मिति॒ सं - चि॒त॒म् । ए॒तैः । उपेति॑ । ति॒ष्ठ॒ते॒ । न॒म॒स्का॒रैरिति॑ नमः-का॒रैः । ए॒व । ए॒न॒म् । श॒म॒य॒ति॒ । ये । अ॒ग्नयः॑ ।  \newline




\markright{ TS 5.5.7.5  \hfill https://www.vedavms.in \hfill}

\section{ TS 5.5.7.5 }

\textbf{TS 5.5.7.5 } \newline
\textbf{Samhita Paata} \newline

पुरी॒ष्याः᳚ प्रवि॑ष्टाः पृथि॒वीमनु॑ । तेषां॒ त्वम॑स्युत्त॒मः प्रणो॑ जी॒वात॑वे सुव ॥ आपं॑ त्वाऽग्ने॒ मन॒सा ऽऽपं॑ त्वाऽग्ने॒ तप॒सा ऽऽपं॑ त्वाऽग्ने दी॒क्षया ऽऽपं॑ त्वाऽग्न उप॒सद्भि॒रापं॑ त्वाऽग्ने सु॒त्ययाऽऽपं॑ त्वाऽग्ने॒ दक्षि॑णाभि॒रापं॑ त्वाऽग्ने ऽवभृ॒थेनापं॑ त्वाऽग्ने व॒शया ऽऽपं॑ त्वाऽग्ने स्वगाका॒रेणेत्या॑है॒ ( ) षा वा अ॒ग्नेराप्ति॒स्तयै॒वैन॑माप्नोति ॥ \newline

\textbf{Pada Paata} \newline

पु॒री॒ष्याः᳚ । प्रवि॑ष्टा॒ इति॒ प्र - वि॒ष्टाः॒ । पृ॒थि॒वीम् । अनु॑ ॥ तेषा᳚म् । त्वम् । अ॒सि॒ । उ॒त्त॒म इत्यु॑त् - त॒मः । प्रेति॑ । नः॒ । जी॒वात॑वे । सु॒व॒ ॥ आप᳚म् । त्वा॒ । अ॒ग्ने॒ । मन॑सा । आप᳚म् । त्वा॒ । अ॒ग्ने॒ । तप॑सा । आप᳚म् । त्वा॒ । अ॒ग्ने॒ । दी॒क्षया᳚ । आप᳚म् । त्वा॒ । अ॒ग्ने॒ । उ॒प॒सद्भि॒रित्यु॑प॒सत् - भिः॒ । आप᳚म् । त्वा॒ । अ॒ग्ने॒ । सु॒त्यया᳚ । आप᳚म् । त्वा॒ । अ॒ग्ने॒ । दक्षि॑णाभिः । आप᳚म् । त्वा॒ । अ॒ग्ने॒ । अ॒व॒भृ॒थेनेत्य॑व - भृ॒थेन॑ । आप᳚म् । त्वा॒ । अ॒ग्ने॒ । व॒शया᳚ । आप᳚म् । त्वा॒ । अ॒ग्ने॒ । स्व॒गा॒का॒रेणेति॑ स्वगा - का॒रेण॑ । इति॑ । आ॒ह॒ ( ) । ए॒षा । वै । अ॒ग्नेः । आप्तिः॑ । तया᳚ । ए॒व । ए॒न॒म् । आ॒प्नो॒ति॒ ॥  \newline




\markright{ TS 5.5.8.1  \hfill https://www.vedavms.in \hfill}

\section{ TS 5.5.8.1 }

\textbf{TS 5.5.8.1 } \newline
\textbf{Samhita Paata} \newline

गा॒य॒त्रेण॑ पु॒रस्ता॒दुप॑ तिष्ठते प्रा॒णमे॒वास्मि॑न् दधाति बृहद्-रथंत॒राभ्यां᳚ प॒क्षावोज॑ ए॒वास्मि॑न् दधात्यृतु॒स्थाय॑ज्ञा-य॒ज्ञिये॑न॒ पुच्छ॑मृ॒तुष्वे॒व प्रति॑ तिष्ठति पृ॒ष्ठैरुप॑ तिष्ठते॒ तेजो॒ वै पृ॒ष्ठानि॒ तेज॑ ए॒वास्मि॑न् दधाति प्र॒जाप॑तिर॒ग्निम॑सृजत॒ सो᳚ऽस्माथ् सृ॒ष्टः परा॑ङै॒त् तं ॅवा॑रव॒न्तीये॑ना-वारयत॒ तद्-वा॑रव॒न्तीय॑स्य वारवन्तीय॒त्वꣳ श्यै॒तेन॑ श्ये॒ती अ॑कुरुत॒ तच्छ्यै॒तस्य॑ श्यैत॒त्वं - [  ] \newline

\textbf{Pada Paata} \newline

गा॒य॒त्रेण॑ । पु॒रस्ता᳚त् । उपेति॑ । ति॒ष्ठ॒ते॒ । प्रा॒णमिति॑ प्र - अ॒नम् । ए॒व । अ॒स्मि॒न्न् । द॒धा॒ति॒ । बृ॒ह॒द्र॒थ॒न्त॒राभ्या॒मिति॑ बृहत् - र॒थ॒न्त॒राभ्या᳚म् । प॒क्षौ । ओजः॑ । ए॒व । अ॒स्मि॒न्न् । द॒धा॒ति॒ । ऋ॒तु॒स्थाय॑ज्ञाय॒ज्ञिये॑न । पुच्छ᳚म् । ऋ॒तुषु॑ । ए॒व । प्रतीति॑ । ति॒ष्ठ॒ति॒ । पृ॒ष्ठैः । उपेति॑ । ति॒ष्ठ॒ते॒ । तेजः॑ । वै । पृ॒ष्ठानि॑ । तेजः॑ । ए॒व । अ॒स्मि॒न्न् । द॒धा॒ति॒ । प्र॒जाप॑ति॒रिति॑ प्र॒जा - प॒तिः॒ । अ॒ग्निम् । अ॒सृ॒ज॒त॒ । सः । अ॒स्मा॒त् । सृ॒ष्टः । पराङ्॑ । ऐ॒त् । तम् । वा॒र॒व॒न्तीये॒नेति॑ वार - व॒न्तीये॑न । अ॒वा॒र॒य॒त॒ । तत् । वा॒र॒व॒न्तीय॒स्येति॑ वार - व॒न्तीय॑स्य । वा॒र॒व॒न्ती॒य॒त्वमिति॑ वारवन्तीय - त्वम् । श्यै॒तेन॑ । श्ये॒ती । अ॒कु॒रु॒त॒ । तत् । श्यै॒तस्य॑ । श्यै॒त॒त्वमिति॑ श्यैत - त्वम् ।  \newline




\markright{ TS 5.5.8.2  \hfill https://www.vedavms.in \hfill}

\section{ TS 5.5.8.2 }

\textbf{TS 5.5.8.2 } \newline
\textbf{Samhita Paata} \newline

ॅयद्-वा॑रव॒न्तीये॑नोप॒तिष्ठ॑ते वा॒रय॑त ऐ॒वैनꣳ॑ श्यै॒तेन॑ श्ये॒ती कु॑रुते प्र॒जाप॑ते॒र्.हृद॑येना-पिप॒क्षं प्रत्युप॑ तिष्ठते प्रे॒माण॑मे॒वास्य॑ गच्छति॒ प्राच्या᳚ त्वा दि॒शा सा॑दयामि गाय॒त्रेण॒ छन्द॑सा॒ऽग्निना॑ दे॒वत॑या॒ऽग्नेः शीर्ष्णाग्नेः शिर॒ उप॑ दधामि॒ दक्षि॑णया त्वा दि॒शा सा॑दयामि॒ त्रैष्टु॑भेन॒ छन्द॒सेन्द्रे॑ण दे॒वत॑या॒ऽग्नेः प॒क्षेणा॒ग्नेः प॒क्षमुप॑ दधामि प्र॒तीच्या᳚ त्वा दि॒शा सा॑दयामि॒ - [  ] \newline

\textbf{Pada Paata} \newline

यत् । वा॒र॒व॒न्तीये॒नेति॑ वार - व॒न्तीये॑न । उ॒प॒तिष्ठ॑त॒ इत्यु॑प - तिष्ठ॑ते । वा॒रय॑ते । ए॒व । ए॒न॒म् । श्यै॒तेन॑ । श्ये॒ती । कु॒रु॒ते॒ । प्र॒जाप॑ते॒र्.हृद॑येन । अ॒पि॒प॒क्षमित्य॑पि - प॒क्षम् । प्रति॑ । उपेति॑ । ति॒ष्ठ॒ते॒ । प्रे॒माण᳚म् । ए॒व । अ॒स्य॒ । ग॒च्छ॒ति॒ । प्राच्या᳚ । त्वा॒ । दि॒शा । सा॒द॒या॒मि॒ । गा॒य॒त्रेण॑ । छन्द॑सा । अ॒ग्निना᳚ । दे॒वत॑या । अ॒ग्नेः । शी॒र्ष्णा । अ॒ग्नेः । शिरः॑ । उपेति॑ । द॒धा॒मि॒ । दक्षि॑णया । त्वा॒ । दि॒शा । सा॒द॒या॒मि॒ । त्रैष्टु॑भेन । छन्द॑सा । इन्द्रे॑ण । दे॒वत॑या । अ॒ग्नेः । प॒क्षेण॑ । अ॒ग्नेः । प॒क्षम् । उपेति॑ । द॒धा॒मि॒ । प्र॒तीच्या᳚ । त्वा॒ । दि॒शा । सा॒द॒या॒मि॒ ।  \newline




\markright{ TS 5.5.8.3  \hfill https://www.vedavms.in \hfill}

\section{ TS 5.5.8.3 }

\textbf{TS 5.5.8.3 } \newline
\textbf{Samhita Paata} \newline

जाग॑तेन॒ छन्द॑सा सवि॒त्रा दे॒वत॑या॒ऽग्नेः पुच्छे॑ना॒ग्नेः पुच्छ॒मुप॑ दधा॒म्युदी᳚च्या त्वा दि॒शा सा॑दया॒म्यानु॑ष्टुभेन॒ छन्द॑सा मि॒त्रावरु॑णाभ्यां ए॒वत॑या॒ऽग्नेः प॒क्षेणा॒ग्नेः प॒क्षमुप॑ दधाम्यू॒र्द्ध्वया᳚ त्वा दि॒शा सा॑दयामि॒ पाङ्क्ते॑न॒ छन्द॑सा॒ बृह॒स्पति॑ना दे॒वत॑या॒ऽग्नेः पृ॒ष्ठेना॒ग्नेः पृ॒ष्ठमुप॑ दधामि॒ योऽवा अपा᳚त्मानम॒ग्निं चि॑नु॒तेऽपा᳚त्मा॒ऽमुष्मि॑न् ॅलो॒के भ॑वति॒ यः सात्मा॑नं चिनु॒ते ( ) सात्मा॒ऽमुष्मि॑न् ॅलो॒के भ॑वत्यात्मेष्ट॒का उप॑ दधात्ये॒ष वा अ॒ग्नेरा॒त्मा सात्मा॑नमे॒वाग्निं चि॑नुते॒ सात्मा॒ऽमुष्मि॑न् ॅलो॒के भ॑वति॒ य ए॒वं ॅवेद॑ ॥ \newline

\textbf{Pada Paata} \newline

जाग॑तेन । छन्द॑सा । स॒वि॒त्रा । दे॒वत॑या । अ॒ग्नेः । पुच्छे॑न । अ॒ग्नेः । पुच्छ᳚म् । उपेति॑ । द॒धा॒मि॒ । उदी᳚च्या । त्वा॒ । दि॒शा । सा॒द॒या॒मि॒ । आनु॑ष्टुभे॒नेत्यानु॑ - स्तु॒भे॒न॒ । छन्द॑सा । मि॒त्रावरु॑णाभ्या॒मिति॑ मि॒त्रा-वरु॑णाभ्याम् । दे॒वत॑या । अ॒ग्नेः । प॒क्षेण॑ । अ॒ग्नेः । प॒क्षम् । उपेति॑ । द॒धा॒मि॒ । ऊ॒द्‌र्ध्वया᳚ । त्वा॒ । दि॒शा । सा॒द॒या॒मि॒ । पाङ्क्ते॑न । छन्द॑सा । बृह॒स्पति॑ना । दे॒वत॑या । अ॒ग्नेः । पृ॒ष्ठेन॑ । अ॒ग्नेः । पृ॒ष्ठम् । उपेति॑ । द॒धा॒मि॒ । यः । वै । अपा᳚त्मान॒मित्यप॑-आ॒त्मा॒न॒म् । अ॒ग्निम् । चि॒नु॒ते । अपा॒त्मेत्यप॑ - आ॒त्मा॒ । अ॒मुष्मिन्न्॑ । लो॒के । भ॒व॒ति॒ । यः । सात्मा॑न॒मिति॒ स - आ॒त्मा॒न॒म् । चि॒नु॒ते ( ) । सात्मेति॒ स-आ॒त्मा॒ । अ॒मुष्मिन्न्॑ । लो॒के । भ॒व॒ति॒ । आ॒त्मे॒ष्ट॒का इत्या᳚त्म-इ॒ष्ट॒काः । उपेति॑ । द॒धा॒ति॒ । ए॒षः । वै । अ॒ग्नेः । आ॒त्मा । सात्मा॑न॒मिति॒ स-आ॒त्मा॒न॒म् । ए॒व । अ॒ग्निम् । चि॒नु॒ते॒ । सात्मेति॒ स - आ॒त्मा॒ । अ॒मुष्मिन्न्॑ । लो॒के । भ॒व॒ति॒ । यः । ए॒वम् । वेद॑ ॥  \newline




\markright{ TS 5.5.9.1  \hfill https://www.vedavms.in \hfill}

\section{ TS 5.5.9.1 }

\textbf{TS 5.5.9.1 } \newline
\textbf{Samhita Paata} \newline

अग्न॑ उदधे॒ या त॒ इषु॑र्यु॒वा नाम॒ तया॑ नो मृड॒ तस्या᳚स्ते॒ नम॒स्तस्या᳚स्त॒ उप॒ जीव॑न्तो भूया॒स्माग्ने॑ दुद्ध्र गह्य किꣳशिल वन्य॒ या त॒ इषु॑र्यु॒वा नाम॒ तया॑ नो मृड॒ तस्या᳚स्ते॒ नम॒स्तस्या᳚स्त॒ उप॒ जीव॑न्तो भूयास्म॒ पञ्च॒ वा ए॒ते᳚ऽग्नयो॒ यच्चित॑य उद॒धिरे॒व नाम॑ प्रथ॒मो दु॒द्ध्रो - [  ] \newline

\textbf{Pada Paata} \newline

अग्ने᳚ । उ॒द॒ध॒ इत्यु॑द - धे॒ । या । ते॒ । इषुः॑ । यु॒वा । नाम॑ । तया᳚ । नः॒ । मृ॒ड॒ । तस्याः᳚ । ते॒ । नमः॑ । तस्याः᳚ । ते॒ । उपेति॑ । जीव॑न्तः । भू॒या॒स्म॒ । अग्ने᳚ । दु॒द्ध्र॒ । ग॒ह्य॒ । किꣳ॒॒शि॒ल॒ । व॒न्य॒ । या । ते॒ । इषुः॑ । यु॒वा । नाम॑ । तया᳚ । नः॒ । मृ॒ड॒ । तस्याः᳚ । ते॒ । नमः॑ । तस्याः᳚ । ते॒ । उपेति॑ । जीव॑न्तः । भू॒या॒स्म॒ । पञ्च॑ । वै । ए॒ते । अ॒ग्नयः॑ । यत् । चित॑यः । उ॒द॒धिरित्यु॑द - धिः । ए॒व । नाम॑ । प्र॒थ॒मः । दु॒द्ध्रः ।  \newline




\markright{ TS 5.5.9.2  \hfill https://www.vedavms.in \hfill}

\section{ TS 5.5.9.2 }

\textbf{TS 5.5.9.2 } \newline
\textbf{Samhita Paata} \newline

द्वि॒तीयो॒ गह्य॑स्तृ॒तीयः॑ किꣳशि॒लश्च॑तु॒र्थो वन्यः॑ पञ्च॒मस्तेभ्यो॒ यदाहु॑ती॒र्न जु॑हु॒याद॑द्ध्व॒र्युं च॒ यज॑मानं च॒ प्र द॑हेयु॒र्यदे॒ता आहु॑तीर्जु॒होति॑ भाग॒धेये॑नै॒वैना᳚ञ्छमयति॒ नाऽऽ*र्ति॒मार्च्छ॑त्यद्ध्व॒र्युर्न यज॑मानो॒ वाङ्म॑ आ॒सन् न॒सोः प्रा॒णो᳚ऽक्ष्योश्चक्षुः॒ कर्ण॑योः॒ श्रोत्रं॑ बाहु॒वोर्बल॑-मूरु॒वोरोजोऽरि॑ष्टा॒ विश्वा॒न्यङ्गा॑नि त॒नू - [  ] \newline

\textbf{Pada Paata} \newline

द्वि॒तीयः॑ । गह्यः॑ । तृ॒तीयः॑ । किꣳ॒॒शि॒लः । च॒तु॒र्थः । वन्यः॑ । प॒ञ्च॒मः । तेभ्यः॑ । यत् । आहु॑ती॒रित्या - हु॒तीः॒ । न । जु॒हु॒यात् । अ॒द्ध्व॒र्युम् । च॒ । यज॑मानम् । च॒ । प्रेति॑ । द॒हे॒युः॒ । यत् । ए॒ताः । आहु॑ती॒रित्या - हु॒तीः॒ । जु॒होति॑ । भा॒ग॒धेये॒नेति॑ भाग-धेये॑न । ए॒व । ए॒ना॒न् । श॒म॒य॒ति॒ । न । आर्ति᳚म् । एति॑ । ऋ॒च्छ॒ति॒ । अ॒द्ध्व॒र्युः । न । यज॑मानः । वाक् । मे॒ । आ॒सन्न् । न॒सोः । प्रा॒ण इति॑ प्र - अ॒नः । अ॒क्ष्योः । चक्षुः॑ । कर्ण॑योः । श्रोत्र᳚म् । बा॒हु॒वोः । बल᳚म् । ऊ॒रु॒वोः । ओजः॑ । अरि॑ष्टा । विश्वा॑नि । अङ्गा॑नि । त॒नूः ।  \newline




\markright{ TS 5.5.9.3  \hfill https://www.vedavms.in \hfill}

\section{ TS 5.5.9.3 }

\textbf{TS 5.5.9.3 } \newline
\textbf{Samhita Paata} \newline

-स्त॒नुवा॑ मे स॒ह नम॑स्ते अस्तु॒ मा मा॑ हिꣳसी॒रप॒ वा ए॒तस्मा᳚त् प्रा॒णाः क्रा॑मन्ति॒ यो᳚ऽग्निं चि॒न्वन्न॑धि॒ क्राम॑ति॒ वाङ्म॑ आ॒सन् न॒सोः प्रा॒ण इत्या॑ह प्रा॒णाने॒वा*ऽऽत्मन् ध॑त्ते॒ यो रु॒द्रो अ॒ग्नौ यो अ॒फ्सु य ओष॑धीषु॒ यो रु॒द्रो विश्वा॒ भुव॑नाऽऽवि॒वेश॒ तस्मै॑ रु॒द्राय॒ नमो॑ अ॒स्त्वाहु॑तिभागा॒ वा अ॒न्ये रु॒द्रा ह॒विर्भा॑गा - [  ] \newline

\textbf{Pada Paata} \newline

त॒नुवा᳚ । मे॒ । स॒ह । नमः॑ । ते॒ । अ॒स्तु॒ । मा । मा॒ । हिꣳ॒॒सीः॒ । अपेति॑ । वै । ए॒तस्मा᳚त् । प्रा॒णा इति॑ प्र - अ॒नाः । क्रा॒म॒न्ति॒ । यः । अ॒ग्निम् । चि॒न्वन्न् । अ॒धि॒क्राम॒तीत्य॑धि - क्राम॑ति । वाक् । मे॒ । आ॒सन्न् । न॒सोः । प्रा॒ण इति॑ प्र - अ॒नः । इति॑ । आ॒ह॒ । प्रा॒णानिति॑ प्र - अ॒नान् । ए॒व । आ॒त्मन्न् । ध॒त्ते॒ । यः । रु॒द्र ः । अ॒ग्नौ । यः । अ॒फ्स्वित्य॑प् - सु । यः । ओष॑धीषु । यः । रु॒द्रः । विश्वा᳚ । भुव॑ना । आ॒वि॒वेशेत्या᳚ - वि॒वेश॑ । तस्मै᳚ । रु॒द्राय॑ । नमः॑ । अ॒स्तु॒ । आहु॑तिभागा॒ इत्याहु॑ति - भा॒गाः॒ । वै । अ॒न्ये । रु॒द्राः । ह॒विर्भा॑गा॒ इति॑ ह॒विः - भा॒गाः॒ ।  \newline




\markright{ TS 5.5.9.4  \hfill https://www.vedavms.in \hfill}

\section{ TS 5.5.9.4 }

\textbf{TS 5.5.9.4 } \newline
\textbf{Samhita Paata} \newline

अ॒न्ये श॑तरु॒द्रीयꣳ॑ हु॒त्वा गा॑वीधु॒कं च॒रुमे॒तेन॒ यजु॑षा चर॒माया॒मिष्ट॑कायां॒ नि द॑द्ध्याद्-भाग॒धेये॑नै॒वैनꣳ॑ शमयति॒ तस्य॒ त्वै श॑तरु॒द्रीयꣳ॑ हु॒तमित्या॑हु॒र्यस्यै॒तद॒ग्नौ क्रि॒यत॒ इति॒ वस॑वस्त्वा रु॒द्रैः पु॒रस्ता᳚त् पान्तु पि॒तर॑स्त्वा य॒मरा॑जानः पि॒तृभि॑र्दक्षिण॒तः पा᳚न्त्वादि॒त्यास्त्वा॒ विश्वै᳚र्दे॒वैः प॒श्चात् पा᳚न्तु द्युता॒नस्त्वा॑ मारु॒तो म॒रुद्भि॑रुत्तर॒तः पा॑तु - [  ] \newline

\textbf{Pada Paata} \newline

अ॒न्ये । श॒त॒रु॒द्रीय॒मिति॑ शत - रु॒द्रीय᳚म् । हु॒त्वा । गा॒वी॒धु॒कम् । च॒रुम् । ए॒तेन॑ । यजु॑षा । च॒र॒माया᳚म् । इष्ट॑कायाम् । नीति॑ । द॒द्ध्या॒त् । भा॒ग॒धेये॒नेति॑ भाग - धेये॑न । ए॒व । ए॒न॒म् । श॒म॒य॒ति॒ । तस्य॑ । तु । वै । श॒त॒रु॒द्रीय॒मिति॑ शत - रु॒द्रीय᳚म् । हु॒तम् । इति॑ । आ॒हुः॒ । यस्य॑ । ए॒तत् । अ॒ग्नौ । क्रि॒यते᳚ । इति॑ । वस॑वः । त्वा॒ । रु॒द्रैः । पु॒रस्ता᳚त् । पा॒न्तु॒ । पि॒तरः॑ । त्वा॒ । य॒मरा॑जान॒ इति॑ य॒म - रा॒जा॒नः॒ । पि॒तृभि॒रिति॑ पि॒तृ - भिः॒ । द॒क्षि॒ण॒तः । पा॒न्तु॒ । आ॒दि॒त्याः । त्वा॒ । विश्वैः᳚ । दे॒वैः । प॒श्चात् । पा॒न्तु॒ । द्यु॒ता॒नः । त्वा॒ । मा॒रु॒तः । म॒रुद्भि॒रिति॑ म॒रुत् - भिः॒ । उ॒त्त॒र॒त इत्यु॑त् - त॒र॒तः । पा॒तु॒ ।  \newline




\markright{ TS 5.5.9.5  \hfill https://www.vedavms.in \hfill}

\section{ TS 5.5.9.5 }

\textbf{TS 5.5.9.5 } \newline
\textbf{Samhita Paata} \newline

दे॒वास्त्वेन्द्र॑ज्येष्ठा॒ वरु॑णराजानो॒ ऽधस्ता᳚च्चो॒-परि॑ष्ठाच्च पान्तु॒ न वा ए॒तेन॑ पू॒तो न मेद्ध्यो॒ न प्रोक्षि॑तो॒ यदे॑न॒मतः॑ प्रा॒चीनं॑ प्रो॒क्षति॒ यथ् संचि॑त॒माज्ये॑न प्रो॒क्षति॒ तेन॑ पू॒तस्तेन॒ मेद्ध्य॒स्तेन॒ प्रोक्षि॑तः ॥ \newline

\textbf{Pada Paata} \newline

दे॒वाः । त्वा॒ । इन्द्र॑ज्येष्ठा॒ इतीन्द्र॑ - ज्ये॒ष्ठाः॒ । वरु॑णराजान॒ इति॒ वरु॑ण - रा॒जा॒नः॒ । अ॒धस्ता᳚त् । च॒ । उ॒परि॑ष्ठात् । च॒ । पा॒न्तु॒ । न । वै । ए॒तेन॑ । पू॒तः । न । मेद्ध्यः॑ । न । प्रोक्षि॑त॒ इति॒ प्र - उ॒क्षि॒तः॒ । यत् । ए॒न॒म् । अतः॑ । प्रा॒चीन᳚म् । प्रो॒क्षतीति॑ प्र - उ॒क्षति॑ । यथ् । सञ्चि॑त॒मिति॒ सं - चि॒त॒म् । आज्ये॑न । प्रो॒क्षतीति॑ प्र - उ॒क्षति॑ । तेन॑ । पू॒तः । तेन॑ । मेद्ध्यः॑ । तेन॑ । प्रोक्षि॑त॒ इति॒ प्र-उ॒क्षि॒तः॒ ॥  \newline




\markright{ TS 5.5.10.1  \hfill https://www.vedavms.in \hfill}

\section{ TS 5.5.10.1 }

\textbf{TS 5.5.10.1 } \newline
\textbf{Samhita Paata} \newline

स॒॒मीची॒॒ नामा॑सि॒ प्राची॒ दिक्तस्या᳚स्ते॒ ऽग्निरधि॑पति रसि॒तो र॑क्षि॒ता यश्चाधि॑पति॒ र्यश्च॑ गो॒प्ता ताभ्यां॒ नम॒स्तौनो॑ मृडयतां॒ ते यं द्वि॒ष्मो यश्च॑ नो॒ द्वेष्टि॒ तं ॅवां॒ जंभे॑ दधाम्योज॒स्विनी॒ नामा॑सि दक्षि॒णा दिक् तस्या᳚स्त॒ इन्द्रोऽधि॑पतिः॒ पृदा॑कुः॒ प्राची॒ नामा॑सि प्र॒तीची॒ दिक् तस्या᳚स्ते॒ - [  ] \newline

\textbf{Pada Paata} \newline

स॒मीची᳚ । नाम॑ । अ॒सि॒ । प्राची᳚ । दिक् । तस्याः᳚ । ते॒ । अ॒ग्निः । अधि॑पति॒रित्यधि॑ - प॒तिः॒ । अ॒सि॒तः । र॒क्षि॒ता । यः । च॒ । अधि॑पति॒रित्यधि॑-प॒तिः॒ । यः । च॒ । गो॒प्ता । ताभ्या᳚म् । नमः॑ । तौ । नः॒ । मृ॒ड॒य॒ता॒म् । ते । यम् । द्वि॒ष्मः । यः । च॒ । नः॒ । द्वेष्टि॑ । तम् । वा॒म् । जम्भे᳚ । द॒धा॒मि॒ । ओ॒ज॒स्विनी᳚ । नाम॑ । अ॒सि॒ । द॒क्षि॒णा । दिक् । तस्याः᳚ । ते॒ । इन्द्रः॑ । अधि॑पति॒रित्यधि॑ - प॒तिः॒ । पृदा॑कुः । प्राची᳚ । नाम॑ । अ॒सि॒ । प्र॒तीची᳚ । दिक् । तस्याः᳚ । ते॒ ।  \newline




\markright{ TS 5.5.10.2  \hfill https://www.vedavms.in \hfill}

\section{ TS 5.5.10.2 }

\textbf{TS 5.5.10.2 } \newline
\textbf{Samhita Paata} \newline

सोमोऽधि॑पतिः स्व॒जो॑ ऽव॒स्थावा॒ नामा॒-स्युदी॑ची॒ दिक् तस्या᳚स्ते॒ वरु॒णोऽधि॑पति-स्ति॒रश्च॑ राजि॒-रधि॑पत्नी॒ नामा॑सि बृह॒ती दिक् तस्या᳚स्ते॒ बृह॒स्पति॒-रधि॑पतिः श्वि॒त्रो व॒शिनी॒ नामा॑सी॒यं दिक् तस्या᳚स्ते य॒मोऽधि॑पतिः क॒ल्माष॑ ग्रीवो रक्षि॒ता यश्चाधि॑पति॒ र्यश्च॑ गो॒प्ता ताभ्यां॒ नम॒स्तौ नो॑ मृडयतां॒ ते यं द्वि॒ष्मो यश्च॑ - [  ] \newline

\textbf{Pada Paata} \newline

सोमः॑ । अधि॑पति॒रित्यधि॑ - प॒तिः॒ । स्व॒ज इति॑ स्व - जः । अ॒व॒स्थावेत्य॑व - स्थावा᳚ । नाम॑ । अ॒सि॒ । उदी॑ची । दिक् । तस्याः᳚ । ते॒ । वरु॑णः । अधि॑पति॒रित्यधि॑-प॒तिः॒ । ति॒रश्व॑राजि॒रिति॑ ति॒रश्व॑-रा॒जिः॒ । अधि॑प॒त्नीत्यधि॑-प॒त्नी॒ । नाम॑ । अ॒सि॒ । बृ॒ह॒ती । दिक् । तस्याः᳚ । ते॒ । बृह॒स्पतिः॑ । अधि॑पति॒रित्यधि॑ - प॒तिः॒ । श्वि॒त्रः । व॒शिनी᳚ । नाम॑ । अ॒सि॒ । इ॒यं । दिक् । तस्याः᳚ । ते॒ । य॒मः । अधि॑पति॒रित्यधि॑-प॒तिः॒ । क॒ल्माष॑ग्रीव॒ इति॑ क॒ल्माष॑ - ग्री॒वः॒ । र॒क्षि॒ता । यः । च॒ । अधि॑पति॒रित्यधि॑-प॒तिः॒ । यः । च॒ । गो॒प्ता । ताभ्या᳚म् । नमः॑ । तौ । नः॒ । मृ॒ड॒य॒ता॒म् । ते । यम् । द्वि॒ष्मः । यः । च॒ ।  \newline




\markright{ TS 5.5.10.3  \hfill https://www.vedavms.in \hfill}

\section{ TS 5.5.10.3 }

\textbf{TS 5.5.10.3 } \newline
\textbf{Samhita Paata} \newline

नो॒ द्वेष्टि॒ तं ॅवां॒ जंभे॑ दधाम्ये॒ता वै दे॒वता॑ अ॒ग्निं चि॒तꣳ र॑क्षन्ति॒ ताभ्यो॒ यदाहु॑ती॒र्न जु॑हु॒याद॑द्ध्व॒र्युं च॒ यज॑मानं च ध्यायेयु॒र्यदे॒ता आहु॑तीर्जु॒होति॑ भाग॒धेये॑नै-वै॒ना᳚ञ्छमयति॒ नाऽऽ*र्ति॒मार्च्छ॑त्यद्ध्व॒र्युर्न यज॑मानो हे॒तयो॒ नाम॑ स्थ॒ तेषां᳚ ॅवः पु॒रो गृ॒हा अ॒ग्निर्व॒ इष॑वः सलि॒लो नि॑लि॒पां नाम॑ - [  ] \newline

\textbf{Pada Paata} \newline

नः॒ । द्वेष्टि॑ । तम् । वा॒म् । जम्भे᳚ । द॒धा॒मि॒ । ए॒ताः । वै । दे॒वताः᳚ । अ॒ग्निम् । चि॒तम् । र॒क्ष॒न्ति॒ । ताभ्यः॑ । यत् । आहु॑ती॒रित्या - हु॒तीः॒ । न । जु॒हु॒यात् । अ॒द्ध्व॒र्युम् । च॒ । यज॑मानम् । च॒ । ध्या॒ये॒युः॒ । यत् । ए॒ताः । आहु॑ती॒रित्या - हु॒तीः॒ । जु॒होति॑ । भा॒ग॒धेये॒नेति॑ भाग-धेये॑न । ए॒व । ए॒ना॒न् । श॒म॒य॒ति॒ । न । आर्ति᳚म् । एति॑ । ऋ॒च्छ॒ति॒ । अ॒द्ध्व॒र्युः । न । यज॑मानः । हे॒तयः॑ । नाम॑ । स्थ॒ । तेषा᳚म् । वः॒ । पु॒रः । गृ॒हाः । अ॒ग्निः । वः॒ । इष॑वः । स॒लि॒लः । नि॒लि॒पां इति॑ नि-लि॒पांः । नाम॑ ।  \newline




\markright{ TS 5.5.10.4  \hfill https://www.vedavms.in \hfill}

\section{ TS 5.5.10.4 }

\textbf{TS 5.5.10.4 } \newline
\textbf{Samhita Paata} \newline

स्थ॒ तेषां᳚ ॅवो दक्षि॒णा गृ॒हाः पि॒तरो॑ व॒ इष॑वः॒ सग॑रो व॒ज्रिणो॒ नाम॑ स्थ॒ तेषां᳚ ॅवः प॒श्चाद् गृ॒हाः स्वप्नो॑ व॒ इष॑वो॒ गह्व॑रो ऽव॒स्थावा॑नो॒ नाम॑ स्थ॒ तेषां᳚ ॅव उत्त॒राद् गृ॒हा आपो॑ व॒ इष॑वः समु॒द्रो-ऽधि॑पतयो॒ नाम॑ स्थ॒ तेषां᳚ ॅव उ॒परि॑ गृ॒हा व॒र्॒.षं ॅव॒ इष॒वोऽव॑स्वान् क्र॒व्या नाम॑ स्थ॒ पार्त्थि॑वा॒-स्तेषां᳚ ॅव इ॒ह गृ॒हा - [  ] \newline

\textbf{Pada Paata} \newline

स्थ॒ । तेषा᳚म् । वः॒ । द॒क्षि॒णा । गृ॒हाः । पि॒तरः॑ । वः॒ । इष॑वः । सग॑रः । व॒ज्रिणः॑ । नाम॑ । स्थ॒ । तेषा᳚म् । वः॒ । प॒श्चात् । गृ॒हाः । स्वप्नः॑ । वः॒ । इष॑वः । गह्व॑रः । अ॒व॒स्थावा॑न॒ इत्य॑व - स्थावा॑नः । नाम॑ । स्थ॒ । तेषा᳚म् । वः॒ । उ॒त्त॒रादित्यु॑त् - त॒रात् । गृ॒हाः । आपः॑ । वः॒ । इष॑वः । स॒मु॒द्रः । अधि॑पतय॒ इत्यधि॑ - प॒त॒यः॒ । नाम॑ । स्थ॒ । तेषा᳚म् । वः॒ । उ॒परि॑ । गृ॒हाः । व॒र्.॒षम् । वः॒ । इष॑वः । अव॑स्वान् । क्र॒व्याः । नाम॑ । स्थ॒ । पार्थि॑वाः । तेषा᳚म् । वः॒ । इ॒ह । गृ॒हाः ।  \newline




\markright{ TS 5.5.10.5  \hfill https://www.vedavms.in \hfill}

\section{ TS 5.5.10.5 }

\textbf{TS 5.5.10.5 } \newline
\textbf{Samhita Paata} \newline

अन्नं॑ ॅव॒ इष॑वो ऽनिमि॒षो वा॑तना॒मं तेभ्यो॑ वो॒ नम॒स्ते नो॑ मृडयत॒ ते यं द्वि॒ष्मो यश्च॑ नो॒ द्वेष्टि॒ तं ॅवो॒ जंभे॑ दधामि हु॒तादो॒ वा अ॒न्ये दे॒वा अ॑हु॒तादो॒ऽन्ये तान॑ग्नि॒चिदे॒वोभया᳚न् प्रीणाति द॒द्ध्ना म॑धुमि॒श्रेणै॒ता आहु॑तीर्जुहोति भाग॒धेये॑नै॒वैना᳚न् प्रीणा॒त्यथो॒ खल्वा॑हु॒रिष्ट॑का॒ वै दे॒वा अ॑हु॒ताद॒ इत्य॑ - [  ] \newline

\textbf{Pada Paata} \newline

अन्न᳚म् । वः॒ । इष॑वः । नि॒मि॒ष इति॑ नि - मि॒षः । वा॒त॒ना॒ममिति॑ वात - ना॒मम् । तेभ्यः॑ । वः॒ । नमः॑ । ते । नः॒ । मृ॒ड॒य॒त॒ । ते । यम् । द्वि॒ष्मः । यः । च॒ । नः॒ । द्वेष्टि॑ । तम् । वः॒ । जम्भे᳚ । द॒धा॒मि॒ । हु॒ताद॒ इति॑ हुत - अदः॑ । वै । अ॒न्ये । दे॒वाः । अ॒हु॒ताद॒ इत्य॑हुत - अदः॑ । अ॒न्ये । तान् । अ॒ग्नि॒चिदित्य॑ग्नि - चित् । ए॒व । उ॒भयान्॑ । प्री॒णा॒ति॒ । द॒द्ध्ना । म॒धु॒मि॒श्रेणेति॑ मधु - मि॒श्रेण॑ । ए॒ताः । आहु॑ती॒रित्या-हु॒तीः॒ । जु॒हो॒ति॒ । भा॒ग॒धेये॒नेति॑ भाग - धेये॑न । ए॒व । ए॒ना॒न् । प्री॒णा॒ति॒ । अथो॒ इति॑ । खलु॑ । आ॒हुः॒ । इष्ट॑काः । वै । दे॒वाः । अ॒हु॒ताद॒ इत्य॑हुत - अदः॑ । इति॑ ।  \newline




\markright{ TS 5.5.10.6  \hfill https://www.vedavms.in \hfill}

\section{ TS 5.5.10.6 }

\textbf{TS 5.5.10.6 } \newline
\textbf{Samhita Paata} \newline

-नुपरि॒क्रामं॑ जुहो॒त्यप॑रिवर्गमे॒वैना᳚न् प्रीणाती॒मꣳ स्तन॒मूर्ज॑स्वन्तं धया॒पां प्रप्या॑तमग्ने सरि॒रस्य॒ मद्ध्ये᳚ । उथ्सं॑ जुषस्व॒ मधु॑मन्तमूर्व समु॒द्रियꣳ॒॒ सद॑न॒मा वि॑शस्व ॥ यो वा अ॒ग्निं प्र॒युज्य॒ न वि॑मु॒ञ्चति॒ यथाऽश्वो॑ यु॒क्तोऽवि॑मुच्यमानः॒ क्षुद्ध्य॑न् परा॒भव॑त्ये॒वम॑स्या॒ग्निः परा॑ भवति॒ तं प॑रा॒भव॑न्तं॒ ॅयज॑मा॒नोऽनु॒ परा॑ भवति॒ सो᳚ऽग्निं चि॒त्वा लू॒क्षो - [  ] \newline

\textbf{Pada Paata} \newline

अ॒नु॒प॒रि॒क्राम॒मित्य॑नु - प॒रि॒क्राम᳚म् । जु॒हो॒ति॒ । अप॑रिवर्ग॒मित्यप॑रि - व॒र्ग॒म् । ए॒व । ए॒ना॒न् । प्री॒णा॒ति॒ । इ॒मम् । स्तन᳚म् । ऊर्ज॑स्वन्तम् । ध॒य॒ । अ॒पाम् । प्रप्या॑त॒मिति॒ प्र-प्या॒त॒म् । अ॒ग्ने॒ । स॒रि॒रस्य॑ । मद्ध्ये᳚ ॥ उथ्स᳚म् । जु॒ष॒स्व॒ । मधु॑मन्त॒मिति॒ मधु॑-म॒न्त॒म् । ऊ॒र्व॒ । स॒मु॒द्रिय᳚म् । सद॑नम् । एति॑ । वि॒श॒स्व॒ ॥ यः । वै । अ॒ग्निम् । प्र॒युज्येति॑ प्र - युज्य॑ । न । वि॒मु॒ञ्चतीति॑ वि - मु॒ञ्चति॑ । यथा᳚ । अश्वः॑ । यु॒क्तः । अवि॑मुच्यमान॒ इत्यवि॑ - मु॒च्य॒मा॒नः॒ । क्षुद्ध्यन्न्॑ । प॒रा॒भव॒तीति॑ परा - भव॑ति । ए॒वम् । अ॒स्य॒ । अ॒ग्निः । परेति॑ । भ॒व॒ति॒ । तम् । प॒रा॒भव॑न्त॒मिति॑ परा - भव॑न्तम् । यज॑मानः । अनु॑ । परेति॑ । भ॒व॒ति॒ । सः । अ॒ग्निम् । चि॒त्वा । लू॒क्षः ।  \newline




\markright{ TS 5.5.10.7  \hfill https://www.vedavms.in \hfill}

\section{ TS 5.5.10.7 }

\textbf{TS 5.5.10.7 } \newline
\textbf{Samhita Paata} \newline

भ॑वती॒मꣳ स्तन॒मूर्ज॑स्वन्तं धया॒पामित्याज्य॑स्य पू॒र्णाꣳ स्रुचं॑ जुहोत्ये॒ष वा अ॒ग्नेर्वि॑मो॒को वि॒मुच्यै॒वास्मा॒ अन्न॒मपि॑ दधाति॒ तस्मा॑दाहु॒र्यश्चै॒वं ॅवेद॒ यश्च॒ न सु॒धायꣳ॑ ह॒ वै वा॒जी सुहि॑तो दधा॒तीत्य॒ग्निर्वाव वा॒जी तमे॒व तत् प्री॑णाति॒ स ए॑नं प्री॒तः प्री॑णाति॒ वसी॑यान् भवति ( ) ॥ \newline

\textbf{Pada Paata} \newline

भ॒व॒ति॒ । इ॒मम् । स्तन᳚म् । ऊर्ज॑स्वन्तम् । ध॒य॒ । अ॒पाम् । इति॑ । आज्य॑स्य । पू॒र्णाम् । स्रुच᳚म् । जु॒हो॒ति॒ । ए॒षः । वै । अ॒ग्नेः । वि॒मो॒क इति॑ वि - मो॒कः । वि॒मुच्येति॑ वि - मुच्य॑ । ए॒व । अ॒स्मै॒ । अन्न᳚म् । अपीति॑ । द॒धा॒ति॒ । तस्मा᳚त् । आ॒हुः॒ । यः । च॒ । ए॒वम् । वेद॑ । यः । च॒ । न । सु॒धाय॒मिति॑ सु - धाय᳚म् । ह॒ । वै । वा॒जी । सुहि॑त॒ इति॒ सु - हि॒तः॒ । द॒धा॒ति॒ । इति॑ । अ॒ग्निः । वाव । वा॒जी । तम् । ए॒व । तत् । प्री॒णा॒ति॒ । सः । ए॒न॒म् । प्री॒तः । प्री॒णा॒ति॒ । वसी॑यान् । भ॒व॒ति॒ ( ) ॥  \newline




\markright{ TS 5.5.11.1  \hfill https://www.vedavms.in \hfill}

\section{ TS 5.5.11.1 }

\textbf{TS 5.5.11.1 } \newline

\textbf{Pada Paata} \newline

इन्द्रा॑य । राज्ञे᳚ । सू॒क॒रः । वरु॑णाय । राज्ञे᳚ । कृष्णः॑ । य॒माय॑ । राज्ञे᳚ । ऋश्यः॑ । ऋ॒ष॒भाय॑ । राज्ञे᳚ । ग॒व॒यः । शा॒र्दू॒लाय॑ । राज्ञे᳚ । गौ॒रः । पु॒रु॒ष॒रा॒जायेति॑ पुरुष - रा॒जाय॑ । म॒र्कटः॑ । क्षि॒प्र॒श्ये॒नस्येति॑ क्षिप्र - श्ये॒नस्य॑ । वर्ति॑का । नील॑ङ्गोः । क्रिमिः॑ । सोम॑स्य । राज्ञ्ः॑ । कु॒लु॒ङ्गः । सिन्धोः᳚ । शिꣳ॒॒शु॒मारः॑ । हि॒मव॑त॒ इति॑ हि॒म - व॒तः॒ । ह॒स्ती ॥  \newline




\markright{ TS 5.5.12.1  \hfill https://www.vedavms.in \hfill}

\section{ TS 5.5.12.1 }

\textbf{TS 5.5.12.1 } \newline
\textbf{Samhita Paata} \newline

म॒युः प्रा॑जाप॒त्य ऊ॒लो हली᳚क्ष्णो वृषदꣳ॒॒शस्ते धा॒तुः सर॑स्वत्यै॒ शारिः॑ श्ये॒ता पु॑रुष॒वाख् सर॑स्वते॒ शुकः॑ श्ये॒तः पु॑रुष॒वागा॑र॒ण्यो॑ऽजो न॑कु॒लः शका॒ ते पौ॒ष्णा वा॒चे क्रौ॒ञ्चः ॥ \newline

\textbf{Pada Paata} \newline

म॒युः । प्रा॒जा॒प॒त्य इति॑ प्राजा - प॒त्यः । ऊ॒लः । हली᳚क्ष्णः । वृ॒ष॒दꣳ॒॒शः । ते । धा॒तुः । सर॑स्वत्यै । शारिः॑ । श्ये॒ता । पु॒रु॒ष॒वागिति॑ पुरुष-वाक् । सर॑स्वते । शुकः॑ । श्ये॒तः । पु॒रु॒ष॒वागिति॑ पुरुष-वाक् । आ॒र॒ण्यः । अ॒जः । न॒कु॒लः । शका᳚ । ते । पौ॒ष्णाः । वा॒चे । क्रौ॒ञ्चः ॥  \newline




\markright{ TS 5.5.13.1  \hfill https://www.vedavms.in \hfill}

\section{ TS 5.5.13.1 }

\textbf{TS 5.5.13.1 } \newline
\textbf{Samhita Paata} \newline

अ॒पां नप्त्रे॑ ज॒षो ना॒क्रो मक॑रः कुली॒कय॒स्तेऽकू॑पारस्य वा॒चे पै᳚ङ्गरा॒जो भगा॑य कु॒षीत॑क आ॒ती वा॑ह॒सो दर्वि॑दा॒ ते वा॑य॒व्या॑ दि॒ग्भ्यश्च॑क्रवा॒कः ॥ \newline

\textbf{Pada Paata} \newline

अ॒पाम् । नप्त्रे᳚ । ज॒षः । ना॒क्रः । मक॑रः । कु॒ली॒कयः॑ । ते । अकू॑पारस्य । वा॒चे । पै॒ङ्ग॒रा॒ज इति॑ पैङ्ग-रा॒जः । भगा॑य । कु॒षीत॑कः । आ॒ती । वा॒ह॒सः । दर्वि॑दा । ते । वा॒य॒व्याः᳚ । दि॒ग्भ्य इति॑ दिक्-भ्यः । च॒क्र॒वा॒कः ॥  \newline




\markright{ TS 5.5.14.1  \hfill https://www.vedavms.in \hfill}

\section{ TS 5.5.14.1 }

\textbf{TS 5.5.14.1 } \newline
\textbf{Samhita Paata} \newline

बला॑याजग॒र आ॒खुः सृ॑ज॒या श॒यण्ड॑क॒स्ते मै॒त्रा मृ॒त्यवे॑ऽसि॒तो म॒न्यवे᳚ स्व॒जः कुं॑भी॒नसः॑ पुष्करसा॒दो लो॑हिता॒हिस्ते त्वा॒ष्ट्राः प्र॑ति॒श्रुत्का॑यै वाह॒सः ॥ \newline

\textbf{Pada Paata} \newline

बला॑य । अ॒ज॒ग॒रः । आ॒खुः । सृ॒ज॒या । श॒यण्ड॑कः । ते । मै॒त्राः । मृ॒त्यवे᳚ । अ॒सि॒तः । म॒न्यवे᳚ । स्व॒ज इति॑ स्व - जः । कु॒भीं॒नस॒ इति॑ कुंभी - नसः॑ । पु॒ष्क॒र॒सा॒द इति॑ पुष्कर - सा॒दः । लो॒हि॒ता॒हिरिति॑ लोहित-अ॒हिः । ते । त्वा॒ष्ट्राः । प्र॒ति॒श्रुत्का॑या॒ इति॑ प्रति - श्रुत्का॑यै । वा॒ह॒सः ॥  \newline




\markright{ TS 5.5.15.1  \hfill https://www.vedavms.in \hfill}

\section{ TS 5.5.15.1 }

\textbf{TS 5.5.15.1 } \newline
\textbf{Samhita Paata} \newline

पु॒रु॒ष॒मृ॒गश्च॒न्द्रम॑से गो॒धा काल॑का दार्वाघा॒टस्ते वन॒स्पती॑नामे॒ण्यह्ने॒ कृष्णो॒ रात्रि॑यै पि॒कः क्ष्विङ्का॒ नील॑शीर्ष्णी॒ ते᳚ऽर्य॒म्णे धा॒तुः क॑त्क॒टः ॥ \newline

\textbf{Pada Paata} \newline

पु॒रु॒ष॒मृ॒ग इति॑ पुरुष - मृ॒गः । च॒न्द्रम॑से । गो॒धा । काल॑का । दा॒र्वा॒घा॒ट इति॑ दारु-आ॒घा॒तः । ते । वन॒स्पती॑नाम् । ए॒णी । अह्ने᳚ । कृष्णः॑ । रात्रि॑यै । पि॒कः । क्ष्विङ्काः᳚ । नील॑शी॒र्ष्णीति॒ नील॑ - शी॒र्ष्णी॒ । ते । अ॒र्य॒म्णे । धा॒तुः । क॒त्क॒टः ॥  \newline




\markright{ TS 5.5.16.1  \hfill https://www.vedavms.in \hfill}

\section{ TS 5.5.16.1 }

\textbf{TS 5.5.16.1 } \newline
\textbf{Samhita Paata} \newline

सौ॒री ब॒लाकर्श्यो॑ म॒यूरः॑ श्ये॒नस्ते ग॑न्ध॒र्वाणां॒ ॅवसू॑नां क॒पिञ्ज॑लो रु॒द्राणां᳚ तित्ति॒री रो॒हित् कु॑ण्डृ॒णाची॑ गो॒लत्ति॑का॒ ता अ॑फ्स॒रसा॒-मर॑ण्याय सृम॒रः ॥ \newline

\textbf{Pada Paata} \newline

सौ॒री । ब॒लाका᳚ । ऋश्यः॑ । म॒यूरः॑ । श्ये॒नः । ते । ग॒न्ध॒र्वाणा᳚म् । वसू॑नाम् । क॒पिञ्ज॑लः । रु॒द्राणा᳚म् । ति॒त्ति॒रिः । रो॒हित् । कु॒ण्डृ॒णाची᳚ । गो॒लत्ति॑का । ताः । अ॒फ्स॒रसा᳚म् । अर॑ण्याय । सृ॒म॒रः ॥  \newline




\markright{ TS 5.5.17.1  \hfill https://www.vedavms.in \hfill}

\section{ TS 5.5.17.1 }

\textbf{TS 5.5.17.1 } \newline
\textbf{Samhita Paata} \newline

पृ॒ष॒तो वै᳚श्वदे॒वः पि॒त्वो न्यङ्कुः॒ कश॒स्तेऽनु॑मत्या अन्यवा॒पो᳚ऽर्द्धमा॒सानां᳚ मा॒सां क॒श्यपः॒ क्वयिः॑ कु॒टरु॑र्दात्यौ॒हस्ते सि॑नीवा॒ल्यै बृह॒स्पत॑ये शित्पु॒टः ॥ \newline

\textbf{Pada Paata} \newline

पृ॒ष॒तः । वै॒श्व॒दे॒व इति॑ वैश्व - दे॒वः । पि॒त्वः । न्यङ्कुः॑ । कशः॑ । ते । अनु॑मत्या॒ इत्यनु॑ - म॒त्यै॒ । अ॒न्य॒वा॒प इत्य॑न्य - वा॒पः । अ॒द्‌र्ध॒मा॒साना॒मित्य॑द्‌र्ध - मा॒साना᳚म् । मा॒साम् । क॒श्यपः॑ । क्वयिः॑ । कु॒टरुः॑ । दा॒त्यौ॒हः । ते । सि॒नी॒वा॒ल्यै । बृह॒स्पत॑ये । शि॒त्पु॒टः ॥  \newline




\markright{ TS 5.5.18.1  \hfill https://www.vedavms.in \hfill}

\section{ TS 5.5.18.1 }

\textbf{TS 5.5.18.1 } \newline
\textbf{Samhita Paata} \newline

शका॑ भौ॒मी पा॒न्त्रः कशो॑ मान्थी॒लव॒स्ते पि॑तृ॒णामृ॑तू॒नां जह॑का संॅवथ्स॒राय॒ लोपा॑ क॒पोत॒ उलू॑कः श॒शस्ते न॑र्.ऋ॒ताः कृ॑क॒वाकुः॑ सावि॒त्रः ॥ \newline

\textbf{Pada Paata} \newline

शका᳚ । भौ॒मी । पा॒न्त्रः । कशः॑ । मा॒न्थी॒लवः॑ । ते । पि॒तृ॒णाम् । ऋ॒तू॒नाम् । जह॑का । सं॒ॅव॒थ्स॒रायेति॑ सं-व॒थ्स॒राय॑ । लोपा᳚ । क॒पोतः॑ । उलू॑कः । श॒शः । ते । नै॒र्.॒ऋ॒ता इति॑ नैः - ऋ॒ताः । कृ॒क॒वाकुः॑ । सा॒वि॒त्रः ॥  \newline




\markright{ TS 5.5.19.1  \hfill https://www.vedavms.in \hfill}

\section{ TS 5.5.19.1 }

\textbf{TS 5.5.19.1 } \newline
\textbf{Samhita Paata} \newline

रुरू॑ रौ॒द्रः कृ॑कला॒सः श॒कुनिः॒ पिप्प॑का॒ ते श॑र॒व्या॑यै हरि॒णो मा॑रु॒तो ब्रह्म॑णे शा॒र्गस्त॒रक्षुः॑ कृ॒ष्णः श्वा च॑तुर॒क्षो ग॑र्द॒भस्त इ॑तरज॒नाना॑म॒ग्नये॒ धूङ्क्ष्णा᳚ ॥ \newline

\textbf{Pada Paata} \newline

रुरुः॑ । रौ॒द्रः । कृ॒क॒ला॒सः । श॒कुनिः॑ । पिप्प॑का । ते । श॒र॒व्या॑यै । ह॒रि॒णः । मा॒रु॒तः । ब्रह्म॑णे । शा॒र्गः । त॒रक्षुः॑ । कृ॒ष्णः । श्वा । च॒तु॒र॒क्ष इति॑ चतुः - अ॒क्षः । ग॒र्द॒भः । ते । इ॒त॒र॒ज॒नाना॒मिती॑तर-ज॒नाना᳚म् । अ॒ग्नये᳚ । धूंक्ष्णा᳚ ॥  \newline




\markright{ TS 5.5.20.1  \hfill https://www.vedavms.in \hfill}

\section{ TS 5.5.20.1 }

\textbf{TS 5.5.20.1 } \newline
\textbf{Samhita Paata} \newline

अ॒ल॒ज आ᳚न्तरि॒क्ष उ॒द्रो म॒द्गुः प्ल॒वस्ते॑ऽपामदि॑त्यै हꣳस॒साचि॑रिन्द्रा॒ण्यै कीर्.शा॒ गृद्ध्रः॑ शितिक॒क्षी वा᳚र्द्ध्राण॒सस्ते दि॒व्या द्या॑वापृथि॒व्या᳚ श्वा॒वित् ॥ \newline

\textbf{Pada Paata} \newline

अ॒ल॒जः । आ॒न्त॒रि॒क्षः । उ॒द्रः । म॒द्गुः । प्ल॒वः । ते । अ॒पाम् । अदि॑त्यै । हꣳ॒॒स॒साचि॒रिति॑ हꣳस - साचिः॑ । इ॒न्द्रा॒ण्यै । कीर्.शा᳚ । गृध्रः॑ । शि॒ति॒क॒क्षीति॑ शिति - क॒क्षी । वा॒द्‌र्ध्रा॒ण॒सः । ते । दि॒व्याः । द्या॒वा॒पृ॒थि॒व्येति॑ द्यावा - पृ॒थि॒व्या᳚ । श्वा॒विदिति॑ श्व - वित् ॥  \newline




\markright{ TS 5.5.21.1  \hfill https://www.vedavms.in \hfill}

\section{ TS 5.5.21.1 }

\textbf{TS 5.5.21.1 } \newline
\textbf{Samhita Paata} \newline

सु॒प॒र्णः पा᳚र्ज॒न्यो हꣳ॒॒सो वृको॑ वृषदꣳ॒॒शस्त ऐ॒न्द्रा अ॒पामु॒द्रो᳚ ऽर्य॒म्णे लो॑पा॒शः सिꣳ॒॒हो न॑कु॒लो व्या॒घ्रस्ते म॑हे॒न्द्राय॒ कामा॑य॒ पर॑स्वान् ॥ \newline

\textbf{Pada Paata} \newline

सु॒प॒र्ण इति॑ सु - प॒र्णः । पा॒र्ज॒न्यः । हꣳ॒॒सः । वृकः॑ । वृ॒ष॒दꣳ॒॒शः । ते । ऐ॒न्द्राः । अ॒पाम् । उ॒द्रः । अ॒र्य॒म्णे । लो॒पा॒शः । सिꣳ॒॒हः । न॒कु॒लः । व्या॒घ्रः । ते । म॒हे॒न्द्रायेति॑ महा - इ॒न्द्राय॑ । कामा॑य । पर॑स्वान् ॥  \newline




\markright{ TS 5.5.22.1  \hfill https://www.vedavms.in \hfill}

\section{ TS 5.5.22.1 }

\textbf{TS 5.5.22.1 } \newline
\textbf{Samhita Paata} \newline

आ॒ग्ने॒यः कृ॒ष्णग्री॑वः सारस्व॒ती मे॒षी ब॒भ्रुः सौ॒म्यः पौ॒ष्णः श्या॒मः शि॑तिपृ॒ष्ठो बा॑र्.हस्प॒त्यः शि॒ल्पो वै᳚श्वदे॒व ऐ॒न्द्रो॑ऽरु॒णो मा॑रु॒तः क॒ल्माष॑ ऐन्द्रा॒ग्नः सꣳ॑हि॒तो॑ ऽधोरा॑मः सावि॒त्रो वा॑रु॒णः पेत्वः॑ ॥ \newline

\textbf{Pada Paata} \newline

आ॒ग्ने॒यः । कृ॒ष्णग्री॑व॒ इति॑ कृ॒ष्ण - ग्री॒वः॒ । सा॒र॒स्व॒ती । मे॒षी । ब॒भ्रुः । सौ॒म्यः । पौ॒ष्णः । श्या॒मः । शि॒ति॒पृ॒ष्ठ इति॑ शिति -पृ॒ष्ठः । बा॒र्.॒ह॒स्प॒त्यः । शि॒ल्पः । वै॒श्व॒दे॒व इति॑ वैश्व - दे॒वः । ऐ॒न्द्रः । अ॒रु॒णः । मा॒रु॒तः । क॒ल्माषः॑ । ऐ॒न्द्रा॒ग्न इत्यै᳚न्द्र - अ॒ग्नः । सꣳ॒॒हि॒त इति॑ सं - हि॒तः । अ॒धोरा॑म॒ इत्य॒धः - रा॒मः॒ । सा॒वि॒त्रः । वा॒रु॒णः । पेत्वः॑ ॥  \newline




\markright{ TS 5.5.23.1  \hfill https://www.vedavms.in \hfill}

\section{ TS 5.5.23.1 }

\textbf{TS 5.5.23.1 } \newline
\textbf{Samhita Paata} \newline

अश्व॑स्तूप॒रो गो॑मृ॒गस्ते प्रा॑जाप॒त्या आ᳚ग्ने॒यौ कृ॒ष्णग्री॑वौ त्वा॒ष्ट्रौ लो॑मशस॒क्थौ शि॑तिपृ॒ष्ठौ बा॑र्.हस्प॒त्यौ धा॒त्रे पृ॑षोद॒रः सौ॒र्यो ब॒लक्षः॒ पेत्वः॑ ॥ \newline

\textbf{Pada Paata} \newline

अश्वः॑ । तू॒प॒रः । गो॒मृ॒ग इति॑ गो - मृ॒गः । ते । प्रा॒जा॒प॒त्या इति॑ प्राजा - प॒त्याः । आ॒ग्ने॒यौ । कृ॒ष्णग्री॑वा॒विति॑ कृ॒ष्ण - ग्री॒वौ॒ । त्वा॒ष्ट्रौ । लो॒म॒श॒स॒क्थाविति॑ लोमश - स॒क्थौ । शि॒ति॒पृ॒ष्ठाविति॑ शिति - पृ॒ष्ठौ । बा॒र्.॒ह॒स्प॒त्यौ । धा॒त्रे । पृ॒षो॒द॒र इति॑ पृष - उ॒द॒रः । सौ॒र्यः । ब॒लक्षः॑ । पेत्वः॑ ॥  \newline




\markright{ TS 5.5.24.1  \hfill https://www.vedavms.in \hfill}

\section{ TS 5.5.24.1 }

\textbf{TS 5.5.24.1 } \newline
\textbf{Samhita Paata} \newline

अ॒ग्नयेऽनी॑कवते॒ रोहि॑ताञ्जिरन॒ड्वान॒धोरा॑मौ सावि॒त्रौ पौ॒ष्णौ र॑ज॒तना॑भी वैश्वदे॒वौ पि॒शङ्गौ॑ तूप॒रौ मा॑रु॒तः क॒ल्माष॑ आग्ने॒यः कृ॒ष्णो॑ऽजः सा॑रस्व॒ती मे॒षी वा॑रु॒णः कृ॒ष्ण एक॑शितिपा॒त् पेत्वः॑ ॥ \newline

\textbf{Pada Paata} \newline

अ॒ग्नये᳚ । अनी॑कवत॒ इत्यनी॑क-व॒ते॒ । रोहि॑ताञ्जि॒रिति॒ रोहि॑त-अ॒ञ्जिः॒ । अ॒न॒ड्वान् । अ॒धोरा॑मा॒वित्य॒धः - रा॒मौ॒ । सा॒वि॒त्रौ । पौ॒ष्णौ । र॒ज॒तना॑भी॒ इति॑ रज॒त - ना॒भी॒ । वै॒श्व॒दे॒वाविति॑ वैश्व-दे॒वौ । पि॒शङ्गौ᳚ । तू॒प॒रौ । मा॒रु॒तः । क॒ल्माषः॑ । आ॒ग्ने॒यः । कृ॒ष्णः । अ॒जः । सा॒र॒स्व॒ती । मे॒षी । वा॒रु॒णः । कृ॒ष्णः । एक॑शितिपा॒दित्येक॑-शि॒ति॒पा॒त् । पेत्वः॑ ॥  \newline






\end{document}