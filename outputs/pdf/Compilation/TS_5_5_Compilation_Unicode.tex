\documentclass[17pt]{extarticle}
\usepackage{babel}
\usepackage{fontspec}
\usepackage{polyglossia}
\usepackage{extsizes}

\usepackage{color}   %May be necessary if you want to color links
\usepackage{hyperref}
\hypersetup{
    colorlinks=true, %set true if you want colored links
    linktoc=all,     %set to all if you want both sections and subsections linked
    linkcolor=black,  %choose some color if you want links to stand out
}

\setmainlanguage{sanskrit}
\setotherlanguages{english} %% or other languages
\setlength{\parindent}{0pt}
\pagestyle{myheadings}
\newfontfamily\devanagarifont[Script=Devanagari]{AdishilaVedic}
\renewcommand{\theHsection}{\thepart.section.\thesection}

\newcommand{\VAR}[1]{}
\newcommand{\BLOCK}[1]{}




\begin{document}
\begin{titlepage}
    \begin{center}
 
\begin{sanskrit}
    { \Large
    कृष्ण यजुर्वेदीय तैत्तिरीय संहिता,पद,जटा,घन पाठः 
    }
    \\
    \vspace{2.5cm}
    \mbox{ \Large
    5.5      पञ्चमकाण्डे पञ्चमः प्रश्नः - वायव्यपश्वाद्यानं निरूपणं   }
\end{sanskrit}
\end{center}

\end{titlepage}
\tableofcontents
\phantomsection
\pagebreak

\markright{ TS 5.5.1.1  \hfill https://www.vedavms.in \hfill}

\section{ TS 5.5.1.1 }

\textbf{TS 5.5.1.1 } \newline
\textbf{Samhita Paata} \newline

यदेके॑न सꣳ स्था॒पय॑ति य॒ज्ञ्स्य॒ संत॑त्या॒ अवि॑च्छेदायै॒न्द्राः प॒शवो॒ ये मु॑ष्क॒रा यदै॒न्द्राः सन्तो॒ऽग्निभ्य॑ आल॒भ्यन्ते॑ दे॒वता᳚भ्यः स॒मदं॑ दधात्याग्ने॒यीस्त्रि॒ष्टुभो॑ याज्यानुवा॒क्याः᳚ कुर्या॒द्-यदा᳚ग्ने॒यीस्तेना᳚ ऽऽ*ग्ने॒या यत् त्रि॒ष्टुभ॒स्तेनै॒न्द्राः समृ॑द्ध्यै॒ न दे॒वता᳚भ्यः स॒मदं॑ दधाति वा॒यवे॑ नि॒युत्व॑ते तूप॒रमा ल॑भते॒ तेजो॒ऽग्नेर्वा॒युस्तेज॑स ए॒ष आ ल॑भ्यते॒ तस्मा᳚द्-य॒द्रिय॑ङ् वा॒यु - [  ] \newline

\textbf{Pada Paata} \newline

यत् । एके॑न । सꣳ॒॒स्था॒पय॒तीति॑ सं - स्था॒पय॑ति । य॒ज्ञ्स्य॑ । सन्त॑त्या॒ इति॒ सं - त॒त्यै॒ । अवि॑च्छेदा॒येत्यवि॑ - छे॒दा॒य॒ । ऐ॒न्द्राः । प॒शवः॑ । ये । मु॒ष्क॒राः । यत् । ऐ॒न्द्राः । सन्तः॑ । अ॒ग्निभ्य॒ इत्य॒ग्नि - भ्यः॒ । आ॒ल॒भ्यन्त॒ इत्या᳚ - ल॒भ्यन्ते᳚ । दे॒वता᳚भ्यः । स॒मद॒मिति॑ स - मद᳚म् । द॒धा॒ति॒ । आ॒ग्ने॒यीः । त्रि॒ष्टुभः॑ । या॒ज्या॒नु॒वा॒क्या॑ इति॑ याज्या - अ॒नु॒वा॒क्याः᳚ । कु॒र्या॒त् । यत् । आ॒ग्ने॒यीः । तेन॑ । आ॒ग्ने॒याः । यत् । त्रि॒ष्टुभः॑ । तेन॑ । ऐ॒न्द्राः । समृ॑द्ध्या॒ इति॒ सं - ऋ॒द्ध्यै॒ । न । दे॒वता᳚भ्यः । स॒मद॒मिति॑ स - मद᳚म् । द॒धा॒ति॒ । वा॒यवे᳚ । नि॒युत्व॑त॒ इति॑ नि-युत्व॑ते । तू॒प॒रम् । एति॑ । ल॒भ॒ते॒ । तेजः॑ । अ॒ग्नेः । वा॒युः । तेज॑से । ए॒षः । एति॑ । ल॒भ्य॒ते॒ । तस्मा᳚त् । य॒द्रियङ्॑ । वा॒युः ।  \newline


\textbf{Krama Paata} \newline

यदेके॑न । एके॑न सꣳस्था॒पय॑ति । सꣳ॒॒स्था॒पय॑ति य॒ज्ञ्स्य॑ । सꣳ॒॒स्था॒पय॒तीति॑ सम् - स्था॒पय॑ति । य॒ज्ञ्स्य॒ सन्त॑त्यै । सन्त॑त्या॒ अवि॑च्छेदाय । सन्त॑त्या॒ इति॒ सम् - त॒त्यै॒ । अवि॑च्छेदायै॒न्द्राः । अवि॑च्छेदा॒येत्यवि॑ - छे॒दा॒य॒ । ऐ॒न्द्राः प॒शवः॑ । प॒शवो॒ ये । 
ये मु॑ष्क॒राः । मु॒ष्क॒रा यत् । यदै॒न्द्राः । ऐ॒न्द्राः सन्तः॑ । सन्तो॒ऽग्निभ्यः॑ । अ॒ग्निभ्य॑ आल॒भ्यन्ते᳚ । अ॒ग्निभ्य॒ इत्य॒ग्नि - भ्यः॒ । आ॒ल॒भ्यन्ते॑ दे॒वता᳚भ्यः । आ॒ल॒भ्यन्त॒ इत्या᳚ - ल॒भ्यन्ते᳚ । दे॒वता᳚भ्यः स॒मद᳚म् । स॒मद॑म् दधाति । स॒मद॒मिति॑ स - मद᳚म् । द॒धा॒त्या॒ग्ने॒यीः । आ॒ग्ने॒यीस्त्रि॒ष्टुभः॑ । त्रि॒ष्टुभो॑ याज्यानुवा॒क्याः᳚ । या॒ज्या॒नु॒वा॒क्याः᳚ कुर्यात् । या॒ज्या॒नु॒वा॒क्या॑ इति॑ याज्या - अ॒नु॒वा॒क्याः᳚ । कु॒र्या॒द् यत् । यदा᳚ग्ने॒यीः । आ॒ग्ने॒यीस्तेन॑ । तेना᳚ग्ने॒याः । आ॒ग्ने॒या यत् । यत् त्रि॒ष्टुभः॑ । त्रि॒ष्टुभ॒स्तेन॑ । तेनै॒न्द्राः । ऐ॒न्द्राः समृ॑द्ध्यै । समृ॑द्ध्यै॒ न । समृ॑द्ध्या॒ इति॒ सम् - ऋ॒द्ध्यै॒ । न दे॒वता᳚भ्यः । दे॒वता᳚भ्यः स॒मद᳚म् । स॒मद॑म् दधाति । स॒मद॒मिति॑ स - मद᳚म् । द॒धा॒ति॒ वा॒यवे᳚ । वा॒यवे॑ नि॒युत्व॑ते । नि॒युत्व॑ते तूप॒रम् । नि॒युत्व॑त॒ इति॑ नि - युत्व॑ते । तू॒प॒रमा । आ ल॑भते । ल॒भ॒ते॒ तेजः॑ । तेजो॒ऽग्नेः । अ॒ग्नेर् वा॒युः । वा॒युस्तेज॑से । तेज॑स ए॒षः । ए॒ष आ । आ ल॑भ्यते । ल॒भ्य॒ते॒ तस्मा᳚त् । तस्मा᳚द् य॒द्रियङ्ङ्॑ । य॒द्रिय॑ङ्. वा॒युः । वा॒युर् वाति॑ \newline

\textbf{Jatai Paata} \newline

1. यदेके॒ नैके॑न॒ यद् यदेके॑न । \newline
2. एके॑न सꣳस्था॒पय॑ति सꣳस्था॒पय॒ त्येके॒ नैके॑न 
सꣳस्था॒पय॑ति । \newline
3. सꣳ॒॒स्था॒पय॑ति य॒ज्ञ्स्य॑ य॒ज्ञ्स्य॑ सꣳस्था॒पय॑ति सꣳस्था॒पय॑ति य॒ज्ञ्स्य॑ । \newline
4. सꣳ॒॒स्था॒पय॒तीति॑ सं - स्था॒पय॑ति । \newline
5. य॒ज्ञ्स्य॒ सन्त॑त्यै॒ सन्त॑त्यै य॒ज्ञ्स्य॑ य॒ज्ञ्स्य॒ सन्त॑त्यै । \newline
6. सन्त॑त्या॒ अवि॑च्छेदा॒या वि॑च्छेदाय॒ सन्त॑त्यै॒ सन्त॑त्या॒ अवि॑च्छेदाय । \newline
7. सन्त॑त्या॒ इति॒ सं - त॒त्यै॒ । \newline
8. अवि॑च्छेदा यै॒न्द्रा ऐ॒न्द्रा अवि॑च्छेदा॒या वि॑च्छेदा यै॒न्द्राः । \newline
9. अवि॑च्छेदा॒येत्यवि॑ - छे॒दा॒य॒ । \newline
10. ऐ॒न्द्राः प॒शवः॑ प॒शव॑ ऐ॒न्द्रा ऐ॒न्द्राः प॒शवः॑ । \newline
11. प॒शवो॒ ये ये प॒शवः॑ प॒शवो॒ ये । \newline
12. ये मु॑ष्क॒रा मु॑ष्क॒रा ये ये मु॑ष्क॒राः । \newline
13. मु॒ष्क॒रा यद् यन् मु॑ष्क॒रा मु॑ष्क॒रा यत् । \newline
14. यदै॒न्द्रा ऐ॒न्द्रा यद् यदै॒न्द्राः । \newline
15. ऐ॒न्द्राः सन्तः॒ सन्त॑ ऐ॒न्द्रा ऐ॒न्द्राः सन्तः॑ । \newline
16. सन्तो॒ ऽग्निभ्यो॒ ऽग्निभ्यः॒ सन्तः॒ सन्तो॒ ऽग्निभ्यः॑ । \newline
17. अ॒ग्निभ्य॑ आल॒भ्यन्त॑ आल॒भ्यन्ते॒ ऽग्निभ्यो॒ ऽग्निभ्य॑ आल॒भ्यन्ते᳚ । \newline
18. अ॒ग्निभ्य॒ इत्य॒ग्नि - भ्यः॒ । \newline
19. आ॒ल॒भ्यन्ते॑ दे॒वता᳚भ्यो दे॒वता᳚भ्य आल॒भ्यन्त॑ आल॒भ्यन्ते॑ दे॒वता᳚भ्यः । \newline
20. आ॒ल॒भ्यन्त॒ इत्या᳚ - ल॒भ्यन्ते᳚ । \newline
21. दे॒वता᳚भ्यः स॒मदꣳ॑ स॒मद॑म् दे॒वता᳚भ्यो दे॒वता᳚भ्यः स॒मद᳚म् । \newline
22. स॒मद॑म् दधाति दधाति स॒मदꣳ॑ स॒मद॑म् दधाति । \newline
23. स॒मद॒मिति॑ स - मद᳚म् । \newline
24. द॒धा॒ त्या॒ग्ने॒यी रा᳚ग्ने॒यीर् द॑धाति दधा त्याग्ने॒यीः । \newline
25. आ॒ग्ने॒यी स्त्रि॒ष्टुभ॑ स्त्रि॒ष्टुभ॑ आग्ने॒यी रा᳚ग्ने॒यी स्त्रि॒ष्टुभः॑ । \newline
26. त्रि॒ष्टुभो॑ याज्यानुवा॒क्या॑ याज्यानुवा॒क्या᳚ स्त्रि॒ष्टुभ॑ स्त्रि॒ष्टुभो॑ याज्यानुवा॒क्याः᳚ । \newline
27. या॒ज्या॒नु॒वा॒क्याः᳚ कुर्यात् कुर्याद् याज्यानुवा॒क्या॑ याज्यानुवा॒क्याः᳚ कुर्यात् । \newline
28. या॒ज्या॒नु॒वा॒क्या॑ इति॑ याज्या - अ॒नु॒वा॒क्याः᳚ । \newline
29. कु॒र्या॒द् यद् यत् कु॑र्यात् कुर्या॒द् यत् । \newline
30. यदा᳚ग्ने॒यी रा᳚ग्ने॒यीर् यद् यदा᳚ग्ने॒यीः । \newline
31. आ॒ग्ने॒यी स्तेन॒ तेना᳚ग्ने॒यी रा᳚ग्ने॒यी स्तेन॑ । \newline
32. तेना᳚ ग्ने॒या आ᳚ग्ने॒या स्तेन॒ तेना᳚ ग्ने॒याः । \newline
33. आ॒ग्ने॒या यद् यदा᳚ ग्ने॒या आ᳚ग्ने॒या यत् । \newline
34. यत् त्रि॒ष्टुभ॑ स्त्रि॒ष्टुभो॒ यद् यत् त्रि॒ष्टुभः॑ । \newline
35. त्रि॒ष्टुभ॒ स्तेन॒ तेन॑ त्रि॒ष्टुभ॑ स्त्रि॒ष्टुभ॒ स्तेन॑ । \newline
36. तेनै॒न्द्रा ऐ॒न्द्रा स्तेन॒ तेनै॒न्द्राः । \newline
37. ऐ॒न्द्राः समृ॑द्ध्यै॒ समृ॑द्ध्या ऐ॒न्द्रा ऐ॒न्द्राः समृ॑द्ध्यै । \newline
38. समृ॑द्ध्यै॒ न न समृ॑द्ध्यै॒ समृ॑द्ध्यै॒ न । \newline
39. समृ॑द्ध्या॒ इति॒ सं - ऋ॒द्ध्यै॒ । \newline
40. न दे॒वता᳚भ्यो दे॒वता᳚भ्यो॒ न न दे॒वता᳚भ्यः । \newline
41. दे॒वता᳚भ्यः स॒मदꣳ॑ स॒मद॑म् दे॒वता᳚भ्यो दे॒वता᳚भ्यः स॒मद᳚म् । \newline
42. स॒मद॑म् दधाति दधाति स॒मदꣳ॑ स॒मद॑म् दधाति । \newline
43. स॒मद॒मिति॑ स - मद᳚म् । \newline
44. द॒धा॒ति॒ वा॒यवे॑ वा॒यवे॑ दधाति दधाति वा॒यवे᳚ । \newline
45. वा॒यवे॑ नि॒युत्व॑ते नि॒युत्व॑ते वा॒यवे॑ वा॒यवे॑ नि॒युत्व॑ते । \newline
46. नि॒युत्व॑ते तूप॒रम् तू॑प॒रन्नि॒युत्व॑ते नि॒युत्व॑ते तूप॒रम् । \newline
47. नि॒युत्व॑त॒ इति॑ नि - युत्व॑ते । \newline
48. तू॒प॒र मा तू॑प॒रम् तू॑प॒र मा । \newline
49. आ ल॑भते लभत॒ आ ल॑भते । \newline
50. ल॒भ॒ते॒ तेज॒ स्तेजो॑ लभते लभते॒ तेजः॑ । \newline
51. तेजो॒ ऽग्ने र॒ग्ने स्तेज॒ स्तेजो॒ ऽग्नेः । \newline
52. अ॒ग्नेर् वा॒युर् वा॒यु र॒ग्ने र॒ग्नेर् वा॒युः । \newline
53. वा॒यु स्तेज॑से॒ तेज॑से वा॒युर् वा॒यु स्तेज॑से । \newline
54. तेज॑स ए॒ष ए॒ष तेज॑से॒ तेज॑स ए॒षः । \newline
55. ए॒ष ऐष ए॒ष आ । \newline
56. आ ल॑भ्यते लभ्यत॒ आ ल॑भ्यते । \newline
57. ल॒भ्य॒ते॒ तस्मा॒त् तस्मा᳚ल् लभ्यते लभ्यते॒ तस्मा᳚त् । \newline
58. तस्मा᳚द् य॒द्रिय॑ङ् य॒द्रिय॒ङ् तस्मा॒त् तस्मा᳚द् य॒द्रियङ्॑ । \newline
59. य॒द्रिय॑ङ्. वा॒युर् वा॒युर् य॒द्रिय॑ङ् य॒द्रिय॑ङ्. वा॒युः । \newline
60. वा॒युर् वाति॒ वाति॑ वा॒युर् वा॒युर् वाति॑ । \newline

\textbf{Ghana Paata } \newline

1. यदेके॒ नैके॑न॒ यद् यदेके॑न सꣳस्था॒पय॑ति सꣳस्था॒पय॒ त्येके॑न॒ यद् यदेके॑न सꣳस्था॒पय॑ति । \newline
2. एके॑न सꣳस्था॒पय॑ति सꣳस्था॒पय॒ त्येके॒ नैके॑न सꣳस्था॒पय॑ति य॒ज्ञ्स्य॑ य॒ज्ञ्स्य॑ सꣳस्था॒पय॒ त्येके॒ नैके॑न सꣳस्था॒पय॑ति य॒ज्ञ्स्य॑ । \newline
3. सꣳ॒॒स्था॒पय॑ति य॒ज्ञ्स्य॑ य॒ज्ञ्स्य॑ सꣳस्था॒पय॑ति सꣳस्था॒पय॑ति य॒ज्ञ्स्य॒ सन्त॑त्यै॒ सन्त॑त्यै य॒ज्ञ्स्य॑ सꣳस्था॒पय॑ति सꣳस्था॒पय॑ति य॒ज्ञ्स्य॒ सन्त॑त्यै । \newline
4. सꣳ॒॒स्था॒पय॒तीति॑ सं - स्था॒पय॑ति । \newline
5. य॒ज्ञ्स्य॒ सन्त॑त्यै॒ सन्त॑त्यै य॒ज्ञ्स्य॑ य॒ज्ञ्स्य॒ सन्त॑त्या॒ अवि॑च्छेदा॒या वि॑च्छेदाय॒ सन्त॑त्यै य॒ज्ञ्स्य॑ य॒ज्ञ्स्य॒ सन्त॑त्या॒ अवि॑च्छेदाय । \newline
6. सन्त॑त्या॒ अवि॑च्छेदा॒या वि॑च्छेदाय॒ सन्त॑त्यै॒ सन्त॑त्या॒ अवि॑च्छेदा यै॒न्द्रा ऐ॒न्द्रा अवि॑च्छेदाय॒ सन्त॑त्यै॒ सन्त॑त्या॒ अवि॑च्छेदायै॒न्द्राः । \newline
7. सन्त॑त्या॒ इति॒ सं - त॒त्यै॒ । \newline
8. अवि॑च्छेदा यै॒न्द्रा ऐ॒न्द्रा अवि॑च्छेदा॒या वि॑च्छेदा यै॒न्द्राः प॒शवः॑ प॒शव॑ ऐ॒न्द्रा अवि॑च्छेदा॒या वि॑च्छेदा यै॒न्द्राः प॒शवः॑ । \newline
9. अवि॑च्छेदा॒येत्यवि॑ - छे॒दा॒य॒ । \newline
10. ऐ॒न्द्राः प॒शवः॑ प॒शव॑ ऐ॒न्द्रा ऐ॒न्द्राः प॒शवो॒ ये ये प॒शव॑ ऐ॒न्द्रा ऐ॒न्द्राः प॒शवो॒ ये । \newline
11. प॒शवो॒ ये ये प॒शवः॑ प॒शवो॒ ये मु॑ष्क॒रा मु॑ष्क॒रा ये प॒शवः॑ प॒शवो॒ ये मु॑ष्क॒राः । \newline
12. ये मु॑ष्क॒रा मु॑ष्क॒रा ये ये मु॑ष्क॒रा यद् यन् मु॑ष्क॒रा ये ये मु॑ष्क॒रा यत् । \newline
13. मु॒ष्क॒रा यद् यन् मु॑ष्क॒रा मु॑ष्क॒रा यदै॒न्द्रा ऐ॒न्द्रा यन् मु॑ष्क॒रा मु॑ष्क॒रा यदै॒न्द्राः । \newline
14. यदै॒न्द्रा ऐ॒न्द्रा यद् यदै॒न्द्राः सन्तः॒ सन्त॑ ऐ॒न्द्रा यद् यदै॒न्द्राः सन्तः॑ । \newline
15. ऐ॒न्द्राः सन्तः॒ सन्त॑ ऐ॒न्द्रा ऐ॒न्द्राः सन्तो॒ ऽग्निभ्यो॒ ऽग्निभ्यः॒ सन्त॑ ऐ॒न्द्रा ऐ॒न्द्राः सन्तो॒ ऽग्निभ्यः॑ । \newline
16. सन्तो॒ ऽग्निभ्यो॒ ऽग्निभ्यः॒ सन्तः॒ सन्तो॒ ऽग्निभ्य॑ आल॒भ्यन्त॑ आल॒भ्यन्ते॒ ऽग्निभ्यः॒ सन्तः॒ सन्तो॒ ऽग्निभ्य॑ आल॒भ्यन्ते᳚ । \newline
17. अ॒ग्निभ्य॑ आल॒भ्यन्त॑ आल॒भ्यन्ते॒ ऽग्निभ्यो॒ ऽग्निभ्य॑ आल॒भ्यन्ते॑ दे॒वता᳚भ्यो दे॒वता᳚भ्य आल॒भ्यन्ते॒ ऽग्निभ्यो॒ ऽग्निभ्य॑ आल॒भ्यन्ते॑ दे॒वता᳚भ्यः । \newline
18. अ॒ग्निभ्य॒ इत्य॒ग्नि - भ्यः॒ । \newline
19. आ॒ल॒भ्यन्ते॑ दे॒वता᳚भ्यो दे॒वता᳚भ्य आल॒भ्यन्त॑ आल॒भ्यन्ते॑ दे॒वता᳚भ्यः स॒मदꣳ॑ स॒मद॑म् दे॒वता᳚भ्य आल॒भ्यन्त॑ आल॒भ्यन्ते॑ दे॒वता᳚भ्यः स॒मद᳚म् । \newline
20. आ॒ल॒भ्यन्त॒ इत्या᳚ - ल॒भ्यन्ते᳚ । \newline
21. दे॒वता᳚भ्यः स॒मदꣳ॑ स॒मद॑म् दे॒वता᳚भ्यो दे॒वता᳚भ्यः स॒मद॑म् दधाति दधाति स॒मद॑म् दे॒वता᳚भ्यो दे॒वता᳚भ्यः स॒मद॑म् दधाति । \newline
22. स॒मद॑म् दधाति दधाति स॒मदꣳ॑ स॒मद॑म् दधा त्याग्ने॒यी रा᳚ग्ने॒यीर् द॑धाति स॒मदꣳ॑ स॒मद॑म् दधा त्याग्ने॒यीः । \newline
23. स॒मद॒मिति॑ स - मद᳚म् । \newline
24. द॒धा॒ त्या॒ग्ने॒यी रा᳚ग्ने॒यीर् द॑धाति दधा त्याग्ने॒यी स्त्रि॒ष्टुभ॑ स्त्रि॒ष्टुभ॑ आग्ने॒यीर् द॑धाति दधा त्याग्ने॒यी स्त्रि॒ष्टुभः॑ । \newline
25. आ॒ग्ने॒यी स्त्रि॒ष्टुभ॑ स्त्रि॒ष्टुभ॑ आग्ने॒यी रा᳚ग्ने॒यी स्त्रि॒ष्टुभो॑ याज्यानुवा॒क्या॑ याज्यानुवा॒क्या᳚ स्त्रि॒ष्टुभ॑ आग्ने॒यी रा᳚ग्ने॒यी स्त्रि॒ष्टुभो॑ याज्यानुवा॒क्याः᳚ । \newline
26. त्रि॒ष्टुभो॑ याज्यानुवा॒क्या॑ याज्यानुवा॒क्या᳚ स्त्रि॒ष्टुभ॑ स्त्रि॒ष्टुभो॑ याज्यानुवा॒क्याः᳚ कुर्यात् कुर्याद् याज्यानुवा॒क्या᳚ स्त्रि॒ष्टुभ॑ स्त्रि॒ष्टुभो॑ याज्यानुवा॒क्याः᳚ कुर्यात् । \newline
27. या॒ज्या॒नु॒वा॒क्याः᳚ कुर्यात् कुर्याद् याज्यानुवा॒क्या॑ याज्यानुवा॒क्याः᳚ कुर्या॒द् यद् यत् कु॑र्याद् याज्यानुवा॒क्या॑ याज्यानुवा॒क्याः᳚ कुर्या॒द् यत् । \newline
28. या॒ज्या॒नु॒वा॒क्या॑ इति॑ याज्या - अ॒नु॒वा॒क्याः᳚ । \newline
29. कु॒र्या॒द् यद् यत् कु॑र्यात् कुर्या॒द् यदा᳚ग्ने॒यी रा᳚ग्ने॒यीर् यत् कु॑र्यात् कुर्या॒द् यदा᳚ग्ने॒यीः । \newline
30. यदा᳚ग्ने॒यी रा᳚ग्ने॒यीर् यद् यदा᳚ग्ने॒यी स्तेन॒ तेना᳚ग्ने॒यीर् यद् यदा᳚ग्ने॒यी स्तेन॑ । \newline
31. आ॒ग्ने॒यी स्तेन॒ तेना᳚ग्ने॒यी रा᳚ग्ने॒यी स्तेना᳚ग्ने॒या आ᳚ग्ने॒या स्तेना᳚ग्ने॒यी रा᳚ग्ने॒यी स्तेना᳚ग्ने॒याः । \newline
32. तेना᳚ग्ने॒या आ᳚ग्ने॒या स्तेन॒ तेना᳚ग्ने॒या यद् यदा᳚ग्ने॒या स्तेन॒ तेना᳚ग्ने॒या यत् । \newline
33. आ॒ग्ने॒या यद् यदा᳚ग्ने॒या आ᳚ग्ने॒या यत् त्रि॒ष्टुभ॑ स्त्रि॒ष्टुभो॒ यदा᳚ग्ने॒या आ᳚ग्ने॒या यत् त्रि॒ष्टुभः॑ । \newline
34. यत् त्रि॒ष्टुभ॑ स्त्रि॒ष्टुभो॒ यद् यत् त्रि॒ष्टुभ॒ स्तेन॒ तेन॑ त्रि॒ष्टुभो॒ यद् यत् त्रि॒ष्टुभ॒ स्तेन॑ । \newline
35. त्रि॒ष्टुभ॒ स्तेन॒ तेन॑ त्रि॒ष्टुभ॑ स्त्रि॒ष्टुभ॒ स्तेनै॒न्द्रा ऐ॒न्द्रा स्तेन॑ त्रि॒ष्टुभ॑ स्त्रि॒ष्टुभ॒ स्तेनै॒न्द्राः । \newline
36. तेनै॒न्द्रा ऐ॒न्द्रा स्तेन॒ तेनै॒न्द्राः समृ॑द्ध्यै॒ समृ॑द्ध्या ऐ॒न्द्रा स्तेन॒ तेनै॒न्द्राः समृ॑द्ध्यै । \newline
37. ऐ॒न्द्राः समृ॑द्ध्यै॒ समृ॑द्ध्या ऐ॒न्द्रा ऐ॒न्द्राः समृ॑द्ध्यै॒ न न समृ॑द्ध्या ऐ॒न्द्रा ऐ॒न्द्राः समृ॑द्ध्यै॒ न । \newline
38. समृ॑द्ध्यै॒ न न समृ॑द्ध्यै॒ समृ॑द्ध्यै॒ न दे॒वता᳚भ्यो दे॒वता᳚भ्यो॒ न समृ॑द्ध्यै॒ समृ॑द्ध्यै॒ न दे॒वता᳚भ्यः । \newline
39. समृ॑द्ध्या॒ इति॒ सं - ऋ॒द्ध्यै॒ । \newline
40. न दे॒वता᳚भ्यो दे॒वता᳚भ्यो॒ न न दे॒वता᳚भ्यः स॒मदꣳ॑ स॒मद॑म् दे॒वता᳚भ्यो॒ न न दे॒वता᳚भ्यः स॒मद᳚म् । \newline
41. दे॒वता᳚भ्यः स॒मदꣳ॑ स॒मद॑म् दे॒वता᳚भ्यो दे॒वता᳚भ्यः स॒मद॑म् दधाति दधाति स॒मद॑म् दे॒वता᳚भ्यो दे॒वता᳚भ्यः स॒मद॑म् दधाति । \newline
42. स॒मद॑म् दधाति दधाति स॒मदꣳ॑ स॒मद॑म् दधाति वा॒यवे॑ वा॒यवे॑ दधाति स॒मदꣳ॑ स॒मद॑म् दधाति वा॒यवे᳚ । \newline
43. स॒मद॒मिति॑ स - मद᳚म् । \newline
44. द॒धा॒ति॒ वा॒यवे॑ वा॒यवे॑ दधाति दधाति वा॒यवे॑ नि॒युत्व॑ते नि॒युत्व॑ते वा॒यवे॑ दधाति दधाति वा॒यवे॑ नि॒युत्व॑ते । \newline
45. वा॒यवे॑ नि॒युत्व॑ते नि॒युत्व॑ते वा॒यवे॑ वा॒यवे॑ नि॒युत्व॑ते तूप॒रम् तू॑प॒रम् नि॒युत्व॑ते वा॒यवे॑ वा॒यवे॑ नि॒युत्व॑ते तूप॒रम् । \newline
46. नि॒युत्व॑ते तूप॒रम् तू॑प॒रम् नि॒युत्व॑ते नि॒युत्व॑ते तूप॒र मा तू॑प॒रम् नि॒युत्व॑ते नि॒युत्व॑ते तूप॒र मा । \newline
47. नि॒युत्व॑त॒ इति॑ नि - युत्व॑ते । \newline
48. तू॒प॒र मा तू॑प॒रम् तू॑प॒र मा ल॑भते लभत॒ आ तू॑प॒रम् तू॑प॒र मा ल॑भते । \newline
49. आ ल॑भते लभत॒ आ ल॑भते॒ तेज॒ स्तेजो॑ लभत॒ आ ल॑भते॒ तेजः॑ । \newline
50. ल॒भ॒ते॒ तेज॒ स्तेजो॑ लभते लभते॒ तेजो॒ ऽग्ने र॒ग्ने स्तेजो॑ लभते लभते॒ तेजो॒ ऽग्नेः । \newline
51. तेजो॒ ऽग्ने र॒ग्ने स्तेज॒ स्तेजो॒ ऽग्नेर् वा॒युर् वा॒यु र॒ग्ने स्तेज॒ स्तेजो॒ ऽग्नेर् वा॒युः । \newline
52. अ॒ग्नेर् वा॒युर् वा॒यु र॒ग्ने र॒ग्नेर् वा॒यु स्तेज॑से॒ तेज॑से वा॒यु र॒ग्ने र॒ग्नेर् वा॒यु स्तेज॑से । \newline
53. वा॒यु स्तेज॑से॒ तेज॑से वा॒युर् वा॒यु स्तेज॑स ए॒ष ए॒ष तेज॑से वा॒युर् वा॒यु स्तेज॑स ए॒षः । \newline
54. तेज॑स ए॒ष ए॒ष तेज॑से॒ तेज॑स ए॒ष ऐष तेज॑से॒ तेज॑स ए॒ष आ । \newline
55. ए॒ष ऐष ए॒ष आ ल॑भ्यते लभ्यत॒ ऐष ए॒ष आ ल॑भ्यते । \newline
56. आ ल॑भ्यते लभ्यत॒ आ ल॑भ्यते॒ तस्मा॒त् तस्मा᳚ल् लभ्यत॒ आ ल॑भ्यते॒ तस्मा᳚त् । \newline
57. ल॒भ्य॒ते॒ तस्मा॒त् तस्मा᳚ल् लभ्यते लभ्यते॒ तस्मा᳚द् य॒द्रिय॑ङ् य॒द्रिय॒ङ् तस्मा᳚ल् लभ्यते लभ्यते॒ तस्मा᳚द् य॒द्रियङ्॑ । \newline
58. तस्मा᳚द् य॒द्रिय॑ङ् य॒द्रिय॒ङ् तस्मा॒त् तस्मा᳚द् य॒द्रिय॑ङ्. वा॒युर् वा॒युर् य॒द्रिय॒ङ् तस्मा॒त् तस्मा᳚द् य॒द्रिय॑ङ्. वा॒युः । \newline
59. य॒द्रिय॑ङ्. वा॒युर् वा॒युर् य॒द्रिय॑ङ् य॒द्रिय॑ङ्. वा॒युर् वाति॒ वाति॑ वा॒युर् य॒द्रिय॑ङ् य॒द्रिय॑ङ्. वा॒युर् वाति॑ । \newline
60. वा॒युर् वाति॒ वाति॑ वा॒युर् वा॒युर् वाति॑ त॒द्रिय॑ङ् त॒द्रिय॒ङ्. वाति॑ वा॒युर् वा॒युर् वाति॑ त॒द्रियङ्॑ । \newline
\pagebreak
\markright{ TS 5.5.1.2  \hfill https://www.vedavms.in \hfill}

\section{ TS 5.5.1.2 }

\textbf{TS 5.5.1.2 } \newline
\textbf{Samhita Paata} \newline

-र्वाति॑ त॒द्रिय॑ङ्ङ॒-ग्निर्द॑हति॒ स्वमे॒व तत् तेजोऽन्वे॑ति॒ यन्न नि॒युत्व॑ते॒ स्यादुन्मा᳚द्ये॒द्-यज॑मानो नि॒युत्व॑ते भवति॒ यज॑मान॒स्याऽ*नु॑न्मादाय वायु॒मती᳚ श्वे॒तव॑ती याज्यानुवा॒क्ये॑ भवतः सतेज॒स्त्वाय॑ हिरण्यग॒र्भः सम॑वर्त॒ताग्र॒ इत्या॑घा॒रमा घा॑रयति प्र॒जाप॑ति॒र्वै हि॑रण्यग॒र्भः प्र॒जाप॑तेरनुरूप॒त्वाय॒ सर्वा॑णि॒ वा ए॒ष रू॒पाणि॑ पशू॒नां प्रत्या ल॑भ्यते॒ यच्छ्म॑श्रु॒णस्तत् - [  ] \newline

\textbf{Pada Paata} \newline

वाति॑ । त॒द्रियङ्॑ । अ॒ग्निः । द॒ह॒ति॒ । स्वम् । ए॒व । तत् । तेजः॑ । अन्विति॑ । ए॒ति॒ । यत् । न । नि॒युत्व॑त॒ इति॑ नि - युत्व॑ते । स्यात् । उदिति॑ । मा॒द्ये॒त् । यज॑मानः । नि॒युत्व॑त॒ इति॑ नि - युत्व॑ते । भ॒व॒ति॒ । यज॑मानस्य । अनु॑न्मादा॒येत्यनु॑त् - मा॒दा॒य॒ । वा॒यु॒मती॒ इति॑ वायु - मती᳚ । श्वे॒तव॑ती॒ इति॑ श्वे॒त - व॒ती॒ । या॒ज्या॒नु॒वा॒क्ये॑ इति॑ याज्या - अ॒नु॒वा॒क्ये᳚ । भ॒व॒तः॒ । स॒ते॒ज॒स्त्वायेति॑ सतेजः - त्वाय॑ । हि॒र॒ण्य॒ग॒र्भ इति॑ हिरण्य - ग॒र्भः । समिति॑ । अ॒व॒र्त॒त॒ । अग्रे᳚ । इति॑ । आ॒घा॒रमित्या᳚ - घा॒रम् । एति॑ । घा॒र॒य॒ति॒ । प्र॒जाप॑ति॒रिति॑ प्र॒जा-प॒तिः॒ । वै । हि॒र॒ण्य॒ग॒र्भ इति॑ हिरण्य - ग॒र्भः । प्र॒जाप॑ते॒रिति॑ प्र॒जा - प॒तेः॒ । अ॒नु॒रू॒प॒त्वायेत्य॑नुरूप - त्वाय॑ । सर्वा॑णि । वै । ए॒षः । रू॒पाणि॑ । प॒शू॒नाम् । प्रति॑ । एति॑ । ल॒भ्य॒ते॒ । यत् । श्म॒श्रु॒णः । तत् ।  \newline


\textbf{Krama Paata} \newline

वाति॑ त॒द्रियङ्ङ्॑ । त॒द्रिय॑ङ्ङ॒ग्निः । अ॒ग्निर् द॑हति । द॒ह॒ति॒ स्वम् । स्वमे॒व । ए॒व तत् । तत् तेजः॑ । तेजोऽनु॑ । अन्वे॑ति । ए॒ति॒ यत् । यन् न । न नि॒युत्व॑ते । नि॒युत्व॑ते॒ स्यात् । नि॒युत्व॑त॒ इति॑ नि - युत्व॑ते । स्यादुत् । उन् मा᳚द्येत् । मा॒द्ये॒द् यज॑मानः । यज॑मानो नि॒युत्व॑ते । नि॒युत्व॑ते भवति । नि॒युत्व॑त॒ इति॑ नि - युत्व॑ते । भ॒व॒ति॒ यज॑मानस्य । यज॑मान॒स्यानु॑न्मादाय । अनु॑न्मादाय वायु॒मती᳚ । अनु॑न्मादा॒येत्यनु॑त् - मा॒दा॒य॒ । वा॒यु॒मती᳚ श्वे॒तव॑ती । वा॒यु॒मती॒ इति॑ वायु - मती᳚ । श्वे॒तव॑ती याज्यानुवा॒क्ये᳚ । श्वे॒तव॑ती॒ इति॑ श्वे॒त - व॒ती॒ । या॒ज्या॒नु॒वा॒क्ये॑ भवतः । या॒ज्या॒नु॒वा॒क्ये॑ इति॑ याज्या - अ॒नु॒वा॒क्ये᳚ । भ॒व॒तः॒ स॒ते॒ज॒स्त्वाय॑ । स॒ते॒ज॒स्त्वाय॑ हिरण्यग॒र्भः । स॒ते॒ज॒स्त्वायेति॑ सतेजः - त्वाय॑ । हि॒र॒ण्य॒ग॒र्भः सम् । हि॒र॒ण्य॒ग॒र्भ इति॑ हिरण्य - ग॒र्भः । सम॑वर्तत । अ॒व॒र्त॒ताग्रे᳚ । अग्र॒ इति॑ । इत्या॑घा॒रम् । आ॒घा॒रमा । आ॒घा॒रमित्या᳚ - घा॒रम् । आ घा॑रयति । घा॒र॒य॒ति॒ प्र॒जाप॑तिः । प्र॒जाप॑ति॒र् वै । प्र॒जाप॑ति॒रिति॑ प्र॒जा - प॒तिः॒ । वै हि॑रण्यग॒र्भः । हि॒र॒ण्य॒ग॒र्भः प्र॒जाप॑तेः । हि॒र॒ण्य॒ग॒र्भ इति॑ हिरण्य - ग॒र्भः । प्र॒जाप॑तेरनुरूप॒त्वाय॑ । प्र॒जाप॑ते॒रिति॑ प्र॒जा - प॒तेः॒ । अ॒नु॒रू॒प॒त्वाय॒ सर्वा॑णि । अ॒नु॒रू॒प॒त्वायेत्य॑नुरूप - त्वाय॑ । सर्वा॑णि॒ वै । वा ए॒षः । ए॒ष रू॒पाणि॑ । रू॒पाणि॑ पशू॒नाम् । प॒शू॒नाम् प्रति॑ । प्रत्या । आ ल॑भ्यते । ल॒भ्य॒ते॒ यत् । यच्छ्म॑श्रु॒णः । श्म॒श्रु॒णस्तत् । तत् पुरु॑षाणाम् \newline

\textbf{Jatai Paata} \newline

1. वाति॑ त॒द्रिय॑ङ् त॒द्रिय॒ङ् वाति॒ वाति॑ त॒द्रियङ्॑ । \newline
2. त॒द्रिय॑ङ् ङ॒ग्नि र॒ग्नि स्त॒द्रिय॑ङ् त॒द्रिय॑ङ् ङ॒ग्निः । \newline
3. अ॒ग्निर् द॑हति दह त्य॒ग्नि र॒ग्निर् द॑हति । \newline
4. द॒ह॒ति॒ स्वꣳ स्वम् द॑हति दहति॒ स्वम् । \newline
5. स्व मे॒वैव स्वꣳ स्व मे॒व । \newline
6. ए॒व तत् तदे॒ वैव तत् । \newline
7. तत् तेज॒ स्तेज॒ स्तत् तत् तेजः॑ । \newline
8. तेजो ऽन्वनु॒ तेज॒ स्तेजो ऽनु॑ । \newline
9. अन्वे᳚ त्ये॒ त्यन्वन् वे॑ति । \newline
10. ए॒ति॒ यद् यदे᳚ त्येति॒ यत् । \newline
11. यन् न न यद् यन् न । \newline
12. न नि॒युत्व॑ते नि॒युत्व॑ते॒ न न नि॒युत्व॑ते । \newline
13. नि॒युत्व॑ते॒ स्याथ् स्यान् नि॒युत्व॑ते नि॒युत्व॑ते॒ स्यात् । \newline
14. नि॒युत्व॑त॒ इति॑ नि - युत्व॑ते । \newline
15. स्या दुदुथ् स्याथ् स्या दुत् । \newline
16. उन् मा᳚द्येन् माद्ये॒ दु दुन् मा᳚द्येत् । \newline
17. मा॒द्ये॒द् यज॑मानो॒ यज॑मानो माद्येन् माद्ये॒द् यज॑मानः । \newline
18. यज॑मानो नि॒युत्व॑ते नि॒युत्व॑ते॒ यज॑मानो॒ यज॑मानो नि॒युत्व॑ते । \newline
19. नि॒युत्व॑ते भवति भवति नि॒युत्व॑ते नि॒युत्व॑ते भवति । \newline
20. नि॒युत्व॑त॒ इति॑ नि - युत्व॑ते । \newline
21. भ॒व॒ति॒ यज॑मानस्य॒ यज॑मानस्य भवति भवति॒ यज॑मानस्य । \newline
22. यज॑मान॒स्या नु॑न्मादा॒या नु॑न्मादाय॒ यज॑मानस्य॒ यज॑मान॒स्या नु॑न्मादाय । \newline
23. अनु॑न्मादाय वायु॒मती॑ वायु॒मती॒ अनु॑न्मादा॒या नु॑न्मादाय वायु॒मती᳚ । \newline
24. अनु॑न्मादा॒येत्यनु॑त् - मा॒दा॒य॒ । \newline
25. वा॒यु॒मती᳚ श्वे॒तव॑ती श्वे॒तव॑ती वायु॒मती॑ वायु॒मती᳚ श्वे॒तव॑ती । \newline
26. वा॒यु॒मती॒ इति॑ वायु - मती᳚ । \newline
27. श्वे॒तव॑ती याज्यानुवा॒क्ये॑ याज्यानुवा॒क्ये᳚ श्वे॒तव॑ती श्वे॒तव॑ती याज्यानुवा॒क्ये᳚ । \newline
28. श्वे॒तव॑ती॒ इति॑ श्वे॒त - व॒ती॒ । \newline
29. या॒ज्या॒नु॒वा॒क्ये॑ भवतो भवतो याज्यानुवा॒क्ये॑ याज्यानुवा॒क्ये॑ भवतः । \newline
30. या॒ज्या॒नु॒वा॒क्ये॑ इति॑ याज्या - अ॒नु॒वा॒क्ये᳚ । \newline
31. भ॒व॒तः॒ स॒ते॒ज॒स्त्वाय॑ सतेज॒स्त्वाय॑ भवतो भवतः सतेज॒स्त्वाय॑ । \newline
32. स॒ते॒ज॒स्त्वाय॑ हिरण्यग॒र्भो हि॑रण्यग॒र्भः स॑तेज॒स्त्वाय॑ सतेज॒स्त्वाय॑ हिरण्यग॒र्भः । \newline
33. स॒ते॒ज॒स्त्वायेति॑ सतेजः - त्वाय॑ । \newline
34. हि॒र॒ण्य॒ग॒र्भः सꣳ सꣳ हि॑रण्यग॒र्भो हि॑रण्यग॒र्भः सम् । \newline
35. हि॒र॒ण्य॒ग॒र्भ इति॑ हिरण्य - ग॒र्भः । \newline
36. स म॑वर्तता वर्तत॒ सꣳ स म॑वर्तत । \newline
37. अ॒व॒र्त॒ ताग्रे॒ अग्रे॑ ऽवर्तता वर्त॒ताग्रे᳚ । \newline
38. अग्र॒ इतीत्यग्रे᳚ ऽग्र॒ इति॑ । \newline
39. इत्या॑घा॒र मा॑घा॒र मिती त्या॑घा॒रम् । \newline
40. आ॒घा॒र मा ऽऽघा॒र मा॑घा॒र मा । \newline
41. आ॒घा॒रमित्या᳚ - घा॒रम् । \newline
42. आ घा॑रयति घारय॒ त्याघा॑रयति । \newline
43. घा॒र॒य॒ति॒ प्र॒जाप॑तिः प्र॒जाप॑तिर् घारयति घारयति प्र॒जाप॑तिः । \newline
44. प्र॒जाप॑ति॒र् वै वै प्र॒जाप॑तिः प्र॒जाप॑ति॒र् वै । \newline
45. प्र॒जाप॑ति॒रिति॑ प्र॒जा - प॒तिः॒ । \newline
46. वै हि॑रण्यग॒र्भो हि॑रण्यग॒र्भो वै वै हि॑रण्यग॒र्भः । \newline
47. हि॒र॒ण्य॒ग॒र्भः प्र॒जाप॑तेः प्र॒जाप॑तेर्. हिरण्यग॒र्भो हि॑रण्यग॒र्भः प्र॒जाप॑तेः । \newline
48. हि॒र॒ण्य॒ग॒र्भ इति॑ हिरण्य - ग॒र्भः । \newline
49. प्र॒जाप॑ते रनुरूप॒त्वाया॑ नुरूप॒त्वाय॑ प्र॒जाप॑तेः प्र॒जाप॑ते रनुरूप॒त्वाय॑ । \newline
50. प्र॒जाप॑ते॒रिति॑ प्र॒जा - प॒तेः॒ । \newline
51. अ॒नु॒रू॒प॒त्वाय॒ सर्वा॑णि॒ सर्वा᳚ ण्यनुरूप॒त्वाया॑ नुरूप॒त्वाय॒ सर्वा॑णि । \newline
52. अ॒नु॒रू॒प॒त्वायेत्य॑नुरूप - त्वाय॑ । \newline
53. सर्वा॑णि॒ वै वै सर्वा॑णि॒ सर्वा॑णि॒ वै । \newline
54. वा ए॒ष ए॒ष वै वा ए॒षः । \newline
55. ए॒ष रू॒पाणि॑ रू॒पा ण्ये॒ष ए॒ष रू॒पाणि॑ । \newline
56. रू॒पाणि॑ पशू॒नाम् प॑शू॒नाꣳ रू॒पाणि॑ रू॒पाणि॑ पशू॒नाम् । \newline
57. प॒शू॒नाम् प्रति॒ प्रति॑ पशू॒नाम् प॑शू॒नाम् प्रति॑ । \newline
58. प्रत्या प्रति॒ प्रत्या । \newline
59. आ ल॑भ्यते लभ्यत॒ आ ल॑भ्यते । \newline
60. ल॒भ्य॒ते॒ यद् यल् ल॑भ्यते लभ्यते॒ यत् । \newline
61. यच्छ्म॑श्रु॒णः श्म॑श्रु॒णो यद् यच्छ्म॑श्रु॒णः । \newline
62. श्म॒श्रु॒ण स्तत् तच्छ्म॑श्रु॒णः श्म॑श्रु॒ण स्तत् । \newline
63. तत् पुरु॑षाणा॒म् पुरु॑षाणा॒म् तत् तत् पुरु॑षाणाम् । \newline

\textbf{Ghana Paata } \newline

1. वाति॑ त॒द्रिय॑ङ् त॒द्रिय॒ङ्. वाति॒ वाति॑ त॒द्रिय॑ङ् ङ॒ग्नि र॒ग्नि स्त॒द्रिय॒ङ् वाति॒ वाति॑ त॒द्रिय॑ङ् ङ॒ग्निः । \newline
2. त॒द्रिय॑ङ् ङ॒ग्नि र॒ग्नि स्त॒द्रिय॑ङ् त॒द्रिय॑ङ् ङ॒ग्निर् द॑हति दह त्य॒ग्नि स्त॒द्रिय॑ङ् त॒द्रिय॑ङ् ङ॒ग्निर् द॑हति । \newline
3. अ॒ग्निर् द॑हति दह त्य॒ग्नि र॒ग्निर् द॑हति॒ स्वꣳ स्वम् द॑ह त्य॒ग्नि र॒ग्निर् द॑हति॒ स्वम् । \newline
4. द॒ह॒ति॒ स्वꣳ स्वम् द॑हति दहति॒ स्व मे॒वैव स्वम् द॑हति दहति॒ स्व मे॒व । \newline
5. स्व मे॒वैव स्वꣳ स्व मे॒व तत् तदे॒व स्वꣳ स्व मे॒व तत् । \newline
6. ए॒व तत् तदे॒ वैव तत् तेज॒ स्तेज॒ स्तदे॒वैव तत् तेजः॑ । \newline
7. तत् तेज॒ स्तेज॒ स्तत् तत् तेजो ऽन्वनु॒ तेज॒ स्तत् तत् तेजो ऽनु॑ । \newline
8. तेजो ऽन्वनु॒ तेज॒ स्तेजो ऽन्वे᳚ त्ये॒ त्यनु॒ तेज॒ स्तेजो ऽन्वे॑ति । \newline
9. अन्वे᳚त्ये॒ त्यन्वन् वे॑ति॒ यद् यदे॒ त्यन् वन् वे॑ति॒ यत् । \newline
10. ए॒ति॒ यद् यदे᳚ त्येति॒ यन् न न यदे᳚ त्येति॒ यन् न । \newline
11. यन् न न यद् यन् न नि॒युत्व॑ते नि॒युत्व॑ते॒ न यद् यन् न नि॒युत्व॑ते । \newline
12. न नि॒युत्व॑ते नि॒युत्व॑ते॒ न न नि॒युत्व॑ते॒ स्याथ् स्यान् नि॒युत्व॑ते॒ न न नि॒युत्व॑ते॒ स्यात् । \newline
13. नि॒युत्व॑ते॒ स्याथ् स्यान् नि॒युत्व॑ते नि॒युत्व॑ते॒ स्या दुदुथ् स्यान् नि॒युत्व॑ते नि॒युत्व॑ते॒ स्या दुत् । \newline
14. नि॒युत्व॑त॒ इति॑ नि - युत्व॑ते । \newline
15. स्यादु दुथ् स्याथ् स्यादुन् मा᳚द्येन् माद्ये॒ दुथ् स्याथ् स्यादुन् मा᳚द्येत् । \newline
16. उन् मा᳚द्येन् माद्ये॒ दुदुन् मा᳚द्ये॒द् यज॑मानो॒ यज॑मानो माद्ये॒ दुदुन् मा᳚द्ये॒द् यज॑मानः । \newline
17. मा॒द्ये॒द् यज॑मानो॒ यज॑मानो माद्येन् माद्ये॒द् यज॑मानो नि॒युत्व॑ते नि॒युत्व॑ते॒ यज॑मानो माद्येन् माद्ये॒द् यज॑मानो नि॒युत्व॑ते । \newline
18. यज॑मानो नि॒युत्व॑ते नि॒युत्व॑ते॒ यज॑मानो॒ यज॑मानो नि॒युत्व॑ते भवति भवति नि॒युत्व॑ते॒ यज॑मानो॒ यज॑मानो नि॒युत्व॑ते भवति । \newline
19. नि॒युत्व॑ते भवति भवति नि॒युत्व॑ते नि॒युत्व॑ते भवति॒ यज॑मानस्य॒ यज॑मानस्य भवति नि॒युत्व॑ते नि॒युत्व॑ते भवति॒ यज॑मानस्य । \newline
20. नि॒युत्व॑त॒ इति॑ नि - युत्व॑ते । \newline
21. भ॒व॒ति॒ यज॑मानस्य॒ यज॑मानस्य भवति भवति॒ यज॑मान॒स्या नु॑न्मादा॒या नु॑न्मादाय॒ यज॑मानस्य भवति भवति॒ यज॑मान॒स्या नु॑न्मादाय । \newline
22. यज॑मान॒स्या नु॑न्मादा॒या नु॑न्मादाय॒ यज॑मानस्य॒ यज॑मान॒स्या नु॑न्मादाय वायु॒मती॑ 
वायु॒मती॒ अनु॑न्मादाय॒ यज॑मानस्य॒ यज॑मान॒स्या नु॑न्मादाय वायु॒मती᳚ । \newline
23. अनु॑न्मादाय वायु॒मती॑ वायु॒मती॒ अनु॑न्मादा॒या नु॑न्मादाय वायु॒मती᳚ श्वे॒तव॑ती श्वे॒तव॑ती 
वायु॒मती॒ अनु॑न्मादा॒या नु॑न्मादाय वायु॒मती᳚ श्वे॒तव॑ती । \newline
24. अनु॑न्मादा॒येत्यनु॑त् - मा॒दा॒य॒ । \newline
25. वा॒यु॒मती᳚ श्वे॒तव॑ती श्वे॒तव॑ती वायु॒मती॑ वायु॒मती᳚ श्वे॒तव॑ती याज्यानुवा॒क्ये॑ याज्यानुवा॒क्ये᳚ श्वे॒तव॑ती वायु॒मती॑ वायु॒मती᳚ श्वे॒तव॑ती याज्यानुवा॒क्ये᳚ । \newline
26. वा॒यु॒मती॒ इति॑ वायु - मती᳚ । \newline
27. श्वे॒तव॑ती याज्यानुवा॒क्ये॑ याज्यानुवा॒क्ये᳚ श्वे॒तव॑ती श्वे॒तव॑ती याज्यानुवा॒क्ये॑ भवतो भवतो याज्यानुवा॒क्ये᳚ श्वे॒तव॑ती श्वे॒तव॑ती याज्यानुवा॒क्ये॑ भवतः । \newline
28. श्वे॒तव॑ती॒ इति॑ श्वे॒त - व॒ती॒ । \newline
29. या॒ज्या॒नु॒वा॒क्ये॑ भवतो भवतो याज्यानुवा॒क्ये॑ याज्यानुवा॒क्ये॑ भवतः सतेज॒स्त्वाय॑ सतेज॒स्त्वाय॑ भवतो याज्यानुवा॒क्ये॑ याज्यानुवा॒क्ये॑ भवतः सतेज॒स्त्वाय॑ । \newline
30. या॒ज्या॒नु॒वा॒क्ये॑ इति॑ याज्या - अ॒नु॒वा॒क्ये᳚ । \newline
31. भ॒व॒तः॒ स॒ते॒ज॒स्त्वाय॑ सतेज॒स्त्वाय॑ भवतो भवतः सतेज॒स्त्वाय॑ हिरण्यग॒र्भो हि॑रण्यग॒र्भः स॑तेज॒स्त्वाय॑ भवतो भवतः सतेज॒स्त्वाय॑ हिरण्यग॒र्भः । \newline
32. स॒ते॒ज॒स्त्वाय॑ हिरण्यग॒र्भो हि॑रण्यग॒र्भः स॑तेज॒स्त्वाय॑ सतेज॒स्त्वाय॑ हिरण्यग॒र्भः सꣳ सꣳ हि॑रण्यग॒र्भः स॑तेज॒स्त्वाय॑ सतेज॒स्त्वाय॑ हिरण्यग॒र्भः सम् । \newline
33. स॒ते॒ज॒स्त्वायेति॑ सतेजः - त्वाय॑ । \newline
34. हि॒र॒ण्य॒ग॒र्भः सꣳ सꣳ हि॑रण्यग॒र्भो हि॑रण्यग॒र्भः स म॑वर्तता वर्तत॒ सꣳ हि॑रण्यग॒र्भो हि॑रण्यग॒र्भः स म॑वर्तत । \newline
35. हि॒र॒ण्य॒ग॒र्भ इति॑ हिरण्य - ग॒र्भः । \newline
36. स म॑वर्तता वर्तत॒ सꣳ स म॑वर्त॒ ताग्रे॒ अग्रे॑ ऽवर्तत॒ सꣳ स म॑वर्त॒ ताग्रे᳚ । \newline
37. अ॒व॒र्त॒ ताग्रे॒ अग्रे॑ ऽवर्तता वर्त॒ताग्र॒ इती त्यग्रे॑ ऽवर्तता वर्त॒ताग्र॒ इति॑ । \newline
38. अग्र॒ इती त्यग्रे᳚ ऽग्र॒ इत्या॑घा॒र मा॑घा॒र मित्यग्रे᳚ ऽग्र॒ इत्या॑घा॒रम् । \newline
39. इत्या॑घा॒र मा॑घा॒र मिती त्या॑घा॒र मा ऽऽघा॒र मिती त्या॑घा॒र मा । \newline
40. आ॒घा॒र मा ऽऽघा॒र मा॑घा॒र मा घा॑रयति घारय॒त्या ऽऽघा॒र मा॑घा॒र मा घा॑रयति । \newline
41. आ॒घा॒रमित्या᳚ - घा॒रम् । \newline
42. आ घा॑रयति घारय॒त्या घा॑रयति प्र॒जाप॑तिः प्र॒जाप॑तिर् घारय॒त्या घा॑रयति प्र॒जाप॑तिः । \newline
43. घा॒र॒य॒ति॒ प्र॒जाप॑तिः प्र॒जाप॑तिर् घारयति घारयति प्र॒जाप॑ति॒र् वै वै प्र॒जाप॑तिर् घारयति घारयति प्र॒जाप॑ति॒र् वै । \newline
44. प्र॒जाप॑ति॒र् वै वै प्र॒जाप॑तिः प्र॒जाप॑ति॒र् वै हि॑रण्यग॒र्भो हि॑रण्यग॒र्भो वै प्र॒जाप॑तिः प्र॒जाप॑ति॒र् वै हि॑रण्यग॒र्भः । \newline
45. प्र॒जाप॑ति॒रिति॑ प्र॒जा - प॒तिः॒ । \newline
46. वै हि॑रण्यग॒र्भो हि॑रण्यग॒र्भो वै वै हि॑रण्यग॒र्भः प्र॒जाप॑तेः प्र॒जाप॑तेर्. हिरण्यग॒र्भो वै वै हि॑रण्यग॒र्भः प्र॒जाप॑तेः । \newline
47. हि॒र॒ण्य॒ग॒र्भः प्र॒जाप॑तेः प्र॒जाप॑तेर्. हिरण्यग॒र्भो हि॑रण्यग॒र्भः प्र॒जाप॑ते रनुरूप॒त्वाया॑नु रूप॒त्वाय॑ प्र॒जाप॑तेर्. हिरण्यग॒र्भो हि॑रण्यग॒र्भः प्र॒जाप॑ते रनुरूप॒त्वाय॑ । \newline
48. हि॒र॒ण्य॒ग॒र्भ इति॑ हिरण्य - ग॒र्भः । \newline
49. प्र॒जाप॑ते रनुरूप॒त्वाया॑ नुरूप॒त्वाय॑ प्र॒जाप॑तेः प्र॒जाप॑ते रनुरूप॒त्वाय॒ सर्वा॑णि॒ सर्वा᳚
ण्यनुरूप॒त्वाय॑ प्र॒जाप॑तेः प्र॒जाप॑ते रनुरूप॒त्वाय॒ सर्वा॑णि । \newline
50. प्र॒जाप॑ते॒रिति॑ प्र॒जा - प॒तेः॒ । \newline
51. अ॒नु॒रू॒प॒त्वाय॒ सर्वा॑णि॒ सर्वा᳚ ण्यनुरूप॒त्वाया॑ नुरूप॒त्वाय॒ सर्वा॑णि॒ वै वै सर्वा᳚ ण्यनुरूप॒त्वाया॑ नुरूप॒त्वाय॒ सर्वा॑णि॒ वै । \newline
52. अ॒नु॒रू॒प॒त्वायेत्य॑नुरूप - त्वाय॑ । \newline
53. सर्वा॑णि॒ वै वै सर्वा॑णि॒ सर्वा॑णि॒ वा ए॒ष ए॒ष वै सर्वा॑णि॒ सर्वा॑णि॒ वा ए॒षः । \newline
54. वा ए॒ष ए॒ष वै वा ए॒ष रू॒पाणि॑ रू॒पाण्ये॒ष वै वा ए॒ष रू॒पाणि॑ । \newline
55. ए॒ष रू॒पाणि॑ रू॒पाण्ये॒ष ए॒ष रू॒पाणि॑ पशू॒नाम् प॑शू॒नाꣳ रू॒पाण्ये॒ष ए॒ष रू॒पाणि॑ पशू॒नाम् । \newline
56. रू॒पाणि॑ पशू॒नाम् प॑शू॒नाꣳ रू॒पाणि॑ रू॒पाणि॑ पशू॒नाम् प्रति॒ प्रति॑ पशू॒नाꣳ रू॒पाणि॑ रू॒पाणि॑ पशू॒नाम् प्रति॑ । \newline
57. प॒शू॒नाम् प्रति॒ प्रति॑ पशू॒नाम् प॑शू॒नाम् प्रत्या प्रति॑ पशू॒नाम् प॑शू॒नाम् प्रत्या । \newline
58. प्रत्या प्रति॒ प्रत्या ल॑भ्यते लभ्यत॒ आ प्रति॒ प्रत्या ल॑भ्यते । \newline
59. आ ल॑भ्यते लभ्यत॒ आ ल॑भ्यते॒ यद् यल्ल॑भ्यत॒ आ ल॑भ्यते॒ यत् । \newline
60. ल॒भ्य॒ते॒ यद् यल्ल॑भ्यते लभ्यते॒ यच्छ्म॑श्रु॒णः श्म॑श्रु॒णो यल्ल॑भ्यते लभ्यते॒ यच्छ्म॑श्रु॒णः । \newline
61. यच्छ्म॑श्रु॒णः श्म॑श्रु॒णो यद् यच्छ्म॑श्रु॒ण स्तत् तच्छ्म॑श्रु॒णो यद् यच्छ्म॑श्रु॒ण स्तत् । \newline
62. श्म॒श्रु॒ण स्तत् तच्छ्म॑श्रु॒णः श्म॑श्रु॒ण स्तत् पुरु॑षाणा॒म् पुरु॑षाणा॒म् तच्छ्म॑श्रु॒णः श्म॑श्रु॒ण स्तत् पुरु॑षाणाम् । \newline
63. तत् पुरु॑षाणा॒म् पुरु॑षाणा॒म् तत् तत् पुरु॑षाणाꣳ रू॒पꣳ रू॒पम् पुरु॑षाणा॒म् तत् तत् पुरु॑षाणाꣳ रू॒पम् । \newline
\pagebreak
\markright{ TS 5.5.1.3  \hfill https://www.vedavms.in \hfill}

\section{ TS 5.5.1.3 }

\textbf{TS 5.5.1.3 } \newline
\textbf{Samhita Paata} \newline

पुरु॑षाणाꣳ रू॒पं ॅयत् तू॑प॒रस्तदश्वा॑नां॒ ॅयद॒न्यतो॑द॒न् तद्-गवां॒ ॅयदव्या॑ इव श॒फास्तदवी॑नां॒ ॅयद॒जस्तद॒जानां᳚ ॅवा॒युर्वै प॑शू॒नां प्रि॒यं धाम॒ यद्-वा॑य॒व्यो॑ भव॑त्ये॒तमे॒वैन॑म॒भि स॑जांना॒नाः प॒शव॒ उप॑ तिष्ठन्ते वाय॒व्यः॑ का॒र्या(3)ः प्रा॑जाप॒त्या(3) इत्या॑हु॒-र्यद्-वा॑य॒व्यं॑ कु॒र्यात् प्र॒जाप॑तेरिया॒द्यत् प्रा॑जाप॒त्यं कु॒र्याद्-वा॒यो - [  ] \newline

\textbf{Pada Paata} \newline

पुरु॑षाणाम् । रू॒पम् । यत् । तू॒प॒रः । तत् । अश्वा॑नाम् । यत् । अ॒न्यतो॑द॒न्नित्य॒न्यतः॑ - द॒न्न् । तत् । गवा᳚म् । यत् । अव्याः᳚ । इ॒व॒ । श॒फाः । तत् । अवी॑नाम् । यत् । अ॒जः । तत् । अ॒जाना᳚म् । वा॒युः । वै । प॒शू॒नाम् । प्रि॒यम् । धाम॑ । यत् । वा॒य॒व्यः॑ । भव॑ति । ए॒तम् । ए॒व । ए॒न॒म् । अ॒भीति॑ । स॒ञ्जा॒ना॒ना इति॑ सं - जा॒ना॒नाः । प॒शवः॑ । उपेति॑ । ति॒ष्ठ॒न्ते॒ । वा॒य॒व्यः॑ । का॒र्या(3)ः । प्रा॒जा॒प॒त्या(3) इति॑ प्राजा - प॒त्या(3)ः । इति॑ । आ॒हुः॒ । यत् । वा॒य॒व्य᳚म् । कु॒र्यात् । प्र॒जाप॑ते॒रिति॑ प्र॒जा - प॒तेः॒ । इ॒या॒त् । यत् । प्रा॒जा॒प॒त्यमिति॑ प्राजा - प॒त्यम् । कु॒र्यात् । वा॒योः ।  \newline


\textbf{Krama Paata} \newline

पुरु॑षाणाꣳ रू॒पम् । रू॒पम् ॅयत् । यत् तू॑प॒रः । तू॒प॒रस्तत् । तदश्वा॑नाम् । अश्वा॑ना॒म् ॅयत् । यद॒न्यतो॑दन्न् । अ॒न्यतो॑द॒न् तत् । अ॒न्यतो॑द॒न्नित्य॒न्यतः॑ - द॒न्न्॒ । तद् गवा᳚म् । गवा॒म् ॅयत् । यदव्याः᳚ । अव्या॑ इव । इ॒व॒ श॒फाः । श॒फास्तत् । तदवी॑नाम् । अवी॑ना॒म् ॅयत् । यद॒जः । अ॒जस्तत् । तद॒जाना᳚म् । अ॒जाना᳚म् ॅवा॒युः । वा॒युर् वै । वै प॑शू॒नाम् । प॒शू॒नाम् प्रि॒यम् । प्रि॒यम् धाम॑ । धाम॒ यत् । यद् वा॑य॒व्यः॑ । वा॒य॒व्यो॑ भव॑ति । भव॑त्ये॒तम् । ए॒तमे॒व । ए॒वैन᳚म् । ए॒न॒म॒भि । अ॒भि स॑ञ्जाना॒नाः । स॒ञ्जा॒ना॒नाः प॒शवः॑ । स॒ञ्जा॒ना॒ना इति॑ सम् - जा॒ना॒नाः । प॒शव॒ उप॑ । उप॑ तिष्ठन्ते । ति॒ष्ठ॒न्ते॒ वा॒य॒व्यः॑ । वा॒य॒व्यः॑ का॒र्या(3)ः । का॒र्या(3)ः प्रा॑जाप॒त्या(3)ः । प्रा॒जा॒प॒त्या(3) इति॑ । प्रा॒जा॒प॒त्या(3) इति॑ प्राजा - प॒त्या(3)ः । इत्या॑हुः । आ॒हु॒र् यत् । यद् वा॑य॒व्य᳚म् । वा॒य॒व्य॑म् कु॒र्यात् । कु॒र्यात् प्र॒जाप॑तेः । प्र॒जाप॑तेरियात् । प्र॒जाप॑ते॒रिति॑ प्र॒जा - प॒तेः॒ । इ॒या॒द् यत् । यत् प्रा॑जाप॒त्यम् । प्रा॒जा॒प॒त्यम् कु॒र्यात् । प्रा॒जा॒प॒त्यमिति॑ प्राजा - प॒त्यम् । कु॒र्याद् वा॒योः । वा॒योरि॑यात् \newline

\textbf{Jatai Paata} \newline

1. पुरु॑षाणाꣳ रू॒पꣳ रू॒पम् पुरु॑षाणा॒म् पुरु॑षाणाꣳ रू॒पम् । \newline
2. रू॒पं ॅयद् यद् रू॒पꣳ रू॒पं ॅयत् । \newline
3. यत् तू॑प॒र स्तू॑प॒रो यद् यत् तू॑प॒रः । \newline
4. तू॒प॒र स्तत् तत् तू॑प॒र स्तू॑प॒र स्तत् । \newline
5. तदश्वा॑ना॒ मश्वा॑ना॒म् तत् तदश्वा॑नाम् । \newline
6. अश्वा॑नां॒ ॅयद् यदश्वा॑ना॒ मश्वा॑नां॒ ॅयत् । \newline
7. यद॒न्यतो॑दन् न॒न्यतो॑द॒न्॒. यद् यद॒न्यतो॑दन्न् । \newline
8. अ॒न्यतो॑द॒न् तत् तद॒न्यतो॑दन् न॒न्यतो॑द॒न् तत् । \newline
9. अ॒न्यतो॑द॒न्नित्य॒न्यतः॑ - द॒न्न् । \newline
10. तद् गवा॒म् गवा॒म् तत् तद् गवा᳚म् । \newline
11. गवां॒ ॅयद् यद् गवा॒म् गवां॒ ॅयत् । \newline
12. यदव्या॒ अव्या॒ यद् यदव्याः᳚ । \newline
13. अव्या॑ इवे॒ वाव्या॒ अव्या॑ इव । \newline
14. इ॒व॒ श॒फाः श॒फा इ॑वे व श॒फाः । \newline
15. श॒फा स्तत् तच्छ॒फाः श॒फा स्तत् । \newline
16. तदवी॑ना॒ मवी॑ना॒म् तत् तदवी॑नाम् । \newline
17. अवी॑नां॒ ॅयद् यदवी॑ना॒ मवी॑नां॒ ॅयत् । \newline
18. यद॒जो॑ ऽजो यद् यद॒जः । \newline
19. अ॒ज स्तत् तद॒जो॑ ऽज स्तत् । \newline
20. तद॒जाना॑ म॒जाना॒म् तत् तद॒जाना᳚म् । \newline
21. अ॒जानां᳚ ॅवा॒युर् वा॒यु र॒जाना॑ म॒जानां᳚ ॅवा॒युः । \newline
22. वा॒युर् वै वै वा॒युर् वा॒युर् वै । \newline
23. वै प॑शू॒नाम् प॑शू॒नां ॅवै वै प॑शू॒नाम् । \newline
24. प॒शू॒नाम् प्रि॒यम् प्रि॒यम् प॑शू॒नाम् प॑शू॒नाम् प्रि॒यम् । \newline
25. प्रि॒यम् धाम॒ धाम॑ प्रि॒यम् प्रि॒यम् धाम॑ । \newline
26. धाम॒ यद् यद् धाम॒ धाम॒ यत् । \newline
27. यद् वा॑य॒व्यो॑ वाय॒व्यो॑ यद् यद् वा॑य॒व्यः॑ । \newline
28. वा॒य॒व्यो॑ भव॑ति॒ भव॑ति वाय॒व्यो॑ वाय॒व्यो॑ भव॑ति । \newline
29. भव॑ त्ये॒त मे॒तम् भव॑ति॒ भव॑ त्ये॒तम् । \newline
30. ए॒त मे॒वैवैत मे॒त मे॒व । \newline
31. ए॒वैन॑ मेन मे॒वै वैन᳚म् । \newline
32. ए॒न॒ म॒भ्या᳚(1॒)भ्ये॑न मेन म॒भि । \newline
33. अ॒भि स॑ञ्जाना॒नाः स॑ञ्जाना॒ना अ॒भ्य॑भि स॑ञ्जाना॒नाः । \newline
34. स॒ञ्जा॒ना॒नाः प॒शवः॑ प॒शवः॑ सञ्जाना॒नाः स॑ञ्जाना॒नाः प॒शवः॑ । \newline
35. स॒ञ्जा॒ना॒ना इति॑ सं - जा॒ना॒नाः । \newline
36. प॒शव॒ उपोप॑ प॒शवः॑ प॒शव॒ उप॑ । \newline
37. उप॑ तिष्ठन्ते तिष्ठन्त॒ उपोप॑ तिष्ठन्ते । \newline
38. ति॒ष्ठ॒न्ते॒ वा॒य॒व्यो॑ वाय॒व्य॑ स्तिष्ठन्ते तिष्ठन्ते वाय॒व्यः॑ । \newline
39. वा॒य॒व्यः॑ का॒र्या(3)ः का॒र्या(3) वा॑य॒व्यो॑ वाय॒व्यः॑ का॒र्या(3)ः । \newline
40. का॒र्या(3)ः प्रा॑जाप॒त्या(3)ः प्रा॑जाप॒त्या(3)ः का॒र्या(3)ः का॒र्या(3)ः प्रा॑जाप॒त्या(3)ः । \newline
41. प्रा॒जा॒प॒त्या(3) इतीति॑ प्राजाप॒त्या(3)ः प्रा॑जाप॒त्या(3) इति॑ । \newline
42. प्रा॒जा॒प॒त्या(3) इति॑ प्राजा - प॒त्या(3)ः । \newline
43. इत्या॑हु राहु॒रि तीत्या॑हुः । \newline
44. आ॒हु॒र् यद् यदा॑हु राहु॒र् यत् । \newline
45. यद् वा॑य॒व्यं॑ ॅवाय॒व्यं॑ ॅयद् यद् वा॑य॒व्य᳚म् । \newline
46. वा॒य॒व्य॑म् कु॒र्यात् कु॒र्याद् वा॑य॒व्यं॑ ॅवाय॒व्य॑म् कु॒र्यात् । \newline
47. कु॒र्यात् प्र॒जाप॑तेः प्र॒जाप॑तेः कु॒र्यात् कु॒र्यात् प्र॒जाप॑तेः । \newline
48. प्र॒जाप॑ते रियादियात् प्र॒जाप॑तेः प्र॒जाप॑ते रियात् । \newline
49. प्र॒जाप॑ते॒रिति॑ प्र॒जा - प॒तेः॒ । \newline
50. इ॒या॒द् यद् यदि॑या दिया॒द् यत् । \newline
51. यत् प्रा॑जाप॒त्यम् प्रा॑जाप॒त्यं ॅयद् यत् प्रा॑जाप॒त्यम् । \newline
52. प्रा॒जा॒प॒त्यम् कु॒र्यात् कु॒र्यात् प्रा॑जाप॒त्यम् प्रा॑जाप॒त्यम् कु॒र्यात् । \newline
53. प्रा॒जा॒प॒त्यमिति॑ प्राजा - प॒त्यम् । \newline
54. कु॒र्याद् वा॒योर् वा॒योः कु॒र्यात् कु॒र्याद् वा॒योः । \newline
55. वा॒यो रि॑या दियाद् वा॒योर् वा॒यो रि॑यात् । \newline

\textbf{Ghana Paata } \newline

1. पुरु॑षाणाꣳ रू॒पꣳ रू॒पम् पुरु॑षाणा॒म् पुरु॑षाणाꣳ रू॒पं ॅयद् यद् रू॒पम् पुरु॑षाणा॒म् पुरु॑षाणाꣳ रू॒पं ॅयत् । \newline
2. रू॒पं ॅयद् यद् रू॒पꣳ रू॒पं ॅयत् तू॑प॒र स्तू॑प॒रो यद् रू॒पꣳ रू॒पं ॅयत् तू॑प॒रः । \newline
3. यत् तू॑प॒र स्तू॑प॒रो यद् यत् तू॑प॒र स्तत् तत् तू॑प॒रो यद् यत् तू॑प॒र स्तत् । \newline
4. तू॒प॒र स्तत् तत् तू॑प॒र स्तू॑प॒र स्तदश्वा॑ना॒ मश्वा॑ना॒म् तत् तू॑प॒र स्तू॑प॒र स्तदश्वा॑नाम् । \newline
5. तदश्वा॑ना॒ मश्वा॑ना॒म् तत् तदश्वा॑नां॒ ॅयद् यदश्वा॑ना॒म् तत् तदश्वा॑नां॒ ॅयत् । \newline
6. अश्वा॑नां॒ ॅयद् यदश्वा॑ना॒ मश्वा॑नां॒ ॅयद॒न्यतो॑दन् न॒न्यतो॑द॒न्॒. यदश्वा॑ना॒ मश्वा॑नां॒ ॅयद॒न्यतो॑दन्न् । \newline
7. यद॒न्यतो॑दन् न॒न्यतो॑द॒न्॒. यद् यद॒न्यतो॑द॒न् तत् तद॒न्यतो॑द॒न्॒. यद् यद॒न्यतो॑द॒न् तत् । \newline
8. अ॒न्यतो॑द॒न् तत् तद॒न्यतो॑दन् न॒न्यतो॑द॒न् तद् गवा॒म् गवा॒म् तद॒न्यतो॑दन् न॒न्यतो॑द॒न् तद् गवा᳚म् । \newline
9. अ॒न्यतो॑द॒न्नित्य॒न्यतः॑ - द॒न्न् । \newline
10. तद् गवा॒म् गवा॒म् तत् तद् गवां॒ ॅयद् यद् गवा॒म् तत् तद् गवां॒ ॅयत् । \newline
11. गवां॒ ॅयद् यद् गवा॒म् गवां॒ ॅयदव्या॒ अव्या॒ यद् गवा॒म् गवां॒ ॅयदव्याः᳚ । \newline
12. यदव्या॒ अव्या॒ यद् यदव्या॑ इवे॒ वाव्या॒ यद् यदव्या॑ इव । \newline
13. अव्या॑ इवे॒ वाव्या॒ अव्या॑ इव श॒फाः श॒फा इ॒वाव्या॒ अव्या॑ इव श॒फाः । \newline
14. इ॒व॒ श॒फाः श॒फा इ॑वेव श॒फा स्तत् तच्छ॒फा इ॑वेव श॒फा स्तत् । \newline
15. श॒फा स्तत् तच्छ॒फाः श॒फा स्तदवी॑ना॒ मवी॑ना॒म् तच्छ॒फाः श॒फा स्तदवी॑नाम् । \newline
16. तदवी॑ना॒ मवी॑ना॒म् तत् तदवी॑नां॒ ॅयद् यदवी॑ना॒म् तत् तदवी॑नां॒ ॅयत् । \newline
17. अवी॑नां॒ ॅयद् यदवी॑ना॒ मवी॑नां॒ ॅयद॒जो॑ ऽजो यदवी॑ना॒ मवी॑नां॒ ॅयद॒जः । \newline
18. यद॒जो॑ ऽजो यद् यद॒ज स्तत् तद॒जो यद् यद॒ज स्तत् । \newline
19. अ॒ज स्तत् तद॒जो॑ ऽजस्त द॒जाना॑ म॒जाना॒म् तद॒जो॑ ऽजस्त द॒जाना᳚म् । \newline
20. तद॒जाना॑ म॒जाना॒म् तत् तद॒जानां᳚ ॅवा॒युर् वा॒यु र॒जाना॒म् तत् तद॒जानां᳚ ॅवा॒युः । \newline
21. अ॒जानां᳚ ॅवा॒युर् वा॒यु र॒जाना॑ म॒जानां᳚ ॅवा॒युर् वै वै वा॒यु र॒जाना॑ म॒जानां᳚ ॅवा॒युर् वै । \newline
22. वा॒युर् वै वै वा॒युर् वा॒युर् वै प॑शू॒नाम् प॑शू॒नां ॅवै वा॒युर् वा॒युर् वै प॑शू॒नाम् । \newline
23. वै प॑शू॒नाम् प॑शू॒नां ॅवै वै प॑शू॒नाम् प्रि॒यम् प्रि॒यम् प॑शू॒नां ॅवै वै प॑शू॒नाम् प्रि॒यम् । \newline
24. प॒शू॒नाम् प्रि॒यम् प्रि॒यम् प॑शू॒नाम् प॑शू॒नाम् प्रि॒यम् धाम॒ धाम॑ प्रि॒यम् प॑शू॒नाम् प॑शू॒नाम् प्रि॒यम् धाम॑ । \newline
25. प्रि॒यम् धाम॒ धाम॑ प्रि॒यम् प्रि॒यम् धाम॒ यद् यद् धाम॑ प्रि॒यम् प्रि॒यम् धाम॒ यत् । \newline
26. धाम॒ यद् यद् धाम॒ धाम॒ यद् वा॑य॒व्यो॑ वाय॒व्यो॑ यद् धाम॒ धाम॒ यद् वा॑य॒व्यः॑ । \newline
27. यद् वा॑य॒व्यो॑ वाय॒व्यो॑ यद् यद् वा॑य॒व्यो॑ भव॑ति॒ भव॑ति वाय॒व्यो॑ यद् यद् वा॑य॒व्यो॑ भव॑ति । \newline
28. वा॒य॒व्यो॑ भव॑ति॒ भव॑ति वाय॒व्यो॑ वाय॒व्यो॑ भव॑ त्ये॒त मे॒तम् भव॑ति वाय॒व्यो॑ वाय॒व्यो॑ भव॑ त्ये॒तम् । \newline
29. भव॑ त्ये॒त मे॒तम् भव॑ति॒ भव॑ त्ये॒त मे॒वै वैतम् भव॑ति॒ भव॑ त्ये॒त मे॒व । \newline
30. ए॒त मे॒वै वैत मे॒त मे॒वैन॑ मेन मे॒वैत मे॒त मे॒वैन᳚म् । \newline
31. ए॒वैन॑ मेन मे॒वै वैन॑ म॒भ्या᳚(1॒)भ्ये॑न मे॒वै वैन॑ म॒भि । \newline
32. ए॒न॒ म॒भ्या᳚(1॒)भ्ये॑न मेन म॒भि स॑ञ्जाना॒नाः स॑ञ्जाना॒ना अ॒भ्ये॑न मेन म॒भि स॑ञ्जाना॒नाः । \newline
33. अ॒भि स॑ञ्जाना॒नाः स॑ञ्जाना॒ना अ॒भ्य॑भि स॑ञ्जाना॒नाः प॒शवः॑ प॒शवः॑ सञ्जाना॒ना अ॒भ्य॑भि स॑ञ्जाना॒नाः प॒शवः॑ । \newline
34. स॒ञ्जा॒ना॒नाः प॒शवः॑ प॒शवः॑ सञ्जाना॒नाः स॑ञ्जाना॒नाः प॒शव॒ उपोप॑ प॒शवः॑ सञ्जाना॒नाः स॑ञ्जाना॒नाः प॒शव॒ उप॑ । \newline
35. स॒ञ्जा॒ना॒ना इति॑ सं - जा॒ना॒नाः । \newline
36. प॒शव॒ उपोप॑ प॒शवः॑ प॒शव॒ उप॑ तिष्ठन्ते तिष्ठन्त॒ उप॑ प॒शवः॑ प॒शव॒ उप॑ तिष्ठन्ते । \newline
37. उप॑ तिष्ठन्ते तिष्ठन्त॒ उपोप॑ तिष्ठन्ते वाय॒व्यो॑ वाय॒व्य॑ स्तिष्ठन्त॒ उपोप॑ तिष्ठन्ते वाय॒व्यः॑ । \newline
38. ति॒ष्ठ॒न्ते॒ वा॒य॒व्यो॑ वाय॒व्य॑ स्तिष्ठन्ते तिष्ठन्ते वाय॒व्यः॑ का॒र्या(3)ः का॒र्या(3) वा॑य॒व्य॑ स्तिष्ठन्ते तिष्ठन्ते वाय॒व्यः॑ का॒र्या(3)ः । \newline
39. वा॒य॒व्यः॑ का॒र्या(3)ः का॒र्या(3) वा॑य॒व्यो॑ वाय॒व्यः॑ का॒र्या(3)ः प्रा॑जाप॒त्या(3)ः प्रा॑जाप॒त्या(3)ः का॒र्या(3) वा॑य॒व्यो॑ वाय॒व्यः॑ का॒र्या(3)ः प्रा॑जाप॒त्या(3)ः । \newline
40. का॒र्या(3)ः प्रा॑जाप॒त्या(3)ः प्रा॑जाप॒त्या(3)ः का॒र्या(3)ः का॒र्या(3)ः प्रा॑जाप॒त्या(3) इतीति॑ प्राजाप॒त्या(3)ः का॒र्या(3)ः का॒र्या(3)ः प्रा॑जाप॒त्या(3) इति॑ । \newline
41. प्रा॒जा॒प॒त्या(3) इतीति॑ प्राजाप॒त्या(3)ः प्रा॑जाप॒त्या(3) इत्या॑हु राहु॒रिति॑ प्राजाप॒त्या(3)ः प्रा॑जाप॒त्या(3) इत्या॑हुः । \newline
42. प्रा॒जा॒प॒त्या(3) इति॑ प्राजा - प॒त्या(3)ः । \newline
43. इत्या॑हु राहु॒रि तीत्या॑हु॒र् यद् यदा॑ हु॒रिती त्या॑हु॒र् यत् । \newline
44. आ॒हु॒र् यद् यदा॑हु राहु॒र् यद् वा॑य॒व्यं॑ ॅवाय॒व्यं॑ ॅयदा॑हु राहु॒र् यद् वा॑य॒व्य᳚म् । \newline
45. यद् वा॑य॒व्यं॑ ॅवाय॒व्यं॑ ॅयद् यद् वा॑य॒व्य॑म् कु॒र्यात् कु॒र्याद् वा॑य॒व्यं॑ ॅयद् यद् वा॑य॒व्य॑म् कु॒र्यात् । \newline
46. वा॒य॒व्य॑म् कु॒र्यात् कु॒र्याद् वा॑य॒व्यं॑ ॅवाय॒व्य॑म् कु॒र्यात् प्र॒जाप॑तेः प्र॒जाप॑तेः कु॒र्याद् वा॑य॒व्यं॑ ॅवाय॒व्य॑म् कु॒र्यात् प्र॒जाप॑तेः । \newline
47. कु॒र्यात् प्र॒जाप॑तेः प्र॒जाप॑तेः कु॒र्यात् कु॒र्यात् प्र॒जाप॑ते रिया दियात् प्र॒जाप॑तेः कु॒र्यात् कु॒र्यात् प्र॒जाप॑ते रियात् । \newline
48. प्र॒जाप॑ते रिया दियात् प्र॒जाप॑तेः प्र॒जाप॑ते रिया॒द् यद् यदि॑यात् प्र॒जाप॑तेः प्र॒जाप॑ते रिया॒द् यत् । \newline
49. प्र॒जाप॑ते॒रिति॑ प्र॒जा - प॒तेः॒ । \newline
50. इ॒या॒द् यद् यदि॑या दिया॒द् यत् प्रा॑जाप॒त्यम् प्रा॑जाप॒त्यं ॅयदि॑या दिया॒द् यत् प्रा॑जाप॒त्यम् । \newline
51. यत् प्रा॑जाप॒त्यम् प्रा॑जाप॒त्यं ॅयद् यत् प्रा॑जाप॒त्यम् कु॒र्यात् कु॒र्यात् प्रा॑जाप॒त्यं ॅयद् यत् प्रा॑जाप॒त्यम् कु॒र्यात् । \newline
52. प्रा॒जा॒प॒त्यम् कु॒र्यात् कु॒र्यात् प्रा॑जाप॒त्यम् प्रा॑जाप॒त्यम् कु॒र्याद् वा॒योर् वा॒योः कु॒र्यात् प्रा॑जाप॒त्यम् प्रा॑जाप॒त्यम् कु॒र्याद् वा॒योः । \newline
53. प्रा॒जा॒प॒त्यमिति॑ प्राजा - प॒त्यम् । \newline
54. कु॒र्याद् वा॒योर् वा॒योः कु॒र्यात् कु॒र्याद् वा॒यो रि॑या दियाद् वा॒योः कु॒र्यात् कु॒र्याद् वा॒यो रि॑यात् । \newline
55. वा॒यो रि॑या दियाद् वा॒योर् वा॒यो रि॑या॒द् यद् यदि॑याद् वा॒योर् वा॒यो रि॑या॒द् यत् । \newline
\pagebreak
\markright{ TS 5.5.1.4  \hfill https://www.vedavms.in \hfill}

\section{ TS 5.5.1.4 }

\textbf{TS 5.5.1.4 } \newline
\textbf{Samhita Paata} \newline

-रि॑या॒द्-यद्-वा॑य॒व्यः॑ प॒शुर्भव॑ति॒ तेन॑ वा॒योर्नैति॒ यत् प्रा॑जाप॒त्यः पु॑रो॒डाशो॒ भव॑ति॒ तेन॑ प्रा॒जाप॑ते॒र्नैति॒ यद् द्वाद॑शकपाल॒स्तेन॑ वैश्वान॒रान्नैत्या᳚ग्ना वैष्ण॒वमेका॑दश-कपालं॒ निर्व॑पति दीक्षि॒ष्यमा॑णो॒ऽग्निः सर्वा॑ दे॒वता॒ विष्णु॑र्य॒ज्ञो दे॒वता᳚श्चै॒व य॒ज्ञ्ं चाऽऽ* र॑भते॒ऽग्निर॑व॒मो दे॒वता॑नां॒ ॅविष्णुः॑ पर॒मो यदा᳚ग्ना-वैष्ण॒व-मेका॑दशकपालं नि॒र्वपति दे॒वता॑ - [  ] \newline

\textbf{Pada Paata} \newline

इ॒या॒त् । यत् । वा॒य॒व्यः॑ । प॒शुः । भव॑ति । तेन॑ । वा॒योः । न । ए॒ति॒ । यत् । प्रा॒जा॒प॒त्य इति॑ प्राजा - प॒त्यः । पु॒रो॒डाशः॑ । भव॑ति । तेन॑ । प्र॒जाप॑ते॒रिति॑ प्र॒जा - प॒तेः॒ । न । ए॒ति॒ । यत् । द्वाद॑शकपाल॒ इति॒ द्वाद॑श - क॒पा॒लः॒ । तेन॑ । वै॒श्वा॒न॒रात् । न । ए॒ति॒ । आ॒ग्ना॒वै॒ष्ण॒वमित्या᳚ग्ना - वै॒ष्ण॒वम् । एका॑दशकपाल॒मित्येका॑दश- क॒पा॒ल॒म् । निरिति॑ । व॒प॒ति॒ । दी॒क्षि॒ष्यमा॑णः । अ॒ग्निः । सर्वाः᳚ । दे॒वताः᳚ । विष्णुः॑ । य॒ज्ञ्ः । दे॒वताः᳚ । च॒ । ए॒व । य॒ज्ञ्म् । च॒ । एति॑ । र॒भ॒ते॒ । अ॒ग्निः । अ॒व॒मः । दे॒वता॑नाम् । विष्णुः॑ । प॒र॒मः । यत् । आ॒ग्ना॒वै॒ष्ण॒वमित्या᳚ग्ना - वै॒ष्ण॒वम् । एका॑दशकपाल॒मित्येका॑दश-क॒पा॒ल॒म् । नि॒र्वप॒तीति॑ निः - वप॑ति । दे॒वताः᳚ ।  \newline


\textbf{Krama Paata} \newline

इ॒या॒द् यत् । यद् वा॑य॒व्यः॑ । वा॒य॒व्यः॑ प॒शुः । प॒शुर् भव॑ति । भव॑ति॒ तेन॑ । तेन॑ वा॒योः । वा॒योर् न । नैति॑ । ए॒ति॒ यत् । यत् प्रा॑जाप॒त्यः । प्रा॒जा॒प॒त्यः पु॑रो॒डाशः॑ । प्रा॒जा॒प॒त्य इति॑ प्राजा - प॒त्यः । पु॒रो॒डाशो॒ भव॑ति । भव॑ति॒ तेन॑ । तेन॑ प्र॒जाप॑तेः । प्र॒जाप॑ते॒र् न । प्र॒जाप॑ते॒रिति॑ प्र॒जा - प॒तेः॒ । नैति॑ । ए॒ति॒ यत् । यद् द्वाद॑शकपालः । द्वाद॑शकपाल॒स्तेन॑ । द्वाद॑शकपाल॒ इति॒ द्वाद॑श - क॒पा॒लः॒ । तेन॑ वैश्वान॒रात् । वै॒श्वा॒न॒रान् न । नैति॑ । ए॒त्या॒ग्ना॒वै॒ष्ण॒वम् । आ॒ग्ना॒वै॒ष्ण॒वमेका॑दशकपालम् । आ॒ग्ना॒वै॒ष्ण॒वमित्या᳚ग्ना - वै॒ष्ण॒वम् । एका॑दशकपाल॒म् निः । एका॑दशकपाल॒मित्येका॑दश - क॒पा॒ल॒म् । निर् व॑पति । व॒प॒ति॒ दी॒क्षि॒ष्यमा॑णः । दी॒क्षि॒ष्यमा॑णो॒ऽग्निः । अ॒ग्निः सर्वाः᳚ । सर्वा॑ दे॒वताः᳚ । दे॒वता॒ विष्णुः॑ । विष्णु॑र् य॒ज्ञ्ः । य॒ज्ञो दे॒वताः᳚ । दे॒वता᳚श्च । चै॒व । ए॒व य॒ज्ञ्म् । य॒ज्ञ्म् च॑ । चा । आ र॑भते । र॒भ॒ते॒ऽग्निः । अ॒ग्निर॑व॒मः । अ॒व॒मो दे॒वता॑नाम् । दे॒वता॑ना॒म् ॅविष्णुः॑ । विष्णुः॑ पर॒मः । प॒र॒मो यत् । यदा᳚ग्नावैष्ण॒वम् । आ॒ग्ना॒वै॒ष्ण॒व,मेका॑दशकपालम् । आ॒ग्ना॒वै॒ष्ण॒वमित्या᳚ग्ना - वै॒ष्ण॒वम् । एका॑दशकपालम् नि॒र्वप॑ति । एका॑दशकपाल॒मित्येका॑दश - क॒पा॒ल॒म् । नि॒र्वप॑ति दे॒वताः᳚ । नि॒र्वप॒तीति॑ निः - वप॑ति । दे॒वता॑ ए॒व \newline

\textbf{Jatai Paata} \newline

1. इ॒या॒द् यद् यदि॑या दिया॒द् यत् । \newline
2. यद् वा॑य॒व्यो॑ वाय॒व्यो॑ यद् यद् वा॑य॒व्यः॑ । \newline
3. वा॒य॒व्यः॑ प॒शुः प॒शुर् वा॑य॒व्यो॑ वाय॒व्यः॑ प॒शुः । \newline
4. प॒शुर् भव॑ति॒ भव॑ति प॒शुः प॒शुर् भव॑ति । \newline
5. भव॑ति॒ तेन॒ तेन॒ भव॑ति॒ भव॑ति॒ तेन॑ । \newline
6. तेन॑ वा॒योर् वा॒यो स्तेन॒ तेन॑ वा॒योः । \newline
7. वा॒योर् न न वा॒योर् वा॒योर् न । \newline
8. नैत्ये॑ति॒ न नैति॑ । \newline
9. ए॒ति॒ यद् यदे᳚त्येति॒ यत् । \newline
10. यत् प्रा॑जाप॒त्यः प्रा॑जाप॒त्यो यद् यत् प्रा॑जाप॒त्यः । \newline
11. प्रा॒जा॒प॒त्यः पु॑रो॒डाशः॑ पुरो॒डाशः॑ प्राजाप॒त्यः प्रा॑जाप॒त्यः पु॑रो॒डाशः॑ । \newline
12. प्रा॒जा॒प॒त्य इति॑ प्राजा - प॒त्यः । \newline
13. पु॒रो॒डाशो॒ भव॑ति॒ भव॑ति पुरो॒डाशः॑ पुरो॒डाशो॒ भव॑ति । \newline
14. भव॑ति॒ तेन॒ तेन॒ भव॑ति॒ भव॑ति॒ तेन॑ । \newline
15. तेन॑ प्र॒जाप॑तेः प्र॒जाप॑ते॒ स्तेन॒ तेन॑ प्र॒जाप॑तेः । \newline
16. प्र॒जाप॑ते॒र् न न प्र॒जाप॑तेः प्र॒जाप॑ते॒र् न । \newline
17. प्र॒जाप॑ते॒रिति॑ प्र॒जा - प॒तेः॒ । \newline
18. नैत्ये॑ति॒ न नैति॑ । \newline
19. ए॒ति॒ यद् यदे᳚त्येति॒ यत् । \newline
20. यद् द्वाद॑शकपालो॒ द्वाद॑शकपालो॒ यद् यद् द्वाद॑शकपालः । \newline
21. द्वाद॑शकपाल॒ स्तेन॒ तेन॒ द्वाद॑शकपालो॒ द्वाद॑शकपाल॒ स्तेन॑ । \newline
22. द्वाद॑शकपाल॒ इति॒ द्वाद॑श - क॒पा॒लः॒ । \newline
23. तेन॑ वैश्वान॒राद् वै᳚श्वान॒रात् तेन॒ तेन॑ वैश्वान॒रात् । \newline
24. वै॒श्वा॒न॒रान् न न वै᳚श्वान॒राद् वै᳚श्वान॒रान् न । \newline
25. नैत्ये॑ति॒ न नैति॑ । \newline
26. ए॒त्या॒ग्ना॒वै॒ष्ण॒व मा᳚ग्नावैष्ण॒व मे᳚त्ये त्याग्नावैष्ण॒वम् । \newline
27. आ॒ग्ना॒वै॒ष्ण॒व मेका॑दशकपाल॒ मेका॑दशकपाल माग्नावैष्ण॒व मा᳚ग्नावैष्ण॒व मेका॑दशकपालम् । \newline
28. आ॒ग्ना॒वै॒ष्ण॒वमित्या᳚ग्ना - वै॒ष्ण॒वम् । \newline
29. एका॑दशकपाल॒म् निर् णिरेका॑दशकपाल॒ मेका॑दशकपाल॒म् निः । \newline
30. एका॑दशकपाल॒मित्येका॑दश - क॒पा॒ल॒म् । \newline
31. निर् व॑पति वपति॒ निर् णिर् व॑पति । \newline
32. व॒प॒ति॒ दी॒क्षि॒ष्यमा॑णो दीक्षि॒ष्यमा॑णो वपति वपति दीक्षि॒ष्यमा॑णः । \newline
33. दी॒क्षि॒ष्यमा॑णो॒ ऽग्नि र॒ग्निर् दी᳚क्षि॒ष्यमा॑णो दीक्षि॒ष्यमा॑णो॒ ऽग्निः । \newline
34. अ॒ग्निः सर्वाः॒ सर्वा॑ अ॒ग्नि र॒ग्निः सर्वाः᳚ । \newline
35. सर्वा॑ दे॒वता॑ दे॒वताः॒ सर्वाः॒ सर्वा॑ दे॒वताः᳚ । \newline
36. दे॒वता॒ विष्णु॒र् विष्णु॑र् दे॒वता॑ दे॒वता॒ विष्णुः॑ । \newline
37. विष्णु॑र् य॒ज्ञो य॒ज्ञो विष्णु॒र् विष्णु॑र् य॒ज्ञ्ः । \newline
38. य॒ज्ञो दे॒वता॑ दे॒वता॑ य॒ज्ञो य॒ज्ञो दे॒वताः᳚ । \newline
39. दे॒वता᳚श्च च दे॒वता॑ दे॒वता᳚श्च । \newline
40. चै॒वैव च॑ चै॒व । \newline
41. ए॒व य॒ज्ञ्ं ॅय॒ज्ञ् मे॒वैव य॒ज्ञ्म् । \newline
42. य॒ज्ञ्म् च॑ च य॒ज्ञ्ं ॅय॒ज्ञ्म् च॑ । \newline
43. चा च॒ चा । \newline
44. आ र॑भते रभत॒ आ र॑भते । \newline
45. र॒भ॒ते॒ ऽग्नि र॒ग्नी र॑भते रभते॒ ऽग्निः । \newline
46. अ॒ग्नि र॑व॒मो॑ ऽव॒मो᳚ ऽग्नि र॒ग्नि र॑व॒मः । \newline
47. अ॒व॒मो दे॒वता॑नाम् दे॒वता॑ना मव॒मो॑ ऽव॒मो दे॒वता॑नाम् । \newline
48. दे॒वता॑नां॒ ॅविष्णु॒र् विष्णु॑र् दे॒वता॑नाम् दे॒वता॑नां॒ ॅविष्णुः॑ । \newline
49. विष्णुः॑ पर॒मः प॑र॒मो विष्णु॒र् विष्णुः॑ पर॒मः । \newline
50. प॒र॒मो यद् यत् प॑र॒मः प॑र॒मो यत् । \newline
51. यदा᳚ग्नावैष्ण॒व मा᳚ग्नावैष्ण॒वं ॅयद् यदा᳚ग्नावैष्ण॒वम् । \newline
52. आ॒ग्ना॒वै॒ष्ण॒व मेका॑दशकपाल॒ मेका॑दशकपाल माग्नावैष्ण॒व मा᳚ग्नावैष्ण॒व मेका॑दशकपालम् । \newline
53. आ॒ग्ना॒वै॒ष्ण॒वमित्या᳚ग्ना - वै॒ष्ण॒वम् । \newline
54. एका॑दशकपालन् नि॒र्वप॒ति नि॒र्वप॒ त्येका॑दशकपाल॒ मेका॑दशकपालन् नि॒र्वप॒ति । \newline
55. एका॑दशकपाल॒मित्येका॑दश - क॒पा॒ल॒म् । \newline
56. नि॒र्वप॒ति दे॒वता॑ दे॒वता॑ नि॒र्वप॒ति नि॒र्वप॒ति दे॒वताः᳚ । \newline
57. नि॒र्वप॒तीति॑ निः - वप॑ति । \newline
58. दे॒वता॑ ए॒वैव दे॒वता॑ दे॒वता॑ ए॒व । \newline

\textbf{Ghana Paata } \newline

1. इ॒या॒द् यद् यदि॑या दिया॒द् यद् वा॑य॒व्यो॑ वाय॒व्यो॑ यदि॑या दिया॒द् यद् वा॑य॒व्यः॑ । \newline
2. यद् वा॑य॒व्यो॑ वाय॒व्यो॑ यद् यद् वा॑य॒व्यः॑ प॒शुः प॒शुर् वा॑य॒व्यो॑ यद् यद् वा॑य॒व्यः॑ प॒शुः । \newline
3. वा॒य॒व्यः॑ प॒शुः प॒शुर् वा॑य॒व्यो॑ वाय॒व्यः॑ प॒शुर् भव॑ति॒ भव॑ति प॒शुर् वा॑य॒व्यो॑ वाय॒व्यः॑ प॒शुर् भव॑ति । \newline
4. प॒शुर् भव॑ति॒ भव॑ति प॒शुः प॒शुर् भव॑ति॒ तेन॒ तेन॒ भव॑ति प॒शुः प॒शुर् भव॑ति॒ तेन॑ । \newline
5. भव॑ति॒ तेन॒ तेन॒ भव॑ति॒ भव॑ति॒ तेन॑ वा॒योर् वा॒यो स्तेन॒ भव॑ति॒ भव॑ति॒ तेन॑ वा॒योः । \newline
6. तेन॑ वा॒योर् वा॒यो स्तेन॒ तेन॑ वा॒योर् न न वा॒यो स्तेन॒ तेन॑ वा॒योर् न । \newline
7. वा॒योर् न न वा॒योर् वा॒योर् नैत्ये॑ति॒ न वा॒योर् वा॒योर् नैति॑ । \newline
8. नैत्ये॑ति॒ न नैति॒ यद् यदे॑ति॒ न नैति॒ यत् । \newline
9. ए॒ति॒ यद् यदे᳚त्येति॒ यत् प्रा॑जाप॒त्यः प्रा॑जाप॒त्यो यदे᳚त्येति॒ यत् प्रा॑जाप॒त्यः । \newline
10. यत् प्रा॑जाप॒त्यः प्रा॑जाप॒त्यो यद् यत् प्रा॑जाप॒त्यः पु॑रो॒डाशः॑ पुरो॒डाशः॑ प्राजाप॒त्यो यद् यत् प्रा॑जाप॒त्यः पु॑रो॒डाशः॑ । \newline
11. प्रा॒जा॒प॒त्यः पु॑रो॒डाशः॑ पुरो॒डाशः॑ प्राजाप॒त्यः प्रा॑जाप॒त्यः पु॑रो॒डाशो॒ भव॑ति॒ भव॑ति पुरो॒डाशः॑ प्राजाप॒त्यः प्रा॑जाप॒त्यः पु॑रो॒डाशो॒ भव॑ति । \newline
12. प्रा॒जा॒प॒त्य इति॑ प्राजा - प॒त्यः । \newline
13. पु॒रो॒डाशो॒ भव॑ति॒ भव॑ति पुरो॒डाशः॑ पुरो॒डाशो॒ भव॑ति॒ तेन॒ तेन॒ भव॑ति पुरो॒डाशः॑ पुरो॒डाशो॒ भव॑ति॒ तेन॑ । \newline
14. भव॑ति॒ तेन॒ तेन॒ भव॑ति॒ भव॑ति॒ तेन॑ प्र॒जाप॑तेः प्र॒जाप॑ते॒ स्तेन॒ भव॑ति॒ भव॑ति॒ तेन॑ प्र॒जाप॑तेः । \newline
15. तेन॑ प्र॒जाप॑तेः प्र॒जाप॑ते॒ स्तेन॒ तेन॑ प्र॒जाप॑ते॒र् न न प्र॒जाप॑ते॒ स्तेन॒ तेन॑ प्र॒जाप॑ते॒र् न । \newline
16. प्र॒जाप॑ते॒र् न न प्र॒जाप॑तेः प्र॒जाप॑ते॒र् नैत्ये॑ति॒ न प्र॒जाप॑तेः प्र॒जाप॑ते॒र् नैति॑ । \newline
17. प्र॒जाप॑ते॒रिति॑ प्र॒जा - प॒तेः॒ । \newline
18. नैत्ये॑ति॒ न नैति॒ यद् यदे॑ति॒ न नैति॒ यत् । \newline
19. ए॒ति॒ यद् यदे᳚त्येति॒ यद् द्वाद॑शकपालो॒ द्वाद॑शकपालो॒ यदे᳚त्येति॒ यद् द्वाद॑शकपालः । \newline
20. यद् द्वाद॑शकपालो॒ द्वाद॑शकपालो॒ यद् यद् द्वाद॑शकपाल॒ स्तेन॒ तेन॒ द्वाद॑शकपालो॒ यद् यद् द्वाद॑शकपाल॒ स्तेन॑ । \newline
21. द्वाद॑शकपाल॒ स्तेन॒ तेन॒ द्वाद॑शकपालो॒ द्वाद॑शकपाल॒ स्तेन॑ वैश्वान॒राद् वै᳚श्वान॒रात् तेन॒ द्वाद॑शकपालो॒ द्वाद॑शकपाल॒ स्तेन॑ वैश्वान॒रात् । \newline
22. द्वाद॑शकपाल॒ इति॒ द्वाद॑श - क॒पा॒लः॒ । \newline
23. तेन॑ वैश्वान॒राद् वै᳚श्वान॒रात् तेन॒ तेन॑ वैश्वान॒रान् न न वै᳚श्वान॒रात् तेन॒ तेन॑ वैश्वान॒रान् न । \newline
24. वै॒श्वा॒न॒रान् न न वै᳚श्वान॒राद् वै᳚श्वान॒रान् नैत्ये॑ति॒ न वै᳚श्वान॒राद् वै᳚श्वान॒रान् नैति॑ । \newline
25. नैत्ये॑ति॒ न नैत्या᳚ ग्नावैष्ण॒व मा᳚ग्नावैष्ण॒व मे॑ति॒ न नैत्या᳚ ग्नावैष्ण॒वम् । \newline
26. ए॒त्या॒ग्ना॒वै॒ष्ण॒व मा᳚ग्नावैष्ण॒व मे᳚त्येत्या ग्नावैष्ण॒व मेका॑दशकपाल॒ मेका॑दशकपाल माग्नावैष्ण॒व मे᳚त्ये त्याग्नावैष्ण॒व मेका॑दशकपालम् । \newline
27. आ॒ग्ना॒वै॒ष्ण॒व मेका॑दशकपाल॒ मेका॑दशकपाल माग्नावैष्ण॒व मा᳚ग्नावैष्ण॒व मेका॑दशकपाल॒म् निर् णिरेका॑दशकपाल माग्नावैष्ण॒व मा᳚ग्नावैष्ण॒व मेका॑दशकपाल॒म् निः । \newline
28. आ॒ग्ना॒वै॒ष्ण॒वमित्या᳚ग्ना - वै॒ष्ण॒वम् । \newline
29. एका॑दशकपाल॒म् निर् णिरेका॑दशकपाल॒ मेका॑दशकपाल॒म् निर् व॑पति वपति॒ निरेका॑दशकपाल॒ मेका॑दशकपाल॒म् निर् व॑पति । \newline
30. एका॑दशकपाल॒मित्येका॑दश - क॒पा॒ल॒म् । \newline
31. निर् व॑पति वपति॒ निर् णिर् व॑पति दीक्षि॒ष्यमा॑णो दीक्षि॒ष्यमा॑णो वपति॒ निर् णिर् व॑पति दीक्षि॒ष्यमा॑णः । \newline
32. व॒प॒ति॒ दी॒क्षि॒ष्यमा॑णो दीक्षि॒ष्यमा॑णो वपति वपति दीक्षि॒ष्यमा॑णो॒ ऽग्नि र॒ग्निर् दी᳚क्षि॒ष्यमा॑णो वपति वपति दीक्षि॒ष्यमा॑णो॒ ऽग्निः । \newline
33. दी॒क्षि॒ष्यमा॑णो॒ ऽग्नि र॒ग्निर् दी᳚क्षि॒ष्यमा॑णो दीक्षि॒ष्यमा॑णो॒ ऽग्निः सर्वाः॒ सर्वा॑ अ॒ग्निर् दी᳚क्षि॒ष्यमा॑णो दीक्षि॒ष्यमा॑णो॒ ऽग्निः सर्वाः᳚ । \newline
34. अ॒ग्निः सर्वाः॒ सर्वा॑ अ॒ग्नि र॒ग्निः सर्वा॑ दे॒वता॑ दे॒वताः॒ सर्वा॑ अ॒ग्नि र॒ग्निः सर्वा॑ दे॒वताः᳚ । \newline
35. सर्वा॑ दे॒वता॑ दे॒वताः॒ सर्वाः॒ सर्वा॑ दे॒वता॒ विष्णु॒र् विष्णु॑र् दे॒वताः॒ सर्वाः॒ सर्वा॑ दे॒वता॒ विष्णुः॑ । \newline
36. दे॒वता॒ विष्णु॒र् विष्णु॑र् दे॒वता॑ दे॒वता॒ विष्णु॑र् य॒ज्ञो य॒ज्ञो विष्णु॑र् दे॒वता॑ दे॒वता॒ विष्णु॑र् य॒ज्ञ्ः । \newline
37. विष्णु॑र् य॒ज्ञो य॒ज्ञो विष्णु॒र् विष्णु॑र् य॒ज्ञो दे॒वता॑ दे॒वता॑ य॒ज्ञो विष्णु॒र् विष्णु॑र् य॒ज्ञो दे॒वताः᳚ । \newline
38. य॒ज्ञो दे॒वता॑ दे॒वता॑ य॒ज्ञो य॒ज्ञो दे॒वता᳚श्च च दे॒वता॑ य॒ज्ञो य॒ज्ञो दे॒वता᳚श्च । \newline
39. दे॒वता᳚श्च च दे॒वता॑ दे॒वता᳚श्चै॒ वैव च॑ दे॒वता॑ दे॒वता᳚श्चै॒व । \newline
40. चै॒वैव च॑ चै॒व य॒ज्ञ्ं ॅय॒ज्ञ् मे॒व च॑ चै॒व य॒ज्ञ्म् । \newline
41. ए॒व य॒ज्ञ्ं ॅय॒ज्ञ् मे॒वैव य॒ज्ञ्म् च॑ च य॒ज्ञ् मे॒वैव य॒ज्ञ्म् च॑ । \newline
42. य॒ज्ञ्म् च॑ च य॒ज्ञ्ं ॅय॒ज्ञ्म् चा च॑ य॒ज्ञ्ं ॅय॒ज्ञ्म् चा । \newline
43. चा च॒ चा र॑भते रभत॒ आ च॒ चा र॑भते । \newline
44. आ र॑भते रभत॒ आ र॑भते॒ ऽग्नि र॒ग्नी र॑भत॒ आ र॑भते॒ ऽग्निः । \newline
45. र॒भ॒ते॒ ऽग्नि र॒ग्नी र॑भते रभते॒ ऽग्नि र॑व॒मो॑ ऽव॒मो᳚ ऽग्नी र॑भते रभते॒ ऽग्नि र॑व॒मः । \newline
46. आ॒ग्नि र॑व॒मो॑ ऽव॒मो᳚ ऽग्नि र॒ग्नि र॑व॒मो दे॒वता॑नाम् दे॒वता॑ना मव॒मो᳚ ऽग्नि र॒ग्नि र॑व॒मो दे॒वता॑नाम् । \newline
47. अ॒व॒मो दे॒वता॑नाम् दे॒वता॑ना मव॒मो॑ ऽव॒मो दे॒वता॑नां॒ ॅविष्णु॒र् विष्णु॑र् दे॒वता॑ना मव॒मो॑ ऽव॒मो दे॒वता॑नां॒ ॅविष्णुः॑ । \newline
48. दे॒वता॑नां॒ ॅविष्णु॒र् विष्णु॑र् दे॒वता॑नाम् दे॒वता॑नां॒ ॅविष्णुः॑ पर॒मः प॑र॒मो विष्णु॑र् दे॒वता॑नाम् दे॒वता॑नां॒ ॅविष्णुः॑ पर॒मः । \newline
49. विष्णुः॑ पर॒मः प॑र॒मो विष्णु॒र् विष्णुः॑ पर॒मो यद् यत् प॑र॒मो विष्णु॒र् विष्णुः॑ पर॒मो यत् । \newline
50. प॒र॒मो यद् यत् प॑र॒मः प॑र॒मो यदा᳚ग्नावैष्ण॒व मा᳚ग्नावैष्ण॒वं ॅयत् प॑र॒मः प॑र॒मो यदा᳚ग्नावैष्ण॒वम् । \newline
51. यदा᳚ग्नावैष्ण॒व मा᳚ग्नावैष्ण॒वं ॅयद् यदा᳚ग्नावैष्ण॒व मेका॑दशकपाल॒ मेका॑दशकपाल माग्नावैष्ण॒वं ॅयद् यदा᳚ग्नावैष्ण॒व मेका॑दशकपालम् । \newline
52. आ॒ग्ना॒वै॒ष्ण॒व मेका॑दशकपाल॒ मेका॑दशकपाल माग्नावैष्ण॒व मा᳚ग्नावैष्ण॒व मेका॑दशकपालन् नि॒र्वप॒ति नि॒र्वप॒ त्येका॑दशकपाल माग्नावैष्ण॒व मा᳚ग्नावैष्ण॒व मेका॑दशकपालन् नि॒र्वप॒ति । \newline
53. आ॒ग्ना॒वै॒ष्ण॒वमित्या᳚ग्ना - वै॒ष्ण॒वम् । \newline
54. एका॑दशकपालन् नि॒र्वप॒ति नि॒र्वप॒ त्येका॑दशकपाल॒ मेका॑दशकपालन् नि॒र्वप॒ति दे॒वता॑ दे॒वता॑ नि॒र्वप॒ त्येका॑दशकपाल॒ मेका॑दशकपालन् नि॒र्वप॒ति दे॒वताः᳚ । \newline
55. एका॑दशकपाल॒मित्येका॑दश - क॒पा॒ल॒म् । \newline
56. नि॒र्वप॒ति दे॒वता॑ दे॒वता॑ नि॒र्वप॒ति नि॒र्वप॒ति दे॒वता॑ ए॒वैव दे॒वता॑ नि॒र्वप॒ति नि॒र्वप॒ति दे॒वता॑ ए॒व । \newline
57. नि॒र्वप॒तीति॑ निः - वप॑ति । \newline
58. दे॒वता॑ ए॒वैव दे॒वता॑ दे॒वता॑ ए॒वोभ॒यत॑ उभ॒यत॑ ए॒व दे॒वता॑ दे॒वता॑ ए॒वोभ॒यतः॑ । \newline
\pagebreak
\markright{ TS 5.5.1.5  \hfill https://www.vedavms.in \hfill}

\section{ TS 5.5.1.5 }

\textbf{TS 5.5.1.5 } \newline
\textbf{Samhita Paata} \newline

ए॒वोभ॒यतः॑ परि॒गृह्य॒ यज॑मा॒नोऽव॑ रुन्धे पुरो॒डाशे॑न॒ वै दे॒वा अ॒मुष्मि॑न् ॅलो॒क आ᳚र्द्ध्नुवन् च॒रुणा॒ऽस्मिन्. यः का॒मये॑ता॒ऽमुष्मि॑न् ॅलो॒क ऋ॑द्ध्नुया॒मिति॒ स पु॑रो॒डाशं॑ कुर्वीता॒ऽमुष्मि॑न्ने॒व लो॒क ऋ॑द्ध्नोति॒ यद॒ष्टाक॑पाल॒स्तेना᳚ऽऽ*ग्ने॒यो यत् त्रि॑कपा॒लस्तेन॑ वैष्ण॒वः समृ॑द्ध्यै॒ यः का॒मये॑ता॒स्मिन् ॅलो॒क ऋ॑द्ध्नुया॒मिति॒ स च॒रुं कु॑र्वीता॒ग्नेर्घृ॒तं ॅविष्णो᳚स्तण्डु॒ला-स्तस्मा᳚ - [  ] \newline

\textbf{Pada Paata} \newline

ए॒व । उ॒भ॒यतः॑ । प॒रि॒गृह्येति॑ परि-गृह्य॑ । यज॑मानः । अवेति॑ । रु॒न्धे॒ । पु॒रो॒डाशे॑न । वै । दे॒वाः । अ॒मुष्मिन्न्॑ । लो॒के । आ॒द्र्ध्नु॒व॒न्न् । च॒रुणा᳚ । अ॒स्मिन्न् । यः । का॒मये॑त । अ॒मुष्मिन्न्॑ । लो॒के । ऋ॒द्ध्नु॒या॒म् । इति॑ । सः । पु॒रो॒डाश᳚म् । कु॒र्वी॒त॒ । अ॒मुष्मिन्न्॑ । ए॒व । लो॒के । ऋ॒द्ध्नो॒ति॒ । यत् । अ॒ष्टाक॑पाल॒ इत्य॒ष्टा - क॒पा॒लः॒ । तेन॑ । आ॒ग्ने॒यः । यत् । त्रि॒क॒पा॒ल इति॑ त्रि - क॒पा॒लः । तेन॑ । वै॒ष्ण॒वः । समृ॑द्ध्या॒ इति॒ सं - ऋ॒द्ध्यै॒ । यः । का॒मये॑त । अ॒स्मिन्न् । लो॒के । ऋ॒द्ध्नु॒या॒म् । इति॑ । सः । च॒रुम् । कु॒र्वी॒त॒ । अ॒ग्नेः । घृ॒तम् । विष्णोः᳚ । त॒ण्डु॒लाः । तस्मा᳚त् ।  \newline


\textbf{Krama Paata} \newline

ए॒वोभ॒यतः॑ । उ॒भ॒यतः॑ परि॒गृह्य॑ । प॒रि॒गृह्य॒ यज॑मानः । प॒रि॒गृह्येति॑ परि - गृह्य॑ । यज॑मा॒नोऽव॑ । अव॑ रुन्धे । रु॒न्धे॒ पु॒रो॒डाशे॑न । पु॒रो॒डाशे॑न॒ वै । वै दे॒वाः । दे॒वा अ॒मुष्मिन्न्॑ । अ॒मुष्मि॑न् ॅलो॒के । लो॒क आ᳚र्द्ध्नुवन्न् । आ॒र्द्ध्नु॒व॒न् च॒रुणा᳚ । च॒रुणा॒ऽस्मिन्न् । अ॒स्मिन्. यः । यः का॒मये॑त । का॒मये॑ता॒मुष्मिन्न्॑ । अ॒मुष्मि॑न् ॅलो॒के । लो॒क ऋ॑द्ध्नुयाम् । ऋ॒द्ध्नु॒या॒मिति॑ । इति॒ सः । स पु॑रो॒डाश᳚म् । पु॒रो॒डाश॑म् कुर्वीत । कु॒र्वी॒ता॒मुष्मिन्न्॑ । अ॒मुष्मि॑न्ने॒व । ए॒व लो॒के । लो॒क ऋ॑द्ध्नोति । ऋ॒द्ध्नो॒ति॒ यत् । यद॒ष्टाक॑पालः । अ॒ष्टाक॑पाल॒स्तेन॑ । अ॒ष्टाक॑पाल॒ इत्य॒ष्टा - क॒पा॒लः॒ । तेना᳚ग्ने॒यः । आ॒ग्ने॒यो यत् । यत् त्रि॑कपा॒लः । त्रि॒क॒पा॒लस्तेन॑ । त्रि॒क॒पा॒ल इति॑ त्रि - क॒पा॒लः । तेन॑ वैष्ण॒वः । वै॒ष्ण॒वः समृ॑द्ध्यै । समृ॑द्ध्यै॒ यः । समृ॑द्ध्या॒ इति॒ सम् - ऋ॒द्ध्यै॒ । यः का॒मये॑त । का॒मये॑ता॒स्मिन्न् । अ॒स्मिन् ॅलो॒के । लो॒क ऋ॑द्ध्नुयाम् । ऋ॒द्ध्नु॒या॒मिति॑ । इति॒ सः । स च॒रुम् । च॒रुम् कु॑र्वीत । कु॒र्वी॒ता॒ग्नेः । अ॒ग्नेर् घृ॒तम् । घृ॒तम् ॅविष्णोः᳚ । विष्णो᳚स्तण्डु॒लाः । त॒ण्डु॒लास्तस्मा᳚त् । तस्मा᳚च् च॒रुः \newline

\textbf{Jatai Paata} \newline

1. ए॒वोभ॒यत॑ उभ॒यत॑ ए॒वै वोभ॒यतः॑ । \newline
2. उ॒भ॒यतः॑ परि॒गृह्य॑ परि॒गृ ह्यो॑भ॒यत॑ उभ॒यतः॑ परि॒गृह्य॑ । \newline
3. प॒रि॒गृह्य॒ यज॑मानो॒ यज॑मानः परि॒गृह्य॑ परि॒गृह्य॒ यज॑मानः । \newline
4. प॒रि॒गृह्येति॑ परि - गृह्य॑ । \newline
5. यज॑मा॒नो ऽवाव॒ यज॑मानो॒ यज॑मा॒नो ऽव॑ । \newline
6. अव॑ रुन्धे रु॒न्धे ऽवाव॑ रुन्धे । \newline
7. रु॒न्धे॒ पु॒रो॒डाशे॑न पुरो॒डाशे॑न रुन्धे रुन्धे पुरो॒डाशे॑न । \newline
8. पु॒रो॒डाशे॑न॒ वै वै पु॑रो॒डाशे॑न पुरो॒डाशे॑न॒ वै । \newline
9. वै दे॒वा दे॒वा वै वै दे॒वाः । \newline
10. दे॒वा अ॒मुष्मि॑न् न॒मुष्मि॑न् दे॒वा दे॒वा अ॒मुष्मिन्न्॑ । \newline
11. अ॒मुष्मि॑न् ॅलो॒के लो॒के॑ ऽमुष्मि॑न् न॒मुष्मि॑न् ॅलो॒के । \newline
12. लो॒क आ᳚र्द्ध्नुवन् नार्द्ध्नुवन् ॅलो॒के लो॒क आ᳚र्द्ध्नुवन्न् । \newline
13. आ॒र्द्ध्नु॒व॒न् च॒रुणा॑ च॒रुणा᳚ ऽऽर्द्ध्नुवन् नार्द्ध्नुवन् च॒रुणा᳚ । \newline
14. च॒रुणा॒ ऽस्मिन् न॒स्मिꣳश् च॒रुणा॑ च॒रुणा॒ ऽस्मिन्न् । \newline
15. अ॒स्मिन्. यो यो᳚ ऽस्मिन् न॒स्मिन्. यः । \newline
16. यः का॒मये॑त का॒मये॑त॒ यो यः का॒मये॑त । \newline
17. का॒मये॑ता॒ मुष्मि॑न् न॒मुष्मि॑न् का॒मये॑त का॒मये॑ता॒ मुष्मिन्न्॑ । \newline
18. अ॒मुष्मि॑न् ॅलो॒के लो॒के॑ ऽमुष्मि॑न् न॒मुष्मि॑न् ॅलो॒के । \newline
19. लो॒क ऋ॑द्ध्नुया मृद्ध्नुयाम् ॅलो॒के लो॒क ऋ॑द्ध्नुयाम् । \newline
20. ऋ॒द्ध्नु॒या॒ मिती त्यृ॑द्ध्नुया मृद्ध्नुया॒ मिति॑ । \newline
21. इति॒ स स इतीति॒ सः । \newline
22. स पु॑रो॒डाश॑म् पुरो॒डाशꣳ॒॒ स स पु॑रो॒डाश᳚म् । \newline
23. पु॒रो॒डाश॑म् कुर्वीत कुर्वीत पुरो॒डाश॑म् पुरो॒डाश॑म् कुर्वीत । \newline
24. कु॒र्वी॒ता॒ मुष्मि॑न् न॒मुष्मि॑न् कुर्वीत कुर्वीता॒ मुष्मिन्न्॑ । \newline
25. अ॒मुष्मि॑न् ने॒वै वामुष्मि॑न् न॒मुष्मि॑न् ने॒व । \newline
26. ए॒व लो॒के लो॒क ए॒वैव लो॒के । \newline
27. लो॒क ऋ॑द्ध्नो त्यृद्ध्नोति लो॒के लो॒क ऋ॑द्ध्नोति । \newline
28. ऋ॒द्ध्नो॒ति॒ यद् यदृ॑द्ध्नो त्यृद्ध्नोति॒ यत् । \newline
29. यद॒ष्टाक॑पालो॒ ऽष्टाक॑पालो॒ यद् यद॒ष्टाक॑पालः । \newline
30. अ॒ष्टाक॑पाल॒ स्तेन॒ तेना॒ष्टाक॑पालो॒ ऽष्टाक॑पाल॒ स्तेन॑ । \newline
31. अ॒ष्टाक॑पाल॒ इत्य॒ष्टा - क॒पा॒लः॒ । \newline
32. तेना᳚ग्ने॒य आ᳚ग्ने॒य स्तेन॒ तेना᳚ग्ने॒यः । \newline
33. आ॒ग्ने॒यो यद् यदा᳚ग्ने॒य आ᳚ग्ने॒यो यत् । \newline
34. यत् त्रि॑कपा॒ल स्त्रि॑कपा॒लो यद् यत् त्रि॑कपा॒लः । \newline
35. त्रि॒क॒पा॒ल स्तेन॒ तेन॑ त्रिकपा॒ल स्त्रि॑कपा॒ल स्तेन॑ । \newline
36. त्रि॒क॒पा॒ल इति॑ त्रि - क॒पा॒लः । \newline
37. तेन॑ वैष्ण॒वो वै᳚ष्ण॒व स्तेन॒ तेन॑ वैष्ण॒वः । \newline
38. वै॒ष्ण॒वः समृ॑द्ध्यै॒ समृ॑द्ध्यै वैष्ण॒वो वै᳚ष्ण॒वः समृ॑द्ध्यै । \newline
39. समृ॑द्ध्यै॒ यो यः समृ॑द्ध्यै॒ समृ॑द्ध्यै॒ यः । \newline
40. समृ॑द्ध्या॒ इति॒ सं - ऋ॒द्ध्यै॒ । \newline
41. यः का॒मये॑त का॒मये॑त॒ यो यः का॒मये॑त । \newline
42. का॒मये॑ता॒ स्मिन् न॒स्मिन् का॒मये॑त का॒मये॑ता॒ स्मिन्न् । \newline
43. अ॒स्मिन् ॅलो॒के लो॒के᳚ ऽस्मिन् न॒स्मिन् ॅलो॒के । \newline
44. लो॒क ऋ॑द्ध्नुया मृद्ध्नुयाम् ॅलो॒के लो॒क ऋ॑द्ध्नुयाम् । \newline
45. ऋ॒द्ध्नु॒या॒ मिती त्यृ॑द्ध्नुया मृद्ध्नुया॒ मिति॑ । \newline
46. इति॒ स स इतीति॒ सः । \newline
47. स च॒रुम् च॒रुꣳ स स च॒रुम् । \newline
48. च॒रुम् कु॑र्वीत कुर्वीत च॒रुम् च॒रुम् कु॑र्वीत । \newline
49. कु॒र्वी॒ता॒ग्ने र॒ग्नेः कु॑र्वीत कुर्वीता॒ग्नेः । \newline
50. अ॒ग्नेर् घृ॒तम् घृ॒त म॒ग्ने र॒ग्नेर् घृ॒तम् । \newline
51. घृ॒तं ॅविष्णो॒र् विष्णो᳚र् घृ॒तम् घृ॒तं ॅविष्णोः᳚ । \newline
52. विष्णो᳚ स्तण्डु॒ला स्त॑ण्डु॒ला विष्णो॒र् विष्णो᳚ स्तण्डु॒लाः । \newline
53. त॒ण्डु॒ला स्तस्मा॒त् तस्मा᳚त् तण्डु॒ला स्त॑ण्दु॒ला स्तस्मा᳚त् । \newline
54. तस्मा᳚च् च॒रु श्च॒रु स्तस्मा॒त् तस्मा᳚च् च॒रुः । \newline

\textbf{Ghana Paata } \newline

1. ए॒वोभ॒यत॑ उभ॒यत॑ ए॒वैवो भ॒यतः॑ परि॒गृह्य॑ परि॒गृह्यो॑ भ॒यत॑ ए॒वैवो भ॒यतः॑ परि॒गृह्य॑ । \newline
2. उ॒भ॒यतः॑ परि॒गृह्य॑ परि॒गृह्यो॑ भ॒यत॑ उभ॒यतः॑ परि॒गृह्य॒ यज॑मानो॒ यज॑मानः परि॒गृह्यो॑ भ॒यत॑ उभ॒यतः॑ परि॒गृह्य॒ यज॑मानः । \newline
3. प॒रि॒गृह्य॒ यज॑मानो॒ यज॑मानः परि॒गृह्य॑ परि॒गृह्य॒ यज॑मा॒नो ऽवाव॒ यज॑मानः परि॒गृह्य॑ परि॒गृह्य॒ यज॑मा॒नो ऽव॑ । \newline
4. प॒रि॒गृह्येति॑ परि - गृह्य॑ । \newline
5. यज॑मा॒नो ऽवाव॒ यज॑मानो॒ यज॑मा॒नो ऽव॑ रुन्धे रु॒न्धे ऽव॒ यज॑मानो॒ यज॑मा॒नो ऽव॑ रुन्धे । \newline
6. अव॑ रुन्धे रु॒न्धे ऽवाव॑ रुन्धे पुरो॒डाशे॑न पुरो॒डाशे॑न रु॒न्धे ऽवाव॑ रुन्धे पुरो॒डाशे॑न । \newline
7. रु॒न्धे॒ पु॒रो॒डाशे॑न पुरो॒डाशे॑न रुन्धे रुन्धे पुरो॒डाशे॑न॒ वै वै पु॑रो॒डाशे॑न रुन्धे रुन्धे पुरो॒डाशे॑न॒ वै । \newline
8. पु॒रो॒डाशे॑न॒ वै वै पु॑रो॒डाशे॑न पुरो॒डाशे॑न॒ वै दे॒वा दे॒वा वै पु॑रो॒डाशे॑न पुरो॒डाशे॑न॒ वै दे॒वाः । \newline
9. वै दे॒वा दे॒वा वै वै दे॒वा अ॒मुष्मि॑न् न॒मुष्मि॑न् दे॒वा वै वै दे॒वा अ॒मुष्मिन्न्॑ । \newline
10. दे॒वा अ॒मुष्मि॑न् न॒मुष्मि॑न् दे॒वा दे॒वा अ॒मुष्मि॑न् ॅलो॒के लो॒के॑ ऽमुष्मि॑न् दे॒वा दे॒वा अ॒मुष्मि॑न् ॅलो॒के । \newline
11. अ॒मुष्मि॑न् ॅलो॒के लो॒के॑ ऽमुष्मि॑न् न॒मुष्मि॑न् ॅलो॒क आ᳚र्द्ध्नुवन् नार्द्ध्नुवन् ॅलो॒के॑ ऽमुष्मि॑न् न॒मुष्मि॑न् ॅलो॒क आ᳚र्द्ध्नुवन्न् । \newline
12. लो॒क आ᳚र्द्ध्नुवन् नार्द्ध्नुवन् ॅलो॒के लो॒क आ᳚र्द्ध्नुवन् च॒रुणा॑ च॒रुणा᳚ ऽऽर्द्ध्नुवन् ॅलो॒के लो॒क आ᳚र्द्ध्नुवन् च॒रुणा᳚ । \newline
13. आ॒र्द्ध्नु॒व॒न् च॒रुणा॑ च॒रुणा᳚ ऽऽर्द्ध्नुवन् नार्द्ध्नुवन् च॒रुणा॒ ऽस्मिन् न॒स्मिꣳ श्च॒रुणा᳚ ऽऽर्द्ध्नुवन् नार्द्ध्नुवन् च॒रुणा॒ ऽस्मिन्न् । \newline
14. च॒रुणा॒ ऽस्मिन् न॒स्मिꣳ श्च॒रुणा॑ च॒रुणा॒ ऽस्मिन्. यो यो᳚ ऽस्मिꣳ श्च॒रुणा॑ च॒रुणा॒ ऽस्मिन्. यः । \newline
15. अ॒स्मिन्. यो यो᳚ ऽस्मिन् न॒स्मिन्. यः का॒मये॑त का॒मये॑त॒ यो᳚ ऽस्मिन् न॒स्मिन्. यः का॒मये॑त । \newline
16. यः का॒मये॑त का॒मये॑त॒ यो यः का॒मये॑ता॒ मुष्मि॑न् न॒मुष्मि॑न् का॒मये॑त॒ यो यः का॒मये॑ता॒ मुष्मिन्न्॑ । \newline
17. का॒मये॑ता॒ मुष्मि॑न् न॒मुष्मि॑न् का॒मये॑त का॒मये॑ता॒ मुष्मि॑न् ॅलो॒के लो॒के॑ ऽमुष्मि॑न् का॒मये॑त का॒मये॑ता॒ मुष्मि॑न् ॅलो॒के । \newline
18. अ॒मुष्मि॑न् ॅलो॒के लो॒के॑ ऽमुष्मि॑न् न॒मुष्मि॑न् ॅलो॒क ऋ॑द्ध्नुया मृद्ध्नुयाम् ॅलो॒के॑ ऽमुष्मि॑न् न॒मुष्मि॑न् ॅलो॒क ऋ॑द्ध्नुयाम् । \newline
19. लो॒क ऋ॑द्ध्नुया मृद्ध्नुयाम् ॅलो॒के लो॒क ऋ॑द्ध्नुया॒ मिती त्यृ॑द्ध्नुयाम् ॅलो॒के लो॒क ऋ॑द्ध्नुया॒ मिति॑ । \newline
20. ऋ॒द्ध्नु॒या॒ मिती त्यृ॑द्ध्नुया मृद्ध्नुया॒ मिति॒ स स इत्यृ॑द्ध्नुया मृद्ध्नुया॒ मिति॒ सः । \newline
21. इति॒ स स इतीति॒ स पु॑रो॒डाश॑म् पुरो॒डाशꣳ॒॒ स इतीति॒ स पु॑रो॒डाश᳚म् । \newline
22. स पु॑रो॒डाश॑म् पुरो॒डाशꣳ॒॒ स स पु॑रो॒डाश॑म् कुर्वीत कुर्वीत पुरो॒डाशꣳ॒॒ स स पु॑रो॒डाश॑म् कुर्वीत । \newline
23. पु॒रो॒डाश॑म् कुर्वीत कुर्वीत पुरो॒डाश॑म् पुरो॒डाश॑म् कुर्वीता॒ मुष्मि॑न् न॒मुष्मि॑न् कुर्वीत पुरो॒डाश॑म् पुरो॒डाश॑म् कुर्वीता॒ मुष्मिन्न्॑ । \newline
24. कु॒र्वी॒ता॒ मुष्मि॑न् न॒मुष्मि॑न् कुर्वीत कुर्वीता॒ मुष्मि॑न् ने॒वैवा मुष्मि॑न् कुर्वीत कुर्वीता॒ मुष्मि॑न् ने॒व । \newline
25. अ॒मुष्मि॑न् ने॒वैवा मुष्मि॑न् न॒मुष्मि॑न् ने॒व लो॒के लो॒क ए॒वामुष्मि॑न् न॒मुष्मि॑न् ने॒व लो॒के । \newline
26. ए॒व लो॒के लो॒क ए॒वैव लो॒क ऋ॑द्ध्नो त्यृद्ध्नोति लो॒क ए॒वैव लो॒क ऋ॑द्ध्नोति । \newline
27. लो॒क ऋ॑द्ध्नो त्यृद्ध्नोति लो॒के लो॒क ऋ॑द्ध्नोति॒ यद् यदृ॑द्ध्नोति लो॒के लो॒क ऋ॑द्ध्नोति॒ यत् । \newline
28. ऋ॒द्ध्नो॒ति॒ यद् यदृ॑द्ध्नो त्यृद्ध्नोति॒ यद॒ष्टाक॑पालो॒ ऽष्टाक॑पालो॒ यदृ॑द्ध्नो त्यृद्ध्नोति॒ यद॒ष्टाक॑पालः । \newline
29. यद॒ष्टाक॑पालो॒ ऽष्टाक॑पालो॒ यद् यद॒ष्टाक॑पाल॒ स्तेन॒ तेना॒ष्टाक॑पालो॒ यद् यद॒ष्टाक॑पाल॒ स्तेन॑ । \newline
30. अ॒ष्टाक॑पाल॒ स्तेन॒ तेना॒ष्टाक॑पालो॒ ऽष्टाक॑पाल॒ स्तेना᳚ ग्ने॒य आ᳚ग्ने॒य स्तेना॒ष्टाक॑पालो॒ ऽष्टाक॑पाल॒ स्तेना᳚ ग्ने॒यः । \newline
31. अ॒ष्टाक॑पाल॒ इत्य॒ष्टा - क॒पा॒लः॒ । \newline
32. तेना᳚ ग्ने॒य आ᳚ग्ने॒य स्तेन॒ तेना᳚ ग्ने॒यो यद् यदा᳚ ग्ने॒य स्तेन॒ तेना᳚ ग्ने॒यो यत् । \newline
33. आ॒ग्ने॒यो यद् यदा᳚ ग्ने॒य आ᳚ग्ने॒यो यत् त्रि॑कपा॒ल स्त्रि॑कपा॒लो यदा᳚ग्ने॒य आ᳚ग्ने॒यो यत् त्रि॑कपा॒लः । \newline
34. यत् त्रि॑कपा॒ल स्त्रि॑कपा॒लो यद् यत् त्रि॑कपा॒ल स्तेन॒ तेन॑ त्रिकपा॒लो यद् यत् त्रि॑कपा॒ल स्तेन॑ । \newline
35. त्रि॒क॒पा॒ल स्तेन॒ तेन॑ त्रिकपा॒ल स्त्रि॑कपा॒ल स्तेन॑ वैष्ण॒वो वै᳚ष्ण॒व स्तेन॑ त्रिकपा॒ल स्त्रि॑कपा॒ल स्तेन॑ वैष्ण॒वः । \newline
36. त्रि॒क॒पा॒ल इति॑ त्रि - क॒पा॒लः । \newline
37. तेन॑ वैष्ण॒वो वै᳚ष्ण॒व स्तेन॒ तेन॑ वैष्ण॒वः समृ॑द्ध्यै॒ समृ॑द्ध्यै वैष्ण॒व स्तेन॒ तेन॑ वैष्ण॒वः समृ॑द्ध्यै । \newline
38. वै॒ष्ण॒वः समृ॑द्ध्यै॒ समृ॑द्ध्यै वैष्ण॒वो वै᳚ष्ण॒वः समृ॑द्ध्यै॒ यो यः समृ॑द्ध्यै वैष्ण॒वो वै᳚ष्ण॒वः समृ॑द्ध्यै॒ यः । \newline
39. समृ॑द्ध्यै॒ यो यः समृ॑द्ध्यै॒ समृ॑द्ध्यै॒ यः का॒मये॑त का॒मये॑त॒ यः समृ॑द्ध्यै॒ समृ॑द्ध्यै॒ यः का॒मये॑त । \newline
40. समृ॑द्ध्या॒ इति॒ सं - ऋ॒द्ध्यै॒ । \newline
41. यः का॒मये॑त का॒मये॑त॒ यो यः का॒मये॑ता॒स्मिन् न॒स्मिन् का॒मये॑त॒ यो यः का॒मये॑ता॒स्मिन्न् । \newline
42. का॒मये॑ता॒स्मिन् न॒स्मिन् का॒मये॑त का॒मये॑ता॒स्मिन् ॅलो॒के लो॒के᳚ ऽस्मिन् का॒मये॑त का॒मये॑ता॒स्मिन् ॅलो॒के । \newline
43. अ॒स्मिन् ॅलो॒के लो॒के᳚ ऽस्मिन् न॒स्मिन् ॅलो॒क ऋ॑द्ध्नुया मृद्ध्नुयाम् ॅलो॒के᳚ ऽस्मिन् न॒स्मिन् ॅलो॒क ऋ॑द्ध्नुयाम् । \newline
44. लो॒क ऋ॑द्ध्नुया मृद्ध्नुयाम् ॅलो॒के लो॒क ऋ॑द्ध्नुया॒ मिती त्यृ॑द्ध्नुयाम् ॅलो॒के लो॒क ऋ॑द्ध्नुया॒ मिति॑ । \newline
45. ऋ॒द्ध्नु॒या॒ मिती त्यृ॑द्ध्नुया मृद्ध्नुया॒ मिति॒ स स इत्यृ॑द्ध्नुया मृद्ध्नुया॒ मिति॒ सः । \newline
46. इति॒ स स इतीति॒ स च॒रुम् च॒रुꣳ स इतीति॒ स च॒रुम् । \newline
47. स च॒रुम् च॒रुꣳ स स च॒रुम् कु॑र्वीत कुर्वीत च॒रुꣳ स स च॒रुम् कु॑र्वीत । \newline
48. च॒रुम् कु॑र्वीत कुर्वीत च॒रुम् च॒रुम् कु॑र्वीता॒ग्ने र॒ग्नेः कु॑र्वीत च॒रुम् च॒रुम् कु॑र्वीता॒ग्नेः । \newline
49. कु॒र्वी॒ता॒ग्ने र॒ग्नेः कु॑र्वीत कुर्वीता॒ग्नेर् घृ॒तम् घृ॒त म॒ग्नेः कु॑र्वीत कुर्वीता॒ग्नेर् घृ॒तम् । \newline
50. अ॒ग्नेर् घृ॒तम् घृ॒त म॒ग्ने र॒ग्नेर् घृ॒तं ॅविष्णो॒र् विष्णो᳚र् घृ॒त म॒ग्ने र॒ग्नेर् घृ॒तं ॅविष्णोः᳚ । \newline
51. घृ॒तं ॅविष्णो॒र् विष्णो᳚र् घृ॒तम् घृ॒तं ॅविष्णो᳚ स्तण्डु॒ला स्त॑ण्डु॒ला विष्णो᳚र् घृ॒तम् घृ॒तं ॅविष्णो᳚ स्तण्डु॒लाः । \newline
52. विष्णो᳚ स्तण्डु॒ला स्त॑ण्डु॒ला विष्णो॒र् विष्णो᳚ स्तण्डु॒ला स्तस्मा॒त् तस्मा᳚त् तण्डु॒ला विष्णो॒र् विष्णो᳚ स्तण्डु॒ला स्तस्मा᳚त् । \newline
53. त॒ण्डु॒ला स्तस्मा॒त् तस्मा᳚त् तण्डु॒ला स्त॑ण्दु॒ला स्तस्मा᳚च् च॒रु श्च॒रु स्तस्मा᳚त् तण्डु॒ला स्त॑ण्दु॒ला स्तस्मा᳚च् च॒रुः । \newline
54. तस्मा᳚च् च॒रु श्च॒रु स्तस्मा॒त् तस्मा᳚च् च॒रुः का॒र्यः॑ का॒र्य॑ श्च॒रु स्तस्मा॒त् तस्मा᳚च् च॒रुः का॒र्यः॑ । \newline
\pagebreak
\markright{ TS 5.5.1.6  \hfill https://www.vedavms.in \hfill}

\section{ TS 5.5.1.6 }

\textbf{TS 5.5.1.6 } \newline
\textbf{Samhita Paata} \newline

-च्च॒रुः का॒र्यो᳚ऽस्मिन्ने॒व लो॒क ऋ॑द्ध्नोत्यादि॒त्यो भ॑वती॒ यं ॅवा अदि॑तिर॒स्यामे॒व प्रति॑ तिष्ठ॒त्यथो॑ अ॒स्यामे॒वाधि॑ य॒ज्ञ्ं त॑नुते॒ यो वै सं॑ॅवथ्स॒रमुख्य॒-मभृ॑त्वा॒ऽग्निं चि॑नु॒ते यथा॑ सा॒मि गर्भो॑ऽव॒पद्य॑ते ता॒दृगे॒व तदार्ति॒मार्च्छे᳚द्-वैश्वान॒रं द्वाद॑शकपालं पु॒रस्ता॒न्निर्व॑पेथ् संॅवथ्स॒रो वा अ॒ग्निर्-वै᳚श्वान॒रो यथा॑ संवथ्स॒रमा॒प्त्वा - [  ] \newline

\textbf{Pada Paata} \newline

च॒रुः । का॒र्यः॑ । अ॒स्मिन्न् । ए॒व । लो॒के । ऋ॒द्ध्नो॒ति॒ । आ॒दि॒त्यः । भ॒व॒ति॒ । इ॒यम् । वै । अदि॑तिः । अ॒स्याम् । ए॒व । प्रतीति॑ । ति॒ष्ठ॒ति॒ । अथो॒ इति॑ । अ॒स्याम् । ए॒व । अधीति॑ । य॒ज्ञ्म् । त॒नु॒ते॒ । यः । वै । सं॒ॅव॒थ्स॒रमिति॑ सं - व॒थ्स॒रम् । उख्य᳚म् । अभृ॑त्वा । अ॒ग्निम् । चि॒नु॒ते । यथा᳚ । सा॒मि । गर्भः॑ । अ॒व॒पद्य॑त॒ इत्य॑व - पद्य॑ते । ता॒दृक् । ए॒व । तत् । आर्ति᳚म् । एति॑ । ऋ॒च्छे॒त् । वै॒श्वा॒न॒रम् । द्वाद॑शकपाल॒मिति॒ द्वाद॑श - क॒पा॒ल॒म् । पु॒रस्ता᳚त् । निरिति॑ । व॒पे॒त् । सं॒ॅव॒थ्स॒र इति॑ सं - व॒थ्स॒रः । वै । अ॒ग्निः । वै॒श्वा॒न॒रः । यथा᳚ । सं॒ॅव॒थ्स॒रमिति॑ सं - व॒थ्स॒रम् । आ॒प्त्वा ।  \newline


\textbf{Krama Paata} \newline

च॒रुः का॒र्यः॑ । का॒र्यो᳚ऽस्मिन्न् । अ॒स्मिन्ने॒व । ए॒व लो॒के । लो॒क ऋ॑द्ध्नोति । ऋ॒द्ध्नो॒त्या॒दि॒त्यः । आ॒दि॒त्यो भ॑वति । भ॒व॒ती॒यम् । इ॒यम् ॅवै । वा अदि॑तिः । अदि॑तिर॒स्याम् । अ॒स्यामे॒व । ए॒व प्रति॑ । प्रति॑ तिष्ठति । ति॒ष्ठ॒त्यथो᳚ । अथो॑ अ॒स्याम् । अथो॒ इत्यथो᳚ । अ॒स्यामे॒व । ए॒वाधि॑ । अधि॑ य॒ज्ञ्म् । य॒ज्ञ्म् त॑नुते । त॒नु॒ते॒ यः । यो वै । वै स॑म्ॅवथ्स॒रम् । स॒ॅव॒थ्स॒रमुख्य᳚म् । स॒म्ॅव॒थ्स॒रमिति॑ सम् - व॒थ्स॒रम् । उख्य॒मभृ॑त्वा । अभृ॑त्वा॒ऽग्निम् । अ॒ग्निम् चि॑नु॒ते । चि॒नु॒ते यथा᳚ । यथा॑ सा॒मि । सा॒मि गर्भः॑ । गर्भो॑ऽव॒पद्य॑ते । अ॒व॒पद्य॑ते ता॒दृक् । अ॒व॒पद्य॑त॒ इत्य॑व - पद्य॑ते । ता॒दृगे॒व । ए॒व तत् । तदार्ति᳚म् । आर्ति॒मा । आर्च्छे᳚त् । ऋ॒च्छे॒द् वै॒श्वा॒न॒रम् । वै॒श्वा॒न॒रम् द्वाद॑शकपालम् । द्वाद॑शकपालम् पु॒रस्ता᳚त् । द्वाद॑शकपाल॒मिति॒ द्वाद॑श - क॒पा॒ल॒म् । पु॒रस्ता॒न् निः । निर् व॑पेत् । व॒पे॒थ् स॒म्ॅव॒थ्स॒रः । स॒म्ॅव॒थ्स॒रो वै । स॒म्ॅव॒थ्स॒र इति॑ सम् - व॒थ्स॒रः । वा अ॒ग्निः । अ॒ग्निर् वै᳚श्वान॒रः । वै॒श्वा॒न॒रो यथा᳚ । यथा॑ सम्ॅवथ्स॒रम् । स॒म्ॅव॒थ्स॒रमा॒प्त्वा ( ) । स॒म्ॅव॒थ्स॒रमिति॑ सम् - व॒थ्स॒रम् । आ॒प्त्वा का॒ले \newline

\textbf{Jatai Paata} \newline

1. च॒रुः का॒र्यः॑ का॒र्य॑ श्च॒रु श्च॒रुः का॒र्यः॑ । \newline
2. का॒र्यो᳚ ऽस्मिन् न॒स्मिन् का॒र्यः॑ का॒र्यो᳚ ऽस्मिन्न् । \newline
3. आ॒स्मिन् ने॒वैवास्मिन् न॒स्मिन् ने॒व । \newline
4. ए॒व लो॒के लो॒क ए॒वैव लो॒के । \newline
5. लो॒क ऋ॑द्ध्नो त्यृद्ध्नोति लो॒के लो॒क ऋ॑द्ध्नोति । \newline
6. ऋ॒द्ध्नो॒ त्या॒दि॒त्य आ॑दि॒त्य ऋ॑द्ध्नो त्यृद्ध्नो त्यादि॒त्यः । \newline
7. आ॒दि॒त्यो भ॑वति भव त्यादि॒त्य आ॑दि॒त्यो भ॑वति । \newline
8. भ॒व॒ती॒य मि॒यम् भ॑वति भवती॒यम् । \newline
9. इ॒यं ॅवै वा इ॒य मि॒यं ॅवै । \newline
10. वा अदि॑ति॒ रदि॑ति॒र् वै वा अदि॑तिः । \newline
11. अदि॑ति र॒स्या म॒स्या मदि॑ति॒ रदि॑ति र॒स्याम् । \newline
12. अ॒स्या मे॒वै वास्या म॒स्या मे॒व । \newline
13. ए॒व प्रति॒ प्रत्ये॒ वैव प्रति॑ । \newline
14. प्रति॑ तिष्ठति तिष्ठति॒ प्रति॒ प्रति॑ तिष्ठति । \newline
15. ति॒ष्ठ॒ त्यथो॒ अथो॑ तिष्ठति तिष्ठ॒ त्यथो᳚ । \newline
16. अथो॑ अ॒स्या म॒स्या मथो॒ अथो॑ अ॒स्याम् । \newline
17. अथो॒ इत्यथो᳚ । \newline
18. अ॒स्या मे॒वै वास्या म॒स्या मे॒व । \newline
19. ए॒वाध्यध्ये॒ वैवाधि॑ । \newline
20. अधि॑ य॒ज्ञ्ं ॅय॒ज्ञ् मध्यधि॑ य॒ज्ञ्म् । \newline
21. य॒ज्ञ्म् त॑नुते तनुते य॒ज्ञ्ं ॅय॒ज्ञ्म् त॑नुते । \newline
22. त॒नु॒ते॒ यो य स्त॑नुते तनुते॒ यः । \newline
23. यो वै वै यो यो वै । \newline
24. वै सं॑ॅवथ्स॒रꣳ सं॑ॅवथ्स॒रं ॅवै वै सं॑ॅवथ्स॒रम् । \newline
25. सं॒ॅव॒थ्स॒र मुख्य॒ मुख्यꣳ॑ संॅवथ्स॒रꣳ सं॑ॅवथ्स॒र मुख्य᳚म् । \newline
26. सं॒ॅव॒थ्स॒रमिति॑ सं - व॒थ्स॒रम् । \newline
27. उख्य॒ मभृ॒त्वा ऽभृ॒त्वोख्य॒ मुख्य॒ मभृ॑त्वा । \newline
28. अभृ॑त्वा॒ ऽग्नि म॒ग्नि मभृ॒त्वा ऽभृ॑त्वा॒ ऽग्निम् । \newline
29. अ॒ग्निम् चि॑नु॒ते चि॑नु॒ते᳚ ऽग्नि म॒ग्निम् चि॑नु॒ते । \newline
30. चि॒नु॒ते यथा॒ यथा॑ चिनु॒ते चि॑नु॒ते यथा᳚ । \newline
31. यथा॑ सा॒मि सा॒मि यथा॒ यथा॑ सा॒मि । \newline
32. सा॒मि गर्भो॒ गर्भः॑ सा॒मि सा॒मि गर्भः॑ । \newline
33. गर्भो॑ ऽव॒पद्य॑ते ऽव॒पद्य॑ते॒ गर्भो॒ गर्भो॑ ऽव॒पद्य॑ते । \newline
34. अ॒व॒पद्य॑ते ता॒दृक् ता॒दृ ग॑व॒पद्य॑ते ऽव॒पद्य॑ते ता॒दृक् । \newline
35. अ॒व॒पद्य॑त॒ इत्य॑व - पद्य॑ते । \newline
36. ता॒दृ गे॒वैव ता॒दृक् ता॒दृ गे॒व । \newline
37. ए॒व तत् तदे॒ वैव तत् । \newline
38. तदार्ति॒ मार्ति॒म् तत् तदार्ति᳚म् । \newline
39. आर्ति॒ मा ऽऽर्ति॒ मार्ति॒ मा । \newline
40. आर्च्छे॑ दृच्छे॒ दार्च्छे᳚त् । \newline
41. ऋ॒च्छे॒द् वै॒श्वा॒न॒रं ॅवै᳚श्वान॒र मृ॑च्छे दृच्छेद् वैश्वान॒रम् । \newline
42. वै॒श्वा॒न॒रम् द्वाद॑शकपाल॒म् द्वाद॑शकपालं ॅवैश्वान॒रं ॅवै᳚श्वान॒रम् द्वाद॑शकपालम् । \newline
43. द्वाद॑शकपालम् पु॒रस्ता᳚त् पु॒रस्ता॒द् द्वाद॑शकपाल॒म् द्वाद॑शकपालम् पु॒रस्ता᳚त् । \newline
44. द्वाद॑शकपाल॒मिति॒ द्वाद॑श - क॒पा॒ल॒म् । \newline
45. पु॒रस्ता॒न् निर् णिष् पु॒रस्ता᳚त् पु॒रस्ता॒न् निः । \newline
46. निर् व॑पेद् वपे॒न् निर् णिर् व॑पेत् । \newline
47. व॒पे॒थ् सं॒ॅव॒थ्स॒रः सं॑ॅवथ्स॒रो व॑पेद् वपेथ् संॅवथ्स॒रः । \newline
48. सं॒ॅव॒थ्स॒रो वै वै सं॑ॅवथ्स॒रः सं॑ॅवथ्स॒रो वै । \newline
49. सं॒ॅव॒थ्स॒र इति॑ सं - व॒थ्स॒रः । \newline
50. वा अ॒ग्नि र॒ग्निर् वै वा अ॒ग्निः । \newline
51. अ॒ग्निर् वै᳚श्वान॒रो वै᳚श्वान॒रो᳚ ऽग्नि र॒ग्निर् वै᳚श्वान॒रः । \newline
52. वै॒श्वा॒न॒रो यथा॒ यथा॑ वैश्वान॒रो वै᳚श्वान॒रो यथा᳚ । \newline
53. यथा॑ संॅवथ्स॒रꣳ सं॑ॅवथ्स॒रं ॅयथा॒ यथा॑ संॅवथ्स॒रम् । \newline
54. सं॒ॅव॒थ्स॒र मा॒प्त्वा ऽऽप्त्वा सं॑ॅवथ्स॒रꣳ सं॑ॅवथ्स॒र मा॒प्त्वा । \newline
55. सं॒ॅव॒थ्स॒रमिति॑ सं - व॒थ्स॒रम् । \newline
56. आ॒प्त्वा का॒ले का॒ल आ॒प्त्वा ऽऽप्त्वा का॒ले । \newline

\textbf{Ghana Paata } \newline

1. च॒रुः का॒र्यः॑ का॒र्य॑ श्च॒रु श्च॒रुः का॒र्यो᳚ ऽस्मिन् न॒स्मिन् का॒र्य॑ श्च॒रु श्च॒रुः का॒र्यो᳚ ऽस्मिन्न् । \newline
2. का॒र्यो᳚ ऽस्मिन् न॒स्मिन् का॒र्यः॑ का॒र्यो᳚ ऽस्मिन् ने॒वै वास्मिन् का॒र्यः॑ का॒र्यो᳚ ऽस्मिन् ने॒व । \newline
3. आ॒स्मिन् ने॒वै वास्मिन् न॒स्मिन् ने॒व लो॒के लो॒क ए॒वास्मिन् न॒स्मिन् ने॒व लो॒के । \newline
4. ए॒व लो॒के लो॒क ए॒वैव लो॒क ऋ॑द्ध्नो त्यृद्ध्नोति लो॒क ए॒वैव लो॒क ऋ॑द्ध्नोति । \newline
5. लो॒क ऋ॑द्ध्नो त्यृद्ध्नोति लो॒के लो॒क ऋ॑द्ध्नो त्यादि॒त्य आ॑दि॒त्य ऋ॑द्ध्नोति लो॒के लो॒क ऋ॑द्ध्नो त्यादि॒त्यः । \newline
6. ऋ॒द्ध्नो॒ त्या॒दि॒त्य आ॑दि॒त्य ऋ॑द्ध्नो त्यृद्ध्नो त्यादि॒त्यो भ॑वति भव त्यादि॒त्य ऋ॑द्ध्नो त्यृद्ध्नो त्यादि॒त्यो भ॑वति । \newline
7. आ॒दि॒त्यो भ॑वति भव त्यादि॒त्य आ॑दि॒त्यो भ॑वती॒य मि॒यम् भ॑व त्यादि॒त्य आ॑दि॒त्यो भ॑वती॒यम् । \newline
8. भ॒व॒ती॒य मि॒यम् भ॑वति भवती॒यं ॅवै वा इ॒यम् भ॑वति भवती॒यं ॅवै । \newline
9. इ॒यं ॅवै वा इ॒य मि॒यं ॅवा अदि॑ति॒ रदि॑ति॒र् वा इ॒य मि॒यं ॅवा अदि॑तिः । \newline
10. वा अदि॑ति॒ रदि॑ति॒र् वै वा अदि॑ति र॒स्या म॒स्या मदि॑ति॒र् वै वा अदि॑ति र॒स्याम् । \newline
11. अदि॑ति र॒स्या म॒स्या मदि॑ति॒ रदि॑ति र॒स्या मे॒वै वास्या मदि॑ति॒ रदि॑ति र॒स्या मे॒व । \newline
12. अ॒स्या मे॒वै वास्या म॒स्या मे॒व प्रति॒ प्रत्ये॒ वास्या म॒स्या मे॒व प्रति॑ । \newline
13. ए॒व प्रति॒ प्रत्ये॒वैव प्रति॑ तिष्ठति तिष्ठति॒ प्रत्ये॒वैव प्रति॑ तिष्ठति । \newline
14. प्रति॑ तिष्ठति तिष्ठति॒ प्रति॒ प्रति॑ तिष्ठ॒ त्यथो॒ अथो॑ तिष्ठति॒ प्रति॒ प्रति॑ तिष्ठ॒ त्यथो᳚ । \newline
15. ति॒ष्ठ॒ त्यथो॒ अथो॑ तिष्ठति तिष्ठ॒ त्यथो॑ अ॒स्या म॒स्या मथो॑ तिष्ठति तिष्ठ॒ त्यथो॑ अ॒स्याम् । \newline
16. अथो॑ अ॒स्या म॒स्या मथो॒ अथो॑ अ॒स्या मे॒वै वास्या मथो॒ अथो॑ अ॒स्या मे॒व । \newline
17. अथो॒ इत्यथो᳚ । \newline
18. अ॒स्या मे॒वै वास्या म॒स्या मे॒वाध्य ध्ये॒ वास्या म॒स्या मे॒वाधि॑ । \newline
19. ए॒वाध्यध्ये॒ वैवाधि॑ य॒ज्ञ्ं ॅय॒ज्ञ् मध्ये॒वैवाधि॑ य॒ज्ञ्म् । \newline
20. अधि॑ य॒ज्ञ्ं ॅय॒ज्ञ् मध्यधि॑ य॒ज्ञ्म् त॑नुते तनुते य॒ज्ञ् मध्यधि॑ य॒ज्ञ्म् त॑नुते । \newline
21. य॒ज्ञ्म् त॑नुते तनुते य॒ज्ञ्ं ॅय॒ज्ञ्म् त॑नुते॒ यो यस्त॑नुते य॒ज्ञ्ं ॅय॒ज्ञ्म् त॑नुते॒ यः । \newline
22. त॒नु॒ते॒ यो यस्त॑नुते तनुते॒ यो वै वै यस्त॑नुते तनुते॒ यो वै । \newline
23. यो वै वै यो यो वै सं॑ॅवथ्स॒रꣳ सं॑ॅवथ्स॒रं ॅवै यो यो वै सं॑ॅवथ्स॒रम् । \newline
24. वै सं॑ॅवथ्स॒रꣳ सं॑ॅवथ्स॒रं ॅवै वै सं॑ॅवथ्स॒र मुख्य॒ मुख्यꣳ॑ संॅवथ्स॒रं ॅवै वै सं॑ॅवथ्स॒र मुख्य᳚म् । \newline
25. सं॒ॅव॒थ्स॒र मुख्य॒ मुख्यꣳ॑ संॅवथ्स॒रꣳ सं॑ॅवथ्स॒र मुख्य॒ मभृ॒त्वा ऽभ्रु॒त्वोख्यꣳ॑ संॅवथ्स॒रꣳ सं॑ॅवथ्स॒र मुख्य॒ मभृ॑त्वा । \newline
26. सं॒ॅव॒थ्स॒रमिति॑ सं - व॒थ्स॒रम् । \newline
27. उख्य॒ मभृ॒त्वा ऽभृ॒त्वोख्य॒ मुख्य॒ मभृ॑त्वा॒ ऽग्नि म॒ग्नि मभृ॒त्वोख्य॒ मुख्य॒ मभृ॑त्वा॒ ऽग्निम् । \newline
28. अभृ॑त्वा॒ ऽग्नि म॒ग्नि मभृ॒त्वा ऽभृ॑त्वा॒ ऽग्निम् चि॑नु॒ते चि॑नु॒ते᳚ ऽग्नि मभृ॒त्वा ऽभृ॑त्वा॒ ऽग्निम् चि॑नु॒ते । \newline
29. अ॒ग्निम् चि॑नु॒ते चि॑नु॒ते᳚ ऽग्नि म॒ग्निम् चि॑नु॒ते यथा॒ यथा॑ चिनु॒ते᳚ ऽग्नि म॒ग्निम् चि॑नु॒ते यथा᳚ । \newline
30. चि॒नु॒ते यथा॒ यथा॑ चिनु॒ते चि॑नु॒ते यथा॑ सा॒मि सा॒मि यथा॑ चिनु॒ते चि॑नु॒ते यथा॑ सा॒मि । \newline
31. यथा॑ सा॒मि सा॒मि यथा॒ यथा॑ सा॒मि गर्भो॒ गर्भः॑ सा॒मि यथा॒ यथा॑ सा॒मि गर्भः॑ । \newline
32. सा॒मि गर्भो॒ गर्भः॑ सा॒मि सा॒मि गर्भो॑ ऽव॒पद्य॑ते ऽव॒पद्य॑ते॒ गर्भः॑ सा॒मि सा॒मि गर्भो॑ ऽव॒पद्य॑ते । \newline
33. गर्भो॑ ऽव॒पद्य॑ते ऽव॒पद्य॑ते॒ गर्भो॒ गर्भो॑ ऽव॒पद्य॑ते ता॒दृक् ता॒दृ ग॑व॒पद्य॑ते॒ गर्भो॒ गर्भो॑ ऽव॒पद्य॑ते ता॒दृक् । \newline
34. अ॒व॒पद्य॑ते ता॒दृक् ता॒दृ ग॑व॒पद्य॑ते ऽव॒पद्य॑ते ता॒दृ गे॒वैव ता॒दृ ग॑व॒पद्य॑ते ऽव॒पद्य॑ते ता॒दृ गे॒व । \newline
35. अ॒व॒पद्य॑त॒ इत्य॑व - पद्य॑ते । \newline
36. ता॒दृगे॒ वैव ता॒दृक् ता॒दृ गे॒व तत् तदे॒व ता॒दृक् ता॒दृ गे॒व तत् । \newline
37. ए॒व तत् तदे॒वैव तदार्ति॒ मार्ति॒म् तदे॒वैव तदार्ति᳚म् । \newline
38. तदार्ति॒ मार्ति॒म् तत् तदार्ति॒ मा ऽऽर्ति॒म् तत् तदार्ति॒ मा । \newline
39. आर्ति॒ मा ऽऽर्ति॒ मार्ति॒ मार्च्छे॑ दृच्छे॒दा ऽऽर्ति॒ मार्ति॒ मार्च्छे᳚त् । \newline
40. आर्च्छे॑ दृच्छे॒ दार्च्छे᳚द् वैश्वान॒रं ॅवै᳚श्वान॒र मृ॑च्छे॒ दार्च्छे᳚द् वैश्वान॒रम् । \newline
41. ऋ॒च्छे॒द् वै॒श्वा॒न॒रं ॅवै᳚श्वान॒र मृ॑च्छे दृच्छेद् वैश्वान॒रम् द्वाद॑शकपाल॒म् द्वाद॑शकपालं ॅवैश्वान॒र मृ॑च्छे दृच्छेद् वैश्वान॒रम् द्वाद॑शकपालम् । \newline
42. वै॒श्वा॒न॒रम् द्वाद॑शकपाल॒म् द्वाद॑शकपालं ॅवैश्वान॒रं ॅवै᳚श्वान॒रम् द्वाद॑शकपालम् पु॒रस्ता᳚त् पु॒रस्ता॒द् द्वाद॑शकपालं ॅवैश्वान॒रं ॅवै᳚श्वान॒रम् द्वाद॑शकपालम् पु॒रस्ता᳚त् । \newline
43. द्वाद॑शकपालम् पु॒रस्ता᳚त् पु॒रस्ता॒द् द्वाद॑शकपाल॒म् द्वाद॑शकपालम् पु॒रस्ता॒न् निर् णिष् पु॒रस्ता॒द् द्वाद॑शकपाल॒म् द्वाद॑शकपालम् पु॒रस्ता॒न् निः । \newline
44. द्वाद॑शकपाल॒मिति॒ द्वाद॑श - क॒पा॒ल॒म् । \newline
45. पु॒रस्ता॒न् निर् णिष् पु॒रस्ता᳚त् पु॒रस्ता॒न् निर् व॑पेद् वपे॒न् निष् पु॒रस्ता᳚त् पु॒रस्ता॒न् निर् व॑पेत् । \newline
46. निर् व॑पेद् वपे॒न् निर् णिर् व॑पेथ् संॅवथ्स॒रः सं॑ॅवथ्स॒रो व॑पे॒न् निर् णिर् व॑पेथ् संॅवथ्स॒रः । \newline
47. व॒पे॒थ् सं॒ॅव॒थ्स॒रः सं॑ॅवथ्स॒रो व॑पेद् वपेथ् संॅवथ्स॒रो वै वै सं॑ॅवथ्स॒रो व॑पेद् वपेथ् संॅवथ्स॒रो वै । \newline
48. सं॒ॅव॒थ्स॒रो वै वै सं॑ॅवथ्स॒रः सं॑ॅवथ्स॒रो वा अ॒ग्नि र॒ग्निर् वै सं॑ॅवथ्स॒रः सं॑ॅवथ्स॒रो वा अ॒ग्निः । \newline
49. सं॒ॅव॒थ्स॒र इति॑ सं - व॒थ्स॒रः । \newline
50. वा अ॒ग्नि र॒ग्निर् वै वा अ॒ग्निर् वै᳚श्वान॒रो वै᳚श्वान॒रो᳚ ऽग्निर् वै वा अ॒ग्निर् वै᳚श्वान॒रः । \newline
51. अ॒ग्निर् वै᳚श्वान॒रो वै᳚श्वान॒रो᳚ ऽग्नि र॒ग्निर् वै᳚श्वान॒रो यथा॒ यथा॑ वैश्वान॒रो᳚ ऽग्नि र॒ग्निर् वै᳚श्वान॒रो यथा᳚ । \newline
52. वै॒श्वा॒न॒रो यथा॒ यथा॑ वैश्वान॒रो वै᳚श्वान॒रो यथा॑ संॅवथ्स॒रꣳ सं॑ॅवथ्स॒रं ॅयथा॑ वैश्वान॒रो वै᳚श्वान॒रो यथा॑ संॅवथ्स॒रम् । \newline
53. यथा॑ संॅवथ्स॒रꣳ सं॑ॅवथ्स॒रं ॅयथा॒ यथा॑ संॅवथ्स॒र मा॒प्त्वा ऽऽप्त्वा सं॑ॅवथ्स॒रं ॅयथा॒ यथा॑ संॅवथ्स॒र मा॒प्त्वा । \newline
54. सं॒ॅव॒थ्स॒र मा॒प्त्वा ऽऽप्त्वा सं॑ॅवथ्स॒रꣳ सं॑ॅवथ्स॒र मा॒प्त्वा का॒ले का॒ल आ॒प्त्वा सं॑ॅवथ्स॒रꣳ सं॑ॅवथ्स॒र मा॒प्त्वा का॒ले । \newline
55. सं॒ॅव॒थ्स॒रमिति॑ सं - व॒थ्स॒रम् । \newline
56. आ॒प्त्वा का॒ले का॒ल आ॒प्त्वा ऽऽप्त्वा का॒ल आग॑त॒ आग॑ते का॒ल आ॒प्त्वा ऽऽप्त्वा का॒ल आग॑ते । \newline
\pagebreak
\markright{ TS 5.5.1.7  \hfill https://www.vedavms.in \hfill}

\section{ TS 5.5.1.7 }

\textbf{TS 5.5.1.7 } \newline
\textbf{Samhita Paata} \newline

का॒ल आग॑ते वि॒जाय॑त ए॒वमे॒व सं॑ॅवथ्स॒रमा॒प्त्वा का॒ल आग॑ते॒ऽग्निं चि॑नुते॒ नाऽऽ*र्ति॒मार्च्छ॑त्ये॒षा वा अ॒ग्नेः प्रि॒या त॒नूर्यद्-वै᳚श्वान॒रः प्रि॒यामे॒वास्य॑ त॒नुव॒मव॑ रुन्धे॒ त्रीण्ये॒तानि॑ ह॒वीꣳषि॑ भवन्ति॒ त्रय॑ इ॒मे लो॒का ए॒षां ॅलो॒कानाꣳ॒॒ रोहा॑य ॥ \newline

\textbf{Pada Paata} \newline

का॒ले । आग॑त॒ इत्या - ग॒ते॒ । वि॒जाय॑त॒ इति॑ वि - जाय॑ते । ए॒वम् । ए॒व । सं॒ॅव॒थ्स॒रमिति॑ सं - व॒थ्स॒रम् । आ॒प्त्वा । का॒ले । आग॑त॒ इत्या - ग॒ते॒ । अ॒ग्निम् । चि॒नु॒ते॒ । न । आर्ति᳚म् । एति॑ । ऋ॒च्छ॒ति॒ । ए॒षा । वै । अ॒ग्नेः । प्रि॒या । त॒नूः । यत् । वै॒श्वा॒न॒रः । प्रि॒याम् । ए॒व । अ॒स्य॒ । त॒नुव᳚म् । अवेति॑ । रु॒न्धे॒ । त्रीणि॑ । ए॒तानि॑ । ह॒वीꣳ षि॑ । भ॒व॒न्ति॒ । त्रयः॑ । इ॒मे । लो॒काः । ए॒षाम् । लो॒काना᳚म् । रोहा॑य ॥  \newline


\textbf{Krama Paata} \newline

का॒ल आग॑ते । आग॑ते वि॒जाय॑ते । आग॑त॒ इत्या - ग॒ते॒ । वि॒जाय॑त ए॒वम् । वि॒जाय॑त॒ इति॑ वि - जाय॑ते । ए॒वमे॒व । ए॒व स॑म्ॅवथ्स॒रम् । स॒म्ॅव॒थ्स॒रमा॒प्त्वा । स॒म्ॅव॒थ्स॒रमिति॑ सम् - व॒थ्स॒रम् । आ॒प्त्वा का॒ले । का॒ल आग॑ते । आग॑ते॒ऽग्निम् । आग॑त॒ इत्या - ग॒ते॒ । अ॒ग्निम् चि॑नुते । चि॒नु॒ते॒ न । नार्ति᳚म् । आर्ति॒मा । आर्च्छ॑ति । ऋ॒च्छ॒त्ये॒षा । ए॒षा वै । वा अ॒ग्नेः । अ॒ग्नेः प्रि॒या । प्रि॒या त॒नूः । त॒नूर् यत् । यद् वै᳚श्वान॒रः । वै॒श्वा॒न॒रः प्रि॒याम् । प्रि॒यामे॒व । ए॒वास्य॑ । अ॒स्य॒ त॒नुव᳚म् । त॒नुव॒मव॑ । अव॑ रुन्धे । रु॒न्धे॒ त्रीणि॑ । त्रीण्ये॒तानि॑ । ए॒तानि॑ ह॒वीꣳषि॑ । ह॒वीꣳषि॑ भवन्ति । भ॒व॒न्ति॒ त्रयः॑ । त्रय॑ इ॒मे । इ॒मे लो॒काः । लो॒का ए॒षाम् । ए॒षाम् ॅलो॒काना᳚म् । लो॒कानाꣳ॒॒ रोहा॑य । रोहा॒येति॒ रोहा॑य । \newline

\textbf{Jatai Paata} \newline

1. का॒ल आग॑त॒ आग॑ते का॒ले का॒ल आग॑ते । \newline
2. आग॑ते वि॒जाय॑ते वि॒जाय॑त॒ आग॑त॒ आग॑ते वि॒जाय॑ते । \newline
3. आग॑त॒ इत्या - ग॒ते॒ । \newline
4. वि॒जाय॑त ए॒व मे॒वं ॅवि॒जाय॑ते वि॒जाय॑त ए॒वम् । \newline
5. वि॒जाय॑त॒ इति॑ वि - जाय॑ते । \newline
6. ए॒व मे॒वै वैव मे॒व मे॒व । \newline
7. ए॒व सं॑ॅवथ्स॒रꣳ सं॑ॅवथ्स॒र मे॒वैव सं॑ॅवथ्स॒रम् । \newline
8. सं॒ॅव॒थ्स॒र मा॒प्त्वा ऽऽप्त्वा सं॑ॅवथ्स॒रꣳ सं॑ॅवथ्स॒र मा॒प्त्वा । \newline
9. सं॒ॅव॒थ्स॒रमिति॑ सं - व॒थ्स॒रम् । \newline
10. आ॒प्त्वा का॒ले का॒ल आ॒प्त्वा ऽऽप्त्वा का॒ले । \newline
11. का॒ल आग॑त॒ आग॑ते का॒ले का॒ल आग॑ते । \newline
12. आग॑ते॒ ऽग्नि म॒ग्नि माग॑त॒ आग॑ते॒ ऽग्निम् । \newline
13. आग॑त॒ इत्या - ग॒ते॒ । \newline
14. अ॒ग्निम् चि॑नुते चिनुते॒ ऽग्नि म॒ग्निम् चि॑नुते । \newline
15. चि॒नु॒ते॒ न न चि॑नुते चिनुते॒ न । \newline
16. नार्ति॒ मार्ति॒म् न नार्ति᳚म् । \newline
17. आर्ति॒ मा ऽऽर्ति॒ मार्ति॒ मा । \newline
18. आर्च्छ॑ त्यृच्छ॒ त्यार्च्छ॑ति । \newline
19. ऋ॒च्छ॒ त्ये॒षैष र्च्छ॑ त्यृच्छ त्ये॒षा । \newline
20. ए॒षा वै वा ए॒षैषा वै । \newline
21. वा अ॒ग्ने र॒ग्नेर् वै वा अ॒ग्नेः । \newline
22. अ॒ग्नेः प्रि॒या प्रि॒या ऽग्ने र॒ग्नेः प्रि॒या । \newline
23. प्रि॒या त॒नू स्त॒नूः प्रि॒या प्रि॒या त॒नूः । \newline
24. त॒नूर् यद् यत् त॒नू स्त॒नूर् यत् । \newline
25. यद् वै᳚श्वान॒रो वै᳚श्वान॒रो यद् यद् वै᳚श्वान॒रः । \newline
26. वै॒श्वा॒न॒रः प्रि॒याम् प्रि॒यां ॅवै᳚श्वान॒रो वै᳚श्वान॒रः प्रि॒याम् । \newline
27. प्रि॒या मे॒वैव प्रि॒याम् प्रि॒या मे॒व । \newline
28. ए॒वास्या᳚ स्यै॒वै वास्य॑ । \newline
29. अ॒स्य॒ त॒नुव॑म् त॒नुव॑ मस्यास्य त॒नुव᳚म् । \newline
30. त॒नुव॒ मवाव॑ त॒नुव॑म् त॒नुव॒ मव॑ । \newline
31. अव॑ रुन्धे रु॒न्धे ऽवाव॑ रुन्धे । \newline
32. रु॒न्धे॒ त्रीणि॒ त्रीणि॑ रुन्धे रुन्धे॒ त्रीणि॑ । \newline
33. त्रीण्ये॒ता न्ये॒तानि॒ त्रीणि॒ त्रीण्ये॒तानि॑ । \newline
34. ए॒तानि॑ ह॒वीꣳषि॑ ह॒वीꣳ ष्ये॒ता न्ये॒तानि॑ ह॒वीꣳषि॑ । \newline
35. ह॒वीꣳषि॑ भवन्ति भवन्ति ह॒वीꣳषि॑ ह॒वीꣳषि॑ भवन्ति । \newline
36. भ॒व॒न्ति॒ त्रय॒ स्त्रयो॑ भवन्ति भवन्ति॒ त्रयः॑ । \newline
37. त्रय॑ इ॒म इ॒मे त्रय॒ स्त्रय॑ इ॒मे । \newline
38. इ॒मे लो॒का लो॒का इ॒म इ॒मे लो॒काः । \newline
39. लो॒का ए॒षा मे॒षाम् ॅलो॒का लो॒का ए॒षाम् । \newline
40. ए॒षाम् ॅलो॒काना᳚म् ॅलो॒काना॑ मे॒षा मे॒षाम् ॅलो॒काना᳚म् । \newline
41. लो॒कानाꣳ॒॒ रोहा॑य॒ रोहा॑य लो॒काना᳚म् ॅलो॒कानाꣳ॒॒ रोहा॑य । \newline
42. रोहा॒येति॒ रोहा॑य । \newline

\textbf{Ghana Paata } \newline

1. का॒ल आग॑त॒ आग॑ते का॒ले का॒ल आग॑ते वि॒जाय॑ते वि॒जाय॑त॒ आग॑ते का॒ले का॒ल आग॑ते वि॒जाय॑ते । \newline
2. आग॑ते वि॒जाय॑ते वि॒जाय॑त॒ आग॑त॒ आग॑ते वि॒जाय॑त ए॒व मे॒वं ॅवि॒जाय॑त॒ आग॑त॒ आग॑ते वि॒जाय॑त ए॒वम् । \newline
3. आग॑त॒ इत्या - ग॒ते॒ । \newline
4. वि॒जाय॑त ए॒व मे॒वं ॅवि॒जाय॑ते वि॒जाय॑त ए॒व मे॒वै वैवं ॅवि॒जाय॑ते वि॒जाय॑त ए॒व मे॒व । \newline
5. वि॒जाय॑त॒ इति॑ वि - जाय॑ते । \newline
6. ए॒व मे॒वै वैव मे॒व मे॒व सं॑ॅवथ्स॒रꣳ सं॑ॅवथ्स॒र मे॒वैव मे॒व मे॒व सं॑ॅवथ्स॒रम् । \newline
7. ए॒व सं॑ॅवथ्स॒रꣳ सं॑ॅवथ्स॒र मे॒वैव सं॑ॅवथ्स॒र मा॒प्त्वा ऽऽप्त्वा सं॑ॅवथ्स॒र मे॒वैव सं॑ॅवथ्स॒र मा॒प्त्वा । \newline
8. सं॒ॅव॒थ्स॒र मा॒प्त्वा ऽऽप्त्वा सं॑ॅवथ्स॒रꣳ सं॑ॅवथ्स॒र मा॒प्त्वा का॒ले का॒ल आ॒प्त्वा सं॑ॅवथ्स॒रꣳ सं॑ॅवथ्स॒र मा॒प्त्वा का॒ले । \newline
9. सं॒ॅव॒थ्स॒रमिति॑ सं - व॒थ्स॒रम् । \newline
10. आ॒प्त्वा का॒ले का॒ल आ॒प्त्वा ऽऽप्त्वा का॒ल आग॑त॒ आग॑ते का॒ल आ॒प्त्वा ऽऽप्त्वा का॒ल आग॑ते । \newline
11. का॒ल आग॑त॒ आग॑ते का॒ले का॒ल आग॑ते॒ ऽग्नि म॒ग्नि माग॑ते का॒ले का॒ल आग॑ते॒ ऽग्निम् । \newline
12. आग॑ते॒ ऽग्नि म॒ग्नि माग॑त॒ आग॑ते॒ ऽग्निम् चि॑नुते चिनुते॒ ऽग्नि माग॑त॒ आग॑ते॒ ऽग्निम् चि॑नुते । \newline
13. आग॑त॒ इत्या - ग॒ते॒ । \newline
14. अ॒ग्निम् चि॑नुते चिनुते॒ ऽग्नि म॒ग्निम् चि॑नुते॒ न न चि॑नुते॒ ऽग्नि म॒ग्निम् चि॑नुते॒ न । \newline
15. चि॒नु॒ते॒ न न चि॑नुते चिनुते॒ नार्ति॒ मार्ति॒म् न चि॑नुते चिनुते॒ नार्ति᳚म् । \newline
16. नार्ति॒ मार्ति॒म् न नार्ति॒ मा ऽऽर्ति॒म् न नार्ति॒ मा । \newline
17. आर्ति॒ मा ऽऽर्ति॒ मार्ति॒ मार्च्छ॑ त्यृच्छ॒ त्याऽऽर्ति॒ मार्ति॒ मार्च्छ॑ति । \newline
18. आर्च्छ॑ त्यृच्छ॒ त्यार्च्छ॑ त्ये॒षैष र्‌च्छ॒ त्यार्च्छ॑ त्ये॒षा । \newline
19. ऋ॒च्छ॒ त्ये॒षैष र्च्छ॑ त्यृच्छ त्ये॒षा वै वा ए॒ष र्च्छ॑ त्यृच्छ त्ये॒षा वै । \newline
20. ए॒षा वै वा ए॒षैषा वा अ॒ग्ने र॒ग्नेर् वा ए॒षैषा वा अ॒ग्नेः । \newline
21. वा अ॒ग्ने र॒ग्नेर् वै वा अ॒ग्नेः प्रि॒या प्रि॒या ऽग्नेर् वै वा अ॒ग्नेः प्रि॒या । \newline
22. अ॒ग्नेः प्रि॒या प्रि॒या ऽग्ने र॒ग्नेः प्रि॒या त॒नू स्त॒नूः प्रि॒या ऽग्ने र॒ग्नेः प्रि॒या त॒नूः । \newline
23. प्रि॒या त॒नू स्त॒नूः प्रि॒या प्रि॒या त॒नूर् यद् यत् त॒नूः प्रि॒या प्रि॒या त॒नूर् यत् । \newline
24. त॒नूर् यद् यत् त॒नू स्त॒नूर् यद् वै᳚श्वान॒रो वै᳚श्वान॒रो यत् त॒नू स्त॒नूर् यद् वै᳚श्वान॒रः । \newline
25. यद् वै᳚श्वान॒रो वै᳚श्वान॒रो यद् यद् वै᳚श्वान॒रः प्रि॒याम् प्रि॒यां ॅवै᳚श्वान॒रो यद् यद् वै᳚श्वान॒रः प्रि॒याम् । \newline
26. वै॒श्वा॒न॒रः प्रि॒याम् प्रि॒यां ॅवै᳚श्वान॒रो वै᳚श्वान॒रः प्रि॒या मे॒वैव प्रि॒यां ॅवै᳚श्वान॒रो वै᳚श्वान॒रः प्रि॒या मे॒व । \newline
27. प्रि॒या मे॒वैव प्रि॒याम् प्रि॒या मे॒वास्या᳚ स्यै॒व प्रि॒याम् प्रि॒या मे॒वास्य॑ । \newline
28. ए॒वास्या᳚ स्यै॒वै वास्य॑ त॒नुव॑म् त॒नुव॑ मस्यै॒वै वास्य॑ त॒नुव᳚म् । \newline
29. अ॒स्य॒ त॒नुव॑म् त॒नुव॑ मस्यास्य त॒नुव॒ मवाव॑ त॒नुव॑ मस्यास्य त॒नुव॒ मव॑ । \newline
30. त॒नुव॒ मवाव॑ त॒नुव॑म् त॒नुव॒ मव॑ रुन्धे रु॒न्धे ऽव॑ त॒नुव॑म् त॒नुव॒ मव॑ रुन्धे । \newline
31. अव॑ रुन्धे रु॒न्धे ऽवाव॑ रुन्धे॒ त्रीणि॒ त्रीणि॑ रु॒न्धे ऽवाव॑ रुन्धे॒ त्रीणि॑ । \newline
32. रु॒न्धे॒ त्रीणि॒ त्रीणि॑ रुन्धे रुन्धे॒ त्रीण्ये॒ता न्ये॒तानि॒ त्रीणि॑ रुन्धे रुन्धे॒ त्रीण्ये॒तानि॑ । \newline
33. त्रीण्ये॒ता न्ये॒तानि॒ त्रीणि॒ त्रीण्ये॒तानि॑ ह॒वीꣳषि॑ ह॒वीꣳ ष्ये॒तानि॒ त्रीणि॒ त्रीण्ये॒तानि॑ ह॒वीꣳषि॑ । \newline
34. ए॒तानि॑ ह॒वीꣳषि॑ ह॒वीꣳ ष्ये॒ता न्ये॒तानि॑ ह॒वीꣳषि॑ भवन्ति भवन्ति ह॒वीꣳ ष्ये॒ता न्ये॒तानि॑ 
ह॒वीꣳषि॑ भवन्ति । \newline
35. ह॒वीꣳषि॑ भवन्ति भवन्ति ह॒वीꣳषि॑ ह॒वीꣳषि॑ भवन्ति॒ त्रय॒ स्त्रयो॑ भवन्ति ह॒वीꣳषि॑ ह॒वीꣳषि॑ भवन्ति॒ त्रयः॑ । \newline
36. भ॒व॒न्ति॒ त्रय॒ स्त्रयो॑ भवन्ति भवन्ति॒ त्रय॑ इ॒म इ॒मे त्रयो॑ भवन्ति भवन्ति॒ त्रय॑ इ॒मे । \newline
37. त्रय॑ इ॒म इ॒मे त्रय॒ स्त्रय॑ इ॒मे लो॒का लो॒का इ॒मे त्रय॒ स्त्रय॑ इ॒मे लो॒काः । \newline
38. इ॒मे लो॒का लो॒का इ॒म इ॒मे लो॒का ए॒षा मे॒षाम् ॅलो॒का इ॒म इ॒मे लो॒का ए॒षाम् । \newline
39. लो॒का ए॒षा मे॒षाम् ॅलो॒का लो॒का ए॒षाम् ॅलो॒काना᳚म् ॅलो॒काना॑ मे॒षाम् ॅलो॒का लो॒का ए॒षाम् ॅलो॒काना᳚म् । \newline
40. ए॒षाम् ॅलो॒काना᳚म् ॅलो॒काना॑ मे॒षा मे॒षाम् ॅलो॒कानाꣳ॒॒ रोहा॑य॒ रोहा॑य लो॒काना॑ मे॒षा मे॒षाम् ॅलो॒कानाꣳ॒॒ रोहा॑य । \newline
41. लो॒कानाꣳ॒॒ रोहा॑य॒ रोहा॑य लो॒काना᳚म् ॅलो॒कानाꣳ॒॒ रोहा॑य । \newline
42. रोहा॒येति॒ रोहा॑य । \newline
\pagebreak
\markright{ TS 5.5.2.1  \hfill https://www.vedavms.in \hfill}

\section{ TS 5.5.2.1 }

\textbf{TS 5.5.2.1 } \newline
\textbf{Samhita Paata} \newline

प्र॒जाप॑तिः प्र॒जाः सृ॒ष्ट्वा प्रे॒णाऽनु॒ प्रावि॑श॒त् ताभ्यः॒ पुनः॒ संभ॑वितुं॒ नाश॑क्नो॒थ् सो᳚ऽब्रवीदृ॒द्ध्नव॒दिथ् स यो मे॒तः पुनः॑ संचि॒नव॒दिति॒ तं दे॒वाः सम॑चिन्व॒न् ततो॒ वै त आ᳚र्द्ध्नुव॒न्॒ यथ् स॒मचि॑न्व॒न् तच्चित्य॑स्य चित्य॒त्वं ॅय ए॒वं ॅवि॒द्वान॒ग्निं चि॑नु॒त ऋ॒द्ध्नोत्ये॒व कस्मै॒ कम॒ग्निश्ची॑यत॒ इत्या॑हुरग्नि॒वा - [  ] \newline

\textbf{Pada Paata} \newline

प्र॒जाप॑ति॒रिति॑ प्र॒जा - प॒तिः॒ । प्र॒जा इति॑ प्र - जाः । सृ॒ष्ट्वा । प्रे॒णा । अनु॑ । प्रेति॑ । अ॒वि॒श॒त् । ताभ्यः॑ । पुनः॑ । संभ॑वितु॒मिति॒ सं - भ॒वि॒तु॒म् । न । अ॒श॒क्नो॒त् । सः । अ॒ब्र॒वी॒त् । ऋ॒द्ध्नव॑त् । इत् । सः । यः । मा॒ । इ॒तः । पुनः॑ । स॒चिं॒नव॒दिति॑ सं - चि॒नव॑त् । इति॑ । तम् । दे॒वाः । समिति॑ । अ॒चि॒न्व॒न्न् । ततः॑ । वै । ते । आ॒द्‌र्ध्नु॒व॒न्न् । यत् । स॒मचि॑न्व॒न्निति॑ सं - अचि॑न्वन्न् । तत् । चित्य॑स्य । चि॒त्य॒त्वमिति॑ चित्य - त्वम् । यः । ए॒वम् । वि॒द्वान् । अ॒ग्निम् । चि॒नु॒ते । ऋ॒द्ध्नोति॑ । ए॒व । कस्मै᳚ । कम् । अ॒ग्निः । ची॒य॒ते॒ । इति॑ । आ॒हुः॒ । अ॒ग्नि॒वानित्य॑ग्नि - वान् ।  \newline


\textbf{Krama Paata} \newline

प्र॒जाप॑तिः प्र॒जाः । प्र॒जाप॑ति॒रिति॑ प्र॒जा - प॒तिः॒ । प्र॒जाः सृ॒ष्ट्वा । प्र॒जा इति॑ प्र - जाः । सृ॒ष्ट्वा प्रे॒णा । प्रे॒णाऽनु॑ । अनु॒ प्र । प्रावि॑शत् । अ॒वि॒श॒त् ताभ्यः॑ । ताभ्यः॒ पुनः॑ । पुनः॒ सम्भ॑वितुम् । सम्भ॑वितु॒म् न । सम्भ॑वितु॒मिति॒ सम् - भ॒वि॒तु॒म् । नाश॑क्नोत् । अ॒श॒क्नो॒थ् सः । सो᳚ऽब्रवीत् । अ॒ब्र॒वी॒दृ॒द्ध्नव॑त् । ऋ॒द्ध्नव॒दित् । इथ् सः । स यः । यो मा᳚ । मे॒तः । इ॒तः पुनः॑ । पुनः॑ सञ्चि॒नव॑त् । स॒ञ्चि॒नव॒दिति॑ । स॒ञ्चि॒नव॒दिति॑ सम् - चि॒नव॑त् । इति॒ तम् । तम् दे॒वाः । दे॒वाः सम् । सम॑चिन्वन्न् । अ॒चि॒न्व॒न् ततः॑ । ततो॒ वै । वै ते । त आ᳚र्द्ध्नुवन्न् । आ॒र्द्ध्नु॒व॒न्॒. यत् । यथ् स॒मचि॑न्वन्न् । स॒मचि॑न्व॒न् तत् । स॒मचि॑न्व॒न्निति॑ सम् - अचि॑न्वन्न् । तच् चित्य॑स्य । चित्य॑स्य चित्य॒त्वम् । चि॒त्य॒त्वम् ॅयः । चि॒त्य॒त्वमिति॑ चित्य - त्वम् । य ए॒वम् । ए॒वम् ॅवि॒द्वान् । वि॒द्वान॒ग्निम् । अ॒ग्निम् चि॑नु॒ते । चि॒नु॒त ऋ॒द्ध्नोति॑ । ऋ॒द्ध्नोत्ये॒व । ए॒व कस्मै᳚ । कस्मै॒ कम् । कम॒ग्निः । अ॒ग्निश्ची॑यते । ची॒य॒त॒ इति॑ । इत्या॑हुः । आ॒हु॒र॒ग्नि॒वान् । अ॒ग्नि॒वान॑सानि । अ॒ग्नि॒वानित्य॑ग्नि - वान् \newline

\textbf{Jatai Paata} \newline

1. प्र॒जाप॑तिः प्र॒जाः प्र॒जाः प्र॒जाप॑तिः प्र॒जाप॑तिः प्र॒जाः । \newline
2. प्र॒जाप॑ति॒रिति॑ प्र॒जा - प॒तिः॒ । \newline
3. प्र॒जाः सृ॒ष्ट्वा सृ॒ष्ट्वा प्र॒जाः प्र॒जाः सृ॒ष्ट्वा । \newline
4. प्र॒जा इति॑ प्र - जाः । \newline
5. सृ॒ष्ट्वा प्रे॒णा प्रे॒णा सृ॒ष्ट्वा सृ॒ष्ट्वा प्रे॒णा । \newline
6. प्रे॒णा ऽन्वनु॑ प्रे॒णा प्रे॒णा ऽनु॑ । \newline
7. अनु॒ प्र प्राण्वनु॒ प्र । \newline
8. प्रावि॑श दविश॒त् प्र प्रावि॑शत् । \newline
9. अ॒वि॒श॒त् ताभ्य॒ स्ताभ्यो॑ ऽविश दविश॒त् ताभ्यः॑ । \newline
10. ताभ्यः॒ पुनः॒ पुन॒ स्ताभ्य॒ स्ताभ्यः॒ पुनः॑ । \newline
11. पुनः॒ संभ॑वितुꣳ॒॒ संभ॑वितु॒म् पुनः॒ पुनः॒ संभ॑वितुम् । \newline
12. संभ॑वितु॒म् न न संभ॑वितुꣳ॒॒ संभ॑वितु॒म् न । \newline
13. संभ॑वितु॒मिति॒ सं - भ॒वि॒तु॒म् । \newline
14. नाश॑क्नो दशक्नो॒न् न नाश॑क्नोत् । \newline
15. अ॒श॒क्नो॒थ् स सो॑ ऽशक्नो दशक्नो॒थ् सः । \newline
16. सो᳚ ऽब्रवी दब्रवी॒थ् स सो᳚ ऽब्रवीत् । \newline
17. अ॒ब्र॒वी॒ दृ॒द्ध्नव॑ दृ॒द्ध्नव॑ दब्रवी दब्रवी दृ॒द्ध्नव॑त् । \newline
18. ऋ॒द्ध्नव॒ दिदि दृ॒द्ध्नव॑ दृ॒द्ध्नव॒ दित् । \newline
19. इथ् स सेदिथ् सः । \newline
20. स यो यः स स यः । \newline
21. यो मा॑ मा॒ यो यो मा᳚ । \newline
22. मे॒त इ॒तो मा॑ मे॒तः । \newline
23. इ॒तः पुनः॒ पुन॑ रि॒त इ॒तः पुनः॑ । \newline
24. पुनः॑ सञ्चि॒नव॑थ् सञ्चि॒नव॒त् पुनः॒ पुनः॑ सञ्चि॒नव॑त् । \newline
25. स॒ञ्चि॒नव॒ दितीति॑ सञ्चि॒नव॑थ् सञ्चि॒नव॒ दिति॑ । \newline
26. स॒ञ्चि॒नव॒दिति॑ सं - चि॒नव॑त् । \newline
27. इति॒ तम् तमितीति॒ तम् । \newline
28. तम् दे॒वा दे॒वा स्तम् तम् दे॒वाः । \newline
29. दे॒वाः सꣳ सम् दे॒वा दे॒वाः सम् । \newline
30. स म॑चिन्वन् नचिन्व॒न् थ्सꣳ स म॑चिन्वन्न् । \newline
31. अ॒चि॒न्व॒न् तत॒ स्ततो॑ ऽचिन्वन् नचिन्व॒न् ततः॑ । \newline
32. ततो॒ वै वै तत॒ स्ततो॒ वै । \newline
33. वै ते ते वै वै ते । \newline
34. त आ᳚र्द्ध्नुवन् नार्द्ध्नुव॒न् ते त आ᳚र्द्ध्नुवन्न् । \newline
35. आ॒र्द्ध्नु॒व॒न्॒. यद् यदा᳚र्द्ध्नुवन् नार्द्ध्नुव॒न्॒. यत् । \newline
36. यथ् स॒मचि॑न्वन् थ्स॒मचि॑न्व॒न्॒. यद् यथ् स॒मचि॑न्वन्न् । \newline
37. स॒मचि॑न्व॒न् तत् तथ् स॒मचि॑न्वन् थ्स॒मचि॑न्व॒न् तत् । \newline
38. स॒मचि॑न्व॒न्निति॑ सं - अचि॑न्वन्न् । \newline
39. तच् चित्य॑स्य॒ चित्य॑स्य॒ तत् तच् चित्य॑स्य । \newline
40. चित्य॑स्य चित्य॒त्वम् चि॑त्य॒त्वम् चित्य॑स्य॒ चित्य॑स्य चित्य॒त्वम् । \newline
41. चि॒त्य॒त्वं ॅयो यश्चि॑त्य॒त्वम् चि॑त्य॒त्वं ॅयः । \newline
42. चि॒त्य॒त्वमिति॑ चित्य - त्वम् । \newline
43. य ए॒व मे॒वं ॅयो य ए॒वम् । \newline
44. ए॒वं ॅवि॒द्वान्. वि॒द्वा ने॒व मे॒वं ॅवि॒द्वान् । \newline
45. वि॒द्वा न॒ग्नि म॒ग्निं ॅवि॒द्वान्. वि॒द्वा न॒ग्निम् । \newline
46. अ॒ग्निम् चि॑नु॒ते चि॑नु॒ते᳚ ऽग्नि म॒ग्निम् चि॑नु॒ते । \newline
47. चि॒नु॒त ऋ॒द्ध्नो त्यृ॒द्ध्नोति॑ चिनु॒ते चि॑नु॒त ऋ॒द्ध्नोति॑ । \newline
48. ऋ॒द्ध्नो त्ये॒वैव र्द्ध्नो त्यृ॒द्ध्नो त्ये॒व । \newline
49. ए॒व कस्मै॒ कस्मा॑ ए॒वैव कस्मै᳚ । \newline
50. कस्मै॒ कम् कम् कस्मै॒ कस्मै॒ कम् । \newline
51. क म॒ग्नि र॒ग्निः कम् क म॒ग्निः । \newline
52. अ॒ग्नि श्ची॑यते चीयते॒ ऽग्नि र॒ग्नि श्ची॑यते । \newline
53. ची॒य॒त॒ इतीति॑ चीयते चीयत॒ इति॑ । \newline
54. इत्या॑हु राहु॒ रिती त्या॑हुः । \newline
55. आ॒हु॒ र॒ग्नि॒वा न॑ग्नि॒वा ना॑हु राहु रग्नि॒वान् । \newline
56. अ॒ग्नि॒वा न॑सान्यसा न्यग्नि॒वा न॑ग्नि॒वा न॑सानि । \newline
57. अ॒ग्नि॒वानित्य॑ग्नि - वान् । \newline

\textbf{Ghana Paata } \newline

1. प्र॒जाप॑तिः प्र॒जाः प्र॒जाः प्र॒जाप॑तिः प्र॒जाप॑तिः प्र॒जाः सृ॒ष्ट्वा सृ॒ष्ट्वा प्र॒जाः प्र॒जाप॑तिः प्र॒जाप॑तिः प्र॒जाः सृ॒ष्ट्वा । \newline
2. प्र॒जाप॑ति॒रिति॑ प्र॒जा - प॒तिः॒ । \newline
3. प्र॒जाः सृ॒ष्ट्वा सृ॒ष्ट्वा प्र॒जाः प्र॒जाः सृ॒ष्ट्वा प्रे॒णा प्रे॒णा सृ॒ष्ट्वा प्र॒जाः प्र॒जाः सृ॒ष्ट्वा प्रे॒णा । \newline
4. प्र॒जा इति॑ प्र - जाः । \newline
5. सृ॒ष्ट्वा प्रे॒णा प्रे॒णा सृ॒ष्ट्वा सृ॒ष्ट्वा प्रे॒णा ऽन्वनु॑ प्रे॒णा सृ॒ष्ट्वा सृ॒ष्ट्वा प्रे॒णा ऽनु॑ । \newline
6. प्रे॒णा ऽन्वनु॑ प्रे॒णा प्रे॒णा ऽनु॒ प्र प्राणु॑ प्रे॒णा प्रे॒णा ऽनु॒ प्र । \newline
7. अनु॒ प्र प्राण्वनु॒ प्रावि॑श दविश॒त् प्राण्वनु॒ प्रावि॑शत् । \newline
8. प्रावि॑श दविश॒त् प्र प्रावि॑श॒त् ताभ्य॒ स्ताभ्यो॑ ऽविश॒त् प्र प्रावि॑श॒त् ताभ्यः॑ । \newline
9. अ॒वि॒श॒त् ताभ्य॒ स्ताभ्यो॑ ऽविश दविश॒त् ताभ्यः॒ पुनः॒ पुन॒ स्ताभ्यो॑ ऽविश दविश॒त् ताभ्यः॒ पुनः॑ । \newline
10. ताभ्यः॒ पुनः॒ पुन॒ स्ताभ्य॒ स्ताभ्यः॒ पुनः॒ संभ॑वितुꣳ॒॒ संभ॑वितु॒म् पुन॒ स्ताभ्य॒ स्ताभ्यः॒ पुनः॒ संभ॑वितुम् । \newline
11. पुनः॒ संभ॑वितुꣳ॒॒ संभ॑वितु॒म् पुनः॒ पुनः॒ संभ॑वितु॒म् न न संभ॑वितु॒म् पुनः॒ पुनः॒ संभ॑वितु॒म् न । \newline
12. संभ॑वितु॒म् न न संभ॑वितुꣳ॒॒ संभ॑वितु॒म् नाश॑क्नो दशक्नो॒म् न संभ॑वितुꣳ॒॒ संभ॑वितु॒म् नाश॑क्नोत् । \newline
13. संभ॑वितु॒मिति॒ सं - भ॒वि॒तु॒म् । \newline
14. नाश॑क्नो दशक्नो॒न् न नाश॑क्नो॒थ् स सो॑ ऽशक्नो॒न् न नाश॑क्नो॒थ् सः । \newline
15. अ॒श॒क्नो॒थ् स सो॑ ऽशक्नो दशक्नो॒थ् सो᳚ ऽब्रवी दब्रवी॒थ् सो॑ ऽशक्नो दशक्नो॒थ् सो᳚ ऽब्रवीत् । \newline
16. सो᳚ ऽब्रवी दब्रवी॒थ् स सो᳚ ऽब्रवी दृ॒द्ध्नव॑ दृ॒द्ध्नव॑ दब्रवी॒थ् स सो᳚ ऽब्रवी दृ॒द्ध्नव॑त् । \newline
17. अ॒ब्र॒वी॒ दृ॒द्ध्नव॑ दृ॒द्ध्नव॑ दब्रवी दब्रवी दृ॒द्ध्नव॒ दिदि दृ॒द्ध्नव॑ दब्रवी दब्रवी दृ॒द्ध्नव॒दित् । \newline
18. ऋ॒द्ध्नव॒ दिदि दृ॒द्ध्नव॑ दृ॒द्ध्नव॒दिथ् स स इदृ॒द्ध्नव॑ दृ॒द्ध्नव॒दिथ् सः । \newline
19. इथ् स स इदिथ् स यो यः स इदिथ् स यः । \newline
20. स यो यः स स यो मा॑ मा॒ यः स स यो मा᳚ । \newline
21. यो मा॑ मा॒ यो यो मे॒त इ॒तो मा॒ यो यो मे॒तः । \newline
22. मे॒त इ॒तो मा॑ मे॒तः पुनः॒ पुन॑ रि॒तो मा॑ मे॒तः पुनः॑ । \newline
23. इ॒तः पुनः॒ पुन॑ रि॒त इ॒तः पुनः॑ सञ्चि॒नव॑थ् सञ्चि॒नव॒त् पुन॑ रि॒त इ॒तः पुनः॑ सञ्चि॒नव॑त् । \newline
24. पुनः॑ सञ्चि॒नव॑थ् सञ्चि॒नव॒त् पुनः॒ पुनः॑ सञ्चि॒नव॒ दितीति॑ सञ्चि॒नव॒त् पुनः॒ पुनः॑ सञ्चि॒नव॒ दिति॑ । \newline
25. स॒ञ्चि॒नव॒ दितीति॑ सञ्चि॒नव॑थ् सञ्चि॒नव॒ दिति॒ तम् त मिति॑ सञ्चि॒नव॑थ् सञ्चि॒नव॒ दिति॒ तम् । \newline
26. स॒ञ्चि॒नव॒दिति॑ सं - चि॒नव॑त् । \newline
27. इति॒ तम् त मितीति॒ तम् दे॒वा दे॒वा स्त मितीति॒ तम् दे॒वाः । \newline
28. तम् दे॒वा दे॒वा स्तम् तम् दे॒वाः सꣳ सम् दे॒वा स्तम् तम् दे॒वाः सम् । \newline
29. दे॒वाः सꣳ सम् दे॒वा दे॒वाः स म॑चिन्वन् नचिन्व॒न् थ्सम् दे॒वा दे॒वाः स म॑चिन्वन्न् । \newline
30. स म॑चिन्वन् नचिन्व॒न् थ्सꣳ स म॑चिन्व॒न् तत॒ स्ततो॑ ऽचिन्व॒न् थ्सꣳ स म॑चिन्व॒न् ततः॑ । \newline
31. अ॒चि॒न्व॒न् तत॒ स्ततो॑ ऽचिन्वन् नचिन्व॒न् ततो॒ वै वै ततो॑ ऽचिन्वन् नचिन्व॒न् ततो॒ वै । \newline
32. ततो॒ वै वै तत॒ स्ततो॒ वै ते ते वै तत॒ स्ततो॒ वै ते । \newline
33. वै ते ते वै वै त आ᳚र्द्ध्नुवन् नार्द्ध्नुव॒न् ते वै वै त आ᳚र्द्ध्नुवन्न् । \newline
34. त आ᳚र्द्ध्नुवन् नार्द्ध्नुव॒न् ते त आ᳚र्द्ध्नुव॒न्॒. यद् यदा᳚र्द्ध्नुव॒न् ते त आ᳚र्द्ध्नुव॒न्॒. यत् । \newline
35. आ॒र्द्ध्नु॒व॒न्॒. यद् यदा᳚र्द्ध्नुवन् नार्द्ध्नुव॒न्॒. यथ् स॒मचि॑न्वन् थ्स॒मचि॑न्व॒न्॒. यदा᳚र्द्ध्नुवन् नार्द्ध्नुव॒न्॒. यथ् स॒मचि॑न्वन्न् । \newline
36. यथ् स॒मचि॑न्वन् थ्स॒मचि॑न्व॒न्॒. यद् यथ् स॒मचि॑न्व॒न् तत् तथ् स॒मचि॑न्व॒न्॒. यद् यथ् स॒मचि॑न्व॒न् तत् । \newline
37. स॒मचि॑न्व॒न् तत् तथ् स॒मचि॑न्वन् थ्स॒मचि॑न्व॒न् तच् चित्य॑स्य॒ चित्य॑स्य॒ तथ् स॒मचि॑न्वन् थ्स॒मचि॑न्व॒न् तच् चित्य॑स्य । \newline
38. स॒मचि॑न्व॒न्निति॑ सं - अचि॑न्वन्न् । \newline
39. तच् चित्य॑स्य॒ चित्य॑स्य॒ तत् तच् चित्य॑स्य चित्य॒त्वम् चि॑त्य॒त्वम् चित्य॑स्य॒ तत् तच् चित्य॑स्य चित्य॒त्वम् । \newline
40. चित्य॑स्य चित्य॒त्वम् चि॑त्य॒त्वम् चित्य॑स्य॒ चित्य॑स्य चित्य॒त्वं ॅयो यश्चि॑त्य॒त्वम् चित्य॑स्य॒ चित्य॑स्य चित्य॒त्वं ॅयः । \newline
41. चि॒त्य॒त्वं ॅयो यश्चि॑त्य॒त्वम् चि॑त्य॒त्वं ॅय ए॒व मे॒वं ॅयश्चि॑त्य॒त्वम् चि॑त्य॒त्वं ॅय ए॒वम् । \newline
42. चि॒त्य॒त्वमिति॑ चित्य - त्वम् । \newline
43. य ए॒व मे॒वं ॅयो य ए॒वं ॅवि॒द्वान्. वि॒द्वा ने॒वं ॅयो य ए॒वं ॅवि॒द्वान् । \newline
44. ए॒वं ॅवि॒द्वान्. वि॒द्वा ने॒व मे॒वं ॅवि॒द्वा न॒ग्नि म॒ग्निं ॅवि॒द्वा ने॒व मे॒वं ॅवि॒द्वा न॒ग्निम् । \newline
45. वि॒द्वा न॒ग्नि म॒ग्निं ॅवि॒द्वान्. वि॒द्वा न॒ग्निम् चि॑नु॒ते चि॑नु॒ते᳚ ऽग्निं ॅवि॒द्वान्. वि॒द्वा न॒ग्निम् चि॑नु॒ते । \newline
46. अ॒ग्निम् चि॑नु॒ते चि॑नु॒ते᳚ ऽग्नि म॒ग्निम् चि॑नु॒त ऋ॒द्ध्नो त्यृ॒द्ध्नोति॑ चिनु॒ते᳚ ऽग्नि म॒ग्निम् चि॑नु॒त ऋ॒द्ध्नोति॑ । \newline
47. चि॒नु॒त ऋ॒द्ध्नो त्यृ॒द्ध्नोति॑ चिनु॒ते चि॑नु॒त ऋ॒द्ध्नो त्ये॒वैव र्‌द्ध्नोति॑ चिनु॒ते चि॑नु॒त ऋ॒द्ध्नो त्ये॒व । \newline
48. ऋ॒द्ध्नो त्ये॒वैव र्‌द्ध्नो त्यृ॒द्ध्नो त्ये॒व कस्मै॒ कस्मा॑ ए॒व र्‌द्ध्नो त्यृ॒द्ध्नो त्ये॒व कस्मै᳚ । \newline
49. ए॒व कस्मै॒ कस्मा॑ ए॒वैव कस्मै॒ कम् कम् कस्मा॑ ए॒वैव कस्मै॒ कम् । \newline
50. कस्मै॒ कम् कम् कस्मै॒ कस्मै॒ क म॒ग्नि र॒ग्निः कम् कस्मै॒ कस्मै॒ क म॒ग्निः । \newline
51. क म॒ग्नि र॒ग्निः कम् क म॒ग्नि श्ची॑यते चीयते॒ ऽग्निः कम् क म॒ग्नि श्ची॑यते । \newline
52. आ॒ग्नि श्ची॑यते चीयते॒ ऽग्नि र॒ग्नि श्ची॑यत॒ इतीति॑ चीयते॒ ऽग्नि र॒ग्नि श्ची॑यत॒ इति॑ । \newline
53. ची॒य॒त॒ इतीति॑ चीयते चीयत॒ इत्या॑हु राहु॒रिति॑ चीयते चीयत॒ इत्या॑हुः । \newline
54. इत्या॑हु राहु॒रिती त्या॑हु रग्नि॒वा न॑ग्नि॒वा ना॑हु॒रिती त्या॑हु रग्नि॒वान् । \newline
55. आ॒हु॒ र॒ग्नि॒वा न॑ग्नि॒वा ना॑हु राहु रग्नि॒वा न॑सा न्यसा न्यग्नि॒वा ना॑हु राहु रग्नि॒वा न॑सानि । \newline
56. अ॒ग्नि॒वा न॑सा न्यसा न्यग्नि॒वा न॑ग्नि॒वा न॑सा॒नीती त्य॑सा न्यग्नि॒वा न॑ग्नि॒वा न॑सा॒नीति॑ । \newline
57. अ॒ग्नि॒वानित्य॑ग्नि - वान् । \newline
\pagebreak
\markright{ TS 5.5.2.2  \hfill https://www.vedavms.in \hfill}

\section{ TS 5.5.2.2 }

\textbf{TS 5.5.2.2 } \newline
\textbf{Samhita Paata} \newline

-न॑सा॒नीति॒ वा अ॒ग्निश्ची॑यते ऽग्नि॒वाने॒व भ॑वति॒ कस्मै॒ कम॒ग्निश्ची॑यत॒ इत्या॑हुर्दे॒वा मा॑ वेद॒न्निति॒ वा अ॒ग्निश्ची॑यते वि॒दुरे॑नं दे॒वाः कस्मै॒ कम॒ग्निश्ची॑यत॒ इत्या॑हुर्गृ॒ह्य॑सा॒नीति॒ वा अ॒ग्निश्ची॑यते गृ॒ह्ये॑व भ॑वति॒ कस्मै॒ कम॒ग्निश्ची॑यत॒ इत्या॑हुः पशु॒मान॑सा॒नीति॒ वा अ॒ग्नि - [  ] \newline

\textbf{Pada Paata} \newline

अ॒सा॒नि॒ । इति॑ । वै । अ॒ग्निः । ची॒य॒ते॒ । अ॒ग्नि॒वानित्य॑ग्नि - वान् । ए॒व । भ॒व॒ति॒ । कस्मै᳚ । कम् । अ॒ग्निः । ची॒य॒ते॒ । इति॑ । आ॒हुः॒ । दे॒वाः । मा॒ । वे॒द॒न्न् । इति॑ । वै । अ॒ग्निः । ची॒य॒ते॒ । वि॒दुः । ए॒न॒म् । दे॒वाः । कस्मै᳚ । कम् । अ॒ग्निः । ची॒य॒ते॒ । इति॑ । आ॒हुः॒ । गृ॒ही । अ॒सा॒नि॒ । इति॑ । वै । अ॒ग्निः । ची॒य॒ते॒ । गृ॒ही । ए॒व । भ॒व॒ति॒ । कस्मै᳚ । कम् । अ॒ग्निः । ची॒य॒ते॒ । इति॑ । आ॒हुः॒ । प॒शु॒मानिति॑ पशु - मान् । अ॒सा॒नि॒ । इति॑ । वै । अ॒ग्निः ।  \newline


\textbf{Krama Paata} \newline

अ॒सा॒नीति॑ । इति॒ वै । वा अ॒ग्निः । अ॒ग्निश्ची॑यते । ची॒य॒ते॒ऽग्नि॒वान् । अ॒ग्नि॒वाने॒व । अ॒ग्नि॒वानित्य॑ग्नि - वान् । ए॒व भ॑वति । भ॒व॒ति॒ कस्मै᳚ । कस्मै॒ कम् । कम॒ग्निः । अ॒ग्निश्ची॑यते । ची॒य॒त॒ इति॑ । इत्या॑हुः । आ॒हु॒र् दे॒वाः । दे॒वा मा᳚ । मा॒ वे॒द॒न्न् । वे॒द॒न्निति॑ । इति॒ वै । वा अ॒ग्निः । अ॒ग्निश्ची॑यते । ची॒य॒ते॒ वि॒दुः । वि॒दुरे॑नम् । ए॒न॒म् दे॒वाः । दे॒वाः कस्मै᳚ । कस्मै॒ कम् । कम॒ग्निः । अ॒ग्निश्ची॑यते । ची॒य॒त॒ इति॑ । इत्या॑हुः । आ॒हु॒र् गृ॒ही । गृ॒ह्य॑सानि । अ॒सा॒नीति॑ । इति॒ वै । वा अ॒ग्निः । अ॒ग्निश्ची॑यते । ची॒य॒ते॒ गृ॒ही । गृ॒ह्ये॑व । ए॒व भ॑वति । भ॒व॒ति॒ कस्मै᳚ । कस्मै॒ कम् । कम॒ग्निः । अ॒ग्निश्चि॑यते । ची॒य॒त॒ इति॑ । इत्या॑हुः । आ॒हुः॒ प॒शु॒मान् । प॒शु॒मान॑सानि । प॒शु॒मानिति॑ पशु - मान् । अ॒सा॒नीति॑ । इति॒ वै । वा अ॒ग्निः । अ॒ग्निश्ची॑यते \newline

\textbf{Jatai Paata} \newline

1. अ॒सा॒नी तीत्य॑सा न्यसा॒ नीति॑ । \newline
2. इति॒ वै वा इतीति॒ वै । \newline
3. वा अ॒ग्नि र॒ग्निर् वै वा अ॒ग्निः । \newline
4. अ॒ग्नि श्ची॑यते चीयते॒ ऽग्नि र॒ग्नि श्ची॑यते । \newline
5. ची॒य॒ते॒ ऽग्नि॒वा न॑ग्नि॒वाꣳ श्ची॑यते चीयते ऽग्नि॒वान् । \newline
6. अ॒ग्नि॒वा ने॒वै वाग्नि॒वा न॑ग्नि॒वा ने॒व । \newline
7. अ॒ग्नि॒वानित्य॑ग्नि - वान् । \newline
8. ए॒व भ॑वति भव त्ये॒वैव भ॑वति । \newline
9. भ॒व॒ति॒ कस्मै॒ कस्मै॑ भवति भवति॒ कस्मै᳚ । \newline
10. कस्मै॒ कम् कम् कस्मै॒ कस्मै॒ कम् । \newline
11. क म॒ग्नि र॒ग्निः कम् क म॒ग्निः । \newline
12. अ॒ग्नि श्ची॑यते चीयते॒ ऽग्नि र॒ग्नि श्ची॑यते । \newline
13. ची॒य॒त॒ इतीति॑ चीयते चीयत॒ इति॑ । \newline
14. इत्या॑हु राहु॒रिती त्या॑हुः । \newline
15. आ॒हु॒र् दे॒वा दे॒वा आ॑हु राहुर् दे॒वाः । \newline
16. दे॒वा मा॑ मा दे॒वा दे॒वा मा᳚ । \newline
17. मा॒ वे॒द॒न्॒. वे॒द॒न् मा॒ मा॒ वे॒द॒न्न् । \newline
18. ꣡ए॒द॒न् नितीति॑ वेदन्. वेद॒न् निति॑ । \newline
19. इति॒ वै वा इतीति॒ वै । \newline
20. वा अ॒ग्नि र॒ग्निर् वै वा अ॒ग्निः । \newline
21. अ॒ग्नि श्ची॑यते चीयते॒ ऽग्नि र॒ग्नि श्ची॑यते । \newline
22. ची॒य॒ते॒ वि॒दुर् वि॒दु श्ची॑यते चीयते वि॒दुः । \newline
23. वि॒दु रे॑न मेनं ॅवि॒दुर् वि॒दु रे॑नम् । \newline
24. ए॒न॒म् दे॒वा दे॒वा ए॑न मेनम् दे॒वाः । \newline
25. दे॒वाः कस्मै॒ कस्मै॑ दे॒वा दे॒वाः कस्मै᳚ । \newline
26. कस्मै॒ कम् कम् कस्मै॒ कस्मै॒ कम् । \newline
27. क म॒ग्नि र॒ग्निः कम् क म॒ग्निः । \newline
28. अ॒ग्नि श्ची॑यते चीयते॒ ऽग्नि र॒ग्नि श्ची॑यते । \newline
29. ची॒य॒त॒ इतीति॑ चीयते चीयत॒ इति॑ । \newline
30. इत्या॑हु राहु॒रिती त्या॑हुः । \newline
31. आ॒हु॒र् गृ॒ही गृ॒ह्या॑हु राहुर् गृ॒ही । \newline
32. गृ॒ह्य॑सा न्यसानि गृ॒ही गृ॒ह्य॑सानि । \newline
33. अ॒सा॒नी तीत्य॑सा न्यसा॒नीति॑ । \newline
34. इति॒ वै वा इतीति॒ वै । \newline
35. वा अ॒ग्नि र॒ग्निर् वै वा अ॒ग्निः । \newline
36. अ॒ग्नि श्ची॑यते चीयते॒ ऽग्नि र॒ग्नि श्ची॑यते । \newline
37. ची॒य॒ते॒ गृ॒ही गृ॒ही ची॑यते चीयते गृ॒ही । \newline
38. गृ॒ह्ये॑ वैव गृ॒ही गृ॒ह्ये॑व । \newline
39. ए॒व भ॑वति भव त्ये॒वैव भ॑वति । \newline
40. भ॒व॒ति॒ कस्मै॒ कस्मै॑ भवति भवति॒ कस्मै᳚ । \newline
41. कस्मै॒ कम् कम् कस्मै॒ कस्मै॒ कम् । \newline
42. क म॒ग्नि र॒ग्निः कम् क म॒ग्निः । \newline
43. अ॒ग्नि श्ची॑यते चीयते॒ ऽग्नि र॒ग्नि श्ची॑यते । \newline
44. ची॒य॒त॒ इतीति॑ चीयते चीयत॒ इति॑ । \newline
45. इत्या॑हु राहु॒रिती त्या॑हुः । \newline
46. आ॒हुः॒ प॒शु॒मान् प॑शु॒मा ना॑हु राहुः पशु॒मान् । \newline
47. प॒शु॒मा न॑सा न्यसानि पशु॒मान् प॑शु॒मा न॑सानि । \newline
48. प॒शु॒मानिति॑ पशु - मान् । \newline
49. अ॒सा॒नी तीत्य॑सा न्यसा॒नीति॑ । \newline
50. इति॒ वै वा इतीति॒ वै । \newline
51. वा अ॒ग्नि र॒ग्निर् वै वा अ॒ग्निः । \newline
52. अ॒ग्नि श्ची॑यते चीयते॒ ऽग्नि र॒ग्नि श्ची॑यते । \newline

\textbf{Ghana Paata } \newline

1. अ॒सा॒नी तीत्य॑सा न्यसा॒नीति॒ वै वा इत्य॑सा न्यसा॒नीति॒ वै । \newline
2. इति॒ वै वा इतीति॒ वा अ॒ग्नि र॒ग्निर् वा इतीति॒ वा अ॒ग्निः । \newline
3. वा अ॒ग्नि र॒ग्निर् वै वा अ॒ग्नि श्ची॑यते चीयते॒ ऽग्निर् वै वा अ॒ग्नि श्ची॑यते । \newline
4. आ॒ग्नि श्ची॑यते चीयते॒ ऽग्नि र॒ग्नि श्ची॑यते ऽग्नि॒वा न॑ग्नि॒वाꣳ श्ची॑यते॒ ऽग्नि र॒ग्नि श्ची॑यते ऽग्नि॒वान् । \newline
5. ची॒य॒ते॒ ऽग्नि॒वा न॑ग्नि॒वाꣳ श्ची॑यते चीयते ऽग्नि॒वा ने॒वै वाग्नि॒वाꣳ श्ची॑यते चीयते ऽग्नि॒वा ने॒व । \newline
6. अ॒ग्नि॒वा ने॒वै वाग्नि॒वा न॑ग्नि॒वा ने॒व भ॑वति भव त्ये॒वाग्नि॒वा न॑ग्नि॒वा ने॒व भ॑वति । \newline
7. अ॒ग्नि॒वानित्य॑ग्नि - वान् । \newline
8. ए॒व भ॑वति भव त्ये॒वैव भ॑वति॒ कस्मै॒ कस्मै॑ भव त्ये॒वैव भ॑वति॒ कस्मै᳚ । \newline
9. भ॒व॒ति॒ कस्मै॒ कस्मै॑ भवति भवति॒ कस्मै॒ कम् कम् कस्मै॑ भवति भवति॒ कस्मै॒ कम् । \newline
10. कस्मै॒ कम् कम् कस्मै॒ कस्मै॒ क म॒ग्नि र॒ग्निः कम् कस्मै॒ कस्मै॒ क म॒ग्निः । \newline
11. क म॒ग्नि र॒ग्निः कम् क म॒ग्नि श्ची॑यते चीयते॒ ऽग्निः कम् क म॒ग्नि श्ची॑यते । \newline
12. अ॒ग्नि श्ची॑यते चीयते॒ ऽग्नि र॒ग्नि श्ची॑यत॒ इतीति॑ चीयते॒ ऽग्नि र॒ग्नि श्ची॑यत॒ इति॑ । \newline
13. ची॒य॒त॒ इतीति॑ चीयते चीयत॒ इत्या॑हु राहु॒रिति॑ चीयते चीयत॒ इत्या॑हुः । \newline
14. इत्या॑हु राहु॒रि तीत्या॑हुर् दे॒वा दे॒वा आ॑हु॒ रिती त्या॑हुर् दे॒वाः । \newline
15. आ॒हु॒र् दे॒वा दे॒वा आ॑हु राहुर् दे॒वा मा॑ मा दे॒वा आ॑हु राहुर् दे॒वा मा᳚ । \newline
16. दे॒वा मा॑ मा दे॒वा दे॒वा मा॑ वेदन्. वेदन् मा दे॒वा दे॒वा मा॑ वेदन्न् । \newline
17. मा॒ वे॒द॒न्॒. वे॒द॒न् मा॒ मा॒ वे॒द॒न् नितीति॑ वेदन् मा मा वेद॒न् निति॑ । \newline
18. ꣡ए॒द॒न् नितीति॑ वेदन्. वेद॒न् निति॒ वै वा इति॑ वेदन्. वेद॒न् निति॒ वै । \newline
19. इति॒ वै वा इतीति॒ वा अ॒ग्नि र॒ग्निर् वा इतीति॒ वा अ॒ग्निः । \newline
20. वा अ॒ग्नि र॒ग्निर् वै वा अ॒ग्नि श्ची॑यते चीयते॒ ऽग्निर् वै वा अ॒ग्नि श्ची॑यते । \newline
21. अ॒ग्नि श्ची॑यते चीयते॒ ऽग्नि र॒ग्नि श्ची॑यते वि॒दुर् वि॒दु श्ची॑यते॒ ऽग्नि र॒ग्नि श्ची॑यते वि॒दुः । \newline
22. ची॒य॒ते॒ वि॒दुर् वि॒दु श्ची॑यते चीयते वि॒दु रे॑न मेनं ॅवि॒दु श्ची॑यते चीयते वि॒दु रे॑नम् । \newline
23. वि॒दु रे॑न मेनं ॅवि॒दुर् वि॒दु रे॑नम् दे॒वा दे॒वा ए॑नं ॅवि॒दुर् वि॒दु रे॑नम् दे॒वाः । \newline
24. ए॒न॒म् दे॒वा दे॒वा ए॑न मेनम् दे॒वाः कस्मै॒ कस्मै॑ दे॒वा ए॑न मेनम् दे॒वाः कस्मै᳚ । \newline
25. दे॒वाः कस्मै॒ कस्मै॑ दे॒वा दे॒वाः कस्मै॒ कम् कम् कस्मै॑ दे॒वा दे॒वाः कस्मै॒ कम् । \newline
26. कस्मै॒ कम् कम् कस्मै॒ कस्मै॒ क म॒ग्नि र॒ग्निः कम् कस्मै॒ कस्मै॒ क म॒ग्निः । \newline
27. क म॒ग्नि र॒ग्निः कम् क म॒ग्नि श्ची॑यते चीयते॒ ऽग्निः कम् क म॒ग्नि श्ची॑यते । \newline
28. अ॒ग्नि श्ची॑यते चीयते॒ ऽग्नि र॒ग्नि श्ची॑यत॒ इतीति॑ चीयते॒ ऽग्नि र॒ग्नि श्ची॑यत॒ इति॑ । \newline
29. ची॒य॒त॒ इतीति॑ चीयते चीयत॒ इत्या॑हु राहु॒रिति॑ चीयते चीयत॒ इत्या॑हुः । \newline
30. इत्या॑हु राहु॒रि तीत्या॑हुर् गृ॒ही गृ॒ह्या॑हु॒ रिती त्या॑हुर् गृ॒ही । \newline
31. आ॒हु॒र् गृ॒ही गृ॒ह्या॑हु राहुर् गृ॒ह्य॑ सान्यसानि गृ॒ह्या॑हु राहुर् गृ॒ह्य॑सानि । \newline
32. गृ॒ह्य॑सा न्यसानि गृ॒ही गृ॒ह्य॑सा॒नी तीत्य॑सानि गृ॒ही गृ॒ह्य॑ सा॒नीति॑ । \newline
33. अ॒सा॒नीती त्य॑सा न्यसा॒नीति॒ वै वा इत्य॑सा न्यसा॒नीति॒ वै । \newline
34. इति॒ वै वा इतीति॒ वा अ॒ग्नि र॒ग्निर् वा इतीति॒ वा अ॒ग्निः । \newline
35. वा अ॒ग्नि र॒ग्निर् वै वा अ॒ग्नि श्ची॑यते चीयते॒ ऽग्निर् वै वा अ॒ग्नि श्ची॑यते । \newline
36. अ॒ग्नि श्ची॑यते चीयते॒ ऽग्नि र॒ग्नि श्ची॑यते गृ॒ही गृ॒ही ची॑यते॒ ऽग्नि र॒ग्नि श्ची॑यते गृ॒ही । \newline
37. ची॒य॒ते॒ गृ॒ही गृ॒ही ची॑यते चीयते गृ॒ह्ये॑वैव गृ॒ही ची॑यते चीयते गृ॒ह्ये॑व । \newline
38. गृ॒ह्ये॑ वैव गृ॒ही गृ॒ह्ये॑व भ॑वति भव त्ये॒व गृ॒ही गृ॒ह्ये॑व भ॑वति । \newline
39. ए॒व भ॑वति भव त्ये॒वैव भ॑वति॒ कस्मै॒ कस्मै॑ भव त्ये॒वैव भ॑वति॒ कस्मै᳚ । \newline
40. भ॒व॒ति॒ कस्मै॒ कस्मै॑ भवति भवति॒ कस्मै॒ कम् कम् कस्मै॑ भवति भवति॒ कस्मै॒ कम् । \newline
41. कस्मै॒ कम् कम् कस्मै॒ कस्मै॒ क म॒ग्नि र॒ग्निः कम् कस्मै॒ कस्मै॒ क म॒ग्निः । \newline
42. क म॒ग्नि र॒ग्निः कम् क म॒ग्नि श्ची॑यते चीयते॒ ऽग्निः कम् क म॒ग्नि श्ची॑यते । \newline
43. अ॒ग्नि श्ची॑यते चीयते॒ ऽग्नि र॒ग्नि श्ची॑यत॒ इतीति॑ चीयते॒ ऽग्नि र॒ग्नि श्ची॑यत॒ इति॑ । \newline
44. ची॒य॒त॒ इतीति॑ चीयते चीयत॒ इत्या॑हु राहु॒रिति॑ चीयते चीयत॒ इत्या॑हुः । \newline
45. इत्या॑हु राहु॒ रितीत्या॑हुः पशु॒मान् प॑शु॒मा ना॑हु॒ रितीत्या॑हुः पशु॒मान् । \newline
46. आ॒हुः॒ प॒शु॒मान् प॑शु॒मा ना॑हु राहुः पशु॒मा न॑सा न्यसानि पशु॒मा ना॑हु राहुः पशु॒मा न॑सानि । \newline
47. प॒शु॒मा न॑सा न्यसानि पशु॒मान् प॑शु॒मा न॑सा॒नी तीत्य॑सानि पशु॒मान् प॑शु॒मा न॑सा॒नीति॑ । \newline
48. प॒शु॒मानिति॑ पशु - मान् । \newline
49. अ॒सा॒नी तीत्य॑सा न्यसा॒नीति॒ वै वा इत्य॑सा न्यसा॒नीति॒ वै । \newline
50. इति॒ वै वा इतीति॒ वा अ॒ग्नि र॒ग्निर् वा इतीति॒ वा अ॒ग्निः । \newline
51. वा अ॒ग्नि र॒ग्निर् वै वा अ॒ग्नि श्ची॑यते चीयते॒ ऽग्निर् वै वा अ॒ग्नि श्ची॑यते । \newline
52. अ॒ग्नि श्ची॑यते चीयते॒ ऽग्नि र॒ग्नि श्ची॑यते पशु॒मान् प॑शु॒माꣳ श्ची॑यते॒ ऽग्नि र॒ग्नि श्ची॑यते पशु॒मान् । \newline
\pagebreak
\markright{ TS 5.5.2.3  \hfill https://www.vedavms.in \hfill}

\section{ TS 5.5.2.3 }

\textbf{TS 5.5.2.3 } \newline
\textbf{Samhita Paata} \newline

-श्ची॑यते पशु॒माने॒व भ॑वति॒ कस्मै॒ कम॒ग्निश्ची॑यत॒ इत्या॑हुः स॒प्त मा॒ पुरु॑षा॒ उप॑ जीवा॒निति॒ वा अ॒ग्निश्ची॑यते॒ त्रयः॒ प्राञ्च॒स्त्रयः॑ प्र॒त्यञ्च॑ आ॒त्मा स॑प्त॒म ए॒ताव॑न्त ए॒वैन॑म॒मुष्मि॑न् ॅलो॒क उप॑ जीवन्ति प्र॒जाप॑तिर॒ग्निम॑चिकीषत॒ तं पृ॑थि॒व्य॑ब्रवी॒न्न मय्य॒ग्निं चे᳚ष्य॒सेऽति॑ मा धक्ष्यति॒ सा त्वा॑ऽतिद॒ह्यमा॑ना॒ वि ध॑विष्ये॒ - [  ] \newline

\textbf{Pada Paata} \newline

ची॒य॒ते॒ । प॒शु॒मानिति॑ पशु - मान् । ए॒व । भ॒व॒ति॒ । कस्मै᳚ । कम् । अ॒ग्निः । ची॒य॒ते॒ । इति॑ । आ॒हुः॒ । स॒प्त । मा॒ । पुरु॑षाः । उपेति॑ । जी॒वा॒न् । इति॑ । वै । अ॒ग्निः । ची॒य॒ते॒ । त्रयः॑ । प्राञ्चः॑ । त्रयः॑ । प्र॒त्यञ्चः॑ । आ॒त्मा । स॒प्त॒मः । ए॒ताव॑न्तः । ए॒व । ए॒न॒म् । अ॒मुष्मिन्न्॑ । लो॒के । उपेति॑ । जी॒व॒न्ति॒ । प्र॒जाप॑ति॒रिति॑ प्र॒जा - प॒तिः॒ । अ॒ग्निम् । अ॒चि॒की॒ष॒त॒ । तम् । पृ॒थि॒वी । अ॒ब्र॒वी॒त् । न । मयि॑ । अ॒ग्निम् । चे॒ष्य॒से॒ । अतीति॑ । मा॒ । ध॒क्ष्य॒ति॒ । सा । त्वा॒ । अ॒ति॒द॒ह्यमा॒नेत्य॑ति - द॒ह्यमा॑ना । वीति॑ । ध॒वि॒ष्ये॒ ।  \newline


\textbf{Krama Paata} \newline

ची॒य॒ते॒ प॒शु॒मान् । प॒शु॒माने॒व । प॒शु॒मानिति॑ पशु - मान् । ए॒व भ॑वति । भ॒व॒ति॒ कस्मै᳚ । कस्मै॒ कम् । कम॒ग्निः । अ॒ग्निश्ची॑यते । ची॒य॒त॒ इति॑ । इत्या॑हुः । आ॒हुः॒ स॒प्त । स॒प्त मा᳚ । मा॒ पुरु॑षाः । पुरु॑षा॒ उप॑ । उप॑ जीवान् । जी॒वा॒निति॑ । इति॒ वै । वा अ॒ग्निः । अ॒ग्निश्ची॑यते । ची॒य॒ते॒ त्रयः॑ । त्रयः॒ प्राञ्चः॑ । प्राञ्च॒स्त्रयः॑ । त्रयः॑ प्र॒त्यञ्चः॑ । प्र॒त्यञ्च॑ आ॒त्मा । आ॒त्मा स॑प्त॒मः । स॒प्त॒म ए॒ताव॑न्तः । ए॒ताव॑न्त ए॒व । ए॒वैन᳚म् । ए॒न॒म॒मुष्मिन्न्॑ । अ॒मुष्मि॑न् ॅलो॒के । लो॒क उप॑ । उप॑ जीवन्ति । जी॒व॒न्ति॒ प्र॒जाप॑तिः । प्र॒जाप॑तिर॒ग्निम् । प्र॒जाप॑ति॒रिति॑ प्र॒जा - प॒तिः॒ । अ॒ग्निम॑चिकीषत । अ॒चि॒की॒ष॒त॒ तम् । तम् पृ॑थि॒वि । पृ॒थि॒व्य॑ब्रवीत् । अ॒ब्र॒वी॒न् न । न मयि॑ । मय्य॒ग्निम् । अ॒ग्निम् चे᳚ष्यसे । चे॒ष्य॒सेऽति॑ । अति॑ मा । मा॒ ध॒क्ष्य॒ति॒ । ध॒क्ष्य॒ति॒ सा । सा त्वा᳚ । त्वा॒ऽति॒द॒ह्यमा॑ना । अ॒ति॒द॒ह्यमा॑ना॒ वि । अ॒ति॒द॒ह्यमा॒नेत्य॑ति - द॒ह्यमा॑ना । वि ध॑विष्ये । ध॒वि॒ष्ये॒ सः \newline

\textbf{Jatai Paata} \newline

1. ची॒य॒ते॒ प॒शु॒मान् प॑शु॒माꣳ श्ची॑यते चीयते पशु॒मान् । \newline
2. प॒शु॒मा ने॒वैव प॑शु॒मान् प॑शु॒मा ने॒व । \newline
3. प॒शु॒मानिति॑ पशु - मान् । \newline
4. ए॒व भ॑वति भव त्ये॒वैव भ॑वति । \newline
5. भ॒व॒ति॒ कस्मै॒ कस्मै॑ भवति भवति॒ कस्मै᳚ । \newline
6. कस्मै॒ कम् कम् कस्मै॒ कस्मै॒ कम् । \newline
7. क म॒ग्नि र॒ग्निः कम् क म॒ग्निः । \newline
8. अ॒ग्नि श्ची॑यते चीयते॒ ऽग्नि र॒ग्नि श्ची॑यते । \newline
9. ची॒य॒त॒ इतीति॑ चीयते चीयत॒ इति॑ । \newline
10. इत्या॑हु राहु॒रिती त्या॑हुः । \newline
11. आ॒हुः॒ स॒प्त स॒प्ताहु॑ राहुः स॒प्त । \newline
12. स॒प्त मा॑ मा स॒प्त स॒प्त मा᳚ । \newline
13. मा॒ पुरु॑षाः॒ पुरु॑षा मा मा॒ पुरु॑षाः । \newline
14. पुरु॑षा॒ उपोप॒ पुरु॑षाः॒ पुरु॑षा॒ उप॑ । \newline
15. उप॑ जीवान् जीवा॒नुपोप॑ जीवान् । \newline
16. जी॒वा॒ नितीति॑ जीवान् जीवा॒निति॑ । \newline
17. इति॒ वै वा इतीति॒ वै । \newline
18. वा अ॒ग्नि र॒ग्निर् वै वा अ॒ग्निः । \newline
19. अ॒ग्नि श्ची॑यते चीयते॒ ऽग्नि र॒ग्नि श्ची॑यते । \newline
20. ची॒य॒ते॒ त्रय॒ स्त्रय॑ श्चीयते चीयते॒ त्रयः॑ । \newline
21. त्रयः॒ प्राञ्चः॒ प्राञ्च॒ स्त्रय॒ स्त्रयः॒ प्राञ्चः॑ । \newline
22. प्राञ्च॒ स्त्रय॒ स्त्रयः॒ प्राञ्चः॒ प्राञ्च॒ स्त्रयः॑ । \newline
23. त्रयः॑ प्र॒त्यञ्चः॑ प्र॒त्यञ्च॒ स्त्रय॒ स्त्रयः॑ प्र॒त्यञ्चः॑ । \newline
24. प्र॒त्यञ्च॑ आ॒त्मा ऽऽत्मा प्र॒त्यञ्चः॑ प्र॒त्यञ्च॑ आ॒त्मा । \newline
25. आ॒त्मा स॑प्त॒मः स॑प्त॒म आ॒त्मा ऽऽत्मा स॑प्त॒मः । \newline
26. स॒प्त॒म ए॒ताव॑न्त ए॒ताव॑न्तः सप्त॒मः स॑प्त॒म ए॒ताव॑न्तः । \newline
27. ए॒ताव॑न्त ए॒वै वैताव॑न्त ए॒ताव॑न्त ए॒व । \newline
28. ए॒वैन॑ मेन मे॒वै वैन᳚म् । \newline
29. ए॒न॒ म॒मुष्मि॑न् न॒मुष्मि॑न् नेन मेन म॒मुष्मिन्न्॑ । \newline
30. अ॒मुष्मि॑न् ॅलो॒के लो॒के॑ ऽमुष्मि॑न् न॒मुष्मि॑न् ॅलो॒के । \newline
31. लो॒क उपोप॑ लो॒के लो॒क उप॑ । \newline
32. उप॑ जीवन्ति जीव॒न् त्युपोप॑ जीवन्ति । \newline
33. जी॒व॒न्ति॒ प्र॒जाप॑तिः प्र॒जाप॑तिर् जीवन्ति जीवन्ति प्र॒जाप॑तिः । \newline
34. प्र॒जाप॑ति र॒ग्नि म॒ग्निम् प्र॒जाप॑तिः प्र॒जाप॑ति र॒ग्निम् । \newline
35. प्र॒जाप॑ति॒रिति॑ प्र॒जा - प॒तिः॒ । \newline
36. अ॒ग्नि म॑चिकीषता चिकीषता॒ग्नि म॒ग्नि म॑चिकीषत । \newline
37. अ॒चि॒की॒ष॒त॒ तम् त म॑चिकीषता चिकीषत॒ तम् । \newline
38. तम् पृ॑थि॒वी पृ॑थि॒वी तम् तम् पृ॑थि॒वी । \newline
39. पृ॒थि॒व्य॑ ब्रवी दब्रवीत् पृथि॒वी पृ॑थि॒व्य॑ ब्रवीत् । \newline
40. अ॒ब्र॒वी॒न् न नाब्र॑वी दब्रवी॒न् न । \newline
41. न मयि॒ मयि॒ न न मयि॑ । \newline
42. मय्य॒ग्नि म॒ग्निम् मयि॒ मय्य॒ग्निम् । \newline
43. अ॒ग्निम् चे᳚ष्यसे चेष्यसे॒ ऽग्नि म॒ग्निम् चे᳚ष्यसे । \newline
44. चे॒ष्य॒से ऽत्यति॑ चेष्यसे चेष्य॒से ऽति॑ । \newline
45. अति॑ मा॒ मा ऽत्यति॑ मा । \newline
46. मा॒ ध॒क्ष्य॒ति॒ ध॒क्ष्य॒ति॒ मा॒ मा॒ ध॒क्ष्य॒ति॒ । \newline
47. ध॒क्ष्य॒ति॒ सा सा ध॑क्ष्यति धक्ष्यति॒ सा । \newline
48. सा त्वा᳚ त्वा॒ सा सा त्वा᳚ । \newline
49. त्वा॒ ऽति॒द॒ह्यमा॑ना ऽतिद॒ह्यमा॑ना त्वा त्वा ऽतिद॒ह्यमा॑ना । \newline
50. अ॒ति॒द॒ह्यमा॑ना॒ वि व्य॑तिद॒ह्यमा॑ना ऽतिद॒ह्यमा॑ना॒ वि । \newline
51. अ॒ति॒द॒ह्यमा॒नेत्य॑ति - द॒ह्यमा॑ना । \newline
52. वि ध॑विष्ये धविष्ये॒ वि वि ध॑विष्ये । \newline
53. ध॒वि॒ष्ये॒ स स ध॑विष्ये धविष्ये॒ सः । \newline

\textbf{Ghana Paata } \newline

1. ची॒य॒ते॒ प॒शु॒मान् प॑शु॒माꣳ श्ची॑यते चीयते पशु॒मा ने॒वैव प॑शु॒माꣳ श्ची॑यते चीयते पशु॒मा ने॒व । \newline
2. प॒शु॒मा ने॒वैव प॑शु॒मान् प॑शु॒मा ने॒व भ॑वति भव त्ये॒व प॑शु॒मान् प॑शु॒मा ने॒व भ॑वति । \newline
3. प॒शु॒मानिति॑ पशु - मान् । \newline
4. ए॒व भ॑वति भव त्ये॒वैव भ॑वति॒ कस्मै॒ कस्मै॑ भव त्ये॒वैव भ॑वति॒ कस्मै᳚ । \newline
5. भ॒व॒ति॒ कस्मै॒ कस्मै॑ भवति भवति॒ कस्मै॒ कम् कम् कस्मै॑ भवति भवति॒ कस्मै॒ कम् । \newline
6. कस्मै॒ कम् कम् कस्मै॒ कस्मै॒ क म॒ग्नि र॒ग्निः कम् कस्मै॒ कस्मै॒ क म॒ग्निः । \newline
7. क म॒ग्नि र॒ग्निः कम् क म॒ग्नि श्ची॑यते चीयते॒ ऽग्निः कम् क म॒ग्नि श्ची॑यते । \newline
8. अ॒ग्नि श्ची॑यते चीयते॒ ऽग्नि र॒ग्नि श्ची॑यत॒ इतीति॑ चीयते॒ ऽग्नि र॒ग्नि श्ची॑यत॒ इति॑ । \newline
9. ची॒य॒त॒ इतीति॑ चीयते चीयत॒ इत्या॑हु राहु॒रिति॑ चीयते चीयत॒ इत्या॑हुः । \newline
10. इत्या॑हु राहु॒रि तीत्या॑हुः स॒प्त स॒प्ता हु॒रि तीत्या॑हुः स॒प्त । \newline
11. आ॒हुः॒ स॒प्त स॒प्ताहु॑ राहुः स॒प्त मा॑ मा स॒प्ताहु॑ राहुः स॒प्त मा᳚ । \newline
12. स॒प्त मा॑ मा स॒प्त स॒प्त मा॒ पुरु॑षाः॒ पुरु॑षा मा स॒प्त स॒प्त मा॒ पुरु॑षाः । \newline
13. मा॒ पुरु॑षाः॒ पुरु॑षा मा मा॒ पुरु॑षा॒ उपोप॒ पुरु॑षा मा मा॒ पुरु॑षा॒ उप॑ । \newline
14. पुरु॑षा॒ उपोप॒ पुरु॑षाः॒ पुरु॑षा॒ उप॑ जीवान् जीवा॒ नुप॒ पुरु॑षाः॒ पुरु॑षा॒ उप॑ जीवान् । \newline
15. उप॑ जीवान् जीवा॒ नुपोप॑ जीवा॒ नितीति॑ जीवा॒ नुपोप॑ जीवा॒ निति॑ । \newline
16. जी॒वा॒ नितीति॑ जीवान् जीवा॒ निति॒ वै वा इति॑ जीवान् जीवा॒ निति॒ वै । \newline
17. इति॒ वै वा इतीति॒ वा अ॒ग्नि र॒ग्निर् वा इतीति॒ वा अ॒ग्निः । \newline
18. वा अ॒ग्नि र॒ग्निर् वै वा अ॒ग्नि श्ची॑यते चीयते॒ ऽग्निर् वै वा अ॒ग्नि श्ची॑यते । \newline
19. आ॒ग्नि श्ची॑यते चीयते॒ ऽग्नि र॒ग्नि श्ची॑यते॒ त्रय॒ स्त्रय॑ श्चीयते॒ ऽग्नि र॒ग्नि श्ची॑यते॒ त्रयः॑ । \newline
20. ची॒य॒ते॒ त्रय॒ स्त्रय॑ श्चीयते चीयते॒ त्रयः॒ प्राञ्चः॒ प्राञ्च॒ स्त्रय॑ श्चीयते चीयते॒ त्रयः॒ प्राञ्चः॑ । \newline
21. त्रयः॒ प्राञ्चः॒ प्राञ्च॒ स्त्रय॒ स्त्रयः॒ प्राञ्च॒ स्त्रय॒ स्त्रयः॒ प्राञ्च॒ स्त्रय॒ स्त्रयः॒ प्राञ्च॒ स्त्रयः॑ । \newline
22. प्राञ्च॒ स्त्रय॒ स्त्रयः॒ प्राञ्चः॒ प्राञ्च॒ स्त्रयः॑ प्र॒त्यञ्चः॑ प्र॒त्यञ्च॒ स्त्रयः॒ प्राञ्चः॒ प्राञ्च॒ स्त्रयः॑ प्र॒त्यञ्चः॑ । \newline
23. त्रयः॑ प्र॒त्यञ्चः॑ प्र॒त्यञ्च॒ स्त्रय॒ स्त्रयः॑ प्र॒त्यञ्च॑ आ॒त्मा ऽऽत्मा प्र॒त्यञ्च॒ स्त्रय॒ स्त्रयः॑ प्र॒त्यञ्च॑ आ॒त्मा । \newline
24. प्र॒त्यञ्च॑ आ॒त्मा ऽऽत्मा प्र॒त्यञ्चः॑ प्र॒त्यञ्च॑ आ॒त्मा स॑प्त॒मः स॑प्त॒म आ॒त्मा प्र॒त्यञ्चः॑ प्र॒त्यञ्च॑ आ॒त्मा स॑प्त॒मः । \newline
25. आ॒त्मा स॑प्त॒मः स॑प्त॒म आ॒त्मा ऽऽत्मा स॑प्त॒म ए॒ताव॑न्त ए॒ताव॑न्तः सप्त॒म आ॒त्मा ऽऽत्मा स॑प्त॒म ए॒ताव॑न्तः । \newline
26. स॒प्त॒म ए॒ताव॑न्त ए॒ताव॑न्तः सप्त॒मः स॑प्त॒म ए॒ताव॑न्त ए॒वै वैताव॑न्तः सप्त॒मः स॑प्त॒म ए॒ताव॑न्त ए॒व । \newline
27. ए॒ताव॑न्त ए॒वै वैताव॑न्त ए॒ताव॑न्त ए॒वैन॑ मेन मे॒वैताव॑न्त ए॒ताव॑न्त ए॒वैन᳚म् । \newline
28. ए॒वैन॑ मेन मे॒वै वैन॑ म॒मुष्मि॑न् न॒मुष्मि॑न् नेन मे॒वै वैन॑ म॒मुष्मिन्न्॑ । \newline
29. ए॒न॒ म॒मुष्मि॑न् न॒मुष्मि॑न् नेन मेन म॒मुष्मि॑न् ॅलो॒के लो॒के॑ ऽमुष्मि॑न् नेन मेन म॒मुष्मि॑न् ॅलो॒के । \newline
30. अ॒मुष्मि॑न् ॅलो॒के लो॒के॑ ऽमुष्मि॑न् न॒मुष्मि॑न् ॅलो॒क उपोप॑ लो॒के॑ ऽमुष्मि॑न् न॒मुष्मि॑न् ॅलो॒क उप॑ । \newline
31. लो॒क उपोप॑ लो॒के लो॒क उप॑ जीवन्ति जीव॒न् त्युप॑ लो॒के लो॒क उप॑ जीवन्ति । \newline
32. उप॑ जीवन्ति जीव॒न् त्युपोप॑ जीवन्ति प्र॒जाप॑तिः प्र॒जाप॑तिर् जीव॒न् त्युपोप॑ जीवन्ति प्र॒जाप॑तिः । \newline
33. जी॒व॒न्ति॒ प्र॒जाप॑तिः प्र॒जाप॑तिर् जीवन्ति जीवन्ति प्र॒जाप॑ति र॒ग्नि म॒ग्निम् प्र॒जाप॑तिर् जीवन्ति जीवन्ति प्र॒जाप॑ति र॒ग्निम् । \newline
34. प्र॒जाप॑ति र॒ग्नि म॒ग्निम् प्र॒जाप॑तिः प्र॒जाप॑ति र॒ग्नि म॑चिकीषता चिकीषता॒ग्निम् प्र॒जाप॑तिः प्र॒जाप॑ति र॒ग्नि म॑चिकीषत । \newline
35. प्र॒जाप॑ति॒रिति॑ प्र॒जा - प॒तिः॒ । \newline
36. अ॒ग्नि म॑चिकीषता चिकीषता॒ग्नि म॒ग्नि म॑चिकीषत॒ तम् त म॑चिकीषता॒ग्नि म॒ग्नि म॑चिकीषत॒ तम् । \newline
37. अ॒चि॒की॒ष॒त॒ तम् त म॑चिकीषता चिकीषत॒ तम् पृ॑थि॒वी पृ॑थि॒वी त म॑चिकीषता चिकीषत॒ तम् पृ॑थि॒वी । \newline
38. तम् पृ॑थि॒वी पृ॑थि॒वी तम् तम् पृ॑थि॒ व्य॑ब्रवी दब्रवीत् पृथि॒वी तम् तम् पृ॑थि॒ व्य॑ब्रवीत् । \newline
39. पृ॒थि॒व्य॑ब्रवी दब्रवीत् पृथि॒वी पृ॑थि॒ व्य॑ब्रवी॒न् न नाब्र॑वीत् पृथि॒वी पृ॑थि॒ व्य॑ब्रवी॒न् न । \newline
40. अ॒ब्र॒वी॒न् न नाब्र॑वी दब्रवी॒न् न मयि॒ मयि॒ नाब्र॑वी दब्रवी॒न् न मयि॑ । \newline
41. न मयि॒ मयि॒ न न मय्य॒ग्नि म॒ग्निम् मयि॒ न न मय्य॒ग्निम् । \newline
42. मय्य॒ग्नि म॒ग्निम् मयि॒ मय्य॒ग्निम् चे᳚ष्यसे चेष्यसे॒ ऽग्निम् मयि॒ मय्य॒ग्निम् चे᳚ष्यसे । \newline
43. अ॒ग्निम् चे᳚ष्यसे चेष्यसे॒ ऽग्नि म॒ग्निम् चे᳚ष्य॒से ऽत्यति॑ चेष्यसे॒ ऽग्नि म॒ग्निम् चे᳚ष्य॒से ऽति॑ । \newline
44. चे॒ष्य॒से ऽत्यति॑ चेष्यसे चेष्य॒से ऽति॑ मा॒ मा ऽति॑ चेष्यसे चेष्य॒से ऽति॑ मा । \newline
45. अति॑ मा॒ मा ऽत्यति॑ मा धक्ष्यति धक्ष्यति॒ मा ऽत्यति॑ मा धक्ष्यति । \newline
46. मा॒ ध॒क्ष्य॒ति॒ ध॒क्ष्य॒ति॒ मा॒ मा॒ ध॒क्ष्य॒ति॒ सा सा ध॑क्ष्यति मा मा धक्ष्यति॒ सा । \newline
47. ध॒क्ष्य॒ति॒ सा सा ध॑क्ष्यति धक्ष्यति॒ सा त्वा᳚ त्वा॒ सा ध॑क्ष्यति धक्ष्यति॒ सा त्वा᳚ । \newline
48. सा त्वा᳚ त्वा॒ सा सा त्वा॑ ऽतिद॒ह्यमा॑ना ऽतिद॒ह्यमा॑ना त्वा॒ सा सा त्वा॑ ऽतिद॒ह्यमा॑ना । \newline
49. त्वा॒ ऽति॒द॒ह्यमा॑ना ऽतिद॒ह्यमा॑ना त्वा त्वा ऽतिद॒ह्यमा॑ना॒ वि व्य॑तिद॒ह्यमा॑ना त्वा त्वा ऽतिद॒ह्यमा॑ना॒ वि । \newline
50. अ॒ति॒द॒ह्यमा॑ना॒ वि व्य॑तिद॒ह्यमा॑ना ऽतिद॒ह्यमा॑ना॒ वि ध॑विष्ये धविष्ये॒ व्य॑तिद॒ह्यमा॑ना ऽतिद॒ह्यमा॑ना॒ वि ध॑विष्ये । \newline
51. अ॒ति॒द॒ह्यमा॒नेत्य॑ति - द॒ह्यमा॑ना । \newline
52. वि ध॑विष्ये धविष्ये॒ वि वि ध॑विष्ये॒ स स ध॑विष्ये॒ वि वि ध॑विष्ये॒ सः । \newline
53. ध॒वि॒ष्ये॒ स स ध॑विष्ये धविष्ये॒ स पापी॑या॒न् पापी॑या॒न् थ्स ध॑विष्ये धविष्ये॒ स पापी॑यान् । \newline
\pagebreak
\markright{ TS 5.5.2.4  \hfill https://www.vedavms.in \hfill}

\section{ TS 5.5.2.4 }

\textbf{TS 5.5.2.4 } \newline
\textbf{Samhita Paata} \newline

स पापी॑यान् भविष्य॒सीति॒ सो᳚ऽब्रवी॒त् तथा॒ वा अ॒हं क॑रिष्यामि॒ यथा᳚ त्वा॒ नाति॑ध॒क्ष्यतीति॒ स इ॒माम॒भ्य॑मृशत् प्र॒जाप॑तिस्त्वा सादयतु॒ तया॑ दे॒वत॑याऽङ्गिर॒स्वद् ध्रु॒वा सी॒देती॒मामे॒वेष्ट॑कां कृ॒त्वोपा॑-ध॒त्ता-न॑तिदाहाय॒ यत् प्रत्य॒ग्निं चि॑न्वी॒त तद॒भि मृ॑शेत् प्र॒जाप॑तिस्त्वा सादयतु॒ तया॑ दे॒वत॑याऽङ्गिर॒स्वद् ध्रु॒वा सी॒दे - [  ] \newline

\textbf{Pada Paata} \newline

सः । पापी॑यान् । भ॒वि॒ष्य॒सि॒ । इति॑ । सः । अ॒ब्र॒वी॒त् । तथा᳚ । वै । अ॒हम् । क॒रि॒ष्या॒मि॒ । यथा᳚ । त्वा॒ । न । अ॒ति॒ध॒क्ष्यतीत्य॑ति-ध॒क्ष्यति॑ । इति॑ । सः । इ॒माम् । अ॒भीति॑ । अ॒मृ॒श॒त् । प्र॒जाप॑ति॒रिति॑ प्र॒जा-प॒तिः॒ । त्वा॒ । सा॒द॒य॒त॒ । तया᳚ । दे॒वत॑या । अ॒ङ्गि॒र॒स्वत् । ध्रु॒वा । सी॒द॒ । इति॑ । इ॒माम् । ए॒व । इष्ट॑काम् । कृ॒त्वा । उपेति॑ । अ॒ध॒त्त॒ । अन॑तिदाहा॒येत्यन॑ति - दा॒हा॒य॒ । यत् । प्रतीति॑ । अ॒ग्निम् । चि॒न्वी॒त । तत् । अ॒भीति॑ । मृ॒शे॒त् । प्र॒जाप॑ति॒रिति॑ प्र॒जा-प॒तिः॒ । त्वा॒ । सा॒द॒य॒तु॒ । तया᳚ । दे॒वत॑या । अ॒ङ्गि॒र॒स्वत् । ध्रु॒वा । सी॒द॒ ।  \newline


\textbf{Krama Paata} \newline

स पापी॑यान् । पापी॑यान् भविष्यसि । भ॒वि॒ष्य॒सीति॑ । इति॒ सः । सो᳚ऽब्रवीत् । अ॒ब्र॒वी॒त् तथा᳚ । तथा॒ वै । वा अ॒हम् । अ॒हम् क॑रिष्यामि । क॒रि॒ष्या॒मि॒ यथा᳚ । यथा᳚ त्वा । त्वा॒ न । नाति॑ध॒क्ष्यति॑ । अ॒ति॒ध॒क्ष्यतीति॑ । अ॒ति॒ध॒क्ष्यतीत्य॑ति - ध॒क्ष्यति॑ । इति॒ सः । स इ॒माम् । इ॒माम॒भि । अ॒भ्य॑मृशत् । अ॒मृ॒श॒त् प्र॒जाप॑तिः । प्र॒जाप॑तिस्त्वा । प्र॒जाप॑ति॒रिति॑ प्र॒जा - प॒तिः॒ । त्वा॒ सा॒द॒य॒तु॒ । सा॒द॒य॒तु॒ तया᳚ । तया॑ दे॒वत॑या । दे॒वत॑याऽङ्गिर॒स्वत् । 
अ॒ङ्गि॒र॒स्वद् ध्रु॒वा । ध्रु॒वा सी॑द । सी॒देति॑ । इती॒माम् । इ॒मामे॒व । ए॒वेष्ट॑काम् । इष्ट॑काम् कृ॒त्वा । कृ॒त्वोप॑ । उपा॑धत्त । अ॒ध॒त्तान॑तिदाहाय । अन॑तिदाहाय॒ यत् । अन॑तिदाहा॒येत्यन॑ति - दा॒हा॒य॒ । यत् प्रति॑ । प्रत्य॒ग्निम् । अ॒ग्निम् चि॑न्वी॒त । चि॒न्वी॒त तत् । तद॒भि । अ॒भि मृ॑शेत् । मृ॒शे॒त् प्र॒जाप॑तिः । प्र॒जाप॑तिस्त्वा । प्र॒जाप॑ति॒रिति॑ प्र॒जा - प॒तिः॒ । त्वा॒ सा॒द॒य॒तु॒ । सा॒द॒य॒तु॒ तया᳚ । तया॑ दे॒वत॑या । दे॒वत॑याऽङ्गिर॒स्वत् । अ॒ङ्गि॒र॒स्वद् ध्रु॒वा । ध्रु॒वा सी॑द । सी॒देति॑ \newline

\textbf{Jatai Paata} \newline

1. स पापी॑या॒न् पापी॑या॒न् थ्स स पापी॑यान् । \newline
2. पापी॑यान् भविष्यसि भविष्यसि॒ पापी॑या॒न् पापी॑यान् भविष्यसि । \newline
3. भ॒वि॒ष्य॒सीतीति॑ भविष्यसि भविष्य॒सीति॑ । \newline
4. इति॒ स स इतीति॒ सः । \newline
5. सो᳚ ऽब्रवी दब्रवी॒थ् स सो᳚ ऽब्रवीत् । \newline
6. अ॒ब्र॒वी॒त् तथा॒ तथा᳚ ऽब्रवी दब्रवी॒त् तथा᳚ । \newline
7. तथा॒ वै वै तथा॒ तथा॒ वै । \newline
8. वा अ॒ह म॒हं ॅवै वा अ॒हम् । \newline
9. अ॒हम् क॑रिष्यामि करिष्याम्य॒ह म॒हम् क॑रिष्यामि । \newline
10. क॒रि॒ष्या॒मि॒ यथा॒ यथा॑ करिष्यामि करिष्यामि॒ यथा᳚ । \newline
11. यथा᳚ त्वा त्वा॒ यथा॒ यथा᳚ त्वा । \newline
12. त्वा॒ न न त्वा᳚ त्वा॒ न । \newline
13. नाति॑ध॒क्ष्य त्य॑तिध॒क्ष्यति॒ न नाति॑ध॒क्ष्यति॑ । \newline
14. अ॒ति॒ध॒क्ष्य तीती त्य॑तिध॒क्ष्य त्य॑तिध॒क्ष्यतीति॑ । \newline
15. अ॒ति॒ध॒क्ष्यतीत्य॑ति - ध॒क्ष्यति॑ । \newline
16. इति॒ स स इतीति॒ सः । \newline
17. स इ॒मा मि॒माꣳ स स इ॒माम् । \newline
18. इ॒मा म॒भ्य॑भीमा मि॒मा म॒भि । \newline
19. अ॒भ्य॑मृश दमृश द॒भ्या᳚(1॒)भ्य॑मृशत् । \newline
20. अ॒मृ॒श॒त् प्र॒जाप॑तिः प्र॒जाप॑ति रमृश दमृशत् प्र॒जाप॑तिः । \newline
21. प्र॒जाप॑ति स्त्वा त्वा प्र॒जाप॑तिः प्र॒जाप॑ति स्त्वा । \newline
22. प्र॒जाप॑ति॒रिति॑ प्र॒जा - प॒तिः॒ । \newline
23. त्वा॒ सा॒द॒य॒तु॒ सा॒द॒य॒तु॒ त्वा॒ त्वा॒ सा॒द॒य॒तु॒ । \newline
24. सा॒द॒य॒तु॒ तया॒ तया॑ सादयतु सादयतु॒ तया᳚ । \newline
25. तया॑ दे॒वत॑या दे॒वत॑या॒ तया॒ तया॑ दे॒वत॑या । \newline
26. दे॒वत॑या ऽङ्गिर॒स्व द॑ङ्गिर॒स्वद् दे॒वत॑या दे॒वत॑या ऽङ्गिर॒स्वत् । \newline
27. अ॒ङ्गि॒र॒स्वद् ध्रु॒वा ध्रु॒वा ऽङ्गि॑र॒स्व द॑ङ्गिर॒स्वद् ध्रु॒वा । \newline
28. ध्रु॒वा सी॑द सीद ध्रु॒वा ध्रु॒वा सी॑द । \newline
29. सी॒दे तीति॑ सीद सी॒देति॑ । \newline
30. इती॒मा मि॒मा मितीती॒माम् । \newline
31. इ॒मा मे॒वैवेमा मि॒मा मे॒व । \newline
32. ए॒वेष्ट॑का॒ मिष्ट॑का मे॒वैवेष्ट॑काम् । \newline
33. इष्ट॑काम् कृ॒त्वा कृ॒त्वेष्ट॑का॒ मिष्ट॑काम् कृ॒त्वा । \newline
34. कृ॒त्वोपोप॑ कृ॒त्वा कृ॒त्वोप॑ । \newline
35. उपा॑धत्ता ध॒त्तो पोपा॑ धत्त । \newline
36. अ॒ध॒त्ता न॑तिदाहा॒या न॑तिदाहाया धत्ता ध॒त्ता न॑तिदाहाय । \newline
37. अन॑तिदाहाय॒ यद् यदन॑तिदाहा॒या न॑तिदाहाय॒ यत् । \newline
38. अन॑तिदाहा॒येत्यन॑ति - दा॒हा॒य॒ । \newline
39. यत् प्रति॒ प्रति॒ यद् यत् प्रति॑ । \newline
40. प्रत्य॒ग्नि म॒ग्निम् प्रति॒ प्रत्य॒ग्निम् । \newline
41. अ॒ग्निम् चि॑न्वी॒त चि॑न्वी॒ताग्नि म॒ग्निम् चि॑न्वी॒त । \newline
42. चि॒न्वी॒त तत् तच् चि॑न्वी॒त चि॑न्वी॒त तत् । \newline
43. तद॒भ्य॑भि तत् तद॒भि । \newline
44. अ॒भि मृ॑शेन् मृशे द॒भ्य॑भि मृ॑शेत् । \newline
45. मृ॒शे॒त् प्र॒जाप॑तिः प्र॒जाप॑तिर् मृशेन् मृशेत् प्र॒जाप॑तिः । \newline
46. प्र॒जाप॑ति स्त्वा त्वा प्र॒जाप॑तिः प्र॒जाप॑ति स्त्वा । \newline
47. प्र॒जाप॑ति॒रिति॑ प्र॒जा - प॒तिः॒ । \newline
48. त्वा॒ सा॒द॒य॒तु॒ सा॒द॒य॒तु॒ त्वा॒ त्वा॒ सा॒द॒य॒तु॒ । \newline
49. सा॒द॒य॒तु॒ तया॒ तया॑ सादयतु सादयतु॒ तया᳚ । \newline
50. तया॑ दे॒वत॑या दे॒वत॑या॒ तया॒ तया॑ दे॒वत॑या । \newline
51. दे॒वत॑या ऽङ्गिर॒स्व द॑ङ्गिर॒स्वद् दे॒वत॑या दे॒वत॑या ऽङ्गिर॒स्वत् । \newline
52. अ॒ङ्गि॒र॒स्वद् ध्रु॒वा ध्रु॒वा ऽङ्गि॑र॒स्व द॑ङ्गिर॒स्वद् ध्रु॒वा । \newline
53. ध्रु॒वा सी॑द सीद ध्रु॒वा ध्रु॒वा सी॑द । \newline
54. सी॒दे तीति॑ सीद सी॒देति॑ । \newline

\textbf{Ghana Paata } \newline

1. स पापी॑या॒न् पापी॑या॒न् थ्स स पापी॑यान् भविष्यसि भविष्यसि॒ पापी॑या॒न् थ्स स पापी॑यान् भविष्यसि । \newline
2. पापी॑यान् भविष्यसि भविष्यसि॒ पापी॑या॒न् पापी॑यान् भविष्य॒सीतीति॑ भविष्यसि॒ पापी॑या॒न् पापी॑यान् भविष्य॒सीति॑ । \newline
3. भ॒वि॒ष्य॒ सीतीति॑ भविष्यसि भविष्य॒सीति॒ स स इति॑ भविष्यसि भविष्य॒सीति॒ सः । \newline
4. इति॒ स स इतीति॒ सो᳚ ऽब्रवी दब्रवी॒थ् स इतीति॒ सो᳚ ऽब्रवीत् । \newline
5. सो᳚ ऽब्रवी दब्रवी॒थ् स सो᳚ ऽब्रवी॒त् तथा॒ तथा᳚ ऽब्रवी॒थ् स सो᳚ ऽब्रवी॒त् तथा᳚ । \newline
6. अ॒ब्र॒वी॒त् तथा॒ तथा᳚ ऽब्रवी दब्रवी॒त् तथा॒ वै वै तथा᳚ ऽब्रवी दब्रवी॒त् तथा॒ वै । \newline
7. तथा॒ वै वै तथा॒ तथा॒ वा अ॒ह म॒हं ॅवै तथा॒ तथा॒ वा अ॒हम् । \newline
8. वा अ॒ह म॒हं ॅवै वा अ॒हम् क॑रिष्यामि करिष्या म्य॒हं ॅवै वा अ॒हम् क॑रिष्यामि । \newline
9. अ॒हम् क॑रिष्यामि करिष्या म्य॒ह म॒हम् क॑रिष्यामि॒ यथा॒ यथा॑ करिष्या म्य॒ह म॒हम् क॑रिष्यामि॒ यथा᳚ । \newline
10. क॒रि॒ष्या॒मि॒ यथा॒ यथा॑ करिष्यामि करिष्यामि॒ यथा᳚ त्वा त्वा॒ यथा॑ करिष्यामि करिष्यामि॒ यथा᳚ त्वा । \newline
11. यथा᳚ त्वा त्वा॒ यथा॒ यथा᳚ त्वा॒ न न त्वा॒ यथा॒ यथा᳚ त्वा॒ न । \newline
12. त्वा॒ न न त्वा᳚ त्वा॒ नाति॑ध॒क्ष्य त्य॑तिध॒क्ष्यति॒ न त्वा᳚ त्वा॒ नाति॑ध॒क्ष्यति॑ । \newline
13. नाति॑ध॒क्ष्य त्य॑तिध॒क्ष्यति॒ न नाति॑ध॒क्ष्यती तीत्य॑तिध॒क्ष्यति॒ न नाति॑ध॒क्ष्यतीति॑ । \newline
14. अ॒ति॒ध॒क्ष्यतीती त्य॑तिध॒क्ष्य त्य॑तिध॒क्ष्यतीति॒ स स इत्य॑ तिध॒क्ष्य त्य॑तिध॒क्ष्यतीति॒ सः । \newline
15. अ॒ति॒ध॒क्ष्यतीत्य॑ति - ध॒क्ष्यति॑ । \newline
16. इति॒ स स इतीति॒ स इ॒मा मि॒माꣳ स इतीति॒ स इ॒माम् । \newline
17. स इ॒मा मि॒माꣳ स स इ॒मा म॒भ्य॑ भीमाꣳ स स इ॒मा म॒भि । \newline
18. इ॒मा म॒भ्य॑भीमा मि॒मा म॒भ्य॑ मृश दमृश द॒भीमा मि॒मा म॒भ्य॑मृशत् । \newline
19. अ॒भ्य॑ मृश दमृश द॒भ्या᳚(1॒)भ्य॑मृशत् प्र॒जाप॑तिः प्र॒जाप॑ति रमृश द॒भ्या᳚(1॒)भ्य॑मृशत् प्र॒जाप॑तिः । \newline
20. अ॒मृ॒श॒त् प्र॒जाप॑तिः प्र॒जाप॑ति रमृश दमृशत् प्र॒जाप॑ति स्त्वा त्वा प्र॒जाप॑ति रमृश दमृशत् प्र॒जाप॑ति स्त्वा । \newline
21. प्र॒जाप॑ति स्त्वा त्वा प्र॒जाप॑तिः प्र॒जाप॑ति स्त्वा सादयतु सादयतु त्वा प्र॒जाप॑तिः प्र॒जाप॑ति स्त्वा सादयतु । \newline
22. प्र॒जाप॑ति॒रिति॑ प्र॒जा - प॒तिः॒ । \newline
23. त्वा॒ सा॒द॒य॒तु॒ सा॒द॒य॒तु॒ त्वा॒ त्वा॒ सा॒द॒य॒तु॒ तया॒ तया॑ सादयतु त्वा त्वा सादयतु॒ तया᳚ । \newline
24. सा॒द॒य॒तु॒ तया॒ तया॑ सादयतु सादयतु॒ तया॑ दे॒वत॑या दे॒वत॑या॒ तया॑ सादयतु सादयतु॒ तया॑ दे॒वत॑या । \newline
25. तया॑ दे॒वत॑या दे॒वत॑या॒ तया॒ तया॑ दे॒वत॑या ऽङ्गिर॒स्व द॑ङ्गिर॒स्वद् दे॒वत॑या॒ तया॒ तया॑ दे॒वत॑या ऽङ्गिर॒स्वत् । \newline
26. दे॒वत॑या ऽङ्गिर॒स्व द॑ङ्गिर॒स्वद् दे॒वत॑या दे॒वत॑या ऽङ्गिर॒स्वद् ध्रु॒वा ध्रु॒वा ऽङ्गि॑र॒स्वद् दे॒वत॑या दे॒वत॑या ऽङ्गिर॒स्वद् ध्रु॒वा । \newline
27. अ॒ङ्गि॒र॒स्वद् ध्रु॒वा ध्रु॒वा ऽङ्गि॑र॒स्व द॑ङ्गिर॒स्वद् ध्रु॒वा सी॑द सीद ध्रु॒वा ऽङ्गि॑र॒स्व द॑ङ्गिर॒स्वद् ध्रु॒वा सी॑द । \newline
28. ध्रु॒वा सी॑द सीद ध्रु॒वा ध्रु॒वा सी॒दे तीति॑ सीद ध्रु॒वा ध्रु॒वा सी॒देति॑ । \newline
29. सी॒दे तीति॑ सीद सी॒दे ती॒मा मि॒मा मिति॑ सीद सी॒दे ती॒माम् । \newline
30. इती॒मा मि॒मा मितीती॒मा मे॒वैवेमा मितीती॒मा मे॒व । \newline
31. इ॒मा मे॒वैवेमा मि॒मा मे॒वेष्ट॑का॒ मिष्ट॑का मे॒वेमा मि॒मा मे॒वेष्ट॑काम् । \newline
32. ए॒वेष्ट॑का॒ मिष्ट॑का मे॒वैवेष्ट॑काम् कृ॒त्वा कृ॒त्वेष्ट॑का मे॒वैवेष्ट॑काम् कृ॒त्वा । \newline
33. इष्ट॑काम् कृ॒त्वा कृ॒त्वेष्ट॑का॒ मिष्ट॑काम् कृ॒त्वोपोप॑ कृ॒त्वेष्ट॑का॒ मिष्ट॑काम् कृ॒त्वोप॑ । \newline
34. कृ॒त्वो पोप॑ कृ॒त्वा कृ॒त्वो पा॑धत्ता ध॒त्तोप॑ कृ॒त्वा कृ॒त्वोपा॑ धत्त । \newline
35. उपा॑धत्ता ध॒त्तोपोपा॑ ध॒त्ता न॑तिदाहा॒या न॑तिदाहाया ध॒त्तोपोपा॑ ध॒त्ता न॑तिदाहाय । \newline
36. अ॒ध॒त्ता न॑तिदाहा॒या न॑तिदाहाया धत्ता ध॒त्ता न॑तिदाहाय॒ यद् यदन॑तिदाहाया धत्ताध॒त्ता न॑तिदाहाय॒ यत् । \newline
37. अन॑तिदाहाय॒ यद् यदन॑तिदाहा॒या न॑तिदाहाय॒ यत् प्रति॒ प्रति॒ यदन॑तिदाहा॒या न॑तिदाहाय॒ यत् प्रति॑ । \newline
38. अन॑तिदाहा॒येत्यन॑ति - दा॒हा॒य॒ । \newline
39. यत् प्रति॒ प्रति॒ यद् यत् प्रत्य॒ग्नि म॒ग्निम् प्रति॒ यद् यत् प्रत्य॒ग्निम् । \newline
40. प्रत्य॒ग्नि म॒ग्निम् प्रति॒ प्रत्य॒ग्निम् चि॑न्वी॒त चि॑न्वी॒ताग्निम् प्रति॒ प्रत्य॒ग्निम् चि॑न्वी॒त । \newline
41. अ॒ग्निम् चि॑न्वी॒त चि॑न्वी॒ताग्नि म॒ग्निम् चि॑न्वी॒त तत् तच् चि॑न्वी॒ताग्नि म॒ग्निम् चि॑न्वी॒त तत् । \newline
42. चि॒न्वी॒त तत् तच् चि॑न्वी॒त चि॑न्वी॒त तद॒भ्य॑भि तच् चि॑न्वी॒त चि॑न्वी॒त तद॒भि । \newline
43. तद॒भ्य॑भि तत् तद॒भि मृ॑शेन् मृशे द॒भि तत् तद॒भि मृ॑शेत् । \newline
44. अ॒भि मृ॑शेन् मृशे द॒भ्य॑भि मृ॑शेत् प्र॒जाप॑तिः प्र॒जाप॑तिर् मृशे द॒भ्य॑भि मृ॑शेत् प्र॒जाप॑तिः । \newline
45. मृ॒शे॒त् प्र॒जाप॑तिः प्र॒जाप॑तिर् मृशेन् मृशेत् प्र॒जाप॑ति स्त्वा त्वा प्र॒जाप॑तिर् मृशेन् मृशेत् प्र॒जाप॑ति स्त्वा । \newline
46. प्र॒जाप॑ति स्त्वा त्वा प्र॒जाप॑तिः प्र॒जाप॑ति स्त्वा सादयतु सादयतु त्वा प्र॒जाप॑तिः प्र॒जाप॑ति स्त्वा सादयतु । \newline
47. प्र॒जाप॑ति॒रिति॑ प्र॒जा - प॒तिः॒ । \newline
48. त्वा॒ सा॒द॒य॒तु॒ सा॒द॒य॒तु॒ त्वा॒ त्वा॒ सा॒द॒य॒तु॒ तया॒ तया॑ सादयतु त्वा त्वा सादयतु॒ तया᳚ । \newline
49. सा॒द॒य॒तु॒ तया॒ तया॑ सादयतु सादयतु॒ तया॑ दे॒वत॑या दे॒वत॑या॒ तया॑ सादयतु सादयतु॒ तया॑ दे॒वत॑या । \newline
50. तया॑ दे॒वत॑या दे॒वत॑या॒ तया॒ तया॑ दे॒वत॑या ऽङ्गिर॒स्व द॑ङ्गिर॒स्वद् दे॒वत॑या॒ तया॒ तया॑ दे॒वत॑या ऽङ्गिर॒स्वत् । \newline
51. दे॒वत॑या ऽङ्गिर॒स्व द॑ङ्गिर॒स्वद् दे॒वत॑या दे॒वत॑या ऽङ्गिर॒स्वद् ध्रु॒वा ध्रु॒वा ऽङ्गि॑र॒स्वद् दे॒वत॑या दे॒वत॑या ऽङ्गिर॒स्वद् ध्रु॒वा । \newline
52. अ॒ङ्गि॒र॒स्वद् ध्रु॒वा ध्रु॒वा ऽङ्गि॑र॒स्व द॑ङ्गिर॒स्वद् ध्रु॒वा सी॑द सीद ध्रु॒वा ऽङ्गि॑र॒स्व द॑ङ्गिर॒स्वद् ध्रु॒वा सी॑द । \newline
53. ध्रु॒वा सी॑द सीद ध्रु॒वा ध्रु॒वा सी॒दे तीति॑ सीद ध्रु॒वा ध्रु॒वा सी॒देति॑ । \newline
54. सी॒दे तीति॑ सीद सी॒देती॒मा मि॒मा मिति॑ सीद सी॒देती॒माम् । \newline
\pagebreak
\markright{ TS 5.5.2.5  \hfill https://www.vedavms.in \hfill}

\section{ TS 5.5.2.5 }

\textbf{TS 5.5.2.5 } \newline
\textbf{Samhita Paata} \newline

-ती॒मामे॒वेष्ट॑कां कृ॒त्वोप॑ ध॒त्तेऽन॑तिदाहाय प्र॒जाप॑तिरकामयत॒ प्रजा॑ये॒येति॒ स ए॒तमुख्य॑मपश्य॒त् तꣳ सं॑ॅवथ्स॒रम॑बिभ॒स्ततो॑ वै स प्राजा॑यत॒ तस्मा᳚थ् संॅवथ्स॒रं भा॒र्यः॑ प्रैव जा॑यते॒ तं ॅवस॑वोऽब्रुव॒न् प्र त्वम॑जनिष्ठा व॒यं प्रजा॑यामहा॒ इति॒ तं ॅवसु॑भ्यः॒ प्राय॑च्छ॒त् तं त्रीण्यहा᳚न्यबिभरु॒स्तेन॒ - [  ] \newline

\textbf{Pada Paata} \newline

इति॑ । इ॒माम् । ए॒व । इष्ट॑काम् । कृ॒त्वा । उपेति॑ । ध॒त्ते॒ । अन॑तिदाहा॒येत्यन॑ति - दा॒हा॒य॒ । प्र॒जाप॑ति॒रिति॑ प्र॒जा - प॒तिः॒ । अ॒का॒म॒य॒त॒ । प्रेति॑ । जा॒ये॒य॒ । इति॑ । सः । ए॒तम् । उख्य᳚म् । अ॒प॒श्य॒त् । तम् । सं॒ॅव॒थ्स॒रमिति॑ सं - व॒थ्स॒रम् । अ॒बि॒भः॒ । ततः॑ । वै । सः । प्रेति॑ । अ॒जा॒य॒त॒ । तस्मा᳚त् । सं॒ॅव॒थ्स॒रमिति॑ सं-व॒थ्स॒रम् । भा॒र्यः॑ । प्रेति॑ । ए॒व । जा॒य॒ते॒ । तम् । वस॑वः । अ॒ब्रु॒व॒न्न् । प्रेति॑ । त्वम् । अ॒ज॒नि॒ष्ठाः॒ । व॒यम् । प्रेति॑ । जा॒या॒म॒है॒ । इति॑ । तम् । वसु॑भ्य॒ इति॒ वसु॑ - भ्यः॒ । प्रेति॑ । अ॒य॒च्छ॒त् । तम् । त्रीणि॑ । अहा॑नि । अ॒बि॒भ॒रुः॒ । तेन॑ ।  \newline


\textbf{Krama Paata} \newline

इती॒माम् । इ॒मामे॒व । ए॒वेष्ट॑काम् । इष्ट॑काम् कृ॒त्वा । कृ॒त्वोप॑ । उप॑ धत्ते । ध॒त्तेऽन॑तिदाहाय । अन॑तिदाहाय प्र॒जाप॑तिः । अन॑तिदाहा॒येत्यन॑ति - दा॒हा॒य॒ । प्र॒जाप॑तिरकामयत । प्र॒जाप॑ति॒रिति॑ प्र॒जा - प॒तिः॒ । अ॒का॒म॒य॒त॒ प्र । प्र जा॑येय । जा॒ये॒येति॑ । इति॒ सः । स ए॒तम् । ए॒तमुख्य᳚म् । उख्य॑मपश्यत् । अ॒प॒श्य॒त् तम् । तꣳ स॑म्ॅवथ्स॒रम् । स॒म्ॅव॒थ्स॒रम॑बिभः । स॒म्ॅव॒थ्स॒रमिति॑ सम् - व॒थ्स॒र॒म् । अ॒बि॒भ॒स्ततः॑ । ततो॒ वै । वै सः । स प्र । प्राजा॑यत । अ॒जा॒य॒त॒ तस्मा᳚त् । तस्मा᳚थ् सम्ॅवथ्स॒रम् । स॒म्ॅव॒थ्स॒रम् भा॒र्यः॑ । स॒म्ॅव॒थ्स॒रमिति॑ सम् - व॒थ्स॒रम् । भा॒र्यः॑ प्र । प्रैव । ए॒व जा॑यते । जा॒य॒ते॒ तम् । तम् ॅवस॑वः । वस॑वोऽब्रुवन्न् । अ॒ब्रु॒व॒न् प्र । प्र त्वम् । त्वम॑जनिष्ठाः । अ॒ज॒नि॒ष्ठा॒ व॒यम् । व॒यम् प्र । प्र जा॑यामहै । जा॒या॒म॒हा॒ इति॑ । इति॒ तम् । तम् ॅवसु॑भ्यः । वसु॑भ्यः॒ प्र । वसु॑भ्य॒ इति॒ वसु॑ - भ्यः॒ । प्राय॑च्छत् । अ॒य॒च्छ॒त् तम् । तम् त्रीणि॑ । त्रीण्यहा॑नि । अहा᳚न्यबिभरुः । अ॒बि॒भ॒रु॒स्तेन॑ । तेन॒ त्रीणि॑ \newline

\textbf{Jatai Paata} \newline

1. इती॒मा मि॒मा मितीती॒माम् । \newline
2. इ॒मा मे॒वैवेमा मि॒मा मे॒व । \newline
3. ए॒वेष्ट॑का॒ मिष्ट॑का मे॒वैवेष्ट॑काम् । \newline
4. इष्ट॑काम् कृ॒त्वा कृ॒त्वेष्ट॑का॒ मिष्ट॑काम् कृ॒त्वा । \newline
5. कृ॒त्वोपोप॑ कृ॒त्वा कृ॒त्वोप॑ । \newline
6. उप॑ धत्ते धत्त॒ उपोप॑ धत्ते । \newline
7. ध॒त्ते ऽन॑तिदाहा॒या न॑तिदाहाय धत्ते ध॒त्ते ऽन॑तिदाहाय । \newline
8. अन॑तिदाहाय प्र॒जाप॑तिः प्र॒जाप॑ति॒ रन॑तिदाहा॒या न॑तिदाहाय प्र॒जाप॑तिः । \newline
9. अन॑तिदाहा॒येत्यन॑ति - दा॒हा॒य॒ । \newline
10. प्र॒जाप॑ति रकामयता कामयत प्र॒जाप॑तिः प्र॒जाप॑ति रकामयत । \newline
11. प्र॒जाप॑ति॒रिति॑ प्र॒जा - प॒तिः॒ । \newline
12. अ॒का॒म॒य॒त॒ प्र प्राका॑मयता कामयत॒ प्र । \newline
13. प्र जा॑येय जायेय॒ प्र प्र जा॑येय । \newline
14. जा॒ये॒ये तीति॑ जायेय जाये॒येति॑ । \newline
15. इति॒ स स इतीति॒ सः । \newline
16. स ए॒त मे॒तꣳ स स ए॒तम् । \newline
17. ए॒त मुख्य॒ मुख्य॑ मे॒त मे॒त मुख्य᳚म् । \newline
18. उख्य॑ मपश्य दपश्य॒ दुख्य॒ मुख्य॑ मपश्यत् । \newline
19. अ॒प॒श्य॒त् तम् त म॑पश्य दपश्य॒त् तम् । \newline
20. तꣳ सं॑ॅवथ्स॒रꣳ सं॑ॅवथ्स॒रम् तम् तꣳ सं॑ॅवथ्स॒रम् । \newline
21. सं॒ॅव॒थ्स॒र म॑बिभ रबिभः संॅवथ्स॒रꣳ सं॑ॅवथ्स॒र म॑बिभः । \newline
22. सं॒ॅव॒थ्स॒रमिति॑ सं - व॒थ्स॒रम् । \newline
23. अ॒बि॒भ॒ स्तत॒ स्ततो॑ ऽबिभ रबिभ॒ स्ततः॑ । \newline
24. ततो॒ वै वै तत॒ स्ततो॒ वै । \newline
25. वै स स वै वै सः । \newline
26. स प्र प्र स स प्र । \newline
27. प्राजा॑यता जायत॒ प्र प्राजा॑यत । \newline
28. अ॒जा॒य॒त॒ तस्मा॒त् तस्मा॑ दजायता जायत॒ तस्मा᳚त् । \newline
29. तस्मा᳚थ् संॅवथ्स॒रꣳ सं॑ॅवथ्स॒रम् तस्मा॒त् तस्मा᳚थ् संॅवथ्स॒रम् । \newline
30. सं॒ॅव॒थ्स॒रम् भा॒र्यो॑ भा॒र्यः॑ संॅवथ्स॒रꣳ सं॑ॅवथ्स॒रम् भा॒र्यः॑ । \newline
31. सं॒ॅव॒थ्स॒रमिति॑ सं - व॒थ्स॒रम् । \newline
32. भा॒र्यः॑ प्र प्र भा॒र्यो॑ भा॒र्यः॑ प्र । \newline
33. प्रैवैव प्र प्रैव । \newline
34. ए॒व जा॑यते जायत ए॒वैव जा॑यते । \newline
35. जा॒य॒ते॒ तम् तम् जा॑यते जायते॒ तम् । \newline
36. तं ॅवस॑वो॒ वस॑व॒ स्तम् तं ॅवस॑वः । \newline
37. वस॑वो ऽब्रुवन् नब्रुव॒न्॒. वस॑वो॒ वस॑वो ऽब्रुवन्न् । \newline
38. अ॒ब्रु॒व॒न् प्र प्राब्रु॑वन् नब्रुव॒न् प्र । \newline
39. प्र त्वम् त्वम् प्र प्र त्वम् । \newline
40. त्व म॑जनिष्ठा अजनिष्ठा॒ स्त्वम् त्व म॑जनिष्ठाः । \newline
41. अ॒ज॒नि॒ष्ठा॒ व॒यं ॅव॒य म॑जनिष्ठा अजनिष्ठा व॒यम् । \newline
42. व॒यम् प्र प्र व॒यं ॅव॒यम् प्र । \newline
43. प्र जा॑यामहै जायामहै॒ प्र प्र जा॑यामहै । \newline
44. जा॒या॒म॒हा॒ इतीति॑ जायामहै जायामहा॒ इति॑ । \newline
45. इति॒ तम् तमितीति॒ तम् । \newline
46. तं ॅवसु॑भ्यो॒ वसु॑भ्य॒ स्तम् तं ॅवसु॑भ्यः । \newline
47. वसु॑भ्यः॒ प्र प्र वसु॑भ्यो॒ वसु॑भ्यः॒ प्र । \newline
48. वसु॑भ्य॒ इति॒ वसु॑ - भ्यः॒ । \newline
49. प्राय॑च्छ दयच्छ॒त् प्र प्राय॑च्छत् । \newline
50. अ॒य॒च्छ॒त् तम् त म॑यच्छ दयच्छ॒त् तम् । \newline
51. तम् त्रीणि॒ त्रीणि॒ तम् तम् त्रीणि॑ । \newline
52. त्रीण्यहा॒ न्यहा॑नि॒ त्रीणि॒ त्रीण्यहा॑नि । \newline
53. अहा᳚ न्यबिभरु रबिभरु॒ रहा॒ न्यहा᳚ न्यबिभरुः । \newline
54. अ॒बि॒भ॒रु॒ स्तेन॒ तेना॑बिभरु रबिभरु॒ स्तेन॑ । \newline
55. तेन॒ त्रीणि॒ त्रीणि॒ तेन॒ तेन॒ त्रीणि॑ । \newline

\textbf{Ghana Paata } \newline

1. इती॒मा मि॒मा मितीती॒मा मे॒वैवेमा मितीती॒मा मे॒व । \newline
2. इ॒मा मे॒वैवेमा मि॒मा मे॒वेष्ट॑का॒ मिष्ट॑का मे॒वेमा मि॒मा मे॒वेष्ट॑काम् । \newline
3. ए॒वेष्ट॑का॒ मिष्ट॑का मे॒वैवेष्ट॑काम् कृ॒त्वा कृ॒त्वेष्ट॑का मे॒वैवेष्ट॑काम् कृ॒त्वा । \newline
4. इष्ट॑काम् कृ॒त्वा कृ॒त्वेष्ट॑का॒ मिष्ट॑काम् कृ॒त्वो पोप॑ कृ॒त्वेष्ट॑का॒ मिष्ट॑काम् कृ॒त्वोप॑ । \newline
5. कृ॒त्वो पोप॑ कृ॒त्वा कृ॒त्वोप॑ धत्ते धत्त॒ उप॑ कृ॒त्वा कृ॒त्वोप॑ धत्ते । \newline
6. उप॑ धत्ते धत्त॒ उपोप॑ ध॒त्ते ऽन॑तिदाहा॒या न॑तिदाहाय धत्त॒ उपोप॑ ध॒त्ते ऽन॑तिदाहाय । \newline
7. ध॒त्ते ऽन॑तिदाहा॒या न॑तिदाहाय धत्ते ध॒त्ते ऽन॑तिदाहाय प्र॒जाप॑तिः प्र॒जाप॑ति॒ रन॑तिदाहाय धत्ते ध॒त्ते ऽन॑तिदाहाय प्र॒जाप॑तिः । \newline
8. अन॑तिदाहाय प्र॒जाप॑तिः प्र॒जाप॑ति॒ रन॑तिदाहा॒या न॑तिदाहाय प्र॒जाप॑ति रकामयता कामयत प्र॒जाप॑ति॒ रन॑तिदाहा॒या न॑तिदाहाय प्र॒जाप॑ति रकामयत । \newline
9. अन॑तिदाहा॒येत्यन॑ति - दा॒हा॒य॒ । \newline
10. प्र॒जाप॑ति रकामयता कामयत प्र॒जाप॑तिः प्र॒जाप॑ति रकामयत॒ प्र प्राका॑मयत प्र॒जाप॑तिः प्र॒जाप॑ति रकामयत॒ प्र । \newline
11. प्र॒जाप॑ति॒रिति॑ प्र॒जा - प॒तिः॒ । \newline
12. अ॒का॒म॒य॒त॒ प्र प्राका॑मयता कामयत॒ प्र जा॑येय जायेय॒ प्राका॑मयता कामयत॒ प्र जा॑येय । \newline
13. प्र जा॑येय जायेय॒ प्र प्र जा॑ये॒ये तीति॑ जायेय॒ प्र प्र जा॑ये॒येति॑ । \newline
14. जा॒ये॒ये तीति॑ जायेय जाये॒येति॒ स स इति॑ जायेय जाये॒येति॒ सः । \newline
15. इति॒ स स इतीति॒ स ए॒त मे॒तꣳ स इतीति॒ स ए॒तम् । \newline
16. स ए॒त मे॒तꣳ स स ए॒त मुख्य॒ मुख्य॑ मे॒तꣳ स स ए॒त मुख्य᳚म् । \newline
17. ए॒त मुख्य॒ मुख्य॑ मे॒त मे॒त मुख्य॑ मपश्य दपश्य॒ दुख्य॑ मे॒त मे॒त मुख्य॑ मपश्यत् । \newline
18. उख्य॑ मपश्य दपश्य॒ दुख्य॒ मुख्य॑ मपश्य॒त् तम् त म॑पश्य॒ दुख्य॒ मुख्य॑ मपश्य॒त् तम् । \newline
19. अ॒प॒श्य॒त् तम् त म॑पश्य दपश्य॒त् तꣳ सं॑ॅवथ्स॒रꣳ सं॑ॅवथ्स॒रम् त म॑पश्य दपश्य॒त् तꣳ सं॑ॅवथ्स॒रम् । \newline
20. तꣳ सं॑ॅवथ्स॒रꣳ सं॑ॅवथ्स॒रम् तम् तꣳ सं॑ॅवथ्स॒र म॑बिभ रबिभः संॅवथ्स॒रम् तम् तꣳ सं॑ॅवथ्स॒र म॑बिभः । \newline
21. सं॒ॅव॒थ्स॒र म॑बिभ रबिभः संॅवथ्स॒रꣳ सं॑ॅवथ्स॒र म॑बिभ॒ स्तत॒ स्ततो॑ ऽबिभः संॅवथ्स॒रꣳ सं॑ॅवथ्स॒र म॑बिभ॒ स्ततः॑ । \newline
22. सं॒ॅव॒थ्स॒रमिति॑ सं - व॒थ्स॒रम् । \newline
23. अ॒बि॒भ॒ स्तत॒ स्ततो॑ ऽबिभ रबिभ॒ स्ततो॒ वै वै ततो॑ ऽबिभ रबिभ॒ स्ततो॒ वै । \newline
24. ततो॒ वै वै तत॒ स्ततो॒ वै स स वै तत॒ स्ततो॒ वै सः । \newline
25. वै स स वै वै स प्र प्र स वै वै स प्र । \newline
26. स प्र प्र स स प्राजा॑यता जायत॒ प्र स स प्राजा॑यत । \newline
27. प्राजा॑यता जायत॒ प्र प्राजा॑यत॒ तस्मा॒त् तस्मा॑ दजायत॒ प्र प्राजा॑यत॒ तस्मा᳚त् । \newline
28. अ॒जा॒य॒त॒ तस्मा॒त् तस्मा॑ दजायता जायत॒ तस्मा᳚थ् संॅवथ्स॒रꣳ सं॑ॅवथ्स॒रम् तस्मा॑ दजायता जायत॒ तस्मा᳚थ् संॅवथ्स॒रम् । \newline
29. तस्मा᳚थ् संॅवथ्स॒रꣳ सं॑ॅवथ्स॒रम् तस्मा॒त् तस्मा᳚थ् संॅवथ्स॒रम् भा॒र्यो॑ भा॒र्यः॑ संॅवथ्स॒रम् तस्मा॒त् तस्मा᳚थ् संॅवथ्स॒रम् भा॒र्यः॑ । \newline
30. सं॒ॅव॒थ्स॒रम् भा॒र्यो॑ भा॒र्यः॑ संॅवथ्स॒रꣳ सं॑ॅवथ्स॒रम् भा॒र्यः॑ प्र प्र भा॒र्यः॑ संॅवथ्स॒रꣳ सं॑ॅवथ्स॒रम् भा॒र्यः॑ प्र । \newline
31. सं॒ॅव॒थ्स॒रमिति॑ सं - व॒थ्स॒रम् । \newline
32. भा॒र्यः॑ प्र प्र भा॒र्यो॑ भा॒र्यः॑ प्रैवैव प्र भा॒र्यो॑ भा॒र्यः॑ प्रैव । \newline
33. प्रैवैव प्र प्रैव जा॑यते जायत ए॒व प्र प्रैव जा॑यते । \newline
34. ए॒व जा॑यते जायत ए॒वैव जा॑यते॒ तम् तम् जा॑यत ए॒वैव जा॑यते॒ तम् । \newline
35. जा॒य॒ते॒ तम् तम् जा॑यते जायते॒ तं ॅवस॑वो॒ वस॑व॒ स्तम् जा॑यते जायते॒ तं ॅवस॑वः । \newline
36. तं ॅवस॑वो॒ वस॑व॒ स्तम् तं ॅवस॑वो ऽब्रुवन् नब्रुव॒न्॒. वस॑व॒ स्तम् तं ॅवस॑वो ऽब्रुवन्न् । \newline
37. वस॑वो ऽब्रुवन् नब्रुव॒न्॒. वस॑वो॒ वस॑वो ऽब्रुव॒न् प्र प्राब्रु॑व॒न्॒. वस॑वो॒ वस॑वो ऽब्रुव॒न् प्र । \newline
38. अ॒ब्रु॒व॒न् प्र प्राब्रु॑वन् नब्रुव॒न् प्र त्वम् त्वम् प्राब्रु॑वन् नब्रुव॒न् प्र त्वम् । \newline
39. प्र त्वम् त्वम् प्र प्र त्व म॑जनिष्ठा अजनिष्ठा॒ स्त्वम् प्र प्र त्व म॑जनिष्ठाः । \newline
40. त्व म॑जनिष्ठा अजनिष्ठा॒ स्त्वम् त्व म॑जनिष्ठा व॒यं ॅव॒य म॑जनिष्ठा॒ स्त्वम् त्व म॑जनिष्ठा व॒यम् । \newline
41. अ॒ज॒नि॒ष्ठा॒ व॒यं ॅव॒य म॑जनिष्ठा अजनिष्ठा व॒यम् प्र प्र व॒य म॑जनिष्ठा अजनिष्ठा व॒यम् प्र । \newline
42. व॒यम् प्र प्र व॒यं ॅव॒यम् प्र जा॑यामहै जायामहै॒ प्र व॒यं ॅव॒यम् प्र जा॑यामहै । \newline
43. प्र जा॑यामहै जायामहै॒ प्र प्र जा॑यामहा॒ इतीति॑ जायामहै॒ प्र प्र जा॑यामहा॒ इति॑ । \newline
44. जा॒या॒म॒हा॒ इतीति॑ जायामहै जायामहा॒ इति॒ तम् त मिति॑ जायामहै जायामहा॒ इति॒ तम् । \newline
45. इति॒ तम् त मितीति॒ तं ॅवसु॑भ्यो॒ वसु॑भ्य॒ स्त मितीति॒ तं ॅवसु॑भ्यः । \newline
46. तं ॅवसु॑भ्यो॒ वसु॑भ्य॒ स्तम् तं ॅवसु॑भ्यः॒ प्र प्र वसु॑भ्य॒ स्तम् तं ॅवसु॑भ्यः॒ प्र । \newline
47. वसु॑भ्यः॒ प्र प्र वसु॑भ्यो॒ वसु॑भ्यः॒ प्राय॑च्छ दयच्छ॒त् प्र वसु॑भ्यो॒ वसु॑भ्यः॒ प्राय॑च्छत् । \newline
48. वसु॑भ्य॒ इति॒ वसु॑ - भ्यः॒ । \newline
49. प्राय॑च्छ दयच्छ॒त् प्र प्राय॑च्छ॒त् तम् त म॑यच्छ॒त् प्र प्राय॑च्छ॒त् तम् । \newline
50. अ॒य॒च्छ॒त् तम् त म॑यच्छ दयच्छ॒त् तम् त्रीणि॒ त्रीणि॒ त म॑यच्छ दयच्छ॒त् तम् त्रीणि॑ । \newline
51. तम् त्रीणि॒ त्रीणि॒ तम् तम् त्रीण्यहा॒ न्यहा॑नि॒ त्रीणि॒ तम् तम् त्रीण्यहा॑नि । \newline
52. त्रीण्यहा॒ न्यहा॑नि॒ त्रीणि॒ त्रीण्यहा᳚ न्यबिभरु रबिभरु॒ रहा॑नि॒ त्रीणि॒ त्रीण्यहा᳚ न्यबिभरुः । \newline
53. अहा᳚न्य बिभरु रबिभरु॒ रहा॒ न्यहा᳚ न्यबिभरु॒ स्तेन॒ तेना॑बिभरु॒ रहा॒ न्यहा᳚ न्यबिभरु॒ स्तेन॑ । \newline
54. अ॒बि॒भ॒रु॒ स्तेन॒ तेना॑ बिभरु रबिभरु॒ स्तेन॒ त्रीणि॒ त्रीणि॒ तेना॑ बिभरु रबिभरु॒ स्तेन॒ त्रीणि॑ । \newline
55. तेन॒ त्रीणि॒ त्रीणि॒ तेन॒ तेन॒ त्रीणि॑ च च॒ त्रीणि॒ तेन॒ तेन॒ त्रीणि॑ च । \newline
\pagebreak
\markright{ TS 5.5.2.6  \hfill https://www.vedavms.in \hfill}

\section{ TS 5.5.2.6 }

\textbf{TS 5.5.2.6 } \newline
\textbf{Samhita Paata} \newline

त्रीणि॑ च श॒तान्यसृ॑जन्त॒ त्रय॑स्त्रिꣳशतं च॒ तस्मा᳚त् त्र्य॒हं भा॒र्यः॑ प्रैव जा॑यते॒ तान् रु॒द्रा अ॑ब्रुव॒न् प्र यू॒यम॑जनिढ्वं ॅव॒यं प्रजा॑यामहा॒ इति॒ तꣳ रु॒द्रेभ्यः॒ प्राय॑च्छ॒न् तꣳ षडहा᳚न्यबिभरु॒स्तेन॒ त्रीणि॑ च श॒तान्यसृ॑जन्त॒ त्रय॑स्त्रिꣳशतं च॒ तस्मा᳚थ् षड॒हं भा॒र्यः॑ प्रैव जा॑यते॒ ताना॑दि॒त्या अ॑ब्रुव॒न् प्र यू॒यम॑जनिढ्वं ॅव॒यं - [  ] \newline

\textbf{Pada Paata} \newline

त्रीणि॑ । च॒ । श॒ताति॑ । असृ॑जन्त । त्रय॑स्त्रिꣳशत॒मिति॒ त्रयः॑- त्रिꣳ॒॒श॒त॒म् । च॒ । तस्मा᳚त् । त्र्य॒हमिति॑ त्रि - अ॒हम् । भा॒र्यः॑ । प्रेति॑ । ए॒व । जा॒य॒ते॒ । तान् । रु॒द्राः । अ॒ब्रु॒व॒न्न् । प्रेति॑ । यू॒यम् । अ॒ज॒नि॒ढ्व॒म् । व॒यम् । प्रेति॑ । जा॒या॒म॒है॒ । इति॑ । तम् । रु॒द्रेभ्यः॑ । प्रेति॑ । अ॒य॒च्छ॒न्न् । तम् । षट् । अहा॑नि । अ॒बि॒भ॒रुः॒ । तेन॑ । त्रीणि॑ । च॒ । श॒तानि॑ । असृ॑जन्त । त्रय॑स्त्रिꣳशत॒मिति॒ त्रयः॑ - त्रिꣳ॒॒श॒त॒म् । च॒ । तस्मा᳚त् । ष॒ड॒हमिति॑ षट् - अ॒हम् । भा॒र्यः॑ । प्रेति॑ । ए॒व । जा॒य॒ते॒ । तान् । आ॒दि॒त्याः । अ॒ब्रु॒व॒न्न् । प्रेति॑ । यू॒यम् । अ॒ज॒नि॒ढ्व॒म् । व॒यम् ।  \newline


\textbf{Krama Paata} \newline

त्रीणि॑ च । च॒ श॒तानि॑ । श॒तान्यसृ॑जन्त । असृ॑जन्त॒ त्रय॑स्त्रिꣳशतम् । त्रय॑स्त्रिꣳशतम् च । त्रय॑स्त्रिꣳशत॒मिति॒ त्रयः॑ - त्रिꣳ॒॒श॒त॒म् । च॒ तस्मा᳚त् । तस्मा᳚त् त्र्य॒हम् । त्र्य॒हम् भा॒र्यः॑ । त्र्य॒हमिति॑ त्रि - अ॒हम् । भा॒र्यः॑ प्र । प्रैव । ए॒व जा॑यते । जा॒य॒ते॒ तान् । तान् रु॒द्राः । रु॒द्रा अ॑ब्रुवन्न् । अ॒ब्रु॒व॒न् प्र । प्र यू॒यम् । यू॒यम॑जनिढ्वम् । अ॒ज॒नि॒ढ्व॒म् ॅव॒यम् । व॒यम् प्र । प्र जा॑यामहै । जा॒या॒म॒हा॒ इति॑ । इति॒ तम् । तꣳ रु॒द्रेभ्यः॑ । रु॒द्रेभ्यः॒ प्र । प्राय॑च्छन्न् । अ॒य॒च्छ॒न् तम् । तꣳ षट् । षडहा॑नि । अहा᳚न्यबिभरुः । अ॒बि॒भ॒रु॒स्तेन॑ । तेन॒ त्रीणि॑ । त्रीणि॑ च । च॒ श॒तानि॑ । श॒तान्यसृ॑जन्त । असृ॑जन्त॒ त्रय॑स्त्रिꣳशतम् । त्रय॑स्त्रिꣳशतम् च । त्रय॑स्त्रिꣳशत॒मिति॒ त्रयः॑ - त्रिꣳ॒॒श॒त॒म् । च॒ तस्मा᳚त् । तस्मा᳚थ् षड॒हम् । ष॒ड॒हम् भा॒र्यः॑ । ष॒ड॒हमिति॑ षट् - अ॒हम् । भा॒र्यः॑ प्र । प्रैव । ए॒व जा॑यते । जा॒य॒ते॒ तान् । ताना॑दि॒त्याः । आ॒दि॒त्या अ॑ब्रुवन्न् । अ॒ब्रु॒व॒न् प्र । प्र यू॒यम् । यू॒यम॑जनिढ्वम् । अ॒ज॒नि॒ढ्व॒म् ॅव॒यम् ( ) । 
व॒यम् प्र \newline

\textbf{Jatai Paata} \newline

1. त्रीणि॑ च च॒ त्रीणि॒ त्रीणि॑ च । \newline
2. च॒ श॒तानि॑ श॒तानि॑ च च श॒तानि॑ । \newline
3. श॒ता न्यसृ॑ज॒न्ता सृ॑जन्त श॒तानि॑ श॒ता न्यसृ॑जन्त । \newline
4. असृ॑जन्त॒ त्रय॑स्त्रिꣳशत॒म् त्रय॑स्त्रिꣳशत॒ मसृ॑ज॒न्ता सृ॑जन्त॒ त्रय॑स्त्रिꣳशतम् । \newline
5. त्रय॑स्त्रिꣳशतम् च च॒ त्रय॑स्त्रिꣳशत॒म् त्रय॑स्त्रिꣳशतम् च । \newline
6. त्रय॑स्त्रिꣳशत॒मिति॒ त्रयः॑ - त्रिꣳ॒॒श॒त॒म् । \newline
7. च॒ तस्मा॒त् तस्मा᳚च् च च॒ तस्मा᳚त् । \newline
8. तस्मा᳚त् त्र्य॒हम् त्र्य॒हम् तस्मा॒त् तस्मा᳚त् त्र्य॒हम् । \newline
9. त्र्य॒हम् भा॒र्यो॑ भा॒र्य॑ स्त्र्य॒हम् त्र्य॒हम् भा॒र्यः॑ । \newline
10. त्र्य॒हमिति॑ त्रि - अ॒हम् । \newline
11. भा॒र्यः॑ प्र प्र भा॒र्यो॑ भा॒र्यः॑ प्र । \newline
12. प्रैवैव प्र प्रैव । \newline
13. ए॒व जा॑यते जायत ए॒वैव जा॑यते । \newline
14. जा॒य॒ते॒ ताꣳ स्तान् जा॑यते जायते॒ तान् । \newline
15. तान् रु॒द्रा रु॒द्रा स्ताꣳ स्तान् रु॒द्राः । \newline
16. रु॒द्रा अ॑ब्रुवन् नब्रुवन् रु॒द्रा रु॒द्रा अ॑ब्रुवन्न् । \newline
17. अ॒ब्रु॒व॒न् प्र प्राब्रु॑वन् नब्रुव॒न् प्र । \newline
18. प्र यू॒यं ॅयू॒यम् प्र प्र यू॒यम् । \newline
19. यू॒य म॑जनिढ्व मजनिढ्वं ॅयू॒यं ॅयू॒य म॑जनिढ्वम् । \newline
20. अ॒ज॒नि॒ढ्वं॒ ॅव॒यं ॅव॒य म॑जनिढ्व मजनिढ्वं ॅव॒यम् । \newline
21. व॒यम् प्र प्र व॒यं ॅव॒यम् प्र । \newline
22. प्र जा॑यामहै जायामहै॒ प्र प्र जा॑यामहै । \newline
23. जा॒या॒म॒हा॒ इतीति॑ जायामहै जायामहा॒ इति॑ । \newline
24. इति॒ तम् त मितीति॒ तम् । \newline
25. तꣳ रु॒द्रेभ्यो॑ रु॒द्रेभ्य॒ स्तम् तꣳ रु॒द्रेभ्यः॑ । \newline
26. रु॒द्रेभ्यः॒ प्र प्र रु॒द्रेभ्यो॑ रु॒द्रेभ्यः॒ प्र । \newline
27. प्राय॑च्छन् नयच्छ॒न् प्र प्राय॑च्छन्न् । \newline
28. अ॒य॒च्छ॒न् तम् त म॑यच्छन् नयच्छ॒न् तम् । \newline
29. तꣳ षट् थ्षट् तम् तꣳ षट् । \newline
30. षडहा॒ न्यहा॑नि॒ षट् थ्षडहा॑नि । \newline
31. अहा᳚ न्यबिभरु रबिभरु॒ रहा॒ न्यहा᳚ न्यबिभरुः । \newline
32. अ॒बि॒भ॒रु॒ स्तेन॒ तेना॑बिभरु रबिभरु॒ स्तेन॑ । \newline
33. तेन॒ त्रीणि॒ त्रीणि॒ तेन॒ तेन॒ त्रीणि॑ । \newline
34. त्रीणि॑ च च॒ त्रीणि॒ त्रीणि॑ च । \newline
35. च॒ श॒तानि॑ श॒तानि॑ च च श॒तानि॑ । \newline
36. श॒तान्य सृ॑ज॒न्ता सृ॑जन्त श॒तानि॑ श॒तान्य सृ॑जन्त । \newline
37. असृ॑जन्त॒ त्रय॑स्त्रिꣳशत॒म् त्रय॑स्त्रिꣳशत॒ मसृ॑ज॒न्ता सृ॑जन्त॒ त्रय॑स्त्रिꣳशतम् । \newline
38. त्रय॑स्त्रिꣳशतम् च च॒ त्रय॑स्त्रिꣳशत॒म् त्रय॑स्त्रिꣳशतम् च । \newline
39. त्रय॑स्त्रिꣳशत॒मिति॒ त्रयः॑ - त्रिꣳ॒॒श॒त॒म् । \newline
40. च॒ तस्मा॒त् तस्मा᳚च् च च॒ तस्मा᳚त् । \newline
41. तस्मा᳚ थ्षड॒हꣳ ष॑ड॒हम् तस्मा॒त् तस्मा᳚थ् षड॒हम् । \newline
42. ष॒ड॒हम् भा॒र्यो॑ भा॒र्य॑ ष्षड॒हꣳ ष॑ड॒हम् भा॒र्यः॑ । \newline
43. ष॒ड॒हमिति॑ षट् - अ॒हम् । \newline
44. भा॒र्यः॑ प्र प्र भा॒र्यो॑ भा॒र्यः॑ प्र । \newline
45. प्रैवैव प्र प्रैव । \newline
46. ए॒व जा॑यते जायत ए॒वैव जा॑यते । \newline
47. जा॒य॒ते॒ ताꣳ स्तान् जा॑यते जायते॒ तान् । \newline
48. ताना॑दि॒त्या आ॑दि॒त्या स्ताꣳ स्ताना॑दि॒त्याः । \newline
49. आ॒दि॒त्या अ॑ब्रुवन् नब्रुवन् नादि॒त्या आ॑दि॒त्या अ॑ब्रुवन्न् । \newline
50. अ॒ब्रु॒व॒न् प्र प्राब्रु॑वन् नब्रुव॒न् प्र । \newline
51. प्र यू॒यं ॅयू॒यम् प्र प्र यू॒यम् । \newline
52. यू॒य म॑जनिढ्व मजनिढ्वं ॅयू॒यं ॅयू॒य म॑जनिढ्वम् । \newline
53. अ॒ज॒नि॒ढ्वं॒ ॅव॒यं ॅव॒य म॑जनिढ्व मजनिढ्वं ॅव॒यम् । \newline
54. व॒यम् प्र प्र व॒यं ॅव॒यम् प्र । \newline

\textbf{Ghana Paata } \newline

1. त्रीणि॑ च च॒ त्रीणि॒ त्रीणि॑ च श॒तानि॑ श॒तानि॑ च॒ त्रीणि॒ त्रीणि॑ च श॒तानि॑ । \newline
2. च॒ श॒तानि॑ श॒तानि॑ च च श॒ता न्यसृ॑ज॒न्ता सृ॑जन्त श॒तानि॑ च च श॒ता न्यसृ॑जन्त । \newline
3. श॒ता न्यसृ॑ज॒न्ता सृ॑जन्त श॒तानि॑ श॒ता न्यसृ॑जन्त॒ त्रय॑स्त्रिꣳशत॒म् त्रय॑स्त्रिꣳशत॒ मसृ॑जन्त श॒तानि॑ श॒ता न्यसृ॑जन्त॒ त्रय॑स्त्रिꣳशतम् । \newline
4. असृ॑जन्त॒ त्रय॑स्त्रिꣳशत॒म् त्रय॑स्त्रिꣳशत॒ मसृ॑ज॒न्ता सृ॑जन्त॒ त्रय॑स्त्रिꣳशतम् च च॒ त्रय॑स्त्रिꣳशत॒ मसृ॑ज॒न्ता सृ॑जन्त॒ त्रय॑स्त्रिꣳशतम् च । \newline
5. त्रय॑स्त्रिꣳशतम् च च॒ त्रय॑स्त्रिꣳशत॒म् त्रय॑स्त्रिꣳशतम् च॒ तस्मा॒त् तस्मा᳚च् च॒ त्रय॑स्त्रिꣳशत॒म् त्रय॑स्त्रिꣳशतम् च॒ तस्मा᳚त् । \newline
6. त्रय॑स्त्रिꣳशत॒मिति॒ त्रयः॑ - त्रिꣳ॒॒श॒त॒म् । \newline
7. च॒ तस्मा॒त् तस्मा᳚च् च च॒ तस्मा᳚त् त्र्य॒हम् त्र्य॒हम् तस्मा᳚च् च च॒ तस्मा᳚त् त्र्य॒हम् । \newline
8. तस्मा᳚त् त्र्य॒हम् त्र्य॒हम् तस्मा॒त् तस्मा᳚त् त्र्य॒हम् भा॒र्यो॑ भा॒र्य॑ स्त्र्य॒हम् तस्मा॒त् तस्मा᳚त् त्र्य॒हम् भा॒र्यः॑ । \newline
9. त्र्य॒हम् भा॒र्यो॑ भा॒र्य॑ स्त्र्य॒हम् त्र्य॒हम् भा॒र्यः॑ प्र प्र भा॒र्य॑ स्त्र्य॒हम् त्र्य॒हम् भा॒र्यः॑ प्र । \newline
10. त्र्य॒हमिति॑ त्रि - अ॒हम् । \newline
11. भा॒र्यः॑ प्र प्र भा॒र्यो॑ भा॒र्यः॑ प्रैवैव प्र भा॒र्यो॑ भा॒र्यः॑ प्रैव । \newline
12. प्रैवैव प्र प्रैव जा॑यते जायत ए॒व प्र प्रैव जा॑यते । \newline
13. ए॒व जा॑यते जायत ए॒वैव जा॑यते॒ ताꣳ स्तान् जा॑यत ए॒वैव जा॑यते॒ तान् । \newline
14. जा॒य॒ते॒ ताꣳ स्तान् जा॑यते जायते॒ तान् रु॒द्रा रु॒द्रा स्तान् जा॑यते जायते॒ तान् रु॒द्राः । \newline
15. तान् रु॒द्रा रु॒द्रा स्ताꣳ स्तान् रु॒द्रा अ॑ब्रुवन् नब्रुवन् रु॒द्रा स्ताꣳ स्तान् रु॒द्रा अ॑ब्रुवन्न् । \newline
16. रु॒द्रा अ॑ब्रुवन् नब्रुवन् रु॒द्रा रु॒द्रा अ॑ब्रुव॒न् प्र प्राब्रु॑वन् रु॒द्रा रु॒द्रा अ॑ब्रुव॒न् प्र । \newline
17. अ॒ब्रु॒व॒न् प्र प्राब्रु॑वन् नब्रुव॒न् प्र यू॒यं ॅयू॒यम् प्राब्रु॑वन् नब्रुव॒न् प्र यू॒यम् । \newline
18. प्र यू॒यं ॅयू॒यम् प्र प्र यू॒य म॑जनिढ्व मजनिढ्वं ॅयू॒यम् प्र प्र यू॒य म॑जनिढ्वम् । \newline
19. यू॒य म॑जनिढ्व मजनिढ्वं ॅयू॒यं ॅयू॒य म॑जनिढ्वं ॅव॒यं ॅव॒य म॑जनिढ्वं ॅयू॒यं ॅयू॒य म॑जनिढ्वं ॅव॒यम् । \newline
20. अ॒ज॒नि॒ढ्वं॒ ॅव॒यं ॅव॒य म॑जनिढ्व मजनिढ्वं ॅव॒यम् प्र प्र व॒य म॑जनिढ्व मजनिढ्वं ॅव॒यम् प्र । \newline
21. व॒यम् प्र प्र व॒यं ॅव॒यम् प्र जा॑यामहै जायामहै॒ प्र व॒यं ॅव॒यम् प्र जा॑यामहै । \newline
22. प्र जा॑यामहै जायामहै॒ प्र प्र जा॑यामहा॒ इतीति॑ जायामहै॒ प्र प्र जा॑यामहा॒ इति॑ । \newline
23. जा॒या॒म॒हा॒ इतीति॑ जायामहै जायामहा॒ इति॒ तम् त मिति॑ जायामहै जायामहा॒ इति॒ तम् । \newline
24. इति॒ तम् त मितीति॒ तꣳ रु॒द्रेभ्यो॑ रु॒द्रेभ्य॒ स्त मितीति॒ तꣳ रु॒द्रेभ्यः॑ । \newline
25. तꣳ रु॒द्रेभ्यो॑ रु॒द्रेभ्य॒ स्तम् तꣳ रु॒द्रेभ्यः॒ प्र प्र रु॒द्रेभ्य॒ स्तम् तꣳ रु॒द्रेभ्यः॒ प्र । \newline
26. रु॒द्रेभ्यः॒ प्र प्र रु॒द्रेभ्यो॑ रु॒द्रेभ्यः॒ प्राय॑च्छन् नयच्छ॒न् प्र रु॒द्रेभ्यो॑ रु॒द्रेभ्यः॒ प्राय॑च्छन्न् । \newline
27. प्राय॑च्छन् नयच्छ॒न् प्र प्राय॑च्छ॒न् तम् त म॑यच्छ॒न् प्र प्राय॑च्छ॒न् तम् । \newline
28. अ॒य॒च्छ॒न् तम् त म॑यच्छन् नयच्छ॒न् तꣳ षट् थ्षट् त म॑यच्छन् नयच्छ॒न् तꣳ षट् । \newline
29. तꣳ षट् थ्षट् तम् तꣳ षड हा॒न्य हा॑नि॒ षट् तम् तꣳ षडहा॑नि । \newline
30. षडहा॒ न्यहा॑नि॒ षट् थ्षड हा᳚न्यबिभरु रबिभरु॒ रहा॑नि॒ षट् थ्षड हा᳚न्यबिभरुः । \newline
31. अहा᳚ न्यबिभरु रबिभरु॒ रहा॒ न्यहा᳚ न्यबिभ रु॒स्तेन॒ तेना॑ बिभरु॒ रहा॒ न्यहा᳚ न्यबिभरु॒ स्तेन॑ । \newline
32. अ॒बि॒भ॒रु॒ स्तेन॒ तेना॑ बिभरु रबिभरु॒ स्तेन॒ त्रीणि॒ त्रीणि॒ तेना॑ बिभरु रबिभरु॒ स्तेन॒ त्रीणि॑ । \newline
33. तेन॒ त्रीणि॒ त्रीणि॒ तेन॒ तेन॒ त्रीणि॑ च च॒ त्रीणि॒ तेन॒ तेन॒ त्रीणि॑ च । \newline
34. त्रीणि॑ च च॒ त्रीणि॒ त्रीणि॑ च श॒तानि॑ श॒तानि॑ च॒ त्रीणि॒ त्रीणि॑ च श॒तानि॑ । \newline
35. च॒ श॒तानि॑ श॒तानि॑ च च श॒ता न्यसृ॑ज॒न्ता सृ॑जन्त श॒तानि॑ च च श॒ता न्यसृ॑जन्त । \newline
36. श॒ता न्यसृ॑ज॒न्ता सृ॑जन्त श॒तानि॑ श॒ता न्यसृ॑जन्त॒ त्रय॑स्त्रिꣳशत॒म् त्रय॑स्त्रिꣳशत॒ मसृ॑जन्त श॒तानि॑ श॒ता न्यसृ॑जन्त॒ त्रय॑स्त्रिꣳशतम् । \newline
37. असृ॑जन्त॒ त्रय॑स्त्रिꣳशत॒म् त्रय॑स्त्रिꣳशत॒ मसृ॑ज॒न्ता सृ॑जन्त॒ त्रय॑स्त्रिꣳशतम् च च॒ त्रय॑स्त्रिꣳशत॒ मसृ॑ज॒न्ता सृ॑जन्त॒ त्रय॑स्त्रिꣳशतम् च । \newline
38. त्रय॑स्त्रिꣳशतम् च च॒ त्रय॑स्त्रिꣳशत॒म् त्रय॑स्त्रिꣳशतम् च॒ तस्मा॒त् तस्मा᳚च् च॒ त्रय॑स्त्रिꣳशत॒म् त्रय॑स्त्रिꣳशतम् च॒ तस्मा᳚त् । \newline
39. त्रय॑स्त्रिꣳशत॒मिति॒ त्रयः॑ - त्रिꣳ॒॒श॒त॒म् । \newline
40. च॒ तस्मा॒त् तस्मा᳚च् च च॒ तस्मा᳚ थ्षड॒हꣳ ष॑ड॒हम् तस्मा᳚च् च च॒ तस्मा᳚ थ्षड॒हम् । \newline
41. तस्मा᳚ थ्षड॒हꣳ ष॑ड॒हम् तस्मा॒त् तस्मा᳚ थ्षड॒हम् भा॒र्यो॑ भा॒र्य॑ ष्षड॒हम् तस्मा॒त् तस्मा᳚ थ्षड॒हम् भा॒र्यः॑ । \newline
42. ष॒ड॒हम् भा॒र्यो॑ भा॒र्य॑ ष्षड॒हꣳ ष॑ड॒हम् भा॒र्यः॑ प्र प्र भा॒र्य॑ ष्षड॒हꣳ ष॑ड॒हम् भा॒र्यः॑ प्र । \newline
43. ष॒ड॒हमिति॑ षट् - अ॒हम् । \newline
44. भा॒र्यः॑ प्र प्र भा॒र्यो॑ भा॒र्यः॑ प्रैवैव प्र भा॒र्यो॑ भा॒र्यः॑ प्रैव । \newline
45. प्रैवैव प्र प्रैव जा॑यते जायत ए॒व प्र प्रैव जा॑यते । \newline
46. ए॒व जा॑यते जायत ए॒वैव जा॑यते॒ ताꣳ स्तान् जा॑यत ए॒वैव जा॑यते॒ तान् । \newline
47. जा॒य॒ते॒ ताꣳ स्तान् जा॑यते जायते॒ ताना॑दि॒त्या आ॑दि॒त्या स्तान् जा॑यते जायते॒ ताना॑दि॒त्याः । \newline
48. ताना॑दि॒त्या आ॑दि॒त्या स्ताꣳ स्ताना॑दि॒त्या अ॑ब्रुवन् नब्रुवन् नादि॒त्या स्ताꣳ स्ताना॑दि॒त्या अ॑ब्रुवन्न् । \newline
49. आ॒दि॒त्या अ॑ब्रुवन् नब्रुवन् नादि॒त्या आ॑दि॒त्या अ॑ब्रुव॒न् प्र प्राब्रु॑वन् नादि॒त्या आ॑दि॒त्या अ॑ब्रुव॒न् प्र । \newline
50. अ॒ब्रु॒व॒न् प्र प्राब्रु॑वन् नब्रुव॒न् प्र यू॒यं ॅयू॒यम् प्राब्रु॑वन् नब्रुव॒न् प्र यू॒यम् । \newline
51. प्र यू॒यं ॅयू॒यम् प्र प्र यू॒य म॑जनिढ्व मजनिढ्वं ॅयू॒यम् प्र प्र यू॒य म॑जनिढ्वम् । \newline
52. यू॒य म॑जनिढ्व मजनिढ्वं ॅयू॒यं ॅयू॒य म॑जनिढ्वं ॅव॒यं ॅव॒य म॑जनिढ्वं ॅयू॒यं ॅयू॒य म॑जनिढ्वं ॅव॒यम् । \newline
53. अ॒ज॒नि॒ढ्वं॒ ॅव॒यं ॅव॒य म॑जनिढ्व मजनिढ्वं ॅव॒यम् प्र प्र व॒य म॑जनिढ्व मजनिढ्वं ॅव॒यम् प्र । \newline
54. व॒यम् प्र प्र व॒यं ॅव॒यम् प्र जा॑यामहै जायामहै॒ प्र व॒यं ॅव॒यम् प्र जा॑यामहै । \newline
\pagebreak
\markright{ TS 5.5.2.7  \hfill https://www.vedavms.in \hfill}

\section{ TS 5.5.2.7 }

\textbf{TS 5.5.2.7 } \newline
\textbf{Samhita Paata} \newline

प्र जा॑यामहा॒ इति॒ तमा॑दि॒त्येभ्यः॒ प्राय॑च्छ॒न् तं द्वाद॒शाहा᳚न्यबिभरु॒स्तेन॒ त्रीणि॑ च श॒तान्यसृ॑जन्त॒ त्रय॑स्त्रिꣳशतं च॒ तस्मा᳚द् द्वादशा॒हं भा॒र्यः॑ प्रैव जा॑यते॒ तेन॒ वै ते स॒हस्र॑मसृजन्तो॒खाꣳ स॑हस्रत॒मीं ॅय ए॒वमुख्यꣳ॑ साह॒स्रं ॅवेद॒ प्र स॒हस्रं॑ प॒शूना᳚प्नोति ॥ \newline

\textbf{Pada Paata} \newline

प्रेति॑ । जा॒या॒म॒है॒ । इति॑ । तम् । आ॒दि॒त्येभ्यः॑ । प्रेति॑ । अ॒य॒च्छ॒न्न् । तम् । द्वाद॑श । अहा॑नि । अ॒बि॒भ॒रुः॒ । तेन॑ । त्रीणि॑ । च॒ । श॒तानि॑ । असृ॑जन्त । त्रय॑स्त्रिꣳशत॒मिति॒ त्रयः॑ - त्रिꣳ॒॒श॒त॒म् । च॒ । तस्मा᳚त् । द्वा॒द॒शा॒हमिति॑ द्वादश - अ॒हम् । भा॒र्यः॑ । प्रेति॑ । ए॒व । जा॒य॒ते॒ । तेन॑ । वै । ते । स॒हस्र᳚म् । अ॒सृ॒ज॒न्त॒ । उ॒खाम् । स॒ह॒स्र॒त॒मीमिति॑ सहस्र - त॒मीम् । यः । ए॒वम् । उख्य᳚म् । सा॒ह॒स्रम् । वेद॑ । प्रेति॑ । स॒हस्र᳚म् । प॒शून् । आ॒प्नो॒ति॒ ॥  \newline


\textbf{Krama Paata} \newline

प्र जा॑यामहै । जा॒या॒म॒हा॒ इति॑ । इति॒ तम् । तमा॑दि॒त्येभ्यः॑ । आ॒दि॒त्येभ्यः॒ प्र । प्राय॑च्छन्न् । अ॒य॒च्छ॒न् तम् । तम् द्वाद॑श । द्वाद॒शाहा॑नि । अहा᳚न्यबिभरुः । अ॒बि॒भ॒रु॒स्तेन॑ । तेन॒ त्रीणि॑ । त्रीणि॑ च । च॒ श॒तानि॑ । श॒तान्यसृ॑जन्त । असृ॑जन्त॒ त्रय॑स्त्रिꣳशतम् । त्रय॑स्त्रिꣳशतम् च । त्रय॑स्त्रिꣳशत॒मिति॒ त्रयः॑ - त्रिꣳ॒॒श॒त॒म् । च॒ तस्मा᳚त् । तस्मा᳚द् द्वादशा॒हम् । द्वा॒द॒शा॒हम् भा॒र्यः॑ । द्वा॒द॒शा॒हमिति॑ द्वादश - अ॒हम् । भा॒र्यः॑ प्र । प्रैव । ए॒व जा॑यते । जा॒य॒ते॒ तेन॑ । तेन॒ वै । वै ते । ते स॒हस्र᳚म् । स॒हस्र॑मसृजन्त । अ॒सृ॒ज॒न्तो॒खाम् । उ॒खाꣳ स॑हस्रत॒मीम् । स॒ह॒स्र॒त॒मीम् ॅयः । स॒ह॒स्र॒त॒मीमिति॑ सहस्र - त॒मीम् । य ए॒वम् । ए॒वमुख्य᳚म् । उख्यꣳ॑ साह॒स्रम् । सा॒ह॒स्रम् ॅवेद॑ । वेद॒ प्र । प्र स॒हस्र᳚म् । स॒हस्र॑म् प॒शून् । प॒शूना᳚प्नोति । आ॒प्नो॒तीत्या᳚प्नोति । \newline

\textbf{Jatai Paata} \newline

1. प्र जा॑यामहै जायामहै॒ प्र प्र जा॑यामहै । \newline
2. जा॒या॒म॒हा॒ इतीति॑ जायामहै जायामहा॒ इति॑ । \newline
3. इति॒ तम् त मितीति॒ तम् । \newline
4. त मा॑दि॒त्येभ्य॑ आदि॒त्येभ्य॒ स्तम् त मा॑दि॒त्येभ्यः॑ । \newline
5. आ॒दि॒त्येभ्यः॒ प्र प्रादि॒त्येभ्य॑ आदि॒त्येभ्यः॒ प्र । \newline
6. प्राय॑च्छन् नयच्छ॒न् प्र प्राय॑च्छन्न् । \newline
7. अ॒य॒च्छ॒न् तम् त म॑यच्छन् नयच्छ॒न् तम् । \newline
8. तम् द्वाद॑श॒ द्वाद॑श॒ तम् तम् द्वाद॑श । \newline
9. द्वाद॒शाहा॒ न्यहा॑नि॒ द्वाद॑श॒ द्वाद॒शाहा॑नि । \newline
10. अहा᳚ न्यबिभरु रबिभरु॒ रहा॒ न्यहा᳚ न्यबिभरुः । \newline
11. अ॒बि॒भ॒रु॒ स्तेन॒ तेना॑बिभरु रबिभरु॒ स्तेन॑ । \newline
12. तेन॒ त्रीणि॒ त्रीणि॒ तेन॒ तेन॒ त्रीणि॑ । \newline
13. त्रीणि॑ च च॒ त्रीणि॒ त्रीणि॑ च । \newline
14. च॒ श॒तानि॑ श॒तानि॑ च च श॒तानि॑ । \newline
15. श॒तान्य सृ॑ज॒न्ता सृ॑जन्त श॒तानि॑ श॒तान्य सृ॑जन्त । \newline
16. असृ॑जन्त॒ त्रय॑स्त्रिꣳशत॒म् त्रय॑स्त्रिꣳशत॒ मसृ॑ज॒न्ता सृ॑जन्त॒ त्रय॑स्त्रिꣳशतम् । \newline
17. त्रय॑स्त्रिꣳशतम् च च॒ त्रय॑स्त्रिꣳशत॒म् त्रय॑स्त्रिꣳशतम् च । \newline
18. त्रय॑स्त्रिꣳशत॒मिति॒ त्रयः॑ - त्रिꣳ॒॒श॒त॒म् । \newline
19. च॒ तस्मा॒त् तस्मा᳚च् च च॒ तस्मा᳚त् । \newline
20. तस्मा᳚द् द्वादशा॒हम् द्वा॑दशा॒हम् तस्मा॒त् तस्मा᳚द् द्वादशा॒हम् । \newline
21. द्वा॒द॒शा॒हम् भा॒र्यो॑ भा॒र्यो᳚ द्वादशा॒हम् द्वा॑दशा॒हम् भा॒र्यः॑ । \newline
22. द्वा॒द॒शा॒हमिति॑ द्वादश - अ॒हम् । \newline
23. भा॒र्यः॑ प्र प्र भा॒र्यो॑ भा॒र्यः॑ प्र । \newline
24. प्रैवैव प्र प्रैव । \newline
25. ए॒व जा॑यते जायत ए॒वैव जा॑यते । \newline
26. जा॒य॒ते॒ तेन॒ तेन॑ जायते जायते॒ तेन॑ । \newline
27. तेन॒ वै वै तेन॒ तेन॒ वै । \newline
28. वै ते ते वै वै ते । \newline
29. ते स॒हस्रꣳ॑ स॒हस्र॒म् ते ते स॒हस्र᳚म् । \newline
30. स॒हस्र॑ मसृजन्ता सृजन्त स॒हस्रꣳ॑ स॒हस्र॑ मसृजन्त । \newline
31. अ॒सृ॒ज॒न्तो॒खा मु॒खा म॑सृजन्ता सृजन्तो॒खाम् । \newline
32. उ॒खाꣳ स॑हस्रत॒मीꣳ स॑हस्रत॒मी मु॒खा मु॒खाꣳ स॑हस्रत॒मीम् । \newline
33. स॒ह॒स्र॒त॒मीं ॅयो यः स॑हस्रत॒मीꣳ स॑हस्रत॒मीं ॅयः । \newline
34. स॒ह॒स्र॒त॒मीमिति॑ सहस्र - त॒मीम् । \newline
35. य ए॒व मे॒वं ॅयो य ए॒वम् । \newline
36. ए॒व मुख्य॒ मुख्य॑ मे॒व मे॒व मुख्य᳚म् । \newline
37. उख्यꣳ॑ साह॒स्रꣳ सा॑ह॒स्र मुख्य॒ मुख्यꣳ॑ साह॒स्रम् । \newline
38. सा॒ह॒स्रं ॅवेद॒ वेद॑ साह॒स्रꣳ सा॑ह॒स्रं ॅवेद॑ । \newline
39. वेद॒ प्र प्र वेद॒ वेद॒ प्र । \newline
40. प्र स॒हस्रꣳ॑ स॒हस्र॒म् प्र प्र स॒हस्र᳚म् । \newline
41. स॒हस्र॑म् प॒शून् प॒शून् थ्स॒हस्रꣳ॑ स॒हस्र॑म् प॒शून् । \newline
42. प॒शू ना᳚प्नो त्याप्नोति प॒शून् प॒शू ना᳚प्नोति । \newline
43. आ॒प्नो॒तीत्या᳚प्नोति । \newline

\textbf{Ghana Paata } \newline

1. प्र जा॑यामहै जायामहै॒ प्र प्र जा॑यामहा॒ इतीति॑ जायामहै॒ प्र प्र जा॑यामहा॒ इति॑ । \newline
2. जा॒या॒म॒हा॒ इतीति॑ जायामहै जायामहा॒ इति॒ तम् त मिति॑ जायामहै जायामहा॒ इति॒ तम् । \newline
3. इति॒ तम् त मितीति॒ त मा॑दि॒त्येभ्य॑ आदि॒त्येभ्य॒ स्त मितीति॒ त मा॑दि॒त्येभ्यः॑ । \newline
4. त मा॑दि॒त्येभ्य॑ आदि॒त्येभ्य॒ स्तम् त मा॑दि॒त्येभ्यः॒ प्र प्रादि॒त्येभ्य॒ स्तम् त मा॑दि॒त्येभ्यः॒ प्र । \newline
5. आ॒दि॒त्येभ्यः॒ प्र प्रादि॒त्येभ्य॑ आदि॒त्येभ्यः॒ प्राय॑च्छन् नयच्छ॒न् प्रादि॒त्येभ्य॑ आदि॒त्येभ्यः॒ प्राय॑च्छन्न् । \newline
6. प्राय॑च्छन् नयच्छ॒न् प्र प्राय॑च्छ॒न् तम् त म॑यच्छ॒न् प्र प्राय॑च्छ॒न् तम् । \newline
7. अ॒य॒च्छ॒न् तम् तम॑यच्छन् नयच्छ॒न् तम् द्वाद॑श॒ द्वाद॑श॒ त म॑यच्छन् नयच्छ॒न् तम् द्वाद॑श । \newline
8. तम् द्वाद॑श॒ द्वाद॑श॒ तम् तम् द्वाद॒शा हा॒न्यहा॑नि॒ द्वाद॑श॒ तम् तम् द्वाद॒शा हा॑नि । \newline
9. द्वाद॒शा हा॒न्यहा॑नि॒ द्वाद॑श॒ द्वाद॒शाहा᳚ न्यबिभरु रबिभरु॒ रहा॑नि॒ द्वाद॑श॒ द्वाद॒शा हा᳚न्यबिभरुः । \newline
10. अहा᳚ न्यबिभरु रबिभरु॒ रहा॒न्यहा᳚ न्यबिभरु॒ स्तेन॒ तेना॑ बिभरु॒ रहा॒न्यहा᳚ न्यबिभरु॒ स्तेन॑ । \newline
11. अ॒बि॒भ॒रु॒ स्तेन॒ तेना॑ बिभरु रबिभरु॒ स्तेन॒ त्रीणि॒ त्रीणि॒ तेना॑ बिभरु रबिभरु॒ स्तेन॒ त्रीणि॑ । \newline
12. तेन॒ त्रीणि॒ त्रीणि॒ तेन॒ तेन॒ त्रीणि॑ च च॒ त्रीणि॒ तेन॒ तेन॒ त्रीणि॑ च । \newline
13. त्रीणि॑ च च॒ त्रीणि॒ त्रीणि॑ च श॒तानि॑ श॒तानि॑ च॒ त्रीणि॒ त्रीणि॑ च श॒तानि॑ । \newline
14. च॒ श॒तानि॑ श॒तानि॑ च च श॒तान्य सृ॑ज॒ न्तासृ॑जन्त श॒तानि॑ च च श॒ता न्यसृ॑जन्त । \newline
15. श॒ता न्यसृ॑ज॒न्ता सृ॑जन्त श॒तानि॑ श॒ता न्यसृ॑जन्त॒ त्रय॑स्त्रिꣳशत॒म् त्रय॑स्त्रिꣳशत॒ मसृ॑जन्त श॒तानि॑ श॒ता न्यसृ॑जन्त॒ त्रय॑स्त्रिꣳशतम् । \newline
16. असृ॑जन्त॒ त्रय॑स्त्रिꣳशत॒म् त्रय॑स्त्रिꣳशत॒ मसृ॑ज॒न्ता सृ॑जन्त॒ त्रय॑स्त्रिꣳशतम् च च॒ त्रय॑स्त्रिꣳशत॒ मसृ॑ज॒न्ता सृ॑जन्त॒ त्रय॑स्त्रिꣳशतम् च । \newline
17. त्रय॑स्त्रिꣳशतम् च च॒ त्रय॑स्त्रिꣳशत॒म् त्रय॑स्त्रिꣳशतम् च॒ तस्मा॒त् तस्मा᳚च् च॒ त्रय॑स्त्रिꣳशत॒म् त्रय॑स्त्रिꣳशतम् च॒ तस्मा᳚त् । \newline
18. त्रय॑स्त्रिꣳशत॒मिति॒ त्रयः॑ - त्रिꣳ॒॒श॒त॒म् । \newline
19. च॒ तस्मा॒त् तस्मा᳚च् च च॒ तस्मा᳚द् द्वादशा॒हम् द्वा॑दशा॒हम् तस्मा᳚च् च च॒ तस्मा᳚द् द्वादशा॒हम् । \newline
20. तस्मा᳚द् द्वादशा॒हम् द्वा॑दशा॒हम् तस्मा॒त् तस्मा᳚द् द्वादशा॒हम् भा॒र्यो॑ भा॒र्यो᳚ द्वादशा॒हम् तस्मा॒त् तस्मा᳚द् द्वादशा॒हम् भा॒र्यः॑ । \newline
21. द्वा॒द॒शा॒हम् भा॒र्यो॑ भा॒र्यो᳚ द्वादशा॒हम् द्वा॑दशा॒हम् भा॒र्यः॑ प्र प्र भा॒र्यो᳚ द्वादशा॒हम् द्वा॑दशा॒हम् भा॒र्यः॑ प्र । \newline
22. द्वा॒द॒शा॒हमिति॑ द्वादश - अ॒हम् । \newline
23. भा॒र्यः॑ प्र प्र भा॒र्यो॑ भा॒र्यः॑ प्रैवैव प्र भा॒र्यो॑ भा॒र्यः॑ प्रैव । \newline
24. प्रैवैव प्र प्रैव जा॑यते जायत ए॒व प्र प्रैव जा॑यते । \newline
25. ए॒व जा॑यते जायत ए॒वैव जा॑यते॒ तेन॒ तेन॑ जायत ए॒वैव जा॑यते॒ तेन॑ । \newline
26. जा॒य॒ते॒ तेन॒ तेन॑ जायते जायते॒ तेन॒ वै वै तेन॑ जायते जायते॒ तेन॒ वै । \newline
27. तेन॒ वै वै तेन॒ तेन॒ वै ते ते वै तेन॒ तेन॒ वै ते । \newline
28. वै ते ते वै वै ते स॒हस्रꣳ॑ स॒हस्र॒म् ते वै वै ते स॒हस्र᳚म् । \newline
29. ते स॒हस्रꣳ॑ स॒हस्र॒म् ते ते स॒हस्र॑ मसृजन्ता सृजन्त स॒हस्र॒म् ते ते स॒हस्र॑ मसृजन्त । \newline
30. स॒हस्र॑ मसृजन्ता सृजन्त स॒हस्रꣳ॑ स॒हस्र॑ मसृजन्तो॒खा मु॒खा म॑सृजन्त स॒हस्रꣳ॑ स॒हस्र॑ मसृजन्तो॒खाम् । \newline
31. अ॒सृ॒ज॒न्तो॒खा मु॒खा म॑सृजन्ता सृजन्तो॒खाꣳ स॑हस्रत॒मीꣳ स॑हस्रत॒मी मु॒खा म॑सृजन्ता सृजन्तो॒खाꣳ स॑हस्रत॒मीम् । \newline
32. उ॒खाꣳ स॑हस्रत॒मीꣳ स॑हस्रत॒मी मु॒खा मु॒खाꣳ स॑हस्रत॒मीं ॅयो यः स॑हस्रत॒मी मु॒खा मु॒खाꣳ स॑हस्रत॒मीं ॅयः । \newline
33. स॒ह॒स्र॒त॒मीं ॅयो यः स॑हस्रत॒मीꣳ स॑हस्रत॒मीं ॅय ए॒व मे॒वं ॅयः स॑हस्रत॒मीꣳ स॑हस्रत॒मीं ॅय ए॒वम् । \newline
34. स॒ह॒स्र॒त॒मीमिति॑ सहस्र - त॒मीम् । \newline
35. य ए॒व मे॒वं ॅयो य ए॒व मुख्य॒ मुख्य॑ मे॒वं ॅयो य ए॒व मुख्य᳚म् । \newline
36. ए॒व मुख्य॒ मुख्य॑ मे॒व मे॒व मुख्यꣳ॑ साह॒स्रꣳ सा॑ह॒स्र मुख्य॑ मे॒व मे॒व मुख्यꣳ॑ साह॒स्रम् । \newline
37. उख्यꣳ॑ साह॒स्रꣳ सा॑ह॒स्र मुख्य॒ मुख्यꣳ॑ साह॒स्रं ॅवेद॒ वेद॑ साह॒स्र मुख्य॒ मुख्यꣳ॑ साह॒स्रं ॅवेद॑ । \newline
38. सा॒ह॒स्रं ॅवेद॒ वेद॑ साह॒स्रꣳ सा॑ह॒स्रं ॅवेद॒ प्र प्र वेद॑ साह॒स्रꣳ सा॑ह॒स्रं ॅवेद॒ प्र । \newline
39. वेद॒ प्र प्र वेद॒ वेद॒ प्र स॒हस्रꣳ॑ स॒हस्र॒म् प्र वेद॒ वेद॒ प्र स॒हस्र᳚म् । \newline
40. प्र स॒हस्रꣳ॑ स॒हस्र॒म् प्र प्र स॒हस्र॑म् प॒शून् प॒शून् थ्स॒हस्र॒म् प्र प्र स॒हस्र॑म् प॒शून् । \newline
41. स॒हस्र॑म् प॒शून् प॒शून् थ्स॒हस्रꣳ॑ स॒हस्र॑म् प॒शू ना᳚प्नो त्याप्नोति प॒शून् थ्स॒हस्रꣳ॑ स॒हस्र॑म् प॒शू ना᳚प्नोति । \newline
42. प॒शूना᳚ प्नो त्याप्नोति प॒शून् प॒शू ना᳚प्नोति । \newline
43. आ॒प्नो॒तीत्या᳚प्नोति । \newline
\pagebreak
\markright{ TS 5.5.3.1  \hfill https://www.vedavms.in \hfill}

\section{ TS 5.5.3.1 }

\textbf{TS 5.5.3.1 } \newline
\textbf{Samhita Paata} \newline

यजु॑षा॒ वा ए॒षा क्रि॑यते॒ यजु॑षा पच्यते॒ यजु॑षा॒ वि मु॑च्यते॒ यदु॒खा सा वा ए॒षैतर्.हि॑ या॒तया᳚म्नी॒ सा न पुनः॑ प्र॒युज्येत्या॑हु॒रग्ने॑ यु॒क्ष्वा हि ये तव॑ यु॒क्ष्वा हि दे॑व॒हूत॑माꣳ॒॒ इत्यु॒खायां᳚ जुहोति॒ तेनै॒वैनां॒ पुनः॒ प्रयु॑ङ्क्ते॒ तेनाया॑तयाम्नी॒ यो वा अ॒ग्निं ॅयोग॒ आग॑ते यु॒नक्ति॑ यु॒ङ्क्ते यु॑ञ्जा॒नेष्वग्ने॑ - [  ] \newline

\textbf{Pada Paata} \newline

यजु॑षा । वै । ए॒षा । क्रि॒य॒ते॒ । यजु॑षा । प॒च्य॒ते॒ । यजु॑षा । वीति॑ । मु॒च्य॒ते॒ । यत् । उ॒खा । सा । वै । ए॒षा । ए॒तर्.हि॑ । या॒तया॒म्नीति॑ या॒त - या॒म्नी॒ । सा । न । पुनः॑ । प्र॒युज्येति॑ प्र - युज्या᳚ । इति॑ । आ॒हुः॒ । अग्ने᳚ । यु॒क्ष्व । हि । ये । तव॑ । यु॒क्ष्व । हि । दे॒व॒हूत॑मा॒निति॑ देव - हूत॑मान् । इति॑ । उ॒खाया᳚म् । जु॒हो॒ति॒ । तेन॑ । ए॒व । ए॒ना॒म् । पुनः॑ । प्रेति॑ । यु॒ङ्क्ते॒ । तेन॑ । अया॑तया॒म्नीत्यया॑त - या॒म्नी॒ । यः । वै । अ॒ग्निम् । योगे᳚ । आग॑त॒ इत्या - ग॒ते॒ । यु॒नक्ति॑ । यु॒ङ्क्ते । यु॒ञ्जा॒नेषु॑ । अग्ने᳚ ।  \newline


\textbf{Krama Paata} \newline

यजु॑षा॒ वै । वा ए॒षा । ए॒षा क्रि॑यते । क्रि॒य॒ते॒ यजु॑षा । यजु॑षा पच्यते । प॒च्य॒ते॒ यजु॑षा । यजु॑षा॒ वि । वि मु॑च्यते । मु॒च्य॒ते॒ यत् । यदु॒खा । उ॒खा सा । सा वै । वा ए॒षा । ए॒षैतर्.हि॑ । ए॒तर्.हि॑ या॒तया᳚म्नी । या॒तया᳚म्नी॒ सा । या॒तया॒म्नीति॑ या॒त - या॒म्नी॒ । सा न । न पुनः॑ । पुनः॑ प्र॒युज्या᳚ । प्र॒युज्येति॑ । प्र॒युज्येति॑ प्र - युज्या᳚ । इत्या॑हुः । आ॒हु॒रग्ने᳚ । अग्ने॑ यु॒क्ष्व । यु॒क्ष्वा हि । हि ये । ये तव॑ । तव॑ यु॒क्ष्व । यु॒क्ष्वा हि । हि दे॑व॒हूत॑मान् । दे॒व॒हूत॑माꣳ॒॒ इति॑ । दे॒व॒हूत॑मा॒निति॑ देव - हूत॑मान् । इत्यु॒खाया᳚म् । उ॒खाया᳚म् जुहोति । जु॒हो॒ति॒ तेन॑ । तेनै॒व । ए॒वैना᳚म् । ए॒ना॒म् पुनः॑ । पुनः॒ प्र । प्र यु॑ङ्क्ते । यु॒ङ्क्ते॒ तेन॑ । तेनाया॑तयाम्नी । अया॑तयाम्नी॒ यः । अया॑तया॒म्नीत्यया॑त - या॒म्नी॒ । यो वै । वा अ॒ग्निम् । अ॒ग्निम् ॅयोगे᳚ । योग॒ आग॑ते । आग॑ते यु॒नक्ति॑ । आग॑त॒ इत्या - ग॒ते॒ । यु॒नक्ति॑ यु॒ङ्क्ते । यु॒ङ्क्ते यु॑ञ्जा॒नेषु॑ । यु॒ञ्जा॒नेष्वग्ने᳚ । अग्ने॑ यु॒क्ष्व \newline

\textbf{Jatai Paata} \newline

1. यजु॑षा॒ वै वै यजु॑षा॒ यजु॑षा॒ वै । \newline
2. वा ए॒षैषा वै वा ए॒षा । \newline
3. ए॒षा क्रि॑यते क्रियत ए॒षैषा क्रि॑यते । \newline
4. क्रि॒य॒ते॒ यजु॑षा॒ यजु॑षा क्रियते क्रियते॒ यजु॑षा । \newline
5. यजु॑षा पच्यते पच्यते॒ यजु॑षा॒ यजु॑षा पच्यते । \newline
6. प॒च्य॒ते॒ यजु॑षा॒ यजु॑षा पच्यते पच्यते॒ यजु॑षा । \newline
7. यजु॑षा॒ वि वि यजु॑षा॒ यजु॑षा॒ वि । \newline
8. वि मु॑च्यते मुच्यते॒ वि वि मु॑च्यते । \newline
9. मु॒च्य॒ते॒ यद् यन् मु॑च्यते मुच्यते॒ यत् । \newline
10. यदु॒खोखा यद् यदु॒खा । \newline
11. उ॒खा सा सोखोखा सा । \newline
12. सा वै वै सा सा वै । \newline
13. वा ए॒षैषा वै वा ए॒षा । \newline
14. ए॒षै तर् ह्ये॒तर् ह्ये॒षै षैतर्.हि॑ । \newline
15. ए॒तर्.हि॑ या॒तया᳚म्नी या॒तया᳚ म्न्ये॒तर् ह्ये॒तर्.हि॑ या॒तया᳚म्नी । \newline
16. या॒तया᳚म्नी॒ सा सा या॒तया᳚म्नी या॒तया᳚म्नी॒ सा । \newline
17. या॒तया॒म्नीति॑ या॒त - या॒म्नी॒ । \newline
18. सा न न सा सा न । \newline
19. न पुनः॒ पुन॒र् न न पुनः॑ । \newline
20. पुनः॑ प्र॒युज्या᳚ प्र॒युज्या॒ पुनः॒ पुनः॑ प्र॒युज्या᳚ । \newline
21. प्र॒युज्येतीति॑ प्र॒युज्या᳚ प्र॒युज्येति॑ । \newline
22. प्र॒युज्येति॑ प्र - युज्या᳚ । \newline
23. इत्या॑हु राहु॒रिती त्या॑हुः । \newline
24. आ॒हु॒ रग्ने ऽग्न॑ आहु राहु॒ रग्ने᳚ । \newline
25. अग्ने॑ यु॒क्ष्व यु॒क्ष्वाग्ने ऽग्ने॑ यु॒क्ष्व । \newline
26. यु॒क्ष्वा हि हि यु॒क्ष्व यु॒क्ष्वा हि । \newline
27. हि ये ये हि हि ये । \newline
28. ये तव॒ तव॒ ये ये तव॑ । \newline
29. तव॑ यु॒क्ष्व यु॒क्ष्व तव॒ तव॑ यु॒क्ष्व । \newline
30. यु॒क्ष्वा हि हि यु॒क्ष्व यु॒क्ष्वा हि । \newline
31. हि दे॑व॒हूत॑मान् देव॒हूत॑मा॒न्॒. हि हि दे॑व॒हूत॑मान् । \newline
32. दे॒व॒हूत॑माꣳ॒॒ इतीति॑ देव॒हूत॑मान् देव॒हूत॑माꣳ॒॒ इति॑ । \newline
33. दे॒व॒हूत॑मा॒निति॑ देव - हूत॑मान् । \newline
34. इत्यु॒खाया॑ मु॒खाया॒ मिती त्यु॒खाया᳚म् । \newline
35. उ॒खाया᳚म् जुहोति जुहो त्यु॒खाया॑ मु॒खाया᳚म् जुहोति । \newline
36. जु॒हो॒ति॒ तेन॒ तेन॑ जुहोति जुहोति॒ तेन॑ । \newline
37. तेनै॒ वैव तेन॒ तेनै॒व । \newline
38. ए॒वैना॑ मेना मे॒वै वैना᳚म् । \newline
39. ए॒ना॒म् पुनः॒ पुन॑ रेना मेना॒म् पुनः॑ । \newline
40. पुनः॒ प्र प्र पुनः॒ पुनः॒ प्र । \newline
41. प्र यु॑ङ्क्ते युङ्क्ते॒ प्र प्र यु॑ङ्क्ते । \newline
42. यु॒ङ्क्ते॒ तेन॒ तेन॑ युङ्क्ते युङ्क्ते॒ तेन॑ । \newline
43. तेना या॑तया॒ म्न्यया॑तया॒म्नी तेन॒ तेनाया॑तया॒म्नी । \newline
44. अया॑तया॒म्नी यो यो ऽया॑तया॒ म्न्यया॑तया॒म्नी यः । \newline
45. अया॑तया॒म्नीत्यया॑त - या॒म्नी॒ । \newline
46. यो वै वै यो यो वै । \newline
47. वा अ॒ग्नि म॒ग्निं ॅवै वा अ॒ग्निम् । \newline
48. अ॒ग्निं ॅयोगे॒ योगे॒ ऽग्नि म॒ग्निं ॅयोगे᳚ । \newline
49. योग॒ आग॑त॒ आग॑ते॒ योगे॒ योग॒ आग॑ते । \newline
50. आग॑ते यु॒नक्ति॑ यु॒नक्त्या ग॑त॒ आग॑ते यु॒नक्ति॑ । \newline
51. आग॑त॒ इत्या - ग॒ते॒ । \newline
52. यु॒नक्ति॑ यु॒ङ्क्ते यु॒ङ्क्ते यु॒नक्ति॑ यु॒नक्ति॑ यु॒ङ्क्ते । \newline
53. यु॒ङ्क्ते यु॑ञ्जा॒नेषु॑ युञ्जा॒नेषु॑ यु॒ङ्क्ते यु॒ङ्क्ते यु॑ञ्जा॒नेषु॑ । \newline
54. यु॒ञ्जा॒ने ष्वग्ने ऽग्ने॑ युञ्जा॒नेषु॑ युञ्जा॒ने ष्वग्ने᳚ । \newline
55. अग्ने॑ यु॒क्ष्व यु॒क्ष्वाग्ने ऽग्ने॑ यु॒क्ष्व । \newline

\textbf{Ghana Paata } \newline

1. यजु॑षा॒ वै वै यजु॑षा॒ यजु॑षा॒ वा ए॒षैषा वै यजु॑षा॒ यजु॑षा॒ वा ए॒षा । \newline
2. वा ए॒षैषा वै वा ए॒षा क्रि॑यते क्रियत ए॒षा वै वा ए॒षा क्रि॑यते । \newline
3. ए॒षा क्रि॑यते क्रियत ए॒षैषा क्रि॑यते॒ यजु॑षा॒ यजु॑षा क्रियत ए॒षैषा क्रि॑यते॒ यजु॑षा । \newline
4. क्रि॒य॒ते॒ यजु॑षा॒ यजु॑षा क्रियते क्रियते॒ यजु॑षा पच्यते पच्यते॒ यजु॑षा क्रियते क्रियते॒ यजु॑षा पच्यते । \newline
5. यजु॑षा पच्यते पच्यते॒ यजु॑षा॒ यजु॑षा पच्यते॒ यजु॑षा॒ यजु॑षा पच्यते॒ यजु॑षा॒ यजु॑षा पच्यते॒ यजु॑षा । \newline
6. प॒च्य॒ते॒ यजु॑षा॒ यजु॑षा पच्यते पच्यते॒ यजु॑षा॒ वि वि यजु॑षा पच्यते पच्यते॒ यजु॑षा॒ वि । \newline
7. यजु॑षा॒ वि वि यजु॑षा॒ यजु॑षा॒ वि मु॑च्यते मुच्यते॒ वि यजु॑षा॒ यजु॑षा॒ वि मु॑च्यते । \newline
8. वि मु॑च्यते मुच्यते॒ वि वि मु॑च्यते॒ यद् यन् मु॑च्यते॒ वि वि मु॑च्यते॒ यत् । \newline
9. मु॒च्य॒ते॒ यद् यन् मु॑च्यते मुच्यते॒ यदु॒खोखा यन् मु॑च्यते मुच्यते॒ यदु॒खा । \newline
10. यदु॒खोखा यद् यदु॒खा सा सोखा यद् यदु॒खा सा । \newline
11. उ॒खा सा सोखोखा सा वै वै सोखोखा सा वै । \newline
12. सा वै वै सा सा वा ए॒षैषा वै सा सा वा ए॒षा । \newline
13. वा ए॒षैषा वै वा ए॒षैतर्‌ ह्ये॒तर्‌ ह्ये॒षा वै वा ए॒षैतर्.हि॑ । \newline
14. ए॒षैतर्‌ ह्ये॒तर् ह्ये॒षै षैतर्.हि॑ या॒तया᳚म्नी या॒तया᳚ म्न्ये॒तर् ह्ये॒षै षैतर्.हि॑ या॒तया᳚म्नी । \newline
15. ए॒तर्.हि॑ या॒तया᳚म्नी या॒तया᳚ म्न्ये॒तर् ह्ये॒तर्.हि॑ या॒तया᳚म्नी॒ सा सा या॒तया᳚ म्न्ये॒तर् ह्ये॒तर्.हि॑ या॒तया᳚म्नी॒ सा । \newline
16. या॒तया᳚म्नी॒ सा सा या॒तया᳚म्नी या॒तया᳚म्नी॒ सा न न सा या॒तया᳚म्नी या॒तया᳚म्नी॒ सा न । \newline
17. या॒तया॒म्नीति॑ या॒त - या॒म्नी॒ । \newline
18. सा न न सा सा न पुनः॒ पुन॒र् न सा सा न पुनः॑ । \newline
19. न पुनः॒ पुन॒र् न न पुनः॑ प्र॒युज्या᳚ प्र॒युज्या॒ पुन॒र् न न पुनः॑ प्र॒युज्या᳚ । \newline
20. पुनः॑ प्र॒युज्या᳚ प्र॒युज्या॒ पुनः॒ पुनः॑ प्र॒युज्येतीति॑ प्र॒युज्या॒ पुनः॒ पुनः॑ प्र॒युज्येति॑ । \newline
21. प्र॒युज्येतीति॑ प्र॒युज्या᳚ प्र॒युज्ये त्या॑हु राहु॒रिति॑ प्र॒युज्या᳚ प्र॒युज्ये त्या॑हुः । \newline
22. प्र॒युज्येति॑ प्र - युज्या᳚ । \newline
23. इत्या॑हु राहु॒ रितीत्या॑हु॒ रग्ने ऽग्न॑ आहु॒ रितीत्या॑हु॒ रग्ने᳚ । \newline
24. आ॒हु॒ रग्ने ऽग्न॑ आहु राहु॒ रग्ने॑ यु॒क्ष्व यु॒क्ष्वाग्न॑ आहु राहु॒ रग्ने॑ यु॒क्ष्व । \newline
25. अग्ने॑ यु॒क्ष्व यु॒क्ष्वाग्ने ऽग्ने॑ यु॒क्ष्वा हि हि यु॒क्ष्वाग्ने ऽग्ने॑ यु॒क्ष्वा हि । \newline
26. यु॒क्ष्वा हि हि यु॒क्ष्व यु॒क्ष्वा हि ये ये हि यु॒क्ष्व यु॒क्ष्वा हि ये । \newline
27. हि ये ये हि हि ये तव॒ तव॒ ये हि हि ये तव॑ । \newline
28. ये तव॒ तव॒ ये ये तव॑ यु॒क्ष्व यु॒क्ष्व तव॒ ये ये तव॑ यु॒क्ष्व । \newline
29. तव॑ यु॒क्ष्व यु॒क्ष्व तव॒ तव॑ यु॒क्ष्वा हि हि यु॒क्ष्व तव॒ तव॑ यु॒क्ष्वा हि । \newline
30. यु॒क्ष्वा हि हि यु॒क्ष्व यु॒क्ष्वा हि दे॑व॒हूत॑मान् देव॒हूत॑मा॒न्॒. हि यु॒क्ष्व यु॒क्ष्वा हि दे॑व॒हूत॑मान् । \newline
31. हि दे॑व॒हूत॑मान् देव॒हूत॑मा॒न्॒. हि हि दे॑व॒हूत॑माꣳ॒॒ इतीति॑ देव॒हूत॑मा॒न्॒. हि हि दे॑व॒हूत॑माꣳ॒॒ इति॑ । \newline
32. दे॒व॒हूत॑माꣳ॒॒ इतीति॑ देव॒हूत॑मान् देव॒हूत॑माꣳ॒॒ इत्यु॒खाया॑ मु॒खाया॒ मिति॑ देव॒हूत॑मान् देव॒हूत॑माꣳ॒॒ इत्यु॒खाया᳚म् । \newline
33. दे॒व॒हूत॑मा॒निति॑ देव - हूत॑मान् । \newline
34. इत्यु॒खाया॑ मु॒खाया॒ मिती त्यु॒खाया᳚म् जुहोति जुहो त्यु॒खाया॒ मिती त्यु॒खाया᳚म् जुहोति । \newline
35. उ॒खाया᳚म् जुहोति जुहो त्यु॒खाया॑ मु॒खाया᳚म् जुहोति॒ तेन॒ तेन॑ जुहो त्यु॒खाया॑ मु॒खाया᳚म् जुहोति॒ तेन॑ । \newline
36. जु॒हो॒ति॒ तेन॒ तेन॑ जुहोति जुहोति॒ तेनै॒ वैव तेन॑ जुहोति जुहोति॒ तेनै॒व । \newline
37. तेनै॒ वैव तेन॒ तेनै॒ वैना॑ मेना मे॒व तेन॒ तेनै॒ वैना᳚म् । \newline
38. ए॒वैना॑ मेना मे॒वै वैना॒म् पुनः॒ पुन॑ रेना मे॒वै वैना॒म् पुनः॑ । \newline
39. ए॒ना॒म् पुनः॒ पुन॑ रेना मेना॒म् पुनः॒ प्र प्र पुन॑ रेना मेना॒म् पुनः॒ प्र । \newline
40. पुनः॒ प्र प्र पुनः॒ पुनः॒ प्र यु॑ङ्क्ते युङ्क्ते॒ प्र पुनः॒ पुनः॒ प्र यु॑ङ्क्ते । \newline
41. प्र यु॑ङ्क्ते युङ्क्ते॒ प्र प्र यु॑ङ्क्ते॒ तेन॒ तेन॑ युङ्क्ते॒ प्र प्र यु॑ङ्क्ते॒ तेन॑ । \newline
42. यु॒ङ्क्ते॒ तेन॒ तेन॑ युङ्क्ते युङ्क्ते॒ तेनाया॑तया॒ म्न्यया॑तया॒म्नी तेन॑ युङ्क्ते युङ्क्ते॒ तेनाया॑तया॒म्नी । \newline
43. तेनाया॑तया॒ म्न्यया॑तया॒म्नी तेन॒ तेनाया॑तया॒म्नी यो यो ऽया॑तया॒म्नी तेन॒ तेनाया॑तया॒म्नी यः । \newline
44. अया॑तया॒म्नी यो यो ऽया॑तया॒ म्न्यया॑तया॒म्नी यो वै वै यो ऽया॑तया॒ म्न्यया॑तया॒म्नी यो वै । \newline
45. अया॑तया॒म्नीत्यया॑त - या॒म्नी॒ । \newline
46. यो वै वै यो यो वा अ॒ग्नि म॒ग्निं ॅवै यो यो वा अ॒ग्निम् । \newline
47. वा अ॒ग्नि म॒ग्निं ॅवै वा अ॒ग्निं ॅयोगे॒ योगे॒ ऽग्निं ॅवै वा अ॒ग्निं ॅयोगे᳚ । \newline
48. अ॒ग्निं ॅयोगे॒ योगे॒ ऽग्नि म॒ग्निं ॅयोग॒ आग॑त॒ आग॑ते॒ योगे॒ ऽग्नि म॒ग्निं ॅयोग॒ आग॑ते । \newline
49. योग॒ आग॑त॒ आग॑ते॒ योगे॒ योग॒ आग॑ते यु॒नक्ति॑ यु॒नक्त्याग॑ते॒ योगे॒ योग॒ आग॑ते यु॒नक्ति॑ । \newline
50. आग॑ते यु॒नक्ति॑ यु॒नक्त्याग॑त॒ आग॑ते यु॒नक्ति॑ यु॒ङ्क्ते यु॒ङ्क्ते यु॒नक्त्याग॑त॒ आग॑ते यु॒नक्ति॑ यु॒ङ्क्ते । \newline
51. आग॑त॒ इत्या - ग॒ते॒ । \newline
52. यु॒नक्ति॑ यु॒ङ्क्ते यु॒ङ्क्ते यु॒नक्ति॑ यु॒नक्ति॑ यु॒ङ्क्ते यु॑ञ्जा॒नेषु॑ युञ्जा॒नेषु॑ यु॒ङ्क्ते यु॒नक्ति॑ यु॒नक्ति॑ यु॒ङ्क्ते यु॑ञ्जा॒नेषु॑ । \newline
53. यु॒ङ्क्ते यु॑ञ्जा॒नेषु॑ युञ्जा॒नेषु॑ यु॒ङ्क्ते यु॒ङ्क्ते यु॑ञ्जा॒ने ष्वग्ने ऽग्ने॑ युञ्जा॒नेषु॑ यु॒ङ्क्ते यु॒ङ्क्ते यु॑ञ्जा॒ने ष्वग्ने᳚ । \newline
54. यु॒ञ्जा॒ने ष्वग्ने ऽग्ने॑ युञ्जा॒नेषु॑ युञ्जा॒ने ष्वग्ने॑ यु॒क्ष्व यु॒क्ष्वाग्ने॑ युञ्जा॒नेषु॑ युञ्जा॒ने ष्वग्ने॑ यु॒क्ष्व । \newline
55. अग्ने॑ यु॒क्ष्व यु॒क्ष्वाग्ने ऽग्ने॑ यु॒क्ष्वा हि हि यु॒क्ष्वाग्ने ऽग्ने॑ यु॒क्ष्वा हि । \newline
\pagebreak
\markright{ TS 5.5.3.2  \hfill https://www.vedavms.in \hfill}

\section{ TS 5.5.3.2 }

\textbf{TS 5.5.3.2 } \newline
\textbf{Samhita Paata} \newline

यु॒क्ष्वा हि ये तव॑ यु॒क्ष्वा हि दे॑व॒हूत॑माꣳ॒॒ इत्या॑है॒ष वा अ॒ग्नेर्योग॒स्तेनै॒वैनं॑ ॅयुनक्ति यु॒ङ्क्ते यु॑ञ्जा॒नेषु॑ ब्रह्मवा॒दिनो॑ वदन्ति॒ न्य॑ङ्ङ॒ग्निश्चे॑त॒व्या(3) उ॑त्ता॒ना(3) इति॒ वय॑सां॒ ॅवा ए॒ष प्र॑ति॒मया॑ चीयते॒ यद॒ग्निर्यन्न्य॑ञ्चं चिनु॒यात् पृ॑ष्टि॒त ए॑न॒माहु॑तय ऋच्छेयु॒र्यदु॑त्ता॒नं न पति॑तुꣳ शक्नुया॒दसु॑वर्ग्योऽस्य स्यात् प्रा॒चीन॑मुत्ता॒नं - [  ] \newline

\textbf{Pada Paata} \newline

यु॒क्ष्व । हि । ये । तव॑ । यु॒क्ष्व । हि । दे॒व॒हूत॑मा॒निति॑ देव - हूत॑मान् । इति॑ । आ॒ह॒ । ए॒षः । वै । अ॒ग्नेः । योगः॑ । तेन॑ । ए॒व । ए॒न॒म् । यु॒न॒क्ति॒ । यु॒ङ्क्ते । यु॒ञ्जा॒नेषु॑ । ब्र॒ह्म॒वा॒दिन॒ इति॑ ब्रह्म - वा॒दिनः॑ । व॒द॒न्ति॒ । न्यङ्॑ । अ॒ग्निः । चे॒त॒व्या(3)ः । उ॒त्ता॒ना(3) इत्यु॑त् - ता॒ना(3)ः । इति॑ । वय॑साम् । वै । ए॒षः । प्र॒ति॒मयेति॑ प्रति - मया᳚ । ची॒य॒ते॒ । यत् । अ॒ग्निः । यत् । न्य॑ञ्चम् । चि॒नु॒यात् । पृ॒ष्टि॒तः । ए॒न॒म् । आहु॑तय॒ इत्या - हु॒त॒यः॒ । ऋ॒च्छे॒युः॒ । यत् । उ॒त्ता॒नमित्यु॑त्-ता॒नम् । न । पति॑तुम् । श॒क्नु॒या॒त् । असु॑वर्ग्य॒ इत्यसु॑वः-ग्यः॒ । अ॒स्य॒ । स्या॒त् । प्रा॒चीन᳚म् । उ॒त्ता॒नमित्यु॑त्-ता॒नम् ।  \newline


\textbf{Krama Paata} \newline

यु॒क्ष्वा हि । हि ये । ये तव॑ । तव॑ यु॒क्ष्व । यु॒क्ष्वा हि । हि दे॑व॒हूत॑मान् । दे॒व॒हूत॑माꣳ॒॒ इति॑ । दे॒व॒हूत॑मा॒निति॑ देव - हूत॑मान् । इत्या॑ह । आ॒है॒षः । ए॒ष वै । वा अ॒ग्नेः । अ॒ग्नेर् योगः॑ । योग॒स्तेन॑ । तेनै॒व । ए॒वैन᳚म् । ए॒न॒म् ॅयु॒न॒क्ति॒ । यु॒न॒क्ति॒ यु॒ङ्क्ते । यु॒ङ्क्ते यु॑ञ्जा॒नेषु॑ । यु॒ञ्जा॒नेषु॑ ब्रह्मवा॒दिनः॑ । ब्र॒ह्म॒वा॒दिनो॑ वदन्ति । ब्र॒ह्म॒वा॒दिन॒ इति॑ ब्रह्म - वा॒दिनः॑ । व॒द॒न्ति॒ न्यङ्ङ्॑ । न्य॑ङ्ङ॒ग्निः । अ॒ग्निश्चे॑त॒व्या(3)ः । चे॒त॒व्या(3) उ॑त्ता॒ना(3)ः । उ॒त्ता॒ना(3) इति॑ । उ॒त्ता॒ना(3) इत्यु॑त् - ता॒ना(3)ः । इति॒ वय॑साम् । वय॑सा॒म् ॅवै । वा ए॒षः । ए॒ष प्र॑ति॒मया᳚ । प्र॒ति॒मया॑ चीयते । प्र॒ति॒मयेति॑ प्रति - मया᳚ । ची॒य॒ते॒ यत् । यद॒ग्निः । अ॒ग्निर् यत् । यन् न्य॑ञ्चम् । न्य॑ञ्चम् चिनु॒यात् । चि॒नु॒यात् पृ॑ष्टि॒तः । पृ॒ष्टि॒त ए॑नम् । ए॒न॒माहु॑तयः । आहु॑तय ऋच्छेयुः । आहु॑तय॒ इत्या - हु॒त॒यः॒ । ऋ॒च्छे॒यु॒र् यत् । यदु॑त्ता॒नम् । उ॒त्ता॒नम् न । उ॒त्ता॒नमित्यु॑त् - ता॒नम् । न पति॑तुम् । पति॑तुꣳ शक्नुयात् । श॒क्नु॒या॒दसु॑वर्ग्यः । असु॑वर्ग्योऽस्य । असु॑वर्ग्य॒ इत्यसु॑वः - ग्यः॒ । अ॒स्य॒ स्या॒त्॒ । स्या॒त् प्रा॒चीन᳚म् । प्रा॒चीन॑मुत्ता॒नम् । उ॒त्ता॒नम् पु॑रुषशी॒र्॒.षम् । उ॒त्ता॒नमित्यु॑त् - ता॒नम् \newline

\textbf{Jatai Paata} \newline

1. यु॒क्ष्वा हि हि यु॒क्ष्व यु॒क्ष्वा हि । \newline
2. हि ये ये हि हि ये । \newline
3. ये तव॒ तव॒ ये ये तव॑ । \newline
4. तव॑ यु॒क्ष्व यु॒क्ष्व तव॒ तव॑ यु॒क्ष्व । \newline
5. यु॒क्ष्वा हि हि यु॒क्ष्व यु॒क्ष्वा हि । \newline
6. हि दे॑व॒हूत॑मान् देव॒हूत॑मा॒न्॒. हि हि दे॑व॒हूत॑मान् । \newline
7. दे॒व॒हूत॑माꣳ॒॒ इतीति॑ देव॒हूत॑मान् देव॒हूत॑माꣳ॒॒ इति॑ । \newline
8. दे॒व॒हूत॑मा॒निति॑ देव - हूत॑मान् । \newline
9. इत्या॑हा॒हे तीत्या॑ह । \newline
10. आ॒है॒ष ए॒ष आ॑हा है॒षः । \newline
11. ए॒ष वै वा ए॒ष ए॒ष वै । \newline
12. वा अ॒ग्ने र॒ग्नेर् वै वा अ॒ग्नेः । \newline
13. अ॒ग्नेर् योगो॒ योगो॒ ऽग्ने र॒ग्नेर् योगः॑ । \newline
14. योग॒ स्तेन॒ तेन॒ योगो॒ योग॒ स्तेन॑ । \newline
15. तेनै॒ वैव तेन॒ तेनै॒व । \newline
16. ए॒वैन॑ मेन मे॒वै वैन᳚म् । \newline
17. ए॒नं॒ ॅयु॒न॒क्ति॒ यु॒न॒क्त्ये॒न॒ मे॒नं॒ ॅयु॒न॒क्ति॒ । \newline
18. यु॒न॒क्ति॒ यु॒ङ्क्ते यु॒ङ्क्ते यु॑नक्ति युनक्ति यु॒ङ्क्ते । \newline
19. यु॒ङ्क्ते यु॑ञ्जा॒नेषु॑ युञ्जा॒नेषु॑ यु॒ङ्क्ते यु॒ङ्क्ते यु॑ञ्जा॒नेषु॑ । \newline
20. यु॒ञ्जा॒नेषु॑ ब्रह्मवा॒दिनो᳚ ब्रह्मवा॒दिनो॑ युञ्जा॒नेषु॑ युञ्जा॒नेषु॑ ब्रह्मवा॒दिनः॑ । \newline
21. ब्र॒ह्म॒वा॒दिनो॑ वदन्ति वदन्ति ब्रह्मवा॒दिनो᳚ ब्रह्मवा॒दिनो॑ वदन्ति । \newline
22. ब्र॒ह्म॒वा॒दिन॒ इति॑ ब्रह्म - वा॒दिनः॑ । \newline
23. व॒द॒न्ति॒ न्या᳚(1॒)ङ् न्य॑ङ्. वदन्ति वदन्ति॒ न्यङ्॑ । \newline
24. न्य॑ङ् ङ॒ग्नि र॒ग्निर् न्या᳚(1॒)ङ् न्य॑ङ् ङ॒ग्निः । \newline
25. अ॒ग्नि श्चे॑त॒व्या(3) श्चे॑त॒व्या(3) अ॒ग्नि र॒ग्नि श्चे॑त॒व्या(3)ः । \newline
26. चे॒त॒व्या(3) उ॑त्ता॒ना(3) उ॑त्ता॒ना(3) श्चे॑त॒व्या(3) श्चे॑त॒व्या(3) उ॑त्ता॒ना(3)ः । \newline
27. उ॒त्ता॒ना(3) इतीत्यु॑त्ता॒ना(3) उ॑त्ता॒ना(3) इति॑ । \newline
28. उ॒त्ता॒ना(3) इत्यु॑त् - ता॒ना(3)ः । \newline
29. इति॒ वय॑सां॒ ॅवय॑सा॒ मितीति॒ वय॑साम् । \newline
30. वय॑सां॒ ॅवै वै वय॑सां॒ ॅवय॑सां॒ ॅवै । \newline
31. वा ए॒ष ए॒ष वै वा ए॒षः । \newline
32. ए॒ष प्र॑ति॒मया᳚ प्रति॒मयै॒ष ए॒ष प्र॑ति॒मया᳚ । \newline
33. प्र॒ति॒मया॑ चीयते चीयते प्रति॒मया᳚ प्रति॒मया॑ चीयते । \newline
34. प्र॒ति॒मयेति॑ प्रति - मया᳚ । \newline
35. ची॒य॒ते॒ यद् यच् ची॑यते चीयते॒ यत् । \newline
36. यद॒ग्नि र॒ग्निर् यद् यद॒ग्निः । \newline
37. अ॒ग्निर् यद् यद॒ग्नि र॒ग्निर् यत् । \newline
38. यन् न्य॑ञ्च॒म् न्य॑ञ्चं॒ ॅयद् यन् न्य॑ञ्चम् । \newline
39. न्य॑ञ्चम् चिनु॒याच् चि॑नु॒यान् न्य॑ञ्च॒म् न्य॑ञ्चम् चिनु॒यात् । \newline
40. चि॒नु॒यात् पृ॑ष्टि॒तः पृ॑ष्टि॒त श्चि॑नु॒याच् चि॑नु॒यात् पृ॑ष्टि॒तः । \newline
41. पृ॒ष्टि॒त ए॑न मेनम् पृष्टि॒तः पृ॑ष्टि॒त ए॑नम् । \newline
42. ए॒न॒ माहु॑तय॒ आहु॑तय एन मेन॒ माहु॑तयः । \newline
43. आहु॑तय ऋच्छेयुर्. ऋच्छेयु॒ राहु॑तय॒ आहु॑तय ऋच्छेयुः । \newline
44. आहु॑तय॒ इत्या - हु॒त॒यः॒ । \newline
45. ऋ॒च्छे॒यु॒र् यद् यदृ॑च्छेयुर्. ऋच्छेयु॒र् यत् । \newline
46. यदु॑त्ता॒न मु॑त्ता॒नं ॅयद् यदु॑त्ता॒नम् । \newline
47. उ॒त्ता॒नम् न नोत्ता॒न मु॑त्ता॒नम् न । \newline
48. उ॒त्ता॒नमित्यु॑त् - ता॒नम् । \newline
49. न पति॑तु॒म् पति॑तु॒न् न न पति॑तुम् । \newline
50. पति॑तुꣳ शक्नुया च्छक्नुया॒त् पति॑तु॒म् पति॑तुꣳ शक्नुयात् । \newline
51. श॒क्नु॒या॒ दसु॑व॒र्ग्यो ऽसु॑वर्ग्यः शक्नुया च्छक्नुया॒ दसु॑वर्ग्यः । \newline
52. असु॑वर्ग्यो ऽस्या॒स्या सु॑व॒र्ग्यो ऽसु॑वर्ग्यो ऽस्य । \newline
53. असु॑वर्ग्य॒ इत्यसु॑वः - ग्यः॒ । \newline
54. अ॒स्य॒ स्या॒थ् स्या॒द॒ स्या॒स्य॒ स्या॒त् । \newline
55. स्या॒त् प्रा॒चीन॑म् प्रा॒चीनꣳ॑ स्याथ् स्यात् प्रा॒चीन᳚म् । \newline
56. प्रा॒चीन॑ मुत्ता॒न मु॑त्ता॒नम् प्रा॒चीन॑म् प्रा॒चीन॑ मुत्ता॒नम् । \newline
57. उ॒त्ता॒नम् पु॑रुषशी॒र्॒.षम् पु॑रुषशी॒र्॒.ष मु॑त्ता॒न मु॑त्ता॒नम् पु॑रुषशी॒र्॒.षम् । \newline
58. उ॒त्ता॒नमित्यु॑त् - ता॒नम् । \newline

\textbf{Ghana Paata } \newline

1. यु॒क्ष्वा हि हि यु॒क्ष्व यु॒क्ष्वा हि ये ये हि यु॒क्ष्व यु॒क्ष्वा हि ये । \newline
2. हि ये ये हि हि ये तव॒ तव॒ ये हि हि ये तव॑ । \newline
3. ये तव॒ तव॒ ये ये तव॑ यु॒क्ष्व यु॒क्ष्व तव॒ ये ये तव॑ यु॒क्ष्व । \newline
4. तव॑ यु॒क्ष्व यु॒क्ष्व तव॒ तव॑ यु॒क्ष्वा हि हि यु॒क्ष्व तव॒ तव॑ यु॒क्ष्वा हि । \newline
5. यु॒क्ष्वा हि हि यु॒क्ष्व यु॒क्ष्वा हि दे॑व॒हूत॑मान् देव॒हूत॑मा॒न्॒. हि यु॒क्ष्व यु॒क्ष्वा हि दे॑व॒हूत॑मान् । \newline
6. हि दे॑व॒हूत॑मान् देव॒हूत॑मा॒न्॒. हि हि दे॑व॒हूत॑माꣳ॒॒ इतीति॑ देव॒हूत॑मा॒न्॒. हि हि दे॑व॒हूत॑माꣳ॒॒ इति॑ । \newline
7. दे॒व॒हूत॑माꣳ॒॒ इतीति॑ देव॒हूत॑मान् देव॒हूत॑माꣳ॒॒ इत्या॑हा॒हेति॑ देव॒हूत॑मान् देव॒हूत॑माꣳ॒॒ इत्या॑ह । \newline
8. दे॒व॒हूत॑मा॒निति॑ देव - हूत॑मान् । \newline
9. इत्या॑हा॒हे तीत्या॑ है॒ष ए॒ष आ॒हे तीत्या॑ है॒षः । \newline
10. आ॒है॒ष ए॒ष आ॑हाहै॒ष वै वा ए॒ष आ॑हाहै॒ष वै । \newline
11. ए॒ष वै वा ए॒ष ए॒ष वा अ॒ग्ने र॒ग्नेर् वा ए॒ष ए॒ष वा अ॒ग्नेः । \newline
12. वा अ॒ग्ने र॒ग्नेर् वै वा अ॒ग्नेर् योगो॒ योगो॒ ऽग्नेर् वै वा अ॒ग्नेर् योगः॑ । \newline
13. अ॒ग्नेर् योगो॒ योगो॒ ऽग्ने र॒ग्नेर् योग॒ स्तेन॒ तेन॒ योगो॒ ऽग्ने र॒ग्नेर् योग॒ स्तेन॑ । \newline
14. Yओग॒ स्तेन॒ तेन॒ योगो॒ योग॒ स्तेनै॒ वैव तेन॒ योगो॒ योग॒ स्तेनै॒व । \newline
15. तेनै॒वैव तेन॒ तेनै॒ वैन॑ मेन मे॒व तेन॒ तेनै॒ वैन᳚म् । \newline
16. ए॒वैन॑ मेन मे॒वै वैनं॑ ॅयुनक्ति युनक्त्येन मे॒वै वैनं॑ ॅयुनक्ति । \newline
17. ए॒नं॒ ॅयु॒न॒क्ति॒ यु॒न॒क्त्ये॒न॒ मे॒नं॒ ॅयु॒न॒क्ति॒ यु॒ङ्क्ते यु॒ङ्क्ते यु॑नक्त्येन मेनं ॅयुनक्ति यु॒ङ्क्ते । \newline
18. यु॒न॒क्ति॒ यु॒ङ्क्ते यु॒ङ्क्ते यु॑नक्ति युनक्ति यु॒ङ्क्ते यु॑ञ्जा॒नेषु॑ युञ्जा॒नेषु॑ यु॒ङ्क्ते यु॑नक्ति युनक्ति यु॒ङ्क्ते यु॑ञ्जा॒नेषु॑ । \newline
19. यु॒ङ्क्ते यु॑ञ्जा॒नेषु॑ युञ्जा॒नेषु॑ यु॒ङ्क्ते यु॒ङ्क्ते यु॑ञ्जा॒नेषु॑ ब्रह्मवा॒दिनो᳚ ब्रह्मवा॒दिनो॑ युञ्जा॒नेषु॑ यु॒ङ्क्ते यु॒ङ्क्ते यु॑ञ्जा॒नेषु॑ ब्रह्मवा॒दिनः॑ । \newline
20. यु॒ञ्जा॒नेषु॑ ब्रह्मवा॒दिनो᳚ ब्रह्मवा॒दिनो॑ युञ्जा॒नेषु॑ युञ्जा॒नेषु॑ ब्रह्मवा॒दिनो॑ वदन्ति वदन्ति ब्रह्मवा॒दिनो॑ युञ्जा॒नेषु॑ युञ्जा॒नेषु॑ ब्रह्मवा॒दिनो॑ वदन्ति । \newline
21. ब्र॒ह्म॒वा॒दिनो॑ वदन्ति वदन्ति ब्रह्मवा॒दिनो᳚ ब्रह्मवा॒दिनो॑ वदन्ति॒ न्या᳚(1॒)ङ् न्य॑ङ्. वदन्ति ब्रह्मवा॒दिनो᳚ ब्रह्मवा॒दिनो॑ वदन्ति॒ न्यङ्॑ । \newline
22. ब्र॒ह्म॒वा॒दिन॒ इति॑ ब्रह्म - वा॒दिनः॑ । \newline
23. व॒द॒न्ति॒ न्या᳚(1॒)ङ् न्य॑ङ्. वदन्ति वदन्ति॒ न्य॑ङ् ङ॒ग्नि र॒ग्निर् न्य॑ङ्. वदन्ति वदन्ति॒ न्य॑ङ् ङ॒ग्निः । \newline
24. न्य॑ङ् ङ॒ग्नि र॒ग्निर् न्या᳚(1॒)ङ् न्य॑ङ् ङ॒ग्नि श्चे॑त॒व्या(3) श्चे॑त॒व्या(3) अ॒ग्निर् न्या᳚(1॒)ङ् न्य॑ङ् ङ॒ग्नि श्चे॑त॒व्या(3)ः । \newline
25. अ॒ग्नि श्चे॑त॒व्या(3) श्चे॑त॒व्या(3) अ॒ग्नि र॒ग्नि श्चे॑त॒व्या(3) उ॑त्ता॒ना(3) उ॑त्ता॒ना(3) श्चे॑त॒व्या(3) अ॒ग्नि र॒ग्नि श्चे॑त॒व्या(3) उ॑त्ता॒ना(3)ः । \newline
26. चे॒त॒व्या(3) उ॑त्ता॒ना(3) उ॑त्ता॒ना(3) श्चे॑त॒व्या(3) श्चे॑त॒व्या(3) उ॑त्ता॒ना(3) इती त्यु॑त्ता॒ना(3) श्चे॑त॒व्या(3) श्चे॑त॒व्या(3) उ॑त्ता॒ना(3) इति॑ । \newline
27. उ॒त्ता॒ना(3) इती त्यु॑त्ता॒ना(3) उ॑त्ता॒ना(3) इति॒ वय॑सां॒ ॅवय॑सा॒ मित्यु॑त्ता॒ना(3) उ॑त्ता॒ना(3) इति॒ वय॑साम् । \newline
28. उ॒त्ता॒ना(3) इत्यु॑त् - ता॒ना(3)ः । \newline
29. इति॒ वय॑सां॒ ॅवय॑सा॒ मितीति॒ वय॑सां॒ ॅवै वै वय॑सा॒ मितीति॒ वय॑सां॒ ॅवै । \newline
30. वय॑सां॒ ॅवै वै वय॑सां॒ ॅवय॑सां॒ ॅवा ए॒ष ए॒ष वै वय॑सां॒ ॅवय॑सां॒ ॅवा ए॒षः । \newline
31. वा ए॒ष ए॒ष वै वा ए॒ष प्र॑ति॒मया᳚ प्रति॒मयै॒ष वै वा ए॒ष प्र॑ति॒मया᳚ । \newline
32. ए॒ष प्र॑ति॒मया᳚ प्रति॒मयै॒ष ए॒ष प्र॑ति॒मया॑ चीयते चीयते प्रति॒मयै॒ष ए॒ष प्र॑ति॒मया॑ चीयते । \newline
33. प्र॒ति॒मया॑ चीयते चीयते प्रति॒मया᳚ प्रति॒मया॑ चीयते॒ यद् यच् ची॑यते प्रति॒मया᳚ प्रति॒मया॑ चीयते॒ यत् । \newline
34. प्र॒ति॒मयेति॑ प्रति - मया᳚ । \newline
35. ची॒य॒ते॒ यद् यच् ची॑यते चीयते॒ यद॒ग्नि र॒ग्निर् यच् ची॑यते चीयते॒ यद॒ग्निः । \newline
36. यद॒ग्नि र॒ग्निर् यद् यद॒ग्निर् यद् यद॒ग्निर् यद् यद॒ग्निर् यत् । \newline
37. अ॒ग्निर् यद् यद॒ग्नि र॒ग्निर् यन् न्य॑ञ्च॒म् न्य॑ञ्चं॒ ॅयद॒ग्नि र॒ग्निर् यन् न्य॑ञ्चम् । \newline
38. यन् न्य॑ञ्च॒म् न्य॑ञ्चं॒ ॅयद् यन् न्य॑ञ्चम् चिनु॒याच् चि॑नु॒यान् न्य॑ञ्चं॒ ॅयद् यन् न्य॑ञ्चम् चिनु॒यात् । \newline
39. न्य॑ञ्चम् चिनु॒याच् चि॑नु॒यान् न्य॑ञ्च॒न् न्य॑ञ्चम् चिनु॒यात् पृ॑ष्टि॒तः पृ॑ष्टि॒त श्चि॑नु॒यान् न्य॑ञ्च॒न् न्य॑ञ्चम् चिनु॒यात् पृ॑ष्टि॒तः । \newline
40. चि॒नु॒यात् पृ॑ष्टि॒तः पृ॑ष्टि॒त श्चि॑नु॒याच् चि॑नु॒यात् पृ॑ष्टि॒त ए॑न मेनम् पृष्टि॒त श्चि॑नु॒याच् चि॑नु॒यात् पृ॑ष्टि॒त ए॑नम् । \newline
41. पृ॒ष्टि॒त ए॑न मेनम् पृष्टि॒तः पृ॑ष्टि॒त ए॑न॒ माहु॑तय॒ आहु॑तय एनम् पृष्टि॒तः पृ॑ष्टि॒त ए॑न॒ माहु॑तयः । \newline
42. ए॒न॒ माहु॑तय॒ आहु॑तय एन मेन॒ माहु॑तय ऋच्छेयुर्. ऋच्छेयु॒ राहु॑तय एन मेन॒ माहु॑तय ऋच्छेयुः । \newline
43. आहु॑तय ऋच्छेयुर्. ऋच्छेयु॒ राहु॑तय॒ आहु॑तय ऋच्छेयु॒र् यद् यदृ॑च्छेयु॒ राहु॑तय॒ आहु॑तय ऋच्छेयु॒र् यत् । \newline
44. आहु॑तय॒ इत्या - हु॒त॒यः॒ । \newline
45. ऋ॒च्छे॒यु॒र् यद् यदृ॑च्छेयुर्. ऋच्छेयु॒र् यदु॑त्ता॒न मु॑त्ता॒नं ॅयदृ॑च्छेयुर्. ऋच्छेयु॒र् यदु॑त्ता॒नम् । \newline
46. यदु॑त्ता॒न मु॑त्ता॒नं ॅयद् यदु॑त्ता॒नम् न नोत्ता॒नं ॅयद् यदु॑त्ता॒नम् न । \newline
47. उ॒त्ता॒नम् न नोत्ता॒न मु॑त्ता॒नम् न पति॑तु॒म् पति॑तु॒म् नोत्ता॒न मु॑त्ता॒नम् न पति॑तुम् । \newline
48. उ॒त्ता॒नमित्यु॑त् - ता॒नम् । \newline
49. न पति॑तु॒म् पति॑तु॒म् न न पति॑तुꣳ शक्नुया च्छक्नुया॒त् पति॑तु॒म् न न पति॑तुꣳ शक्नुयात् । \newline
50. पति॑तुꣳ शक्नुया च्छक्नुया॒त् पति॑तु॒म् पति॑तुꣳ शक्नुया॒ दसु॑व॒र्ग्यो ऽसु॑वर्ग्यः शक्नुया॒त् पति॑तु॒म् पति॑तुꣳ शक्नुया॒ दसु॑वर्ग्यः । \newline
51. श॒क्नु॒या॒ दसु॑व॒र्ग्यो ऽसु॑वर्ग्यः शक्नुया च्छक्नुया॒ दसु॑वर्ग्यो ऽस्या॒स्या सु॑वर्ग्यः शक्नुया च्छक्नुया॒ दसु॑वर्ग्यो ऽस्य । \newline
52. असु॑वर्ग्यो ऽस्या॒ स्या सु॑व॒र्ग्यो ऽसु॑वर्ग्यो ऽस्य स्याथ् स्या द॒स्या सु॑व॒र्ग्यो ऽसु॑वर्ग्यो ऽस्य स्यात् । \newline
53. असु॑वर्ग्य॒ इत्यसु॑वः - ग्यः॒ । \newline
54. अ॒स्य॒ स्या॒थ् स्या॒ द॒स्या॒ स्य॒ स्या॒त् प्रा॒चीन॑म् प्रा॒चीनꣳ॑ स्या दस्यास्य स्यात् प्रा॒चीन᳚म् । \newline
55. स्या॒त् प्रा॒चीन॑म् प्रा॒चीनꣳ॑ स्याथ् स्यात् प्रा॒चीन॑ मुत्ता॒न मु॑त्ता॒नम् प्रा॒चीनꣳ॑ स्याथ् स्यात् प्रा॒चीन॑ मुत्ता॒नम् । \newline
56. प्रा॒चीन॑ मुत्ता॒न मु॑त्ता॒नम् प्रा॒चीन॑म् प्रा॒चीन॑ मुत्ता॒नम् पु॑रुषशी॒र्॒.षम् पु॑रुषशी॒र्॒.ष मु॑त्ता॒नम् प्रा॒चीन॑म् प्रा॒चीन॑ मुत्ता॒नम् पु॑रुषशी॒र्॒.षम् । \newline
57. उ॒त्ता॒नम् पु॑रुषशी॒र्॒.षम् पु॑रुषशी॒र्॒.ष मु॑त्ता॒न मु॑त्ता॒नम् पु॑रुषशी॒र्॒.ष मुपोप॑ पुरुषशी॒र्॒.ष मु॑त्ता॒न मु॑त्ता॒नम् पु॑रुषशी॒र्॒.ष मुप॑ । \newline
58. उ॒त्ता॒नमित्यु॑त् - ता॒नम् । \newline
\pagebreak
\markright{ TS 5.5.3.3  \hfill https://www.vedavms.in \hfill}

\section{ TS 5.5.3.3 }

\textbf{TS 5.5.3.3 } \newline
\textbf{Samhita Paata} \newline

पु॑रुषशी॒र्॒.षमुप॑ दधाति मुख॒त ए॒वैन॒माहु॑तय ऋच्छन्ति॒ नोत्ता॒नं चि॑नुते सुव॒र्ग्यो᳚ऽस्य भवति सौ॒र्या जु॑होति॒ चक्षु॑रे॒वास्मि॒न् प्रति॑ दधाति॒ द्विर्जु॑होति॒ द्वे हि चक्षु॑षी समा॒न्या जु॑होति समा॒नꣳ हि चक्षुः॒ समृ॑द्ध्यै देवासु॒राः संॅय॑त्ता आस॒न् ते वा॒मं ॅवसु॒ सं न्य॑दधत॒ तद्दे॒वा वा॑म॒भृता॑ऽवृञ्जत॒ तद्वा॑म॒भृतो॑ वामभृ॒त्त्वं ॅयद्वा॑म॒भृत॑ ( ) मुप॒दधा॑ति वा॒ममे॒व तया॒ वसु॒ यज॑मानो॒ भ्रातृ॑व्यस्य वृङ्क्ते॒ हिर॑ण्यमूर्द्ध्नी भवति॒ ज्योति॒र्वै हिर॑ण्यं॒ ज्योति॑र्वा॒मं ज्योति॑षै॒वास्य॒ ज्योति॑र्वा॒मं ॅवृ॑ङ्क्ते द्विय॒जुर्भ॑वति॒ प्रति॑ष्ठित्यै ॥ \newline

\textbf{Pada Paata} \newline

पु॒रु॒ष॒शी॒र्॒.षमिति॑ पुरुष-शी॒र्॒.षम् । उपेति॑ । द॒धा॒ति॒ । मु॒ख॒तः । ए॒व । ए॒न॒म् । आहु॑तय॒ इत्या - हु॒त॒यः॒ । ऋ॒च्छ॒न्ति॒ । न । उ॒त्ता॒नमित्यु॑त्- ता॒नम् । चि॒नु॒ते॒ । सु॒व॒र्ग्य॑ इति॑ सुवः - ग्यः॑ । अ॒स्य॒ । भ॒व॒ति॒ । सौ॒र्या । जु॒हो॒ति॒ । चक्षुः॑ । ए॒व । अ॒स्मि॒न्न् । प्रतीति॑ । द॒धा॒ति॒ । द्विः । जु॒हो॒ति॒ । द्वे इति॑ । हि । चक्षु॑षी॒ इति॑ । स॒मा॒न्या । जु॒हो॒ति॒ । स॒मा॒नम् । हि । चक्षुः॑ । समृ॑द्ध्या॒ इति॒ सं - ऋ॒द्ध्यै॒ । दे॒वा॒सु॒रा इति॑ देव - अ॒सु॒राः । संॅय॑त्ता॒ इति॒ सं - य॒त्ताः॒ । आ॒स॒न्न् । ते । वा॒मम् । वसु॑ । सम् । नीति॑ । अ॒द॒ध॒त॒ । तत् । दे॒वाः । वा॒म॒भृतेति॑ वाम - भृता᳚ । अ॒वृ॒ञ्ज॒त॒ । तत् । वा॒म॒भृत॒ इति॑ वाम - भृतः॑ । वा॒म॒भृ॒त्त्वमिति॑ वामभृत् - त्वम् । यत् । वा॒म॒भृत॒मिति॑ वाम - भृत᳚म् ( ) । उ॒प॒दधा॒तीत्यु॑प - दधा॑ति । वा॒मम् । ए॒व । तया᳚ । वसु॑ । यज॑मानः । भ्रातृ॑व्यस्य । वृ॒ङ्क्ते॒ । हिर॑ण्यमू॒द्‌र्ध्नीति॒ हिर॑ण्य - मू॒द्‌र्ध्नी॒ । भ॒व॒ति॒ । ज्योतिः॑ । वै । हिर॑ण्यम् । ज्योतिः॑ । वा॒मम् । ज्योति॑षा । ए॒व । अ॒स्य॒ । ज्योतिः॑ । वा॒मम् । वृ॒ङ्क्ते॒ । द्वि॒य॒जुरिति॑ द्वि-य॒जुः । भ॒व॒ति॒ । प्रति॑ष्ठित्या॒ इति॒ प्रति॑-स्थि॒त्यै॒ ॥  \newline


\textbf{Krama Paata} \newline

पु॒रु॒ष॒शी॒र्॒.षमुप॑ । पु॒रु॒ष॒शी॒र्॒.षमिति॑ पुरुष - शी॒र्॒.षम् । उप॑ दधाति । द॒धा॒ति॒ मु॒ख॒तः । मु॒ख॒त ए॒व । ए॒वैन᳚म् । ए॒न॒माहु॑तयः । आहु॑तय ऋच्छन्ति । आहु॑तय॒ इत्या - हु॒त॒यः॒ । ऋ॒च्छ॒न्ति॒ न । नोत्ता॒नम् । उ॒त्ता॒नम् चि॑नुते । उ॒त्ता॒नमित्यु॑त् - ता॒नम् । चि॒नु॒ते॒ सु॒व॒र्ग्यः॑ । सु॒व॒र्ग्यो᳚ऽस्य । सु॒व॒र्ग्य॑ इति॑ सुवः - ग्यः॑ । अ॒स्य॒ भ॒व॒ति॒ । भ॒व॒ति॒ सौ॒र्या । सौ॒र्या जु॑होति । जु॒हो॒ति॒ चक्षुः॑ । चक्षु॑रे॒व । ए॒वास्मिन्न्॑ । अ॒स्मि॒न् प्रति॑ । प्रति॑ दधाति । द॒धा॒ति॒ द्विः । द्विर् जु॑होति । जु॒हो॒ति॒ द्वे । द्वे हि । द्वे इति॒ द्वे । हि चक्षु॑षी । चक्षु॑षी समा॒न्या । चक्षु॑षी॒ इति॒ चक्षु॑षी । स॒मा॒न्या जु॑होति । जु॒हो॒ति॒ स॒मा॒नम् । स॒मा॒नꣳ हि । हि चक्षुः॑ । चक्षुः॒ समृ॑द्ध्यै । समृ॑द्ध्यै देवासु॒राः । समृ॑द्ध्या॒ इति॒ सम् - ऋ॒द्ध्यै॒ । दे॒वा॒सु॒राः सम्ॅय॑त्ताः । दे॒वा॒सु॒रा इति॑ देव - अ॒सु॒राः । सम्ॅय॑त्ता आसन्न् । सम्ॅय॑त्ता॒ इति॒ सम् - य॒त्ताः॒ । आ॒स॒न् ते । ते वा॒मम् । वा॒मम् ॅवसु॑ । वसु॒ सम् । सम् नि । न्य॑दधत । अ॒द॒ध॒त॒ तत् । तद् दे॒वाः । दे॒वा वा॑म॒भृता᳚ । वा॒म॒भृता॑ऽवृञ्जत । वा॒म॒भृतेति॑ वाम - भृता᳚ । अ॒वृ॒ञ्ज॒त॒ तत् । तद् वा॑म॒भृतः॑ । वा॒म॒भृतो॑ वामभृ॒त्त्वम् । वा॒म॒भृत॒ इति॑ वाम - भृतः॑ । वा॒म॒भृ॒त्त्वम् ॅयत् । वा॒म॒भृ॒त्त्वमिति॑ वामभृत् - त्वम् । यद् वा॑म॒भृत᳚म् ( ) । वा॒म॒भृत॑मुप॒दधा॑ति । वा॒म॒भृत॒मिति॑ वाम - भृत᳚म् । उ॒प॒दधा॑ति वा॒मम् । उ॒प॒दधा॒तीत्यु॑प - दधा॑ति । वा॒ममे॒व । ए॒व तया᳚ । तया॒ वसु॑ । वसु॒ यज॑मानः । यज॑मानो॒ भ्रातृ॑व्यस्य । भातृ॑व्यस्य वृङ्क्ते । वृ॒ङ्क्ते॒ हिर॑ण्यमूर्द्ध्नी । हिर॑ण्यमूर्द्ध्नी भवति । हिर॑ण्यमू॒र्द्ध्नीति॒ हिर॑ण्य - मू॒र्द्ध्नी॒ । भ॒व॒ति॒ ज्योतिः॑ । ज्योति॒र् वै । वै हिर॑ण्यम् । हिर॑ण्य॒म् ज्योतिः॑ । ज्योति॑र् वा॒मम् । वा॒मम् ज्योति॑षा । ज्योति॑षै॒व । ए॒वास्य॑ । अ॒स्य॒ ज्योतिः॑ । ज्योति॑र् वा॒मम् । वा॒मम् ॅवृ॑ङ्क्ते । वृ॒ङ्क्ते॒ द्वि॒य॒जुः । द्वि॒य॒जुर् भ॑वति । द्वि॒य॒जुरिति॑ द्वि - य॒जुः । भ॒व॒ति॒ प्रति॑ष्ठित्यै । प्रति॑ष्ठित्या॒ इति॒ प्रति॑ - स्थि॒त्यै॒ । \newline

\textbf{Jatai Paata} \newline

1. पु॒रु॒ष॒शी॒र्॒.ष मुपोप॑ पुरुषशी॒र्॒.षम् पु॑रुषशी॒र्॒.ष मुप॑ । \newline
2. पु॒रु॒ष॒शी॒र्॒.षमिति॑ पुरुष - शी॒र्॒.षम् । \newline
3. उप॑ दधाति दधा॒ त्युपोप॑ दधाति । \newline
4. द॒धा॒ति॒ मु॒ख॒तो मु॑ख॒तो द॑धाति दधाति मुख॒तः । \newline
5. मु॒ख॒त ए॒वैव मु॑ख॒तो मु॑ख॒त ए॒व । \newline
6. ए॒वैन॑ मेन मे॒वै वैन᳚म् । \newline
7. ए॒न॒ माहु॑तय॒ आहु॑तय एन मेन॒ माहु॑तयः । \newline
8. आहु॑तय ऋच्छन् त्यृच्छ॒न् त्याहु॑तय॒ आहु॑तय ऋच्छन्ति । \newline
9. आहु॑तय॒ इत्या - हु॒त॒यः॒ । \newline
10. ऋ॒च्छ॒न्ति॒ न न र्च्छ॑न् त्यृच्छन्ति॒ न । \newline
11. नोत्ता॒न मु॑त्ता॒नम् न नोत्ता॒नम् । \newline
12. उ॒त्ता॒नम् चि॑नुते चिनुत उत्ता॒न मु॑त्ता॒नम् चि॑नुते । \newline
13. उ॒त्ता॒नमित्यु॑त् - ता॒नम् । \newline
14. चि॒नु॒ते॒ सु॒व॒र्ग्यः॑ सुव॒र्ग्य॑ श्चिनुते चिनुते सुव॒र्ग्यः॑ । \newline
15. सु॒व॒र्ग्यो᳚ ऽस्यास्य सुव॒र्ग्यः॑ सुव॒र्ग्यो᳚ ऽस्य । \newline
16. सु॒व॒र्ग्य॑ इति॑ सुवः - ग्यः॑ । \newline
17. अ॒स्य॒ भ॒व॒ति॒ भ॒व॒ त्य॒स्या॒स्य॒ भ॒व॒ति॒ । \newline
18. भ॒व॒ति॒ सौ॒र्या सौ॒र्या भ॑वति भवति सौ॒र्या । \newline
19. सौ॒र्या जु॑होति जुहोति सौ॒र्या सौ॒र्या जु॑होति । \newline
20. जु॒हो॒ति॒ चक्षु॒ श्चक्षु॑र् जुहोति जुहोति॒ चक्षुः॑ । \newline
21. चक्षु॑ रे॒वैव चक्षु॒ श्चक्षु॑ रे॒व । \newline
22. ए॒वास्मि॑न् नस्मिन् ने॒वै वास्मिन्न्॑ । \newline
23. अ॒स्मि॒न् प्रति॒ प्रत्य॑स्मिन् नस्मि॒न् प्रति॑ । \newline
24. प्रति॑ दधाति दधाति॒ प्रति॒ प्रति॑ दधाति । \newline
25. द॒धा॒ति॒ द्विर् द्विर् द॑धाति दधाति॒ द्विः । \newline
26. द्विर् जु॑होति जुहोति॒ द्विर् द्विर् जु॑होति । \newline
27. जु॒हो॒ति॒ द्वे द्वे जु॑होति जुहोति॒ द्वे । \newline
28. द्वे हि हि द्वे द्वे हि । \newline
29. द्वे इति॒ द्वे । \newline
30. हि चक्षु॑षी॒ चक्षु॑षी॒ हि हि चक्षु॑षी । \newline
31. चक्षु॑षी समा॒न्या स॑मा॒न्या चक्षु॑षी॒ चक्षु॑षी समा॒न्या । \newline
32. चक्षु॑षी॒ इति॒ चक्षु॑षी । \newline
33. स॒मा॒न्या जु॑होति जुहोति समा॒न्या स॑मा॒न्या जु॑होति । \newline
34. जु॒हो॒ति॒ स॒मा॒नꣳ स॑मा॒नम् जु॑होति जुहोति समा॒नम् । \newline
35. स॒मा॒नꣳ हि हि स॑मा॒नꣳ स॑मा॒नꣳ हि । \newline
36. हि चक्षु॒ श्चक्षु॒र्॒. हि हि चक्षुः॑ । \newline
37. चक्षुः॒ समृ॑द्ध्यै॒ समृ॑द्ध्यै॒ चक्षु॒ श्चक्षुः॒ समृ॑द्ध्यै । \newline
38. समृ॑द्ध्यै देवासु॒रा दे॑वासु॒राः समृ॑द्ध्यै॒ समृ॑द्ध्यै देवासु॒राः । \newline
39. समृ॑द्ध्या॒ इति॒ सं - ऋ॒द्ध्यै॒ । \newline
40. दे॒वा॒सु॒राः संॅय॑त्ताः॒ संॅय॑त्ता देवासु॒रा दे॑वासु॒राः संॅय॑त्ताः । \newline
41. दे॒वा॒सु॒रा इति॑ देव - अ॒सु॒राः । \newline
42. संॅय॑त्ता आसन् नास॒न् थ्संॅय॑त्ताः॒ संॅय॑त्ता आसन्न् । \newline
43. संॅय॑त्ता॒ इति॒ सं - य॒त्ताः॒ । \newline
44. आ॒स॒न् ते त आ॑सन् नास॒न् ते । \newline
45. ते वा॒मं ॅवा॒मम् ते ते वा॒मम् । \newline
46. वा॒मं ॅवसु॒ वसु॑ वा॒मं ॅवा॒मं ॅवसु॑ । \newline
47. वसु॒ सꣳ सं ॅवसु॒ वसु॒ सम् । \newline
48. सम् नि नि सꣳ सम् नि । \newline
49. न्य॑दधता दधत॒ नि न्य॑दधत । \newline
50. अ॒द॒ध॒त॒ तत् तद॑दधता दधत॒ तत् । \newline
51. तद् दे॒वा दे॒वा स्तत् तद् दे॒वाः । \newline
52. दे॒वा वा॑म॒भृता॑ वाम॒भृता॑ दे॒वा दे॒वा वा॑म॒भृता᳚ । \newline
53. वा॒म॒भृता॑ ऽवृञ्जता वृञ्जत वाम॒भृता॑ वाम॒भृता॑ ऽवृञ्जत । \newline
54. वा॒म॒भृतेति॑ वाम - भृता᳚ । \newline
55. अ॒वृ॒ञ्ज॒त॒ तत् तद॑वृञ्जता वृञ्जत॒ तत् । \newline
56. तद् वा॑म॒भृतो॑ वाम॒भृत॒ स्तत् तद् वा॑म॒भृतः॑ । \newline
57. वा॒म॒भृतो॑ वामभृ॒त्त्वं ॅवा॑मभृ॒त्त्वं ॅवा॑म॒भृतो॑ वाम॒भृतो॑ वामभृ॒त्त्वम् । \newline
58. वा॒म॒भृत॒ इति॑ वाम - भृतः॑ । \newline
59. वा॒म॒भृ॒त्त्वं ॅयद् यद् वा॑मभृ॒त्त्वं ॅवा॑मभृ॒त्त्वं ॅयत् । \newline
60. वा॒म॒भृ॒त्त्वमिति॑ वामभृत् - त्वम् । \newline
61. यद् वा॑म॒भृतं॑ ॅवाम॒भृतं॒ ॅयद् यद् वा॑म॒भृत᳚म् । \newline
62. वा॒म॒भृत॑ मुप॒दधा᳚ त्युप॒दधा॑ति वाम॒भृतं॑ ॅवाम॒भृत॑ मुप॒दधा॑ति । \newline
63. वा॒म॒भृत॒मिति॑ वाम - भृत᳚म् । \newline
64. उ॒प॒दधा॑ति वा॒मं ॅवा॒म मु॑प॒दधा᳚ त्युप॒दधा॑ति वा॒मम् । \newline
65. उ॒प॒दधा॒तीत्यु॑प - दधा॑ति । \newline
66. वा॒म मे॒वैव वा॒मं ॅवा॒म मे॒व । \newline
67. ए॒व तया॒ तयै॒ वैव तया᳚ । \newline
68. तया॒ वसु॒ वसु॒ तया॒ तया॒ वसु॑ । \newline
69. वसु॒ यज॑मानो॒ यज॑मानो॒ वसु॒ वसु॒ यज॑मानः । \newline
70. यज॑मानो॒ भ्रातृ॑व्यस्य॒ भ्रातृ॑व्यस्य॒ यज॑मानो॒ यज॑मानो॒ भ्रातृ॑व्यस्य । \newline
71. भ्रातृ॑व्यस्य वृङ्क्ते वृङ्क्ते॒ भ्रातृ॑व्यस्य॒ भ्रातृ॑व्यस्य वृङ्क्ते । \newline
72. वृ॒ङ्क्ते॒ हिर॑ण्यमूर्द्ध्नी॒ हिर॑ण्यमूर्द्ध्नी वृङ्क्ते वृङ्क्ते॒ हिर॑ण्यमूर्द्ध्नी । \newline
73. हिर॑ण्यमूर्द्ध्नी भवति भवति॒ हिर॑ण्यमूर्द्ध्नी॒ हिर॑ण्यमूर्द्ध्नी भवति । \newline
74. हिर॑ण्यमू॒र्द्ध्नीति॒ हिर॑ण्य - मू॒र्द्ध्नी॒ । \newline
75. भ॒व॒ति॒ ज्योति॒र् ज्योति॑र् भवति भवति॒ ज्योतिः॑ । \newline
76. ज्योति॒र् वै वै ज्योति॒र् ज्योति॒र् वै । \newline
77. वै हिर॑ण्यꣳ॒॒ हिर॑ण्यं॒ ॅवै वै हिर॑ण्यम् । \newline
78. हिर॑ण्य॒म् ज्योति॒र् ज्योति॒र्॒. हिर॑ण्यꣳ॒॒ हिर॑ण्य॒म् ज्योतिः॑ । \newline
79. ज्योति॑र् वा॒मं ॅवा॒मम् ज्योति॒र् ज्योति॑र् वा॒मम् । \newline
80. वा॒मम् ज्योति॑षा॒ ज्योति॑षा वा॒मं ॅवा॒मम् ज्योति॑षा । \newline
81. ज्योति॑ षै॒वैव ज्योति॑षा॒ ज्योति॑षै॒व । \newline
82. ए॒वास्या᳚ स्यै॒वै वास्य॑ । \newline
83. अ॒स्य॒ ज्योति॒र् ज्योति॑र स्यास्य॒ ज्योतिः॑ । \newline
84. ज्योति॑र् वा॒मं ॅवा॒मम् ज्योति॒र् ज्योति॑र् वा॒मम् । \newline
85. वा॒मं ॅवृ॑ङ्क्ते वृङ्क्ते वा॒मं ॅवा॒मं ॅवृ॑ङ्क्ते । \newline
86. वृ॒ङ्क्ते॒ द्वि॒य॒जुर् द्वि॑य॒जुर् वृ॑ङ्क्ते वृङ्क्ते द्विय॒जुः । \newline
87. द्वि॒य॒जुर् भ॑वति भवति द्विय॒जुर् द्वि॑य॒जुर् भ॑वति । \newline
88. द्वि॒य॒जुरिति॑ द्वि - य॒जुः । \newline
89. भ॒व॒ति॒ प्रति॑ष्ठित्यै॒ प्रति॑ष्ठित्यै भवति भवति॒ प्रति॑ष्ठित्यै । \newline
90. प्रति॑ष्ठित्या॒ इति॒ प्रति॑ - स्थि॒त्यै॒ । \newline

\textbf{Ghana Paata } \newline

1. पु॒रु॒ष॒शी॒र्॒.ष मुपोप॑ पुरुषशी॒र्॒.षम् पु॑रुषशी॒र्॒.ष मुप॑ दधाति दधा॒ त्युप॑ पुरुषशी॒र्॒.षम् पु॑रुषशी॒र्॒.ष मुप॑ दधाति । \newline
2. पु॒रु॒ष॒शी॒र्॒.षमिति॑ पुरुष - शी॒र्॒.षम् । \newline
3. उप॑ दधाति दधा॒ त्युपोप॑ दधाति मुख॒तो मु॑ख॒तो द॑धा॒ त्युपोप॑ दधाति मुख॒तः । \newline
4. द॒धा॒ति॒ मु॒ख॒तो मु॑ख॒तो द॑धाति दधाति मुख॒त ए॒वैव मु॑ख॒तो द॑धाति दधाति मुख॒त ए॒व । \newline
5. मु॒ख॒त ए॒वैव मु॑ख॒तो मु॑ख॒त ए॒वैन॑ मेन मे॒व मु॑ख॒तो मु॑ख॒त ए॒वैन᳚म् । \newline
6. ए॒वैन॑ मेन मे॒वै वैन॒ माहु॑तय॒ आहु॑तय एन मे॒वै वैन॒ माहु॑तयः । \newline
7. ए॒न॒ माहु॑तय॒ आहु॑तय एन मेन॒ माहु॑तय ऋच्छन् त्यृच्छ॒न् त्याहु॑तय एन मेन॒ माहु॑तय ऋच्छन्ति । \newline
8. आहु॑तय ऋच्छन् त्यृच्छ॒न् त्याहु॑तय॒ आहु॑तय ऋच्छन्ति॒ न न र्च्छ॒न्त्याहु॑तय॒ आहु॑तय ऋच्छन्ति॒ न । \newline
9. आहु॑तय॒ इत्या - हु॒त॒यः॒ । \newline
10. ऋ॒च्छ॒न्ति॒ न न र्च्छ॑न् त्यृच्छन्ति॒ नोत्ता॒न मु॑त्ता॒नम् न र्च्छ॑न् त्यृच्छन्ति॒ नोत्ता॒नम् । \newline
11. नोत्ता॒न मु॑त्ता॒नम् न नोत्ता॒नम् चि॑नुते चिनुत उत्ता॒नम् न नोत्ता॒नम् चि॑नुते । \newline
12. उ॒त्ता॒नम् चि॑नुते चिनुत उत्ता॒न मु॑त्ता॒नम् चि॑नुते सुव॒र्ग्यः॑ सुव॒र्ग्य॑ श्चिनुत उत्ता॒न मु॑त्ता॒नम् चि॑नुते सुव॒र्ग्यः॑ । \newline
13. उ॒त्ता॒नमित्यु॑त् - ता॒नम् । \newline
14. चि॒नु॒ते॒ सु॒व॒र्ग्यः॑ सुव॒र्ग्य॑ श्चिनुते चिनुते सुव॒र्ग्यो᳚ ऽस्यास्य सुव॒र्ग्य॑ श्चिनुते चिनुते सुव॒र्ग्यो᳚ ऽस्य । \newline
15. सु॒व॒र्ग्यो᳚ ऽस्यास्य सुव॒र्ग्यः॑ सुव॒र्ग्यो᳚ ऽस्य भवति भव त्यस्य सुव॒र्ग्यः॑ सुव॒र्ग्यो᳚ ऽस्य भवति । \newline
16. सु॒व॒र्ग्य॑ इति॑ सुवः - ग्यः॑ । \newline
17. अ॒स्य॒ भ॒व॒ति॒ भ॒व॒ त्य॒स्या॒स्य॒ भ॒व॒ति॒ सौ॒र्या सौ॒र्या भ॑व त्यस्यास्य भवति सौ॒र्या । \newline
18. भ॒व॒ति॒ सौ॒र्या सौ॒र्या भ॑वति भवति सौ॒र्या जु॑होति जुहोति सौ॒र्या भ॑वति भवति सौ॒र्या जु॑होति । \newline
19. सौ॒र्या जु॑होति जुहोति सौ॒र्या सौ॒र्या जु॑होति॒ चक्षु॒ श्चक्षु॑र् जुहोति सौ॒र्या सौ॒र्या जु॑होति॒ चक्षुः॑ । \newline
20. जु॒हो॒ति॒ चक्षु॒ श्चक्षु॑र् जुहोति जुहोति॒ चक्षु॑ रे॒वैव चक्षु॑र् जुहोति जुहोति॒ चक्षु॑ रे॒व । \newline
21. चक्षु॑रे॒ वैव चक्षु॒ श्चक्षु॑ रे॒वास्मि॑न् नस्मिन् ने॒व चक्षु॒ श्चक्षु॑ रे॒वास्मिन्न्॑ । \newline
22. ए॒वास्मि॑न् नस्मिन् ने॒वै वास्मि॒न् प्रति॒ प्रत्य॑स्मिन् ने॒वै वास्मि॒न् प्रति॑ । \newline
23. अ॒स्मि॒न् प्रति॒ प्रत्य॑स्मिन् नस्मि॒न् प्रति॑ दधाति दधाति॒ प्रत्य॑स्मिन् नस्मि॒न् प्रति॑ दधाति । \newline
24. प्रति॑ दधाति दधाति॒ प्रति॒ प्रति॑ दधाति॒ द्विर् द्विर् द॑धाति॒ प्रति॒ प्रति॑ दधाति॒ द्विः । \newline
25. द॒धा॒ति॒ द्विर् द्विर् द॑धाति दधाति॒ द्विर् जु॑होति जुहोति॒ द्विर् द॑धाति दधाति॒ द्विर् जु॑होति । \newline
26. द्विर् जु॑होति जुहोति॒ द्विर् द्विर् जु॑होति॒ द्वे द्वे जु॑होति॒ द्विर् द्विर् जु॑होति॒ द्वे । \newline
27. जु॒हो॒ति॒ द्वे द्वे जु॑होति जुहोति॒ द्वे हि हि द्वे जु॑होति जुहोति॒ द्वे हि । \newline
28. द्वे हि हि द्वे द्वे हि चक्षु॑षी॒ चक्षु॑षी॒ हि द्वे द्वे हि चक्षु॑षी । \newline
29. द्वे इति॒ द्वे । \newline
30. हि चक्षु॑षी॒ चक्षु॑षी॒ हि हि चक्षु॑षी समा॒न्या स॑मा॒न्या चक्षु॑षी॒ हि हि चक्षु॑षी समा॒न्या । \newline
31. चक्षु॑षी समा॒न्या स॑मा॒न्या चक्षु॑षी॒ चक्षु॑षी समा॒न्या जु॑होति जुहोति समा॒न्या चक्षु॑षी॒ चक्षु॑षी समा॒न्या जु॑होति । \newline
32. चक्षु॑षी॒ इति॒ चक्षु॑षी । \newline
33. स॒मा॒न्या जु॑होति जुहोति समा॒न्या स॑मा॒न्या जु॑होति समा॒नꣳ स॑मा॒नम् जु॑होति समा॒न्या स॑मा॒न्या जु॑होति समा॒नम् । \newline
34. जु॒हो॒ति॒ स॒मा॒नꣳ स॑मा॒नम् जु॑होति जुहोति समा॒नꣳ हि हि स॑मा॒नम् जु॑होति जुहोति समा॒नꣳ हि । \newline
35. स॒मा॒नꣳ हि हि स॑मा॒नꣳ स॑मा॒नꣳ हि चक्षु॒ श्चक्षु॒र्॒. हि स॑मा॒नꣳ स॑मा॒नꣳ हि चक्षुः॑ । \newline
36. हि चक्षु॒ श्चक्षु॒र्॒. हि हि चक्षुः॒ समृ॑द्ध्यै॒ समृ॑द्ध्यै॒ चक्षु॒र्॒. हि हि चक्षुः॒ समृ॑द्ध्यै । \newline
37. चक्षुः॒ समृ॑द्ध्यै॒ समृ॑द्ध्यै॒ चक्षु॒ श्चक्षुः॒ समृ॑द्ध्यै देवासु॒रा दे॑वासु॒राः समृ॑द्ध्यै॒ चक्षु॒ श्चक्षुः॒ समृ॑द्ध्यै देवासु॒राः । \newline
38. समृ॑द्ध्यै देवासु॒रा दे॑वासु॒राः समृ॑द्ध्यै॒ समृ॑द्ध्यै देवासु॒राः संॅय॑त्ताः॒ संॅय॑त्ता देवासु॒राः समृ॑द्ध्यै॒ समृ॑द्ध्यै देवासु॒राः संॅय॑त्ताः । \newline
39. समृ॑द्ध्या॒ इति॒ सं - ऋ॒द्ध्यै॒ । \newline
40. दे॒वा॒सु॒राः संॅय॑त्ताः॒ संॅय॑त्ता देवासु॒रा दे॑वासु॒राः संॅय॑त्ता आसन् नास॒न् थ्संॅय॑त्ता देवासु॒रा दे॑वासु॒राः संॅय॑त्ता आसन्न् । \newline
41. दे॒वा॒सु॒रा इति॑ देव - अ॒सु॒राः । \newline
42. संॅय॑त्ता आसन् नास॒न् थ्संॅय॑त्ताः॒ संॅय॑त्ता आस॒न् ते त आ॑स॒न् थ्संॅय॑त्ताः॒ संॅय॑त्ता आस॒न् ते । \newline
43. संॅय॑त्ता॒ इति॒ सं - य॒त्ताः॒ । \newline
44. आ॒स॒न् ते त आ॑सन् नास॒न् ते वा॒मं ॅवा॒मम् त आ॑सन् नास॒न् ते वा॒मम् । \newline
45. ते वा॒मं ॅवा॒मम् ते ते वा॒मं ॅवसु॒ वसु॑ वा॒मम् ते ते वा॒मं ॅवसु॑ । \newline
46. वा॒मं ॅवसु॒ वसु॑ वा॒मं ॅवा॒मं ॅवसु॒ सꣳ सं ॅवसु॑ वा॒मं ॅवा॒मं ॅवसु॒ सम् । \newline
47. वसु॒ सꣳ सं ॅवसु॒ वसु॒ सम् नि नि सं ॅवसु॒ वसु॒ सम् नि । \newline
48. सम् नि नि सꣳ सम् न्य॑दधता दधत॒ नि सꣳ सम् न्य॑दधत । \newline
49. न्य॑दधता दधत॒ नि न्य॑दधत॒ तत् तद॑दधत॒ नि न्य॑दधत॒ तत् । \newline
50. अ॒द॒ध॒त॒ तत् तद॑दधता दधत॒ तद् दे॒वा दे॒वा स्तद॑ दधता दधत॒ तद् दे॒वाः । \newline
51. तद् दे॒वा दे॒वा स्तत् तद् दे॒वा वा॑म॒भृता॑ वाम॒भृता॑ दे॒वा स्तत् तद् दे॒वा वा॑म॒भृता᳚ । \newline
52. दे॒वा वा॑म॒भृता॑ वाम॒भृता॑ दे॒वा दे॒वा वा॑म॒भृता॑ ऽवृञ्जता वृञ्जत वाम॒भृता॑ दे॒वा दे॒वा वा॑म॒भृता॑ ऽवृञ्जत । \newline
53. वा॒म॒भृता॑ ऽवृञ्जता वृञ्जत वाम॒भृता॑ वाम॒भृता॑ ऽवृञ्जत॒ तत् तद॑वृञ्जत वाम॒भृता॑ वाम॒भृता॑ ऽवृञ्जत॒ तत् । \newline
54. वा॒म॒भृतेति॑ वाम - भृता᳚ । \newline
55. अ॒वृ॒ञ्ज॒त॒ तत् तद॑वृञ्जता वृञ्जत॒ तद् वा॑म॒भृतो॑ वाम॒भृत॒ स्तद॑वृञ्जता वृञ्जत॒ तद् वा॑म॒भृतः॑ । \newline
56. तद् वा॑म॒भृतो॑ वाम॒भृत॒ स्तत् तद् वा॑म॒भृतो॑ वामभृ॒त्त्वं ॅवा॑मभृ॒त्त्वं ॅवा॑म॒भृत॒ स्तत् तद् वा॑म॒भृतो॑ वामभृ॒त्त्वम् । \newline
57. वा॒म॒भृतो॑ वामभृ॒त्त्वं ॅवा॑मभृ॒त्त्वं ॅवा॑म॒भृतो॑ वाम॒भृतो॑ वामभृ॒त्त्वं ॅयद् यद् वा॑मभृ॒त्त्वं ॅवा॑म॒भृतो॑ वाम॒भृतो॑ वामभृ॒त्त्वं ॅयत् । \newline
58. वा॒म॒भृत॒ इति॑ वाम - भृतः॑ । \newline
59. वा॒म॒भृ॒त्त्वं ॅयद् यद् वा॑मभृ॒त्त्वं ॅवा॑मभृ॒त्त्वं ॅयद् वा॑म॒भृतं॑ ॅवाम॒भृतं॒ ॅयद् वा॑मभृ॒त्त्वं ॅवा॑मभृ॒त्त्वं ॅयद् वा॑म॒भृत᳚म् । \newline
60. वा॒म॒भृ॒त्त्वमिति॑ वामभृत् - त्वम् । \newline
61. यद् वा॑म॒भृतं॑ ॅवाम॒भृतं॒ ॅयद् यद् वा॑म॒भृत॑ मुप॒दधा᳚ त्युप॒दधा॑ति वाम॒भृतं॒ ॅयद् यद् वा॑म॒भृत॑ मुप॒दधा॑ति । \newline
62. वा॒म॒भृत॑ मुप॒दधा᳚ त्युप॒दधा॑ति वाम॒भृतं॑ ॅवाम॒भृत॑ मुप॒दधा॑ति वा॒मं ॅवा॒म मु॑प॒दधा॑ति वाम॒भृतं॑ ॅवाम॒भृत॑ मुप॒दधा॑ति वा॒मम् । \newline
63. वा॒म॒भृत॒मिति॑ वाम - भृत᳚म् । \newline
64. उ॒प॒दधा॑ति वा॒मं ॅवा॒म मु॑प॒दधा᳚ त्युप॒दधा॑ति वा॒म मे॒वैव वा॒म मु॑प॒दधा᳚ त्युप॒दधा॑ति वा॒म मे॒व । \newline
65. उ॒प॒दधा॒तीत्यु॑प - दधा॑ति । \newline
66. वा॒म मे॒वैव वा॒मं ॅवा॒म मे॒व तया॒ तयै॒व वा॒मं ॅवा॒म मे॒व तया᳚ । \newline
67. ए॒व तया॒ तयै॒ वैव तया॒ वसु॒ वसु॒ तयै॒ वैव तया॒ वसु॑ । \newline
68. तया॒ वसु॒ वसु॒ तया॒ तया॒ वसु॒ यज॑मानो॒ यज॑मानो॒ वसु॒ तया॒ तया॒ वसु॒ यज॑मानः । \newline
69. वसु॒ यज॑मानो॒ यज॑मानो॒ वसु॒ वसु॒ यज॑मानो॒ भ्रातृ॑व्यस्य॒ भ्रातृ॑व्यस्य॒ यज॑मानो॒ वसु॒ वसु॒ यज॑मानो॒ भ्रातृ॑व्यस्य । \newline
70. यज॑मानो॒ भ्रातृ॑व्यस्य॒ भ्रातृ॑व्यस्य॒ यज॑मानो॒ यज॑मानो॒ भ्रातृ॑व्यस्य वृङ्क्ते वृङ्क्ते॒ भ्रातृ॑व्यस्य॒ यज॑मानो॒ यज॑मानो॒ भ्रातृ॑व्यस्य वृङ्क्ते । \newline
71. भ्रातृ॑व्यस्य वृङ्क्ते वृङ्क्ते॒ भ्रातृ॑व्यस्य॒ भ्रातृ॑व्यस्य वृङ्क्ते॒ हिर॑ण्यमूर्द्ध्नी॒ हिर॑ण्यमूर्द्ध्नी वृङ्क्ते॒ भ्रातृ॑व्यस्य॒ भ्रातृ॑व्यस्य वृङ्क्ते॒ हिर॑ण्यमूर्द्ध्नी । \newline
72. वृ॒ङ्क्ते॒ हिर॑ण्यमूर्द्ध्नी॒ हिर॑ण्यमूर्द्ध्नी वृङ्क्ते वृङ्क्ते॒ हिर॑ण्यमूर्द्ध्नी भवति भवति॒ हिर॑ण्यमूर्द्ध्नी वृङ्क्ते वृङ्क्ते॒ हिर॑ण्यमूर्द्ध्नी भवति । \newline
73. हिर॑ण्यमूर्द्ध्नी भवति भवति॒ हिर॑ण्यमूर्द्ध्नी॒ हिर॑ण्यमूर्द्ध्नी भवति॒ ज्योति॒र् ज्योति॑र् भवति॒ हिर॑ण्यमूर्द्ध्नी॒ हिर॑ण्यमूर्द्ध्नी भवति॒ ज्योतिः॑ । \newline
74. हिर॑ण्यमू॒र्द्ध्नीति॒ हिर॑ण्य - मू॒र्द्ध्नी॒ । \newline
75. भ॒व॒ति॒ ज्योति॒र् ज्योति॑र् भवति भवति॒ ज्योति॒र् वै वै ज्योति॑र् भवति भवति॒ ज्योति॒र् वै । \newline
76. ज्योति॒र् वै वै ज्योति॒र् ज्योति॒र् वै हिर॑ण्यꣳ॒॒ हिर॑ण्यं॒ ॅवै ज्योति॒र् ज्योति॒र् वै हिर॑ण्यम् । \newline
77. वै हिर॑ण्यꣳ॒॒ हिर॑ण्यं॒ ॅवै वै हिर॑ण्य॒म् ज्योति॒र् ज्योति॒र्॒. हिर॑ण्यं॒ ॅवै वै हिर॑ण्य॒म् ज्योतिः॑ । \newline
78. हिर॑ण्य॒म् ज्योति॒र् ज्योति॒र्॒. हिर॑ण्यꣳ॒॒ हिर॑ण्य॒म् ज्योति॑र् वा॒मं ॅवा॒मम् ज्योति॒र्॒. हिर॑ण्यꣳ॒॒ हिर॑ण्य॒म् ज्योति॑र् वा॒मम् । \newline
79. ज्योति॑र् वा॒मं ॅवा॒मम् ज्योति॒र् ज्योति॑र् वा॒मम् ज्योति॑षा॒ ज्योति॑षा वा॒मम् ज्योति॒र् ज्योति॑र् वा॒मम् ज्योति॑षा । \newline
80. वा॒मम् ज्योति॑षा॒ ज्योति॑षा वा॒मं ॅवा॒मम् ज्योति॑षै॒वैव ज्योति॑षा वा॒मं ॅवा॒मम् ज्योति॑षै॒व । \newline
81. ज्योति॑षै॒वैव ज्योति॑षा॒ ज्योति॑षै॒ वास्या᳚ स्यै॒व ज्योति॑षा॒ ज्योति॑षै॒ वास्य॑ । \newline
82. ए॒वास्या᳚ स्यै॒वै वास्य॒ ज्योति॒र् ज्योति॑ रस्यै॒ वैवास्य॒ ज्योतिः॑ । \newline
83. अ॒स्य॒ ज्योति॒र् ज्योति॑ रस्यास्य॒ ज्योति॑र् वा॒मं ॅवा॒मम् ज्योति॑ रस्यास्य॒ ज्योति॑र् वा॒मम् । \newline
84. ज्योति॑र् वा॒मं ॅवा॒मम् ज्योति॒र् ज्योति॑र् वा॒मं ॅवृ॑ङ्क्ते वृङ्क्ते वा॒मम् ज्योति॒र् ज्योति॑र् वा॒मं ॅवृ॑ङ्क्ते । \newline
85. वा॒मं ॅवृ॑ङ्क्ते वृङ्क्ते वा॒मं ॅवा॒मं ॅवृ॑ङ्क्ते द्विय॒जुर् द्वि॑य॒जुर् वृ॑ङ्क्ते वा॒मं ॅवा॒मं ॅवृ॑ङ्क्ते द्विय॒जुः । \newline
86. वृ॒ङ्क्ते॒ द्वि॒य॒जुर् द्वि॑य॒जुर् वृ॑ङ्क्ते वृङ्क्ते द्विय॒जुर् भ॑वति भवति द्विय॒जुर् वृ॑ङ्क्ते वृङ्क्ते द्विय॒जुर् भ॑वति । \newline
87. द्वि॒य॒जुर् भ॑वति भवति द्विय॒जुर् द्वि॑य॒जुर् भ॑वति॒ प्रति॑ष्ठित्यै॒ प्रति॑ष्ठित्यै भवति द्विय॒जुर् द्वि॑य॒जुर् भ॑वति॒ प्रति॑ष्ठित्यै । \newline
88. द्वि॒य॒जुरिति॑ द्वि - य॒जुः । \newline
89. भ॒व॒ति॒ प्रति॑ष्ठित्यै॒ प्रति॑ष्ठित्यै भवति भवति॒ प्रति॑ष्ठित्यै । \newline
90. प्रति॑ष्ठित्या॒ इति॒ प्रति॑ - स्थि॒त्यै॒ । \newline
\pagebreak
\markright{ TS 5.5.4.1  \hfill https://www.vedavms.in \hfill}

\section{ TS 5.5.4.1 }

\textbf{TS 5.5.4.1 } \newline
\textbf{Samhita Paata} \newline

आपो॒ वरु॑णस्य॒ पत्न॑य आस॒न् ता अ॒ग्निर॒भ्य॑द्ध्याय॒त् ताः सम॑भव॒त् तस्य॒ रेतः॒ परा॑ऽपत॒त् तदि॒यम॑भव॒द्यद् द्वि॒तीयं॑ प॒राऽप॑त॒त् तद॒साव॑भवदि॒यं ॅवै वि॒राड॒सौ स्व॒राड् यद्-वि॒राजा॑वुप॒दधा॑ती॒मे ए॒वोप॑ धत्ते॒ यद्वा अ॒सौ रेतः॑ सि॒ञ्चति॒ तद॒स्यां प्रति॑ तिष्ठति॒ तत् प्र जा॑यते॒ ता ओष॑धयो - [  ] \newline

\textbf{Pada Paata} \newline

आपः॑ । वरु॑णस्य । पत्न॑यः । आ॒स॒न्न् । ताः । अ॒ग्निः । अ॒भीति॑ । अ॒द्ध्या॒य॒त् । ताः । समिति॑ । अ॒भ॒व॒त् । तस्य॑ । रेतः॑ । परेति॑ । अ॒प॒त॒त् । तत् । इ॒यम् । अ॒भ॒व॒त् । यत् । द्वि॒तीय᳚म् । प॒राप॑त॒दिति॑ परा - अप॑तत् । तत् । अ॒सौ । अ॒भ॒व॒त् । इ॒यम् । वै । वि॒राडिति॑ वि - राट् । अ॒सौ । स्व॒राडिति॑ स्व - राट् । यत् । वि॒राजा॒विति॑ वि - राजौ᳚ । उ॒प॒दधा॒तीत्यु॑प-दधा॑ति । इ॒मे इति॑ । ए॒व । उपेति॑ । ध॒त्ते॒ । यत् । वै । अ॒सौ । रेतः॑ । सि॒ञ्चति॑ । तत् । अ॒स्याम् । प्रतीति॑ । ति॒ष्ठ॒ति॒ । तत् । प्रेति॑ । जा॒य॒ते॒ । ताः । ओष॑धयः ।  \newline


\textbf{Krama Paata} \newline

आपो॒ वरु॑णस्य । वरु॑णस्य॒ पत्न॑यः । पत्न॑य आसन्न् । आ॒स॒न् ताः । ता अ॒ग्निः । अ॒ग्निर॒भि । अ॒भ्य॑द्ध्यायत् । अ॒द्ध्या॒य॒त् ताः । ताः सम् । सम॑भवत् । अ॒भ॒व॒त् तस्य॑ । तस्य॒ रेतः॑ । रेतः॒ परा᳚ । परा॑ऽपतत् । अ॒प॒त॒त् तत् । तदि॒यम् । इ॒यम॑भवत् । अ॒भ॒व॒द् यत् । यद् द्वि॒तीय᳚म् । द्वि॒तीय॑म् प॒राप॑तत् । प॒राप॑त॒त् तत् । प॒राप॑त॒दिति॑ परा - अप॑तत् । तद॒सौ । अ॒साव॑भवत् । अ॒भ॒व॒दि॒यम् । इ॒यम् ॅवै । वै वि॒राट् । वि॒राड॒सौ । वि॒राडिति॑ वि - राट् । अ॒सौ स्व॒राट् । स्व॒राड् यत् । स्व॒राडिति॑ स्व - राट् । यद् वि॒राजौ᳚ । वि॒राजा॑वुप॒दधा॑ति । वि॒राजा॒विति॑ वि - राजौ᳚ । उ॒प॒दधा॑ती॒मे । उ॒प॒दधा॒तीत्यु॑प - दधा॑ति । इ॒मे ए॒व । इ॒मे इती॒मे । ए॒वोप॑ । उप॑ धत्ते । ध॒त्ते॒ यत् । यद् वै । वा अ॒सौ । अ॒सौ रेतः॑ । रेतः॑ सि॒ञ्चति॑ । सि॒ञ्चति॒ तत् । तद॒स्याम् । अ॒स्याम् प्रति॑ । प्रति॑ तिष्ठति । ति॒ष्ठ॒ति॒ तत् । तत् प्र । प्र जा॑यते । जा॒य॒ते॒ ताः । ता ओष॑धयः । ओष॑धयो वी॒रुधः॑ \newline

\textbf{Jatai Paata} \newline

1. आपो॒ वरु॑णस्य॒ वरु॑ण॒स्याप॒ आपो॒ वरु॑णस्य । \newline
2. वरु॑णस्य॒ पत्न॑यः॒ पत्न॑यो॒ वरु॑णस्य॒ वरु॑णस्य॒ पत्न॑यः । \newline
3. पत्न॑य आसन् नास॒न् पत्न॑यः॒ पत्न॑य आसन्न् । \newline
4. आ॒स॒न् ता स्ता आ॑सन् नास॒न् ताः । \newline
5. ता अ॒ग्नि र॒ग्नि स्ता स्ता अ॒ग्निः । \newline
6. अ॒ग्नि र॒भ्या᳚(1॒)भ्य॑ग्नि र॒ग्नि र॒भि । \newline
7. अ॒भ्य॑द्ध्याय दद्ध्याय द॒भ्या᳚(1॒)भ्य॑द्ध्यायत् । \newline
8. अ॒द्ध्या॒य॒त् ता स्ता अ॑द्ध्याय दद्ध्याय॒त् ताः । \newline
9. ताः सꣳ सम् ता स्ताः सम् । \newline
10. स म॑भव दभव॒थ् सꣳ स म॑भवत् । \newline
11. अ॒भ॒व॒त् तस्य॒ तस्या॑भव दभव॒त् तस्य॑ । \newline
12. तस्य॒ रेतो॒ रेत॒ स्तस्य॒ तस्य॒ रेतः॑ । \newline
13. रेतः॒ परा॒ परा॒ रेतो॒ रेतः॒ परा᳚ । \newline
14. परा॑ ऽपत दपत॒त् परा॒ परा॑ ऽपतत् । \newline
15. अ॒प॒त॒त् तत् तद॑पत दपत॒त् तत् । \newline
16. तदि॒य मि॒यम् तत् तदि॒यम् । \newline
17. इ॒य म॑भव दभव दि॒य मि॒य म॑भवत् । \newline
18. अ॒भ॒व॒द् यद् यद॑भव दभव॒द् यत् । \newline
19. यद् द्वि॒तीय॑म् द्वि॒तीयं॒ ॅयद् यद् द्वि॒तीय᳚म् । \newline
20. द्वि॒तीय॑म् प॒राप॑तत् प॒राप॑तद् द्वि॒तीय॑म् द्वि॒तीय॑म् प॒राप॑तत् । \newline
21. प॒राप॑त॒त् तत् तत् प॒राप॑तत् प॒राप॑त॒त् तत् । \newline
22. प॒राप॑त॒दिति॑ परा - अप॑तत् । \newline
23. तद॒सा व॒सौ तत् तद॒सौ । \newline
24. अ॒सा व॑भव दभव द॒सा व॒सा व॑भवत् । \newline
25. अ॒भ॒व॒ दि॒य मि॒य म॑भव दभव दि॒यम् । \newline
26. इ॒यं ॅवै वा इ॒य मि॒यं ॅवै । \newline
27. वै वि॒राड् वि॒राड् वै वै वि॒राट् । \newline
28. वि॒रा ड॒सा व॒सौ वि॒राड् वि॒रा ड॒सौ । \newline
29. वि॒राडिति॑ वि - राट् । \newline
30. अ॒सौ स्व॒राट् थ् स्व॒रा ड॒सा व॒सौ स्व॒राट् । \newline
31. स्व॒राड् यद् यथ् स्व॒राट् थ्स्व॒राड् यत् । \newline
32. स्व॒राडिति॑ स्व - राट् । \newline
33. यद् वि॒राजौ॑ वि॒राजौ॒ यद् यद् वि॒राजौ᳚ । \newline
34. वि॒राजा॑ वुप॒दधा᳚ त्युप॒दधा॑ति वि॒राजौ॑ वि॒राजा॑ वुप॒दधा॑ति । \newline
35. वि॒राजा॒विति॑ वि - राजौ᳚ । \newline
36. उ॒प॒दधा॑ ती॒मे इ॒मे उ॑प॒दधा᳚ त्युप॒दधा॑ती॒मे । \newline
37. उ॒प॒दधा॒तीत्यु॑प - दधा॑ति । \newline
38. इ॒मे ए॒वैवेमे इ॒मे ए॒व । \newline
39. इ॒मे इती॒मे । \newline
40. ए॒वोपो पै॒वै वोप॑ । \newline
41. उप॑ धत्ते धत्त॒ उपोप॑ धत्ते । \newline
42. ध॒त्ते॒ यद् यद् ध॑त्ते धत्ते॒ यत् । \newline
43. यद् वै वै यद् यद् वै । \newline
44. वा अ॒सा व॒सौ वै वा अ॒सौ । \newline
45. अ॒सौ रेतो॒ रेतो॒ ऽसा व॒सौ रेतः॑ । \newline
46. रेतः॑ सि॒ञ्चति॑ सि॒ञ्चति॒ रेतो॒ रेतः॑ सि॒ञ्चति॑ । \newline
47. सि॒ञ्चति॒ तत् तथ् सि॒ञ्चति॑ सि॒ञ्चति॒ तत् । \newline
48. तद॒स्या म॒स्याम् तत् तद॒स्याम् । \newline
49. अ॒स्याम् प्रति॒ प्रत्य॒स्या म॒स्याम् प्रति॑ । \newline
50. प्रति॑ तिष्ठति तिष्ठति॒ प्रति॒ प्रति॑ तिष्ठति । \newline
51. ति॒ष्ठ॒ति॒ तत् तत् ति॑ष्ठति तिष्ठति॒ तत् । \newline
52. तत् प्र प्र तत् तत् प्र । \newline
53. प्र जा॑यते जायते॒ प्र प्र जा॑यते । \newline
54. जा॒य॒ते॒ ता स्ता जा॑यते जायते॒ ताः । \newline
55. ता ओष॑धय॒ ओष॑धय॒ स्ता स्ता ओष॑धयः । \newline
56. ओष॑धयो वी॒रुधो॑ वी॒रुध॒ ओष॑धय॒ ओष॑धयो वी॒रुधः॑ । \newline

\textbf{Ghana Paata } \newline

1. आपो॒ वरु॑णस्य॒ वरु॑ण॒स्याप॒ आपो॒ वरु॑णस्य॒ पत्न॑यः॒ पत्न॑यो॒ वरु॑ण॒स्याप॒ आपो॒ वरु॑णस्य॒ पत्न॑यः । \newline
2. वरु॑णस्य॒ पत्न॑यः॒ पत्न॑यो॒ वरु॑णस्य॒ वरु॑णस्य॒ पत्न॑य आसन् नास॒न् पत्न॑यो॒ वरु॑णस्य॒ वरु॑णस्य॒ पत्न॑य आसन्न् । \newline
3. पत्न॑य आसन् नास॒न् पत्न॑यः॒ पत्न॑य आस॒न् ता स्ता आ॑स॒न् पत्न॑यः॒ पत्न॑य आस॒न् ताः । \newline
4. आ॒स॒न् ता स्ता आ॑सन् नास॒न् ता अ॒ग्नि र॒ग्नि स्ता आ॑सन् नास॒न् ता अ॒ग्निः । \newline
5. ता अ॒ग्नि र॒ग्नि स्ता स्ता अ॒ग्नि र॒भ्या᳚(1॒)भ्य॑ग्नि स्ता स्ता अ॒ग्नि र॒भि । \newline
6. अ॒ग्नि र॒भ्या᳚(1॒)भ्य॑ग्नि र॒ग्नि र॒भ्य॑ द्ध्याय दद्ध्याय द॒भ्य॑ग्नि र॒ग्नि र॒भ्य॑द्ध्यायत् । \newline
7. अ॒भ्य॑द्ध्याय दद्ध्याय द॒भ्या᳚(1॒)भ्य॑द्ध्याय॒त् ता स्ता अ॑द्ध्याय द॒भ्या᳚(1॒)भ्य॑द्ध्याय॒त् ताः । \newline
8. अ॒द्ध्या॒य॒त् ता स्ता अ॑द्ध्याय दद्ध्याय॒त् ताः सꣳ सम् ता अ॑द्ध्याय दद्ध्याय॒त् ताः सम् । \newline
9. ताः सꣳ सम् ता स्ताः स म॑भव दभव॒थ् सम् ता स्ताः स म॑भवत् । \newline
10. स म॑भव दभव॒थ् सꣳ स म॑भव॒त् तस्य॒ तस्या॑ भव॒थ् सꣳ स म॑भव॒त् तस्य॑ । \newline
11. अ॒भ॒व॒त् तस्य॒ तस्या॑भव दभव॒त् तस्य॒ रेतो॒ रेत॒ स्तस्या॑भव दभव॒त् तस्य॒ रेतः॑ । \newline
12. तस्य॒ रेतो॒ रेत॒ स्तस्य॒ तस्य॒ रेतः॒ परा॒ परा॒ रेत॒ स्तस्य॒ तस्य॒ रेतः॒ परा᳚ । \newline
13. रेतः॒ परा॒ परा॒ रेतो॒ रेतः॒ परा॑ ऽपत दपत॒त् परा॒ रेतो॒ रेतः॒ परा॑ ऽपतत् । \newline
14. परा॑ ऽपत दपत॒त् परा॒ परा॑ ऽपत॒त् तत् तद॑पत॒त् परा॒ परा॑ ऽपत॒त् तत् । \newline
15. अ॒प॒त॒त् तत् तद॑पत दपत॒त् तदि॒य मि॒यम् तद॑पत दपत॒त् तदि॒यम् । \newline
16. तदि॒य मि॒यम् तत् तदि॒य म॑भव दभव दि॒यम् तत् तदि॒य म॑भवत् । \newline
17. इ॒य म॑भव दभवदि॒य मि॒य म॑भव॒द् यद् यद॑भव दि॒य मि॒य म॑भव॒द् यत् । \newline
18. अ॒भ॒व॒द् यद् यद॑भव दभव॒द् यद् द्वि॒तीय॑म् द्वि॒तीयं॒ ॅयद॑भव दभव॒द् यद् द्वि॒तीय᳚म् । \newline
19. यद् द्वि॒तीय॑म् द्वि॒तीयं॒ ॅयद् यद् द्वि॒तीय॑म् प॒राप॑तत् प॒राप॑तद् द्वि॒तीयं॒ ॅयद् यद् द्वि॒तीय॑म् प॒राप॑तत् । \newline
20. द्वि॒तीय॑म् प॒राप॑तत् प॒राप॑तद् द्वि॒तीय॑म् द्वि॒तीय॑म् प॒राप॑त॒त् तत् तत् प॒राप॑तद् द्वि॒तीय॑म् द्वि॒तीय॑म् प॒राप॑त॒त् तत् । \newline
21. प॒राप॑त॒त् तत् तत् प॒राप॑तत् प॒राप॑त॒त् तद॒सा व॒सौ तत् प॒राप॑तत् प॒राप॑त॒त् तद॒सौ । \newline
22. प॒राप॑त॒दिति॑ परा - अप॑तत् । \newline
23. तद॒सा व॒सौ तत् तद॒सा व॑भव दभव द॒सौ तत् तद॒सा व॑भवत् । \newline
24. अ॒सा व॑भव दभव द॒सा व॒सा व॑भव दि॒य मि॒य म॑भव द॒सा व॒सा व॑भव दि॒यम् । \newline
25. अ॒भ॒व॒ दि॒य मि॒य म॑भव दभव दि॒यं ॅवै वा इ॒य म॑भव दभव दि॒यं ॅवै । \newline
26. इ॒यं ॅवै वा इ॒य मि॒यं ॅवै वि॒राड् वि॒राड् वा इ॒य मि॒यं ॅवै वि॒राट् । \newline
27. वै वि॒राड् वि॒राड् वै वै वि॒रा ड॒सा व॒सौ वि॒राड् वै वै वि॒रा ड॒सौ । \newline
28. वि॒रा ड॒सा व॒सौ वि॒राड् वि॒रा ड॒सौ स्व॒राट् थ्स्व॒रा ड॒सौ वि॒राड् वि॒रा ड॒सौ स्व॒राट् । \newline
29. वि॒राडिति॑ वि - राट् । \newline
30. अ॒सौ स्व॒राट् थ्स्व॒रा ड॒सा व॒सौ स्व॒राड् यद् यथ् स्व॒रा ड॒सा व॒सौ स्व॒राड् यत् । \newline
31. स्व॒राड् यद् यथ् स्व॒राट् थ्स्व॒राड् यद् वि॒राजौ॑ वि॒राजौ॒ यथ् स्व॒राट् थ्स्व॒राड् यद् वि॒राजौ᳚ । \newline
32. स्व॒राडिति॑ स्व - राट् । \newline
33. यद् वि॒राजौ॑ वि॒राजौ॒ यद् यद् वि॒राजा॑ वुप॒दधा᳚ त्युप॒दधा॑ति वि॒राजौ॒ यद् यद् वि॒राजा॑ वुप॒दधा॑ति । \newline
34. वि॒राजा॑ वुप॒दधा᳚ त्युप॒दधा॑ति वि॒राजौ॑ वि॒राजा॑ वुप॒दधा॑ती॒मे इ॒मे उ॑प॒दधा॑ति वि॒राजौ॑ वि॒राजा॑ वुप॒दधा॑ती॒मे । \newline
35. वि॒राजा॒विति॑ वि - राजौ᳚ । \newline
36. उ॒प॒दधा॑ती॒मे इ॒मे उ॑प॒दधा᳚ त्युप॒दधा॑ती॒मे ए॒वैवेमे उ॑प॒दधा᳚ त्युप॒दधा॑ती॒मे ए॒व । \newline
37. उ॒प॒दधा॒तीत्यु॑प - दधा॑ति । \newline
38. इ॒मे ए॒वैवेमे इ॒मे ए॒वोपो पै॒वेमे इ॒मे ए॒वोप॑ । \newline
39. इ॒मे इती॒मे । \newline
40. ए॒वोपो पै॒वैवोप॑ धत्ते धत्त॒ उपै॒वै वोप॑ धत्ते । \newline
41. उप॑ धत्ते धत्त॒ उपोप॑ धत्ते॒ यद् यद् ध॑त्त॒ उपोप॑ धत्ते॒ यत् । \newline
42. ध॒त्ते॒ यद् यद् ध॑त्ते धत्ते॒ यद् वै वै यद् ध॑त्ते धत्ते॒ यद् वै । \newline
43. यद् वै वै यद् यद् वा अ॒सा व॒सौ वै यद् यद् वा अ॒सौ । \newline
44. वा अ॒सा व॒सौ वै वा अ॒सौ रेतो॒ रेतो॒ ऽसौ वै वा अ॒सौ रेतः॑ । \newline
45. अ॒सौ रेतो॒ रेतो॒ ऽसा व॒सौ रेतः॑ सि॒ञ्चति॑ सि॒ञ्चति॒ रेतो॒ ऽसा व॒सौ रेतः॑ सि॒ञ्चति॑ । \newline
46. रेतः॑ सि॒ञ्चति॑ सि॒ञ्चति॒ रेतो॒ रेतः॑ सि॒ञ्चति॒ तत् तथ् सि॒ञ्चति॒ रेतो॒ रेतः॑ सि॒ञ्चति॒ तत् । \newline
47. सि॒ञ्चति॒ तत् तथ् सि॒ञ्चति॑ सि॒ञ्चति॒ तद॒स्या म॒स्याम् तथ् सि॒ञ्चति॑ सि॒ञ्चति॒ तद॒स्याम् । \newline
48. तद॒स्या म॒स्याम् तत् तद॒स्याम् प्रति॒ प्रत्य॒स्याम् तत् तद॒स्याम् प्रति॑ । \newline
49. अ॒स्याम् प्रति॒ प्रत्य॒स्या म॒स्याम् प्रति॑ तिष्ठति तिष्ठति॒ प्रत्य॒स्या म॒स्याम् प्रति॑ तिष्ठति । \newline
50. प्रति॑ तिष्ठति तिष्ठति॒ प्रति॒ प्रति॑ तिष्ठति॒ तत् तत् ति॑ष्ठति॒ प्रति॒ प्रति॑ तिष्ठति॒ तत् । \newline
51. ति॒ष्ठ॒ति॒ तत् तत् ति॑ष्ठति तिष्ठति॒ तत् प्र प्र तत् ति॑ष्ठति तिष्ठति॒ तत् प्र । \newline
52. तत् प्र प्र तत् तत् प्र जा॑यते जायते॒ प्र तत् तत् प्र जा॑यते । \newline
53. प्र जा॑यते जायते॒ प्र प्र जा॑यते॒ ता स्ता जा॑यते॒ प्र प्र जा॑यते॒ ताः । \newline
54. जा॒य॒ते॒ ता स्ता जा॑यते जायते॒ ता ओष॑धय॒ ओष॑धय॒ स्ता जा॑यते जायते॒ ता ओष॑धयः । \newline
55. ता ओष॑धय॒ ओष॑धय॒ स्ता स्ता ओष॑धयो वी॒रुधो॑ वी॒रुध॒ ओष॑धय॒ स्ता स्ता ओष॑धयो वी॒रुधः॑ । \newline
56. ओष॑धयो वी॒रुधो॑ वी॒रुध॒ ओष॑धय॒ ओष॑धयो वी॒रुधो॑ भवन्ति भवन्ति वी॒रुध॒ ओष॑धय॒ ओष॑धयो वी॒रुधो॑ भवन्ति । \newline
\pagebreak
\markright{ TS 5.5.4.2  \hfill https://www.vedavms.in \hfill}

\section{ TS 5.5.4.2 }

\textbf{TS 5.5.4.2 } \newline
\textbf{Samhita Paata} \newline

वी॒रुधो॑ भवन्ति॒ ता अ॒ग्निर॑त्ति॒ य ए॒वं ॅवेद॒ प्रैव जा॑यतेऽन्ना॒दो भ॑वति॒ यो रे॑त॒स्वी स्यात् प्र॑थ॒मायां॒ तस्य॒ चित्या॑मु॒भे उप॑ दद्ध्यादि॒मे ए॒वास्मै॑ स॒मीची॒ रेतः॑ सिञ्चतो॒ यः सि॒क्तरे॑ताः॒ स्यात् प्र॑थ॒मायां॒ तस्य॒ चित्या॑म॒न्यामुप॑ दद्ध्यादुत्त॒माया॑म॒न्याꣳ रेत॑ ए॒वास्य॑ सि॒क्तमा॒भ्यामु॑भ॒यतः॒ परि॑ गृह्णाति संॅवथ्स॒रं न क - [  ] \newline

\textbf{Pada Paata} \newline

वी॒रुधः॑ । भ॒व॒न्ति॒ । ताः । अ॒ग्निः । अ॒त्ति॒ । यः । ए॒वम् । वेद॑ । प्रेति॑ । ए॒व । जा॒य॒ते॒ । अ॒न्ना॒द इत्य॑न्न - अ॒दः । भ॒व॒ति॒ । यः । रे॒त॒स्वी । स्यात् । प्र॒थ॒माया᳚म् । तस्य॑ । चित्या᳚म् । उ॒भे इति॑ । उपेति॑ । द॒द्ध्या॒त् । इ॒मे इति॑ । ए॒व । अ॒स्मै॒ । स॒मीची॒ इति॑ । रेतः॑ । सि॒ञ्च॒तः॒ । यः । सि॒क्तरे॑ता॒ इति॑ सि॒क्त - रे॒ताः॒ । स्यात् । प्र॒थ॒माया᳚म् । तस्य॑ । चित्या᳚म् । अ॒न्याम् । उपेति॑ । द॒द्ध्या॒त् । उ॒त्त॒माया॒मित्यु॑त्-त॒माया᳚म् । अ॒न्याम् । रेतः॑ । ए॒व । अ॒स्य॒ । सि॒क्तम् । आ॒भ्याम् । उ॒भ॒यतः॑ । परीति॑ । गृ॒ह्णा॒ति॒ । सं॒ॅव॒थ्स॒रमिति॑ सं - व॒थ्स॒रम् । न । कम् ।  \newline


\textbf{Krama Paata} \newline

वी॒रुधो॑ भवन्ति । भ॒व॒न्ति॒ ताः । ता अ॒ग्निः । अ॒ग्निर॑त्ति । अ॒त्ति॒ यः । य ए॒वम् । ए॒वम् ॅवेद॑ । वेद॒ प्र । प्रैव । ए॒व जा॑यते । जा॒य॒ते॒ऽन्ना॒दः । अ॒न्ना॒दो भ॑वति । अ॒न्ना॒द इत्य॑न्न - अ॒दः । भ॒व॒ति॒ यः । यो रे॑त॒स्वी । रे॒त॒स्वी स्यात् । स्यात् प्र॑थ॒माया᳚म् । प्र॒थ॒माया॒म् तस्य॑ । तस्य॒ चित्या᳚म् । चित्या॑मु॒भे । उ॒भे उप॑ । उ॒भे इत्यु॒भे । उप॑ दद्ध्यात् । द॒द्ध्या॒दि॒मे । इ॒मे ए॒व । इ॒मे इती॒मे । ए॒वास्मै᳚ । अ॒स्मै॒ स॒मीची᳚ । स॒मीची॒ रेतः॑ । स॒मीची॒ इति॑ स॒मीची᳚ । रेतः॑ सिञ्चतः । सि॒ञ्च॒तो॒ यः । यः सि॒क्तरे॑ताः । सि॒क्तरे॑ताः॒ स्यात् । सि॒क्तरे॑ता॒ इति॑ सि॒क्त - रे॒ताः॒ । स्यात् प्र॑थ॒माया᳚म् । प्र॒थ॒माया॒म् तस्य॑ । तस्य॒ चित्या᳚म् । चित्या॑म॒न्याम् । अ॒न्यामुप॑ । उप॑ दद्ध्यात् । द॒द्ध्या॒दु॒त्त॒माया᳚म् । उ॒त्त॒माया॑म॒न्याम् । उ॒त्त॒माया॒मित्यु॑त् - त॒माया᳚म् । अ॒न्याꣳ रेतः॑ । रेत॑ ए॒व । ए॒वास्य॑ । अ॒स्य॒ सि॒क्तम् । सि॒क्तमा॒भ्याम् । आ॒भ्यामु॑भ॒यतः॑ । उ॒भ॒यतः॒ परि॑ । परि॑ गृह्णाति । गृ॒ह्णा॒ति॒ स॒म्ॅव॒थ्स॒रम् । स॒म्ॅव॒थ्स॒रम् न । स॒म्ॅव॒थ्स॒रमिति॑ सम् - व॒थ्स॒रम् । न कम् । कम् च॒न \newline

\textbf{Jatai Paata} \newline

1. वी॒रुधो॑ भवन्ति भवन्ति वी॒रुधो॑ वी॒रुधो॑ भवन्ति । \newline
2. भ॒व॒न्ति॒ ता स्ता भ॑वन्ति भवन्ति॒ ताः । \newline
3. ता अ॒ग्नि र॒ग्नि स्ता स्ता अ॒ग्निः । \newline
4. अ॒ग्नि र॑त्त्यत् त्य॒ग्नि र॒ग्नि र॑त्ति । \newline
5. अ॒त्ति॒ यो यो᳚ ऽत्त्यत्ति॒ यः । \newline
6. य ए॒व मे॒वं ॅयो य ए॒वम् । \newline
7. ए॒वं ॅवेद॒ वेदै॒व मे॒वं ॅवेद॑ । \newline
8. वेद॒ प्र प्र वेद॒ वेद॒ प्र । \newline
9. प्रैवैव प्र प्रैव । \newline
10. ए॒व जा॑यते जायत ए॒वैव जा॑यते । \newline
11. जा॒य॒ते॒ ऽन्ना॒दो᳚ ऽन्ना॒दो जा॑यते जायते ऽन्ना॒दः । \newline
12. अ॒न्ना॒दो भ॑वति भव त्यन्ना॒दो᳚ ऽन्ना॒दो भ॑वति । \newline
13. अ॒न्ना॒द इत्य॑न्न - अ॒दः । \newline
14. भ॒व॒ति॒ यो यो भ॑वति भवति॒ यः । \newline
15. यो रे॑त॒स्वी रे॑त॒स्वी यो यो रे॑त॒स्वी । \newline
16. रे॒त॒स्वी स्याथ् स्याद् रे॑त॒स्वी रे॑त॒स्वी स्यात् । \newline
17. स्यात् प्र॑थ॒माया᳚म् प्रथ॒मायाꣳ॒॒ स्याथ् स्यात् प्र॑थ॒माया᳚म् । \newline
18. प्र॒थ॒माया॒म् तस्य॒ तस्य॑ प्रथ॒माया᳚म् प्रथ॒माया॒म् तस्य॑ । \newline
19. तस्य॒ चित्या॒म् चित्या॒म् तस्य॒ तस्य॒ चित्या᳚म् । \newline
20. चित्या॑ मु॒भे उ॒भे चित्या॒म् चित्या॑ मु॒भे । \newline
21. उ॒भे उपोपो॒भे उ॒भे उप॑ । \newline
22. उ॒भे इत्यु॒भे । \newline
23. उप॑ दद्ध्याद् दद्ध्या॒ दुपोप॑ दद्ध्यात् । \newline
24. द॒द्ध्या॒ दि॒मे इ॒मे द॑द्ध्याद् दद्ध्यादि॒मे । \newline
25. इ॒मे ए॒वैवे मे इ॒मे ए॒व । \newline
26. इ॒मे इती॒मे । \newline
27. ए॒वास्मा॑ अस्मा ए॒वै वास्मै᳚ । \newline
28. अ॒स्मै॒ स॒मीची॑ स॒मीची॑ अस्मा अस्मै स॒मीची᳚ । \newline
29. स॒मीची॒ रेतो॒ रेतः॑ स॒मीची॑ स॒मीची॒ रेतः॑ । \newline
30. स॒मीची॒ इति॑ स॒मीची᳚ । \newline
31. रेतः॑ सिञ्चतः सिञ्चतो॒ रेतो॒ रेतः॑ सिञ्चतः । \newline
32. सि॒ञ्च॒तो॒ यो यः सि॑ञ्चतः सिञ्चतो॒ यः । \newline
33. यः सि॒क्तरे॑ताः सि॒क्तरे॑ता॒ यो यः सि॒क्तरे॑ताः । \newline
34. सि॒क्तरे॑ताः॒ स्याथ् स्याथ् सि॒क्तरे॑ताः सि॒क्तरे॑ताः॒ स्यात् । \newline
35. सि॒क्तरे॑ता॒ इति॑ सि॒क्त - रे॒ताः॒ । \newline
36. स्यात् प्र॑थ॒माया᳚म् प्रथ॒मायाꣳ॒॒ स्याथ् स्यात् प्र॑थ॒माया᳚म् । \newline
37. प्र॒थ॒माया॒म् तस्य॒ तस्य॑ प्रथ॒माया᳚म् प्रथ॒माया॒म् तस्य॑ । \newline
38. तस्य॒ चित्या॒म् चित्या॒म् तस्य॒ तस्य॒ चित्या᳚म् । \newline
39. चित्या॑ म॒न्या म॒न्याम् चित्या॒म् चित्या॑ म॒न्याम् । \newline
40. अ॒न्या मुपोपा॒न्या म॒न्या मुप॑ । \newline
41. उप॑ दद्ध्याद् दद्ध्या॒ दुपोप॑ दद्ध्यात् । \newline
42. द॒द्ध्या॒ दु॒त्त॒माया॑ मुत्त॒माया᳚म् दद्ध्याद् दद्ध्या दुत्त॒माया᳚म् । \newline
43. उ॒त्त॒माया॑ म॒न्या म॒न्या मु॑त्त॒माया॑ मुत्त॒माया॑ म॒न्याम् । \newline
44. उ॒त्त॒माया॒मित्यु॑त् - त॒माया᳚म् । \newline
45. अ॒न्याꣳ रेतो॒ रेतो॒ ऽन्या म॒न्याꣳ रेतः॑ । \newline
46. रेत॑ ए॒वैव रेतो॒ रेत॑ ए॒व । \newline
47. ए॒वास्या᳚ स्यै॒वै वास्य॑ । \newline
48. अ॒स्य॒ सि॒क्तꣳ सि॒क्त म॑स्यास्य सि॒क्तम् । \newline
49. सि॒क्त मा॒भ्या मा॒भ्याꣳ सि॒क्तꣳ सि॒क्त मा॒भ्याम् । \newline
50. आ॒भ्या मु॑भ॒यत॑ उभ॒यत॑ आ॒भ्या मा॒भ्या मु॑भ॒यतः॑ । \newline
51. उ॒भ॒यतः॒ परि॒ पर्यु॑भ॒यत॑ उभ॒यतः॒ परि॑ । \newline
52. परि॑ गृह्णाति गृह्णाति॒ परि॒ परि॑ गृह्णाति । \newline
53. गृ॒ह्णा॒ति॒ सं॒ॅव॒थ्स॒रꣳ सं॑ॅवथ्स॒रम् गृ॑ह्णाति गृह्णाति संॅवथ्स॒रम् । \newline
54. सं॒ॅव॒थ्स॒रम् न न सं॑ॅवथ्स॒रꣳ सं॑ॅवथ्स॒रम् न । \newline
55. सं॒ॅव॒थ्स॒रमिति॑ सं - व॒थ्स॒रम् । \newline
56. न कम् कम् न न कम् । \newline
57. कम् च॒न च॒न कम् कम् च॒न । \newline

\textbf{Ghana Paata } \newline

1. वी॒रुधो॑ भवन्ति भवन्ति वी॒रुधो॑ वी॒रुधो॑ भवन्ति॒ ता स्ता भ॑वन्ति वी॒रुधो॑ वी॒रुधो॑ भवन्ति॒ ताः । \newline
2. भ॒व॒न्ति॒ ता स्ता भ॑वन्ति भवन्ति॒ ता अ॒ग्नि र॒ग्नि स्ता भ॑वन्ति भवन्ति॒ ता अ॒ग्निः । \newline
3. ता अ॒ग्नि र॒ग्नि स्ता स्ता अ॒ग्नि र॑त्त्य त्त्य॒ग्नि स्ता स्ता अ॒ग्नि र॑त्ति । \newline
4. अ॒ग्नि र॑त्त्य त्त्य॒ग्नि र॒ग्नि र॑त्ति॒ यो यो᳚ ऽत्त्य॒ग्नि र॒ग्नि र॑त्ति॒ यः । \newline
5. अ॒त्ति॒ यो यो᳚ ऽत्त्यत्ति॒ य ए॒व मे॒वं ॅयो᳚ ऽत्त्यत्ति॒ य ए॒वम् । \newline
6. य ए॒व मे॒वं ॅयो य ए॒वं ॅवेद॒ वेदै॒वं ॅयो य ए॒वं ॅवेद॑ । \newline
7. ए॒वं ॅवेद॒ वेदै॒व मे॒वं ॅवेद॒ प्र प्र वेदै॒व मे॒वं ॅवेद॒ प्र । \newline
8. वेद॒ प्र प्र वेद॒ वेद॒ प्रैवैव प्र वेद॒ वेद॒ प्रैव । \newline
9. प्रैवैव प्र प्रैव जा॑यते जायत ए॒व प्र प्रैव जा॑यते । \newline
10. ए॒व जा॑यते जायत ए॒वैव जा॑यते ऽन्ना॒दो᳚ ऽन्ना॒दो जा॑यत ए॒वैव जा॑यते ऽन्ना॒दः । \newline
11. जा॒य॒ते॒ ऽन्ना॒दो᳚ ऽन्ना॒दो जा॑यते जायते ऽन्ना॒दो भ॑वति भव त्यन्ना॒दो जा॑यते जायते ऽन्ना॒दो भ॑वति । \newline
12. अ॒न्ना॒दो भ॑वति भव त्यन्ना॒दो᳚ ऽन्ना॒दो भ॑वति॒ यो यो भ॑व त्यन्ना॒दो᳚ ऽन्ना॒दो भ॑वति॒ यः । \newline
13. अ॒न्ना॒द इत्य॑न्न - अ॒दः । \newline
14. भ॒व॒ति॒ यो यो भ॑वति भवति॒ यो रे॑त॒स्वी रे॑त॒स्वी यो भ॑वति भवति॒ यो रे॑त॒स्वी । \newline
15. यो रे॑त॒स्वी रे॑त॒स्वी यो यो रे॑त॒स्वी स्याथ् स्याद् रे॑त॒स्वी यो यो रे॑त॒स्वी स्यात् । \newline
16. रे॒त॒स्वी स्याथ् स्याद् रे॑त॒स्वी रे॑त॒स्वी स्यात् प्र॑थ॒माया᳚म् प्रथ॒मायाꣳ॒॒ स्याद् रे॑त॒स्वी रे॑त॒स्वी स्यात् प्र॑थ॒माया᳚म् । \newline
17. स्यात् प्र॑थ॒माया᳚म् प्रथ॒मायाꣳ॒॒ स्याथ् स्यात् प्र॑थ॒माया॒म् तस्य॒ तस्य॑ प्रथ॒मायाꣳ॒॒ स्याथ् स्यात् प्र॑थ॒माया॒म् तस्य॑ । \newline
18. प्र॒थ॒माया॒म् तस्य॒ तस्य॑ प्रथ॒माया᳚म् प्रथ॒माया॒म् तस्य॒ चित्या॒म् चित्या॒म् तस्य॑ प्रथ॒माया᳚म् प्रथ॒माया॒म् तस्य॒ चित्या᳚म् । \newline
19. तस्य॒ चित्या॒म् चित्या॒म् तस्य॒ तस्य॒ चित्या॑ मु॒भे उ॒भे चित्या॒म् तस्य॒ तस्य॒ चित्या॑ मु॒भे । \newline
20. चित्या॑ मु॒भे उ॒भे चित्या॒म् चित्या॑ मु॒भे उपोपो॒भे चित्या॒म् चित्या॑ मु॒भे उप॑ । \newline
21. उ॒भे उपोपो॒भे उ॒भे उप॑ दद्ध्याद् दद्ध्या॒ दुपो॒भे उ॒भे उप॑ दद्ध्यात् । \newline
22. उ॒भे इत्यु॒भे । \newline
23. उप॑ दद्ध्याद् दद्ध्या॒ दुपोप॑ दद्ध्या दि॒मे इ॒मे द॑द्ध्या॒ दुपोप॑ दद्ध्या दि॒मे । \newline
24. द॒द्ध्या॒ दि॒मे इ॒मे द॑द्ध्याद् दद्ध्या दि॒मे ए॒वैवेमे द॑द्ध्याद् दद्ध्यादि॒मे ए॒व । \newline
25. इ॒मे ए॒वैवेमे इ॒मे ए॒वास्मा॑ अस्मा ए॒वेमे इ॒मे ए॒वास्मै᳚ । \newline
26. इ॒मे इती॒मे । \newline
27. ए॒वास्मा॑ अस्मा ए॒वै वास्मै॑ स॒मीची॑ स॒मीची॑ अस्मा ए॒वै वास्मै॑ स॒मीची᳚ । \newline
28. अ॒स्मै॒ स॒मीची॑ स॒मीची॑ अस्मा अस्मै स॒मीची॒ रेतो॒ रेतः॑ स॒मीची॑ अस्मा अस्मै स॒मीची॒ रेतः॑ । \newline
29. स॒मीची॒ रेतो॒ रेतः॑ स॒मीची॑ स॒मीची॒ रेतः॑ सिञ्चतः सिञ्चतो॒ रेतः॑ स॒मीची॑ स॒मीची॒ रेतः॑ सिञ्चतः । \newline
30. स॒मीची॒ इति॑ स॒मीची᳚ । \newline
31. रेतः॑ सिञ्चतः सिञ्चतो॒ रेतो॒ रेतः॑ सिञ्चतो॒ यो यः सि॑ञ्चतो॒ रेतो॒ रेतः॑ सिञ्चतो॒ यः । \newline
32. सि॒ञ्च॒तो॒ यो यः सि॑ञ्चतः सिञ्चतो॒ यः सि॒क्तरे॑ताः सि॒क्तरे॑ता॒ यः सि॑ञ्चतः सिञ्चतो॒ यः सि॒क्तरे॑ताः । \newline
33. यः सि॒क्तरे॑ताः सि॒क्तरे॑ता॒ यो यः सि॒क्तरे॑ताः॒ स्याथ् स्याथ् सि॒क्तरे॑ता॒ यो यः सि॒क्तरे॑ताः॒ स्यात् । \newline
34. सि॒क्तरे॑ताः॒ स्याथ् स्याथ् सि॒क्तरे॑ताः सि॒क्तरे॑ताः॒ स्यात् प्र॑थ॒माया᳚म् प्रथ॒मायाꣳ॒॒ स्याथ् सि॒क्तरे॑ताः सि॒क्तरे॑ताः॒ स्यात् प्र॑थ॒माया᳚म् । \newline
35. सि॒क्तरे॑ता॒ इति॑ सि॒क्त - रे॒ताः॒ । \newline
36. स्यात् प्र॑थ॒माया᳚म् प्रथ॒मायाꣳ॒॒ स्याथ् स्यात् प्र॑थ॒माया॒म् तस्य॒ तस्य॑ प्रथ॒मायाꣳ॒॒ स्याथ् स्यात् प्र॑थ॒माया॒म् तस्य॑ । \newline
37. प्र॒थ॒माया॒म् तस्य॒ तस्य॑ प्रथ॒माया᳚म् प्रथ॒माया॒म् तस्य॒ चित्या॒म् चित्या॒म् तस्य॑ प्रथ॒माया᳚म् प्रथ॒माया॒म् तस्य॒ चित्या᳚म् । \newline
38. तस्य॒ चित्या॒म् चित्या॒म् तस्य॒ तस्य॒ चित्या॑ म॒न्या म॒न्याम् चित्या॒म् तस्य॒ तस्य॒ चित्या॑ म॒न्याम् । \newline
39. चित्या॑ म॒न्या म॒न्याम् चित्या॒म् चित्या॑ म॒न्या मुपोपा॒न्याम् चित्या॒म् चित्या॑ म॒न्या मुप॑ । \newline
40. अ॒न्या मुपोपा॒न्या म॒न्या मुप॑ दद्ध्याद् दद्ध्या॒ दुपा॒ न्या म॒न्या मुप॑ दद्ध्यात् । \newline
41. उप॑ दद्ध्याद् दद्ध्या॒ दुपोप॑ दद्ध्या दुत्त॒माया॑ मुत्त॒माया᳚म् दद्ध्या॒ दुपोप॑ दद्ध्या दुत्त॒माया᳚म् । \newline
42. द॒द्ध्या॒ दु॒त्त॒माया॑ मुत्त॒माया᳚म् दद्ध्याद् दद्ध्या दुत्त॒माया॑ म॒न्या म॒न्या मु॑त्त॒माया᳚म् दद्ध्याद् दद्ध्या दुत्त॒माया॑ म॒न्याम् । \newline
43. उ॒त्त॒माया॑ म॒न्या म॒न्या मु॑त्त॒माया॑ मुत्त॒माया॑ म॒न्याꣳ रेतो॒ रेतो॒ ऽन्या मु॑त्त॒माया॑ मुत्त॒माया॑ म॒न्याꣳ रेतः॑ । \newline
44. उ॒त्त॒माया॒मित्यु॑त् - त॒माया᳚म् । \newline
45. अ॒न्याꣳ रेतो॒ रेतो॒ ऽन्या म॒न्याꣳ रेत॑ ए॒वैव रेतो॒ ऽन्या म॒न्याꣳ रेत॑ ए॒व । \newline
46. रेत॑ ए॒वैव रेतो॒ रेत॑ ए॒वास्या᳚ स्यै॒व रेतो॒ रेत॑ ए॒वास्य॑ । \newline
47. ए॒वास्या᳚ स्यै॒वै वास्य॑ सि॒क्तꣳ सि॒क्त म॑स्यै॒ वैवास्य॑ सि॒क्तम् । \newline
48. अ॒स्य॒ सि॒क्तꣳ सि॒क्त म॑स्यास्य सि॒क्त मा॒भ्या मा॒भ्याꣳ सि॒क्त म॑स्यास्य सि॒क्त मा॒भ्याम् । \newline
49. सि॒क्त मा॒भ्या मा॒भ्याꣳ सि॒क्तꣳ सि॒क्त मा॒भ्या मु॑भ॒यत॑ उभ॒यत॑ आ॒भ्याꣳ सि॒क्तꣳ सि॒क्त मा॒भ्या मु॑भ॒यतः॑ । \newline
50. आ॒भ्या मु॑भ॒यत॑ उभ॒यत॑ आ॒भ्या मा॒भ्या मु॑भ॒यतः॒ परि॒ पर्यु॑भ॒यत॑ आ॒भ्या मा॒भ्या मु॑भ॒यतः॒ परि॑ । \newline
51. उ॒भ॒यतः॒ परि॒ पर्यु॑भ॒यत॑ उभ॒यतः॒ परि॑ गृह्णाति गृह्णाति॒ पर्यु॑भ॒यत॑ उभ॒यतः॒ परि॑ गृह्णाति । \newline
52. परि॑ गृह्णाति गृह्णाति॒ परि॒ परि॑ गृह्णाति संॅवथ्स॒रꣳ सं॑ॅवथ्स॒रम् गृ॑ह्णाति॒ परि॒ परि॑ गृह्णाति संॅवथ्स॒रम् । \newline
53. गृ॒ह्णा॒ति॒ सं॒ॅव॒थ्स॒रꣳ सं॑ॅवथ्स॒रम् गृ॑ह्णाति गृह्णाति संॅवथ्स॒रम् न न सं॑ॅवथ्स॒रम् गृ॑ह्णाति गृह्णाति संॅवथ्स॒रम् न । \newline
54. सं॒ॅव॒थ्स॒रम् न न सं॑ॅवथ्स॒रꣳ सं॑ॅवथ्स॒रम् न कम् कम् न सं॑ॅवथ्स॒रꣳ सं॑ॅवथ्स॒रम् न कम् । \newline
55. सं॒ॅव॒थ्स॒रमिति॑ सं - व॒थ्स॒रम् । \newline
56. न कम् कम् न न कम् च॒न च॒न कम् न न कम् च॒न । \newline
57. कम् च॒न च॒न कम् कम् च॒न प्र॒त्यव॑रोहेत् प्र॒त्यव॑रोहेच् च॒न कम् कम् च॒न प्र॒त्यव॑रोहेत् । \newline
\pagebreak
\markright{ TS 5.5.4.3  \hfill https://www.vedavms.in \hfill}

\section{ TS 5.5.4.3 }

\textbf{TS 5.5.4.3 } \newline
\textbf{Samhita Paata} \newline

ञ्च॒न प्र॒त्यव॑रोहे॒न्न हीमे कञ्च॒न प्र॑त्यव॒रोह॑त॒स्तदे॑नयोर्व्र॒तं ॅयो वा अप॑ शीर्.षाणम॒ग्निं चि॑नु॒तेऽप॑शीर्.षा॒ऽमुष्मि॑न् ॅलो॒के भ॑वति॒ यः सशी॑र्.षाणं चिनु॒ते सशी॑र्.षा॒ ऽमुष्मि॑न् ॅलो॒के भ॑वति॒ चित्तिं॑ जुहोमि॒ मन॑सा घृ॒तेन॒ यथा॑ दे॒वा इ॒हाऽऽ*गम॑न् वी॒तिहो᳚त्रा ऋता॒वृधः॑ समु॒द्रस्य॑ व॒युन॑स्य॒ पत्म॑न् जु॒होमि॑ वि॒श्वक॑र्मणे॒ विश्वाऽहाऽम॑र्त्यꣳ ह॒विरिति॑ स्वयमातृ॒ण्णामु॑प॒धाय॑ जुहोत्ये॒ - [  ] \newline

\textbf{Pada Paata} \newline

च॒न । प्र॒त्यव॑रोहे॒दिति॑ प्रति - अव॑रोहेत् । न । हि । इ॒मे इति॑ । कम् । च॒न । प्र॒त्य॒व॒रोह॑त॒ इति॑ प्रति - अ॒व॒रोह॑तः । तत् । ए॒न॒योः॒ । व्र॒तम् । यः । वै । अप॑शीर्.षाण॒मित्यप॑ - शी॒र्.॒षा॒ण॒म् । अ॒ग्निम् । चि॒नु॒ते । अप॑शी॒र्॒.षेत्यप॑ - शी॒र्.॒षा॒ । अ॒मुष्मिन्न्॑ । लो॒के । भ॒व॒ति॒ । यः । सशी॑र्.षाण॒मिति॒ स - शी॒र्.॒षा॒ण॒म् । चि॒नु॒ते । सशी॒र्॒.षेति॒ स - शी॒र्.॒षा॒ । अ॒मुष्मिन्न्॑ । लो॒के । भ॒व॒ति॒ । चित्ति᳚म् । जु॒हो॒मि॒ । मन॑सा । घृ॒तेन॑ । यथा᳚ । दे॒वाः । इ॒ह । आ॒गम॒न्नित्या᳚ - गमन्न्॑ । वी॒तिहो᳚त्रा॒ इति॑ वी॒ति - हो॒त्राः॒ । ऋ॒ता॒वृध॒ इत्यृ॑त - वृधः॑ । स॒मु॒द्रस्य॑ । व॒युन॑स्य । पत्मन्न्॑ । जु॒होमि॑ । वि॒श्वक॑र्मण॒ इति॑ वि॒श्व - क॒र्म॒णे॒ । विश्वा᳚ । अहा᳚ । अम॑र्त्यम् । ह॒विः । इति॑ । स्व॒य॒मा॒तृ॒ण्णामिति॑ स्वयं - आ॒तृ॒ण्णाम् । उ॒प॒धायेत्यु॑प - धाय॑ । जु॒हो॒ति॒ ।  \newline


\textbf{Krama Paata} \newline

च॒न प्र॒त्यव॑रोहेत् । प्र॒त्यव॑रोहे॒न् न । प्र॒त्यव॑रोहे॒दिति॑ प्रति - अव॑रोहेत् । न हि । हीमे । इ॒मे कम् । इ॒मे इती॒मे । कम् च॒न । च॒न प्र॑त्यव॒रोह॑तः । प्र॒त्य॒व॒रोह॑त॒स्तत् । प्र॒त्य॒व॒रोह॑त॒ इति॑ प्रति - अ॒व॒रोह॑तः । तदे॑नयोः । ए॒न॒यो॒र् व्र॒तम् । व्र॒तम् ॅयः । यो वै । वा अप॑शीर्.षाणम् । अप॑शीर्.षाणम॒ग्निम् । अप॑शीर्.षाण॒मित्यप॑ - शी॒र्॒.षा॒ण॒म् । अ॒ग्निम् चि॑नु॒ते । चि॒नु॒तेऽप॑शीर्.षा । अप॑शिर्.षा॒ऽमुष्मिन्न्॑ । अप॑शी॒र्॒.षेत्यप॑ - शी॒र्॒.षा॒ । अ॒मुष्मि॑न् ॅलो॒के । लो॒के भ॑वति । भ॒व॒ति॒ यः । यः सशी॑र्.षाणम् । सशी॑र्.षाणम् चिनु॒ते । सशी॑र्.षाण॒मिति॒ स - शी॒र्॒.षा॒ण॒म् । चि॒नु॒ते सशी॑र्.षा । सशी॑र्.षा॒ऽमुष्मिन्न्॑ । सशी॒र्.षेति॒ स - शी॒र्॒.षा॒ । अ॒मुष्मि॑न् ॅलो॒के । लो॒के भ॑वति । भ॒व॒ति॒ चित्ति᳚म् । चित्ति॑म् जुहोमि । जु॒हो॒मि॒ मन॑सा । मन॑सा घृ॒तेन॑ । घृ॒तेन॒ यथा᳚ । यथा॑ दे॒वाः । दे॒वा इ॒ह । इ॒हागमन्न्॑ । आ॒गम॑न् वी॒तिहो᳚त्राः । आ॒गम॒न्नित्या᳚ - गमन्न्॑ । वी॒तिहो᳚त्रा ऋता॒वृधः॑ । वी॒तिहो᳚त्रा॒ इति॑ वी॒ति - हो॒त्राः॒ । ऋ॒ता॒वृधः॑ समु॒द्रस्य॑ । ऋ॒ता॒वृध॒ इत्यृ॑त - वृधः॑ । स॒मु॒द्रस्य॑ व॒युन॑स्य । व॒युन॑स्य॒ पत्मन्न्॑ । पत्म॑न् जु॒होमि॑ । जु॒होमि॑ वि॒श्वक॑र्मणे । वि॒श्वक॑र्मणे॒ विश्वा᳚ । वि॒श्वक॑र्मण॒ इति॑ वि॒श्व - क॒र्म॒णे॒ । विश्वाऽहा᳚ । अहाऽम॑र्त्यम् । अम॑र्त्यꣳ ह॒विः । ह॒विरिति॑ । इति॑ स्वयमातृ॒ण्णाम् । स्व॒य॒मा॒तृ॒ण्णामु॑प॒धाय॑ । स्व॒य॒मा॒तृ॒ण्णामिति॑ स्वयम् - आ॒तृ॒ण्णाम् । उ॒प॒धाय॑ जुहोति । उ॒प॒धायेत्यु॑प - धाय॑ । जु॒हो॒त्ये॒तत् \newline

\textbf{Jatai Paata} \newline

1. च॒न प्र॒त्यव॑रोहेत् प्र॒त्यव॑रोहेच् च॒न च॒न प्र॒त्यव॑रोहेत् । \newline
2. प्र॒त्यव॑रोहे॒न् न न प्र॒त्यव॑रोहेत् प्र॒त्यव॑रोहे॒न् न । \newline
3. प्र॒त्यव॑रोहे॒दिति॑ प्रति - अव॑रोहेत् । \newline
4. न हि हि न न हि । \newline
5. हीमे इ॒मे हि हीमे । \newline
6. इ॒मे कम् क मि॒मे इ॒मे कम् । \newline
7. इ॒मे इती॒मे । \newline
8. कम् च॒न च॒न कम् कम् च॒न । \newline
9. च॒न प्र॑त्यव॒रोह॑तः प्रत्यव॒रोह॑त श्च॒न च॒न प्र॑त्यव॒रोह॑तः । \newline
10. प्र॒त्य॒व॒रोह॑त॒ स्तत् तत् प्र॑त्यव॒रोह॑तः प्रत्यव॒रोह॑त॒ स्तत् । \newline
11. प्र॒त्य॒व॒रोह॑त॒ इति॑ प्रति - अ॒व॒रोह॑तः । \newline
12. तदे॑नयो रेनयो॒ स्तत् तदे॑नयोः । \newline
13. ए॒न॒यो॒र् व्र॒तं ॅव्र॒त मे॑नयो रेनयोर् व्र॒तम् । \newline
14. व्र॒तं ॅयो यो व्र॒तं ॅव्र॒तं ॅयः । \newline
15. यो वै वै यो यो वै । \newline
16. वा अप॑शीर्.षाण॒ मप॑शीर्.षाणं॒ ॅवै वा अप॑शीर्.षाणम् । \newline
17. अप॑शीर्.षाण म॒ग्नि म॒ग्नि मप॑शीर्.षाण॒ मप॑शीर्.षाण म॒ग्निम् । \newline
18. अप॑शीर्.षाण॒मित्यप॑ - शी॒र्.॒षा॒ण॒म् । \newline
19. अ॒ग्निम् चि॑नु॒ते चि॑नु॒ते᳚ ऽग्नि म॒ग्निम् चि॑नु॒ते । \newline
20. चि॒नु॒ते ऽप॑शी॒र्॒.षा ऽप॑शीर्.षा चिनु॒ते चि॑नु॒ते ऽप॑शीर्.षा । \newline
21. अप॑शीर्.षा॒ ऽमुष्मि॑न् न॒मुष्मि॒न् नप॑शी॒र्॒.षा ऽप॑शीर्.षा॒ ऽमुष्मिन्न्॑ । \newline
22. अप॑शी॒र्॒.षेत्यप॑ - शी॒र्.॒षा॒ । \newline
23. अ॒मुष्मि॑न् ॅलो॒के लो॒के॑ ऽमुष्मि॑न् न॒मुष्मि॑न् ॅलो॒के । \newline
24. लो॒के भ॑वति भवति लो॒के लो॒के भ॑वति । \newline
25. भ॒व॒ति॒ यो यो भ॑वति भवति॒ यः । \newline
26. यः सशी॑र्.षाणꣳ॒॒ सशी॑र्.षाणं॒ ॅयो यः सशी॑र्.षाणम् । \newline
27. सशी॑र्.षाणम् चिनु॒ते चि॑नु॒ते सशी॑र्.षाणꣳ॒॒ सशी॑र्.षाणम् चिनु॒ते । \newline
28. सशी॑र्.षाण॒मिति॒ स - शी॒र्.॒षा॒ण॒म् । \newline
29. चि॒नु॒ते सशी॑र्.षा॒ सशी॑र्.षा चिनु॒ते चि॑नु॒ते सशी॑र्.षा । \newline
30. सशी॑र्.षा॒ ऽमुष्मि॑न् न॒मुष्मि॒न् थ्सशी॑र्.षा॒ सशी॑र्.षा॒ ऽमुष्मिन्न्॑ । \newline
31. सशी॒र्॒.षेति॒ स - शी॒र्.॒षा॒ । \newline
32. अ॒मुष्मि॑न् ॅलो॒के लो॒के॑ ऽमुष्मि॑न् न॒मुष्मि॑न् ॅलो॒के । \newline
33. लो॒के भ॑वति भवति लो॒के लो॒के भ॑वति । \newline
34. भ॒व॒ति॒ चित्ति॒म् चित्ति॑म् भवति भवति॒ चित्ति᳚म् । \newline
35. चित्ति॑म् जुहोमि जुहोमि॒ चित्ति॒म् चित्ति॑म् जुहोमि । \newline
36. जु॒हो॒मि॒ मन॑सा॒ मन॑सा जुहोमि जुहोमि॒ मन॑सा । \newline
37. मन॑सा घृ॒तेन॑ घृ॒तेन॒ मन॑सा॒ मन॑सा घृ॒तेन॑ । \newline
38. घृ॒तेन॒ यथा॒ यथा॑ घृ॒तेन॑ घृ॒तेन॒ यथा᳚ । \newline
39. यथा॑ दे॒वा दे॒वा यथा॒ यथा॑ दे॒वाः । \newline
40. दे॒वा इ॒हेह दे॒वा दे॒वा इ॒ह । \newline
41. इ॒हागम॑न् ना॒गम॑न् नि॒हे हागमन्न्॑ । \newline
42. आ॒गम॑न् वी॒तिहो᳚त्रा वी॒तिहो᳚त्रा आ॒गम॑न् ना॒गम॑न् वी॒तिहो᳚त्राः । \newline
43. आ॒गम॒न्नित्या᳚ - गमन्न्॑ । \newline
44. वी॒तिहो᳚त्रा ऋता॒वृध॑ ऋता॒वृधो॑ वी॒तिहो᳚त्रा वी॒तिहो᳚त्रा 
ऋता॒वृधः॑ । \newline
45. वी॒तिहो᳚त्रा॒ इति॑ वी॒ति - हो॒त्राः॒ । \newline
46. ऋ॒ता॒वृधः॑ समु॒द्रस्य॑ समु॒द्रस्य॑ र्‌ता॒वृध॑ ऋता॒वृधः॑ समु॒द्रस्य॑ । \newline
47. ऋ॒ता॒वृध॒ इत्यृ॑त - वृधः॑ । \newline
48. स॒मु॒द्रस्य॑ व॒युन॑स्य व॒युन॑स्य समु॒द्रस्य॑ समु॒द्रस्य॑ व॒युन॑स्य । \newline
49. व॒युन॑स्य॒ पत्म॒न् पत्म॑न्. व॒युन॑स्य व॒युन॑स्य॒ पत्मन्न्॑ । \newline
50. पत्म॑न् जु॒होमि॑ जु॒होमि॒ पत्म॒न् पत्म॑न् जु॒होमि॑ । \newline
51. जु॒होमि॑ वि॒श्वक॑र्मणे वि॒श्वक॑र्मणे जु॒होमि॑ जु॒होमि॑ वि॒श्वक॑र्मणे । \newline
52. वि॒श्वक॑र्मणे॒ विश्वा॒ विश्वा॑ वि॒श्वक॑र्मणे वि॒श्वक॑र्मणे॒ विश्वा᳚ । \newline
53. वि॒श्वक॑र्मण॒ इति॑ वि॒श्व - क॒र्म॒णे॒ । \newline
54. विश्वा ऽहा ऽहा॒ विश्वा॒ विश्वा ऽहा᳚ । \newline
55. अहा ऽम॑र्त्य॒ मम॑र्त्य॒ महा ऽहा ऽम॑र्त्यम् । \newline
56. अम॑र्त्यꣳ ह॒विर्. ह॒वि रम॑र्त्य॒ मम॑र्त्यꣳ ह॒विः । \newline
57. ह॒वि रितीति॑ ह॒विर्. ह॒विरिति॑ । \newline
58. इति॑ स्वयमातृ॒ण्णाꣳ स्व॑यमातृ॒ण्णा मितीति॑ स्वयमातृ॒ण्णाम् । \newline
59. स्व॒य॒मा॒तृ॒ण्णा मु॑प॒धायो॑प॒धाय॑ स्वयमातृ॒ण्णाꣳ स्व॑यमातृ॒ण्णा मु॑प॒धाय॑ । \newline
60. स्व॒य॒मा॒तृ॒ण्णामिति॑ स्वयं - आ॒तृ॒ण्णाम् । \newline
61. उ॒प॒धाय॑ जुहोति जुहो त्युप॒धा यो॑प॒धाय॑ जुहोति । \newline
62. उ॒प॒धायेत्यु॑प - धाय॑ । \newline
63. जु॒हो॒ त्ये॒ तदे॒तज् जु॑होति जुहो त्ये॒तत् । \newline

\textbf{Ghana Paata } \newline

1. च॒न प्र॒त्यव॑रोहेत् प्र॒त्यव॑रोहेच् च॒न च॒न प्र॒त्यव॑रोहे॒न् न न प्र॒त्यव॑रोहेच् च॒न च॒न प्र॒त्यव॑रोहे॒न् न । \newline
2. प्र॒त्यव॑रोहे॒न् न न प्र॒त्यव॑रोहेत् प्र॒त्यव॑रोहे॒न् न हि हि न प्र॒त्यव॑रोहेत् प्र॒त्यव॑रोहे॒न् न हि । \newline
3. प्र॒त्यव॑रोहे॒दिति॑ प्रति - अव॑रोहेत् । \newline
4. न हि हि न न हीमे इ॒मे हि न न हीमे । \newline
5. हीमे इ॒मे हि हीमे कम् कमि॒मे हि हीमे कम् । \newline
6. इ॒मे कम् क मि॒मे इ॒मे कम् च॒न च॒न क मि॒मे इ॒मे कम् च॒न । \newline
7. इ॒मे इती॒मे । \newline
8. कम् च॒न च॒न कम् कम् च॒न प्र॑त्यव॒रोह॑तः प्रत्यव॒रोह॑त श्च॒न कम् कम् च॒न प्र॑त्यव॒रोह॑तः । \newline
9. च॒न प्र॑त्यव॒रोह॑तः प्रत्यव॒रोह॑त श्च॒न च॒न प्र॑त्यव॒रोह॑त॒ स्तत् तत् प्र॑त्यव॒रोह॑त श्च॒न च॒न प्र॑त्यव॒रोह॑त॒ स्तत् । \newline
10. प्र॒त्य॒व॒रोह॑त॒ स्तत् तत् प्र॑त्यव॒रोह॑तः प्रत्यव॒रोह॑त॒ स्तदे॑नयो रेनयो॒ स्तत् प्र॑त्यव॒रोह॑तः प्रत्यव॒रोह॑त॒ स्तदे॑नयोः । \newline
11. प्र॒त्य॒व॒रोह॑त॒ इति॑ प्रति - अ॒व॒रोह॑तः । \newline
12. तदे॑नयो रेनयो॒ स्तत् तदे॑नयोर् व्र॒तं ॅव्र॒त मे॑नयो॒ स्तत् तदे॑नयोर् व्र॒तम् । \newline
13. ए॒न॒यो॒र् व्र॒तं ॅव्र॒त मे॑नयो रेनयोर् व्र॒तं ॅयो यो व्र॒त मे॑नयो रेनयोर् व्र॒तं ॅयः । \newline
14. व्र॒तं ॅयो यो व्र॒तं ॅव्र॒तं ॅयो वै वै यो व्र॒तं ॅव्र॒तं ॅयो वै । \newline
15. यो वै वै यो यो वा अप॑शीर्.षाण॒ मप॑शीर्.षाणं॒ ॅवै यो यो वा अप॑शीर्.षाणम् । \newline
16. वा अप॑शीर्.षाण॒ मप॑शीर्.षाणं॒ ॅवै वा अप॑शीर्.षाण म॒ग्नि म॒ग्नि मप॑शीर्.षाणं॒ ॅवै वा अप॑शीर्.षाण म॒ग्निम् । \newline
17. अप॑शीर्.षाण म॒ग्नि म॒ग्नि मप॑शीर्.षाण॒ मप॑शीर्.षाण म॒ग्निम् चि॑नु॒ते चि॑नु॒ते᳚ ऽग्नि मप॑शीर्.षाण॒ मप॑शीर्.षाण म॒ग्निम् चि॑नु॒ते । \newline
18. अप॑शीर्.षाण॒मित्यप॑ - शी॒र्.॒षा॒ण॒म् । \newline
19. अ॒ग्निम् चि॑नु॒ते चि॑नु॒ते᳚ ऽग्नि म॒ग्निम् चि॑नु॒ते ऽप॑शी॒र्॒.षा ऽप॑शीर्.षा चिनु॒ते᳚ ऽग्नि म॒ग्निम् चि॑नु॒ते ऽप॑शीर्.षा । \newline
20. चि॒नु॒ते ऽप॑शी॒र्॒.षा ऽप॑शीर्.षा चिनु॒ते चि॑नु॒ते ऽप॑शीर्.षा॒ ऽमुष्मि॑न् न॒मुष्मि॒न् नप॑शीर्.षा चिनु॒ते चि॑नु॒ते ऽप॑शीर्.षा॒ ऽमुष्मिन्न्॑ । \newline
21. अप॑शीर्.षा॒ ऽमुष्मि॑न् न॒मुष्मि॒न् नप॑शी॒र्॒.षा ऽप॑शीर्.षा॒ ऽमुष्मि॑न् ॅलो॒के लो॒के॑ ऽमुष्मि॒न् नप॑शी॒र्॒.षा ऽप॑शीर्.षा॒ ऽमुष्मि॑न् ॅलो॒के । \newline
22. अप॑शी॒र्॒.षेत्यप॑ - शी॒र्.॒षा॒ । \newline
23. अ॒मुष्मि॑न् ॅलो॒के लो॒के॑ ऽमुष्मि॑न् न॒मुष्मि॑न् ॅलो॒के भ॑वति भवति लो॒के॑ ऽमुष्मि॑न् न॒मुष्मि॑न् ॅलो॒के भ॑वति । \newline
24. लो॒के भ॑वति भवति लो॒के लो॒के भ॑वति॒ यो यो भ॑वति लो॒के लो॒के भ॑वति॒ यः । \newline
25. भ॒व॒ति॒ यो यो भ॑वति भवति॒ यः सशी॑र्.षाणꣳ॒॒ सशी॑र्.षाणं॒ ॅयो भ॑वति भवति॒ यः सशी॑र्.षाणम् । \newline
26. यः सशी॑र्.षाणꣳ॒॒ सशी॑र्.षाणं॒ ॅयो यः सशी॑र्.षाणम् चिनु॒ते चि॑नु॒ते सशी॑र्.षाणं॒ ॅयो यः सशी॑र्.षाणम् चिनु॒ते । \newline
27. सशी॑र्.षाणम् चिनु॒ते चि॑नु॒ते सशी॑र्.षाणꣳ॒॒ सशी॑र्.षाणम् चिनु॒ते सशी॑र्.षा॒ सशी॑र्.षा चिनु॒ते सशी॑र्.षाणꣳ॒॒ सशी॑र्.षाणम् चिनु॒ते सशी॑र्.षा । \newline
28. सशी॑र्.षाण॒मिति॒ स - शी॒र्.॒षा॒ण॒म् । \newline
29. चि॒नु॒ते सशी॑र्.षा॒ सशी॑र्.षा चिनु॒ते चि॑नु॒ते सशी॑र्.षा॒ ऽमुष्मि॑न् न॒मुष्मि॒न् थ्सशी॑र्.षा चिनु॒ते चि॑नु॒ते सशी॑र्.षा॒ ऽमुष्मिन्न्॑ । \newline
30. सशी॑र्.षा॒ ऽमुष्मि॑न् न॒मुष्मि॒न् थ्सशी॑र्.षा॒ सशी॑र्.षा॒ ऽमुष्मि॑न् ॅलो॒के लो॒के॑ ऽमुष्मि॒न् थ्सशी॑र्.षा॒ सशी॑र्.षा॒ ऽमुष्मि॑न् ॅलो॒के । \newline
31. सशी॒र्॒.षेति॒ स - शी॒र्.॒षा॒ । \newline
32. अ॒मुष्मि॑न् ॅलो॒के लो॒के॑ ऽमुष्मि॑न् न॒मुष्मि॑न् ॅलो॒के भ॑वति भवति लो॒के॑ ऽमुष्मि॑न् न॒मुष्मि॑न् ॅलो॒के भ॑वति । \newline
33. लो॒के भ॑वति भवति लो॒के लो॒के भ॑वति॒ चित्ति॒म् चित्ति॑म् भवति लो॒के लो॒के भ॑वति॒ चित्ति᳚म् । \newline
34. भ॒व॒ति॒ चित्ति॒म् चित्ति॑म् भवति भवति॒ चित्ति॑म् जुहोमि जुहोमि॒ चित्ति॑म् भवति भवति॒ चित्ति॑म् जुहोमि । \newline
35. चित्ति॑म् जुहोमि जुहोमि॒ चित्ति॒म् चित्ति॑म् जुहोमि॒ मन॑सा॒ मन॑सा जुहोमि॒ चित्ति॒म् चित्ति॑म् जुहोमि॒ मन॑सा । \newline
36. जु॒हो॒मि॒ मन॑सा॒ मन॑सा जुहोमि जुहोमि॒ मन॑सा घृ॒तेन॑ घृ॒तेन॒ मन॑सा जुहोमि जुहोमि॒ मन॑सा घृ॒तेन॑ । \newline
37. मन॑सा घृ॒तेन॑ घृ॒तेन॒ मन॑सा॒ मन॑सा घृ॒तेन॒ यथा॒ यथा॑ घृ॒तेन॒ मन॑सा॒ मन॑सा घृ॒तेन॒ यथा᳚ । \newline
38. घृ॒तेन॒ यथा॒ यथा॑ घृ॒तेन॑ घृ॒तेन॒ यथा॑ दे॒वा दे॒वा यथा॑ घृ॒तेन॑ घृ॒तेन॒ यथा॑ दे॒वाः । \newline
39. यथा॑ दे॒वा दे॒वा यथा॒ यथा॑ दे॒वा इ॒हेह दे॒वा यथा॒ यथा॑ दे॒वा इ॒ह । \newline
40. दे॒वा इ॒हेह दे॒वा दे॒वा इ॒हागम॑न् ना॒गम॑न् नि॒ह दे॒वा दे॒वा इ॒हागमन्न्॑ । \newline
41. इ॒हागम॑न् ना॒गम॑न् नि॒हे हागम॑न् वी॒तिहो᳚त्रा वी॒तिहो᳚त्रा आ॒गम॑न् नि॒हे हागम॑न् वी॒तिहो᳚त्राः । \newline
42. आ॒गम॑न् वी॒तिहो᳚त्रा वी॒तिहो᳚त्रा आ॒गम॑न् ना॒गम॑न् वी॒तिहो᳚त्रा ऋता॒वृध॑ ऋता॒वृधो॑ वी॒तिहो᳚त्रा आ॒गम॑न् ना॒गम॑न् वी॒तिहो᳚त्रा ऋता॒वृधः॑ । \newline
43. आ॒गम॒न्नित्या᳚ - गमन्न्॑ । \newline
44. वी॒तिहो᳚त्रा ऋता॒वृध॑ ऋता॒वृधो॑ वी॒तिहो᳚त्रा वी॒तिहो᳚त्रा ऋता॒वृधः॑ समु॒द्रस्य॑ समु॒द्रस्य॑ र्‌ता॒वृधो॑ वी॒तिहो᳚त्रा वी॒तिहो᳚त्रा ऋता॒वृधः॑ समु॒द्रस्य॑ । \newline
45. वी॒तिहो᳚त्रा॒ इति॑ वी॒ति - हो॒त्राः॒ । \newline
46. ऋ॒ता॒वृधः॑ समु॒द्रस्य॑ समु॒द्रस्य॑ र्‌ता॒वृध॑ ऋता॒वृधः॑ समु॒द्रस्य॑ व॒युन॑स्य व॒युन॑स्य समु॒द्रस्य॑ र्‌ता॒वृध॑ ऋता॒वृधः॑ समु॒द्रस्य॑ व॒युन॑स्य । \newline
47. ऋ॒ता॒वृध॒ इत्यृ॑त - वृधः॑ । \newline
48. स॒मु॒द्रस्य॑ व॒युन॑स्य व॒युन॑स्य समु॒द्रस्य॑ समु॒द्रस्य॑ व॒युन॑स्य॒ पत्म॒न् पत्म॑न्. व॒युन॑स्य समु॒द्रस्य॑ समु॒द्रस्य॑ व॒युन॑स्य॒ पत्मन्न्॑ । \newline
49. व॒युन॑स्य॒ पत्म॒न् पत्म॑न्. व॒युन॑स्य व॒युन॑स्य॒ पत्म॑न् जु॒होमि॑ जु॒होमि॒ पत्म॑न्. व॒युन॑स्य व॒युन॑स्य॒ पत्म॑न् जु॒होमि॑ । \newline
50. पत्म॑न् जु॒होमि॑ जु॒होमि॒ पत्म॒न् पत्म॑न् जु॒होमि॑ वि॒श्वक॑र्मणे वि॒श्वक॑र्मणे जु॒होमि॒ पत्म॒न् पत्म॑न् जु॒होमि॑ वि॒श्वक॑र्मणे । \newline
51. जु॒होमि॑ वि॒श्वक॑र्मणे वि॒श्वक॑र्मणे जु॒होमि॑ जु॒होमि॑ वि॒श्वक॑र्मणे॒ विश्वा॒ विश्वा॑ वि॒श्वक॑र्मणे जु॒होमि॑ जु॒होमि॑ वि॒श्वक॑र्मणे॒ विश्वा᳚ । \newline
52. वि॒श्वक॑र्मणे॒ विश्वा॒ विश्वा॑ वि॒श्वक॑र्मणे वि॒श्वक॑र्मणे॒ विश्वा ऽहा ऽहा॒ विश्वा॑ वि॒श्वक॑र्मणे वि॒श्वक॑र्मणे॒ विश्वा ऽहा᳚ । \newline
53. वि॒श्वक॑र्मण॒ इति॑ वि॒श्व - क॒र्म॒णे॒ । \newline
54. विश्वा ऽहा ऽहा॒ विश्वा॒ विश्वा ऽहा ऽम॑र्त्य॒ मम॑र्त्य॒ महा॒ विश्वा॒ विश्वा ऽहा ऽम॑र्त्यम् । \newline
55. अहा ऽम॑र्त्य॒ मम॑र्त्य॒ महा ऽहा ऽम॑र्त्यꣳ ह॒विर्. ह॒वि रम॑र्त्य॒ महा ऽहा ऽम॑र्त्यꣳ ह॒विः । \newline
56. अम॑र्त्यꣳ ह॒विर्. ह॒वि रम॑र्त्य॒ मम॑र्त्यꣳ ह॒वि रितीति॑ ह॒वि रम॑र्त्य॒ मम॑र्त्यꣳ ह॒वि रिति॑ । \newline
57. ह॒वि रितीति॑ ह॒विर्. ह॒विरिति॑ स्वयमातृ॒ण्णाꣳ स्व॑यमातृ॒ण्णा मिति॑ ह॒विर्. ह॒विरिति॑ स्वयमातृ॒ण्णाम् । \newline
58. इति॑ स्वयमातृ॒ण्णाꣳ स्व॑यमातृ॒ण्णा मितीति॑ स्वयमातृ॒ण्णा मु॑प॒धा यो॑प॒धाय॑ स्वयमातृ॒ण्णा मितीति॑ स्वयमातृ॒ण्णा मु॑प॒धाय॑ । \newline
59. स्व॒य॒मा॒तृ॒ण्णा मु॑प॒धा यो॑प॒धाय॑ स्वयमातृ॒ण्णाꣳ स्व॑यमातृ॒ण्णा मु॑प॒धाय॑ जुहोति जुहो त्युप॒धाय॑ स्वयमातृ॒ण्णाꣳ स्व॑यमातृ॒ण्णा मु॑प॒धाय॑ जुहोति । \newline
60. स्व॒य॒मा॒तृ॒ण्णामिति॑ स्वयं - आ॒तृ॒ण्णाम् । \newline
61. उ॒प॒धाय॑ जुहोति जुहो त्युप॒धा यो॑प॒धाय॑ जुहो त्ये॒त दे॒तज् जु॑हो त्युप॒धा यो॑प॒धाय॑ जुहो त्ये॒तत् । \newline
62. उ॒प॒धायेत्यु॑प - धाय॑ । \newline
63. जु॒हो॒ त्ये॒त दे॒तज् जु॑होति जुहो त्ये॒तद् वै वा ए॒तज् जु॑होति जुहो त्ये॒तद् वै । \newline
\pagebreak
\markright{ TS 5.5.4.4  \hfill https://www.vedavms.in \hfill}

\section{ TS 5.5.4.4 }

\textbf{TS 5.5.4.4 } \newline
\textbf{Samhita Paata} \newline

तद्वा अ॒ग्नेः शिरः॒ सशी॑र्.षाणमे॒वाग्निं चि॑नुते॒ सशी॑र्.षा॒ऽमुष्मि॑न् ॅलो॒के भ॑वति॒ य ए॒वं ॅवेद॑ सुव॒र्गाय॒ वा ए॒ष लो॒काय॑ चीयते॒ यद॒ग्निस्तस्य॒ यदय॑थापूर्वं क्रि॒यते ऽसु॑वर्ग्यमस्य॒ तथ् सु॑व॒र्ग्यो᳚ ऽग्निश्चिति॑मुप॒धाया॒भि मृ॑शे॒च्चित्ति॒मचि॑त्तिं चिनव॒द्वि वि॒द्वान् पृ॒ष्ठेव॑ वी॒ता वृ॑जि॒ना च॒ मर्ता᳚न् रा॒ये च॑ नः स्वप॒त्याय॑ ( ) देव॒ दितिं॑ च॒ रास्वा-दि॑तिमुरु॒ष्येति॑ यथापू॒र्वमे॒वैना॒मुप॑ धत्ते॒ प्राञ्च॑मेनं चिनुते सुव॒र्ग्यो᳚ऽस्य भवति ॥ \newline

\textbf{Pada Paata} \newline

ए॒तत् । वै । अ॒ग्नेः । शिरः॑ । सशी॑र्.षाण॒मिति॒ स - शी॒र्.॒षा॒ण॒म् । ए॒व । अ॒ग्निम् । चि॒नु॒ते॒ । सशी॒र्.॒षेति॒ स - शी॒र्.॒षा॒ । अ॒मुष्मिन्न्॑ । लो॒के । भ॒व॒ति॒ । यः । ए॒वम् । वेद॑ । सु॒व॒र्गायेति॑ सुवः - गाय॑ । वै । ए॒षः । लो॒काय॑ । ची॒य॒ते॒ । यत् । अ॒ग्निः । तस्य॑ । यत् । अय॑थापूर्व॒मित्यय॑था - पू॒र्व॒म् । क्रि॒यते᳚ । असु॑वर्ग्य॒मित्यसु॑वः-ग्य॒म् । अ॒स्य॒ । तत् । सु॒व॒ग्य॑ इति॑ सुवः - ग्यः॑ । अ॒ग्निः । चिति᳚म् । उ॒प॒धायेत्यु॑प - धाय॑ । अ॒भीति॑ । मृ॒शे॒त् । चित्ति᳚म् । अचि॑त्तिम् । चि॒न॒व॒त् । वीति॑ । वि॒द्वान् । पृ॒ष्ठा । इ॒व॒ । वी॒ता । वृ॒जि॒ना । च॒ । मर्तान्॑ । रा॒ये । च॒ । नः॒ । स्व॒प॒त्यायेति॑ सु - अ॒प॒त्याय॑ ( ) । दे॒व॒ । दिति᳚म् । च॒ । रास्व॑ । अदि॑तिम् । उ॒रु॒ष्य॒ । इति॑ । य॒था॒पू॒र्वमिति॑ यथा - पू॒र्वम् । ए॒व । ए॒ना॒म् । उपेति॑ । ध॒त्ते॒ । प्राञ्च᳚म् । ए॒न॒म् । चि॒नु॒ते॒ । सु॒व॒र्ग्य॑ इति॑ सुवः - ग्यः॑ । अ॒स्य॒ । भ॒व॒ति॒ ॥  \newline


\textbf{Krama Paata} \newline

ए॒तद् वै । वा अ॒ग्नेः । अ॒ग्नेः शिरः॑ । शिरः॒ सशी॑र्.षाणम् । सशी॑र्.षाणमे॒व । सशी॑र्.षाण॒मिति॒ स - शी॒र्॒.षा॒ण॒म् । ए॒वाग्निम् । अ॒ग्निम् चि॑नुते । चि॒नु॒ते॒ सशी॑र्.षा । सशी॑र्.षा॒ऽमुष्मिन्न्॑ । सशी॒र्॒.षेति॒ स - शी॒र्॒.षा॒ । अ॒मुष्मि॑न् ॅलो॒के । लो॒के भ॑वति । भ॒व॒ति॒ यः । य ए॒वम् । ए॒वम् ॅवेद॑ । वेद॑ सुव॒र्गाय॑ । सु॒व॒र्गाय॒ वै । सु॒व॒र्गायेति॑ सुवः - गाय॑ । वा ए॒षः । ए॒ष लो॒काय॑ । लो॒काय॑ चीयते । ची॒य॒ते॒ यत् । यद॒ग्निः । अ॒ग्निस्तस्य॑ । तस्य॒ यत् । यदय॑थापूर्वम् । अय॑थापूर्वम् क्रि॒यते᳚ । अय॑थापूर्व॒मित्यय॑था - पू॒र्व॒म् । क्रि॒यतेऽसु॑वर्ग्यम् । असु॑वर्ग्यमस्य । असु॑वर्ग्य॒मित्यसु॑वः - ग्य॒म् । अ॒स्य॒ तत् । तथ् सु॑व॒र्ग्यः॑ । सु॒व॒र्ग्यो᳚ऽग्निः । सु॒व॒र्ग्य॑ इति॑ सुवः - ग्यः॑ । अ॒ग्निश्चिति᳚म् । चिति॑मुप॒धाय॑ । उ॒प॒धाया॒भि । उ॒प॒धायेत्यु॑प - धाय॑ । अ॒भि मृ॑शेत् । मृ॒शे॒च् चित्ति᳚म् । चित्ति॒मचि॑त्तिम् । अचि॑त्तिम् चिनवत् । चि॒न॒व॒द् वि । वि वि॒द्वान् । वि॒द्वान् पृ॒ष्ठा । पृ॒ष्ठेव॑ । इ॒व॒ वी॒ता । वी॒ता वृ॑जि॒ना । वृ॒जि॒ना च॑ । च॒ मर्तान्॑ । मर्ता᳚न् रा॒ये । रा॒ये च॑ । च॒ नः॒ । नः॒ स्व॒प॒त्याय॑ ( ) । स्व॒प॒त्याय॑ देव । स्व॒प॒त्यायेति॑ सु - अ॒प॒त्याय॑ । दे॒व॒ दिति᳚म् । दिति॑म् च । च॒ रास्व॑ । रास्वादि॑तिम् । अदि॑तिमुरुष्य । उ॒रु॒ष्येति॑ । इति॑ यथापू॒र्वम् । य॒था॒पू॒र्वमे॒व । य॒था॒पू॒र्वमिति॑ यथा - पू॒र्वम् । ए॒वैना᳚म् । ए॒ना॒मुप॑ । उप॑ धत्ते । ध॒त्ते॒ प्राञ्च᳚म् । प्राञ्च॑मेनम् । ए॒न॒म् चि॒नु॒ते॒ । चि॒नु॒ते॒ सु॒व॒र्ग्यः॑ । सु॒व॒र्ग्यो᳚ऽस्य । सु॒व॒र्ग्य॑ इति॑ सुवः - ग्यः॑ । अ॒स्य॒ भ॒व॒ति॒ । भ॒व॒तीति॑ भवति । \newline

\textbf{Jatai Paata} \newline

1. ए॒तद् वै वा ए॒त दे॒तद् वै । \newline
2. वा अ॒ग्ने र॒ग्नेर् वै वा अ॒ग्नेः । \newline
3. अ॒ग्नेः शिरः॒ शिरो॒ ऽग्ने र॒ग्नेः शिरः॑ । \newline
4. शिरः॒ सशी॑र्.षाणꣳ॒॒ सशी॑र्.षाणꣳ॒॒ शिरः॒ शिरः॒ सशी॑र्.षाणम् । \newline
5. सशी॑र्.षाण मे॒वैव सशी॑र्.षाणꣳ॒॒ सशी॑र्.षाण मे॒व । \newline
6. सशी॑र्.षाण॒मिति॒ स - शी॒र्.॒षा॒ण॒म् । \newline
7. ए॒वाग्नि म॒ग्नि मे॒वै वाग्निम् । \newline
8. अ॒ग्निम् चि॑नुते चिनुते॒ ऽग्नि म॒ग्निम् चि॑नुते । \newline
9. चि॒नु॒ते॒ सशी॑र्.षा॒ सशी॑र्.षा चिनुते चिनुते॒ सशी॑र्.षा । \newline
10. सशी॑र्.षा॒ ऽमुष्मि॑न् न॒मुष्मि॒न् थ्सशी॑र्.षा॒ सशी॑र्.षा॒ ऽमुष्मिन्न्॑ । \newline
11. सशी॒र्.॒षेति॒ स - शी॒र्.॒षा॒ । \newline
12. अ॒मुष्मि॑न् ॅलो॒के लो॒के॑ ऽमुष्मि॑न् न॒मुष्मि॑न् ॅलो॒के । \newline
13. लो॒के भ॑वति भवति लो॒के लो॒के भ॑वति । \newline
14. भ॒व॒ति॒ यो यो भ॑वति भवति॒ यः । \newline
15. य ए॒व मे॒वं ॅयो य ए॒वम् । \newline
16. ए॒वं ॅवेद॒ वेदै॒व मे॒वं ॅवेद॑ । \newline
17. वेद॑ सुव॒र्गाय॑ सुव॒र्गाय॒ वेद॒ वेद॑ सुव॒र्गाय॑ । \newline
18. सु॒व॒र्गाय॒ वै वै सु॑व॒र्गाय॑ सुव॒र्गाय॒ वै । \newline
19. सु॒व॒र्गायेति॑ सुवः - गाय॑ । \newline
20. वा ए॒ष ए॒ष वै वा ए॒षः । \newline
21. ए॒ष लो॒काय॑ लो॒कायै॒ष ए॒ष लो॒काय॑ । \newline
22. लो॒काय॑ चीयते चीयते लो॒काय॑ लो॒काय॑ चीयते । \newline
23. ची॒य॒ते॒ यद् यच् ची॑यते चीयते॒ यत् । \newline
24. Yअद॒ग्नि र॒ग्निर् यद् यद॒ग्निः । \newline
25. अ॒ग्नि स्तस्य॒ तस्या॒ग्नि र॒ग्नि स्तस्य॑ । \newline
26. तस्य॒ यद् यत् तस्य॒ तस्य॒ यत् । \newline
27. यदय॑थापूर्व॒ मय॑थापूर्वं॒ ॅयद् यदय॑थापूर्वम् । \newline
28. अय॑थापूर्वम् क्रि॒यते᳚ क्रि॒यते ऽय॑थापूर्व॒ मय॑थापूर्वम् क्रि॒यते᳚ । \newline
29. अय॑थापूर्व॒मित्यय॑था - पू॒र्व॒म् । \newline
30. क्रि॒यते ऽसु॑वर्ग्य॒ मसु॑वर्ग्यम् क्रि॒यते᳚ क्रि॒यते ऽसु॑वर्ग्यम् । \newline
31. असु॑वर्ग्य मस्या॒स्या सु॑वर्ग्य॒ मसु॑वर्ग्य मस्य । \newline
32. असु॑वर्ग्य॒मित्यसु॑वः - ग्य॒म् । \newline
33. अ॒स्य॒ तत् तद॑स्यास्य॒ तत् । \newline
34. तथ् सु॑व॒ग्यः॑ सुव॒ग्य॑ स्तत् तथ् सु॑व॒ग्यः॑ । \newline
35. सु॒व॒ग्यो᳚ ऽग्नि र॒ग्निः सु॑व॒ग्यः॑ सुव॒ग्यो᳚ ऽग्निः । \newline
36. सु॒व॒ग्य॑ इति॑ सुवः - ग्यः॑ । \newline
37. अ॒ग्नि श्चिति॒म् चिति॑ म॒ग्नि र॒ग्नि श्चिति᳚म् । \newline
38. चिति॑ मुप॒धायो॑ प॒धाय॒ चिति॒म् चिति॑ मुप॒धाय॑ । \newline
39. उ॒प॒धा या॒भ्या᳚(1॒)भ्यु॑प॒धा यो॑प॒धाया॒भि । \newline
40. उ॒प॒धायेत्यु॑प - धाय॑ । \newline
41. अ॒भि मृ॑शेन् मृशे द॒भ्य॑भि मृ॑शेत् । \newline
42. मृ॒शे॒च् चित्ति॒म् चित्ति॑म् मृशेन् मृशे॒च् चित्ति᳚म् । \newline
43. चित्ति॒ मचि॑त्ति॒ मचि॑त्ति॒म् चित्ति॒म् चित्ति॒ मचि॑त्तिम् । \newline
44. अचि॑त्तिम् चिनवच् चिनव॒ दचि॑त्ति॒ मचि॑त्तिम् चिनवत् । \newline
45. चि॒न॒व॒द् वि वि चि॑नवच् चिनव॒द् वि । \newline
46. वि वि॒द्वान्. वि॒द्वान्. वि वि वि॒द्वान् । \newline
47. वि॒द्वान् पृ॒ष्ठा पृ॒ष्ठा वि॒द्वान्. वि॒द्वान् पृ॒ष्ठा । \newline
48. पृ॒ष्ठेवे॑व पृ॒ष्ठा पृ॒ष्ठेव॑ । \newline
49. इ॒व॒ वी॒ता वी॒तेवे॑व वी॒ता । \newline
50. वी॒ता वृ॑जि॒ना वृ॑जि॒ना वी॒ता वी॒ता वृ॑जि॒ना । \newline
51. वृ॒जि॒ना च॑ च वृजि॒ना वृ॑जि॒ना च॑ । \newline
52. च॒ मर्ता॒न् मर्ताꣳ॑श्च च॒ मर्तान्॑ । \newline
53. मर्ता᳚न् रा॒ये रा॒ये मर्ता॒न् मर्ता᳚न् रा॒ये । \newline
54. रा॒ये च॑ च रा॒ये रा॒ये च॑ । \newline
55. च॒ नो॒ न॒श्च॒ च॒ नः॒ । \newline
56. नः॒ स्व॒प॒त्याय॑ स्वप॒त्याय॑ नो नः स्वप॒त्याय॑ । \newline
57. स्व॒प॒त्याय॑ देव देव स्वप॒त्याय॑ स्वप॒त्याय॑ देव । \newline
58. स्व॒प॒त्यायेति॑ सु - अ॒प॒त्याय॑ । \newline
59. दे॒व॒ दिति॒म् दिति॑म् देव देव॒ दिति᳚म् । \newline
60. दिति॑म् च च॒ दिति॒म् दिति॑म् च । \newline
61. च॒ रास्व॒ रास्व॑ च च॒ रास्व॑ । \newline
62. रास्वादि॑ति॒ मदि॑तिꣳ॒॒ रास्व॒ रास्वादि॑तिम् । \newline
63. अदि॑ति मुरुष्यो रु॒ष्यादि॑ति॒ मदि॑ति मुरुष्य । \newline
64. उ॒रु॒ष्ये तीत्यु॑ रुष्यो रु॒ष्येति॑ । \newline
65. इति॑ यथापू॒र्वं ॅय॑थापू॒र्व मितीति॑ यथापू॒र्वम् । \newline
66. य॒था॒पू॒र्व मे॒वैव य॑थापू॒र्वं ॅय॑थापू॒र्व मे॒व । \newline
67. य॒था॒पू॒र्वमिति॑ यथा - पू॒र्वम् । \newline
68. ए॒वैना॑ मेना मे॒वै वैना᳚म् । \newline
69. ए॒ना॒ मुपोपै॑ना मेना॒ मुप॑ । \newline
70. उप॑ धत्ते धत्त॒ उपोप॑ धत्ते । \newline
71. ध॒त्ते॒ प्राञ्च॒म् प्राञ्च॑म् धत्ते धत्ते॒ प्राञ्च᳚म् । \newline
72. प्राञ्च॑ मेन मेन॒म् प्राञ्च॒म् प्राञ्च॑ मेनम् । \newline
73. ए॒न॒म् चि॒नु॒ते॒ चि॒नु॒त॒ ए॒न॒ मे॒न॒म् चि॒नु॒ते॒ । \newline
74. चि॒नु॒ते॒ सु॒व॒र्ग्यः॑ सुव॒र्ग्य॑ श्चिनुते चिनुते सुव॒र्ग्यः॑ । \newline
75. सु॒व॒र्ग्यो᳚ ऽस्यास्य सुव॒र्ग्यः॑ सुव॒र्ग्यो᳚ ऽस्य । \newline
76. सु॒व॒र्ग्य॑ इति॑ सुवः - ग्यः॑ । \newline
77. अ॒स्य॒ भ॒व॒ति॒ भ॒व॒ त्य॒स्या॒स्य॒ भ॒व॒ति॒ । \newline
78. भ॒व॒तीति॑ भवति । \newline

\textbf{Ghana Paata } \newline

1. ए॒तद् वै वा ए॒त दे॒तद् वा अ॒ग्ने र॒ग्नेर् वा ए॒त दे॒तद् वा अ॒ग्नेः । \newline
2. वा अ॒ग्ने र॒ग्नेर् वै वा अ॒ग्नेः शिरः॒ शिरो॒ ऽग्नेर् वै वा अ॒ग्नेः शिरः॑ । \newline
3. अ॒ग्नेः शिरः॒ शिरो॒ ऽग्ने र॒ग्नेः शिरः॒ सशी॑र्.षाणꣳ॒॒ सशी॑र्.षाणꣳ॒॒ शिरो॒ ऽग्ने र॒ग्नेः शिरः॒ सशी॑र्.षाणम् । \newline
4. शिरः॒ सशी॑र्.षाणꣳ॒॒ सशी॑र्.षाणꣳ॒॒ शिरः॒ शिरः॒ सशी॑र्.षाण मे॒वैव सशी॑र्.षाणꣳ॒॒ शिरः॒ शिरः॒ सशी॑र्.षाण मे॒व । \newline
5. सशी॑र्.षाण मे॒वैव सशी॑र्.षाणꣳ॒॒ सशी॑र्.षाण मे॒वाग्नि म॒ग्नि मे॒व सशी॑र्.षाणꣳ॒॒ सशी॑र्.षाण मे॒वाग्निम् । \newline
6. सशी॑र्.षाण॒मिति॒ स - शी॒र्.॒षा॒ण॒म् । \newline
7. ए॒वाग्नि म॒ग्नि मे॒वै वाग्निम् चि॑नुते चिनुते॒ ऽग्नि मे॒वै वाग्निम् चि॑नुते । \newline
8. अ॒ग्निम् चि॑नुते चिनुते॒ ऽग्नि म॒ग्निम् चि॑नुते॒ सशी॑र्.षा॒ सशी॑र्.षा चिनुते॒ ऽग्नि म॒ग्निम् चि॑नुते॒ सशी॑र्.षा । \newline
9. चि॒नु॒ते॒ सशी॑र्.षा॒ सशी॑र्.षा चिनुते चिनुते॒ सशी॑र्.षा॒ ऽमुष्मि॑न् न॒मुष्मि॒न् थ्सशी॑र्.षा चिनुते चिनुते॒ सशी॑र्.षा॒ ऽमुष्मिन्न्॑ । \newline
10. सशी॑र्.षा॒ ऽमुष्मि॑न् न॒मुष्मि॒न् थ्सशी॑र्.षा॒ सशी॑र्.षा॒ ऽमुष्मि॑न् ॅलो॒के लो॒के॑ ऽमुष्मि॒न् थ्सशी॑र्.षा॒ सशी॑र्.षा॒ ऽमुष्मि॑न् ॅलो॒के । \newline
11. सशी॒र्.॒षेति॒ स - शी॒र्.॒षा॒ । \newline
12. अ॒मुष्मि॑न् ॅलो॒के लो॒के॑ ऽमुष्मि॑न् न॒मुष्मि॑न् ॅलो॒के भ॑वति भवति लो॒के॑ ऽमुष्मि॑न् न॒मुष्मि॑न् ॅलो॒के भ॑वति । \newline
13. लो॒के भ॑वति भवति लो॒के लो॒के भ॑वति॒ यो यो भ॑वति लो॒के लो॒के भ॑वति॒ यः । \newline
14. भ॒व॒ति॒ यो यो भ॑वति भवति॒ य ए॒व मे॒वं ॅयो भ॑वति भवति॒ य ए॒वम् । \newline
15. य ए॒व मे॒वं ॅयो य ए॒वं ॅवेद॒ वेदै॒वं ॅयो य ए॒वं ॅवेद॑ । \newline
16. ए॒वं ॅवेद॒ वेदै॒व मे॒वं ॅवेद॑ सुव॒र्गाय॑ सुव॒र्गाय॒ वेदै॒व मे॒वं ॅवेद॑ सुव॒र्गाय॑ । \newline
17. वेद॑ सुव॒र्गाय॑ सुव॒र्गाय॒ वेद॒ वेद॑ सुव॒र्गाय॒ वै वै सु॑व॒र्गाय॒ वेद॒ वेद॑ सुव॒र्गाय॒ वै । \newline
18. सु॒व॒र्गाय॒ वै वै सु॑व॒र्गाय॑ सुव॒र्गाय॒ वा ए॒ष ए॒ष वै सु॑व॒र्गाय॑ सुव॒र्गाय॒ वा ए॒षः । \newline
19. सु॒व॒र्गायेति॑ सुवः - गाय॑ । \newline
20. वा ए॒ष ए॒ष वै वा ए॒ष लो॒काय॑ लो॒कायै॒ष वै वा ए॒ष लो॒काय॑ । \newline
21. ए॒ष लो॒काय॑ लो॒कायै॒ष ए॒ष लो॒काय॑ चीयते चीयते लो॒कायै॒ष ए॒ष लो॒काय॑ चीयते । \newline
22. लो॒काय॑ चीयते चीयते लो॒काय॑ लो॒काय॑ चीयते॒ यद् यच् ची॑यते लो॒काय॑ लो॒काय॑ चीयते॒ यत् । \newline
23. ची॒य॒ते॒ यद् यच् ची॑यते चीयते॒ यद॒ग्नि र॒ग्निर् यच् ची॑यते चीयते॒ यद॒ग्निः । \newline
24. यद॒ग्नि र॒ग्निर् यद् यद॒ग्नि स्तस्य॒ तस्या॒ग्निर् यद् यद॒ग्नि स्तस्य॑ । \newline
25. अ॒ग्नि स्तस्य॒ तस्या॒ग्नि र॒ग्नि स्तस्य॒ यद् यत् तस्या॒ग्नि र॒ग्नि स्तस्य॒ यत् । \newline
26. तस्य॒ यद् यत् तस्य॒ तस्य॒ यदय॑थापूर्व॒ मय॑थापूर्वं॒ ॅयत् तस्य॒ तस्य॒ यदय॑थापूर्वम् । \newline
27. यदय॑थापूर्व॒ मय॑थापूर्वं॒ ॅयद् यदय॑थापूर्वम् क्रि॒यते᳚ क्रि॒यते ऽय॑थापूर्वं॒ ॅयद् यदय॑थापूर्वम् क्रि॒यते᳚ । \newline
28. अय॑थापूर्वम् क्रि॒यते᳚ क्रि॒यते ऽय॑थापूर्व॒ मय॑थापूर्वम् क्रि॒यते ऽसु॑वर्ग्य॒ मसु॑वर्ग्यम् क्रि॒यते ऽय॑थापूर्व॒ मय॑थापूर्वम् क्रि॒यते ऽसु॑वर्ग्यम् । \newline
29. अय॑थापूर्व॒मित्यय॑था - पू॒र्व॒म् । \newline
30. क्रि॒यते ऽसु॑वर्ग्य॒ मसु॑वर्ग्यम् क्रि॒यते᳚ क्रि॒यते ऽसु॑वर्ग्य मस्या॒स्या सु॑वर्ग्यम् क्रि॒यते᳚ क्रि॒यते ऽसु॑वर्ग्य मस्य । \newline
31. असु॑वर्ग्य मस्या॒स्या सु॑वर्ग्य॒ मसु॑वर्ग्य मस्य॒ तत् तद॒स्या सु॑वर्ग्य॒ मसु॑वर्ग्य मस्य॒ तत् । \newline
32. असु॑वर्ग्य॒मित्यसु॑वः - ग्य॒म् । \newline
33. अ॒स्य॒ तत् तद॑स्या स्य॒ तथ् सु॑व॒ग्यः॑ सुव॒ग्य॑ स्तद॑स्या स्य॒ तथ् सु॑व॒ग्यः॑ । \newline
34. तथ् सु॑व॒ग्यः॑ सुव॒ग्य॑ स्तत् तथ् सु॑व॒ग्यो᳚ ऽग्नि र॒ग्निः सु॑व॒ग्य॑ स्तत् तथ् सु॑व॒ग्यो᳚ ऽग्निः । \newline
35. सु॒व॒ग्यो᳚ ऽग्नि र॒ग्निः सु॑व॒ग्यः॑ सुव॒ग्यो᳚ ऽग्नि श्चिति॒म् चिति॑ म॒ग्निः सु॑व॒ग्यः॑ सुव॒ग्यो᳚ ऽग्नि श्चिति᳚म् । \newline
36. सु॒व॒ग्य॑ इति॑ सुवः - ग्यः॑ । \newline
37. आ॒ग्नि श्चिति॒म् चिति॑ म॒ग्नि र॒ग्नि श्चिति॑ मुप॒धा यो॑प॒धाय॒ चिति॑ म॒ग्नि र॒ग्नि श्चिति॑ मुप॒धाय॑ । \newline
38. चिति॑ मुप॒धा यो॑प॒धाय॒ चिति॒म् चिति॑ मुप॒धाया॒ भ्या᳚(1॒) भ्यु॑प॒धाय॒ चिति॒म् चिति॑ मुप॒धाया॒भि । \newline
39. उ॒प॒धाया॒ भ्या᳚(1॒) भ्यु॑प॒धा यो॑प॒धाया॒भि मृ॑शेन् मृशे द॒भ्यु॑प॒धा यो॑प॒धाया॒भि मृ॑शेत् । \newline
40. उ॒प॒धायेत्यु॑प - धाय॑ । \newline
41. अ॒भि मृ॑शेन् मृशे द॒भ्य॑भि मृ॑शे॒च् चित्ति॒म् चित्ति॑म् मृशे द॒भ्य॑भि मृ॑शे॒च् चित्ति᳚म् । \newline
42. मृ॒शे॒च् चित्ति॒म् चित्ति॑म् मृशेन् मृशे॒च् चित्ति॒ मचि॑त्ति॒ मचि॑त्ति॒म् चित्ति॑म् मृशेन् मृशे॒च् चित्ति॒ मचि॑त्तिम् । \newline
43. चित्ति॒ मचि॑त्ति॒ मचि॑त्ति॒म् चित्ति॒म् चित्ति॒ मचि॑त्तिम् चिनवच् चिनव॒ दचि॑त्ति॒म् चित्ति॒म् चित्ति॒ मचि॑त्तिम् चिनवत् । \newline
44. अचि॑त्तिम् चिनवच् चिनव॒ दचि॑त्ति॒ मचि॑त्तिम् चिनव॒द् वि वि चि॑नव॒ दचि॑त्ति॒ मचि॑त्तिम् चिनव॒द् वि । \newline
45. चि॒न॒व॒द् वि वि चि॑नवच् चिनव॒द् वि वि॒द्वान्. वि॒द्वान्. वि चि॑नवच् चिनव॒द् वि वि॒द्वान् । \newline
46. वि वि॒द्वान्. वि॒द्वान्. वि वि वि॒द्वान् पृ॒ष्ठा पृ॒ष्ठा वि॒द्वान्. वि वि वि॒द्वान् पृ॒ष्ठा । \newline
47. वि॒द्वान् पृ॒ष्ठा पृ॒ष्ठा वि॒द्वान्. वि॒द्वान् पृ॒ष्ठेवे॑व पृ॒ष्ठा वि॒द्वान्. वि॒द्वान् पृ॒ष्ठेव॑ । \newline
48. पृ॒ष्ठेवे॑व पृ॒ष्ठा पृ॒ष्ठेव॑ वी॒ता वी॒तेव॑ पृ॒ष्ठा पृ॒ष्ठेव॑ वी॒ता । \newline
49. इ॒व॒ वी॒ता वी॒तेवे॑व वी॒ता वृ॑जि॒ना वृ॑जि॒ना वी॒तेवे॑व वी॒ता वृ॑जि॒ना । \newline
50. वी॒ता वृ॑जि॒ना वृ॑जि॒ना वी॒ता वी॒ता वृ॑जि॒ना च॑ च वृजि॒ना वी॒ता वी॒ता वृ॑जि॒ना च॑ । \newline
51. वृ॒जि॒ना च॑ च वृजि॒ना वृ॑जि॒ना च॒ मर्ता॒न् मर्ताꣳ॑श्च वृजि॒ना वृ॑जि॒ना च॒ मर्तान्॑ । \newline
52. च॒ मर्ता॒न् मर्ताꣳ॑श्च च॒ मर्ता᳚न् रा॒ये रा॒ये मर्ताꣳ॑श्च च॒ मर्ता᳚न् रा॒ये । \newline
53. मर्ता᳚न् रा॒ये रा॒ये मर्ता॒न् मर्ता᳚न् रा॒ये च॑ च रा॒ये मर्ता॒न् मर्ता᳚न् रा॒ये च॑ । \newline
54. रा॒ये च॑ च रा॒ये रा॒ये च॑ नो नश्च रा॒ये रा॒ये च॑ नः । \newline
55. च॒ नो॒ न॒श्च॒ च॒ नः॒ स्व॒प॒त्याय॑ स्वप॒त्याय॑ नश्च च नः स्वप॒त्याय॑ । \newline
56. नः॒ स्व॒प॒त्याय॑ स्वप॒त्याय॑ नो नः स्वप॒त्याय॑ देव देव स्वप॒त्याय॑ नो नः स्वप॒त्याय॑ देव । \newline
57. स्व॒प॒त्याय॑ देव देव स्वप॒त्याय॑ स्वप॒त्याय॑ देव॒ दिति॒म् दिति॑म् देव स्वप॒त्याय॑ स्वप॒त्याय॑ देव॒ दिति᳚म् । \newline
58. स्व॒प॒त्यायेति॑ सु - अ॒प॒त्याय॑ । \newline
59. दे॒व॒ दिति॒म् दिति॑म् देव देव॒ दिति॑म् च च॒ दिति॑म् देव देव॒ दिति॑म् च । \newline
60. दिति॑म् च च॒ दिति॒म् दिति॑म् च॒ रास्व॒ रास्व॑ च॒ दिति॒म् दिति॑म् च॒ रास्व॑ । \newline
61. च॒ रास्व॒ रास्व॑ च च॒ रास्वादि॑ति॒ मदि॑तिꣳ॒॒ रास्व॑ च च॒ रास्वादि॑तिम् । \newline
62. रास्वादि॑ति॒ मदि॑तिꣳ॒॒ रास्व॒ रास्वादि॑ति मुरुष्यो रु॒ष्या दि॑तिꣳ॒॒ रास्व॒ रास्वादि॑ति मुरुष्य । \newline
63. अदि॑ति मुरुष्यो रु॒ष्या दि॑ति॒ मदि॑ति मुरु॒ष्ये तीत्यु॑रु॒ष्या दि॑ति॒ मदि॑ति मुरु॒ष्येति॑ । \newline
64. उ॒रु॒ष्ये तीत्यु॑रुष्यो रु॒ष्येति॑ यथापू॒र्वं ॅय॑थापू॒र्व मित्यु॑रु ष्योरु॒ष्येति॑ यथापू॒र्वम् । \newline
65. इति॑ यथापू॒र्वं ॅय॑थापू॒र्व मितीति॑ यथापू॒र्व मे॒वैव य॑थापू॒र्व मितीति॑ यथापू॒र्व मे॒व । \newline
66. य॒था॒पू॒र्व मे॒वैव य॑थापू॒र्वं ॅय॑थापू॒र्व मे॒वैना॑ मेना मे॒व य॑थापू॒र्वं ॅय॑थापू॒र्व मे॒वैना᳚म् । \newline
67. य॒था॒पू॒र्वमिति॑ यथा - पू॒र्वम् । \newline
68. ए॒वैना॑ मेना मे॒वै वैना॒ मुपो पै॑ना मे॒वै वैना॒ मुप॑ । \newline
69. ए॒ना॒ मुपो पै॑ना मेना॒ मुप॑ धत्ते धत्त॒ उपै॑ना मेना॒ मुप॑ धत्ते । \newline
70. उप॑ धत्ते धत्त॒ उपोप॑ धत्ते॒ प्राञ्च॒म् प्राञ्च॑म् धत्त॒ उपोप॑ धत्ते॒ प्राञ्च᳚म् । \newline
71. ध॒त्ते॒ प्राञ्च॒म् प्राञ्च॑म् धत्ते धत्ते॒ प्राञ्च॑ मेन मेन॒म् प्राञ्च॑म् धत्ते धत्ते॒ प्राञ्च॑ मेनम् । \newline
72. प्राञ्च॑ मेन मेन॒म् प्राञ्च॒म् प्राञ्च॑ मेनम् चिनुते चिनुत एन॒म् प्राञ्च॒म् प्राञ्च॑ मेनम् चिनुते । \newline
73. ए॒न॒म् चि॒नु॒ते॒ चि॒नु॒त॒ ए॒न॒ मे॒न॒म् चि॒नु॒ते॒ सु॒व॒र्ग्यः॑ सुव॒र्ग्य॑ श्चिनुत एन मेनम् चिनुते सुव॒र्ग्यः॑ । \newline
74. चि॒नु॒ते॒ सु॒व॒र्ग्यः॑ सुव॒र्ग्य॑ श्चिनुते चिनुते सुव॒र्ग्यो᳚ ऽस्यास्य सुव॒र्ग्य॑ श्चिनुते चिनुते सुव॒र्ग्यो᳚ ऽस्य । \newline
75. सु॒व॒र्ग्यो᳚ ऽस्यास्य सुव॒र्ग्यः॑ सुव॒र्ग्यो᳚ ऽस्य भवति भव त्यस्य सुव॒र्ग्यः॑ सुव॒र्ग्यो᳚ ऽस्य भवति । \newline
76. सु॒व॒र्ग्य॑ इति॑ सुवः - ग्यः॑ । \newline
77. अ॒स्य॒ भ॒व॒ति॒ भ॒व॒ त्य॒स्या॒स्य॒ भ॒व॒ति॒ । \newline
78. भ॒व॒तीति॑ भवति । \newline
\pagebreak
\markright{ TS 5.5.5.1  \hfill https://www.vedavms.in \hfill}

\section{ TS 5.5.5.1 }

\textbf{TS 5.5.5.1 } \newline
\textbf{Samhita Paata} \newline

वि॒श्वक॑र्मा दि॒शां पतिः॒ स नः॑ प॒शून् पा॑तु॒ सो᳚ऽस्मान् पा॑तु॒ तस्मै॒ नमः॑ प्र॒जाप॑ती रु॒द्रो वरु॑णो॒ ऽग्निर्दि॒शां पतिः॒ स नः॑ प॒शून् पा॑तु॒ सो᳚ऽस्मान् पा॑तु॒ तस्मै॒ नम॑ ए॒ता वै दे॒वता॑ ए॒तेषां᳚ पशू॒ना-मधि॑पतय॒-स्ताभ्यो॒ वा ए॒ष आ वृ॑श्च्यते॒ यः प॑शुशी॒र्॒.षाण्यु॑प॒ दधा॑ति हिरण्येष्ट॒का उप॑ दधात्ये॒ताभ्य॑ ए॒व दे॒वता᳚भ्यो॒ नम॑स्करोति ब्रह्मवा॒दिनो॑ - [  ] \newline

\textbf{Pada Paata} \newline

वि॒श्वक॒र्मेति॑ वि॒श्व - क॒र्मा॒ । दि॒शाम् । पतिः॑ । सः । नः॒ । प॒शून् । पा॒तु॒ । सः । अ॒स्मान् । पा॒तु॒ । तस्मै᳚ । नमः॑ । प्र॒जाप॑ति॒रिति॑ प्र॒जा - प॒तिः॒ । रु॒द्रः । वरु॑णः । अ॒ग्निः । दि॒शाम् । पतिः॑ । सः । नः॒ । प॒शून् । पा॒तु॒ । सः । अ॒स्मान् । पा॒तु॒ । तस्मै᳚ । नमः॑ । ए॒ताः । वै । दे॒वताः᳚ । ए॒तेषा᳚म् । प॒शू॒नाम् । अधि॑पतय॒ इत्यधि॑ - प॒त॒यः॒ । ताभ्यः॑ । वै । ए॒षः । एति॑ । वृ॒श्च्य॒ते॒ । यः । प॒शु॒शी॒र्॒.षाणीति॑ पशु - शी॒र्॒.षाणि॑ । उ॒प॒दधा॒तीत्यु॑प - दधा॑ति । हि॒र॒ण्ये॒ष्ट॒का इति॑ हिरण्य - इ॒ष्ट॒काः । उपेति॑ । द॒धा॒ति॒ । ए॒ताभ्यः॑ । ए॒व । दे॒वता᳚भ्यः । नमः॑ । क॒रो॒ति॒ । ब्र॒ह्म॒वा॒दिन॒ इति॑ ब्रह्म - वा॒दिनः॑ ।  \newline


\textbf{Krama Paata} \newline

वि॒श्वक॑र्मा दि॒शाम् । वि॒श्वक॒र्मेति॑ वि॒श्व - क॒र्मा॒ । दि॒शाम् पतिः॑ । पतिः॒ सः । स नः॑ । नः॒ प॒शून् । प॒शून् पा॑तु । पा॒तु॒ सः । सो᳚ऽस्मान् । अ॒स्मान् पा॑तु । पा॒तु॒ तस्मै᳚ । तस्मै॒ नमः॑ । नमः॑ प्र॒जाप॑तिः । प्र॒जाप॑ती रु॒द्रः । प्र॒जाप॑ति॒रिति॑ प्र॒जा - प॒तिः॒ । रु॒द्रो वरु॑णः । वरु॑णो॒ऽग्निः । अ॒ग्निर् दि॒शाम् । दि॒शाम् पतिः॑ । पतिः॒ सः । स नः॑ । नः॒ प॒शून् । प॒शून् पा॑तु । पा॒तु॒ सः । सो᳚ऽस्मान् । अ॒स्मान् पा॑तु । पा॒तु॒ तस्मै᳚ । तस्मै॒ नमः॑ । नम॑ ए॒ताः । ए॒ता वै । वै दे॒वताः᳚ । दे॒वता॑ ए॒तेषा᳚म् । ए॒तेषा᳚म् पशू॒नाम् । प॒शू॒नामधि॑पतयः । अधि॑पतय॒ स्ताभ्यः॑ । अधि॑पतय॒ इत्यधि॑ - प॒त॒यः॒ । ताभ्यो॒ वै । वा ए॒षः । ए॒ष आ । आ वृ॑श्च्यते । वृ॒श्च्य॒ते॒ यः । यः प॑शुशी॒र्॒.षाणि॑ । प॒शु॒शी॒र्॒.षाण्यु॑प॒दधा॑ति । प॒शु॒शी॒र्॒.षाणीति॑ पशु - शी॒र्॒.षाणि॑ । उ॒प॒दधा॑ति हिरण्येष्ट॒काः । उ॒प॒दधा॒तीत्यु॑प - दधा॑ति । हि॒र॒ण्ये॒ष्ट॒का उप॑ । हि॒र॒ण्ये॒ष्ट॒का इति॑ हिरण्य - इ॒ष्ट॒काः । उप॑ दधाति । द॒धा॒त्ये॒ताभ्यः॑ । ए॒ताभ्य॑ ए॒व । ए॒व दे॒वता᳚भ्यः । दे॒वता᳚भ्यो॒ नमः॑ । नम॑स्करोति । क॒रो॒ति॒ ब्र॒ह्म॒वा॒दिनः॑ । ब्र॒ह्म॒वा॒दिनो॑ वदन्ति । ब्र॒ह्म॒वा॒दिन॒ इति॑ ब्रह्म - वा॒दिनः॑ \newline

\textbf{Jatai Paata} \newline

1. वि॒श्वक॑र्मा दि॒शाम् दि॒शां ॅवि॒श्वक॑र्मा वि॒श्वक॑र्मा दि॒शाम् । \newline
2. वि॒श्वक॒र्मेति॑ वि॒श्व - क॒र्मा॒ । \newline
3. दि॒शाम् पति॒ष् पति॑र् दि॒शाम् दि॒शाम् पतिः॑ । \newline
4. पतिः॒ स स पति॒ष् पतिः॒ सः । \newline
5. स नो॑ नः॒ स स नः॑ । \newline
6. नः॒ प॒शून् प॒शून् नो॑ नः प॒शून् । \newline
7. प॒शून् पा॑तु पातु प॒शून् प॒शून् पा॑तु । \newline
8. पा॒तु॒ स स पा॑तु पातु॒ सः । \newline
9. सो᳚ ऽस्मा न॒स्मान् थ्स सो᳚ ऽस्मान् । \newline
10. अ॒स्मान् पा॑तु पात्व॒स्मा न॒स्मान् पा॑तु । \newline
11. पा॒तु॒ तस्मै॒ तस्मै॑ पातु पातु॒ तस्मै᳚ । \newline
12. तस्मै॒ नमो॒ नम॒ स्तस्मै॒ तस्मै॒ नमः॑ । \newline
13. नमः॑ प्र॒जाप॑तिः प्र॒जाप॑ति॒र् नमो॒ नमः॑ प्र॒जाप॑तिः । \newline
14. प्र॒जाप॑ती रु॒द्रो रु॒द्रः प्र॒जाप॑तिः प्र॒जाप॑ती रु॒द्रः । \newline
15. प्र॒जाप॑ति॒रिति॑ प्र॒जा - प॒तिः॒ । \newline
16. रु॒द्रो वरु॑णो॒ वरु॑णो रु॒द्रो रु॒द्रो वरु॑णः । \newline
17. वरु॑णो॒ ऽग्नि र॒ग्निर् वरु॑णो॒ वरु॑णो॒ ऽग्निः । \newline
18. अ॒ग्निर् दि॒शाम् दि॒शा म॒ग्नि र॒ग्निर् दि॒शाम् । \newline
19. दि॒शाम् पति॒ष् पति॑र् दि॒शाम् दि॒शाम् पतिः॑ । \newline
20. पतिः॒ स स पति॒ष् पतिः॒ सः । \newline
21. स नो॑ नः॒ स स नः॑ । \newline
22. नः॒ प॒शून् प॒शून् नो॑ नः प॒शून् । \newline
23. प॒शून् पा॑तु पातु प॒शून् प॒शून् पा॑तु । \newline
24. पा॒तु॒ स स पा॑तु पातु॒ सः । \newline
25. सो᳚ ऽस्मा न॒स्मान् थ्स सो᳚ ऽस्मान् । \newline
26. अ॒स्मान् पा॑तु पात्व॒स्मा न॒स्मान् पा॑तु । \newline
27. पा॒तु॒ तस्मै॒ तस्मै॑ पातु पातु॒ तस्मै᳚ । \newline
28. तस्मै॒ नमो॒ नम॒ स्तस्मै॒ तस्मै॒ नमः॑ । \newline
29. नम॑ ए॒ता ए॒ता नमो॒ नम॑ ए॒ताः । \newline
30. ए॒ता वै वा ए॒ता ए॒ता वै । \newline
31. वै दे॒वता॑ दे॒वता॒ वै वै दे॒वताः᳚ । \newline
32. दे॒वता॑ ए॒तेषा॑ मे॒तेषा᳚म् दे॒वता॑ दे॒वता॑ ए॒तेषा᳚म् । \newline
33. ए॒तेषा᳚म् पशू॒नाम् प॑शू॒ना मे॒तेषा॑ मे॒तेषा᳚म् पशू॒नाम् । \newline
34. प॒शू॒ना मधि॑पत॒यो ऽधि॑पतयः पशू॒नाम् प॑शू॒ना मधि॑पतयः । \newline
35. अधि॑पतय॒ स्ताभ्य॒ स्ताभ्यो ऽधि॑पत॒यो ऽधि॑पतय॒ स्ताभ्यः॑ । \newline
36. अधि॑पतय॒ इत्यधि॑ - प॒त॒यः॒ । \newline
37. ताभ्यो॒ वै वै ताभ्य॒ स्ताभ्यो॒ वै । \newline
38. वा ए॒ष ए॒ष वै वा ए॒षः । \newline
39. ए॒ष ऐष ए॒ष आ । \newline
40. आ वृ॑श्च्यते वृश्च्यत॒ आ वृ॑श्च्यते । \newline
41. वृ॒श्च्य॒ते॒ यो यो वृ॑श्च्यते वृश्च्यते॒ यः । \newline
42. यः प॑शुशी॒र्॒.षाणि॑ पशुशी॒र्॒.षाणि॒ यो यः प॑शुशी॒र्॒.षाणि॑ । \newline
43. प॒शु॒शी॒र्॒.षा ण्यु॑प॒दधा᳚ त्युप॒दधा॑ति पशुशी॒र्॒.षाणि॑ पशुशी॒र्॒.षा ण्यु॑प॒दधा॑ति । \newline
44. प॒शु॒शी॒र्॒.षाणीति॑ पशु - शी॒र्॒.षाणि॑ । \newline
45. उ॒प॒दधा॑ति हिरण्येष्ट॒का हि॑रण्येष्ट॒का उ॑प॒दधा᳚ त्युप॒दधा॑ति हिरण्येष्ट॒काः । \newline
46. उ॒प॒दधा॒तीत्यु॑प - दधा॑ति । \newline
47. हि॒र॒ण्ये॒ष्ट॒का उपोप॑ हिरण्येष्ट॒का हि॑रण्येष्ट॒का उप॑ । \newline
48. हि॒र॒ण्ये॒ष्ट॒का इति॑ हिरण्य - इ॒ष्ट॒काः । \newline
49. उप॑ दधाति दधा॒ त्युपोप॑ दधाति । \newline
50. द॒धा॒ त्ये॒ताभ्य॑ ए॒ताभ्यो॑ दधाति दधा त्ये॒ताभ्यः॑ । \newline
51. ए॒ताभ्य॑ ए॒वैवै ताभ्य॑ ए॒ताभ्य॑ ए॒व । \newline
52. ए॒व दे॒वता᳚भ्यो दे॒वता᳚भ्य ए॒वैव दे॒वता᳚भ्यः । \newline
53. दे॒वता᳚भ्यो॒ नमो॒ नमो॑ दे॒वता᳚भ्यो दे॒वता᳚भ्यो॒ नमः॑ । \newline
54. नम॑स् करोति करोति॒ नमो॒ नम॑स् करोति । \newline
55. क॒रो॒ति॒ ब्र॒ह्म॒वा॒दिनो᳚ ब्रह्मवा॒दिनः॑ करोति करोति ब्रह्मवा॒दिनः॑ । \newline
56. ब्र॒ह्म॒वा॒दिनो॑ वदन्ति वदन्ति ब्रह्मवा॒दिनो᳚ ब्रह्मवा॒दिनो॑ वदन्ति । \newline
57. ब्र॒ह्म॒वा॒दिन॒ इति॑ ब्रह्म - वा॒दिनः॑ । \newline

\textbf{Ghana Paata } \newline

1. वि॒श्वक॑र्मा दि॒शाम् दि॒शां ॅवि॒श्वक॑र्मा वि॒श्वक॑र्मा दि॒शाम् पति॒ष् पति॑र् दि॒शां ॅवि॒श्वक॑र्मा वि॒श्वक॑र्मा दि॒शाम् पतिः॑ । \newline
2. वि॒श्वक॒र्मेति॑ वि॒श्व - क॒र्मा॒ । \newline
3. दि॒शाम् पति॒ष् पति॑र् दि॒शाम् दि॒शाम् पतिः॒ स स पति॑र् दि॒शाम् दि॒शाम् पतिः॒ सः । \newline
4. पतिः॒ स स पति॒ष् पतिः॒ स नो॑ नः॒ स पति॒ष् पतिः॒ स नः॑ । \newline
5. स नो॑ नः॒ स स नः॑ प॒शून् प॒शून् नः॒ स स नः॑ प॒शून् । \newline
6. नः॒ प॒शून् प॒शून् नो॑ नः प॒शून् पा॑तु पातु प॒शून् नो॑ नः प॒शून् पा॑तु । \newline
7. प॒शून् पा॑तु पातु प॒शून् प॒शून् पा॑तु॒ स स पा॑तु प॒शून् प॒शून् पा॑तु॒ सः । \newline
8. पा॒तु॒ स स पा॑तु पातु॒ सो᳚ ऽस्मा न॒स्मान् थ्स पा॑तु पातु॒ सो᳚ ऽस्मान् । \newline
9. सो᳚ ऽस्मा न॒स्मान् थ्स सो᳚ ऽस्मान् पा॑तु पात्व॒स्मान् थ्स सो᳚ ऽस्मान् पा॑तु । \newline
10. अ॒स्मान् पा॑तु पात्व॒स्मा न॒स्मान् पा॑तु॒ तस्मै॒ तस्मै॑ पात्व॒स्मा न॒स्मान् पा॑तु॒ तस्मै᳚ । \newline
11. पा॒तु॒ तस्मै॒ तस्मै॑ पातु पातु॒ तस्मै॒ नमो॒ नम॒ स्तस्मै॑ पातु पातु॒ तस्मै॒ नमः॑ । \newline
12. तस्मै॒ नमो॒ नम॒ स्तस्मै॒ तस्मै॒ नमः॑ प्र॒जाप॑तिः प्र॒जाप॑ति॒र् नम॒ स्तस्मै॒ तस्मै॒ नमः॑ प्र॒जाप॑तिः । \newline
13. नमः॑ प्र॒जाप॑तिः प्र॒जाप॑ति॒र् नमो॒ नमः॑ प्र॒जाप॑ती रु॒द्रो रु॒द्रः प्र॒जाप॑ति॒र् नमो॒ नमः॑ प्र॒जाप॑ती रु॒द्रः । \newline
14. प्र॒जाप॑ती रु॒द्रो रु॒द्रः प्र॒जाप॑तिः प्र॒जाप॑ती रु॒द्रो वरु॑णो॒ वरु॑णो रु॒द्रः प्र॒जाप॑तिः प्र॒जाप॑ती रु॒द्रो वरु॑णः । \newline
15. प्र॒जाप॑ति॒रिति॑ प्र॒जा - प॒तिः॒ । \newline
16. रु॒द्रो वरु॑णो॒ वरु॑णो रु॒द्रो रु॒द्रो वरु॑णो॒ ऽग्नि र॒ग्निर् वरु॑णो रु॒द्रो रु॒द्रो वरु॑णो॒ ऽग्निः । \newline
17. वरु॑णो॒ ऽग्नि र॒ग्निर् वरु॑णो॒ वरु॑णो॒ ऽग्निर् दि॒शाम् दि॒शा म॒ग्निर् वरु॑णो॒ वरु॑णो॒ ऽग्निर् दि॒शाम् । \newline
18. अ॒ग्निर् दि॒शाम् दि॒शा म॒ग्नि र॒ग्निर् दि॒शाम् पति॒ष् पति॑र् दि॒शा म॒ग्नि र॒ग्निर् दि॒शाम् पतिः॑ । \newline
19. दि॒शाम् पति॒ष् पति॑र् दि॒शाम् दि॒शाम् पतिः॒ स स पति॑र् दि॒शाम् दि॒शाम् पतिः॒ सः । \newline
20. पतिः॒ स स पति॒ष् पतिः॒ स नो॑ नः॒ स पति॒ष् पतिः॒ स नः॑ । \newline
21. स नो॑ नः॒ स स नः॑ प॒शून् प॒शून् नः॒ स स नः॑ प॒शून् । \newline
22. नः॒ प॒शून् प॒शून् नो॑ नः प॒शून् पा॑तु पातु प॒शून् नो॑ नः प॒शून् पा॑तु । \newline
23. प॒शून् पा॑तु पातु प॒शून् प॒शून् पा॑तु॒ स स पा॑तु प॒शून् प॒शून् पा॑तु॒ सः । \newline
24. पा॒तु॒ स स पा॑तु पातु॒ सो᳚ ऽस्मा न॒स्मान् थ्स पा॑तु पातु॒ सो᳚ ऽस्मान् । \newline
25. सो᳚ ऽस्मा न॒स्मान् थ्स सो᳚ ऽस्मान् पा॑तु पात्व॒स्मान् थ्स सो᳚ ऽस्मान् पा॑तु । \newline
26. अ॒स्मान् पा॑तु पात्व॒स्मा न॒स्मान् पा॑तु॒ तस्मै॒ तस्मै॑ पात्व॒स्मा न॒स्मान् पा॑तु॒ तस्मै᳚ । \newline
27. पा॒तु॒ तस्मै॒ तस्मै॑ पातु पातु॒ तस्मै॒ नमो॒ नम॒ स्तस्मै॑ पातु पातु॒ तस्मै॒ नमः॑ । \newline
28. तस्मै॒ नमो॒ नम॒ स्तस्मै॒ तस्मै॒ नम॑ ए॒ता ए॒ता नम॒ स्तस्मै॒ तस्मै॒ नम॑ ए॒ताः । \newline
29. नम॑ ए॒ता ए॒ता नमो॒ नम॑ ए॒ता वै वा ए॒ता नमो॒ नम॑ ए॒ता वै । \newline
30. ए॒ता वै वा ए॒ता ए॒ता वै दे॒वता॑ दे॒वता॒ वा ए॒ता ए॒ता वै दे॒वताः᳚ । \newline
31. वै दे॒वता॑ दे॒वता॒ वै वै दे॒वता॑ ए॒तेषा॑ मे॒तेषा᳚म् दे॒वता॒ वै वै दे॒वता॑ ए॒तेषा᳚म् । \newline
32. दे॒वता॑ ए॒तेषा॑ मे॒तेषा᳚म् दे॒वता॑ दे॒वता॑ ए॒तेषा᳚म् पशू॒नाम् प॑शू॒ना मे॒तेषा᳚म् दे॒वता॑ दे॒वता॑ ए॒तेषा᳚म् पशू॒नाम् । \newline
33. ए॒तेषा᳚म् पशू॒नाम् प॑शू॒ना मे॒तेषा॑ मे॒तेषा᳚म् पशू॒ना मधि॑पत॒यो ऽधि॑पतयः पशू॒ना मे॒तेषा॑ मे॒तेषा᳚म् पशू॒ना मधि॑पतयः । \newline
34. प॒शू॒ना मधि॑पत॒यो ऽधि॑पतयः पशू॒नाम् प॑शू॒ना मधि॑पतय॒ स्ताभ्य॒ स्ताभ्यो ऽधि॑पतयः पशू॒नाम् प॑शू॒ना मधि॑पतय॒ स्ताभ्यः॑ । \newline
35. अधि॑पतय॒ स्ताभ्य॒ स्ताभ्यो ऽधि॑पत॒यो ऽधि॑पतय॒ स्ताभ्यो॒ वै वै ताभ्यो ऽधि॑पत॒यो ऽधि॑पतय॒ स्ताभ्यो॒ वै । \newline
36. अधि॑पतय॒ इत्यधि॑ - प॒त॒यः॒ । \newline
37. ताभ्यो॒ वै वै ताभ्य॒ स्ताभ्यो॒ वा ए॒ष ए॒ष वै ताभ्य॒ स्ताभ्यो॒ वा ए॒षः । \newline
38. वा ए॒ष ए॒ष वै वा ए॒ष ऐष वै वा ए॒ष आ । \newline
39. ए॒ष ऐष ए॒ष आ वृ॑श्च्यते वृश्च्यत॒ ऐष ए॒ष आ वृ॑श्च्यते । \newline
40. आ वृ॑श्च्यते वृश्च्यत॒ आ वृ॑श्च्यते॒ यो यो वृ॑श्च्यत॒ आ वृ॑श्च्यते॒ यः । \newline
41. वृ॒श्च्य॒ते॒ यो यो वृ॑श्च्यते वृश्च्यते॒ यः प॑शुशी॒र्॒.षाणि॑ पशुशी॒र्॒.षाणि॒ यो वृ॑श्च्यते वृश्च्यते॒ यः प॑शुशी॒र्॒.षाणि॑ । \newline
42. यः प॑शुशी॒र्॒.षाणि॑ पशुशी॒र्॒.षाणि॒ यो यः प॑शुशी॒र्॒.षा ण्यु॑प॒दधा᳚ त्युप॒दधा॑ति पशुशी॒र्॒.षाणि॒ यो यः प॑शुशी॒र्॒.षा ण्यु॑प॒दधा॑ति । \newline
43. प॒शु॒शी॒र्॒.षा ण्यु॑प॒दधा᳚ त्युप॒दधा॑ति पशुशी॒र्॒.षाणि॑ पशुशी॒र्॒.षा ण्यु॑प॒दधा॑ति हिरण्येष्ट॒का हि॑रण्येष्ट॒का उ॑प॒दधा॑ति पशुशी॒र्॒.षाणि॑ पशुशी॒र्॒.षा ण्यु॑प॒दधा॑ति हिरण्येष्ट॒काः । \newline
44. प॒शु॒शी॒र्॒.षाणीति॑ पशु - शी॒र्॒.षाणि॑ । \newline
45. उ॒प॒दधा॑ति हिरण्येष्ट॒का हि॑रण्येष्ट॒का उ॑प॒दधा᳚ त्युप॒दधा॑ति हिरण्येष्ट॒का उपोप॑ हिरण्येष्ट॒का उ॑प॒दधा᳚ त्युप॒दधा॑ति हिरण्येष्ट॒का उप॑ । \newline
46. उ॒प॒दधा॒तीत्यु॑प - दधा॑ति । \newline
47. हि॒र॒ण्ये॒ष्ट॒का उपोप॑ हिरण्येष्ट॒का हि॑रण्येष्ट॒का उप॑ दधाति दधा॒ त्युप॑ हिरण्येष्ट॒का हि॑रण्येष्ट॒का उप॑ दधाति । \newline
48. हि॒र॒ण्ये॒ष्ट॒का इति॑ हिरण्य - इ॒ष्ट॒काः । \newline
49. उप॑ दधाति दधा॒ त्युपोप॑ दधा त्ये॒ताभ्य॑ ए॒ताभ्यो॑ दधा॒ त्युपोप॑ दधा त्ये॒ताभ्यः॑ । \newline
50. द॒धा॒ त्ये॒ताभ्य॑ ए॒ताभ्यो॑ दधाति दधा त्ये॒ताभ्य॑ ए॒वैवै ताभ्यो॑ दधाति दधा त्ये॒ताभ्य॑ ए॒व । \newline
51. ए॒ताभ्य॑ ए॒वैवैताभ्य॑ ए॒ताभ्य॑ ए॒व दे॒वता᳚भ्यो दे॒वता᳚भ्य ए॒वैताभ्य॑ ए॒ताभ्य॑ ए॒व दे॒वता᳚भ्यः । \newline
52. ए॒व दे॒वता᳚भ्यो दे॒वता᳚भ्य ए॒वैव दे॒वता᳚भ्यो॒ नमो॒ नमो॑ दे॒वता᳚भ्य ए॒वैव दे॒वता᳚भ्यो॒ नमः॑ । \newline
53. दे॒वता᳚भ्यो॒ नमो॒ नमो॑ दे॒वता᳚भ्यो दे॒वता᳚भ्यो॒ नम॑ स्करोति करोति॒ नमो॑ दे॒वता᳚भ्यो दे॒वता᳚भ्यो॒ नम॑
स्करोति । \newline
54. नम॑ स्करोति करोति॒ नमो॒ नम॑ स्करोति ब्रह्मवा॒दिनो᳚ ब्रह्मवा॒दिनः॑ करोति॒ नमो॒ नम॑ स्करोति ब्रह्मवा॒दिनः॑ । \newline
55. क॒रो॒ति॒ ब्र॒ह्म॒वा॒दिनो᳚ ब्रह्मवा॒दिनः॑ करोति करोति ब्रह्मवा॒दिनो॑ वदन्ति वदन्ति ब्रह्मवा॒दिनः॑ करोति करोति ब्रह्मवा॒दिनो॑ वदन्ति । \newline
56. ब्र॒ह्म॒वा॒दिनो॑ वदन्ति वदन्ति ब्रह्मवा॒दिनो᳚ ब्रह्मवा॒दिनो॑ वदन् त्य॒ग्ना व॒ग्नौ व॑दन्ति ब्रह्मवा॒दिनो᳚ ब्रह्मवा॒दिनो॑ वदन् त्य॒ग्नौ । \newline
57. ब्र॒ह्म॒वा॒दिन॒ इति॑ ब्रह्म - वा॒दिनः॑ । \newline
\pagebreak
\markright{ TS 5.5.5.2  \hfill https://www.vedavms.in \hfill}

\section{ TS 5.5.5.2 }

\textbf{TS 5.5.5.2 } \newline
\textbf{Samhita Paata} \newline

वदन्त्य॒ग्नौ ग्रा॒म्यान् प॒शून् प्र द॑धाति शु॒चाऽऽर॒ण्यान॑र्पयति॒ किं तत॒ उच्छिꣳ॑ष॒तीति॒ यद्धि॑रण्येष्ट॒का उ॑प॒दधा᳚त्य॒मृतं॒ ॅवै हिर॑ण्यम॒मृते॑नै॒व ग्रा॒म्येभ्यः॑ प॒शुभ्यो॑ भेष॒जं क॑रोति॒ नैनान्॑ हिनस्ति प्रा॒णो वै प्र॑थ॒मा स्व॑यमातृ॒ण्णा व्या॒नो द्वि॒तीया॑ऽपा॒नस्तृ॒तीयाऽनु॒ प्राऽ*ण्या᳚त् प्रथ॒माꣳ स्व॑यमातृ॒ण्णामु॑प॒धाय॑ प्रा॒णेनै॒व प्रा॒णꣳ सम॑र्द्धयति॒ व्य॑न्याद् - [  ] \newline

\textbf{Pada Paata} \newline

व॒द॒न्ति॒ । अ॒ग्नौ । ग्रा॒म्यान् । प॒शून् । प्रेति॑ । द॒धा॒ति॒ । शु॒चा । आ॒र॒ण्यान् । अ॒र्प॒य॒ति॒ । किम् । ततः॑ । उदिति॑ । शिꣳ॒॒ष॒ति॒ । इति॑ । यत् । हि॒र॒ण्ये॒ष्ट॒का इति॑ हिरण्य - इ॒ष्ट॒काः । उ॒प॒दधा॒तीत्यु॑प-दधा॑ति । अ॒मृत᳚म् । वै । हिर॑ण्यम् । अ॒मृते॑न । ए॒व । ग्रा॒म्येभ्यः॑ । प॒शुभ्य॒ इति॑ प॒शु - भ्यः॒ । भे॒ष॒जम् । क॒रो॒ति॒ । न । ए॒ना॒न् । हि॒न॒स्ति॒ । प्रा॒ण इति॑ प्र - अ॒नः । वै । प्र॒थ॒मा । स्व॒य॒मा॒तृ॒ण्णेति॑ स्वयं -  आ॒तृ॒ण्णा । व्या॒न इति॑ वि - अ॒नः । द्वि॒तीया᳚ । अ॒पा॒न इत्य॑प-अ॒नः । तृ॒तीया᳚ । अनु॑ । प्रेति॑ । अ॒न्या॒त् । प्र॒थ॒माम् । स्व॒य॒मा॒तृ॒ण्णामिति॑ स्वयं - आ॒तृ॒ण्णाम् । उ॒प॒धायेत्यु॑प - धाय॑ । प्रा॒णेनेति॑ प्र - अ॒नेन॑ । ए॒व । प्रा॒णमिति॑ प्र - अ॒नम् । समिति॑ । अ॒द्‌र्ध॒य॒ति॒ । वीति॑ । अ॒न्या॒त् ।  \newline


\textbf{Krama Paata} \newline

व॒द॒न्त्य॒ग्नौ । अ॒ग्नौ ग्रा॒म्यान् । ग्रा॒म्यान् प॒शून् । प॒शून् प्र । प्र द॑धाति । द॒धा॒ति॒ शु॒चा । शु॒चाऽऽर॒ण्यान् । आ॒र॒ण्यान॑र्पयति । अ॒र्प॒य॒ति॒ किम् । किम् ततः॑ । तत॒ उत् । उच्छिꣳ॑षति । शिꣳ॒॒ष॒तीति॑ । इति॒ यत् । यद्धि॑रण्येष्ट॒काः । हि॒र॒ण्ये॒ष्ट॒का उ॑प॒दधा॑ति । हि॒र॒ण्ये॒ष्ट॒का इति॑ हिरण्य - इ॒ष्ट॒काः । उ॒प॒दधा᳚त्य॒मृत᳚म् । उ॒प॒दधा॒तीत्यु॑प - दधा॑ति । अ॒मृत॒म् ॅवै । वै हिर॑ण्यम् । हिर॑ण्यम॒मृते॑न । अ॒मृते॑नै॒व । ए॒व ग्रा॒म्येभ्यः॑ । ग्रा॒म्येभ्यः॑ प॒शुभ्यः॑ । प॒शुभ्यो॑ भेष॒जम् । प॒शुभ्य॒ इति॑ प॒शु - भ्यः॒ । भे॒ष॒जम् क॑रोति । क॒रो॒ति॒ न । नैनान्॑ । ए॒ना॒न्॒. हि॒न॒स्ति॒ । हि॒न॒स्ति॒ प्रा॒णः । प्रा॒णो वै । प्रा॒ण इति॑ प्र - अ॒नः । वै प्र॑थ॒मा । प्र॒थ॒मा स्व॑यमातृ॒ण्णा । स्व॒य॒मा॒तृ॒ण्णा व्या॒नः । स्व॒य॒मा॒तृ॒ण्णेति॑ स्वयम् - आ॒तृ॒ण्णा । व्या॒नो द्वि॒तीया᳚ । व्या॒न इति॑ वि - अ॒नः । द्वि॒तीया॑ऽपा॒नः । अ॒पा॒न स्तृ॒तीया᳚ । अ॒पा॒न इत्य॑प - अ॒नः । तृ॒तीयाऽनु॑ । अनु॒ प्र । प्राण्या᳚त् । अ॒न्या॒त् प्र॒थ॒माम् । प्र॒थ॒माꣳ स्व॑यमातृ॒ण्णाम् । स्व॒य॒मा॒तृ॒ण्णामु॑प॒धाय॑ । स्व॒य॒मा॒तृ॒ण्णामिति॑ स्वयम् - आ॒तृ॒ण्णाम् । उ॒प॒धाय॑ प्रा॒णेन॑ । उ॒प॒धायेत्यु॑प - धाय॑ । प्रा॒णेनै॒व । प्रा॒णेनेति॑ प्र - अ॒नेन॑ । ए॒व प्रा॒णम् । प्रा॒णꣳ सम् । प्रा॒णमिति॑ प्र - अ॒नम् । सम॑र्द्धयति । अ॒र्द्ध॒य॒ति॒ वि । व्य॑न्यात् । अ॒न्या॒द् द्वि॒तीया᳚म् \newline

\textbf{Jatai Paata} \newline

1. व॒द॒न् त्य॒ग्ना व॒ग्नौ व॑दन्ति वदन् त्य॒ग्नौ । \newline
2. अ॒ग्नौ ग्रा॒म्यान् ग्रा॒म्या न॒ग्ना व॒ग्नौ ग्रा॒म्यान् । \newline
3. ग्रा॒म्यान् प॒शून् प॒शून् ग्रा॒म्यान् ग्रा॒म्यान् प॒शून् । \newline
4. प॒शून् प्र प्र प॒शून् प॒शून् प्र । \newline
5. प्र द॑धाति दधाति॒ प्र प्र द॑धाति । \newline
6. द॒धा॒ति॒ शु॒चा शु॒चा द॑धाति दधाति शु॒चा । \newline
7. शु॒चा ऽऽर॒ण्या ना॑र॒ण्याञ् छु॒चा शु॒चा ऽऽर॒ण्यान् । \newline
8. आ॒र॒ण्या न॑र्पय त्यर्पयत्या र॒ण्या ना॑र॒ण्या न॑र्पयति । \newline
9. अ॒र्प॒य॒ति॒ किम् कि म॑र्पय त्यर्पयति॒ किम् । \newline
10. किम् तत॒ स्ततः॒ किम् किम् ततः॑ । \newline
11. तत॒ उदुत् तत॒ स्तत॒ उत् । \newline
12. उच्छिꣳ॑षति शिꣳष॒ त्युदु च्छिꣳ॑षति । \newline
13. शिꣳ॒॒ष॒तीतीति॑ शिꣳषति शिꣳष॒तीति॑ । \newline
14. इति॒ यद् यदितीति॒ यत् । \newline
15. यद्धि॑रण्येष्ट॒का हि॑रण्येष्ट॒का यद् यद्धि॑रण्येष्ट॒काः । \newline
16. हि॒र॒ण्ये॒ष्ट॒का उ॑प॒दधा᳚ त्युप॒दधा॑ति हिरण्येष्ट॒का हि॑रण्येष्ट॒का उ॑प॒दधा॑ति । \newline
17. हि॒र॒ण्ये॒ष्ट॒का इति॑ हिरण्य - इ॒ष्ट॒काः । \newline
18. उ॒प॒दधा᳚ त्य॒मृत॑ म॒मृत॑ मुप॒दधा᳚ त्युप॒दधा᳚ त्य॒मृत᳚म् । \newline
19. उ॒प॒दधा॒तीत्यु॑प - दधा॑ति । \newline
20. अ॒मृतं॒ ॅवै वा अ॒मृत॑ म॒मृतं॒ ॅवै । \newline
21. वै हिर॑ण्यꣳ॒॒ हिर॑ण्यं॒ ॅवै वै हिर॑ण्यम् । \newline
22. हिर॑ण्य म॒मृते॑ना॒ मृते॑न॒ हिर॑ण्यꣳ॒॒ हिर॑ण्य म॒मृते॑न । \newline
23. अ॒मृते॑नै॒ वैवा मृते॑ना॒ मृते॑नै॒व । \newline
24. ए॒व ग्रा॒म्येभ्यो᳚ ग्रा॒म्येभ्य॑ ए॒वैव ग्रा॒म्येभ्यः॑ । \newline
25. ग्रा॒म्येभ्यः॑ प॒शुभ्यः॑ प॒शुभ्यो᳚ ग्रा॒म्येभ्यो᳚ ग्रा॒म्येभ्यः॑ प॒शुभ्यः॑ । \newline
26. प॒शुभ्यो॑ भेष॒जम् भे॑ष॒जम् प॒शुभ्यः॑ प॒शुभ्यो॑ भेष॒जम् । \newline
27. प॒शुभ्य॒ इति॑ प॒शु - भ्यः॒ । \newline
28. भे॒ष॒जम् क॑रोति करोति भेष॒जम् भे॑ष॒जम् क॑रोति । \newline
29. क॒रो॒ति॒ न न क॑रोति करोति॒ न । \newline
30. नैना॑ नेना॒न् न नैनान्॑ । \newline
31. ए॒ना॒न्॒. हि॒न॒स्ति॒ हि॒न॒ स्त्ये॒ना॒ ने॒ना॒न्॒. हि॒न॒स्ति॒ । \newline
32. हि॒न॒स्ति॒ प्रा॒णः प्रा॒णो हि॑नस्ति हिनस्ति प्रा॒णः । \newline
33. प्रा॒णो वै वै प्रा॒णः प्रा॒णो वै । \newline
34. प्रा॒ण इति॑ प्र - अ॒नः । \newline
35. वै प्र॑थ॒मा प्र॑थ॒मा वै वै प्र॑थ॒मा । \newline
36. प्र॒थ॒मा स्व॑यमातृ॒ण्णा स्व॑यमातृ॒ण्णा प्र॑थ॒मा प्र॑थ॒मा स्व॑यमातृ॒ण्णा । \newline
37. स्व॒य॒मा॒तृ॒ण्णा व्या॒नो व्या॒नः स्व॑यमातृ॒ण्णा स्व॑यमातृ॒ण्णा व्या॒नः । \newline
38. स्व॒य॒मा॒तृ॒ण्णेति॑ स्वयं - आ॒तृ॒ण्णा । \newline
39. व्या॒नो द्वि॒तीया᳚ द्वि॒तीया᳚ व्या॒नो व्या॒नो द्वि॒तीया᳚ । \newline
40. व्या॒न इति॑ वि - अ॒नः । \newline
41. द्वि॒तीया॑ ऽपा॒नो॑ ऽपा॒नो द्वि॒तीया᳚ द्वि॒तीया॑ ऽपा॒नः । \newline
42. अ॒पा॒न स्तृ॒तीया॑ तृ॒तीया॑ ऽपा॒नो॑ ऽपा॒न स्तृ॒तीया᳚ । \newline
43. अ॒पा॒न इत्य॑प - अ॒नः । \newline
44. तृ॒तीया ऽन्वनु॑ तृ॒तीया॑ तृ॒तीया ऽनु॑ । \newline
45. अनु॒ प्र प्राण्वनु॒ प्र । \newline
46. प्राण्या॑ दन्या॒त् प्र प्राण्या᳚त् । \newline
47. अ॒न्या॒त् प्र॒थ॒माम् प्र॑थ॒मा म॑न्या दन्यात् प्रथ॒माम् । \newline
48. प्र॒थ॒माꣳ स्व॑यमातृ॒ण्णाꣳ स्व॑यमातृ॒ण्णाम् प्र॑थ॒माम् प्र॑थ॒माꣳ स्व॑यमातृ॒ण्णाम् । \newline
49. स्व॒य॒मा॒तृ॒ण्णा मु॑प॒धायो॑ प॒धाय॑ स्वयमातृ॒ण्णाꣳ स्व॑यमातृ॒ण्णा मु॑प॒धाय॑ । \newline
50. स्व॒य॒मा॒तृ॒ण्णामिति॑ स्वयं - आ॒तृ॒ण्णाम् । \newline
51. उ॒प॒धाय॑ प्रा॒णेन॑ प्रा॒णेनो॑ प॒धायो॑ प॒धाय॑ प्रा॒णेन॑ । \newline
52. उ॒प॒धायेत्यु॑प - धाय॑ । \newline
53. प्रा॒णेनै॒ वैव प्रा॒णेन॑ प्रा॒णे नै॒व । \newline
54. प्रा॒णेनेति॑ प्र - अ॒नेन॑ । \newline
55. ए॒व प्रा॒णम् प्रा॒ण मे॒वैव प्रा॒णम् । \newline
56. प्रा॒णꣳ सꣳ सम् प्रा॒णम् प्रा॒णꣳ सम् । \newline
57. प्रा॒णमिति॑ प्र - अ॒नम् । \newline
58. स म॑र्द्धय त्यर्द्धयति॒ सꣳ स म॑र्द्धयति । \newline
59. अ॒र्द्ध॒य॒ति॒ वि व्य॑र्द्धय त्यर्द्धयति॒ वि । \newline
60. व्य॑न्या दन्या॒द् वि व्य॑न्यात् । \newline
61. अ॒न्या॒द् द्वि॒तीया᳚म् द्वि॒तीया॑ मन्या दन्याद् द्वि॒तीया᳚म् । \newline

\textbf{Ghana Paata } \newline

1. व॒द॒न् त्य॒ग्ना व॒ग्नौ व॑दन्ति वदन् त्य॒ग्नौ ग्रा॒म्यान् ग्रा॒म्या न॒ग्नौ व॑दन्ति वदन् त्य॒ग्नौ ग्रा॒म्यान् । \newline
2. अ॒ग्नौ ग्रा॒म्यान् ग्रा॒म्या न॒ग्ना व॒ग्नौ ग्रा॒म्यान् प॒शून् प॒शून् ग्रा॒म्या न॒ग्ना व॒ग्नौ ग्रा॒म्यान् प॒शून् । \newline
3. ग्रा॒म्यान् प॒शून् प॒शून् ग्रा॒म्यान् ग्रा॒म्यान् प॒शून् प्र प्र प॒शून् ग्रा॒म्यान् ग्रा॒म्यान् प॒शून् प्र । \newline
4. प॒शून् प्र प्र प॒शून् प॒शून् प्र द॑धाति दधाति॒ प्र प॒शून् प॒शून् प्र द॑धाति । \newline
5. प्र द॑धाति दधाति॒ प्र प्र द॑धाति शु॒चा शु॒चा द॑धाति॒ प्र प्र द॑धाति शु॒चा । \newline
6. द॒धा॒ति॒ शु॒चा शु॒चा द॑धाति दधाति शु॒चा ऽऽर॒ण्या ना॑र॒ण्याञ् छु॒चा द॑धाति दधाति शु॒चा ऽऽर॒ण्यान् । \newline
7. शु॒चा ऽऽर॒ण्या ना॑र॒ण्याञ् छु॒चा शु॒चा ऽऽर॒ण्या न॑र्पय त्यर्पय त्यार॒ण्याञ् छु॒चा शु॒चा ऽऽर॒ण्या न॑र्पयति । \newline
8. आ॒र॒ण्या न॑र्पय त्यर्पय त्यार॒ण्या ना॑र॒ण्या न॑र्पयति॒ किम् कि म॑र्पय त्यार॒ण्या ना॑र॒ण्या न॑र्पयति॒ किम् । \newline
9. अ॒र्प॒य॒ति॒ किम् कि म॑र्पय त्यर्पयति॒ किम् तत॒ स्ततः॒ कि म॑र्पय त्यर्पयति॒ किम् ततः॑ । \newline
10. किम् तत॒ स्ततः॒ किम् किम् तत॒ उदुत् ततः॒ किम् किम् तत॒ उत् । \newline
11. तत॒ उदुत् तत॒ स्तत॒ उच्छिꣳ॑षति शिꣳष॒ त्युत् तत॒ स्तत॒ उच्छिꣳ॑षति । \newline
12. उच्छिꣳ॑षति शिꣳष॒ त्युदुच्छिꣳ॑ष॒ती तीति॑ शिꣳष॒ त्युदुच्छिꣳ॑ष॒तीति॑ । \newline
13. शिꣳ॒॒ष॒तीतीति॑ शिꣳषति शिꣳष॒तीति॒ यद् यदिति॑ शिꣳषति शिꣳष॒तीति॒ यत् । \newline
14. इति॒ यद् यदितीति॒ यद्धि॑रण्येष्ट॒का हि॑रण्येष्ट॒का यदितीति॒ यद्धि॑रण्येष्ट॒काः । \newline
15. यद्धि॑रण्येष्ट॒का हि॑रण्येष्ट॒का यद् यद्धि॑रण्येष्ट॒का उ॑प॒दधा᳚ त्युप॒दधा॑ति हिरण्येष्ट॒का यद् 
यद्धि॑रण्येष्ट॒का उ॑प॒दधा॑ति । \newline
16. हि॒र॒ण्ये॒ष्ट॒का उ॑प॒दधा᳚ त्युप॒दधा॑ति हिरण्येष्ट॒का हि॑रण्येष्ट॒का उ॑प॒दधा᳚ त्य॒मृत॑ म॒मृत॑ मुप॒दधा॑ति हिरण्येष्ट॒का हि॑रण्येष्ट॒का उ॑प॒दधा᳚ त्य॒मृत᳚म् । \newline
17. हि॒र॒ण्ये॒ष्ट॒का इति॑ हिरण्य - इ॒ष्ट॒काः । \newline
18. उ॒प॒दधा᳚ त्य॒मृत॑ म॒मृत॑ मुप॒दधा᳚ त्युप॒दधा᳚ त्य॒मृतं॒ ॅवै वा अ॒मृत॑ मुप॒दधा᳚ त्युप॒दधा᳚ त्य॒मृतं॒ ॅवै । \newline
19. उ॒प॒दधा॒तीत्यु॑प - दधा॑ति । \newline
20. अ॒मृतं॒ ॅवै वा अ॒मृत॑ म॒मृतं॒ ॅवै हिर॑ण्यꣳ॒॒ हिर॑ण्यं॒ ॅवा अ॒मृत॑ म॒मृतं॒ ॅवै हिर॑ण्यम् । \newline
21. वै हिर॑ण्यꣳ॒॒ हिर॑ण्यं॒ ॅवै वै हिर॑ण्य म॒मृते॑ना॒ मृते॑न॒ हिर॑ण्यं॒ ॅवै वै हिर॑ण्य म॒मृते॑न । \newline
22. हिर॑ण्य म॒मृते॑ना॒ मृते॑न॒ हिर॑ण्यꣳ॒॒ हिर॑ण्य म॒मृते॑ नै॒वैवा मृते॑न॒ हिर॑ण्यꣳ॒॒ हिर॑ण्य म॒मृते॑ नै॒व । \newline
23. अ॒मृते॑ नै॒वैवा मृते॑ना॒ मृते॑नै॒व ग्रा॒म्येभ्यो᳚ ग्रा॒म्येभ्य॑ ए॒वामृते॑ना॒ मृते॑नै॒व ग्रा॒म्येभ्यः॑ । \newline
24. ए॒व ग्रा॒म्येभ्यो᳚ ग्रा॒म्येभ्य॑ ए॒वैव ग्रा॒म्येभ्यः॑ प॒शुभ्यः॑ प॒शुभ्यो᳚ ग्रा॒म्येभ्य॑ ए॒वैव ग्रा॒म्येभ्यः॑ प॒शुभ्यः॑ । \newline
25. ग्रा॒म्येभ्यः॑ प॒शुभ्यः॑ प॒शुभ्यो᳚ ग्रा॒म्येभ्यो᳚ ग्रा॒म्येभ्यः॑ प॒शुभ्यो॑ भेष॒जम् भे॑ष॒जम् प॒शुभ्यो᳚ ग्रा॒म्येभ्यो᳚ ग्रा॒म्येभ्यः॑ प॒शुभ्यो॑ भेष॒जम् । \newline
26. प॒शुभ्यो॑ भेष॒जम् भे॑ष॒जम् प॒शुभ्यः॑ प॒शुभ्यो॑ भेष॒जम् क॑रोति करोति भेष॒जम् प॒शुभ्यः॑ प॒शुभ्यो॑ भेष॒जम् क॑रोति । \newline
27. प॒शुभ्य॒ इति॑ प॒शु - भ्यः॒ । \newline
28. भे॒ष॒जम् क॑रोति करोति भेष॒जम् भे॑ष॒जम् क॑रोति॒ न न क॑रोति भेष॒जम् भे॑ष॒जम् क॑रोति॒ न । \newline
29. क॒रो॒ति॒ न न क॑रोति करोति॒ नैना॑ नेना॒न् न क॑रोति करोति॒ नैनान्॑ । \newline
30. नैना॑ नेना॒न् न नैनान्॑. हिनस्ति हिन स्त्येना॒न् न नैनान्॑. हिनस्ति । \newline
31. ए॒ना॒न्॒. हि॒न॒स्ति॒ हि॒न॒स्त्ये॒ना॒ ने॒ना॒न्॒. हि॒न॒स्ति॒ प्रा॒णः प्रा॒णो हि॑न स्त्येना नेनान्. हिनस्ति प्रा॒णः । \newline
32. हि॒न॒स्ति॒ प्रा॒णः प्रा॒णो हि॑नस्ति हिनस्ति प्रा॒णो वै वै प्रा॒णो हि॑नस्ति हिनस्ति प्रा॒णो वै । \newline
33. प्रा॒णो वै वै प्रा॒णः प्रा॒णो वै प्र॑थ॒मा प्र॑थ॒मा वै प्रा॒णः प्रा॒णो वै प्र॑थ॒मा । \newline
34. प्रा॒ण इति॑ प्र - अ॒नः । \newline
35. वै प्र॑थ॒मा प्र॑थ॒मा वै वै प्र॑थ॒मा स्व॑यमातृ॒ण्णा स्व॑यमातृ॒ण्णा प्र॑थ॒मा वै वै प्र॑थ॒मा स्व॑यमातृ॒ण्णा । \newline
36. प्र॒थ॒मा स्व॑यमातृ॒ण्णा स्व॑यमातृ॒ण्णा प्र॑थ॒मा प्र॑थ॒मा स्व॑यमातृ॒ण्णा व्या॒नो व्या॒नः स्व॑यमातृ॒ण्णा प्र॑थ॒मा प्र॑थ॒मा स्व॑यमातृ॒ण्णा व्या॒नः । \newline
37. स्व॒य॒मा॒तृ॒ण्णा व्या॒नो व्या॒नः स्व॑यमातृ॒ण्णा स्व॑यमातृ॒ण्णा व्या॒नो द्वि॒तीया᳚ द्वि॒तीया᳚ व्या॒नः स्व॑यमातृ॒ण्णा स्व॑यमातृ॒ण्णा व्या॒नो द्वि॒तीया᳚ । \newline
38. स्व॒य॒मा॒तृ॒ण्णेति॑ स्वयं - आ॒तृ॒ण्णा । \newline
39. व्या॒नो द्वि॒तीया᳚ द्वि॒तीया᳚ व्या॒नो व्या॒नो द्वि॒तीया॑ ऽपा॒नो॑ ऽपा॒नो द्वि॒तीया᳚ व्या॒नो व्या॒नो द्वि॒तीया॑ ऽपा॒नः । \newline
40. व्या॒न इति॑ वि - अ॒नः । \newline
41. द्वि॒तीया॑ ऽपा॒नो॑ ऽपा॒नो द्वि॒तीया᳚ द्वि॒तीया॑ ऽपा॒न स्तृ॒तीया॑ तृ॒तीया॑ ऽपा॒नो द्वि॒तीया᳚ द्वि॒तीया॑ ऽपा॒न स्तृ॒तीया᳚ । \newline
42. अ॒पा॒न स्तृ॒तीया॑ तृ॒तीया॑ ऽपा॒नो॑ ऽपा॒न स्तृ॒तीया ऽन्वनु॑ तृ॒तीया॑ ऽपा॒नो॑ ऽपा॒न स्तृ॒तीया ऽनु॑ । \newline
43. अ॒पा॒न इत्य॑प - अ॒नः । \newline
44. तृ॒तीया ऽन्वनु॑ तृ॒तीया॑ तृ॒तीया ऽनु॒ प्र प्राणु॑ तृ॒तीया॑ तृ॒तीया ऽनु॒ प्र । \newline
45. अनु॒ प्र प्राण्वनु॒ प्राण्या॑ दन्या॒त् प्राण्वनु॒ प्राण्या᳚त् । \newline
46. प्राण्या॑ दन्या॒त् प्र प्राण्या᳚त् प्रथ॒माम् प्र॑थ॒मा म॑न्या॒त् प्र प्राण्या᳚त् प्रथ॒माम् । \newline
47. अ॒न्या॒त् प्र॒थ॒माम् प्र॑थ॒मा म॑न्या दन्यात् प्रथ॒माꣳ स्व॑यमातृ॒ण्णाꣳ स्व॑यमातृ॒ण्णाम् प्र॑थ॒मा म॑न्या दन्यात् प्रथ॒माꣳ स्व॑यमातृ॒ण्णाम् । \newline
48. प्र॒थ॒माꣳ स्व॑यमातृ॒ण्णाꣳ स्व॑यमातृ॒ण्णाम् प्र॑थ॒माम् प्र॑थ॒माꣳ स्व॑यमातृ॒ण्णा मु॑प॒धा यो॑प॒धाय॑ स्वयमातृ॒ण्णाम् प्र॑थ॒माम् प्र॑थ॒माꣳ स्व॑यमातृ॒ण्णा मु॑प॒धाय॑ । \newline
49. स्व॒य॒मा॒तृ॒ण्णा मु॑प॒धा यो॑प॒धाय॑ स्वयमातृ॒ण्णाꣳ स्व॑यमातृ॒ण्णा मु॑प॒धाय॑ प्रा॒णेन॑ प्रा॒णेनो॑प॒धाय॑ स्वयमातृ॒ण्णाꣳ स्व॑यमातृ॒ण्णा मु॑प॒धाय॑ प्रा॒णेन॑ । \newline
50. स्व॒य॒मा॒तृ॒ण्णामिति॑ स्वयं - आ॒तृ॒ण्णाम् । \newline
51. उ॒प॒धाय॑ प्रा॒णेन॑ प्रा॒णे नो॑प॒धायो॑ प॒धाय॑ प्रा॒णेनै॒वैव प्रा॒णे नो॑प॒धायो॑ प॒धाय॑ प्रा॒णेनै॒व । \newline
52. उ॒प॒धायेत्यु॑प - धाय॑ । \newline
53. प्रा॒णेनै॒वैव प्रा॒णेन॑ प्रा॒णेनै॒व प्रा॒णम् प्रा॒ण मे॒व प्रा॒णेन॑ प्रा॒णेनै॒व प्रा॒णम् । \newline
54. प्रा॒णेनेति॑ प्र - अ॒नेन॑ । \newline
55. ए॒व प्रा॒णम् प्रा॒ण मे॒वैव प्रा॒णꣳ सꣳ सम् प्रा॒ण मे॒वैव प्रा॒णꣳ सम् । \newline
56. प्रा॒णꣳ सꣳ सम् प्रा॒णम् प्रा॒णꣳ स म॑र्द्धय त्यर्द्धयति॒ सम् प्रा॒णम् प्रा॒णꣳ स म॑र्द्धयति । \newline
57. प्रा॒णमिति॑ प्र - अ॒नम् । \newline
58. स म॑र्द्धय त्यर्द्धयति॒ सꣳ स म॑र्द्धयति॒ वि व्य॑र्द्धयति॒ सꣳ स म॑र्द्धयति॒ वि । \newline
59. अ॒र्द्ध॒य॒ति॒ वि व्य॑र्द्धय त्यर्द्धयति॒ व्य॑न्या दन्या॒द् व्य॑र्द्धय त्यर्द्धयति॒ व्य॑न्यात् । \newline
60. व्य॑न्या दन्या॒द् वि व्य॑न्याद् द्वि॒तीया᳚म् द्वि॒तीया॑ मन्या॒द् वि व्य॑न्याद् द्वि॒तीया᳚म् । \newline
61. अ॒न्या॒द् द्वि॒तीया᳚म् द्वि॒तीया॑ मन्या दन्याद् द्वि॒तीया॑ मुप॒धा यो॑प॒धाय॑ द्वि॒तीया॑ मन्या दन्याद् द्वि॒तीया॑ मुप॒धाय॑ । \newline
\pagebreak
\markright{ TS 5.5.5.3  \hfill https://www.vedavms.in \hfill}

\section{ TS 5.5.5.3 }

\textbf{TS 5.5.5.3 } \newline
\textbf{Samhita Paata} \newline

द्वि॒तीया॑मुप॒धाय॑ व्या॒नेनै॒व व्या॒नꣳ सम॑र्द्धय॒त्यपा᳚न् यात्तृ॒तीया॑मुप॒धाया॑-पा॒नेनै॒वापा॒नꣳ सम॑र्द्धय॒त्यथो᳚ प्रा॒णैरे॒वैनꣳ॒॒ समि॑न्धे॒ भूर्भुवः॒ सुव॒रिति॑ स्वयमातृ॒ण्णा उप॑ दधाती॒मे वै लो॒काः स्व॑यमातृ॒ण्णा ए॒ताभिः॒ खलु॒वै व्याहृ॑तीभिः प्र॒जाप॑तिः॒ प्रा*ऽजा॑यत॒ यदे॒ताभि॒र्व्याहृ॑तीभिः स्वयमातृ॒ण्णा उ॑प॒दधा॑ती॒माने॒व लो॒कानु॑प॒धायै॒षु - [  ] \newline

\textbf{Pada Paata} \newline

द्वि॒तीया᳚म् । उ॒प॒धायेत्यु॑प - धाय॑ । व्या॒नेनेति॑ वि - अ॒नेन॑ । ए॒व । व्या॒नमिति॑ वि - अ॒नम् । समिति॑ । अ॒द्‌र्ध॒य॒ति॒ । अपेति॑ । अ॒न्या॒त् । तृ॒तीया᳚म् । उ॒प॒धायेत्यु॑प - धाय॑ । अ॒पा॒नेनेत्य॑प - अ॒नेन॑ । ए॒व । अ॒पा॒नमित्य॑प - अ॒नम् । समिति॑ । अ॒द्‌र्ध॒य॒ति॒ । अथो॒ इति॑ । प्रा॒णैरिति॑ प्र - अ॒नैः । ए॒व । ए॒न॒म् । समिति॑ । इ॒न्धे॒ । भूः । भुवः॑ । सुवः॑ । इति॑ । स्व॒य॒मा॒तृ॒ण्णा इति॑ स्वयं-आ॒तृ॒ण्णाः । उपेति॑ । द॒धा॒ति॒ । इ॒मे । वै । लो॒काः । स्व॒य॒मा॒तृ॒ण्णा इति॑ स्वयं-आ॒तृ॒ण्णाः । ए॒ताभिः॑ । खलु॑ । वै । व्याहृ॑तीभि॒रिति॒ व्याहृ॑ति - भिः॒ । प्र॒जाप॑ति॒रिति॑ प्र॒जा - प॒तिः॒ । प्रेति॑ । अ॒जा॒य॒त॒ । यत् । ए॒ताभिः॑ । व्याहृ॑तीभि॒रिति॒ व्याहृ॑ति - भिः । स्व॒य॒मा॒तृ॒ण्णा इति॑ स्वयं - आ॒तृ॒ण्णाः । उ॒प॒दधा॒तीत्यु॑प - दधा॑ति । इ॒मान् । ए॒व । लो॒कान् । उ॒प॒धायेत्यु॑प - धाय॑ । ए॒षु ।  \newline


\textbf{Krama Paata} \newline

द्वि॒तीया॑मुप॒धाय॑ । उ॒प॒धाय॑ व्या॒नेन॑ । उ॒प॒धायेत्यु॑प - धाय॑ । व्या॒नेनै॒व । व्या॒नेनेति॑ वि - अ॒नेन॑ । ए॒व व्या॒नम् । व्या॒नꣳ सम् । व्या॒नमिति॑ वि - अ॒नम् । सम॑र्द्धयति । अ॒र्द्ध॒य॒त्यप॑ । अपा᳚न्यात् । अ॒न्या॒त् तृ॒तीया᳚म् । तृ॒तीया॑मुप॒धाय॑ । उ॒प॒धाया॑पा॒नेन॑ । उ॒प॒धायेत्यु॑प - धाय॑ । अ॒पा॒नेनै॒व । अ॒पा॒नेनेत्य॑प - अ॒नेन॑ । ए॒वापा॒नम् । अ॒पा॒नꣳ सम् । अ॒पा॒नमित्य॑प - अ॒नम् । सम॑र्द्धयति । अ॒र्द्ध॒य॒त्यथो᳚ । अथो᳚ प्रा॒णैः । अथो॒ इत्यथो᳚ । प्रा॒णैरे॒व । प्रा॒णैरिति॑ प्र - अ॒नैः । ए॒वैन᳚म् । ए॒नꣳ॒॒ सम् । समि॑न्धे । इ॒न्धे॒ भूः । भूर्भुवः॑ । भुवः॒ सुवः॑ । सुव॒रिति॑ । इति॑ स्वयमातृ॒ण्णाः । स्व॒य॒मा॒तृ॒ण्णा उप॑ । स्व॒य॒मा॒तृ॒ण्णा इति॑ स्वयम् - आ॒तृ॒ण्णाः । उप॑ दधाति । द॒धा॒ती॒मे । इ॒मे वै । वै लो॒काः । लो॒काः स्व॑यमातृ॒ण्णाः । स्व॒य॒मा॒तृ॒ण्णा ए॒ताभिः॑ । स्व॒य॒मा॒तृ॒ण्णा इति॑ स्वयम् - आ॒तृ॒ण्णाः । ए॒ताभिः॒ खलु॑ । खलु॒ वै । वै व्याहृ॑तीभिः । व्याहृ॑तीभिः प्र॒जाप॑तिः । व्याहृ॑तीभि॒रिति॒ व्याहृ॑ति - भिः॒ । प्र॒जाप॑तिः॒ प्र । प्र॒जाप॑ति॒रिति॑ प्र॒जा - प॒तिः॒ । प्राजा॑यत । अ॒जा॒य॒त॒ यत् । यदे॒ताभिः॑ । ए॒ताभि॒र् व्याहृ॑तीभिः । व्याहृ॑तीभिः स्वयमातृ॒ण्णाः । व्याहृ॑तीभि॒रिति॒ व्याहृ॑ति - भिः॒ । स्व॒य॒मा॒तृ॒ण्णा उ॑प॒दधा॑ति । स्व॒य॒मा॒तृ॒ण्णा इति॑ स्वयम् - आ॒तृ॒ण्णाः । उ॒प॒दधा॑ती॒मान् । उ॒प॒दधा॒तीत्यु॑प - दधा॑ति । इ॒माने॒व । ए॒व लो॒कान् । लो॒कानु॑प॒धाय॑ । उ॒प॒धायै॒षु । उ॒प॒धायेत्यु॑प - धाय॑ । ए॒षु लो॒केषु॑ \newline

\textbf{Jatai Paata} \newline

1. द्वि॒तीया॑ मुप॒धायो॑ प॒धाय॑ द्वि॒तीया᳚म् द्वि॒तीया॑ मुप॒धाय॑ । \newline
2. उ॒प॒धाय॑ व्या॒नेन॑ व्या॒नेनो॑ प॒धायो॑ प॒धाय॑ व्या॒नेन॑ । \newline
3. उ॒प॒धायेत्यु॑प - धाय॑ । \newline
4. व्या॒नेनै॒ वैव व्या॒नेन॑ व्या॒ने नै॒व । \newline
5. व्या॒नेनेति॑ वि - अ॒नेन॑ । \newline
6. ए॒व व्या॒नं ॅव्या॒न मे॒वैव व्या॒नम् । \newline
7. व्या॒नꣳ सꣳ सं ॅव्या॒नं ॅव्या॒नꣳ सम् । \newline
8. व्या॒नमिति॑ वि - अ॒नम् । \newline
9. स म॑र्द्धय त्यर्द्धयति॒ सꣳ स म॑र्द्धयति । \newline
10. अ॒र्द्ध॒य॒ त्यपापा᳚ र्द्धय त्यर्द्धय॒ त्यप॑ । \newline
11. अपा᳚न्या दन्या॒ दपापा᳚ न्यात् । \newline
12. अ॒न्या॒त् तृ॒तीया᳚म् तृ॒तीया॑ मन्या दन्यात् तृ॒तीया᳚म् । \newline
13. तृ॒तीया॑ मुप॒धा यो॑प॒धाय॑ तृ॒तीया᳚म् तृ॒तीया॑ मुप॒धाय॑ । \newline
14. उ॒प॒धाया॑ पा॒नेना॑ पा॒ने नो॑प॒धा यो॑प॒धाया॑ पा॒नेन॑ । \newline
15. उ॒प॒धायेत्यु॑प - धाय॑ । \newline
16. अ॒पा॒नेनै ॒वैवा पा॒नेना॑ पा॒नेनै॒व । \newline
17. अ॒पा॒नेनेत्य॑प - अ॒नेन॑ । \newline
18. ए॒वापा॒न म॑पा॒न मे॒वै वापा॒नम् । \newline
19. अ॒पा॒नꣳ सꣳ स म॑पा॒न म॑पा॒नꣳ सम् । \newline
20. अ॒पा॒नमित्य॑प - अ॒नम् । \newline
21. स म॑र्द्धय त्यर्द्धयति॒ सꣳ स म॑र्द्धयति । \newline
22. अ॒र्द्ध॒य॒ त्यथो॒ अथो॑ अर्द्धय त्यर्द्धय॒ त्यथो᳚ । \newline
23. अथो᳚ प्रा॒णैः प्रा॒णै रथो॒ अथो᳚ प्रा॒णैः । \newline
24. अथो॒ इत्यथो᳚ । \newline
25. प्रा॒णै रे॒वैव प्रा॒णैः प्रा॒णै रे॒व । \newline
26. प्रा॒णैरिति॑ प्र - अ॒नैः । \newline
27. ए॒वैन॑ मेन मे॒वै वैन᳚म् । \newline
28. ए॒नꣳ॒॒ सꣳ स मे॑न मेनꣳ॒॒ सम् । \newline
29. स मि॑न्ध इन्धे॒ सꣳ स मि॑न्धे । \newline
30. इ॒न्धे॒ भूर् भू रि॑न्ध इन्धे॒ भूः । \newline
31. भूर् भुवो॒ भुवो॒ भूर् भूर् भुवः॑ । \newline
32. भुवः॒ सुवः॒ सुव॒र् भुवो॒ भुवः॒ सुवः॑ । \newline
33. सुव॒ रितीति॒ सुवः॒ सुव॒ रिति॑ । \newline
34. इति॑ स्वयमातृ॒ण्णाः स्व॑यमातृ॒ण्णा इतीति॑ स्वयमातृ॒ण्णाः । \newline
35. स्व॒य॒मा॒तृ॒ण्णा उपोप॑ स्वयमातृ॒ण्णाः स्व॑यमातृ॒ण्णा उप॑ । \newline
36. स्व॒य॒मा॒तृ॒ण्णा इति॑ स्वयं - आ॒तृ॒ण्णाः । \newline
37. उप॑ दधाति दधा॒ त्युपोप॑ दधाति । \newline
38. द॒धा॒ ती॒म इ॒मे द॑धाति दधा ती॒मे । \newline
39. इ॒मे वै वा इ॒म इ॒मे वै । \newline
40. वै लो॒का लो॒का वै वै लो॒काः । \newline
41. लो॒काः स्व॑यमातृ॒ण्णाः स्व॑यमातृ॒ण्णा लो॒का लो॒काः स्व॑यमातृ॒ण्णाः । \newline
42. स्व॒य॒मा॒तृ॒ण्णा ए॒ताभि॑ रे॒ताभिः॑ स्वयमातृ॒ण्णाः स्व॑यमातृ॒ण्णा ए॒ताभिः॑ । \newline
43. स्व॒य॒मा॒तृ॒ण्णा इति॑ स्वयं - आ॒तृ॒ण्णाः । \newline
44. ए॒ताभिः॒ खलु॒ खल्वे॒ ताभि॑ रे॒ताभिः॒ खलु॑ । \newline
45. खलु॒ वै वै खलु॒ खलु॒ वै । \newline
46. वै व्याहृ॑तीभि॒र् व्याहृ॑तीभि॒र् वै वै व्याहृ॑तीभिः । \newline
47. व्याहृ॑तीभिः प्र॒जाप॑तिः प्र॒जाप॑ति॒र् व्याहृ॑तीभि॒र् व्याहृ॑तीभिः प्र॒जाप॑तिः । \newline
48. व्याहृ॑तीभि॒रिति॒ व्याहृ॑ति - भिः॒ । \newline
49. प्र॒जाप॑तिः॒ प्र प्र प्र॒जाप॑तिः प्र॒जाप॑तिः॒ प्र । \newline
50. प्र॒जाप॑ति॒रिति॑ प्र॒जा - प॒तिः॒ । \newline
51. प्राजा॑यता जायत॒ प्र प्राजा॑यत । \newline
52. अ॒जा॒य॒त॒ यद् यद॑जायता जायत॒ यत् । \newline
53. यदे॒ताभि॑ रे॒ताभि॒र् यद् यदे॒ताभिः॑ । \newline
54. ए॒ताभि॒र् व्याहृ॑तीभि॒र् व्याहृ॑तीभि रे॒ताभि॑ रे॒ताभि॒र् व्याहृ॑तीभिः । \newline
55. व्याहृ॑तीभिः स्वयमातृ॒ण्णाः स्व॑यमातृ॒ण्णा व्याहृ॑तीभि॒र् व्याहृ॑तीभिः स्वयमातृ॒ण्णाः । \newline
56. व्याहृ॑तीभि॒रिति॒ व्याहृ॑ति - भिः॒ । \newline
57. स्व॒य॒मा॒तृ॒ण्णा उ॑प॒दधा᳚ त्युप॒दधा॑ति स्वयमातृ॒ण्णाः स्व॑यमातृ॒ण्णा उ॑प॒दधा॑ति । \newline
58. स्व॒य॒मा॒तृ॒ण्णा इति॑ स्वयं - आ॒तृ॒ण्णाः । \newline
59. उ॒प॒दधा॑ ती॒मा नि॒मा नु॑प॒दधा᳚ त्युप॒दधा॑ ती॒मान् । \newline
60. उ॒प॒दधा॒तीत्यु॑प - दधा॑ति । \newline
61. इ॒मा ने॒वैवे मानि॒मा ने॒व । \newline
62. ए॒व लो॒कान् ॅलो॒का ने॒वैव लो॒कान् । \newline
63. लो॒का नु॑प॒धा यो॑प॒धाय॑ लो॒कान् ॅलो॒का नु॑प॒धाय॑ । \newline
64. उ॒प॒धा यै॒ष्वे᳚(1॒)षू॑प॒धा यो॑प॒धायै॒षु । \newline
65. उ॒प॒धायेत्यु॑प - धाय॑ । \newline
66. ए॒षु लो॒केषु॑ लो॒के ष्वे॒ ष्वे॑षु लो॒केषु॑ । \newline

\textbf{Ghana Paata } \newline

1. द्वि॒तीया॑ मुप॒धा यो॑प॒धाय॑ द्वि॒तीया᳚म् द्वि॒तीया॑ मुप॒धाय॑ व्या॒नेन॑ व्या॒ने नो॑प॒धाय॑ द्वि॒तीया᳚म् द्वि॒तीया॑ मुप॒धाय॑ व्या॒नेन॑ । \newline
2. उ॒प॒धाय॑ व्या॒नेन॑ व्या॒ने नो॑प॒धा यो॑प॒धाय॑ व्या॒नेनै॒ वैव व्या॒ने नो॑प॒धा यो॑प॒धाय॑ व्या॒नेनै॒व । \newline
3. उ॒प॒धायेत्यु॑प - धाय॑ । \newline
4. व्या॒नेनै॒ वैव व्या॒नेन॑ व्या॒नेनै॒व व्या॒नं ॅव्या॒न मे॒व व्या॒नेन॑ व्या॒नेनै॒व व्या॒नम् । \newline
5. व्या॒नेनेति॑ वि - अ॒नेन॑ । \newline
6. ए॒व व्या॒नं ॅव्या॒न मे॒वैव व्या॒नꣳ सꣳ सं ॅव्या॒न मे॒वैव व्या॒नꣳ सम् । \newline
7. व्या॒नꣳ सꣳ सं ॅव्या॒नं ॅव्या॒नꣳ स म॑र्द्धय त्यर्द्धयति॒ सं ॅव्या॒नं ॅव्या॒नꣳ स म॑र्द्धयति । \newline
8. व्या॒नमिति॑ वि - अ॒नम् । \newline
9. स म॑र्द्धय त्यर्द्धयति॒ सꣳ स म॑र्द्धय॒ त्यपापा᳚ र्द्धयति॒ सꣳ स म॑र्द्धय॒ त्यप॑ । \newline
10. आ॒र्द्ध॒य॒ त्यपापा᳚ र्द्धय त्यर्द्धय॒ त्यपा᳚ न्या दन्या॒ दपा᳚र्द्धय त्यर्द्धय॒ त्यपा᳚न्यात् । \newline
11. अपा᳚ न्या दन्या॒ दपापा᳚न्यात् तृ॒तीया᳚म् तृ॒तीया॑ मन्या॒ दपापा᳚न्यात् तृ॒तीया᳚म् । \newline
12. अ॒न्या॒त् तृ॒तीया᳚म् तृ॒तीया॑ मन्या दन्यात् तृ॒तीया॑ मुप॒धा यो॑प॒धाय॑ तृ॒तीया॑ मन्या दन्यात् तृ॒तीया॑ मुप॒धाय॑ । \newline
13. तृ॒तीया॑ मुप॒धा यो॑प॒धाय॑ तृ॒तीया᳚म् तृ॒तीया॑ मुप॒धाया॑ पा॒नेना॑ पा॒ने नो॑प॒धाय॑ तृ॒तीया᳚म् तृ॒तीया॑ मुप॒धाया॑ पा॒नेन॑ । \newline
14. उ॒प॒धाया॑ पा॒नेना॑ पा॒ने नो॑प॒धा यो॑प॒धाया॑ पा॒नेनै॒वैवा पा॒ने नो॑प॒धा यो॑प॒धाया॑ पा॒नेनै॒व । \newline
15. उ॒प॒धायेत्यु॑प - धाय॑ । \newline
16. अ॒पा॒ने नै॒वैवा पा॒नेना॑ पा॒नेनै॒वा पा॒न म॑पा॒न मे॒वा पा॒नेना॑ पा॒ने नै॒वापा॒नम् । \newline
17. अ॒पा॒नेनेत्य॑प - अ॒नेन॑ । \newline
18. ए॒वापा॒न म॑पा॒न मे॒वैवा पा॒नꣳ सꣳ स म॑पा॒न मे॒वैवा पा॒नꣳ सम् । \newline
19. अ॒पा॒नꣳ सꣳ स म॑पा॒न म॑पा॒नꣳ स म॑र्द्धय त्यर्द्धयति॒ स म॑पा॒न म॑पा॒नꣳ स म॑र्द्धयति । \newline
20. अ॒पा॒नमित्य॑प - अ॒नम् । \newline
21. स म॑र्द्धय त्यर्द्धयति॒ सꣳ स म॑र्द्धय॒ त्यथो॒ अथो॑ अर्द्धयति॒ सꣳ स म॑र्द्धय॒ त्यथो᳚ । \newline
22. अ॒र्द्ध॒य॒ त्यथो॒ अथो॑ अर्द्धयत्य र्द्धय॒ त्यथो᳚ प्रा॒णैः प्रा॒णै रथो॑ अर्द्धय त्यर्द्धय॒ त्यथो᳚ प्रा॒णैः । \newline
23. अथो᳚ प्रा॒णैः प्रा॒णै रथो॒ अथो᳚ प्रा॒णै रे॒वैव प्रा॒णै रथो॒ अथो᳚ प्रा॒णैरे॒व । \newline
24. अथो॒ इत्यथो᳚ । \newline
25. प्रा॒णै रे॒वैव प्रा॒णैः प्रा॒णै रे॒वैन॑ मेन मे॒व प्रा॒णैः प्रा॒णै रे॒वैन᳚म् । \newline
26. प्रा॒णैरिति॑ प्र - अ॒नैः । \newline
27. ए॒वैन॑ मेन मे॒वै वैनꣳ॒॒ सꣳ स मे॑न मे॒वै वैनꣳ॒॒ सम् । \newline
28. ए॒नꣳ॒॒ सꣳ स मे॑न मेनꣳ॒॒ स मि॑न्ध इन्धे॒ स मे॑न मेनꣳ॒॒ स मि॑न्धे । \newline
29. स मि॑न्ध इन्धे॒ सꣳ स मि॑न्धे॒ भूर् भू रि॑न्धे॒ सꣳ स मि॑न्धे॒ भूः । \newline
30. इ॒न्धे॒ भूर् भू रि॑न्ध इन्धे॒ भूर् भुवो॒ भुवो॒ भू रि॑न्ध इन्धे॒ भूर् भुवः॑ । \newline
31. भूर् भुवो॒ भुवो॒ भूर् भूर् भुवः॒ सुवः॒ सुव॒र् भुवो॒ भूर् भूर् भुवः॒ सुवः॑ । \newline
32. भुवः॒ सुवः॒ सुव॒र् भुवो॒ भुवः॒ सुव॒ रितीति॒ सुव॒र् भुवो॒ भुवः॒ सुव॒ रिति॑ । \newline
33. शुव॒ रितीति॒ सुवः॒ सुव॒ रिति॑ स्वयमातृ॒ण्णाः स्व॑यमातृ॒ण्णा इति॒ सुवः॒ सुव॒ रिति॑ स्वयमातृ॒ण्णाः । \newline
34. इति॑ स्वयमातृ॒ण्णाः स्व॑यमातृ॒ण्णा इतीति॑ स्वयमातृ॒ण्णा उपोप॑ स्वयमातृ॒ण्णा इतीति॑ स्वयमातृ॒ण्णा उप॑ । \newline
35. स्व॒य॒मा॒तृ॒ण्णा उपोप॑ स्वयमातृ॒ण्णाः स्व॑यमातृ॒ण्णा उप॑ दधाति दधा॒ त्युप॑ स्वयमातृ॒ण्णाः स्व॑यमातृ॒ण्णा उप॑ दधाति । \newline
36. स्व॒य॒मा॒तृ॒ण्णा इति॑ स्वयं - आ॒तृ॒ण्णाः । \newline
37. उप॑ दधाति दधा॒ त्युपोप॑ दधाती॒म इ॒मे द॑धा॒ त्युपोप॑ दधाती॒मे । \newline
38. द॒धा॒ती॒म इ॒मे द॑धाति दधाती॒मे वै वा इ॒मे द॑धाति दधाती॒मे वै । \newline
39. इ॒मे वै वा इ॒म इ॒मे वै लो॒का लो॒का वा इ॒म इ॒मे वै लो॒काः । \newline
40. वै लो॒का लो॒का वै वै लो॒काः स्व॑यमातृ॒ण्णाः स्व॑यमातृ॒ण्णा लो॒का वै वै लो॒काः स्व॑यमातृ॒ण्णाः । \newline
41. लो॒काः स्व॑यमातृ॒ण्णाः स्व॑यमातृ॒ण्णा लो॒का लो॒काः स्व॑यमातृ॒ण्णा ए॒ताभि॑ रे॒ताभिः॑ स्वयमातृ॒ण्णा लो॒का लो॒काः स्व॑यमातृ॒ण्णा ए॒ताभिः॑ । \newline
42. स्व॒य॒मा॒तृ॒ण्णा ए॒ताभि॑ रे॒ताभिः॑ स्वयमातृ॒ण्णाः स्व॑यमातृ॒ण्णा ए॒ताभिः॒ खलु॒ खल्वे॒ताभिः॑ स्वयमातृ॒ण्णाः स्व॑यमातृ॒ण्णा ए॒ताभिः॒ खलु॑ । \newline
43. स्व॒य॒मा॒तृ॒ण्णा इति॑ स्वयं - आ॒तृ॒ण्णाः । \newline
44. ए॒ताभिः॒ खलु॒ खल्वे॒ताभि॑ रे॒ताभिः॒ खलु॒ वै वै खल्वे॒ताभि॑ रे॒ताभिः॒ खलु॒ वै । \newline
45. खलु॒ वै वै खलु॒ खलु॒ वै व्याहृ॑तीभि॒र् व्याहृ॑तीभि॒र् वै खलु॒ खलु॒ वै व्याहृ॑तीभिः । \newline
46. वै व्याहृ॑तीभि॒र् व्याहृ॑तीभि॒र् वै वै व्याहृ॑तीभिः प्र॒जाप॑तिः प्र॒जाप॑ति॒र् व्याहृ॑तीभि॒र् वै वै व्याहृ॑तीभिः प्र॒जाप॑तिः । \newline
47. व्याहृ॑तीभिः प्र॒जाप॑तिः प्र॒जाप॑ति॒र् व्याहृ॑तीभि॒र् व्याहृ॑तीभिः प्र॒जाप॑तिः॒ प्र प्र प्र॒जाप॑ति॒र् व्याहृ॑तीभि॒र् व्याहृ॑तीभिः प्र॒जाप॑तिः॒ प्र । \newline
48. व्याहृ॑तीभि॒रिति॒ व्याहृ॑ति - भिः॒ । \newline
49. प्र॒जाप॑तिः॒ प्र प्र प्र॒जाप॑तिः प्र॒जाप॑तिः॒ प्राजा॑यता जायत॒ प्र प्र॒जाप॑तिः प्र॒जाप॑तिः॒ प्राजा॑यत । \newline
50. प्र॒जाप॑ति॒रिति॑ प्र॒जा - प॒तिः॒ । \newline
51. प्राजा॑यता जायत॒ प्र प्राजा॑यत॒ यद् यद॑जायत॒ प्र प्राजा॑यत॒ यत् । \newline
52. अ॒जा॒य॒त॒ यद् यद॑जायता जायत॒ यदे॒ताभि॑ रे॒ताभि॒र् यद॑जायता जायत॒ यदे॒ताभिः॑ । \newline
53. यदे॒ताभि॑ रे॒ताभि॒र् यद् यदे॒ताभि॒र् व्याहृ॑तीभि॒र् व्याहृ॑तीभि रे॒ताभि॒र् यद् यदे॒ताभि॒र् व्याहृ॑तीभिः । \newline
54. ए॒ताभि॒र् व्याहृ॑तीभि॒र् व्याहृ॑तीभि रे॒ताभि॑ रे॒ताभि॒र् व्याहृ॑तीभिः स्वयमातृ॒ण्णाः स्व॑यमातृ॒ण्णा व्याहृ॑तीभि रे॒ताभि॑ रे॒ताभि॒र् व्याहृ॑तीभिः स्वयमातृ॒ण्णाः । \newline
55. व्याहृ॑तीभिः स्वयमातृ॒ण्णाः स्व॑यमातृ॒ण्णा व्याहृ॑तीभि॒र् व्याहृ॑तीभिः स्वयमातृ॒ण्णा उ॑प॒दधा᳚ त्युप॒दधा॑ति स्वयमातृ॒ण्णा व्याहृ॑तीभि॒र् व्याहृ॑तीभिः स्वयमातृ॒ण्णा उ॑प॒दधा॑ति । \newline
56. व्याहृ॑तीभि॒रिति॒ व्याहृ॑ति - भिः॒ । \newline
57. स्व॒य॒मा॒तृ॒ण्णा उ॑प॒दधा᳚ त्युप॒दधा॑ति स्वयमातृ॒ण्णाः स्व॑यमातृ॒ण्णा उ॑प॒दधा॑ती॒मा नि॒मा नु॑प॒दधा॑ति स्वयमातृ॒ण्णाः स्व॑यमातृ॒ण्णा उ॑प॒दधा॑ती॒मान् । \newline
58. स्व॒य॒मा॒तृ॒ण्णा इति॑ स्वयं - आ॒तृ॒ण्णाः । \newline
59. उ॒प॒दधा॑ ती॒मा नि॒मा नु॑प॒दधा᳚ त्युप॒दधा॑ ती॒मा ने॒वैवेमा नु॑प॒दधा᳚ त्युप॒दधा॑ ती॒मा ने॒व । \newline
60. उ॒प॒दधा॒तीत्यु॑प - दधा॑ति । \newline
61. इ॒मा ने॒वैवेमा नि॒मा ने॒व लो॒कान् ॅलो॒का ने॒वेमा नि॒मा ने॒व लो॒कान् । \newline
62. ए॒व लो॒कान् ॅलो॒का ने॒वैव लो॒का नु॑प॒धा यो॑प॒धाय॑ लो॒का ने॒वैव लो॒का नु॑प॒धाय॑ । \newline
63. लो॒का नु॑प॒धा यो॑प॒धाय॑ लो॒कान् ॅलो॒का नु॑प॒धायै॒ ष्वे᳚(1॒)षू॑ प॒धाय॑ लो॒कान् ॅलो॒का नु॑प॒धायै॒षु । \newline
64. उ॒प॒धा यै॒ष्वे᳚(1॒)षू॑ प॒धायो॑ प॒धायै॒षु लो॒केषु॑ लो॒के ष्वे॒षू॑ प॒धा यो॑प॒धायै॒षु लो॒केषु॑ । \newline
65. उ॒प॒धायेत्यु॑प - धाय॑ । \newline
66. ए॒षु लो॒केषु॑ लो॒के ष्वे॒ ष्वे॑षु लो॒के ष्वध्यधि॑ लो॒के ष्वे॒ ष्वे॑षु लो॒के ष्वधि॑ । \newline
\pagebreak
\markright{ TS 5.5.5.4  \hfill https://www.vedavms.in \hfill}

\section{ TS 5.5.5.4 }

\textbf{TS 5.5.5.4 } \newline
\textbf{Samhita Paata} \newline

लो॒केष्वधि॒ प्रजा॑यते प्रा॒णाय॑ व्या॒नाया॑पा॒नाय॑ वा॒चे त्वा॒ चक्षु॑षे त्वा॒ तया॑ दे॒वत॑याऽङ्गिर॒स्वद् ध्रु॒वा सी॑दा॒ग्निना॒ वै दे॒वाः सु॑व॒र्गं ॅलो॒कम॑जिगाꣳस॒न् तेन॒ पति॑तुं॒ नाश॑क्नुव॒न् त ए॒ताश्चत॑स्रः स्वयमातृ॒ण्णा अ॑पश्य॒न् ता दि॒क्षूपा॑दधत॒ तेन॑ स॒र्वत॑श्चक्षुषा सुव॒र्गं ॅलो॒कमा॑य॒न्॒ यच्चत॑स्रः स्वयमातृ॒ण्णा दि॒क्षू॑प॒दधा॑ति स॒र्वत॑श्चक्षुषै॒व तद॒ग्निना॒ यज॑मानः ( ) सुव॒र्गं ॅलो॒कमे॑ति ॥ \newline

\textbf{Pada Paata} \newline

लो॒केषु॑ । अधि॑ । प्रेति॑ । जा॒य॒ते॒ । प्रा॒णायेति॑ प्र - अ॒नाय॑ । व्या॒नायेति॑ वि - अ॒नाय॑ । अ॒पा॒नायेत्य॑प - अ॒नाय॑ । वा॒चे । त्वा॒ । चक्षु॑षे । त्वा॒ । तया᳚ । दे॒वत॑या । अ॒ङ्गि॒र॒स्वत् । ध्रु॒वा । सी॒द॒ । अ॒ग्निना᳚ । वै । दे॒वाः । सु॒व॒र्गमिति॑ सुवः - गम् । लो॒कम् । अ॒जि॒गाꣳ॒॒स॒न्न् । तेन॑ । पति॑तुम् । न । अ॒श॒क्नु॒व॒न्न् । ते । ए॒ताः । चत॑स्रः । स्व॒य॒मा॒तृ॒ण्णा इति॑ स्वयं - अ॒तृ॒ण्णाः । अ॒प॒श्य॒न्न् । ताः । दि॒क्षु । उपेति॑ । अ॒द॒ध॒त॒ । तेन॑ । स॒र्वत॑श्चक्षु॒षेति॑ स॒र्वतः॑ - च॒क्षु॒षा॒ । सु॒व॒र्गमिति॑ सुवः - गम् । लो॒कम् । आ॒य॒न्न् । यत् । चत॑स्रः । स्व॒य॒मा॒तृ॒ण्णा इति॑ स्वयं - आ॒तृ॒ण्णाः । दि॒क्षु । उ॒प॒दधा॒तीत्यु॑प - दधा॑ति । सर्वत॑श्चक्षु॒षेति॑ स॒र्वतः॑ - च॒क्षु॒षा॒ । ए॒व । तत् । अ॒ग्निना᳚ । यज॑मानः ( ) । सु॒व॒र्गमिति॑ सुवः - गम् । लो॒कम् । ए॒ति॒ ॥  \newline


\textbf{Krama Paata} \newline

लो॒केष्वधि॑ । अधि॒ प्र । प्र जा॑यते । जा॒य॒ते॒ प्रा॒णाय॑ । प्रा॒णाय॑ व्या॒नाय॑ । प्रा॒णायेति॑ प्र - अ॒नाय॑ । व्या॒नाया॑पा॒नाय॑ । व्या॒नायेति॑ वि - अ॒नाय॑ । अ॒पा॒नाय॑ वा॒चे । अ॒पा॒नायेत्य॑प - अ॒नाय॑ । वा॒चे त्वा᳚ । त्वा॒ चक्षु॑षे । चक्षु॑षे त्वा । त्वा॒ तया᳚ । तया॑ दे॒वत॑या । दे॒वत॑याऽङ्गिर॒स्वत् । अ॒ङ्गि॒र॒स्वद् ध्रु॒वा । ध्रु॒वा सी॑द । सी॒दा॒ग्निना᳚ । अ॒ग्निना॒ वै । वै दे॒वाः । दे॒वाः सु॑व॒र्गम् । सु॒व॒र्गम् ॅलो॒कम् । सु॒व॒र्गमिति॑ सुवः - गम् । लो॒कम॑जिगाꣳसन्न् । अ॒जि॒गाꣳ॒॒स॒न् तेन॑ । तेन॒ पति॑तुम् । पति॑तु॒म् न । नाश॑क्नुवन्न् । अ॒श॒क्नु॒व॒न् ते । त ए॒ताः । ए॒ताश्चत॑स्रः । चत॑स्रः स्वयमातृ॒ण्णाः । स्व॒य॒मा॒तृ॒ण्णा अ॑पश्यन्न् । स्व॒य॒मा॒तृ॒ण्णा इति॑ स्वयम् - आ॒तृ॒ण्णाः । अ॒प॒श्य॒न् ताः । ता दि॒क्षु । दि॒क्षूप॑ । उपा॑दधत । अ॒द॒ध॒त॒ तेन॑ । तेन॑ स॒र्वत॑श्चक्षुषा । स॒र्वत॑श्चक्षुषा सुव॒र्गम् । स॒र्वत॑श्चक्षु॒षेति॑ स॒र्वतः॑ - च॒क्षु॒षा॒ । सु॒व॒र्गम् ॅलो॒कम् । सु॒व॒र्गमिति॑ सुवः - गम् । लो॒कमा॑यन्न् । आ॒य॒न्॒. यत् । यच् चत॑स्रः । चत॑स्रः स्वयमातृ॒ण्णाः । स्व॒य॒मा॒तृ॒ण्णा दि॒क्षु । स्व॒य॒मा॒तृ॒ण्णा इति॑ स्वयम् - आ॒तृ॒ण्णाः । दि॒क्षू॑प॒दधा॑ति । उ॒प॒दधा॑ति स॒र्वत॑श्चक्षुषा । उ॒प॒दधा॒तीत्यु॑प दधा॑ति । स॒र्वत॑श्चक्षुषै॒व । स॒र्वत॑श्चक्षु॒षेति॑ स॒र्वतः॑ - च॒क्षु॒षा॒ । ए॒व तत् । तद॒ग्निना᳚ । अ॒ग्निना॒ यज॑मानः ( ) । यज॑मानः सुव॒र्गम् । सु॒व॒र्गम् ॅलो॒कम् । सु॒व॒र्गमिति॑ सुवः - गम् । लो॒कमे॑ति । ए॒तीत्ये॑ति । \newline

\textbf{Jatai Paata} \newline

1. लो॒के ष्वध्यधि॑ लो॒केषु॑ लो॒के ष्वधि॑ । \newline
2. अधि॒ प्र प्राध्यधि॒ प्र । \newline
3. प्र जा॑यते जायते॒ प्र प्र जा॑यते । \newline
4. जा॒य॒ते॒ प्रा॒णाय॑ प्रा॒णाय॑ जायते जायते प्रा॒णाय॑ । \newline
5. प्रा॒णाय॑ व्या॒नाय॑ व्या॒नाय॑ प्रा॒णाय॑ प्रा॒णाय॑ व्या॒नाय॑ । \newline
6. प्रा॒णायेति॑ प्र - अ॒नाय॑ । \newline
7. व्या॒नाया॑ पा॒नाया॑ पा॒नाय॑ व्या॒नाय॑ व्या॒नाया॑ पा॒नाय॑ । \newline
8. व्या॒नायेति॑ वि - अ॒नाय॑ । \newline
9. अ॒पा॒नाय॑ वा॒चे वा॒चे॑ ऽपा॒नाया॑ पा॒नाय॑ वा॒चे । \newline
10. अ॒पा॒नायेत्य॑प - अ॒नाय॑ । \newline
11. वा॒चे त्वा᳚ त्वा वा॒चे वा॒चे त्वा᳚ । \newline
12. त्वा॒ चक्षु॑षे॒ चक्षु॑षे त्वा त्वा॒ चक्षु॑षे । \newline
13. चक्षु॑षे त्वा त्वा॒ चक्षु॑षे॒ चक्षु॑षे त्वा । \newline
14. त्वा॒ तया॒ तया᳚ त्वा त्वा॒ तया᳚ । \newline
15. तया॑ दे॒वत॑या दे॒वत॑या॒ तया॒ तया॑ दे॒वत॑या । \newline
16. दे॒वत॑या ऽङ्गिर॒स्व द॑ङ्गिर॒स्वद् दे॒वत॑या दे॒वत॑या ऽङ्गिर॒स्वत् । \newline
17. अ॒ङ्गि॒र॒स्वद् ध्रु॒वा ध्रु॒वा ऽङ्गि॑र॒स्व द॑ङ्गिर॒स्वद् ध्रु॒वा । \newline
18. ध्रु॒वा सी॑द सीद ध्रु॒वा ध्रु॒वा सी॑द । \newline
19. सी॒दा॒ग्निना॒ ऽग्निना॑ सीद सीदा॒ग्निना᳚ । \newline
20. अ॒ग्निना॒ वै वा अ॒ग्निना॒ ऽग्निना॒ वै । \newline
21. वै दे॒वा दे॒वा वै वै दे॒वाः । \newline
22. दे॒वाः सु॑व॒र्गꣳ सु॑व॒र्गम् दे॒वा दे॒वाः सु॑व॒र्गम् । \newline
23. सु॒व॒र्गम् ॅलो॒कम् ॅलो॒कꣳ सु॑व॒र्गꣳ सु॑व॒र्गम् ॅलो॒कम् । \newline
24. सु॒व॒र्गमिति॑ सुवः - गम् । \newline
25. लो॒क म॑जिगाꣳसन् नजिगाꣳसन् ॅलो॒कम् ॅलो॒क म॑जिगाꣳसन्न् । \newline
26. अ॒जि॒गाꣳ॒॒स॒न् तेन॒ तेना॑जिगाꣳसन् नजिगाꣳस॒न् तेन॑ । \newline
27. तेन॒ पति॑तु॒म् पति॑तु॒म् तेन॒ तेन॒ पति॑तुम् । \newline
28. पति॑तु॒म् न न पति॑तु॒म् पति॑तु॒म् न । \newline
29. नाश॑क्नुवन् नशक्नुव॒न् न नाश॑क्नुवन्न् । \newline
30. अ॒श॒क्नु॒व॒न् ते ते॑ ऽशक्नुवन् नशक्नुव॒न् ते । \newline
31. त ए॒ता ए॒ता स्ते त ए॒ताः । \newline
32. ए॒ता श्चत॑स्र॒ श्चत॑स्र ए॒ता ए॒ता श्चत॑स्रः । \newline
33. चत॑स्रः स्वयमातृ॒ण्णाः स्व॑यमातृ॒ण्णा श्चत॑स्र॒ श्चत॑स्रः स्वयमातृ॒ण्णाः । \newline
34. स्व॒य॒मा॒तृ॒ण्णा अ॑पश्यन् नपश्यन् थ्स्वयमातृ॒ण्णाः स्व॑यमातृ॒ण्णा अ॑पश्यन्न् । \newline
35. स्व॒य॒मा॒तृ॒ण्णा इति॑ स्वयं - आ॒तृ॒ण्णाः । \newline
36. अ॒प॒श्य॒न् ता स्ता अ॑पश्यन् नपश्य॒न् ताः । \newline
37. ता दि॒क्षु दि॒क्षु ता स्ता दि॒क्षु । \newline
38. दि॒क्षूपोप॑ दि॒क्षु दि॒क्षूप॑ । \newline
39. उपा॑दधता दध॒तोपोपा॑ दधत । \newline
40. अ॒द॒ध॒त॒ तेन॒ तेना॑ दधता दधत॒ तेन॑ । \newline
41. तेन॑ स॒र्वत॑श्चक्षुषा स॒र्वत॑श्चक्षुषा॒ तेन॒ तेन॑ स॒र्वत॑श्चक्षुषा । \newline
42. स॒र्वत॑श्चक्षुषा सुव॒र्गꣳ सु॑व॒र्गꣳ स॒र्वत॑श्चक्षुषा स॒र्वत॑श्चक्षुषा सुव॒र्गम् । \newline
43. स॒र्वत॑श्चक्षु॒षेति॑ स॒र्वतः॑ - च॒क्षु॒षा॒ । \newline
44. सु॒व॒र्गम् ॅलो॒कम् ॅलो॒कꣳ सु॑व॒र्गꣳ सु॑व॒र्गम् ॅलो॒कम् । \newline
45. सु॒व॒र्गमिति॑ सुवः - गम् । \newline
46. लो॒क मा॑यन् नायन् ॅलो॒कम् ॅलो॒क मा॑यन्न् । \newline
47. आ॒य॒न्॒. यद् यदा॑यन् नाय॒न्॒. यत् । \newline
48. यच् चत॑स्र॒ श्चत॑स्रो॒ यद् यच् चत॑स्रः । \newline
49. चत॑स्रः स्वयमातृ॒ण्णाः स्व॑यमातृ॒ण्णा श्चत॑स्र॒ श्चत॑स्रः स्वयमातृ॒ण्णाः । \newline
50. स्व॒य॒मा॒तृ॒ण्णा दि॒क्षु दि॒क्षु स्व॑यमातृ॒ण्णाः स्व॑यमातृ॒ण्णा दि॒क्षु । \newline
51. स्व॒य॒मा॒तृ॒ण्णा इति॑ स्वयं - आ॒तृ॒ण्णाः । \newline
52. दि॒क्षू॑ प॒दधा᳚ त्युप॒दधा॑ति दि॒क्षु दि॒क्षू॑ प॒दधा॑ति । \newline
53. उ॒प॒दधा॑ति॒ सर्वत॑श्चक्षुषा॒ सर्वत॑श्चक्षुषो प॒दधा᳚ त्युप॒दधा॑ति॒ सर्वत॑श्चक्षुषा । \newline
54. उ॒प॒दधा॒तीत्यु॑प - दधा॑ति । \newline
55. सर्वत॑श्चक्षु षै॒वैव सर्वत॑श्चक्षुषा॒ सर्वत॑श्चक्षुषै॒व । \newline
56. सर्वत॑श्चक्षु॒षेति॑ स॒र्वतः॑ - च॒क्षु॒षा॒ । \newline
57. ए॒व तत् तदे॒ वैव तत् । \newline
58. तद॒ग्निना॒ ऽग्निना॒ तत् तद॒ग्निना᳚ । \newline
59. अ॒ग्निना॒ यज॑मानो॒ यज॑मानो॒ ऽग्निना॒ ऽग्निना॒ यज॑मानः । \newline
60. यज॑मानः सुव॒र्गꣳ सु॑व॒र्गं ॅयज॑मानो॒ यज॑मानः सुव॒र्गम् । \newline
61. सु॒व॒र्गम् ॅलो॒कम् ॅलो॒कꣳ सु॑व॒र्गꣳ सु॑व॒र्गम् ॅलो॒कम् । \newline
62. सु॒व॒र्गमिति॑ सुवः - गम् । \newline
63. लो॒क मे᳚त्येति लो॒कम् ॅलो॒क मे॑ति । \newline
64. ए॒तीत्ये॑ति । \newline

\textbf{Ghana Paata } \newline

1. लो॒केष्व ध्यधि॑ लो॒केषु॑ लो॒केष्वधि॒ प्र प्राधि॑ लो॒केषु॑ लो॒के ष्वधि॒ प्र । \newline
2. अधि॒ प्र प्राध्यधि॒ प्र जा॑यते जायते॒ प्राध्यधि॒ प्र जा॑यते । \newline
3. प्र जा॑यते जायते॒ प्र प्र जा॑यते प्रा॒णाय॑ प्रा॒णाय॑ जायते॒ प्र प्र जा॑यते प्रा॒णाय॑ । \newline
4. जा॒य॒ते॒ प्रा॒णाय॑ प्रा॒णाय॑ जायते जायते प्रा॒णाय॑ व्या॒नाय॑ व्या॒नाय॑ प्रा॒णाय॑ जायते जायते प्रा॒णाय॑ व्या॒नाय॑ । \newline
5. प्रा॒णाय॑ व्या॒नाय॑ व्या॒नाय॑ प्रा॒णाय॑ प्रा॒णाय॑ व्या॒नाया॑ पा॒नाया॑ पा॒नाय॑ व्या॒नाय॑ प्रा॒णाय॑ प्रा॒णाय॑ व्या॒नाया॑ पा॒नाय॑ । \newline
6. प्रा॒णायेति॑ प्र - अ॒नाय॑ । \newline
7. व्या॒नाया॑ पा॒नाया॑ पा॒नाय॑ व्या॒नाय॑ व्या॒नाया॑ पा॒नाय॑ वा॒चे वा॒चे॑ ऽपा॒नाय॑ व्या॒नाय॑ व्या॒नाया॑ पा॒नाय॑ वा॒चे । \newline
8. व्या॒नायेति॑ वि - अ॒नाय॑ । \newline
9. अ॒पा॒नाय॑ वा॒चे वा॒चे॑ ऽपा॒नाया॑ पा॒नाय॑ वा॒चे त्वा᳚ त्वा वा॒चे॑ ऽपा॒नाया॑ पा॒नाय॑ वा॒चे त्वा᳚ । \newline
10. अ॒पा॒नायेत्य॑प - अ॒नाय॑ । \newline
11. वा॒चे त्वा᳚ त्वा वा॒चे वा॒चे त्वा॒ चक्षु॑षे॒ चक्षु॑षे त्वा वा॒चे वा॒चे त्वा॒ चक्षु॑षे । \newline
12. त्वा॒ चक्षु॑षे॒ चक्षु॑षे त्वा त्वा॒ चक्षु॑षे त्वा त्वा॒ चक्षु॑षे त्वा त्वा॒ चक्षु॑षे त्वा । \newline
13. चक्षु॑षे त्वा त्वा॒ चक्षु॑षे॒ चक्षु॑षे त्वा॒ तया॒ तया᳚ त्वा॒ चक्षु॑षे॒ चक्षु॑षे त्वा॒ तया᳚ । \newline
14. त्वा॒ तया॒ तया᳚ त्वा त्वा॒ तया॑ दे॒वत॑या दे॒वत॑या॒ तया᳚ त्वा त्वा॒ तया॑ दे॒वत॑या । \newline
15. तया॑ दे॒वत॑या दे॒वत॑या॒ तया॒ तया॑ दे॒वत॑या ऽङ्गिर॒स्व द॑ङ्गिर॒स्वद् दे॒वत॑या॒ तया॒ तया॑ दे॒वत॑या ऽङ्गिर॒स्वत् । \newline
16. दे॒वत॑या ऽङ्गिर॒स्व द॑ङ्गिर॒स्वद् दे॒वत॑या दे॒वत॑या ऽङ्गिर॒स्वद् ध्रु॒वा ध्रु॒वा ऽङ्गि॑र॒स्वद् दे॒वत॑या दे॒वत॑या ऽङ्गिर॒स्वद् ध्रु॒वा । \newline
17. अ॒ङ्गि॒र॒स्वद् ध्रु॒वा ध्रु॒वा ऽङ्गि॑र॒स्व द॑ङ्गिर॒स्वद् ध्रु॒वा सी॑द सीद ध्रु॒वा ऽङ्गि॑र॒स्व द॑ङ्गिर॒स्वद् ध्रु॒वा सी॑द । \newline
18. ध्रु॒वा सी॑द सीद ध्रु॒वा ध्रु॒वा सी॑दा॒ ग्निना॒ ऽग्निना॑ सीद ध्रु॒वा ध्रु॒वा सी॑दा॒ग्निना᳚ । \newline
19. सी॒दा॒ ग्निना॒ ऽग्निना॑ सीद सीदा॒ग्निना॒ वै वा अ॒ग्निना॑ सीद सीदा॒ग्निना॒ वै । \newline
20. अ॒ग्निना॒ वै वा अ॒ग्निना॒ ऽग्निना॒ वै दे॒वा दे॒वा वा अ॒ग्निना॒ ऽग्निना॒ वै दे॒वाः । \newline
21. वै दे॒वा दे॒वा वै वै दे॒वाः सु॑व॒र्गꣳ सु॑व॒र्गम् दे॒वा वै वै दे॒वाः सु॑व॒र्गम् । \newline
22. दे॒वाः सु॑व॒र्गꣳ सु॑व॒र्गम् दे॒वा दे॒वाः सु॑व॒र्गम् ॅलो॒कम् ॅलो॒कꣳ सु॑व॒र्गम् दे॒वा दे॒वाः सु॑व॒र्गम् ॅलो॒कम् । \newline
23. सु॒व॒र्गम् ॅलो॒कम् ॅलो॒कꣳ सु॑व॒र्गꣳ सु॑व॒र्गम् ॅलो॒क म॑जिगाꣳसन् नजिगाꣳसन् ॅलो॒कꣳ सु॑व॒र्गꣳ सु॑व॒र्गम् ॅलो॒क म॑जिगाꣳसन्न् । \newline
24. सु॒व॒र्गमिति॑ सुवः - गम् । \newline
25. लो॒क म॑जिगाꣳसन् नजिगाꣳसन् ॅलो॒कम् ॅलो॒क म॑जिगाꣳस॒न् तेन॒ तेना॑जिगाꣳसन् ॅलो॒कम् ॅलो॒क म॑जिगाꣳस॒न् तेन॑ । \newline
26. अ॒जि॒गाꣳ॒॒स॒न् तेन॒ तेना॑जिगाꣳसन् नजिगाꣳस॒न् तेन॒ पति॑तु॒म् पति॑तु॒म् तेना॑जिगाꣳसन् नजिगाꣳस॒न् तेन॒ पति॑तुम् । \newline
27. तेन॒ पति॑तु॒म् पति॑तु॒म् तेन॒ तेन॒ पति॑तु॒म् न न पति॑तु॒म् तेन॒ तेन॒ पति॑तु॒म् न । \newline
28. पति॑तु॒म् न न पति॑तु॒म् पति॑तु॒म् नाश॑क्नुवन् नशक्नुव॒न् न पति॑तु॒म् पति॑तु॒म् नाश॑क्नुवन्न् । \newline
29. नाश॑क्नुवन् नशक्नुव॒न् न नाश॑क्नुव॒न् ते ते॑ ऽशक्नुव॒न् न नाश॑क्नुव॒न् ते । \newline
30. अ॒श॒क्नु॒व॒न् ते ते॑ ऽशक्नुवन् नशक्नुव॒न् त ए॒ता ए॒ता स्ते॑ ऽशक्नुवन् नशक्नुव॒न् त ए॒ताः । \newline
31. त ए॒ता ए॒ता स्ते त ए॒ता श्चत॑स्र॒ श्चत॑स्र ए॒ता स्ते त ए॒ता श्चत॑स्रः । \newline
32. ए॒ता श्चत॑स्र॒ श्चत॑स्र ए॒ता ए॒ता श्चत॑स्रः स्वयमातृ॒ण्णाः स्व॑यमातृ॒ण्णा श्चत॑स्र ए॒ता ए॒ता श्चत॑स्रः स्वयमातृ॒ण्णाः । \newline
33. चत॑स्रः स्वयमातृ॒ण्णाः स्व॑यमातृ॒ण्णा श्चत॑स्र॒ श्चत॑स्रः स्वयमातृ॒ण्णा अ॑पश्यन् नपश्यन् थ्स्वयमातृ॒ण्णा श्चत॑स्र॒ श्चत॑स्रः स्वयमातृ॒ण्णा अ॑पश्यन्न् । \newline
34. स्व॒य॒मा॒तृ॒ण्णा अ॑पश्यन् नपश्यन् थ्स्वयमातृ॒ण्णाः स्व॑यमातृ॒ण्णा अ॑पश्य॒न् ता स्ता अ॑पश्यन् थ्स्वयमातृ॒ण्णाः स्व॑यमातृ॒ण्णा अ॑पश्य॒न् ताः । \newline
35. स्व॒य॒मा॒तृ॒ण्णा इति॑ स्वयं - आ॒तृ॒ण्णाः । \newline
36. अ॒प॒श्य॒न् ता स्ता अ॑पश्यन् नपश्य॒न् ता दि॒क्षु दि॒क्षु ता अ॑पश्यन् नपश्य॒न् ता दि॒क्षु । \newline
37. ता दि॒क्षु दि॒क्षु ता स्ता दि॒क्षूपोप॑ दि॒क्षु ता स्ता दि॒क्षूप॑ । \newline
38. दि॒क्षूपोप॑ दि॒क्षु दि॒क्षूपा॑ दधता दध॒तोप॑ दि॒क्षु दि॒क्षूपा॑ दधत । \newline
39. उपा॑दधता दध॒तोपोपा॑ दधत॒ तेन॒ तेना॑ दध॒तोपोपा॑ दधत॒ तेन॑ । \newline
40. अ॒द॒ध॒त॒ तेन॒ तेना॑ दधता दधत॒ तेन॑ स॒र्वत॑श्चक्षुषा स॒र्वत॑श्चक्षुषा॒ तेना॑ दधता दधत॒ तेन॑ स॒र्वत॑श्चक्षुषा । \newline
41. तेन॑ स॒र्वत॑श्चक्षुषा स॒र्वत॑श्चक्षुषा॒ तेन॒ तेन॑ स॒र्वत॑श्चक्षुषा सुव॒र्गꣳ सु॑व॒र्गꣳ स॒र्वत॑श्चक्षुषा॒ तेन॒ तेन॑ स॒र्वत॑श्चक्षुषा सुव॒र्गम् । \newline
42. स॒र्वत॑श्चक्षुषा सुव॒र्गꣳ सु॑व॒र्गꣳ स॒र्वत॑श्चक्षुषा स॒र्वत॑श्चक्षुषा सुव॒र्गम् ॅलो॒कम् ॅलो॒कꣳ सु॑व॒र्गꣳ स॒र्वत॑श्चक्षुषा स॒र्वत॑श्चक्षुषा सुव॒र्गम् ॅलो॒कम् । \newline
43. स॒र्वत॑श्चक्षु॒षेति॑ स॒र्वतः॑ - च॒क्षु॒षा॒ । \newline
44. सु॒व॒र्गम् ॅलो॒कम् ॅलो॒कꣳ सु॑व॒र्गꣳ सु॑व॒र्गम् ॅलो॒क मा॑यन् ‍नायन् ॅलो॒कꣳ सु॑व॒र्गꣳ सु॑व॒र्गम् ॅलो॒क मा॑यन्न् । \newline
45. सु॒व॒र्गमिति॑ सुवः - गम् । \newline
46. लो॒क मा॑यन् नायन् ॅलो॒कम् ॅलो॒क मा॑य॒न्॒. यद् यदा॑यन् ॅलो॒कम् ॅलो॒क मा॑य॒न्॒. यत् । \newline
47. आ॒य॒न्॒. यद् यदा॑यन् नाय॒न्॒. यच् चत॑स्र॒ श्चत॑स्रो॒ यदा॑यन् नाय॒न्॒. यच् चत॑स्रः । \newline
48. यच् चत॑स्र॒ श्चत॑स्रो॒ यद् यच् चत॑स्रः स्वयमातृ॒ण्णाः स्व॑यमातृ॒ण्णा श्चत॑स्रो॒ यद् यच् चत॑स्रः स्वयमातृ॒ण्णाः । \newline
49. चत॑स्रः स्वयमातृ॒ण्णाः स्व॑यमातृ॒ण्णा श्चत॑स्र॒ श्चत॑स्रः स्वयमातृ॒ण्णा दि॒क्षु दि॒क्षु स्व॑यमातृ॒ण्णा श्चत॑स्र॒ श्चत॑स्रः स्वयमातृ॒ण्णा दि॒क्षु । \newline
50. स्व॒य॒मा॒तृ॒ण्णा दि॒क्षु दि॒क्षु स्व॑यमातृ॒ण्णाः स्व॑यमातृ॒ण्णा दि॒क्षू॑प॒दधा᳚ त्युप॒दधा॑ति दि॒क्षु स्व॑यमातृ॒ण्णाः स्व॑यमातृ॒ण्णा दि॒क्षू॑प॒दधा॑ति । \newline
51. स्व॒य॒मा॒तृ॒ण्णा इति॑ स्वयं - आ॒तृ॒ण्णाः । \newline
52. दि॒क्षू॑ प॒दधा᳚ त्युप॒दधा॑ति दि॒क्षु दि॒क्षू॑ प॒दधा॑ति॒ सर्वत॑श्चक्षुषा॒ सर्वत॑श्चक्षुषो प॒दधा॑ति दि॒क्षु दि॒क्षू॑प॒दधा॑ति॒ सर्वत॑श्चक्षुषा । \newline
53. उ॒प॒दधा॑ति॒ सर्वत॑श्चक्षुषा॒ सर्वत॑श्चक्षुषो प॒दधा᳚ त्युप॒दधा॑ति॒ सर्वत॑श्चक्षु षै॒वैव सर्वत॑श्चक्षुषो प॒दधा᳚ त्युप॒दधा॑ति॒ सर्वत॑श्चक्षु षै॒व । \newline
54. उ॒प॒दधा॒तीत्यु॑प - दधा॑ति । \newline
55. सर्वत॑श्चक्षु षै॒वैव सर्वत॑श्चक्षुषा॒ सर्वत॑श्चक्षुषै॒व तत् तदे॒व सर्वत॑श्चक्षुषा॒ सर्वत॑श्चक्षुषै॒व तत् । \newline
56. सर्वत॑श्चक्षु॒षेति॑ स॒र्वतः॑ - च॒क्षु॒षा॒ । \newline
57. ए॒व तत् तदे॒वैव तद॒ग्निना॒ ऽग्निना॒ तदे॒वैव तद॒ग्निना᳚ । \newline
58. तद॒ग्निना॒ ऽग्निना॒ तत् तद॒ग्निना॒ यज॑मानो॒ यज॑मानो॒ ऽग्निना॒ तत् तद॒ग्निना॒ यज॑मानः । \newline
59. अ॒ग्निना॒ यज॑मानो॒ यज॑मानो॒ ऽग्निना॒ ऽग्निना॒ यज॑मानः सुव॒र्गꣳ सु॑व॒र्गं ॅयज॑मानो॒ ऽग्निना॒ ऽग्निना॒ यज॑मानः सुव॒र्गम् । \newline
60. यज॑मानः सुव॒र्गꣳ सु॑व॒र्गं ॅयज॑मानो॒ यज॑मानः सुव॒र्गम् ॅलो॒कम् ॅलो॒कꣳ सु॑व॒र्गं ॅयज॑मानो॒ यज॑मानः सुव॒र्गम् ॅलो॒कम् । \newline
61. सु॒व॒र्गम् ॅलो॒कम् ॅलो॒कꣳ सु॑व॒र्गꣳ सु॑व॒र्गम् ॅलो॒क मे᳚त्येति लो॒कꣳ सु॑व॒र्गꣳ सु॑व॒र्गम् ॅलो॒क मे॑ति । \newline
62. सु॒व॒र्गमिति॑ सुवः - गम् । \newline
63. लो॒क मे᳚त्येति लो॒कम् ॅलो॒क मे॑ति । \newline
64. ए॒तीत्ये॑ति । \newline
\pagebreak
\markright{ TS 5.5.6.1  \hfill https://www.vedavms.in \hfill}

\section{ TS 5.5.6.1 }

\textbf{TS 5.5.6.1 } \newline
\textbf{Samhita Paata} \newline

अग्न॒ आ या॑हि वी॒तय॒ इत्या॒हा-ह्व॑तै॒वैन॑-म॒ग्निं दू॒तं ॅवृ॑णीमह॒ इत्या॑ह हू॒त्वैवैनं॑ ॅवृणीते॒ ऽग्निना॒ऽग्निः समि॑द्ध्यत॒ इत्या॑ह॒ समि॑न्ध ए॒वैन॑म॒ग्निर्वृ॒त्राणि॑ जङ्घन॒दित्या॑ह॒ समि॑द्ध ए॒वास्मि॑न्निन्द्रि॒यं द॑धात्य॒ग्नेः स्तोमं॑ मनामह॒ इत्या॑ह मनु॒त ए॒वैन॑मे॒तानि॒ वा अह्नाꣳ॑ रू॒पाण्य॑ - [  ] \newline

\textbf{Pada Paata} \newline

अग्ने᳚ । एति॑ । या॒हि॒ । वी॒तये᳚ । इति॑ । आ॒ह॒ । अह्व॑त । ए॒व । ए॒न॒म् । अ॒ग्निम् । दू॒तम् । वृ॒णी॒म॒हे॒ । इति॑ । आ॒ह॒ । हू॒त्वा । ए॒व । ए॒न॒म् । वृ॒णी॒ते॒ । अ॒ग्निना᳚ । अ॒ग्निः । समिति॑ । इ॒द्ध्य॒ते॒ । इति॑ । आ॒ह॒ । समिति॑ । इ॒न्धे॒ । ए॒व । ए॒न॒म् । अ॒ग्निः । वृ॒त्राणि॑ । ज॒ङ्घ॒न॒त् । इति॑ । आ॒ह॒ । समि॑द्ध॒ इति॒ सं - इ॒द्धे॒ । ए॒व । अ॒स्मि॒न्न् । इ॒न्द्रि॒यम् । द॒धा॒ति॒ । अ॒ग्नेः । स्तोम᳚म् । म॒ना॒म॒हे॒ । इति॑ । आ॒ह॒ । म॒नु॒ते । ए॒व । ए॒न॒म् । ए॒तानि॑ । वै । अह्ना᳚म् । रू॒पाणि॑ ।  \newline


\textbf{Krama Paata} \newline

अग्न॒ आ । आ या॑हि । या॒हि॒ वी॒तये᳚ । वी॒तय॒ इति॑ । इत्या॑ह । आ॒हाह्व॑त । अह्व॑तै॒व । ए॒वैन᳚म् । ए॒न॒म॒ग्निम् । अ॒ग्निम् दू॒तम् । दू॒तम् ॅवृ॑णीमहे । वृ॒णी॒म॒ह॒ इति॑ । इत्या॑ह । आ॒ह॒ हू॒त्वा । हू॒त्वैव । ए॒वैन᳚म् । ए॒न॒म् वृ॒णी॒ते॒ । वृ॒णी॒ते॒ऽग्निना᳚ । अ॒ग्निना॒ऽग्निः । अ॒ग्निः सम् । समि॑द्ध्यते । इ॒द्ध्य॒त॒ इति॑ । इत्या॑ह । आ॒ह॒ सम् । समि॑न्धे । इ॒न्ध॒ ए॒व । ए॒वैन᳚म् । ए॒न॒म॒ग्निः । अ॒ग्निर् वृ॒त्राणि॑ । वृ॒त्राणि॑ जङ्घनत् । ज॒ङ्घ॒न॒दिति॑ । इत्या॑ह । आ॒ह॒ समि॑द्धे । समि॑द्ध ए॒व । समि॑द्ध॒ इति॒ सम् - इ॒द्धे॒ । ए॒वास्मिन्न्॑ । अ॒स्मि॒न्नि॒न्द्रि॒यम् । इ॒न्द्रि॒यम् द॑धाति । द॒धा॒त्य॒ग्नेः । अ॒ग्नेः स्तोम᳚म् । स्तोम॑म् मनामहे । म॒ना॒म॒ह॒ इति॑ । इत्या॑ह । आ॒ह॒ म॒नु॒ते । म॒नु॒त ए॒व । ए॒वैन᳚म् । ए॒न॒मे॒तानि॑ । ए॒तानि॒ वै । वा अह्ना᳚म् । अह्नाꣳ॑ रू॒पाणि॑ । रू॒पाण्य॑न्व॒हम् \newline

\textbf{Jatai Paata} \newline

1. अग्न॒ आ ऽग्ने ऽग्न॒ आ । \newline
2. आ या॑हि या॒ह्या या॑हि । \newline
3. या॒हि॒ वी॒तये॑ वी॒तये॑ याहि याहि वी॒तये᳚ । \newline
4. वी॒तय॒ इतीति॑ वी॒तये॑ वी॒तय॒ इति॑ । \newline
5. इत्या॑हा॒हे तीत्या॑ह । \newline
6. आ॒हा ह्व॒ता ह्व॑ता हा॒हा ह्व॑त । \newline
7. अह्व॑तै॒ वैवा ह्व॒ता ह्व॑तै॒व । \newline
8. ए॒वैन॑ मेन मे॒वैवैन᳚म् । \newline
9. ए॒न॒ म॒ग्नि म॒ग्नि मे॑न मेन म॒ग्निम् । \newline
10. अ॒ग्निम् दू॒तम् दू॒त म॒ग्नि म॒ग्निम् दू॒तम् । \newline
11. दू॒तं ॅवृ॑णीमहे वृणीमहे दू॒तम् दू॒तं ॅवृ॑णीमहे । \newline
12. वृ॒णी॒म॒ह॒ इतीति॑ वृणीमहे वृणीमह॒ इति॑ । \newline
13. इत्या॑हा॒हे तीत्या॑ह । \newline
14. आ॒ह॒ हू॒त्वा हू॒त्वा ऽऽहा॑ह हू॒त्वा । \newline
15. हू॒त्वै वैव हू॒त्वा हू॒त्वैव । \newline
16. ए॒वैन॑ मेन मे॒वै वैन᳚म् । \newline
17. ए॒नं॒ ॅवृ॒णी॒ते॒ वृ॒णी॒त॒ ए॒न॒ मे॒नं॒ ॅवृ॒णी॒ते॒ । \newline
18. वृ॒णी॒ते॒ ऽग्निना॒ ऽग्निना॑ वृणीते वृणीते॒ ऽग्निना᳚ । \newline
19. अ॒ग्निना॒ ऽग्नि र॒ग्नि र॒ग्निना॒ ऽग्निना॒ ऽग्निः । \newline
20. अ॒ग्निः सꣳ सम॒ग्नि र॒ग्निः सम् । \newline
21. स मि॑द्ध्यत इद्ध्यते॒ सꣳ स मि॑द्ध्यते । \newline
22. इ॒द्ध्य॒त॒ इतीती᳚द्ध्यत इद्ध्यत॒ इति॑ । \newline
23. इत्या॑हा॒हे तीत्या॑ह । \newline
24. आ॒ह॒ सꣳ समा॑हाह॒ सम् । \newline
25. स मि॑न्ध इन्धे॒ सꣳ स मि॑न्धे । \newline
26. इ॒न्ध॒ ए॒वैवेन्ध॑ इन्ध ए॒व । \newline
27. ए॒वैन॑ मेन मे॒वै वैन᳚म् । \newline
28. ए॒न॒ म॒ग्नि र॒ग्नि रे॑न मेन म॒ग्निः । \newline
29. अ॒ग्निर् वृ॒त्राणि॑ वृ॒त्रा ण्य॒ग्नि र॒ग्निर् वृ॒त्राणि॑ । \newline
30. वृ॒त्राणि॑ जङ्घनज् जङ्घनद् वृ॒त्राणि॑ वृ॒त्राणि॑ जङ्घनत् । \newline
31. ज॒ङ्घ॒न॒दितीति॑ जङ्घनज् जङ्घन॒दिति॑ । \newline
32. इत्या॑हा॒हे तीत्या॑ह । \newline
33. आ॒ह॒ समि॑द्धे॒ समि॑द्ध आहाह॒ समि॑द्धे । \newline
34. समि॑द्ध ए॒वैव समि॑द्धे॒ समि॑द्ध ए॒व । \newline
35. समि॑द्ध॒ इति॒ सं - इ॒द्धे॒ । \newline
36. ए॒वास्मि॑न् नस्मिन् ने॒वै वास्मिन्न्॑ । \newline
37. आ॒स्मि॒न् नि॒न्द्रि॒य मि॑न्द्रि॒य म॑स्मिन् नस्मिन् निन्द्रि॒यम् । \newline
38. इ॒न्द्रि॒यम् द॑धाति दधातीन्द्रि॒य मि॑न्द्रि॒यम् द॑धाति । \newline
39. द॒धा॒ त्य॒ग्ने र॒ग्नेर् द॑धाति दधा त्य॒ग्नेः । \newline
40. अ॒ग्नेः स्तोमꣳ॒॒ स्तोम॑ म॒ग्ने र॒ग्नेः स्तोम᳚म् । \newline
41. स्तोम॑म् मनामहे मनामहे॒ स्तोमꣳ॒॒ स्तोम॑म् मनामहे । \newline
42. म॒ना॒म॒ह॒ इतीति॑ मनामहे मनामह॒ इति॑ । \newline
43. इत्या॑हा॒हे तीत्या॑ह । \newline
44. आ॒ह॒ म॒नु॒ते म॑नु॒त आ॑हाह मनु॒ते । \newline
45. म॒नु॒त ए॒वैव म॑नु॒ते म॑नु॒त ए॒व । \newline
46. ए॒वैन॑ मेन मे॒वै वैन᳚म् । \newline
47. ए॒न॒ मे॒ता न्ये॒ता न्ये॑न मेन मे॒तानि॑ । \newline
48. ए॒तानि॒ वै वा ए॒ता न्ये॒तानि॒ वै । \newline
49. वा अह्ना॒ मह्नां॒ ॅवै वा अह्ना᳚म् । \newline
50. अह्नाꣳ॑ रू॒पाणि॑ रू॒पाण्यह्ना॒ मह्नाꣳ॑ रू॒पाणि॑ । \newline
51. रू॒पा ण्य॑न्व॒ह म॑न्व॒हꣳ रू॒पाणि॑ रू॒पा ण्य॑न्व॒हम् । \newline

\textbf{Ghana Paata } \newline

1. अग्न॒ आ ऽग्ने ऽग्न॒ आ या॑हि या॒ह्या ऽग्ने ऽग्न॒ आ या॑हि । \newline
2. आ या॑हि या॒ह्या या॑हि वी॒तये॑ वी॒तये॑ या॒ह्या या॑हि वी॒तये᳚ । \newline
3. या॒हि॒ वी॒तये॑ वी॒तये॑ याहि याहि वी॒तय॒ इतीति॑ वी॒तये॑ याहि याहि वी॒तय॒ इति॑ । \newline
4. वी॒तय॒ इतीति॑ वी॒तये॑ वी॒तय॒ इत्या॑हा॒हेति॑ वी॒तये॑ वी॒तय॒ इत्या॑ह । \newline
5. इत्या॑हा॒हे तीत्या॒हा ह्व॒ता ह्व॑ता॒ हेती त्या॒हा ह्व॑त । \newline
6. आ॒हा ह्व॒ता ह्व॑ता हा॒हा ह्व॑तै॒वैवा ह्व॑ता हा॒हा ह्व॑तै॒व । \newline
7. अह्व॑तै॒ वैवा ह्व॒ता ह्व॑तै॒ वैन॑ मेन मे॒वा ह्व॒ता ह्व॑तै॒ वैन᳚म् । \newline
8. ए॒वैन॑ मेन मे॒वै वैन॑ म॒ग्नि म॒ग्नि मे॑न मे॒वै वैन॑ म॒ग्निम् । \newline
9. ए॒न॒ म॒ग्नि म॒ग्नि मे॑न मेन म॒ग्निम् दू॒तम् दू॒त म॒ग्नि मे॑न मेन म॒ग्निम् दू॒तम् । \newline
10. अ॒ग्निम् दू॒तम् दू॒त म॒ग्नि म॒ग्निम् दू॒तं ॅवृ॑णीमहे वृणीमहे दू॒त म॒ग्नि म॒ग्निम् दू॒तं ॅवृ॑णीमहे । \newline
11. दू॒तं ॅवृ॑णीमहे वृणीमहे दू॒तम् दू॒तं ॅवृ॑णीमह॒ इतीति॑ वृणीमहे दू॒तम् दू॒तं ॅवृ॑णीमह॒ इति॑ । \newline
12. वृ॒णी॒म॒ह॒ इतीति॑ वृणीमहे वृणीमह॒ इत्या॑हा॒हेति॑ वृणीमहे वृणीमह॒ इत्या॑ह । \newline
13. इत्या॑हा॒हे तीत्या॑ह हू॒त्वा हू॒त्वा ऽऽहे तीत्या॑ह हू॒त्वा । \newline
14. आ॒ह॒ हू॒त्वा हू॒त्वा ऽऽहा॑ह हू॒त्वै वैव हू॒त्वा ऽऽहा॑ह हू॒त्वैव । \newline
15. हू॒त्वै वैव हू॒त्वा हू॒त्वै वैन॑ मेन मे॒व हू॒त्वा हू॒त्वै वैन᳚म् । \newline
16. ए॒वैन॑ मेन मे॒वै वैनं॑ ॅवृणीते वृणीत एन मे॒वै वैनं॑ ॅवृणीते । \newline
17. ए॒नं॒ ॅवृ॒णी॒ते॒ वृ॒णी॒त॒ ए॒न॒ मे॒नं॒ ॅवृ॒णी॒ते॒ ऽग्निना॒ ऽग्निना॑ वृणीत एन मेनं ॅवृणीते॒ ऽग्निना᳚ । \newline
18. वृ॒णी॒ते॒ ऽग्निना॒ ऽग्निना॑ वृणीते वृणीते॒ ऽग्निना॒ ऽग्नि र॒ग्नि र॒ग्निना॑ वृणीते वृणीते॒ ऽग्निना॒ ऽग्निः । \newline
19. अ॒ग्निना॒ ऽग्नि र॒ग्नि र॒ग्निना॒ ऽग्निना॒ ऽग्निः सꣳ स म॒ग्नि र॒ग्निना॒ ऽग्निना॒ ऽग्निः सम् । \newline
20. अ॒ग्निः सꣳ स म॒ग्नि र॒ग्निः स मि॑द्ध्यत इद्ध्यते॒ स म॒ग्नि र॒ग्निः स मि॑द्ध्यते । \newline
21. स मि॑द्ध्यत इद्ध्यते॒ सꣳ स मि॑द्ध्यत॒ इतीती᳚द्ध्यते॒ सꣳ स मि॑द्ध्यत॒ इति॑ । \newline
22. इ॒द्ध्य॒त॒ इतीती᳚द्ध्यत इद्ध्यत॒ इत्या॑हा॒हे ती᳚द्ध्यत इद्ध्यत॒ इत्या॑ह । \newline
23. इत्या॑हा॒हे तीत्या॑ह॒ सꣳ स मा॒हे तीत्या॑ह॒ सम् । \newline
24. आ॒ह॒ सꣳ स मा॑हाह॒ स मि॑न्ध इन्धे॒ स मा॑हाह॒ स मि॑न्धे । \newline
25. स मि॑न्ध इन्धे॒ सꣳ स मि॑न्ध ए॒वैवेन्धे॒ सꣳ स मि॑न्ध ए॒व । \newline
26. इ॒न्ध॒ ए॒वैवेन्ध॑ इन्ध ए॒वैन॑ मेन मे॒वेन्ध॑ इन्ध ए॒वैन᳚म् । \newline
27. ए॒वैन॑ मेन मे॒वै वैन॑ म॒ग्नि र॒ग्नि रे॑न मे॒वै वैन॑ म॒ग्निः । \newline
28. ए॒न॒ म॒ग्नि र॒ग्नि रे॑न मेन म॒ग्निर् वृ॒त्राणि॑ वृ॒त्राण्य॒ ग्नि रे॑न मेन म॒ग्निर् वृ॒त्राणि॑ । \newline
29. अ॒ग्निर् वृ॒त्राणि॑ वृ॒त्राण्य॒ग्नि र॒ग्निर् वृ॒त्राणि॑ जङ्घनज् जङ्घनद् वृ॒त्राण्य॒ग्नि र॒ग्निर् वृ॒त्राणि॑ जङ्घनत् । \newline
30. वृ॒त्राणि॑ जङ्घनज् जङ्घनद् वृ॒त्राणि॑ वृ॒त्राणि॑ जङ्घन॒ दितीति॑ जङ्घनद् वृ॒त्राणि॑ वृ॒त्राणि॑ जङ्घन॒दिति॑ । \newline
31. ज॒ङ्घ॒न॒ दितीति॑ जङ्घनज् जङ्घन॒ दित्या॑हा॒हेति॑ जङ्घनज् जङ्घन॒ दित्या॑ह । \newline
32. इत्या॑हा॒हे तीत्या॑ह॒ समि॑द्धे॒ समि॑द्ध आ॒हे तीत्या॑ह॒ समि॑द्धे । \newline
33. आ॒ह॒ समि॑द्धे॒ समि॑द्ध आहाह॒ समि॑द्ध ए॒वैव समि॑द्ध आहाह॒ समि॑द्ध ए॒व । \newline
34. समि॑द्ध ए॒वैव समि॑द्धे॒ समि॑द्ध ए॒वास्मि॑न् नस्मिन् ने॒व समि॑द्धे॒ समि॑द्ध ए॒वास्मिन्न्॑ । \newline
35. समि॑द्ध॒ इति॒ सं - इ॒द्धे॒ । \newline
36. ए॒वास्मि॑न् नस्मिन् ने॒वै वास्मि॑न् निन्द्रि॒य मि॑न्द्रि॒य म॑स्मिन् ने॒वै वास्मि॑न् निन्द्रि॒यम् । \newline
37. आ॒स्मि॒न् नि॒न्द्रि॒य मि॑न्द्रि॒य म॑स्मिन् नस्मिन् निन्द्रि॒यम् द॑धाति दधातीन्द्रि॒य म॑स्मिन् नस्मिन् निन्द्रि॒यम् द॑धाति । \newline
38. इ॒न्द्रि॒यम् द॑धाति दधातीन्द्रि॒य मि॑न्द्रि॒यम् द॑धा त्य॒ग्ने र॒ग्नेर् द॑धातीन्द्रि॒य मि॑न्द्रि॒यम् द॑धा त्य॒ग्नेः । \newline
39. द॒धा॒ त्य॒ग्ने र॒ग्नेर् द॑धाति दधा त्य॒ग्नेः स्तोमꣳ॒॒ स्तोम॑ म॒ग्नेर् द॑धाति दधा त्य॒ग्नेः स्तोम᳚म् । \newline
40. अ॒ग्नेः स्तोमꣳ॒॒ स्तोम॑ म॒ग्ने र॒ग्नेः स्तोम॑म् मनामहे मनामहे॒ स्तोम॑ म॒ग्ने र॒ग्नेः स्तोम॑म् मनामहे । \newline
41. स्तोम॑म् मनामहे मनामहे॒ स्तोमꣳ॒॒ स्तोम॑म् मनामह॒ इतीति॑ मनामहे॒ स्तोमꣳ॒॒ स्तोम॑म् मनामह॒ इति॑ । \newline
42. म॒ना॒म॒ह॒ इतीति॑ मनामहे मनामह॒ इत्या॑हा॒हेति॑ मनामहे मनामह॒ इत्या॑ह । \newline
43. इत्या॑हा॒हे तीत्या॑ह मनु॒ते म॑नु॒त आ॒हे तीत्या॑ह मनु॒ते । \newline
44. आ॒ह॒ म॒नु॒ते म॑नु॒त आ॑हाह मनु॒त ए॒वैव म॑नु॒त आ॑हाह मनु॒त ए॒व । \newline
45. म॒नु॒त ए॒वैव म॑नु॒ते म॑नु॒त ए॒वैन॑ मेन मे॒व म॑नु॒ते म॑नु॒त ए॒वैन᳚म् । \newline
46. ए॒वैन॑ मेन मे॒वै वैन॑ मे॒ता न्ये॒ता न्ये॑न मे॒वै वैन॑ मे॒तानि॑ । \newline
47. ए॒न॒ मे॒ता न्ये॒ता न्ये॑न मेन मे॒तानि॒ वै वा ए॒ता न्ये॑न मेन मे॒तानि॒ वै । \newline
48. ए॒तानि॒ वै वा ए॒ता न्ये॒तानि॒ वा अह्ना॒ मह्नां॒ ॅवा ए॒ता न्ये॒तानि॒ वा अह्ना᳚म् । \newline
49. वा अह्ना॒ मह्नां॒ ॅवै वा अह्नाꣳ॑ रू॒पाणि॑ रू॒पाण्यह्नां॒ ॅवै वा अह्नाꣳ॑ रू॒पाणि॑ । \newline
50. अह्नाꣳ॑ रू॒पाणि॑ रू॒पाण्यह्ना॒ मह्नाꣳ॑ रू॒पा ण्य॑न्व॒ह म॑न्व॒हꣳ रू॒पा ण्यह्ना॒ मह्नाꣳ॑ रू॒पा ण्य॑न्व॒हम् । \newline
51. रू॒पा ण्य॑न्व॒ह म॑न्व॒हꣳ रू॒पाणि॑ रू॒पा ण्य॑न्व॒ह मे॒वैवान्व॒हꣳ रू॒पाणि॑ रू॒पा ण्य॑न्व॒ह मे॒व । \newline
\pagebreak
\markright{ TS 5.5.6.2  \hfill https://www.vedavms.in \hfill}

\section{ TS 5.5.6.2 }

\textbf{TS 5.5.6.2 } \newline
\textbf{Samhita Paata} \newline

-न्व॒हमे॒वैनं॑ चिनु॒ते ऽवाह्नाꣳ॑ रू॒पाणि॑ रुन्धे ब्रह्मवा॒दिनो॑ वदन्ति॒ कस्मा᳚थ् स॒त्याद्या॒तया᳚म्नीर॒न्या इष्ट॑का॒ अया॑तयाम्नी लोकं पृ॒णेत्यै᳚न्द्रा॒ग्नी हि बा॑र्.हस्प॒त्येति॑ ब्रूयादिन्द्रा॒ग्नी च॒ हि दे॒वानां॒ बृह॒स्पति॒श्चा-या॑तयामानो ऽनुच॒रव॑ती भव॒त्यजा॑मित्वाया -नु॒ष्टुभाऽनु॑ चरत्या॒त्मा वै लो॑कं पृ॒णा प्रा॒णो॑ ऽनु॒ष्टुप् तस्मा᳚त् प्रा॒णः सर्वा॒ण्यङ्गा॒न्यनु॑ चरति॒ ता अ॑स्य॒ सूद॑दोहस॒ - [  ] \newline

\textbf{Pada Paata} \newline

अ॒न्व॒हमित्य॑नु - अ॒हम् । ए॒व । ए॒न॒म् । चि॒नु॒ते॒ । अवेति॑ । अह्ना᳚म् । रू॒पाणि॑ । रु॒न्धे॒ । ब्र॒ह्म॒वा॒दिन॒ इति॑ ब्रह्म-वा॒दिनः॑ । व॒द॒न्ति॒ । कस्मा᳚त् । स॒त्यात् । या॒तया᳚म्नी॒रिति॑ या॒त - या॒म्नीः॒ । अ॒न्याः । इष्ट॑काः । अया॑तया॒म्नीत्यया॑त-या॒म्नी॒ । लो॒क॒पृं॒णेति॑ लोकं - पृ॒णा । इति॑ । ऐ॒न्द्रा॒ग्नीत्यै᳚न्द्र-अ॒ग्नी । हि । बा॒र्.॒ह॒स्प॒त्या । इति॑ । ब्रू॒या॒त् । इ॒न्द्रा॒ग्नी इती᳚न्द्र - अ॒ग्नी । च॒ । हि । दे॒वाना᳚म् । बृह॒स्पतिः॑ । च॒ । अया॑तयामान॒ इत्यया॑त - या॒मा॒नः॒ । अ॒नु॒च॒रव॒तीत्य॑नुच॒र - व॒ती॒ । भ॒व॒ति॒ । अजा॑मित्वा॒येत्यजा॑मि-त्वा॒य॒ । अ॒नु॒ष्टुभेत्य॑नु - स्तुभा᳚ । अन्विति॑ । च॒र॒ति॒ । आ॒त्मा । वै । लो॒क॒पृं॒णेति॑ लोकं - पृ॒णा । प्रा॒ण इति॑ प्र - अ॒नः । अ॒नु॒ष्टुबित्य॑नु - स्तुप् । तस्मा᳚त् । प्रा॒ण इति॑ प्र - अ॒नः । सर्वा॑णि । अङ्गा॑नि । अन्विति॑ । च॒र॒ति॒ । ताः । अ॒स्य॒ । सूद॑दोहस॒ इति॒ सूद॑ - दो॒ह॒सः॒ ।  \newline


\textbf{Krama Paata} \newline

अ॒न्व॒हमे॒व । अ॒न्व॒हमित्य॑नु - अ॒हम् । ए॒वैन᳚म् । ए॒न॒म् चि॒नु॒ते॒ । चि॒नु॒तेऽव॑ । अवाह्ना᳚म् । अह्नाꣳ॑ रू॒पाणि॑ । रू॒पाणि॑ रुन्धे । रु॒न्धे॒ ब्र॒ह्म॒वा॒दिनः॑ । ब्र॒ह्म॒वा॒दिनो॑ वदन्ति । ब्र॒ह्म॒वा॒दिन॒ इति॑ ब्रह्म - वा॒दिनः॑ । व॒द॒न्ति॒ कस्मा᳚त् । कस्मा᳚थ् स॒त्यात् । स॒त्याद् या॒तया᳚म्नीः । या॒तया᳚म्नीर॒न्याः । या॒तया᳚म्नी॒रिति॑ या॒त - या॒म्नीः॒ । अ॒न्या इष्ट॑काः । इष्ट॑का॒ अया॑तयाम्नी । अया॑तयाम्नी लोकम्पृ॒णा । अया॑तया॒म्नीत्यया॑त - या॒म्नी॒ । लो॒क॒म्पृ॒णेति॑ । लो॒क॒म्पृ॒णेति॑ लोकम् - पृ॒णा । इत्यै᳚न्द्रा॒ग्नी । ऐ॒न्द्रा॒ग्नी हि । ऐ॒न्द्रा॒ग्नीत्यै᳚न्द्र - अ॒ग्नी । हि बा॑र्.हस्प॒त्या । बा॒र्॒.ह॒स्प॒त्येति॑ । इति॑ ब्रूयात् । ब्रू॒या॒दि॒न्द्रा॒ग्नी । इ॒न्द्रा॒ग्नी च॑ । इ॒न्द्रा॒ग्नी इती᳚न्द्र - अ॒ग्नी । च॒ हि । हि दे॒वाना᳚म् । दे॒वाना॒म् बृह॒स्पतिः॑ । बृह॒स्पति॑श्च । चाया॑तयामानः । अया॑तयामानोऽनुच॒रव॑ती । अया॑तयामान॒ इत्यया॑त - या॒मा॒नः॒ । अ॒नु॒च॒रव॑ती भवति । अ॒नु॒च॒रव॒तीत्य॑नुच॒र - व॒ती॒ । भ॒व॒त्यजा॑मित्वाय । अजा॑मित्वायानु॒ष्टुभा᳚ । अजा॑मित्वा॒येत्यजा॑मि - त्वा॒य॒ । अ॒नु॒ष्टुभाऽनु॑ । अ॒नु॒ष्टुभेत्य॑नु - स्तुभा᳚ । अनु॑ चरति । च॒र॒त्या॒त्मा । आ॒त्मा वै । वै लो॑कम्पृ॒णा । लो॒क॒म्पृ॒णा प्रा॒णः । लो॒क॒म्पृ॒णेति॑ लोकम् - पृ॒णा । प्रा॒णो॑ऽनु॒ष्टुप् । प्रा॒ण इति॑ प्र - अ॒नः । अ॒नु॒ष्टुप् तस्मा᳚त् । अ॒नु॒ष्टुबित्य॑नु - स्तुप् । तस्मा᳚त् प्रा॒णः । प्रा॒णः सर्वा॑णि । प्रा॒ण इति॑ प्र - अ॒नः । सर्वा॒ण्यङ्गा॑नि । अङ्गा॒न्यनु॑ । अनु॑ चरति । च॒र॒ति॒ ताः । ता अ॑स्य । 
अ॒स्य॒ सूद॑दोहसः । सूद॑दोहस॒ इति॑ । सूद॑दोहस॒ इति॒ सूद॑ - दो॒ह॒सः॒ \newline

\textbf{Jatai Paata} \newline

1. अ॒न्व॒ह मे॒वै वान्व॒ह म॑न्व॒ह मे॒व । \newline
2. अ॒न्व॒हमित्य॑नु - अ॒हम् । \newline
3. ए॒वैन॑ मेन मे॒वै वैन᳚म् । \newline
4. ए॒न॒म् चि॒नु॒ते॒ चि॒नु॒त॒ ए॒न॒ मे॒न॒म् चि॒नु॒ते॒ । \newline
5. चि॒नु॒ते ऽवाव॑ चिनुते चिनु॒ते ऽव॑ । \newline
6. अवाह्ना॒ मह्ना॒ मवा वाह्ना᳚म् । \newline
7. अह्नाꣳ॑ रू॒पाणि॑ रू॒पा ण्यह्ना॒ मह्नाꣳ॑ रू॒पाणि॑ । \newline
8. रू॒पाणि॑ रुन्धे रुन्धे रू॒पाणि॑ रू॒पाणि॑ रुन्धे । \newline
9. रु॒न्धे॒ ब्र॒ह्म॒वा॒दिनो᳚ ब्रह्मवा॒दिनो॑ रुन्धे रुन्धे ब्रह्मवा॒दिनः॑ । \newline
10. ब्र॒ह्म॒वा॒दिनो॑ वदन्ति वदन्ति ब्रह्मवा॒दिनो᳚ ब्रह्मवा॒दिनो॑ वदन्ति । \newline
11. ब्र॒ह्म॒वा॒दिन॒ इति॑ ब्रह्म - वा॒दिनः॑ । \newline
12. व॒द॒न्ति॒ कस्मा॒त् कस्मा᳚द् वदन्ति वदन्ति॒ कस्मा᳚त् । \newline
13. कस्मा᳚थ् स॒त्याथ् स॒त्यात् कस्मा॒त् कस्मा᳚थ् स॒त्यात् । \newline
14. स॒त्याद् या॒तया᳚म्नीर् या॒तया᳚म्नीः स॒त्याथ् स॒त्याद् या॒तया᳚म्नीः । \newline
15. या॒तया᳚म्नीर॒न्या अ॒न्या या॒तया᳚म्नीर् या॒तया᳚म्नी र॒न्याः । \newline
16. या॒तया᳚म्नी॒रिति॑ या॒त - या॒म्नीः॒ । \newline
17. अ॒न्या इष्ट॑का॒ इष्ट॑का अ॒न्या अ॒न्या इष्ट॑काः । \newline
18. इष्ट॑का॒ अया॑तया॒ म्न्यया॑तया॒ म्नीष्ट॑का॒ इष्ट॑का॒ अया॑तया॒म्नी । \newline
19. अया॑तया॒म्नी लो॑कंपृ॒णा लो॑कंपृ॒णा ऽया॑तया॒ म्न्यया॑तया॒म्नी लो॑कंपृ॒णा । \newline
20. अया॑तया॒म्नीत्यया॑त - या॒म्नी॒ । \newline
21. लो॒कं॒पृ॒णेतीति॑ लोकंपृ॒णा लो॑कंपृ॒णेति॑ । \newline
22. लो॒कं॒पृ॒णेति॑ लोकं - पृ॒णा । \newline
23. इत्यै᳚न्द्रा॒ग् न्यै᳚न्द्रा॒ग्नी तीत्यै᳚न्द्रा॒ग्नी । \newline
24. ऐ॒न्द्रा॒ग्नी हि ह्यै᳚न्द्रा॒ ग्न्यै᳚न्द्रा॒ग्नी हि । \newline
25. ऐ॒न्द्रा॒ग्नीत्यै᳚न्द्र - अ॒ग्नी । \newline
26. हि बा॑र्.हस्प॒त्या बा॑र्.हस्प॒त्या हि हि बा॑र्.हस्प॒त्या । \newline
27. बा॒र्॒.ह॒स्प॒त्येतीति॑ बार्.हस्प॒त्या बा॑र्.हस्प॒त्येति॑ । \newline
28. इति॑ ब्रूयाद् ब्रूया॒ दितीति॑ ब्रूयात् । \newline
29. ब्रू॒या॒ दि॒न्द्रा॒ग्नी इ॑न्द्रा॒ग्नी ब्रू॑याद् ब्रूया दिन्द्रा॒ग्नी । \newline
30. इ॒न्द्रा॒ग्नी च॑ चेन्द्रा॒ग्नी इ॑न्द्रा॒ग्नी च॑ । \newline
31. इ॒न्द्रा॒ग्नी इती᳚न्द्र - अ॒ग्नी । \newline
32. च॒ हि हि च॑ च॒ हि । \newline
33. हि दे॒वाना᳚म् दे॒वानाꣳ॒॒ हि हि दे॒वाना᳚म् । \newline
34. दे॒वाना॒म् बृह॒स्पति॒र् बृह॒स्पति॑र् दे॒वाना᳚म् दे॒वाना॒म् बृह॒स्पतिः॑ । \newline
35. बृह॒स्पति॑श्च च॒ बृह॒स्पति॒र् बृह॒स्पति॑श्च । \newline
36. चाया॑तयामा॒नो ऽया॑तयामानश्च॒ चाया॑तयामानः । \newline
37. अया॑तयामानो ऽनुच॒रव॑ त्यनुच॒रव॒ त्यया॑तयामा॒नो ऽया॑तयामानो ऽनुच॒रव॑ती । \newline
38. अया॑तयामान॒ इत्यया॑त - या॒मा॒नः॒ । \newline
39. अ॒नु॒च॒रव॑ती भवति भव त्यनुच॒रव॑ त्यनुच॒रव॑ती भवति । \newline
40. अ॒नु॒च॒रव॒तीत्य॑नुच॒र - व॒ती॒ । \newline
41. भ॒व॒त्य जा॑मित्वा॒या जा॑मित्वाय भवति भव॒त्य जा॑मित्वाय । \newline
42. अजा॑मित्वाया नु॒ष्टुभा॑ ऽनु॒ष्टुभा ऽजा॑मित्वा॒या जा॑मित्वाया नु॒ष्टुभा᳚ । \newline
43. अजा॑मित्वा॒येत्यजा॑मि - त्वा॒य॒ । \newline
44. अ॒नु॒ष्टुभा ऽन्वन् व॑नु॒ष्टुभा॑ ऽनु॒ष्टुभा ऽनु॑ । \newline
45. अ॒नु॒ष्टुभेत्य॑नु - स्तुभा᳚ । \newline
46. अनु॑ चरति चर॒ त्यन्वनु॑ चरति । \newline
47. च॒र॒ त्या॒त्मा ऽऽत्मा च॑रति चर त्या॒त्मा । \newline
48. आ॒त्मा वै वा आ॒त्मा ऽऽत्मा वै । \newline
49. वै लो॑कंपृ॒णा लो॑कंपृ॒णा वै वै लो॑कंपृ॒णा । \newline
50. लो॒कं॒पृ॒णा प्रा॒णः प्रा॒णो लो॑कंपृ॒णा लो॑कंपृ॒णा प्रा॒णः । \newline
51. लो॒कं॒पृ॒णेति॑ लोकं - पृ॒णा । \newline
52. प्रा॒णो॑ ऽनु॒ष्टु ब॑नु॒ष्टुप् प्रा॒णः प्रा॒णो॑ ऽनु॒ष्टुप् । \newline
53. प्रा॒ण इति॑ प्र - अ॒नः । \newline
54. अ॒नु॒ष्टुप् तस्मा॒त् तस्मा॑ दनु॒ष्टु ब॑नु॒ष्टुप् तस्मा᳚त् । \newline
55. अ॒नु॒ष्टुबित्य॑नु - स्तुप् । \newline
56. तस्मा᳚त् प्रा॒णः प्रा॒ण स्तस्मा॒त् तस्मा᳚त् प्रा॒णः । \newline
57. प्रा॒णः सर्वा॑णि॒ सर्वा॑णि प्रा॒णः प्रा॒णः सर्वा॑णि । \newline
58. प्रा॒ण इति॑ प्र - अ॒नः । \newline
59. सर्वा॒ ण्यङ्गा॒ न्यङ्गा॑नि॒ सर्वा॑णि॒ सर्वा॒ ण्यङ्गा॑नि । \newline
60. अङ्गा॒ न्यन् वन् वङ्गा॒ न्यङ्गा॒ न्यनु॑ । \newline
61. अनु॑ चरति चर॒ त्यन्वनु॑ चरति । \newline
62. च॒र॒ति॒ ता स्ता श्च॑रति चरति॒ ताः । \newline
63. ता अ॑स्यास्य॒ ता स्ता अ॑स्य । \newline
64. अ॒स्य॒ सूद॑दोहसः॒ सूद॑दोहसो अस्यास्य॒ सूद॑दोहसः । \newline
65. सूद॑दोहस॒ इतीति॒ सूद॑दोहसः॒ सूद॑दोहस॒ इति॑ । \newline
66. सूद॑दोहस॒ इति॒ सूद॑ - दो॒ह॒सः॒ । \newline

\textbf{Ghana Paata } \newline

1. अ॒न्व॒ह मे॒वै वान्व॒ह म॑न्व॒ह मे॒वैन॑ मेन मे॒वान्व॒ह म॑न्व॒ह मे॒वैन᳚म् । \newline
2. अ॒न्व॒हमित्य॑नु - अ॒हम् । \newline
3. ए॒वैन॑ मेन मे॒वै वैन॑म् चिनुते चिनुत एन मे॒वै वैन॑म् चिनुते । \newline
4. ए॒न॒म् चि॒नु॒ते॒ चि॒नु॒त॒ ए॒न॒ मे॒न॒म् चि॒नु॒ते ऽवाव॑ चिनुत एन मेनम् चिनु॒ते ऽव॑ । \newline
5. चि॒नु॒ते ऽवाव॑ चिनुते चिनु॒ते ऽवाह्ना॒ मह्ना॒ मव॑ चिनुते चिनु॒ते ऽवाह्ना᳚म् । \newline
6. अवाह्ना॒ मह्ना॒ मवावा ह्नाꣳ॑ रू॒पाणि॑ रू॒पाण्यह्ना॒ मवावा ह्नाꣳ॑ रू॒पाणि॑ । \newline
7. अह्नाꣳ॑ रू॒पाणि॑ रू॒पा ण्यह्ना॒ मह्नाꣳ॑ रू॒पाणि॑ रुन्धे रुन्धे रू॒पा ण्यह्ना॒ मह्नाꣳ॑ रू॒पाणि॑ रुन्धे । \newline
8. रू॒पाणि॑ रुन्धे रुन्धे रू॒पाणि॑ रू॒पाणि॑ रुन्धे ब्रह्मवा॒दिनो᳚ ब्रह्मवा॒दिनो॑ रुन्धे रू॒पाणि॑ रू॒पाणि॑ रुन्धे ब्रह्मवा॒दिनः॑ । \newline
9. रु॒न्धे॒ ब्र॒ह्म॒वा॒दिनो᳚ ब्रह्मवा॒दिनो॑ रुन्धे रुन्धे ब्रह्मवा॒दिनो॑ वदन्ति वदन्ति ब्रह्मवा॒दिनो॑ रुन्धे रुन्धे ब्रह्मवा॒दिनो॑ वदन्ति । \newline
10. ब्र॒ह्म॒वा॒दिनो॑ वदन्ति वदन्ति ब्रह्मवा॒दिनो᳚ ब्रह्मवा॒दिनो॑ वदन्ति॒ कस्मा॒त् कस्मा᳚द् वदन्ति ब्रह्मवा॒दिनो᳚ ब्रह्मवा॒दिनो॑ वदन्ति॒ कस्मा᳚त् । \newline
11. ब्र॒ह्म॒वा॒दिन॒ इति॑ ब्रह्म - वा॒दिनः॑ । \newline
12. व॒द॒न्ति॒ कस्मा॒त् कस्मा᳚द् वदन्ति वदन्ति॒ कस्मा᳚थ् स॒त्याथ् स॒त्यात् कस्मा᳚द् वदन्ति वदन्ति॒ कस्मा᳚थ् स॒त्यात् । \newline
13. कस्मा᳚थ् स॒त्याथ् स॒त्यात् कस्मा॒त् कस्मा᳚थ् स॒त्याद् या॒तया᳚म्नीर् या॒तया᳚म्नीः स॒त्यात् कस्मा॒त् कस्मा᳚थ् स॒त्याद् या॒तया᳚म्नीः । \newline
14. स॒त्याद् या॒तया᳚म्नीर् या॒तया᳚म्नीः स॒त्याथ् स॒त्याद् या॒तया᳚म्नी र॒न्या अ॒न्या या॒तया᳚म्नीः स॒त्याथ् स॒त्याद् या॒तया᳚म्नी र॒न्याः । \newline
15. या॒तया᳚म्नी र॒न्या अ॒न्या या॒तया᳚म्नीर् या॒तया᳚म्नी र॒न्या इष्ट॑का॒ इष्ट॑का अ॒न्या या॒तया᳚म्नीर् या॒तया᳚म्नी र॒न्या इष्ट॑काः । \newline
16. या॒तया᳚म्नी॒रिति॑ या॒त - या॒म्नीः॒ । \newline
17. अ॒न्या इष्ट॑का॒ इष्ट॑का अ॒न्या अ॒न्या इष्ट॑का॒ अया॑तया॒ म्न्यया॑तया॒ म्नीष्ट॑का अ॒न्या अ॒न्या इष्ट॑का॒ अया॑तया॒म्नी । \newline
18. इष्ट॑का॒ अया॑तया॒ म्न्यया॑तया॒म्नी ष्ट॑का॒ इष्ट॑का॒ अया॑तया॒म्नी लो॑कंपृ॒णा लो॑कंपृ॒णा ऽया॑तया॒म्नी ष्ट॑का॒ इष्ट॑का॒ अया॑तया॒म्नी लो॑कंपृ॒णा । \newline
19. अया॑तया॒म्नी लो॑कंपृ॒णा लो॑कंपृ॒णा ऽया॑तया॒ म्न्यया॑तया॒म्नी लो॑कंपृ॒णेतीति॑ लोकंपृ॒णा ऽया॑तया॒ म्न्यया॑तया॒म्नी लो॑कंपृ॒णेति॑ । \newline
20. अया॑तया॒म्नीत्यया॑त - या॒म्नी॒ । \newline
21. लो॒कं॒पृ॒णेतीति॑ लोकंपृ॒णा लो॑कंपृ॒णे त्यै᳚न्द्रा॒ ग्न्यै᳚न्द्रा॒ग्नीति॑ लोकंपृ॒णा लो॑कंपृ॒णे त्यै᳚न्द्रा॒ग्नी । \newline
22. लो॒कं॒पृ॒णेति॑ लोकं - पृ॒णा । \newline
23. इत्यै᳚न्द्रा॒ ग्न्यै᳚न्द्रा॒ग्नीती त्यै᳚न्द्रा॒ग्नी हि ह्यै᳚न्द्रा॒ग्नीती त्यै᳚न्द्रा॒ग्नी हि । \newline
24. ऐ॒न्द्रा॒ग्नी हि ह्यै᳚न्द्रा॒ ग्न्यै᳚न्द्रा॒ग्नी हि बा॑र्.हस्प॒त्या बा॑र्.हस्प॒त्या ह्यै᳚न्द्रा॒ ग्न्यै᳚न्द्रा॒ग्नी हि बा॑र्.हस्प॒त्या । \newline
25. ऐ॒न्द्रा॒ग्नीत्यै᳚न्द्र - अ॒ग्नी । \newline
26. हि बा॑र्.हस्प॒त्या बा॑र्.हस्प॒त्या हि हि बा॑र्.हस्प॒ त्येतीति॑ बार्.हस्प॒त्या हि हि बा॑र्.हस्प॒त्येति॑ । \newline
27. बा॒र्॒.ह॒स्प॒ त्येतीति॑ बार्.हस्प॒त्या बा॑र्.हस्प॒त्येति॑ ब्रूयाद् ब्रूया॒दिति॑ बार्.हस्प॒त्या बा॑र्.हस्प॒त्येति॑ ब्रूयात् । \newline
28. इति॑ ब्रूयाद् ब्रूया॒दितीति॑ ब्रूया दिन्द्रा॒ग्नी इ॑न्द्रा॒ग्नी ब्रू॑या॒ दितीति॑ ब्रूया दिन्द्रा॒ग्नी । \newline
29. ब्रू॒या॒ दि॒न्द्रा॒ग्नी इ॑न्द्रा॒ग्नी ब्रू॑याद् ब्रूया दिन्द्रा॒ग्नी च॑ चेन्द्रा॒ग्नी ब्रू॑याद् ब्रूया दिन्द्रा॒ग्नी च॑ । \newline
30. इ॒न्द्रा॒ग्नी च॑ चेन्द्रा॒ग्नी इ॑न्द्रा॒ग्नी च॒ हि हि चे᳚न्द्रा॒ग्नी इ॑न्द्रा॒ग्नी च॒ हि । \newline
31. इ॒न्द्रा॒ग्नी इती᳚न्द्र - अ॒ग्नी । \newline
32. च॒ हि हि च॑ च॒ हि दे॒वाना᳚म् दे॒वानाꣳ॒॒ हि च॑ च॒ हि दे॒वाना᳚म् । \newline
33. हि दे॒वाना᳚म् दे॒वानाꣳ॒॒ हि हि दे॒वाना॒म् बृह॒स्पति॒र् बृह॒स्पति॑र् दे॒वानाꣳ॒॒ हि हि दे॒वाना॒म् बृह॒स्पतिः॑ । \newline
34. दे॒वाना॒म् बृह॒स्पति॒र् बृह॒स्पति॑र् दे॒वाना᳚म् दे॒वाना॒म् बृह॒स्पति॑श्च च॒ बृह॒स्पति॑र् दे॒वाना᳚म् दे॒वाना॒म् बृह॒स्पति॑श्च । \newline
35. बृह॒स्पति॑श्च च॒ बृह॒स्पति॒र् बृह॒स्पति॒ श्चाया॑तयामा॒नो ऽया॑तयामानश्च॒ बृह॒स्पति॒र् बृह॒स्पति॒ श्चाया॑तयामानः । \newline
36. चाया॑तयामा॒नो ऽया॑तयामानश्च॒ चाया॑तयामानो ऽनुच॒रव॑ त्यनुच॒रव॒ त्यया॑तयामानश्च॒ चाया॑तयामानो ऽनुच॒रव॑ती । \newline
37. अया॑तयामानो ऽनुच॒रव॑ त्यनुच॒रव॒ त्यया॑तयामा॒नो ऽया॑तयामानो ऽनुच॒रव॑ती भवति भव त्यनुच॒रव॒ त्यया॑तयामा॒नो ऽया॑तयामानो ऽनुच॒रव॑ती भवति । \newline
38. अया॑तयामान॒ इत्यया॑त - या॒मा॒नः॒ । \newline
39. अ॒नु॒च॒रव॑ती भवति भव त्यनुच॒रव॑ त्यनुच॒रव॑ती भव॒ त्यजा॑मित्वा॒या जा॑मित्वाय भव त्यनुच॒रव॑ त्यनुच॒रव॑ती भव॒ त्यजा॑मित्वाय । \newline
40. अ॒नु॒च॒रव॒तीत्य॑नुच॒र - व॒ती॒ । \newline
41. भ॒व॒त्य जा॑मित्वा॒या जा॑मित्वाय भवति भव॒ त्यजा॑मित्वाया नु॒ष्टुभा॑ ऽनु॒ष्टुभा ऽजा॑मित्वाय भवति भव॒
त्यजा॑मित्वाया नु॒ष्टुभा᳚ । \newline
42. अजा॑मित्वाया नु॒ष्टुभा॑ ऽनु॒ष्टुभा ऽजा॑मित्वा॒या जा॑मित्वाया नु॒ष्टुभा ऽन्वन् व॑नु॒ष्टुभा ऽजा॑मित्वा॒या जा॑मित्वाया नु॒ष्टुभा ऽनु॑ । \newline
43. अजा॑मित्वा॒येत्यजा॑मि - त्वा॒य॒ । \newline
44. अ॒नु॒ष्टुभा ऽन्वन् व॑नु॒ष्टुभा॑ ऽनु॒ष्टुभा ऽनु॑ चरति चर॒ त्यन् व॑नु॒ष्टुभा॑ ऽनु॒ष्टुभा ऽनु॑ चरति । \newline
45. अ॒नु॒ष्टुभेत्य॑नु - स्तुभा᳚ । \newline
46. अनु॑ चरति चर॒ त्यन् वनु॑ चर त्या॒त्मा ऽऽत्मा च॑र॒ त्यन् वनु॑ चर त्या॒त्मा । \newline
47. च॒र॒ त्या॒त्मा ऽऽत्मा च॑रति चर त्या॒त्मा वै वा आ॒त्मा च॑रति चर त्या॒त्मा वै । \newline
48. आ॒त्मा वै वा आ॒त्मा ऽऽत्मा वै लो॑कंपृ॒णा लो॑कंपृ॒णा वा आ॒त्मा ऽऽत्मा वै लो॑कंपृ॒णा । \newline
49. वै लो॑कंपृ॒णा लो॑कंपृ॒णा वै वै लो॑कंपृ॒णा प्रा॒णः प्रा॒णो लो॑कंपृ॒णा वै वै लो॑कंपृ॒णा प्रा॒णः । \newline
50. लो॒कं॒पृ॒णा प्रा॒णः प्रा॒णो लो॑कंपृ॒णा लो॑कंपृ॒णा प्रा॒णो॑ ऽनु॒ष्टु ब॑नु॒ष्टुप् प्रा॒णो लो॑कंपृ॒णा लो॑कंपृ॒णा प्रा॒णो॑ ऽनु॒ष्टुप् । \newline
51. लो॒कं॒पृ॒णेति॑ लोकं - पृ॒णा । \newline
52. प्रा॒णो॑ ऽनु॒ष्टु ब॑नु॒ष्टुप् प्रा॒णः प्रा॒णो॑ ऽनु॒ष्टुप् तस्मा॒त् तस्मा॑ दनु॒ष्टुप् प्रा॒णः प्रा॒णो॑ ऽनु॒ष्टुप् तस्मा᳚त् । \newline
53. प्रा॒ण इति॑ प्र - अ॒नः । \newline
54. अ॒नु॒ष्टुप् तस्मा॒त् तस्मा॑ दनु॒ष्टु ब॑नु॒ष्टुप् तस्मा᳚त् प्रा॒णः प्रा॒ण स्तस्मा॑ दनु॒ष्टु ब॑नु॒ष्टुप् तस्मा᳚त् प्रा॒णः । \newline
55. अ॒नु॒ष्टुबित्य॑नु - स्तुप् । \newline
56. तस्मा᳚त् प्रा॒णः प्रा॒ण स्तस्मा॒त् तस्मा᳚त् प्रा॒णः सर्वा॑णि॒ सर्वा॑णि प्रा॒ण स्तस्मा॒त् तस्मा᳚त् प्रा॒णः सर्वा॑णि । \newline
57. प्रा॒णः सर्वा॑णि॒ सर्वा॑णि प्रा॒णः प्रा॒णः सर्वा॒ ण्यङ्गा॒ न्यङ्गा॑नि॒ सर्वा॑णि प्रा॒णः प्रा॒णः सर्वा॒
ण्यङ्गा॑नि । \newline
58. प्रा॒ण इति॑ प्र - अ॒नः । \newline
59. सर्वा॒ ण्यङ्गा॒ न्यङ्गा॑नि॒ सर्वा॑णि॒ सर्वा॒ ण्यङ्गा॒ न्यन् वन् वङ्गा॑नि॒ सर्वा॑णि॒ सर्वा॒ ण्यङ्गा॒ न्यनु॑ । \newline
60. अङ्गा॒ न्यन्वन् वङ्गा॒ न्यङ्गा॒ न्यनु॑ चरति चर॒ त्यन् वङ्गा॒ न्यङ्गा॒ न्यनु॑ चरति । \newline
61. अनु॑ चरति चर॒ त्यन् वनु॑ चरति॒ ता स्ता श्च॑र॒ त्यन् वनु॑ चरति॒ ताः । \newline
62. च॒र॒ति॒ ता स्ता श्च॑रति चरति॒ ता अ॑स्यास्य॒ ता श्च॑रति चरति॒ ता अ॑स्य । \newline
63. ता अ॑स्यास्य॒ ता स्ता अ॑स्य॒ सूद॑दोहसः॒ सूद॑दोहसो अस्य॒ ता स्ता अ॑स्य॒ सूद॑दोहसः । \newline
64. अ॒स्य॒ सूद॑दोहसः॒ सूद॑दोहसो अस्यास्य॒ सूद॑दोहस॒ इतीति॒ सूद॑दोहसो अस्यास्य॒ सूद॑दोहस॒ इति॑ । \newline
65. सूद॑दोहस॒ इतीति॒ सूद॑दोहसः॒ सूद॑दोहस॒ इत्या॑हा॒हेति॒ सूद॑दोहसः॒ सूद॑दोहस॒ इत्या॑ह । \newline
66. सूद॑दोहस॒ इति॒ सूद॑ - दो॒ह॒सः॒ । \newline
\pagebreak
\markright{ TS 5.5.6.3  \hfill https://www.vedavms.in \hfill}

\section{ TS 5.5.6.3 }

\textbf{TS 5.5.6.3 } \newline
\textbf{Samhita Paata} \newline

इत्या॑ह॒ तस्मा॒त् परु॑षिपरुषि॒ रसः॒ सोमꣳ॑ श्रीणन्ति॒ पृश्न॑य॒ इत्या॒हान्नं॒ ॅवै पृश्न्यन्न॑मे॒वाव॑ रुन्धे॒ऽर्को वा अ॒ग्निर॒र्कोऽन्न॒मन्न॑मे॒वाव॑ रुन्धे॒ जन्म॑न् दे॒वानां॒ ॅविश॑स्त्रि॒ष्वा रो॑च॒ने दि॒व इत्या॑हे॒माने॒वास्मै॑ लो॒कान् ज्योति॑ष्मतः करोति॒ यो वा इष्ट॑कानां प्रति॒ष्ठां ॅवेद॒ प्रत्ये॒व ति॑ष्ठति॒ तया॑ ( ) दे॒वत॑याऽङ्गिर॒स्वद् ध्रु॒वा सी॒देत्या॑है॒षा वा इष्ट॑कानां प्रति॒ष्ठा य ए॒वं ॅवेद॒ प्रत्ये॒वति॑ष्ठति ॥ \newline

\textbf{Pada Paata} \newline

इति॑ । आ॒ह॒ । तस्मा᳚त् । परु॑षिपरु॒षीति॒ परु॑षि - प॒रु॒षि॒ । रसः॑ । सोम᳚म् । श्री॒ण॒न्ति॒ । पृश्न॑यः । इति॑ । आ॒ह॒ । अन्न᳚म् । वै । पृश्नि॑ । अन्न᳚म् । ए॒व । अवेति॑ । रु॒न्धे॒ । अ॒र्कः । वै । अ॒ग्निः । अ॒र्कः । अन्न᳚म् । अन्न᳚म् । ए॒व । अवेति॑ । रु॒न्धे॒ । जन्मन्न्॑ । दे॒वाना᳚म् । विशः॑ । त्रि॒षु । एति॑ । रो॒च॒ने । दि॒वः । इति॑ । आ॒ह॒ । इ॒मान् । ए॒व । अ॒स्मै॒ । लो॒कान् । ज्योति॑ष्मतः । क॒रो॒ति॒ । यः । वै । इष्ट॑कानाम् । प्र॒ति॒ष्ठामिति॑ प्रति - स्थाम् । वेद॑ । प्रतीति॑ । ए॒व । ति॒ष्ठ॒ति॒ । तया᳚ ( ) । दे॒वत॑या । अ॒ङ्गि॒र॒स्वत् । ध्रु॒वा । सी॒द॒ । इति॑ । आ॒ह॒ । ए॒षा । वै । इष्ट॑कानाम् । प्र॒ति॒ष्ठेति॑ प्रति - स्था । यः । ए॒वम् । वेद॑ । प्रतीति॑ । ए॒व । ति॒ष्ठ॒ति॒ ॥  \newline


\textbf{Krama Paata} \newline

इत्या॑ह । आ॒ह॒ तस्मा᳚त् । तस्मा॒त् परु॑षिपरुषि । परु॑षिपरुषि॒ रसः॑ । परु॑षिपरु॒षीति॒ परु॑षि - प॒रु॒षि॒ । रसः॒ सोम᳚म् । सोमꣳ॑ श्रीणन्ति । श्री॒ण॒न्ति॒ पृश्ञ॑यः । पृश्ञ॑य॒ इति॑ । इत्या॑ह । आ॒हान्न᳚म् । अन्न॒म् ॅवै । वै पृश्ञि॑ । पृश्ञ्यन्न᳚म् । अन्न॑मे॒व । ए॒वाव॑ । अव॑ रुन्धे । रु॒न्धे॒ऽर्कः । अ॒र्को वै । वा अ॒ग्निः । अ॒ग्निर॒र्कः । अ॒र्कोऽन्न᳚म् । अन्न॒मन्न᳚म् । अन्न॑मे॒व । ए॒वाव॑ । अव॑ रुन्धे । रु॒न्धे॒ जन्मन्न्॑ । जन्म॑न् दे॒वाना᳚म् । दे॒वाना॒म् ॅविशः॑ । विश॑स्त्रि॒षु । त्रि॒ष्वा । आ रो॑च॒ने । रो॒च॒ने दि॒वः । दि॒व इति॑ । इत्या॑ह । आ॒हे॒मान् । इ॒माने॒व । ए॒वास्मै᳚ । अ॒स्मै॒ लो॒कान् । लो॒कान् ज्योति॑ष्मतः । ज्योति॑ष्मतः करोति । क॒रो॒ति॒ यः । यो वै । वा इष्ट॑कानाम् । इष्ट॑कानाम् प्रति॒ष्ठाम् । प्र॒ति॒ष्ठाम् ॅवेद॑ । प्र॒ति॒ष्ठामिति॑ प्रति - स्थाम् । वेद॒ प्रति॑ । प्रत्ये॒व । ए॒व ति॑ष्ठति । ति॒ष्ठ॒ति॒ तया᳚ ( ) । तया॑ दे॒वत॑या । दे॒वत॑याऽङ्गिर॒स्वत् । अ॒ङ्गि॒र॒स्वद् ध्रु॒वा । ध्रु॒वा सी॑द । सी॒देति॑ । इत्या॑ह । आ॒है॒षा । ए॒षा वै । वा इष्ट॑कानाम् । इष्ट॑कानाम् प्रति॒ष्ठा । प्र॒ति॒ष्ठा यः । प्र॒ति॒ष्ठेति॑ प्रति - स्था । य ए॒वम् । ए॒वम् ॅवेद॑ । वेद॒ प्रति॑ । प्रत्ये॒व । ए॒व ति॑ष्ठति । ति॒ष्ठ॒तीति॑ तिष्ठति । \newline

\textbf{Jatai Paata} \newline

1. इत्या॑हा॒हे तीत्या॑ह । \newline
2. आ॒ह॒ तस्मा॒त् तस्मा॑ दाहाह॒ तस्मा᳚त् । \newline
3. तस्मा॒त् परु॑षिपरुषि॒ परु॑षिपरुषि॒ तस्मा॒त् तस्मा॒त् परु॑षिपरुषि । \newline
4. परु॑षिपरुषि॒ रसो॒ रसः॒ परु॑षिपरुषि॒ परु॑षिपरुषि॒ रसः॑ । \newline
5. परु॑षिपरु॒षीति॒ परु॑षि - प॒रु॒षि॒ । \newline
6. रसः॒ सोमꣳ॒॒ सोमꣳ॒॒ रसो॒ रसः॒ सोम᳚म् । \newline
7. सोमꣳ॑ श्रीणन्ति श्रीणन्ति॒ सोमꣳ॒॒ सोमꣳ॑ श्रीणन्ति । \newline
8. श्री॒ण॒न्ति॒ पृश्ञ॑यः॒ पृश्ञ॑यः श्रीणन्ति श्रीणन्ति॒ पृश्ञ॑यः । \newline
9. पृश्ञ॑य॒ इतीति॒ पृश्ञ॑यः॒ पृश्ञ॑य॒ इति॑ । \newline
10. इत्या॑हा॒हे तीत्या॑ह । \newline
11. आ॒हान्न॒ मन्न॑ माहा॒ हान्न᳚म् । \newline
12. अन्नं॒ ॅवै वा अन्न॒ मन्नं॒ ॅवै । \newline
13. वै पृश्ञि॒ पृश्ञि॒ वै वै पृश्ञि॑ । \newline
14. पृश्ञ्यन्न॒ मन्न॒म् पृश्ञि॒ पृश्ञ्यन्न᳚म् । \newline
15. अन्न॑ मे॒वै वान्न॒ मन्न॑ मे॒व । \newline
16. ए॒वावा वै॒वै वाव॑ । \newline
17. अव॑ रुन्धे रु॒न्धे ऽवाव॑ रुन्धे । \newline
18. रु॒न्धे॒ ऽर्को᳚ ऽर्को रु॑न्धे रुन्धे॒ ऽर्कः । \newline
19. अ॒र्को वै वा अ॒र्को᳚ ऽर्को वै । \newline
20. वा अ॒ग्नि र॒ग्निर् वै वा अ॒ग्निः । \newline
21. अ॒ग्नि र॒र्को᳚(1॒) ऽर्को᳚ ऽग्नि र॒ग्नि र॒र्कः । \newline
22. अ॒र्को ऽन्न॒ मन्न॑ म॒र्को᳚ ऽर्को ऽन्न᳚म् । \newline
23. अन्न॒ मन्न᳚म् । \newline
24. अन्न॑ मे॒वै वान्न॒ मन्न॑ मे॒व । \newline
25. ए॒वावा वै॒वै वाव॑ । \newline
26. अव॑ रुन्धे रु॒न्धे ऽवाव॑ रुन्धे । \newline
27. रु॒न्धे॒ जन्म॒न् जन्म॑न् रुन्धे रुन्धे॒ जन्मन्न्॑ । \newline
28. जन्म॑न् दे॒वाना᳚म् दे॒वाना॒म् जन्म॒न् जन्म॑न् दे॒वाना᳚म् । \newline
29. दे॒वानां॒ ॅविशो॒ विशो॑ दे॒वाना᳚म् दे॒वानां॒ ॅविशः॑ । \newline
30. विश॑ स्त्रि॒षुत् त्रि॒षु विशो॒ विश॑ स्त्रि॒षु । \newline
31. त्रि॒ष्वा त्रि॒षुत् त्रि॒ष्वा । \newline
32. आ रो॑च॒ने रो॑च॒न आ रो॑च॒ने । \newline
33. रो॒च॒ने दि॒वो दि॒वो रो॑च॒ने रो॑च॒ने दि॒वः । \newline
34. दि॒व इतीति॑ दि॒वो दि॒व इति॑ । \newline
35. इत्या॑हा॒हे तीत्या॑ह । \newline
36. आ॒हे॒ मानि॒मा ना॑हाहे॒ मान् । \newline
37. इ॒मा ने॒वैवे मा नि॒मा ने॒व । \newline
38. ए॒वास्मा॑ अस्मा ए॒वै वास्मै᳚ । \newline
39. अ॒स्मै॒ लो॒कान् ॅलो॒का न॑स्मा अस्मै लो॒कान् । \newline
40. लो॒कान् ज्योति॑ष्मतो॒ ज्योति॑ष्मतो लो॒कान् ॅलो॒कान् ज्योति॑ष्मतः । \newline
41. ज्योति॑ष्मतः करोति करोति॒ ज्योति॑ष्मतो॒ ज्योति॑ष्मतः करोति । \newline
42. क॒रो॒ति॒ यो यः क॑रोति करोति॒ यः । \newline
43. यो वै वै यो यो वै । \newline
44. वा इष्ट॑काना॒ मिष्ट॑कानां॒ ॅवै वा इष्ट॑कानाम् । \newline
45. इष्ट॑कानाम् प्रति॒ष्ठाम् प्र॑ति॒ष्ठा मिष्ट॑काना॒ मिष्ट॑कानाम् प्रति॒ष्ठाम् । \newline
46. प्र॒ति॒ष्ठां ॅवेद॒ वेद॑ प्रति॒ष्ठाम् प्र॑ति॒ष्ठां ॅवेद॑ । \newline
47. प्र॒ति॒ष्ठामिति॑ प्रति - स्थाम् । \newline
48. वेद॒ प्रति॒ प्रति॒ वेद॒ वेद॒ प्रति॑ । \newline
49. प्रत्ये॒वैव प्रति॒ प्रत्ये॒व । \newline
50. ए॒व ति॑ष्ठति तिष्ठ त्ये॒वैव ति॑ष्ठति । \newline
51. ति॒ष्ठ॒ति॒ तया॒ तया॑ तिष्ठति तिष्ठति॒ तया᳚ । \newline
52. तया॑ दे॒वत॑या दे॒वत॑या॒ तया॒ तया॑ दे॒वत॑या । \newline
53. दे॒वत॑या ऽङ्गिर॒स्व द॑ङ्गिर॒स्वद् दे॒वत॑या दे॒वत॑या ऽङ्गिर॒स्वत् । \newline
54. अ॒ङ्गि॒र॒स्वद् ध्रु॒वा ध्रु॒वा ऽङ्गि॑र॒स्व द॑ङ्गिर॒स्वद् ध्रु॒वा । \newline
55. ध्रु॒वा सी॑द सीद ध्रु॒वा ध्रु॒वा सी॑द । \newline
56. सी॒दे तीति॑ सीद सी॒देति॑ । \newline
57. इत्या॑हा॒हे तीत्या॑ह । \newline
58. आ॒है॒ षैषा ऽऽहा॑है॒षा । \newline
59. ए॒षा वै वा ए॒षैषा वै । \newline
60. वा इष्ट॑काना॒ मिष्ट॑कानां॒ ॅवै वा इष्ट॑कानाम् । \newline
61. इष्ट॑कानाम् प्रति॒ष्ठा प्र॑ति॒ष्ठे ष्ट॑काना॒ मिष्ट॑कानाम् प्रति॒ष्ठा । \newline
62. प्र॒ति॒ष्ठा यो यः प्र॑ति॒ष्ठा प्र॑ति॒ष्ठा यः । \newline
63. प्र॒ति॒ष्ठेति॑ प्रति - स्था । \newline
64. य ए॒व मे॒वं ॅयो य ए॒वम् । \newline
65. ए॒वं ॅवेद॒ वेदै॒व मे॒वं ॅवेद॑ । \newline
66. वेद॒ प्रति॒ प्रति॒ वेद॒ वेद॒ प्रति॑ । \newline
67. प्रत्ये॒वैव प्रति॒ प्रत्ये॒व । \newline
68. ए॒व ति॑ष्ठति तिष्ठ त्ये॒वैव ति॑ष्ठति । \newline
69. ति॒ष्ठ॒तीति॑ तिष्ठति । \newline

\textbf{Ghana Paata } \newline

1. इत्या॑हा॒हे तीत्या॑ह॒ तस्मा॒त् तस्मा॑ दा॒हे तीत्या॑ह॒ तस्मा᳚त् । \newline
2. आ॒ह॒ तस्मा॒त् तस्मा॑ दाहाह॒ तस्मा॒त् परु॑षिपरुषि॒ परु॑षिपरुषि॒ तस्मा॑ दाहाह॒ तस्मा॒त् परु॑षिपरुषि । \newline
3. तस्मा॒त् परु॑षिपरुषि॒ परु॑षिपरुषि॒ तस्मा॒त् तस्मा॒त् परु॑षिपरुषि॒ रसो॒ रसः॒ परु॑षिपरुषि॒ तस्मा॒त् तस्मा॒त् परु॑षिपरुषि॒ रसः॑ । \newline
4. परु॑षिपरुषि॒ रसो॒ रसः॒ परु॑षिपरुषि॒ परु॑षिपरुषि॒ रसः॒ सोमꣳ॒॒ सोमꣳ॒॒ रसः॒ परु॑षिपरुषि॒ परु॑षिपरुषि॒ रसः॒ सोम᳚म् । \newline
5. परु॑षिपरु॒षीति॒ परु॑षि - प॒रु॒षि॒ । \newline
6. रसः॒ सोमꣳ॒॒ सोमꣳ॒॒ रसो॒ रसः॒ सोमꣳ॑ श्रीणन्ति श्रीणन्ति॒ सोमꣳ॒॒ रसो॒ रसः॒ सोमꣳ॑ श्रीणन्ति । \newline
7. सोमꣳ॑ श्रीणन्ति श्रीणन्ति॒ सोमꣳ॒॒ सोमꣳ॑ श्रीणन्ति॒ पृश्ञ॑यः॒ पृश्ञ॑यः श्रीणन्ति॒ सोमꣳ॒॒ सोमꣳ॑ श्रीणन्ति॒ पृश्ञ॑यः । \newline
8. श्री॒ण॒न्ति॒ पृश्ञ॑यः॒ पृश्ञ॑यः श्रीणन्ति श्रीणन्ति॒ पृश्ञ॑य॒ इतीति॒ पृश्ञ॑यः श्रीणन्ति श्रीणन्ति॒ पृश्ञ॑य॒ इति॑ । \newline
9. पृश्ञ॑य॒ इतीति॒ पृश्ञ॑यः॒ पृश्ञ॑य॒ इत्या॑हा॒हेति॒ पृश्ञ॑यः॒ पृश्ञ॑य॒ इत्या॑ह । \newline
10. इत्या॑हा॒हे तीत्या॒हान्न॒ मन्न॑ मा॒हे तीत्या॒हान्न᳚म् । \newline
11. आ॒हान्न॒ मन्न॑ माहा॒ हान्नं॒ ॅवै वा अन्न॑ माहा॒ हान्नं॒ ॅवै । \newline
12. अन्नं॒ ॅवै वा अन्न॒ मन्नं॒ ॅवै पृश्ञि॒ पृश्ञि॒ वा अन्न॒ मन्नं॒ ॅवै पृश्ञि॑ । \newline
13. वै पृश्ञि॒ पृश्ञि॒ वै वै पृश्ञ्यन्न॒ मन्न॒म् पृश्ञि॒ वै वै पृश्ञ्यन्न᳚म् । \newline
14. पृश्ञ्यन्न॒ मन्न॒म् पृश्ञि॒ पृश्ञ्यन्न॑ मे॒वै वान्न॒म् पृश्ञि॒ पृश्ञ्यन्न॑ मे॒व । \newline
15. अन्न॑ मे॒वै वान्न॒ मन्न॑ मे॒वावा वै॒वान्न॒ मन्न॑ मे॒वाव॑ । \newline
16. ए॒वावा वै॒वै वाव॑ रुन्धे रु॒न्धे ऽवै॒वै वाव॑ रुन्धे । \newline
17. अव॑ रुन्धे रु॒न्धे ऽवाव॑ रुन्धे॒ ऽर्को᳚ ऽर्को रु॒न्धे ऽवाव॑ रुन्धे॒ ऽर्कः । \newline
18. रु॒न्धे॒ ऽर्को᳚ ऽर्को रु॑न्धे रुन्धे॒ ऽर्को वै वा अ॒र्को रु॑न्धे रुन्धे॒ ऽर्को वै । \newline
19. अ॒र्को वै वा अ॒र्को᳚ ऽर्को वा अ॒ग्नि र॒ग्निर् वा अ॒र्को᳚ ऽर्को वा अ॒ग्निः । \newline
20. वा अ॒ग्नि र॒ग्निर् वै वा अ॒ग्नि र॒र्को᳚(1॒) ऽर्को᳚ ऽग्निर् वै वा अ॒ग्नि र॒र्कः । \newline
21. अ॒ग्नि र॒र्को᳚(1॒) ऽर्को᳚ ऽग्नि र॒ग्नि र॒र्को ऽन्न॒ मन्न॑ म॒र्को᳚ ऽग्नि र॒ग्नि र॒र्को ऽन्न᳚म् । \newline
22. अ॒र्को ऽन्न॒ मन्न॑ म॒र्को᳚ ऽर्को ऽन्न᳚म् । \newline
23. अन्न॒ मन्न᳚म् । \newline
24. अन्न॑ मे॒वै वान्न॒ मन्न॑ मे॒वा वावै॒ वान्न॒ मन्न॑ मे॒वाव॑ । \newline
25. ए॒वावा वै॒वै वाव॑ रुन्धे रु॒न्धे ऽवै॒वै वाव॑ रुन्धे । \newline
26. अव॑ रुन्धे रु॒न्धे ऽवाव॑ रुन्धे॒ जन्म॒न् जन्म॑न् रु॒न्धे ऽवाव॑ रुन्धे॒ जन्मन्न्॑ । \newline
27. रु॒न्धे॒ जन्म॒न् जन्म॑न् रुन्धे रुन्धे॒ जन्म॑न् दे॒वाना᳚म् दे॒वाना॒म् जन्म॑न् रुन्धे रुन्धे॒ जन्म॑न् दे॒वाना᳚म् । \newline
28. जन्म॑न् दे॒वाना᳚म् दे॒वाना॒म् जन्म॒न् जन्म॑न् दे॒वानां॒ ॅविशो॒ विशो॑ दे॒वाना॒म् जन्म॒न् जन्म॑न् दे॒वानां॒ ॅविशः॑ । \newline
29. दे॒वानां॒ ॅविशो॒ विशो॑ दे॒वाना᳚म् दे॒वानां॒ ॅविश॑ स्त्रि॒षुत् त्रि॒षु विशो॑ दे॒वाना᳚म् दे॒वानां॒ ॅविश॑ स्त्रि॒षु । \newline
30. विश॑ स्त्रि॒षुत् त्रि॒षु विशो॒ विश॑ स्त्रि॒ष्वा त्रि॒षु विशो॒ विश॑ स्त्रि॒ष्वा । \newline
31. त्रि॒ष्वा त्रि॒षुत् त्रि॒ष्वा रो॑च॒ने रो॑च॒न आ त्रि॒षुत् त्रि॒ष्वा रो॑च॒ने । \newline
32. आ रो॑च॒ने रो॑च॒न आ रो॑च॒ने दि॒वो दि॒वो रो॑च॒न आ रो॑च॒ने दि॒वः । \newline
33. रो॒च॒ने दि॒वो दि॒वो रो॑च॒ने रो॑च॒ने दि॒व इतीति॑ दि॒वो रो॑च॒ने रो॑च॒ने दि॒व इति॑ । \newline
34. दि॒व इतीति॑ दि॒वो दि॒व इत्या॑हा॒हेति॑ दि॒वो दि॒व इत्या॑ह । \newline
35. इत्या॑हा॒हे तीत्या॑हे॒ मानि॒मा ना॒हे तीत्या॑हे॒मान् । \newline
36. आ॒हे॒ मानि॒मा ना॑हा हे॒माने॒वैवे माना॑हाहे॒ माने॒व । \newline
37. इ॒मा ने॒वैवे मानि॒मा ने॒वास्मा॑ अस्मा ए॒वे मानि॒मा ने॒वास्मै᳚ । \newline
38. ए॒वास्मा॑ अस्मा ए॒वै वास्मै॑ लो॒कान् ॅलो॒का न॑स्मा ए॒वै वास्मै॑ लो॒कान् । \newline
39. अ॒स्मै॒ लो॒कान् ॅलो॒का न॑स्मा अस्मै लो॒कान् ज्योति॑ष्मतो॒ ज्योति॑ष्मतो लो॒का न॑स्मा अस्मै लो॒कान् ज्योति॑ष्मतः । \newline
40. लो॒कान् ज्योति॑ष्मतो॒ ज्योति॑ष्मतो लो॒कान् ॅलो॒कान् ज्योति॑ष्मतः करोति करोति॒ ज्योति॑ष्मतो लो॒कान् ॅलो॒कान् ज्योति॑ष्मतः करोति । \newline
41. ज्योति॑ष्मतः करोति करोति॒ ज्योति॑ष्मतो॒ ज्योति॑ष्मतः करोति॒ यो यः क॑रोति॒ ज्योति॑ष्मतो॒ ज्योति॑ष्मतः करोति॒ यः । \newline
42. क॒रो॒ति॒ यो यः क॑रोति करोति॒ यो वै वै यः क॑रोति करोति॒ यो वै । \newline
43. यो वै वै यो यो वा इष्ट॑काना॒ मिष्ट॑कानां॒ ॅवै यो यो वा इष्ट॑कानाम् । \newline
44. वा इष्ट॑काना॒ मिष्ट॑कानां॒ ॅवै वा इष्ट॑कानाम् प्रति॒ष्ठाम् प्र॑ति॒ष्ठा मिष्ट॑कानां॒ ॅवै वा इष्ट॑कानाम् प्रति॒ष्ठाम् । \newline
45. इष्ट॑कानाम् प्रति॒ष्ठाम् प्र॑ति॒ष्ठा मिष्ट॑काना॒ मिष्ट॑कानाम् प्रति॒ष्ठां ॅवेद॒ वेद॑ प्रति॒ष्ठा मिष्ट॑काना॒ मिष्ट॑कानाम् प्रति॒ष्ठां ॅवेद॑ । \newline
46. प्र॒ति॒ष्ठां ॅवेद॒ वेद॑ प्रति॒ष्ठाम् प्र॑ति॒ष्ठां ॅवेद॒ प्रति॒ प्रति॒ वेद॑ प्रति॒ष्ठाम् प्र॑ति॒ष्ठां ॅवेद॒ प्रति॑ । \newline
47. प्र॒ति॒ष्ठामिति॑ प्रति - स्थाम् । \newline
48. वेद॒ प्रति॒ प्रति॒ वेद॒ वेद॒ प्रत्ये॒वैव प्रति॒ वेद॒ वेद॒ प्रत्ये॒व । \newline
49. प्रत्ये॒वैव प्रति॒ प्रत्ये॒व ति॑ष्ठति तिष्ठ त्ये॒व प्रति॒ प्रत्ये॒व ति॑ष्ठति । \newline
50. ए॒व ति॑ष्ठति तिष्ठ त्ये॒वैव ति॑ष्ठति॒ तया॒ तया॑ तिष्ठ त्ये॒वैव ति॑ष्ठति॒ तया᳚ । \newline
51. ति॒ष्ठ॒ति॒ तया॒ तया॑ तिष्ठति तिष्ठति॒ तया॑ दे॒वत॑या दे॒वत॑या॒ तया॑ तिष्ठति तिष्ठति॒ तया॑ दे॒वत॑या । \newline
52. तया॑ दे॒वत॑या दे॒वत॑या॒ तया॒ तया॑ दे॒वत॑या ऽङ्गिर॒स्व द॑ङ्गिर॒स्वद् दे॒वत॑या॒ तया॒ तया॑ दे॒वत॑या ऽङ्गिर॒स्वत् । \newline
53. दे॒वत॑या ऽङ्गिर॒स्व द॑ङ्गिर॒स्वद् दे॒वत॑या दे॒वत॑या ऽङ्गिर॒स्वद् ध्रु॒वा ध्रु॒वा ऽङ्गि॑र॒स्वद् दे॒वत॑या दे॒वत॑या ऽङ्गिर॒स्वद् ध्रु॒वा । \newline
54. अ॒ङ्गि॒र॒स्वद् ध्रु॒वा ध्रु॒वा ऽङ्गि॑र॒स्व द॑ङ्गिर॒स्वद् ध्रु॒वा सी॑द सीद ध्रु॒वा ऽङ्गि॑र॒स्व द॑ङ्गिर॒स्वद् ध्रु॒वा सी॑द । \newline
55. ध्रु॒वा सी॑द सीद ध्रु॒वा ध्रु॒वा सी॒दे तीति॑ सीद ध्रु॒वा ध्रु॒वा सी॒देति॑ । \newline
56. सी॒दे तीति॑ सीद सी॒दे त्या॑हा॒हेति॑ सीद सी॒दे त्या॑ह । \newline
57. इत्या॑हा॒हे तीत्या॑ है॒षैषा ऽऽहे तीत्या॑है॒षा । \newline
58. आ॒है॒ षैषा ऽऽहा॑है॒षा वै वा ए॒षा ऽऽहा॑है॒षा वै । \newline
59. ए॒षा वै वा ए॒षैषा वा इष्ट॑काना॒ मिष्ट॑कानां॒ ॅवा ए॒षैषा वा इष्ट॑कानाम् । \newline
60. वा इष्ट॑काना॒ मिष्ट॑कानां॒ ॅवै वा इष्ट॑कानाम् प्रति॒ष्ठा प्र॑ति॒ष्ठेष्ट॑कानां॒ ॅवै वा इष्ट॑कानाम् प्रति॒ष्ठा । \newline
61. इष्ट॑कानाम् प्रति॒ष्ठा प्र॑ति॒ष्ठेष्ट॑काना॒ मिष्ट॑कानाम् प्रति॒ष्ठा यो यः प्र॑ति॒ष्ठेष्ट॑काना॒ मिष्ट॑कानाम् प्रति॒ष्ठा यः । \newline
62. प्र॒ति॒ष्ठा यो यः प्र॑ति॒ष्ठा प्र॑ति॒ष्ठा य ए॒व मे॒वं ॅयः प्र॑ति॒ष्ठा प्र॑ति॒ष्ठा य ए॒वम् । \newline
63. प्र॒ति॒ष्ठेति॑ प्रति - स्था । \newline
64. य ए॒व मे॒वं ॅयो य ए॒वं ॅवेद॒ वेदै॒वं ॅयो य ए॒वं ॅवेद॑ । \newline
65. ए॒वं ॅवेद॒ वेदै॒व मे॒वं ॅवेद॒ प्रति॒ प्रति॒ वेदै॒व मे॒वं ॅवेद॒ प्रति॑ । \newline
66. वेद॒ प्रति॒ प्रति॒ वेद॒ वेद॒ प्रत्ये॒वैव प्रति॒ वेद॒ वेद॒ प्रत्ये॒व । \newline
67. प्रत्ये॒वैव प्रति॒ प्रत्ये॒व ति॑ष्ठति तिष्ठ त्ये॒व प्रति॒ प्रत्ये॒व ति॑ष्ठति । \newline
68. ए॒व ति॑ष्ठति तिष्ठ त्ये॒वैव ति॑ष्ठति । \newline
69. ति॒ष्ठ॒तीति॑ तिष्ठति । \newline
\pagebreak
\markright{ TS 5.5.7.1  \hfill https://www.vedavms.in \hfill}

\section{ TS 5.5.7.1 }

\textbf{TS 5.5.7.1 } \newline
\textbf{Samhita Paata} \newline

सु॒व॒र्गाय॒ वा ए॒ष लो॒काय॑ चीयते॒ यद॒ग्निर्वज्र॑ एकाद॒शिनी॒ यद॒ग्नावे॑काद॒शिनीं᳚ मिनु॒याद्-वज्रे॑णैनꣳ सुव॒र्गाल्लो॒काद॒न्तर्द॑द्ध्या॒द्यन्न मि॑नु॒याथ् स्वरु॑भिः प॒शून् व्य॑र्द्धयेदेकयू॒पं मि॑नोति॒ नैनं॒ ॅवज्रे॑ण सुव॒र्गाल्लो॒काद॑न्त॒र्दधा॑ति॒ न स्वरु॑भिः प॒शून् व्य॑र्द्धयति॒ वि वा ए॒ष इ॑न्द्रि॒येण॑ वी॒र्ये॑णर्द्ध्यते॒ यो᳚ऽग्निं चि॒न्व-न्न॑धि॒क्राम॑त्यैन्द्रि॒य - [  ] \newline

\textbf{Pada Paata} \newline

सु॒व॒र्गायेति॑ सुवः - गाय॑ । वै । ए॒षः । लो॒काय॑ । ची॒य॒ते॒ । यत् । अ॒ग्निः । वज्रः॑ । ए॒का॒द॒शिनी᳚ । यत् । अ॒ग्नौ । ए॒का॒द॒शिनी᳚म् । मि॒नु॒यात् । वज्रे॑ण । ए॒न॒म् । सु॒व॒र्गादिति॑ सुवः - गात् । लो॒कात् । अ॒न्तः । द॒द्ध्या॒त् । यत् । न । मि॒नु॒यात् । स्वरु॑भि॒रिति॒ स्वरु॑-भिः॒ । प॒शून् । वीति॑ । अ॒द्‌र्ध॒ये॒त् । ए॒क॒यू॒पमित्ये॑क - यू॒पम् । मि॒नो॒ति॒ । न । ए॒न॒म् । वज्रे॑ण । सु॒व॒र्गादिति॑ सुवः - गात् । लो॒कात् । अ॒न्त॒र्दधा॒तीत्य॑न्तः - दधा॑ति । न । स्वरु॑भि॒रिति॒ स्वरु॑-भिः॒ । प॒शून् । वीति॑ । अ॒द्‌र्ध॒य॒ति॒ । वीति॑ । वै । ए॒षः । इ॒न्द्रि॒येण॑ । वी॒र्ये॑ण । ऋ॒द्ध्य॒ते॒ । यः । अ॒ग्निम् । चि॒न्वन्न् । अ॒धि॒क्राम॒तीत्य॑धि - क्राम॑ति । ऐ॒न्द्रि॒या ।  \newline


\textbf{Krama Paata} \newline

सु॒व॒र्गाय॒ वै । सु॒व॒र्गायेति॑ सुवः - गाय॑ । वा ए॒षः । ए॒ष लो॒काय॑ । लो॒काय॑ चीयते । ची॒य॒ते॒ यत् । यद॒ग्निः । अ॒ग्निर् वज्रः॑ । वज्र॑ एकाद॒शिनी᳚ । ए॒का॒द॒शिनी॒ यत् । यद॒ग्नौ । अ॒ग्नावे॑काद॒शिनी᳚म् । ए॒का॒द॒शिनी᳚म् मिनु॒यात् । मि॒नु॒याद् वज्रे॑ण । वज्रे॑णैनम् । ए॒नꣳ॒॒ सु॒व॒र्गात् । सु॒व॒र्गाल्लो॒कात् । सु॒व॒र्गादिति॑ सुवः - गात् । लो॒काद॒न्तः । अ॒न्तर् द॑द्ध्यात् । द॒द्ध्या॒द् यत् । यन् न । न मि॑नु॒यात् । मि॒नु॒याथ् स्वरु॑भिः । स्वरु॑भिः प॒शून् । स्वरु॑भि॒रिति॒ स्वरु॑ - भिः॒ । प॒शून्. वि । व्य॑र्द्धयेत् । अ॒र्द्ध॒ये॒दे॒क॒यू॒पम् । ए॒क॒यू॒पम् मि॑नोति । ए॒क॒यू॒पमित्ये॑क - यू॒पम् । मि॒नो॒ति॒ न । नैन᳚म् । ए॒न॒म् ॅवज्रे॑ण । वज्रे॑ण सुव॒र्गात् । सु॒व॒र्गाल्लो॒कात् । सु॒व॒र्गादिति॑ सुवः - गात् । लो॒काद॑न्त॒र्दधा॑ति । अ॒न्त॒र्दधा॑ति॒ न । अ॒न्त॒र्दधा॒तीत्य॑न्तः - दधा॑ति । न स्वरु॑भिः । स्वरु॑भिः प॒शून् । स्वरु॑भि॒रिति॒ स्वरु॑ - भिः॒ । प॒शून्. वि । व्य॑र्द्धयति । अ॒र्द्ध॒य॒ति॒ वि । वि वै । वा ए॒षः । ए॒ष इ॑न्द्रि॒येण॑ । इ॒न्द्रि॒येण॑ वी॒र्ये॑ण । वी॒र्ये॑णर्द्ध्यते । ऋ॒द्ध्य॒ते॒ यः । यो᳚ऽग्निम् । अ॒ग्निम् चि॒न्वन्न् । चि॒न्वन्न॑धि॒क्राम॑ति । अ॒धि॒क्राम॑त्यैन्द्रि॒या । अ॒धि॒क्राम॒तीत्य॑धि - क्राम॑ति । ऐ॒न्द्रि॒यर्चा \newline

\textbf{Jatai Paata} \newline

1. सु॒व॒र्गाय॒ वै वै सु॑व॒र्गाय॑ सुव॒र्गाय॒ वै । \newline
2. सु॒व॒र्गायेति॑ सुवः - गाय॑ । \newline
3. वा ए॒ष ए॒ष वै वा ए॒षः । \newline
4. ए॒ष लो॒काय॑ लो॒कायै॒ष ए॒ष लो॒काय॑ । \newline
5. लो॒काय॑ चीयते चीयते लो॒काय॑ लो॒काय॑ चीयते । \newline
6. ची॒य॒ते॒ यद् यच् ची॑यते चीयते॒ यत् । \newline
7. यद॒ग्नि र॒ग्निर् यद् यद॒ग्निः । \newline
8. अ॒ग्निर् वज्रो॒ वज्रो॒ ऽग्नि र॒ग्निर् वज्रः॑ । \newline
9. वज्र॑ एकाद॒शि न्ये॑काद॒शिनी॒ वज्रो॒ वज्र॑ एकाद॒शिनी᳚ । \newline
10. ए॒का॒द॒शिनी॒ यद् यदे॑काद॒शि न्ये॑काद॒शिनी॒ यत् । \newline
11. यद॒ग्ना व॒ग्नौ यद् यद॒ग्नौ । \newline
12. अ॒ग्ना वे॑काद॒शिनी॑ मेकाद॒शिनी॑ म॒ग्ना व॒ग्ना वे॑काद॒शिनी᳚म् । \newline
13. ए॒का॒द॒शिनी᳚म् मिनु॒यान् मि॑नु॒या दे॑काद॒शिनी॑ मेकाद॒शिनी᳚म् मिनु॒यात् । \newline
14. मि॒नु॒याद् वज्रे॑ण॒ वज्रे॑ण मिनु॒यान् मि॑नु॒याद् वज्रे॑ण । \newline
15. वज्रे॑णैन मेनं॒ ॅवज्रे॑ण॒ वज्रे॑णैनम् । \newline
16. ए॒नꣳ॒॒ सु॒व॒र्गाथ् सु॑व॒र्गादे॑न मेनꣳ सुव॒र्गात् । \newline
17. सु॒व॒र्गाल् लो॒काल् लो॒काथ् सु॑व॒र्गाथ् सु॑व॒र्गाल् लो॒कात् । \newline
18. सु॒व॒र्गादिति॑ सुवः - गात् । \newline
19. लो॒का द॒न्त र॒न्तर् लो॒काल् लो॒का द॒न्तः । \newline
20. अ॒न्तर् द॑द्ध्याद् दद्ध्या द॒न्त र॒न्तर् द॑द्ध्यात् । \newline
21. द॒द्ध्या॒द् यद् यद् द॑द्ध्याद् दद्ध्या॒द् यत् । \newline
22. यन् न न यद् यन् न । \newline
23. न मि॑नु॒यान् मि॑नु॒यान् न न मि॑नु॒यात् । \newline
24. मि॒नु॒याथ् स्वरु॑भिः॒ स्वरु॑भिर् मिनु॒यान् मि॑नु॒याथ् स्वरु॑भिः । \newline
25. स्वरु॑भिः प॒शून् प॒शून् थ्स्वरु॑भिः॒ स्वरु॑भिः प॒शून् । \newline
26. स्वरु॑भि॒रिति॒ स्वरु॑ - भिः॒ । \newline
27. प॒शून्. वि वि प॒शून् प॒शून्. वि । \newline
28. व्य॑र्द्धये दर्द्धये॒द् वि व्य॑र्द्धयेत् । \newline
29. अ॒र्द्ध॒ये॒ दे॒क॒यू॒प मे॑कयू॒प म॑र्द्धये दर्द्धये देकयू॒पम् । \newline
30. ए॒क॒यू॒पम् मि॑नोति मिनो त्येकयू॒प मे॑कयू॒पम् मि॑नोति । \newline
31. ए॒क॒यू॒पमित्ये॑क - यू॒पम् । \newline
32. मि॒नो॒ति॒ न न मि॑नोति मिनोति॒ न । \newline
33. नैन॑ मेन॒म् न नैन᳚म् । \newline
34. ए॒नं॒ ॅवज्रे॑ण॒ वज्रे॑णैन मेनं॒ ॅवज्रे॑ण । \newline
35. वज्रे॑ण सुव॒र्गाथ् सु॑व॒र्गाद् वज्रे॑ण॒ वज्रे॑ण सुव॒र्गात् । \newline
36. सु॒व॒र्गाल् लो॒काल् लो॒काथ् सु॑व॒र्गाथ् सु॑व॒र्गाल् लो॒कात् । \newline
37. सु॒व॒र्गादिति॑ सुवः - गात् । \newline
38. लो॒का द॑न्त॒र्दधा᳚ त्यन्त॒र्दधा॑ति लो॒काल् लो॒का द॑न्त॒र्दधा॑ति । \newline
39. अ॒न्त॒र्दधा॑ति॒ न नान्त॒र्दधा᳚ त्यन्त॒र्दधा॑ति॒ न । \newline
40. अ॒न्त॒र्दधा॒तीत्य॑न्तः - दधा॑ति । \newline
41. न स्वरु॑भिः॒ स्वरु॑भि॒र् न न स्वरु॑भिः । \newline
42. स्वरु॑भिः प॒शून् प॒शून् थ्स्वरु॑भिः॒ स्वरु॑भिः प॒शून् । \newline
43. स्वरु॑भि॒रिति॒ स्वरु॑ - भिः॒ । \newline
44. प॒शून्. वि वि प॒शून् प॒शून्. वि । \newline
45. व्य॑र्द्धय त्यर्द्धयति॒ वि व्य॑र्द्धयति । \newline
46. अ॒र्द्ध॒य॒ति॒ वि व्य॑र्द्धय त्यर्द्धयति॒ वि । \newline
47. वि वै वै वि वि वै । \newline
48. वा ए॒ष ए॒ष वै वा ए॒षः । \newline
49. ए॒ष इ॑न्द्रि॒ये णे᳚न्द्रि॒ये णै॒ष ए॒ष इ॑न्द्रि॒येण॑ । \newline
50. इ॒न्द्रि॒येण॑ वी॒र्ये॑ण वी॒र्ये॑ णेन्द्रि॒ये णे᳚न्द्रि॒येण॑ वी॒र्ये॑ण । \newline
51. वी॒र्ये॑ण र्‌द्ध्यत ऋद्ध्यते वी॒र्ये॑ण वी॒र्ये॑ण र्‌द्ध्यते । \newline
52. ऋ॒द्ध्य॒ते॒ यो य ऋ॑द्ध्यत ऋद्ध्यते॒ यः । \newline
53. यो᳚ ऽग्नि म॒ग्निं ॅयो यो᳚ ऽग्निम् । \newline
54. अ॒ग्निम् चि॒न्वꣳ श्चि॒न्वन् न॒ग्नि म॒ग्निम् चि॒न्वन्न् । \newline
55. चि॒न्वन् न॑धि॒क्राम॑ त्यधि॒क्राम॑ति चि॒न्वꣳ श्चि॒न्वन् न॑धि॒क्राम॑ति । \newline
56. अ॒धि॒क्राम॑ त्यैन्द्रि॒ यैन्द्रि॒या ऽधि॒क्राम॑ त्यधि॒क्राम॑ त्यैन्द्रि॒या । \newline
57. अ॒धि॒क्राम॒तीत्य॑धि - क्राम॑ति । \newline
58. ऐ॒न्द्रि॒य र्‌च र्‌चैन्द्रि॒ यैन्द्रि॒य र्‌चा । \newline

\textbf{Ghana Paata } \newline

1. सु॒व॒र्गाय॒ वै वै सु॑व॒र्गाय॑ सुव॒र्गाय॒ वा ए॒ष ए॒ष वै सु॑व॒र्गाय॑ सुव॒र्गाय॒ वा ए॒षः । \newline
2. सु॒व॒र्गायेति॑ सुवः - गाय॑ । \newline
3. वा ए॒ष ए॒ष वै वा ए॒ष लो॒काय॑ लो॒कायै॒ष वै वा ए॒ष लो॒काय॑ । \newline
4. ए॒ष लो॒काय॑ लो॒कायै॒ष ए॒ष लो॒काय॑ चीयते चीयते लो॒कायै॒ष ए॒ष लो॒काय॑ चीयते । \newline
5. लो॒काय॑ चीयते चीयते लो॒काय॑ लो॒काय॑ चीयते॒ यद् यच् ची॑यते लो॒काय॑ लो॒काय॑ चीयते॒ यत् । \newline
6. ची॒य॒ते॒ यद् यच् ची॑यते चीयते॒ यद॒ग्नि र॒ग्निर् यच् ची॑यते चीयते॒ यद॒ग्निः । \newline
7. यद॒ग्नि र॒ग्निर् यद् यद॒ग्निर् वज्रो॒ वज्रो॒ ऽग्निर् यद् यद॒ग्निर् वज्रः॑ । \newline
8. अ॒ग्निर् वज्रो॒ वज्रो॒ ऽग्नि र॒ग्निर् वज्र॑ एकाद॒शि न्ये॑काद॒शिनी॒ वज्रो॒ ऽग्नि र॒ग्निर् वज्र॑ एकाद॒शिनी᳚ । \newline
9. वज्र॑ एकाद॒शि न्ये॑काद॒शिनी॒ वज्रो॒ वज्र॑ एकाद॒शिनी॒ यद् यदे॑काद॒शिनी॒ वज्रो॒ वज्र॑ एकाद॒शिनी॒ यत् । \newline
10. ए॒का॒द॒शिनी॒ यद् यदे॑काद॒शि न्ये॑काद॒शिनी॒ यद॒ग्ना व॒ग्नौ यदे॑काद॒शि न्ये॑काद॒शिनी॒ यद॒ग्नौ । \newline
11. यद॒ग्ना व॒ग्नौ यद् यद॒ग्ना वे॑काद॒शिनी॑ मेकाद॒शिनी॑ म॒ग्नौ यद् यद॒ग्ना वे॑काद॒शिनी᳚म् । \newline
12. अ॒ग्ना वे॑काद॒शिनी॑ मेकाद॒शिनी॑ म॒ग्ना व॒ग्ना वे॑काद॒शिनी᳚म् मिनु॒यान् मि॑नु॒या दे॑काद॒शिनी॑ म॒ग्ना व॒ग्ना वे॑काद॒शिनी᳚म् मिनु॒यात् । \newline
13. ए॒का॒द॒शिनी᳚म् मिनु॒यान् मि॑नु॒या दे॑काद॒शिनी॑ मेकाद॒शिनी᳚म् मिनु॒याद् वज्रे॑ण॒ वज्रे॑ण मिनु॒या दे॑काद॒शिनी॑ मेकाद॒शिनी᳚म् मिनु॒याद् वज्रे॑ण । \newline
14. मि॒नु॒याद् वज्रे॑ण॒ वज्रे॑ण मिनु॒यान् मि॑नु॒याद् वज्रे॑ णैन मेनं॒ ॅवज्रे॑ण मिनु॒यान् मि॑नु॒याद् वज्रे॑णैनम् । \newline
15. वज्रे॑णैन मेनं॒ ॅवज्रे॑ण॒ वज्रे॑णैनꣳ सुव॒र्गाथ् सु॑व॒र्गा दे॑नं॒ ॅवज्रे॑ण॒ वज्रे॑णैनꣳ सुव॒र्गात् । \newline
16. ए॒नꣳ॒॒ सु॒व॒र्गाथ् सु॑व॒र्गा दे॑न मेनꣳ सुव॒र्गाल् लो॒काल् लो॒काथ् सु॑व॒र्गा दे॑न मेनꣳ सुव॒र्गाल् लो॒कात् । \newline
17. सु॒व॒र्गाल् लो॒काल् लो॒काथ् सु॑व॒र्गाथ् सु॑व॒र्गाल् लो॒का द॒न्त र॒न्तर् लो॒काथ् सु॑व॒र्गाथ् सु॑व॒र्गाल् लो॒का द॒न्तः । \newline
18. सु॒व॒र्गादिति॑ सुवः - गात् । \newline
19. लो॒का द॒न्त र॒न्तर् लो॒काल् लो॒का द॒न्तर् द॑द्ध्याद् दद्ध्या द॒न्तर् लो॒काल् लो॒का द॒न्तर् द॑द्ध्यात् । \newline
20. अ॒न्तर् द॑द्ध्याद् दद्ध्या द॒न्त र॒न्तर् द॑द्ध्या॒द् यद् यद् द॑द्ध्या द॒न्त र॒न्तर् द॑द्ध्या॒द् यत् । \newline
21. द॒द्ध्या॒द् यद् यद् द॑द्ध्याद् दद्ध्या॒द् यन् न न यद् द॑द्ध्याद् दद्ध्या॒द् यन् न । \newline
22. यन् न न यद् यन् न मि॑नु॒यान् मि॑नु॒यान् न यद् यन् न मि॑नु॒यात् । \newline
23. न मि॑नु॒यान् मि॑नु॒यान् न न मि॑नु॒याथ् स्वरु॑भिः॒ स्वरु॑भिर् मिनु॒यान् न न मि॑नु॒याथ् स्वरु॑भिः । \newline
24. मि॒नु॒याथ् स्वरु॑भिः॒ स्वरु॑भिर् मिनु॒यान् मि॑नु॒याथ् स्वरु॑भिः प॒शून् प॒शून् थ्स्वरु॑भिर् मिनु॒यान् मि॑नु॒याथ् स्वरु॑भिः प॒शून् । \newline
25. स्वरु॑भिः प॒शून् प॒शून् थ्स्वरु॑भिः॒ स्वरु॑भिः प॒शून्. वि वि प॒शून् थ्स्वरु॑भिः॒ स्वरु॑भिः प॒शून्. वि । \newline
26. स्वरु॑भि॒रिति॒ स्वरु॑ - भिः॒ । \newline
27. प॒शून्. वि वि प॒शून् प॒शून् व्य॑र्द्धये दर्द्धये॒द् वि प॒शून् प॒शून् व्य॑र्द्धयेत् । \newline
28. व्य॑र्द्धये दर्द्धये॒द् वि व्य॑र्द्धये देकयू॒प मे॑कयू॒प म॑र्द्धये॒द् वि व्य॑र्द्धये देकयू॒पम् । \newline
29. अ॒र्द्ध॒ये॒ दे॒क॒यू॒प मे॑कयू॒प म॑र्द्धये दर्द्धये देकयू॒पम् मि॑नोति मिनो त्येकयू॒प म॑र्द्धये दर्द्धये देकयू॒पम् मि॑नोति । \newline
30. ए॒क॒यू॒पम् मि॑नोति मिनो त्येकयू॒प मे॑कयू॒पम् मि॑नोति॒ न न मि॑नो त्येकयू॒प मे॑कयू॒पम् मि॑नोति॒ न । \newline
31. ए॒क॒यू॒पमित्ये॑क - यू॒पम् । \newline
32. मि॒नो॒ति॒ न न मि॑नोति मिनोति॒ नैन॑ मेन॒म् न मि॑नोति मिनोति॒ नैन᳚म् । \newline
33. नैन॑ मेन॒म् न नैनं॒ ॅवज्रे॑ण॒ वज्रे॑णैन॒म् न नैनं॒ ॅवज्रे॑ण । \newline
34. ए॒नं॒ ॅवज्रे॑ण॒ वज्रे॑णैन मेनं॒ ॅवज्रे॑ण सुव॒र्गाथ् सु॑व॒र्गाद् वज्रे॑णैन मेनं॒ ॅवज्रे॑ण सुव॒र्गात् । \newline
35. वज्रे॑ण सुव॒र्गाथ् सु॑व॒र्गाद् वज्रे॑ण॒ वज्रे॑ण सुव॒र्गाल् लो॒काल् लो॒काथ् सु॑व॒र्गाद् वज्रे॑ण॒ वज्रे॑ण सुव॒र्गाल् लो॒कात् । \newline
36. सु॒व॒र्गाल् लो॒काल् लो॒काथ् सु॑व॒र्गाथ् सु॑व॒र्गाल् लो॒का द॑न्त॒र्दधा᳚ त्यन्त॒र्दधा॑ति लो॒काथ् सु॑व॒र्गाथ् सु॑व॒र्गाल् लो॒का द॑न्त॒र्दधा॑ति । \newline
37. सु॒व॒र्गादिति॑ सुवः - गात् । \newline
38. लो॒का द॑न्त॒र्दधा᳚ त्यन्त॒र्दधा॑ति लो॒काल् लो॒का द॑न्त॒र्दधा॑ति॒ न नान्त॒र्दधा॑ति लो॒काल् लो॒का द॑न्त॒र्दधा॑ति॒ न । \newline
39. अ॒न्त॒र्दधा॑ति॒ न नान्त॒र्दधा᳚ त्यन्त॒र्दधा॑ति॒ न स्वरु॑भिः॒ स्वरु॑भि॒र् नान्त॒र्दधा᳚ त्यन्त॒र्दधा॑ति॒ न स्वरु॑भिः । \newline
40. अ॒न्त॒र्दधा॒तीत्य॑न्तः - दधा॑ति । \newline
41. न स्वरु॑भिः॒ स्वरु॑भि॒र् न न स्वरु॑भिः प॒शून् प॒शून् थ्स्वरु॑भि॒र् न न स्वरु॑भिः प॒शून् । \newline
42. स्वरु॑भिः प॒शून् प॒शून् थ्स्वरु॑भिः॒ स्वरु॑भिः प॒शून्. वि वि प॒शून् थ्स्वरु॑भिः॒ स्वरु॑भिः प॒शून्. वि । \newline
43. स्वरु॑भि॒रिति॒ स्वरु॑ - भिः॒ । \newline
44. प॒शून्. वि वि प॒शून् प॒शून् व्य॑र्द्धय त्यर्द्धयति॒ वि प॒शून् प॒शून् व्य॑र्द्धयति । \newline
45. व्य॑र्द्धय त्यर्द्धयति॒ वि व्य॑र्द्धयति॒ वि व्य॑र्द्धयति॒ वि व्य॑र्द्धयति॒ वि । \newline
46. अ॒र्द्ध॒य॒ति॒ वि व्य॑र्द्धय त्यर्द्धयति॒ वि वै वै व्य॑र्द्धय त्यर्द्धयति॒ वि वै । \newline
47. वि वै वै वि वि वा ए॒ष ए॒ष वै वि वि वा ए॒षः । \newline
48. वा ए॒ष ए॒ष वै वा ए॒ष इ॑न्द्रि॒ येणे᳚न्द्रि॒येणै॒ष वै वा ए॒ष इ॑न्द्रि॒येण॑ । \newline
49. ए॒ष इ॑न्द्रि॒ये णे᳚न्द्रि॒ये णै॒ष ए॒ष इ॑न्द्रि॒येण॑ वी॒र्ये॑ण वी॒र्ये॑ णेन्द्रि॒ये णै॒ष ए॒ष इ॑न्द्रि॒येण॑ वी॒र्ये॑ण । \newline
50. इ॒न्द्रि॒येण॑ वी॒र्ये॑ण वी॒र्ये॑ णेन्द्रि॒ये णे᳚न्द्रि॒येण॑ वी॒र्ये॑ण र्‌द्ध्यत ऋद्ध्यते वी॒र्ये॑ णेन्द्रि॒ये 
णे᳚न्द्रि॒येण॑ वी॒र्ये॑ण र्‌द्ध्यते । \newline
51. वी॒र्ये॑ण र्‌द्ध्यत ऋद्ध्यते वी॒र्ये॑ण वी॒र्ये॑ण र्‌द्ध्यते॒ यो य ऋ॑द्ध्यते वी॒र्ये॑ण वी॒र्ये॑ण र्‌द्ध्यते॒ यः । \newline
52. ऋ॒द्ध्य॒ते॒ यो य ऋ॑द्ध्यत ऋद्ध्यते॒ यो᳚ ऽग्नि म॒ग्निं ॅय ऋ॑द्ध्यत ऋद्ध्यते॒ यो᳚ ऽग्निम् । \newline
53. यो᳚ ऽग्नि म॒ग्निं ॅयो यो᳚ ऽग्निम् चि॒न्वꣳ श्चि॒न्वन् न॒ग्निं ॅयो यो᳚ ऽग्निम् चि॒न्वन्न् । \newline
54. अ॒ग्निम् चि॒न्वꣳ श्चि॒न्वन् न॒ग्नि म॒ग्निम् चि॒न्वन् न॑धि॒क्राम॑ त्यधि॒क्राम॑ति चि॒न्वन् न॒ग्नि म॒ग्निम् चि॒न्वन् न॑धि॒क्राम॑ति । \newline
55. चि॒न्वन् न॑धि॒क्राम॑ त्यधि॒क्राम॑ति चि॒न्वꣳ श्चि॒न्वन्  न॑धि॒क्राम॑ त्यैन्द्रि॒ यैन्द्रि॒या ऽधि॒क्राम॑ति चि॒न्वꣳश्चि॒न्वन् न॑धि॒क्राम॑ त्यैन्द्रि॒या । \newline
56. अ॒धि॒क्राम॑ त्यैन्द्रि॒ यैन्द्रि॒या ऽधि॒क्राम॑ त्यधि॒क्राम॑ त्यैन्द्रि॒य र्‌च र्‌चैन्द्रि॒या ऽधि॒क्राम॑ त्यधि॒क्राम॑ त्यैन्द्रि॒य र्‌चा । \newline
57. अ॒धि॒क्राम॒तीत्य॑धि - क्राम॑ति । \newline
58. ऐ॒न्द्रि॒य र्‌च र्‌चैन्द्रि॒ यैन्द्रि॒य र्‌चा ऽऽक्रम॑ण मा॒क्रम॑ण मृ॒चैन्द्रि॒ यैन्द्रि॒य र्‌चा ऽऽक्रम॑णम् । \newline
\pagebreak
\markright{ TS 5.5.7.2  \hfill https://www.vedavms.in \hfill}

\section{ TS 5.5.7.2 }

\textbf{TS 5.5.7.2 } \newline
\textbf{Samhita Paata} \newline

-र्चा ऽऽक्रम॑णं॒ प्रतीष्ट॑का॒मुप॑ दद्ध्या॒न्नेन्द्रि॒येण॑ वी॒र्ये॑ण॒ व्यृ॑द्ध्यते रु॒द्रो वा ए॒ष यद॒ग्निस्तस्य॑ ति॒स्रः श॑र॒व्याः᳚ प्र॒तीची॑ ति॒रश्च्य॒नूची॒ ताभ्यो॒ वा ए॒ष आ वृ॑श्च्यते॒ यो᳚ऽग्निं चि॑नु॒ते᳚ ऽग्निं चि॒त्वा ति॑सृध॒न्वमया॑चितं ब्राह्म॒णाय॑ दद्या॒त् ताभ्य॑ ए॒व नम॑स्करो॒त्यथो॒ ताभ्य॑ ए॒वाऽऽत्मानं॒ निष्क्री॑णीते॒ यत्ते॑ रुद्र पु॒रो - [  ] \newline

\textbf{Pada Paata} \newline

ऋ॒चा । आ॒क्रम॑ण॒मित्या᳚ - क्रम॑णम् । प्रतीति॑ । इष्ट॑काम् । उपेति॑ । द॒द्ध्या॒त् । न । इ॒न्द्रि॒येण॑ । वी॒र्ये॑ण । वीति॑ । ऋ॒द्ध्य॒ते॒ । रु॒द्रः । वै । ए॒षः । यत् । अ॒ग्निः । तस्य॑ । ति॒स्रः । श॒र॒व्याः᳚ । प्र॒तीची᳚ । ति॒रश्ची᳚ । अ॒नूची᳚ । ताभ्यः॑ । वै । ए॒षः । एति॑ । वृ॒श्च्य॒ते॒ । यः । अ॒ग्निम् । चि॒नु॒ते । अ॒ग्निम् । चि॒त्वा । ति॒सृ॒ध॒न्वमिति॑ तिसृ - ध॒न्वम् । अया॑चितम् । ब्रा॒ह्म॒णाय॑ । द॒द्या॒त् । ताभ्यः॑ । ए॒व । नमः॑ । क॒रो॒ति॒ । अथो॒ इति॑ । ताभ्यः॑ । ए॒व । आ॒त्मान᳚म् । निरिति॑ । क्री॒णी॒ते॒ । यत् । ते॒ । रु॒द्र॒ । पु॒रः ।  \newline


\textbf{Krama Paata} \newline

ऋ॒चाऽऽक्रम॑णम् । आ॒क्रम॑ण॒म् प्रति॑ । आ॒क्रम॑ण॒मित्या᳚ - क्रम॑णम् । प्रतीष्ट॑काम् । इष्ट॑का॒मुप॑ । उप॑ दद्ध्यात् । द॒द्ध्या॒न् न । नेन्द्रि॒येण॑ । इ॒न्द्रि॒येण॑ वी॒र्ये॑ण । वी॒र्ये॑ण॒ वि । व्यृ॑द्ध्यते । ऋ॒द्ध्य॒ते॒ रु॒द्रः । रु॒द्रो वै । वा ए॒षः । ए॒ष यत् । यद॒ग्निः । अ॒ग्निस्तस्य॑ । तस्य॑ ति॒स्रः । ति॒स्रः श॑र॒व्याः᳚ । श॒र॒व्याः᳚ प्र॒तीची᳚ । प्र॒तीची॑ ति॒रश्ची᳚ । ति॒रश्च्य॒नूची᳚ । अ॒नूची॒ ताभ्यः॑ । ताभ्यो॒ वै । वा ए॒षः । ए॒ष आ । आ वृ॑श्च्यते । वृ॒श्च्य॒ते॒ यः । यो᳚ऽग्निम् । अ॒ग्निम् चि॑नु॒ते । चि॒नु॒ते᳚ऽग्निम् । अ॒ग्निम् चि॒त्वा । चि॒त्वा ति॑सृध॒न्वम् । ति॒सृ॒ध॒न्वमया॑चितम् । ति॒सृ॒ध॒न्वमिति॑ तिसृ - ध॒न्वम् । अया॑चितम् ब्राह्म॒णाय॑ । ब्रा॒ह्म॒णाय॑ दद्यात् । द॒द्या॒त् ताभ्यः॑ । ताभ्य॑ ए॒व । ए॒व नमः॑ । नम॑स्करोति । क॒रो॒त्यथो᳚ । अथो॒ ताभ्यः॑ । अथो॒ इत्यथो᳚ । ताभ्य॑ ए॒व । ए॒वात्मान᳚म् । आ॒त्मान॒म् निः । निष्क्री॑णीते । क्री॒णी॒ते॒ यत् । यत् ते᳚ । ते॒ रु॒द्र॒ । रु॒द्र॒ पु॒रः । पु॒रो धनुः॑ \newline

\textbf{Jatai Paata} \newline

1. ऋ॒चा ऽऽक्रम॑ण मा॒क्रम॑ण मृ॒च र्‌चा ऽऽक्रम॑णम् । \newline
2. आ॒क्रम॑ण॒म् प्रति॒ प्रत्या॒क्रम॑ण मा॒क्रम॑ण॒म् प्रति॑ । \newline
3. आ॒क्रम॑ण॒मित्या᳚ - क्रम॑णम् । \newline
4. प्रतीष्ट॑का॒ मिष्ट॑का॒म् प्रति॒ प्रतीष्ट॑काम् । \newline
5. इष्ट॑का॒ मुपोपेष्ट॑का॒ मिष्ट॑का॒ मुप॑ । \newline
6. उप॑ दद्ध्याद् दद्ध्या॒ दुपोप॑ दद्ध्यात् । \newline
7. द॒द्ध्या॒न् न न द॑द्ध्याद् दद्ध्या॒न् न । \newline
8. नेन्द्रि॒ये णे᳚न्द्रि॒येण॒ न नेन्द्रि॒येण॑ । \newline
9. इ॒न्द्रि॒येण॑ वी॒र्ये॑ण वी॒र्ये॑ णेन्द्रि॒ये णे᳚न्द्रि॒येण॑ वी॒र्ये॑ण । \newline
10. वी॒र्ये॑ण॒ वि वि वी॒र्ये॑ण वी॒र्ये॑ण॒ वि । \newline
11. व्यृ॑द्ध्यत ऋद्ध्यते॒ वि व्यृ॑द्ध्यते । \newline
12. ऋ॒द्ध्य॒ते॒ रु॒द्रो रु॒द्र ऋ॑द्ध्यत ऋद्ध्यते रु॒द्रः । \newline
13. रु॒द्रो वै वै रु॒द्रो रु॒द्रो वै । \newline
14. वा ए॒ष ए॒ष वै वा ए॒षः । \newline
15. ए॒ष यद् यदे॒ष ए॒ष यत् । \newline
16. यद॒ग्नि र॒ग्निर् यद् यद॒ग्निः । \newline
17. अ॒ग्नि स्तस्य॒ तस्या॒ग्नि र॒ग्नि स्तस्य॑ । \newline
18. तस्य॑ ति॒स्र स्ति॒स्र स्तस्य॒ तस्य॑ ति॒स्रः । \newline
19. ति॒स्रः श॑र॒व्याः᳚ शर॒व्या᳚ स्ति॒स्र स्ति॒स्रः श॑र॒व्याः᳚ । \newline
20. श॒र॒व्याः᳚ प्र॒तीची᳚ प्र॒तीची॑ शर॒व्याः᳚ शर॒व्याः᳚ प्र॒तीची᳚ । \newline
21. प्र॒तीची॑ ति॒रश्ची॑ ति॒रश्ची᳚ प्र॒तीची᳚ प्र॒तीची॑ ति॒रश्ची᳚ । \newline
22. ति॒रश्च्य॒ नू च्य॒नूची॑ ति॒रश्ची॑ ति॒र श्च्य॒नूची᳚ । \newline
23. अ॒नूची॒ ताभ्य॒ स्ताभ्यो॒ ऽनू च्य॒नूची॒ ताभ्यः॑ । \newline
24. ताभ्यो॒ वै वै ताभ्य॒ स्ताभ्यो॒ वै । \newline
25. वा ए॒ष ए॒ष वै वा ए॒षः । \newline
26. ए॒ष ऐष ए॒ष आ । \newline
27. आ वृ॑श्च्यते वृश्च्यत॒ आ वृ॑श्च्यते । \newline
28. वृ॒श्च्य॒ते॒ यो यो वृ॑श्च्यते वृश्च्यते॒ यः । \newline
29. यो᳚ ऽग्नि म॒ग्निं ॅयो यो᳚ ऽग्निम् । \newline
30. अ॒ग्निम् चि॑नु॒ते चि॑नु॒ते᳚ ऽग्नि म॒ग्निम् चि॑नु॒ते । \newline
31. चि॒नु॒ते᳚ ऽग्नि म॒ग्निम् चि॑नु॒ते चि॑नु॒ते᳚ ऽग्निम् । \newline
32. अ॒ग्निम् चि॒त्वा चि॒त्वा ऽग्नि म॒ग्निम् चि॒त्वा । \newline
33. चि॒त्वा ति॑सृध॒न्वम् ति॑सृध॒न्वम् चि॒त्वा चि॒त्वा ति॑सृध॒न्वम् । \newline
34. ति॒सृ॒ध॒न्व मया॑चित॒ मया॑चितम् तिसृध॒न्वम् ति॑सृध॒न्व मया॑चितम् । \newline
35. ति॒सृ॒ध॒न्वमिति॑ तिसृ - ध॒न्वम् । \newline
36. अया॑चितम् ब्राह्म॒णाय॑ ब्राह्म॒णाया या॑चित॒ मया॑चितम् ब्राह्म॒णाय॑ । \newline
37. ब्रा॒ह्म॒णाय॑ दद्याद् दद्याद् ब्राह्म॒णाय॑ ब्राह्म॒णाय॑ दद्यात् । \newline
38. द॒द्या॒त् ताभ्य॒ स्ताभ्यो॑ दद्याद् दद्या॒त् ताभ्यः॑ । \newline
39. ताभ्य॑ ए॒वैव ताभ्य॒ स्ताभ्य॑ ए॒व । \newline
40. ए॒व नमो॒ नम॑ ए॒वैव नमः॑ । \newline
41. नम॑स् करोति करोति॒ नमो॒ नम॑ स्करोति । \newline
42. क॒रो॒ त्यथो॒ अथो॑ करोति करो॒ त्यथो᳚ । \newline
43. अथो॒ ताभ्य॒ स्ताभ्यो ऽथो॒ अथो॒ ताभ्यः॑ । \newline
44. अथो॒ इत्यथो᳚ । \newline
45. ताभ्य॑ ए॒वैव ताभ्य॒ स्ताभ्य॑ ए॒व । \newline
46. ए॒वात्मान॑ मा॒त्मान॑ मे॒वै वात्मान᳚म् । \newline
47. आ॒त्मान॒म् निर् णिरा॒त्मान॑ मा॒त्मान॒म् निः । \newline
48. नि ष्क्री॑णीते क्रीणीते॒ निर् णि ष्क्री॑णीते । \newline
49. क्री॒णी॒ते॒ यद् यत् क्री॑णीते क्रीणीते॒ यत् । \newline
50. यत् ते॑ ते॒ यद् यत् ते᳚ । \newline
51. ते॒ रु॒द्र॒ रु॒द्र॒ ते॒ ते॒ रु॒द्र॒ । \newline
52. रु॒द्र॒ पु॒रः पु॒रो रु॑द्र रुद्र पु॒रः । \newline
53. पु॒रो धनु॒र् धनुः॑ पु॒रः पु॒रो धनुः॑ । \newline

\textbf{Ghana Paata } \newline

1. ऋ॒चा ऽऽक्रम॑ण मा॒क्रम॑ण मृ॒च र्‌चा ऽऽक्रम॑ण॒म् प्रति॒ प्रत्या॒क्रम॑ण मृ॒च र्‌चा ऽऽक्रम॑ण॒म् प्रति॑ । \newline
2. आ॒क्रम॑ण॒म् प्रति॒ प्रत्या॒क्रम॑ण मा॒क्रम॑ण॒म् प्रतीष्ट॑का॒ मिष्ट॑का॒म् प्रत्या॒क्रम॑ण मा॒क्रम॑ण॒म् प्रतीष्ट॑काम् । \newline
3. आ॒क्रम॑ण॒मित्या᳚ - क्रम॑णम् । \newline
4. प्रतीष्ट॑का॒ मिष्ट॑का॒म् प्रति॒ प्रतीष्ट॑का॒ मुपो पेष्ट॑का॒म् प्रति॒ प्रतीष्ट॑का॒ मुप॑ । \newline
5. इष्ट॑का॒ मुपोपेष्ट॑का॒ मिष्ट॑का॒ मुप॑ दद्ध्याद् दद्ध्या॒ दुपेष्ट॑का॒ मिष्ट॑का॒ मुप॑ दद्ध्यात् । \newline
6. उप॑ दद्ध्याद् दद्ध्या॒ दुपोप॑ दद्ध्या॒न् न न द॑द्ध्या॒ दुपोप॑ दद्ध्या॒न् न । \newline
7. द॒द्ध्या॒न् न न द॑द्ध्याद् दद्ध्या॒न् नेन्द्रि॒ये णे᳚न्द्रि॒येण॒ न द॑द्ध्याद् दद्ध्या॒न् नेन्द्रि॒येण॑ । \newline
8. नेन्द्रि॒ये णे᳚न्द्रि॒येण॒ न नेन्द्रि॒येण॑ वी॒र्ये॑ण वी॒र्ये॑ णेन्द्रि॒येण॒ न नेन्द्रि॒येण॑ वी॒र्ये॑ण । \newline
9. इ॒न्द्रि॒येण॑ वी॒र्ये॑ण वी॒र्ये॑ णेन्द्रि॒ये णे᳚न्द्रि॒येण॑ वी॒र्ये॑ण॒ वि वि वी॒र्ये॑ णेन्द्रि॒ये णे᳚न्द्रि॒येण॑ वी॒र्ये॑ण॒ वि । \newline
10. वी॒र्ये॑ण॒ वि वि वी॒र्ये॑ण वी॒र्ये॑ण॒ व्यृ॑द्ध्यत ऋद्ध्यते॒ वि वी॒र्ये॑ण वी॒र्ये॑ण॒ व्यृ॑द्ध्यते । \newline
11. व्यृ॑द्ध्यत ऋद्ध्यते॒ वि व्यृ॑द्ध्यते रु॒द्रो रु॒द्र ऋ॑द्ध्यते॒ वि व्यृ॑द्ध्यते रु॒द्रः । \newline
12. ऋ॒द्ध्य॒ते॒ रु॒द्रो रु॒द्र ऋ॑द्ध्यत ऋद्ध्यते रु॒द्रो वै वै रु॒द्र ऋ॑द्ध्यत ऋद्ध्यते रु॒द्रो वै । \newline
13. रु॒द्रो वै वै रु॒द्रो रु॒द्रो वा ए॒ष ए॒ष वै रु॒द्रो रु॒द्रो वा ए॒षः । \newline
14. वा ए॒ष ए॒ष वै वा ए॒ष यद् यदे॒ष वै वा ए॒ष यत् । \newline
15. ए॒ष यद् यदे॒ष ए॒ष यद॒ग्नि र॒ग्निर् यदे॒ष ए॒ष यद॒ग्निः । \newline
16. यद॒ग्नि र॒ग्निर् यद् यद॒ग्नि स्तस्य॒ तस्या॒ ग्निर् यद् यद॒ग्नि स्तस्य॑ । \newline
17. अ॒ग्नि स्तस्य॒ तस्या॒ग्नि र॒ग्नि स्तस्य॑ ति॒स्र स्ति॒स्र स्तस्या॒ग्नि र॒ग्नि स्तस्य॑ ति॒स्रः । \newline
18. तस्य॑ ति॒स्र स्ति॒स्र स्तस्य॒ तस्य॑ ति॒स्रः श॑र॒व्याः᳚ शर॒व्या᳚ स्ति॒स्र स्तस्य॒ तस्य॑ ति॒स्रः श॑र॒व्याः᳚ । \newline
19. ति॒स्रः श॑र॒व्याः᳚ शर॒व्या᳚ स्ति॒स्र स्ति॒स्रः श॑र॒व्याः᳚ प्र॒तीची᳚ प्र॒तीची॑ शर॒व्या᳚ स्ति॒स्र स्ति॒स्रः श॑र॒व्याः᳚ प्र॒तीची᳚ । \newline
20. श॒र॒व्याः᳚ प्र॒तीची᳚ प्र॒तीची॑ शर॒व्याः᳚ शर॒व्याः᳚ प्र॒तीची॑ ति॒रश्ची॑ ति॒रश्ची᳚ प्र॒तीची॑ शर॒व्याः᳚ शर॒व्याः᳚ प्र॒तीची॑ ति॒रश्ची᳚ । \newline
21. प्र॒तीची॑ ति॒रश्ची॑ ति॒रश्ची᳚ प्र॒तीची᳚ प्र॒तीची॑ ति॒रश्च्य॒ नूच्य॒नूची॑ ति॒रश्ची᳚ प्र॒तीची᳚ प्र॒तीची॑ ति॒र श्च्य॒नूची᳚ । \newline
22. ति॒रश्च्य॒नू च्य॒नूची॑ ति॒रश्ची॑ ति॒र श्च्य॒नूची॒ ताभ्य॒ स्ताभ्यो॒ ऽनूची॑ ति॒रश्ची॑ ति॒र श्च्य॒नूची॒ ताभ्यः॑ । \newline
23. अ॒नूची॒ ताभ्य॒ स्ताभ्यो॒ ऽनूच्य॒नूची॒ ताभ्यो॒ वै वै ताभ्यो॒ ऽनूच्य॒नूची॒ ताभ्यो॒ वै । \newline
24. ताभ्यो॒ वै वै ताभ्य॒ स्ताभ्यो॒ वा ए॒ष ए॒ष वै ताभ्य॒ स्ताभ्यो॒ वा ए॒षः । \newline
25. वा ए॒ष ए॒ष वै वा ए॒ष ऐष वै वा ए॒ष आ । \newline
26. ए॒ष ऐष ए॒ष आ वृ॑श्च्यते वृश्च्यत॒ ऐष ए॒ष आ वृ॑श्च्यते । \newline
27. आ वृ॑श्च्यते वृश्च्यत॒ आ वृ॑श्च्यते॒ यो यो वृ॑श्च्यत॒ आ वृ॑श्च्यते॒ यः । \newline
28. वृ॒श्च्य॒ते॒ यो यो वृ॑श्च्यते वृश्च्यते॒ यो᳚ ऽग्नि म॒ग्निं ॅयो वृ॑श्च्यते वृश्च्यते॒ यो᳚ ऽग्निम् । \newline
29. यो᳚ ऽग्नि म॒ग्निं ॅयो यो᳚ ऽग्निम् चि॑नु॒ते चि॑नु॒ते᳚ ऽग्निं ॅयो यो᳚ ऽग्निम् चि॑नु॒ते । \newline
30. अ॒ग्निम् चि॑नु॒ते चि॑नु॒ते᳚ ऽग्नि म॒ग्निम् चि॑नु॒ते᳚ ऽग्नि म॒ग्निम् चि॑नु॒ते᳚ ऽग्नि म॒ग्निम् चि॑नु॒ते᳚ ऽग्निम् । \newline
31. चि॒नु॒ते᳚ ऽग्नि म॒ग्निम् चि॑नु॒ते चि॑नु॒ते᳚ ऽग्निम् चि॒त्वा चि॒त्वा ऽग्निम् चि॑नु॒ते चि॑नु॒ते᳚ ऽग्निम् चि॒त्वा । \newline
32. अ॒ग्निम् चि॒त्वा चि॒त्वा ऽग्नि म॒ग्निम् चि॒त्वा ति॑सृध॒न्वम् ति॑सृध॒न्वम् चि॒त्वा ऽग्नि म॒ग्निम् चि॒त्वा ति॑सृध॒न्वम् । \newline
33. चि॒त्वा ति॑सृध॒न्वम् ति॑सृध॒न्वम् चि॒त्वा चि॒त्वा ति॑सृध॒न्व मया॑चित॒ मया॑चितम् तिसृध॒न्वम् चि॒त्वा चि॒त्वा ति॑सृध॒न्व मया॑चितम् । \newline
34. ति॒सृ॒ध॒न्व मया॑चित॒ मया॑चितम् तिसृध॒न्वम् ति॑सृध॒न्व मया॑चितम् ब्राह्म॒णाय॑ ब्राह्म॒णाया या॑चितम् तिसृध॒न्वम् ति॑सृध॒न्व मया॑चितम् ब्राह्म॒णाय॑ । \newline
35. ति॒सृ॒ध॒न्वमिति॑ तिसृ - ध॒न्वम् । \newline
36. अया॑चितम् ब्राह्म॒णाय॑ ब्राह्म॒णाया या॑चित॒ मया॑चितम् ब्राह्म॒णाय॑ दद्याद् दद्याद् ब्राह्म॒णाया या॑चित॒ मया॑चितम् ब्राह्म॒णाय॑ दद्यात् । \newline
37. ब्रा॒ह्म॒णाय॑ दद्याद् दद्याद् ब्राह्म॒णाय॑ ब्राह्म॒णाय॑ दद्या॒त् ताभ्य॒ स्ताभ्यो॑ दद्याद् ब्राह्म॒णाय॑ ब्राह्म॒णाय॑ दद्या॒त् ताभ्यः॑ । \newline
38. द॒द्या॒त् ताभ्य॒ स्ताभ्यो॑ दद्याद् दद्या॒त् ताभ्य॑ ए॒वैव ताभ्यो॑ दद्याद् दद्या॒त् ताभ्य॑ ए॒व । \newline
39. ताभ्य॑ ए॒वैव ताभ्य॒ स्ताभ्य॑ ए॒व नमो॒ नम॑ ए॒व ताभ्य॒ स्ताभ्य॑ ए॒व नमः॑ । \newline
40. ए॒व नमो॒ नम॑ ए॒वैव नम॑ स्करोति करोति॒ नम॑ ए॒वैव नम॑ स्करोति । \newline
41. नम॑ स्करोति करोति॒ नमो॒ नम॑ स्करो॒ त्यथो॒ अथो॑ करोति॒ नमो॒ नम॑ स्करो॒ त्यथो᳚ । \newline
42. क॒रो॒ त्यथो॒ अथो॑ करोति करो॒ त्यथो॒ ताभ्य॒ स्ताभ्यो ऽथो॑ करोति करो॒ त्यथो॒ ताभ्यः॑ । \newline
43. अथो॒ ताभ्य॒ स्ताभ्यो ऽथो॒ अथो॒ ताभ्य॑ ए॒वैव ताभ्यो ऽथो॒ अथो॒ ताभ्य॑ ए॒व । \newline
44. अथो॒ इत्यथो᳚ । \newline
45. ताभ्य॑ ए॒वैव ताभ्य॒ स्ताभ्य॑ ए॒वात्मान॑ मा॒त्मान॑ मे॒व ताभ्य॒ स्ताभ्य॑ ए॒वात्मान᳚म् । \newline
46. ए॒वात्मान॑ मा॒त्मान॑ मे॒वै वात्मान॒म् निर् णिरा॒त्मान॑ मे॒वै वात्मान॒म् निः । \newline
47. आ॒त्मान॒म् निर् णिरा॒त्मान॑ मा॒त्मान॒म् निष्क्री॑णीते क्रीणीते॒ निरा॒त्मान॑ मा॒त्मान॒म् निष्क्री॑णीते । \newline
48. नि ष्क्री॑णीते क्रीणीते॒ निर् णि ष्क्री॑णीते॒ यद् यत् क्री॑णीते॒ निर् णि ष्क्री॑णीते॒ यत् । \newline
49. क्री॒णी॒ते॒ यद् यत् क्री॑णीते क्रीणीते॒ यत् ते॑ ते॒ यत् क्री॑णीते क्रीणीते॒ यत् ते᳚ । \newline
50. यत् ते॑ ते॒ यद् यत् ते॑ रुद्र रुद्र ते॒ यद् यत् ते॑ रुद्र । \newline
51. ते॒ रु॒द्र॒ रु॒द्र॒ ते॒ ते॒ रु॒द्र॒ पु॒रः पु॒रो रु॑द्र ते ते रुद्र पु॒रः । \newline
52. रु॒द्र॒ पु॒रः पु॒रो रु॑द्र रुद्र पु॒रो धनु॒र् धनुः॑ पु॒रो रु॑द्र रुद्र पु॒रो धनुः॑ । \newline
53. पु॒रो धनु॒र् धनुः॑ पु॒रः पु॒रो धनु॒ स्तत् तद् धनुः॑ पु॒रः पु॒रो धनु॒ स्तत् । \newline
\pagebreak
\markright{ TS 5.5.7.3  \hfill https://www.vedavms.in \hfill}

\section{ TS 5.5.7.3 }

\textbf{TS 5.5.7.3 } \newline
\textbf{Samhita Paata} \newline

धनु॒स्तद्-वातो॒ अनु॑ वातु ते॒ तस्मै॑ ते॒ रुद्र संॅवथ्स॒रेण॒ नम॑स्करोमि॒ यत्ते॑ रुद्र दक्षि॒णा धनु॒स्तद्-वातो॒ अनु॑ वातु ते॒ तस्मै॑ ते रुद्र परिवथ्स॒रेण॒ नम॑स्करोमि॒ यत्ते॑ रुद्र प॒श्चाद्धनु॒स्तद्-वातो॒ अनु॑ वातु ते॒ तस्मै॑ ते रुद्रेदावथ्स॒रेण॒ नम॑स्करोमि॒ यत्ते॑ रुद्रोत्त॒राद्धनु॒स्त - [  ] \newline

\textbf{Pada Paata} \newline

धनुः॑ । तत् । वातः॑ । अन्विति॑ । वा॒तु॒ । ते॒ । तस्मै᳚ । ते॒ । रु॒द्र॒ । सं॒ॅव॒थ्स॒रेणेति॑ सं - व॒थ्स॒रेण॑ । नमः॑ । क॒रो॒मि॒ । यत् । ते॒ । रु॒द्र॒ । द॒क्षि॒णा । धनुः॑ । तत् । वातः॑ । अन्विति॑ । वा॒तु॒ । ते॒ । तस्मै᳚ । ते॒ । रु॒द्र॒ । प॒रि॒व॒थ्स॒रेणेति॑ परि - व॒थ्स॒रेण॑ । नमः॑ । क॒रो॒मि॒ । यत् । ते॒ । रु॒द्र॒ । प॒श्चात् । धनुः॑ । तत् । वातः॑ । अन्विति॑ । वा॒तु॒ । ते॒ । तस्मै᳚ । ते॒ । रु॒द्र॒ । इ॒दा॒व॒थ्स॒रेण॑ । नमः॑ । क॒रो॒मि॒ । यत् । ते॒ । रु॒द्र॒ । उ॒त्त॒रादित्यु॑त् - त॒रात् । धनुः॑ । तत् ।  \newline


\textbf{Krama Paata} \newline

धनु॒स्तत् । तद् वातः॑ । वातो॒ अनु॑ । अनु॑ वातु । वा॒तु॒ ते॒ । ते॒ तस्मै᳚ । तस्मै॑ ते । ते॒ रु॒द्र॒ । रु॒द्र॒ स॒म्ॅव॒थ्स॒रेण॑ । स॒म्ॅव॒थ्स॒रेण॒ नमः॑ । स॒म्ॅव॒थ्स॒रेणेति॑ सम् - व॒थ्स॒रेण॑ । नम॑स्करोमि । क॒रो॒मि॒ यत् । यत् ते᳚ । ते॒ रु॒द्र॒ । रु॒द्र॒ द॒क्षि॒णा । द॒क्षि॒णा धनुः॑ । धनु॒स्तत् । तद् वातः॑ । वातो॒ अनु॑ । अनु॑ वातु । वा॒तु॒ ते॒ । ते॒ तस्मै᳚ । तस्मै॑ ते । ते॒ रु॒द्र॒ । रु॒द्र॒ प॒रि॒व॒थ्स॒रेण॑ । प॒रि॒व॒थ्स॒रेण॒ नमः॑ । प॒रि॒व॒थ्स॒रेणेति॑ परि - व॒थ्स॒रेण॑ । नम॑स्करोमि । क॒रो॒मि॒ यत् । यत् ते᳚ । ते॒ रु॒द्र॒ । रु॒द्र॒ प॒श्चात् । प॒श्चाद् धनुः॑ । धनु॒स्तत् । तद् वातः॑ । वातो॒ अनु॑ । अनु॑ वातु । वा॒तु॒ ते॒ । ते॒ तस्मै᳚ । तस्मै॑ ते । ते॒ रु॒द्र॒ । रु॒द्रे॒दा॒व॒थ्स॒रेण॑ । इ॒दा॒व॒थ्स॒रेण॒ नमः॑ । नम॑स्करोमि । क॒रो॒मि॒ यत् । यत् ते᳚ । ते॒ रु॒द्र॒ । रु॒द्रो॒त्त॒रात् । उ॒त्त॒राद् धनुः॑ । उ॒त्त॒रादित्यु॑त् - त॒रात् । धनु॒ स्तत् । तद् वातः॑ \newline

\textbf{Jatai Paata} \newline

1. धनु॒ स्तत् तद् धनु॒र् धनु॒ स्तत् । \newline
2. तद् वातो॒ वात॒ स्तत् तद् वातः॑ । \newline
3. वातो॒ अन्वनु॒ वातो॒ वातो॒ अनु॑ । \newline
4. अनु॑ वातु वा॒त्वन् वनु॑ वातु । \newline
5. वा॒तु॒ ते॒ ते॒ वा॒तु॒ वा॒तु॒ ते॒ । \newline
6. ते॒ तस्मै॒ तस्मै॑ ते ते॒ तस्मै᳚ । \newline
7. तस्मै॑ ते ते॒ तस्मै॒ तस्मै॑ ते । \newline
8. ते॒ रु॒द्र॒ रु॒द्र॒ ते॒ ते॒ रु॒द्र॒ । \newline
9. रु॒द्र॒ सं॒ॅव॒थ्स॒रेण॑ संॅवथ्स॒रेण॑ रुद्र रुद्र संॅवथ्स॒रेण॑ । \newline
10. सं॒ॅव॒थ्स॒रेण॒ नमो॒ नमः॑ संॅवथ्स॒रेण॑ संॅवथ्स॒रेण॒ नमः॑ । \newline
11. सं॒ॅव॒थ्स॒रेणेति॑ सं - व॒थ्स॒रेण॑ । \newline
12. नम॑ स्करोमि करोमि॒ नमो॒ नम॑ स्करोमि । \newline
13. क॒रो॒मि॒ यद् यत् क॑रोमि करोमि॒ यत् । \newline
14. यत् ते॑ ते॒ यद् यत् ते᳚ । \newline
15. ते॒ रु॒द्र॒ रु॒द्र॒ ते॒ ते॒ रु॒द्र॒ । \newline
16. रु॒द्र॒ द॒क्षि॒णा द॑क्षि॒णा रु॑द्र रुद्र दक्षि॒णा । \newline
17. द॒क्षि॒णा धनु॒र् धनु॑र् दक्षि॒णा द॑क्षि॒णा धनुः॑ । \newline
18. धनु॒ स्तत् तद् धनु॒र् धनु॒ स्तत् । \newline
19. तद् वातो॒ वात॒ स्तत् तद् वातः॑ । \newline
20. वातो॒ अन्वनु॒ वातो॒ वातो॒ अनु॑ । \newline
21. अनु॑ वातु वा॒त्वन् वनु॑ वातु । \newline
22. वा॒तु॒ ते॒ ते॒ वा॒तु॒ वा॒तु॒ ते॒ । \newline
23. ते॒ तस्मै॒ तस्मै॑ ते ते॒ तस्मै᳚ । \newline
24. तस्मै॑ ते ते॒ तस्मै॒ तस्मै॑ ते । \newline
25. ते॒ रु॒द्र॒ रु॒द्र॒ ते॒ ते॒ रु॒द्र॒ । \newline
26. रु॒द्र॒ प॒रि॒व॒थ्स॒रेण॑ परिवथ्स॒रेण॑ रुद्र रुद्र परिवथ्स॒रेण॑ । \newline
27. प॒रि॒व॒थ्स॒रेण॒ नमो॒ नमः॑ परिवथ्स॒रेण॑ परिवथ्स॒रेण॒ नमः॑ । \newline
28. प॒रि॒व॒थ्स॒रेणेति॑ परि - व॒थ्स॒रेण॑ । \newline
29. नम॑ स्करोमि करोमि॒ नमो॒ नम॑ स्करोमि । \newline
30. क॒रो॒मि॒ यद् यत् क॑रोमि करोमि॒ यत् । \newline
31. यत् ते॑ ते॒ यद् यत् ते᳚ । \newline
32. ते॒ रु॒द्र॒ रु॒द्र॒ ते॒ ते॒ रु॒द्र॒ । \newline
33. रु॒द्र॒ प॒श्चात् प॒श्चाद् रु॑द्र रुद्र प॒श्चात् । \newline
34. प॒श्चाद् धनु॒र् धनुः॑ प॒श्चात् प॒श्चाद् धनुः॑ । \newline
35. धनु॒ स्तत् तद् धनु॒र् धनु॒ स्तत् । \newline
36. तद् वातो॒ वात॒ स्तत् तद् वातः॑ । \newline
37. वातो॒ अन्वनु॒ वातो॒ वातो॒ अनु॑ । \newline
38. अनु॑ वातु वा॒त्वन् वनु॑ वातु । \newline
39. वा॒तु॒ ते॒ ते॒ वा॒तु॒ वा॒तु॒ ते॒ । \newline
40. ते॒ तस्मै॒ तस्मै॑ ते ते॒ तस्मै᳚ । \newline
41. तस्मै॑ ते ते॒ तस्मै॒ तस्मै॑ ते । \newline
42. ते॒ रु॒द्र॒ रु॒द्र॒ ते॒ ते॒ रु॒द्र॒ । \newline
43. रु॒द्रे॒ दा॒व॒थ्स॒रे णे॑दावथ्स॒रेण॑ रुद्र रुद्रे दावथ्स॒रेण॑ । \newline
44. इ॒दा॒व॒थ्स॒रेण॒ नमो॒ नम॑ इदावथ्स॒रे णे॑दावथ्स॒रेण॒ नमः॑ । \newline
45. नम॑ स्करोमि करोमि॒ नमो॒ नम॑ स्करोमि । \newline
46. क॒रो॒मि॒ यद् यत् क॑रोमि करोमि॒ यत् । \newline
47. यत् ते॑ ते॒ यद् यत् ते᳚ । \newline
48. ते॒ रु॒द्र॒ रु॒द्र॒ ते॒ ते॒ रु॒द्र॒ । \newline
49. रु॒द्रो॒ त्त॒रा दु॑त्त॒राद् रु॑द्र रुद्रोत्त॒रात् । \newline
50. उ॒त्त॒राद् धनु॒र् धनु॑ रुत्त॒रा दु॑त्त॒राद् धनुः॑ । \newline
51. उ॒त्त॒रादित्यु॑त् - त॒रात् । \newline
52. धनु॒ स्तत् तद् धनु॒र् धनु॒ स्तत् । \newline
53. तद् वातो॒ वात॒ स्तत् तद् वातः॑ । \newline

\textbf{Ghana Paata } \newline

1. ढनु॒ स्तत् तद् धनु॒र् धनु॒ स्तद् वातो॒ वात॒ स्तद् धनु॒र् धनु॒ स्तद् वातः॑ । \newline
2. तद् वातो॒ वात॒ स्तत् तद् वातो॒ अन्वनु॒ वात॒ स्तत् तद् वातो॒ अनु॑ । \newline
3. वातो॒ अन्वनु॒ वातो॒ वातो॒ अनु॑ वातु वा॒त्वनु॒ वातो॒ वातो॒ अनु॑ वातु । \newline
4. अनु॑ वातु वा॒त्वन् वनु॑ वातु ते ते वा॒त्वन् वनु॑ वातु ते । \newline
5. वा॒तु॒ ते॒ ते॒ वा॒तु॒ वा॒तु॒ ते॒ तस्मै॒ तस्मै॑ ते वातु वातु ते॒ तस्मै᳚ । \newline
6. ते॒ तस्मै॒ तस्मै॑ ते ते॒ तस्मै॑ ते ते॒ तस्मै॑ ते ते॒ तस्मै॑ ते । \newline
7. तस्मै॑ ते ते॒ तस्मै॒ तस्मै॑ ते रुद्र रुद्र ते॒ तस्मै॒ तस्मै॑ ते रुद्र । \newline
8. ते॒ रु॒द्र॒ रु॒द्र॒ ते॒ ते॒ रु॒द्र॒ सं॒ॅव॒थ्स॒रेण॑ संॅवथ्स॒रेण॑ रुद्र ते ते रुद्र संॅवथ्स॒रेण॑ । \newline
9. रु॒द्र॒ सं॒ॅव॒थ्स॒रेण॑ संॅवथ्स॒रेण॑ रुद्र रुद्र संॅवथ्स॒रेण॒ नमो॒ नमः॑ संॅवथ्स॒रेण॑ रुद्र रुद्र संॅवथ्स॒रेण॒ नमः॑ । \newline
10. सं॒ॅव॒थ्स॒रेण॒ नमो॒ नमः॑ संॅवथ्स॒रेण॑ संॅवथ्स॒रेण॒ नम॑ स्करोमि करोमि॒ नमः॑ संॅवथ्स॒रेण॑ संॅवथ्स॒रेण॒ नम॑ स्करोमि । \newline
11. सं॒ॅव॒थ्स॒रेणेति॑ सं - व॒थ्स॒रेण॑ । \newline
12. नम॑ स्करोमि करोमि॒ नमो॒ नम॑ स्करोमि॒ यद् यत् क॑रोमि॒ नमो॒ नम॑ स्करोमि॒ यत् । \newline
13. क॒रो॒मि॒ यद् यत् क॑रोमि करोमि॒ यत् ते॑ ते॒ यत् क॑रोमि करोमि॒ यत् ते᳚ । \newline
14. यत् ते॑ ते॒ यद् यत् ते॑ रुद्र रुद्र ते॒ यद् यत् ते॑ रुद्र । \newline
15. ते॒ रु॒द्र॒ रु॒द्र॒ ते॒ ते॒ रु॒द्र॒ द॒क्षि॒णा द॑क्षि॒णा रु॑द्र ते ते रुद्र दक्षि॒णा । \newline
16. रु॒द्र॒ द॒क्षि॒णा द॑क्षि॒णा रु॑द्र रुद्र दक्षि॒णा धनु॒र् धनु॑र् दक्षि॒णा रु॑द्र रुद्र दक्षि॒णा धनुः॑ । \newline
17. द॒क्षि॒णा धनु॒र् धनु॑र् दक्षि॒णा द॑क्षि॒णा धनु॒ स्तत् तद् धनु॑र् दक्षि॒णा द॑क्षि॒णा धनु॒ स्तत् । \newline
18. ढनु॒ स्तत् तद् धनु॒र् धनु॒ स्तद् वातो॒ वात॒ स्तद् धनु॒र् धनु॒ स्तद् वातः॑ । \newline
19. तद् वातो॒ वात॒ स्तत् तद् वातो॒ अन्वनु॒ वात॒ स्तत् तद् वातो॒ अनु॑ । \newline
20. वातो॒ अन्वनु॒ वातो॒ वातो॒ अनु॑ वातु वा॒त्वनु॒ वातो॒ वातो॒ अनु॑ वातु । \newline
21. अनु॑ वातु वा॒त्वन् वनु॑ वातु ते ते वा॒त्वन् वनु॑ वातु ते । \newline
22. वा॒तु॒ ते॒ ते॒ वा॒तु॒ वा॒तु॒ ते॒ तस्मै॒ तस्मै॑ ते वातु वातु ते॒ तस्मै᳚ । \newline
23. ते॒ तस्मै॒ तस्मै॑ ते ते॒ तस्मै॑ ते ते॒ तस्मै॑ ते ते॒ तस्मै॑ ते । \newline
24. तस्मै॑ ते ते॒ तस्मै॒ तस्मै॑ ते रुद्र रुद्र ते॒ तस्मै॒ तस्मै॑ ते रुद्र । \newline
25. ते॒ रु॒द्र॒ रु॒द्र॒ ते॒ ते॒ रु॒द्र॒ प॒रि॒व॒थ्स॒रेण॑ परिवथ्स॒रेण॑ रुद्र ते ते रुद्र परिवथ्स॒रेण॑ । \newline
26. रु॒द्र॒ प॒रि॒व॒थ्स॒रेण॑ परिवथ्स॒रेण॑ रुद्र रुद्र परिवथ्स॒रेण॒ नमो॒ नमः॑ परिवथ्स॒रेण॑ रुद्र रुद्र परिवथ्स॒रेण॒ नमः॑ । \newline
27. प॒रि॒व॒थ्स॒रेण॒ नमो॒ नमः॑ परिवथ्स॒रेण॑ परिवथ्स॒रेण॒ नम॑ स्करोमि करोमि॒ नमः॑ परिवथ्स॒रेण॑ परिवथ्स॒रेण॒ नम॑ स्करोमि । \newline
28. प॒रि॒व॒थ्स॒रेणेति॑ परि - व॒थ्स॒रेण॑ । \newline
29. नम॑ स्करोमि करोमि॒ नमो॒ नम॑ स्करोमि॒ यद् यत् क॑रोमि॒ नमो॒ नम॑ स्करोमि॒ यत् । \newline
30. क॒रो॒मि॒ यद् यत् क॑रोमि करोमि॒ यत् ते॑ ते॒ यत् क॑रोमि करोमि॒ यत् ते᳚ । \newline
31. यत् ते॑ ते॒ यद् यत् ते॑ रुद्र रुद्र ते॒ यद् यत् ते॑ रुद्र । \newline
32. ते॒ रु॒द्र॒ रु॒द्र॒ ते॒ ते॒ रु॒द्र॒ प॒श्चात् प॒श्चाद् रु॑द्र ते ते रुद्र प॒श्चात् । \newline
33. रु॒द्र॒ प॒श्चात् प॒श्चाद् रु॑द्र रुद्र प॒श्चाद् धनु॒र् धनुः॑ प॒श्चाद् रु॑द्र रुद्र प॒श्चाद् धनुः॑ । \newline
34. प॒श्चाद् धनु॒र् धनुः॑ प॒श्चात् प॒श्चाद् धनु॒ स्तत् तद् धनुः॑ प॒श्चात् प॒श्चाद् धनु॒ स्तत् । \newline
35. ढनु॒ स्तत् तद् धनु॒र् धनु॒ स्तद् वातो॒ वात॒ स्तद् धनु॒र् धनु॒ स्तद् वातः॑ । \newline
36. तद् वातो॒ वात॒ स्तत् तद् वातो॒ अन्वनु॒ वात॒ स्तत् तद् वातो॒ अनु॑ । \newline
37. वातो॒ अन्वनु॒ वातो॒ वातो॒ अनु॑ वातु वा॒त्वनु॒ वातो॒ वातो॒ अनु॑ वातु । \newline
38. अनु॑ वातु वा॒त्वन् वनु॑ वातु ते ते वा॒त्वन् वनु॑ वातु ते । \newline
39. वा॒तु॒ ते॒ ते॒ वा॒तु॒ वा॒तु॒ ते॒ तस्मै॒ तस्मै॑ ते वातु वातु ते॒ तस्मै᳚ । \newline
40. ते॒ तस्मै॒ तस्मै॑ ते ते॒ तस्मै॑ ते ते॒ तस्मै॑ ते ते॒ तस्मै॑ ते । \newline
41. तस्मै॑ ते ते॒ तस्मै॒ तस्मै॑ ते रुद्र रुद्र ते॒ तस्मै॒ तस्मै॑ ते रुद्र । \newline
42. ते॒ रु॒द्र॒ रु॒द्र॒ ते॒ ते॒ रु॒द्रे॒ दा॒व॒थ्स॒रेणे॑ दावथ्स॒रेण॑ रुद्र ते ते रुद्रे दावथ्स॒रेण॑ । \newline
43. रु॒द्रे॒ दा॒व॒थ्स॒रेणे॑ दावथ्स॒रेण॑ रुद्र रुद्रे दावथ्स॒रेण॒ नमो॒ नम॑ इदावथ्स॒रेण॑ रुद्र रुद्रे दावथ्स॒रेण॒ नमः॑ । \newline
44. इ॒दा॒व॒थ्स॒रेण॒ नमो॒ नम॑ इदावथ्स॒रे णे॑दावथ्स॒रेण॒ नम॑ स्करोमि करोमि॒ नम॑ इदावथ्स॒रे
णे॑दावथ्स॒रेण॒ नम॑ स्करोमि । \newline
45. नम॑ स्करोमि करोमि॒ नमो॒ नम॑ स्करोमि॒ यद् यत् क॑रोमि॒ नमो॒ नम॑ स्करोमि॒ यत् । \newline
46. क॒रो॒मि॒ यद् यत् क॑रोमि करोमि॒ यत् ते॑ ते॒ यत् क॑रोमि करोमि॒ यत् ते᳚ । \newline
47. यत् ते॑ ते॒ यद् यत् ते॑ रुद्र रुद्र ते॒ यद् यत् ते॑ रुद्र । \newline
48. ते॒ रु॒द्र॒ रु॒द्र॒ ते॒ ते॒ रु॒द्रो॒त्त॒रा दु॑त्त॒राद् रु॑द्र ते ते रुद्रोत्त॒रात् । \newline
49. रु॒द्रो॒त्त॒रा दु॑त्त॒राद् रु॑द्र रुद्रोत्त॒राद् धनु॒र् धनु॑ रुत्त॒राद् रु॑द्र रुद्रोत्त॒राद् धनुः॑ । \newline
50. उ॒त्त॒राद् धनु॒र् धनु॑ रुत्त॒रा दु॑त्त॒राद् धनु॒ स्तत् तद् धनु॑ रुत्त॒रा दु॑त्त॒राद् धनु॒ स्तत् । \newline
51. उ॒त्त॒रादित्यु॑त् - त॒रात् । \newline
52. ढनु॒ स्तत् तद् धनु॒र् धनु॒ स्तद् वातो॒ वात॒ स्तद् धनु॒र् धनु॒ स्तद् वातः॑ । \newline
53. तद् वातो॒ वात॒ स्तत् तद् वातो॒ अन्वनु॒ वात॒ स्तत् तद् वातो॒ अनु॑ । \newline
\pagebreak
\markright{ TS 5.5.7.4  \hfill https://www.vedavms.in \hfill}

\section{ TS 5.5.7.4 }

\textbf{TS 5.5.7.4 } \newline
\textbf{Samhita Paata} \newline

द्वातो॒ अनु॑ वातु ते॒ तस्मै॑ ते रुद्रेदुवथ्स॒रेण॒ नम॑स्करोमि॒ यत्ते॑ रुद्रो॒परि॒ धनु॒स्तद्-वातो॒ अनु॑ वातु ते॒ तस्मै॑ ते रुद्र वथ्स॒रेण॒ नम॑स्करोमि रु॒द्रो वा ए॒ष यद॒ग्निः स यथा᳚ व्या॒घ्रः क्रु॒द्धस्तिष्ठ॑त्ये॒वं ॅवा ए॒ष ए॒तर्.हि॒ सञ्चि॑तमे॒तैरुप॑ तिष्ठते नमस्का॒रैरे॒वैनꣳ॑ शमयति॒ ये᳚ऽग्नयः॑ - [  ] \newline

\textbf{Pada Paata} \newline

वातः॑ । अन्विति॑ । वा॒तु॒ । ते॒ । तस्मै᳚ । ते॒ । रु॒द्र॒ । इ॒दु॒व॒थ्स॒रेणेती॑दु- व॒थ्स॒रेण॑ । नमः॑ । क॒रो॒मि॒ । यत् । ते॒ । रु॒द्र॒ । उ॒परि॑ । धनुः॑ । तत् । वातः॑ । अन्विति॑ । वा॒तु॒ । ते॒ । तस्मै᳚ । ते॒ । रु॒द्र॒ । व॒थ्स॒रेण॑ । नमः॑ । क॒रो॒मि॒ । रु॒द्रः । वै । ए॒षः । यत् । अ॒ग्निः । सः । यथा᳚ । व्या॒घ्रः । क्रु॒द्धः । तिष्ठ॑ति । ए॒वम् । वै । ए॒षः । ए॒तर्.हि॑ । सञ्चि॑त॒मिति॒ सं - चि॒त॒म् । ए॒तैः । उपेति॑ । ति॒ष्ठ॒ते॒ । न॒म॒स्का॒रैरिति॑ नमः-का॒रैः । ए॒व । ए॒न॒म् । श॒म॒य॒ति॒ । ये । अ॒ग्नयः॑ ।  \newline


\textbf{Krama Paata} \newline

वातो॒ अनु॑ । अनु॑ वातु । वा॒तु॒ ते॒ । ते॒ तस्मै᳚ । तस्मै॑ ते । ते॒ रु॒द्र॒ । रु॒द्रे॒दु॒व॒थ्स॒रेण॑ । इ॒दु॒व॒थ्स॒रेण॒ नमः॑ । इ॒दु॒व॒थ्स॒रेणेती॑दु - व॒थ्स॒रेण॑ । नम॑स्करोमि । क॒रो॒मि॒ यत् । यत् ते᳚ । ते॒ रु॒द्र॒ । रु॒द्रो॒परि॑ । उ॒परि॒ धनुः॑ । धनु॒स्तत् । तद् वातः॑ । वातो॒ अनु॑ । अनु॑ वातु । वा॒तु॒ ते॒ । ते॒ तस्मै᳚ । तस्मै॑ ते । ते॒ रु॒द्र॒ । रु॒द्र॒ व॒थ्स॒रेण॑ । व॒थ्स॒रेण॒ नमः॑ । नम॑स्करोमि । क॒रो॒मि॒ रु॒द्रः । रु॒द्रो वै । वा ए॒षः । ए॒ष यत् । यद॒ग्निः । अ॒ग्निः सः । स यथा᳚ । यथा᳚ व्या॒घ्रः । व्या॒घ्रः क्रु॒द्धः । क्रु॒द्धस्तिष्ठ॑ति । तिष्ठ॑त्ये॒वम् । ए॒वम् ॅवै । वा ए॒षः । ए॒ष ए॒तर्.हि॑ । ए॒तर्.हि॒ सञ्चि॑तम् । सञ्चि॑तमे॒तैः । सञ्चि॑त॒मिति॒ सम् - चि॒त॒म् । ए॒तैरुप॑ । उप॑ तिष्ठते । ति॒ष्ठ॒ते॒ न॒म॒स्का॒रैः । न॒म॒स्का॒रैरे॒व । न॒म॒स्का॒रैरिति॑ नमः - का॒रैः । ए॒वैन᳚म् । ए॒नꣳ॒॒ श॒म॒य॒ति॒ । श॒म॒य॒ति॒ ये । ये᳚ऽग्नयः॑ । अ॒ग्नयः॑ पुरी॒ष्याः᳚ \newline

\textbf{Jatai Paata} \newline

1. वातो॒ अन्वनु॒ वातो॒ वातो॒ अनु॑ । \newline
2. अनु॑ वातु वा॒त्वन् वनु॑ वातु । \newline
3. वा॒तु॒ ते॒ ते॒ वा॒तु॒ वा॒तु॒ ते॒ । \newline
4. ते॒ तस्मै॒ तस्मै॑ ते ते॒ तस्मै᳚ । \newline
5. तस्मै॑ ते ते॒ तस्मै॒ तस्मै॑ ते । \newline
6. ते॒ रु॒द्र॒ रु॒द्र॒ ते॒ ते॒ रु॒द्र॒ । \newline
7. रु॒द्रे॒ दु॒व॒थ्स॒रेणे॑ दुवथ्स॒रेण॑ रुद्र रुद्रे दुवथ्स॒रेण॑ । \newline
8. इ॒दु॒व॒थ्स॒रेण॒ नमो॒ नम॑ इदुवथ्स॒रेणे॑ दुवथ्स॒रेण॒ नमः॑ । \newline
9. इ॒दु॒व॒थ्स॒रेणेती॑दु - व॒थ्स॒रेण॑ । \newline
10. नम॑ स्करोमि करोमि॒ नमो॒ नम॑ स्करोमि । \newline
11. क॒रो॒मि॒ यद् यत् क॑रोमि करोमि॒ यत् । \newline
12. यत् ते॑ ते॒ यद् यत् ते᳚ । \newline
13. ते॒ रु॒द्र॒ रु॒द्र॒ ते॒ ते॒ रु॒द्र॒ । \newline
14. रु॒द्रो॒ पर्यु॒परि॑ रुद्र रुद्रो॒परि॑ । \newline
15. उ॒परि॒ धनु॒र् धनु॑ रु॒पर्यु॒परि॒ धनुः॑ । \newline
16. धनु॒ स्तत् तद् धनु॒र् धनु॒ स्तत् । \newline
17. तद् वातो॒ वात॒ स्तत् तद् वातः॑ । \newline
18. वातो॒ अन्वनु॒ वातो॒ वातो॒ अनु॑ । \newline
19. अनु॑ वातु वा॒त्वन् वनु॑ वातु । \newline
20. वा॒तु॒ ते॒ ते॒ वा॒तु॒ वा॒तु॒ ते॒ । \newline
21. ते॒ तस्मै॒ तस्मै॑ ते ते॒ तस्मै᳚ । \newline
22. तस्मै॑ ते ते॒ तस्मै॒ तस्मै॑ ते । \newline
23. ते॒ रु॒द्र॒ रु॒द्र॒ ते॒ ते॒ रु॒द्र॒ । \newline
24. रु॒द्र॒ व॒थ्स॒रेण॑ वथ्स॒रेण॑ रुद्र रुद्र वथ्स॒रेण॑ । \newline
25. व॒थ्स॒रेण॒ नमो॒ नमो॑ वथ्स॒रेण॑ वथ्स॒रेण॒ नमः॑ । \newline
26. नम॑ स्करोमि करोमि॒ नमो॒ नम॑ स्करोमि । \newline
27. क॒रो॒मि॒ रु॒द्रो रु॒द्रः क॑रोमि करोमि रु॒द्रः । \newline
28. रु॒द्रो वै वै रु॒द्रो रु॒द्रो वै । \newline
29. वा ए॒ष ए॒ष वै वा ए॒षः । \newline
30. ए॒ष यद् यदे॒ष ए॒ष यत् । \newline
31. यद॒ग्नि र॒ग्निर् यद् यद॒ग्निः । \newline
32. अ॒ग्निः स सो᳚ ऽग्नि र॒ग्निः सः । \newline
33. स यथा॒ यथा॒ स स यथा᳚ । \newline
34. यथा᳚ व्या॒घ्रो व्या॒घ्रो यथा॒ यथा᳚ व्या॒घ्रः । \newline
35. व्या॒घ्रः क्रु॒द्धः क्रु॒द्धो व्या॒घ्रो व्या॒घ्रः क्रु॒द्धः । \newline
36. क्रु॒द्ध स्तिष्ठ॑ति॒ तिष्ठ॑ति क्रु॒द्धः क्रु॒द्ध स्तिष्ठ॑ति । \newline
37. तिष्ठ॑ त्ये॒व मे॒वम् तिष्ठ॑ति॒ तिष्ठ॑ त्ये॒वम् । \newline
38. ए॒वं ॅवै वा ए॒व मे॒वं ॅवै । \newline
39. वा ए॒ष ए॒ष वै वा ए॒षः । \newline
40. ए॒ष ए॒तर् ह्ये॒तर् ह्ये॒ष ए॒ष ए॒तर्.हि॑ । \newline
41. ए॒तर्.हि॒ सञ्चि॑तꣳ॒॒ सञ्चि॑त मे॒तर् ह्ये॒तर्.हि॒ सञ्चि॑तम् । \newline
42. सञ्चि॑त मे॒तै रे॒तैः सञ्चि॑तꣳ॒॒ सञ्चि॑त मे॒तैः । \newline
43. सञ्चि॑त॒मिति॒ सं - चि॒त॒म् । \newline
44. ए॒तै रुपोपै॒तै रे॒तै रुप॑ । \newline
45. उप॑ तिष्ठते तिष्ठत॒ उपोप॑ तिष्ठते । \newline
46. ति॒ष्ठ॒ते॒ न॒म॒स्का॒रैर् न॑मस्का॒रै स्ति॑ष्ठते तिष्ठते नमस्का॒रैः । \newline
47. न॒म॒स्का॒रै रे॒वैव न॑मस्का॒रैर् न॑मस्का॒रै रे॒व । \newline
48. न॒म॒स्का॒रैरिति॑ नमः - का॒रैः । \newline
49. ए॒वैन॑ मेन मे॒वै वैन᳚म् । \newline
50. ए॒नꣳ॒॒ श॒म॒य॒ति॒ श॒म॒य॒ त्ये॒न॒ मे॒नꣳ॒॒ श॒म॒य॒ति॒ । \newline
51. श॒म॒य॒ति॒ ये ये श॑मयति शमयति॒ ये । \newline
52. ये᳚ ऽग्नयो॒ ऽग्नयो॒ ये ये᳚ ऽग्नयः॑ । \newline
53. अ॒ग्नयः॑ पुरी॒ष्याः᳚ पुरी॒ष्या॑ अ॒ग्नयो॒ ऽग्नयः॑ पुरी॒ष्याः᳚ । \newline

\textbf{Ghana Paata } \newline

1. वातो॒ अन्वनु॒ वातो॒ वातो॒ अनु॑ वातु वा॒त्वनु॒ वातो॒ वातो॒ अनु॑ वातु । \newline
2. अनु॑ वातु वा॒त्वन् वनु॑ वातु ते ते वा॒त्वन् वनु॑ वातु ते । \newline
3. वा॒तु॒ ते॒ ते॒ वा॒तु॒ वा॒तु॒ ते॒ तस्मै॒ तस्मै॑ ते वातु वातु ते॒ तस्मै᳚ । \newline
4. ते॒ तस्मै॒ तस्मै॑ ते ते॒ तस्मै॑ ते ते॒ तस्मै॑ ते ते॒ तस्मै॑ ते । \newline
5. तस्मै॑ ते ते॒ तस्मै॒ तस्मै॑ ते रुद्र रुद्र ते॒ तस्मै॒ तस्मै॑ ते रुद्र । \newline
6. ते॒ रु॒द्र॒ रु॒द्र॒ ते॒ ते॒ रु॒द्रे॒ दु॒व॒थ्स॒रे णे॑दुवथ्स॒रेण॑ रुद्र ते ते रुद्रे दुवथ्स॒रेण॑ । \newline
7. रु॒द्रे॒ दु॒व॒थ्स॒रे णे॑दुवथ्स॒रेण॑ रुद्र रुद्रे दुवथ्स॒रेण॒ नमो॒ नम॑ इदुवथ्स॒रेण॑ रुद्र रुद्रे दुवथ्स॒रेण॒ नमः॑ । \newline
8. इ॒दु॒व॒थ्स॒रेण॒ नमो॒ नम॑ इदुवथ्स॒रे णे॑दुवथ्स॒रेण॒ नम॑ स्करोमि करोमि॒ नम॑ इदुवथ्स॒रे
णे॑दुवथ्स॒रेण॒ नम॑ स्करोमि । \newline
9. इ॒दु॒व॒थ्स॒रेणेती॑दु - व॒थ्स॒रेण॑ । \newline
10. नम॑ स्करोमि करोमि॒ नमो॒ नम॑ स्करोमि॒ यद् यत् क॑रोमि॒ नमो॒ नम॑ स्करोमि॒ यत् । \newline
11. क॒रो॒मि॒ यद् यत् क॑रोमि करोमि॒ यत् ते॑ ते॒ यत् क॑रोमि करोमि॒ यत् ते᳚ । \newline
12. यत् ते॑ ते॒ यद् यत् ते॑ रुद्र रुद्र ते॒ यद् यत् ते॑ रुद्र । \newline
13. ते॒ रु॒द्र॒ रु॒द्र॒ ते॒ ते॒ रु॒द्रो॒पर् यु॒परि॑ रुद्र ते ते रुद्रो॒परि॑ । \newline
14. रु॒द्रो॒ पर् यु॒परि॑ रुद्र रुद्रो॒ परि॒ धनु॒र् धनु॑ रु॒परि॑ रुद्र रुद्रो॒ परि॒ धनुः॑ । \newline
15. उ॒परि॒ धनु॒र् धनु॑ रु॒पर् यु॒परि॒ धनु॒ स्तत् तद् धनु॑ रु॒पर् यु॒परि॒ धनु॒ स्तत् । \newline
16. ढनु॒ स्तत् तद् धनु॒र् धनु॒ स्तद् वातो॒ वात॒ स्तद् धनु॒र् धनु॒ स्तद् वातः॑ । \newline
17. तद् वातो॒ वात॒ स्तत् तद् वातो॒ अन्वनु॒ वात॒ स्तत् तद् वातो॒ अनु॑ । \newline
18. वातो॒ अन्वनु॒ वातो॒ वातो॒ अनु॑ वातु वा॒त्वनु॒ वातो॒ वातो॒ अनु॑ वातु । \newline
19. अनु॑ वातु वा॒त्वन् वनु॑ वातु ते ते वा॒त्वन् वनु॑ वातु ते । \newline
20. वा॒तु॒ ते॒ ते॒ वा॒तु॒ वा॒तु॒ ते॒ तस्मै॒ तस्मै॑ ते वातु वातु ते॒ तस्मै᳚ । \newline
21. ते॒ तस्मै॒ तस्मै॑ ते ते॒ तस्मै॑ ते ते॒ तस्मै॑ ते ते॒ तस्मै॑ ते । \newline
22. तस्मै॑ ते ते॒ तस्मै॒ तस्मै॑ ते रुद्र रुद्र ते॒ तस्मै॒ तस्मै॑ ते रुद्र । \newline
23. ते॒ रु॒द्र॒ रु॒द्र॒ ते॒ ते॒ रु॒द्र॒ व॒थ्स॒रेण॑ वथ्स॒रेण॑ रुद्र ते ते रुद्र वथ्स॒रेण॑ । \newline
24. रु॒द्र॒ व॒थ्स॒रेण॑ वथ्स॒रेण॑ रुद्र रुद्र वथ्स॒रेण॒ नमो॒ नमो॑ वथ्स॒रेण॑ रुद्र रुद्र वथ्स॒रेण॒ नमः॑ । \newline
25. व॒थ्स॒रेण॒ नमो॒ नमो॑ वथ्स॒रेण॑ वथ्स॒रेण॒ नम॑ स्करोमि करोमि॒ नमो॑ वथ्स॒रेण॑ वथ्स॒रेण॒ नम॑ स्करोमि । \newline
26. नम॑ स्करोमि करोमि॒ नमो॒ नम॑ स्करोमि रु॒द्रो रु॒द्रः क॑रोमि॒ नमो॒ नम॑ स्करोमि रु॒द्रः । \newline
27. क॒रो॒मि॒ रु॒द्रो रु॒द्रः क॑रोमि करोमि रु॒द्रो वै वै रु॒द्रः क॑रोमि करोमि रु॒द्रो वै । \newline
28. रु॒द्रो वै वै रु॒द्रो रु॒द्रो वा ए॒ष ए॒ष वै रु॒द्रो रु॒द्रो वा ए॒षः । \newline
29. वा ए॒ष ए॒ष वै वा ए॒ष यद् यदे॒ष वै वा ए॒ष यत् । \newline
30. ए॒ष यद् यदे॒ष ए॒ष यद॒ग्नि र॒ग्निर् यदे॒ष ए॒ष यद॒ग्निः । \newline
31. यद॒ग्नि र॒ग्निर् यद् यद॒ग्निः स सो᳚ ऽग्निर् यद् यद॒ग्निः सः । \newline
32. अ॒ग्निः स सो᳚ ऽग्नि र॒ग्निः स यथा॒ यथा॒ सो᳚ ऽग्नि र॒ग्निः स यथा᳚ । \newline
33. स यथा॒ यथा॒ स स यथा᳚ व्या॒घ्रो व्या॒घ्रो यथा॒ स स यथा᳚ व्या॒घ्रः । \newline
34. यथा᳚ व्या॒घ्रो व्या॒घ्रो यथा॒ यथा᳚ व्या॒घ्रः क्रु॒द्धः क्रु॒द्धो व्या॒घ्रो यथा॒ यथा᳚ व्या॒घ्रः क्रु॒द्धः । \newline
35. व्या॒घ्रः क्रु॒द्धः क्रु॒द्धो व्या॒घ्रो व्या॒घ्रः क्रु॒द्ध स्तिष्ठ॑ति॒ तिष्ठ॑ति क्रु॒द्धो व्या॒घ्रो व्या॒घ्रः क्रु॒द्ध स्तिष्ठ॑ति । \newline
36. क्रु॒द्ध स्तिष्ठ॑ति॒ तिष्ठ॑ति क्रु॒द्धः क्रु॒द्ध स्तिष्ठ॑ त्ये॒व मे॒वम् तिष्ठ॑ति क्रु॒द्धः क्रु॒द्ध स्तिष्ठ॑ त्ये॒वम् । \newline
37. तिष्ठ॑ त्ये॒व मे॒वम् तिष्ठ॑ति॒ तिष्ठ॑ त्ये॒वं ॅवै वा ए॒वम् तिष्ठ॑ति॒ तिष्ठ॑ त्ये॒वं ॅवै । \newline
38. ए॒वं ॅवै वा ए॒व मे॒वं ॅवा ए॒ष ए॒ष वा ए॒व मे॒वं ॅवा ए॒षः । \newline
39. वा ए॒ष ए॒ष वै वा ए॒ष ए॒तर् ह्ये॒तर् ह्ये॒ष वै वा ए॒ष ए॒तर्.हि॑ । \newline
40. ए॒ष ए॒तर् ह्ये॒तर् ह्ये॒ष ए॒ष ए॒तर्.हि॒ सञ्चि॑तꣳ॒॒ सञ्चि॑त मे॒तर् ह्ये॒ष ए॒ष ए॒तर्.हि॒ सञ्चि॑तम् । \newline
41. ए॒तर्.हि॒ सञ्चि॑तꣳ॒॒ सञ्चि॑त मे॒तर् ह्ये॒तर्.हि॒ सञ्चि॑त मे॒तै रे॒तैः सञ्चि॑त मे॒तर् ह्ये॒तर्.हि॒ सञ्चि॑त मे॒तैः । \newline
42. सञ्चि॑त मे॒तै रे॒तैः सञ्चि॑तꣳ॒॒ सञ्चि॑त मे॒तै रुपोपै॒तैः सञ्चि॑तꣳ॒॒ सञ्चि॑त मे॒तै रुप॑ । \newline
43. सञ्चि॑त॒मिति॒ सं - चि॒त॒म् । \newline
44. ए॒तै रुपोपै॒तै रे॒तै रुप॑ तिष्ठते तिष्ठत॒ उपै॒तै रे॒तै रुप॑ तिष्ठते । \newline
45. उप॑ तिष्ठते तिष्ठत॒ उपोप॑ तिष्ठते नमस्का॒रैर् न॑मस्का॒रै स्ति॑ष्ठत॒ उपोप॑ तिष्ठते नमस्का॒रैः । \newline
46. ति॒ष्ठ॒ते॒ न॒म॒स्का॒रैर् न॑मस्का॒रै स्ति॑ष्ठते तिष्ठते नमस्का॒रै रे॒वैव न॑मस्का॒रै स्ति॑ष्ठते तिष्ठते नमस्का॒रै रे॒व । \newline
47. न॒म॒स्का॒रै रे॒वैव न॑मस्का॒रैर् न॑मस्का॒रै रे॒वैन॑ मेन मे॒व न॑मस्का॒रैर् न॑मस्का॒रै रे॒वैन᳚म् । \newline
48. न॒म॒स्का॒रैरिति॑ नमः - का॒रैः । \newline
49. ए॒वैन॑ मेन मे॒वै वैनꣳ॑ शमयति शमय त्येन मे॒वै वैनꣳ॑ शमयति । \newline
50. ए॒नꣳ॒॒ श॒म॒य॒ति॒ श॒म॒य॒ त्ये॒न॒ मे॒नꣳ॒॒ श॒म॒य॒ति॒ ये ये श॑मय त्येन मेनꣳ शमयति॒ ये । \newline
51. श॒म॒य॒ति॒ ये ये श॑मयति शमयति॒ ये᳚ ऽग्नयो॒ ऽग्नयो॒ ये श॑मयति शमयति॒ ये᳚ ऽग्नयः॑ । \newline
52. ये᳚ ऽग्नयो॒ ऽग्नयो॒ ये ये᳚ ऽग्नयः॑ पुरी॒ष्याः᳚ पुरी॒ष्या॑ अ॒ग्नयो॒ ये ये᳚ ऽग्नयः॑ पुरी॒ष्याः᳚ । \newline
53. अ॒ग्नयः॑ पुरी॒ष्याः᳚ पुरी॒ष्या॑ अ॒ग्नयो॒ ऽग्नयः॑ पुरी॒ष्याः᳚ प्रवि॑ष्टाः॒ प्रवि॑ष्टाः पुरी॒ष्या॑ अ॒ग्नयो॒ ऽग्नयः॑ पुरी॒ष्याः᳚ प्रवि॑ष्टाः । \newline
\pagebreak
\markright{ TS 5.5.7.5  \hfill https://www.vedavms.in \hfill}

\section{ TS 5.5.7.5 }

\textbf{TS 5.5.7.5 } \newline
\textbf{Samhita Paata} \newline

पुरी॒ष्याः᳚ प्रवि॑ष्टाः पृथि॒वीमनु॑ । तेषां॒ त्वम॑स्युत्त॒मः प्रणो॑ जी॒वात॑वे सुव ॥ आपं॑ त्वाऽग्ने॒ मन॒सा ऽऽपं॑ त्वाऽग्ने॒ तप॒सा ऽऽपं॑ त्वाऽग्ने दी॒क्षया ऽऽपं॑ त्वाऽग्न उप॒सद्भि॒रापं॑ त्वाऽग्ने सु॒त्ययाऽऽपं॑ त्वाऽग्ने॒ दक्षि॑णाभि॒रापं॑ त्वाऽग्ने ऽवभृ॒थेनापं॑ त्वाऽग्ने व॒शया ऽऽपं॑ त्वाऽग्ने स्वगाका॒रेणेत्या॑है॒ ( ) षा वा अ॒ग्नेराप्ति॒स्तयै॒वैन॑माप्नोति ॥ \newline

\textbf{Pada Paata} \newline

पु॒री॒ष्याः᳚ । प्रवि॑ष्टा॒ इति॒ प्र - वि॒ष्टाः॒ । पृ॒थि॒वीम् । अनु॑ ॥ तेषा᳚म् । त्वम् । अ॒सि॒ । उ॒त्त॒म इत्यु॑त् - त॒मः । प्रेति॑ । नः॒ । जी॒वात॑वे । सु॒व॒ ॥ आप᳚म् । त्वा॒ । अ॒ग्ने॒ । मन॑सा । आप᳚म् । त्वा॒ । अ॒ग्ने॒ । तप॑सा । आप᳚म् । त्वा॒ । अ॒ग्ने॒ । दी॒क्षया᳚ । आप᳚म् । त्वा॒ । अ॒ग्ने॒ । उ॒प॒सद्भि॒रित्यु॑प॒सत् - भिः॒ । आप᳚म् । त्वा॒ । अ॒ग्ने॒ । सु॒त्यया᳚ । आप᳚म् । त्वा॒ । अ॒ग्ने॒ । दक्षि॑णाभिः । आप᳚म् । त्वा॒ । अ॒ग्ने॒ । अ॒व॒भृ॒थेनेत्य॑व - भृ॒थेन॑ । आप᳚म् । त्वा॒ । अ॒ग्ने॒ । व॒शया᳚ । आप᳚म् । त्वा॒ । अ॒ग्ने॒ । स्व॒गा॒का॒रेणेति॑ स्वगा - का॒रेण॑ । इति॑ । आ॒ह॒ ( ) । ए॒षा । वै । अ॒ग्नेः । आप्तिः॑ । तया᳚ । ए॒व । ए॒न॒म् । आ॒प्नो॒ति॒ ॥  \newline


\textbf{Krama Paata} \newline

पु॒री॒ष्याः᳚ प्रवि॑ष्टाः । प्रवि॑ष्टाः पृथि॒वीम् । प्रवि॑ष्टा॒ इति॒ प्र - वि॒ष्टाः॒ । पृ॒थि॒वीमनु॑ । अन्वित्यनु॑ ॥ तेषा॒म् त्वम् । त्वम॑सि । अ॒स्यु॒त्त॒मः । उ॒त्त॒मः प्र । उ॒त्त॒म इत्यु॑त् - त॒मः । प्र णः॑ । नो॒ जी॒वात॑वे । जी॒वात॑वे सुव । सु॒वेति॑ सुव ॥ आप॑म् त्वा । त्वा॒ऽग्ने॒ । अ॒ग्ने॒ मन॑सा । मन॒साऽऽप᳚म् । आप॑म् त्वा । त्वा॒ऽग्ने॒ । अ॒ग्ने॒ तप॑सा । तप॒साऽऽप᳚म् । आप॑म् त्वा । त्वा॒ऽग्ने॒ । अ॒ग्ने॒ दी॒क्षया᳚ । दी॒क्षयाऽऽप᳚म् । आप॑म् त्वा । त्वा॒ऽग्ने॒ । अ॒ग्न॒ उ॒प॒सद्भिः॑ । उ॒प॒सद्भि॒राप᳚म् । उ॒प॒सद्भि॒रित्यु॑प॒सत् - भिः॒ । आप॑म् त्वा । त्वा॒ऽग्ने॒ । अ॒ग्ने॒ सु॒त्यया᳚ । सु॒त्ययाऽऽप᳚म् । आप॑म् त्वा । त्वा॒ऽग्ने॒ । अ॒ग्ने॒ दक्षि॑णाभिः । दक्षि॑णाभि॒राप᳚म् । आप॑म् त्वा । त्वा॒ऽग्ने॒ । अ॒ग्ने॒ऽव॒भृ॒थेन॑ । अ॒व॒भृ॒थेनाऽऽप᳚म् । अ॒व॒भृ॒थेनेत्य॑व - भृ॒थेन॑ । आप॑म् त्वा । त्वा॒ऽग्ने॒ । अ॒ग्ने॒ व॒शया᳚ । व॒शयाऽऽप᳚म् । आप॑म् त्वा । त्वा॒ऽग्ने॒ । अ॒ग्ने॒ स्व॒गा॒का॒रेण॑ । स्व॒गा॒का॒रेणेति॑ । स्व॒गा॒का॒रेणेति॑ स्वगा - का॒रेण॑ । इत्या॑ह ( ) । आ॒है॒षा । ए॒षा वै । वा अ॒ग्नेः । अ॒ग्नेराप्तिः॑ । आ॒प्तिस्तया᳚ । तयै॒व । ए॒वैन᳚म् । ए॒न॒मा॒प्नो॒ति॒ । आ॒प्नो॒तीत्या᳚प्नोति । \newline

\textbf{Jatai Paata} \newline

1. पु॒री॒ष्याः᳚ प्रवि॑ष्टाः॒ प्रवि॑ष्टाः पुरी॒ष्याः᳚ पुरी॒ष्याः᳚ प्रवि॑ष्टाः । \newline
2. प्रवि॑ष्टाः पृथि॒वीम् पृ॑थि॒वीम् प्रवि॑ष्टाः॒ प्रवि॑ष्टाः पृथि॒वीम् । \newline
3. प्रवि॑ष्टा॒ इति॒ प्र - वि॒ष्टाः॒ । \newline
4. पृ॒थि॒वी मन्वनु॑ पृथि॒वीम् पृ॑थि॒वी मनु॑ । \newline
5. अन्वित्यनु॑ । \newline
6. तेषा॒म् त्वम् त्वम् तेषा॒म् तेषा॒म् त्वम् । \newline
7. त्व म॑स्यसि॒ त्वम् त्व म॑सि । \newline
8. अ॒स्यु॒त्त॒म उ॑त्त॒मो᳚ ऽस्य स्युत्त॒मः । \newline
9. उ॒त्त॒मः प्र प्रोत्त॒म उ॑त्त॒मः प्र । \newline
10. उ॒त्त॒म इत्यु॑त् - त॒मः । \newline
11. प्र णो॑ नः॒ प्र प्र णः॑ । \newline
12. नो॒ जी॒वात॑वे जी॒वात॑वे नो नो जी॒वात॑वे । \newline
13. जी॒वात॑वे सुव सुव जी॒वात॑वे जी॒वात॑वे सुव । \newline
14. सु॒वेति॑ सुव । \newline
15. आप॑म् त्वा॒ त्वा ऽऽप॒ माप॑म् त्वा । \newline
16. त्वा॒ ऽग्ने॒ ऽग्ने॒ त्वा॒ त्वा॒ ऽग्ने॒ । \newline
17. अ॒ग्ने॒ मन॑सा॒ मन॑सा ऽग्ने ऽग्ने॒ मन॑सा । \newline
18. मन॒सा ऽऽप॒ माप॒म् मन॑सा॒ मन॒सा ऽऽप᳚म् । \newline
19. आप॑म् त्वा॒ त्वा ऽऽप॒ माप॑म् त्वा । \newline
20. त्वा॒ ऽग्ने॒ ऽग्ने॒ त्वा॒ त्वा॒ ऽग्ने॒ । \newline
21. अ॒ग्ने॒ तप॑सा॒ तप॑सा ऽग्ने ऽग्ने॒ तप॑सा । \newline
22. तप॒सा ऽऽप॒ माप॒म् तप॑सा॒ तप॒सा ऽऽप᳚म् । \newline
23. आप॑म् त्वा॒ त्वा ऽऽप॒ माप॑म् त्वा । \newline
24. त्वा॒ ऽग्ने॒ ऽग्ने॒ त्वा॒ त्वा॒ ऽग्ने॒ । \newline
25. अ॒ग्ने॒ दी॒क्षया॑ दी॒क्षया᳚ ऽग्ने ऽग्ने दी॒क्षया᳚ । \newline
26. दी॒क्षया ऽऽप॒ माप॑म् दी॒क्षया॑ दी॒क्षया ऽऽप᳚म् । \newline
27. आप॑म् त्वा॒ त्वा ऽऽप॒ माप॑म् त्वा । \newline
28. त्वा॒ ऽग्ने॒ ऽग्ने॒ त्वा॒ त्वा॒ ऽग्ने॒ । \newline
29. अ॒ग्न॒ उ॒प॒सद्भि॑ रुप॒सद्भि॑ रग्ने ऽग्न उप॒सद्भिः॑ । \newline
30. उ॒प॒सद्भि॒ राप॒ माप॑ मुप॒सद्भि॑ रुप॒सद्भि॒ राप᳚म् । \newline
31. उ॒प॒सद्भि॒रित्यु॑प॒सत् - भिः॒ । \newline
32. आप॑म् त्वा॒ त्वा ऽऽप॒ माप॑म् त्वा । \newline
33. त्वा॒ ऽग्ने॒ ऽग्ने॒ त्वा॒ त्वा॒ ऽग्ने॒ । \newline
34. अ॒ग्ने॒ सु॒त्यया॑ सु॒त्यया᳚ ऽग्ने ऽग्ने सु॒त्यया᳚ । \newline
35. सु॒त्यया ऽऽप॒ मापꣳ॑ सु॒त्यया॑ सु॒त्यया ऽऽप᳚म् । \newline
36. आप॑म् त्वा॒ त्वा ऽऽप॒ माप॑म् त्वा । \newline
37. त्वा॒ ऽग्ने॒ ऽग्ने॒ त्वा॒ त्वा॒ ऽग्ने॒ । \newline
38. अ॒ग्ने॒ दक्षि॑णाभि॒र् दक्षि॑णाभि रग्ने ऽग्ने॒ दक्षि॑णाभिः । \newline
39. दक्षि॑णाभि॒ राप॒ माप॒म् दक्षि॑णाभि॒र् दक्षि॑णाभि॒ राप᳚म् । \newline
40. आप॑म् त्वा॒ त्वा ऽऽप॒ माप॑म् त्वा । \newline
41. त्वा॒ ऽग्ने॒ ऽग्ने॒ त्वा॒ त्वा॒ ऽग्ने॒ । \newline
42. अ॒ग्ने॒ ऽव॒भृ॒थेना॑ वभृ॒थेना᳚ग्ने ऽग्ने ऽवभृ॒थेन॑ । \newline
43. अ॒व॒भृ॒थेनाप॒ माप॑ मवभृ॒थेना॑ वभृ॒थेनाप᳚म् । \newline
44. अ॒व॒भृ॒थेनेत्य॑व - भृ॒थेन॑ । \newline
45. आप॑म् त्वा॒ त्वा ऽऽप॒ माप॑म् त्वा । \newline
46. त्वा॒ ऽग्ने॒ ऽग्ने॒ त्वा॒ त्वा॒ ऽग्ने॒ । \newline
47. अ॒ग्ने॒ व॒शया॑ व॒शया᳚ ऽग्ने ऽग्ने व॒शया᳚ । \newline
48. व॒शया ऽऽप॒ मापं॑ ॅव॒शया॑ व॒शया ऽऽप᳚म् । \newline
49. आप॑म् त्वा॒ त्वा ऽऽप॒ माप॑म् त्वा । \newline
50. त्वा॒ ऽग्ने॒ ऽग्ने॒ त्वा॒ त्वा॒ ऽग्ने॒ । \newline
51. अ॒ग्ने॒ स्व॒गा॒का॒रेण॑ स्वगाका॒रेणा᳚ग्ने ऽग्ने स्वगाका॒रेण॑ । \newline
52. स्व॒गा॒का॒रेणे तीति॑ स्वगाका॒रेण॑ स्वगाका॒रेणेति॑ । \newline
53. स्व॒गा॒का॒रेणेति॑ स्वगा - का॒रेण॑ । \newline
54. इत्या॑हा॒हे तीत्या॑ह । \newline
55. आ॒है॒षैषा ऽऽहा॑है॒षा । \newline
56. ए॒षा वै वा ए॒षैषा वै । \newline
57. वा अ॒ग्ने र॒ग्नेर् वै वा अ॒ग्नेः । \newline
58. आ॒ग्ने राप्ति॒ राप्ति॑ र॒ग्ने र॒ग्ने राप्तिः॑ । \newline
59. आप्ति॒ स्तया॒ तया ऽऽप्ति॒ राप्ति॒ स्तया᳚ । \newline
60. तयै॒ वैव तया॒ तयै॒व । \newline
61. ए॒वैन॑ मेन मे॒वै वैन᳚म् । \newline
62. ए॒न॒ मा॒प्नो॒ त्या॒प्नो॒ त्ये॒न॒ मे॒न॒ मा॒प्नो॒ति॒ । \newline
63. आ॒प्नो॒तीत्या᳚प्नोति । \newline

\textbf{Ghana Paata } \newline

1. पु॒री॒ष्याः᳚ प्रवि॑ष्टाः॒ प्रवि॑ष्टाः पुरी॒ष्याः᳚ पुरी॒ष्याः᳚ प्रवि॑ष्टाः पृथि॒वीम् पृ॑थि॒वीम् प्रवि॑ष्टाः पुरी॒ष्याः᳚ पुरी॒ष्याः᳚ प्रवि॑ष्टाः पृथि॒वीम् । \newline
2. प्रवि॑ष्टाः पृथि॒वीम् पृ॑थि॒वीम् प्रवि॑ष्टाः॒ प्रवि॑ष्टाः पृथि॒वी मन्वनु॑ पृथि॒वीम् प्रवि॑ष्टाः॒ प्रवि॑ष्टाः पृथि॒वी मनु॑ । \newline
3. प्रवि॑ष्टा॒ इति॒ प्र - वि॒ष्टाः॒ । \newline
4. पृ॒थि॒वी मन्वनु॑ पृथि॒वीम् पृ॑थि॒वी मनु॑ । \newline
5. अन्वित्यनु॑ । \newline
6. तेषा॒म् त्वम् त्वम् तेषा॒म् तेषा॒म् त्व म॑स्यसि॒ त्वम् तेषा॒म् तेषा॒म् त्व म॑सि । \newline
7. त्व म॑स्यसि॒ त्वम् त्व म॑स्युत्त॒म उ॑त्त॒मो॑ ऽसि॒ त्वम् त्व म॑स्युत्त॒मः । \newline
8. अ॒स्यु॒त्त॒म उ॑त्त॒मो᳚ ऽस्य स्युत्त॒मः प्र प्रोत्त॒मो᳚ ऽस्य स्युत्त॒मः प्र । \newline
9. उ॒त्त॒मः प्र प्रोत्त॒म उ॑त्त॒मः प्र णो॑ नः॒ प्रोत्त॒म उ॑त्त॒मः प्र णः॑ । \newline
10. उ॒त्त॒म इत्यु॑त् - त॒मः । \newline
11. प्र णो॑ नः॒ प्र प्र णो॑ जी॒वात॑वे जी॒वात॑वे नः॒ प्र प्र णो॑ जी॒वात॑वे । \newline
12. नो॒ जी॒वात॑वे जी॒वात॑वे नो नो जी॒वात॑वे सुव सुव जी॒वात॑वे नो नो जी॒वात॑वे सुव । \newline
13. जी॒वात॑वे सुव सुव जी॒वात॑वे जी॒वात॑वे सुव । \newline
14. सु॒वेति॑ सुव । \newline
15. आप॑म् त्वा॒ त्वा ऽऽप॒ माप॑म् त्वा ऽग्ने ऽग्ने॒ त्वा ऽऽप॒ माप॑म् त्वा ऽग्ने । \newline
16. त्वा॒ ऽग्ने॒ ऽग्ने॒ त्वा॒ त्वा॒ ऽग्ने॒ मन॑सा॒ मन॑सा ऽग्ने त्वा त्वा ऽग्ने॒ मन॑सा । \newline
17. अ॒ग्ने॒ मन॑सा॒ मन॑सा ऽग्ने ऽग्ने॒ मन॒सा ऽऽप॒ माप॒म् मन॑सा ऽग्ने ऽग्ने॒ मन॒सा ऽऽप᳚म् । \newline
18. मन॒सा ऽऽप॒ माप॒म् मन॑सा॒ मन॒सा ऽऽप॑म् त्वा॒ त्वा ऽऽप॒म् मन॑सा॒ मन॒सा ऽऽप॑म् त्वा । \newline
19. आप॑म् त्वा॒ त्वा ऽऽप॒ माप॑म् त्वा ऽग्ने ऽग्ने॒ त्वा ऽऽप॒ माप॑म् त्वा ऽग्ने । \newline
20. त्वा॒ ऽग्ने॒ ऽग्ने॒ त्वा॒ त्वा॒ ऽग्ने॒ तप॑सा॒ तप॑सा ऽग्ने त्वा त्वा ऽग्ने॒ तप॑सा । \newline
21. अ॒ग्ने॒ तप॑सा॒ तप॑सा ऽग्ने ऽग्ने॒ तप॒सा ऽऽप॒ माप॒म् तप॑सा ऽग्ने ऽग्ने॒ तप॒सा ऽऽप᳚म् । \newline
22. तप॒सा ऽऽप॒ माप॒म् तप॑सा॒ तप॒सा ऽऽप॑म् त्वा॒ त्वा ऽऽप॒म् तप॑सा॒ तप॒सा ऽऽप॑म् त्वा । \newline
23. आप॑म् त्वा॒ त्वा ऽऽप॒ माप॑म् त्वा ऽग्ने ऽग्ने॒ त्वा ऽऽप॒ माप॑म् त्वा ऽग्ने । \newline
24. त्वा॒ ऽग्ने॒ ऽग्ने॒ त्वा॒ त्वा॒ ऽग्ने॒ दी॒क्षया॑ दी॒क्षया᳚ ऽग्ने त्वा त्वा ऽग्ने दी॒क्षया᳚ । \newline
25. अ॒ग्ने॒ दी॒क्षया॑ दी॒क्षया᳚ ऽग्ने ऽग्ने दी॒क्षया ऽऽप॒ माप॑म् दी॒क्षया᳚ ऽग्ने ऽग्ने दी॒क्षया ऽऽप᳚म् । \newline
26. दी॒क्षया ऽऽप॒ माप॑म् दी॒क्षया॑ दी॒क्षया ऽऽप॑म् त्वा॒ त्वा ऽऽप॑म् दी॒क्षया॑ दी॒क्षया ऽऽप॑म् त्वा । \newline
27. आप॑म् त्वा॒ त्वा ऽऽप॒ माप॑म् त्वा ऽग्ने ऽग्ने॒ त्वा ऽऽप॒ माप॑म् त्वा ऽग्ने । \newline
28. त्वा॒ ऽग्ने॒ ऽग्ने॒ त्वा॒ त्वा॒ ऽग्न॒ उ॒प॒सद्भि॑ रुप॒सद्भि॑ रग्ने त्वा त्वा ऽग्न उप॒सद्भिः॑ । \newline
29. अ॒ग्न॒ उ॒प॒सद्भि॑ रुप॒सद्भि॑ रग्ने ऽग्न उप॒सद्भि॒ राप॒ माप॑ मुप॒सद्भि॑ रग्ने ऽग्न उप॒सद्भि॒ राप᳚म् । \newline
30. उ॒प॒सद्भि॒ राप॒ माप॑ मुप॒सद्भि॑ रुप॒सद्भि॒ राप॑म् त्वा॒ त्वा ऽऽप॑ मुप॒सद्भि॑ रुप॒सद्भि॒ राप॑म् त्वा । \newline
31. उ॒प॒सद्भि॒रित्यु॑प॒सत् - भिः॒ । \newline
32. आप॑म् त्वा॒ त्वा ऽऽप॒ माप॑म् त्वा ऽग्ने ऽग्ने॒ त्वा ऽऽप॒ माप॑म् त्वा ऽग्ने । \newline
33. त्वा॒ ऽग्ने॒ ऽग्ने॒ त्वा॒ त्वा॒ ऽग्ने॒ सु॒त्यया॑ सु॒त्यया᳚ ऽग्ने त्वा त्वा ऽग्ने सु॒त्यया᳚ । \newline
34. अ॒ग्ने॒ सु॒त्यया॑ सु॒त्यया᳚ ऽग्ने ऽग्ने सु॒त्यया ऽऽप॒ मापꣳ॑ सु॒त्यया᳚ ऽग्ने ऽग्ने सु॒त्यया ऽऽप᳚म् । \newline
35. सु॒त्यया ऽऽप॒ मापꣳ॑ सु॒त्यया॑ सु॒त्यया ऽऽप॑म् त्वा॒ त्वा ऽऽपꣳ॑ सु॒त्यया॑ सु॒त्यया ऽऽप॑म् त्वा । \newline
36. आप॑म् त्वा॒ त्वा ऽऽप॒ माप॑म् त्वा ऽग्ने ऽग्ने॒ त्वा ऽऽप॒ माप॑म् त्वा ऽग्ने । \newline
37. त्वा॒ ऽग्ने॒ ऽग्ने॒ त्वा॒ त्वा॒ ऽग्ने॒ दक्षि॑णाभि॒र् दक्षि॑णाभि रग्ने त्वा त्वा ऽग्ने॒ दक्षि॑णाभिः । \newline
38. अ॒ग्ने॒ दक्षि॑णाभि॒र् दक्षि॑णाभि रग्ने ऽग्ने॒ दक्षि॑णाभि॒ राप॒ माप॒म् दक्षि॑णाभि रग्ने ऽग्ने॒ दक्षि॑णाभि॒ राप᳚म् । \newline
39. दक्षि॑णाभि॒ राप॒ माप॒म् दक्षि॑णाभि॒र् दक्षि॑णाभि॒ राप॑म् त्वा॒ त्वा ऽऽप॒म् दक्षि॑णाभि॒र् दक्षि॑णाभि॒ राप॑म् त्वा । \newline
40. आप॑म् त्वा॒ त्वा ऽऽप॒ माप॑म् त्वा ऽग्ने ऽग्ने॒ त्वा ऽऽप॒ माप॑म् त्वा ऽग्ने । \newline
41. त्वा॒ ऽग्ने॒ ऽग्ने॒ त्वा॒ त्वा॒ ऽग्ने॒ ऽव॒भृ॒थेना॑ वभृ॒थेना᳚ग्ने त्वा त्वा ऽग्ने ऽवभृ॒थेन॑ । \newline
42. अ॒ग्ने॒ ऽव॒भृ॒थेना॑ वभृ॒थेना᳚ग्ने ऽग्ने ऽवभृ॒थेनाप॒ माप॑ मवभृ॒थेना᳚ग्ने ऽग्ने ऽवभृ॒थेनाप᳚म् । \newline
43. अ॒व॒भृ॒थेनाप॒ माप॑ मवभृ॒थेना॑ वभृ॒थेनाप॑म् त्वा॒ त्वा ऽऽप॑ मवभृ॒थेना॑ वभृ॒थेनाप॑म् त्वा । \newline
44. अ॒व॒भृ॒थेनेत्य॑व - भृ॒थेन॑ । \newline
45. आप॑म् त्वा॒ त्वा ऽऽप॒ माप॑म् त्वा ऽग्ने ऽग्ने॒ त्वा ऽऽप॒ माप॑म् त्वा ऽग्ने । \newline
46. त्वा॒ ऽग्ने॒ ऽग्ने॒ त्वा॒ त्वा॒ ऽग्ने॒ व॒शया॑ व॒शया᳚ ऽग्ने त्वा त्वा ऽग्ने व॒शया᳚ । \newline
47. अ॒ग्ने॒ व॒शया॑ व॒शया᳚ ऽग्ने ऽग्ने व॒शया ऽऽप॒ मापं॑ ॅव॒शया᳚ ऽग्ने ऽग्ने व॒शया ऽऽप᳚म् । \newline
48. व॒शया ऽऽप॒ मापं॑ ॅव॒शया॑ व॒शया ऽऽप॑म् त्वा॒ त्वा ऽऽपं॑ ॅव॒शया॑ व॒शया ऽऽप॑म् त्वा । \newline
49. आप॑म् त्वा॒ त्वा ऽऽप॒ माप॑म् त्वा ऽग्ने ऽग्ने॒ त्वा ऽऽप॒ माप॑म् त्वा ऽग्ने । \newline
50. त्वा॒ ऽग्ने॒ ऽग्ने॒ त्वा॒ त्वा॒ ऽग्ने॒ स्व॒गा॒का॒रेण॑ स्वगाका॒रेणा᳚ग्ने त्वा त्वा ऽग्ने स्वगाका॒रेण॑ । \newline
51. अ॒ग्ने॒ स्व॒गा॒का॒रेण॑ स्वगाका॒रेणा᳚ग्ने ऽग्ने स्वगाका॒रेणे तीति॑ स्वगाका॒रेणा᳚ग्ने ऽग्ने स्वगाका॒रेणेति॑ । \newline
52. स्व॒गा॒का॒रे णेतीति॑ स्वगाका॒रेण॑ स्वगाका॒रेणे त्या॑हा॒हेति॑ स्वगाका॒रेण॑ स्वगाका॒रे णेत्या॑ह । \newline
53. स्व॒गा॒का॒रेणेति॑ स्वगा - का॒रेण॑ । \newline
54. इत्या॑हा॒हे तीत्या॑है॒ षैषा ऽऽहे तीत्या॑ है॒षा । \newline
55. आ॒है॒ षैषा ऽऽहा॑है॒षा वै वा ए॒षा ऽऽहा॑है॒षा वै । \newline
56. ए॒षा वै वा ए॒षैषा वा अ॒ग्ने र॒ग्नेर् वा ए॒षैषा वा अ॒ग्नेः । \newline
57. वा अ॒ग्ने र॒ग्नेर् वै वा अ॒ग्ने राप्ति॒ राप्ति॑ र॒ग्नेर् वै वा अ॒ग्ने राप्तिः॑ । \newline
58. अ॒ग्ने राप्ति॒ राप्ति॑ र॒ग्ने र॒ग्ने राप्ति॒ स्तया॒ तया ऽऽप्ति॑ र॒ग्ने र॒ग्ने राप्ति॒ स्तया᳚ । \newline
59. आप्ति॒ स्तया॒ तया ऽऽप्ति॒ राप्ति॒ स्तयै॒ वैव तया ऽऽप्ति॒ राप्ति॒ स्तयै॒व । \newline
60. तयै॒ वैव तया॒ तयै॒ वैन॑ मेन मे॒व तया॒ तयै॒ वैन᳚म् । \newline
61. ए॒वैन॑ मेन मे॒वै वैन॑ माप्नो त्याप्नो त्येन मे॒वै वैन॑ माप्नोति । \newline
62. ए॒न॒ मा॒प्नो॒ त्या॒प्नो॒  त्ये॒न॒ मे॒न॒ मा॒प्नो॒ति॒ । \newline
63. आ॒प्नो॒तीत्या᳚प्नोति । \newline
\pagebreak
\markright{ TS 5.5.8.1  \hfill https://www.vedavms.in \hfill}

\section{ TS 5.5.8.1 }

\textbf{TS 5.5.8.1 } \newline
\textbf{Samhita Paata} \newline

गा॒य॒त्रेण॑ पु॒रस्ता॒दुप॑ तिष्ठते प्रा॒णमे॒वास्मि॑न् दधाति बृहद्-रथंत॒राभ्यां᳚ प॒क्षावोज॑ ए॒वास्मि॑न् दधात्यृतु॒स्थाय॑ज्ञा-य॒ज्ञिये॑न॒ पुच्छ॑मृ॒तुष्वे॒व प्रति॑ तिष्ठति पृ॒ष्ठैरुप॑ तिष्ठते॒ तेजो॒ वै पृ॒ष्ठानि॒ तेज॑ ए॒वास्मि॑न् दधाति प्र॒जाप॑तिर॒ग्निम॑सृजत॒ सो᳚ऽस्माथ् सृ॒ष्टः परा॑ङै॒त् तं ॅवा॑रव॒न्तीये॑ना-वारयत॒ तद्-वा॑रव॒न्तीय॑स्य वारवन्तीय॒त्वꣳ श्यै॒तेन॑ श्ये॒ती अ॑कुरुत॒ तच्छ्यै॒तस्य॑ श्यैत॒त्वं - [  ] \newline

\textbf{Pada Paata} \newline

गा॒य॒त्रेण॑ । पु॒रस्ता᳚त् । उपेति॑ । ति॒ष्ठ॒ते॒ । प्रा॒णमिति॑ प्र - अ॒नम् । ए॒व । अ॒स्मि॒न्न् । द॒धा॒ति॒ । बृ॒ह॒द्र॒थ॒न्त॒राभ्या॒मिति॑ बृहत् - र॒थ॒न्त॒राभ्या᳚म् । प॒क्षौ । ओजः॑ । ए॒व । अ॒स्मि॒न्न् । द॒धा॒ति॒ । ऋ॒तु॒स्थाय॑ज्ञाय॒ज्ञिये॑न । पुच्छ᳚म् । ऋ॒तुषु॑ । ए॒व । प्रतीति॑ । ति॒ष्ठ॒ति॒ । पृ॒ष्ठैः । उपेति॑ । ति॒ष्ठ॒ते॒ । तेजः॑ । वै । पृ॒ष्ठानि॑ । तेजः॑ । ए॒व । अ॒स्मि॒न्न् । द॒धा॒ति॒ । प्र॒जाप॑ति॒रिति॑ प्र॒जा - प॒तिः॒ । अ॒ग्निम् । अ॒सृ॒ज॒त॒ । सः । अ॒स्मा॒त् । सृ॒ष्टः । पराङ्॑ । ऐ॒त् । तम् । वा॒र॒व॒न्तीये॒नेति॑ वार - व॒न्तीये॑न । अ॒वा॒र॒य॒त॒ । तत् । वा॒र॒व॒न्तीय॒स्येति॑ वार - व॒न्तीय॑स्य । वा॒र॒व॒न्ती॒य॒त्वमिति॑ वारवन्तीय - त्वम् । श्यै॒तेन॑ । श्ये॒ती । अ॒कु॒रु॒त॒ । तत् । श्यै॒तस्य॑ । श्यै॒त॒त्वमिति॑ श्यैत - त्वम् ।  \newline


\textbf{Krama Paata} \newline

गा॒य॒त्रेण॑ पु॒रस्ता᳚त् । पु॒रस्ता॒दुप॑ । उप॑ तिष्ठते । ति॒ष्ठ॒ते॒ प्रा॒णम् । प्रा॒णमे॒व । प्रा॒णमिति॑ प्र - अ॒नम् । ए॒वास्मिन्न्॑ । अ॒स्मि॒न् द॒धा॒ति॒ । द॒धा॒ति॒ बृ॒ह॒द्र॒थ॒न्त॒राभ्या᳚म् । बृ॒ह॒द्र॒थ॒न्त॒राभ्या᳚म् प॒क्षौ । बृ॒ह॒द्र॒न्त॒राभ्या॒मिति॑ बृहत् - र॒थ॒न्त॒राभ्या᳚म् । प॒क्षावोजः॑ । ओज॑ ए॒व । ए॒वास्मिन्न्॑ । अ॒स्मि॒न् द॒धा॒ति॒ । द॒धा॒त्यृ॒तु॒स्थाय॑ज्ञाय॒ज्ञिये॑न । ऋ॒तु॒स्थाय॑ज्ञाय॒ज्ञिये॑न॒ पुच्छ᳚म् । पुच्छ॑मृ॒तुषु॑ । ऋ॒तुष्वे॒व । ए॒व प्रति॑ । प्रति॑ तिष्ठति । ति॒ष्ठ॒ति॒ पृ॒ष्ठैः । पृ॒ष्ठैरुप॑ । उप॑ तिष्ठते । ति॒ष्ठ॒ते॒ तेजः॑ । तेजो॒ वै । वै पृ॒ष्ठानि॑ । पृ॒ष्ठानि॒ तेजः॑ । तेज॑ ए॒व । ए॒वास्मिन्न्॑ । अ॒स्मि॒न् द॒धा॒ति॒ । द॒धा॒ति॒ प्र॒जाप॑तिः । प्र॒जाप॑तिर॒ग्निम् । प्र॒जाप॑ति॒रिति॑ प्र॒जा - प॒तिः॒ । अ॒ग्निम॑सृजत । अ॒सृ॒ज॒त॒ सः । सो᳚ऽस्मात् । अ॒स्मा॒थ् सृ॒ष्टः । सृ॒ष्टः पराङ्॑ । परा॑ङैत् । ऐ॒त् तम् । तम् ॅवा॑रव॒न्तीये॑न । वा॒र॒व॒न्तीये॑नावारयत । वा॒र॒व॒न्तीये॒नेति॑ वार - व॒न्तीये॑न । अ॒वा॒र॒य॒त॒ तत् । तद् वा॑रव॒न्तीय॑स्य । वा॒र॒व॒न्तीय॑स्य वारवन्तीय॒त्वम् । वा॒र॒व॒न्तीय॒स्येति॑ वार - व॒न्तीय॑स्य । वा॒र॒व॒न्ती॒य॒त्वꣳ श्यै॒तेन॑ । वा॒र॒व॒न्ती॒य॒त्वमिति॑ वारवन्तीय - त्वम् । श्यै॒तेन॑ श्ये॒ती । श्ये॒ती अ॑कुरुत । अ॒कु॒रु॒त॒ तत् । तच्छ्यै॒तस्य॑ । श्यै॒तस्य॑ शैत॒त्वम् । श्यै॒त॒त्वम् ॅयत् । श्यै॒त॒त्वमिति॑ श्यैत - त्वम् \newline

\textbf{Jatai Paata} \newline

1. गा॒य॒त्रेण॑ पु॒रस्ता᳚त् पु॒रस्ता᳚द् गाय॒त्रेण॑ गाय॒त्रेण॑ पु॒रस्ता᳚त् । \newline
2. पु॒रस्ता॒ दुपोप॑ पु॒रस्ता᳚त् पु॒रस्ता॒ दुप॑ । \newline
3. उप॑ तिष्ठते तिष्ठत॒ उपोप॑ तिष्ठते । \newline
4. ति॒ष्ठ॒ते॒ प्रा॒णम् प्रा॒णम् ति॑ष्ठते तिष्ठते प्रा॒णम् । \newline
5. प्रा॒ण मे॒वैव प्रा॒णम् प्रा॒ण मे॒व । \newline
6. प्रा॒णमिति॑ प्र - अ॒नम् । \newline
7. ए॒वास्मि॑न् नस्मिन् ने॒वै वास्मिन्न्॑ । \newline
8. अ॒स्मि॒न् द॒धा॒ति॒ द॒धा॒ त्य॒स्मि॒न् न॒स्मि॒न् द॒धा॒ति॒ । \newline
9. द॒धा॒ति॒ बृ॒ह॒द्र॒थ॒न्त॒राभ्या᳚म् बृहद्रथन्त॒राभ्या᳚म् दधाति दधाति बृहद्रथन्त॒राभ्या᳚म् । \newline
10. बृ॒ह॒द्र॒थ॒न्त॒राभ्या᳚म् प॒क्षौ प॒क्षौ बृ॑हद्रथन्त॒राभ्या᳚म् बृहद्रथन्त॒राभ्या᳚म् प॒क्षौ । \newline
11. बृ॒ह॒द्र॒थ॒न्त॒राभ्या॒मिति॑ बृहत् - र॒थ॒न्त॒राभ्या᳚म् । \newline
12. प॒क्षा वोज॒ ओजः॑ प॒क्षौ प॒क्षा वोजः॑ । \newline
13. ओज॑ ए॒वै वौज॒ ओज॑ ए॒व । \newline
14. ए॒वास्मि॑न् नस्मिन् ने॒वै वास्मिन्न्॑ । \newline
15. अ॒स्मि॒न् द॒धा॒ति॒ द॒धा॒ त्य॒स्मि॒न् न॒स्मि॒न् द॒धा॒ति॒ । \newline
16. द॒धा॒ त्यृ॒तु॒स्थाय॑ज्ञाय॒ज्ञिये॑न र्‌तु॒स्थाय॑ज्ञाय॒ज्ञिये॑न दधाति दधा त्यृतु॒स्थाय॑ज्ञाय॒ज्ञिये॑न । \newline
17. ऋ॒तु॒स्थाय॑ज्ञाय॒ज्ञिये॑न॒ पुच्छ॒म् पुच्छ॑ मृतु॒स्थाय॑ज्ञाय॒ज्ञिये॑न र्‌तु॒स्थाय॑ज्ञाय॒ज्ञिये॑न॒ पुच्छ᳚म् । \newline
18. पुच्छ॑ मृ॒तु ष्वृ॒तुषु॒ पुच्छ॒म् पुच्छ॑ मृ॒तुषु॑ । \newline
19. ऋ॒तु ष्वे॒वैव र्‌तुष्वृ॒तु ष्वे॒व । \newline
20. ए॒व प्रति॒ प्रत्ये॒वैव प्रति॑ । \newline
21. प्रति॑ तिष्ठति तिष्ठति॒ प्रति॒ प्रति॑ तिष्ठति । \newline
22. ति॒ष्ठ॒ति॒ पृ॒ष्ठैः पृ॒ष्ठै स्ति॑ष्ठति तिष्ठति पृ॒ष्ठैः । \newline
23. पृ॒ष्ठै रुपोप॑ पृ॒ष्ठैः पृ॒ष्ठै रुप॑ । \newline
24. उप॑ तिष्ठते तिष्ठत॒ उपोप॑ तिष्ठते । \newline
25. ति॒ष्ठ॒ते॒ तेज॒ स्तेज॑ स्तिष्ठते तिष्ठते॒ तेजः॑ । \newline
26. तेजो॒ वै वै तेज॒ स्तेजो॒ वै । \newline
27. वै पृ॒ष्ठानि॑ पृ॒ष्ठानि॒ वै वै पृ॒ष्ठानि॑ । \newline
28. पृ॒ष्ठानि॒ तेज॒ स्तेजः॑ पृ॒ष्ठानि॑ पृ॒ष्ठानि॒ तेजः॑ । \newline
29. तेज॑ ए॒वैव तेज॒ स्तेज॑ ए॒व । \newline
30. ए॒वास्मि॑न् नस्मिन् ने॒वै वास्मिन्न्॑ । \newline
31. अ॒स्मि॒न् द॒धा॒ति॒ द॒धा॒ त्य॒स्मि॒न् न॒स्मि॒न् द॒धा॒ति॒ । \newline
32. द॒धा॒ति॒ प्र॒जाप॑तिः प्र॒जाप॑तिर् दधाति दधाति प्र॒जाप॑तिः । \newline
33. प्र॒जाप॑ति र॒ग्नि म॒ग्निम् प्र॒जाप॑तिः प्र॒जाप॑ति र॒ग्निम् । \newline
34. प्र॒जाप॑ति॒रिति॑ प्र॒जा - प॒तिः॒ । \newline
35. अ॒ग्नि म॑सृजता सृजता॒ग्नि म॒ग्नि म॑सृजत । \newline
36. अ॒सृ॒ज॒त॒ स सो॑ ऽसृजता सृजत॒ सः । \newline
37. सो᳚ ऽस्मा दस्मा॒ थ्स सो᳚ ऽस्मात् । \newline
38. अ॒स्मा॒थ् सृ॒ष्टः सृ॒ष्टो᳚ ऽस्मा दस्मा थ्सृ॒ष्टः । \newline
39. सृ॒ष्टः परा॒ङ् परा᳚ङ् ख्सृ॒ष्टः सृ॒ष्टः पराङ्॑ । \newline
40. परा॑ ङैदै॒त् परा॒ङ् परा॑ ङैत् । \newline
41. ऐ॒त् तम् त मै॑दै॒त् तम् । \newline
42. तं ॅवा॑रव॒न्तीये॑न वारव॒न्तीये॑न॒ तम् तं ॅवा॑रव॒न्तीये॑न । \newline
43. वा॒र॒व॒न्तीये॑ना वारयता वारयत वारव॒न्तीये॑न वारव॒न्तीये॑ नावारयत । \newline
44. वा॒र॒व॒न्तीये॒नेति॑ वार - व॒न्तीये॑न । \newline
45. अ॒वा॒र॒य॒त॒ तत् तद॑वारयता वारयत॒ तत् । \newline
46. तद् वा॑रव॒न्तीय॑स्य वारव॒न्तीय॑स्य॒ तत् तद् वा॑रव॒न्तीय॑स्य । \newline
47. वा॒र॒व॒न्तीय॑स्य वारवन्तीय॒त्वं ॅवा॑रवन्तीय॒त्वं ॅवा॑रव॒न्तीय॑स्य वारव॒न्तीय॑स्य वारवन्तीय॒त्वम् । \newline
48. वा॒र॒व॒न्तीय॒स्येति॑ वार - व॒न्तीय॑स्य । \newline
49. वा॒र॒व॒न्ती॒य॒त्वꣳ श्यै॒तेन॑ श्यै॒तेन॑ वारवन्तीय॒त्वं ॅवा॑रवन्तीय॒त्वꣳ श्यै॒तेन॑ । \newline
50. वा॒र॒व॒न्ती॒य॒त्वमिति॑ वारवन्तीय - त्वम् । \newline
51. श्यै॒तेन॑ श्ये॒ती श्ये॒ती श्यै॒तेन॑ श्यै॒तेन॑ श्ये॒ती । \newline
52. श्ये॒ती अ॑कुरुता कुरुत श्ये॒ती श्ये॒ती अ॑कुरुत । \newline
53. अ॒कु॒रु॒त॒ तत् तद॑कुरुता कुरुत॒ तत् । \newline
54. तच्छ्यै॒तस्य॑ श्यै॒तस्य॒ तत् तच्छ्यै॒तस्य॑ । \newline
55. श्यै॒तस्य॑ श्यैत॒त्वꣳ श्यै॑त॒त्वꣳ श्यै॒तस्य॑ श्यै॒तस्य॑ श्यैत॒त्वम् । \newline
56. श्यै॒त॒त्वं ॅयद् यच्छ्यै॑त॒त्वꣳ श्यै॑त॒त्वं ॅयत् । \newline
57. श्यै॒त॒त्वमिति॑ श्यैत - त्वम् । \newline

\textbf{Ghana Paata } \newline

1. गा॒य॒त्रेण॑ पु॒रस्ता᳚त् पु॒रस्ता᳚द् गाय॒त्रेण॑ गाय॒त्रेण॑ पु॒रस्ता॒ दुपोप॑ पु॒रस्ता᳚द् गाय॒त्रेण॑ गाय॒त्रेण॑ पु॒रस्ता॒ दुप॑ । \newline
2. पु॒रस्ता॒ दुपोप॑ पु॒रस्ता᳚त् पु॒रस्ता॒ दुप॑ तिष्ठते तिष्ठत॒ उप॑ पु॒रस्ता᳚त् पु॒रस्ता॒ दुप॑ तिष्ठते । \newline
3. उप॑ तिष्ठते तिष्ठत॒ उपोप॑ तिष्ठते प्रा॒णम् प्रा॒णम् ति॑ष्ठत॒ उपोप॑ तिष्ठते प्रा॒णम् । \newline
4. ति॒ष्ठ॒ते॒ प्रा॒णम् प्रा॒णम् ति॑ष्ठते तिष्ठते प्रा॒ण मे॒वैव प्रा॒णम् ति॑ष्ठते तिष्ठते प्रा॒ण मे॒व । \newline
5. प्रा॒ण मे॒वैव प्रा॒णम् प्रा॒ण मे॒वास्मि॑न् नस्मिन् ने॒व प्रा॒णम् प्रा॒ण मे॒वास्मिन्न्॑ । \newline
6. प्रा॒णमिति॑ प्र - अ॒नम् । \newline
7. ए॒वास्मि॑न् नस्मिन् ने॒वै वास्मि॑न् दधाति दधा त्यस्मिन् ने॒वै वास्मि॑न् दधाति । \newline
8. अ॒स्मि॒न् द॒धा॒ति॒ द॒धा॒ त्य॒स्मि॒न् न॒स्मि॒न् द॒धा॒ति॒ बृ॒ह॒द्र॒थ॒न्त॒राभ्या᳚म् बृहद्रथन्त॒राभ्या᳚म् दधात्यस्मिन् नस्मिन् दधाति बृहद्रथन्त॒राभ्या᳚म् । \newline
9. द॒धा॒ति॒ बृ॒ह॒द्र॒थ॒न्त॒राभ्या᳚म् बृहद्रथन्त॒राभ्या᳚म् दधाति दधाति बृहद्रथन्त॒राभ्या᳚म् प॒क्षौ प॒क्षौ बृ॑हद्रथन्त॒राभ्या᳚म् दधाति दधाति बृहद्रथन्त॒राभ्या᳚म् प॒क्षौ । \newline
10. बृ॒ह॒द्र॒थ॒न्त॒राभ्या᳚म् प॒क्षौ प॒क्षौ बृ॑हद्रथन्त॒राभ्या᳚म् बृहद्रथन्त॒राभ्या᳚म् प॒क्षा वोज॒ ओजः॑ प॒क्षौ बृ॑हद्रथन्त॒राभ्या᳚म् बृहद्रथन्त॒राभ्या᳚म् प॒क्षा वोजः॑ । \newline
11. बृ॒ह॒द्र॒थ॒न्त॒राभ्या॒मिति॑ बृहत् - र॒थ॒न्त॒राभ्या᳚म् । \newline
12. प॒क्षा वोज॒ ओजः॑ प॒क्षौ प॒क्षा वोज॑ ए॒वै वौजः॑ प॒क्षौ प॒क्षा वोज॑ ए॒व । \newline
13. ओज॑ ए॒वै वौज॒ ओज॑ ए॒वास्मि॑न् नस्मिन् ने॒वौज॒ ओज॑ ए॒वास्मिन्न्॑ । \newline
14. ए॒वास्मि॑न् नस्मिन् ने॒वै वास्मि॑न् दधाति दधा त्यस्मिन् ने॒वै वास्मि॑न् दधाति । \newline
15. अ॒स्मि॒न् द॒धा॒ति॒ द॒धा॒ त्य॒स्मि॒न् न॒स्मि॒न् द॒धा॒ त्यृ॒तु॒स्थाय॑ज्ञाय॒ज्ञिये॑न र्‌तु॒स्थाय॑ज्ञाय॒ज्ञिये॑न दधा त्यस्मिन् नस्मिन् दधा त्यृतु॒स्थाय॑ज्ञाय॒ज्ञिये॑न । \newline
16. द॒धा॒ त्यृ॒तु॒स्थाय॑ज्ञाय॒ज्ञिये॑न र्‌तु॒स्थाय॑ज्ञाय॒ज्ञिये॑न दधाति दधा त्यृतु॒स्थाय॑ज्ञाय॒ज्ञिये॑न॒ पुच्छ॒म् पुच्छ॑ मृतु॒स्थाय॑ज्ञाय॒ज्ञिये॑न दधाति दधा त्यृतु॒स्थाय॑ज्ञाय॒ज्ञिये॑न॒ पुच्छ᳚म् । \newline
17. ऋ॒तु॒स्थाय॑ज्ञाय॒ज्ञिये॑न॒ पुच्छ॒म् पुच्छ॑ मृतु॒स्थाय॑ज्ञाय॒ज्ञिये॑न र्‌तु॒स्थाय॑ज्ञाय॒ज्ञिये॑न॒ पुच्छ॑ मृ॒तु ष्वृ॒तुषु॒ पुच्छ॑ मृतु॒स्थाय॑ज्ञाय॒ज्ञिये॑न र्‌तु॒स्थाय॑ज्ञाय॒ज्ञिये॑न॒ पुच्छ॑ मृ॒तुषु॑ । \newline
18. पुच्छ॑ मृ॒तु ष्वृ॒तुषु॒ पुच्छ॒म् पुच्छ॑ मृ॒तु ष्वे॒वैव र्‌तुषु॒ पुच्छ॒म् पुच्छ॑ मृ॒तु ष्वे॒व । \newline
19. ऋ॒तु ष्वे॒वैव र्‌तु ष्वृ॒तु ष्वे॒व प्रति॒ प्रत्ये॒व र्‌तु ष्वृ॒तु ष्वे॒व प्रति॑ । \newline
20. ए॒व प्रति॒ प्रत्ये॒वैव प्रति॑ तिष्ठति तिष्ठति॒ प्रत्ये॒वैव प्रति॑ तिष्ठति । \newline
21. प्रति॑ तिष्ठति तिष्ठति॒ प्रति॒ प्रति॑ तिष्ठति पृ॒ष्ठैः पृ॒ष्ठै स्ति॑ष्ठति॒ प्रति॒ प्रति॑ तिष्ठति पृ॒ष्ठैः । \newline
22. ति॒ष्ठ॒ति॒ पृ॒ष्ठैः पृ॒ष्ठै स्ति॑ष्ठति तिष्ठति पृ॒ष्ठै रुपोप॑ पृ॒ष्ठै स्ति॑ष्ठति तिष्ठति पृ॒ष्ठै रुप॑ । \newline
23. पृ॒ष्ठै रुपोप॑ पृ॒ष्ठैः पृ॒ष्ठै रुप॑ तिष्ठते तिष्ठत॒ उप॑ पृ॒ष्ठैः पृ॒ष्ठै रुप॑ तिष्ठते । \newline
24. उप॑ तिष्ठते तिष्ठत॒ उपोप॑ तिष्ठते॒ तेज॒ स्तेज॑ स्तिष्ठत॒ उपोप॑ तिष्ठते॒ तेजः॑ । \newline
25. ति॒ष्ठ॒ते॒ तेज॒ स्तेज॑ स्तिष्ठते तिष्ठते॒ तेजो॒ वै वै तेज॑ स्तिष्ठते तिष्ठते॒ तेजो॒ वै । \newline
26. तेजो॒ वै वै तेज॒ स्तेजो॒ वै पृ॒ष्ठानि॑ पृ॒ष्ठानि॒ वै तेज॒ स्तेजो॒ वै पृ॒ष्ठानि॑ । \newline
27. वै पृ॒ष्ठानि॑ पृ॒ष्ठानि॒ वै वै पृ॒ष्ठानि॒ तेज॒ स्तेजः॑ पृ॒ष्ठानि॒ वै वै पृ॒ष्ठानि॒ तेजः॑ । \newline
28. पृ॒ष्ठानि॒ तेज॒ स्तेजः॑ पृ॒ष्ठानि॑ पृ॒ष्ठानि॒ तेज॑ ए॒वैव तेजः॑ पृ॒ष्ठानि॑ पृ॒ष्ठानि॒ तेज॑ ए॒व । \newline
29. तेज॑ ए॒वैव तेज॒ स्तेज॑ ए॒वास्मि॑न् नस्मिन् ने॒व तेज॒ स्तेज॑ ए॒वास्मिन्न्॑ । \newline
30. ए॒वास्मि॑न् नस्मिन् ने॒वै वास्मि॑न् दधाति दधा त्यस्मिन् ने॒वै वास्मि॑न् दधाति । \newline
31. अ॒स्मि॒न् द॒धा॒ति॒ द॒धा॒ त्य॒स्मि॒न् न॒स्मि॒न् द॒धा॒ति॒ प्र॒जाप॑तिः प्र॒जाप॑तिर् दधा त्यस्मिन् नस्मिन् दधाति प्र॒जाप॑तिः । \newline
32. द॒धा॒ति॒ प्र॒जाप॑तिः प्र॒जाप॑तिर् दधाति दधाति प्र॒जाप॑ति र॒ग्नि म॒ग्निम् प्र॒जाप॑तिर् दधाति दधाति प्र॒जाप॑ति र॒ग्निम् । \newline
33. प्र॒जाप॑ति र॒ग्नि म॒ग्निम् प्र॒जाप॑तिः प्र॒जाप॑ति र॒ग्नि म॑सृजता सृजता॒ग्निम् प्र॒जाप॑तिः प्र॒जाप॑ति र॒ग्नि म॑सृजत । \newline
34. प्र॒जाप॑ति॒रिति॑ प्र॒जा - प॒तिः॒ । \newline
35. अ॒ग्नि म॑सृजता सृजता॒ग्नि म॒ग्नि म॑सृजत॒ स सो॑ ऽसृजता॒ग्नि म॒ग्नि म॑सृजत॒ सः । \newline
36. अ॒सृ॒ज॒त॒ स सो॑ ऽसृजता सृजत॒ सो᳚ ऽस्मा दस्मा॒थ् सो॑ ऽसृजता सृजत॒ सो᳚ ऽस्मात् । \newline
37. सो᳚ ऽस्मा दस्मा॒थ् स सो᳚ ऽस्माथ् सृ॒ष्टः सृ॒ष्टो᳚ ऽस्मा॒थ् स सो᳚ ऽस्माथ् सृ॒ष्टः । \newline
38. अ॒स्मा॒थ् सृ॒ष्टः सृ॒ष्टो᳚ ऽस्मा दस्माथ् सृ॒ष्टः परा॒ङ् परा᳚ङ् ख्सृ॒ष्टो᳚ ऽस्मा दस्माथ् सृ॒ष्टः पराङ्॑ । \newline
39. सृ॒ष्टः परा॒ङ् परा᳚ङ् ख्‌सृ॒ष्टः सृ॒ष्टः परा॑ ङैदै॒त् परा᳚ङ् ख्‌सृ॒ष्टः सृ॒ष्टः 
परा॑ ङैत् । \newline
40. परा॑ ङैदै॒त् परा॒ङ् परा॑ ङै॒त् तम् त मै॒त् परा॒ङ् परा॑ ङै॒त् तम् । \newline
41. ऐ॒त् तम् त मै॑दै॒त् तं ॅवा॑रव॒न्तीये॑न वारव॒न्तीये॑न॒ त मै॑दै॒त् तं ॅवा॑रव॒न्तीये॑न । \newline
42. तं ॅवा॑रव॒न्तीये॑न वारव॒न्तीये॑न॒ तम् तं ॅवा॑रव॒न्तीये॑ना वारयता वारयत वारव॒न्तीये॑न॒ तम् तं ॅवा॑रव॒न्तीये॑ना वारयत । \newline
43. वा॒र॒व॒न्तीये॑ना वारयता वारयत वारव॒न्तीये॑न वारव॒न्तीये॑ना वारयत॒ तत् तद॑वारयत वारव॒न्तीये॑न वारव॒न्तीये॑ना वारयत॒ तत् । \newline
44. वा॒र॒व॒न्तीये॒नेति॑ वार - व॒न्तीये॑न । \newline
45. अ॒वा॒र॒य॒त॒ तत् तद॑वारयता वारयत॒ तद् वा॑रव॒न्तीय॑स्य वारव॒न्तीय॑स्य॒ तद॑वारयता वारयत॒ तद् वा॑रव॒न्तीय॑स्य । \newline
46. तद् वा॑रव॒न्तीय॑स्य वारव॒न्तीय॑स्य॒ तत् तद् वा॑रव॒न्तीय॑स्य वारवन्तीय॒त्वं ॅवा॑रवन्तीय॒त्वं ॅवा॑रव॒न्तीय॑स्य॒ तत् तद् वा॑रव॒न्तीय॑स्य वारवन्तीय॒त्वम् । \newline
47. वा॒र॒व॒न्तीय॑स्य वारवन्तीय॒त्वं ॅवा॑रवन्तीय॒त्वं ॅवा॑रव॒न्तीय॑स्य वारव॒न्तीय॑स्य वारवन्तीय॒त्वꣳ श्यै॒तेन॑ श्यै॒तेन॑ वारवन्तीय॒त्वं ॅवा॑रव॒न्तीय॑स्य वारव॒न्तीय॑स्य वारवन्तीय॒त्वꣳ श्यै॒तेन॑ । \newline
48. वा॒र॒व॒न्तीय॒स्येति॑ वार - व॒न्तीय॑स्य । \newline
49. वा॒र॒व॒न्ती॒य॒त्वꣳ श्यै॒तेन॑ श्यै॒तेन॑ वारवन्तीय॒त्वं ॅवा॑रवन्तीय॒त्वꣳ श्यै॒तेन॑ श्ये॒ती श्ये॒ती श्यै॒तेन॑ वारवन्तीय॒त्वं ॅवा॑रवन्तीय॒त्वꣳ श्यै॒तेन॑ श्ये॒ती । \newline
50. वा॒र॒व॒न्ती॒य॒त्वमिति॑ वारवन्तीय - त्वम् । \newline
51. श्यै॒तेन॑ श्ये॒ती श्ये॒ती श्यै॒तेन॑ श्यै॒तेन॑ श्ये॒ती अ॑कुरुता कुरुत श्ये॒ती श्यै॒तेन॑ श्यै॒तेन॑ श्ये॒ती अ॑कुरुत । \newline
52. श्ये॒ती अ॑कुरुता कुरुत श्ये॒ती श्ये॒ती अ॑कुरुत॒ तत् तद॑कुरुत श्ये॒ती श्ये॒ती अ॑कुरुत॒ तत् । \newline
53. अ॒कु॒रु॒त॒ तत् तद॑कुरुता कुरुत॒ तच्छ्यै॒तस्य॑ श्यै॒तस्य॒ तद॑कुरुता कुरुत॒ तच्छ्यै॒तस्य॑ । \newline
54. टच्छ्यै॒तस्य॑ श्यै॒तस्य॒ तत् तच्छ्यै॒तस्य॑ श्यैत॒त्वꣳ श्यै॑त॒त्वꣳ श्यै॒तस्य॒ तत् तच्छ्यै॒तस्य॑ श्यैत॒त्वम् । \newline
55. श्यै॒तस्य॑ श्यैत॒त्वꣳ श्यै॑त॒त्वꣳ श्यै॒तस्य॑ श्यै॒तस्य॑ श्यैत॒त्वं ॅयद् यच्छ्यै॑त॒त्वꣳ श्यै॒तस्य॑ श्यै॒तस्य॑ श्यैत॒त्वं ॅयत् । \newline
56. श्यै॒त॒त्वं ॅयद् यच्छ्यै॑त॒त्वꣳ श्यै॑त॒त्वं ॅयद् वा॑रव॒न्तीये॑न वारव॒न्तीये॑न॒ यच्छ्यै॑त॒त्वꣳ श्यै॑त॒त्वं ॅयद् वा॑रव॒न्तीये॑न । \newline
57. श्यै॒त॒त्वमिति॑ श्यैत - त्वम् । \newline
\pagebreak
\markright{ TS 5.5.8.2  \hfill https://www.vedavms.in \hfill}

\section{ TS 5.5.8.2 }

\textbf{TS 5.5.8.2 } \newline
\textbf{Samhita Paata} \newline

ॅयद्-वा॑रव॒न्तीये॑नोप॒तिष्ठ॑ते वा॒रय॑त ऐ॒वैनꣳ॑ श्यै॒तेन॑ श्ये॒ती कु॑रुते प्र॒जाप॑ते॒र्.हृद॑येना-पिप॒क्षं प्रत्युप॑ तिष्ठते प्रे॒माण॑मे॒वास्य॑ गच्छति॒ प्राच्या᳚ त्वा दि॒शा सा॑दयामि गाय॒त्रेण॒ छन्द॑सा॒ऽग्निना॑ दे॒वत॑या॒ऽग्नेः शीर्ष्णाग्नेः शिर॒ उप॑ दधामि॒ दक्षि॑णया त्वा दि॒शा सा॑दयामि॒ त्रैष्टु॑भेन॒ छन्द॒सेन्द्रे॑ण दे॒वत॑या॒ऽग्नेः प॒क्षेणा॒ग्नेः प॒क्षमुप॑ दधामि प्र॒तीच्या᳚ त्वा दि॒शा सा॑दयामि॒ - [  ] \newline

\textbf{Pada Paata} \newline

यत् । वा॒र॒व॒न्तीये॒नेति॑ वार - व॒न्तीये॑न । उ॒प॒तिष्ठ॑त॒ इत्यु॑प - तिष्ठ॑ते । वा॒रय॑ते । ए॒व । ए॒न॒म् । श्यै॒तेन॑ । श्ये॒ती । कु॒रु॒ते॒ । प्र॒जाप॑ते॒र्.हृद॑येन । अ॒पि॒प॒क्षमित्य॑पि - प॒क्षम् । प्रति॑ । उपेति॑ । ति॒ष्ठ॒ते॒ । प्रे॒माण᳚म् । ए॒व । अ॒स्य॒ । ग॒च्छ॒ति॒ । प्राच्या᳚ । त्वा॒ । दि॒शा । सा॒द॒या॒मि॒ । गा॒य॒त्रेण॑ । छन्द॑सा । अ॒ग्निना᳚ । दे॒वत॑या । अ॒ग्नेः । शी॒र्ष्णा । अ॒ग्नेः । शिरः॑ । उपेति॑ । द॒धा॒मि॒ । दक्षि॑णया । त्वा॒ । दि॒शा । सा॒द॒या॒मि॒ । त्रैष्टु॑भेन । छन्द॑सा । इन्द्रे॑ण । दे॒वत॑या । अ॒ग्नेः । प॒क्षेण॑ । अ॒ग्नेः । प॒क्षम् । उपेति॑ । द॒धा॒मि॒ । प्र॒तीच्या᳚ । त्वा॒ । दि॒शा । सा॒द॒या॒मि॒ ।  \newline


\textbf{Krama Paata} \newline

यद् वा॑रव॒न्तिये॑न । वा॒र॒व॒न्तीये॑नोप॒तिष्ठ॑ते । वा॒र॒व॒न्तीये॒नेति॑ वार - व॒न्तिये॑न । उ॒प॒तिष्ठ॑ते वा॒रय॑ते । उ॒प॒तिष्ठ॑त॒ इत्यु॑प - तिष्ठ॑ते । वा॒रय॑त ए॒व । ए॒वैन᳚म् । ए॒नꣳ॒॒ श्यै॒तेन॑ । श्यै॒तेन॑ श्ये॒ती । श्ये॒ती कु॑रुते । कु॒रु॒ते॒ प्र॒जाप॑ते॒र्॒.हृद॑येन । प्र॒जाप॑ते॒र्॒.हृद॑येनापिप॒क्षम् । अ॒पि॒प॒क्षम् प्रति॑ । अ॒पि॒प॒क्षमित्य॑पि - प॒क्षम् । प्रत्युप॑ । उप॑ तिष्ठते । ति॒ष्ठ॒ते॒ प्रे॒माण᳚म् । प्रे॒माण॑मे॒व । ए॒वास्य॑ । अ॒स्य॒ ग॒च्छ॒ति॒ । ग॒च्छ॒ति॒ प्राच्या᳚ । प्राच्या᳚ त्वा । त्वा॒ दि॒शा । दि॒शा सा॑दयामि । सा॒द॒या॒मि॒ गा॒य॒त्रेण॑ । गा॒य॒त्रेण॒ छन्द॑सा । छन्द॑सा॒ऽग्निना᳚ । अ॒ग्निना॑ दे॒वत॑या । दे॒वत॑या॒ऽग्नेः । अ॒ग्नेः शी॒र्ष्णा । शी॒र्ष्णाऽग्नेः । अ॒ग्नेः शिरः॑ । शिर॒ उप॑ । उप॑ दधामि । द॒धा॒मि॒ दक्षि॑णया । दक्षि॑णया त्वा । त्वा॒ दि॒शा । दि॒शा सा॑दयामि । सा॒द॒या॒मि॒ त्रैष्टु॑भेन । त्रैष्टु॑भेन॒ छन्द॑सा । छन्द॒सेन्द्रे॑ण । इन्द्रे॑ण दे॒वत॑या । दे॒वत॑या॒ऽग्नेः । अ॒ग्नेः प॒क्षेण॑ । प॒क्षेणा॒ग्नेः । अ॒ग्नेः प॒क्षम् । प॒क्षमुप॑ । उप॑ दधामि । द॒धा॒मि॒ प्र॒तीच्या᳚ । प्र॒तीच्या᳚ त्वा । त्वा॒ दि॒शा । दि॒शा सा॑दयामि । सा॒द॒या॒मि॒ जाग॑तेन \newline

\textbf{Jatai Paata} \newline

1. यद् वा॑रव॒न्तीये॑न वारव॒न्तीये॑न॒ यद् यद् वा॑रव॒न्तीये॑न । \newline
2. वा॒र॒व॒न्तीये॑ नोप॒तिष्ठ॑त उप॒तिष्ठ॑ते वारव॒न्तीये॑न वारव॒न्तीये॑ नोप॒तिष्ठ॑ते । \newline
3. वा॒र॒व॒न्तीये॒नेति॑ वार - व॒न्तीये॑न । \newline
4. उ॒प॒तिष्ठ॑ते वा॒रय॑ते वा॒रय॑त उप॒तिष्ठ॑त उप॒तिष्ठ॑ते वा॒रय॑ते । \newline
5. उ॒प॒तिष्ठ॑त॒ इत्यु॑प - तिष्ठ॑ते । \newline
6. वा॒रय॑त ए॒वैव वा॒रय॑ते वा॒रय॑त ए॒व । \newline
7. ए॒वैन॑ मेन मे॒वै वैन᳚म् । \newline
8. ए॒नꣳ॒॒ श्यै॒तेन॑ श्यै॒ते नै॑न मेनꣳ श्यै॒तेन॑ । \newline
9. श्यै॒तेन॑ श्ये॒ती श्ये॒ती श्यै॒तेन॑ श्यै॒तेन॑ श्ये॒ती । \newline
10. श्ये॒ती कु॑रुते कुरुते श्ये॒ती श्ये॒ती कु॑रुते । \newline
11. कु॒रु॒ते॒ प्र॒जाप॑ते॒र्॒.हृद॑येन प्र॒जाप॑ते॒र्॒.हृद॑येन कुरुते कुरुते प्र॒जाप॑ते॒र्॒.हृद॑येन । \newline
12. प्र॒जाप॑ते॒र्॒.हृद॑येना पिप॒क्ष म॑पिप॒क्षम् प्र॒जाप॑ते॒र्॒.हृद॑येन प्र॒जाप॑ते॒र्॒.हृद॑येना पिप॒क्षम् । \newline
13. अ॒पि॒प॒क्षम् प्रति॒ प्रत्य॑पिप॒क्ष म॑पिप॒क्षम् प्रति॑ । \newline
14. अ॒पि॒प॒क्षमित्य॑पि - प॒क्षम् । \newline
15. प्रत्युपोप॒ प्रति॒ प्रत्युप॑ । \newline
16. उप॑ तिष्ठते तिष्ठत॒ उपोप॑ तिष्ठते । \newline
17. ति॒ष्ठ॒ते॒ प्रे॒माण॑म् प्रे॒माण॑म् तिष्ठते तिष्ठते प्रे॒माण᳚म् । \newline
18. प्रे॒माण॑ मे॒वैव प्रे॒माण॑म् प्रे॒माण॑ मे॒व । \newline
19. ए॒वास्या᳚ स्यै॒वै वास्य॑ । \newline
20. अ॒स्य॒ ग॒च्छ॒ति॒ ग॒च्छ॒ त्य॒स्या॒स्य॒ ग॒च्छ॒ति॒ । \newline
21. ग॒च्छ॒ति॒ प्राच्या॒ प्राच्या॑ गच्छति गच्छति॒ प्राच्या᳚ । \newline
22. प्राच्या᳚ त्वा त्वा॒ प्राच्या॒ प्राच्या᳚ त्वा । \newline
23. त्वा॒ दि॒शा दि॒शा त्वा᳚ त्वा दि॒शा । \newline
24. दि॒शा सा॑दयामि सादयामि दि॒शा दि॒शा सा॑दयामि । \newline
25. सा॒द॒या॒मि॒ गा॒य॒त्रेण॑ गाय॒त्रेण॑ सादयामि सादयामि गाय॒त्रेण॑ । \newline
26. गा॒य॒त्रेण॒ छन्द॑सा॒ छन्द॑सा गाय॒त्रेण॑ गाय॒त्रेण॒ छन्द॑सा । \newline
27. छन्द॑सा॒ ऽग्निना॒ ऽग्निना॒ छन्द॑सा॒ छन्द॑सा॒ ऽग्निना᳚ । \newline
28. अ॒ग्निना॑ दे॒वत॑या दे॒वत॑या॒ ऽग्निना॒ ऽग्निना॑ दे॒वत॑या । \newline
29. दे॒वत॑या॒ ऽग्ने र॒ग्नेर् दे॒वत॑या दे॒वत॑या॒ ऽग्नेः । \newline
30. अ॒ग्नेः शी॒र्ष्णा शी॒र्ष्णा ऽग्ने र॒ग्नेः शी॒र्ष्णा । \newline
31. शी॒र्ष्णा ऽग्ने र॒ग्नेः शी॒र्ष्णा शी॒र्ष्णा ऽग्नेः । \newline
32. अ॒ग्नेः शिरः॒ शिरो॒ ऽग्ने र॒ग्नेः शिरः॑ । \newline
33. शिर॒ उपोप॒ शिरः॒ शिर॒ उप॑ । \newline
34. उप॑ दधामि दधा॒ म्युपोप॑ दधामि । \newline
35. द॒धा॒मि॒ दक्षि॑णया॒ दक्षि॑णया दधामि दधामि॒ दक्षि॑णया । \newline
36. दक्षि॑णया त्वा त्वा॒ दक्षि॑णया॒ दक्षि॑णया त्वा । \newline
37. त्वा॒ दि॒शा दि॒शा त्वा᳚ त्वा दि॒शा । \newline
38. दि॒शा सा॑दयामि सादयामि दि॒शा दि॒शा सा॑दयामि । \newline
39. सा॒द॒या॒मि॒ त्रैष्टु॑भेन॒ त्रैष्टु॑भेन सादयामि सादयामि॒ त्रैष्टु॑भेन । \newline
40. त्रैष्टु॑भेन॒ छन्द॑सा॒ छन्द॑सा॒ त्रैष्टु॑भेन॒ त्रैष्टु॑भेन॒ छन्द॑सा । \newline
41. छन्द॒ सेन्द्रे॒ णेन्द्रे॑ण॒ छन्द॑सा॒ छन्द॒ सेन्द्रे॑ण । \newline
42. इन्द्रे॑ण दे॒वत॑या दे॒वत॒ येन्द्रे॒ णेन्द्रे॑ण दे॒वत॑या । \newline
43. दे॒वत॑या॒ ऽग्ने र॒ग्नेर् दे॒वत॑या दे॒वत॑या॒ ऽग्नेः । \newline
44. अ॒ग्नेः प॒क्षेण॑ प॒क्षेणा॒ग्ने र॒ग्नेः प॒क्षेण॑ । \newline
45. प॒क्षेणा॒ग्ने र॒ग्नेः प॒क्षेण॑ प॒क्षेणा॒ग्नेः । \newline
46. अ॒ग्नेः प॒क्षम् प॒क्ष म॒ग्ने र॒ग्नेः प॒क्षम् । \newline
47. प॒क्ष मुपोप॑ प॒क्षम् प॒क्ष मुप॑ । \newline
48. उप॑ दधामि दधा॒ म्युपोप॑ दधामि । \newline
49. द॒धा॒मि॒ प्र॒तीच्या᳚ प्र॒तीच्या॑ दधामि दधामि प्र॒तीच्या᳚ । \newline
50. प्र॒तीच्या᳚ त्वा त्वा प्र॒तीच्या᳚ प्र॒तीच्या᳚ त्वा । \newline
51. त्वा॒ दि॒शा दि॒शा त्वा᳚ त्वा दि॒शा । \newline
52. दि॒शा सा॑दयामि सादयामि दि॒शा दि॒शा सा॑दयामि । \newline
53. सा॒द॒या॒मि॒ जाग॑तेन॒ जाग॑तेन सादयामि सादयामि॒ जाग॑तेन । \newline

\textbf{Ghana Paata } \newline

1. यद् वा॑रव॒न्तीये॑न वारव॒न्तीये॑न॒ यद् यद् वा॑रव॒न्तीये॑ नोप॒तिष्ठ॑त उप॒तिष्ठ॑ते वारव॒न्तीये॑न॒ यद् यद् वा॑रव॒न्तीये॑ नोप॒तिष्ठ॑ते । \newline
2. वा॒र॒व॒न्तीये॑ नोप॒तिष्ठ॑त उप॒तिष्ठ॑ते वारव॒न्तीये॑न वारव॒न्तीये॑ नोप॒तिष्ठ॑ते वा॒रय॑ते वा॒रय॑त उप॒तिष्ठ॑ते वारव॒न्तीये॑न वारव॒न्तीये॑ नोप॒तिष्ठ॑ते वा॒रय॑ते । \newline
3. वा॒र॒व॒न्तीये॒नेति॑ वार - व॒न्तीये॑न । \newline
4. उ॒प॒तिष्ठ॑ते वा॒रय॑ते वा॒रय॑त उप॒तिष्ठ॑त उप॒तिष्ठ॑ते वा॒रय॑त ए॒वैव वा॒रय॑त उप॒तिष्ठ॑त उप॒तिष्ठ॑ते वा॒रय॑त ए॒व । \newline
5. उ॒प॒तिष्ठ॑त॒ इत्यु॑प - तिष्ठ॑ते । \newline
6. वा॒रय॑त ए॒वैव वा॒रय॑ते वा॒रय॑त ए॒वैन॑ मेन मे॒व वा॒रय॑ते वा॒रय॑त ए॒वैन᳚म् । \newline
7. ए॒वैन॑ मेन मे॒वै वैनꣳ॑ श्यै॒तेन॑ श्यै॒ते नै॑न मे॒वै वैनꣳ॑ श्यै॒तेन॑ । \newline
8. ए॒नꣳ॒॒ श्यै॒तेन॑ श्यै॒ते नै॑न मेनꣳ श्यै॒तेन॑ श्ये॒ती श्ये॒ती श्यै॒ते नै॑न मेनꣳ श्यै॒तेन॑ श्ये॒ती । \newline
9. श्यै॒तेन॑ श्ये॒ती श्ये॒ती श्यै॒तेन॑ श्यै॒तेन॑ श्ये॒ती कु॑रुते कुरुते श्ये॒ती श्यै॒तेन॑ श्यै॒तेन॑ श्ये॒ती कु॑रुते । \newline
10. श्ये॒ती कु॑रुते कुरुते श्ये॒ती श्ये॒ती कु॑रुते प्र॒जाप॑ते॒र्॒.हृद॑येन प्र॒जाप॑ते॒र्॒.हृद॑येन कुरुते श्ये॒ती श्ये॒ती कु॑रुते प्र॒जाप॑ते॒र्॒.हृद॑येन । \newline
11. कु॒रु॒ते॒ प्र॒जाप॑ते॒र्॒.हृद॑येन प्र॒जाप॑ते॒र्॒.हृद॑येन कुरुते कुरुते प्र॒जाप॑ते॒र्॒.हृद॑येना पिप॒क्ष म॑पिप॒क्षम् प्र॒जाप॑ते॒र्॒.हृद॑येन कुरुते कुरुते प्र॒जाप॑ते॒र्॒.हृद॑येना पिप॒क्षम् । \newline
12. प्र॒जाप॑ते॒र्॒.हृद॑येना पिप॒क्ष म॑पिप॒क्षम् प्र॒जाप॑ते॒र्॒.हृद॑येन प्र॒जाप॑ते॒र्॒.हृद॑येना पिप॒क्षम् प्रति॒ प्रत्य॑पिप॒क्षम् प्र॒जाप॑ते॒र्॒.हृद॑येन प्र॒जाप॑ते॒र्॒.हृद॑येना पिप॒क्षम् प्रति॑ । \newline
13. अ॒पि॒प॒क्षम् प्रति॒ प्रत्य॑पिप॒क्ष म॑पिप॒क्षम् प्रत्युपोप॒ प्रत्य॑पिप॒क्ष म॑पिप॒क्षम् प्रत्युप॑ । \newline
14. अ॒पि॒प॒क्षमित्य॑पि - प॒क्षम् । \newline
15. प्रत्युपोप॒ प्रति॒ प्रत्युप॑ तिष्ठते तिष्ठत॒ उप॒ प्रति॒ प्रत्युप॑ तिष्ठते । \newline
16. उप॑ तिष्ठते तिष्ठत॒ उपोप॑ तिष्ठते प्रे॒माण॑म् प्रे॒माण॑म् तिष्ठत॒ उपोप॑ तिष्ठते प्रे॒माण᳚म् । \newline
17. ति॒ष्ठ॒ते॒ प्रे॒माण॑म् प्रे॒माण॑म् तिष्ठते तिष्ठते प्रे॒माण॑ मे॒वैव प्रे॒माण॑म् तिष्ठते तिष्ठते प्रे॒माण॑ मे॒व । \newline
18. प्रे॒माण॑ मे॒वैव प्रे॒माण॑म् प्रे॒माण॑ मे॒वास्या᳚ स्यै॒व प्रे॒माण॑म् प्रे॒माण॑ मे॒वास्य॑ । \newline
19. ए॒वास्या᳚ स्यै॒वै वास्य॑ गच्छति गच्छ त्यस्यै॒वैवास्य॑ गच्छति । \newline
20. अ॒स्य॒ ग॒च्छ॒ति॒ ग॒च्छ॒ त्य॒स्या॒स्य॒ ग॒च्छ॒ति॒ प्राच्या॒ प्राच्या॑ गच्छ त्यस्यास्य गच्छति॒ प्राच्या᳚ । \newline
21. ग॒च्छ॒ति॒ प्राच्या॒ प्राच्या॑ गच्छति गच्छति॒ प्राच्या᳚ त्वा त्वा॒ प्राच्या॑ गच्छति गच्छति॒ प्राच्या᳚ त्वा । \newline
22. प्राच्या᳚ त्वा त्वा॒ प्राच्या॒ प्राच्या᳚ त्वा दि॒शा दि॒शा त्वा॒ प्राच्या॒ प्राच्या᳚ त्वा दि॒शा । \newline
23. त्वा॒ दि॒शा दि॒शा त्वा᳚ त्वा दि॒शा सा॑दयामि सादयामि दि॒शा त्वा᳚ त्वा दि॒शा सा॑दयामि । \newline
24. दि॒शा सा॑दयामि सादयामि दि॒शा दि॒शा सा॑दयामि गाय॒त्रेण॑ गाय॒त्रेण॑ सादयामि दि॒शा दि॒शा सा॑दयामि गाय॒त्रेण॑ । \newline
25. सा॒द॒या॒मि॒ गा॒य॒त्रेण॑ गाय॒त्रेण॑ सादयामि सादयामि गाय॒त्रेण॒ छन्द॑सा॒ छन्द॑सा गाय॒त्रेण॑ सादयामि सादयामि गाय॒त्रेण॒ छन्द॑सा । \newline
26. गा॒य॒त्रेण॒ छन्द॑सा॒ छन्द॑सा गाय॒त्रेण॑ गाय॒त्रेण॒ छन्द॑सा॒ ऽग्निना॒ ऽग्निना॒ छन्द॑सा गाय॒त्रेण॑ गाय॒त्रेण॒ छन्द॑सा॒ ऽग्निना᳚ । \newline
27. छन्द॑सा॒ ऽग्निना॒ ऽग्निना॒ छन्द॑सा॒ छन्द॑सा॒ ऽग्निना॑ दे॒वत॑या दे॒वत॑या॒ ऽग्निना॒ छन्द॑सा॒ छन्द॑सा॒ ऽग्निना॑ दे॒वत॑या । \newline
28. अ॒ग्निना॑ दे॒वत॑या दे॒वत॑या॒ ऽग्निना॒ ऽग्निना॑ दे॒वत॑या॒ ऽग्ने र॒ग्नेर् दे॒वत॑या॒ ऽग्निना॒ ऽग्निना॑ दे॒वत॑या॒ ऽग्नेः । \newline
29. दे॒वत॑या॒ ऽग्ने र॒ग्नेर् दे॒वत॑या दे॒वत॑या॒ ऽग्नेः शी॒र्ष्णा शी॒र्ष्णा ऽग्नेर् दे॒वत॑या दे॒वत॑या॒ ऽग्नेः शी॒र्ष्णा । \newline
30. अ॒ग्नेः शी॒र्ष्णा शी॒र्ष्णा ऽग्ने र॒ग्नेः शी॒र्ष्णा ऽग्ने र॒ग्नेः शी॒र्ष्णा ऽग्ने र॒ग्नेः शी॒र्ष्णा ऽग्नेः । \newline
31. शी॒र्ष्णा ऽग्ने र॒ग्नेः शी॒र्ष्णा शी॒र्ष्णा ऽग्नेः शिरः॒ शिरो॒ ऽग्नेः शी॒र्ष्णा शी॒र्ष्णा ऽग्नेः शिरः॑ । \newline
32. अ॒ग्नेः शिरः॒ शिरो॒ ऽग्ने र॒ग्नेः शिर॒ उपोप॒ शिरो॒ ऽग्ने र॒ग्नेः शिर॒ उप॑ । \newline
33. शिर॒ उपोप॒ शिरः॒ शिर॒ उप॑ दधामि दधा॒ म्युप॒ शिरः॒ शिर॒ उप॑ दधामि । \newline
34. उप॑ दधामि दधा॒ म्युपोप॑ दधामि॒ दक्षि॑णया॒ दक्षि॑णया दधा॒ म्युपोप॑ दधामि॒ दक्षि॑णया । \newline
35. द॒धा॒मि॒ दक्षि॑णया॒ दक्षि॑णया दधामि दधामि॒ दक्षि॑णया त्वा त्वा॒ दक्षि॑णया दधामि दधामि॒ दक्षि॑णया त्वा । \newline
36. दक्षि॑णया त्वा त्वा॒ दक्षि॑णया॒ दक्षि॑णया त्वा दि॒शा दि॒शा त्वा॒ दक्षि॑णया॒ दक्षि॑णया त्वा दि॒शा । \newline
37. त्वा॒ दि॒शा दि॒शा त्वा᳚ त्वा दि॒शा सा॑दयामि सादयामि दि॒शा त्वा᳚ त्वा दि॒शा सा॑दयामि । \newline
38. दि॒शा सा॑दयामि सादयामि दि॒शा दि॒शा सा॑दयामि॒ त्रैष्टु॑भेन॒ त्रैष्टु॑भेन सादयामि दि॒शा दि॒शा सा॑दयामि॒ त्रैष्टु॑भेन । \newline
39. सा॒द॒या॒मि॒ त्रैष्टु॑भेन॒ त्रैष्टु॑भेन सादयामि सादयामि॒ त्रैष्टु॑भेन॒ छन्द॑सा॒ छन्द॑सा॒ त्रैष्टु॑भेन सादयामि सादयामि॒ त्रैष्टु॑भेन॒ छन्द॑सा । \newline
40. त्रैष्टु॑भेन॒ छन्द॑सा॒ छन्द॑सा॒ त्रैष्टु॑भेन॒ त्रैष्टु॑भेन॒ छन्द॒सेन्द्रे॒ णेन्द्रे॑ण॒ छन्द॑सा॒ त्रैष्टु॑भेन॒ त्रैष्टु॑भेन॒ छन्द॒सेन्द्रे॑ण । \newline
41. छन्द॒सेन्द्रे॒ णेन्द्रे॑ण॒ छन्द॑सा॒ छन्द॒सेन्द्रे॑ण दे॒वत॑या दे॒वत॒येन्द्रे॑ण॒ छन्द॑सा॒ छन्द॒सेन्द्रे॑ण दे॒वत॑या । \newline
42. इन्द्रे॑ण दे॒वत॑या दे॒वत॒येन्द्रे॒ णेन्द्रे॑ण दे॒वत॑या॒ ऽग्ने र॒ग्नेर् दे॒वत॒येन्द्रे॒ णेन्द्रे॑ण दे॒वत॑या॒ ऽग्नेः । \newline
43. दे॒वत॑या॒ ऽग्ने र॒ग्नेर् दे॒वत॑या दे॒वत॑या॒ ऽग्नेः प॒क्षेण॑ प॒क्षेणा॒ग्नेर् दे॒वत॑या दे॒वत॑या॒ ऽग्नेः प॒क्षेण॑ । \newline
44. अ॒ग्नेः प॒क्षेण॑ प॒क्षेणा॒ ग्ने र॒ग्नेः प॒क्षेणा॒ ग्ने र॒ग्नेः प॒क्षेणा॒ ग्ने र॒ग्नेः प॒क्षेणा॒ग्नेः । \newline
45. प॒क्षेणा॒ ग्ने र॒ग्नेः प॒क्षेण॑ प॒क्षेणा॒ग्नेः प॒क्षम् प॒क्ष म॒ग्नेः प॒क्षेण॑ प॒क्षेणा॒ग्नेः प॒क्षम् । \newline
46. अ॒ग्नेः प॒क्षम् प॒क्ष म॒ग्ने र॒ग्नेः प॒क्ष मुपोप॑ प॒क्ष म॒ग्ने र॒ग्नेः प॒क्ष मुप॑ । \newline
47. प॒क्ष मुपोप॑ प॒क्षम् प॒क्ष मुप॑ दधामि दधा॒ म्युप॑ प॒क्षम् प॒क्ष मुप॑ दधामि । \newline
48. उप॑ दधामि दधा॒ म्युपोप॑ दधामि प्र॒तीच्या᳚ प्र॒तीच्या॑ दधा॒ म्युपोप॑ दधामि प्र॒तीच्या᳚ । \newline
49. द॒धा॒मि॒ प्र॒तीच्या᳚ प्र॒तीच्या॑ दधामि दधामि प्र॒तीच्या᳚ त्वा त्वा प्र॒तीच्या॑ दधामि दधामि प्र॒तीच्या᳚ त्वा । \newline
50. प्र॒तीच्या᳚ त्वा त्वा प्र॒तीच्या᳚ प्र॒तीच्या᳚ त्वा दि॒शा दि॒शा त्वा᳚ प्र॒तीच्या᳚ प्र॒तीच्या᳚ त्वा दि॒शा । \newline
51. त्वा॒ दि॒शा दि॒शा त्वा᳚ त्वा दि॒शा सा॑दयामि सादयामि दि॒शा त्वा᳚ त्वा दि॒शा सा॑दयामि । \newline
52. दि॒शा सा॑दयामि सादयामि दि॒शा दि॒शा सा॑दयामि॒ जाग॑तेन॒ जाग॑तेन सादयामि दि॒शा दि॒शा सा॑दयामि॒ जाग॑तेन । \newline
53. सा॒द॒या॒मि॒ जाग॑तेन॒ जाग॑तेन सादयामि सादयामि॒ जाग॑तेन॒ छन्द॑सा॒ छन्द॑सा॒ जाग॑तेन सादयामि सादयामि॒ जाग॑तेन॒ छन्द॑सा । \newline
\pagebreak
\markright{ TS 5.5.8.3  \hfill https://www.vedavms.in \hfill}

\section{ TS 5.5.8.3 }

\textbf{TS 5.5.8.3 } \newline
\textbf{Samhita Paata} \newline

जाग॑तेन॒ छन्द॑सा सवि॒त्रा दे॒वत॑या॒ऽग्नेः पुच्छे॑ना॒ग्नेः पुच्छ॒मुप॑ दधा॒म्युदी᳚च्या त्वा दि॒शा सा॑दया॒म्यानु॑ष्टुभेन॒ छन्द॑सा मि॒त्रावरु॑णाभ्यां ए॒वत॑या॒ऽग्नेः प॒क्षेणा॒ग्नेः प॒क्षमुप॑ दधाम्यू॒र्द्ध्वया᳚ त्वा दि॒शा सा॑दयामि॒ पाङ्क्ते॑न॒ छन्द॑सा॒ बृह॒स्पति॑ना दे॒वत॑या॒ऽग्नेः पृ॒ष्ठेना॒ग्नेः पृ॒ष्ठमुप॑ दधामि॒ योऽवा अपा᳚त्मानम॒ग्निं चि॑नु॒तेऽपा᳚त्मा॒ऽमुष्मि॑न् ॅलो॒के भ॑वति॒ यः सात्मा॑नं चिनु॒ते ( ) सात्मा॒ऽमुष्मि॑न् ॅलो॒के भ॑वत्यात्मेष्ट॒का उप॑ दधात्ये॒ष वा अ॒ग्नेरा॒त्मा सात्मा॑नमे॒वाग्निं चि॑नुते॒ सात्मा॒ऽमुष्मि॑न् ॅलो॒के भ॑वति॒ य ए॒वं ॅवेद॑ ॥ \newline

\textbf{Pada Paata} \newline

जाग॑तेन । छन्द॑सा । स॒वि॒त्रा । दे॒वत॑या । अ॒ग्नेः । पुच्छे॑न । अ॒ग्नेः । पुच्छ᳚म् । उपेति॑ । द॒धा॒मि॒ । उदी᳚च्या । त्वा॒ । दि॒शा । सा॒द॒या॒मि॒ । आनु॑ष्टुभे॒नेत्यानु॑ - स्तु॒भे॒न॒ । छन्द॑सा । मि॒त्रावरु॑णाभ्या॒मिति॑ मि॒त्रा-वरु॑णाभ्याम् । दे॒वत॑या । अ॒ग्नेः । प॒क्षेण॑ । अ॒ग्नेः । प॒क्षम् । उपेति॑ । द॒धा॒मि॒ । ऊ॒द्‌र्ध्वया᳚ । त्वा॒ । दि॒शा । सा॒द॒या॒मि॒ । पाङ्क्ते॑न । छन्द॑सा । बृह॒स्पति॑ना । दे॒वत॑या । अ॒ग्नेः । पृ॒ष्ठेन॑ । अ॒ग्नेः । पृ॒ष्ठम् । उपेति॑ । द॒धा॒मि॒ । यः । वै । अपा᳚त्मान॒मित्यप॑-आ॒त्मा॒न॒म् । अ॒ग्निम् । चि॒नु॒ते । अपा॒त्मेत्यप॑ - आ॒त्मा॒ । अ॒मुष्मिन्न्॑ । लो॒के । भ॒व॒ति॒ । यः । सात्मा॑न॒मिति॒ स - आ॒त्मा॒न॒म् । चि॒नु॒ते ( ) । सात्मेति॒ स-आ॒त्मा॒ । अ॒मुष्मिन्न्॑ । लो॒के । भ॒व॒ति॒ । आ॒त्मे॒ष्ट॒का इत्या᳚त्म-इ॒ष्ट॒काः । उपेति॑ । द॒धा॒ति॒ । ए॒षः । वै । अ॒ग्नेः । आ॒त्मा । सात्मा॑न॒मिति॒ स-आ॒त्मा॒न॒म् । ए॒व । अ॒ग्निम् । चि॒नु॒ते॒ । सात्मेति॒ स - आ॒त्मा॒ । अ॒मुष्मिन्न्॑ । लो॒के । भ॒व॒ति॒ । यः । ए॒वम् । वेद॑ ॥  \newline


\textbf{Krama Paata} \newline

जाग॑तेन॒ छन्द॑सा । छन्द॑सा सवि॒त्रा । स॒वि॒त्रा दे॒वत॑या । दे॒वत॑या॒ऽग्नेः । अ॒ग्नेः पुच्छे॑न । पुच्छे॑ना॒ग्नेः । अ॒ग्नेः पुच्छ᳚म् । पुच्छ॒मुप॑ । उप॑ दधामि । द॒धा॒म्युदी᳚च्या । उदी᳚च्या त्वा । त्वा॒ दि॒शा । दि॒शा सा॑दयामि । सा॒द॒या॒म्यानु॑ष्टुभेन । आनु॑ष्टुभेन॒ छन्द॑सा । आनु॑ष्टभे॒नेत्यानु॑ - स्तु॒भे॒न॒ । छन्द॑सा मि॒त्रावरु॑णाभ्याम् । मि॒त्रावरु॑णाभ्याम् दे॒वत॑या । मि॒त्रावरु॑णाभ्या॒मिति॑ मि॒त्रा - वरु॑णाभ्याम् । दे॒वत॑या॒ऽग्नेः । अ॒ग्नेः प॒क्षेण॑ । प॒क्षेणा॒ग्नेः । अ॒ग्नेः प॒क्षम् । प॒क्षमुप॑ । उप॑ दधामि । द॒धा॒म्यू॒र्द्ध्वया᳚ । ऊ॒र्द्ध्वया᳚ त्वा । त्वा॒ दि॒शा । दि॒शा सा॑दयामि । सा॒द॒या॒मि॒ पाङ्क्ते॑न । पाङ्क्ते॑न॒ छन्द॑सा । छन्द॑सा॒ बृह॒स्पति॑ना । बृह॒स्पति॑ना दे॒वत॑या । दे॒वत॑या॒ऽग्नेः । अ॒ग्नेः पृ॒ष्ठेन॑ । पृ॒ष्ठेना॒ग्नेः । अ॒ग्नेः पृ॒ष्ठम् । पृ॒ष्ठमुप॑ । उप॑ दधामि । द॒धा॒मि॒ यः । यो वै । वा अपा᳚त्मानम् । अपा᳚त्मानम॒ग्निम् । अपा᳚त्मान॒मित्यप॑ - आ॒त्मा॒न॒म् । अ॒ग्निम् चि॑नु॒ते । चि॒नु॒तेऽपा᳚त्मा । अपा᳚त्मा॒ऽमुष्मिन्न्॑ । अपा॒त्मेत्यप॑ - आ॒त्मा॒ । अ॒मुष्मि॑न् ॅलो॒के । लो॒के भ॑वति । भ॒व॒ति॒ यः । यः सात्मा॑नम् । सात्मा॑नम् चिनु॒ते ( ) । सात्मा॑न॒मिति॒ स - आ॒त्मा॒न॒म् । चि॒नु॒ते सात्मा᳚ । सात्मा॒ऽमुष्मिन्न्॑ । सात्मेति॒ स - आ॒त्मा॒ । अ॒मुष्मि॑न् ॅलो॒के । लो॒के भ॑वति । भ॒व॒त्या॒त्मे॒ष्ट॒काः । आ॒त्मे॒ष्ट॒का उप॑ । आ॒त्मे॒ष्ट॒का इत्या᳚त्म - इ॒ष्ट॒काः । उप॑ दधाति । द॒धा॒त्ये॒षः । ए॒ष वै । वा अ॒ग्नेः । अ॒ग्नेरा॒त्मा । आ॒त्मा सात्मा॑नम् । सात्मा॑नमे॒व । सात्मा॑न॒मिति॒ स - आ॒त्मा॒न॒म् । ए॒वाग्निम् । अ॒ग्निम् चि॑नुते । चि॒नु॒ते॒ सात्मा᳚ । सात्मा॒ऽमुष्मिन्न्॑ । सात्मेति॒ स - आ॒त्मा॒ । अ॒मुष्मि॑न् ॅलो॒के । लो॒के भ॑वति । भ॒व॒ति॒ यः । य ए॒वम् । ए॒वम् ॅवेद॑ । वेदेति॒ वेद॑ । \newline

\textbf{Jatai Paata} \newline

1. जाग॑तेन॒ छन्द॑सा॒ छन्द॑सा॒ जाग॑तेन॒ जाग॑तेन॒ छन्द॑सा । \newline
2. छन्द॑सा सवि॒त्रा स॑वि॒त्रा छन्द॑सा॒ छन्द॑सा सवि॒त्रा । \newline
3. स॒वि॒त्रा दे॒वत॑या दे॒वत॑या सवि॒त्रा स॑वि॒त्रा दे॒वत॑या । \newline
4. दे॒वत॑या॒ ऽग्ने र॒ग्नेर् दे॒वत॑या दे॒वत॑या॒ ऽग्नेः । \newline
5. अ॒ग्नेः पुच्छे॑न॒ पुच्छे॑ना॒ग्ने र॒ग्नेः पुच्छे॑न । \newline
6. पुच्छे॑ना॒ग्ने र॒ग्नेः पुच्छे॑न॒ पुच्छे॑ना॒ग्नेः । \newline
7. अ॒ग्नेः पुच्छ॒म् पुच्छ॑ म॒ग्ने र॒ग्नेः पुच्छ᳚म् । \newline
8. पुच्छ॒ मुपोप॒ पुच्छ॒म् पुच्छ॒ मुप॑ । \newline
9. उप॑ दधामि दधा॒ म्युपोप॑ दधामि । \newline
10. द॒धा॒ म्युदी॒च्यो दी᳚च्या दधामि दधा॒म्यु दी᳚च्या । \newline
11. उदी᳚च्या त्वा॒ त्वोदी॒च्यो दी᳚च्या त्वा । \newline
12. त्वा॒ दि॒शा दि॒शा त्वा᳚ त्वा दि॒शा । \newline
13. दि॒शा सा॑दयामि सादयामि दि॒शा दि॒शा सा॑दयामि । \newline
14. सा॒द॒या॒ म्यानु॑ष्टुभे॒ना नु॑ष्टुभेन सादयामि सादया॒ म्यानु॑ष्टुभेन । \newline
15. आनु॑ष्टुभेन॒ छन्द॑सा॒ छन्द॒सा ऽऽनु॑ष्टुभे॒ना नु॑ष्टुभेन॒ छन्द॑सा । \newline
16. आनु॑ष्टुभे॒नेत्यानु॑ - स्तु॒भे॒न॒ । \newline
17. छन्द॑सा मि॒त्रावरु॑णाभ्याम् मि॒त्रावरु॑णाभ्या॒म् छन्द॑सा॒ छन्द॑सा मि॒त्रावरु॑णाभ्याम् । \newline
18. मि॒त्रावरु॑णाभ्याम् दे॒वत॑या दे॒वत॑या मि॒त्रावरु॑णाभ्याम् मि॒त्रावरु॑णाभ्याम् दे॒वत॑या । \newline
19. मि॒त्रावरु॑णाभ्या॒मिति॑ मि॒त्रा - वरु॑णाभ्याम् । \newline
20. दे॒वत॑या॒ ऽग्ने र॒ग्नेर् दे॒वत॑या दे॒वत॑या॒ ऽग्नेः । \newline
21. अ॒ग्नेः प॒क्षेण॑ प॒क्षेणा॒ग्ने र॒ग्नेः प॒क्षेण॑ । \newline
22. प॒क्षेणा॒ग्ने र॒ग्नेः प॒क्षेण॑ प॒क्षेणा॒ग्नेः । \newline
23. अ॒ग्नेः प॒क्षम् प॒क्ष म॒ग्ने र॒ग्नेः प॒क्षम् । \newline
24. प॒क्ष मुपोप॑ प॒क्षम् प॒क्ष मुप॑ । \newline
25. उप॑ दधामि दधा॒ म्युपोप॑ दधामि । \newline
26. द॒धा॒ म्यू॒र्द्ध्व यो॒र्द्ध्वया॑ दधामि दधा म्यू॒र्द्ध्वया᳚ । \newline
27. ऊ॒र्द्ध्वया᳚ त्वा त्वो॒र्द्ध्व यो॒र्द्ध्वया᳚ त्वा । \newline
28. त्वा॒ दि॒शा दि॒शा त्वा᳚ त्वा दि॒शा । \newline
29. दि॒शा सा॑दयामि सादयामि दि॒शा दि॒शा सा॑दयामि । \newline
30. सा॒द॒या॒मि॒ पाङ्क्ते॑न॒ पाङ्क्ते॑न सादयामि सादयामि॒ पाङ्क्ते॑न । \newline
31. पाङ्क्ते॑न॒ छन्द॑सा॒ छन्द॑सा॒ पाङ्क्ते॑न॒ पाङ्क्ते॑न॒ छन्द॑सा । \newline
32. छन्द॑सा॒ बृह॒स्पति॑ना॒ बृह॒स्पति॑ना॒ छन्द॑सा॒ छन्द॑सा॒ बृह॒स्पति॑ना । \newline
33. बृह॒स्पति॑ना दे॒वत॑या दे॒वत॑या॒ बृह॒स्पति॑ना॒ बृह॒स्पति॑ना दे॒वत॑या । \newline
34. दे॒वत॑या॒ ऽग्ने र॒ग्नेर् दे॒वत॑या दे॒वत॑या॒ ऽग्नेः । \newline
35. अ॒ग्नेः पृ॒ष्ठेन॑ पृ॒ष्ठेना॒ग्ने र॒ग्नेः पृ॒ष्ठेन॑ । \newline
36. पृ॒ष्ठेना॒ग्ने र॒ग्नेः पृ॒ष्ठेन॑ पृ॒ष्ठेना॒ग्नेः । \newline
37. अ॒ग्नेः पृ॒ष्ठम् पृ॒ष्ठ म॒ग्ने र॒ग्नेः पृ॒ष्ठम् । \newline
38. पृ॒ष्ठ मुपोप॑ पृ॒ष्ठम् पृ॒ष्ठ मुप॑ । \newline
39. उप॑ दधामि दधा॒ म्युपोप॑ दधामि । \newline
40. द॒धा॒मि॒ यो यो द॑धामि दधामि॒ यः । \newline
41. यो वै वै यो यो वै । \newline
42. वा अपा᳚त्मान॒ मपा᳚त्मानं॒ ॅवै वा अपा᳚त्मानम् । \newline
43. अपा᳚त्मान म॒ग्नि म॒ग्नि मपा᳚त्मान॒ मपा᳚त्मान म॒ग्निम् । \newline
44. अपा᳚त्मान॒मित्यप॑ - आ॒त्मा॒न॒म् । \newline
45. अ॒ग्निम् चि॑नु॒ते चि॑नु॒ते᳚ ऽग्नि म॒ग्निम् चि॑नु॒ते । \newline
46. चि॒नु॒ते ऽपा॒त्मा ऽपा᳚त्मा॒ चिनु॒ते चि॑नु॒ते ऽपा᳚त्मा । \newline
47. अपा᳚त्मा॒ ऽमुष्मि॑न् न॒मुष्मि॒न् नपा॒त्मा ऽपा᳚त्मा॒ ऽमुष्मिन्न्॑ । \newline
48. अपा॒त्मेत्यप॑ - आ॒त्मा॒ । \newline
49. अ॒मुष्मि॑न् ॅलो॒के लो॒के॑ ऽमुष्मि॑न् न॒मुष्मि॑न् ॅलो॒के । \newline
50. लो॒के भ॑वति भवति लो॒के लो॒के भ॑वति । \newline
51. भ॒व॒ति॒ यो यो भ॑वति भवति॒ यः । \newline
52. यः सात्मा॑नꣳ॒॒ सात्मा॑नं॒ ॅयो यः सात्मा॑नम् । \newline
53. सात्मा॑नम् चिनु॒ते चि॑नु॒ते सात्मा॑नꣳ॒॒ सात्मा॑नम् चिनु॒ते । \newline
54. सात्मा॑न॒मिति॒ स - आ॒त्मा॒न॒म् । \newline
55. चि॒नु॒ते सात्मा॒ सात्मा॑ चिनु॒ते चि॑नु॒ते सात्मा᳚ । \newline
56. सात्मा॒ ऽमुष्मि॑न् न॒मुष्मि॒न् थ्सात्मा॒ सात्मा॒ ऽमुष्मिन्न्॑ । \newline
57. सात्मेति॒ स - आ॒त्मा॒ । \newline
58. अ॒मुष्मि॑न् ॅलो॒के लो॒के॑ ऽमुष्मि॑न् न॒मुष्मि॑न् ॅलो॒के । \newline
59. लो॒के भ॑वति भवति लो॒के लो॒के भ॑वति । \newline
60. भ॒व॒ त्या॒त्मे॒ष्ट॒का आ᳚त्मेष्ट॒का भ॑वति भव त्यात्मेष्ट॒काः । \newline
61. आ॒त्मे॒ष्ट॒का उपोपा᳚ त्मेष्ट॒का आ᳚त्मेष्ट॒का उप॑ । \newline
62. आ॒त्मे॒ष्ट॒का इत्या᳚त्म - इ॒ष्ट॒काः । \newline
63. उप॑ दधाति दधा॒ त्युपोप॑ दधाति । \newline
64. द॒धा॒ त्ये॒ष ए॒ष द॑धाति दधा त्ये॒षः । \newline
65. ए॒ष वै वा ए॒ष ए॒ष वै । \newline
66. वा अ॒ग्ने र॒ग्नेर् वै वा अ॒ग्नेः । \newline
67. अ॒ग्ने रा॒त्मा ऽऽत्मा ऽग्ने र॒ग्ने रा॒त्मा । \newline
68. आ॒त्मा सात्मा॑नꣳ॒॒ सात्मा॑न मा॒त्मा ऽऽत्मा सात्मा॑नम् । \newline
69. सात्मा॑न मे॒वैव सात्मा॑नꣳ॒॒ सात्मा॑न मे॒व । \newline
70. सात्मा॑न॒मिति॒ स - आ॒त्मा॒न॒म् । \newline
71. ए॒वाग्नि म॒ग्नि मे॒वै वाग्निम् । \newline
72. अ॒ग्निम् चि॑नुते चिनुते॒ ऽग्नि म॒ग्निम् चि॑नुते । \newline
73. चि॒नु॒ते॒ सात्मा॒ सात्मा॑ चिनुते चिनुते॒ सात्मा᳚ । \newline
74. सात्मा॒ ऽमुष्मि॑न् न॒मुष्मि॒न् थ्सात्मा॒ सात्मा॒ ऽमुष्मिन्न्॑ । \newline
75. सात्मेति॒ स - आ॒त्मा॒ । \newline
76. अ॒मुष्मि॑न् ॅलो॒के लो॒के॑ ऽमुष्मि॑न् न॒मुष्मि॑न् ॅलो॒के । \newline
77. लो॒के भ॑वति भवति लो॒के लो॒के भ॑वति । \newline
78. भ॒व॒ति॒ यो यो भ॑वति भवति॒ यः । \newline
79. य ए॒व मे॒वं ॅयो य ए॒वम् । \newline
80. ए॒वं ॅवेद॒ वेदै॒व मे॒वं ॅवेद॑ । \newline
81. वेदेति॒ वेद॑ । \newline

\textbf{Ghana Paata } \newline

1. जाग॑तेन॒ छन्द॑सा॒ छन्द॑सा॒ जाग॑तेन॒ जाग॑तेन॒ छन्द॑सा सवि॒त्रा स॑वि॒त्रा छन्द॑सा॒ जाग॑तेन॒ जाग॑तेन॒ छन्द॑सा सवि॒त्रा । \newline
2. छन्द॑सा सवि॒त्रा स॑वि॒त्रा छन्द॑सा॒ छन्द॑सा सवि॒त्रा दे॒वत॑या दे॒वत॑या सवि॒त्रा छन्द॑सा॒ छन्द॑सा सवि॒त्रा दे॒वत॑या । \newline
3. स॒वि॒त्रा दे॒वत॑या दे॒वत॑या सवि॒त्रा स॑वि॒त्रा दे॒वत॑या॒ ऽग्ने र॒ग्नेर् दे॒वत॑या सवि॒त्रा स॑वि॒त्रा दे॒वत॑या॒ ऽग्नेः । \newline
4. दे॒वत॑या॒ ऽग्ने र॒ग्नेर् दे॒वत॑या दे॒वत॑या॒ ऽग्नेः पुच्छे॑न॒ पुच्छे॑ना॒ग्नेर् दे॒वत॑या दे॒वत॑या॒ ऽग्नेः पुच्छे॑न । \newline
5. अ॒ग्नेः पुच्छे॑न॒ पुच्छे॑ना॒ग्ने र॒ग्नेः पुच्छे॑ना॒ग्ने र॒ग्नेः पुच्छे॑ना॒ग्ने र॒ग्नेः पुच्छे॑ना॒ग्नेः । \newline
6. पुच्छे॑ना॒ग्ने र॒ग्नेः पुच्छे॑न॒ पुच्छे॑ना॒ग्नेः पुच्छ॒म् पुच्छ॑ म॒ग्नेः पुच्छे॑न॒ पुच्छे॑ना॒ग्नेः पुच्छ᳚म् । \newline
7. अ॒ग्नेः पुच्छ॒म् पुच्छ॑ म॒ग्ने र॒ग्नेः पुच्छ॒ मुपोप॒ पुच्छ॑ म॒ग्ने र॒ग्नेः पुच्छ॒ मुप॑ । \newline
8. पुच्छ॒ मुपोप॒ पुच्छ॒म् पुच्छ॒ मुप॑ दधामि दधा॒ म्युप॒ पुच्छ॒म् पुच्छ॒ मुप॑ दधामि । \newline
9. उप॑ दधामि दधा॒ म्युपोप॑ दधा॒ म्युदी॒ च्योदी᳚च्या दधा॒ म्युपोप॑ दधा॒ म्युदी᳚च्या । \newline
10. द॒धा॒ म्युदी॒ च्योदी᳚च्या दधामि दधा॒ म्युदी᳚च्या त्वा॒ त्वोदी᳚च्या दधामि दधा॒ म्युदी᳚च्या त्वा । \newline
11. उदी᳚च्या त्वा॒ त्वोदी॒ च्योदी᳚च्या त्वा दि॒शा दि॒शा त्वोदी॒ च्योदी᳚च्या त्वा दि॒शा । \newline
12. त्वा॒ दि॒शा दि॒शा त्वा᳚ त्वा दि॒शा सा॑दयामि सादयामि दि॒शा त्वा᳚ त्वा दि॒शा सा॑दयामि । \newline
13. दि॒शा सा॑दयामि सादयामि दि॒शा दि॒शा सा॑दया॒ म्यानु॑ष्टुभे॒ना नु॑ष्टुभेन सादयामि दि॒शा दि॒शा सा॑दया॒
म्यानु॑ष्टुभेन । \newline
14. सा॒द॒या॒ म्यानु॑ष्टुभे॒ना नु॑ष्टुभेन सादयामि सादया॒ म्यानु॑ष्टुभेन॒ छन्द॑सा॒ छन्द॒सा ऽऽनु॑ष्टुभेन सादयामि सादया॒ म्यानु॑ष्टुभेन॒ छन्द॑सा । \newline
15. आनु॑ष्टुभेन॒ छन्द॑सा॒ छन्द॒सा ऽऽनु॑ष्टुभे॒ना नु॑ष्टुभेन॒ छन्द॑सा मि॒त्रावरु॑णाभ्याम् मि॒त्रावरु॑णाभ्या॒म् छन्द॒सा ऽऽनु॑ष्टुभे॒ना नु॑ष्टुभेन॒ छन्द॑सा मि॒त्रावरु॑णाभ्याम् । \newline
16. आनु॑ष्टुभे॒नेत्यानु॑ - स्तु॒भे॒न॒ । \newline
17. छन्द॑सा मि॒त्रावरु॑णाभ्याम् मि॒त्रावरु॑णाभ्या॒म् छन्द॑सा॒ छन्द॑सा मि॒त्रावरु॑णाभ्याम् दे॒वत॑या दे॒वत॑या मि॒त्रावरु॑णाभ्या॒म् छन्द॑सा॒ छन्द॑सा मि॒त्रावरु॑णाभ्याम् दे॒वत॑या । \newline
18. मि॒त्रावरु॑णाभ्याम् दे॒वत॑या दे॒वत॑या मि॒त्रावरु॑णाभ्याम् मि॒त्रावरु॑णाभ्याम् दे॒वत॑या॒ ऽग्ने र॒ग्नेर् दे॒वत॑या मि॒त्रावरु॑णाभ्याम् मि॒त्रावरु॑णाभ्याम् दे॒वत॑या॒ ऽग्नेः । \newline
19. मि॒त्रावरु॑णाभ्या॒मिति॑ मि॒त्रा - वरु॑णाभ्याम् । \newline
20. दे॒वत॑या॒ ऽग्ने र॒ग्नेर् दे॒वत॑या दे॒वत॑या॒ ऽग्नेः प॒क्षेण॑ प॒क्षेणा॒ग्नेर् दे॒वत॑या दे॒वत॑या॒ ऽग्नेः प॒क्षेण॑ । \newline
21. अ॒ग्नेः प॒क्षेण॑ प॒क्षेणा॒ग्ने र॒ग्नेः प॒क्षेणा॒ग्ने र॒ग्नेः प॒क्षेणा॒ग्ने र॒ग्नेः प॒क्षेणा॒ग्नेः । \newline
22. प॒क्षेणा॒ग्ने र॒ग्नेः प॒क्षेण॑ प॒क्षेणा॒ग्नेः प॒क्षम् प॒क्ष म॒ग्नेः प॒क्षेण॑ प॒क्षेणा॒ग्नेः प॒क्षम् । \newline
23. अ॒ग्नेः प॒क्षम् प॒क्ष म॒ग्ने र॒ग्नेः प॒क्ष मुपोप॑ प॒क्ष म॒ग्ने र॒ग्नेः प॒क्ष मुप॑ । \newline
24. प॒क्ष मुपोप॑ प॒क्षम् प॒क्ष मुप॑ दधामि दधा॒ म्युप॑ प॒क्षम् प॒क्ष मुप॑ दधामि । \newline
25. उप॑ दधामि दधा॒ म्युपोप॑ दधा म्यू॒र्द्ध्व यो॒र्द्ध्वया॑ दधा॒ म्युपोप॑ दधा म्यू॒र्द्ध्वया᳚ । \newline
26. द॒धा॒ म्यू॒र्द्ध्व यो॒र्द्ध्वया॑ दधामि दधा म्यू॒र्द्ध्वया᳚ त्वा त्वो॒र्द्ध्वया॑ दधामि दधा म्यू॒र्द्ध्वया᳚ त्वा । \newline
27. ऊ॒र्द्ध्वया᳚ त्वा त्वो॒र्द्ध्व यो॒र्द्ध्वया᳚ त्वा दि॒शा दि॒शा त्वो॒र्द्ध्व यो॒र्द्ध्वया᳚ त्वा दि॒शा । \newline
28. त्वा॒ दि॒शा दि॒शा त्वा᳚ त्वा दि॒शा सा॑दयामि सादयामि दि॒शा त्वा᳚ त्वा दि॒शा सा॑दयामि । \newline
29. दि॒शा सा॑दयामि सादयामि दि॒शा दि॒शा सा॑दयामि॒ पाङ्क्ते॑न॒ पाङ्क्ते॑न सादयामि दि॒शा दि॒शा सा॑दयामि॒ पाङ्क्ते॑न । \newline
30. सा॒द॒या॒मि॒ पाङ्क्ते॑न॒ पाङ्क्ते॑न सादयामि सादयामि॒ पाङ्क्ते॑न॒ छन्द॑सा॒ छन्द॑सा॒ पाङ्क्ते॑न सादयामि सादयामि॒ पाङ्क्ते॑न॒ छन्द॑सा । \newline
31. पाङ्क्ते॑न॒ छन्द॑सा॒ छन्द॑सा॒ पाङ्क्ते॑न॒ पाङ्क्ते॑न॒ छन्द॑सा॒ बृह॒स्पति॑ना॒ बृह॒स्पति॑ना॒ छन्द॑सा॒ पाङ्क्ते॑न॒ पाङ्क्ते॑न॒ छन्द॑सा॒ बृह॒स्पति॑ना । \newline
32. छन्द॑सा॒ बृह॒स्पति॑ना॒ बृह॒स्पति॑ना॒ छन्द॑सा॒ छन्द॑सा॒ बृह॒स्पति॑ना दे॒वत॑या दे॒वत॑या॒ बृह॒स्पति॑ना॒ छन्द॑सा॒ छन्द॑सा॒ बृह॒स्पति॑ना दे॒वत॑या । \newline
33. बृह॒स्पति॑ना दे॒वत॑या दे॒वत॑या॒ बृह॒स्पति॑ना॒ बृह॒स्पति॑ना दे॒वत॑या॒ ऽग्ने र॒ग्नेर् दे॒वत॑या॒ बृह॒स्पति॑ना॒ बृह॒स्पति॑ना दे॒वत॑या॒ ऽग्नेः । \newline
34. दे॒वत॑या॒ ऽग्ने र॒ग्नेर् दे॒वत॑या दे॒वत॑या॒ ऽग्नेः पृ॒ष्ठेन॑ पृ॒ष्ठेना॒ग्नेर् दे॒वत॑या दे॒वत॑या॒ ऽग्नेः पृ॒ष्ठेन॑ । \newline
35. अ॒ग्नेः पृ॒ष्ठेन॑ पृ॒ष्ठेना॒ग्ने र॒ग्नेः पृ॒ष्ठेना॒ग्ने र॒ग्नेः पृ॒ष्ठेना॒ग्ने र॒ग्नेः पृ॒ष्ठेना॒ग्नेः । \newline
36. पृ॒ष्ठेना॒ग्ने र॒ग्नेः पृ॒ष्ठेन॑ पृ॒ष्ठेना॒ग्नेः पृ॒ष्ठम् पृ॒ष्ठ म॒ग्नेः पृ॒ष्ठेन॑ पृ॒ष्ठेना॒ग्नेः पृ॒ष्ठम् । \newline
37. अ॒ग्नेः पृ॒ष्ठम् पृ॒ष्ठ म॒ग्ने र॒ग्नेः पृ॒ष्ठ मुपोप॑ पृ॒ष्ठ म॒ग्ने र॒ग्नेः पृ॒ष्ठ मुप॑ । \newline
38. पृ॒ष्ठ मुपोप॑ पृ॒ष्ठम् पृ॒ष्ठ मुप॑ दधामि दधा॒ म्युप॑ पृ॒ष्ठम् पृ॒ष्ठ मुप॑ दधामि । \newline
39. उप॑ दधामि दधा॒ म्युपोप॑ दधामि॒ यो यो द॑धा॒ म्युपोप॑ दधामि॒ यः । \newline
40. द॒धा॒मि॒ यो यो द॑धामि दधामि॒ यो वै वै यो द॑धामि दधामि॒ यो वै । \newline
41. यो वै वै यो यो वा अपा᳚त्मान॒ मपा᳚त्मानं॒ ॅवै यो यो वा अपा᳚त्मानम् । \newline
42. वा अपा᳚त्मान॒ मपा᳚त्मानं॒ ॅवै वा अपा᳚त्मान म॒ग्नि म॒ग्नि मपा᳚त्मानं॒ ॅवै वा अपा᳚त्मान म॒ग्निम् । \newline
43. अपा᳚त्मान म॒ग्नि म॒ग्नि मपा᳚त्मान॒ मपा᳚त्मान म॒ग्निम् चि॑नु॒ते चि॑नु॒ते᳚ ऽग्नि मपा᳚त्मान॒ मपा᳚त्मान म॒ग्निम् चि॑नु॒ते । \newline
44. अपा᳚त्मान॒मित्यप॑ - आ॒त्मा॒न॒म् । \newline
45. अ॒ग्निम् चि॑नु॒ते चि॑नु॒ते᳚ ऽग्नि म॒ग्निम् चि॑नु॒ते ऽपा॒त्मा ऽपा᳚त्मा॒ चिनु॒ते᳚ ऽग्नि म॒ग्निम् चि॑नु॒ते ऽपा᳚त्मा । \newline
46. चि॒नु॒ते ऽपा॒त्मा ऽपा᳚त्मा॒ चिनु॒ते चि॑नु॒ते ऽपा᳚त्मा॒ ऽमुष्मि॑न् न॒मुष्मि॒न् नपा᳚त्मा॒ चिनु॒ते चि॑नु॒ते ऽपा᳚त्मा॒ ऽमुष्मिन्न्॑ । \newline
47. अपा᳚त्मा॒ ऽमुष्मि॑न् न॒मुष्मि॒न् नपा॒त्मा ऽपा᳚त्मा॒ ऽमुष्मि॑न् ॅलो॒के लो॒के॑ ऽमुष्मि॒न् नपा॒त्मा ऽपा᳚त्मा॒ ऽमुष्मि॑न् ॅलो॒के । \newline
48. अपा॒त्मेत्यप॑ - आ॒त्मा॒ । \newline
49. अ॒मुष्मि॑न् ॅलो॒के लो॒के॑ ऽमुष्मि॑न् न॒मुष्मि॑न् ॅलो॒के भ॑वति भवति लो॒के॑ ऽमुष्मि॑न् न॒मुष्मि॑न् ॅलो॒के भ॑वति । \newline
50. लो॒के भ॑वति भवति लो॒के लो॒के भ॑वति॒ यो यो भ॑वति लो॒के लो॒के भ॑वति॒ यः । \newline
51. भ॒व॒ति॒ यो यो भ॑वति भवति॒ यः सात्मा॑नꣳ॒॒ सात्मा॑नं॒ ॅयो भ॑वति भवति॒ यः सात्मा॑नम् । \newline
52. यः सात्मा॑नꣳ॒॒ सात्मा॑नं॒ ॅयो यः सात्मा॑नम् चिनु॒ते चि॑नु॒ते सात्मा॑नं॒ ॅयो यः सात्मा॑नम् चिनु॒ते । \newline
53. सात्मा॑नम् चिनु॒ते चि॑नु॒ते सात्मा॑नꣳ॒॒ सात्मा॑नम् चिनु॒ते सात्मा॒ सात्मा॑ चिनु॒ते सात्मा॑नꣳ॒॒ सात्मा॑नम् चिनु॒ते सात्मा᳚ । \newline
54. सात्मा॑न॒मिति॒ स - आ॒त्मा॒न॒म् । \newline
55. चि॒नु॒ते सात्मा॒ सात्मा॑ चिनु॒ते चि॑नु॒ते सात्मा॒ ऽमुष्मि॑न् न॒मुष्मि॒न् थ्सात्मा॑ चिनु॒ते चि॑नु॒ते सात्मा॒ ऽमुष्मिन्न्॑ । \newline
56. सात्मा॒ ऽमुष्मि॑न् न॒मुष्मि॒न् थ्सात्मा॒ सात्मा॒ ऽमुष्मि॑न् ॅलो॒के लो॒के॑ ऽमुष्मि॒न् थ्सात्मा॒ सात्मा॒ ऽमुष्मि॑न् ॅलो॒के । \newline
57. सात्मेति॒ स - आ॒त्मा॒ । \newline
58. अ॒मुष्मि॑न् ॅलो॒के लो॒के॑ ऽमुष्मि॑न् न॒मुष्मि॑न् ॅलो॒के भ॑वति भवति लो॒के॑ ऽमुष्मि॑न् न॒मुष्मि॑न् ॅलो॒के भ॑वति । \newline
59. लो॒के भ॑वति भवति लो॒के लो॒के भ॑व त्यात्मेष्ट॒का आ᳚त्मेष्ट॒का भ॑वति लो॒के लो॒के भ॑व त्यात्मेष्ट॒काः । \newline
60. भ॒व॒ त्या॒त्मे॒ष्ट॒का आ᳚त्मेष्ट॒का भ॑वति भव त्यात्मेष्ट॒का उपोपा᳚त्मेष्ट॒का भ॑वति भव त्यात्मेष्ट॒का उप॑ । \newline
61. आ॒त्मे॒ष्ट॒का उपोपा᳚त्मेष्ट॒का आ᳚त्मेष्ट॒का उप॑ दधाति दधा॒ त्युपा᳚त्मेष्ट॒का आ᳚त्मेष्ट॒का उप॑ दधाति । \newline
62. आ॒त्मे॒ष्ट॒का इत्या᳚त्म - इ॒ष्ट॒काः । \newline
63. उप॑ दधाति दधा॒ त्युपोप॑ दधा त्ये॒ष ए॒ष द॑धा॒ त्युपोप॑ दधा त्ये॒षः । \newline
64. द॒धा॒ त्ये॒ष ए॒ष द॑धाति दधा त्ये॒ष वै वा ए॒ष द॑धाति दधा त्ये॒ष वै । \newline
65. ए॒ष वै वा ए॒ष ए॒ष वा अ॒ग्ने र॒ग्नेर् वा ए॒ष ए॒ष वा अ॒ग्नेः । \newline
66. वा अ॒ग्ने र॒ग्नेर् वै वा अ॒ग्ने रा॒त्मा ऽऽत्मा ऽग्नेर् वै वा अ॒ग्ने रा॒त्मा । \newline
67. अ॒ग्ने रा॒त्मा ऽऽत्मा ऽग्ने र॒ग्ने रा॒त्मा सात्मा॑नꣳ॒॒ सात्मा॑न मा॒त्मा ऽग्ने र॒ग्ने रा॒त्मा सात्मा॑नम् । \newline
68. आ॒त्मा सात्मा॑नꣳ॒॒ सात्मा॑न मा॒त्मा ऽऽत्मा सात्मा॑न मे॒वैव सात्मा॑न मा॒त्मा ऽऽत्मा सात्मा॑न मे॒व । \newline
69. सात्मा॑न मे॒वैव सात्मा॑नꣳ॒॒ सात्मा॑न मे॒वाग्नि म॒ग्नि मे॒व सात्मा॑नꣳ॒॒ सात्मा॑न मे॒वाग्निम् । \newline
70. सात्मा॑न॒मिति॒ स - आ॒त्मा॒न॒म् । \newline
71. ए॒वाग्नि म॒ग्नि मे॒वै वाग्निम् चि॑नुते चिनुते॒ ऽग्नि मे॒वै वाग्निम् चि॑नुते । \newline
72. अ॒ग्निम् चि॑नुते चिनुते॒ ऽग्नि म॒ग्निम् चि॑नुते॒ सात्मा॒ सात्मा॑ चिनुते॒ ऽग्नि म॒ग्निम् चि॑नुते॒ सात्मा᳚ । \newline
73. चि॒नु॒ते॒ सात्मा॒ सात्मा॑ चिनुते चिनुते॒ सात्मा॒ ऽमुष्मि॑न् न॒मुष्मि॒न् थ्सात्मा॑ चिनुते चिनुते॒ सात्मा॒ ऽमुष्मिन्न्॑ । \newline
74. सात्मा॒ ऽमुष्मि॑न् न॒मुष्मि॒न् थ्सात्मा॒ सात्मा॒ ऽमुष्मि॑न् ॅलो॒के लो॒के॑ ऽमुष्मि॒न् थ्सात्मा॒ सात्मा॒ ऽमुष्मि॑न् ॅलो॒के । \newline
75. सात्मेति॒ स - आ॒त्मा॒ । \newline
76. अ॒मुष्मि॑न् ॅलो॒के लो॒के॑ ऽमुष्मि॑न् न॒मुष्मि॑न् ॅलो॒के भ॑वति भवति लो॒के॑ ऽमुष्मि॑न् न॒मुष्मि॑न् ॅलो॒के भ॑वति । \newline
77. लो॒के भ॑वति भवति लो॒के लो॒के भ॑वति॒ यो यो भ॑वति लो॒के लो॒के भ॑वति॒ यः । \newline
78. भ॒व॒ति॒ यो यो भ॑वति भवति॒ य ए॒व मे॒वं ॅयो भ॑वति भवति॒ य ए॒वम् । \newline
79. य ए॒व मे॒वं ॅयो य ए॒वं ॅवेद॒ वेदै॒वं ॅयो य ए॒वं ॅवेद॑ । \newline
80. ए॒वं ॅवेद॒ वेदै॒व मे॒वं ॅवेद॑ । \newline
81. वेदेति॒ वेद॑ । \newline
\pagebreak
\markright{ TS 5.5.9.1  \hfill https://www.vedavms.in \hfill}

\section{ TS 5.5.9.1 }

\textbf{TS 5.5.9.1 } \newline
\textbf{Samhita Paata} \newline

अग्न॑ उदधे॒ या त॒ इषु॑र्यु॒वा नाम॒ तया॑ नो मृड॒ तस्या᳚स्ते॒ नम॒स्तस्या᳚स्त॒ उप॒ जीव॑न्तो भूया॒स्माग्ने॑ दुद्ध्र गह्य किꣳशिल वन्य॒ या त॒ इषु॑र्यु॒वा नाम॒ तया॑ नो मृड॒ तस्या᳚स्ते॒ नम॒स्तस्या᳚स्त॒ उप॒ जीव॑न्तो भूयास्म॒ पञ्च॒ वा ए॒ते᳚ऽग्नयो॒ यच्चित॑य उद॒धिरे॒व नाम॑ प्रथ॒मो दु॒द्ध्रो - [  ] \newline

\textbf{Pada Paata} \newline

अग्ने᳚ । उ॒द॒ध॒ इत्यु॑द - धे॒ । या । ते॒ । इषुः॑ । यु॒वा । नाम॑ । तया᳚ । नः॒ । मृ॒ड॒ । तस्याः᳚ । ते॒ । नमः॑ । तस्याः᳚ । ते॒ । उपेति॑ । जीव॑न्तः । भू॒या॒स्म॒ । अग्ने᳚ । दु॒द्ध्र॒ । ग॒ह्य॒ । किꣳ॒॒शि॒ल॒ । व॒न्य॒ । या । ते॒ । इषुः॑ । यु॒वा । नाम॑ । तया᳚ । नः॒ । मृ॒ड॒ । तस्याः᳚ । ते॒ । नमः॑ । तस्याः᳚ । ते॒ । उपेति॑ । जीव॑न्तः । भू॒या॒स्म॒ । पञ्च॑ । वै । ए॒ते । अ॒ग्नयः॑ । यत् । चित॑यः । उ॒द॒धिरित्यु॑द - धिः । ए॒व । नाम॑ । प्र॒थ॒मः । दु॒द्ध्रः ।  \newline


\textbf{Krama Paata} \newline

अग्न॑ उदधे । उ॒द॒धे॒ या । उ॒द॒ध॒ इत्यु॑द - धे॒ । या ते᳚ । त॒ इषुः॑ । इषु॑र् यु॒वा । यु॒वा नाम॑ । नाम॒ तया᳚ । तया॑ नः । नो॒ मृ॒ड॒ । मृ॒ड॒ तस्याः᳚ । तस्या᳚स्ते । ते॒ नमः॑ । नम॒स्तस्याः᳚ । तस्या᳚स्ते । त॒ उप॑ । उप॒ जीव॑न्तः । जीव॑न्तो भूयास्म । भू॒या॒स्माग्ने᳚ । अग्ने॑ दुद्ध्र । दु॒द्ध्र॒ ग॒ह्य॒ । ग॒ह्य॒ किꣳ॒॒शि॒ल॒ । किꣳ॒॒शि॒ल॒ व॒न्य॒ । व॒न्य॒ या । या ते᳚ । त॒ इषुः॑ । इषु॑र् यु॒वा । यु॒वा नाम॑ । नाम॒ तया᳚ । तया॑ नः । नो॒ मृ॒ड॒ । मृ॒ड॒ तस्याः᳚ । तस्या᳚स्ते । ते॒ नमः॑ । नम॒स्तस्याः᳚ । तस्या᳚स्ते । त॒ उप॑ । उप॒ जीव॑न्तः । जीव॑न्तो भूयास्म । भू॒या॒स्म॒ पञ्च॑ । पञ्च॒ वै । वा ए॒ते । ए॒ते᳚ऽग्नयः॑ । अ॒ग्नयो॒ यत् । यच् चित॑यः । चित॑य उद॒धिः । उ॒द॒धिरे॒व । उ॒द॒धिरित्यु॑द - धिः । ए॒व नाम॑ । नाम॑ प्रथ॒मः । प्र॒थ॒मो दु॒द्ध्रः । दु॒द्ध्रो द्वि॒तीयः॑ \newline

\textbf{Jatai Paata} \newline

1. अग्न॑ उदध उद॒धे ऽग्ने ऽग्न॑ उदधे । \newline
2. उ॒द॒धे॒ या योद॑ध उदधे॒ या । \newline
3. उ॒द॒ध॒ इत्यु॑द - धे॒ । \newline
4. या ते॑ ते॒ या या ते᳚ । \newline
5. त॒ इषु॒ रिषु॑ स्ते त॒ इषुः॑ । \newline
6. इषु॑र् यु॒वा यु॒वेषु॒ रिषु॑र् यु॒वा । \newline
7. यु॒वा नाम॒ नाम॑ यु॒वा यु॒वा नाम॑ । \newline
8. नाम॒ तया॒ तया॒ नाम॒ नाम॒ तया᳚ । \newline
9. तया॑ नो न॒ स्तया॒ तया॑ नः । \newline
10. नो॒ मृ॒ड॒ मृ॒ड॒ नो॒ नो॒ मृ॒ड॒ । \newline
11. मृ॒ड॒ तस्या॒ स्तस्या॑ मृड मृड॒ तस्याः᳚ । \newline
12. तस्या᳚ स्ते ते॒ तस्या॒ स्तस्या᳚ स्ते । \newline
13. ते॒ नमो॒ नम॑ स्ते ते॒ नमः॑ । \newline
14. नम॒ स्तस्या॒ स्तस्या॒ नमो॒ नम॒ स्तस्याः᳚ । \newline
15. तस्या᳚ स्ते ते॒ तस्या॒ स्तस्या᳚ स्ते । \newline
16. त॒ उपोप॑ ते त॒ उप॑ । \newline
17. उप॒ जीव॑न्तो॒ जीव॑न्त॒ उपोप॒ जीव॑न्तः । \newline
18. जीव॑न्तो भूयास्म भूयास्म॒ जीव॑न्तो॒ जीव॑न्तो भूयास्म । \newline
19. भू॒या॒स्माग्ने ऽग्ने॑ भूयास्म भूया॒स्माग्ने᳚ । \newline
20. अग्ने॑ दुद्ध्र दु॒द्ध्राग्ने ऽग्ने॑ दुद्ध्र । \newline
21. दु॒द्ध्र॒ ग॒ह्य॒ ग॒ह्य॒ दु॒द्ध्र॒ दु॒द्ध्र॒ ग॒ह्य॒ । \newline
22. ग॒ह्य॒ किꣳ॒॒शि॒ल॒ किꣳ॒॒शि॒ल॒ ग॒ह्य॒ ग॒ह्य॒ किꣳ॒॒शि॒ल॒ । \newline
23. किꣳ॒॒शि॒ल॒ व॒न्य॒ व॒न्य॒ किꣳ॒॒शि॒ल॒ किꣳ॒॒शि॒ल॒ व॒न्य॒ । \newline
24. व॒न्य॒ या या व॑न्य वन्य॒ या । \newline
25. या ते॑ ते॒ या या ते᳚ । \newline
26. त॒ इषु॒ रिषु॑ स्ते त॒ इषुः॑ । \newline
27. इषु॑र् यु॒वा यु॒वेषु॒ रिषु॑र् यु॒वा । \newline
28. यु॒वा नाम॒ नाम॑ यु॒वा यु॒वा नाम॑ । \newline
29. नाम॒ तया॒ तया॒ नाम॒ नाम॒ तया᳚ । \newline
30. तया॑ नो न॒ स्तया॒ तया॑ नः । \newline
31. नो॒ मृ॒ड॒ मृ॒ड॒ नो॒ नो॒ मृ॒ड॒ । \newline
32. मृ॒ड॒ तस्या॒ स्तस्या॑ मृड मृड॒ तस्याः᳚ । \newline
33. तस्या᳚ स्ते ते॒ तस्या॒ स्तस्या᳚ स्ते । \newline
34. ते॒ नमो॒ नम॑ स्ते ते॒ नमः॑ । \newline
35. णम॒ स्तस्या॒ स्तस्या॒ नमो॒ नम॒ स्तस्याः᳚ । \newline
36. तस्या᳚ स्ते ते॒ तस्या॒ स्तस्या᳚ स्ते । \newline
37. त॒ उपोप॑ ते त॒ उप॑ । \newline
38. उप॒ जीव॑न्तो॒ जीव॑न्त॒ उपोप॒ जीव॑न्तः । \newline
39. जीव॑न्तो भूयास्म भूयास्म॒ जीव॑न्तो॒ जीव॑न्तो भूयास्म । \newline
40. भू॒या॒स्म॒ पञ्च॒ पञ्च॑ भूयास्म भूयास्म॒ पञ्च॑ । \newline
41. पञ्च॒ वै वै पञ्च॒ पञ्च॒ वै । \newline
42. वा ए॒त ए॒ते वै वा ए॒ते । \newline
43. ए॒ते᳚ ऽग्नयो॒ ऽग्नय॑ ए॒त ए॒ते᳚ ऽग्नयः॑ । \newline
44. अ॒ग्नयो॒ यद् यद॒ग्नयो॒ ऽग्नयो॒ यत् । \newline
45. यच् चित॑य॒ श्चित॑यो॒ यद् यच् चित॑यः । \newline
46. चित॑य उद॒धि रु॑द॒धि श्चित॑य॒ श्चित॑य उद॒धिः । \newline
47. उ॒द॒धि रे॒वैवो द॒धि रु॑द॒धि रे॒व । \newline
48. उ॒द॒धिरित्यु॑द - धिः । \newline
49. ए॒व नाम॒ नामै॒ वैव नाम॑ । \newline
50. नाम॑ प्रथ॒मः प्र॑थ॒मो नाम॒ नाम॑ प्रथ॒मः । \newline
51. प्र॒थ॒मो दु॒द्ध्रो दु॒द्ध्रः प्र॑थ॒मः प्र॑थ॒मो दु॒द्ध्रः । \newline
52. दु॒द्ध्रो द्वि॒तीयो᳚ द्वि॒तीयो॑ दु॒द्ध्रो दु॒द्ध्रो द्वि॒तीयः॑ । \newline

\textbf{Ghana Paata } \newline

1. अग्न॑ उदध उद॒धे ऽग्ने ऽग्न॑ उदधे॒ या योद॒धे ऽग्ने ऽग्न॑ उदधे॒ या । \newline
2. उ॒द॒धे॒ या योद॑ध उदधे॒ या ते॑ ते॒ योद॑ध उदधे॒ या ते᳚ । \newline
3. उ॒द॒ध॒ इत्यु॑द - धे॒ । \newline
4. या ते॑ ते॒ या या त॒ इषु॒ रिषु॑ स्ते॒ या या त॒ इषुः॑ । \newline
5. त॒ इषु॒ रिषु॑ स्ते त॒ इषु॑र् यु॒वा यु॒वेषु॑ स्ते त॒ इषु॑र् यु॒वा । \newline
6. इषु॑र् यु॒वा यु॒वेषु॒ रिषु॑र् यु॒वा नाम॒ नाम॑ यु॒वेषु॒ रिषु॑र् यु॒वा नाम॑ । \newline
7. यु॒वा नाम॒ नाम॑ यु॒वा यु॒वा नाम॒ तया॒ तया॒ नाम॑ यु॒वा यु॒वा नाम॒ तया᳚ । \newline
8. नाम॒ तया॒ तया॒ नाम॒ नाम॒ तया॑ नो न॒ स्तया॒ नाम॒ नाम॒ तया॑ नः । \newline
9. तया॑ नो न॒ स्तया॒ तया॑ नो मृड मृड न॒ स्तया॒ तया॑ नो मृड । \newline
10. नो॒ मृ॒ड॒ मृ॒ड॒ नो॒ नो॒ मृ॒ड॒ तस्या॒ स्तस्या॑ मृड नो नो मृड॒ तस्याः᳚ । \newline
11. मृ॒ड॒ तस्या॒ स्तस्या॑ मृड मृड॒ तस्या᳚ स्ते ते॒ तस्या॑ मृड मृड॒ तस्या᳚ स्ते । \newline
12. तस्या᳚ स्ते ते॒ तस्या॒ स्तस्या᳚ स्ते॒ नमो॒ नम॑ स्ते॒ तस्या॒ स्तस्या᳚ स्ते॒ नमः॑ । \newline
13. ते॒ नमो॒ नम॑ स्ते ते॒ नम॒ स्तस्या॒ स्तस्या॒ नम॑ स्ते ते॒ नम॒ स्तस्याः᳚ । \newline
14. नम॒ स्तस्या॒ स्तस्या॒ नमो॒ नम॒ स्तस्या᳚ स्ते ते॒ तस्या॒ नमो॒ नम॒ स्तस्या᳚ स्ते । \newline
15. तस्या᳚ स्ते ते॒ तस्या॒ स्तस्या᳚ स्त॒ उपोप॑ ते॒ तस्या॒ स्तस्या᳚ स्त॒ उप॑ । \newline
16. त॒ उपोप॑ ते त॒ उप॒ जीव॑न्तो॒ जीव॑न्त॒ उप॑ ते त॒ उप॒ जीव॑न्तः । \newline
17. उप॒ जीव॑न्तो॒ जीव॑न्त॒ उपोप॒ जीव॑न्तो भूयास्म भूयास्म॒ जीव॑न्त॒ उपोप॒ जीव॑न्तो भूयास्म । \newline
18. जीव॑न्तो भूयास्म भूयास्म॒ जीव॑न्तो॒ जीव॑न्तो भूया॒स्माग्ने ऽग्ने॑ भूयास्म॒ जीव॑न्तो॒ जीव॑न्तो भूया॒स्माग्ने᳚ । \newline
19. भू॒या॒स्माग्ने ऽग्ने॑ भूयास्म भूया॒स्माग्ने॑ दुद्ध्र दु॒द्ध्राग्ने॑ भूयास्म भूया॒स्माग्ने॑ दुद्ध्र । \newline
20. अग्ने॑ दुद्ध्र दु॒द्ध्राग्ने ऽग्ने॑ दुद्ध्र गह्य गह्य दु॒द्ध्राग्ने ऽग्ने॑ दुद्ध्र गह्य । \newline
21. दु॒द्ध्र॒ ग॒ह्य॒ ग॒ह्य॒ दु॒द्ध्र॒ दु॒द्ध्र॒ ग॒ह्य॒ किꣳ॒॒शि॒ल॒ किꣳ॒॒शि॒ल॒ ग॒ह्य॒ दु॒द्ध्र॒ दु॒द्ध्र॒ ग॒ह्य॒ किꣳ॒॒शि॒ल॒ । \newline
22. ग॒ह्य॒ किꣳ॒॒शि॒ल॒ किꣳ॒॒शि॒ल॒ ग॒ह्य॒ ग॒ह्य॒ किꣳ॒॒शि॒ल॒ व॒न्य॒ व॒न्य॒ किꣳ॒॒शि॒ल॒ ग॒ह्य॒ ग॒ह्य॒ किꣳ॒॒शि॒ल॒ व॒न्य॒ । \newline
23. किꣳ॒॒शि॒ल॒ व॒न्य॒ व॒न्य॒ किꣳ॒॒शि॒ल॒ किꣳ॒॒शि॒ल॒ व॒न्य॒ या या व॑न्य किꣳशिल किꣳशिल वन्य॒ या । \newline
24. व॒न्य॒ या या व॑न्य वन्य॒ या ते॑ ते॒ या व॑न्य वन्य॒ या ते᳚ । \newline
25. या ते॑ ते॒ या या त॒ इषु॒ रिषु॑ स्ते॒ या या त॒ इषुः॑ । \newline
26. त॒ इषु॒ रिषु॑ स्ते त॒ इषु॑र् यु॒वा यु॒वेषु॑ स्ते त॒ इषु॑र् यु॒वा । \newline
27. इषु॑र् यु॒वा यु॒वेषु॒ रिषु॑र् यु॒वा नाम॒ नाम॑ यु॒वेषु॒ रिषु॑र् यु॒वा नाम॑ । \newline
28. यु॒वा नाम॒ नाम॑ यु॒वा यु॒वा नाम॒ तया॒ तया॒ नाम॑ यु॒वा यु॒वा नाम॒ तया᳚ । \newline
29. नाम॒ तया॒ तया॒ नाम॒ नाम॒ तया॑ नो न॒ स्तया॒ नाम॒ नाम॒ तया॑ नः । \newline
30. तया॑ नो न॒ स्तया॒ तया॑ नो मृड मृड न॒ स्तया॒ तया॑ नो मृड । \newline
31. नो॒ मृ॒ड॒ मृ॒ड॒ नो॒ नो॒ मृ॒ड॒ तस्या॒ स्तस्या॑ मृड नो नो मृड॒ तस्याः᳚ । \newline
32. मृ॒ड॒ तस्या॒ स्तस्या॑ मृड मृड॒ तस्या᳚ स्ते ते॒ तस्या॑ मृड मृड॒ तस्या᳚ स्ते । \newline
33. तस्या᳚ स्ते ते॒ तस्या॒ स्तस्या᳚ स्ते॒ नमो॒ नम॑ स्ते॒ तस्या॒ स्तस्या᳚ स्ते॒ नमः॑ । \newline
34. ते॒ नमो॒ नम॑ स्ते ते॒ नम॒ स्तस्या॒ स्तस्या॒ नम॑ स्ते ते॒ नम॒ स्तस्याः᳚ । \newline
35. नम॒ स्तस्या॒ स्तस्या॒ नमो॒ नम॒ स्तस्या᳚ स्ते ते॒ तस्या॒ नमो॒ नम॒ स्तस्या᳚ स्ते । \newline
36. तस्या᳚ स्ते ते॒ तस्या॒ स्तस्या᳚ स्त॒ उपोप॑ ते॒ तस्या॒ स्तस्या᳚ स्त॒ उप॑ । \newline
37. त॒ उपोप॑ ते त॒ उप॒ जीव॑न्तो॒ जीव॑न्त॒ उप॑ ते त॒ उप॒ जीव॑न्तः । \newline
38. उप॒ जीव॑न्तो॒ जीव॑न्त॒ उपोप॒ जीव॑न्तो भूयास्म भूयास्म॒ जीव॑न्त॒ उपोप॒ जीव॑न्तो भूयास्म । \newline
39. जीव॑न्तो भूयास्म भूयास्म॒ जीव॑न्तो॒ जीव॑न्तो भूयास्म॒ पञ्च॒ पञ्च॑ भूयास्म॒ जीव॑न्तो॒ जीव॑न्तो भूयास्म॒ पञ्च॑ । \newline
40. भू॒या॒स्म॒ पञ्च॒ पञ्च॑ भूयास्म भूयास्म॒ पञ्च॒ वै वै पञ्च॑ भूयास्म भूयास्म॒ पञ्च॒ वै । \newline
41. पञ्च॒ वै वै पञ्च॒ पञ्च॒ वा ए॒त ए॒ते वै पञ्च॒ पञ्च॒ वा ए॒ते । \newline
42. वा ए॒त ए॒ते वै वा ए॒ते᳚ ऽग्नयो॒ ऽग्नय॑ ए॒ते वै वा ए॒ते᳚ ऽग्नयः॑ । \newline
43. ए॒ते᳚ ऽग्नयो॒ ऽग्नय॑ ए॒त ए॒ते᳚ ऽग्नयो॒ यद् यद॒ग्नय॑ ए॒त ए॒ते᳚ ऽग्नयो॒ यत् । \newline
44. अ॒ग्नयो॒ यद् यद॒ग्नयो॒ ऽग्नयो॒ यच् चित॑य॒ श्चित॑यो॒ यद॒ग्नयो॒ ऽग्नयो॒ यच् चित॑यः । \newline
45. यच् चित॑य॒ श्चित॑यो॒ यद् यच् चित॑य उद॒धि रु॑द॒धि श्चित॑यो॒ यद् यच् चित॑य उद॒धिः । \newline
46. चित॑य उद॒धि रु॑द॒धि श्चित॑य॒ श्चित॑य उद॒धि रे॒वैवो द॒धि श्चित॑य॒ श्चित॑य उद॒धि रे॒व । \newline
47. उ॒द॒धि रे॒वैवो द॒धि रु॑द॒धि रे॒व नाम॒ नामै॒वो द॒धि रु॑द॒धि रे॒व नाम॑ । \newline
48. उ॒द॒धिरित्यु॑द - धिः । \newline
49. ए॒व नाम॒ नामै॒ वैव नाम॑ प्रथ॒मः प्र॑थ॒मो नामै॒ वैव नाम॑ प्रथ॒मः । \newline
50. नाम॑ प्रथ॒मः प्र॑थ॒मो नाम॒ नाम॑ प्रथ॒मो दु॒द्ध्रो दु॒द्ध्रः प्र॑थ॒मो नाम॒ नाम॑ प्रथ॒मो दु॒द्ध्रः । \newline
51. प्र॒थ॒मो दु॒द्ध्रो दु॒द्ध्रः प्र॑थ॒मः प्र॑थ॒मो दु॒द्ध्रो द्वि॒तीयो᳚ द्वि॒तीयो॑ दु॒द्ध्रः प्र॑थ॒मः प्र॑थ॒मो दु॒द्ध्रो द्वि॒तीयः॑ । \newline
52. दु॒द्ध्रो द्वि॒तीयो᳚ द्वि॒तीयो॑ दु॒द्ध्रो दु॒द्ध्रो द्वि॒तीयो॒ गह्यो॒ गह्यो᳚ द्वि॒तीयो॑ दु॒द्ध्रो दु॒द्ध्रो द्वि॒तीयो॒ गह्यः॑ । \newline
\pagebreak
\markright{ TS 5.5.9.2  \hfill https://www.vedavms.in \hfill}

\section{ TS 5.5.9.2 }

\textbf{TS 5.5.9.2 } \newline
\textbf{Samhita Paata} \newline

द्वि॒तीयो॒ गह्य॑स्तृ॒तीयः॑ किꣳशि॒लश्च॑तु॒र्थो वन्यः॑ पञ्च॒मस्तेभ्यो॒ यदाहु॑ती॒र्न जु॑हु॒याद॑द्ध्व॒र्युं च॒ यज॑मानं च॒ प्र द॑हेयु॒र्यदे॒ता आहु॑तीर्जु॒होति॑ भाग॒धेये॑नै॒वैना᳚ञ्छमयति॒ नाऽऽ*र्ति॒मार्च्छ॑त्यद्ध्व॒र्युर्न यज॑मानो॒ वाङ्म॑ आ॒सन् न॒सोः प्रा॒णो᳚ऽक्ष्योश्चक्षुः॒ कर्ण॑योः॒ श्रोत्रं॑ बाहु॒वोर्बल॑-मूरु॒वोरोजोऽरि॑ष्टा॒ विश्वा॒न्यङ्गा॑नि त॒नू - [  ] \newline

\textbf{Pada Paata} \newline

द्वि॒तीयः॑ । गह्यः॑ । तृ॒तीयः॑ । किꣳ॒॒शि॒लः । च॒तु॒र्थः । वन्यः॑ । प॒ञ्च॒मः । तेभ्यः॑ । यत् । आहु॑ती॒रित्या - हु॒तीः॒ । न । जु॒हु॒यात् । अ॒द्ध्व॒र्युम् । च॒ । यज॑मानम् । च॒ । प्रेति॑ । द॒हे॒युः॒ । यत् । ए॒ताः । आहु॑ती॒रित्या - हु॒तीः॒ । जु॒होति॑ । भा॒ग॒धेये॒नेति॑ भाग-धेये॑न । ए॒व । ए॒ना॒न् । श॒म॒य॒ति॒ । न । आर्ति᳚म् । एति॑ । ऋ॒च्छ॒ति॒ । अ॒द्ध्व॒र्युः । न । यज॑मानः । वाक् । मे॒ । आ॒सन्न् । न॒सोः । प्रा॒ण इति॑ प्र - अ॒नः । अ॒क्ष्योः । चक्षुः॑ । कर्ण॑योः । श्रोत्र᳚म् । बा॒हु॒वोः । बल᳚म् । ऊ॒रु॒वोः । ओजः॑ । अरि॑ष्टा । विश्वा॑नि । अङ्गा॑नि । त॒नूः ।  \newline


\textbf{Krama Paata} \newline

द्वि॒तीयो॒ गह्यः॑ । गह्य॑स्तृ॒तीयः॑ । तृ॒तीयः॑ किꣳशि॒लः । किꣳ॒॒शि॒लश्च॑तु॒र्त्थः । च॒तु॒र्त्थो वन्यः॑ । वन्यः॑ पञ्च॒मः । प॒ञ्च॒मस्तेभ्यः॑ । तेभ्यो॒ यत् । यदाहु॑तीः । आहु॑ती॒र् न । आहु॑ती॒रित्या - हु॒तीः॒ । न जु॑हु॒यात् । जु॒हु॒याद॑द्ध्व॒र्युम् । अ॒द्ध्व॒र्युम् च॑ । च॒ यज॑मानम् । यज॑मानम् च । च॒ प्र । प्र द॑हेयुः । द॒हे॒यु॒र् यत् । यदे॒ताः । ए॒ता आहु॑तीः । आहु॑तीर् जु॒होति॑ । आहु॑ती॒रित्या - हु॒तीः॒ । जु॒होति॑ भाग॒धेये॑न । भा॒ग॒धेये॑नै॒व । भा॒ग॒धेये॒नेति॑ भाग - धेये॑न । ए॒वैनान्॑ । ए॒ना॒ञ्छ॒म॒य॒ति॒ । श॒म॒य॒ति॒ न । नार्ति᳚म् । आर्ति॒मा । आर्च्छ॑ति । ऋ॒च्छ॒त्य॒द्ध्व॒र्युः । अ॒द्ध्व॒र्युर् न । न यज॑मानः । यज॑मानो॒ वाक् । वाङ् मे᳚ । म॒ आ॒सन्न् । आ॒सन् न॒सोः । न॒सोः प्रा॒णः । प्रा॒णो᳚ऽक्ष्योः । प्रा॒ण इति॑ प्र - अ॒नः । अ॒क्ष्योश्चक्षुः॑ । चक्षुः॒ कर्ण॑योः । कर्ण॑योः॒ श्रोत्र᳚म् । श्रोत्र॑म् बाहु॒वोः । बा॒हु॒वोर् बल᳚म् । बल॑मूरु॒वोः । ऊ॒रु॒वोरोजः॑ । ओजोऽरि॑ष्टा । अरि॑ष्टा॒ विश्वा॑नि । विश्वा॒न्यङ्गा॑नि । अङ्गा॑नि त॒नूः । त॒नूस्त॒नुवा᳚ \newline

\textbf{Jatai Paata} \newline

1. द्वि॒तीयो॒ गह्यो॒ गह्यो᳚ द्वि॒तीयो᳚ द्वि॒तीयो॒ गह्यः॑ । \newline
2. गह्य॑ स्तृ॒तीय॑ स्तृ॒तीयो॒ गह्यो॒ गह्य॑ स्तृ॒तीयः॑ । \newline
3. तृ॒तीयः॑ किꣳशि॒लः किꣳ॑शि॒ल स्तृ॒तीय॑ स्तृ॒तीयः॑ किꣳशि॒लः । \newline
4. किꣳ॒॒शि॒ल श्च॑तु॒र्थ श्च॑तु॒र्थः किꣳ॑शि॒लः किꣳ॑शि॒ल श्च॑तु॒र्थः । \newline
5. च॒तु॒र्थो वन्यो॒ वन्य॑ श्चतु॒र्थ श्च॑तु॒र्थो वन्यः॑ । \newline
6. वन्यः॑ पञ्च॒मः प॑ञ्च॒मो वन्यो॒ वन्यः॑ पञ्च॒मः । \newline
7. प॒ञ्च॒म स्तेभ्य॒ स्तेभ्यः॑ पञ्च॒मः प॑ञ्च॒म स्तेभ्यः॑ । \newline
8. तेभ्यो॒ यद् यत् तेभ्य॒ स्तेभ्यो॒ यत् । \newline
9. यदाहु॑ती॒ राहु॑ती॒र् यद् यदाहु॑तीः । \newline
10. आहु॑ती॒र् न नाहु॑ती॒ राहु॑ती॒र् न । \newline
11. आहु॑ती॒रित्या - हु॒तीः॒ । \newline
12. न जु॑हु॒याज् जु॑हु॒यान् न न जु॑हु॒यात् । \newline
13. जु॒हु॒या द॑द्ध्व॒र्यु म॑द्ध्व॒र्युम् जु॑हु॒याज् जु॑हु॒या द॑द्ध्व॒र्युम् । \newline
14. अ॒द्ध्व॒र्युम् च॑ चाद्ध्व॒र्यु म॑द्ध्व॒र्युम् च॑ । \newline
15. च॒ यज॑मानं॒ ॅयज॑मानम् च च॒ यज॑मानम् । \newline
16. यज॑मानम् च च॒ यज॑मानं॒ ॅयज॑मानम् च । \newline
17. च॒ प्र प्र च॑ च॒ प्र । \newline
18. प्र द॑हेयुर् दहेयुः॒ प्र प्र द॑हेयुः । \newline
19. द॒हे॒यु॒र् यद् यद् द॑हेयुर् दहेयु॒र् यत् । \newline
20. यदे॒ता ए॒ता यद् यदे॒ताः । \newline
21. ए॒ता आहु॑ती॒ राहु॑ती रे॒ता ए॒ता आहु॑तीः । \newline
22. आहु॑तीर् जु॒होति॑ जु॒हो त्याहु॑ती॒ राहु॑तीर् जु॒होति॑ । \newline
23. आहु॑ती॒रित्या - हु॒तीः॒ । \newline
24. जु॒होति॑ भाग॒धेये॑न भाग॒धेये॑न जु॒होति॑ जु॒होति॑ भाग॒धेये॑न । \newline
25. भा॒ग॒धेये॑ नै॒वैव भा॑ग॒धेये॑न भाग॒धेये॑नै॒व । \newline
26. भा॒ग॒धेये॒नेति॑ भाग - धेये॑न । \newline
27. ए॒वैना॑ नेना ने॒वै वैनान्॑ । \newline
28. ए॒ना॒ञ् छ॒म॒य॒ति॒ श॒म॒य॒ त्ये॒ना॒ ने॒ना॒ञ् छ॒म॒य॒ति॒ । \newline
29. श॒म॒य॒ति॒ न न श॑मयति शमयति॒ न । \newline
30. नार्ति॒ मार्ति॒म् न नार्ति᳚म् । \newline
31. आर्ति॒ मा ऽऽर्ति॒ मार्ति॒ मा । \newline
32. आर्च्छ॑ त्यृच्छ॒ त्यार्च्छ॑ति । \newline
33. ऋ॒च्छ॒ त्य॒द्ध्व॒र्यु र॑द्ध्व॒र्युर्. ऋ॑च्छ त्यृच्छ त्यद्ध्व॒र्युः । \newline
34. अ॒द्ध्व॒र्युर् न नाद्ध्व॒र्यु र॑द्ध्व॒र्युर् न । \newline
35. न यज॑मानो॒ यज॑मानो॒ न न यज॑मानः । \newline
36. यज॑मानो॒ वाग् वाग् यज॑मानो॒ यज॑मानो॒ वाक् । \newline
37. वाङ् मे॑ मे॒ वाग् वाङ् मे᳚ । \newline
38. म॒ आ॒सन् ना॒सन् मे॑ म आ॒सन्न् । \newline
39. आ॒सन् न॒सोर् न॒सो रा॒सन् ना॒सन् न॒सोः । \newline
40. न॒सोः प्रा॒णः प्रा॒णो न॒सोर् न॒सोः प्रा॒णः । \newline
41. प्रा॒णो᳚ ऽक्ष्यो र॒क्ष्योः प्रा॒णः प्रा॒णो᳚ ऽक्ष्योः । \newline
42. प्रा॒ण इति॑ प्र - अ॒नः । \newline
43. अ॒क्ष्यो श्चक्षु॒ श्चक्षु॑ र॒क्ष्यो र॒क्ष्यो श्चक्षुः॑ । \newline
44. चक्षुः॒ कर्ण॑योः॒ कर्ण॑यो॒ श्चक्षु॒ श्चक्षुः॒ कर्ण॑योः । \newline
45. कर्ण॑योः॒ श्रोत्रꣳ॒॒ श्रोत्र॒म् कर्ण॑योः॒ कर्ण॑योः॒ श्रोत्र᳚म् । \newline
46. श्रोत्र॑म् बाहु॒वोर् बा॑हु॒वोः श्रोत्रꣳ॒॒ श्रोत्र॑म् बाहु॒वोः । \newline
47. बा॒हु॒वोर् बल॒म् बल॑म् बाहु॒वोर् बा॑हु॒वोर् बल᳚म् । \newline
48. बल॑ मूरु॒वो रू॑रु॒वोर् बल॒म् बल॑ मूरु॒वोः । \newline
49. ऊ॒रु॒वो रोज॒ ओज॑ ऊरु॒वो रू॑रु॒वो रोजः॑ । \newline
50. ओजो ऽरि॒ष्टा ऽरि॒ष्टौज॒ ओजो ऽरि॑ष्टा । \newline
51. अरि॑ष्टा॒ विश्वा॑नि॒ विश्वा॒ न्यरि॒ष्टा ऽरि॑ष्टा॒ विश्वा॑नि । \newline
52. विश्वा॒ न्यङ्गा॒ न्यङ्गा॑नि॒ विश्वा॑नि॒ विश्वा॒ न्यङ्गा॑नि । \newline
53. अङ्गा॑नि त॒नू स्त॒नू रङ्गा॒ न्यङ्गा॑नि त॒नूः । \newline
54. त॒नू स्त॒नुवा॑ त॒नुवा॑ त॒नू स्त॒नू स्त॒नुवा᳚ । \newline

\textbf{Ghana Paata } \newline

1. द्वि॒तीयो॒ गह्यो॒ गह्यो᳚ द्वि॒तीयो᳚ द्वि॒तीयो॒ गह्य॑ स्तृ॒तीय॑ स्तृ॒तीयो॒ गह्यो᳚ द्वि॒तीयो᳚ द्वि॒तीयो॒ गह्य॑ स्तृ॒तीयः॑ । \newline
2. गह्य॑ स्तृ॒तीय॑ स्तृ॒तीयो॒ गह्यो॒ गह्य॑ स्तृ॒तीयः॑ किꣳशि॒लः किꣳ॑शि॒ल स्तृ॒तीयो॒ गह्यो॒ गह्य॑ स्तृ॒तीयः॑ किꣳशि॒लः । \newline
3. तृ॒तीयः॑ किꣳशि॒लः किꣳ॑शि॒ल स्तृ॒तीय॑ स्तृ॒तीयः॑ किꣳशि॒ल श्च॑तु॒र्थ श्च॑तु॒र्थः किꣳ॑शि॒ल स्तृ॒तीय॑ स्तृ॒तीयः॑ किꣳशि॒ल श्च॑तु॒र्थः । \newline
4. किꣳ॒॒शि॒ल श्च॑तु॒र्थ श्च॑तु॒र्थः किꣳ॑शि॒लः किꣳ॑शि॒ल श्च॑तु॒र्थो वन्यो॒ वन्य॑ श्चतु॒र्थः किꣳ॑शि॒लः किꣳ॑शि॒ल श्च॑तु॒र्थो वन्यः॑ । \newline
5. च॒तु॒र्थो वन्यो॒ वन्य॑ श्चतु॒र्थ श्च॑तु॒र्थो वन्यः॑ पञ्च॒मः प॑ञ्च॒मो वन्य॑ श्चतु॒र्थ श्च॑तु॒र्थो वन्यः॑ पञ्च॒मः । \newline
6. वन्यः॑ पञ्च॒मः प॑ञ्च॒मो वन्यो॒ वन्यः॑ पञ्च॒म स्तेभ्य॒ स्तेभ्यः॑ पञ्च॒मो वन्यो॒ वन्यः॑ पञ्च॒म स्तेभ्यः॑ । \newline
7. प॒ञ्च॒म स्तेभ्य॒ स्तेभ्यः॑ पञ्च॒मः प॑ञ्च॒म स्तेभ्यो॒ यद् यत् तेभ्यः॑ पञ्च॒मः प॑ञ्च॒म स्तेभ्यो॒ यत् । \newline
8. तेभ्यो॒ यद् यत् तेभ्य॒ स्तेभ्यो॒ यदाहु॑ती॒ राहु॑ती॒र् यत् तेभ्य॒ स्तेभ्यो॒ यदाहु॑तीः । \newline
9. यदाहु॑ती॒ राहु॑ती॒र् यद् यदाहु॑ती॒र् न नाहु॑ती॒र् यद् यदाहु॑ती॒र् न । \newline
10. आहु॑ती॒र् न नाहु॑ती॒ राहु॑ती॒र् न जु॑हु॒याज् जु॑हु॒यान् नाहु॑ती॒ राहु॑ती॒र् न जु॑हु॒यात् । \newline
11. आहु॑ती॒रित्या - हु॒तीः॒ । \newline
12. न जु॑हु॒याज् जु॑हु॒यान् न न जु॑हु॒या द॑द्ध्व॒र्यु म॑द्ध्व॒र्युम् जु॑हु॒यान् न न जु॑हु॒या द॑द्ध्व॒र्युम् । \newline
13. जु॒हु॒या द॑द्ध्व॒र्यु म॑द्ध्व॒र्युम् जु॑हु॒याज् जु॑हु॒या द॑द्ध्व॒र्युम् च॑ चाद्ध्व॒र्युम् जु॑हु॒याज् जु॑हु॒या द॑द्ध्व॒र्युम् च॑ । \newline
14. अ॒द्ध्व॒र्युम् च॑ चाद्ध्व॒र्यु म॑द्ध्व॒र्युम् च॒ यज॑मानं॒ ॅयज॑मानम् चाद्ध्व॒र्यु म॑द्ध्व॒र्युम् च॒ यज॑मानम् । \newline
15. च॒ यज॑मानं॒ ॅयज॑मानम् च च॒ यज॑मानम् च च॒ यज॑मानम् च च॒ यज॑मानम् च । \newline
16. यज॑मानम् च च॒ यज॑मानं॒ ॅयज॑मानम् च॒ प्र प्र च॒ यज॑मानं॒ ॅयज॑मानम् च॒ प्र । \newline
17. च॒ प्र प्र च॑ च॒ प्र द॑हेयुर् दहेयुः॒ प्र च॑ च॒ प्र द॑हेयुः । \newline
18. प्र द॑हेयुर् दहेयुः॒ प्र प्र द॑हेयु॒र् यद् यद् द॑हेयुः॒ प्र प्र द॑हेयु॒र् यत् । \newline
19. द॒हे॒यु॒र् यद् यद् द॑हेयुर् दहेयु॒र् यदे॒ता ए॒ता यद् द॑हेयुर् दहेयु॒र् यदे॒ताः । \newline
20. यदे॒ता ए॒ता यद् यदे॒ता आहु॑ती॒ राहु॑ती रे॒ता यद् यदे॒ता आहु॑तीः । \newline
21. ए॒ता आहु॑ती॒ राहु॑ती रे॒ता ए॒ता आहु॑तीर् जु॒होति॑ जु॒हो त्याहु॑ती रे॒ता ए॒ता आहु॑तीर् जु॒होति॑ । \newline
22. आहु॑तीर् जु॒होति॑ जु॒हो त्याहु॑ती॒ राहु॑तीर् जु॒होति॑ भाग॒धेये॑न भाग॒धेये॑न जु॒हो त्याहु॑ती॒ राहु॑तीर् जु॒होति॑ भाग॒धेये॑न । \newline
23. आहु॑ती॒रित्या - हु॒तीः॒ । \newline
24. जु॒होति॑ भाग॒धेये॑न भाग॒धेये॑न जु॒होति॑ जु॒होति॑ भाग॒धेये॑ नै॒वैव भा॑ग॒धेये॑न जु॒होति॑ जु॒होति॑ भाग॒धेये॑नै॒व । \newline
25. भा॒ग॒धेये॑ नै॒वैव भा॑ग॒धेये॑न भाग॒धेये॑ नै॒वैना॑ नेना ने॒व भा॑ग॒धेये॑न भाग॒धेये॑
नै॒वैनान्॑ । \newline
26. भा॒ग॒धेये॒नेति॑ भाग - धेये॑न । \newline
27. ए॒वैना॑ नेना ने॒वै वैना᳚ञ् छमयति शमय त्येना ने॒वै वैना᳚ञ् छमयति । \newline
28. ए॒ना॒ञ् छ॒म॒य॒ति॒ श॒म॒य॒ त्ये॒ना॒ ने॒ना॒ञ् छ॒म॒य॒ति॒ न न श॑मय त्येना नेनाञ् छमयति॒ न । \newline
29. श॒म॒य॒ति॒ न न श॑मयति शमयति॒ नार्ति॒ मार्ति॒म् न श॑मयति शमयति॒ नार्ति᳚म् । \newline
30. नार्ति॒ मार्ति॒म् न नार्ति॒ मा ऽऽर्ति॒म् न नार्ति॒ मा । \newline
31. आर्ति॒ मा ऽऽर्ति॒ मार्ति॒ मार्च्छ॑ त्यृच्छ॒ त्याऽऽर्ति॒ मार्ति॒ मार्च्छ॑ति । \newline
32. आर्च्छ॑ त्यृच्छ॒ त्यार्च्छ॑ त्यद्ध्व॒र्यु र॑द्ध्व॒र्युर्. ऋ॑च्छ॒ त्यार्च्छ॑ त्यद्ध्व॒र्युः । \newline
33. ऋ॒च्छ॒त्य॒ द्ध्व॒र्यु र॑द्ध्व॒र्युर्. ऋ॑च्छ त्यृच्छ त्यद्ध्व॒र्युर् न नाद्ध्व॒र्युर्. ऋ॑च्छ त्यृच्छ त्यद्ध्व॒र्युर् न । \newline
34. अ॒द्ध्व॒र्युर् न नाद्ध्व॒र्यु र॑द्ध्व॒र्युर् न यज॑मानो॒ यज॑मानो॒ नाद्ध्व॒र्यु र॑द्ध्व॒र्युर् न यज॑मानः । \newline
35. न यज॑मानो॒ यज॑मानो॒ न न यज॑मानो॒ वाग् वाग् यज॑मानो॒ न न यज॑मानो॒ वाक् । \newline
36. यज॑मानो॒ वाग् वाग् यज॑मानो॒ यज॑मानो॒ वाङ् मे॑ मे॒ वाग् यज॑मानो॒ यज॑मानो॒ वाङ् मे᳚ । \newline
37. वाङ् मे॑ मे॒ वाग् वाङ् म॑ आ॒सन् ना॒सन् मे॒ वाग् वाङ् म॑ आ॒सन्न् । \newline
38. म॒ आ॒सन् ना॒सन् मे॑ म आ॒सन् न॒सोर् न॒सो रा॒सन् मे॑ म आ॒सन् न॒सोः । \newline
39. आ॒सन् न॒सोर् न॒सो रा॒सन् ना॒सन् न॒सोः प्रा॒णः प्रा॒णो न॒सो रा॒सन् ना॒सन् न॒सोः प्रा॒णः । \newline
40. न॒सोः प्रा॒णः प्रा॒णो न॒सोर् न॒सोः प्रा॒णो᳚ ऽक्ष्यो र॒क्ष्योः प्रा॒णो न॒सोर् न॒सोः प्रा॒णो᳚ ऽक्ष्योः । \newline
41. प्रा॒णो᳚ ऽक्ष्यो र॒क्ष्योः प्रा॒णः प्रा॒णो᳚ ऽक्ष्यो श्चक्षु॒ श्चक्षु॑ र॒क्ष्योः प्रा॒णः प्रा॒णो᳚ ऽक्ष्यो श्चक्षुः॑ । \newline
42. प्रा॒ण इति॑ प्र - अ॒नः । \newline
43. अ॒क्ष्यो श्चक्षु॒ श्चक्षु॑ र॒क्ष्यो र॒क्ष्यो श्चक्षुः॒ कर्ण॑योः॒ कर्ण॑यो॒ श्चक्षु॑ र॒क्ष्यो र॒क्ष्यो श्चक्षुः॒ कर्ण॑योः । \newline
44. चक्षुः॒ कर्ण॑योः॒ कर्ण॑यो॒ श्चक्षु॒ श्चक्षुः॒ कर्ण॑योः॒ श्रोत्रꣳ॒॒ श्रोत्र॒म् कर्ण॑यो॒ श्चक्षु॒ श्चक्षुः॒ कर्ण॑योः॒ श्रोत्र᳚म् । \newline
45. कर्ण॑योः॒ श्रोत्रꣳ॒॒ श्रोत्र॒म् कर्ण॑योः॒ कर्ण॑योः॒ श्रोत्र॑म् बाहु॒वोर् बा॑हु॒वोः श्रोत्र॒म् कर्ण॑योः॒ कर्ण॑योः॒ श्रोत्र॑म् बाहु॒वोः । \newline
46. श्रोत्र॑म् बाहु॒वोर् बा॑हु॒वोः श्रोत्रꣳ॒॒ श्रोत्र॑म् बाहु॒वोर् बल॒म् बल॑म् बाहु॒वोः श्रोत्रꣳ॒॒ श्रोत्र॑म् बाहु॒वोर् बल᳚म् । \newline
47. बा॒हु॒वोर् बल॒म् बल॑म् बाहु॒वोर् बा॑हु॒वोर् बल॑ मूरु॒वो रू॑रु॒वोर् बल॑म् बाहु॒वोर् बा॑हु॒वोर् बल॑ मूरु॒वोः । \newline
48. बल॑ मूरु॒वो रू॑रु॒वोर् बल॒म् बल॑ मूरु॒वो रोज॒ ओज॑ ऊरु॒वोर् बल॒म् बल॑ मूरु॒वो रोजः॑ । \newline
49. ऊ॒रु॒वो रोज॒ ओज॑ ऊरु॒वो रू॑रु॒वो रोजो ऽरि॒ष्टा ऽरि॒ष्टौज॑ ऊरु॒वो रू॑रु॒वो रोजो ऽरि॑ष्टा । \newline
50. ओजो ऽरि॒ष्टा ऽरि॒ष्टौज॒ ओजो ऽरि॑ष्टा॒ विश्वा॑नि॒ विश्वा॒ न्यरि॒ष्टौज॒ ओजो ऽरि॑ष्टा॒ विश्वा॑नि । \newline
51. अरि॑ष्टा॒ विश्वा॑नि॒ विश्वा॒ न्यरि॒ष्टा ऽरि॑ष्टा॒ विश्वा॒ न्यङ्गा॒ न्यङ्गा॑नि॒ विश्वा॒ न्यरि॒ष्टा ऽरि॑ष्टा॒ विश्वा॒ न्यङ्गा॑नि । \newline
52. विश्वा॒ न्यङ्गा॒ न्यङ्गा॑नि॒ विश्वा॑नि॒ विश्वा॒ न्यङ्गा॑नि त॒नू स्त॒नू रङ्गा॑नि॒ विश्वा॑नि॒ विश्वा॒ न्यङ्गा॑नि त॒नूः । \newline
53. अङ्गा॑नि त॒नू स्त॒नू रङ्गा॒ न्यङ्गा॑नि त॒नू स्त॒नुवा॑ त॒नुवा॑ त॒नू रङ्गा॒ न्यङ्गा॑नि त॒नू स्त॒नुवा᳚ । \newline
54. त॒नू स्त॒नुवा॑ त॒नुवा॑ त॒नू स्त॒नू स्त॒नुवा॑ मे मे त॒नुवा॑ त॒नू स्त॒नू स्त॒नुवा॑ मे । \newline
\pagebreak
\markright{ TS 5.5.9.3  \hfill https://www.vedavms.in \hfill}

\section{ TS 5.5.9.3 }

\textbf{TS 5.5.9.3 } \newline
\textbf{Samhita Paata} \newline

-स्त॒नुवा॑ मे स॒ह नम॑स्ते अस्तु॒ मा मा॑ हिꣳसी॒रप॒ वा ए॒तस्मा᳚त् प्रा॒णाः क्रा॑मन्ति॒ यो᳚ऽग्निं चि॒न्वन्न॑धि॒ क्राम॑ति॒ वाङ्म॑ आ॒सन् न॒सोः प्रा॒ण इत्या॑ह प्रा॒णाने॒वा*ऽऽत्मन् ध॑त्ते॒ यो रु॒द्रो अ॒ग्नौ यो अ॒फ्सु य ओष॑धीषु॒ यो रु॒द्रो विश्वा॒ भुव॑नाऽऽवि॒वेश॒ तस्मै॑ रु॒द्राय॒ नमो॑ अ॒स्त्वाहु॑तिभागा॒ वा अ॒न्ये रु॒द्रा ह॒विर्भा॑गा - [  ] \newline

\textbf{Pada Paata} \newline

त॒नुवा᳚ । मे॒ । स॒ह । नमः॑ । ते॒ । अ॒स्तु॒ । मा । मा॒ । हिꣳ॒॒सीः॒ । अपेति॑ । वै । ए॒तस्मा᳚त् । प्रा॒णा इति॑ प्र - अ॒नाः । क्रा॒म॒न्ति॒ । यः । अ॒ग्निम् । चि॒न्वन्न् । अ॒धि॒क्राम॒तीत्य॑धि - क्राम॑ति । वाक् । मे॒ । आ॒सन्न् । न॒सोः । प्रा॒ण इति॑ प्र - अ॒नः । इति॑ । आ॒ह॒ । प्रा॒णानिति॑ प्र - अ॒नान् । ए॒व । आ॒त्मन्न् । ध॒त्ते॒ । यः । रु॒द्र ः । अ॒ग्नौ । यः । अ॒फ्स्वित्य॑प् - सु । यः । ओष॑धीषु । यः । रु॒द्रः । विश्वा᳚ । भुव॑ना । आ॒वि॒वेशेत्या᳚ - वि॒वेश॑ । तस्मै᳚ । रु॒द्राय॑ । नमः॑ । अ॒स्तु॒ । आहु॑तिभागा॒ इत्याहु॑ति - भा॒गाः॒ । वै । अ॒न्ये । रु॒द्राः । ह॒विर्भा॑गा॒ इति॑ ह॒विः - भा॒गाः॒ ।  \newline


\textbf{Krama Paata} \newline

त॒नुवा॑ मे । मे॒ स॒ह । स॒ह नमः॑ । नम॑स्ते । ते॒ अ॒स्तु॒ । अ॒स्तु॒ मा । मा मा᳚ । मा॒ हिꣳ॒॒सीः॒ । हिꣳ॒॒सी॒रप॑ । अप॒ वै । वा ए॒तस्मा᳚त् । ए॒तस्मा᳚त् प्रा॒णाः । प्रा॒णाः क्रा॑मन्ति । प्रा॒णा इति॑ प्र - अ॒नाः । क्रा॒म॒न्ति॒ यः । यो᳚ऽग्निम् । अ॒ग्निम् चि॒न्वन्न् । चि॒न्वन्न॑धि॒क्राम॑ति । अ॒धि॒क्राम॑ति॒ वाक् । अ॒धि॒क्राम॒तीत्य॑धि - क्राम॑ति । वाङ् मे᳚ । म॒ आ॒सन्न् । आ॒सन् न॒सोः । न॒सोः प्रा॒णः । प्रा॒ण इति॑ । प्रा॒ण इति॑ प्र - अ॒नः । इत्या॑ह । आ॒ह॒ प्रा॒णान् । प्रा॒णाने॒व । प्रा॒णानिति॑ प्र - अ॒नान् । ए॒वात्मन्न् । आ॒त्मन् ध॑त्ते । ध॒त्ते॒ यः । यो रु॒द्रः । रु॒द्रो अ॒ग्नौ । अ॒ग्नौ यः । यो अ॒फ्सु । अ॒फ्सु यः । अ॒फ्स्वित्य॑प् - सु । य ओष॑धीषु । ओष॑धीषु॒ यः । यो रु॒द्रः । रु॒द्रो विश्वा᳚ । विश्वा॒ भुव॑ना । भुव॑नाऽऽवि॒वेश॑ । आ॒वि॒वेश॒ तस्मै᳚ । आ॒वि॒वेशेत्या᳚ - वि॒वेश॑ । तस्मै॑ रु॒द्राय॑ । रु॒द्राय॒ नमः॑ । नमो॑ अस्तु । अ॒स्त्वाहु॑तिभागाः । आहु॑तिभागा॒ वै । आहु॑तिभागा॒ इत्या॑हुति - भा॒गः॒ । वा अ॒न्ये । अ॒न्ये रु॒द्राः । रु॒द्रा ह॒विर्भा॑गाः । ह॒विर्भा॑गा अ॒न्ये । ह॒विर्भा॑गा॒ इति॑ ह॒विः - भा॒गाः॒ \newline

\textbf{Jatai Paata} \newline

1. त॒नुवा॑ मे मे त॒नुवा॑ त॒नुवा॑ मे । \newline
2. मे॒ स॒ह स॒ह मे॑ मे स॒ह । \newline
3. स॒ह नमो॒ नमः॑ स॒ह स॒ह नमः॑ । \newline
4. नम॑ स्ते ते॒ नमो॒ नम॑ स्ते । \newline
5. ते॒ अ॒स्त्व॒स्तु॒ ते॒ ते॒ अ॒स्तु॒ । \newline
6. अ॒स्तु॒ मा मा अ॑स्त्वस्तु॒ मा । \newline
7. मा मा॑ मा॒ मा मा मा᳚ । \newline
8. मा॒ हिꣳ॒॒सी॒र्॒. हिꣳ॒॒सी॒र् मा॒ मा॒ हिꣳ॒॒सीः॒ । \newline
9. हिꣳ॒॒सी॒ रपाप॑ हिꣳसीर्. हिꣳसी॒ रप॑ । \newline
10. अप॒ वै वा अपाप॒ वै । \newline
11. वा ए॒तस्मा॑ दे॒तस्मा॒द् वै वा ए॒तस्मा᳚त् । \newline
12. ए॒तस्मा᳚त् प्रा॒णः प्रा॒ण ए॒तस्मा॑ दे॒तस्मा᳚त् प्रा॒णः । \newline
13. प्रा॒णः क्रा॑मन्ति क्रामन्ति प्रा॒णः प्रा॒णः क्रा॑मन्ति । \newline
14. प्रा॒णा इति॑ प्र - अ॒नाः । \newline
15. क्रा॒म॒न्ति॒ यो यः क्रा॑मन्ति क्रामन्ति॒ यः । \newline
16. यो᳚ ऽग्नि म॒ग्निं ॅयो यो᳚ ऽग्निम् । \newline
17. अ॒ग्निम् चि॒न्वꣳ श्चि॒न्वन् न॒ग्नि म॒ग्निम् चि॒न्वन्न् । \newline
18. छि॒न्वन् न॑धि॒क्राम॑ त्यधि॒क्राम॑ति चि॒न्वꣳ श्चि॒न्वन् न॑धि॒क्राम॑ति । \newline
19. अ॒धि॒क्राम॑ति॒ वाग् वाग॑धि॒क्राम॑ त्यधि॒क्राम॑ति॒ वाक् । \newline
20. अ॒धि॒क्राम॒तीत्य॑धि - क्राम॑ति । \newline
21. वाङ् मे॑ मे॒ वाग् वाङ् मे᳚ । \newline
22. म॒ आ॒सन् ना॒सन् मे॑ म आ॒सन्न् । \newline
23. आ॒सन् न॒सोर् न॒सो रा॒सन् ना॒सन् न॒सोः । \newline
24. न॒सोः प्रा॒णः प्रा॒णो न॒सोर् न॒सोः प्रा॒णः । \newline
25. प्रा॒ण इतीति॑ प्रा॒णः प्रा॒ण इति॑ । \newline
26. प्रा॒ण इति॑ प्र - अ॒नः । \newline
27. इत्या॑हा॒हे तीत्या॑ह । \newline
28. आ॒ह॒ प्रा॒णान् प्रा॒णा ना॑हाह प्रा॒णान् । \newline
29. प्रा॒णा ने॒वैव प्रा॒णान् प्रा॒णा ने॒व । \newline
30. प्रा॒णानिति॑ प्र - अ॒नान् । \newline
31. ए॒वात्मन् ना॒त्मन् ने॒वै वात्मन्न् । \newline
32. आ॒त्मन् ध॑त्ते धत्त आ॒त्मन् ना॒त्मन् ध॑त्ते । \newline
33. ध॒त्ते॒ यो यो ध॑त्ते धत्ते॒ यः । \newline
34. यो रु॒द्रो रु॒द्रो यो यो रु॒द्रः । \newline
35. रु॒द्रो अ॒ग्ना व॒ग्नौ रु॒द्रो रु॒द्रो अ॒ग्नौ । \newline
36. अ॒ग्नौ यो यो᳚ ऽग्ना व॒ग्नौ यः । \newline
37. यो अ॒फ्स्व॑फ्सु यो यो अ॒फ्सु । \newline
38. अ॒फ्सु यो यो᳚ (1॒)फ्स्व॑फ्सु यः । \newline
39. अ॒फ्स्वित्य॑प् - सु । \newline
40. य ओष॑धी॒ ष्वोष॑धीषु॒ यो य ओष॑धीषु । \newline
41. ओष॑धीषु॒ यो य ओष॑धी॒ ष्वोष॑धीषु॒ यः । \newline
42. यो रु॒द्रो रु॒द्रो यो यो रु॒द्रः । \newline
43. रु॒द्रो विश्वा॒ विश्वा॑ रु॒द्रो रु॒द्रो विश्वा᳚ । \newline
44. विश्वा॒ भुव॑ना॒ भुव॑ना॒ विश्वा॒ विश्वा॒ भुव॑ना । \newline
45. भुव॑ना ऽऽवि॒वेशा॑ वि॒वेश॒ भुव॑ना॒ भुव॑ना ऽऽवि॒वेश॑ । \newline
46. आ॒वि॒वेश॒ तस्मै॒ तस्मा॑ आवि॒वेशा॑ वि॒वेश॒ तस्मै᳚ । \newline
47. आ॒वि॒वेशेत्या᳚ - वि॒वेश॑ । \newline
48. तस्मै॑ रु॒द्राय॑ रु॒द्राय॒ तस्मै॒ तस्मै॑ रु॒द्राय॑ । \newline
49. रु॒द्राय॒ नमो॒ नमो॑ रु॒द्राय॑ रु॒द्राय॒ नमः॑ । \newline
50. नमो॑ अस्त्वस्तु॒ नमो॒ नमो॑ अस्तु । \newline
51. अ॒स्त्वाहु॑तिभागा॒ आहु॑तिभागा अस्त्व॒स्त्वा हु॑तिभागाः । \newline
52. आहु॑तिभागा॒ वै वा आहु॑तिभागा॒ आहु॑तिभागा॒ वै । \newline
53. आहु॑तिभागा॒ इत्याहु॑ति - भा॒गाः॒ । \newline
54. वा अ॒न्ये᳚ ऽन्ये वै वा अ॒न्ये । \newline
55. अ॒न्ये रु॒द्रा रु॒द्रा अ॒न्ये᳚ ऽन्ये रु॒द्राः । \newline
56. रु॒द्रा ह॒विर्भा॑गा ह॒विर्भा॑गा रु॒द्रा रु॒द्रा ह॒विर्भा॑गाः । \newline
57. ह॒विर्भा॑गा अ॒न्ये᳚ ऽन्ये ह॒विर्भा॑गा ह॒विर्भा॑गा अ॒न्ये । \newline
58. ह॒विर्भा॑गा॒ इति॑ ह॒विः - भा॒गाः॒ । \newline

\textbf{Ghana Paata } \newline

1. त॒नुवा॑ मे मे त॒नुवा॑ त॒नुवा॑ मे स॒ह स॒ह मे॑ त॒नुवा॑ त॒नुवा॑ मे स॒ह । \newline
2. मे॒ स॒ह स॒ह मे॑ मे स॒ह नमो॒ नमः॑ स॒ह मे॑ मे स॒ह नमः॑ । \newline
3. स॒ह नमो॒ नमः॑ स॒ह स॒ह नम॑ स्ते ते॒ नमः॑ स॒ह स॒ह नम॑ स्ते । \newline
4. नम॑ स्ते ते॒ नमो॒ नम॑ स्ते अस्त्वस्तु ते॒ नमो॒ नम॑ स्ते अस्तु । \newline
5. ते॒ अ॒स्त्व॒स्तु॒ ते॒ ते॒ अ॒स्तु॒ मा मा अ॑स्तु ते ते अस्तु॒ मा । \newline
6. अ॒स्तु॒ मा मा अ॑स्त्वस्तु॒ मा मा॑ मा॒ मा अ॑स्त्वस्तु॒ मा मा᳚ । \newline
7. मा मा॑ मा॒ मा मा मा॑ हिꣳसीर्. हिꣳसीर् मा॒ मा मा मा॑ हिꣳसीः । \newline
8. मा॒ हिꣳ॒॒सी॒र्॒. हिꣳ॒॒सी॒र् मा॒ मा॒ हिꣳ॒॒सी॒ रपाप॑ हिꣳसीर् मा मा हिꣳसी॒रप॑ । \newline
9. हिꣳ॒॒सी॒ रपाप॑ हिꣳसीर्. हिꣳसी॒ रप॒ वै वा अप॑ हिꣳसीर्. हिꣳसी॒ रप॒ वै । \newline
10. अप॒ वै वा अपाप॒ वा ए॒तस्मा॑ दे॒तस्मा॒द् वा अपाप॒ वा ए॒तस्मा᳚त् । \newline
11. वा ए॒तस्मा॑ दे॒तस्मा॒द् वै वा ए॒तस्मा᳚त् प्रा॒णः प्रा॒ण ए॒तस्मा॒द् वै वा ए॒तस्मा᳚त् प्रा॒णः । \newline
12. ए॒तस्मा᳚त् प्रा॒णः प्रा॒ण ए॒तस्मा॑ दे॒तस्मा᳚त् प्रा॒णः क्रा॑मन्ति क्रामन्ति प्रा॒ण ए॒तस्मा॑ दे॒तस्मा᳚त् प्रा॒णः क्रा॑मन्ति । \newline
13. प्रा॒णः क्रा॑मन्ति क्रामन्ति प्रा॒णः प्रा॒णः क्रा॑मन्ति॒ यो यः क्रा॑मन्ति प्रा॒णः प्रा॒णः क्रा॑मन्ति॒ यः । \newline
14. प्रा॒णा इति॑ प्र - अ॒नाः । \newline
15. क्रा॒म॒न्ति॒ यो यः क्रा॑मन्ति क्रामन्ति॒ यो᳚ ऽग्नि म॒ग्निं ॅयः क्रा॑मन्ति क्रामन्ति॒ यो᳚ ऽग्निम् । \newline
16. यो᳚ ऽग्नि म॒ग्निं ॅयो यो᳚ ऽग्निम् चि॒न्वꣳ श्चि॒न्वन् न॒ग्निं ॅयो यो᳚ ऽग्निम् चि॒न्वन्न् । \newline
17. अ॒ग्निम् चि॒न्वꣳ श्चि॒न्वन् न॒ग्नि म॒ग्निम् चि॒न्वन् न॑धि॒क्राम॑ त्यधि॒क्राम॑ति चि॒न्वन् न॒ग्नि म॒ग्निम् चि॒न्वन् न॑धि॒क्राम॑ति । \newline
18. छि॒न्वन् न॑धि॒क्राम॑ त्यधि॒क्राम॑ति चि॒न्वꣳ श्चि॒न्वन् न॑धि॒क्राम॑ति॒ वाग् वाग॑धि॒क्राम॑ति चि॒न्वꣳ श्चि॒न्वन् न॑धि॒क्राम॑ति॒ वाक् । \newline
19. अ॒धि॒क्राम॑ति॒ वाग् वाग॑धि॒क्राम॑ त्यधि॒क्राम॑ति॒ वाङ् मे॑ मे॒ वाग॑धि॒क्राम॑ त्यधि॒क्राम॑ति॒ वाङ् मे᳚ । \newline
20. अ॒धि॒क्राम॒तीत्य॑धि - क्राम॑ति । \newline
21. वाङ् मे॑ मे॒ वाग् वाङ् म॑ आ॒सन् ना॒सन् मे॒ वाग् वाङ् म॑ आ॒सन्न् । \newline
22. म॒ आ॒सन् ना॒सन् मे॑ म आ॒सन् न॒सोर् न॒सो रा॒सन् मे॑ म आ॒सन् न॒सोः । \newline
23. आ॒सन् न॒सोर् न॒सो रा॒सन् ना॒सन् न॒सोः प्रा॒णः प्रा॒णो न॒सो रा॒सन् ना॒सन् न॒सोः प्रा॒णः । \newline
24. न॒सोः प्रा॒णः प्रा॒णो न॒सोर् न॒सोः प्रा॒ण इतीति॑ प्रा॒णो न॒सोर् न॒सोः प्रा॒ण इति॑ । \newline
25. प्रा॒ण इतीति॑ प्रा॒णः प्रा॒ण इत्या॑हा॒हेति॑ प्रा॒णः प्रा॒ण इत्या॑ह । \newline
26. प्रा॒ण इति॑ प्र - अ॒नः । \newline
27. इत्या॑हा॒हे तीत्या॑ह प्रा॒णान् प्रा॒णा ना॒हे तीत्या॑ह प्रा॒णान् । \newline
28. आ॒ह॒ प्रा॒णान् प्रा॒णा ना॑हाह प्रा॒णा ने॒वैव प्रा॒णा ना॑हाह प्रा॒णा ने॒व । \newline
29. प्रा॒णा ने॒वैव प्रा॒णान् प्रा॒णा ने॒वात्मन् ना॒त्मन् ने॒व प्रा॒णान् प्रा॒णा ने॒वात्मन्न् । \newline
30. प्रा॒णानिति॑ प्र - अ॒नान् । \newline
31. ए॒वात्मन् ना॒त्मन् ने॒वै वात्मन् ध॑त्ते धत्त आ॒त्मन् ने॒वै वात्मन् ध॑त्ते । \newline
32. आ॒त्मन् ध॑त्ते धत्त आ॒त्मन् ना॒त्मन् ध॑त्ते॒ यो यो ध॑त्त आ॒त्मन् ना॒त्मन् ध॑त्ते॒ यः । \newline
33. ध॒त्ते॒ यो यो ध॑त्ते धत्ते॒ यो रु॒द्रो रु॒द्रो यो ध॑त्ते धत्ते॒ यो रु॒द्रः । \newline
34. यो रु॒द्रो रु॒द्रो यो यो रु॒द्रो अ॒ग्ना व॒ग्नौ रु॒द्रो यो यो रु॒द्रो अ॒ग्नौ । \newline
35. रु॒द्रो अ॒ग्ना व॒ग्नौ रु॒द्रो रु॒द्रो अ॒ग्नौ यो यो᳚ ऽग्नौ रु॒द्रो रु॒द्रो अ॒ग्नौ यः । \newline
36. अ॒ग्नौ यो यो᳚ ऽग्ना व॒ग्नौ यो अ॒फ्स्व॑फ्सु यो᳚ ऽग्ना व॒ग्नौ यो अ॒फ्सु । \newline
37. यो अ॒फ्स्व॑फ्सु यो यो अ॒फ्सु यो यो अ॒फ्सु यो यो अ॒फ्सु यः । \newline
38. अ॒फ्सु यो यो᳚ (1॒)फ्स्व॑फ्सु य ओष॑धी॒ ष्वोष॑धीषु॒ यो᳚ (1॒) फ्स्व॑फ्सु य ओष॑धीषु । \newline
39. अ॒फ्स्वित्य॑प् - सु । \newline
40. य ओष॑धी॒ ष्वोष॑धीषु॒ यो य ओष॑धीषु॒ यो य ओष॑धीषु॒ यो य ओष॑धीषु॒ यः । \newline
41. ओष॑धीषु॒ यो य ओष॑धी॒ ष्वोष॑धीषु॒ यो रु॒द्रो रु॒द्रो य ओष॑धी॒ ष्वोष॑धीषु॒ यो रु॒द्रः । \newline
42. यो रु॒द्रो रु॒द्रो यो यो रु॒द्रो विश्वा॒ विश्वा॑ रु॒द्रो यो यो रु॒द्रो विश्वा᳚ । \newline
43. रु॒द्रो विश्वा॒ विश्वा॑ रु॒द्रो रु॒द्रो विश्वा॒ भुव॑ना॒ भुव॑ना॒ विश्वा॑ रु॒द्रो रु॒द्रो विश्वा॒ भुव॑ना । \newline
44. विश्वा॒ भुव॑ना॒ भुव॑ना॒ विश्वा॒ विश्वा॒ भुव॑ना ऽऽवि॒वेशा॑ वि॒वेश॒ भुव॑ना॒ विश्वा॒ विश्वा॒ भुव॑ना ऽऽवि॒वेश॑ । \newline
45. भुव॑ना ऽऽवि॒वेशा॑ वि॒वेश॒ भुव॑ना॒ भुव॑ना ऽऽवि॒वेश॒ तस्मै॒ तस्मा॑ आवि॒वेश॒ भुव॑ना॒ भुव॑ना ऽऽवि॒वेश॒ तस्मै᳚ । \newline
46. आ॒वि॒वेश॒ तस्मै॒ तस्मा॑ आवि॒वेशा॑ वि॒वेश॒ तस्मै॑ रु॒द्राय॑ रु॒द्राय॒ तस्मा॑ आवि॒वेशा॑ वि॒वेश॒ तस्मै॑ रु॒द्राय॑ । \newline
47. आ॒वि॒वेशेत्या᳚ - वि॒वेश॑ । \newline
48. तस्मै॑ रु॒द्राय॑ रु॒द्राय॒ तस्मै॒ तस्मै॑ रु॒द्राय॒ नमो॒ नमो॑ रु॒द्राय॒ तस्मै॒ तस्मै॑ रु॒द्राय॒ नमः॑ । \newline
49. रु॒द्राय॒ नमो॒ नमो॑ रु॒द्राय॑ रु॒द्राय॒ नमो॑ अस्त्वस्तु॒ नमो॑ रु॒द्राय॑ रु॒द्राय॒ नमो॑ अस्तु । \newline
50. नमो॑ अस्त्वस्तु॒ नमो॒ नमो॑ अ॒स्त्वाहु॑तिभागा॒ आहु॑तिभागा अस्तु॒ नमो॒ नमो॑ अ॒स्त्वाहु॑तिभागाः । \newline
51. अ॒स्त्वा हु॑तिभागा॒ आहु॑तिभागा अस्त्व॒ स्त्वाहु॑तिभागा॒ वै वा आहु॑तिभागा अस्त्व॒ स्त्वाहु॑तिभागा॒ वै । \newline
52. आहु॑तिभागा॒ वै वा आहु॑तिभागा॒ आहु॑तिभागा॒ वा अ॒न्ये᳚ ऽन्ये वा आहु॑तिभागा॒ आहु॑तिभागा॒ वा अ॒न्ये । \newline
53. आहु॑तिभागा॒ इत्याहु॑ति - भा॒गाः॒ । \newline
54. वा अ॒न्ये᳚ ऽन्ये वै वा अ॒न्ये रु॒द्रा रु॒द्रा अ॒न्ये वै वा अ॒न्ये रु॒द्राः । \newline
55. अ॒न्ये रु॒द्रा रु॒द्रा अ॒न्ये᳚ ऽन्ये रु॒द्रा ह॒विर्भा॑गा ह॒विर्भा॑गा रु॒द्रा अ॒न्ये᳚ ऽन्ये रु॒द्रा ह॒विर्भा॑गाः । \newline
56. रु॒द्रा ह॒विर्भा॑गा ह॒विर्भा॑गा रु॒द्रा रु॒द्रा ह॒विर्भा॑गा अ॒न्ये᳚ ऽन्ये ह॒विर्भा॑गा रु॒द्रा रु॒द्रा ह॒विर्भा॑गा अ॒न्ये । \newline
57. ह॒विर्भा॑गा अ॒न्ये᳚ ऽन्ये ह॒विर्भा॑गा ह॒विर्भा॑गा अ॒न्ये श॑तरु॒द्रीयꣳ॑ शतरु॒द्रीय॑ म॒न्ये ह॒विर्भा॑गा ह॒विर्भा॑गा अ॒न्ये श॑तरु॒द्रीय᳚म् । \newline
58. ह॒विर्भा॑गा॒ इति॑ ह॒विः - भा॒गाः॒ । \newline
\pagebreak
\markright{ TS 5.5.9.4  \hfill https://www.vedavms.in \hfill}

\section{ TS 5.5.9.4 }

\textbf{TS 5.5.9.4 } \newline
\textbf{Samhita Paata} \newline

अ॒न्ये श॑तरु॒द्रीयꣳ॑ हु॒त्वा गा॑वीधु॒कं च॒रुमे॒तेन॒ यजु॑षा चर॒माया॒मिष्ट॑कायां॒ नि द॑द्ध्याद्-भाग॒धेये॑नै॒वैनꣳ॑ शमयति॒ तस्य॒ त्वै श॑तरु॒द्रीयꣳ॑ हु॒तमित्या॑हु॒र्यस्यै॒तद॒ग्नौ क्रि॒यत॒ इति॒ वस॑वस्त्वा रु॒द्रैः पु॒रस्ता᳚त् पान्तु पि॒तर॑स्त्वा य॒मरा॑जानः पि॒तृभि॑र्दक्षिण॒तः पा᳚न्त्वादि॒त्यास्त्वा॒ विश्वै᳚र्दे॒वैः प॒श्चात् पा᳚न्तु द्युता॒नस्त्वा॑ मारु॒तो म॒रुद्भि॑रुत्तर॒तः पा॑तु - [  ] \newline

\textbf{Pada Paata} \newline

अ॒न्ये । श॒त॒रु॒द्रीय॒मिति॑ शत - रु॒द्रीय᳚म् । हु॒त्वा । गा॒वी॒धु॒कम् । च॒रुम् । ए॒तेन॑ । यजु॑षा । च॒र॒माया᳚म् । इष्ट॑कायाम् । नीति॑ । द॒द्ध्या॒त् । भा॒ग॒धेये॒नेति॑ भाग - धेये॑न । ए॒व । ए॒न॒म् । श॒म॒य॒ति॒ । तस्य॑ । तु । वै । श॒त॒रु॒द्रीय॒मिति॑ शत - रु॒द्रीय᳚म् । हु॒तम् । इति॑ । आ॒हुः॒ । यस्य॑ । ए॒तत् । अ॒ग्नौ । क्रि॒यते᳚ । इति॑ । वस॑वः । त्वा॒ । रु॒द्रैः । पु॒रस्ता᳚त् । पा॒न्तु॒ । पि॒तरः॑ । त्वा॒ । य॒मरा॑जान॒ इति॑ य॒म - रा॒जा॒नः॒ । पि॒तृभि॒रिति॑ पि॒तृ - भिः॒ । द॒क्षि॒ण॒तः । पा॒न्तु॒ । आ॒दि॒त्याः । त्वा॒ । विश्वैः᳚ । दे॒वैः । प॒श्चात् । पा॒न्तु॒ । द्यु॒ता॒नः । त्वा॒ । मा॒रु॒तः । म॒रुद्भि॒रिति॑ म॒रुत् - भिः॒ । उ॒त्त॒र॒त इत्यु॑त् - त॒र॒तः । पा॒तु॒ ।  \newline


\textbf{Krama Paata} \newline

अ॒न्ये श॑तरु॒द्रीय᳚म् । श॒त॒रु॒द्रीयꣳ॑ हु॒त्वा । श॒त॒रु॒द्रीय॒मिति॑ शत - रु॒द्रीय᳚म् । हु॒त्वा गा॑वीधु॒कम् । गा॒वी॒धु॒कम् च॒रुम् । च॒रुमे॒तेन॑ । ए॒तेन॒ यजु॑षा । यजु॑षा चर॒माया᳚म् । च॒र॒माया॒मिष्ट॑कायाम् । इष्ट॑काया॒म् नि । नि द॑द्ध्यात् । द॒द्ध्या॒द् भा॒ग॒धेये॑न । भा॒ग॒धेये॑नै॒व । भा॒ग॒धेये॒नेति॑ भाग - धेये॑न । ए॒वैन᳚म् । ए॒नꣳ॒॒ श॒म॒य॒ति॒ । श॒म॒य॒ति॒ तस्य॑ । तस्य॒ तु । त्वै । वै श॑तरु॒द्रीय᳚म् । श॒त॒रु॒द्रीयꣳ॑ हु॒तम् । श॒त॒रु॒द्रीय॒मिति॑ शत - रु॒द्रीय᳚म् । हु॒तमिति॑ । इत्या॑हुः । आ॒हु॒र्यस्य॑ । यस्यै॒तत् । ए॒तद॒ग्नौ । अ॒ग्नौ क्रि॒यते᳚ । क्रि॒यत॒ इति॑ । इति॒ वस॑वः । वस॑वस्त्वा । त्वा॒ रु॒द्रैः । रु॒द्रैः पु॒रस्ता᳚त् । पु॒रस्ता᳚त् पान्तु । पा॒न्तु॒ पि॒तरः॑ । पि॒तर॑स्त्वा । त्वा॒ य॒मरा॑जानः । य॒मरा॑जानः पि॒तृभिः॑ । य॒मरा॑जान॒ इति॑ य॒म - रा॒जा॒नः॒ । पि॒तृभि॑र् दक्षिण॒तः । पि॒तृभि॒रिति॑ पि॒तृ - भिः॒ । द॒क्षि॒ण॒तः पा᳚न्तु । पा॒न्त्वा॒दि॒त्याः । आ॒दि॒त्यास्त्वा᳚ । त्वा॒ विश्वैः᳚ । विश्वै᳚र् दे॒वैः । दे॒वैः प॒श्चात् । प॒श्चात् पा᳚न्तु । पा॒न्तु॒ द्यु॒ता॒नः । द्यु॒ता॒नस्त्वा᳚ । त्वा॒ मा॒रु॒तः । मा॒रु॒तो म॒रुद्भिः॑ । म॒रुद्भि॑रुत्तर॒तः । म॒रुद्भि॒रिति॑ म॒रुत् - भिः॒ । उ॒त्त॒र॒तः पा॑तु ( ) । उ॒त्त॒र॒त इत्यु॑त् - त॒र॒तः । पा॒तु॒ दे॒वाः \newline

\textbf{Jatai Paata} \newline

1. अ॒न्ये श॑तरु॒द्रीयꣳ॑ शतरु॒द्रीय॑ म॒न्ये᳚ ऽन्ये श॑तरु॒द्रीय᳚म् । \newline
2. श॒त॒रु॒द्रीयꣳ॑ हु॒त्वा हु॒त्वा श॑तरु॒द्रीयꣳ॑ शतरु॒द्रीयꣳ॑ हु॒त्वा । \newline
3. श॒त॒रु॒द्रीय॒मिति॑ शत - रु॒द्रीय᳚म् । \newline
4. हु॒त्वा गा॑वीधु॒कम् गा॑वीधु॒कꣳ हु॒त्वा हु॒त्वा गा॑वीधु॒कम् । \newline
5. गा॒वी॒धु॒कम् च॒रुम् च॒रुम् गा॑वीधु॒कम् गा॑वीधु॒कम् च॒रुम् । \newline
6. च॒रु मे॒तेनै॒ तेन॑ च॒रुम् च॒रु मे॒तेन॑ । \newline
7. ए॒तेन॒ यजु॑षा॒ यजु॑षै॒ तेनै॒ तेन॒ यजु॑षा । \newline
8. यजु॑षा चर॒माया᳚म् चर॒मायां॒ ॅयजु॑षा॒ यजु॑षा चर॒माया᳚म् । \newline
9. च॒र॒माया॒ मिष्ट॑काया॒ मिष्ट॑कायाम् चर॒माया᳚म् चर॒माया॒ मिष्ट॑कायाम् । \newline
10. इष्ट॑काया॒म् नि नीष्ट॑काया॒ मिष्ट॑काया॒म् नि । \newline
11. नि द॑द्ध्याद् दद्ध्या॒न् नि नि द॑द्ध्यात् । \newline
12. द॒द्ध्या॒द् भा॒ग॒धेये॑न भाग॒धेये॑न दद्ध्याद् दद्ध्याद् भाग॒धेये॑न । \newline
13. भा॒ग॒धेये॑ नै॒वैव भा॑ग॒धेये॑न भाग॒धेये॑नै॒व । \newline
14. भा॒ग॒धेये॒नेति॑ भाग - धेये॑न । \newline
15. ए॒वैन॑ मेन मे॒वै वैन᳚म् । \newline
16. ए॒नꣳ॒॒ श॒म॒य॒ति॒ श॒म॒य॒ त्ये॒न॒ मे॒नꣳ॒॒ श॒म॒य॒ति॒ । \newline
17. श॒म॒य॒ति॒ तस्य॒ तस्य॑ शमयति शमयति॒ तस्य॑ । \newline
18. तस्य॒ तु तु तस्य॒ तस्य॒ तु । \newline
19. त्वै वै तु त्वै । \newline
20. वै श॑तरु॒द्रीयꣳ॑ शतरु॒द्रीयं॒ ॅवै वै श॑तरु॒द्रीय᳚म् । \newline
21. श॒त॒रु॒द्रीयꣳ॑ हु॒तꣳ हु॒तꣳ श॑तरु॒द्रीयꣳ॑ शतरु॒द्रीयꣳ॑ हु॒तम् । \newline
22. श॒त॒रु॒द्रीय॒मिति॑ शत - रु॒द्रीय᳚म् । \newline
23. हु॒त मितीति॑ हु॒तꣳ हु॒त मिति॑ । \newline
24. इत्या॑हु राहु॒ रिती त्या॑हुः । \newline
25. आ॒हु॒र् यस्य॒ यस्या॑हु राहु॒र् यस्य॑ । \newline
26. यस्यै॒त दे॒तद् यस्य॒ यस्यै॒तत् । \newline
27. ए॒त द॒ग्ना व॒ग्ना वे॒त दे॒त द॒ग्नौ । \newline
28. अ॒ग्नौ क्रि॒यते᳚ क्रि॒यते॒ ऽग्ना व॒ग्नौ क्रि॒यते᳚ । \newline
29. क्रि॒यत॒ इतीति॑ क्रि॒यते᳚ क्रि॒यत॒ इति॑ । \newline
30. इति॒ वस॑वो॒ वस॑व॒ इतीति॒ वस॑वः । \newline
31. वस॑व स्त्वा त्वा॒ वस॑वो॒ वस॑व स्त्वा । \newline
32. त्वा॒ रु॒द्रै रु॒द्रै स्त्वा᳚ त्वा रु॒द्रैः । \newline
33. रु॒द्रैः पु॒रस्ता᳚त् पु॒रस्ता᳚द् रु॒द्रै रु॒द्रैः पु॒रस्ता᳚त् । \newline
34. पु॒रस्ता᳚त् पान्तु पान्तु पु॒रस्ता᳚त् पु॒रस्ता᳚त् पान्तु । \newline
35. पा॒न्तु॒ पि॒तरः॑ पि॒तरः॑ पान्तु पान्तु पि॒तरः॑ । \newline
36. पि॒तर॑ स्त्वा त्वा पि॒तरः॑ पि॒तर॑ स्त्वा । \newline
37. त्वा॒ य॒मरा॑जानो य॒मरा॑जान स्त्वा त्वा य॒मरा॑जानः । \newline
38. य॒मरा॑जानः पि॒तृभिः॑ पि॒तृभि॑र् य॒मरा॑जानो य॒मरा॑जानः पि॒तृभिः॑ । \newline
39. य॒मरा॑जान॒ इति॑ य॒म - रा॒जा॒नः॒ । \newline
40. पि॒तृभि॑र् दक्षिण॒तो द॑क्षिण॒तः पि॒तृभिः॑ पि॒तृभि॑र् दक्षिण॒तः । \newline
41. पि॒तृभि॒रिति॑ पि॒तृ - भिः॒ । \newline
42. द॒क्षि॒ण॒तः पा᳚न्तु पान्तु दक्षिण॒तो द॑क्षिण॒तः पा᳚न्तु । \newline
43. पा॒न् त्वा॒दि॒त्या आ॑दि॒त्याः पा᳚न्तु पान् त्वादि॒त्याः । \newline
44. आ॒दि॒त्या स्त्वा᳚ त्वा ऽऽदि॒त्या आ॑दि॒त्या स्त्वा᳚ । \newline
45. त्वा॒ विश्वै॒र् विश्वै᳚ स्त्वा त्वा॒ विश्वैः᳚ । \newline
46. विश्वै᳚र् दे॒वैर् दे॒वैर् विश्वै॒र् विश्वै᳚र् दे॒वैः । \newline
47. दे॒वैः प॒श्चात् प॒श्चाद् दे॒वैर् दे॒वैः प॒श्चात् । \newline
48. प॒श्चात् पा᳚न्तु पान्तु प॒श्चात् प॒श्चात् पा᳚न्तु । \newline
49. पा॒न्तु॒ द्यु॒ता॒नो द्यु॑ता॒नः पा᳚न्तु पान्तु द्युता॒नः । \newline
50. द्यु॒ता॒न स्त्वा᳚ त्वा द्युता॒नो द्यु॑ता॒न स्त्वा᳚ । \newline
51. त्वा॒ मा॒रु॒तो मा॑रु॒त स्त्वा᳚ त्वा मारु॒तः । \newline
52. मा॒रु॒तो म॒रुद्भि॑र् म॒रुद्भि॑र् मारु॒तो मा॑रु॒तो म॒रुद्भिः॑ । \newline
53. म॒रुद्भि॑ रुत्तर॒त उ॑त्तर॒तो म॒रुद्भि॑र् म॒रुद्भि॑ रुत्तर॒तः । \newline
54. म॒रुद्भि॒रिति॑ म॒रुत् - भिः॒ । \newline
55. उ॒त्त॒र॒तः पा॑तु पातूत्तर॒त उ॑त्तर॒तः पा॑तु । \newline
56. उ॒त्त॒र॒त इत्यु॑त् - त॒र॒तः । \newline
57. पा॒तु॒ दे॒वा दे॒वाः पा॑तु पातु दे॒वाः । \newline

\textbf{Ghana Paata } \newline

1. अ॒न्ये श॑तरु॒द्रीयꣳ॑ शतरु॒द्रीय॑ म॒न्ये᳚ ऽन्ये श॑तरु॒द्रीयꣳ॑ हु॒त्वा हु॒त्वा श॑तरु॒द्रीय॑ म॒न्ये᳚ ऽन्ये श॑तरु॒द्रीयꣳ॑ हु॒त्वा । \newline
2. श॒त॒रु॒द्रीयꣳ॑ हु॒त्वा हु॒त्वा श॑तरु॒द्रीयꣳ॑ शतरु॒द्रीयꣳ॑ हु॒त्वा गा॑वीधु॒कम् गा॑वीधु॒कꣳ हु॒त्वा श॑तरु॒द्रीयꣳ॑ शतरु॒द्रीयꣳ॑ हु॒त्वा गा॑वीधु॒कम् । \newline
3. श॒त॒रु॒द्रीय॒मिति॑ शत - रु॒द्रीय᳚म् । \newline
4. हु॒त्वा गा॑वीधु॒कम् गा॑वीधु॒कꣳ हु॒त्वा हु॒त्वा गा॑वीधु॒कम् च॒रुम् च॒रुम् गा॑वीधु॒कꣳ हु॒त्वा हु॒त्वा गा॑वीधु॒कम् च॒रुम् । \newline
5. गा॒वी॒धु॒कम् च॒रुम् च॒रुम् गा॑वीधु॒कम् गा॑वीधु॒कम् च॒रु मे॒ते नै॒तेन॑ च॒रुम् गा॑वीधु॒कम् गा॑वीधु॒कम् च॒रु मे॒तेन॑ । \newline
6. च॒रु मे॒ते नै॒तेन॑ च॒रुम् च॒रु मे॒तेन॒ यजु॑षा॒ यजु॑षै॒तेन॑ च॒रुम् च॒रु मे॒तेन॒ यजु॑षा । \newline
7. ए॒तेन॒ यजु॑षा॒ यजु॑षै॒ते नै॒तेन॒ यजु॑षा चर॒माया᳚म् चर॒मायां॒ ॅयजु॑षै॒ते नै॒तेन॒ यजु॑षा चर॒माया᳚म् । \newline
8. यजु॑षा चर॒माया᳚म् चर॒मायां॒ ॅयजु॑षा॒ यजु॑षा चर॒माया॒ मिष्ट॑काया॒ मिष्ट॑कायाम् चर॒मायां॒ ॅयजु॑षा॒ यजु॑षा चर॒माया॒ मिष्ट॑कायाम् । \newline
9. च॒र॒माया॒ मिष्ट॑काया॒ मिष्ट॑कायाम् चर॒माया᳚म् चर॒माया॒ मिष्ट॑काया॒म् नि नीष्ट॑कायाम् चर॒माया᳚म् चर॒माया॒ मिष्ट॑काया॒म् नि । \newline
10. इष्ट॑काया॒म् नि नीष्ट॑काया॒ मिष्ट॑काया॒म् नि द॑द्ध्याद् दद्ध्या॒न् नीष्ट॑काया॒ मिष्ट॑काया॒म् नि द॑द्ध्यात् । \newline
11. नि द॑द्ध्याद् दद्ध्या॒न् नि नि द॑द्ध्याद् भाग॒धेये॑न भाग॒धेये॑न दद्ध्या॒न् नि नि द॑द्ध्याद् भाग॒धेये॑न । \newline
12. द॒द्ध्या॒द् भा॒ग॒धेये॑न भाग॒धेये॑न दद्ध्याद् दद्ध्याद् भाग॒धेये॑ नै॒वैव भा॑ग॒धेये॑न दद्ध्याद् दद्ध्याद् भाग॒धेये॑नै॒व । \newline
13. भा॒ग॒धेये॑ नै॒वैव भा॑ग॒धेये॑न भाग॒धेये॑ नै॒वैन॑ मेन मे॒व भा॑ग॒धेये॑न भाग॒धेये॑
नै॒वैन᳚म् । \newline
14. भा॒ग॒धेये॒नेति॑ भाग - धेये॑न । \newline
15. ए॒वैन॑ मेन मे॒वै वैनꣳ॑ शमयति शमय त्येन मे॒वै वैनꣳ॑ शमयति । \newline
16. ए॒नꣳ॒॒ श॒म॒य॒ति॒ श॒म॒य॒ त्ये॒न॒ मे॒नꣳ॒॒ श॒म॒य॒ति॒ तस्य॒ तस्य॑ शमय त्येन मेनꣳ शमयति॒ तस्य॑ । \newline
17. श॒म॒य॒ति॒ तस्य॒ तस्य॑ शमयति शमयति॒ तस्य॒ तु तु तस्य॑ शमयति शमयति॒ तस्य॒ तु । \newline
18. तस्य॒ तु तु तस्य॒ तस्य॒ त्वै वै तु तस्य॒ तस्य॒ त्वै । \newline
19. त्वै वै तु त्वै श॑तरु॒द्रीयꣳ॑ शतरु॒द्रीयं॒ ॅवै तु त्वै श॑तरु॒द्रीय᳚म् । \newline
20. वै श॑तरु॒द्रीयꣳ॑ शतरु॒द्रीयं॒ ॅवै वै श॑तरु॒द्रीयꣳ॑ हु॒तꣳ हु॒तꣳ श॑तरु॒द्रीयं॒ ॅवै वै श॑तरु॒द्रीयꣳ॑ हु॒तम् । \newline
21. श॒त॒रु॒द्रीयꣳ॑ हु॒तꣳ हु॒तꣳ श॑तरु॒द्रीयꣳ॑ शतरु॒द्रीयꣳ॑ हु॒त मितीति॑ हु॒तꣳ श॑तरु॒द्रीयꣳ॑ शतरु॒द्रीयꣳ॑ हु॒त मिति॑ । \newline
22. श॒त॒रु॒द्रीय॒मिति॑ शत - रु॒द्रीय᳚म् । \newline
23. हु॒त मितीति॑ हु॒तꣳ हु॒त मित्या॑हु राहु॒रिति॑ हु॒तꣳ हु॒त मित्या॑हुः । \newline
24. इत्या॑हु राहु॒ रिती त्या॑हु॒र् यस्य॒ यस्या॑हु॒ रिती त्या॑हु॒र् यस्य॑ । \newline
25. आ॒हु॒र् यस्य॒ यस्या॑हु राहु॒र् यस्यै॒त दे॒तद् यस्या॑हु राहु॒र् यस्यै॒तत् । \newline
26. यस्यै॒त दे॒तद् यस्य॒ यस्यै॒त द॒ग्ना व॒ग्ना वे॒तद् यस्य॒ यस्यै॒त द॒ग्नौ । \newline
27. ए॒तद॒ग्ना व॒ग्ना वे॒त दे॒त द॒ग्नौ क्रि॒यते᳚ क्रि॒यते॒ ऽग्ना वे॒त दे॒त द॒ग्नौ क्रि॒यते᳚ । \newline
28. अ॒ग्नौ क्रि॒यते᳚ क्रि॒यते॒ ऽग्ना व॒ग्नौ क्रि॒यत॒ इतीति॑ क्रि॒यते॒ ऽग्ना व॒ग्नौ क्रि॒यत॒ इति॑ । \newline
29. क्रि॒यत॒ इतीति॑ क्रि॒यते᳚ क्रि॒यत॒ इति॒ वस॑वो॒ वस॑व॒ इति॑ क्रि॒यते᳚ क्रि॒यत॒ इति॒ वस॑वः । \newline
30. इति॒ वस॑वो॒ वस॑व॒ इतीति॒ वस॑व स्त्वा त्वा॒ वस॑व॒ इतीति॒ वस॑व स्त्वा । \newline
31. वस॑व स्त्वा त्वा॒ वस॑वो॒ वस॑व स्त्वा रु॒द्रै रु॒द्रै स्त्वा॒ वस॑वो॒ वस॑व स्त्वा रु॒द्रैः । \newline
32. त्वा॒ रु॒द्रै रु॒द्रै स्त्वा᳚ त्वा रु॒द्रैः पु॒रस्ता᳚त् पु॒रस्ता᳚द् रु॒द्रै स्त्वा᳚ त्वा रु॒द्रैः पु॒रस्ता᳚त् । \newline
33. रु॒द्रैः पु॒रस्ता᳚त् पु॒रस्ता᳚द् रु॒द्रै रु॒द्रैः पु॒रस्ता᳚त् पान्तु पान्तु पु॒रस्ता᳚द् रु॒द्रै रु॒द्रैः पु॒रस्ता᳚त् पान्तु । \newline
34. पु॒रस्ता᳚त् पान्तु पान्तु पु॒रस्ता᳚त् पु॒रस्ता᳚त् पान्तु पि॒तरः॑ पि॒तरः॑ पान्तु पु॒रस्ता᳚त् पु॒रस्ता᳚त् पान्तु पि॒तरः॑ । \newline
35. पा॒न्तु॒ पि॒तरः॑ पि॒तरः॑ पान्तु पान्तु पि॒तर॑ स्त्वा त्वा पि॒तरः॑ पान्तु पान्तु पि॒तर॑ स्त्वा । \newline
36. पि॒तर॑ स्त्वा त्वा पि॒तरः॑ पि॒तर॑ स्त्वा य॒मरा॑जानो य॒मरा॑जान स्त्वा पि॒तरः॑ पि॒तर॑ स्त्वा य॒मरा॑जानः । \newline
37. त्वा॒ य॒मरा॑जानो य॒मरा॑जान स्त्वा त्वा य॒मरा॑जानः पि॒तृभिः॑ पि॒तृभि॑र् य॒मरा॑जान स्त्वा त्वा य॒मरा॑जानः पि॒तृभिः॑ । \newline
38. य॒मरा॑जानः पि॒तृभिः॑ पि॒तृभि॑र् य॒मरा॑जानो य॒मरा॑जानः पि॒तृभि॑र् दक्षिण॒तो द॑क्षिण॒तः पि॒तृभि॑र् य॒मरा॑जानो य॒मरा॑जानः पि॒तृभि॑र् दक्षिण॒तः । \newline
39. य॒मरा॑जान॒ इति॑ य॒म - रा॒जा॒नः॒ । \newline
40. पि॒तृभि॑र् दक्षिण॒तो द॑क्षिण॒तः पि॒तृभिः॑ पि॒तृभि॑र् दक्षिण॒तः पा᳚न्तु पान्तु दक्षिण॒तः पि॒तृभिः॑ पि॒तृभि॑र् दक्षिण॒तः पा᳚न्तु । \newline
41. पि॒तृभि॒रिति॑ पि॒तृ - भिः॒ । \newline
42. द॒क्षि॒ण॒तः पा᳚न्तु पान्तु दक्षिण॒तो द॑क्षिण॒तः पा᳚न्त्वादि॒त्या आ॑दि॒त्याः पा᳚न्तु दक्षिण॒तो द॑क्षिण॒तः पा᳚न्त्वादि॒त्याः । \newline
43. पा॒न्त्वा॒दि॒त्या आ॑दि॒त्याः पा᳚न्तु पान्त्वादि॒त्या स्त्वा᳚ त्वा ऽऽदि॒त्याः पा᳚न्तु पान्त्वादि॒त्या स्त्वा᳚ । \newline
44. आ॒दि॒त्या स्त्वा᳚ त्वा ऽऽदि॒त्या आ॑दि॒त्या स्त्वा॒ विश्वै॒र् विश्वै᳚ स्त्वा ऽऽदि॒त्या आ॑दि॒त्या स्त्वा॒ विश्वैः᳚ । \newline
45. त्वा॒ विश्वै॒र् विश्वै᳚ स्त्वा त्वा॒ विश्वै᳚र् दे॒वैर् दे॒वैर् विश्वै᳚ स्त्वा त्वा॒ विश्वै᳚र् दे॒वैः । \newline
46. विश्वै᳚र् दे॒वैर् दे॒वैर् विश्वै॒र् विश्वै᳚र् दे॒वैः प॒श्चात् प॒श्चाद् दे॒वैर् विश्वै॒र् विश्वै᳚र् दे॒वैः प॒श्चात् । \newline
47. दे॒वैः प॒श्चात् प॒श्चाद् दे॒वैर् दे॒वैः प॒श्चात् पा᳚न्तु पान्तु प॒श्चाद् दे॒वैर् दे॒वैः प॒श्चात् पा᳚न्तु । \newline
48. प॒श्चात् पा᳚न्तु पान्तु प॒श्चात् प॒श्चात् पा᳚न्तु द्युता॒नो द्यु॑ता॒नः पा᳚न्तु प॒श्चात् प॒श्चात् पा᳚न्तु द्युता॒नः । \newline
49. पा॒न्तु॒ द्यु॒ता॒नो द्यु॑ता॒नः पा᳚न्तु पान्तु द्युता॒न स्त्वा᳚ त्वा द्युता॒नः पा᳚न्तु पान्तु द्युता॒न स्त्वा᳚ । \newline
50. द्यु॒ता॒न स्त्वा᳚ त्वा द्युता॒नो द्यु॑ता॒न स्त्वा॑ मारु॒तो मा॑रु॒त स्त्वा᳚ द्युता॒नो द्यु॑ता॒न स्त्वा॑ मारु॒तः । \newline
51. त्वा॒ मा॒रु॒तो मा॑रु॒त स्त्वा᳚ त्वा मारु॒तो म॒रुद्भि॑र् म॒रुद्भि॑र् मारु॒त स्त्वा᳚ त्वा मारु॒तो म॒रुद्भिः॑ । \newline
52. मा॒रु॒तो म॒रुद्भि॑र् म॒रुद्भि॑र् मारु॒तो मा॑रु॒तो म॒रुद्भि॑ रुत्तर॒त उ॑त्तर॒तो म॒रुद्भि॑र् मारु॒तो मा॑रु॒तो म॒रुद्भि॑ रुत्तर॒तः । \newline
53. म॒रुद्भि॑ रुत्तर॒त उ॑त्तर॒तो म॒रुद्भि॑र् म॒रुद्भि॑ रुत्तर॒तः पा॑तु पातूत्तर॒तो म॒रुद्भि॑र् म॒रुद्भि॑ रुत्तर॒तः पा॑तु । \newline
54. म॒रुद्भि॒रिति॑ म॒रुत् - भिः॒ । \newline
55. उ॒त्त॒र॒तः पा॑तु पातूत्तर॒त उ॑त्तर॒तः पा॑तु दे॒वा दे॒वाः पा॑तूत्तर॒त उ॑त्तर॒तः पा॑तु दे॒वाः । \newline
56. उ॒त्त॒र॒त इत्यु॑त् - त॒र॒तः । \newline
57. पा॒तु॒ दे॒वा दे॒वाः पा॑तु पातु दे॒वा स्त्वा᳚ त्वा दे॒वाः पा॑तु पातु दे॒वा स्त्वा᳚ । \newline
\pagebreak
\markright{ TS 5.5.9.5  \hfill https://www.vedavms.in \hfill}

\section{ TS 5.5.9.5 }

\textbf{TS 5.5.9.5 } \newline
\textbf{Samhita Paata} \newline

दे॒वास्त्वेन्द्र॑ज्येष्ठा॒ वरु॑णराजानो॒ ऽधस्ता᳚च्चो॒-परि॑ष्ठाच्च पान्तु॒ न वा ए॒तेन॑ पू॒तो न मेद्ध्यो॒ न प्रोक्षि॑तो॒ यदे॑न॒मतः॑ प्रा॒चीनं॑ प्रो॒क्षति॒ यथ् संचि॑त॒माज्ये॑न प्रो॒क्षति॒ तेन॑ पू॒तस्तेन॒ मेद्ध्य॒स्तेन॒ प्रोक्षि॑तः ॥ \newline

\textbf{Pada Paata} \newline

दे॒वाः । त्वा॒ । इन्द्र॑ज्येष्ठा॒ इतीन्द्र॑ - ज्ये॒ष्ठाः॒ । वरु॑णराजान॒ इति॒ वरु॑ण - रा॒जा॒नः॒ । अ॒धस्ता᳚त् । च॒ । उ॒परि॑ष्ठात् । च॒ । पा॒न्तु॒ । न । वै । ए॒तेन॑ । पू॒तः । न । मेद्ध्यः॑ । न । प्रोक्षि॑त॒ इति॒ प्र - उ॒क्षि॒तः॒ । यत् । ए॒न॒म् । अतः॑ । प्रा॒चीन᳚म् । प्रो॒क्षतीति॑ प्र - उ॒क्षति॑ । यथ् । सञ्चि॑त॒मिति॒ सं - चि॒त॒म् । आज्ये॑न । प्रो॒क्षतीति॑ प्र - उ॒क्षति॑ । तेन॑ । पू॒तः । तेन॑ । मेद्ध्यः॑ । तेन॑ । प्रोक्षि॑त॒ इति॒ प्र-उ॒क्षि॒तः॒ ॥  \newline


\textbf{Krama Paata} \newline

दे॒वास्त्वा᳚ । त्वेन्द्र॑ज्येष्ठाः । इन्द्र॑ज्येष्ठा॒ वरु॑णराजानः । इन्द्र॑ज्येष्ठा॒ इतीन्द्र॑ - ज्ये॒ष्ठाः॒ । वरु॑णराजानो॒ऽधस्ता᳚त् । वरु॑णराजान॒ इति॒ वरु॑ण - रा॒जा॒नः॒ । अ॒धस्ता᳚च् च । चो॒परि॑ष्टात् । उ॒परि॑ष्टाच् च । च॒ पा॒न्तु॒ । पा॒न्तु॒ न । न वै । वा ए॒तेन॑ । ए॒तेन॑ पू॒तः । पू॒तो न । न मेद्ध्यः॑ । मेद्ध्यो॒ न । न प्रोक्षि॑तः । प्रोक्षि॑तो॒ यत् । प्रोक्षि॑त॒ इति॒ प्र - उ॒क्षि॒तः॒ । यदे॑नम् । ए॒न॒मतः॑ । अतः॑ प्रा॒चीन᳚म् । प्रा॒चीन॑म् प्रो॒क्षति॑ । प्रो॒क्षति॒ यत् । प्रो॒क्षतीति॑ प्र - उ॒क्षति॑ । यथ् सञ्चि॑तम् । सञ्चि॑त॒माज्ये॑न । सञ्चि॑त॒मिति॒ सम् - चि॒त॒म् । आज्ये॑न प्रो॒क्षति॑ । प्रो॒क्षति॒ तेन॑ । प्रो॒क्षतीति॑ प्र - उ॒क्षति॑ । तेन॑ पू॒तः । पू॒तस्तेन॑ । तेन॒ मेद्ध्यः॑ । मेद्ध्य॒स्तेन॑ । तेन॒ प्रोक्षि॑तः । प्रोक्षि॑त॒ इति॒ प्र - उ॒क्षि॒तः॒ । \newline

\textbf{Jatai Paata} \newline

1. दे॒वा स्त्वा᳚ त्वा दे॒वा दे॒वा स्त्वा᳚ । \newline
2. त्वेन्द्र॑ज्येष्ठा॒ इन्द्र॑ज्येष्ठा स्त्वा॒ त्वेन्द्र॑ज्येष्ठाः । \newline
3. इन्द्र॑ज्येष्ठा॒ वरु॑णराजानो॒ वरु॑णराजान॒ इन्द्र॑ज्येष्ठा॒ इन्द्र॑ज्येष्ठा॒ वरु॑णराजानः । \newline
4. इन्द्र॑ज्येष्ठा॒ इतीन्द्र॑ - ज्ये॒ष्ठाः॒ । \newline
5. वरु॑णराजानो॒ ऽधस्ता॑द॒ धस्ता॒द् वरु॑णराजानो॒ वरु॑णराजानो॒ ऽधस्ता᳚त् । \newline
6. वरु॑णराजान॒ इति॒ वरु॑ण - रा॒जा॒नः॒ । \newline
7. अ॒धस्ता᳚च् च चा॒धस्ता॑ द॒धस्ता᳚च् च । \newline
8. चो॒परि॑ष्ठा दु॒परि॑ष्ठाच् च चो॒परि॑ष्ठात् । \newline
9. उ॒परि॑ष्ठाच् च चो॒परि॑ष्ठा दु॒परि॑ष्ठाच् च । \newline
10. च॒ पा॒न्तु॒ पा॒न्तु॒ च॒ च॒ पा॒न्तु॒ । \newline
11. पा॒न्तु॒ न न पा᳚न्तु पान्तु॒ न । \newline
12. न वै वै न न वै । \newline
13. वा ए॒तेनै॒ तेन॒ वै वा ए॒तेन॑ । \newline
14. ए॒तेन॑ पू॒तः पू॒त ए॒तेनै॒ तेन॑ पू॒तः । \newline
15. पू॒तो न न पू॒तः पू॒तो न । \newline
16. न मेद्ध्यो॒ मेद्ध्यो॒ न न मेद्ध्यः॑ । \newline
17. मेद्ध्यो॒ न न मेद्ध्यो॒ मेद्ध्यो॒ न । \newline
18. न प्रोक्षि॑तः॒ प्रोक्षि॑तो॒ न न प्रोक्षि॑तः । \newline
19. प्रोक्षि॑तो॒ यद् यत् प्रोक्षि॑तः॒ प्रोक्षि॑तो॒ यत् । \newline
20. प्रोक्षि॑त॒ इति॒ प्र - उ॒क्षि॒तः॒ । \newline
21. यदे॑न मेनं॒ ॅयद् यदे॑नम् । \newline
22. ए॒न॒ मतो ऽत॑ एन मेन॒ मतः॑ । \newline
23. अतः॑ प्रा॒चीन॑म् प्रा॒चीन॒ मतो ऽतः॑ प्रा॒चीन᳚म् । \newline
24. प्रा॒चीन॑म् प्रो॒क्षति॑ प्रो॒क्षति॑ प्रा॒चीन॑म् प्रा॒चीन॑म् प्रो॒क्षति॑ । \newline
25. प्रो॒क्षति॒ यद् यत् प्रो॒क्षति॑ प्रो॒क्षति॒ यत् । \newline
26. प्रो॒क्षतीति॑ प्र - उ॒क्षति॑ । \newline
27. यथ् सञ्चि॑तꣳ॒॒ सञ्चि॑तं॒ ॅयद् यथ् सञ्चि॑तम् । \newline
28. सञ्चि॑त॒ माज्ये॒ना ज्ये॑न॒ सञ्चि॑तꣳ॒॒ सञ्चि॑त॒ माज्ये॑न । \newline
29. सञ्चि॑त॒मिति॒ सं - चि॒त॒म् । \newline
30. आज्ये॑न प्रो॒क्षति॑ प्रो॒क्ष त्याज्ये॒ना ज्ये॑न प्रो॒क्षति॑ । \newline
31. प्रो॒क्षति॒ तेन॒ तेन॑ प्रो॒क्षति॑ प्रो॒क्षति॒ तेन॑ । \newline
32. प्रो॒क्षतीति॑ प्र - उ॒क्षति॑ । \newline
33. तेन॑ पू॒तः पू॒त स्तेन॒ तेन॑ पू॒तः । \newline
34. पू॒त स्तेन॒ तेन॑ पू॒तः पू॒त स्तेन॑ । \newline
35. तेन॒ मेद्ध्यो॒ मेद्ध्य॒ स्तेन॒ तेन॒ मेद्ध्यः॑ । \newline
36. मेद्ध्य॒ स्तेन॒ तेन॒ मेद्ध्यो॒ मेद्ध्य॒ स्तेन॑ । \newline
37. तेन॒ प्रोक्षि॑तः॒ प्रोक्षि॑त॒ स्तेन॒ तेन॒ प्रोक्षि॑तः । \newline
38. प्रोक्षि॑त॒ इति॒ प्र - उ॒क्षि॒तः॒ । \newline

\textbf{Ghana Paata } \newline

1. दे॒वा स्त्वा᳚ त्वा दे॒वा दे॒वा स्त्वेन्द्र॑ज्येष्ठा॒ इन्द्र॑ज्येष्ठा स्त्वा दे॒वा दे॒वा स्त्वेन्द्र॑ज्येष्ठाः । \newline
2. त्वेन्द्र॑ज्येष्ठा॒ इन्द्र॑ज्येष्ठा स्त्वा॒ त्वेन्द्र॑ज्येष्ठा॒ वरु॑णराजानो॒ वरु॑णराजान॒ इन्द्र॑ज्येष्ठा स्त्वा॒ त्वेन्द्र॑ज्येष्ठा॒ वरु॑णराजानः । \newline
3. इन्द्र॑ज्येष्ठा॒ वरु॑णराजानो॒ वरु॑णराजान॒ इन्द्र॑ज्येष्ठा॒ इन्द्र॑ज्येष्ठा॒ वरु॑णराजानो॒ ऽधस्ता॑ द॒धस्ता॒द् वरु॑णराजान॒ इन्द्र॑ज्येष्ठा॒ इन्द्र॑ज्येष्ठा॒ वरु॑णराजानो॒ ऽधस्ता᳚त् । \newline
4. इन्द्र॑ज्येष्ठा॒ इतीन्द्र॑ - ज्ये॒ष्ठाः॒ । \newline
5. वरु॑णराजानो॒ ऽधस्ता॑ द॒धस्ता॒द् वरु॑णराजानो॒ वरु॑णराजानो॒ ऽधस्ता᳚च् च चा॒धस्ता॒द् वरु॑णराजानो॒ वरु॑णराजानो॒ ऽधस्ता᳚च् च । \newline
6. वरु॑णराजान॒ इति॒ वरु॑ण - रा॒जा॒नः॒ । \newline
7. अ॒धस्ता᳚च् च चा॒धस्ता॑ द॒धस्ता᳚च् चो॒परि॑ष्ठा दु॒परि॑ष्ठाच् चा॒धस्ता॑ द॒धस्ता᳚च् चो॒परि॑ष्ठात् । \newline
8. चो॒परि॑ष्ठा दु॒परि॑ष्ठाच् च चो॒परि॑ष्ठाच् च चो॒परि॑ष्ठाच् च चो॒परि॑ष्ठाच् च । \newline
9. उ॒परि॑ष्ठाच् च चो॒परि॑ष्ठा दु॒परि॑ष्ठाच् च पान्तु पान्तु चो॒परि॑ष्ठा दु॒परि॑ष्ठाच् च पान्तु । \newline
10. च॒ पा॒न्तु॒ पा॒न्तु॒ च॒ च॒ पा॒न्तु॒ न न पा᳚न्तु च च पान्तु॒ न । \newline
11. पा॒न्तु॒ न न पा᳚न्तु पान्तु॒ न वै वै न पा᳚न्तु पान्तु॒ न वै । \newline
12. न वै वै न न वा ए॒ते नै॒तेन॒ वै न न वा ए॒तेन॑ । \newline
13. वा ए॒ते नै॒तेन॒ वै वा ए॒तेन॑ पू॒तः पू॒त ए॒तेन॒ वै वा ए॒तेन॑ पू॒तः । \newline
14. ए॒तेन॑ पू॒तः पू॒त ए॒ते नै॒तेन॑ पू॒तो न न पू॒त ए॒ते नै॒तेन॑ पू॒तो न । \newline
15. पू॒तो न न पू॒तः पू॒तो न मेद्ध्यो॒ मेद्ध्यो॒ न पू॒तः पू॒तो न मेद्ध्यः॑ । \newline
16. न मेद्ध्यो॒ मेद्ध्यो॒ न न मेद्ध्यो॒ न न मेद्ध्यो॒ न न मेद्ध्यो॒ न । \newline
17. मेद्ध्यो॒ न न मेद्ध्यो॒ मेद्ध्यो॒ न प्रोक्षि॑तः॒ प्रोक्षि॑तो॒ न मेद्ध्यो॒ मेद्ध्यो॒ न प्रोक्षि॑तः । \newline
18. न प्रोक्षि॑तः॒ प्रोक्षि॑तो॒ न न प्रोक्षि॑तो॒ यद् यत् प्रोक्षि॑तो॒ न न प्रोक्षि॑तो॒ यत् । \newline
19. प्रोक्षि॑तो॒ यद् यत् प्रोक्षि॑तः॒ प्रोक्षि॑तो॒ यदे॑न मेनं॒ ॅयत् प्रोक्षि॑तः॒ प्रोक्षि॑तो॒ यदे॑नम् । \newline
20. प्रोक्षि॑त॒ इति॒ प्र - उ॒क्षि॒तः॒ । \newline
21. यदे॑न मेनं॒ ॅयद् यदे॑न॒ मतो ऽत॑ एनं॒ ॅयद् यदे॑न॒ मतः॑ । \newline
22. ए॒न॒ मतो ऽत॑ एन मेन॒ मतः॑ प्रा॒चीन॑म् प्रा॒चीन॒ मत॑ एन मेन॒ मतः॑ प्रा॒चीन᳚म् । \newline
23. अतः॑ प्रा॒चीन॑म् प्रा॒चीन॒ मतो ऽतः॑ प्रा॒चीन॑म् प्रो॒क्षति॑ प्रो॒क्षति॑ प्रा॒चीन॒ मतो ऽतः॑ प्रा॒चीन॑म् प्रो॒क्षति॑ । \newline
24. प्रा॒चीन॑म् प्रो॒क्षति॑ प्रो॒क्षति॑ प्रा॒चीन॑म् प्रा॒चीन॑म् प्रो॒क्षति॒ यद् यत् प्रो॒क्षति॑ प्रा॒चीन॑म् प्रा॒चीन॑म् प्रो॒क्षति॒ यत् । \newline
25. प्रो॒क्षति॒ यद् यत् प्रो॒क्षति॑ प्रो॒क्षति॒ यथ् सञ्चि॑तꣳ॒॒ सञ्चि॑तं॒ ॅयत् प्रो॒क्षति॑ प्रो॒क्षति॒ यथ् सञ्चि॑तम् । \newline
26. प्रो॒क्षतीति॑ प्र - उ॒क्षति॑ । \newline
27. यथ् सञ्चि॑तꣳ॒॒ सञ्चि॑तं॒ ॅयद् यथ् सञ्चि॑त॒ माज्ये॒ना ज्ये॑न॒ सञ्चि॑तं॒ ॅयद् यथ् सञ्चि॑त॒ माज्ये॑न । \newline
28. सञ्चि॑त॒ माज्ये॒ना ज्ये॑न॒ सञ्चि॑तꣳ॒॒ सञ्चि॑त॒ माज्ये॑न प्रो॒क्षति॑ प्रो॒क्ष त्याज्ये॑न॒ सञ्चि॑तꣳ॒॒ सञ्चि॑त॒ माज्ये॑न प्रो॒क्षति॑ । \newline
29. सञ्चि॑त॒मिति॒ सं - चि॒त॒म् । \newline
30. आज्ये॑न प्रो॒क्षति॑ प्रो॒क्ष त्याज्ये॒ना ज्ये॑न प्रो॒क्षति॒ तेन॒ तेन॑ प्रो॒क्ष त्याज्ये॒ना ज्ये॑न प्रो॒क्षति॒ तेन॑ । \newline
31. प्रो॒क्षति॒ तेन॒ तेन॑ प्रो॒क्षति॑ प्रो॒क्षति॒ तेन॑ पू॒तः पू॒त स्तेन॑ प्रो॒क्षति॑ प्रो॒क्षति॒ तेन॑ पू॒तः । \newline
32. प्रो॒क्षतीति॑ प्र - उ॒क्षति॑ । \newline
33. तेन॑ पू॒तः पू॒त स्तेन॒ तेन॑ पू॒त स्तेन॒ तेन॑ पू॒त स्तेन॒ तेन॑ पू॒त स्तेन॑ । \newline
34. पू॒त स्तेन॒ तेन॑ पू॒तः पू॒त स्तेन॒ मेद्ध्यो॒ मेद्ध्य॒ स्तेन॑ पू॒तः पू॒त स्तेन॒ मेद्ध्यः॑ । \newline
35. तेन॒ मेद्ध्यो॒ मेद्ध्य॒ स्तेन॒ तेन॒ मेद्ध्य॒ स्तेन॒ तेन॒ मेद्ध्य॒ स्तेन॒ तेन॒ मेद्ध्य॒ स्तेन॑ । \newline
36. मेद्ध्य॒ स्तेन॒ तेन॒ मेद्ध्यो॒ मेद्ध्य॒ स्तेन॒ प्रोक्षि॑तः॒ प्रोक्षि॑त॒ स्तेन॒ मेद्ध्यो॒ मेद्ध्य॒ स्तेन॒ प्रोक्षि॑तः । \newline
37. तेन॒ प्रोक्षि॑तः॒ प्रोक्षि॑त॒ स्तेन॒ तेन॒ प्रोक्षि॑तः । \newline
38. प्रोक्षि॑त॒ इति॒ प्र - उ॒क्षि॒तः॒ । \newline
\pagebreak
\markright{ TS 5.5.10.1  \hfill https://www.vedavms.in \hfill}

\section{ TS 5.5.10.1 }

\textbf{TS 5.5.10.1 } \newline
\textbf{Samhita Paata} \newline

स॒॒मीची॒॒ नामा॑सि॒ प्राची॒ दिक्तस्या᳚स्ते॒ ऽग्निरधि॑पति रसि॒तो र॑क्षि॒ता यश्चाधि॑पति॒ र्यश्च॑ गो॒प्ता ताभ्यां॒ नम॒स्तौनो॑ मृडयतां॒ ते यं द्वि॒ष्मो यश्च॑ नो॒ द्वेष्टि॒ तं ॅवां॒ जंभे॑ दधाम्योज॒स्विनी॒ नामा॑सि दक्षि॒णा दिक् तस्या᳚स्त॒ इन्द्रोऽधि॑पतिः॒ पृदा॑कुः॒ प्राची॒ नामा॑सि प्र॒तीची॒ दिक् तस्या᳚स्ते॒ - [  ] \newline

\textbf{Pada Paata} \newline

स॒मीची᳚ । नाम॑ । अ॒सि॒ । प्राची᳚ । दिक् । तस्याः᳚ । ते॒ । अ॒ग्निः । अधि॑पति॒रित्यधि॑ - प॒तिः॒ । अ॒सि॒तः । र॒क्षि॒ता । यः । च॒ । अधि॑पति॒रित्यधि॑-प॒तिः॒ । यः । च॒ । गो॒प्ता । ताभ्या᳚म् । नमः॑ । तौ । नः॒ । मृ॒ड॒य॒ता॒म् । ते । यम् । द्वि॒ष्मः । यः । च॒ । नः॒ । द्वेष्टि॑ । तम् । वा॒म् । जम्भे᳚ । द॒धा॒मि॒ । ओ॒ज॒स्विनी᳚ । नाम॑ । अ॒सि॒ । द॒क्षि॒णा । दिक् । तस्याः᳚ । ते॒ । इन्द्रः॑ । अधि॑पति॒रित्यधि॑ - प॒तिः॒ । पृदा॑कुः । प्राची᳚ । नाम॑ । अ॒सि॒ । प्र॒तीची᳚ । दिक् । तस्याः᳚ । ते॒ ।  \newline


\textbf{Krama Paata} \newline

स॒मीची॒ नाम॑ । नामा॑सि । अ॒सि॒ प्राची᳚ । प्राची॒ दिक् । दिक् तस्याः᳚ । तस्या᳚स्ते । ते॒ऽग्निः । अ॒ग्निरधि॑पतिः । अधि॑पतिरसि॒तः । अधि॑पति॒रित्यधि॑ - प॒तिः॒ । अ॒सि॒तो र॑क्षि॒ता । र॒क्षि॒ता यः । यश्च॑ । चाधि॑पतिः । अधि॑पति॒र् यः । अधि॑पति॒रित्यधि॑ - प॒तिः॒ । यश्च॑ । च॒ गो॒प्ता । गो॒प्ता ताभ्या᳚म् । ताभ्या॒म् नमः॑ । नम॒स्तौ । तौ नः॑ । नो॒ मृ॒ड॒य॒ता॒म् । मृ॒ड॒य॒ता॒म् ते । ते यम् । यम् द्वि॒ष्मः । द्वि॒ष्मो यः । यश्च॑ । च॒ नः॒ । नो॒ द्वेष्टि॑ । द्वेष्टि॒ तम् । तम् ॅवा᳚म् । वा॒म् जम्भे᳚ । जम्भे॑ दधामि । द॒धा॒म्यो॒ज॒स्विनी᳚ । ओ॒ज॒स्विनी॒ नाम॑ । नामा॑सि । अ॒सि॒ द॒क्षि॒णा । द॒क्षि॒णा दिक् । दिक् तस्याः᳚ । तस्या᳚स्ते । त॒ इन्द्रः॑ । इन्द्रोऽधि॑पतिः । अधि॑पतिः॒ पृदा॑कुः । अधि॑पति॒रित्यधि॑ - प॒तिः॒ । पृदा॑कुः॒ प्राची᳚ । प्राची॒ नाम॑ । नामा॑सि । अ॒सि॒ प्र॒तीची᳚ । प्र॒तीची॒ दिक् । दिक् तस्याः᳚ । तस्या᳚स्ते । ते॒ सोमः॑ \newline

\textbf{Jatai Paata} \newline

1. स॒मीची॒ नाम॒ नाम॑ स॒मीची॑ स॒मीची॒ नाम॑ । \newline
2. नामा᳚स्यसि॒ नाम॒ नामा॑सि । \newline
3. अ॒सि॒ प्राची॒ प्राच्य॑स्यसि॒ प्राची᳚ । \newline
4. प्राची॒ दिग् दिक् प्राची॒ प्राची॒ दिक् । \newline
5. दिक् तस्या॒ स्तस्या॒ दिग् दिक् तस्याः᳚ । \newline
6. तस्या᳚ स्ते ते॒ तस्या॒ स्तस्या᳚ स्ते । \newline
7. ते॒ ऽग्नि र॒ग्नि स्ते॑ ते॒ ऽग्निः । \newline
8. अ॒ग्निरधि॑पति॒ रधि॑पति र॒ग्नि र॒ग्नि रधि॑पतिः । \newline
9. अधि॑पति रसि॒तो॑ ऽसि॒तो ऽधि॑पति॒ रधि॑पति रसि॒तः । \newline
10. अधि॑पति॒रित्यधि॑ - प॒तिः॒ । \newline
11. अ॒सि॒तो र॑क्षि॒ता र॑क्षि॒ता ऽसि॒तो॑ ऽसि॒तो र॑क्षि॒ता । \newline
12. र॒क्षि॒ता यो यो र॑क्षि॒ता र॑क्षि॒ता यः । \newline
13. यश्च॑ च॒ यो यश्च॑ । \newline
14. चाधि॑पति॒ रधि॑पति श्च॒ चाधि॑पतिः । \newline
15. अधि॑पति॒र् यो यो ऽधि॑पति॒ रधि॑पति॒र् यः । \newline
16. अधि॑पति॒रित्यधि॑ - प॒तिः॒ । \newline
17. यश्च॑ च॒ यो यश्च॑ । \newline
18. च॒ गो॒प्ता गो॒प्ता च॑ च गो॒प्ता । \newline
19. गो॒प्ता ताभ्या॒म् ताभ्या᳚म् गो॒प्ता गो॒प्ता ताभ्या᳚म् । \newline
20. ताभ्या॒म् नमो॒ नम॒ स्ताभ्या॒म् ताभ्या॒म् नमः॑ । \newline
21. नम॒ स्तौ तौ नमो॒ नम॒ स्तौ । \newline
22. तौ नो॑ न॒ स्तौ तौ नः॑ । \newline
23. नो॒ मृ॒ड॒य॒ता॒म् मृ॒ड॒य॒ता॒म् नो॒ नो॒ मृ॒ड॒य॒ता॒म् । \newline
24. मृ॒ड॒य॒ता॒म् ते ते मृ॑डयताम् मृडयता॒म् ते । \newline
25. ते यं ॅयम् ते ते यम् । \newline
26. यम् द्वि॒ष्मो द्वि॒ष्मो यं ॅयम् द्वि॒ष्मः । \newline
27. द्वि॒ष्मो यो यो द्वि॒ष्मो द्वि॒ष्मो यः । \newline
28. यश्च॑ च॒ यो यश्च॑ । \newline
29. च॒ नो॒ न॒श्च॒ च॒ नः॒ । \newline
30. नो॒ द्वेष्टि॒ द्वेष्टि॑ नो नो॒ द्वेष्टि॑ । \newline
31. द्वेष्टि॒ तम् तम् द्वेष्टि॒ द्वेष्टि॒ तम् । \newline
32. तं ॅवां᳚ ॅवा॒म् तम् तं ॅवा᳚म् । \newline
33. वा॒म् जम्भे॒ जम्भे॑ वां ॅवा॒म् जम्भे᳚ । \newline
34. जम्भे॑ दधामि दधामि॒ जम्भे॒ जम्भे॑ दधामि । \newline
35. द॒धा॒ म्यो॒ज॒स्वि न्यो॑ज॒स्विनी॑ दधामि दधा म्योज॒स्विनी᳚ । \newline
36. ओ॒ज॒स्विनी॒ नाम॒ नामौ॑ ज॒स्वि न्यो॑ज॒स्विनी॒ नाम॑ । \newline
37. नामा᳚स्यसि॒ नाम॒ नामा॑सि । \newline
38. अ॒सि॒ द॒क्षि॒णा द॑क्षि॒णा ऽस्य॑सि दक्षि॒णा । \newline
39. द॒क्षि॒णा दिग् दिग् द॑क्षि॒णा द॑क्षि॒णा दिक् । \newline
40. दिक् तस्या॒ स्तस्या॒ दिग् दिक् तस्याः᳚ । \newline
41. तस्या᳚ स्ते ते॒ तस्या॒ स्तस्या᳚ स्ते । \newline
42. त॒ इन्द्र॒ इन्द्र॑ स्ते त॒ इन्द्रः॑ । \newline
43. इन्द्रो ऽधि॑पति॒ रधि॑पति॒ रिन्द्र॒ इन्द्रो ऽधि॑पतिः । \newline
44. अधि॑पतिः॒ पृदा॑कुः॒ पृदा॑कु॒ रधि॑पति॒ रधि॑पतिः॒ पृदा॑कुः । \newline
45. अधि॑पति॒रित्यधि॑ - प॒तिः॒ । \newline
46. पृदा॑कुः॒ प्राची॒ प्राची॒ पृदा॑कुः॒ पृदा॑कुः॒ प्राची᳚ । \newline
47. प्राची॒ नाम॒ नाम॒ प्राची॒ प्राची॒ नाम॑ । \newline
48. नामा᳚ स्यसि॒ नाम॒ नामा॑सि । \newline
49. अ॒सि॒ प्र॒तीची᳚ प्र॒तीच्य॑स्यसि प्र॒तीची᳚ । \newline
50. प्र॒तीची॒ दिग् दिक् प्र॒तीची᳚ प्र॒तीची॒ दिक् । \newline
51. दिक् तस्या॒ स्तस्या॒ दिग् दिक् तस्याः᳚ । \newline
52. तस्या᳚ स्ते ते॒ तस्या॒ स्तस्या᳚ स्ते । \newline
53. ते॒ सोमः॒ सोम॑ स्ते ते॒ सोमः॑ । \newline

\textbf{Ghana Paata } \newline

1. स॒मीची॒ नाम॒ नाम॑ स॒मीची॑ स॒मीची॒ नामा᳚स्यसि॒ नाम॑ स॒मीची॑ स॒मीची॒ नामा॑सि । \newline
2. नामा᳚स्यसि॒ नाम॒ नामा॑सि॒ प्राची॒ प्राच्य॑सि॒ नाम॒ नामा॑सि॒ प्राची᳚ । \newline
3. अ॒सि॒ प्राची॒ प्राच्य॑ स्यसि॒ प्राची॒ दिग् दिक् प्राच्य॑स्यसि॒ प्राची॒ दिक् । \newline
4. प्राची॒ दिग् दिक् प्राची॒ प्राची॒ दिक् तस्या॒ स्तस्या॒ दिक् प्राची॒ प्राची॒ दिक् तस्याः᳚ । \newline
5. दिक् तस्या॒ स्तस्या॒ दिग् दिक् तस्या᳚ स्ते ते॒ तस्या॒ दिग् दिक् तस्या᳚ स्ते । \newline
6. तस्या᳚ स्ते ते॒ तस्या॒ स्तस्या᳚ स्ते॒ ऽग्नि र॒ग्नि स्ते॒ तस्या॒ स्तस्या᳚ स्ते॒ ऽग्निः । \newline
7. ते॒ ऽग्नि र॒ग्नि स्ते॑ ते॒ ऽग्नि रधि॑पति॒ रधि॑पति र॒ग्नि स्ते॑ ते॒ ऽग्नि रधि॑पतिः । \newline
8. अ॒ग्नि रधि॑पति॒ रधि॑पति र॒ग्नि र॒ग्नि रधि॑पति रसि॒तो॑ ऽसि॒तो ऽधि॑पति र॒ग्नि र॒ग्नि रधि॑पति रसि॒तः । \newline
9. अधि॑पति रसि॒तो॑ ऽसि॒तो ऽधि॑पति॒ रधि॑पति रसि॒तो र॑क्षि॒ता र॑क्षि॒ता ऽसि॒तो ऽधि॑पति॒ रधि॑पति रसि॒तो र॑क्षि॒ता । \newline
10. अधि॑पति॒रित्यधि॑ - प॒तिः॒ । \newline
11. अ॒सि॒तो र॑क्षि॒ता र॑क्षि॒ता ऽसि॒तो॑ ऽसि॒तो र॑क्षि॒ता यो यो र॑क्षि॒ता ऽसि॒तो॑ ऽसि॒तो र॑क्षि॒ता यः । \newline
12. र॒क्षि॒ता यो यो र॑क्षि॒ता र॑क्षि॒ता यश्च॑ च॒ यो र॑क्षि॒ता र॑क्षि॒ता यश्च॑ । \newline
13. यश्च॑ च॒ यो यश्चा धि॑पति॒ रधि॑पतिश्च॒ यो यश्चाधि॑पतिः । \newline
14. चाधि॑पति॒ रधि॑पतिश्च॒ चाधि॑पति॒र् यो यो ऽधि॑पतिश्च॒ चाधि॑पति॒र् यः । \newline
15. अधि॑पति॒र् यो यो ऽधि॑पति॒ रधि॑पति॒र् यश्च॑ च॒ यो ऽधि॑पति॒ रधि॑पति॒र् यश्च॑ । \newline
16. अधि॑पति॒रित्यधि॑ - प॒तिः॒ । \newline
17. यश्च॑ च॒ यो यश्च॑ गो॒प्ता गो॒प्ता च॒ यो यश्च॑ गो॒प्ता । \newline
18. च॒ गो॒प्ता गो॒प्ता च॑ च गो॒प्ता ताभ्या॒म् ताभ्या᳚म् गो॒प्ता च॑ च गो॒प्ता ताभ्या᳚म् । \newline
19. गो॒प्ता ताभ्या॒म् ताभ्या᳚म् गो॒प्ता गो॒प्ता ताभ्या॒म् नमो॒ नम॒ स्ताभ्या᳚म् गो॒प्ता गो॒प्ता ताभ्या॒म् नमः॑ । \newline
20. ताभ्या॒म् नमो॒ नम॒ स्ताभ्या॒म् ताभ्या॒म् नम॒ स्तौ तौ नम॒ स्ताभ्या॒म् ताभ्या॒म् नम॒ स्तौ । \newline
21. नम॒ स्तौ तौ नमो॒ नम॒ स्तौ नो॑ न॒ स्तौ नमो॒ नम॒ स्तौ नः॑ । \newline
22. तौ नो॑ न॒ स्तौ तौ नो॑ मृडयताम् मृडयताम् न॒ स्तौ तौ नो॑ मृडयताम् । \newline
23. नो॒ मृ॒ड॒य॒ता॒म् मृ॒ड॒य॒ता॒म् नो॒ नो॒ मृ॒ड॒य॒ता॒म् ते ते मृ॑डयताम् नो नो मृडयता॒म् ते । \newline
24. मृ॒ड॒य॒ता॒म् ते ते मृ॑डयताम् मृडयता॒म् ते यं ॅयम् ते मृ॑डयताम् मृडयता॒म् ते यम् । \newline
25. ते यं ॅयम् ते ते यम् द्वि॒ष्मो द्वि॒ष्मो यम् ते ते यम् द्वि॒ष्मः । \newline
26. यम् द्वि॒ष्मो द्वि॒ष्मो यं ॅयम् द्वि॒ष्मो यो यो द्वि॒ष्मो यं ॅयम् द्वि॒ष्मो यः । \newline
27. द्वि॒ष्मो यो यो द्वि॒ष्मो द्वि॒ष्मो यश्च॑ च॒ यो द्वि॒ष्मो द्वि॒ष्मो यश्च॑ । \newline
28. यश्च॑ च॒ यो यश्च॑ नो नश्च॒ यो यश्च॑ नः । \newline
29. च॒ नो॒ न॒श्च॒ च॒ नो॒ द्वेष्टि॒ द्वेष्टि॑ नश्च च नो॒ द्वेष्टि॑ । \newline
30. नो॒ द्वेष्टि॒ द्वेष्टि॑ नो नो॒ द्वेष्टि॒ तम् तम् द्वेष्टि॑ नो नो॒ द्वेष्टि॒ तम् । \newline
31. द्वेष्टि॒ तम् तम् द्वेष्टि॒ द्वेष्टि॒ तं ॅवां᳚ ॅवा॒म् तम् द्वेष्टि॒ द्वेष्टि॒ तं ॅवा᳚म् । \newline
32. तं ॅवां᳚ ॅवा॒म् तम् तं ॅवा॒म् जम्भे॒ जम्भे॑ वा॒म् तम् तं ॅवा॒म् जम्भे᳚ । \newline
33. वा॒म् जम्भे॒ जम्भे॑ वां ॅवा॒म् जम्भे॑ दधामि दधामि॒ जम्भे॑ वां ॅवा॒म् जम्भे॑ दधामि । \newline
34. जम्भे॑ दधामि दधामि॒ जम्भे॒ जम्भे॑ दधा म्योज॒स्वि न्यो॑ज॒स्विनी॑ दधामि॒ जम्भे॒ जम्भे॑ दधा म्योज॒स्विनी᳚ । \newline
35. द॒धा॒ म्यो॒ज॒स्वि न्यो॑ज॒स्विनी॑ दधामि दधा म्योज॒स्विनी॒ नाम॒ नामौ॑ज॒स्विनी॑ दधामि दधा म्योज॒स्विनी॒ नाम॑ । \newline
36. ओ॒ज॒स्विनी॒ नाम॒ नामौ॑ ज॒स्वि न्यो॑ज॒स्विनी॒ नामा᳚ स्यसि॒ नामौ॑ ज॒स्वि न्यो॑ज॒स्विनी॒ नामा॑सि । \newline
37. नामा᳚स्यसि॒ नाम॒ नामा॑सि दक्षि॒णा द॑क्षि॒णा ऽसि॒ नाम॒ नामा॑सि दक्षि॒णा । \newline
38. अ॒सि॒ द॒क्षि॒णा द॑क्षि॒णा ऽस्य॑सि दक्षि॒णा दिग् दिग् द॑क्षि॒णा ऽस्य॑सि दक्षि॒णा दिक् । \newline
39. द॒क्षि॒णा दिग् दिग् द॑क्षि॒णा द॑क्षि॒णा दिक् तस्या॒ स्तस्या॒ दिग् द॑क्षि॒णा द॑क्षि॒णा दिक् तस्याः᳚ । \newline
40. दिक् तस्या॒ स्तस्या॒ दिग् दिक् तस्या᳚ स्ते ते॒ तस्या॒ दिग् दिक् तस्या᳚ स्ते । \newline
41. तस्या᳚ स्ते ते॒ तस्या॒ स्तस्या᳚ स्त॒ इन्द्र॒ इन्द्र॑ स्ते॒ तस्या॒ स्तस्या᳚ स्त॒ इन्द्रः॑ । \newline
42. त॒ इन्द्र॒ इन्द्र॑ स्ते त॒ इन्द्रो ऽधि॑पति॒ रधि॑पति॒ रिन्द्र॑ स्ते त॒ इन्द्रो ऽधि॑पतिः । \newline
43. इन्द्रो ऽधि॑पति॒ रधि॑पति॒ रिन्द्र॒ इन्द्रो ऽधि॑पतिः॒ पृदा॑कुः॒ पृदा॑कु॒ रधि॑पति॒ रिन्द्र॒ इन्द्रो ऽधि॑पतिः॒ पृदा॑कुः । \newline
44. अधि॑पतिः॒ पृदा॑कुः॒ पृदा॑कु॒ रधि॑पति॒ रधि॑पतिः॒ पृदा॑कुः॒ प्राची॒ प्राची॒ पृदा॑कु॒ रधि॑पति॒ रधि॑पतिः॒ पृदा॑कुः॒ प्राची᳚ । \newline
45. अधि॑पति॒रित्यधि॑ - प॒तिः॒ । \newline
46. पृदा॑कुः॒ प्राची॒ प्राची॒ पृदा॑कुः॒ पृदा॑कुः॒ प्राची॒ नाम॒ नाम॒ प्राची॒ पृदा॑कुः॒ पृदा॑कुः॒ प्राची॒ नाम॑ । \newline
47. प्राची॒ नाम॒ नाम॒ प्राची॒ प्राची॒ नामा᳚ स्यसि॒ नाम॒ प्राची॒ प्राची॒ नामा॑सि । \newline
48. नामा᳚स्यसि॒ नाम॒ नामा॑सि प्र॒तीची᳚ प्र॒तीच्य॑सि॒ नाम॒ नामा॑सि प्र॒तीची᳚ । \newline
49. अ॒सि॒ प्र॒तीची᳚ प्र॒तीच्य॑ स्यसि प्र॒तीची॒ दिग् दिक् प्र॒तीच्य॑ स्यसि प्र॒तीची॒ दिक् । \newline
50. प्र॒तीची॒ दिग् दिक् प्र॒तीची᳚ प्र॒तीची॒ दिक् तस्या॒ स्तस्या॒ दिक् प्र॒तीची᳚ प्र॒तीची॒ दिक् तस्याः᳚ । \newline
51. दिक् तस्या॒ स्तस्या॒ दिग् दिक् तस्या᳚ स्ते ते॒ तस्या॒ दिग् दिक् तस्या᳚ स्ते । \newline
52. तस्या᳚ स्ते ते॒ तस्या॒ स्तस्या᳚ स्ते॒ सोमः॒ सोम॑ स्ते॒ तस्या॒ स्तस्या᳚ स्ते॒ सोमः॑ । \newline
53. ते॒ सोमः॒ सोम॑ स्ते ते॒ सोमो ऽधि॑पति॒ रधि॑पतिः॒ सोम॑ स्ते ते॒ सोमो ऽधि॑पतिः । \newline
\pagebreak
\markright{ TS 5.5.10.2  \hfill https://www.vedavms.in \hfill}

\section{ TS 5.5.10.2 }

\textbf{TS 5.5.10.2 } \newline
\textbf{Samhita Paata} \newline

सोमोऽधि॑पतिः स्व॒जो॑ ऽव॒स्थावा॒ नामा॒-स्युदी॑ची॒ दिक् तस्या᳚स्ते॒ वरु॒णोऽधि॑पति-स्ति॒रश्च॑ राजि॒-रधि॑पत्नी॒ नामा॑सि बृह॒ती दिक् तस्या᳚स्ते॒ बृह॒स्पति॒-रधि॑पतिः श्वि॒त्रो व॒शिनी॒ नामा॑सी॒यं दिक् तस्या᳚स्ते य॒मोऽधि॑पतिः क॒ल्माष॑ ग्रीवो रक्षि॒ता यश्चाधि॑पति॒ र्यश्च॑ गो॒प्ता ताभ्यां॒ नम॒स्तौ नो॑ मृडयतां॒ ते यं द्वि॒ष्मो यश्च॑ - [  ] \newline

\textbf{Pada Paata} \newline

सोमः॑ । अधि॑पति॒रित्यधि॑ - प॒तिः॒ । स्व॒ज इति॑ स्व - जः । अ॒व॒स्थावेत्य॑व - स्थावा᳚ । नाम॑ । अ॒सि॒ । उदी॑ची । दिक् । तस्याः᳚ । ते॒ । वरु॑णः । अधि॑पति॒रित्यधि॑-प॒तिः॒ । ति॒रश्व॑राजि॒रिति॑ ति॒रश्व॑-रा॒जिः॒ । अधि॑प॒त्नीत्यधि॑-प॒त्नी॒ । नाम॑ । अ॒सि॒ । बृ॒ह॒ती । दिक् । तस्याः᳚ । ते॒ । बृह॒स्पतिः॑ । अधि॑पति॒रित्यधि॑ - प॒तिः॒ । श्वि॒त्रः । व॒शिनी᳚ । नाम॑ । अ॒सि॒ । इ॒यं । दिक् । तस्याः᳚ । ते॒ । य॒मः । अधि॑पति॒रित्यधि॑-प॒तिः॒ । क॒ल्माष॑ग्रीव॒ इति॑ क॒ल्माष॑ - ग्री॒वः॒ । र॒क्षि॒ता । यः । च॒ । अधि॑पति॒रित्यधि॑-प॒तिः॒ । यः । च॒ । गो॒प्ता । ताभ्या᳚म् । नमः॑ । तौ । नः॒ । मृ॒ड॒य॒ता॒म् । ते । यम् । द्वि॒ष्मः । यः । च॒ ।  \newline


\textbf{Krama Paata} \newline

सोमोऽधि॑पतिः । अधि॑पतिः स्व॒जः । अधि॑पति॒रित्यधि॑ - प॒तिः॒ । स्व॒जो॑ऽव॒स्थावा᳚ । स्व॒ज इति॑ स्व - जः । अ॒व॒स्थावा॒ नाम॑ । अ॒व॒स्थावेत्य॑व - स्थावा᳚ । नामा॑सि । अ॒स्युदी॑चि । उदी॑ची॒ दिक् । दिक् तस्याः᳚ । तस्या᳚स्ते । ते॒ वरु॑णः । वरु॒णोऽधि॑पतिः । अधि॑पतिस्ति॒रश्च॑राजिः । अधि॑पति॒रित्यधि॑ - प॒तिः॒ । ति॒रश्च॑राजि॒रधि॑पत्नी । ति॒रश्च॑राजि॒रिति॑ ति॒रश्च॑ - रा॒जिः॒ । अधि॑पत्नी॒ नाम॑ । अधि॑प॒त्नीत्यधि॑ - प॒त्नी॒ । नामा॑सि । अ॒सि॒ बृ॒ह॒ती । बृ॒ह॒ती दिक् । दिक् तस्याः᳚ । तस्या᳚स्ते । ते॒ बृह॒स्पतिः॑ । बृह॒स्पति॒रधि॑पतिः । अधि॑पतिः श्वि॒त्रः । अधि॑पति॒रित्यधि॑ - प॒तिः॒ । श्वि॒त्रो व॒शिनी᳚ । व॒शीनी॒ नाम॑ । नामा॑सि । अ॒सी॒यम् । इ॒यम् दिक् । दिक् तस्याः᳚ । तस्या᳚स्ते । ते॒ य॒मः । य॒मोऽधि॑पतिः । अधि॑पतिः क॒ल्माष॑ग्रीवः । अधि॑पति॒रित्यधि॑ - प॒तिः॒ । क॒ल्माष॑ग्रीवो रक्षि॒ता । क॒ल्माष॑ग्रीव॒ इति॑ क॒ल्माष॑ - ग्री॒वः॒ । र॒क्षि॒ता यः । यश्च॑ । चाधि॑पतिः । अधि॑पति॒र् यः । अधि॑पति॒रित्यधि॑ - प॒तिः॒ । यश्च॑ । च॒ गो॒प्ता । गो॒प्ता ताभ्या᳚म् । ताभ्या॒म् नमः॑ । नम॒स्तौ । तौ नः॑ । नो॒ मृ॒ड॒य॒ता॒म् । मृ॒ड॒य॒ता॒म् ते । ते यम् । यम् द्वि॒ष्मः । द्वि॒ष्मो यः । यश्च॑ । च॒ नः॒ \newline

\textbf{Jatai Paata} \newline

1. सोमो ऽधि॑पति॒ रधि॑पतिः॒ सोमः॒ सोमो ऽधि॑पतिः । \newline
2. अधि॑पतिः स्व॒जः स्व॒जो ऽधि॑पति॒ रधि॑पतिः स्व॒जः । \newline
3. अधि॑पति॒रित्यधि॑ - प॒तिः॒ । \newline
4. स्व॒जो॑ ऽव॒स्थावा॑ ऽव॒स्थावा᳚ स्व॒जः स्व॒जो॑ ऽव॒स्थावा᳚ । \newline
5. स्व॒ज इति॑ स्व - जः । \newline
6. अ॒व॒स्थावा॒ नाम॒ नामा॑ व॒स्थावा॑ ऽव॒स्थावा॒ नाम॑ । \newline
7. अ॒व॒स्थावेत्य॑व - स्थावा᳚ । \newline
8. नामा᳚स्यसि॒ नाम॒ नामा॑सि । \newline
9. अ॒स्युदी॒ च्युदी᳚ च्यस्य॒ स्युदी॑ची । \newline
10. उदी॑ची॒ दिग् दिगुदी॒ च्युदी॑ची॒ दिक् । \newline
11. दिक् तस्या॒ स्तस्या॒ दिग् दिक् तस्याः᳚ । \newline
12. तस्या᳚ स्ते ते॒ तस्या॒ स्तस्या᳚ स्ते । \newline
13. ते॒ वरु॑णो॒ वरु॑ण स्ते ते॒ वरु॑णः । \newline
14. वरु॒णो ऽधि॑पति॒ रधि॑पति॒र् वरु॑णो॒ वरु॒णो ऽधि॑पतिः । \newline
15. अधि॑पति स्ति॒रश्व॑राजि स्ति॒रश्व॑राजि॒ रधि॑पति॒ रधि॑पति स्ति॒रश्व॑राजिः । \newline
16. अधि॑पति॒रित्यधि॑ - प॒तिः॒ । \newline
17. ति॒रश्व॑राजि॒ रधि॑प॒त्न्य धि॑पत्नी ति॒रश्व॑राजि स्ति॒रश्व॑राजि॒ रधि॑पत्नी । \newline
18. ति॒रश्व॑राजि॒रिति॑ ति॒रश्व॑ - रा॒जिः॒ । \newline
19. अधि॑पत्नी॒ नाम॒ नामा धि॑प॒त्न्य धि॑पत्नी॒ नाम॑ । \newline
20. अधि॑प॒त्नीत्यधि॑ - प॒त्नी॒ । \newline
21. नामा᳚स्यसि॒ नाम॒ नामा॑सि । \newline
22. अ॒सि॒ बृ॒ह॒ती बृ॑ह॒ त्य॑स्यसि बृह॒ती । \newline
23. बृ॒ह॒ती दिग् दिग् बृ॑ह॒ती बृ॑ह॒ती दिक् । \newline
24. दिक् तस्या॒ स्तस्या॒ दिग् दिक् तस्याः᳚ । \newline
25. तस्या᳚ स्ते ते॒ तस्या॒ स्तस्या᳚ स्ते । \newline
26. ते॒ बृह॒स्पति॒र् बृह॒स्पति॑ स्ते ते॒ बृह॒स्पतिः॑ । \newline
27. बृह॒स्पति॒ रधि॑पति॒ रधि॑पति॒र् बृह॒स्पति॒र् बृह॒स्पति॒ रधि॑पतिः । \newline
28. अधि॑पतिः श्वि॒त्रः श्वि॒त्रो ऽधि॑पति॒ रधि॑पतिः श्वि॒त्रः । \newline
29. अधि॑पति॒रित्यधि॑ - प॒तिः॒ । \newline
30. श्वि॒त्रो व॒शिनी॑ व॒शिनी᳚ श्वि॒त्रः श्वि॒त्रो व॒शिनी᳚ । \newline
31. व॒शिनी॒ नाम॒ नाम॑ व॒शिनी॑ व॒शिनी॒ नाम॑ । \newline
32. नामा᳚स्यसि॒ नाम॒ नामा॑सि । \newline
33. अ॒सी॒य मि॒य म॑स्य सी॒यम् । \newline
34. इ॒यम् दिग् दिगि॒य मि॒यम् दिक् । \newline
35. दिक् तस्या॒ स्तस्या॒ दिग् दिक् तस्याः᳚ । \newline
36. तस्या᳚ स्ते ते॒ तस्या॒ स्तस्या᳚ स्ते । \newline
37. ते॒ य॒मो य॒म स्ते॑ ते य॒मः । \newline
38. य॒मो ऽधि॑पति॒ रधि॑पतिर् य॒मो य॒मो ऽधि॑पतिः । \newline
39. अधि॑पतिः क॒ल्माष॑ग्रीवः क॒ल्माष॑ग्री॒वो ऽधि॑पति॒ रधि॑पतिः क॒ल्माष॑ग्रीवः । \newline
40. अधि॑पति॒रित्यधि॑ - प॒तिः॒ । \newline
41. क॒ल्माष॑ग्रीवो रक्षि॒ता र॑क्षि॒ता क॒ल्माष॑ग्रीवः क॒ल्माष॑ग्रीवो रक्षि॒ता । \newline
42. क॒ल्माष॑ग्रीव॒ इति॑ क॒ल्माष॑ - ग्री॒वः॒ । \newline
43. र॒क्षि॒ता यो यो र॑क्षि॒ता र॑क्षि॒ता यः । \newline
44. यश्च॑ च॒ यो यश्च॑ । \newline
45. चाधि॑पति॒ रधि॑पतिश्च॒ चाधि॑पतिः । \newline
46. अधि॑पति॒र् यो यो ऽधि॑पति॒ रधि॑पति॒र् यः । \newline
47. अधि॑पति॒रित्यधि॑ - प॒तिः॒ । \newline
48. यश्च॑ च॒ यो यश्च॑ । \newline
49. च॒ गो॒प्ता गो॒प्ता च॑ च गो॒प्ता । \newline
50. गो॒प्ता ताभ्या॒म् ताभ्या᳚म् गो॒प्ता गो॒प्ता ताभ्या᳚म् । \newline
51. ताभ्या॒न् नमो॒ नम॒ स्ताभ्या॒म् ताभ्या॒न् नमः॑ । \newline
52. नम॒ स्तौ तौ नमो॒ नम॒ स्तौ । \newline
53. तौ नो॑ न॒ स्तौ तौ नः॑ । \newline
54. नो॒ मृ॒ड॒य॒ता॒म् मृ॒ड॒य॒ता॒न् नो॒ नो॒ मृ॒ड॒य॒ता॒म् । \newline
55. मृ॒ड॒य॒ता॒म् ते ते मृ॑डयताम् मृडयता॒म् ते । \newline
56. ते यं ॅयम् ते ते यम् । \newline
57. यम् द्वि॒ष्मो द्वि॒ष्मो यं ॅयम् द्वि॒ष्मः । \newline
58. द्वि॒ष्मो यो यो द्वि॒ष्मो द्वि॒ष्मो यः । \newline
59. यश्च॑ च॒ यो यश्च॑ । \newline
60. च॒ नो॒ न॒श्च॒ च॒ नः॒ । \newline

\textbf{Ghana Paata } \newline

1. सोमो ऽधि॑पति॒ रधि॑पतिः॒ सोमः॒ सोमो ऽधि॑पतिः स्व॒जः स्व॒जो ऽधि॑पतिः॒ सोमः॒ सोमो ऽधि॑पतिः स्व॒जः । \newline
2. अधि॑पतिः स्व॒जः स्व॒जो ऽधि॑पति॒ रधि॑पतिः स्व॒जो॑ ऽव॒स्थावा॑ ऽव॒स्थावा᳚ स्व॒जो ऽधि॑पति॒ रधि॑पतिः स्व॒जो॑ ऽव॒स्थावा᳚ । \newline
3. अधि॑पति॒रित्यधि॑ - प॒तिः॒ । \newline
4. स्व॒जो॑ ऽव॒स्थावा॑ ऽव॒स्थावा᳚ स्व॒जः स्व॒जो॑ ऽव॒स्थावा॒ नाम॒ नामा॑ व॒स्थावा᳚ स्व॒जः स्व॒जो॑ 
ऽव॒स्थावा॒॒ नाम॑ । \newline
5. स्व॒ज इति॑ स्व - जः । \newline
6. अ॒व॒स्थावा॒ नाम॒ नामा॑ व॒स्थावा॑ ऽव॒स्थावा॒ नामा᳚स्यसि॒ नामा॑ व॒स्थावा॑ ऽव॒स्थावा॒ नामा॑सि । \newline
7. अ॒व॒स्थावेत्य॑व - स्थावा᳚ । \newline
8. नामा᳚ स्यसि॒ नाम॒ नामा॒ स्युदी॒ च्युदी᳚ च्यसि॒ नाम॒ नामा॒ स्युदी॑ची । \newline
9. अ॒स्युदी॒ च्युदी᳚ च्यस्य॒ स्युदी॑ची॒ दिग् दिगुदी᳚ च्यस्य॒ स्युदी॑ची॒ दिक् । \newline
10. उदी॑ची॒ दिग् दिगुदी॒ च्युदी॑ची॒ दिक् तस्या॒ स्तस्या॒ दिगुदी॒ च्युदी॑ची॒ दिक् तस्याः᳚ । \newline
11. दिक् तस्या॒ स्तस्या॒ दिग् दिक् तस्या᳚ स्ते ते॒ तस्या॒ दिग् दिक् तस्या᳚ स्ते । \newline
12. तस्या᳚ स्ते ते॒ तस्या॒ स्तस्या᳚ स्ते॒ वरु॑णो॒ वरु॑ण स्ते॒ तस्या॒ स्तस्या᳚ स्ते॒ वरु॑णः । \newline
13. ते॒ वरु॑णो॒ वरु॑ण स्ते ते॒ वरु॒णो ऽधि॑पति॒ रधि॑पति॒र् वरु॑ण स्ते ते॒ वरु॒णो ऽधि॑पतिः । \newline
14. वरु॒णो ऽधि॑पति॒ रधि॑पति॒र् वरु॑णो॒ वरु॒णो ऽधि॑पति स्ति॒रश्व॑राजि स्ति॒रश्व॑राजि॒ रधि॑पति॒र् वरु॑णो॒ वरु॒णो ऽधि॑पति स्ति॒रश्व॑राजिः । \newline
15. अधि॑पति स्ति॒रश्व॑राजि स्ति॒रश्व॑राजि॒ रधि॑पति॒ रधि॑पति स्ति॒रश्व॑राजि॒ रधि॑प॒त्न्यधि॑पत्नी ति॒रश्व॑राजि॒ रधि॑पति॒ रधि॑पति स्ति॒रश्व॑राजि॒ रधि॑पत्नी । \newline
16. अधि॑पति॒रित्यधि॑ - प॒तिः॒ । \newline
17. ति॒रश्व॑राजि॒ रधि॑प॒त्न्य धि॑पत्नी ति॒रश्व॑राजि स्ति॒रश्व॑राजि॒ रधि॑पत्नी॒ नाम॒ नामाधि॑पत्नी ति॒रश्व॑राजि स्ति॒रश्व॑राजि॒ रधि॑पत्नी॒ नाम॑ । \newline
18. ति॒रश्व॑राजि॒रिति॑ ति॒रश्व॑ - रा॒जिः॒ । \newline
19. अधि॑पत्नी॒ नाम॒ नामाधि॑ प॒त्न्यधि॑पत्नी॒ नामा᳚स्यसि॒ नामाधि॑प॒त्न्य धि॑पत्नी॒ नामा॑सि । \newline
20. अधि॑प॒त्नीत्यधि॑ - प॒त्नी॒ । \newline
21. नामा᳚स्यसि॒ नाम॒ नामा॑सि बृह॒ती बृ॑ह॒ त्य॑सि॒ नाम॒ नामा॑सि बृह॒ती । \newline
22. अ॒सि॒ बृ॒ह॒ती बृ॑ह॒ त्य॑स्यसि बृह॒ती दिग् दिग् बृ॑ह॒ त्य॑स्यसि बृह॒ती दिक् । \newline
23. बृ॒ह॒ती दिग् दिग् बृ॑ह॒ती बृ॑ह॒ती दिक् तस्या॒ स्तस्या॒ दिग् बृ॑ह॒ती बृ॑ह॒ती दिक् तस्याः᳚ । \newline
24. दिक् तस्या॒ स्तस्या॒ दिग् दिक् तस्या᳚ स्ते ते॒ तस्या॒ दिग् दिक् तस्या᳚ स्ते । \newline
25. तस्या᳚ स्ते ते॒ तस्या॒ स्तस्या᳚ स्ते॒ बृह॒स्पति॒र् बृह॒स्पति॑ स्ते॒ तस्या॒ स्तस्या᳚ स्ते॒ बृह॒स्पतिः॑ । \newline
26. ते॒ बृह॒स्पति॒र् बृह॒स्पति॑ स्ते ते॒ बृह॒स्पति॒ रधि॑पति॒ रधि॑पति॒र् बृह॒स्पति॑ स्ते ते॒ बृह॒स्पति॒ रधि॑पतिः । \newline
27. बृह॒स्पति॒ रधि॑पति॒ रधि॑पति॒र् बृह॒स्पति॒र् बृह॒स्पति॒ रधि॑पतिः श्वि॒त्रः श्वि॒त्रो ऽधि॑पति॒र् बृह॒स्पति॒र् बृह॒स्पति॒ रधि॑पतिः श्वि॒त्रः । \newline
28. अधि॑पतिः श्वि॒त्रः श्वि॒त्रो ऽधि॑पति॒ रधि॑पतिः श्वि॒त्रो व॒शिनी॑ व॒शिनी᳚ श्वि॒त्रो ऽधि॑पति॒ रधि॑पतिः श्वि॒त्रो व॒शिनी᳚ । \newline
29. अधि॑पति॒रित्यधि॑ - प॒तिः॒ । \newline
30. श्वि॒त्रो व॒शिनी॑ व॒शिनी᳚ श्वि॒त्रः श्वि॒त्रो व॒शिनी॒ नाम॒ नाम॑ व॒शिनी᳚ श्वि॒त्रः श्वि॒त्रो व॒शिनी॒ नाम॑ । \newline
31. व॒शिनी॒ नाम॒ नाम॑ व॒शिनी॑ व॒शिनी॒ नामा᳚स्यसि॒ नाम॑ व॒शिनी॑ व॒शिनी॒ नामा॑सि । \newline
32. नामा᳚स्यसि॒ नाम॒ नामा॑सी॒य मि॒य म॑सि॒ नाम॒ नामा॑सी॒यम् । \newline
33. अ॒सी॒य मि॒य म॑स्य सी॒यम् दिग् दिगि॒य म॑स्यसी॒यम् दिक् । \newline
34. इ॒यम् दिग् दिगि॒य मि॒यम् दिक् तस्या॒ स्तस्या॒ दिगि॒य मि॒यम् दिक् तस्याः᳚ । \newline
35. दिक् तस्या॒ स्तस्या॒ दिग् दिक् तस्या᳚ स्ते ते॒ तस्या॒ दिग् दिक् तस्या᳚ स्ते । \newline
36. तस्या᳚ स्ते ते॒ तस्या॒ स्तस्या᳚ स्ते य॒मो य॒म स्ते॒ तस्या॒ स्तस्या᳚ स्ते य॒मः । \newline
37. ते॒ य॒मो य॒म स्ते॑ ते य॒मो ऽधि॑पति॒ रधि॑पतिर् य॒म स्ते॑ ते य॒मो ऽधि॑पतिः । \newline
38. य॒मो ऽधि॑पति॒ रधि॑पतिर् य॒मो य॒मो ऽधि॑पतिः क॒ल्माष॑ग्रीवः क॒ल्माष॑ग्री॒वो ऽधि॑पतिर् य॒मो य॒मो ऽधि॑पतिः क॒ल्माष॑ग्रीवः । \newline
39. अधि॑पतिः क॒ल्माष॑ग्रीवः क॒ल्माष॑ग्री॒वो ऽधि॑पति॒ रधि॑पतिः क॒ल्माष॑ग्रीवो रक्षि॒ता र॑क्षि॒ता क॒ल्माष॑ग्री॒वो ऽधि॑पति॒ रधि॑पतिः क॒ल्माष॑ग्रीवो रक्षि॒ता । \newline
40. अधि॑पति॒रित्यधि॑ - प॒तिः॒ । \newline
41. क॒ल्माष॑ग्रीवो रक्षि॒ता र॑क्षि॒ता क॒ल्माष॑ग्रीवः क॒ल्माष॑ग्रीवो रक्षि॒ता यो यो र॑क्षि॒ता क॒ल्माष॑ग्रीवः क॒ल्माष॑ग्रीवो रक्षि॒ता यः । \newline
42. क॒ल्माष॑ग्रीव॒ इति॑ क॒ल्माष॑ - ग्री॒वः॒ । \newline
43. र॒क्षि॒ता यो यो र॑क्षि॒ता र॑क्षि॒ता यश्च॑ च॒ यो र॑क्षि॒ता र॑क्षि॒ता यश्च॑ । \newline
44. यश्च॑ च॒ यो यश्चाधि॑पति॒ रधि॑पतिश्च॒ यो यश्चाधि॑पतिः । \newline
45. चाधि॑पति॒ रधि॑पतिश्च॒ चाधि॑पति॒र् यो यो ऽधि॑पतिश्च॒ चाधि॑पति॒र् यः । \newline
46. अधि॑पति॒र् यो यो ऽधि॑पति॒ रधि॑पति॒र् यश्च॑ च॒ यो ऽधि॑पति॒ रधि॑पति॒र् यश्च॑ । \newline
47. अधि॑पति॒रित्यधि॑ - प॒तिः॒ । \newline
48. यश्च॑ च॒ यो यश्च॑ गो॒प्ता गो॒प्ता च॒ यो यश्च॑ गो॒प्ता । \newline
49. च॒ गो॒प्ता गो॒प्ता च॑ च गो॒प्ता ताभ्या॒म् ताभ्या᳚म् गो॒प्ता च॑ च गो॒प्ता ताभ्या᳚म् । \newline
50. गो॒प्ता ताभ्या॒म् ताभ्या᳚म् गो॒प्ता गो॒प्ता ताभ्या॒म् नमो॒ नम॒ स्ताभ्या᳚म् गो॒प्ता गो॒प्ता ताभ्या॒म् नमः॑ । \newline
51. ताभ्या॒म् नमो॒ नम॒ स्ताभ्या॒म् ताभ्या॒म् नम॒ स्तौ तौ नम॒ स्ताभ्या॒म् ताभ्या॒म् नम॒ स्तौ । \newline
52. नम॒ स्तौ तौ नमो॒ नम॒ स्तौ नो॑ न॒ स्तौ नमो॒ नम॒ स्तौ नः॑ । \newline
53. तौ नो॑ न॒ स्तौ तौ नो॑ मृडयताम् मृडयताम् न॒ स्तौ तौ नो॑ मृडयताम् । \newline
54. नो॒ मृ॒ड॒य॒ता॒म् मृ॒ड॒य॒ता॒म् नो॒ नो॒ मृ॒ड॒य॒ता॒म् ते ते मृ॑डयताम् नो नो मृडयता॒म् ते । \newline
55. मृ॒ड॒य॒ता॒म् ते ते मृ॑डयताम् मृडयता॒म् ते यं ॅयम् ते मृ॑डयताम् मृडयता॒म् ते यम् । \newline
56. ते यं ॅयम् ते ते यम् द्वि॒ष्मो द्वि॒ष्मो यम् ते ते यम् द्वि॒ष्मः । \newline
57. यम् द्वि॒ष्मो द्वि॒ष्मो यं ॅयम् द्वि॒ष्मो यो यो द्वि॒ष्मो यं ॅयम् द्वि॒ष्मो यः । \newline
58. द्वि॒ष्मो यो यो द्वि॒ष्मो द्वि॒ष्मो यश्च॑ च॒ यो द्वि॒ष्मो द्वि॒ष्मो यश्च॑ । \newline
59. यश्च॑ च॒ यो यश्च॑ नो नश्च॒ यो यश्च॑ नः । \newline
60. च॒ नो॒ न॒श्च॒ च॒ नो॒ द्वेष्टि॒ द्वेष्टि॑ नश्च च नो॒ द्वेष्टि॑ । \newline
\pagebreak
\markright{ TS 5.5.10.3  \hfill https://www.vedavms.in \hfill}

\section{ TS 5.5.10.3 }

\textbf{TS 5.5.10.3 } \newline
\textbf{Samhita Paata} \newline

नो॒ द्वेष्टि॒ तं ॅवां॒ जंभे॑ दधाम्ये॒ता वै दे॒वता॑ अ॒ग्निं चि॒तꣳ र॑क्षन्ति॒ ताभ्यो॒ यदाहु॑ती॒र्न जु॑हु॒याद॑द्ध्व॒र्युं च॒ यज॑मानं च ध्यायेयु॒र्यदे॒ता आहु॑तीर्जु॒होति॑ भाग॒धेये॑नै-वै॒ना᳚ञ्छमयति॒ नाऽऽ*र्ति॒मार्च्छ॑त्यद्ध्व॒र्युर्न यज॑मानो हे॒तयो॒ नाम॑ स्थ॒ तेषां᳚ ॅवः पु॒रो गृ॒हा अ॒ग्निर्व॒ इष॑वः सलि॒लो नि॑लि॒पां नाम॑ - [  ] \newline

\textbf{Pada Paata} \newline

नः॒ । द्वेष्टि॑ । तम् । वा॒म् । जम्भे᳚ । द॒धा॒मि॒ । ए॒ताः । वै । दे॒वताः᳚ । अ॒ग्निम् । चि॒तम् । र॒क्ष॒न्ति॒ । ताभ्यः॑ । यत् । आहु॑ती॒रित्या - हु॒तीः॒ । न । जु॒हु॒यात् । अ॒द्ध्व॒र्युम् । च॒ । यज॑मानम् । च॒ । ध्या॒ये॒युः॒ । यत् । ए॒ताः । आहु॑ती॒रित्या - हु॒तीः॒ । जु॒होति॑ । भा॒ग॒धेये॒नेति॑ भाग-धेये॑न । ए॒व । ए॒ना॒न् । श॒म॒य॒ति॒ । न । आर्ति᳚म् । एति॑ । ऋ॒च्छ॒ति॒ । अ॒द्ध्व॒र्युः । न । यज॑मानः । हे॒तयः॑ । नाम॑ । स्थ॒ । तेषा᳚म् । वः॒ । पु॒रः । गृ॒हाः । अ॒ग्निः । वः॒ । इष॑वः । स॒लि॒लः । नि॒लि॒पां इति॑ नि-लि॒पांः । नाम॑ ।  \newline


\textbf{Krama Paata} \newline

नो॒ द्वेष्टि॑ । द्वेष्टि॒ तम् । तम् ॅवा᳚म् । वा॒म् जम्भे᳚ । जम्भे॑ दधामि । द॒धा॒म्ये॒ताः । ए॒ता वै । वै दे॒वताः᳚ । दे॒वता॑ अ॒ग्निम् । अ॒ग्निम् चि॒तम् । चि॒तꣳ र॑क्षन्ति । र॒क्ष॒न्ति॒ ताभ्यः॑ । ताभ्यो॒ यत् । यदाहु॑तीः । आहु॑ती॒र् न । आहु॑ती॒रित्या - हु॒तीः॒ । न जु॑हु॒यात् । जु॒हु॒याद॑द्ध्व॒र्युम् । अ॒द्ध्व॒र्युम् च॑ । च॒ यज॑मानम् । यज॑मानम् च । च॒ ध्या॒ये॒युः॒ । ध्या॒ये॒यु॒र् यत् । यदे॒ताः । ए॒ता आहु॑तीः । आहु॑तीर् जु॒होति॑ । आहु॑ती॒रित्या - हु॒तीः॒ । जु॒होति॑ भाग॒धेये॑न । भा॒ग॒धेये॑नै॒व । भा॒ग॒धेये॒नेति॑ भाग - धेये॑न । ए॒वैनान्॑ । ए॒ना॒ञ्छ॒॒म॒य॒ति॒ । श॒म॒य॒ति॒ न । नार्ति᳚म् । आर्ति॒मा । आर्च्छ॑ति । ऋ॒च्छ॒त्य॒द्ध्व॒र्युः । अ॒द्ध्व॒र्युर् न । न यज॑मानः । यज॑मानो हे॒तयः॑ । हे॒तयो॒ नाम॑ । नाम॑ स्थ । स्थ॒ तेषा᳚म् । तेषा᳚म् ॅवः । वः॒ पु॒रः । पु॒रो गृ॒हाः । गृ॒हा अ॒ग्निः । अ॒ग्निर् वः॑ । व॒ इष॑वः । इष॑वः सलि॒लः । स॒लि॒लो नि॑लि॒म्पाः । नि॒लि॒म्पा नाम॑ । नि॒लि॒म्पा इति॑ नि - लि॒म्पाः । नाम॑ स्थ \newline

\textbf{Jatai Paata} \newline

1. नो॒ द्वेष्टि॒ द्वेष्टि॑ नो नो॒ द्वेष्टि॑ । \newline
2. द्वेष्टि॒ तम् तम् द्वेष्टि॒ द्वेष्टि॒ तम् । \newline
3. तं ॅवां᳚ ॅवा॒म् तम् तं ॅवा᳚म् । \newline
4. वा॒म् जम्भे॒ जम्भे॑ वां ॅवा॒म् जम्भे᳚ । \newline
5. जम्भे॑ दधामि दधामि॒ जम्भे॒ जम्भे॑ दधामि । \newline
6. द॒धा॒ म्ये॒ता ए॒ता द॑धामि दधा म्ये॒ताः । \newline
7. ए॒ता वै वा ए॒ता ए॒ता वै । \newline
8. वै दे॒वता॑ दे॒वता॒ वै वै दे॒वताः᳚ । \newline
9. दे॒वता॑ अ॒ग्नि म॒ग्निम् दे॒वता॑ दे॒वता॑ अ॒ग्निम् । \newline
10. अ॒ग्निम् चि॒तम् चि॒त म॒ग्नि म॒ग्निम् चि॒तम् । \newline
11. चि॒तꣳ र॑क्षन्ति रक्षन्ति चि॒तम् चि॒तꣳ र॑क्षन्ति । \newline
12. र॒क्ष॒न्ति॒ ताभ्य॒ स्ताभ्यो॑ रक्षन्ति रक्षन्ति॒ ताभ्यः॑ । \newline
13. ताभ्यो॒ यद् यत् ताभ्य॒ स्ताभ्यो॒ यत् । \newline
14. यदाहु॑ती॒ राहु॑ती॒र् यद् यदाहु॑तीः । \newline
15. आहु॑ती॒र् न नाहु॑ती॒ राहु॑ती॒र् न । \newline
16. आहु॑ती॒रित्या - हु॒तीः॒ । \newline
17. न जु॑हु॒याज् जु॑हु॒यान् न न जु॑हु॒यात् । \newline
18. जु॒हु॒या द॑द्ध्व॒र्यु म॑द्ध्व॒र्युम् जु॑हु॒याज् जु॑हु॒या द॑द्ध्व॒र्युम् । \newline
19. अ॒द्ध्व॒र्युम् च॑ चाद्ध्व॒र्यु म॑द्ध्व॒र्युम् च॑ । \newline
20. च॒ यज॑मानं॒ ॅयज॑मानम् च च॒ यज॑मानम् । \newline
21. यज॑मानम् च च॒ यज॑मानं॒ ॅयज॑मानम् च । \newline
22. च॒ ध्या॒ये॒यु॒र् ध्या॒ये॒यु॒ श्च॒ च॒ ध्या॒ये॒युः॒ । \newline
23. ध्या॒ये॒यु॒र् यद् यद् ध्या॑येयुर् ध्यायेयु॒र् यत् । \newline
24. यदे॒ता ए॒ता यद् यदे॒ताः । \newline
25. ए॒ता आहु॑ती॒ राहु॑ती रे॒ता ए॒ता आहु॑तीः । \newline
26. आहु॑तीर् जु॒होति॑ जु॒हो त्याहु॑ती॒ राहु॑तीर् जु॒होति॑ । \newline
27. आहु॑ती॒रित्या - हु॒तीः॒ । \newline
28. जु॒होति॑ भाग॒धेये॑न भाग॒धेये॑न जु॒होति॑ जु॒होति॑ भाग॒धेये॑न । \newline
29. भा॒ग॒धेये॑नै॒ वैव भा॑ग॒धेये॑न भाग॒धेये॑नै॒व । \newline
30. भा॒ग॒धेये॒नेति॑ भाग - धेये॑न । \newline
31. ए॒वैना॑ नेना ने॒वै वैनान्॑ । \newline
32. ए॒ना॒ञ् छ॒म॒य॒ति॒ श॒म॒य॒ त्ये॒ना॒ ने॒ना॒ञ् छ॒म॒य॒ति॒ । \newline
33. श॒म॒य॒ति॒ न न श॑मयति शमयति॒ न । \newline
34. नार्ति॒ मार्ति॒न् न नार्ति᳚म् । \newline
35. आर्ति॒ मा ऽऽर्ति॒ मार्ति॒ मा । \newline
36. आर्च्छ॑ त्यृच्छ॒ त्यार्च्छ॑ति । \newline
37. ऋ॒च्छ॒ त्य॒द्ध्व॒र्यु र॑द्ध्व॒र्युर्.ऋ॑च्छ त्यृच्छ त्यद्ध्व॒र्युः । \newline
38. अ॒द्ध्व॒र्युर् न नाद्ध्व॒र्यु र॑द्ध्व॒र्युर् न । \newline
39. न यज॑मानो॒ यज॑मानो॒ न न यज॑मानः । \newline
40. यज॑मानो हे॒तयो॑ हे॒तयो॒ यज॑मानो॒ यज॑मानो हे॒तयः॑ । \newline
41. हे॒तयो॒ नाम॒ नाम॑ हे॒तयो॑ हे॒तयो॒ नाम॑ । \newline
42. नाम॑ स्थ स्थ॒ नाम॒ नाम॑ स्थ । \newline
43. स्थ॒ तेषा॒म् तेषाꣳ॑ स्थ स्थ॒ तेषा᳚म् । \newline
44. तेषां᳚ ॅवो व॒ स्तेषा॒म् तेषां᳚ ॅवः । \newline
45. वः॒ पु॒रः पु॒रो वो॑ वः पु॒रः । \newline
46. पु॒रो गृ॒हा गृ॒हाः पु॒रः पु॒रो गृ॒हाः । \newline
47. गृ॒हा अ॒ग्नि र॒ग्निर् गृ॒हा गृ॒हा अ॒ग्निः । \newline
48. अ॒ग्निर् वो॑ वो॒ ऽग्नि र॒ग्निर् वः॑ । \newline
49. व॒ इष॑व॒ इष॑वो वो व॒ इष॑वः । \newline
50. इष॑वः सलि॒लः स॑लि॒ल इष॑व॒ इष॑वः सलि॒लः । \newline
51. स॒लि॒लो नि॑लि॒म्पा नि॑लि॒म्पाः स॑लि॒लः स॑लि॒लो नि॑लि॒म्पाः । \newline
52. नि॒लि॒म्पा नाम॒ नाम॑ निलि॒म्पा नि॑लि॒म्पा नाम॑ । \newline
53. नि॒लि॒म्पा इति॑ नि - लि॒म्पाः । \newline
54. नाम॑ स्थ स्थ॒ नाम॒ नाम॑ स्थ । \newline

\textbf{Ghana Paata } \newline

1. नो॒ द्वेष्टि॒ द्वेष्टि॑ नो नो॒ द्वेष्टि॒ तम् तम् द्वेष्टि॑ नो नो॒ द्वेष्टि॒ तम् । \newline
2. द्वेष्टि॒ तम् तम् द्वेष्टि॒ द्वेष्टि॒ तं ॅवां᳚ ॅवा॒म् तम् द्वेष्टि॒ द्वेष्टि॒ तं ॅवा᳚म् । \newline
3. तं ॅवां᳚ ॅवा॒म् तम् तं ॅवा॒म् जम्भे॒ जम्भे॑ वा॒म् तम् तं ॅवा॒म् जम्भे᳚ । \newline
4. वा॒म् जम्भे॒ जम्भे॑ वां ॅवा॒म् जम्भे॑ दधामि दधामि॒ जम्भे॑ वां ॅवा॒म् जम्भे॑ दधामि । \newline
5. जम्भे॑ दधामि दधामि॒ जम्भे॒ जम्भे॑ दधा म्ये॒ता ए॒ता द॑धामि॒ जम्भे॒ जम्भे॑ दधा म्ये॒ताः । \newline
6. द॒धा॒ म्ये॒ता ए॒ता द॑धामि दधा म्ये॒ता वै वा ए॒ता द॑धामि दधा म्ये॒ता वै । \newline
7. ए॒ता वै वा ए॒ता ए॒ता वै दे॒वता॑ दे॒वता॒ वा ए॒ता ए॒ता वै दे॒वताः᳚ । \newline
8. वै दे॒वता॑ दे॒वता॒ वै वै दे॒वता॑ अ॒ग्नि म॒ग्निम् दे॒वता॒ वै वै दे॒वता॑ अ॒ग्निम् । \newline
9. दे॒वता॑ अ॒ग्नि म॒ग्निम् दे॒वता॑ दे॒वता॑ अ॒ग्निम् चि॒तम् चि॒त म॒ग्निम् दे॒वता॑ दे॒वता॑ अ॒ग्निम् चि॒तम् । \newline
10. अ॒ग्निम् चि॒तम् चि॒त म॒ग्नि म॒ग्निम् चि॒तꣳ र॑क्षन्ति रक्षन्ति चि॒त म॒ग्नि म॒ग्निम् चि॒तꣳ र॑क्षन्ति । \newline
11. चि॒तꣳ र॑क्षन्ति रक्षन्ति चि॒तम् चि॒तꣳ र॑क्षन्ति॒ ताभ्य॒ स्ताभ्यो॑ रक्षन्ति चि॒तम् चि॒तꣳ र॑क्षन्ति॒ ताभ्यः॑ । \newline
12. र॒क्ष॒न्ति॒ ताभ्य॒ स्ताभ्यो॑ रक्षन्ति रक्षन्ति॒ ताभ्यो॒ यद् यत् ताभ्यो॑ रक्षन्ति रक्षन्ति॒ ताभ्यो॒ यत् । \newline
13. ताभ्यो॒ यद् यत् ताभ्य॒ स्ताभ्यो॒ यदा हु॑ती॒ राहु॑ती॒र् यत् ताभ्य॒ स्ताभ्यो॒ यदा हु॑तीः । \newline
14. यदाहु॑ती॒ राहु॑ती॒र् यद् यदाहु॑ती॒र् न नाहु॑ती॒र् यद् यदाहु॑ती॒र् न । \newline
15. आहु॑ती॒र् न नाहु॑ती॒ राहु॑ती॒र् न जु॑हु॒याज् जु॑हु॒यान् नाहु॑ती॒ राहु॑ती॒र् न जु॑हु॒यात् । \newline
16. आहु॑ती॒रित्या - हु॒तीः॒ । \newline
17. न जु॑हु॒याज् जु॑हु॒यान् न न जु॑हु॒या द॑द्ध्व॒र्यु म॑द्ध्व॒र्युम् जु॑हु॒यान् न न जु॑हु॒या द॑द्ध्व॒र्युम् । \newline
18. जु॒हु॒या द॑द्ध्व॒र्यु म॑द्ध्व॒र्युम् जु॑हु॒याज् जु॑हु॒या द॑द्ध्व॒र्युम् च॑ चाद्ध्व॒र्युम् जु॑हु॒याज् जु॑हु॒या द॑द्ध्व॒र्युम् च॑ । \newline
19. अ॒द्ध्व॒र्युम् च॑ चाद्ध्व॒र्यु म॑द्ध्व॒र्युम् च॒ यज॑मानं॒ ॅयज॑मानम् चाद्ध्व॒र्यु म॑द्ध्व॒र्युम् च॒ यज॑मानम् । \newline
20. च॒ यज॑मानं॒ ॅयज॑मानम् च च॒ यज॑मानम् च च॒ यज॑मानम् च च॒ यज॑मानम् च । \newline
21. यज॑मानम् च च॒ यज॑मानं॒ ॅयज॑मानम् च ध्यायेयुर् ध्यायेयुश्च॒ यज॑मानं॒ ॅयज॑मानम् च ध्यायेयुः । \newline
22. च॒ ध्या॒ये॒यु॒र् ध्या॒ये॒यु॒श्च॒ च॒ ध्या॒ये॒यु॒र् यद् यद् ध्या॑येयुश्च च ध्यायेयु॒र् यत् । \newline
23. ध्या॒ये॒यु॒र् यद् यद् ध्या॑येयुर् ध्यायेयु॒र् यदे॒ता ए॒ता यद् ध्या॑येयुर् ध्यायेयु॒र् यदे॒ताः । \newline
24. यदे॒ता ए॒ता यद् यदे॒ता आहु॑ती॒ राहु॑ती रे॒ता यद् यदे॒ता आहु॑तीः । \newline
25. ए॒ता आहु॑ती॒ राहु॑ती रे॒ता ए॒ता आहु॑तीर् जु॒होति॑ जु॒हो त्याहु॑ती रे॒ता ए॒ता आहु॑तीर् जु॒होति॑ । \newline
26. आहु॑तीर् जु॒होति॑ जु॒हो त्याहु॑ती॒ राहु॑तीर् जु॒होति॑ भाग॒धेये॑न भाग॒धेये॑न जु॒हो त्याहु॑ती॒ राहु॑तीर् जु॒होति॑ भाग॒धेये॑न । \newline
27. आहु॑ती॒रित्या - हु॒तीः॒ । \newline
28. जु॒होति॑ भाग॒धेये॑न भाग॒धेये॑न जु॒होति॑ जु॒होति॑ भाग॒धेये॑ नै॒वैव भा॑ग॒धेये॑न जु॒होति॑ जु॒होति॑ भाग॒धेये॑नै॒व । \newline
29. भा॒ग॒धेये॑नै॒ वैव भा॑ग॒धेये॑न भाग॒धेये॑ नै॒वैना॑ नेना ने॒व भा॑ग॒धेये॑न भाग॒धेये॑
नै॒वैनान्॑ । \newline
30. भा॒ग॒धेये॒नेति॑ भाग - धेये॑न । \newline
31. ए॒वैना॑ नेना ने॒वै वैना᳚ञ् छमयति शमय त्येना ने॒वै वैना᳚ञ् छमयति । \newline
32. ए॒ना॒ञ् छ॒म॒य॒ति॒ श॒म॒य॒ त्ये॒ना॒ ने॒ना॒ञ् छ॒म॒य॒ति॒ न न श॑मय त्येना नेनाञ् छमयति॒ न । \newline
33. श॒म॒य॒ति॒ न न श॑मयति शमयति॒ नार्ति॒ मार्ति॒म् न श॑मयति शमयति॒ नार्ति᳚म् । \newline
34. नार्ति॒ मार्ति॒म् न नार्ति॒ मा ऽऽर्ति॒म् न नार्ति॒ मा । \newline
35. आर्ति॒ मा ऽऽर्ति॒ मार्ति॒ मार्च्छ॑ त्यृच्छ॒ त्याऽऽर्ति॒ मार्ति॒ मार्च्छ॑ति । \newline
36. आर्च्छ॑ त्यृच्छ॒ त्यार्च्छ॑ त्यद्ध्व॒र्यु र॑द्ध्व॒र्युर्. ऋ॑च्छ॒ त्यार्च्छ॑ त्यद्ध्व॒र्युः । \newline
37. ऋ॒च्छ॒ त्य॒द्ध्व॒र्यु र॑द्ध्व॒र्युर्. ऋ॑च्छ त्यृच्छ त्यद्ध्व॒र्युर् न नाद्ध्व॒र्युर्. ऋ॑च्छ त्यृच्छ त्यद्ध्व॒र्युर् न । \newline
38. अ॒द्ध्व॒र्युर् न नाद्ध्व॒र्यु र॑द्ध्व॒र्युर् न यज॑मानो॒ यज॑मानो॒ नाद्ध्व॒र्यु र॑द्ध्व॒र्युर् न यज॑मानः । \newline
39. न यज॑मानो॒ यज॑मानो॒ न न यज॑मानो हे॒तयो॑ हे॒तयो॒ यज॑मानो॒ न न यज॑मानो हे॒तयः॑ । \newline
40. यज॑मानो हे॒तयो॑ हे॒तयो॒ यज॑मानो॒ यज॑मानो हे॒तयो॒ नाम॒ नाम॑ हे॒तयो॒ यज॑मानो॒ यज॑मानो हे॒तयो॒ नाम॑ । \newline
41. हे॒तयो॒ नाम॒ नाम॑ हे॒तयो॑ हे॒तयो॒ नाम॑ स्थ स्थ॒ नाम॑ हे॒तयो॑ हे॒तयो॒ नाम॑ स्थ । \newline
42. नाम॑ स्थ स्थ॒ नाम॒ नाम॑ स्थ॒ तेषा॒म् तेषाꣳ॑ स्थ॒ नाम॒ नाम॑ स्थ॒ तेषा᳚म् । \newline
43. स्थ॒ तेषा॒म् तेषाꣳ॑ स्थ स्थ॒ तेषां᳚ ॅवो व॒स्तेषाꣳ॑ स्थ स्थ॒ तेषां᳚ ॅवः । \newline
44. तेषां᳚ ॅवो व॒ स्तेषा॒म् तेषां᳚ ॅवः पु॒रः पु॒रो व॒ स्तेषा॒म् तेषां᳚ ॅवः पु॒रः । \newline
45. वः॒ पु॒रः पु॒रो वो॑ वः पु॒रो गृ॒हा गृ॒हाः पु॒रो वो॑ वः पु॒रो गृ॒हाः । \newline
46. पु॒रो गृ॒हा गृ॒हाः पु॒रः पु॒रो गृ॒हा अ॒ग्नि र॒ग्निर् गृ॒हाः पु॒रः पु॒रो गृ॒हा अ॒ग्निः । \newline
47. गृ॒हा अ॒ग्नि र॒ग्निर् गृ॒हा गृ॒हा अ॒ग्निर् वो॑ वो॒ ऽग्निर् गृ॒हा गृ॒हा अ॒ग्निर् वः॑ । \newline
48. अ॒ग्निर् वो॑ वो॒ ऽग्नि र॒ग्निर् व॒ इष॑व॒ इष॑वो वो॒ ऽग्नि र॒ग्निर् व॒ इष॑वः । \newline
49. व॒ इष॑व॒ इष॑वो वो व॒ इष॑वः सलि॒लः स॑लि॒ल इष॑वो वो व॒ इष॑वः सलि॒लः । \newline
50. इष॑वः सलि॒लः स॑लि॒ल इष॑व॒ इष॑वः सलि॒लो नि॑लि॒म्पा नि॑लि॒म्पाः स॑लि॒ल इष॑व॒ इष॑वः सलि॒लो नि॑लि॒म्पाः । \newline
51. स॒लि॒लो नि॑लि॒म्पा नि॑लि॒म्पाः स॑लि॒लः स॑लि॒लो नि॑लि॒म्पा नाम॒ नाम॑ निलि॒म्पाः स॑लि॒लः स॑लि॒लो नि॑लि॒म्पा नाम॑ । \newline
52. नि॒लि॒म्पा नाम॒ नाम॑ निलि॒म्पा नि॑लि॒म्पा नाम॑ स्थ स्थ॒ नाम॑ निलि॒म्पा नि॑लि॒म्पा नाम॑ स्थ । \newline
53. नि॒लि॒म्पा इति॑ नि - लि॒म्पाः । \newline
54. नाम॑ स्थ स्थ॒ नाम॒ नाम॑ स्थ॒ तेषा॒म् तेषाꣳ॑ स्थ॒ नाम॒ नाम॑ स्थ॒ तेषा᳚म् । \newline
\pagebreak
\markright{ TS 5.5.10.4  \hfill https://www.vedavms.in \hfill}

\section{ TS 5.5.10.4 }

\textbf{TS 5.5.10.4 } \newline
\textbf{Samhita Paata} \newline

स्थ॒ तेषां᳚ ॅवो दक्षि॒णा गृ॒हाः पि॒तरो॑ व॒ इष॑वः॒ सग॑रो व॒ज्रिणो॒ नाम॑ स्थ॒ तेषां᳚ ॅवः प॒श्चाद् गृ॒हाः स्वप्नो॑ व॒ इष॑वो॒ गह्व॑रो ऽव॒स्थावा॑नो॒ नाम॑ स्थ॒ तेषां᳚ ॅव उत्त॒राद् गृ॒हा आपो॑ व॒ इष॑वः समु॒द्रो-ऽधि॑पतयो॒ नाम॑ स्थ॒ तेषां᳚ ॅव उ॒परि॑ गृ॒हा व॒र्॒.षं ॅव॒ इष॒वोऽव॑स्वान् क्र॒व्या नाम॑ स्थ॒ पार्त्थि॑वा॒-स्तेषां᳚ ॅव इ॒ह गृ॒हा - [  ] \newline

\textbf{Pada Paata} \newline

स्थ॒ । तेषा᳚म् । वः॒ । द॒क्षि॒णा । गृ॒हाः । पि॒तरः॑ । वः॒ । इष॑वः । सग॑रः । व॒ज्रिणः॑ । नाम॑ । स्थ॒ । तेषा᳚म् । वः॒ । प॒श्चात् । गृ॒हाः । स्वप्नः॑ । वः॒ । इष॑वः । गह्व॑रः । अ॒व॒स्थावा॑न॒ इत्य॑व - स्थावा॑नः । नाम॑ । स्थ॒ । तेषा᳚म् । वः॒ । उ॒त्त॒रादित्यु॑त् - त॒रात् । गृ॒हाः । आपः॑ । वः॒ । इष॑वः । स॒मु॒द्रः । अधि॑पतय॒ इत्यधि॑ - प॒त॒यः॒ । नाम॑ । स्थ॒ । तेषा᳚म् । वः॒ । उ॒परि॑ । गृ॒हाः । व॒र्.॒षम् । वः॒ । इष॑वः । अव॑स्वान् । क्र॒व्याः । नाम॑ । स्थ॒ । पार्थि॑वाः । तेषा᳚म् । वः॒ । इ॒ह । गृ॒हाः ।  \newline


\textbf{Krama Paata} \newline

स्थ॒ तेषा᳚म् । तेषा᳚म् ॅवः । वो॒ द॒क्षि॒णा । द॒क्षि॒णा गृ॒हाः । गृ॒हाः पि॒तरः॑ । पि॒तरो॑ वः । व॒ इष॑वः । इष॑वः॒ सग॑रः । सग॑रो व॒ज्रिणः॑ । व॒ज्रिणो॒ नाम॑ । नाम॑ स्थ । स्थ॒ तेषा᳚म् । तेषा᳚म् ॅवः । वः॒ प॒श्चात् । प॒श्चाद् गृ॒हाः । गृ॒हाः स्वप्नः॑ । स्वप्नो॑ वः । व॒ इष॑वः । इष॑वो॒ गह्व॑रः । गह्व॑रोऽव॒स्थावा॑नः । अ॒व॒स्थावा॑नो॒ नाम॑ । अ॒व॒स्थावा॑न॒ इत्य॑व - स्थावा॑नः । नाम॑ स्थ । स्थ॒ तेषा᳚म् । तेषा᳚म् ॅवः । व॒ उ॒त्त॒रात् । उ॒त्त॒राद् गृ॒हाः । उ॒त्त॒रादित्यु॑त् - त॒रात् । गृ॒हा आपः॑ । आपो॑ वः । व॒ इष॑वः । इष॑वः समु॒द्रः । स॒मु॒द्रोऽधि॑पतयः । अधि॑पतयो॒ नाम॑ । अधि॑पतय॒ इत्यधि॑ - प॒त॒यः॒ । नाम॑ स्थ । स्थ॒ तेषा᳚म् । तेषा᳚म् ॅवः । व॒ उ॒परि॑ । उ॒परि॑ गृ॒हाः । गृ॒हा व॒र्॒.षम् । व॒र्॒.षम् ॅवः॑ । व॒ इष॑वः । इष॒वोऽव॑स्वान् । अव॑स्वान् क्र॒व्याः । क्र॒व्या नाम॑ । नाम॑ स्थ । स्थ॒ पार्त्थि॑वाः । पार्त्थि॑वा॒स्तेषा᳚म् । तेषा᳚म् ॅवः । व॒ इ॒ह । इ॒ह गृ॒हाः । गृ॒हा अन्न᳚म् \newline

\textbf{Jatai Paata} \newline

1. स्थ॒ तेषा॒म् तेषाꣳ॑ स्थ स्थ॒ तेषा᳚म् । \newline
2. तेषां᳚ ॅवो व॒ स्तेषा॒म् तेषां᳚ ॅवः । \newline
3. वो॒ द॒क्षि॒णा द॑क्षि॒णा वो॑ वो दक्षि॒णा । \newline
4. द॒क्षि॒णा गृ॒हा गृ॒हा द॑क्षि॒णा द॑क्षि॒णा गृ॒हाः । \newline
5. गृ॒हाः पि॒तरः॑ पि॒तरो॑ गृ॒हा गृ॒हाः पि॒तरः॑ । \newline
6. पि॒तरो॑ वो वः पि॒तरः॑ पि॒तरो॑ वः । \newline
7. व॒ इष॑व॒ इष॑वो वो व॒ इष॑वः । \newline
8. इष॑वः॒ सग॑रः॒ सग॑र॒ इष॑व॒ इष॑वः॒ सग॑रः । \newline
9. सग॑रो व॒ज्रिणो॑ व॒ज्रिणः॒ सग॑रः॒ सग॑रो व॒ज्रिणः॑ । \newline
10. व॒ज्रिणो॒ नाम॒ नाम॑ व॒ज्रिणो॑ व॒ज्रिणो॒ नाम॑ । \newline
11. नाम॑ स्थ स्थ॒ नाम॒ नाम॑ स्थ । \newline
12. स्थ॒ तेषा॒म् तेषाꣳ॑ स्थ स्थ॒ तेषा᳚म् । \newline
13. तेषां᳚ ॅवो व॒ स्तेषा॒म् तेषां᳚ ॅवः । \newline
14. वः॒ प॒श्चात् प॒श्चाद् वो॑ वः प॒श्चात् । \newline
15. प॒श्चाद् गृ॒हा गृ॒हाः प॒श्चात् प॒श्चाद् गृ॒हाः । \newline
16. गृ॒हाः स्वप्नः॒ स्वप्नो॑ गृ॒हा गृ॒हाः स्वप्नः॑ । \newline
17. स्वप्नो॑ वो वः॒ स्वप्नः॒ स्वप्नो॑ वः । \newline
18. व॒ इष॑व॒ इष॑वो वो व॒ इष॑वः । \newline
19. इष॑वो॒ गह्व॑रो॒ गह्व॑र॒ इष॑व॒ इष॑वो॒ गह्व॑रः । \newline
20. गह्व॑रो ऽव॒स्थावा॑नो ऽव॒स्थावा॑नो॒ गह्व॑रो॒ गह्व॑रो ऽव॒स्थावा॑नः । \newline
21. अ॒व॒स्थावा॑नो॒ नाम॒ नामा॑ व॒स्थावा॑नो ऽव॒स्थावा॑नो॒ नाम॑ । \newline
22. अ॒व॒स्थावा॑न॒ इत्य॑व - स्थावा॑नः । \newline
23. नाम॑ स्थ स्थ॒ नाम॒ नाम॑ स्थ । \newline
24. स्थ॒ तेषा॒म् तेषाꣳ॑ स्थ स्थ॒ तेषा᳚म् । \newline
25. तेषां᳚ ॅवो व॒ स्तेषा॒म् तेषां᳚ ॅवः । \newline
26. व॒ उ॒त्त॒रा दु॑त्त॒राद् वो॑ व उत्त॒रात् । \newline
27. उ॒त्त॒राद् गृ॒हा गृ॒हा उ॑त्त॒रा दु॑त्त॒राद् गृ॒हाः । \newline
28. उ॒त्त॒रादित्यु॑त् - त॒रात् । \newline
29. गृ॒हा आप॒ आपो॑ गृ॒हा गृ॒हा आपः॑ । \newline
30. आपो॑ वो व॒ आप॒ आपो॑ वः । \newline
31. व॒ इष॑व॒ इष॑वो वो व॒ इष॑वः । \newline
32. इष॑वः समु॒द्रः स॑मु॒द्र इष॑व॒ इष॑वः समु॒द्रः । \newline
33. स॒मु॒द्रो ऽधि॑पत॒यो ऽधि॑पतयः समु॒द्रः स॑मु॒द्रो ऽधि॑पतयः । \newline
34. अधि॑पतयो॒ नाम॒ नामा धि॑पत॒यो ऽधि॑पतयो॒ नाम॑ । \newline
35. अधि॑पतय॒ इत्यधि॑ - प॒त॒यः॒ । \newline
36. नाम॑ स्थ स्थ॒ नाम॒ नाम॑ स्थ । \newline
37. स्थ॒ तेषा॒म् तेषाꣳ॑ स्थ स्थ॒ तेषा᳚म् । \newline
38. तेषां᳚ ॅवो व॒ स्तेषा॒म् तेषां᳚ ॅवः । \newline
39. व॒ उ॒पर्यु॒परि॑ वो व उ॒परि॑ । \newline
40. उ॒परि॑ गृ॒हा गृ॒हा उ॒पर्यु॒परि॑ गृ॒हाः । \newline
41. गृ॒हा व॒र्॒.षं ॅव॒र्॒.षम् गृ॒हा गृ॒हा व॒र्॒.षम् । \newline
42. व॒र्॒.षं ॅवो॑ वो व॒र्॒.षं ॅव॒र्॒.षं ॅवः॑ । \newline
43. व॒ इष॑व॒ इष॑वो वो व॒ इष॑वः । \newline
44. इष॒वो ऽव॑स्वा॒ नव॑स्वा॒ निष॑व॒ इष॒वो ऽव॑स्वान् । \newline
45. अव॑स्वान् क्र॒व्याः क्र॒व्या अव॑स्वा॒ नव॑स्वान् क्र॒व्याः । \newline
46. क्र॒व्या नाम॒ नाम॑ क्र॒व्याः क्र॒व्या नाम॑ । \newline
47. नाम॑ स्थ स्थ॒ नाम॒ नाम॑ स्थ । \newline
48. स्थ॒ पार्थि॑वाः॒ पार्थि॑वाः स्थ स्थ॒ पार्थि॑वाः । \newline
49. पार्थि॑वा॒ स्तेषा॒म् तेषा॒म् पार्थि॑वाः॒ पार्थि॑वा॒ स्तेषा᳚म् । \newline
50. तेषां᳚ ॅवो व॒ स्तेषा॒म् तेषां᳚ ॅवः । \newline
51. व॒ इ॒हेह वो॑ व इ॒ह । \newline
52. इ॒ह गृ॒हा गृ॒हा इ॒हेह गृ॒हाः । \newline
53. गृ॒हा अन्न॒ मन्न॑म् गृ॒हा गृ॒हा अन्न᳚म् । \newline

\textbf{Ghana Paata } \newline

1. स्थ॒ तेषा॒म् तेषाꣳ॑ स्थ स्थ॒ तेषां᳚ ॅवो व॒ स्तेषाꣳ॑ स्थ स्थ॒ तेषां᳚ ॅवः । \newline
2. तेषां᳚ ॅवो व॒ स्तेषा॒म् तेषां᳚ ॅवो दक्षि॒णा द॑क्षि॒णा व॒ स्तेषा॒म् तेषां᳚ ॅवो दक्षि॒णा । \newline
3. वो॒ द॒क्षि॒णा द॑क्षि॒णा वो॑ वो दक्षि॒णा गृ॒हा गृ॒हा द॑क्षि॒णा वो॑ वो दक्षि॒णा गृ॒हाः । \newline
4. द॒क्षि॒णा गृ॒हा गृ॒हा द॑क्षि॒णा द॑क्षि॒णा गृ॒हाः पि॒तरः॑ पि॒तरो॑ गृ॒हा द॑क्षि॒णा द॑क्षि॒णा गृ॒हाः पि॒तरः॑ । \newline
5. गृ॒हाः पि॒तरः॑ पि॒तरो॑ गृ॒हा गृ॒हाः पि॒तरो॑ वो वः पि॒तरो॑ गृ॒हा गृ॒हाः पि॒तरो॑ वः । \newline
6. पि॒तरो॑ वो वः पि॒तरः॑ पि॒तरो॑ व॒ इष॑व॒ इष॑वो वः पि॒तरः॑ पि॒तरो॑ व॒ इष॑वः । \newline
7. व॒ इष॑व॒ इष॑वो वो व॒ इष॑वः॒ सग॑रः॒ सग॑र॒ इष॑वो वो व॒ इष॑वः॒ सग॑रः । \newline
8. इष॑वः॒ सग॑रः॒ सग॑र॒ इष॑व॒ इष॑वः॒ सग॑रो व॒ज्रिणो॑ व॒ज्रिणः॒ सग॑र॒ इष॑व॒ इष॑वः॒ सग॑रो व॒ज्रिणः॑ । \newline
9. सग॑रो व॒ज्रिणो॑ व॒ज्रिणः॒ सग॑रः॒ सग॑रो व॒ज्रिणो॒ नाम॒ नाम॑ व॒ज्रिणः॒ सग॑रः॒ सग॑रो व॒ज्रिणो॒ नाम॑ । \newline
10. व॒ज्रिणो॒ नाम॒ नाम॑ व॒ज्रिणो॑ व॒ज्रिणो॒ नाम॑ स्थ स्थ॒ नाम॑ व॒ज्रिणो॑ व॒ज्रिणो॒ नाम॑ स्थ । \newline
11. नाम॑ स्थ स्थ॒ नाम॒ नाम॑ स्थ॒ तेषा॒म् तेषाꣳ॑ स्थ॒ नाम॒ नाम॑ स्थ॒ तेषा᳚म् । \newline
12. स्थ॒ तेषा॒म् तेषाꣳ॑ स्थ स्थ॒ तेषां᳚ ॅवो व॒ स्तेषाꣳ॑ स्थ स्थ॒ तेषां᳚ ॅवः । \newline
13. तेषां᳚ ॅवो व॒ स्तेषा॒म् तेषां᳚ ॅवः प॒श्चात् प॒श्चाद् व॒ स्तेषा॒म् तेषां᳚ ॅवः प॒श्चात् । \newline
14. वः॒ प॒श्चात् प॒श्चाद् वो॑ वः प॒श्चाद् गृ॒हा गृ॒हाः प॒श्चाद् वो॑ वः प॒श्चाद् गृ॒हाः । \newline
15. प॒श्चाद् गृ॒हा गृ॒हाः प॒श्चात् प॒श्चाद् गृ॒हाः स्वप्नः॒ स्वप्नो॑ गृ॒हाः प॒श्चात् प॒श्चाद् गृ॒हाः स्वप्नः॑ । \newline
16. गृ॒हाः स्वप्नः॒ स्वप्नो॑ गृ॒हा गृ॒हाः स्वप्नो॑ वो वः॒ स्वप्नो॑ गृ॒हा गृ॒हाः स्वप्नो॑ वः । \newline
17. स्वप्नो॑ वो वः॒ स्वप्नः॒ स्वप्नो॑ व॒ इष॑व॒ इष॑वो वः॒ स्वप्नः॒ स्वप्नो॑ व॒ इष॑वः । \newline
18. व॒ इष॑व॒ इष॑वो वो व॒ इष॑वो॒ गह्व॑रो॒ गह्व॑र॒ इष॑वो वो व॒ इष॑वो॒ गह्व॑रः । \newline
19. इष॑वो॒ गह्व॑रो॒ गह्व॑र॒ इष॑व॒ इष॑वो॒ गह्व॑रो ऽव॒स्थावा॑नो ऽव॒स्थावा॑नो॒ गह्व॑र॒ इष॑व॒ इष॑वो॒ गह्व॑रो ऽव॒स्थावा॑नः । \newline
20. गह्व॑रो ऽव॒स्थावा॑नो ऽव॒स्थावा॑नो॒ गह्व॑रो॒ गह्व॑रो ऽव॒स्थावा॑नो॒ नाम॒ नामा॑ व॒स्थावा॑नो॒ गह्व॑रो॒ गह्व॑रो ऽव॒स्थावा॑नो॒ नाम॑ । \newline
21. अ॒व॒स्थावा॑नो॒ नाम॒ नामा॑ व॒स्थावा॑नो ऽव॒स्थावा॑नो॒ नाम॑ स्थ स्थ॒ नामा॑ व॒स्थावा॑नो ऽव॒स्थावा॑नो॒ नाम॑ स्थ । \newline
22. अ॒व॒स्थावा॑न॒ इत्य॑व - स्थावा॑नः । \newline
23. नाम॑ स्थ स्थ॒ नाम॒ नाम॑ स्थ॒ तेषा॒म् तेषाꣳ॑ स्थ॒ नाम॒ नाम॑ स्थ॒ तेषा᳚म् । \newline
24. स्थ॒ तेषा॒म् तेषाꣳ॑ स्थ स्थ॒ तेषां᳚ ॅवो व॒ स्तेषाꣳ॑ स्थ स्थ॒ तेषां᳚ ॅवः । \newline
25. तेषां᳚ ॅवो व॒ स्तेषा॒म् तेषां᳚ ॅव उत्त॒रा दु॑त्त॒राद् व॒ स्तेषा॒म् तेषां᳚ ॅव उत्त॒रात् । \newline
26. व॒ उ॒त्त॒रा दु॑त्त॒राद् वो॑ व उत्त॒राद् गृ॒हा गृ॒हा उ॑त्त॒राद् वो॑ व उत्त॒राद् गृ॒हाः । \newline
27. उ॒त्त॒राद् गृ॒हा गृ॒हा उ॑त्त॒रा दु॑त्त॒राद् गृ॒हा आप॒ आपो॑ गृ॒हा उ॑त्त॒रा दु॑त्त॒राद् गृ॒हा आपः॑ । \newline
28. उ॒त्त॒रादित्यु॑त् - त॒रात् । \newline
29. गृ॒हा आप॒ आपो॑ गृ॒हा गृ॒हा आपो॑ वो व॒ आपो॑ गृ॒हा गृ॒हा आपो॑ वः । \newline
30. आपो॑ वो व॒ आप॒ आपो॑ व॒ इष॑व॒ इष॑वो व॒ आप॒ आपो॑ व॒ इष॑वः । \newline
31. व॒ इष॑व॒ इष॑वो वो व॒ इष॑वः समु॒द्रः स॑मु॒द्र इष॑वो वो व॒ इष॑वः समु॒द्रः । \newline
32. इष॑वः समु॒द्रः स॑मु॒द्र इष॑व॒ इष॑वः समु॒द्रो ऽधि॑पत॒यो ऽधि॑पतयः समु॒द्र इष॑व॒ इष॑वः समु॒द्रो ऽधि॑पतयः । \newline
33. स॒मु॒द्रो ऽधि॑पत॒यो ऽधि॑पतयः समु॒द्रः स॑मु॒द्रो ऽधि॑पतयो॒ नाम॒ नामा धि॑पतयः समु॒द्रः स॑मु॒द्रो ऽधि॑पतयो॒ नाम॑ । \newline
34. अधि॑पतयो॒ नाम॒ नामा धि॑पत॒यो ऽधि॑पतयो॒ नाम॑ स्थ स्थ॒ नामा धि॑पत॒यो ऽधि॑पतयो॒ नाम॑ स्थ । \newline
35. अधि॑पतय॒ इत्यधि॑ - प॒त॒यः॒ । \newline
36. नाम॑ स्थ स्थ॒ नाम॒ नाम॑ स्थ॒ तेषा॒म् तेषाꣳ॑ स्थ॒ नाम॒ नाम॑ स्थ॒ तेषा᳚म् । \newline
37. स्थ॒ तेषा॒म् तेषाꣳ॑ स्थ स्थ॒ तेषां᳚ ॅवो व॒ स्तेषाꣳ॑ स्थ स्थ॒ तेषां᳚ ॅवः । \newline
38. तेषां᳚ ॅवो व॒ स्तेषा॒म् तेषां᳚ ॅव उ॒पर्यु॒परि॑ व॒ स्तेषा॒म् तेषां᳚ ॅव उ॒परि॑ । \newline
39. व॒ उ॒पर्यु॒परि॑ वो व उ॒परि॑ गृ॒हा गृ॒हा उ॒परि॑ वो व उ॒परि॑ गृ॒हाः । \newline
40. उ॒परि॑ गृ॒हा गृ॒हा उ॒पर्यु॒परि॑ गृ॒हा व॒र्॒.षं ॅव॒र्॒.षम् गृ॒हा उ॒पर्यु॒परि॑ गृ॒हा व॒र्॒.षम् । \newline
41. गृ॒हा व॒र्॒.षं ॅव॒र्॒.षम् गृ॒हा गृ॒हा व॒र्॒.षं ॅवो॑ वो व॒र्॒.षम् गृ॒हा गृ॒हा व॒र्॒.षं ॅवः॑ । \newline
42. व॒र्॒.षं ॅवो॑ वो व॒र्॒.षं ॅव॒र्॒.षं ॅव॒ इष॑व॒ इष॑वो वो व॒र्॒.षं ॅव॒र्॒.षं ॅव॒ इष॑वः । \newline
43. व॒ इष॑व॒ इष॑वो वो व॒ इष॒वो ऽव॑स्वा॒ नव॑स्वा॒ निष॑वो वो व॒ इष॒वो ऽव॑स्वान् । \newline
44. इष॒वो ऽव॑स्वा॒ नव॑स्वा॒ निष॑व॒ इष॒वो ऽव॑स्वान् क्र॒व्याः क्र॒व्या अव॑स्वा॒ निष॑व॒ इष॒वो ऽव॑स्वान् क्र॒व्याः । \newline
45. अव॑स्वान् क्र॒व्याः क्र॒व्या अव॑स्वा॒ नव॑स्वान् क्र॒व्या नाम॒ नाम॑ क्र॒व्या अव॑स्वा॒ नव॑स्वान् क्र॒व्या नाम॑ । \newline
46. क्र॒व्या नाम॒ नाम॑ क्र॒व्याः क्र॒व्या नाम॑ स्थ स्थ॒ नाम॑ क्र॒व्याः क्र॒व्या नाम॑ स्थ । \newline
47. नाम॑ स्थ स्थ॒ नाम॒ नाम॑ स्थ॒ पार्थि॑वाः॒ पार्थि॑वाः स्थ॒ नाम॒ नाम॑ स्थ॒ पार्थि॑वाः । \newline
48. स्थ॒ पार्थि॑वाः॒ पार्थि॑वाः स्थ स्थ॒ पार्थि॑वा॒ स्तेषा॒म् तेषा॒म् पार्थि॑वाः स्थ स्थ॒ पार्थि॑वा॒ स्तेषा᳚म् । \newline
49. पार्थि॑वा॒ स्तेषा॒म् तेषा॒म् पार्थि॑वाः॒ पार्थि॑वा॒ स्तेषां᳚ ॅवो व॒ स्तेषा॒म् पार्थि॑वाः॒ पार्थि॑वा॒ स्तेषां᳚ ॅवः । \newline
50. तेषां᳚ ॅवो व॒ स्तेषा॒म् तेषां᳚ ॅव इ॒हेह व॒ स्तेषा॒म् तेषां᳚ ॅव इ॒ह । \newline
51. व॒ इ॒हेह वो॑ व इ॒ह गृ॒हा गृ॒हा इ॒ह वो॑ व इ॒ह गृ॒हाः । \newline
52. इ॒ह गृ॒हा गृ॒हा इ॒हेह गृ॒हा अन्न॒ मन्न॑म् गृ॒हा इ॒हेह गृ॒हा अन्न᳚म् । \newline
53. गृ॒हा अन्न॒ मन्न॑म् गृ॒हा गृ॒हा अन्नं॑ ॅवो॒ वो ऽन्न॑म् गृ॒हा गृ॒हा अन्नं॑ ॅवः । \newline
\pagebreak
\markright{ TS 5.5.10.5  \hfill https://www.vedavms.in \hfill}

\section{ TS 5.5.10.5 }

\textbf{TS 5.5.10.5 } \newline
\textbf{Samhita Paata} \newline

अन्नं॑ ॅव॒ इष॑वो ऽनिमि॒षो वा॑तना॒मं तेभ्यो॑ वो॒ नम॒स्ते नो॑ मृडयत॒ ते यं द्वि॒ष्मो यश्च॑ नो॒ द्वेष्टि॒ तं ॅवो॒ जंभे॑ दधामि हु॒तादो॒ वा अ॒न्ये दे॒वा अ॑हु॒तादो॒ऽन्ये तान॑ग्नि॒चिदे॒वोभया᳚न् प्रीणाति द॒द्ध्ना म॑धुमि॒श्रेणै॒ता आहु॑तीर्जुहोति भाग॒धेये॑नै॒वैना᳚न् प्रीणा॒त्यथो॒ खल्वा॑हु॒रिष्ट॑का॒ वै दे॒वा अ॑हु॒ताद॒ इत्य॑ - [  ] \newline

\textbf{Pada Paata} \newline

अन्न᳚म् । वः॒ । इष॑वः । नि॒मि॒ष इति॑ नि - मि॒षः । वा॒त॒ना॒ममिति॑ वात - ना॒मम् । तेभ्यः॑ । वः॒ । नमः॑ । ते । नः॒ । मृ॒ड॒य॒त॒ । ते । यम् । द्वि॒ष्मः । यः । च॒ । नः॒ । द्वेष्टि॑ । तम् । वः॒ । जम्भे᳚ । द॒धा॒मि॒ । हु॒ताद॒ इति॑ हुत - अदः॑ । वै । अ॒न्ये । दे॒वाः । अ॒हु॒ताद॒ इत्य॑हुत - अदः॑ । अ॒न्ये । तान् । अ॒ग्नि॒चिदित्य॑ग्नि - चित् । ए॒व । उ॒भयान्॑ । प्री॒णा॒ति॒ । द॒द्ध्ना । म॒धु॒मि॒श्रेणेति॑ मधु - मि॒श्रेण॑ । ए॒ताः । आहु॑ती॒रित्या-हु॒तीः॒ । जु॒हो॒ति॒ । भा॒ग॒धेये॒नेति॑ भाग - धेये॑न । ए॒व । ए॒ना॒न् । प्री॒णा॒ति॒ । अथो॒ इति॑ । खलु॑ । आ॒हुः॒ । इष्ट॑काः । वै । दे॒वाः । अ॒हु॒ताद॒ इत्य॑हुत - अदः॑ । इति॑ ।  \newline


\textbf{Krama Paata} \newline

अन्न॑म् ॅवः । व॒ इष॑वः । इष॑वो निमि॒षः । नि॒मि॒षो वा॑तना॒मम् । नि॒मि॒ष इति॑ नि - मि॒षः । वा॒त॒ना॒मम् तेभ्यः॑ । वा॒त॒ना॒ममिति॑ वात - ना॒मम् । तेभ्यो॑ वः । वो॒ नमः॑ । नम॒स्ते । ते नः॑ । नो॒ मृ॒ड॒य॒त॒ । मृ॒ड॒य॒त॒ ते । ते यम् । यम् द्वि॒ष्मः । द्वि॒ष्मो यः । यश्च॑ । च॒ नः॒ । नो॒ द्वेष्टि॑ । द्वेष्टि॒ तम् । तम् ॅवः॑ । वो॒ जम्भे᳚ । जम्भे॑ दधामि । द॒धा॒मि॒ हु॒तादः॑ । हु॒तादो॒ वै । हु॒ताद॒ इति॑ हुत - अदः॑ । वा अ॒न्ये । अ॒न्ये दे॒वाः । दे॒वा अ॑हु॒तादः॑ । अ॒हु॒तादो॒ऽन्ये । अ॒हु॒ताद॒ इत्य॑हुत - अदः॑ । अ॒न्ये तान् । तान॑ग्नि॒चित् । अ॒ग्नि॒चिदे॒व । अ॒ग्नि॒चिदित्य॑ग्नि - चित् । ए॒वोभयान्॑ । उ॒भया᳚न् प्रीणाति । प्री॒णा॒ति॒ द॒द्ध्ना । द॒द्ध्ना म॑धुमि॒श्रेण॑ । म॒धु॒मि॒श्रेणै॒ताः । म॒धु॒मि॒श्रेणेति॑ मधु - मि॒श्रेण॑ । ए॒ता आहु॑तीः । आहु॑तीर् जुहोति । आहु॑ती॒रित्या - हु॒तीः॒ । जु॒हो॒ति॒ भा॒ग॒धेये॑न । भा॒ग॒धेये॑नै॒व । भा॒ग॒धेये॒नेति॑ भाग - धेये॑न । ए॒वैनान्॑ । ए॒ना॒न् प्री॒णा॒ति॒ । प्री॒णा॒त्यथो᳚ । अथो॒ खलु॑ । अथो॒ इत्यथो᳚ । खल्वा॑हुः । आ॒हु॒रिष्ट॑काः । इष्ट॑का॒ वै । वै दे॒वाः । दे॒वा अ॑हु॒तादः॑ । अ॒हु॒ताद॒ इति॑ । अ॒हु॒ताद॒ इत्य॑हुत - अदः॑ । इत्य॑नुपरि॒क्राम᳚म् \newline

\textbf{Jatai Paata} \newline

1. अन्नं॑ ॅवो॒ वो ऽन्न॒ मन्नं॑ ॅवः । \newline
2. व॒ इष॑व॒ इष॑वो वो व॒ इष॑वः । \newline
3. इष॑वो निमि॒षो नि॑मि॒ष इष॑व॒ इष॑वो निमि॒षः । \newline
4. नि॒मि॒षो वा॑तना॒मं ॅवा॑तना॒मम् नि॑मि॒षो नि॑मि॒षो वा॑तना॒मम् । \newline
5. नि॒मि॒ष इति॑ नि - मि॒षः । \newline
6. वा॒त॒ना॒मम् तेभ्य॒ स्तेभ्यो॑ वातना॒मं ॅवा॑तना॒मम् तेभ्यः॑ । \newline
7. वा॒त॒ना॒ममिति॑ वात - ना॒मम् । \newline
8. तेभ्यो॑ वो व॒ स्तेभ्य॒ स्तेभ्यो॑ वः । \newline
9. वो॒ नमो॒ नमो॑ वो वो॒ नमः॑ । \newline
10. नम॒ स्ते ते नमो॒ नम॒ स्ते । \newline
11. ते नो॑ न॒ स्ते ते नः॑ । \newline
12. नो॒ मृ॒ड॒य॒त॒ मृ॒ड॒य॒त॒ नो॒ नो॒ मृ॒ड॒य॒त॒ । \newline
13. मृ॒ड॒य॒त॒ ते ते मृ॑डयत मृडयत॒ ते । \newline
14. ते यं ॅयम् ते ते यम् । \newline
15. यम् द्वि॒ष्मो द्वि॒ष्मो यं ॅयम् द्वि॒ष्मः । \newline
16. द्वि॒ष्मो यो यो द्वि॒ष्मो द्वि॒ष्मो यः । \newline
17. यश्च॑ च॒ यो यश्च॑ । \newline
18. च॒ नो॒ न॒श्च॒ च॒ नः॒ । \newline
19. नो॒ द्वेष्टि॒ द्वेष्टि॑ नो नो॒ द्वेष्टि॑ । \newline
20. द्वेष्टि॒ तम् तम् द्वेष्टि॒ द्वेष्टि॒ तम् । \newline
21. तं ॅवो॑ व॒ स्तम् तं ॅवः॑ । \newline
22. वो॒ जम्भे॒ जम्भे॑ वो वो॒ जम्भे᳚ । \newline
23. जम्भे॑ दधामि दधामि॒ जम्भे॒ जम्भे॑ दधामि । \newline
24. द॒धा॒मि॒ हु॒तादो॑ हु॒तादो॑ दधामि दधामि हु॒तादः॑ । \newline
25. हु॒तादो॒ वै वै हु॒तादो॑ हु॒तादो॒ वै । \newline
26. हु॒ताद॒ इति॑ हुत - अदः॑ । \newline
27. वा अ॒न्ये᳚ ऽन्ये वै वा अ॒न्ये । \newline
28. अ॒न्ये दे॒वा दे॒वा अ॒न्ये᳚ ऽन्ये दे॒वाः । \newline
29. दे॒वा अ॑हु॒तादो॑ ऽहु॒तादो॑ दे॒वा दे॒वा अ॑हु॒तादः॑ । \newline
30. अ॒हु॒तादो॒ ऽन्ये᳚(1॒) ऽन्ये॑ ऽहु॒तादो॑ ऽहु॒तादो॒ ऽन्ये । \newline
31. अ॒हु॒ताद॒ इत्य॑हुत - अदः॑ । \newline
32. अ॒न्ये ताꣳ स्तान॒ न्ये᳚ ऽन्ये तान् । \newline
33. तान॑ग्नि॒चि द॑ग्नि॒चित् ताꣳ स्तान॑ग्नि॒चित् । \newline
34. अ॒ग्नि॒चि दे॒वै वाग्नि॒चि द॑ग्नि॒चि दे॒व । \newline
35. अ॒ग्नि॒चिदित्य॑ग्नि - चित् । \newline
36. ए॒वोभया॑ नु॒भया॑ ने॒वै वोभयान्॑ । \newline
37. उ॒भया᳚न् प्रीणाति प्रीणा त्यु॒भया॑ नु॒भया᳚न् प्रीणाति । \newline
38. प्री॒णा॒ति॒ द॒द्ध्ना द॒द्ध्ना प्री॑णाति प्रीणाति द॒द्ध्ना । \newline
39. द॒द्ध्ना म॑धुमि॒श्रेण॑ मधुमि॒श्रेण॑ द॒द्ध्ना द॒द्ध्ना म॑धुमि॒श्रेण॑ । \newline
40. म॒धु॒मि॒श्रे णै॒ता ए॒ता म॑धुमि॒श्रेण॑ मधुमि॒श्रे णै॒ताः । \newline
41. म॒धु॒मि॒श्रेणेति॑ मधु - मि॒श्रेण॑ । \newline
42. ए॒ता आहु॑ती॒ राहु॑ती रे॒ता ए॒ता आहु॑तीः । \newline
43. आहु॑तीर् जुहोति जुहो॒ त्याहु॑ती॒ राहु॑तीर् जुहोति । \newline
44. आहु॑ती॒रित्या - हु॒तीः॒ । \newline
45. जु॒हो॒ति॒ भा॒ग॒धेये॑न भाग॒धेये॑न जुहोति जुहोति भाग॒धेये॑न । \newline
46. भा॒ग॒धेये॑ नै॒वैव भा॑ग॒धेये॑न भाग॒धेये॑नै॒व । \newline
47. भा॒ग॒धेये॒नेति॑ भाग - धेये॑न । \newline
48. ए॒वैना॑ नेना ने॒वै वैनान्॑ । \newline
49. ए॒ना॒न् प्री॒णा॒ति॒ प्री॒णा॒ त्ये॒ना॒ ने॒ना॒न् प्री॒णा॒ति॒ । \newline
50. प्री॒णा॒ त्यथो॒ अथो᳚ प्रीणाति प्रीणा॒ त्यथो᳚ । \newline
51. अथो॒ खलु॒ खल्वथो॒ अथो॒ खलु॑ । \newline
52. अथो॒ इत्यथो᳚ । \newline
53. खल्वा॑हु राहुः॒ खलु॒ खल्वा॑हुः । \newline
54. आ॒हु॒ रिष्ट॑का॒ इष्ट॑का आहु राहु॒ रिष्ट॑काः । \newline
55. इष्ट॑का॒ वै वा इष्ट॑का॒ इष्ट॑का॒ वै । \newline
56. वै दे॒वा दे॒वा वै वै दे॒वाः । \newline
57. दे॒वा अ॑हु॒तादो॑ ऽहु॒तादो॑ दे॒वा दे॒वा अ॑हु॒तादः॑ । \newline
58. अ॒हु॒ताद॒ इती त्य॑हु॒तादो॑ ऽहु॒ताद॒ इति॑ । \newline
59. अ॒हु॒ताद॒ इत्य॑हुत - अदः॑ । \newline
60. इत्य॑नुपरि॒क्राम॑ मनुपरि॒क्राम॒ मिती त्य॑नुपरि॒क्राम᳚म् । \newline

\textbf{Ghana Paata } \newline

1. अन्नं॑ ॅवो॒ वो ऽन्न॒ मन्नं॑ ॅव॒ इष॑व॒ इष॑वो॒ वो ऽन्न॒ मन्नं॑ ॅव॒ इष॑वः । \newline
2. व॒ इष॑व॒ इष॑वो वो व॒ इष॑वो निमि॒षो नि॑मि॒ष इष॑वो वो व॒ इष॑वो निमि॒षः । \newline
3. इष॑वो निमि॒षो नि॑मि॒ष इष॑व॒ इष॑वो निमि॒षो वा॑तना॒मं ॅवा॑तना॒मम् नि॑मि॒ष इष॑व॒ इष॑वो निमि॒षो वा॑तना॒मम् । \newline
4. नि॒मि॒षो वा॑तना॒मं ॅवा॑तना॒मम् नि॑मि॒षो नि॑मि॒षो वा॑तना॒मम् तेभ्य॒ स्तेभ्यो॑ वातना॒मम् नि॑मि॒षो नि॑मि॒षो वा॑तना॒मम् तेभ्यः॑ । \newline
5. नि॒मि॒ष इति॑ नि - मि॒षः । \newline
6. वा॒त॒ना॒मम् तेभ्य॒ स्तेभ्यो॑ वातना॒मं ॅवा॑तना॒मम् तेभ्यो॑ वो व॒ स्तेभ्यो॑ वातना॒मं ॅवा॑तना॒मम् तेभ्यो॑ वः । \newline
7. वा॒त॒ना॒ममिति॑ वात - ना॒मम् । \newline
8. तेभ्यो॑ वो व॒ स्तेभ्य॒ स्तेभ्यो॑ वो॒ नमो॒ नमो॑ व॒ स्तेभ्य॒ स्तेभ्यो॑ वो॒ नमः॑ । \newline
9. वो॒ नमो॒ नमो॑ वो वो॒ नम॒ स्ते ते नमो॑ वो वो॒ नम॒ स्ते । \newline
10. नम॒ स्ते ते नमो॒ नम॒ स्ते नो॑ न॒ स्ते नमो॒ नम॒ स्ते नः॑ । \newline
11. ते नो॑ न॒ स्ते ते नो॑ मृडयत मृडयत न॒ स्ते ते नो॑ मृडयत । \newline
12. नो॒ मृ॒ड॒य॒त॒ मृ॒ड॒य॒त॒ नो॒ नो॒ मृ॒ड॒य॒त॒ ते ते मृ॑डयत नो नो मृडयत॒ ते । \newline
13. मृ॒ड॒य॒त॒ ते ते मृ॑डयत मृडयत॒ ते यं ॅयम् ते मृ॑डयत मृडयत॒ ते यम् । \newline
14. ते यं ॅयम् ते ते यम् द्वि॒ष्मो द्वि॒ष्मो यम् ते ते यम् द्वि॒ष्मः । \newline
15. यम् द्वि॒ष्मो द्वि॒ष्मो यं ॅयम् द्वि॒ष्मो यो यो द्वि॒ष्मो यं ॅयम् द्वि॒ष्मो यः । \newline
16. द्वि॒ष्मो यो यो द्वि॒ष्मो द्वि॒ष्मो यश्च॑ च॒ यो द्वि॒ष्मो द्वि॒ष्मो यश्च॑ । \newline
17. यश्च॑ च॒ यो यश्च॑ नो नश्च॒ यो यश्च॑ नः । \newline
18. च॒ नो॒ न॒श्च॒ च॒ नो॒ द्वेष्टि॒ द्वेष्टि॑ नश्च च नो॒ द्वेष्टि॑ । \newline
19. नो॒ द्वेष्टि॒ द्वेष्टि॑ नो नो॒ द्वेष्टि॒ तम् तम् द्वेष्टि॑ नो नो॒ द्वेष्टि॒ तम् । \newline
20. द्वेष्टि॒ तम् तम् द्वेष्टि॒ द्वेष्टि॒ तं ॅवो॑ व॒ स्तम् द्वेष्टि॒ द्वेष्टि॒ तं ॅवः॑ । \newline
21. तं ॅवो॑ व॒ स्तम् तं ॅवो॒ जम्भे॒ जम्भे॑ व॒ स्तम् तं ॅवो॒ जम्भे᳚ । \newline
22. वो॒ जम्भे॒ जम्भे॑ वो वो॒ जम्भे॑ दधामि दधामि॒ जम्भे॑ वो वो॒ जम्भे॑ दधामि । \newline
23. जम्भे॑ दधामि दधामि॒ जम्भे॒ जम्भे॑ दधामि हु॒तादो॑ हु॒तादो॑ दधामि॒ जम्भे॒ जम्भे॑ दधामि हु॒तादः॑ । \newline
24. द॒धा॒मि॒ हु॒तादो॑ हु॒तादो॑ दधामि दधामि हु॒तादो॒ वै वै हु॒तादो॑ दधामि दधामि हु॒तादो॒ वै । \newline
25. हु॒तादो॒ वै वै हु॒तादो॑ हु॒तादो॒ वा अ॒न्ये᳚ ऽन्ये वै हु॒तादो॑ हु॒तादो॒ वा अ॒न्ये । \newline
26. हु॒ताद॒ इति॑ हुत - अदः॑ । \newline
27. वा अ॒न्ये᳚ ऽन्ये वै वा अ॒न्ये दे॒वा दे॒वा अ॒न्ये वै वा अ॒न्ये दे॒वाः । \newline
28. अ॒न्ये दे॒वा दे॒वा अ॒न्ये᳚ ऽन्ये दे॒वा अ॑हु॒तादो॑ ऽहु॒तादो॑ दे॒वा अ॒न्ये᳚ ऽन्ये दे॒वा अ॑हु॒तादः॑ । \newline
29. दे॒वा अ॑हु॒तादो॑ ऽहु॒तादो॑ दे॒वा दे॒वा अ॑हु॒तादो॒ ऽन्ये᳚(1॒) ऽन्ये॑ ऽहु॒तादो॑ दे॒वा दे॒वा अ॑हु॒तादो॒ ऽन्ये । \newline
30. अ॒हु॒तादो॒ ऽन्ये᳚(1॒) ऽन्ये॑ ऽहु॒तादो॑ ऽहु॒तादो॒ ऽन्ये ताꣳ स्ता न॒न्ये॑ ऽहु॒तादो॑ ऽहु॒तादो॒ ऽन्ये तान् । \newline
31. अ॒हु॒ताद॒ इत्य॑हुत - अदः॑ । \newline
32. अ॒न्ये ताꣳ स्ता न॒न्ये᳚ ऽन्ये तान॑ग्नि॒चि द॑ग्नि॒चित् ता न॒न्ये᳚ ऽन्ये तान॑ग्नि॒चित् । \newline
33. तान॑ग्नि॒चि द॑ग्नि॒चित् ताꣳ स्तान॑ग्नि॒चि दे॒वै वाग्नि॒चित् ताꣳ स्ता न॑ग्नि॒चि दे॒व । \newline
34. आ॒ग्नि॒चि दे॒वैवाग्नि॒चि द॑ग्नि॒चि दे॒वोभया॑ नु॒भया॑ ने॒वाग्नि॒चि द॑ग्नि॒चि दे॒वोभयान्॑ । \newline
35. अ॒ग्नि॒चिदित्य॑ग्नि - चित् । \newline
36. ए॒वोभया॑ नु॒भया॑ ने॒वै वोभया᳚न् प्रीणाति प्रीणा त्यु॒भया॑ ने॒वै वोभया᳚न् प्रीणाति । \newline
37. उ॒भया᳚न् प्रीणाति प्रीणा त्यु॒भया॑ नु॒भया᳚न् प्रीणाति द॒द्ध्ना द॒द्ध्ना प्री॑णा त्यु॒भया॑ नु॒भया᳚न् प्रीणाति द॒द्ध्ना । \newline
38. प्री॒णा॒ति॒ द॒द्ध्ना द॒द्ध्ना प्री॑णाति प्रीणाति द॒द्ध्ना म॑धुमि॒श्रेण॑ मधुमि॒श्रेण॑ द॒द्ध्ना प्री॑णाति प्रीणाति द॒द्ध्ना म॑धुमि॒श्रेण॑ । \newline
39. द॒द्ध्ना म॑धुमि॒श्रेण॑ मधुमि॒श्रेण॑ द॒द्ध्ना द॒द्ध्ना म॑धुमि॒श्रे णै॒ता ए॒ता म॑धुमि॒श्रेण॑ द॒द्ध्ना द॒द्ध्ना म॑धुमि॒श्रेणै॒ताः । \newline
40. म॒धु॒मि॒श्रे णै॒ता ए॒ता म॑धुमि॒श्रेण॑ मधुमि॒श्रेणै॒ता आहु॑ती॒ राहु॑ती रे॒ता म॑धुमि॒श्रेण॑ मधुमि॒श्रे णै॒ता आहु॑तीः । \newline
41. म॒धु॒मि॒श्रेणेति॑ मधु - मि॒श्रेण॑ । \newline
42. ए॒ता आहु॑ती॒ राहु॑ती रे॒ता ए॒ता आहु॑तीर् जुहोति जुहो॒ त्याहु॑ती रे॒ता ए॒ता आहु॑तीर् जुहोति । \newline
43. आहु॑तीर् जुहोति जुहो॒ त्याहु॑ती॒ राहु॑तीर् जुहोति भाग॒धेये॑न भाग॒धेये॑न जुहो॒ त्याहु॑ती॒ राहु॑तीर् जुहोति भाग॒धेये॑न । \newline
44. आहु॑ती॒रित्या - हु॒तीः॒ । \newline
45. जु॒हो॒ति॒ भा॒ग॒धेये॑न भाग॒धेये॑न जुहोति जुहोति भाग॒धेये॑ नै॒वैव भा॑ग॒धेये॑न जुहोति जुहोति भाग॒धेये॑नै॒व । \newline
46. भा॒ग॒धेये॑ नै॒वैव भा॑ग॒धेये॑न भाग॒धेये॑ नै॒वैना॑ नेना ने॒व भा॑ग॒धेये॑न भाग॒धेये॑
नै॒वैनान्॑ । \newline
47. भा॒ग॒धेये॒नेति॑ भाग - धेये॑न । \newline
48. ए॒वैना॑ नेना ने॒वैवैना᳚न् प्रीणाति प्रीणा त्येना ने॒वैवैना᳚न् प्रीणाति । \newline
49. ए॒ना॒न् प्री॒णा॒ति॒ प्री॒णा॒ त्ये॒ना॒ ने॒ना॒न् प्री॒णा॒ त्यथो॒ अथो᳚ प्रीणा त्येना नेनान् प्रीणा॒ त्यथो᳚ । \newline
50. प्री॒णा॒ त्यथो॒ अथो᳚ प्रीणाति प्रीणा॒ त्यथो॒ खलु॒ खल्वथो᳚ प्रीणाति प्रीणा॒ त्यथो॒ खलु॑ । \newline
51. अथो॒ खलु॒ खल्वथो॒ अथो॒ खल्वा॑हु राहुः॒ खल्वथो॒ अथो॒ खल्वा॑हुः । \newline
52. अथो॒ इत्यथो᳚ । \newline
53. खल्वा॑हु राहुः॒ खलु॒ खल्वा॑हु॒ रिष्ट॑का॒ इष्ट॑का आहुः॒ खलु॒ खल्वा॑हु॒ रिष्ट॑काः । \newline
54. आ॒हु॒ रिष्ट॑का॒ इष्ट॑का आहु राहु॒ रिष्ट॑का॒ वै वा इष्ट॑का आहु राहु॒ रिष्ट॑का॒ वै । \newline
55. इष्ट॑का॒ वै वा इष्ट॑का॒ इष्ट॑का॒ वै दे॒वा दे॒वा वा इष्ट॑का॒ इष्ट॑का॒ वै दे॒वाः । \newline
56. वै दे॒वा दे॒वा वै वै दे॒वा अ॑हु॒तादो॑ ऽहु॒तादो॑ दे॒वा वै वै दे॒वा अ॑हु॒तादः॑ । \newline
57. दे॒वा अ॑हु॒तादो॑ ऽहु॒तादो॑ दे॒वा दे॒वा अ॑हु॒ताद॒ इती त्य॑हु॒तादो॑ दे॒वा दे॒वा अ॑हु॒ताद॒ इति॑ । \newline
58. अ॒हु॒ताद॒ इती त्य॑हु॒तादो॑ ऽहु॒ताद॒ इत्य॑नु परि॒क्राम॑ मनुपरि॒क्राम॒ मित्य॑ हु॒तादो॑ ऽहु॒ताद॒ इत्य॑नु परि॒क्राम᳚म् । \newline
59. अ॒हु॒ताद॒ इत्य॑हुत - अदः॑ । \newline
60. इत्य॑नु परि॒क्राम॑ मनुपरि॒क्राम॒ मिती त्य॑नुपरि॒क्राम॑म् जुहोति जुहो त्यनुपरि॒क्राम॒ मिती त्य॑नुपरि॒क्राम॑म् जुहोति । \newline
\pagebreak
\markright{ TS 5.5.10.6  \hfill https://www.vedavms.in \hfill}

\section{ TS 5.5.10.6 }

\textbf{TS 5.5.10.6 } \newline
\textbf{Samhita Paata} \newline

-नुपरि॒क्रामं॑ जुहो॒त्यप॑रिवर्गमे॒वैना᳚न् प्रीणाती॒मꣳ स्तन॒मूर्ज॑स्वन्तं धया॒पां प्रप्या॑तमग्ने सरि॒रस्य॒ मद्ध्ये᳚ । उथ्सं॑ जुषस्व॒ मधु॑मन्तमूर्व समु॒द्रियꣳ॒॒ सद॑न॒मा वि॑शस्व ॥ यो वा अ॒ग्निं प्र॒युज्य॒ न वि॑मु॒ञ्चति॒ यथाऽश्वो॑ यु॒क्तोऽवि॑मुच्यमानः॒ क्षुद्ध्य॑न् परा॒भव॑त्ये॒वम॑स्या॒ग्निः परा॑ भवति॒ तं प॑रा॒भव॑न्तं॒ ॅयज॑मा॒नोऽनु॒ परा॑ भवति॒ सो᳚ऽग्निं चि॒त्वा लू॒क्षो - [  ] \newline

\textbf{Pada Paata} \newline

अ॒नु॒प॒रि॒क्राम॒मित्य॑नु - प॒रि॒क्राम᳚म् । जु॒हो॒ति॒ । अप॑रिवर्ग॒मित्यप॑रि - व॒र्ग॒म् । ए॒व । ए॒ना॒न् । प्री॒णा॒ति॒ । इ॒मम् । स्तन᳚म् । ऊर्ज॑स्वन्तम् । ध॒य॒ । अ॒पाम् । प्रप्या॑त॒मिति॒ प्र-प्या॒त॒म् । अ॒ग्ने॒ । स॒रि॒रस्य॑ । मद्ध्ये᳚ ॥ उथ्स᳚म् । जु॒ष॒स्व॒ । मधु॑मन्त॒मिति॒ मधु॑-म॒न्त॒म् । ऊ॒र्व॒ । स॒मु॒द्रिय᳚म् । सद॑नम् । एति॑ । वि॒श॒स्व॒ ॥ यः । वै । अ॒ग्निम् । प्र॒युज्येति॑ प्र - युज्य॑ । न । वि॒मु॒ञ्चतीति॑ वि - मु॒ञ्चति॑ । यथा᳚ । अश्वः॑ । यु॒क्तः । अवि॑मुच्यमान॒ इत्यवि॑ - मु॒च्य॒मा॒नः॒ । क्षुद्ध्यन्न्॑ । प॒रा॒भव॒तीति॑ परा - भव॑ति । ए॒वम् । अ॒स्य॒ । अ॒ग्निः । परेति॑ । भ॒व॒ति॒ । तम् । प॒रा॒भव॑न्त॒मिति॑ परा - भव॑न्तम् । यज॑मानः । अनु॑ । परेति॑ । भ॒व॒ति॒ । सः । अ॒ग्निम् । चि॒त्वा । लू॒क्षः ।  \newline


\textbf{Krama Paata} \newline

अ॒नु॒प॒रि॒क्राम॑म् जुहोति । अ॒नु॒प॒रि॒क्राम॒मित्य॑नु - प॒रि॒क्राम᳚म् । जु॒हो॒त्यप॑रिवर्गम् । अप॑रिवर्गमे॒व । अप॑रिवर्ग॒मित्यप॑रि - व॒र्ग॒म् । ए॒वैनान्॑ । ए॒ना॒न् प्री॒णा॒ति॒ । प्री॒णा॒ती॒मम् । इ॒मꣳ स्तन᳚म् । स्तन॒मूर्ज॑स्वन्तम् । ऊर्ज॑स्वन्तम् धय । ध॒या॒पाम् । अ॒पाम् प्रप्या॑तम् । प्रप्या॑तमग्ने । प्रप्या॑त॒मिति॒ प्र - प्या॒त॒म् । अ॒ग्ने॒ स॒रि॒रस्य॑ । स॒रि॒रस्य॒ मद्ध्ये᳚ । मद्ध्य॒ इति॒ मद्ध्ये᳚ ॥ उथ्स॑म् जुषस्व । जु॒ष॒स्व॒ मधु॑मन्तम् । मधु॑मन्तमू॒र्व । मधु॑मन्त॒मिति॒ मधु॑ - म॒न्त॒म् । ऊ॒र्व॒ स॒मु॒द्रिय᳚म् । स॒मु॒द्रियꣳ॒॒ सद॑नम् । सद॑न॒मा । आ वि॑शस्व । वि॒श॒स्वेति॑ विशस्व ॥ यो वै । वा अ॒ग्निम् । अ॒ग्निम् प्र॒युज्य॑ । प्र॒युज्य॒ न । प्र॒युज्येति॑ प्र - युज्य॑ । न वि॑मु॒ञ्चति॑ । वि॒मु॒ञ्चति॒ यथा᳚ । वि॒मु॒ञ्चतीति॑ वि - मु॒ञ्चति॑ । यथाऽश्वः॑ । अश्वो॑ यु॒क्तः । यु॒क्तोऽवि॑मुच्यमानः । अवि॑मुच्यमानः॒ क्षुद्ध्यन्न्॑ । अवि॑मुच्यमान॒ इत्यवि॑ - मु॒च्य॒मा॒नः॒ । क्षुद्ध्य॑न् परा॒भव॑ति । प॒रा॒भव॑त्ये॒वम् । प॒रा॒भव॒तीति॑ परा - भव॑ति । ए॒वम॑स्य । अ॒स्या॒ग्निः । अ॒ग्निः परा᳚ । परा॑ भवति । भ॒व॒ति॒ तम् । तम् प॑रा॒भव॑न्तम् । प॒रा॒भव॑न्त॒म् ॅयज॑मानः । प॒रा॒भव॑न्त॒मिति॑ परा - भव॑न्तम् । यज॑मा॒नोऽनु॑ । अनु॒ परा᳚ । परा॑ भवति । भ॒व॒ति॒ सः । सो᳚ऽग्निम् । अ॒ग्निम् चि॒त्वा । चि॒त्वा लू॒क्षः । लू॒क्षो भ॑वति \newline

\textbf{Jatai Paata} \newline

1. अ॒नु॒प॒रि॒क्राम॑म् जुहोति जुहो त्यनुपरि॒क्राम॑ मनुपरि॒क्राम॑म् जुहोति । \newline
2. अ॒नु॒प॒रि॒क्राम॒मित्य॑नु - प॒रि॒क्राम᳚म् । \newline
3. जु॒हो॒ त्यप॑रिवर्ग॒ मप॑रिवर्गम् जुहोति जुहो॒ त्यप॑रिवर्गम् । \newline
4. अप॑रिवर्ग मे॒वैवा प॑रिवर्ग॒ मप॑रिवर्ग मे॒व । \newline
5. अप॑रिवर्ग॒मित्यप॑रि - व॒र्ग॒म् । \newline
6. ए॒वैना॑ नेना ने॒वै वैनान्॑ । \newline
7. ए॒ना॒न् प्री॒णा॒ति॒ प्री॒णा॒ त्ये॒ना॒ ने॒ना॒न् प्री॒णा॒ति॒ । \newline
8. प्री॒णा॒ती॒म मि॒मम् प्री॑णाति प्रीणाती॒मम् । \newline
9. इ॒मꣳ स्तनꣳ॒॒ स्तन॑ मि॒म मि॒मꣳ स्तन᳚म् । \newline
10. स्तन॒ मूर्ज॑स्वन्त॒ मूर्ज॑स्वन्तꣳ॒॒ स्तनꣳ॒॒ स्तन॒ मूर्ज॑स्वन्तम् । \newline
11. ऊर्ज॑स्वन्तम् धय ध॒यो र्ज॑स्वन्त॒ मूर्ज॑स्वन्तम् धय । \newline
12. ध॒या॒पा म॒पाम् ध॑य धया॒पाम् । \newline
13. अ॒पाम् प्रप्या॑त॒म् प्रप्या॑त म॒पा म॒पाम् प्रप्या॑तम् । \newline
14. प्रप्या॑त मग्ने ऽग्ने॒ प्रप्या॑त॒म् प्रप्या॑त मग्ने । \newline
15. प्रप्या॑त॒मिति॒ प्र - प्या॒त॒म् । \newline
16. अ॒ग्ने॒ स॒रि॒रस्य॑ सरि॒रस्या᳚ग्ने ऽग्ने सरि॒रस्य॑ । \newline
17. स॒रि॒रस्य॒ मद्ध्ये॒ मद्ध्ये॑ सरि॒रस्य॑ सरि॒रस्य॒ मद्ध्ये᳚ । \newline
18. मद्ध्य॒ इति॒ मद्ध्ये᳚ । \newline
19. उथ्स॑म् जुषस्व जुष॒स्वोथ्स॒ मुथ्स॑म् जुषस्व । \newline
20. जु॒ष॒स्व॒ मधु॑मन्त॒म् मधु॑मन्तम् जुषस्व जुषस्व॒ मधु॑मन्तम् । \newline
21. मधु॑मन्त मूर्वोर्व॒ मधु॑मन्त॒म् मधु॑मन्त मूर्व । \newline
22. मधु॑मन्त॒मिति॒ मधु॑ - म॒न्त॒म् । \newline
23. ऊ॒र्व॒ स॒मु॒द्रियꣳ॑ समु॒द्रिय॑ मूर्वोर्व समु॒द्रिय᳚म् । \newline
24. स॒मु॒द्रियꣳ॒॒ सद॑नꣳ॒॒ सद॑नꣳ समु॒द्रियꣳ॑ समु॒द्रियꣳ॒॒ सद॑नम् । \newline
25. सद॑न॒ मा सद॑नꣳ॒॒ सद॑न॒ मा । \newline
26. आ वि॑शस्व विश॒स्वा वि॑शस्व । \newline
27. वि॒श॒स्वेति॑ विशस्व । \newline
28. यो वै वै यो यो वै । \newline
29. वा अ॒ग्नि म॒ग्निं ॅवै वा अ॒ग्निम् । \newline
30. अ॒ग्निम् प्र॒युज्य॑ प्र॒युज्या॒ग्नि म॒ग्निम् प्र॒युज्य॑ । \newline
31. प्र॒युज्य॒ न न प्र॒युज्य॑ प्र॒युज्य॒ न । \newline
32. प्र॒युज्येति॑ प्र - युज्य॑ । \newline
33. न वि॑मु॒ञ्चति॑ विमु॒ञ्चति॒ न न वि॑मु॒ञ्चति॑ । \newline
34. वि॒मु॒ञ्चति॒ यथा॒ यथा॑ विमु॒ञ्चति॑ विमु॒ञ्चति॒ यथा᳚ । \newline
35. वि॒मु॒ञ्चतीति॑ वि - मु॒ञ्चति॑ । \newline
36. यथा ऽश्वो ऽश्वो॒ यथा॒ यथा ऽश्वः॑ । \newline
37. अश्वो॑ यु॒क्तो यु॒क्तो ऽश्वो ऽश्वो॑ यु॒क्तः । \newline
38. यु॒क्तो ऽवि॑मुच्यमा॒नो ऽवि॑मुच्यमानो यु॒क्तो यु॒क्तो ऽवि॑मुच्यमानः । \newline
39. अवि॑मुच्यमानः॒ क्षुद्ध्य॒न् क्षुद्ध्य॒न् नवि॑मुच्यमा॒नो ऽवि॑मुच्यमानः॒ क्षुद्ध्यन्न्॑ । \newline
40. अवि॑मुच्यमान॒ इत्यवि॑ - मु॒च्य॒मा॒नः॒ । \newline
41. क्षुद्ध्य॑न् परा॒भव॑ति परा॒भव॑ति॒ क्षुद्ध्य॒न् क्षुद्ध्य॑न् परा॒भव॑ति । \newline
42. प॒रा॒भव॑ त्ये॒व मे॒वम् प॑रा॒भव॑ति परा॒भव॑ त्ये॒वम् । \newline
43. प॒रा॒भव॒तीति॑ परा - भव॑ति । \newline
44. ए॒व म॑स्या स्यै॒व मे॒व म॑स्य । \newline
45. अ॒स्या॒ग्नि र॒ग्नि र॑स्या स्या॒ग्निः । \newline
46. अ॒ग्निः परा॒ परा॒ ऽग्नि र॒ग्निः परा᳚ । \newline
47. परा॑ भवति भवति॒ परा॒ परा॑ भवति । \newline
48. भ॒व॒ति॒ तम् तम् भ॑वति भवति॒ तम् । \newline
49. तम् प॑रा॒भव॑न्तम् परा॒भव॑न्त॒म् तम् तम् प॑रा॒भव॑न्तम् । \newline
50. प॒रा॒भव॑न्तं॒ ॅयज॑मानो॒ यज॑मानः परा॒भव॑न्तम् परा॒भव॑न्तं॒ ॅयज॑मानः । \newline
51. प॒रा॒भव॑न्त॒मिति॑ परा - भव॑न्तम् । \newline
52. यज॑मा॒नो ऽन्वनु॒ यज॑मानो॒ यज॑मा॒नो ऽनु॑ । \newline
53. अनु॒ परा॒ परा ऽन्वनु॒ परा᳚ । \newline
54. परा॑ भवति भवति॒ परा॒ परा॑ भवति । \newline
55. भ॒व॒ति॒ स स भ॑वति भवति॒ सः । \newline
56. सो᳚ ऽग्नि म॒ग्निꣳ स सो᳚ ऽग्निम् । \newline
57. अ॒ग्निम् चि॒त्वा चि॒त्वा ऽग्नि म॒ग्निम् चि॒त्वा । \newline
58. चि॒त्वा लू॒क्षो लू॒क्ष श्चि॒त्वा चि॒त्वा लू॒क्षः । \newline
59. लू॒क्षो भ॑वति भवति लू॒क्षो लू॒क्षो भ॑वति । \newline

\textbf{Ghana Paata } \newline

1. अ॒नु॒प॒रि॒क्राम॑म् जुहोति जुहो त्यनुपरि॒क्राम॑ मनुपरि॒क्राम॑म् जुहो॒ त्यप॑रिवर्ग॒ मप॑रिवर्गम् जुहो
त्यनुपरि॒क्राम॑ मनुपरि॒क्राम॑म् जुहो॒ त्यप॑रिवर्गम् । \newline
2. अ॒नु॒प॒रि॒क्राम॒मित्य॑नु - प॒रि॒क्राम᳚म् । \newline
3. जु॒हो॒ त्यप॑रिवर्ग॒ मप॑रिवर्गम् जुहोति जुहो॒ त्यप॑रिवर्ग मे॒वैवा प॑रिवर्गम् जुहोति जुहो॒ त्यप॑रिवर्ग मे॒व । \newline
4. अप॑रिवर्ग मे॒वै वाप॑रिवर्ग॒ मप॑रिवर्ग मे॒वैना॑ नेना ने॒वा प॑रिवर्ग॒ मप॑रिवर्ग मे॒वैनान्॑ । \newline
5. अप॑रिवर्ग॒मित्यप॑रि - व॒र्ग॒म् । \newline
6. ए॒वैना॑ नेना ने॒वैवैना᳚न् प्रीणाति प्रीणा त्येना ने॒वैवैना᳚न् प्रीणाति । \newline
7. ए॒ना॒न् प्री॒णा॒ति॒ प्री॒णा॒ त्ये॒ना॒ ने॒ना॒न् प्री॒णा॒ती॒म मि॒मम् प्री॑ण त्येना नेनान् प्रीणाती॒मम् । \newline
8. प्री॒णा॒ती॒म मि॒मम् प्री॑णाति प्रीणाती॒मꣳ स्तनꣳ॒॒ स्तन॑ मि॒मम् प्री॑णाति प्रीणाती॒मꣳ स्तन᳚म् । \newline
9. इ॒मꣳ स्तनꣳ॒॒ स्तन॑ मि॒म मि॒मꣳ स्तन॒ मूर्ज॑स्वन्त॒ मूर्ज॑स्वन्तꣳ॒॒ स्तन॑ मि॒म मि॒मꣳ स्तन॒ मूर्ज॑स्वन्तम् । \newline
10. स्तन॒ मूर्ज॑स्वन्त॒ मूर्ज॑स्वन्तꣳ॒॒ स्तनꣳ॒॒ स्तन॒ मूर्ज॑स्वन्तम् धय ध॒योर्ज॑स्वन्तꣳ॒॒ स्तनꣳ॒॒ स्तन॒ मूर्ज॑स्वन्तम् धय । \newline
11. ऊर्ज॑स्वन्तम् धय ध॒योर्ज॑स्वन्त॒ मूर्ज॑स्वन्तम् धया॒पा म॒पाम् ध॒योर्ज॑स्वन्त॒ मूर्ज॑स्वन्तम् धया॒पाम् । \newline
12. ध॒या॒पा म॒पाम् ध॑य धया॒पाम् प्रप्या॑त॒म् प्रप्या॑त म॒पाम् ध॑य धया॒पाम् प्रप्या॑तम् । \newline
13. अ॒पाम् प्रप्या॑त॒म् प्रप्या॑त म॒पा म॒पाम् प्रप्या॑त मग्ने ऽग्ने॒ प्रप्या॑त म॒पा म॒पाम् प्रप्या॑त मग्ने । \newline
14. प्रप्या॑त मग्ने ऽग्ने॒ प्रप्या॑त॒म् प्रप्या॑त मग्ने सरि॒रस्य॑ सरि॒रस्या᳚ग्ने॒ प्रप्या॑त॒म् प्रप्या॑त मग्ने सरि॒रस्य॑ । \newline
15. प्रप्या॑त॒मिति॒ प्र - प्या॒त॒म् । \newline
16. अ॒ग्ने॒ स॒रि॒रस्य॑ सरि॒रस्या᳚ग्ने ऽग्ने सरि॒रस्य॒ मद्ध्ये॒ मद्ध्ये॑ सरि॒रस्या᳚ग्ने ऽग्ने सरि॒रस्य॒ मद्ध्ये᳚ । \newline
17. स॒रि॒रस्य॒ मद्ध्ये॒ मद्ध्ये॑ सरि॒रस्य॑ सरि॒रस्य॒ मद्ध्ये᳚ । \newline
18. मद्ध्य॒ इति॒ मद्ध्ये᳚ । \newline
19. उथ्स॑म् जुषस्व जुष॒स्वोथ्स॒ मुथ्स॑म् जुषस्व॒ मधु॑मन्त॒म् मधु॑मन्तम् जुष॒स्वोथ्स॒ मुथ्स॑म् जुषस्व॒ मधु॑मन्तम् । \newline
20. जु॒ष॒स्व॒ मधु॑मन्त॒म् मधु॑मन्तम् जुषस्व जुषस्व॒ मधु॑मन्त मूर्वोर्व॒ मधु॑मन्तम् जुषस्व जुषस्व॒ मधु॑मन्त मूर्व । \newline
21. मधु॑मन्त मूर्वोर्व॒ मधु॑मन्त॒म् मधु॑मन्त मूर्व समु॒द्रियꣳ॑ समु॒द्रिय॑ मूर्व॒ मधु॑मन्त॒म् मधु॑मन्त मूर्व समु॒द्रिय᳚म् । \newline
22. मधु॑मन्त॒मिति॒ मधु॑ - म॒न्त॒म् । \newline
23. ऊ॒र्व॒ स॒मु॒द्रियꣳ॑ समु॒द्रिय॑ मूर्वोर्व समु॒द्रियꣳ॒॒ सद॑नꣳ॒॒ सद॑नꣳ समु॒द्रिय॑ मूर्वोर्व समु॒द्रियꣳ॒॒ सद॑नम् । \newline
24. स॒मु॒द्रियꣳ॒॒ सद॑नꣳ॒॒ सद॑नꣳ समु॒द्रियꣳ॑ समु॒द्रियꣳ॒॒ सद॑न॒ मा सद॑नꣳ समु॒द्रियꣳ॑ समु॒द्रियꣳ॒॒ सद॑न॒ मा । \newline
25. सद॑न॒ मा सद॑नꣳ॒॒ सद॑न॒ मा वि॑शस्व विश॒स्वा सद॑नꣳ॒॒ सद॑न॒ मा वि॑शस्व । \newline
26. आ वि॑शस्व विश॒स्वा वि॑शस्व । \newline
27. वि॒श॒स्वेति॑ विशस्व । \newline
28. यो वै वै यो यो वा अ॒ग्नि म॒ग्निं ॅवै यो यो वा अ॒ग्निम् । \newline
29. वा अ॒ग्नि म॒ग्निं ॅवै वा अ॒ग्निम् प्र॒युज्य॑ प्र॒युज्या॒ग्निं ॅवै वा अ॒ग्निम् प्र॒युज्य॑ । \newline
30. अ॒ग्निम् प्र॒युज्य॑ प्र॒युज्या॒ग्नि म॒ग्निम् प्र॒युज्य॒ न न प्र॒युज्या॒ग्नि म॒ग्निम् प्र॒युज्य॒ न । \newline
31. प्र॒युज्य॒ न न प्र॒युज्य॑ प्र॒युज्य॒ न वि॑मु॒ञ्चति॑ विमु॒ञ्चति॒ न प्र॒युज्य॑ प्र॒युज्य॒ न वि॑मु॒ञ्चति॑ । \newline
32. प्र॒युज्येति॑ प्र - युज्य॑ । \newline
33. न वि॑मु॒ञ्चति॑ विमु॒ञ्चति॒ न न वि॑मु॒ञ्चति॒ यथा॒ यथा॑ विमु॒ञ्चति॒ न न वि॑मु॒ञ्चति॒ यथा᳚ । \newline
34. वि॒मु॒ञ्चति॒ यथा॒ यथा॑ विमु॒ञ्चति॑ विमु॒ञ्चति॒ यथा ऽश्वो ऽश्वो॒ यथा॑ विमु॒ञ्चति॑ विमु॒ञ्चति॒ यथा ऽश्वः॑ । \newline
35. वि॒मु॒ञ्चतीति॑ वि - मु॒ञ्चति॑ । \newline
36. यथा ऽश्वो ऽश्वो॒ यथा॒ यथा ऽश्वो॑ यु॒क्तो यु॒क्तो ऽश्वो॒ यथा॒ यथा ऽश्वो॑ यु॒क्तः । \newline
37. अश्वो॑ यु॒क्तो यु॒क्तो ऽश्वो ऽश्वो॑ यु॒क्तो ऽवि॑मुच्यमा॒नो ऽवि॑मुच्यमानो यु॒क्तो ऽश्वो ऽश्वो॑ यु॒क्तो ऽवि॑मुच्यमानः । \newline
38. यु॒क्तो ऽवि॑मुच्यमा॒नो ऽवि॑मुच्यमानो यु॒क्तो यु॒क्तो ऽवि॑मुच्यमानः॒ क्षुद्ध्य॒न् क्षुद्ध्य॒न् नवि॑मुच्यमानो यु॒क्तो यु॒क्तो ऽवि॑मुच्यमानः॒ क्षुद्ध्यन्न्॑ । \newline
39. अवि॑मुच्यमानः॒ क्षुद्ध्य॒न् क्षुद्ध्य॒न् नवि॑मुच्यमा॒नो ऽवि॑मुच्यमानः॒ क्षुद्ध्य॑न् परा॒भव॑ति परा॒भव॑ति॒ क्षुद्ध्य॒न् नवि॑मुच्यमा॒नो ऽवि॑मुच्यमानः॒ क्षुद्ध्य॑न् परा॒भव॑ति । \newline
40. अवि॑मुच्यमान॒ इत्यवि॑ - मु॒च्य॒मा॒नः॒ । \newline
41. क्षुद्ध्य॑न् परा॒भव॑ति परा॒भव॑ति॒ क्षुद्ध्य॒न् क्षुद्ध्य॑न् परा॒भव॑ त्ये॒व मे॒वम् प॑रा॒भव॑ति॒ क्षुद्ध्य॒न् क्षुद्ध्य॑न् परा॒भव॑ त्ये॒वम् । \newline
42. प॒रा॒भव॑ त्ये॒व मे॒वम् प॑रा॒भव॑ति परा॒भव॑ त्ये॒व म॑स्या स्यै॒वम् प॑रा॒भव॑ति परा॒भव॑ त्ये॒व म॑स्य । \newline
43. प॒रा॒भव॒तीति॑ परा - भव॑ति । \newline
44. ए॒व म॑स्या स्यै॒व मे॒व म॑स्या॒ग्नि र॒ग्नि र॑स्यै॒व मे॒व म॑स्या॒ग्निः । \newline
45. अ॒स्या॒ग्नि र॒ग्नि र॑स्या स्या॒ग्निः परा॒ परा॒ ऽग्नि र॑स्या स्या॒ग्निः परा᳚ । \newline
46. अ॒ग्निः परा॒ परा॒ ऽग्नि र॒ग्निः परा॑ भवति भवति॒ परा॒ ऽग्नि र॒ग्निः परा॑ भवति । \newline
47. परा॑ भवति भवति॒ परा॒ परा॑ भवति॒ तम् तम् भ॑वति॒ परा॒ परा॑ भवति॒ तम् । \newline
48. भ॒व॒ति॒ तम् तम् भ॑वति भवति॒ तम् प॑रा॒भव॑न्तम् परा॒भव॑न्त॒म् तम् भ॑वति भवति॒ तम् प॑रा॒भव॑न्तम् । \newline
49. तम् प॑रा॒भव॑न्तम् परा॒भव॑न्त॒म् तम् तम् प॑रा॒भव॑न्तं॒ ॅयज॑मानो॒ यज॑मानः परा॒भव॑न्त॒म् तम् तम् प॑रा॒भव॑न्तं॒ ॅयज॑मानः । \newline
50. प॒रा॒भव॑न्तं॒ ॅयज॑मानो॒ यज॑मानः परा॒भव॑न्तम् परा॒भव॑न्तं॒ ॅयज॑मा॒नो ऽन्वनु॒ यज॑मानः परा॒भव॑न्तम् परा॒भव॑न्तं॒ ॅयज॑मा॒नो ऽनु॑ । \newline
51. प॒रा॒भव॑न्त॒मिति॑ परा - भव॑न्तम् । \newline
52. यज॑मा॒नो ऽन्वनु॒ यज॑मानो॒ यज॑मा॒नो ऽनु॒ परा॒ परा ऽनु॒ यज॑मानो॒ यज॑मा॒नो ऽनु॒ परा᳚ । \newline
53. अनु॒ परा॒ परा ऽन्वनु॒ परा॑ भवति भवति॒ परा ऽन्वनु॒ परा॑ भवति । \newline
54. परा॑ भवति भवति॒ परा॒ परा॑ भवति॒ स स भ॑वति॒ परा॒ परा॑ भवति॒ सः । \newline
55. भ॒व॒ति॒ स स भ॑वति भवति॒ सो᳚ ऽग्नि म॒ग्निꣳ स भ॑वति भवति॒ सो᳚ ऽग्निम् । \newline
56. सो᳚ ऽग्नि म॒ग्निꣳ स सो᳚ ऽग्निम् चि॒त्वा चि॒त्वा ऽग्निꣳ स सो᳚ ऽग्निम् चि॒त्वा । \newline
57. अ॒ग्निम् चि॒त्वा चि॒त्वा ऽग्नि म॒ग्निम् चि॒त्वा लू॒क्षो लू॒क्ष श्चि॒त्वा ऽग्नि म॒ग्निम् चि॒त्वा लू॒क्षः । \newline
58. चि॒त्वा लू॒क्षो लू॒क्ष श्चि॒त्वा चि॒त्वा लू॒क्षो भ॑वति भवति लू॒क्ष श्चि॒त्वा चि॒त्वा लू॒क्षो भ॑वति । \newline
59. लू॒क्षो भ॑वति भवति लू॒क्षो लू॒क्षो भ॑वती॒म मि॒मम् भ॑वति लू॒क्षो लू॒क्षो भ॑वती॒मम् । \newline
\pagebreak
\markright{ TS 5.5.10.7  \hfill https://www.vedavms.in \hfill}

\section{ TS 5.5.10.7 }

\textbf{TS 5.5.10.7 } \newline
\textbf{Samhita Paata} \newline

भ॑वती॒मꣳ स्तन॒मूर्ज॑स्वन्तं धया॒पामित्याज्य॑स्य पू॒र्णाꣳ स्रुचं॑ जुहोत्ये॒ष वा अ॒ग्नेर्वि॑मो॒को वि॒मुच्यै॒वास्मा॒ अन्न॒मपि॑ दधाति॒ तस्मा॑दाहु॒र्यश्चै॒वं ॅवेद॒ यश्च॒ न सु॒धायꣳ॑ ह॒ वै वा॒जी सुहि॑तो दधा॒तीत्य॒ग्निर्वाव वा॒जी तमे॒व तत् प्री॑णाति॒ स ए॑नं प्री॒तः प्री॑णाति॒ वसी॑यान् भवति ( ) ॥ \newline

\textbf{Pada Paata} \newline

भ॒व॒ति॒ । इ॒मम् । स्तन᳚म् । ऊर्ज॑स्वन्तम् । ध॒य॒ । अ॒पाम् । इति॑ । आज्य॑स्य । पू॒र्णाम् । स्रुच᳚म् । जु॒हो॒ति॒ । ए॒षः । वै । अ॒ग्नेः । वि॒मो॒क इति॑ वि - मो॒कः । वि॒मुच्येति॑ वि - मुच्य॑ । ए॒व । अ॒स्मै॒ । अन्न᳚म् । अपीति॑ । द॒धा॒ति॒ । तस्मा᳚त् । आ॒हुः॒ । यः । च॒ । ए॒वम् । वेद॑ । यः । च॒ । न । सु॒धाय॒मिति॑ सु - धाय᳚म् । ह॒ । वै । वा॒जी । सुहि॑त॒ इति॒ सु - हि॒तः॒ । द॒धा॒ति॒ । इति॑ । अ॒ग्निः । वाव । वा॒जी । तम् । ए॒व । तत् । प्री॒णा॒ति॒ । सः । ए॒न॒म् । प्री॒तः । प्री॒णा॒ति॒ । वसी॑यान् । भ॒व॒ति॒ ( ) ॥  \newline


\textbf{Krama Paata} \newline

भ॒व॒ती॒मम् । इ॒मꣳ स्तन᳚म् । स्तन॒मूर्ज॑स्वन्तम् । ऊर्ज॑स्वन्तम् धय । ध॒या॒पाम् । अ॒पामिति॑ । इत्याज्य॑स्य । आज्य॑स्य पू॒र्णाम् । पू॒र्णाꣳ स्रुच᳚म् । स्रुच॑म् जुहोति । जु॒हो॒त्ये॒षः । ए॒ष वै । वा अ॒ग्नेः । अ॒ग्नेर् वि॑मो॒कः । वि॒मो॒को वि॒मुच्य॑ । वि॒मो॒क इति॑ वि - मो॒कः । वि॒मुच्यै॒व । वि॒मुच्येति॑ वि - मुच्य॑ । ए॒वास्मै᳚ । अ॒स्मा॒ अन्न᳚म् । अन्न॒मपि॑ । अपि॑ दधाति । द॒धा॒ति॒ तस्मा᳚त् । तस्मा॑दाहुः । आ॒हु॒र् यः । यश्च॑ । चै॒वम् । ए॒वम् ॅवेद॑ । वेद॒ यः । यश्च॑ । च॒ न । न सु॒धाय᳚म् । सु॒धायꣳ॑ ह । सु॒धाय॒मिति॑ सु - धाय᳚म् । ह॒ वै । वै वा॒जी । वा॒जी सुहि॑तः । सुहि॑तो दधाति । सुहि॑त॒ इति॒ सु - हि॒तः॒ । द॒धा॒तीति॑ । इत्य॒ग्निः । अ॒ग्निर् वाव । वाव वा॒जी । वा॒जी तम् । तमे॒व । ए॒व तत् । तत् प्री॑णाति । प्री॒णा॒ति॒ सः । स ए॑नम् । ए॒न॒म् प्री॒तः । प्री॒तः प्री॑णाति । प्री॒णा॒ति॒ वसी॑यान् । वसी॑यान् भवति ( ) । भ॒व॒तीति॑ भवति । \newline

\textbf{Jatai Paata} \newline

1. भ॒व॒ती॒म मि॒मम् भ॑वति भवती॒मम् । \newline
2. इ॒मꣳ स्तनꣳ॒॒ स्तन॑ मि॒म मि॒मꣳ स्तन᳚म् । \newline
3. स्तन॒ मूर्ज॑स्वन्त॒ मूर्ज॑स्वन्तꣳ॒॒ स्तनꣳ॒॒ स्तन॒ मूर्ज॑स्वन्तम् । \newline
4. ऊर्ज॑स्वन्तम् धय ध॒यो र्ज॑स्वन्त॒ मूर्ज॑स्वन्तम् धय । \newline
5. ध॒या॒पा म॒पाम् ध॑य धया॒पाम् । \newline
6. अ॒पा मितीत्य॒पा म॒पा मिति॑ । \newline
7. इत्या ज्य॒स्या ज्य॒स्येती त्याज्य॑स्य । \newline
8. आज्य॑स्य पू॒र्णाम् पू॒र्णा माज्य॒स्या ज्य॑स्य पू॒र्णाम् । \newline
9. पू॒र्णाꣳ स्रुचꣳ॒॒ स्रुच॑म् पू॒र्णाम् पू॒र्णाꣳ स्रुच᳚म् । \newline
10. स्रुच॑म् जुहोति जुहोति॒ स्रुचꣳ॒॒ स्रुच॑म् जुहोति । \newline
11. जु॒हो॒ त्ये॒ष ए॒ष जु॑होति जुहो त्ये॒षः । \newline
12. ए॒ष वै वा ए॒ष ए॒ष वै । \newline
13. वा अ॒ग्ने र॒ग्नेर् वै वा अ॒ग्नेः । \newline
14. अ॒ग्नेर् वि॑मो॒को वि॑मो॒को᳚ ऽग्ने र॒ग्नेर् वि॑मो॒कः । \newline
15. वि॒मो॒को वि॒मुच्य॑ वि॒मुच्य॑ विमो॒को वि॑मो॒को वि॒मुच्य॑ । \newline
16. वि॒मो॒क इति॑ वि - मो॒कः । \newline
17. वि॒मुच्यै॒ वैव वि॒मुच्य॑ वि॒मुच्यै॒व । \newline
18. वि॒मुच्येति॑ वि - मुच्य॑ । \newline
19. ए॒वास्मा॑ अस्मा ए॒वै वास्मै᳚ । \newline
20. अ॒स्मा॒ अन्न॒ मन्न॑ मस्मा अस्मा॒ अन्न᳚म् । \newline
21. अन्न॒ मप्य प्यन्न॒ मन्न॒ मपि॑ । \newline
22. अपि॑ दधाति दधा॒ त्यप्यपि॑ दधाति । \newline
23. द॒धा॒ति॒ तस्मा॒त् तस्मा᳚द् दधाति दधाति॒ तस्मा᳚त् । \newline
24. तस्मा॑ दाहु राहु॒ स्तस्मा॒त् तस्मा॑ दाहुः । \newline
25. आ॒हु॒र् यो य आ॑हु राहु॒र् यः । \newline
26. यश्च॑ च॒ यो यश्च॑ । \newline
27. चै॒व मे॒वम् च॑ चै॒वम् । \newline
28. ए॒वं ॅवेद॒ वेदै॒व मे॒वं ॅवेद॑ । \newline
29. वेद॒ यो यो वेद॒ वेद॒ यः । \newline
30. यश्च॑ च॒ यो यश्च॑ । \newline
31. च॒ न न च॑ च॒ न । \newline
32. न सु॒धायꣳ॑ सु॒धाय॒म् न न सु॒धाय᳚म् । \newline
33. सु॒धायꣳ॑ ह ह सु॒धायꣳ॑ सु॒धायꣳ॑ ह । \newline
34. सु॒धाय॒मिति॑ सु - धाय᳚म् । \newline
35. ह॒ वै वै ह॑ ह॒ वै । \newline
36. वै वा॒जी वा॒जी वै वै वा॒जी । \newline
37. वा॒जी सुहि॑तः॒ सुहि॑तो वा॒जी वा॒जी सुहि॑तः । \newline
38. सुहि॑तो दधाति दधाति॒ सुहि॑तः॒ सुहि॑तो दधाति । \newline
39. सुहि॑त॒ इति॒ सु - हि॒तः॒ । \newline
40. द॒धा॒तीतीति॑ दधाति दधा॒तीति॑ । \newline
41. इत्य॒ग्नि र॒ग्नि रिती त्य॒ग्निः । \newline
42. अ॒ग्निर् वाव वावाग्नि र॒ग्निर् वाव । \newline
43. वाव वा॒जी वा॒जी वाव वाव वा॒जी । \newline
44. वा॒जी तम् तं ॅवा॒जी वा॒जी तम् । \newline
45. त मे॒वैव तम् त मे॒व । \newline
46. ए॒व तत् तदे॒ वैव तत् । \newline
47. तत् प्री॑णाति प्रीणाति॒ तत् तत् प्री॑णाति । \newline
48. प्री॒णा॒ति॒ स स प्री॑णाति प्रीणाति॒ सः । \newline
49. स ए॑न मेनꣳ॒॒ स स ए॑नम् । \newline
50. ए॒न॒म् प्री॒तः प्री॒त ए॑न मेनम् प्री॒तः । \newline
51. प्री॒तः प्री॑णाति प्रीणाति प्री॒तः प्री॒तः प्री॑णाति । \newline
52. प्री॒णा॒ति॒ वसी॑या॒न्॒. वसी॑यान् प्रीणाति प्रीणाति॒ वसी॑यान् । \newline
53. वसी॑यान् भवति भवति॒ वसी॑या॒न्॒. वसी॑यान् भवति । \newline
54. भ॒व॒तीति॑ भवति । \newline

\textbf{Ghana Paata } \newline

1. भ॒व॒ती॒म मि॒मम् भ॑वति भवती॒मꣳ स्तनꣳ॒॒ स्तन॑ मि॒मम् भ॑वति भवती॒मꣳ स्तन᳚म् । \newline
2. इ॒मꣳ स्तनꣳ॒॒ स्तन॑ मि॒म मि॒मꣳ स्तन॒ मूर्ज॑स्वन्त॒ मूर्ज॑स्वन्तꣳ॒॒ स्तन॑ मि॒म मि॒मꣳ स्तन॒ मूर्ज॑स्वन्तम् । \newline
3. स्तन॒ मूर्ज॑स्वन्त॒ मूर्ज॑स्वन्तꣳ॒॒ स्तनꣳ॒॒ स्तन॒ मूर्ज॑स्वन्तम् धय ध॒योर्ज॑स्वन्तꣳ॒॒ स्तनꣳ॒॒ स्तन॒ मूर्ज॑स्वन्तम् धय । \newline
4. ऊर्ज॑स्वन्तम् धय ध॒योर्ज॑स्वन्त॒ मूर्ज॑स्वन्तम् धया॒पा म॒पाम् ध॒योर्ज॑स्वन्त॒ मूर्ज॑स्वन्तम् धया॒पाम् । \newline
5. ध॒या॒पा म॒पाम् ध॑य धया॒पा मिती त्य॒पाम् ध॑य धया॒पा मिति॑ । \newline
6. अ॒पा मिती त्य॒पा म॒पा मित्याज्य॒ स्याज्य॒ स्येत्य॒पा म॒पा मित्याज्य॑स्य । \newline
7. इत्याज्य॒ स्याज्य॒ स्येती त्याज्य॑स्य पू॒र्णाम् पू॒र्णा माज्य॒ स्येती त्याज्य॑स्य पू॒र्णाम् । \newline
8. आज्य॑स्य पू॒र्णाम् पू॒र्णा माज्य॒ स्याज्य॑स्य पू॒र्णाꣳ स्रुचꣳ॒॒ स्रुच॑म् पू॒र्णा माज्य॒ स्याज्य॑स्य पू॒र्णाꣳ स्रुच᳚म् । \newline
9. पू॒र्णाꣳ स्रुचꣳ॒॒ स्रुच॑म् पू॒र्णाम् पू॒र्णाꣳ स्रुच॑म् जुहोति जुहोति॒ स्रुच॑म् पू॒र्णाम् पू॒र्णाꣳ स्रुच॑म् जुहोति । \newline
10. स्रुच॑म् जुहोति जुहोति॒ स्रुचꣳ॒॒ स्रुच॑म् जुहो त्ये॒ष ए॒ष जु॑होति॒ स्रुचꣳ॒॒ स्रुच॑म् जुहो त्ये॒षः । \newline
11. जु॒हो॒ त्ये॒ष ए॒ष जु॑होति जुहो त्ये॒ष वै वा ए॒ष जु॑होति जुहो त्ये॒ष वै । \newline
12. ए॒ष वै वा ए॒ष ए॒ष वा अ॒ग्ने र॒ग्नेर् वा ए॒ष ए॒ष वा अ॒ग्नेः । \newline
13. वा अ॒ग्ने र॒ग्नेर् वै वा अ॒ग्नेर् वि॑मो॒को वि॑मो॒को᳚ ऽग्नेर् वै वा अ॒ग्नेर् वि॑मो॒कः । \newline
14. अ॒ग्नेर् वि॑मो॒को वि॑मो॒को᳚ ऽग्ने र॒ग्नेर् वि॑मो॒को वि॒मुच्य॑ वि॒मुच्य॑ विमो॒को᳚ ऽग्ने र॒ग्नेर् वि॑मो॒को वि॒मुच्य॑ । \newline
15. वि॒मो॒को वि॒मुच्य॑ वि॒मुच्य॑ विमो॒को वि॑मो॒को वि॒मुच्यै॒वैव वि॒मुच्य॑ विमो॒को वि॑मो॒को वि॒मुच्यै॒व । \newline
16. वि॒मो॒क इति॑ वि - मो॒कः । \newline
17. वि॒मुच्यै॒वैव वि॒मुच्य॑ वि॒मुच्यै॒ वास्मा॑ अस्मा ए॒व वि॒मुच्य॑ वि॒मुच्यै॒ वास्मै᳚ । \newline
18. वि॒मुच्येति॑ वि - मुच्य॑ । \newline
19. ए॒वास्मा॑ अस्मा ए॒वै वास्मा॒ अन्न॒ मन्न॑ मस्मा ए॒वै वास्मा॒ अन्न᳚म् । \newline
20. अ॒स्मा॒ अन्न॒ मन्न॑ मस्मा अस्मा॒ अन्न॒ मप्य प्यन्न॑ मस्मा अस्मा॒ अन्न॒ मपि॑ । \newline
21. अन्न॒ मप्य प्यन्न॒ मन्न॒ मपि॑ दधाति दधा॒ त्यप्यन्न॒ मन्न॒ मपि॑ दधाति । \newline
22. अपि॑ दधाति दधा॒ त्यप्यपि॑ दधाति॒ तस्मा॒त् तस्मा᳚द् दधा॒ त्यप्यपि॑ दधाति॒ तस्मा᳚त् । \newline
23. द॒धा॒ति॒ तस्मा॒त् तस्मा᳚द् दधाति दधाति॒ तस्मा॑ दाहु राहु॒ स्तस्मा᳚द् दधाति दधाति॒ तस्मा॑ दाहुः । \newline
24. तस्मा॑ दाहु राहु॒ स्तस्मा॒त् तस्मा॑ दाहु॒र् यो य आ॑हु॒ स्तस्मा॒त् तस्मा॑ दाहु॒र् यः । \newline
25. आ॒हु॒र् यो य आ॑हु राहु॒र् यश्च॑ च॒ य आ॑हु राहु॒र् यश्च॑ । \newline
26. यश्च॑ च॒ यो यश्चै॒व मे॒वम् च॒ यो यश्चै॒वम् । \newline
27. चै॒व मे॒वम् च॑ चै॒वं ॅवेद॒ वेदै॒वम् च॑ चै॒वं ॅवेद॑ । \newline
28. ए॒वं ॅवेद॒ वेदै॒व मे॒वं ॅवेद॒ यो यो वेदै॒व मे॒वं ॅवेद॒ यः । \newline
29. वेद॒ यो यो वेद॒ वेद॒ यश्च॑ च॒ यो वेद॒ वेद॒ यश्च॑ । \newline
30. यश्च॑ च॒ यो यश्च॒ न न च॒ यो यश्च॒ न । \newline
31. च॒ न न च॑ च॒ न सु॒धायꣳ॑ सु॒धाय॒म् न च॑ च॒ न सु॒धाय᳚म् । \newline
32. न सु॒धायꣳ॑ सु॒धाय॒म् न न सु॒धायꣳ॑ ह ह सु॒धाय॒म् न न सु॒धायꣳ॑ ह । \newline
33. सु॒धायꣳ॑ ह ह सु॒धायꣳ॑ सु॒धायꣳ॑ ह॒ वै वै ह॑ सु॒धायꣳ॑ सु॒धायꣳ॑ ह॒ वै । \newline
34. सु॒धाय॒मिति॑ सु - धाय᳚म् । \newline
35. ह॒ वै वै ह॑ ह॒ वै वा॒जी वा॒जी वै ह॑ ह॒ वै वा॒जी । \newline
36. वै वा॒जी वा॒जी वै वै वा॒जी सुहि॑तः॒ सुहि॑तो वा॒जी वै वै वा॒जी सुहि॑तः । \newline
37. वा॒जी सुहि॑तः॒ सुहि॑तो वा॒जी वा॒जी सुहि॑तो दधाति दधाति॒ सुहि॑तो वा॒जी वा॒जी सुहि॑तो दधाति । \newline
38. सुहि॑तो दधाति दधाति॒ सुहि॑तः॒ सुहि॑तो दधा॒ती तीति॑ दधाति॒ सुहि॑तः॒ सुहि॑तो दधा॒ तीति॑ । \newline
39. सुहि॑त॒ इति॒ सु - हि॒तः॒ । \newline
40. द॒धा॒ती तीति॑ दधाति दधा॒ तीत्य॒ग्नि र॒ग्नि रिति॑ दधाति दधा॒ तीत्य॒ग्निः । \newline
41. इत्य॒ग्नि र॒ग्नि रिती त्य॒ग्निर् वाव वावाग्नि रिती त्य॒ग्निर् वाव । \newline
42. अ॒ग्निर् वाव वावाग्नि र॒ग्निर् वाव वा॒जी वा॒जी वावाग्नि र॒ग्निर् वाव वा॒जी । \newline
43. वाव वा॒जी वा॒जी वाव वाव वा॒जी तम् तं ॅवा॒जी वाव वाव वा॒जी तम् । \newline
44. वा॒जी तम् तं ॅवा॒जी वा॒जी तमे॒वैव तं ॅवा॒जी वा॒जी तमे॒व । \newline
45. त मे॒वैव तम् त मे॒व तत् तदे॒व तम् त मे॒व तत् । \newline
46. ए॒व तत् तदे॒वैव तत् प्री॑णाति प्रीणाति॒ तदे॒वैव तत् प्री॑णाति । \newline
47. तत् प्री॑णाति प्रीणाति॒ तत् तत् प्री॑णाति॒ स स प्री॑णाति॒ तत् तत् प्री॑णाति॒ सः । \newline
48. प्री॒णा॒ति॒ स स प्री॑णाति प्रीणाति॒ स ए॑न मेनꣳ॒॒ स प्री॑णाति प्रीणाति॒ स ए॑नम् । \newline
49. स ए॑न मेनꣳ॒॒ स स ए॑नम् प्री॒तः प्री॒त ए॑नꣳ॒॒ स स ए॑नम् प्री॒तः । \newline
50. ए॒न॒म् प्री॒तः प्री॒त ए॑न मेनम् प्री॒तः प्री॑णाति प्रीणाति प्री॒त ए॑न मेनम् प्री॒तः प्री॑णाति । \newline
51. प्री॒तः प्री॑णाति प्रीणाति प्री॒तः प्री॒तः प्री॑णाति॒ वसी॑या॒न्॒. वसी॑यान् प्रीणाति प्री॒तः प्री॒तः प्री॑णाति॒ वसी॑यान् । \newline
52. प्री॒णा॒ति॒ वसी॑या॒न्॒. वसी॑यान् प्रीणाति प्रीणाति॒ वसी॑यान् भवति भवति॒ वसी॑यान् प्रीणाति प्रीणाति॒ वसी॑यान् भवति । \newline
53. वसी॑यान् भवति भवति॒ वसी॑या॒न्॒. वसी॑यान् भवति । \newline
54. भ॒व॒तीति॑ भवति । \newline
\pagebreak
\markright{ TS 5.5.11.1  \hfill https://www.vedavms.in \hfill}

\section{ TS 5.5.11.1 }

\textbf{TS 5.5.11.1 } \newline

\textbf{Pada Paata} \newline

इन्द्रा॑य । राज्ञे᳚ । सू॒क॒रः । वरु॑णाय । राज्ञे᳚ । कृष्णः॑ । य॒माय॑ । राज्ञे᳚ । ऋश्यः॑ । ऋ॒ष॒भाय॑ । राज्ञे᳚ । ग॒व॒यः । शा॒र्दू॒लाय॑ । राज्ञे᳚ । गौ॒रः । पु॒रु॒ष॒रा॒जायेति॑ पुरुष - रा॒जाय॑ । म॒र्कटः॑ । क्षि॒प्र॒श्ये॒नस्येति॑ क्षिप्र - श्ये॒नस्य॑ । वर्ति॑का । नील॑ङ्गोः । क्रिमिः॑ । सोम॑स्य । राज्ञ्ः॑ । कु॒लु॒ङ्गः । सिन्धोः᳚ । शिꣳ॒॒शु॒मारः॑ । हि॒मव॑त॒ इति॑ हि॒म - व॒तः॒ । ह॒स्ती ॥  \newline


\textbf{Krama Paata} \newline

इन्द्रा॑य॒ राज्ञे᳚ । राज्ञे॑ सूक॒रः । सू॒क॒रो वरु॑णाय । वरु॑णाय॒ राज्ञे᳚ । राज्ञे॒ कृष्णः॑ । कृष्णो॑ य॒माय॑ । य॒माय॒ राज्ञे᳚ । राज्ञ्॒ ऋश्यः॑ । ऋश्य॑ ऋष॒भाय॑ । ऋ॒ष॒भाय॒ राज्ञे᳚ । राज्ञे॑ गव॒यः । ग॒व॒यः शा᳚र्दू॒लाय॑ । शा॒र्दू॒लाय॒ राज्ञे᳚ । राज्ञे॑ गौ॒रः । गौ॒रः पु॑रुषरा॒जाय॑ । पु॒रु॒ष॒रा॒जाय॑ म॒र्कटः॑ । पु॒रु॒ष॒रा॒जायेति॑ पुरुष - रा॒जाय॑ । म॒र्कटः॑ क्षिप्रश्ये॒नस्य॑ । क्षि॒प्र॒श्ये॒नस्य॒ वर्ति॑का । क्षि॒प्र॒श्ये॒नस्येति॑ क्षिप्र - श्ये॒नस्य॑ । वर्ति॑का॒ नील॑ङ्गोः । नील॑ङ्गोः॒ क्रिमिः॑ । क्रिमिः॒ सोम॑स्य । सोम॑स्य॒ राज्ञ्ः॑ । राज्ञ्ः॑ कुलु॒ङ्गः । कु॒लु॒ङ्गः सिन्धो᳚ । सिन्धोः᳚ शिꣳशु॒मारः॑ । शिꣳ॒॒शु॒मारो॑ हि॒मव॑तः । हि॒मव॑तो ह॒स्ती । हि॒मव॑त॒ इति॑ हि॒म - व॒तः॒ । ह॒स्तीति॑ ह॒स्ती । \newline

\textbf{Jatai Paata} \newline

1. इन्द्रा॑य॒ राज्ञे॒ राज्ञ्॒ इन्द्रा॒ येन्द्रा॑य॒ राज्ञे᳚ । \newline
2. राज्ञे॑ सूक॒रः सू॑क॒रो राज्ञे॒ राज्ञे॑ सूक॒रः । \newline
3. सू॒क॒रो वरु॑णाय॒ वरु॑णाय सूक॒रः सू॑क॒रो वरु॑णाय । \newline
4. वरु॑णाय॒ राज्ञे॒ राज्ञे॒ वरु॑णाय॒ वरु॑णाय॒ राज्ञे᳚ । \newline
5. राज्ञे॒ कृष्णः॒ कृष्णो॒ राज्ञे॒ राज्ञे॒ कृष्णः॑ । \newline
6. कृष्णो॑ य॒माय॑ य॒माय॒ कृष्णः॒ कृष्णो॑ य॒माय॑ । \newline
7. य॒माय॒ राज्ञे॒ राज्ञे॑ य॒माय॑ य॒माय॒ राज्ञे᳚ । \newline
8. राज्ञ्॒ ऋश्य॒ ऋश्यो॒ राज्ञे॒ राज्ञ्॒ ऋश्यः॑ । \newline
9. ऋश्य॑ ऋष॒भाय॑ र्.ष॒भाय र्श्य॒ ऋश्य॑ ऋष॒भाय॑ । \newline
10. ऋ॒ष॒भाय॒ राज्ञे॒ राज्ञ्॑ ऋष॒भाय॑ र्.ष॒भाय॒ राज्ञे᳚ । \newline
11. राज्ञे॑ गव॒यो ग॑व॒यो राज्ञे॒ राज्ञे॑ गव॒यः । \newline
12. ग॒व॒यः शा᳚र्दू॒लाय॑ शार्दू॒लाय॑ गव॒यो ग॑व॒यः शा᳚र्दू॒लाय॑ । \newline
13. शा॒र्दू॒लाय॒ राज्ञे॒ राज्ञे॑ शार्दू॒लाय॑ शार्दू॒लाय॒ राज्ञे᳚ । \newline
14. राज्ञे॑ गौ॒रो गौ॒रो राज्ञे॒ राज्ञे॑ गौ॒रः । \newline
15. गौ॒रः पु॑रुषरा॒जाय॑ पुरुषरा॒जाय॑ गौ॒रो गौ॒रः पु॑रुषरा॒जाय॑ । \newline
16. पु॒रु॒ष॒रा॒जाय॑ म॒र्कटो॑ म॒र्कटः॑ पुरुषरा॒जाय॑ पुरुषरा॒जाय॑ म॒र्कटः॑ । \newline
17. पु॒रु॒ष॒रा॒जायेति॑ पुरुष - रा॒जाय॑ । \newline
18. म॒र्कटः॑ क्षिप्रश्ये॒नस्य॑ क्षिप्रश्ये॒नस्य॑ म॒र्कटो॑ म॒र्कटः॑ क्षिप्रश्ये॒नस्य॑ । \newline
19. क्षि॒प्र॒श्ये॒नस्य॒ वर्ति॑का॒ वर्ति॑का क्षिप्रश्ये॒नस्य॑ क्षिप्रश्ये॒नस्य॒ वर्ति॑का । \newline
20. क्षि॒प्र॒श्ये॒नस्येति॑ क्षिप्र - श्ये॒नस्य॑ । \newline
21. वर्ति॑का॒ नील॑ङ्गो॒र् नील॑ङ्गो॒र् वर्ति॑का॒ वर्ति॑का॒ नील॑ङ्गोः । \newline
22. नील॑ङ्गोः॒ क्रिमिः॒ क्रिमि॒र् नील॑ङ्गो॒र् नील॑ङ्गोः॒ क्रिमिः॑ । \newline
23. क्रिमिः॒ सोम॑स्य॒ सोम॑स्य॒ क्रिमिः॒ क्रिमिः॒ सोम॑स्य । \newline
24. सोम॑स्य॒ राज्ञो॒ राज्ञ्ः॒ सोम॑स्य॒ सोम॑स्य॒ राज्ञ्ः॑ । \newline
25. राज्ञ्ः॑ कुलु॒ङ्गः कु॑लु॒ङ्गो राज्ञो॒ राज्ञ्ः॑ कुलु॒ङ्गः । \newline
26. कु॒लु॒ङ्गः सिन्धोः॒ सिन्धोः᳚ कुलु॒ङ्गः कु॑लु॒ङ्गः सिन्धोः᳚ । \newline
27. सिन्धोः᳚ शिꣳशु॒मारः॑ शिꣳशु॒मारः॒ सिन्धोः॒ सिन्धोः᳚ शिꣳशु॒मारः॑ । \newline
28. शिꣳ॒॒शु॒मारो॑ हि॒मव॑तो हि॒मव॑तः शिꣳशु॒मारः॑ शिꣳशु॒मारो॑ हि॒मव॑तः । \newline
29. हि॒मव॑तो ह॒स्ती ह॒स्ती हि॒मव॑तो हि॒मव॑तो ह॒स्ती । \newline
30. हि॒मव॑त॒ इति॑ हि॒म - व॒तः॒ । \newline
31. ह॒स्तीति॑ ह॒स्ती । \newline

\textbf{Ghana Paata } \newline

1. इन्द्रा॑य॒ राज्ञे॒ राज्ञ्॒ इन्द्रा॒ येन्द्रा॑य॒ राज्ञे॑ सूक॒रः सू॑क॒रो राज्ञ्॒ इन्द्रा॒ येन्द्रा॑य॒ राज्ञे॑ सूक॒रः । \newline
2. राज्ञे॑ सूक॒रः सू॑क॒रो राज्ञे॒ राज्ञे॑ सूक॒रो वरु॑णाय॒ वरु॑णाय सूक॒रो राज्ञे॒ राज्ञे॑ सूक॒रो वरु॑णाय । \newline
3. सू॒क॒रो वरु॑णाय॒ वरु॑णाय सूक॒रः सू॑क॒रो वरु॑णाय॒ राज्ञे॒ राज्ञे॒ वरु॑णाय सूक॒रः सू॑क॒रो वरु॑णाय॒ राज्ञे᳚ । \newline
4. वरु॑णाय॒ राज्ञे॒ राज्ञे॒ वरु॑णाय॒ वरु॑णाय॒ राज्ञे॒ कृष्णः॒ कृष्णो॒ राज्ञे॒ वरु॑णाय॒ वरु॑णाय॒ राज्ञे॒ कृष्णः॑ । \newline
5. राज्ञे॒ कृष्णः॒ कृष्णो॒ राज्ञे॒ राज्ञे॒ कृष्णो॑ य॒माय॑ य॒माय॒ कृष्णो॒ राज्ञे॒ राज्ञे॒ कृष्णो॑ य॒माय॑ । \newline
6. कृष्णो॑ य॒माय॑ य॒माय॒ कृष्णः॒ कृष्णो॑ य॒माय॒ राज्ञे॒ राज्ञे॑ य॒माय॒ कृष्णः॒ कृष्णो॑ य॒माय॒ राज्ञे᳚ । \newline
7. य॒माय॒ राज्ञे॒ राज्ञे॑ य॒माय॑ य॒माय॒ राज्ञ्॒ ऋश्य॒ ऋश्यो॒ राज्ञे॑ य॒माय॑ य॒माय॒ राज्ञ्॒ ऋश्यः॑ । \newline
8. राज्ञ्॒ ऋश्य॒ ऋश्यो॒ राज्ञे॒ राज्ञ्॒ ऋश्य॑ ऋष॒भाय॑ र्.ष॒भाय र्‌श्यो॒ राज्ञे॒ राज्ञ्॒ ऋश्य॑ ऋष॒भाय॑ । \newline
9. ऋश्य॑ ऋष॒भाय॑ र्.ष॒भाय र्‌श्य॒ ऋश्य॑ ऋष॒भाय॒ राज्ञे॒ राज्ञ्॑ ऋष॒भाय र्‌श्य॒ ऋश्य॑ ऋष॒भाय॒ राज्ञे᳚ । \newline
10. ऋ॒ष॒भाय॒ राज्ञे॒ राज्ञ्॑ ऋष॒भाय॑ र्.ष॒भाय॒ राज्ञे॑ गव॒यो ग॑व॒यो राज्ञ्॑ ऋष॒भाय॑ र्.ष॒भाय॒ राज्ञे॑ गव॒यः । \newline
11. राज्ञे॑ गव॒यो ग॑व॒यो राज्ञे॒ राज्ञे॑ गव॒यः शा᳚र्दू॒लाय॑ शार्दू॒लाय॑ गव॒यो राज्ञे॒ राज्ञे॑ गव॒यः शा᳚र्दू॒लाय॑ । \newline
12. ग॒व॒यः शा᳚र्दू॒लाय॑ शार्दू॒लाय॑ गव॒यो ग॑व॒यः शा᳚र्दू॒लाय॒ राज्ञे॒ राज्ञे॑ शार्दू॒लाय॑ गव॒यो ग॑व॒यः शा᳚र्दू॒लाय॒ राज्ञे᳚ । \newline
13. शा॒र्दू॒लाय॒ राज्ञे॒ राज्ञे॑ शार्दू॒लाय॑ शार्दू॒लाय॒ राज्ञे॑ गौ॒रो गौ॒रो राज्ञे॑ शार्दू॒लाय॑ शार्दू॒लाय॒ राज्ञे॑ गौ॒रः । \newline
14. राज्ञे॑ गौ॒रो गौ॒रो राज्ञे॒ राज्ञे॑ गौ॒रः पु॑रुषरा॒जाय॑ पुरुषरा॒जाय॑ गौ॒रो राज्ञे॒ राज्ञे॑ गौ॒रः पु॑रुषरा॒जाय॑ । \newline
15. गौ॒रः पु॑रुषरा॒जाय॑ पुरुषरा॒जाय॑ गौ॒रो गौ॒रः पु॑रुषरा॒जाय॑ म॒र्कटो॑ म॒र्कटः॑ पुरुषरा॒जाय॑ गौ॒रो गौ॒रः पु॑रुषरा॒जाय॑ म॒र्कटः॑ । \newline
16. पु॒रु॒ष॒रा॒जाय॑ म॒र्कटो॑ म॒र्कटः॑ पुरुषरा॒जाय॑ पुरुषरा॒जाय॑ म॒र्कटः॑ क्षिप्रश्ये॒नस्य॑ क्षिप्रश्ये॒नस्य॑ म॒र्कटः॑ पुरुषरा॒जाय॑ पुरुषरा॒जाय॑ म॒र्कटः॑ क्षिप्रश्ये॒नस्य॑ । \newline
17. पु॒रु॒ष॒रा॒जायेति॑ पुरुष - रा॒जाय॑ । \newline
18. म॒र्कटः॑ क्षिप्रश्ये॒नस्य॑ क्षिप्रश्ये॒नस्य॑ म॒र्कटो॑ म॒र्कटः॑ क्षिप्रश्ये॒नस्य॒ वर्ति॑का॒ वर्ति॑का क्षिप्रश्ये॒नस्य॑ म॒र्कटो॑ म॒र्कटः॑ क्षिप्रश्ये॒नस्य॒ वर्ति॑का । \newline
19. क्षि॒प्र॒श्ये॒नस्य॒ वर्ति॑का॒ वर्ति॑का क्षिप्रश्ये॒नस्य॑ क्षिप्रश्ये॒नस्य॒ वर्ति॑का॒ नील॑ङ्गो॒र् नील॑ङ्गो॒र् वर्ति॑का क्षिप्रश्ये॒नस्य॑ क्षिप्रश्ये॒नस्य॒ वर्ति॑का॒ नील॑ङ्गोः । \newline
20. क्षि॒प्र॒श्ये॒नस्येति॑ क्षिप्र - श्ये॒नस्य॑ । \newline
21. वर्ति॑का॒ नील॑ङ्गो॒र् नील॑ङ्गो॒र् वर्ति॑का॒ वर्ति॑का॒ नील॑ङ्गोः॒ क्रिमिः॒ क्रिमि॒र् नील॑ङ्गो॒र् वर्ति॑का॒ वर्ति॑का॒ नील॑ङ्गोः॒ क्रिमिः॑ । \newline
22. नील॑ङ्गोः॒ क्रिमिः॒ क्रिमि॒र् नील॑ङ्गो॒र् नील॑ङ्गोः॒ क्रिमिः॒ सोम॑स्य॒ सोम॑स्य॒ क्रिमि॒र् नील॑ङ्गो॒र् नील॑ङ्गोः॒ क्रिमिः॒ सोम॑स्य । \newline
23. क्रिमिः॒ सोम॑स्य॒ सोम॑स्य॒ क्रिमिः॒ क्रिमिः॒ सोम॑स्य॒ राज्ञो॒ राज्ञ्ः॒ सोम॑स्य॒ क्रिमिः॒ क्रिमिः॒ सोम॑स्य॒ राज्ञ्ः॑ । \newline
24. सोम॑स्य॒ राज्ञो॒ राज्ञ्ः॒ सोम॑स्य॒ सोम॑स्य॒ राज्ञ्ः॑ कुलु॒ङ्गः कु॑लु॒ङ्गो राज्ञ्ः॒ सोम॑स्य॒ सोम॑स्य॒ राज्ञ्ः॑ कुलु॒ङ्गः । \newline
25. राज्ञ्ः॑ कुलु॒ङ्गः कु॑लु॒ङ्गो राज्ञो॒ राज्ञ्ः॑ कुलु॒ङ्गः सिन्धोः॒ सिन्धोः᳚ कुलु॒ङ्गो राज्ञो॒ राज्ञ्ः॑ कुलु॒ङ्गः सिन्धोः᳚ । \newline
26. कु॒लु॒ङ्गः सिन्धोः॒ सिन्धोः᳚ कुलु॒ङ्गः कु॑लु॒ङ्गः सिन्धोः᳚ शिꣳशु॒मारः॑ शिꣳशु॒मारः॒ सिन्धोः᳚ कुलु॒ङ्गः कु॑लु॒ङ्गः सिन्धोः᳚ शिꣳशु॒मारः॑ । \newline
27. सिन्धोः᳚ शिꣳशु॒मारः॑ शिꣳशु॒मारः॒ सिन्धोः॒ सिन्धोः᳚ शिꣳशु॒मारो॑ हि॒मव॑तो हि॒मव॑तः शिꣳशु॒मारः॒ सिन्धोः॒ सिन्धोः᳚ शिꣳशु॒मारो॑ हि॒मव॑तः । \newline
28. शिꣳ॒॒शु॒मारो॑ हि॒मव॑तो हि॒मव॑तः शिꣳशु॒मारः॑ शिꣳशु॒मारो॑ हि॒मव॑तो ह॒स्ती ह॒स्ती हि॒मव॑तः शिꣳशु॒मारः॑ शिꣳशु॒मारो॑ हि॒मव॑तो ह॒स्ती । \newline
29. हि॒मव॑तो ह॒स्ती ह॒स्ती हि॒मव॑तो हि॒मव॑तो ह॒स्ती । \newline
30. हि॒मव॑त॒ इति॑ हि॒म - व॒तः॒ । \newline
31. ह॒स्तीति॑ ह॒स्ती । \newline
\pagebreak
\markright{ TS 5.5.12.1  \hfill https://www.vedavms.in \hfill}

\section{ TS 5.5.12.1 }

\textbf{TS 5.5.12.1 } \newline
\textbf{Samhita Paata} \newline

म॒युः प्रा॑जाप॒त्य ऊ॒लो हली᳚क्ष्णो वृषदꣳ॒॒शस्ते धा॒तुः सर॑स्वत्यै॒ शारिः॑ श्ये॒ता पु॑रुष॒वाख् सर॑स्वते॒ शुकः॑ श्ये॒तः पु॑रुष॒वागा॑र॒ण्यो॑ऽजो न॑कु॒लः शका॒ ते पौ॒ष्णा वा॒चे क्रौ॒ञ्चः ॥ \newline

\textbf{Pada Paata} \newline

म॒युः । प्रा॒जा॒प॒त्य इति॑ प्राजा - प॒त्यः । ऊ॒लः । हली᳚क्ष्णः । वृ॒ष॒दꣳ॒॒शः । ते । धा॒तुः । सर॑स्वत्यै । शारिः॑ । श्ये॒ता । पु॒रु॒ष॒वागिति॑ पुरुष-वाक् । सर॑स्वते । शुकः॑ । श्ये॒तः । पु॒रु॒ष॒वागिति॑ पुरुष-वाक् । आ॒र॒ण्यः । अ॒जः । न॒कु॒लः । शका᳚ । ते । पौ॒ष्णाः । वा॒चे । क्रौ॒ञ्चः ॥  \newline


\textbf{Krama Paata} \newline

म॒युः प्रा॑जाप॒त्यः । प्रा॒जा॒प॒त्य ऊ॒लः । प्रा॒जा॒प॒त्य इति॑ प्राजा - प॒त्यः । ऊ॒लो हली᳚क्ष्णः । हली᳚क्ष्णो वृषदꣳ॒॒शः । वृ॒ष॒दꣳ॒॒शस्ते । ते धा॒तुः । धा॒तुः सर॑स्वत्यै । सर॑स्वत्यै॒ शारिः॑ । शारिः॑ श्ये॒ता । श्ये॒ता पु॑रुष॒वाक् । पु॒रु॒ष॒वाख् सर॑स्वते । पु॒रु॒ष॒वागिति॑ पुरुष - वाक् । सर॑स्वते॒ शुकः॑ । शुकः॑ श्ये॒तः । श्ये॒तः पु॑रुष॒वाक् । पु॒रु॒ष॒वागा॑र॒ण्यः । पु॒रु॒ष॒वागिति॑ पुरुष - वाक् । आ॒र॒ण्यो॑ऽजः । अ॒जो न॑कु॒लः । न॒कु॒लः शका᳚ । शका॒ ते । ते पौ॒ष्णाः । पौ॒ष्णा वा॒चे । वा॒चे क्रौ॒ञ्चः । क्रौ॒ञ्च इति॑ क्रौ॒ञ्चः । \newline

\textbf{Jatai Paata} \newline

1. म॒युः प्रा॑जाप॒त्यः प्रा॑जाप॒त्यो म॒युर् म॒युः प्रा॑जाप॒त्यः । \newline
2. प्रा॒जा॒प॒त्य ऊ॒ल ऊ॒लः प्रा॑जाप॒त्यः प्रा॑जाप॒त्य ऊ॒लः । \newline
3. प्रा॒जा॒प॒त्य इति॑ प्राजा - प॒त्यः । \newline
4. ऊ॒लो हली᳚क्ष्णो॒ हली᳚क्ष्ण ऊ॒ल ऊ॒लो हली᳚क्ष्णः । \newline
5. हली᳚क्ष्णो वृषदꣳ॒॒शो वृ॑षदꣳ॒॒शो हली᳚क्ष्णो॒ हली᳚क्ष्णो वृषदꣳ॒॒शः । \newline
6. वृ॒ष॒दꣳ॒॒श स्ते ते वृ॑षदꣳ॒॒शो वृ॑षदꣳ॒॒श स्ते । \newline
7. ते धा॒तुर् धा॒तु स्ते ते धा॒तुः । \newline
8. धा॒तुः सर॑स्वत्यै॒ सर॑स्वत्यै धा॒तुर् धा॒तुः सर॑स्वत्यै । \newline
9. सर॑स्वत्यै॒ शारिः॒ शारिः॒ सर॑स्वत्यै॒ सर॑स्वत्यै॒ शारिः॑ । \newline
10. शारिः॑ श्ये॒ता श्ये॒ता शारिः॒ शारिः॑ श्ये॒ता । \newline
11. श्ये॒ता पु॑रुष॒वाक् पु॑रुष॒वाक् छ्ये॒ता श्ये॒ता पु॑रुष॒वाक् । \newline
12. पु॒रु॒ष॒वाख् सर॑स्वते॒ सर॑स्वते पुरुष॒वाक् पु॑रुष॒वाख् सर॑स्वते । \newline
13. पु॒रु॒ष॒वागिति॑ पुरुष - वाक् । \newline
14. सर॑स्वते॒ शुकः॒ शुकः॒ सर॑स्वते॒ सर॑स्वते॒ शुकः॑ । \newline
15. शुकः॑ श्ये॒तः श्ये॒तः शुकः॒ शुकः॑ श्ये॒तः । \newline
16. श्ये॒तः पु॑रुष॒वाक् पु॑रुष॒वाक् छ्ये॒तः श्ये॒तः पु॑रुष॒वाक् । \newline
17. पु॒रु॒ष॒वा गा॑र॒ण्य आ॑र॒ण्यः पु॑रुष॒वाक् पु॑रुष॒वा गा॑र॒ण्यः । \newline
18. पु॒रु॒ष॒वागिति॑ पुरुष - वाक् । \newline
19. आ॒र॒ण्यो᳚(1॒) ऽजो॑ ऽज आ॑र॒ण्य आ॑र॒ण्यो॑ ऽजः । \newline
20. अ॒जो न॑कु॒लो न॑कु॒लो᳚(1॒) ऽजो॑ ऽजो न॑कु॒लः । \newline
21. न॒कु॒लः शका॒ शका॑ नकु॒लो न॑कु॒लः शका᳚ । \newline
22. शका॒ ते ते शका॒ शका॒ ते । \newline
23. ते पौ॒ष्णाः पौ॒ष्णा स्ते ते पौ॒ष्णाः । \newline
24. पौ॒ष्णा वा॒चे वा॒चे पौ॒ष्णाः पौ॒ष्णा वा॒चे । \newline
25. वा॒चे क्रौ॒ञ्चः क्रौ॒ञ्चो वा॒चे वा॒चे क्रौ॒ञ्चः । \newline
26. क्रौ॒ञ्च इति॑ क्रौ॒ञ्चः । \newline

\textbf{Ghana Paata } \newline

1. म॒युः प्रा॑जाप॒त्यः प्रा॑जाप॒त्यो म॒युर् म॒युः प्रा॑जाप॒त्य ऊ॒ल ऊ॒लः प्रा॑जाप॒त्यो म॒युर् म॒युः प्रा॑जाप॒त्य ऊ॒लः । \newline
2. प्रा॒जा॒प॒त्य ऊ॒ल ऊ॒लः प्रा॑जाप॒त्यः प्रा॑जाप॒त्य ऊ॒लो हली᳚क्ष्णो॒ हली᳚क्ष्ण ऊ॒लः प्रा॑जाप॒त्यः प्रा॑जाप॒त्य ऊ॒लो हली᳚क्ष्णः । \newline
3. प्रा॒जा॒प॒त्य इति॑ प्राजा - प॒त्यः । \newline
4. ऊ॒लो हली᳚क्ष्णो॒ हली᳚क्ष्ण ऊ॒ल ऊ॒लो हली᳚क्ष्णो वृषदꣳ॒॒शो वृ॑षदꣳ॒॒शो हली᳚क्ष्ण ऊ॒ल ऊ॒लो हली᳚क्ष्णो वृषदꣳ॒॒शः । \newline
5. हली᳚क्ष्णो वृषदꣳ॒॒शो वृ॑षदꣳ॒॒शो हली᳚क्ष्णो॒ हली᳚क्ष्णो वृषदꣳ॒॒श स्ते ते वृ॑षदꣳ॒॒शो हली᳚क्ष्णो॒ हली᳚क्ष्णो वृषदꣳ॒॒श स्ते । \newline
6. वृ॒ष॒दꣳ॒॒श स्ते ते वृ॑षदꣳ॒॒शो वृ॑षदꣳ॒॒श स्ते धा॒तुर् धा॒तु स्ते वृ॑षदꣳ॒॒शो वृ॑षदꣳ॒॒श स्ते धा॒तुः । \newline
7. ते धा॒तुर् धा॒तु स्ते ते धा॒तुः सर॑स्वत्यै॒ सर॑स्वत्यै धा॒तु स्ते ते धा॒तुः सर॑स्वत्यै । \newline
8. धा॒तुः सर॑स्वत्यै॒ सर॑स्वत्यै धा॒तुर् धा॒तुः सर॑स्वत्यै॒ शारिः॒ शारिः॒ सर॑स्वत्यै धा॒तुर् धा॒तुः सर॑स्वत्यै॒ शारिः॑ । \newline
9. सर॑स्वत्यै॒ शारिः॒ शारिः॒ सर॑स्वत्यै॒ सर॑स्वत्यै॒ शारिः॑ श्ये॒ता श्ये॒ता शारिः॒ सर॑स्वत्यै॒ सर॑स्वत्यै॒ शारिः॑ श्ये॒ता । \newline
10. शारिः॑ श्ये॒ता श्ये॒ता शारिः॒ शारिः॑ श्ये॒ता पु॑रुष॒वाक् पु॑रुष॒वाक् छ्ये॒ता शारिः॒ शारिः॑ श्ये॒ता पु॑रुष॒वाक् । \newline
11. श्ये॒ता पु॑रुष॒वाक् पु॑रुष॒वाक् छ्ये॒ता श्ये॒ता पु॑रुष॒वाख् सर॑स्वते॒ सर॑स्वते पुरुष॒वाक् छ्ये॒ता श्ये॒ता पु॑रुष॒वाख् सर॑स्वते । \newline
12. पु॒रु॒ष॒वाख् सर॑स्वते॒ सर॑स्वते पुरुष॒वाक् पु॑रुष॒वाख् सर॑स्वते॒ शुकः॒ शुकः॒ सर॑स्वते पुरुष॒वाक् पु॑रुष॒वाख् सर॑स्वते॒ शुकः॑ । \newline
13. पु॒रु॒ष॒वागिति॑ पुरुष - वाक् । \newline
14. सर॑स्वते॒ शुकः॒ शुकः॒ सर॑स्वते॒ सर॑स्वते॒ शुकः॑ श्ये॒तः श्ये॒तः शुकः॒ सर॑स्वते॒ सर॑स्वते॒ शुकः॑ श्ये॒तः । \newline
15. शुकः॑ श्ये॒तः श्ये॒तः शुकः॒ शुकः॑ श्ये॒तः पु॑रुष॒वाक् पु॑रुष॒वाक् छ्ये॒तः शुकः॒ शुकः॑ श्ये॒तः पु॑रुष॒वाक् । \newline
16. श्ये॒तः पु॑रुष॒वाक् पु॑रुष॒वाक् छ्ये॒तः श्ये॒तः पु॑रुष॒वा गा॑र॒ण्य आ॑र॒ण्यः पु॑रुष॒वाक् छ्ये॒तः श्ये॒तः पु॑रुष॒वा गा॑र॒ण्यः । \newline
17. पु॒रु॒ष॒वा गा॑र॒ण्य आ॑र॒ण्यः पु॑रुष॒वाक् पु॑रुष॒वा गा॑र॒ण्यो᳚(1॒) ऽजो॑ ऽज आ॑र॒ण्यः पु॑रुष॒वाक् पु॑रुष॒वा गा॑र॒ण्यो॑ ऽजः । \newline
18. पु॒रु॒ष॒वागिति॑ पुरुष - वाक् । \newline
19. आ॒र॒ण्यो᳚(1॒) ऽजो॑ ऽज आ॑र॒ण्य आ॑र॒ण्यो॑ ऽजो न॑कु॒लो न॑कु॒लो॑ ऽज आ॑र॒ण्य आ॑र॒ण्यो॑ ऽजो न॑कु॒लः । \newline
20. अ॒जो न॑कु॒लो न॑कु॒लो᳚(1॒) ऽजो॑ ऽजो न॑कु॒लः शका॒ शका॑ नकु॒लो᳚(1॒) ऽजो॑ ऽजो न॑कु॒लः शका᳚ । \newline
21. न॒कु॒लः शका॒ शका॑ नकु॒लो न॑कु॒लः शका॒ ते ते शका॑ नकु॒लो न॑कु॒लः शका॒ ते । \newline
22. शका॒ ते ते शका॒ शका॒ ते पौ॒ष्णाः पौ॒ष्णा स्ते शका॒ शका॒ ते पौ॒ष्णाः । \newline
23. ते पौ॒ष्णाः पौ॒ष्णा स्ते ते पौ॒ष्णा वा॒चे वा॒चे पौ॒ष्णा स्ते ते पौ॒ष्णा वा॒चे । \newline
24. पौ॒ष्णा वा॒चे वा॒चे पौ॒ष्णाः पौ॒ष्णा वा॒चे क्रौ॒ञ्चः क्रौ॒ञ्चो वा॒चे पौ॒ष्णाः पौ॒ष्णा वा॒चे क्रौ॒ञ्चः । \newline
25. वा॒चे क्रौ॒ञ्चः क्रौ॒ञ्चो वा॒चे वा॒चे क्रौ॒ञ्चः । \newline
26. क्रौ॒ञ्च इति॑ क्रौ॒ञ्चः । \newline
\pagebreak
\markright{ TS 5.5.13.1  \hfill https://www.vedavms.in \hfill}

\section{ TS 5.5.13.1 }

\textbf{TS 5.5.13.1 } \newline
\textbf{Samhita Paata} \newline

अ॒पां नप्त्रे॑ ज॒षो ना॒क्रो मक॑रः कुली॒कय॒स्तेऽकू॑पारस्य वा॒चे पै᳚ङ्गरा॒जो भगा॑य कु॒षीत॑क आ॒ती वा॑ह॒सो दर्वि॑दा॒ ते वा॑य॒व्या॑ दि॒ग्भ्यश्च॑क्रवा॒कः ॥ \newline

\textbf{Pada Paata} \newline

अ॒पाम् । नप्त्रे᳚ । ज॒षः । ना॒क्रः । मक॑रः । कु॒ली॒कयः॑ । ते । अकू॑पारस्य । वा॒चे । पै॒ङ्ग॒रा॒ज इति॑ पैङ्ग-रा॒जः । भगा॑य । कु॒षीत॑कः । आ॒ती । वा॒ह॒सः । दर्वि॑दा । ते । वा॒य॒व्याः᳚ । दि॒ग्भ्य इति॑ दिक्-भ्यः । च॒क्र॒वा॒कः ॥  \newline


\textbf{Krama Paata} \newline

अ॒पान् नप्त्रे᳚ । नप्त्रे॑ ज॒षः । ज॒षो ना॒क्रः । ना॒क्रो मक॑रः । मक॑रः कुली॒कयः॑ । कु॒ली॒कय॒स्ते । तेऽकू॑पारस्य । अकू॑पारस्य वा॒चे । वा॒चे पै᳚ङ्गरा॒जः । पै॒ङ्ग॒रा॒जो भगा॑य । पै॒ङ्ग॒रा॒ज इति॑ पैङ्ग - रा॒जः । भगा॑य कु॒षीत॑कः । कु॒षीत॑क आ॒ती । आ॒ती वा॑ह॒सः । वा॒ह॒सो दर्वि॑दा । दर्वि॑दा॒ ते । ते वा॑य॒व्याः᳚ । वा॒य॒व्या॑ दि॒ग्भ्यः । दि॒ग्भ्यश्च॑क्रवा॒कः । दि॒ग्भ्य इति॑ दिक् - भ्यः । च॒क्र॒वा॒क इति॑ चक्रवा॒कः । \newline

\textbf{Jatai Paata} \newline

1. अ॒पान् नप्त्रे॒ नप्त्रे॒ ऽपा म॒पान् नप्त्रे᳚ । \newline
2. नप्त्रे॑ ज॒षो ज॒षो नप्त्रे॒ नप्त्रे॑ ज॒षः । \newline
3. ज॒षो ना॒क्रो ना॒क्रो ज॒षो ज॒षो ना॒क्रः । \newline
4. ना॒क्रो मक॑रो॒ मक॑रो ना॒क्रो ना॒क्रो मक॑रः । \newline
5. मक॑रः कुली॒कयः॑ कुली॒कयो॒ मक॑रो॒ मक॑रः कुली॒कयः॑ । \newline
6. कु॒ली॒कय॒ स्ते ते कु॑ली॒कयः॑ कुली॒कय॒ स्ते । \newline
7. ते ऽकू॑पार॒स्या कू॑पारस्य॒ ते ते ऽकू॑पारस्य । \newline
8. अकू॑पारस्य वा॒चे वा॒चे ऽकू॑पार॒स्या कू॑पारस्य वा॒चे । \newline
9. वा॒चे पै᳚ङ्गरा॒जः पै᳚ङ्गरा॒जो वा॒चे वा॒चे पै᳚ङ्गरा॒जः । \newline
10. पै॒ङ्ग॒रा॒जो भगा॑य॒ भगा॑य पैङ्गरा॒जः पै᳚ङ्गरा॒जो भगा॑य । \newline
11. पै॒ङ्ग॒रा॒ज इति॑ पैङ्ग - रा॒जः । \newline
12. भगा॑य कु॒षीत॑कः कु॒षीत॑को॒ भगा॑य॒ भगा॑य कु॒षीत॑कः । \newline
13. कु॒षीत॑क आ॒त्या॑ती कु॒षीत॑कः कु॒षीत॑क आ॒ती । \newline
14. आ॒ती वा॑ह॒सो वा॑ह॒स आ॒त्या॑ती वा॑ह॒सः । \newline
15. वा॒ह॒सो दर्वि॑दा॒ दर्वि॑दा वाह॒सो वा॑ह॒सो दर्वि॑दा । \newline
16. दर्वि॑दा॒ ते ते दर्वि॑दा॒ दर्वि॑दा॒ ते । \newline
17. ते वा॑य॒व्या॑ वाय॒व्या᳚ स्ते ते वा॑य॒व्याः᳚ । \newline
18. वा॒य॒व्या॑ दि॒ग्भ्यो दि॒ग्भ्यो वा॑य॒व्या॑ वाय॒व्या॑ दि॒ग्भ्यः । \newline
19. दि॒ग्भ्य श्च॑क्रवा॒क श्च॑क्रवा॒को दि॒ग्भ्यो दि॒ग्भ्य श्च॑क्रवा॒कः । \newline
20. दि॒ग्भ्य इति॑ दिक् - भ्यः । \newline
21. च॒क्र॒वा॒क इति॑ चक्रवा॒कः । \newline

\textbf{Ghana Paata } \newline

1. अ॒पान् नप्त्रे॒ नप्त्रे॒ ऽपा म॒पाम् नप्त्रे॑ ज॒षो ज॒षो नप्त्रे॒ ऽपा म॒पाम् नप्त्रे॑ ज॒षः । \newline
2. नप्त्रे॑ ज॒षो ज॒षो नप्त्रे॒ नप्त्रे॑ ज॒षो ना॒क्रो ना॒क्रो ज॒षो नप्त्रे॒ नप्त्रे॑ ज॒षो ना॒क्रः । \newline
3. ज॒षो ना॒क्रो ना॒क्रो ज॒षो ज॒षो ना॒क्रो मक॑रो॒ मक॑रो ना॒क्रो ज॒षो ज॒षो ना॒क्रो मक॑रः । \newline
4. ना॒क्रो मक॑रो॒ मक॑रो ना॒क्रो ना॒क्रो मक॑रः कुली॒कयः॑ कुली॒कयो॒ मक॑रो ना॒क्रो ना॒क्रो मक॑रः कुली॒कयः॑ । \newline
5. मक॑रः कुली॒कयः॑ कुली॒कयो॒ मक॑रो॒ मक॑रः कुली॒कय॒ स्ते ते कु॑ली॒कयो॒ मक॑रो॒ मक॑रः कुली॒कय॒ स्ते । \newline
6. कु॒ली॒कय॒ स्ते ते कु॑ली॒कयः॑ कुली॒कय॒ स्ते ऽकू॑पार॒स्या कू॑पारस्य॒ ते कु॑ली॒कयः॑ कुली॒कय॒ स्ते ऽकू॑पारस्य । \newline
7. ते ऽकू॑पार॒स्या कू॑पारस्य॒ ते ते ऽकू॑पारस्य वा॒चे वा॒चे ऽकू॑पारस्य॒ ते ते ऽकू॑पारस्य वा॒चे । \newline
8. अकू॑पारस्य वा॒चे वा॒चे ऽकू॑पार॒स्या कू॑पारस्य वा॒चे पै᳚ङ्गरा॒जः पै᳚ङ्गरा॒जो वा॒चे ऽकू॑पार॒स्या कू॑पारस्य वा॒चे पै᳚ङ्गरा॒जः । \newline
9. वा॒चे पै᳚ङ्गरा॒जः पै᳚ङ्गरा॒जो वा॒चे वा॒चे पै᳚ङ्गरा॒जो भगा॑य॒ भगा॑य पैङ्गरा॒जो वा॒चे वा॒चे पै᳚ङ्गरा॒जो भगा॑य । \newline
10. पै॒ङ्ग॒रा॒जो भगा॑य॒ भगा॑य पैङ्गरा॒जः पै᳚ङ्गरा॒जो भगा॑य कु॒षीत॑कः कु॒षीत॑को॒ भगा॑य पैङ्गरा॒जः पै᳚ङ्गरा॒जो भगा॑य कु॒षीत॑कः । \newline
11. पै॒ङ्ग॒रा॒ज इति॑ पैङ्ग - रा॒जः । \newline
12. भगा॑य कु॒षीत॑कः कु॒षीत॑को॒ भगा॑य॒ भगा॑य कु॒षीत॑क आ॒त्या॑ती कु॒षीत॑को॒ भगा॑य॒ भगा॑य कु॒षीत॑क आ॒ती । \newline
13. कु॒षीत॑क आ॒त्या॑ती कु॒षीत॑कः कु॒षीत॑क आ॒ती वा॑ह॒सो वा॑ह॒स आ॒ती कु॒षीत॑कः कु॒षीत॑क आ॒ती वा॑ह॒सः । \newline
14. आ॒ती वा॑ह॒सो वा॑ह॒स आ॒त्या॑ती वा॑ह॒सो दर्वि॑दा॒ दर्वि॑दा वाह॒स आ॒त्या॑ती वा॑ह॒सो दर्वि॑दा । \newline
15. वा॒ह॒सो दर्वि॑दा॒ दर्वि॑दा वाह॒सो वा॑ह॒सो दर्वि॑दा॒ ते ते दर्वि॑दा वाह॒सो वा॑ह॒सो दर्वि॑दा॒ ते । \newline
16. दर्वि॑दा॒ ते ते दर्वि॑दा॒ दर्वि॑दा॒ ते वा॑य॒व्या॑ वाय॒व्या᳚ स्ते दर्वि॑दा॒ दर्वि॑दा॒ ते वा॑य॒व्याः᳚ । \newline
17. ते वा॑य॒व्या॑ वाय॒व्या᳚ स्ते ते वा॑य॒व्या॑ दि॒ग्भ्यो दि॒ग्भ्यो वा॑य॒व्या᳚ स्ते ते वा॑य॒व्या॑ दि॒ग्भ्यः । \newline
18. वा॒य॒व्या॑ दि॒ग्भ्यो दि॒ग्भ्यो वा॑य॒व्या॑ वाय॒व्या॑ दि॒ग्भ्य श्च॑क्रवा॒क श्च॑क्रवा॒को दि॒ग्भ्यो वा॑य॒व्या॑ वाय॒व्या॑ दि॒ग्भ्य श्च॑क्रवा॒कः । \newline
19. दि॒ग्भ्य श्च॑क्रवा॒क श्च॑क्रवा॒को दि॒ग्भ्यो दि॒ग्भ्य श्च॑क्रवा॒कः । \newline
20. दि॒ग्भ्य इति॑ दिक् - भ्यः । \newline
21. च॒क्र॒वा॒क इति॑ चक्रवा॒कः । \newline
\pagebreak
\markright{ TS 5.5.14.1  \hfill https://www.vedavms.in \hfill}

\section{ TS 5.5.14.1 }

\textbf{TS 5.5.14.1 } \newline
\textbf{Samhita Paata} \newline

बला॑याजग॒र आ॒खुः सृ॑ज॒या श॒यण्ड॑क॒स्ते मै॒त्रा मृ॒त्यवे॑ऽसि॒तो म॒न्यवे᳚ स्व॒जः कुं॑भी॒नसः॑ पुष्करसा॒दो लो॑हिता॒हिस्ते त्वा॒ष्ट्राः प्र॑ति॒श्रुत्का॑यै वाह॒सः ॥ \newline

\textbf{Pada Paata} \newline

बला॑य । अ॒ज॒ग॒रः । आ॒खुः । सृ॒ज॒या । श॒यण्ड॑कः । ते । मै॒त्राः । मृ॒त्यवे᳚ । अ॒सि॒तः । म॒न्यवे᳚ । स्व॒ज इति॑ स्व - जः । कु॒भीं॒नस॒ इति॑ कुंभी - नसः॑ । पु॒ष्क॒र॒सा॒द इति॑ पुष्कर - सा॒दः । लो॒हि॒ता॒हिरिति॑ लोहित-अ॒हिः । ते । त्वा॒ष्ट्राः । प्र॒ति॒श्रुत्का॑या॒ इति॑ प्रति - श्रुत्का॑यै । वा॒ह॒सः ॥  \newline


\textbf{Krama Paata} \newline

बला॑याजग॒रः । अ॒ज॒ग॒र आ॒खुः । आ॒खुः सृ॑ज॒या । सृ॒ज॒या श॒यण्ड॑कः । श॒यण्ड॑क॒स्ते । ते मै॒त्राः । मै॒त्रा मृ॒त्यवे᳚ । मृ॒त्यवे॑ऽसि॒तः । अ॒सि॒तो म॒न्यवे᳚ । म॒न्यवे᳚ स्व॒जः । स्व॒जः कु॑म्भी॒नसः॑ । स्व॒ज इति॑ स्व - जः । कु॒म्भी॒नसः॑ पुष्करसा॒दः । कु॒म्भी॒नस॒ इति॑ कुम्भी - नसः॑ । पु॒ष्क॒र॒सा॒दो लो॑हिता॒हिः । पु॒ष्क॒र॒सा॒द इति॑ पुष्कर - सा॒दः । लो॒हि॒ता॒हिस्ते । लो॒हि॒ता॒हिरिति॑ लोहित - अ॒हिः । ते त्वा॒ष्ट्राः । त्वा॒ष्ट्राः प्र॑ति॒श्रुत्का॑यै । प्र॒ति॒श्रुत्का॑यै वाह॒सः । प्र॒ति॒श्रुत्का॑या॒ इति॑ प्रति - श्रुत्का॑यै । वा॒ह॒स इति॑ वाह॒सः । \newline

\textbf{Jatai Paata} \newline

1. बला॑या जग॒रो॑ ऽजग॒रो बला॑य॒ बला॑या जग॒रः । \newline
2. अ॒ज॒ग॒र आ॒खु रा॒खु र॑जग॒रो॑ ऽजग॒र आ॒खुः । \newline
3. आ॒खुः सृ॑ज॒या सृ॑ज॒या ऽऽखु रा॒खुः सृ॑ज॒या । \newline
4. सृ॒ज॒या श॒यण्ड॑कः श॒यण्ड॑कः सृज॒या सृ॑ज॒या श॒यण्ड॑कः । \newline
5. श॒यण्ड॑क॒ स्ते ते श॒यण्ड॑कः श॒यण्ड॑क॒ स्ते । \newline
6. ते मै॒त्रा मै॒त्रा स्ते ते मै॒त्राः । \newline
7. मै॒त्रा मृ॒त्यवे॑ मृ॒त्यवे॑ मै॒त्रा मै॒त्रा मृ॒त्यवे᳚ । \newline
8. मृ॒त्यवे॑ ऽसि॒तो॑ ऽसि॒तो मृ॒त्यवे॑ मृ॒त्यवे॑ ऽसि॒तः । \newline
9. अ॒सि॒तो म॒न्यवे॑ म॒न्यवे॑ ऽसि॒तो॑ ऽसि॒तो म॒न्यवे᳚ । \newline
10. म॒न्यवे᳚ स्व॒जः स्व॒जो म॒न्यवे॑ म॒न्यवे᳚ स्व॒जः । \newline
11. स्व॒जः कुं॑भी॒नसः॑ कुंभी॒नसः॑ स्व॒जः स्व॒जः कुं॑भी॒नसः॑ । \newline
12. स्व॒ज इति॑ स्व - जः । \newline
13. कुं॒भी॒नसः॑ पुष्करसा॒दः पु॑ष्करसा॒दः कुं॑भी॒नसः॑ कुंभी॒नसः॑ पुष्करसा॒दः । \newline
14. कुं॒भी॒नस॒ इति॑ कुंभी - नसः॑ । \newline
15. पु॒ष्क॒र॒सा॒दो लो॑हिता॒हिर् लो॑हिता॒हिः पु॑ष्करसा॒दः पु॑ष्करसा॒दो लो॑हिता॒हिः । \newline
16. पु॒ष्क॒र॒सा॒द इति॑ पुष्कर - सा॒दः । \newline
17. लो॒हि॒ता॒हि स्ते ते लो॑हिता॒हिर् लो॑हिता॒हि स्ते । \newline
18. लो॒हि॒ता॒हिरिति॑ लोहित - अ॒हिः । \newline
19. ते त्वा॒ष्ट्रा स्त्वा॒ष्ट्रा स्ते ते त्वा॒ष्ट्राः । \newline
20. त्वा॒ष्ट्राः प्र॑ति॒श्रुत्का॑यै प्रति॒श्रुत्का॑यै त्वा॒ष्ट्रा स्त्वा॒ष्ट्राः प्र॑ति॒श्रुत्का॑यै । \newline
21. प्र॒ति॒श्रुत्का॑यै वाह॒सो वा॑ह॒सः प्र॑ति॒श्रुत्का॑यै प्रति॒श्रुत्का॑यै वाह॒सः । \newline
22. प्र॒ति॒श्रुत्का॑या॒ इति॑ प्रति - श्रुत्का॑यै । \newline
23. वा॒ह॒स इति॑ वाह॒सः । \newline

\textbf{Ghana Paata } \newline

1. बला॑याजग॒रो॑ ऽजग॒रो बला॑य॒ बला॑याजग॒र आ॒खु रा॒खु र॑जग॒रो बला॑य॒ बला॑याजग॒र आ॒खुः । \newline
2. अ॒ज॒ग॒र आ॒खु रा॒खु र॑जग॒रो॑ ऽजग॒र आ॒खुः सृ॑ज॒या सृ॑ज॒या ऽऽखु र॑जग॒रो॑ ऽजग॒र आ॒खुः सृ॑ज॒या । \newline
3. आ॒खुः सृ॑ज॒या सृ॑ज॒या ऽऽखु रा॒खुः सृ॑ज॒या श॒यण्ड॑कः श॒यण्ड॑कः सृज॒या ऽऽखु रा॒खुः सृ॑ज॒या श॒यण्ड॑कः । \newline
4. सृ॒ज॒या श॒यण्ड॑कः श॒यण्ड॑कः सृज॒या सृ॑ज॒या श॒यण्ड॑क॒ स्ते ते श॒यण्ड॑कः सृज॒या सृ॑ज॒या श॒यण्ड॑क॒ स्ते । \newline
5. श॒यण्ड॑क॒ स्ते ते श॒यण्ड॑कः श॒यण्ड॑क॒ स्ते मै॒त्रा मै॒त्रा स्ते श॒यण्ड॑कः श॒यण्ड॑क॒ स्ते मै॒त्राः । \newline
6. ते मै॒त्रा मै॒त्रा स्ते ते मै॒त्रा मृ॒त्यवे॑ मृ॒त्यवे॑ मै॒त्रा स्ते ते मै॒त्रा मृ॒त्यवे᳚ । \newline
7. मै॒त्रा मृ॒त्यवे॑ मृ॒त्यवे॑ मै॒त्रा मै॒त्रा मृ॒त्यवे॑ ऽसि॒तो॑ ऽसि॒तो मृ॒त्यवे॑ मै॒त्रा मै॒त्रा मृ॒त्यवे॑ ऽसि॒तः । \newline
8. मृ॒त्यवे॑ ऽसि॒तो॑ ऽसि॒तो मृ॒त्यवे॑ मृ॒त्यवे॑ ऽसि॒तो म॒न्यवे॑ म॒न्यवे॑ ऽसि॒तो मृ॒त्यवे॑ मृ॒त्यवे॑ ऽसि॒तो म॒न्यवे᳚ । \newline
9. अ॒सि॒तो म॒न्यवे॑ म॒न्यवे॑ ऽसि॒तो॑ ऽसि॒तो म॒न्यवे᳚ स्व॒जः स्व॒जो म॒न्यवे॑ ऽसि॒तो॑ ऽसि॒तो म॒न्यवे᳚ स्व॒जः । \newline
10. म॒न्यवे᳚ स्व॒जः स्व॒जो म॒न्यवे॑ म॒न्यवे᳚ स्व॒जः कुं॑भी॒नसः॑ कुंभी॒नसः॑ स्व॒जो म॒न्यवे॑ म॒न्यवे᳚ स्व॒जः कुं॑भी॒नसः॑ । \newline
11. स्व॒जः कुं॑भी॒नसः॑ कुंभी॒नसः॑ स्व॒जः स्व॒जः कुं॑भी॒नसः॑ पुष्करसा॒दः पु॑ष्करसा॒दः कुं॑भी॒नसः॑ स्व॒जः स्व॒जः कुं॑भी॒नसः॑ पुष्करसा॒दः । \newline
12. स्व॒ज इति॑ स्व - जः । \newline
13. कुं॒भी॒नसः॑ पुष्करसा॒दः पु॑ष्करसा॒दः कुं॑भी॒नसः॑ कुंभी॒नसः॑ पुष्करसा॒दो लो॑हिता॒हिर् लो॑हिता॒हिः पु॑ष्करसा॒दः कुं॑भी॒नसः॑ कुंभी॒नसः॑ पुष्करसा॒दो लो॑हिता॒हिः । \newline
14. कुं॒भी॒नस॒ इति॑ कुंभी - नसः॑ । \newline
15. पु॒ष्क॒र॒सा॒दो लो॑हिता॒हिर् लो॑हिता॒हिः पु॑ष्करसा॒दः पु॑ष्करसा॒दो लो॑हिता॒हि स्ते ते लो॑हिता॒हिः पु॑ष्करसा॒दः पु॑ष्करसा॒दो लो॑हिता॒हि स्ते । \newline
16. पु॒ष्क॒र॒सा॒द इति॑ पुष्कर - सा॒दः । \newline
17. लो॒हि॒ता॒हि स्ते ते लो॑हिता॒हिर् लो॑हिता॒हि स्ते त्वा॒ष्ट्रा स्त्वा॒ष्ट्रा स्ते लो॑हिता॒हिर् लो॑हिता॒हि स्ते त्वा॒ष्ट्राः । \newline
18. लो॒हि॒ता॒हिरिति॑ लोहित - अ॒हिः । \newline
19. ते त्वा॒ष्ट्रा स्त्वा॒ष्ट्रा स्ते ते त्वा॒ष्ट्राः प्र॑ति॒श्रुत्का॑यै प्रति॒श्रुत्का॑यै त्वा॒ष्ट्रा स्ते ते त्वा॒ष्ट्राः प्र॑ति॒श्रुत्का॑यै । \newline
20. त्वा॒ष्ट्राः प्र॑ति॒श्रुत्का॑यै प्रति॒श्रुत्का॑यै त्वा॒ष्ट्रा स्त्वा॒ष्ट्राः प्र॑ति॒श्रुत्का॑यै वाह॒सो वा॑ह॒सः प्र॑ति॒श्रुत्का॑यै त्वा॒ष्ट्रा स्त्वा॒ष्ट्राः प्र॑ति॒श्रुत्का॑यै वाह॒सः । \newline
21. प्र॒ति॒श्रुत्का॑यै वाह॒सो वा॑ह॒सः प्र॑ति॒श्रुत्का॑यै प्रति॒श्रुत्का॑यै वाह॒सः । \newline
22. प्र॒ति॒श्रुत्का॑या॒ इति॑ प्रति - श्रुत्का॑यै । \newline
23. वा॒ह॒स इति॑ वाह॒सः । \newline
\pagebreak
\markright{ TS 5.5.15.1  \hfill https://www.vedavms.in \hfill}

\section{ TS 5.5.15.1 }

\textbf{TS 5.5.15.1 } \newline
\textbf{Samhita Paata} \newline

पु॒रु॒ष॒मृ॒गश्च॒न्द्रम॑से गो॒धा काल॑का दार्वाघा॒टस्ते वन॒स्पती॑नामे॒ण्यह्ने॒ कृष्णो॒ रात्रि॑यै पि॒कः क्ष्विङ्का॒ नील॑शीर्ष्णी॒ ते᳚ऽर्य॒म्णे धा॒तुः क॑त्क॒टः ॥ \newline

\textbf{Pada Paata} \newline

पु॒रु॒ष॒मृ॒ग इति॑ पुरुष - मृ॒गः । च॒न्द्रम॑से । गो॒धा । काल॑का । दा॒र्वा॒घा॒ट इति॑ दारु-आ॒घा॒तः । ते । वन॒स्पती॑नाम् । ए॒णी । अह्ने᳚ । कृष्णः॑ । रात्रि॑यै । पि॒कः । क्ष्विङ्काः᳚ । नील॑शी॒र्ष्णीति॒ नील॑ - शी॒र्ष्णी॒ । ते । अ॒र्य॒म्णे । धा॒तुः । क॒त्क॒टः ॥  \newline


\textbf{Krama Paata} \newline

पु॒रु॒ष॒मृ॒ग,श्च॒न्द्रम॑से । पु॒रु॒ष॒मृ॒ग इति॑ पुरुष - मृ॒गः । च॒न्द्रम॑से गो॒धा । गो॒धा काल॑का । काल॑का दार्वाघा॒टः । दा॒र्वा॒घा॒टस्ते । दा॒र्वा॒घा॒ट इति॑ दारु - आ॒घा॒तः । ते वन॒स्पती॑नाम् । वन॒स्पती॑नामे॒णी । ए॒ण्यह्ने᳚ । अह्ने॒ कृष्णः॑ । कृष्णो॒ रात्रि॑यै । रात्रि॑यै पि॒कः । पि॒कः क्ष्विङ्का᳚ । क्ष्विङ्का॒ नील॑शीर्ष्णी । नील॑शीर्ष्णी॒ ते । नील॑शी॒र्ष्णीति॒ नील॑ - शी॒र्ष्णी॒ । ते᳚ऽर्य॒म्णे । अ॒र्य॒म्णे धा॒तुः । धा॒तुः क॑त्क॒टः । क॒त्क॒ट इति॑ कत्क॒टः । \newline

\textbf{Jatai Paata} \newline

1. पु॒रु॒ष॒मृ॒ग श्च॒न्द्रम॑से च॒न्द्रम॑से पुरुषमृ॒गः पु॑रुषमृ॒ग श्च॒न्द्रम॑से । \newline
2. पु॒रु॒ष॒मृ॒ग इति॑ पुरुष - मृ॒गः । \newline
3. च॒न्द्रम॑से गो॒धा गो॒धा च॒न्द्रम॑से च॒न्द्रम॑से गो॒धा । \newline
4. गो॒धा काल॑का॒ काल॑का गो॒धा गो॒धा काल॑का । \newline
5. काल॑का दार्वाघा॒टो दा᳚र्वाघा॒टः काल॑का॒ काल॑का दार्वाघा॒टः । \newline
6. दा॒र्वा॒घा॒ट स्ते ते दा᳚र्वाघा॒टो दा᳚र्वाघा॒ट स्ते । \newline
7. दा॒र्वा॒घा॒ट इति॑ दारु - आ॒घा॒तः । \newline
8. ते वन॒स्पती॑नां॒ ॅवन॒स्पती॑ना॒म् ते ते वन॒स्पती॑नाम् । \newline
9. वन॒स्पती॑ना मे॒ण्ये॑णी वन॒स्पती॑नां॒ ॅवन॒स्पती॑ना मे॒णी । \newline
10. ए॒ण्यह्ने ऽह्न॑ ए॒ण्ये᳚ण्यह्ने᳚ । \newline
11. अह्ने॒ कृष्णः॒ कृष्णो ऽह्ने ऽह्ने॒ कृष्णः॑ । \newline
12. कृष्णो॒ रात्रि॑यै॒ रात्रि॑यै॒ कृष्णः॒ कृष्णो॒ रात्रि॑यै । \newline
13. रात्रि॑यै पि॒कः पि॒को रात्रि॑यै॒ रात्रि॑यै पि॒कः । \newline
14. पि॒कः क्ष्विङ्का॒ क्ष्विङ्का॑ पि॒कः पि॒कः क्ष्विङ्का᳚ । \newline
15. क्ष्विङ्का॒ नील॑शी॒र्ष्णी नील॑शी॒र्ष्णी क्ष्विङ्का॒ क्ष्विङ्का॒ नील॑शी॒र्ष्णी । \newline
16. नील॑शी॒र्ष्णी ते ते नील॑शी॒र्ष्णी नील॑शी॒र्ष्णी ते । \newline
17. नील॑शी॒र्ष्णीति॒ नील॑ - शी॒र्ष्णी॒ । \newline
18. ते᳚ ऽर्य॒म्णे᳚ ऽर्य॒म्णे ते ते᳚ ऽर्य॒म्णे । \newline
19. अ॒र्य॒म्णे धा॒तुर् धा॒तु र॑र्य॒म्णे᳚ ऽर्य॒म्णे धा॒तुः । \newline
20. धा॒तुः क॑त्क॒टः क॑त्क॒टो धा॒तुर् धा॒तुः क॑त्क॒टः । \newline
21. क॒त्क॒ट इति॑ कत्क॒टः । \newline

\textbf{Ghana Paata } \newline

1. पु॒रु॒ष॒मृ॒ग श्च॒न्द्रम॑से च॒न्द्रम॑से पुरुषमृ॒गः पु॑रुषमृ॒ग श्च॒न्द्रम॑से गो॒धा गो॒धा च॒न्द्रम॑से पुरुषमृ॒गः पु॑रुषमृ॒ग श्च॒न्द्रम॑से गो॒धा । \newline
2. पु॒रु॒ष॒मृ॒ग इति॑ पुरुष - मृ॒गः । \newline
3. च॒न्द्रम॑से गो॒धा गो॒धा च॒न्द्रम॑से च॒न्द्रम॑से गो॒धा काल॑का॒ काल॑का गो॒धा च॒न्द्रम॑से च॒न्द्रम॑से गो॒धा काल॑का । \newline
4. गो॒धा काल॑का॒ काल॑का गो॒धा गो॒धा काल॑का दार्वाघा॒टो दा᳚र्वाघा॒टः काल॑का गो॒धा गो॒धा काल॑का दार्वाघा॒टः । \newline
5. काल॑का दार्वाघा॒टो दा᳚र्वाघा॒टः काल॑का॒ काल॑का दार्वाघा॒ट स्ते ते दा᳚र्वाघा॒टः काल॑का॒ काल॑का दार्वाघा॒ट स्ते । \newline
6. दा॒र्वा॒घा॒ट स्ते ते दा᳚र्वाघा॒टो दा᳚र्वाघा॒ट स्ते वन॒स्पती॑नां॒ ॅवन॒स्पती॑ना॒म् ते दा᳚र्वाघा॒टो दा᳚र्वाघा॒ट स्ते वन॒स्पती॑नाम् । \newline
7. दा॒र्वा॒घा॒ट इति॑ दारु - आ॒घा॒तः । \newline
8. ते वन॒स्पती॑नां॒ ॅवन॒स्पती॑ना॒म् ते ते वन॒स्पती॑ना मे॒ण्ये॑णी वन॒स्पती॑ना॒म् ते ते वन॒स्पती॑ना मे॒णी । \newline
9. वन॒स्पती॑ना मे॒ण्ये॑णी वन॒स्पती॑नां॒ ॅवन॒स्पती॑ना मे॒ण्यह्ने ऽह्न॑ ए॒णी वन॒स्पती॑नां॒ ॅवन॒स्पती॑ना मे॒ण्यह्ने᳚ । \newline
10. ए॒ण्यह्ने ऽह्न॑ ए॒ण्ये᳚ण्यह्ने॒ कृष्णः॒ कृष्णो ऽह्न॑ ए॒ण्ये᳚ण्यह्ने॒ कृष्णः॑ । \newline
11. अह्ने॒ कृष्णः॒ कृष्णो ऽह्ने ऽह्ने॒ कृष्णो॒ रात्रि॑यै॒ रात्रि॑यै॒ कृष्णो ऽह्ने ऽह्ने॒ कृष्णो॒ रात्रि॑यै । \newline
12. कृष्णो॒ रात्रि॑यै॒ रात्रि॑यै॒ कृष्णः॒ कृष्णो॒ रात्रि॑यै पि॒कः पि॒को रात्रि॑यै॒ कृष्णः॒ कृष्णो॒ रात्रि॑यै पि॒कः । \newline
13. रात्रि॑यै पि॒कः पि॒को रात्रि॑यै॒ रात्रि॑यै पि॒कः क्ष्विङ्का॒ क्ष्विङ्का॑ पि॒को रात्रि॑यै॒ रात्रि॑यै पि॒कः क्ष्विङ्का᳚ । \newline
14. पि॒कः क्ष्विङ्का॒ क्ष्विङ्का॑ पि॒कः पि॒कः क्ष्विङ्का॒ नील॑शी॒र्ष्णी नील॑शी॒र्ष्णी क्ष्विङ्का॑ पि॒कः पि॒कः क्ष्विङ्का॒ नील॑शी॒र्ष्णी । \newline
15. क्ष्विङ्का॒ नील॑शी॒र्ष्णी नील॑शी॒र्ष्णी क्ष्विङ्का॒ क्ष्विङ्का॒ नील॑शी॒र्ष्णी ते ते नील॑शी॒र्ष्णी क्ष्विङ्का॒ क्ष्विङ्का॒ नील॑शी॒र्ष्णी ते । \newline
16. नील॑शी॒र्ष्णी ते ते नील॑शी॒र्ष्णी नील॑शी॒र्ष्णी ते᳚ ऽर्य॒म्णे᳚ ऽर्य॒म्णे ते नील॑शी॒र्ष्णी नील॑शी॒र्ष्णी ते᳚ ऽर्य॒म्णे । \newline
17. नील॑शी॒र्ष्णीति॒ नील॑ - शी॒र्ष्णी॒ । \newline
18. ते᳚ ऽर्य॒म्णे᳚ ऽर्य॒म्णे ते ते᳚ ऽर्य॒म्णे धा॒तुर् धा॒तु र॑र्य॒म्णे ते ते᳚ ऽर्य॒म्णे धा॒तुः । \newline
19. अ॒र्य॒म्णे धा॒तुर् धा॒तु र॑र्य॒म्णे᳚ ऽर्य॒म्णे धा॒तुः क॑त्क॒टः क॑त्क॒टो धा॒तु र॑र्य॒म्णे᳚ ऽर्य॒म्णे धा॒तुः क॑त्क॒टः । \newline
20. धा॒तुः क॑त्क॒टः क॑त्क॒टो धा॒तुर् धा॒तुः क॑त्क॒टः । \newline
21. क॒त्क॒ट इति॑ कत्क॒टः । \newline
\pagebreak
\markright{ TS 5.5.16.1  \hfill https://www.vedavms.in \hfill}

\section{ TS 5.5.16.1 }

\textbf{TS 5.5.16.1 } \newline
\textbf{Samhita Paata} \newline

सौ॒री ब॒लाकर्श्यो॑ म॒यूरः॑ श्ये॒नस्ते ग॑न्ध॒र्वाणां॒ ॅवसू॑नां क॒पिञ्ज॑लो रु॒द्राणां᳚ तित्ति॒री रो॒हित् कु॑ण्डृ॒णाची॑ गो॒लत्ति॑का॒ ता अ॑फ्स॒रसा॒-मर॑ण्याय सृम॒रः ॥ \newline

\textbf{Pada Paata} \newline

सौ॒री । ब॒लाका᳚ । ऋश्यः॑ । म॒यूरः॑ । श्ये॒नः । ते । ग॒न्ध॒र्वाणा᳚म् । वसू॑नाम् । क॒पिञ्ज॑लः । रु॒द्राणा᳚म् । ति॒त्ति॒रिः । रो॒हित् । कु॒ण्डृ॒णाची᳚ । गो॒लत्ति॑का । ताः । अ॒फ्स॒रसा᳚म् । अर॑ण्याय । सृ॒म॒रः ॥  \newline


\textbf{Krama Paata} \newline

सौ॒री ब॒लाका᳚ । ब॒लाकर्श्यः॑ । ऋश्यो॑ म॒यूरः॑ । म॒यूरः॑ श्ये॒नः । श्ये॒नस्ते । ते ग॑न्ध॒र्वाणा᳚म् । ग॒न्ध॒र्वाणा॒म् ॅवसू॑नाम् । वसू॑नाम् क॒पिञ्ज॑लः । क॒पिञ्ज॑लो रु॒द्राणा᳚म् । रु॒द्राणा᳚म् तित्ति॒रिः । ति॒त्ति॒री रो॒हित् । रो॒हित् कु॑ण्डृ॒णाची᳚ । कु॒ण्डृ॒णाची॑ गो॒लत्ति॑का । गो॒लत्ति॑का॒ ताः । ता अ॑फ्स॒रसा᳚म् । अ॒फ्स॒रसा॒मर॑ण्याय । अर॑ण्याय सृम॒रः । सृ॒म॒र इति॑ सृम॒रः । \newline

\textbf{Jatai Paata} \newline

1. सौ॒री ब॒लाका॑ ब॒लाका॑ सौ॒री सौ॒री ब॒लाका᳚ । \newline
2. ब॒लाक र्‌श्य॒ ऋश्यो॑ ब॒लाका॑ ब॒लाक र्‌श्यः॑ । \newline
3. ऋश्यो॑ म॒यूरो॑ म॒यूर॒ ऋश्य॒ ऋश्यो॑ म॒यूरः॑ । \newline
4. म॒यूरः॑ श्ये॒नः श्ये॒नो म॒यूरो॑ म॒यूरः॑ श्ये॒नः । \newline
5. श्ये॒न स्ते ते श्ये॒नः श्ये॒न स्ते । \newline
6. ते ग॑न्ध॒र्वाणा᳚म् गन्ध॒र्वाणा॒म् ते ते ग॑न्ध॒र्वाणा᳚म् । \newline
7. ग॒न्ध॒र्वाणां॒ ॅवसू॑नां॒ ॅवसू॑नाम् गन्ध॒र्वाणा᳚म् गन्ध॒र्वाणां॒ ॅवसू॑नाम् । \newline
8. वसू॑नाम् क॒पिञ्ज॑लः क॒पिञ्ज॑लो॒ वसू॑नां॒ ॅवसू॑नाम् क॒पिञ्ज॑लः । \newline
9. क॒पिञ्ज॑लो रु॒द्राणाꣳ॑ रु॒द्राणा᳚म् क॒पिञ्ज॑लः क॒पिञ्ज॑लो रु॒द्राणा᳚म् । \newline
10. रु॒द्राणा᳚म् तित्ति॒रि स्ति॑त्ति॒री रु॒द्राणाꣳ॑ रु॒द्राणा᳚म् तित्ति॒रिः । \newline
11. ति॒त्ति॒री रो॒हिद् रो॒हित् ति॑त्ति॒रि स्ति॑त्ति॒री रो॒हित् । \newline
12. रो॒हित् कु॑ण्डृ॒णाची॑ कुण्डृ॒णाची॑ रो॒हिद् रो॒हित् कु॑ण्डृ॒णाची᳚ । \newline
13. कु॒ण्डृ॒णाची॑ गो॒लत्ति॑का गो॒लत्ति॑का कुण्डृ॒णाची॑ कुण्डृ॒णाची॑ गो॒लत्ति॑का । \newline
14. गो॒लत्ति॑का॒ ता स्ता गो॒लत्ति॑का गो॒लत्ति॑का॒ ताः । \newline
15. ता अ॑फ्स॒रसा॑ मफ्स॒रसा॒म् ता स्ता अ॑फ्स॒रसा᳚म् । \newline
16. अ॒फ्स॒रसा॒ मर॑ण्या॒यार॑ण्याया फ्स॒रसा॑ मफ्स॒रसा॒ मर॑ण्याय । \newline
17. अर॑ण्याय सृम॒रः सृ॑म॒रो ऽर॑ण्या॒या र॑ण्याय सृम॒रः । \newline
18. सृ॒म॒र इति॑ सृम॒रः । \newline

\textbf{Ghana Paata } \newline

1. सौ॒री ब॒लाका॑ ब॒लाका॑ सौ॒री सौ॒री ब॒लाक र्‌श्य॒ ऋश्यो॑ ब॒लाका॑ सौ॒री सौ॒री ब॒लाक र्‌श्यः॑ । \newline
2. ब॒लाक र्‌श्य॒ ऋश्यो॑ ब॒लाका॑ ब॒लाक र्‌श्यो॑ म॒यूरो॑ म॒यूर॒ ऋश्यो॑ ब॒लाका॑ ब॒लाक र्‌श्यो॑ म॒यूरः॑ । \newline
3. ऋश्यो॑ म॒यूरो॑ म॒यूर॒ ऋश्य॒ ऋश्यो॑ म॒यूरः॑ श्ये॒नः श्ये॒नो म॒यूर॒ ऋश्य॒ ऋश्यो॑ म॒यूरः॑ श्ये॒नः । \newline
4. म॒यूरः॑ श्ये॒नः श्ये॒नो म॒यूरो॑ म॒यूरः॑ श्ये॒न स्ते ते श्ये॒नो म॒यूरो॑ म॒यूरः॑ श्ये॒न स्ते । \newline
5. श्ये॒न स्ते ते श्ये॒नः श्ये॒न स्ते ग॑न्ध॒र्वाणा᳚म् गन्ध॒र्वाणा॒म् ते श्ये॒नः श्ये॒न स्ते ग॑न्ध॒र्वाणा᳚म् । \newline
6. ते ग॑न्ध॒र्वाणा᳚म् गन्ध॒र्वाणा॒म् ते ते ग॑न्ध॒र्वाणां॒ ॅवसू॑नां॒ ॅवसू॑नाम् गन्ध॒र्वाणा॒म् ते ते ग॑न्ध॒र्वाणां॒ ॅवसू॑नाम् । \newline
7. ग॒न्ध॒र्वाणां॒ ॅवसू॑नां॒ ॅवसू॑नाम् गन्ध॒र्वाणा᳚म् गन्ध॒र्वाणां॒ ॅवसू॑नाम् क॒पिञ्ज॑लः क॒पिञ्ज॑लो॒ वसू॑नाम् गन्ध॒र्वाणा᳚म् गन्ध॒र्वाणां॒ ॅवसू॑नाम् क॒पिञ्ज॑लः । \newline
8. वसू॑नाम् क॒पिञ्ज॑लः क॒पिञ्ज॑लो॒ वसू॑नां॒ ॅवसू॑नाम् क॒पिञ्ज॑लो रु॒द्राणाꣳ॑ रु॒द्राणा᳚म् क॒पिञ्ज॑लो॒ वसू॑नां॒ ॅवसू॑नाम् क॒पिञ्ज॑लो रु॒द्राणा᳚म् । \newline
9. क॒पिञ्ज॑लो रु॒द्राणाꣳ॑ रु॒द्राणा᳚म् क॒पिञ्ज॑लः क॒पिञ्ज॑लो रु॒द्राणा᳚म् तित्ति॒रि स्ति॑त्ति॒री रु॒द्राणा᳚म् क॒पिञ्ज॑लः क॒पिञ्ज॑लो रु॒द्राणा᳚म् तित्ति॒रिः । \newline
10. रु॒द्राणा᳚म् तित्ति॒रि स्ति॑त्ति॒री रु॒द्राणाꣳ॑ रु॒द्राणा᳚म् तित्ति॒री रो॒हिद् रो॒हित् ति॑त्ति॒री रु॒द्राणाꣳ॑ रु॒द्राणा᳚म् तित्ति॒री रो॒हित् । \newline
11. ति॒त्ति॒री रो॒हिद् रो॒हित् ति॑त्ति॒रि स्ति॑त्ति॒री रो॒हित् कु॑ण्डृ॒णाची॑ कुण्डृ॒णाची॑ रो॒हित् ति॑त्ति॒रि स्ति॑त्ति॒री रो॒हित् कु॑ण्डृ॒णाची᳚ । \newline
12. रो॒हित् कु॑ण्डृ॒णाची॑ कुण्डृ॒णाची॑ रो॒हिद् रो॒हित् कु॑ण्डृ॒णाची॑ गो॒लत्ति॑का गो॒लत्ति॑का कुण्डृ॒णाची॑ रो॒हिद् रो॒हित् कु॑ण्डृ॒णाची॑ गो॒लत्ति॑का । \newline
13. कु॒ण्डृ॒णाची॑ गो॒लत्ति॑का गो॒लत्ति॑का कुण्डृ॒णाची॑ कुण्डृ॒णाची॑ गो॒लत्ति॑का॒ ता स्ता गो॒लत्ति॑का कुण्डृ॒णाची॑ कुण्डृ॒णाची॑ गो॒लत्ति॑का॒ ताः । \newline
14. गो॒लत्ति॑का॒ ता स्ता गो॒लत्ति॑का गो॒लत्ति॑का॒ ता अ॑फ्स॒रसा॑ मफ्स॒रसा॒म् ता गो॒लत्ति॑का गो॒लत्ति॑का॒ ता अ॑फ्स॒रसा᳚म् । \newline
15. ता अ॑फ्स॒रसा॑ मफ्स॒रसा॒म् ता स्ता अ॑फ्स॒रसा॒ मर॑ण्या॒या र॑ण्याया फ्स॒रसा॒म् ता स्ता अ॑फ्स॒रसा॒ मर॑ण्याय । \newline
16. अ॒फ्स॒रसा॒ मर॑ण्या॒या र॑ण्याया फ्स॒रसा॑ मफ्स॒रसा॒ मर॑ण्याय सृम॒रः सृ॑म॒रो ऽर॑ण्याया फ्स॒रसा॑ मफ्स॒रसा॒ मर॑ण्याय सृम॒रः । \newline
17. अर॑ण्याय सृम॒रः सृ॑म॒रो ऽर॑ण्या॒या र॑ण्याय सृम॒रः । \newline
18. सृ॒म॒र इति॑ सृम॒रः । \newline
\pagebreak
\markright{ TS 5.5.17.1  \hfill https://www.vedavms.in \hfill}

\section{ TS 5.5.17.1 }

\textbf{TS 5.5.17.1 } \newline
\textbf{Samhita Paata} \newline

पृ॒ष॒तो वै᳚श्वदे॒वः पि॒त्वो न्यङ्कुः॒ कश॒स्तेऽनु॑मत्या अन्यवा॒पो᳚ऽर्द्धमा॒सानां᳚ मा॒सां क॒श्यपः॒ क्वयिः॑ कु॒टरु॑र्दात्यौ॒हस्ते सि॑नीवा॒ल्यै बृह॒स्पत॑ये शित्पु॒टः ॥ \newline

\textbf{Pada Paata} \newline

पृ॒ष॒तः । वै॒श्व॒दे॒व इति॑ वैश्व - दे॒वः । पि॒त्वः । न्यङ्कुः॑ । कशः॑ । ते । अनु॑मत्या॒ इत्यनु॑ - म॒त्यै॒ । अ॒न्य॒वा॒प इत्य॑न्य - वा॒पः । अ॒द्‌र्ध॒मा॒साना॒मित्य॑द्‌र्ध - मा॒साना᳚म् । मा॒साम् । क॒श्यपः॑ । क्वयिः॑ । कु॒टरुः॑ । दा॒त्यौ॒हः । ते । सि॒नी॒वा॒ल्यै । बृह॒स्पत॑ये । शि॒त्पु॒टः ॥  \newline


\textbf{Krama Paata} \newline

पृ॒ष॒तो वै᳚श्वदे॒वः । वै॒श्व॒दे॒वः पि॒त्वः । वै॒श्व॒दे॒व इति॑ वैश्व - दे॒वः । पि॒त्वो न्यङ्कुः॑ । न्यङ्कुः॒ कशः॑ । कश॒स्ते । तेऽनु॑मत्यै । अनु॑मत्या अन्यवा॒पः । अनु॑मत्या॒ इत्यनु॑ - म॒त्यै॒ । अ॒न्य॒वा॒पो᳚ऽर्द्धमा॒साना᳚म् । अ॒न्य॒वा॒प इत्य॑न्य - वा॒पः । अ॒र्द्ध॒मा॒साना᳚म् मा॒साम् । अ॒र्द्ध॒मा॒साना॒मित्य॑र्द्ध - मा॒साना᳚म् । मा॒साम् क॒श्यपः॑ । क॒श्यपः॒ क्वयिः॑ । क्वयिः॑ कु॒टरुः॑ । कु॒टरु॑र् दात्यौ॒हः । दा॒त्यौ॒हस्ते । ते सि॑नीवा॒ल्यै । सि॒नी॒वा॒ल्यै बृह॒स्पत॑ये । बृह॒स्पत॑ये शित्पु॒टः । शि॒त्पु॒ट इति॑ शित्पु॒टः । \newline

\textbf{Jatai Paata} \newline

1. पृ॒ष॒तो वै᳚श्वदे॒वो वै᳚श्वदे॒वः पृ॑ष॒तः पृ॑ष॒तो वै᳚श्वदे॒वः । \newline
2. वै॒श्व॒दे॒वः पि॒त्वः पि॒त्वो वै᳚श्वदे॒वो वै᳚श्वदे॒वः पि॒त्वः । \newline
3. वै॒श्व॒दे॒व इति॑ वैश्व - दे॒वः । \newline
4. पि॒त्वो न्यङ्कु॒र् न्यङ्कुः॑ पि॒त्वः पि॒त्वो न्यङ्कुः॑ । \newline
5. न्यङ्कुः॒ कशः॒ कशो॒ न्यङ्कु॒र् न्यङ्कुः॒ कशः॑ । \newline
6. कश॒ स्ते ते कशः॒ कश॒ स्ते । \newline
7. ते ऽनु॑मत्या॒ अनु॑मत्यै॒ ते ते ऽनु॑मत्यै । \newline
8. अनु॑मत्या अन्यवा॒पो᳚ ऽन्यवा॒पो ऽनु॑मत्या॒ अनु॑मत्या अन्यवा॒पः । \newline
9. अनु॑मत्या॒ इत्यनु॑ - म॒त्यै॒ । \newline
10. अ॒न्य॒वा॒पो᳚ ऽर्द्धमा॒साना॑ मर्द्धमा॒साना॑ मन्यवा॒पो᳚ ऽन्यवा॒पो᳚ ऽर्द्धमा॒साना᳚म् । \newline
11. अ॒न्य॒वा॒प इत्य॑न्य - वा॒पः । \newline
12. अ॒र्द्ध॒मा॒साना᳚म् मा॒साम् मा॒सा म॑र्द्धमा॒साना॑ मर्द्धमा॒साना᳚म् मा॒साम् । \newline
13. अ॒र्द्ध॒मा॒साना॒मित्य॑र्द्ध - मा॒साना᳚म् । \newline
14. मा॒साम् क॒श्यपः॑ क॒श्यपो॑ मा॒साम् मा॒साम् क॒श्यपः॑ । \newline
15. क॒श्यपः॒ क्वयिः॒ क्वयिः॑ क॒श्यपः॑ क॒श्यपः॒ क्वयिः॑ । \newline
16. क्वयिः॑ कु॒टरुः॑ कु॒टरुः॒ क्वयिः॒ क्वयिः॑ कु॒टरुः॑ । \newline
17. कु॒टरु॑र् दात्यौ॒हो दा᳚त्यौ॒हः कु॒टरुः॑ कु॒टरु॑र् दात्यौ॒हः । \newline
18. दा॒त्यौ॒ह स्ते ते दा᳚त्यौ॒हो दा᳚त्यौ॒ह स्ते । \newline
19. ते सि॑नीवा॒ल्यै सि॑नीवा॒ल्यै ते ते सि॑नीवा॒ल्यै । \newline
20. सि॒नी॒वा॒ल्यै बृह॒स्पत॑ये॒ बृह॒स्पत॑ये सिनीवा॒ल्यै सि॑नीवा॒ल्यै बृह॒स्पत॑ये । \newline
21. बृह॒स्पत॑ये शित्पु॒टः शि॑त्पु॒टो बृह॒स्पत॑ये॒ बृह॒स्पत॑ये शित्पु॒टः । \newline
22. शि॒त्पु॒ट इति॑ शित्पु॒टः । \newline

\textbf{Ghana Paata } \newline

1. पृ॒ष॒तो वै᳚श्वदे॒वो वै᳚श्वदे॒वः पृ॑ष॒तः पृ॑ष॒तो वै᳚श्वदे॒वः पि॒त्वः पि॒त्वो वै᳚श्वदे॒वः पृ॑ष॒तः पृ॑ष॒तो वै᳚श्वदे॒वः पि॒त्वः । \newline
2. वै॒श्व॒दे॒वः पि॒त्वः पि॒त्वो वै᳚श्वदे॒वो वै᳚श्वदे॒वः पि॒त्वो न्यङ्कु॒र् न्यङ्कुः॑ पि॒त्वो वै᳚श्वदे॒वो वै᳚श्वदे॒वः पि॒त्वो न्यङ्कुः॑ । \newline
3. वै॒श्व॒दे॒व इति॑ वैश्व - दे॒वः । \newline
4. पि॒त्वो न्यङ्कु॒र् न्यङ्कुः॑ पि॒त्वः पि॒त्वो न्यङ्कुः॒ कशः॒ कशो॒ न्यङ्कुः॑ पि॒त्वः पि॒त्वो न्यङ्कुः॒ कशः॑ । \newline
5. न्यङ्कुः॒ कशः॒ कशो॒ न्यङ्कु॒र् न्यङ्कुः॒ कश॒ स्ते ते कशो॒ न्यङ्कु॒र् न्यङ्कुः॒ कश॒ स्ते । \newline
6. कश॒ स्ते ते कशः॒ कश॒ स्ते ऽनु॑मत्या॒ अनु॑मत्यै॒ ते कशः॒ कश॒ स्ते ऽनु॑मत्यै । \newline
7. ते ऽनु॑मत्या॒ अनु॑मत्यै॒ ते ते ऽनु॑मत्या अन्यवा॒पो᳚ ऽन्यवा॒पो ऽनु॑मत्यै॒ ते ते ऽनु॑मत्या अन्यवा॒पः । \newline
8. अनु॑मत्या अन्यवा॒पो᳚ ऽन्यवा॒पो ऽनु॑मत्या॒ अनु॑मत्या अन्यवा॒पो᳚ ऽर्द्धमा॒साना॑ मर्द्धमा॒साना॑ मन्यवा॒पो ऽनु॑मत्या॒ अनु॑मत्या अन्यवा॒पो᳚ ऽर्द्धमा॒साना᳚म् । \newline
9. अनु॑मत्या॒ इत्यनु॑ - म॒त्यै॒ । \newline
10. अ॒न्य॒वा॒पो᳚ ऽर्द्धमा॒साना॑ मर्द्धमा॒साना॑ मन्यवा॒पो᳚ ऽन्यवा॒पो᳚ ऽर्द्धमा॒साना᳚म् मा॒साम् मा॒सा म॑र्द्धमा॒साना॑ मन्यवा॒पो᳚ ऽन्यवा॒पो᳚ ऽर्द्धमा॒साना᳚म् मा॒साम् । \newline
11. अ॒न्य॒वा॒प इत्य॑न्य - वा॒पः । \newline
12. अ॒र्द्ध॒मा॒साना᳚म् मा॒साम् मा॒सा म॑र्द्धमा॒साना॑ मर्द्धमा॒साना᳚म् मा॒साम् क॒श्यपः॑ क॒श्यपो॑ मा॒सा म॑र्द्धमा॒साना॑ मर्द्धमा॒साना᳚म् मा॒साम् क॒श्यपः॑ । \newline
13. अ॒र्द्ध॒मा॒साना॒मित्य॑र्द्ध - मा॒साना᳚म् । \newline
14. मा॒साम् क॒श्यपः॑ क॒श्यपो॑ मा॒साम् मा॒साम् क॒श्यपः॒ क्वयिः॒ क्वयिः॑ क॒श्यपो॑ मा॒साम् मा॒साम् क॒श्यपः॒ क्वयिः॑ । \newline
15. क॒श्यपः॒ क्वयिः॒ क्वयिः॑ क॒श्यपः॑ क॒श्यपः॒ क्वयिः॑ कु॒टरुः॑ कु॒टरुः॒ क्वयिः॑ क॒श्यपः॑ क॒श्यपः॒ क्वयिः॑ कु॒टरुः॑ । \newline
16. क्वयिः॑ कु॒टरुः॑ कु॒टरुः॒ क्वयिः॒ क्वयिः॑ कु॒टरु॑र् दात्यौ॒हो दा᳚त्यौ॒हः कु॒टरुः॒ क्वयिः॒ क्वयिः॑ कु॒टरु॑र् दात्यौ॒हः । \newline
17. कु॒टरु॑र् दात्यौ॒हो दा᳚त्यौ॒हः कु॒टरुः॑ कु॒टरु॑र् दात्यौ॒ह स्ते ते दा᳚त्यौ॒हः कु॒टरुः॑ कु॒टरु॑र् दात्यौ॒ह स्ते । \newline
18. दा॒त्यौ॒ह स्ते ते दा᳚त्यौ॒हो दा᳚त्यौ॒ह स्ते सि॑नीवा॒ल्यै सि॑नीवा॒ल्यै ते दा᳚त्यौ॒हो दा᳚त्यौ॒ह स्ते सि॑नीवा॒ल्यै । \newline
19. ते सि॑नीवा॒ल्यै सि॑नीवा॒ल्यै ते ते सि॑नीवा॒ल्यै बृह॒स्पत॑ये॒ बृह॒स्पत॑ये सिनीवा॒ल्यै ते ते सि॑नीवा॒ल्यै बृह॒स्पत॑ये । \newline
20. सि॒नी॒वा॒ल्यै बृह॒स्पत॑ये॒ बृह॒स्पत॑ये सिनीवा॒ल्यै सि॑नीवा॒ल्यै बृह॒स्पत॑ये शित्पु॒टः शि॑त्पु॒टो बृह॒स्पत॑ये सिनीवा॒ल्यै सि॑नीवा॒ल्यै बृह॒स्पत॑ये शित्पु॒टः । \newline
21. बृह॒स्पत॑ये शित्पु॒टः शि॑त्पु॒टो बृह॒स्पत॑ये॒ बृह॒स्पत॑ये शित्पु॒टः । \newline
22. शि॒त्पु॒ट इति॑ शित्पु॒टः । \newline
\pagebreak
\markright{ TS 5.5.18.1  \hfill https://www.vedavms.in \hfill}

\section{ TS 5.5.18.1 }

\textbf{TS 5.5.18.1 } \newline
\textbf{Samhita Paata} \newline

शका॑ भौ॒मी पा॒न्त्रः कशो॑ मान्थी॒लव॒स्ते पि॑तृ॒णामृ॑तू॒नां जह॑का संॅवथ्स॒राय॒ लोपा॑ क॒पोत॒ उलू॑कः श॒शस्ते न॑र्.ऋ॒ताः कृ॑क॒वाकुः॑ सावि॒त्रः ॥ \newline

\textbf{Pada Paata} \newline

शका᳚ । भौ॒मी । पा॒न्त्रः । कशः॑ । मा॒न्थी॒लवः॑ । ते । पि॒तृ॒णाम् । ऋ॒तू॒नाम् । जह॑का । सं॒ॅव॒थ्स॒रायेति॑ सं-व॒थ्स॒राय॑ । लोपा᳚ । क॒पोतः॑ । उलू॑कः । श॒शः । ते । नै॒र्.॒ऋ॒ता इति॑ नैः - ऋ॒ताः । कृ॒क॒वाकुः॑ । सा॒वि॒त्रः ॥  \newline


\textbf{Krama Paata} \newline

शका॑ भौ॒मी । भौ॒मी पा॒न्त्रः । पा॒न्त्रः कशः॑ । कशो॑ मान्थी॒लवः॑ । मा॒न्थी॒लव॒स्ते । ते पि॑तृ॒णाम् । पि॒तृ॒णामृ॑तू॒नाम् । ऋ॒तू॒नाम् जह॑का । जह॑का सम्ॅवथ्स॒राय॑ । स॒म्ॅव॒थ्स॒राय॒ लोपा᳚ । स॒म्ॅव॒थ्स॒रायेति॑ सम् - व॒थ्स॒राय॑ । लोपा॑ क॒पोतः॑ । क॒पोत॒ उलू॑कः । उलू॑कः श॒शः । श॒शस्ते । ते नैर्.॑ऋ॒ताः । नै॒र्॒.ऋ॒ताः कृ॑क॒वाकुः॑ । नै॒र्॒.ऋ॒ता इति॑ नैः - ऋ॒ताः । कृ॒क॒वाकुः॑ सावि॒त्रः । सा॒वि॒त्र इति॑ सावि॒त्रः । \newline

\textbf{Jatai Paata} \newline

1. शका॑ भौ॒मी भौ॒मी शका॒ शका॑ भौ॒मी । \newline
2. भौ॒मी पा॒न्त्रः पा॒न्त्रो भौ॒मी भौ॒मी पा॒न्त्रः । \newline
3. पा॒न्त्रः कशः॒ कशः॑ पा॒न्त्रः पा॒न्त्रः कशः॑ । \newline
4. कशो॑ मान्थी॒लवो॑ मान्थी॒लवः॒ कशः॒ कशो॑ मान्थी॒लवः॑ । \newline
5. मा॒न्थी॒लव॒ स्ते ते मा᳚न्थी॒लवो॑ मान्थी॒लव॒ स्ते । \newline
6. ते पि॑तृ॒णाम् पि॑तृ॒णाम् ते ते पि॑तृ॒णाम् । \newline
7. पि॒तृ॒णा मृ॑तू॒ना मृ॑तू॒नाम् पि॑तृ॒णाम् पि॑तृ॒णा मृ॑तू॒नाम् । \newline
8. ऋ॒तू॒नाम् जह॑का॒ जह॑क र्‌तू॒ना मृ॑तू॒नाम् जह॑का । \newline
9. जह॑का संॅवथ्स॒राय॑ संॅवथ्स॒राय॒ जह॑का॒ जह॑का संॅवथ्स॒राय॑ । \newline
10. सं॒ॅव॒थ्स॒राय॒ लोपा॒ लोपा॑ संॅवथ्स॒राय॑ संॅवथ्स॒राय॒ लोपा᳚ । \newline
11. सं॒ॅव॒थ्स॒रायेति॑ सं - व॒थ्स॒राय॑ । \newline
12. लोपा॑ क॒पोतः॑ क॒पोतो॒ लोपा॒ लोपा॑ क॒पोतः॑ । \newline
13. क॒पोत॒ उलू॑क॒ उलू॑कः क॒पोतः॑ क॒पोत॒ उलू॑कः । \newline
14. उलू॑कः श॒शः श॒श उलू॑क॒ उलू॑कः श॒शः । \newline
15. श॒श स्ते ते श॒शः श॒श स्ते । \newline
16. ते नैर्॑.ऋ॒ता नैर्॑.ऋ॒ता स्ते ते नैर्॑.ऋ॒ताः । \newline
17. नै॒र्॒.ऋ॒ताः कृ॑क॒वाकुः॑ कृक॒वाकु॑र् नैर्.ऋ॒ता नैर्॑.ऋ॒ताः कृ॑क॒वाकुः॑ । \newline
18. नै॒र्.॒ऋ॒ता इति॑ नैः - ऋ॒ताः । \newline
19. कृ॒क॒वाकुः॑ सावि॒त्रः सा॑वि॒त्रः कृ॑क॒वाकुः॑ कृक॒वाकुः॑ सावि॒त्रः । \newline
20. सा॒वि॒त्र इति॑ सावि॒त्रः । \newline

\textbf{Ghana Paata } \newline

1. शका॑ भौ॒मी भौ॒मी शका॒ शका॑ भौ॒मी पा॒न्त्रः पा॒न्त्रो भौ॒मी शका॒ शका॑ भौ॒मी पा॒न्त्रः । \newline
2. भौ॒मी पा॒न्त्रः पा॒न्त्रो भौ॒मी भौ॒मी पा॒न्त्रः कशः॒ कशः॑ पा॒न्त्रो भौ॒मी भौ॒मी पा॒न्त्रः कशः॑ । \newline
3. पा॒न्त्रः कशः॒ कशः॑ पा॒न्त्रः पा॒न्त्रः कशो॑ मान्थी॒लवो॑ मान्थी॒लवः॒ कशः॑ पा॒न्त्रः पा॒न्त्रः कशो॑ मान्थी॒लवः॑ । \newline
4. कशो॑ मान्थी॒लवो॑ मान्थी॒लवः॒ कशः॒ कशो॑ मान्थी॒लव॒ स्ते ते मा᳚न्थी॒लवः॒ कशः॒ कशो॑ मान्थी॒लव॒ स्ते । \newline
5. मा॒न्थी॒लव॒ स्ते ते मा᳚न्थी॒लवो॑ मान्थी॒लव॒ स्ते पि॑तृ॒णाम् पि॑तृ॒णाम् ते मा᳚न्थी॒लवो॑ मान्थी॒लव॒ स्ते पि॑तृ॒णाम् । \newline
6. ते पि॑तृ॒णाम् पि॑तृ॒णाम् ते ते पि॑तृ॒णा मृ॑तू॒ना मृ॑तू॒नाम् पि॑तृ॒णाम् ते ते पि॑तृ॒णा मृ॑तू॒नाम् । \newline
7. पि॒तृ॒णा मृ॑तू॒ना मृ॑तू॒नाम् पि॑तृ॒णाम् पि॑तृ॒णा मृ॑तू॒नाम् जह॑का॒ जह॑क र्‌तू॒नाम् पि॑तृ॒णाम् पि॑तृ॒णा मृ॑तू॒नाम् जह॑का । \newline
8. ऋ॒तू॒नाम् जह॑का॒ जह॑क र्‌तू॒ना मृ॑तू॒नाम् जह॑का संॅवथ्स॒राय॑ संॅवथ्स॒राय॒ जह॑क र्‌तू॒ना मृ॑तू॒नाम् जह॑का संॅवथ्स॒राय॑ । \newline
9. जह॑का संॅवथ्स॒राय॑ संॅवथ्स॒राय॒ जह॑का॒ जह॑का संॅवथ्स॒राय॒ लोपा॒ लोपा॑ संॅवथ्स॒राय॒ जह॑का॒ जह॑का संॅवथ्स॒राय॒ लोपा᳚ । \newline
10. सं॒ॅव॒थ्स॒राय॒ लोपा॒ लोपा॑ संॅवथ्स॒राय॑ संॅवथ्स॒राय॒ लोपा॑ क॒पोतः॑ क॒पोतो॒ लोपा॑ संॅवथ्स॒राय॑ संॅवथ्स॒राय॒ लोपा॑ क॒पोतः॑ । \newline
11. सं॒ॅव॒थ्स॒रायेति॑ सं - व॒थ्स॒राय॑ । \newline
12. लोपा॑ क॒पोतः॑ क॒पोतो॒ लोपा॒ लोपा॑ क॒पोत॒ उलू॑क॒ उलू॑कः क॒पोतो॒ लोपा॒ लोपा॑ क॒पोत॒ उलू॑कः । \newline
13. क॒पोत॒ उलू॑क॒ उलू॑कः क॒पोतः॑ क॒पोत॒ उलू॑कः श॒शः श॒श उलू॑कः क॒पोतः॑ क॒पोत॒ उलू॑कः श॒शः । \newline
14. उलू॑कः श॒शः श॒श उलू॑क॒ उलू॑कः श॒श स्ते ते श॒श उलू॑क॒ उलू॑कः श॒श स्ते । \newline
15. श॒श स्ते ते श॒शः श॒श स्ते नैर्॑.ऋ॒ता नैर्॑.ऋ॒ता स्ते श॒शः श॒श स्ते नैर्॑.ऋ॒ताः । \newline
16. ते नैर्॑.ऋ॒ता नैर्॑.ऋ॒ता स्ते ते नैर्॑.ऋ॒ताः कृ॑क॒वाकुः॑ कृक॒वाकु॑र् नैर्.ऋ॒ता स्ते ते नैर्॑.ऋ॒ताः कृ॑क॒वाकुः॑ । \newline
17. नै॒र्॒.ऋ॒ताः कृ॑क॒वाकुः॑ कृक॒वाकु॑र् नैर्.ऋ॒ता नैर्॑.ऋ॒ताः कृ॑क॒वाकुः॑ सावि॒त्रः सा॑वि॒त्रः कृ॑क॒वाकु॑र् नैर्.ऋ॒ता नैर्॑.ऋ॒ताः कृ॑क॒वाकुः॑ सावि॒त्रः । \newline
18. नै॒र्.॒ऋ॒ता इति॑ नैः - ऋ॒ताः । \newline
19. कृ॒क॒वाकुः॑ सावि॒त्रः सा॑वि॒त्रः कृ॑क॒वाकुः॑ कृक॒वाकुः॑ सावि॒त्रः । \newline
20. सा॒वि॒त्र इति॑ सावि॒त्रः । \newline
\pagebreak
\markright{ TS 5.5.19.1  \hfill https://www.vedavms.in \hfill}

\section{ TS 5.5.19.1 }

\textbf{TS 5.5.19.1 } \newline
\textbf{Samhita Paata} \newline

रुरू॑ रौ॒द्रः कृ॑कला॒सः श॒कुनिः॒ पिप्प॑का॒ ते श॑र॒व्या॑यै हरि॒णो मा॑रु॒तो ब्रह्म॑णे शा॒र्गस्त॒रक्षुः॑ कृ॒ष्णः श्वा च॑तुर॒क्षो ग॑र्द॒भस्त इ॑तरज॒नाना॑म॒ग्नये॒ धूङ्क्ष्णा᳚ ॥ \newline

\textbf{Pada Paata} \newline

रुरुः॑ । रौ॒द्रः । कृ॒क॒ला॒सः । श॒कुनिः॑ । पिप्प॑का । ते । श॒र॒व्या॑यै । ह॒रि॒णः । मा॒रु॒तः । ब्रह्म॑णे । शा॒र्गः । त॒रक्षुः॑ । कृ॒ष्णः । श्वा । च॒तु॒र॒क्ष इति॑ चतुः - अ॒क्षः । ग॒र्द॒भः । ते । इ॒त॒र॒ज॒नाना॒मिती॑तर-ज॒नाना᳚म् । अ॒ग्नये᳚ । धूंक्ष्णा᳚ ॥  \newline


\textbf{Krama Paata} \newline

रुरू॑ रौ॒द्रः । रौ॒द्रः कृ॑कला॒सः । कृ॒क॒ला॒सः श॒कुनिः॑ । श॒कुनिः॒ पिप्प॑का । पिप्प॑का॒ ते । ते श॑र॒व्या॑यै । श॒र॒व्या॑यै हरि॒णः । ह॒रि॒णो मा॑रु॒तः । मा॒रु॒तो ब्रह्म॑णे । ब्रह्म॑णे शा॒र्गः । शा॒र्गस्त॒रक्षुः॑ । त॒रक्षुः॑ कृ॒ष्णः । कृ॒ष्णः श्वा । श्वा च॑तुर॒क्षः । च॒तु॒र॒क्षो ग॑र्द॒भः । च॒तु॒र॒क्ष इति॑ चतुः - अ॒क्षः । ग॒र्द॒भस्ते । त इ॑तरज॒नाना᳚म् । इ॒त॒र॒ज॒नाना॑म॒ग्नये᳚ । इ॒त॒र॒ज॒नाना॒मिती॑तर - ज॒नाना᳚म् । अ॒ग्नये॒ धूङ्क्ष्णा᳚ । धूङ्क्ष्णेति॒ धूङ्क्ष्णा᳚ । \newline

\textbf{Jatai Paata} \newline

1. रुरू॑ रौ॒द्रो रौ॒द्रो रुरू॒ रुरू॑ रौ॒द्रः । \newline
2. रौ॒द्रः कृ॑कला॒सः कृ॑कला॒सो रौ॒द्रो रौ॒द्रः कृ॑कला॒सः । \newline
3. कृ॒क॒ला॒सः श॒कुनिः॑ श॒कुनिः॑ कृकला॒सः कृ॑कला॒सः श॒कुनिः॑ । \newline
4. श॒कुनिः॒ पिप्प॑का॒ पिप्प॑का श॒कुनिः॑ श॒कुनिः॒ पिप्प॑का । \newline
5. पिप्प॑का॒ ते ते पिप्प॑का॒ पिप्प॑का॒ ते । \newline
6. ते श॑र॒व्या॑यै शर॒व्या॑यै॒ ते ते श॑र॒व्या॑यै । \newline
7. श॒र॒व्या॑यै हरि॒णो ह॑रि॒णः श॑र॒व्या॑यै शर॒व्या॑यै हरि॒णः । \newline
8. ह॒रि॒णो मा॑रु॒तो मा॑रु॒तो ह॑रि॒णो ह॑रि॒णो मा॑रु॒तः । \newline
9. मा॒रु॒तो ब्रह्म॑णे॒ ब्रह्म॑णे मारु॒तो मा॑रु॒तो ब्रह्म॑णे । \newline
10. ब्रह्म॑णे शा॒र्गः शा॒र्गो ब्रह्म॑णे॒ ब्रह्म॑णे शा॒र्गः । \newline
11. शा॒र्ग स्त॒रक्षु॑ स्त॒रक्षुः॑ शा॒र्गः शा॒र्ग स्त॒रक्षुः॑ । \newline
12. त॒रक्षुः॑ कृ॒ष्णः कृ॒ष्ण स्त॒रक्षु॑ स्त॒रक्षुः॑ कृ॒ष्णः । \newline
13. कृ॒ष्णः श्वा श्वा कृ॒ष्णः कृ॒ष्णः श्वा । \newline
14. श्वा च॑तुर॒क्ष श्च॑तुर॒क्षः श्वा श्वा च॑तुर॒क्षः । \newline
15. च॒तु॒र॒क्षो ग॑र्द॒भो ग॑र्द॒भ श्च॑तुर॒क्ष श्च॑तुर॒क्षो ग॑र्द॒भः । \newline
16. च॒तु॒र॒क्ष इति॑ चतुः - अ॒क्षः । \newline
17. ग॒र्द॒भ स्ते ते ग॑र्द॒भो ग॑र्द॒भ स्ते । \newline
18. त इ॑तरज॒नाना॑ मितरज॒नाना॒म् ते त इ॑तरज॒नाना᳚म् । \newline
19. इ॒त॒र॒ज॒नाना॑ म॒ग्नये॒ ऽग्नय॑ इतरज॒नाना॑ मितरज॒नाना॑ म॒ग्नये᳚ । \newline
20. इ॒त॒र॒ज॒नाना॒मिती॑तर - ज॒नाना᳚म् । \newline
21. अ॒ग्नये॒ धूङ्क्ष्णा॒ धूङ्क्ष्णा॒ ऽग्नये॒ ऽग्नये॒ धूङ्क्ष्णा᳚ । \newline
22. धूङ्‍क्ष्णेति॒ धूङ्‍क्ष्णा᳚ । \newline

\textbf{Ghana Paata } \newline

1. रुरू॑ रौ॒द्रो रौ॒द्रो रुरू॒ रुरू॑ रौ॒द्रः कृ॑कला॒सः कृ॑कला॒सो रौ॒द्रो रुरू॒ रुरू॑ रौ॒द्रः कृ॑कला॒सः । \newline
2. रौ॒द्रः कृ॑कला॒सः कृ॑कला॒सो रौ॒द्रो रौ॒द्रः कृ॑कला॒सः श॒कुनिः॑ श॒कुनिः॑ कृकला॒सो रौ॒द्रो रौ॒द्रः कृ॑कला॒सः श॒कुनिः॑ । \newline
3. कृ॒क॒ला॒सः श॒कुनिः॑ श॒कुनिः॑ कृकला॒सः कृ॑कला॒सः श॒कुनिः॒ पिप्प॑का॒ पिप्प॑का श॒कुनिः॑ कृकला॒सः कृ॑कला॒सः श॒कुनिः॒ पिप्प॑का । \newline
4. श॒कुनिः॒ पिप्प॑का॒ पिप्प॑का श॒कुनिः॑ श॒कुनिः॒ पिप्प॑का॒ ते ते पिप्प॑का श॒कुनिः॑ श॒कुनिः॒ पिप्प॑का॒ ते । \newline
5. पिप्प॑का॒ ते ते पिप्प॑का॒ पिप्प॑का॒ ते श॑र॒व्या॑यै शर॒व्या॑यै॒ ते पिप्प॑का॒ पिप्प॑का॒ ते श॑र॒व्या॑यै । \newline
6. ते श॑र॒व्या॑यै शर॒व्या॑यै॒ ते ते श॑र॒व्या॑यै हरि॒णो ह॑रि॒णः श॑र॒व्या॑यै॒ ते ते श॑र॒व्या॑यै हरि॒णः । \newline
7. श॒र॒व्या॑यै हरि॒णो ह॑रि॒णः श॑र॒व्या॑यै शर॒व्या॑यै हरि॒णो मा॑रु॒तो मा॑रु॒तो ह॑रि॒णः श॑र॒व्या॑यै शर॒व्या॑यै हरि॒णो मा॑रु॒तः । \newline
8. ह॒रि॒णो मा॑रु॒तो मा॑रु॒तो ह॑रि॒णो ह॑रि॒णो मा॑रु॒तो ब्रह्म॑णे॒ ब्रह्म॑णे मारु॒तो ह॑रि॒णो ह॑रि॒णो मा॑रु॒तो ब्रह्म॑णे । \newline
9. मा॒रु॒तो ब्रह्म॑णे॒ ब्रह्म॑णे मारु॒तो मा॑रु॒तो ब्रह्म॑णे शा॒र्गः शा॒र्गो ब्रह्म॑णे मारु॒तो मा॑रु॒तो ब्रह्म॑णे शा॒र्गः । \newline
10. ब्रह्म॑णे शा॒र्गः शा॒र्गो ब्रह्म॑णे॒ ब्रह्म॑णे शा॒र्ग स्त॒रक्षु॑ स्त॒रक्षुः॑ शा॒र्गो ब्रह्म॑णे॒ ब्रह्म॑णे शा॒र्ग स्त॒रक्षुः॑ । \newline
11. शा॒र्ग स्त॒रक्षु॑ स्त॒रक्षुः॑ शा॒र्गः शा॒र्ग स्त॒रक्षुः॑ कृ॒ष्णः कृ॒ष्ण स्त॒रक्षुः॑ शा॒र्गः शा॒र्ग स्त॒रक्षुः॑ कृ॒ष्णः । \newline
12. त॒रक्षुः॑ कृ॒ष्णः कृ॒ष्ण स्त॒रक्षु॑ स्त॒रक्षुः॑ कृ॒ष्णः श्वा श्वा कृ॒ष्ण स्त॒रक्षु॑ स्त॒रक्षुः॑ कृ॒ष्णः श्वा । \newline
13. कृ॒ष्णः श्वा श्वा कृ॒ष्णः कृ॒ष्णः श्वा च॑तुर॒क्ष श्च॑तुर॒क्षः श्वा कृ॒ष्णः कृ॒ष्णः श्वा च॑तुर॒क्षः । \newline
14. श्वा च॑तुर॒क्ष श्च॑तुर॒क्षः श्वा श्वा च॑तुर॒क्षो ग॑र्द॒भो ग॑र्द॒भ श्च॑तुर॒क्षः श्वा श्वा च॑तुर॒क्षो ग॑र्द॒भः । \newline
15. च॒तु॒र॒क्षो ग॑र्द॒भो ग॑र्द॒भ श्च॑तुर॒क्ष श्च॑तुर॒क्षो ग॑र्द॒भ स्ते ते ग॑र्द॒भ श्च॑तुर॒क्ष श्च॑तुर॒क्षो ग॑र्द॒भ स्ते । \newline
16. च॒तु॒र॒क्ष इति॑ चतुः - अ॒क्षः । \newline
17. ग॒र्द॒भ स्ते ते ग॑र्द॒भो ग॑र्द॒भ स्त इ॑तरज॒नाना॑ मितरज॒नाना॒म् ते ग॑र्द॒भो ग॑र्द॒भ स्त इ॑तरज॒नाना᳚म् । \newline
18. त इ॑तरज॒नाना॑ मितरज॒नाना॒म् ते त इ॑तरज॒नाना॑ म॒ग्नये॒ ऽग्नय॑ इतरज॒नाना॒म् ते त इ॑तरज॒नाना॑ म॒ग्नये᳚ । \newline
19. इ॒त॒र॒ज॒नाना॑ म॒ग्नये॒ ऽग्नय॑ इतरज॒नाना॑ मितरज॒नाना॑ म॒ग्नये॒ धूङ्क्ष्णा॒ धूङ्क्ष्णा॒ ऽग्नय॑ इतरज॒नाना॑ मितरज॒नाना॑ म॒ग्नये॒ धूङ्क्ष्णा᳚ । \newline
20. इ॒त॒र॒ज॒नाना॒मिती॑तर - ज॒नाना᳚म् । \newline
21. अ॒ग्नये॒ धूङ्क्ष्णा॒ धूङ्क्ष्णा॒ ऽग्नये॒ ऽग्नये॒ धूङ्क्ष्णा᳚ । \newline
22. धूङ्‍क्ष्णेति॒ धूङ्‍क्ष्णा᳚ । \newline
\pagebreak
\markright{ TS 5.5.20.1  \hfill https://www.vedavms.in \hfill}

\section{ TS 5.5.20.1 }

\textbf{TS 5.5.20.1 } \newline
\textbf{Samhita Paata} \newline

अ॒ल॒ज आ᳚न्तरि॒क्ष उ॒द्रो म॒द्गुः प्ल॒वस्ते॑ऽपामदि॑त्यै हꣳस॒साचि॑रिन्द्रा॒ण्यै कीर्.शा॒ गृद्ध्रः॑ शितिक॒क्षी वा᳚र्द्ध्राण॒सस्ते दि॒व्या द्या॑वापृथि॒व्या᳚ श्वा॒वित् ॥ \newline

\textbf{Pada Paata} \newline

अ॒ल॒जः । आ॒न्त॒रि॒क्षः । उ॒द्रः । म॒द्गुः । प्ल॒वः । ते । अ॒पाम् । अदि॑त्यै । हꣳ॒॒स॒साचि॒रिति॑ हꣳस - साचिः॑ । इ॒न्द्रा॒ण्यै । कीर्.शा᳚ । गृध्रः॑ । शि॒ति॒क॒क्षीति॑ शिति - क॒क्षी । वा॒द्‌र्ध्रा॒ण॒सः । ते । दि॒व्याः । द्या॒वा॒पृ॒थि॒व्येति॑ द्यावा - पृ॒थि॒व्या᳚ । श्वा॒विदिति॑ श्व - वित् ॥  \newline


\textbf{Krama Paata} \newline

अ॒ल॒ज आ᳚न्तरि॒क्षः । आ॒न्त॒रि॒क्ष उ॒द्रः । उ॒द्रो म॒द्गुः । म॒द्गुः प्ल॒वः । प्ल॒वस्ते । ते॑ऽपाम् । अ॒पामदि॑त्यै । अदि॑त्यै हꣳस॒साचिः॑ । हꣳ॒॒स॒साचि॑रिन्द्रा॒ण्यै । हꣳ॒॒स॒साचि॒रिति॑ हꣳस - साचिः॑ । इ॒न्दा॒ण्यै कीर्.शा᳚ । कीर्.शा॒ गृद्ध्रः॑ । गृद्ध्रः॑ शितिक॒क्षी । शि॒ति॒क॒क्षी वा᳚र्द्ध्राण॒सः । शि॒ति॒क॒क्षीति॑ शिति - क॒क्षी । वा॒र्द्ध्रा॒ण॒सस्ते । ते दि॒व्या । दि॒व्या द्या॑वापृथि॒व्या᳚ । द्या॒वा॒पृ॒थि॒व्या᳚ श्वा॒वित् । द्या॒वा॒पृ॒थि॒व्येति॑ द्यावा - पृ॒थि॒व्या᳚ । श्वा॒विदिति॑ श्व - वित् । \newline

\textbf{Jatai Paata} \newline

1. अ॒ल॒ज आ᳚न्तरि॒क्ष आ᳚न्तरि॒क्षो॑ ऽल॒जो॑ ऽल॒ज आ᳚न्तरि॒क्षः । \newline
2. आ॒न्त॒रि॒क्ष उ॒द्र उ॒द्र आ᳚न्तरि॒क्ष आ᳚न्तरि॒क्ष उ॒द्रः । \newline
3. उ॒द्रो म॒द्‌गुर् म॒द्‌गु रु॒द्र उ॒द्रो म॒द्‌गुः । \newline
4. म॒द्‌गुः प्ल॒वः प्ल॒वो म॒द्‌गुर् म॒द्‌गुः प्ल॒वः । \newline
5. प्ल॒व स्ते ते प्ल॒वः प्ल॒व स्ते । \newline
6. ते॑ ऽपा म॒पाम् ते ते॑ ऽपाम् । \newline
7. अ॒पा मदि॑त्या॒ अदि॑त्या अ॒पा म॒पा मदि॑त्यै । \newline
8. अदि॑त्यै हꣳस॒साचि॑र्. हꣳस॒साचि॒ रदि॑त्या॒ अदि॑त्यै हꣳस॒साचिः॑ । \newline
9. हꣳ॒॒स॒साचि॑ रिन्द्रा॒ण्या इ॑न्द्रा॒ण्यै हꣳ॑स॒साचि॑र्. हꣳस॒साचि॑ रिन्द्रा॒ण्यै । \newline
10. हꣳ॒॒स॒साचि॒रिति॑ हꣳस - साचिः॑ । \newline
11. इ॒न्द्रा॒ण्यै कीर्.शा॒ कीर्.शे᳚न्द्रा॒ण्या इ॑न्द्रा॒ण्यै कीर्.शा᳚ । \newline
12. कीर्.शा॒ गृध्रो॒ गृध्रः॒ कीर्.शा॒ कीर्.शा॒ गृध्रः॑ । \newline
13. गृध्रः॑ शितिक॒क्षी शि॑तिक॒क्षी गृध्रो॒ गृध्रः॑ शितिक॒क्षी । \newline
14. शि॒ति॒क॒क्षी वा᳚र्द्ध्राण॒सो वा᳚र्द्ध्राण॒सः शि॑तिक॒क्षी शि॑तिक॒क्षी वा᳚र्द्ध्राण॒सः । \newline
15. शि॒ति॒क॒क्षीति॑ शिति - क॒क्षी । \newline
16. वा॒र्द्ध्रा॒ण॒स स्ते ते वा᳚र्द्ध्राण॒सो वा᳚र्द्ध्राण॒स स्ते । \newline
17. ते दि॒व्या दि॒व्या स्ते ते दि॒व्याः । \newline
18. दि॒व्या द्या॑वापृथि॒व्या᳚ द्यावापृथि॒व्या॑ दि॒व्या दि॒व्या द्या॑वापृथि॒व्या᳚ । \newline
19. द्या॒वा॒पृ॒थि॒व्या᳚ श्वा॒वि च्छ्वा॒विद् द्या॑वापृथि॒व्या᳚ द्यावापृथि॒व्या᳚ श्वा॒वित् । \newline
20. द्या॒वा॒पृ॒थि॒व्येति॑ द्यावा - पृ॒थि॒व्या᳚ । \newline
21. श्वा॒विदिति॑ श्व - वित् । \newline

\textbf{Ghana Paata } \newline

1. अ॒ल॒ज आ᳚न्तरि॒क्ष आ᳚न्तरि॒क्षो॑ ऽल॒जो॑ ऽल॒ज आ᳚न्तरि॒क्ष उ॒द्र उ॒द्र आ᳚न्तरि॒क्षो॑ ऽल॒जो॑ ऽल॒ज आ᳚न्तरि॒क्ष उ॒द्रः । \newline
2. आ॒न्त॒रि॒क्ष उ॒द्र उ॒द्र आ᳚न्तरि॒क्ष आ᳚न्तरि॒क्ष उ॒द्रो म॒द्‌गुर् म॒द्‌गु रु॒द्र आ᳚न्तरि॒क्ष आ᳚न्तरि॒क्ष उ॒द्रो म॒द्‌गुः । \newline
3. उ॒द्रो म॒द्‌गुर् म॒द्‌गु रु॒द्र उ॒द्रो म॒द्‌गुः प्ल॒वः प्ल॒वो म॒द्‌गु रु॒द्र उ॒द्रो म॒द्‌गुः प्ल॒वः । \newline
4. म॒द्‌गुः प्ल॒वः प्ल॒वो म॒द्‌गुर् म॒द्‌गुः प्ल॒व स्ते ते प्ल॒वो म॒द्‌गुर् म॒द्‌गुः प्ल॒व स्ते । \newline
5. प्ल॒व स्ते ते प्ल॒वः प्ल॒व स्ते॑ ऽपा म॒पाम् ते प्ल॒वः प्ल॒व स्ते॑ ऽपाम् । \newline
6. ते॑ ऽपा म॒पाम् ते ते॑ ऽपा मदि॑त्या॒ अदि॑त्या अ॒पाम् ते ते॑ ऽपा मदि॑त्यै । \newline
7. अ॒पा मदि॑त्या॒ अदि॑त्या अ॒पा म॒पा मदि॑त्यै हꣳस॒साचि॑र्. हꣳस॒साचि॒ रदि॑त्या अ॒पा म॒पा मदि॑त्यै हꣳस॒साचिः॑ । \newline
8. अदि॑त्यै हꣳस॒साचि॑र्. हꣳस॒साचि॒ रदि॑त्या॒ अदि॑त्यै हꣳस॒साचि॑ रिन्द्रा॒ण्या इ॑न्द्रा॒ण्यै हꣳ॑स॒साचि॒ रदि॑त्या॒ अदि॑त्यै हꣳस॒साचि॑ रिन्द्रा॒ण्यै । \newline
9. हꣳ॒॒स॒साचि॑ रिन्द्रा॒ण्या इ॑न्द्रा॒ण्यै हꣳ॑स॒साचि॑र्. हꣳस॒साचि॑ रिन्द्रा॒ण्यै कीर्.शा॒ कीर्.शे᳚न्द्रा॒ण्यै हꣳ॑स॒साचि॑र्. हꣳस॒साचि॑ रिन्द्रा॒ण्यै कीर्.शा᳚ । \newline
10. हꣳ॒॒स॒साचि॒रिति॑ हꣳस - साचिः॑ । \newline
11. इ॒न्द्रा॒ण्यै कीर्.शा॒ कीर्.शे᳚न्द्रा॒ण्या इ॑न्द्रा॒ण्यै कीर्.शा॒ गृध्रो॒ गृध्रः॒ कीर्.शे᳚न्द्रा॒ण्या इ॑न्द्रा॒ण्यै कीर्.शा॒ गृध्रः॑ । \newline
12. कीर्.शा॒ गृध्रो॒ गृध्रः॒ कीर्.शा॒ कीर्.शा॒ गृध्रः॑ शितिक॒क्षी शि॑तिक॒क्षी गृध्रः॒ कीर्.शा॒ कीर्.शा॒ गृध्रः॑ शितिक॒क्षी । \newline
13. गृध्रः॑ शितिक॒क्षी शि॑तिक॒क्षी गृध्रो॒ गृध्रः॑ शितिक॒क्षी वा᳚र्द्ध्राण॒सो वा᳚र्द्ध्राण॒सः शि॑तिक॒क्षी गृध्रो॒ गृध्रः॑ शितिक॒क्षी वा᳚र्द्ध्राण॒सः । \newline
14. शि॒ति॒क॒क्षी वा᳚र्द्ध्राण॒सो वा᳚र्द्ध्राण॒सः शि॑तिक॒क्षी शि॑तिक॒क्षी वा᳚र्द्ध्राण॒स स्ते ते वा᳚र्द्ध्राण॒सः शि॑तिक॒क्षी शि॑तिक॒क्षी वा᳚र्द्ध्राण॒स स्ते । \newline
15. शि॒ति॒क॒क्षीति॑ शिति - क॒क्षी । \newline
16. वा॒र्द्ध्रा॒ण॒स स्ते ते वा᳚र्द्ध्राण॒सो वा᳚र्द्ध्राण॒स स्ते दि॒व्या दि॒व्या स्ते वा᳚र्द्ध्राण॒सो वा᳚र्द्ध्राण॒स स्ते दि॒व्याः । \newline
17. ते दि॒व्या दि॒व्या स्ते ते दि॒व्या द्या॑वापृथि॒व्या᳚ द्यावापृथि॒व्या॑ दि॒व्या स्ते ते दि॒व्या द्या॑वापृथि॒व्या᳚ । \newline
18. दि॒व्या द्या॑वापृथि॒व्या᳚ द्यावापृथि॒व्या॑ दि॒व्या दि॒व्या द्या॑वापृथि॒व्या᳚ श्वा॒वि च्छ्वा॒विद् द्या॑वापृथि॒व्या॑ दि॒व्या दि॒व्या द्या॑वापृथि॒व्या᳚ श्वा॒वित् । \newline
19. द्या॒वा॒पृ॒थि॒व्या᳚ श्वा॒वि च्छ्वा॒विद् द्या॑वापृथि॒व्या᳚ द्यावापृथि॒व्या᳚ श्वा॒वित् । \newline
20. द्या॒वा॒पृ॒थि॒व्येति॑ द्यावा - पृ॒थि॒व्या᳚ । \newline
21. श्वा॒विदिति॑ श्व - वित् । \newline
\pagebreak
\markright{ TS 5.5.21.1  \hfill https://www.vedavms.in \hfill}

\section{ TS 5.5.21.1 }

\textbf{TS 5.5.21.1 } \newline
\textbf{Samhita Paata} \newline

सु॒प॒र्णः पा᳚र्ज॒न्यो हꣳ॒॒सो वृको॑ वृषदꣳ॒॒शस्त ऐ॒न्द्रा अ॒पामु॒द्रो᳚ ऽर्य॒म्णे लो॑पा॒शः सिꣳ॒॒हो न॑कु॒लो व्या॒घ्रस्ते म॑हे॒न्द्राय॒ कामा॑य॒ पर॑स्वान् ॥ \newline

\textbf{Pada Paata} \newline

सु॒प॒र्ण इति॑ सु - प॒र्णः । पा॒र्ज॒न्यः । हꣳ॒॒सः । वृकः॑ । वृ॒ष॒दꣳ॒॒शः । ते । ऐ॒न्द्राः । अ॒पाम् । उ॒द्रः । अ॒र्य॒म्णे । लो॒पा॒शः । सिꣳ॒॒हः । न॒कु॒लः । व्या॒घ्रः । ते । म॒हे॒न्द्रायेति॑ महा - इ॒न्द्राय॑ । कामा॑य । पर॑स्वान् ॥  \newline


\textbf{Krama Paata} \newline

सु॒प॒र्णः पा᳚र्ज॒न्यः । सु॒प॒र्ण इति॑ सु - प॒र्णः । पा॒र्ज॒न्यो हꣳ॒॒सः । हꣳ॒॒सो वृकः॑ । वृको॑ वृषदꣳ॒॒शः । वृ॒ष॒दꣳ॒॒शस्ते । त ऐ॒न्द्राः । ऐ॒न्द्रा अ॒पाम् । अ॒पामु॒द्रः । उ॒द्रो᳚ऽर्य॒म्णे । अ॒र्य॒म्णे लो॑पा॒शः । लो॒पा॒शः सिꣳ॒॒हः । सिꣳ॒॒हो न॑कु॒लः । न॒कु॒लो व्या॒घ्रः । व्या॒घ्रस्ते । ते म॑हे॒न्द्राय॑ । म॒हे॒न्द्राय॒ कामा॑य । म॒हे॒न्द्रायेति॑ महा - इ॒न्द्राय॑ । कामा॑य॒ पर॑स्वान् । पर॑स्वा॒निति॒ पर॑स्वान् । \newline

\textbf{Jatai Paata} \newline

1. सु॒प॒र्णः पा᳚र्ज॒न्यः पा᳚र्ज॒न्यः सु॑प॒र्णः सु॑प॒र्णः पा᳚र्ज॒न्यः । \newline
2. सु॒प॒र्ण इति॑ सु - प॒र्णः । \newline
3. पा॒र्ज॒न्यो हꣳ॒॒सो हꣳ॒॒सः पा᳚र्ज॒न्यः पा᳚र्ज॒न्यो हꣳ॒॒सः । \newline
4. हꣳ॒॒सो वृको॒ वृको॑ हꣳ॒॒सो हꣳ॒॒सो वृकः॑ । \newline
5. वृको॑ वृषदꣳ॒॒शो वृ॑षदꣳ॒॒शो वृको॒ वृको॑ वृषदꣳ॒॒शः । \newline
6. वृ॒ष॒दꣳ॒॒श स्ते ते वृ॑षदꣳ॒॒शो वृ॑षदꣳ॒॒श स्ते । \newline
7. त ऐ॒न्द्रा ऐ॒न्द्रा स्ते त ऐ॒न्द्राः । \newline
8. ऐ॒न्द्रा अ॒पा म॒पा मै॒न्द्रा ऐ॒न्द्रा अ॒पाम् । \newline
9. अ॒पा मु॒द्र उ॒द्रो॑ ऽपा म॒पा मु॒द्रः । \newline
10. उ॒द्रो᳚ ऽर्य॒म्णे᳚ ऽर्य॒म्ण उ॒द्र उ॒द्रो᳚ ऽर्य॒म्णे । \newline
11. अ॒र्य॒म्णे लो॑पा॒शो लो॑पा॒शो᳚ ऽर्य॒म्णे᳚ ऽर्य॒म्णे लो॑पा॒शः । \newline
12. लो॒पा॒शः सिꣳ॒॒हः सिꣳ॒॒हो लो॑पा॒शो लो॑पा॒शः सिꣳ॒॒हः । \newline
13. सिꣳ॒॒हो न॑कु॒लो न॑कु॒लः सिꣳ॒॒हः सिꣳ॒॒हो न॑कु॒लः । \newline
14. न॒कु॒लो व्या॒घ्रो व्या॒घ्रो न॑कु॒लो न॑कु॒लो व्या॒घ्रः । \newline
15. व्या॒घ्र स्ते ते व्या॒घ्रो व्या॒घ्र स्ते । \newline
16. ते म॑हे॒न्द्राय॑ महे॒न्द्राय॒ ते ते म॑हे॒न्द्राय॑ । \newline
17. म॒हे॒न्द्राय॒ कामा॑य॒ कामा॑य महे॒न्द्राय॑ महे॒न्द्राय॒ कामा॑य । \newline
18. म॒हे॒न्द्रायेति॑ महा - इ॒न्द्राय॑ । \newline
19. कामा॑य॒ पर॑स्वा॒न् पर॑स्वा॒न् कामा॑य॒ कामा॑य॒ पर॑स्वान् । \newline
20. पर॑स्वा॒निति॒ पर॑स्वान्न् । \newline

\textbf{Ghana Paata } \newline

1. सु॒प॒र्णः पा᳚र्ज॒न्यः पा᳚र्ज॒न्यः सु॑प॒र्णः सु॑प॒र्णः पा᳚र्ज॒न्यो हꣳ॒॒सो हꣳ॒॒सः पा᳚र्ज॒न्यः सु॑प॒र्णः सु॑प॒र्णः पा᳚र्ज॒न्यो हꣳ॒॒सः । \newline
2. सु॒प॒र्ण इति॑ सु - प॒र्णः । \newline
3. पा॒र्ज॒न्यो हꣳ॒॒सो हꣳ॒॒सः पा᳚र्ज॒न्यः पा᳚र्ज॒न्यो हꣳ॒॒सो वृको॒ वृको॑ हꣳ॒॒सः पा᳚र्ज॒न्यः पा᳚र्ज॒न्यो हꣳ॒॒सो वृकः॑ । \newline
4. हꣳ॒॒सो वृको॒ वृको॑ हꣳ॒॒सो हꣳ॒॒सो वृको॑ वृषदꣳ॒॒शो वृ॑षदꣳ॒॒शो वृको॑ हꣳ॒॒सो हꣳ॒॒सो वृको॑ वृषदꣳ॒॒शः । \newline
5. वृको॑ वृषदꣳ॒॒शो वृ॑षदꣳ॒॒शो वृको॒ वृको॑ वृषदꣳ॒॒श स्ते ते वृ॑षदꣳ॒॒शो वृको॒ वृको॑ वृषदꣳ॒॒श स्ते । \newline
6. वृ॒ष॒दꣳ॒॒श स्ते ते वृ॑षदꣳ॒॒शो वृ॑षदꣳ॒॒श स्त ऐ॒न्द्रा ऐ॒न्द्रा स्ते वृ॑षदꣳ॒॒शो वृ॑षदꣳ॒॒श स्त ऐ॒न्द्राः । \newline
7. त ऐ॒न्द्रा ऐ॒न्द्रा स्ते त ऐ॒न्द्रा अ॒पा म॒पा मै॒न्द्रा स्ते त ऐ॒न्द्रा अ॒पाम् । \newline
8. ऐ॒न्द्रा अ॒पा म॒पा मै॒न्द्रा ऐ॒न्द्रा अ॒पा मु॒द्र उ॒द्रो॑ ऽपा मै॒न्द्रा ऐ॒न्द्रा अ॒पा मु॒द्रः । \newline
9. अ॒पा मु॒द्र उ॒द्रो॑ ऽपा म॒पा मु॒द्रो᳚ ऽर्य॒म्णे᳚ ऽर्य॒म्ण उ॒द्रो॑ ऽपा म॒पा मु॒द्रो᳚ ऽर्य॒म्णे । \newline
10. उ॒द्रो᳚ ऽर्य॒म्णे᳚ ऽर्य॒म्ण उ॒द्र उ॒द्रो᳚ ऽर्य॒म्णे लो॑पा॒शो लो॑पा॒शो᳚ ऽर्य॒म्ण उ॒द्र उ॒द्रो᳚ ऽर्य॒म्णे लो॑पा॒शः । \newline
11. अ॒र्य॒म्णे लो॑पा॒शो लो॑पा॒शो᳚ ऽर्य॒म्णे᳚ ऽर्य॒म्णे लो॑पा॒शः सिꣳ॒॒हः सिꣳ॒॒हो लो॑पा॒शो᳚ ऽर्य॒म्णे᳚ ऽर्य॒म्णे लो॑पा॒शः सिꣳ॒॒हः । \newline
12. लो॒पा॒शः सिꣳ॒॒हः सिꣳ॒॒हो लो॑पा॒शो लो॑पा॒शः सिꣳ॒॒हो न॑कु॒लो न॑कु॒लः सिꣳ॒॒हो लो॑पा॒शो लो॑पा॒शः सिꣳ॒॒हो न॑कु॒लः । \newline
13. सिꣳ॒॒हो न॑कु॒लो न॑कु॒लः सिꣳ॒॒हः सिꣳ॒॒हो न॑कु॒लो व्या॒घ्रो व्या॒घ्रो न॑कु॒लः सिꣳ॒॒हः सिꣳ॒॒हो न॑कु॒लो व्या॒घ्रः । \newline
14. न॒कु॒लो व्या॒घ्रो व्या॒घ्रो न॑कु॒लो न॑कु॒लो व्या॒घ्र स्ते ते व्या॒घ्रो न॑कु॒लो न॑कु॒लो व्या॒घ्र स्ते । \newline
15. व्या॒घ्र स्ते ते व्या॒घ्रो व्या॒घ्र स्ते म॑हे॒न्द्राय॑ महे॒न्द्राय॒ ते व्या॒घ्रो व्या॒घ्र स्ते म॑हे॒न्द्राय॑ । \newline
16. ते म॑हे॒न्द्राय॑ महे॒न्द्राय॒ ते ते म॑हे॒न्द्राय॒ कामा॑य॒ कामा॑य महे॒न्द्राय॒ ते ते म॑हे॒न्द्राय॒ कामा॑य । \newline
17. म॒हे॒न्द्राय॒ कामा॑य॒ कामा॑य महे॒न्द्राय॑ महे॒न्द्राय॒ कामा॑य॒ पर॑स्वा॒न् पर॑स्वा॒न् कामा॑य महे॒न्द्राय॑ महे॒न्द्राय॒ कामा॑य॒ पर॑स्वान् । \newline
18. म॒हे॒न्द्रायेति॑ महा - इ॒न्द्राय॑ । \newline
19. कामा॑य॒ पर॑स्वा॒न् पर॑स्वा॒न् कामा॑य॒ कामा॑य॒ पर॑स्वान् । \newline
20. पर॑स्वा॒निति॒ पर॑स्वान्न् । \newline
\pagebreak
\markright{ TS 5.5.22.1  \hfill https://www.vedavms.in \hfill}

\section{ TS 5.5.22.1 }

\textbf{TS 5.5.22.1 } \newline
\textbf{Samhita Paata} \newline

आ॒ग्ने॒यः कृ॒ष्णग्री॑वः सारस्व॒ती मे॒षी ब॒भ्रुः सौ॒म्यः पौ॒ष्णः श्या॒मः शि॑तिपृ॒ष्ठो बा॑र्.हस्प॒त्यः शि॒ल्पो वै᳚श्वदे॒व ऐ॒न्द्रो॑ऽरु॒णो मा॑रु॒तः क॒ल्माष॑ ऐन्द्रा॒ग्नः सꣳ॑हि॒तो॑ ऽधोरा॑मः सावि॒त्रो वा॑रु॒णः पेत्वः॑ ॥ \newline

\textbf{Pada Paata} \newline

आ॒ग्ने॒यः । कृ॒ष्णग्री॑व॒ इति॑ कृ॒ष्ण - ग्री॒वः॒ । सा॒र॒स्व॒ती । मे॒षी । ब॒भ्रुः । सौ॒म्यः । पौ॒ष्णः । श्या॒मः । शि॒ति॒पृ॒ष्ठ इति॑ शिति -पृ॒ष्ठः । बा॒र्.॒ह॒स्प॒त्यः । शि॒ल्पः । वै॒श्व॒दे॒व इति॑ वैश्व - दे॒वः । ऐ॒न्द्रः । अ॒रु॒णः । मा॒रु॒तः । क॒ल्माषः॑ । ऐ॒न्द्रा॒ग्न इत्यै᳚न्द्र - अ॒ग्नः । सꣳ॒॒हि॒त इति॑ सं - हि॒तः । अ॒धोरा॑म॒ इत्य॒धः - रा॒मः॒ । सा॒वि॒त्रः । वा॒रु॒णः । पेत्वः॑ ॥  \newline


\textbf{Krama Paata} \newline

आ॒ग्ने॒यः कृ॒ष्णग्री॑वः । कृ॒ष्णग्री॑वः सारस्व॒ती । कृ॒ष्णग्री॑व॒ इति॑ कृ॒ष्ण - ग्री॒वः॒ । सा॒र॒स्व॒ती मे॒षी । मे॒षी ब॒भ्रुः । ब॒भ्रुः सौ॒म्यः । सौ॒म्यः पौ॒ष्णः । पौ॒ष्णः श्या॒मः । श्या॒मः शि॑तिपृ॒ष्ठः । शि॒ति॒पृ॒ष्ठो बा॑र्.हस्प॒त्यः । शि॒ति॒पृ॒ष्ठ इति॑ शिति - पृ॒ष्ठः । बा॒र्॒.ह॒स्प॒त्य शि॒ल्पः । शि॒ल्पो वै᳚श्वदे॒वः । वै॒श्व॒दे॒व ऐ॒न्द्रः । वै॒श्व॒दे॒व इति॑ वैश्व - दे॒वः । ऐ॒न्द्रो॑ऽरु॒णः । अ॒रु॒णो मा॑रु॒तः । मा॒रु॒तः क॒ल्माषः॑ । क॒ल्माष॑ ऐन्द्रा॒ग्नः । ऐ॒न्द्रा॒ग्नः सꣳ॑हि॒तः । ऐ॒न्द्रा॒ग्न इत्यै᳚न्द्र - अ॒ग्नः । सꣳ॒॒हि॒तो॑ऽधोरा॑मः । सꣳ॒॒हि॒त इति॑ सम् - हि॒तः । अ॒धोरा॑मः सावि॒त्रः । अ॒धोरा॑म॒ इत्य॒धः - रा॒मः॒ । सा॒वि॒त्रो वा॑रु॒णः । वा॒रु॒णः पेत्वः॑ । पेत्व॒ इति॒ पेत्वः॑ । \newline

\textbf{Jatai Paata} \newline

1. आ॒ग्ने॒यः कृ॒ष्णग्री॑वः कृ॒ष्णग्री॑व आग्ने॒य आ᳚ग्ने॒यः कृ॒ष्णग्री॑वः । \newline
2. कृ॒ष्णग्री॑वः सारस्व॒ती सा॑रस्व॒ती कृ॒ष्णग्री॑वः कृ॒ष्णग्री॑वः सारस्व॒ती । \newline
3. कृ॒ष्णग्री॑व॒ इति॑ कृ॒ष्ण - ग्री॒वः॒ । \newline
4. सा॒र॒स्व॒ती मे॒षी मे॒षी सा॑रस्व॒ती सा॑रस्व॒ती मे॒षी । \newline
5. मे॒षी ब॒भ्रुर् ब॒भ्रुर् मे॒षी मे॒षी ब॒भ्रुः । \newline
6. ब॒भ्रुः सौ॒म्यः सौ॒म्यो ब॒भ्रुर् ब॒भ्रुः सौ॒म्यः । \newline
7. सौ॒म्यः पौ॒ष्णः पौ॒ष्णः सौ॒म्यः सौ॒म्यः पौ॒ष्णः । \newline
8. पौ॒ष्णः श्या॒मः श्या॒मः पौ॒ष्णः पौ॒ष्णः श्या॒मः । \newline
9. श्या॒मः शि॑तिपृ॒ष्ठः शि॑तिपृ॒ष्ठः श्या॒मः श्या॒मः शि॑तिपृ॒ष्ठः । \newline
10. शि॒ति॒पृ॒ष्ठो बा॑र्.हस्प॒त्यो बा॑र्.हस्प॒त्यः शि॑तिपृ॒ष्ठः शि॑तिपृ॒ष्ठो बा॑र्.हस्प॒त्यः । \newline
11. शि॒ति॒पृ॒ष्ठ इति॑ शिति - पृ॒ष्ठः । \newline
12. बा॒र्॒.ह॒स्प॒त्यः शि॒ल्पः शि॒ल्पो बा॑र्.हस्प॒त्यो बा॑र्.हस्प॒त्यः शि॒ल्पः । \newline
13. शि॒ल्पो वै᳚श्वदे॒वो वै᳚श्वदे॒वः शि॒ल्पः शि॒ल्पो वै᳚श्वदे॒वः । \newline
14. वै॒श्व॒दे॒व ऐ॒न्द्र ऐ॒न्द्रो वै᳚श्वदे॒वो वै᳚श्वदे॒व ऐ॒न्द्रः । \newline
15. वै॒श्व॒दे॒व इति॑ वैश्व - दे॒वः । \newline
16. ऐ॒न्द्रो॑ ऽरु॒णो॑ ऽरु॒ण ऐ॒न्द्र ऐ॒न्द्रो॑ ऽरु॒णः । \newline
17. अ॒रु॒णो मा॑रु॒तो मा॑रु॒तो॑ ऽरु॒णो॑ ऽरु॒णो मा॑रु॒तः । \newline
18. मा॒रु॒तः क॒ल्माषः॑ क॒ल्माषो॑ मारु॒तो मा॑रु॒तः क॒ल्माषः॑ । \newline
19. क॒ल्माष॑ ऐन्द्रा॒ग्न ऐ᳚न्द्रा॒ग्नः क॒ल्माषः॑ क॒ल्माष॑ ऐन्द्रा॒ग्नः । \newline
20. ऐ॒न्द्रा॒ग्नः सꣳ॑हि॒तः सꣳ॑हि॒त ऐ᳚न्द्रा॒ग्न ऐ᳚न्द्रा॒ग्नः सꣳ॑हि॒तः । \newline
21. ऐ॒न्द्रा॒ग्न इत्यै᳚न्द्र - अ॒ग्नः । \newline
22. सꣳ॒॒हि॒तो॑ ऽधोरा॑मो॒ ऽधोरा॑मः सꣳहि॒तः सꣳ॑हि॒तो॑ ऽधोरा॑मः । \newline
23. सꣳ॒॒हि॒त इति॑ सं - हि॒तः । \newline
24. अ॒धोरा॑मः सावि॒त्रः सा॑वि॒त्रो॑ ऽधोरा॑मो॒ ऽधोरा॑मः सावि॒त्रः । \newline
25. अ॒धोरा॑म॒ इत्य॒धः - रा॒मः॒ । \newline
26. सा॒वि॒त्रो वा॑रु॒णो वा॑रु॒णः सा॑वि॒त्रः सा॑वि॒त्रो वा॑रु॒णः । \newline
27. वा॒रु॒णः पेत्वः॒ पेत्वो॑ वारु॒णो वा॑रु॒णः पेत्वः॑ । \newline
28. पेत्व॒ इति॒ पेत्वः॑ । \newline

\textbf{Ghana Paata } \newline

1. आ॒ग्ने॒यः कृ॒ष्णग्री॑वः कृ॒ष्णग्री॑व आग्ने॒य आ᳚ग्ने॒यः कृ॒ष्णग्री॑वः सारस्व॒ती सा॑रस्व॒ती कृ॒ष्णग्री॑व आग्ने॒य आ᳚ग्ने॒यः कृ॒ष्णग्री॑वः सारस्व॒ती । \newline
2. कृ॒ष्णग्री॑वः सारस्व॒ती सा॑रस्व॒ती कृ॒ष्णग्री॑वः कृ॒ष्णग्री॑वः सारस्व॒ती मे॒षी मे॒षी सा॑रस्व॒ती कृ॒ष्णग्री॑वः कृ॒ष्णग्री॑वः सारस्व॒ती मे॒षी । \newline
3. कृ॒ष्णग्री॑व॒ इति॑ कृ॒ष्ण - ग्री॒वः॒ । \newline
4. सा॒र॒स्व॒ती मे॒षी मे॒षी सा॑रस्व॒ती सा॑रस्व॒ती मे॒षी ब॒भ्रुर् ब॒भ्रुर् मे॒षी सा॑रस्व॒ती सा॑रस्व॒ती मे॒षी ब॒भ्रुः । \newline
5. मे॒षी ब॒भ्रुर् ब॒भ्रुर् मे॒षी मे॒षी ब॒भ्रुः सौ॒म्यः सौ॒म्यो ब॒भ्रुर् मे॒षी मे॒षी ब॒भ्रुः सौ॒म्यः । \newline
6. ब॒भ्रुः सौ॒म्यः सौ॒म्यो ब॒भ्रुर् ब॒भ्रुः सौ॒म्यः पौ॒ष्णः पौ॒ष्णः सौ॒म्यो ब॒भ्रुर् ब॒भ्रुः सौ॒म्यः पौ॒ष्णः । \newline
7. सौ॒म्यः पौ॒ष्णः पौ॒ष्णः सौ॒म्यः सौ॒म्यः पौ॒ष्णः श्या॒मः श्या॒मः पौ॒ष्णः सौ॒म्यः सौ॒म्यः पौ॒ष्णः श्या॒मः । \newline
8. पौ॒ष्णः श्या॒मः श्या॒मः पौ॒ष्णः पौ॒ष्णः श्या॒मः शि॑तिपृ॒ष्ठः शि॑तिपृ॒ष्ठः श्या॒मः पौ॒ष्णः पौ॒ष्णः श्या॒मः शि॑तिपृ॒ष्ठः । \newline
9. श्या॒मः शि॑तिपृ॒ष्ठः शि॑तिपृ॒ष्ठः श्या॒मः श्या॒मः शि॑तिपृ॒ष्ठो बा॑र्.हस्प॒त्यो बा॑र्.हस्प॒त्यः शि॑तिपृ॒ष्ठः श्या॒मः श्या॒मः शि॑तिपृ॒ष्ठो बा॑र्.हस्प॒त्यः । \newline
10. शि॒ति॒पृ॒ष्ठो बा॑र्.हस्प॒त्यो बा॑र्.हस्प॒त्यः शि॑तिपृ॒ष्ठः शि॑तिपृ॒ष्ठो बा॑र्.हस्प॒त्यः शि॒ल्पः शि॒ल्पो बा॑र्.हस्प॒त्यः शि॑तिपृ॒ष्ठः शि॑तिपृ॒ष्ठो बा॑र्.हस्प॒त्यः शि॒ल्पः । \newline
11. शि॒ति॒पृ॒ष्ठ इति॑ शिति - पृ॒ष्ठः । \newline
12. बा॒र्॒.ह॒स्प॒त्यः शि॒ल्पः शि॒ल्पो बा॑र्.हस्प॒त्यो बा॑र्.हस्प॒त्यः शि॒ल्पो वै᳚श्वदे॒वो वै᳚श्वदे॒वः शि॒ल्पो बा॑र्.हस्प॒त्यो बा॑र्.हस्प॒त्यः शि॒ल्पो वै᳚श्वदे॒वः । \newline
13. शि॒ल्पो वै᳚श्वदे॒वो वै᳚श्वदे॒वः शि॒ल्पः शि॒ल्पो वै᳚श्वदे॒व ऐ॒न्द्र ऐ॒न्द्रो वै᳚श्वदे॒वः शि॒ल्पः शि॒ल्पो वै᳚श्वदे॒व ऐ॒न्द्रः । \newline
14. वै॒श्व॒दे॒व ऐ॒न्द्र ऐ॒न्द्रो वै᳚श्वदे॒वो वै᳚श्वदे॒व ऐ॒न्द्रो॑ ऽरु॒णो॑ ऽरु॒ण ऐ॒न्द्रो वै᳚श्वदे॒वो वै᳚श्वदे॒व ऐ॒न्द्रो॑ ऽरु॒णः । \newline
15. वै॒श्व॒दे॒व इति॑ वैश्व - दे॒वः । \newline
16. ऐ॒न्द्रो॑ ऽरु॒णो॑ ऽरु॒ण ऐ॒न्द्र ऐ॒न्द्रो॑ ऽरु॒णो मा॑रु॒तो मा॑रु॒तो॑ ऽरु॒ण ऐ॒न्द्र ऐ॒न्द्रो॑ ऽरु॒णो मा॑रु॒तः । \newline
17. अ॒रु॒णो मा॑रु॒तो मा॑रु॒तो॑ ऽरु॒णो॑ ऽरु॒णो मा॑रु॒तः क॒ल्माषः॑ क॒ल्माषो॑ मारु॒तो॑ ऽरु॒णो॑ ऽरु॒णो मा॑रु॒तः क॒ल्माषः॑ । \newline
18. मा॒रु॒तः क॒ल्माषः॑ क॒ल्माषो॑ मारु॒तो मा॑रु॒तः क॒ल्माष॑ ऐन्द्रा॒ग्न ऐ᳚न्द्रा॒ग्नः क॒ल्माषो॑ मारु॒तो मा॑रु॒तः क॒ल्माष॑ ऐन्द्रा॒ग्नः । \newline
19. क॒ल्माष॑ ऐन्द्रा॒ग्न ऐ᳚न्द्रा॒ग्नः क॒ल्माषः॑ क॒ल्माष॑ ऐन्द्रा॒ग्नः सꣳ॑हि॒तः सꣳ॑हि॒त ऐ᳚न्द्रा॒ग्नः क॒ल्माषः॑ क॒ल्माष॑ ऐन्द्रा॒ग्नः सꣳ॑हि॒तः । \newline
20. ऐ॒न्द्रा॒ग्नः सꣳ॑हि॒तः सꣳ॑हि॒त ऐ᳚न्द्रा॒ग्न ऐ᳚न्द्रा॒ग्नः सꣳ॑हि॒तो॑ ऽधोरा॑मो॒ ऽधोरा॑मः सꣳहि॒त ऐ᳚न्द्रा॒ग्न ऐ᳚न्द्रा॒ग्नः सꣳ॑हि॒तो॑ ऽधोरा॑मः । \newline
21. ऐ॒न्द्रा॒ग्न इत्यै᳚न्द्र - अ॒ग्नः । \newline
22. सꣳ॒॒हि॒तो॑ ऽधोरा॑मो॒ ऽधोरा॑मः सꣳहि॒तः सꣳ॑हि॒तो॑ ऽधोरा॑मः सावि॒त्रः सा॑वि॒त्रो॑ ऽधोरा॑मः सꣳहि॒तः सꣳ॑हि॒तो॑ ऽधोरा॑मः सावि॒त्रः । \newline
23. सꣳ॒॒हि॒त इति॑ सं - हि॒तः । \newline
24. अ॒धोरा॑मः सावि॒त्रः सा॑वि॒त्रो॑ ऽधोरा॑मो॒ ऽधोरा॑मः सावि॒त्रो वा॑रु॒णो वा॑रु॒णः सा॑वि॒त्रो॑ ऽधोरा॑मो॒ ऽधोरा॑मः सावि॒त्रो वा॑रु॒णः । \newline
25. अ॒धोरा॑म॒ इत्य॒धः - रा॒मः॒ । \newline
26. सा॒वि॒त्रो वा॑रु॒णो वा॑रु॒णः सा॑वि॒त्रः सा॑वि॒त्रो वा॑रु॒णः पेत्वः॒ पेत्वो॑ वारु॒णः सा॑वि॒त्रः सा॑वि॒त्रो वा॑रु॒णः पेत्वः॑ । \newline
27. वा॒रु॒णः पेत्वः॒ पेत्वो॑ वारु॒णो वा॑रु॒णः पेत्वः॑ । \newline
28. पेत्व॒ इति॒ पेत्वः॑ । \newline
\pagebreak
\markright{ TS 5.5.23.1  \hfill https://www.vedavms.in \hfill}

\section{ TS 5.5.23.1 }

\textbf{TS 5.5.23.1 } \newline
\textbf{Samhita Paata} \newline

अश्व॑स्तूप॒रो गो॑मृ॒गस्ते प्रा॑जाप॒त्या आ᳚ग्ने॒यौ कृ॒ष्णग्री॑वौ त्वा॒ष्ट्रौ लो॑मशस॒क्थौ शि॑तिपृ॒ष्ठौ बा॑र्.हस्प॒त्यौ धा॒त्रे पृ॑षोद॒रः सौ॒र्यो ब॒लक्षः॒ पेत्वः॑ ॥ \newline

\textbf{Pada Paata} \newline

अश्वः॑ । तू॒प॒रः । गो॒मृ॒ग इति॑ गो - मृ॒गः । ते । प्रा॒जा॒प॒त्या इति॑ प्राजा - प॒त्याः । आ॒ग्ने॒यौ । कृ॒ष्णग्री॑वा॒विति॑ कृ॒ष्ण - ग्री॒वौ॒ । त्वा॒ष्ट्रौ । लो॒म॒श॒स॒क्थाविति॑ लोमश - स॒क्थौ । शि॒ति॒पृ॒ष्ठाविति॑ शिति - पृ॒ष्ठौ । बा॒र्.॒ह॒स्प॒त्यौ । धा॒त्रे । पृ॒षो॒द॒र इति॑ पृष - उ॒द॒रः । सौ॒र्यः । ब॒लक्षः॑ । पेत्वः॑ ॥  \newline


\textbf{Krama Paata} \newline

अश्व॑स्तूप॒रः । तू॒प॒रो गो॑मृ॒गः । गो॒मृ॒गस्ते । गो॒मृ॒ग इति॑ गो - मृ॒गः । ते प्रा॑जाप॒त्याः । प्रा॒जा॒प॒त्या आ᳚ग्ने॒यौ । प्रा॒जा॒प॒त्या इति॑ प्राजा - प॒त्याः । आ॒ग्ने॒यौ कृ॒ष्णग्री॑वौ । कृ॒ष्णग्री॑वौ त्वा॒ष्ट्रौ । कृ॒ष्णग्री॑वा॒विति॑ कृ॒ष्ण - ग्री॒वौ॒ । त्वा॒ष्ट्रौ लो॑मशस॒क्थौ । लो॒म॒श॒स॒क्थौ शि॑तिपृ॒ष्ठौ । लो॒म॒श॒स॒क्थाविति॑ लोमश - स॒क्थौ । शि॒ति॒पृ॒ष्ठौ बा॑र्.हस्प॒त्यौ । शि॒ति॒पृ॒ष्ठाविति॑ शिति - पृ॒ष्ठौ । बा॒र्॒.ह॒स्प॒त्यौ धा॒त्रे । धा॒त्रे पृ॑षोद॒रः । पृ॒षो॒द॒रः सौ॒र्यः । पृ॒षो॒द॒र इति॑ पृष - उ॒द॒रः । सौ॒र्यो ब॒लक्षः॑ । ब॒लक्षः॒ पेत्वः॑ । पेत्व॒ इति॒ पेत्वः॑ । \newline

\textbf{Jatai Paata} \newline

1. अश्व॑ स्तूप॒र स्तू॑प॒रो ऽश्वो ऽश्व॑ स्तूप॒रः । \newline
2. तू॒प॒रो गो॑मृ॒गो गो॑मृ॒ग स्तू॑प॒र स्तू॑प॒रो गो॑मृ॒गः । \newline
3. गो॒मृ॒ग स्ते ते गो॑मृ॒गो गो॑मृ॒ग स्ते । \newline
4. गो॒मृ॒ग इति॑ गो - मृ॒गः । \newline
5. ते प्रा॑जाप॒त्याः प्रा॑जाप॒त्यास्ते ते प्रा॑जाप॒त्याः । \newline
6. प्रा॒जा॒प॒त्या आ᳚ग्ने॒या वा᳚ग्ने॒यौ प्रा॑जाप॒त्याः प्रा॑जाप॒त्या आ᳚ग्ने॒यौ । \newline
7. प्रा॒जा॒प॒त्या इति॑ प्राजा - प॒त्याः । \newline
8. आ॒ग्ने॒यौ कृ॒ष्णग्री॑वौ कृ॒ष्णग्री॑वा वाग्ने॒या वा᳚ग्ने॒यौ कृ॒ष्णग्री॑वौ । \newline
9. कृ॒ष्णग्री॑वौ त्वा॒ष्ट्रौ त्वा॒ष्ट्रौ कृ॒ष्णग्री॑वौ कृ॒ष्णग्री॑वौ त्वा॒ष्ट्रौ । \newline
10. कृ॒ष्णग्री॑वा॒विति॑ कृ॒ष्ण - ग्री॒वौ॒ । \newline
11. त्वा॒ष्ट्रौ लो॑मशस॒क्थौ लो॑मशस॒क्थौ त्वा॒ष्ट्रौ त्वा॒ष्ट्रौ लो॑मशस॒क्थौ । \newline
12. लो॒म॒श॒स॒क्थौ शि॑तिपृ॒ष्ठौ शि॑तिपृ॒ष्ठौ लो॑मशस॒क्थौ लो॑मशस॒क्थौ शि॑तिपृ॒ष्ठौ । \newline
13. लो॒म॒श॒स॒क्थाविति॑ लोमश - स॒क्थौ । \newline
14. शि॒ति॒पृ॒ष्ठौ बा॑र्.हस्प॒त्यौ बा॑र्.हस्प॒त्यौ शि॑तिपृ॒ष्ठौ शि॑तिपृ॒ष्ठौ बा॑र्.हस्प॒त्यौ । \newline
15. शि॒ति॒पृ॒ष्ठाविति॑ शिति - पृ॒ष्ठौ । \newline
16. बा॒र्॒.ह॒स्प॒त्यौ धा॒त्रे धा॒त्रे बा॑र्.हस्प॒त्यौ बा॑र्.हस्प॒त्यौ धा॒त्रे । \newline
17. धा॒त्रे पृ॑षोद॒रः पृ॑षोद॒रो धा॒त्रे धा॒त्रे पृ॑षोद॒रः । \newline
18. पृ॒षो॒द॒रः सौ॒र्यः सौ॒र्यः पृ॑षोद॒रः पृ॑षोद॒रः सौ॒र्यः । \newline
19. पृ॒षो॒द॒र इति॑ पृष - उ॒द॒रः । \newline
20. सौ॒र्यो ब॒लक्षो॑ ब॒लक्षः॑ सौ॒र्यः सौ॒र्यो ब॒लक्षः॑ । \newline
21. ब॒लक्षः॒ पेत्वः॒ पेत्वो॑ ब॒लक्षो॑ ब॒लक्षः॒ पेत्वः॑ । \newline
22. पेत्व॒ इति॒ पेत्वः॑ । \newline

\textbf{Ghana Paata } \newline

1. अश्व॑ स्तूप॒र स्तू॑प॒रो ऽश्वो ऽश्व॑ स्तूप॒रो गो॑मृ॒गो गो॑मृ॒ग स्तू॑प॒रो ऽश्वो ऽश्व॑ स्तूप॒रो गो॑मृ॒गः । \newline
2. तू॒प॒रो गो॑मृ॒गो गो॑मृ॒ग स्तू॑प॒र स्तू॑प॒रो गो॑मृ॒ग स्ते ते गो॑मृ॒ग स्तू॑प॒र स्तू॑प॒रो गो॑मृ॒ग स्ते । \newline
3. गो॒मृ॒ग स्ते ते गो॑मृ॒गो गो॑मृ॒ग स्ते प्रा॑जाप॒त्याः प्रा॑जाप॒त्या स्ते गो॑मृ॒गो गो॑मृ॒ग स्ते प्रा॑जाप॒त्याः । \newline
4. गो॒मृ॒ग इति॑ गो - मृ॒गः । \newline
5. ते प्रा॑जाप॒त्याः प्रा॑जाप॒त्या स्ते ते प्रा॑जाप॒त्या आ᳚ग्ने॒या वा᳚ग्ने॒यौ प्रा॑जाप॒त्या स्ते ते प्रा॑जाप॒त्या आ᳚ग्ने॒यौ । \newline
6. प्रा॒जा॒प॒त्या आ᳚ग्ने॒या वा᳚ग्ने॒यौ प्रा॑जाप॒त्याः प्रा॑जाप॒त्या आ᳚ग्ने॒यौ कृ॒ष्णग्री॑वौ कृ॒ष्णग्री॑वा वाग्ने॒यौ प्रा॑जाप॒त्याः प्रा॑जाप॒त्या आ᳚ग्ने॒यौ कृ॒ष्णग्री॑वौ । \newline
7. प्रा॒जा॒प॒त्या इति॑ प्राजा - प॒त्याः । \newline
8. आ॒ग्ने॒यौ कृ॒ष्णग्री॑वौ कृ॒ष्णग्री॑वा वाग्ने॒या वा᳚ग्ने॒यौ कृ॒ष्णग्री॑वौ त्वा॒ष्ट्रौ त्वा॒ष्ट्रौ कृ॒ष्णग्री॑वा वाग्ने॒या वा᳚ग्ने॒यौ कृ॒ष्णग्री॑वौ त्वा॒ष्ट्रौ । \newline
9. कृ॒ष्णग्री॑वौ त्वा॒ष्ट्रौ त्वा॒ष्ट्रौ कृ॒ष्णग्री॑वौ कृ॒ष्णग्री॑वौ त्वा॒ष्ट्रौ लो॑मशस॒क्थौ लो॑मशस॒क्थौ त्वा॒ष्ट्रौ कृ॒ष्णग्री॑वौ कृ॒ष्णग्री॑वौ त्वा॒ष्ट्रौ लो॑मशस॒क्थौ । \newline
10. कृ॒ष्णग्री॑वा॒विति॑ कृ॒ष्ण - ग्री॒वौ॒ । \newline
11. त्वा॒ष्ट्रौ लो॑मशस॒क्थौ लो॑मशस॒क्थौ त्वा॒ष्ट्रौ त्वा॒ष्ट्रौ लो॑मशस॒क्थौ शि॑तिपृ॒ष्ठौ शि॑तिपृ॒ष्ठौ लो॑मशस॒क्थौ त्वा॒ष्ट्रौ त्वा॒ष्ट्रौ लो॑मशस॒क्थौ शि॑तिपृ॒ष्ठौ । \newline
12. लो॒म॒श॒स॒क्थौ शि॑तिपृ॒ष्ठौ शि॑तिपृ॒ष्ठौ लो॑मशस॒क्थौ लो॑मशस॒क्थौ शि॑तिपृ॒ष्ठौ बा॑र्.हस्प॒त्यौ बा॑र्.हस्प॒त्यौ शि॑तिपृ॒ष्ठौ लो॑मशस॒क्थौ लो॑मशस॒क्थौ शि॑तिपृ॒ष्ठौ बा॑र्.हस्प॒त्यौ । \newline
13. लो॒म॒श॒स॒क्थाविति॑ लोमश - स॒क्थौ । \newline
14. शि॒ति॒पृ॒ष्ठौ बा॑र्.हस्प॒त्यौ बा॑र्.हस्प॒त्यौ शि॑तिपृ॒ष्ठौ शि॑तिपृ॒ष्ठौ बा॑र्.हस्प॒त्यौ धा॒त्रे धा॒त्रे बा॑र्.हस्प॒त्यौ शि॑तिपृ॒ष्ठौ शि॑तिपृ॒ष्ठौ बा॑र्.हस्प॒त्यौ धा॒त्रे । \newline
15. शि॒ति॒पृ॒ष्ठाविति॑ शिति - पृ॒ष्ठौ । \newline
16. बा॒र्॒.ह॒स्प॒त्यौ धा॒त्रे धा॒त्रे बा॑र्.हस्प॒त्यौ बा॑र्.हस्प॒त्यौ धा॒त्रे पृ॑षोद॒रः पृ॑षोद॒रो धा॒त्रे बा॑र्.हस्प॒त्यौ बा॑र्.हस्प॒त्यौ धा॒त्रे पृ॑षोद॒रः । \newline
17. धा॒त्रे पृ॑षोद॒रः पृ॑षोद॒रो धा॒त्रे धा॒त्रे पृ॑षोद॒रः सौ॒र्यः सौ॒र्यः पृ॑षोद॒रो धा॒त्रे धा॒त्रे पृ॑षोद॒रः सौ॒र्यः । \newline
18. पृ॒षो॒द॒रः सौ॒र्यः सौ॒र्यः पृ॑षोद॒रः पृ॑षोद॒रः सौ॒र्यो ब॒लक्षो॑ ब॒लक्षः॑ सौ॒र्यः पृ॑षोद॒रः पृ॑षोद॒रः सौ॒र्यो ब॒लक्षः॑ । \newline
19. पृ॒षो॒द॒र इति॑ पृष - उ॒द॒रः । \newline
20. सौ॒र्यो ब॒लक्षो॑ ब॒लक्षः॑ सौ॒र्यः सौ॒र्यो ब॒लक्षः॒ पेत्वः॒ पेत्वो॑ ब॒लक्षः॑ सौ॒र्यः सौ॒र्यो ब॒लक्षः॒ पेत्वः॑ । \newline
21. ब॒लक्षः॒ पेत्वः॒ पेत्वो॑ ब॒लक्षो॑ ब॒लक्षः॒ पेत्वः॑ । \newline
22. पेत्व॒ इति॒ पेत्वः॑ । \newline
\pagebreak
\markright{ TS 5.5.24.1  \hfill https://www.vedavms.in \hfill}

\section{ TS 5.5.24.1 }

\textbf{TS 5.5.24.1 } \newline
\textbf{Samhita Paata} \newline

अ॒ग्नयेऽनी॑कवते॒ रोहि॑ताञ्जिरन॒ड्वान॒धोरा॑मौ सावि॒त्रौ पौ॒ष्णौ र॑ज॒तना॑भी वैश्वदे॒वौ पि॒शङ्गौ॑ तूप॒रौ मा॑रु॒तः क॒ल्माष॑ आग्ने॒यः कृ॒ष्णो॑ऽजः सा॑रस्व॒ती मे॒षी वा॑रु॒णः कृ॒ष्ण एक॑शितिपा॒त् पेत्वः॑ ॥ \newline

\textbf{Pada Paata} \newline

अ॒ग्नये᳚ । अनी॑कवत॒ इत्यनी॑क-व॒ते॒ । रोहि॑ताञ्जि॒रिति॒ रोहि॑त-अ॒ञ्जिः॒ । अ॒न॒ड्वान् । अ॒धोरा॑मा॒वित्य॒धः - रा॒मौ॒ । सा॒वि॒त्रौ । पौ॒ष्णौ । र॒ज॒तना॑भी॒ इति॑ रज॒त - ना॒भी॒ । वै॒श्व॒दे॒वाविति॑ वैश्व-दे॒वौ । पि॒शङ्गौ᳚ । तू॒प॒रौ । मा॒रु॒तः । क॒ल्माषः॑ । आ॒ग्ने॒यः । कृ॒ष्णः । अ॒जः । सा॒र॒स्व॒ती । मे॒षी । वा॒रु॒णः । कृ॒ष्णः । एक॑शितिपा॒दित्येक॑-शि॒ति॒पा॒त् । पेत्वः॑ ॥  \newline


\textbf{Krama Paata} \newline

अ॒ग्नयेऽनी॑कवते । अनी॑कवते॒ रोहि॑ताञ्जिः । अनी॑कवत॒ इत्यनी॑क - व॒ते॒ । रोहि॑ताञ्जिरन॒ड्वान् । रोहि॑ताञ्जि॒रिति॒ रोहि॑त - अ॒ञ्जिः॒ । अ॒न॒ड्वान॒धोरा॑मौ । अ॒धोरा॑मौ सावि॒त्रौ । अ॒धोरा॑मा॒वित्य॒धः - रा॒मौ॒ । सा॒वि॒त्रौ पौ॒ष्णौ । पौ॒ष्णौ र॑ज॒तना॑भी । र॒ज॒तना॑भी वैश्वदे॒वौ । र॒ज॒तना॑भी॒ इति॑ रज॒त - ना॒भी॒ । वै॒श्व॒दे॒वौ पि॒शङ्गौ᳚ । वै॒श्व॒दे॒वाविति॑ वैश्व - दे॒वौ । पि॒शङ्गौ॑ तूप॒रौ । तू॒प॒रौ मा॑रु॒तः । मा॒रु॒तः क॒ल्माषः॑ । क॒ल्माष॑ आग्ने॒यः । आ॒ग्ने॒यः कृ॒ष्णः । कृ॒ष्णो॑ऽजः । अ॒जः सा॑रस्व॒ती । सा॒र॒स्व॒ती मे॒षी । मे॒षी वा॑रु॒णः । वा॒रु॒णः कृ॒ष्णः । कृ॒ष्ण एक॑शितिपात् । एक॑शितिपा॒त् पेत्वः॑ । एक॑शितिपा॒दित्येक॑ - शि॒ति॒पा॒त् । पेत्व॒ इति॒ पेत्वः॑ । \newline

\textbf{Jatai Paata} \newline

1. अ॒ग्नये ऽनी॑कव॒ते ऽनी॑कवते॒ ऽग्नये॒ ऽग्नये ऽनी॑कवते । \newline
2. अनी॑कवते॒ रोहि॑ताञ्जी॒ रोहि॑ताञ्जि॒ रनी॑कव॒ते ऽनी॑कवते॒ रोहि॑ताञ्जिः । \newline
3. अनी॑कवत॒ इत्यनी॑क - व॒ते॒ । \newline
4. रोहि॑ताञ्जि रन॒ड्वा न॑न॒ड्वान् रोहि॑ताञ्जी॒ रोहि॑ताञ्जि रन॒ड्वान् । \newline
5. रोहि॑ताञ्जि॒रिति॒ रोहि॑त - अ॒ञ्जिः॒ । \newline
6. अ॒न॒ड्वा न॒धोरा॑मा व॒धोरा॑मा वन॒ड्वा न॑न॒ड्वा न॒धोरा॑मौ । \newline
7. अ॒धोरा॑मौ सावि॒त्रौ सा॑वि॒त्रा व॒धोरा॑मा व॒धोरा॑मौ सावि॒त्रौ । \newline
8. अ॒धोरा॑मा॒वित्य॒धः - रा॒मौ॒ । \newline
9. सा॒वि॒त्रौ पौ॒ष्णौ पौ॒ष्णौ सा॑वि॒त्रौ सा॑वि॒त्रौ पौ॒ष्णौ । \newline
10. पौ॒ष्णौ र॑ज॒तना॑भी रज॒तना॑भी पौ॒ष्णौ पौ॒ष्णौ र॑ज॒तना॑भी । \newline
11. र॒ज॒तना॑भी वैश्वदे॒वौ वै᳚श्वदे॒वौ र॑ज॒तना॑भी रज॒तना॑भी वैश्वदे॒वौ । \newline
12. र॒ज॒तना॑भी॒ इति॑ रज॒त - ना॒भी॒ । \newline
13. वै॒श्व॒दे॒वौ पि॒शङ्गौ॑ पि॒शङ्गौ॑ वैश्वदे॒वौ वै᳚श्वदे॒वौ पि॒शङ्गौ᳚ । \newline
14. वै॒श्व॒दे॒वाविति॑ वैश्व - दे॒वौ । \newline
15. पि॒शङ्गौ॑ तूप॒रौ तू॑प॒रौ पि॒शङ्गौ॑ पि॒शङ्गौ॑ तूप॒रौ । \newline
16. तू॒प॒रौ मा॑रु॒तो मा॑रु॒त स्तू॑प॒रौ तू॑प॒रौ मा॑रु॒तः । \newline
17. मा॒रु॒तः क॒ल्माषः॑ क॒ल्माषो॑ मारु॒तो मा॑रु॒तः क॒ल्माषः॑ । \newline
18. क॒ल्माष॑ आग्ने॒य आ᳚ग्ने॒यः क॒ल्माषः॑ क॒ल्माष॑ आग्ने॒यः । \newline
19. आ॒ग्ने॒यः कृ॒ष्णः कृ॒ष्ण आ᳚ग्ने॒य आ᳚ग्ने॒यः कृ॒ष्णः । \newline
20. कृ॒ष्णो᳚(1॒) ऽजो॑ ऽजः कृ॒ष्णः कृ॒ष्णो॑ ऽजः । \newline
21. अ॒जः सा॑रस्व॒ती सा॑रस्व॒ त्या᳚(1॒)जो॑ ऽजः सा॑रस्व॒ती । \newline
22. सा॒र॒स्व॒ती मे॒षी मे॒षी सा॑रस्व॒ती सा॑रस्व॒ती मे॒षी । \newline
23. मे॒षी वा॑रु॒णो वा॑रु॒णो मे॒षी मे॒षी वा॑रु॒णः । \newline
24. वा॒रु॒णः कृ॒ष्णः कृ॒ष्णो वा॑रु॒णो वा॑रु॒णः कृ॒ष्णः । \newline
25. कृ॒ष्ण एक॑शितिपा॒ देक॑शितिपात् कृ॒ष्णः कृ॒ष्ण एक॑शितिपात् । \newline
26. एक॑शितिपा॒त् पेत्वः॒ पेत्व॒ एक॑शितिपा॒ देक॑शितिपा॒त् पेत्वः॑ । \newline
27. एक॑शितिपा॒दित्येक॑ - शि॒ति॒पा॒त् । \newline
28. पेत्व॒ इति॒ पेत्वः॑ । \newline

\textbf{Ghana Paata } \newline

1. अ॒ग्नये ऽनी॑कव॒ते ऽनी॑कवते॒ ऽग्नये॒ ऽग्नये ऽनी॑कवते॒ रोहि॑ताञ्जी॒ रोहि॑ताञ्जि॒ रनी॑कवते॒ ऽग्नये॒ ऽग्नये ऽनी॑कवते॒ रोहि॑ताञ्जिः । \newline
2. अनी॑कवते॒ रोहि॑ताञ्जी॒ रोहि॑ताञ्जि॒ रनी॑कव॒ते ऽनी॑कवते॒ रोहि॑ताञ्जि रन॒ड्वा न॑न॒ड्वान् रोहि॑ताञ्जि॒ रनी॑कव॒ते ऽनी॑कवते॒ रोहि॑ताञ्जि रन॒ड्वान् । \newline
3. अनी॑कवत॒ इत्यनी॑क - व॒ते॒ । \newline
4. रोहि॑ताञ्जि रन॒ड्वा न॑न॒ड्वान् रोहि॑ताञ्जी॒ रोहि॑ताञ्जि रन॒ड्वा न॒धोरा॑मा व॒धोरा॑मा वन॒ड्वान् रोहि॑ताञ्जी॒ रोहि॑ताञ्जि रन॒ड्वा न॒धोरा॑मौ । \newline
5. रोहि॑ताञ्जि॒रिति॒ रोहि॑त - अ॒ञ्जिः॒ । \newline
6. अ॒न॒ड्वा न॒धोरा॑मा व॒धोरा॑मा वन॒ड्वा न॑न॒ड्वा न॒धोरा॑मौ सावि॒त्रौ सा॑वि॒त्रा व॒धोरा॑मा वन॒ड्वा न॑न॒ड्वा न॒धोरा॑मौ सावि॒त्रौ । \newline
7. अ॒धोरा॑मौ सावि॒त्रौ सा॑वि॒त्रा व॒धोरा॑मा व॒धोरा॑मौ सावि॒त्रौ पौ॒ष्णौ पौ॒ष्णौ सा॑वि॒त्रा व॒धोरा॑मा व॒धोरा॑मौ सावि॒त्रौ पौ॒ष्णौ । \newline
8. अ॒धोरा॑मा॒वित्य॒धः - रा॒मौ॒ । \newline
9. सा॒वि॒त्रौ पौ॒ष्णौ पौ॒ष्णौ सा॑वि॒त्रौ सा॑वि॒त्रौ पौ॒ष्णौ र॑ज॒तना॑भी रज॒तना॑भी पौ॒ष्णौ सा॑वि॒त्रौ सा॑वि॒त्रौ पौ॒ष्णौ र॑ज॒तना॑भी । \newline
10. पौ॒ष्णौ र॑ज॒तना॑भी रज॒तना॑भी पौ॒ष्णौ पौ॒ष्णौ र॑ज॒तना॑भी वैश्वदे॒वौ वै᳚श्वदे॒वौ र॑ज॒तना॑भी पौ॒ष्णौ पौ॒ष्णौ र॑ज॒तना॑भी वैश्वदे॒वौ । \newline
11. र॒ज॒तना॑भी वैश्वदे॒वौ वै᳚श्वदे॒वौ र॑ज॒तना॑भी रज॒तना॑भी वैश्वदे॒वौ पि॒शङ्गौ॑ पि॒शङ्गौ॑ वैश्वदे॒वौ र॑ज॒तना॑भी रज॒तना॑भी वैश्वदे॒वौ पि॒शङ्गौ᳚ । \newline
12. र॒ज॒तना॑भी॒ इति॑ रज॒त - ना॒भी॒ । \newline
13. वै॒श्व॒दे॒वौ पि॒शङ्गौ॑ पि॒शङ्गौ॑ वैश्वदे॒वौ वै᳚श्वदे॒वौ पि॒शङ्गौ॑ तूप॒रौ तू॑प॒रौ पि॒शङ्गौ॑ वैश्वदे॒वौ वै᳚श्वदे॒वौ पि॒शङ्गौ॑ तूप॒रौ । \newline
14. वै॒श्व॒दे॒वाविति॑ वैश्व - दे॒वौ । \newline
15. पि॒शङ्गौ॑ तूप॒रौ तू॑प॒रौ पि॒शङ्गौ॑ पि॒शङ्गौ॑ तूप॒रौ मा॑रु॒तो मा॑रु॒त स्तू॑प॒रौ पि॒शङ्गौ॑ पि॒शङ्गौ॑ तूप॒रौ मा॑रु॒तः । \newline
16. तू॒प॒रौ मा॑रु॒तो मा॑रु॒त स्तू॑प॒रौ तू॑प॒रौ मा॑रु॒तः क॒ल्माषः॑ क॒ल्माषो॑ मारु॒त स्तू॑प॒रौ तू॑प॒रौ मा॑रु॒तः क॒ल्माषः॑ । \newline
17. मा॒रु॒तः क॒ल्माषः॑ क॒ल्माषो॑ मारु॒तो मा॑रु॒तः क॒ल्माष॑ आग्ने॒य आ᳚ग्ने॒यः क॒ल्माषो॑ मारु॒तो मा॑रु॒तः क॒ल्माष॑ आग्ने॒यः । \newline
18. क॒ल्माष॑ आग्ने॒य आ᳚ग्ने॒यः क॒ल्माषः॑ क॒ल्माष॑ आग्ने॒यः कृ॒ष्णः कृ॒ष्ण आ᳚ग्ने॒यः क॒ल्माषः॑ क॒ल्माष॑ आग्ने॒यः कृ॒ष्णः । \newline
19. आ॒ग्ने॒यः कृ॒ष्णः कृ॒ष्ण आ᳚ग्ने॒य आ᳚ग्ने॒यः कृ॒ष्णो᳚(1॒) ऽजो॑ ऽजः कृ॒ष्ण आ᳚ग्ने॒य आ᳚ग्ने॒यः कृ॒ष्णो॑ ऽजः । \newline
20. कृ॒ष्णो᳚(1॒) ऽजो॑ ऽजः कृ॒ष्णः कृ॒ष्णो॑ ऽजः सा॑रस्व॒ती सा॑रस्व॒ त्य॑जः कृ॒ष्णः कृ॒ष्णो॑ ऽजः सा॑रस्व॒ती । \newline
21. अ॒जः सा॑रस्व॒ती सा॑रस्व॒त्या᳚(1॒)जो॑ ऽजः सा॑रस्व॒ती मे॒षी मे॒षी सा॑रस्व॒त्या᳚(1॒)जो॑ ऽजः सा॑रस्व॒ती मे॒षी । \newline
22. सा॒र॒स्व॒ती मे॒षी मे॒षी सा॑रस्व॒ती सा॑रस्व॒ती मे॒षी वा॑रु॒णो वा॑रु॒णो मे॒षी सा॑रस्व॒ती सा॑रस्व॒ती मे॒षी वा॑रु॒णः । \newline
23. मे॒षी वा॑रु॒णो वा॑रु॒णो मे॒षी मे॒षी वा॑रु॒णः कृ॒ष्णः कृ॒ष्णो वा॑रु॒णो मे॒षी मे॒षी वा॑रु॒णः कृ॒ष्णः । \newline
24. वा॒रु॒णः कृ॒ष्णः कृ॒ष्णो वा॑रु॒णो वा॑रु॒णः कृ॒ष्ण एक॑शितिपा॒ देक॑शितिपात् कृ॒ष्णो वा॑रु॒णो वा॑रु॒णः कृ॒ष्ण एक॑शितिपात् । \newline
25. कृ॒ष्ण एक॑शितिपा॒ देक॑शितिपात् कृ॒ष्णः कृ॒ष्ण एक॑शितिपा॒त् पेत्वः॒ पेत्व॒ एक॑शितिपात् कृ॒ष्णः कृ॒ष्ण एक॑शितिपा॒त् पेत्वः॑ । \newline
26. एक॑शितिपा॒त् पेत्वः॒ पेत्व॒ एक॑शितिपा॒ देक॑शितिपा॒त् पेत्वः॑ । \newline
27. एक॑शितिपा॒दित्येक॑ - शि॒ति॒पा॒त् । \newline
28. पेत्व॒ इति॒ पेत्वः॑ । \newline
\pagebreak


\end{document}