\documentclass[17pt]{extarticle}
\usepackage{babel}
\usepackage{fontspec}
\usepackage{polyglossia}
\usepackage{extsizes}

\usepackage{color}   %May be necessary if you want to color links
\usepackage{hyperref}
\hypersetup{
    colorlinks=true, %set true if you want colored links
    linktoc=all,     %set to all if you want both sections and subsections linked
    linkcolor=black,  %choose some color if you want links to stand out
}

\setmainlanguage{sanskrit}
\setotherlanguages{english} %% or other languages
\setlength{\parindent}{0pt}
\pagestyle{myheadings}
\newfontfamily\devanagarifont[Script=Devanagari]{AdishilaVedic}
\renewcommand{\theHsection}{\thepart.section.\thesection}

\newcommand{\VAR}[1]{}
\newcommand{\BLOCK}[1]{}




\begin{document}
\begin{titlepage}
    \begin{center}
 
\begin{sanskrit}
    { \Large
    कृष्ण यजुर्वेदीय तैत्तिरीय संहिता,पद,जटा,घन पाठः 
    }
    \\
    \vspace{2.5cm}
    \mbox{ \Large
    4.3      चतुर्थकाण्डे तृतीयः प्रश्नः - चितिवर्णनं   }
\end{sanskrit}
\end{center}

\end{titlepage}
\tableofcontents
\phantomsection
\pagebreak

\markright{ TS 4.3.1.1  \hfill https://www.vedavms.in \hfill}

\section{ TS 4.3.1.1 }

\textbf{TS 4.3.1.1 } \newline
\textbf{Samhita Paata} \newline

अ॒पां त्वेम᳚न्थ् सादयाम्य॒पां त्वोद्म᳚न्थ् सादयाम्य॒पां त्वा॒ भस्म᳚न्थ् सादयाम्य॒पां त्वा॒ ज्योति॑षि सादयाम्य॒पां त्वाऽय॑ने सादयाम्यर्ण॒वे सद॑ने सीद समु॒द्रे सद॑ने सीद सलि॒ले सद॑ने सीदा॒पां क्षये॑ सीदा॒पाꣳ सधि॑षि सीदा॒पां त्वा॒ सद॑ने सादयाम्य॒पां त्वा॑ स॒धस्थे॑ सादयाम्य॒पां त्वा॒ पुरी॑षे सादयाम्य॒पां त्वा॒ योनौ॑ ( ) सादयाम्य॒पां त्वा॒ पाथ॑सि सादयामि गाय॒त्री छन्द॑-स्त्रि॒ष्टुप् छन्दो॒ जग॑ती॒ छन्दो॑ऽनु॒ष्टुप् छन्दः॑ प॒ङ्क्तिश्छन्दः॑ ॥ \newline

\textbf{Pada Paata} \newline

अ॒पाम् । त्वा॒ । एमन्न्॑ । सा॒द॒या॒मि॒ । अ॒पाम् । त्वा॒ । ओद्मन्न्॑ । सा॒द॒या॒मि॒ । अ॒पाम् । त्वा॒ । भस्मन्न्॑ । सा॒द॒या॒मि॒ । अ॒पाम् । त्वा॒ । ज्योति॑षि । सा॒द॒या॒मि॒ । अ॒पाम् । त्वा॒ । अय॑ने । सा॒द॒या॒मि॒ । अ॒र्ण॒वे । सद॑ने । सी॒द॒ । स॒मु॒द्रे । सद॑ने । सी॒द॒ । स॒लि॒ले । सद॑ने । सी॒द॒ । अ॒पाम् । क्षये᳚ । सी॒द॒ । अ॒पाम् । सधि॑षि । सी॒द॒ । अ॒पाम् । त्वा॒ । सद॑ने । सा॒द॒या॒मि॒ । अ॒पाम् । त्वा॒ । स॒धस्थ॒ इति॑ स॒ध - स्थे॒ । सा॒द॒या॒मि॒ । अ॒पाम् । त्वा॒ । पुरी॑षे । सा॒द॒या॒मि॒ । अ॒पाम् । त्वा॒ । योनौ᳚ ( ) । सा॒द॒या॒मि॒ । अ॒पाम् । त्वा॒ । पाथ॑सि । सा॒द॒या॒मि॒ । गा॒य॒त्री । छन्दः॑ । त्रि॒ष्टुप् । छन्दः॑ । जग॑ती । छन्दः॑ । अ॒नु॒ष्टुबित्य॑नु - स्तुप् । छन्दः॑ । प॒ङ्क्तिः । छन्दः॑ ॥  \newline


\textbf{Krama Paata} \newline

अ॒पाम् त्वा᳚ । त्वेमन्न्॑ । एम᳚न्थ् सादयामि । सा॒द॒या॒म्य॒पाम् । अ॒पाम् त्वा᳚ । त्वोद्मन्न्॑ । ओद्म᳚न्थ् सादयामि । सा॒द॒या॒म्य॒पाम् । अ॒पाम् त्वा᳚ । त्वा॒ भस्मन्न्॑ । भस्म᳚न्थ् सादयामि । सा॒द॒या॒म्य॒पाम् । अ॒पाम् त्वा᳚ । त्वा॒ ज्योति॑षि । ज्योति॑षि सादयामि । सा॒द॒या॒म्य॒पाम् । अ॒पाम् त्वा᳚ । त्वाऽय॑ने । अय॑ने सादयामि । सा॒द॒या॒म्य॒र्ण॒वे । अ॒र्ण॒वे सद॑ने । सद॑ने सीद । सी॒द॒ स॒मु॒द्रे । स॒मु॒द्रे सद॑ने । सद॑ने सीद । सी॒द॒ स॒लि॒ले । स॒लि॒ले सद॑ने । सद॑ने सीद । सी॒दा॒पाम् । अ॒पाम् क्षये᳚ । क्षये॑ सीद । सी॒दा॒पाम् । अ॒पाꣳ सधि॑षि । सधि॑षि सीद । सी॒दा॒पाम् । अ॒पाम् त्वा᳚ । त्वा॒ सद॑ने । सद॑ने सादयामि । सा॒द॒या॒म्य॒पाम् । अ॒पाम् त्वा᳚ । त्वा॒ स॒धस्थे᳚ । स॒धस्थे॑ सादयामि । स॒धस्थ॒ इति॑ स॒ध - स्थे॒ । सा॒द॒या॒म्य॒पाम् । अ॒पाम् त्वा᳚ । त्वा॒ पुरी॑षे । पुरी॑षे सादयामि । सा॒द॒या॒म्य॒पाम् । अ॒पाम् त्वा᳚ । त्वा॒ योनौ᳚ ( ) । योनौ॑ सादयामि । सा॒द॒या॒म्य॒पाम् । अ॒पाम् त्वा᳚ । त्वा॒ पाथ॑सि । पाथ॑सि सादयामि । सा॒द॒या॒मि॒ गा॒य॒त्री । गा॒य॒त्री छन्दः॑ । छन्द॑स्त्रि॒ष्टुप् । त्रि॒ष्टुप् छन्दः॑ । छन्दो॒ जग॑ती । जग॑ती॒ छन्दः॑ । छन्दो॑ऽनु॒ष्टुप् । अ॒नु॒ष्टुप् छन्दः॑ । अ॒नु॒ष्टुबित्य॑नु - स्तुप् । छन्दः॑ प॒ङ्क्तिः । प॒ङ्क्ति श्छन्दः॑ । छन्द॒ इति॒ छन्दः॑ । \newline

\textbf{Jatai Paata} \newline

1. अ॒पाम् त्वा᳚ त्वा॒ ऽपा म॒पाम् त्वा᳚ । \newline
2. त्वेम॒न् नेम॑न् त्वा॒ त्वेमन्न्॑ । \newline
3. एमन्᳚ थ्सादयामि सादया॒ म्येम॒न् नेमन्᳚ थ्सादयामि । \newline
4. सा॒द॒या॒ म्य॒पा म॒पाꣳ सा॑दयामि सादया म्य॒पाम् । \newline
5. अ॒पाम् त्वा᳚ त्वा॒ ऽपा म॒पाम् त्वा᳚ । \newline
6. त्वोद्म॒न् नोद्म॑न् त्वा॒ त्वोद्मन्न्॑ । \newline
7. ओद्मन्᳚ थ्सादयामि सादया॒ म्योद्म॒न् नोद्मन्᳚ थ्सादयामि । \newline
8. सा॒द॒या॒ म्य॒पा म॒पाꣳ सा॑दयामि सादया म्य॒पाम् । \newline
9. अ॒पाम् त्वा᳚ त्वा॒ ऽपा म॒पाम् त्वा᳚ । \newline
10. त्वा॒ भस्म॒न् भस्म॑न् त्वा त्वा॒ भस्मन्न्॑ । \newline
11. भस्मन्᳚ थ्सादयामि सादयामि॒ भस्म॒न् भस्मन्᳚ थ्सादयामि । \newline
12. सा॒द॒या॒ म्य॒पा म॒पाꣳ सा॑दयामि सादया म्य॒पाम् । \newline
13. अ॒पाम् त्वा᳚ त्वा॒ ऽपा म॒पाम् त्वा᳚ । \newline
14. त्वा॒ ज्योति॑षि॒ ज्योति॑षि त्वा त्वा॒ ज्योति॑षि । \newline
15. ज्योति॑षि सादयामि सादयामि॒ ज्योति॑षि॒ ज्योति॑षि सादयामि । \newline
16. सा॒द॒या॒ म्य॒पा म॒पाꣳ सा॑दयामि सादया म्य॒पाम् । \newline
17. अ॒पाम् त्वा᳚ त्वा॒ ऽपा म॒पाम् त्वा᳚ । \newline
18. त्वा ऽय॒ने ऽय॑ने त्वा॒ त्वा ऽय॑ने । \newline
19. अय॑ने सादयामि सादया॒ म्यय॒ने ऽय॑ने सादयामि । \newline
20. सा॒द॒या॒ म्य॒र्ण॒वे᳚ ऽर्ण॒वे सा॑दयामि सादया म्यर्ण॒वे । \newline
21. अ॒र्ण॒वे सद॑ने॒ सद॑ने ऽर्ण॒वे᳚ ऽर्ण॒वे सद॑ने । \newline
22. सद॑ने सीद सीद॒ सद॑ने॒ सद॑ने सीद । \newline
23. सी॒द॒ स॒मु॒द्रे स॑मु॒द्रे सी॑द सीद समु॒द्रे । \newline
24. स॒मु॒द्रे सद॑ने॒ सद॑ने समु॒द्रे स॑मु॒द्रे सद॑ने । \newline
25. सद॑ने सीद सीद॒ सद॑ने॒ सद॑ने सीद । \newline
26. सी॒द॒ स॒लि॒ले स॑लि॒ले सी॑द सीद सलि॒ले । \newline
27. स॒लि॒ले सद॑ने॒ सद॑ने सलि॒ले स॑लि॒ले सद॑ने । \newline
28. सद॑ने सीद सीद॒ सद॑ने॒ सद॑ने सीद । \newline
29. सी॒दा॒पा म॒पाꣳ सी॑द सीदा॒पाम् । \newline
30. अ॒पाम् क्षये॒ क्षये॒ ऽपा म॒पाम् क्षये᳚ । \newline
31. क्षये॑ सीद सीद॒ क्षये॒ क्षये॑ सीद । \newline
32. सी॒दा॒पा म॒पाꣳ सी॑द सीदा॒पाम् । \newline
33. अ॒पाꣳ सधि॑षि॒ सधि॑ ष्य॒पा म॒पाꣳ सधि॑षि । \newline
34. सधि॑षि सीद सीद॒ सधि॑षि॒ सधि॑षि सीद । \newline
35. सी॒दा॒पा म॒पाꣳ सी॑द सीदा॒पाम् । \newline
36. अ॒पाम् त्वा᳚ त्वा॒ ऽपा म॒पाम् त्वा᳚ । \newline
37. त्वा॒ सद॑ने॒ सद॑ने त्वा त्वा॒ सद॑ने । \newline
38. सद॑ने सादयामि सादयामि॒ सद॑ने॒ सद॑ने सादयामि । \newline
39. सा॒द॒या॒ म्य॒पा म॒पाꣳ सा॑दयामि सादया म्य॒पाम् । \newline
40. अ॒पाम् त्वा᳚ त्वा॒ ऽपा म॒पाम् त्वा᳚ । \newline
41. त्वा॒ स॒धस्थे॑ स॒धस्थे᳚ त्वा त्वा स॒धस्थे᳚ । \newline
42. स॒धस्थे॑ सादयामि सादयामि स॒धस्थे॑ स॒धस्थे॑ सादयामि । \newline
43. स॒धस्थ॒ इति॑ स॒ध - स्थे॒ । \newline
44. सा॒द॒या॒ म्य॒पा म॒पाꣳ सा॑दयामि सादया म्य॒पाम् । \newline
45. अ॒पाम् त्वा᳚ त्वा॒ ऽपा म॒पाम् त्वा᳚ । \newline
46. त्वा॒ पुरी॑षे॒ पुरी॑षे त्वा त्वा॒ पुरी॑षे । \newline
47. पुरी॑षे सादयामि सादयामि॒ पुरी॑षे॒ पुरी॑षे सादयामि । \newline
48. सा॒द॒या॒ म्य॒पा म॒पाꣳ सा॑दयामि सादया म्य॒पाम् । \newline
49. अ॒पाम् त्वा᳚ त्वा॒ ऽपा म॒पाम् त्वा᳚ । \newline
50. त्वा॒ योनौ॒ योनौ᳚ त्वा त्वा॒ योनौ᳚ । \newline
51. योनौ॑ सादयामि सादयामि॒ योनौ॒ योनौ॑ सादयामि । \newline
52. सा॒द॒या॒ म्य॒पा म॒पाꣳ सा॑दयामि सादया म्य॒पाम् । \newline
53. अ॒पाम् त्वा᳚ त्वा॒ ऽपा म॒पाम् त्वा᳚ । \newline
54. त्वा॒ पाथ॑सि॒ पाथ॑सि त्वा त्वा॒ पाथ॑सि । \newline
55. पाथ॑सि सादयामि सादयामि॒ पाथ॑सि॒ पाथ॑सि सादयामि । \newline
56. सा॒द॒या॒मि॒ गा॒य॒त्री गा॑य॒त्री सा॑दयामि सादयामि गाय॒त्री । \newline
57. गा॒य॒त्री छन्द॒ श्छन्दो॑ गाय॒त्री गा॑य॒त्री छन्दः॑ । \newline
58. छन्द॑ स्त्रि॒ष्टुप् त्रि॒ष्टुप् छन्द॒ श्छन्द॑ स्त्रि॒ष्टुप् । \newline
59. त्रि॒ष्टुप् छन्द॒ श्छन्द॑ स्त्रि॒ष्टुप् त्रि॒ष्टुप् छन्दः॑ । \newline
60. छन्दो॒ जग॑ती॒ जग॑ती॒ छन्द॒ श्छन्दो॒ जग॑ती । \newline
61. जग॑ती॒ छन्द॒ श्छन्दो॒ जग॑ती॒ जग॑ती॒ छन्दः॑ । \newline
62. छन्दो॑ ऽनु॒ष्टु ब॑नु॒ष्टुप् छन्द॒ श्छन्दो॑ ऽनु॒ष्टुप् । \newline
63. अ॒नु॒ष्टुप् छन्द॒ श्छन्दो॑ ऽनु॒ष्टु ब॑नु॒ष्टुप् छन्दः॑ । \newline
64. अ॒नु॒ष्टुबित्य॑नु - स्तुप् । \newline
65. छन्दः॑ प॒ङ्क्तिः प॒ङ्क्ति श्छन्द॒ श्छन्दः॑ प॒ङ्क्तिः । \newline
66. प॒ङ्क्ति श्छन्द॒ श्छन्दः॑ प॒ङ्क्तिः प॒ङ्क्ति श्छन्दः॑ । \newline
67. छन्द॒ इति॒ छन्दः॑ । \newline

\textbf{Ghana Paata } \newline

1. अ॒पाम् त्वा᳚ त्वा॒ ऽपा म॒पाम् त्वेम॒न् नेम॑न् त्वा॒ ऽपा म॒पाम् त्वेमन्न्॑ । \newline
2. त्वेम॒न् नेम॑न् त्वा॒ त्वेमन्᳚ थ्सादयामि सादया॒ म्येम॑न् त्वा॒ त्वेमन्᳚ थ्सादयामि । \newline
3. एमन्᳚ थ्सादयामि सादया॒ म्येम॒न् नेमन्᳚ थ्सादया म्य॒पा म॒पाꣳ सा॑दया॒ म्येम॒न् नेमन्᳚ थ्सादया म्य॒पाम् । \newline
4. सा॒द॒या॒ म्य॒पा म॒पाꣳ सा॑दयामि सादया म्य॒पाम् त्वा᳚ त्वा॒ ऽपाꣳ सा॑दयामि सादया म्य॒पाम् त्वा᳚ । \newline
5. अ॒पाम् त्वा᳚ त्वा॒ ऽपा म॒पाम् त्वोद्म॒न् नोद्म॑न् त्वा॒ ऽपा म॒पाम् त्वोद्मन्न्॑ । \newline
6. त्वोद्म॒न् नोद्म॑न् त्वा॒ त्वोद्मन्᳚ थ्सादयामि सादया॒ म्योद्म॑न् त्वा॒ त्वोद्मन्᳚ थ्सादयामि । \newline
7. ओद्मन्᳚ थ्सादयामि सादया॒ म्योद्म॒न् नोद्मन्᳚ थ्सादया म्य॒पा म॒पाꣳ सा॑दया॒ म्योद्म॒न् नोद्मन्᳚ थ्सादया म्य॒पाम् । \newline
8. सा॒द॒या॒ म्य॒पा म॒पाꣳ सा॑दयामि सादया म्य॒पाम् त्वा᳚ त्वा॒ ऽपाꣳ सा॑दयामि सादया म्य॒पाम् त्वा᳚ । \newline
9. अ॒पाम् त्वा᳚ त्वा॒ ऽपा म॒पाम् त्वा॒ भस्म॒न् भस्म॑न् त्वा॒ ऽपा म॒पाम् त्वा॒ भस्मन्न्॑ । \newline
10. त्वा॒ भस्म॒न् भस्म॑न् त्वा त्वा॒ भस्मन्᳚ थ्सादयामि सादयामि॒ भस्म॑न् त्वा त्वा॒ भस्मन्᳚ थ्सादयामि । \newline
11. भस्मन्᳚ थ्सादयामि सादयामि॒ भस्म॒न् भस्मन्᳚ थ्सादया म्य॒पा म॒पाꣳ सा॑दयामि॒ भस्म॒न् भस्मन्᳚ थ्सादया म्य॒पाम् । \newline
12. सा॒द॒या॒ म्य॒पा म॒पाꣳ सा॑दयामि सादया म्य॒पाम् त्वा᳚ त्वा॒ ऽपाꣳ सा॑दयामि सादया म्य॒पाम् त्वा᳚ । \newline
13. अ॒पाम् त्वा᳚ त्वा॒ ऽपा म॒पाम् त्वा॒ ज्योति॑षि॒ ज्योति॑षि त्वा॒ ऽपा म॒पाम् त्वा॒ ज्योति॑षि । \newline
14. त्वा॒ ज्योति॑षि॒ ज्योति॑षि त्वा त्वा॒ ज्योति॑षि सादयामि सादयामि॒ ज्योति॑षि त्वा त्वा॒ ज्योति॑षि सादयामि । \newline
15. ज्योति॑षि सादयामि सादयामि॒ ज्योति॑षि॒ ज्योति॑षि सादया म्य॒पा म॒पाꣳ सा॑दयामि॒ ज्योति॑षि॒ ज्योति॑षि सादया म्य॒पाम् । \newline
16. सा॒द॒या॒ म्य॒पा म॒पाꣳ सा॑दयामि सादया म्य॒पाम् त्वा᳚ त्वा॒ ऽपाꣳ सा॑दयामि सादया म्य॒पाम् त्वा᳚ । \newline
17. अ॒पाम् त्वा᳚ त्वा॒ ऽपा म॒पाम् त्वा ऽय॒ने ऽय॑ने त्वा॒ ऽपा म॒पाम् त्वा ऽय॑ने । \newline
18. त्वा ऽय॒ने ऽय॑ने त्वा॒ त्वा ऽय॑ने सादयामि सादया॒ म्यय॑ने त्वा॒ त्वा ऽय॑ने सादयामि । \newline
19. अय॑ने सादयामि सादया॒ म्यय॒ने ऽय॑ने सादया म्यर्ण॒वे᳚ ऽर्ण॒वे सा॑दया॒ म्यय॒ने ऽय॑ने सादया म्यर्ण॒वे । \newline
20. सा॒द॒या॒ म्य॒र्ण॒वे᳚ ऽर्ण॒वे सा॑दयामि सादया म्यर्ण॒वे सद॑ने॒ सद॑ने ऽर्ण॒वे सा॑दयामि सादया म्यर्ण॒वे सद॑ने । \newline
21. अ॒र्ण॒वे सद॑ने॒ सद॑ने ऽर्ण॒वे᳚ ऽर्ण॒वे सद॑ने सीद सीद॒ सद॑ने ऽर्ण॒वे᳚ ऽर्ण॒वे सद॑ने सीद । \newline
22. सद॑ने सीद सीद॒ सद॑ने॒ सद॑ने सीद समु॒द्रे स॑मु॒द्रे सी॑द॒ सद॑ने॒ सद॑ने सीद समु॒द्रे । \newline
23. सी॒द॒ स॒मु॒द्रे स॑मु॒द्रे सी॑द सीद समु॒द्रे सद॑ने॒ सद॑ने समु॒द्रे सी॑द सीद समु॒द्रे सद॑ने । \newline
24. स॒मु॒द्रे सद॑ने॒ सद॑ने समु॒द्रे स॑मु॒द्रे सद॑ने सीद सीद॒ सद॑ने समु॒द्रे स॑मु॒द्रे सद॑ने सीद । \newline
25. सद॑ने सीद सीद॒ सद॑ने॒ सद॑ने सीद सलि॒ले स॑लि॒ले सी॑द॒ सद॑ने॒ सद॑ने सीद सलि॒ले । \newline
26. सी॒द॒ स॒लि॒ले स॑लि॒ले सी॑द सीद सलि॒ले सद॑ने॒ सद॑ने सलि॒ले सी॑द सीद सलि॒ले सद॑ने । \newline
27. स॒लि॒ले सद॑ने॒ सद॑ने सलि॒ले स॑लि॒ले सद॑ने सीद सीद॒ सद॑ने सलि॒ले स॑लि॒ले सद॑ने सीद । \newline
28. सद॑ने सीद सीद॒ सद॑ने॒ सद॑ने सीदा॒पा म॒पाꣳ सी॑द॒ सद॑ने॒ सद॑ने सीदा॒पाम् । \newline
29. सी॒दा॒पा म॒पाꣳ सी॑द सीदा॒पाम् क्षये॒ क्षये॒ ऽपाꣳ सी॑द सीदा॒पाम् क्षये᳚ । \newline
30. अ॒पाम् क्षये॒ क्षये॒ ऽपा म॒पाम् क्षये॑ सीद सीद॒ क्षये॒ ऽपा म॒पाम् क्षये॑ सीद । \newline
31. क्षये॑ सीद सीद॒ क्षये॒ क्षये॑ सीदा॒पा म॒पाꣳ सी॑द॒ क्षये॒ क्षये॑ सीदा॒पाम् । \newline
32. सी॒दा॒पा म॒पाꣳ सी॑द सीदा॒पाꣳ सधि॑षि॒ सधि॑ ष्य॒पाꣳ सी॑द सीदा॒पाꣳ सधि॑षि । \newline
33. अ॒पाꣳ सधि॑षि॒ सधि॑ ष्य॒पा म॒पाꣳ सधि॑षि सीद सीद॒ सधि॑ ष्य॒पा म॒पाꣳ सधि॑षि सीद । \newline
34. सधि॑षि सीद सीद॒ सधि॑षि॒ सधि॑षि सीदा॒पा म॒पाꣳ सी॑द॒ सधि॑षि॒ सधि॑षि सीदा॒पाम् । \newline
35. सी॒दा॒पा म॒पाꣳ सी॑द सीदा॒पाम् त्वा᳚ त्वा॒ ऽपाꣳ सी॑द सीदा॒पाम् त्वा᳚ । \newline
36. अ॒पाम् त्वा᳚ त्वा॒ ऽपा म॒पाम् त्वा॒ सद॑ने॒ सद॑ने त्वा॒ ऽपा म॒पाम् त्वा॒ सद॑ने । \newline
37. त्वा॒ सद॑ने॒ सद॑ने त्वा त्वा॒ सद॑ने सादयामि सादयामि॒ सद॑ने त्वा त्वा॒ सद॑ने सादयामि । \newline
38. सद॑ने सादयामि सादयामि॒ सद॑ने॒ सद॑ने सादया म्य॒पा म॒पाꣳ सा॑दयामि॒ सद॑ने॒ सद॑ने सादया म्य॒पाम् । \newline
39. सा॒द॒या॒ म्य॒पा म॒पाꣳ सा॑दयामि सादया म्य॒पाम् त्वा᳚ त्वा॒ ऽपाꣳ सा॑दयामि सादया म्य॒पाम् त्वा᳚ । \newline
40. अ॒पाम् त्वा᳚ त्वा॒ ऽपा म॒पाम् त्वा॑ स॒धस्थे॑ स॒धस्थे᳚ त्वा॒ ऽपा म॒पाम् त्वा॑ स॒धस्थे᳚ । \newline
41. त्वा॒ स॒धस्थे॑ स॒धस्थे᳚ त्वा त्वा स॒धस्थे॑ सादयामि सादयामि स॒धस्थे᳚ त्वा त्वा स॒धस्थे॑ सादयामि । \newline
42. स॒धस्थे॑ सादयामि सादयामि स॒धस्थे॑ स॒धस्थे॑ सादया म्य॒पा म॒पाꣳ सा॑दयामि स॒धस्थे॑ स॒धस्थे॑ सादया म्य॒पाम् । \newline
43. स॒धस्थ॒ इति॑ स॒ध - स्थे॒ । \newline
44. सा॒द॒या॒ म्य॒पा म॒पाꣳ सा॑दयामि सादया म्य॒पाम् त्वा᳚ त्वा॒ ऽपाꣳ सा॑दयामि सादया म्य॒पाम् त्वा᳚ । \newline
45. अ॒पाम् त्वा᳚ त्वा॒ ऽपा म॒पाम् त्वा॒ पुरी॑षे॒ पुरी॑षे त्वा॒ ऽपा म॒पाम् त्वा॒ पुरी॑षे । \newline
46. त्वा॒ पुरी॑षे॒ पुरी॑षे त्वा त्वा॒ पुरी॑षे सादयामि सादयामि॒ पुरी॑षे त्वा त्वा॒ पुरी॑षे सादयामि । \newline
47. पुरी॑षे सादयामि सादयामि॒ पुरी॑षे॒ पुरी॑षे सादया म्य॒पा म॒पाꣳ सा॑दयामि॒ पुरी॑षे॒ पुरी॑षे सादया म्य॒पाम् । \newline
48. सा॒द॒या॒ म्य॒पा म॒पाꣳ सा॑दयामि सादया म्य॒पाम् त्वा᳚ त्वा॒ ऽपाꣳ सा॑दयामि सादया म्य॒पाम् त्वा᳚ । \newline
49. अ॒पाम् त्वा᳚ त्वा॒ ऽपा म॒पाम् त्वा॒ योनौ॒ योनौ᳚ त्वा॒ ऽपा म॒पाम् त्वा॒ योनौ᳚ । \newline
50. त्वा॒ योनौ॒ योनौ᳚ त्वा त्वा॒ योनौ॑ सादयामि सादयामि॒ योनौ᳚ त्वा त्वा॒ योनौ॑ सादयामि । \newline
51. योनौ॑ सादयामि सादयामि॒ योनौ॒ योनौ॑ सादया म्य॒पा म॒पाꣳ सा॑दयामि॒ योनौ॒ योनौ॑ सादया म्य॒पाम् । \newline
52. सा॒द॒या॒ म्य॒पा म॒पाꣳ सा॑दयामि सादया म्य॒पाम् त्वा᳚ त्वा॒ ऽपाꣳ सा॑दयामि सादया म्य॒पाम् त्वा᳚ । \newline
53. अ॒पाम् त्वा᳚ त्वा॒ ऽपा म॒पाम् त्वा॒ पाथ॑सि॒ पाथ॑सि त्वा॒ ऽपा म॒पाम् त्वा॒ पाथ॑सि । \newline
54. त्वा॒ पाथ॑सि॒ पाथ॑सि त्वा त्वा॒ पाथ॑सि सादयामि सादयामि॒ पाथ॑सि त्वा त्वा॒ पाथ॑सि सादयामि । \newline
55. पाथ॑सि सादयामि सादयामि॒ पाथ॑सि॒ पाथ॑सि सादयामि गाय॒त्री गा॑य॒त्री सा॑दयामि॒ पाथ॑सि॒ पाथ॑सि सादयामि गाय॒त्री । \newline
56. सा॒द॒या॒मि॒ गा॒य॒त्री गा॑य॒त्री सा॑दयामि सादयामि गाय॒त्री छन्द॒ श्छन्दो॑ गाय॒त्री सा॑दयामि सादयामि गाय॒त्री छन्दः॑ । \newline
57. गा॒य॒त्री छन्द॒ श्छन्दो॑ गाय॒त्री गा॑य॒त्री छन्द॑ स्त्रि॒ष्टुप् त्रि॒ष्टुप् छन्दो॑ गाय॒त्री गा॑य॒त्री छन्द॑ स्त्रि॒ष्टुप् । \newline
58. छन्द॑ स्त्रि॒ष्टुप् त्रि॒ष्टुप् छन्द॒ श्छन्द॑ स्त्रि॒ष्टुप् छन्द॒ श्छन्द॑ स्त्रि॒ष्टुप् छन्द॒ श्छन्द॑ स्त्रि॒ष्टुप् छन्दः॑ । \newline
59. त्रि॒ष्टुप् छन्द॒ श्छन्द॑ स्त्रि॒ष्टुप् त्रि॒ष्टुप् छन्दो॒ जग॑ती॒ जग॑ती॒ छन्द॑ स्त्रि॒ष्टुप् त्रि॒ष्टुप् छन्दो॒ जग॑ती । \newline
60. छन्दो॒ जग॑ती॒ जग॑ती॒ छन्द॒ श्छन्दो॒ जग॑ती॒ छन्द॒ श्छन्दो॒ जग॑ती॒ छन्द॒ श्छन्दो॒ जग॑ती॒ छन्दः॑ । \newline
61. जग॑ती॒ छन्द॒ श्छन्दो॒ जग॑ती॒ जग॑ती॒ छन्दो॑ ऽनु॒ष्टु ब॑नु॒ष्टुप् छन्दो॒ जग॑ती॒ जग॑ती॒ छन्दो॑ ऽनु॒ष्टुप् । \newline
62. छन्दो॑ ऽनु॒ष्टु ब॑नु॒ष्टुप् छन्द॒ श्छन्दो॑ ऽनु॒ष्टुप् छन्द॒ श्छन्दो॑ ऽनु॒ष्टुप् छन्द॒ श्छन्दो॑ ऽनु॒ष्टुप् छन्दः॑ । \newline
63. अ॒नु॒ष्टुप् छन्द॒ श्छन्दो॑ ऽनु॒ष्टु ब॑नु॒ष्टुप् छन्दः॑ प॒ङ्क्तिः प॒ङ्क्ति श्छन्दो॑ ऽनु॒ष्टु ब॑नु॒ष्टुप् छन्दः॑ प॒ङ्क्तिः । \newline
64. अ॒नु॒ष्टुबित्य॑नु - स्तुप् । \newline
65. छन्दः॑ प॒ङ्क्तिः प॒ङ्क्ति श्छन्द॒ श्छन्दः॑ प॒ङ्क्ति श्छन्द॒ श्छन्दः॑ प॒ङ्क्ति श्छन्द॒ श्छन्दः॑ प॒ङ्क्ति श्छन्दः॑ । \newline
66. प॒ङ्क्ति श्छन्द॒ श्छन्दः॑ प॒ङ्क्तिः प॒ङ्क्ति श्छन्दः॑ । \newline
67. छन्द॒ इति॒ छन्दः॑ । \newline
\pagebreak
\markright{ TS 4.3.2.1  \hfill https://www.vedavms.in \hfill}

\section{ TS 4.3.2.1 }

\textbf{TS 4.3.2.1 } \newline
\textbf{Samhita Paata} \newline

अ॒यं पु॒रो भुव॒स्तस्य॑ प्रा॒णो भौ॑वाय॒नो व॑स॒न्तः प्रा॑णाय॒नो गा॑य॒त्री वा॑स॒न्ती गा॑यत्रि॒यै गा॑य॒त्रं गा॑य॒त्रादु॑पाꣳ॒॒ शुरु॑पाꣳ॒॒ शोस्त्रि॒वृत् त्रि॒वृतो॑ रथन्त॒रꣳ र॑थन्त॒राद्-वसि॑ष्ठ॒ ऋषिः॑ प्र॒जाप॑ति गृहीतया॒ त्वया᳚ प्रा॒णं गृ॑ह्णामि प्र॒जाभ्यो॒ऽयं द॑क्षि॒णा वि॒श्वक॑र्मा॒ तस्य॒ मनो॑ वैश्वकर्म॒णं ग्री॒ष्मो मा॑न॒सस्त्रि॒ष्टुग्ग्रै॒ष्मी त्रि॒ष्टुभ॑ ऐ॒डमै॒डा-द॑न्तर्या॒मो᳚ ऽन्तर्या॒मात् प॑ञ्चद॒शः प॑ञ्चद॒शाद्-बृ॒हद्-बृ॑ह॒तो भ॒रद्वा॑ज॒ ऋषिः॑ प्र॒जाप॑ति गृहीतया॒ त्वया॒ मनो॑ - [  ] \newline

\textbf{Pada Paata} \newline

अ॒यम् । पु॒रः । भुवः॑ । तस्य॑ । प्रा॒ण इति॑ प्र - अ॒नः । भौ॒वा॒य॒नः । व॒स॒न्तः । प्रा॒णा॒य॒नः । गा॒य॒त्री । वा॒स॒न्ती । गा॒य॒त्रि॒यै । गा॒य॒त्रम् । गा॒य॒त्रात् । उ॒पाꣳ॒॒शुरित्यु॑प - अꣳ॒॒शुः । उ॒पाꣳ॒॒शोरित्यु॑प-अꣳ॒॒शोः । त्रि॒वृदिति॑ त्रि - वृत् । त्रि॒वृत॒ इति॑ त्रि - वृतः॑ । र॒थ॒न्त॒रमिति॑ रथं - त॒रम् । र॒थ॒न्त॒रादिति॑ रथं - त॒रात् । वसि॑ष्ठः । ऋषिः॑ । प्र॒जाप॑तिगृहीत॒येति॑ प्र॒जाप॑ति - गृ॒ही॒त॒या॒ । त्वया᳚ । प्रा॒णमिति॑ प्र - अ॒नम् । गृ॒ह्णा॒मि॒ । प्र॒जाभ्य॒ इति॑ प्र - जाभ्यः॑ । अ॒यम् । द॒क्षि॒णा । वि॒श्वक॒र्मेति॑ वि॒श्व - क॒र्मा॒ । तस्य॑ । मनः॑ । वै॒श्व॒क॒र्म॒णमिति॑ वैश्व - क॒र्म॒णम् । ग्री॒ष्मः । मा॒न॒सः । त्रि॒ष्टुक् । ग्रै॒ष्मी । त्रि॒ष्टुभः॑ । ऐ॒डम् । ऐ॒डात् । अ॒न्त॒र्या॒म इत्य॑न्तः - या॒मः । अ॒न्त॒र्या॒मादित्य॑न्तः - या॒मात् । प॒ञ्च॒द॒श इति॑ पञ्च - द॒शः । प॒ञ्च॒द॒शादिति॑ पञ्च-द॒शात् । बृ॒हत् । बृ॒ह॒तः । भ॒रद्वा॑जः । ऋषिः॑ । प्र॒जाप॑तिगृहीत॒येति॑ प्र॒जाप॑ति - गृ॒ही॒त॒या॒ । त्वया᳚ । मनः॑ ।  \newline


\textbf{Krama Paata} \newline

अ॒यम् पु॒रः । पु॒रो भुवः॑ । भुव॒स्तस्य॑ । तस्य॑ प्रा॒णः । प्रा॒णो भौ॑वाय॒नः । प्रा॒ण इति॑ प्र - अ॒नः । भौ॒वा॒य॒नो व॑स॒न्तः । व॒स॒न्तः प्रा॑णाय॒नः । प्रा॒णा॒य॒नो गा॑य॒त्री । गा॒य॒त्री वा॑स॒न्ती । वा॒स॒न्ती गा॑यत्रि॒यै । गा॒य॒त्रि॒यै गा॑य॒त्रम् । गा॒य॒त्रम् गा॑य॒त्रात् । गा॒य॒त्रादु॑पाꣳ॒॒शुः । उ॒पाꣳ॒॒शुरु॑पाꣳ॒॒शोः । उ॒पाꣳ॒॒शुरित्यु॑प - अꣳ॒॒शुः । उ॒पाꣳ॒॒शोस्त्रि॒वृत् । उ॒पाꣳ॒॒शोरित्यु॑प - अꣳ॒॒शोः । त्रि॒वृत् त्रि॒वृतः॑ । त्रि॒वृदिति॑ त्रि - वृत् । त्रि॒वृतो॑ रथन्त॒रम् । त्रि॒वृत॒ इति॑ त्रि - वृतः॑ । र॒थ॒न्त॒रꣳ र॑थन्त॒रात् । र॒थ॒न्त॒रमिति॑ रथम् - त॒रम् । र॒थ॒न्त॒राद् वसि॑ष्ठः । र॒थ॒न्त॒रादिति॑ रथम् - त॒रात् । वसि॑ष्ठ॒ ऋषिः॑ । ऋषिः॑ प्र॒जाप॑तिगृहीतया । प्र॒जाप॑तिगृहीतया॒ त्वया᳚ । प्र॒जाप॑तिगृहीत॒येति॑ प्र॒जाप॑ति - गृ॒ही॒त॒या॒ । त्वया᳚ प्रा॒णम् । प्रा॒णम् गृ॑ह्णामि । प्रा॒णमिति॑ प्र - अ॒नम् । गृ॒ह्णा॒मि॒ प्र॒जाभ्यः॑ । प्र॒जाभ्यो॒ऽयम् । प्र॒जाभ्य॒ इति॑ प्र - जाभ्यः॑ । अ॒यम् द॑क्षि॒णा । द॒क्षि॒णा वि॒श्वक॑र्मा । वि॒श्वक॑र्मा॒ तस्य॑ । वि॒श्वक॒र्मेति॑ वि॒श्व - क॒र्मा॒ । तस्य॒ मनः॑ । मनो॑ वैश्वकर्म॒णम् । वै॒श्व॒क॒र्म॒णम् ग्री॒ष्मः । वै॒श्व॒क॒र्म॒णमिति॑ वैश्व - क॒र्म॒णम् । ग्री॒ष्मो मा॑न॒सः । मा॒न॒सस्त्रि॒ष्टुक् । त्रि॒ष्टुग् ग्रै॒ष्मी । ग्रै॒ष्मी त्रि॒ष्टुभः॑ । त्रि॒ष्टुभ॑ ऐ॒डम् । ऐ॒डमै॒डात् । ऐ॒डाद॑न्तर्या॒मः । अ॒न्त॒र्या॒मो᳚ऽन्तर्या॒मात् । अ॒न्त॒र्या॒म इत्य॑न्तः - या॒मः । अ॒न्त॒र्या॒मात् प॑ञ्चद॒शः । अ॒न्त॒र्या॒मादित्य॑न्तः - या॒मात् । प॒ञ्च॒द॒शः प॑ञ्चद॒शात् । प॒ञ्च॒द॒श इति॑ पञ्च - द॒शः । प॒ञ्च॒द॒शाद् बृ॒हत् । प॒ञ्च॒द॒शादिति॑ पञ्च - द॒शात् । बृ॒हद् बृ॑ह॒तः । बृ॒ह॒तो भ॒रद्वा॑जः । भ॒रद्वा॑ज॒ ऋषिः॑ । ऋषिः॑ प्र॒जाप॑तिगृहीतया । प्र॒जाप॑तिगृहीतया॒ त्वया᳚ । प्र॒जाप॑तिगृहीत॒येति॑ प्र॒जाप॑ति - गृ॒ही॒त॒या॒ । त्वया॒ मनः॑ । मनो॑ गृह्णामि \newline

\textbf{Jatai Paata} \newline

1. अ॒यम् पु॒रः पु॒रो॑ ऽय म॒यम् पु॒रः । \newline
2. पु॒रो भुवो॒ भुवः॑ पु॒रः पु॒रो भुवः॑ । \newline
3. भुव॒ स्तस्य॒ तस्य॒ भुवो॒ भुव॒ स्तस्य॑ । \newline
4. तस्य॑ प्रा॒णः प्रा॒ण स्तस्य॒ तस्य॑ प्रा॒णः । \newline
5. प्रा॒णो भौ॑वाय॒नो भौ॑वाय॒नः प्रा॒णः प्रा॒णो भौ॑वाय॒नः । \newline
6. प्रा॒ण इति॑ प्र - अ॒नः । \newline
7. भौ॒वा॒य॒नो व॑स॒न्तो व॑स॒न्तो भौ॑वाय॒नो भौ॑वाय॒नो व॑स॒न्तः । \newline
8. व॒स॒न्तः प्रा॑णाय॒नः प्रा॑णाय॒नो व॑स॒न्तो व॑स॒न्तः प्रा॑णाय॒नः । \newline
9. प्रा॒णा॒य॒नो गा॑य॒त्री गा॑य॒त्री प्रा॑णाय॒नः प्रा॑णाय॒नो गा॑य॒त्री । \newline
10. गा॒य॒त्री वा॑स॒न्ती वा॑स॒न्ती गा॑य॒त्री गा॑य॒त्री वा॑स॒न्ती । \newline
11. वा॒स॒न्ती गा॑यत्रि॒यै गा॑यत्रि॒यै वा॑स॒न्ती वा॑स॒न्ती गा॑यत्रि॒यै । \newline
12. गा॒य॒त्रि॒यै गा॑य॒त्रम् गा॑य॒त्रम् गा॑यत्रि॒यै गा॑यत्रि॒यै गा॑य॒त्रम् । \newline
13. गा॒य॒त्रम् गा॑य॒त्राद् गा॑य॒त्राद् गा॑य॒त्रम् गा॑य॒त्रम् गा॑य॒त्रात् । \newline
14. गा॒य॒त्रा दु॑पाꣳ॒॒शु रु॑पाꣳ॒॒शुर् गा॑य॒त्राद् गा॑य॒त्रा दु॑पाꣳ॒॒शुः । \newline
15. उ॒पाꣳ॒॒शु रु॑पाꣳ॒॒शो रु॑पाꣳ॒॒शो रु॑पाꣳ॒॒शु रु॑पाꣳ॒॒शु रु॑पाꣳ॒॒शोः । \newline
16. उ॒पाꣳ॒॒शुरित्यु॑प - अꣳ॒॒शुः । \newline
17. उ॒पाꣳ॒॒शो स्त्रि॒वृत् त्रि॒वृ दु॑पाꣳ॒॒शो रु॑पाꣳ॒॒शो स्त्रि॒वृत् । \newline
18. उ॒पाꣳ॒॒शोरित्यु॑प - अꣳ॒॒शोः । \newline
19. त्रि॒वृत् त्रि॒वृत॑ स्त्रि॒वृत॑ स्त्रि॒वृत् त्रि॒वृत् त्रि॒वृतः॑ । \newline
20. त्रि॒वृदिति॑ त्रि - वृत् । \newline
21. त्रि॒वृतो॑ रथन्त॒रꣳ र॑थन्त॒रम् त्रि॒वृत॑ स्त्रि॒वृतो॑ रथन्त॒रम् । \newline
22. त्रि॒वृत॒ इति॑ त्रि - वृतः॑ । \newline
23. र॒थ॒न्त॒रꣳ र॑थन्त॒राद् र॑थन्त॒राद् र॑थन्त॒रꣳ र॑थन्त॒रꣳ र॑थन्त॒रात् । \newline
24. र॒थ॒न्त॒रमिति॑ रथं - त॒रम् । \newline
25. र॒थ॒न्त॒राद् वसि॑ष्ठो॒ वसि॑ष्ठो रथन्त॒राद् र॑थन्त॒राद् वसि॑ष्ठः । \newline
26. र॒थ॒न्त॒रादिति॑ रथं - त॒रात् । \newline
27. वसि॑ष्ठ॒ ऋषि॒र्॒. ऋषि॒र् वसि॑ष्ठो॒ वसि॑ष्ठ॒ ऋषिः॑ । \newline
28. ऋषिः॑ प्र॒जाप॑तिगृहीतया प्र॒जाप॑तिगृहीत॒य र्.षि॒र्॒.ऋषिः॑ प्र॒जाप॑तिगृहीतया । \newline
29. प्र॒जाप॑तिगृहीतया॒ त्वया॒ त्वया᳚ प्र॒जाप॑तिगृहीतया प्र॒जाप॑तिगृहीतया॒ त्वया᳚ । \newline
30. प्र॒जाप॑तिगृहीत॒येति॑ प्र॒जाप॑ति - गृ॒ही॒त॒या॒ । \newline
31. त्वया᳚ प्रा॒णम् प्रा॒णम् त्वया॒ त्वया᳚ प्रा॒णम् । \newline
32. प्रा॒णम् गृ॑ह्णामि गृह्णामि प्रा॒णम् प्रा॒णम् गृ॑ह्णामि । \newline
33. प्रा॒णमिति॑ प्र - अ॒नम् । \newline
34. गृ॒ह्णा॒मि॒ प्र॒जाभ्यः॑ प्र॒जाभ्यो॑ गृह्णामि गृह्णामि प्र॒जाभ्यः॑ । \newline
35. प्र॒जाभ्यो॒ ऽय म॒यम् प्र॒जाभ्यः॑ प्र॒जाभ्यो॒ ऽयम् । \newline
36. प्र॒जाभ्य॒ इति॑ प्र - जाभ्यः॑ । \newline
37. अ॒यम् द॑क्षि॒णा द॑क्षि॒णा ऽय म॒यम् द॑क्षि॒णा । \newline
38. द॒क्षि॒णा वि॒श्वक॑र्मा वि॒श्वक॑र्मा दक्षि॒णा द॑क्षि॒णा वि॒श्वक॑र्मा । \newline
39. वि॒श्वक॑र्मा॒ तस्य॒ तस्य॑ वि॒श्वक॑र्मा वि॒श्वक॑र्मा॒ तस्य॑ । \newline
40. वि॒श्वक॒र्मेति॑ वि॒श्व - क॒र्मा॒ । \newline
41. तस्य॒ मनो॒ मन॒ स्तस्य॒ तस्य॒ मनः॑ । \newline
42. मनो॑ वैश्वकर्म॒णं ॅवै᳚श्वकर्म॒णम् मनो॒ मनो॑ वैश्वकर्म॒णम् । \newline
43. वै॒श्व॒क॒र्म॒णम् ग्री॒ष्मो ग्री॒ष्मो वै᳚श्वकर्म॒णं ॅवै᳚श्वकर्म॒णम् ग्री॒ष्मः । \newline
44. वै॒श्व॒क॒र्म॒णमिति॑ वैश्व - क॒र्म॒णम् । \newline
45. ग्री॒ष्मो मा॑न॒सो मा॑न॒सो ग्री॒ष्मो ग्री॒ष्मो मा॑न॒सः । \newline
46. मा॒न॒स स्त्रि॒ष्टुक् त्रि॒ष्टुङ् मा॑न॒सो मा॑न॒ सस्त्रि॒ष्टुक् । \newline
47. त्रि॒ष्टुग् ग्रै॒ष्मी ग्रै॒ष्मी त्रि॒ष्टुक् त्रि॒ष्टुग् ग्रै॒ष्मी । \newline
48. ग्रै॒ष्मी त्रि॒ष्टुभ॑ स्त्रि॒ष्टुभो᳚ ग्रै॒ष्मी ग्रै॒ष्मी त्रि॒ष्टुभः॑ । \newline
49. त्रि॒ष्टुभ॑ ऐ॒ड मै॒डम् त्रि॒ष्टुभ॑ स्त्रि॒ष्टुभ॑ ऐ॒डम् । \newline
50. ऐ॒ड मै॒डा दै॒डा दै॒ड मै॒ड मै॒डात् । \newline
51. ऐ॒डा द॑न्तर्या॒मो᳚ ऽन्तर्या॒म ऐ॒डा दै॒डा द॑न्तर्या॒मः । \newline
52. अ॒न्त॒र्या॒मो᳚ ऽन्तर्या॒मा द॑न्तर्या॒मा द॑न्तर्या॒मो᳚ ऽन्तर्या॒मो᳚ ऽन्तर्या॒मात् । \newline
53. अ॒न्त॒र्या॒म इत्य॑न्तः - या॒मः । \newline
54. अ॒न्त॒र्या॒मात् प॑ञ्चद॒शः प॑ञ्चद॒शो᳚ ऽन्तर्या॒मा द॑न्तर्या॒मात् प॑ञ्चद॒शः । \newline
55. अ॒न्त॒र्या॒मादित्य॑न्तः - या॒मात् । \newline
56. प॒ञ्च॒द॒शः प॑ञ्चद॒शात् प॑ञ्चद॒शात् प॑ञ्चद॒शः प॑ञ्चद॒शः प॑ञ्चद॒शात् । \newline
57. प॒ञ्च॒द॒श इति॑ पञ्च - द॒शः । \newline
58. प॒ञ्च॒द॒शाद् बृ॒हद् बृ॒हत् प॑ञ्चद॒शात् प॑ञ्चद॒शाद् बृ॒हत् । \newline
59. प॒ञ्च॒द॒शादिति॑ पञ्च - द॒शात् । \newline
60. बृ॒हद् बृ॑ह॒तो बृ॑ह॒तो बृ॒हद् बृ॒हद् बृ॑ह॒तः । \newline
61. बृ॒ह॒तो भ॒रद्वा॑जो भ॒रद्वा॑जो बृह॒तो बृ॑ह॒तो भ॒रद्वा॑जः । \newline
62. भ॒रद्वा॑ज॒ ऋषि॒र्॒. ऋषि॑र् भ॒रद्वा॑जो भ॒रद्वा॑ज॒ ऋषिः॑ । \newline
63. ऋषिः॑ प्र॒जाप॑तिगृहीतया प्र॒जाप॑तिगृहीत॒य र्.षि॒र्॒. ऋषिः॑ प्र॒जाप॑तिगृहीतया । \newline
64. प्र॒जाप॑तिगृहीतया॒ त्वया॒ त्वया᳚ प्र॒जाप॑तिगृहीतया प्र॒जाप॑तिगृहीतया॒ त्वया᳚ । \newline
65. प्र॒जाप॑तिगृहीत॒येति॑ प्र॒जाप॑ति - गृ॒ही॒त॒या॒ । \newline
66. त्वया॒ मनो॒ मन॒ स्त्वया॒ त्वया॒ मनः॑ । \newline
67. मनो॑ गृह्णामि गृह्णामि॒ मनो॒ मनो॑ गृह्णामि । \newline

\textbf{Ghana Paata } \newline

1. अ॒यम् पु॒रः पु॒रो॑ ऽय म॒यम् पु॒रो भुवो॒ भुवः॑ पु॒रो॑ ऽय म॒यम् पु॒रो भुवः॑ । \newline
2. पु॒रो भुवो॒ भुवः॑ पु॒रः पु॒रो भुव॒ स्तस्य॒ तस्य॒ भुवः॑ पु॒रः पु॒रो भुव॒ स्तस्य॑ । \newline
3. भुव॒ स्तस्य॒ तस्य॒ भुवो॒ भुव॒ स्तस्य॑ प्रा॒णः प्रा॒ण स्तस्य॒ भुवो॒ भुव॒ स्तस्य॑ प्रा॒णः । \newline
4. तस्य॑ प्रा॒णः प्रा॒ण स्तस्य॒ तस्य॑ प्रा॒णो भौ॑वाय॒नो भौ॑वाय॒नः प्रा॒ण स्तस्य॒ तस्य॑ प्रा॒णो भौ॑वाय॒नः । \newline
5. प्रा॒णो भौ॑वाय॒नो भौ॑वाय॒नः प्रा॒णः प्रा॒णो भौ॑वाय॒नो व॑स॒न्तो व॑स॒न्तो भौ॑वाय॒नः प्रा॒णः प्रा॒णो भौ॑वाय॒नो व॑स॒न्तः । \newline
6. प्रा॒ण इति॑ प्र - अ॒नः । \newline
7. भौ॒वा॒य॒नो व॑स॒न्तो व॑स॒न्तो भौ॑वाय॒नो भौ॑वाय॒नो व॑स॒न्तः प्रा॑णाय॒नः प्रा॑णाय॒नो व॑स॒न्तो भौ॑वाय॒नो भौ॑वाय॒नो व॑स॒न्तः प्रा॑णाय॒नः । \newline
8. व॒स॒न्तः प्रा॑णाय॒नः प्रा॑णाय॒नो व॑स॒न्तो व॑स॒न्तः प्रा॑णाय॒नो गा॑य॒त्री गा॑य॒त्री प्रा॑णाय॒नो व॑स॒न्तो व॑स॒न्तः प्रा॑णाय॒नो गा॑य॒त्री । \newline
9. प्रा॒णा॒य॒नो गा॑य॒त्री गा॑य॒त्री प्रा॑णाय॒नः प्रा॑णाय॒नो गा॑य॒त्री वा॑स॒न्ती वा॑स॒न्ती गा॑य॒त्री प्रा॑णाय॒नः प्रा॑णाय॒नो गा॑य॒त्री वा॑स॒न्ती । \newline
10. गा॒य॒त्री वा॑स॒न्ती वा॑स॒न्ती गा॑य॒त्री गा॑य॒त्री वा॑स॒न्ती गा॑यत्रि॒यै गा॑यत्रि॒यै वा॑स॒न्ती गा॑य॒त्री गा॑य॒त्री वा॑स॒न्ती गा॑यत्रि॒यै । \newline
11. वा॒स॒न्ती गा॑यत्रि॒यै गा॑यत्रि॒यै वा॑स॒न्ती वा॑स॒न्ती गा॑यत्रि॒यै गा॑य॒त्रम् गा॑य॒त्रम् गा॑यत्रि॒यै वा॑स॒न्ती वा॑स॒न्ती गा॑यत्रि॒यै गा॑य॒त्रम् । \newline
12. गा॒य॒त्रि॒यै गा॑य॒त्रम् गा॑य॒त्रम् गा॑यत्रि॒यै गा॑यत्रि॒यै गा॑य॒त्रम् गा॑य॒त्राद् गा॑य॒त्राद् गा॑य॒त्रम् गा॑यत्रि॒यै गा॑यत्रि॒यै गा॑य॒त्रम् गा॑य॒त्रात् । \newline
13. गा॒य॒त्रम् गा॑य॒त्राद् गा॑य॒त्राद् गा॑य॒त्रम् गा॑य॒त्रम् गा॑य॒त्रा दु॑पाꣳ॒॒शु रु॑पाꣳ॒॒शुर् गा॑य॒त्राद् गा॑य॒त्रम् गा॑य॒त्रम् गा॑य॒त्रा दु॑पाꣳ॒॒शुः । \newline
14. गा॒य॒त्रा दु॑पाꣳ॒॒शु रु॑पाꣳ॒॒शुर् गा॑य॒त्राद् गा॑य॒त्रा दु॑पाꣳ॒॒शु रु॑पाꣳ॒॒शो रु॑पाꣳ॒॒शो रु॑पाꣳ॒॒शुर् गा॑य॒त्राद् गा॑य॒त्रा दु॑पाꣳ॒॒शु रु॑पाꣳ॒॒शोः । \newline
15. उ॒पाꣳ॒॒शु रु॑पाꣳ॒॒शो रु॑पाꣳ॒॒शो रु॑पाꣳ॒॒शु रु॑पाꣳ॒॒शु रु॑पाꣳ॒॒शो स्त्रि॒वृत् त्रि॒वृ 
दु॑पाꣳ॒॒शो रु॑पाꣳ॒॒शु रु॑पाꣳ॒॒शु रु॑पाꣳ॒॒शो स्त्रि॒वृत् । \newline
16. उ॒पाꣳ॒॒शुरित्यु॑प - अꣳ॒॒शुः । \newline
17. उ॒पाꣳ॒॒शो स्त्रि॒वृत् त्रि॒वृदु॑पाꣳ॒॒शो रु॑पाꣳ॒॒शो स्त्रि॒वृत् त्रि॒वृत॑ स्त्रि॒वृत॑ स्त्रि॒वृ दु॑पाꣳ॒॒शो रु॑पाꣳ॒॒शो स्त्रि॒वृत् त्रि॒वृतः॑ । \newline
18. उ॒पाꣳ॒॒शोरित्यु॑प - अꣳ॒॒शोः । \newline
19. त्रि॒वृत् त्रि॒वृत॑ स्त्रि॒वृत॑ स्त्रि॒वृत् त्रि॒वृत् त्रि॒वृतो॑ रथन्त॒रꣳ र॑थन्त॒रम् त्रि॒वृत॑ स्त्रि॒वृत् त्रि॒वृत् त्रि॒वृतो॑ रथन्त॒रम् । \newline
20. त्रि॒वृदिति॑ त्रि - वृत् । \newline
21. त्रि॒वृतो॑ रथन्त॒रꣳ र॑थन्त॒रम् त्रि॒वृत॑ स्त्रि॒वृतो॑ रथन्त॒रꣳ र॑थन्त॒राद् र॑थन्त॒राद् र॑थन्त॒रम् त्रि॒वृत॑ स्त्रि॒वृतो॑ रथन्त॒रꣳ र॑थन्त॒रात् । \newline
22. त्रि॒वृत॒ इति॑ त्रि - वृतः॑ । \newline
23. र॒थ॒न्त॒रꣳ र॑थन्त॒राद् र॑थन्त॒राद् र॑थन्त॒रꣳ र॑थन्त॒रꣳ र॑थन्त॒राद् वसि॑ष्ठो॒ वसि॑ष्ठो रथन्त॒राद् र॑थन्त॒रꣳ र॑थन्त॒रꣳ र॑थन्त॒राद् वसि॑ष्ठः । \newline
24. र॒थ॒न्त॒रमिति॑ रथं - त॒रम् । \newline
25. र॒थ॒न्त॒राद् वसि॑ष्ठो॒ वसि॑ष्ठो रथन्त॒राद् र॑थन्त॒राद् वसि॑ष्ठ॒ ऋषि॒र्॒. ऋषि॒र् वसि॑ष्ठो रथन्त॒राद् र॑थन्त॒राद् वसि॑ष्ठ॒ ऋषिः॑ । \newline
26. र॒थ॒न्त॒रादिति॑ रथं - त॒रात् । \newline
27. वसि॑ष्ठ॒ ऋषि॒र्॒. ऋषि॒र् वसि॑ष्ठो॒ वसि॑ष्ठ॒ ऋषिः॑ प्र॒जाप॑तिगृहीतया प्र॒जाप॑तिगृहीत॒य र्.षि॒र् वसि॑ष्ठो॒ वसि॑ष्ठ॒ ऋषिः॑ प्र॒जाप॑तिगृहीतया । \newline
28. ऋषिः॑ प्र॒जाप॑तिगृहीतया प्र॒जाप॑तिगृहीत॒य र्.षि॒र्॒. ऋषिः॑ प्र॒जाप॑तिगृहीतया॒ त्वया॒ त्वया᳚ प्र॒जाप॑तिगृहीत॒य र्.षि॒र्॒. ऋषिः॑ प्र॒जाप॑तिगृहीतया॒ त्वया᳚ । \newline
29. प्र॒जाप॑तिगृहीतया॒ त्वया॒ त्वया᳚ प्र॒जाप॑तिगृहीतया प्र॒जाप॑तिगृहीतया॒ त्वया᳚ प्रा॒णम् प्रा॒णम् त्वया᳚ प्र॒जाप॑तिगृहीतया प्र॒जाप॑तिगृहीतया॒ त्वया᳚ प्रा॒णम् । \newline
30. प्र॒जाप॑तिगृहीत॒येति॑ प्र॒जाप॑ति - गृ॒ही॒त॒या॒ । \newline
31. त्वया᳚ प्रा॒णम् प्रा॒णम् त्वया॒ त्वया᳚ प्रा॒णम् गृ॑ह्णामि गृह्णामि प्रा॒णम् त्वया॒ त्वया᳚ प्रा॒णम् गृ॑ह्णामि । \newline
32. प्रा॒णम् गृ॑ह्णामि गृह्णामि प्रा॒णम् प्रा॒णम् गृ॑ह्णामि प्र॒जाभ्यः॑ प्र॒जाभ्यो॑ गृह्णामि प्रा॒णम् प्रा॒णम् गृ॑ह्णामि प्र॒जाभ्यः॑ । \newline
33. प्रा॒णमिति॑ प्र - अ॒नम् । \newline
34. गृ॒ह्णा॒मि॒ प्र॒जाभ्यः॑ प्र॒जाभ्यो॑ गृह्णामि गृह्णामि प्र॒जाभ्यो॒ ऽय म॒यम् प्र॒जाभ्यो॑ गृह्णामि गृह्णामि प्र॒जाभ्यो॒ ऽयम् । \newline
35. प्र॒जाभ्यो॒ ऽय म॒यम् प्र॒जाभ्यः॑ प्र॒जाभ्यो॒ ऽयम् द॑क्षि॒णा द॑क्षि॒णा ऽयम् प्र॒जाभ्यः॑ प्र॒जाभ्यो॒ ऽयम् द॑क्षि॒णा । \newline
36. प्र॒जाभ्य॒ इति॑ प्र - जाभ्यः॑ । \newline
37. अ॒यम् द॑क्षि॒णा द॑क्षि॒णा ऽय म॒यम् द॑क्षि॒णा वि॒श्वक॑र्मा वि॒श्वक॑र्मा दक्षि॒णा ऽय म॒यम् द॑क्षि॒णा वि॒श्वक॑र्मा । \newline
38. द॒क्षि॒णा वि॒श्वक॑र्मा वि॒श्वक॑र्मा दक्षि॒णा द॑क्षि॒णा वि॒श्वक॑र्मा॒ तस्य॒ तस्य॑ वि॒श्वक॑र्मा दक्षि॒णा द॑क्षि॒णा वि॒श्वक॑र्मा॒ तस्य॑ । \newline
39. वि॒श्वक॑र्मा॒ तस्य॒ तस्य॑ वि॒श्वक॑र्मा वि॒श्वक॑र्मा॒ तस्य॒ मनो॒ मन॒ स्तस्य॑ वि॒श्वक॑र्मा वि॒श्वक॑र्मा॒ तस्य॒ मनः॑ । \newline
40. वि॒श्वक॒र्मेति॑ वि॒श्व - क॒र्मा॒ । \newline
41. तस्य॒ मनो॒ मन॒ स्तस्य॒ तस्य॒ मनो॑ वैश्वकर्म॒णं ॅवै᳚श्वकर्म॒णम् मन॒ स्तस्य॒ तस्य॒ मनो॑ वैश्वकर्म॒णम् । \newline
42. मनो॑ वैश्वकर्म॒णं ॅवै᳚श्वकर्म॒णम् मनो॒ मनो॑ वैश्वकर्म॒णम् ग्री॒ष्मो ग्री॒ष्मो वै᳚श्वकर्म॒णम् मनो॒ मनो॑ वैश्वकर्म॒णम् ग्री॒ष्मः । \newline
43. वै॒श्व॒क॒र्म॒णम् ग्री॒ष्मो ग्री॒ष्मो वै᳚श्वकर्म॒णं ॅवै᳚श्वकर्म॒णम् ग्री॒ष्मो मा॑न॒सो मा॑न॒सो ग्री॒ष्मो वै᳚श्वकर्म॒णं ॅवै᳚श्वकर्म॒णम् ग्री॒ष्मो मा॑न॒सः । \newline
44. वै॒श्व॒क॒र्म॒णमिति॑ वैश्व - क॒र्म॒णम् । \newline
45. ग्री॒ष्मो मा॑न॒सो मा॑न॒सो ग्री॒ष्मो ग्री॒ष्मो मा॑न॒स स्त्रि॒ष्टुक् त्रि॒ष्टुङ् मा॑न॒सो ग्री॒ष्मो ग्री॒ष्मो मा॑न॒स स्त्रि॒ष्टुक् । \newline
46. मा॒न॒स स्त्रि॒ष्टुक् त्रि॒ष्टुङ् मा॑न॒सो मा॑न॒स स्त्रि॒ष्टुग् ग्रै॒ष्मी ग्रै॒ष्मी त्रि॒ष्टुङ् मा॑न॒सो मा॑न॒स स्त्रि॒ष्टुग् ग्रै॒ष्मी । \newline
47. त्रि॒ष्टुग् ग्रै॒ष्मी ग्रै॒ष्मी त्रि॒ष्टुक् त्रि॒ष्टुग् ग्रै॒ष्मी त्रि॒ष्टुभ॑ स्त्रि॒ष्टुभो᳚ ग्रै॒ष्मी त्रि॒ष्टुक् त्रि॒ष्टुग् ग्रै॒ष्मी त्रि॒ष्टुभः॑ । \newline
48. ग्रै॒ष्मी त्रि॒ष्टुभ॑ स्त्रि॒ष्टुभो᳚ ग्रै॒ष्मी ग्रै॒ष्मी त्रि॒ष्टुभ॑ ऐ॒ड मै॒डम् त्रि॒ष्टुभो᳚ ग्रै॒ष्मी ग्रै॒ष्मी त्रि॒ष्टुभ॑ ऐ॒डम् । \newline
49. त्रि॒ष्टुभ॑ ऐ॒ड मै॒डम् त्रि॒ष्टुभ॑ स्त्रि॒ष्टुभ॑ ऐ॒ड मै॒ड दै॒डा दै॒डम् त्रि॒ष्टुभ॑ स्त्रि॒ष्टुभ॑ ऐ॒ड मै॒डात् । \newline
50. ऐ॒ड मै॒डा दै॒डा दै॒ड मै॒ड मै॒डा द॑न्तर्या॒मो᳚ ऽन्तर्या॒म ऐ॒डा दै॒ड मै॒ड मै॒डा द॑न्तर्या॒मः । \newline
51. ऐ॒डा द॑न्तर्या॒मो᳚ ऽन्तर्या॒म ऐ॒डा दै॒डा द॑न्तर्या॒मो᳚ ऽन्तर्या॒मा द॑न्तर्या॒मा द॑न्तर्या॒म ऐ॒डा दै॒डा द॑न्तर्या॒मो᳚ ऽन्तर्या॒मात् । \newline
52. अ॒न्त॒र्या॒मो᳚ ऽन्तर्या॒मा द॑न्तर्या॒मा द॑न्तर्या॒मो᳚ ऽन्तर्या॒मो᳚ ऽन्तर्या॒मात् प॑ञ्चद॒शः प॑ञ्चद॒शो᳚ ऽन्तर्या॒मा द॑न्तर्या॒मो᳚ ऽन्तर्या॒मो᳚ ऽन्तर्या॒मात् प॑ञ्चद॒शः । \newline
53. अ॒न्त॒र्या॒म इत्य॑न्तः - या॒मः । \newline
54. अ॒न्त॒र्या॒मात् प॑ञ्चद॒शः प॑ञ्चद॒शो᳚ ऽन्तर्या॒मा द॑न्तर्या॒मात् प॑ञ्चद॒शः प॑ञ्चद॒शात् प॑ञ्चद॒शात् प॑ञ्चद॒शो᳚ ऽन्तर्या॒मा द॑न्तर्या॒मात् प॑ञ्चद॒शः प॑ञ्चद॒शात् । \newline
55. अ॒न्त॒र्या॒मादित्य॑न्तः - या॒मात् । \newline
56. प॒ञ्च॒द॒शः प॑ञ्चद॒शात् प॑ञ्चद॒शात् प॑ञ्चद॒शः प॑ञ्चद॒शः प॑ञ्चद॒शाद् बृ॒हद् बृ॒हत् प॑ञ्चद॒शात् प॑ञ्चद॒शः प॑ञ्चद॒शः प॑ञ्चद॒शाद् बृ॒हत् । \newline
57. प॒ञ्च॒द॒श इति॑ पञ्च - द॒शः । \newline
58. प॒ञ्च॒द॒शाद् बृ॒हद् बृ॒हत् प॑ञ्चद॒शात् प॑ञ्चद॒शाद् बृ॒हद् बृ॑ह॒तो बृ॑ह॒तो बृ॒हत् प॑ञ्चद॒शात् प॑ञ्चद॒शाद् बृ॒हद् बृ॑ह॒तः । \newline
59. प॒ञ्च॒द॒शादिति॑ पञ्च - द॒शात् । \newline
60. बृ॒हद् बृ॑ह॒तो बृ॑ह॒तो बृ॒हद् बृ॒हद् बृ॑ह॒तो भ॒रद्वा॑जो भ॒रद्वा॑जो बृह॒तो बृ॒हद् बृ॒हद् बृ॑ह॒तो भ॒रद्वा॑जः । \newline
61. बृ॒ह॒तो भ॒रद्वा॑जो भ॒रद्वा॑जो बृह॒तो बृ॑ह॒तो भ॒रद्वा॑ज॒ ऋषि॒र्॒. ऋषि॑र् भ॒रद्वा॑जो बृह॒तो बृ॑ह॒तो भ॒रद्वा॑ज॒ ऋषिः॑ । \newline
62. भ॒रद्वा॑ज॒ ऋषि॒र्॒. ऋषि॑र् भ॒रद्वा॑जो भ॒रद्वा॑ज॒ ऋषिः॑ प्र॒जाप॑तिगृहीतया प्र॒जाप॑तिगृहीत॒य र्.षि॑र् भ॒रद्वा॑जो भ॒रद्वा॑ज॒ ऋषिः॑ प्र॒जाप॑तिगृहीतया । \newline
63. ऋषिः॑ प्र॒जाप॑तिगृहीतया प्र॒जाप॑तिगृहीत॒य र्.षि॒र्॒. ऋषिः॑ प्र॒जाप॑तिगृहीतया॒ त्वया॒ त्वया᳚ प्र॒जाप॑तिगृहीत॒य र्.षि॒र्॒. ऋषिः॑ प्र॒जाप॑तिगृहीतया॒ त्वया᳚ । \newline
64. प्र॒जाप॑तिगृहीतया॒ त्वया॒ त्वया᳚ प्र॒जाप॑तिगृहीतया प्र॒जाप॑तिगृहीतया॒ त्वया॒ मनो॒ मन॒ स्त्वया᳚ प्र॒जाप॑तिगृहीतया प्र॒जाप॑तिगृहीतया॒ त्वया॒ मनः॑ । \newline
65. प्र॒जाप॑तिगृहीत॒येति॑ प्र॒जाप॑ति - गृ॒ही॒त॒या॒ । \newline
66. त्वया॒ मनो॒ मन॒ स्त्वया॒ त्वया॒ मनो॑ गृह्णामि गृह्णामि॒ मन॒ स्त्वया॒ त्वया॒ मनो॑ गृह्णामि । \newline
67. मनो॑ गृह्णामि गृह्णामि॒ मनो॒ मनो॑ गृह्णामि प्र॒जाभ्यः॑ प्र॒जाभ्यो॑ गृह्णामि॒ मनो॒ मनो॑ गृह्णामि प्र॒जाभ्यः॑ । \newline
\pagebreak
\markright{ TS 4.3.2.2  \hfill https://www.vedavms.in \hfill}

\section{ TS 4.3.2.2 }

\textbf{TS 4.3.2.2 } \newline
\textbf{Samhita Paata} \newline

गृह्णामि प्र॒जाभ्यो॒ऽयं प॒श्चाद्-वि॒श्वव्य॑चा॒स्तस्य॒ चक्षु॑र्वैश्वव्यच॒सं ॅव॒र्॒.षाणि॑ चाक्षु॒षाणि॒ जग॑ती वा॒र्॒.षी जग॑त्या॒ ऋक्ष॑म॒मृक्ष॑माच्छु॒क्रः शु॒क्राथ् स॑प्तद॒शः स॑प्तद॒शाद्-वै॑रू॒पं ॅवै॑रू॒पाद्-वि॒श्वामि॑त्र॒ ऋषिः॑ प्र॒जाप॑ति गृहीतया॒ त्वया॒ चक्षु॑र्गृह्णामि प्र॒जाभ्य॑ इ॒दमु॑त्त॒राथ् सुव॒स्तस्य॒ श्रोत्रꣳ॑ सौ॒वꣳ श॒रच्छ्रौ॒त्र्य॑नु॒ष्टुप्-छा॑र॒द्य॑नु॒ष्टुभः॑ स्वा॒रꣳ स्वा॒रान्म॒न्थी म॒न्थिन॑ एकविꣳ॒॒श ए॑कविꣳ॒॒शाद् वै॑रा॒जं ॅवै॑रा॒जाज्ज॒मद॑ग्नि॒र्॒. ऋषिः॑ प्र॒जाप॑ति गृहीतया॒ - [  ] \newline

\textbf{Pada Paata} \newline

गृ॒ह्णा॒मि॒ । प्र॒जाभ्य॒ इति॑ प्र - जाभ्यः॑ । अ॒यम् । प॒श्चात् । वि॒श्वव्य॑चा॒ इति॑ वि॒श्व - व्य॒चाः॒ । तस्य॑ । चक्षुः॑ । वै॒श्व॒व्य॒च॒समिति॑ वैश्व - व्य॒च॒सम् । व॒र्॒.षाणि॑ । चा॒क्षु॒षाणि॑ । जग॑ती । वा॒र्॒.षी । जग॑त्याः । ऋक्ष॑मम् । ऋक्ष॑मात् । शु॒क्रः । शु॒क्रात् । स॒प्त॒द॒श इति॑ सप्त - द॒शः । स॒प्त॒द॒शादिति॑ सप्त-द॒शात् । वै॒रू॒पम् । वै॒रू॒पात् । वि॒श्वामि॑त्र॒ इति॑ वि॒श्व - मि॒त्रः॒ । ऋषिः॑ । प्र॒जाप॑तिगृहीत॒येति॑ प्र॒जाप॑ति - गृ॒ही॒त॒या॒ । त्वया᳚ । चक्षुः॑ । गृ॒ह्णा॒मि॒ । प्र॒जाभ्य॒ इति॑ प्र - जाभ्यः॑ । इ॒दम् । उ॒त्त॒रादित्यु॑त् - त॒रात् । सुवः॑ । तस्य॑ । श्रोत्र᳚म् । सौ॒वम् । श॒रत् । श्रौ॒त्री । अ॒नु॒ष्टुबित्य॑नु - स्तुप् । शा॒र॒दी । अ॒नु॒ष्टुभ॒ इत्य॑नु - स्तुभः॑ । स्वा॒रम् । स्वा॒रात् । म॒न्थी । म॒न्थिनः॑ । ए॒क॒विꣳ॒॒श इत्ये॑क - विꣳ॒॒शः । ए॒क॒विꣳ॒॒शादित्ये॑क - विꣳ॒॒शात् । वै॒रा॒जम् । वै॒रा॒जात् । ज॒मद॑ग्निः । ऋषिः॑ । प्र॒जाप॑तिगृहीत॒येति॑ प्र॒जाप॑ति - गृ॒ही॒त॒या॒ ।  \newline


\textbf{Krama Paata} \newline

गृ॒ह्णा॒मि॒ प्र॒जाभ्यः॑ । प्र॒जाभ्यो॒ऽयम् । प्र॒जाभ्य॒ इति॑ प्र - जाभ्यः॑ । अ॒यम् प॒श्चात् । प॒श्चाद् वि॒श्वव्य॑चाः । वि॒श्वव्य॑चा॒स्तस्य॑ । वि॒श्वव्य॑चा॒ इति॑ वि॒श्व - व्य॒चाः॒ । तस्य॒ चक्षुः॑ । चक्षु॑र् वैश्वव्यच॒सम् । वै॒श्व॒व्य॒च॒सम् ॅव॒र्॒.षाणि॑ । वै॒श्व॒व्य॒च॒समिति॑ वैश्व - व्य॒च॒सम् । व॒र्.॒षाणि॑ चाक्षु॒षाणि॑ । चा॒क्षु॒षाणि॒ जग॑ती । जग॑ती वा॒र्.॒षी । वा॒र्.॒षी जग॑त्याः । जग॑त्या॒ ऋक्ष॑मम् । ऋक्ष॑म॒मृक्ष॑मात् । ऋक्ष॑माच्छु॒क्रः । शु॒क्रः शु॒क्रात् । शु॒क्राथ् स॑प्तद॒शः । स॒प्त॒द॒शः स॑प्तद॒शात् । स॒प्त॒द॒श इति॑ सप्त - द॒शः । स॒प्त॒द॒शाद् वै॑रू॒पम् । स॒प्त॒द॒शादिति॑ सप्त - द॒शात् । वै॒रू॒पं ॅवै॑रू॒पात् । वै॒रू॒पाद् वि॒श्वामि॑त्रः । वि॒श्वामि॑त्र॒ ऋषिः॑ । वि॒श्वामि॑त्र॒ इति॑ वि॒श्व - मि॒त्रः॒ । ऋषिः॑ प्र॒जाप॑तिगृहीतया । प्र॒जाप॑तिगृहीतया॒ त्वया᳚ । प्र॒जाप॑तिगृहीत॒येति॑ प्र॒जाप॑ति - गृ॒ही॒त॒या॒ । त्वया॒ चक्षुः॑ । चक्षु॑र् गृह्णामि । गृ॒ह्णा॒मि॒ प्र॒जाभ्यः॑ । प्र॒जाभ्य॑ इ॒दम् । प्र॒जाभ्य॒ इति॑ प्र - जाभ्यः॑ । इ॒दमु॑त्त॒रात् । उ॒त्त॒राथ् सुवः॑ । उ॒त्त॒रादित्यु॑त् - त॒रात् । सुव॒स्तस्य॑ । तस्य॒ श्रोत्र᳚म् । श्रोत्रꣳ॑ सौ॒वम् । सौ॒वꣳ श॒रत् । श॒रच्छ्रौ॒त्री । श्रौ॒त्र्य॑नु॒ष्टुप् । अ॒नु॒ष्टुप्छा॑र॒दी । अ॒नु॒ष्टुबित्य॑नु - स्तुप् । शा॒र॒द्य॑नु॒ष्टुभः॑ । अ॒नु॒ष्टुभः॑ स्वा॒रम् । अ॒नु॒ष्टुभ॒ इत्य॑नु - स्तुभः॑ । स्वा॒रꣳ स्वा॒रात् । स्वा॒रान् म॒न्थी । म॒न्थी म॒न्थिनः॑ । म॒न्थिन॑ एकविꣳ॒॒शः । ए॒क॒विꣳ॒॒श ए॑क॒विꣳ॒॒शात् । ए॒क॒विꣳ॒॒श इत्ये॑क - विꣳ॒॒शः । ए॒क॒विꣳ॒॒शाद् वै॑रा॒जम् । ए॒क॒विꣳ॒॒शादित्ये॑क - विꣳ॒॒शात् । वै॒रा॒जं ॅवै॑रा॒जात् । वै॒रा॒जाज् ज॒मद॑ग्निः । ज॒मद॑ग्नि॒र्. ऋषिः॑ । ऋषिः॑ प्र॒जाप॑तिगृहीतया ( ) । प्र॒जाप॑तिगृहीतया॒ त्वया᳚ । प्र॒जाप॑तिगृहीत॒येति॑ प्र॒जाप॑ति - गृ॒ही॒त॒या॒ \newline

\textbf{Jatai Paata} \newline

1. गृ॒ह्णा॒मि॒ प्र॒जाभ्यः॑ प्र॒जाभ्यो॑ गृह्णामि गृह्णामि प्र॒जाभ्यः॑ । \newline
2. प्र॒जाभ्यो॒ ऽय म॒यम् प्र॒जाभ्यः॑ प्र॒जाभ्यो॒ ऽयम् । \newline
3. प्र॒जाभ्य॒ इति॑ प्र - जाभ्यः॑ । \newline
4. अ॒यम् प॒श्चात् प॒श्चा द॒य म॒यम् प॒श्चात् । \newline
5. प॒श्चाद् वि॒श्वव्य॑चा वि॒श्वव्य॑चाः प॒श्चात् प॒श्चाद् वि॒श्वव्य॑चाः । \newline
6. वि॒श्वव्य॑चा॒ स्तस्य॒ तस्य॑ वि॒श्वव्य॑चा वि॒श्वव्य॑चा॒ स्तस्य॑ । \newline
7. वि॒श्वव्य॑चा॒ इति॑ वि॒श्व - व्य॒चाः॒ । \newline
8. तस्य॒ चक्षु॒ श्चक्षु॒ स्तस्य॒ तस्य॒ चक्षुः॑ । \newline
9. चक्षु॑र् वैश्वव्यच॒सं ॅवै᳚श्वव्यच॒सम् चक्षु॒ श्चक्षु॑र् वैश्वव्यच॒सम् । \newline
10. वै॒श्व॒व्य॒च॒सं ॅव॒र्॒.षाणि॑ व॒र्॒.षाणि॑ वैश्वव्यच॒सं ॅवै᳚श्वव्यच॒सं ॅव॒र्॒.षाणि॑ । \newline
11. वै॒श्व॒व्य॒च॒समिति॑ वैश्व - व्य॒च॒सम् । \newline
12. व॒र्॒.षाणि॑ चाक्षु॒षाणि॑ चाक्षु॒षाणि॑ व॒र्॒.षाणि॑ व॒र्॒.षाणि॑ चाक्षु॒षाणि॑ । \newline
13. चा॒क्षु॒षाणि॒ जग॑ती॒ जग॑ती चाक्षु॒षाणि॑ चाक्षु॒षाणि॒ जग॑ती । \newline
14. जग॑ती वा॒र्॒.षी वा॒र्॒.षी जग॑ती॒ जग॑ती वा॒र्॒.षी । \newline
15. वा॒र्॒.षी जग॑त्या॒ जग॑त्या वा॒र्॒.षी वा॒र्॒.षी जग॑त्याः । \newline
16. जग॑त्या॒ ऋक्ष॑म॒ मृक्ष॑म॒म् जग॑त्या॒ जग॑त्या॒ ऋक्ष॑मम् । \newline
17. ऋक्ष॑म॒ मृक्ष॑मा॒ दृक्ष॑मा॒ दृक्ष॑म॒ मृक्ष॑म॒ मृक्ष॑मात् । \newline
18. ऋक्ष॑मा च्छु॒क्रः शु॒क्र ऋक्ष॑मा॒ दृक्ष॑मा च्छु॒क्रः । \newline
19. शु॒क्रः शु॒क्रा च्छु॒क्रा च्छु॒क्रः शु॒क्रः शु॒क्रात् । \newline
20. शु॒क्राथ् स॑प्तद॒शः स॑प्तद॒शः शु॒क्रा च्छु॒क्राथ् स॑प्तद॒शः । \newline
21. स॒प्त॒द॒शः स॑प्तद॒शाथ् स॑प्तद॒शाथ् स॑प्तद॒शः स॑प्तद॒शः स॑प्तद॒शात् । \newline
22. स॒प्त॒द॒श इति॑ सप्त - द॒शः । \newline
23. स॒प्त॒द॒शाद् वै॑रू॒पं ॅवै॑रू॒पꣳ स॑प्तद॒शाथ् स॑प्तद॒शाद् वै॑रू॒पम् । \newline
24. स॒प्त॒द॒शादिति॑ सप्त - द॒शात् । \newline
25. वै॒रू॒पं ॅवै॑रू॒पाद् वै॑रू॒पाद् वै॑रू॒पं ॅवै॑रू॒पं ॅवै॑रू॒पात् । \newline
26. वै॒रू॒पाद् वि॒श्वामि॑त्रो वि॒श्वामि॑त्रो वैरू॒पाद् वै॑रू॒पाद् वि॒श्वामि॑त्रः । \newline
27. वि॒श्वामि॑त्र॒ ऋषि॒र्॒. ऋषि॑र् वि॒श्वामि॑त्रो वि॒श्वामि॑त्र॒ ऋषिः॑ । \newline
28. वि॒श्वामि॑त्र॒ इति॑ वि॒श्व - मि॒त्रः॒ । \newline
29. ऋषिः॑ प्र॒जाप॑तिगृहीतया प्र॒जाप॑तिगृहीत॒ यर्.षि॒र्॒.ऋषिः॑ प्र॒जाप॑तिगृहीतया । \newline
30. प्र॒जाप॑तिगृहीतया॒ त्वया॒ त्वया᳚ प्र॒जाप॑तिगृहीतया प्र॒जाप॑तिगृहीतया॒ त्वया᳚ । \newline
31. प्र॒जाप॑तिगृहीत॒येति॑ प्र॒जाप॑ति - गृ॒ही॒त॒या॒ । \newline
32. त्वया॒ चक्षु॒ श्चक्षु॒ स्त्वया॒ त्वया॒ चक्षुः॑ । \newline
33. चक्षु॑र् गृह्णामि गृह्णामि॒ चक्षु॒ श्चक्षु॑र् गृह्णामि । \newline
34. गृ॒ह्णा॒मि॒ प्र॒जाभ्यः॑ प्र॒जाभ्यो॑ गृह्णामि गृह्णामि प्र॒जाभ्यः॑ । \newline
35. प्र॒जाभ्य॑ इ॒द मि॒दम् प्र॒जाभ्यः॑ प्र॒जाभ्य॑ इ॒दम् । \newline
36. प्र॒जाभ्य॒ इति॑ प्र - जाभ्यः॑ । \newline
37. इ॒द मु॑त्त॒रा दु॑त्त॒रा दि॒द मि॒द मु॑त्त॒रात् । \newline
38. उ॒त्त॒राथ् सुवः॒ सुव॑ रुत्त॒रा दु॑त्त॒राथ् सुवः॑ । \newline
39. उ॒त्त॒रादित्यु॑त् - त॒रात् । \newline
40. सुव॒ स्तस्य॒ तस्य॒ सुवः॒ सुव॒ स्तस्य॑ । \newline
41. तस्य॒ श्रोत्रꣳ॒॒ श्रोत्र॒म् तस्य॒ तस्य॒ श्रोत्र᳚म् । \newline
42. श्रोत्रꣳ॑ सौ॒वꣳ सौ॒वꣳ श्रोत्रꣳ॒॒ श्रोत्रꣳ॑ सौ॒वम् । \newline
43. सौ॒वꣳ श॒र च्छ॒रथ् सौ॒वꣳ सौ॒वꣳ श॒रत् । \newline
44. श॒रच् छ्रौ॒त्री श्रौ॒त्री श॒र च्छ॒र च्छ्रौ॒त्री । \newline
45. श्रौ॒त्र्य॑ नु॒ष्टु ब॑नु॒ष्टुप् छ्रौ॒त्री श्रौ॒त्र्य॑ नु॒ष्टुप् । \newline
46. अ॒नु॒ष्टुप् छा॑र॒दी शा॑र॒द्य॑ नु॒ष्टु ब॑नु॒ष्टुप् छा॑र॒दी । \newline
47. अ॒नु॒ष्टुबित्य॑नु - स्तुप् । \newline
48. शा॒र॒ द्य॑नु॒ष्टुभो॑ ऽनु॒ष्टुभः॑ शार॒दी शा॑र॒ द्य॑नु॒ष्टुभः॑ । \newline
49. अ॒नु॒ष्टुभः॑ स्वा॒रꣳ स्वा॒र म॑नु॒ष्टुभो॑ ऽनु॒ष्टुभः॑ स्वा॒रम् । \newline
50. अ॒नु॒ष्टुभ॒ इत्य॑नु - स्तुभः॑ । \newline
51. स्वा॒रꣳ स्वा॒राथ् स्वा॒राथ् स्वा॒रꣳ स्वा॒रꣳ स्वा॒रात् । \newline
52. स्वा॒रान् म॒न्थी म॒न्थी स्वा॒राथ् स्वा॒रान् म॒न्थी । \newline
53. म॒न्थी म॒न्थिनो॑ म॒न्थिनो॑ म॒न्थी म॒न्थी म॒न्थिनः॑ । \newline
54. म॒न्थिन॑ एकविꣳ॒॒श ए॑कविꣳ॒॒शो म॒न्थिनो॑ म॒न्थिन॑ एकविꣳ॒॒शः । \newline
55. ए॒क॒विꣳ॒॒श ए॑कविꣳ॒॒शा दे॑कविꣳ॒॒शा दे॑कविꣳ॒॒श ए॑कविꣳ॒॒श ए॑कविꣳ॒॒शात् । \newline
56. ए॒क॒विꣳ॒॒श इत्ये॑क - विꣳ॒॒शः । \newline
57. ए॒क॒विꣳ॒॒शाद् वै॑रा॒जं ॅवै॑रा॒ज मे॑कविꣳ॒॒शा दे॑कविꣳ॒॒शाद् वै॑रा॒जम् । \newline
58. ए॒क॒विꣳ॒॒शादित्ये॑क - विꣳ॒॒शात् । \newline
59. वै॒रा॒जं ॅवै॑रा॒जाद् वै॑रा॒जाद् वै॑रा॒जं ॅवै॑रा॒जं ॅवै॑रा॒जात् । \newline
60. वै॒रा॒जाज् ज॒मद॑ग्निर् ज॒मद॑ग्निर् वैरा॒जाद् वै॑रा॒जाज् ज॒मद॑ग्निः । \newline
61. ज॒मद॑ग्नि॒र्॒. ऋषि॒र्॒. ऋषि॑र् ज॒मद॑ग्निर् ज॒मद॑ग्नि॒र्॒.ऋषिः॑ । \newline
62. ऋषिः॑ प्र॒जाप॑तिगृहीतया प्र॒जाप॑तिगृहीत॒यर्.षि॒र्॒.ऋषिः॑ प्र॒जाप॑तिगृहीतया । \newline
63. प्र॒जाप॑तिगृहीतया॒ त्वया॒ त्वया᳚ प्र॒जाप॑तिगृहीतया प्र॒जाप॑तिगृहीतया॒ त्वया᳚ । \newline
64. प्र॒जाप॑तिगृहीत॒येति॑ प्र॒जाप॑ति - गृ॒ही॒त॒या॒ । \newline

\textbf{Ghana Paata } \newline

1. गृ॒ह्णा॒मि॒ प्र॒जाभ्यः॑ प्र॒जाभ्यो॑ गृह्णामि गृह्णामि प्र॒जाभ्यो॒ ऽय म॒यम् प्र॒जाभ्यो॑ गृह्णामि गृह्णामि प्र॒जाभ्यो॒ ऽयम् । \newline
2. प्र॒जाभ्यो॒ ऽय म॒यम् प्र॒जाभ्यः॑ प्र॒जाभ्यो॒ ऽयम् प॒श्चात् प॒श्चा द॒यम् प्र॒जाभ्यः॑ प्र॒जाभ्यो॒ ऽयम् प॒श्चात् । \newline
3. प्र॒जाभ्य॒ इति॑ प्र - जाभ्यः॑ । \newline
4. अ॒यम् प॒श्चात् प॒श्चा द॒य म॒यम् प॒श्चाद् वि॒श्वव्य॑चा वि॒श्वव्य॑चाः प॒श्चा द॒य म॒यम् प॒श्चाद् वि॒श्वव्य॑चाः । \newline
5. प॒श्चाद् वि॒श्वव्य॑चा वि॒श्वव्य॑चाः प॒श्चात् प॒श्चाद् वि॒श्वव्य॑चा॒ स्तस्य॒ तस्य॑ वि॒श्वव्य॑चाः प॒श्चात् प॒श्चाद् वि॒श्वव्य॑चा॒ स्तस्य॑ । \newline
6. वि॒श्वव्य॑चा॒ स्तस्य॒ तस्य॑ वि॒श्वव्य॑चा वि॒श्वव्य॑चा॒ स्तस्य॒ चक्षु॒ श्चक्षु॒ स्तस्य॑ वि॒श्वव्य॑चा वि॒श्वव्य॑चा॒ स्तस्य॒ चक्षुः॑ । \newline
7. वि॒श्वव्य॑चा॒ इति॑ वि॒श्व - व्य॒चाः॒ । \newline
8. तस्य॒ चक्षु॒ श्चक्षु॒ स्तस्य॒ तस्य॒ चक्षु॑र् वैश्वव्यच॒सं ॅवै᳚श्वव्यच॒सम् चक्षु॒ स्तस्य॒ तस्य॒ चक्षु॑र् वैश्वव्यच॒सम् । \newline
9. चक्षु॑र् वैश्वव्यच॒सं ॅवै᳚श्वव्यच॒सम् चक्षु॒ श्चक्षु॑र् वैश्वव्यच॒सं ॅव॒र्॒.षाणि॑ व॒र्॒.षाणि॑ वैश्वव्यच॒सम् चक्षु॒ श्चक्षु॑र् वैश्वव्यच॒सं ॅव॒र्॒.षाणि॑ । \newline
10. वै॒श्व॒व्य॒च॒सं ॅव॒र्॒.षाणि॑ व॒र्॒.षाणि॑ वैश्वव्यच॒सं ॅवै᳚श्वव्यच॒सं ॅव॒र्॒.षाणि॑ चाक्षु॒षाणि॑ चाक्षु॒षाणि॑ व॒र्॒.षाणि॑ वैश्वव्यच॒सं ॅवै᳚श्वव्यच॒सं ॅव॒र्॒.षाणि॑ चाक्षु॒षाणि॑ । \newline
11. वै॒श्व॒व्य॒च॒समिति॑ वैश्व - व्य॒च॒सम् । \newline
12. व॒र्॒.षाणि॑ चाक्षु॒षाणि॑ चाक्षु॒षाणि॑ व॒र्॒.षाणि॑ व॒र्॒.षाणि॑ चाक्षु॒षाणि॒ जग॑ती॒ जग॑ती चाक्षु॒षाणि॑ व॒र्॒.षाणि॑ व॒र्॒.षाणि॑ चाक्षु॒षाणि॒ जग॑ती । \newline
13. चा॒क्षु॒षाणि॒ जग॑ती॒ जग॑ती चाक्षु॒षाणि॑ चाक्षु॒षाणि॒ जग॑ती वा॒र्॒.षी वा॒र्॒.षी जग॑ती चाक्षु॒षाणि॑ चाक्षु॒षाणि॒ जग॑ती वा॒र्॒.षी । \newline
14. जग॑ती वा॒र्॒.षी वा॒र्॒.षी जग॑ती॒ जग॑ती वा॒र्॒.षी जग॑त्या॒ जग॑त्या वा॒र्॒.षी जग॑ती॒ जग॑ती वा॒र्॒.षी जग॑त्याः । \newline
15. वा॒र्॒.षी जग॑त्या॒ जग॑त्या वा॒र्॒.षी वा॒र्॒.षी जग॑त्या॒ ऋक्ष॑म॒ मृक्ष॑म॒म् जग॑त्या वा॒र्॒.षी वा॒र्॒.षी जग॑त्या॒ ऋक्ष॑मम् । \newline
16. जग॑त्या॒ ऋक्ष॑म॒ मृक्ष॑म॒म् जग॑त्या॒ जग॑त्या॒ ऋक्ष॑म॒ मृक्ष॑मा॒ दृक्ष॑मा॒ दृक्ष॑म॒म् जग॑त्या॒ जग॑त्या॒ ऋक्ष॑म॒ मृक्ष॑मात् । \newline
17. ऋक्ष॑म॒ मृक्ष॑मा॒ दृक्ष॑मा॒ दृक्ष॑म॒ मृक्ष॑म॒ मृक्ष॑मा च्छु॒क्रः शु॒क्र ऋक्ष॑मा॒ दृक्ष॑म॒ मृक्ष॑म॒ मृक्ष॑मा च्छु॒क्रः । \newline
18. ऋक्ष॑मा च्छु॒क्रः शु॒क्र ऋक्ष॑मा॒ दृक्ष॑मा च्छु॒क्रः शु॒क्रा च्छु॒क्रा च्छु॒क्र ऋक्ष॑मा॒ दृक्ष॑मा च्छु॒क्रः शु॒क्रात् । \newline
19. शु॒क्रः शु॒क्रा च्छु॒क्रा च्छु॒क्रः शु॒क्रः शु॒क्राथ् स॑प्तद॒शः स॑प्तद॒शः शु॒क्रा च्छु॒क्रः शु॒क्रः शु॒क्राथ् स॑प्तद॒शः । \newline
20. शु॒क्राथ् स॑प्तद॒शः स॑प्तद॒शः शु॒क्रा च्छु॒क्राथ् स॑प्तद॒शः स॑प्तद॒शाथ् स॑प्तद॒शाथ् स॑प्तद॒शः शु॒क्रा च्छु॒क्राथ् स॑प्तद॒शः स॑प्तद॒शात् । \newline
21. स॒प्त॒द॒शः स॑प्तद॒शाथ् स॑प्तद॒शाथ् स॑प्तद॒शः स॑प्तद॒शः स॑प्तद॒शाद् वै॑रू॒पं ॅवै॑रू॒पꣳ स॑प्तद॒शाथ् स॑प्तद॒शः स॑प्तद॒शः स॑प्तद॒शाद् वै॑रू॒पम् । \newline
22. स॒प्त॒द॒श इति॑ सप्त - द॒शः । \newline
23. स॒प्त॒द॒शाद् वै॑रू॒पं ॅवै॑रू॒पꣳ स॑प्तद॒शाथ् स॑प्तद॒शाद् वै॑रू॒पं ॅवै॑रू॒पाद् वै॑रू॒पाद् वै॑रू॒पꣳ स॑प्तद॒शाथ् स॑प्तद॒शाद् वै॑रू॒पं ॅवै॑रू॒पात् । \newline
24. स॒प्त॒द॒शादिति॑ सप्त - द॒शात् । \newline
25. वै॒रू॒पं ॅवै॑रू॒पाद् वै॑रू॒पाद् वै॑रू॒पं ॅवै॑रू॒पं ॅवै॑रू॒पाद् वि॒श्वामि॑त्रो वि॒श्वामि॑त्रो वैरू॒पाद् वै॑रू॒पं ॅवै॑रू॒पं ॅवै॑रू॒पाद् वि॒श्वामि॑त्रः । \newline
26. वै॒रू॒पाद् वि॒श्वामि॑त्रो वि॒श्वामि॑त्रो वैरू॒पाद् वै॑रू॒पाद् वि॒श्वामि॑त्र॒ ऋषि॒र्॒. ऋषि॑र् वि॒श्वामि॑त्रो वैरू॒पाद् वै॑रू॒पाद् वि॒श्वामि॑त्र॒ ऋषिः॑ । \newline
27. वि॒श्वामि॑त्र॒ ऋषि॒र्॒. ऋषि॑र् वि॒श्वामि॑त्रो वि॒श्वामि॑त्र॒ ऋषिः॑ प्र॒जाप॑तिगृहीतया प्र॒जाप॑तिगृहीत॒य र्.षि॑र् वि॒श्वामि॑त्रो वि॒श्वामि॑त्र॒ ऋषिः॑ प्र॒जाप॑तिगृहीतया । \newline
28. वि॒श्वामि॑त्र॒ इति॑ वि॒श्व - मि॒त्रः॒ । \newline
29. ऋषिः॑ प्र॒जाप॑तिगृहीतया प्र॒जाप॑तिगृहीत॒य र्.षि॒र्॒. ऋषिः॑ प्र॒जाप॑तिगृहीतया॒ त्वया॒ त्वया᳚ प्र॒जाप॑तिगृहीत॒य र्.षि॒र्॒. ऋषिः॑ प्र॒जाप॑तिगृहीतया॒ त्वया᳚ । \newline
30. प्र॒जाप॑तिगृहीतया॒ त्वया॒ त्वया᳚ प्र॒जाप॑तिगृहीतया प्र॒जाप॑तिगृहीतया॒ त्वया॒ चक्षु॒ श्चक्षु॒ स्त्वया᳚ प्र॒जाप॑तिगृहीतया प्र॒जाप॑तिगृहीतया॒ त्वया॒ चक्षुः॑ । \newline
31. प्र॒जाप॑तिगृहीत॒येति॑ प्र॒जाप॑ति - गृ॒ही॒त॒या॒ । \newline
32. त्वया॒ चक्षु॒ श्चक्षु॒ स्त्वया॒ त्वया॒ चक्षु॑र् गृह्णामि गृह्णामि॒ चक्षु॒ स्त्वया॒ त्वया॒ चक्षु॑र् गृह्णामि । \newline
33. चक्षु॑र् गृह्णामि गृह्णामि॒ चक्षु॒ श्चक्षु॑र् गृह्णामि प्र॒जाभ्यः॑ प्र॒जाभ्यो॑ गृह्णामि॒ चक्षु॒ श्चक्षु॑र् गृह्णामि प्र॒जाभ्यः॑ । \newline
34. गृ॒ह्णा॒मि॒ प्र॒जाभ्यः॑ प्र॒जाभ्यो॑ गृह्णामि गृह्णामि प्र॒जाभ्य॑ इ॒द मि॒दम् प्र॒जाभ्यो॑ गृह्णामि गृह्णामि प्र॒जाभ्य॑ इ॒दम् । \newline
35. प्र॒जाभ्य॑ इ॒द मि॒दम् प्र॒जाभ्यः॑ प्र॒जाभ्य॑ इ॒द मु॑त्त॒रा दु॑त्त॒रा दि॒दम् प्र॒जाभ्यः॑ प्र॒जाभ्य॑ इ॒द मु॑त्त॒रात् । \newline
36. प्र॒जाभ्य॒ इति॑ प्र - जाभ्यः॑ । \newline
37. ई॒द मु॑त्त॒रा दु॑त्त॒रा दि॒द मि॒द मु॑त्त॒राथ् सुवः॒ सुव॑-रुत्त॒रा दि॒द मि॒द मु॑त्त॒राथ् सुवः॑ । \newline
38. उ॒त्त॒राथ् सुवः॒ सुव॑ रुत्त॒रा दु॑त्त॒राथ् सुव॒ स्तस्य॒ तस्य॒ सुव॑-रुत्त॒रा दु॑त्त॒राथ् सुव॒ स्तस्य॑ । \newline
39. उ॒त्त॒रादित्यु॑त् - त॒रात् । \newline
40. सुव॒ स्तस्य॒ तस्य॒ सुवः॒ सुव॒ स्तस्य॒ श्रोत्रꣳ॒॒ श्रोत्र॒म् तस्य॒ सुवः॒ सुव॒ स्तस्य॒ श्रोत्र᳚म् । \newline
41. तस्य॒ श्रोत्रꣳ॒॒ श्रोत्र॒म् तस्य॒ तस्य॒ श्रोत्रꣳ॑ सौ॒वꣳ सौ॒वꣳ श्रोत्र॒म् तस्य॒ तस्य॒ श्रोत्रꣳ॑ सौ॒वम् । \newline
42. श्रोत्रꣳ॑ सौ॒वꣳ सौ॒वꣳ श्रोत्रꣳ॒॒ श्रोत्रꣳ॑ सौ॒वꣳ श॒र-च्छ॒रथ् सौ॒वꣳ श्रोत्रꣳ॒॒ श्रोत्रꣳ॑ सौ॒वꣳ श॒रत् । \newline
43. सौ॒वꣳ श॒र-च्छ॒रथ् सौ॒वꣳ सौ॒वꣳ श॒र-च्छ्रौ॒त्री श्रौ॒त्री श॒रथ् सौ॒वꣳ सौ॒वꣳ श॒र-च्छ्रौ॒त्री । \newline
44. श॒र च्छ्रौ॒त्री श्रौ॒त्री श॒र-च्छ॒र-च्छ्रौ॒ त्र्य॑नु॒ष्टु ब॑नु॒ष्टुप् छ्रौ॒त्री श॒र च्छ॒र च्छ्रौ॒ त्र्य॑नु॒ष्टुप् । \newline
45. श्रौ॒ त्र्य॑नु॒ष्टु ब॑नु॒ष्टुप् छ्रौ॒त्री श्रौ॒ त्र्य॑नु॒ष्टुप् छा॑र॒दी शा॑र॒ द्य॑नु॒ष्टुप् छ्रौ॒त्री श्रौ॒ त्र्य॑नु॒ष्टुप् छा॑र॒दी । \newline
46. अ॒नु॒ष्टुप् छा॑र॒दी शा॑र॒ द्य॑नु॒ष्टु ब॑नु॒ष्टुप् छा॑र॒ द्य॑नु॒ष्टुभो॑ ऽनु॒ष्टुभः॑ शर॒ द्य॑नु॒ष्टु ब॑नु॒ष्टुप् छा॑र॒ द्य॑नु॒ष्टुभः॑ । \newline
47. अ॒नु॒ष्टुबित्य॑नु - स्तुप् । \newline
48. शा॒र॒ द्य॑नु॒ष्टुभो॑ ऽनु॒ष्टुभः॑ शार॒दी शा॑र॒ द्य॑नु॒ष्टुभः॑ स्वा॒रꣳ स्वा॒र म॑नु॒ष्टुभः॑ शार॒दी शा॑र॒ द्य॑नु॒ष्टुभः॑ स्वा॒रम् । \newline
49. अ॒नु॒ष्टुभः॑ स्वा॒रꣳ स्वा॒र म॑नु॒ष्टुभो॑ ऽनु॒ष्टुभः॑ स्वा॒रꣳ स्वा॒राथ् स्वा॒राथ् स्वा॒र म॑नु॒ष्टुभो॑ ऽनु॒ष्टुभः॑ स्वा॒रꣳ स्वा॒रात् । \newline
50. अ॒नु॒ष्टुभ॒ इत्य॑नु - स्तुभः॑ । \newline
51. स्वा॒रꣳ स्वा॒राथ् स्वा॒राथ् स्वा॒रꣳ स्वा॒रꣳ स्वा॒रान् म॒न्थी म॒न्थी स्वा॒राथ् स्वा॒रꣳ स्वा॒रꣳ स्वा॒रान् म॒न्थी । \newline
52. स्वा॒रान् म॒न्थी म॒न्थी स्वा॒राथ् स्वा॒रान् म॒न्थी म॒न्थिनो॑ म॒न्थिनो॑ म॒न्थी स्वा॒राथ् स्वा॒रान् म॒न्थी म॒न्थिनः॑ । \newline
53. म॒न्थी म॒न्थिनो॑ म॒न्थिनो॑ म॒न्थी म॒न्थी म॒न्थिन॑ एकविꣳ॒॒श ए॑कविꣳ॒॒शो म॒न्थिनो॑ म॒न्थी म॒न्थी म॒न्थिन॑ एकविꣳ॒॒शः । \newline
54. म॒न्थिन॑ एकविꣳ॒॒श ए॑कविꣳ॒॒शो म॒न्थिनो॑ म॒न्थिन॑ एकविꣳ॒॒श ए॑कविꣳ॒॒ शादे॑कविꣳ॒॒शा दे॑कविꣳ॒॒शो म॒न्थिनो॑ म॒न्थिन॑ एकविꣳ॒॒श ए॑कविꣳ॒॒शात् । \newline
55. ए॒क॒विꣳ॒॒श ए॑कविꣳ॒॒शा दे॑कविꣳ॒॒शा दे॑कविꣳ॒॒श ए॑कविꣳ॒॒श ए॑कविꣳ॒॒शाद् वै॑रा॒जं ॅवै॑रा॒ज मे॑कविꣳ॒॒शा दे॑कविꣳ॒॒श ए॑कविꣳ॒॒श ए॑कविꣳ॒॒शाद् वै॑रा॒जम् । \newline
56. ए॒क॒विꣳ॒॒श इत्ये॑क - विꣳ॒॒शः । \newline
57. ए॒क॒विꣳ॒॒शाद् वै॑रा॒जं ॅवै॑रा॒ज मे॑कविꣳ॒॒शा दे॑कविꣳ॒॒शाद् वै॑रा॒जं ॅवै॑रा॒जाद् वै॑रा॒जाद् वै॑रा॒ज मे॑कविꣳ॒॒शा दे॑कविꣳ॒॒शाद् वै॑रा॒जं ॅवै॑रा॒जात् । \newline
58. ए॒क॒विꣳ॒॒शादित्ये॑क - विꣳ॒॒शात् । \newline
59. वै॒रा॒जं ॅवै॑रा॒जाद् वै॑रा॒जाद् वै॑रा॒जं ॅवै॑रा॒जं ॅवै॑रा॒जाज् ज॒मद॑ग्निर् ज॒मद॑ग्निर् वैरा॒जाद् वै॑रा॒जं ॅवै॑रा॒जं ॅवै॑रा॒जाज् ज॒मद॑ग्निः । \newline
60. वै॒रा॒जाज् ज॒मद॑ग्निर् ज॒मद॑ग्निर् वैरा॒जाद् वै॑रा॒जाज् ज॒मद॑ग्नि॒र्॒. ऋषि॒र्॒. ऋषि॑र् ज॒मद॑ग्निर् वैरा॒जाद् वै॑रा॒जाज् ज॒मद॑ग्नि॒र्॒. ऋषिः॑ । \newline
61. ज॒मद॑ग्नि॒र्॒. ऋषि॒र्॒. ऋषि॑र् ज॒मद॑ग्निर् ज॒मद॑ग्नि॒र्॒. ऋषिः॑ प्र॒जाप॑तिगृहीतया प्र॒जाप॑तिगृहीत॒य र्.षि॑र् ज॒मद॑ग्निर् ज॒मद॑ग्नि॒र्॒. ऋषिः॑ प्र॒जाप॑तिगृहीतया । \newline
62. ऋषिः॑ प्र॒जाप॑तिगृहीतया प्र॒जाप॑तिगृहीत॒य र्.षि॒र्॒. ऋषिः॑ प्र॒जाप॑तिगृहीतया॒ त्वया॒ त्वया᳚ प्र॒जाप॑तिगृहीत॒य र्.षि॒र्॒. ऋषिः॑ प्र॒जाप॑तिगृहीतया॒ त्वया᳚ । \newline
63. प्र॒जाप॑तिगृहीतया॒ त्वया॒ त्वया᳚ प्र॒जाप॑तिगृहीतया प्र॒जाप॑तिगृहीतया॒ त्वया॒ श्रोत्रꣳ॒॒ श्रोत्र॒म् त्वया᳚ प्र॒जाप॑तिगृहीतया प्र॒जाप॑तिगृहीतया॒ त्वया॒ श्रोत्र᳚म् । \newline
64. प्र॒जाप॑तिगृहीत॒येति॑ प्र॒जाप॑ति - गृ॒ही॒त॒या॒ । \newline
\pagebreak
\markright{ TS 4.3.2.3  \hfill https://www.vedavms.in \hfill}

\section{ TS 4.3.2.3 }

\textbf{TS 4.3.2.3 } \newline
\textbf{Samhita Paata} \newline

त्वया॒ श्रोत्रं॑ गृह्णामि प्र॒जाभ्य॑ इ॒यमु॒परि॑ म॒तिस्तस्यै॒ वाङ्मा॒ती हे॑म॒न्तो वा᳚च्याय॒नः प॒ङ्क्तिर्.है॑म॒न्ती प॒क्त्यैं नि॒धन॑वन्नि॒धन॑वत आग्रय॒ण आ᳚ग्रय॒णात् त्रि॑णवत्रयस्त्रिꣳ॒॒शौ त्रि॑णवत्रयस्त्रिꣳ॒॒शाभ्याꣳ॑ शाक्वररैव॒ते शा᳚क्वररैव॒ताभ्यां᳚ ॅवि॒श्वक॒र्मर्.षिः॑ प्र॒जाप॑ति गृहीतया॒ त्वया॒ वाचं॑ गृह्णामि प्र॒जाभ्यः॑ ॥ \newline

\textbf{Pada Paata} \newline

त्वया᳚ । श्रोत्र᳚म् । गृ॒ह्णा॒मि॒ । प्र॒जाभ्य॒ इति॑ प्र-जाभ्यः॑ । इ॒यम् । उ॒परि॑ । म॒तिः । तस्यै᳚ । वाक् । मा॒ती । हे॒म॒न्तः । वा॒च्या॒य॒नः । प॒ङ्क्तिः । है॒म॒न्ती । प॒ङ्क्त्यै । नि॒धन॑व॒दिति॑ नि॒धन॑ - व॒त् । नि॒धन॑वत॒ इति॑ नि॒धन॑ - व॒तः॒ । आ॒ग्र॒य॒णः । आ॒ग्र॒य॒णात् । त्रि॒ण॒व॒त्र॒य॒स्त्रिꣳ॒॒शाविति॑ त्रिणव - त्र॒य॒स्त्रिꣳ॒॒शौ । त्रि॒ण॒व॒त्र॒य॒स्त्रिꣳ॒॒शाभ्या॒मिति॑ त्रिणव - त्र॒य॒स्त्रिꣳ॒॒शाभ्या᳚म् । शा॒क्व॒र॒रै॒व॒ते इति॑ शाक्वर - रै॒व॒ते । शा॒क्व॒र॒रै॒व॒ताभ्या॒मिति॑ शाक्वर - रै॒व॒ताभ्या᳚म् । वि॒श्वक॒र्मेति॑ वि॒श्व - क॒र्मा॒ । ऋषिः॑ । प्र॒जाप॑तिगृहीत॒येति॑ प्र॒जाप॑ति - गृ॒ही॒त॒या॒ । त्वया᳚ । वाच᳚म् । गृ॒ह्णा॒मि॒ । प्र॒जाभ्य॒ इति॑ प्र - जाभ्यः॑ ॥  \newline


\textbf{Krama Paata} \newline

त्वया॒ श्रोत्र᳚म् । श्रोत्र॑म् गृह्णामि । गृ॒ह्णा॒मि॒ प्र॒जाभ्यः॑ । प्र॒जाभ्य॑ इ॒यम् । प्र॒जाभ्य॒ इति॑ प्र - जाभ्यः॑ । इ॒यमु॒परि॑ । उ॒परि॑ म॒तिः । म॒तिस्तस्यै᳚ । तस्यै॒ वाक् । वाङ् मा॒ती । मा॒ती हे॑म॒न्तः । हे॒म॒न्तो वा᳚च्याय॒नः । वा॒च्या॒य॒नः प॒ङ्क्तिः । प॒ङ्क्तिर्. है॑म॒न्ती । है॒म॒न्ती प॒ङ्क्त्यै । प॒ङ्क्त्यै नि॒धन॑वत् । नि॒धन॑वन् नि॒धन॑वतः । नि॒धन॑व॒दिति॑ नि॒धन॑ - व॒त्॒ । नि॒धन॑वत आग्रय॒णः । नि॒धन॑वत॒ इति॑ नि॒धन॑ - व॒तः॒ । आ॒ग्र॒य॒ण आ᳚ग्रय॒णात् । आ॒ग्र॒य॒णात् त्रि॑णवत्रयस्त्रिꣳ॒॒शौ । त्रि॒ण॒व॒त्र॒य॒स्त्रिꣳ॒॒शौ त्रि॑णवत्रयस्त्रिꣳ॒॒शाभ्या᳚म् । त्रि॒ण॒व॒त्र॒य॒स्त्रिꣳ॒॒शाविति॑ त्रिणव - त्र॒य॒स्त्रिꣳ॒॒शौ । त्रि॒ण॒व॒त्र॒य॒स्त्रिꣳ॒॒शाभ्याꣳ॑ शाक्वररैव॒ते । त्रि॒ण॒व॒त्र॒य॒स्त्रिꣳ॒॒शाभ्या॒मिति॑ त्रिणव - त्र॒य॒स्त्रिꣳ॒॒शाभ्या᳚म् । शा॒क्व॒र॒रै॒व॒ते शा᳚क्वररैव॒ताभ्या᳚म् । शा॒क्व॒र॒रै॒व॒ते इति॑ शाक्वर - रै॒व॒ते । शा॒क्व॒र॒रै॒व॒ताभ्यां᳚ ॅवि॒श्वक॑र्मा । शा॒क्व॒र॒रै॒व॒ताभ्या॒मिति॑ शाक्वर - रै॒व॒ताभ्या᳚म् । वि॒श्वक॒र्मर्षिः॑ । वि॒श्वक॒र्मेति॑ वि॒श्व - क॒र्मा॒ । ऋषिः॑ प्र॒जाप॑तिगृहीतया । प्र॒जाप॑तिगृहीतया॒ त्वया᳚ । प्र॒जाप॑तिगृहीत॒येति॑ प्र॒जाप॑ति - गृ॒ही॒त॒या॒ । त्वया॒ वाच᳚म् । वाच॑म् गृह्णामि । गृ॒ह्णा॒मि॒ प्र॒जाभ्यः॑ । प्र॒जाभ्य॒ इति॑ प्र - जाभ्यः॑ । \newline

\textbf{Jatai Paata} \newline

1. त्वया॒ श्रोत्रꣳ॒॒ श्रोत्र॒म् त्वया॒ त्वया॒ श्रोत्र᳚म् । \newline
2. श्रोत्र॑म् गृह्णामि गृह्णामि॒ श्रोत्रꣳ॒॒ श्रोत्र॑म् गृह्णामि । \newline
3. गृ॒ह्णा॒मि॒ प्र॒जाभ्यः॑ प्र॒जाभ्यो॑ गृह्णामि गृह्णामि प्र॒जाभ्यः॑ । \newline
4. प्र॒जाभ्य॑ इ॒य मि॒यम् प्र॒जाभ्यः॑ प्र॒जाभ्य॑ इ॒यम् । \newline
5. प्र॒जाभ्य॒ इति॑ प्र - जाभ्यः॑ । \newline
6. इ॒य मु॒पर्यु॒प री॒य मि॒य मु॒परि॑ । \newline
7. उ॒परि॑ म॒तिर् म॒ति रु॒पर्यु॒परि॑ म॒तिः । \newline
8. म॒ति स्तस्यै॒ तस्यै॑ म॒तिर् म॒ति स्तस्यै᳚ । \newline
9. तस्यै॒ वाग् वाक् तस्यै॒ तस्यै॒ वाक् । \newline
10. वाङ् मा॒ती मा॒ती वाग् वाङ् मा॒ती । \newline
11. मा॒ती हे॑म॒न्तो हे॑म॒न्तो मा॒ती मा॒ती हे॑म॒न्तः । \newline
12. हे॒म॒न्तो वा᳚च्याय॒नो वा᳚च्याय॒नो हे॑म॒न्तो हे॑म॒न्तो वा᳚च्याय॒नः । \newline
13. वा॒च्या॒य॒नः प॒ङ्क्तिः प॒ङ्क्तिर् वा᳚च्याय॒नो वा᳚च्याय॒नः प॒ङ्क्तिः । \newline
14. प॒ङ्क्तिर्. है॑म॒न्ती है॑म॒न्ती प॒ङ्क्तिः प॒ङ्क्तिर्. है॑म॒न्ती । \newline
15. है॒म॒न्ती प॒ङ्क्त्यै प॒ङ्क्त्यै है॑म॒न्ती है॑म॒न्ती प॒ङ्क्त्यै । \newline
16. प॒ङ्क्त्यै नि॒धन॑वन् नि॒धन॑वत् प॒ङ्क्त्यै प॒ङ्क्त्यै नि॒धन॑वत् । \newline
17. नि॒धन॑वन् नि॒धन॑वतो नि॒धन॑वतो नि॒धन॑वन् नि॒धन॑वन् नि॒धन॑वतः । \newline
18. नि॒धन॑व॒दिति॑ नि॒धन॑ - व॒त् । \newline
19. नि॒धन॑वत आग्रय॒ण आ᳚ग्रय॒णो नि॒धन॑वतो नि॒धन॑वत आग्रय॒णः । \newline
20. नि॒धन॑वत॒ इति॑ नि॒धन॑ - व॒तः॒ । \newline
21. आ॒ग्र॒य॒ण आ᳚ग्रय॒णा दा᳚ग्रय॒णा दा᳚ग्रय॒ण आ᳚ग्रय॒ण आ᳚ग्रय॒णात् । \newline
22. आ॒ग्र॒य॒णात् त्रि॑णवत्रयस्त्रिꣳ॒॒शौ त्रि॑णवत्रयस्त्रिꣳ॒॒शा वा᳚ग्रय॒णा दा᳚ग्रय॒णात् त्रि॑णवत्रयस्त्रिꣳ॒॒शौ । \newline
23. त्रि॒ण॒व॒त्र॒य॒स्त्रिꣳ॒॒शौ त्रि॑णवत्रयस्त्रिꣳ॒॒शाभ्या᳚म् त्रिणवत्रयस्त्रिꣳ॒॒शाभ्या᳚म् त्रिणवत्रयस्त्रिꣳ॒॒शौ त्रि॑णवत्रयस्त्रिꣳ॒॒शौ त्रि॑णवत्रयस्त्रिꣳ॒॒शाभ्या᳚म् । \newline
24. त्रि॒ण॒व॒त्र॒य॒स्त्रिꣳ॒॒शाविति॑ त्रिणव - त्र॒य॒स्त्रिꣳ॒॒शौ । \newline
25. त्रि॒ण॒व॒त्र॒य॒स्त्रिꣳ॒॒शाभ्याꣳ॑ शाक्वररैव॒ते शा᳚क्वररैव॒ते त्रि॑णवत्रयस्त्रिꣳ॒॒शाभ्या᳚म् त्रिणवत्रयस्त्रिꣳ॒॒शाभ्याꣳ॑ शाक्वररैव॒ते । \newline
26. त्रि॒ण॒व॒त्र॒य॒स्त्रिꣳ॒॒शाभ्या॒मिति॑ त्रिणव - त्र॒य॒स्त्रिꣳ॒॒शाभ्या᳚म् । \newline
27. शा॒क्व॒र॒रै॒व॒ते शा᳚क्वररैव॒ताभ्याꣳ॑ शाक्वररैव॒ताभ्याꣳ॑ शाक्वररैव॒ते शा᳚क्वररैव॒ते शा᳚क्वररैव॒ताभ्या᳚म् । \newline
28. शा॒क्व॒र॒रै॒व॒ते इति॑ शाक्वर - रै॒व॒ते । \newline
29. शा॒क्व॒र॒रै॒व॒ताभ्यां᳚ ॅवि॒श्वक॑र्मा वि॒श्वक॑र्मा शाक्वररैव॒ताभ्याꣳ॑ शाक्वररैव॒ताभ्यां᳚ ॅवि॒श्वक॑र्मा । \newline
30. शा॒क्व॒र॒रै॒व॒ताभ्या॒मिति॑ शाक्वर - रै॒व॒ताभ्या᳚म् । \newline
31. वि॒श्वक॒र्म र्.षि॒र्॒. ऋषि॑र् वि॒श्वक॑र्मा वि॒श्वक॒र्म र्.षिः॑ । \newline
32. वि॒श्वक॒र्मेति॑ वि॒श्व - क॒र्मा॒ । \newline
33. ऋषिः॑ प्र॒जाप॑तिगृहीतया प्र॒जाप॑तिगृहीत॒य र्.षि॒र्॒. ऋषिः॑ प्र॒जाप॑तिगृहीतया । \newline
34. प्र॒जाप॑तिगृहीतया॒ त्वया॒ त्वया᳚ प्र॒जाप॑तिगृहीतया प्र॒जाप॑तिगृहीतया॒ त्वया᳚ । \newline
35. प्र॒जाप॑तिगृहीत॒येति॑ प्र॒जाप॑ति - गृ॒ही॒त॒या॒ । \newline
36. त्वया॒ वाचं॒ ॅवाच॒म् त्वया॒ त्वया॒ वाच᳚म् । \newline
37. वाच॑म् गृह्णामि गृह्णामि॒ वाचं॒ ॅवाच॑म् गृह्णामि । \newline
38. गृ॒ह्णा॒मि॒ प्र॒जाभ्यः॑ प्र॒जाभ्यो॑ गृह्णामि गृह्णामि प्र॒जाभ्यः॑ । \newline
39. प्र॒जाभ्य॒ इति॑ प्र - जाभ्यः॑ । \newline

\textbf{Ghana Paata } \newline

1. त्वया॒ श्रोत्रꣳ॒॒ श्रोत्र॒म् त्वया॒ त्वया॒ श्रोत्र॑म् गृह्णामि गृह्णामि॒ श्रोत्र॒म् त्वया॒ त्वया॒ श्रोत्र॑म् गृह्णामि । \newline
2. श्रोत्र॑म् गृह्णामि गृह्णामि॒ श्रोत्रꣳ॒॒ श्रोत्र॑म् गृह्णामि प्र॒जाभ्यः॑ प्र॒जाभ्यो॑ गृह्णामि॒ श्रोत्रꣳ॒॒ श्रोत्र॑म् गृह्णामि प्र॒जाभ्यः॑ । \newline
3. गृ॒ह्णा॒मि॒ प्र॒जाभ्यः॑ प्र॒जाभ्यो॑ गृह्णामि गृह्णामि प्र॒जाभ्य॑ इ॒य मि॒यम् प्र॒जाभ्यो॑ गृह्णामि गृह्णामि प्र॒जाभ्य॑ इ॒यम् । \newline
4. प्र॒जाभ्य॑ इ॒य मि॒यम् प्र॒जाभ्यः॑ प्र॒जाभ्य॑ इ॒य मु॒पर्यु॒परी॒यम् प्र॒जाभ्यः॑ प्र॒जाभ्य॑ इ॒य मु॒परि॑ । \newline
5. प्र॒जाभ्य॒ इति॑ प्र - जाभ्यः॑ । \newline
6. ई॒य मु॒पर्यु॒ परी॒य मि॒य मु॒परि॑ म॒तिर् म॒ति रु॒प री॒य मि॒य मु॒परि॑ म॒तिः । \newline
7. उ॒परि॑ म॒तिर् म॒ति रु॒प र्यु॒परि॑ म॒ति स्तस्यै॒ तस्यै॑ म॒ति रु॒प र्यु॒परि॑ म॒ति स्तस्यै᳚ । \newline
8. ंअ॒ति स्तस्यै॒ तस्यै॑ म॒तिर् म॒ति स्तस्यै॒ वाग् वाक् तस्यै॑ म॒तिर् म॒ति स्तस्यै॒ वाक् । \newline
9. तस्यै॒ वाग् वाक् तस्यै॒ तस्यै॒ वाङ् मा॒ती मा॒ती वाक् तस्यै॒ तस्यै॒ वाङ् मा॒ती । \newline
10. वाङ् मा॒ती मा॒ती वाग् वाङ् मा॒ती हे॑म॒न्तो हे॑म॒न्तो मा॒ती वाग् वाङ् मा॒ती हे॑म॒न्तः । \newline
11. मा॒ती हे॑म॒न्तो हे॑म॒न्तो मा॒ती मा॒ती हे॑म॒न्तो वा᳚च्याय॒नो वा᳚च्याय॒नो हे॑म॒न्तो मा॒ती मा॒ती हे॑म॒न्तो वा᳚च्याय॒नः । \newline
12. हे॒म॒न्तो वा᳚च्याय॒नो वा᳚च्याय॒नो हे॑म॒न्तो हे॑म॒न्तो वा᳚च्याय॒नः प॒ङ्क्तिः प॒ङ्क्तिर् वा᳚च्याय॒नो हे॑म॒न्तो हे॑म॒न्तो वा᳚च्याय॒नः प॒ङ्क्तिः । \newline
13. वा॒च्या॒य॒नः प॒ङ्क्तिः प॒ङ्क्तिर् वा᳚च्याय॒नो वा᳚च्याय॒नः प॒ङ्क्तिर्. है॑म॒न्ती है॑म॒न्ती प॒ङ्क्तिर् वा᳚च्याय॒नो वा᳚च्याय॒नः प॒ङ्क्तिर्. है॑म॒न्ती । \newline
14. प॒ङ्क्तिर्. है॑म॒न्ती है॑म॒न्ती प॒ङ्क्तिः प॒ङ्क्तिर्. है॑म॒न्ती प॒ङ्क्त्यै प॒ङ्क्त्यै है॑म॒न्ती प॒ङ्क्तिः प॒ङ्क्तिर्. है॑म॒न्ती प॒ङ्क्त्यै । \newline
15. है॒म॒न्ती प॒ङ्क्त्यै प॒ङ्क्त्यै है॑म॒न्ती है॑म॒न्ती प॒ङ्क्त्यै नि॒धन॑वन् नि॒धन॑वत् प॒ङ्क्त्यै है॑म॒न्ती है॑म॒न्ती प॒ङ्क्त्यै नि॒धन॑वत् । \newline
16. प॒ङ्क्त्यै नि॒धन॑वन् नि॒धन॑वत् प॒ङ्क्त्यै प॒ङ्क्त्यै नि॒धन॑वन् नि॒धन॑वतो नि॒धन॑वतो नि॒धन॑वत् प॒ङ्क्त्यै प॒ङ्क्त्यै नि॒धन॑वन् नि॒धन॑वतः । \newline
17. नि॒धन॑वन् नि॒धन॑वतो नि॒धन॑वतो नि॒धन॑वन् नि॒धन॑वन् नि॒धन॑वत आग्रय॒ण आ᳚ग्रय॒णो नि॒धन॑वतो नि॒धन॑वन् नि॒धन॑वन् नि॒धन॑वत आग्रय॒णः । \newline
18. नि॒धन॑व॒दिति॑ नि॒धन॑ - व॒त् । \newline
19. नि॒धन॑वत आग्रय॒ण आ᳚ग्रय॒णो नि॒धन॑वतो नि॒धन॑वत आग्रय॒ण आ᳚ग्रय॒णा दा᳚ग्रय॒णा दा᳚ग्रय॒णो नि॒धन॑वतो नि॒धन॑वत आग्रय॒ण आ᳚ग्रय॒णात् । \newline
20. नि॒धन॑वत॒ इति॑ नि॒धन॑ - व॒तः॒ । \newline
21. आ॒ग्र॒य॒ण आ᳚ग्रय॒णा दा᳚ग्रय॒णा दा᳚ग्रय॒ण आ᳚ग्रय॒ण आ᳚ग्रय॒णात् त्रि॑णवत्रयस्त्रिꣳ॒॒शौ त्रि॑णवत्रयस्त्रिꣳ॒॒शा वा᳚ग्रय॒णा दा᳚ग्रय॒ण आ᳚ग्रय॒ण आ᳚ग्रय॒णात् त्रि॑णवत्रयस्त्रिꣳ॒॒शौ । \newline
22. आ॒ग्र॒य॒णात् त्रि॑णवत्रयस्त्रिꣳ॒॒शौ त्रि॑णवत्रयस्त्रिꣳ॒॒शा वा᳚ग्रय॒णा दा᳚ग्रय॒णात् त्रि॑णवत्रयस्त्रिꣳ॒॒शौ त्रि॑णवत्रयस्त्रिꣳ॒॒शाभ्या᳚म् त्रिणवत्रयस्त्रिꣳ॒॒शाभ्या᳚म् त्रिणवत्रयस्त्रिꣳ॒॒शा वा᳚ग्रय॒णा दा᳚ग्रय॒णात् त्रि॑णवत्रयस्त्रिꣳ॒॒शौ त्रि॑णवत्रयस्त्रिꣳ॒॒शाभ्या᳚म् । \newline
23. त्रि॒ण॒व॒त्र॒य॒स्त्रिꣳ॒॒शौ त्रि॑णवत्रयस्त्रिꣳ॒॒शाभ्या᳚म् त्रिणवत्रयस्त्रिꣳ॒॒शाभ्या᳚म् त्रिणवत्रयस्त्रिꣳ॒॒शौ त्रि॑णवत्रयस्त्रिꣳ॒॒शौ त्रि॑णवत्रयस्त्रिꣳ॒॒शाभ्याꣳ॑ शाक्वररैव॒ते शा᳚क्वररैव॒ते त्रि॑णवत्रयस्त्रिꣳ॒॒शाभ्या᳚म् त्रिणवत्रयस्त्रिꣳ॒॒शौ त्रि॑णवत्रयस्त्रिꣳ॒॒शौ त्रि॑णवत्रयस्त्रिꣳ॒॒शाभ्याꣳ॑ शाक्वररैव॒ते । \newline
24. त्रि॒ण॒व॒त्र॒य॒स्त्रिꣳ॒॒शाविति॑ त्रिणव - त्र॒य॒स्त्रिꣳ॒॒शौ । \newline
25. त्रि॒ण॒व॒त्र॒य॒स्त्रिꣳ॒॒शाभ्याꣳ॑ शाक्वररैव॒ते शा᳚क्वररैव॒ते त्रि॑णवत्रयस्त्रिꣳ॒॒शाभ्या᳚म् त्रिणवत्रयस्त्रिꣳ॒॒शाभ्याꣳ॑ शाक्वररैव॒ते शा᳚क्वररैव॒ताभ्याꣳ॑ शाक्वररैव॒ताभ्याꣳ॑ शाक्वररैव॒ते त्रि॑णवत्रयस्त्रिꣳ॒॒शाभ्या᳚म् त्रिणवत्रयस्त्रिꣳ॒॒शाभ्याꣳ॑ शाक्वररैव॒ते शा᳚क्वररैव॒ताभ्या᳚म् । \newline
26. त्रि॒ण॒व॒त्र॒य॒स्त्रिꣳ॒॒शाभ्या॒मिति॑ त्रिणव - त्र॒य॒स्त्रिꣳ॒॒शाभ्या᳚म् । \newline
27. शा॒क्व॒र॒रै॒व॒ते शा᳚क्वररैव॒ताभ्याꣳ॑ शाक्वररैव॒ताभ्याꣳ॑ शाक्वररैव॒ते शा᳚क्वररैव॒ते शा᳚क्वररैव॒ताभ्यां᳚ ॅवि॒श्वक॑र्मा वि॒श्वक॑र्मा शाक्वररैव॒ताभ्याꣳ॑ शाक्वररैव॒ते शा᳚क्वररैव॒ते शा᳚क्वररैव॒ताभ्यां᳚ ॅवि॒श्वक॑र्मा । \newline
28. शा॒क्व॒र॒रै॒व॒ते इति॑ शाक्वर - रै॒व॒ते । \newline
29. शा॒क्व॒र॒रै॒व॒ताभ्यां᳚ ॅवि॒श्वक॑र्मा वि॒श्वक॑र्मा शाक्वररैव॒ताभ्याꣳ॑ शाक्वररैव॒ताभ्यां᳚ ॅवि॒श्वक॒र्म र्.षि॒र्॒. ऋषि॑र् वि॒श्वक॑र्मा शाक्वररैव॒ताभ्याꣳ॑ शाक्वररैव॒ताभ्यां᳚ ॅवि॒श्वक॒र्मर्.षिः॑ । \newline
30. शा॒क्व॒र॒रै॒व॒ताभ्या॒मिति॑ शाक्वर - रै॒व॒ताभ्या᳚म् । \newline
31. वि॒श्वक॒र्म र्.षि॒र्॒. ऋषि॑र् वि॒श्वक॑र्मा वि॒श्वक॒र्मर्.षिः॑ प्र॒जाप॑तिगृहीतया प्र॒जाप॑तिगृहीत॒य र्.षि॑र् वि॒श्वक॑र्मा वि॒श्वक॒र्मर्.षिः॑ प्र॒जाप॑तिगृहीतया । \newline
32. वि॒श्वक॒र्मेति॑ वि॒श्व - क॒र्मा॒ । \newline
33. ऋषिः॑ प्र॒जाप॑तिगृहीतया प्र॒जाप॑तिगृहीत॒य र्.षि॒र्॒. ऋषिः॑ प्र॒जाप॑तिगृहीतया॒ त्वया॒ त्वया᳚ प्र॒जाप॑तिगृहीत॒य र्.षि॒र्॒. ऋषिः॑ प्र॒जाप॑तिगृहीतया॒ त्वया᳚ । \newline
34. प्र॒जाप॑तिगृहीतया॒ त्वया॒ त्वया᳚ प्र॒जाप॑तिगृहीतया प्र॒जाप॑तिगृहीतया॒ त्वया॒ वाचं॒ ॅवाच॒म् त्वया᳚ प्र॒जाप॑तिगृहीतया प्र॒जाप॑तिगृहीतया॒ त्वया॒ वाच᳚म् । \newline
35. प्र॒जाप॑तिगृहीत॒येति॑ प्र॒जाप॑ति - गृ॒ही॒त॒या॒ । \newline
36. त्वया॒ वाचं॒ ॅवाच॒म् त्वया॒ त्वया॒ वाच॑म् गृह्णामि गृह्णामि॒ वाच॒म् त्वया॒ त्वया॒ वाच॑म् गृह्णामि । \newline
37. वाच॑म् गृह्णामि गृह्णामि॒ वाचं॒ ॅवाच॑म् गृह्णामि प्र॒जाभ्यः॑ प्र॒जाभ्यो॑ गृह्णामि॒ वाचं॒ ॅवाच॑म् गृह्णामि प्र॒जाभ्यः॑ । \newline
38. गृ॒ह्णा॒मि॒ प्र॒जाभ्यः॑ प्र॒जाभ्यो॑ गृह्णामि गृह्णामि प्र॒जाभ्यः॑ । \newline
39. प्र॒जाभ्य॒ इति॑ प्र - जाभ्यः॑ । \newline
\pagebreak
\markright{ TS 4.3.3.1  \hfill https://www.vedavms.in \hfill}

\section{ TS 4.3.3.1 }

\textbf{TS 4.3.3.1 } \newline
\textbf{Samhita Paata} \newline

प्राची॑ दि॒शां ॅव॑स॒न्त ऋ॑तू॒नाम॒ग्निर्दे॒वता॒ ब्रह्म॒ द्रवि॑णं त्रि॒वृथ् स्तोमः॒ स उ॑ पञ्चद॒शव॑र्तनि॒-स्त्र्यवि॒र्वयः॑ कृ॒तमया॑नां पुरोवा॒तो वातः॒ सान॑ग॒ ऋषि॑र्दक्षि॒णा दि॒शां ग्री॒ष्म ऋ॑तू॒नामिन्द्रो॑ दे॒वता᳚ क्ष॒त्रं द्रवि॑णं पञ्चद॒शः स्तोमः॒ स उ॑ सप्तद॒श व॑र्तनि-र्दि॑त्य॒वाड्-वय॒स्त्रेताऽया॑नां दक्षिणाद्वा॒तो वातः॑ सना॒तन॒ ऋषिः॑ प्र॒तीची॑ दि॒शां ॅव॒र्॒.षा ऋ॑तू॒नां ॅविश्वे॑ दे॒वा दे॒वता॒ विड्-[  ] \newline

\textbf{Pada Paata} \newline

प्राची᳚ । दि॒शाम् । व॒स॒न्तः । ऋ॒तू॒नाम् । अ॒ग्निः । दे॒वता᳚ । ब्रह्म॑ । द्रवि॑णम् । त्रि॒वृदिति॑ त्रि - वृत् । स्तोमः॑ । सः । उ॒ । प॒ञ्च॒द॒शव॑र्तनि॒रिति॑ पञ्चद॒श - व॒र्त॒निः॒ । त्र्यवि॒रिति॑ त्रि - अविः॑ । वयः॑ । कृ॒तम् । अया॑नाम् । पु॒रो॒वा॒त इति॑ पुरः - वा॒तः । वातः॑ । सान॑गः । ऋषिः॑ । द॒क्षि॒णा । दि॒शाम् । ग्री॒ष्मः । ऋ॒तू॒नाम् । इन्द्रः॑ । दे॒वता᳚ । क्ष॒त्रम् । द्रवि॑णम् । प॒ञ्च॒द॒श इति॑ पञ्च - द॒शः । स्तोमः॑ । सः । उ॒ । स॒प्त॒द॒शव॑र्तनि॒रिति॑ सप्तद॒श - व॒र्त॒निः॒ । दि॒त्य॒वाडिति॑ दित्य - वाट् । वयः॑ । त्रेता᳚ । अया॑नाम् । द॒क्षि॒णा॒द्वा॒त इति॑ दक्षिणात् - वा॒तः । वातः॑ । स॒ना॒तन॒ इति॑ सना - तनः॑ । ऋषिः॑ । प्र॒तीची᳚ । दि॒शाम् । व॒र्॒.षाः । ऋ॒तू॒नाम् । विश्वे᳚ । दे॒वाः । दे॒वता᳚ । विट् ।  \newline


\textbf{Krama Paata} \newline

प्राची॑ दि॒शाम् । दि॒शां ॅव॑स॒न्तः । व॒स॒न्त ऋ॑तू॒नाम् । ऋ॒तू॒नाम॒ग्निः । अ॒ग्निर् दे॒वता᳚ । दे॒वता॒ ब्रह्म॑ । ब्रह्म॒ द्रवि॑णम् । द्रवि॑णम् त्रि॒वृत् । त्रि॒वृथ् स्तोमः॑ । त्रि॒वृदिति॑ त्रि - वृत् । स्तोमः॒ सः । स उ॑ । उ॒ प॒ञ्च॒द॒शव॑र्तनिः । प॒ञ्च॒द॒शव॑र्तनि॒ स्त्र्यविः॑ । प॒ञ्च॒द॒शव॑र्तनि॒रिति॑ पञ्चद॒श - व॒र्त॒निः॒ । त्र्यवि॒र् वयः॑ । त्र्यवि॒रिति॑ त्रि - अविः॑ । वयः॑ कृ॒तम् । कृ॒तमया॑नाम् । अया॑नाम् पुरोवा॒तः । पु॒रो॒वा॒तो वातः॑ । पु॒रो॒वा॒त इति॑ पुरः - वा॒तः । वातः॒ सान॑गः । सान॑ग॒ ऋषिः॑ । ऋषि॑र् दक्षि॒णा । द॒क्षि॒णा दि॒शाम् । दि॒शाम् ग्री॒ष्मः । ग्री॒ष्म ऋ॑तू॒नाम् । ऋ॒तू॒नामिन्द्रः॑ । इन्द्रो॑ दे॒वता᳚ । दे॒वता᳚ क्ष॒त्रम् । क्ष॒त्रम् द्रवि॑णम् । द्रवि॑णम् पञ्चद॒शः । प॒ञ्च॒द॒शः स्तोमः॑ । प॒ञ्च॒द॒श इति॑ पञ्च - द॒शः । स्तोमः॒ सः । स उ॑ । उ॒ स॒प्त॒द॒शव॑र्तनिः । स॒प्त॒द॒शव॑र्तनिर् दित्य॒वाट् । स॒प्त॒द॒शव॑र्तनि॒रिति॑ सप्तद॒श - व॒र्त॒निः॒ । दि॒त्य॒वाड् वयः॑ । दि॒त्य॒वाडिति॑ दित्य - वाट् । वय॒स्त्रेता᳚ । त्रेताऽया॑नाम् । अया॑नाम् दक्षिणाद्वा॒तः । द॒क्षि॒णा॒द्वा॒तो वातः॑ । द॒क्षि॒णा॒द्वा॒त इति॑ दक्षिणात् - वा॒तः । वातः॑ सना॒तनः॑ । स॒ना॒तन॒ ऋषिः॑ । स॒ना॒तन॒ इति॑ सना - तनः॑ । ऋषिः॑ प्र॒तीची᳚ । प्र॒तीची॑ दि॒शाम् । दि॒शां ॅव॒र्.॒षाः । व॒र्.॒षा ऋ॑तू॒नाम् । ऋ॒तू॒नां ॅविश्वे᳚ । विश्वे॑ दे॒वाः । दे॒वा दे॒वता᳚ । दे॒वता॒ विट् । विड् द्रवि॑णम् \newline

\textbf{Jatai Paata} \newline

1. प्राची॑ दि॒शाम् दि॒शाम् प्राची॒ प्राची॑ दि॒शाम् । \newline
2. दि॒शां ॅव॑स॒न्तो व॑स॒न्तो दि॒शाम् दि॒शां ॅव॑स॒न्तः । \newline
3. व॒स॒न्त ऋ॑तू॒ना मृ॑तू॒नां ॅव॑स॒न्तो व॑स॒न्त ऋ॑तू॒नाम् । \newline
4. ऋ॒तू॒ना म॒ग्नि र॒ग्निर्. ऋ॑तू॒ना मृ॑तू॒ना म॒ग्निः । \newline
5. अ॒ग्निर् दे॒वता॑ दे॒वता॒ ऽग्नि र॒ग्निर् दे॒वता᳚ । \newline
6. दे॒वता॒ ब्रह्म॒ ब्रह्म॑ दे॒वता॑ दे॒वता॒ ब्रह्म॑ । \newline
7. ब्रह्म॒ द्रवि॑ण॒म् द्रवि॑ण॒म् ब्रह्म॒ ब्रह्म॒ द्रवि॑णम् । \newline
8. द्रवि॑णम् त्रि॒वृत् त्रि॒वृद् द्रवि॑ण॒म् द्रवि॑णम् त्रि॒वृत् । \newline
9. त्रि॒वृथ् स्तोमः॒ स्तोम॑ स्त्रि॒वृत् त्रि॒वृथ् स्तोमः॑ । \newline
10. त्रि॒वृदिति॑ त्रि - वृत् । \newline
11. स्तोमः॒ स स स्तोमः॒ स्तोमः॒ सः । \newline
12. स उ॑ वु॒ स स उ॑ । \newline
13. उ॒ प॒ञ्च॒द॒शव॑र्तनिः पञ्चद॒शव॑र्तनिरु वु पञ्चद॒शव॑र्तनिः । \newline
14. प॒ञ्च॒द॒शव॑र्तनि॒ स्त्र्यवि॒ स्त्र्यविः॑ पञ्चद॒शव॑र्तनिः पञ्चद॒शव॑र्तनि॒ स्त्र्यविः॑ । \newline
15. प॒ञ्च॒द॒शव॑र्तनि॒रिति॑ पञ्चद॒श - व॒र्त॒निः॒ । \newline
16. त्र्यवि॒र् वयो॒ वय॒ स्त्र्यवि॒ स्त्र्यवि॒र् वयः॑ । \newline
17. त्र्यवि॒रिति॑ त्रि - अविः॑ । \newline
18. वयः॑ कृ॒तम् कृ॒तं ॅवयो॒ वयः॑ कृ॒तम् । \newline
19. कृ॒त मया॑ना॒ मया॑नाम् कृ॒तम् कृ॒त मया॑नाम् । \newline
20. अया॑नाम् पुरोवा॒तः पु॑रोवा॒तो ऽया॑ना॒ मया॑नाम् पुरोवा॒तः । \newline
21. पु॒रो॒वा॒तो वातो॒ वातः॑ पुरोवा॒तः पु॑रोवा॒तो वातः॑ । \newline
22. पु॒रो॒वा॒त इति॑ पुरः - वा॒तः । \newline
23. वातः॒ सान॑गः॒ सान॑गो॒ वातो॒ वातः॒ सान॑गः । \newline
24. सान॑ग॒ ऋषि॒र्॒. ऋषिः॒ सान॑गः॒ सान॑ग॒ ऋषिः॑ । \newline
25. ऋषि॑र् दक्षि॒णा द॑क्षि॒ण र्.षि॒र्॒. ऋषि॑र् दक्षि॒णा । \newline
26. द॒क्षि॒णा दि॒शाम् दि॒शाम् द॑क्षि॒णा द॑क्षि॒णा दि॒शाम् । \newline
27. दि॒शाम् ग्री॒ष्मो ग्री॒ष्मो दि॒शाम् दि॒शाम् ग्री॒ष्मः । \newline
28. ग्री॒ष्म ऋ॑तू॒ना मृ॑तू॒नाम् ग्री॒ष्मो ग्री॒ष्म ऋ॑तू॒नाम् । \newline
29. ऋ॒तू॒ना मिन्द्र॒ इन्द्र॑ ऋतू॒ना मृ॑तू॒ना मिन्द्रः॑ । \newline
30. इन्द्रो॑ दे॒वता॑ दे॒व तेन्द्र॒ इन्द्रो॑ दे॒वता᳚ । \newline
31. दे॒वता᳚ क्ष॒त्रम् क्ष॒त्रम् दे॒वता॑ दे॒वता᳚ क्ष॒त्रम् । \newline
32. क्ष॒त्रम् द्रवि॑ण॒म् द्रवि॑णम् क्ष॒त्रम् क्ष॒त्रम् द्रवि॑णम् । \newline
33. द्रवि॑णम् पञ्चद॒शः प॑ञ्चद॒शो द्रवि॑ण॒म् द्रवि॑णम् पञ्चद॒शः । \newline
34. प॒ञ्च॒द॒शः स्तोमः॒ स्तोमः॑ पञ्चद॒शः प॑ञ्चद॒शः स्तोमः॑ । \newline
35. प॒ञ्च॒द॒श इति॑ पञ्च - द॒शः । \newline
36. स्तोमः॒ स स स्तोमः॒ स्तोमः॒ सः । \newline
37. स उ॑ वु॒ स स उ॑ । \newline
38. उ॒ स॒प्त॒द॒शव॑र्तनिः सप्तद॒शव॑र्तनिरु  वु सप्तद॒शव॑र्तनिः । \newline
39. स॒प्त॒द॒शव॑र्तनिर् दित्य॒वाड् दि॑त्य॒वाट् थ् स॑प्तद॒शव॑र्तनिः सप्तद॒शव॑र्तनिर् दित्य॒वाट् । \newline
40. स॒प्त॒द॒शव॑र्तनि॒रिति॑ सप्तद॒श - व॒र्त॒निः॒ । \newline
41. दि॒त्य॒वाड् वयो॒ वयो॑ दित्य॒वाड् दि॑त्य॒वाड् वयः॑ । \newline
42. दि॒त्य॒वाडिति॑ दित्य - वाट् । \newline
43. वय॒ स्त्रेता॒ त्रेता॒ वयो॒ वय॒ स्त्रेता᳚ । \newline
44. त्रेता ऽया॑ना॒ मया॑ना॒म् त्रेता॒ त्रेता ऽया॑नाम् । \newline
45. अया॑नाम् दक्षिणाद्वा॒तो द॑क्षिणाद्वा॒तो ऽया॑ना॒ मया॑नाम् दक्षिणाद्वा॒तः । \newline
46. द॒क्षि॒णा॒द्वा॒तो वातो॒ वातो॑ दक्षिणाद्वा॒तो द॑क्षिणाद्वा॒तो वातः॑ । \newline
47. द॒क्षि॒णा॒द्वा॒त इति॑ दक्षिणात् - वा॒तः । \newline
48. वातः॑ सना॒तनः॑ सना॒तनो॒ वातो॒ वातः॑ सना॒तनः॑ । \newline
49. स॒ना॒तन॒ ऋषि॒र्॒. ऋषिः॑ सना॒तनः॑ सना॒तन॒ ऋषिः॑ । \newline
50. स॒ना॒तन॒ इति॑ सना - तनः॑ । \newline
51. ऋषिः॑ प्र॒तीची᳚ प्र॒तीच्यृ षि॒र्॒. ऋषिः॑ प्र॒तीची᳚ । \newline
52. प्र॒तीची॑ दि॒शाम् दि॒शाम् प्र॒तीची᳚ प्र॒तीची॑ दि॒शाम् । \newline
53. दि॒शां ॅव॒र्॒.षा व॒र्॒.षा दि॒शाम् दि॒शां ॅव॒र्॒.षाः । \newline
54. व॒र्॒.षा ऋ॑तू॒ना मृ॑तू॒नां ॅव॒र्॒.षा व॒र्॒.षा ऋ॑तू॒नाम् । \newline
55. ऋ॒तू॒नां ॅविश्वे॒ विश्व॑ ऋतू॒ना मृ॑तू॒नां ॅविश्वे᳚ । \newline
56. विश्वे॑ दे॒वा दे॒वा विश्वे॒ विश्वे॑ दे॒वाः । \newline
57. दे॒वा दे॒वता॑ दे॒वता॑ दे॒वा दे॒वा दे॒वता᳚ । \newline
58. दे॒वता॒ विड् विड् दे॒वता॑ दे॒वता॒ विट् । \newline
59. विड् द्रवि॑ण॒म् द्रवि॑णं॒ ॅविड् विड् द्रवि॑णम् । \newline

\textbf{Ghana Paata } \newline

1. प्राची॑ दि॒शाम् दि॒शाम् प्राची॒ प्राची॑ दि॒शां ॅव॑स॒न्तो व॑स॒न्तो दि॒शाम् प्राची॒ प्राची॑ दि॒शां ॅव॑स॒न्तः । \newline
2. दि॒शां ॅव॑स॒न्तो व॑स॒न्तो दि॒शाम् दि॒शां ॅव॑स॒न्त ऋ॑तू॒ना मृ॑तू॒नां ॅव॑स॒न्तो दि॒शाम् दि॒शां ॅव॑स॒न्त ऋ॑तू॒नाम् । \newline
3. व॒स॒न्त ऋ॑तू॒ना मृ॑तू॒नां ॅव॑स॒न्तो व॑स॒न्त ऋ॑तू॒ना म॒ग्नि र॒ग्निर्. ऋ॑तू॒नां ॅव॑स॒न्तो व॑स॒न्त ऋ॑तू॒ना म॒ग्निः । \newline
4. ऋ॒तू॒ना म॒ग्नि र॒ग्निर्. ऋ॑तू॒ना मृ॑तू॒ना म॒ग्निर् दे॒वता॑ दे॒वता॒ ऽग्निर्. ऋ॑तू॒ना मृ॑तू॒ना म॒ग्निर् दे॒वता᳚ । \newline
5. अ॒ग्निर् दे॒वता॑ दे॒वता॒ ऽग्नि र॒ग्निर् दे॒वता॒ ब्रह्म॒ ब्रह्म॑ दे॒वता॒ ऽग्नि र॒ग्निर् दे॒वता॒ ब्रह्म॑ । \newline
6. दे॒वता॒ ब्रह्म॒ ब्रह्म॑ दे॒वता॑ दे॒वता॒ ब्रह्म॒ द्रवि॑ण॒म् द्रवि॑ण॒म् ब्रह्म॑ दे॒वता॑ दे॒वता॒ ब्रह्म॒ द्रवि॑णम् । \newline
7. ब्रह्म॒ द्रवि॑ण॒म् द्रवि॑ण॒म् ब्रह्म॒ ब्रह्म॒ द्रवि॑णम् त्रि॒वृत् त्रि॒वृद् द्रवि॑ण॒म् ब्रह्म॒ ब्रह्म॒ द्रवि॑णम् त्रि॒वृत् । \newline
8. द्रवि॑णम् त्रि॒वृत् त्रि॒वृद् द्रवि॑ण॒म् द्रवि॑णम् त्रि॒वृथ् स्तोमः॒ स्तोम॑ स्त्रि॒वृद् द्रवि॑ण॒म् द्रवि॑णम् त्रि॒वृथ् स्तोमः॑ । \newline
9. त्रि॒वृथ् स्तोमः॒ स्तोम॑ स्त्रि॒वृत् त्रि॒वृथ् स्तोमः॒ स स स्तोम॑ स्त्रि॒वृत् त्रि॒वृथ् स्तोमः॒ सः । \newline
10. त्रि॒वृदिति॑ त्रि - वृत् । \newline
11. स्तोमः॒ स स स्तोमः॒ स्तोमः॒ स उ॑ वु॒ स स्तोमः॒ स्तोमः॒ स उ॑ । \newline
12. स उ॑  वु॒ स स उ॑ पञ्चद॒शव॑र्तनिः पञ्चद॒शव॑र्तनिरु॒ स स उ॑ पञ्चद॒शव॑र्तनिः । \newline
13. उ॒ प॒ञ्च॒द॒शव॑र्तनिः पञ्चद॒शव॑र्तनिरु वु पञ्चद॒शव॑र्तनि॒ स्त्र्यवि॒ स्त्र्यविः॑ पञ्चद॒शव॑र्तनिरु वु पञ्चद॒शव॑र्तनि॒ स्त्र्यविः॑ । \newline
14. प॒ञ्च॒द॒शव॑र्तनि॒ स्त्र्यवि॒ स्त्र्यविः॑ पञ्चद॒शव॑र्तनिः पञ्चद॒शव॑र्तनि॒ स्त्र्यवि॒र् वयो॒ वय॒ स्त्र्यविः॑ पञ्चद॒शव॑र्तनिः पञ्चद॒शव॑र्तनि॒ स्त्र्यवि॒र् वयः॑ । \newline
15. प॒ञ्च॒द॒शव॑र्तनि॒रिति॑ पञ्चद॒श - व॒र्त॒निः॒ । \newline
16. त्र्यवि॒र् वयो॒ वय॒ स्त्र्यवि॒ स्त्र्यवि॒र् वयः॑ कृ॒तम् कृ॒तं ॅवय॒ स्त्र्यवि॒ स्त्र्यवि॒र् वयः॑ कृ॒तम् । \newline
17. त्र्यवि॒रिति॑ त्रि - अविः॑ । \newline
18. वयः॑ कृ॒तम् कृ॒तं ॅवयो॒ वयः॑ कृ॒त मया॑ना॒ मया॑नाम् कृ॒तं ॅवयो॒ वयः॑ कृ॒त मया॑नाम् । \newline
19. कृ॒त मया॑ना॒ मया॑नाम् कृ॒तम् कृ॒त मया॑नाम् पुरोवा॒तः पु॑रोवा॒तो ऽया॑नाम् कृ॒तम् कृ॒त मया॑नाम् पुरोवा॒तः । \newline
20. अया॑नाम् पुरोवा॒तः पु॑रोवा॒तो ऽया॑ना॒ मया॑नाम् पुरोवा॒तो वातो॒ वातः॑ पुरोवा॒तो ऽया॑ना॒ मया॑नाम् पुरोवा॒तो वातः॑ । \newline
21. पु॒रो॒वा॒तो वातो॒ वातः॑ पुरोवा॒तः पु॑रोवा॒तो वातः॒ सान॑गः॒ सान॑गो॒ वातः॑ पुरोवा॒तः पु॑रोवा॒तो वातः॒ सान॑गः । \newline
22. पु॒रो॒वा॒त इति॑ पुरः - वा॒तः । \newline
23. वातः॒ सान॑गः॒ सान॑गो॒ वातो॒ वातः॒ सान॑ग॒ ऋषि॒र्॒. ऋषिः॒ सान॑गो॒ वातो॒ वातः॒ सान॑ग॒ ऋषिः॑ । \newline
24. सान॑ग॒ ऋषि॒र्॒. ऋषिः॒ सान॑गः॒ सान॑ग॒ ऋषि॑र् दक्षि॒णा द॑क्षि॒ण र्.षिः॒ सान॑गः॒ सान॑ग॒ ऋषि॑र् दक्षि॒णा । \newline
25. ऋषि॑र् दक्षि॒णा द॑क्षि॒ण र्.षि॒र्॒. ऋषि॑र् दक्षि॒णा दि॒शाम् दि॒शाम् द॑क्षि॒ण र्.षि॒र्॒. ऋषि॑र् दक्षि॒णा दि॒शाम् । \newline
26. द॒क्षि॒णा दि॒शाम् दि॒शाम् द॑क्षि॒णा द॑क्षि॒णा दि॒शाम् ग्री॒ष्मो ग्री॒ष्मो दि॒शाम् द॑क्षि॒णा द॑क्षि॒णा दि॒शाम् ग्री॒ष्मः । \newline
27. दि॒शाम् ग्री॒ष्मो ग्री॒ष्मो दि॒शाम् दि॒शाम् ग्री॒ष्म ऋ॑तू॒ना मृ॑तू॒नाम् ग्री॒ष्मो दि॒शाम् दि॒शाम् ग्री॒ष्म ऋ॑तू॒नाम् । \newline
28. ग्री॒ष्म ऋ॑तू॒ना मृ॑तू॒नाम् ग्री॒ष्मो ग्री॒ष्म ऋ॑तू॒ना मिन्द्र॒ इन्द्र॑ ऋतू॒नाम् ग्री॒ष्मो ग्री॒ष्म ऋ॑तू॒ना मिन्द्रः॑ । \newline
29. ऋ॒तू॒ना मिन्द्र॒ इन्द्र॑ ऋतू॒ना मृ॑तू॒ना मिन्द्रो॑ दे॒वता॑ दे॒वतेन्द्र॑ ऋतू॒ना मृ॑तू॒ना मिन्द्रो॑ दे॒वता᳚ । \newline
30. इन्द्रो॑ दे॒वता॑ दे॒वतेन्द्र॒ इन्द्रो॑ दे॒वता᳚ क्ष॒त्रम् क्ष॒त्रम् दे॒वतेन्द्र॒ इन्द्रो॑ दे॒वता᳚ क्ष॒त्रम् । \newline
31. दे॒वता᳚ क्ष॒त्रम् क्ष॒त्रम् दे॒वता॑ दे॒वता᳚ क्ष॒त्रम् द्रवि॑ण॒म् द्रवि॑णम् क्ष॒त्रम् दे॒वता॑ दे॒वता᳚ क्ष॒त्रम् द्रवि॑णम् । \newline
32. क्ष॒त्रम् द्रवि॑ण॒म् द्रवि॑णम् क्ष॒त्रम् क्ष॒त्रम् द्रवि॑णम् पञ्चद॒शः प॑ञ्चद॒शो द्रवि॑णम् क्ष॒त्रम् क्ष॒त्रम् द्रवि॑णम् पञ्चद॒शः । \newline
33. द्रवि॑णम् पञ्चद॒शः प॑ञ्चद॒शो द्रवि॑ण॒म् द्रवि॑णम् पञ्चद॒शः स्तोमः॒ स्तोमः॑ पञ्चद॒शो द्रवि॑ण॒म् द्रवि॑णम् पञ्चद॒शः स्तोमः॑ । \newline
34. प॒ञ्च॒द॒शः स्तोमः॒ स्तोमः॑ पञ्चद॒शः प॑ञ्चद॒शः स्तोमः॒ स स स्तोमः॑ पञ्चद॒शः प॑ञ्चद॒शः स्तोमः॒ सः । \newline
35. प॒ञ्च॒द॒श इति॑ पञ्च - द॒शः । \newline
36. स्तोमः॒ स स स्तोमः॒ स्तोमः॒ स उ॑ वु॒ स स्तोमः॒ स्तोमः॒ स उ॑ । \newline
37. स उ॑ वु॒ स स उ॑ सप्तद॒शव॑र्तनिः सप्तद॒शव॑र्तनिरु॒ स स उ॑ सप्तद॒शव॑र्तनिः । \newline
38. उ॒ स॒प्त॒द॒शव॑र्तनिः सप्तद॒शव॑र्तनिरु वु सप्तद॒शव॑र्तनिर् दित्य॒वाड् दि॑त्य॒वाट् थ्स॑प्तद॒शव॑र्तनिरु वु सप्तद॒शव॑र्तनिर् दित्य॒वाट् । \newline
39. स॒प्त॒द॒शव॑र्तनिर् दित्य॒वाड् दि॑त्य॒वाट् थ्स॑प्तद॒शव॑र्तनिः सप्तद॒शव॑र्तनिर् दित्य॒वाड् वयो॒ वयो॑ दित्य॒वाट् थ्स॑प्तद॒शव॑र्तनिः सप्तद॒शव॑र्तनिर् दित्य॒वाड् वयः॑ । \newline
40. स॒प्त॒द॒शव॑र्तनि॒रिति॑ सप्तद॒श - व॒र्त॒निः॒ । \newline
41. दि॒त्य॒वाड् वयो॒ वयो॑ दित्य॒वाड् दि॑त्य॒वाड् वय॒ स्त्रेता॒ त्रेता॒ वयो॑ दित्य॒वाड् दि॑त्य॒वाड् वय॒ स्त्रेता᳚ । \newline
42. दि॒त्य॒वाडिति॑ दित्य - वाट् । \newline
43. वय॒ स्त्रेता॒ त्रेता॒ वयो॒ वय॒ स्त्रेता ऽया॑ना॒ मया॑ना॒म् त्रेता॒ वयो॒ वय॒ स्त्रेता ऽया॑नाम् । \newline
44. त्रेता ऽया॑ना॒ मया॑ना॒म् त्रेता॒ त्रेता ऽया॑नाम् दक्षिणाद्वा॒तो द॑क्षिणाद्वा॒तो ऽया॑ना॒म् त्रेता॒ त्रेता ऽया॑नाम् दक्षिणाद्वा॒तः । \newline
45. अया॑नाम् दक्षिणाद्वा॒तो द॑क्षिणाद्वा॒तो ऽया॑ना॒ मया॑नाम् दक्षिणाद्वा॒तो वातो॒ वातो॑ दक्षिणाद्वा॒तो ऽया॑ना॒ मया॑नाम् दक्षिणाद्वा॒तो वातः॑ । \newline
46. द॒क्षि॒णा॒द्वा॒तो वातो॒ वातो॑ दक्षिणाद्वा॒तो द॑क्षिणाद्वा॒तो वातः॑ सना॒तनः॑ सना॒तनो॒ वातो॑ दक्षिणाद्वा॒तो द॑क्षिणाद्वा॒तो वातः॑ सना॒तनः॑ । \newline
47. द॒क्षि॒णा॒द्वा॒त इति॑ दक्षिणात् - वा॒तः । \newline
48. वातः॑ सना॒तनः॑ सना॒तनो॒ वातो॒ वातः॑ सना॒तन॒ ऋषि॒र्॒. ऋषिः॑ सना॒तनो॒ वातो॒ वातः॑ सना॒तन॒ ऋषिः॑ । \newline
49. स॒ना॒तन॒ ऋषि॒र्॒. ऋषिः॑ सना॒तनः॑ सना॒तन॒ ऋषिः॑ प्र॒तीची᳚ प्र॒तीच्यृषिः॑ सना॒तनः॑ सना॒तन॒ ऋषिः॑ प्र॒तीची᳚ । \newline
50. स॒ना॒तन॒ इति॑ सना - तनः॑ । \newline
51. ऋषिः॑ प्र॒तीची᳚ प्र॒ती च्यृषि॒र्॒. ऋषिः॑ प्र॒तीची॑ दि॒शाम् दि॒शाम् प्र॒तीच्यृषि॒र्॒. ऋषिः॑ प्र॒तीची॑ दि॒शाम् । \newline
52. प्र॒तीची॑ दि॒शाम् दि॒शाम् प्र॒तीची᳚ प्र॒तीची॑ दि॒शां ॅव॒र्॒.षा व॒र्॒.षा दि॒शाम् प्र॒तीची᳚ प्र॒तीची॑ दि॒शां ॅव॒र्॒.षाः । \newline
53. दि॒शां ॅव॒र्॒.षा व॒र्॒.षा दि॒शाम् दि॒शां ॅव॒र्॒.षा ऋ॑तू॒ना मृ॑तू॒नां ॅव॒र्॒.षा दि॒शाम् दि॒शां ॅव॒र्॒.षा ऋ॑तू॒नाम् । \newline
54. व॒र्॒.षा ऋ॑तू॒ना मृ॑तू॒नां ॅव॒र्॒.षा व॒र्॒.षा ऋ॑तू॒नां ॅविश्वे॒ विश्व॑ ऋतू॒नां ॅव॒र्॒.षा व॒र्॒.षा ऋ॑तू॒नां ॅविश्वे᳚ । \newline
55. ऋ॒तू॒नां ॅविश्वे॒ विश्व॑ ऋतू॒ना मृ॑तू॒नां ॅविश्वे॑ दे॒वा दे॒वा विश्व॑ ऋतू॒ना मृ॑तू॒नां ॅविश्वे॑ दे॒वाः । \newline
56. विश्वे॑ दे॒वा दे॒वा विश्वे॒ विश्वे॑ दे॒वा दे॒वता॑ दे॒वता॑ दे॒वा विश्वे॒ विश्वे॑ दे॒वा दे॒वता᳚ । \newline
57. दे॒वा दे॒वता॑ दे॒वता॑ दे॒वा दे॒वा दे॒वता॒ विड् विड् दे॒वता॑ दे॒वा दे॒वा दे॒वता॒ विट् । \newline
58. दे॒वता॒ विड् विड् दे॒वता॑ दे॒वता॒ विड् द्रवि॑ण॒म् द्रवि॑णं॒ ॅविड् दे॒वता॑ दे॒वता॒ विड् द्रवि॑णम् । \newline
59. विड् द्रवि॑ण॒म् द्रवि॑णं॒ ॅविड् विड् द्रवि॑णꣳ सप्तद॒शः स॑प्तद॒शो द्रवि॑णं॒ ॅविड् विड् द्रवि॑णꣳ सप्तद॒शः । \newline
\pagebreak
\markright{ TS 4.3.3.2  \hfill https://www.vedavms.in \hfill}

\section{ TS 4.3.3.2 }

\textbf{TS 4.3.3.2 } \newline
\textbf{Samhita Paata} \newline

द्रवि॑णꣳ सप्तद॒श स्तोमः॒ स उ॑ वेकविꣳ॒॒ शव॑र्तनि-स्त्रिव॒थ्सो वयो᳚ द्वाप॒रोऽया॑नां पश्चाद्वा॒तो वातो॑ऽह॒भून॒ ऋषि॒रुदी॑ची दि॒शाꣳ श॒रद्-ऋ॑तू॒नां मि॒त्रावरु॑णौ दे॒वता॑ पु॒ष्टं द्रवि॑णमेकविꣳ॒॒शः स्तोमः॒ स उ॑ त्रिण॒वव॑र्तनि-स्तुर्य॒वाड् वय॑ आस्क॒न्दो ऽया॑नामुत्तराद्-वा॒तो वातः॑ प्र॒त्न ऋषि॑रू॒र्द्ध्वा दि॒शाꣳ हे॑मन्तशिशि॒रावृ॑तू॒नां बृह॒स्पति॑र्दे॒वता॒ वर्चो॒ द्रवि॑णं त्रिण॒व स्तोमः॒ स उ॑ त्रयस्त्रिꣳ॒॒शव॑र्तनिः पष्ठ॒वाद्वयो॑ ( ) ऽभि॒भूरया॑नां ॅविष्वग्वा॒तो वातः॑ सुप॒र्ण ऋषिः॑ पि॒तरः॑ पिताम॒हाः परेऽव॑रे॒ ते नः॑ पान्तु॒ ते नो॑ऽवन्त्व॒स्मिन् ब्रह्म॑न्न॒स्मिन् क्ष॒त्रे᳚ऽस्या-मा॒शिष्य॒स्यां पु॑रो॒धाया॑म॒स्मिन् कर्म॑न्न॒स्यां दे॒वहू᳚त्यां ॥ \newline

\textbf{Pada Paata} \newline

द्रवि॑णम् । स॒प्त॒द॒श इति॑ सप्त-द॒शः । स्तोमः॑ । सः । उ॒ । ए॒क॒विꣳ॒॒शव॑र्तनि॒रित्ये॑कविꣳ॒॒श - व॒र्त॒निः॒ । त्रि॒व॒थ्स इति॑ त्रि-व॒थ्सः । वयः॑ । द्वा॒प॒रः । अया॑नाम् । प॒श्चा॒द्वा॒त इति॑ पश्चात्-वा॒तः । वातः॑ । अ॒ह॒भूनः॑ । ऋषिः॑ । उदी॑ची । दि॒शाम् । श॒रत् । ऋ॒तू॒नाम् । मि॒त्रावरु॑णा॒विति॑ मि॒त्रा - वरु॑णौ । दे॒वता᳚ । पु॒ष्टम् । द्रवि॑णम् । ए॒क॒विꣳ॒॒श इत्ये॑क - विꣳ॒॒शः । स्तोमः॑ । सः । उ॒ । त्रि॒ण॒वव॑र्तनि॒रिति॑ त्रिण॒व - व॒र्त॒निः॒ । तु॒र्य॒वाडिति॑ तुर्य - वाट् । वयः॑ । आ॒स्क॒न्द इत्या᳚ - स्क॒न्दः । अया॑नाम् । उ॒त्त॒रा॒द्वा॒त इत्यु॑त्तरात् - वा॒तः । वातः॑ । प्र॒त्नः । ऋषिः॑ । ऊ॒द्‌र्ध्वा । दि॒शाम् । हे॒म॒न्त॒शि॒शि॒राविति॑ हेमन्त - शि॒शि॒रौ । ऋ॒तू॒नाम् । बृह॒स्पतिः॑ । दे॒वता᳚ । वर्चः॑ । द्रवि॑णम् । त्रि॒ण॒व इति त्रि॑ - न॒वः । स्तोमः॑ । सः । उ॒ । त्र॒य॒स्त्रिꣳ॒॒शव॑र्तनि॒रिति॑ त्रयस्त्रिꣳ॒॒श - व॒र्त॒निः॒ । प॒ष्ठ॒वादिति॑ पष्ठ - वात् । वयः॑ ( ) । अ॒भि॒भूरित्य॑भि - भूः । अया॑नाम् । वि॒ष्व॒ग्वा॒त इति॑ विष्वक् - वा॒तः । वातः॑ । सु॒प॒र्ण इति॑ सु - प॒र्णः । ऋषिः॑ । पि॒तरः॑ । पि॒ता॒म॒हाः । परे᳚ । अव॑रे । ते । नः॒ । पा॒न्तु॒ । ते । नः॒ । अ॒व॒न्तु॒ । अ॒स्मिन्न् । ब्रह्मन्न्॑ । अ॒स्मिन्न् । क्ष॒त्रे । अ॒स्याम् । आ॒शिषीत्या᳚ - शिषि॑ । अ॒स्याम् । पु॒रो॒धाया॒मिति॑ पुरः - धाया᳚म् । अ॒स्मिन्न् । कर्मन्न्॑ । अ॒स्याम् । दे॒वहू᳚त्या॒मिति॑ दे॒व-हू॒त्या॒म् ॥  \newline


\textbf{Krama Paata} \newline

द्रवि॑णꣳ सप्तद॒शः । स॒प्त॒द॒शः स्तोमः॑ । स॒प्त॒द॒श इति॑ सप्त - द॒शः । स्तोमः॒ सः । स उ॑ । 
उ॒वे॒क॒विꣳ॒॒शव॑र्तनिः । ए॒क॒विꣳ॒॒शव॑र्तनि स्त्रिव॒थ्सः । ए॒क॒विꣳ॒॒शव॑र्तनि॒रित्ये॑कविꣳ॒॒श - व॒र्त॒निः॒ । त्रि॒व॒थ्सो वयः॑ । त्रि॒व॒थ्स इति॑ त्रि - व॒थ्सः । वयो᳚ द्वाप॒रः । द्वा॒प॒रोऽया॑नाम् । अया॑नाम् पश्चाद्वा॒तः । प॒श्चा॒द्वा॒तो वातः॑ । प॒श्चा॒द्वा॒त इति॑ पश्चात् - वा॒तः । वातो॑ऽह॒भूनः॑ । अ॒ह॒भून॒ ऋषिः॑ । ऋषि॒रुदी॑ची । उदी॑ची दि॒शाम् । दि॒शाꣳ श॒रत् । श॒रदृ॑तू॒नाम् । ऋ॒तू॒नाम् मि॒त्रावरु॑णौ । मि॒त्रावरु॑णौ दे॒वता᳚ । मि॒त्रावरु॑णा॒विति॑ मि॒त्रा - वरु॑णौ । दे॒वता॑ पु॒ष्टम् । पु॒ष्टम् द्रवि॑णम् । द्रवि॑णमेकविꣳ॒॒शः । ए॒क॒विꣳ॒॒शः स्तोमः॑ । ए॒क॒विꣳ॒॒श इत्ये॑क - विꣳ॒॒शः । स्तोमः॒ सः । स उ॑ । उ॒ त्रि॒ण॒वव॑र्तनिः । त्रि॒ण॒वव॑र्तनि,स्तुर्य॒वाट् । त्रि॒ण॒वव॑र्तनि॒रिति॑ त्रिण॒व - व॒र्त॒निः॒ । तु॒र्य॒वाड् वयः॑ । तु॒र्य॒वाडिति॑ तुर्य - वाट् । वय॑ आस्क॒न्दः । आ॒स्क॒न्दोऽया॑नाम् । आ॒स्क॒न्द इत्या᳚ - स्क॒न्दः । अया॑नामुत्तराद्वा॒तः । उ॒त्त॒रा॒द्वा॒तो वातः॑ । उ॒त्त॒रा॒द्वा॒त इत्यु॑त्तरात् - वा॒तः । वातः॑ प्र॒त्नः । प्र॒त्न ऋषिः॑ । ऋषि॑रू॒र्द्ध्वा । ऊ॒र्द्ध्वा दि॒शाम् । दि॒शाꣳ हे॑मन्तशिशि॒रौ । हे॒म॒न्त॒शि॒शि॒रावृ॑तू॒नाम् । हे॒म॒न्त॒शि॒शि॒राविति॑ हेमन्त - शि॒शि॒रौ । ऋ॒तू॒नाम् बृह॒स्पतिः॑ । बृह॒स्पति॑र् दे॒वता᳚ । दे॒वता॒ वर्चः॑ । वर्चो॒ द्रवि॑णम् । द्रवि॑णम् त्रिण॒वः । त्रि॒ण॒वः स्तोमः॑ । त्रि॒ण॒व इति॑ त्रि - न॒वः । स्तोमः॒ सः । स उ॑ । उ॒ त्र॒य॒स्त्रिꣳ॒॒शव॑र्तनिः । त्र॒य॒स्त्रिꣳ॒॒शव॑र्तनिः पष्ठ॒वात् । त्र॒य॒स्त्रिꣳ॒॒शव॑र्तनि॒रिति॑ त्रयस्त्रिꣳ॒॒श - व॒र्त॒निः॒ । प॒ष्ठ॒वाद् वयः॑ ( ) । प॒ष्ठ॒वादिति॑ पष्ठ - वात् । वयो॑ऽभि॒भूः । अ॒भि॒भूरया॑नाम् । अ॒भि॒भूरित्य॑भि - भूः । अया॑नाम् ॅविष्वग्वा॒तः । वि॒ष्व॒ग्वा॒तो वातः॑ । वि॒ष्व॒ग्वा॒त इति॑ विष्वक् - वा॒तः । वातः॑ सुप॒र्णः । सु॒प॒र्ण ऋषिः॑ । सु॒प॒र्ण इति॑ सु - प॒र्णः । ऋषिः॑ पि॒तरः॑ । पि॒तरः॑ पिताम॒हाः । पि॒ता॒म॒हाः परे᳚ । परेऽव॑रे । अव॑रे॒ ते । ते नः॑ । नः॒ पा॒न्तु॒ । पा॒न्तु॒ ते । ते नः॑ । नो॒ऽव॒न्तु॒ । अ॒व॒न्त्व॒स्मिन्न् । अ॒स्मिन् ब्रह्मन्न्॑ । ब्रह्म॑न्न॒स्मिन्न् । अ॒स्मिन् क्ष॒त्रे । क्ष॒त्रे᳚ऽस्याम् । अ॒स्यामा॒शिषि॑ । आ॒शिष्य॒स्याम् । आ॒शिषीत्या᳚ - शिषि॑ । अ॒स्याम् पु॑रो॒धाया᳚म् । पु॒रो॒धाया॑म॒स्मिन्न् । पु॒रो॒धाया॒मिति॑ पुरः - धाया᳚म् । अ॒स्मिन् कर्मन्न्॑ । कर्म॑न्न॒स्याम् । अ॒स्याम् दे॒वहू᳚त्याम् । दे॒वहू᳚त्या॒मिति॑ दे॒व - हू॒त्या॒म् । \newline

\textbf{Jatai Paata} \newline

1. द्रवि॑णꣳ सप्तद॒शः स॑प्तद॒शो द्रवि॑ण॒म् द्रवि॑णꣳ सप्तद॒शः । \newline
2. स॒प्त॒द॒शः स्तोमः॒ स्तोमः॑ सप्तद॒शः स॑प्तद॒शः स्तोमः॑ । \newline
3. स॒प्त॒द॒श इति॑ सप्त - द॒शः । \newline
4. स्तोमः॒ स स स्तोमः॒ स्तोमः॒ सः । \newline
5. स उ॑ वु॒ स स उ॑ । \newline
6. उ॒ वे॒क॒विꣳ॒॒शव॑र्तनि रेकविꣳ॒॒शव॑र्तनिरु उ वेकविꣳ॒॒शव॑र्तनिः । \newline
7. ए॒क॒विꣳ॒॒शव॑र्तनि स्त्रिव॒थ्स स्त्रि॑व॒थ्स ए॑कविꣳ॒॒शव॑र्तनि रेकविꣳ॒॒शव॑र्तनि स्त्रिव॒थ्सः । \newline
8. ए॒क॒विꣳ॒॒शव॑र्तनि॒रित्ये॑कविꣳ॒॒श - व॒र्त॒निः॒ । \newline
9. त्रि॒व॒थ्सो वयो॒ वय॑ स्त्रिव॒थ्स स्त्रि॑व॒थ्सो वयः॑ । \newline
10. त्रि॒व॒थ्स इति॑ त्रि - व॒थ्सः । \newline
11. वयो᳚ द्वाप॒रो द्वा॑प॒रो वयो॒ वयो᳚ द्वाप॒रः । \newline
12. द्वा॒प॒रो ऽया॑ना॒ मया॑नाम् द्वाप॒रो द्वा॑प॒रो ऽया॑नाम् । \newline
13. अया॑नाम् पश्चाद्वा॒तः प॑श्चाद्वा॒तो ऽया॑ना॒ मया॑नाम् पश्चाद्वा॒तः । \newline
14. प॒श्चा॒द्वा॒तो वातो॒ वातः॑ पश्चाद्वा॒तः प॑श्चाद्वा॒तो वातः॑ । \newline
15. प॒श्चा॒द्वा॒त इति॑ पश्चात् - वा॒तः । \newline
16. वातो॑ ऽह॒भूनो॑ ऽह॒भूनो॒ वातो॒ वातो॑ ऽह॒भूनः॑ । \newline
17. अ॒ह॒भून॒ ऋषि॒र्॒. ऋषि॑ रह॒भूनो॑ ऽह॒भून॒ ऋषिः॑ । \newline
18. ऋषि॒ रुदी॒ च्युदी॒ च्यृषि॒र्॒. ऋषि॒ रुदी॑ची । \newline
19. उदी॑ची दि॒शाम् दि॒शा मुदी॒ च्युदी॑ची दि॒शाम् । \newline
20. दि॒शाꣳ श॒र च्छ॒रद् दि॒शाम् दि॒शाꣳ श॒रत् । \newline
21. श॒र दृ॑तू॒ना मृ॑तू॒नाꣳ श॒र च्छ॒र दृ॑तू॒नाम् । \newline
22. ऋ॒तू॒नाम् मि॒त्रावरु॑णौ मि॒त्रावरु॑णा वृतू॒ना मृ॑तू॒नाम् मि॒त्रावरु॑णौ । \newline
23. मि॒त्रावरु॑णौ दे॒वता॑ दे॒वता॑ मि॒त्रावरु॑णौ मि॒त्रावरु॑णौ दे॒वता᳚ । \newline
24. मि॒त्रावरु॑णा॒विति॑ मि॒त्रा - वरु॑णौ । \newline
25. दे॒वता॑ पु॒ष्टम् पु॒ष्टम् दे॒वता॑ दे॒वता॑ पु॒ष्टम् । \newline
26. पु॒ष्टम् द्रवि॑ण॒म् द्रवि॑णम् पु॒ष्टम् पु॒ष्टम् द्रवि॑णम् । \newline
27. द्रवि॑ण मेकविꣳ॒॒श ए॑कविꣳ॒॒शो द्रवि॑ण॒म् द्रवि॑ण मेकविꣳ॒॒शः । \newline
28. ए॒क॒विꣳ॒॒शः स्तोमः॒ स्तोम॑ एकविꣳ॒॒श ए॑कविꣳ॒॒शः स्तोमः॑ । \newline
29. ए॒क॒विꣳ॒॒श इत्ये॑क - विꣳ॒॒शः । \newline
30. स्तोमः॒ स स स्तोमः॒ स्तोमः॒ सः । \newline
31. स उ॑ वु॒ स स उ॑ । \newline
32. उ॒ त्रि॒ण॒वव॑र्तनिस्त्रिण॒वव॑र्तनिरु वु त्रिण॒वव॑र्तनिः । \newline
33. त्रि॒ण॒वव॑र्तनि स्तुर्य॒वाट् तु॑र्य॒वाट् त्रि॑ण॒वव॑र्तनि स्त्रिण॒वव॑र्तनि स्तुर्य॒वाट् । \newline
34. त्रि॒ण॒वव॑र्तनि॒रिति॑ त्रिण॒व - व॒र्त॒निः॒ । \newline
35. तु॒र्य॒वाड् वयो॒ वय॑ स्तुर्य॒वाट् तु॑र्य॒वाड् वयः॑ । \newline
36. तु॒र्य॒वाडिति॑ तुर्य - वाट् । \newline
37. वय॑ आस्क॒न्द आ᳚स्क॒न्दो वयो॒ वय॑ आस्क॒न्दः । \newline
38. आ॒स्क॒न्दो ऽया॑ना॒ मया॑ना मास्क॒न्द आ᳚स्क॒न्दो ऽया॑नाम् । \newline
39. आ॒स्क॒न्द इत्या᳚ - स्क॒न्दः । \newline
40. अया॑ना मुत्तराद्वा॒त उ॑त्तराद्वा॒तो ऽया॑ना॒ मया॑ना मुत्तराद्वा॒तः । \newline
41. उ॒त्त॒रा॒द्वा॒तो वातो॒ वात॑ उत्तराद्वा॒त उ॑त्तराद्वा॒तो वातः॑ । \newline
42. उ॒त्त॒रा॒द्वा॒त इत्यु॑त्तरात् - वा॒तः । \newline
43. वातः॑ प्र॒त्नः प्र॒त्नो वातो॒ वातः॑ प्र॒त्नः । \newline
44. प्र॒त्न ऋषि॒र्॒. ऋषिः॑ प्र॒त्नः प्र॒त्न ऋषिः॑ । \newline
45. ऋषि॑ रू॒र्द्ध्वो र्द्ध्वर्. षि॒र्॒. ऋषि॑ रू॒र्द्ध्वा । \newline
46. ऊ॒र्द्ध्वा दि॒शाम् दि॒शा मू॒र्द्ध्वो र्द्ध्वा दि॒शाम् । \newline
47. दि॒शाꣳ हे॑मन्तशिशि॒रौ हे॑मन्तशिशि॒रौ दि॒शाम् दि॒शाꣳ हे॑मन्तशिशि॒रौ । \newline
48. हे॒म॒न्त॒शि॒शि॒रा वृ॑तू॒ना मृ॑तू॒नाꣳ हे॑मन्तशिशि॒रौ हे॑मन्तशिशि॒रा वृ॑तू॒नाम् । \newline
49. हे॒म॒न्त॒शि॒शि॒राविति॑ हेमन्त - शि॒शि॒रौ । \newline
50. ऋ॒तू॒नाम् बृह॒स्पति॒र् बृह॒स्पति॑र्. ऋतू॒ना मृ॑तू॒नाम् बृह॒स्पतिः॑ । \newline
51. बृह॒स्पति॑र् दे॒वता॑ दे॒वता॒ बृह॒स्पति॒र् बृह॒स्पति॑र् दे॒वता᳚ । \newline
52. दे॒वता॒ वर्चो॒ वर्चो॑ दे॒वता॑ दे॒वता॒ वर्चः॑ । \newline
53. वर्चो॒ द्रवि॑ण॒म् द्रवि॑णं॒ ॅवर्चो॒ वर्चो॒ द्रवि॑णम् । \newline
54. द्रवि॑णम् त्रिण॒व स्त्रि॑ण॒वो द्रवि॑ण॒म् द्रवि॑णम् त्रिण॒वः । \newline
55. त्रि॒ण॒वः स्तोमः॒ स्तोम॑ स्त्रिण॒व स्त्रि॑ण॒वः स्तोमः॑ । \newline
56. त्रि॒ण॒व इति त्रि॑ - न॒वः । \newline
57. स्तोमः॒ स स स्तोमः॒ स्तोमः॒ सः । \newline
58. स उ॑ वु॒ स स उ॑ । \newline
59. उ॒ त्र॒य॒स्त्रिꣳ॒॒शव॑र्तनि स्त्रयस्त्रिꣳ॒॒शव॑र्तनिरु वु त्रयस्त्रिꣳ॒॒शव॑र्तनिः । \newline
60. त्र॒य॒स्त्रिꣳ॒॒शव॑र्तनिः पष्ठ॒वात् प॑ष्ठ॒वात् त्र॑यस्त्रिꣳ॒॒शव॑र्तनि स्त्रयस्त्रिꣳ॒॒शव॑र्तनिः पष्ठ॒वात् । \newline
61. त्र॒य॒स्त्रिꣳ॒॒शव॑र्तनि॒रिति॑ त्रयस्त्रिꣳ॒॒श - व॒र्त॒निः॒ । \newline
62. प॒ष्ठ॒वाद् वयो॒ वयः॑ पष्ठ॒वात् प॑ष्ठ॒वाद् वयः॑ । \newline
63. प॒ष्ठ॒वादिति॑ पष्ठ - वात् । \newline
64. वयो॑ ऽभि॒भू र॑भि॒भूर् वयो॒ वयो॑ ऽभि॒भूः । \newline
65. अ॒भि॒भू रया॑ना॒ मया॑ना मभि॒भू र॑भि॒भू रया॑नाम् । \newline
66. अ॒भि॒भूरित्य॑भि - भूः । \newline
67. अया॑नां ॅविष्वग्वा॒तो वि॑ष्वग्वा॒तो ऽया॑ना॒ मया॑नां ॅविष्वग्वा॒तः । \newline
68. वि॒ष्व॒ग्वा॒तो वातो॒ वातो॑ विष्वग्वा॒तो वि॑ष्वग्वा॒तो वातः॑ । \newline
69. वि॒ष्व॒ग्वा॒त इति॑ विष्वक् - वा॒तः । \newline
70. वातः॑ सुप॒र्णः सु॑प॒र्णो वातो॒ वातः॑ सुप॒र्णः । \newline
71. सु॒प॒र्ण ऋषि॒र्॒. ऋषिः॑ सुप॒र्णः सु॑प॒र्ण ऋषिः॑ । \newline
72. सु॒प॒र्ण इति॑ सु - प॒र्णः । \newline
73. ऋषिः॑ पि॒तरः॑ पि॒तर॒ ऋषि॒र्॒. ऋषिः॑ पि॒तरः॑ । \newline
74. पि॒तरः॑ पिताम॒हाः पि॑ताम॒हाः पि॒तरः॑ पि॒तरः॑ पिताम॒हाः । \newline
75. पि॒ता॒म॒हाः परे॒ परे॑ पिताम॒हाः पि॑ताम॒हाः परे᳚ । \newline
76. परे ऽव॒रे ऽव॑रे॒ परे॒ परे ऽव॑रे । \newline
77. अव॑रे॒ ते ते ऽव॒रे ऽव॑रे॒ ते । \newline
78. ते नो॑ न॒ स्ते ते नः॑ । \newline
79. नः॒ पा॒न्तु॒ पा॒न्तु॒ नो॒ नः॒ पा॒न्तु॒ । \newline
80. पा॒न्तु॒ ते ते पा᳚न्तु पान्तु॒ ते । \newline
81. ते नो॑ न॒ स्ते ते नः॑ । \newline
82. नो॒ ऽव॒न् त्व॒ व॒न्तु॒ नो॒ नो॒ ऽव॒न्तु॒ । \newline
83. अ॒व॒न् त्व॒स्मिन् न॒स्मिन् न॑वन् त्ववन् त्व॒स्मिन्न् । \newline
84. अ॒स्मिन् ब्रह्म॒न् ब्रह्म॑न् न॒स्मिन् न॒स्मिन् ब्रह्मन्न्॑ । \newline
85. ब्रह्म॑न् न॒स्मिन् न॒स्मिन् ब्रह्म॒न् ब्रह्म॑न् न॒स्मिन्न् । \newline
86. अ॒स्मिन् क्ष॒त्रे क्ष॒त्रे᳚ ऽस्मिन् न॒स्मिन् क्ष॒त्रे । \newline
87. क्ष॒त्रे᳚ ऽस्या म॒स्याम् क्ष॒त्रे क्ष॒त्रे᳚ ऽस्याम् । \newline
88. अ॒स्या मा॒शिष्या॒ शिष्य॒स्या म॒स्या मा॒शिषि॑ । \newline
89. आ॒शिष्य॒स्या म॒स्या मा॒शिष्या॒ शिष्य॒स्याम् । \newline
90. आ॒शिषीत्या᳚ - शिषि॑ । \newline
91. अ॒स्याम् पु॑रो॒धाया᳚म् पुरो॒धाया॑ म॒स्या म॒स्याम् पु॑रो॒धाया᳚म् । \newline
92. पु॒रो॒धाया॑ म॒स्मिन् न॒स्मिन् पु॑रो॒धाया᳚म् पुरो॒धाया॑ म॒स्मिन्न् । \newline
93. पु॒रो॒धाया॒मिति॑ पुरः - धाया᳚म् । \newline
94. अ॒स्मिन् कर्म॒न् कर्म॑न् न॒स्मिन् न॒स्मिन् कर्मन्न्॑ । \newline
95. कर्म॑न् न॒स्या म॒स्याम् कर्म॒न् कर्म॑न् न॒स्याम् । \newline
96. अ॒स्याम् दे॒वहू᳚त्याम् दे॒वहू᳚त्या म॒स्या म॒स्याम् दे॒वहू᳚त्याम् । \newline
97. दे॒वहू᳚त्या॒मिति॑ दे॒व - हू॒त्या॒म् । \newline

\textbf{Ghana Paata } \newline

1. द्रवि॑णꣳ सप्तद॒शः स॑प्तद॒शो द्रवि॑ण॒म् द्रवि॑णꣳ सप्तद॒शः स्तोमः॒ स्तोमः॑ सप्तद॒शो द्रवि॑ण॒म् द्रवि॑णꣳ सप्तद॒शः स्तोमः॑ । \newline
2. स॒प्त॒द॒शः स्तोमः॒ स्तोमः॑ सप्तद॒शः स॑प्तद॒शः स्तोमः॒ स स स्तोमः॑ सप्तद॒शः स॑प्तद॒शः स्तोमः॒ सः । \newline
3. स॒प्त॒द॒श इति॑ सप्त - द॒शः । \newline
4. स्तोमः॒ स स स्तोमः॒ स्तोमः॒ स उ॑ वु॒ स स्तोमः॒ स्तोमः॒ स उ॑ । \newline
5. स उ॑ वु॒ स स उ॑ वेकविꣳ॒॒शव॑र्तनि रेकविꣳ॒॒शव॑र्तनिरु॒ स स उ॑ वेकविꣳ॒॒शव॑र्तनिः । \newline
6. उ॒ वे॒क॒विꣳ॒॒शव॑र्तनि रेकविꣳ॒॒शव॑र्तनिरु वु वेकविꣳ॒॒शव॑र्तनि स्त्रिव॒थ्स स्त्रि॑व॒थ्स ए॑कविꣳ॒॒शव॑र्तनिरु वु वेकविꣳ॒॒शव॑र्तनि स्त्रिव॒थ्सः । \newline
7. ए॒क॒विꣳ॒॒शव॑र्तनि स्त्रिव॒थ्सस्त्रि॑व॒थ्स ए॑कविꣳ॒॒शव॑र्तनि रेकविꣳ॒॒शव॑र्तनि स्त्रिव॒थ्सो वयो॒ वय॑ स्त्रिव॒थ्स ए॑कविꣳ॒॒शव॑र्तनि रेकविꣳ॒॒शव॑र्तनि स्त्रिव॒थ्सो वयः॑ । \newline
8. ए॒क॒विꣳ॒॒शव॑र्तनि॒रित्ये॑कविꣳ॒॒श - व॒र्त॒निः॒ । \newline
9. त्रि॒व॒थ्सो वयो॒ वय॑ स्त्रिव॒थ्स स्त्रि॑व॒थ्सो वयो᳚ द्वाप॒रो द्वा॑प॒रो वय॑ स्त्रिव॒थ्स स्त्रि॑व॒थ्सो वयो᳚ द्वाप॒रः । \newline
10. त्रि॒व॒थ्स इति॑ त्रि - व॒थ्सः । \newline
11. वयो᳚ द्वाप॒रो द्वा॑प॒रो वयो॒ वयो᳚ द्वाप॒रो ऽया॑ना॒ मया॑नाम् द्वाप॒रो वयो॒ वयो᳚ द्वाप॒रो ऽया॑नाम् । \newline
12. द्वा॒प॒रो ऽया॑ना॒ मया॑नाम् द्वाप॒रो द्वा॑प॒रो ऽया॑नाम् पश्चाद्वा॒तः प॑श्चाद्वा॒तो ऽया॑नाम् द्वाप॒रो द्वा॑प॒रो ऽया॑नाम् पश्चाद्वा॒तः । \newline
13. अया॑नाम् पश्चाद्वा॒तः प॑श्चाद्वा॒तो ऽया॑ना॒ मया॑नाम् पश्चाद्वा॒तो वातो॒ वातः॑ पश्चाद्वा॒तो ऽया॑ना॒ मया॑नाम् पश्चाद्वा॒तो वातः॑ । \newline
14. प॒श्चा॒द्वा॒तो वातो॒ वातः॑ पश्चाद्वा॒तः प॑श्चाद्वा॒तो वातो॑ ऽह॒भूनो॑ ऽह॒भूनो॒ वातः॑ पश्चाद्वा॒तः प॑श्चाद्वा॒तो वातो॑ ऽह॒भूनः॑ । \newline
15. प॒श्चा॒द्वा॒त इति॑ पश्चात् - वा॒तः । \newline
16. वातो॑ ऽह॒भूनो॑ ऽह॒भूनो॒ वातो॒ वातो॑ ऽह॒भून॒ ऋषि॒र्॒. ऋषि॑ रह॒भूनो॒ वातो॒ वातो॑ ऽह॒भून॒ ऋषिः॑ । \newline
17. अ॒ह॒भून॒ ऋषि॒र्॒. ऋषि॑ रह॒भूनो॑ ऽह॒भून॒ ऋषि॒ रुदी॒ च्युदी॒ च्यृषि॑ रह॒भूनो॑ ऽह॒भून॒ ऋषि॒ रुदी॑ची । \newline
18. ऋषि॒ रुदी॒ च्युदी॒ च्यृषि॒र्॒. ऋषि॒ रुदी॑ची दि॒शाम् दि॒शा मुदी॒ च्यृषि॒र्॒. ऋषि॒ रुदी॑ची दि॒शाम् । \newline
19. उदी॑ची दि॒शाम् दि॒शा मुदी॒ च्युदी॑ची दि॒शाꣳ श॒र-च्छ॒रद् दि॒शा मुदी॒ च्युदी॑ची दि॒शाꣳ श॒रत् । \newline
20. दि॒शाꣳ श॒र च्छ॒रद् दि॒शाम् दि॒शाꣳ श॒र दृ॑तू॒ना मृ॑तू॒नाꣳ श॒रद् दि॒शाम् दि॒शाꣳ श॒र दृ॑तू॒नाम् । \newline
21. श॒र दृ॑तू॒ना मृ॑तू॒नाꣳ श॒र च्छ॒र दृ॑तू॒नाम् मि॒त्रावरु॑णौ मि॒त्रावरु॑णा वृतू॒नाꣳ 
श॒र च्छ॒र दृ॑तू॒नाम् मि॒त्रावरु॑णौ । \newline
22. ऋ॒तू॒नाम् मि॒त्रावरु॑णौ मि॒त्रावरु॑णा वृतू॒ना मृ॑तू॒नाम् मि॒त्रावरु॑णौ दे॒वता॑ दे॒वता॑ मि॒त्रावरु॑णा वृतू॒ना मृ॑तू॒नाम् मि॒त्रावरु॑णौ दे॒वता᳚ । \newline
23. मि॒त्रावरु॑णौ दे॒वता॑ दे॒वता॑ मि॒त्रावरु॑णौ मि॒त्रावरु॑णौ दे॒वता॑ पु॒ष्टम् पु॒ष्टम् दे॒वता॑ मि॒त्रावरु॑णौ मि॒त्रावरु॑णौ दे॒वता॑ पु॒ष्टम् । \newline
24. मि॒त्रावरु॑णा॒विति॑ मि॒त्रा - वरु॑णौ । \newline
25. दे॒वता॑ पु॒ष्टम् पु॒ष्टम् दे॒वता॑ दे॒वता॑ पु॒ष्टम् द्रवि॑ण॒म् द्रवि॑णम् पु॒ष्टम् दे॒वता॑ दे॒वता॑ पु॒ष्टम् द्रवि॑णम् । \newline
26. पु॒ष्टम् द्रवि॑ण॒म् द्रवि॑णम् पु॒ष्टम् पु॒ष्टम् द्रवि॑ण मेकविꣳ॒॒श ए॑कविꣳ॒॒शो द्रवि॑णम् पु॒ष्टम् पु॒ष्टम् द्रवि॑ण मेकविꣳ॒॒शः । \newline
27. द्रवि॑ण मेकविꣳ॒॒श ए॑कविꣳ॒॒शो द्रवि॑ण॒म् द्रवि॑ण मेकविꣳ॒॒शः स्तोमः॒ स्तोम॑ एकविꣳ॒॒शो द्रवि॑ण॒म् द्रवि॑ण मेकविꣳ॒॒शः स्तोमः॑ । \newline
28. ए॒क॒विꣳ॒॒शः स्तोमः॒ स्तोम॑ एकविꣳ॒॒श ए॑कविꣳ॒॒शः स्तोमः॒ स स स्तोम॑ एकविꣳ॒॒श ए॑कविꣳ॒॒शः स्तोमः॒ सः । \newline
29. ए॒क॒विꣳ॒॒श इत्ये॑क - विꣳ॒॒शः । \newline
30. स्तोमः॒ स स स्तोमः॒ स्तोमः॒ स उ॑ वु॒ स स्तोमः॒ स्तोमः॒ स उ॑ । \newline
31. स उ॑ वु॒ स स उ॑ त्रिण॒वव॑र्तनि स्त्रिण॒वव॑र्तनिरु॒ स स उ॑ त्रिण॒वव॑र्तनिः । \newline
32. उ॒ त्रि॒ण॒वव॑र्तनि स्त्रिण॒वव॑र्तनिरु वु त्रिण॒वव॑र्तनि स्तुर्य॒वाट् तु॑र्य॒वाट् त्रि॑ण॒वव॑र्तनिरु वु त्रिण॒वव॑र्तनि स्तुर्य॒वाट् । \newline
33. त्रि॒ण॒वव॑र्तनि स्तुर्य॒वाट् तु॑र्य॒वाट् त्रि॑ण॒वव॑र्तनि स्त्रिण॒वव॑र्तनि स्तुर्य॒वाड् वयो॒ वय॑ स्तुर्य॒वाट् त्रि॑ण॒वव॑र्तनि स्त्रिण॒वव॑र्तनि स्तुर्य॒वाड् वयः॑ । \newline
34. त्रि॒ण॒वव॑र्तनि॒रिति॑ त्रिण॒व - व॒र्त॒निः॒ । \newline
35. तु॒र्य॒वाड् वयो॒ वय॑ स्तुर्य॒वाट् तु॑र्य॒वाड् वय॑ आस्क॒न्द आ᳚स्क॒न्दो वय॑ स्तुर्य॒वाट् तु॑र्य॒वाड् वय॑ आस्क॒न्दः । \newline
36. तु॒र्य॒वाडिति॑ तुर्य - वाट् । \newline
37. वय॑ आस्क॒न्द आ᳚स्क॒न्दो वयो॒ वय॑ आस्क॒न्दो ऽया॑ना॒ मया॑ना मास्क॒न्दो वयो॒ वय॑ आस्क॒न्दो ऽया॑नाम् । \newline
38. आ॒स्क॒न्दो ऽया॑ना॒ मया॑ना मास्क॒न्द आ᳚स्क॒न्दो ऽया॑ना मुत्तराद्वा॒त उ॑त्तराद्वा॒तो ऽया॑ना मास्क॒न्द आ᳚स्क॒न्दो ऽया॑ना मुत्तराद्वा॒तः । \newline
39. आ॒स्क॒न्द इत्या᳚ - स्क॒न्दः । \newline
40. अया॑ना मुत्तराद्वा॒त उ॑त्तराद्वा॒तो ऽया॑ना॒ मया॑ना मुत्तराद्वा॒तो वातो॒ वात॑ उत्तराद्वा॒तो ऽया॑ना॒ मया॑ना मुत्तराद्वा॒तो वातः॑ । \newline
41. उ॒त्त॒रा॒द्वा॒तो वातो॒ वात॑ उत्तराद्वा॒त उ॑त्तराद्वा॒तो वातः॑ प्र॒त्नः प्र॒त्नो वात॑ उत्तराद्वा॒त उ॑त्तराद्वा॒तो वातः॑ प्र॒त्नः । \newline
42. उ॒त्त॒रा॒द्वा॒त इत्यु॑त्तरात् - वा॒तः । \newline
43. वातः॑ प्र॒त्नः प्र॒त्नो वातो॒ वातः॑ प्र॒त्न ऋषि॒र्॒. ऋषिः॑ प्र॒त्नो वातो॒ वातः॑ प्र॒त्न ऋषिः॑ । \newline
44. प्र॒त्न ऋषि॒र्॒. ऋषिः॑ प्र॒त्नः प्र॒त्न ऋषि॑ रू॒र्द्ध्वो र्द्ध्वर्.षिः॑ प्र॒त्नः प्र॒त्न ऋषि॑ रू॒र्द्ध्वा । \newline
45. ऋषि॑ रू॒र्द्ध्वो र्द्ध्वर्.षि॒र्॒. ऋषि॑ रू॒र्द्ध्वा दि॒शाम् दि॒शा मू॒र्द्ध्व र्.षि॒र्॒. ऋषि॑ रू॒र्द्ध्वा दि॒शाम् । \newline
46. ऊ॒र्द्ध्वा दि॒शाम् दि॒शा मू॒र्द्ध्वो र्द्ध्वा दि॒शाꣳ हे॑मन्तशिशि॒रौ हे॑मन्तशिशि॒रौ दि॒शा मू॒र्द्ध्वो र्द्ध्वा दि॒शाꣳ हे॑मन्तशिशि॒रौ । \newline
47. दि॒शाꣳ हे॑मन्तशिशि॒रौ हे॑मन्तशिशि॒रौ दि॒शाम् दि॒शाꣳ हे॑मन्तशिशि॒रा वृ॑तू॒ना मृ॑तू॒नाꣳ हे॑मन्तशिशि॒रौ दि॒शाम् दि॒शाꣳ हे॑मन्तशिशि॒रा वृ॑तू॒नाम् । \newline
48. हे॒म॒न्त॒शि॒शि॒रा वृ॑तू॒ना मृ॑तू॒नाꣳ हे॑मन्तशिशि॒रौ हे॑मन्तशिशि॒रा वृ॑तू॒नाम् बृह॒स्पति॒र् बृह॒स्पति॑र्. ऋतू॒नाꣳ हे॑मन्तशिशि॒रौ हे॑मन्तशिशि॒रा वृ॑तू॒नाम् बृह॒स्पतिः॑ । \newline
49. हे॒म॒न्त॒शि॒शि॒राविति॑ हेमन्त - शि॒शि॒रौ । \newline
50. ऋ॒तू॒नाम् बृह॒स्पति॒र् बृह॒स्पति॑र्. ऋतू॒ना मृ॑तू॒नाम् बृह॒स्पति॑र् दे॒वता॑ दे॒वता॒ बृह॒स्पति॑र्. ऋतू॒ना मृ॑तू॒नाम् बृह॒स्पति॑र् दे॒वता᳚ । \newline
51. बृह॒स्पति॑र् दे॒वता॑ दे॒वता॒ बृह॒स्पति॒र् बृह॒स्पति॑र् दे॒वता॒ वर्चो॒ वर्चो॑ दे॒वता॒ बृह॒स्पति॒र् बृह॒स्पति॑र् दे॒वता॒ वर्चः॑ । \newline
52. दे॒वता॒ वर्चो॒ वर्चो॑ दे॒वता॑ दे॒वता॒ वर्चो॒ द्रवि॑ण॒म् द्रवि॑णं॒ ॅवर्चो॑ दे॒वता॑ दे॒वता॒ वर्चो॒ द्रवि॑णम् । \newline
53. वर्चो॒ द्रवि॑ण॒म् द्रवि॑णं॒ ॅवर्चो॒ वर्चो॒ द्रवि॑णम् त्रिण॒व स्त्रि॑ण॒वो द्रवि॑णं॒ ॅवर्चो॒ वर्चो॒ द्रवि॑णम् त्रिण॒वः । \newline
54. द्रवि॑णम् त्रिण॒व स्त्रि॑ण॒वो द्रवि॑ण॒म् द्रवि॑णम् त्रिण॒वः स्तोमः॒ स्तोम॑ स्त्रिण॒वो द्रवि॑ण॒म् द्रवि॑णम् त्रिण॒वः स्तोमः॑ । \newline
55. त्रि॒ण॒वः स्तोमः॒ स्तोम॑ स्त्रिण॒व स्त्रि॑ण॒वः स्तोमः॒ स स स्तोम॑ स्त्रिण॒व स्त्रि॑ण॒वः स्तोमः॒ सः । \newline
56. त्रि॒ण॒व इति त्रि॑ - न॒वः । \newline
57. स्तोमः॒ स स स्तोमः॒ स्तोमः॒ स उ॑ वु॒ स स्तोमः॒ स्तोमः॒ स उ॑ । \newline
58. स उ॑ वु॒ स स उ॑ त्रयस्त्रिꣳ॒॒शव॑र्तनि स्त्रयस्त्रिꣳ॒॒शव॑र्तनि रु॒ स स उ॑ त्रयस्त्रिꣳ॒॒शव॑र्तनिः । \newline
59. उ॒ त्र॒य॒स्त्रिꣳ॒॒शव॑र्तनि स्त्रयस्त्रिꣳ॒॒शव॑र्तनिरु वु त्रयस्त्रिꣳ॒॒शव॑र्तनिः पष्ठ॒वात् प॑ष्ठ॒वात् त्र॑यस्त्रिꣳ॒॒शव॑र्तनिरु वु त्रयस्त्रिꣳ॒॒शव॑र्तनिः पष्ठ॒वात् । \newline
60. त्र॒य॒स्त्रिꣳ॒॒शव॑र्तनिः पष्ठ॒वात् प॑ष्ठ॒वात् त्र॑यस्त्रिꣳ॒॒शव॑र्तनि स्त्रयस्त्रिꣳ॒॒शव॑र्तनिः पष्ठ॒वाद् वयो॒ वयः॑ पष्ठ॒वात् त्र॑यस्त्रिꣳ॒॒शव॑र्तनि स्त्रयस्त्रिꣳ॒॒शव॑र्तनिः पष्ठ॒वाद् वयः॑ । \newline
61. त्र॒य॒स्त्रिꣳ॒॒शव॑र्तनि॒रिति॑ त्रयस्त्रिꣳ॒॒श - व॒र्त॒निः॒ । \newline
62. प॒ष्ठ॒वाद् वयो॒ वयः॑ पष्ठ॒वात् प॑ष्ठ॒वाद् वयो॑ ऽभि॒भू र॑भि॒भूर् वयः॑ पष्ठ॒वात् प॑ष्ठ॒वाद् वयो॑ ऽभि॒भूः । \newline
63. प॒ष्ठ॒वादिति॑ पष्ठ - वात् । \newline
64. वयो॑ ऽभि॒भू र॑भि॒भूर् वयो॒ वयो॑ ऽभि॒भू रया॑ना॒ मया॑ना मभि॒भूर् वयो॒ वयो॑ ऽभि॒भू रया॑नाम् । \newline
65. अ॒भि॒भू रया॑ना॒ मया॑ना मभि॒भू र॑भि॒भू रया॑नां ॅविष्वग्वा॒तो वि॑ष्वग्वा॒तो ऽया॑ना मभि॒भू र॑भि॒भू रया॑नां ॅविष्वग्वा॒तः । \newline
66. अ॒भि॒भूरित्य॑भि - भूः । \newline
67. अया॑नां ॅविष्वग्वा॒तो वि॑ष्वग्वा॒तो ऽया॑ना॒ मया॑नां ॅविष्वग्वा॒तो वातो॒ वातो॑ विष्वग्वा॒तो ऽया॑ना॒ मया॑नां ॅविष्वग्वा॒तो वातः॑ । \newline
68. वि॒ष्व॒ग्वा॒तो वातो॒ वातो॑ विष्वग्वा॒तो वि॑ष्वग्वा॒तो वातः॑ सुप॒र्णः सु॑प॒र्णो वातो॑ विष्वग्वा॒तो वि॑ष्वग्वा॒तो वातः॑ सुप॒र्णः । \newline
69. वि॒ष्व॒ग्वा॒त इति॑ विष्वक् - वा॒तः । \newline
70. वातः॑ सुप॒र्णः सु॑प॒र्णो वातो॒ वातः॑ सुप॒र्ण ऋषि॒र्॒. ऋषिः॑ सुप॒र्णो वातो॒ वातः॑ सुप॒र्ण ऋषिः॑ । \newline
71. सु॒प॒र्ण ऋषि॒र्॒. ऋषिः॑ सुप॒र्णः सु॑प॒र्ण ऋषिः॑ पि॒तरः॑ पि॒तर॒ ऋषिः॑ सुप॒र्णः सु॑प॒र्ण ऋषिः॑ पि॒तरः॑ । \newline
72. सु॒प॒र्ण इति॑ सु - प॒र्णः । \newline
73. ऋषिः॑ पि॒तरः॑ पि॒तर॒ ऋषि॒र्॒. ऋषिः॑ पि॒तरः॑ पिताम॒हाः पि॑ताम॒हाः पि॒तर॒ ऋषि॒र्॒. ऋषिः॑ पि॒तरः॑ पिताम॒हाः । \newline
74. पि॒तरः॑ पिताम॒हाः पि॑ताम॒हाः पि॒तरः॑ पि॒तरः॑ पिताम॒हाः परे॒ परे॑ पिताम॒हाः पि॒तरः॑ पि॒तरः॑ पिताम॒हाः परे᳚ । \newline
75. पि॒ता॒म॒हाः परे॒ परे॑ पिताम॒हाः पि॑ताम॒हाः परे ऽव॒रे ऽव॑रे॒ परे॑ पिताम॒हाः पि॑ताम॒हाः परे ऽव॑रे । \newline
76. परे ऽव॒रे ऽव॑रे॒ परे॒ परे ऽव॑रे॒ ते ते ऽव॑रे॒ परे॒ परे ऽव॑रे॒ ते । \newline
77. अव॑रे॒ ते ते ऽव॒रे ऽव॑रे॒ ते नो॑ न॒ स्ते ऽव॒रे ऽव॑रे॒ ते नः॑ । \newline
78. ते नो॑ न॒ स्ते ते नः॑ पान्तु पान्तु न॒ स्ते ते नः॑ पान्तु । \newline
79. नः॒ पा॒न्तु॒ पा॒न्तु॒ नो॒ नः॒ पा॒न्तु॒ ते ते पा᳚न्तु नो नः पान्तु॒ ते । \newline
80. पा॒न्तु॒ ते ते पा᳚न्तु पान्तु॒ ते नो॑ न॒ स्ते पा᳚न्तु पान्तु॒ ते नः॑ । \newline
81. ते नो॑ न॒ स्ते ते नो॑ ऽवन् त्व वन्तु न॒स्ते ते नो॑ ऽवन्तु । \newline
82. नो॒ ऽव॒न् त्व॒ व॒न्तु॒ नो॒ नो॒ ऽव॒न् त्व॒स्मिन् न॒स्मिन् न॑वन्तु नो नो ऽवन्त् व॒स्मिन्न् । \newline
83. अ॒व॒न् त्व॒स्मिन् न॒स्मिन् न॑वन् त्ववन् त्व॒स्मिन् ब्रह्म॒न् ब्रह्म॑न् न॒स्मिन् न॑वन् त्ववन् त्व॒स्मिन् ब्रह्मन्न्॑ । \newline
84. अ॒स्मिन् ब्रह्म॒न् ब्रह्म॑न् न॒स्मिन् न॒स्मिन् ब्रह्म॑न् न॒स्मिन् न॒स्मिन् ब्रह्म॑न् न॒स्मिन् न॒स्मिन् ब्रह्म॑न् न॒स्मिन्न् । \newline
85. ब्रह्म॑न् न॒स्मिन् न॒स्मिन् ब्रह्म॒न् ब्रह्म॑न् न॒स्मिन् क्ष॒त्रे क्ष॒त्रे᳚ ऽस्मिन् ब्रह्म॒न् ब्रह्म॑न् न॒स्मिन् क्ष॒त्रे । \newline
86. अ॒स्मिन् क्ष॒त्रे क्ष॒त्रे᳚ ऽस्मिन् न॒स्मिन् क्ष॒त्रे᳚ ऽस्या म॒स्याम् क्ष॒त्रे᳚ ऽस्मिन् न॒स्मिन् क्ष॒त्रे᳚ ऽस्याम् । \newline
87. क्ष॒त्रे᳚ ऽस्या म॒स्याम् क्ष॒त्रे क्ष॒त्रे᳚ ऽस्या मा॒शि ष्या॒शि ष्य॒स्याम् क्ष॒त्रे क्ष॒त्रे᳚ ऽस्या मा॒शिषि॑ । \newline
88. अ॒स्या मा॒शि ष्या॒शि ष्य॒स्या म॒स्या मा॒शि ष्य॒स्या म॒स्या मा॒शि ष्य॒स्या म॒स्या मा॒शि ष्य॒स्याम् । \newline
89. आ॒शि ष्य॒स्या म॒स्या मा॒शिष्या॒ शिष्य॒स्याम् पु॑रो॒धाया᳚म् पुरो॒धाया॑ म॒स्या मा॒शि ष्या॒शि ष्य॒स्याम् पु॑रो॒धाया᳚म् । \newline
90. आ॒शिषीत्या᳚ - शिषि॑ । \newline
91. अ॒स्याम् पु॑रो॒धाया᳚म् पुरो॒धाया॑ म॒स्या म॒स्याम् पु॑रो॒धाया॑ म॒स्मिन् न॒स्मिन् पु॑रो॒धाया॑ म॒स्या म॒स्याम् पु॑रो॒धाया॑ म॒स्मिन्न् । \newline
92. पु॒रो॒धाया॑ म॒स्मिन् न॒स्मिन् पु॑रो॒धाया᳚म् पुरो॒धाया॑ म॒स्मिन् कर्म॒न् कर्म॑न् न॒स्मिन् पु॑रो॒धाया᳚म् पुरो॒धाया॑ म॒स्मिन् कर्मन्न्॑ । \newline
93. पु॒रो॒धाया॒मिति॑ पुरः - धाया᳚म् । \newline
94. अ॒स्मिन् कर्म॒न् कर्म॑न् न॒स्मिन् न॒स्मिन् कर्म॑न् न॒स्या म॒स्याम् कर्म॑न् न॒स्मिन् न॒स्मिन् कर्म॑न् न॒स्याम् । \newline
95. कर्म॑न् न॒स्या म॒स्याम् कर्म॒न् कर्म॑न् न॒स्याम् दे॒वहू᳚त्याम् दे॒वहू᳚त्या म॒स्याम् कर्म॒न् कर्म॑न् न॒स्याम् दे॒वहू᳚त्याम् । \newline
96. अ॒स्याम् दे॒वहू᳚त्याम् दे॒वहू᳚त्या म॒स्या म॒स्याम् दे॒वहू᳚त्याम् । \newline
97. दे॒वहू᳚त्या॒मिति॑ दे॒व - हू॒त्या॒म् । \newline
\pagebreak
\markright{ TS 4.3.4.1  \hfill https://www.vedavms.in \hfill}

\section{ TS 4.3.4.1 }

\textbf{TS 4.3.4.1 } \newline
\textbf{Samhita Paata} \newline

ध्रु॒वक्षि॑ति -र्ध्रु॒वयो॑नि-र्ध्रु॒वाऽसि॑ ध्रु॒वां ॅयोनि॒मा सी॑द सा॒द्ध्या । उख्य॑स्य के॒तुं प्र॑थ॒मं पु॒रस्ता॑द॒श्विना᳚ऽद्ध्व॒र्यू सा॑दयतामि॒ह त्वा᳚ ॥ स्वे दक्षे॒ दक्ष॑पिते॒ह सी॑द देव॒त्रा पृ॑थि॒वी बृ॑ह॒ती ररा॑णा । स्वा॒स॒स्था त॒नुवा॒ सं ॅवि॑शस्व पि॒तेवै॑धि सू॒नव॒ आ सु॒शेवा॒ऽश्विना᳚द्ध्व॒र्यू सा॑दयतामि॒ह त्वा᳚ ॥ कु॒ला॒यिनी॒ वसु॑मती वयो॒धा र॒यिं नो॑ वर्द्ध बहु॒लꣳ सु॒वीरं᳚ । \newline

\textbf{Pada Paata} \newline

ध्रु॒वक्षि॑ती॒रिति॑ ध्रु॒व - क्षि॒तिः॒ । ध्रु॒वयो॑नि॒रिति॑ ध्रु॒व-यो॒निः॒ । ध्रु॒वा । अ॒सि॒ । ध्रु॒वाम् । योनि᳚म् । एति॑ । सी॒द॒ । सा॒द्ध्या ॥ उख्य॑स्य । के॒तुम् । प्र॒थ॒मम् । पु॒रस्ता᳚त् । अ॒श्विना᳚ । अ॒द्ध्व॒र्यू इति॑ । सा॒द॒य॒ता॒म् । इ॒ह । त्वा॒ ॥ स्वे । दक्षे᳚ । दक्ष॑पि॒तेति॒ दक्ष॑ - पि॒ता॒ । इ॒ह । सी॒द॒ । दे॒व॒त्रेति॑ देव - त्रा । पृ॒थि॒वी । बृ॒ह॒ती । ररा॑णा ॥ स्वा॒स॒स्थेति॑ सु - आ॒स॒स्था । त॒नुवा᳚ । समिति॑ । वि॒श॒स्व॒ । पि॒ता । इ॒व॒ । ए॒धि॒ । सू॒नवे᳚ । एति॑ । सु॒शेवेति॑ सु - शेवा᳚ । अ॒श्विना᳚ । अ॒द्ध्व॒र्यू इति॑ । सा॒द॒य॒ता॒म् । इ॒ह । त्वा॒ ॥ कु॒ला॒यिनी᳚ । वसु॑म॒तीति॒ वसु॑ - म॒ती॒ । व॒यो॒धा इति॑ वयः - धाः । र॒यिम् । नः॒ । व॒द्‌र्ध॒ । ब॒हु॒लम् । सु॒वीर॒मिति॑ सु - वीर᳚म् ॥  \newline


\textbf{Krama Paata} \newline

ध्रु॒वक्षि॑तिर् ध्रु॒वयो॑निः । ध्रु॒वक्षि॑ति॒रिति॑ ध्रु॒व - क्षि॒तिः॒ । ध्रु॒वयो॑निर् ध्रु॒वा । ध्रु॒वयो॑नि॒रिति॑ ध्रु॒व - यो॒निः॒ । ध्रु॒वाऽसि॑ । अ॒सि॒ ध्रु॒वाम् । ध्रु॒वां ॅयोनि᳚म् । योनि॒मा । आ सी॑द । सी॒द॒ सा॒द्ध्या । सा॒द्ध्येति॑ सा॒द्ध्या ॥ उक्ष॑स्य के॒तुम् । के॒तुम् प्र॑थ॒मम् । प्र॒थ॒मम् पु॒रस्ता᳚त् । पु॒रस्ता॑द॒श्विना᳚ । अ॒श्विना᳚ऽद्ध्व॒र्यू । अ॒द्ध्व॒र्यू सा॑दयताम् । अ॒द्ध्व॒र्यू इत्य॑द्ध्व॒र्यू । सा॒द॒य॒ता॒मि॒ह । इ॒ह त्वा᳚ । त्वेति॑ त्वा ॥ स्वे दक्षे᳚ । दक्षे॒ दक्ष॑पिता । दक्ष॑पिते॒ह । दक्ष॑पि॒तेति॒ दक्ष॑ - पि॒ता॒ । इ॒ह सी॑द । सी॒द॒ दे॒व॒त्रा । दे॒व॒त्रा पृ॑थि॒वी । दे॒व॒त्रेति॑ देव - त्रा । पृ॒थि॒वी बृ॑ह॒ती । बृ॒ह॒ती ररा॑णा । ररा॒णेति॒ ररा॑णा ॥ स्वा॒स॒स्था त॒नुवा᳚ । स्वा॒स॒स्थेति॑ सु - आ॒स॒स्था । त॒नुवा॒ सम् । सं ॅवि॑शस्व । वि॒श॒स्व॒ पि॒ता । पि॒तेव॑ । इ॒वै॒धि॒ । ए॒धि॒ सू॒नवे᳚ । सू॒नव॒ आ । आ सु॒शेवा᳚ । सु॒शेवा॒ऽश्विना᳚ । सु॒शेवेति॑ सु - शेवा᳚ । अ॒श्विना᳚ऽद्ध्व॒र्यू । अ॒द्ध्व॒र्यू सा॑दयताम् । अ॒द्ध्व॒र्यू इत्य॑द्ध्व॒र्यू । सा॒द॒य॒ता॒मि॒ह । इ॒ह त्वा᳚ । त्वेति॑ त्वा ॥ कु॒ला॒यिनी॒ वसु॑मती । वसु॑मती वयो॒धाः । वसु॑म॒तीति॒ वसु॑ - म॒ती॒ । व॒यो॒धा र॒यिम् । व॒यो॒धा इति॑ वयः - धाः । र॒यिम् नः॑ । नो॒ व॒र्द्ध॒ । व॒र्द्ध॒ ब॒हु॒लम् । ब॒हु॒लꣳ सु॒वीर᳚म् । सु॒वीर॒मिति॑ सु - वीर᳚म् । \newline

\textbf{Jatai Paata} \newline

1. ध्रु॒वक्षि॑तिर् ध्रु॒वयो॑निर् ध्रु॒वयो॑निर् ध्रु॒वक्षि॑तिर् ध्रु॒वक्षि॑तिर् ध्रु॒वयो॑निः । \newline
2. ध्रु॒वक्षि॑ति॒रिति॑ ध्रु॒व - क्षि॒तिः॒ । \newline
3. ध्रु॒वयो॑निर् ध्रु॒वा ध्रु॒वा ध्रु॒वयो॑निर् ध्रु॒वयो॑निर् ध्रु॒वा । \newline
4. ध्रु॒वयो॑नि॒रिति॑ ध्रु॒व - यो॒निः॒ । \newline
5. ध्रु॒वा ऽस्य॑सि ध्रु॒वा ध्रु॒वा ऽसि॑ । \newline
6. अ॒सि॒ ध्रु॒वाम् ध्रु॒वा म॑स्यसि ध्रु॒वाम् । \newline
7. ध्रु॒वां ॅयोनिं॒ ॅयोनि॑म् ध्रु॒वाम् ध्रु॒वां ॅयोनि᳚म् । \newline
8. योनि॒ मा योनिं॒ ॅयोनि॒ मा । \newline
9. आ सी॑द सी॒दा सी॑द । \newline
10. सी॒द॒ सा॒द्ध्या सा॒द्ध्या सी॑द सीद सा॒द्ध्या । \newline
11. सा॒द्ध्येति॑ सा॒द्ध्या । \newline
12. उख्य॑स्य के॒तुम् के॒तु मुख्य॒ स्योख्य॑स्य के॒तुम् । \newline
13. के॒तुम् प्र॑थ॒मम् प्र॑थ॒मम् के॒तुम् के॒तुम् प्र॑थ॒मम् । \newline
14. प्र॒थ॒मम् पु॒रस्ता᳚त् पु॒रस्ता᳚त् प्रथ॒मम् प्र॑थ॒मम् पु॒रस्ता᳚त् । \newline
15. पु॒रस्ता॑ द॒श्विना॒ ऽश्विना॑ पु॒रस्ता᳚त् पु॒रस्ता॑ द॒श्विना᳚ । \newline
16. अ॒श्विना᳚ ऽद्ध्व॒र्यू अ॑द्ध्व॒र्यू अ॒श्विना॒ ऽश्विना᳚ ऽद्ध्व॒र्यू । \newline
17. अ॒द्ध्व॒र्यू सा॑दयताꣳ सादयता मद्ध्व॒र्यू अ॑द्ध्व॒र्यू सा॑दयताम् । \newline
18. अ॒द्ध्व॒र्यू इत्य॑द्ध्व॒र्यू । \newline
19. सा॒द॒य॒ता॒ मि॒हे ह सा॑दयताꣳ सादयता मि॒ह । \newline
20. इ॒ह त्वा᳚ त्वे॒ हेह त्वा᳚ । \newline
21. त्वेति॑ त्वा । \newline
22. स्वे दक्षे॒ दक्षे॒ स्वे स्वे दक्षे᳚ । \newline
23. दक्षे॒ दक्ष॑पिता॒ दक्ष॑पिता॒ दक्षे॒ दक्षे॒ दक्ष॑पिता । \newline
24. दक्ष॑पिते॒ हेह दक्ष॑पिता॒ दक्ष॑पिते॒ह । \newline
25. दक्ष॑पि॒तेति॒ दक्ष॑ - पि॒ता॒ । \newline
26. इ॒ह सी॑द सीदे॒ हेह सी॑द । \newline
27. सी॒द॒ दे॒व॒त्रा दे॑व॒त्रा सी॑द सीद देव॒त्रा । \newline
28. दे॒व॒त्रा पृ॑थि॒वी पृ॑थि॒वी दे॑व॒त्रा दे॑व॒त्रा पृ॑थि॒वी । \newline
29. दे॒व॒त्रेति॑ देव - त्रा । \newline
30. पृ॒थि॒वी बृ॑ह॒ती बृ॑ह॒ती पृ॑थि॒वी पृ॑थि॒वी बृ॑ह॒ती । \newline
31. बृ॒ह॒ती ररा॑णा॒ ररा॑णा बृह॒ती बृ॑ह॒ती ररा॑णा । \newline
32. ररा॒णेति॒ ररा॑णा । \newline
33. स्वा॒स॒स्था त॒नुवा॑ त॒नुवा᳚ स्वास॒स्था स्वा॑स॒स्था त॒नुवा᳚ । \newline
34. स्वा॒स॒स्थेति॑ सु - आ॒स॒स्था । \newline
35. त॒नुवा॒ सꣳ सम् त॒नुवा॑ त॒नुवा॒ सम् । \newline
36. सं ॅवि॑शस्व विशस्व॒ सꣳ सं ॅवि॑शस्व । \newline
37. वि॒श॒स्व॒ पि॒ता पि॒ता वि॑शस्व विशस्व पि॒ता । \newline
38. पि॒तेवे॑व पि॒ता पि॒तेव॑ । \newline
39. इ॒वै॒ ध्ये॒ धी॒वे॒ वै॒धि॒ । \newline
40. ए॒धि॒ सू॒नवे॑ सू॒नव॑ एध्येधि सू॒नवे᳚ । \newline
41. सू॒नव॒ आ सू॒नवे॑ सू॒नव॒ आ । \newline
42. आ सु॒शेवा॑ सु॒शेवा ऽऽसु॒शेवा᳚ । \newline
43. सु॒शेवा॒ ऽश्विना॒ ऽश्विना॑ सु॒शेवा॑ सु॒शेवा॒ ऽश्विना᳚ । \newline
44. सु॒शेवेति॑ सु - शेवा᳚ । \newline
45. अ॒श्विना᳚ ऽद्ध्व॒र्यू अ॑द्ध्व॒र्यू अ॒श्विना॒ ऽश्विना᳚ ऽद्ध्व॒र्यू । \newline
46. अ॒द्ध्व॒र्यू सा॑दयताꣳ सादयता मद्ध्व॒र्यू अ॑द्ध्व॒र्यू सा॑दयताम् । \newline
47. अ॒द्ध्व॒र्यू इत्य॑द्ध्व॒र्यू । \newline
48. सा॒द॒य॒ता॒ मि॒हेह सा॑दयताꣳ सादयता मि॒ह । \newline
49. इ॒ह त्वा᳚ त्वे॒हेह त्वा᳚ । \newline
50. त्वेति॑ त्वा । \newline
51. कु॒ला॒यिनी॒ वसु॑मती॒ वसु॑मती कुला॒यिनी॑ कुला॒यिनी॒ वसु॑मती । \newline
52. वसु॑मती वयो॒धा व॑यो॒धा वसु॑मती॒ वसु॑मती वयो॒धाः । \newline
53. वसु॑म॒तीति॒ वसु॑ - म॒ती॒ । \newline
54. व॒यो॒धा र॒यिꣳ र॒यिं ॅव॑यो॒धा व॑यो॒धा र॒यिम् । \newline
55. व॒यो॒धा इति॑ वयः - धाः । \newline
56. र॒यिन् नो॑ नो र॒यिꣳ र॒यिन् नः॑ । \newline
57. नो॒ व॒र्द्ध॒ व॒र्द्ध॒ नो॒ नो॒ व॒र्द्ध॒ । \newline
58. व॒र्द्ध॒ ब॒हु॒लम् ब॑हु॒लं ॅव॑र्द्ध वर्द्ध बहु॒लम् । \newline
59. ब॒हु॒लꣳ सु॒वीरꣳ॑ सु॒वीर॑म् बहु॒लम् ब॑हु॒लꣳ सु॒वीर᳚म् । \newline
60. सु॒वीर॒मिति॑ सु - वीर᳚म् । \newline

\textbf{Ghana Paata } \newline

1. ध्रु॒वक्षि॑तिर् ध्रु॒वयो॑निर् ध्रु॒वयो॑निर् ध्रु॒वक्षि॑तिर् ध्रु॒वक्षि॑तिर् ध्रु॒वयो॑निर् ध्रु॒वा ध्रु॒वा ध्रु॒वयो॑निर् ध्रु॒वक्षि॑तिर् ध्रु॒वक्षि॑तिर् ध्रु॒वयो॑निर् ध्रु॒वा । \newline
2. ध्रु॒वक्षि॑ति॒रिति॑ ध्रु॒व - क्षि॒तिः॒ । \newline
3. ध्रु॒वयो॑निर् ध्रु॒वा ध्रु॒वा ध्रु॒वयो॑निर् ध्रु॒वयो॑निर् ध्रु॒वा ऽस्य॑सि ध्रु॒वा ध्रु॒वयो॑निर् ध्रु॒वयो॑निर् ध्रु॒वा ऽसि॑ । \newline
4. ध्रु॒वयो॑नि॒रिति॑ ध्रु॒व - यो॒निः॒ । \newline
5. ध्रु॒वा ऽस्य॑सि ध्रु॒वा ध्रु॒वा ऽसि॑ ध्रु॒वाम् ध्रु॒वा म॑सि ध्रु॒वा ध्रु॒वा ऽसि॑ ध्रु॒वाम् । \newline
6. अ॒सि॒ ध्रु॒वाम् ध्रु॒वा म॑स्यसि ध्रु॒वां ॅयोनिं॒ ॅयोनि॑म् ध्रु॒वा म॑स्यसि ध्रु॒वां ॅयोनि᳚म् । \newline
7. ध्रु॒वां ॅयोनिं॒ ॅयोनि॑म् ध्रु॒वाम् ध्रु॒वां ॅयोनि॒ मा योनि॑म् ध्रु॒वाम् ध्रु॒वां ॅयोनि॒ मा । \newline
8. योनि॒ मा योनिं॒ ॅयोनि॒ मा सी॑द सी॒दा योनिं॒ ॅयोनि॒ मा सी॑द । \newline
9. आ सी॑द सी॒दा सी॑द सा॒द्ध्या सा॒द्ध्या सी॒दा सी॑द सा॒द्ध्या । \newline
10. सी॒द॒ सा॒द्ध्या सा॒द्ध्या सी॑द सीद सा॒द्ध्या । \newline
11. सा॒द्ध्येति॑ सा॒द्ध्या । \newline
12. उख्य॑स्य के॒तुम् के॒तु मुख्य॒स्यो ख्य॑स्य के॒तुम् प्र॑थ॒मम् प्र॑थ॒मम् के॒तु मुख्य॒स्यो ख्य॑स्य के॒तुम् प्र॑थ॒मम् । \newline
13. के॒तुम् प्र॑थ॒मम् प्र॑थ॒मम् के॒तुम् के॒तुम् प्र॑थ॒मम् पु॒रस्ता᳚त् पु॒रस्ता᳚त् प्रथ॒मम् के॒तुम् के॒तुम् प्र॑थ॒मम् पु॒रस्ता᳚त् । \newline
14. प्र॒थ॒मम् पु॒रस्ता᳚त् पु॒रस्ता᳚त् प्रथ॒मम् प्र॑थ॒मम् पु॒रस्ता॑ द॒श्विना॒ ऽश्विना॑ पु॒रस्ता᳚त् प्रथ॒मम् प्र॑थ॒मम् पु॒रस्ता॑ द॒श्विना᳚ । \newline
15. पु॒रस्ता॑ द॒श्विना॒ ऽश्विना॑ पु॒रस्ता᳚त् पु॒रस्ता॑ द॒श्विना᳚ ऽद्ध्व॒र्यू अ॑द्ध्व॒र्यू अ॒श्विना॑ पु॒रस्ता᳚त् पु॒रस्ता॑ द॒श्विना᳚ ऽद्ध्व॒र्यू । \newline
16. अ॒श्विना᳚ ऽद्ध्व॒र्यू अ॑द्ध्व॒र्यू अ॒श्विना॒ ऽश्विना᳚ ऽद्ध्व॒र्यू सा॑दयताꣳ सादयता मद्ध्व॒र्यू अ॒श्विना॒ ऽश्विना᳚ ऽद्ध्व॒र्यू सा॑दयताम् । \newline
17. अ॒द्ध्व॒र्यू सा॑दयताꣳ सादयता मद्ध्व॒र्यू अ॑द्ध्व॒र्यू सा॑दयता मि॒हेह सा॑दयता मद्ध्व॒र्यू अ॑द्ध्व॒र्यू सा॑दयता मि॒ह । \newline
18. अ॒द्ध्व॒र्यू इत्य॑द्ध्व॒र्यू । \newline
19. सा॒द॒य॒ता॒ मि॒हे ह सा॑दयताꣳ सादयता मि॒ह त्वा᳚ त्वे॒ह सा॑दयताꣳ सादयता मि॒ह त्वा᳚ । \newline
20. इ॒ह त्वा᳚ त्वे॒हेह त्वा᳚ । \newline
21. त्वेति॑ त्वा । \newline
22. स्वे दक्षे॒ दक्षे॒ स्वे स्वे दक्षे॒ दक्ष॑पिता॒ दक्ष॑पिता॒ दक्षे॒ स्वे स्वे दक्षे॒ दक्ष॑पिता । \newline
23. दक्षे॒ दक्ष॑पिता॒ दक्ष॑पिता॒ दक्षे॒ दक्षे॒ दक्ष॑पिते॒हेह दक्ष॑पिता॒ दक्षे॒ दक्षे॒ दक्ष॑पिते॒ह । \newline
24. दक्ष॑पिते॒हेह दक्ष॑पिता॒ दक्ष॑पिते॒ह सी॑द सीदे॒ह दक्ष॑पिता॒ दक्ष॑पिते॒ह सी॑द । \newline
25. दक्ष॑पि॒तेति॒ दक्ष॑ - पि॒ता॒ । \newline
26. इ॒ह सी॑द सीदे॒ हेह सी॑द देव॒त्रा दे॑व॒त्रा सी॑दे॒ हेह सी॑द देव॒त्रा । \newline
27. सी॒द॒ दे॒व॒त्रा दे॑व॒त्रा सी॑द सीद देव॒त्रा पृ॑थि॒वी पृ॑थि॒वी दे॑व॒त्रा सी॑द सीद देव॒त्रा पृ॑थि॒वी । \newline
28. दे॒व॒त्रा पृ॑थि॒वी पृ॑थि॒वी दे॑व॒त्रा दे॑व॒त्रा पृ॑थि॒वी बृ॑ह॒ती बृ॑ह॒ती पृ॑थि॒वी दे॑व॒त्रा दे॑व॒त्रा पृ॑थि॒वी बृ॑ह॒ती । \newline
29. दे॒व॒त्रेति॑ देव - त्रा । \newline
30. पृ॒थि॒वी बृ॑ह॒ती बृ॑ह॒ती पृ॑थि॒वी पृ॑थि॒वी बृ॑ह॒ती ररा॑णा॒ ररा॑णा बृह॒ती पृ॑थि॒वी पृ॑थि॒वी बृ॑ह॒ती ररा॑णा । \newline
31. बृ॒ह॒ती ररा॑णा॒ ररा॑णा बृह॒ती बृ॑ह॒ती ररा॑णा । \newline
32. ररा॒णेति॒ ररा॑णा । \newline
33. स्वा॒स॒स्था त॒नुवा॑ त॒नुवा᳚ स्वास॒स्था स्वा॑स॒स्था त॒नुवा॒ सꣳ सम् त॒नुवा᳚ स्वास॒स्था स्वा॑स॒स्था त॒नुवा॒ सम् । \newline
34. स्वा॒स॒स्थेति॑ सु - आ॒स॒स्था । \newline
35. त॒नुवा॒ सꣳ सम् त॒नुवा॑ त॒नुवा॒ सं ॅवि॑शस्व विशस्व॒ सम् त॒नुवा॑ त॒नुवा॒ सं ॅवि॑शस्व । \newline
36. सं ॅवि॑शस्व विशस्व॒ सꣳ सं ॅवि॑शस्व पि॒ता पि॒ता वि॑शस्व॒ सꣳ सं ॅवि॑शस्व पि॒ता । \newline
37. वि॒श॒स्व॒ पि॒ता पि॒ता वि॑शस्व विशस्व पि॒तेवे॑व पि॒ता वि॑शस्व विशस्व पि॒तेव॑ । \newline
38. पि॒तेवे॑व पि॒ता पि॒ते वै᳚ध्ये धीव पि॒ता पि॒ते वै॑धि । \newline
39. इ॒वै॒ ध्ये॒ धी॒वे॒ वै॒धि॒ सू॒नवे॑ सू॒नव॑ एधीवे वैधि सू॒नवे᳚ । \newline
40. ए॒धि॒ सू॒नवे॑ सू॒नव॑ एध्येधि सू॒नव॒ आ सू॒नव॑ एध्येधि सू॒नव॒ आ । \newline
41. सू॒नव॒ आ सू॒नवे॑ सू॒नव॒ आ सु॒शेवा॑ सु॒शेवा ऽऽसू॒नवे॑ सू॒नव॒ आ सु॒शेवा᳚ । \newline
42. आ सु॒शेवा॑ सु॒शेवा ऽऽसु॒शेवा॒ ऽश्विना॒ ऽश्विना॑ सु॒शेवा ऽऽसु॒शेवा॒ ऽश्विना᳚ । \newline
43. सु॒शेवा॒ ऽश्विना॒ ऽश्विना॑ सु॒शेवा॑ सु॒शेवा॒ ऽश्विना᳚ ऽद्ध्व॒र्यू अ॑द्ध्व॒र्यू अ॒श्विना॑ सु॒शेवा॑ सु॒शेवा॒ ऽश्विना᳚ ऽद्ध्व॒र्यू । \newline
44. सु॒शेवेति॑ सु - शेवा᳚ । \newline
45. अ॒श्विना᳚ ऽद्ध्व॒र्यू अ॑द्ध्व॒र्यू अ॒श्विना॒ ऽश्विना᳚ ऽद्ध्व॒र्यू सा॑दयताꣳ सादयता मद्ध्व॒र्यू अ॒श्विना॒ ऽश्विना᳚ ऽद्ध्व॒र्यू सा॑दयताम् । \newline
46. अ॒द्ध्व॒र्यू सा॑दयताꣳ सादयता मद्ध्व॒र्यू अ॑द्ध्व॒र्यू सा॑दयता मि॒हेह सा॑दयता मद्ध्व॒र्यू अ॑द्ध्व॒र्यू सा॑दयता मि॒ह । \newline
47. अ॒द्ध्व॒र्यू इत्य॑द्ध्व॒र्यू । \newline
48. सा॒द॒य॒ता॒ मि॒हेह सा॑दयताꣳ सादयता मि॒ह त्वा᳚ त्वे॒ह सा॑दयताꣳ सादयता मि॒ह त्वा᳚ । \newline
49. इ॒ह त्वा᳚ त्वे॒हेह त्वा᳚ । \newline
50. त्वेति॑ त्वा । \newline
51. कु॒ला॒यिनी॒ वसु॑मती॒ वसु॑मती कुला॒यिनी॑ कुला॒यिनी॒ वसु॑मती वयो॒धा व॑यो॒धा वसु॑मती कुला॒यिनी॑ कुला॒यिनी॒ वसु॑मती वयो॒धाः । \newline
52. वसु॑मती वयो॒धा व॑यो॒धा वसु॑मती॒ वसु॑मती वयो॒धा र॒यिꣳ र॒यिं ॅव॑यो॒धा वसु॑मती॒ वसु॑मती वयो॒धा र॒यिम् । \newline
53. वसु॑म॒तीति॒ वसु॑ - म॒ती॒ । \newline
54. व॒यो॒धा र॒यिꣳ र॒यिं ॅव॑यो॒धा व॑यो॒धा र॒यिन्नो॑ नो र॒यिं ॅव॑यो॒धा व॑यो॒धा र॒यिन्नः॑ । \newline
55. व॒यो॒धा इति॑ वयः - धाः । \newline
56. र॒यिन्नो॑ नो र॒यिꣳ र॒यिन्नो॑ वर्द्ध वर्द्ध नो र॒यिꣳ र॒यिन्नो॑ वर्द्ध । \newline
57. नो॒ व॒र्द्ध॒ व॒र्द्ध॒ नो॒ नो॒ व॒र्द्ध॒ ब॒हु॒लम् ब॑हु॒लं ॅव॑र्द्ध नो नो वर्द्ध बहु॒लम् । \newline
58. व॒र्द्ध॒ ब॒हु॒लम् ब॑हु॒लं ॅव॑र्द्ध वर्द्ध बहु॒लꣳ सु॒वीरꣳ॑ सु॒वीर॑म् बहु॒लं ॅव॑र्द्ध वर्द्ध बहु॒लꣳ सु॒वीर᳚म् । \newline
59. ब॒हु॒लꣳ सु॒वीरꣳ॑ सु॒वीर॑म् बहु॒लम् ब॑हु॒लꣳ सु॒वीर᳚म् । \newline
60. सु॒वीर॒मिति॑ सु - वीर᳚म् । \newline
\pagebreak
\markright{ TS 4.3.4.2  \hfill https://www.vedavms.in \hfill}

\section{ TS 4.3.4.2 }

\textbf{TS 4.3.4.2 } \newline
\textbf{Samhita Paata} \newline

अपाम॑तिं दुर्म॒तिं बाध॑माना रा॒यस्पोषे॑ य॒ज्ञ्प॑तिमा॒भज॑न्ती॒ सुव॑र्द्धेहि॒ यज॑मानाय॒ पोष॑म॒श्विना᳚ऽद्ध्व॒र्यू सा॑दयतामि॒ह त्वा᳚ ॥ अ॒ग्नेः पुरी॑षमसि देव॒यानी॒ तां त्वा॒ विश्वे॑ अ॒भि गृ॑णन्तु दे॒वाः । स्तोम॑पृष्ठा घृ॒तव॑ती॒ह सी॑द प्र॒जाव॑द॒स्मे द्रवि॒णा ऽऽ*य॑जस्वा॒श्विना᳚ ऽद्ध्व॒र्यू सा॑दयतामि॒ह त्वा᳚ ॥ दि॒वो मू॒र्द्धाऽसि॑ पृथि॒व्या नाभि॑र्वि॒ष्टंभ॑नी दि॒शामधि॑पत्नी॒ भुव॑नानां । \newline

\textbf{Pada Paata} \newline

अपेति॑ । अम॑तिम् । दु॒र्म॒तिमिति॑ दुः - म॒तिम् । बाध॑माना । रा॒यः । पोषे᳚ । य॒ज्ञ्प॑ति॒मिति॑ य॒ज्ञ् - प॒ति॒म् । आ॒भज॒न्तीत्या᳚ - भज॑न्ती । सुवः॑ । धे॒हि॒ । यज॑मानाय । पोष᳚म् । अ॒श्विना᳚ । अ॒द्ध्व॒र्यू इति॑ । सा॒द॒य॒ता॒म् । इ॒ह । त्वा॒ ॥ अ॒ग्नेः । पुरी॑षम् । अ॒सि॒ । दे॒व॒यानीति॑ देव - यानी᳚ । ताम् । त्वा॒ । विश्वे᳚ । अ॒भीति॑ । गृ॒ण॒न्तु॒ । दे॒वाः ॥ स्तोम॑पृ॒ष्ठेति॒ स्तोम॑ - पृ॒ष्ठा॒ । घृ॒तव॒तीति॑ घृ॒त - व॒ती॒ । इ॒ह । सी॒द॒ । प्र॒जाव॒दिति॑ प्र॒जा - व॒त् । अ॒स्मे इति॑ । द्रवि॑णा । एति॑ । य॒ज॒स्व॒ । आ॒श्विना᳚ । अ॒द्ध्व॒र्यू इति॑ । सा॒द॒य॒ता॒म् । इ॒ह । त्वा॒ ॥ दि॒वः । मू॒द्‌र्धा । अ॒सि॒ । पृ॒थि॒व्याः । नाभिः॑ । वि॒ष्टंभ॒नीति॑ वि - स्तंभ॑नी । दि॒शाम् । अधि॑प॒त्नीत्यधि॑ - प॒त्नी॒ । भुव॑नानाम् ॥  \newline


\textbf{Krama Paata} \newline

अपाम॑तिम् । अम॑तिम् दुर्म॒तिम् । दु॒र्म॒तिम् बाध॑माना । दु॒र्म॒तिमिति॑ दुः - म॒तिम् । बाध॑माना रा॒यः । रा॒यस्पोषे᳚ । पोषे॑ य॒ज्ञ्प॑तिम् । य॒ज्ञ्प॑तिमा॒भज॑न्ती । य॒ज्ञ्प॑ति॒मिति॑ य॒ज्ञ् - प॒ति॒म् । आ॒भज॑न्ती॒ सुवः॑ । आ॒भज॒न्तीत्या᳚ - भज॑न्ती । सुव॑र् धेहि । धे॒हि॒ यज॑मानाय । यज॑मनाय॒ पोष᳚म् । पोष॑म॒श्विना᳚ । अ॒श्विना᳚ऽद्ध्व॒र्यू । अ॒द्ध्व॒र्यू सा॑दयताम् । अ॒द्ध्व॒र्यू इत्य॑द्ध्व॒र्यू । सा॒द॒य॒ता॒मि॒ह । इ॒ह त्वा᳚ । त्वेति॑ त्वा ॥ अ॒ग्नेः पुरी॑षम् । पुरी॑षमसि । अ॒सि॒ दे॒व॒यानी᳚ । दे॒व॒यानी॒ ताम् । दे॒व॒यानीति॑ देव - यानी᳚ । ताम् त्वा᳚ । त्वा॒ विश्वे᳚ । विश्वे॑ अ॒भि । अ॒भि गृ॑णन्तु । गृ॒ण॒न्तु॒ दे॒वाः । दे॒वा इति॑ दे॒वाः ॥ स्तोम॑पृष्ठा घृ॒तव॑ती । स्तोम॑पृ॒ष्ठेति॒ स्तोम॑ - पृ॒ष्ठा॒ । घृ॒तव॑ती॒ह । घृ॒तव॒तीति॑ घृ॒त - व॒ती॒ । इ॒ह सी॑द । सी॒द॒ प्र॒जाव॑त् । प्र॒जाव॑द॒स्मे । प्र॒जाव॒दिति॑ प्र॒जा - व॒त्॒ । अ॒स्मे द्रवि॑णा । अ॒स्मे इत्य॒स्मे । द्रवि॒णा । आ य॑जस्व । य॒ज॒स्वा॒श्विना᳚ । अ॒श्विना᳚ऽद्ध्व॒र्यू । अ॒द्ध्व॒र्यू सा॑दयताम् । अ॒द्ध्व॒र्यू इत्य॑द्ध्व॒र्यू । सा॒द॒य॒ता॒मि॒ह । इ॒ह त्वा᳚ । त्वेति॑ त्वा ॥ दि॒वो मू॒र्द्धा । मू॒र्द्धाऽसि॑ । अ॒सि॒ पृ॒थि॒व्याः । पृ॒थि॒व्या नाभिः॑ । नाभि॑र् वि॒ष्टम्भ॑नी । वि॒ष्टम्भ॑नी दि॒शाम् । वि॒ष्टम्भ॒नीति॑ वि - स्तम्भ॑नी । दि॒शामधि॑पत्नी । अधि॑पत्नी॒ भुव॑नानाम् । अधि॑प॒त्नीत्यधि॑ - प॒त्नी॒ । भुव॑नाना॒मिति॒ भुव॑नानाम् । \newline

\textbf{Jatai Paata} \newline

1. अपाम॑ति॒ मम॑ति॒ मपापा म॑तिम् । \newline
2. अम॑तिम् दुर्म॒तिम् दु॑र्म॒ति मम॑ति॒ मम॑तिम् दुर्म॒तिम् । \newline
3. दु॒र्म॒तिम् बाध॑माना॒ बाध॑माना दुर्म॒तिम् दु॑र्म॒तिम् बाध॑माना । \newline
4. दु॒र्म॒तिमिति॑ दुः - म॒तिम् । \newline
5. बाध॑माना रा॒यो रा॒यो बाध॑माना॒ बाध॑माना रा॒यः । \newline
6. रा॒य स्पोषे॒ पोषे॑ रा॒यो रा॒य स्पोषे᳚ । \newline
7. पोषे॑ य॒ज्ञ्प॑तिं ॅय॒ज्ञ्प॑ति॒म् पोषे॒ पोषे॑ य॒ज्ञ्प॑तिम् । \newline
8. य॒ज्ञ्प॑ति मा॒भज॑न्त्या॒ भज॑न्ती य॒ज्ञ्प॑तिं ॅय॒ज्ञ्प॑ति मा॒भज॑न्ती । \newline
9. य॒ज्ञ्प॑ति॒मिति॑ य॒ज्ञ् - प॒ति॒म् । \newline
10. आ॒भज॑न्ती॒ सुवः॒ सुव॑ रा॒भज॑न्त्या॒ भज॑न्ती॒ सुवः॑ । \newline
11. आ॒भज॒न्तीत्या᳚ - भज॑न्ती । \newline
12. सुव॑र् धेहि धेहि॒ सुवः॒ सुव॑र् धेहि । \newline
13. धे॒हि॒ यज॑मानाय॒ यज॑मानाय धेहि धेहि॒ यज॑मानाय । \newline
14. यज॑मानाय॒ पोष॒म् पोषं॒ ॅयज॑मानाय॒ यज॑मानाय॒ पोष᳚म् । \newline
15. पोष॑ म॒श्विना॒ ऽश्विना॒ पोष॒म् पोष॑ म॒श्विना᳚ । \newline
16. अ॒श्विना᳚ ऽद्ध्व॒र्यू अ॑द्ध्व॒र्यू अ॒श्विना॒ ऽश्विना᳚ ऽद्ध्व॒र्यू । \newline
17. अ॒द्ध्व॒र्यू सा॑दयताꣳ सादयता मद्ध्व॒र्यू अ॑द्ध्व॒र्यू सा॑दयताम् । \newline
18. अ॒द्ध्व॒र्यू इत्य॑द्ध्व॒र्यू । \newline
19. सा॒द॒य॒ता॒ मि॒हेह सा॑दयताꣳ सादयता मि॒ह । \newline
20. इ॒ह त्वा᳚ त्वे॒हेह त्वा᳚ । \newline
21. त्वेति॑ त्वा । \newline
22. अ॒ग्नेः पुरी॑ष॒म् पुरी॑ष म॒ग्ने र॒ग्नेः पुरी॑षम् । \newline
23. पुरी॑ष मस्यसि॒ पुरी॑ष॒म् पुरी॑ष मसि । \newline
24. अ॒सि॒ दे॒व॒यानी॑ देव॒या न्य॑स्यसि देव॒यानी᳚ । \newline
25. दे॒व॒यानी॒ ताम् ताम् दे॑व॒यानी॑ देव॒यानी॒ ताम् । \newline
26. दे॒व॒यानीति॑ देव - यानी᳚ । \newline
27. ताम् त्वा᳚ त्वा॒ ताम् ताम् त्वा᳚ । \newline
28. त्वा॒ विश्वे॒ विश्वे᳚ त्वा त्वा॒ विश्वे᳚ । \newline
29. विश्वे॑ अ॒भ्य॑भि विश्वे॒ विश्वे॑ अ॒भि । \newline
30. अ॒भि गृ॑णन्तु गृणन्त्व॒ भ्य॑भि गृ॑णन्तु । \newline
31. गृ॒ण॒न्तु॒ दे॒वा दे॒वा गृ॑णन्तु गृणन्तु दे॒वाः । \newline
32. दे॒वा इति॑ दे॒वाः । \newline
33. स्तोम॑पृष्ठा घृ॒तव॑ती घृ॒तव॑ती॒ स्तोम॑पृष्ठा॒ स्तोम॑पृष्ठा घृ॒तव॑ती । \newline
34. स्तोम॑पृ॒ष्ठेति॒ स्तोम॑ - पृ॒ष्ठा॒ । \newline
35. घृ॒तव॑ती॒ हेह घृ॒तव॑ती घृ॒तव॑ती॒ह । \newline
36. घृ॒तव॒तीति॑ घृ॒त - व॒ती॒ । \newline
37. इ॒ह सी॑द सीदे॒ हेह सी॑द । \newline
38. सी॒द॒ प्र॒जाव॑त् प्र॒जाव॑थ् सीद सीद प्र॒जाव॑त् । \newline
39. प्र॒जाव॑ द॒स्मे अ॒स्मे प्र॒जाव॑त् प्र॒जाव॑ द॒स्मे । \newline
40. प्र॒जाव॒दिति॑ प्र॒जा - व॒त् । \newline
41. अ॒स्मे द्रवि॑णा॒ द्रवि॑णा॒ ऽस्मे अ॒स्मे द्रवि॑णा । \newline
42. अ॒स्मे इत्य॒स्मे । \newline
43. द्रवि॒णा ऽऽद्रवि॑णा॒ द्रवि॒णा । \newline
44. आ य॑जस्व यज॒स्वा य॑जस्व । \newline
45. य॒ज॒ स्वा॒श्विना॒ ऽश्विना॑ यजस्व यज स्वा॒श्विना᳚ । \newline
46. अ॒श्विना᳚ ऽद्ध्व॒र्यू अ॑द्ध्व॒र्यू अ॒श्विना॒ ऽश्विना᳚ ऽद्ध्व॒र्यू । \newline
47. अ॒द्ध्व॒र्यू सा॑दयताꣳ सादयता मद्ध्व॒र्यू अ॑द्ध्व॒र्यू सा॑दयताम् । \newline
48. अ॒द्ध्व॒र्यू इत्य॑द्ध्व॒र्यू । \newline
49. सा॒द॒य॒ता॒ मि॒हेह सा॑दयताꣳ सादयता मि॒ह । \newline
50. इ॒ह त्वा᳚ त्वे॒हेह त्वा᳚ । \newline
51. त्वेति॑ त्वा । \newline
52. दि॒वो मू॒र्द्धा मू॒र्द्धा दि॒वो दि॒वो मू॒र्द्धा । \newline
53. मू॒र्द्धा ऽस्य॑सि मू॒र्द्धा मू॒र्द्धा ऽसि॑ । \newline
54. अ॒सि॒ पृ॒थि॒व्याः पृ॑थि॒व्या अ॑स्यसि पृथि॒व्याः । \newline
55. पृ॒थि॒व्या नाभि॒र् नाभिः॑ पृथि॒व्याः पृ॑थि॒व्या नाभिः॑ । \newline
56. नाभि॑र् वि॒ष्टंभ॑नी वि॒ष्टंभ॑नी॒ नाभि॒र् नाभि॑र् वि॒ष्टंभ॑नी । \newline
57. वि॒ष्टंभ॑नी दि॒शाम् दि॒शां ॅवि॒ष्टंभ॑नी वि॒ष्टंभ॑नी दि॒शाम् । \newline
58. वि॒ष्टंभ॒नीति॑ वि - स्तंभ॑नी । \newline
59. दि॒शा मधि॑प॒त्न्य धि॑पत्नी दि॒शाम् दि॒शा मधि॑पत्नी । \newline
60. अधि॑पत्नी॒ भुव॑नाना॒म् भुव॑नाना॒ मधि॑प॒त्न्य धि॑पत्नी॒ भुव॑नानाम् । \newline
61. अधि॑प॒त्नीत्यधि॑ - प॒त्नी॒ । \newline
62. भुव॑नाना॒मिति॒ भुव॑नानाम् । \newline

\textbf{Ghana Paata } \newline

1. अपाम॑ति॒ मम॑ति॒ मपापा म॑तिम् दुर्म॒तिम् दु॑र्म॒ति मम॑ति॒ मपापा म॑तिम् दुर्म॒तिम् । \newline
2. अम॑तिम् दुर्म॒तिम् दु॑र्म॒ति मम॑ति॒ मम॑तिम् दुर्म॒तिम् बाध॑माना॒ बाध॑माना दुर्म॒ति मम॑ति॒ मम॑तिम् दुर्म॒तिम् बाध॑माना । \newline
3. दु॒र्म॒तिम् बाध॑माना॒ बाध॑माना दुर्म॒तिम् दु॑र्म॒तिम् बाध॑माना रा॒यो रा॒यो बाध॑माना दुर्म॒तिम् दु॑र्म॒तिम् बाध॑माना रा॒यः । \newline
4. दु॒र्म॒तिमिति॑ दुः - म॒तिम् । \newline
5. बाध॑माना रा॒यो रा॒यो बाध॑माना॒ बाध॑माना रा॒य स्पोषे॒ पोषे॑ रा॒यो बाध॑माना॒ बाध॑माना 
रा॒य स्पोषे᳚ । \newline
6. रा॒य स्पोषे॒ पोषे॑ रा॒यो रा॒य स्पोषे॑ य॒ज्ञ्प॑तिं ॅय॒ज्ञ्प॑ति॒म् पोषे॑ रा॒यो रा॒य स्पोषे॑ य॒ज्ञ्प॑तिम् । \newline
7. पोषे॑ य॒ज्ञ्प॑तिं ॅय॒ज्ञ्प॑ति॒म् पोषे॒ पोषे॑ य॒ज्ञ्प॑ति मा॒भज॑न् त्या॒भज॑न्ती य॒ज्ञ्प॑ति॒म् पोषे॒ पोषे॑ य॒ज्ञ्प॑ति मा॒भज॑न्ती । \newline
8. य॒ज्ञ्प॑ति मा॒भज॑न् त्या॒भज॑न्ती य॒ज्ञ्प॑तिं ॅय॒ज्ञ्प॑ति मा॒भज॑न्ती॒ सुवः॒ सुव॑-रा॒भज॑न्ती य॒ज्ञ्प॑तिं ॅय॒ज्ञ्प॑ति मा॒भज॑न्ती॒ सुवः॑ । \newline
9. य॒ज्ञ्प॑ति॒मिति॑ य॒ज्ञ् - प॒ति॒म् । \newline
10. आ॒भज॑न्ती॒ सुवः॒ सुव॑ रा॒भज॑न् त्या॒भज॑न्ती॒ सुव॑र् धेहि धेहि॒ सुव॑-रा॒ भज॑न् त्या॒भज॑न्ती॒ सुव॑र् धेहि । \newline
11. आ॒भज॒न्तीत्या᳚ - भज॑न्ती । \newline
12. सुव॑र् धेहि धेहि॒ सुवः॒ सुव॑र् धेहि॒ यज॑मानाय॒ यज॑मानाय धेहि॒ सुवः॒ सुव॑र् धेहि॒ यज॑मानाय । \newline
13. धे॒हि॒ यज॑मानाय॒ यज॑मानाय धेहि धेहि॒ यज॑मानाय॒ पोष॒म् पोषं॒ ॅयज॑मानाय धेहि धेहि॒ यज॑मानाय॒ पोष᳚म् । \newline
14. यज॑मानाय॒ पोष॒म् पोषं॒ ॅयज॑मानाय॒ यज॑मानाय॒ पोष॑ म॒श्विना॒ ऽश्विना॒ पोषं॒ ॅयज॑मानाय॒ यज॑मानाय॒ पोष॑ म॒श्विना᳚ । \newline
15. पोष॑ म॒श्विना॒ ऽश्विना॒ पोष॒म् पोष॑ म॒श्विना᳚ ऽद्ध्व॒र्यू अ॑द्ध्व॒र्यू अ॒श्विना॒ पोष॒म् पोष॑ म॒श्विना᳚ ऽद्ध्व॒र्यू । \newline
16. अ॒श्विना᳚ ऽद्ध्व॒र्यू अ॑द्ध्व॒र्यू अ॒श्विना॒ ऽश्विना᳚ ऽद्ध्व॒र्यू सा॑दयताꣳ सादयता मद्ध्व॒र्यू अ॒श्विना॒ ऽश्विना᳚ ऽद्ध्व॒र्यू सा॑दयताम् । \newline
17. अ॒द्ध्व॒र्यू सा॑दयताꣳ सादयता मद्ध्व॒र्यू अ॑द्ध्व॒र्यू सा॑दयता मि॒हेह सा॑दयता मद्ध्व॒र्यू अ॑द्ध्व॒र्यू सा॑दयता मि॒ह । \newline
18. अ॒द्ध्व॒र्यू इत्य॑द्ध्व॒र्यू । \newline
19. सा॒द॒य॒ता॒ मि॒हेह सा॑दयताꣳ सादयता मि॒ह त्वा᳚ त्वे॒ह सा॑दयताꣳ सादयता मि॒ह त्वा᳚ । \newline
20. इ॒ह त्वा᳚ त्वे॒हेह त्वा᳚ । \newline
21. त्वेति॑ त्वा । \newline
22. अ॒ग्नेः पुरी॑ष॒म् पुरी॑ष म॒ग्ने र॒ग्नेः पुरी॑ष मस्यसि॒ पुरी॑ष म॒ग्ने र॒ग्नेः पुरी॑ष मसि । \newline
23. पुरी॑ष मस्यसि॒ पुरी॑ष॒म् पुरी॑ष मसि देव॒यानी॑ देव॒या न्य॑सि॒ पुरी॑ष॒म् पुरी॑ष मसि देव॒यानी᳚ । \newline
24. अ॒सि॒ दे॒व॒यानी॑ देव॒या न्य॑स्यसि देव॒यानी॒ ताम् ताम् दे॑व॒या न्य॑स्यसि देव॒यानी॒ ताम् । \newline
25. दे॒व॒यानी॒ ताम् ताम् दे॑व॒यानी॑ देव॒यानी॒ ताम् त्वा᳚ त्वा॒ ताम् दे॑व॒यानी॑ देव॒यानी॒ ताम् त्वा᳚ । \newline
26. दे॒व॒यानीति॑ देव - यानी᳚ । \newline
27. ताम् त्वा᳚ त्वा॒ ताम् ताम् त्वा॒ विश्वे॒ विश्वे᳚ त्वा॒ ताम् ताम् त्वा॒ विश्वे᳚ । \newline
28. त्वा॒ विश्वे॒ विश्वे᳚ त्वा त्वा॒ विश्वे॑ अ॒भ्य॑भि विश्वे᳚ त्वा त्वा॒ विश्वे॑ अ॒भि । \newline
29. विश्वे॑ अ॒भ्य॑भि विश्वे॒ विश्वे॑ अ॒भि गृ॑णन्तु गृणन्त्व॒भि विश्वे॒ विश्वे॑ अ॒भि गृ॑णन्तु । \newline
30. अ॒भि गृ॑णन्तु गृणन् त्व॒भ्य॑भि गृ॑णन्तु दे॒वा दे॒वा गृ॑णन् त्व॒भ्य॑भि गृ॑णन्तु दे॒वाः । \newline
31. गृ॒ण॒न्तु॒ दे॒वा दे॒वा गृ॑णन्तु गृणन्तु दे॒वाः । \newline
32. दे॒वा इति॑ दे॒वाः । \newline
33. स्तोम॑पृष्ठा घृ॒तव॑ती घृ॒तव॑ती॒ स्तोम॑पृष्ठा॒ स्तोम॑पृष्ठा घृ॒तव॑ती॒हेह घृ॒तव॑ती॒ स्तोम॑पृष्ठा॒ स्तोम॑पृष्ठा घृ॒तव॑ती॒ह । \newline
34. स्तोम॑पृ॒ष्ठेति॒ स्तोम॑ - पृ॒ष्ठा॒ । \newline
35. घृ॒तव॑ती॒हेह घृ॒तव॑ती घृ॒तव॑ ती॒ह सी॑द सीदे॒ह घृ॒तव॑ती घृ॒तव॑ ती॒ह सी॑द । \newline
36. घृ॒तव॒तीति॑ घृ॒त - व॒ती॒ । \newline
37. इ॒ह सी॑द सीदे॒ हेह सी॑द प्र॒जाव॑त् प्र॒जाव॑थ् सी॑दे॒ हेह सी॑द प्र॒जाव॑त् । \newline
38. सी॒द॒ प्र॒जाव॑त् प्र॒जाव॑थ् सीद सीद प्र॒जाव॑ द॒स्मे अ॒स्मे प्र॒जाव॑थ् सीद सीद प्र॒जाव॑ द॒स्मे । \newline
39. प्र॒जाव॑ द॒स्मे अ॒स्मे प्र॒जाव॑त् प्र॒जाव॑ द॒स्मे द्रवि॑णा॒ द्रवि॑णा॒ ऽस्मे प्र॒जाव॑त् प्र॒जाव॑ द॒स्मे द्रवि॑णा । \newline
40. प्र॒जाव॒दिति॑ प्र॒जा - व॒त् । \newline
41. अ॒स्मे द्रवि॑णा॒ द्रवि॑णा॒ ऽस्मे अ॒स्मे द्रवि॒णा ऽऽद्रवि॑णा॒ ऽस्मे अ॒स्मे द्रवि॒णा । \newline
42. अ॒स्मे इत्य॒स्मे । \newline
43. द्रवि॒णा ऽऽद्रवि॑णा॒ द्रवि॒णा ऽऽय॑जस्व यज॒स्वा द्रवि॑णा॒ द्रवि॒णा ऽऽय॑जस्व । \newline
44. आ य॑जस्व यज॒स्वा य॑जस्वा॒ श्विना॒ ऽश्विना॑ यज॒स्वा य॑जस्वा॒ श्विना᳚ । \newline
45. य॒ज॒स्वा॒ श्विना॒ ऽश्विना॑ यजस्व यजस्वा॒ श्विना᳚ ऽद्ध्व॒र्यू अ॑द्ध्व॒र्यू अ॒श्विना॑ यजस्व यजस्वा॒ श्विना᳚ ऽद्ध्व॒र्यू । \newline
46. अ॒श्विना᳚ ऽद्ध्व॒र्यू अ॑द्ध्व॒र्यू अ॒श्विना॒ ऽश्विना᳚ ऽद्ध्व॒र्यू सा॑दयताꣳ सादयता मद्ध्व॒र्यू अ॒श्विना॒ ऽश्विना᳚ ऽद्ध्व॒र्यू सा॑दयताम् । \newline
47. अ॒द्ध्व॒र्यू सा॑दयताꣳ सादयता मद्ध्व॒र्यू अ॑द्ध्व॒र्यू सा॑दयता मि॒हेह सा॑दयता मद्ध्व॒र्यू अ॑द्ध्व॒र्यू सा॑दयता मि॒ह । \newline
48. अ॒द्ध्व॒र्यू इत्य॑द्ध्व॒र्यू । \newline
49. सा॒द॒य॒ता॒ मि॒हेह सा॑दयताꣳ सादयता मि॒ह त्वा᳚ त्वे॒ह सा॑दयताꣳ सादयता मि॒ह त्वा᳚ । \newline
50. इ॒ह त्वा᳚ त्वे॒हेह त्वा᳚ । \newline
51. त्वेति॑ त्वा । \newline
52. दि॒वो मू॒र्द्धा मू॒र्द्धा दि॒वो दि॒वो मू॒र्द्धा ऽस्य॑सि मू॒र्द्धा दि॒वो दि॒वो मू॒र्द्धा ऽसि॑ । \newline
53. मू॒र्द्धा ऽस्य॑सि मू॒र्द्धा मू॒र्द्धा ऽसि॑ पृथि॒व्याः पृ॑थि॒व्या अ॑सि मू॒र्द्धा मू॒र्द्धा ऽसि॑ पृथि॒व्याः । \newline
54. अ॒सि॒ पृ॒थि॒व्याः पृ॑थि॒व्या अ॑स्यसि पृथि॒व्या नाभि॒र् नाभिः॑ पृथि॒व्या अ॑स्यसि पृथि॒व्या नाभिः॑ । \newline
55. पृ॒थि॒व्या नाभि॒र् नाभिः॑ पृथि॒व्याः पृ॑थि॒व्या नाभि॑र् वि॒ष्टंभ॑नी वि॒ष्टंभ॑नी॒ नाभिः॑ पृथि॒व्याः पृ॑थि॒व्या नाभि॑र् वि॒ष्टंभ॑नी । \newline
56. नाभि॑र् वि॒ष्टंभ॑नी वि॒ष्टंभ॑नी॒ नाभि॒र् नाभि॑र् वि॒ष्टंभ॑नी दि॒शाम् दि॒शां ॅवि॒ष्टंभ॑नी॒ नाभि॒र् नाभि॑र् वि॒ष्टंभ॑नी दि॒शाम् । \newline
57. वि॒ष्टंभ॑नी दि॒शाम् दि॒शां ॅवि॒ष्टंभ॑नी वि॒ष्टंभ॑नी दि॒शा मधि॑प॒ त्न्यधि॑पत्नी दि॒शां ॅवि॒ष्टंभ॑नी वि॒ष्टंभ॑नी दि॒शा मधि॑पत्नी । \newline
58. वि॒ष्टंभ॒नीति॑ वि - स्तंभ॑नी । \newline
59. दि॒शा मधि॑प॒ त्न्यधि॑पत्नी दि॒शाम् दि॒शा मधि॑पत्नी॒ भुव॑नाना॒म् भुव॑नाना॒ मधि॑पत्नी दि॒शाम् दि॒शा मधि॑पत्नी॒ भुव॑नानाम् । \newline
60. अधि॑पत्नी॒ भुव॑नाना॒म् भुव॑नाना॒ मधि॑प॒ त्न्यधि॑पत्नी॒ भुव॑नानाम् । \newline
61. अधि॑प॒त्नीत्यधि॑ - प॒त्नी॒ । \newline
62. भुव॑नाना॒मिति॒ भुव॑नानाम् । \newline
\pagebreak
\markright{ TS 4.3.4.3  \hfill https://www.vedavms.in \hfill}

\section{ TS 4.3.4.3 }

\textbf{TS 4.3.4.3 } \newline
\textbf{Samhita Paata} \newline

ऊ॒र्मिर्द्र॒फ्सो अ॒पाम॑सि वि॒श्वक॑र्मा त॒ ऋषि॑र॒श्विना᳚ऽद्ध्व॒र्यू सा॑दयतामि॒ह त्वा᳚ ॥ स॒जूर्.ऋ॒तुभिः॑ स॒जूर्वि॒धाभिः॑ स॒जूर्वसु॑भिः स॒जू रु॒द्रैः स॒जूरा॑दि॒त्यैः स॒जूर्विश्वै᳚र्दे॒वैः स॒जूर्दे॒वैः स॒जूर्दे॒वैर्व॑यो-ना॒धैर॒ग्नये᳚ त्वा वैश्वान॒राया॒श्विना᳚ऽद्ध्व॒र्यू सा॑दयतामि॒ह त्वा᳚ ॥ प्रा॒णं मे॑ पाह्यपा॒नं मे॑ पाहि व्या॒नं मे॑ पाहि॒ चक्षु॑र्म उ॒र्व्या ( ) वि भा॑हि॒ श्रोत्रं॑ मे श्लोकया॒प-स्पि॒न्वौष॑धीर्जिन्व द्वि॒पात् पा॑हि॒ चतु॑ष्पादव दि॒वो वृष्टि॒मेर॑य ॥ \newline

\textbf{Pada Paata} \newline

ऊ॒र्मिः । द्र॒फ्सः । अ॒पाम् । अ॒सि॒ । वि॒श्वक॒र्मेति॑ वि॒श्व - क॒र्मा॒ । ते॒ । ऋषिः॑ । अ॒श्विना᳚ । अ॒द्ध्व॒र्यू इति॑ । सा॒द॒य॒ता॒म् । इ॒ह । त्वा॒ ॥ स॒जूरिति॑ स - जूः । ऋ॒तुभि॒रित्यृ॒तु - भिः॒ । स॒जूरिति॑ स - जूः । वि॒धाभि॒रिति॑ वि-धाभिः॑ । स॒जूरिति॑ स-जूः । वसु॑भि॒रिति॒ वसु॑-भिः॒ । स॒जूरिति॑ स - जूः । रु॒द्रैः । स॒जूरिति॑ स - जूः । आ॒दि॒त्यैः । स॒जूरिति॑ स - जूः । विश्वैः᳚ । दे॒वैः । स॒जूरिति॑ स - जूः । दे॒वैः । स॒जूरिति॑ स - जूः । दे॒वैः । व॒यो॒ना॒धैरिति॑ वयः - ना॒धैः । अ॒ग्नये᳚ । त्वा॒ । वै॒श्वा॒न॒राय॑ । अ॒श्विना᳚ । अ॒द्ध्व॒र्यू इति॑ । सा॒द॒य॒ता॒म् । इ॒ह । त्वा॒ ॥ प्रा॒णमिति॑ प्र - अ॒नम् । मे॒ । पा॒हि॒ । अ॒पा॒नमित्य॑प - अ॒नम् । मे॒ । पा॒हि॒ । व्या॒नमिति॑ वि-अ॒नम् । मे॒ । पा॒हि॒ । चक्षुः॑ । मे॒ । उ॒र्व्या ( ) । वीति॑ । भा॒हि॒ । श्रोत्र᳚म् । मे॒ । श्लो॒क॒य॒ । अ॒पः । पि॒न्व॒ । ओष॑धीः । जि॒न्व॒ । द्वि॒पादिति॑ द्वि - पात् । पा॒हि॒ । चतु॑ष्पा॒दिति॒ चतुः॑ - पा॒त् । अ॒व॒ । दि॒वः । वृष्टि᳚म् । एति॑ । ई॒र॒य॒ ॥  \newline


\textbf{Krama Paata} \newline

ऊ॒र्मिर् द्र॒फ्सः । द्र॒फ्सो अ॒पाम् । अ॒पाम॑सि । अ॒सि॒ वि॒श्वक॑र्मा । वि॒श्वक॑र्मा ते । वि॒श्वक॒र्मेति॑ वि॒श्व - क॒र्मा॒ । त॒ ऋषिः॑ । ऋषि॑र॒श्विना᳚ । अ॒श्विना᳚ऽद्ध्व॒र्यू । अ॒द्ध्व॒र्यू सा॑दयताम् । अ॒द्ध्व॒र्यू इत्य॑द्ध्व॒र्यू । सा॒द॒य॒ता॒मि॒ह । इ॒ह त्वा᳚ । त्वेति॑ त्वा ॥ स॒जूर्. ऋ॒तुभिः॑ । स॒जूरिति॑ स - जूः । ऋ॒तुभिः॑ स॒जूः । ऋ॒तुभि॒रित्यृ॒तु - भिः॒ । स॒जूर् वि॒धाभिः॑ । स॒जूरिति॑ स - जूः । वि॒धाभिः॑ स॒जूः । वि॒धाभि॒रिति॑ वि - धाभिः॑ । स॒जूर् वसु॑भिः । स॒जूरिति॑ स - जूः । वसु॑भिः स॒जूः । वसु॑भि॒रिति॒ वसु॑ - भिः॒ । स॒जू रु॒द्रैः । स॒जूरिति॑ स - जूः । रु॒द्रैः स॒जूः । स॒जूरा॑दि॒त्यैः । स॒जूरिति॑ स - जूः । आ॒दि॒त्यैः स॒जूः । स॒जूर् विश्वैः᳚ । स॒जूरिति॑ स - जूः । विश्वै᳚र् दे॒वैः । दे॒वैः स॒जूः । स॒जूर् दे॒वैः । स॒जूरिति॑ स - जूः । दे॒वैः स॒जूः । स॒जूर् दे॒वैः । स॒जूरिति॑ स - जूः । दे॒वैर् व॑योना॒धैः । व॒यो॒ना॒धैर॒ग्नये᳚ । व॒यो॒ना॒धैरिति॑ वयः - ना॒धैः । अ॒ग्नये᳚ त्वा । 
त्वा॒ वै॒श्वा॒न॒राय॑ । वै॒श्वा॒न॒राया॒श्विना᳚ । अ॒श्विना᳚ऽद्ध्व॒र्यू । अ॒द्ध्व॒र्यू सा॑दयताम् । अ॒द्ध्व॒र्यू इत्य॑द्ध्व॒र्यू । 
सा॒द॒य॒ता॒मि॒ह । इ॒ह त्वा᳚ । त्वेति॑ त्वा ॥ प्रा॒णम् मे᳚ । प्रा॒णमिति॑ प्र - अ॒नम् । मे॒ पा॒हि॒ । पा॒ह्य॒पा॒नम् । अ॒पा॒नम् मे᳚ । अ॒पा॒नमित्य॑प - अ॒नम् । मे॒ पा॒हि॒ । पा॒हि॒ व्या॒नम् । व्या॒नम् मे᳚ । व्या॒नमिति॑ वि - अ॒नम् । मे॒ पा॒हि॒ । पा॒हि॒ चक्षुः॑ । चक्षु॑र् मे । म॒ उ॒र्व्या ( ) । उ॒र्व्या वि । वि भा॑हि । भा॒हि॒ श्रोत्र᳚म् । श्रोत्र॑म् मे । मे॒ श्लो॒क॒य॒ । श्लो॒क॒या॒पः । अ॒पस्पि॑न्व । पि॒न्वौष॑धीः । ओष॑धीर् जिन्व । जि॒न्व॒ द्वि॒पात् । द्वि॒पात् पा॑हि । द्वि॒पादिति॑ द्वि - पात् । पा॒हि॒ चतु॑ष्पात् । चतु॑ष्पादव । चतु॑ष्पा॒दिति॒ चतुः॑ - पा॒त्॒ । अ॒व॒ दि॒वः । दि॒वो वृष्टि᳚म् । वृष्टि॒मा । एर॑य । ई॒र॒येती॑रय । \newline

\textbf{Jatai Paata} \newline

1. ऊ॒र्मिर् द्र॒फ्सो द्र॒फ्स ऊ॒र्मि रू॒र्मिर् द्र॒फ्सः । \newline
2. द्र॒फ्सो अ॒पा म॒पाम् द्र॒फ्सो द्र॒फ्सो अ॒पाम् । \newline
3. अ॒पा म॑स्य स्य॒पा म॒पा म॑सि । \newline
4. अ॒सि॒ वि॒श्वक॑र्मा वि॒श्वक॑र्मा ऽस्यसि वि॒श्वक॑र्मा । \newline
5. वि॒श्वक॑र्मा ते ते वि॒श्वक॑र्मा वि॒श्वक॑र्मा ते । \newline
6. वि॒श्वक॒र्मेति॑ वि॒श्व - क॒र्मा॒ । \newline
7. त॒ ऋषि॒र्॒. ऋषि॑ स्ते त॒ ऋषिः॑ । \newline
8. ऋषि॑ र॒श्विना॒ ऽश्विन र्.षि॒र्॒. ऋषि॑ र॒श्विना᳚ । \newline
9. अ॒श्विना᳚ ऽद्ध्व॒र्यू अ॑द्ध्व॒र्यू अ॒श्विना॒ ऽश्विना᳚ ऽद्ध्व॒र्यू । \newline
10. अ॒द्ध्व॒र्यू सा॑दयताꣳ सादयता मद्ध्व॒र्यू अ॑द्ध्व॒र्यू सा॑दयताम् । \newline
11. अ॒द्ध्व॒र्यू इत्य॑द्ध्व॒र्यू । \newline
12. सा॒द॒य॒ता॒ मि॒हेह सा॑दयताꣳ सादयता मि॒ह । \newline
13. इ॒ह त्वा᳚ त्वे॒हेह त्वा᳚ । \newline
14. त्वेति॑ त्वा । \newline
15. स॒जूर्. ऋ॒तुभि॑र्. ऋ॒तुभिः॑ स॒जूः स॒जूर्. ऋ॒तुभिः॑ । \newline
16. स॒जूरिति॑ स - जूः । \newline
17. ऋ॒तुभिः॑ स॒जूः स॒जूर्. ऋ॒तुभि॑र्. ऋ॒तुभिः॑ स॒जूः । \newline
18. ऋ॒तुभि॒रित्यृ॒तु - भिः॒ । \newline
19. स॒जूर् वि॒धाभि॑र् वि॒धाभिः॑ स॒जूः स॒जूर् वि॒धाभिः॑ । \newline
20. स॒जूरिति॑ स - जूः । \newline
21. वि॒धाभिः॑ स॒जूः स॒जूर् वि॒धाभि॑र् वि॒धाभिः॑ स॒जूः । \newline
22. वि॒धाभि॒रिति॑ वि - धाभिः॑ । \newline
23. स॒जूर् वसु॑भि॒र् वसु॑भिः स॒जूः स॒जूर् वसु॑भिः । \newline
24. स॒जूरिति॑ स - जूः । \newline
25. वसु॑भिः स॒जूः स॒जूर् वसु॑भि॒र् वसु॑भिः स॒जूः । \newline
26. वसु॑भि॒रिति॒ वसु॑ - भिः॒ । \newline
27. स॒जू रु॒द्रै रु॒द्रैः स॒जूः स॒जू रु॒द्रैः । \newline
28. स॒जूरिति॑ स - जूः । \newline
29. रु॒द्रैः स॒जूः स॒जू रु॒द्रै रु॒द्रैः स॒जूः । \newline
30. स॒जू रा॑दि॒त्यै रा॑दि॒त्यैः स॒जूः स॒जू रा॑दि॒त्यैः । \newline
31. स॒जूरिति॑ स - जूः । \newline
32. आ॒दि॒त्यैः स॒जूः स॒जू रा॑दि॒त्यै रा॑दि॒त्यैः स॒जूः । \newline
33. स॒जूर् विश्वै॒र् विश्वैः᳚ स॒जूः स॒जूर् विश्वैः᳚ । \newline
34. स॒जूरिति॑ स - जूः । \newline
35. विश्वै᳚र् दे॒वैर् दे॒वैर् विश्वै॒र् विश्वै᳚र् दे॒वैः । \newline
36. दे॒वैः स॒जूः स॒जूर् दे॒वैर् दे॒वैः स॒जूः । \newline
37. स॒जूर् दे॒वैर् दे॒वैः स॒जूः स॒जूर् दे॒वैः । \newline
38. स॒जूरिति॑ स - जूः । \newline
39. दे॒वैः स॒जूः स॒जूर् दे॒वैर् दे॒वैः स॒जूः । \newline
40. स॒जूर् दे॒वैर् दे॒वैः स॒जूः स॒जूर् दे॒वैः । \newline
41. स॒जूरिति॑ स - जूः । \newline
42. दे॒वैर् व॑योना॒धैर् व॑योना॒धैर् दे॒वैर् दे॒वैर् व॑योना॒धैः । \newline
43. व॒यो॒ना॒धै र॒ग्नये॑ अ॒ग्नये॑ वयोना॒धैर् व॑योना॒धै र॒ग्नये᳚ । \newline
44. व॒यो॒ना॒धैरिति॑ वयः - ना॒धैः । \newline
45. अ॒ग्नये᳚ त्वा त्वा॒ ऽग्नये॑ अ॒ग्नये᳚ त्वा । \newline
46. त्वा॒ वै॒श्वा॒न॒राय॑ वैश्वान॒राय॑ त्वा त्वा वैश्वान॒राय॑ । \newline
47. वै॒श्वा॒न॒राया॒ श्विना॒ ऽश्विना॑ वैश्वान॒राय॑ वैश्वान॒राया॒ श्विना᳚ । \newline
48. अ॒श्विना᳚ ऽद्ध्व॒र्यू अ॑द्ध्व॒र्यू अ॒श्विना॒ ऽश्विना᳚ ऽद्ध्व॒र्यू । \newline
49. अ॒द्ध्व॒र्यू सा॑दयताꣳ सादयता मद्ध्व॒र्यू अ॑द्ध्व॒र्यू सा॑दयताम् । \newline
50. अ॒द्ध्व॒र्यू इत्य॑द्ध्व॒र्यू । \newline
51. सा॒द॒य॒ता॒ मि॒हेह सा॑दयताꣳ सादयता मि॒ह । \newline
52. इ॒ह त्वा᳚ त्वे॒हेह त्वा᳚ । \newline
53. त्वेति॑ त्वा । \newline
54. प्रा॒णम् मे॑ मे प्रा॒णम् प्रा॒णम् मे᳚ । \newline
55. प्रा॒णमिति॑ प्र - अ॒नम् । \newline
56. मे॒ पा॒हि॒ पा॒हि॒ मे॒ मे॒ पा॒हि॒ । \newline
57. पा॒ह्य॒पा॒न म॑पा॒नम् पा॑हि पाह्यपा॒नम् । \newline
58. अ॒पा॒नम् मे॑ मे अपा॒न म॑पा॒नम् मे᳚ । \newline
59. अ॒पा॒नमित्य॑प - अ॒नम् । \newline
60. मे॒ पा॒हि॒ पा॒हि॒ मे॒ मे॒ पा॒हि॒ । \newline
61. पा॒हि॒ व्या॒नं ॅव्या॒नम् पा॑हि पाहि व्या॒नम् । \newline
62. व्या॒नम् मे॑ मे व्या॒नं ॅव्या॒नम् मे᳚ । \newline
63. व्या॒नमिति॑ वि - अ॒नम् । \newline
64. मे॒ पा॒हि॒ पा॒हि॒ मे॒ मे॒ पा॒हि॒ । \newline
65. पा॒हि॒ चक्षु॒ श्चक्षुः॑ पाहि पाहि॒ चक्षुः॑ । \newline
66. चक्षु॑र् मे मे॒ चक्षु॒ श्चक्षु॑र् मे । \newline
67. म॒ उ॒र्व्यो र्व्या मे॑ म उ॒र्व्या । \newline
68. उ॒र्व्या वि व्यु॑र् व्योर् व्या वि । \newline
69. वि भा॑हि भाहि॒ वि वि भा॑हि । \newline
70. भा॒हि॒ श्रोत्रꣳ॒॒ श्रोत्र॑म् भाहि भाहि॒ श्रोत्र᳚म् । \newline
71. श्रोत्र॑म् मे मे॒ श्रोत्रꣳ॒॒ श्रोत्र॑म् मे । \newline
72. मे॒ श्लो॒क॒य॒ श्लो॒क॒य॒ मे॒ मे॒ श्लो॒क॒य॒ । \newline
73. श्लो॒क॒या॒पो अ॒पः श्लो॑कय श्लोकया॒पः । \newline
74. अ॒प स्पि॑न्व पिन्वा॒पो अ॒प स्पि॑न्व । \newline
75. पि॒न्वौष॑धी॒ रोष॑धीष् पिन्व पि॒न्वौष॑धीः । \newline
76. ओष॑धीर् जिन्व जि॒न्वौष॑धी॒ रोष॑धीर् जिन्व । \newline
77. जि॒न्व॒ द्वि॒पाद् द्वि॒पाज् जि॑न्व जिन्व द्वि॒पात् । \newline
78. द्वि॒पात् पा॑हि पाहि द्वि॒पाद् द्वि॒पात् पा॑हि । \newline
79. द्वि॒पादिति॑ द्वि - पात् । \newline
80. पा॒हि॒ चतु॑ष्पा॒च् चतु॑ष्पात् पाहि पाहि॒ चतु॑ष्पात् । \newline
81. चतु॑ष्पा दवाव॒ चतु॑ष्पा॒च् चतु॑ष्पा दव । \newline
82. चतु॑ष्पा॒दिति॒ चतुः॑ - पा॒त् । \newline
83. अ॒व॒ दि॒वो दि॒वो॑ ऽवाव दि॒वः । \newline
84. दि॒वो वृष्टिं॒ ॅवृष्टि॑म् दि॒वो दि॒वो वृष्टि᳚म् । \newline
85. वृष्टि॒ मा वृष्टिं॒ ॅवृष्टि॒ मा । \newline
86. एर॑ येर॒ येर॑य । \newline
87. ई॒र॒येती॑रय । \newline

\textbf{Ghana Paata } \newline

1. ऊ॒र्मिर् द्र॒फ्सो द्र॒फ्स ऊ॒र्मि रू॒र्मिर् द्र॒फ्सो अ॒पा म॒पाम् द्र॒फ्स ऊ॒र्मि रू॒र्मिर् द्र॒फ्सो अ॒पाम् । \newline
2. द्र॒फ्सो अ॒पा म॒पाम् द्र॒फ्सो द्र॒फ्सो अ॒पा म॑स्य स्य॒पाम् द्र॒फ्सो द्र॒फ्सो अ॒पा म॑सि । \newline
3. अ॒पा म॑स्य स्य॒पा म॒पा म॑सि वि॒श्वक॑र्मा वि॒श्वक॑र्मा ऽस्य॒पा म॒पा म॑सि वि॒श्वक॑र्मा । \newline
4. अ॒सि॒ वि॒श्वक॑र्मा वि॒श्वक॑र्मा ऽस्यसि वि॒श्वक॑र्मा ते ते वि॒श्वक॑र्मा ऽस्यसि वि॒श्वक॑र्मा ते । \newline
5. वि॒श्वक॑र्मा ते ते वि॒श्वक॑र्मा वि॒श्वक॑र्मा त॒ ऋषि॒र्॒. ऋषि॑ स्ते वि॒श्वक॑र्मा वि॒श्वक॑र्मा त॒ ऋषिः॑ । \newline
6. वि॒श्वक॒र्मेति॑ वि॒श्व - क॒र्मा॒ । \newline
7. त॒ ऋषि॒र्॒. ऋषि॑ स्ते त॒ ऋषि॑ र॒श्विना॒ ऽश्विन र्.षि॑स्ते त॒ ऋषि॑ र॒श्विना᳚ । \newline
8. ऋषि॑ र॒श्विना॒ ऽश्विन र्.षि॒र्॒. ऋषि॑ र॒श्विना᳚ ऽद्ध्व॒र्यू अ॑द्ध्व॒र्यू अ॒श्विन र्.षि॒र्॒. ऋषि॑ र॒श्विना᳚ ऽद्ध्व॒र्यू । \newline
9. अ॒श्विना᳚ ऽद्ध्व॒र्यू अ॑द्ध्व॒र्यू अ॒श्विना॒ ऽश्विना᳚ ऽद्ध्व॒र्यू सा॑दयताꣳ सादयता मद्ध्व॒र्यू अ॒श्विना॒ ऽश्विना᳚ ऽद्ध्व॒र्यू सा॑दयताम् । \newline
10. अ॒द्ध्व॒र्यू सा॑दयताꣳ सादयता मद्ध्व॒र्यू अ॑द्ध्व॒र्यू सा॑दयता मि॒हेह सा॑दयता मद्ध्व॒र्यू अ॑द्ध्व॒र्यू सा॑दयता मि॒ह । \newline
11. अ॒द्ध्व॒र्यू इत्य॑द्ध्व॒र्यू । \newline
12. सा॒द॒य॒ता॒ मि॒हेह सा॑दयताꣳ सादयता मि॒ह त्वा᳚ त्वे॒ह सा॑दयताꣳ सादयता मि॒ह त्वा᳚ । \newline
13. इ॒ह त्वा᳚ त्वे॒हेह त्वा᳚ । \newline
14. त्वेति॑ त्वा । \newline
15. स॒जूर्. ऋ॒तुभि॑र्. ऋ॒तुभिः॑ स॒जूः स॒जूर्. ऋ॒तुभिः॑ स॒जूः स॒जूर्. ऋ॒तुभिः॑ स॒जूः स॒जूर्. ऋ॒तुभिः॑ स॒जूः । \newline
16. स॒जूरिति॑ स - जूः । \newline
17. ऋ॒तुभिः॑ स॒जूः स॒जूर्. ऋ॒तुभि॑र्. ऋ॒तुभिः॑ स॒जूर् वि॒धाभि॑र् वि॒धाभिः॑ स॒जूर्. ऋ॒तुभि॑र्. ऋ॒तुभिः॑ स॒जूर् वि॒धाभिः॑ । \newline
18. ऋ॒तुभि॒रित्यृ॒तु - भिः॒ । \newline
19. स॒जूर् वि॒धाभि॑र् वि॒धाभिः॑ स॒जूः स॒जूर् वि॒धाभिः॑ स॒जूः स॒जूर् वि॒धाभिः॑ स॒जूः स॒जूर् वि॒धाभिः॑ स॒जूः । \newline
20. स॒जूरिति॑ स - जूः । \newline
21. वि॒धाभिः॑ स॒जूः स॒जूर् वि॒धाभि॑र् वि॒धाभिः॑ स॒जूर् वसु॑भि॒र् वसु॑भिः स॒जूर् वि॒धाभि॑र् वि॒धाभिः॑ स॒जूर् वसु॑भिः । \newline
22. वि॒धाभि॒रिति॑ वि - धाभिः॑ । \newline
23. स॒जूर् वसु॑भि॒र् वसु॑भिः स॒जूः स॒जूर् वसु॑भिः स॒जूः स॒जूर् वसु॑भिः स॒जूः स॒जूर् वसु॑भिः स॒जूः । \newline
24. स॒जूरिति॑ स - जूः । \newline
25. वसु॑भिः स॒जूः स॒जूर् वसु॑भि॒र् वसु॑भिः स॒जू रु॒द्रै रु॒द्रैः स॒जूर् वसु॑भि॒र् वसु॑भिः स॒जू रु॒द्रैः । \newline
26. वसु॑भि॒रिति॒ वसु॑ - भिः॒ । \newline
27. स॒जू रु॒द्रै रु॒द्रैः स॒जूः स॒जू रु॒द्रैः स॒जूः स॒जू रु॒द्रैः स॒जूः स॒जू रु॒द्रैः स॒जूः । \newline
28. स॒जूरिति॑ स - जूः । \newline
29. रु॒द्रैः स॒जूः स॒जू रु॒द्रै रु॒द्रैः स॒जू रा॑दि॒त्यै रा॑दि॒त्यैः स॒जू रु॒द्रै रु॒द्रैः स॒जू रा॑दि॒त्यैः । \newline
30. स॒जू रा॑दि॒त्यै रा॑दि॒त्यैः स॒जूः स॒जू रा॑दि॒त्यैः स॒जूः स॒जू रा॑दि॒त्यैः स॒जूः स॒जू रा॑दि॒त्यैः स॒जूः । \newline
31. स॒जूरिति॑ स - जूः । \newline
32. आ॒दि॒त्यैः स॒जूः स॒जू रा॑दि॒त्यै रा॑दि॒त्यैः स॒जूर् विश्वै॒र् विश्वैः᳚ स॒जू रा॑दि॒त्यै रा॑दि॒त्यैः स॒जूर् विश्वैः᳚ । \newline
33. स॒जूर् विश्वै॒र् विश्वैः᳚ स॒जूः स॒जूर् विश्वै᳚र् दे॒वैर् दे॒वैर् विश्वैः᳚ स॒जूः स॒जूर् विश्वै᳚र् दे॒वैः । \newline
34. स॒जूरिति॑ स - जूः । \newline
35. विश्वै᳚र् दे॒वैर् दे॒वैर् विश्वै॒र् विश्वै᳚र् दे॒वैः स॒जूः स॒जूर् दे॒वैर् विश्वै॒र् विश्वै᳚र् दे॒वैः स॒जूः । \newline
36. दे॒वैः स॒जूः स॒जूर् दे॒वैर् दे॒वैः स॒जूर् दे॒वैर् दे॒वैः स॒जूर् दे॒वैर् दे॒वैः स॒जूर् दे॒वैः । \newline
37. स॒जूर् दे॒वैर् दे॒वैः स॒जूः स॒जूर् दे॒वैः स॒जूः स॒जूर् दे॒वैः स॒जूः स॒जूर् दे॒वैः स॒जूः । \newline
38. स॒जूरिति॑ स - जूः । \newline
39. दे॒वैः स॒जूः स॒जूर् दे॒वैर् दे॒वैः स॒जूर् दे॒वैर् दे॒वैः स॒जूर् दे॒वैर् दे॒वैः स॒जूर् दे॒वैः । \newline
40. स॒जूर् दे॒वैर् दे॒वैः स॒जूः स॒जूर् दे॒वैर् व॑योना॒धैर् व॑योना॒धैर् दे॒वैः स॒जूः स॒जूर् दे॒वैर् व॑योना॒धैः । \newline
41. स॒जूरिति॑ स - जूः । \newline
42. दे॒वैर् व॑योना॒धैर् व॑योना॒धैर् दे॒वैर् दे॒वैर् व॑योना॒धै र॒ग्नये॑ अ॒ग्नये॑ वयोना॒धैर् दे॒वैर् दे॒वैर् व॑योना॒धै र॒ग्नये᳚ । \newline
43. व॒यो॒ना॒धै र॒ग्नये॑ अ॒ग्नये॑ वयोना॒धैर् व॑योना॒धै र॒ग्नये᳚ त्वा त्वा॒ ऽग्नये॑ वयोना॒धैर् व॑योना॒धै र॒ग्नये᳚ त्वा । \newline
44. व॒यो॒ना॒धैरिति॑ वयः - ना॒धैः । \newline
45. अ॒ग्नये᳚ त्वा त्वा॒ ऽग्नये॑ अ॒ग्नये᳚ त्वा वैश्वान॒राय॑ वैश्वान॒राय॑ त्वा॒ ऽग्नये॑ अ॒ग्नये᳚ त्वा वैश्वान॒राय॑ । \newline
46. त्वा॒ वै॒श्वा॒न॒राय॑ वैश्वान॒राय॑ त्वा त्वा वैश्वान॒राया॒ श्विना॒ ऽश्विना॑ वैश्वान॒राय॑ त्वा त्वा वैश्वान॒राया॒ श्विना᳚ । \newline
47. वै॒श्वा॒न॒राया॒ श्विना॒ ऽश्विना॑ वैश्वान॒राय॑ वैश्वान॒राया॒ श्विना᳚ ऽद्ध्व॒र्यू अ॑द्ध्व॒र्यू अ॒श्विना॑ वैश्वान॒राय॑ वैश्वान॒राया॒ श्विना᳚ ऽद्ध्व॒र्यू । \newline
48. अ॒श्विना᳚ ऽद्ध्व॒र्यू अ॑द्ध्व॒र्यू अ॒श्विना॒ ऽश्विना᳚ ऽद्ध्व॒र्यू सा॑दयताꣳ सादयता मद्ध्व॒र्यू अ॒श्विना॒ ऽश्विना᳚ ऽद्ध्व॒र्यू सा॑दयताम् । \newline
49. अ॒द्ध्व॒र्यू सा॑दयताꣳ सादयता मद्ध्व॒र्यू अ॑द्ध्व॒र्यू सा॑दयता मि॒हेह सा॑दयता मद्ध्व॒र्यू अ॑द्ध्व॒र्यू सा॑दयता मि॒ह । \newline
50. अ॒द्ध्व॒र्यू इत्य॑द्ध्व॒र्यू । \newline
51. सा॒द॒य॒ता॒ मि॒हेह सा॑दयताꣳ सादयता मि॒ह त्वा᳚ त्वे॒ह सा॑दयताꣳ सादयता मि॒ह त्वा᳚ । \newline
52. इ॒ह त्वा᳚ त्वे॒हेह त्वा᳚ । \newline
53. त्वेति॑ त्वा । \newline
54. प्रा॒णम् मे॑ मे प्रा॒णम् प्रा॒णम् मे॑ पाहि पाहि मे प्रा॒णम् प्रा॒णम् मे॑ पाहि । \newline
55. प्रा॒णमिति॑ प्र - अ॒नम् । \newline
56. मे॒ पा॒हि॒ पा॒हि॒ मे॒ मे॒ पा॒ह्य॒पा॒न म॑पा॒नम् पा॑हि मे मे पाह्यपा॒नम् । \newline
57. पा॒ह्य॒पा॒न म॑पा॒नम् पा॑हि पाह्यपा॒नम् मे॑ मे अपा॒नम् पा॑हि पाह्यपा॒नम् मे᳚ । \newline
58. अ॒पा॒नम् मे॑ मे अपा॒न म॑पा॒नम् मे॑ पाहि पाहि मे अपा॒न म॑पा॒नम् मे॑ पाहि । \newline
59. अ॒पा॒नमित्य॑प - अ॒नम् । \newline
60. मे॒ पा॒हि॒ पा॒हि॒ मे॒ मे॒ पा॒हि॒ व्या॒नं ॅव्या॒नम् पा॑हि मे मे पाहि व्या॒नम् । \newline
61. पा॒हि॒ व्या॒नं ॅव्या॒नम् पा॑हि पाहि व्या॒नम् मे॑ मे व्या॒नम् पा॑हि पाहि व्या॒नम् मे᳚ । \newline
62. व्या॒नम् मे॑ मे व्या॒नं ॅव्या॒नम् मे॑ पाहि पाहि मे व्या॒नं ॅव्या॒नम् मे॑ पाहि । \newline
63. व्या॒नमिति॑ वि - अ॒नम् । \newline
64. मे॒ पा॒हि॒ पा॒हि॒ मे॒ मे॒ पा॒हि॒ चक्षु॒ श्चक्षुः॑ पाहि मे मे पाहि॒ चक्षुः॑ । \newline
65. पा॒हि॒ चक्षु॒ श्चक्षुः॑ पाहि पाहि॒ चक्षु॑र् मे मे॒ चक्षुः॑ पाहि पाहि॒ चक्षु॑र् मे । \newline
66. चक्षु॑र् मे मे॒ चक्षु॒ श्चक्षु॑र् म उ॒र्व्योर्व्या मे॒ चक्षु॒ श्चक्षु॑र् म उ॒र्व्या । \newline
67. म॒ उ॒र्व्योर्व्या मे॑ म उ॒र्व्या वि व्यु॑र्व्या मे॑ म उ॒र्व्या वि । \newline
68. उ॒र्व्या वि व्यु॑र्व्योर्व्या वि भा॑हि भाहि॒ व्यु॑र्व्योर्व्या वि भा॑हि । \newline
69. वि भा॑हि भाहि॒ वि वि भा॑हि॒ श्रोत्रꣳ॒॒ श्रोत्र॑म् भाहि॒ वि वि भा॑हि॒ श्रोत्र᳚म् । \newline
70. भा॒हि॒ श्रोत्रꣳ॒॒ श्रोत्र॑म् भाहि भाहि॒ श्रोत्र॑म् मे मे॒ श्रोत्र॑म् भाहि भाहि॒ श्रोत्र॑म् मे । \newline
71. श्रोत्र॑म् मे मे॒ श्रोत्रꣳ॒॒ श्रोत्र॑म् मे श्लोकय श्लोकय मे॒ श्रोत्रꣳ॒॒ श्रोत्र॑म् मे श्लोकय । \newline
72. मे॒ श्लो॒क॒य॒ श्लो॒क॒य॒ मे॒ मे॒ श्लो॒क॒या॒पो अ॒पः श्लो॑कय मे मे श्लोकया॒पः । \newline
73. श्लो॒क॒या॒पो अ॒पः श्लो॑कय श्लोकया॒प स्पि॑न्व पिन्वा॒पः श्लो॑कय श्लोकया॒प स्पि॑न्व । \newline
74. अ॒प स्पि॑न्व पिन्वा॒पो अ॒प स्पि॒न्वौष॑धी॒ रोष॑धीष् पिन्वा॒पो अ॒प स्पि॒न्वौष॑धीः । \newline
75. पि॒न्वौ ष॑धी॒ रोष॑धीष् पिन्व पि॒न्वौ ष॑धीर् जिन्व जि॒न्वौ ष॑धीष् पिन्व पि॒न्वौ ष॑धीर् जिन्व । \newline
76. ओष॑धीर् जिन्व जि॒न्वौ ष॑धी॒ रोष॑धीर् जिन्व द्वि॒पाद् द्वि॒पाज् जि॒न्वौ ष॑धी॒ रोष॑धीर् जिन्व द्वि॒पात् । \newline
77. जि॒न्व॒ द्वि॒पाद् द्वि॒पाज् जि॑न्व जिन्व द्वि॒पात् पा॑हि पाहि द्वि॒पाज् जि॑न्व जिन्व द्वि॒पात् पा॑हि । \newline
78. द्वि॒पात् पा॑हि पाहि द्वि॒पाद् द्वि॒पात् पा॑हि॒ चतु॑ष्पा॒च् चतु॑ष्पात् पाहि द्वि॒पाद् द्वि॒पात् पा॑हि॒ चतु॑ष्पात् । \newline
79. द्वि॒पादिति॑ द्वि - पात् । \newline
80. पा॒हि॒ चतु॑ष्पा॒च् चतु॑ष्पात् पाहि पाहि॒ चतु॑ष्पा दवाव॒ चतु॑ष्पात् पाहि पाहि॒ चतु॑ष्पा दव । \newline
81. चतु॑ष्पा दवाव॒ चतु॑ष्पा॒च् चतु॑ष्पा दव दि॒वो दि॒वो॑ ऽव॒ चतु॑ष्पा॒च् चतु॑ष्पा दव दि॒वः । \newline
82. चतु॑ष्पा॒दिति॒ चतुः॑ - पा॒त् । \newline
83. अ॒व॒ दि॒वो दि॒वो॑ ऽवाव दि॒वो वृष्टिं॒ ॅवृष्टि॑म् दि॒वो॑ ऽवाव दि॒वो वृष्टि᳚म् । \newline
84. दि॒वो वृष्टिं॒ ॅवृष्टि॑म् दि॒वो दि॒वो वृष्टि॒ मा वृष्टि॑म् दि॒वो दि॒वो वृष्टि॒ मा । \newline
85. वृष्टि॒ मा वृष्टिं॒ ॅवृष्टि॒ मेर॑ येर॒या वृष्टिं॒ ॅवृष्टि॒ मेर॑य । \newline
86. एर॑ येर॒ येर॑य । \newline
87. ई॒र॒येती॑रय । \newline
\pagebreak
\markright{ TS 4.3.5.1  \hfill https://www.vedavms.in \hfill}

\section{ TS 4.3.5.1 }

\textbf{TS 4.3.5.1 } \newline
\textbf{Samhita Paata} \newline

त्र्यवि॒र्वय॑स्त्रि॒ष्टुप् छन्दो॑ दित्य॒वाड् वयो॑ वि॒राट् छन्दः॒ पञ्चा॑वि॒र्वयो॑ गाय॒त्री छन्द॑स्त्रिव॒थ्सो वय॑ उ॒ष्णिहा॒ छन्द॑ स्तुर्य॒वाड् वयो॑ऽनु॒ष्टुप् छन्दः॑ पष्ठ॒वाद् वयो॑ बृह॒ती छन्द॑ उ॒क्षा वयः॑ स॒तोबृ॑हती॒ छन्द॑ ऋष॒भो वयः॑ क॒कुच्छन्दो॑ धे॒नुर्वयो॒ जग॑ती॒ छन्दो॑ऽन॒ड्वान्. वयः॑ प॒ङ्क्ति श्छन्दो॑ ब॒स्तो वयो॑ विव॒लं छन्दो॑ वृ॒ष्णिर्वयो॑ विशा॒लं छन्दः॒ पुरु॑षो॒ वय॑ ( ) स्त॒न्द्रं छन्दो᳚ व्या॒घ्रो वयोऽना॑धृष्टं॒ छन्दः॑ सिꣳ॒॒हो वय॑ श्छ॒दि श्छन्दो॑ विष्ट॒भ्ॐ ॅवयोऽधि॑पति॒ श्छन्दः॑ क्ष॒त्रं ॅवयो॒ मय॑न्दं॒ छन्दो॑ वि॒श्वक॑र्मा॒ वयः॑ परमे॒ष्ठी छन्दो॑ मू॒र्द्धा वयः॑ प्र॒जाप॑ति॒ श्छन्दः॑ ॥ \newline

\textbf{Pada Paata} \newline

त्र्यवि॒रिति॑ त्रि - अविः॑ । वयः॑ । त्रि॒ष्टुप् । छन्दः॑ । दि॒त्य॒वाडिति॑ दित्य - वाट् । वयः॑ । वि॒राडिति॑ वि - राट् । छन्दः॑ । पञ्चा॑वि॒रिति॒ पञ्च॑-अ॒विः॒ । वयः॑ । गा॒य॒त्री । छन्दः॑ । त्रि॒व॒थ्स इति॑ त्रि - व॒थ्सः । वयः॑ । उ॒ष्णिहा᳚ । छन्दः॑ । तु॒र्य॒वाडिति॑ तुर्य - वाट् । वयः॑ । अ॒नु॒ष्टुबित्य॑नु - स्तुप् । छन्दः॑ । प॒ष्ठ॒वादिति॑ पष्ठ - वात् । वयः॑ । बृ॒ह॒ती । छन्दः॑ । उ॒क्षा । वयः॑ । स॒तोबृ॑ह॒तीति॑ स॒तः - बृ॒ह॒ती॒ । छन्दः॑ । ऋ॒ष॒भः । वयः॑ । क॒कुत् । छन्दः॑ । धे॒नुः । वयः॑ । जग॑ती । छन्दः॑ । अ॒न॒ड्वान् । वयः॑ । प॒ङ्क्तिः । छन्दः॑ । ब॒स्तः । वयः॑ । वि॒व॒लमिति॑ वि - व॒लम् । छन्दः॑ । वृ॒ष्णिः । वयः॑ । वि॒शा॒लमिति॑ वि-शा॒लम् । छन्दः॑ । पुरु॑षः । वयः॑ ( ) । त॒न्द्रम् । छन्दः॑ । व्या॒घ्रः । वयः॑ । अना॑धृष्ट॒मित्यना᳚ - धृ॒ष्ट॒म् । छन्दः॑ । सिꣳ॒॒हः । वयः॑ । छ॒दिः । छन्दः॑ । वि॒ष्ट॒भं इति॑ वि - स्त॒भंः । वयः॑ । अधि॑पति॒रित्यधि॑-प॒तिः॒ । छन्दः॑ । क्ष॒त्रम् । वयः॑ । मय॑न्दम् । छन्दः॑ । वि॒श्वक॒र्मेति॑ वि॒श्व - क॒र्मा॒ । वयः॑ । प॒र॒मे॒ष्ठी । छन्दः॑ । मू॒द्‌र्धा । वयः॑ । प्र॒जाप॑ति॒रिति॑ प्र॒जा - प॒तिः॒ । छन्दः॑ ॥  \newline


\textbf{Krama Paata} \newline

त्र्यवि॒र् वयः॑ । त्र्यवि॒रिति॑ त्रि - अविः॑ । वय॑स्त्रि॒ष्टुप् । त्रि॒ष्टुप् छन्दः॑ । छन्दो॑ दित्य॒वाट् । दि॒त्य॒वाड् वयः॑ । दि॒त्य॒वाडिति॑ दित्य - वाट् । वयो॑ वि॒राट् । वि॒राट् छन्दः॑ । वि॒राडिति॑ वि - राट् । छन्दः॒ पञ्चा॑विः । पञ्चा॑वि॒र् वयः॑ । पञ्चा॑वि॒रिति॒ पञ्च॑ - अ॒विः॒ । वयो॑ गाय॒त्री । गा॒य॒त्री छन्दः॑ । छन्द॑स्त्रिव॒थ्सः । त्रि॒व॒थ्सो वयः॑ । त्रि॒व॒थ्स इति॑ त्रि - व॒थ्सः । वय॑ उ॒ष्णिहा᳚ । उ॒ष्णिहा॒ छन्दः॑ । छन्द॑स्तुर्य॒वाट् । तु॒र्य॒वाड् वयः॑ । तु॒र्य॒वाडिति॑ तुर्य - वाट् । वयो॑ऽनु॒ष्टुप् । अ॒नु॒ष्टुप् छन्दः॑ । अ॒नु॒ष्टुबित्य॑नु - स्तुप् । छन्दः॑ पष्ठ॒वात् । प॒ष्ठ॒वाद् वयः॑ । प॒ष्ठ॒वादिति॑ पष्ठ - वात् । वयो॑ बृह॒ती । बृ॒ह॒ती छन्दः॑ । छन्द॑ उ॒क्षा । उ॒क्षा वयः॑ । वयः॑ स॒तोबृ॑हती । स॒तोबृ॑हती॒ छन्दः॑ । स॒तोबृ॑ह॒तीति॑ स॒तः - बृ॒ह॒ती॒ । छन्द॑ ऋष॒भः । ऋ॒ष॒भो वयः॑ । वयः॑ क॒कुत् । क॒कुच्छन्दः॑ । छन्दो॑ धे॒नुः । धे॒नुर् वयः॑ । वयो॒ जग॑ती । जग॑ती॒ छन्दः॑ । छन्दो॑ऽन॒ड्वान् । अ॒न॒ड्वान्. वयः॑ । वयः॑ प॒ङ्क्तिः । प॒ङ्क्ति श्छन्दः॑ । छन्दो॑ ब॒स्तः । ब॒स्तो वयः॑ । वयो॑ विव॒लम् । वि॒व॒लम् छन्दः॑ । वि॒व॒लमिति॑ वि - व॒लम् । छन्दो॑ वृ॒ष्णिः । वृ॒ष्णिर् वयः॑ । वयो॑ विशा॒लम् । वि॒शा॒लम् छन्दः॑ । वि॒शा॒लमिति॑ वि - शा॒लम् । छन्दः॒ पुरु॑षः । पुरु॑षो॒ वयः॑ ( ) । वय॑स्त॒न्द्रम् । त॒न्द्रम् छन्दः॑ । छन्दो᳚ व्या॒घ्रः । व्या॒घ्रो वयः॑ । वयोऽना॑धृष्टम् । अना॑धृष्ट॒म् छन्दः॑ । अना॑धृष्ट॒मित्यना᳚ - धृ॒ष्ट॒म् । छन्दः॑ सिꣳ॒॒हः । सिꣳ॒॒हो वयः॑ । वय॑ श्छ॒दिः । छ॒दि श्छन्दः॑ । छन्दो॑ विष्ट॒म्भः । वि॒ष्ट॒म्भो वयः॑ । वि॒ष्ट॒म्भ इति॑ वि - स्त॒म्भः । वयोऽधि॑पतिः । अधि॑पति॒ श्छन्दः॑ । अधि॑पति॒रित्यधि॑ - प॒तिः॒ । छन्दः॑ क्ष॒त्रम् । क्ष॒त्रम् ॅवयः॑ । वयो॒ मय॑न्दम् । मय॑न्द॒म् छन्दः॑ । छन्दो॑ वि॒श्वक॑र्मा । वि॒श्वक॑र्मा॒ वयः॑ । वि॒श्वक॒र्मेति॑ वि॒श्व - क॒र्मा॒ । वयः॑ परमे॒ष्ठी । प॒र॒मे॒ष्ठी छन्दः॑ । छन्दो॑ मू॒र्द्धा । मू॒र्द्धा वयः॑ । वयः॑ प्र॒जाप॑तिः । प्र॒जाप॑ति॒ श्छन्दः॑ । प्र॒जाप॑ति॒रिति॑ प्र॒जा - प॒तिः॒ । छन्द॒ इति॒ छन्दः॑ । \newline

\textbf{Jatai Paata} \newline

1. त्र्यवि॒र् वयो॒ वय॒ स्त्र्यवि॒ स्त्र्यवि॒र् वयः॑ । \newline
2. त्र्यवि॒रिति॑ त्रि - अविः॑ । \newline
3. वय॑ स्त्रि॒ष्टुप् त्रि॒ष्टुब् वयो॒ वय॑ स्त्रि॒ष्टुप् । \newline
4. त्रि॒ष्टुप् छन्द॒ श्छन्द॑ स्त्रि॒ष्टुप् त्रि॒ष्टुप् छन्दः॑ । \newline
5. छन्दो॑ दित्य॒वाड् दि॑त्य॒वाट् छन्द॒ श्छन्दो॑ दित्य॒वाट् । \newline
6. दि॒त्य॒वाड् वयो॒ वयो॑ दित्य॒वाड् दि॑त्य॒वाड् वयः॑ । \newline
7. दि॒त्य॒वाडिति॑ दित्य - वाट् । \newline
8. वयो॑ वि॒राड् वि॒राड् वयो॒ वयो॑ वि॒राट् । \newline
9. वि॒राट् छन्द॒ श्छन्दो॑ वि॒राड् वि॒राट् छन्दः॑ । \newline
10. वि॒राडिति॑ वि - राट् । \newline
11. छन्दः॒ पञ्चा॑विः॒ पञ्चा॑वि॒ श्छन्द॒ श्छन्दः॒ पञ्चा॑विः । \newline
12. पञ्चा॑वि॒र् वयो॒ वयः॒ पञ्चा॑विः॒ पञ्चा॑वि॒र् वयः॑ । \newline
13. पञ्चा॑वि॒रिति॒ पञ्च॑ - अ॒विः॒ । \newline
14. वयो॑ गाय॒त्री गा॑य॒त्री वयो॒ वयो॑ गाय॒त्री । \newline
15. गा॒य॒त्री छन्द॒ श्छन्दो॑ गाय॒त्री गा॑य॒त्री छन्दः॑ । \newline
16. छन्द॑ स्त्रिव॒थ्स स्त्रि॑व॒थ्स श्छन्द॒ श्छन्द॑ स्त्रिव॒थ्सः । \newline
17. त्रि॒व॒थ्सो वयो॒ वय॑ स्त्रिव॒थ् सस्त्रि॑व॒थ्सो वयः॑ । \newline
18. त्रि॒व॒थ्स इति॑ त्रि - व॒थ्सः । \newline
19. वय॑ उ॒ष्णि हो॒ष्णिहा॒ वयो॒ वय॑ उ॒ष्णिहा᳚ । \newline
20. उ॒ष्णिहा॒ छन्द॒ श्छन्द॑ उ॒ष्णि हो॒ष्णिहा॒ छन्दः॑ । \newline
21. छन्द॑ स्तुर्य॒वाट् तु॑र्य॒वाट् छन्द॒ श्छन्द॑ स्तुर्य॒वाट् । \newline
22. तु॒र्य॒वाड् वयो॒ वय॑ स्तुर्य॒वाट् तु॑र्य॒वाड् वयः॑ । \newline
23. तु॒र्य॒वाडिति॑ तुर्य - वाट् । \newline
24. वयो॑ ऽनु॒ष्टु ब॑नु॒ष्टुब् वयो॒ वयो॑ ऽनु॒ष्टुप् । \newline
25. अ॒नु॒ष्टुप् छन्द॒ श्छन्दो॑ ऽनु॒ष्टु ब॑नु॒ष्टुप् छन्दः॑ । \newline
26. अ॒नु॒ष्टुबित्य॑नु - स्तुप् । \newline
27. छन्दः॑ पष्ठ॒वात् प॑ष्ठ॒वाच् छन्द॒ श्छन्दः॑ पष्ठ॒वात् । \newline
28. प॒ष्ठ॒वाद् वयो॒ वयः॑ पष्ठ॒वात् प॑ष्ठ॒वाद् वयः॑ । \newline
29. प॒ष्ठ॒वादिति॑ पष्ठ - वात् । \newline
30. वयो॑ बृह॒ती बृ॑ह॒ती वयो॒ वयो॑ बृह॒ती । \newline
31. बृ॒ह॒ती छन्द॒ श्छन्दो॑ बृह॒ती बृ॑ह॒ती छन्दः॑ । \newline
32. छन्द॑ उ॒क्षोक्षा छन्द॒ श्छन्द॑ उ॒क्षा । \newline
33. उ॒क्षा वयो॒ वय॑ उ॒क्षोक्षा वयः॑ । \newline
34. वयः॑ स॒तोबृ॑हती स॒तोबृ॑हती॒ वयो॒ वयः॑ स॒तोबृ॑हती । \newline
35. स॒तोबृ॑हती॒ छन्द॒ श्छन्दः॑ स॒तोबृ॑हती स॒तोबृ॑हती॒ छन्दः॑ । \newline
36. स॒तोबृ॑ह॒तीति॑ स॒तः - बृ॒ह॒ती॒ । \newline
37. छन्द॑ ऋष॒भ ऋ॑ष॒भ श्छन्द॒ श्छन्द॑ ऋष॒भः । \newline
38. ऋ॒ष॒भो वयो॒ वय॑ ऋष॒भ ऋ॑ष॒भो वयः॑ । \newline
39. वयः॑ क॒कुत् क॒कुद् वयो॒ वयः॑ क॒कुत् । \newline
40. क॒कुच् छन्द॒ श्छन्दः॑ क॒कुत् क॒कुच् छन्दः॑ । \newline
41. छन्दो॑ धे॒नुर् धे॒नु श्छन्द॒ श्छन्दो॑ धे॒नुः । \newline
42. धे॒नुर् वयो॒ वयो॑ धे॒नुर् धे॒नुर् वयः॑ । \newline
43. वयो॒ जग॑ती॒ जग॑ती॒ वयो॒ वयो॒ जग॑ती । \newline
44. जग॑ती॒ छन्द॒ श्छन्दो॒ जग॑ती॒ जग॑ती॒ छन्दः॑ । \newline
45. छन्दो॑ ऽन॒ड्वा-न॑न॒ड्वान् छन्द॒ श्छन्दो॑ ऽन॒ड्वान् । \newline
46. अ॒न॒ड्वान्. वयो॒ वयो॑ ऽन॒ड्वा-न॑न॒ड्वान्. वयः॑ । \newline
47. वयः॑ प॒ङ्क्तिः प॒ङ्क्तिर् वयो॒ वयः॑ प॒ङ्क्तिः । \newline
48. प॒ङ्क्ति श्छन्द॒ श्छन्दः॑ प॒ङ्क्तिः प॒ङ्क्ति श्छन्दः॑ । \newline
49. छन्दो॑ ब॒स्तो ब॒स्त श्छन्द॒ श्छन्दो॑ ब॒स्तः । \newline
50. ब॒स्तो वयो॒ वयो॑ ब॒स्तो ब॒स्तो वयः॑ । \newline
51. वयो॑ विव॒लं ॅवि॑व॒लं ॅवयो॒ वयो॑ विव॒लम् । \newline
52. वि॒व॒लम् छन्द॒ श्छन्दो॑ विव॒लं ॅवि॑व॒लम् छन्दः॑ । \newline
53. वि॒व॒लमिति॑ वि - व॒लम् । \newline
54. छन्दो॑ वृ॒ष्णिर् वृ॒ष्णि श्छन्द॒ श्छन्दो॑ वृ॒ष्णिः । \newline
55. वृ॒ष्णिर् वयो॒ वयो॑ वृ॒ष्णिर् वृ॒ष्णिर् वयः॑ । \newline
56. वयो॑ विशा॒लं ॅवि॑शा॒लं ॅवयो॒ वयो॑ विशा॒लम् । \newline
57. वि॒शा॒लम् छन्द॒ श्छन्दो॑ विशा॒लं ॅवि॑शा॒लम् छन्दः॑ । \newline
58. वि॒शा॒लमिति॑ वि - शा॒लम् । \newline
59. छन्दः॒ पुरु॑षः॒ पुरु॑ष॒ श्छन्द॒ श्छन्दः॒ पुरु॑षः । \newline
60. पुरु॑षो॒ वयो॒ वयः॒ पुरु॑षः॒ पुरु॑षो॒ वयः॑ । \newline
61. वय॑ स्त॒न्द्रम् त॒न्द्रं ॅवयो॒ वय॑ स्त॒न्द्रम् । \newline
62. त॒न्द्रम् छन्द॒ श्छन्द॑ स्त॒न्द्रम् त॒न्द्रम् छन्दः॑ । \newline
63. छन्दो᳚ व्या॒घ्रो व्या॒घ्र श्छन्द॒ श्छन्दो᳚ व्या॒घ्रः । \newline
64. व्या॒घ्रो वयो॒ वयो᳚ व्या॒घ्रो व्या॒घ्रो वयः॑ । \newline
65. वयो ऽना॑धृष्ट॒ मना॑धृष्टं॒ ॅवयो॒ वयो ऽना॑धृष्टम् । \newline
66. अना॑धृष्ट॒म् छन्द॒ श्छन्दो ऽना॑धृष्ट॒ मना॑धृष्ट॒म् छन्दः॑ । \newline
67. अना॑धृष्ट॒मित्यना᳚ - धृ॒ष्ट॒म् । \newline
68. छन्दः॑ सिꣳ॒॒हः सिꣳ॒॒ह श्छन्द॒ श्छन्दः॑ सिꣳ॒॒हः । \newline
69. सिꣳ॒॒हो वयो॒ वयः॑ सिꣳ॒॒हः सिꣳ॒॒हो वयः॑ । \newline
70. वय॑ श्छ॒दि श्छ॒दिर् वयो॒ वय॑ श्छ॒दिः । \newline
71. छ॒दि श्छन्द॒ श्छन्द॑ श्छ॒दि श्छ॒दि श्छन्दः॑ । \newline
72. छन्दो॑ विष्टं॒भो वि॑ष्टं॒भ श्छन्द॒ श्छन्दो॑ विष्टं॒भः । \newline
73. वि॒ष्टं॒भो वयो॒ वयो॑ विष्टं॒भो वि॑ष्टं॒भो वयः॑ । \newline
74. वि॒ष्टं॒भ इति॑ वि - स्तं॒भः । \newline
75. वयो ऽधि॑पति॒ रधि॑पति॒र् वयो॒ वयो ऽधि॑पतिः । \newline
76. अधि॑पति॒ श्छन्द॒ श्छन्दो ऽधि॑पति॒ रधि॑पति॒ श्छन्दः॑ । \newline
77. अधि॑पति॒रित्यधि॑ - प॒तिः॒ । \newline
78. छन्दः॑ क्ष॒त्रम् क्ष॒त्रम् छन्द॒ श्छन्दः॑ क्ष॒त्रम् । \newline
79. क्ष॒त्रं ॅवयो॒ वयः॑ क्ष॒त्रम् क्ष॒त्रं ॅवयः॑ । \newline
80. वयो॒ मय॑न्द॒म् मय॑न्दं॒ ॅवयो॒ वयो॒ मय॑न्दम् । \newline
81. मय॑न्द॒म् छन्द॒ श्छन्दो॒ मय॑न्द॒म् मय॑न्द॒म् छन्दः॑ । \newline
82. छन्दो॑ वि॒श्वक॑र्मा वि॒श्वक॑र्मा॒ छन्द॒ श्छन्दो॑ वि॒श्वक॑र्मा । \newline
83. वि॒श्वक॑र्मा॒ वयो॒ वयो॑ वि॒श्वक॑र्मा वि॒श्वक॑र्मा॒ वयः॑ । \newline
84. वि॒श्वक॒र्मेति॑ वि॒श्व - क॒र्मा॒ । \newline
85. वयः॑ परमे॒ष्ठी प॑रमे॒ष्ठी वयो॒ वयः॑ परमे॒ष्ठी । \newline
86. प॒र॒मे॒ष्ठी छन्द॒ श्छन्दः॑ परमे॒ष्ठी प॑रमे॒ष्ठी छन्दः॑ । \newline
87. छन्दो॑ मू॒र्द्धा मू॒र्द्धा छन्द॒ श्छन्दो॑ मू॒र्द्धा । \newline
88. मू॒र्द्धा वयो॒ वयो॑ मू॒र्द्धा मू॒र्द्धा वयः॑ । \newline
89. वयः॑ प्र॒जाप॑तिः प्र॒जाप॑ति॒र् वयो॒ वयः॑ प्र॒जाप॑तिः । \newline
90. प्र॒जाप॑ति॒ श्छन्द॒ श्छन्दः॑ प्र॒जाप॑तिः प्र॒जाप॑ति॒ श्छन्दः॑ । \newline
91. प्र॒जाप॑ति॒रिति॑ प्र॒जा - प॒तिः॒ । \newline
92. छन्द॒ इति॒ छन्दः॑ । \newline

\textbf{Ghana Paata } \newline

1. त्र्यवि॒र् वयो॒ वय॒ स्त्र्यवि॒ स्त्र्यवि॒र् वय॑ स्त्रि॒ष्टुप् त्रि॒ष्टुब् वय॒ स्त्र्यवि॒ स्त्र्यवि॒र् वय॑ स्त्रि॒ष्टुप् । \newline
2. त्र्यवि॒रिति॑ त्रि - अविः॑ । \newline
3. वय॑ स्त्रि॒ष्टुप् त्रि॒ष्टुब् वयो॒ वय॑ स्त्रि॒ष्टुप् छन्द॒ श्छन्द॑ स्त्रि॒ष्टुब् वयो॒ वय॑ स्त्रि॒ष्टुप् छन्दः॑ । \newline
4. त्रि॒ष्टुप् छन्द॒ श्छन्द॑ स्त्रि॒ष्टुप् त्रि॒ष्टुप् छन्दो॑ दित्य॒वाड् दि॑त्य॒वाट् छन्द॑ स्त्रि॒ष्टुप् त्रि॒ष्टुप् छन्दो॑ दित्य॒वाट् । \newline
5. छन्दो॑ दित्य॒वाड् दि॑त्य॒वाट् छन्द॒ श्छन्दो॑ दित्य॒वाड् वयो॒ वयो॑ दित्य॒वाट् छन्द॒ श्छन्दो॑ दित्य॒वाड् वयः॑ । \newline
6. दि॒त्य॒वाड् वयो॒ वयो॑ दित्य॒वाड् दि॑त्य॒वाड् वयो॑ वि॒राड् वि॒राड् वयो॑ दित्य॒वाड् दि॑त्य॒वाड् वयो॑ वि॒राट् । \newline
7. दि॒त्य॒वाडिति॑ दित्य - वाट् । \newline
8. वयो॑ वि॒राड् वि॒राड् वयो॒ वयो॑ वि॒राट् छन्द॒ श्छन्दो॑ वि॒राड् वयो॒ वयो॑ वि॒राट् छन्दः॑ । \newline
9. वि॒राट् छन्द॒ श्छन्दो॑ वि॒राड् वि॒राट् छन्दः॒ पञ्चा॑विः॒ पञ्चा॑वि॒ श्छन्दो॑ वि॒राड् वि॒राट् छन्दः॒ पञ्चा॑विः । \newline
10. वि॒राडिति॑ वि - राट् । \newline
11. छन्दः॒ पञ्चा॑विः॒ पञ्चा॑वि॒ श्छन्द॒ श्छन्दः॒ पञ्चा॑वि॒र् वयो॒ वयः॒ पञ्चा॑वि॒ श्छन्द॒ श्छन्दः॒ पञ्चा॑वि॒र् वयः॑ । \newline
12. पञ्चा॑वि॒र् वयो॒ वयः॒ पञ्चा॑विः॒ पञ्चा॑वि॒र् वयो॑ गाय॒त्री गा॑य॒त्री वयः॒ पञ्चा॑विः॒ पञ्चा॑वि॒र् वयो॑ गाय॒त्री । \newline
13. पञ्चा॑वि॒रिति॒ पञ्च॑ - अ॒विः॒ । \newline
14. वयो॑ गाय॒त्री गा॑य॒त्री वयो॒ वयो॑ गाय॒त्री छन्द॒ श्छन्दो॑ गाय॒त्री वयो॒ वयो॑ गाय॒त्री छन्दः॑ । \newline
15. गा॒य॒त्री छन्द॒ श्छन्दो॑ गाय॒त्री गा॑य॒त्री छन्द॑ स्त्रिव॒ थ्सस्त्रि॑व॒ थ्सश्छन्दो॑ गाय॒त्री गा॑य॒त्री छन्द॑ स्त्रिव॒थ्सः । \newline
16. छन्द॑ स्त्रिव॒ थ्सस्त्रि॑व॒थ्स श्छन्द॒ श्छन्द॑ स्त्रिव॒थ्सो वयो॒ वय॑ स्त्रिव॒ थ्सश्छन्द॒ श्छन्द॑ स्त्रिव॒थ्सो वयः॑ । \newline
17. त्रि॒व॒थ्सो वयो॒ वय॑ स्त्रिव॒ थ्सस्त्रि॑व॒थ्सो वय॑ उ॒ष्णि हो॒ष्णिहा॒ वय॑ स्त्रिव॒ थ्सस्त्रि॑व॒थ्सो वय॑ उ॒ष्णिहा᳚ । \newline
18. त्रि॒व॒थ्स इति॑ त्रि - व॒थ्सः । \newline
19. वय॑ उ॒ष्णि हो॒ष्णिहा॒ वयो॒ वय॑ उ॒ष्णिहा॒ छन्द॒ श्छन्द॑ उ॒ष्णिहा॒ वयो॒ वय॑ उ॒ष्णिहा॒ छन्दः॑ । \newline
20. उ॒ष्णिहा॒ छन्द॒ श्छन्द॑ उ॒ष्णि हो॒ष्णिहा॒ छन्द॑ स्तुर्य॒वाट् तु॑र्य॒वाट् छन्द॑ उ॒ष्णि हो॒ष्णिहा॒ छन्द॑ स्तुर्य॒वाट् । \newline
21. छन्द॑ स्तुर्य॒वाट् तु॑र्य॒वाट् छन्द॒ श्छन्द॑ स्तुर्य॒वाड् वयो॒ वय॑ स्तुर्य॒वाट् छन्द॒ श्छन्द॑ स्तुर्य॒वाड् वयः॑ । \newline
22. तु॒र्य॒वाड् वयो॒ वय॑ स्तुर्य॒वाट् तु॑र्य॒वाड् वयो॑ ऽनु॒ष्टु ब॑नु॒ष्टुब् वय॑ स्तुर्य॒वाट् तु॑र्य॒वाड् वयो॑ ऽनु॒ष्टुप् । \newline
23. तु॒र्य॒वाडिति॑ तुर्य - वाट् । \newline
24. वयो॑ ऽनु॒ष्टु ब॑नु॒ष्टुब् वयो॒ वयो॑ ऽनु॒ष्टुप् छन्द॒ श्छन्दो॑ ऽनु॒ष्टुब् वयो॒ वयो॑ ऽनु॒ष्टुप् छन्दः॑ । \newline
25. अ॒नु॒ष्टुप् छन्द॒ श्छन्दो॑ ऽनु॒ष्टु ब॑नु॒ष्टुप् छन्दः॑ पष्ठ॒वात् प॑ष्ठ॒वाच् छन्दो॑ ऽनु॒ष्टु ब॑नु॒ष्टुप् छन्दः॑ पष्ठ॒वात् । \newline
26. अ॒नु॒ष्टुबित्य॑नु - स्तुप् । \newline
27. छन्दः॑ पष्ठ॒वात् प॑ष्ठ॒वाच् छन्द॒ श्छन्दः॑ पष्ठ॒वाद् वयो॒ वयः॑ पष्ठ॒वाच् छन्द॒ श्छन्दः॑ पष्ठ॒वाद् वयः॑ । \newline
28. प॒ष्ठ॒वाद् वयो॒ वयः॑ पष्ठ॒वात् प॑ष्ठ॒वाद् वयो॑ बृह॒ती बृ॑ह॒ती वयः॑ पष्ठ॒वात् प॑ष्ठ॒वाद् वयो॑ बृह॒ती । \newline
29. प॒ष्ठ॒वादिति॑ पष्ठ - वात् । \newline
30. वयो॑ बृह॒ती बृ॑ह॒ती वयो॒ वयो॑ बृह॒ती छन्द॒ श्छन्दो॑ बृह॒ती वयो॒ वयो॑ बृह॒ती छन्दः॑ । \newline
31. बृ॒ह॒ती छन्द॒ श्छन्दो॑ बृह॒ती बृ॑ह॒ती छन्द॑ उ॒क्षोक्षा छन्दो॑ बृह॒ती बृ॑ह॒ती छन्द॑ उ॒क्षा । \newline
32. छन्द॑ उ॒क्षोक्षा छन्द॒ श्छन्द॑ उ॒क्षा वयो॒ वय॑ उ॒क्षा छन्द॒ श्छन्द॑ उ॒क्षा वयः॑ । \newline
33. उ॒क्षा वयो॒ वय॑ उ॒क्षोक्षा वयः॑ स॒तोबृ॑हती स॒तोबृ॑हती॒ वय॑ उ॒क्षोक्षा वयः॑ स॒तोबृ॑हती । \newline
34. वयः॑ स॒तोबृ॑हती स॒तोबृ॑हती॒ वयो॒ वयः॑ स॒तोबृ॑हती॒ छन्द॒ श्छन्दः॑ स॒तोबृ॑हती॒ वयो॒ वयः॑ स॒तोबृ॑हती॒ छन्दः॑ । \newline
35. स॒तोबृ॑हती॒ छन्द॒ श्छन्दः॑ स॒तोबृ॑हती स॒तोबृ॑हती॒ छन्द॑ ऋष॒भ ऋ॑ष॒भ श्छन्दः॑ स॒तोबृ॑हती स॒तोबृ॑हती॒ छन्द॑ ऋष॒भः । \newline
36. स॒तोबृ॑ह॒तीति॑ स॒तः - बृ॒ह॒ती॒ । \newline
37. छन्द॑ ऋष॒भ ऋ॑ष॒भ श्छन्द॒ श्छन्द॑ ऋष॒भो वयो॒ वय॑ ऋष॒भ श्छन्द॒ श्छन्द॑ ऋष॒भो वयः॑ । \newline
38. ऋ॒ष॒भो वयो॒ वय॑ ऋष॒भ ऋ॑ष॒भो वयः॑ क॒कुत् क॒कुद् वय॑ ऋष॒भ ऋ॑ष॒भो वयः॑ क॒कुत् । \newline
39. वयः॑ क॒कुत् क॒कुद् वयो॒ वयः॑ क॒कुच् छन्द॒ श्छन्दः॑ क॒कुद् वयो॒ वयः॑ क॒कुच् छन्दः॑ । \newline
40. क॒कुच् छन्द॒ श्छन्दः॑ क॒कुत् क॒कुच् छन्दो॑ धे॒नुर् धे॒नु श्छन्दः॑ क॒कुत् क॒कुच् छन्दो॑ धे॒नुः । \newline
41. छन्दो॑ धे॒नुर् धे॒नु श्छन्द॒ श्छन्दो॑ धे॒नुर् वयो॒ वयो॑ धे॒नु श्छन्द॒ श्छन्दो॑ धे॒नुर् वयः॑ । \newline
42. धे॒नुर् वयो॒ वयो॑ धे॒नुर् धे॒नुर् वयो॒ जग॑ती॒ जग॑ती॒ वयो॑ धे॒नुर् धे॒नुर् वयो॒ जग॑ती । \newline
43. वयो॒ जग॑ती॒ जग॑ती॒ वयो॒ वयो॒ जग॑ती॒ छन्द॒ श्छन्दो॒ जग॑ती॒ वयो॒ वयो॒ जग॑ती॒ छन्दः॑ । \newline
44. जग॑ती॒ छन्द॒ श्छन्दो॒ जग॑ती॒ जग॑ती॒ छन्दो॑ ऽन॒ड्वा न॑न॒ड्वान् छन्दो॒ जग॑ती॒ जग॑ती॒ छन्दो॑ ऽन॒ड्वान् । \newline
45. छन्दो॑ ऽन॒ड्वा न॑न॒ड्वान् छन्द॒ श्छन्दो॑ ऽन॒ड्वान्. वयो॒ वयो॑ ऽन॒ड्वान् छन्द॒ श्छन्दो॑ ऽन॒ड्वान्. वयः॑ । \newline
46. अ॒न॒ड्वान्. वयो॒ वयो॑ ऽन॒ड्वा न॑न॒ड्वान्. वयः॑ प॒ङ्क्तिः प॒ङ्क्तिर् वयो॑ ऽन॒ड्वा न॑न॒ड्वान्. वयः॑ प॒ङ्क्तिः । \newline
47. वयः॑ प॒ङ्क्तिः प॒ङ्क्तिर् वयो॒ वयः॑ प॒ङ्क्ति श्छन्द॒ श्छन्दः॑ प॒ङ्क्तिर् वयो॒ वयः॑ प॒ङ्क्ति श्छन्दः॑ । \newline
48. प॒ङ्क्ति श्छन्द॒ श्छन्दः॑ प॒ङ्क्तिः प॒ङ्क्ति श्छन्दो॑ ब॒स्तो ब॒स्त श्छन्दः॑ प॒ङ्क्तिः प॒ङ्क्ति श्छन्दो॑ ब॒स्तः । \newline
49. छन्दो॑ ब॒स्तो ब॒स्त श्छन्द॒ श्छन्दो॑ ब॒स्तो वयो॒ वयो॑ ब॒स्त श्छन्द॒ श्छन्दो॑ ब॒स्तो वयः॑ । \newline
50. ब॒स्तो वयो॒ वयो॑ ब॒स्तो ब॒स्तो वयो॑ विव॒लं ॅवि॑व॒लं ॅवयो॑ ब॒स्तो ब॒स्तो वयो॑ विव॒लम् । \newline
51. वयो॑ विव॒लं ॅवि॑व॒लं ॅवयो॒ वयो॑ विव॒लम् छन्द॒ श्छन्दो॑ विव॒लं ॅवयो॒ वयो॑ विव॒लम् छन्दः॑ । \newline
52. वि॒व॒लम् छन्द॒ श्छन्दो॑ विव॒लं ॅवि॑व॒लम् छन्दो॑ वृ॒ष्णिर् वृ॒ष्णि श्छन्दो॑ विव॒लं ॅवि॑व॒लम् छन्दो॑ वृ॒ष्णिः । \newline
53. वि॒व॒लमिति॑ वि - व॒लम् । \newline
54. छन्दो॑ वृ॒ष्णिर् वृ॒ष्णि श्छन्द॒ श्छन्दो॑ वृ॒ष्णिर् वयो॒ वयो॑ वृ॒ष्णि श्छन्द॒ श्छन्दो॑ वृ॒ष्णिर् वयः॑ । \newline
55. वृ॒ष्णिर् वयो॒ वयो॑ वृ॒ष्णिर् वृ॒ष्णिर् वयो॑ विशा॒लं ॅवि॑शा॒लं ॅवयो॑ वृ॒ष्णिर् वृ॒ष्णिर् वयो॑ विशा॒लम् । \newline
56. वयो॑ विशा॒लं ॅवि॑शा॒लं ॅवयो॒ वयो॑ विशा॒लम् छन्द॒ श्छन्दो॑ विशा॒लं ॅवयो॒ वयो॑ विशा॒लम् छन्दः॑ । \newline
57. वि॒शा॒लम् छन्द॒ श्छन्दो॑ विशा॒लं ॅवि॑शा॒लम् छन्दः॒ पुरु॑षः॒ पुरु॑ष॒ श्छन्दो॑ विशा॒लं ॅवि॑शा॒लम् छन्दः॒ पुरु॑षः । \newline
58. वि॒शा॒लमिति॑ वि - शा॒लम् । \newline
59. छन्दः॒ पुरु॑षः॒ पुरु॑ष॒ श्छन्द॒ श्छन्दः॒ पुरु॑षो॒ वयो॒ वयः॒ पुरु॑ष॒ श्छन्द॒ श्छन्दः॒ पुरु॑षो॒ वयः॑ । \newline
60. पुरु॑षो॒ वयो॒ वयः॒ पुरु॑षः॒ पुरु॑षो॒ वय॑ स्त॒न्द्रम् त॒न्द्रं ॅवयः॒ पुरु॑षः॒ पुरु॑षो॒ वय॑ स्त॒न्द्रम् । \newline
61. वय॑ स्त॒न्द्रम् त॒न्द्रं ॅवयो॒ वय॑ स्त॒न्द्रम् छन्द॒ श्छन्द॑ स्त॒न्द्रं ॅवयो॒ वय॑ स्त॒न्द्रम् छन्दः॑ । \newline
62. त॒न्द्रम् छन्द॒ श्छन्द॑ स्त॒न्द्रम् त॒न्द्रम् छन्दो᳚ व्या॒घ्रो व्या॒घ्र श्छन्द॑ स्त॒न्द्रम् त॒न्द्रम् छन्दो᳚ व्या॒घ्रः । \newline
63. छन्दो᳚ व्या॒घ्रो व्या॒घ्र श्छन्द॒ श्छन्दो᳚ व्या॒घ्रो वयो॒ वयो᳚ व्या॒घ्र श्छन्द॒ श्छन्दो᳚ व्या॒घ्रो वयः॑ । \newline
64. व्या॒घ्रो वयो॒ वयो᳚ व्या॒घ्रो व्या॒घ्रो वयो ऽना॑धृष्ट॒ मना॑धृष्टं॒ ॅवयो᳚ व्या॒घ्रो व्या॒घ्रो वयो ऽना॑धृष्टम् । \newline
65. वयो ऽना॑धृष्ट॒ मना॑धृष्टं॒ ॅवयो॒ वयो ऽना॑धृष्ट॒म् छन्द॒ श्छन्दो ऽना॑धृष्टं॒ ॅवयो॒ वयो ऽना॑धृष्ट॒म् छन्दः॑ । \newline
66. अना॑धृष्ट॒म् छन्द॒ श्छन्दो ऽना॑धृष्ट॒ मना॑धृष्ट॒म् छन्दः॑ सिꣳ॒॒हः सिꣳ॒॒ह श्छन्दो ऽना॑धृष्ट॒ मना॑धृष्ट॒म् छन्दः॑ सिꣳ॒॒हः । \newline
67. अना॑धृष्ट॒मित्यना᳚ - धृ॒ष्ट॒म् । \newline
68. छन्दः॑ सिꣳ॒॒हः सिꣳ॒॒ह श्छन्द॒ श्छन्दः॑ सिꣳ॒॒हो वयो॒ वयः॑ सिꣳ॒॒ह श्छन्द॒ श्छन्दः॑ सिꣳ॒॒हो वयः॑ । \newline
69. सिꣳ॒॒हो वयो॒ वयः॑ सिꣳ॒॒हः सिꣳ॒॒हो वय॑ श्छ॒दि श्छ॒दिर् वयः॑ सिꣳ॒॒हः सिꣳ॒॒हो वय॑ श्छ॒दिः । \newline
70. वय॑ श्छ॒दि श्छ॒दिर् वयो॒ वय॑ श्छ॒दि श्छन्द॒ श्छन्द॑ श्छ॒दिर् वयो॒ वय॑ श्छ॒दि श्छन्दः॑ । \newline
71. छ॒दि श्छन्द॒ श्छन्द॑ श्छ॒दि श्छ॒दि श्छन्दो॑ विष्टं॒भो वि॑ष्टं॒भ श्छन्द॑ श्छ॒दि श्छ॒दि श्छन्दो॑ विष्टं॒भः । \newline
72. छन्दो॑ विष्टं॒भो वि॑ष्टं॒भ श्छन्द॒ श्छन्दो॑ विष्टं॒भो वयो॒ वयो॑ विष्टं॒भ श्छन्द॒ श्छन्दो॑ विष्टं॒भो वयः॑ । \newline
73. वि॒ष्टं॒भो वयो॒ वयो॑ विष्टं॒भो वि॑ष्टं॒भो वयो ऽधि॑पति॒ रधि॑पति॒र् वयो॑ विष्टं॒भो वि॑ष्टं॒भो वयो ऽधि॑पतिः । \newline
74. वि॒ष्टं॒भ इति॑ वि - स्तं॒भः । \newline
75. वयो ऽधि॑पति॒ रधि॑पति॒र् वयो॒ वयो ऽधि॑पति॒ श्छन्द॒ श्छन्दो ऽधि॑पति॒र् वयो॒ वयो ऽधि॑पति॒ श्छन्दः॑ । \newline
76. अधि॑पति॒ श्छन्द॒ श्छन्दो ऽधि॑पति॒ रधि॑पति॒ श्छन्दः॑ क्ष॒त्रम् क्ष॒त्रम् छन्दो ऽधि॑पति॒ रधि॑पति॒ श्छन्दः॑ क्ष॒त्रम् । \newline
77. अधि॑पति॒रित्यधि॑ - प॒तिः॒ । \newline
78. छन्दः॑ क्ष॒त्रम् क्ष॒त्रम् छन्द॒ श्छन्दः॑ क्ष॒त्रं ॅवयो॒ वयः॑ क्ष॒त्रम् छन्द॒ श्छन्दः॑ क्ष॒त्रं ॅवयः॑ । \newline
79. क्ष॒त्रं ॅवयो॒ वयः॑ क्ष॒त्रम् क्ष॒त्रं ॅवयो॒ मय॑न्द॒म् मय॑न्दं॒ ॅवयः॑ क्ष॒त्रम् क्ष॒त्रं ॅवयो॒ मय॑न्दम् । \newline
80. वयो॒ मय॑न्द॒म् मय॑न्दं॒ ॅवयो॒ वयो॒ मय॑न्द॒म् छन्द॒ श्छन्दो॒ मय॑न्दं॒ ॅवयो॒ वयो॒ मय॑न्द॒म् छन्दः॑ । \newline
81. मय॑न्द॒म् छन्द॒ श्छन्दो॒ मय॑न्द॒म् मय॑न्द॒म् छन्दो॑ वि॒श्वक॑र्मा वि॒श्वक॑र्मा॒ छन्दो॒ मय॑न्द॒म् मय॑न्द॒म् छन्दो॑ वि॒श्वक॑र्मा । \newline
82. छन्दो॑ वि॒श्वक॑र्मा वि॒श्वक॑र्मा॒ छन्द॒ श्छन्दो॑ वि॒श्वक॑र्मा॒ वयो॒ वयो॑ वि॒श्वक॑र्मा॒ छन्द॒ श्छन्दो॑ वि॒श्वक॑र्मा॒ वयः॑ । \newline
83. वि॒श्वक॑र्मा॒ वयो॒ वयो॑ वि॒श्वक॑र्मा वि॒श्वक॑र्मा॒ वयः॑ परमे॒ष्ठी प॑रमे॒ष्ठी वयो॑ वि॒श्वक॑र्मा वि॒श्वक॑र्मा॒ वयः॑ परमे॒ष्ठी । \newline
84. वि॒श्वक॒र्मेति॑ वि॒श्व - क॒र्मा॒ । \newline
85. वयः॑ परमे॒ष्ठी प॑रमे॒ष्ठी वयो॒ वयः॑ परमे॒ष्ठी छन्द॒ श्छन्दः॑ परमे॒ष्ठी वयो॒ वयः॑ परमे॒ष्ठी छन्दः॑ । \newline
86. प॒र॒मे॒ष्ठी छन्द॒ श्छन्दः॑ परमे॒ष्ठी प॑रमे॒ष्ठी छन्दो॑ मू॒र्द्धा मू॒र्द्धा छन्दः॑ परमे॒ष्ठी प॑रमे॒ष्ठी छन्दो॑ मू॒र्द्धा । \newline
87. छन्दो॑ मू॒र्द्धा मू॒र्द्धा छन्द॒ श्छन्दो॑ मू॒र्द्धा वयो॒ वयो॑ मू॒र्द्धा छन्द॒ श्छन्दो॑ मू॒र्द्धा वयः॑ । \newline
88. मू॒र्द्धा वयो॒ वयो॑ मू॒र्द्धा मू॒र्द्धा वयः॑ प्र॒जाप॑तिः प्र॒जाप॑ति॒र् वयो॑ मू॒र्द्धा मू॒र्द्धा वयः॑ प्र॒जाप॑तिः । \newline
89. वयः॑ प्र॒जाप॑तिः प्र॒जाप॑ति॒र् वयो॒ वयः॑ प्र॒जाप॑ति॒ श्छन्द॒ श्छन्दः॑ प्र॒जाप॑ति॒र् वयो॒ वयः॑ प्र॒जाप॑ति॒ श्छन्दः॑ । \newline
90. प्र॒जाप॑ति॒ श्छन्द॒ श्छन्दः॑ प्र॒जाप॑तिः प्र॒जाप॑ति॒ श्छन्दः॑ । \newline
91. प्र॒जाप॑ति॒रिति॑ प्र॒जा - प॒तिः॒ । \newline
92. छन्द॒ इति॒ छन्दः॑ । \newline
\pagebreak
\markright{ TS 4.3.6.1  \hfill https://www.vedavms.in \hfill}

\section{ TS 4.3.6.1 }

\textbf{TS 4.3.6.1 } \newline
\textbf{Samhita Paata} \newline

इन्द्रा᳚ग्नी॒ अव्य॑थमाना॒मिष्ट॑कां दृꣳहतं ॅयु॒वं । पृ॒ष्ठेन॒ द्यावा॑पृथि॒वी अ॒न्तरि॑क्षं च॒ वि बा॑धतां ॥ वि॒श्वक॑र्मा त्वा सादयत्व॒न्तरि॑क्षस्य पृ॒ष्ठे व्यच॑स्वतीं॒ प्रथ॑स्वतीं॒ भास्व॑तीꣳ सूरि॒मती॒मा या द्यां भास्या पृ॑थि॒वीमोर्व॑न्तरि॑क्ष-म॒न्तरि॑क्षं ॅयच्छा॒न्तरि॑क्षं दृꣳहा॒न्तरि॑क्षं॒ मा हिꣳ॑सी॒ र्विश्व॑स्मै प्रा॒णाया॑पा॒नाय॑ व्या॒नायो॑दा॒नाय॑ प्रति॒ष्ठायै॑ च॒रित्रा॑य वा॒युस्त्वा॒ऽभि पा॑तु म॒ह्या स्व॒स्त्या छ॒र्दिषा॒ - [  ] \newline

\textbf{Pada Paata} \newline

इन्द्रा᳚ग्नी॒ इतीन्द्र॑ - अ॒ग्नी॒ । अव्य॑थमानाम् । इष्ट॑काम् । दृꣳ॒॒ह॒त॒म् । यु॒वम् ॥ पृ॒ष्ठेन॑ । द्यावा॑पृथि॒वी इति॒ द्यावा᳚ -पृ॒थि॒वी । अ॒न्तरि॑क्षम् । च॒ । वीति॑ । बा॒ध॒ता॒म् ॥ वि॒श्वक॒र्मेति॑ वि॒श्व - क॒र्मा॒ । त्वा॒ । सा॒द॒य॒तु॒ । अ॒न्तरि॑क्षस्य । पृ॒ष्ठे । व्यच॑स्वतीम् । प्रथ॑स्वतीम् । भास्व॑तीम् । सू॒रि॒मती॒मिति॑ सूरि - मती᳚म् । एति॑ । या । द्याम् । भासि॑ । एति॑ । पृ॒थि॒वीम् । एति॑ । उ॒रु । अ॒न्तरि॑क्षम् । अ॒न्तरि॑क्षम् । य॒च्छ॒ । अ॒न्तरि॑क्षम् । दृꣳ॒॒ह॒ । अ॒न्तरि॑क्षम् । मा । हिꣳ॒॒सीः॒ । विश्व॑स्मै । प्रा॒णायेति॑ प्रा - अ॒नाय॑ । अ॒पा॒नायेत्य॑प - अ॒नाय॑ । व्या॒नायेति॑ वि - अ॒नाय॑ । उ॒दा॒नायेत्यु॑त् - अ॒नाय॑ । प्र॒ति॒ष्ठाया॒ इति॑ प्रति - स्थायै᳚ । च॒रित्रा॑य । वा॒युः । त्वा॒ । अ॒भीति॑ । पा॒तु॒ । म॒ह्या । स्व॒स्त्या । छ॒र्दिषा᳚ ।  \newline


\textbf{Krama Paata} \newline

इन्द्रा᳚ग्नी॒ अव्य॑थमानाम् । इन्द्रा᳚ग्नी॒ इतीन्द्र॑ - अ॒ग्नी॒ । अव्य॑थमाना॒मिष्ट॑काम् । इष्ट॑काम् दृꣳहतम् । दृꣳ॒॒ह॒तं॒ ॅयु॒वम् । यु॒वमिति॑ यु॒वम् ॥ पृ॒ष्ठेन॒ द्यावा॑पृथि॒वी । द्यावा॑पृथि॒वी अ॒न्तरि॑क्षम् । द्यावा॑पृथि॒वी इति॒ द्यावा᳚ - पृ॒थि॒वी । अ॒न्तरि॑क्षम् च । च॒ वि । वि बा॑धताम् । बा॒ध॒ता॒मिति॑ बाधताम् ॥ वि॒श्वक॑र्मा त्वा । वि॒श्वक॒र्मेति॑ वि॒श्व - क॒र्मा॒ । त्वा॒ सा॒द॒य॒तु॒ । सा॒द॒य॒त्व॒न्तरि॑क्षस्य । अ॒न्तरि॑क्षस्य पृ॒ष्ठे । पृ॒ष्ठे व्यच॑स्वतीम् । व्यच॑स्वती॒म् प्रथ॑स्वतीम् । प्रथ॑स्वती॒म् भास्व॑तीम् । भास्व॑तीꣳ सूरि॒मती᳚म् । सू॒रि॒मती॒मा । सू॒रि॒मती॒मिति॑ सूरि - मती᳚म् । आ या । या द्याम् । द्याम् भासि॑ । भास्या । आ पृ॑थि॒वीम् । पृ॒थि॒वीमा । ओरु । उ॒र्व॑न्तरि॑क्षम् । अ॒न्तरि॑क्षम॒न्तरि॑क्षम् । अ॒न्तरि॑क्षं ॅयच्छ । यच्छा॒न्तरि॑क्षम् । अ॒न्तरि॑क्षम् दृꣳह । दृꣳ॒॒हा॒न्तरि॑क्षम् । अ॒न्तरि॑क्ष॒म् मा । मा हिꣳ॑सीः । हिꣳ॒॒सी॒र् विश्व॑स्मै । विश्व॑स्मै प्रा॒णाय॑ । प्रा॒णाया॑पा॒नाय॑ । प्रा॒णायेति॑ प्र - अ॒नाय॑ । अ॒पा॒नाय॑ व्या॒नाय॑ । अ॒पा॒नायेत्य॑प - अ॒नाय॑ । व्या॒नायो॑दा॒नाय॑ । व्या॒नायेति॑ वि - अ॒नाय॑ । उ॒दा॒नाय॑ प्रति॒ष्ठायै᳚ । उ॒दा॒नायेत्यु॑त् - अ॒नाय॑ । प्र॒ति॒ष्ठायै॑ च॒रित्रा॑य । प्र॒ति॒ष्ठाया॒ इति॑ प्रति - स्थायै᳚ । च॒रित्रा॑य वा॒युः । वा॒युस्त्वा᳚ । त्वा॒ऽभि । अ॒भि पा॑तु । पा॒तु॒ म॒ह्या । म॒ह्या स्व॒स्त्या । स्व॒स्त्या छ॒र्दिषा᳚ । छ॒र्दिषा॒ शन्त॑मेन \newline

\textbf{Jatai Paata} \newline

1. इन्द्रा᳚ग्नी॒ अव्य॑थमाना॒ मव्य॑थमाना॒ मिन्द्रा᳚ग्नी॒ इन्द्रा᳚ग्नी॒ अव्य॑थमानाम् । \newline
2. इन्द्रा᳚ग्नी॒ इतीन्द्र॑ - अ॒ग्नी॒ । \newline
3. अव्य॑थमाना॒ मिष्ट॑का॒ मिष्ट॑का॒ मव्य॑थमाना॒ मव्य॑थमाना॒ मिष्ट॑काम् । \newline
4. इष्ट॑काम् दृꣳहतम् दृꣳहत॒ मिष्ट॑का॒ मिष्ट॑काम् दृꣳहतम् । \newline
5. दृꣳ॒॒ह॒तं॒ ॅयु॒वं ॅयु॒वम् दृꣳ॑हतम् दृꣳहतं ॅयु॒वम् । \newline
6. यु॒वमिति॑ यु॒वम् । \newline
7. पृ॒ष्ठेन॒ द्यावा॑पृथि॒वी द्यावा॑पृथि॒वी पृ॒ष्ठेन॑ पृ॒ष्ठेन॒ द्यावा॑पृथि॒वी । \newline
8. द्यावा॑पृथि॒वी अ॒न्तरि॑क्ष म॒न्तरि॑क्ष॒म् द्यावा॑पृथि॒वी द्यावा॑पृथि॒वी अ॒न्तरि॑क्षम् । \newline
9. द्यावा॑पृथि॒वी इति॒ द्यावा᳚ - पृ॒थि॒वी । \newline
10. अ॒न्तरि॑क्षम् च चा॒न्तरि॑क्ष म॒न्तरि॑क्षम् च । \newline
11. च॒ वि वि च॑ च॒ वि । \newline
12. वि बा॑धताम् बाधतां॒ ॅवि वि बा॑धताम् । \newline
13. बा॒ध॒ता॒मिति॑ बाधताम् । \newline
14. वि॒श्वक॑र्मा त्वा त्वा वि॒श्वक॑र्मा वि॒श्वक॑र्मा त्वा । \newline
15. वि॒श्वक॒र्मेति॑ वि॒श्व - क॒र्मा॒ । \newline
16. त्वा॒ सा॒द॒य॒तु॒ सा॒द॒य॒तु॒ त्वा॒ त्वा॒ सा॒द॒य॒तु॒ । \newline
17. सा॒द॒य॒ त्व॒न्तरि॑क्षस्या॒ न्तरि॑क्षस्य सादयतु सादय त्व॒न्तरि॑क्षस्य । \newline
18. अ॒न्तरि॑क्षस्य पृ॒ष्ठे पृ॒ष्ठे अ॒न्तरि॑क्षस्या॒ न्तरि॑क्षस्य पृ॒ष्ठे । \newline
19. पृ॒ष्ठे व्यच॑स्वतीं॒ ॅव्यच॑स्वतीम् पृ॒ष्ठे पृ॒ष्ठे व्यच॑स्वतीम् । \newline
20. व्यच॑स्वती॒म् प्रथ॑स्वती॒म् प्रथ॑स्वतीं॒ ॅव्यच॑स्वतीं॒ ॅव्यच॑स्वती॒म् प्रथ॑स्वतीम् । \newline
21. प्रथ॑स्वती॒म् भास्व॑ती॒म् भास्व॑ती॒म् प्रथ॑स्वती॒म् प्रथ॑स्वती॒म् भास्व॑तीम् । \newline
22. भास्व॑तीꣳ सूरि॒मतीꣳ॑ सूरि॒मती॒म् भास्व॑ती॒म् भास्व॑तीꣳ सूरि॒मती᳚म् । \newline
23. सू॒रि॒मती॒ मा सू॑रि॒मतीꣳ॑ सूरि॒मती॒ मा । \newline
24. सू॒रि॒मती॒मिति॑ सूरि - मती᳚म् । \newline
25. आ या या ऽऽया । \newline
26. या द्याम् द्यां ॅया या द्याम् । \newline
27. द्याम् भासि॒ भासि॒ द्याम् द्याम् भासि॑ । \newline
28. भास्या भासि॒ भास्या । \newline
29. आ पृ॑थि॒वीम् पृ॑थि॒वी मा पृ॑थि॒वीम् । \newline
30. पृ॒थि॒वी मा पृ॑थि॒वीम् पृ॑थि॒वी मा । \newline
31. ओरू᳚र् वोरु । \newline
32. उ॒र्व॑न्तरि॑क्ष म॒न्तरि॑क्ष मु॒रू᳚(1॒)र्व॑न्तरि॑क्षम् । \newline
33. अ॒न्तरि॑क्ष म॒न्तरि॑क्षम् । \newline
34. अ॒न्तरि॑क्षं ॅयच्छ यच्छा॒न्तरि॑क्ष म॒न्तरि॑क्षं ॅयच्छ । \newline
35. य॒च्छा॒ न्तरि॑क्ष म॒न्तरि॑क्षं ॅयच्छ यच्छा॒ न्तरि॑क्षम् । \newline
36. अ॒न्तरि॑क्षम् दृꣳह दृꣳहा॒ न्तरि॑क्ष म॒न्तरि॑क्षम् दृꣳह । \newline
37. दृꣳ॒॒हा॒ न्तरि॑क्ष म॒न्तरि॑क्षम् दृꣳह दृꣳहा॒ न्तरि॑क्षम् । \newline
38. अ॒न्तरि॑क्ष॒म् मा मा ऽन्तरि॑क्ष म॒न्तरि॑क्ष॒म् मा । \newline
39. मा हिꣳ॑सीर्. हिꣳसी॒र् मा मा हिꣳ॑सीः । \newline
40. हिꣳ॒॒सी॒र् विश्व॑स्मै॒ विश्व॑स्मै हिꣳसीर्. हिꣳसी॒र् विश्व॑स्मै । \newline
41. विश्व॑स्मै प्रा॒णाय॑ प्रा॒णाय॒ विश्व॑स्मै॒ विश्व॑स्मै प्रा॒णाय॑ । \newline
42. प्रा॒णाया॑ पा॒नाया॑ पा॒नाय॑ प्रा॒णाय॑ प्रा॒णाया॑ पा॒नाय॑ । \newline
43. प्रा॒णायेति॑ प्र - अ॒नाय॑ । \newline
44. अ॒पा॒नाय॑ व्या॒नाय॑ व्या॒नाया॑ पा॒नाया॑ पा॒नाय॑ व्या॒नाय॑ । \newline
45. अ॒पा॒नायेत्य॑प - अ॒नाय॑ । \newline
46. व्या॒नायो॑ दा॒नायो॑ दा॒नाय॑ व्या॒नाय॑ व्या॒नायो॑ दा॒नाय॑ । \newline
47. व्या॒नायेति॑ वि - अ॒नाय॑ । \newline
48. उ॒दा॒नाय॑ प्रति॒ष्ठायै᳚ प्रति॒ष्ठाया॑ उदा॒नायो॑ दा॒नाय॑ प्रति॒ष्ठायै᳚ । \newline
49. उ॒दा॒नायेत्यु॑त् - अ॒नाय॑ । \newline
50. प्र॒ति॒ष्ठायै॑ च॒रित्रा॑य च॒रित्रा॑य प्रति॒ष्ठायै᳚ प्रति॒ष्ठायै॑ च॒रित्रा॑य । \newline
51. प्र॒ति॒ष्ठाया॒ इति॑ प्रति - स्थायै᳚ । \newline
52. च॒रित्रा॑य वा॒युर् वा॒यु श्च॒रित्रा॑य च॒रित्रा॑य वा॒युः । \newline
53. वा॒यु स्त्वा᳚ त्वा वा॒युर् वा॒यु स्त्वा᳚ । \newline
54. त्वा॒ ऽभ्य॑भि त्वा᳚ त्वा॒ ऽभि । \newline
55. अ॒भि पा॑तु पात्व॒ भ्य॑भि पा॑तु । \newline
56. पा॒तु॒ म॒ह्या म॒ह्या पा॑तु पातु म॒ह्या । \newline
57. म॒ह्या स्व॒स्त्या स्व॒स्त्या म॒ह्या म॒ह्या स्व॒स्त्या । \newline
58. स्व॒स्त्या छ॒र्दिषा॑ छ॒र्दिषा᳚ स्व॒स्त्या स्व॒स्त्या छ॒र्दिषा᳚ । \newline
59. छ॒र्दिषा॒ शन्त॑मेन॒ शन्त॑मेन छ॒र्दिषा॑ छ॒र्दिषा॒ शन्त॑मेन । \newline

\textbf{Ghana Paata } \newline

1. इन्द्रा᳚ग्नी॒ अव्य॑थमाना॒ मव्य॑थमाना॒ मिन्द्रा᳚ग्नी॒ इन्द्रा᳚ग्नी॒ अव्य॑थमाना॒ मिष्ट॑का॒ मिष्ट॑का॒ मव्य॑थमाना॒ मिन्द्रा᳚ग्नी॒ इन्द्रा᳚ग्नी॒ अव्य॑थमाना॒ मिष्ट॑काम् । \newline
2. इन्द्रा᳚ग्नी॒ इतीन्द्र॑ - अ॒ग्नी॒ । \newline
3. अव्य॑थमाना॒ मिष्ट॑का॒ मिष्ट॑का॒ मव्य॑थमाना॒ मव्य॑थमाना॒ मिष्ट॑काम् दृꣳहतम् दृꣳहत॒ मिष्ट॑का॒ मव्य॑थमाना॒ मव्य॑थमाना॒ मिष्ट॑काम् दृꣳहतम् । \newline
4. इष्ट॑काम् दृꣳहतम् दृꣳहत॒ मिष्ट॑का॒ मिष्ट॑काम् दृꣳहतं ॅयु॒वं ॅयु॒वम् दृꣳ॑हत॒ मिष्ट॑का॒ मिष्ट॑काम् दृꣳहतं ॅयु॒वम् । \newline
5. दृꣳ॒॒ह॒तं॒ ॅयु॒वं ॅयु॒वम् दृꣳ॑हतम् दृꣳहतं ॅयु॒वम् । \newline
6. यु॒वमिति॑ यु॒वम् । \newline
7. पृ॒ष्ठेन॒ द्यावा॑पृथि॒वी द्यावा॑पृथि॒वी पृ॒ष्ठेन॑ पृ॒ष्ठेन॒ द्यावा॑पृथि॒वी अ॒न्तरि॑क्ष म॒न्तरि॑क्ष॒म् द्यावा॑पृथि॒वी पृ॒ष्ठेन॑ पृ॒ष्ठेन॒ द्यावा॑पृथि॒वी अ॒न्तरि॑क्षम् । \newline
8. द्यावा॑पृथि॒वी अ॒न्तरि॑क्ष म॒न्तरि॑क्ष॒म् द्यावा॑पृथि॒वी द्यावा॑पृथि॒वी अ॒न्तरि॑क्षम् च चा॒न्तरि॑क्ष॒म् द्यावा॑पृथि॒वी द्यावा॑पृथि॒वी अ॒न्तरि॑क्षम् च । \newline
9. द्यावा॑पृथि॒वी इति॒ द्यावा᳚ - पृ॒थि॒वी । \newline
10. अ॒न्तरि॑क्षम् च चा॒न्तरि॑क्ष म॒न्तरि॑क्षम् च॒ वि वि चा॒न्तरि॑क्ष म॒न्तरि॑क्षम् च॒ वि । \newline
11. च॒ वि वि च॑ च॒ वि बा॑धताम् बाधतां॒ ॅवि च॑ च॒ वि बा॑धताम् । \newline
12. वि बा॑धताम् बाधतां॒ ॅवि वि बा॑धताम् । \newline
13. बा॒ध॒ता॒मिति॑ बाधताम् । \newline
14. वि॒श्वक॑र्मा त्वा त्वा वि॒श्वक॑र्मा वि॒श्वक॑र्मा त्वा सादयतु सादयतु त्वा वि॒श्वक॑र्मा वि॒श्वक॑र्मा त्वा सादयतु । \newline
15. वि॒श्वक॒र्मेति॑ वि॒श्व - क॒र्मा॒ । \newline
16. त्वा॒ सा॒द॒य॒तु॒ सा॒द॒य॒तु॒ त्वा॒ त्वा॒ सा॒द॒य॒ त्व॒न्तरि॑क्षस्या॒ न्तरि॑क्षस्य सादयतु त्वा त्वा सादय त्व॒न्तरि॑क्षस्य । \newline
17. सा॒द॒य॒ त्व॒न्तरि॑क्षस्या॒ न्तरि॑क्षस्य सादयतु सादय त्व॒न्तरि॑क्षस्य पृ॒ष्ठे पृ॒ष्ठे अ॒न्तरि॑क्षस्य सादयतु सादय त्व॒न्तरि॑क्षस्य पृ॒ष्ठे । \newline
18. अ॒न्तरि॑क्षस्य पृ॒ष्ठे पृ॒ष्ठे अ॒न्तरि॑क्षस्या॒ न्तरि॑क्षस्य पृ॒ष्ठे व्यच॑स्वतीं॒ ॅव्यच॑स्वतीम् पृ॒ष्ठे अ॒न्तरि॑क्षस्या॒ न्तरि॑क्षस्य पृ॒ष्ठे व्यच॑स्वतीम् । \newline
19. पृ॒ष्ठे व्यच॑स्वतीं॒ ॅव्यच॑स्वतीम् पृ॒ष्ठे पृ॒ष्ठे व्यच॑स्वती॒म् प्रथ॑स्वती॒म् प्रथ॑स्वतीं॒ ॅव्यच॑स्वतीम् पृ॒ष्ठे पृ॒ष्ठे व्यच॑स्वती॒म् प्रथ॑स्वतीम् । \newline
20. व्यच॑स्वती॒म् प्रथ॑स्वती॒म् प्रथ॑स्वतीं॒ ॅव्यच॑स्वतीं॒ ॅव्यच॑स्वती॒म् प्रथ॑स्वती॒म् भास्व॑ती॒म् भास्व॑ती॒म् प्रथ॑स्वतीं॒ ॅव्यच॑स्वतीं॒ ॅव्यच॑स्वती॒म् प्रथ॑स्वती॒म् भास्व॑तीम् । \newline
21. प्रथ॑स्वती॒म् भास्व॑ती॒म् भास्व॑ती॒म् प्रथ॑स्वती॒म् प्रथ॑स्वती॒म् भास्व॑तीꣳ सूरि॒मतीꣳ॑ सूरि॒मती॒म् भास्व॑ती॒म् प्रथ॑स्वती॒म् प्रथ॑स्वती॒म् भास्व॑तीꣳ सूरि॒मती᳚म् । \newline
22. भास्व॑तीꣳ सूरि॒मतीꣳ॑ सूरि॒मती॒म् भास्व॑ती॒म् भास्व॑तीꣳ सूरि॒मती॒ मा सू॑रि॒मती॒म् भास्व॑ती॒म् भास्व॑तीꣳ सूरि॒मती॒ मा । \newline
23. सू॒रि॒मती॒ मा सू॑रि॒मतीꣳ॑ सूरि॒मती॒ मा या या ऽऽसू॑रि॒मतीꣳ॑ सूरि॒मती॒ मा या । \newline
24. सू॒रि॒मती॒मिति॑ सूरि - मती᳚म् । \newline
25. आ या या ऽऽया द्याम् द्यां ॅया ऽऽया द्याम् । \newline
26. या द्याम् द्यां ॅया या द्याम् भासि॒ भासि॒ द्यां ॅया या द्याम् भासि॑ । \newline
27. द्याम् भासि॒ भासि॒ द्याम् द्याम् भास्या भासि॒ द्याम् द्याम् भास्या । \newline
28. भास्या भासि॒ भास्या पृ॑थि॒वीम् पृ॑थि॒वी मा भासि॒ भास्या पृ॑थि॒वीम् । \newline
29. आ पृ॑थि॒वीम् पृ॑थि॒वी मा पृ॑थि॒वी मा पृ॑थि॒वी मा पृ॑थि॒वी मा । \newline
30. पृ॒थि॒वी मा पृ॑थि॒वीम् पृ॑थि॒वी मोरू᳚र्वा पृ॑थि॒वीम् पृ॑थि॒वी मोरु । \newline
31. ओरू᳚र्वो र्व॑न्तरि॑क्ष म॒न्तरि॑क्ष मु॒र्वो र्व॑न्तरि॑क्षम् । \newline
32. उ॒र्व॑न्तरि॑क्ष म॒न्तरि॑क्ष मु॒रू᳚(1॒)र्व॑न्तरि॑क्षम् । \newline
33. अ॒न्तरि॑क्ष म॒न्तरि॑क्षम् । \newline
34. अ॒न्तरि॑क्षं ॅयच्छ यच्छा॒न्तरि॑क्ष म॒न्तरि॑क्षं ॅयच्छा॒न्तरि॑क्ष म॒न्तरि॑क्षं ॅयच्छा॒न्तरि॑क्ष म॒न्तरि॑क्षं ॅयच्छा॒न्तरि॑क्षम् । \newline
35. य॒च्छा॒न्तरि॑क्ष म॒न्तरि॑क्षं ॅयच्छ यच्छा॒न्तरि॑क्षम् दृꣳह दृꣳहा॒न्तरि॑क्षं ॅयच्छ यच्छा॒न्तरि॑क्षम् दृꣳह । \newline
36. अ॒न्तरि॑क्षम् दृꣳह दृꣳहा॒न्तरि॑क्ष म॒न्तरि॑क्षम् दृꣳहा॒न्तरि॑क्ष म॒न्तरि॑क्षम् दृꣳहा॒न्तरि॑क्ष म॒न्तरि॑क्षम् दृꣳहा॒न्तरि॑क्षम् । \newline
37. दृꣳ॒॒हा॒न्तरि॑क्ष म॒न्तरि॑क्षम् दृꣳह दृꣳहा॒न्तरि॑क्ष॒म् मा मा ऽन्तरि॑क्षम् दृꣳह दृꣳहा॒न्तरि॑क्ष॒म् मा । \newline
38. अ॒न्तरि॑क्ष॒म् मा मा ऽन्तरि॑क्ष म॒न्तरि॑क्ष॒म् मा हिꣳ॑सीर्. हिꣳसी॒र् मा ऽन्तरि॑क्ष म॒न्तरि॑क्ष॒म् मा हिꣳ॑सीः । \newline
39. मा हिꣳ॑सीर्. हिꣳसी॒र् मा मा हिꣳ॑सी॒र् विश्व॑स्मै॒ विश्व॑स्मै हिꣳसी॒र् मा मा हिꣳ॑सी॒र् विश्व॑स्मै । \newline
40. हिꣳ॒॒सी॒र् विश्व॑स्मै॒ विश्व॑स्मै हिꣳसीर्. हिꣳसी॒र् विश्व॑स्मै प्रा॒णाय॑ प्रा॒णाय॒ विश्व॑स्मै हिꣳसीर्. हिꣳसी॒र् 
विश्व॑स्मै प्रा॒णाय॑ । \newline
41. विश्व॑स्मै प्रा॒णाय॑ प्रा॒णाय॒ विश्व॑स्मै॒ विश्व॑स्मै प्रा॒णाया॑ पा॒नाया॑ पा॒नाय॑ प्रा॒णाय॒ विश्व॑स्मै॒ विश्व॑स्मै प्रा॒णाया॑ पा॒नाय॑ । \newline
42. प्रा॒णाया॑ पा॒नाया॑ पा॒नाय॑ प्रा॒णाय॑ प्रा॒णाया॑ पा॒नाय॑ व्या॒नाय॑ व्या॒नाया॑ पा॒नाय॑ प्रा॒णाय॑ प्रा॒णाया॑ पा॒नाय॑ व्या॒नाय॑ । \newline
43. प्रा॒णायेति॑ प्र - अ॒नाय॑ । \newline
44. अ॒पा॒नाय॑ व्या॒नाय॑ व्या॒नाया॑ पा॒नाया॑ पा॒नाय॑ व्या॒नायो॑ दा॒नायो॑ दा॒नाय॑ व्या॒नाया॑ पा॒नाया॑ पा॒नाय॑ व्या॒नायो॑ दा॒नाय॑ । \newline
45. अ॒पा॒नायेत्य॑प - अ॒नाय॑ । \newline
46. व्या॒नायो॑ दा॒नायो॑ दा॒नाय॑ व्या॒नाय॑ व्या॒नायो॑ दा॒नाय॑ प्रति॒ष्ठायै᳚ प्रति॒ष्ठाया॑ उदा॒नाय॑ व्या॒नाय॑ व्या॒नायो॑ दा॒नाय॑ प्रति॒ष्ठायै᳚ । \newline
47. व्या॒नायेति॑ वि - अ॒नाय॑ । \newline
48. उ॒दा॒नाय॑ प्रति॒ष्ठायै᳚ प्रति॒ष्ठाया॑ उदा॒नायो॑ दा॒नाय॑ प्रति॒ष्ठायै॑ च॒रित्रा॑य च॒रित्रा॑य प्रति॒ष्ठाया॑ उदा॒नायो॑ दा॒नाय॑ प्रति॒ष्ठायै॑ च॒रित्रा॑य । \newline
49. उ॒दा॒नायेत्यु॑त् - अ॒नाय॑ । \newline
50. प्र॒ति॒ष्ठायै॑ च॒रित्रा॑य च॒रित्रा॑य प्रति॒ष्ठायै᳚ प्रति॒ष्ठायै॑ च॒रित्रा॑य वा॒युर् वा॒यु श्च॒रित्रा॑य प्रति॒ष्ठायै᳚ प्रति॒ष्ठायै॑ च॒रित्रा॑य वा॒युः । \newline
51. प्र॒ति॒ष्ठाया॒ इति॑ प्रति - स्थायै᳚ । \newline
52. च॒रित्रा॑य वा॒युर् वा॒यु श्च॒रित्रा॑य च॒रित्रा॑य वा॒यु स्त्वा᳚ त्वा वा॒यु श्च॒रित्रा॑य च॒रित्रा॑य वा॒यु स्त्वा᳚ । \newline
53. वा॒यु स्त्वा᳚ त्वा वा॒युर् वा॒यु स्त्वा॒ ऽभ्य॑भि त्वा॑ वा॒युर् वा॒यु स्त्वा॒ ऽभि । \newline
54. त्वा॒ ऽभ्य॑भि त्वा᳚ त्वा॒ ऽभि पा॑तु पात्व॒भि त्वा᳚ त्वा॒ ऽभि पा॑तु । \newline
55. अ॒भि पा॑तु पात्व॒ भ्य॑भि पा॑तु म॒ह्या म॒ह्या पा᳚त्व॒ भ्य॑भि पा॑तु म॒ह्या । \newline
56. पा॒तु॒ म॒ह्या म॒ह्या पा॑तु पातु म॒ह्या स्व॒स्त्या स्व॒स्त्या म॒ह्या पा॑तु पातु म॒ह्या स्व॒स्त्या । \newline
57. म॒ह्या स्व॒स्त्या स्व॒स्त्या म॒ह्या म॒ह्या स्व॒स्त्या छ॒र्दिषा॑ छ॒र्दिषा᳚ स्व॒स्त्या म॒ह्या म॒ह्या स्व॒स्त्या छ॒र्दिषा᳚ । \newline
58. स्व॒स्त्या छ॒र्दिषा॑ छ॒र्दिषा᳚ स्व॒स्त्या स्व॒स्त्या छ॒र्दिषा॒ शन्त॑मेन॒ शन्त॑मेन छ॒र्दिषा᳚ स्व॒स्त्या स्व॒स्त्या छ॒र्दिषा॒ शन्त॑मेन । \newline
59. छ॒र्दिषा॒ शन्त॑मेन॒ शन्त॑मेन छ॒र्दिषा॑ छ॒र्दिषा॒ शन्त॑मेन॒ तया॒ तया॒ शन्त॑मेन छ॒र्दिषा॑ छ॒र्दिषा॒ शन्त॑मेन॒ तया᳚ । \newline
\pagebreak
\markright{ TS 4.3.6.2  \hfill https://www.vedavms.in \hfill}

\section{ TS 4.3.6.2 }

\textbf{TS 4.3.6.2 } \newline
\textbf{Samhita Paata} \newline

शन्त॑मेन॒ तया॑ दे॒व॑तयाऽङ्गिर॒स्वद्-ध्रु॒वा सी॑द ॥ राज्ञ्य॑सि॒ प्राची॒ दिग्-वि॒राड॑सि दक्षि॒णा दिख् स॒म्राड॑सि प्र॒तीची॒ दिख्-स्व॒राड॒स्युदी॑ची॒ दिगधि॑पत्न्यसि बृह॒ती दिगायु॑र्मे पाहि प्रा॒णं मे॑ पाह्यपा॒नं मे॑ पाहि व्या॒नं मे॑ पाहि॒ चक्षु॑र्मे पाहि॒ श्रोत्रं॑ मे पाहि॒ मनो॑ मे जिन्व॒ वाचं॑ मे पिन्वा॒ ( ) ऽऽत्मानं॑ मे पाहि॒ ज्योति॑र्मे यच्छ ॥ \newline

\textbf{Pada Paata} \newline

शन्त॑मे॒नेति॒ शं - त॒मे॒न॒ । तया᳚ । दे॒वत॑या । अ॒ङ्गि॒र॒स्वत् । ध्रु॒वा । सी॒द॒ ॥ राज्ञी᳚ । अ॒सि॒ । प्राची᳚ । दिक् । वि॒राडिति॑ वि-राट् । अ॒सि॒ । द॒क्षि॒णा । दिक् । स॒म्राडिति॑ सं - राट् । अ॒सि॒ । प्र॒तीची᳚ । दिक् । स्व॒राडिति॑ स्व - राट् । अ॒सि॒ । उदी॑ची । दिक् । अधि॑प॒त्नीत्यधि॑-प॒त्नी॒ । अ॒सि॒ । बृ॒ह॒ती । दिक् । आयुः॑ । मे॒ । पा॒हि॒ । प्रा॒णमिति॑ प्र - अ॒नम् । मे॒ । पा॒हि॒ । अ॒पा॒नमित्य॑प - अ॒नम् । मे॒ । पा॒हि॒ । व्या॒नमिति॑ वि - अ॒नम् । मे॒ । पा॒हि॒ । चक्षुः॑ । मे॒ । पा॒हि॒ । श्रोत्र᳚म् । मे॒ । पा॒हि॒ । मनः॑ । मे॒ । जि॒न्व॒ । वाच᳚म् । मे॒ । पि॒न्व॒ ( ) । आ॒त्मान᳚म् । मे॒ । पा॒हि॒ । ज्योतिः॑ । मे॒ । य॒च्छ॒ ॥  \newline


\textbf{Krama Paata} \newline

शन्त॑मेन॒ तया᳚ । शन्त॑मे॒नेति॒ शम् - त॒मे॒न॒ । तया॑ दे॒वत॑या । दे॒वत॑याऽङ्गिर॒स्वत् । अ॒ङ्गि॒र॒स्वद् ध्रु॒वा । ध्रु॒वा सी॑द । सी॒देति॑ सीद ॥ राज्ञ्य॑सि । अ॒सि॒ प्राची᳚ । प्राची॒ दिक् । दिग् वि॒राट् । वि॒राड॑सि । वि॒राडिति॑ वि - राट् । अ॒सि॒ द॒क्षि॒णा । द॒क्षि॒णा दिक् । दिख् स॒म्राट् । स॒म्राड॑सि । स॒म्राडिति॑ सम् - राट् । अ॒सि॒ प्र॒तीची᳚ । प्र॒तीची॒ दिक् । दिख् स्व॒राट् । स्व॒राड॑सि । स्व॒राडिति॑ स्व - राट् । अ॒स्युदी॑ची । उदी॑ची॒ दिक् । दिगधि॑पत्नी । अधि॑पत्न्यसि । अधि॑प॒त्नीत्यधि॑ - प॒त्नी॒ । अ॒सि॒ बृ॒ह॒ती । बृ॒ह॒ती दिक् । दिगायुः॑ । आयु॑र् मे । मे॒ पा॒हि॒ । पा॒हि॒ प्रा॒णम् । प्रा॒णम् मे᳚ । प्रा॒णमिति॑ प्र - अ॒नम् । मे॒ पा॒हि॒ । पा॒ह्य॒पा॒नम् । अ॒पा॒नम् मे᳚ । अ॒पा॒नमित्य॑प - अ॒नम् । मे॒ पा॒हि॒ । पा॒हि॒ व्या॒नम् । व्या॒नम् मे᳚ । व्या॒नमिति॑ वि - अ॒नम् । मे॒ पा॒हि॒ । पा॒हि॒ चक्षुः॑ । चर्क्षु॑र् मे । मे॒ पा॒हि॒ । पा॒हि॒ श्रोत्र᳚म् । श्रोत्र॑म् मे । मे॒ पा॒हि॒ । पा॒हि॒ मनः॑ । मनो॑ मे । मे॒ जि॒न्व॒ । जि॒न्व॒ वाच᳚म् । वाच॑म् मे । मे॒ पि॒न्व॒ ( ) । पि॒न्वा॒त्मान᳚म् । आ॒त्मान॑म् मे । मे॒ पा॒हि॒ । पा॒हि॒ ज्योतिः॑ । ज्योति॑र् मे । मे॒ य॒च्छ॒ । य॒च्छेति॑ यच्छ । \newline

\textbf{Jatai Paata} \newline

1. शन्त॑मेन॒ तया॒ तया॒ शन्त॑मेन॒ शन्त॑मेन॒ तया᳚ । \newline
2. शन्त॑मे॒नेति॒ शं - त॒मे॒न॒ । \newline
3. तया॑ दे॒वत॑या दे॒वत॑या॒ तया॒ तया॑ दे॒वत॑या । \newline
4. दे॒वत॑या ऽङ्गिर॒स्व द॑ङ्गिर॒स्वद् दे॒वत॑या दे॒वत॑या ऽङ्गिर॒स्वत् । \newline
5. अ॒ङ्गि॒र॒स्वद् ध्रु॒वा ध्रु॒वा ऽङ्गि॑र॒स्व द॑ङ्गिर॒स्वद् ध्रु॒वा । \newline
6. ध्रु॒वा सी॑द सीद ध्रु॒वा ध्रु॒वा सी॑द । \newline
7. सी॒देति॑ सीद । \newline
8. राज्ञ्य॑स्यसि॒ राज्ञी॒ राज्ञ्य॑सि । \newline
9. अ॒सि॒ प्राची॒ प्राच्य॑ स्यसि॒ प्राची᳚ । \newline
10. प्राची॒ दिग् दिक् प्राची॒ प्राची॒ दिक् । \newline
11. दिग् वि॒राड् वि॒राड् दिग् दिग् वि॒राट् । \newline
12. वि॒रा ड॑स्यसि वि॒राड् वि॒रा ड॑सि । \newline
13. वि॒राडिति॑ वि - राट् । \newline
14. अ॒सि॒ द॒क्षि॒णा द॑क्षि॒णा ऽस्य॑सि दक्षि॒णा । \newline
15. द॒क्षि॒णा दिग् दिग् द॑क्षि॒णा द॑क्षि॒णा दिक् । \newline
16. दिख् स॒म्राट् थ्स॒म्राड् दिग् दिख् स॒म्राट् । \newline
17. स॒म्रा ड॑स्यसि स॒म्राट् थ्स॒म्राड॑सि । \newline
18. स॒म्राडिति॑ सं - राट् । \newline
19. अ॒सि॒ प्र॒तीची᳚ प्र॒तीच्य॑ स्यसि प्र॒तीची᳚ । \newline
20. प्र॒तीची॒ दिग् दिक् प्र॒तीची᳚ प्र॒तीची॒ दिक् । \newline
21. दिख् स्व॒राट् थ् स्व॒राड् दिग् दिख् स्व॒राट् । \newline
22. स्व॒रा ड॑स्यसि स्व॒राट् थ् स्व॒राड॑सि । \newline
23. स्व॒राडिति॑ स्व - राट् । \newline
24. अ॒स्युदी॒ च्युदी᳚ च्यस्य॒ स्युदी॑ची । \newline
25. उदी॑ची॒ दिग् दिगुदी॒ च्युदी॑ची॒ दिक् । \newline
26. दिगधि॑प॒त्न्य धि॑पत्नी॒ दिग् दिगधि॑पत्नी । \newline
27. अधि॑पत्न्य स्य॒स्य धि॑प॒त्न्यधि॑पत्न्यसि । \newline
28. अधि॑प॒त्नीत्यधि॑ - प॒त्नी॒ । \newline
29. अ॒सि॒ बृ॒ह॒ती बृ॑ह॒ त्य॑स्यसि बृह॒ती । \newline
30. बृ॒ह॒ती दिग् दिग् बृ॑ह॒ती बृ॑ह॒ती दिक् । \newline
31. दिग् आयु॒ रायु॒र् दिग् दिगायुः॑ । \newline
32. आयु॑र् मे म॒ आयु॒ रायु॑र् मे । \newline
33. मे॒ पा॒हि॒ पा॒हि॒ मे॒ मे॒ पा॒हि॒ । \newline
34. पा॒हि॒ प्रा॒णम् प्रा॒णम् पा॑हि पाहि प्रा॒णम् । \newline
35. प्रा॒णम् मे॑ मे प्रा॒णम् प्रा॒णम् मे᳚ । \newline
36. प्रा॒णमिति॑ प्र - अ॒नम् । \newline
37. मे॒ पा॒हि॒ पा॒हि॒ मे॒ मे॒ पा॒हि॒ । \newline
38. पा॒ह्य॒ पा॒न म॑पा॒नम् पा॑हि पाह्य पा॒नम् । \newline
39. अ॒पा॒नम् मे॑ मे अपा॒न म॑पा॒नम् मे᳚ । \newline
40. अ॒पा॒नमित्य॑प - अ॒नम् । \newline
41. मे॒ पा॒हि॒ पा॒हि॒ मे॒ मे॒ पा॒हि॒ । \newline
42. पा॒हि॒ व्या॒नं ॅव्या॒नम् पा॑हि पाहि व्या॒नम् । \newline
43. व्या॒नम् मे॑ मे व्या॒नं ॅव्या॒नम् मे᳚ । \newline
44. व्या॒नमिति॑ वि - अ॒नम् । \newline
45. मे॒ पा॒हि॒ पा॒हि॒ मे॒ मे॒ पा॒हि॒ । \newline
46. पा॒हि॒ चक्षु॒ श्चक्षुः॑ पाहि पाहि॒ चक्षुः॑ । \newline
47. चक्षु॑र् मे मे॒ चक्षु॒ श्चक्षु॑र् मे । \newline
48. मे॒ पा॒हि॒ पा॒हि॒ मे॒ मे॒ पा॒हि॒ । \newline
49. पा॒हि॒ श्रोत्रꣳ॒॒ श्रोत्र॑म् पाहि पाहि॒ श्रोत्र᳚म् । \newline
50. श्रोत्र॑म् मे मे॒ श्रोत्रꣳ॒॒ श्रोत्र॑म् मे । \newline
51. मे॒ पा॒हि॒ पा॒हि॒ मे॒ मे॒ पा॒हि॒ । \newline
52. पा॒हि॒ मनो॒ मनः॑ पाहि पाहि॒ मनः॑ । \newline
53. मनो॑ मे मे॒ मनो॒ मनो॑ मे । \newline
54. मे॒ जि॒न्व॒ जि॒न्व॒ मे॒ मे॒ जि॒न्व॒ । \newline
55. जि॒न्व॒ वाचं॒ ॅवाच॑म् जिन्व जिन्व॒ वाच᳚म् । \newline
56. वाच॑म् मे मे॒ वाचं॒ ॅवाच॑म् मे । \newline
57. मे॒ पि॒न्व॒ पि॒न्व॒ मे॒ मे॒ पि॒न्व॒ । \newline
58. पि॒न्वा॒त्मान॑ मा॒त्मान॑म् पिन्व पिन्वा॒त्मान᳚म् । \newline
59. आ॒त्मान॑म् मे म आ॒त्मान॑ मा॒त्मान॑म् मे । \newline
60. मे॒ पा॒हि॒ पा॒हि॒ मे॒ मे॒ पा॒हि॒ । \newline
61. पा॒हि॒ ज्योति॒र् ज्योतिः॑ पाहि पाहि॒ ज्योतिः॑ । \newline
62. ज्योति॑र् मे मे॒ ज्योति॒र् ज्योति॑र् मे । \newline
63. मे॒ य॒च्छ॒ य॒च्छ॒ मे॒ मे॒ य॒च्छ॒ । \newline
64. य॒च्छेति॑ यच्छ । \newline

\textbf{Ghana Paata } \newline

1. शन्त॑मेन॒ तया॒ तया॒ शन्त॑मेन॒ शन्त॑मेन॒ तया॑ दे॒वत॑या दे॒वत॑या॒ तया॒ शन्त॑मेन॒ शन्त॑मेन॒ तया॑ दे॒वत॑या । \newline
2. शन्त॑मे॒नेति॒ शं - त॒मे॒न॒ । \newline
3. तया॑ दे॒वत॑या दे॒वत॑या॒ तया॒ तया॑ दे॒वत॑या ऽङ्गिर॒स्व द॑ङ्गिर॒स्वद् दे॒वत॑या॒ तया॒ तया॑ दे॒वत॑या ऽङ्गिर॒स्वत् । \newline
4. दे॒वत॑या ऽङ्गिर॒स्व द॑ङ्गिर॒स्वद् दे॒वत॑या दे॒वत॑या ऽङ्गिर॒स्वद् ध्रु॒वा ध्रु॒वा ऽङ्गि॑र॒स्वद् दे॒वत॑या दे॒वत॑या ऽङ्गिर॒स्वद् ध्रु॒वा । \newline
5. अ॒ङ्गि॒र॒स्वद् ध्रु॒वा ध्रु॒वा ऽङ्गि॑र॒ स्वद॑ङ्गिर॒स्वद् ध्रु॒वा सी॑द सीद ध्रु॒वा ऽङ्गि॑र॒स्व द॑ङ्गिर॒स्वद् ध्रु॒वा सी॑द । \newline
6. ध्रु॒वा सी॑द सीद ध्रु॒वा ध्रु॒वा सी॑द । \newline
7. सी॒देति॑ सीद । \newline
8. राज्ञ्य॑स्यसि॒ राज्ञी॒ राज्ञ्य॑सि॒ प्राची॒ प्राच्य॑सि॒ राज्ञी॒ राज्ञ्य॑सि॒ प्राची᳚ । \newline
9. अ॒सि॒ प्राची॒ प्राच्य॑ स्यसि॒ प्राची॒ दिग् दिक् प्राच्य॑ स्यसि॒ प्राची॒ दिक् । \newline
10. प्राची॒ दिग् दिक् प्राची॒ प्राची॒ दिग् वि॒राड् वि॒राड् दिक् प्राची॒ प्राची॒ दिग् वि॒राट् । \newline
11. दिग् वि॒राड् वि॒राड् दिग् दिग् वि॒रा ड॑स्यसि वि॒राड् दिग् दिग् वि॒राड॑सि । \newline
12. वि॒रा ड॑स्यसि वि॒राड् वि॒राड॑सि दक्षि॒णा द॑क्षि॒णा ऽसि॑ वि॒राड् वि॒राड॑सि दक्षि॒णा । \newline
13. वि॒राडिति॑ वि - राट् । \newline
14. अ॒सि॒ द॒क्षि॒णा द॑क्षि॒णा ऽस्य॑सि दक्षि॒णा दिग् दिग् द॑क्षि॒णा ऽस्य॑सि दक्षि॒णा दिक् । \newline
15. द॒क्षि॒णा दिग् दिग् द॑क्षि॒णा द॑क्षि॒णा दिख् स॒म्राट् थ्स॒म्राड् दिग् द॑क्षि॒णा द॑क्षि॒णा दिख् स॒म्राट् । \newline
16. दिख् स॒म्राट् थ्स॒म्राड् दिग् दिख् स॒म्रा ड॑स्यसि स॒म्राड् दिग् दिख् स॒म्राड॑सि । \newline
17. स॒म्रा ड॑स्यसि स॒म्राट् थ्स॒म्रा ड॑सि प्र॒तीची᳚ प्र॒तीच्य॑सि स॒म्राट् थ्स॒म्रा ड॑सि प्र॒तीची᳚ । \newline
18. स॒म्राडिति॑ सं - राट् । \newline
19. अ॒सि॒ प्र॒तीची᳚ प्र॒ती च्य॑स्यसि प्र॒तीची॒ दिग् दिक् प्र॒ती च्य॑स्यसि प्र॒तीची॒ दिक् । \newline
20. प्र॒तीची॒ दिग् दिक् प्र॒तीची᳚ प्र॒तीची॒ दिख् स्व॒राट् थ्स्व॒राड् दिक् प्र॒तीची᳚ प्र॒तीची॒ दिख् स्व॒राट् । \newline
21. दिख् स्व॒राट् थ्स्व॒राड् दिग् दिख् स्व॒रा ड॑स्यसि स्व॒राड् दिग् दिख् स्व॒रा ड॑सि । \newline
22. स्व॒रा ड॑स्यसि स्व॒राट् थ्स्व॒रा ड॒स्युदी॒ च्युदी᳚च्यसि स्व॒राट् थ्स्व॒रा ड॒स्युदी॑ची । \newline
23. स्व॒राडिति॑ स्व - राट् । \newline
24. अ॒स्युदी॒ च्युदी᳚च्य स्य॒ स्युदी॑ची॒ दिग् दिगुदी᳚च्य स्य॒ स्युदी॑ची॒ दिक् । \newline
25. उदी॑ची॒ दिग् दिगुदी॒ च्युदी॑ची॒ दिगधि॑प॒त्न्य धि॑पत्नी॒ दिगुदी॒ च्युदी॑ची॒ दिगधि॑पत्नी । \newline
26. दिगधि॑प॒त् न्यधि॑पत्नी॒ दिग् दिगधि॑पत्न्य स्य॒ स्य धि॑पत्नी॒ दिग् दिगधि॑ पत्न्यसि । \newline
27. अधि॑पत् न्यस्य॒स्य धि॑प॒त्न्यधि॑ पत्न्यसि बृह॒ती बृ॑ह॒ त्य॑स्य धि॑प॒त्न्य धि॑पत्न्यसि बृह॒ती । \newline
28. अधि॑प॒त्नीत्यधि॑ - प॒त्नी॒ । \newline
29. अ॒सि॒ बृ॒ह॒ती बृ॑ह॒ त्य॑स्यसि बृह॒ती दिग् दिग् बृ॑ह॒ त्य॑स्यसि बृह॒ती दिक् । \newline
30. बृ॒ह॒ती दिग् दिग् बृ॑ह॒ती बृ॑ह॒ती दिगायु॒ रायु॒र् दिग् बृ॑ह॒ती बृ॑ह॒ती दिगायुः॑ । \newline
31. दिगायु॒ रायु॒र् दिग् दिगायु॑र् मे म॒ आयु॒र् दिग् दिगायु॑र् मे । \newline
32. आयु॑र् मे म॒ आयु॒ रायु॑र् मे पाहि पाहि म॒ आयु॒ रायु॑र् मे पाहि । \newline
33. मे॒ पा॒हि॒ पा॒हि॒ मे॒ मे॒ पा॒हि॒ प्रा॒णम् प्रा॒णम् पा॑हि मे मे पाहि प्रा॒णम् । \newline
34. पा॒हि॒ प्रा॒णम् प्रा॒णम् पा॑हि पाहि प्रा॒णम् मे॑ मे प्रा॒णम् पा॑हि पाहि प्रा॒णम् मे᳚ । \newline
35. प्रा॒णम् मे॑ मे प्रा॒णम् प्रा॒णम् मे॑ पाहि पाहि मे प्रा॒णम् प्रा॒णम् मे॑ पाहि । \newline
36. प्रा॒णमिति॑ प्र - अ॒नम् । \newline
37. मे॒ पा॒हि॒ पा॒हि॒ मे॒ मे॒ पा॒ह्य॒पा॒न म॑पा॒नम् पा॑हि मे मे पाह्यपा॒नम् । \newline
38. पा॒ह्य॒पा॒न म॑पा॒नम् पा॑हि पाह्यपा॒नम् मे॑ मे अपा॒नम् पा॑हि पाह्यपा॒नम् मे᳚ । \newline
39. अ॒पा॒नम् मे॑ मे अपा॒न म॑पा॒नम् मे॑ पाहि पाहि मे अपा॒न म॑पा॒नम् मे॑ पाहि । \newline
40. अ॒पा॒नमित्य॑प - अ॒नम् । \newline
41. मे॒ पा॒हि॒ पा॒हि॒ मे॒ मे॒ पा॒हि॒ व्या॒नं ॅव्या॒नम् पा॑हि मे मे पाहि व्या॒नम् । \newline
42. पा॒हि॒ व्या॒नं ॅव्या॒नम् पा॑हि पाहि व्या॒नम् मे॑ मे व्या॒नम् पा॑हि पाहि व्या॒नम् मे᳚ । \newline
43. व्या॒नम् मे॑ मे व्या॒नं ॅव्या॒नम् मे॑ पाहि पाहि मे व्या॒नं ॅव्या॒नम् मे॑ पाहि । \newline
44. व्या॒नमिति॑ वि - अ॒नम् । \newline
45. मे॒ पा॒हि॒ पा॒हि॒ मे॒ मे॒ पा॒हि॒ चक्षु॒ श्चक्षुः॑ पाहि मे मे पाहि॒ चक्षुः॑ । \newline
46. पा॒हि॒ चक्षु॒ श्चक्षुः॑ पाहि पाहि॒ चक्षु॑र् मे मे॒ चक्षुः॑ पाहि पाहि॒ चक्षु॑र् मे । \newline
47. चक्षु॑र् मे मे॒ चक्षु॒ श्चक्षु॑र् मे पाहि पाहि मे॒ चक्षु॒ श्चक्षु॑र् मे पाहि । \newline
48. मे॒ पा॒हि॒ पा॒हि॒ मे॒ मे॒ पा॒हि॒ श्रोत्रꣳ॒॒ श्रोत्र॑म् पाहि मे मे पाहि॒ श्रोत्र᳚म् । \newline
49. पा॒हि॒ श्रोत्रꣳ॒॒ श्रोत्र॑म् पाहि पाहि॒ श्रोत्र॑म् मे मे॒ श्रोत्र॑म् पाहि पाहि॒ श्रोत्र॑म् मे । \newline
50. श्रोत्र॑म् मे मे॒ श्रोत्रꣳ॒॒ श्रोत्र॑म् मे पाहि पाहि मे॒ श्रोत्रꣳ॒॒ श्रोत्र॑म् मे पाहि । \newline
51. मे॒ पा॒हि॒ पा॒हि॒ मे॒ मे॒ पा॒हि॒ मनो॒ मनः॑ पाहि मे मे पाहि॒ मनः॑ । \newline
52. पा॒हि॒ मनो॒ मनः॑ पाहि पाहि॒ मनो॑ मे मे॒ मनः॑ पाहि पाहि॒ मनो॑ मे । \newline
53. मनो॑ मे मे॒ मनो॒ मनो॑ मे जिन्व जिन्व मे॒ मनो॒ मनो॑ मे जिन्व । \newline
54. मे॒ जि॒न्व॒ जि॒न्व॒ मे॒ मे॒ जि॒न्व॒ वाचं॒ ॅवाच॑म् जिन्व मे मे जिन्व॒ वाच᳚म् । \newline
55. जि॒न्व॒ वाचं॒ ॅवाच॑म् जिन्व जिन्व॒ वाच॑म् मे मे॒ वाच॑म् जिन्व जिन्व॒ वाच॑म् मे । \newline
56. वाच॑म् मे मे॒ वाचं॒ ॅवाच॑म् मे पिन्व पिन्व मे॒ वाचं॒ ॅवाच॑म् मे पिन्व । \newline
57. मे॒ पि॒न्व॒ पि॒न्व॒ मे॒ मे॒ पि॒न्वा॒ त्मान॑ मा॒त्मान॑म् पिन्व मे मे पिन्वा॒ त्मान᳚म् । \newline
58. पि॒न्वा॒ त्मान॑ मा॒त्मान॑म् पिन्व पिन्वा॒ त्मान॑म् मे म आ॒त्मान॑म् पिन्व पिन्वा॒ त्मान॑म् मे । \newline
59. आ॒त्मान॑म् मे म आ॒त्मान॑ मा॒त्मान॑म् मे पाहि पाहि म आ॒त्मान॑ मा॒त्मान॑म् मे पाहि । \newline
60. मे॒ पा॒हि॒ पा॒हि॒ मे॒ मे॒ पा॒हि॒ ज्योति॒र् ज्योतिः॑ पाहि मे मे पाहि॒ ज्योतिः॑ । \newline
61. पा॒हि॒ ज्योति॒र् ज्योतिः॑ पाहि पाहि॒ ज्योति॑र् मे मे॒ ज्योतिः॑ पाहि पाहि॒ ज्योति॑र् मे । \newline
62. ज्योति॑र् मे मे॒ ज्योति॒र् ज्योति॑र् मे यच्छ यच्छ मे॒ ज्योति॒र् ज्योति॑र् मे यच्छ । \newline
63. मे॒ य॒च्छ॒ य॒च्छ॒ मे॒ मे॒ य॒च्छ॒ । \newline
64. य॒च्छेति॑ यच्छ । \newline
\pagebreak
\markright{ TS 4.3.7.1  \hfill https://www.vedavms.in \hfill}

\section{ TS 4.3.7.1 }

\textbf{TS 4.3.7.1 } \newline
\textbf{Samhita Paata} \newline

मा छन्दः॑ प्र॒मा छन्दः॑ प्रति॒मा छन्दो᳚ऽस्री॒वि श्छन्दः॑ प॒ङ्क्ति श्छन्द॑ उ॒ष्णिहा॒ छन्दो॑ बृह॒ती छन्दो॑ऽनु॒ष्टुप् छन्दो॑ वि॒राट् छन्दो॑ गाय॒त्री छन्द॑-स्त्रि॒ष्टुप् छन्दो॒ जग॑ती॒ छन्दः॑ पृथि॒वी छन्दो॒ ऽन्तरि॑क्षं॒ छन्दो॒ द्यौ श्छन्दः॒ समा॒ श्छन्दो॒ नक्ष॑त्राणि॒ छन्दो॒ मन॒ श्छन्दो॒ वाक् छन्दः॑ कृ॒षि श्छन्दो॒ हिर॑ण्यं॒ छन्दो॒ गौ श्छन्दो॒ ऽजा छन्दो ऽश्व॒ श्छन्दः॑ ॥ अ॒ग्निर्दे॒वता॒ - [  ] \newline

\textbf{Pada Paata} \newline

मा । छन्दः॑ । प्र॒मेति॑ प्र - मा । छन्दः॑ । प्र॒ति॒मेति॑ प्रति-मा । छन्दः॑ । अ॒स्री॒विः । छन्दः॑ । प॒ङ्क्तिः । छन्दः॑ । उ॒ष्णिहा᳚ । छन्दः॑ । बृ॒ह॒ती । छन्दः॑ । अ॒नु॒ष्टुबित्य॑नु - स्तुप् । छन्दः॑ । वि॒राडिति॑ वि - राट् । छन्दः॑ । गा॒य॒त्री । छन्दः॑ । त्रि॒ष्टुप् । छन्दः॑ । जग॑ती । छन्दः॑ । पृ॒थि॒वी । छन्दः॑ । अ॒न्तरि॑क्षम् । छन्दः॑ । द्यौः । छन्दः॑ । समाः᳚ । छन्दः॑ । नक्ष॑त्राणि । छन्दः॑ । मनः॑ । छन्दः॑ । वाक् । छन्दः॑ । कृ॒षिः । छन्दः॑ । हिर॑ण्यम् । छन्दः॑ । गौः । छन्दः॑ । अ॒जा । छन्दः॑ । अश्वः॑ । छन्दः॑ ॥ अ॒ग्निः । दे॒वता᳚ ।  \newline


\textbf{Krama Paata} \newline

मा छन्दः॑ । छन्दः॑ प्र॒मा । प्र॒मा छन्दः॑ । प्र॒मेति॑ प्र - मा । छन्दः॑ प्रति॒मा । प्र॒ति॒मा छन्दः॑ । प्र॒ति॒मेति॑ प्रति - मा । छन्दो᳚ऽस्री॒विः । अ॒स्री॒वि श्छन्दः॑ । छन्दः॑ प॒ङ्क्तिः । प॒ङ्क्ति श्छन्दः॑ । छन्द॑ उ॒ष्णिहा᳚ । उ॒ष्णिहा॒ छन्दः॑ । छन्दो॑ बृह॒ती । बृ॒ह॒ती छन्दः॑ । छन्दो॑ऽनु॒ष्टुप् । अ॒नु॒ष्टुप् छन्दः॑ । अ॒नु॒ष्टुबित्य॑नु - स्तुप् । छन्दो॑ वि॒राट् । वि॒राट् छन्दः॑ । वि॒राडिति॑ वि - राट् । छन्दो॑ गाय॒त्री । गा॒य॒त्री छन्दः॑ । छन्द॑स्त्रि॒ष्टुप् । त्रि॒ष्टुप् छन्दः॑ । छन्दो॒ जग॑ती । जग॑ती॒ छन्दः॑ । छन्दः॑ पृथि॒वी । पृ॒थि॒वी छन्दः॑ । छन्दो॒ऽन्तरि॑क्षम् । अ॒न्तरि॑क्ष॒म् छन्दः॑ । छन्दो॒ द्यौः । द्यौ श्चन्दः॑ । छन्दः॒ समाः᳚ । समा॒ श्छन्दः॑ । छन्दो॒ नक्ष॑त्राणि । नक्ष॑त्राणि॒ छन्दः॑ । छन्दो॒ मनः॑ । मन॒ श्छन्दः॑ । छन्दो॒ वाक् । वाक् छन्दः॑ । छन्दः॑ कृ॒षिः । कृ॒षि श्छन्दः॑ । छन्दो॒ हिर॑ण्यम् । हिर॑ण्य॒म् छन्दः॑ । छन्दो॒ गौः । गौ श्छन्दः॑ । छन्दो॒ऽजा । अ॒जा छन्दः॑ । छन्दो ऽश्वः॑ । अश्व॒ श्छन्दः॑ । छन्द॒ इति॒ छन्दः॑ ॥ अ॒ग्निर् दे॒वता᳚ । दे॒वता॒ वातः॑ \newline

\textbf{Jatai Paata} \newline

1. मा छन्द॒ श्छन्दो॒ मा मा छन्दः॑ । \newline
2. छन्दः॑ प्र॒मा प्र॒मा छन्द॒ श्छन्दः॑ प्र॒मा । \newline
3. प्र॒मा छन्द॒ श्छन्दः॑ प्र॒मा प्र॒मा छन्दः॑ । \newline
4. प्र॒मेति॑ प्र - मा । \newline
5. छन्दः॑ प्रति॒मा प्र॑ति॒मा छन्द॒ श्छन्दः॑ प्रति॒मा । \newline
6. प्र॒ति॒मा छन्द॒ श्छन्दः॑ प्रति॒मा प्र॑ति॒मा छन्दः॑ । \newline
7. प्र॒ति॒मेति॑ प्रति - मा । \newline
8. छन्दो᳚ ऽस्री॒वि र॑स्री॒वि श्छन्द॒ श्छन्दो᳚ ऽस्री॒विः । \newline
9. अ॒स्री॒वि श्छन्द॒ श्छन्दो᳚ ऽस्री॒वि र॑स्री॒वि श्छन्दः॑ । \newline
10. छन्दः॑ प॒ङ्क्तिः प॒ङ्क्ति श्छन्द॒ श्छन्दः॑ प॒ङ्क्तिः । \newline
11. प॒ङ्क्ति श्छन्द॒ श्छन्दः॑ प॒ङ्क्तिः प॒ङ्क्ति श्छन्दः॑ । \newline
12. छन्द॑ उ॒ष्णि हो॒ष्णिहा॒ छन्द॒ श्छन्द॑ उ॒ष्णिहा᳚ । \newline
13. उ॒ष्णिहा॒ छन्द॒ श्छन्द॑ उ॒ष्णि हो॒ष्णिहा॒ छन्दः॑ । \newline
14. छन्दो॑ बृह॒ती बृ॑ह॒ती छन्द॒ श्छन्दो॑ बृह॒ती । \newline
15. बृ॒ह॒ती छन्द॒ श्छन्दो॑ बृह॒ती बृ॑ह॒ती छन्दः॑ । \newline
16. छन्दो॑ ऽनु॒ष्टु ब॑नु॒ष्टुप् छन्द॒ श्छन्दो॑ ऽनु॒ष्टुप् । \newline
17. अ॒नु॒ष्टुप् छन्द॒ श्छन्दो॑ ऽनु॒ष्टु ब॑नु॒ष्टुप् छन्दः॑ । \newline
18. अ॒नु॒ष्टुबित्य॑नु - स्तुप् । \newline
19. छन्दो॑ वि॒राड् वि॒राट् छन्द॒ श्छन्दो॑ वि॒राट् । \newline
20. वि॒राट् छन्द॒ श्छन्दो॑ वि॒राड् वि॒राट् छन्दः॑ । \newline
21. वि॒राडिति॑ वि - राट् । \newline
22. छन्दो॑ गाय॒त्री गा॑य॒त्री छन्द॒ श्छन्दो॑ गाय॒त्री । \newline
23. गा॒य॒त्री छन्द॒ श्छन्दो॑ गाय॒त्री गा॑य॒त्री छन्दः॑ । \newline
24. छन्द॑ स्त्रि॒ष्टुप् त्रि॒ष्टुप् छन्द॒ श्छन्द॑ स्त्रि॒ष्टुप् । \newline
25. त्रि॒ष्टुप् छन्द॒ श्छन्द॑ स्त्रि॒ष्टुप् त्रि॒ष्टुप् छन्दः॑ । \newline
26. छन्दो॒ जग॑ती॒ जग॑ती॒ छन्द॒ श्छन्दो॒ जग॑ती । \newline
27. जग॑ती॒ छन्द॒ श्छन्दो॒ जग॑ती॒ जग॑ती॒ छन्दः॑ । \newline
28. छन्दः॑ पृथि॒वी पृ॑थि॒वी छन्द॒ श्छन्दः॑ पृथि॒वी । \newline
29. पृ॒थि॒वी छन्द॒ श्छन्दः॑ पृथि॒वी पृ॑थि॒वी छन्दः॑ । \newline
30. छन्दो॒ ऽन्तरि॑क्ष म॒न्तरि॑क्ष॒म् छन्द॒ श्छन्दो॒ ऽन्तरि॑क्षम् । \newline
31. अ॒न्तरि॑क्ष॒म् छन्द॒ श्छन्दो॒ ऽन्तरि॑क्ष म॒न्तरि॑क्ष॒म् छन्दः॑ । \newline
32. छन्दो॒ द्यौर् द्यौ श्छन्द॒ श्छन्दो॒ द्यौः । \newline
33. द्यौ श्छन्द॒ श्छन्दो॒ द्यौर् द्यौ श्छन्दः॑ । \newline
34. छन्दः॒ समाः॒ समा॒ श्छन्द॒ श्छन्दः॒ समाः᳚ । \newline
35. समा॒ श्छन्द॒ श्छन्दः॒ समाः॒ समा॒ श्छन्दः॑ । \newline
36. छन्दो॒ नक्ष॑त्राणि॒ नक्ष॑त्राणि॒ छन्द॒ श्छन्दो॒ नक्ष॑त्राणि । \newline
37. नक्ष॑त्राणि॒ छन्द॒ श्छन्दो॒ नक्ष॑त्राणि॒ नक्ष॑त्राणि॒ छन्दः॑ । \newline
38. छन्दो॒ मनो॒ मन॒ श्छन्द॒ श्छन्दो॒ मनः॑ । \newline
39. मन॒ श्छन्द॒ श्छन्दो॒ मनो॒ मन॒ श्छन्दः॑ । \newline
40. छन्दो॒ वाग् वाक् छन्द॒ श्छन्दो॒ वाक् । \newline
41. वाक् छन्द॒ श्छन्दो॒ वाग् वाक् छन्दः॑ । \newline
42. छन्दः॑ कृ॒षिः कृ॒षि श्छन्द॒ श्छन्दः॑ कृ॒षिः । \newline
43. कृ॒षि श्छन्द॒ श्छन्दः॑ कृ॒षिः कृ॒षि श्छन्दः॑ । \newline
44. छन्दो॒ हिर॑ण्यꣳ॒॒ हिर॑ण्य॒म् छन्द॒ श्छन्दो॒ हिर॑ण्यम् । \newline
45. हिर॑ण्य॒म् छन्द॒ श्छन्दो॒ हिर॑ण्यꣳ॒॒ हिर॑ण्य॒म् छन्दः॑ । \newline
46. छन्दो॒ गौर् गौ श्छन्द॒ श्छन्दो॒ गौः । \newline
47. गौ श्छन्द॒ श्छन्दो॒ गौर् गौ श्छन्दः॑ । \newline
48. छन्दो॒ ऽजा ऽजा छन्द॒ श्छन्दो॒ ऽजा । \newline
49. अ॒जा छन्द॒ श्छन्दो॒ ऽजा ऽजा छन्दः॑ । \newline
50. छन्दो ऽश्वो ऽश्व॒ श्छन्द॒ श्छन्दो ऽश्वः॑ । \newline
51. अश्व॒ श्छन्द॒ श्छन्दो ऽश्वो ऽश्व॒ श्छन्दः॑ । \newline
52. छन्द॒ इति॒ छन्दः॑ । \newline
53. अ॒ग्निर् दे॒वता॑ दे॒वता॒ ऽग्नि र॒ग्निर् दे॒वता᳚ । \newline
54. दे॒वता॒ वातो॒ वातो॑ दे॒वता॑ दे॒वता॒ वातः॑ । \newline

\textbf{Ghana Paata } \newline

1. मा छन्द॒ श्छन्दो॒ मा मा छन्दः॑ प्र॒मा प्र॒मा छन्दो॒ मा मा छन्दः॑ प्र॒मा । \newline
2. छन्दः॑ प्र॒मा प्र॒मा छन्द॒ श्छन्दः॑ प्र॒मा छन्द॒ श्छन्दः॑ प्र॒मा छन्द॒ श्छन्दः॑ प्र॒मा छन्दः॑ । \newline
3. प्र॒मा छन्द॒ श्छन्दः॑ प्र॒मा प्र॒मा छन्दः॑ प्रति॒मा प्र॑ति॒मा छन्दः॑ प्र॒मा प्र॒मा छन्दः॑ प्रति॒मा । \newline
4. प्र॒मेति॑ प्र - मा । \newline
5. छन्दः॑ प्रति॒मा प्र॑ति॒मा छन्द॒ श्छन्दः॑ प्रति॒मा छन्द॒ श्छन्दः॑ प्रति॒मा छन्द॒ श्छन्दः॑ प्रति॒मा छन्दः॑ । \newline
6. प्र॒ति॒मा छन्द॒ श्छन्दः॑ प्रति॒मा प्र॑ति॒मा छन्दो᳚ ऽस्री॒वि र॑स्री॒वि श्छन्दः॑ प्रति॒मा प्र॑ति॒मा छन्दो᳚ ऽस्री॒विः । \newline
7. प्र॒ति॒मेति॑ प्रति - मा । \newline
8. छन्दो᳚ ऽस्री॒वि र॑स्री॒वि श्छन्द॒ श्छन्दो᳚ ऽस्री॒वि श्छन्द॒ श्छन्दो᳚ ऽस्री॒वि श्छन्द॒ श्छन्दो᳚ ऽस्री॒वि श्छन्दः॑ । \newline
9. अ॒स्री॒वि श्छन्द॒ श्छन्दो᳚ ऽस्री॒वि र॑स्री॒वि श्छन्दः॑ प॒ङ्क्तिः प॒ङ्क्ति श्छन्दो᳚ ऽस्री॒वि र॑स्री॒वि श्छन्दः॑ प॒ङ्क्तिः । \newline
10. छन्दः॑ प॒ङ्क्तिः प॒ङ्क्ति श्छन्द॒ श्छन्दः॑ प॒ङ्क्ति श्छन्द॒ श्छन्दः॑ प॒ङ्क्ति श्छन्द॒ श्छन्दः॑ प॒ङ्क्ति श्छन्दः॑ । \newline
11. प॒ङ्क्ति श्छन्द॒ श्छन्दः॑ प॒ङ्क्तिः प॒ङ्क्ति श्छन्द॑ उ॒ष्णि हो॒ष्णिहा॒ छन्दः॑ प॒ङ्क्तिः प॒ङ्क्ति श्छन्द॑ उ॒ष्णिहा᳚ । \newline
12. छन्द॑ उ॒ष्णि हो॒ष्णिहा॒ छन्द॒ श्छन्द॑ उ॒ष्णिहा॒ छन्द॒ श्छन्द॑ उ॒ष्णिहा॒ छन्द॒ श्छन्द॑ उ॒ष्णिहा॒ छन्दः॑ । \newline
13. उ॒ष्णिहा॒ छन्द॒ श्छन्द॑ उ॒ष्णि हो॒ष्णिहा॒ छन्दो॑ बृह॒ती बृ॑ह॒ती छन्द॑ उ॒ष्णि हो॒ष्णिहा॒ छन्दो॑ बृह॒ती । \newline
14. छन्दो॑ बृह॒ती बृ॑ह॒ती छन्द॒ श्छन्दो॑ बृह॒ती छन्द॒ श्छन्दो॑ बृह॒ती छन्द॒ श्छन्दो॑ बृह॒ती छन्दः॑ । \newline
15. बृ॒ह॒ती छन्द॒ श्छन्दो॑ बृह॒ती बृ॑ह॒ती छन्दो॑ ऽनु॒ष्टु ब॑नु॒ष्टुप् छन्दो॑ बृह॒ती बृ॑ह॒ती छन्दो॑ ऽनु॒ष्टुप् । \newline
16. छन्दो॑ ऽनु॒ष्टु ब॑नु॒ष्टुप् छन्द॒ श्छन्दो॑ ऽनु॒ष्टुप् छन्द॒ श्छन्दो॑ ऽनु॒ष्टुप् छन्द॒ श्छन्दो॑ ऽनु॒ष्टुप् छन्दः॑ । \newline
17. अ॒नु॒ष्टुप् छन्द॒ श्छन्दो॑ ऽनु॒ष्टु ब॑नु॒ष्टुप् छन्दो॑ वि॒राड् वि॒राट् छन्दो॑ ऽनु॒ष्टु ब॑नु॒ष्टुप् छन्दो॑ वि॒राट् । \newline
18. अ॒नु॒ष्टुबित्य॑नु - स्तुप् । \newline
19. छन्दो॑ वि॒राड् वि॒राट् छन्द॒ श्छन्दो॑ वि॒राट् छन्द॒ श्छन्दो॑ वि॒राट् छन्द॒ श्छन्दो॑ वि॒राट् छन्दः॑ । \newline
20. वि॒राट् छन्द॒ श्छन्दो॑ वि॒राड् वि॒राट् छन्दो॑ गाय॒त्री गा॑य॒त्री छन्दो॑ वि॒राड् वि॒राट् छन्दो॑ गाय॒त्री । \newline
21. वि॒राडिति॑ वि - राट् । \newline
22. छन्दो॑ गाय॒त्री गा॑य॒त्री छन्द॒ श्छन्दो॑ गाय॒त्री छन्द॒ श्छन्दो॑ गाय॒त्री छन्द॒ श्छन्दो॑ गाय॒त्री छन्दः॑ । \newline
23. गा॒य॒त्री छन्द॒ श्छन्दो॑ गाय॒त्री गा॑य॒त्री छन्द॑ स्त्रि॒ष्टुप् त्रि॒ष्टुप् छन्दो॑ गाय॒त्री गा॑य॒त्री छन्द॑ स्त्रि॒ष्टुप् । \newline
24. छन्द॑ स्त्रि॒ष्टुप् त्रि॒ष्टुप् छन्द॒ श्छन्द॑ स्त्रि॒ष्टुप् छन्द॒ श्छन्द॑ स्त्रि॒ष्टुप् छन्द॒ श्छन्द॑ स्त्रि॒ष्टुप् छन्दः॑ । \newline
25. त्रि॒ष्टुप् छन्द॒ श्छन्द॑ स्त्रि॒ष्टुप् त्रि॒ष्टुप् छन्दो॒ जग॑ती॒ जग॑ती॒ छन्द॑ स्त्रि॒ष्टुप् त्रि॒ष्टुप् छन्दो॒ जग॑ती । \newline
26. छन्दो॒ जग॑ती॒ जग॑ती॒ छन्द॒ श्छन्दो॒ जग॑ती॒ छन्द॒ श्छन्दो॒ जग॑ती॒ छन्द॒ श्छन्दो॒ जग॑ती॒ छन्दः॑ । \newline
27. जग॑ती॒ छन्द॒ श्छन्दो॒ जग॑ती॒ जग॑ती॒ छन्दः॑ पृथि॒वी पृ॑थि॒वी छन्दो॒ जग॑ती॒ जग॑ती॒ छन्दः॑ पृथि॒वी । \newline
28. छन्दः॑ पृथि॒वी पृ॑थि॒वी छन्द॒ श्छन्दः॑ पृथि॒वी छन्द॒ श्छन्दः॑ पृथि॒वी छन्द॒ श्छन्दः॑ पृथि॒वी छन्दः॑ । \newline
29. पृ॒थि॒वी छन्द॒ श्छन्दः॑ पृथि॒वी पृ॑थि॒वी छन्दो॒ ऽन्तरि॑क्ष म॒न्तरि॑क्ष॒म् छन्दः॑ पृथि॒वी पृ॑थि॒वी छन्दो॒ ऽन्तरि॑क्षम् । \newline
30. छन्दो॒ ऽन्तरि॑क्ष म॒न्तरि॑क्ष॒म् छन्द॒ श्छन्दो॒ ऽन्तरि॑क्ष॒म् छन्द॒ श्छन्दो॒ ऽन्तरि॑क्ष॒म् छन्द॒ श्छन्दो॒ ऽन्तरि॑क्ष॒म् छन्दः॑ । \newline
31. अ॒न्तरि॑क्ष॒म् छन्द॒ श्छन्दो॒ ऽन्तरि॑क्ष म॒न्तरि॑क्ष॒म् छन्दो॒ द्यौर् द्यौ श्छन्दो॒ ऽन्तरि॑क्ष म॒न्तरि॑क्ष॒म् छन्दो॒ द्यौः । \newline
32. छन्दो॒ द्यौर् द्यौ श्छन्द॒ श्छन्दो॒ द्यौ श्छन्द॒ श्छन्दो॒ द्यौ श्छन्द॒ श्छन्दो॒ द्यौ श्छन्दः॑ । \newline
33. ड्यौ श्छन्द॒ श्छन्दो॒ द्यौर् द्यौ श्छन्दः॒ समाः॒ समा॒ श्छन्दो॒ द्यौर् द्यौ श्छन्दः॒ समाः᳚ । \newline
34. छन्दः॒ समाः॒ समा॒ श्छन्द॒ श्छन्दः॒ समा॒ श्छन्द॒ श्छन्दः॒ समा॒ श्छन्द॒ श्छन्दः॒ समा॒ श्छन्दः॑ । \newline
35. समा॒ श्छन्द॒ श्छन्दः॒ समाः॒ समा॒ श्छन्दो॒ नक्ष॑त्राणि॒ नक्ष॑त्राणि॒ छन्दः॒ समाः॒ समा॒ श्छन्दो॒ नक्ष॑त्राणि । \newline
36. छन्दो॒ नक्ष॑त्राणि॒ नक्ष॑त्राणि॒ छन्द॒ श्छन्दो॒ नक्ष॑त्राणि॒ छन्द॒ श्छन्दो॒ नक्ष॑त्राणि॒ छन्द॒ श्छन्दो॒ नक्ष॑त्राणि॒ छन्दः॑ । \newline
37. नक्ष॑त्राणि॒ छन्द॒ श्छन्दो॒ नक्ष॑त्राणि॒ नक्ष॑त्राणि॒ छन्दो॒ मनो॒ मन॒ श्छन्दो॒ नक्ष॑त्राणि॒ नक्ष॑त्राणि॒ छन्दो॒ मनः॑ । \newline
38. छन्दो॒ मनो॒ मन॒ श्छन्द॒ श्छन्दो॒ मन॒ श्छन्द॒ श्छन्दो॒ मन॒ श्छन्द॒ श्छन्दो॒ मन॒ श्छन्दः॑ । \newline
39. मन॒ श्छन्द॒ श्छन्दो॒ मनो॒ मन॒ श्छन्दो॒ वाग् वाक् छन्दो॒ मनो॒ मन॒ श्छन्दो॒ वाक् । \newline
40. छन्दो॒ वाग् वाक् छन्द॒ श्छन्दो॒ वाक् छन्द॒ श्छन्दो॒ वाक् छन्द॒ श्छन्दो॒ वाक् छन्दः॑ । \newline
41. वाक् छन्द॒ श्छन्दो॒ वाग् वाक् छन्दः॑ कृ॒षिः कृ॒षि श्छन्दो॒ वाग् वाक् छन्दः॑ कृ॒षिः । \newline
42. छन्दः॑ कृ॒षिः कृ॒षि श्छन्द॒ श्छन्दः॑ कृ॒षि श्छन्द॒ श्छन्दः॑ कृ॒षि श्छन्द॒ श्छन्दः॑ कृ॒षि श्छन्दः॑ । \newline
43. कृ॒षि श्छन्द॒ श्छन्दः॑ कृ॒षिः कृ॒षि श्छन्दो॒ हिर॑ण्यꣳ॒॒ हिर॑ण्य॒म् छन्दः॑ कृ॒षिः कृ॒षि श्छन्दो॒ हिर॑ण्यम् । \newline
44. छन्दो॒ हिर॑ण्यꣳ॒॒ हिर॑ण्य॒म् छन्द॒ श्छन्दो॒ हिर॑ण्य॒म् छन्द॒ श्छन्दो॒ हिर॑ण्य॒म् छन्द॒ श्छन्दो॒ हिर॑ण्य॒म् छन्दः॑ । \newline
45. हिर॑ण्य॒म् छन्द॒ श्छन्दो॒ हिर॑ण्यꣳ॒॒ हिर॑ण्य॒म् छन्दो॒ गौर् गौ श्छन्दो॒ हिर॑ण्यꣳ॒॒ हिर॑ण्य॒म् छन्दो॒ गौः । \newline
46. छन्दो॒ गौर् गौ श्छन्द॒ श्छन्दो॒ गौ श्छन्द॒ श्छन्दो॒ गौ श्छन्द॒ श्छन्दो॒ गौ श्छन्दः॑ । \newline
47. गौ श्छन्द॒ श्छन्दो॒ गौर् गौ श्छन्दो॒ ऽजा ऽजा छन्दो॒ गौर् गौ श्छन्दो॒ ऽजा । \newline
48. छन्दो॒ ऽजा ऽजा छन्द॒ श्छन्दो॒ ऽजा छन्द॒ श्छन्दो॒ ऽजा छन्द॒ श्छन्दो॒ ऽजा छन्दः॑ । \newline
49. अ॒जा छन्द॒ श्छन्दो॒ ऽजा ऽजा छन्दो ऽश्वो ऽश्व॒ श्छन्दो॒ ऽजा ऽजा छन्दो ऽश्वः॑ । \newline
50. छन्दो ऽश्वो ऽश्व॒ श्छन्द॒ श्छन्दो ऽश्व॒ श्छन्द॒ श्छन्दो ऽश्व॒ श्छन्द॒ श्छन्दो ऽश्व॒ श्छन्दः॑ । \newline
51. अश्व॒ श्छन्द॒ श्छन्दो ऽश्वो ऽश्व॒ श्छन्दः॑ । \newline
52. छन्द॒ इति॒ छन्दः॑ । \newline
53. अ॒ग्निर् दे॒वता॑ दे॒वता॒ ऽग्नि र॒ग्निर् दे॒वता॒ वातो॒ वातो॑ दे॒वता॒ ऽग्नि र॒ग्निर् दे॒वता॒ वातः॑ । \newline
54. दे॒वता॒ वातो॒ वातो॑ दे॒वता॑ दे॒वता॒ वातो॑ दे॒वता॑ दे॒वता॒ वातो॑ दे॒वता॑ दे॒वता॒ वातो॑ दे॒वता᳚ । \newline
\pagebreak
\markright{ TS 4.3.7.2  \hfill https://www.vedavms.in \hfill}

\section{ TS 4.3.7.2 }

\textbf{TS 4.3.7.2 } \newline
\textbf{Samhita Paata} \newline

वातो॑ दे॒वता॒ सूर्यो॑ दे॒वता॑ च॒न्द्रमा॑ दे॒वता॒ वस॑वो दे॒वता॑ रु॒द्रा दे॒वता॑ ऽऽदि॒त्या दे॒वता॒ विश्वे॑ दे॒वा दे॒वता॑ म॒रुतो॑ दे॒वता॒ बृह॒स्पति॑ र्दे॒वतेन्द्रो॑ दे॒वता॒ वरु॑णो दे॒वता॑ मू॒र्द्धाऽसि॒ राड् ध्रु॒वाऽसि॑ ध॒रुणा॑ य॒न्त्र्य॑सि॒ यमि॑त्री॒षे त्वो॒र्जे त्वा॑ कृ॒ष्यै त्वा॒ क्षेमा॑य त्वा॒ यन्त्री॒ राड् ध्रु॒वाऽसि॒ धर॑णी ध॒र्त्र्य॑सि॒ धरि॒त्र्यायु॑षे त्वा॒ ( ) वर्च॑से॒ त्वौज॑से त्वा॒ बला॑य त्वा ॥ \newline

\textbf{Pada Paata} \newline

वातः॑ । दे॒वता᳚ । सूर्यः॑ । दे॒वता᳚ । च॒न्द्रमाः᳚ । दे॒वता᳚ । वस॑वः । दे॒वता᳚ । रु॒द्राः । दे॒वता᳚ । आ॒दि॒त्याः । दे॒वता᳚ । विश्वे᳚ । दे॒वाः । दे॒वता᳚ । म॒रुतः॑ । दे॒वता᳚ । बृह॒स्पतिः॑ । दे॒वता᳚ । इन्द्रः॑ । दे॒वता᳚ । वरु॑णः । दे॒वता᳚ । मू॒द्‌र्धा । अ॒सि॒ । राट् । ध्रु॒वा । अ॒सि॒ । ध॒रुणा᳚ । य॒न्त्री । अ॒सि॒ । यमि॑त्री । इ॒षे । त्वा॒ । ऊ॒र्जे । त्वा॒ । कृ॒ष्यै । त्वा॒ । क्षेमा॑य । त्वा॒ । यन्त्री᳚ । राट् । ध्रु॒वा । अ॒सि॒ । धर॑णी । ध॒र्त्री । अ॒सि॒ । धरि॑त्री । आयु॑षे । त्वा॒ ( ) । वर्च॑से । त्वा॒ । ओज॑से । त्वा॒ । बला॑य । त्वा॒ ॥ 14  \newline


\textbf{Krama Paata} \newline

वातो॑ दे॒वता᳚ । दे॒वता॒ सूर्यः॑ । सूर्यो॑ दे॒वता᳚ । दे॒वता॑ च॒न्द्रमाः᳚ । च॒न्द्रमा॑ दे॒वता᳚ । दे॒वता॒ वस॑वः । वस॑वो दे॒वता᳚ । दे॒वता॑ रु॒द्राः । रु॒द्रा दे॒वता᳚ । दे॒वता॑ऽऽदि॒त्याः । आ॒दि॒त्या दे॒वता᳚ । दे॒वता॒ विश्वे᳚ । विश्वे॑ दे॒वाः । दे॒वा दे॒वता᳚ । दे॒वता॑ म॒रुतः॑ । म॒रुतो॑ दे॒वता᳚ । दे॒वता॒ बृह॒स्पतिः॑ । बृह॒स्पति॑र् दे॒वता᳚ । दे॒वतेन्द्रः॑ । इन्द्रो॑ दे॒वता᳚ । दे॒वता॒ वरु॑णः । वरु॑णो दे॒वता᳚ । दे॒वता॑ मू॒र्द्धा । मू॒र्द्धाऽसि॑ । अ॒सि॒ राट् । राड् ध्रु॒वा । ध्रु॒वा ऽसि॑ । अ॒सि॒ ध॒रुणा᳚ । ध॒रुणा॑ य॒न्त्री । य॒न्त्र्य॑सि । अ॒सि॒ यमि॑त्री । यमि॑त्री॒षे । इ॒षे त्वा᳚ । त्वो॒र्जे । ऊ॒र्जे त्वा᳚ । त्वा॒ कृ॒ष्यै । कृ॒ष्यै त्वा᳚ । त्वा॒ क्षेमा॑य । क्षेमा॑य त्वा । त्वा॒ यन्त्री᳚ । यन्त्री॒ राट् । राड् ध्रु॒वा । ध्रु॒वाऽसि॑ । अ॒सि॒ धर॑णी । धर॑णी ध॒र्त्री । ध॒र्त्र्य॑सि । अ॒सि॒ धरि॑त्री । धरि॒त्य्रायु॑षे ( ) । आयु॑षे त्वा । त्वा॒ वर्च॑से । वर्च॑से त्वा । त्वौज॑से । ओज॑से त्वा । त्वा॒ बला॑य । बला॑य त्वा । त्वेति॑ त्वा । \newline

\textbf{Jatai Paata} \newline

1. वातो॑ दे॒वता॑ दे॒वता॒ वातो॒ वातो॑ दे॒वता᳚ । \newline
2. दे॒वता॒ सूर्यः॒ सूर्यो॑ दे॒वता॑ दे॒वता॒ सूर्यः॑ । \newline
3. सूर्यो॑ दे॒वता॑ दे॒वता॒ सूर्यः॒ सूर्यो॑ दे॒वता᳚ । \newline
4. दे॒वता॑ च॒न्द्रमा᳚ श्च॒न्द्रमा॑ दे॒वता॑ दे॒वता॑ च॒न्द्रमाः᳚ । \newline
5. च॒न्द्रमा॑ दे॒वता॑ दे॒वता॑ च॒न्द्रमा᳚ श्च॒न्द्रमा॑ दे॒वता᳚ । \newline
6. दे॒वता॒ वस॑वो॒ वस॑वो दे॒वता॑ दे॒वता॒ वस॑वः । \newline
7. वस॑वो दे॒वता॑ दे॒वता॒ वस॑वो॒ वस॑वो दे॒वता᳚ । \newline
8. दे॒वता॑ रु॒द्रा रु॒द्रा दे॒वता॑ दे॒वता॑ रु॒द्राः । \newline
9. रु॒द्रा दे॒वता॑ दे॒वता॑ रु॒द्रा रु॒द्रा दे॒वता᳚ । \newline
10. दे॒वता॑ ऽऽदि॒त्या आ॑दि॒त्या दे॒वता॑ दे॒वता॑ ऽऽदि॒त्याः । \newline
11. आ॒दि॒त्या दे॒वता॑ दे॒वता॑ ऽऽदि॒त्या आ॑दि॒त्या दे॒वता᳚ । \newline
12. दे॒वता॒ विश्वे॒ विश्वे॑ दे॒वता॑ दे॒वता॒ विश्वे᳚ । \newline
13. विश्वे॑ दे॒वा दे॒वा विश्वे॒ विश्वे॑ दे॒वाः । \newline
14. दे॒वा दे॒वता॑ दे॒वता॑ दे॒वा दे॒वा दे॒वता᳚ । \newline
15. दे॒वता॑ म॒रुतो॑ म॒रुतो॑ दे॒वता॑ दे॒वता॑ म॒रुतः॑ । \newline
16. म॒रुतो॑ दे॒वता॑ दे॒वता॑ म॒रुतो॑ म॒रुतो॑ दे॒वता᳚ । \newline
17. दे॒वता॒ बृह॒स्पति॒र् बृह॒स्पति॑र् दे॒वता॑ दे॒वता॒ बृह॒स्पतिः॑ । \newline
18. बृह॒स्पति॑र् दे॒वता॑ दे॒वता॒ बृह॒स्पति॒र् बृह॒स्पति॑र् दे॒वता᳚ । \newline
19. दे॒वतेन्द्र॒ इन्द्रो॑ दे॒वता॑ दे॒वतेन्द्रः॑ । \newline
20. इन्द्रो॑ दे॒वता॑ दे॒वतेन्द्र॒ इन्द्रो॑ दे॒वता᳚ । \newline
21. दे॒वता॒ वरु॑णो॒ वरु॑णो दे॒वता॑ दे॒वता॒ वरु॑णः । \newline
22. वरु॑णो दे॒वता॑ दे॒वता॒ वरु॑णो॒ वरु॑णो दे॒वता᳚ । \newline
23. दे॒वता॑ मू॒र्द्धा मू॒र्द्धा दे॒वता॑ दे॒वता॑ मू॒र्द्धा । \newline
24. मू॒र्द्धा ऽस्य॑सि मू॒र्द्धा मू॒र्द्धा ऽसि॑ । \newline
25. अ॒सि॒ राड् राड॑स्यसि॒ राट् । \newline
26. राड् ध्रु॒वा ध्रु॒वा राड् राड् ध्रु॒वा । \newline
27. ध्रु॒वा ऽस्य॑सि ध्रु॒वा ध्रु॒वा ऽसि॑ । \newline
28. अ॒सि॒ ध॒रुणा॑ ध॒रुणा᳚ ऽस्यसि ध॒रुणा᳚ । \newline
29. ध॒रुणा॑ य॒न्त्री य॒न्त्री ध॒रुणा॑ ध॒रुणा॑ य॒न्त्री । \newline
30. य॒न्त्र्य॑स्यसि य॒न्त्री य॒न्त्र्य॑सि । \newline
31. अ॒सि॒ यमि॑त्री॒ यमि॑त्र्यस्यसि॒ यमि॑त्री । \newline
32. यमि॑त्री॒ष इ॒षे यमि॑त्री॒ यमि॑त्री॒षे । \newline
33. इ॒षे त्वा᳚ त्वे॒ष इ॒षे त्वा᳚ । \newline
34. त्वो॒र्ज ऊ॒र्जे त्वा᳚ त्वो॒र्जे । \newline
35. ऊ॒र्जे त्वा᳚ त्वो॒र्ज ऊ॒र्जे त्वा᳚ । \newline
36. त्वा॒ कृ॒ष्यै कृ॒ष्यै त्वा᳚ त्वा कृ॒ष्यै । \newline
37. कृ॒ष्यै त्वा᳚ त्वा कृ॒ष्यै कृ॒ष्यै त्वा᳚ । \newline
38. त्वा॒ क्षेमा॑य॒ क्षेमा॑य त्वा त्वा॒ क्षेमा॑य । \newline
39. क्षेमा॑य त्वा त्वा॒ क्षेमा॑य॒ क्षेमा॑य त्वा । \newline
40. त्वा॒ यन्त्री॒ यन्त्री᳚ त्वा त्वा॒ यन्त्री᳚ । \newline
41. यन्त्री॒ राड् राड् यन्त्री॒ यन्त्री॒ राट् । \newline
42. राड् ध्रु॒वा ध्रु॒वा राड् राड् ध्रु॒वा । \newline
43. ध्रु॒वा ऽस्य॑सि ध्रु॒वा ध्रु॒वा ऽसि॑ । \newline
44. अ॒सि॒ धर॑णी॒ धर॑ण्य स्यसि॒ धर॑णी । \newline
45. धर॑णी ध॒र्त्री ध॒र्त्री धर॑णी॒ धर॑णी ध॒र्त्री । \newline
46. ध॒र्त्र्य॑ स्यसि ध॒र्त्री ध॒र्त्र्य॑सि । \newline
47. अ॒सि॒ धरि॑त्री॒ धरि॑त्र्य स्यसि॒ धरि॑त्री । \newline
48. धरि॒त्र्यायु॑ष॒ आयु॑षे॒ धरि॑त्री॒ धरि॒त्र्यायु॑षे । \newline
49. आयु॑षे त्वा॒ त्वा ऽऽयु॑ष॒ आयु॑षे त्वा । \newline
50. त्वा॒ वर्च॑से॒ वर्च॑से त्वा त्वा॒ वर्च॑से । \newline
51. वर्च॑से त्वा त्वा॒ वर्च॑से॒ वर्च॑से त्वा । \newline
52. त्वौज॑स॒ ओज॑से त्वा॒ त्वौज॑से । \newline
53. ओज॑से त्वा॒ त्वौज॑स॒ ओज॑से त्वा । \newline
54. त्वा॒ बला॑य॒ बला॑य त्वा त्वा॒ बला॑य । \newline
55. बला॑य त्वा त्वा॒ बला॑य॒ बला॑य त्वा । \newline
56. त्वेति॑ त्वा । \newline

\textbf{Ghana Paata } \newline

1. वातो॑ दे॒वता॑ दे॒वता॒ वातो॒ वातो॑ दे॒वता॒ सूर्यः॒ सूर्यो॑ दे॒वता॒ वातो॒ वातो॑ दे॒वता॒ सूर्यः॑ । \newline
2. दे॒वता॒ सूर्यः॒ सूर्यो॑ दे॒वता॑ दे॒वता॒ सूर्यो॑ दे॒वता॑ दे॒वता॒ सूर्यो॑ दे॒वता॑ दे॒वता॒ सूर्यो॑ दे॒वता᳚ । \newline
3. सूर्यो॑ दे॒वता॑ दे॒वता॒ सूर्यः॒ सूर्यो॑ दे॒वता॑ च॒न्द्रमा᳚ श्च॒न्द्रमा॑ दे॒वता॒ सूर्यः॒ सूर्यो॑ दे॒वता॑ च॒न्द्रमाः᳚ । \newline
4. दे॒वता॑ च॒न्द्रमा᳚ श्च॒न्द्रमा॑ दे॒वता॑ दे॒वता॑ च॒न्द्रमा॑ दे॒वता॑ दे॒वता॑ च॒न्द्रमा॑ दे॒वता॑ दे॒वता॑ च॒न्द्रमा॑ दे॒वता᳚ । \newline
5. च॒न्द्रमा॑ दे॒वता॑ दे॒वता॑ च॒न्द्रमा᳚ श्च॒न्द्रमा॑ दे॒वता॒ वस॑वो॒ वस॑वो दे॒वता॑ च॒न्द्रमा᳚ श्च॒न्द्रमा॑ दे॒वता॒ वस॑वः । \newline
6. दे॒वता॒ वस॑वो॒ वस॑वो दे॒वता॑ दे॒वता॒ वस॑वो दे॒वता॑ दे॒वता॒ वस॑वो दे॒वता॑ दे॒वता॒ वस॑वो दे॒वता᳚ । \newline
7. वस॑वो दे॒वता॑ दे॒वता॒ वस॑वो॒ वस॑वो दे॒वता॑ रु॒द्रा रु॒द्रा दे॒वता॒ वस॑वो॒ वस॑वो दे॒वता॑ रु॒द्राः । \newline
8. दे॒वता॑ रु॒द्रा रु॒द्रा दे॒वता॑ दे॒वता॑ रु॒द्रा दे॒वता॑ दे॒वता॑ रु॒द्रा दे॒वता॑ दे॒वता॑ रु॒द्रा दे॒वता᳚ । \newline
9. रु॒द्रा दे॒वता॑ दे॒वता॑ रु॒द्रा रु॒द्रा दे॒वता॑ ऽऽदि॒त्या आ॑दि॒त्या दे॒वता॑ रु॒द्रा रु॒द्रा दे॒वता॑ ऽऽदि॒त्याः । \newline
10. दे॒वता॑ ऽऽदि॒त्या आ॑दि॒त्या दे॒वता॑ दे॒वता॑ ऽऽदि॒त्या दे॒वता॑ दे॒वता॑ ऽऽदि॒त्या दे॒वता॑ दे॒वता॑ ऽऽदि॒त्या दे॒वता᳚ । \newline
11. आ॒दि॒त्या दे॒वता॑ दे॒वता॑ ऽऽदि॒त्या आ॑दि॒त्या दे॒वता॒ विश्वे॒ विश्वे॑ दे॒वता॑ ऽऽदि॒त्या आ॑दि॒त्या दे॒वता॒ विश्वे᳚ । \newline
12. दे॒वता॒ विश्वे॒ विश्वे॑ दे॒वता॑ दे॒वता॒ विश्वे॑ दे॒वा दे॒वा विश्वे॑ दे॒वता॑ दे॒वता॒ विश्वे॑ दे॒वाः । \newline
13. विश्वे॑ दे॒वा दे॒वा विश्वे॒ विश्वे॑ दे॒वा दे॒वता॑ दे॒वता॑ दे॒वा विश्वे॒ विश्वे॑ दे॒वा दे॒वता᳚ । \newline
14. दे॒वा दे॒वता॑ दे॒वता॑ दे॒वा दे॒वा दे॒वता॑ म॒रुतो॑ म॒रुतो॑ दे॒वता॑ दे॒वा दे॒वा दे॒वता॑ म॒रुतः॑ । \newline
15. दे॒वता॑ म॒रुतो॑ म॒रुतो॑ दे॒वता॑ दे॒वता॑ म॒रुतो॑ दे॒वता॑ दे॒वता॑ म॒रुतो॑ दे॒वता॑ दे॒वता॑ म॒रुतो॑ दे॒वता᳚ । \newline
16. म॒रुतो॑ दे॒वता॑ दे॒वता॑ म॒रुतो॑ म॒रुतो॑ दे॒वता॒ बृह॒स्पति॒र् बृह॒स्पति॑र् दे॒वता॑ म॒रुतो॑ म॒रुतो॑ दे॒वता॒ बृह॒स्पतिः॑ । \newline
17. दे॒वता॒ बृह॒स्पति॒र् बृह॒स्पति॑र् दे॒वता॑ दे॒वता॒ बृह॒स्पति॑र् दे॒वता॑ दे॒वता॒ बृह॒स्पति॑र् दे॒वता॑ दे॒वता॒ बृह॒स्पति॑र् दे॒वता᳚ । \newline
18. बृह॒स्पति॑र् दे॒वता॑ दे॒वता॒ बृह॒स्पति॒र् बृह॒स्पति॑र् दे॒वतेन्द्र॒ इन्द्रो॑ दे॒वता॒ बृह॒स्पति॒र् बृह॒स्पति॑र् दे॒वतेन्द्रः॑ । \newline
19. दे॒वतेन्द्र॒ इन्द्रो॑ दे॒वता॑ दे॒वतेन्द्रो॑ दे॒वता॑ दे॒वतेन्द्रो॑ दे॒वता॑ दे॒वतेन्द्रो॑ दे॒वता᳚ । \newline
20. इन्द्रो॑ दे॒वता॑ दे॒वतेन्द्र॒ इन्द्रो॑ दे॒वता॒ वरु॑णो॒ वरु॑णो दे॒वतेन्द्र॒ इन्द्रो॑ दे॒वता॒ वरु॑णः । \newline
21. दे॒वता॒ वरु॑णो॒ वरु॑णो दे॒वता॑ दे॒वता॒ वरु॑णो दे॒वता॑ दे॒वता॒ वरु॑णो दे॒वता॑ दे॒वता॒ वरु॑णो दे॒वता᳚ । \newline
22. वरु॑णो दे॒वता॑ दे॒वता॒ वरु॑णो॒ वरु॑णो दे॒वता॑ मू॒र्द्धा मू॒र्द्धा दे॒वता॒ वरु॑णो॒ वरु॑णो दे॒वता॑ मू॒र्द्धा । \newline
23. दे॒वता॑ मू॒र्द्धा मू॒र्द्धा दे॒वता॑ दे॒वता॑ मू॒र्द्धा ऽस्य॑सि मू॒र्द्धा दे॒वता॑ दे॒वता॑ मू॒र्द्धा ऽसि॑ । \newline
24. मू॒र्द्धा ऽस्य॑सि मू॒र्द्धा मू॒र्द्धा ऽसि॒ राड् राड॑सि मू॒र्द्धा मू॒र्द्धा ऽसि॒ राट् । \newline
25. अ॒सि॒ राड् राड॑ स्यसि॒ राड् ध्रु॒वा ध्रु॒वा राड॑ स्यसि॒ राड् ध्रु॒वा । \newline
26. राड् ध्रु॒वा ध्रु॒वा राड् राड् ध्रु॒वा ऽस्य॑सि ध्रु॒वा राड् राड् ध्रु॒वा ऽसि॑ । \newline
27. ध्रु॒वा ऽस्य॑सि ध्रु॒वा ध्रु॒वा ऽसि॑ ध॒रुणा॑ ध॒रुणा॑ ऽसि ध्रु॒वा ध्रु॒वा ऽसि॑ ध॒रुणा᳚ । \newline
28. अ॒सि॒ ध॒रुणा॑ ध॒रुणा᳚ ऽस्यसि ध॒रुणा॑ य॒न्त्री य॒न्त्री ध॒रुणा᳚ ऽस्यसि ध॒रुणा॑ य॒न्त्री । \newline
29. ध॒रुणा॑ य॒न्त्री य॒न्त्री ध॒रुणा॑ ध॒रुणा॑ य॒न्त्र्य॑ स्यसि य॒न्त्री ध॒रुणा॑ ध॒रुणा॑ य॒न्त्र्य॑सि । \newline
30. य॒न्त्र्य॑ स्यसि य॒न्त्री य॒न्त्र्य॑सि॒ यमि॑त्री॒ यमि॑त्र्यसि य॒न्त्री य॒न्त्र्य॑सि॒ यमि॑त्री । \newline
31. अ॒सि॒ यमि॑त्री॒ यमि॑त्र्य स्यसि॒ यमि॑त्री॒ष इ॒षे यमि॑त्र्य स्यसि॒ यमि॑त्री॒षे । \newline
32. यमि॑त्री॒ष इ॒षे यमि॑त्री॒ यमि॑त्री॒षे त्वा᳚ त्वे॒षे यमि॑त्री॒ यमि॑त्री॒षे त्वा᳚ । \newline
33. इ॒षे त्वा᳚ त्वे॒ष इ॒षे त्वो॒र्ज ऊ॒र्जे त्वे॒ष इ॒षे त्वो॒र्जे । \newline
34. त्वो॒र्ज ऊ॒र्जे त्वा᳚ त्वो॒र्जे त्वा᳚ त्वो॒र्जे त्वा᳚ त्वो॒र्जे त्वा᳚ । \newline
35. ऊ॒र्जे त्वा᳚ त्वो॒र्ज ऊ॒र्जे त्वा॑ कृ॒ष्यै कृ॒ष्यै त्वो॒र्ज ऊ॒र्जे त्वा॑ कृ॒ष्यै । \newline
36. त्वा॒ कृ॒ष्यै कृ॒ष्यै त्वा᳚ त्वा कृ॒ष्यै त्वा᳚ त्वा कृ॒ष्यै त्वा᳚ त्वा कृ॒ष्यै त्वा᳚ । \newline
37. कृ॒ष्यै त्वा᳚ त्वा कृ॒ष्यै कृ॒ष्यै त्वा॒ क्षेमा॑य॒ क्षेमा॑य त्वा कृ॒ष्यै कृ॒ष्यै त्वा॒ क्षेमा॑य । \newline
38. त्वा॒ क्षेमा॑य॒ क्षेमा॑य त्वा त्वा॒ क्षेमा॑य त्वा त्वा॒ क्षेमा॑य त्वा त्वा॒ क्षेमा॑य त्वा । \newline
39. क्षेमा॑य त्वा त्वा॒ क्षेमा॑य॒ क्षेमा॑य त्वा॒ यन्त्री॒ यन्त्री᳚ त्वा॒ क्षेमा॑य॒ क्षेमा॑य त्वा॒ यन्त्री᳚ । \newline
40. त्वा॒ यन्त्री॒ यन्त्री᳚ त्वा त्वा॒ यन्त्री॒ राड् राड् यन्त्री᳚ त्वा त्वा॒ यन्त्री॒ राट् । \newline
41. यन्त्री॒ राड् राड् यन्त्री॒ यन्त्री॒ राड् ध्रु॒वा ध्रु॒वा राड् यन्त्री॒ यन्त्री॒ राड् ध्रु॒वा । \newline
42. राड् ध्रु॒वा ध्रु॒वा राड् राड् ध्रु॒वा ऽस्य॑सि ध्रु॒वा राड् राड् ध्रु॒वा ऽसि॑ । \newline
43. ध्रु॒वा ऽस्य॑सि ध्रु॒वा ध्रु॒वा ऽसि॒ धर॑णी॒ धर॑ण्यसि ध्रु॒वा ध्रु॒वा ऽसि॒ धर॑णी । \newline
44. अ॒सि॒ धर॑णी॒ धर॑ण्य स्यसि॒ धर॑णी ध॒र्त्री ध॒र्त्री धर॑ण्य स्यसि॒ धर॑णी ध॒र्त्री । \newline
45. धर॑णी ध॒र्त्री ध॒र्त्री धर॑णी॒ धर॑णी ध॒र्त्र्य॑ स्यसि ध॒र्त्री धर॑णी॒ धर॑णी ध॒र्त्र्य॑सि । \newline
46. ध॒र्त्र्य॑ स्यसि ध॒र्त्री ध॒र्त्र्य॑सि॒ धरि॑त्री॒ धरि॑त्र्यसि ध॒र्त्री ध॒र्त्र्य॑सि॒ धरि॑त्री । \newline
47. अ॒सि॒ धरि॑त्री॒ धरि॑त्र्य स्यसि॒ धरि॒त्र्या यु॑ष॒ आयु॑षे॒ धरि॑त्र्य स्यसि॒ धरि॒त्र्या यु॑षे । \newline
48. धरि॒त्र्या यु॑ष॒ आयु॑षे॒ धरि॑त्री॒ धरि॒त्र्या यु॑षे त्वा॒ त्वा ऽऽयु॑षे॒ धरि॑त्री॒ धरि॒त्र्या यु॑षे त्वा । \newline
49. आयु॑षे त्वा॒ त्वा ऽऽयु॑ष॒ आयु॑षे त्वा॒ वर्च॑से॒ वर्च॑से॒ त्वा ऽऽयु॑ष॒ आयु॑षे त्वा॒ वर्च॑से । \newline
50. त्वा॒ वर्च॑से॒ वर्च॑से त्वा त्वा॒ वर्च॑से त्वा त्वा॒ वर्च॑से त्वा त्वा॒ वर्च॑से त्वा । \newline
51. वर्च॑से त्वा त्वा॒ वर्च॑से॒ वर्च॑से॒ त्वौज॑स॒ ओज॑से त्वा॒ वर्च॑से॒ वर्च॑से॒ त्वौज॑से । \newline
52. त्वौज॑स॒ ओज॑से त्वा॒ त्वौज॑से त्वा॒ त्वौज॑से त्वा॒ त्वौज॑से त्वा । \newline
53. ओज॑से त्वा॒ त्वौज॑स॒ ओज॑से त्वा॒ बला॑य॒ बला॑य॒ त्वौज॑स॒ ओज॑से त्वा॒ बला॑य । \newline
54. त्वा॒ बला॑य॒ बला॑य त्वा त्वा॒ बला॑य त्वा त्वा॒ बला॑य त्वा त्वा॒ बला॑य त्वा । \newline
55. बला॑य त्वा त्वा॒ बला॑य॒ बला॑य त्वा । \newline
56. त्वेति॑ त्वा । \newline
\pagebreak
\markright{ TS 4.3.8.1  \hfill https://www.vedavms.in \hfill}

\section{ TS 4.3.8.1 }

\textbf{TS 4.3.8.1 } \newline
\textbf{Samhita Paata} \newline

आ॒शुस्त्रि॒वृद्-भा॒न्तः प॑ञ्चद॒शो व्यो॑म सप्तद॒शः प्रतू᳚र्तिरष्टाद॒श स्तपो॑ नवद॒शो॑ ऽभिव॒र्तः स॑विꣳ॒॒शो ध॒रुण॑ एकविꣳ॒॒शो वर्चो᳚ द्वाविꣳ॒॒शः सं॒भर॑णस्त्रयोविꣳ॒॒शो योनि॑श्चतुर्विꣳ॒॒शो गर्भाः᳚ पञ्चविꣳ॒॒श ओज॑स्त्रिण॒वः क्रतु॑रेकत्रिꣳ॒॒शः प्र॑ति॒ष्ठा त्र॑यस्त्रिꣳ॒॒शो ब्र॒द्ध्नस्य॑ वि॒ष्टपं॑ चतुस्त्रिꣳ॒॒शो नाकः॑ षट्त्रिꣳ॒॒शो वि॑व॒र्तो᳚ऽष्टाचत्वारिꣳ॒॒शो ध॒र्त्रश्च॑तुष्टो॒मः ॥ \newline

\textbf{Pada Paata} \newline

आ॒शुः । त्रि॒वृदिति॑ त्रि-वृत् । भा॒न्तः । प॒ञ्च॒द॒श इति॑ पञ्च-द॒शः । व्यो॑मेति॒ वि - ओ॒म॒ । स॒प्त॒द॒श इति॑ सप्त - द॒शः । प्रतू᳚र्ति॒रिति॒ प्र-तू॒र्तिः॒ । अ॒ष्टा॒द॒श इत्य॑ष्टा-द॒शः । तपः॑ । न॒व॒द॒श इति॑ नव-द॒शः । अ॒भि॒व॒र्त इत्य॑भि - व॒र्तः । स॒विꣳ॒॒श इति॑ स - विꣳ॒॒शः । ध॒रुणः॑ । ए॒क॒विꣳ॒॒श इत्ये॑क - विꣳ॒॒शः । वर्चः॑ । द्वा॒विꣳ॒॒शः । स॒भंर॑ण॒ इति॑ सं - भर॑णः । त्र॒यो॒विꣳ॒॒श इति॑ त्रयः - विꣳ॒॒शः । योनिः॑ । च॒तु॒र्विꣳ॒॒श इति॑ चतुः - विꣳ॒॒शः । गर्भाः᳚ । प॒ञ्च॒विꣳ॒॒श इति॑ पञ्च - विꣳ॒॒शः । ओजः॑ । त्रि॒ण॒व इति॑ त्रि - न॒वः । क्रतुः॑ । ए॒क॒त्रिꣳ॒॒श इत्ये॑क - त्रिꣳ॒॒शः । प्र॒ति॒ष्ठेति॑ प्रति - स्था । त्र॒य॒स्त्रिꣳ॒॒श इति॑ त्रयः - त्रिꣳ॒॒शः । ब्र॒द्ध्नस्य॑ । वि॒ष्टप᳚म् । च॒तु॒स्त्रिꣳ॒॒श इति॑ चतुः - त्रिꣳ॒॒शः । नाकः॑ । ष॒ट्त्रिꣳ॒॒श इति॑ षट् - त्रिꣳ॒॒शः । वि॒व॒र्त इति॑ वि - व॒र्तः । अ॒ष्टा॒च॒त्वा॒रिꣳ॒॒श इत्य॑ष्टा-च॒त्वा॒रिꣳ॒॒शः । ध॒र्त्रः । च॒तु॒ष्टो॒म इति॑ चतुः-स्तो॒मः ॥  \newline


\textbf{Krama Paata} \newline

आ॒शुस्त्रि॒वृत् । त्रि॒वृद् भा॒न्तः । त्रि॒वृदिति॑ त्रि - वृत् । भा॒न्तः प॑ञ्चद॒शः । प॒ञ्च॒द॒शो व्यो॑म । प॒ञ्च॒द॒श इति॑ पञ्च - द॒शः । व्यो॑म सप्तद॒शः । व्यो॑मेति॒ वि - ओ॒म॒ । स॒प्त॒द॒शः प्रतू᳚र्तिः । स॒प्त॒द॒श इति॑ सप्त - द॒शः । प्रतू᳚र्तिरष्टाद॒शः । प्रतू᳚र्ति॒रिति॒ प्र - तू॒र्तिः॒ । अ॒ष्टा॒द॒श स्तपः॑ । अ॒ष्टा॒द॒श इत्य॑ष्टा - द॒शः । तपो॑ नवद॒शः । न॒व॒द॒शो॑ऽभिव॒र्तः । न॒व॒द॒श इति॑ नव - द॒शः । अ॒भि॒व॒र्तः स॑विꣳ॒॒शः । अ॒भि॒व॒र्त इत्य॑भि - व॒र्तः । स॒विꣳ॒॒शो ध॒रुणः॑ । स॒विꣳ॒॒श इति॑ स - विꣳ॒॒शः । ध॒रुण॑ एकविꣳ॒॒शः । ए॒क॒विꣳ॒॒शो वर्चः॑ । ए॒क॒विꣳ॒॒श इत्ये॑क - विꣳ॒॒शः । वर्चो᳚ द्वाविꣳ॒॒शः । द्वा॒विꣳ॒॒शः स॒म्भर॑णः । स॒म्भर॑णस्त्रयोविꣳ॒॒शः । स॒म्भर॑ण॒ इति॑ सम् - भर॑णः । त्र॒यो॒विꣳ॒॒शो योनिः॑ । त्र॒यो॒विꣳ॒॒श इति॑ त्रयः - विꣳ॒॒शः । योनि॑श्चतुर्विꣳ॒॒शः । च॒तु॒र्विꣳ॒॒शो गर्भाः᳚ । च॒तु॒र्विꣳ॒॒श इति॑ चतुः - विꣳ॒॒शः । गर्भाः᳚ पञ्चविꣳ॒॒शः । प॒ञ्च॒विꣳ॒॒श ओजः॑ । प॒ञ्च॒विꣳ॒॒श इति॑ पञ्च - विꣳ॒॒शः । ओज॑स्त्रिण॒वः । त्रि॒ण॒वः क्रतुः॑ । त्रि॒ण॒व इति॑ त्रि - न॒वः । क्रतु॑रेकत्रिꣳ॒॒शः । ए॒क॒त्रिꣳ॒॒शः प्र॑ति॒ष्ठा । ए॒क॒त्रिꣳ॒॒श इत्ये॑क - त्रिꣳ॒॒शः । प्र॒ति॒ष्ठा त्र॑यस्त्रिꣳ॒॒शः । प्र॒ति॒ष्ठेति॑ प्रति - स्था । त्र॒य॒स्त्रिꣳ॒॒शो ब्र॒द्ध्नस्य॑ । त्र॒य॒स्त्रिꣳ॒॒श इति॑ त्रयः - त्रिꣳ॒॒शः । ब्र॒द्ध्नस्य॑ वि॒ष्टप᳚म् । वि॒ष्टप॑म् चतुस्त्रिꣳ॒॒शः । च॒तु॒स्तिꣳ॒॒शो नाकः॑ । च॒तु॒स्त्रिꣳ॒॒श इति॑ चतुः - त्रिꣳ॒॒शः । नाकः॑ षट्त्रिꣳ॒॒शः । ष॒ट्त्रिꣳ॒॒शो वि॑व॒र्तः । ष॒ट्॒त्रिꣳ॒॒श इति॑ षट् - त्रिꣳ॒॒शः । वि॒व॒र्तो᳚ऽष्टाचत्वारिꣳ॒॒शः । वि॒व॒र्त इति॑ वि - व॒र्तः । अ॒ष्टा॒च॒त्वा॒रिꣳ॒॒शो ध॒र्त्रः । अ॒ष्टा॒च॒त्वा॒रिꣳ॒॒श इत्य॑ष्टा - च॒त्वा॒रिꣳ॒॒शः । ध॒र्त्रश्च॑तुष्टो॒मः । च॒तु॒ष्टो॒म इति॑ चतुः - स्तो॒मः । \newline

\textbf{Jatai Paata} \newline

1. आ॒शु स्त्रि॒वृत् त्रि॒वृ दा॒शु रा॒शु स्त्रि॒वृत् । \newline
2. त्रि॒वृद् भा॒न्तो भा॒न्त स्त्रि॒वृत् त्रि॒वृद् भा॒न्तः । \newline
3. त्रि॒वृदिति॑ त्रि - वृत् । \newline
4. भा॒न्तः प॑ञ्चद॒शः प॑ञ्चद॒शो भा॒न्तो भा॒न्तः प॑ञ्चद॒शः । \newline
5. प॒ञ्च॒द॒शो व्यो॑म॒ व्यो॑म पञ्चद॒शः प॑ञ्चद॒शो व्यो॑म । \newline
6. प॒ञ्च॒द॒श इति॑ पञ्च - द॒शः । \newline
7. व्यो॑म सप्तद॒शः स॑प्तद॒शो व्यो॑म॒ व्यो॑म सप्तद॒शः । \newline
8. व्यो॑मेति॒ वि - ओ॒म॒ । \newline
9. स॒प्त॒द॒शः प्रतू᳚र्तिः॒ प्रतू᳚र्तिः सप्तद॒शः स॑प्तद॒शः प्रतू᳚र्तिः । \newline
10. स॒प्त॒द॒श इति॑ सप्त - द॒शः । \newline
11. प्रतू᳚र्ति रष्टाद॒शो᳚ ऽष्टाद॒शः प्रतू᳚र्तिः॒ प्रतू᳚र्ति रष्टाद॒शः । \newline
12. प्रतू᳚र्ति॒रिति॒ प्र - तू॒र्तिः॒ । \newline
13. अ॒ष्टा॒द॒श स्तप॒ स्तपो᳚ ऽष्टाद॒शो᳚ ऽष्टाद॒श स्तपः॑ । \newline
14. अ॒ष्टा॒द॒श इत्य॑ष्टा - द॒शः । \newline
15. तपो॑ नवद॒शो न॑वद॒श स्तप॒ स्तपो॑ नवद॒शः । \newline
16. न॒व॒द॒शो॑ ऽभिव॒र्तो॑ ऽभिव॒र्तो न॑वद॒शो न॑वद॒शो॑ ऽभिव॒र्तः । \newline
17. न॒व॒द॒श इति॑ नव - द॒शः । \newline
18. अ॒भि॒व॒र्तः स॑विꣳ॒॒शः स॑विꣳ॒॒शो॑ ऽभिव॒र्तो॑ ऽभिव॒र्तः स॑विꣳ॒॒शः । \newline
19. अ॒भि॒व॒र्त इत्य॑भि - व॒र्तः । \newline
20. स॒विꣳ॒॒शो ध॒रुणो॑ ध॒रुणः॑ सविꣳ॒॒शः स॑विꣳ॒॒शो ध॒रुणः॑ । \newline
21. स॒विꣳ॒॒श इति॑ स - विꣳ॒॒शः । \newline
22. ध॒रुण॑ एकविꣳ॒॒श ए॑कविꣳ॒॒शो ध॒रुणो॑ ध॒रुण॑ एकविꣳ॒॒शः । \newline
23. ए॒क॒विꣳ॒॒शो वर्चो॒ वर्च॑ एकविꣳ॒॒श ए॑कविꣳ॒॒शो वर्चः॑ । \newline
24. ए॒क॒विꣳ॒॒श इत्ये॑क - विꣳ॒॒शः । \newline
25. वर्चो᳚ द्वाविꣳ॒॒शो द्वा॑विꣳ॒॒शो वर्चो॒ वर्चो᳚ द्वाविꣳ॒॒शः । \newline
26. द्वा॒विꣳ॒॒शः सं॒भर॑णः सं॒भर॑णो द्वाविꣳ॒॒शो द्वा॑विꣳ॒॒शः सं॒भर॑णः । \newline
27. सं॒भर॑ण स्त्रयोविꣳ॒॒श स्त्र॑योविꣳ॒॒शः सं॒भर॑णः सं॒भर॑ण स्त्रयोविꣳ॒॒शः । \newline
28. सं॒भर॑ण॒ इति॑ सं - भर॑णः । \newline
29. त्र॒यो॒विꣳ॒॒शो योनि॒र् योनि॑ स्त्रयोविꣳ॒॒श स्त्र॑योविꣳ॒॒शो योनिः॑ । \newline
30. त्र॒यो॒विꣳ॒॒श इति॑ त्रयः - विꣳ॒॒शः । \newline
31. योनि॑ श्चतुर्विꣳ॒॒श श्च॑तुर्विꣳ॒॒शो योनि॒र् योनि॑ श्चतुर्विꣳ॒॒शः । \newline
32. च॒तु॒र्विꣳ॒॒शो गर्भा॒ गर्भा᳚ श्चतुर्विꣳ॒॒श श्च॑तुर्विꣳ॒॒शो गर्भाः᳚ । \newline
33. च॒तु॒र्विꣳ॒॒श इति॑ चतुः - विꣳ॒॒शः । \newline
34. गर्भाः᳚ पञ्चविꣳ॒॒शः प॑ञ्चविꣳ॒॒शो गर्भा॒ गर्भाः᳚ पञ्चविꣳ॒॒शः । \newline
35. प॒ञ्च॒विꣳ॒॒श ओज॒ ओजः॑ पञ्चविꣳ॒॒शः प॑ञ्चविꣳ॒॒श ओजः॑ । \newline
36. प॒ञ्च॒विꣳ॒॒श इति॑ पञ्च - विꣳ॒॒शः । \newline
37. ओज॑ स्त्रिण॒व स्त्रि॑ण॒व ओज॒ ओज॑ स्त्रिण॒वः । \newline
38. त्रि॒ण॒वः क्रतुः॒ क्रतु॑ स्त्रिण॒व स्त्रि॑ण॒वः क्रतुः॑ । \newline
39. त्रि॒ण॒व इति॑ त्रि - न॒वः । \newline
40. क्रतु॑ रेकत्रिꣳ॒॒श ए॑कत्रिꣳ॒॒शः क्रतुः॒ क्रतु॑ रेकत्रिꣳ॒॒शः । \newline
41. ए॒क॒त्रिꣳ॒॒शः प्र॑ति॒ष्ठा प्र॑ति॒ष्ठैक॑त्रिꣳ॒॒श ए॑कत्रिꣳ॒॒शः प्र॑ति॒ष्ठा । \newline
42. ए॒क॒त्रिꣳ॒॒श इत्ये॑क - त्रिꣳ॒॒शः । \newline
43. प्र॒ति॒ष्ठा त्र॑यस्त्रिꣳ॒॒श स्त्र॑यस्त्रिꣳ॒॒शः प्र॑ति॒ष्ठा प्र॑ति॒ष्ठा त्र॑यस्त्रिꣳ॒॒शः । \newline
44. प्र॒ति॒ष्ठेति॑ प्रति - स्था । \newline
45. त्र॒य॒स्त्रिꣳ॒॒शो ब्र॒द्ध्नस्य॑ ब्र॒द्ध्नस्य॑ त्रयस्त्रिꣳ॒॒श स्त्र॑यस्त्रिꣳ॒॒शो ब्र॒द्ध्नस्य॑ । \newline
46. त्र॒य॒स्त्रिꣳ॒॒श इति॑ त्रयः - त्रिꣳ॒॒शः । \newline
47. ब्र॒द्ध्नस्य॑ वि॒ष्टपं॑ ॅवि॒ष्टप॑म् ब्र॒द्ध्नस्य॑ ब्र॒द्ध्नस्य॑ वि॒ष्टप᳚म् । \newline
48. वि॒ष्टप॑म् चतुस्त्रिꣳ॒॒श श्च॑तुस्त्रिꣳ॒॒शो वि॒ष्टपं॑ ॅवि॒ष्टप॑म् चतुस्त्रिꣳ॒॒शः । \newline
49. च॒तु॒स्त्रिꣳ॒॒शो नाको॒ नाक॑ श्चतुस्त्रिꣳ॒॒श श्च॑तुस्त्रिꣳ॒॒शो नाकः॑ । \newline
50. च॒तु॒स्त्रिꣳ॒॒श इति॑ चतुः - त्रिꣳ॒॒शः । \newline
51. नाक॑ ष्षट्त्रिꣳ॒॒श ष्ष॑ट्त्रिꣳ॒॒शो नाको॒ नाक॑ष्षट्त्रिꣳ॒॒शः । \newline
52. ष॒ट्त्रिꣳ॒॒शो वि॑व॒र्तो वि॑व॒र्त ष्ष॑ट्त्रिꣳ॒॒श ष्ष॑ट्त्रिꣳ॒॒शो वि॑व॒र्तः । \newline
53. ष॒ट्त्रिꣳ॒॒श इति॑ षट् - त्रिꣳ॒॒शः । \newline
54. वि॒व॒र्तो᳚ ऽष्टाचत्वारिꣳ॒॒शो᳚ ऽष्टाचत्वारिꣳ॒॒शो वि॑व॒र्तो वि॑व॒र्तो᳚ ऽष्टाचत्वारिꣳ॒॒शः । \newline
55. वि॒व॒र्त इति॑ वि - व॒र्तः । \newline
56. अ॒ष्टा॒च॒त्वा॒रिꣳ॒॒शो ध॒र्त्रो ध॒र्त्रो᳚ ऽष्टाचत्वारिꣳ॒॒शो᳚ ऽष्टाचत्वारिꣳ॒॒शो ध॒र्त्रः । \newline
57. अ॒ष्टा॒च॒त्वा॒रिꣳ॒॒श इत्य॑ष्टा - च॒त्वा॒रिꣳ॒॒शः । \newline
58. ध॒र्त्र श्च॑तुष्टो॒म श्च॑तुष्टो॒मो ध॒र्त्रो ध॒र्त्र श्च॑तुष्टो॒मः । \newline
59. च॒तु॒ष्टो॒म इति॑ चतुः - स्तो॒मः । \newline

\textbf{Ghana Paata } \newline

1. आ॒शु स्त्रि॒वृत् त्रि॒वृ दा॒शु रा॒शु स्त्रि॒वृद् भा॒न्तो भा॒न्त स्त्रि॒वृ दा॒शु रा॒शु स्त्रि॒वृद् भा॒न्तः । \newline
2. त्रि॒वृद् भा॒न्तो भा॒न्त स्त्रि॒वृत् त्रि॒वृद् भा॒न्तः प॑ञ्चद॒शः प॑ञ्चद॒शो भा॒न्त स्त्रि॒वृत् त्रि॒वृद् भा॒न्तः प॑ञ्चद॒शः । \newline
3. त्रि॒वृदिति॑ त्रि - वृत् । \newline
4. भा॒न्तः प॑ञ्चद॒शः प॑ञ्चद॒शो भा॒न्तो भा॒न्तः प॑ञ्चद॒शो व्यो॑म॒ व्यो॑म पञ्चद॒शो भा॒न्तो भा॒न्तः प॑ञ्चद॒शो व्यो॑म । \newline
5. प॒ञ्च॒द॒शो व्यो॑म॒ व्यो॑म पञ्चद॒शः प॑ञ्चद॒शो व्यो॑म सप्तद॒शः स॑प्तद॒शो व्यो॑म पञ्चद॒शः प॑ञ्चद॒शो व्यो॑म सप्तद॒शः । \newline
6. प॒ञ्च॒द॒श इति॑ पञ्च - द॒शः । \newline
7. व्यो॑म सप्तद॒शः स॑प्तद॒शो व्यो॑म॒ व्यो॑म सप्तद॒शः प्रतू᳚र्तिः॒ प्रतू᳚र्तिः सप्तद॒शो व्यो॑म॒ व्यो॑म सप्तद॒शः प्रतू᳚र्तिः । \newline
8. व्यो॑मेति॒ वि - ओ॒म॒ । \newline
9. स॒प्त॒द॒शः प्रतू᳚र्तिः॒ प्रतू᳚र्तिः सप्तद॒शः स॑प्तद॒शः प्रतू᳚र्ति रष्टाद॒शो᳚ ऽष्टाद॒शः प्रतू᳚र्तिः सप्तद॒शः स॑प्तद॒शः प्रतू᳚र्ति रष्टाद॒शः । \newline
10. स॒प्त॒द॒श इति॑ सप्त - द॒शः । \newline
11. प्रतू᳚र्ति रष्टाद॒शो᳚ ऽष्टाद॒शः प्रतू᳚र्तिः॒ प्रतू᳚र्ति रष्टाद॒श स्तप॒ स्तपो᳚ ऽष्टाद॒शः प्रतू᳚र्तिः॒ प्रतू᳚र्ति रष्टाद॒श स्तपः॑ । \newline
12. प्रतू᳚र्ति॒रिति॒ प्र - तू॒र्तिः॒ । \newline
13. अ॒ष्टा॒द॒श स्तप॒ स्तपो᳚ ऽष्टाद॒शो᳚ ऽष्टाद॒श स्तपो॑ नवद॒शो न॑वद॒श स्तपो᳚ ऽष्टाद॒शो᳚ ऽष्टाद॒श स्तपो॑ नवद॒शः । \newline
14. अ॒ष्टा॒द॒श इत्य॑ष्टा - द॒शः । \newline
15. तपो॑ नवद॒शो न॑वद॒श स्तप॒ स्तपो॑ नवद॒शो॑ ऽभिव॒र्तो॑ ऽभिव॒र्तो न॑वद॒श स्तप॒ स्तपो॑ नवद॒शो॑ ऽभिव॒र्तः । \newline
16. न॒व॒द॒शो॑ ऽभिव॒र्तो॑ ऽभिव॒र्तो न॑वद॒शो न॑वद॒शो॑ ऽभिव॒र्तः स॑विꣳ॒॒शः स॑विꣳ॒॒शो॑ ऽभिव॒र्तो न॑वद॒शो न॑वद॒शो॑ ऽभिव॒र्तः स॑विꣳ॒॒शः । \newline
17. न॒व॒द॒श इति॑ नव - द॒शः । \newline
18. अ॒भि॒व॒र्तः स॑विꣳ॒॒शः स॑विꣳ॒॒शो॑ ऽभिव॒र्तो॑ ऽभिव॒र्तः स॑विꣳ॒॒शो ध॒रुणो॑ ध॒रुणः॑ सविꣳ॒॒शो॑ ऽभिव॒र्तो॑ ऽभिव॒र्तः स॑विꣳ॒॒शो ध॒रुणः॑ । \newline
19. अ॒भि॒व॒र्त इत्य॑भि - व॒र्तः । \newline
20. स॒विꣳ॒॒शो ध॒रुणो॑ ध॒रुणः॑ सविꣳ॒॒शः स॑विꣳ॒॒शो ध॒रुण॑ एकविꣳ॒॒श ए॑कविꣳ॒॒शो ध॒रुणः॑ सविꣳ॒॒शः स॑विꣳ॒॒शो ध॒रुण॑ एकविꣳ॒॒शः । \newline
21. स॒विꣳ॒॒श इति॑ स - विꣳ॒॒शः । \newline
22. ध॒रुण॑ एकविꣳ॒॒श ए॑कविꣳ॒॒शो ध॒रुणो॑ ध॒रुण॑ एकविꣳ॒॒शो वर्चो॒ वर्च॑ एकविꣳ॒॒शो ध॒रुणो॑ ध॒रुण॑ एकविꣳ॒॒शो वर्चः॑ । \newline
23. ए॒क॒विꣳ॒॒शो वर्चो॒ वर्च॑ एकविꣳ॒॒श ए॑कविꣳ॒॒शो वर्चो᳚ द्वाविꣳ॒॒शो द्वा॑विꣳ॒॒शो वर्च॑ एकविꣳ॒॒श ए॑कविꣳ॒॒शो वर्चो᳚ द्वाविꣳ॒॒शः । \newline
24. ए॒क॒विꣳ॒॒श इत्ये॑क - विꣳ॒॒शः । \newline
25. वर्चो᳚ द्वाविꣳ॒॒शो द्वा॑विꣳ॒॒शो वर्चो॒ वर्चो᳚ द्वाविꣳ॒॒शः सं॒भर॑णः सं॒भर॑णो द्वाविꣳ॒॒शो वर्चो॒ वर्चो᳚ द्वाविꣳ॒॒शः सं॒भर॑णः । \newline
26. द्वा॒विꣳ॒॒शः सं॒भर॑णः सं॒भर॑णो द्वाविꣳ॒॒शो द्वा॑विꣳ॒॒शः सं॒भर॑ण स्त्रयोविꣳ॒॒श स्त्र॑योविꣳ॒॒शः सं॒भर॑णो द्वाविꣳ॒॒शो द्वा॑विꣳ॒॒शः सं॒भर॑ण स्त्रयोविꣳ॒॒शः । \newline
27. सं॒भर॑ण स्त्रयोविꣳ॒॒श स्त्र॑योविꣳ॒॒शः सं॒भर॑णः सं॒भर॑ण स्त्रयोविꣳ॒॒शो योनि॒र् योनि॑ स्त्रयोविꣳ॒॒शः सं॒भर॑णः सं॒भर॑ण स्त्रयोविꣳ॒॒शो योनिः॑ । \newline
28. सं॒भर॑ण॒ इति॑ सं - भर॑णः । \newline
29. त्र॒यो॒विꣳ॒॒शो योनि॒र् योनि॑ स्त्रयोविꣳ॒॒श स्त्र॑योविꣳ॒॒शो योनि॑ श्चतुर्विꣳ॒॒श श्च॑तुर्विꣳ॒॒शो योनि॑ स्त्रयोविꣳ॒॒श स्त्र॑योविꣳ॒॒शो योनि॑ श्चतुर्विꣳ॒॒शः । \newline
30. त्र॒यो॒विꣳ॒॒श इति॑ त्रयः - विꣳ॒॒शः । \newline
31. योनि॑ श्चतुर्विꣳ॒॒श श्च॑तुर्विꣳ॒॒शो योनि॒र् योनि॑ श्चतुर्विꣳ॒॒शो गर्भा॒ गर्भा᳚ श्चतुर्विꣳ॒॒शो योनि॒र् योनि॑ श्चतुर्विꣳ॒॒शो गर्भाः᳚ । \newline
32. च॒तु॒र्विꣳ॒॒शो गर्भा॒ गर्भा᳚ श्चतुर्विꣳ॒॒श श्च॑तुर्विꣳ॒॒शो गर्भाः᳚ पञ्चविꣳ॒॒शः प॑ञ्चविꣳ॒॒शो गर्भा᳚ श्चतुर्विꣳ॒॒श श्च॑तुर्विꣳ॒॒शो गर्भाः᳚ पञ्चविꣳ॒॒शः । \newline
33. च॒तु॒र्विꣳ॒॒श इति॑ चतुः - विꣳ॒॒शः । \newline
34. गर्भाः᳚ पञ्चविꣳ॒॒शः प॑ञ्चविꣳ॒॒शो गर्भा॒ गर्भाः᳚ पञ्चविꣳ॒॒श ओज॒ ओजः॑ पञ्चविꣳ॒॒शो गर्भा॒ गर्भाः᳚ पञ्चविꣳ॒॒श ओजः॑ । \newline
35. प॒ञ्च॒विꣳ॒॒श ओज॒ ओजः॑ पञ्चविꣳ॒॒शः प॑ञ्चविꣳ॒॒श ओज॑ स्त्रिण॒व स्त्रि॑ण॒व ओजः॑ पञ्चविꣳ॒॒शः प॑ञ्चविꣳ॒॒श ओज॑ स्त्रिण॒वः । \newline
36. प॒ञ्च॒विꣳ॒॒श इति॑ पञ्च - विꣳ॒॒शः । \newline
37. ओज॑ स्त्रिण॒व स्त्रि॑ण॒व ओज॒ ओज॑ स्त्रिण॒वः क्रतुः॒ क्रतु॑ स्त्रिण॒व ओज॒ ओज॑ स्त्रिण॒वः क्रतुः॑ । \newline
38. त्रि॒ण॒वः क्रतुः॒ क्रतु॑ स्त्रिण॒व स्त्रि॑ण॒वः क्रतु॑ रेकत्रिꣳ॒॒श ए॑कत्रिꣳ॒॒शः क्रतु॑ स्त्रिण॒व स्त्रि॑ण॒वः क्रतु॑ रेकत्रिꣳ॒॒शः । \newline
39. त्रि॒ण॒व इति॑ त्रि - न॒वः । \newline
40. क्रतु॑ रेकत्रिꣳ॒॒श ए॑कत्रिꣳ॒॒शः क्रतुः॒ क्रतु॑रेकत्रिꣳ॒॒शः प्र॑ति॒ष्ठा प्र॑ति॒ ष्ठैक॑त्रिꣳ॒॒शः क्रतुः॒ क्रतु॑ रेकत्रिꣳ॒॒शः प्र॑ति॒ष्ठा । \newline
41. ए॒क॒त्रिꣳ॒॒शः प्र॑ति॒ष्ठा प्र॑ति॒ ष्ठैक॑त्रिꣳ॒॒श ए॑कत्रिꣳ॒॒शः प्र॑ति॒ष्ठा 
त्र॑यस्त्रिꣳ॒॒श स्त्र॑यस्त्रिꣳ॒॒शः प्र॑ति॒ष्ठै क॑त्रिꣳ॒॒श ए॑कत्रिꣳ॒॒शः प्र॑ति॒ष्ठा त्र॑यस्त्रिꣳ॒॒शः । \newline
42. ए॒क॒त्रिꣳ॒॒श इत्ये॑क - त्रिꣳ॒॒शः । \newline
43. प्र॒ति॒ष्ठा त्र॑यस्त्रिꣳ॒॒श स्त्र॑यस्त्रिꣳ॒॒शः प्र॑ति॒ष्ठा प्र॑ति॒ष्ठा त्र॑यस्त्रिꣳ॒॒शो ब्र॒द्ध्नस्य॑ ब्र॒द्ध्नस्य॑ त्रयस्त्रिꣳ॒॒शः प्र॑ति॒ष्ठा प्र॑ति॒ष्ठा त्र॑यस्त्रिꣳ॒॒शो ब्र॒द्ध्नस्य॑ । \newline
44. प्र॒ति॒ष्ठेति॑ प्रति - स्था । \newline
45. त्र॒य॒स्त्रिꣳ॒॒शो ब्र॒द्ध्नस्य॑ ब्र॒द्ध्नस्य॑ त्रयस्त्रिꣳ॒॒श स्त्र॑यस्त्रिꣳ॒॒शो ब्र॒द्ध्नस्य॑ वि॒ष्टपं॑ ॅवि॒ष्टप॑म् ब्र॒द्ध्नस्य॑ त्रयस्त्रिꣳ॒॒श स्त्र॑यस्त्रिꣳ॒॒शो ब्र॒द्ध्नस्य॑ वि॒ष्टप᳚म् । \newline
46. त्र॒य॒स्त्रिꣳ॒॒श इति॑ त्रयः - त्रिꣳ॒॒शः । \newline
47. ब्र॒द्ध्नस्य॑ वि॒ष्टपं॑ ॅवि॒ष्टप॑म् ब्र॒द्ध्नस्य॑ ब्र॒द्ध्नस्य॑ वि॒ष्टप॑म् चतुस्त्रिꣳ॒॒श श्च॑तुस्त्रिꣳ॒॒शो वि॒ष्टप॑म् ब्र॒द्ध्नस्य॑ ब्र॒द्ध्नस्य॑ वि॒ष्टप॑म् चतुस्त्रिꣳ॒॒शः । \newline
48. वि॒ष्टप॑म् चतुस्त्रिꣳ॒॒श श्च॑तुस्त्रिꣳ॒॒शो वि॒ष्टपं॑ ॅवि॒ष्टप॑म् चतुस्त्रिꣳ॒॒शो नाको॒ नाक॑श्चतु स्त्रिꣳ॒॒शो वि॒ष्टपं॑ ॅवि॒ष्टप॑म् चतुस्त्रिꣳ॒॒शो नाकः॑ । \newline
49. च॒तु॒स्त्रिꣳ॒॒शो नाको॒ नाक॑ श्चतुस्त्रिꣳ॒॒श श्च॑तुस्त्रिꣳ॒॒शो नाक॑ ष्षट्त्रिꣳ॒॒श ष्ष॑ट्त्रिꣳ॒॒शो नाक॑श् चतुस्त्रिꣳ॒॒श श्च॑तुस्त्रिꣳ॒॒शो नाक॑ ष्षट्त्रिꣳ॒॒शः । \newline
50. च॒तु॒स्त्रिꣳ॒॒श इति॑ चतुः - त्रिꣳ॒॒शः । \newline
51. नाक॑ ष्षट्त्रिꣳ॒॒श ष्ष॑ट्त्रिꣳ॒॒शो नाको॒ नाक॑ ष्षट्त्रिꣳ॒॒शो वि॑व॒र्तो वि॑व॒र्त ष्ष॑ट्त्रिꣳ॒॒शो नाको॒ नाक॑ ष्षट्त्रिꣳ॒॒शो वि॑व॒र्तः । \newline
52. ष॒ट्त्रिꣳ॒॒शो वि॑व॒र्तो वि॑व॒र्त ष्ष॑ट्त्रिꣳ॒॒श ष्ष॑ट्त्रिꣳ॒॒शो वि॑व॒र्तो᳚ ऽष्टाचत्वारिꣳ॒॒शो᳚ ऽष्टाचत्वारिꣳ॒॒शो वि॑व॒र्त ष्ष॑ट्त्रिꣳ॒॒श ष्ष॑ट्त्रिꣳ॒॒शो वि॑व॒र्तो᳚ ऽष्टाचत्वारिꣳ॒॒शः । \newline
53. ष॒ट्त्रिꣳ॒॒श इति॑ षट् - त्रिꣳ॒॒शः । \newline
54. वि॒व॒र्तो᳚ ऽष्टाचत्वारिꣳ॒॒शो᳚ ऽष्टाचत्वारिꣳ॒॒शो वि॑व॒र्तो वि॑व॒र्तो᳚ ऽष्टाचत्वारिꣳ॒॒शो ध॒र्त्रो ध॒र्त्रो᳚ ऽष्टाचत्वारिꣳ॒॒शो वि॑व॒र्तो वि॑व॒र्तो᳚ ऽष्टाचत्वारिꣳ॒॒शो ध॒र्त्रः । \newline
55. वि॒व॒र्त इति॑ वि - व॒र्तः । \newline
56. अ॒ष्टा॒च॒त्वा॒रिꣳ॒॒शो ध॒र्त्रो ध॒र्त्रो᳚ ऽष्टाचत्वारिꣳ॒॒शो᳚ ऽष्टाचत्वारिꣳ॒॒शो ध॒र्त्र श्च॑तुष्टो॒म श्च॑तुष्टो॒मो ध॒र्त्रो᳚ ऽष्टाचत्वारिꣳ॒॒शो᳚ ऽष्टाचत्वारिꣳ॒॒शो ध॒र्त्र श्च॑तुष्टो॒मः । \newline
57. अ॒ष्टा॒च॒त्वा॒रिꣳ॒॒श इत्य॑ष्टा - च॒त्वा॒रिꣳ॒॒शः । \newline
58. ध॒र्त्र श्च॑तुष्टो॒म श्च॑तुष्टो॒मो ध॒र्त्रो ध॒र्त्र श्च॑तुष्टो॒मः । \newline
59. च॒तु॒ष्टो॒म इति॑ चतुः - स्तो॒मः । \newline
\pagebreak
\markright{ TS 4.3.9.1  \hfill https://www.vedavms.in \hfill}

\section{ TS 4.3.9.1 }

\textbf{TS 4.3.9.1 } \newline
\textbf{Samhita Paata} \newline

अ॒ग्नेर्भा॒गो॑ऽसि दी॒क्षाया॒ आधि॑पत्यं॒ ब्रह्म॑ स्पृ॒तं त्रि॒वृथ् स्तोम॒ इन्द्र॑स्य भा॒गो॑ऽसि॒ विष्णो॒राधि॑पत्यं क्ष॒त्रꣳ स्पृ॒तं प॑ञ्चद॒शः स्तोमो॑ नृ॒चक्ष॑सां भा॒गो॑ऽसि धा॒तुराधि॑पत्यं ज॒नित्रꣳ॑ स्पृ॒तꣳ स॑प्तद॒शः स्तोमो॑ मि॒त्रस्य॑ भा॒गो॑ऽसि॒ वरु॑ण॒स्याऽऽधि॑पत्यं दि॒वो वृ॒ष्टिर्वाताः᳚ स्पृ॒ता ए॑कविꣳ॒॒शः स्तोमोऽदि॑त्यै भा॒गो॑ऽसि पू॒ष्ण आधि॑पत्य॒मोजः॑ स्पृ॒तं त्रि॑ण॒वः स्तोमो॒ वसू॑नां भा॒गो॑ऽसि- [  ] \newline

\textbf{Pada Paata} \newline

अ॒ग्नेः । भा॒गः । अ॒सि॒ । दी॒क्षायाः᳚ । आधि॑पत्य॒मित्याधि॑ - प॒त्य॒म् । ब्रह्म॑ । स्पृ॒तम् । त्रि॒वृदिति॑ त्रि - वृत् । स्तोमः॑ । इन्द्र॑स्य । भा॒गः । अ॒सि॒ । विष्णोः᳚ । आधि॑पत्य॒मित्याधि॑ - प॒त्य॒म् । क्ष॒त्रम् । स्पृ॒तम् । प॒ञ्च॒द॒श इति॑ पञ्च - द॒शः । स्तोमः॑ । नृ॒चक्ष॑सा॒मिति॑ नृ - चक्ष॑साम् । भा॒गः । अ॒सि॒ । धा॒तुः । आधि॑पत्य॒मित्याधि॑ - प॒त्य॒म् । ज॒नित्र᳚म् । स्पृ॒तम् । स॒प्त॒द॒श इति॑ सप्त - द॒शः । स्तोमः॑ । मि॒त्रस्य॑ । भा॒गः । अ॒सि॒ । वरु॑णस्य । आधि॑पत्य॒मित्याधि॑ - प॒त्य॒म् । दि॒वः । वृ॒ष्टिः । वाताः᳚ । स्पृ॒ताः । ए॒क॒विꣳ॒॒श इत्ये॑क - विꣳ॒॒शः । स्तोमः॑ । अदि॑त्यै । भा॒गः । अ॒सि॒ । पू॒ष्णः । आधि॑पत्य॒मित्याधि॑ - प॒त्य॒म् । ओजः॑ । स्पृ॒तम् । त्रि॒ण॒व इति॑ त्रि - न॒वः । स्तोमः॑ । वसू॑नाम् । भा॒गः । अ॒सि॒ ।  \newline


\textbf{Krama Paata} \newline

अ॒ग्नेर् भा॒गः । भा॒गो॑ऽसि । अ॒सि॒ दी॒क्षायाः᳚ । दी॒क्षाया॒ आधि॑पत्यम् । आधि॑पत्य॒म् ब्रह्म॑ । आधि॑पत्य॒मित्याधि॑ - प॒त्य॒म् । ब्रह्म॑ स्पृ॒तम् । स्पृ॒तम् त्रि॒वृत् । त्रि॒वृथ् स्तोमः॑ । त्रि॒वृदिति॑ त्रि - वृत् । स्तोम॒ इन्द्र॑स्य । इन्द्र॑स्य भा॒गः । भा॒गो॑ऽसि । अ॒सि॒ विष्णोः᳚ । विष्णो॒राधि॑पत्यम् । आधि॑पत्यम् क्ष॒त्रम् । आधि॑पत्य॒मित्याधि॑ - प॒त्य॒म् । क्ष॒त्रꣳ स्पृ॒तम् । स्पृ॒तम् प॑ञ्चद॒शः । प॒ञ्च॒द॒शः स्तोमः॑ । प॒ञ्च॒द॒श इति॑ पञ्च - द॒शः । स्तोमो॑ नृ॒चक्ष॑साम् । नृ॒चक्ष॑साम् भा॒गः । नृ॒चक्ष॑सा॒मिति॑ नृ - चक्ष॑साम् । भा॒गो॑ऽसि । अ॒सि॒ धा॒तुः । धा॒तुराधि॑पत्यम् । आधि॑पत्यम् ज॒नित्र᳚म् । आधि॑पत्य॒मित्याधि॑ - प॒त्य॒म् । ज॒नित्रꣳ॑ स्पृ॒तम् । स्पृ॒तꣳ स॑प्तद॒शः । स॒प्त॒द॒शः स्तोमः॑ । स॒प्त॒द॒श इति॑ सप्त - द॒शः । स्तोमो॑ मि॒त्रस्य॑ । मि॒त्रस्य॑ भा॒गः । भा॒गो॑ऽसि । अ॒सि॒ वरु॑णस्य । वरु॑ण॒स्याधि॑पत्यम् । आधि॑पत्यम् दि॒वः । आधि॑पत्य॒मित्याधि॑ - प॒त्य॒म् । दि॒वो वृ॒ष्टिः । वृ॒ष्टिर् वाताः᳚ । वाताः᳚ स्पृ॒ताः । स्पृ॒ता ए॑कविꣳ॒॒शः । ए॒क॒विꣳ॒॒शः स्तोमः॑ । ए॒क॒विꣳ॒॒श इत्ये॑क - विꣳ॒॒शः । स्तोमोऽदि॑त्यै । अदि॑त्यै भा॒गः । भा॒गो॑ऽसि । अ॒सि॒ पू॒ष्णः । पू॒ष्ण आधि॑पत्यम् । आधि॑पत्य॒मोजः॑ । आधि॑पत्य॒मित्याधि॑ - प॒त्य॒म् । ओजः॑ स्पृ॒तम् । स्पृ॒तम् त्रि॑ण॒वः । त्रि॒ण॒वः स्तोमः॑ । त्रि॒ण॒व इति॑ त्रि - न॒वः । स्तोमो॒ वसू॑नाम् । वसू॑नाम् भा॒गः । भा॒गो॑ऽसि ( ) । अ॒सि॒ रु॒द्राणा᳚म् \newline

\textbf{Jatai Paata} \newline

1. अ॒ग्नेर् भा॒गो भा॒गो᳚ ऽग्ने र॒ग्नेर् भा॒गः । \newline
2. भा॒गो᳚ ऽस्यसि भा॒गो भा॒गो॑ ऽसि । \newline
3. अ॒सि॒ दी॒क्षाया॑ दी॒क्षाया॑ अस्यसि दी॒क्षायाः᳚ । \newline
4. दी॒क्षाया॒ आधि॑पत्य॒ माधि॑पत्यम् दी॒क्षाया॑ दी॒क्षाया॒ आधि॑पत्यम् । \newline
5. आधि॑पत्य॒म् ब्रह्म॒ ब्रह्माधि॑पत्य॒ माधि॑पत्य॒म् ब्रह्म॑ । \newline
6. आधि॑पत्य॒मित्याधि॑ - प॒त्य॒म् । \newline
7. ब्रह्म॑ स्पृ॒तꣳ स्पृ॒तम् ब्रह्म॒ ब्रह्म॑ स्पृ॒तम् । \newline
8. स्पृ॒तम् त्रि॒वृत् त्रि॒वृथ् स्पृ॒तꣳ स्पृ॒तम् त्रि॒वृत् । \newline
9. त्रि॒वृथ् स्तोमः॒ स्तोम॑ स्त्रि॒वृत् त्रि॒वृथ् स्तोमः॑ । \newline
10. त्रि॒वृदिति॑ त्रि - वृत् । \newline
11. स्तोम॒ इन्द्र॒ स्येन्द्र॑स्य॒ स्तोमः॒ स्तोम॒ इन्द्र॑स्य । \newline
12. इन्द्र॑स्य भा॒गो भा॒ग इन्द्र॒ स्येन्द्र॑स्य भा॒गः । \newline
13. भा॒गो᳚ ऽस्यसि भा॒गो भा॒गो॑ ऽसि । \newline
14. अ॒सि॒ विष्णो॒र् विष्णो॑ रस्यसि॒ विष्णोः᳚ । \newline
15. विष्णो॒ राधि॑पत्य॒ माधि॑पत्यं॒ ॅविष्णो॒र् विष्णो॒ राधि॑पत्यम् । \newline
16. आधि॑पत्यम् क्ष॒त्रम् क्ष॒त्र माधि॑पत्य॒ माधि॑पत्यम् क्ष॒त्रम् । \newline
17. आधि॑पत्य॒मित्याधि॑ - प॒त्य॒म् । \newline
18. क्ष॒त्रꣳ स्पृ॒तꣳ स्पृ॒तम् क्ष॒त्रम् क्ष॒त्रꣳ स्पृ॒तम् । \newline
19. स्पृ॒तम् प॑ञ्चद॒शः प॑ञ्चद॒शः स्पृ॒तꣳ स्पृ॒तम् प॑ञ्चद॒शः । \newline
20. प॒ञ्च॒द॒शः स्तोमः॒ स्तोमः॑ पञ्चद॒शः प॑ञ्चद॒शः स्तोमः॑ । \newline
21. प॒ञ्च॒द॒श इति॑ पञ्च - द॒शः । \newline
22. स्तोमो॑ नृ॒चक्ष॑सान् नृ॒चक्ष॑साꣳ॒॒ स्तोमः॒ स्तोमो॑ नृ॒चक्ष॑साम् । \newline
23. नृ॒चक्ष॑साम् भा॒गो भा॒गो नृ॒चक्ष॑साम् नृ॒चक्ष॑साम् भा॒गः । \newline
24. नृ॒चक्ष॑सा॒मिति॑ नृ - चक्ष॑साम् । \newline
25. भा॒गो᳚ ऽस्यसि भा॒गो भा॒गो॑ ऽसि । \newline
26. अ॒सि॒ धा॒तुर् धा॒तु र॑स्यसि धा॒तुः । \newline
27. धा॒तु राधि॑पत्य॒ माधि॑पत्यम् धा॒तुर् धा॒तु राधि॑पत्यम् । \newline
28. आधि॑पत्यम् ज॒नित्र॑म् ज॒नित्र॒ माधि॑पत्य॒ माधि॑पत्यम् ज॒नित्र᳚म् । \newline
29. आधि॑पत्य॒मित्याधि॑ - प॒त्य॒म् । \newline
30. ज॒नित्रꣳ॑ स्पृ॒तꣳ स्पृ॒तम् ज॒नित्र॑म् ज॒नित्रꣳ॑ स्पृ॒तम् । \newline
31. स्पृ॒तꣳ स॑प्तद॒शः स॑प्तद॒शः स्पृ॒तꣳ स्पृ॒तꣳ स॑प्तद॒शः । \newline
32. स॒प्त॒द॒शः स्तोमः॒ स्तोमः॑ सप्तद॒शः स॑प्तद॒शः स्तोमः॑ । \newline
33. स॒प्त॒द॒श इति॑ सप्त - द॒शः । \newline
34. स्तोमो॑ मि॒त्रस्य॑ मि॒त्रस्य॒ स्तोमः॒ स्तोमो॑ मि॒त्रस्य॑ । \newline
35. मि॒त्रस्य॑ भा॒गो भा॒गो मि॒त्रस्य॑ मि॒त्रस्य॑ भा॒गः । \newline
36. भा॒गो᳚ ऽस्यसि भा॒गो भा॒गो॑ ऽसि । \newline
37. अ॒सि॒ वरु॑णस्य॒ वरु॑ण स्यास्यसि॒ वरु॑णस्य । \newline
38. वरु॑ण॒स्या धि॑पत्य॒ माधि॑पत्यं॒ ॅवरु॑णस्य॒ वरु॑ण॒स्या धि॑पत्यम् । \newline
39. आधि॑पत्यम् दि॒वो दि॒व आधि॑पत्य॒ माधि॑पत्यम् दि॒वः । \newline
40. आधि॑पत्य॒मित्याधि॑ - प॒त्य॒म् । \newline
41. दि॒वो वृ॒ष्टिर् वृ॒ष्टिर् दि॒वो दि॒वो वृ॒ष्टिः । \newline
42. वृ॒ष्टिर् वाता॒ वाता॑ वृ॒ष्टिर् वृ॒ष्टिर् वाताः᳚ । \newline
43. वाताः᳚ स्पृ॒ताः स्पृ॒ता वाता॒ वाताः᳚ स्पृ॒ताः । \newline
44. स्पृ॒ता ए॑कविꣳ॒॒श ए॑कविꣳ॒॒शः स्पृ॒ताः स्पृ॒ता ए॑कविꣳ॒॒शः । \newline
45. ए॒क॒विꣳ॒॒शः स्तोमः॒ स्तोम॑ एकविꣳ॒॒श ए॑कविꣳ॒॒शः स्तोमः॑ । \newline
46. ए॒क॒विꣳ॒॒श इत्ये॑क - विꣳ॒॒शः । \newline
47. स्तोमो ऽदि॑त्या॒ अदि॑त्यै॒ स्तोमः॒ स्तोमो ऽदि॑त्यै । \newline
48. अदि॑त्यै भा॒गो भा॒गो ऽदि॑त्या॒ अदि॑त्यै भा॒गः । \newline
49. भा॒गो᳚ ऽस्यसि भा॒गो भा॒गो॑ ऽसि । \newline
50. अ॒सि॒ पू॒ष्णः पू॒ष्णो᳚ ऽस्यसि पू॒ष्णः । \newline
51. पू॒ष्ण आधि॑पत्य॒ माधि॑पत्यम् पू॒ष्णः पू॒ष्ण आधि॑पत्यम् । \newline
52. आधि॑पत्य॒ मोज॒ ओज॒ आधि॑पत्य॒ माधि॑पत्य॒ मोजः॑ । \newline
53. आधि॑पत्य॒मित्याधि॑ - प॒त्य॒म् । \newline
54. ओजः॑ स्पृ॒तꣳ स्पृ॒त मोज॒ ओजः॑ स्पृ॒तम् । \newline
55. स्पृ॒तम् त्रि॑ण॒व स्त्रि॑ण॒वः स्पृ॒तꣳ स्पृ॒तम् त्रि॑ण॒वः । \newline
56. त्रि॒ण॒वः स्तोमः॒ स्तोम॑ स्त्रिण॒व स्त्रि॑ण॒वः स्तोमः॑ । \newline
57. त्रि॒ण॒व इति॑ त्रि - न॒वः । \newline
58. स्तोमो॒ वसू॑नां॒ ॅवसू॑नाꣳ॒॒ स्तोमः॒ स्तोमो॒ वसू॑नाम् । \newline
59. वसू॑नाम् भा॒गो भा॒गो वसू॑नां॒ ॅवसू॑नाम् भा॒गः । \newline
60. भा॒गो᳚ ऽस्यसि भा॒गो भा॒गो॑ ऽसि । \newline
61. अ॒सि॒ रु॒द्राणाꣳ॑ रु॒द्राणा॑ मस्यसि रु॒द्राणा᳚म् । \newline

\textbf{Ghana Paata } \newline

1. अ॒ग्नेर् भा॒गो भा॒गो᳚ ऽग्ने र॒ग्नेर् भा॒गो᳚ ऽस्यसि भा॒गो᳚ ऽग्ने र॒ग्नेर् भा॒गो॑ ऽसि । \newline
2. भा॒गो᳚ ऽस्यसि भा॒गो भा॒गो॑ ऽसि दी॒क्षाया॑ दी॒क्षाया॑ असि भा॒गो भा॒गो॑ ऽसि दी॒क्षायाः᳚ । \newline
3. अ॒सि॒ दी॒क्षाया॑ दी॒क्षाया॑ अस्यसि दी॒क्षाया॒ आधि॑पत्य॒ माधि॑पत्यम् दी॒क्षाया॑ अस्यसि दी॒क्षाया॒ आधि॑पत्यम् । \newline
4. दी॒क्षाया॒ आधि॑पत्य॒ माधि॑पत्यम् दी॒क्षाया॑ दी॒क्षाया॒ आधि॑पत्य॒म् ब्रह्म॒ ब्रह्माधि॑पत्यम् दी॒क्षाया॑ दी॒क्षाया॒ आधि॑पत्य॒म् ब्रह्म॑ । \newline
5. आधि॑पत्य॒म् ब्रह्म॒ ब्रह्माधि॑पत्य॒ माधि॑पत्य॒म् ब्रह्म॑ स्पृ॒तꣳ स्पृ॒तम् ब्रह्माधि॑पत्य॒ माधि॑पत्य॒म् 
ब्रह्म॑ स्पृ॒तम् । \newline
6. आधि॑पत्य॒मित्याधि॑ - प॒त्य॒म् । \newline
7. ब्रह्म॑ स्पृ॒तꣳ स्पृ॒तम् ब्रह्म॒ ब्रह्म॑ स्पृ॒तम् त्रि॒वृत् त्रि॒वृथ् स्पृ॒तम् ब्रह्म॒ ब्रह्म॑ स्पृ॒तम् त्रि॒वृत् । \newline
8. स्पृ॒तम् त्रि॒वृत् त्रि॒वृथ् स्पृ॒तꣳ स्पृ॒तम् त्रि॒वृथ् स्तोमः॒ स्तोम॑ स्त्रि॒वृथ् स्पृ॒तꣳ स्पृ॒तम् त्रि॒वृथ् स्तोमः॑ । \newline
9. त्रि॒वृथ् स्तोमः॒ स्तोम॑ स्त्रि॒वृत् त्रि॒वृथ् स्तोम॒ इन्द्र॒ स्येन्द्र॑स्य॒ स्तोम॑ स्त्रि॒वृत् त्रि॒वृथ् स्तोम॒ इन्द्र॑स्य । \newline
10. त्रि॒वृदिति॑ त्रि - वृत् । \newline
11. स्तोम॒ इन्द्र॒ स्येन्द्र॑स्य॒ स्तोमः॒ स्तोम॒ इन्द्र॑स्य भा॒गो भा॒ग इन्द्र॑स्य॒ स्तोमः॒ स्तोम॒ इन्द्र॑स्य भा॒गः । \newline
12. इन्द्र॑स्य भा॒गो भा॒ग इन्द्र॒ स्येन्द्र॑स्य भा॒गो᳚ ऽस्यसि भा॒ग इन्द्र॒ स्येन्द्र॑स्य भा॒गो॑ ऽसि । \newline
13. भा॒गो᳚ ऽस्यसि भा॒गो भा॒गो॑ ऽसि॒ विष्णो॒र् विष्णो॑ रसि भा॒गो भा॒गो॑ ऽसि॒ विष्णोः᳚ । \newline
14. अ॒सि॒ विष्णो॒र् विष्णो॑ रस्यसि॒ विष्णो॒ राधि॑पत्य॒ माधि॑पत्यं॒ ॅविष्णो॑ रस्यसि॒ विष्णो॒ राधि॑पत्यम् । \newline
15. विष्णो॒ राधि॑पत्य॒ माधि॑पत्यं॒ ॅविष्णो॒र् विष्णो॒ राधि॑पत्यम् क्ष॒त्रम् क्ष॒त्र माधि॑पत्यं॒ ॅविष्णो॒र् विष्णो॒ राधि॑पत्यम् क्ष॒त्रम् । \newline
16. आधि॑पत्यम् क्ष॒त्रम् क्ष॒त्र माधि॑पत्य॒ माधि॑पत्यम् क्ष॒त्रꣳ स्पृ॒तꣳ स्पृ॒तम् क्ष॒त्र माधि॑पत्य॒ माधि॑पत्यम् क्ष॒त्रꣳ स्पृ॒तम् । \newline
17. आधि॑पत्य॒मित्याधि॑ - प॒त्य॒म् । \newline
18. क्ष॒त्रꣳ स्पृ॒तꣳ स्पृ॒तम् क्ष॒त्रम् क्ष॒त्रꣳ स्पृ॒तम् प॑ञ्चद॒शः प॑ञ्चद॒शः स्पृ॒तम् क्ष॒त्रम् क्ष॒त्रꣳ स्पृ॒तम् प॑ञ्चद॒शः । \newline
19. स्पृ॒तम् प॑ञ्चद॒शः प॑ञ्चद॒शः स्पृ॒तꣳ स्पृ॒तम् प॑ञ्चद॒शः स्तोमः॒ स्तोमः॑ पञ्चद॒शः स्पृ॒तꣳ स्पृ॒तम् प॑ञ्चद॒शः स्तोमः॑ । \newline
20. प॒ञ्च॒द॒शः स्तोमः॒ स्तोमः॑ पञ्चद॒शः प॑ञ्चद॒शः स्तोमो॑ नृ॒चक्ष॑साम् नृ॒चक्ष॑साꣳ॒॒ स्तोमः॑ पञ्चद॒शः प॑ञ्चद॒शः स्तोमो॑ नृ॒चक्ष॑साम् । \newline
21. प॒ञ्च॒द॒श इति॑ पञ्च - द॒शः । \newline
22. स्तोमो॑ नृ॒चक्ष॑साम् नृ॒चक्ष॑साꣳ॒॒ स्तोमः॒ स्तोमो॑ नृ॒चक्ष॑साम् भा॒गो भा॒गो नृ॒चक्ष॑साꣳ॒॒ स्तोमः॒ स्तोमो॑ नृ॒चक्ष॑साम् भा॒गः । \newline
23. नृ॒चक्ष॑साम् भा॒गो भा॒गो नृ॒चक्ष॑साम् नृ॒चक्ष॑साम् भा॒गो᳚ ऽस्यसि भा॒गो नृ॒चक्ष॑साम् नृ॒चक्ष॑साम् भा॒गो॑ ऽसि । \newline
24. नृ॒चक्ष॑सा॒मिति॑ नृ - चक्ष॑साम् । \newline
25. भा॒गो᳚ ऽस्यसि भा॒गो भा॒गो॑ ऽसि धा॒तुर् धा॒तु र॑सि भा॒गो भा॒गो॑ ऽसि धा॒तुः । \newline
26. अ॒सि॒ धा॒तुर् धा॒तु र॑स्यसि धा॒तु राधि॑पत्य॒ माधि॑पत्यम् धा॒तु र॑स्यसि धा॒तु राधि॑पत्यम् । \newline
27. धा॒तु राधि॑पत्य॒ माधि॑पत्यम् धा॒तुर् धा॒तु राधि॑पत्यम् ज॒नित्र॑म् ज॒नित्र॒ माधि॑पत्यम् धा॒तुर् धा॒तु राधि॑पत्यम् ज॒नित्र᳚म् । \newline
28. आधि॑पत्यम् ज॒नित्र॑म् ज॒नित्र॒ माधि॑पत्य॒ माधि॑पत्यम् ज॒नित्रꣳ॑ स्पृ॒तꣳ स्पृ॒तम् ज॒नित्र॒ माधि॑पत्य॒ माधि॑पत्यम् ज॒नित्रꣳ॑ स्पृ॒तम् । \newline
29. आधि॑पत्य॒मित्याधि॑ - प॒त्य॒म् । \newline
30. ज॒नित्रꣳ॑ स्पृ॒तꣳ स्पृ॒तम् ज॒नित्र॑म् ज॒नित्रꣳ॑ स्पृ॒तꣳ स॑प्तद॒शः स॑प्तद॒शः स्पृ॒तम् ज॒नित्र॑म् ज॒नित्रꣳ॑ स्पृ॒तꣳ स॑प्तद॒शः । \newline
31. स्पृ॒तꣳ स॑प्तद॒शः स॑प्तद॒शः स्पृ॒तꣳ स्पृ॒तꣳ स॑प्तद॒शः स्तोमः॒ स्तोमः॑ सप्तद॒शः स्पृ॒तꣳ स्पृ॒तꣳ स॑प्तद॒शः स्तोमः॑ । \newline
32. स॒प्त॒द॒शः स्तोमः॒ स्तोमः॑ सप्तद॒शः स॑प्तद॒शः स्तोमो॑ मि॒त्रस्य॑ मि॒त्रस्य॒ स्तोमः॑ सप्तद॒शः स॑प्तद॒शः स्तोमो॑ मि॒त्रस्य॑ । \newline
33. स॒प्त॒द॒श इति॑ सप्त - द॒शः । \newline
34. स्तोमो॑ मि॒त्रस्य॑ मि॒त्रस्य॒ स्तोमः॒ स्तोमो॑ मि॒त्रस्य॑ भा॒गो भा॒गो मि॒त्रस्य॒ स्तोमः॒ स्तोमो॑ मि॒त्रस्य॑ भा॒गः । \newline
35. मि॒त्रस्य॑ भा॒गो भा॒गो मि॒त्रस्य॑ मि॒त्रस्य॑ भा॒गो᳚ ऽस्यसि भा॒गो मि॒त्रस्य॑ मि॒त्रस्य॑ भा॒गो॑ ऽसि । \newline
36. भा॒गो᳚ ऽस्यसि भा॒गो भा॒गो॑ ऽसि॒ वरु॑णस्य॒ वरु॑णस्यासि भा॒गो भा॒गो॑ ऽसि॒ वरु॑णस्य । \newline
37. अ॒सि॒ वरु॑णस्य॒ वरु॑ण स्यास्यसि॒ वरु॑ण॒स्या धि॑पत्य॒ माधि॑पत्यं॒ ॅवरु॑ण स्यास्यसि॒ वरु॑ण॒स्या धि॑पत्यम् । \newline
38. वरु॑ण॒स्या धि॑पत्य॒ माधि॑पत्यं॒ ॅवरु॑णस्य॒ वरु॑ण॒स्या धि॑पत्यम् दि॒वो दि॒व आधि॑पत्यं॒ ॅवरु॑णस्य॒ वरु॑ण॒स्या धि॑पत्यम् दि॒वः । \newline
39. आधि॑पत्यम् दि॒वो दि॒व आधि॑पत्य॒ माधि॑पत्यम् दि॒वो वृ॒ष्टिर् वृ॒ष्टिर् दि॒व आधि॑पत्य॒ माधि॑पत्यम् दि॒वो वृ॒ष्टिः । \newline
40. आधि॑पत्य॒मित्याधि॑ - प॒त्य॒म् । \newline
41. दि॒वो वृ॒ष्टिर् वृ॒ष्टिर् दि॒वो दि॒वो वृ॒ष्टिर् वाता॒ वाता॑ वृ॒ष्टिर् दि॒वो दि॒वो वृ॒ष्टिर् वाताः᳚ । \newline
42. वृ॒ष्टिर् वाता॒ वाता॑ वृ॒ष्टिर् वृ॒ष्टिर् वाताः᳚ स्पृ॒ताः स्पृ॒ता वाता॑ वृ॒ष्टिर् वृ॒ष्टिर् वाताः᳚ स्पृ॒ताः । \newline
43. वाताः᳚ स्पृ॒ताः स्पृ॒ता वाता॒ वाताः᳚ स्पृ॒ता ए॑कविꣳ॒॒श ए॑कविꣳ॒॒शः स्पृ॒ता वाता॒ वाताः᳚ स्पृ॒ता ए॑कविꣳ॒॒शः । \newline
44. स्पृ॒ता ए॑कविꣳ॒॒श ए॑कविꣳ॒॒शः स्पृ॒ताः स्पृ॒ता ए॑कविꣳ॒॒शः स्तोमः॒ स्तोम॑ एकविꣳ॒॒शः स्पृ॒ताः स्पृ॒ता ए॑कविꣳ॒॒शः स्तोमः॑ । \newline
45. ए॒क॒विꣳ॒॒शः स्तोमः॒ स्तोम॑ एकविꣳ॒॒श ए॑कविꣳ॒॒शः स्तोमो ऽदि॑त्या॒ अदि॑त्यै॒ स्तोम॑ एकविꣳ॒॒श ए॑कविꣳ॒॒शः स्तोमो ऽदि॑त्यै । \newline
46. ए॒क॒विꣳ॒॒श इत्ये॑क - विꣳ॒॒शः । \newline
47. स्तोमो ऽदि॑त्या॒ अदि॑त्यै॒ स्तोमः॒ स्तोमो ऽदि॑त्यै भा॒गो भा॒गो ऽदि॑त्यै॒ स्तोमः॒ स्तोमो ऽदि॑त्यै भा॒गः । \newline
48. अदि॑त्यै भा॒गो भा॒गो ऽदि॑त्या॒ अदि॑त्यै भा॒गो᳚ ऽस्यसि भा॒गो ऽदि॑त्या॒ अदि॑त्यै भा॒गो॑ ऽसि । \newline
49. भा॒गो᳚ ऽस्यसि भा॒गो भा॒गो॑ ऽसि पू॒ष्णः पू॒ष्णो॑ ऽसि भा॒गो भा॒गो॑ ऽसि पू॒ष्णः । \newline
50. अ॒सि॒ पू॒ष्णः पू॒ष्णो᳚ ऽस्यसि पू॒ष्ण आधि॑पत्य॒ माधि॑पत्यम् पू॒ष्णो᳚ ऽस्यसि पू॒ष्ण आधि॑पत्यम् । \newline
51. पू॒ष्ण आधि॑पत्य॒ माधि॑पत्यम् पू॒ष्णः पू॒ष्ण आधि॑पत्य॒ मोज॒ ओज॒ आधि॑पत्यम् पू॒ष्णः पू॒ष्ण आधि॑पत्य॒ मोजः॑ । \newline
52. आधि॑पत्य॒ मोज॒ ओज॒ आधि॑पत्य॒ माधि॑पत्य॒ मोजः॑ स्पृ॒तꣳ स्पृ॒त मोज॒ आधि॑पत्य॒ माधि॑पत्य॒ मोजः॑ स्पृ॒तम् । \newline
53. आधि॑पत्य॒मित्याधि॑ - प॒त्य॒म् । \newline
54. ओजः॑ स्पृ॒तꣳ स्पृ॒त मोज॒ ओजः॑ स्पृ॒तम् त्रि॑ण॒व स्त्रि॑ण॒वः स्पृ॒त मोज॒ ओजः॑ स्पृ॒तम् त्रि॑ण॒वः । \newline
55. स्पृ॒तम् त्रि॑ण॒व स्त्रि॑ण॒वः स्पृ॒तꣳ स्पृ॒तम् त्रि॑ण॒वः स्तोमः॒ स्तोम॑ स्त्रिण॒वः स्पृ॒तꣳ स्पृ॒तम् त्रि॑ण॒वः स्तोमः॑ । \newline
56. त्रि॒ण॒वः स्तोमः॒ स्तोम॑ स्त्रिण॒व स्त्रि॑ण॒वः स्तोमो॒ वसू॑नां॒ ॅवसू॑नाꣳ॒॒ स्तोम॑ स्त्रिण॒व स्त्रि॑ण॒वः स्तोमो॒ वसू॑नाम् । \newline
57. त्रि॒ण॒व इति॑ त्रि - न॒वः । \newline
58. स्तोमो॒ वसू॑नां॒ ॅवसू॑नाꣳ॒॒ स्तोमः॒ स्तोमो॒ वसू॑नाम् भा॒गो भा॒गो वसू॑नाꣳ॒॒ स्तोमः॒ स्तोमो॒ वसू॑नाम् भा॒गः । \newline
59. वसू॑नाम् भा॒गो भा॒गो वसू॑नां॒ ॅवसू॑नाम् भा॒गो᳚ ऽस्यसि भा॒गो वसू॑नां॒ ॅवसू॑नाम् भा॒गो॑ ऽसि । \newline
60. भा॒गो᳚ ऽस्यसि भा॒गो भा॒गो॑ ऽसि रु॒द्राणाꣳ॑ रु॒द्राणा॑ मसि भा॒गो भा॒गो॑ ऽसि रु॒द्राणा᳚म् । \newline
61. अ॒सि॒ रु॒द्राणाꣳ॑ रु॒द्राणा॑ मस्यसि रु॒द्राणा॒ माधि॑पत्य॒ माधि॑पत्यꣳ रु॒द्राणा॑ मस्यसि रु॒द्राणा॒ माधि॑पत्यम् । \newline
\pagebreak
\markright{ TS 4.3.9.2  \hfill https://www.vedavms.in \hfill}

\section{ TS 4.3.9.2 }

\textbf{TS 4.3.9.2 } \newline
\textbf{Samhita Paata} \newline

रु॒द्राणा॒माधि॑पत्यं॒ चतु॑ष्पाथ् स्पृ॒तं च॑तुर्विꣳ॒॒शः स्तोम॑ आदि॒त्यानां᳚ भा॒गो॑ऽसि म॒रुता॒माधि॑पत्यं॒ गर्भाः᳚ स्पृ॒ताः प॑ञ्चविꣳ॒॒शः स्तोमो॑ दे॒वस्य॑ सवि॒तुर्भा॒गो॑ऽसि॒ बृह॒स्पते॒राधि॑पत्यꣳ स॒मीची॒र्दिशः॑ स्पृ॒ताश्च॑तुष्टो॒मः स्तोमो॒ यावा॑नां भा॒गो᳚ऽस्यया॑वाना॒माधि॑पत्यं प्र॒जाः स्पृ॒ता-श्च॑तु-श्चत्वारिꣳ॒॒शः स्तोम॑ ऋभू॒णां भा॒गो॑ऽसि॒ विश्वे॑षां दे॒वाना॒माधि॑पत्यं भू॒तं निशा᳚न्तꣳ स्पृ॒तं त्र॑यस्त्रिꣳ॒॒शः स्तोमः॑ ॥ \newline

\textbf{Pada Paata} \newline

रु॒द्राणा᳚म् । आधि॑पत्य॒मित्याधि॑ - प॒त्य॒म् । चतु॑ष्पा॒दिति॒ चतुः॑-पा॒त् । स्पृ॒तम् । च॒तु॒र्विꣳ॒॒श इति॑ चतुः-विꣳ॒॒शः । स्तोमः॑ । आ॒दि॒त्याना᳚म् । भा॒गः । अ॒सि॒ । म॒रुता᳚म् । आधि॑पत्य॒मित्याधि॑ - प॒त्य॒म् । गर्भाः᳚ । स्पृ॒ताः । प॒ञ्च॒विꣳ॒॒श इति॑ पञ्च - विꣳ॒॒शः । स्तोमः॑ । दे॒वस्य॑ । स॒वि॒तुः । भा॒गः । अ॒सि॒ । बृह॒स्पतेः᳚ । आधि॑पत्य॒मित्याधि॑ - प॒त्य॒म् । स॒मीचीः᳚ । दिशः॑ । स्पृ॒ताः । च॒तु॒ष्टो॒म इति॑ चतुः-स्तो॒मः । स्तोमः॑ । यावा॑नाम् । भा॒गः । अ॒सि॒ । अया॑वानाम् । आधि॑पत्य॒मित्याधि॑ - प॒त्य॒म् । प्र॒जा इति॑ प्र - जाः । स्पृ॒ताः । च॒तु॒श्च॒त्वा॒रिꣳ॒॒श इति॑ चतुः - च॒त्वा॒रिꣳ॒॒शः । स्तोमः॑ । ऋ॒भू॒णाम् । भा॒गः । अ॒सि॒ । विश्वे॑षाम् । दे॒वाना᳚म् । आधि॑पत्य॒मित्याधि॑-प॒त्य॒म् । भू॒तम् । निशा᳚न्त॒मिति॒ नि - शा॒न्त॒म् । स्पृ॒तम् । त्र॒य॒स्त्रिꣳ॒॒श इति॑ त्रयः - त्रिꣳ॒॒शः । स्तोमः॑ ॥  \newline


\textbf{Krama Paata} \newline

रु॒द्राणा॒माधि॑पत्यम् । आधि॑पत्य॒म् चतु॑ष्पात् । आधि॑पत्य॒मित्याधि॑ - प॒त्य॒म् । चतु॑ष्पाथ् स्पृ॒तम् । चतु॑ष्पा॒दिति॒ चतुः॑ - पा॒त्॒ । स्पृ॒तम् च॑तुर्विꣳ॒॒शः । च॒तु॒र्विꣳ॒॒शः स्तोमः॑ । च॒तु॒र्विꣳ॒॒श इति॑ चतुः - विꣳ॒॒शः । स्तोम॑ आदि॒त्याना᳚म् । आ॒दि॒त्याना᳚म् भा॒गः । भा॒गो॑ऽसि । अ॒सि॒ म॒रुता᳚म् । म॒रुता॒माधि॑पत्यम् । आधि॑पत्य॒म् गर्भाः᳚ । आधि॑पत्य॒मित्याधि॑ - प॒त्य॒म् । गर्भाः᳚ स्पृ॒ताः । स्पृ॒ताः प॑ञ्चविꣳ॒॒शः । प॒ञ्च॒विꣳ॒॒शः स्तोमः॑ । प॒ञ्च॒विꣳ॒॒शः इति॑ पञ्च - विꣳ॒॒शः । स्तोमो॑ दे॒वस्य॑ । दे॒वस्य॑ सवि॒तुः । स॒वि॒तुर् भा॒गः । भा॒गो॑ऽसि । अ॒सि॒ बृह॒स्पतेः᳚ । बृह॒स्पते॒राधि॑पत्यम् । आधि॑पत्यꣳ स॒मीचीः᳚ । आधि॑पत्य॒मित्याधि॑ - प॒त्य॒म् । स॒मीची॒र् दिशः॑ । दिशः॑ स्पृ॒ताः । स्पृ॒ताश्च॑तुष्टो॒मः । च॒तु॒ष्टो॒मः स्तोमः॑ । च॒तु॒ष्टो॒म इति॑ चतुः - स्तो॒मः । स्तोमो॒ यावा॑नाम् । यावा॑नाम् भा॒गः । भा॒गो॑ऽसि । अ॒स्यया॑वानाम् । अया॑वाना॒माधि॑पत्यम् । आधि॑पत्यम् प्र॒जाः । आधि॑पत्य॒मित्याधि॑ - प॒त्य॒म् । प्र॒जाः स्पृ॒ताः । प्र॒जा इति॑ प्र - जाः । स्पृ॒ताश्च॑तुश्चत्वारिꣳ॒॒शः । च॒तु॒श्च॒त्वा॒रिꣳ॒॒शः स्तोमः॑ । च॒तु॒श्च॒त्वा॒रिꣳ॒॒श इति॑ चतुः - च॒त्वा॒रिꣳ॒॒शः । स्तोम॑ ऋभू॒णाम् । ऋ॒भू॒णाम् भा॒गः । भा॒गो॑ऽसि । अ॒सि॒ विश्वे॑षाम् । विश्वे॑षाम् दे॒वाना᳚म् । दे॒वाना॒माधि॑पत्यम् । आधि॑पत्यम् भू॒तम् । आधि॑पत्य॒मित्याधि॑ - प॒त्य॒म् । भू॒तम् निशा᳚न्तम् । निशा᳚न्तꣳ स्पृ॒तम् । निशा᳚न्त॒मिति॒ नि - शा॒न्त॒म् । स्पृ॒तम् त्र॑यस्त्रिꣳ॒॒शः । त्र॒य॒स्त्रिꣳ॒॒शः स्तोमः॑ । त्र॒य॒स्त्रिꣳ॒॒श इति॑ त्रयः - त्रिꣳ॒॒शः । स्तोम॒ इति॒ स्तोमः॑ । \newline

\textbf{Jatai Paata} \newline

1. रु॒द्राणा॒ माधि॑पत्य॒ माधि॑पत्यꣳ रु॒द्राणाꣳ॑ रु॒द्राणा॒ माधि॑पत्यम् । \newline
2. आधि॑पत्य॒म् चतु॑ष्पा॒च् चतु॑ष्पा॒ दाधि॑पत्य॒ माधि॑पत्य॒म् चतु॑ष्पात् । \newline
3. आधि॑पत्य॒मित्याधि॑ - प॒त्य॒म् । \newline
4. चतु॑ष्पाथ् स्पृ॒तꣳ स्पृ॒तम् चतु॑ष्पा॒च् चतु॑ष्पाथ् स्पृ॒तम् । \newline
5. चतु॑ष्पा॒दिति॒ चतुः॑ - पा॒त् । \newline
6. स्पृ॒तम् च॑तुर्विꣳ॒॒श श्च॑तुर्विꣳ॒॒शः स्पृ॒तꣳ स्पृ॒तम् च॑तुर्विꣳ॒॒शः । \newline
7. च॒तु॒र्विꣳ॒॒शः स्तोमः॒ स्तोम॑ श्चतुर्विꣳ॒॒श श्च॑तुर्विꣳ॒॒शः स्तोमः॑ । \newline
8. च॒तु॒र्विꣳ॒॒श इति॑ चतुः - विꣳ॒॒शः । \newline
9. स्तोम॑ आदि॒त्याना॑ मादि॒त्यानाꣳ॒॒ स्तोमः॒ स्तोम॑ आदि॒त्याना᳚म् । \newline
10. आ॒दि॒त्याना᳚म् भा॒गो भा॒ग आ॑दि॒त्याना॑ मादि॒त्याना᳚म् भा॒गः । \newline
11. भा॒गो᳚ ऽस्यसि भा॒गो भा॒गो॑ ऽसि । \newline
12. अ॒सि॒ म॒रुता᳚म् म॒रुता॑ मस्यसि म॒रुता᳚म् । \newline
13. म॒रुता॒ माधि॑पत्य॒ माधि॑पत्यम् म॒रुता᳚म् म॒रुता॒ माधि॑पत्यम् । \newline
14. आधि॑पत्य॒म् गर्भा॒ गर्भा॒ आधि॑पत्य॒ माधि॑पत्य॒म् गर्भाः᳚ । \newline
15. आधि॑पत्य॒मित्याधि॑ - प॒त्य॒म् । \newline
16. गर्भाः᳚ स्पृ॒ताः स्पृ॒ता गर्भा॒ गर्भाः᳚ स्पृ॒ताः । \newline
17. स्पृ॒ताः प॑ञ्चविꣳ॒॒शः प॑ञ्चविꣳ॒॒शः स्पृ॒ताः स्पृ॒ताः प॑ञ्चविꣳ॒॒शः । \newline
18. प॒ञ्च॒विꣳ॒॒शः स्तोमः॒ स्तोमः॑ पञ्चविꣳ॒॒शः प॑ञ्चविꣳ॒॒शः स्तोमः॑ । \newline
19. प॒ञ्च॒विꣳ॒॒श इति॑ पञ्च - विꣳ॒॒शः । \newline
20. स्तोमो॑ दे॒वस्य॑ दे॒वस्य॒ स्तोमः॒ स्तोमो॑ दे॒वस्य॑ । \newline
21. दे॒वस्य॑ सवि॒तुः स॑वि॒तुर् दे॒वस्य॑ दे॒वस्य॑ सवि॒तुः । \newline
22. स॒वि॒तुर् भा॒गो भा॒गः स॑वि॒तुः स॑वि॒तुर् भा॒गः । \newline
23. भा॒गो᳚ ऽस्यसि भा॒गो भा॒गो॑ ऽसि । \newline
24. अ॒सि॒ बृह॒स्पते॒र् बृह॒स्पते॑ रस्यसि॒ बृह॒स्पतेः᳚ । \newline
25. बृह॒स्पते॒ राधि॑पत्य॒ माधि॑पत्य॒म् बृह॒स्पते॒र् बृह॒स्पते॒ राधि॑पत्यम् । \newline
26. आधि॑पत्यꣳ स॒मीचीः᳚ स॒मीची॒ राधि॑पत्य॒ माधि॑पत्यꣳ स॒मीचीः᳚ । \newline
27. आधि॑पत्य॒मित्याधि॑ - प॒त्य॒म् । \newline
28. स॒मीची॒र् दिशो॒ दिशः॑ स॒मीचीः᳚ स॒मीची॒र् दिशः॑ । \newline
29. दिशः॑ स्पृ॒ताः स्पृ॒ता दिशो॒ दिशः॑ स्पृ॒ताः । \newline
30. स्पृ॒ता श्च॑तुष्टो॒म श्च॑तुष्टो॒मः स्पृ॒ताः स्पृ॒ता श्च॑तुष्टो॒मः । \newline
31. च॒तु॒ष्टो॒मः स्तोमः॒ स्तोम॑ श्चतुष्टो॒म श्च॑तुष्टो॒मः स्तोमः॑ । \newline
32. च॒तु॒ष्टो॒म इति॑ चतुः - स्तो॒मः । \newline
33. स्तोमो॒ यावा॑नां॒ ॅयावा॑नाꣳ॒॒ स्तोमः॒ स्तोमो॒ यावा॑नाम् । \newline
34. यावा॑नाम् भा॒गो भा॒गो यावा॑नां॒ ॅयावा॑नाम् भा॒गः । \newline
35. भा॒गो᳚ ऽस्यसि भा॒गो भा॒गो॑ ऽसि । \newline
36. अ॒स्य या॑वाना॒ मया॑वाना मस्य॒स्य या॑वानाम् । \newline
37. अया॑वाना॒ माधि॑पत्य॒ माधि॑पत्य॒ मया॑वाना॒ मया॑वाना॒ माधि॑पत्यम् । \newline
38. आधि॑पत्यम् प्र॒जाः प्र॒जा आधि॑पत्य॒ माधि॑पत्यम् प्र॒जाः । \newline
39. आधि॑पत्य॒मित्याधि॑ - प॒त्य॒म् । \newline
40. प्र॒जाः स्पृ॒ताः स्पृ॒ताः प्र॒जाः प्र॒जाः स्पृ॒ताः । \newline
41. प्र॒जा इति॑ प्र - जाः । \newline
42. स्पृ॒ता श्च॑तुश्चत्वारिꣳ॒॒श श्च॑तुश्चत्वारिꣳ॒॒शः स्पृ॒ताः स्पृ॒ताश्च॑तु श्चत्वारिꣳ॒॒शः । \newline
43. च॒तु॒श्च॒त्वा॒रिꣳ॒॒शः स्तोमः॒ स्तोम॑ श्चतुश्चत्वारिꣳ॒॒श श्च॑तुश्चत्वारिꣳ॒॒शः स्तोमः॑ । \newline
44. च॒तु॒श्च॒त्वा॒रिꣳ॒॒श इति॑ चतुः - च॒त्वा॒रिꣳ॒॒शः । \newline
45. स्तोम॑ ऋभू॒णा मृ॑भू॒णाꣳ स्तोमः॒ स्तोम॑ ऋभू॒णाम् । \newline
46. ऋ॒भू॒णाम् भा॒गो भा॒ग ऋ॑भू॒णा मृ॑भू॒णाम् भा॒गः । \newline
47. भा॒गो᳚ ऽस्यसि भा॒गो भा॒गो॑ ऽसि । \newline
48. अ॒सि॒ विश्वे॑षां॒ ॅविश्वे॑षा मस्यसि॒ विश्वे॑षाम् । \newline
49. विश्वे॑षाम् दे॒वाना᳚म् दे॒वानां॒ ॅविश्वे॑षां॒ ॅविश्वे॑षाम् दे॒वाना᳚म् । \newline
50. दे॒वाना॒ माधि॑पत्य॒ माधि॑पत्यम् दे॒वाना᳚म् दे॒वाना॒ माधि॑पत्यम् । \newline
51. आधि॑पत्यम् भू॒तम् भू॒त माधि॑पत्य॒ माधि॑पत्यम् भू॒तम् । \newline
52. आधि॑पत्य॒मित्याधि॑ - प॒त्य॒म् । \newline
53. भू॒तम् निशा᳚न्त॒न् निशा᳚न्तम् भू॒तम् भू॒तम् निशा᳚न्तम् । \newline
54. निशा᳚न्तꣳ स्पृ॒तꣳ स्पृ॒तन् निशा᳚न्त॒म् निशा᳚न्तꣳ स्पृ॒तम् । \newline
55. निशा᳚न्त॒मिति॒ नि - शा॒न्त॒म् । \newline
56. स्पृ॒तम् त्र॑यस्त्रिꣳ॒॒श स्त्र॑यस्त्रिꣳ॒॒शः स्पृ॒तꣳ स्पृ॒तम् त्र॑यस्त्रिꣳ॒॒शः । \newline
57. त्र॒य॒स्त्रिꣳ॒॒शः स्तोमः॒ स्तोम॑ स्त्रयस्त्रिꣳ॒॒श स्त्र॑यस्त्रिꣳ॒॒शः स्तोमः॑ । \newline
58. त्र॒य॒स्त्रिꣳ॒॒श इति॑ त्रयः - त्रिꣳ॒॒शः । \newline
59. स्तोम॒ इति॒ स्तोमः॑ । \newline

\textbf{Ghana Paata } \newline

1. रु॒द्राणा॒ माधि॑पत्य॒ माधि॑पत्यꣳ रु॒द्राणाꣳ॑ रु॒द्राणा॒ माधि॑पत्य॒म् चतु॑ष्पा॒च् चतु॑ष्पा॒ दाधि॑पत्यꣳ रु॒द्राणाꣳ॑ रु॒द्राणा॒ माधि॑पत्य॒म् चतु॑ष्पात् । \newline
2. आधि॑पत्य॒म् चतु॑ष्पा॒च् चतु॑ष्पा॒ दाधि॑पत्य॒ माधि॑पत्य॒म् चतु॑ष्पाथ् स्पृ॒तꣳ स्पृ॒तम् चतु॑ष्पा॒ दाधि॑पत्य॒ माधि॑पत्य॒म् चतु॑ष्पाथ् स्पृ॒तम् । \newline
3. आधि॑पत्य॒मित्याधि॑ - प॒त्य॒म् । \newline
4. चतु॑ष्पाथ् स्पृ॒तꣳ स्पृ॒तम् चतु॑ष्पा॒च् चतु॑ष्पाथ् स्पृ॒तम् च॑तुर्विꣳ॒॒श श्च॑तुर्विꣳ॒॒शः स्पृ॒तम् चतु॑ष्पा॒च् चतु॑ष्पाथ् स्पृ॒तम् च॑तुर्विꣳ॒॒शः । \newline
5. चतु॑ष्पा॒दिति॒ चतुः॑ - पा॒त् । \newline
6. स्पृ॒तम् च॑तुर्विꣳ॒॒श श्च॑तुर्विꣳ॒॒शः स्पृ॒तꣳ स्पृ॒तम् च॑तुर्विꣳ॒॒शः स्तोमः॒ स्तोम॑ श्चतुर्विꣳ॒॒शः स्पृ॒तꣳ स्पृ॒तम् च॑तुर्विꣳ॒॒शः स्तोमः॑ । \newline
7. च॒तु॒र्विꣳ॒॒शः स्तोमः॒ स्तोम॑ श्चतुर्विꣳ॒॒श श्च॑तुर्विꣳ॒॒शः स्तोम॑ आदि॒त्याना॑ मादि॒त्यानाꣳ॒॒ स्तोम॑ श्चतुर्विꣳ॒॒श श्च॑तुर्विꣳ॒॒शः स्तोम॑ आदि॒त्याना᳚म् । \newline
8. च॒तु॒र्विꣳ॒॒श इति॑ चतुः - विꣳ॒॒शः । \newline
9. स्तोम॑ आदि॒त्याना॑ मादि॒त्यानाꣳ॒॒ स्तोमः॒ स्तोम॑ आदि॒त्याना᳚म् भा॒गो भा॒ग आ॑दि॒त्यानाꣳ॒॒ स्तोमः॒ स्तोम॑ आदि॒त्याना᳚म् भा॒गः । \newline
10. आ॒दि॒त्याना᳚म् भा॒गो भा॒ग आ॑दि॒त्याना॑ मादि॒त्याना᳚म् भा॒गो᳚ ऽस्यसि भा॒ग आ॑दि॒त्याना॑ मादि॒त्याना᳚म् भा॒गो॑ ऽसि । \newline
11. भा॒गो᳚ ऽस्यसि भा॒गो भा॒गो॑ ऽसि म॒रुता᳚म् म॒रुता॑ मसि भा॒गो भा॒गो॑ ऽसि म॒रुता᳚म् । \newline
12. अ॒सि॒ म॒रुता᳚म् म॒रुता॑ मस्यसि म॒रुता॒ माधि॑पत्य॒ माधि॑पत्यम् म॒रुता॑ मस्यसि म॒रुता॒ माधि॑पत्यम् । \newline
13. म॒रुता॒ माधि॑पत्य॒ माधि॑पत्यम् म॒रुता᳚म् म॒रुता॒ माधि॑पत्य॒म् गर्भा॒ गर्भा॒ आधि॑पत्यम् म॒रुता᳚म् म॒रुता॒ माधि॑पत्य॒म् गर्भाः᳚ । \newline
14. आधि॑पत्य॒म् गर्भा॒ गर्भा॒ आधि॑पत्य॒ माधि॑पत्य॒म् गर्भाः᳚ स्पृ॒ताः स्पृ॒ता गर्भा॒ आधि॑पत्य॒ माधि॑पत्य॒म् गर्भाः᳚ स्पृ॒ताः । \newline
15. आधि॑पत्य॒मित्याधि॑ - प॒त्य॒म् । \newline
16. गर्भाः᳚ स्पृ॒ताः स्पृ॒ता गर्भा॒ गर्भाः᳚ स्पृ॒ताः प॑ञ्चविꣳ॒॒शः प॑ञ्चविꣳ॒॒शः स्पृ॒ता गर्भा॒ गर्भाः᳚ 
स्पृ॒ताः प॑ञ्चविꣳ॒॒शः । \newline
17. स्पृ॒ताः प॑ञ्चविꣳ॒॒शः प॑ञ्चविꣳ॒॒शः स्पृ॒ताः स्पृ॒ताः प॑ञ्चविꣳ॒॒शः स्तोमः॒ स्तोमः॑ पञ्चविꣳ॒॒शः स्पृ॒ताः स्पृ॒ताः प॑ञ्चविꣳ॒॒शः स्तोमः॑ । \newline
18. प॒ञ्च॒विꣳ॒॒शः स्तोमः॒ स्तोमः॑ पञ्चविꣳ॒॒शः प॑ञ्चविꣳ॒॒शः स्तोमो॑ दे॒वस्य॑ दे॒वस्य॒ स्तोमः॑ पञ्चविꣳ॒॒शः प॑ञ्चविꣳ॒॒शः स्तोमो॑ दे॒वस्य॑ । \newline
19. प॒ञ्च॒विꣳ॒॒श इति॑ पञ्च - विꣳ॒॒शः । \newline
20. स्तोमो॑ दे॒वस्य॑ दे॒वस्य॒ स्तोमः॒ स्तोमो॑ दे॒वस्य॑ सवि॒तुः स॑वि॒तुर् दे॒वस्य॒ स्तोमः॒ स्तोमो॑ दे॒वस्य॑ सवि॒तुः । \newline
21. दे॒वस्य॑ सवि॒तुः स॑वि॒तुर् दे॒वस्य॑ दे॒वस्य॑ सवि॒तुर् भा॒गो भा॒गः स॑वि॒तुर् दे॒वस्य॑ दे॒वस्य॑ सवि॒तुर् भा॒गः । \newline
22. स॒वि॒तुर् भा॒गो भा॒गः स॑वि॒तुः स॑वि॒तुर् भा॒गो᳚ ऽस्यसि भा॒गः स॑वि॒तुः स॑वि॒तुर् भा॒गो॑ ऽसि । \newline
23. भा॒गो᳚ ऽस्यसि भा॒गो भा॒गो॑ ऽसि॒ बृह॒स्पते॒र् बृह॒स्पते॑ रसि भा॒गो भा॒गो॑ ऽसि॒ बृह॒स्पतेः᳚ । \newline
24. अ॒सि॒ बृह॒स्पते॒र् बृह॒स्पते॑ रस्यसि॒ बृह॒स्पते॒ राधि॑पत्य॒ माधि॑पत्य॒म् बृह॒स्पते॑ रस्यसि॒ बृह॒स्पते॒ राधि॑पत्यम् । \newline
25. बृह॒स्पते॒ राधि॑पत्य॒ माधि॑पत्य॒म् बृह॒स्पते॒र् बृह॒स्पते॒ राधि॑पत्यꣳ स॒मीचीः᳚ स॒मीची॒ राधि॑पत्य॒म् बृह॒स्पते॒र् बृह॒स्पते॒ राधि॑पत्यꣳ स॒मीचीः᳚ । \newline
26. आधि॑पत्यꣳ स॒मीचीः᳚ स॒मीची॒ राधि॑पत्य॒ माधि॑पत्यꣳ स॒मीची॒र् दिशो॒ दिशः॑ स॒मीची॒ राधि॑पत्य॒ माधि॑पत्यꣳ स॒मीची॒र् दिशः॑ । \newline
27. आधि॑पत्य॒मित्याधि॑ - प॒त्य॒म् । \newline
28. स॒मीची॒र् दिशो॒ दिशः॑ स॒मीचीः᳚ स॒मीची॒र् दिशः॑ स्पृ॒ताः स्पृ॒ता दिशः॑ स॒मीचीः᳚ स॒मीची॒र् दिशः॑ स्पृ॒ताः । \newline
29. दिशः॑ स्पृ॒ताः स्पृ॒ता दिशो॒ दिशः॑ स्पृ॒ता श्च॑तुष्टो॒म श्च॑तुष्टो॒मः स्पृ॒ता दिशो॒ दिशः॑ स्पृ॒ता श्च॑तुष्टो॒मः । \newline
30. स्पृ॒ता श्च॑तुष्टो॒म श्च॑तुष्टो॒मः स्पृ॒ताः स्पृ॒ता श्च॑तुष्टो॒मः स्तोमः॒ स्तोम॑ श्चतुष्टो॒मः स्पृ॒ताः स्पृ॒ता श्च॑तुष्टो॒मः स्तोमः॑ । \newline
31. च॒तु॒ष्टो॒मः स्तोमः॒ स्तोम॑ श्चतुष्टो॒म श्च॑तुष्टो॒मः स्तोमो॒ यावा॑नां॒ ॅयावा॑नाꣳ॒॒ स्तोम॑ श्चतुष्टो॒म श्च॑तुष्टो॒मः स्तोमो॒ यावा॑नाम् । \newline
32. च॒तु॒ष्टो॒म इति॑ चतुः - स्तो॒मः । \newline
33. स्तोमो॒ यावा॑नां॒ ॅयावा॑नाꣳ॒॒ स्तोमः॒ स्तोमो॒ यावा॑नाम् भा॒गो भा॒गो यावा॑नाꣳ॒॒ स्तोमः॒ स्तोमो॒ यावा॑नाम् भा॒गः । \newline
34. यावा॑नाम् भा॒गो भा॒गो यावा॑नां॒ ॅयावा॑नाम् भा॒गो᳚ ऽस्यसि भा॒गो यावा॑नां॒ ॅयावा॑नाम् भा॒गो॑ ऽसि । \newline
35. भा॒गो᳚ ऽस्यसि भा॒गो भा॒गो᳚ ऽस्यया॑वाना॒ मया॑वाना मसि भा॒गो भा॒गो᳚ ऽस्यया॑वानाम् । \newline
36. अ॒स्यया॑वाना॒ मया॑वाना मस्य॒स्य या॑वाना॒ माधि॑पत्य॒ माधि॑पत्य॒ मया॑वाना मस्य॒स्य या॑वाना॒ माधि॑पत्यम् । \newline
37. अया॑वाना॒ माधि॑पत्य॒ माधि॑पत्य॒ मया॑वाना॒ मया॑वाना॒ माधि॑पत्यम् प्र॒जाः प्र॒जा आधि॑पत्य॒ मया॑वाना॒ मया॑वाना॒ माधि॑पत्यम् प्र॒जाः । \newline
38. आधि॑पत्यम् प्र॒जाः प्र॒जा आधि॑पत्य॒ माधि॑पत्यम् प्र॒जाः स्पृ॒ताः स्पृ॒ताः प्र॒जा आधि॑पत्य॒ माधि॑पत्यम् प्र॒जाः स्पृ॒ताः । \newline
39. आधि॑पत्य॒मित्याधि॑ - प॒त्य॒म् । \newline
40. प्र॒जाः स्पृ॒ताः स्पृ॒ताः प्र॒जाः प्र॒जाः स्पृ॒ता श्च॑तुश्चत्वारिꣳ॒॒श श्च॑तुश्चत्वारिꣳ॒॒शः स्पृ॒ताः प्र॒जाः प्र॒जाः स्पृ॒ता श्च॑तुश्चत्वारिꣳ॒॒शः । \newline
41. प्र॒जा इति॑ प्र - जाः । \newline
42. स्पृ॒ता श्च॑तुश्चत्वारिꣳ॒॒श श्च॑तुश्चत्वारिꣳ॒॒शः स्पृ॒ताः स्पृ॒ता श्च॑तुश्चत्वारिꣳ॒॒शः स्तोमः॒ स्तोम॑ श्चतुश्चत्वारिꣳ॒॒शः स्पृ॒ताः स्पृ॒ता श्च॑तुश्चत्वारिꣳ॒॒शः स्तोमः॑ । \newline
43. च॒तु॒श्च॒त्वा॒रिꣳ॒॒शः स्तोमः॒ स्तोम॑ श्चतुश्चत्वारिꣳ॒॒श श्च॑तुश्चत्वारिꣳ॒॒शः स्तोम॑ ऋभू॒णा मृ॑भू॒णाꣳ स्तोम॑ श्चतुश्चत्वारिꣳ॒॒श श्च॑तुश्चत्वारिꣳ॒॒शः स्तोम॑ ऋभू॒णाम् । \newline
44. च॒तु॒श्च॒त्वा॒रिꣳ॒॒श इति॑ चतुः - च॒त्वा॒रिꣳ॒॒शः । \newline
45. स्तोम॑ ऋभू॒णा मृ॑भू॒णाꣳ स्तोमः॒ स्तोम॑ ऋभू॒णाम् भा॒गो भा॒ग ऋ॑भू॒णाꣳ स्तोमः॒ स्तोम॑ ऋभू॒णाम् भा॒गः । \newline
46. ऋ॒भू॒णाम् भा॒गो भा॒ग ऋ॑भू॒णा मृ॑भू॒णाम् भा॒गो᳚ ऽस्यसि भा॒ग ऋ॑भू॒णा मृ॑भू॒णाम् भा॒गो॑ ऽसि । \newline
47. भा॒गो᳚ ऽस्यसि भा॒गो भा॒गो॑ ऽसि॒ विश्वे॑षां॒ ॅविश्वे॑षा मसि भा॒गो भा॒गो॑ ऽसि॒ विश्वे॑षाम् । \newline
48. अ॒सि॒ विश्वे॑षां॒ ॅविश्वे॑षा मस्यसि॒ विश्वे॑षाम् दे॒वाना᳚म् दे॒वानां॒ ॅविश्वे॑षा मस्यसि॒ विश्वे॑षाम् दे॒वाना᳚म् । \newline
49. विश्वे॑षाम् दे॒वाना᳚म् दे॒वानां॒ ॅविश्वे॑षां॒ ॅविश्वे॑षाम् दे॒वाना॒ माधि॑पत्य॒ माधि॑पत्यम् दे॒वानां॒ ॅविश्वे॑षां॒ ॅविश्वे॑षाम् दे॒वाना॒ माधि॑पत्यम् । \newline
50. दे॒वाना॒ माधि॑पत्य॒ माधि॑पत्यम् दे॒वाना᳚म् दे॒वाना॒ माधि॑पत्यम् भू॒तम् भू॒त माधि॑पत्यम् दे॒वाना᳚म् दे॒वाना॒ माधि॑पत्यम् भू॒तम् । \newline
51. आधि॑पत्यम् भू॒तम् भू॒त माधि॑पत्य॒ माधि॑पत्यम् भू॒तन् निशा᳚न्त॒न् निशा᳚न्तम् भू॒त माधि॑पत्य॒ माधि॑पत्यम् भू॒तन् निशा᳚न्तम् । \newline
52. आधि॑पत्य॒मित्याधि॑ - प॒त्य॒म् । \newline
53. भू॒तम् निशा᳚न्त॒म् निशा᳚न्तम् भू॒तम् भू॒तम् निशा᳚न्तꣳ स्पृ॒तꣳ स्पृ॒तम् निशा᳚न्तम् भू॒तम् भू॒तम् निशा᳚न्तꣳ स्पृ॒तम् । \newline
54. निशा᳚न्तꣳ स्पृ॒तꣳ स्पृ॒तम् निशा᳚न्त॒न् निशा᳚न्तꣳ स्पृ॒तम् त्र॑यस्त्रिꣳ॒॒श स्त्र॑यस्त्रिꣳ॒॒शः स्पृ॒तम् निशा᳚न्त॒म् निशा᳚न्तꣳ स्पृ॒तम् त्र॑यस्त्रिꣳ॒॒शः । \newline
55. निशा᳚न्त॒मिति॒ नि - शा॒न्त॒म् । \newline
56. स्पृ॒तम् त्र॑यस्त्रिꣳ॒॒श स्त्र॑यस्त्रिꣳ॒॒शः स्पृ॒तꣳ स्पृ॒तम् त्र॑यस्त्रिꣳ॒॒शः स्तोमः॒ स्तोम॑ स्त्रयस्त्रिꣳ॒॒शः स्पृ॒तꣳ स्पृ॒तम् त्र॑यस्त्रिꣳ॒॒शः स्तोमः॑ । \newline
57. त्र॒य॒स्त्रिꣳ॒॒शः स्तोमः॒ स्तोम॑ स्त्रयस्त्रिꣳ॒॒श स्त्र॑यस्त्रिꣳ॒॒शः स्तोमः॑ । \newline
58. त्र॒य॒स्त्रिꣳ॒॒श इति॑ त्रयः - त्रिꣳ॒॒शः । \newline
59. स्तोम॒ इति॒ स्तोमः॑ । \newline
\pagebreak
\markright{ TS 4.3.10.1  \hfill https://www.vedavms.in \hfill}

\section{ TS 4.3.10.1 }

\textbf{TS 4.3.10.1 } \newline
\textbf{Samhita Paata} \newline

एक॑याऽस्तुवत प्र॒जा अ॑धीयन्त प्र॒जाप॑ति॒रधि॑पतिरासीत् ति॒सृभि॑रस्तुवत॒ ब्रह्मा॑सृज्यत॒ ब्रह्म॑ण॒स्पति॒-रधि॑पतिरासीत् प॒ञ्चभि॑रस्तुवत भू॒तान्य॑सृज्यन्त भू॒तानां॒ पति॒रधि॑पतिरासीथ् स॒प्तभि॑रस्तुवत सप्त॒र्॒.षयो॑ऽसृज्यन्त धा॒ता-धि॑पतिरासी-न्न॒वभि॑रस्तुवत पि॒तरो॑-ऽसृज्य॒न्ता-ऽ*दि॑ति॒रधि॑पत्न्यासी-देकाद॒शभि॑-रस्तुवत॒र्तवो॑ऽ सृज्यन्ता- ऽऽर्त॒वो-ऽधि॑पति-रासीत् त्रयोद॒शभि॑-रस्तुवत॒ मासा॑ असृज्यन्त संॅवथ्स॒रोऽधि॑पति- [  ] \newline

\textbf{Pada Paata} \newline

एक॑या । अ॒स्तु॒व॒त॒ । प्र॒जा इति॑ प्र - जाः । अ॒धी॒य॒न्त॒ । प्र॒जाप॑ति॒रिति॑ प्र॒जा - प॒तिः॒ । अधि॑पति॒रित्यधि॑ - प॒तिः॒ । आ॒सी॒त् । ति॒सृभि॒रिति॑ ति॒सृ - भिः॒ । अ॒स्तु॒व॒त॒ । ब्रह्म॑ । अ॒सृ॒ज्य॒त॒ । ब्रह्म॑णः । पतिः॑ । अधि॑पति॒रित्यधि॑-प॒तिः॒ । आ॒सी॒त् । प॒ञ्चभि॒रिति॑ प॒ञ्च-भिः॒ । अ॒स्तु॒व॒त॒ । भू॒तानि॑ । अ॒सृ॒ज्य॒न्त॒ । भू॒ताना᳚म् । पतिः॑ । अधि॑पति॒रित्यधि॑ - प॒तिः॒ । आ॒सी॒त् । स॒प्तभि॒रिति॑ स॒प्त - भिः॒ । अ॒स्तु॒व॒त॒ । स॒प्त॒र्॒.षय॒ इति॑ सप्त - ऋ॒षयः॑ । अ॒सृ॒ज्य॒न्त॒ । धा॒ता । अधि॑पति॒रित्यधि॑ - प॒तिः॒ । आ॒सी॒त् । न॒वभि॒रिति॑ न॒व - भिः॒ । अ॒स्तु॒व॒त॒ । पि॒तरः॑ । अ॒सृ॒ज्य॒न्त॒ । अदि॑तिः । अधि॑प॒त्नीत्यधि॑-प॒त्नी॒ । आ॒सी॒त् । ए॒का॒द॒शभि॒रित्ये॑काद॒श - भिः॒ । अ॒स्तु॒व॒त॒ । ऋ॒तवः॑ । अ॒सृ॒ज्य॒न्त॒ । आ॒र्त॒वः । अधि॑पति॒रित्यधि॑ - प॒तिः॒ । आ॒सी॒त् । त्र॒यो॒द॒शभि॒रिति॑ त्रयोद॒श - भिः॒ । अ॒स्तु॒व॒त॒ । मासाः᳚ । अ॒सृ॒ज्य॒न्त॒ । सं॒ॅव॒थ्स॒र इति॑ सं - व॒थ्स॒रः । अधि॑पति॒रित्यधि॑ - प॒तिः॒ ।  \newline


\textbf{Krama Paata} \newline

एक॑याऽस्तुवत । अ॒स्तु॒व॒त॒ प्र॒जाः । प्र॒जा अ॑धीयन्त । प्र॒जा इति॑ प्र - जाः । अ॒धी॒य॒न्त॒ प्र॒जाप॑तिः । प्र॒जाप॑ति॒रधि॑पतिः । प्र॒जाप॑ति॒रिति॑ प्र॒जा - प॒तिः॒ । अधि॑पतिरासीत् । अधि॑पति॒रित्यधि॑ - प॒तिः॒ । आ॒सी॒त् ति॒सृभिः॑ । ति॒सृभि॑रस्तुवत । ति॒सृभि॒रिति॑ ति॒सृ - भिः॒ । अ॒स्तु॒व॒त॒ ब्रह्म॑ । ब्रह्मा॑सृज्यत । अ॒सृ॒ज्य॒त॒ ब्रह्म॑णः । ब्रह्म॑ण॒स्पतिः॑ । पति॒रधि॑पतिः । अधि॑पतिरासीत् । अधि॑पति॒रित्यधि॑ - प॒तिः॒ । आ॒सी॒त् प॒ञ्चभिः॑ । प॒ञ्चभि॑रस्तुवत । प॒ञ्चभि॒रिति॑ प॒ञ्च - भिः॒ । अ॒स्तु॒व॒त॒ भू॒तानि॑ । भू॒तान्य॑सृज्यन्त । अ॒सृ॒ज्य॒न्त॒ भू॒ताना᳚म् । भू॒ताना॒म् पतिः॑ । पति॒रधि॑पतिः । अधि॑पतिरासीत् । अधि॑पति॒रित्यधि॑ - प॒तिः॒ । आ॒सी॒थ् स॒प्तभिः॑ । स॒प्तभि॑रस्तुवत । स॒प्तभि॒रिति॑ स॒प्त - भिः॒ । अ॒स्तु॒व॒त॒ स॒प्त॒र्.॒षयः॑ । स॒प्त॒र्.॒षयो॑ऽसृज्यन्त । स॒प्त॒र्.॒षय॒ इति॑ सप्त - ऋ॒षयः॑ । अ॒सृ॒ज्य॒न्त॒ धा॒ता । धा॒ताऽधि॑पतिः । अधि॑पतिरासीत् । अधि॑पति॒रित्यधि॑ - प॒तिः॒ । आ॒सी॒न् न॒वभिः॑ । न॒वभि॑रस्तुवत । न॒वभि॒रिति॑ न॒व - भिः॒ । अ॒स्तु॒व॒त॒ पि॒तरः॑ । पि॒तरो॑ऽसृज्यन्त । अ॒सृ॒ज्य॒न्तादि॑तिः । अदि॑ति॒रधि॑पत्नी । अधि॑पत्न्यासीत् । अधि॑प॒त्नीत्यधि॑ - प॒त्नी॒ । आ॒सी॒दे॒का॒द॒शभिः॑ । ए॒का॒द॒शभि॑रस्तुवत । ए॒का॒द॒शभि॒रित्ये॑काद॒श - भिः॒ । अ॒स्तु॒व॒त॒र्तवः॑ । ऋ॒तवो॑ऽसृज्यन्त । अ॒सृ॒ज्य॒न्ता॒र्त॒वः । आ॒र्त॒वोऽधि॑पतिः । अधि॑पतिरासीत् । अधि॑पति॒रित्यधि॑ - प॒तिः॒ । आ॒सी॒त् त्र॒यो॒द॒शभिः॑ । त्र॒यो॒द॒शभि॑रस्तुवत । त्र॒यो॒द॒शभि॒रिति॑ त्रयोद॒श - भिः॒ । अ॒स्तु॒व॒त॒ मासाः᳚ । मासा॑ असृज्यन्त । अ॒सृ॒ज्य॒न्त॒ स॒म्ॅव॒थ्स॒रः । स॒म्ॅव॒थ्स॒रोऽधि॑पतिः । स॒म्ॅव॒थ्स॒र इति॑ सं - व॒थ्स॒रः । अधि॑पतिरासीत् । अधि॑पति॒रित्यधि॑ - प॒तिः॒ \newline

\textbf{Jatai Paata} \newline

1. एक॑या अस्तुवता स्तुव॒तैक॒ यैक॑या अस्तुवत । \newline
2. अ॒स्तु॒व॒त॒ प्र॒जाः प्र॒जा अ॑स्तुवता स्तुवत प्र॒जाः । \newline
3. प्र॒जा अ॑धीयन्ता धीयन्त प्र॒जाः प्र॒जा अ॑धीयन्त । \newline
4. प्र॒जा इति॑ प्र - जाः । \newline
5. अ॒धी॒य॒न्त॒ प्र॒जाप॑तिः प्र॒जाप॑ति रधीयन्ता धीयन्त प्र॒जाप॑तिः । \newline
6. प्र॒जाप॑ति॒ रधि॑पति॒ रधि॑पतिः प्र॒जाप॑तिः प्र॒जाप॑ति॒ रधि॑पतिः । \newline
7. प्र॒जाप॑ति॒रिति॑ प्र॒जा - प॒तिः॒ । \newline
8. अधि॑पति रासी दासी॒ दधि॑पति॒ रधि॑पति रासीत् । \newline
9. अधि॑पति॒रित्यधि॑ - प॒तिः॒ । \newline
10. आ॒सी॒त् ति॒सृभि॑ स्ति॒सृभि॑ रासी दासीत् ति॒सृभिः॑ । \newline
11. ति॒सृभि॑ रस्तुवता स्तुवत ति॒सृभि॑ स्ति॒सृभि॑ रस्तुवत । \newline
12. ति॒सृभि॒रिति॑ ति॒सृ - भिः॒ । \newline
13. अ॒स्तु॒व॒त॒ ब्रह्म॒ ब्रह्मा᳚ स्तुवता स्तुवत॒ ब्रह्म॑ । \newline
14. ब्रह्मा॑ सृज्यता सृज्यत॒ ब्रह्म॒ ब्रह्मा॑ सृज्यत । \newline
15. अ॒सृ॒ज्य॒त॒ ब्रह्म॑णो॒ ब्रह्म॑णो ऽसृज्यता सृज्यत॒ ब्रह्म॑णः । \newline
16. ब्रह्म॑ण॒ स्पति॒ष् पति॒र् ब्रह्म॑णो॒ ब्रह्म॑ण॒ स्पतिः॑ । \newline
17. पति॒ रधि॑पति॒ रधि॑पति॒ष् पति॒ष् पति॒ रधि॑पतिः । \newline
18. अधि॑पति रासी दासी॒ दधि॑पति॒ रधि॑पति रासीत् । \newline
19. अधि॑पति॒रित्यधि॑ - प॒तिः॒ । \newline
20. आ॒सी॒त् प॒ञ्चभिः॑ प॒ञ्चभि॑ रासी दासीत् प॒ञ्चभिः॑ । \newline
21. प॒ञ्चभि॑ रस्तुवता स्तुवत प॒ञ्चभिः॑ प॒ञ्चभि॑ रस्तुवत । \newline
22. प॒ञ्चभि॒रिति॑ प॒ञ्च - भिः॒ । \newline
23. अ॒स्तु॒व॒त॒ भू॒तानि॑ भू॒तान्य॑ स्तुवता स्तुवत भू॒तानि॑ । \newline
24. भू॒तान्य॑ सृज्यन्ता सृज्यन्त भू॒तानि॑ भू॒तान्य॑ सृज्यन्त । \newline
25. अ॒सृ॒ज्य॒न्त॒ भू॒ताना᳚म् भू॒ताना॑ मसृज्यन्ता सृज्यन्त भू॒ताना᳚म् । \newline
26. भू॒ताना॒म् पति॒ष् पति॑र् भू॒ताना᳚म् भू॒ताना॒म् पतिः॑ । \newline
27. पति॒ रधि॑पति॒ रधि॑पति॒ष् पति॒ष् पति॒ रधि॑पतिः । \newline
28. अधि॑पति रासी दासी॒ दधि॑पति॒ रधि॑पति रासीत् । \newline
29. अधि॑पति॒रित्यधि॑ - प॒तिः॒ । \newline
30. आ॒सी॒थ् स॒प्तभिः॑ स॒प्तभि॑ रासी दासीथ् स॒प्तभिः॑ । \newline
31. स॒प्तभि॑ रस्तुवता स्तुवत स॒प्तभिः॑ स॒प्तभि॑ रस्तुवत । \newline
32. स॒प्तभि॒रिति॑ स॒प्त - भिः॒ । \newline
33. अ॒स्तु॒व॒त॒ स॒प्त॒र्॒.षयः॑ सप्त॒र्॒.षयो᳚ ऽस्तुवता स्तुवत सप्त॒र्॒.षयः॑ । \newline
34. स॒प्त॒र्॒.षयो॑ ऽसृज्यन्ता सृज्यन्त सप्त॒र्॒.षयः॑ सप्त॒र्॒.षयो॑ ऽसृज्यन्त । \newline
35. स॒प्त॒र्॒.षय॒ इति॑ सप्त - ऋ॒षयः॑ । \newline
36. अ॒सृ॒ज्य॒न्त॒ धा॒ता धा॒ता ऽसृ॑ज्यन्ता सृज्यन्त धा॒ता । \newline
37. धा॒ता ऽधि॑पति॒ रधि॑पतिर् धा॒ता धा॒ता ऽधि॑पतिः । \newline
38. अधि॑पति रासी दासी॒ दधि॑पति॒ रधि॑पति रासीत् । \newline
39. अधि॑पति॒रित्यधि॑ - प॒तिः॒ । \newline
40. आ॒सी॒न् न॒वभि॑र् न॒वभि॑ रासी दासीन् न॒वभिः॑ । \newline
41. न॒वभि॑ रस्तुवता स्तुवत न॒वभि॑र् न॒वभि॑ रस्तुवत । \newline
42. न॒वभि॒रिति॑ न॒व - भिः॒ । \newline
43. अ॒स्तु॒व॒त॒ पि॒तरः॑ पि॒तरो᳚ ऽस्तुवता स्तुवत पि॒तरः॑ । \newline
44. पि॒तरो॑ ऽसृज्यन्ता सृज्यन्त पि॒तरः॑ पि॒तरो॑ ऽसृज्यन्त । \newline
45. अ॒सृ॒ज्य॒न्ता दि॑ति॒ रदि॑ति रसृज्यन्ता सृज्य॒न्ता दि॑तिः । \newline
46. अदि॑ति॒ रधि॑प॒ त्न्यधि॑प॒ त्न्यदि॑ति॒ रदि॑ति॒ रधि॑पत्नी । \newline
47. अधि॑पत्न्या सीदासी॒ दधि॑प॒ त्न्यधि॑प त्न्यासीत् । \newline
48. अधि॑प॒त्नीत्यधि॑ - प॒त्नी॒ । \newline
49. आ॒सी॒ दे॒का॒द॒शभि॑ रेकाद॒शभि॑ रासी दासी देकाद॒शभिः॑ । \newline
50. ए॒का॒द॒शभि॑ रस्तुवता स्तुवतै काद॒शभि॑ रेकाद॒शभि॑ रस्तुवत । \newline
51. ए॒का॒द॒शभि॒रित्ये॑काद॒श - भिः॒ । \newline
52. अ॒स्तु॒व॒त॒ र्‌तव॑ ऋ॒तवो᳚ ऽस्तुवता स्तुवत॒ र्‌तवः॑ । \newline
53. ऋ॒तवो॑ ऽसृज्यन्ता सृज्यन्त॒ र्‌तव॑ ऋ॒तवो॑ ऽसृज्यन्त । \newline
54. अ॒सृ॒ज्य॒न्ता॒ र्‌त॒व आ᳚र्त॒वो॑ ऽसृज्यन्ता सृज्यन्ता र्‌त॒वः । \newline
55. आ॒र्त॒वो ऽधि॑पति॒ रधि॑पति रार्त॒व आ᳚र्त॒वो ऽधि॑पतिः । \newline
56. अधि॑पति रासी दासी॒ दधि॑पति॒ रधि॑पति रासीत् । \newline
57. अधि॑पति॒रित्यधि॑ - प॒तिः॒ । \newline
58. आ॒सी॒त् त्र॒यो॒द॒शभि॑ स्त्रयोद॒शभि॑ रासी दासीत् त्रयोद॒शभिः॑ । \newline
59. त्र॒यो॒द॒शभि॑ रस्तुवता स्तुवत त्रयोद॒शभि॑ स्त्रयोद॒शभि॑ रस्तुवत । \newline
60. त्र॒यो॒द॒शभि॒रिति॑ त्रयोद॒श - भिः॒ । \newline
61. अ॒स्तु॒व॒त॒ मासा॒ मासा॑ अस्तुवता स्तुवत॒ मासाः᳚ । \newline
62. मासा॑ असृज्यन्ता सृज्यन्त॒ मासा॒ मासा॑ असृज्यन्त । \newline
63. अ॒सृ॒ज्य॒न्त॒ सं॒ॅव॒थ्स॒रः सं॑ॅवथ्स॒रो॑ ऽसृज्यन्ता सृज्यन्त संॅवथ्स॒रः । \newline
64. सं॒ॅव॒थ्स॒रो ऽधि॑पति॒ रधि॑पतिः संॅवथ्स॒रः सं॑ॅवथ्स॒रो ऽधि॑पतिः । \newline
65. सं॒ॅव॒थ्स॒र इति॑ सं - व॒थ्स॒रः । \newline
66. अधि॑पति रासी दासी॒ दधि॑पति॒ रधि॑पति रासीत् । \newline
67. अधि॑पति॒रित्यधि॑ - प॒तिः॒ । \newline

\textbf{Ghana Paata } \newline

1. एक॑या अस्तुवता स्तुव॒ तैक॒यैक॑या अस्तुवत प्र॒जाः प्र॒जा अ॑स्तुव॒ तैक॒यैक॑या अस्तुवत प्र॒जाः । \newline
2. अ॒स्तु॒व॒त॒ प्र॒जाः प्र॒जा अ॑स्तुवता स्तुवत प्र॒जा अ॑धीयन्ता धीयन्त प्र॒जा अ॑स्तुवता स्तुवत प्र॒जा अ॑धीयन्त । \newline
3. प्र॒जा अ॑धीयन्ता धीयन्त प्र॒जाः प्र॒जा अ॑धीयन्त प्र॒जाप॑तिः प्र॒जाप॑ति रधीयन्त प्र॒जाः प्र॒जा अ॑धीयन्त प्र॒जाप॑तिः । \newline
4. प्र॒जा इति॑ प्र - जाः । \newline
5. अ॒धी॒य॒न्त॒ प्र॒जाप॑तिः प्र॒जाप॑ति रधीयन्ता धीयन्त प्र॒जाप॑ति॒ रधि॑पति॒ रधि॑पतिः प्र॒जाप॑ति रधीयन्ता धीयन्त प्र॒जाप॑ति॒ रधि॑पतिः । \newline
6. प्र॒जाप॑ति॒ रधि॑पति॒ रधि॑पतिः प्र॒जाप॑तिः प्र॒जाप॑ति॒ रधि॑पति रासी दासी॒ दधि॑पतिः प्र॒जाप॑तिः प्र॒जाप॑ति॒ रधि॑पति रासीत् । \newline
7. प्र॒जाप॑ति॒रिति॑ प्र॒जा - प॒तिः॒ । \newline
8. अधि॑पति रासी दासी॒ दधि॑पति॒ रधि॑पति रासीत् ति॒सृभि॑ स्ति॒सृभि॑ रासी॒ दधि॑पति॒ रधि॑पति रासीत् ति॒सृभिः॑ । \newline
9. अधि॑पति॒रित्यधि॑ - प॒तिः॒ । \newline
10. आ॒सी॒त् ति॒सृभि॑ स्ति॒सृभि॑ रासी दासीत् ति॒सृभि॑ रस्तुवता स्तुवत ति॒सृभि॑ रासी दासीत् ति॒सृभि॑ रस्तुवत । \newline
11. ति॒सृभि॑ रस्तुवता स्तुवत ति॒सृभि॑ स्ति॒सृभि॑ रस्तुवत॒ ब्रह्म॒ ब्रह्मा᳚ स्तुवत ति॒सृभि॑ स्ति॒सृभि॑ रस्तुवत॒ ब्रह्म॑ । \newline
12. ति॒सृभि॒रिति॑ ति॒सृ - भिः॒ । \newline
13. अ॒स्तु॒व॒त॒ ब्रह्म॒ ब्रह्मा᳚ स्तुवता स्तुवत॒ ब्रह्मा॑ सृज्यता सृज्यत॒ ब्रह्मा᳚ स्तुवता स्तुवत॒ ब्रह्मा॑ सृज्यत । \newline
14. ब्रह्मा॑ सृज्यता सृज्यत॒ ब्रह्म॒ ब्रह्मा॑ सृज्यत॒ ब्रह्म॑णो॒ ब्रह्म॑णो ऽसृज्यत॒ ब्रह्म॒ ब्रह्मा॑ सृज्यत॒ ब्रह्म॑णः । \newline
15. अ॒सृ॒ज्य॒त॒ ब्रह्म॑णो॒ ब्रह्म॑णो ऽसृज्यता सृज्यत॒ ब्रह्म॑ण॒ स्पति॒ष् पति॒र् ब्रह्म॑णो ऽसृज्यता सृज्यत॒ ब्रह्म॑ण॒ स्पतिः॑ । \newline
16. ब्रह्म॑ण॒ स्पति॒ष् पति॒र् ब्रह्म॑णो॒ ब्रह्म॑ण॒ स्पति॒ रधि॑पति॒ रधि॑पति॒ष् पति॒र् ब्रह्म॑णो॒ ब्रह्म॑ण॒ स्पति॒ रधि॑पतिः । \newline
17. पति॒ रधि॑पति॒ रधि॑पति॒ष् पति॒ष् पति॒ रधि॑पति रासी दासी॒ दधि॑पति॒ष् पति॒ष् पति॒ रधि॑पति रासीत् । \newline
18. अधि॑पति रासी दासी॒ दधि॑पति॒ रधि॑पति रासीत् प॒ञ्चभिः॑ प॒ञ्चभि॑ रासी॒ दधि॑पति॒ रधि॑पति रासीत् प॒ञ्चभिः॑ । \newline
19. अधि॑पति॒रित्यधि॑ - प॒तिः॒ । \newline
20. आ॒सी॒त् प॒ञ्चभिः॑ प॒ञ्चभि॑ रासी दासीत् प॒ञ्चभि॑ रस्तुवता स्तुवत प॒ञ्चभि॑ रासी दासीत् प॒ञ्चभि॑ रस्तुवत । \newline
21. प॒ञ्चभि॑ रस्तुवता स्तुवत प॒ञ्चभिः॑ प॒ञ्चभि॑ रस्तुवत भू॒तानि॑ भू॒ता न्य॑स्तुवत प॒ञ्चभिः॑ प॒ञ्चभि॑ रस्तुवत भू॒तानि॑ । \newline
22. प॒ञ्चभि॒रिति॑ प॒ञ्च - भिः॒ । \newline
23. अ॒स्तु॒व॒त॒ भू॒तानि॑ भू॒ता न्य॑स्तुवता स्तुवत भू॒ता न्य॑सृज्यन्ता सृज्यन्त भू॒ता न्य॑स्तुवता स्तुवत 
भू॒ता न्य॑सृज्यन्त । \newline
24. भू॒ता न्य॑सृज्यन्ता सृज्यन्त भू॒तानि॑ भू॒ता न्य॑सृज्यन्त भू॒ताना᳚म् भू॒ताना॑ मसृज्यन्त भू॒तानि॑ 
भू॒ता न्य॑सृज्यन्त भू॒ताना᳚म् । \newline
25. अ॒सृ॒ज्य॒न्त॒ भू॒ताना᳚म् भू॒ताना॑ मसृज्यन्ता सृज्यन्त भू॒ताना॒म् पति॒ष् पति॑र् भू॒ताना॑ मसृज्यन्ता सृज्यन्त भू॒ताना॒म् पतिः॑ । \newline
26. भू॒ताना॒म् पति॒ष् पति॑र् भू॒ताना᳚म् भू॒ताना॒म् पति॒ रधि॑पति॒ रधि॑पति॒ष् पति॑र् भू॒ताना᳚म् भू॒ताना॒म् पति॒ रधि॑पतिः । \newline
27. पति॒ रधि॑पति॒ रधि॑पति॒ष् पति॒ष् पति॒ रधि॑पति रासी दासी॒ दधि॑पति॒ष् पति॒ष् पति॒ रधि॑पति रासीत् । \newline
28. अधि॑पति रासी दासी॒ दधि॑पति॒ रधि॑पति रासीथ् स॒प्तभिः॑ स॒प्तभि॑ रासी॒ दधि॑पति॒ रधि॑पति रासीथ् स॒प्तभिः॑ । \newline
29. अधि॑पति॒रित्यधि॑ - प॒तिः॒ । \newline
30. आ॒सी॒थ् स॒प्तभिः॑ स॒प्तभि॑ रासी दासीथ् स॒प्तभि॑ रस्तुवता स्तुवत स॒प्तभि॑ रासी दासीथ् स॒प्तभि॑ रस्तुवत । \newline
31. स॒प्तभि॑ रस्तुवता स्तुवत स॒प्तभिः॑ स॒प्तभि॑ रस्तुवत सप्त॒र्॒.षयः॑ सप्त॒र्॒.षयो᳚ ऽस्तुवत स॒प्तभिः॑ स॒प्तभि॑ रस्तुवत सप्त॒र्॒.षयः॑ । \newline
32. स॒प्तभि॒रिति॑ स॒प्त - भिः॒ । \newline
33. अ॒स्तु॒व॒त॒ स॒प्त॒र्॒.षयः॑ सप्त॒र्॒.षयो᳚ ऽस्तुवता स्तुवत सप्त॒र्॒.षयो॑ ऽसृज्यन्ता सृज्यन्त सप्त॒र्॒.षयो᳚ ऽस्तुवता स्तुवत सप्त॒र्॒.षयो॑ ऽसृज्यन्त । \newline
34. स॒प्त॒र्॒.षयो॑ ऽसृज्यन्ता सृज्यन्त सप्त॒र्॒.षयः॑ सप्त॒र्॒.षयो॑ ऽसृज्यन्त धा॒ता धा॒ता ऽसृ॑ज्यन्त सप्त॒र्॒.षयः॑ सप्त॒र्॒.षयो॑ ऽसृज्यन्त धा॒ता । \newline
35. स॒प्त॒र्॒.षय॒ इति॑ सप्त - ऋ॒षयः॑ । \newline
36. अ॒सृ॒ज्य॒न्त॒ धा॒ता धा॒ता ऽसृ॑ज्यन्ता सृज्यन्त धा॒ता ऽधि॑पति॒ रधि॑पतिर् धा॒ता ऽसृ॑ज्यन्ता सृज्यन्त धा॒ता ऽधि॑पतिः । \newline
37. धा॒ता ऽधि॑पति॒ रधि॑पतिर् धा॒ता धा॒ता ऽधि॑पति रासी दासी॒ दधि॑पतिर् धा॒ता धा॒ता ऽधि॑पति रासीत् । \newline
38. अधि॑पति रासी दासी॒ दधि॑पति॒ रधि॑पति रासीन् न॒वभि॑र् न॒वभि॑ रासी॒ दधि॑पति॒ रधि॑पति रासीन् न॒वभिः॑ । \newline
39. अधि॑पति॒रित्यधि॑ - प॒तिः॒ । \newline
40. आ॒सी॒न् न॒वभि॑र् न॒वभि॑ रासी दासीन् न॒वभि॑ रस्तुवता स्तुवत न॒वभि॑ रासी दासीन् न॒वभि॑ रस्तुवत । \newline
41. न॒वभि॑ रस्तुवता स्तुवत न॒वभि॑र् न॒वभि॑ रस्तुवत पि॒तरः॑ पि॒तरो᳚ ऽस्तुवत न॒वभि॑र् न॒वभि॑ रस्तुवत पि॒तरः॑ । \newline
42. न॒वभि॒रिति॑ न॒व - भिः॒ । \newline
43. अ॒स्तु॒व॒त॒ पि॒तरः॑ पि॒तरो᳚ ऽस्तुवता स्तुवत पि॒तरो॑ ऽसृज्यन्ता सृज्यन्त पि॒तरो᳚ ऽस्तुवता स्तुवत पि॒तरो॑ ऽसृज्यन्त । \newline
44. पि॒तरो॑ ऽसृज्यन्ता सृज्यन्त पि॒तरः॑ पि॒तरो॑ ऽसृज्य॒न्ता दि॑ति॒ रदि॑ति रसृज्यन्त पि॒तरः॑ पि॒तरो॑ ऽसृज्य॒न्ता दि॑तिः । \newline
45. अ॒सृ॒ज्य॒न्ता दि॑ति॒ रदि॑ति रसृज्यन्ता सृज्य॒न्ता दि॑ति॒ रधि॑प॒त्न्य धि॑प॒त्न्यदि॑ति रसृज्यन्ता सृज्य॒न्ता दि॑ति॒ रधि॑पत्नी । \newline
46. अदि॑ति॒ रधि॑प॒त्न्य धि॑प॒त्न्य दि॑ति॒ रदि॑ति॒ रधि॑पत्न्यासी दासी॒ दधि॑प॒त्न्य दि॑ति॒ रदि॑ति॒ रधि॑पत्न्यासीत् । \newline
47. अधि॑पत्न्यासी दासी॒ दधि॑प॒त्न्य धि॑पत्न्यासी देकाद॒शभि॑ रेकाद॒शभि॑ रासी॒ दधि॑प॒त्न्य धि॑पत्न्यासी देकाद॒शभिः॑ । \newline
48. अधि॑प॒त्नीत्यधि॑ - प॒त्नी॒ । \newline
49. आ॒सी॒ दे॒का॒द॒शभि॑ रेकाद॒शभि॑ रासी दासी देकाद॒शभि॑ रस्तुवता स्तुव तैकाद॒शभि॑ रासीदा सीदेकाद॒शभि॑ रस्तुवत । \newline
50. ए॒का॒द॒शभि॑ रस्तुवता स्तुवतैकाद॒शभि॑ रेकाद॒शभि॑ रस्तुवत॒ र्‌तव॑ ऋ॒तवो᳚ ऽस्तुव तैकाद॒शभि॑ रेकाद॒शभि॑ रस्तुवत॒ र्‌तवः॑ । \newline
51. ए॒का॒द॒शभि॒रित्ये॑काद॒श - भिः॒ । \newline
52. अ॒स्तु॒व॒त॒ र्‌तव॑ ऋ॒तवो᳚ ऽस्तुवता स्तुवत॒ र्‌तवो॑ ऽसृज्यन्ता सृज्यन्त॒ र्‌तवो᳚ ऽस्तुवता स्तुवत॒ र्‌तवो॑ ऽसृज्यन्त । \newline
53. ऋ॒तवो॑ ऽसृज्यन्ता सृज्यन्त॒ र्‌तव॑ ऋ॒तवो॑ ऽसृज्यन्ता र्‌त॒व आ᳚र्त॒वो॑ ऽसृज्यन्त॒ र्‌तव॑ ऋ॒तवो॑ 
ऽसृज्यन्तार्त॒वः । \newline
54. अ॒सृ॒ज्य॒न् ता॒र्त॒व आ᳚र्त॒वो॑ ऽसृज्यन्ता सृज्यन्तार्त॒वो ऽधि॑पति॒ रधि॑पति रार्त॒वो॑ ऽसृज्यन्ता सृज्यन् तार्त॒वो ऽधि॑पतिः । \newline
55. आ॒र्त॒वो ऽधि॑पति॒ रधि॑पति रार्त॒व आ᳚र्त॒वो ऽधि॑पति रासी दासी॒ दधि॑पति रार्त॒व आ᳚र्त॒वो ऽधि॑पति रासीत् । \newline
56. अधि॑पति रासी दासी॒ दधि॑पति॒ रधि॑पति रासीत् त्रयोद॒शभि॑ स्त्रयोद॒शभि॑ रासी॒ दधि॑पति॒ रधि॑पति रासीत् त्रयोद॒शभिः॑ । \newline
57. अधि॑पति॒रित्यधि॑ - प॒तिः॒ । \newline
58. आ॒सी॒त् त्र॒यो॒द॒शभि॑ स्त्रयोद॒शभि॑ रासी दासीत् त्रयोद॒शभि॑ रस्तुवता स्तुवत त्रयोद॒शभि॑ रासी दासीत् त्रयोद॒शभि॑ रस्तुवत । \newline
59. त्र॒यो॒द॒शभि॑ रस्तुवता स्तुवत त्रयोद॒शभि॑ स्त्रयोद॒शभि॑ रस्तुवत॒ मासा॒ मासा॑ अस्तुवत त्रयोद॒शभि॑ स्त्रयोद॒शभि॑ रस्तुवत॒ मासाः᳚ । \newline
60. त्र॒यो॒द॒शभि॒रिति॑ त्रयोद॒श - भिः॒ । \newline
61. अ॒स्तु॒व॒त॒ मासा॒ मासा॑ अस्तुवता स्तुवत॒ मासा॑ असृज्यन्ता सृज्यन्त॒ मासा॑ अस्तुवता स्तुवत॒ मासा॑ असृज्यन्त । \newline
62. मासा॑ असृज्यन्ता सृज्यन्त॒ मासा॒ मासा॑ असृज्यन्त संॅवथ्स॒रः सं॑ॅवथ्स॒रो॑ ऽसृज्यन्त॒ मासा॒ मासा॑ असृज्यन्त संॅवथ्स॒रः । \newline
63. अ॒सृ॒ज्य॒न्त॒ सं॒ॅव॒थ्स॒रः सं॑ॅवथ्स॒रो॑ ऽसृज्यन्ता सृज्यन्त संॅवथ्स॒रो ऽधि॑पति॒ रधि॑पतिः संॅवथ्स॒रो॑ ऽसृज्यन्ता सृज्यन्त संॅवथ्स॒रो ऽधि॑पतिः । \newline
64. सं॒ॅव॒थ्स॒रो ऽधि॑पति॒ रधि॑पतिः संॅवथ्स॒रः सं॑ॅवथ्स॒रो ऽधि॑पति रासी दासी॒ दधि॑पतिः संॅवथ्स॒रः सं॑ॅवथ्स॒रो ऽधि॑पति रासीत् । \newline
65. सं॒ॅव॒थ्स॒र इति॑ सं - व॒थ्स॒रः । \newline
66. अधि॑पति रासी दासी॒ दधि॑पति॒ रधि॑पति रासीत् पञ्चद॒शभिः॑ पञ्चद॒शभि॑ रासी॒ दधि॑पति॒ रधि॑पति रासीत् पञ्चद॒शभिः॑ । \newline
67. अधि॑पति॒रित्यधि॑ - प॒तिः॒ । \newline
\pagebreak
\markright{ TS 4.3.10.2  \hfill https://www.vedavms.in \hfill}

\section{ TS 4.3.10.2 }

\textbf{TS 4.3.10.2 } \newline
\textbf{Samhita Paata} \newline

-रासीत् पञ्चद॒शभि॑रस्तुवत क्ष॒त्रम॑सृज्य॒तेन्द्रो ऽधि॑पतिरासीथ्-सप्तद॒शभि॑रस्तुवत प॒शवो॑ऽसृज्यन्त॒ बृह॒स्पति॒रधि॑पति-रासीन्नवद॒शभि॑-रस्तुवत शूद्रा॒र्याव॑सृज्येतामहोरा॒त्रे अधि॑पत्नी आस्ता॒मेक॑विꣳ शत्याऽस्तुव॒तैक॑शफाः प॒शवो॑ऽसृज्यन्त॒ वरु॒णो ऽधि॑पतिरासी॒त् त्रयो॑विꣳशत्याऽस्तुवत क्षु॒द्राः प॒शवो॑ऽसृज्यन्त पू॒षा ऽधि॑पतिरासी॒त् पञ्च॑विꣳशत्या ऽस्तुवता*ऽऽर॒ण्याः प॒शवो॑ऽसृज्यन्त वा॒युरधि॑पतिरासीथ् स॒प्तविꣳ॑शत्याऽस्तुवत॒ द्यावा॑पृथि॒वी व्यै॑- [  ] \newline

\textbf{Pada Paata} \newline

आ॒सी॒त् । प॒ञ्च॒द॒शभि॒रिति॑ पञ्चद॒श -भिः॒ । अ॒स्तु॒व॒त॒ । क्ष॒त्रम् । अ॒सृ॒ज्य॒त॒ । इन्द्रः॑ । अधि॑पति॒रित्यधि॑ - प॒तिः॒ । आ॒सी॒त् । स॒प्त॒द॒शभि॒रिति॑ सप्तद॒श-भिः॒ । अ॒स्तु॒व॒त॒ । प॒शवः॑ । अ॒सृ॒ज्य॒न्त॒ । बृह॒स्पतिः॑ । अधि॑पति॒रित्यधि॑ - प॒तिः॒ । आ॒सी॒त् । न॒व॒द॒शभि॒रिति॑ नवद॒श -भिः॒ । अ॒स्तु॒व॒त॒ । शू॒द्रा॒र्याविति॑ शूद्र - अ॒र्यौ । अ॒सृ॒ज्ये॒ता॒म् । अ॒हो॒रा॒त्रे इत्य॑हः - रा॒त्रे । अधि॑पत्नी॒ इत्यधि॑- प॒त्नी॒ । आ॒स्ता॒म् । एक॑विꣳश॒त्येत्येक॑ - विꣳ॒॒श॒त्या॒ । अ॒स्तु॒व॒त । एक॑शफा॒ इत्येक॑ - श॒फाः॒ । प॒शवः॑ । अ॒सृ॒ज्य॒न्त॒ । वरु॑णः । अधि॑पति॒रित्यधि॑ - प॒तिः॒ । आ॒सी॒त् । त्रयो॑विꣳश॒त्येति॒ त्रयः॑-विꣳ॒॒श॒त्या॒ । अ॒स्तु॒व॒त॒ । क्षु॒द्राः । प॒शवः॑ । अ॒सृ॒ज्य॒न्त॒ । पू॒षा । अधि॑पति॒रित्यधि॑ - प॒तिः॒ । आ॒सी॒त् । पञ्च॑विꣳश॒त्येति॒ पञ्च॑ - विꣳ॒॒श॒त्या॒ । अ॒स्तु॒व॒त॒ । आ॒र॒ण्याः । प॒शवः॑ । अ॒सृ॒ज्य॒न्त॒ । वा॒युः । अधि॑पति॒रित्यधि॑ - प॒तिः॒ । आ॒सी॒त् । स॒प्तविꣳ॑श॒त्येति॑ स॒प्त - विꣳ॒॒श॒त्या॒ । अ॒स्तु॒व॒त॒ । द्यावा॑पृथि॒वी इति॒ द्यावा᳚ - पृ॒थि॒वी । वीति॑ ।  \newline


\textbf{Krama Paata} \newline

आ॒सी॒त् प॒ञ्च॒द॒शभिः॑ । प॒ञ्च॒द॒शभि॑रस्तुवत । प॒ञ्च॒द॒शभि॒रिति॑ पञ्चद॒श - भिः॒ । अ॒स्तु॒व॒त॒ क्ष॒त्रम् । क्ष॒त्रम॑सृज्यत । अ॒सृ॒ज्य॒तेन्द्रः॑ । इन्द्रोऽधि॑पतिः । अधि॑पतिरासीत् । अधि॑पति॒रित्यधि॑ - प॒तिः॒ । आ॒सी॒थ् स॒प्त॒द॒शभिः॑ । स॒प्त॒द॒शभि॑रस्तुवत । स॒प्त॒द॒शभि॒रिति॑ सप्तद॒श - भिः॒ । अ॒स्तु॒व॒त॒ प॒शवः॑ । प॒शवो॑ऽसृज्यन्त । अ॒सृ॒ज्य॒न्त॒ बृह॒स्पतिः॑ । बृह॒स्पति॒रधि॑पतिः । अधि॑पतिरासीत् । अधि॑पति॒रित्यधि॑ - प॒तिः॒ । आ॒सी॒न् न॒व॒द॒शभिः॑ । न॒व॒द॒शभि॑रस्तुवत । न॒व॒द॒शभि॒रिति॑ नवद॒श - भिः॒ । अ॒स्तु॒व॒त॒ शू॒द्रा॒र्यौ । शू॒द्रा॒र्याव॑सृज्येताम् । शू॒द्रा॒र्याविति॑ शूद्र - अ॒र्यौ । अ॒सृ॒ज्ये॒ता॒म॒हो॒रा॒त्रे । अ॒हो॒रा॒त्रे अधि॑पत्नी । अ॒हो॒रा॒त्रे इत्य॑हः - रा॒त्रे । अधि॑पत्नी आस्ताम् । अधि॑पत्नी॒ इत्यधि॑ - प॒त्नी॒ । आ॒स्ता॒मेक॑विꣳशत्या । एक॑विꣳशत्याऽस्तुवत । एक॑विꣳश॒त्येत्येक॑ - विꣳ॒॒श॒त्या॒ । अ॒स्तु॒व॒तैक॑शफाः । एक॑शफाः प॒शवः॑ । एक॑शफा॒ इत्येक॑ - श॒फाः॒ । प॒शवो॑ऽसृज्यन्त । अ॒सृ॒ज्य॒न्त॒ वरु॑णः । वरु॒णोऽधि॑पतिः । अधि॑पतिरासीत् । अधि॑पति॒रित्यधि॑ - प॒तिः॒ । आ॒सी॒त् त्रयो॑विꣳशत्या । त्रयो॑विꣳशत्याऽस्तुवत । त्रयो॑विꣳश॒त्येति॒ त्रयः॑ - विꣳ॒॒श॒त्या॒ । अ॒स्तु॒व॒त॒ क्षु॒द्राः । क्षु॒द्राः प॒शवः॑ । प॒शवो॑ऽसृज्यन्त । अ॒सृ॒ज्य॒न्त॒ पू॒षा । पू॒षाऽधि॑पतिः । अधि॑पतिरासीत् । अधि॑पति॒रित्यधि॑ - प॒तिः॒ । आ॒सी॒त् पञ्च॑विꣳशत्या । पञ्च॑विꣳशत्याऽस्तुवत । पञ्च॑विꣳश॒त्येति॒ पञ्च॑ - विꣳ॒॒श॒त्या॒ । अ॒स्तु॒व॒ता॒र॒ण्याः । आ॒र॒ण्याः प॒शवः॑ । प॒शवो॑ऽसृज्यन्त । अ॒सृ॒ज्य॒न्त॒ वा॒युः । वा॒युरधि॑पतिः । अधि॑पतिरासीत् । अधि॑पति॒रित्यधि॑ - प॒तिः॒ । आ॒सी॒थ् स॒प्तविꣳ॑शत्या । स॒प्तविꣳ॑शत्याऽस्तुवत । स॒प्तविꣳ॑श॒त्येति॑ स॒प्त - विꣳ॒॒श॒त्या॒ । अ॒स्तु॒व॒त॒ द्यावा॑पृथि॒वी । द्यावा॑पृथि॒वी वि ( ) । द्यावा॑पृथि॒वी इति॒ द्यावा᳚ - पृ॒थि॒वी । व्यै॑ताम् \newline

\textbf{Jatai Paata} \newline

1. आ॒सी॒त् प॒ञ्च॒द॒शभिः॑ पञ्चद॒शभि॑ रासी दासीत् पञ्चद॒शभिः॑ । \newline
2. प॒ञ्च॒द॒शभि॑ रस्तुवता स्तुवत पञ्चद॒शभिः॑ पञ्चद॒शभि॑ रस्तुवत । \newline
3. प॒ञ्च॒द॒शभि॒रिति॑ पञ्चद॒श - भिः॒ । \newline
4. अ॒स्तु॒व॒त॒ क्ष॒त्रम् क्ष॒त्र म॑स्तुवता स्तुवत क्ष॒त्रम् । \newline
5. क्ष॒त्र म॑सृज्यता सृज्यत क्ष॒त्रम् क्ष॒त्र म॑सृज्यत । \newline
6. अ॒सृ॒ज्य॒तेन्द्र॒ इन्द्रो॑ ऽसृज्यता सृज्य॒तेन्द्रः॑ । \newline
7. इन्द्रो ऽधि॑पति॒ रधि॑पति॒ रिन्द्र॒ इन्द्रो ऽधि॑पतिः । \newline
8. अधि॑पति रासी दासी॒ दधि॑पति॒ रधि॑पति रासीत् । \newline
9. अधि॑पति॒रित्यधि॑ - प॒तिः॒ । \newline
10. आ॒सी॒थ् स॒प्त॒द॒शभिः॑ सप्तद॒शभि॑ रासी दासीथ् सप्तद॒शभिः॑ । \newline
11. स॒प्त॒द॒शभि॑ रस्तुवता स्तुवत सप्तद॒शभिः॑ सप्तद॒शभि॑ रस्तुवत । \newline
12. स॒प्त॒द॒शभि॒रिति॑ सप्तद॒श - भिः॒ । \newline
13. अ॒स्तु॒व॒त॒ प॒शवः॑ प॒शवो᳚ ऽस्तुवता स्तुवत प॒शवः॑ । \newline
14. प॒शवो॑ ऽसृज्यन्ता सृज्यन्त प॒शवः॑ प॒शवो॑ ऽसृज्यन्त । \newline
15. अ॒सृ॒ज्य॒न्त॒ बृह॒स्पति॒र् बृह॒स्पति॑ रसृज्यन्ता सृज्यन्त॒ बृह॒स्पतिः॑ । \newline
16. बृह॒स्पति॒ रधि॑पति॒ रधि॑पति॒र् बृह॒स्पति॒र् बृह॒स्पति॒ रधि॑पतिः । \newline
17. अधि॑पति रासी दासी॒ दधि॑पति॒ रधि॑पति रासीत् । \newline
18. अधि॑पति॒रित्यधि॑ - प॒तिः॒ । \newline
19. आ॒सी॒न् न॒व॒द॒शभि॑र् नवद॒शभि॑ रासी दासीन् नवद॒शभिः॑ । \newline
20. न॒व॒द॒शभि॑ रस्तुवता स्तुवत नवद॒शभि॑र् नवद॒शभि॑ रस्तुवत । \newline
21. न॒व॒द॒शभि॒रिति॑ नवद॒श - भिः॒ । \newline
22. अ॒स्तु॒व॒त॒ शू॒द्रा॒र्यौ शू᳚द्रा॒र्या व॑स्तुवता स्तुवत शूद्रा॒र्यौ । \newline
23. शू॒द्रा॒र्या व॑सृज्येता मसृज्येताꣳ शूद्रा॒र्यौ शू᳚द्रा॒र्या व॑सृज्येताम् । \newline
24. शू॒द्रा॒र्याविति॑ शूद्र - अ॒र्यौ । \newline
25. अ॒सृ॒ज्ये॒ता॒ म॒हो॒रा॒त्रे अ॑होरा॒त्रे अ॑सृज्येता मसृज्येता महोरा॒त्रे । \newline
26. अ॒हो॒रा॒त्रे अधि॑पत्नी॒ अधि॑पत्नी अहोरा॒त्रे अ॑होरा॒त्रे अधि॑पत्नी । \newline
27. अ॒हो॒रा॒त्रे इत्य॑हः - रा॒त्रे । \newline
28. अधि॑पत्नी आस्ता मास्ता॒ मधि॑पत्नी॒ अधि॑पत्नी आस्ताम् । \newline
29. अधि॑पत्नी॒ इत्यधि॑ - प॒त्नी॒ । \newline
30. आ॒स्ता॒ मेक॑विꣳश॒त्यै क॑विꣳशत्या ऽऽस्ता मास्ता॒ मेक॑विꣳशत्या । \newline
31. एक॑विꣳशत्या ऽस्तुव॒ता स्तु॑व॒तै क॑विꣳश॒ त्यैक॑विꣳशत्या ऽस्तुव॒त । \newline
32. एक॑विꣳश॒त्येत्येक॑ - विꣳ॒॒श॒त्या॒ । \newline
33. अ॒स्तु॒व॒तै क॑शफा॒ एक॑शफा अस्तुव॒ता स्तु॑व॒तै क॑शफाः । \newline
34. एक॑शफाः प॒शवः॑ प॒शव॒ एक॑शफा॒ एक॑शफाः प॒शवः॑ । \newline
35. एक॑शफा॒ इत्येक॑ - श॒फाः॒ । \newline
36. प॒शवो॑ ऽसृज्यन्ता सृज्यन्त प॒शवः॑ प॒शवो॑ ऽसृज्यन्त । \newline
37. अ॒सृ॒ज्य॒न्त॒ वरु॑णो॒ वरु॑णो ऽसृज्यन्ता सृज्यन्त॒ वरु॑णः । \newline
38. वरु॒णो ऽधि॑पति॒ रधि॑पति॒र् वरु॑णो॒ वरु॒णो ऽधि॑पतिः । \newline
39. अधि॑पति रासी दासी॒ दधि॑पति॒ रधि॑पति रासीत् । \newline
40. अधि॑पति॒रित्यधि॑ - प॒तिः॒ । \newline
41. आ॒सी॒त् त्रयो॑विꣳशत्या॒ त्रयो॑विꣳशत्या ऽऽसीदासी॒त् त्रयो॑विꣳशत्या । \newline
42. त्रयो॑विꣳशत्या ऽस्तुवता स्तुवत॒ त्रयो॑विꣳशत्या॒ त्रयो॑विꣳशत्या ऽस्तुवत । \newline
43. त्रयो॑विꣳश॒त्येति॒ त्रयः॑ - विꣳ॒॒श॒त्या॒ । \newline
44. अ॒स्तु॒व॒त॒ क्षु॒द्राः क्षु॒द्रा अ॑स्तुवता स्तुवत क्षु॒द्राः । \newline
45. क्षु॒द्राः प॒शवः॑ प॒शवः॑ क्षु॒द्राः क्षु॒द्राः प॒शवः॑ । \newline
46. प॒शवो॑ ऽसृज्यन्ता सृज्यन्त प॒शवः॑ प॒शवो॑ ऽसृज्यन्त । \newline
47. अ॒सृ॒ज्य॒न्त॒ पू॒षा पू॒षा ऽसृ॑ज्यन्ता सृज्यन्त पू॒षा । \newline
48. पू॒षा ऽधि॑पति॒ रधि॑पतिः पू॒षा पू॒षा ऽधि॑पतिः । \newline
49. अधि॑पति रासी दासी॒ दधि॑पति॒ रधि॑पति रासीत् । \newline
50. अधि॑पति॒रित्यधि॑ - प॒तिः॒ । \newline
51. आ॒सी॒त् पञ्च॑विꣳशत्या॒ पञ्च॑विꣳशत्या ऽऽसी दासी॒त् पञ्च॑विꣳशत्या । \newline
52. पञ्च॑विꣳशत्या ऽस्तुवता स्तुवत॒ पञ्च॑विꣳशत्या॒ पञ्च॑विꣳशत्या ऽस्तुवत । \newline
53. पञ्च॑विꣳश॒त्येति॒ पञ्च॑ - विꣳ॒॒श॒त्या॒ । \newline
54. अ॒स्तु॒व॒ता॒ र॒ण्या आ॑र॒ण्या अ॑स्तुवता स्तुवता र॒ण्याः । \newline
55. आ॒र॒ण्याः प॒शवः॑ प॒शव॑ आर॒ण्या आ॑र॒ण्याः प॒शवः॑ । \newline
56. प॒शवो॑ ऽसृज्यन्ता सृज्यन्त प॒शवः॑ प॒शवो॑ ऽसृज्यन्त । \newline
57. अ॒सृ॒ज्य॒न्त॒ वा॒युर् वा॒यु र॑सृज्यन्ता सृज्यन्त वा॒युः । \newline
58. वा॒यु रधि॑पति॒ रधि॑पतिर् वा॒युर् वा॒यु रधि॑पतिः । \newline
59. अधि॑पति रासी दासी॒ दधि॑पति॒ रधि॑पति रासीत् । \newline
60. अधि॑पति॒रित्यधि॑ - प॒तिः॒ । \newline
61. आ॒सी॒थ् स॒प्तविꣳ॑शत्या स॒प्तविꣳ॑शत्या ऽऽसी दासीथ् स॒प्तविꣳ॑शत्या । \newline
62. स॒प्तविꣳ॑शत्या ऽस्तुवता स्तुवत स॒प्तविꣳ॑शत्या स॒प्तविꣳ॑शत्या ऽस्तुवत । \newline
63. स॒प्तविꣳ॑श॒त्येति॑ स॒प्त - विꣳ॒॒श॒त्या॒ । \newline
64. अ॒स्तु॒व॒त॒ द्यावा॑पृथि॒वी द्यावा॑पृथि॒वी अ॑स्तुवता स्तुवत॒ द्यावा॑पृथि॒वी । \newline
65. द्यावा॑पृथि॒वी वि वि द्यावा॑पृथि॒वी द्यावा॑पृथि॒वी वि । \newline
66. द्यावा॑पृथि॒वी इति॒ द्यावा᳚ - पृ॒थि॒वी । \newline
67. व्यै॑ता मैतां॒ ॅवि व्यै॑ताम् । \newline

\textbf{Ghana Paata } \newline

1. आ॒सी॒त् प॒ञ्च॒द॒शभिः॑ पञ्चद॒शभि॑ रासी दासीत् पञ्चद॒शभि॑ रस्तुवता स्तुवत पञ्चद॒शभि॑ रासी दासीत् पञ्चद॒शभि॑ रस्तुवत । \newline
2. प॒ञ्च॒द॒शभि॑ रस्तुवता स्तुवत पञ्चद॒शभिः॑ पञ्चद॒शभि॑ रस्तुवत क्ष॒त्रम् क्ष॒त्र म॑स्तुवत पञ्चद॒शभिः॑ पञ्चद॒शभि॑ रस्तुवत क्ष॒त्रम् । \newline
3. प॒ञ्च॒द॒शभि॒रिति॑ पञ्चद॒श - भिः॒ । \newline
4. अ॒स्तु॒व॒त॒ क्ष॒त्रम् क्ष॒त्र म॑स्तुवता स्तुवत क्ष॒त्र म॑सृज्यता सृज्यत क्ष॒त्र म॑स्तुवता स्तुवत क्ष॒त्र म॑सृज्यत । \newline
5. क्ष॒त्र म॑सृज्यता सृज्यत क्ष॒त्रम् क्ष॒त्र म॑सृज्य॒तेन्द्र॒ इन्द्रो॑ ऽसृज्यत क्ष॒त्रम् क्ष॒त्र 
म॑सृज्य॒तेन्द्रः॑ । \newline
6. अ॒सृ॒ज्य॒तेन्द्र॒ इन्द्रो॑ ऽसृज्यता सृज्य॒तेन्द्रो ऽधि॑पति॒ रधि॑पति॒ रिन्द्रो॑ ऽसृज्यता सृज्य॒तेन्द्रो ऽधि॑पतिः । \newline
7. इन्द्रो ऽधि॑पति॒ रधि॑पति॒ रिन्द्र॒ इन्द्रो ऽधि॑पति रासी दासी॒ दधि॑पति॒ रिन्द्र॒ इन्द्रो ऽधि॑पति रासीत् । \newline
8. अधि॑पति रासी दासी॒ दधि॑पति॒ रधि॑पति रासीथ् सप्तद॒शभिः॑ सप्तद॒शभि॑ रासी॒ दधि॑पति॒ रधि॑पति रासीथ् सप्तद॒शभिः॑ । \newline
9. अधि॑पति॒रित्यधि॑ - प॒तिः॒ । \newline
10. आ॒सी॒थ् स॒प्त॒द॒शभिः॑ सप्त द॒शभि॑ रासीदासीथ् सप्तद॒शभि॑ रस्तुवता स्तुवत सप्तद॒शभि॑ रासी दासीथ् सप्तद॒शभि॑ रस्तुवत । \newline
11. स॒प्त॒द॒शभि॑ रस्तुवता स्तुवत सप्तद॒शभिः॑ सप्तद॒शभि॑ रस्तुवत प॒शवः॑ प॒शवो᳚ ऽस्तुवत सप्तद॒शभिः॑ सप्तद॒शभि॑ रस्तुवत प॒शवः॑ । \newline
12. स॒प्त॒द॒शभि॒रिति॑ सप्तद॒श - भिः॒ । \newline
13. अ॒स्तु॒व॒त॒ प॒शवः॑ प॒शवो᳚ ऽस्तुवता स्तुवत प॒शवो॑ ऽसृज्यन्ता सृज्यन्त प॒शवो᳚ ऽस्तुवता स्तुवत प॒शवो॑ ऽसृज्यन्त । \newline
14. प॒शवो॑ ऽसृज्यन्ता सृज्यन्त प॒शवः॑ प॒शवो॑ ऽसृज्यन्त॒ बृह॒स्पति॒र् बृह॒स्पति॑ रसृज्यन्त प॒शवः॑ प॒शवो॑ ऽसृज्यन्त॒ बृह॒स्पतिः॑ । \newline
15. अ॒सृ॒ज्य॒न्त॒ बृह॒स्पति॒र् बृह॒स्पति॑ रसृज्यन्ता सृज्यन्त॒ बृह॒स्पति॒ रधि॑पति॒ रधि॑पति॒र् बृह॒स्पति॑ रसृज्यन्ता सृज्यन्त॒ बृह॒स्पति॒ रधि॑पतिः । \newline
16. बृह॒स्पति॒ रधि॑पति॒ रधि॑पति॒र् बृह॒स्पति॒र् बृह॒स्पति॒ रधि॑पति रासी दासी॒ दधि॑पति॒र् बृह॒स्पति॒र् बृह॒स्पति॒ रधि॑पति रासीत् । \newline
17. अधि॑पति रासी दासी॒ दधि॑पति॒ रधि॑पति रासीन् नवद॒शभि॑र् नवद॒शभि॑ रासी॒ दधि॑पति॒ रधि॑पति रासीन् नवद॒शभिः॑ । \newline
18. अधि॑पति॒रित्यधि॑ - प॒तिः॒ । \newline
19. आ॒सी॒न् न॒व॒द॒शभि॑र् नवद॒शभि॑ रासी दासीन् नवद॒शभि॑ रस्तुवता स्तुवत नवद॒शभि॑ रासी दासीन् नव द॒शभि॑ रस्तुवत । \newline
20. न॒व॒द॒शभि॑ रस्तुवता स्तुवत नवद॒शभि॑र् नवद॒शभि॑ रस्तुवत शूद्रा॒र्यौ शू᳚द्रा॒र्या व॑स्तुवत नवद॒शभि॑र् नवद॒शभि॑ रस्तुवत शूद्रा॒र्यौ । \newline
21. न॒व॒द॒शभि॒रिति॑ नवद॒श - भिः॒ । \newline
22. अ॒स्तु॒व॒त॒ शू॒द्रा॒र्यौ शू᳚द्रा॒र्या व॑स्तुवता स्तुवत शूद्रा॒र्या व॑सृज्येता मसृज्येताꣳ शूद्रा॒र्या व॑स्तुवता स्तुवत शूद्रा॒र्या व॑सृज्येताम् । \newline
23. शू॒द्रा॒र्या व॑सृज्येता मसृज्येताꣳ शूद्रा॒र्यौ शू᳚द्रा॒र्या व॑सृज्येता महोरा॒त्रे अ॑होरा॒त्रे अ॑सृज्येताꣳ शूद्रा॒र्यौ शू᳚द्रा॒र्या व॑सृज्येता महोरा॒त्रे । \newline
24. शू॒द्रा॒र्याविति॑ शूद्र - अ॒र्यौ । \newline
25. अ॒सृ॒ज्ये॒ता॒ म॒हो॒रा॒त्रे अ॑होरा॒त्रे अ॑सृज्येता मसृज्येता महोरा॒त्रे अधि॑पत्नी॒ अधि॑पत्नी अहोरा॒त्रे अ॑सृज्येता मसृज्येता महोरा॒त्रे अधि॑पत्नी । \newline
26. अ॒हो॒रा॒त्रे अधि॑पत्नी॒ अधि॑पत्नी अहोरा॒त्रे अ॑होरा॒त्रे अधि॑पत्नी आस्ता मास्ता॒ मधि॑पत्नी अहोरा॒त्रे अ॑होरा॒त्रे अधि॑पत्नी आस्ताम् । \newline
27. अ॒हो॒रा॒त्रे इत्य॑हः - रा॒त्रे । \newline
28. अधि॑पत्नी आस्ता मास्ता॒ मधि॑पत्नी॒ अधि॑पत्नी आस्ता॒ मेक॑विꣳश॒ त्यैक॑विꣳशत्या ऽऽस्ता॒ मधि॑पत्नी॒ अधि॑पत्नी आस्ता॒ मेक॑विꣳशत्या । \newline
29. अधि॑पत्नी॒ इत्यधि॑ - प॒त्नी॒ । \newline
30. आ॒स्ता॒ मेक॑विꣳश॒ त्यैक॑विꣳशत्या ऽऽस्ता मास्ता॒ मेक॑विꣳशत्या ऽस्तुव॒ता स्तु॑व॒ तैक॑विꣳशत्या ऽऽस्ता मास्ता॒ मेक॑विꣳशत्या ऽस्तुव॒त । \newline
31. एक॑विꣳशत्या ऽस्तुव॒ता स्तु॑व॒ तैक॑विꣳश॒ त्यैक॑विꣳशत्या ऽस्तुव॒ तैक॑शफा॒ एक॑शफा 
अस्तुव॒ तैक॑विꣳश॒ त्यैक॑विꣳशत्या ऽस्तुव॒ तैक॑शफाः । \newline
32. एक॑विꣳश॒त्येत्येक॑ - विꣳ॒॒श॒त्या॒ । \newline
33. अ॒स्तु॒व॒ तैक॑शफा॒ एक॑शफा अस्तुव॒ता स्तु॑व॒ तैक॑शफाः प॒शवः॑ प॒शव॒ एक॑शफा अस्तुव॒ता 
स्तु॑व॒ तैक॑शफाः प॒शवः॑ । \newline
34. एक॑शफाः प॒शवः॑ प॒शव॒ एक॑शफा॒ एक॑शफाः प॒शवो॑ ऽसृज्यन्ता सृज्यन्त प॒शव॒ एक॑शफा॒ एक॑शफाः प॒शवो॑ ऽसृज्यन्त । \newline
35. एक॑शफा॒ इत्येक॑ - श॒फाः॒ । \newline
36. प॒शवो॑ ऽसृज्यन्ता सृज्यन्त प॒शवः॑ प॒शवो॑ ऽसृज्यन्त॒ वरु॑णो॒ वरु॑णो ऽसृज्यन्त प॒शवः॑ प॒शवो॑ ऽसृज्यन्त॒ वरु॑णः । \newline
37. अ॒सृ॒ज्य॒न्त॒ वरु॑णो॒ वरु॑णो ऽसृज्यन्ता सृज्यन्त॒ वरु॒णो ऽधि॑पति॒ रधि॑पति॒र् वरु॑णो ऽसृज्यन्ता सृज्यन्त॒ वरु॒णो ऽधि॑पतिः । \newline
38. वरु॒णो ऽधि॑पति॒ रधि॑पति॒र् वरु॑णो॒ वरु॒णो ऽधि॑पति रासी दासी॒ दधि॑पति॒र् वरु॑णो॒ वरु॒णो ऽधि॑पति रासीत् । \newline
39. अधि॑पति रासी दासी॒ दधि॑पति॒ रधि॑पति रासी॒त् त्रयो॑विꣳशत्या॒ त्रयो॑विꣳशत्या ऽऽसी॒ दधि॑पति॒ रधि॑पति रासी॒त् त्रयो॑विꣳशत्या । \newline
40. अधि॑पति॒रित्यधि॑ - प॒तिः॒ । \newline
41. आ॒सी॒त् त्रयो॑विꣳशत्या॒ त्रयो॑विꣳशत्या ऽऽसी दासी॒त् त्रयो॑विꣳशत्या ऽस्तुवता स्तुवत॒ त्रयो॑विꣳशत्या ऽऽसी दासी॒त् त्रयो॑विꣳशत्या ऽस्तुवत । \newline
42. त्रयो॑विꣳशत्या ऽस्तुवता स्तुवत॒ त्रयो॑विꣳशत्या॒ त्रयो॑विꣳशत्या ऽस्तुवत क्षु॒द्राः क्षु॒द्रा अ॑स्तुवत॒ त्रयो॑विꣳशत्या॒ त्रयो॑विꣳशत्या ऽस्तुवत क्षु॒द्राः । \newline
43. त्रयो॑विꣳश॒त्येति॒ त्रयः॑ - विꣳ॒॒श॒त्या॒ । \newline
44. अ॒स्तु॒व॒त॒ क्षु॒द्राः क्षु॒द्रा अ॑स्तुवता स्तुवत क्षु॒द्राः प॒शवः॑ प॒शवः॑ क्षु॒द्रा अ॑स्तुवता स्तुवत क्षु॒द्राः प॒शवः॑ । \newline
45. क्षु॒द्राः प॒शवः॑ प॒शवः॑ क्षु॒द्राः क्षु॒द्राः प॒शवो॑ ऽसृज्यन्ता सृज्यन्त प॒शवः॑ क्षु॒द्राः क्षु॒द्राः प॒शवो॑ ऽसृज्यन्त । \newline
46. प॒शवो॑ ऽसृज्यन्ता सृज्यन्त प॒शवः॑ प॒शवो॑ ऽसृज्यन्त पू॒षा पू॒षा ऽसृ॑ज्यन्त प॒शवः॑ प॒शवो॑ ऽसृज्यन्त पू॒षा । \newline
47. अ॒सृ॒ज्य॒न्त॒ पू॒षा पू॒षा ऽसृ॑ज्यन्ता सृज्यन्त पू॒षा ऽधि॑पति॒ रधि॑पतिः पू॒षा ऽसृ॑ज्यन्ता सृज्यन्त पू॒षा ऽधि॑पतिः । \newline
48. पू॒षा ऽधि॑पति॒ रधि॑पतिः पू॒षा पू॒षा ऽधि॑पति रासी दासी॒ दधि॑पतिः पू॒षा पू॒षा ऽधि॑पति रासीत् । \newline
49. अधि॑पति रासी दासी॒ दधि॑पति॒ रधि॑पति रासी॒त् पञ्च॑विꣳशत्या॒ पञ्च॑विꣳशत्या ऽऽसी॒ दधि॑पति॒ रधि॑पति रासी॒त् पञ्च॑विꣳशत्या । \newline
50. अधि॑पति॒रित्यधि॑ - प॒तिः॒ । \newline
51. आ॒सी॒त् पञ्च॑विꣳशत्या॒ पञ्च॑विꣳशत्या ऽऽसी दासी॒त् पञ्च॑विꣳशत्या ऽस्तुवता स्तुवत॒ पञ्च॑विꣳशत्या ऽऽसी दासी॒त् पञ्च॑विꣳशत्या ऽस्तुवत । \newline
52. पञ्च॑विꣳशत्या ऽस्तुवता स्तुवत॒ पञ्च॑विꣳशत्या॒ पञ्च॑विꣳशत्या ऽस्तुवता र॒ण्या आ॑र॒ण्या अ॑स्तुवत॒ पञ्च॑विꣳशत्या॒ पञ्च॑विꣳशत्या ऽस्तुवता र॒ण्याः । \newline
53. पञ्च॑विꣳश॒त्येति॒ पञ्च॑ - विꣳ॒॒श॒त्या॒ । \newline
54. अ॒स्तु॒व॒ता॒ र॒ण्या आ॑र॒ण्या अ॑स्तुवता स्तुवता र॒ण्याः प॒शवः॑ प॒शव॑ आर॒ण्या अ॑स्तुवता स्तुवता र॒ण्याः प॒शवः॑ । \newline
55. आ॒र॒ण्याः प॒शवः॑ प॒शव॑ आर॒ण्या आ॑र॒ण्याः प॒शवो॑ ऽसृज्यन्ता सृज्यन्त प॒शव॑ आर॒ण्या आ॑र॒ण्याः प॒शवो॑ ऽसृज्यन्त । \newline
56. प॒शवो॑ ऽसृज्यन्ता सृज्यन्त प॒शवः॑ प॒शवो॑ ऽसृज्यन्त वा॒युर् वा॒यु र॑सृज्यन्त प॒शवः॑ प॒शवो॑ ऽसृज्यन्त वा॒युः । \newline
57. अ॒सृ॒ज्य॒न्त॒ वा॒युर् वा॒यु र॑सृज्यन्ता सृज्यन्त वा॒यु रधि॑पति॒ रधि॑पतिर् वा॒यु र॑सृज्यन्ता सृज्यन्त वा॒यु रधि॑पतिः । \newline
58. वा॒यु रधि॑पति॒ रधि॑पतिर् वा॒युर् वा॒यु रधि॑पति रासी दासी॒ दधि॑पतिर् वा॒युर् वा॒यु रधि॑पति रासीत् । \newline
59. अधि॑पति रासी दासी॒ दधि॑पति॒ रधि॑पति रासीथ् स॒प्तविꣳ॑शत्या स॒प्तविꣳ॑शत्या ऽऽसी॒ दधि॑पति॒ रधि॑पति रासीथ् स॒प्तविꣳ॑शत्या । \newline
60. अधि॑पति॒रित्यधि॑ - प॒तिः॒ । \newline
61. आ॒सी॒थ् स॒प्तविꣳ॑शत्या स॒प्तविꣳ॑शत्या ऽऽसी दासीथ् स॒प्तविꣳ॑शत्या ऽस्तुवता स्तुवत स॒प्तविꣳ॑शत्या ऽऽसी दासीथ् स॒प्तविꣳ॑शत्या ऽस्तुवत । \newline
62. स॒प्तविꣳ॑शत्या ऽस्तुवता स्तुवत स॒प्तविꣳ॑शत्या स॒प्तविꣳ॑शत्या ऽस्तुवत॒ द्यावा॑पृथि॒वी द्यावा॑पृथि॒वी अ॑स्तुवत स॒प्तविꣳ॑शत्या स॒प्तविꣳ॑शत्या ऽस्तुवत॒ द्यावा॑पृथि॒वी । \newline
63. स॒प्तविꣳ॑श॒त्येति॑ स॒प्त - विꣳ॒॒श॒त्या॒ । \newline
64. अ॒स्तु॒व॒त॒ द्यावा॑पृथि॒वी द्यावा॑पृथि॒वी अ॑स्तुवता स्तुवत॒ द्यावा॑पृथि॒वी वि वि द्यावा॑पृथि॒वी अ॑स्तुवता स्तुवत॒ द्यावा॑पृथि॒वी वि । \newline
65. द्यावा॑पृथि॒वी वि वि द्यावा॑पृथि॒वी द्यावा॑पृथि॒वी व्यै॑ता मैतां॒ ॅवि द्यावा॑पृथि॒वी द्यावा॑पृथि॒वी व्यै॑ताम् । \newline
66. द्यावा॑पृथि॒वी इति॒ द्यावा᳚ - पृ॒थि॒वी । \newline
67. व्यै॑ता मैतां॒ ॅवि व्यै॑तां॒ ॅवस॑वो॒ वस॑व ऐतां॒ ॅवि व्यै॑तां॒ ॅवस॑वः । \newline
\pagebreak
\markright{ TS 4.3.10.3  \hfill https://www.vedavms.in \hfill}

\section{ TS 4.3.10.3 }

\textbf{TS 4.3.10.3 } \newline
\textbf{Samhita Paata} \newline

-तां॒ ॅवस॑वो रु॒द्रा आ॑दि॒त्या अनु॒ व्या॑य॒न् तेषा॒माधि॑पत्यमासी॒-न्नव॑विꣳ शत्याऽस्तुवत॒ वन॒स्पत॑योऽसृज्यन्त॒ सोमो ऽधि॑पतिरासी॒-देक॑त्रिꣳशता ऽस्तुवत प्र॒जा अ॑सृज्यन्त॒ यावा॑नां॒ चाया॑वानां॒ चाऽऽधि॑पत्यमासी॒त् त्रय॑स्त्रिꣳशता ऽस्तुवत भू॒तान्य॑शाम्यन् प्र॒जाप॑तिः परमे॒ष्ठ्यधि॑पतिरासीत् ॥ \newline

\textbf{Pada Paata} \newline

ए॒ता॒म् । वस॑वः । रु॒द्राः । आ॒दि॒त्याः । अनु॑ । वीति॑ । आ॒य॒न्न् । तेषा᳚म् । आधि॑पत्य॒मित्याधि॑ - प॒त्य॒म् । आ॒सी॒त् । नव॑विꣳश॒त्येति॒ नव॑ - विꣳ॒॒श॒त्या॒ । अ॒स्तु॒व॒त॒ । वन॒स्पत॑यः । अ॒सृ॒ज्य॒न्त॒ । सोमः॑ । अधि॑पति॒रित्यधि॑ - प॒तिः॒ । आ॒सी॒त् । एक॑त्रिꣳश॒तेत्ये॑क-त्रिꣳ॒॒श॒ता॒ । अ॒स्तु॒व॒त॒ । प्र॒जा इति॑ प्र - जाः । अ॒सृ॒ज्य॒न्त॒ । यावा॑नाम् । च॒ । अया॑वानाम् । च॒ । आधि॑पत्य॒मित्याधि॑ - प॒त्य॒म् । आ॒सी॒त् । त्रय॑स्त्रिꣳश॒तेति॒ त्रयः॑-त्रिꣳ॒॒श॒ता॒ । अ॒स्तु॒व॒त॒ । भू॒तानि॑ । अ॒शा॒म्य॒न्न् । प्र॒जाप॑ति॒रिति॑ प्र॒जा - प॒तिः॒ । प॒र॒मे॒ष्ठी । अधि॑पति॒रित्यधि॑ - प॒तिः॒ । आ॒सी॒त् ॥  \newline


\textbf{Krama Paata} \newline

ऐ॒तां॒ ॅवस॑वः । वस॑वो रु॒द्राः । रु॒द्रा आ॑दि॒त्याः । आ॒दि॒त्या अनु॑ । अनु॒ वि । व्या॑यन्न् । आ॒य॒न् तेषा᳚म् । तेषा॒माधि॑पत्यम् । आधि॑पत्यमासीत् । आधि॑पत्य॒मित्याधि॑ - प॒त्य॒म् । आ॒सी॒न् नव॑विꣳशत्या । नव॑विꣳशत्याऽस्तुवत । नव॑विꣳश॒त्येति॒ नव॑ - विꣳ॒॒श॒त्या॒ । अ॒स्तु॒व॒त॒ वन॒स्पत॑यः । वन॒स्पत॑योऽसृज्यन्त । अ॒सृ॒ज्य॒न्त॒ सोमः॑ । सोमोऽधि॑पतिः । अधि॑पतिरासीत् । अधि॑पति॒रत्यधि॑ - प॒तिः॒ । आ॒सी॒देक॑त्रिꣳशताः । एक॑त्रिꣳशताऽस्तुवत । एक॑त्रिꣳश॒तेत्येक॑ - त्रिꣳ॒॒श॒ता॒ । अ॒स्तु॒व॒त॒ प्र॒जाः । प्र॒जा अ॑सृज्यन्त । प्र॒जा इति॑ प्र - जाः । अ॒सृ॒ज्य॒न्त॒ यावा॑नाम् । यावा॑नाम् च । चाया॑वानाम् । 
अया॑वानाम् च । चाधि॑पत्यम् । आधि॑पत्यमासीत् । आधि॑पत्य॒मित्याधि॑ - प॒त्य॒म् । आ॒सी॒त् त्रय॑स्त्रिꣳशता । त्रय॑स्त्रिꣳशताऽस्तुवत । त्रय॑स्त्रिꣳश॒तेति॒ त्रयः॑ - त्रिꣳ॒॒श॒ता॒ । अ॒स्तु॒व॒त॒ भू॒तानि॑ । भू॒तान्य॑शाम्यन्न् । अ॒शा॒म्य॒न् प्र॒जाप॑तिः । प्र॒जाप॑तिः 
परमे॒ष्ठी । प्र॒जाप॑ति॒रिति॑ प्र॒जा - प॒तिः॒ । 
प॒र॒मे॒ष्ठ्यधि॑पतिः । अधि॑पतिरासीत् । अधि॑पति॒रित्यधि॑ - प॒तिः॒ । आ॒सी॒दित्या॑सीत् । \newline

\textbf{Jatai Paata} \newline

1. ऐ॒तां॒ ॅवस॑वो॒ वस॑व ऐता मैतां॒ ॅवस॑वः । \newline
2. वस॑वो रु॒द्रा रु॒द्रा वस॑वो॒ वस॑वो रु॒द्राः । \newline
3. रु॒द्रा आ॑दि॒त्या आ॑दि॒त्या रु॒द्रा रु॒द्रा आ॑दि॒त्याः । \newline
4. आ॒दि॒त्या अन्वन् वा॑दि॒त्या आ॑दि॒त्या अनु॑ । \newline
5. अनु॒ वि व्यन् वनु॒ वि । \newline
6. व्या॑यन् नाय॒न्॒. वि व्या॑यन्न् । \newline
7. आ॒य॒न् तेषा॒म् तेषा॑ मायन् नाय॒न् तेषा᳚म् । \newline
8. तेषा॒ माधि॑पत्य॒ माधि॑पत्य॒म् तेषा॒म् तेषा॒ माधि॑पत्यम् । \newline
9. आधि॑पत्य मासी दासी॒ दाधि॑पत्य॒ माधि॑पत्य मासीत् । \newline
10. आधि॑पत्य॒मित्याधि॑ - प॒त्य॒म् । \newline
11. आ॒सी॒न् नव॑विꣳशत्या॒ नव॑विꣳशत्या ऽऽसी दासी॒न् नव॑विꣳशत्या । \newline
12. नव॑विꣳशत्या ऽस्तुवता स्तुवत॒ नव॑विꣳशत्या॒ नव॑विꣳशत्या ऽस्तुवत । \newline
13. नव॑विꣳश॒त्येति॒ नव॑ - विꣳ॒॒श॒त्या॒ । \newline
14. अ॒स्तु॒व॒त॒ वन॒स्पत॑यो॒ वन॒स्पत॑यो ऽस्तुवता स्तुवत॒ वन॒स्पत॑यः । \newline
15. वन॒स्पत॑यो ऽसृज्यन्ता सृज्यन्त॒ वन॒स्पत॑यो॒ वन॒स्पत॑यो ऽसृज्यन्त । \newline
16. अ॒सृ॒ज्य॒न्त॒ सोमः॒ सोमो॑ ऽसृज्यन्ता सृज्यन्त॒ सोमः॑ । \newline
17. सोमो ऽधि॑पति॒ रधि॑पतिः॒ सोमः॒ सोमो ऽधि॑पतिः । \newline
18. अधि॑पति रासी दासी॒ दधि॑पति॒ रधि॑पति रासीत् । \newline
19. अधि॑पति॒रित्यधि॑ - प॒तिः॒ । \newline
20. आ॒सी॒ देक॑त्रिꣳश॒ तैक॑त्रिꣳशता ऽऽसी दासी॒ देक॑त्रिꣳशता । \newline
21. एक॑त्रिꣳशता ऽस्तुवता स्तुव॒ तैक॑त्रिꣳश॒ तैक॑त्रिꣳशता ऽस्तुवत । \newline
22. एक॑त्रिꣳश॒तेत्येक॑ - त्रिꣳ॒॒श॒ता॒ । \newline
23. अ॒स्तु॒व॒त॒ प्र॒जाः प्र॒जा अ॑स्तुवता स्तुवत प्र॒जाः । \newline
24. प्र॒जा अ॑सृज्यन्ता सृज्यन्त प्र॒जाः प्र॒जा अ॑सृज्यन्त । \newline
25. प्र॒जा इति॑ प्र - जाः । \newline
26. अ॒सृ॒ज्य॒न्त॒ यावा॑नां॒ ॅयावा॑ना मसृज्यन्ता सृज्यन्त॒ यावा॑नाम् । \newline
27. यावा॑नाम् च च॒ यावा॑नां॒ ॅयावा॑नाम् च । \newline
28. चाया॑वाना॒ मया॑वानाम् च॒ चाया॑वानाम् । \newline
29. अया॑वानाम् च॒ चाया॑वाना॒ मया॑वानाम् च । \newline
30. चाधि॑पत्य॒ माधि॑पत्यम् च॒ चाधि॑पत्यम् । \newline
31. आधि॑पत्य मासी दासी॒ दाधि॑पत्य॒ माधि॑पत्य मासीत् । \newline
32. आधि॑पत्य॒मित्याधि॑ - प॒त्य॒म् । \newline
33. आ॒सी॒त् त्रय॑स्त्रिꣳशता॒ त्रय॑स्त्रिꣳशता ऽऽसी दासी॒त् त्रय॑स्त्रिꣳशता । \newline
34. त्रय॑स्त्रिꣳशता ऽस्तुवता स्तुवत॒ त्रय॑स्त्रिꣳशता॒ त्रय॑स्त्रिꣳशता ऽस्तुवत । \newline
35. त्रय॑स्त्रिꣳश॒तेति॒ त्रयः॑ - त्रिꣳ॒॒श॒ता॒ । \newline
36. अ॒स्तु॒व॒त॒ भू॒तानि॑ भू॒तान्य॑ स्तुवता स्तुवत भू॒तानि॑ । \newline
37. भू॒ता न्य॑शाम्यन् नशाम्यन् भू॒तानि॑ भू॒ता न्य॑शाम्यन्न् । \newline
38. अ॒शा॒म्य॒न् प्र॒जाप॑तिः प्र॒जाप॑ तिरशाम्यन् नशाम्यन् प्र॒जाप॑तिः । \newline
39. प्र॒जाप॑तिः परमे॒ष्ठी प॑रमे॒ष्ठी प्र॒जाप॑तिः प्र॒जाप॑तिः परमे॒ष्ठी । \newline
40. प्र॒जाप॑ति॒रिति॑ प्र॒जा - प॒तिः॒ । \newline
41. प॒र॒मे॒ ष्ठ्यधि॑पति॒ रधि॑पतिः परमे॒ष्ठी प॑रमे॒ष्ठ्य धि॑पतिः । \newline
42. अधि॑पति रासी दासी॒ दधि॑पति॒ रधि॑पति रासीत् । \newline
43. अधि॑पति॒रित्यधि॑ - प॒तिः॒ । \newline
44. आ॒सी॒दित्या॑सीत् । \newline

\textbf{Ghana Paata } \newline

1. ऐ॒तां॒ ॅवस॑वो॒ वस॑व ऐता मैतां॒ ॅवस॑वो रु॒द्रा रु॒द्रा वस॑व ऐता मैतां॒ ॅवस॑वो रु॒द्राः । \newline
2. वस॑वो रु॒द्रा रु॒द्रा वस॑वो॒ वस॑वो रु॒द्रा आ॑दि॒त्या आ॑दि॒त्या रु॒द्रा वस॑वो॒ वस॑वो रु॒द्रा आ॑दि॒त्याः । \newline
3. रु॒द्रा आ॑दि॒त्या आ॑दि॒त्या रु॒द्रा रु॒द्रा आ॑दि॒त्या अन्वन् वा॑दि॒त्या रु॒द्रा रु॒द्रा आ॑दि॒त्या अनु॑ । \newline
4. आ॒दि॒त्या अन्वन् वा॑दि॒त्या आ॑दि॒त्या अनु॒ वि व्यन्वा॑ दि॒त्या आ॑दि॒त्या अनु॒ वि । \newline
5. अनु॒ वि व्यन् वन् व्या॑यन् नाय॒न् व्यन् वनु॒ व्या॑यन्न् । \newline
6. व्या॑यन् नाय॒न्॒. वि व्या॑य॒न् तेषा॒म् तेषा॑ माय॒न्॒. वि व्या॑य॒न् तेषा᳚म् । \newline
7. आ॒य॒न् तेषा॒म् तेषा॑ मायन् नाय॒न् तेषा॒ माधि॑पत्य॒ माधि॑पत्य॒म् तेषा॑ मायन् नाय॒न् तेषा॒ माधि॑पत्यम् । \newline
8. तेषा॒ माधि॑पत्य॒ माधि॑पत्य॒म् तेषा॒म् तेषा॒ माधि॑पत्य मासी दासी॒ दाधि॑पत्य॒म् तेषा॒म् तेषा॒ माधि॑पत्य मासीत् । \newline
9. आधि॑पत्य मासी दासी॒ दाधि॑पत्य॒ माधि॑पत्य मासी॒न् नव॑विꣳशत्या॒ नव॑विꣳशत्या ऽऽसी॒ दाधि॑पत्य॒ माधि॑पत्य मासी॒न् नव॑विꣳशत्या । \newline
10. आधि॑पत्य॒मित्याधि॑ - प॒त्य॒म् । \newline
11. आ॒सी॒न् नव॑विꣳशत्या॒ नव॑विꣳशत्या ऽऽसी दासी॒न् नव॑विꣳशत्या ऽस्तुवता स्तुवत॒ नव॑विꣳशत्या ऽऽसीदासी॒न् नव॑विꣳशत्या ऽस्तुवत । \newline
12. नव॑विꣳशत्या ऽस्तुवता स्तुवत॒ नव॑विꣳशत्या॒ नव॑विꣳशत्या ऽस्तुवत॒ वन॒स्पत॑यो॒ 
वन॒स्पत॑यो ऽस्तुवत॒ नव॑विꣳशत्या॒ नव॑विꣳशत्या ऽस्तुवत॒ वन॒स्पत॑यः । \newline
13. नव॑विꣳश॒त्येति॒ नव॑ - विꣳ॒॒श॒त्या॒ । \newline
14. अ॒स्तु॒व॒त॒ वन॒स्पत॑यो॒ वन॒स्पत॑यो ऽस्तुवता स्तुवत॒ वन॒स्पत॑यो ऽसृज्यन्ता सृज्यन्त॒ वन॒स्पत॑यो ऽस्तुवता स्तुवत॒ वन॒स्पत॑यो ऽसृज्यन्त । \newline
15. वन॒स्पत॑यो ऽसृज्यन्ता सृज्यन्त॒ वन॒स्पत॑यो॒ वन॒स्पत॑यो ऽसृज्यन्त॒ सोमः॒ सोमो॑ ऽसृज्यन्त॒ वन॒स्पत॑यो॒ वन॒स्पत॑यो ऽसृज्यन्त॒ सोमः॑ । \newline
16. अ॒सृ॒ज्य॒न्त॒ सोमः॒ सोमो॑ ऽसृज्यन्ता सृज्यन्त॒ सोमो ऽधि॑पति॒ रधि॑पतिः॒ सोमो॑ ऽसृज्यन्ता सृज्यन्त॒ सोमो ऽधि॑पतिः । \newline
17. सोमो ऽधि॑पति॒ रधि॑पतिः॒ सोमः॒ सोमो ऽधि॑पति रासी दासी॒ दधि॑पतिः॒ सोमः॒ सोमो ऽधि॑पति रासीत् । \newline
18. अधि॑पति रासी दासी॒ दधि॑पति॒ रधि॑पति रासी॒ देक॑त्रिꣳश॒ तैक॑त्रिꣳशता ऽऽसी॒ दधि॑पति॒ रधि॑पति रासी॒ देक॑त्रिꣳशता । \newline
19. अधि॑पति॒रित्यधि॑ - प॒तिः॒ । \newline
20. आ॒सी॒ देक॑त्रिꣳश॒ तैक॑त्रिꣳशता ऽऽसी दासी॒ देक॑त्रिꣳशता ऽस्तुवता स्तुव॒ तैक॑त्रिꣳशता ऽऽसी दासी॒ देक॑त्रिꣳशता ऽस्तुवत । \newline
21. एक॑त्रिꣳशता ऽस्तुवता स्तुव॒ तैक॑त्रिꣳश॒ तैक॑त्रिꣳशता ऽस्तुवत प्र॒जाः प्र॒जा अ॑स्तुव॒ तैक॑त्रिꣳश॒
तैक॑त्रिꣳशता ?स्तुवत प्र॒जाः । \newline
22. एक॑त्रिꣳश॒तेत्येक॑ - त्रिꣳ॒॒श॒ता॒ । \newline
23. अ॒स्तु॒व॒त॒ प्र॒जाः प्र॒जा अ॑स्तुवता स्तुवत प्र॒जा अ॑सृज्यन्ता सृज्यन्त प्र॒जा अ॑स्तुवता स्तुवत प्र॒जा अ॑सृज्यन्त । \newline
24. प्र॒जा अ॑सृज्यन्ता सृज्यन्त प्र॒जाः प्र॒जा अ॑सृज्यन्त॒ यावा॑नां॒ ॅयावा॑ना मसृज्यन्त प्र॒जाः प्र॒जा अ॑सृज्यन्त॒ यावा॑नाम् । \newline
25. प्र॒जा इति॑ प्र - जाः । \newline
26. अ॒सृ॒ज्य॒न्त॒ यावा॑नां॒ ॅयावा॑ना मसृज्यन्ता सृज्यन्त॒ यावा॑नाम् च च॒ यावा॑ना मसृज्यन्ता सृज्यन्त॒ यावा॑नाम् च । \newline
27. यावा॑नाम् च च॒ यावा॑नां॒ ॅयावा॑ना॒म् चा या॑वाना॒ मया॑वानाम् च॒ यावा॑नां॒ ॅयावा॑ना॒म् चा या॑वानाम् । \newline
28. चा या॑वाना॒ मया॑वानाम् च॒ चा या॑वानाम् च॒ चा या॑वानाम् च॒ चा या॑वानाम् च । \newline
29. अया॑वानाम् च॒ चा या॑वाना॒ मया॑वाना॒म् चाधि॑पत्य॒ माधि॑पत्य॒म् चा या॑वाना॒ मया॑वाना॒म् चाधि॑पत्यम् । \newline
30. चाधि॑पत्य॒ माधि॑पत्यम् च॒ चाधि॑पत्य मासी दासी॒ दाधि॑पत्यम् च॒ चाधि॑पत्य मासीत् । \newline
31. आधि॑पत्य मासी दासी॒ दाधि॑पत्य॒ माधि॑पत्य मासी॒त् त्रय॑स्त्रिꣳशता॒ त्रय॑स्त्रिꣳशता ऽऽसी॒ दाधि॑पत्य॒ माधि॑पत्य मासी॒त् त्रय॑स्त्रिꣳशता । \newline
32. आधि॑पत्य॒मित्याधि॑ - प॒त्य॒म् । \newline
33. आ॒सी॒त् त्रय॑स्त्रिꣳशता॒ त्रय॑स्त्रिꣳशता ऽऽसी दासी॒त् त्रय॑स्त्रिꣳशता ऽस्तुवता स्तुवत॒ त्रय॑स्त्रिꣳशता ऽऽसीदासी॒त् त्रय॑स्त्रिꣳशता ऽस्तुवत । \newline
34. त्रय॑स्त्रिꣳशता ऽस्तुवता स्तुवत॒ त्रय॑स्त्रिꣳशता॒ त्रय॑स्त्रिꣳशता ऽस्तुवत भू॒तानि॑ भू॒ता न्य॑स्तुवत॒ त्रय॑स्त्रिꣳशता॒ त्रय॑स्त्रिꣳशता ऽस्तुवत भू॒तानि॑ । \newline
35. त्रय॑स्त्रिꣳश॒तेति॒ त्रयः॑ - त्रिꣳ॒॒श॒ता॒ । \newline
36. अ॒स्तु॒व॒त॒ भू॒तानि॑ भू॒ता न्य॑स्तुवता स्तुवत भू॒ता न्य॑शाम्यन् नशाम्यन् भू॒ता न्य॑स्तुवता स्तुवत 
भू॒ता न्य॑शाम्यन्न् । \newline
37. भू॒ता न्य॑शाम्यन् नशाम्यन् भू॒तानि॑ भू॒ता न्य॑शाम्यन् प्र॒जाप॑तिः प्र॒जाप॑ति रशाम्यन् भू॒तानि॑ 
भू॒ता न्य॑शाम्यन् प्र॒जाप॑तिः । \newline
38. अ॒शा॒म्य॒न् प्र॒जाप॑तिः प्र॒जाप॑ति रशाम्यन् नशाम्यन् प्र॒जाप॑तिः परमे॒ष्ठी प॑रमे॒ष्ठी प्र॒जाप॑ति रशाम्यन् नशाम्यन् प्र॒जाप॑तिः परमे॒ष्ठी । \newline
39. प्र॒जाप॑तिः परमे॒ष्ठी प॑रमे॒ष्ठी प्र॒जाप॑तिः प्र॒जाप॑तिः परमे॒ ष्ठ्यधि॑पति॒ रधि॑पतिः परमे॒ष्ठी प्र॒जाप॑तिः प्र॒जाप॑तिः परमे॒ ष्ठ्यधि॑पतिः । \newline
40. प्र॒जाप॑ति॒रिति॑ प्र॒जा - प॒तिः॒ । \newline
41. प॒र॒मे॒ ष्ठ्यधि॑पति॒ रधि॑पतिः परमे॒ष्ठी प॑रमे॒ ष्ठ्यधि॑पति रासी दासी॒ दधि॑पतिः परमे॒ष्ठी प॑रमे॒ ष्ठ्यधि॑पति रासीत् । \newline
42. अधि॑पति रासी दासी॒ दधि॑पति॒ रधि॑पति रासीत् । \newline
43. अधि॑पति॒रित्यधि॑ - प॒तिः॒ । \newline
44. आ॒सी॒दित्या॑सीत् । \newline
\pagebreak
\markright{ TS 4.3.11.1  \hfill https://www.vedavms.in \hfill}

\section{ TS 4.3.11.1 }

\textbf{TS 4.3.11.1 } \newline
\textbf{Samhita Paata} \newline

इ॒यमे॒व सा या प्र॑थ॒मा व्यौच्छ॑द॒न्तर॒स्यां च॑रति॒ प्रवि॑ष्टा । व॒धूर्ज॑जान नव॒गज्जनि॑त्री॒ त्रय॑ एनां महि॒मानः॑ सचन्ते ॥ छन्द॑स्वती उ॒षसा॒ पेपि॑शाने समा॒नं ॅयोनि॒मनु॑ स॒ञ्चर॑न्ती । सूर्य॑पत्नी॒ वि च॑रतः प्रजान॒ती के॒तुं कृ॑ण्वा॒ने अ॒जर॒ भूरि॑रेतसा ॥ ऋ॒तस्य॒ पन्था॒मनु॑ ति॒स्र आऽगु॒स्त्रयो॑ घ॒र्मासो॒ अनु॒ ज्योति॒षाऽऽगुः॑ । प्र॒जामेका॒ रक्ष॒त्यूर्ज॒मेका᳚ -[  ] \newline

\textbf{Pada Paata} \newline

इ॒यम् । ए॒व । सा । या । प्र॒थ॒मा । व्यौच्छ॒दिति॑ वि - औच्छ॑त् । अ॒न्तः । अ॒स्याम् । च॒र॒ति॒ । प्रवि॒ष्टेति॒ प्र - वि॒ष्टा॒ ॥ व॒धूः । ज॒जा॒न॒ । न॒व॒गदिति॑ नव - गत् । जनि॑त्री । त्रयः॑ । ए॒ना॒म् । म॒हि॒मानः॑ । स॒च॒न्ते॒ ॥ छन्द॑स्वती॒ इति॑ । उ॒षसा᳚ । पेपि॑शाने॒ इति॑ । स॒मा॒नम् । योनि᳚म् । अन्विति॑ । स॒ञ्चर॑न्ती॒ इति॑ सं - चर॑न्ती ॥ सूर्य॑पत्नी॒ इति॒ सूर्य॑ - प॒त्नी॒ । वीति॑ । च॒र॒तः॒ । प्र॒जा॒न॒ती इति॑ प्र-जा॒न॒ती । के॒तुम् । कृ॒ण्वा॒ने इति॑ । अ॒जरे॒ इति॑ । भूरि॑रे॒तसेति॒ भूरि॑ - रे॒त॒सा॒ ॥ ऋ॒तस्य॑ । पन्था᳚म् । अन्विति॑ । ति॒स्रः । एति॑ । अ॒गुः॒ । त्रयः॑ । घ॒र्मासः॑ । अन्विति॑ । ज्योति॑षा । एति॑ । अ॒गुः॒ ॥ प्र॒जामिति॑ प्र - जाम् । एका᳚ । रक्ष॑ति । ऊर्ज᳚म् । एका᳚ ।  \newline


\textbf{Krama Paata} \newline

इ॒यमे॒व । ए॒व सा । सा या । या प्र॑थ॒मा । प्र॒थ॒मा व्यौच्छ॑त् । व्यौच्छ॑द॒न्तः । व्यौच्छ॒दिति॑ वि - औच्छ॑त् । अ॒न्तर॒स्याम् । अ॒स्याम् च॑रति । च॒र॒ति॒ प्रवि॑ष्टा । प्रवि॒ष्टेति॒ प्र - वि॒ष्टा॒ ॥ व॒धूर् ज॑जान । ज॒जा॒न॒ न॒व॒गत् । न॒व॒गज् जनि॑त्री । न॒व॒गदिति॑ नव - गत् । जनि॑त्री॒ त्रयः॑ । त्रय॑ एनाम् । ए॒ना॒म् म॒हि॒मानः॑ । म॒हि॒मानः॑ सचन्ते । स॒च॒न्त॒ इति॑ सचन्ते ॥ छन्द॑स्वती उ॒षसा᳚ । छन्द॑स्वती॒ इति॒ छन्द॑स्वती । उ॒षसा॒ पेपि॑शाने । पेपि॑शाने समा॒नम् । पेपि॑शाने॒ इति॒ पेपि॑शाने । स॒मा॒नं ॅयोनि᳚म् । योनि॒मनु॑ । अनु॑ स॒ञ्चर॑न्ती । स॒ञ्चर॑न्ती॒ इति॑ सम् - चर॑न्ती ॥ सूर्य॑पत्नी॒ वि । सूर्य॑पत्नी॒ इति॒ सूर्य॑ - प॒त्नी॒ । वि च॑रतः । च॒र॒तः॒ प्र॒जा॒न॒ती । प्र॒जा॒न॒ती के॒तुम् । प्र॒जा॒न॒ती इति॑ प्र - जा॒न॒ती । के॒तुम् कृ॑ण्वा॒ने । कृ॒ण्वा॒ने अ॒जरे᳚ । कृ॒ण्वा॒ने इति॑ कृण्वा॒ने । अ॒जरे॒ भूरि॑रेतसा । अ॒जरे॒ इत्य॒जरे᳚ । भूरि॑रेत॒सेति॒ भूरि॑ - रे॒त॒सा॒ ॥ ऋ॒तस्य॒ पन्था᳚म् । पन्था॒ मनु॑ । अनु॑ ति॒स्रः । ति॒स्र आ । आऽगुः॑ । अ॒गु॒स्त्रयः॑ । त्रयो॑ घ॒र्मासः॑ । घ॒र्मासो॒ अनु॑ । अनु॒ ज्योति॑षा । ज्योति॒षा । आऽगुः॑ । अ॒गु॒रित्य॑गुः ॥ प्र॒जामेका᳚ । प्र॒जामिति॑ प्र - जाम् । एका॒ रक्ष॑ति । रक्ष॒त्यूर्ज᳚म् । ऊर्ज॒मेका᳚ । एका᳚ व्र॒तम् \newline

\textbf{Jatai Paata} \newline

1. इ॒य मे॒वै वेय मि॒य मे॒व । \newline
2. ए॒व सा सैवैव सा । \newline
3. सा या या सा सा या । \newline
4. या प्र॑थ॒मा प्र॑थ॒मा या या प्र॑थ॒मा । \newline
5. प्र॒थ॒मा व्यौच्छ॒द् व्यौच्छ॑त् प्रथ॒मा प्र॑थ॒मा व्यौच्छ॑त् । \newline
6. व्यौच्छ॑ द॒न्त र॒न्तर् व्यौच्छ॒द् व्यौच्छ॑ द॒न्तः । \newline
7. व्यौच्छ॒दिति॑ वि - औच्छ॑त् । \newline
8. अ॒न्त र॒स्या म॒स्या म॒न्त र॒न्त र॒स्याम् । \newline
9. अ॒स्याम् च॑रति चरत्य॒ स्या म॒स्याम् च॑रति । \newline
10. च॒र॒ति॒ प्रवि॑ष्टा॒ प्रवि॑ष्टा चरति चरति॒ प्रवि॑ष्टा । \newline
11. प्रवि॒ष्टेति॒ प्र - वि॒ष्टा॒ । \newline
12. व॒धूर् ज॑जान जजान व॒धूर् व॒धूर् ज॑जान । \newline
13. ज॒जा॒न॒ न॒व॒गन् न॑व॒गज् ज॑जान जजान नव॒गत् । \newline
14. न॒व॒गज् जनि॑त्री॒ जनि॑त्री नव॒गन् न॑व॒गज् जनि॑त्री । \newline
15. न॒व॒गदिति॑ नव - गत् । \newline
16. जनि॑त्री॒ त्रय॒ स्त्रयो॒ जनि॑त्री॒ जनि॑त्री॒ त्रयः॑ । \newline
17. त्रय॑ एना मेना॒म् त्रय॒ स्त्रय॑ एनाम् । \newline
18. ए॒ना॒म् म॒हि॒मानो॑ महि॒मान॑ एना मेनाम् महि॒मानः॑ । \newline
19. म॒हि॒मानः॑ सचन्ते सचन्ते महि॒मानो॑ महि॒मानः॑ सचन्ते । \newline
20. स॒च॒न्त॒ इति॑ सचन्ते । \newline
21. छन्द॑स्वती उ॒षसो॒ षसा॒ छन्द॑स्वती॒ छन्द॑स्वती उ॒षसा᳚ । \newline
22. छन्द॑स्वती॒ इति॒ छन्द॑स्वती । \newline
23. उ॒षसा॒ पेपि॑शाने॒ पेपि॑शाने उ॒षसो॒ षसा॒ पेपि॑शाने । \newline
24. पेपि॑शाने समा॒नꣳ स॑मा॒नम् पेपि॑शाने॒ पेपि॑शाने समा॒नम् । \newline
25. पेपि॑शाने॒ इति॒ पेपि॑शाने । \newline
26. स॒मा॒नं ॅयोनिं॒ ॅयोनिꣳ॑ समा॒नꣳ स॑मा॒नं ॅयोनि᳚म् । \newline
27. योनि॒ मन्वनु॒ योनिं॒ ॅयोनि॒ मनु॑ । \newline
28. अनु॑ स॒ञ्चर॑न्ती स॒ञ्चर॑न्ती॒ अन्वनु॑ स॒ञ्चर॑न्ती । \newline
29. स॒ञ्चर॑न्ती॒ इति॑ सं - चर॑न्ती । \newline
30. सूर्य॑पत्नी॒ वि वि सूर्य॑पत्नी॒ सूर्य॑पत्नी॒ वि । \newline
31. सूर्य॑पत्नी॒ इति॒ सूर्य॑ - प॒त्नी॒ । \newline
32. वि च॑रत श्चरतो॒ वि वि च॑रतः । \newline
33. च॒र॒तः॒ प्र॒जा॒न॒ती प्र॑जान॒ती च॑रत श्चरतः प्रजान॒ती । \newline
34. प्र॒जा॒न॒ती के॒तुम् के॒तुम् प्र॑जान॒ती प्र॑जान॒ती के॒तुम् । \newline
35. प्र॒जा॒न॒ती इति॑ प्र - जा॒न॒ती । \newline
36. के॒तुम् कृ॑ण्वा॒ने कृ॑ण्वा॒ने के॒तुम् के॒तुम् कृ॑ण्वा॒ने । \newline
37. कृ॒ण्वा॒ने अ॒जरे॑ अ॒जरे॑ कृण्वा॒ने कृ॑ण्वा॒ने अ॒जरे᳚ । \newline
38. कृ॒ण्वा॒ने इति॑ कृण्वा॒ने । \newline
39. अ॒जरे॒ भूरि॑रेतसा॒ भूरि॑ रेतसा॒ ऽजरे॑ अ॒जरे॒ भूरि॑ रेतसा । \newline
40. अ॒जरे॒ इत्य॒जरे᳚ । \newline
41. भूरि॑रे॒तसेति॒ भूरि॑ - रे॒त॒सा॒ । \newline
42. ऋ॒तस्य॒ पन्था॒म् पन्था॑ मृ॒तस्य॒ र्‌तस्य॒ पन्था᳚म् । \newline
43. पन्था॒ मन्वनु॒ पन्था॒म् पन्था॒ मनु॑ । \newline
44. अनु॑ ति॒स्र स्ति॒स्रो अन्वनु॑ ति॒स्रः । \newline
45. ति॒स्र आ ति॒स्र स्ति॒स्र आ । \newline
46. आ ऽगु॑ रगु॒रा ऽगुः॑ । \newline
47. अ॒गु॒ स्त्रय॒ स्त्रयो॑ अगु रगु॒ स्त्रयः॑ । \newline
48. त्रयो॑ घ॒र्मासो॑ घ॒र्मास॒ स्त्रय॒ स्त्रयो॑ घ॒र्मासः॑ । \newline
49. घ॒र्मासो॒ अन्वनु॑ घ॒र्मासो॑ घ॒र्मासो॒ अनु॑ । \newline
50. अनु॒ ज्योति॑षा॒ ज्योति॑षा॒ ऽन्वनु॒ ज्योति॑षा । \newline
51. ज्योति॒षा ऽऽज्योति॑षा॒ ज्योति॒षा । \newline
52. आ ऽगु॑ रगु॒रा ऽगुः॑ । \newline
53. अ॒गु॒रित्य॑गुः । \newline
54. प्र॒जा मेकैका᳚ प्र॒जाम् प्र॒जा मेका᳚ । \newline
55. प्र॒जामिति॑ प्र - जाम् । \newline
56. एका॒ रक्ष॑ति॒ रक्ष॒ त्येकैका॒ रक्ष॑ति । \newline
57. रक्ष॒त्यूर्ज॒ मूर्जꣳ॒॒ रक्ष॑ति॒ रक्ष॒ त्यूर्ज᳚म् । \newline
58. ऊर्ज॒ मेकै कोर्ज॒ मूर्ज॒ मेका᳚ । \newline
59. एका᳚ व्र॒तं ॅव्र॒त मेकैका᳚ व्र॒तम् । \newline

\textbf{Ghana Paata } \newline

1. इ॒य मे॒वैवे य मि॒य मे॒व सा सैवे य मि॒य मे॒व सा । \newline
2. ए॒व सा सै वैव सा या या सै वैव सा या । \newline
3. सा या या सा सा या प्र॑थ॒मा प्र॑थ॒मा या सा सा या प्र॑थ॒मा । \newline
4. या प्र॑थ॒मा प्र॑थ॒मा या या प्र॑थ॒मा व्यौच्छ॒द् व्यौच्छ॑त् प्रथ॒मा या या प्र॑थ॒मा व्यौच्छ॑त् । \newline
5. प्र॒थ॒मा व्यौच्छ॒द् व्यौच्छ॑त् प्रथ॒मा प्र॑थ॒मा व्यौच्छ॑ द॒न्त र॒न्तर् व्यौच्छ॑त् प्रथ॒मा प्र॑थ॒मा व्यौच्छ॑ द॒न्तः । \newline
6. व्यौच्छ॑ द॒न्त र॒न्तर् व्यौच्छ॒द् व्यौच्छ॑ द॒न्त र॒स्या म॒स्या म॒न्तर् व्यौच्छ॒द् व्यौच्छ॑ द॒न्त र॒स्याम् । \newline
7. व्यौच्छ॒दिति॑ वि - औच्छ॑त् । \newline
8. अ॒न्त र॒स्या म॒स्या म॒न्त र॒न्त र॒स्याम् च॑रति चर त्य॒स्या म॒न्त र॒न्त र॒स्याम् च॑रति । \newline
9. अ॒स्याम् च॑रति चर त्य॒स्या म॒स्याम् च॑रति॒ प्रवि॑ष्टा॒ प्रवि॑ष्टा चर त्य॒स्या म॒स्याम् च॑रति॒ प्रवि॑ष्टा । \newline
10. च॒र॒ति॒ प्रवि॑ष्टा॒ प्रवि॑ष्टा चरति चरति॒ प्रवि॑ष्टा । \newline
11. प्रवि॒ष्टेति॒ प्र - वि॒ष्टा॒ । \newline
12. व॒धूर् ज॑जान जजान व॒धूर् व॒धूर् ज॑जान नव॒गन् न॑व॒गज् ज॑जान व॒धूर् व॒धूर् ज॑जान नव॒गत् । \newline
13. ज॒जा॒न॒ न॒व॒गन् न॑व॒गज् ज॑जान जजान नव॒गज् जनि॑त्री॒ जनि॑त्री नव॒गज् ज॑जान जजान नव॒गज् जनि॑त्री । \newline
14. न॒व॒गज् जनि॑त्री॒ जनि॑त्री नव॒गन् न॑व॒गज् जनि॑त्री॒ त्रय॒ स्त्रयो॒ जनि॑त्री नव॒गन् न॑व॒गज् जनि॑त्री॒ त्रयः॑ । \newline
15. न॒व॒गदिति॑ नव - गत् । \newline
16. जनि॑त्री॒ त्रय॒ स्त्रयो॒ जनि॑त्री॒ जनि॑त्री॒ त्रय॑ एना मेना॒म् त्रयो॒ जनि॑त्री॒ जनि॑त्री॒ त्रय॑ एनाम् । \newline
17. त्रय॑ एना मेना॒म् त्रय॒ स्त्रय॑ एनाम् महि॒मानो॑ महि॒मान॑ एना॒म् त्रय॒ स्त्रय॑ एनाम् महि॒मानः॑ । \newline
18. ए॒ना॒म् म॒हि॒मानो॑ महि॒मान॑ एना मेनाम् महि॒मानः॑ सचन्ते सचन्ते महि॒मान॑ एना मेनाम् महि॒मानः॑ सचन्ते । \newline
19. म॒हि॒मानः॑ सचन्ते सचन्ते महि॒मानो॑ महि॒मानः॑ सचन्ते । \newline
20. स॒च॒न्त॒ इति॑ सचन्ते । \newline
21. छन्द॑स्वती उ॒षसो ॒षसा॒ छन्द॑स्वती॒ छन्द॑स्वती उ॒षसा॒ पेपि॑शाने॒ पेपि॑शाने उ॒षसा॒ छन्द॑स्वती॒ छन्द॑स्वती उ॒षसा॒ पेपि॑शाने । \newline
22. छन्द॑स्वती॒ इति॒ छन्द॑स्वती । \newline
23. उ॒षसा॒ पेपि॑शाने॒ पेपि॑शाने उ॒षसो॒ षसा॒ पेपि॑शाने समा॒नꣳ स॑मा॒नम् पेपि॑शाने उ॒षसो॒षसा॒ पेपि॑शाने समा॒नम् । \newline
24. पेपि॑शाने समा॒नꣳ स॑मा॒नम् पेपि॑शाने॒ पेपि॑शाने समा॒नं ॅयोनिं॒ ॅयोनिꣳ॑ समा॒नम् पेपि॑शाने॒ पेपि॑शाने समा॒नं ॅयोनि᳚म् । \newline
25. पेपि॑शाने॒ इति॒ पेपि॑शाने । \newline
26. स॒मा॒नं ॅयोनिं॒ ॅयोनिꣳ॑ समा॒नꣳ स॑मा॒नं ॅयोनि॒ मन्वनु॒ योनिꣳ॑ समा॒नꣳ स॑मा॒नं ॅयोनि॒ मनु॑ । \newline
27. योनि॒ मन्वनु॒ योनिं॒ ॅयोनि॒ मनु॑ स॒ञ्चर॑न्ती स॒ञ्चर॑न्ती॒ अनु॒ योनिं॒ ॅयोनि॒ मनु॑ स॒ञ्चर॑न्ती । \newline
28. अनु॑ स॒ञ्चर॑न्ती स॒ञ्चर॑न्ती॒ अन्वनु॑ स॒ञ्चर॑न्ती । \newline
29. स॒ञ्चर॑न्ती॒ इति॑ सं - चर॑न्ती । \newline
30. सूर्य॑पत्नी॒ वि वि सूर्य॑पत्नी॒ सूर्य॑पत्नी॒ वि च॑रत श्चरतो॒ वि सूर्य॑पत्नी॒ सूर्य॑पत्नी॒ वि च॑रतः । \newline
31. सूर्य॑पत्नी॒ इति॒ सूर्य॑ - प॒त्नी॒ । \newline
32. वि च॑रत श्चरतो॒ वि वि च॑रतः प्रजान॒ती प्र॑जान॒ती च॑रतो॒ वि वि च॑रतः प्रजान॒ती । \newline
33. च॒र॒तः॒ प्र॒जा॒न॒ती प्र॑जान॒ती च॑रत श्चरतः प्रजान॒ती के॒तुम् के॒तुम् प्र॑जान॒ती च॑रत श्चरतः प्रजान॒ती के॒तुम् । \newline
34. प्र॒जा॒न॒ती के॒तुम् के॒तुम् प्र॑जान॒ती प्र॑जान॒ती के॒तुम् कृ॑ण्वा॒ने कृ॑ण्वा॒ने के॒तुम् प्र॑जान॒ती प्र॑जान॒ती के॒तुम् कृ॑ण्वा॒ने । \newline
35. प्र॒जा॒न॒ती इति॑ प्र - जा॒न॒ती । \newline
36. के॒तुम् कृ॑ण्वा॒ने कृ॑ण्वा॒ने के॒तुम् के॒तुम् कृ॑ण्वा॒ने अ॒जरे॑ अ॒जरे॑ कृण्वा॒ने के॒तुम् के॒तुम् कृ॑ण्वा॒ने अ॒जरे᳚ । \newline
37. कृ॒ण्वा॒ने अ॒जरे॑ अ॒जरे॑ कृण्वा॒ने कृ॑ण्वा॒ने अ॒जरे॒ भूरि॑रेतसा॒ भूरि॑रेतसा॒ ऽजरे॑ कृण्वा॒ने कृ॑ण्वा॒ने अ॒जरे॒ भूरि॑रेतसा । \newline
38. कृ॒ण्वा॒ने इति॑ कृण्वा॒ने । \newline
39. अ॒जरे॒ भूरि॑रेतसा॒ भूरि॑रेतसा॒ ऽजरे॑ अ॒जरे॒ भूरि॑रेतसा । \newline
40. अ॒जरे॒ इत्य॒जरे᳚ । \newline
41. भूरि॑रे॒तसेति॒ भूरि॑ - रे॒त॒सा॒ । \newline
42. ऋ॒तस्य॒ पन्था॒म् पन्था॑ मृ॒तस्य॒ र्‌तस्य॒ पन्था॒ मन्वनु॒ पन्था॑ मृ॒तस्य॒ र्‌तस्य॒ पन्था॒ मनु॑ । \newline
43. पन्था॒ मन्वनु॒ पन्था॒म् पन्था॒ मनु॑ ति॒स्र स्ति॒स्रो ऽनु॒ पन्था॒म् पन्था॒ मनु॑ ति॒स्रः । \newline
44. अनु॑ ति॒स्र स्ति॒स्रो अन्वनु॑ ति॒स्र आ ति॒स्रो अन्वनु॑ ति॒स्र आ । \newline
45. ति॒स्र आ ति॒स्र स्ति॒स्र आ ऽगु॑ रगु॒रा ति॒स्र स्ति॒स्र आ ऽगुः॑ । \newline
46. आ ऽगु॑ रगु॒रा ऽगु॒ स्त्रय॒ स्त्रयो॑ अगु॒रा ऽगु॒ स्त्रयः॑ । \newline
47. अ॒गु॒ स्त्रय॒ स्त्रयो॑ अगु रगु॒ स्त्रयो॑ घ॒र्मासो॑ घ॒र्मास॒ स्त्रयो॑ अगु रगु॒ स्त्रयो॑ घ॒र्मासः॑ । \newline
48. त्रयो॑ घ॒र्मासो॑ घ॒र्मास॒ स्त्रय॒ स्त्रयो॑ घ॒र्मासो॒ अन्वनु॑ घ॒र्मास॒ स्त्रय॒ स्त्रयो॑ घ॒र्मासो॒ अनु॑ । \newline
49. घ॒र्मासो॒ अन्वनु॑ घ॒र्मासो॑ घ॒र्मासो॒ अनु॒ ज्योति॑षा॒ ज्योति॑षा ऽनु घ॒र्मासो॑ घ॒र्मासो॒ अनु॒ ज्योति॑षा । \newline
50. अनु॒ ज्योति॑षा॒ ज्योति॑षा॒ ऽन्वनु॒ ज्योति॒षा ऽऽज्योति॑षा॒ ऽन्वनु॒ ज्योति॒षा । \newline
51. ज्योति॒षा ऽऽज्योति॑षा॒ ज्योति॒षा ऽगु॑ रगु॒रा ज्योति॑षा॒ ज्योति॒षा ऽगुः॑ । \newline
52. आ ऽगु॑ रगु॒रा ऽगुः॑ । \newline
53. अ॒गु॒रित्य॑गुः । \newline
54. प्र॒जा मेकैका᳚ प्र॒जाम् प्र॒जा मेका॒ रक्ष॑ति॒ रक्ष॒ त्येका᳚ प्र॒जाम् प्र॒जा मेका॒ रक्ष॑ति । \newline
55. प्र॒जामिति॑ प्र - जाम् । \newline
56. एका॒ रक्ष॑ति॒ रक्ष॒ त्येकैका॒ रक्ष॒ त्यूर्ज॒ मूर्जꣳ॒॒ रक्ष॒ त्येकैका॒ रक्ष॒ त्यूर्ज᳚म् । \newline
57. रक्ष॒ त्यूर्ज॒ मूर्जꣳ॒॒ रक्ष॑ति॒ रक्ष॒ त्यूर्ज॒ मेकै कोर्जꣳ॒॒ रक्ष॑ति॒ रक्ष॒ त्यूर्ज॒ मेका᳚ । \newline
58. ऊर्ज॒ मेकै कोर्ज॒ मूर्ज॒ मेका᳚ व्र॒तं ॅव्र॒त मेकोर्ज॒ मूर्ज॒ मेका᳚ व्र॒तम् । \newline
59. एका᳚ व्र॒तं ॅव्र॒त मेकैका᳚ व्र॒त मेकैका᳚ व्र॒त मेकैका᳚ व्र॒त मेका᳚ । \newline
\pagebreak
\markright{ TS 4.3.11.2  \hfill https://www.vedavms.in \hfill}

\section{ TS 4.3.11.2 }

\textbf{TS 4.3.11.2 } \newline
\textbf{Samhita Paata} \newline

व्र॒तमेका॑ रक्षति देवयू॒नां ॥ च॒तु॒ष्टो॒मो अ॑भव॒द्या तु॒रीया॑ य॒ज्ञ्स्य॑ प॒क्षावृ॑षयो॒ भव॑न्ती । गा॒य॒त्रीं त्रि॒ष्टुभं॒ ज॑गतीमनु॒ष्टुभं॑ बृ॒हद॒र्कं ॅयु॑ञ्जा॒नाः सुव॒राऽभ॑रन्नि॒दं ॥ प॒ञ्चभि॑र्द्धा॒ता वि द॑धावि॒दं ॅयत् तासाꣳ॒॒ स्वसॄ॑रजनय॒त् पञ्च॑पञ्च । तासा॑मु यन्ति प्रय॒वेण॒ पञ्च॒ नाना॑ रू॒पाणि॒ क्रत॑वो॒ वसा॑नाः ॥ त्रिꣳ॒॒शथ् स्वसा॑र॒ उप॑यन्ति निष्कृ॒तꣳ स॑मा॒नं के॒तुं प्र॑तिमु॒ञ्चमा॑नाः । \newline

\textbf{Pada Paata} \newline

व्र॒तम् । एका᳚ । र॒क्ष॒ति॒ । दे॒व॒यू॒नामिति॑ देव - यू॒नाम् ॥ च॒तु॒ष्टो॒म इति॑ चतुः-स्तो॒मः । अ॒भ॒व॒त् । या । तु॒रीया᳚ । य॒ज्ञ्स्य॑ । प॒क्षौ । ऋ॒ष॒यः॒ । भव॑न्ती ॥ गा॒य॒त्रीम् । त्रि॒ष्टुभ᳚म् । जग॑तीम् । अ॒नु॒ष्टुभ॒मित्य॑नु-स्तुभ᳚म् । बृ॒हत् । अ॒र्कम् । यु॒ञ्जा॒नाः । सुवः॑ । एति॑ । अ॒भ॒र॒न्न् । इ॒दम् ॥ प॒ञ्चभि॒रिति॑ प॒ञ्च - भिः॒ । धा॒ता । वीति॑ । द॒धौ॒ । इ॒दम् । यत् । तासा᳚म् । स्वसॄः᳚ । अ॒ज॒न॒य॒त् । पञ्च॑प॒ञ्चेति॒ पञ्च॑ - प॒ञ्च॒ ॥ तासा᳚म् । उ॒ । य॒न्ति॒ । प्र॒य॒वेणेति॑ प्र - य॒वेन॑ । पञ्च॑ । नाना᳚ । रू॒पाणि॑ । क्रत॑वः । वसा॑नाः ॥ त्रिꣳ॒॒शत् । स्वसा॑रः । उपेति॑ । य॒न्ति॒ । नि॒ष्कृ॒तमिति॑ निः - कृ॒तम् । स॒मा॒नम् । के॒तुम् । प्र॒ति॒मु॒ञ्चमा॑ना॒ इति॑ प्रति - मु॒ञ्चमा॑नाः ॥  \newline


\textbf{Krama Paata} \newline

व्र॒तमेका᳚ । एका॑ रक्षति । र॒क्ष॒ति॒ दे॒व॒यू॒नाम् । दे॒व॒यू॒नामिति॑ देव - यू॒नाम् ॥ च॒तु॒ष्टो॒मो अ॑भवत् । च॒तु॒ष्टो॒म इति॑ चतुः - स्तो॒मः । अ॒भ॒व॒द् या । या तु॒रीया᳚ । तु॒रीया॑ य॒ज्ञ्स्य॑ । य॒ज्ञ्स्य॑ प॒क्षौ । प॒क्षावृ॑षयः । ऋ॒ष॒यो॒ भव॑न्ती । भव॒न्तीति॒ भव॑न्ती ॥ गा॒य॒त्रीम् त्रि॒ष्टुभ᳚म् । त्रि॒ष्टुभ॒म् जग॑तीम् । जग॑तीमनु॒ष्टुभ᳚म् । अ॒नु॒ष्टुभ॑म् बृ॒हत् । अ॒नु॒ष्टुभ॒मित्य॑नु - स्तुभ᳚म् । बृ॒हद॒र्कम् । अ॒र्कं ॅयु॑ञ्जा॒नाः । यु॒ञ्जा॒नाः सुवः॑ । सुव॒रा । आऽभ॑रन्न् । अ॒भ॒र॒न्नि॒दम् । इ॒दमिती॒दम् ॥ प॒ञ्चभि॑र् धा॒ता । प॒ञ्चभि॒रिति॑ प॒ञ्च - भिः॒ । धा॒ता वि । वि द॑धौ । द॒धा॒वि॒दम् । इ॒दं ॅयत् । यत् तासा᳚म् । तासाꣳ॒॒ स्वसॄः᳚ । स्वसॄ॑रजनयत् । अ॒ज॒न॒य॒त् पञ्च॑पञ्च । पञ्च॑प॒ञ्चेति॒ पञ्च॑ - प॒ञ्च॒ ॥ तासा॑मु । उ॒ य॒न्ति॒ । य॒न्ति॒ प्र॒य॒वेण॑ । प्र॒य॒वेण॒ पञ्च॑ । प्र॒य॒वेणेति॑ प्र - य॒वेन॑ । पञ्च॒ नाना᳚ । नाना॑ रू॒पाणि॑ । रू॒पाणि॒ क्रत॑वः । क्रत॑वो॒ वसा॑नाः । वसा॑ना॒ इति॒ वसा॑नाः ॥ त्रिꣳ॒॒शथ् स्वसा॑रः । स्वसा॑र॒ उप॑ । उप॑ यन्ति । य॒न्ति॒ नि॒ष्कृ॒तम् । नि॒ष्कृ॒तꣳ स॑मा॒नम् । नि॒ष्कृ॒तमिति॑ निः - कृ॒तम् । स॒मा॒नम् के॒तुम् । के॒तुम् प्र॑तिमु॒ञ्चमा॑नाः । प्र॒ति॒मु॒ञ्चमा॑ना॒ इति॑ प्रति - मु॒ञ्चमा॑नाः । \newline

\textbf{Jatai Paata} \newline

1. व्र॒त मेकैका᳚ व्र॒तं ॅव्र॒त मेका᳚ । \newline
2. एका॑ रक्षति रक्ष॒ त्येकैका॑ रक्षति । \newline
3. र॒क्ष॒ति॒ दे॒व॒यू॒नाम् दे॑वयू॒नाꣳ र॑क्षति रक्षति देवयू॒नाम् । \newline
4. दे॒व॒यू॒नामिति॑ देव - यू॒नाम् । \newline
5. च॒तु॒ष्टो॒मो अ॑भव दभवच् चतुष्टो॒म श्च॑तुष्टो॒मो अ॑भवत् । \newline
6. च॒तु॒ष्टो॒म इति॑ चतुः - स्तो॒मः । \newline
7. अ॒भ॒व॒द् या या ऽभ॑व दभव॒द् या । \newline
8. या तु॒रीया॑ तु॒रीया॒ या या तु॒रीया᳚ । \newline
9. तु॒रीया॑ य॒ज्ञ्स्य॑ य॒ज्ञ्स्य॑ तु॒रीया॑ तु॒रीया॑ य॒ज्ञ्स्य॑ । \newline
10. य॒ज्ञ्स्य॑ प॒क्षौ प॒क्षौ य॒ज्ञ्स्य॑ य॒ज्ञ्स्य॑ प॒क्षौ । \newline
11. प॒क्षा वृ॑षय ऋषयः प॒क्षौ प॒क्षा वृ॑षयः । \newline
12. ऋ॒ष॒यो॒ भव॑न्ती॒ भव॑न् त्यृषय ऋषयो॒ भव॑न्ती । \newline
13. भव॒न्तीति॒ भव॑न्ती । \newline
14. गा॒य॒त्रीम् त्रि॒ष्टुभ॑म् त्रि॒ष्टुभ॑म् गाय॒त्रीम् गा॑य॒त्रीम् त्रि॒ष्टुभ᳚म् । \newline
15. त्रि॒ष्टुभ॒म् जग॑ती॒म् जग॑तीम् त्रि॒ष्टुभ॑म् त्रि॒ष्टुभ॒म् जग॑तीम् । \newline
16. जग॑ती मनु॒ष्टुभ॑ मनु॒ष्टुभ॒म् जग॑ती॒म् जग॑ती मनु॒ष्टुभ᳚म् । \newline
17. अ॒नु॒ष्टुभ॑म् बृ॒हद् बृ॒ह द॑नु॒ष्टुभ॑ मनु॒ष्टुभ॑म् बृ॒हत् । \newline
18. अ॒नु॒ष्टुभ॒मित्य॑नु - स्तुभ᳚म् । \newline
19. बृ॒ह द॒र्क म॒र्कम् बृ॒हद् बृ॒ह द॒र्कम् । \newline
20. अ॒र्कं ॅयु॑ञ्जा॒ना यु॑ञ्जा॒ना अ॒र्क म॒र्कं ॅयु॑ञ्जा॒नाः । \newline
21. यु॒ञ्जा॒नाः सुवः॒ सुव॑र् युञ्जा॒ना यु॑ञ्जा॒नाः सुवः॑ । \newline
22. सुव॒रा सुवः॒ सुव॒रा । \newline
23. आ ऽभ॑रन् नभर॒न् ना ऽभ॑रन्न् । \newline
24. अ॒भ॒र॒न् नि॒द मि॒द म॑भरन् नभरन् नि॒दम् । \newline
25. इ॒दमिती॒दम् । \newline
26. प॒ञ्चभि॑र् धा॒ता धा॒ता प॒ञ्चभिः॑ प॒ञ्चभि॑र् धा॒ता । \newline
27. प॒ञ्चभि॒रिति॑ प॒ञ्च - भिः॒ । \newline
28. धा॒ता वि वि धा॒ता धा॒ता वि । \newline
29. वि द॑धौ दधौ॒ वि वि द॑धौ । \newline
30. द॒धा॒ वि॒द मि॒दम् द॑धौ दधा वि॒दम् । \newline
31. इ॒दं ॅयद् यदि॒द मि॒दं ॅयत् । \newline
32. यत् तासा॒म् तासां॒ ॅयद् यत् तासा᳚म् । \newline
33. तासाꣳ॒॒ स्वसॄः॒ स्वसॄ॒ स्तासा॒म् तासाꣳ॒॒ स्वसॄः᳚ । \newline
34. स्वसॄ॑ रजनय दजनय॒थ् स्वसॄः॒ स्वसॄ॑ रजनयत् । \newline
35. अ॒ज॒न॒य॒त् पञ्च॑पञ्च॒ पञ्च॑पञ्चा जनय दजनय॒त् पञ्च॑पञ्च । \newline
36. पञ्च॑प॒ञ्चेति॒ पञ्च॑ - प॒ञ्च॒ । \newline
37. तासा॑ मु वु॒ तासा॒म् तासा॑ मु । \newline
38. उ॒ य॒न्ति॒ य॒न्ति॒ उ॒ वु॒ य॒न्ति॒ । \newline
39. य॒न्ति॒ प्र॒य॒वेण॑ प्रय॒वेण॑ यन्ति यन्ति प्रय॒वेण॑ । \newline
40. प्र॒य॒वेण॒ पञ्च॒ पञ्च॑ प्रय॒वेण॑ प्रय॒वेण॒ पञ्च॑ । \newline
41. प्र॒य॒वेणेति॑ प्र - य॒वेन॑ । \newline
42. पञ्च॒ नाना॒ नाना॒ पञ्च॒ पञ्च॒ नाना᳚ । \newline
43. नाना॑ रू॒पाणि॑ रू॒पाणि॒ नाना॒ नाना॑ रू॒पाणि॑ । \newline
44. रू॒पाणि॒ क्रत॑वः॒ क्रत॑वो रू॒पाणि॑ रू॒पाणि॒ क्रत॑वः । \newline
45. क्रत॑वो॒ वसा॑ना॒ वसा॑नाः॒ क्रत॑वः॒ क्रत॑वो॒ वसा॑नाः । \newline
46. वसा॑ना॒ इति॒ वसा॑नाः । \newline
47. त्रिꣳ॒॒शथ् स्वसा॑रः॒ स्वसा॑र स्त्रिꣳ॒॒शत् त्रिꣳ॒॒शथ् स्वसा॑रः । \newline
48. स्वसा॑र॒ उपोप॒ स्वसा॑रः॒ स्वसा॑र॒ उप॑ । \newline
49. उप॑ यन्ति य॒न् त्युपोप॑ यन्ति । \newline
50. य॒न्ति॒ नि॒ष्कृ॒तन् नि॑ष्कृ॒तं ॅय॑न्ति यन्ति निष्कृ॒तम् । \newline
51. नि॒ष्कृ॒तꣳ स॑मा॒नꣳ स॑मा॒नन् नि॑ष्कृ॒तन् नि॑ष्कृ॒तꣳ स॑मा॒नम् । \newline
52. नि॒ष्कृ॒तमिति॑ निः - कृ॒तम् । \newline
53. स॒मा॒नम् के॒तुम् के॒तुꣳ स॑मा॒नꣳ स॑मा॒नम् के॒तुम् । \newline
54. के॒तुम् प्र॑तिमु॒ञ्चमा॑नाः प्रतिमु॒ञ्चमा॑नाः के॒तुम् के॒तुम् प्र॑तिमु॒ञ्चमा॑नाः । \newline
55. प्र॒ति॒मु॒ञ्चमा॑ना॒ इति॑ प्रति - मु॒ञ्चमा॑नाः । \newline

\textbf{Ghana Paata } \newline

1. व्र॒त मेकैका᳚ व्र॒तं ॅव्र॒त मेका॑ रक्षति रक्ष॒ त्येका᳚ व्र॒तं ॅव्र॒त मेका॑ रक्षति । \newline
2. एका॑ रक्षति रक्ष॒ त्येकैका॑ रक्षति देवयू॒नाम् दे॑वयू॒नाꣳ र॑क्ष॒ त्येकैका॑ रक्षति देवयू॒नाम् । \newline
3. र॒क्ष॒ति॒ दे॒व॒यू॒नाम् दे॑वयू॒नाꣳ र॑क्षति रक्षति देवयू॒नाम् । \newline
4. दे॒व॒यू॒नामिति॑ देव - यू॒नाम् । \newline
5. च॒तु॒ष्टो॒मो अ॑भव दभवच् चतुष्टो॒म श्च॑तुष्टो॒मो अ॑भव॒द् या या ऽभ॑वच् चतुष्टो॒म श्च॑तुष्टो॒मो अ॑भव॒द् या । \newline
6. च॒तु॒ष्टो॒म इति॑ चतुः - स्तो॒मः । \newline
7. अ॒भ॒व॒द् या या ऽभ॑व दभव॒द् या तु॒रीया॑ तु॒रीया॒ या ऽभ॑व दभव॒द् या तु॒रीया᳚ । \newline
8. या तु॒रीया॑ तु॒रीया॒ या या तु॒रीया॑ य॒ज्ञ्स्य॑ य॒ज्ञ्स्य॑ तु॒रीया॒ या या तु॒रीया॑ य॒ज्ञ्स्य॑ । \newline
9. तु॒रीया॑ य॒ज्ञ्स्य॑ य॒ज्ञ्स्य॑ तु॒रीया॑ तु॒रीया॑ य॒ज्ञ्स्य॑ प॒क्षौ प॒क्षौ य॒ज्ञ्स्य॑ तु॒रीया॑ तु॒रीया॑ य॒ज्ञ्स्य॑ प॒क्षौ । \newline
10. य॒ज्ञ्स्य॑ प॒क्षौ प॒क्षौ य॒ज्ञ्स्य॑ य॒ज्ञ्स्य॑ प॒क्षा वृ॑षय ऋषयः प॒क्षौ य॒ज्ञ्स्य॑ य॒ज्ञ्स्य॑ प॒क्षा वृ॑षयः । \newline
11. प॒क्षा वृ॑षय ऋषयः प॒क्षौ प॒क्षा वृ॑षयो॒ भव॑न्ती॒ भव॑न् त्यृषयः प॒क्षौ प॒क्षा वृ॑षयो॒ भव॑न्ती । \newline
12. ऋ॒ष॒यो॒ भव॑न्ती॒ भव॑न् त्यृषय ऋषयो॒ भव॑न्ती । \newline
13. भव॒न्तीति॒ भव॑न्ती । \newline
14. गा॒य॒त्रीम् त्रि॒ष्टुभ॑म् त्रि॒ष्टुभ॑म् गाय॒त्रीम् गा॑य॒त्रीम् त्रि॒ष्टुभ॒म् जग॑ती॒म् जग॑तीम् त्रि॒ष्टुभ॑म् गाय॒त्रीम् गा॑य॒त्रीम् त्रि॒ष्टुभ॒म् जग॑तीम् । \newline
15. त्रि॒ष्टुभ॒म् जग॑ती॒म् जग॑तीम् त्रि॒ष्टुभ॑म् त्रि॒ष्टुभ॒म् जग॑ती मनु॒ष्टुभ॑ मनु॒ष्टुभ॒म् जग॑तीम् त्रि॒ष्टुभ॑म् त्रि॒ष्टुभ॒म् जग॑ती मनु॒ष्टुभ᳚म् । \newline
16. जग॑ती मनु॒ष्टुभ॑ मनु॒ष्टुभ॒म् जग॑ती॒म् जग॑ती मनु॒ष्टुभ॑म् बृ॒हद् बृ॒ह द॑नु॒ष्टुभ॒म् जग॑ती॒म् जग॑ती मनु॒ष्टुभ॑म् बृ॒हत् । \newline
17. अ॒नु॒ष्टुभ॑म् बृ॒हद् बृ॒ह द॑नु॒ष्टुभ॑ मनु॒ष्टुभ॑म् बृ॒ह द॒र्क म॒र्कम् बृ॒ह द॑नु॒ष्टुभ॑ मनु॒ष्टुभ॑म् बृ॒ह द॒र्कम् । \newline
18. अ॒नु॒ष्टुभ॒मित्य॑नु - स्तुभ᳚म् । \newline
19. बृ॒ह द॒र्क म॒र्कम् बृ॒हद् बृ॒ह द॒र्कं ॅयु॑ञ्जा॒ना यु॑ञ्जा॒ना अ॒र्कम् बृ॒हद् बृ॒ह द॒र्कं ॅयु॑ञ्जा॒नाः । \newline
20. अ॒र्कं ॅयु॑ञ्जा॒ना यु॑ञ्जा॒ना अ॒र्क म॒र्कं ॅयु॑ञ्जा॒नाः सुवः॒ सुव॑र् युञ्जा॒ना अ॒र्क म॒र्कं ॅयु॑ञ्जा॒नाः सुवः॑ । \newline
21. यु॒ञ्जा॒नाः सुवः॒ सुव॑र् युञ्जा॒ना यु॑ञ्जा॒नाः सुव॒रा सुव॑र् युञ्जा॒ना यु॑ञ्जा॒नाः सुव॒रा । \newline
22. सुव॒रा सुवः॒ सुव॒रा ऽभ॑रन् नभर॒न् ना सुवः॒ सुव॒रा ऽभ॑रन्न् । \newline
23. आ ऽभ॑रन् नभर॒न्ना ऽभ॑रन् नि॒द मि॒द म॑भर॒न्ना ऽभ॑रन् नि॒दम् । \newline
24. अ॒भ॒र॒न् नि॒द मि॒द म॑भरन् नभरन् नि॒दम् । \newline
25. इ॒दमिती॒दम् । \newline
26. प॒ञ्चभि॑र् धा॒ता धा॒ता प॒ञ्चभिः॑ प॒ञ्चभि॑र् धा॒ता वि वि धा॒ता प॒ञ्चभिः॑ प॒ञ्चभि॑र् धा॒ता वि । \newline
27. प॒ञ्चभि॒रिति॑ प॒ञ्च - भिः॒ । \newline
28. धा॒ता वि वि धा॒ता धा॒ता वि द॑धौ दधौ॒ वि धा॒ता धा॒ता वि द॑धौ । \newline
29. वि द॑धौ दधौ॒ वि वि द॑धा वि॒द मि॒दम् द॑धौ॒ वि वि द॑धा वि॒दम् । \newline
30. द॒धा॒ वि॒द मि॒दम् द॑धौ दधा वि॒दं ॅयद् यदि॒दम् द॑धौ दधा वि॒दं ॅयत् । \newline
31. इ॒दं ॅयद् यदि॒द मि॒दं ॅयत् तासा॒म् तासां॒ ॅयदि॒द मि॒दं ॅयत् तासा᳚म् । \newline
32. यत् तासा॒म् तासां॒ ॅयद् यत् तासाꣳ॒॒ स्वसॄः॒ स्वसॄ॒ स्तासां॒ ॅयद् यत् तासाꣳ॒॒ स्वसॄः᳚ । \newline
33. तासाꣳ॒॒ स्वसॄः॒ स्वसॄ॒ स्तासा॒म् तासाꣳ॒॒ स्वसॄ॑ रजनय दजनय॒थ् स्वसॄ॒ स्तासा॒म् तासाꣳ॒॒ स्वसॄ॑ रजनयत् । \newline
34. स्वसॄ॑ रजनय दजनय॒थ् स्वसॄः॒ स्वसॄ॑ रजनय॒त् पञ्च॑पञ्च॒ पञ्च॑पञ्चा जनय॒थ् स्वसॄः॒ स्वसॄ॑ रजनय॒त् पञ्च॑पञ्च । \newline
35. अ॒ज॒न॒य॒त् पञ्च॑पञ्च॒ पञ्च॑पञ्चा जनय दजनय॒त् पञ्च॑पञ्च । \newline
36. पञ्च॑प॒ञ्चेति॒ पञ्च॑ - प॒ञ्च॒ । \newline
37. तासा॑ मु वु॒ तासा॒म् तासा॑ मु यन्ति यन्त्यु॒ तासा॒म् तासा॑ मु यन्ति । \newline
38. उ॒ य॒न्ति॒ य॒न्ति॒ उ॒ वु॒ य॒न्ति॒ प्र॒य॒वेण॑ प्रय॒वेण॑ यन्ति उ वु यन्ति प्रय॒वेण॑ । \newline
39. य॒न्ति॒ प्र॒य॒वेण॑ प्रय॒वेण॑ यन्ति यन्ति प्रय॒वेण॒ पञ्च॒ पञ्च॑ प्रय॒वेण॑ यन्ति यन्ति प्रय॒वेण॒ पञ्च॑ । \newline
40. प्र॒य॒वेण॒ पञ्च॒ पञ्च॑ प्रय॒वेण॑ प्रय॒वेण॒ पञ्च॒ नाना॒ नाना॒ पञ्च॑ प्रय॒वेण॑ प्रय॒वेण॒ पञ्च॒ नाना᳚ । \newline
41. प्र॒य॒वेणेति॑ प्र - य॒वेन॑ । \newline
42. पञ्च॒ नाना॒ नाना॒ पञ्च॒ पञ्च॒ नाना॑ रू॒पाणि॑ रू॒पाणि॒ नाना॒ पञ्च॒ पञ्च॒ नाना॑ रू॒पाणि॑ । \newline
43. नाना॑ रू॒पाणि॑ रू॒पाणि॒ नाना॒ नाना॑ रू॒पाणि॒ क्रत॑वः॒ क्रत॑वो रू॒पाणि॒ नाना॒ नाना॑ रू॒पाणि॒ क्रत॑वः । \newline
44. रू॒पाणि॒ क्रत॑वः॒ क्रत॑वो रू॒पाणि॑ रू॒पाणि॒ क्रत॑वो॒ वसा॑ना॒ वसा॑नाः॒ क्रत॑वो रू॒पाणि॑ रू॒पाणि॒ क्रत॑वो॒ वसा॑नाः । \newline
45. क्रत॑वो॒ वसा॑ना॒ वसा॑नाः॒ क्रत॑वः॒ क्रत॑वो॒ वसा॑नाः । \newline
46. वसा॑ना॒ इति॒ वसा॑नाः । \newline
47. त्रिꣳ॒॒शथ् स्वसा॑रः॒ स्वसा॑र स्त्रिꣳ॒॒शत् त्रिꣳ॒॒शथ् स्वसा॑र॒ उपोप॒ स्वसा॑र स्त्रिꣳ॒॒शत् त्रिꣳ॒॒शथ् स्वसा॑र॒ उप॑ । \newline
48. स्वसा॑र॒ उपोप॒ स्वसा॑रः॒ स्वसा॑र॒ उप॑ यन्ति य॒न्त्युप॒ स्वसा॑रः॒ स्वसा॑र॒ उप॑ यन्ति । \newline
49. उप॑ यन्ति य॒न्त्युपोप॑ यन्ति निष्कृ॒तन् नि॑ष्कृ॒तं ॅय॒न्त्युपोप॑ यन्ति निष्कृ॒तम् । \newline
50. य॒न्ति॒ नि॒ष्कृ॒तन् नि॑ष्कृ॒तं ॅय॑न्ति यन्ति निष्कृ॒तꣳ स॑मा॒नꣳ स॑मा॒नम् नि॑ष्कृ॒तं ॅय॑न्ति यन्ति निष्कृ॒तꣳ स॑मा॒नम् । \newline
51. नि॒ष्कृ॒तꣳ स॑मा॒नꣳ स॑मा॒नन् नि॑ष्कृ॒तन् नि॑ष्कृ॒तꣳ स॑मा॒नम् के॒तुम् के॒तुꣳ स॑मा॒नम् नि॑ष्कृ॒तन् नि॑ष्कृ॒तꣳ स॑मा॒नम् के॒तुम् । \newline
52. नि॒ष्कृ॒तमिति॑ निः - कृ॒तम् । \newline
53. स॒मा॒नम् के॒तुम् के॒तुꣳ स॑मा॒नꣳ स॑मा॒नम् के॒तुम् प्र॑तिमु॒ञ्चमा॑नाः प्रतिमु॒ञ्चमा॑नाः के॒तुꣳ स॑मा॒नꣳ स॑मा॒नम् के॒तुम् प्र॑तिमु॒ञ्चमा॑नाः । \newline
54. के॒तुम् प्र॑तिमु॒ञ्चमा॑नाः प्रतिमु॒ञ्चमा॑नाः के॒तुम् के॒तुम् प्र॑तिमु॒ञ्चमा॑नाः । \newline
55. प्र॒ति॒मु॒ञ्चमा॑ना॒ इति॑ प्रति - मु॒ञ्चमा॑नाः । \newline
\pagebreak
\markright{ TS 4.3.11.3  \hfill https://www.vedavms.in \hfill}

\section{ TS 4.3.11.3 }

\textbf{TS 4.3.11.3 } \newline
\textbf{Samhita Paata} \newline

ऋ॒तूꣳस्त॑न्वते क॒वयः॑ प्रजान॒तीर्मद्ध्ये॑छन्दसः॒ परि॑ यन्ति॒ भास्व॑तीः ॥ ज्योति॑ष्मती॒ प्रति॑ मुञ्चते॒ नभो॒ रात्री॑ दे॒वी सूर्य॑स्य व्र॒तानि॑ । वि प॑श्यन्ति प॒शवो॒ जाय॑माना॒ नाना॑रूपा मा॒तुर॒स्या उ॒पस्थे᳚ ॥ ए॒का॒ष्ट॒का तप॑सा॒ तप्य॑माना ज॒जान॒ गर्भं॑ महि॒मान॒मिन्द्रं᳚ । तेन॒ दस्यू॒न् व्य॑सहन्त दे॒वा ह॒न्ताऽसु॑राणा-मभव॒च्छची॑भिः ॥ अना॑नुजामनु॒जां माम॑कर्त स॒त्यं ॅवद॒न्त्यन्वि॑च्छ ए॒तत् । भू॒यास॑ -[  ] \newline

\textbf{Pada Paata} \newline

ऋ॒तून् । त॒न्व॒ते॒ । क॒वयः॑ । प्र॒जा॒न॒तीरिति॑ प्र - जा॒न॒तीः । मद्ध्ये॑छन्दस॒ इति॒ मद्ध्ये᳚ - छ॒न्द॒सः॒ । परीति॑ । य॒न्ति॒ । भास्व॑तीः ॥ ज्योति॑ष्मती । प्रतीति॑ । मु॒ञ्च॒ते॒ । नभः॑ । रात्री᳚ । दे॒वी । सूर्य॑स्य । व्र॒तानि॑ ॥ वीति॑ । प॒श्य॒न्ति॒ । प॒शवः॑ । जाय॑मानाः । नाना॑रूपा॒ इति॒ नाना᳚ - रू॒पाः॒ । मा॒तुः । अ॒स्याः । उ॒पस्थ॒ इत्यु॒प - स्थे॒ ॥ ए॒का॒ष्ट॒केत्ये॑क- अ॒ष्ट॒का । तप॑सा । तप्य॑माना । ज॒जान॑ । गर्भ᳚म् । म॒हि॒मान᳚म् । इन्द्र᳚म् ॥ तेन॑ । दस्यून्॑ । वीति॑ । अ॒स॒ह॒न्त॒ । दे॒वाः । ह॒न्ता । असु॑राणाम् । अ॒भ॒व॒त् । शची॑भि॒रिति॒ शचि॑ - भिः॒ ॥ अना॑नुजा॒मित्यना॑नु - जा॒म् । अ॒नु॒जामित्य॑नु-जाम् । माम् । अ॒क॒र्त॒ । स॒त्यम् । वद॑न्ती । अन्विति॑ । इ॒च्छे॒ । ए॒तत् ॥ भू॒यास᳚म् ।  \newline


\textbf{Krama Paata} \newline

ऋ॒तूꣳ स्त॑न्वते । त॒न्व॒ते॒ क॒वयः॑ । क॒वयः॑ प्रजान॒तीः । प्र॒जा॒न॒तीर् मद्ध्ये॑छन्दसः । प्र॒जा॒न॒तीरिति॑ प्र - जा॒न॒तीः । मद्ध्ये॑छन्दसः॒ परि॑ । मद्ध्ये॑छन्दस॒ इति॒ मद्ध्ये᳚ - छ॒न्द॒सः॒ । परि॑ यन्ति । य॒न्ति॒ भास्व॑तीः । भास्व॑ती॒रिति॒ भास्व॑तीः ॥ ज्योति॑ष्मती॒ प्रति॑ । प्रति॑ मुञ्चते । मु॒ञ्च॒ते॒ नभः॑ । नभो॒ रात्री᳚ । रात्री॑ दे॒वी । दे॒वी सूर्य॑स्य । सूर्य॑स्य व्र॒तानि॑ । व्र॒तानीति॑ व्र॒तानि॑ ॥ वि प॑श्यन्ति । प॒श्य॒न्ति॒ प॒शवः॑ । प॒शवो॒ जाय॑मानाः । जाय॑माना॒ नाना॑रूपाः । नाना॑रूपा मा॒तुः । नाना॑रूपा॒ इति॒ नाना᳚ - रू॒पाः॒ । मा॒तुर॒स्याः । अ॒स्या उ॒पस्थे᳚ । उ॒पस्थ॒ इत्यु॒प - स्थे॒ ॥ ए॒का॒ष्ट॒का तप॑सा । ए॒का॒ष्ट॒केत्ये॑क - अ॒ष्ट॒का । तप॑सा॒ तप्य॑माना । तप्य॑माना ज॒जान॑ । ज॒जान॒ गर्भ᳚म् । गर्भ॑म् महि॒मान᳚म् । म॒हि॒मान॒मिन्द्र᳚म् । इन्द्र॒मितीन्द्र᳚म् ॥ तेन॒ दस्यून्॑ । दस्यू॒न्॒. वि । व्य॑सहन्त । अ॒स॒ह॒न्त॒ दे॒वाः । दे॒वा ह॒न्ता । ह॒न्ताऽसु॑राणाम् । असु॑राणामभवत् । अ॒भ॒व॒च्छची॑भिः । शची॑भि॒रिति॒ शचि॑ - भिः॒ ॥ अना॑नुजामनु॒जाम् । अना॑नुजा॒मित्यना॑नु - जा॒म् । अ॒नु॒जाम् माम् । अ॒नु॒जामित्य॑नु - जाम् । माम॑कर्त । अ॒क॒र्त॒ स॒त्यम् । स॒त्यं ॅवद॑न्ती । वद॒न्त्यनु॑ । अन्वि॑च्छे । इ॒च्छ॒ ए॒तत् । ए॒तदित्ये॒तत् ॥ भू॒यास॑मस्य \newline

\textbf{Jatai Paata} \newline

1. ऋ॒तूꣳ स्त॑न्वते तन्वत ऋ॒तू-नृ॒तूꣳ स्त॑न्वते । \newline
2. त॒न्व॒ते॒ क॒वयः॑ क॒वय॑ स्तन्वते तन्वते क॒वयः॑ । \newline
3. क॒वयः॑ प्रजान॒तीः प्र॑जान॒तीः क॒वयः॑ क॒वयः॑ प्रजान॒तीः । \newline
4. प्र॒जा॒न॒तीर् मद्ध्ये॑छन्दसो॒ मद्ध्ये॑छन्दसः प्रजान॒तीः प्र॑जान॒तीर् मद्ध्ये॑छन्दसः । \newline
5. प्र॒जा॒न॒तीरिति॑ प्र - जा॒न॒तीः । \newline
6. मद्ध्ये॑छन्दसः॒ परि॒ परि॒ मद्ध्ये॑छन्दसो॒ मद्ध्ये॑छन्दसः॒ परि॑ । \newline
7. मद्ध्ये॑छन्दस॒ इति॒ मद्ध्ये᳚ - छ॒न्द॒सः॒ । \newline
8. परि॑ यन्ति यन्ति॒ परि॒ परि॑ यन्ति । \newline
9. य॒न्ति॒ भास्व॑ती॒र् भास्व॑तीर् यन्ति यन्ति॒ भास्व॑तीः । \newline
10. भास्व॑ती॒रिति॒ भास्व॑तीः । \newline
11. ज्योति॑ष्मती॒ प्रति॒ प्रति॒ ज्योति॑ष्मती॒ ज्योति॑ष्मती॒ प्रति॑ । \newline
12. प्रति॑ मुञ्चते मुञ्चते॒ प्रति॒ प्रति॑ मुञ्चते । \newline
13. मु॒ञ्च॒ते॒ नभो॒ नभो॑ मुञ्चते मुञ्चते॒ नभः॑ । \newline
14. नभो॒ रात्री॒ रात्री॒ नभो॒ नभो॒ रात्री᳚ । \newline
15. रात्री॑ दे॒वी दे॒वी रात्री॒ रात्री॑ दे॒वी । \newline
16. दे॒वी सूर्य॑स्य॒ सूर्य॑स्य दे॒वी दे॒वी सूर्य॑स्य । \newline
17. सूर्य॑स्य व्र॒तानि॑ व्र॒तानि॒ सूर्य॑स्य॒ सूर्य॑स्य व्र॒तानि॑ । \newline
18. व्र॒तानीति॑ व्र॒तानि॑ । \newline
19. वि प॑श्यन्ति पश्यन्ति॒ वि वि प॑श्यन्ति । \newline
20. प॒श्य॒न्ति॒ प॒शवः॑ प॒शवः॑ पश्यन्ति पश्यन्ति प॒शवः॑ । \newline
21. प॒शवो॒ जाय॑माना॒ जाय॑मानाः प॒शवः॑ प॒शवो॒ जाय॑मानाः । \newline
22. जाय॑माना॒ नाना॑रूपा॒ नाना॑रूपा॒ जाय॑माना॒ जाय॑माना॒ नाना॑रूपाः । \newline
23. नाना॑रूपा मा॒तुर् मा॒तुर् नाना॑रूपा॒ नाना॑रूपा मा॒तुः । \newline
24. नाना॑रूपा॒ इति॒ नाना᳚ - रू॒पाः॒ । \newline
25. मा॒तु र॒स्या अ॒स्या मा॒तुर् मा॒तु र॒स्याः । \newline
26. अ॒स्या उ॒पस्थ॑ उ॒पस्थे॑ अ॒स्या अ॒स्या उ॒पस्थे᳚ । \newline
27. उ॒पस्थ॒ इत्यु॒प - स्थे॒ । \newline
28. ए॒का॒ष्ट॒का तप॑सा॒ तप॑सै काष्ट॒कै का᳚ष्ट॒का तप॑सा । \newline
29. ए॒का॒ष्ट॒केत्ये॑क - अ॒ष्ट॒का । \newline
30. तप॑सा॒ तप्य॑माना॒ तप्य॑माना॒ तप॑सा॒ तप॑सा॒ तप्य॑माना । \newline
31. तप्य॑माना ज॒जान॑ ज॒जान॒ तप्य॑माना॒ तप्य॑माना ज॒जान॑ । \newline
32. ज॒जान॒ गर्भ॒म् गर्भ॑म् ज॒जान॑ ज॒जान॒ गर्भ᳚म् । \newline
33. गर्भ॑म् महि॒मान॑म् महि॒मान॒म् गर्भ॒म् गर्भ॑म् महि॒मान᳚म् । \newline
34. म॒हि॒मान॒ मिन्द्र॒ मिन्द्र॑म् महि॒मान॑म् महि॒मान॒ मिन्द्र᳚म् । \newline
35. इन्द्र॒मितीन्द्र᳚म् । \newline
36. तेन॒ दस्यू॒न् दस्यू॒न् तेन॒ तेन॒ दस्यून्॑ । \newline
37. दस्यू॒न्॒. वि वि दस्यू॒न् दस्यू॒न्॒. वि । \newline
38. व्य॑सहन्ता सहन्त॒ वि व्य॑सहन्त । \newline
39. अ॒स॒ह॒न्त॒ दे॒वा दे॒वा अ॑सहन्ता सहन्त दे॒वाः । \newline
40. दे॒वा ह॒न्ता ह॒न्ता दे॒वा दे॒वा ह॒न्ता । \newline
41. ह॒न्ता ऽसु॑राणा॒ मसु॑राणाꣳ ह॒न्ता ह॒न्ता ऽसु॑राणाम् । \newline
42. असु॑राणा मभव दभव॒ दसु॑राणा॒ मसु॑राणा मभवत् । \newline
43. अ॒भ॒व॒ च्छची॑भिः॒ शची॑भि रभव दभव॒ च्छची॑भिः । \newline
44. शची॑भि॒रिति॒ शचि॑ - भिः॒ । \newline
45. अना॑नुजा मनु॒जा म॑नु॒जा मना॑नुजा॒ मना॑नुजा मनु॒जाम् । \newline
46. अना॑नुजा॒मित्यना॑नु - जा॒म् । \newline
47. अ॒नु॒जाम् माम् मा म॑नु॒जा म॑नु॒जाम् माम् । \newline
48. अ॒नु॒जामित्य॑नु - जाम् । \newline
49. मा म॑कर्ता कर्त॒ माम् मा म॑कर्त । \newline
50. अ॒क॒र्त॒ स॒त्यꣳ स॒त्य म॑कर्ता कर्त स॒त्यम् । \newline
51. स॒त्यं ॅवद॑न्ती॒ वद॑न्ती स॒त्यꣳ स॒त्यं ॅवद॑न्ती । \newline
52. वद॒न् त्यन्वनु॒ वद॑न्ती॒ वद॒न् त्यनु॑ । \newline
53. अन्वि॑च्छ इच्छे॒ अन्व न्वि॑च्छे । \newline
54. इ॒च्छ॒ ए॒त दे॒त दि॑च्छ इच्छ ए॒तत् । \newline
55. ए॒तदित्ये॒तत् । \newline
56. भू॒यास॑ मस्यास्य भू॒यास॑म् भू॒यास॑ मस्य । \newline

\textbf{Ghana Paata } \newline

1. ऋ॒तूꣳ स्त॑न्वते तन्वत ऋ॒तू नृ॒तूꣳ स्त॑न्वते क॒वयः॑ क॒वय॑ स्तन्वत ऋ॒तू नृ॒तूꣳ स्त॑न्वते क॒वयः॑ । \newline
2. त॒न्व॒ते॒ क॒वयः॑ क॒वय॑ स्तन्वते तन्वते क॒वयः॑ प्रजान॒तीः प्र॑जान॒तीः क॒वय॑ स्तन्वते तन्वते क॒वयः॑ प्रजान॒तीः । \newline
3. क॒वयः॑ प्रजान॒तीः प्र॑जान॒तीः क॒वयः॑ क॒वयः॑ प्रजान॒तीर् मद्ध्ये॑छन्दसो॒ मद्ध्ये॑छन्दसः प्रजान॒तीः क॒वयः॑ क॒वयः॑ प्रजान॒तीर् मद्ध्ये॑छन्दसः । \newline
4. प्र॒जा॒न॒तीर् मद्ध्ये॑छन्दसो॒ मद्ध्ये॑छन्दसः प्रजान॒तीः प्र॑जान॒तीर् मद्ध्ये॑छन्दसः॒ परि॒ परि॒ मद्ध्ये॑छन्दसः प्रजान॒तीः प्र॑जान॒तीर् मद्ध्ये॑छन्दसः॒ परि॑ । \newline
5. प्र॒जा॒न॒तीरिति॑ प्र - जा॒न॒तीः । \newline
6. मद्ध्ये॑छन्दसः॒ परि॒ परि॒ मद्ध्ये॑छन्दसो॒ मद्ध्ये॑छन्दसः॒ परि॑ यन्ति यन्ति॒ परि॒ मद्ध्ये॑छन्दसो॒ मद्ध्ये॑छन्दसः॒ परि॑ यन्ति । \newline
7. मद्ध्ये॑छन्दस॒ इति॒ मद्ध्ये᳚ - छ॒न्द॒सः॒ । \newline
8. परि॑ यन्ति यन्ति॒ परि॒ परि॑ यन्ति॒ भास्व॑ती॒र् भास्व॑तीर् यन्ति॒ परि॒ परि॑ यन्ति॒ भास्व॑तीः । \newline
9. य॒न्ति॒ भास्व॑ती॒र् भास्व॑तीर् यन्ति यन्ति॒ भास्व॑तीः । \newline
10. भास्व॑ती॒रिति॒ भास्व॑तीः । \newline
11. ज्योति॑ष्मती॒ प्रति॒ प्रति॒ ज्योति॑ष्मती॒ ज्योति॑ष्मती॒ प्रति॑ मुञ्चते मुञ्चते॒ प्रति॒ ज्योति॑ष्मती॒ ज्योति॑ष्मती॒ प्रति॑ मुञ्चते । \newline
12. प्रति॑ मुञ्चते मुञ्चते॒ प्रति॒ प्रति॑ मुञ्चते॒ नभो॒ नभो॑ मुञ्चते॒ प्रति॒ प्रति॑ मुञ्चते॒ नभः॑ । \newline
13. मु॒ञ्च॒ते॒ नभो॒ नभो॑ मुञ्चते मुञ्चते॒ नभो॒ रात्री॒ रात्री॒ नभो॑ मुञ्चते मुञ्चते॒ नभो॒ रात्री᳚ । \newline
14. नभो॒ रात्री॒ रात्री॒ नभो॒ नभो॒ रात्री॑ दे॒वी दे॒वी रात्री॒ नभो॒ नभो॒ रात्री॑ दे॒वी । \newline
15. रात्री॑ दे॒वी दे॒वी रात्री॒ रात्री॑ दे॒वी सूर्य॑स्य॒ सूर्य॑स्य दे॒वी रात्री॒ रात्री॑ दे॒वी सूर्य॑स्य । \newline
16. दे॒वी सूर्य॑स्य॒ सूर्य॑स्य दे॒वी दे॒वी सूर्य॑स्य व्र॒तानि॑ व्र॒तानि॒ सूर्य॑स्य दे॒वी दे॒वी सूर्य॑स्य व्र॒तानि॑ । \newline
17. सूर्य॑स्य व्र॒तानि॑ व्र॒तानि॒ सूर्य॑स्य॒ सूर्य॑स्य व्र॒तानि॑ । \newline
18. व्र॒तानीति॑ व्र॒तानि॑ । \newline
19. वि प॑श्यन्ति पश्यन्ति॒ वि वि प॑श्यन्ति प॒शवः॑ प॒शवः॑ पश्यन्ति॒ वि वि प॑श्यन्ति प॒शवः॑ । \newline
20. प॒श्य॒न्ति॒ प॒शवः॑ प॒शवः॑ पश्यन्ति पश्यन्ति प॒शवो॒ जाय॑माना॒ जाय॑मानाः प॒शवः॑ पश्यन्ति पश्यन्ति प॒शवो॒ जाय॑मानाः । \newline
21. प॒शवो॒ जाय॑माना॒ जाय॑मानाः प॒शवः॑ प॒शवो॒ जाय॑माना॒ नाना॑रूपा॒ नाना॑रूपा॒ जाय॑मानाः प॒शवः॑ प॒शवो॒ जाय॑माना॒ नाना॑रूपाः । \newline
22. जाय॑माना॒ नाना॑रूपा॒ नाना॑रूपा॒ जाय॑माना॒ जाय॑माना॒ नाना॑रूपा मा॒तुर् मा॒तुर् नाना॑रूपा॒ जाय॑माना॒ जाय॑माना॒ नाना॑रूपा मा॒तुः । \newline
23. नाना॑रूपा मा॒तुर् मा॒तुर् नाना॑रूपा॒ नाना॑रूपा मा॒तु र॒स्या अ॒स्या मा॒तुर् नाना॑रूपा॒ नाना॑रूपा मा॒तुर॒स्याः । \newline
24. नाना॑रूपा॒ इति॒ नाना᳚ - रू॒पाः॒ । \newline
25. मा॒तु र॒स्या अ॒स्या मा॒तुर् मा॒तु र॒स्या उ॒पस्थ॑ उ॒पस्थे॑ अ॒स्या मा॒तुर् मा॒तु र॒स्या उ॒पस्थे᳚ । \newline
26. अ॒स्या उ॒पस्थ॑ उ॒पस्थे॑ अ॒स्या अ॒स्या उ॒पस्थे᳚ । \newline
27. उ॒पस्थ॒ इत्यु॒प - स्थे॒ । \newline
28. ए॒का॒ष्ट॒का तप॑सा॒ तप॑ सैकाष्ट॒ कैका᳚ष्ट॒का तप॑सा॒ तप्य॑माना॒ तप्य॑माना॒ तप॑ सैकाष्ट॒ 
कैका᳚ष्ट॒का तप॑सा॒ तप्य॑माना । \newline
29. ए॒का॒ष्ट॒केत्ये॑क - अ॒ष्ट॒का । \newline
30. तप॑सा॒ तप्य॑माना॒ तप्य॑माना॒ तप॑सा॒ तप॑सा॒ तप्य॑माना ज॒जान॑ ज॒जान॒ तप्य॑माना॒ तप॑सा॒ तप॑सा॒ तप्य॑माना ज॒जान॑ । \newline
31. तप्य॑माना ज॒जान॑ ज॒जान॒ तप्य॑माना॒ तप्य॑माना ज॒जान॒ गर्भ॒म् गर्भ॑म् ज॒जान॒ तप्य॑माना॒ तप्य॑माना ज॒जान॒ गर्भ᳚म् । \newline
32. ज॒जान॒ गर्भ॒म् गर्भ॑म् ज॒जान॑ ज॒जान॒ गर्भ॑म् महि॒मान॑म् महि॒मान॒म् गर्भ॑म् ज॒जान॑ ज॒जान॒ गर्भ॑म् महि॒मान᳚म् । \newline
33. गर्भ॑म् महि॒मान॑म् महि॒मान॒म् गर्भ॒म् गर्भ॑म् महि॒मान॒ मिन्द्र॒ मिन्द्र॑म् महि॒मान॒म् गर्भ॒म् गर्भ॑म् महि॒मान॒ मिन्द्र᳚म् । \newline
34. म॒हि॒मान॒ मिन्द्र॒ मिन्द्र॑म् महि॒मान॑म् महि॒मान॒ मिन्द्र᳚म् । \newline
35. इन्द्र॒मितीन्द्र᳚म् । \newline
36. तेन॒ दस्यू॒न् दस्यू॒न् तेन॒ तेन॒ दस्यू॒न्॒. वि वि दस्यू॒न् तेन॒ तेन॒ दस्यू॒न्॒. वि । \newline
37. दस्यू॒न्॒. वि वि दस्यू॒न् दस्यू॒न् व्य॑सहन्ता सहन्त॒ वि दस्यू॒न् दस्यू॒न् व्य॑सहन्त । \newline
38. व्य॑सहन्ता सहन्त॒ वि व्य॑सहन्त दे॒वा दे॒वा अ॑सहन्त॒ वि व्य॑सहन्त दे॒वाः । \newline
39. अ॒स॒ह॒न्त॒ दे॒वा दे॒वा अ॑सहन्ता सहन्त दे॒वा ह॒न्ता ह॒न्ता दे॒वा अ॑सहन्ता सहन्त दे॒वा ह॒न्ता । \newline
40. दे॒वा ह॒न्ता ह॒न्ता दे॒वा दे॒वा ह॒न्ता ऽसु॑राणा॒ मसु॑राणाꣳ ह॒न्ता दे॒वा दे॒वा ह॒न्ता ऽसु॑राणाम् । \newline
41. ह॒न्ता ऽसु॑राणा॒ मसु॑राणाꣳ ह॒न्ता ह॒न्ता ऽसु॑राणा मभव दभव॒ दसु॑राणाꣳ ह॒न्ता ह॒न्ता ऽसु॑राणा मभवत् । \newline
42. असु॑राणा मभव दभव॒ दसु॑राणा॒ मसु॑राणा मभव॒ च्छची॑भिः॒ शची॑भि रभव॒ दसु॑राणा॒ मसु॑राणा मभव॒ च्छची॑भिः । \newline
43. अ॒भ॒व॒ च्छची॑भिः॒ शची॑भि रभव दभव॒ च्छची॑भिः । \newline
44. शची॑भि॒रिति॒ शचि॑ - भिः॒ । \newline
45. अना॑नुजा मनु॒जा म॑नु॒जा मना॑नुजा॒ मना॑नुजा मनु॒जाम् माम् मा म॑नु॒जा मना॑नुजा॒ मना॑नुजा मनु॒जाम् माम् । \newline
46. अना॑नुजा॒मित्यना॑नु - जा॒म् । \newline
47. अ॒नु॒जाम् माम् मा म॑नु॒जा म॑नु॒जाम् मा म॑कर्ता कर्त॒ मा म॑नु॒जा म॑नु॒जाम् मा म॑कर्त । \newline
48. अ॒नु॒जामित्य॑नु - जाम् । \newline
49. मा म॑कर्ता कर्त॒ माम् मा म॑कर्त स॒त्यꣳ स॒त्य म॑कर्त॒ माम् मा म॑कर्त स॒त्यम् । \newline
50. अ॒क॒र्त॒ स॒त्यꣳ स॒त्य म॑कर्ता कर्त स॒त्यं ॅवद॑न्ती॒ वद॑न्ती स॒त्य म॑कर्ता कर्त स॒त्यं ॅवद॑न्ती । \newline
51. स॒त्यं ॅवद॑न्ती॒ वद॑न्ती स॒त्यꣳ स॒त्यं ॅवद॒न् त्यन्वनु॒ वद॑न्ती स॒त्यꣳ स॒त्यं ॅवद॒न्त्यनु॑ । \newline
52. वद॒न् त्यन्वनु॒ वद॑न्ती॒ वद॒न् त्यन्‌ वि॑च्छ इच्छे॒ अनु॒ वद॑न्ती॒ वद॒न् त्यन् वि॑च्छे । \newline
53. अन्वि॑च्छ इच्छे॒ अन्वन्वि॑च्छ ए॒त दे॒त दि॑च्छे॒ अन्वन्वि॑च्छ ए॒तत् । \newline
54. इ॒च्छ॒ ए॒त दे॒त दि॑च्छ इच्छ ए॒तत् । \newline
55. ए॒तदित्ये॒तत् । \newline
56. भू॒यास॑ मस्यास्य भू॒यास॑म् भू॒यास॑ मस्य सुम॒तौ सु॑म॒ता व॑स्य भू॒यास॑म् भू॒यास॑ मस्य सुम॒तौ । \newline
\pagebreak
\markright{ TS 4.3.11.4  \hfill https://www.vedavms.in \hfill}

\section{ TS 4.3.11.4 }

\textbf{TS 4.3.11.4 } \newline
\textbf{Samhita Paata} \newline

मस्य सुम॒तौ यथा॑ यू॒यम॒न्या वो॑ अ॒न्यामति॒ मा प्र यु॑क्त ॥ अभू॒न्मम॑ सुम॒तौ वि॒श्ववे॑दा॒ आष्ट॑ प्रति॒ष्ठामवि॑द॒द्धि गा॒धं । भू॒यास॑मस्य सुम॒तौ यथा॑ यू॒यम॒न्या वो॑ अ॒न्यामति॒ मा प्रयु॑क्त ॥ पञ्च॒ व्यु॑ष्टी॒रनु॒ पञ्च॒ दोहा॒ गां पञ्च॑नाम्नीमृ॒तवोऽनु॒ पञ्च॑ । पञ्च॒ दिशः॑ पञ्चद॒शेन॑ क्लृ॒प्ताः स॑मा॒नमू᳚र्द्ध्नीर॒भि लो॒कमेकं᳚ ॥ \newline

\textbf{Pada Paata} \newline

अ॒स्य॒ । सु॒म॒ताविति॑ सु - म॒तौ । यथा᳚ । यू॒यम् । अ॒न्या । वः॒ । अ॒न्याम् । अतीति॑ । मा । प्रेति॑ । यु॒क्त॒ ॥ अभू᳚त् । मम॑ । सु॒म॒ताविति॑ सु - म॒तौ । वि॒श्ववे॑दा॒ इति॑ वि॒श्व - वे॒दाः॒ । आष्ट॑ । प्र॒ति॒ष्ठामिति॑ प्रति - स्थाम् । अवि॑दत् । हि । गा॒धम् ॥ भू॒यास᳚म् । अ॒स्य॒ । सु॒म॒ताविति॑ सु - म॒तौ । यथा᳚ । यू॒यम् । अ॒न्या । वः॒ । अ॒न्याम् । अतीति॑ । मा । प्रेति॑ । यु॒क्त॒ ॥ पञ्च॑ । व्यु॑ष्टी॒रिति॒ वि - उ॒ष्टीः॒ । अन्विति॑ । पञ्च॑ । दोहाः᳚ । गाम् । पञ्च॑नाम्नी॒मिति॒ पञ्च॑ - ना॒म्नी॒म् । ऋ॒तवः॑ । अन्विति॑ । पञ्च॑ ॥ पञ्च॑ । दिशः॑ । प॒ञ्च॒द॒शेनेति॑ पञ्च - द॒शेन॑ । क्लृ॒प्ताः । स॒मा॒नमू᳚द्‌र्ध्नी॒रिति॑ समा॒न - मू॒द्‌र्ध्नीः॒ । अ॒भीति॑ । लो॒कम् । एक᳚म् ॥  \newline


\textbf{Krama Paata} \newline

अ॒स्य॒ सु॒म॒तौ । सु॒म॒तौ यथा᳚ । सु॒म॒ताविति॑ सु - म॒तौ । यथा॑ यू॒यम् । यू॒यम॒न्या । अ॒न्या वः॑ । वो॒ अ॒न्याम् । अ॒न्यामति॑ । अति॒ मा । मा प्र । प्र यु॑क्त । यु॒क्तेति॑ युक्त ॥ अभू॒न् मम॑ । मम॑ सुम॒तौ । सु॒म॒तौ वि॒श्ववे॑दाः । सु॒म॒ताविति॑ सु - म॒तौ । वि॒श्ववे॑दा॒ आष्ट॑ । वि॒श्ववे॑दा॒ इति॑ वि॒श्व - वे॒दाः॒ । आष्ट॑ प्रति॒ष्ठाम् । प्र॒ति॒ष्ठामवि॑दत् । प्र॒ति॒ष्ठामिति॑ प्रति - स्थाम् । अवि॑द॒द्धि । हि गा॒धम् । गा॒धमिति॑ गा॒धम् ॥ भू॒यास॑मस्य । अ॒स्य॒ सु॒म॒तौ । सु॒म॒तौ यथा᳚ । सु॒म॒ताविति॑ सु - म॒तौ । यथा॑ यू॒यम् । यू॒यम॒न्या । अ॒न्या वः॑ । वो॒ अ॒न्याम् । अ॒न्यामति॑ । अति॒ मा । मा प्र । प्र यु॑क्त । यु॒क्तेति॑ युक्त ॥ पञ्च॒ व्यु॑ष्टीः । व्यु॑ष्टी॒रनु॑ । व्यु॑ष्टी॒रिति॒ वि - उ॒ष्टीः॒ । अनु॒ पञ्च॑ । पञ्च॒ दोहाः᳚ । दोहा॒ गाम् । गाम् पञ्च॑नाम्नीम् । पञ्च॑नाम्नीमृ॒तवः॑ । पञ्च॑नाम्नी॒मिति॒ पञ्च॑ - ना॒म्नी॒म् । ऋ॒तवोऽनु॑ । अनु॒ पञ्च॑ । पञ्चेति॒ पञ्च॑ ॥ पञ्च॒ दिशः॑ । दिशः॑ पञ्चद॒शेन॑ । प॒ञ्च॒द॒शेन॑ क्लृ॒प्ताः । प॒ञ्च॒द॒शेनेति॑ पञ्च - द॒शेन॑ । क्लृ॒प्ताः स॑मा॒नमू᳚र्द्ध्नीः । स॒मा॒नमू᳚र्द्ध्नीर॒भि । स॒मा॒नमू᳚र्द्ध्नी॒रिति॑ समा॒न - मू॒र्द्ध्नीः॒ । अ॒भि लो॒कम् । लो॒कमेक᳚म् । एक॒मित्येक᳚म् । \newline

\textbf{Jatai Paata} \newline

1. अ॒स्य॒ सु॒म॒तौ सु॑म॒ता व॑स्यास्य सुम॒तौ । \newline
2. सु॒म॒तौ यथा॒ यथा॑ सुम॒तौ सु॑म॒तौ यथा᳚ । \newline
3. सु॒म॒ताविति॑ सु - म॒तौ । \newline
4. यथा॑ यू॒यं ॅयू॒यं ॅयथा॒ यथा॑ यू॒यम् । \newline
5. यू॒य म॒न्या ऽन्या यू॒यं ॅयू॒य म॒न्या । \newline
6. अ॒न्या वो॑ वो अ॒न्या ऽन्या वः॑ । \newline
7. वो॒ अ॒न्या म॒न्यां ॅवो॑ वो अ॒न्याम् । \newline
8. अ॒न्या मत्य त्य॒न्या म॒न्या मति॑ । \newline
9. अति॒ मा मा ऽत्यति॒ मा । \newline
10. मा प्र प्र मा मा प्र । \newline
11. प्र यु॑क्त युक्त॒ प्र प्र यु॑क्त । \newline
12. यु॒क्तेति॑ युक्त । \newline
13. अभू॒न् मम॒ ममा भू॒ दभू॒न् मम॑ । \newline
14. मम॑ सुम॒तौ सु॑म॒तौ मम॒ मम॑ सुम॒तौ । \newline
15. सु॒म॒तौ वि॒श्ववे॑दा वि॒श्ववे॑दाः सुम॒तौ सु॑म॒तौ वि॒श्ववे॑दाः । \newline
16. सु॒म॒ताविति॑ सु - म॒तौ । \newline
17. वि॒श्ववे॑दा॒ आष्टाष्ट॑ वि॒श्ववे॑दा वि॒श्ववे॑दा॒ आष्ट॑ । \newline
18. वि॒श्ववे॑दा॒ इति॑ वि॒श्व - वे॒दाः॒ । \newline
19. आष्ट॑ प्रति॒ष्ठाम् प्र॑ति॒ष्ठा माष्टाष्ट॑ प्रति॒ष्ठाम् । \newline
20. प्र॒ति॒ष्ठा मवि॑द॒ दवि॑दत् प्रति॒ष्ठाम् प्र॑ति॒ष्ठा मवि॑दत् । \newline
21. प्र॒ति॒ष्ठामिति॑ प्रति - स्थाम् । \newline
22. अवि॑द॒ द्धि ह्यवि॑द॒ दवि॑द॒ द्धि । \newline
23. हि गा॒धम् गा॒धꣳ हि हि गा॒धम् । \newline
24. गा॒धमिति॑ गा॒धम् । \newline
25. भू॒यास॑ मस्यास्य भू॒यास॑म् भू॒यास॑ मस्य । \newline
26. अ॒स्य॒ सु॒म॒तौ सु॑म॒ता व॑स्यास्य सुम॒तौ । \newline
27. सु॒म॒तौ यथा॒ यथा॑ सुम॒तौ सु॑म॒तौ यथा᳚ । \newline
28. सु॒म॒ताविति॑ सु - म॒तौ । \newline
29. यथा॑ यू॒यं ॅयू॒यं ॅयथा॒ यथा॑ यू॒यम् । \newline
30. यू॒य म॒न्या ऽन्या यू॒यं ॅयू॒य म॒न्या । \newline
31. अ॒न्या वो॑ वो अ॒न्या ऽन्या वः॑ । \newline
32. वो॒ अ॒न्या म॒न्यां ॅवो॑ वो अ॒न्याम् । \newline
33. अ॒न्या मत्य त्य॒न्या म॒न्या मति॑ । \newline
34. अति॒ मा मा ऽत्यति॒ मा । \newline
35. मा प्र प्र मा मा प्र । \newline
36. प्र यु॑क्त युक्त॒ प्र प्र यु॑क्त । \newline
37. यु॒क्तेति॑ युक्त । \newline
38. पञ्च॒ व्यु॑ष्टी॒र् व्यु॑ष्टीः॒ पञ्च॒ पञ्च॒ व्यु॑ष्टीः । \newline
39. व्यु॑ष्टी॒ रन्वनु॒ व्यु॑ष्टी॒र् व्यु॑ष्टी॒ रनु॑ । \newline
40. व्यु॑ष्टी॒रिति॒ वि - उ॒ष्टीः॒ । \newline
41. अनु॒ पञ्च॒ पञ्चान् वनु॒ पञ्च॑ । \newline
42. पञ्च॒ दोहा॒ दोहाः॒ पञ्च॒ पञ्च॒ दोहाः᳚ । \newline
43. दोहा॒ गाम् गाम् दोहा॒ दोहा॒ गाम् । \newline
44. गाम् पञ्च॑नाम्नी॒म् पञ्च॑नाम्नी॒म् गाम् गाम् पञ्च॑नाम्नीम् । \newline
45. पञ्च॑नाम्नी मृ॒तव॑ ऋ॒तवः॒ पञ्च॑नाम्नी॒म् पञ्च॑नाम्नी मृ॒तवः॑ । \newline
46. पञ्च॑नाम्नी॒मिति॒ पञ्च॑ - ना॒म्नी॒म् । \newline
47. ऋ॒तवो ऽन्वन् वृ॒तव॑ ऋ॒तवो ऽनु॑ । \newline
48. अनु॒ पञ्च॒ पञ्चान्वनु॒ पञ्च॑ । \newline
49. पञ्चेति॒ पञ्च॑ । \newline
50. पञ्च॒ दिशो॒ दिशः॒ पञ्च॒ पञ्च॒ दिशः॑ । \newline
51. दिशः॑ पञ्चद॒शेन॑ पञ्चद॒शेन॒ दिशो॒ दिशः॑ पञ्चद॒शेन॑ । \newline
52. प॒ञ्च॒द॒शेन॑ क्लृ॒प्ताः क्लृ॒प्ताः प॑ञ्चद॒शेन॑ पञ्चद॒शेन॑ क्लृ॒प्ताः । \newline
53. प॒ञ्च॒द॒शेनेति॑ पञ्च - द॒शेन॑ । \newline
54. क्लृ॒प्ताः स॑मा॒नमू᳚र्द्ध्नीः समा॒नमू᳚र्द्ध्नीः क्लृ॒प्ताः क्लृ॒प्ताः स॑मा॒नमू᳚र्द्ध्नीः । \newline
55. स॒मा॒नमू᳚र्द्ध्नी र॒भ्य॑भि स॑मा॒नमू᳚र्द्ध्नीः समा॒नमू᳚र्द्ध्नी र॒भि । \newline
56. स॒मा॒नमू᳚र्द्ध्नी॒रिति॑ समा॒न - मू॒र्द्ध्नीः॒ । \newline
57. अ॒भि लो॒कम् ॅलो॒क म॒भ्य॑भि लो॒कम् । \newline
58. लो॒क मेक॒ मेक॑म् ॅलो॒कम् ॅलो॒क मेक᳚म् । \newline
59. एक॒मित्येक᳚म् । \newline

\textbf{Ghana Paata } \newline

1. अ॒स्य॒ सु॒म॒तौ सु॑म॒ता व॑स्यास्य सुम॒तौ यथा॒ यथा॑ सुम॒ता व॑स्यास्य सुम॒तौ यथा᳚ । \newline
2. सु॒म॒तौ यथा॒ यथा॑ सुम॒तौ सु॑म॒तौ यथा॑ यू॒यं ॅयू॒यं ॅयथा॑ सुम॒तौ सु॑म॒तौ यथा॑ यू॒यम् । \newline
3. सु॒म॒ताविति॑ सु - म॒तौ । \newline
4. यथा॑ यू॒यं ॅयू॒यं ॅयथा॒ यथा॑ यू॒य म॒न्या ऽन्या यू॒यं ॅयथा॒ यथा॑ यू॒य म॒न्या । \newline
5. यू॒य म॒न्या ऽन्या यू॒यं ॅयू॒य म॒न्या वो॑ वो अ॒न्या यू॒यं ॅयू॒य म॒न्या वः॑ । \newline
6. अ॒न्या वो॑ वो अ॒न्या ऽन्या वो॑ अ॒न्या म॒न्यां ॅवो॑ अ॒न्या ऽन्या वो॑ अ॒न्याम् । \newline
7. वो॒ अ॒न्या म॒न्यां ॅवो॑ वो अ॒न्या मत्य त्य॒न्यां ॅवो॑ वो अ॒न्या मति॑ । \newline
8. अ॒न्या मत्य त्य॒न्या म॒न्या मति॒ मा मा ऽत्य॒न्या म॒न्या मति॒ मा । \newline
9. अति॒ मा मा ऽत्यति॒ मा प्र प्र मा ऽत्यति॒ मा प्र । \newline
10. मा प्र प्र मा मा प्र यु॑क्त युक्त॒ प्र मा मा प्र यु॑क्त । \newline
11. प्र यु॑क्त युक्त॒ प्र प्र यु॑क्त । \newline
12. यु॒क्तेति॑ युक्त । \newline
13. अभू॒न् मम॒ ममाभू॒ दभू॒न् मम॑ सुम॒तौ सु॑म॒तौ ममाभू॒ दभू॒न् मम॑ सुम॒तौ । \newline
14. मम॑ सुम॒तौ सु॑म॒तौ मम॒ मम॑ सुम॒तौ वि॒श्ववे॑दा वि॒श्ववे॑दाः सुम॒तौ मम॒ मम॑ सुम॒तौ वि॒श्ववे॑दाः । \newline
15. सु॒म॒तौ वि॒श्ववे॑दा वि॒श्ववे॑दाः सुम॒तौ सु॑म॒तौ वि॒श्ववे॑दा॒ आष्टाष्ट॑ वि॒श्ववे॑दाः सुम॒तौ सु॑म॒तौ वि॒श्ववे॑दा॒ आष्ट॑ । \newline
16. सु॒म॒ताविति॑ सु - म॒तौ । \newline
17. वि॒श्ववे॑दा॒ आष्टाष्ट॑ वि॒श्ववे॑दा वि॒श्ववे॑दा॒ आष्ट॑ प्रति॒ष्ठाम् प्र॑ति॒ष्ठा माष्ट॑ वि॒श्ववे॑दा वि॒श्ववे॑दा॒ आष्ट॑ प्रति॒ष्ठाम् । \newline
18. वि॒श्ववे॑दा॒ इति॑ वि॒श्व - वे॒दाः॒ । \newline
19. आष्ट॑ प्रति॒ष्ठाम् प्र॑ति॒ष्ठा माष्टाष्ट॑ प्रति॒ष्ठा मवि॑द॒ दवि॑दत् प्रति॒ष्ठा माष्टाष्ट॑ प्रति॒ष्ठा मवि॑दत् । \newline
20. प्र॒ति॒ष्ठा मवि॑द॒ दवि॑दत् प्रति॒ष्ठाम् प्र॑ति॒ष्ठा मवि॑द॒द्धि ह्यवि॑दत् प्रति॒ष्ठाम् प्र॑ति॒ष्ठा मवि॑द॒द्धि । \newline
21. प्र॒ति॒ष्ठामिति॑ प्रति - स्थाम् । \newline
22. अवि॑द॒द्धि ह्यवि॑द॒ दवि॑द॒द्धि गा॒धम् गा॒धꣳ ह्यवि॑द॒ दवि॑द॒द्धि गा॒धम् । \newline
23. हि गा॒धम् गा॒धꣳ हि हि गा॒धम् । \newline
24. गा॒धमिति॑ गा॒धम् । \newline
25. भू॒यास॑ मस्यास्य भू॒यास॑म् भू॒यास॑ मस्य सुम॒तौ सु॑म॒ता व॑स्य भू॒यास॑म् भू॒यास॑ मस्य सुम॒तौ । \newline
26. अ॒स्य॒ सु॒म॒तौ सु॑म॒ता व॑स्यास्य सुम॒तौ यथा॒ यथा॑ सुम॒ता व॑स्यास्य सुम॒तौ यथा᳚ । \newline
27. सु॒म॒तौ यथा॒ यथा॑ सुम॒तौ सु॑म॒तौ यथा॑ यू॒यं ॅयू॒यं ॅयथा॑ सुम॒तौ सु॑म॒तौ यथा॑ यू॒यम् । \newline
28. सु॒म॒ताविति॑ सु - म॒तौ । \newline
29. यथा॑ यू॒यं ॅयू॒यं ॅयथा॒ यथा॑ यू॒य म॒न्या ऽन्या यू॒यं ॅयथा॒ यथा॑ यू॒य म॒न्या । \newline
30. यू॒य म॒न्या ऽन्या यू॒यं ॅयू॒य म॒न्या वो॑ वो अ॒न्या यू॒यं ॅयू॒य म॒न्या वः॑ । \newline
31. अ॒न्या वो॑ वो अ॒न्या ऽन्या वो॑ अ॒न्या म॒न्यां ॅवो॑ अ॒न्या ऽन्या वो॑ अ॒न्याम् । \newline
32. वो॒ अ॒न्या म॒न्यां ॅवो॑ वो अ॒न्या मत्य त्य॒न्यां ॅवो॑ वो अ॒न्या मति॑ । \newline
33. अ॒न्या मत्य त्य॒न्या म॒न्या मति॒ मा मा ऽत्य॒न्या म॒न्या मति॒ मा । \newline
34. अति॒ मा मा ऽत्यति॒ मा प्र प्र मा ऽत्यति॒ मा प्र । \newline
35. मा प्र प्र मा मा प्र यु॑क्त युक्त॒ प्र मा मा प्र यु॑क्त । \newline
36. प्र यु॑क्त युक्त॒ प्र प्र यु॑क्त । \newline
37. यु॒क्तेति॑ युक्त । \newline
38. पञ्च॒ व्यु॑ष्टी॒र् व्यु॑ष्टीः॒ पञ्च॒ पञ्च॒ व्यु॑ष्टी॒ रन्वनु॒ व्यु॑ष्टीः॒ पञ्च॒ पञ्च॒ व्यु॑ष्टी॒ रनु॑ । \newline
39. व्यु॑ष्टी॒ रन्वनु॒ व्यु॑ष्टी॒र् व्यु॑ष्टी॒ रनु॒ पञ्च॒ पञ्चानु॒ व्यु॑ष्टी॒र् व्यु॑ष्टी॒ रनु॒ पञ्च॑ । \newline
40. व्यु॑ष्टी॒रिति॒ वि - उ॒ष्टीः॒ । \newline
41. अनु॒ पञ्च॒ पञ्चान्वनु॒ पञ्च॒ दोहा॒ दोहाः॒ पञ्चान्वनु॒ पञ्च॒ दोहाः᳚ । \newline
42. पञ्च॒ दोहा॒ दोहाः॒ पञ्च॒ पञ्च॒ दोहा॒ गाम् गाम् दोहाः॒ पञ्च॒ पञ्च॒ दोहा॒ गाम् । \newline
43. दोहा॒ गाम् गाम् दोहा॒ दोहा॒ गाम् पञ्च॑नाम्नी॒म् पञ्च॑नाम्नी॒म् गाम् दोहा॒ दोहा॒ गाम् पञ्च॑नाम्नीम् । \newline
44. गाम् पञ्च॑नाम्नी॒म् पञ्च॑नाम्नी॒म् गाम् गाम् पञ्च॑नाम्नी मृ॒तव॑ ऋ॒तवः॒ पञ्च॑नाम्नी॒म् गाम् गाम् पञ्च॑नाम्नी मृ॒तवः॑ । \newline
45. पञ्च॑नाम्नी मृ॒तव॑ ऋ॒तवः॒ पञ्च॑नाम्नी॒म् पञ्च॑नाम्नी मृ॒तवो ऽन्वन् वृ॒तवः॒ पञ्च॑नाम्नी॒म् पञ्च॑नाम्नी मृ॒तवो ऽनु॑ । \newline
46. पञ्च॑नाम्नी॒मिति॒ पञ्च॑ - ना॒म्नी॒म् । \newline
47. ऋ॒तवो ऽन्वन्वृ॒तव॑ ऋ॒तवो ऽनु॒ पञ्च॒ पञ्चान्वृ॒तव॑ ऋ॒तवो ऽनु॒ पञ्च॑ । \newline
48. अनु॒ पञ्च॒ पञ्चान्वनु॒ पञ्च॑ । \newline
49. पञ्चेति॒ पञ्च॑ । \newline
50. पञ्च॒ दिशो॒ दिशः॒ पञ्च॒ पञ्च॒ दिशः॑ पञ्चद॒शेन॑ पञ्चद॒शेन॒ दिशः॒ पञ्च॒ पञ्च॒ दिशः॑ पञ्चद॒शेन॑ । \newline
51. दिशः॑ पञ्चद॒शेन॑ पञ्चद॒शेन॒ दिशो॒ दिशः॑ पञ्चद॒शेन॑ क्लृ॒प्ताः क्लृ॒प्ताः प॑ञ्चद॒शेन॒ दिशो॒ दिशः॑ पञ्चद॒शेन॑ क्लृ॒प्ताः । \newline
52. प॒ञ्च॒द॒शेन॑ क्लृ॒प्ताः क्लृ॒प्ताः प॑ञ्चद॒शेन॑ पञ्चद॒शेन॑ क्लृ॒प्ताः स॑मा॒नमू᳚र्द्ध्नीः समा॒नमू᳚र्द्ध्नीः क्लृ॒प्ताः प॑ञ्चद॒शेन॑ पञ्चद॒शेन॑ क्लृ॒प्ताः स॑मा॒नमू᳚र्द्ध्नीः । \newline
53. प॒ञ्च॒द॒शेनेति॑ पञ्च - द॒शेन॑ । \newline
54. क्लृ॒प्ताः स॑मा॒नमू᳚र्द्ध्नीः समा॒नमू᳚र्द्ध्नीः क्लृ॒प्ताः क्लृ॒प्ताः स॑मा॒नमू᳚र्द्ध्नी र॒भ्य॑भि स॑मा॒नमू᳚र्द्ध्नीः क्लृ॒प्ताः क्लृ॒प्ताः स॑मा॒नमू᳚र्द्ध्नी र॒भि । \newline
55. स॒मा॒नमू᳚र्द्ध्नी र॒भ्य॑भि स॑मा॒नमू᳚र्द्ध्नीः समा॒नमू᳚र्द्ध्नी र॒भि लो॒कम् ॅलो॒क म॒भि स॑मा॒नमू᳚र्द्ध्नीः समा॒नमू᳚र्द्ध्नी र॒भि लो॒कम् । \newline
56. स॒मा॒नमू᳚र्द्ध्नी॒रिति॑ समा॒न - मू॒र्द्ध्नीः॒ । \newline
57. अ॒भि लो॒कम् ॅलो॒क म॒भ्य॑भि लो॒क मेक॒ मेक॑म् ॅलो॒क म॒भ्य॑भि लो॒क मेक᳚म् । \newline
58. लो॒क मेक॒ मेक॑म् ॅलो॒कम् ॅलो॒क मेक᳚म् । \newline
59. एक॒मित्येक᳚म् । \newline
\pagebreak
\markright{ TS 4.3.11.5  \hfill https://www.vedavms.in \hfill}

\section{ TS 4.3.11.5 }

\textbf{TS 4.3.11.5 } \newline
\textbf{Samhita Paata} \newline

ऋ॒तस्य॒ गर्भः॑ प्रथ॒मा व्यू॒षुष्य॒पामेका॑ महि॒मानं॑ बिभर्ति । सूर्य॒स्यैका॒ चर॑ति निष्कृ॒तेषु॑ घ॒र्मस्यैका॑ सवि॒तैकां॒ नि य॑च्छति ॥ या प्र॑थ॒मा व्यौच्छ॒थ् सा धे॒नुर॑भवद्य॒मे । सा नः॒ पय॑स्वती धु॒क्ष्वोत्त॑रामुत्तराꣳ॒॒ समां᳚ ॥ शु॒क्रर्.ष॑भा॒ नभ॑सा॒ ज्योति॒षा ऽऽ*गा᳚द्-वि॒श्वरू॑पा शब॒लीर॒ग्निके॑तुः । स॒मा॒नमर्थꣳ॑ स्वप॒स्यमा॑ना॒ बिभ्र॑ती ज॒राम॑जर उष॒ आऽगाः᳚ ॥ ऋ॒तू॒नां पत्नी᳚ ( ) प्रथ॒मेयमाऽगा॒दह्नां᳚ ने॒त्री ज॑नि॒त्री प्र॒जानां᳚ । एका॑ स॒ती ब॑हु॒धोषो॒ व्यु॑च्छ॒स्यजी᳚र्णा॒ त्वं ज॑रयसि॒ सर्व॑म॒न्यत् ॥ \newline

\textbf{Pada Paata} \newline

ऋ॒तस्य॑ । गर्भः॑ । प्र॒थ॒मा । व्यू॒षुषीति॑ वि - ऊ॒षुषी᳚ । अ॒पाम् । एका᳚ । म॒हि॒मान᳚म् । बि॒भ॒र्ति॒ ॥ सूर्य॑स्य । एका᳚ । चर॑ति । नि॒ष्कृ॒तेष्विति॑ निः-कृ॒तेषु॑ । घ॒र्मस्य॑ । एका᳚ । स॒वि॒ता । एका᳚म् । नीति॑ । य॒च्छ॒ति॒ ॥ या । प्र॒थ॒मा । व्यौच्छ॒दिति॑ वि - औच्छ॑त् । सा । धे॒नुः । अ॒भ॒व॒त् । य॒मे ॥ सा । नः॒ । पय॑स्वती । धु॒क्ष्व॒ । उत्त॑रामुत्तरा॒मित्युत्त॑रां-उ॒त्त॒रा॒म् । समा᳚म् ॥ शु॒क्रर्.ष॒भेति॑ शु॒क्र - ऋ॒ष॒भा॒ । नभ॑सा । ज्योति॑षा । एति॑ । अ॒गा॒त् । वि॒श्वरू॒पेति॑ वि॒श्व - रू॒पा॒ । श॒ब॒लीः । अ॒ग्निके॑तु॒रित्य॒ग्नि - के॒तुः॒ ॥ स॒मा॒नम् । अर्थ᳚म् । स्व॒प॒स्यमा॒नेति॑ सु-अ॒प॒स्यमा॑ना । बिभ्र॑ती । ज॒राम् । अ॒ज॒रे॒ । उ॒षः॒ । एति॑ । अ॒गाः॒ ॥ ऋ॒तू॒नाम् । पत्नी᳚ ( ) । प्र॒थ॒मा । इ॒यम् । एति॑ । अ॒गा॒त् । अह्ना᳚म् । ने॒त्री । ज॒नि॒त्री । प्र॒जाना॒मिति॑ प्र - जाना᳚म् ॥ एका᳚ । स॒ती । ब॒हु॒धेति॑ बहु - धा । उ॒षः॒ । वीति॑ । उ॒च्छ॒सि॒ । अजी᳚र्णा । त्वम् । ज॒र॒य॒सि॒ । सर्व᳚म् । अ॒न्यत् ॥  \newline


\textbf{Krama Paata} \newline

ऋ॒तस्य॒ गर्भः॑ । गर्भः॑ प्रथ॒मा । प्र॒थ॒मा व्यू॒षुषी᳚ । व्यू॒षुष्य॒पाम् । व्यू॒षुषीति॑ वि - ऊ॒षुषी᳚ । अ॒पामेका᳚ । एका॑ महि॒मान᳚म् । म॒हि॒मान॑म् बिभर्ति । बि॒भ॒र्तीति॑ बिभर्ति ॥ सूर्य॒स्यैका᳚ । एका॒ चर॑ति । चर॑ति निष्कृ॒तेषु॑ । नि॒ष्कृ॒तेषु॑ घ॒र्मस्य॑ । नि॒ष्कृ॒तेष्विति॑ निः - कृ॒तेषु॑ । घ॒र्मस्यैका᳚ । एका॑ सवि॒ता । स॒वि॒तैका᳚म् । एका॒म् नि । नि य॑च्छति । य॒च्छ॒तीति॑ यच्छति ॥ या प्र॑थ॒मा । प्र॒थ॒मा व्यौच्छ॑त् । व्यौच्छ॒थ् सा । व्यौच्छ॒दिति॑ वि - औच्छ॑त् । सा धे॒नुः । धे॒नुर॑भवत् । अ॒भ॒व॒द् य॒मे । य॒म इति॑ य॒मे ॥ सा नः॑ । नः॒ पय॑स्वती । पय॑स्वती धु॒क्ष्व । धु॒क्ष्वोत्त॑रामुत्तराम् । उत्त॑रामुत्तराꣳ॒॒ समा᳚म् । उत्त॑रामुत्तरा॒मित्युत्त॑राम् - उ॒त्त॒रा॒म् । समा॒मिति॒ समा᳚म् ॥ शु॒क्रर्.ष॑भा॒ नभ॑सा । शु॒क्रर्.ष॒भेति॑ शु॒क्र - ऋ॒ष॒भा॒ । नभ॑सा॒ ज्योति॑षा । ज्योति॒षा । आऽगा᳚त् । अ॒गा॒द् वि॒श्वरू॑पा । वि॒श्वरू॑पा शब॒लीः । वि॒श्वरू॒पेति॑ वि॒श्व - रू॒पा॒ । श॒ब॒लीर॒ग्निके॑तुः । अ॒ग्निके॑तु॒रित्य॒ग्नि - के॒तुः॒ ॥ स॒मा॒नमर्त्थ᳚म् । अर्त्थꣳ॑ स्वप॒स्यमा॑ना । स्व॒प॒स्यमा॑ना॒ बिभ्र॑ती । स्व॒प॒स्यमा॒नेति॑ सु - अ॒प॒स्यमा॑ना । बिभ्र॑ती ज॒राम् । ज॒राम॑जरे । अ॒ज॒र॒ उ॒षः॒ । उ॒ष॒ आ । आऽगाः᳚ । अ॒गा॒ इत्य॑गाः ॥ ऋ॒तू॒नाम् पत्नी᳚ ( ) । पत्नी᳚ प्रथ॒मा । प्र॒थ॒मेयम् । इ॒यमा । आऽगा᳚त् । अ॒गा॒दह्ना᳚म् । अह्ना᳚म् ने॒त्री । ने॒त्री ज॑नि॒त्री । ज॒नि॒त्री प्र॒जाना᳚म् । प्र॒जाना॒मिति॑ प्र - जाना᳚म् ॥ एका॑ स॒ती । स॒ती ब॑हु॒धा । ब॒हु॒धोषः॑ । ब॒हु॒धेति॑ बहु - धा । उ॒षो॒ वि । व्यु॑च्छसि । उ॒च्छ॒स्यजी᳚र्णा । अजी᳚र्णा॒ त्वम् । त्वम् ज॑रयसि । ज॒र॒य॒सि॒ सर्व᳚म् । सर्व॑म॒न्यत् । अ॒न्यदित्य॒न्यत् । \newline

\textbf{Jatai Paata} \newline

1. ऋ॒तस्य॒ गर्भो॒ गर्भ॑ ऋ॒तस्य॒ र्‌तस्य॒ गर्भः॑ । \newline
2. गर्भः॑ प्रथ॒मा प्र॑थ॒मा गर्भो॒ गर्भः॑ प्रथ॒मा । \newline
3. प्र॒थ॒मा व्यू॒षुषी᳚ व्यू॒षुषी᳚ प्रथ॒मा प्र॑थ॒मा व्यू॒षुषी᳚ । \newline
4. व्यू॒षुष्य॒पा म॒पां ॅव्यू॒षुषी᳚ व्यू॒षुष्य॒पाम् । \newline
5. व्यू॒षुषीति॑ वि - ऊ॒षुषी᳚ । \newline
6. अ॒पा मेकैका॒ ऽपा म॒पा मेका᳚ । \newline
7. एका॑ महि॒मान॑म् महि॒मान॒ मेकैका॑ महि॒मान᳚म् । \newline
8. म॒हि॒मान॑म् बिभर्ति बिभर्ति महि॒मान॑म् महि॒मान॑म् बिभर्ति । \newline
9. बि॒भ॒र्तीति॑ बिभर्ति । \newline
10. सूर्य॒ स्यैकैका॒ सूर्य॑स्य॒ सूर्य॒ स्यैका᳚ । \newline
11. एका॒ चर॑ति॒ चर॒ त्येकैका॒ चर॑ति । \newline
12. चर॑ति निष्कृ॒तेषु॑ निष्कृ॒तेषु॒ चर॑ति॒ चर॑ति निष्कृ॒तेषु॑ । \newline
13. नि॒ष्कृ॒तेषु॑ घ॒र्मस्य॑ घ॒र्मस्य॑ निष्कृ॒तेषु॑ निष्कृ॒तेषु॑ घ॒र्मस्य॑ । \newline
14. नि॒ष्कृ॒तेष्विति॑ निः - कृ॒तेषु॑ । \newline
15. घ॒र्म स्यैकैका॑ घ॒र्मस्य॑ घ॒र्म स्यैका᳚ । \newline
16. एका॑ सवि॒ता स॑वि॒तै कैका॑ सवि॒ता । \newline
17. स॒वि॒तैका॒ मेकाꣳ॑ सवि॒ता स॑वि॒ तैका᳚म् । \newline
18. एका॒म् नि न्येका॒ मेका॒म् नि । \newline
19. नि य॑च्छति यच्छति॒ नि नि य॑च्छति । \newline
20. य॒च्छ॒तीति॑ यच्छति । \newline
21. या प्र॑थ॒मा प्र॑थ॒मा या या प्र॑थ॒मा । \newline
22. प्र॒थ॒मा व्यौच्छ॒द् व्यौच्छ॑त् प्रथ॒मा प्र॑थ॒मा व्यौच्छ॑त् । \newline
23. व्यौच्छ॒थ् सा सा व्यौच्छ॒द् व्यौच्छ॒थ् सा । \newline
24. व्यौच्छ॒दिति॑ वि - औच्छ॑त् । \newline
25. सा धे॒नुर् धे॒नुः सा सा धे॒नुः । \newline
26. धे॒नु र॑भव दभवद् धे॒नुर् धे॒नु र॑भवत् । \newline
27. अ॒भ॒व॒द् य॒मे य॒मे अ॑भव दभवद् य॒मे । \newline
28. य॒म इति॑ य॒मे । \newline
29. सा नो॑ नः॒ सा सा नः॑ । \newline
30. नः॒ पय॑स्वती॒ पय॑स्वती नो नः॒ पय॑स्वती । \newline
31. पय॑स्वती धुक्ष्व धुक्ष्व॒ पय॑स्वती॒ पय॑स्वती धुक्ष्व । \newline
32. धु॒क्ष्वो त्त॑रामुत्तरा॒ मुत्त॑रामुत्तराम् धुक्ष्व धु॒क्ष्वो त्त॑रामुत्तराम् । \newline
33. उत्त॑रामुत्तराꣳ॒॒ समाꣳ॒॒ समा॒ मुत्त॑रामुत्तरा॒ मुत्त॑रामुत्तराꣳ॒॒ समा᳚म् । \newline
34. उत्त॑रामुत्तरा॒मित्युत्त॑रां - उ॒त्त॒रा॒म् । \newline
35. समा॒मिति॒ समा᳚म् । \newline
36. शु॒क्रर्.ष॑भा॒ नभ॑सा॒ नभ॑सा शु॒क्रर्.ष॑भा शु॒क्रर्.ष॑भा॒ नभ॑सा । \newline
37. शु॒क्रर्.ष॒भेति॑ शु॒क्र - ऋ॒ष॒भा॒ । \newline
38. नभ॑सा॒ ज्योति॑षा॒ ज्योति॑षा॒ नभ॑सा॒ नभ॑सा॒ ज्योति॑षा । \newline
39. ज्योति॒षा ऽऽज्योति॑षा॒ ज्योति॒षा । \newline
40. आ ऽगा॑ दगा॒दा ऽगा᳚त् । \newline
41. अ॒गा॒द् वि॒श्वरू॑पा वि॒श्वरू॑पा ऽगादगाद् वि॒श्वरू॑पा । \newline
42. वि॒श्वरू॑पा शब॒लीः श॑ब॒लीर् वि॒श्वरू॑पा वि॒श्वरू॑पा शब॒लीः । \newline
43. वि॒श्वरू॒पेति॑ वि॒श्व - रू॒पा॒ । \newline
44. श॒ब॒ली र॒ग्निके॑तु र॒ग्निके॑तुः शब॒लीः श॑ब॒ली र॒ग्निके॑तुः । \newline
45. अ॒ग्निके॑तु॒रित्य॒ग्नि - के॒तुः॒ । \newline
46. स॒मा॒न मर्थ॒ मर्थꣳ॑ समा॒नꣳ स॑मा॒न मर्थ᳚म् । \newline
47. अर्थꣳ॑ स्वप॒स्यमा॑ना स्वप॒स्यमा॒ना ऽर्थ॒ मर्थꣳ॑ स्वप॒स्यमा॑ना । \newline
48. स्व॒प॒स्यमा॑ना॒ बिभ्र॑ती॒ बिभ्र॑ती स्वप॒स्यमा॑ना स्वप॒स्यमा॑ना॒ बिभ्र॑ती । \newline
49. स्व॒प॒स्यमा॒नेति॑ सु - अ॒प॒स्यमा॑ना । \newline
50. बिभ्र॑ती ज॒राम् ज॒राम् बिभ्र॑ती॒ बिभ्र॑ती ज॒राम् । \newline
51. ज॒रा म॑जरे अजरे ज॒राम् ज॒रा म॑जरे । \newline
52. अ॒ज॒र॒ उ॒ष॒ उ॒षो॒ अ॒ज॒रे॒ अ॒ज॒र॒ उ॒षः॒ । \newline
53. उ॒ष॒ ओष॑ उष॒ आ । \newline
54. आ ऽगा॑ अगा॒ आ ऽगाः᳚ । \newline
55. अ॒गा॒ इत्य॑गाः । \newline
56. ऋ॒तू॒नाम् पत्नी॒ पत्न्यृ॑तू॒ना मृ॑तू॒नाम् पत्नी᳚ । \newline
57. पत्नी᳚ प्रथ॒मा प्र॑थ॒मा पत्नी॒ पत्नी᳚ प्रथ॒मा । \newline
58. प्र॒थ॒मेय मि॒यम् प्र॑थ॒मा प्र॑थ॒मेयम् । \newline
59. इ॒य मेय मि॒य मा । \newline
60. आ ऽगा॑ दगा॒ दा ऽगा᳚त् । \newline
61. अ॒गा॒ दह्ना॒ मह्ना॑ मगादगा॒ दह्ना᳚म् । \newline
62. अह्ना᳚न् ने॒त्री ने॒त्र्यह्ना॒ मह्ना᳚न् ने॒त्री । \newline
63. ने॒त्री ज॑नि॒त्री ज॑नि॒त्री ने॒त्री ने॒त्री ज॑नि॒त्री । \newline
64. ज॒नि॒त्री प्र॒जाना᳚म् प्र॒जाना᳚म् जनि॒त्री ज॑नि॒त्री प्र॒जाना᳚म् । \newline
65. प्र॒जाना॒मिति॑ प्र - जाना᳚म् । \newline
66. एका॑ स॒ती स॒त्येकैका॑ स॒ती । \newline
67. स॒ती ब॑हु॒धा ब॑हु॒धा स॒ती स॒ती ब॑हु॒धा । \newline
68. ब॒हु॒ धोष॑ उषो बहु॒धा ब॑हु॒ धोषः॑ । \newline
69. ब॒हु॒धेति॑ बहु - धा । \newline
70. उ॒षो॒ वि व्यु॑ष उषो॒ वि । \newline
71. व्यु॑च्छ स्युच्छसि॒ वि व्यु॑च्छसि । \newline
72. उ॒च्छ॒ स्यजी॒र्णा ऽजी᳚र्णो च्छ स्युच्छ॒ स्यजी᳚र्णा । \newline
73. अजी᳚र्णा॒ त्वम् त्व मजी॒र्णा ऽजी᳚र्णा॒ त्वम् । \newline
74. त्वम् ज॑रयसि जरयसि॒ त्वम् त्वम् ज॑रयसि । \newline
75. ज॒र॒य॒सि॒ सर्वꣳ॒॒ सर्व॑म् जरयसि जरयसि॒ सर्व᳚म् । \newline
76. सर्व॑ म॒न्य द॒न्यथ् सर्वꣳ॒॒ सर्व॑ म॒न्यत् । \newline
77. अ॒न्यदित्य॒न्यत् । \newline

\textbf{Ghana Paata } \newline

1. ऋ॒तस्य॒ गर्भो॒ गर्भ॑ ऋ॒तस्य॒ र्‌तस्य॒ गर्भः॑ प्रथ॒मा प्र॑थ॒मा गर्भ॑ ऋ॒तस्य॒ र्‌तस्य॒ गर्भः॑ प्रथ॒मा । \newline
2. गर्भः॑ प्रथ॒मा प्र॑थ॒मा गर्भो॒ गर्भः॑ प्रथ॒मा व्यू॒षुषी᳚ व्यू॒षुषी᳚ प्रथ॒मा गर्भो॒ गर्भः॑ प्रथ॒मा व्यू॒षुषी᳚ । \newline
3. प्र॒थ॒मा व्यू॒षुषी᳚ व्यू॒षुषी᳚ प्रथ॒मा प्र॑थ॒मा व्यू॒षुष्य॒पा म॒पां ॅव्यू॒षुषी᳚ प्रथ॒मा प्र॑थ॒मा व्यू॒षुष्य॒पाम् । \newline
4. व्यू॒षुष्य॒पा म॒पां ॅव्यू॒षुषी᳚ व्यू॒षुष्य॒पा मेकैका॒ ऽपां ॅव्यू॒षुषी᳚ व्यू॒षुष्य॒पा मेका᳚ । \newline
5. व्यू॒षुषीति॑ वि - ऊ॒षुषी᳚ । \newline
6. अ॒पा मेकैका॒ ऽपा म॒पा मेका॑ महि॒मान॑म् महि॒मान॒ मेका॒ ऽपा म॒पा मेका॑ महि॒मान᳚म् । \newline
7. एका॑ महि॒मान॑म् महि॒मान॒ मेकैका॑ महि॒मान॑म् बिभर्ति बिभर्ति महि॒मान॒ मेकैका॑ महि॒मान॑म् बिभर्ति । \newline
8. म॒हि॒मान॑म् बिभर्ति बिभर्ति महि॒मान॑म् महि॒मान॑म् बिभर्ति । \newline
9. बि॒भ॒र्तीति॑ बिभर्ति । \newline
10. सूर्य॒ स्यैकैका॒ सूर्य॑स्य॒ सूर्य॒स्यैका॒ चर॑ति॒ चर॒त्येका॒ सूर्य॑स्य॒ सूर्य॒ स्यैका॒ चर॑ति । \newline
11. एका॒ चर॑ति॒ चर॒ त्येकैका॒ चर॑ति निष्कृ॒तेषु॑ निष्कृ॒तेषु॒ चर॒ त्येकैका॒ चर॑ति निष्कृ॒तेषु॑ । \newline
12. चर॑ति निष्कृ॒तेषु॑ निष्कृ॒तेषु॒ चर॑ति॒ चर॑ति निष्कृ॒तेषु॑ घ॒र्मस्य॑ घ॒र्मस्य॑ निष्कृ॒तेषु॒ चर॑ति॒ चर॑ति निष्कृ॒तेषु॑ घ॒र्मस्य॑ । \newline
13. नि॒ष्कृ॒तेषु॑ घ॒र्मस्य॑ घ॒र्मस्य॑ निष्कृ॒तेषु॑ निष्कृ॒तेषु॑ घ॒र्मस्यैकैका॑ घ॒र्मस्य॑ निष्कृ॒तेषु॑ निष्कृ॒तेषु॑ घ॒र्मस्यैका᳚ । \newline
14. नि॒ष्कृ॒तेष्विति॑ निः - कृ॒तेषु॑ । \newline
15. घ॒र्म स्यैकैका॑ घ॒र्मस्य॑ घ॒र्म स्यैका॑ सवि॒ता स॑वि॒ तैका॑ घ॒र्मस्य॑ घ॒र्म स्यैका॑ सवि॒ता । \newline
16. एका॑ सवि॒ता स॑वि॒ तैकैका॑ सवि॒ तैका॒ मेकाꣳ॑ सवि॒ तैकैका॑ सवि॒तैका᳚म् । \newline
17. स॒वि॒ तैका॒ मेकाꣳ॑ सवि॒ता स॑वि॒ तैका॒न्नि न्येकाꣳ॑ सवि॒ता स॑वि॒ तैका॒न्नि । \newline
18. एका॒न्नि न्येका॒ मेका॒न्नि य॑च्छति यच्छति॒ न्येका॒ मेका॒न्नि य॑च्छति । \newline
19. नि य॑च्छति यच्छति॒ नि नि य॑च्छति । \newline
20. य॒च्छ॒तीति॑ यच्छति । \newline
21. या प्र॑थ॒मा प्र॑थ॒मा या या प्र॑थ॒मा व्यौच्छ॒द् व्यौच्छ॑त् प्रथ॒मा या या प्र॑थ॒मा व्यौच्छ॑त् । \newline
22. प्र॒थ॒मा व्यौच्छ॒द् व्यौच्छ॑त् प्रथ॒मा प्र॑थ॒मा व्यौच्छ॒थ् सा सा व्यौच्छ॑त् प्रथ॒मा प्र॑थ॒मा व्यौच्छ॒थ् सा । \newline
23. व्यौच्छ॒थ् सा सा व्यौच्छ॒द् व्यौच्छ॒थ् सा धे॒नुर् धे॒नुः सा व्यौच्छ॒द् व्यौच्छ॒थ् सा धे॒नुः । \newline
24. व्यौच्छ॒दिति॑ वि - औच्छ॑त् । \newline
25. सा धे॒नुर् धे॒नुः सा सा धे॒नु र॑भव दभवद् धे॒नुः सा सा धे॒नु र॑भवत् । \newline
26. धे॒नु र॑भव दभवद् धे॒नुर् धे॒नु र॑भवद् य॒मे य॒मे अ॑भवद् धे॒नुर् धे॒नु र॑भवद् य॒मे । \newline
27. अ॒भ॒व॒द् य॒मे य॒मे अ॑भव दभवद् य॒मे । \newline
28. य॒म इति॑ य॒मे । \newline
29. सा नो॑ नः॒ सा सा नः॒ पय॑स्वती॒ पय॑स्वती नः॒ सा सा नः॒ पय॑स्वती । \newline
30. नः॒ पय॑स्वती॒ पय॑स्वती नो नः॒ पय॑स्वती धुक्ष्व धुक्ष्व॒ पय॑स्वती नो नः॒ पय॑स्वती धुक्ष्व । \newline
31. पय॑स्वती धुक्ष्व धुक्ष्व॒ पय॑स्वती॒ पय॑स्वती धु॒क्ष्वोत् त॑रामुत्तरा॒ मुत्त॑रामुत्तराम् धुक्ष्व॒ पय॑स्वती॒ पय॑स्वती धु॒क्ष्वोत् त॑रामुत्तराम् । \newline
32. धु॒क्ष्वोत् त॑रामुत्तरा॒ मुत्त॑रामुत्तराम् धुक्ष्व धु॒क्ष्वोत् त॑रामुत्तराꣳ॒॒ समाꣳ॒॒ समा॒ मुत्त॑रामुत्तराम् धुक्ष्व धु॒क्ष्वोत् त॑रामुत्तराꣳ॒॒ समा᳚म् । \newline
33. उत्त॑रामुत्तराꣳ॒॒ समाꣳ॒॒ समा॒ मुत्त॑रामुत्तरा॒ मुत्त॑रामुत्तराꣳ॒॒ समा᳚म् । \newline
34. उत्त॑रामुत्तरा॒मित्युत्त॑रां - उ॒त्त॒रा॒म् । \newline
35. समा॒मिति॒ समा᳚म् । \newline
36. शु॒क्रर्.ष॑भा॒ नभ॑सा॒ नभ॑सा शु॒क्रर्.ष॑भा शु॒क्रर्.ष॑भा॒ नभ॑सा॒ ज्योति॑षा॒ ज्योति॑षा॒ नभ॑सा शु॒क्रर्.ष॑भा शु॒क्रर्.ष॑भा॒ नभ॑सा॒ ज्योति॑षा । \newline
37. शु॒क्रर्.ष॒भेति॑ शु॒क्र - ऋ॒ष॒भा॒ । \newline
38. नभ॑सा॒ ज्योति॑षा॒ ज्योति॑षा॒ नभ॑सा॒ नभ॑सा॒ ज्योति॒षा ऽऽज्योति॑षा॒ नभ॑सा॒ नभ॑सा॒ ज्योति॒षा । \newline
39. ज्योति॒षा ऽऽज्योति॑षा॒ ज्योति॒षा ऽगा॑ दगा॒ दा ज्योति॑षा॒ ज्योति॒षा ऽगा᳚त् । \newline
40. आ ऽगा॑ दगा॒ दा ऽगा᳚द् वि॒श्वरू॑पा वि॒श्वरू॑पा ऽगा॒दा ऽगा᳚द् वि॒श्वरू॑पा । \newline
41. अ॒गा॒द् वि॒श्वरू॑पा वि॒श्वरू॑पा ऽगा दगाद् वि॒श्वरू॑पा शब॒लीः श॑ब॒लीर् वि॒श्वरू॑पा ऽगादगाद् वि॒श्वरू॑पा शब॒लीः । \newline
42. वि॒श्वरू॑पा शब॒लीः श॑ब॒लीर् वि॒श्वरू॑पा वि॒श्वरू॑पा शब॒ली र॒ग्निके॑तु र॒ग्निके॑तुः शब॒लीर् वि॒श्वरू॑पा वि॒श्वरू॑पा शब॒ली र॒ग्निके॑तुः । \newline
43. वि॒श्वरू॒पेति॑ वि॒श्व - रू॒पा॒ । \newline
44. श॒ब॒ली र॒ग्निके॑तु र॒ग्निके॑तुः शब॒लीः श॑ब॒ली र॒ग्निके॑तुः । \newline
45. अ॒ग्निके॑तु॒रित्य॒ग्नि - के॒तुः॒ । \newline
46. स॒मा॒न मर्थ॒ मर्थꣳ॑ समा॒नꣳ स॑मा॒न मर्थꣳ॑ स्वप॒स्यमा॑ना स्वप॒स्यमा॒ना ऽर्थꣳ॑ समा॒नꣳ स॑मा॒न मर्थꣳ॑ स्वप॒स्यमा॑ना । \newline
47. अर्थꣳ॑ स्वप॒स्यमा॑ना स्वप॒स्यमा॒ना ऽर्थ॒ मर्थꣳ॑ स्वप॒स्यमा॑ना॒ बिभ्र॑ती॒ बिभ्र॑ती स्वप॒स्यमा॒ना ऽर्थ॒ मर्थꣳ॑ स्वप॒स्यमा॑ना॒ बिभ्र॑ती । \newline
48. स्व॒प॒स्यमा॑ना॒ बिभ्र॑ती॒ बिभ्र॑ती स्वप॒स्यमा॑ना स्वप॒स्यमा॑ना॒ बिभ्र॑ती ज॒राम् ज॒राम् बिभ्र॑ती स्वप॒स्यमा॑ना स्वप॒स्यमा॑ना॒ बिभ्र॑ती ज॒राम् । \newline
49. स्व॒प॒स्यमा॒नेति॑ सु - अ॒प॒स्यमा॑ना । \newline
50. बिभ्र॑ती ज॒राम् ज॒राम् बिभ्र॑ती॒ बिभ्र॑ती ज॒रा म॑जरे अजरे ज॒राम् बिभ्र॑ती॒ बिभ्र॑ती ज॒रा म॑जरे । \newline
51. ज॒रा म॑जरे अजरे ज॒राम् ज॒रा म॑जर उष उषो अजरे ज॒राम् ज॒रा म॑जर उषः । \newline
52. अ॒ज॒र॒ उ॒ष॒ उ॒षो॒ अ॒ज॒रे॒ अ॒ज॒र॒ उ॒ष॒ ओषो॑ अजरे अजर उष॒ आ । \newline
53. उ॒ष॒ ओष॑ उष॒ आ ऽगा॑ अगा॒ ओष॑ उष॒ आ ऽगाः᳚ । \newline
54. आ ऽगा॑ अगा॒ आ ऽगाः᳚ । \newline
55. अ॒गा॒ इत्य॑गाः । \newline
56. ऋ॒तू॒नाम् पत्नी॒ पत्न्यृ॑तू॒ना मृ॑तू॒नाम् पत्नी᳚ प्रथ॒मा प्र॑थ॒मा पत्न्यृ॑तू॒ना मृ॑तू॒नाम् पत्नी᳚ प्रथ॒मा । \newline
57. पत्नी᳚ प्रथ॒मा प्र॑थ॒मा पत्नी॒ पत्नी᳚ प्रथ॒मेय मि॒यम् प्र॑थ॒मा पत्नी॒ पत्नी᳚ प्रथ॒मेयम् । \newline
58. प्र॒थ॒मेय मि॒यम् प्र॑थ॒मा प्र॑थ॒मेय मेयम् प्र॑थ॒मा प्र॑थ॒मेय मा । \newline
59. इ॒य मेय मि॒य मा ऽगा॑ दगा॒ देय मि॒य मा ऽगा᳚त् । \newline
60. आ ऽगा॑ दगा॒दा ऽगा॒दह्ना॒ मह्ना॑ मगा॒दा ऽगा॒दह्ना᳚म् । \newline
61. अ॒गा॒ दह्ना॒ मह्ना॑ मगा दगा॒दह्ना᳚म् ने॒त्री ने॒त्र्यह्ना॑ मगादगा॒ दह्ना᳚म् ने॒त्री । \newline
62. अह्ना᳚म् ने॒त्री ने॒त्र्यह्ना॒ मह्ना᳚म् ने॒त्री ज॑नि॒त्री ज॑नि॒त्री ने॒त्र्यह्ना॒ मह्ना᳚म् ने॒त्री ज॑नि॒त्री । \newline
63. ने॒त्री ज॑नि॒त्री ज॑नि॒त्री ने॒त्री ने॒त्री ज॑नि॒त्री प्र॒जाना᳚म् प्र॒जाना᳚म् जनि॒त्री ने॒त्री ने॒त्री ज॑नि॒त्री प्र॒जाना᳚म् । \newline
64. ज॒नि॒त्री प्र॒जाना᳚म् प्र॒जाना᳚म् जनि॒त्री ज॑नि॒त्री प्र॒जाना᳚म् । \newline
65. प्र॒जाना॒मिति॑ प्र - जाना᳚म् । \newline
66. एका॑ स॒ती स॒त्ये कैका॑ स॒ती ब॑हु॒धा ब॑हु॒धा स॒त्ये कैका॑ स॒ती ब॑हु॒धा । \newline
67. स॒ती ब॑हु॒धा ब॑हु॒धा स॒ती स॒ती ब॑हु॒धोष॑ उषो बहु॒धा स॒ती स॒ती ब॑हु॒धोषः॑ । \newline
68. ब॒हु॒धोष॑ उषो बहु॒धा ब॑हु॒धोषो॒ वि व्यु॑षो बहु॒धा ब॑हु॒धोषो॒ वि । \newline
69. ब॒हु॒धेति॑ बहु - धा । \newline
70. उ॒षो॒ वि व्यु॑ष उषो॒ व्यु॑च्छ स्युच्छसि॒ व्यु॑ष उषो॒ व्यु॑च्छसि । \newline
71. व्यु॑च्छ स्युच्छसि॒ वि व्यु॑च्छ॒ स्यजी॒र्णा ऽजी᳚र्णोच्छसि॒ वि व्यु॑च्छ॒ स्यजी᳚र्णा । \newline
72. उ॒च्छ॒ स्यजी॒र्णा ऽजी᳚र्णोच्छ स्युच्छ॒ स्यजी᳚र्णा॒ त्वम् त्व मजी᳚र्णोच्छ स्युच्छ॒ स्यजी᳚र्णा॒ त्वम् । \newline
73. अजी᳚र्णा॒ त्वम् त्व मजी॒र्णा ऽजी᳚र्णा॒ त्वम् ज॑रयसि जरयसि॒ त्व मजी॒र्णा ऽजी᳚र्णा॒ त्वम् ज॑रयसि । \newline
74. त्वम् ज॑रयसि जरयसि॒ त्वम् त्वम् ज॑रयसि॒ सर्वꣳ॒॒ सर्व॑म् जरयसि॒ त्वम् त्वम् ज॑रयसि॒ सर्व᳚म् । \newline
75. ज॒र॒य॒सि॒ सर्वꣳ॒॒ सर्व॑म् जरयसि जरयसि॒ सर्व॑ म॒न्य द॒न्यथ् सर्व॑म् जरयसि जरयसि॒ सर्व॑ म॒न्यत् । \newline
76. सर्व॑ म॒न्य द॒न्यथ् सर्वꣳ॒॒ सर्व॑ म॒न्यत् । \newline
77. अ॒न्यदित्य॒न्यत् । \newline
\pagebreak
\markright{ TS 4.3.12.1  \hfill https://www.vedavms.in \hfill}

\section{ TS 4.3.12.1 }

\textbf{TS 4.3.12.1 } \newline
\textbf{Samhita Paata} \newline

अग्ने॑ जा॒तान् प्रणु॑दा नः स॒पत्ना॒न् प्रत्यजा॑ताञ्जातवेदो नुदस्व । अ॒स्मे दी॑दिहि सु॒मना॒ अहे॑ड॒न् तव॑ स्याꣳ॒॒ शर्म॑न् त्रि॒वरू॑थ उ॒द्भित् ॥ सह॑सा जा॒तान् प्रणु॑दानः स॒पत्ना॒न् प्रत्यजा॑ताञ्जातवेदो नुदस्व । अधि॑ नो ब्रूहि सुमन॒स्यमा॑नो व॒यꣳ स्या॑म॒ प्रणु॑दा नः स॒पत्नान्॑ ॥ च॒तु॒श्च॒त्वा॒रिꣳ॒॒शः स्तोमो॒ वर्चो॒ द्रवि॑णꣳ षोड॒शः स्तोम॒ ओजो॒ द्रवि॑णं पृथि॒व्याः पुरी॑षम॒स्य- [  ] \newline

\textbf{Pada Paata} \newline

अग्ने᳚ । जा॒तान् । प्रेति॑ । नु॒द॒ । नः॒ । स॒पत्नान्॑ । प्रतीति॑ । अजा॑तान् । जा॒त॒वे॒द॒ इति॑ जात - वे॒दः॒ । नु॒द॒स्व॒ ॥ अ॒स्मे इति॑ । दी॒दि॒हि॒ । सु॒मना॒ इति॑ सु - मनाः᳚ । अहे॑डन्न् । तव॑ । स्या॒म् । शर्मन्न्॑ । त्रि॒वरू॑थ॒ इति॑ त्रि-वरू॑थः । उ॒द्भिदित्यु॑त् - भित् ॥ सह॑सा । जा॒तान् । प्रेति॑ । नु॒द॒ । नः॒ । स॒पत्नान्॑ । प्रतीति॑ । अजा॑तान् । जा॒त॒वे॒द॒ इति॑ जात - वे॒दः॒ । नु॒द॒स्व॒ ॥ अधीति॑ । नः॒ । ब्रू॒हि॒ । सु॒म॒न॒स्यमा॑न॒ इति॑ सु - म॒न॒स्यमा॑नः । व॒यम् । स्या॒म॒ । प्रेति॑ । नु॒द॒ । नः॒ । स॒पत्नान्॑ ॥ च॒तु॒श्च॒त्वा॒रिꣳ॒॒श इति॑ चतुः - च॒त्वा॒रिꣳ॒॒शः । स्तोमः॑ । वर्चः॑ । द्रवि॑णम् । षो॒ड॒शः । स्तोमः॑ । ओजः॑ । द्रवि॑णम् । पृ॒थि॒व्याः । पुरी॑षम् । अ॒सि॒ ।  \newline


\textbf{Krama Paata} \newline

अग्ने॑ जा॒तान् । जा॒तान् प्र । प्र णु॑द । नु॒दा॒ नः॒ । नः॒ स॒पत्नान्॑ । स॒पत्ना॒न् प्रति॑ । प्रत्यजा॑तान् । अजा॑तान् जातवेदः । जा॒त॒वे॒दो॒ नु॒द॒स्व॒ । जा॒त॒वे॒द॒ इति॑ जात - वे॒दः॒ । नु॒द॒स्वेति॑ नुदस्व ॥ अ॒स्मे दी॑दिहि । अ॒स्मे इत्य॒स्मे । दी॒दि॒हि॒ सु॒मनाः᳚ । सु॒मना॒ अहे॑डन्न् । सु॒मना॒ इति॑ सु - मनाः᳚ । अहे॑ड॒न् तव॑ । तव॑ स्याम् । स्याꣳ॒॒ शर्मन्न्॑ । 
शर्म॑न् त्रि॒वरू॑थः । त्रि॒वरू॑थ उ॒द्भित् । त्रि॒वरू॑थ॒ इति॑ त्रि - वरू॑थः । उ॒द्भिदित्यु॑त् - भित् ॥ सह॑सा जा॒तान् । जा॒तान् प्र । 
प्र णु॑द । नु॒दा॒ नः॒ । नः॒ स॒पत्नान्॑ । स॒पत्ना॒न् प्रति॑ । प्रत्यजा॑तान् । अजा॑तान् जातवेदः । जा॒त॒वे॒दो॒ नु॒द॒स्व॒ । जा॒त॒वे॒द॒ इति॑ जात - वे॒दः॒ । नु॒द॒स्वेति॑ नुदस्व ॥ अधि॑ नः । नो॒ ब्रू॒हि॒ । ब्रू॒हि॒ सु॒म॒न॒स्यमा॑नः । सु॒म॒न॒स्यमा॑नो व॒यम् । सु॒म॒न॒स्यमा॑न॒ इति॑ सु - म॒न॒स्यमा॑नः । व॒यꣳ स्या॑म । स्या॒म॒ प्र । प्र णु॑द । नु॒दा॒ नः॒ । नः॒ स॒पत्नान्॑ । स॒पत्ना॒निति॑ स॒पत्नान्॑ ॥ च॒तु॒श्च॒त्वा॒रिꣳ॒॒शः स्तोमः॑ । च॒तु॒श्च॒त्वा॒रिꣳ॒॒श इति॑ चतुः - च॒त्वा॒रिꣳ॒॒शः । स्तोमो॒ वर्चः॑ । वर्चो॒ द्रवि॑णम् । द्रवि॑णꣳ षोड॒शः । षो॒ड॒शः स्तोमः॑ । स्तोम॒ ओजः॑ । ओजो॒ द्रवि॑णम् । द्रवि॑णम् पृथि॒व्याः । पृ॒थि॒व्याः पुरी॑षम् । पुरी॑षमसि । अ॒स्यफ्सः॑ \newline

\textbf{Jatai Paata} \newline

1. अग्ने॑ जा॒तान् जा॒ता-नग्ने ऽग्ने॑ जा॒तान् । \newline
2. जा॒तान् प्र प्र जा॒तान् जा॒तान् प्र । \newline
3. प्र णु॑द नुद॒ प्र प्र णु॑द । \newline
4. नु॒दा॒ नो॒ नो॒ नु॒द॒ नु॒दा॒ नः॒ । \newline
5. नः॒ स॒पत्ना᳚न् थ्स॒पत्ना᳚न् नो नः स॒पत्नान्॑ । \newline
6. स॒पत्ना॒न् प्रति॒ प्रति॑ स॒पत्ना᳚न् थ्स॒पत्ना॒न् प्रति॑ । \newline
7. प्रत्यजा॑ता॒-नजा॑ता॒न् प्रति॒ प्रत्यजा॑तान् । \newline
8. अजा॑तान् जातवेदो जातवे॒दो ऽजा॑ता॒-नजा॑तान् जातवेदः । \newline
9. जा॒त॒वे॒दो॒ नु॒द॒स्व॒ नु॒द॒स्व॒ जा॒त॒वे॒दो॒ जा॒त॒वे॒दो॒ नु॒द॒स्व॒ । \newline
10. जा॒त॒वे॒द॒ इति॑ जात - वे॒दः॒ । \newline
11. नु॒द॒स्वेति॑ नुदस्व । \newline
12. अ॒स्मे दी॑दिहि दीदि ह्य॒स्मे अ॒स्मे दी॑दिहि । \newline
13. अ॒स्मे इत्य॒स्मे । \newline
14. दी॒दि॒हि॒ सु॒मनाः᳚ सु॒मना॑ दीदिहि दीदिहि सु॒मनाः᳚ । \newline
15. सु॒मना॒ अहे॑ड॒न् नहे॑डन् थ्सु॒मनाः᳚ सु॒मना॒ अहे॑डन्न् । \newline
16. सु॒मना॒ इति॑ सु - मनाः᳚ । \newline
17. अहे॑ड॒न् तव॒ तवा हे॑ड॒न् नहे॑ड॒न् तव॑ । \newline
18. तव॑ स्याꣳ स्या॒म् तव॒ तव॑ स्याम् । \newline
19. स्याꣳ॒॒ शर्म॒ञ् छर्मन्᳚ थ्स्याꣳ स्याꣳ॒॒ शर्मन्न्॑ । \newline
20. शर्म॑न् त्रि॒वरू॑थ स्त्रि॒वरू॑थः॒ शर्म॒ञ् छर्म॑न् त्रि॒वरू॑थः । \newline
21. त्रि॒वरू॑थ उ॒द्भि दु॒द्भित् त्रि॒वरू॑थ स्त्रि॒वरू॑थ उ॒द्भित् । \newline
22. त्रि॒वरू॑थ॒ इति॑ त्रि - वरू॑थः । \newline
23. उ॒द्भिदित्यु॑त् - भित् । \newline
24. सह॑सा जा॒तान् जा॒तान् थ्सह॑सा॒ सह॑सा जा॒तान् । \newline
25. जा॒तान् प्र प्र जा॒तान् जा॒तान् प्र । \newline
26. प्र णु॑द नुद॒ प्र प्र णु॑द । \newline
27. नु॒दा॒ नो॒ नो॒ नु॒द॒ नु॒दा॒ नः॒ । \newline
28. नः॒ स॒पत्ना᳚न् थ्स॒पत्ना᳚न् नो नः स॒पत्नान्॑ । \newline
29. स॒पत्ना॒न् प्रति॒ प्रति॑ स॒पत्ना᳚न् थ्स॒पत्ना॒न् प्रति॑ । \newline
30. प्रत्यजा॑ता॒-नजा॑ता॒न् प्रति॒ प्रत्यजा॑तान् । \newline
31. अजा॑तान् जातवेदो जातवे॒दो ऽजा॑ता॒-नजा॑तान् जातवेदः । \newline
32. जा॒त॒वे॒दो॒ नु॒द॒स्व॒ नु॒द॒स्व॒ जा॒त॒वे॒दो॒ जा॒त॒वे॒दो॒ नु॒द॒स्व॒ । \newline
33. जा॒त॒वे॒द॒ इति॑ जात - वे॒दः॒ । \newline
34. नु॒द॒स्वेति॑ नुदस्व । \newline
35. अधि॑ नो नो॒ ऽध्यधि॑ नः । \newline
36. नो॒ ब्रू॒हि॒ ब्रू॒हि॒ नो॒ नो॒ ब्रू॒हि॒ । \newline
37. ब्रू॒हि॒ सु॒म॒न॒स्यमा॑नः सुमन॒स्यमा॑नो ब्रूहि ब्रूहि सुमन॒स्यमा॑नः । \newline
38. सु॒म॒न॒स्यमा॑नो व॒यं ॅव॒यꣳ सु॑मन॒स्यमा॑नः सुमन॒स्यमा॑नो व॒यम् । \newline
39. सु॒म॒न॒स्यमा॑न॒ इति॑ सु - म॒न॒स्यमा॑नः । \newline
40. व॒यꣳ स्या॑म स्याम व॒यं ॅव॒यꣳ स्या॑म । \newline
41. स्या॒म॒ प्र प्र स्या॑म स्याम॒ प्र । \newline
42. प्र णु॑द नुद॒ प्र प्र णु॑द । \newline
43. नु॒दा॒ नो॒ नो॒ नु॒द॒ नु॒दा॒ नः॒ । \newline
44. नः॒ स॒पत्ना᳚न् थ्स॒पत्ना᳚न् नो नः स॒पत्नान्॑ । \newline
45. स॒पत्ना॒निति॑ स॒पत्नान्॑ । \newline
46. च॒तु॒श्च॒त्वा॒रिꣳ॒॒शः स्तोमः॒ स्तोम॑ श्चतुश्चत्वारिꣳ॒॒श श्च॑तुश्चत्वारिꣳ॒॒शः स्तोमः॑ । \newline
47. च॒तु॒श्च॒त्वा॒रिꣳ॒॒श इति॑ चतुः - च॒त्वा॒रिꣳ॒॒शः । \newline
48. स्तोमो॒ वर्चो॒ वर्चः॒ स्तोमः॒ स्तोमो॒ वर्चः॑ । \newline
49. वर्चो॒ द्रवि॑ण॒म् द्रवि॑णं॒ ॅवर्चो॒ वर्चो॒ द्रवि॑णम् । \newline
50. द्रवि॑णꣳ षोड॒श ष्षो॑ड॒शो द्रवि॑ण॒म् द्रवि॑णꣳ षोड॒शः । \newline
51. षो॒ड॒शः स्तोमः॒ स्तोम॑ ष्षोड॒श ष्षो॑ड॒शः स्तोमः॑ । \newline
52. स्तोम॒ ओज॒ ओजः॒ स्तोमः॒ स्तोम॒ ओजः॑ । \newline
53. ओजो॒ द्रवि॑ण॒म् द्रवि॑ण॒ मोज॒ ओजो॒ द्रवि॑णम् । \newline
54. द्रवि॑णम् पृथि॒व्याः पृ॑थि॒व्या द्रवि॑ण॒म् द्रवि॑णम् पृथि॒व्याः । \newline
55. पृ॒थि॒व्याः पुरी॑ष॒म् पुरी॑षम् पृथि॒व्याः पृ॑थि॒व्याः पुरी॑षम् । \newline
56. पुरी॑ष मस्यसि॒ पुरी॑ष॒म् पुरी॑ष मसि । \newline
57. अ॒स्यफ्सो ऽफ्सो᳚ ऽस्य॒ स्यफ्सः॑ । \newline

\textbf{Ghana Paata } \newline

1. अग्ने॑ जा॒तान् जा॒ता नग्ने ऽग्ने॑ जा॒तान् प्र प्र जा॒ता नग्ने ऽग्ने॑ जा॒तान् प्र । \newline
2. जा॒तान् प्र प्र जा॒तान् जा॒तान् प्र णु॑दा नुद॒ प्र जा॒तान् जा॒तान् प्र णु॑दा । \newline
3. प्र णु॑दा नुद॒ प्र प्र णु॑दा नो नो नुद॒ प्र प्र णु॑दा नः । \newline
4. नु॒दा॒ नो॒ नो॒ नु॒द॒ नु॒दा॒ नः॒ स॒पत्ना᳚न् थ्स॒पत्ना᳚न् नो नुद नुदा नः स॒पत्नान्॑ । \newline
5. नः॒ स॒पत्ना᳚न् थ्स॒पत्ना᳚न् नो नः स॒पत्ना॒न् प्रति॒ प्रति॑ स॒पत्ना᳚न् नो नः स॒पत्ना॒न् प्रति॑ । \newline
6. स॒पत्ना॒न् प्रति॒ प्रति॑ स॒पत्ना᳚न् थ्स॒पत्ना॒न् प्रत्यजा॑ता॒ नजा॑ता॒न् प्रति॑ स॒पत्ना᳚न् थ्स॒पत्ना॒न् प्रत्यजा॑तान् । \newline
7. प्रत्यजा॑ता॒ नजा॑ता॒न् प्रति॒ प्रत्यजा॑तान् जातवेदो जातवे॒दो ऽजा॑ता॒न् प्रति॒ प्रत्यजा॑तान् जातवेदः । \newline
8. अजा॑तान् जातवेदो जातवे॒दो ऽजा॑ता॒ नजा॑तान् जातवेदो नुदस्व नुदस्व जातवे॒दो ऽजा॑ता॒ नजा॑तान् जातवेदो नुदस्व । \newline
9. जा॒त॒वे॒दो॒ नु॒द॒स्व॒ नु॒द॒स्व॒ जा॒त॒वे॒दो॒ जा॒त॒वे॒दो॒ नु॒द॒स्व॒ । \newline
10. जा॒त॒वे॒द॒ इति॑ जात - वे॒दः॒ । \newline
11. नु॒द॒स्वेति॑ नुदस्व । \newline
12. अ॒स्मे दी॑दिहि दीदिह्य॒स्मे अ॒स्मे दी॑दिहि सु॒मनाः᳚ सु॒मना॑ दीदिह्य॒स्मे अ॒स्मे दी॑दिहि सु॒मनाः᳚ । \newline
13. अ॒स्मे इत्य॒स्मे । \newline
14. दी॒दि॒हि॒ सु॒मनाः᳚ सु॒मना॑ दीदिहि दीदिहि सु॒मना॒ अहे॑ड॒न् नहे॑डन् थ्सु॒मना॑ दीदिहि दीदिहि सु॒मना॒ अहे॑डन्न् । \newline
15. सु॒मना॒ अहे॑ड॒न् नहे॑डन् थ्सु॒मनाः᳚ सु॒मना॒ अहे॑ड॒न् तव॒ तवाहे॑डन् थ्सु॒मनाः᳚ सु॒मना॒ अहे॑ड॒न् तव॑ । \newline
16. सु॒मना॒ इति॑ सु - मनाः᳚ । \newline
17. अहे॑ड॒न् तव॒ तवाहे॑ड॒न् नहे॑ड॒न् तव॑ स्याꣳ स्या॒म् तवाहे॑ड॒न् नहे॑ड॒न् तव॑ स्याम् । \newline
18. तव॑ स्याꣳ स्या॒म् तव॒ तव॑ स्याꣳ॒॒ शर्म॒ञ् छर्मन्᳚ थ्स्या॒म् तव॒ तव॑ स्याꣳ॒॒ शर्मन्न्॑ । \newline
19. स्याꣳ॒॒ शर्म॒ञ् छर्मन्᳚ थ्स्याꣳ स्याꣳ॒॒ शर्म॑न् त्रि॒वरू॑थ स्त्रि॒वरू॑थः॒ शर्मन्᳚ थ्स्याꣳ स्याꣳ॒॒ शर्म॑न् त्रि॒वरू॑थः । \newline
20. शर्म॑न् त्रि॒वरू॑थ स्त्रि॒वरू॑थः॒ शर्म॒ञ् छर्म॑न् त्रि॒वरू॑थ उ॒द्भि दु॒द्भित् त्रि॒वरू॑थः॒ शर्म॒ञ् छर्म॑न् त्रि॒वरू॑थ उ॒द्भित् । \newline
21. त्रि॒वरू॑थ उ॒द्भि दु॒द्भित् त्रि॒वरू॑थ स्त्रि॒वरू॑थ उ॒द्भित् । \newline
22. त्रि॒वरू॑थ॒ इति॑ त्रि - वरू॑थः । \newline
23. उ॒द्भिदित्यु॑त् - भित् । \newline
24. सह॑सा जा॒तान् जा॒तान् थ्सह॑सा॒ सह॑सा जा॒तान् प्र प्र जा॒तान् थ्सह॑सा॒ सह॑सा जा॒तान् प्र । \newline
25. जा॒तान् प्र प्र जा॒तान् जा॒तान् प्र णु॑द नुद॒ प्र जा॒तान् जा॒तान् प्र णु॑द । \newline
26. प्र णु॑द नुद॒ प्र प्र णु॑दा नो नो नुद॒ प्र प्र णु॑दा नः । \newline
27. नु॒दा॒ नो॒ नो॒ नु॒द॒ नु॒दा॒ नः॒ स॒पत्ना᳚न् थ्स॒पत्ना᳚न् नो नुद नुदा नः स॒पत्नान्॑ । \newline
28. नः॒ स॒पत्ना᳚न् थ्स॒पत्ना᳚न् नो नः स॒पत्ना॒न् प्रति॒ प्रति॑ स॒पत्ना᳚न् नो नः स॒पत्ना॒न् प्रति॑ । \newline
29. स॒पत्ना॒न् प्रति॒ प्रति॑ स॒पत्ना᳚न् थ्स॒पत्ना॒न् प्रत्यजा॑ता॒ नजा॑ता॒न् प्रति॑ स॒पत्ना᳚न् थ्स॒पत्ना॒न् प्रत्यजा॑तान् । \newline
30. प्रत्यजा॑ता॒ नजा॑ता॒न् प्रति॒ प्रत्यजा॑तान् जातवेदो जातवे॒दो ऽजा॑ता॒न् प्रति॒ प्रत्यजा॑तान् जातवेदः । \newline
31. अजा॑तान् जातवेदो जातवे॒दो ऽजा॑ता॒ नजा॑तान् जातवेदो नुदस्व नुदस्व जातवे॒दो ऽजा॑ता॒ नजा॑तान् जातवेदो नुदस्व । \newline
32. जा॒त॒वे॒दो॒ नु॒द॒स्व॒ नु॒द॒स्व॒ जा॒त॒वे॒दो॒ जा॒त॒वे॒दो॒ नु॒द॒स्व॒ । \newline
33. जा॒त॒वे॒द॒ इति॑ जात - वे॒दः॒ । \newline
34. नु॒द॒स्वेति॑ नुदस्व । \newline
35. अधि॑ नो नो॒ ऽध्यधि॑ नो ब्रूहि ब्रूहि नो॒ ऽध्यधि॑ नो ब्रूहि । \newline
36. नो॒ ब्रू॒हि॒ ब्रू॒हि॒ नो॒ नो॒ ब्रू॒हि॒ सु॒म॒न॒स्यमा॑नः सुमन॒स्यमा॑नो ब्रूहि नो नो ब्रूहि सुमन॒स्यमा॑नः । \newline
37. ब्रू॒हि॒ सु॒म॒न॒स्यमा॑नः सुमन॒स्यमा॑नो ब्रूहि ब्रूहि सुमन॒स्यमा॑नो व॒यं ॅव॒यꣳ सु॑मन॒स्यमा॑नो ब्रूहि ब्रूहि सुमन॒स्यमा॑नो व॒यम् । \newline
38. सु॒म॒न॒स्यमा॑नो व॒यं ॅव॒यꣳ सु॑मन॒स्यमा॑नः सुमन॒स्यमा॑नो व॒यꣳ स्या॑म स्याम व॒यꣳ सु॑मन॒स्यमा॑नः सुमन॒स्यमा॑नो व॒यꣳ स्या॑म । \newline
39. सु॒म॒न॒स्यमा॑न॒ इति॑ सु - म॒न॒स्यमा॑नः । \newline
40. व॒यꣳ स्या॑म स्याम व॒यं ॅव॒यꣳ स्या॑म॒ प्र प्र स्या॑म व॒यं ॅव॒यꣳ स्या॑म॒ प्र । \newline
41. स्या॒म॒ प्र प्र स्या॑म स्याम॒ प्र णु॑द नुद॒ प्र स्या॑म स्याम॒ प्र णु॑द । \newline
42. प्र णु॑द नुद॒ प्र प्र णु॑दा नो नो नुद॒ प्र प्र णु॑दा नः । \newline
43. नु॒दा॒ नो॒ नो॒ नु॒द॒ नु॒दा॒ नः॒ स॒पत्ना᳚न् थ्स॒पत्ना᳚न् नो नुद नुदा नः स॒पत्नान्॑ । \newline
44. नः॒ स॒पत्ना᳚न् थ्स॒पत्ना᳚न् नो नः स॒पत्नान्॑ । \newline
45. स॒पत्ना॒निति॑ स॒पत्नान्॑ । \newline
46. च॒तु॒श्च॒त्वा॒रिꣳ॒॒शः स्तोमः॒ स्तोम॑ श्चतुश्चत्वारिꣳ॒॒श श्च॑तुश्चत्वारिꣳ॒॒शः स्तोमो॒ वर्चो॒ वर्चः॒ स्तोम॑ श्चतुश्चत्वारिꣳ॒॒श श्च॑तुश्चत्वारिꣳ॒॒शः स्तोमो॒ वर्चः॑ । \newline
47. च॒तु॒श्च॒त्वा॒रिꣳ॒॒श इति॑ चतुः - च॒त्वा॒रिꣳ॒॒शः । \newline
48. स्तोमो॒ वर्चो॒ वर्चः॒ स्तोमः॒ स्तोमो॒ वर्चो॒ द्रवि॑ण॒म् द्रवि॑णं॒ ॅवर्चः॒ स्तोमः॒ स्तोमो॒ वर्चो॒ द्रवि॑णम् । \newline
49. वर्चो॒ द्रवि॑ण॒म् द्रवि॑णं॒ ॅवर्चो॒ वर्चो॒ द्रवि॑णꣳ षोड॒श ष्षो॑ड॒शो द्रवि॑णं॒ ॅवर्चो॒ वर्चो॒ द्रवि॑णꣳ षोड॒शः । \newline
50. द्रवि॑णꣳ षोड॒श ष्षो॑ड॒शो द्रवि॑ण॒म् द्रवि॑णꣳ षोड॒शः स्तोमः॒ स्तोम॑ ष्षोड॒शो द्रवि॑ण॒म् द्रवि॑णꣳ षोड॒शः स्तोमः॑ । \newline
51. षो॒ड॒शः स्तोमः॒ स्तोम॑ ष्षोड॒श ष्षो॑ड॒शः स्तोम॒ ओज॒ ओजः॒ स्तोम॑ ष्षोड॒श ष्षो॑ड॒शः स्तोम॒ ओजः॑ । \newline
52. स्तोम॒ ओज॒ ओजः॒ स्तोमः॒ स्तोम॒ ओजो॒ द्रवि॑ण॒म् द्रवि॑ण॒ मोजः॒ स्तोमः॒ स्तोम॒ ओजो॒ द्रवि॑णम् । \newline
53. ओजो॒ द्रवि॑ण॒म् द्रवि॑ण॒ मोज॒ ओजो॒ द्रवि॑णम् पृथि॒व्याः पृ॑थि॒व्या द्रवि॑ण॒ मोज॒ ओजो॒ द्रवि॑णम् पृथि॒व्याः । \newline
54. द्रवि॑णम् पृथि॒व्याः पृ॑थि॒व्या द्रवि॑ण॒म् द्रवि॑णम् पृथि॒व्याः पुरी॑ष॒म् पुरी॑षम् पृथि॒व्या द्रवि॑ण॒म् द्रवि॑णम् पृथि॒व्याः पुरी॑षम् । \newline
55. पृ॒थि॒व्याः पुरी॑ष॒म् पुरी॑षम् पृथि॒व्याः पृ॑थि॒व्याः पुरी॑ष मस्यसि॒ पुरी॑षम् पृथि॒व्याः पृ॑थि॒व्याः पुरी॑ष मसि । \newline
56. पुरी॑ष मस्यसि॒ पुरी॑ष॒म् पुरी॑ष म॒स्यफ्सो ऽफ्सो॑ ऽसि॒ पुरी॑ष॒म् पुरी॑ष म॒स्यफ्सः॑ । \newline
57. अ॒स्यफ्सो ऽफ्सो᳚ ऽस्य॒ स्यफ्सो॒ नाम॒ नामाफ्सो᳚ ऽस्य॒ स्यफ्सो॒ नाम॑ । \newline
\pagebreak
\markright{ TS 4.3.12.2  \hfill https://www.vedavms.in \hfill}

\section{ TS 4.3.12.2 }

\textbf{TS 4.3.12.2 } \newline
\textbf{Samhita Paata} \newline

-फ्सो॒ नाम॑ । एव॒ श्छन्दो॒ वरि॑व॒ श्छन्दः॑ श॒भूं श्छन्दः॑ परि॒भू श्छन्द॑ आ॒च्छच्छन्दो॒ मन॒ श्छन्दो॒ व्यच॒ श्छन्दः॒ सिन्धु॒ श्छन्दः॑ समु॒द्रं छन्दः॑ सलि॒लं छन्दः॑ सं॒ॅयच्छन्दो॑ वि॒यच्छन्दो॑ बृ॒हच्छन्दो॑ रथन्त॒रं छन्दो॑ निका॒य श्छन्दो॑ विव॒ध श्छन्दो॒ गिर॒ श्छन्दो॒ भ्रज॒ श्छन्दः॑ स॒ष्टुप् छन्दो॑ ऽनु॒ष्टुप् छन्दः॑ क॒कुच्छन्द॑ स्त्रिक॒कुच्छन्दः॑ का॒व्यं छन्दो᳚ -ऽङ्कु॒पं छन्दः॑ - [  ] \newline

\textbf{Pada Paata} \newline

अफ्सः॑ । नाम॑ ॥ एवः॑ । छन्दः॑ । वरि॑वः । छन्दः॑ । श॒भूंरिति॑ शं - भूः । छन्दः॑ । प॒रि॒भूरिति॑ परि - भूः । छन्दः॑ । आ॒च्छत् । छन्दः॑ । मनः॑ । छन्दः॑ । व्यचः॑ । छन्दः॑ । सिन्धुः॑ । छन्दः॑ । स॒मु॒द्रम् । छन्दः॑ । स॒लि॒लम् । छन्दः॑ । सं॒ॅयदिति॑ सं - यत् । छन्दः॑ । वि॒यदिति॑ वि - यत् । छन्दः॑ । बृ॒हत् । छन्दः॑ । र॒थ॒न्त॒रमिति॑ रथं - त॒रम् । छन्दः॑ । नि॒का॒य इति॑ नि - का॒यः । छन्दः॑ । वि॒व॒ध इति॑ वि-व॒धः । छन्दः॑ । गिरः॑ । छन्दः॑ । भ्रजः॑ । छन्दः॑ । स॒ष्टुबिति॑ स - स्तुप् । छन्दः॑ । अ॒नु॒ष्टुबित्य॑नु - स्तुप् । छन्दः॑ । क॒कुत् । छन्दः॑ । त्रि॒क॒कुदिति॑ त्रि - क॒कुत् । छन्दः॑ । का॒व्यम् । छन्दः॑ । अ॒ङ्कु॒पम् । छन्दः॑ ।  \newline


\textbf{Krama Paata} \newline

अफ्सो॒ नाम॑ । नामेति॒ नाम॑ ॥ एव॒ श्छन्दः॑ । छन्दो॒ वरि॑वः । वरि॑व॒ श्छन्दः॑ । छन्दः॑ श॒म्भूः । श॒म्भू श्छन्दः॑ । श॒म्भूरिति॑ शम् - भूः । छन्दः॑ परि॒भूः । प॒रि॒भू श्छन्दः॑ । प॒रि॒भूरिति॑ परि - भूः । छन्द॑ आ॒च्छत् । आ॒च्छच्छन्दः॑ । छन्दो॒ मनः॑ । मन॒ श्छन्दः॑ । छन्दो॒ व्यचः॑ । व्यच॒ श्छन्दः॑ । छन्दः॒ सिन्धुः॑ । सिन्धु॒ श्छन्दः॑ । छन्दः॑ समु॒द्रम् । स॒मु॒द्रम् छन्दः॑ । छन्दः॑ सलि॒लम् । स॒लि॒लम् छन्दः॑ । छन्दः॑ स॒म्ॅयत् । स॒म्ॅयच्छन्दः॑ । स॒म्ॅयदिति॑ सम् - यत् । छन्दो॑ वि॒यत् । वि॒यच्छन्दः॑ । वि॒यदिति॑ वि - यत् । छन्दो॑ बृ॒हत् । बृ॒हच्छन्दः॑ । छन्दो॑ रथन्त॒रम् । र॒थ॒न्त॒रम् छन्दः॑ । र॒थ॒न्त॒रमिति॑ रथम् - त॒रम् । छन्दो॑ निका॒यः । नि॒का॒य श्छन्दः॑ । नि॒का॒य इति॑ नि - का॒यः । छन्दो॑ विव॒धः । वि॒व॒ध श्छन्दः॑ । वि॒व॒ध इति॑ वि - व॒धः । छन्दो॒ गिरः॑ । गिर॒ श्छन्दः॑ । छन्दो॒ भ्रजः॑ । भ्रज॒ श्छन्दः॑ । छन्दः॑ स॒ष्टुप् । स॒ष्टुप् छन्दः॑ । स॒ष्टुबिति॑ स - स्तुप् । छन्दो॑ऽनु॒ष्टुप् । अ॒नु॒ष्टुप् छन्दः॑ । अ॒नु॒ष्टुबित्य॑नु - स्तुप् । छन्दः॑ क॒कुत् । क॒कुच्छन्दः॑ । छन्द॑स्त्रिक॒कुत् । त्रि॒क॒कुच्छन्दः॑ । त्रि॒क॒कुदिति॑ त्रि - क॒कुत् । छन्दः॑ का॒व्यम् । का॒व्यम् छन्दः॑ । छन्दो᳚ऽङ्कु॒पम् । अ॒ङ्कु॒पम् छन्दः॑ ( ) । छन्दः॑ प॒दप॑ङ्क्तिः \newline

\textbf{Jatai Paata} \newline

1. अफ्सो॒ नाम॒ नामाफ्सो ऽफ्सो॒ नाम॑ । \newline
2. नामेति॒ नाम॑ । \newline
3. एव॒ श्छन्द॒ श्छन्द॒ एव॒ एव॒ श्छन्दः॑ । \newline
4. छन्दो॒ वरि॑वो॒ वरि॑व॒ श्छन्द॒ श्छन्दो॒ वरि॑वः । \newline
5. वरि॑व॒ श्छन्द॒ श्छन्दो॒ वरि॑वो॒ वरि॑व॒ श्छन्दः॑ । \newline
6. छन्दः॑ शं॒भूः शं॒भू श्छन्द॒ श्छन्दः॑ शं॒भूः । \newline
7. शं॒भू श्छन्द॒ श्छन्दः॑ शं॒भूः शं॒भू श्छन्दः॑ । \newline
8. शं॒भूरिति॑ शं - भूः । \newline
9. छन्दः॑ परि॒भूः प॑रि॒भू श्छन्द॒ श्छन्दः॑ परि॒भूः । \newline
10. प॒रि॒भू श्छन्द॒ श्छन्दः॑ परि॒भूः प॑रि॒भू श्छन्दः॑ । \newline
11. प॒रि॒भूरिति॑ परि - भूः । \newline
12. छन्द॑ आ॒च्छ दा॒च्छच् छन्द॒ श्छन्द॑ आ॒च्छत् । \newline
13. आ॒च्छच् छन्द॒ श्छन्द॑ आ॒च्छ दा॒च्छच् छन्दः॑ । \newline
14. छन्दो॒ मनो॒ मन॒ श्छन्द॒ श्छन्दो॒ मनः॑ । \newline
15. मन॒ श्छन्द॒ श्छन्दो॒ मनो॒ मन॒ श्छन्दः॑ । \newline
16. छन्दो॒ व्यचो॒ व्यच॒ श्छन्द॒ श्छन्दो॒ व्यचः॑ । \newline
17. व्यच॒ श्छन्द॒ श्छन्दो॒ व्यचो॒ व्यच॒ श्छन्दः॑ । \newline
18. छन्दः॒ सिन्धुः॒ सिन्धु॒ श्छन्द॒ श्छन्दः॒ सिन्धुः॑ । \newline
19. सिन्धु॒ श्छन्द॒ श्छन्दः॒ सिन्धुः॒ सिन्धु॒ श्छन्दः॑ । \newline
20. छन्दः॑ समु॒द्रꣳ स॑मु॒द्रम् छन्द॒ श्छन्दः॑ समु॒द्रम् । \newline
21. स॒मु॒द्रम् छन्द॒ श्छन्दः॑ समु॒द्रꣳ स॑मु॒द्रम् छन्दः॑ । \newline
22. छन्दः॑ सलि॒लꣳ स॑लि॒लम् छन्द॒ श्छन्दः॑ सलि॒लम् । \newline
23. स॒लि॒लम् छन्द॒ श्छन्दः॑ सलि॒लꣳ स॑लि॒लम् छन्दः॑ । \newline
24. छन्दः॑ सं॒ॅयथ् सं॒ॅयच् छन्द॒ श्छन्दः॑ सं॒ॅयत् । \newline
25. सं॒ॅयच् छन्द॒ श्छन्दः॑ सं॒ॅयथ् सं॒ॅयच् छन्दः॑ । \newline
26. सं॒ॅयदिति॑ सं - यत् । \newline
27. छन्दो॑ वि॒यद् वि॒यच् छन्द॒ श्छन्दो॑ वि॒यत् । \newline
28. वि॒यच् छन्द॒ श्छन्दो॑ वि॒यद् वि॒यच् छन्दः॑ । \newline
29. वि॒यदिति॑ वि - यत् । \newline
30. छन्दो॑ बृ॒हद् बृ॒हच् छन्द॒ श्छन्दो॑ बृ॒हत् । \newline
31. बृ॒हच् छन्द॒ श्छन्दो॑ बृ॒हद् बृ॒हच् छन्दः॑ । \newline
32. छन्दो॑ रथन्त॒रꣳ र॑थन्त॒रम् छन्द॒ श्छन्दो॑ रथन्त॒रम् । \newline
33. र॒थ॒न्त॒रम् छन्द॒ श्छन्दो॑ रथन्त॒रꣳ र॑थन्त॒रम् छन्दः॑ । \newline
34. र॒थ॒न्त॒रमिति॑ रथं - त॒रम् । \newline
35. छन्दो॑ निका॒यो नि॑का॒य श्छन्द॒ श्छन्दो॑ निका॒यः । \newline
36. नि॒का॒य श्छन्द॒ श्छन्दो॑ निका॒यो नि॑का॒य श्छन्दः॑ । \newline
37. नि॒का॒य इति॑ नि - का॒यः । \newline
38. छन्दो॑ विव॒धो वि॑व॒ध श्छन्द॒ श्छन्दो॑ विव॒धः । \newline
39. ꣡इ॒व॒ध श्छन्द॒ श्छन्दो॑ विव॒धो वि॑व॒ध श्छन्दः॑ । \newline
40. वि॒व॒ध इति॑ वि - व॒धः । \newline
41. छन्दो॒ गिरो॒ गिर॒ श्छन्द॒ श्छन्दो॒ गिरः॑ । \newline
42. गिर॒ श्छन्द॒ श्छन्दो॒ गिरो॒ गिर॒ श्छन्दः॑ । \newline
43. छन्दो॒ भ्रजो॒ भ्रज॒ श्छन्द॒ श्छन्दो॒ भ्रजः॑ । \newline
44. भ्रज॒ श्छन्द॒ श्छन्दो॒ भ्रजो॒ भ्रज॒ श्छन्दः॑ । \newline
45. छन्दः॑ स॒ष्टुफ् स॒ष्टुप् छन्द॒ श्छन्दः॑ स॒ष्टुप् । \newline
46. स॒ष्टुप् छन्द॒ श्छन्दः॑ स॒ष्टुफ् स॒ष्टुप् छन्दः॑ । \newline
47. स॒ष्टुबिति॑ स - स्तुप् । \newline
48. छन्दो॑ ऽनु॒ष्टु ब॑नु॒ष्टुप् छन्द॒ श्छन्दो॑ ऽनु॒ष्टुप् । \newline
49. अ॒नु॒ष्टुप् छन्द॒ श्छन्दो॑ ऽनु॒ष्टु ब॑नु॒ष्टुप् छन्दः॑ । \newline
50. अ॒नु॒ष्टुबित्य॑नु - स्तुप् । \newline
51. छन्दः॑ क॒कुत् क॒कुच् छन्द॒ श्छन्दः॑ क॒कुत् । \newline
52. क॒कुच् छन्द॒ श्छन्दः॑ क॒कुत् क॒कुच् छन्दः॑ । \newline
53. छन्द॑ स्त्रिक॒कुत् त्रि॑क॒कुच् छन्द॒ श्छन्द॑ स्त्रिक॒कुत् । \newline
54. त्रि॒क॒कुच् छन्द॒ श्छन्द॑ स्त्रिक॒कुत् त्रि॑क॒कुच् छन्दः॑ । \newline
55. त्रि॒क॒कुदिति॑ त्रि - क॒कुत् । \newline
56. छन्दः॑ का॒व्यम् का॒व्यम् छन्द॒ श्छन्दः॑ का॒व्यम् । \newline
57. का॒व्यम् छन्द॒ श्छन्दः॑ का॒व्यम् का॒व्यम् छन्दः॑ । \newline
58. छन्दो᳚ ऽङ्कु॒प म॑ङ्कु॒पम् छन्द॒ श्छन्दो᳚ ऽङ्कु॒पम् । \newline
59. अ॒ङ्कु॒पम् छन्द॒ श्छन्दो᳚ ऽङ्कु॒प म॑ङ्कु॒पम् छन्दः॑ । \newline
60. छन्दः॑ प॒दप॑ङ्क्तिः प॒दप॑ङ्क्ति॒ श्छन्द॒ श्छन्दः॑ प॒दप॑ङ्क्तिः । \newline

\textbf{Ghana Paata } \newline

1. अफ्सो॒ नाम॒ नामाफ्सो ऽफ्सो॒ नाम॑ । \newline
2. नामेति॒ नाम॑ । \newline
3. एव॒ श्छन्द॒ श्छन्द॒ एव॒ एव॒ श्छन्दो॒ वरि॑वो॒ वरि॑व॒ श्छन्द॒ एव॒ एव॒ श्छन्दो॒ वरि॑वः । \newline
4. छन्दो॒ वरि॑वो॒ वरि॑व॒ श्छन्द॒ श्छन्दो॒ वरि॑व॒ श्छन्द॒ श्छन्दो॒ वरि॑व॒ श्छन्द॒ श्छन्दो॒ वरि॑व॒ श्छन्दः॑ । \newline
5. वरि॑व॒ श्छन्द॒ श्छन्दो॒ वरि॑वो॒ वरि॑व॒ श्छन्दः॑ शं॒भूः शं॒भू श्छन्दो॒ वरि॑वो॒ वरि॑व॒ श्छन्दः॑ शं॒भूः । \newline
6. छन्दः॑ शं॒भूः शं॒भू श्छन्द॒ श्छन्दः॑ शं॒भू श्छन्द॒ श्छन्दः॑ शं॒भू श्छन्द॒ श्छन्दः॑ शं॒भू श्छन्दः॑ । \newline
7. शं॒भू श्छन्द॒ श्छन्दः॑ शं॒भूः शं॒भू श्छन्दः॑ परि॒भूः प॑रि॒भू श्छन्दः॑ शं॒भूः शं॒भू श्छन्दः॑ परि॒भूः । \newline
8. शं॒भूरिति॑ शं - भूः । \newline
9. छन्दः॑ परि॒भूः प॑रि॒भू श्छन्द॒ श्छन्दः॑ परि॒भू श्छन्द॒ श्छन्दः॑ परि॒भू श्छन्द॒ श्छन्दः॑ परि॒भू श्छन्दः॑ । \newline
10. प॒रि॒भू श्छन्द॒ श्छन्दः॑ परि॒भूः प॑रि॒भू श्छन्द॑ आ॒च्छ दा॒च्छच् छन्दः॑ परि॒भूः प॑रि॒भू श्छन्द॑ आ॒च्छत् । \newline
11. प॒रि॒भूरिति॑ परि - भूः । \newline
12. छन्द॑ आ॒च्छ दा॒च्छच् छन्द॒ श्छन्द॑ आ॒च्छच् छन्द॒ श्छन्द॑ आ॒च्छच् छन्द॒ श्छन्द॑ आ॒च्छच् छन्दः॑ । \newline
13. आ॒च्छच् छन्द॒ श्छन्द॑ आ॒च्छ दा॒च्छच् छन्दो॒ मनो॒ मन॒ श्छन्द॑ आ॒च्छ दा॒च्छच् छन्दो॒ मनः॑ । \newline
14. छन्दो॒ मनो॒ मन॒ श्छन्द॒ श्छन्दो॒ मन॒ श्छन्द॒ श्छन्दो॒ मन॒ श्छन्द॒ श्छन्दो॒ मन॒ श्छन्दः॑ । \newline
15. मन॒ श्छन्द॒ श्छन्दो॒ मनो॒ मन॒ श्छन्दो॒ व्यचो॒ व्यच॒ श्छन्दो॒ मनो॒ मन॒ श्छन्दो॒ व्यचः॑ । \newline
16. छन्दो॒ व्यचो॒ व्यच॒ श्छन्द॒ श्छन्दो॒ व्यच॒ श्छन्द॒ श्छन्दो॒ व्यच॒ श्छन्द॒ श्छन्दो॒ व्यच॒ श्छन्दः॑ । \newline
17. व्यच॒ श्छन्द॒ श्छन्दो॒ व्यचो॒ व्यच॒ श्छन्दः॒ सिन्धुः॒ सिन्धु॒ श्छन्दो॒ व्यचो॒ व्यच॒ श्छन्दः॒ सिन्धुः॑ । \newline
18. छन्दः॒ सिन्धुः॒ सिन्धु॒ श्छन्द॒ श्छन्दः॒ सिन्धु॒ श्छन्द॒ श्छन्दः॒ सिन्धु॒ श्छन्द॒ श्छन्दः॒ सिन्धु॒ श्छन्दः॑ । \newline
19. सिन्धु॒ श्छन्द॒ श्छन्दः॒ सिन्धुः॒ सिन्धु॒ श्छन्दः॑ समु॒द्रꣳ स॑मु॒द्रम् छन्दः॒ सिन्धुः॒ सिन्धु॒ श्छन्दः॑ समु॒द्रम् । \newline
20. छन्दः॑ समु॒द्रꣳ स॑मु॒द्रम् छन्द॒ श्छन्दः॑ समु॒द्रम् छन्द॒ श्छन्दः॑ समु॒द्रम् छन्द॒ श्छन्दः॑ समु॒द्रम् छन्दः॑ । \newline
21. स॒मु॒द्रम् छन्द॒ श्छन्दः॑ समु॒द्रꣳ स॑मु॒द्रम् छन्दः॑ सलि॒लꣳ स॑लि॒लम् छन्दः॑ समु॒द्रꣳ स॑मु॒द्रम् छन्दः॑ सलि॒लम् । \newline
22. छन्दः॑ सलि॒लꣳ स॑लि॒लम् छन्द॒ श्छन्दः॑ सलि॒लम् छन्द॒ श्छन्दः॑ सलि॒लम् छन्द॒ श्छन्दः॑ सलि॒लम् छन्दः॑ । \newline
23. स॒लि॒लम् छन्द॒ श्छन्दः॑ सलि॒लꣳ स॑लि॒लम् छन्दः॑ सं॒ॅयथ् सं॒ॅयच् छन्दः॑ सलि॒लꣳ स॑लि॒लम् छन्दः॑ सं॒ॅयत् । \newline
24. छन्दः॑ सं॒ॅयथ् सं॒ॅयच् छन्द॒ श्छन्दः॑ सं॒ॅयच् छन्द॒ श्छन्दः॑ सं॒ॅयच् छन्द॒ श्छन्दः॑ सं॒ॅयच् छन्दः॑ । \newline
25. सं॒ॅयच् छन्द॒ श्छन्दः॑ सं॒ॅयथ् सं॒ॅयच् छन्दो॑ वि॒यद् वि॒यच् छन्दः॑ सं॒ॅयथ् सं॒ॅयच् छन्दो॑ वि॒यत् । \newline
26. सं॒ॅयदिति॑ सं - यत् । \newline
27. छन्दो॑ वि॒यद् वि॒यच् छन्द॒ श्छन्दो॑ वि॒यच् छन्द॒ श्छन्दो॑ वि॒यच् छन्द॒ श्छन्दो॑ वि॒यच् छन्दः॑ । \newline
28. वि॒यच् छन्द॒ श्छन्दो॑ वि॒यद् वि॒यच् छन्दो॑ बृ॒हद् बृ॒हच् छन्दो॑ वि॒यद् वि॒यच् छन्दो॑ बृ॒हत् । \newline
29. वि॒यदिति॑ वि - यत् । \newline
30. छन्दो॑ बृ॒हद् बृ॒हच् छन्द॒ श्छन्दो॑ बृ॒हच् छन्द॒ श्छन्दो॑ बृ॒हच् छन्द॒ श्छन्दो॑ बृ॒हच् छन्दः॑ । \newline
31. बृ॒हच् छन्द॒ श्छन्दो॑ बृ॒हद् बृ॒हच् छन्दो॑ रथन्त॒रꣳ र॑थन्त॒रम् छन्दो॑ बृ॒हद् बृ॒हच् छन्दो॑ रथन्त॒रम् । \newline
32. छन्दो॑ रथन्त॒रꣳ र॑थन्त॒रम् छन्द॒ श्छन्दो॑ रथन्त॒रम् छन्द॒ श्छन्दो॑ रथन्त॒रम् छन्द॒ श्छन्दो॑ रथन्त॒रम् छन्दः॑ । \newline
33. र॒थ॒न्त॒रम् छन्द॒ श्छन्दो॑ रथन्त॒रꣳ र॑थन्त॒रम् छन्दो॑ निका॒यो नि॑का॒य श्छन्दो॑ रथन्त॒रꣳ र॑थन्त॒रम् छन्दो॑ निका॒यः । \newline
34. र॒थ॒न्त॒रमिति॑ रथं - त॒रम् । \newline
35. छन्दो॑ निका॒यो नि॑का॒य श्छन्द॒ श्छन्दो॑ निका॒य श्छन्द॒ श्छन्दो॑ निका॒य श्छन्द॒ श्छन्दो॑ निका॒य श्छन्दः॑ । \newline
36. नि॒का॒य श्छन्द॒ श्छन्दो॑ निका॒यो नि॑का॒य श्छन्दो॑ विव॒धो वि॑व॒ध श्छन्दो॑ निका॒यो नि॑का॒य श्छन्दो॑ विव॒धः । \newline
37. नि॒का॒य इति॑ नि - का॒यः । \newline
38. छन्दो॑ विव॒धो वि॑व॒ध श्छन्द॒ श्छन्दो॑ विव॒ध श्छन्द॒ श्छन्दो॑ विव॒ध श्छन्द॒ श्छन्दो॑ विव॒ध श्छन्दः॑ । \newline
39. वि॒व॒ध श्छन्द॒ श्छन्दो॑ विव॒धो वि॑व॒ध श्छन्दो॒ गिरो॒ गिर॒ श्छन्दो॑ विव॒धो वि॑व॒ध श्छन्दो॒ गिरः॑ । \newline
40. वि॒व॒ध इति॑ वि - व॒धः । \newline
41. छन्दो॒ गिरो॒ गिर॒ श्छन्द॒ श्छन्दो॒ गिर॒ श्छन्द॒ श्छन्दो॒ गिर॒ श्छन्द॒ श्छन्दो॒ गिर॒ श्छन्दः॑ । \newline
42. गिर॒ श्छन्द॒ श्छन्दो॒ गिरो॒ गिर॒ श्छन्दो॒ भ्रजो॒ भ्रज॒ श्छन्दो॒ गिरो॒ गिर॒ श्छन्दो॒ भ्रजः॑ । \newline
43. छन्दो॒ भ्रजो॒ भ्रज॒ श्छन्द॒ श्छन्दो॒ भ्रज॒ श्छन्द॒ श्छन्दो॒ भ्रज॒ श्छन्द॒ श्छन्दो॒ भ्रज॒ श्छन्दः॑ । \newline
44. भ्रज॒ श्छन्द॒ श्छन्दो॒ भ्रजो॒ भ्रज॒ श्छन्दः॑ स॒ष्टुफ् स॒ष्टुप् छन्दो॒ भ्रजो॒ भ्रज॒ श्छन्दः॑ स॒ष्टुप् । \newline
45. छन्दः॑ स॒ष्टुफ् स॒ष्टुप् छन्द॒ श्छन्दः॑ स॒ष्टुप् छन्द॒ श्छन्दः॑ स॒ष्टुप् छन्द॒ श्छन्दः॑ स॒ष्टुप् छन्दः॑ । \newline
46. स॒ष्टुप् छन्द॒ श्छन्दः॑ स॒ष्टुफ् स॒ष्टुप् छन्दो॑ ऽनु॒ष्टु ब॑नु॒ष्टुप् छन्दः॑ स॒ष्टुफ् स॒ष्टुप् छन्दो॑ ऽनु॒ष्टुप् । \newline
47. स॒ष्टुबिति॑ स - स्तुप् । \newline
48. छन्दो॑ ऽनु॒ष्टु ब॑नु॒ष्टुप् छन्द॒ श्छन्दो॑ ऽनु॒ष्टुप् छन्द॒ श्छन्दो॑ ऽनु॒ष्टुप् छन्द॒ श्छन्दो॑ ऽनु॒ष्टुप् छन्दः॑ । \newline
49. अ॒नु॒ष्टुप् छन्द॒ श्छन्दो॑ ऽनु॒ष्टु ब॑नु॒ष्टुप् छन्दः॑ क॒कुत् क॒कुच् छन्दो॑ ऽनु॒ष्टु ब॑नु॒ष्टुप् छन्दः॑ क॒कुत् । \newline
50. अ॒नु॒ष्टुबित्य॑नु - स्तुप् । \newline
51. छन्दः॑ क॒कुत् क॒कुच् छन्द॒ श्छन्दः॑ क॒कुच् छन्द॒ श्छन्दः॑ क॒कुच् छन्द॒ श्छन्दः॑ क॒कुच् छन्दः॑ । \newline
52. क॒कुच् छन्द॒ श्छन्दः॑ क॒कुत् क॒कुच् छन्द॑ स्त्रिक॒कुत् त्रि॑क॒कुच् छन्दः॑ क॒कुत् क॒कुच् छन्द॑ स्त्रिक॒कुत् । \newline
53. छन्द॑ स्त्रिक॒कुत् त्रि॑क॒कुच् छन्द॒ श्छन्द॑ स्त्रिक॒कुच् छन्द॒ श्छन्द॑ स्त्रिक॒कुच् छन्द॒ श्छन्द॑ स्त्रिक॒कुच् छन्दः॑ । \newline
54. त्रि॒क॒कुच् छन्द॒ श्छन्द॑ स्त्रिक॒कुत् त्रि॑क॒कुच् छन्दः॑ का॒व्यम् का॒व्यम् छन्द॑ स्त्रिक॒कुत् त्रि॑क॒कुच् छन्दः॑ का॒व्यम् । \newline
55. त्रि॒क॒कुदिति॑ त्रि - क॒कुत् । \newline
56. छन्दः॑ का॒व्यम् का॒व्यम् छन्द॒ श्छन्दः॑ का॒व्यम् छन्द॒ श्छन्दः॑ का॒व्यम् छन्द॒ श्छन्दः॑ का॒व्यम् छन्दः॑ । \newline
57. का॒व्यम् छन्द॒ श्छन्दः॑ का॒व्यम् का॒व्यम् छन्दो᳚ ऽङ्कु॒प म॑ङ्कु॒पम् छन्दः॑ का॒व्यम् का॒व्यम् छन्दो᳚ ऽङ्कु॒पम् । \newline
58. छन्दो᳚ ऽङ्कु॒प म॑ङ्कु॒पम् छन्द॒ श्छन्दो᳚ ऽङ्कु॒पम् छन्द॒ श्छन्दो᳚ ऽङ्कु॒पम् छन्द॒ श्छन्दो᳚ ऽङ्कु॒पम् छन्दः॑ । \newline
59. अ॒ङ्कु॒पम् छन्द॒ श्छन्दो᳚ ऽङ्कु॒प म॑ङ्कु॒पम् छन्दः॑ प॒दप॑ङ्क्तिः प॒दप॑ङ्क्ति॒ श्छन्दो᳚ ऽङ्कु॒प म॑ङ्कु॒पम् छन्दः॑ प॒दप॑ङ्क्तिः । \newline
60. छन्दः॑ प॒दप॑ङ्क्तिः प॒दप॑ङ्क्ति॒ श्छन्द॒ श्छन्दः॑ प॒दप॑ङ्क्ति॒ श्छन्द॒ श्छन्दः॑ प॒दप॑ङ्क्ति॒ श्छन्द॒ श्छन्दः॑ प॒दप॑ङ्क्ति॒ श्छन्दः॑ । \newline
\pagebreak
\markright{ TS 4.3.12.3  \hfill https://www.vedavms.in \hfill}

\section{ TS 4.3.12.3 }

\textbf{TS 4.3.12.3 } \newline
\textbf{Samhita Paata} \newline

प॒दप॑ङ्क्ति॒ श्छन्दो॒ ऽक्षर॑पङ्क्ति॒ श्छन्दो॑ विष्टा॒रप॑ङ्क्ति॒ श्छन्दः॑ क्षु॒रो भृज्वा॒ञ्छन्दः॑ प्र॒च्छच्छन्दः॑ प॒क्ष श्छन्द॒ एव॒ श्छन्दो॒ वरि॑व॒ श्छन्दो॒ वय॒ श्छन्दो॑ वय॒स्कृच्छन्दो॑ विशा॒लं छन्दो॒ विष्प॑र्द्धा॒ श्छन्द॑ श्छ॒दि श्छन्दो॑ दूरोह॒णं छन्द॑स्त॒न्द्रं छन्दो᳚ ऽङ्का॒ङ्कं छन्दः॑ ॥ \newline

\textbf{Pada Paata} \newline

प॒दप॑ङ्क्ति॒रिति॑ प॒द - प॒ङ्क्तिः॒ । छन्दः॑ । अ॒क्षर॑पङ्क्ति॒रित्य॒क्षर॑ - प॒ङ्क्तिः॒ । छन्दः॑ । वि॒ष्टा॒रप॑ङ्क्ति॒रिति॑ विष्टा॒र - प॒ङ्क्तिः॒ । छन्दः॑ । क्षु॒रः । भृज्वान्॑ । छन्दः॑ । प्र॒च्छत् । छन्दः॑ । प॒क्षः । छन्दः॑ । एवः॑ । छन्दः॑ । वरि॑वः । छन्दः॑ । वयः॑ । छन्दः॑ । व॒य॒स्कृदिति॑ वयः - कृत् । छन्दः॑ । वि॒शा॒लमिति॑ वि - शा॒लम् । छन्दः॑ । विष्प॑र्धा॒ इति॒ विः-स्प॒द्‌र्धाः॒ । छन्दः॑ । छ॒दिः । छन्दः॑ । दू॒रो॒ह॒णमिति॑ दूः - रो॒ह॒णम् । छन्दः॑ । त॒न्द्रम् । छन्दः॑ । अ॒ङ्का॒ङ्कम् । छन्दः॑ ॥  \newline


\textbf{Krama Paata} \newline

प॒दप॑ङ्क्ति॒ श्छन्दः॑ । प॒दप॑ङ्क्ति॒रिति॑ प॒द - प॒ङ्क्तिः॒ । छन्दो॒ऽक्षर॑पङ्क्तिः । अ॒क्षर॑पङ्क्ति॒ श्छन्दः॑ । अ॒क्षर॑पङ्क्ति॒रित्य॒क्षर॑ - प॒ङ्क्तिः॒ । छन्दो॑ विष्टा॒रप॑ङ्क्तिः । वि॒ष्टा॒रप॑ङ्क्ति॒ श्छन्दः॑ । वि॒ष्टा॒रप॑ङ्क्ति॒रिति॑ विष्टा॒र - प॒ङ्क्तिः॒ । छन्दः॑ क्षु॒रः । क्षु॒रो भृज्वान्॑ । भृज्वा॒न् छन्दः॑ । छन्दः॑ प्र॒च्छत् । प्र॒च्छच्छन्दः॑ । छन्दः॑ प॒क्षः । प॒क्ष श्छन्दः॑ । छन्द॒ एवः॑ । एव॒ श्छन्दः॑ । छन्दो॒ वरि॑वः । वरि॑व॒ श्छन्दः॑ । छन्दो॒ वयः॑ । वय॒ श्छन्दः॑ । छन्दो॑ वय॒स्कृत् । व॒य॒स्कृच्छन्दः॑ । व॒य॒स्कृदिति॑ वयः - कृत् । छन्दो॑ विशा॒लम् । वि॒शा॒लम् छन्दः॑ । वि॒शा॒लमिति॑ वि - शा॒लम् । छन्दो॒ विष्प॑र्द्धाः । विष्प॑र्द्धा॒ श्छन्दः॑ । विष्प॑र्द्धा॒ इति॒ वि - स्प॒र्द्धाः॒ । छन्द॑ श्छ॒दिः । छ॒दि श्छन्दः॑ । छन्दो॑ दूरोह॒णम् । दू॒रो॒ह॒णम् छन्दः॑ । दू॒रो॒ह॒णमिति॑ दुः - रो॒ह॒णम् । छन्द॑स्त॒न्द्रम् । त॒न्द्रम् छन्दः॑ । छन्दो᳚ऽङ्का॒ङ्कम् । अ॒ङ्का॒ङ्कम् छन्दः॑ । छन्द॒ इति॒ छन्दः॑ । \newline

\textbf{Jatai Paata} \newline

1. प॒दप॑ङ्क्ति॒ श्छन्द॒ श्छन्दः॑ प॒दप॑ङ्क्तिः प॒दप॑ङ्क्ति॒ श्छन्दः॑ । \newline
2. प॒दप॑ङ्क्ति॒रिति॑ प॒द - प॒ङ्क्तिः॒ । \newline
3. छन्दो॒ ऽक्षर॑पङ्क्ति र॒क्षर॑पङ्क्ति॒ श्छन्द॒ श्छन्दो॒ ऽक्षर॑पङ्क्तिः । \newline
4. अ॒क्षर॑पङ्क्ति॒ श्छन्द॒ श्छन्दो॒ ऽक्षर॑पङ्क्ति र॒क्षर॑पङ्क्ति॒ श्छन्दः॑ । \newline
5. अ॒क्षर॑पङ्क्ति॒रित्य॒क्षर॑ - प॒ङ्क्तिः॒ । \newline
6. छन्दो॑ विष्टा॒रप॑ङ्क्तिर् विष्टा॒रप॑ङ्क्ति॒ श्छन्द॒ श्छन्दो॑ विष्टा॒रप॑ङ्क्तिः । \newline
7. वि॒ष्टा॒रप॑ङ्क्ति॒ श्छन्द॒ श्छन्दो॑ विष्टा॒रप॑ङ्क्तिर् विष्टा॒रप॑ङ्क्ति॒ श्छन्दः॑ । \newline
8. वि॒ष्टा॒रप॑ङ्क्ति॒रिति॑ विष्टा॒र - प॒ङ्क्तिः॒ । \newline
9. छन्दः॑ क्षु॒रः क्षु॒र श्छन्द॒ श्छन्दः॑ क्षु॒रः । \newline
10. क्षु॒रो भृज्वा॒न् भृज्वा᳚न् क्षु॒रः क्षु॒रो भृज्वान्॑ । \newline
11. भृज्वा॒न् छन्द॒ श्छन्दो॒ भृज्वा॒न् भृज्वा॒न् छन्दः॑ । \newline
12. छन्दः॑ प्र॒च्छत् प्र॒च्छच् छन्द॒ श्छन्दः॑ प्र॒च्छत् । \newline
13. प्र॒च्छच् छन्द॒ श्छन्दः॑ प्र॒च्छत् प्र॒च्छच् छन्दः॑ । \newline
14. छन्दः॑ प॒क्षः प॒क्ष श्छन्द॒ श्छन्दः॑ प॒क्षः । \newline
15. प॒क्ष श्छन्द॒ श्छन्दः॑ प॒क्षः प॒क्ष श्छन्दः॑ । \newline
16. छन्द॒ एव॒ एव॒ श्छन्द॒ श्छन्द॒ एवः॑ । \newline
17. एव॒ श्छन्द॒ श्छन्द॒ एव॒ एव॒ श्छन्दः॑ । \newline
18. छन्दो॒ वरि॑वो॒ वरि॑व॒ श्छन्द॒ श्छन्दो॒ वरि॑वः । \newline
19. वरि॑व॒ श्छन्द॒ श्छन्दो॒ वरि॑वो॒ वरि॑व॒ श्छन्दः॑ । \newline
20. छन्दो॒ वयो॒ वय॒ श्छन्द॒ श्छन्दो॒ वयः॑ । \newline
21. वय॒ श्छन्द॒ श्छन्दो॒ वयो॒ वय॒ श्छन्दः॑ । \newline
22. छन्दो॑ वय॒स्कृद् व॑य॒स्कृच् छन्द॒ श्छन्दो॑ वय॒स्कृत् । \newline
23. व॒य॒स्कृच् छन्द॒ श्छन्दो॑ वय॒स्कृद् व॑य॒स्कृच् छन्दः॑ । \newline
24. व॒य॒स्कृदिति॑ वयः - कृत् । \newline
25. छन्दो॑ विशा॒लं ॅवि॑शा॒लम् छन्द॒ श्छन्दो॑ विशा॒लम् । \newline
26. वि॒शा॒लम् छन्द॒ श्छन्दो॑ विशा॒लं ॅवि॑शा॒लम् छन्दः॑ । \newline
27. वि॒शा॒लमिति॑ वि - शा॒लम् । \newline
28. छन्दो॒ विष्प॑र्धा॒ विष्प॑र्धा॒ श्छन्द॒ श्छन्दो॒ विष्प॑र्धाः । \newline
29. विष्प॑र्धा॒ श्छन्द॒ श्छन्दो॒ विष्प॑र्धा॒ विष्प॑र्धा॒ श्छन्दः॑ । \newline
30. विष्प॑र्धा॒ इति॒ वि - स्प॒र्द्धाः॒ । \newline
31. छन्द॑ श्छ॒दि श्छ॒दि श्छन्द॒ श्छन्द॑ श्छ॒दिः । \newline
32. छ॒दि श्छन्द॒ श्छन्द॑ श्छ॒दि श्छ॒दि श्छन्दः॑ । \newline
33. छन्दो॑ दूरोह॒णम् दू॑रोह॒णम् छन्द॒ श्छन्दो॑ दूरोह॒णम् । \newline
34. दू॒रो॒ह॒णम् छन्द॒ श्छन्दो॑ दूरोह॒णम् दू॑रोह॒णम् छन्दः॑ । \newline
35. दू॒रो॒ह॒णमिति॑ दुः - रो॒ह॒णम् । \newline
36. छन्द॑ स्त॒न्द्रम् त॒न्द्रम् छन्द॒ श्छन्द॑ स्त॒न्द्रम् । \newline
37. त॒न्द्रम् छन्द॒ श्छन्द॑ स्त॒न्द्रम् त॒न्द्रम् छन्दः॑ । \newline
38. छन्दो᳚ ऽङ्का॒ङ्क म॑ङ्का॒ङ्कम् छन्द॒ श्छन्दो᳚ ऽङ्का॒ङ्कम् । \newline
39. अ॒ङ्का॒ङ्कम् छन्द॒ श्छन्दो᳚ ऽङ्का॒ङ्क म॑ङ्का॒ङ्कम् छन्दः॑ । \newline
40. छन्द॒ इति॒ छन्दः॑ । \newline

\textbf{Ghana Paata } \newline

1. प॒दप॑ङ्क्ति॒ श्छन्द॒ श्छन्दः॑ प॒दप॑ङ्क्तिः प॒दप॑ङ्क्ति॒ श्छन्दो॒ ऽक्षर॑पङ्क्ति र॒क्षर॑पङ्क्ति॒ श्छन्दः॑ प॒दप॑ङ्क्तिः प॒दप॑ङ्क्ति॒ श्छन्दो॒ ऽक्षर॑पङ्क्तिः । \newline
2. प॒दप॑ङ्क्ति॒रिति॑ प॒द - प॒ङ्क्तिः॒ । \newline
3. छन्दो॒ ऽक्षर॑पङ्क्ति र॒क्षर॑पङ्क्ति॒ श्छन्द॒ श्छन्दो॒ ऽक्षर॑पङ्क्ति॒ श्छन्द॒ श्छन्दो॒ ऽक्षर॑पङ्क्ति॒ श्छन्द॒ श्छन्दो॒ ऽक्षर॑पङ्क्ति॒ श्छन्दः॑ । \newline
4. अ॒क्षर॑पङ्क्ति॒ श्छन्द॒ श्छन्दो॒ ऽक्षर॑पङ्क्ति र॒क्षर॑पङ्क्ति॒ श्छन्दो॑ विष्टा॒रप॑ङ्क्तिर् विष्टा॒रप॑ङ्क्ति॒ श्छन्दो॒ ऽक्षर॑पङ्क्ति र॒क्षर॑पङ्क्ति॒ श्छन्दो॑ विष्टा॒रप॑ङ्क्तिः । \newline
5. अ॒क्षर॑पङ्क्ति॒रित्य॒क्षर॑ - प॒ङ्क्तिः॒ । \newline
6. छन्दो॑ विष्टा॒रप॑ङ्क्तिर् विष्टा॒रप॑ङ्क्ति॒ श्छन्द॒ श्छन्दो॑ विष्टा॒रप॑ङ्क्ति॒ श्छन्द॒ श्छन्दो॑ विष्टा॒रप॑ङ्क्ति॒ श्छन्द॒ श्छन्दो॑ विष्टा॒रप॑ङ्क्ति॒ श्छन्दः॑ । \newline
7. वि॒ष्टा॒रप॑ङ्क्ति॒ श्छन्द॒ श्छन्दो॑ विष्टा॒रप॑ङ्क्तिर् विष्टा॒रप॑ङ्क्ति॒ श्छन्दः॑ क्षु॒रः क्षु॒र श्छन्दो॑ विष्टा॒रप॑ङ्क्तिर् विष्टा॒रप॑ङ्क्ति॒ श्छन्दः॑ क्षु॒रः । \newline
8. वि॒ष्टा॒रप॑ङ्क्ति॒रिति॑ विष्टा॒र - प॒ङ्क्तिः॒ । \newline
9. छन्दः॑ क्षु॒रः क्षु॒र श्छन्द॒ श्छन्दः॑ क्षु॒रो भृज्वा॒न् भृज्वा᳚न् क्षु॒र श्छन्द॒ श्छन्दः॑ क्षु॒रो भृज्वान्॑ । \newline
10. क्षु॒रो भृज्वा॒न् भृज्वा᳚न् क्षु॒रः क्षु॒रो भृज्वा॒न् छन्द॒ श्छन्दो॒ भृज्वा᳚न् क्षु॒रः क्षु॒रो भृज्वा॒न् छन्दः॑ । \newline
11. भृज्वा॒न् छन्द॒ श्छन्दो॒ भृज्वा॒न् भृज्वा॒न् छन्दः॑ प्र॒च्छत् प्र॒च्छच् छन्दो॒ भृज्वा॒न् भृज्वा॒न् छन्दः॑ प्र॒च्छत् । \newline
12. छन्दः॑ प्र॒च्छत् प्र॒च्छच् छन्द॒ श्छन्दः॑ प्र॒च्छच् छन्द॒ श्छन्दः॑ प्र॒च्छच् छन्द॒ श्छन्दः॑ प्र॒च्छच् छन्दः॑ । \newline
13. प्र॒च्छच् छन्द॒ श्छन्दः॑ प्र॒च्छत् प्र॒च्छच् छन्दः॑ प॒क्षः प॒क्ष श्छन्दः॑ प्र॒च्छत् प्र॒च्छच् छन्दः॑ प॒क्षः । \newline
14. छन्दः॑ प॒क्षः प॒क्ष श्छन्द॒ श्छन्दः॑ प॒क्ष श्छन्द॒ श्छन्दः॑ प॒क्ष श्छन्द॒ श्छन्दः॑ प॒क्ष श्छन्दः॑ । \newline
15. प॒क्ष श्छन्द॒ श्छन्दः॑ प॒क्षः प॒क्ष श्छन्द॒ एव॒ एव॒ श्छन्दः॑ प॒क्षः प॒क्ष श्छन्द॒ एवः॑ । \newline
16. छन्द॒ एव॒ एव॒ श्छन्द॒ श्छन्द॒ एव॒ श्छन्द॒ श्छन्द॒ एव॒ श्छन्द॒ श्छन्द॒ एव॒ श्छन्दः॑ । \newline
17. एव॒ श्छन्द॒ श्छन्द॒ एव॒ एव॒ श्छन्दो॒ वरि॑वो॒ वरि॑व॒ श्छन्द॒ एव॒ एव॒ श्छन्दो॒ वरि॑वः । \newline
18. छन्दो॒ वरि॑वो॒ वरि॑व॒ श्छन्द॒ श्छन्दो॒ वरि॑व॒ श्छन्द॒ श्छन्दो॒ वरि॑व॒ श्छन्द॒ श्छन्दो॒ वरि॑व॒ श्छन्दः॑ । \newline
19. वरि॑व॒ श्छन्द॒ श्छन्दो॒ वरि॑वो॒ वरि॑व॒ श्छन्दो॒ वयो॒ वय॒ श्छन्दो॒ वरि॑वो॒ वरि॑व॒ श्छन्दो॒ वयः॑ । \newline
20. छन्दो॒ वयो॒ वय॒ श्छन्द॒ श्छन्दो॒ वय॒ श्छन्द॒ श्छन्दो॒ वय॒ श्छन्द॒ श्छन्दो॒ वय॒ श्छन्दः॑ । \newline
21. वय॒ श्छन्द॒ श्छन्दो॒ वयो॒ वय॒ श्छन्दो॑ वय॒स्कृद् व॑य॒स्कृच् छन्दो॒ वयो॒ वय॒ श्छन्दो॑ वय॒स्कृत् । \newline
22. छन्दो॑ वय॒स्कृद् व॑य॒स्कृच् छन्द॒ श्छन्दो॑ वय॒स्कृच् छन्द॒ श्छन्दो॑ वय॒स्कृच् छन्द॒ श्छन्दो॑ वय॒स्कृच् छन्दः॑ । \newline
23. व॒य॒स्कृच् छन्द॒ श्छन्दो॑ वय॒स्कृद् व॑य॒स्कृच् छन्दो॑ विशा॒लं ॅवि॑शा॒लम् छन्दो॑ वय॒स्कृद् व॑य॒स्कृच् छन्दो॑ विशा॒लम् । \newline
24. व॒य॒स्कृदिति॑ वयः - कृत् । \newline
25. छन्दो॑ विशा॒लं ॅवि॑शा॒लम् छन्द॒ श्छन्दो॑ विशा॒लम् छन्द॒ श्छन्दो॑ विशा॒लम् छन्द॒ श्छन्दो॑ विशा॒लम् छन्दः॑ । \newline
26. वि॒शा॒लम् छन्द॒ श्छन्दो॑ विशा॒लं ॅवि॑शा॒लम् छन्दो॒ विष्प॑र्धा॒ विष्प॑र्धा॒ श्छन्दो॑ विशा॒लं ॅवि॑शा॒लम् छन्दो॒ विष्प॑र्धाः । \newline
27. वि॒शा॒लमिति॑ वि - शा॒लम् । \newline
28. छन्दो॒ विष्प॑र्धा॒ विष्प॑र्धा॒ श्छन्द॒ श्छन्दो॒ विष्प॑र्धा॒ श्छन्द॒ श्छन्दो॒ विष्प॑र्धा॒ श्छन्द॒ श्छन्दो॒ विष्प॑र्धा॒ श्छन्दः॑ । \newline
29. विष्प॑र्धा॒ श्छन्द॒ श्छन्दो॒ विष्प॑र्धा॒ विष्प॑र्धा॒ श्छन्द॑ श्छ॒दि श्छ॒दि श्छन्दो॒ विष्प॑र्धा॒ विष्प॑र्धा॒ श्छन्द॑ श्छ॒दिः । \newline
30. विष्प॑र्धा॒ इति॒ वि - स्प॒र्द्धाः॒ । \newline
31. छन्द॑ श्छ॒दि श्छ॒दि श्छन्द॒ श्छन्द॑ श्छ॒दि श्छन्द॒ श्छन्द॑ श्छ॒दि श्छन्द॒ श्छन्द॑ श्छ॒दि श्छन्दः॑ । \newline
32. छ॒दि श्छन्द॒ श्छन्द॑ श्छ॒दि श्छ॒दि श्छन्दो॑ दूरोह॒णम् दू॑रोह॒णम् छन्द॑ श्छ॒दि श्छ॒दि श्छन्दो॑ दूरोह॒णम् । \newline
33. छन्दो॑ दूरोह॒णम् दू॑रोह॒णम् छन्द॒ श्छन्दो॑ दूरोह॒णम् छन्द॒ श्छन्दो॑ दूरोह॒णम् छन्द॒ श्छन्दो॑ दूरोह॒णम् छन्दः॑ । \newline
34. दू॒रो॒ह॒णम् छन्द॒ श्छन्दो॑ दूरोह॒णम् दू॑रोह॒णम् छन्द॑ स्त॒न्द्रम् त॒न्द्रम् छन्दो॑ दूरोह॒णम् दू॑रोह॒णम् छन्द॑ स्त॒न्द्रम् । \newline
35. दू॒रो॒ह॒णमिति॑ दुः - रो॒ह॒णम् । \newline
36. छन्द॑ स्त॒न्द्रम् त॒न्द्रम् छन्द॒ श्छन्द॑ स्त॒न्द्रम् छन्द॒ श्छन्द॑ स्त॒न्द्रम् छन्द॒ श्छन्द॑ स्त॒न्द्रम् छन्दः॑ । \newline
37. त॒न्द्रम् छन्द॒ श्छन्द॑ स्त॒न्द्रम् त॒न्द्रम् छन्दो᳚ ऽङ्का॒ङ्क म॑ङ्का॒ङ्कम् छन्द॑ स्त॒न्द्रम् त॒न्द्रम् छन्दो᳚ ऽङ्का॒ङ्कम् । \newline
38. छन्दो᳚ ऽङ्का॒ङ्क म॑ङ्का॒ङ्कम् छन्द॒ श्छन्दो᳚ ऽङ्का॒ङ्कम् छन्द॒ श्छन्दो᳚ ऽङ्का॒ङ्कम् छन्द॒ श्छन्दो᳚ ऽङ्का॒ङ्कम् छन्दः॑ । \newline
39. अ॒ङ्का॒ङ्कम् छन्द॒ श्छन्दो᳚ ऽङ्का॒ङ्क म॑ङ्का॒ङ्कम् छन्दः॑ । \newline
40. छन्द॒ इति॒ छन्दः॑ । \newline
\pagebreak
\markright{ TS 4.3.13.1  \hfill https://www.vedavms.in \hfill}

\section{ TS 4.3.13.1 }

\textbf{TS 4.3.13.1 } \newline
\textbf{Samhita Paata} \newline

अ॒ग्निर्वृ॒त्राणि॑ जङ्घनद् द्रविण॒स्युर्वि॑प॒न्यया᳚ । समि॑द्धः शु॒क्र आहु॑तः ॥ त्वꣳ सो॑मासि॒ सत्प॑ति॒स्त्वꣳ राजो॒त वृ॑त्र॒हा । त्वं भ॒द्रो अ॑सि॒ क्रतुः॑ ॥ भ॒द्रा ते॑ अग्ने स्वनीक स॒दृंग्घो॒रस्य॑ स॒तो विषु॑णस्य॒ चारुः॑ । न यत् ते॑ शो॒चिस्तम॑सा॒ वर॑न्त॒ न ध्व॒स्मान॑स्त॒नुवि॒ रेप॒ आ धुः॑ ॥ भ॒द्रं ते॑ अग्ने सहसि॒न्ननी॑कमुपा॒क आ रो॑चते॒ सूर्य॑स्य । \newline

\textbf{Pada Paata} \newline

अ॒ग्निः । वृ॒त्राणि॑ । ज॒ङ्घ॒न॒त् । द्र॒वि॒ण॒स्युः । वि॒प॒न्ययेति॑ वि - प॒न्यया᳚ ॥ समि॑द्ध॒ इति॑ सम् - इ॒द्धः॒ । शु॒क्रः । आहु॑त॒ इत्या - हु॒तः॒ ॥ त्वम् । सो॒म॒ । अ॒सि॒ । सत्प॑ति॒रिति॒ सत् - प॒तिः॒ । त्वम् । राजा᳚ । उ॒त । वृ॒त्र॒हेति॑ वृत्र - हा ॥ त्वम् । भ॒द्रः । अ॒सि॒ । क्रतुः॑ ॥ भ॒द्रा । ते॒ । अ॒ग्ने॒ । स्व॒नी॒केति॑ सु - अ॒नी॒क॒ । स॒न्दृगेति॑ सं - दृक् । घो॒रस्य॑ । स॒तः । विषु॑णस्य । चारुः॑ ॥ न । यत् । ते॒ । शो॒चिः । तम॑सा । वर॑न्त । न । ध्व॒स्मानः॑ । त॒नुवि॑ । रेपः॑ । एति॑ । धुः॒ ॥ भ॒द्रम् । ते॒ । अ॒ग्ने॒ । स॒ह॒सि॒न्न् । अनी॑कम् । उ॒पा॒के । एति॑ । रो॒च॒ते॒ । सूर्य॑स्य ॥  \newline


\textbf{Krama Paata} \newline

अ॒ग्निर् वृ॒त्राणि॑ । वृ॒त्राणि॑ जङ्घनत् । ज॒ङ्घ॒न॒द् द्र॒वि॒ण॒स्युः । द्र॒वि॒ण॒स्युर् वि॑प॒न्यया᳚ । वि॒प॒न्ययेति॑ वि - प॒न्यया᳚ ॥ समि॑द्धः शु॒क्रः । समि॑द्ध॒ इति॒ सम् - इ॒द्धः॒ । शु॒क्र आहु॑तः । आहु॑त॒ इत्या - हु॒तः॒ ॥ त्वꣳ सो॑म । सो॒मा॒सि॒ । अ॒सि॒ सत्प॑तिः । सत्प॑ति॒स्त्वम् । सत्प॑ति॒रिति॒ सत् - प॒तिः॒ । त्वꣳ राजा᳚ । राजो॒त । उ॒त वृ॑त्र॒हा । वृ॒त्र॒हेति॑ वृत्र - हा ॥ त्वम् भ॒द्रः । भ॒द्रो अ॑सि । अ॒सि॒ क्रतुः॑ । क्रतु॒रिति॒ क्रतुः॑ ॥ भ॒द्रा ते᳚ । ते॒ अ॒ग्ने॒ । अ॒ग्ने॒ स्व॒नी॒क॒ । स्व॒नी॒क॒ स॒न्दृक् । स्व॒नी॒केति॑ सु - अ॒नी॒क॒ । स॒न्दृग् घो॒रस्य॑ । स॒न्दृगिति॑ सम् - दृक् । घो॒रस्य॑ स॒तः । स॒तो विषु॑णस्य । विषु॑णस्य॒ चारुः॑ । चारु॒रिति॒ चारुः॑ ॥ न यत् । यत् ते᳚ । ते॒ शो॒चिः । शो॒चिस्तम॑सा । तम॑सा॒ वर॑न्त । वर॑न्त॒ न । न ध्व॒स्मानः॑ । ध्व॒स्मान॑स्त॒नुवि॑ । त॒नुवि॒ रेपः॑ । रेप॒ आ । आ धुः॑ । धु॒रिति॑ धुः ॥ भ॒द्रम् ते᳚ । ते॒ अ॒ग्ने॒ । अ॒ग्ने॒ स॒ह॒सि॒न्न्॒ । स॒ह॒सि॒न्ननी॑कम् । अनी॑कमुपा॒के । उ॒पा॒क आ । आ रो॑चते । रो॒च॒ते॒ सूर्य॑स्य । सूर्य॒स्येति॒ सूर्य॑स्य । \newline

\textbf{Jatai Paata} \newline

1. अ॒ग्निर् वृ॒त्राणि॑ वृ॒त्रा ण्य॒ग्नि र॒ग्निर् वृ॒त्राणि॑ । \newline
2. वृ॒त्राणि॑ जङ्घनज् जङ्घनद् वृ॒त्राणि॑ वृ॒त्राणि॑ जङ्घनत् । \newline
3. ज॒ङ्घ॒न॒द् द्र॒वि॒ण॒स्युर् द्र॑विण॒स्युर् ज॑ङ्घनज् जङ्घनद् द्रविण॒स्युः । \newline
4. द्र॒वि॒ण॒स्युर् वि॑प॒न्यया॑ विप॒न्यया᳚ द्रविण॒स्युर् द्र॑विण॒स्युर् वि॑प॒न्यया᳚ । \newline
5. वि॒प॒न्ययेति॑ वि - प॒न्यया᳚ । \newline
6. समि॑द्धः शु॒क्रः शु॒क्रः समि॑द्धः॒ समि॑द्धः शु॒क्रः । \newline
7. समि॑द्ध॒ इति॒ सम् - इ॒द्धः॒ । \newline
8. शु॒क्र आहु॑त॒ आहु॑तः शु॒क्रः शु॒क्र आहु॑तः । \newline
9. आहु॑त॒ इत्या - हु॒तः॒ । \newline
10. त्वꣳ सो॑म सोम॒ त्वम् त्वꣳ सो॑म । \newline
11. सो॒मा॒ स्य॒सि॒ सो॒म॒ सो॒मा॒सि॒ । \newline
12. अ॒सि॒ सत्प॑तिः॒ सत्प॑ति रस्यसि॒ सत्प॑तिः । \newline
13. सत्प॑ति॒ स्त्वम् त्वꣳ सत्प॑तिः॒ सत्प॑ति॒ स्त्वम् । \newline
14. सत्प॑ति॒रिति॒ सत् - प॒तिः॒ । \newline
15. त्वꣳ राजा॒ राजा॒ त्वम् त्वꣳ राजा᳚ । \newline
16. राजो॒तोत राजा॒ राजो॒त । \newline
17. उ॒त वृ॑त्र॒हा वृ॑त्र॒ होतोत वृ॑त्र॒हा । \newline
18. वृ॒त्र॒हेति॑ वृत्र - हा । \newline
19. त्वम् भ॒द्रो भ॒द्र स्त्वम् त्वम् भ॒द्रः । \newline
20. भ॒द्रो अ॑स्यसि भ॒द्रो भ॒द्रो अ॑सि । \newline
21. अ॒सि॒ क्रतुः॒ क्रतु॑ रस्यसि॒ क्रतुः॑ । \newline
22. क्रतु॒रिति॒ क्रतुः॑ । \newline
23. भ॒द्रा ते॑ ते भ॒द्रा भ॒द्रा ते᳚ । \newline
24. ते॒ अ॒ग्ने॒ अ॒ग्ने॒ ते॒ ते॒ अ॒ग्ने॒ । \newline
25. अ॒ग्ने॒ स्व॒नी॒क॒ स्व॒नी॒ का॒ग्ने॒ अ॒ग्ने॒ स्व॒नी॒क॒ । \newline
26. स्व॒नी॒क॒ स॒न्दृख् स॒न्दृख् स्व॑नीक स्वनीक स॒न्दृक् । \newline
27. स्व॒नी॒केति॑ सु - अ॒नी॒क॒ । \newline
28. स॒न्दृग् घो॒रस्य॑ घो॒रस्य॑ स॒न्दृख् स॒न्दृग् घो॒रस्य॑ । \newline
29. स॒न्दृगिति॑ सं - दृक् । \newline
30. घो॒रस्य॑ स॒तः स॒तो घो॒रस्य॑ घो॒रस्य॑ स॒तः । \newline
31. स॒तो विषु॑णस्य॒ विषु॑णस्य स॒तः स॒तो विषु॑णस्य । \newline
32. विषु॑णस्य॒ चारु॒ श्चारु॒र् विषु॑णस्य॒ विषु॑णस्य॒ चारुः॑ । \newline
33. चारु॒रिति॒ चारुः॑ । \newline
34. न यद् यन् न न यत् । \newline
35. यत् ते॑ ते॒ यद् यत् ते᳚ । \newline
36. ते॒ शो॒चिः शो॒चि स्ते॑ ते शो॒चिः । \newline
37. शो॒चि स्तम॑सा॒ तम॑सा शो॒चिः शो॒चि स्तम॑सा । \newline
38. तम॑सा॒ वर॑न्त॒ वर॑न्त॒ तम॑सा॒ तम॑सा॒ वर॑न्त । \newline
39. वर॑न्त॒ न न वर॑न्त॒ वर॑न्त॒ न । \newline
40. न ध्व॒स्मानो᳚ ध्व॒स्मानो॒ न न ध्व॒स्मानः॑ । \newline
41. ध्व॒स्मान॑ स्त॒नुवि॑ त॒नुवि॑ ध्व॒स्मानो᳚ ध्व॒स्मान॑ स्त॒नुवि॑ । \newline
42. त॒नुवि॒ रेपो॒ रेप॑ स्त॒नुवि॑ त॒नुवि॒ रेपः॑ । \newline
43. रेप॒ आ रेपो॒ रेप॒ आ । \newline
44. आ धु॑र् धु॒रा धुः॑ । \newline
45. धु॒रिति॑ धुः । \newline
46. भ॒द्रम् ते॑ ते भ॒द्रम् भ॒द्रम् ते᳚ । \newline
47. ते॒ अ॒ग्ने॒ अ॒ग्ने॒ ते॒ ते॒ अ॒ग्ने॒ । \newline
48. अ॒ग्ने॒ स॒ह॒सि॒न् थ्स॒ह॒सि॒न् न॒ग्ने॒ अ॒ग्ने॒ स॒ह॒सि॒न्न् । \newline
49. स॒ह॒सि॒न् ननी॑क॒ मनी॑कꣳ सहसिन् थ्सहसि॒न् ननी॑कम् । \newline
50. अनी॑क मुपा॒क उ॑पा॒के ऽनी॑क॒ मनी॑क मुपा॒के । \newline
51. उ॒पा॒क ओपा॒क उ॑पा॒क आ । \newline
52. आ रो॑चते रोचत॒ आ रो॑चते । \newline
53. रो॒च॒ते॒ सूर्य॑स्य॒ सूर्य॑स्य रोचते रोचते॒ सूर्य॑स्य । \newline
54. सूर्य॒स्येति॒ सूर्य॑स्य । \newline

\textbf{Ghana Paata } \newline

1. अ॒ग्निर् वृ॒त्राणि॑ वृ॒त्राण्य॒ग्नि र॒ग्निर् वृ॒त्राणि॑ जङ्घनज् जङ्घनद् वृ॒त्राण्य॒ग्नि र॒ग्निर् वृ॒त्राणि॑ जङ्घनत् । \newline
2. वृ॒त्राणि॑ जङ्घनज् जङ्घनद् वृ॒त्राणि॑ वृ॒त्राणि॑ जङ्घनद् द्रविण॒स्युर् द्र॑विण॒स्युर् ज॑ङ्घनद् वृ॒त्राणि॑ वृ॒त्राणि॑ जङ्घनद् द्रविण॒स्युः । \newline
3. ज॒ङ्घ॒न॒द् द्र॒वि॒ण॒स्युर् द्र॑विण॒स्युर् ज॑ङ्घनज् जङ्घनद् द्रविण॒स्युर् वि॑प॒न्यया॑ विप॒न्यया᳚ द्रविण॒स्युर् ज॑ङ्घनज् जङ्घनद् द्रविण॒स्युर् वि॑प॒न्यया᳚ । \newline
4. द्र॒वि॒ण॒स्युर् वि॑प॒न्यया॑ विप॒न्यया᳚ द्रविण॒स्युर् द्र॑विण॒स्युर् वि॑प॒न्यया᳚ । \newline
5. वि॒प॒न्ययेति॑ वि - प॒न्यया᳚ । \newline
6. समि॑द्धः शु॒क्रः शु॒क्रः समि॑द्धः॒ समि॑द्धः शु॒क्र आहु॑त॒ आहु॑तः शु॒क्रः समि॑द्धः॒ समि॑द्धः शु॒क्र आहु॑तः । \newline
7. समि॑द्ध॒ इति॒ सम् - इ॒द्धः॒ । \newline
8. शु॒क्र आहु॑त॒ आहु॑तः शु॒क्रः शु॒क्र आहु॑तः । \newline
9. आहु॑त॒ इत्या - हु॒तः॒ । \newline
10. त्वꣳ सो॑म सोम॒ त्वम् त्वꣳ सो॑मास्यसि सोम॒ त्वम् त्वꣳ सो॑मासि । \newline
11. सो॒मा॒स्य॒सि॒ सो॒म॒ सो॒मा॒सि॒ सत्प॑तिः॒ सत्प॑तिरसि सोम सोमासि॒ सत्प॑तिः । \newline
12. अ॒सि॒ सत्प॑तिः॒ सत्प॑ति रस्यसि॒ सत्प॑ति॒ स्त्वम् त्वꣳ सत्प॑ति रस्यसि॒ सत्प॑ति॒ स्त्वम् । \newline
13. सत्प॑ति॒ स्त्वम् त्वꣳ सत्प॑तिः॒ सत्प॑ति॒ स्त्वꣳ राजा॒ राजा॒ त्वꣳ सत्प॑तिः॒ सत्प॑ति॒ स्त्वꣳ राजा᳚ । \newline
14. सत्प॑ति॒रिति॒ सत् - प॒तिः॒ । \newline
15. त्वꣳ राजा॒ राजा॒ त्वम् त्वꣳ राजो॒तोत राजा॒ त्वम् त्वꣳ राजो॒त । \newline
16. राजो॒तोत राजा॒ राजो॒त वृ॑त्र॒हा वृ॑त्र॒होत राजा॒ राजो॒त वृ॑त्र॒हा । \newline
17. उ॒त वृ॑त्र॒हा वृ॑त्र॒हो तोत वृ॑त्र॒हा । \newline
18. वृ॒त्र॒हेति॑ वृत्र - हा । \newline
19. त्वम् भ॒द्रो भ॒द्र स्त्वम् त्वम् भ॒द्रो अ॑स्यसि भ॒द्र स्त्वम् त्वम् भ॒द्रो अ॑सि । \newline
20. भ॒द्रो अ॑स्यसि भ॒द्रो भ॒द्रो अ॑सि॒ क्रतुः॒ क्रतु॑ रसि भ॒द्रो भ॒द्रो अ॑सि॒ क्रतुः॑ । \newline
21. अ॒सि॒ क्रतुः॒ क्रतु॑ रस्यसि॒ क्रतुः॑ । \newline
22. क्रतु॒रिति॒ क्रतुः॑ । \newline
23. भ॒द्रा ते॑ ते भ॒द्रा भ॒द्रा ते॑ अग्ने अग्ने ते भ॒द्रा भ॒द्रा ते॑ अग्ने । \newline
24. ते॒ अ॒ग्ने॒ अ॒ग्ने॒ ते॒ ते॒ अ॒ग्ने॒ स्व॒नी॒क॒ स्व॒नी॒ का॒ग्ने॒ ते॒ ते॒ अ॒ग्ने॒ स्व॒नी॒क॒ । \newline
25. अ॒ग्ने॒ स्व॒नी॒क॒ स्व॒नी॒ का॒ग्ने॒ अ॒ग्ने॒ स्व॒नी॒क॒ स॒न्दृख् स॒न्दृख् स्व॑नी काग्ने अग्ने स्वनीक स॒न्दृक् । \newline
26. स्व॒नी॒क॒ स॒न्दृख् स॒न्दृख् स्व॑नीक स्वनीक स॒न्दृग् घो॒रस्य॑ घो॒रस्य॑ स॒न्दृख् स्व॑नीक स्वनीक स॒न्दृग् घो॒रस्य॑ । \newline
27. स्व॒नी॒केति॑ सु - अ॒नी॒क॒ । \newline
28. स॒न्दृग् घो॒रस्य॑ घो॒रस्य॑ स॒न्दृख् स॒न्दृग् घो॒रस्य॑ स॒तः स॒तो घो॒रस्य॑ स॒न्दृख् स॒न्दृग् घो॒रस्य॑ स॒तः । \newline
29. स॒न्दृगिति॑ सं - दृक् । \newline
30. घो॒रस्य॑ स॒तः स॒तो घो॒रस्य॑ घो॒रस्य॑ स॒तो विषु॑णस्य॒ विषु॑णस्य स॒तो घो॒रस्य॑ घो॒रस्य॑ स॒तो विषु॑णस्य । \newline
31. स॒तो विषु॑णस्य॒ विषु॑णस्य स॒तः स॒तो विषु॑णस्य॒ चारु॒ श्चारु॒र् विषु॑णस्य स॒तः स॒तो विषु॑णस्य॒ चारुः॑ । \newline
32. विषु॑णस्य॒ चारु॒ श्चारु॒र् विषु॑णस्य॒ विषु॑णस्य॒ चारुः॑ । \newline
33. चारु॒रिति॒ चारुः॑ । \newline
34. न यद् यन् न न यत् ते॑ ते॒ यन् न न यत् ते᳚ । \newline
35. यत् ते॑ ते॒ यद् यत् ते॑ शो॒चिः शो॒चि स्ते॒ यद् यत् ते॑ शो॒चिः । \newline
36. ते॒ शो॒चिः शो॒चि स्ते॑ ते शो॒चि स्तम॑सा॒ तम॑सा शो॒चि स्ते॑ ते शो॒चि स्तम॑सा । \newline
37. शो॒चि स्तम॑सा॒ तम॑सा शो॒चिः शो॒चि स्तम॑सा॒ वर॑न्त॒ वर॑न्त॒ तम॑सा शो॒चिः शो॒चि स्तम॑सा॒ वर॑न्त । \newline
38. तम॑सा॒ वर॑न्त॒ वर॑न्त॒ तम॑सा॒ तम॑सा॒ वर॑न्त॒ न न वर॑न्त॒ तम॑सा॒ तम॑सा॒ वर॑न्त॒ न । \newline
39. वर॑न्त॒ न न वर॑न्त॒ वर॑न्त॒ न ध्व॒स्मानो᳚ ध्व॒स्मानो॒ न वर॑न्त॒ वर॑न्त॒ न ध्व॒स्मानः॑ । \newline
40. न ध्व॒स्मानो᳚ ध्व॒स्मानो॒ न न ध्व॒स्मान॑ स्त॒नुवि॑ त॒नुवि॑ ध्व॒स्मानो॒ न न ध्व॒स्मान॑ स्त॒नुवि॑ । \newline
41. ध्व॒स्मान॑ स्त॒नुवि॑ त॒नुवि॑ ध्व॒स्मानो᳚ ध्व॒स्मान॑ स्त॒नुवि॒ रेपो॒ रेप॑ स्त॒नुवि॑ ध्व॒स्मानो᳚ ध्व॒स्मान॑ स्त॒नुवि॒ रेपः॑ । \newline
42. त॒नुवि॒ रेपो॒ रेप॑ स्त॒नुवि॑ त॒नुवि॒ रेप॒ आ रेप॑ स्त॒नुवि॑ त॒नुवि॒ रेप॒ आ । \newline
43. रेप॒ आ रेपो॒ रेप॒ आ धु॑र् धु॒रा रेपो॒ रेप॒ आ धुः॑ । \newline
44. आ धु॑र् धु॒रा धुः॑ । \newline
45. धु॒रिति॑ धुः । \newline
46. भ॒द्रम् ते॑ ते भ॒द्रम् भ॒द्रम् ते॑ अग्ने अग्ने ते भ॒द्रम् भ॒द्रम् ते॑ अग्ने । \newline
47. ते॒ अ॒ग्ने॒ अ॒ग्ने॒ ते॒ ते॒ अ॒ग्ने॒ स॒ह॒सि॒न् थ्स॒ह॒सि॒न् न॒ग्ने॒ ते॒ ते॒ अ॒ग्ने॒ स॒ह॒सि॒न्न् । \newline
48. अ॒ग्ने॒ स॒ह॒सि॒न् थ्स॒ह॒सि॒न् न॒ग्ने॒ अ॒ग्ने॒ स॒ह॒सि॒न् ननी॑क॒ मनी॑कꣳ सहसिन् नग्ने अग्ने सहसि॒न् ननी॑कम् । \newline
49. स॒ह॒सि॒न् ननी॑क॒ मनी॑कꣳ सहसिन् थ्सहसि॒न् ननी॑क मुपा॒क उ॑पा॒के ऽनी॑कꣳ सहसिन् थ्सहसि॒न् ननी॑क मुपा॒के । \newline
50. अनी॑क मुपा॒क उ॑पा॒के ऽनी॑क॒ मनी॑क मुपा॒क ओपा॒के ऽनी॑क॒ मनी॑क मुपा॒क आ । \newline
51. उ॒पा॒क ओपा॒क उ॑पा॒क आ रो॑चते रोचत॒ ओपा॒क उ॑पा॒क आ रो॑चते । \newline
52. आ रो॑चते रोचत॒ आ रो॑चते॒ सूर्य॑स्य॒ सूर्य॑स्य रोचत॒ आ रो॑चते॒ सूर्य॑स्य । \newline
53. रो॒च॒ते॒ सूर्य॑स्य॒ सूर्य॑स्य रोचते रोचते॒ सूर्य॑स्य । \newline
54. सूर्य॒स्येति॒ सूर्य॑स्य । \newline
\pagebreak
\markright{ TS 4.3.13.2  \hfill https://www.vedavms.in \hfill}

\section{ TS 4.3.13.2 }

\textbf{TS 4.3.13.2 } \newline
\textbf{Samhita Paata} \newline

रुश॑द्-दृ॒शे द॑दृशे नक्त॒या चि॒दरू᳚क्षितं दृ॒श आ रू॒पे अन्नं᳚ ॥ सैनाऽनी॑केन सुवि॒दत्रो॑ अ॒स्मे यष्टा॑ दे॒वाꣳ आय॑जिष्ठः स्व॒स्ति । अद॑ब्धो गो॒पा उ॒त नः॑ पर॒स्पा अग्ने᳚ द्यु॒मदु॒त रे॒वद् दि॑दीहि ॥ स्व॒स्ति नो॑ दि॒वो अ॑ग्ने पृथि॒व्या वि॒श्वायु॑र्द्धेहि य॒जथा॑य देव । यथ् सी॒महि॑ दिविजात॒ प्रश॑स्तं॒ तद॒स्मासु॒ द्रवि॑णं धेहि चि॒त्रं ॥ यथा॑ होत॒र्मनु॑षो - [  ] \newline

\textbf{Pada Paata} \newline

रुश॑त् । दृ॒शे । द॒दृ॒शे॒ । न॒क्त॒या । चि॒त् । अरू᳚क्षितम् । दृ॒शे । एति॑ । रू॒पे । अन्न᳚म् ॥ सः । ए॒ना । अनी॑केन । सु॒वि॒दत्र॒ इति॑ सु-वि॒दत्रः॑ । अ॒स्मे इति॑ । यष्टा᳚ । दे॒वान् । आय॑जिष्ठ॒ इत्या - य॒जि॒ष्ठः॒ । स्व॒स्ति ॥ अद॑ब्धः । गो॒पा इति गो-पाः । उ॒त । नः॒ । प॒र॒स्पा इति॑ परः-पाः । अग्ने᳚ । द्यु॒मदिति॑ द्यु - मत् । उ॒त । रे॒वत् । दि॒दी॒हि॒ ॥ स्व॒स्ति । नः॒ । दि॒वः । अ॒ग्ने॒ । पृ॒थि॒व्याः । वि॒श्वायु॒रिति॑ वि॒श्व - आ॒युः॒ । धे॒हि॒ । य॒जथा॑य । दे॒व॒ ॥ यत् । सी॒महि॑ । दि॒वि॒जा॒तेति॑ दिवि - जा॒त॒ । प्रश॑स्त॒मिति॒ प्र - श॒स्त॒म् । तत् । अ॒स्मासु॑ । द्रवि॑णम् । धे॒हि॒ । चि॒त्रम् ॥ यथा᳚ । हो॒तः॒ । मनु॑षः ।  \newline


\textbf{Krama Paata} \newline

रुश॑द् दृ॒शे । दृ॒शे द॑दृशे । द॒दृ॒शे॒ न॒क्त॒या । न॒क्त॒या चि॑त् । चि॒दरू᳚क्षितम् । अरू᳚क्षितम् दृ॒शे । दृ॒श आ । आ रू॒पे । रू॒पे अन्न᳚म् । अन्न॒मित्यन्न᳚म् ॥ सैना । ए॒नाऽनी॑केन । अनी॑केन सुवि॒दत्रः॑ । सु॒वि॒दत्रो॑ अ॒स्मे । सु॒वि॒दत्र॒ इति॑ सु - वि॒दत्रः॑ । अ॒स्मे यष्टा᳚ । अ॒स्मे इत्य॒स्मे । यष्टा॑ दे॒वान् । दे॒वाꣳ आय॑जिष्ठः । आय॑जिष्ठः स्व॒स्ति । आय॑जिष्ठ॒ इत्या - य॒जि॒ष्ठः॒ । स्व॒स्तीति॑ स्व॒स्ति ॥ अद॑ब्धो गो॒पाः । गो॒पा उ॒त । गो॒पा इति॑ गो - पाः । उ॒त नः॑ । नः॒ प॒र॒स्पाः । प॒र॒स्पा अग्ने᳚ । प॒र॒स्पा इति॑ परः - पाः । अग्ने᳚ द्यु॒मत् । द्यु॒मदु॒त । द्यु॒मदिति॑ द्यु - मत् । उ॒त रे॒वत् । रे॒वद् दि॑दीहि । दि॒दी॒हीति॑ दिदीहि ॥ स्व॒स्ति नः॑ । नो॒ दि॒वः । दि॒वो अ॑ग्ने । अ॒ग्ने॒ पृ॒थि॒व्याः । पृ॒थि॒व्या वि॒श्वायुः॑ । वि॒श्वायु॑र् धेहि । वि॒श्वायु॒रिति॑ वि॒श्व - आ॒युः॒ । धे॒हि॒ य॒जथा॑य । य॒जथा॑य देव । दे॒वेति॑ देव ॥ यथ् सी॒महि॑ । सी॒महि॑ दिविजात । दि॒वि॒जा॒त॒ प्रश॑स्तम् । दि॒वि॒जा॒तेति॑ दिवि - जा॒त॒ । प्रश॑स्त॒म् तत् । प्रश॑स्त॒मिति॒ प्र - श॒स्त॒म् । तद॒स्मासु॑ । अ॒स्मासु॒ द्रवि॑णम् । द्रवि॑णम् धेहि । धे॒हि॒ चि॒त्रम् । चि॒त्रमिति॑ चि॒त्रम् ॥ यथा॑ होतः । हो॒त॒र् मनु॑षः । मनु॑षो दे॒वता॑ता । \newline

\textbf{Jatai Paata} \newline

1. रुश॑द् दृ॒शे दृ॒शे रुश॒द् रुश॑द् दृ॒शे । \newline
2. दृ॒शे द॑दृशे ददृशे दृ॒शे दृ॒शे द॑दृशे । \newline
3. द॒दृ॒शे॒ न॒क्त॒या न॑क्त॒या द॑दृशे ददृशे नक्त॒या । \newline
4. न॒क्त॒या चि॑च् चिन् नक्त॒या न॑क्त॒या चि॑त् । \newline
5. चि॒दरू᳚क्षित॒ मरू᳚क्षितम् चिच् चि॒दरू᳚क्षितम् । \newline
6. अरू᳚क्षितम् दृ॒शे दृ॒शे अरू᳚क्षित॒ मरू᳚क्षितम् दृ॒शे । \newline
7. दृ॒श आ दृ॒शे दृ॒श आ । \newline
8. आ रू॒पे रू॒प आ रू॒पे । \newline
9. रू॒पे अन्न॒ मन्नꣳ॑ रू॒पे रू॒पे अन्न᳚म् । \newline
10. अन्न॒मित्यन्न᳚म् । \newline
11. सैनैना स सैना । \newline
12. ए॒ना ऽनी॑के॒ना नी॑के नै॒नैना ऽनी॑केन । \newline
13. अनी॑केन सुवि॒दत्रः॑ सुवि॒दत्रो ऽनी॑के॒ना नी॑केन सुवि॒दत्रः॑ । \newline
14. सु॒वि॒दत्रो॑ अ॒स्मे अ॒स्मे सु॑वि॒दत्रः॑ सुवि॒दत्रो॑ अ॒स्मे । \newline
15. सु॒वि॒दत्र॒ इति॑ सु - वि॒दत्रः॑ । \newline
16. अ॒स्मे यष्टा॒ यष्टा॒ ऽस्मे अ॒स्मे यष्टा᳚ । \newline
17. अ॒स्मे इत्य॒स्मे । \newline
18. यष्टा॑ दे॒वान् दे॒वान्. यष्टा॒ यष्टा॑ दे॒वान् । \newline
19. दे॒वाꣳ आय॑जिष्ठ॒ आय॑जिष्ठो दे॒वान् दे॒वाꣳ आय॑जिष्ठः । \newline
20. आय॑जिष्ठः स्व॒स्ति स्व॒स्त्याय॑जिष्ठ॒ आय॑जिष्ठः स्व॒स्ति । \newline
21. आय॑जिष्ठ॒ इत्या - य॒जि॒ष्ठः॒ । \newline
22. स्व॒स्तीति॑ स्व॒स्ति । \newline
23. अद॑ब्धो गो॒पा गो॒पा अद॑ब्धो॒ अद॑ब्धो गो॒पाः । \newline
24. गो॒पा उ॒तोत गो॒पा गो॒पा उ॒त । \newline
25. गो॒पा इति॑ गो - पाः । \newline
26. उ॒त नो॑ न उ॒तोत नः॑ । \newline
27. नः॒ प॒र॒स्पाः प॑र॒स्पा नो॑ नः पर॒स्पाः । \newline
28. प॒र॒स्पा अग्ने ऽग्ने॑ पर॒स्पाः प॑र॒स्पा अग्ने᳚ । \newline
29. प॒र॒स्पा इति॑ परः - पाः । \newline
30. अग्ने᳚ द्यु॒मद् द्यु॒म दग्ने ऽग्ने᳚ द्यु॒मत् । \newline
31. द्यु॒म दु॒तोत द्यु॒मद् द्यु॒म दु॒त । \newline
32. द्यु॒मदिति॑ द्यु - मत् । \newline
33. उ॒त रे॒वद् रे॒व दु॒तोत रे॒वत् । \newline
34. रे॒वद् दि॑दीहि दिदीहि रे॒वद् रे॒वद् दि॑दीहि । \newline
35. दि॒दी॒हीति॑ दिदीहि । \newline
36. स्व॒स्ति नो॑ नः स्व॒स्ति स्व॒स्ति नः॑ । \newline
37. नो॒ दि॒वो दि॒वो नो॑ नो दि॒वः । \newline
38. दि॒वो अ॑ग्ने अग्ने दि॒वो दि॒वो अ॑ग्ने । \newline
39. अ॒ग्ने॒ पृ॒थि॒व्याः पृ॑थि॒व्या अ॑ग्ने अग्ने पृथि॒व्याः । \newline
40. पृ॒थि॒व्या वि॒श्वायु॑र् वि॒श्वायुः॑ पृथि॒व्याः पृ॑थि॒व्या वि॒श्वायुः॑ । \newline
41. वि॒श्वायु॑र् धेहि धेहि वि॒श्वायु॑र् वि॒श्वायु॑र् धेहि । \newline
42. वि॒श्वायु॒रिति॑ वि॒श्व - आ॒युः॒ । \newline
43. धे॒हि॒ य॒जथा॑य य॒जथा॑य धेहि धेहि य॒जथा॑य । \newline
44. य॒जथा॑य देव देव य॒जथा॑य य॒जथा॑य देव । \newline
45. दे॒वेति॑ देव । \newline
46. यथ् सी॒महि॑ सी॒महि॒ यद् यथ् सी॒महि॑ । \newline
47. सी॒महि॑ दिविजात दिविजात सी॒महि॑ सी॒महि॑ दिविजात । \newline
48. दि॒वि॒जा॒त॒ प्रश॑स्त॒म् प्रश॑स्तम् दिविजात दिविजात॒ प्रश॑स्तम् । \newline
49. दि॒वि॒जा॒तेति॑ दिवि - जा॒त॒ । \newline
50. प्रश॑स्त॒म् तत् तत् प्रश॑स्त॒म् प्रश॑स्त॒म् तत् । \newline
51. प्रश॑स्त॒मिति॒ प्र - श॒स्त॒म् । \newline
52. तद॒स्मा स्व॒स्मासु॒ तत् तद॒स्मासु॑ । \newline
53. अ॒स्मासु॒ द्रवि॑ण॒म् द्रवि॑ण म॒स्मा स्व॒स्मासु॒ द्रवि॑णम् । \newline
54. द्रवि॑णम् धेहि धेहि॒ द्रवि॑ण॒म् द्रवि॑णम् धेहि । \newline
55. धे॒हि॒ चि॒त्रम् चि॒त्रम् धे॑हि धेहि चि॒त्रम् । \newline
56. चि॒त्रमिति॑ चि॒त्रम् । \newline
57. यथा॑ होतर्. होत॒र् यथा॒ यथा॑ होतः । \newline
58. हो॒त॒र् मनु॑षो॒ मनु॑षो होतर्. होत॒र् मनु॑षः । \newline
59. मनु॑षो दे॒वता॑ता दे॒वता॑ता॒ मनु॑षो॒ मनु॑षो दे॒वता॑ता । \newline

\textbf{Ghana Paata } \newline

1. रुश॑द् दृ॒शे दृ॒शे रुश॒द् रुश॑द् दृ॒शे द॑दृशे ददृशे दृ॒शे रुश॒द् रुश॑द् दृ॒शे द॑दृशे । \newline
2. दृ॒शे द॑दृशे ददृशे दृ॒शे दृ॒शे द॑दृशे नक्त॒या न॑क्त॒या द॑दृशे दृ॒शे दृ॒शे द॑दृशे नक्त॒या । \newline
3. द॒दृ॒शे॒ न॒क्त॒या न॑क्त॒या द॑दृशे ददृशे नक्त॒या चि॑च् चिन् नक्त॒या द॑दृशे ददृशे नक्त॒या चि॑त् । \newline
4. न॒क्त॒या चि॑च् चिन् नक्त॒या न॑क्त॒या चि॒दरू᳚क्षित॒ मरू᳚क्षितम् चिन् नक्त॒या न॑क्त॒या चि॒दरू᳚क्षितम् । \newline
5. छि॒दरू᳚क्षित॒ मरू᳚क्षितम् चिच् चि॒दरू᳚क्षितम् दृ॒शे दृ॒शे अरू᳚क्षितम् चिच् चि॒दरू᳚क्षितम् दृ॒शे । \newline
6. अरू᳚क्षितम् दृ॒शे दृ॒शे अरू᳚क्षित॒ मरू᳚क्षितम् दृ॒श आ दृ॒शे अरू᳚क्षित॒ मरू᳚क्षितम् दृ॒श आ । \newline
7. दृ॒श आ दृ॒शे दृ॒श आ रू॒पे रू॒प आ दृ॒शे दृ॒श आ रू॒पे । \newline
8. आ रू॒पे रू॒प आ रू॒पे अन्न॒ मन्नꣳ॑ रू॒प आ रू॒पे अन्न᳚म् । \newline
9. रू॒पे अन्न॒ मन्नꣳ॑ रू॒पे रू॒पे अन्न᳚म् । \newline
10. अन्न॒मित्यन्न᳚म् । \newline
11. सैनैना स सैना ऽनी॑के॒ना नी॑के नै॒ना स सैना ऽनी॑केन । \newline
12. ए॒ना ऽनी॑के॒ना नी॑के नै॒नैना ऽनी॑केन सुवि॒दत्रः॑ सुवि॒दत्रो ऽनी॑केनै॒ नैना ऽनी॑केन सुवि॒दत्रः॑ । \newline
13. अनी॑केन सुवि॒दत्रः॑ सुवि॒दत्रो ऽनी॑के॒ना नी॑केन सुवि॒दत्रो॑ अ॒स्मे अ॒स्मे सु॑वि॒दत्रो ऽनी॑के॒ना नी॑केन सुवि॒दत्रो॑ अ॒स्मे । \newline
14. सु॒वि॒दत्रो॑ अ॒स्मे अ॒स्मे सु॑वि॒दत्रः॑ सुवि॒दत्रो॑ अ॒स्मे यष्टा॒ यष्टा॒ ऽस्मे सु॑वि॒दत्रः॑ सुवि॒दत्रो॑ अ॒स्मे यष्टा᳚ । \newline
15. सु॒वि॒दत्र॒ इति॑ सु - वि॒दत्रः॑ । \newline
16. अ॒स्मे यष्टा॒ यष्टा॒ ऽस्मे अ॒स्मे यष्टा॑ दे॒वान् दे॒वान्. यष्टा॒ ऽस्मे अ॒स्मे यष्टा॑ दे॒वान् । \newline
17. अ॒स्मे इत्य॒स्मे । \newline
18. यष्टा॑ दे॒वान् दे॒वान्. यष्टा॒ यष्टा॑ दे॒वाꣳ आय॑जिष्ठ॒ आय॑जिष्ठो दे॒वान्. यष्टा॒ यष्टा॑ दे॒वाꣳ आय॑जिष्ठः । \newline
19. दे॒वाꣳ आय॑जिष्ठ॒ आय॑जिष्ठो दे॒वान् दे॒वाꣳ आय॑जिष्ठः स्व॒स्ति स्व॒स्त्या य॑जिष्ठो दे॒वान् दे॒वाꣳ आय॑जिष्ठः स्व॒स्ति । \newline
20. आय॑जिष्ठः स्व॒स्ति स्व॒स्त्या य॑जिष्ठ॒ आय॑जिष्ठः स्व॒स्ति । \newline
21. आय॑जिष्ठ॒ इत्या - य॒जि॒ष्ठः॒ । \newline
22. स्व॒स्तीति॑ स्व॒स्ति । \newline
23. अद॑ब्धो गो॒पा गो॒पा अद॑ब्धो॒ अद॑ब्धो गो॒पा उ॒तोत गो॒पा अद॑ब्धो॒ अद॑ब्धो गो॒पा उ॒त । \newline
24. गो॒पा उ॒तोत गो॒पा गो॒पा उ॒त नो॑ न उ॒त गो॒पा गो॒पा उ॒त नः॑ । \newline
25. गो॒पा इति॑ गो - पाः । \newline
26. उ॒त नो॑ न उ॒तोत नः॑ पर॒स्पाः प॑र॒स्पा न॑ उ॒तोत नः॑ पर॒स्पाः । \newline
27. नः॒ प॒र॒स्पाः प॑र॒स्पा नो॑ नः पर॒स्पा अग्ने ऽग्ने॑ पर॒स्पा नो॑ नः पर॒स्पा अग्ने᳚ । \newline
28. प॒र॒स्पा अग्ने ऽग्ने॑ पर॒स्पाः प॑र॒स्पा अग्ने᳚ द्यु॒मद् द्यु॒म दग्ने॑ पर॒स्पाः प॑र॒स्पा अग्ने᳚ द्यु॒मत् । \newline
29. प॒र॒स्पा इति॑ परः - पाः । \newline
30. अग्ने᳚ द्यु॒मद् द्यु॒म दग्ने ऽग्ने᳚ द्यु॒म दु॒तोत द्यु॒म दग्ने ऽग्ने᳚ द्यु॒म दु॒त । \newline
31. द्यु॒म दु॒तोत द्यु॒मद् द्यु॒म दु॒त रे॒वद् रे॒व दु॒त द्यु॒मद् द्यु॒म दु॒त रे॒वत् । \newline
32. द्यु॒मदिति॑ द्यु - मत् । \newline
33. उ॒त रे॒वद् रे॒व दु॒तोत रे॒वद् दि॑दीहि दिदीहि रे॒व दु॒तोत रे॒वद् दि॑दीहि । \newline
34. रे॒वद् दि॑दीहि दिदीहि रे॒वद् रे॒वद् दि॑दीहि । \newline
35. दि॒दी॒हीति॑ दिदीहि । \newline
36. स्व॒स्ति नो॑ नः स्व॒स्ति स्व॒स्ति नो॑ दि॒वो दि॒वो नः॑ स्व॒स्ति स्व॒स्ति नो॑ दि॒वः । \newline
37. नो॒ दि॒वो दि॒वो नो॑ नो दि॒वो अ॑ग्ने अग्ने दि॒वो नो॑ नो दि॒वो अ॑ग्ने । \newline
38. दि॒वो अ॑ग्ने अग्ने दि॒वो दि॒वो अ॑ग्ने पृथि॒व्याः पृ॑थि॒व्या अ॑ग्ने दि॒वो दि॒वो अ॑ग्ने पृथि॒व्याः । \newline
39. अ॒ग्ने॒ पृ॒थि॒व्याः पृ॑थि॒व्या अ॑ग्ने अग्ने पृथि॒व्या वि॒श्वायु॑र् वि॒श्वायुः॑ पृथि॒व्या अ॑ग्ने अग्ने पृथि॒व्या वि॒श्वायुः॑ । \newline
40. पृ॒थि॒व्या वि॒श्वायु॑र् वि॒श्वायुः॑ पृथि॒व्याः पृ॑थि॒व्या वि॒श्वायु॑र् धेहि धेहि वि॒श्वायुः॑ पृथि॒व्याः पृ॑थि॒व्या वि॒श्वायु॑र् धेहि । \newline
41. वि॒श्वायु॑र् धेहि धेहि वि॒श्वायु॑र् वि॒श्वायु॑र् धेहि य॒जथा॑य य॒जथा॑य धेहि वि॒श्वायु॑र् वि॒श्वायु॑र् धेहि य॒जथा॑य । \newline
42. वि॒श्वायु॒रिति॑ वि॒श्व - आ॒युः॒ । \newline
43. धे॒हि॒ य॒जथा॑य य॒जथा॑य धेहि धेहि य॒जथा॑य देव देव य॒जथा॑य धेहि धेहि य॒जथा॑य देव । \newline
44. य॒जथा॑य देव देव य॒जथा॑य य॒जथा॑य देव । \newline
45. दे॒वेति॑ देव । \newline
46. यथ् सी॒महि॑ सी॒महि॒ यद् यथ् सी॒महि॑ दिविजात दिविजात सी॒महि॒ यद् यथ् सी॒महि॑ दिविजात । \newline
47. सी॒महि॑ दिविजात दिविजात सी॒महि॑ सी॒महि॑ दिविजात॒ प्रश॑स्त॒म् प्रश॑स्तम् दिविजात सी॒महि॑ सी॒महि॑ दिविजात॒ प्रश॑स्तम् । \newline
48. दि॒वि॒जा॒त॒ प्रश॑स्त॒म् प्रश॑स्तम् दिविजात दिविजात॒ प्रश॑स्त॒म् तत् तत् प्रश॑स्तम् दिविजात दिविजात॒ प्रश॑स्त॒म् तत् । \newline
49. दि॒वि॒जा॒तेति॑ दिवि - जा॒त॒ । \newline
50. प्रश॑स्त॒म् तत् तत् प्रश॑स्त॒म् प्रश॑स्त॒म् तद॒स्मा स्व॒स्मासु॒ तत् प्रश॑स्त॒म् प्रश॑स्त॒म् तद॒स्मासु॑ । \newline
51. प्रश॑स्त॒मिति॒ प्र - श॒स्त॒म् । \newline
52. तद॒स्मा स्व॒स्मासु॒ तत् तद॒स्मासु॒ द्रवि॑ण॒म् द्रवि॑ण म॒स्मासु॒ तत् तद॒स्मासु॒ द्रवि॑णम् । \newline
53. अ॒स्मासु॒ द्रवि॑ण॒म् द्रवि॑ण म॒स्मा स्व॒स्मासु॒ द्रवि॑णम् धेहि धेहि॒ द्रवि॑ण म॒स्मा स्व॒स्मासु॒ द्रवि॑णम् धेहि । \newline
54. द्रवि॑णम् धेहि धेहि॒ द्रवि॑ण॒म् द्रवि॑णम् धेहि चि॒त्रम् चि॒त्रम् धे॑हि॒ द्रवि॑ण॒म् द्रवि॑णम् धेहि चि॒त्रम् । \newline
55. धे॒हि॒ चि॒त्रम् चि॒त्रम् धे॑हि धेहि चि॒त्रम् । \newline
56. चि॒त्रमिति॑ चि॒त्रम् । \newline
57. यथा॑ होतर्. होत॒र् यथा॒ यथा॑ होत॒र् मनु॑षो॒ मनु॑षो होत॒र् यथा॒ यथा॑ होत॒र् मनु॑षः । \newline
58. हो॒त॒र् मनु॑षो॒ मनु॑षो होतर्. होत॒र् मनु॑षो दे॒वता॑ता दे॒वता॑ता॒ मनु॑षो होतर्. होत॒र् मनु॑षो दे॒वता॑ता । \newline
59. मनु॑षो दे॒वता॑ता दे॒वता॑ता॒ मनु॑षो॒ मनु॑षो दे॒वता॑ता य॒ज्ञेभि॑र् य॒ज्ञेभि॑र् दे॒वता॑ता॒ मनु॑षो॒ मनु॑षो दे॒वता॑ता य॒ज्ञेभिः॑ । \newline
\pagebreak
\markright{ TS 4.3.13.3  \hfill https://www.vedavms.in \hfill}

\section{ TS 4.3.13.3 }

\textbf{TS 4.3.13.3 } \newline
\textbf{Samhita Paata} \newline

दे॒वता॑ता य॒ज्ञेभिः॑ सूनो सहसो॒ यजा॑सि । ए॒वा नो॑ अ॒द्य स॑म॒ना स॑मा॒नानु॒-शन्न॑ग्न उश॒तो य॑क्षि दे॒वान् ॥ अ॒ग्निमी॑डे पु॒रोहि॑तं ॅय॒ज्ञ्स्य॑ दे॒वमृ॒त्विजं᳚ । होता॑रꣳ रत्न॒धात॑मं ॥ वृषा॑ सोम द्यु॒माꣳ अ॑सि॒ वृषा॑ देव॒ वृष॑व्रतः । वृषा॒ धर्मा॑णि दधिषे ॥ सान्त॑पना इ॒दꣳ ह॒विर्मरु॑त॒स्तज्जु॑जुष्टन । यु॒ष्माको॒ती रि॑शादसः ॥ यो नो॒ मर्तो॑ वसवो दुर्.हृणा॒युस्ति॒रः स॒त्यानि॑ मरुतो॒ - [  ] \newline

\textbf{Pada Paata} \newline

दे॒वता॒तेति॑ दे॒व - ता॒ता॒ । य॒ज्ञेभिः॑ । सू॒नो॒ इति॑ । स॒ह॒सः॒ । यजा॑सि ॥ ए॒वा । नः॒ । अ॒द्य । स॒म॒ना । स॒मा॒नान् । उ॒शन्न् । अ॒ग्ने॒ । उ॒श॒तः । य॒क्षि॒ । दे॒वान् ॥ अ॒ग्निम् । ई॒डे॒ । पु॒रोहि॑त॒मिति॑ पु॒रः - हि॒त॒म् । य॒ज्ञ्स्य॑ । दे॒वम् । ऋ॒त्विज᳚म् ॥ होता॑रम् । र॒त्न॒धात॑म॒मिति॑ रत्न - धात॑मम् ॥ वृषा᳚ । सो॒म॒ । द्यु॒मानिति॑ द्यु-मान् । अ॒सि॒ । वृषा᳚ । दे॒व॒ । वृष॑व्रत॒ इति॒ वृष॑ - व्र॒तः॒ ॥ वृषा᳚ । धर्मा॑णि । द॒धि॒षे॒ ॥ सान्त॑पना॒ इति॒ सां - त॒प॒नाः॒ । इ॒दम् । ह॒विः । मरु॑तः । तत् । जु॒जु॒ष्ट॒न॒ ॥ यु॒ष्माक॑ । ऊ॒ती । रि॒शा॒द॒स॒ इति॑ रिश - अ॒द॒सः॒ ॥ यः । नः॒ । मर्तः॑ । व॒स॒वः॒ । दु॒र्॒.हृ॒णा॒युरिति॑ दुः - हृ॒णा॒युः । ति॒रः । स॒त्यानि॑ । म॒रु॒तः॒ ।  \newline


\textbf{Krama Paata} \newline

दे॒वता॑ता य॒ज्ञेभिः॑ । दे॒वता॒तेति॑ दे॒व - ता॒ता॒ । य॒ज्ञेभिः॑ सूनो । सू॒नो॒ स॒ह॒सः॒ । सू॒नो॒ इति॑ सूनो । स॒ह॒सो॒ यजा॑सि । यजा॒सीति॒ यजा॑सि ॥ ए॒वा नः॑ । नो॒ अ॒द्य । अ॒द्य स॑म॒ना । स॒म॒ना स॑मा॒नान् । स॒मा॒नानु॒शन्न् । उ॒शन्न॑ग्ने । अ॒ग्न॒ उ॒श॒तः । उ॒श॒तो य॑क्षि । य॒क्षि॒ दे॒वान् । दे॒वानिति॑ दे॒वान् ॥ अ॒ग्निमी॑डे । ई॒डे॒ पु॒रोहि॑तम् । पु॒रोहि॑तं ॅय॒ज्ञ्स्य॑ । पु॒रोहि॑त॒मिति॑ पु॒रः - हि॒त॒म् । य॒ज्ञ्स्य॑ दे॒वम् । दे॒वमृ॒त्विजम्᳚ । ऋ॒त्विज॒मित्यृ॒त्विज᳚म् ॥ होता॑रꣳ रत्न॒धात॑मम् । र॒त्न॒धात॑म॒मिति॑ रत्न - धात॑मम् ॥ वृषा॑ सोम । सो॒म॒ द्यु॒मान् । द्यु॒माꣳ अ॑सि । द्यु॒मानिति॑ द्यु - मान् । अ॒सि॒ वृषा᳚ । वृषा॑ देव । दे॒व॒ वृष॑व्रतः । वृष॑व्रत॒ इति॒ वृष॑ - व्र॒तः॒ ॥ वृषा॒ धर्मा॑णि । धर्मा॑णि दधिषे । द॒धि॒ष॒ इति॑ दधिषे ॥ सान्त॑पना इ॒दम् । सान्त॑पना॒ इति॒ साम् - त॒प॒नाः॒ । इ॒दꣳ ह॒विः । ह॒विर् मरु॑तः । मरु॑त॒स्तत् । तज् जु॑जुष्टन । जु॒जु॒ष्ट॒नेति॑ जुजुष्टन ॥ यु॒ष्माको॒ती । ऊ॒ती रि॑शादसः । रि॒शा॒द॒स॒ इति॑ रिश - अ॒द॒सः॒ ॥ यो नः॑ । नो॒ मर्तः॑ । मर्तो॑ वसवः । व॒स॒वो॒ दु॒र्॒.हृ॒णा॒युः । दु॒र्॒.हृ॒णा॒युस्ति॒रः । दु॒र्॒.हृ॒णा॒युरिति॑ दुः - हृ॒णा॒युः । ति॒रः स॒त्यानि॑ । स॒त्यानि॑ मरुतः । म॒रु॒तो॒ जिघाꣳ॑सात् । \newline

\textbf{Jatai Paata} \newline

1. दे॒वता॑ता य॒ज्ञेभि॑र् य॒ज्ञेभि॑र् दे॒वता॑ता दे॒वता॑ता य॒ज्ञेभिः॑ । \newline
2. दे॒वता॒तेति॑ दे॒व - ता॒ता॒ । \newline
3. य॒ज्ञेभिः॑ सूनो सूनो य॒ज्ञेभि॑र् य॒ज्ञेभिः॑ सूनो । \newline
4. सू॒नो॒ स॒ह॒सः॒ स॒ह॒सः॒ सू॒नो॒ सू॒नो॒ स॒ह॒सः॒ । \newline
5. सू॒नो॒ इति॑ सूनो । \newline
6. स॒ह॒सो॒ यजा॑सि॒ यजा॑सि सहसः सहसो॒ यजा॑सि । \newline
7. यजा॒सीति॒ यजा॑सि । \newline
8. ए॒वा नो॑ न ए॒वैवा नः॑ । \newline
9. नो॒ अ॒द्याद्य नो॑ नो अ॒द्य । \newline
10. अ॒द्य स॑म॒ना स॑म॒ना ऽद्याद्य स॑म॒ना । \newline
11. स॒म॒ना स॑मा॒नान् थ्स॑मा॒नान् थ्स॑म॒ना स॑म॒ना स॑मा॒नान् । \newline
12. स॒मा॒ना-नु॒शन् नु॒शन् थ्स॑मा॒नान् थ्स॑मा॒ना-नु॒शन्न् । \newline
13. उ॒शन् न॑ग्ने अग्न उ॒शन् नु॒शन् न॑ग्ने । \newline
14. अ॒ग्न॒ उ॒श॒त उ॑श॒तो अ॑ग्ने अग्न उश॒तः । \newline
15. उ॒श॒तो य॑क्षि यक्ष्युश॒त उ॑श॒तो य॑क्षि । \newline
16. य॒क्षि॒ दे॒वान् दे॒वान्. य॑क्षि यक्षि दे॒वान् । \newline
17. दे॒वानिति॑ दे॒वान् । \newline
18. अ॒ग्नि मी॑ड ईडे अ॒ग्नि म॒ग्नि मी॑डे । \newline
19. ई॒डे॒ पु॒रोहि॑तम् पु॒रोहि॑त मीड ईडे पु॒रोहि॑तम् । \newline
20. पु॒रोहि॑तं ॅय॒ज्ञ्स्य॑ य॒ज्ञ्स्य॑ पु॒रोहि॑तम् पु॒रोहि॑तं ॅय॒ज्ञ्स्य॑ । \newline
21. पु॒रोहि॑त॒मिति॑ पु॒रः - हि॒त॒म् । \newline
22. य॒ज्ञ्स्य॑ दे॒वम् दे॒वं ॅय॒ज्ञ्स्य॑ य॒ज्ञ्स्य॑ दे॒वम् । \newline
23. दे॒व मृ॒त्विज॑ मृ॒त्विज॑म् दे॒वम् दे॒व मृ॒त्विज᳚म् । \newline
24. ऋ॒त्विज॒मित्यृ॒त्विज᳚म् । \newline
25. होता॑रꣳ रत्न॒धात॑मꣳ रत्न॒धात॑मꣳ॒॒ होता॑रꣳ॒॒ होता॑रꣳ रत्न॒धात॑मम् । \newline
26. र॒त्न॒धात॑म॒मिति॑ रत्न - धात॑मम् । \newline
27. वृषा॑ सोम सोम॒ वृषा॒ वृषा॑ सोम । \newline
28. सो॒म॒ द्यु॒मान् द्यु॒मान् थ्सो॑म सोम द्यु॒मान् । \newline
29. द्यु॒माꣳ अ॑स्यसि द्यु॒मान् द्यु॒माꣳ अ॑सि । \newline
30. द्यु॒मानिति॑ द्यु - मान् । \newline
31. अ॒सि॒ वृषा॒ वृषा᳚ ऽस्यसि॒ वृषा᳚ । \newline
32. वृषा॑ देव देव॒ वृषा॒ वृषा॑ देव । \newline
33. दे॒व॒ वृष॑व्रतो॒ वृष॑व्रतो देव देव॒ वृष॑व्रतः । \newline
34. वृष॑व्रत॒ इति॒ वृष॑ - व्र॒तः॒ । \newline
35. वृषा॒ धर्मा॑णि॒ धर्मा॑णि॒ वृषा॒ वृषा॒ धर्मा॑णि । \newline
36. धर्मा॑णि दधिषे दधिषे॒ धर्मा॑णि॒ धर्मा॑णि दधिषे । \newline
37. द॒धि॒ष॒ इति॑ दधिषे । \newline
38. सान्त॑पना इ॒द मि॒दꣳ सान्त॑पनाः॒ सान्त॑पना इ॒दम् । \newline
39. सान्त॑पना॒ इति॒ सां - त॒प॒नाः॒ । \newline
40. इ॒दꣳ ह॒विर्. ह॒वि रि॒द मि॒दꣳ ह॒विः । \newline
41. ह॒विर् मरु॑तो॒ मरु॑तो ह॒विर्. ह॒विर् मरु॑तः । \newline
42. मरु॑त॒ स्तत् तन् मरु॑तो॒ मरु॑त॒ स्तत् । \newline
43. तज् जु॑जुष्टन जुजुष्टन॒ तत् तज् जु॑जुष्टन । \newline
44. जु॒जु॒ष्ट॒नेति॑ जुजुष्टन । \newline
45. यु॒ष्मा को॒त्यू॑ती यु॒ष्माक॑ यु॒ष्माको॒ती । \newline
46. ऊ॒ती रि॑शादसो रिशादस ऊ॒त्यू॑ती रि॑शादसः । \newline
47. रि॒शा॒द॒स॒ इति॑ रिश - अ॒द॒सः॒ । \newline
48. यो नो॑ नो॒ यो यो नः॑ । \newline
49. नो॒ मर्तो॒ मर्तो॑ नो नो॒ मर्तः॑ । \newline
50. मर्तो॑ वसवो वसवो॒ मर्तो॒ मर्तो॑ वसवः । \newline
51. व॒स॒वो॒ दु॒र्॒.हृ॒णा॒युर् दु॑र्.हृणा॒युर् व॑सवो वसवो दुर्.हृणा॒युः । \newline
52. दु॒र्॒.हृ॒णा॒यु स्ति॒र स्ति॒रो दु॑र्.हृणा॒युर् दु॑र्.हृणा॒यु स्ति॒रः । \newline
53. दु॒र्॒.हृ॒णा॒युरिति॑ दुः - हृ॒णा॒युः । \newline
54. ति॒रः स॒त्यानि॑ स॒त्यानि॑ ति॒र स्ति॒रः स॒त्यानि॑ । \newline
55. स॒त्यानि॑ मरुतो मरुतः स॒त्यानि॑ स॒त्यानि॑ मरुतः । \newline
56. म॒रु॒तो॒ जिघाꣳ॑सा॒ज् जिघाꣳ॑सान् मरुतो मरुतो॒ जिघाꣳ॑सात् । \newline

\textbf{Ghana Paata } \newline

1. दे॒वता॑ता य॒ज्ञेभि॑र् य॒ज्ञेभि॑र् दे॒वता॑ता दे॒वता॑ता य॒ज्ञेभिः॑ सूनो सूनो य॒ज्ञेभि॑र् दे॒वता॑ता दे॒वता॑ता य॒ज्ञेभिः॑ सूनो । \newline
2. दे॒वता॒तेति॑ दे॒व - ता॒ता॒ । \newline
3. य॒ज्ञेभिः॑ सूनो सूनो य॒ज्ञेभि॑र् य॒ज्ञेभिः॑ सूनो सहसः सहसः सूनो य॒ज्ञेभि॑र् य॒ज्ञेभिः॑ सूनो सहसः । \newline
4. सू॒नो॒ स॒ह॒सः॒ स॒ह॒सः॒ सू॒नो॒ सू॒नो॒ स॒ह॒सो॒ यजा॑सि॒ यजा॑सि सहसः सूनो सूनो सहसो॒ यजा॑सि । \newline
5. सू॒नो॒ इति॑ सूनो । \newline
6. स॒ह॒सो॒ यजा॑सि॒ यजा॑सि सहसः सहसो॒ यजा॑सि । \newline
7. यजा॒सीति॒ यजा॑सि । \newline
8. ए॒वा नो॑ न ए॒वैवा नो॑ अ॒द्याद्य न॑ ए॒वैवा नो॑ अ॒द्य । \newline
9. नो॒ अ॒द्याद्य नो॑ नो अ॒द्य स॑म॒ना स॑म॒ना ऽद्य नो॑ नो अ॒द्य स॑म॒ना । \newline
10. अ॒द्य स॑म॒ना स॑म॒ना ऽद्याद्य स॑म॒ना स॑मा॒नान् थ्स॑मा॒नान् थ्स॑म॒ना ऽद्याद्य स॑म॒ना स॑मा॒नान् । \newline
11. स॒म॒ना स॑मा॒नान् थ्स॑मा॒नान् थ्स॑म॒ना स॑म॒ना स॑मा॒ना नु॒शन् नु॒शन् थ्स॑मा॒नान् थ्स॑म॒ना स॑म॒ना स॑मा॒ना नु॒शन्न् । \newline
12. स॒मा॒ना नु॒शन् नु॒शन् थ्स॑मा॒नान् थ्स॑मा॒ना नु॒शन् न॑ग्ने अग्न उ॒शन् थ्स॑मा॒नान् थ्स॑मा॒ना नु॒शन् न॑ग्ने । \newline
13. उ॒शन् न॑ग्ने अग्न उ॒शन् नु॒शन् न॑ग्न उश॒त उ॑श॒तो अ॑ग्न उ॒शन् नु॒शन् न॑ग्न उश॒तः । \newline
14. अ॒ग्न॒ उ॒श॒त उ॑श॒तो अ॑ग्ने अग्न उश॒तो य॑क्षि यक्ष्युश॒तो अ॑ग्ने अग्न उश॒तो य॑क्षि । \newline
15. उ॒श॒तो य॑क्षि यक्ष्युश॒त उ॑श॒तो य॑क्षि दे॒वान् दे॒वान्. य॑क्ष्युश॒त उ॑श॒तो य॑क्षि दे॒वान् । \newline
16. य॒क्षि॒ दे॒वान् दे॒वान्. य॑क्षि यक्षि दे॒वान् । \newline
17. दे॒वानिति॑ दे॒वान् । \newline
18. अ॒ग्नि मी॑ड ईडे अ॒ग्नि म॒ग्नि मी॑डे पु॒रोहि॑तम् पु॒रोहि॑त मीडे अ॒ग्नि म॒ग्नि मी॑डे पु॒रोहि॑तम् । \newline
19. ई॒डे॒ पु॒रोहि॑तम् पु॒रोहि॑त मीड ईडे पु॒रोहि॑तं ॅय॒ज्ञ्स्य॑ य॒ज्ञ्स्य॑ पु॒रोहि॑त मीड ईडे पु॒रोहि॑तं ॅय॒ज्ञ्स्य॑ । \newline
20. पु॒रोहि॑तं ॅय॒ज्ञ्स्य॑ य॒ज्ञ्स्य॑ पु॒रोहि॑तम् पु॒रोहि॑तं ॅय॒ज्ञ्स्य॑ दे॒वम् दे॒वं ॅय॒ज्ञ्स्य॑ पु॒रोहि॑तम् पु॒रोहि॑तं ॅय॒ज्ञ्स्य॑ दे॒वम् । \newline
21. पु॒रोहि॑त॒मिति॑ पु॒रः - हि॒त॒म् । \newline
22. य॒ज्ञ्स्य॑ दे॒वम् दे॒वं ॅय॒ज्ञ्स्य॑ य॒ज्ञ्स्य॑ दे॒व मृ॒त्विज॑ मृ॒त्विज॑म् दे॒वं ॅय॒ज्ञ्स्य॑ य॒ज्ञ्स्य॑ दे॒व मृ॒त्विज᳚म् । \newline
23. दे॒व मृ॒त्विज॑ मृ॒त्विज॑म् दे॒वम् दे॒व मृ॒त्विज᳚म् । \newline
24. ऋ॒त्विज॒मित्यृ॒त्विज᳚म् । \newline
25. होता॑रꣳ रत्न॒धात॑मꣳ रत्न॒धात॑मꣳ॒॒ होता॑रꣳ॒॒ होता॑रꣳ रत्न॒धात॑मम् । \newline
26. र॒त्न॒धात॑म॒मिति॑ रत्न - धात॑मम् । \newline
27. वृषा॑ सोम सोम॒ वृषा॒ वृषा॑ सोम द्यु॒मान् द्यु॒मान् थ्सो॑म॒ वृषा॒ वृषा॑ सोम द्यु॒मान् । \newline
28. सो॒म॒ द्यु॒मान् द्यु॒मान् थ्सो॑म सोम द्यु॒माꣳ अ॑स्यसि द्यु॒मान् थ्सो॑म सोम द्यु॒माꣳ अ॑सि । \newline
29. द्यु॒माꣳ अ॑स्यसि द्यु॒मान् द्यु॒माꣳ अ॑सि॒ वृषा॒ वृषा॑ ऽसि द्यु॒मान् द्यु॒माꣳ अ॑सि॒ वृषा᳚ । \newline
30. द्यु॒मानिति॑ द्यु - मान् । \newline
31. अ॒सि॒ वृषा॒ वृषा᳚ ऽस्यसि॒ वृषा॑ देव देव॒ वृषा᳚ ऽस्यसि॒ वृषा॑ देव । \newline
32. वृषा॑ देव देव॒ वृषा॒ वृषा॑ देव॒ वृष॑व्रतो॒ वृष॑व्रतो देव॒ वृषा॒ वृषा॑ देव॒ वृष॑व्रतः । \newline
33. दे॒व॒ वृष॑व्रतो॒ वृष॑व्रतो देव देव॒ वृष॑व्रतः । \newline
34. वृष॑व्रत॒ इति॒ वृष॑ - व्र॒तः॒ । \newline
35. वृषा॒ धर्मा॑णि॒ धर्मा॑णि॒ वृषा॒ वृषा॒ धर्मा॑णि दधिषे दधिषे॒ धर्मा॑णि॒ वृषा॒ वृषा॒ धर्मा॑णि दधिषे । \newline
36. धर्मा॑णि दधिषे दधिषे॒ धर्मा॑णि॒ धर्मा॑णि दधिषे । \newline
37. द॒धि॒ष॒ इति॑ दधिषे । \newline
38. सान्त॑पना इ॒द मि॒दꣳ सान्त॑पनाः॒ सान्त॑पना इ॒दꣳ ह॒विर्. ह॒वि रि॒दꣳ सान्त॑पनाः॒ सान्त॑पना इ॒दꣳ ह॒विः । \newline
39. सान्त॑पना॒ इति॒ सां - त॒प॒नाः॒ । \newline
40. इ॒दꣳ ह॒विर्. ह॒वि रि॒द मि॒दꣳ ह॒विर् मरु॑तो॒ मरु॑तो ह॒वि रि॒द मि॒दꣳ ह॒विर् मरु॑तः । \newline
41. ह॒विर् मरु॑तो॒ मरु॑तो ह॒विर्. ह॒विर् मरु॑त॒ स्तत् तन् मरु॑तो ह॒विर्. ह॒विर् मरु॑त॒ स्तत् । \newline
42. मरु॑त॒ स्तत् तन् मरु॑तो॒ मरु॑त॒ स्तज् जु॑जुष्टन जुजुष्टन॒ तन् मरु॑तो॒ मरु॑त॒ स्तज् जु॑जुष्टन । \newline
43. तज् जु॑जुष्टन जुजुष्टन॒ तत् तज् जु॑जुष्टन । \newline
44. जु॒जु॒ष्ट॒नेति॑ जुजुष्टन । \newline
45. यु॒ष्माको॒ त्यू॑ती यु॒ष्माक॑ यु॒ष्माको॒ती रि॑शादसो रिशादस ऊ॒ती यु॒ष्माक॑ यु॒ष्माको॒ती रि॑शादसः । \newline
46. ऊ॒ती रि॑शादसो रिशादस ऊ॒त्यू॑ती रि॑शादसः । \newline
47. रि॒शा॒द॒स॒ इति॑ रिश - अ॒द॒सः॒ । \newline
48. यो नो॑ नो॒ यो यो नो॒ मर्तो॒ मर्तो॑ नो॒ यो यो नो॒ मर्तः॑ । \newline
49. नो॒ मर्तो॒ मर्तो॑ नो नो॒ मर्तो॑ वसवो वसवो॒ मर्तो॑ नो नो॒ मर्तो॑ वसवः । \newline
50. मर्तो॑ वसवो वसवो॒ मर्तो॒ मर्तो॑ वसवो दुर्.हृणा॒युर् दु॑र्.हृणा॒युर् व॑सवो॒ मर्तो॒ मर्तो॑ वसवो दुर्.हृणा॒युः । \newline
51. व॒स॒वो॒ दु॒र्॒.हृ॒णा॒युर् दु॑र्.हृणा॒युर् व॑सवो वसवो दुर्.हृणा॒यु स्ति॒र स्ति॒रो दु॑र्.हृणा॒युर् व॑सवो वसवो दुर्.हृणा॒यु स्ति॒रः । \newline
52. दु॒र्॒.हृ॒णा॒यु स्ति॒र स्ति॒रो दु॑र्.हृणा॒युर् दु॑र्.हृणा॒यु स्ति॒रः स॒त्यानि॑ स॒त्यानि॑ ति॒रो दु॑र्.हृणा॒युर् दु॑र्.हृणा॒यु स्ति॒रः स॒त्यानि॑ । \newline
53. दु॒र्॒.हृ॒णा॒युरिति॑ दुः - हृ॒णा॒युः । \newline
54. ति॒रः स॒त्यानि॑ स॒त्यानि॑ ति॒र स्ति॒रः स॒त्यानि॑ मरुतो मरुतः स॒त्यानि॑ ति॒र स्ति॒रः स॒त्यानि॑ मरुतः । \newline
55. स॒त्यानि॑ मरुतो मरुतः स॒त्यानि॑ स॒त्यानि॑ मरुतो॒ जिघाꣳ॑सा॒ज् जिघाꣳ॑सान् मरुतः स॒त्यानि॑ स॒त्यानि॑ मरुतो॒ जिघाꣳ॑सात् । \newline
56. म॒रु॒तो॒ जिघाꣳ॑सा॒ज् जिघाꣳ॑सान् मरुतो मरुतो॒ जिघाꣳ॑सात् । \newline
\pagebreak
\markright{ TS 4.3.13.4  \hfill https://www.vedavms.in \hfill}

\section{ TS 4.3.13.4 }

\textbf{TS 4.3.13.4 } \newline
\textbf{Samhita Paata} \newline

जिघाꣳ॑सात् । द्रु॒हः पाशं॒ प्रति॒ स मु॑चीष्ट॒ तपि॑ष्ठेन॒ तप॑सा हन्तना॒ तं ॥ सं॒ॅव॒थ्स॒रीणा॑ म॒रुतः॑ स्व॒र्का उ॑रु॒क्षयाः॒ सग॑णा॒ मानु॑षेषु । ते᳚ऽस्मत् पाशा॒न् प्र मु॑ञ्च॒न्त्वꣳह॑सः सान्तप॒ना म॑दि॒रा मा॑दयि॒ष्णवः॑ ॥ पि॒प्री॒हि दे॒वाꣳ उ॑श॒तो य॑विष्ठ वि॒द्वाꣳ ऋ॒तूꣳर्.ऋ॑तुपते यजे॒ह । ये दैव्या॑ ऋ॒त्विज॒स्तेभि॑रग्ने॒ त्वꣳ होतॄ॑णाम॒स्याय॑जिष्ठः ॥ अग्ने॒ यद॒द्य वि॒शो अ॑द्ध्वरस्य होतः॒ पाव॑क - [  ] \newline

\textbf{Pada Paata} \newline

जिघाꣳ॑सात् ॥ द्रु॒हः । पाश᳚म् । प्रतीति॑ । सः । मु॒ची॒ष्ट॒ । तपि॑ष्ठेन । तप॑सा । ह॒न्त॒न॒ । तम् ॥ सं॒ॅव॒थ्स॒रीणा॒ इति॑ सं - व॒थ्स॒रीणाः᳚ । म॒रुतः॑ । स्व॒र्का इति॑ सु - अ॒र्काः । उ॒रु॒क्षया॒ इत्यु॑रु - क्षयाः᳚ । सग॑णा॒ इति॒ स - ग॒णाः॒ । मानु॑षेषु ॥ ते । अ॒स्मत् । पाशान्॑ । प्रेति॑ । मु॒ञ्च॒न्तु॒ । अꣳह॑सः । सा॒न्त॒प॒ना इति॑ सां - त॒प॒नाः । म॒दि॒राः । मा॒द॒यि॒ष्णवः॑ ॥ पि॒प्री॒हि । दे॒वान् । उ॒श॒तः । य॒वि॒ष्ठ॒ । वि॒द्वान् । ऋ॒तून् । ऋ॒तु॒प॒त॒ इत्यृ॑तु-प॒ते॒ । य॒ज॒ । इ॒ह ॥ ये । दैव्याः᳚ । ऋ॒त्विजः॑ । तेभिः॑ । अ॒ग्ने॒ । त्वम् । होतॄ॑णाम् । अ॒सि॒ । आय॑जिष्ठ॒ इत्या - य॒जि॒ष्ठः॒ ॥ अग्ने᳚ । यत् । अ॒द्य । वि॒शः । अ॒द्ध्व॒र॒स्य॒ । हो॒तः॒ । पाव॑क ।  \newline


\textbf{Krama Paata} \newline

जिघाꣳ॑सा॒दिति॒ जिघाꣳ॑सात् ॥ द्रु॒हः पाश᳚म् । पाश॒म् प्रति॑ । प्रति॒ सः । स मु॑चीष्ट । मु॒ची॒ष्ट॒ तपि॑ष्ठेन । तपि॑ष्ठेन॒ तप॑सा । तप॑सा हन्तन । ह॒न्त॒ना॒ तम् । तमिति॒ तम् ॥ स॒म्ॅव॒थ्स॒रीणा॑ म॒रुतः॑ । स॒म्ॅव॒थ्स॒रीणा॒ इति॑ सम् - व॒थ्स॒रीणाः᳚ । म॒रुतः॑ स्व॒र्काः । स्व॒र्का उ॑रु॒क्षयाः᳚ । स्व॒र्का इति॑ सु - अ॒र्काः । उ॒रु॒क्षयाः॒ सग॑णाः । उ॒रु॒क्षया॒ इत्यु॑रु - क्षयाः᳚ । सग॑णा॒ मानु॑षेषु । सग॑णा॒ इति॒ स - ग॒णाः॒ । मानु॑षे॒ष्विति॒ मानु॑षेषु ॥ ते᳚ऽस्मत् । अ॒स्मत् पाशान्॑ । पाशा॒न् प्र । प्र मु॑ञ्चन्तु । मु॒ञ्च॒न्त्वꣳह॑सः । अꣳह॑सः सान्तप॒नाः । सा॒न्त॒प॒ना म॑दि॒राः । सा॒न्त॒प॒ना इति॑ साम् - त॒प॒नाः । म॒दि॒रा मा॑दयि॒ष्णवः॑ । मा॒द॒यि॒ष्णव॒ इति॑ मादयि॒ष्णवः॑ ॥ पि॒प्री॒हि दे॒वान् । दे॒वाꣳ उ॑श॒तः । उ॒श॒तो य॑विष्ठ । य॒वि॒ष्ठ॒ वि॒द्वान् । वि॒द्वाꣳर्. ऋ॒तून् । ऋ॒तूꣳ ऋ॑तुपते । ऋ॒तु॒प॒ते॒ य॒ज॒ । ऋ॒तु॒प॒त॒ इत्यृ॑तु - प॒ते॒ । य॒जे॒ह । इ॒हेती॒ह ॥ ये दैव्याः᳚ । दैव्या॑ ऋ॒त्विजः॑ । ऋ॒त्विज॒स्तेभिः॑ । तेभि॑रग्ने । अ॒ग्ने॒ त्वम् । त्वꣳ होतॄ॑णाम् । होतॄ॑णामसि । अ॒स्याय॑जिष्ठः । आय॑जिष्ठ॒ इत्या - य॒जि॒ष्ठः॒ ॥ अग्ने॒ यत् । यद॒द्य । अ॒द्य वि॒शः । वि॒शो अ॑द्ध्वरस्य । अ॒द्ध्व॒र॒स्य॒ हो॒तः॒ । हो॒तः॒ पाव॑क । पाव॑क शोचे । \newline

\textbf{Jatai Paata} \newline

1. जिघाꣳ॑सा॒दिति॒ जिघाꣳ॑सात् । \newline
2. द्रु॒हः पाश॒म् पाश॑म् द्रु॒हो द्रु॒हः पाश᳚म् । \newline
3. पाश॒म् प्रति॒ प्रति॒ पाश॒म् पाश॒म् प्रति॑ । \newline
4. प्रति॒ स स प्रति॒ प्रति॒ सः । \newline
5. स मु॑चीष्ट मुचीष्ट॒ स स मु॑चीष्ट । \newline
6. मु॒ची॒ष्ट॒ तपि॑ष्ठेन॒ तपि॑ष्ठेन मुचीष्ट मुचीष्ट॒ तपि॑ष्ठेन । \newline
7. तपि॑ष्ठेन॒ तप॑सा॒ तप॑सा॒ तपि॑ष्ठेन॒ तपि॑ष्ठेन॒ तप॑सा । \newline
8. तप॑सा हन्तना हन्तन॒ तप॑सा॒ तप॑सा हन्तन । \newline
9. ह॒न्त॒ना॒ तम् तꣳ ह॑न्तन हन्तना॒ तम् । \newline
10. तमिति॒ तम् । \newline
11. सं॒ॅव॒थ्स॒रीणा॑ म॒रुतो॑ म॒रुतः॑ संॅवथ्स॒रीणाः᳚ संॅवथ्स॒रीणा॑ म॒रुतः॑ । \newline
12. सं॒ॅव॒थ्स॒रीणा॒ इति॑ सं - व॒थ्स॒रीणाः᳚ । \newline
13. म॒रुतः॑ स्व॒र्काः स्व॒र्का म॒रुतो॑ म॒रुतः॑ स्व॒र्काः । \newline
14. स्व॒र्का उ॑रु॒क्षया॑ उरु॒क्षयाः᳚ स्व॒र्काः स्व॒र्का उ॑रु॒क्षयाः᳚ । \newline
15. स्व॒र्का इति॑ सु - अ॒र्काः । \newline
16. उ॒रु॒क्षयाः॒ सग॑णाः॒ सग॑णा उरु॒क्षया॑ उरु॒क्षयाः॒ सग॑णाः । \newline
17. उ॒रु॒क्षया॒ इत्यु॑रु - क्षयाः᳚ । \newline
18. सग॑णा॒ मानु॑षेषु॒ मानु॑षेषु॒ सग॑णाः॒ सग॑णा॒ मानु॑षेषु । \newline
19. सग॑णा॒ इति॒ स - ग॒णाः॒ । \newline
20. मानु॑षे॒ष्विति॒ मानु॑षेषु । \newline
21. ते᳚ ऽस्म द॒स्मत् ते ते᳚ ऽस्मत् । \newline
22. अ॒स्मत् पाशा॒न् पाशा॑-न॒स्म द॒स्मत् पाशान्॑ । \newline
23. पाशा॒न् प्र प्र पाशा॒न् पाशा॒न् प्र । \newline
24. प्र मु॑ञ्चन्तु मुञ्चन्तु॒ प्र प्र मु॑ञ्चन्तु । \newline
25. मु॒ञ्च॒न् त्वꣳह॑सो॒ अꣳह॑सो मुञ्चन्तु मुञ्च॒न् त्वꣳह॑सः । \newline
26. अꣳह॑सः सान्तप॒नाः सा᳚न्तप॒ना अꣳह॑सो॒ अꣳह॑सः सान्तप॒नाः । \newline
27. सा॒न्त॒प॒ना म॑दि॒रा म॑दि॒राः सा᳚न्तप॒नाः सा᳚न्तप॒ना म॑दि॒राः । \newline
28. सा॒न्त॒प॒ना इति॑ सां - त॒प॒नाः । \newline
29. म॒दि॒रा मा॑दयि॒ष्णवो॑ मादयि॒ष्णवो॑ मदि॒रा म॑दि॒रा मा॑दयि॒ष्णवः॑ । \newline
30. मा॒द॒यि॒ष्णव॒ इति॑ मादयि॒ष्णवः॑ । \newline
31. पि॒प्री॒हि दे॒वान् दे॒वान् पि॑प्री॒हि पि॑प्री॒हि दे॒वान् । \newline
32. दे॒वाꣳ उ॑श॒त उ॑श॒तो दे॒वान् दे॒वाꣳ उ॑श॒तः । \newline
33. उ॒श॒तो य॑विष्ठ यविष्ठोश॒त उ॑श॒तो य॑विष्ठ । \newline
34. य॒वि॒ष्ठ॒ वि॒द्वान्. वि॒द्वान्. य॑विष्ठ यविष्ठ वि॒द्वान् । \newline
35. वि॒द्वाꣳ ऋ॒तूꣳर्. ऋ॒तून्. वि॒द्वान्. वि॒द्वाꣳ ऋ॒तून् । \newline
36. ऋ॒तूꣳर्. ऋ॑तुपत ऋतुपत ऋ॒तूꣳर्. ऋ॒तूꣳर्. ऋ॑तुपते । \newline
37. ऋ॒तु॒प॒ते॒ य॒ज॒ य॒ज॒ र्‌तु॒प॒त॒ ऋ॒तु॒प॒ते॒ य॒ज॒ । \newline
38. ऋ॒तु॒प॒त॒ इत्यृ॑तु - प॒ते॒ । \newline
39. य॒जे॒ हेह य॑ज यजे॒ह । \newline
40. इ॒हेती॒ह । \newline
41. ये दैव्या॒ दैव्या॒ ये ये दैव्याः᳚ । \newline
42. दैव्या॑ ऋ॒त्विज॑ ऋ॒त्विजो॒ दैव्या॒ दैव्या॑ ऋ॒त्विजः॑ । \newline
43. ऋ॒त्विज॒ स्तेभि॒ स्तेभि॑र्. ऋ॒त्विज॑ ऋ॒त्विज॒ स्तेभिः॑ । \newline
44. तेभि॑ रग्ने अग्ने॒ तेभि॒ स्तेभि॑ रग्ने । \newline
45. अ॒ग्ने॒ त्वम् त्व म॑ग्ने अग्ने॒ त्वम् । \newline
46. त्वꣳ होतॄ॑णाꣳ॒॒ होतॄ॑णा॒म् त्वम् त्वꣳ होतॄ॑णाम् । \newline
47. होतॄ॑णा मस्यसि॒ होतॄ॑णाꣳ॒॒ होतॄ॑णा मसि । \newline
48. अ॒स्या य॑जिष्ठ॒ आय॑जिष्ठो ऽस्य॒स्या य॑जिष्ठः । \newline
49. आय॑जिष्ठ॒ इत्या - य॒जि॒ष्ठः॒ । \newline
50. अग्ने॒ यद् यदग्ने ऽग्ने॒ यत् । \newline
51. यद॒ द्याद्य यद् यद॒द्य । \newline
52. अ॒द्य वि॒शो वि॒शो अ॒द्याद्य वि॒शः । \newline
53. वि॒शो अ॑द्ध्वरस्या द्ध्वरस्य वि॒शो वि॒शो अ॑द्ध्वरस्य । \newline
54. अ॒द्ध्व॒र॒स्य॒ हो॒त॒र्॒. हो॒त॒ र॒द्ध्व॒र॒स्या॒ द्ध्व॒र॒स्य॒ हो॒तः॒ । \newline
55. हो॒तः॒ पाव॑क॒ पाव॑क होतर्. होतः॒ पाव॑क । \newline
56. पाव॑क शोचे शोचे॒ पाव॑क॒ पाव॑क शोचे । \newline

\textbf{Ghana Paata } \newline

1. जिघाꣳ॑सा॒दिति॒ जिघाꣳ॑सात् । \newline
2. द्रु॒हः पाश॒म् पाश॑म् द्रु॒हो द्रु॒हः पाश॒म् प्रति॒ प्रति॒ पाश॑म् द्रु॒हो द्रु॒हः पाश॒म् प्रति॑ । \newline
3. पाश॒म् प्रति॒ प्रति॒ पाश॒म् पाश॒म् प्रति॒ स स प्रति॒ पाश॒म् पाश॒म् प्रति॒ सः । \newline
4. प्रति॒ स स प्रति॒ प्रति॒ स मु॑चीष्ट मुचीष्ट॒ स प्रति॒ प्रति॒ स मु॑चीष्ट । \newline
5. स मु॑चीष्ट मुचीष्ट॒ स स मु॑चीष्ट॒ तपि॑ष्ठेन॒ तपि॑ष्ठेन मुचीष्ट॒ स स मु॑चीष्ट॒ तपि॑ष्ठेन । \newline
6. मु॒ची॒ष्ट॒ तपि॑ष्ठेन॒ तपि॑ष्ठेन मुचीष्ट मुचीष्ट॒ तपि॑ष्ठेन॒ तप॑सा॒ तप॑सा॒ तपि॑ष्ठेन मुचीष्ट मुचीष्ट॒ तपि॑ष्ठेन॒ तप॑सा । \newline
7. तपि॑ष्ठेन॒ तप॑सा॒ तप॑सा॒ तपि॑ष्ठेन॒ तपि॑ष्ठेन॒ तप॑सा हन्तना हन्तन॒ तप॑सा॒ तपि॑ष्ठेन॒ तपि॑ष्ठेन॒ तप॑सा हन्तन । \newline
8. तप॑सा हन्तन हन्तन॒ तप॑सा॒ तप॑सा हन्तना॒ तम् तꣳ ह॑न्तन॒ तप॑सा॒ तप॑सा हन्तना॒ तम् । \newline
9. ह॒न्त॒ना॒ तम् तꣳ ह॑न्तन हन्तना॒ तम् । \newline
10. तमिति॒ तम् । \newline
11. सं॒ॅव॒थ्स॒रीणा॑ म॒रुतो॑ म॒रुतः॑ संॅवथ्स॒रीणाः᳚ संॅवथ्स॒रीणा॑ म॒रुतः॑ स्व॒र्काः स्व॒र्का म॒रुतः॑ संॅवथ्स॒रीणाः᳚ संॅवथ्स॒रीणा॑ म॒रुतः॑ स्व॒र्काः । \newline
12. सं॒ॅव॒थ्स॒रीणा॒ इति॑ सं - व॒थ्स॒रीणाः᳚ । \newline
13. म॒रुतः॑ स्व॒र्काः स्व॒र्का म॒रुतो॑ म॒रुतः॑ स्व॒र्का उ॑रु॒क्षया॑ उरु॒क्षयाः᳚ स्व॒र्का म॒रुतो॑ म॒रुतः॑ स्व॒र्का उ॑रु॒क्षयाः᳚ । \newline
14. स्व॒र्का उ॑रु॒क्षया॑ उरु॒क्षयाः᳚ स्व॒र्काः स्व॒र्का उ॑रु॒क्षयाः॒ सग॑णाः॒ सग॑णा उरु॒क्षयाः᳚ स्व॒र्काः स्व॒र्का उ॑रु॒क्षयाः॒ सग॑णाः । \newline
15. स्व॒र्का इति॑ सु - अ॒र्काः । \newline
16. उ॒रु॒क्षयाः॒ सग॑णाः॒ सग॑णा उरु॒क्षया॑ उरु॒क्षयाः॒ सग॑णा॒ मानु॑षेषु॒ मानु॑षेषु॒ सग॑णा उरु॒क्षया॑ उरु॒क्षयाः॒ सग॑णा॒ मानु॑षेषु । \newline
17. उ॒रु॒क्षया॒ इत्यु॑रु - क्षयाः᳚ । \newline
18. सग॑णा॒ मानु॑षेषु॒ मानु॑षेषु॒ सग॑णाः॒ सग॑णा॒ मानु॑षेषु । \newline
19. सग॑णा॒ इति॒ स - ग॒णाः॒ । \newline
20. मानु॑षे॒ष्विति॒ मानु॑षेषु । \newline
21. ते᳚ ऽस्म द॒स्मत् ते ते᳚ ऽस्मत् पाशा॒न् पाशा॑ न॒स्मत् ते ते᳚ ऽस्मत् पाशान्॑ । \newline
22. अ॒स्मत् पाशा॒न् पाशा॑ न॒स्म द॒स्मत् पाशा॒न् प्र प्र पाशा॑ न॒स्म द॒स्मत् पाशा॒न् प्र । \newline
23. पाशा॒न् प्र प्र पाशा॒न् पाशा॒न् प्र मु॑ञ्चन्तु मुञ्चन्तु॒ प्र पाशा॒न् पाशा॒न् प्र मु॑ञ्चन्तु । \newline
24. प्र मु॑ञ्चन्तु मुञ्चन्तु॒ प्र प्र मु॑ञ्च॒न् त्वꣳह॑सो॒ अꣳह॑सो मुञ्चन्तु॒ प्र प्र मु॑ञ्च॒न् त्वꣳह॑सः । \newline
25. मु॒ञ्च॒न् त्वꣳह॑सो॒ अꣳह॑सो मुञ्चन्तु मुञ्च॒न् त्वꣳह॑सः सान्तप॒नाः सा᳚न्तप॒ना अꣳह॑सो मुञ्चन्तु मुञ्च॒न् त्वꣳह॑सः सान्तप॒नाः । \newline
26. अꣳह॑सः सान्तप॒नाः सा᳚न्तप॒ना अꣳह॑सो॒ अꣳह॑सः सान्तप॒ना म॑दि॒रा म॑दि॒राः सा᳚न्तप॒ना अꣳह॑सो॒ अꣳह॑सः सान्तप॒ना म॑दि॒राः । \newline
27. सा॒न्त॒प॒ना म॑दि॒रा म॑दि॒राः सा᳚न्तप॒नाः सा᳚न्तप॒ना म॑दि॒रा मा॑दयि॒ष्णवो॑ मादयि॒ष्णवो॑ मदि॒राः सा᳚न्तप॒नाः सा᳚न्तप॒ना म॑दि॒रा मा॑दयि॒ष्णवः॑ । \newline
28. सा॒न्त॒प॒ना इति॑ सां - त॒प॒नाः । \newline
29. म॒दि॒रा मा॑दयि॒ष्णवो॑ मादयि॒ष्णवो॑ मदि॒रा म॑दि॒रा मा॑दयि॒ष्णवः॑ । \newline
30. मा॒द॒यि॒ष्णव॒ इति॑ मादयि॒ष्णवः॑ । \newline
31. पि॒प्री॒हि दे॒वान् दे॒वान् पि॑प्री॒हि पि॑प्री॒हि दे॒वाꣳ उ॑श॒त उ॑श॒तो दे॒वान् पि॑प्री॒हि पि॑प्री॒हि दे॒वाꣳ उ॑श॒तः । \newline
32. दे॒वाꣳ उ॑श॒त उ॑श॒तो दे॒वान् दे॒वाꣳ उ॑श॒तो य॑विष्ठ यविष्ठोश॒तो दे॒वान् दे॒वाꣳ उ॑श॒तो य॑विष्ठ । \newline
33. उ॒श॒तो य॑विष्ठ यविष्ठोश॒त उ॑श॒तो य॑विष्ठ वि॒द्वान्. वि॒द्वान्. य॑विष्ठोश॒त उ॑श॒तो य॑विष्ठ वि॒द्वान् । \newline
34. य॒वि॒ष्ठ॒ वि॒द्वान्. वि॒द्वान्. य॑विष्ठ यविष्ठ वि॒द्वाꣳ ऋ॒तूꣳर्. ऋ॒तून्. वि॒द्वान्. य॑विष्ठ यविष्ठ वि॒द्वाꣳ ऋ॒तून् । \newline
35. वि॒द्वाꣳ ऋ॒तूꣳर्. ऋ॒तून्. वि॒द्वान्. वि॒द्वाꣳ ऋ॒तूꣳर्. ऋ॑तुपत ऋतुपत ऋ॒तून्. वि॒द्वान्. वि॒द्वाꣳ ऋ॒तूꣳर्. ऋ॑तुपते । \newline
36. ऋ॒तूꣳर्. ऋ॑तुपत ऋतुपत ऋ॒तूꣳर्. ऋ॒तूꣳर्. ऋ॑तुपते यज यज र्तुपत ऋ॒तूꣳर्. ऋ॒तूꣳर्. ऋ॑तुपते यज । \newline
37. ऋ॒तु॒प॒ते॒ य॒ज॒ य॒ज॒ र्तु॒प॒त॒ ऋ॒तु॒प॒ते॒ य॒जे॒ हेह य॑ज र्तुपत ऋतुपते यजे॒ह । \newline
38. ऋ॒तु॒प॒त॒ इत्यृ॑तु - प॒ते॒ । \newline
39. य॒जे॒ हेह य॑ज यजे॒ह । \newline
40. इ॒हेती॒ह । \newline
41. ये दैव्या॒ दैव्या॒ ये ये दैव्या॑ ऋ॒त्विज॑ ऋ॒त्विजो॒ दैव्या॒ ये ये दैव्या॑ ऋ॒त्विजः॑ । \newline
42. दैव्या॑ ऋ॒त्विज॑ ऋ॒त्विजो॒ दैव्या॒ दैव्या॑ ऋ॒त्विज॒ स्तेभि॒ स्तेभि॑र्. ऋ॒त्विजो॒ दैव्या॒ दैव्या॑ ऋ॒त्विज॒ स्तेभिः॑ । \newline
43. ऋ॒त्विज॒ स्तेभि॒ स्तेभि॑र्. ऋ॒त्विज॑ ऋ॒त्विज॒ स्तेभि॑ रग्ने अग्ने॒ तेभि॑र्. ऋ॒त्विज॑ ऋ॒त्विज॒ स्तेभि॑ रग्ने । \newline
44. तेभि॑ रग्ने अग्ने॒ तेभि॒ स्तेभि॑ रग्ने॒ त्वम् त्व म॑ग्ने॒ तेभि॒ स्तेभि॑ रग्ने॒ त्वम् । \newline
45. अ॒ग्ने॒ त्वम् त्व म॑ग्ने अग्ने॒ त्वꣳ होतॄ॑णाꣳ॒॒ होतॄ॑णा॒म् त्व म॑ग्ने अग्ने॒ त्वꣳ होतॄ॑णाम् । \newline
46. त्वꣳ होतॄ॑णाꣳ॒॒ होतॄ॑णा॒म् त्वम् त्वꣳ होतॄ॑णा मस्यसि॒ होतॄ॑णा॒म् त्वम् त्वꣳ होतॄ॑णा मसि । \newline
47. होतॄ॑णा मस्यसि॒ होतॄ॑णाꣳ॒॒ होतॄ॑णा म॒स्या य॑जिष्ठ॒ आय॑जिष्ठो ऽसि॒ होतॄ॑णाꣳ॒॒ होतॄ॑णा म॒स्याय॑जिष्ठः । \newline
48. अ॒स्या य॑जिष्ठ॒ आय॑जिष्ठो ऽस्य॒स्या य॑जिष्ठः । \newline
49. आय॑जिष्ठ॒ इत्या - य॒जि॒ष्ठः॒ । \newline
50. अग्ने॒ यद् यदग्ने ऽग्ने॒ यद॒ द्याद्य यदग्ने ऽग्ने॒ यद॒द्य । \newline
51. यद॒ द्याद्य यद् यद॒द्य वि॒शो वि॒शो अ॒द्य यद् यद॒द्य वि॒शः । \newline
52. अ॒द्य वि॒शो वि॒शो अ॒द्याद्य वि॒शो अ॑द्ध्वरस्या द्ध्वरस्य वि॒शो अ॒द्याद्य वि॒शो अ॑द्ध्वरस्य । \newline
53. वि॒शो अ॑द्ध्वरस्या द्ध्वरस्य वि॒शो वि॒शो अ॑द्ध्वरस्य होतर्. होत-रद्ध्वरस्य वि॒शो वि॒शो अ॑द्ध्वरस्य होतः । \newline
54. अ॒द्ध्व॒र॒स्य॒ हो॒त॒र्॒. हो॒त॒ र॒द्ध्व॒र॒स्या॒ द्ध्व॒र॒स्य॒ हो॒तः॒ पाव॑क॒ पाव॑क होत रद्ध्वरस्या द्ध्वरस्य होतः॒ पाव॑क । \newline
55. हो॒तः॒ पाव॑क॒ पाव॑क होतर्. होतः॒ पाव॑क शोचे शोचे॒ पाव॑क होतर्. होतः॒ पाव॑क शोचे । \newline
56. पाव॑क शोचे शोचे॒ पाव॑क॒ पाव॑क शोचे॒ वेर् वेः शो॑चे॒ पाव॑क॒ पाव॑क शोचे॒ वेः । \newline
\pagebreak
\markright{ TS 4.3.13.5  \hfill https://www.vedavms.in \hfill}

\section{ TS 4.3.13.5 }

\textbf{TS 4.3.13.5 } \newline
\textbf{Samhita Paata} \newline

शोचे॒ वेष्ट्वꣳ हि यज्वा᳚ । ऋ॒ता य॑जासि महि॒ना वि यद्भूर्.ह॒व्या व॑ह यविष्ठ॒ या ते॑ अ॒द्य ॥ अ॒ग्निना॑ र॒यिम॑श्नव॒त् पोष॑मे॒व दि॒वेदि॑वे । य॒शसं॑ ॅवी॒रव॑त्तमं ॥ ग॒य॒स्फानो॑ अमीव॒हा व॑सु॒वित् पु॑ष्टि॒वर्द्ध॑नः । सु॒मि॒त्रः सो॑म नो भव ॥ गृह॑मेधास॒ आ ग॑त॒ मरु॑तो॒ माऽप॑ भूतन । प्र॒मु॒ञ्चन्तो॑ नो॒ अꣳह॑सः ॥ पू॒र्वीभि॒र्॒.हि द॑दाशि॒म श॒रद्भि॑र्मरुतो व॒यं । महो॑भि- [  ] \newline

\textbf{Pada Paata} \newline

शो॒चे॒ । वेः । त्वम् । हि । यज्वा᳚ ॥ ऋ॒ता । य॒जा॒सि॒ । म॒हि॒ना । वीति॑ । यत् । भूः । ह॒व्या । व॒ह॒ । य॒वि॒ष्ठ॒ । या । ते॒ । अ॒द्य ॥ अ॒ग्निना᳚ । र॒यिम् । अ॒श्न॒व॒त् । पोष᳚म् । ए॒व । दि॒वेदि॑व॒ इति॑ दि॒वे - दि॒वे॒ ॥ य॒शस᳚म् । वी॒रव॑त्तम॒मिति॑ वी॒रव॑त् - त॒म॒म् ॥ ग॒य॒स्फान॒ इति॑ गय - स्फानः॑ । अ॒मी॒व॒हेत्य॑मीव - हा । व॒सु॒विदिति॑ वसु - वित् । पु॒ष्टि॒वद्‌र्ध॑न॒ इति॑ पुष्टि - वर्ध॑नः ॥ सु॒मि॒त्र इति॑ सु - मि॒त्रः । सो॒म॒ । नः॒ । भ॒व॒ ॥ गृह॑मेधास॒ इति॒ गृह॑ - मे॒धा॒सः॒ । एति॑ । ग॒त॒ । मरु॑तः । मा । अपेति॑ । भू॒त॒न॒ ॥ प्र॒मु॒ञ्चन्त॒ इति॑ प्र - मु॒ञ्चन्तः॑ । नः॒ । अꣳह॑सः ॥ पू॒र्वीभिः॑ । हि । द॒दा॒शि॒म । श॒रद्भि॒रिति॑ श॒रत् - भिः॒ । म॒रु॒तः॒ । व॒यम् ॥ महो॑भि॒रिति॒ महः॑ - भिः॒ ।  \newline


\textbf{Krama Paata} \newline

शो॒चे॒ वेः । वेष्ट्वम् । त्वꣳ हि । हि यज्वा᳚ । यज्वेति॒ यज्वा᳚ ॥ ऋ॒ता य॑जासि । य॒जा॒सि॒ म॒हि॒ना । म॒हि॒ना वि । वि यत् । यद् भूः । भूर्. ह॒व्या । ह॒व्या व॑ह । व॒ह॒ य॒वि॒ष्ठ॒ । य॒वि॒ष्ठ॒ या । या ते᳚ । ते॒ अ॒द्य । अ॒द्येत्य॒द्य ॥ अ॒ग्निना॑ र॒यिम् । र॒यिम॑श्ञवत् । अ॒श्ञ॒व॒त् पोष᳚म् । पोष॑मे॒व । ए॒व दि॒वेदि॑वे । दि॒वेदि॑व॒ इति॑ दि॒वे - दि॒वे॒ ॥ य॒शस॑म् ॅवी॒रव॑त्तमम् । वी॒रव॑त्तम॒मिति॑ वी॒रव॑त् - त॒म॒म् ॥ ग॒य॒स्फानो॑ अमीव॒हा । ग॒य॒स्फान॒ इति॑ गय - स्फानः॑ । अ॒मी॒व॒हा व॑सु॒वित् । अ॒मी॒व॒हेत्य॑मीव - हा । व॒सु॒वित् पु॑ष्टि॒वर्द्ध॑नः । व॒सु॒विदिति॑ वसु - वित् । पु॒ष्टि॒वर्द्ध॑न॒ इति॑ पुष्टि - वर्द्ध॑नः ॥ सु॒मि॒त्रः सो॑म । सु॒मि॒त्र इति॑ सु - मि॒त्रः । सो॒म॒ नः॒ । नो॒ भ॒व॒ । भ॒वेति॑ भव ॥ गृह॑मेधास॒ आ । गृह॑मेधास॒ इति॒ गृह॑ - मे॒धा॒सः॒ । आ ग॑त । ग॒त॒ मरु॑तः । मरु॑तो॒ मा । माऽप॑ । अप॑ भूतन । भू॒त॒नेति॑ भूतन ॥ प्र॒मु॒ञ्चन्तो॑ नः । प्र॒मु॒ञ्चन्त॒ इति॑ प्र - मु॒ञ्चन्तः॑ । नो॒ अꣳह॑सः । अꣳह॑स॒ इत्यꣳह॑सः ॥ पू॒र्वीभि॒र्॒. हि । हि द॑दाशि॒म । द॒दा॒शि॒म श॒रद्भिः॑ । श॒रद्भि॑र् मरुतः । श॒रद्भि॒रिति॑ श॒रत् - भिः॒ । म॒रु॒तो॒ व॒यम् । व॒यमिति॑ व॒यम् ॥ महो॑भिश्चर्.षणी॒नाम् । महो॑भि॒रिति॒ महः॑ - भिः॒ । \newline

\textbf{Jatai Paata} \newline

1. शो॒चे॒ वेर् वेः शो॑चे शोचे॒ वेः । \newline
2. वेष्ट्वम् त्वं ॅवेर् वेष्ट्वम् । \newline
3. त्वꣳ हि हि त्वम् त्वꣳ हि । \newline
4. हि यज्वा॒ यज्वा॒ हि हि यज्वा᳚ । \newline
5. यज्वेति॒ यज्वा᳚ । \newline
6. ऋ॒ता य॑जासि यजा स्यृ॒त र्‌ता य॑जासि । \newline
7. य॒जा॒सि॒ म॒हि॒ना म॑हि॒ना य॑जासि यजासि महि॒ना । \newline
8. म॒हि॒ना वि वि म॑हि॒ना म॑हि॒ना वि । \newline
9. वि यद् यद् वि वि यत् । \newline
10. यद् भूर् भूर् यद् यद् भूः । \newline
11. भूर्. ह॒व्या ह॒व्या भूर् भूर्. ह॒व्या । \newline
12. ह॒व्या व॑ह वह ह॒व्या ह॒व्या व॑ह । \newline
13. व॒ह॒ य॒वि॒ष्ठ॒ य॒वि॒ष्ठ॒ व॒ह॒ व॒ह॒ य॒वि॒ष्ठ॒ । \newline
14. य॒वि॒ष्ठ॒ या या य॑विष्ठ यविष्ठ॒ या । \newline
15. या ते॑ ते॒ या या ते᳚ । \newline
16. ते॒ अ॒द्याद्य ते॑ ते अ॒द्य । \newline
17. अ॒द्येत्य॒द्य । \newline
18. अ॒ग्निना॑ र॒यिꣳ र॒यि म॒ग्निना॒ ऽग्निना॑ र॒यिम् । \newline
19. र॒यि म॑श्ञव दश्ञवद् र॒यिꣳ र॒यि म॑श्ञवत् । \newline
20. अ॒श्ञ॒व॒त् पोष॒म् पोष॑ मश्ञव दश्ञव॒त् पोष᳚म् । \newline
21. पोष॑ मे॒वैव पोष॒म् पोष॑ मे॒व । \newline
22. ए॒व दि॒वेदि॑वे दि॒वेदि॑व ए॒वैव दि॒वेदि॑वे । \newline
23. दि॒वेदि॑व॒ इति॑ दि॒वे - दि॒वे॒ । \newline
24. य॒शसं॑ ॅवी॒रव॑त्तमं ॅवी॒रव॑त्तमं ॅय॒शसं॑ ॅय॒शसं॑ ॅवी॒रव॑त्तमम् । \newline
25. वी॒रव॑त्तम॒मिति॑ वी॒रव॑त् - त॒म॒म् । \newline
26. ग॒य॒स्फानो॑ अमीव॒हा ऽमी॑व॒हा ग॑य॒स्फानो॑ गय॒स्फानो॑ अमीव॒हा । \newline
27. ग॒य॒स्फान॒ इति॑ गय - स्फानः॑ । \newline
28. अ॒मी॒व॒हा व॑सु॒विद् व॑सु॒वि द॑मीव॒हा ऽमी॑व॒हा व॑सु॒वित् । \newline
29. अ॒मी॒व॒हेत्य॑मीव - हा । \newline
30. व॒सु॒वित् पु॑ष्टि॒वर्द्ध॑नः पुष्टि॒वर्द्ध॑नो वसु॒विद् व॑सु॒वित् पु॑ष्टि॒वर्द्ध॑नः । \newline
31. व॒सु॒विदिति॑ वसु - वित् । \newline
32. पु॒ष्टि॒वर्द्ध॑न॒ इति॑ पुष्टि - वर्ध॑नः । \newline
33. सु॒मि॒त्रः सो॑म सोम सुमि॒त्रः सु॑मि॒त्रः सो॑म । \newline
34. सु॒मि॒त्र इति॑ सु - मि॒त्रः । \newline
35. सो॒म॒ नो॒ नः॒ सो॒म॒ सो॒म॒ नः॒ । \newline
36. नो॒ भ॒व॒ भ॒व॒ नो॒ नो॒ भ॒व॒ । \newline
37. भ॒वेति॑ भव । \newline
38. गृह॑मेधास॒ आ गृह॑मेधासो॒ गृह॑मेधास॒ आ । \newline
39. गृह॑मेधास॒ इति॒ गृह॑ - मे॒धा॒सः॒ । \newline
40. आ ग॑त ग॒ता ग॑त । \newline
41. ग॒त॒ मरु॑तो॒ मरु॑तो गत गत॒ मरु॑तः । \newline
42. मरु॑तो॒ मा मा मरु॑तो॒ मरु॑तो॒ मा । \newline
43. मा ऽपाप॒ मा मा ऽप॑ । \newline
44. अप॑ भूतन भूत॒ना पाप॑ भूतन । \newline
45. भू॒त॒नेति॑ भूतन । \newline
46. प्र॒मु॒ञ्चन्तो॑ नो नः प्रमु॒ञ्चन्तः॑ प्रमु॒ञ्चन्तो॑ नः । \newline
47. प्र॒मु॒ञ्चन्त॒ इति॑ प्र - मु॒ञ्चन्तः॑ । \newline
48. नो॒ अꣳह॑सो॒ अꣳह॑सो नो नो॒ अꣳह॑सः । \newline
49. अꣳह॑स॒ इत्यꣳह॑सः । \newline
50. पू॒र्वीभि॒र्॒. हि हि पू॒र्वीभिः॑ पू॒र्वीभि॒र्॒. हि । \newline
51. हि द॑दाशि॒म द॑दाशि॒म हि हि द॑दाशि॒म । \newline
52. द॒दा॒शि॒म श॒रद्भिः॑ श॒रद्भि॑र् ददाशि॒म द॑दाशि॒म श॒रद्भिः॑ । \newline
53. श॒रद्भि॑र् मरुतो मरुतः श॒रद्भिः॑ श॒रद्भि॑र् मरुतः । \newline
54. श॒रद्भि॒रिति॑ श॒रत् - भिः॒ । \newline
55. म॒रु॒तो॒ व॒यं ॅव॒यम् म॑रुतो मरुतो व॒यम् । \newline
56. व॒यमिति॑ व॒यम् । \newline
57. महो॑भि श्चर्.षणी॒नाम् च॑र्.षणी॒नाम् महो॑भि॒र् महो॑भि श्चर्.षणी॒नाम् । \newline
58. महो॑भि॒रिति॒ महः॑ - भिः॒ । \newline

\textbf{Ghana Paata } \newline

1. शो॒चे॒ वेर् वेः शो॑चे शोचे॒ वेष्ट्वम् त्वं ॅवेः शो॑चे शोचे॒ वेष्ट्वम् । \newline
2. वेष्ट्वम् त्वं ॅवेर् वेष्ट्वꣳ हि हि त्वं ॅवेर् वेष्ट्वꣳ हि । \newline
3. त्वꣳ हि हि त्वम् त्वꣳ हि यज्वा॒ यज्वा॒ हि त्वम् त्वꣳ हि यज्वा᳚ । \newline
4. हि यज्वा॒ यज्वा॒ हि हि यज्वा᳚ । \newline
5. यज्वेति॒ यज्वा᳚ । \newline
6. ऋ॒ता य॑जासि यजा स्यृ॒त र्‌ता य॑जासि महि॒ना म॑हि॒ना य॑जा स्यृ॒त र्‌ता य॑जासि महि॒ना । \newline
7. य॒जा॒सि॒ म॒हि॒ना म॑हि॒ना य॑जासि यजासि महि॒ना वि वि म॑हि॒ना य॑जासि यजासि महि॒ना वि । \newline
8. म॒हि॒ना वि वि म॑हि॒ना म॑हि॒ना वि यद् यद् वि म॑हि॒ना म॑हि॒ना वि यत् । \newline
9. वि यद् यद् वि वि यद् भूर् भूर् यद् वि वि यद् भूः । \newline
10. यद् भूर् भूर् यद् यद् भूर्. ह॒व्या ह॒व्या भूर् यद् यद् भूर्. ह॒व्या । \newline
11. भूर्. ह॒व्या ह॒व्या भूर् भूर्. ह॒व्या व॑ह वह ह॒व्या भूर् भूर्. ह॒व्या व॑ह । \newline
12. ह॒व्या व॑ह वह ह॒व्या ह॒व्या व॑ह यविष्ठ यविष्ठ वह ह॒व्या ह॒व्या व॑ह यविष्ठ । \newline
13. व॒ह॒ य॒वि॒ष्ठ॒ य॒वि॒ष्ठ॒ व॒ह॒ व॒ह॒ य॒वि॒ष्ठ॒ या या य॑विष्ठ वह वह यविष्ठ॒ या । \newline
14. य॒वि॒ष्ठ॒ या या य॑विष्ठ यविष्ठ॒ या ते॑ ते॒ या य॑विष्ठ यविष्ठ॒ या ते᳚ । \newline
15. या ते॑ ते॒ या या ते॑ अ॒द्याद्य ते॒ या या ते॑ अ॒द्य । \newline
16. ते॒ अ॒द्याद्य ते॑ ते अ॒द्य । \newline
17. अ॒द्येत्य॒द्य । \newline
18. अ॒ग्निना॑ र॒यिꣳ र॒यि म॒ग्निना॒ ऽग्निना॑ र॒यि म॑श्ञव दश्ञवद् र॒यि म॒ग्निना॒ ऽग्निना॑ र॒यि म॑श्ञवत् । \newline
19. र॒यि म॑श्ञव दश्ञवद् र॒यिꣳ र॒यि म॑श्ञव॒त् पोष॒म् पोष॑ मश्ञवद् र॒यिꣳ र॒यि म॑श्ञव॒त् पोष᳚म् । \newline
20. अ॒श्ञ॒व॒त् पोष॒म् पोष॑ मश्ञव दश्ञव॒त् पोष॑ मे॒वैव पोष॑ मञ्नव दश्ञव॒त् पोष॑ मे॒व । \newline
21. पोष॑ मे॒वैव पोष॒म् पोष॑ मे॒व दि॒वेदि॑वे दि॒वेदि॑व ए॒व पोष॒म् पोष॑ मे॒व दि॒वेदि॑वे । \newline
22. ए॒व दि॒वेदि॑वे दि॒वेदि॑व ए॒वैव दि॒वेदि॑वे । \newline
23. दि॒वेदि॑व॒ इति॑ दि॒वे - दि॒वे॒ । \newline
24. य॒शसं॑ ॅवी॒रव॑त्तमं ॅवी॒रव॑त्तमं ॅय॒शसं॑ ॅय॒शसं॑ ॅवी॒रव॑त्तमम् । \newline
25. वी॒रव॑त्तम॒मिति॑ वी॒रव॑त् - त॒म॒म् । \newline
26. ग॒य॒स्फानो॑ अमीव॒हा ऽमी॑व॒हा ग॑य॒स्फानो॑ गय॒स्फानो॑ अमीव॒हा व॑सु॒विद् व॑सु॒वि द॑मीव॒हा ग॑य॒स्फानो॑ गय॒स्फानो॑ अमीव॒हा व॑सु॒वित् । \newline
27. ग॒य॒स्फान॒ इति॑ गय - स्फानः॑ । \newline
28. अ॒मी॒व॒हा व॑सु॒विद् व॑सु॒वि द॑मीव॒हा ऽमी॑व॒हा व॑सु॒वित् पु॑ष्टि॒वर्द्ध॑नः पुष्टि॒वर्द्ध॑नो वसु॒वि द॑मीव॒हा ऽमी॑व॒हा व॑सु॒वित् पु॑ष्टि॒वर्द्ध॑नः । \newline
29. अ॒मी॒व॒हेत्य॑मीव - हा । \newline
30. व॒सु॒वित् पु॑ष्टि॒वर्द्ध॑नः पुष्टि॒वर्द्ध॑नो वसु॒विद् व॑सु॒वित् पु॑ष्टि॒वर्द्ध॑नः । \newline
31. व॒सु॒विदिति॑ वसु - वित् । \newline
32. पु॒ष्टि॒वर्द्ध॑न॒ इति॑ पुष्टि - वर्द्ध॑नः । \newline
33. सु॒मि॒त्रः सो॑म सोम सुमि॒त्रः सु॑मि॒त्रः सो॑म नो नः सोम सुमि॒त्रः सु॑मि॒त्रः सो॑म नः । \newline
34. सु॒मि॒त्र इति॑ सु - मि॒त्रः । \newline
35. सो॒म॒ नो॒ नः॒ सो॒म॒ सो॒म॒ नो॒ भ॒व॒ भ॒व॒ नः॒ सो॒म॒ सो॒म॒ नो॒ भ॒व॒ । \newline
36. नो॒ भ॒व॒ भ॒व॒ नो॒ नो॒ भ॒व॒ । \newline
37. भ॒वेति॑ भव । \newline
38. गृह॑मेधास॒ आ गृह॑मेधासो॒ गृह॑मेधास॒ आ ग॑त ग॒ता गृह॑मेधासो॒ गृह॑मेधास॒ आ ग॑त । \newline
39. गृह॑मेधास॒ इति॒ गृह॑ - मे॒धा॒सः॒ । \newline
40. आ ग॑त ग॒ता ग॑त॒ मरु॑तो॒ मरु॑तो ग॒ता ग॑त॒ मरु॑तः । \newline
41. ग॒त॒ मरु॑तो॒ मरु॑तो गत गत॒ मरु॑तो॒ मा मा मरु॑तो गत गत॒ मरु॑तो॒ मा । \newline
42. मरु॑तो॒ मा मा मरु॑तो॒ मरु॑तो॒ मा ऽपाप॒ मा मरु॑तो॒ मरु॑तो॒ मा ऽप॑ । \newline
43. मा ऽपाप॒ मा मा ऽप॑ भूतन भूत॒नाप॒ मा मा ऽप॑ भूतन । \newline
44. अप॑ भूतन भूत॒ नापाप॑ भूतन । \newline
45. भू॒त॒नेति॑ भूतन । \newline
46. प्र॒मु॒ञ्चन्तो॑ नो नः प्रमु॒ञ्चन्तः॑ प्रमु॒ञ्चन्तो॑ नो॒ अꣳह॑सो॒ अꣳह॑सो नः प्रमु॒ञ्चन्तः॑ प्रमु॒ञ्चन्तो॑ नो॒ अꣳह॑सः । \newline
47. प्र॒मु॒ञ्चन्त॒ इति॑ प्र - मु॒ञ्चन्तः॑ । \newline
48. नो॒ अꣳह॑सो॒ अꣳह॑सो नो नो॒ अꣳह॑सः । \newline
49. अꣳह॑स॒ इत्यꣳह॑सः । \newline
50. पू॒र्वीभि॒र्॒. हि हि पू॒र्वीभिः॑ पू॒र्वीभि॒र्॒. हि द॑दाशि॒म द॑दाशि॒म हि पू॒र्वीभिः॑ पू॒र्वीभि॒र्॒. हि द॑दाशि॒म । \newline
51. हि द॑दाशि॒म द॑दाशि॒म हि हि द॑दाशि॒म श॒रद्भिः॑ श॒रद्भि॑र् ददाशि॒म हि हि द॑दाशि॒म श॒रद्भिः॑ । \newline
52. द॒दा॒शि॒म श॒रद्भिः॑ श॒रद्भि॑र् ददाशि॒म द॑दाशि॒म श॒रद्भि॑र् मरुतो मरुतः श॒रद्भि॑र् ददाशि॒म द॑दाशि॒म श॒रद्भि॑र् मरुतः । \newline
53. श॒रद्भि॑र् मरुतो मरुतः श॒रद्भिः॑ श॒रद्भि॑र् मरुतो व॒यं ॅव॒यम् म॑रुतः श॒रद्भिः॑ श॒रद्भि॑र् मरुतो व॒यम् । \newline
54. श॒रद्भि॒रिति॑ श॒रत् - भिः॒ । \newline
55. म॒रु॒तो॒ व॒यं ॅव॒यम् म॑रुतो मरुतो व॒यम् । \newline
56. व॒यमिति॑ व॒यम् । \newline
57. महो॑भि श्चर्.षणी॒नाम् चर्॑.षणी॒नाम् महो॑भि॒र् महो॑भि श्चर्.षणी॒नाम् । \newline
58. महो॑भि॒रिति॒ महः॑ - भिः॒ । \newline
\pagebreak
\markright{ TS 4.3.13.6  \hfill https://www.vedavms.in \hfill}

\section{ TS 4.3.13.6 }

\textbf{TS 4.3.13.6 } \newline
\textbf{Samhita Paata} \newline

-श्चर्.षणी॒नां ॥ प्रबु॒द्ध्निया॑ ईरते वो॒ महाꣳ॑सि॒ प्रणामा॑नि प्रयज्यवस्तिरद्ध्वं । स॒ह॒स्रियं॒ दम्यं॑ भा॒गमे॒तं गृ॑हमे॒धीयं॑ मरुतो जुषद्ध्वं ॥ उप॒ यमेति॑ युव॒तिः सु॒दक्षं॑ दो॒षा वस्तोर्॑. ह॒विष्म॑ती घृ॒ताची᳚ । उप॒ स्वैन॑म॒रम॑तिर्व-सू॒युः ॥ इ॒मो अ॑ग्ने वी॒तत॑मानि ह॒व्या ऽज॑स्रो वक्षि दे॒वता॑ति॒मच्छ॑ । प्रति॑ न ईꣳ सुर॒भीणि॑ वियन्तु ॥क्री॒डं ॅवः॒ शर्द्धो॒ मारु॑तमन॒र्वाणꣳ॑ रथे॒शुभं᳚ । \newline

\textbf{Pada Paata} \newline

च॒र्॒.ष॒णी॒नाम् ॥ प्रेति॑ । बु॒द्ध्निया᳚ । ई॒र॒ते॒ । वः॒ । महाꣳ॑सि । प्रेति॑ । नामा॑नि । प्र॒य॒ज्य॒व॒ इति॑ प्र - य॒ज्य॒वः॒ । ति॒र॒द्ध्व॒म् ॥ स॒ह॒स्रिय᳚म् । दम्य᳚म् । भा॒गम् । ए॒तम् । गृ॒ह॒मे॒धीय॒मिति॑ गृह - मे॒धीय᳚म् । म॒रु॒तः॒ । जु॒ष॒द्ध्व॒म् ॥ उपेति॑ । यम् । एति॑ । यु॒व॒तिः । सु॒दक्ष॒मिति॑ सु-दक्ष᳚म् । दो॒षा । वस्तोः᳚ । ह॒विष्म॑ती । घृ॒ताची᳚ ॥ उपेति॑ । स्व । ए॒न॒म् । अ॒रम॑तिः । व॒सू॒युरिति॑ वसू - युः ॥ इ॒मो इति॑ । अ॒ग्ने॒ । वी॒तत॑मा॒नीति॑ वी॒त - त॒मा॒नि॒ । ह॒व्या । अज॑स्रः । व॒क्षि॒ । दे॒वता॑ति॒मिति॑ दे॒व - ता॒ति॒म् । अच्छ॑ ॥ प्रतीति॑ । नः॒ । ई॒म् । सु॒र॒भीणि॑ । वि॒य॒न्तु॒ ॥ क्री॒डम् । वः॒ । शद्‌र्धः॑ । मारु॑तम् । अ॒न॒र्वाण᳚म् । र॒थे॒शुभ॒मिति॑ रथे - शुभ᳚म् ॥  \newline


\textbf{Krama Paata} \newline

च॒र्॒.ष॒णी॒नामिति॑ चर्.षणी॒नाम् ॥ प्र बु॒ध्निया᳚ । बु॒ध्निया॑ ईरते । ई॒र॒ते॒ वः॒ । वो॒ महाꣳ॑सि । महाꣳ॑सि॒ प्र । 
प्र णामा॑नि । नामा॑नि प्रयज्यवः । प्र॒य॒ज्य॒व॒स्ति॒र॒द्ध्व॒म् । प्र॒य॒ज्य॒व॒ इति॑ प्र - य॒ज्य॒वः॒ । ति॒र॒द्ध्व॒मिति॑ तिरद्ध्वम् ॥ स॒ह॒स्रिय॒म् दम्य᳚म् । दम्य॑म् भा॒गम् । भा॒गमे॒तम् । ए॒तम् गृ॑हमे॒धीय᳚म् । गृ॒ह॒मे॒धीय॑म् मरुतः । गृ॒ह॒मे॒धीय॒मिति॑ गृह - मे॒धीय᳚म् । म॒रु॒तो॒ जु॒ष॒द्ध्व॒म् । जु॒ष॒द्ध्व॒मिति॑ जुषध्वम् ॥ उप॒ यम् । यमेति॑ । एति॑ युव॒तिः । यु॒व॒तिः सु॒दक्ष᳚म् । सु॒दक्ष॑म् दो॒षा । सु॒दक्ष॒मिति॑ सु - दक्ष᳚म् । दो॒षा वस्तोः᳚ । वस्तोर्॑. ह॒विष्म॑ती । ह॒विष्म॑ती घृ॒ताची᳚ । घृ॒ताचीति॑ घृ॒ताची᳚ ॥ उप॒ स्वा । स्वैन᳚म् । ए॒न॒म॒रम॑तिः । अ॒रम॑तिर् वसू॒युः । व॒सू॒युरिति॑ वसु - युः ॥ इ॒मो अ॑ग्ने । इ॒मो इती॒मो । अ॒ग्ने॒ वी॒तत॑मानि । वी॒तत॑मानि ह॒व्या । वी॒तत॑मा॒नीति॑ वी॒त - त॒मा॒नि॒ । ह॒व्याऽज॑स्रः । अज॑स्रो वक्षि । 
व॒क्षि॒ दे॒वता॑तिम् । दे॒वता॑ति॒मच्छ॑ । दे॒वता॑ति॒मिति॑ दे॒व - 
ता॒ति॒म् । अच्छेत्यच्छ॑ ॥ प्रति॑ नः । न॒ ई॒म् । ईꣳ॒॒ सु॒र॒भीणि॑ । सु॒र॒भीणि॑ वियन्तु । वि॒य॒न्त्विति॑ वियन्तु ॥ क्री॒डं ॅवः॑ । वः॒ शर्द्धः॑ । शर्द्धो॒ मारु॑तम् । मारु॑तमन॒र्वाण᳚म् । अ॒न॒र्वाणꣳ॑ रथे॒शुभ᳚म् । र॒थे॒शुभ॒मिति॑ रथे - शुभ᳚म् । \newline

\textbf{Jatai Paata} \newline

1. च॒र्॒.ष॒णी॒नामिति॑ चर्.षणी॒नाम् । \newline
2. प्र बु॒द्ध्निया॑ बु॒द्ध्निया॒ प्र प्र बु॒द्ध्निया᳚ । \newline
3. बु॒द्ध्निया॑ ईरत ईरते बु॒द्ध्निया॑ बु॒द्ध्निया॑ ईरते । \newline
4. ई॒र॒ते॒ वो॒ व॒ ई॒र॒त॒ ई॒र॒ते॒ वः॒ । \newline
5. वो॒ महाꣳ॑सि॒ महाꣳ॑सि वो वो॒ महाꣳ॑सि । \newline
6. महाꣳ॑सि॒ प्र प्र महाꣳ॑सि॒ महाꣳ॑सि॒ प्र । \newline
7. प्र णामा॑नि॒ नामा॑नि॒ प्र प्र णामा॑नि । \newline
8. नामा॑नि प्रयज्यवः प्रयज्यवो॒ नामा॑नि॒ नामा॑नि प्रयज्यवः । \newline
9. प्र॒य॒ज्य॒व॒ स्ति॒र॒द्ध्व॒म् ति॒र॒द्ध्व॒म् प्र॒य॒ज्य॒वः॒ प्र॒य॒ज्य॒व॒ स्ति॒र॒द्ध्व॒म् । \newline
10. प्र॒य॒ज्य॒व॒ इति॑ प्र - य॒ज्य॒वः॒ । \newline
11. ति॒र॒द्ध्व॒मिति॑ तिरद्ध्वम् । \newline
12. स॒ह॒स्रिय॒म् दम्य॒म् दम्यꣳ॑ सह॒स्रियꣳ॑ सह॒स्रिय॒म् दम्य᳚म् । \newline
13. दम्य॑म् भा॒गम् भा॒गम् दम्य॒म् दम्य॑म् भा॒गम् । \newline
14. भा॒ग मे॒त मे॒तम् भा॒गम् भा॒ग मे॒तम् । \newline
15. ए॒तम् गृ॑हमे॒धीय॑म् गृहमे॒धीय॑ मे॒त मे॒तम् गृ॑हमे॒धीय᳚म् । \newline
16. गृ॒ह॒मे॒धीय॑म् मरुतो मरुतो गृहमे॒धीय॑म् गृहमे॒धीय॑म् मरुतः । \newline
17. गृ॒ह॒मे॒धीय॒मिति॑ गृह - मे॒धीय᳚म् । \newline
18. म॒रु॒तो॒ जु॒ष॒द्ध्व॒म् जु॒ष॒द्ध्व॒म् म॒रु॒तो॒ म॒रु॒तो॒ जु॒ष॒द्ध्व॒म् । \newline
19. जु॒ष॒द्ध्व॒मिति॑ जुषध्वम् । \newline
20. उप॒ यं ॅय मुपोप॒ यम् । \newline
21. य मेत्येति॒ यम् ॅयमेति॑ । \newline
22. एति॑ युव॒तिर् यु॑व॒ति रेत्येति॑ युव॒तिः । \newline
23. यु॒व॒तिः सु॒दक्षꣳ॑ सु॒दक्षं॑ ॅयुव॒तिर् यु॑व॒तिः सु॒दक्ष᳚म् । \newline
24. सु॒दक्ष॑म् दो॒षा दो॒षा सु॒दक्षꣳ॑ सु॒दक्ष॑म् दो॒षा । \newline
25. सु॒दक्ष॒मिति॑ सु - दक्ष᳚म् । \newline
26. दो॒षा वस्तो॒र् वस्तो᳚र् दो॒षा दो॒षा वस्तोः᳚ । \newline
27. वस्तोर्॑. ह॒विष्म॑ती ह॒विष्म॑ती॒ वस्तो॒र् वस्तोर्॑. ह॒विष्म॑ती । \newline
28. ह॒विष्म॑ती घृ॒ताची॑ घृ॒ताची॑ ह॒विष्म॑ती ह॒विष्म॑ती घृ॒ताची᳚ । \newline
29. घृ॒ताचीति॑ घृ॒ताची᳚ । \newline
30. उप॒ स्वा स्वोपोप॒ स्वा । \newline
31. स्वैन॑ मेनꣳ॒॒ स्वा स्वैन᳚म् । \newline
32. ए॒न॒ म॒रम॑ति र॒रम॑ति रेन मेन म॒रम॑तिः । \newline
33. अ॒रम॑तिर् वसू॒युर् व॑सू॒यु र॒रम॑ति र॒रम॑तिर् वसू॒युः । \newline
34. व॒सू॒युरिति॑ वसु - युः । \newline
35. इ॒मो अ॑ग्ने अग्न इ॒मो इ॒मो अ॑ग्ने । \newline
36. इ॒मो इती॒मो । \newline
37. अ॒ग्ने॒ वी॒तत॑मानि वी॒तत॑मा न्यग्ने अग्ने वी॒तत॑मानि । \newline
38. वी॒तत॑मानि ह॒व्या ह॒व्या वी॒तत॑मानि वी॒तत॑मानि ह॒व्या । \newline
39. वी॒तत॑मा॒नीति॑ वी॒त - त॒मा॒नि॒ । \newline
40. ह॒व्या ऽज॒स्रो ऽज॑स्रो ह॒व्या ह॒व्या ऽज॑स्रः । \newline
41. अज॑स्रो वक्षि व॒क्ष्य ज॒स्रो ऽज॑स्रो वक्षि । \newline
42. व॒क्षि॒ दे॒वता॑तिम् दे॒वता॑तिं ॅवक्षि वक्षि दे॒वता॑तिम् । \newline
43. दे॒वता॑ति॒ मच्छाच्छ॑ दे॒वता॑तिम् दे॒वता॑ति॒ मच्छ॑ । \newline
44. दे॒वता॑ति॒मिति॑ दे॒व - ता॒ति॒म् । \newline
45. अच्छेत्यच्छ॑ । \newline
46. प्रति॑ नो नः॒ प्रति॒ प्रति॑ नः । \newline
47. न॒ ई॒ मी॒म् नो॒ न॒ ई॒म् । \newline
48. ईꣳ॒॒ सु॒र॒भीणि॑ सुर॒भीणी॑ मीꣳ सुर॒भीणि॑ । \newline
49. सु॒र॒भीणि॑ वियन्तु वियन्तु सुर॒भीणि॑ सुर॒भीणि॑ वियन्तु । \newline
50. वि॒य॒न्त्विति॑ वियन्तु । \newline
51. क्री॒डं ॅवो॑ वः क्री॒डम् क्री॒डं ॅवः॑ । \newline
52. वः॒ शर्द्धः॒ शर्द्धो॑ वो वः॒ शर्द्धः॑ । \newline
53. शर्द्धो॒ मारु॑त॒म् मारु॑तꣳ॒॒ शर्द्धः॒ शर्द्धो॒ मारु॑तम् । \newline
54. मारु॑त मन॒र्वाण॑ मन॒र्वाण॒म् मारु॑त॒म् मारु॑त मन॒र्वाण᳚म् । \newline
55. अ॒न॒र्वाणꣳ॑ रथे॒शुभꣳ॑ रथे॒शुभ॑ मन॒र्वाण॑ मन॒र्वाणꣳ॑ रथे॒शुभ᳚म् । \newline
56. र॒थे॒शुभ॒मिति॑ रथे - शुभ᳚म् । \newline

\textbf{Ghana Paata } \newline

1. च॒र्॒.ष॒णी॒नामिति॑ चर्.षणी॒नाम् । \newline
2. प्र बु॒द्ध्निया॑ बु॒द्ध्निया॒ प्र प्र बु॒द्ध्निया॑ ईरत ईरते बु॒द्ध्निया॒ प्र प्र बु॒द्ध्निया॑ ईरते । \newline
3. बु॒द्ध्निया॑ ईरत ईरते बु॒द्ध्निया॑ बु॒द्ध्निया॑ ईरते वो व ईरते बु॒द्ध्निया॑ बु॒द्ध्निया॑ ईरते वः । \newline
4. ई॒र॒ते॒ वो॒ व॒ ई॒र॒त॒ ई॒र॒ते॒ वो॒ महाꣳ॑सि॒ महाꣳ॑सि व ईरत ईरते वो॒ महाꣳ॑सि । \newline
5. वो॒ महाꣳ॑सि॒ महाꣳ॑सि वो वो॒ महाꣳ॑सि॒ प्र प्र महाꣳ॑सि वो वो॒ महाꣳ॑सि॒ प्र । \newline
6. महाꣳ॑सि॒ प्र प्र महाꣳ॑सि॒ महाꣳ॑सि॒ प्र णामा॑नि॒ नामा॑नि॒ प्र महाꣳ॑सि॒ महाꣳ॑सि॒ प्र णामा॑नि । \newline
7. प्र णामा॑नि॒ नामा॑नि॒ प्र प्र णामा॑नि प्रयज्यवः प्रयज्यवो॒ नामा॑नि॒ प्र प्र णामा॑नि प्रयज्यवः । \newline
8. नामा॑नि प्रयज्यवः प्रयज्यवो॒ नामा॑नि॒ नामा॑नि प्रयज्यव स्तिरद्ध्वम् तिरद्ध्वम् प्रयज्यवो॒ नामा॑नि॒ नामा॑नि प्रयज्यव स्तिरद्ध्वम् । \newline
9. प्र॒य॒ज्य॒व॒ स्ति॒र॒द्ध्व॒म् ति॒र॒द्ध्व॒म् प्र॒य॒ज्य॒वः॒ प्र॒य॒ज्य॒व॒ स्ति॒र॒द्ध्व॒म् । \newline
10. प्र॒य॒ज्य॒व॒ इति॑ प्र - य॒ज्य॒वः॒ । \newline
11. ति॒र॒द्ध्व॒मिति॑ तिरद्ध्वम् । \newline
12. स॒ह॒स्रिय॒म् दम्य॒म् दम्यꣳ॑ सह॒स्रियꣳ॑ सह॒स्रिय॒म् दम्य॑म् भा॒गम् भा॒गम् दम्यꣳ॑ सह॒स्रियꣳ॑ सह॒स्रिय॒म् दम्य॑म् भा॒गम् । \newline
13. दम्य॑म् भा॒गम् भा॒गम् दम्य॒म् दम्य॑म् भा॒ग मे॒त मे॒तम् भा॒गम् दम्य॒म् दम्य॑म् भा॒ग मे॒तम् । \newline
14. भा॒ग मे॒त मे॒तम् भा॒गम् भा॒ग मे॒तम् गृ॑हमे॒धीय॑म् गृहमे॒धीय॑ मे॒तम् भा॒गम् भा॒ग मे॒तम् गृ॑हमे॒धीय᳚म् । \newline
15. ए॒तम् गृ॑हमे॒धीय॑म् गृहमे॒धीय॑ मे॒त मे॒तम् गृ॑हमे॒धीय॑म् मरुतो मरुतो गृहमे॒धीय॑ मे॒त मे॒तम् गृ॑हमे॒धीय॑म् मरुतः । \newline
16. गृ॒ह॒मे॒धीय॑म् मरुतो मरुतो गृहमे॒धीय॑म् गृहमे॒धीय॑म् मरुतो जुषद्ध्वम् जुषद्ध्वम् मरुतो गृहमे॒धीय॑म् गृहमे॒धीय॑म् मरुतो जुषद्ध्वम् । \newline
17. गृ॒ह॒मे॒धीय॒मिति॑ गृह - मे॒धीय᳚म् । \newline
18. म॒रु॒तो॒ जु॒ष॒द्ध्व॒म् जु॒ष॒द्ध्व॒म् म॒रु॒तो॒ म॒रु॒तो॒ जु॒ष॒द्ध्व॒म् । \newline
19. जु॒ष॒द्ध्व॒मिति॑ जुषद्ध्वम् । \newline
20. उप॒ यम् ॅयम् उपोप॒ यमेत्येति॒ यमुपोप॒ यमेति॑ । \newline
21. यमेत्येति॒ ॅयम् ॅयमेति॑ युव॒तिर् यु॑व॒ति रेति॒ ॅयम् ॅयमेति॑ युव॒तिः । \newline
22. एति॑ युव॒तिर् यु॑व॒ति रेत्येति॑ युव॒तिः सु॒दक्षꣳ॑ सु॒दक्षं॑ ॅयुव॒ति रेत्येति॑ 
युव॒तिः सु॒दक्ष᳚म् । \newline
23. यु॒व॒तिः सु॒दक्षꣳ॑ सु॒दक्षं॑ ॅयुव॒तिर् यु॑व॒तिः सु॒दक्ष॑म् दो॒षा दो॒षा सु॒दक्षं॑ ॅयुव॒तिर् यु॑व॒तिः सु॒दक्ष॑म् दो॒षा । \newline
24. सु॒दक्ष॑म् दो॒षा दो॒षा सु॒दक्षꣳ॑ सु॒दक्ष॑म् दो॒षा वस्तो॒र् वस्तो᳚र् दो॒षा सु॒दक्षꣳ॑ सु॒दक्ष॑म् दो॒षा वस्तोः᳚ । \newline
25. सु॒दक्ष॒मिति॑ सु - दक्ष᳚म् । \newline
26. दो॒षा वस्तो॒र् वस्तो᳚र् दो॒षा दो॒षा वस्तोर्॑. ह॒विष्म॑ती ह॒विष्म॑ती॒ वस्तो᳚र् दो॒षा दो॒षा वस्तोर्॑. ह॒विष्म॑ती । \newline
27. वस्तोर्॑. ह॒विष्म॑ती ह॒विष्म॑ती॒ वस्तो॒र् वस्तोर्॑. ह॒विष्म॑ती घृ॒ताची॑ घृ॒ताची॑ ह॒विष्म॑ती॒ वस्तो॒र् वस्तोर्॑. ह॒विष्म॑ती घृ॒ताची᳚ । \newline
28. ह॒विष्म॑ती घृ॒ताची॑ घृ॒ताची॑ ह॒विष्म॑ती ह॒विष्म॑ती घृ॒ताची᳚ । \newline
29. घृ॒ताचीति॑ घृ॒ताची᳚ । \newline
30. उप॒ स्वा स्वो पोप॒ स्वैन॑ मेनꣳ॒॒ स्वो पोप॒ स्वैन᳚म् । \newline
31. स्वैन॑ मेनꣳ॒॒ स्वा स्वैन॑ म॒रम॑ति र॒रम॑ति रेनꣳ॒॒ स्वा स्वैन॑ म॒रम॑तिः । \newline
32. ए॒न॒ म॒रम॑ति र॒रम॑ति रेन मेन म॒रम॑तिर् वसू॒युर् व॑सू॒यु र॒रम॑ति रेन मेन म॒रम॑तिर् वसू॒युः । \newline
33. अ॒रम॑तिर् वसू॒युर् व॑सू॒यु र॒रम॑ति र॒रम॑तिर् वसू॒युः । \newline
34. व॒सू॒युरिति॑ वसु - युः । \newline
35. इ॒मो अ॑ग्ने अग्न इ॒मो इ॒मो अ॑ग्ने वी॒तत॑मानि वी॒तत॑मा न्यग्न इ॒मो इ॒मो अ॑ग्ने वी॒तत॑मानि । \newline
36. इ॒मो इती॒मो । \newline
37. अ॒ग्ने॒ वी॒तत॑मानि वी॒तत॑मा न्यग्ने अग्ने वी॒तत॑मानि ह॒व्या ह॒व्या वी॒तत॑मा न्यग्ने अग्ने वी॒तत॑मानि ह॒व्या । \newline
38. वी॒तत॑मानि ह॒व्या ह॒व्या वी॒तत॑मानि वी॒तत॑मानि ह॒व्या ऽज॒स्रो ऽज॑स्रो ह॒व्या वी॒तत॑मानि वी॒तत॑मानि ह॒व्या ऽज॑स्रः । \newline
39. वी॒तत॑मा॒नीति॑ वी॒त - त॒मा॒नि॒ । \newline
40. ह॒व्या ऽज॒स्रो ऽज॑स्रो ह॒व्या ह॒व्या ऽज॑स्रो वक्षि व॒क्ष्यज॑स्रो ह॒व्या ह॒व्या ऽज॑स्रो वक्षि । \newline
41. अज॑स्रो वक्षि व॒क्ष्यज॒स्रो ऽज॑स्रो वक्षि दे॒वता॑तिम् दे॒वता॑तिं ॅव॒क्ष्यज॒स्रो ऽज॑स्रो वक्षि दे॒वता॑तिम् । \newline
42. व॒क्षि॒ दे॒वता॑तिम् दे॒वता॑तिं ॅवक्षि वक्षि दे॒वता॑ति॒ मच्छाच्छ॑ दे॒वता॑तिं ॅवक्षि वक्षि दे॒वता॑ति॒ मच्छ॑ । \newline
43. दे॒वता॑ति॒ मच्छाच्छ॑ दे॒वता॑तिम् दे॒वता॑ति॒ मच्छ॑ । \newline
44. दे॒वता॑ति॒मिति॑ दे॒व - ता॒ति॒म् । \newline
45. अच्छेत्यच्छ॑ । \newline
46. प्रति॑ नो नः॒ प्रति॒ प्रति॑ न ई मीन्नः॒ प्रति॒ प्रति॑ न ईम् । \newline
47. न॒ ई॒ मी॒न् नो॒ न॒ ईꣳ॒॒ सु॒र॒भीणि॑ सुर॒भीणी᳚न् नो न ईꣳ सुर॒भीणि॑ । \newline
48. ईꣳ॒॒ सु॒र॒भीणि॑ सुर॒भीणी॑ मीꣳ सुर॒भीणि॑ वियन्तु वियन्तु सुर॒भीणी॑ मीꣳ सुर॒भीणि॑ वियन्तु । \newline
49. सु॒र॒भीणि॑ वियन्तु वियन्तु सुर॒भीणि॑ सुर॒भीणि॑ वियन्तु । \newline
50. वि॒य॒न्त्विति॑ वियन्तु । \newline
51. क्री॒डं ॅवो॑ वः क्री॒डम् क्री॒डं ॅवः॒ शर्द्धः॒ शर्द्धो॑ वः क्री॒डम् क्री॒डं ॅवः॒ शर्द्धः॑ । \newline
52. वः॒ शर्द्धः॒ शर्द्धो॑ वो वः॒ शर्द्धो॒ मारु॑त॒म् मारु॑तꣳ॒॒ शर्द्धो॑ वो वः॒ शर्द्धो॒ मारु॑तम् । \newline
53. शर्द्धो॒ मारु॑त॒म् मारु॑तꣳ॒॒ शर्द्धः॒ शर्द्धो॒ मारु॑त मन॒र्वाण॑ मन॒र्वाण॒म् मारु॑तꣳ॒॒ शर्द्धः॒ 
शर्द्धो॒ मारु॑त मन॒र्वाण᳚म् । \newline
54. मारु॑त मन॒र्वाण॑ मन॒र्वाण॒म् मारु॑त॒म् मारु॑त मन॒र्वाणꣳ॑ रथे॒शुभꣳ॑ रथे॒शुभ॑ मन॒र्वाण॒म् 
मारु॑त॒म् मारु॑त मन॒र्वाणꣳ॑ रथे॒शुभ᳚म् । \newline
55. अ॒न॒र्वाणꣳ॑ रथे॒शुभꣳ॑ रथे॒शुभ॑ मन॒र्वाण॑ मन॒र्वाणꣳ॑ रथे॒शुभ᳚म् । \newline
56. र॒थे॒शुभ॒मिति॑ रथे - शुभ᳚म् । \newline
\pagebreak
\markright{ TS 4.3.13.7  \hfill https://www.vedavms.in \hfill}

\section{ TS 4.3.13.7 }

\textbf{TS 4.3.13.7 } \newline
\textbf{Samhita Paata} \newline

कण्वा॑ अ॒भि प्र गा॑यत ॥ अत्या॑सो॒ न ये म॒रुतः॒ स्वञ्चो॑ यक्ष॒दृशो॒ न शु॒भय॑न्त॒ मर्याः᳚ । ते ह॑र्म्ये॒ष्ठाः शिश॑वो॒ न शु॒भ्रा व॒थ्सासो॒ न प्र॑क्री॒डिनः॑ पयो॒धाः ॥ प्रैषा॒मज्मे॑षु विथु॒रेव॑ रेजते॒ भूमि॒र्यामे॑षु॒ यद्ध॑ यु॒ञ्जते॑ शु॒भे । ते क्री॒डयो॒ धुन॑यो॒ भ्राज॑दृष्टयः स्व॒यं म॑हि॒त्वं प॑नयन्त॒ धूत॑यः ॥ उ॒प॒ह्व॒रेषु॒ यदचि॑द्ध्वं ॅय॒यिं ॅवय॑ इव मरुतः॒ केन॑ - [  ] \newline

\textbf{Pada Paata} \newline

कण्वाः᳚ । अ॒भि । प्रेति॑ । गा॒य॒त॒ ॥ अत्या॑सः । न । ये । म॒रुतः॑ । स्वञ्चः॑ । य॒क्ष॒दृश॒ इति॑ यक्ष - दृशः॑ । न । शु॒भय॑न्त । मर्याः᳚ ॥ ते । ह॒र्म्ये॒ष्ठा इति॑ हर्म्ये-स्थाः । शिश॑वः । न । शु॒भ्राः । व॒थ्सासः॑ । न । प्र॒क्री॒डिन॒ इति॑ प्र - क्री॒डिनः॑ । प॒यो॒धा इति॑ पयः - धाः ॥ प्रेति॑ । ए॒षा॒म् । अज्मे॑षु । वि॒थु॒रा । इ॒व॒ । रे॒ज॒ते॒ । भूमिः॑ । यामे॑षु । यत् । ह॒ । यु॒ञ्जते᳚ । शु॒भे ॥ ते । क्री॒डयः॑ । धुन॑यः । भ्राज॑दृष्टय॒ इति॒ भ्राज॑त् - ऋ॒ष्ट॒यः॒ । स्व॒यम् । म॒हि॒त्वमिति॑ महि - त्वम् । प॒न॒य॒न्त॒ । धूत॑यः ॥ उ॒प॒ह्व॒रेष्वित्यु॑प-ह्व॒रेषु॑ । यत् । अचि॑द्ध्वम् । य॒यिम् । वयः॑ । इ॒व॒ । म॒रु॒तः॒ । केन॑ ।  \newline


\textbf{Krama Paata} \newline

कण्वा॑ अ॒भि । अ॒भि प्र । प्र गा॑यत । गा॒य॒तेति॑ गायत ॥ अत्या॑सो॒ न । न ये । ये म॒रुतः॑ । म॒रुतः॒ स्वञ्चः॑ । स्वञ्चो॑ यक्ष॒दृशः॑ । य॒क्ष॒दृशो॒ न । य॒क्ष॒दृश॒ इति॑ यक्ष - दृशः॑ । न शु॒भय॑न्त । शु॒भय॑न्त॒ मर्याः᳚ । मर्या॒ इति॒ मर्याः᳚ ॥ ते ह॑र्म्ये॒ष्ठाः । ह॒र्म्ये॒ष्ठाः शिश॑वः । ह॒र्म्ये॒ष्ठा इति॑ हर्म्ये - स्थाः । शिश॑वो॒ न । न शु॒भ्राः । शु॒भ्रा व॒थ्सासः॑ । व॒थ्सासो॒ न । न प्र॑क्री॒डिनः॑ । प्र॒क्री॒डिनः॑ पयो॒धाः । प्र॒क्री॒डिन॒ इति॑ प्र - क्री॒डिनः॑ । प॒यो॒धा इति॑ पयः - धाः ॥ प्रैषा᳚म् । ए॒षा॒मज्मे॑षु । अज्मे॑षु विथु॒रा । वि॒थु॒रेव॑ । इ॒व॒ रे॒ज॒ते॒ । रे॒ज॒ते॒ भूमिः॑ । भूमि॒र् यामे॑षु । यामे॑षु॒ यत् । यद्ध॑ । ह॒ यु॒ञ्जते᳚ । यु॒ञ्जते॑ शु॒भे । शु॒भ इति॑ शु॒भे ॥ ते क्री॒डयः॑ । क्री॒डयो॒ धुन॑यः । धुन॑यो॒ भ्राज॑दृष्टयः । भ्राज॑दृष्टयः स्व॒यम् । भ्राज॑दृष्टय॒ इति॒ भ्राज॑त् - ऋ॒ष्ट॒यः॒ । स्व॒यम् म॑हि॒त्वम् । म॒हि॒त्वम् प॑नयन्त । म॒हि॒त्वमिति॑ महि - त्वम् । प॒न॒य॒न्त॒ धूत॑यः । धूत॑य॒ इति॒ धूत॑यः ॥ उ॒प॒ह्व॒रेषु॒ यत् । उ॒प॒ह्व॒रेष्वित्यु॑प - ह्व॒रेषु॑ । यदचि॑द्ध्वम् । अचि॑द्ध्वं ॅय॒यिम् । य॒यिं ॅवयः॑ । वय॑ इव । इ॒व॒ म॒रु॒तः॒ । म॒रु॒तः॒ केन॑ ( ) । केन॑ चित् \newline

\textbf{Jatai Paata} \newline

1. कण्वा॑ अ॒भ्य॑भि कण्वाः॒ कण्वा॑ अ॒भि । \newline
2. अ॒भि प्र प्रा भ्य॑भि प्र । \newline
3. प्र गा॑यत गायत॒ प्र प्र गा॑यत । \newline
4. गा॒य॒तेति॑ गायत । \newline
5. अत्या॑सो॒ न नात्या॑सो॒ अत्या॑सो॒ न । \newline
6. न ये ये न न ये । \newline
7. ये म॒रुतो॑ म॒रुतो॒ ये ये म॒रुतः॑ । \newline
8. म॒रुतः॒ स्वञ्चः॒ स्वञ्चो॑ म॒रुतो॑ म॒रुतः॒ स्वञ्चः॑ । \newline
9. स्वञ्चो॑ यक्ष॒दृशो॑ यक्ष॒दृशः॒ स्वञ्चः॒ स्वञ्चो॑ यक्ष॒दृशः॑ । \newline
10. य॒क्ष॒दृशो॒ न न य॑क्ष॒दृशो॑ यक्ष॒दृशो॒ न । \newline
11. य॒क्ष॒दृश॒ इति॑ यक्ष - दृशः॑ । \newline
12. न शु॒भय॑न्त शु॒भय॑न्त॒ न न शु॒भय॑न्त । \newline
13. शु॒भय॑न्त॒ मर्या॒ मर्याः᳚ शु॒भय॑न्त शु॒भय॑न्त॒ मर्याः᳚ । \newline
14. मर्या॒ इति॒ मर्याः᳚ । \newline
15. ते ह॑र्म्ये॒ष्ठा ह॑र्म्ये॒ष्ठा स्ते ते ह॑र्म्ये॒ष्ठाः । \newline
16. ह॒र्म्ये॒ष्ठाः शिश॑वः॒ शिश॑वो हर्म्ये॒ष्ठा ह॑र्म्ये॒ष्ठाः शिश॑वः । \newline
17. ह॒र्म्ये॒ष्ठा इति॑ हर्म्ये - स्थाः । \newline
18. शिश॑वो॒ न न शिश॑वः॒ शिश॑वो॒ न । \newline
19. न शु॒भ्राः शु॒भ्रा न न शु॒भ्राः । \newline
20. शु॒भ्रा व॒थ्सासो॑ व॒थ्सासः॑ शु॒भ्राः शु॒भ्रा व॒थ्सासः॑ । \newline
21. व॒थ्सासो॒ न न व॒थ्सासो॑ व॒थ्सासो॒ न । \newline
22. न प्र॑क्री॒डिनः॑ प्रक्री॒डिनो॒ न न प्र॑क्री॒डिनः॑ । \newline
23. प्र॒क्री॒डिनः॑ पयो॒धाः प॑यो॒धाः प्र॑क्री॒डिनः॑ प्रक्री॒डिनः॑ पयो॒धाः । \newline
24. प्र॒क्री॒डिन॒ इति॑ प्र - क्री॒डिनः॑ । \newline
25. प॒यो॒धा इति॑ पयः - धाः । \newline
26. प्रैषा॑ मेषा॒म् प्र प्रैषा᳚म् । \newline
27. ए॒षा॒ मज्मे॒ ष्वज्मे᳚ ष्वेषा मेषा॒ मज्मे॑षु । \newline
28. अज्मे॑षु विथु॒रा वि॑थु॒रा ऽज्मे॒ ष्वज्मे॑षु विथु॒रा । \newline
29. वि॒थु॒ रेवे॑व विथु॒रा वि॑थु॒ रेव॑ । \newline
30. इ॒व॒ रे॒ज॒ते॒ रे॒ज॒त॒ इ॒वे॒व॒ रे॒ज॒ते॒ । \newline
31. रे॒ज॒ते॒ भूमि॒र् भूमी॑ रेजते रेजते॒ भूमिः॑ । \newline
32. भूमि॒र् यामे॑षु॒ यामे॑षु॒ भूमि॒र् भूमि॒र् यामे॑षु । \newline
33. यामे॑षु॒ यद् यद् यामे॑षु॒ यामे॑षु॒ यत् । \newline
34. यद्ध॑ ह॒ यद् यद्ध॑ । \newline
35. ह॒ यु॒ञ्जते॑ यु॒ञ्जते॑ ह ह यु॒ञ्जते᳚ । \newline
36. यु॒ञ्जते॑ शु॒भे शु॒भे यु॒ञ्जते॑ यु॒ञ्जते॑ शु॒भे । \newline
37. शु॒भ इति॑ शु॒भे । \newline
38. ते क्री॒डयः॑ क्री॒डय॒ स्ते ते क्री॒डयः॑ । \newline
39. क्री॒डयो॒ धुन॑यो॒ धुन॑यः क्री॒डयः॑ क्री॒डयो॒ धुन॑यः । \newline
40. धुन॑यो॒ भ्राज॑दृष्टयो॒ भ्राज॑दृष्टयो॒ धुन॑यो॒ धुन॑यो॒ भ्राज॑दृष्टयः । \newline
41. भ्राज॑दृष्टयः स्व॒यꣳ स्व॒यम् भ्राज॑दृष्टयो॒ भ्राज॑दृष्टयः स्व॒यम् । \newline
42. भ्राज॑दृष्टय॒ इति॒ भ्राज॑त् - ऋ॒ष्ट॒यः॒ । \newline
43. स्व॒यम् म॑हि॒त्वम् म॑हि॒त्वꣳ स्व॒यꣳ स्व॒यम् म॑हि॒त्वम् । \newline
44. म॒हि॒त्वम् प॑नयन्त पनयन्त महि॒त्वम् म॑हि॒त्वम् प॑नयन्त । \newline
45. म॒हि॒त्वमिति॑ महि - त्वम् । \newline
46. प॒न॒य॒न्त॒ धूत॑यो॒ धूत॑यः पनयन्त पनयन्त॒ धूत॑यः । \newline
47. धूत॑य॒ इति॒ धूत॑यः । \newline
48. उ॒प॒ह्व॒रेषु॒ यद् यदु॑पह्व॒रे षू॑पह्व॒रेषु॒ यत् । \newline
49. उ॒प॒ह्व॒रेष्वित्यु॑प - ह्व॒रेषु॑ । \newline
50. यदचि॑द्ध्व॒ मचि॑द्ध्वं॒ ॅयद् यदचि॑द्ध्वम् । \newline
51. अचि॑द्ध्वं ॅय॒यिं ॅय॒यि मचि॑द्ध्व॒ मचि॑द्ध्वं ॅय॒यिम् । \newline
52. य॒यिं ॅवयो॒ वयो॑ य॒यिं ॅय॒यिं ॅवयः॑ । \newline
53. वय॑ इवेव॒ वयो॒ वय॑ इव । \newline
54. इ॒व॒ म॒रु॒तो॒ म॒रु॒त॒ इ॒वे॒ व॒ म॒रु॒तः॒ । \newline
55. म॒रु॒तः॒ केन॒ केन॑ मरुतो मरुतः॒ केन॑ । \newline
56. केन॑ चिच् चि॒त् केन॒ केन॑ चित् । \newline

\textbf{Ghana Paata } \newline

1. कण्वा॑ अ॒भ्य॑भि कण्वाः॒ कण्वा॑ अ॒भि प्र प्राभि कण्वाः॒ कण्वा॑ अ॒भि प्र । \newline
2. अ॒भि प्र प्राभ्य॑भि प्र गा॑यत गायत॒ प्राभ्य॑भि प्र गा॑यत । \newline
3. प्र गा॑यत गायत॒ प्र प्र गा॑यत । \newline
4. गा॒य॒तेति॑ गायत । \newline
5. अत्या॑सो॒ न नात्या॑सो॒ अत्या॑सो॒ न ये ये नात्या॑सो॒ अत्या॑सो॒ न ये । \newline
6. न ये ये न न ये म॒रुतो॑ म॒रुतो॒ ये न न ये म॒रुतः॑ । \newline
7. ये म॒रुतो॑ म॒रुतो॒ ये ये म॒रुतः॒ स्वञ्चः॒ स्वञ्चो॑ म॒रुतो॒ ये ये म॒रुतः॒ स्वञ्चः॑ । \newline
8. म॒रुतः॒ स्वञ्चः॒ स्वञ्चो॑ म॒रुतो॑ म॒रुतः॒ स्वञ्चो॑ यक्ष॒दृशो॑ यक्ष॒दृशः॒ स्वञ्चो॑ म॒रुतो॑ म॒रुतः॒ स्वञ्चो॑ यक्ष॒दृशः॑ । \newline
9. स्वञ्चो॑ यक्ष॒दृशो॑ यक्ष॒दृशः॒ स्वञ्चः॒ स्वञ्चो॑ यक्ष॒दृशो॒ न न य॑क्ष॒दृशः॒ स्वञ्चः॒ स्वञ्चो॑ यक्ष॒दृशो॒ न । \newline
10. य॒क्ष॒दृशो॒ न न य॑क्ष॒दृशो॑ यक्ष॒दृशो॒ न शु॒भय॑न्त शु॒भय॑न्त॒ न य॑क्ष॒दृशो॑ यक्ष॒दृशो॒ न शु॒भय॑न्त । \newline
11. य॒क्ष॒दृश॒ इति॑ यक्ष - दृशः॑ । \newline
12. न शु॒भय॑न्त शु॒भय॑न्त॒ न न शु॒भय॑न्त॒ मर्या॒ मर्याः᳚ शु॒भय॑न्त॒ न न शु॒भय॑न्त॒ मर्याः᳚ । \newline
13. शु॒भय॑न्त॒ मर्या॒ मर्याः᳚ शु॒भय॑न्त शु॒भय॑न्त॒ मर्याः᳚ । \newline
14. मर्या॒ इति॒ मर्याः᳚ । \newline
15. ते ह॑र्म्ये॒ष्ठा ह॑र्म्ये॒ष्ठा स्ते ते ह॑र्म्ये॒ष्ठाः शिश॑वः॒ शिश॑वो हर्म्ये॒ष्ठा स्ते ते ह॑र्म्ये॒ष्ठाः शिश॑वः । \newline
16. ह॒र्म्ये॒ष्ठाः शिश॑वः॒ शिश॑वो हर्म्ये॒ष्ठा ह॑र्म्ये॒ष्ठाः शिश॑वो॒ न न शिश॑वो हर्म्ये॒ष्ठा ह॑र्म्ये॒ष्ठाः शिश॑वो॒ न । \newline
17. ह॒र्म्ये॒ष्ठा इति॑ हर्म्ये - स्थाः । \newline
18. शिश॑वो॒ न न शिश॑वः॒ शिश॑वो॒ न शु॒भ्राः शु॒भ्रा न शिश॑वः॒ शिश॑वो॒ न शु॒भ्राः । \newline
19. न शु॒भ्राः शु॒भ्रा न न शु॒भ्रा व॒थ्सासो॑ व॒थ्सासः॑ शु॒भ्रा न न शु॒भ्रा व॒थ्सासः॑ । \newline
20. शु॒भ्रा व॒थ्सासो॑ व॒थ्सासः॑ शु॒भ्राः शु॒भ्रा व॒थ्सासो॒ न न व॒थ्सासः॑ शु॒भ्राः शु॒भ्रा व॒थ्सासो॒ न । \newline
21. व॒थ्सासो॒ न न व॒थ्सासो॑ व॒थ्सासो॒ न प्र॑क्री॒डिनः॑ प्रक्री॒डिनो॒ न व॒थ्सासो॑ व॒थ्सासो॒ न प्र॑क्री॒डिनः॑ । \newline
22. न प्र॑क्री॒डिनः॑ प्रक्री॒डिनो॒ न न प्र॑क्री॒डिनः॑ पयो॒धाः प॑यो॒धाः प्र॑क्री॒डिनो॒ न न प्र॑क्री॒डिनः॑ पयो॒धाः । \newline
23. प्र॒क्री॒डिनः॑ पयो॒धाः प॑यो॒धाः प्र॑क्री॒डिनः॑ प्रक्री॒डिनः॑ पयो॒धाः । \newline
24. प्र॒क्री॒डिन॒ इति॑ प्र - क्री॒डिनः॑ । \newline
25. प॒यो॒धा इति॑ पयः - धाः । \newline
26. प्रैषा॑ मेषा॒म् प्र प्रैषा॒ मज्मे॒ ष्वज्मे᳚ ष्वेषा॒म् प्र प्रैषा॒ मज्मे॑षु । \newline
27. ए॒षा॒ मज्मे॒ ष्वज्मे᳚ ष्वेषा मेषा॒ मज्मे॑षु विथु॒रा वि॑थु॒रा ऽज्मे᳚ ष्वेषा मेषा॒ मज्मे॑षु विथु॒रा । \newline
28. अज्मे॑षु विथु॒रा वि॑थु॒रा ऽज्मे॒ ष्वज्मे॑षु विथु॒रेवे॑व विथु॒रा ऽज्मे॒ ष्वज्मे॑षु विथु॒रेव॑ । \newline
29. वि॒थु॒रेवे॑व विथु॒रा वि॑थु॒रेव॑ रेजते रेजत इव विथु॒रा वि॑थु॒रेव॑ रेजते । \newline
30. इ॒व॒ रे॒ज॒ते॒ रे॒ज॒त॒ इ॒वे॒व॒ रे॒ज॒ते॒ भूमि॒र् भूमी॑ रेजत इवेव रेजते॒ भूमिः॑ । \newline
31. रे॒ज॒ते॒ भूमि॒र् भूमी॑ रेजते रेजते॒ भूमि॒र् यामे॑षु॒ यामे॑षु॒ भूमी॑ रेजते रेजते॒ भूमि॒र् यामे॑षु । \newline
32. भूमि॒र् यामे॑षु॒ यामे॑षु॒ भूमि॒र् भूमि॒र् यामे॑षु॒ यद् यद् यामे॑षु॒ भूमि॒र् भूमि॒र् यामे॑षु॒ यत् । \newline
33. यामे॑षु॒ यद् यद् यामे॑षु॒ यामे॑षु॒ यद्ध॑ ह॒ यद् यामे॑षु॒ यामे॑षु॒ यद्ध॑ । \newline
34. यद्ध॑ ह॒ यद् यद्ध॑ यु॒ञ्जते॑ यु॒ञ्जते॑ ह॒ यद् यद्ध॑ यु॒ञ्जते᳚ । \newline
35. ह॒ यु॒ञ्जते॑ यु॒ञ्जते॑ ह ह यु॒ञ्जते॑ शु॒भे शु॒भे यु॒ञ्जते॑ ह ह यु॒ञ्जते॑ शु॒भे । \newline
36. यु॒ञ्जते॑ शु॒भे शु॒भे यु॒ञ्जते॑ यु॒ञ्जते॑ शु॒भे । \newline
37. शु॒भ इति॑ शु॒भे । \newline
38. ते क्री॒डयः॑ क्री॒डय॒ स्ते ते क्री॒डयो॒ धुन॑यो॒ धुन॑यः क्री॒डय॒ स्ते ते क्री॒डयो॒ धुन॑यः । \newline
39. क्री॒डयो॒ धुन॑यो॒ धुन॑यः क्री॒डयः॑ क्री॒डयो॒ धुन॑यो॒ भ्राज॑दृष्टयो॒ भ्राज॑दृष्टयो॒ धुन॑यः क्री॒डयः॑ क्री॒डयो॒ धुन॑यो॒ भ्राज॑दृष्टयः । \newline
40. धुन॑यो॒ भ्राज॑दृष्टयो॒ भ्राज॑दृष्टयो॒ धुन॑यो॒ धुन॑यो॒ भ्राज॑दृष्टयः स्व॒यꣳ स्व॒यम् भ्राज॑दृष्टयो॒ धुन॑यो॒ धुन॑यो॒ भ्राज॑दृष्टयः स्व॒यम् । \newline
41. भ्राज॑दृष्टयः स्व॒यꣳ स्व॒यम् भ्राज॑दृष्टयो॒ भ्राज॑दृष्टयः स्व॒यम् म॑हि॒त्वम् म॑हि॒त्वꣳ स्व॒यम् भ्राज॑दृष्टयो॒ भ्राज॑दृष्टयः स्व॒यम् म॑हि॒त्वम् । \newline
42. भ्राज॑दृष्टय॒ इति॒ भ्राज॑त् - ऋ॒ष्ट॒यः॒ । \newline
43. स्व॒यम् म॑हि॒त्वम् म॑हि॒त्वꣳ स्व॒यꣳ स्व॒यम् म॑हि॒त्वम् प॑नयन्त पनयन्त महि॒त्वꣳ स्व॒यꣳ स्व॒यम् म॑हि॒त्वम् प॑नयन्त । \newline
44. म॒हि॒त्वम् प॑नयन्त पनयन्त महि॒त्वम् म॑हि॒त्वम् प॑नयन्त॒ धूत॑यो॒ धूत॑यः पनयन्त महि॒त्वम् म॑हि॒त्वम् प॑नयन्त॒ धूत॑यः । \newline
45. म॒हि॒त्वमिति॑ महि - त्वम् । \newline
46. प॒न॒य॒न्त॒ धूत॑यो॒ धूत॑यः पनयन्त पनयन्त॒ धूत॑यः । \newline
47. धूत॑य॒ इति॒ धूत॑यः । \newline
48. उ॒प॒ह्व॒रेषु॒ यद् यदु॑पह्व॒रेषू॑ पह्व॒रेषु॒ यदचि॑द्ध्व॒ मचि॑द्ध्वं॒ ॅयदु॑पह्व॒रेषू॑ पह्व॒रेषु॒ यदचि॑द्ध्वम् । \newline
49. उ॒प॒ह्व॒रेष्वित्यु॑प - ह्व॒रेषु॑ । \newline
50. यदचि॑द्ध्व॒ मचि॑द्ध्वं॒ ॅयद् यदचि॑द्ध्वं ॅय॒यिं ॅय॒यि मचि॑द्ध्वं॒ ॅयद् यदचि॑द्ध्वं ॅय॒यिम् । \newline
51. अचि॑द्ध्वं ॅय॒यिं ॅय॒यि मचि॑द्ध्व॒ मचि॑द्ध्वं ॅय॒यिं ॅवयो॒ वयो॑ य॒यि मचि॑द्ध्व॒ मचि॑द्ध्वं ॅय॒यिं ॅवयः॑ । \newline
52. य॒यिं ॅवयो॒ वयो॑ य॒यिं ॅय॒यिं ॅवय॑ इवेव॒ वयो॑ य॒यिं ॅय॒यिं ॅवय॑ इव । \newline
53. वय॑ इवेव॒ वयो॒ वय॑ इव मरुतो मरुत इव॒ वयो॒ वय॑ इव मरुतः । \newline
54. इ॒व॒ म॒रु॒तो॒ म॒रु॒त॒ इ॒वे॒व॒ म॒रु॒तः॒ केन॒ केन॑ मरुत इवेव मरुतः॒ केन॑ । \newline
55. म॒रु॒तः॒ केन॒ केन॑ मरुतो मरुतः॒ केन॑ चिच् चि॒त् केन॑ मरुतो मरुतः॒ केन॑ चित् । \newline
56. केन॑ चिच् चि॒त् केन॒ केन॑ चित् प॒था प॒था चि॒त् केन॒ केन॑ चित् प॒था । \newline
\pagebreak
\markright{ TS 4.3.13.8  \hfill https://www.vedavms.in \hfill}

\section{ TS 4.3.13.8 }

\textbf{TS 4.3.13.8 } \newline
\textbf{Samhita Paata} \newline

चित् प॒था । श्चोत॑न्ति॒ कोशा॒ उप॑ वो॒ रथे॒ष्वा घृ॒तमु॑क्षता॒ मधु॑वर्ण॒मर्च॑ते ॥ अ॒ग्निम॑ग्निꣳ॒॒ हवी॑मभिः॒ सदा॑ हवन्त वि॒श्पतिं᳚ । ह॒व्य॒वाहं॑ पुरुप्रि॒यं ॥ तꣳ हि शश्व॑न्त॒ ईड॑ते स्रु॒चा दे॒वं घृ॑त॒श्चुता᳚ । अ॒ग्निꣳ ह॒व्याय॒ वोढ॑वे ॥ इन्द्रा᳚ग्नी रोच॒ना दि॒वः > 1, श्नथ॑द्वृ॒त्र >2, मिन्द्रं॑ ॅवो वि॒श्वत॒स्परी>3, न्द्रं॒ नरो॒ >4, विश्व॑कर्मन्. ह॒विषा॑ वावृधा॒नो>5,विश्व॑कर्मन्. ह॒विषा॒ वर्ध॑नेन >6 ॥ \newline

\textbf{Pada Paata} \newline

चि॒त् । प॒था ॥ श्चोत॑न्ति । कोशाः᳚ । उपेति॑ । वः॒ । रथे॑षु । एति॑ । घृ॒तम् । उ॒क्ष॒त॒ । मधु॑वर्ण॒मिति॒ मधु॑ - व॒र्ण॒म् । अर्च॑ते ॥ अ॒ग्निम॑ग्नि॒मित्य॒ग्निम् - अ॒ग्नि॒म् । हवी॑मभि॒रिति॒ हवी॑म - भिः॒ । सदा᳚ । ह॒व॒न्त॒ । वि॒श्पति᳚म् ॥ ह॒व्य॒वाह॒मिति॑ हव्य - वाह᳚म् । पु॒रु॒प्रि॒यमिति॑ पुरु - प्रि॒यम् ॥ तम् । हि । शश्व॑न्तः । ईड॑ते । स्रु॒चा । दे॒वम् । घृ॒त॒श्चुतेति॑ घृत - श्चुता᳚ ॥ अ॒ग्निम् । ह॒व्याय॑ । वोढ॑वे ॥ इन्द्रा᳚ग्नी॒ इतीन्द्र॑ - अ॒ग्नी॒ । रो॒च॒ना । दि॒वः । श्नथ॑त् । वृ॒त्रम् । इन्द्र᳚म् । वः॒ । वि॒श्वतः॑ । परीति॑ । इन्द्र᳚म् । नरः॑ । विश्व॑कर्म॒न्निति॒ विश्व॑ - क॒र्म॒न्न् । ह॒विषा᳚ । वा॒वृ॒धा॒नः । विश्व॑कर्म॒न्निति॒ विश्व॑ - क॒र्म॒न्न् । ह॒विषा᳚ । वद्‌र्ध॑नेन ॥  \newline


\textbf{Krama Paata} \newline

चि॒त् प॒था । प॒थेति॑ प॒था ॥ श्चोत॑न्ति॒ कोशाः᳚ । कोशा॒ उप॑ । उप॑ वः । वो॒ रथे॑षु । रथे॒ष्वा । आ घृ॒तम् । घृ॒तमु॑क्षत । उ॒क्ष॒ता॒ मधु॑वर्णम् । मधु॑वर्ण॒मर्च॑ते । मधु॑वर्ण॒मिति॒ मधु॑ - व॒र्ण॒म् । अर्च॑त॒ इत्यर्च॑ते ॥ अ॒ग्निम॑ग्निꣳ॒॒ हवी॑मभिः । अ॒ग्निम॑ग्नि॒मित्य॒ग्निम् - अ॒ग्नि॒म् । हवी॑मभिः॒ सदा᳚ । हवी॑मभि॒रिति॒ हवी॑म - भिः॒ । सदा॑ हवन्त । ह॒व॒न्त॒ वि॒श्पति᳚म् । वि॒श्पति॒मिति॑ वि॒श्पति᳚म् ॥ ह॒व्य॒वाह॑म् पुरुप्रि॒यम् । ह॒व्य॒वाह॒मिति॑ हव्य - वाह᳚म् । पु॒रु॒प्रि॒यमिति॑ पुरु - प्रि॒यम् ॥ तꣳ हि । हि शश्व॑न्तः । शश्व॑न्त॒ ईड॑ते । ईड॑ते स्रु॒चा । स्रु॒चा दे॒वम् । दे॒वम् घृ॑त॒श्चुता᳚ । घृ॒त॒श्चुतेति॑ घृत - श्चुता᳚ ॥ अ॒ग्निꣳ ह॒व्याय॑ । ह॒व्याय॒ वोढ॑वे । वोढ॑व॒ इति॒ वोढ॑वे ॥ इन्द्रा᳚ग्नी रोच॒ना । इन्द्रा᳚ग्नी॒ इतीन्द्र॑ - अ॒ग्नी॒ । रो॒च॒ना दि॒वः । दि॒वः श्ञथ॑त् । श्ञथ॑द् वृ॒त्रम् । वृ॒त्रमिन्द्र᳚म् । इन्द्रं॑ ॅवः । वो॒ वि॒श्वतः॑ । वि॒श्वत॒स्परि॑ । परीन्द्र᳚म् । इन्द्रं॒ नरः॑ । नरो॒ विश्व॑कर्मन्न् । विश्व॑कर्मन्. ह॒विषा᳚ । विश्व॑कर्म॒न्निति॒ विश्व॑ - क॒र्म॒न्न्॒ । ह॒विषा॑ वावृधा॒नः । वा॒वृ॒धा॒नो विश्व॑कर्मन्न् । विश्व॑कर्मन्. ह॒विषा᳚ । विश्व॑कर्म॒न्निति॒ विश्व॑ - क॒र्म॒न्न्॒ । ह॒विषा॒ वर्द्ध॑नेन । वर्द्ध॑ने॒नेति॒ वर्द्ध॑नेन । \newline

\textbf{Jatai Paata} \newline

1. चि॒त् प॒था प॒था चि॑च् चित् प॒था । \newline
2. प॒थेति॑ प॒था । \newline
3. श्चोत॑न्ति॒ कोशाः॒ कोशाः॒ श्चोत॑न्ति॒ श्चोत॑न्ति॒ कोशाः᳚ । \newline
4. कोशा॒ उपोप॒ कोशाः॒ कोशा॒ उप॑ । \newline
5. उप॑ वो व॒ उपोप॑ वः । \newline
6. वो॒ रथे॑षु॒ रथे॑षु वो वो॒ रथे॑षु । \newline
7. रथे॒ष्वा रथे॑षु॒ रथे॒ष्वा । \newline
8. आ घृ॒तम् घृ॒त मा घृ॒तम् । \newline
9. घृ॒त मु॑क्ष तोक्षत घृ॒तम् घृ॒त मु॑क्षत । \newline
10. उ॒क्ष॒ता॒ मधु॑वर्ण॒म् मधु॑वर्ण मुक्ष तोक्षता॒ मधु॑वर्णम् । \newline
11. मधु॑वर्ण॒ मर्च॑ते॒ अर्च॑ते॒ मधु॑वर्ण॒म् मधु॑वर्ण॒ मर्च॑ते । \newline
12. मधु॑वर्ण॒मिति॒ मधु॑ - व॒र्ण॒म् । \newline
13. अर्च॑त॒ इत्यर्च॑ते । \newline
14. अ॒ग्निम॑ग्निꣳ॒॒ हवी॑मभि॒र्॒. हवी॑मभि र॒ग्निम॑ग्नि म॒ग्नि म॑ग्निꣳ॒॒ हवी॑मभिः । \newline
15. अ॒ग्निम॑ग्नि॒मित्य॒ग्निम् - अ॒ग्नि॒म् । \newline
16. हवी॑मभिः॒ सदा॒ सदा॒ हवी॑मभि॒र्॒. हवी॑मभिः॒ सदा᳚ । \newline
17. हवी॑मभि॒रिति॒ हवी॑म - भिः॒ । \newline
18. सदा॑ हवन्त हवन्त॒ सदा॒ सदा॑ हवन्त । \newline
19. ह॒व॒न्त॒ वि॒श्पतिं॑ ॅवि॒श्पतिꣳ॑ हवन्त हवन्त वि॒श्पति᳚म् । \newline
20. वि॒श्पति॒मिति॑ वि॒श्पति᳚म् । \newline
21. ह॒व्य॒वाह॑म् पुरुप्रि॒यम् पु॑रुप्रि॒यꣳ ह॑व्य॒वाहꣳ॑ हव्य॒वाह॑म् पुरुप्रि॒यम् । \newline
22. ह॒व्य॒वाह॒मिति॑ हव्य - वाह᳚म् । \newline
23. पु॒रु॒प्रि॒यमिति॑ पुरु - प्रि॒यम् । \newline
24. तꣳ हि हि तम् तꣳ हि । \newline
25. हि शश्व॑न्तः॒ शश्व॑न्तो॒ हि हि शश्व॑न्तः । \newline
26. शश्व॑न्त॒ ईड॑त॒ ईड॑ते॒ शश्व॑न्तः॒ शश्व॑न्त॒ ईड॑ते । \newline
27. ईड॑ते स्रु॒चा स्रु॒चे ड॑त॒ ईड॑ते स्रु॒चा । \newline
28. स्रु॒चा दे॒वम् दे॒वꣳ स्रु॒चा स्रु॒चा दे॒वम् । \newline
29. दे॒वम् घृ॑त॒श्चुता॑ घृत॒श्चुता॑ दे॒वम् दे॒वम् घृ॑त॒श्चुता᳚ । \newline
30. घृ॒त॒श्चुतेति॑ घृत - श्चुता᳚ । \newline
31. अ॒ग्निꣳ ह॒व्याय॑ ह॒व्याया॒ग्नि म॒ग्निꣳ ह॒व्याय॑ । \newline
32. ह॒व्याय॒ वोढ॑वे॒ वोढ॑वे ह॒व्याय॑ ह॒व्याय॒ वोढ॑वे । \newline
33. वोढ॑व॒ इति॒ वोढ॑वे । \newline
34. इन्द्रा᳚ग्नी रोच॒ना रो॑च॒ने न्द्रा᳚ग्नी॒ इन्द्रा᳚ग्नी रोच॒ना । \newline
35. इन्द्रा᳚ग्नी॒ इतीन्द्र॑ - अ॒ग्नी॒ । \newline
36. रो॒च॒ना दि॒वो दि॒वो रो॑च॒ना रो॑च॒ना दि॒वः । \newline
37. दि॒वः श्ञथ॒ च्छ्ञथ॑द् दि॒वो दि॒वः श्ञथ॑त् । \newline
38. श्ञथ॑द् वृ॒त्रं ॅवृ॒त्रꣳ श्ञथ॒ च्छ्ञथ॑द् वृ॒त्रम् । \newline
39. वृ॒त्र मिन्द्र॒ मिन्द्रं॑ ॅवृ॒त्रं ॅवृ॒त्र मिन्द्र᳚म् । \newline
40. इन्द्रं॑ ॅवो व॒ इन्द्र॒ मिन्द्रं॑ ॅवः । \newline
41. वो॒ वि॒श्वतो॑ वि॒श्वतो॑ वो वो वि॒श्वतः॑ । \newline
42. वि॒श्वत॒ स्परि॒ परि॑ वि॒श्वतो॑ वि॒श्वत॒ स्परि॑ । \newline
43. परीन्द्र॒ मिन्द्र॒म् परि॒ परीन्द्र᳚म् । \newline
44. इन्द्र॒म् नरो॒ नर॒ इन्द्र॒ मिन्द्र॒म् नरः॑ । \newline
45. नरो॒ विश्व॑कर्म॒न्॒. विश्व॑कर्म॒न् नरो॒ नरो॒ विश्व॑कर्मन्न् । \newline
46. विश्व॑कर्मन्. ह॒विषा॑ ह॒विषा॒ विश्व॑कर्म॒न्.॒ विश्व॑कर्मन्. ह॒विषा᳚ । \newline
47. विश्व॑कर्म॒न्निति॒ विश्व॑ - क॒र्म॒न्न् । \newline
48. ह॒विषा॑ वावृधा॒नो वा॑वृधा॒नो ह॒विषा॑ ह॒विषा॑ वावृधा॒नः । \newline
49. वा॒वृ॒धा॒नो विश्व॑कर्म॒न्॒. विश्व॑कर्मन्. वावृधा॒नो वा॑वृधा॒नो विश्व॑कर्मन्न् । \newline
50. विश्व॑कर्मन्. ह॒विषा॑ ह॒विषा॒ विश्व॑कर्म॒न्॒. विश्व॑कर्मन्. ह॒विषा᳚ । \newline
51. विश्व॑कर्म॒न्निति॒ विश्व॑ - क॒र्म॒न्न् । \newline
52. ह॒विषा॒ वर्द्ध॑नेन॒ वर्द्ध॑नेन ह॒विषा॑ 
ह॒विषा॒ वर्द्ध॑नेन । \newline
53. वर्द्ध॑ने॒नेति॒ वर्द्ध॑नेन । \newline

\textbf{Ghana Paata } \newline

1. चि॒त् प॒था प॒था चि॑च् चित् प॒था । \newline
2. प॒थेति॑ प॒था । \newline
3. श्चोत॑न्ति॒ कोशाः॒ कोशाः॒ श्चोत॑न्ति॒ श्चोत॑न्ति॒ कोशा॒ उपोप॒ कोशाः॒ श्चोत॑न्ति॒ श्चोत॑न्ति॒ कोशा॒ उप॑ । \newline
4. कोशा॒ उपोप॒ कोशाः॒ कोशा॒ उप॑ वो व॒ उप॒ कोशाः॒ कोशा॒ उप॑ वः । \newline
5. उप॑ वो व॒ उपोप॑ वो॒ रथे॑षु॒ रथे॑षु व॒ उपोप॑ वो॒ रथे॑षु । \newline
6. वो॒ रथे॑षु॒ रथे॑षु वो वो॒ रथे॒ष्वा रथे॑षु वो वो॒ रथे॒ष्वा । \newline
7. रथे॒ष्वा रथे॑षु॒ रथे॒ष्वा घृ॒तम् घृ॒त मा रथे॑षु॒ रथे॒ष्वा घृ॒तम् । \newline
8. आ घृ॒तम् घृ॒त मा घृ॒त मु॑क्षतोक्षत घृ॒त मा घृ॒त मु॑क्षत । \newline
9. घृ॒त मु॑क्ष तोक्षत घृ॒तम् घृ॒त मु॑क्षता॒ मधु॑वर्ण॒म् मधु॑वर्ण मुक्षत घृ॒तम् घृ॒त मु॑क्षता॒ मधु॑वर्णम् । \newline
10. उ॒क्ष॒ता॒ मधु॑वर्ण॒म् मधु॑वर्ण मुक्ष तोक्षता॒ मधु॑वर्ण॒ मर्च॑ते॒ अर्च॑ते॒ मधु॑वर्ण मुक्ष तोक्षता॒ मधु॑वर्ण॒ मर्च॑ते । \newline
11. मधु॑वर्ण॒ मर्च॑ते॒ अर्च॑ते॒ मधु॑वर्ण॒म् मधु॑वर्ण॒ मर्च॑ते । \newline
12. मधु॑वर्ण॒मिति॒ मधु॑ - व॒र्ण॒म् । \newline
13. अर्च॑त॒ इत्यर्च॑ते । \newline
14. अ॒ग्निम॑ग्निꣳ॒॒ हवी॑मभि॒र्॒. हवी॑मभि र॒ग्निम॑ग्नि म॒ग्निम॑ग्निꣳ॒॒ हवी॑मभिः॒ सदा॒ सदा॒ हवी॑मभि र॒ग्निम॑ग्नि म॒ग्निम॑ग्निꣳ॒॒ हवी॑मभिः॒ सदा᳚ । \newline
15. अ॒ग्निम॑ग्नि॒मित्य॒ग्निम् - अ॒ग्नि॒म् । \newline
16. हवी॑मभिः॒ सदा॒ सदा॒ हवी॑मभि॒र्॒. हवी॑मभिः॒ सदा॑ हवन्त हवन्त॒ सदा॒ हवी॑मभि॒र्॒. हवी॑मभिः॒ सदा॑ हवन्त । \newline
17. हवी॑मभि॒रिति॒ हवी॑म - भिः॒ । \newline
18. सदा॑ हवन्त हवन्त॒ सदा॒ सदा॑ हवन्त वि॒श्पतिं॑ ॅवि॒श्पतिꣳ॑ हवन्त॒ सदा॒ सदा॑ हवन्त वि॒श्पति᳚म् । \newline
19. ह॒व॒न्त॒ वि॒श्पतिं॑ ॅवि॒श्पतिꣳ॑ हवन्त हवन्त वि॒श्पति᳚म् । \newline
20. वि॒श्पति॒मिति॑ वि॒श्पति᳚म् । \newline
21. ह॒व्य॒वाह॑म् पुरुप्रि॒यम् पु॑रुप्रि॒यꣳ ह॑व्य॒वाहꣳ॑ हव्य॒वाह॑म् पुरुप्रि॒यम् । \newline
22. ह॒व्य॒वाह॒मिति॑ हव्य - वाह᳚म् । \newline
23. पु॒रु॒प्रि॒यमिति॑ पुरु - प्रि॒यम् । \newline
24. तꣳ हि हि तम् तꣳ हि शश्व॑न्तः॒ शश्व॑न्तो॒ हि तम् तꣳ हि शश्व॑न्तः । \newline
25. हि शश्व॑न्तः॒ शश्व॑न्तो॒ हि हि शश्व॑न्त॒ ईड॑त॒ ईड॑ते॒ शश्व॑न्तो॒ हि हि शश्व॑न्त॒ ईड॑ते । \newline
26. शश्व॑न्त॒ ईड॑त॒ ईड॑ते॒ शश्व॑न्तः॒ शश्व॑न्त॒ ईड॑ते स्रु॒चा स्रु॒चेड॑ते॒ शश्व॑न्तः॒ शश्व॑न्त॒ ईड॑ते स्रु॒चा । \newline
27. ईड॑ते स्रु॒चा स्रु॒चेड॑त॒ ईड॑ते स्रु॒चा दे॒वम् दे॒वꣳ स्रु॒चेड॑त॒ ईड॑ते स्रु॒चा दे॒वम् । \newline
28. स्रु॒चा दे॒वम् दे॒वꣳ स्रु॒चा स्रु॒चा दे॒वम् घृ॑त॒श्चुता॑ घृत॒श्चुता॑ दे॒वꣳ स्रु॒चा स्रु॒चा दे॒वम् घृ॑त॒श्चुता᳚ । \newline
29. दे॒वम् घृ॑त॒श्चुता॑ घृत॒श्चुता॑ दे॒वम् दे॒वम् घृ॑त॒श्चुता᳚ । \newline
30. घृ॒त॒श्चुतेति॑ घृत - श्चुता᳚ । \newline
31. अ॒ग्निꣳ ह॒व्याय॑ ह॒व्याया॒ग्नि म॒ग्निꣳ ह॒व्याय॒ वोढ॑वे॒ वोढ॑वे ह॒व्याया॒ग्नि म॒ग्निꣳ ह॒व्याय॒ वोढ॑वे । \newline
32. ह॒व्याय॒ वोढ॑वे॒ वोढ॑वे ह॒व्याय॑ ह॒व्याय॒ वोढ॑वे । \newline
33. वोढ॑व॒ इति॒ वोढ॑वे । \newline
34. इन्द्रा᳚ग्नी रोच॒ना रो॑च॒नेन्द्रा᳚ग्नी॒ इन्द्रा᳚ग्नी रोच॒ना दि॒वो दि॒वो रो॑च॒नेन्द्रा᳚ग्नी॒ इन्द्रा᳚ग्नी रोच॒ना दि॒वः । \newline
35. इन्द्रा᳚ग्नी॒ इतीन्द्र॑ - अ॒ग्नी॒ । \newline
36. रो॒च॒ना दि॒वो दि॒वो रो॑च॒ना रो॑च॒ना दि॒वः श्ञथ॒ च्छ्ञथ॑द् दि॒वो रो॑च॒ना रो॑च॒ना दि॒वः श्ञथ॑त् । \newline
37. दि॒वः श्ञथ॒ च्छ्ञथ॑द् दि॒वो दि॒वः श्ञथ॑द् वृ॒त्रं ॅवृ॒त्रꣳ श्ञथ॑द् दि॒वो दि॒वः श्ञथ॑द् वृ॒त्रम् । \newline
38. श्ञथ॑द् वृ॒त्रं ॅवृ॒त्रꣳ श्ञथ॒ च्छ्ञथ॑द् वृ॒त्र मिन्द्र॒ मिन्द्रं॑ ॅवृ॒त्रꣳ श्ञथ॒ च्छ्ञथ॑द् वृ॒त्र मिन्द्र᳚म् । \newline
39. वृ॒त्र मिन्द्र॒ मिन्द्रं॑ ॅवृ॒त्रं ॅवृ॒त्र मिन्द्रं॑ ॅवो व॒ इन्द्रं॑ ॅवृ॒त्रं ॅवृ॒त्र मिन्द्रं॑ ॅवः । \newline
40. इन्द्रं॑ ॅवो व॒ इन्द्र॒ मिन्द्रं॑ ॅवो वि॒श्वतो॑ वि॒श्वतो॑ व॒ इन्द्र॒ मिन्द्रं॑ ॅवो वि॒श्वतः॑ । \newline
41. वो॒ वि॒श्वतो॑ वि॒श्वतो॑ वो वो वि॒श्वत॒ स्परि॒ परि॑ वि॒श्वतो॑ वो वो वि॒श्वत॒ स्परि॑ । \newline
42. वि॒श्वत॒ स्परि॒ परि॑ वि॒श्वतो॑ वि॒श्वत॒ स्परीन्द्र॒ मिन्द्र॒म् परि॑ वि॒श्वतो॑ वि॒श्वत॒ स्परीन्द्र᳚म् । \newline
43. परीन्द्र॒ मिन्द्र॒म् परि॒ परीन्द्र॒म् नरो॒ नर॒ इन्द्र॒म् परि॒ परीन्द्र॒म् नरः॑ । \newline
44. इन्द्र॒म् नरो॒ नर॒ इन्द्र॒ मिन्द्र॒म् नरो॒ विश्व॑कर्म॒न्॒. विश्व॑कर्म॒न् नर॒ इन्द्र॒ मिन्द्र॒म् नरो॒ विश्व॑कर्मन्न् । \newline
45. नरो॒ विश्व॑कर्म॒न्॒. विश्व॑कर्म॒न् नरो॒ नरो॒ विश्व॑कर्मन्. ह॒विषा॑ ह॒विषा॒ विश्व॑कर्म॒न् नरो॒ नरो॒ विश्व॑कर्मन्. ह॒विषा᳚ । \newline
46. विश्व॑कर्मन्. ह॒विषा॑ ह॒विषा॒ विश्व॑कर्म॒न्॒. विश्व॑कर्मन्. ह॒विषा॑ वावृधा॒नो वा॑वृधा॒नो ह॒विषा॒ विश्व॑कर्म॒न्॒. विश्व॑कर्मन्. ह॒विषा॑ वावृधा॒नः । \newline
47. विश्व॑कर्म॒न्निति॒ विश्व॑ - क॒र्म॒न्न् । \newline
48. ह॒विषा॑ वावृधा॒नो वा॑वृधा॒नो ह॒विषा॑ ह॒विषा॑ वावृधा॒नो विश्व॑कर्म॒न्॒. विश्व॑कर्मन्. वावृधा॒नो ह॒विषा॑ ह॒विषा॑ वावृधा॒नो विश्व॑कर्मन्न् । \newline
49. वा॒वृ॒धा॒नो विश्व॑कर्म॒न्॒. विश्व॑कर्मन्. वावृधा॒नो वा॑वृधा॒नो विश्व॑कर्मन्. ह॒विषा॑ ह॒विषा॒ विश्व॑कर्मन्. वावृधा॒नो वा॑वृधा॒नो विश्व॑कर्मन्. ह॒विषा᳚ । \newline
50. विश्व॑कर्मन्. ह॒विषा॑ ह॒विषा॒ विश्व॑कर्म॒न्॒. विश्व॑कर्मन्. ह॒विषा॒ वर्द्ध॑नेन॒ वर्द्ध॑नेन ह॒विषा॒ विश्व॑कर्म॒न्॒. विश्व॑कर्मन्. ह॒विषा॒ वर्द्ध॑नेन । \newline
51. विश्व॑कर्म॒न्निति॒ विश्व॑ - क॒र्म॒न्न् । \newline
52. ह॒विषा॒ वर्द्ध॑नेन॒ वर्द्ध॑नेन ह॒विषा॑ ह॒विषा॒ वर्द्ध॑नेन । \newline
53. वर्द्ध॑ने॒नेति॒ वर्द्ध॑नेन । \newline
\pagebreak


\end{document}