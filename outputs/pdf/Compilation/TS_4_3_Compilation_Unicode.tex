\documentclass[17pt]{extarticle}
\usepackage{babel}
\usepackage{fontspec}
\usepackage{polyglossia}
\usepackage{extsizes}



\setmainlanguage{sanskrit}
\setotherlanguages{english} %% or other languages
\setlength{\parindent}{0pt}
\pagestyle{myheadings}
\newfontfamily\devanagarifont[Script=Devanagari]{AdishilaVedic}


\newcommand{\VAR}[1]{}
\newcommand{\BLOCK}[1]{}




\begin{document}
\begin{titlepage}
    \begin{center}
 
\begin{sanskrit}
    { \Huge
    कृष्ण यजुर्वेदीय तैत्तिरीय संहिता,पद,जटा,घन पाठः 
    }
    \\
    \vspace{2.5cm}
    \mbox{ \Huge
    4.3      चतुर्थकाण्डे तृतीयः प्रश्नः - चितिवर्णनं   }
\end{sanskrit}
\end{center}

\end{titlepage}
\tableofcontents
\pagebreak

\markright{ TS 4.3.1.1  \hfill https://www.vedavms.in \hfill}
\addcontentsline{toc}{section}{ TS 4.3.1.1 }
\section*{ TS 4.3.1.1 }

\textbf{TS 4.3.1.1 } \newline
\textbf{Samhita Paata} \newline

अ॒पां त्वेम᳚न्थ् सादयाम्य॒पां त्वोद्म᳚न्थ् सादयाम्य॒पां त्वा॒ भस्म᳚न्थ् सादयाम्य॒पां त्वा॒ ज्योति॑षि सादयाम्य॒पां त्वाऽय॑ने सादयाम्यर्ण॒वे सद॑ने सीद समु॒द्रे सद॑ने सीद सलि॒ले सद॑ने सीदा॒पां क्षये॑ सीदा॒पाꣳ सधि॑षि सीदा॒पां त्वा॒ सद॑ने सादयाम्य॒पां त्वा॑ स॒धस्थे॑ सादयाम्य॒पां त्वा॒ पुरी॑षे सादयाम्य॒पां त्वा॒ योनौ॑ ( ) सादयाम्य॒पां त्वा॒ पाथ॑सि सादयामि गाय॒त्री छन्द॑-स्त्रि॒ष्टुप् छन्दो॒ जग॑ती॒ छन्दो॑ऽनु॒ष्टुप् छन्दः॑ प॒ङ्क्तिश्छन्दः॑ ॥ \newline

\textbf{Pada Paata} \newline

अ॒पाम् । त्वा॒ । एमन्न्॑ । सा॒द॒या॒मि॒ । अ॒पाम् । त्वा॒ । ओद्मन्न्॑ । सा॒द॒या॒मि॒ । अ॒पाम् । त्वा॒ । भस्मन्न्॑ । सा॒द॒या॒मि॒ । अ॒पाम् । त्वा॒ । ज्योति॑षि । सा॒द॒या॒मि॒ । अ॒पाम् । त्वा॒ । अय॑ने । सा॒द॒या॒मि॒ । अ॒र्ण॒वे । सद॑ने । सी॒द॒ । स॒मु॒द्रे । सद॑ने । सी॒द॒ । स॒लि॒ले । सद॑ने । सी॒द॒ । अ॒पाम् । क्षये᳚ । सी॒द॒ । अ॒पाम् । सधि॑षि । सी॒द॒ । अ॒पाम् । त्वा॒ । सद॑ने । सा॒द॒या॒मि॒ । अ॒पाम् । त्वा॒ । स॒धस्थ॒ इति॑ स॒ध - स्थे॒ । सा॒द॒या॒मि॒ । अ॒पाम् । त्वा॒ । पुरी॑षे । सा॒द॒या॒मि॒ । अ॒पाम् । त्वा॒ । योनौ᳚ ( ) । सा॒द॒या॒मि॒ । अ॒पाम् । त्वा॒ । पाथ॑सि । सा॒द॒या॒मि॒ । गा॒य॒त्री । छन्दः॑ । त्रि॒ष्टुप् । छन्दः॑ । जग॑ती । छन्दः॑ । अ॒नु॒ष्टुबित्य॑नु - स्तुप् । छन्दः॑ । प॒ङ्क्तिः । छन्दः॑ ॥  \newline




\markright{ TS 4.3.2.1  \hfill https://www.vedavms.in \hfill}
\addcontentsline{toc}{section}{ TS 4.3.2.1 }
\section*{ TS 4.3.2.1 }

\textbf{TS 4.3.2.1 } \newline
\textbf{Samhita Paata} \newline

अ॒यं पु॒रो भुव॒स्तस्य॑ प्रा॒णो भौ॑वाय॒नो व॑स॒न्तः प्रा॑णाय॒नो गा॑य॒त्री वा॑स॒न्ती गा॑यत्रि॒यै गा॑य॒त्रं गा॑य॒त्रादु॑पाꣳ॒॒ शुरु॑पाꣳ॒॒ शोस्त्रि॒वृत् त्रि॒वृतो॑ रथन्त॒रꣳ र॑थन्त॒राद्-वसि॑ष्ठ॒ ऋषिः॑ प्र॒जाप॑ति गृहीतया॒ त्वया᳚ प्रा॒णं गृ॑ह्णामि प्र॒जाभ्यो॒ऽयं द॑क्षि॒णा वि॒श्वक॑र्मा॒ तस्य॒ मनो॑ वैश्वकर्म॒णं ग्री॒ष्मो मा॑न॒सस्त्रि॒ष्टुग्ग्रै॒ष्मी त्रि॒ष्टुभ॑ ऐ॒डमै॒डा-द॑न्तर्या॒मो᳚ ऽन्तर्या॒मात् प॑ञ्चद॒शः प॑ञ्चद॒शाद्-बृ॒हद्-बृ॑ह॒तो भ॒रद्वा॑ज॒ ऋषिः॑ प्र॒जाप॑ति गृहीतया॒ त्वया॒ मनो॑ - [  ] \newline

\textbf{Pada Paata} \newline

अ॒यम् । पु॒रः । भुवः॑ । तस्य॑ । प्रा॒ण इति॑ प्र - अ॒नः । भौ॒वा॒य॒नः । व॒स॒न्तः । प्रा॒णा॒य॒नः । गा॒य॒त्री । वा॒स॒न्ती । गा॒य॒त्रि॒यै । गा॒य॒त्रम् । गा॒य॒त्रात् । उ॒पाꣳ॒॒शुरित्यु॑प - अꣳ॒॒शुः । उ॒पाꣳ॒॒शोरित्यु॑प-अꣳ॒॒शोः । त्रि॒वृदिति॑ त्रि - वृत् । त्रि॒वृत॒ इति॑ त्रि - वृतः॑ । र॒थ॒न्त॒रमिति॑ रथं - त॒रम् । र॒थ॒न्त॒रादिति॑ रथं - त॒रात् । वसि॑ष्ठः । ऋषिः॑ । प्र॒जाप॑तिगृहीत॒येति॑ प्र॒जाप॑ति - गृ॒ही॒त॒या॒ । त्वया᳚ । प्रा॒णमिति॑ प्र - अ॒नम् । गृ॒ह्णा॒मि॒ । प्र॒जाभ्य॒ इति॑ प्र - जाभ्यः॑ । अ॒यम् । द॒क्षि॒णा । वि॒श्वक॒र्मेति॑ वि॒श्व - क॒र्मा॒ । तस्य॑ । मनः॑ । वै॒श्व॒क॒र्म॒णमिति॑ वैश्व - क॒र्म॒णम् । ग्री॒ष्मः । मा॒न॒सः । त्रि॒ष्टुक् । ग्रै॒ष्मी । त्रि॒ष्टुभः॑ । ऐ॒डम् । ऐ॒डात् । अ॒न्त॒र्या॒म इत्य॑न्तः - या॒मः । अ॒न्त॒र्या॒मादित्य॑न्तः - या॒मात् । प॒ञ्च॒द॒श इति॑ पञ्च - द॒शः । प॒ञ्च॒द॒शादिति॑ पञ्च-द॒शात् । बृ॒हत् । बृ॒ह॒तः । भ॒रद्वा॑जः । ऋषिः॑ । प्र॒जाप॑तिगृहीत॒येति॑ प्र॒जाप॑ति - गृ॒ही॒त॒या॒ । त्वया᳚ । मनः॑ ।  \newline




\markright{ TS 4.3.2.2  \hfill https://www.vedavms.in \hfill}
\addcontentsline{toc}{section}{ TS 4.3.2.2 }
\section*{ TS 4.3.2.2 }

\textbf{TS 4.3.2.2 } \newline
\textbf{Samhita Paata} \newline

गृह्णामि प्र॒जाभ्यो॒ऽयं प॒श्चाद्-वि॒श्वव्य॑चा॒स्तस्य॒ चक्षु॑र्वैश्वव्यच॒सं ॅव॒र्॒.षाणि॑ चाक्षु॒षाणि॒ जग॑ती वा॒र्॒.षी जग॑त्या॒ ऋक्ष॑म॒मृक्ष॑माच्छु॒क्रः शु॒क्राथ् स॑प्तद॒शः स॑प्तद॒शाद्-वै॑रू॒पं ॅवै॑रू॒पाद्-वि॒श्वामि॑त्र॒ ऋषिः॑ प्र॒जाप॑ति गृहीतया॒ त्वया॒ चक्षु॑र्गृह्णामि प्र॒जाभ्य॑ इ॒दमु॑त्त॒राथ् सुव॒स्तस्य॒ श्रोत्रꣳ॑ सौ॒वꣳ श॒रच्छ्रौ॒त्र्य॑नु॒ष्टुप्-छा॑र॒द्य॑नु॒ष्टुभः॑ स्वा॒रꣳ स्वा॒रान्म॒न्थी म॒न्थिन॑ एकविꣳ॒॒श ए॑कविꣳ॒॒शाद् वै॑रा॒जं ॅवै॑रा॒जाज्ज॒मद॑ग्नि॒र्॒. ऋषिः॑ प्र॒जाप॑ति गृहीतया॒ - [  ] \newline

\textbf{Pada Paata} \newline

गृ॒ह्णा॒मि॒ । प्र॒जाभ्य॒ इति॑ प्र - जाभ्यः॑ । अ॒यम् । प॒श्चात् । वि॒श्वव्य॑चा॒ इति॑ वि॒श्व - व्य॒चाः॒ । तस्य॑ । चक्षुः॑ । वै॒श्व॒व्य॒च॒समिति॑ वैश्व - व्य॒च॒सम् । व॒र्॒.षाणि॑ । चा॒क्षु॒षाणि॑ । जग॑ती । वा॒र्॒.षी । जग॑त्याः । ऋक्ष॑मम् । ऋक्ष॑मात् । शु॒क्रः । शु॒क्रात् । स॒प्त॒द॒श इति॑ सप्त - द॒शः । स॒प्त॒द॒शादिति॑ सप्त-द॒शात् । वै॒रू॒पम् । वै॒रू॒पात् । वि॒श्वामि॑त्र॒ इति॑ वि॒श्व - मि॒त्रः॒ । ऋषिः॑ । प्र॒जाप॑तिगृहीत॒येति॑ प्र॒जाप॑ति - गृ॒ही॒त॒या॒ । त्वया᳚ । चक्षुः॑ । गृ॒ह्णा॒मि॒ । प्र॒जाभ्य॒ इति॑ प्र - जाभ्यः॑ । इ॒दम् । उ॒त्त॒रादित्यु॑त् - त॒रात् । सुवः॑ । तस्य॑ । श्रोत्र᳚म् । सौ॒वम् । श॒रत् । श्रौ॒त्री । अ॒नु॒ष्टुबित्य॑नु - स्तुप् । शा॒र॒दी । अ॒नु॒ष्टुभ॒ इत्य॑नु - स्तुभः॑ । स्वा॒रम् । स्वा॒रात् । म॒न्थी । म॒न्थिनः॑ । ए॒क॒विꣳ॒॒श इत्ये॑क - विꣳ॒॒शः । ए॒क॒विꣳ॒॒शादित्ये॑क - विꣳ॒॒शात् । वै॒रा॒जम् । वै॒रा॒जात् । ज॒मद॑ग्निः । ऋषिः॑ । प्र॒जाप॑तिगृहीत॒येति॑ प्र॒जाप॑ति - गृ॒ही॒त॒या॒ ।  \newline




\markright{ TS 4.3.2.3  \hfill https://www.vedavms.in \hfill}
\addcontentsline{toc}{section}{ TS 4.3.2.3 }
\section*{ TS 4.3.2.3 }

\textbf{TS 4.3.2.3 } \newline
\textbf{Samhita Paata} \newline

त्वया॒ श्रोत्रं॑ गृह्णामि प्र॒जाभ्य॑ इ॒यमु॒परि॑ म॒तिस्तस्यै॒ वाङ्मा॒ती हे॑म॒न्तो वा᳚च्याय॒नः प॒ङ्क्तिर्.है॑म॒न्ती प॒क्त्यैं नि॒धन॑वन्नि॒धन॑वत आग्रय॒ण आ᳚ग्रय॒णात् त्रि॑णवत्रयस्त्रिꣳ॒॒शौ त्रि॑णवत्रयस्त्रिꣳ॒॒शाभ्याꣳ॑ शाक्वररैव॒ते शा᳚क्वररैव॒ताभ्यां᳚ ॅवि॒श्वक॒र्मर्.षिः॑ प्र॒जाप॑ति गृहीतया॒ त्वया॒ वाचं॑ गृह्णामि प्र॒जाभ्यः॑ ॥ \newline

\textbf{Pada Paata} \newline

त्वया᳚ । श्रोत्र᳚म् । गृ॒ह्णा॒मि॒ । प्र॒जाभ्य॒ इति॑ प्र-जाभ्यः॑ । इ॒यम् । उ॒परि॑ । म॒तिः । तस्यै᳚ । वाक् । मा॒ती । हे॒म॒न्तः । वा॒च्या॒य॒नः । प॒ङ्क्तिः । है॒म॒न्ती । प॒ङ्क्त्यै । नि॒धन॑व॒दिति॑ नि॒धन॑ - व॒त् । नि॒धन॑वत॒ इति॑ नि॒धन॑ - व॒तः॒ । आ॒ग्र॒य॒णः । आ॒ग्र॒य॒णात् । त्रि॒ण॒व॒त्र॒य॒स्त्रिꣳ॒॒शाविति॑ त्रिणव - त्र॒य॒स्त्रिꣳ॒॒शौ । त्रि॒ण॒व॒त्र॒य॒स्त्रिꣳ॒॒शाभ्या॒मिति॑ त्रिणव - त्र॒य॒स्त्रिꣳ॒॒शाभ्या᳚म् । शा॒क्व॒र॒रै॒व॒ते इति॑ शाक्वर - रै॒व॒ते । शा॒क्व॒र॒रै॒व॒ताभ्या॒मिति॑ शाक्वर - रै॒व॒ताभ्या᳚म् । वि॒श्वक॒र्मेति॑ वि॒श्व - क॒र्मा॒ । ऋषिः॑ । प्र॒जाप॑तिगृहीत॒येति॑ प्र॒जाप॑ति - गृ॒ही॒त॒या॒ । त्वया᳚ । वाच᳚म् । गृ॒ह्णा॒मि॒ । प्र॒जाभ्य॒ इति॑ प्र - जाभ्यः॑ ॥  \newline




\markright{ TS 4.3.3.1  \hfill https://www.vedavms.in \hfill}
\addcontentsline{toc}{section}{ TS 4.3.3.1 }
\section*{ TS 4.3.3.1 }

\textbf{TS 4.3.3.1 } \newline
\textbf{Samhita Paata} \newline

प्राची॑ दि॒शां ॅव॑स॒न्त ऋ॑तू॒नाम॒ग्निर्दे॒वता॒ ब्रह्म॒ द्रवि॑णं त्रि॒वृथ् स्तोमः॒ स उ॑ पञ्चद॒शव॑र्तनि॒-स्त्र्यवि॒र्वयः॑ कृ॒तमया॑नां पुरोवा॒तो वातः॒ सान॑ग॒ ऋषि॑र्दक्षि॒णा दि॒शां ग्री॒ष्म ऋ॑तू॒नामिन्द्रो॑ दे॒वता᳚ क्ष॒त्रं द्रवि॑णं पञ्चद॒शः स्तोमः॒ स उ॑ सप्तद॒श व॑र्तनि-र्दि॑त्य॒वाड्-वय॒स्त्रेताऽया॑नां दक्षिणाद्वा॒तो वातः॑ सना॒तन॒ ऋषिः॑ प्र॒तीची॑ दि॒शां ॅव॒र्॒.षा ऋ॑तू॒नां ॅविश्वे॑ दे॒वा दे॒वता॒ विड्-[  ] \newline

\textbf{Pada Paata} \newline

प्राची᳚ । दि॒शाम् । व॒स॒न्तः । ऋ॒तू॒नाम् । अ॒ग्निः । दे॒वता᳚ । ब्रह्म॑ । द्रवि॑णम् । त्रि॒वृदिति॑ त्रि - वृत् । स्तोमः॑ । सः । उ॒ । प॒ञ्च॒द॒शव॑र्तनि॒रिति॑ पञ्चद॒श - व॒र्त॒निः॒ । त्र्यवि॒रिति॑ त्रि - अविः॑ । वयः॑ । कृ॒तम् । अया॑नाम् । पु॒रो॒वा॒त इति॑ पुरः - वा॒तः । वातः॑ । सान॑गः । ऋषिः॑ । द॒क्षि॒णा । दि॒शाम् । ग्री॒ष्मः । ऋ॒तू॒नाम् । इन्द्रः॑ । दे॒वता᳚ । क्ष॒त्रम् । द्रवि॑णम् । प॒ञ्च॒द॒श इति॑ पञ्च - द॒शः । स्तोमः॑ । सः । उ॒ । स॒प्त॒द॒शव॑र्तनि॒रिति॑ सप्तद॒श - व॒र्त॒निः॒ । दि॒त्य॒वाडिति॑ दित्य - वाट् । वयः॑ । त्रेता᳚ । अया॑नाम् । द॒क्षि॒णा॒द्वा॒त इति॑ दक्षिणात् - वा॒तः । वातः॑ । स॒ना॒तन॒ इति॑ सना - तनः॑ । ऋषिः॑ । प्र॒तीची᳚ । दि॒शाम् । व॒र्॒.षाः । ऋ॒तू॒नाम् । विश्वे᳚ । दे॒वाः । दे॒वता᳚ । विट् ।  \newline




\markright{ TS 4.3.3.2  \hfill https://www.vedavms.in \hfill}
\addcontentsline{toc}{section}{ TS 4.3.3.2 }
\section*{ TS 4.3.3.2 }

\textbf{TS 4.3.3.2 } \newline
\textbf{Samhita Paata} \newline

द्रवि॑णꣳ सप्तद॒श स्तोमः॒ स उ॑ वेकविꣳ॒॒ शव॑र्तनि-स्त्रिव॒थ्सो वयो᳚ द्वाप॒रोऽया॑नां पश्चाद्वा॒तो वातो॑ऽह॒भून॒ ऋषि॒रुदी॑ची दि॒शाꣳ श॒रद्-ऋ॑तू॒नां मि॒त्रावरु॑णौ दे॒वता॑ पु॒ष्टं द्रवि॑णमेकविꣳ॒॒शः स्तोमः॒ स उ॑ त्रिण॒वव॑र्तनि-स्तुर्य॒वाड् वय॑ आस्क॒न्दो ऽया॑नामुत्तराद्-वा॒तो वातः॑ प्र॒त्न ऋषि॑रू॒र्द्ध्वा दि॒शाꣳ हे॑मन्तशिशि॒रावृ॑तू॒नां बृह॒स्पति॑र्दे॒वता॒ वर्चो॒ द्रवि॑णं त्रिण॒व स्तोमः॒ स उ॑ त्रयस्त्रिꣳ॒॒शव॑र्तनिः पष्ठ॒वाद्वयो॑ ( ) ऽभि॒भूरया॑नां ॅविष्वग्वा॒तो वातः॑ सुप॒र्ण ऋषिः॑ पि॒तरः॑ पिताम॒हाः परेऽव॑रे॒ ते नः॑ पान्तु॒ ते नो॑ऽवन्त्व॒स्मिन् ब्रह्म॑न्न॒स्मिन् क्ष॒त्रे᳚ऽस्या-मा॒शिष्य॒स्यां पु॑रो॒धाया॑म॒स्मिन् कर्म॑न्न॒स्यां दे॒वहू᳚त्यां ॥ \newline

\textbf{Pada Paata} \newline

द्रवि॑णम् । स॒प्त॒द॒श इति॑ सप्त-द॒शः । स्तोमः॑ । सः । उ॒ । ए॒क॒विꣳ॒॒शव॑र्तनि॒रित्ये॑कविꣳ॒॒श - व॒र्त॒निः॒ । त्रि॒व॒थ्स इति॑ त्रि-व॒थ्सः । वयः॑ । द्वा॒प॒रः । अया॑नाम् । प॒श्चा॒द्वा॒त इति॑ पश्चात्-वा॒तः । वातः॑ । अ॒ह॒भूनः॑ । ऋषिः॑ । उदी॑ची । दि॒शाम् । श॒रत् । ऋ॒तू॒नाम् । मि॒त्रावरु॑णा॒विति॑ मि॒त्रा - वरु॑णौ । दे॒वता᳚ । पु॒ष्टम् । द्रवि॑णम् । ए॒क॒विꣳ॒॒श इत्ये॑क - विꣳ॒॒शः । स्तोमः॑ । सः । उ॒ । त्रि॒ण॒वव॑र्तनि॒रिति॑ त्रिण॒व - व॒र्त॒निः॒ । तु॒र्य॒वाडिति॑ तुर्य - वाट् । वयः॑ । आ॒स्क॒न्द इत्या᳚ - स्क॒न्दः । अया॑नाम् । उ॒त्त॒रा॒द्वा॒त इत्यु॑त्तरात् - वा॒तः । वातः॑ । प्र॒त्नः । ऋषिः॑ । ऊ॒द्‌र्ध्वा । दि॒शाम् । हे॒म॒न्त॒शि॒शि॒राविति॑ हेमन्त - शि॒शि॒रौ । ऋ॒तू॒नाम् । बृह॒स्पतिः॑ । दे॒वता᳚ । वर्चः॑ । द्रवि॑णम् । त्रि॒ण॒व इति त्रि॑ - न॒वः । स्तोमः॑ । सः । उ॒ । त्र॒य॒स्त्रिꣳ॒॒शव॑र्तनि॒रिति॑ त्रयस्त्रिꣳ॒॒श - व॒र्त॒निः॒ । प॒ष्ठ॒वादिति॑ पष्ठ - वात् । वयः॑ ( ) । अ॒भि॒भूरित्य॑भि - भूः । अया॑नाम् । वि॒ष्व॒ग्वा॒त इति॑ विष्वक् - वा॒तः । वातः॑ । सु॒प॒र्ण इति॑ सु - प॒र्णः । ऋषिः॑ । पि॒तरः॑ । पि॒ता॒म॒हाः । परे᳚ । अव॑रे । ते । नः॒ । पा॒न्तु॒ । ते । नः॒ । अ॒व॒न्तु॒ । अ॒स्मिन्न् । ब्रह्मन्न्॑ । अ॒स्मिन्न् । क्ष॒त्रे । अ॒स्याम् । आ॒शिषीत्या᳚ - शिषि॑ । अ॒स्याम् । पु॒रो॒धाया॒मिति॑ पुरः - धाया᳚म् । अ॒स्मिन्न् । कर्मन्न्॑ । अ॒स्याम् । दे॒वहू᳚त्या॒मिति॑ दे॒व-हू॒त्या॒म् ॥  \newline




\markright{ TS 4.3.4.1  \hfill https://www.vedavms.in \hfill}
\addcontentsline{toc}{section}{ TS 4.3.4.1 }
\section*{ TS 4.3.4.1 }

\textbf{TS 4.3.4.1 } \newline
\textbf{Samhita Paata} \newline

ध्रु॒वक्षि॑ति -र्ध्रु॒वयो॑नि-र्ध्रु॒वाऽसि॑ ध्रु॒वां ॅयोनि॒मा सी॑द सा॒द्ध्या । उख्य॑स्य के॒तुं प्र॑थ॒मं पु॒रस्ता॑द॒श्विना᳚ऽद्ध्व॒र्यू सा॑दयतामि॒ह त्वा᳚ ॥ स्वे दक्षे॒ दक्ष॑पिते॒ह सी॑द देव॒त्रा पृ॑थि॒वी बृ॑ह॒ती ररा॑णा । स्वा॒स॒स्था त॒नुवा॒ सं ॅवि॑शस्व पि॒तेवै॑धि सू॒नव॒ आ सु॒शेवा॒ऽश्विना᳚द्ध्व॒र्यू सा॑दयतामि॒ह त्वा᳚ ॥ कु॒ला॒यिनी॒ वसु॑मती वयो॒धा र॒यिं नो॑ वर्द्ध बहु॒लꣳ सु॒वीरं᳚ । \newline

\textbf{Pada Paata} \newline

ध्रु॒वक्षि॑ती॒रिति॑ ध्रु॒व - क्षि॒तिः॒ । ध्रु॒वयो॑नि॒रिति॑ ध्रु॒व-यो॒निः॒ । ध्रु॒वा । अ॒सि॒ । ध्रु॒वाम् । योनि᳚म् । एति॑ । सी॒द॒ । सा॒द्ध्या ॥ उख्य॑स्य । के॒तुम् । प्र॒थ॒मम् । पु॒रस्ता᳚त् । अ॒श्विना᳚ । अ॒द्ध्व॒र्यू इति॑ । सा॒द॒य॒ता॒म् । इ॒ह । त्वा॒ ॥ स्वे । दक्षे᳚ । दक्ष॑पि॒तेति॒ दक्ष॑ - पि॒ता॒ । इ॒ह । सी॒द॒ । दे॒व॒त्रेति॑ देव - त्रा । पृ॒थि॒वी । बृ॒ह॒ती । ररा॑णा ॥ स्वा॒स॒स्थेति॑ सु - आ॒स॒स्था । त॒नुवा᳚ । समिति॑ । वि॒श॒स्व॒ । पि॒ता । इ॒व॒ । ए॒धि॒ । सू॒नवे᳚ । एति॑ । सु॒शेवेति॑ सु - शेवा᳚ । अ॒श्विना᳚ । अ॒द्ध्व॒र्यू इति॑ । सा॒द॒य॒ता॒म् । इ॒ह । त्वा॒ ॥ कु॒ला॒यिनी᳚ । वसु॑म॒तीति॒ वसु॑ - म॒ती॒ । व॒यो॒धा इति॑ वयः - धाः । र॒यिम् । नः॒ । व॒द्‌र्ध॒ । ब॒हु॒लम् । सु॒वीर॒मिति॑ सु - वीर᳚म् ॥  \newline




\markright{ TS 4.3.4.2  \hfill https://www.vedavms.in \hfill}
\addcontentsline{toc}{section}{ TS 4.3.4.2 }
\section*{ TS 4.3.4.2 }

\textbf{TS 4.3.4.2 } \newline
\textbf{Samhita Paata} \newline

अपाम॑तिं दुर्म॒तिं बाध॑माना रा॒यस्पोषे॑ य॒ज्ञ्प॑तिमा॒भज॑न्ती॒ सुव॑र्द्धेहि॒ यज॑मानाय॒ पोष॑म॒श्विना᳚ऽद्ध्व॒र्यू सा॑दयतामि॒ह त्वा᳚ ॥ अ॒ग्नेः पुरी॑षमसि देव॒यानी॒ तां त्वा॒ विश्वे॑ अ॒भि गृ॑णन्तु दे॒वाः । स्तोम॑पृष्ठा घृ॒तव॑ती॒ह सी॑द प्र॒जाव॑द॒स्मे द्रवि॒णा ऽऽ*य॑जस्वा॒श्विना᳚ ऽद्ध्व॒र्यू सा॑दयतामि॒ह त्वा᳚ ॥ दि॒वो मू॒र्द्धाऽसि॑ पृथि॒व्या नाभि॑र्वि॒ष्टंभ॑नी दि॒शामधि॑पत्नी॒ भुव॑नानां । \newline

\textbf{Pada Paata} \newline

अपेति॑ । अम॑तिम् । दु॒र्म॒तिमिति॑ दुः - म॒तिम् । बाध॑माना । रा॒यः । पोषे᳚ । य॒ज्ञ्प॑ति॒मिति॑ य॒ज्ञ् - प॒ति॒म् । आ॒भज॒न्तीत्या᳚ - भज॑न्ती । सुवः॑ । धे॒हि॒ । यज॑मानाय । पोष᳚म् । अ॒श्विना᳚ । अ॒द्ध्व॒र्यू इति॑ । सा॒द॒य॒ता॒म् । इ॒ह । त्वा॒ ॥ अ॒ग्नेः । पुरी॑षम् । अ॒सि॒ । दे॒व॒यानीति॑ देव - यानी᳚ । ताम् । त्वा॒ । विश्वे᳚ । अ॒भीति॑ । गृ॒ण॒न्तु॒ । दे॒वाः ॥ स्तोम॑पृ॒ष्ठेति॒ स्तोम॑ - पृ॒ष्ठा॒ । घृ॒तव॒तीति॑ घृ॒त - व॒ती॒ । इ॒ह । सी॒द॒ । प्र॒जाव॒दिति॑ प्र॒जा - व॒त् । अ॒स्मे इति॑ । द्रवि॑णा । एति॑ । य॒ज॒स्व॒ । आ॒श्विना᳚ । अ॒द्ध्व॒र्यू इति॑ । सा॒द॒य॒ता॒म् । इ॒ह । त्वा॒ ॥ दि॒वः । मू॒द्‌र्धा । अ॒सि॒ । पृ॒थि॒व्याः । नाभिः॑ । वि॒ष्टंभ॒नीति॑ वि - स्तंभ॑नी । दि॒शाम् । अधि॑प॒त्नीत्यधि॑ - प॒त्नी॒ । भुव॑नानाम् ॥  \newline




\markright{ TS 4.3.4.3  \hfill https://www.vedavms.in \hfill}
\addcontentsline{toc}{section}{ TS 4.3.4.3 }
\section*{ TS 4.3.4.3 }

\textbf{TS 4.3.4.3 } \newline
\textbf{Samhita Paata} \newline

ऊ॒र्मिर्द्र॒फ्सो अ॒पाम॑सि वि॒श्वक॑र्मा त॒ ऋषि॑र॒श्विना᳚ऽद्ध्व॒र्यू सा॑दयतामि॒ह त्वा᳚ ॥ स॒जूर्.ऋ॒तुभिः॑ स॒जूर्वि॒धाभिः॑ स॒जूर्वसु॑भिः स॒जू रु॒द्रैः स॒जूरा॑दि॒त्यैः स॒जूर्विश्वै᳚र्दे॒वैः स॒जूर्दे॒वैः स॒जूर्दे॒वैर्व॑यो-ना॒धैर॒ग्नये᳚ त्वा वैश्वान॒राया॒श्विना᳚ऽद्ध्व॒र्यू सा॑दयतामि॒ह त्वा᳚ ॥ प्रा॒णं मे॑ पाह्यपा॒नं मे॑ पाहि व्या॒नं मे॑ पाहि॒ चक्षु॑र्म उ॒र्व्या ( ) वि भा॑हि॒ श्रोत्रं॑ मे श्लोकया॒प-स्पि॒न्वौष॑धीर्जिन्व द्वि॒पात् पा॑हि॒ चतु॑ष्पादव दि॒वो वृष्टि॒मेर॑य ॥ \newline

\textbf{Pada Paata} \newline

ऊ॒र्मिः । द्र॒फ्सः । अ॒पाम् । अ॒सि॒ । वि॒श्वक॒र्मेति॑ वि॒श्व - क॒र्मा॒ । ते॒ । ऋषिः॑ । अ॒श्विना᳚ । अ॒द्ध्व॒र्यू इति॑ । सा॒द॒य॒ता॒म् । इ॒ह । त्वा॒ ॥ स॒जूरिति॑ स - जूः । ऋ॒तुभि॒रित्यृ॒तु - भिः॒ । स॒जूरिति॑ स - जूः । वि॒धाभि॒रिति॑ वि-धाभिः॑ । स॒जूरिति॑ स-जूः । वसु॑भि॒रिति॒ वसु॑-भिः॒ । स॒जूरिति॑ स - जूः । रु॒द्रैः । स॒जूरिति॑ स - जूः । आ॒दि॒त्यैः । स॒जूरिति॑ स - जूः । विश्वैः᳚ । दे॒वैः । स॒जूरिति॑ स - जूः । दे॒वैः । स॒जूरिति॑ स - जूः । दे॒वैः । व॒यो॒ना॒धैरिति॑ वयः - ना॒धैः । अ॒ग्नये᳚ । त्वा॒ । वै॒श्वा॒न॒राय॑ । अ॒श्विना᳚ । अ॒द्ध्व॒र्यू इति॑ । सा॒द॒य॒ता॒म् । इ॒ह । त्वा॒ ॥ प्रा॒णमिति॑ प्र - अ॒नम् । मे॒ । पा॒हि॒ । अ॒पा॒नमित्य॑प - अ॒नम् । मे॒ । पा॒हि॒ । व्या॒नमिति॑ वि-अ॒नम् । मे॒ । पा॒हि॒ । चक्षुः॑ । मे॒ । उ॒र्व्या ( ) । वीति॑ । भा॒हि॒ । श्रोत्र᳚म् । मे॒ । श्लो॒क॒य॒ । अ॒पः । पि॒न्व॒ । ओष॑धीः । जि॒न्व॒ । द्वि॒पादिति॑ द्वि - पात् । पा॒हि॒ । चतु॑ष्पा॒दिति॒ चतुः॑ - पा॒त् । अ॒व॒ । दि॒वः । वृष्टि᳚म् । एति॑ । ई॒र॒य॒ ॥  \newline




\markright{ TS 4.3.5.1  \hfill https://www.vedavms.in \hfill}
\addcontentsline{toc}{section}{ TS 4.3.5.1 }
\section*{ TS 4.3.5.1 }

\textbf{TS 4.3.5.1 } \newline
\textbf{Samhita Paata} \newline

त्र्यवि॒र्वय॑स्त्रि॒ष्टुप् छन्दो॑ दित्य॒वाड् वयो॑ वि॒राट् छन्दः॒ पञ्चा॑वि॒र्वयो॑ गाय॒त्री छन्द॑स्त्रिव॒थ्सो वय॑ उ॒ष्णिहा॒ छन्द॑ स्तुर्य॒वाड् वयो॑ऽनु॒ष्टुप् छन्दः॑ पष्ठ॒वाद् वयो॑ बृह॒ती छन्द॑ उ॒क्षा वयः॑ स॒तोबृ॑हती॒ छन्द॑ ऋष॒भो वयः॑ क॒कुच्छन्दो॑ धे॒नुर्वयो॒ जग॑ती॒ छन्दो॑ऽन॒ड्वान्. वयः॑ प॒ङ्क्ति श्छन्दो॑ ब॒स्तो वयो॑ विव॒लं छन्दो॑ वृ॒ष्णिर्वयो॑ विशा॒लं छन्दः॒ पुरु॑षो॒ वय॑ ( ) स्त॒न्द्रं छन्दो᳚ व्या॒घ्रो वयोऽना॑धृष्टं॒ छन्दः॑ सिꣳ॒॒हो वय॑ श्छ॒दि श्छन्दो॑ विष्ट॒भ्ॐ ॅवयोऽधि॑पति॒ श्छन्दः॑ क्ष॒त्रं ॅवयो॒ मय॑न्दं॒ छन्दो॑ वि॒श्वक॑र्मा॒ वयः॑ परमे॒ष्ठी छन्दो॑ मू॒र्द्धा वयः॑ प्र॒जाप॑ति॒ श्छन्दः॑ ॥ \newline

\textbf{Pada Paata} \newline

त्र्यवि॒रिति॑ त्रि - अविः॑ । वयः॑ । त्रि॒ष्टुप् । छन्दः॑ । दि॒त्य॒वाडिति॑ दित्य - वाट् । वयः॑ । वि॒राडिति॑ वि - राट् । छन्दः॑ । पञ्चा॑वि॒रिति॒ पञ्च॑-अ॒विः॒ । वयः॑ । गा॒य॒त्री । छन्दः॑ । त्रि॒व॒थ्स इति॑ त्रि - व॒थ्सः । वयः॑ । उ॒ष्णिहा᳚ । छन्दः॑ । तु॒र्य॒वाडिति॑ तुर्य - वाट् । वयः॑ । अ॒नु॒ष्टुबित्य॑नु - स्तुप् । छन्दः॑ । प॒ष्ठ॒वादिति॑ पष्ठ - वात् । वयः॑ । बृ॒ह॒ती । छन्दः॑ । उ॒क्षा । वयः॑ । स॒तोबृ॑ह॒तीति॑ स॒तः - बृ॒ह॒ती॒ । छन्दः॑ । ऋ॒ष॒भः । वयः॑ । क॒कुत् । छन्दः॑ । धे॒नुः । वयः॑ । जग॑ती । छन्दः॑ । अ॒न॒ड्वान् । वयः॑ । प॒ङ्क्तिः । छन्दः॑ । ब॒स्तः । वयः॑ । वि॒व॒लमिति॑ वि - व॒लम् । छन्दः॑ । वृ॒ष्णिः । वयः॑ । वि॒शा॒लमिति॑ वि-शा॒लम् । छन्दः॑ । पुरु॑षः । वयः॑ ( ) । त॒न्द्रम् । छन्दः॑ । व्या॒घ्रः । वयः॑ । अना॑धृष्ट॒मित्यना᳚ - धृ॒ष्ट॒म् । छन्दः॑ । सिꣳ॒॒हः । वयः॑ । छ॒दिः । छन्दः॑ । वि॒ष्ट॒भं इति॑ वि - स्त॒भंः । वयः॑ । अधि॑पति॒रित्यधि॑-प॒तिः॒ । छन्दः॑ । क्ष॒त्रम् । वयः॑ । मय॑न्दम् । छन्दः॑ । वि॒श्वक॒र्मेति॑ वि॒श्व - क॒र्मा॒ । वयः॑ । प॒र॒मे॒ष्ठी । छन्दः॑ । मू॒द्‌र्धा । वयः॑ । प्र॒जाप॑ति॒रिति॑ प्र॒जा - प॒तिः॒ । छन्दः॑ ॥  \newline




\markright{ TS 4.3.6.1  \hfill https://www.vedavms.in \hfill}
\addcontentsline{toc}{section}{ TS 4.3.6.1 }
\section*{ TS 4.3.6.1 }

\textbf{TS 4.3.6.1 } \newline
\textbf{Samhita Paata} \newline

इन्द्रा᳚ग्नी॒ अव्य॑थमाना॒मिष्ट॑कां दृꣳहतं ॅयु॒वं । पृ॒ष्ठेन॒ द्यावा॑पृथि॒वी अ॒न्तरि॑क्षं च॒ वि बा॑धतां ॥ वि॒श्वक॑र्मा त्वा सादयत्व॒न्तरि॑क्षस्य पृ॒ष्ठे व्यच॑स्वतीं॒ प्रथ॑स्वतीं॒ भास्व॑तीꣳ सूरि॒मती॒मा या द्यां भास्या पृ॑थि॒वीमोर्व॑न्तरि॑क्ष-म॒न्तरि॑क्षं ॅयच्छा॒न्तरि॑क्षं दृꣳहा॒न्तरि॑क्षं॒ मा हिꣳ॑सी॒ र्विश्व॑स्मै प्रा॒णाया॑पा॒नाय॑ व्या॒नायो॑दा॒नाय॑ प्रति॒ष्ठायै॑ च॒रित्रा॑य वा॒युस्त्वा॒ऽभि पा॑तु म॒ह्या स्व॒स्त्या छ॒र्दिषा॒ - [  ] \newline

\textbf{Pada Paata} \newline

इन्द्रा᳚ग्नी॒ इतीन्द्र॑ - अ॒ग्नी॒ । अव्य॑थमानाम् । इष्ट॑काम् । दृꣳ॒॒ह॒त॒म् । यु॒वम् ॥ पृ॒ष्ठेन॑ । द्यावा॑पृथि॒वी इति॒ द्यावा᳚ -पृ॒थि॒वी । अ॒न्तरि॑क्षम् । च॒ । वीति॑ । बा॒ध॒ता॒म् ॥ वि॒श्वक॒र्मेति॑ वि॒श्व - क॒र्मा॒ । त्वा॒ । सा॒द॒य॒तु॒ । अ॒न्तरि॑क्षस्य । पृ॒ष्ठे । व्यच॑स्वतीम् । प्रथ॑स्वतीम् । भास्व॑तीम् । सू॒रि॒मती॒मिति॑ सूरि - मती᳚म् । एति॑ । या । द्याम् । भासि॑ । एति॑ । पृ॒थि॒वीम् । एति॑ । उ॒रु । अ॒न्तरि॑क्षम् । अ॒न्तरि॑क्षम् । य॒च्छ॒ । अ॒न्तरि॑क्षम् । दृꣳ॒॒ह॒ । अ॒न्तरि॑क्षम् । मा । हिꣳ॒॒सीः॒ । विश्व॑स्मै । प्रा॒णायेति॑ प्रा - अ॒नाय॑ । अ॒पा॒नायेत्य॑प - अ॒नाय॑ । व्या॒नायेति॑ वि - अ॒नाय॑ । उ॒दा॒नायेत्यु॑त् - अ॒नाय॑ । प्र॒ति॒ष्ठाया॒ इति॑ प्रति - स्थायै᳚ । च॒रित्रा॑य । वा॒युः । त्वा॒ । अ॒भीति॑ । पा॒तु॒ । म॒ह्या । स्व॒स्त्या । छ॒र्दिषा᳚ ।  \newline




\markright{ TS 4.3.6.2  \hfill https://www.vedavms.in \hfill}
\addcontentsline{toc}{section}{ TS 4.3.6.2 }
\section*{ TS 4.3.6.2 }

\textbf{TS 4.3.6.2 } \newline
\textbf{Samhita Paata} \newline

शन्त॑मेन॒ तया॑ दे॒व॑तयाऽङ्गिर॒स्वद्-ध्रु॒वा सी॑द ॥ राज्ञ्य॑सि॒ प्राची॒ दिग्-वि॒राड॑सि दक्षि॒णा दिख् स॒म्राड॑सि प्र॒तीची॒ दिख्-स्व॒राड॒स्युदी॑ची॒ दिगधि॑पत्न्यसि बृह॒ती दिगायु॑र्मे पाहि प्रा॒णं मे॑ पाह्यपा॒नं मे॑ पाहि व्या॒नं मे॑ पाहि॒ चक्षु॑र्मे पाहि॒ श्रोत्रं॑ मे पाहि॒ मनो॑ मे जिन्व॒ वाचं॑ मे पिन्वा॒ ( ) ऽऽत्मानं॑ मे पाहि॒ ज्योति॑र्मे यच्छ ॥ \newline

\textbf{Pada Paata} \newline

शन्त॑मे॒नेति॒ शं - त॒मे॒न॒ । तया᳚ । दे॒वत॑या । अ॒ङ्गि॒र॒स्वत् । ध्रु॒वा । सी॒द॒ ॥ राज्ञी᳚ । अ॒सि॒ । प्राची᳚ । दिक् । वि॒राडिति॑ वि-राट् । अ॒सि॒ । द॒क्षि॒णा । दिक् । स॒म्राडिति॑ सं - राट् । अ॒सि॒ । प्र॒तीची᳚ । दिक् । स्व॒राडिति॑ स्व - राट् । अ॒सि॒ । उदी॑ची । दिक् । अधि॑प॒त्नीत्यधि॑-प॒त्नी॒ । अ॒सि॒ । बृ॒ह॒ती । दिक् । आयुः॑ । मे॒ । पा॒हि॒ । प्रा॒णमिति॑ प्र - अ॒नम् । मे॒ । पा॒हि॒ । अ॒पा॒नमित्य॑प - अ॒नम् । मे॒ । पा॒हि॒ । व्या॒नमिति॑ वि - अ॒नम् । मे॒ । पा॒हि॒ । चक्षुः॑ । मे॒ । पा॒हि॒ । श्रोत्र᳚म् । मे॒ । पा॒हि॒ । मनः॑ । मे॒ । जि॒न्व॒ । वाच᳚म् । मे॒ । पि॒न्व॒ ( ) । आ॒त्मान᳚म् । मे॒ । पा॒हि॒ । ज्योतिः॑ । मे॒ । य॒च्छ॒ ॥  \newline




\markright{ TS 4.3.7.1  \hfill https://www.vedavms.in \hfill}
\addcontentsline{toc}{section}{ TS 4.3.7.1 }
\section*{ TS 4.3.7.1 }

\textbf{TS 4.3.7.1 } \newline
\textbf{Samhita Paata} \newline

मा छन्दः॑ प्र॒मा छन्दः॑ प्रति॒मा छन्दो᳚ऽस्री॒वि श्छन्दः॑ प॒ङ्क्ति श्छन्द॑ उ॒ष्णिहा॒ छन्दो॑ बृह॒ती छन्दो॑ऽनु॒ष्टुप् छन्दो॑ वि॒राट् छन्दो॑ गाय॒त्री छन्द॑-स्त्रि॒ष्टुप् छन्दो॒ जग॑ती॒ छन्दः॑ पृथि॒वी छन्दो॒ ऽन्तरि॑क्षं॒ छन्दो॒ द्यौ श्छन्दः॒ समा॒ श्छन्दो॒ नक्ष॑त्राणि॒ छन्दो॒ मन॒ श्छन्दो॒ वाक् छन्दः॑ कृ॒षि श्छन्दो॒ हिर॑ण्यं॒ छन्दो॒ गौ श्छन्दो॒ ऽजा छन्दो ऽश्व॒ श्छन्दः॑ ॥ अ॒ग्निर्दे॒वता॒ - [  ] \newline

\textbf{Pada Paata} \newline

मा । छन्दः॑ । प्र॒मेति॑ प्र - मा । छन्दः॑ । प्र॒ति॒मेति॑ प्रति-मा । छन्दः॑ । अ॒स्री॒विः । छन्दः॑ । प॒ङ्क्तिः । छन्दः॑ । उ॒ष्णिहा᳚ । छन्दः॑ । बृ॒ह॒ती । छन्दः॑ । अ॒नु॒ष्टुबित्य॑नु - स्तुप् । छन्दः॑ । वि॒राडिति॑ वि - राट् । छन्दः॑ । गा॒य॒त्री । छन्दः॑ । त्रि॒ष्टुप् । छन्दः॑ । जग॑ती । छन्दः॑ । पृ॒थि॒वी । छन्दः॑ । अ॒न्तरि॑क्षम् । छन्दः॑ । द्यौः । छन्दः॑ । समाः᳚ । छन्दः॑ । नक्ष॑त्राणि । छन्दः॑ । मनः॑ । छन्दः॑ । वाक् । छन्दः॑ । कृ॒षिः । छन्दः॑ । हिर॑ण्यम् । छन्दः॑ । गौः । छन्दः॑ । अ॒जा । छन्दः॑ । अश्वः॑ । छन्दः॑ ॥ अ॒ग्निः । दे॒वता᳚ ।  \newline




\markright{ TS 4.3.7.2  \hfill https://www.vedavms.in \hfill}
\addcontentsline{toc}{section}{ TS 4.3.7.2 }
\section*{ TS 4.3.7.2 }

\textbf{TS 4.3.7.2 } \newline
\textbf{Samhita Paata} \newline

वातो॑ दे॒वता॒ सूर्यो॑ दे॒वता॑ च॒न्द्रमा॑ दे॒वता॒ वस॑वो दे॒वता॑ रु॒द्रा दे॒वता॑ ऽऽदि॒त्या दे॒वता॒ विश्वे॑ दे॒वा दे॒वता॑ म॒रुतो॑ दे॒वता॒ बृह॒स्पति॑ र्दे॒वतेन्द्रो॑ दे॒वता॒ वरु॑णो दे॒वता॑ मू॒र्द्धाऽसि॒ राड् ध्रु॒वाऽसि॑ ध॒रुणा॑ य॒न्त्र्य॑सि॒ यमि॑त्री॒षे त्वो॒र्जे त्वा॑ कृ॒ष्यै त्वा॒ क्षेमा॑य त्वा॒ यन्त्री॒ राड् ध्रु॒वाऽसि॒ धर॑णी ध॒र्त्र्य॑सि॒ धरि॒त्र्यायु॑षे त्वा॒ ( ) वर्च॑से॒ त्वौज॑से त्वा॒ बला॑य त्वा ॥ \newline

\textbf{Pada Paata} \newline

वातः॑ । दे॒वता᳚ । सूर्यः॑ । दे॒वता᳚ । च॒न्द्रमाः᳚ । दे॒वता᳚ । वस॑वः । दे॒वता᳚ । रु॒द्राः । दे॒वता᳚ । आ॒दि॒त्याः । दे॒वता᳚ । विश्वे᳚ । दे॒वाः । दे॒वता᳚ । म॒रुतः॑ । दे॒वता᳚ । बृह॒स्पतिः॑ । दे॒वता᳚ । इन्द्रः॑ । दे॒वता᳚ । वरु॑णः । दे॒वता᳚ । मू॒द्‌र्धा । अ॒सि॒ । राट् । ध्रु॒वा । अ॒सि॒ । ध॒रुणा᳚ । य॒न्त्री । अ॒सि॒ । यमि॑त्री । इ॒षे । त्वा॒ । ऊ॒र्जे । त्वा॒ । कृ॒ष्यै । त्वा॒ । क्षेमा॑य । त्वा॒ । यन्त्री᳚ । राट् । ध्रु॒वा । अ॒सि॒ । धर॑णी । ध॒र्त्री । अ॒सि॒ । धरि॑त्री । आयु॑षे । त्वा॒ ( ) । वर्च॑से । त्वा॒ । ओज॑से । त्वा॒ । बला॑य । त्वा॒ ॥ 14  \newline




\markright{ TS 4.3.8.1  \hfill https://www.vedavms.in \hfill}
\addcontentsline{toc}{section}{ TS 4.3.8.1 }
\section*{ TS 4.3.8.1 }

\textbf{TS 4.3.8.1 } \newline
\textbf{Samhita Paata} \newline

आ॒शुस्त्रि॒वृद्-भा॒न्तः प॑ञ्चद॒शो व्यो॑म सप्तद॒शः प्रतू᳚र्तिरष्टाद॒श स्तपो॑ नवद॒शो॑ ऽभिव॒र्तः स॑विꣳ॒॒शो ध॒रुण॑ एकविꣳ॒॒शो वर्चो᳚ द्वाविꣳ॒॒शः सं॒भर॑णस्त्रयोविꣳ॒॒शो योनि॑श्चतुर्विꣳ॒॒शो गर्भाः᳚ पञ्चविꣳ॒॒श ओज॑स्त्रिण॒वः क्रतु॑रेकत्रिꣳ॒॒शः प्र॑ति॒ष्ठा त्र॑यस्त्रिꣳ॒॒शो ब्र॒द्ध्नस्य॑ वि॒ष्टपं॑ चतुस्त्रिꣳ॒॒शो नाकः॑ षट्त्रिꣳ॒॒शो वि॑व॒र्तो᳚ऽष्टाचत्वारिꣳ॒॒शो ध॒र्त्रश्च॑तुष्टो॒मः ॥ \newline

\textbf{Pada Paata} \newline

आ॒शुः । त्रि॒वृदिति॑ त्रि-वृत् । भा॒न्तः । प॒ञ्च॒द॒श इति॑ पञ्च-द॒शः । व्यो॑मेति॒ वि - ओ॒म॒ । स॒प्त॒द॒श इति॑ सप्त - द॒शः । प्रतू᳚र्ति॒रिति॒ प्र-तू॒र्तिः॒ । अ॒ष्टा॒द॒श इत्य॑ष्टा-द॒शः । तपः॑ । न॒व॒द॒श इति॑ नव-द॒शः । अ॒भि॒व॒र्त इत्य॑भि - व॒र्तः । स॒विꣳ॒॒श इति॑ स - विꣳ॒॒शः । ध॒रुणः॑ । ए॒क॒विꣳ॒॒श इत्ये॑क - विꣳ॒॒शः । वर्चः॑ । द्वा॒विꣳ॒॒शः । स॒भंर॑ण॒ इति॑ सं - भर॑णः । त्र॒यो॒विꣳ॒॒श इति॑ त्रयः - विꣳ॒॒शः । योनिः॑ । च॒तु॒र्विꣳ॒॒श इति॑ चतुः - विꣳ॒॒शः । गर्भाः᳚ । प॒ञ्च॒विꣳ॒॒श इति॑ पञ्च - विꣳ॒॒शः । ओजः॑ । त्रि॒ण॒व इति॑ त्रि - न॒वः । क्रतुः॑ । ए॒क॒त्रिꣳ॒॒श इत्ये॑क - त्रिꣳ॒॒शः । प्र॒ति॒ष्ठेति॑ प्रति - स्था । त्र॒य॒स्त्रिꣳ॒॒श इति॑ त्रयः - त्रिꣳ॒॒शः । ब्र॒द्ध्नस्य॑ । वि॒ष्टप᳚म् । च॒तु॒स्त्रिꣳ॒॒श इति॑ चतुः - त्रिꣳ॒॒शः । नाकः॑ । ष॒ट्त्रिꣳ॒॒श इति॑ षट् - त्रिꣳ॒॒शः । वि॒व॒र्त इति॑ वि - व॒र्तः । अ॒ष्टा॒च॒त्वा॒रिꣳ॒॒श इत्य॑ष्टा-च॒त्वा॒रिꣳ॒॒शः । ध॒र्त्रः । च॒तु॒ष्टो॒म इति॑ चतुः-स्तो॒मः ॥  \newline




\markright{ TS 4.3.9.1  \hfill https://www.vedavms.in \hfill}
\addcontentsline{toc}{section}{ TS 4.3.9.1 }
\section*{ TS 4.3.9.1 }

\textbf{TS 4.3.9.1 } \newline
\textbf{Samhita Paata} \newline

अ॒ग्नेर्भा॒गो॑ऽसि दी॒क्षाया॒ आधि॑पत्यं॒ ब्रह्म॑ स्पृ॒तं त्रि॒वृथ् स्तोम॒ इन्द्र॑स्य भा॒गो॑ऽसि॒ विष्णो॒राधि॑पत्यं क्ष॒त्रꣳ स्पृ॒तं प॑ञ्चद॒शः स्तोमो॑ नृ॒चक्ष॑सां भा॒गो॑ऽसि धा॒तुराधि॑पत्यं ज॒नित्रꣳ॑ स्पृ॒तꣳ स॑प्तद॒शः स्तोमो॑ मि॒त्रस्य॑ भा॒गो॑ऽसि॒ वरु॑ण॒स्याऽऽधि॑पत्यं दि॒वो वृ॒ष्टिर्वाताः᳚ स्पृ॒ता ए॑कविꣳ॒॒शः स्तोमोऽदि॑त्यै भा॒गो॑ऽसि पू॒ष्ण आधि॑पत्य॒मोजः॑ स्पृ॒तं त्रि॑ण॒वः स्तोमो॒ वसू॑नां भा॒गो॑ऽसि- [  ] \newline

\textbf{Pada Paata} \newline

अ॒ग्नेः । भा॒गः । अ॒सि॒ । दी॒क्षायाः᳚ । आधि॑पत्य॒मित्याधि॑ - प॒त्य॒म् । ब्रह्म॑ । स्पृ॒तम् । त्रि॒वृदिति॑ त्रि - वृत् । स्तोमः॑ । इन्द्र॑स्य । भा॒गः । अ॒सि॒ । विष्णोः᳚ । आधि॑पत्य॒मित्याधि॑ - प॒त्य॒म् । क्ष॒त्रम् । स्पृ॒तम् । प॒ञ्च॒द॒श इति॑ पञ्च - द॒शः । स्तोमः॑ । नृ॒चक्ष॑सा॒मिति॑ नृ - चक्ष॑साम् । भा॒गः । अ॒सि॒ । धा॒तुः । आधि॑पत्य॒मित्याधि॑ - प॒त्य॒म् । ज॒नित्र᳚म् । स्पृ॒तम् । स॒प्त॒द॒श इति॑ सप्त - द॒शः । स्तोमः॑ । मि॒त्रस्य॑ । भा॒गः । अ॒सि॒ । वरु॑णस्य । आधि॑पत्य॒मित्याधि॑ - प॒त्य॒म् । दि॒वः । वृ॒ष्टिः । वाताः᳚ । स्पृ॒ताः । ए॒क॒विꣳ॒॒श इत्ये॑क - विꣳ॒॒शः । स्तोमः॑ । अदि॑त्यै । भा॒गः । अ॒सि॒ । पू॒ष्णः । आधि॑पत्य॒मित्याधि॑ - प॒त्य॒म् । ओजः॑ । स्पृ॒तम् । त्रि॒ण॒व इति॑ त्रि - न॒वः । स्तोमः॑ । वसू॑नाम् । भा॒गः । अ॒सि॒ ।  \newline




\markright{ TS 4.3.9.2  \hfill https://www.vedavms.in \hfill}
\addcontentsline{toc}{section}{ TS 4.3.9.2 }
\section*{ TS 4.3.9.2 }

\textbf{TS 4.3.9.2 } \newline
\textbf{Samhita Paata} \newline

रु॒द्राणा॒माधि॑पत्यं॒ चतु॑ष्पाथ् स्पृ॒तं च॑तुर्विꣳ॒॒शः स्तोम॑ आदि॒त्यानां᳚ भा॒गो॑ऽसि म॒रुता॒माधि॑पत्यं॒ गर्भाः᳚ स्पृ॒ताः प॑ञ्चविꣳ॒॒शः स्तोमो॑ दे॒वस्य॑ सवि॒तुर्भा॒गो॑ऽसि॒ बृह॒स्पते॒राधि॑पत्यꣳ स॒मीची॒र्दिशः॑ स्पृ॒ताश्च॑तुष्टो॒मः स्तोमो॒ यावा॑नां भा॒गो᳚ऽस्यया॑वाना॒माधि॑पत्यं प्र॒जाः स्पृ॒ता-श्च॑तु-श्चत्वारिꣳ॒॒शः स्तोम॑ ऋभू॒णां भा॒गो॑ऽसि॒ विश्वे॑षां दे॒वाना॒माधि॑पत्यं भू॒तं निशा᳚न्तꣳ स्पृ॒तं त्र॑यस्त्रिꣳ॒॒शः स्तोमः॑ ॥ \newline

\textbf{Pada Paata} \newline

रु॒द्राणा᳚म् । आधि॑पत्य॒मित्याधि॑ - प॒त्य॒म् । चतु॑ष्पा॒दिति॒ चतुः॑-पा॒त् । स्पृ॒तम् । च॒तु॒र्विꣳ॒॒श इति॑ चतुः-विꣳ॒॒शः । स्तोमः॑ । आ॒दि॒त्याना᳚म् । भा॒गः । अ॒सि॒ । म॒रुता᳚म् । आधि॑पत्य॒मित्याधि॑ - प॒त्य॒म् । गर्भाः᳚ । स्पृ॒ताः । प॒ञ्च॒विꣳ॒॒श इति॑ पञ्च - विꣳ॒॒शः । स्तोमः॑ । दे॒वस्य॑ । स॒वि॒तुः । भा॒गः । अ॒सि॒ । बृह॒स्पतेः᳚ । आधि॑पत्य॒मित्याधि॑ - प॒त्य॒म् । स॒मीचीः᳚ । दिशः॑ । स्पृ॒ताः । च॒तु॒ष्टो॒म इति॑ चतुः-स्तो॒मः । स्तोमः॑ । यावा॑नाम् । भा॒गः । अ॒सि॒ । अया॑वानाम् । आधि॑पत्य॒मित्याधि॑ - प॒त्य॒म् । प्र॒जा इति॑ प्र - जाः । स्पृ॒ताः । च॒तु॒श्च॒त्वा॒रिꣳ॒॒श इति॑ चतुः - च॒त्वा॒रिꣳ॒॒शः । स्तोमः॑ । ऋ॒भू॒णाम् । भा॒गः । अ॒सि॒ । विश्वे॑षाम् । दे॒वाना᳚म् । आधि॑पत्य॒मित्याधि॑-प॒त्य॒म् । भू॒तम् । निशा᳚न्त॒मिति॒ नि - शा॒न्त॒म् । स्पृ॒तम् । त्र॒य॒स्त्रिꣳ॒॒श इति॑ त्रयः - त्रिꣳ॒॒शः । स्तोमः॑ ॥  \newline




\markright{ TS 4.3.10.1  \hfill https://www.vedavms.in \hfill}
\addcontentsline{toc}{section}{ TS 4.3.10.1 }
\section*{ TS 4.3.10.1 }

\textbf{TS 4.3.10.1 } \newline
\textbf{Samhita Paata} \newline

एक॑याऽस्तुवत प्र॒जा अ॑धीयन्त प्र॒जाप॑ति॒रधि॑पतिरासीत् ति॒सृभि॑रस्तुवत॒ ब्रह्मा॑सृज्यत॒ ब्रह्म॑ण॒स्पति॒-रधि॑पतिरासीत् प॒ञ्चभि॑रस्तुवत भू॒तान्य॑सृज्यन्त भू॒तानां॒ पति॒रधि॑पतिरासीथ् स॒प्तभि॑रस्तुवत सप्त॒र्॒.षयो॑ऽसृज्यन्त धा॒ता-धि॑पतिरासी-न्न॒वभि॑रस्तुवत पि॒तरो॑-ऽसृज्य॒न्ता-ऽ*दि॑ति॒रधि॑पत्न्यासी-देकाद॒शभि॑-रस्तुवत॒र्तवो॑ऽ सृज्यन्ता- ऽऽर्त॒वो-ऽधि॑पति-रासीत् त्रयोद॒शभि॑-रस्तुवत॒ मासा॑ असृज्यन्त संॅवथ्स॒रोऽधि॑पति- [  ] \newline

\textbf{Pada Paata} \newline

एक॑या । अ॒स्तु॒व॒त॒ । प्र॒जा इति॑ प्र - जाः । अ॒धी॒य॒न्त॒ । प्र॒जाप॑ति॒रिति॑ प्र॒जा - प॒तिः॒ । अधि॑पति॒रित्यधि॑ - प॒तिः॒ । आ॒सी॒त् । ति॒सृभि॒रिति॑ ति॒सृ - भिः॒ । अ॒स्तु॒व॒त॒ । ब्रह्म॑ । अ॒सृ॒ज्य॒त॒ । ब्रह्म॑णः । पतिः॑ । अधि॑पति॒रित्यधि॑-प॒तिः॒ । आ॒सी॒त् । प॒ञ्चभि॒रिति॑ प॒ञ्च-भिः॒ । अ॒स्तु॒व॒त॒ । भू॒तानि॑ । अ॒सृ॒ज्य॒न्त॒ । भू॒ताना᳚म् । पतिः॑ । अधि॑पति॒रित्यधि॑ - प॒तिः॒ । आ॒सी॒त् । स॒प्तभि॒रिति॑ स॒प्त - भिः॒ । अ॒स्तु॒व॒त॒ । स॒प्त॒र्॒.षय॒ इति॑ सप्त - ऋ॒षयः॑ । अ॒सृ॒ज्य॒न्त॒ । धा॒ता । अधि॑पति॒रित्यधि॑ - प॒तिः॒ । आ॒सी॒त् । न॒वभि॒रिति॑ न॒व - भिः॒ । अ॒स्तु॒व॒त॒ । पि॒तरः॑ । अ॒सृ॒ज्य॒न्त॒ । अदि॑तिः । अधि॑प॒त्नीत्यधि॑-प॒त्नी॒ । आ॒सी॒त् । ए॒का॒द॒शभि॒रित्ये॑काद॒श - भिः॒ । अ॒स्तु॒व॒त॒ । ऋ॒तवः॑ । अ॒सृ॒ज्य॒न्त॒ । आ॒र्त॒वः । अधि॑पति॒रित्यधि॑ - प॒तिः॒ । आ॒सी॒त् । त्र॒यो॒द॒शभि॒रिति॑ त्रयोद॒श - भिः॒ । अ॒स्तु॒व॒त॒ । मासाः᳚ । अ॒सृ॒ज्य॒न्त॒ । सं॒ॅव॒थ्स॒र इति॑ सं - व॒थ्स॒रः । अधि॑पति॒रित्यधि॑ - प॒तिः॒ ।  \newline




\markright{ TS 4.3.10.2  \hfill https://www.vedavms.in \hfill}
\addcontentsline{toc}{section}{ TS 4.3.10.2 }
\section*{ TS 4.3.10.2 }

\textbf{TS 4.3.10.2 } \newline
\textbf{Samhita Paata} \newline

-रासीत् पञ्चद॒शभि॑रस्तुवत क्ष॒त्रम॑सृज्य॒तेन्द्रो ऽधि॑पतिरासीथ्-सप्तद॒शभि॑रस्तुवत प॒शवो॑ऽसृज्यन्त॒ बृह॒स्पति॒रधि॑पति-रासीन्नवद॒शभि॑-रस्तुवत शूद्रा॒र्याव॑सृज्येतामहोरा॒त्रे अधि॑पत्नी आस्ता॒मेक॑विꣳ शत्याऽस्तुव॒तैक॑शफाः प॒शवो॑ऽसृज्यन्त॒ वरु॒णो ऽधि॑पतिरासी॒त् त्रयो॑विꣳशत्याऽस्तुवत क्षु॒द्राः प॒शवो॑ऽसृज्यन्त पू॒षा ऽधि॑पतिरासी॒त् पञ्च॑विꣳशत्या ऽस्तुवता*ऽऽर॒ण्याः प॒शवो॑ऽसृज्यन्त वा॒युरधि॑पतिरासीथ् स॒प्तविꣳ॑शत्याऽस्तुवत॒ द्यावा॑पृथि॒वी व्यै॑- [  ] \newline

\textbf{Pada Paata} \newline

आ॒सी॒त् । प॒ञ्च॒द॒शभि॒रिति॑ पञ्चद॒श -भिः॒ । अ॒स्तु॒व॒त॒ । क्ष॒त्रम् । अ॒सृ॒ज्य॒त॒ । इन्द्रः॑ । अधि॑पति॒रित्यधि॑ - प॒तिः॒ । आ॒सी॒त् । स॒प्त॒द॒शभि॒रिति॑ सप्तद॒श-भिः॒ । अ॒स्तु॒व॒त॒ । प॒शवः॑ । अ॒सृ॒ज्य॒न्त॒ । बृह॒स्पतिः॑ । अधि॑पति॒रित्यधि॑ - प॒तिः॒ । आ॒सी॒त् । न॒व॒द॒शभि॒रिति॑ नवद॒श -भिः॒ । अ॒स्तु॒व॒त॒ । शू॒द्रा॒र्याविति॑ शूद्र - अ॒र्यौ । अ॒सृ॒ज्ये॒ता॒म् । अ॒हो॒रा॒त्रे इत्य॑हः - रा॒त्रे । अधि॑पत्नी॒ इत्यधि॑- प॒त्नी॒ । आ॒स्ता॒म् । एक॑विꣳश॒त्येत्येक॑ - विꣳ॒॒श॒त्या॒ । अ॒स्तु॒व॒त । एक॑शफा॒ इत्येक॑ - श॒फाः॒ । प॒शवः॑ । अ॒सृ॒ज्य॒न्त॒ । वरु॑णः । अधि॑पति॒रित्यधि॑ - प॒तिः॒ । आ॒सी॒त् । त्रयो॑विꣳश॒त्येति॒ त्रयः॑-विꣳ॒॒श॒त्या॒ । अ॒स्तु॒व॒त॒ । क्षु॒द्राः । प॒शवः॑ । अ॒सृ॒ज्य॒न्त॒ । पू॒षा । अधि॑पति॒रित्यधि॑ - प॒तिः॒ । आ॒सी॒त् । पञ्च॑विꣳश॒त्येति॒ पञ्च॑ - विꣳ॒॒श॒त्या॒ । अ॒स्तु॒व॒त॒ । आ॒र॒ण्याः । प॒शवः॑ । अ॒सृ॒ज्य॒न्त॒ । वा॒युः । अधि॑पति॒रित्यधि॑ - प॒तिः॒ । आ॒सी॒त् । स॒प्तविꣳ॑श॒त्येति॑ स॒प्त - विꣳ॒॒श॒त्या॒ । अ॒स्तु॒व॒त॒ । द्यावा॑पृथि॒वी इति॒ द्यावा᳚ - पृ॒थि॒वी । वीति॑ ।  \newline




\markright{ TS 4.3.10.3  \hfill https://www.vedavms.in \hfill}
\addcontentsline{toc}{section}{ TS 4.3.10.3 }
\section*{ TS 4.3.10.3 }

\textbf{TS 4.3.10.3 } \newline
\textbf{Samhita Paata} \newline

-तां॒ ॅवस॑वो रु॒द्रा आ॑दि॒त्या अनु॒ व्या॑य॒न् तेषा॒माधि॑पत्यमासी॒-न्नव॑विꣳ शत्याऽस्तुवत॒ वन॒स्पत॑योऽसृज्यन्त॒ सोमो ऽधि॑पतिरासी॒-देक॑त्रिꣳशता ऽस्तुवत प्र॒जा अ॑सृज्यन्त॒ यावा॑नां॒ चाया॑वानां॒ चाऽऽधि॑पत्यमासी॒त् त्रय॑स्त्रिꣳशता ऽस्तुवत भू॒तान्य॑शाम्यन् प्र॒जाप॑तिः परमे॒ष्ठ्यधि॑पतिरासीत् ॥ \newline

\textbf{Pada Paata} \newline

ए॒ता॒म् । वस॑वः । रु॒द्राः । आ॒दि॒त्याः । अनु॑ । वीति॑ । आ॒य॒न्न् । तेषा᳚म् । आधि॑पत्य॒मित्याधि॑ - प॒त्य॒म् । आ॒सी॒त् । नव॑विꣳश॒त्येति॒ नव॑ - विꣳ॒॒श॒त्या॒ । अ॒स्तु॒व॒त॒ । वन॒स्पत॑यः । अ॒सृ॒ज्य॒न्त॒ । सोमः॑ । अधि॑पति॒रित्यधि॑ - प॒तिः॒ । आ॒सी॒त् । एक॑त्रिꣳश॒तेत्ये॑क-त्रिꣳ॒॒श॒ता॒ । अ॒स्तु॒व॒त॒ । प्र॒जा इति॑ प्र - जाः । अ॒सृ॒ज्य॒न्त॒ । यावा॑नाम् । च॒ । अया॑वानाम् । च॒ । आधि॑पत्य॒मित्याधि॑ - प॒त्य॒म् । आ॒सी॒त् । त्रय॑स्त्रिꣳश॒तेति॒ त्रयः॑-त्रिꣳ॒॒श॒ता॒ । अ॒स्तु॒व॒त॒ । भू॒तानि॑ । अ॒शा॒म्य॒न्न् । प्र॒जाप॑ति॒रिति॑ प्र॒जा - प॒तिः॒ । प॒र॒मे॒ष्ठी । अधि॑पति॒रित्यधि॑ - प॒तिः॒ । आ॒सी॒त् ॥  \newline




\markright{ TS 4.3.11.1  \hfill https://www.vedavms.in \hfill}
\addcontentsline{toc}{section}{ TS 4.3.11.1 }
\section*{ TS 4.3.11.1 }

\textbf{TS 4.3.11.1 } \newline
\textbf{Samhita Paata} \newline

इ॒यमे॒व सा या प्र॑थ॒मा व्यौच्छ॑द॒न्तर॒स्यां च॑रति॒ प्रवि॑ष्टा । व॒धूर्ज॑जान नव॒गज्जनि॑त्री॒ त्रय॑ एनां महि॒मानः॑ सचन्ते ॥ छन्द॑स्वती उ॒षसा॒ पेपि॑शाने समा॒नं ॅयोनि॒मनु॑ स॒ञ्चर॑न्ती । सूर्य॑पत्नी॒ वि च॑रतः प्रजान॒ती के॒तुं कृ॑ण्वा॒ने अ॒जर॒ भूरि॑रेतसा ॥ ऋ॒तस्य॒ पन्था॒मनु॑ ति॒स्र आऽगु॒स्त्रयो॑ घ॒र्मासो॒ अनु॒ ज्योति॒षाऽऽगुः॑ । प्र॒जामेका॒ रक्ष॒त्यूर्ज॒मेका᳚ -[  ] \newline

\textbf{Pada Paata} \newline

इ॒यम् । ए॒व । सा । या । प्र॒थ॒मा । व्यौच्छ॒दिति॑ वि - औच्छ॑त् । अ॒न्तः । अ॒स्याम् । च॒र॒ति॒ । प्रवि॒ष्टेति॒ प्र - वि॒ष्टा॒ ॥ व॒धूः । ज॒जा॒न॒ । न॒व॒गदिति॑ नव - गत् । जनि॑त्री । त्रयः॑ । ए॒ना॒म् । म॒हि॒मानः॑ । स॒च॒न्ते॒ ॥ छन्द॑स्वती॒ इति॑ । उ॒षसा᳚ । पेपि॑शाने॒ इति॑ । स॒मा॒नम् । योनि᳚म् । अन्विति॑ । स॒ञ्चर॑न्ती॒ इति॑ सं - चर॑न्ती ॥ सूर्य॑पत्नी॒ इति॒ सूर्य॑ - प॒त्नी॒ । वीति॑ । च॒र॒तः॒ । प्र॒जा॒न॒ती इति॑ प्र-जा॒न॒ती । के॒तुम् । कृ॒ण्वा॒ने इति॑ । अ॒जरे॒ इति॑ । भूरि॑रे॒तसेति॒ भूरि॑ - रे॒त॒सा॒ ॥ ऋ॒तस्य॑ । पन्था᳚म् । अन्विति॑ । ति॒स्रः । एति॑ । अ॒गुः॒ । त्रयः॑ । घ॒र्मासः॑ । अन्विति॑ । ज्योति॑षा । एति॑ । अ॒गुः॒ ॥ प्र॒जामिति॑ प्र - जाम् । एका᳚ । रक्ष॑ति । ऊर्ज᳚म् । एका᳚ ।  \newline




\markright{ TS 4.3.11.2  \hfill https://www.vedavms.in \hfill}
\addcontentsline{toc}{section}{ TS 4.3.11.2 }
\section*{ TS 4.3.11.2 }

\textbf{TS 4.3.11.2 } \newline
\textbf{Samhita Paata} \newline

व्र॒तमेका॑ रक्षति देवयू॒नां ॥ च॒तु॒ष्टो॒मो अ॑भव॒द्या तु॒रीया॑ य॒ज्ञ्स्य॑ प॒क्षावृ॑षयो॒ भव॑न्ती । गा॒य॒त्रीं त्रि॒ष्टुभं॒ ज॑गतीमनु॒ष्टुभं॑ बृ॒हद॒र्कं ॅयु॑ञ्जा॒नाः सुव॒राऽभ॑रन्नि॒दं ॥ प॒ञ्चभि॑र्द्धा॒ता वि द॑धावि॒दं ॅयत् तासाꣳ॒॒ स्वसॄ॑रजनय॒त् पञ्च॑पञ्च । तासा॑मु यन्ति प्रय॒वेण॒ पञ्च॒ नाना॑ रू॒पाणि॒ क्रत॑वो॒ वसा॑नाः ॥ त्रिꣳ॒॒शथ् स्वसा॑र॒ उप॑यन्ति निष्कृ॒तꣳ स॑मा॒नं के॒तुं प्र॑तिमु॒ञ्चमा॑नाः । \newline

\textbf{Pada Paata} \newline

व्र॒तम् । एका᳚ । र॒क्ष॒ति॒ । दे॒व॒यू॒नामिति॑ देव - यू॒नाम् ॥ च॒तु॒ष्टो॒म इति॑ चतुः-स्तो॒मः । अ॒भ॒व॒त् । या । तु॒रीया᳚ । य॒ज्ञ्स्य॑ । प॒क्षौ । ऋ॒ष॒यः॒ । भव॑न्ती ॥ गा॒य॒त्रीम् । त्रि॒ष्टुभ᳚म् । जग॑तीम् । अ॒नु॒ष्टुभ॒मित्य॑नु-स्तुभ᳚म् । बृ॒हत् । अ॒र्कम् । यु॒ञ्जा॒नाः । सुवः॑ । एति॑ । अ॒भ॒र॒न्न् । इ॒दम् ॥ प॒ञ्चभि॒रिति॑ प॒ञ्च - भिः॒ । धा॒ता । वीति॑ । द॒धौ॒ । इ॒दम् । यत् । तासा᳚म् । स्वसॄः᳚ । अ॒ज॒न॒य॒त् । पञ्च॑प॒ञ्चेति॒ पञ्च॑ - प॒ञ्च॒ ॥ तासा᳚म् । उ॒ । य॒न्ति॒ । प्र॒य॒वेणेति॑ प्र - य॒वेन॑ । पञ्च॑ । नाना᳚ । रू॒पाणि॑ । क्रत॑वः । वसा॑नाः ॥ त्रिꣳ॒॒शत् । स्वसा॑रः । उपेति॑ । य॒न्ति॒ । नि॒ष्कृ॒तमिति॑ निः - कृ॒तम् । स॒मा॒नम् । के॒तुम् । प्र॒ति॒मु॒ञ्चमा॑ना॒ इति॑ प्रति - मु॒ञ्चमा॑नाः ॥  \newline




\markright{ TS 4.3.11.3  \hfill https://www.vedavms.in \hfill}
\addcontentsline{toc}{section}{ TS 4.3.11.3 }
\section*{ TS 4.3.11.3 }

\textbf{TS 4.3.11.3 } \newline
\textbf{Samhita Paata} \newline

ऋ॒तूꣳस्त॑न्वते क॒वयः॑ प्रजान॒तीर्मद्ध्ये॑छन्दसः॒ परि॑ यन्ति॒ भास्व॑तीः ॥ ज्योति॑ष्मती॒ प्रति॑ मुञ्चते॒ नभो॒ रात्री॑ दे॒वी सूर्य॑स्य व्र॒तानि॑ । वि प॑श्यन्ति प॒शवो॒ जाय॑माना॒ नाना॑रूपा मा॒तुर॒स्या उ॒पस्थे᳚ ॥ ए॒का॒ष्ट॒का तप॑सा॒ तप्य॑माना ज॒जान॒ गर्भं॑ महि॒मान॒मिन्द्रं᳚ । तेन॒ दस्यू॒न् व्य॑सहन्त दे॒वा ह॒न्ताऽसु॑राणा-मभव॒च्छची॑भिः ॥ अना॑नुजामनु॒जां माम॑कर्त स॒त्यं ॅवद॒न्त्यन्वि॑च्छ ए॒तत् । भू॒यास॑ -[  ] \newline

\textbf{Pada Paata} \newline

ऋ॒तून् । त॒न्व॒ते॒ । क॒वयः॑ । प्र॒जा॒न॒तीरिति॑ प्र - जा॒न॒तीः । मद्ध्ये॑छन्दस॒ इति॒ मद्ध्ये᳚ - छ॒न्द॒सः॒ । परीति॑ । य॒न्ति॒ । भास्व॑तीः ॥ ज्योति॑ष्मती । प्रतीति॑ । मु॒ञ्च॒ते॒ । नभः॑ । रात्री᳚ । दे॒वी । सूर्य॑स्य । व्र॒तानि॑ ॥ वीति॑ । प॒श्य॒न्ति॒ । प॒शवः॑ । जाय॑मानाः । नाना॑रूपा॒ इति॒ नाना᳚ - रू॒पाः॒ । मा॒तुः । अ॒स्याः । उ॒पस्थ॒ इत्यु॒प - स्थे॒ ॥ ए॒का॒ष्ट॒केत्ये॑क- अ॒ष्ट॒का । तप॑सा । तप्य॑माना । ज॒जान॑ । गर्भ᳚म् । म॒हि॒मान᳚म् । इन्द्र᳚म् ॥ तेन॑ । दस्यून्॑ । वीति॑ । अ॒स॒ह॒न्त॒ । दे॒वाः । ह॒न्ता । असु॑राणाम् । अ॒भ॒व॒त् । शची॑भि॒रिति॒ शचि॑ - भिः॒ ॥ अना॑नुजा॒मित्यना॑नु - जा॒म् । अ॒नु॒जामित्य॑नु-जाम् । माम् । अ॒क॒र्त॒ । स॒त्यम् । वद॑न्ती । अन्विति॑ । इ॒च्छे॒ । ए॒तत् ॥ भू॒यास᳚म् ।  \newline




\markright{ TS 4.3.11.4  \hfill https://www.vedavms.in \hfill}
\addcontentsline{toc}{section}{ TS 4.3.11.4 }
\section*{ TS 4.3.11.4 }

\textbf{TS 4.3.11.4 } \newline
\textbf{Samhita Paata} \newline

मस्य सुम॒तौ यथा॑ यू॒यम॒न्या वो॑ अ॒न्यामति॒ मा प्र यु॑क्त ॥ अभू॒न्मम॑ सुम॒तौ वि॒श्ववे॑दा॒ आष्ट॑ प्रति॒ष्ठामवि॑द॒द्धि गा॒धं । भू॒यास॑मस्य सुम॒तौ यथा॑ यू॒यम॒न्या वो॑ अ॒न्यामति॒ मा प्रयु॑क्त ॥ पञ्च॒ व्यु॑ष्टी॒रनु॒ पञ्च॒ दोहा॒ गां पञ्च॑नाम्नीमृ॒तवोऽनु॒ पञ्च॑ । पञ्च॒ दिशः॑ पञ्चद॒शेन॑ क्लृ॒प्ताः स॑मा॒नमू᳚र्द्ध्नीर॒भि लो॒कमेकं᳚ ॥ \newline

\textbf{Pada Paata} \newline

अ॒स्य॒ । सु॒म॒ताविति॑ सु - म॒तौ । यथा᳚ । यू॒यम् । अ॒न्या । वः॒ । अ॒न्याम् । अतीति॑ । मा । प्रेति॑ । यु॒क्त॒ ॥ अभू᳚त् । मम॑ । सु॒म॒ताविति॑ सु - म॒तौ । वि॒श्ववे॑दा॒ इति॑ वि॒श्व - वे॒दाः॒ । आष्ट॑ । प्र॒ति॒ष्ठामिति॑ प्रति - स्थाम् । अवि॑दत् । हि । गा॒धम् ॥ भू॒यास᳚म् । अ॒स्य॒ । सु॒म॒ताविति॑ सु - म॒तौ । यथा᳚ । यू॒यम् । अ॒न्या । वः॒ । अ॒न्याम् । अतीति॑ । मा । प्रेति॑ । यु॒क्त॒ ॥ पञ्च॑ । व्यु॑ष्टी॒रिति॒ वि - उ॒ष्टीः॒ । अन्विति॑ । पञ्च॑ । दोहाः᳚ । गाम् । पञ्च॑नाम्नी॒मिति॒ पञ्च॑ - ना॒म्नी॒म् । ऋ॒तवः॑ । अन्विति॑ । पञ्च॑ ॥ पञ्च॑ । दिशः॑ । प॒ञ्च॒द॒शेनेति॑ पञ्च - द॒शेन॑ । क्लृ॒प्ताः । स॒मा॒नमू᳚द्‌र्ध्नी॒रिति॑ समा॒न - मू॒द्‌र्ध्नीः॒ । अ॒भीति॑ । लो॒कम् । एक᳚म् ॥  \newline




\markright{ TS 4.3.11.5  \hfill https://www.vedavms.in \hfill}
\addcontentsline{toc}{section}{ TS 4.3.11.5 }
\section*{ TS 4.3.11.5 }

\textbf{TS 4.3.11.5 } \newline
\textbf{Samhita Paata} \newline

ऋ॒तस्य॒ गर्भः॑ प्रथ॒मा व्यू॒षुष्य॒पामेका॑ महि॒मानं॑ बिभर्ति । सूर्य॒स्यैका॒ चर॑ति निष्कृ॒तेषु॑ घ॒र्मस्यैका॑ सवि॒तैकां॒ नि य॑च्छति ॥ या प्र॑थ॒मा व्यौच्छ॒थ् सा धे॒नुर॑भवद्य॒मे । सा नः॒ पय॑स्वती धु॒क्ष्वोत्त॑रामुत्तराꣳ॒॒ समां᳚ ॥ शु॒क्रर्.ष॑भा॒ नभ॑सा॒ ज्योति॒षा ऽऽ*गा᳚द्-वि॒श्वरू॑पा शब॒लीर॒ग्निके॑तुः । स॒मा॒नमर्थꣳ॑ स्वप॒स्यमा॑ना॒ बिभ्र॑ती ज॒राम॑जर उष॒ आऽगाः᳚ ॥ ऋ॒तू॒नां पत्नी᳚ ( ) प्रथ॒मेयमाऽगा॒दह्नां᳚ ने॒त्री ज॑नि॒त्री प्र॒जानां᳚ । एका॑ स॒ती ब॑हु॒धोषो॒ व्यु॑च्छ॒स्यजी᳚र्णा॒ त्वं ज॑रयसि॒ सर्व॑म॒न्यत् ॥ \newline

\textbf{Pada Paata} \newline

ऋ॒तस्य॑ । गर्भः॑ । प्र॒थ॒मा । व्यू॒षुषीति॑ वि - ऊ॒षुषी᳚ । अ॒पाम् । एका᳚ । म॒हि॒मान᳚म् । बि॒भ॒र्ति॒ ॥ सूर्य॑स्य । एका᳚ । चर॑ति । नि॒ष्कृ॒तेष्विति॑ निः-कृ॒तेषु॑ । घ॒र्मस्य॑ । एका᳚ । स॒वि॒ता । एका᳚म् । नीति॑ । य॒च्छ॒ति॒ ॥ या । प्र॒थ॒मा । व्यौच्छ॒दिति॑ वि - औच्छ॑त् । सा । धे॒नुः । अ॒भ॒व॒त् । य॒मे ॥ सा । नः॒ । पय॑स्वती । धु॒क्ष्व॒ । उत्त॑रामुत्तरा॒मित्युत्त॑रां-उ॒त्त॒रा॒म् । समा᳚म् ॥ शु॒क्रर्.ष॒भेति॑ शु॒क्र - ऋ॒ष॒भा॒ । नभ॑सा । ज्योति॑षा । एति॑ । अ॒गा॒त् । वि॒श्वरू॒पेति॑ वि॒श्व - रू॒पा॒ । श॒ब॒लीः । अ॒ग्निके॑तु॒रित्य॒ग्नि - के॒तुः॒ ॥ स॒मा॒नम् । अर्थ᳚म् । स्व॒प॒स्यमा॒नेति॑ सु-अ॒प॒स्यमा॑ना । बिभ्र॑ती । ज॒राम् । अ॒ज॒रे॒ । उ॒षः॒ । एति॑ । अ॒गाः॒ ॥ ऋ॒तू॒नाम् । पत्नी᳚ ( ) । प्र॒थ॒मा । इ॒यम् । एति॑ । अ॒गा॒त् । अह्ना᳚म् । ने॒त्री । ज॒नि॒त्री । प्र॒जाना॒मिति॑ प्र - जाना᳚म् ॥ एका᳚ । स॒ती । ब॒हु॒धेति॑ बहु - धा । उ॒षः॒ । वीति॑ । उ॒च्छ॒सि॒ । अजी᳚र्णा । त्वम् । ज॒र॒य॒सि॒ । सर्व᳚म् । अ॒न्यत् ॥  \newline




\markright{ TS 4.3.12.1  \hfill https://www.vedavms.in \hfill}
\addcontentsline{toc}{section}{ TS 4.3.12.1 }
\section*{ TS 4.3.12.1 }

\textbf{TS 4.3.12.1 } \newline
\textbf{Samhita Paata} \newline

अग्ने॑ जा॒तान् प्रणु॑दा नः स॒पत्ना॒न् प्रत्यजा॑ताञ्जातवेदो नुदस्व । अ॒स्मे दी॑दिहि सु॒मना॒ अहे॑ड॒न् तव॑ स्याꣳ॒॒ शर्म॑न् त्रि॒वरू॑थ उ॒द्भित् ॥ सह॑सा जा॒तान् प्रणु॑दानः स॒पत्ना॒न् प्रत्यजा॑ताञ्जातवेदो नुदस्व । अधि॑ नो ब्रूहि सुमन॒स्यमा॑नो व॒यꣳ स्या॑म॒ प्रणु॑दा नः स॒पत्नान्॑ ॥ च॒तु॒श्च॒त्वा॒रिꣳ॒॒शः स्तोमो॒ वर्चो॒ द्रवि॑णꣳ षोड॒शः स्तोम॒ ओजो॒ द्रवि॑णं पृथि॒व्याः पुरी॑षम॒स्य- [  ] \newline

\textbf{Pada Paata} \newline

अग्ने᳚ । जा॒तान् । प्रेति॑ । नु॒द॒ । नः॒ । स॒पत्नान्॑ । प्रतीति॑ । अजा॑तान् । जा॒त॒वे॒द॒ इति॑ जात - वे॒दः॒ । नु॒द॒स्व॒ ॥ अ॒स्मे इति॑ । दी॒दि॒हि॒ । सु॒मना॒ इति॑ सु - मनाः᳚ । अहे॑डन्न् । तव॑ । स्या॒म् । शर्मन्न्॑ । त्रि॒वरू॑थ॒ इति॑ त्रि-वरू॑थः । उ॒द्भिदित्यु॑त् - भित् ॥ सह॑सा । जा॒तान् । प्रेति॑ । नु॒द॒ । नः॒ । स॒पत्नान्॑ । प्रतीति॑ । अजा॑तान् । जा॒त॒वे॒द॒ इति॑ जात - वे॒दः॒ । नु॒द॒स्व॒ ॥ अधीति॑ । नः॒ । ब्रू॒हि॒ । सु॒म॒न॒स्यमा॑न॒ इति॑ सु - म॒न॒स्यमा॑नः । व॒यम् । स्या॒म॒ । प्रेति॑ । नु॒द॒ । नः॒ । स॒पत्नान्॑ ॥ च॒तु॒श्च॒त्वा॒रिꣳ॒॒श इति॑ चतुः - च॒त्वा॒रिꣳ॒॒शः । स्तोमः॑ । वर्चः॑ । द्रवि॑णम् । षो॒ड॒शः । स्तोमः॑ । ओजः॑ । द्रवि॑णम् । पृ॒थि॒व्याः । पुरी॑षम् । अ॒सि॒ ।  \newline




\markright{ TS 4.3.12.2  \hfill https://www.vedavms.in \hfill}
\addcontentsline{toc}{section}{ TS 4.3.12.2 }
\section*{ TS 4.3.12.2 }

\textbf{TS 4.3.12.2 } \newline
\textbf{Samhita Paata} \newline

-फ्सो॒ नाम॑ । एव॒ श्छन्दो॒ वरि॑व॒ श्छन्दः॑ श॒भूं श्छन्दः॑ परि॒भू श्छन्द॑ आ॒च्छच्छन्दो॒ मन॒ श्छन्दो॒ व्यच॒ श्छन्दः॒ सिन्धु॒ श्छन्दः॑ समु॒द्रं छन्दः॑ सलि॒लं छन्दः॑ सं॒ॅयच्छन्दो॑ वि॒यच्छन्दो॑ बृ॒हच्छन्दो॑ रथन्त॒रं छन्दो॑ निका॒य श्छन्दो॑ विव॒ध श्छन्दो॒ गिर॒ श्छन्दो॒ भ्रज॒ श्छन्दः॑ स॒ष्टुप् छन्दो॑ ऽनु॒ष्टुप् छन्दः॑ क॒कुच्छन्द॑ स्त्रिक॒कुच्छन्दः॑ का॒व्यं छन्दो᳚ -ऽङ्कु॒पं छन्दः॑ - [  ] \newline

\textbf{Pada Paata} \newline

अफ्सः॑ । नाम॑ ॥ एवः॑ । छन्दः॑ । वरि॑वः । छन्दः॑ । श॒भूंरिति॑ शं - भूः । छन्दः॑ । प॒रि॒भूरिति॑ परि - भूः । छन्दः॑ । आ॒च्छत् । छन्दः॑ । मनः॑ । छन्दः॑ । व्यचः॑ । छन्दः॑ । सिन्धुः॑ । छन्दः॑ । स॒मु॒द्रम् । छन्दः॑ । स॒लि॒लम् । छन्दः॑ । सं॒ॅयदिति॑ सं - यत् । छन्दः॑ । वि॒यदिति॑ वि - यत् । छन्दः॑ । बृ॒हत् । छन्दः॑ । र॒थ॒न्त॒रमिति॑ रथं - त॒रम् । छन्दः॑ । नि॒का॒य इति॑ नि - का॒यः । छन्दः॑ । वि॒व॒ध इति॑ वि-व॒धः । छन्दः॑ । गिरः॑ । छन्दः॑ । भ्रजः॑ । छन्दः॑ । स॒ष्टुबिति॑ स - स्तुप् । छन्दः॑ । अ॒नु॒ष्टुबित्य॑नु - स्तुप् । छन्दः॑ । क॒कुत् । छन्दः॑ । त्रि॒क॒कुदिति॑ त्रि - क॒कुत् । छन्दः॑ । का॒व्यम् । छन्दः॑ । अ॒ङ्कु॒पम् । छन्दः॑ ।  \newline




\markright{ TS 4.3.12.3  \hfill https://www.vedavms.in \hfill}
\addcontentsline{toc}{section}{ TS 4.3.12.3 }
\section*{ TS 4.3.12.3 }

\textbf{TS 4.3.12.3 } \newline
\textbf{Samhita Paata} \newline

प॒दप॑ङ्क्ति॒ श्छन्दो॒ ऽक्षर॑पङ्क्ति॒ श्छन्दो॑ विष्टा॒रप॑ङ्क्ति॒ श्छन्दः॑ क्षु॒रो भृज्वा॒ञ्छन्दः॑ प्र॒च्छच्छन्दः॑ प॒क्ष श्छन्द॒ एव॒ श्छन्दो॒ वरि॑व॒ श्छन्दो॒ वय॒ श्छन्दो॑ वय॒स्कृच्छन्दो॑ विशा॒लं छन्दो॒ विष्प॑र्द्धा॒ श्छन्द॑ श्छ॒दि श्छन्दो॑ दूरोह॒णं छन्द॑स्त॒न्द्रं छन्दो᳚ ऽङ्का॒ङ्कं छन्दः॑ ॥ \newline

\textbf{Pada Paata} \newline

प॒दप॑ङ्क्ति॒रिति॑ प॒द - प॒ङ्क्तिः॒ । छन्दः॑ । अ॒क्षर॑पङ्क्ति॒रित्य॒क्षर॑ - प॒ङ्क्तिः॒ । छन्दः॑ । वि॒ष्टा॒रप॑ङ्क्ति॒रिति॑ विष्टा॒र - प॒ङ्क्तिः॒ । छन्दः॑ । क्षु॒रः । भृज्वान्॑ । छन्दः॑ । प्र॒च्छत् । छन्दः॑ । प॒क्षः । छन्दः॑ । एवः॑ । छन्दः॑ । वरि॑वः । छन्दः॑ । वयः॑ । छन्दः॑ । व॒य॒स्कृदिति॑ वयः - कृत् । छन्दः॑ । वि॒शा॒लमिति॑ वि - शा॒लम् । छन्दः॑ । विष्प॑र्धा॒ इति॒ विः-स्प॒द्‌र्धाः॒ । छन्दः॑ । छ॒दिः । छन्दः॑ । दू॒रो॒ह॒णमिति॑ दूः - रो॒ह॒णम् । छन्दः॑ । त॒न्द्रम् । छन्दः॑ । अ॒ङ्का॒ङ्कम् । छन्दः॑ ॥  \newline




\markright{ TS 4.3.13.1  \hfill https://www.vedavms.in \hfill}
\addcontentsline{toc}{section}{ TS 4.3.13.1 }
\section*{ TS 4.3.13.1 }

\textbf{TS 4.3.13.1 } \newline
\textbf{Samhita Paata} \newline

अ॒ग्निर्वृ॒त्राणि॑ जङ्घनद् द्रविण॒स्युर्वि॑प॒न्यया᳚ । समि॑द्धः शु॒क्र आहु॑तः ॥ त्वꣳ सो॑मासि॒ सत्प॑ति॒स्त्वꣳ राजो॒त वृ॑त्र॒हा । त्वं भ॒द्रो अ॑सि॒ क्रतुः॑ ॥ भ॒द्रा ते॑ अग्ने स्वनीक स॒दृंग्घो॒रस्य॑ स॒तो विषु॑णस्य॒ चारुः॑ । न यत् ते॑ शो॒चिस्तम॑सा॒ वर॑न्त॒ न ध्व॒स्मान॑स्त॒नुवि॒ रेप॒ आ धुः॑ ॥ भ॒द्रं ते॑ अग्ने सहसि॒न्ननी॑कमुपा॒क आ रो॑चते॒ सूर्य॑स्य । \newline

\textbf{Pada Paata} \newline

अ॒ग्निः । वृ॒त्राणि॑ । ज॒ङ्घ॒न॒त् । द्र॒वि॒ण॒स्युः । वि॒प॒न्ययेति॑ वि - प॒न्यया᳚ ॥ समि॑द्ध॒ इति॑ सम् - इ॒द्धः॒ । शु॒क्रः । आहु॑त॒ इत्या - हु॒तः॒ ॥ त्वम् । सो॒म॒ । अ॒सि॒ । सत्प॑ति॒रिति॒ सत् - प॒तिः॒ । त्वम् । राजा᳚ । उ॒त । वृ॒त्र॒हेति॑ वृत्र - हा ॥ त्वम् । भ॒द्रः । अ॒सि॒ । क्रतुः॑ ॥ भ॒द्रा । ते॒ । अ॒ग्ने॒ । स्व॒नी॒केति॑ सु - अ॒नी॒क॒ । स॒न्दृगेति॑ सं - दृक् । घो॒रस्य॑ । स॒तः । विषु॑णस्य । चारुः॑ ॥ न । यत् । ते॒ । शो॒चिः । तम॑सा । वर॑न्त । न । ध्व॒स्मानः॑ । त॒नुवि॑ । रेपः॑ । एति॑ । धुः॒ ॥ भ॒द्रम् । ते॒ । अ॒ग्ने॒ । स॒ह॒सि॒न्न् । अनी॑कम् । उ॒पा॒के । एति॑ । रो॒च॒ते॒ । सूर्य॑स्य ॥  \newline




\markright{ TS 4.3.13.2  \hfill https://www.vedavms.in \hfill}
\addcontentsline{toc}{section}{ TS 4.3.13.2 }
\section*{ TS 4.3.13.2 }

\textbf{TS 4.3.13.2 } \newline
\textbf{Samhita Paata} \newline

रुश॑द्-दृ॒शे द॑दृशे नक्त॒या चि॒दरू᳚क्षितं दृ॒श आ रू॒पे अन्नं᳚ ॥ सैनाऽनी॑केन सुवि॒दत्रो॑ अ॒स्मे यष्टा॑ दे॒वाꣳ आय॑जिष्ठः स्व॒स्ति । अद॑ब्धो गो॒पा उ॒त नः॑ पर॒स्पा अग्ने᳚ द्यु॒मदु॒त रे॒वद् दि॑दीहि ॥ स्व॒स्ति नो॑ दि॒वो अ॑ग्ने पृथि॒व्या वि॒श्वायु॑र्द्धेहि य॒जथा॑य देव । यथ् सी॒महि॑ दिविजात॒ प्रश॑स्तं॒ तद॒स्मासु॒ द्रवि॑णं धेहि चि॒त्रं ॥ यथा॑ होत॒र्मनु॑षो - [  ] \newline

\textbf{Pada Paata} \newline

रुश॑त् । दृ॒शे । द॒दृ॒शे॒ । न॒क्त॒या । चि॒त् । अरू᳚क्षितम् । दृ॒शे । एति॑ । रू॒पे । अन्न᳚म् ॥ सः । ए॒ना । अनी॑केन । सु॒वि॒दत्र॒ इति॑ सु-वि॒दत्रः॑ । अ॒स्मे इति॑ । यष्टा᳚ । दे॒वान् । आय॑जिष्ठ॒ इत्या - य॒जि॒ष्ठः॒ । स्व॒स्ति ॥ अद॑ब्धः । गो॒पा इति गो-पाः । उ॒त । नः॒ । प॒र॒स्पा इति॑ परः-पाः । अग्ने᳚ । द्यु॒मदिति॑ द्यु - मत् । उ॒त । रे॒वत् । दि॒दी॒हि॒ ॥ स्व॒स्ति । नः॒ । दि॒वः । अ॒ग्ने॒ । पृ॒थि॒व्याः । वि॒श्वायु॒रिति॑ वि॒श्व - आ॒युः॒ । धे॒हि॒ । य॒जथा॑य । दे॒व॒ ॥ यत् । सी॒महि॑ । दि॒वि॒जा॒तेति॑ दिवि - जा॒त॒ । प्रश॑स्त॒मिति॒ प्र - श॒स्त॒म् । तत् । अ॒स्मासु॑ । द्रवि॑णम् । धे॒हि॒ । चि॒त्रम् ॥ यथा᳚ । हो॒तः॒ । मनु॑षः ।  \newline




\markright{ TS 4.3.13.3  \hfill https://www.vedavms.in \hfill}
\addcontentsline{toc}{section}{ TS 4.3.13.3 }
\section*{ TS 4.3.13.3 }

\textbf{TS 4.3.13.3 } \newline
\textbf{Samhita Paata} \newline

दे॒वता॑ता य॒ज्ञेभिः॑ सूनो सहसो॒ यजा॑सि । ए॒वा नो॑ अ॒द्य स॑म॒ना स॑मा॒नानु॒-शन्न॑ग्न उश॒तो य॑क्षि दे॒वान् ॥ अ॒ग्निमी॑डे पु॒रोहि॑तं ॅय॒ज्ञ्स्य॑ दे॒वमृ॒त्विजं᳚ । होता॑रꣳ रत्न॒धात॑मं ॥ वृषा॑ सोम द्यु॒माꣳ अ॑सि॒ वृषा॑ देव॒ वृष॑व्रतः । वृषा॒ धर्मा॑णि दधिषे ॥ सान्त॑पना इ॒दꣳ ह॒विर्मरु॑त॒स्तज्जु॑जुष्टन । यु॒ष्माको॒ती रि॑शादसः ॥ यो नो॒ मर्तो॑ वसवो दुर्.हृणा॒युस्ति॒रः स॒त्यानि॑ मरुतो॒ - [  ] \newline

\textbf{Pada Paata} \newline

दे॒वता॒तेति॑ दे॒व - ता॒ता॒ । य॒ज्ञेभिः॑ । सू॒नो॒ इति॑ । स॒ह॒सः॒ । यजा॑सि ॥ ए॒वा । नः॒ । अ॒द्य । स॒म॒ना । स॒मा॒नान् । उ॒शन्न् । अ॒ग्ने॒ । उ॒श॒तः । य॒क्षि॒ । दे॒वान् ॥ अ॒ग्निम् । ई॒डे॒ । पु॒रोहि॑त॒मिति॑ पु॒रः - हि॒त॒म् । य॒ज्ञ्स्य॑ । दे॒वम् । ऋ॒त्विज᳚म् ॥ होता॑रम् । र॒त्न॒धात॑म॒मिति॑ रत्न - धात॑मम् ॥ वृषा᳚ । सो॒म॒ । द्यु॒मानिति॑ द्यु-मान् । अ॒सि॒ । वृषा᳚ । दे॒व॒ । वृष॑व्रत॒ इति॒ वृष॑ - व्र॒तः॒ ॥ वृषा᳚ । धर्मा॑णि । द॒धि॒षे॒ ॥ सान्त॑पना॒ इति॒ सां - त॒प॒नाः॒ । इ॒दम् । ह॒विः । मरु॑तः । तत् । जु॒जु॒ष्ट॒न॒ ॥ यु॒ष्माक॑ । ऊ॒ती । रि॒शा॒द॒स॒ इति॑ रिश - अ॒द॒सः॒ ॥ यः । नः॒ । मर्तः॑ । व॒स॒वः॒ । दु॒र्॒.हृ॒णा॒युरिति॑ दुः - हृ॒णा॒युः । ति॒रः । स॒त्यानि॑ । म॒रु॒तः॒ ।  \newline




\markright{ TS 4.3.13.4  \hfill https://www.vedavms.in \hfill}
\addcontentsline{toc}{section}{ TS 4.3.13.4 }
\section*{ TS 4.3.13.4 }

\textbf{TS 4.3.13.4 } \newline
\textbf{Samhita Paata} \newline

जिघाꣳ॑सात् । द्रु॒हः पाशं॒ प्रति॒ स मु॑चीष्ट॒ तपि॑ष्ठेन॒ तप॑सा हन्तना॒ तं ॥ सं॒ॅव॒थ्स॒रीणा॑ म॒रुतः॑ स्व॒र्का उ॑रु॒क्षयाः॒ सग॑णा॒ मानु॑षेषु । ते᳚ऽस्मत् पाशा॒न् प्र मु॑ञ्च॒न्त्वꣳह॑सः सान्तप॒ना म॑दि॒रा मा॑दयि॒ष्णवः॑ ॥ पि॒प्री॒हि दे॒वाꣳ उ॑श॒तो य॑विष्ठ वि॒द्वाꣳ ऋ॒तूꣳर्.ऋ॑तुपते यजे॒ह । ये दैव्या॑ ऋ॒त्विज॒स्तेभि॑रग्ने॒ त्वꣳ होतॄ॑णाम॒स्याय॑जिष्ठः ॥ अग्ने॒ यद॒द्य वि॒शो अ॑द्ध्वरस्य होतः॒ पाव॑क - [  ] \newline

\textbf{Pada Paata} \newline

जिघाꣳ॑सात् ॥ द्रु॒हः । पाश᳚म् । प्रतीति॑ । सः । मु॒ची॒ष्ट॒ । तपि॑ष्ठेन । तप॑सा । ह॒न्त॒न॒ । तम् ॥ सं॒ॅव॒थ्स॒रीणा॒ इति॑ सं - व॒थ्स॒रीणाः᳚ । म॒रुतः॑ । स्व॒र्का इति॑ सु - अ॒र्काः । उ॒रु॒क्षया॒ इत्यु॑रु - क्षयाः᳚ । सग॑णा॒ इति॒ स - ग॒णाः॒ । मानु॑षेषु ॥ ते । अ॒स्मत् । पाशान्॑ । प्रेति॑ । मु॒ञ्च॒न्तु॒ । अꣳह॑सः । सा॒न्त॒प॒ना इति॑ सां - त॒प॒नाः । म॒दि॒राः । मा॒द॒यि॒ष्णवः॑ ॥ पि॒प्री॒हि । दे॒वान् । उ॒श॒तः । य॒वि॒ष्ठ॒ । वि॒द्वान् । ऋ॒तून् । ऋ॒तु॒प॒त॒ इत्यृ॑तु-प॒ते॒ । य॒ज॒ । इ॒ह ॥ ये । दैव्याः᳚ । ऋ॒त्विजः॑ । तेभिः॑ । अ॒ग्ने॒ । त्वम् । होतॄ॑णाम् । अ॒सि॒ । आय॑जिष्ठ॒ इत्या - य॒जि॒ष्ठः॒ ॥ अग्ने᳚ । यत् । अ॒द्य । वि॒शः । अ॒द्ध्व॒र॒स्य॒ । हो॒तः॒ । पाव॑क ।  \newline




\markright{ TS 4.3.13.5  \hfill https://www.vedavms.in \hfill}
\addcontentsline{toc}{section}{ TS 4.3.13.5 }
\section*{ TS 4.3.13.5 }

\textbf{TS 4.3.13.5 } \newline
\textbf{Samhita Paata} \newline

शोचे॒ वेष्ट्वꣳ हि यज्वा᳚ । ऋ॒ता य॑जासि महि॒ना वि यद्भूर्.ह॒व्या व॑ह यविष्ठ॒ या ते॑ अ॒द्य ॥ अ॒ग्निना॑ र॒यिम॑श्नव॒त् पोष॑मे॒व दि॒वेदि॑वे । य॒शसं॑ ॅवी॒रव॑त्तमं ॥ ग॒य॒स्फानो॑ अमीव॒हा व॑सु॒वित् पु॑ष्टि॒वर्द्ध॑नः । सु॒मि॒त्रः सो॑म नो भव ॥ गृह॑मेधास॒ आ ग॑त॒ मरु॑तो॒ माऽप॑ भूतन । प्र॒मु॒ञ्चन्तो॑ नो॒ अꣳह॑सः ॥ पू॒र्वीभि॒र्॒.हि द॑दाशि॒म श॒रद्भि॑र्मरुतो व॒यं । महो॑भि- [  ] \newline

\textbf{Pada Paata} \newline

शो॒चे॒ । वेः । त्वम् । हि । यज्वा᳚ ॥ ऋ॒ता । य॒जा॒सि॒ । म॒हि॒ना । वीति॑ । यत् । भूः । ह॒व्या । व॒ह॒ । य॒वि॒ष्ठ॒ । या । ते॒ । अ॒द्य ॥ अ॒ग्निना᳚ । र॒यिम् । अ॒श्न॒व॒त् । पोष᳚म् । ए॒व । दि॒वेदि॑व॒ इति॑ दि॒वे - दि॒वे॒ ॥ य॒शस᳚म् । वी॒रव॑त्तम॒मिति॑ वी॒रव॑त् - त॒म॒म् ॥ ग॒य॒स्फान॒ इति॑ गय - स्फानः॑ । अ॒मी॒व॒हेत्य॑मीव - हा । व॒सु॒विदिति॑ वसु - वित् । पु॒ष्टि॒वद्‌र्ध॑न॒ इति॑ पुष्टि - वर्ध॑नः ॥ सु॒मि॒त्र इति॑ सु - मि॒त्रः । सो॒म॒ । नः॒ । भ॒व॒ ॥ गृह॑मेधास॒ इति॒ गृह॑ - मे॒धा॒सः॒ । एति॑ । ग॒त॒ । मरु॑तः । मा । अपेति॑ । भू॒त॒न॒ ॥ प्र॒मु॒ञ्चन्त॒ इति॑ प्र - मु॒ञ्चन्तः॑ । नः॒ । अꣳह॑सः ॥ पू॒र्वीभिः॑ । हि । द॒दा॒शि॒म । श॒रद्भि॒रिति॑ श॒रत् - भिः॒ । म॒रु॒तः॒ । व॒यम् ॥ महो॑भि॒रिति॒ महः॑ - भिः॒ ।  \newline




\markright{ TS 4.3.13.6  \hfill https://www.vedavms.in \hfill}
\addcontentsline{toc}{section}{ TS 4.3.13.6 }
\section*{ TS 4.3.13.6 }

\textbf{TS 4.3.13.6 } \newline
\textbf{Samhita Paata} \newline

-श्चर्.षणी॒नां ॥ प्रबु॒द्ध्निया॑ ईरते वो॒ महाꣳ॑सि॒ प्रणामा॑नि प्रयज्यवस्तिरद्ध्वं । स॒ह॒स्रियं॒ दम्यं॑ भा॒गमे॒तं गृ॑हमे॒धीयं॑ मरुतो जुषद्ध्वं ॥ उप॒ यमेति॑ युव॒तिः सु॒दक्षं॑ दो॒षा वस्तोर्॑. ह॒विष्म॑ती घृ॒ताची᳚ । उप॒ स्वैन॑म॒रम॑तिर्व-सू॒युः ॥ इ॒मो अ॑ग्ने वी॒तत॑मानि ह॒व्या ऽज॑स्रो वक्षि दे॒वता॑ति॒मच्छ॑ । प्रति॑ न ईꣳ सुर॒भीणि॑ वियन्तु ॥क्री॒डं ॅवः॒ शर्द्धो॒ मारु॑तमन॒र्वाणꣳ॑ रथे॒शुभं᳚ । \newline

\textbf{Pada Paata} \newline

च॒र्॒.ष॒णी॒नाम् ॥ प्रेति॑ । बु॒द्ध्निया᳚ । ई॒र॒ते॒ । वः॒ । महाꣳ॑सि । प्रेति॑ । नामा॑नि । प्र॒य॒ज्य॒व॒ इति॑ प्र - य॒ज्य॒वः॒ । ति॒र॒द्ध्व॒म् ॥ स॒ह॒स्रिय᳚म् । दम्य᳚म् । भा॒गम् । ए॒तम् । गृ॒ह॒मे॒धीय॒मिति॑ गृह - मे॒धीय᳚म् । म॒रु॒तः॒ । जु॒ष॒द्ध्व॒म् ॥ उपेति॑ । यम् । एति॑ । यु॒व॒तिः । सु॒दक्ष॒मिति॑ सु-दक्ष᳚म् । दो॒षा । वस्तोः᳚ । ह॒विष्म॑ती । घृ॒ताची᳚ ॥ उपेति॑ । स्व । ए॒न॒म् । अ॒रम॑तिः । व॒सू॒युरिति॑ वसू - युः ॥ इ॒मो इति॑ । अ॒ग्ने॒ । वी॒तत॑मा॒नीति॑ वी॒त - त॒मा॒नि॒ । ह॒व्या । अज॑स्रः । व॒क्षि॒ । दे॒वता॑ति॒मिति॑ दे॒व - ता॒ति॒म् । अच्छ॑ ॥ प्रतीति॑ । नः॒ । ई॒म् । सु॒र॒भीणि॑ । वि॒य॒न्तु॒ ॥ क्री॒डम् । वः॒ । शद्‌र्धः॑ । मारु॑तम् । अ॒न॒र्वाण᳚म् । र॒थे॒शुभ॒मिति॑ रथे - शुभ᳚म् ॥  \newline




\markright{ TS 4.3.13.7  \hfill https://www.vedavms.in \hfill}
\addcontentsline{toc}{section}{ TS 4.3.13.7 }
\section*{ TS 4.3.13.7 }

\textbf{TS 4.3.13.7 } \newline
\textbf{Samhita Paata} \newline

कण्वा॑ अ॒भि प्र गा॑यत ॥ अत्या॑सो॒ न ये म॒रुतः॒ स्वञ्चो॑ यक्ष॒दृशो॒ न शु॒भय॑न्त॒ मर्याः᳚ । ते ह॑र्म्ये॒ष्ठाः शिश॑वो॒ न शु॒भ्रा व॒थ्सासो॒ न प्र॑क्री॒डिनः॑ पयो॒धाः ॥ प्रैषा॒मज्मे॑षु विथु॒रेव॑ रेजते॒ भूमि॒र्यामे॑षु॒ यद्ध॑ यु॒ञ्जते॑ शु॒भे । ते क्री॒डयो॒ धुन॑यो॒ भ्राज॑दृष्टयः स्व॒यं म॑हि॒त्वं प॑नयन्त॒ धूत॑यः ॥ उ॒प॒ह्व॒रेषु॒ यदचि॑द्ध्वं ॅय॒यिं ॅवय॑ इव मरुतः॒ केन॑ - [  ] \newline

\textbf{Pada Paata} \newline

कण्वाः᳚ । अ॒भि । प्रेति॑ । गा॒य॒त॒ ॥ अत्या॑सः । न । ये । म॒रुतः॑ । स्वञ्चः॑ । य॒क्ष॒दृश॒ इति॑ यक्ष - दृशः॑ । न । शु॒भय॑न्त । मर्याः᳚ ॥ ते । ह॒र्म्ये॒ष्ठा इति॑ हर्म्ये-स्थाः । शिश॑वः । न । शु॒भ्राः । व॒थ्सासः॑ । न । प्र॒क्री॒डिन॒ इति॑ प्र - क्री॒डिनः॑ । प॒यो॒धा इति॑ पयः - धाः ॥ प्रेति॑ । ए॒षा॒म् । अज्मे॑षु । वि॒थु॒रा । इ॒व॒ । रे॒ज॒ते॒ । भूमिः॑ । यामे॑षु । यत् । ह॒ । यु॒ञ्जते᳚ । शु॒भे ॥ ते । क्री॒डयः॑ । धुन॑यः । भ्राज॑दृष्टय॒ इति॒ भ्राज॑त् - ऋ॒ष्ट॒यः॒ । स्व॒यम् । म॒हि॒त्वमिति॑ महि - त्वम् । प॒न॒य॒न्त॒ । धूत॑यः ॥ उ॒प॒ह्व॒रेष्वित्यु॑प-ह्व॒रेषु॑ । यत् । अचि॑द्ध्वम् । य॒यिम् । वयः॑ । इ॒व॒ । म॒रु॒तः॒ । केन॑ ।  \newline




\markright{ TS 4.3.13.8  \hfill https://www.vedavms.in \hfill}
\addcontentsline{toc}{section}{ TS 4.3.13.8 }
\section*{ TS 4.3.13.8 }

\textbf{TS 4.3.13.8 } \newline
\textbf{Samhita Paata} \newline

चित् प॒था । श्चोत॑न्ति॒ कोशा॒ उप॑ वो॒ रथे॒ष्वा घृ॒तमु॑क्षता॒ मधु॑वर्ण॒मर्च॑ते ॥ अ॒ग्निम॑ग्निꣳ॒॒ हवी॑मभिः॒ सदा॑ हवन्त वि॒श्पतिं᳚ । ह॒व्य॒वाहं॑ पुरुप्रि॒यं ॥ तꣳ हि शश्व॑न्त॒ ईड॑ते स्रु॒चा दे॒वं घृ॑त॒श्चुता᳚ । अ॒ग्निꣳ ह॒व्याय॒ वोढ॑वे ॥ इन्द्रा᳚ग्नी रोच॒ना दि॒वः > 1, श्नथ॑द्वृ॒त्र >2, मिन्द्रं॑ ॅवो वि॒श्वत॒स्परी>3, न्द्रं॒ नरो॒ >4, विश्व॑कर्मन्. ह॒विषा॑ वावृधा॒नो>5,विश्व॑कर्मन्. ह॒विषा॒ वर्ध॑नेन >6 ॥ \newline

\textbf{Pada Paata} \newline

चि॒त् । प॒था ॥ श्चोत॑न्ति । कोशाः᳚ । उपेति॑ । वः॒ । रथे॑षु । एति॑ । घृ॒तम् । उ॒क्ष॒त॒ । मधु॑वर्ण॒मिति॒ मधु॑ - व॒र्ण॒म् । अर्च॑ते ॥ अ॒ग्निम॑ग्नि॒मित्य॒ग्निम् - अ॒ग्नि॒म् । हवी॑मभि॒रिति॒ हवी॑म - भिः॒ । सदा᳚ । ह॒व॒न्त॒ । वि॒श्पति᳚म् ॥ ह॒व्य॒वाह॒मिति॑ हव्य - वाह᳚म् । पु॒रु॒प्रि॒यमिति॑ पुरु - प्रि॒यम् ॥ तम् । हि । शश्व॑न्तः । ईड॑ते । स्रु॒चा । दे॒वम् । घृ॒त॒श्चुतेति॑ घृत - श्चुता᳚ ॥ अ॒ग्निम् । ह॒व्याय॑ । वोढ॑वे ॥ इन्द्रा᳚ग्नी॒ इतीन्द्र॑ - अ॒ग्नी॒ । रो॒च॒ना । दि॒वः । श्नथ॑त् । वृ॒त्रम् । इन्द्र᳚म् । वः॒ । वि॒श्वतः॑ । परीति॑ । इन्द्र᳚म् । नरः॑ । विश्व॑कर्म॒न्निति॒ विश्व॑ - क॒र्म॒न्न् । ह॒विषा᳚ । वा॒वृ॒धा॒नः । विश्व॑कर्म॒न्निति॒ विश्व॑ - क॒र्म॒न्न् । ह॒विषा᳚ । वद्‌र्ध॑नेन ॥  \newline






\end{document}