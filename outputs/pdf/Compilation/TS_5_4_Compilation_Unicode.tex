\documentclass[17pt]{extarticle}
\usepackage{babel}
\usepackage{fontspec}
\usepackage{polyglossia}
\usepackage{extsizes}

\usepackage{color}   %May be necessary if you want to color links
\usepackage{hyperref}
\hypersetup{
    colorlinks=true, %set true if you want colored links
    linktoc=all,     %set to all if you want both sections and subsections linked
    linkcolor=black,  %choose some color if you want links to stand out
}

\setmainlanguage{sanskrit}
\setotherlanguages{english} %% or other languages
\setlength{\parindent}{0pt}
\pagestyle{myheadings}
\newfontfamily\devanagarifont[Script=Devanagari]{AdishilaVedic}
\renewcommand{\theHsection}{\thepart.section.\thesection}

\newcommand{\VAR}[1]{}
\newcommand{\BLOCK}[1]{}




\begin{document}
\begin{titlepage}
    \begin{center}
 
\begin{sanskrit}
    { \Large
    कृष्ण यजुर्वेदीय तैत्तिरीय संहिता,पद,जटा,घन पाठः 
    }
    \\
    \vspace{2.5cm}
    \mbox{ \Large
    5.4      पञ्चमकाण्डे चतुर्थः प्रश्नः इष्टकात्रयाभिधानं   }
\end{sanskrit}
\end{center}

\end{titlepage}
\tableofcontents
\phantomsection
\pagebreak

\markright{ TS 5.4.1.1  \hfill https://www.vedavms.in \hfill}

\section{ TS 5.4.1.1 }

\textbf{TS 5.4.1.1 } \newline
\textbf{Samhita Paata} \newline

दे॒वा॒सु॒राः संॅय॑त्ता आस॒न् ते न व्य॑जयन्त॒ स ए॒ता इन्द्र॑स्त॒नूर॑पश्य॒त् ता उपा॑धत्त॒ ताभि॒र्वै स त॒नुव॑मिन्द्रि॒यं ॅवी॒र्य॑मा॒त्मन्न॑धत्त॒ ततो॑ दे॒वा अभ॑व॒न् पराऽसु॑रा॒ यदि॑न्द्रत॒नूरु॑प॒दधा॑ति त॒नुव॑मे॒व ताभि॑रिन्द्रि॒यं ॅवी॒र्यं॑ ॅयज॑मान आ॒त्मन् ध॒त्तेऽथो॒ सेन्द्र॑मे॒वाग्निꣳ सत॑नुं चिनुते॒ भव॑त्या॒त्मना॒ परा᳚ऽस्य॒ भ्रातृ॑व्यो - [  ] \newline

\textbf{Pada Paata} \newline

दे॒वा॒सु॒रा इति॑ देव - अ॒सु॒राः । संॅय॑त्ता॒ इति॒ सं-य॒त्ताः॒ । आ॒स॒न्न् । ते । न । वीति॑ । अ॒ज॒य॒न्त॒ । सः । ए॒ताः । इन्द्रः॑ । त॒नूः । अ॒प॒श्य॒त् । ताः । उपेति॑ । अ॒ध॒त्त॒ । ताभिः॑ । वै । सः । त॒नुव᳚म् । इ॒न्द्रि॒यम् । वी॒र्य᳚म् । आ॒त्मन्न् । अ॒ध॒त्त॒ । ततः॑ । दे॒वाः । अभ॑वन्न् । परेति॑ । असु॑राः । यत् । इ॒न्द्र॒त॒नूरिती᳚न्द्र - त॒नूः । उ॒प॒दधा॒तीत्यु॑प - दधा॑ति । त॒नुव᳚म् । ए॒व । ताभिः॑ । इ॒न्द्रि॒यम् । वी॒र्य᳚म् । यज॑मानः । आ॒त्मन्न् । ध॒त्ते॒ । अथो॒ इति॑ । सैन्द्र॒मिति॒ स - इ॒न्द्र॒म् । ए॒व । अ॒ग्निम् । सत॑नु॒मिति॒ स - त॒नु॒म् । चि॒नु॒ते॒ । भव॑ति । आ॒त्मना᳚ । परेति॑ । अ॒स्य॒ । भ्रातृ॑व्यः ।  \newline




\markright{ TS 5.4.1.2  \hfill https://www.vedavms.in \hfill}

\section{ TS 5.4.1.2 }

\textbf{TS 5.4.1.2 } \newline
\textbf{Samhita Paata} \newline

भवति य॒ज्ञो दे॒वेभ्योऽपा᳚क्राम॒त् तम॑व॒रुधं॒ नाश॑क्नुव॒न्त ए॒ता य॑ज्ञ्त॒नूर॑पश्य॒न् ता उपा॑दधत॒ ताभि॒र्वै ते य॒ज्ञ्मवा॑रुन्धत॒ यद्-य॑ज्ञ्त॒नूरु॑प॒दधा॑ति य॒ज्ञ्मे॒व ताभि॒र्यज॑मा॒नोऽव॑ रुन्धे॒ त्रय॑स्त्रिꣳ शत॒मुप॑ दधाति॒ त्रय॑स्त्रिꣳश॒द्वै दे॒वता॑ दे॒वता॑ ए॒वाव॑ रु॒न्धे ऽथो॒ सात्मा॑नमे॒वाग्निꣳ सत॑नुं चिनुते॒ सात्मा॒ऽमुष्मि॑न् ॅलो॒के - [  ] \newline

\textbf{Pada Paata} \newline

भ॒व॒ति॒ । य॒ज्ञ्ः । दे॒वेभ्यः॑ । अपेति॑ । अ॒क्रा॒म॒त् । तम् । अ॒व॒रुध॒मित्य॑व - रुध᳚म् । न । अ॒श॒क्नु॒व॒न्न् । ते । ए॒ताः । य॒ज्ञ्॒त॒नूरिति॑ यज्ञ् - त॒नूः । अ॒प॒श्य॒न्न् । ताः । उपेति॑ । अ॒द॒ध॒त॒ । ताभिः॑ । वै । ते । य॒ज्ञ्म् । अवेति॑ । अ॒रु॒न्ध॒त॒ । यत् । य॒ज्ञ्॒त॒नूरिति॑ यज्ञ् - त॒नूः । उ॒प॒दधा॒तीत्यु॑प - दधा॑ति । य॒ज्ञ्म् । ए॒व । ताभिः॑ । यज॑मानः । अवेति॑ । रु॒न्धे॒ । त्रय॑स्त्रिꣳशत॒मिति॒ त्रयः॑ - त्रिꣳ॒॒श॒त॒म् । उपेति॑ । द॒धा॒ति॒ । त्रय॑स्त्रिꣳश॒दिति॒ त्रयः॑ - त्रिꣳ॒॒श॒त् । वै । दे॒वताः᳚ । दे॒वताः᳚ । ए॒व । अवेति॑ । रु॒न्धे॒ । अथो॒ इति॑ । सात्मा॑न॒मिति॒ स - आ॒त्मा॒न॒म् । ए॒व । अ॒ग्निम् । सत॑नु॒मिति॒ स-त॒नु॒म् । चि॒नु॒ते॒ । सात्मेति॒ स - आ॒त्मा॒ । अ॒मुष्मिन्॑ । लो॒के ।  \newline




\markright{ TS 5.4.1.3  \hfill https://www.vedavms.in \hfill}

\section{ TS 5.4.1.3 }

\textbf{TS 5.4.1.3 } \newline
\textbf{Samhita Paata} \newline

भ॑वति॒ य ए॒वं ॅवेद॒ ज्योति॑ष्मती॒रुप॑ दधाति॒ ज्योति॑रे॒वास्मि॑न् दधात्ये॒ताभि॒र्वा अ॒ग्निश्चि॒तो ज्व॑लति॒ ताभि॑रे॒वैनꣳ॒॒ समि॑न्ध उ॒भयो॑रस्मै लो॒कयो॒र्ज्योति॑र्भवति नक्षत्रेष्ट॒का उप॑ दधात्ये॒तानि॒ वै दि॒वो ज्योतीꣳ॑षि॒ तान्ये॒वाव॑ रुन्धे सु॒कृतां॒ ॅवा ए॒तानि॒ ज्योतीꣳ॑षि॒ यन्नक्ष॑त्राणि॒ तान्ये॒वाऽऽ*प्नो॒त्यथो॑ अनूका॒शमे॒वैतानि॒ - [  ] \newline

\textbf{Pada Paata} \newline

भ॒व॒ति॒ । यः । ए॒वम् । वेद॑ । ज्योति॑ष्मतीः । उपेति॑ । द॒धा॒ति॒ । ज्योतिः॑ । ए॒व । अ॒स्मि॒न्न् । द॒धा॒ति॒ । ए॒ताभिः॑ । वै । अ॒ग्निः । चि॒तः । ज्व॒ल॒ति॒ । ताभिः॑ । ए॒व । ए॒न॒म् । समिति॑ । इ॒न्धे॒ । उ॒भयोः᳚ । अ॒स्मै॒ । लो॒कयोः᳚ । ज्योतिः॑ । भ॒व॒ति॒ । न॒क्ष॒त्रे॒ष्ट॒का इति॑ नक्षत्र - इ॒ष्ट॒काः । उपेति॑ । द॒धा॒ति॒ । ए॒तानि॑ । वै । दि॒वः । ज्योतीꣳ॑षि । तानि॑ । ए॒व । अवेति॑ । रु॒न्धे॒ । सु॒कृता॒मिति॑ सु - कृता᳚म् । वै । ए॒तानि॑ । ज्योतीꣳ॑षि । यत् । नक्ष॑त्राणि । तानि॑ । ए॒व । आ॒प्नो॒ति॒ । अथो॒ इति॑ । अ॒नू॒का॒शमित्य॑नु - का॒शम् । ए॒व । ए॒तानि॑ ।  \newline




\markright{ TS 5.4.1.4  \hfill https://www.vedavms.in \hfill}

\section{ TS 5.4.1.4 }

\textbf{TS 5.4.1.4 } \newline
\textbf{Samhita Paata} \newline

ज्योतीꣳ॑षि कुरुते सुव॒र्गस्य॑ लो॒कस्यानु॑ख्यात्यै॒ यथ् सꣳस्पृ॑ष्टा उपद॒द्ध्याद्-वृष्ट्यै॑ लो॒कमपि॑ दद्ध्या॒दव॑र्.षुकः प॒र्जन्यः॑ स्या॒दसꣳ॑स्पृष्टा॒ उप॑ दधाति॒ वृष्ट्या॑ ए॒व लो॒कं क॑रोति॒ वर्.षु॑कः प॒र्जन्यो॑ भवति पु॒रस्ता॑द॒न्याः प्र॒तीची॒रुप॑ दधाति प॒श्चाद॒न्याः प्राची॒स्तस्मा᳚त् प्रा॒चीना॑नि च प्रती॒चीना॑नि च॒ नक्ष॑त्रा॒ण्या व॑र्तन्ते ॥ \newline

\textbf{Pada Paata} \newline

ज्योतीꣳ॑षि । कु॒रु॒ते॒ । सु॒व॒र्गस्येति॑ सुवः - गस्य॑ । लो॒कस्य॑ । अनु॑ख्यात्या॒ इत्यनु॑ - ख्या॒त्यै॒ । यत् । सꣳस्पृ॑ष्टा॒ इति॒ सं - स्पृ॒ष्टाः॒ । उ॒प॒द॒द्ध्यादित्यु॑प-द॒द्ध्यात् । वृष्ट्यै᳚ । लो॒कम् । अपीति॑ । द॒द्ध्या॒त् । अव॑र्.षुकः । प॒र्जन्यः॑ । स्या॒त् । असꣳ॑स्पृष्टा॒ इत्यसं᳚ - स्पृ॒ष्टाः॒ । उपेति॑ । द॒धा॒ति॒ । वृष्ट्यै᳚ । ए॒व । लो॒कम् । क॒रो॒ति॒ । वर्.षु॑कः । प॒र्जन्यः॑ । भ॒व॒ति॒ । पु॒रस्ता᳚त् । अ॒न्याः । प्र॒तीचीः᳚ । उपेति॑ । द॒धा॒ति॒ । प॒श्चात् । अ॒न्याः । प्राचीः᳚ । तस्मा᳚त् । प्रा॒चीना॑नि । च॒ । प्र॒ती॒चीना॑नि । च॒ । नक्ष॑त्राणि । एति॑ । व॒र्त॒न्ते॒ ॥  \newline




\markright{ TS 5.4.2.1  \hfill https://www.vedavms.in \hfill}

\section{ TS 5.4.2.1 }

\textbf{TS 5.4.2.1 } \newline
\textbf{Samhita Paata} \newline

ऋ॒त॒व्या॑ उप॑ दधात्यृतू॒नां क्लृप्त्यै᳚ द्व॒द्वंमुप॑ दधाति॒ तस्मा᳚द् द्व॒न्द्वमृ॒तवो ऽधृ॑तेव॒ वा ए॒षा यन्म॑द्ध्य॒मा चिति॑र॒न्तरि॑क्षमिव॒ वा ए॒षा द्व॒द्वंम॒न्यासु॒ चिती॒षूप॑ दधाति॒ चत॑स्रो॒ मद्ध्ये॒ धृत्या॑ अन्त॒श्श्लेष॑णं॒ ॅवा ए॒ताश्चिती॑नां॒ ॅयदृ॑त॒व्या॑ यदृ॑त॒व्या॑ उप॒दधा॑ति॒ चिती॑नां॒ ॅविधृ॑त्या॒ अव॑का॒मनूप॑ दधात्ये॒षा वा अ॒ग्नेर्योनिः॒ सयो॑नि - [  ] \newline

\textbf{Pada Paata} \newline

ऋ॒त॒व्याः᳚ । उपेति॑ । द॒धा॒ति॒ । ऋ॒तू॒नाम् । क्लृप्त्यै᳚ । द्व॒द्वंमिति॑ द्वं-द्वम् । उपेति॑ । द॒धा॒ति॒ । तस्मा᳚त् । द्व॒द्वंमिति॑ द्वं - द्वम् । ऋ॒तवः॑ । अधृ॑ता । इ॒व॒ । वै । ए॒षा । यत् । म॒द्ध्य॒मा । चितिः॑ । अ॒न्तरि॑क्षम् । इ॒व॒ । वै । ए॒षा । द्व॒द्वंमिति॑ द्वं - द्वम् । अ॒न्यासु॑ । चिती॑षु । उपेति॑ । द॒धा॒ति॒ । चत॑स्रः । मद्ध्ये᳚ । धृत्यै᳚ । अ॒न्त॒श्श्लेष॑ण॒मित्य॑न्तः - श्लेष॑णम् । वै । ए॒ताः । चिती॑नाम् । यत् । ऋ॒त॒व्याः᳚ । यत् । ऋ॒त॒व्याः᳚ । उ॒प॒दधा॒तीत्यु॑प - दधा॑ति । चिती॑नाम् । विधृ॑त्या॒ इति॒ वि - धृ॒त्यै॒ । अव॑काम् । अनु॑ । उपेति॑ । द॒धा॒ति॒ । ए॒षा । वै । अ॒ग्नेः । योनिः॑ । सयो॑नि॒मिति॒ स - यो॒नि॒म् ।  \newline




\markright{ TS 5.4.2.2  \hfill https://www.vedavms.in \hfill}

\section{ TS 5.4.2.2 }

\textbf{TS 5.4.2.2 } \newline
\textbf{Samhita Paata} \newline

-मे॒वाग्निं चि॑नुत उ॒वाच॑ ह वि॒श्वामि॒त्रो ऽद॒दिथ् स ब्रह्म॒णाऽ*न्नं॒ ॅयस्यै॒ता उ॑पधी॒यान्तै॒ य उ॑ चैना ए॒वं ॅवेद॒दिति॑ संॅवथ्स॒रो वा ए॒तं प्र॑ति॒ष्ठायै॑ नुदते॒ यो᳚ऽग्निं चि॒त्वा न प्र॑ति॒तिष्ठ॑ति॒ पञ्च॒ पूर्वा॒श्चित॑यो भव॒न्त्यथ॑ ष॒ष्ठीं चितिं॑ चिनुते॒ षड्वा ऋ॒तवः॑ संॅवथ्स॒र ऋ॒तुष्वे॒व सं॑ॅवथ्स॒रे प्रति॑तिष्ठत्ये॒ ता वा - [  ] \newline

\textbf{Pada Paata} \newline

ए॒व । अ॒ग्निम् । चि॒नु॒ते॒ । उ॒वाच॑ । ह॒ । वि॒श्वामि॑त्र॒ इति॑ वि॒श्व-मि॒त्रः॒ । अद॑त् । इत् । सः । ब्रह्म॑णा । अन्न᳚म् । यस्य॑ । ए॒ताः । उ॒प॒धी॒यान्ता॒ इत्यु॑प - धी॒यान्तै᳚ । यः । उ॒ । च॒ । ए॒नाः॒ । ए॒वम् । वेद॑त् । इति॑ । सं॒ॅव॒थ्स॒र इति॑ सं - व॒थ्स॒रः । वै । ए॒तम् । प्र॒ति॒ष्ठाया॒ इति॑ प्रति - स्थायै᳚ । नु॒द॒ते॒ । यः । अ॒ग्निम् । चि॒त्वा । न । प्र॒ति॒तिष्ठ॒तीति॑ प्रति - तिष्ठ॑ति । पञ्च॑ । पूर्वाः᳚ । चित॑यः । भ॒व॒न्ति॒ । अथ॑ । ष॒ष्ठीम् । चिति᳚म् । चि॒नु॒ते॒ । षट् । वै । ऋ॒तवः॑ । सं॒ॅव॒थ्स॒र इति॑ सं-व॒थ्स॒रः । ऋ॒तुषु॑ । ए॒व । सं॒ॅव॒थ्स॒र इति॑ सं - व॒थ्स॒रे । प्रतीति॑ । ति॒ष्ठ॒ति॒ । ए॒ताः । वै ।  \newline




\markright{ TS 5.4.2.3  \hfill https://www.vedavms.in \hfill}

\section{ TS 5.4.2.3 }

\textbf{TS 5.4.2.3 } \newline
\textbf{Samhita Paata} \newline

अधि॑पत्नी॒र्नामेष्ट॑का॒ यस्यै॒ता उ॑पधी॒यन्तेऽधि॑पतिरे॒व स॑मा॒नानां᳚ भवति॒ यं द्वि॒ष्यात् तमु॑प॒दध॑द् ध्यायेदे॒ताभ्य॑ ए॒वैनं॑ देवता᳚भ्य॒ आ वृ॑श्चति ता॒जगार्ति॒मार्च्छ॒त्यङ्गि॑रसः सुव॒र्गं ॅलो॒कं ॅयन्तो॒ या य॒ज्ञ्स्य॒ निष्कृ॑ति॒रासी॒त् तामृषि॑भ्यः॒ प्रत्यौ॑ह॒न् तद्धिर॑ण्यमभव॒द्य-द्धि॑रण्यश॒ल्कैः प्रो॒क्षति॑ य॒ज्ञ्स्य॒ निष्कृ॑त्या॒ अथो॑ भेष॒जमे॒वास्मै॑ करो॒त्य - [  ] \newline

\textbf{Pada Paata} \newline

अधि॑पत्नी॒रित्यधि॑ - प॒त्नीः॒ । नाम॑ । इष्ट॑काः । यस्य॑ । ए॒ताः । उ॒प॒धी॒यन्त॒ इत्यु॑प - धी॒यन्ते᳚ । अधि॑पति॒रित्यधि॑ - प॒तिः॒ । ए॒व । स॒मा॒नाना᳚म् । भ॒व॒ति॒ । यम् । द्वि॒ष्यात् । तम् । उ॒प॒दध॒दित्यु॑प-दध॑त् । ध्या॒ये॒त् । ए॒ताभ्यः॑ । ए॒व । ए॒न॒म् । दे॒वता᳚भ्यः । एति॑ । वृ॒श्च॒ति॒ । ता॒जक् । आर्ति᳚म् । एति॑ । ऋ॒च्छ॒ति॒ । अङ्गि॑रसः । सु॒व॒र्गमिति॑ सुवः - गम् । लो॒कम् । यन्तः॑ । या । य॒ज्ञ्स्य॑ । निष्कृ॑ति॒रिति॒ निः-कृ॒तिः॒ । आसी᳚त् । ताम् । ऋषि॑भ्य॒ इत्यृषि॑-भ्यः॒ । प्रतीति॑ । औ॒ह॒न्न् । तत् । हिर॑ण्यम् । अ॒भ॒व॒त् । यत् । हि॒र॒ण्य॒श॒ल्कैरिति॑ हिरण्य - श॒ल्कैः । प्रो॒क्षतीति॑ प्र - उ॒क्षति॑ । य॒ज्ञ्स्य॑ । निष्कृ॑त्या॒ इति॒ निः - कृ॒त्यै॒ । अथो॒ इति॑ । भे॒ष॒जम् । ए॒व । अ॒स्मै॒ । क॒रो॒ति॒ ।  \newline




\markright{ TS 5.4.2.4  \hfill https://www.vedavms.in \hfill}

\section{ TS 5.4.2.4 }

\textbf{TS 5.4.2.4 } \newline
\textbf{Samhita Paata} \newline

-थो॑ रू॒पेणै॒वैनꣳ॒॒ सम॑र्द्धय॒त्यथो॒ हिर॑ण्यज्योतिषै॒व सु॑व॒र्गं ॅलो॒कमे॑ति साह॒स्रव॑ता॒ प्रोक्ष॑ति साह॒स्रः प्र॒जाप॑तिः प्र॒जाप॑ते॒राप्त्या॑ इ॒मा मे॑ अग्न॒ इष्ट॑का धे॒नवः॑ स॒न्त्वित्या॑ह धे॒नूरे॒वैनाः᳚ कुरुते॒ ता ए॑नं काम॒दुघा॑ अ॒मुत्रा॒मुष्मि॑न् ॅलो॒क उप॑ तिष्ठन्ते ॥ \newline

\textbf{Pada Paata} \newline

अथो॒ इति॑ । रू॒पेण॑ । ए॒व । ए॒न॒म् । समिति॑ । अ॒द्‌र्ध॒य॒ति॒ । अथो॒ इति॑ । हिर॑ण्यज्योति॒षेति॒ हिर॑ण्य - ज्यो॒ति॒षा॒ । ए॒व । सु॒व॒र्गमिति॑ सुवः - गम् । लो॒कम् । ए॒ति॒ । सा॒ह॒स्रव॒तेति॑ साह॒स्र - व॒ता॒ । प्रेति॑ । उ॒क्ष॒ति॒ । सा॒ह॒स्रः । प्र॒जाप॑ति॒रिति॑ प्र॒जा - प॒तिः॒ । प्र॒जाप॑ते॒रिति॑ प्र॒जा - प॒तेः॒ । आप्त्यै᳚ । इ॒माः । मे॒ । अ॒ग्ने॒ । इष्ट॑काः । धे॒नवः॑ । स॒न्तु॒ । इति॑ । आ॒ह॒ । धे॒नूः । ए॒व । ए॒नाः॒ । कु॒रु॒ते॒ । ताः । ए॒न॒म् । का॒म॒दुघा॒ इति॑ काम - दुघाः᳚ । अ॒मुत्र॑ । अ॒मुष्मिन्न्॑ । लो॒के । उपेति॑ । ति॒ष्ठ॒न्ते॒ ॥  \newline




\markright{ TS 5.4.3.1  \hfill https://www.vedavms.in \hfill}

\section{ TS 5.4.3.1 }

\textbf{TS 5.4.3.1 } \newline
\textbf{Samhita Paata} \newline

रु॒द्रो वा ए॒ष यद॒ग्निः स ए॒तर्.हि॑ जा॒तो यर्.हि॒ सर्व॑श्चि॒तः स यथा॑ व॒थ्सो जा॒तः स्तनं॑ प्रे॒फ्सत्ये॒वं ॅवा ए॒ष ए॒तर्.हि॑ भाग॒धेयं॒ प्रेफ्स॑ति॒ तस्मै॒ यदाहु॑तिं॒ न जु॑हु॒याद॑द्ध्व॒र्युं च॒ यज॑मानं च ध्यायेच्छतरु॒द्रीयं॑ जुहोति भाग॒धेये॑नै॒वैनꣳ॑ शमयति॒ नाऽऽ*र्ति॒मार्च्छ॑त्यद्ध्व॒र्युर्न यज॑मानो॒ यद् ग्रा॒म्याणां᳚ पशू॒नां - [  ] \newline

\textbf{Pada Paata} \newline

रु॒द्रः । वै । ए॒षः । यत् । अ॒ग्निः । सः । ए॒तर्.हि॑ । जा॒तः । यर्.हि॑ । सर्वः॑ । चि॒तः । सः । यथा᳚ । व॒थ्सः । जा॒तः । स्तन᳚म् । प्रे॒फ्सतीति॑ प्र - ई॒फ्सति॑ । ए॒वम् । वै । ए॒षः । ए॒तर्.हि॑ । भा॒ग॒धेय॒मिति॑ भाग-धेय᳚म् । प्रेति॑ । ई॒फ्स॒ति॒ । तस्मै᳚ । यत् । आहु॑ति॒मित्या-हु॒ति॒म् । न । जु॒हु॒यात् । अ॒द्ध्व॒र्युम् । च॒ । यज॑मानम् । च॒ । ध्या॒ये॒त् । श॒त॒रु॒द्रीय॒मिति॑ शत - रु॒द्रीय᳚म् । जु॒हो॒ति॒ । भा॒ग॒धेये॒नेति॑ भाग - धेये॑न । ए॒व । ए॒न॒म् । श॒म॒य॒ति॒ । न । आर्ति᳚म् । एति॑ । ऋ॒च्छ॒ति॒ । अ॒द्ध्व॒र्युः । न । यज॑मानः । यत् । ग्रा॒म्याणा᳚म् । प॒शू॒नाम् ।  \newline




\markright{ TS 5.4.3.2  \hfill https://www.vedavms.in \hfill}

\section{ TS 5.4.3.2 }

\textbf{TS 5.4.3.2 } \newline
\textbf{Samhita Paata} \newline

पय॑सा जुहु॒याद् ग्रा॒म्यान् प॒शूञ्छु॒चा ऽर्पये॒द्-यदा॑र॒ण्याना॑-मार॒ण्यान् ज॑र्तिलयवा॒ग्वा॑ वा जुहु॒याद् ग॑वीधुकयवा॒ग्वा॑ वा॒ न ग्रा॒म्यान् प॒शून्. हि॒नस्ति॒ नाऽऽ*र॒ण्यानथो॒ खल्वा॑हु॒रना॑हुति॒र्वै ज॒र्तिला᳚श्च ग॒वीधु॑का॒श्चेत्य॑ जक्षी॒रेण॑ जुहोत्याग्ने॒यी वा ए॒षा यद॒जाऽऽहु॑त्यै॒व जु॑होति॒ न ग्रा॒म्यान् प॒शून्. हि॒नस्ति॒ नाऽऽ*र॒ण्यानङ्गि॑रसः सुव॒र्गं ॅलो॒कं ॅयन्तो॒ - [  ] \newline

\textbf{Pada Paata} \newline

पय॑सा । जु॒हु॒यात् । ग्रा॒म्यान् । प॒शून् । शु॒चा । अ॒र्प॒ये॒त् । यत् । आ॒र॒ण्याना᳚म् । आ॒र॒ण्यान् । ज॒र्ति॒ल॒य॒वा॒ग्वेति॑ जर्तिल - य॒वा॒ग्वा᳚ । वा॒ । जु॒हु॒यात् । ग॒वी॒धु॒क॒य॒वा॒ग्वेति॑ गवीधुक - य॒वा॒ग्वा᳚ । वा॒ । न । ग्रा॒म्यान् । प॒शून् । हि॒नस्ति॑ । न । आ॒र॒ण्यान् । अथो॒ इति॑ । खलु॑ । आ॒हुः॒ । अना॑हुति॒रित्यना᳚-हु॒तिः॒ । वै । ज॒र्तिलाः᳚ । च॒ । ग॒वीधु॑काः । च॒ । इति॑ । अ॒ज॒क्षी॒रेणेत्य॑ज - क्षी॒रेण॑ । जु॒हो॒ति॒ । आ॒ग्ने॒यी । वै । ए॒षा । यत् । अ॒जा । आहु॒त्येत्या - हु॒त्या॒ । ए॒व । जु॒हो॒ति॒ । न । ग्रा॒म्यान् । प॒शून् । हि॒नस्ति॑ । न । आ॒र॒ण्यान् । अङ्गि॑रसः । सु॒व॒र्गमिति॑ सुवः - गम् । लो॒कम् । यन्तः॑ ।  \newline




\markright{ TS 5.4.3.3  \hfill https://www.vedavms.in \hfill}

\section{ TS 5.4.3.3 }

\textbf{TS 5.4.3.3 } \newline
\textbf{Samhita Paata} \newline

-ऽजायां᳚ घ॒र्मं प्रासि॑ञ्च॒न्थ्सा शोच॑न्ती प॒र्णं परा॑ऽजिहीत॒ सो᳚(1॒)ऽर्को॑ऽभव॒त् तद॒र्कस्या᳚-र्क॒त्वम॑र्कप॒र्णेन॑ जुहोति सयोनि॒त्वायोद॒ङ् तिष्ठ॑न् जुहोत्ये॒षा वै रु॒द्रस्य॒ दिख् स्वाया॑मे॒व दि॒शि रु॒द्रं नि॒रव॑दयते चर॒माया॒मिष्ट॑कायां जुहोत्यन्त॒त ए॒व रु॒द्रं नि॒रव॑दयते त्रेधाविभ॒क्तं जु॑होति॒ त्रय॑ इ॒मे लो॒का इ॒माने॒व लो॒कान्थ् स॒माव॑द्वीर्यान् करो॒तीय॒त्यग्रे॑ जुहो॒त्य - [  ] \newline

\textbf{Pada Paata} \newline

अ॒जाया᳚म् । घ॒र्मम् । प्रेति॑ । अ॒सि॒ञ्च॒न्न् । सा । शोच॑न्ती । प॒र्णम् । परेति॑ । अ॒जि॒ही॒त॒ । सः । अ॒र्कः । अ॒भ॒व॒त् । तत् । अ॒र्कस्य॑ । अ॒र्क॒त्वमित्य॑र्क - त्वम् । अ॒र्क॒प॒र्णेनेत्य॑र्क - प॒र्णेन॑ । जु॒हो॒ति॒ । स॒यो॒नि॒त्वायेति॑ सयोनि - त्वाय॑ । उदङ्॑ । तिष्ठन्न्॑ । जु॒हो॒ति॒ । ए॒षा । वै । रु॒द्रस्य॑ । दिक् । स्वाया᳚म् । ए॒व । दि॒शि । रु॒द्रम् । नि॒रव॑दयत॒ इति॑ निः-अव॑दयते । च॒र॒माया᳚म् । इष्ट॑कायाम् । जु॒हो॒ति॒ । अ॒न्त॒तः । ए॒व । रु॒द्रम् । नि॒रव॑दयत॒ इति॑ निः - अव॑दयते । त्रे॒धा॒वि॒भ॒क्तमिति॑ त्रेधा - वि॒भ॒क्तम् । जु॒हो॒ति॒ । त्रयः॑ । इ॒मे । लो॒काः । इ॒मान् । ए॒व । लो॒कान् । स॒माव॑द्वीर्या॒निति॑ स॒माव॑त् - वी॒र्या॒न् । क॒रो॒ति॒ । इय॑ति । अग्रे᳚ । जु॒हो॒ति॒ ।  \newline




\markright{ TS 5.4.3.4  \hfill https://www.vedavms.in \hfill}

\section{ TS 5.4.3.4 }

\textbf{TS 5.4.3.4 } \newline
\textbf{Samhita Paata} \newline

-थेय॒त्यथेय॑ति॒ त्रय॑ इ॒मे लो॒का ए॒भ्य ए॒वैनं॑ ॅलो॒केभ्यः॑ शमयति ति॒स्र उत्त॑रा॒ आहु॑तीर्जुहोति॒ षट् थ्सं प॑द्यन्ते॒ षड् वा ऋ॒तव॑ ऋ॒तुभि॑रे॒वैनꣳ॑ शमयति॒ यद॑नुपरि॒क्रामं॑ जुहु॒याद॑न्तरवचा॒रिणꣳ॑ रु॒द्रं कु॑र्या॒दथो॒ खल्वा॑हुः॒ कस्यां॒ ॅवाऽह॑ दि॒शि रु॒द्रः कस्यां॒ ॅवेत्य॑नुपरि॒क्राम॑मे॒व हो॑त॒व्य॑-मप॑रिवर्गमे॒वैनꣳ॑ शमयत्ये॒ - [  ] \newline

\textbf{Pada Paata} \newline

अथ॑ । इय॑ति । अथ॑ । इय॑ति । त्रयः॑ । इ॒मे । लो॒काः । ए॒भ्यः । ए॒व । ए॒न॒म् । लो॒केभ्यः॑ । श॒म॒य॒ति॒ । ति॒स्रः । उत्त॑रा॒ इत्युत् - त॒राः॒ । आहु॑ती॒रित्या-हु॒तीः॒ । जु॒हो॒ति॒ । षट् । समिति॑ । प॒द्य॒न्ते॒ । षट् । वै । ऋ॒तवः॑ । ऋ॒तुभि॒रित्यृ॒तु - भिः॒ । ए॒व । ए॒न॒म् । श॒म॒य॒ति॒ । यत् । अ॒नु॒प॒रि॒क्राम॒मित्य॑नु - प॒रि॒क्राम᳚म् । जु॒हु॒यात् । अ॒न्त॒र॒व॒चा॒रिण॒मित्य॑न्तः - अ॒व॒चा॒रिण᳚म् । रु॒द्रम् । कु॒र्या॒त् । अथो॒ इति॑ । खलु॑ । आ॒हुः॒ । कस्या᳚म् । वा॒ । अह॑ । दि॒शि । रु॒द्रः । कस्या᳚म् । वा॒ । इति॑ । अ॒नु॒प॒रि॒क्राम॒मित्य॑नु - प॒रि॒क्राम᳚म् । ए॒व । हो॒त॒व्य᳚म् । अप॑रिवर्ग॒मित्यप॑रि - व॒र्ग॒म् । ए॒व । ए॒न॒म् । श॒म॒य॒ति॒ ।  \newline




\markright{ TS 5.4.3.5  \hfill https://www.vedavms.in \hfill}

\section{ TS 5.4.3.5 }

\textbf{TS 5.4.3.5 } \newline
\textbf{Samhita Paata} \newline

-तावै दे॒वताः᳚ सुव॒र्ग्या॑या उ॑त्त॒मास्ता यज॑मानं ॅवाचयति॒ ताभि॑रे॒वैनꣳ॑ सुव॒र्गं ॅलो॒कं ग॑मयति॒ यं द्वि॒ष्यात् तस्य॑ सञ्च॒रे प॑शू॒नां न्य॑स्ये॒द् यः प्र॑थ॒मः प॒शुर॑भि॒तिष्ठ॑ति॒ स आर्ति॒मार्च्छ॑ति ॥ \newline

\textbf{Pada Paata} \newline

ए॒ताः । वै । दे॒वताः᳚ । सु॒व॒र्ग्या॑ इति॑ सुवः - ग्याः᳚ । याः । उ॒त्त॒मा इत्यु॑त्-त॒माः । ताः । यज॑मानम् । वा॒च॒य॒ति॒ । ताभिः॑ । ए॒व । ए॒न॒म् । सु॒व॒र्गमिति॑ सुवः-गम् । लो॒कम् । ग॒म॒य॒ति॒ । यम् । द्वि॒ष्यात् । तस्य॑ । सं॒च॒र इति॑ सं - च॒रे । प॒शू॒नाम् । नीति॑ । अ॒स्ये॒त् । यः । प्र॒थ॒मः । प॒शुः । अ॒भि॒तिष्ठ॒तीत्य॑भि - तिष्ठ॑ति । सः । आर्ति᳚म् । एति॑ । ऋ॒च्छ॒ति॒ ॥  \newline




\markright{ TS 5.4.4.1  \hfill https://www.vedavms.in \hfill}

\section{ TS 5.4.4.1 }

\textbf{TS 5.4.4.1 } \newline
\textbf{Samhita Paata} \newline

अश्म॒न्नूर्ज॒मिति॒ परि॑ षिञ्चति मा॒र्जय॑त्ये॒वैन॒मथो॑ त॒र्पय॑त्ये॒व स ए॑नं तृ॒प्तो ऽक्षु॑द्ध्य॒-न्नशो॑च-न्न॒मुष्मि॑न् ॅलो॒क उप॑ तिष्ठते॒ तृप्य॑ति प्र॒जया॑ प॒शुभि॒र्य ए॒वं ॅवेद॒ तां न॒ इष॒मूर्जं॑ धत्त मरुतः सꣳररा॒णा इत्या॒हान्नं॒ ॅवा ऊर्गन्नं॑ म॒रुतोऽन्न॑मे॒वाव॑ रु॒न्धे ऽश्मꣳ॑स्ते॒ क्षुद॒मुं ते॒ शु - [  ] \newline

\textbf{Pada Paata} \newline

अश्मन्न्॑ । ऊर्ज᳚म् । इति॑ । परीति॑ । सि॒ञ्च॒ति॒ । मा॒र्जय॑ति । ए॒व । ए॒न॒म् । अथो॒ इति॑ । त॒र्पय॑ति । ए॒व । सः । ए॒न॒म् । तृ॒प्तः । अक्षु॑द्ध्यन्न् । अशो॑चन्न् । अ॒मुष्मिन्न्॑ । लो॒के । उपेति॑ । ति॒ष्ठ॒ते॒ । तृप्य॑ति । प्र॒जयेति॑ प्र - जया᳚ । प॒शुभि॒रिति॑ प॒शु - भिः॒ । यः । ए॒वम् । वेद॑ । ताम् । नः॒ । इष᳚म् । ऊर्ज᳚म् । ध॒त्त॒ । म॒रु॒तः॒ । सꣳ॒॒र॒रा॒णा इति॑ सं - र॒रा॒णाः । इति॑ । आ॒ह॒ । अन्न᳚म् । वै । ऊर्क् । अन्न᳚म् । म॒रुतः॑ । अन्न᳚म् । ए॒व । अवेति॑ । रु॒न्धे॒ । अश्मन्न्॑ । ते॒ । क्षुत् । अ॒मुम् । ते॒ । शुक् ।  \newline




\markright{ TS 5.4.4.2  \hfill https://www.vedavms.in \hfill}

\section{ TS 5.4.4.2 }

\textbf{TS 5.4.4.2 } \newline
\textbf{Samhita Paata} \newline

-गृ॑च्छतु॒ यं द्वि॒ष्म इत्या॑ह॒ यमे॒व द्वेष्टि॒ तम॑स्य क्षु॒धा च॑ शु॒चा चा᳚र्पयति॒ त्रिः प॑रिषि॒ञ्चन् पर्ये॑ति त्रि॒वृद्वा अ॒ग्निर्यावा॑ने॒वा-ग्निस्तस्य॒ शुचꣳ॑ शमयति॒ त्रिः पुनः॒ पर्ये॑ति॒ षट् थ्सं प॑द्यन्ते॒ षड् वा ऋ॒तव॑ ऋ॒तुभि॑रे॒वास्य॒ शुचꣳ॑ शमयत्य॒पां ॅवा ए॒तत् पुष्पं॒ ॅयद्वे॑त॒सो॑ऽपाꣳ - [  ] \newline

\textbf{Pada Paata} \newline

ऋ॒च्छ॒तु॒ । यम् । द्वि॒ष्मः । इति॑ । आ॒ह॒ । यम् । ए॒व । द्वेष्टि॑ । तम् । अ॒स्य॒ । क्षु॒धा । च॒ । शु॒चा । च॒ । अ॒र्प॒य॒ति॒ । त्रिः । प॒रि॒षि॒ञ्चन्निति॑ परि - सि॒ञ्चन्न् । परीति॑ । ए॒ति॒ । त्रि॒वृदिति॑ त्रि-वृत् । वै । अ॒ग्निः । यावान्॑ । ए॒व । अ॒ग्निः । तस्य॑ । शुच᳚म् । श॒म॒य॒ति॒ । त्रिः । पुनः॑ । परीति॑ । ए॒ति॒ । षट् । समिति॑ । प॒द्य॒न्ते॒ । षट् । वै । ऋ॒तवः॑ । ऋ॒तुभि॒रित्यृ॒तु-भिः॒। ए॒व । अ॒स्य॒ । शुच᳚म् । श॒म॒य॒ति॒ । अ॒पाम् । वै । ए॒तत् । पुष्प᳚म् । यत् । वे॒त॒सः । अ॒पाम् ।  \newline




\markright{ TS 5.4.4.3  \hfill https://www.vedavms.in \hfill}

\section{ TS 5.4.4.3 }

\textbf{TS 5.4.4.3 } \newline
\textbf{Samhita Paata} \newline

शरोऽव॑का वेतसशा॒खया॒ चाव॑काभिश्च॒ वि क॑र्.ष॒त्यापो॒ वै शा॒न्ताः शा॒न्ताभि॑रे॒वास्य॒ शुचꣳ॑ शमयति॒ यो वा अ॒ग्निं चि॒तं प्र॑थ॒मः प॒शुर॑धि॒क्राम॑तीश्व॒रो वै तꣳ शु॒चा प्र॒दहो॑ म॒ण्डूके॑न॒ विक॑र्.षत्ये॒ष वै प॑शू॒ना-म॑नुपजीवनी॒यो न वा ए॒ष ग्रा॒म्येषु॑ प॒शुषु॑ हि॒तो नाऽऽ*र॒ण्येषु॒ तमे॒व शु॒चाऽर्प॑यत्यष्टा॒भिर्वि क॑र्.षत्य॒ - [  ] \newline

\textbf{Pada Paata} \newline

शरः॑ । अव॑काः । वे॒त॒स॒शा॒खयेति॑ वेतस - शा॒खया᳚ । च॒ । अव॑काभिः । च॒ । वीति॑ । क॒र्.॒ष॒ति॒ । आपः॑ । वै । शा॒न्ताः । शा॒न्ताभिः॑ । ए॒व । अ॒स्य॒ । शुच᳚म् । श॒म॒य॒ति॒ । यः । वै । अ॒ग्निम् । चि॒तम् । प्र॒थ॒मः । प॒शुः । अ॒धि॒क्राम॒तीत्य॑धि-क्राम॑ति । ई॒श्व॒रः । वै । तम् । शु॒चा । प्र॒दह॒ इति॑ प्र - दहः॑ । म॒ण्डूके॑न । वीति॑ । क॒र्.॒ष॒ति॒ । ए॒षः । वै । प॒शू॒नाम् । अ॒नु॒प॒जी॒व॒नी॒य इत्य॑नुप -  जी॒व॒नी॒यः । न । वै । ए॒षः । ग्रा॒म्येषु॑ । प॒शुषु॑ । हि॒तः । न । आ॒र॒ण्येषु॑ । तम् । ए॒व । शु॒चा । अ॒र्प॒य॒ति॒ । अ॒ष्टा॒भिः । वीति॑ । क॒र्.॒ष॒ति॒ ।  \newline




\markright{ TS 5.4.4.4  \hfill https://www.vedavms.in \hfill}

\section{ TS 5.4.4.4 }

\textbf{TS 5.4.4.4 } \newline
\textbf{Samhita Paata} \newline

-ष्टाक्ष॑रा गाय॒त्री गा॑य॒त्रो᳚ऽग्निर्यावा॑-ने॒वाऽग्निस्तस्य॒ शुचꣳ॑ शमयति पाव॒कव॑तीभि॒रन्नं॒ ॅवै पा॑व॒कोऽन्ने॑नै॒वास्य॒ शुचꣳ॑ शमयति मृ॒त्युर्वा ए॒ष यद॒ग्निर्ब्रह्म॑ण ए॒तद्रू॒पं ॅयत् कृ॑ष्णाजि॒नं कार्ष्णी॑ उपा॒नहा॒वुप॑ मुञ्चते॒ ब्रह्म॑णै॒व मृ॒त्योर॒न्तर्द्ध॑त्ते॒ ऽन्तर्मृ॒त्योर्द्ध॑त्ते॒ ऽन्तर॒न्नाद्या॒-दित्या॑हुर॒न्या-मु॑पमु॒ञ्चते॒ऽन्यां नान्त - [  ] \newline

\textbf{Pada Paata} \newline

अ॒ष्टाक्ष॒रेत्य॒ष्टा-अ॒क्ष॒रा॒ । गा॒य॒त्री । गा॒य॒त्रः । अ॒ग्निः । यावान्॑ । ए॒व । अ॒ग्निः । तस्य॑ । शुच᳚म् । श॒म॒य॒ति॒ । पा॒व॒कव॑तीभि॒रिति॑ पाव॒क - व॒ती॒भिः॒ । अन्न᳚म् । वै । पा॒व॒कः । अन्ने॑न । ए॒व । अ॒स्य॒ । शुच᳚म् । श॒म॒य॒ति॒ । मृ॒त्युः । वै ।   ए॒षः । यत् । अ॒ग्निः । ब्रह्म॑णः । ए॒तत् । रू॒पम् । यत् । कृ॒ष्णा॒जि॒नमिति॑ कृष्ण - अ॒जि॒नम् । कार्ष्णी॒ इति॑ । उ॒पा॒नहौ᳚ । उपेति॑ । मु॒ञ्च॒ते॒ । ब्रह्म॑णा । ए॒व । मृ॒त्योः । अ॒न्तः । ध॒त्ते॒ । अ॒न्तः । मृ॒त्योः । ध॒त्ते॒ । अ॒न्तः । अ॒न्नाद्या॒दित्य॑न्न - अद्या᳚त् । इति॑ । आ॒हुः॒ । अ॒न्याम् । उ॒प॒मु॒ञ्चत॒ इत्यु॑प - मु॒ञ्चते᳚ । अ॒न्याम् । न । अ॒न्तः ।  \newline




\markright{ TS 5.4.4.5  \hfill https://www.vedavms.in \hfill}

\section{ TS 5.4.4.5 }

\textbf{TS 5.4.4.5 } \newline
\textbf{Samhita Paata} \newline

-रे॒व मृ॒त्योर्द्ध॒त्ते ऽवा॒ऽन्नाद्यꣳ॑ रुन्धे॒ नम॑स्ते॒ हर॑से शो॒चिष॒ इत्या॑ह नम॒स्कृत्य॒ हि वसी॑याꣳ समुप॒चर॑न्त्य॒न्यं ते॑ अ॒स्मत् त॑पन्तु हे॒तय॒ इत्या॑ह॒ यमे॒व द्वेष्टि॒ तम॑स्य शु॒चाऽर्प॑यति पाव॒को अ॒स्मभ्यꣳ॑ शि॒वो भ॒वेत्या॒हान्नं॒ ॅवै पा॑व॒कोऽन्न॑मे॒वाव॑ रुन्धे॒ द्वाभ्या॒मधि॑ क्रामति॒ प्रति॑ष्ठित्या अप॒स्य॑वतीभ्याꣳ॒॒ शान्त्यै᳚ ॥ \newline

\textbf{Pada Paata} \newline

ए॒व । मृ॒त्योः । ध॒त्ते॒ । अवेति॑ । अ॒न्नाद्य॒मित्य॑न्न - अद्य᳚म् । रु॒न्धे॒ । नमः॑ । ते॒ । हर॑से । शो॒चिषे᳚ । इति॑ । आ॒ह॒ । न॒म॒स्कृत्येति॑ नमः - कृत्य॑ । हि । वसी॑याꣳसम् । उ॒प॒चर॒न्तीत्यु॑प - चर॑न्ति । अ॒न्यम् । ते॒ । अ॒स्मत् । त॒प॒न्तु॒ । हे॒तयः॑ । इति॑ । आ॒ह॒ । यम् । ए॒व । द्वेष्टि॑ । तम् । अ॒स्य॒ । शु॒चा । अ॒र्प॒य॒ति॒ । पा॒व॒कः । अ॒स्मभ्य॒मित्य॒स्म - भ्य॒म् । शि॒वः । भ॒व॒ । इति॑ । आ॒ह॒ । अन्न᳚म् । वै । पा॒व॒कः । अन्न᳚म् । ए॒व । अवेति॑ । रु॒न्धे॒ । द्वाभ्या᳚म् । अधीति॑ । क्रा॒म॒ति॒ । प्रति॑ष्ठित्या॒ इति॒ प्रति॑ - स्थि॒त्यै॒ । अ॒प॒स्य॑वतीभ्या॒मित्य॑प॒स्य॑ - व॒ती॒भ्या॒म् । शान्त्यै᳚ ॥  \newline




\markright{ TS 5.4.5.1  \hfill https://www.vedavms.in \hfill}

\section{ TS 5.4.5.1 }

\textbf{TS 5.4.5.1 } \newline
\textbf{Samhita Paata} \newline

नृ॒षदे॒ वडिति॒ व्याघा॑रयति प॒ङ्क्त्याऽऽहु॑त्या यज्ञ्मु॒खमा र॑भते ऽक्ष्ण॒या व्याघा॑रयति॒ तस्मा॑दक्ष्ण॒या प॒शवोऽङ्गा॑नि॒ प्रह॑रन्ति॒ प्रति॑ष्ठित्यै॒ यद्व॑षट्कु॒र्याद्-या॒तया॑माऽस्य वषट्का॒रः स्या॒द्यन्न व॑षट्कु॒र्याद्-रक्षाꣳ॑सि य॒ज्ञ्ꣳ ह॑न्यु॒र्वडित्या॑ह प॒रोक्ष॑मे॒व वष॑ट् करोति॒ नास्य॑ या॒तया॑मा वषट्का॒रो भव॑ति॒ न य॒ज्ञ्ꣳ रक्षाꣳ॑सि घ्नन्ति हु॒तादो॒ वा अ॒न्ये दे॒वा - [  ] \newline

\textbf{Pada Paata} \newline

नृ॒षद॒ इति॑ नृ - सदे᳚ । वट् । इति॑ । व्याघा॑रय॒तीति॑ वि-आघा॑रयति । प॒ङ्क्त्या । आहु॒त्येत्या-हु॒त्या॒ । य॒ज्ञ्॒मु॒खमिति॑ यज्ञ् - मु॒खम् । एति॑ । र॒भ॒ते॒ । अ॒क्ष्ण॒या । व्याघा॑रय॒तीति॑ वि - आघा॑रयति । तस्मा᳚त् । अ॒क्ष्ण॒या । प॒शवः॑ । अङ्गा॑नि । प्रेति॑ । ह॒र॒न्ति॒ । प्रति॑ष्ठित्या॒ इति॒ प्रति॑ - स्थि॒त्यै॒ । यत् । व॒ष॒ट्कु॒र्यादिति॑ वषट् - कु॒र्यात् । या॒तया॒मेति॑ या॒त - या॒मा॒ । अ॒स्य॒ । व॒ष॒ट्का॒र इति॑ वषट्-का॒रः । स्या॒त् । यत् । न । व॒ष॒ट्कु॒र्यादिति॑ वषट् - कु॒र्यात् । रक्षाꣳ॑सि । य॒ज्ञ्म् । ह॒न्युः॒ । वट् । इति॑ । आ॒ह॒ । प॒रोक्ष॒मिति॑ परः - अक्ष᳚म् । ए॒व । वष॑ट् । क॒रो॒ति॒ । न । अ॒स्य॒ । या॒तया॒मेति॑ या॒त - या॒मा॒ । व॒ष॒ट्का॒र इति॑ वषट् - का॒रः । भव॑ति । न । य॒ज्ञ्म् । रक्षाꣳ॑सि । घ्न॒न्ति॒ । हु॒ताद॒ इति॑ हुत - अदः॑ । वै । अ॒न्ये । दे॒वाः ।  \newline




\markright{ TS 5.4.5.2  \hfill https://www.vedavms.in \hfill}

\section{ TS 5.4.5.2 }

\textbf{TS 5.4.5.2 } \newline
\textbf{Samhita Paata} \newline

अ॑हु॒तादो॒ऽन्ये तान॑ग्नि॒चिदे॒वोभया᳚न् प्रीणाति॒ ये दे॒वा दे॒वाना॒मिति॑ द॒द्ध्ना म॑धुमि॒श्रेणावो᳚क्षति हु॒ताद॑श्चै॒व दे॒वान॑हु॒ताद॑श्च॒ यज॑मानः प्रीणाति॒ ते यज॑मानं प्रीणन्ति द॒द्ध्नैव हु॒तादः॑ प्री॒णाति॒ मधु॑षा ऽहु॒तादो᳚ ग्रा॒म्यं ॅवा ए॒तदन्नं॒ ॅयद्दद्ध्या॑र॒ण्यं मधु॒ यद्द॒ध्ना म॑धुमि॒श्रेणा॒-वोक्ष॑त्यु॒भय॒स्याऽव॑रुद्ध्यै ग्रुमु॒ष्टिनाऽवो᳚क्षति प्राजाप॒त्यो - [  ] \newline

\textbf{Pada Paata} \newline

अ॒हु॒ताद॒ इत्य॑हुत - अदः॑ । अ॒न्ये । तान् । अ॒ग्नि॒चिदित्य॑ग्नि-चित् । ए॒व । उ॒भयान्॑ । प्री॒णा॒ति॒ । ये । दे॒वाः । दे॒वाना᳚म् । इति॑ । द॒द्ध्ना । म॒धु॒मि॒श्रेणेति॑ मधु - मि॒श्रेण॑ । अवेति॑ । उ॒क्ष॒ति॒ । हु॒ताद॒ इति॑ हुत - अदः॑ । च॒ । ए॒व । दे॒वान् । अ॒हु॒ताद॒ इत्य॑हुत - अदः॑ । च॒ । यज॑मानः । प्री॒णा॒ति॒ । ते । यज॑मानम् । प्री॒ण॒न्ति॒ । द॒द्ध्ना । ए॒व । हु॒ताद॒ इति॑ हुत - अदः॑ । प्री॒णाति॑ । मधु॑षा । अ॒हु॒ताद॒ इत्य॑हुत - अदः॑ । ग्रा॒म्यम् । वै । ए॒तत् । अन्न᳚म् । यत् । दधि॑ । आ॒र॒ण्यम् । मधु॑ । यत् । द॒द्ध्ना । म॒धु॒मि॒श्रेणेति॑ मधु - मि॒श्रेण॑ । अ॒वोक्ष॒तीत्य॑व - उक्ष॑ति । उ॒भय॑स्य । अव॑रुद्ध्या॒ इत्यव॑ - रु॒ध्यै॒ । ग्रु॒मु॒ष्टिना᳚ । अवेति॑ । उ॒क्ष॒ति॒ । प्रा॒जा॒प॒त्य इति॑ प्राजा-प॒त्यः ।  \newline




\markright{ TS 5.4.5.3  \hfill https://www.vedavms.in \hfill}

\section{ TS 5.4.5.3 }

\textbf{TS 5.4.5.3 } \newline
\textbf{Samhita Paata} \newline

वै ग्रु॑मु॒ष्टिः स॑योनि॒त्वाय॒ द्वाभ्यां॒ प्रति॑ष्ठित्या अनुपरि॒चार॒-मवो᳚क्ष॒त्य-प॑रिवर्गमे॒वैना᳚न् प्रीणाति॒ वि वा ए॒ष प्रा॒णैः प्र॒जया॑ प॒शुभि॑र्.ऋद्ध्यते॒ यो᳚ऽग्निं चि॒न्वन्न॑धि॒क्राम॑ति प्राण॒दा अ॑पान॒दा इत्या॑ह प्रा॒णाने॒वाऽऽत्मन् ध॑त्ते वर्चो॒दा व॑रिवो॒दा इत्या॑ह प्र॒जा वै वर्चः॑ प॒शवो॒ वरि॑वः प्र॒जामे॒व प॒शूना॒त्मन् ध॑त्त॒ इन्द्रो॑ वृ॒त्रम॑ह॒न्तं ॅवृ॒त्रो - [  ] \newline

\textbf{Pada Paata} \newline

वै । ग्रु॒मु॒ष्टिः । स॒यो॒नि॒त्वायेति॑ सयोनि - त्वाय॑ । द्वाभ्या᳚म् । प्रति॑ष्ठित्या॒ इति॒ प्रति॑ - स्थि॒त्यै॒ । अ॒नु॒प॒रि॒चार॒मित्य॑नु - प॒रि॒चार᳚म् । अवेति॑ । उ॒क्ष॒ति॒ । अप॑रिवर्ग॒मित्यप॑रि - व॒र्ग॒म् । ए॒व । ए॒ना॒न् । प्री॒णा॒ति॒ । वीति॑ । वै । ए॒षः । प्रा॒णैरिति॑ प्र - अ॒नैः । प्र॒जयेति॑ प्र - जया᳚ । प॒शुभि॒रिति॑ प॒शु - भिः॒ । ऋ॒द्ध्य॒ते॒ । यः । अ॒ग्निम् । चि॒न्वन्न् । अ॒धि॒क्राम॒तीत्य॑धि - क्राम॑ति । प्रा॒ण॒दा इति॑ प्राण - दाः । अ॒पा॒न॒दा इत्य॑पान - दाः । इति॑ । आ॒ह॒ । प्रा॒णानिति॑ प्र - अ॒नान् । ए॒व । आ॒त्मन् । ध॒त्ते॒ । व॒र्चो॒दा इति॑ वर्चः - दाः । व॒रि॒वो॒दा इति॑ वरिवः - दाः । इति॑ । आ॒ह॒ । प्र॒जेति॑ प्र - जा । वै । वर्चः॑ । प॒शवः॑ । वरि॑वः । प्र॒जामिति॑ प्र - जाम् । ए॒व । प॒शून् । आ॒त्मन्न् । ध॒त्ते॒ । इन्द्रः॑ । वृ॒त्रम् । अ॒ह॒न्न् । तम् । वृ॒त्रः ।  \newline




\markright{ TS 5.4.5.4  \hfill https://www.vedavms.in \hfill}

\section{ TS 5.4.5.4 }

\textbf{TS 5.4.5.4 } \newline
\textbf{Samhita Paata} \newline

ह॒तः षो॑ड॒शभि॑र्भो॒गैर॑सिना॒थ् स ए॒ताम॒ग्नयेऽनी॑कवत॒ आहु॑तिमपश्य॒त् ताम॑जुहो॒त् तस्या॒ग्निरनी॑ कवा॒न्थ्स्वेन॑ भाग॒धेये॑न प्री॒तः षो॑डश॒धा वृ॒त्रस्य॑ भो॒गानप्य॑दहद्-वैश्वकर्म॒णेन॑ पा॒प्मनो॒ निर॑मुच्यत॒ यद॒ग्नयेऽनी॑कवत॒ आहु॑तिं जु॒होत्य॒ग्निरे॒वा-ऽस्यानी॑कवा॒न्थ्स्वेन॑ भाग॒धेये॑न प्री॒तः पा॒प्मान॒मपि॑ दहति वैश्वकर्म॒णेन॑ पा॒प्मनो॒ निर्मु॑च्यते॒ यं का॒मये॑त चि॒रं पा॒प्मनो॒ - [  ] \newline

\textbf{Pada Paata} \newline

ह॒तः । षो॒ड॒शभि॒रिति॑ षोड॒श - भिः॒ । भो॒गैः । अ॒सि॒ना॒त् । सः । ए॒ताम् । अ॒ग्नये᳚ । अनी॑कवत॒ इत्यनी॑क - व॒ते॒ । आहु॑ति॒मित्या - हु॒ति॒म् । अ॒प॒श्य॒त् । ताम् । अ॒जु॒हो॒त् । तस्य॑ । अ॒ग्निः । अनी॑कवा॒नित्यनी॑क - वा॒न् । स्वेन॑ । भा॒ग॒धेये॒नेति॑ भाग - धेये॑न । प्री॒तः । षो॒ड॒श॒धेति॑ षोडश-धा । वृ॒त्रस्य॑ । भो॒गान् । अपीति॑ । अ॒द॒ह॒त् । वै॒श्व॒क॒र्म॒णेनेति॑ वैश्व - क॒र्म॒णेन॑ । पा॒प्मनः॑ । निरिति॑ । अ॒मु॒च्य॒त॒ । यत् । अ॒ग्नये᳚ । अनी॑कवत॒ इत्यनी॑क - व॒ते॒ । आहु॑ति॒मित्या - हु॒ति॒म् । जु॒होति॑ । अ॒ग्निः । ए॒व । अ॒स्य॒ । आनी॑कवा॒नित्यनी॑क - वा॒न् । स्वेन॑ । भा॒ग॒धेये॒नेति॑ भाग - धेये॑न । प्री॒तः । पा॒प्मान᳚म् । अपीति॑ । द॒ह॒ति॒ । वै॒श्व॒क॒र्म॒णेनेति॑ वैश्व - क॒र्म॒णेन॑ । पा॒प्मनः॑ । निरिति॑ । मु॒च्य॒ते॒ । यम् । का॒मये॑त । चि॒रम् । पा॒प्मनः॑ ।  \newline




\markright{ TS 5.4.5.5  \hfill https://www.vedavms.in \hfill}

\section{ TS 5.4.5.5 }

\textbf{TS 5.4.5.5 } \newline
\textbf{Samhita Paata} \newline

निर्मु॑च्ये॒तेत्येकै॑कं॒ तस्य॑ जुहुयाच्चि॒रमे॒व पा॒प्मनो॒ निर्मु॑च्यते॒ यं का॒मये॑त ता॒जक् पा॒प्मनो॒ निर्मु॑च्ये॒तेति॒ सर्वा॑णि॒ तस्या॑नु॒द्रुत्य॑ जुहुयात् ता॒जगे॒व पा॒प्मनो॒ निर्मु॑च्य॒तेऽथो॒ खलु॒ नानै॒व सू॒क्ताभ्यां᳚ जुहोति॒ नानै॒व सू॒क्तयो᳚र्वी॒र्यं॑ दधा॒त्यथो॒ प्रति॑ष्ठित्यै ॥ \newline

\textbf{Pada Paata} \newline

निरिति॑ । मु॒च्ये॒त॒ । इति॑ । एकै॑क॒मित्येकं᳚-ए॒क॒म् । तस्य॑ । जु॒हु॒या॒त् । चि॒रम् । ए॒व । पा॒प्मनः॑ । निरिति॑ । मु॒च्य॒ते॒ । यम् । का॒मये॑त । ता॒जक् । पा॒प्मनः॑ । निरिति॑ । मु॒च्ये॒त॒ । इति॑ । सर्वा॑णि । तस्य॑ । अ॒नु॒द्रुत्येत्य॑नु-द्रुत्य॑ । जु॒हु॒या॒त् । ता॒जक् । ए॒व । पा॒प्मनः॑ । निरिति॑ । मु॒च्य॒ते॒ । अथो॒ इति॑ । खलु॑ । नाना᳚ । ए॒व । सू॒क्ताभ्या॒मिति॑ सु - उ॒क्ताभ्या᳚म् । जु॒हो॒ति॒ । नाना᳚ । ए॒व । सू॒क्तयो॒रिति॑ सु - उ॒क्तयोः᳚ । वी॒र्य᳚म् । द॒धा॒ति॒ । अथो॒ इति॑ । प्रति॑ष्ठित्या॒ इति॒ प्रति॑ - स्थि॒त्यै॒ ॥  \newline




\markright{ TS 5.4.6.1  \hfill https://www.vedavms.in \hfill}

\section{ TS 5.4.6.1 }

\textbf{TS 5.4.6.1 } \newline
\textbf{Samhita Paata} \newline

उदे॑नमुत्त॒रां न॒येति॑ स॒मिध॒ आ द॑धाति॒ यथा॒ जनं॑ ॅय॒ते॑ऽव॒सं क॒रोति॑ ता॒दृगे॒व तत् ति॒स्र आ द॑धाति त्रि॒वृद्वा अ॒ग्निर्यावा॑ने॒-वाग्निस्तस्मै॑ भाग॒धेयं॑ करो॒त्यौदु॑म्बरी-र्भव॒न्त्यूर्ग्वा उ॑दु॒म्बर॒ ऊर्ज॑मे॒वास्मा॒ अपि॑ दधा॒त्युदु॑ त्वा॒ विश्वे॑ दे॒वा इत्या॑ह प्रा॒णा वै विश्वे॑ दे॒वाः प्रा॒णै - [  ] \newline

\textbf{Pada Paata} \newline

उदिति॑ । ए॒न॒म् । उ॒त्त॒रामित्यु॑त् - त॒राम् । न॒य॒ । इति॑ । स॒मिध॒ इति॑ सं - इधः॑ । एति॑ । द॒धा॒ति॒ । यथा᳚ । जन᳚म् । य॒ते । अ॒व॒सम् । क॒रोति॑ । ता॒दृक् । ए॒व । तत् । ति॒स्रः । एति॑ । द॒धा॒ति॒ । त्रि॒वृदिति॑ त्रि - वृत् । वै । अ॒ग्निः । यावान्॑ । ए॒व । अ॒ग्निः । तस्मै᳚ । भा॒ग॒धेय॒मिति॑ भाग - धेय᳚म् । क॒रो॒ति॒ । औदु॑बंरीः । भ॒व॒न्ति॒ । ऊर्क् । वै । उ॒दु॒बंरः॑ । ऊर्ज᳚म् । ए॒व । अ॒स्मै॒ । अपीति॑ । द॒धा॒ति॒ । उदिति॑ । उ॒ । त्वा॒ । विश्वे᳚ । दे॒वाः । इति॑ । आ॒ह॒ । प्रा॒णा इति॑ प्र - अ॒नाः । वै । विश्वे᳚ । दे॒वाः । प्रा॒णैरिति॑ प्र - अ॒नैः ।  \newline




\markright{ TS 5.4.6.2  \hfill https://www.vedavms.in \hfill}

\section{ TS 5.4.6.2 }

\textbf{TS 5.4.6.2 } \newline
\textbf{Samhita Paata} \newline

-रे॒वैन॒मुद्य॑च्छ॒ते ऽग्ने॒ भर॑न्तु॒ चित्ति॑भि॒रित्या॑ह॒ यस्मा॑ ए॒वैनं॑ चि॒त्तायो॒द्यच्छ॑ते॒ तेनै॒वैनꣳ॒॒ सम॑र्द्धयति॒ पञ्च॒ दिशो॒ दैवी᳚र्य॒ज्ञ्म॑वन्तु दे॒वीरित्या॑ह॒ दिशो॒ ह्ये॑षोऽनु॑ प्र॒च्यव॒ते ऽपाम॑तिं दुर्म॒तिं बाध॑माना॒ इत्या॑ह॒ रक्ष॑सा॒मप॑हत्यै रा॒यस्पोषे॑ य॒ज्ञ्प॑ति-मा॒भज॑न्ती॒रित्या॑ह प॒शवो॒ वै रा॒यस्पोषः॑ - [  ] \newline

\textbf{Pada Paata} \newline

ए॒व । ए॒न॒म् । उदिति॑ । य॒च्छ॒ते॒ । अग्ने᳚ । भर॑न्तु । चित्ति॑भि॒रिति॒ चित्ति॑ - भिः॒ । इति॑ । आ॒ह॒ । यस्मै᳚ । ए॒व । ए॒न॒म् । चि॒त्ताय॑ । उ॒द्यच्छ॑त॒ इत्यु॑त्-यच्छ॑ते । तेन॑ । ए॒व । ए॒न॒म् । समिति॑ । अ॒र्द्ध॒य॒ति॒ । पञ्च॑ । दिशः॑ । दैवीः᳚ । य॒ज्ञ्म् । अ॒व॒न्तु॒ । दे॒वीः । इति॑ । आ॒ह॒ । दिशः॑ । हि । ए॒षः । अन्विति॑ । प्र॒च्यव॑त॒ इति॑ प्र - च्यव॑ते । अपेति॑ । अम॑तिम् । दु॒र्म॒तिमिति॑ दुः - म॒तिम् । बाध॑मानाः । इति॑ । आ॒ह॒ । रक्ष॑साम् । अप॑हत्या॒ इत्यप॑ - ह॒त्यै॒ । रा॒यः । पोषे᳚ । य॒ज्ञ्प॑ति॒मिति॑ य॒ज्ञ्-प॒ति॒म् । आ॒भज॑न्ती॒रित्या᳚ - भज॑न्तीः । इति॑ । आ॒ह॒ । प॒शवः॑ । वै । रा॒यः । पोषः॑ ।  \newline




\markright{ TS 5.4.6.3  \hfill https://www.vedavms.in \hfill}

\section{ TS 5.4.6.3 }

\textbf{TS 5.4.6.3 } \newline
\textbf{Samhita Paata} \newline

प॒शूने॒वाव॑ रुन्धे ष॒ड्भिर्.ह॑रति॒ षड् वा ऋ॒तव॑ ऋ॒तुभि॑रे॒वैनꣳ॑ हरति॒ द्वे प॑रि॒गृह्य॑वती भवतो॒ रक्ष॑सा॒मप॑हत्यै॒ सूर्य॑रश्मि॒र्॒.हरि॑केशः पु॒रस्ता॒दित्या॑ह॒ प्रसू᳚त्यै॒ ततः॑ पाव॒का आ॒शिषो॑ नो जुषन्ता॒मित्या॒हान्नं॒ ॅवै पा॑व॒कोऽन्न॑मे॒वाव॑ रुन्धे देवासु॒राः संॅय॑त्ता आस॒न् ते दे॒वा ए॒त-दप्र॑तिरथ-मपश्य॒न् तेन॒ वै ते᳚ प्र॒त्य - [  ] \newline

\textbf{Pada Paata} \newline

प॒शून् । ए॒व । अवेति॑ । रु॒न्धे॒ । ष॒ड्भिरिति॑ षट्-भिः । ह॒र॒ति॒ । षट् । वै । ऋ॒तवः॑ । ऋ॒तुभि॒रित्यृ॒तु - भिः॒ । ए॒व । ए॒न॒म् । ह॒र॒ति॒ । द्वे इति॑ । प॒रि॒गृह्य॑वती॒ इति॑ परि॒गृह्य॑ - व॒ती॒ । भ॒व॒तः॒ । रक्ष॑साम् । अप॑हत्या॒ इत्यप॑ - ह॒त्यै॒ । सूर्य॑रश्मि॒रिति॒ सूर्य॑ - र॒श्मिः॒ । हरि॑केश॒ इति॒ हरि॑-के॒शः॒ । पु॒रस्ता᳚त् । इति॑ । आ॒ह॒ । प्रसू᳚त्या॒ इति॒ प्र-सू॒त्यै॒ । ततः॑ । पा॒व॒काः । आ॒शिष॒ इत्या᳚ - शिषः॑ । नः॒ । जु॒ष॒न्ता॒म् । इति॑ । आ॒ह॒ । अन्न᳚म् । वै । पा॒व॒कः । अन्न᳚म् । ए॒व । अवेति॑ । रु॒न्धे॒ । दे॒वा॒सु॒रा इति॑ देव-अ॒सु॒राः । संॅय॑त्ता॒ इति॒ सं - य॒त्ताः॒ । आ॒स॒न्न् । ते । दे॒वाः । ए॒तत् । अप्र॑तिरथ॒मित्यप्र॑ति - र॒थ॒म् । अ॒प॒श्य॒न्न् । तेन॑ । वै । ते । अ॒प्र॒ति ।  \newline




\markright{ TS 5.4.6.4  \hfill https://www.vedavms.in \hfill}

\section{ TS 5.4.6.4 }

\textbf{TS 5.4.6.4 } \newline
\textbf{Samhita Paata} \newline

-सु॑रानजय॒न् तदप्र॑तिरथस्या-प्रतिरथ॒त्वं ॅयदप्र॑तिरथं द्वि॒तीयो॒ होता॒ऽन्वाहा᳚प्र॒त्ये॑व तेन॒ यज॑मानो॒ भ्रातृ॑व्यान् जय॒त्यथो॒ अन॑भिजितमे॒वाभि ज॑यति दश॒र्चं भ॑वति॒ दशा᳚क्षरा वि॒राड् वि॒राजे॒मौ लो॒कौ विधृ॑ता व॒नयो᳚र्लो॒कयो॒र्विधृ॑त्या॒ अथो॒ दशा᳚क्षरा वि॒राडन्नं॑ ॅवि॒राड् वि॒राज्ये॒वान्नाद्ये॒ प्रति॑तिष्ठ॒त्यस॑दिव॒ वा अ॒न्तरि॑क्षम॒न्तरि॑क्षमि॒वा ऽऽ*ग्नी᳚द्ध्र॒माग्नी॒द्ध्रे - [  ] \newline

\textbf{Pada Paata} \newline

असु॑रान् । अ॒ज॒य॒न्न् । तत् । अप्र॑तिरथ॒स्येत्यप्र॑ति - र॒थ॒स्य॒ । अ॒प्र॒ति॒र॒थ॒त्वमित्य॑प्रतिरथ - त्वम् । यत् । अप्र॑तिरथ॒मित्यप्र॑ति-र॒थ॒म् । द्वि॒तीयः॑ । होता᳚ । अ॒न्वाहेत्य॑नु - आह॑ । अ॒प्र॒ति । ए॒व । तेन॑ । यज॑मानः । भ्रातृ॑व्यान् । ज॒य॒ति॒ । अथो॒ इति॑ । अन॑भिजित॒मित्यन॑भि - जि॒त॒म् । ए॒व । अ॒भीति॑ । ज॒य॒ति॒ । द॒श॒र्चमिति॑ दश - ऋ॒चम् । भ॒व॒ति॒ । दशा᳚क्ष॒रेति॒ दश॑ - अ॒क्ष॒रा॒ । वि॒राडिति॑ वि - राट् । वि॒राजेति॑ वि - राजा᳚ । इ॒मौ । लो॒कौ । विधृ॑ता॒विति॒ वि - धृ॒तौ॒ । अ॒नयोः᳚ । लो॒कयोः᳚ । विधृ॑त्या॒ इति॒ वि - धृ॒त्यै॒ । अथो॒ इति॑ । दशा᳚क्ष॒रेति॒ दश॑ - अ॒क्ष॒रा॒ । वि॒राडिति॑ वि - राट् । अन्न᳚म् । वि॒राडिति॑ वि - राट् । वि॒राजीति॑ वि - राजि॑ । ए॒व । अ॒न्नाद्य॒ इत्य॑न्न - अद्ये᳚ । प्रतीति॑ । ति॒ष्ठ॒ति॒ । अस॑त् । इ॒व॒ । वै । अ॒न्तरि॑क्षम् । अ॒न्तरि॑क्षम् । इ॒व॒ । आग्नी᳚ध्र॒मित्याग्नि॑ - इ॒ध्र॒म् । आग्नी᳚द्ध्र॒ इत्याग्नि॑ - इ॒द्ध्रे॒ ।  \newline




\markright{ TS 5.4.6.5  \hfill https://www.vedavms.in \hfill}

\section{ TS 5.4.6.5 }

\textbf{TS 5.4.6.5 } \newline
\textbf{Samhita Paata} \newline

ऽश्मा॑नं॒ नि द॑धाति स॒त्त्वाय॒ द्वाभ्यां॒ प्रति॑ष्ठित्यै वि॒मान॑ ए॒ष दि॒वो मद्ध्य॑ आस्त॒ इत्या॑ह॒ व्ये॑वैतया॑ मिमीते॒ मद्ध्ये॑ दि॒वो निहि॑तः॒ पृश्नि॒रश्मेत्या॒हान्नं॒ ॅवै पृश्न्यन्न॑मे॒वाव॑ रुन्धे चत॒सृभि॒रा पुच्छा॑देति च॒त्वारि॒ छन्दाꣳ॑सि॒ छन्दो॑भिरे॒वेन्द्रं॒ ॅविश्वा॑ अवीवृध॒न्नित्या॑ह॒ वृद्धि॑मे॒वोपाव॑र्तते॒ वाजा॑नाꣳ॒॒ सत्प॑तिं॒ पति॒ - [  ] \newline

\textbf{Pada Paata} \newline

अश्मा॑नम् । नीति॑ । द॒धा॒ति॒ । स॒त्त्वायेति॑ सत् - त्वाय॑ । द्वाभ्या᳚म् । प्रति॑ष्ठित्या॒ इति॒ प्रति॑ - स्थि॒त्यै॒ । वि॒मान॒ इति॑ वि - मानः॑ । ए॒षः । दि॒वः । मद्ध्ये᳚ । आ॒स्ते॒ । इति॑ । आ॒ह॒ । वीति॑ । ए॒व । ए॒तया᳚ । मि॒मी॒ते॒ । मद्ध्ये᳚ । दि॒वः । निहि॑त॒ इति॒ नि - हि॒तः॒ । पृश्निः॑ । अश्मा᳚ । इति॑ । आ॒ह॒ । अन्न᳚म् । वै । पृश्नि॑ । अन्न᳚म् । ए॒व । अवेति॑ । रु॒न्धे॒ । च॒त॒सृभि॒रिति॑ चत॒सृ-भिः॒ । एति॑ । पुच्छा᳚त् । ए॒ति॒ । च॒त्वारि॑ । छन्दाꣳ॑सि । छन्दो॑भि॒रिति॒ छन्दः॑ - भिः॒ । ए॒व । इन्द्र᳚म् । विश्वाः᳚ । अ॒वी॒वृ॒ध॒न्न् । इति॑ । आ॒ह॒ । वृद्धि᳚म् । ए॒व । उ॒पाव॑र्तत॒ इत्यु॑प-आव॑र्तते । वाजा॑नाम् । सत्प॑ति॒मिति॒ सत् - प॒ति॒म् । पति᳚म् ।  \newline




\markright{ TS 5.4.6.6  \hfill https://www.vedavms.in \hfill}

\section{ TS 5.4.6.6 }

\textbf{TS 5.4.6.6 } \newline
\textbf{Samhita Paata} \newline

-मित्या॒हाऽन्नं॒ ॅवै वाजोऽन्न॑मे॒वाव॑ रुन्धे सुम्न॒हूर्य॒ज्ञो दे॒वाꣳ आ च॑ वक्ष॒दित्या॑ह प्र॒जा वै प॒शवः॑ सु॒म्नं प्र॒जामे॒व प॒शूना॒त्मन् ध॑त्ते॒ यक्ष॑द॒ग्निर्दे॒वो दे॒वाꣳ आ च॑ वक्ष॒दित्या॑ह स्व॒गाकृ॑त्यै॒ वाज॑स्य मा प्रस॒वेनो᳚द्-ग्रा॒भेणोद॑ग्रभी॒दित्या॑हा॒सौ वा आ॑दि॒त्य उ॒द्यन्नु॑द्ग्रा॒भ ए॒ष ( ) नि॒म्रोच॑न् निग्रा॒भो ब्रह्म॑णै॒वाऽऽ*त्मान॑मुद्-गृ॒ह्णाति॒ ब्रह्म॑णा॒ भ्रातृ॑व्यं॒ नि गृ॑ह्णाति ॥ \newline

\textbf{Pada Paata} \newline

इति॑ । आ॒ह॒ । अन्न᳚म् । वै । वाजः॑ । अन्न᳚म् । ए॒व । अवेति॑ । रु॒न्धे॒ । सु॒म्न॒हूरिति॑ सुम्न - हूः । य॒ज्ञ्ः । दे॒वान् । एति॑ । च॒ । व॒क्ष॒त् । इति॑ । आ॒ह॒ । प्र॒जेति॑ प्र - जा । वै । प॒शवः॑ । सु॒म्नम् । प्र॒जामिति॑ प्र - जाम् । ए॒व । प॒शून् । आ॒त्मन्न् । ध॒त्ते॒ । यक्ष॑त् । अ॒ग्निः । दे॒वः । दे॒वान् । एति॑ । च॒ । व॒क्ष॒त् । इति॑ । आ॒ह॒ । स्व॒गाकृ॑त्या॒ इति॑ स्व॒गा - कृ॒त्यै॒ । वाज॑स्य । मा॒ । प्र॒स॒वेनेति॑ प्र - स॒वेन॑ । उ॒द्ग्रा॒भेणेत्यू॑त्-ग्रा॒भेण॑ । उदिति॑ । अ॒ग्र॒भी॒त् । इति॑ । आ॒ह॒ । अ॒सौ । वै । आ॒दि॒त्यः । उ॒द्यन्नित्यु॑त् - यन्न् । उ॒द्ग्रा॒भ इत्यु॑त् - ग्रा॒भः । ए॒षः ( ) । नि॒म्रोच॒न्निति॑ नि - म्रोचन्न्॑ । नि॒ग्रा॒भ इति॑ नि - ग्रा॒भः । ब्रह्म॑णा । ए॒व । आ॒त्मान᳚म् । उ॒द्गृ॒ह्णातीत्यु॑त् - गृ॒ह्णाति॑ । ब्रह्म॑णा । भ्रातृ॑व्यम् । नीति॑ । गृ॒ह्णा॒ति॒ ॥  \newline




\markright{ TS 5.4.7.1  \hfill https://www.vedavms.in \hfill}

\section{ TS 5.4.7.1 }

\textbf{TS 5.4.7.1 } \newline
\textbf{Samhita Paata} \newline

प्राची॒मनु॑ प्र॒दिशं॒ प्रेहि॑ वि॒द्वानित्या॑ह देवलो॒क-मे॒वैतयो॒पाव॑र्तते॒ क्रम॑द्ध्वम॒ग्निना॒ नाक॒-मित्या॑हे॒माने॒वैतया॑ लो॒कान् क्र॑मते पृथि॒व्या अ॒हमुद॒न्तरि॑क्ष॒मा ऽरु॑ह॒मित्या॑हे॒माने॒वैतया॑ लो॒कान्थ् स॒मारो॑हति॒ सुव॒र्यन्तो॒ नापे᳚क्षन्त॒ इत्या॑ह सुव॒र्गमे॒वैतया॑ लो॒कमे॒त्यग्ने॒ प्रेहि॑ - [  ] \newline

\textbf{Pada Paata} \newline

प्राची᳚म् । अन्विति॑ । प्र॒दिश॒मिति॑ प्र-दिश᳚म् । प्रेति॑ । इ॒हि॒ । वि॒द्वान् । इति॑ । आ॒ह॒ । दे॒व॒लो॒कमिति॑ देव - लो॒कम् । ए॒व । ए॒तया᳚ । उ॒पाव॑र्तत॒ इत्यु॑प - आव॑र्तते । क्रम॑द्ध्वम् । अ॒ग्निना᳚ । नाक᳚म् । इति॑ । आ॒ह॒ । इ॒मान् । ए॒व । ए॒तया᳚ । लो॒कान् । क्र॒म॒ते॒ । पृ॒थि॒व्याः । अ॒हम् । उदिति॑ । अ॒न्तरि॑क्षम् । एति॑ । अ॒रु॒ह॒म् । इति॑ । आ॒ह॒ । इ॒मान् । ए॒व । ए॒तया᳚ । लो॒कान् । स॒मारो॑ह॒तीति॑ सं - आरो॑हति । सुवः॑ । यन्तः॑ । न । अपेति॑ । ई॒क्ष॒न्ते॒ । इति॑ । आ॒ह॒ । सु॒व॒र्गमिति॑ सुवः - गम् । ए॒व । ए॒तया᳚ । लो॒कम् । ए॒ति॒ । अग्ने᳚ । प्रेति॑ । इ॒हि॒ ।  \newline




\markright{ TS 5.4.7.2  \hfill https://www.vedavms.in \hfill}

\section{ TS 5.4.7.2 }

\textbf{TS 5.4.7.2 } \newline
\textbf{Samhita Paata} \newline

प्रथ॒मो दे॑वय॒ता-मित्या॑हो॒भये᳚ष्वे॒वैतया॑ देवमनु॒ष्येषु॒ चक्षु॑र्दधाति प॒ञ्चभि॒रधि॑ क्रामति॒ पाङ्क्तो॑ य॒ज्ञो यावा॑ने॒व य॒ज्ञ्स्तेन॑ स॒ह सु॑व॒र्गं ॅलो॒कमे॑ति॒ नक्तो॒षासेति॑ पुरोऽनुवा॒क्या॑मन्वा॑ह॒ प्रत्या॒ अग्ने॑ सहस्रा॒क्षेत्या॑ह साह॒स्रः प्र॒जाप॑तिः प्र॒जाप॑ते॒राप्त्यै॒ तस्मै॑ ते विधेम॒ वाजा॑य॒ स्वाहेत्या॒हान्नं॒ ॅवै वाजोऽन्न॑मे॒वाव॑ - [  ] \newline

\textbf{Pada Paata} \newline

प्र॒थ॒मः । दे॒व॒य॒तामिति॑ देव - य॒ताम् । इति॑ । आ॒ह॒ । उ॒भये॑षु । ए॒व । ए॒तया᳚ । दे॒व॒म॒नु॒ष्येष्विति॑ देव - म॒नु॒ष्येषु॑ । चक्षुः॑ । द॒धा॒ति॒ । प॒ञ्चभि॒रिति॑ प॒ञ्च - भिः॒ । अधीति॑ । क्रा॒म॒ति॒ । पाङ्क्तः॑ । य॒ज्ञ्ः । यावान्॑ । ए॒व । य॒ज्ञ्ः । तेन॑ । स॒ह । सु॒व॒र्गमिति॑ सुवः - गम् । लो॒कम् । ए॒ति॒ । नक्तो॒षासा᳚ । इति॑ । पु॒रो॒नु॒वा॒क्या॑मिति॑ पुरः - अ॒नु॒वा॒क्या᳚म् । अन्विति॑ । आ॒ह॒ । प्रत्यै᳚ । अग्ने᳚ । स॒ह॒स्रा॒क्षेति॑ सहस्र - अ॒क्ष॒ । इति॑ । आ॒ह॒ । सा॒ह॒स्रः । प्र॒जाप॑ति॒रिति॑ प्र॒जा - प॒तिः॒ । प्र॒जाप॑ते॒रिति॑ प्र॒जा - प॒तेः॒ । आप्त्यै᳚ । तस्मै᳚ । ते॒ । वि॒धे॒म॒ । वाजा॑य । स्वाहा᳚ । इति॑ । आ॒ह॒ । अन्न᳚म् । वै । वाजः॑ । अन्न᳚म् । ए॒व । अवेति॑ ।  \newline




\markright{ TS 5.4.7.3  \hfill https://www.vedavms.in \hfill}

\section{ TS 5.4.7.3 }

\textbf{TS 5.4.7.3 } \newline
\textbf{Samhita Paata} \newline

रुन्धे द॒द्ध्नः पू॒र्णामौदु॑म्बरीꣳ स्वयमातृ॒ण्णायां᳚ जुहो॒त्यूर्ग्वै दद्ध्यूर्गु॑दु॒म्बरो॒ऽसौ स्व॑यमातृ॒ण्णा ऽमुष्या॑मे॒वोर्जं॑ दधाति॒ तस्मा॑द॒मुतो॒ऽर्वाची॒मूर्ज॒मुप॑ जीवामस्ति॒सृभिः॑ सादयति त्रि॒वृद्वा अ॒ग्निर्यावा॑ने॒वाग्निस्तं प्र॑ति॒ष्ठां ग॑मयति॒ प्रेद्धो॑ अग्ने दीदिहि पु॒रो न॒ इत्यौदु॑म्बरी॒मा द॑धात्ये॒षा वै सू॒र्मी कर्ण॑कावत्ये॒तया॑ ह स्म॒ - [  ] \newline

\textbf{Pada Paata} \newline

रु॒न्धे॒ । द॒द्ध्नः । पू॒र्णाम् । औदु॑बंरीम् । स्व॒य॒मा॒तृ॒ण्णाया॒मिति॑ स्वयं - आ॒तृ॒ण्णाया᳚म् । जु॒हो॒ति॒ । ऊर्क् । वै । दधि॑ । ऊर्क् । उ॒दु॒बंरः॑ । अ॒सौ । स्व॒य॒मा॒तृ॒ण्णेति॑ स्वयं - आ॒तृ॒ण्णा । अ॒मुष्या᳚म् । ए॒व । ऊर्ज᳚म् । द॒धा॒ति॒ । तस्मा᳚त् । अ॒मुतः॑ । अ॒र्वाची᳚म् । ऊर्ज᳚म् । उपेति॑ । जी॒वा॒मः॒ । ति॒सृभि॒रिति॑ ति॒सृ - भिः॒। सा॒द॒य॒ति॒ । त्रि॒वृदिति॑ त्रि - वृत् । वै । अ॒ग्निः । यावान्॑ । ए॒व । अ॒ग्निः । तम् । प्र॒ति॒ष्ठामिति॑ प्रति-स्थाम् । ग॒म॒य॒ति॒ । प्रेद्ध॒ इति॒ प्र - इ॒द्धः॒ । अ॒ग्ने॒ । दी॒दि॒हि॒ । पु॒रः । नः॒ । इति॑ । औदु॑बंरीम् । एति॑ । द॒धा॒ति॒ । ए॒षा । वै । सू॒र्मी । कर्ण॑काव॒तीति॒ कर्ण॑क - व॒ति॒ । ए॒तया᳚ । ह॒ । स्म॒ ।  \newline




\markright{ TS 5.4.7.4  \hfill https://www.vedavms.in \hfill}

\section{ TS 5.4.7.4 }

\textbf{TS 5.4.7.4 } \newline
\textbf{Samhita Paata} \newline

वै दे॒वा असु॑राणाꣳ शतत॒र्॒.हाꣳ स्तृꣳ॑हन्ति॒ यदे॒तया॑ स॒मिध॑मा॒दधा॑ति॒ वज्र॑मे॒वैतच्छ॑त॒घ्नीं ॅयज॑मानो॒ भ्रातृ॑व्याय॒ प्रह॑रति॒ स्तृत्या॒ अच्छ॑म्बट्कारं ॅवि॒धेम॑ ते पर॒मे जन्म॑न्नग्न॒ इति॒ वैक॑ङ्कती॒मा द॑धाति॒ भा ए॒वाव॑ रुन्धे॒ ताꣳ स॑वि॒तुर्वरे᳚ण्यस्य चि॒त्रामिति॑ शमी॒मयीꣳ॒॒ शान्त्या॑ अ॒ग्निर्वा॑ ह॒ वा अ॑ग्नि॒चितं॑ दु॒हे᳚ऽग्नि॒चिद्वा॒ऽग्निं दु॑हे॒ ताꣳ - [  ] \newline

\textbf{Pada Paata} \newline

वै । दे॒वाः । असु॑राणाम् । श॒त॒त॒र्.॒हानिति॑ शत-त॒र्॒.हान् । तृꣳ॒॒ह॒न्ति॒ । यत् । ए॒तया᳚ । स॒मिध॒मिति॑ सं - इध᳚म् । आ॒दधा॒तीत्या᳚ - दधा॑ति । वज्र᳚म् । ए॒व । ए॒तत् । श॒त॒घ्नीमिति॑ शत - घ्नीम् । यज॑मानः । भ्रातृ॑व्याय । प्रेति॑ । ह॒र॒ति॒ । स्तृत्यै᳚ । अच्छ॑बंट्कार॒मित्यछ॑बंट्-का॒र॒म् । वि॒धेम॑ । ते॒ । प॒र॒मे । जन्मन्न्॑ । अ॒ग्ने॒ । इति॑ । वैक॑ङ्कतीम् । एति॑ । द॒धा॒ति॒ । भाः । ए॒व । अवेति॑ । रु॒न्धे॒ । ताम् । स॒वि॒तुः । वरे᳚ण्यस्य । चि॒त्राम् । इति॑ । श॒मी॒मयी॒मिति॑ शमी - मयी᳚म् । शान्त्यै᳚ । अ॒ग्निः । वा॒ । ह॒ । वै । अ॒ग्नि॒चित॒मित्य॑ग्नि - चित᳚म् । दु॒हे । अ॒ग्नि॒चिदित्य॑ग्नि - चित् । वा॒ । अ॒ग्निम् । दु॒हे॒ । ताम् ।  \newline




\markright{ TS 5.4.7.5  \hfill https://www.vedavms.in \hfill}

\section{ TS 5.4.7.5 }

\textbf{TS 5.4.7.5 } \newline
\textbf{Samhita Paata} \newline

स॑वि॒तुर्वरे᳚ण्यस्य चि॒त्रामित्या॑है॒ष वा अ॒ग्नेर्दोह॒स्तम॑स्य॒ कण्व॑ ए॒व श्रा॑य॒सो॑ऽवे॒त् तेन॑ ह स्मैनꣳ॒॒ स दु॑हे॒ यदे॒तया॑ स॒मिध॑-मा॒दधा᳚त्यग्नि॒चिदे॒व तद॒ग्निं दु॑हे स॒प्त ते॑ अग्ने स॒मिधः॑ स॒प्तजि॒ह्वा इत्या॑ह स॒प्तैवास्य॒ साप्ता॑नि प्रीणाति पू॒र्णया॑ जुहोति पू॒र्ण इ॑व॒ हि प्र॒जाप॑तिः प्र॒जाप॑ते॒ - [  ] \newline

\textbf{Pada Paata} \newline

स॒वि॒तुः । वरे᳚ण्यस्य । चि॒त्राम् । इति॑ । आ॒ह॒ । ए॒षः । वै । अ॒ग्नेः । दोहः॑ । तम् । अ॒स्य॒ । कण्वः॑ । ए॒व । श्रा॒य॒सः । अ॒वे॒त् । तेन॑ । ह॒ । स्म॒ । ए॒न॒म् । सः । दु॒हे॒ । यत् । ए॒तया᳚ । स॒मिध॒मिति॑ सं - इध᳚म् । आ॒दधा॒तीत्या᳚ - दधा॑ति । अ॒ग्नि॒चिदित्य॑ग्नि - चित् । ए॒व । तत् । अ॒ग्निम् । दु॒हे॒ । स॒प्त । ते॒ । अ॒ग्ने॒ । स॒मिध॒ इति॑ सं - इधः॑ । स॒प्त । जि॒ह्वाः । इति॑ । आ॒ह॒ । स॒प्त । ए॒व । अ॒स्य॒ । साप्ता॑नि । प्री॒णा॒ति॒ । पू॒र्णया᳚ । जु॒हो॒ति॒ । पू॒र्णः । इ॒व॒ । हि । प्र॒जाप॑ति॒रिति॑ प्र॒जा - प॒तिः॒ । प्र॒जाप॑ते॒रिति॑ प्र॒जा - प॒तेः॒ ।  \newline




\markright{ TS 5.4.7.6  \hfill https://www.vedavms.in \hfill}

\section{ TS 5.4.7.6 }

\textbf{TS 5.4.7.6 } \newline
\textbf{Samhita Paata} \newline

-राप्त्यै॒ न्यू॑नया जुहोति॒ न्यू॑ना॒द्धि प्र॒जाप॑तिः प्र॒जा असृ॑जत प्र॒जानाꣳ॒॒ सृष्ट्या॑ अ॒ग्निर्दे॒वेभ्यो॒ निला॑यत॒ स दिशोऽनु॒ प्राऽवि॑श॒ज्जुह्व॒न्मन॑सा॒ दिशो᳚ द्ध्याये द्दि॒ग्भ्य ए॒वैन॒मव॑ रुन्धे द॒द्ध्ना पु॒रस्ता᳚ज्जुहो॒त्या-ज्ये॑नो॒परि॑ष्टा॒त् तेज॑श्चै॒वास्मा॑ इन्द्रि॒यं च॑ स॒मीची॑ दधाति॒ द्वाद॑शकपालो वैश्वान॒रो भ॑वति॒ द्वाद॑श॒ मासाः᳚ संॅवथ्स॒रः सं॑ॅवथ्स॒रो᳚-ऽग्निर्वै᳚श्वान॒रः सा॒क्षा - [  ] \newline

\textbf{Pada Paata} \newline

आप्त्यै᳚ । न्यू॑न॒येति॒ नि-ऊ॒न॒या॒ । जु॒हो॒ति॒ । न्यू॑ना॒दिति॒ नि - ऊ॒ना॒त् । हि । प्र॒जाप॑ति॒रिति॑ प्र॒जा - प॒तिः॒ । प्र॒जा इति॑ प्र - जाः । असृ॑जत । प्र॒जाना॒मिति॑ प्र - जाना᳚म् । सृष्ट्यै᳚ । अ॒ग्निः । दे॒वेभ्यः॑ । निला॑यत । सः । दिशः॑ । अनु॑ । प्रेति॑ । अ॒वि॒श॒त् । जुह्व॑त् । मन॑सा । दिशः॑ । ध्या॒ये॒त् । दि॒ग्भ्य इति॑ दिक् - भ्यः । ए॒व । ए॒न॒म् । अवेति॑ । रु॒न्धे॒ । द॒द्ध्ना । पु॒रस्ता᳚त् । जु॒हो॒ति॒ । आज्ये॑न । उ॒परि॑ष्टात् । तेजः॑ । च॒ । ए॒व । अ॒स्मै॒ । इ॒न्द्रि॒यम् । च॒ । स॒मीची॒ इति॑ । द॒धा॒ति॒ । द्वाद॑शकपाल॒ इति॒ द्वाद॑श - क॒पा॒लः॒ । वै॒श्वा॒न॒रः । भ॒व॒ति॒ । द्वाद॑श । मासाः᳚ । सं॒ॅव॒थ्स॒र इति॑ सं-व॒थ्स॒रः । सं॒ॅव॒थ्स॒र इति॑ सं-व॒थ्स॒रः । अ॒ग्निः । वै॒श्वा॒न॒रः । सा॒क्षादिति॑ स - अ॒क्षात् ।  \newline




\markright{ TS 5.4.7.7  \hfill https://www.vedavms.in \hfill}

\section{ TS 5.4.7.7 }

\textbf{TS 5.4.7.7 } \newline
\textbf{Samhita Paata} \newline

-दे॒व वै᳚श्वान॒रमव॑ रुन्धे॒ यत् प्र॑याजानूया॒जान् कु॒र्याद्विक॑स्तिः॒ सा य॒ज्ञ्स्य॑ दर्विहो॒मं क॑रोति य॒ज्ञ्स्य॒ प्रति॑ष्ठित्यै रा॒ष्ट्रं ॅवै वै᳚श्वान॒रो विण्म॒रुतो॑ वैश्वान॒रꣳ हु॒त्वा मा॑रु॒तान् जु॑होति रा॒ष्ट्र ए॒व विश॒मनु॑ बद्ध्नात्यु॒च्चै-र्वै᳚श्वान॒रस्याऽऽ श्रा॑वयत्युपाꣳ॒॒शु मा॑रु॒तान् जु॑होति॒ तस्मा᳚द्-रा॒ष्ट्रं ॅविश॒मति॑ वदति मारु॒ता भ॑वन्ति म॒रुतो॒ वै दे॒वानां॒ ॅविशो॑ देववि॒शेनै॒वास्मै॑ मनुष्यवि॒श ( ) -मव॑ रुन्धे स॒प्त भ॑वन्ति स॒प्तग॑णा॒ वै म॒रुतो॑ गण॒श ए॒व विश॒मव॑ रुन्धे ग॒णेन॑ ग॒णम॑नु॒द्रुत्य॑ जुहोति॒ विश॑मे॒वास्मा॒ अनु॑वर्त्मानं करोति ॥ \newline

\textbf{Pada Paata} \newline

ए॒व । वै॒श्वा॒न॒रम् । अवेति॑ । रु॒न्धे॒ । यत् । प्र॒या॒जा॒नू॒या॒जानिति॑ प्रयाज - अ॒नू॒या॒जान् । कु॒र्यात् । विक॑स्ति॒रिति॒ वि - क॒स्तिः॒ । सा । य॒ज्ञ्स्य॑ । द॒र्वि॒हो॒ममिति॑ दर्वि - हो॒मम् । क॒रो॒ति॒ । य॒ज्ञ्स्य॑ । प्रति॑ष्ठित्या॒ इति॒ प्रति॑ - स्थि॒त्यै॒ । रा॒ष्ट्रम् । वै । वै॒श्वा॒न॒रः । विट् । म॒रुतः॑ । वै॒श्वा॒न॒रम् । हु॒त्वा । मा॒रु॒ताम् । जु॒हो॒ति॒ । रा॒ष्ट्रे । ए॒व । विश᳚म् । अन्विति॑ । ब॒द्ध्ना॒ति॒ । उ॒च्चैः । वै॒श्वा॒न॒रस्य॑ । एति॑ । श्रा॒व॒य॒ति॒ । उ॒पाꣳ॒॒श्वित्यु॑प - अꣳ॒॒शु । मा॒रु॒तान् । जु॒हो॒ति॒ । तस्मा᳚त् । रा॒ष्ट्रम् । विश᳚म् । अतीति॑ । व॒द॒ति॒ । मा॒रु॒ताः । भ॒व॒न्ति॒ । म॒रुतः॑ । वै । दे॒वाना᳚म् । विशः॑ । दे॒व॒वि॒शेनेति॑ देव - वि॒शेन॑ । ए॒व । अ॒स्मै॒ । म॒नु॒ष्य॒वि॒शमिति॑ मनुष्य - वि॒शम् ( ) । अवेति॑ । रु॒न्धे॒ । स॒प्त । भ॒व॒न्ति॒ । स॒प्तग॑णा॒ इति॑ स॒प्त - ग॒णाः॒ । वै । म॒रुतः॑ । ग॒ण॒श इति॑ गण - शः । ए॒व । विश᳚म् । अवेति॑ । रु॒न्धे॒ । ग॒णेन॑ । ग॒णम् । अ॒नु॒द्रुत्येत्य॑नु - द्रुत्य॑ । जु॒हो॒ति॒ । विश᳚म् । ए॒व । अ॒स्मै॒ । अनु॑वर्त्मान॒मित्यनु॑ - व॒र्त्मा॒न॒म् । क॒रो॒ति॒ ॥  \newline




\markright{ TS 5.4.8.1  \hfill https://www.vedavms.in \hfill}

\section{ TS 5.4.8.1 }

\textbf{TS 5.4.8.1 } \newline
\textbf{Samhita Paata} \newline

वसो॒र्द्धारां᳚ जुहोति॒ वसो᳚र्मे॒ धारा॑ऽस॒दिति॒ वा ए॒षा हू॑यते घृ॒तस्य॒ वा ए॑नमे॒षा धारा॒ऽमुष्मि॑न् ॅलो॒के पिन्व॑मा॒नोप॑ तिष्ठत॒ आज्ये॑न जुहोति॒ तेजो॒ वा आज्यं॒ तेजो॒ वसो॒र्द्धारा॒ तेज॑सै॒वास्मै॒ तेजोऽव॑ रु॒न्धेऽथो॒ कामा॒ वै वसो॒र्द्धारा॒ कामा॑ने॒वाव॑ रुन्धे॒ यं का॒मये॑त प्रा॒णान॑स्या॒न्नाद्यं॒ ॅवि - [  ] \newline

\textbf{Pada Paata} \newline

वसोः᳚ । धारा᳚म् । जु॒हो॒ति॒ । वसोः᳚ । मे॒ । धारा᳚ । अ॒स॒त् । इति॑ । वै । ए॒षा । हू॒य॒ते॒ । घृ॒तस्य॑ । वै । ए॒न॒म् । ए॒षा । धारा᳚ । अ॒मुष्मिन्न्॑ । लो॒के । पिन्व॑माना । उपेति॑ । ति॒ष्ठ॒ते॒ । आज्ये॑न । जु॒हो॒ति॒ । तेजः॑ । वै । आज्य᳚म् । तेजः॑ । वसोः᳚ । धारा᳚ । तेज॑सा । ए॒व । अ॒स्मै॒ । तेजः॑ । अवेति॑ । रु॒न्धे॒ । अथो॒ इति॑ । कामाः᳚ । वै । वसोः᳚ । धारा᳚ । कामान्॑ । ए॒व । अवेति॑ । रु॒न्धे॒ । यम् । का॒मये॑त । प्रा॒णानिति॑ प्र - अ॒नान् । अ॒स्य॒ । अ॒न्नाद्य॒मित्य॑न्न - अद्य᳚म् । वीति॑ ।  \newline




\markright{ TS 5.4.8.2  \hfill https://www.vedavms.in \hfill}

\section{ TS 5.4.8.2 }

\textbf{TS 5.4.8.2 } \newline
\textbf{Samhita Paata} \newline

च्छि॑न्द्या॒मिति॑ वि॒ग्राहं॒ तस्य॑ जुहुयात् प्रा॒णाने॒वास्या॒न्नाद्यं॒ ॅविच्छि॑नत्ति॒ यं का॒मये॑त प्रा॒णान॑स्या॒न्नाद्यꣳ॒॒ सं त॑नुया॒मिति॒ सं त॑तां॒ तस्य॑ जुहुयात् प्रा॒णाने॒वास्या॒न्नाद्यꣳ॒॒ सं त॑नोति॒ द्वाद॑श द्वाद॒शानि॑ जुहोति॒ द्वाद॑श॒ मासाः᳚ संॅवथ्स॒रः सं॑ॅवथ्स॒रेणै॒ वास्मा॒ अन्न॒मव॑ रु॒न्धे ऽन्नं॑ च॒ मेऽक्षु॑च्च म॒ इत्या॑है॒ तद् वा - [  ] \newline

\textbf{Pada Paata} \newline

छि॒न्द्या॒म् । इति॑ । वि॒ग्राह॒मिति॑ वि - ग्राह᳚म् । तस्य॑ । जु॒हु॒या॒त् । प्रा॒णानिति॑ प्र - अ॒नान् । ए॒व । अ॒स्य॒ । अ॒न्नाद्य॒मित्य॑न्न - अद्य᳚म् । वीति॑ । छि॒न॒त्ति॒ । यम् । का॒मये॑त । प्रा॒णानिति॑ प्र-अ॒नान् । अ॒स्य॒ । अ॒न्नाद्य॒मित्य॑न्न - अद्य᳚म् । समिति॑ । त॒नु॒या॒म् । इति॑ । संत॑ता॒मिति॒ सं - त॒ता॒म् । तस्य॑ । जु॒हु॒या॒त् । प्रा॒णानिति॑ प्र - अ॒नान् । ए॒व । अ॒स्य॒ । अ॒न्नाद्य॒मित्य॑न्न - अद्य᳚म् । समिति॑ । त॒नो॒ति॒ । द्वाद॑श । द्वा॒द॒शानि॑ । जु॒हो॒ति॒ । द्वाद॑श । मासाः᳚ । सं॒ॅव॒थ्स॒र इति॑ सं-व॒थ्स॒रः । सं॒ॅव॒थ्स॒रेणेति॑ सं - व॒थ्स॒रेण॑ । ए॒व । अ॒स्मै॒ । अन्न᳚म् । अवेति॑ । रु॒न्धे॒ । अन्न᳚म् । च॒ । मे॒ । अक्षु॑त् । च॒ । मे॒ । इति॑ । आ॒ह॒ । ए॒तत् । वै ।  \newline




\markright{ TS 5.4.8.3  \hfill https://www.vedavms.in \hfill}

\section{ TS 5.4.8.3 }

\textbf{TS 5.4.8.3 } \newline
\textbf{Samhita Paata} \newline

अन्न॑स्य रू॒पꣳ रू॒पेणै॒वान्न॒मव॑ रुन्धे॒ ऽग्निश्च॑ म॒ आप॑श्च म॒ इत्या॑है॒षा वा अन्न॑स्य॒ योनिः॒ सयो᳚न्ये॒वान्न॒मव॑ रुन्धे-ऽर्द्धे॒न्द्राणि॑ जुहोति दे॒वता॑ ए॒वाव॑ रुन्धे॒ यथ् सर्वे॑षा-म॒र्द्धमिन्द्रः॒ प्रति॒ तस्मा॒दिन्द्रो॑ दे॒वता॑नां भूयिष्ठ॒भाक्त॑म॒ इन्द्र॒मुत्त॑रमाहे-न्द्रि॒यमे॒वास्मि॑-न्नु॒परि॑ष्टाद् दधाति यज्ञायु॒धानि॑ जुहोति य॒ज्ञो - [  ] \newline

\textbf{Pada Paata} \newline

अन्न॑स्य । रू॒पम् । रू॒पेण॑ । ए॒व । अन्न᳚म् । अवेति॑ । रु॒न्धे॒ । अ॒ग्निः । च॒ । मे॒ । आपः॑ । च॒ । मे॒ । इति॑ । आ॒ह॒ । ए॒षा । वै । अन्न॑स्य । योनिः॑ । सयो॒नीति॒ स - यो॒नि॒ । ए॒व । अन्न᳚म् । अवेति॑ । रु॒न्धे॒ । अ॒द्‌र्धे॒न्द्राणीत्य॑द्‌र्ध - इ॒न्द्राणि॑ । जु॒हो॒ति॒ । दे॒वताः᳚ । ए॒व । अवेति॑ । रु॒न्धे॒ । यत् । सर्वे॑षाम् । अ॒द्‌र्धम् । इन्द्रः॑ । प्रतीति॑ । तस्मा᳚त् । इन्द्रः॑ । दे॒वता॑नाम् । भू॒यि॒ष्ठ॒भाक्त॑म॒ इति॑ भूयिष्ठ॒भाक्-त॒मः॒ । इन्द्र᳚म् । उत्त॑र॒मित्युत् -त॒र॒म् । आ॒ह॒ । इ॒न्द्रि॒यम् । ए॒व । अ॒स्मि॒न्न् । उ॒परि॑ष्टात् । द॒धा॒ति॒ । य॒ज्ञा॒यु॒धानीति॑ यज्ञ्-आ॒यु॒धानि॑ । जु॒हो॒ति॒ । य॒ज्ञ्ः ।  \newline




\markright{ TS 5.4.8.4  \hfill https://www.vedavms.in \hfill}

\section{ TS 5.4.8.4 }

\textbf{TS 5.4.8.4 } \newline
\textbf{Samhita Paata} \newline

वै य॑ज्ञायु॒धा॑नि य॒ज्ञ्मे॒वाव॑ रु॒न्धेऽथो॑ ए॒तद्वै य॒ज्ञ्स्य॑ रू॒पꣳ रू॒पेणै॒व य॒ज्ञ्मव॑ रुन्धे ऽवभृ॒थश्च॑ मे स्वगाका॒रश्च॑ म॒ इत्या॑ह स्व॒गाकृ॑त्या अ॒ग्निश्च॑ मे घ॒र्मश्च॑ म॒ इत्या॑है॒तद् वै ब्र॑ह्मवर्च॒सस्य॑ रू॒पꣳ रू॒पेणै॒व ब्र॑ह्मवर्च॒समव॑ रुन्ध॒ ऋक्च॑ मे॒ साम॑ च म॒ इत्या॑है॒ - [  ] \newline

\textbf{Pada Paata} \newline

वै । य॒ज्ञा॒यु॒धानीति॑ यज्ञ्-आ॒यु॒धानि॑ । य॒ज्ञ्म् । ए॒व । अवेति॑ । रु॒न्धे॒ । अथो॒ इति॑ । ए॒तत् । वै । य॒ज्ञ्स्य॑ । रू॒पम् । रू॒पेण॑ । ए॒व । य॒ज्ञ्म् । अवेति॑ । रु॒न्धे॒ । अ॒व॒भृ॒थ इत्य॑व - भृ॒थः । च॒ । मे॒ । स्व॒गा॒का॒र इति॑ स्वगा - का॒रः । च॒ । मे॒ । इति॑ । आ॒ह॒ । स्व॒गाकृ॑त्या॒ इति॑ स्व॒गा - कृ॒त्यै॒ । अ॒ग्निः । च॒ । मे॒ । घ॒र्मः । च॒ । मे॒ । इति॑ । आ॒ह॒ । ए॒तत् । वै । ब्र॒ह्म॒व॒र्च॒सस्येति॑ ब्रह्म - व॒र्च॒सस्य॑ । रू॒पम् । रू॒पेण॑ । ए॒व । ब्र॒ह्म॒व॒र्च॒समिति॑ ब्रह्म - व॒र्च॒सम् । अवेति॑ । रु॒न्धे॒ । ऋक् । च॒ । मे॒ । साम॑ । च॒ । मे॒ । इति॑ । आ॒ह॒ ।  \newline




\markright{ TS 5.4.8.5  \hfill https://www.vedavms.in \hfill}

\section{ TS 5.4.8.5 }

\textbf{TS 5.4.8.5 } \newline
\textbf{Samhita Paata} \newline

-तद्वै छन्द॑साꣳ रू॒पꣳ रू॒पेणै॒व छन्दाꣳ॒॒स्यव॑ रुन्धे॒ गर्भा᳚श्च मे व॒थ्साश्च॑ म॒ इत्या॑है॒तद् वै प॑शू॒नाꣳ रू॒पꣳ रू॒पेणै॒व प॒शूनव॑ रुन्धे॒ कल्पा᳚न् जुहो॒त्य क्लृ॑प्तस्य॒ क्लृप्त्यै॑ युग्मदयु॒जे जु॑होति मिथुन॒त्वायो᳚-त्त॒राव॑ती भवतो॒ऽभिक्रा᳚न्त्या॒ एका॑ च मे ति॒स्रश्च॑ म॒ इत्या॑ह देवछन्द॒सं ॅवा एका॑ च ति॒स्रश्च॑ - [  ] \newline

\textbf{Pada Paata} \newline

ए॒तत् । वै । छन्द॑साम् । रू॒पम् । रू॒पेण॑ । ए॒व । छन्दाꣳ॑सि । अवेति॑ । रु॒न्धे॒ । गर्भाः᳚ । च॒ । मे॒ । व॒थ्साः । च॒ । मे॒ । इति॑ । आ॒ह॒ । ए॒तत् । वै । प॒शू॒नाम् । रू॒पम् । रू॒पेण॑ । ए॒व । प॒शून् । अवेति॑ । रु॒न्धे॒ । कल्पान्॑ । जु॒हो॒ति॒ । अक्लृ॑प्तस्य । क्लृप्त्यै᳚ । यु॒ग्म॒द॒यु॒जे इति॑ युग्मत् - अ॒यु॒जे । जु॒हो॒ति॒ । मि॒थु॒न॒त्वायेति॑ मिथुन - त्वाय॑ । उ॒त्त॒राव॑ती॒ इत्यु॑त्त॒रा-व॒ती॒ । भ॒व॒तः॒ । अ॒भिक्रा᳚न्त्या॒ इत्य॒भि - क्रा॒न्त्यै॒ । एका᳚ । च॒ । मे॒ । ति॒स्रः । च॒ । मे॒ । इति॑ । आ॒ह॒ । दे॒व॒छ॒न्द॒समिति॑ देव - छ॒न्द॒सम् । वै । एका᳚ । च॒ । ति॒स्रः । च॒ ।  \newline




\markright{ TS 5.4.8.6  \hfill https://www.vedavms.in \hfill}

\section{ TS 5.4.8.6 }

\textbf{TS 5.4.8.6 } \newline
\textbf{Samhita Paata} \newline

मनुष्यछन्द॒सं चत॑स्रश्चा॒ष्टौ च॑ देवछन्द॒सं चै॒व म॑नुष्य छन्द॒सश्चाव॑ रुन्ध॒ आ त्रय॑स्त्रिꣳ शतो जुहोति॒ त्रय॑स्त्रिꣳश॒द्वै दे॒वता॑ दे॒वता॑ ए॒वाव॑ रुन्ध॒ आऽष्टाच॑त्वारिꣳशतो जुहोत्य॒ष्टाच॑त्वारिꣳ-शदक्षरा॒ जग॑ती॒ जाग॑ताः प॒शवो॒ जग॑त्यै॒वास्मै॑ प॒शूनव॑ रुन्धे॒ वाज॑श्च प्रस॒वश्चेति॑ द्वाद॒शं जु॑होति॒ द्वाद॑श॒ मासाः᳚ संॅवथ्स॒रः सं॑ॅवथ्स॒र ए॒व प्रति॑ तिष्ठति ॥ \newline

\textbf{Pada Paata} \newline

म॒नु॒ष्य॒छ॒न्द॒समिति॑ मनुष्य - छ॒न्द॒सम् । चत॑स्रः । च॒ । अ॒ष्टौ । च॒ । दे॒व॒छ॒न्द॒समिति॑ देव - छ॒न्द॒सम् । च॒ । ए॒व । म॒नु॒ष्य॒छ॒न्द॒समिति॑ मनुष्य - छ॒न्द॒सम् । च॒ । अवेति॑ । रु॒न्धे॒ । एति॑ । त्रय॑स्त्रिꣳशत॒ इति॒ त्रयः॑ - त्रिꣳ॒॒श॒तः॒ । जु॒हो॒ति॒ । त्रय॑स्त्रिꣳश॒दिति॒ त्रयः॑ - त्रिꣳ॒॒श॒त् । वै । दे॒वताः᳚ । दे॒वताः᳚ । ए॒व । अवेति॑ । रु॒न्धे॒ । एति॑ । अ॒ष्टाच॑त्वारिꣳशत॒ इत्य॒ष्टा - च॒त्वा॒रिꣳ॒॒श॒तः॒ । जु॒हो॒ति॒ । अ॒ष्टाच॑त्वारिꣳशदक्ष॒रेत्य॒ष्टाच॑त्वारिꣳशत्-अ॒क्ष॒रा॒ । जग॑ती । जाग॑ताः । प॒शवः॑ । जग॑त्या । ए॒व । अ॒स्मै॒ । प॒शून् । अवेति॑ । रु॒न्धे॒ । वाजः॑ । च॒ । प्र॒स॒व इति॑ प्र-स॒वः । च॒ । इति॑ । द्वा॒द॒शम् । जु॒हो॒ति॒ । द्वाद॑श । मासाः᳚ । सं॒ॅव॒थ्स॒र इति॑ सं - व॒थ्स॒रः । सं॒ॅव॒थ्स॒र इति॑ सं - व॒थ्स॒रे । ए॒व । प्रतीति॑ । ति॒ष्ठ॒ति॒ ॥  \newline




\markright{ TS 5.4.9.1  \hfill https://www.vedavms.in \hfill}

\section{ TS 5.4.9.1 }

\textbf{TS 5.4.9.1 } \newline
\textbf{Samhita Paata} \newline

अ॒ग्निर्दे॒वेभ्यो ऽपा᳚क्रामद्-भाग॒धेय॑मि॒च्छमा॑न॒स्तं दे॒वा अ॑ब्रुव॒न्नुप॑ न॒ आ व॑र्तस्व ह॒व्यं नो॑ व॒हेति॒ सो᳚ऽब्रवी॒द्-वरं॑ ॅवृणै॒ मह्य॑मे॒व वा॑जप्रस॒वीयं॑ जुहव॒न्निति॒ तस्मा॑द॒ग्नये॑ वाजप्रस॒वीयं॑ जुह्वति॒ यद्-वा॑जप्रस॒वीयं॑ जु॒होत्य॒ग्निमे॒व तद्-भा॑ग॒धेये॑न॒ सम॑र्द्धय॒त्यथो॑ अभिषे॒क ए॒वास्य॒ स च॑तुर्द॒शभि॑र्जुहोति स॒प्त ग्रा॒म्या ओष॑धयः स॒प्ता - [  ] \newline

\textbf{Pada Paata} \newline

अ॒ग्निः । दे॒वेभ्यः॑ । अपेति॑ । अ॒क्रा॒म॒त् । भा॒ग॒धेय॒मिति॑ भाग-धेय᳚म् । इ॒च्छमा॑नः । तम् । दे॒वाः । अ॒ब्रु॒व॒न्न् । उपेति॑ । नः॒ । एति॑ । व॒र्त॒स्व॒ । ह॒व्यम् । नः॒ । व॒ह॒ । इति॑ । सः । अ॒ब्र॒वी॒त् । वर᳚म् । वृ॒णै॒ । मह्य᳚म् । ए॒व । वा॒ज॒प्र॒स॒वीय॒मिति॑ वाज - प्र॒स॒वीय᳚म् । जु॒ह॒व॒न्न् । इति॑ । तस्मा᳚त् । अ॒ग्नये᳚ । वा॒ज॒प्र॒स॒वीय॒मिति॑ वाज - प्र॒स॒वीय᳚म् । जु॒ह॒ति॒ । यत् । वा॒ज॒प्र॒स॒वीय॒मिति॑ वाज - प्र॒स॒वीय᳚म् । जु॒होति॑ । अ॒ग्निम् । ए॒व । तत् । भा॒ग॒धेये॒नेति॑ भाग - धेये॑न । समिति॑ । अ॒द्‌र्ध॒य॒ति॒ । अथो॒ इति॑ । अ॒भि॒षे॒क इत्य॑भि - से॒कः । ए॒व । अ॒स्य॒ । सः । च॒तु॒र्द॒शभि॒रिति॑ चतुर्द॒श - भिः॒ । जु॒हो॒ति॒ । स॒प्त । ग्रा॒म्याः । ओष॑धयः । स॒प्त ।  \newline




\markright{ TS 5.4.9.2  \hfill https://www.vedavms.in \hfill}

\section{ TS 5.4.9.2 }

\textbf{TS 5.4.9.2 } \newline
\textbf{Samhita Paata} \newline

-*ऽऽर॒ण्या उ॒भयी॑षा॒मव॑रुद्ध्या॒ अन्न॑स्यान्नस्य जुहो॒त्यन्न॑स्यान्न॒स्या-व॑रुद्ध्या॒ औदु॑म्बरेण स्रु॒वेण॑ जुहो॒त्यूर्ग्वा उ॑दु॒म्बर॒ ऊर्गन्न॑मू॒र्जैवास्मा॒ ऊर्ज॒मन्न॒मव॑ रुन्धे॒ ऽग्निर्वै दे॒वाना॑-म॒भिषि॑क्तोऽग्नि॒चिन्-म॑नु॒ष्या॑णां॒ तस्मा॑दग्नि॒चिद्-वर्.ष॑ति॒ न धा॑वे॒दव॑रुद्धꣳ॒॒ ह्य॑स्या-न्न॒मन्न॑मिव॒ खलु॒ वै व॒र्॒.षं ॅयद्धावे॑द॒न्नाद्-या᳚द्धावेदु॒पाव॑र्तेता॒-न्नाद्य॑मे॒वाभ्यु॒ - [  ] \newline

\textbf{Pada Paata} \newline

आ॒र॒ण्याः । उ॒भयी॑षाम् । अव॑रुद्ध्या॒ इत्यव॑ - रु॒द्ध्यै॒ । अन्न॑स्यान्न॒स्येत्यन्न॑स्य - अ॒न्न॒स्य॒ । जु॒हो॒ति॒ । अन्न॑स्यान्न॒स्येत्यन्न॑स्य - अ॒न्न॒स्य॒ । अव॑रुद्ध्या॒ इत्यव॑ - रु॒द्ध्यै॒ । औदु॑बंरेण । स्रु॒वेण॑ । जु॒हो॒ति॒ । ऊर्क् । वै । उ॒दु॒बंरः॑ । ऊर्क् । अन्न᳚म् । ऊ॒र्जा । ए॒व । अ॒स्मै॒ । ऊर्ज᳚म् । अन्न᳚म् । अवेति॑ । रु॒न्धे॒ । अ॒ग्निः । वै । दे॒वाना᳚म् । अ॒भिषि॑क्त॒ इत्य॒भि - सि॒क्तः॒ । अ॒ग्नि॒चिदित्य॑ग्नि - चित् । म॒नु॒ष्या॑णाम् । तस्मा᳚त् । अ॒ग्नि॒चिदित्य॑ग्नि - चित् । वर्.ष॑ति । न । धा॒वे॒त् । अव॑रुद्ध॒मित्यव॑ - रु॒द्ध॒म् । हि । अ॒स्य॒ । अन्न᳚म् । अन्न᳚म् । इ॒व॒ । खलु॑ । वै । व॒र्॒.षम् । यत् । धावे᳚त् । अ॒न्नाद्या॒दित्य॑न्न - अद्या᳚त् । धा॒वे॒त् । उ॒पाव॑र्ते॒तेत्य॑प-आव॑र्तेत । अ॒न्नाद्य॒मित्य॑न्न - अद्य᳚म् । ए॒व । अ॒भीति॑ ।  \newline




\markright{ TS 5.4.9.3  \hfill https://www.vedavms.in \hfill}

\section{ TS 5.4.9.3 }

\textbf{TS 5.4.9.3 } \newline
\textbf{Samhita Paata} \newline

-पाव॑र्तते॒ नक्तो॒षासेति॑ कृ॒ष्णायै᳚ श्वे॒तव॑थ्सायै॒ पय॑सा जुहो॒त्यह्नै॒वास्मै॒ रात्रिं॒ प्रदा॑पयति॒ रात्रि॒याऽह॑रहोरा॒त्रे ए॒वास्मै॒ प्रत्ते॒ काम॑म॒न्नाद्यं॑ दुहाते राष्ट्र॒भृतो॑ जुहोति रा॒ष्ट्रमे॒वाव॑ रुन्धे ष॒ड्भिर्जु॑होति॒ षड्वा ऋ॒तव॑ ऋ॒तुष्वे॒व प्रति॑तिष्ठति॒ भुव॑नस्य पत॒ इति॑ रथमु॒खे पञ्चाऽऽ*हु॑तीर्जुहोति॒ वज्रो॒ वै रथो॒ वज्रे॑णै॒व दिशो॒ - [  ] \newline

\textbf{Pada Paata} \newline

उ॒पाव॑र्तत॒ इत्यु॑प - आव॑र्तते । नक्तो॒षासा᳚ । इति॑ । कृ॒ष्णायै᳚ । श्वे॒तव॑थ्साया॒ इति॑ श्वे॒त - व॒थ्सा॒यै॒ । पय॑सा । जु॒हो॒ति॒ । अह्ना᳚ । ए॒व । अ॒स्मै॒ । रात्रि᳚म् । प्रेति॑ । दा॒प॒य॒ति॒ । रात्रि॑या । अहः॑ । अ॒हो॒रा॒त्रे इत्य॑हः - रा॒त्रे । ए॒व । अ॒स्मै॒ । प्रत्ते॒ इति॑ । काम᳚म् । अ॒न्नाद्य॒मित्य॑न्न - अद्य᳚म् । दु॒हा॒ते॒ इति॑ । रा॒ष्ट्र॒भृत॒ इति॑ राष्ट्र - भृतः॑ । जु॒हो॒ति॒ । रा॒ष्ट्रम् । ए॒व । अवेति॑ । रु॒न्धे॒ । ष॒ड्भिरिति॑ षट् - भिः । जु॒हो॒ति॒ । षट् । वै । ऋ॒तवः॑ । ऋ॒तुषु॑ । ए॒व । प्रतीति॑ । ति॒ष्ठ॒ति॒ । भुव॑नस्य । प॒ते॒ । इति॑ । र॒थ॒मु॒ख इति॑ रथ - मु॒खे । पञ्च॑ । आहु॑ती॒रित्या - हु॒तीः॒ । जु॒हो॒ति॒ । वज्रः॑ । वै । रथः॑ । वज्रे॑ण । ए॒व । दिशः॑ ।  \newline




\markright{ TS 5.4.9.4  \hfill https://www.vedavms.in \hfill}

\section{ TS 5.4.9.4 }

\textbf{TS 5.4.9.4 } \newline
\textbf{Samhita Paata} \newline

-ऽभि ज॑यत्यग्नि॒चितꣳ॑ ह॒ वा अ॒मुष्मि॑न् ॅलो॒के वातो॒ऽभि प॑वते वातना॒मानि॑ जुहोत्य॒भ्ये॑वैन॑-म॒मुष्मि॑न् ॅलो॒के वातः॑ पवते॒ त्रीणि॑ जुहोति॒ त्रय॑ इ॒मे लो॒का ए॒भ्य ए॒व लो॒केभ्यो॒ वात॒मव॑ रुन्धे समु॒द्रो॑ऽसि॒ नभ॑स्वा॒नित्या॑है॒तद्वै वात॑स्य रू॒पꣳ रू॒पेणै॒व वात॒मव॑ रुन्धे ऽञ्ज॒लिना॑ जुहोति॒ न ह्ये॑तेषा॑म॒न्यथा ( ) ऽऽहु॑तिरव॒कल्प॑ते ॥ \newline

\textbf{Pada Paata} \newline

अ॒भीति॑ । ज॒य॒ति॒ । अ॒ग्नि॒चित॒मित्य॑ग्नि - चित᳚म् । ह॒ । वै । अ॒मुष्मिन्न्॑ । लो॒के । वातः॑ । अ॒भीति॑ । प॒व॒ते॒ । वा॒त॒ना॒मानीति॑ वात - ना॒मानि॑ । जु॒हो॒ति॒ । अ॒भीति॑ । ए॒व । ए॒न॒म् । अ॒मुष्मिन्न्॑ । लो॒के । वातः॑ । प॒व॒ते॒ । त्रीणि॑ । जु॒हो॒ति॒ । त्रयः॑ । इ॒मे । लो॒काः । ए॒भ्यः । ए॒व । लो॒केभ्यः॑ । वात᳚म् । अवेति॑ । रु॒न्धे॒ । स॒मु॒द्रः । अ॒सि॒ । नभ॑स्वान् । इति॑ । आ॒ह॒ । ए॒तत् । वै । वात॑स्य । रू॒पम् । रू॒पेण॑ । ए॒व । वात᳚म् । अवेति॑ । रु॒न्धे॒ । अ॒ञ्ज॒लिना᳚ । जु॒हो॒ति॒ । न । हि । ए॒तेषा᳚म् । अ॒न्यथा᳚ ( ) । आहु॑ति॒रित्या - हु॒तिः॒ । अ॒व॒कल्प॑त॒ इत्य॑व - कल्प॑ते ॥  \newline




\markright{ TS 5.4.10.1  \hfill https://www.vedavms.in \hfill}

\section{ TS 5.4.10.1 }

\textbf{TS 5.4.10.1 } \newline
\textbf{Samhita Paata} \newline

सु॒व॒र्गाय॒ वै लो॒काय॑ देवर॒थो यु॑ज्यते यत्राकू॒ताय॑ मनुष्यर॒थ ए॒ष खलु॒ वै दे॑वर॒थो यद॒ग्निर॒ग्निं ॅयु॑नज्मि॒ शव॑सा घृ॒तेनेत्या॑ह यु॒नक्त्ये॒वैनꣳ॒॒ स ए॑नं ॅयु॒क्तः सु॑व॒र्गं ॅलो॒कम॒भि व॑हति॒ यथ् सर्वा॑भिः प॒ञ्चभि॑-र्यु॒ञ्ज्याद्-यु॒क्तो᳚ऽस्या॒ऽग्निः प्रच्यु॑तः स्या॒दप्र॑तिष्ठिता॒ आहु॑तयः॒ स्युरप्र॑तिष्ठिताः॒ स्तोमा॒ अप्र॑तिष्ठितान्यु॒क्थानि॑ ति॒सृभिः॑ प्रातस्सव॒ने॑ऽभि मृ॑शति त्रि॒वृ - [  ] \newline

\textbf{Pada Paata} \newline

सु॒व॒र्गायेति॑ सुवः - गाय॑ । वै । लो॒काय॑ । दे॒व॒र॒थ इति॑ देव - र॒थः । यु॒ज्य॒ते॒ । य॒त्रा॒कू॒तायेति॑ यत्र - आ॒कू॒ताय॑ । म॒नु॒ष्य॒र॒थ इति॑ मनुष्य - र॒थः । ए॒षः । खलु॑ । वै । दे॒व॒र॒थ इति॑ देव - र॒थः । यत् । अ॒ग्निः । अ॒ग्निम् । यु॒न॒ज्मि॒ । शव॑सा । घृ॒तेन॑ । इति॑ । आ॒ह॒ । यु॒नक्ति॑ । ए॒व । ए॒न॒म् । सः । ए॒न॒म् । यु॒क्तः । सु॒व॒र्गमिति॑ सुवः-गम् । लो॒कम् । अ॒भीति॑ । व॒ह॒ति॒ । यत् । सर्वा॑भिः । प॒ञ्चभि॒रिति॑ प॒ञ्च - भिः॒ । यु॒ञ्ज्यात् । यु॒क्तः । अ॒स्य॒ । अ॒ग्निः । प्रच्यु॑त॒ इति॒ प्र - च्यु॒तः॒ । स्या॒त् । अप्र॑तिष्ठिता॒ इत्यप्र॑ति - स्थि॒ताः॒ । आहु॑तय॒ इत्या - हु॒त॒यः॒ । स्युः । अप्र॑तिष्ठिता॒ इत्यप्र॑ति - स्थि॒ताः॒ । स्तोमाः᳚ । अप्र॑तिष्ठिता॒नीत्यप्र॑ति - स्थि॒ता॒नि॒ । उ॒क्थानि॑ । ति॒सृभि॒रिति॑ ति॒सृ - भिः॒ । प्रा॒त॒स्स॒व॒न इति॑ प्रातः - स॒व॒ने । अ॒भीति॑ । मृ॒श॒ति॒ । त्रि॒वृदिति॑ त्रि - वृत् ।  \newline




\markright{ TS 5.4.10.2  \hfill https://www.vedavms.in \hfill}

\section{ TS 5.4.10.2 }

\textbf{TS 5.4.10.2 } \newline
\textbf{Samhita Paata} \newline

-द्वा अ॒ग्निर्यावा॑ने॒वा-ग्निस्तं ॅयु॑नक्ति॒ यथाऽन॑सि यु॒क्त आ॑धी॒यत॑ ए॒वमे॒व तत् प्रत्याहु॑तय॒स्तिष्ठ॑न्ति॒ प्रति॒ स्तोमाः॒ प्रत्यु॒क्थानि॑ यज्ञाय॒ज्ञिय॑स्य स्तो॒त्रे द्वाभ्या॑म॒भि मृ॑शत्ये॒तावा॒न्॒ वै य॒ज्ञो यावा॑नग्निष्टो॒मो भू॒मा त्वा अ॒स्यात॑ ऊ॒र्द्ध्वः क्रि॑यते॒ यावा॑ने॒व य॒ज्ञ्स्तम॑न्त॒तो᳚ ऽन्वारो॑हति॒ द्वाभ्यां॒ प्रति॑ष्ठित्या॒ एक॒याऽप्र॑स्तुतं॒ भव॒त्यथा॒ - [  ] \newline

\textbf{Pada Paata} \newline

वै । अ॒ग्निः । यावान्॑ । ए॒व । अ॒ग्निः । तम् । यु॒न॒क्ति॒ । यथा᳚ । अन॑सि । यु॒क्ते । आ॒धी॒यत॒ इत्या᳚ - धी॒यते᳚ । ए॒वम् । ए॒व । तत् । प्रतीति॑ । आहु॑तय॒ इत्या - हु॒त॒यः॒ । तिष्ठ॑न्ति । प्रतीति॑ । स्तोमाः᳚ । प्रतीति॑ । उ॒क्थानि॑ । य॒ज्ञा॒य॒ज्ञिय॑स्य । स्तो॒त्रे । द्वाभ्या᳚म् । अ॒भीति॑ । मृ॒श॒ति॒ । ए॒तावान्॑ । वै । य॒ज्ञ्ः । यावान्॑ । अ॒ग्नि॒ष्टो॒म इत्य॑ग्नि - स्तो॒मः । भू॒मा । तु । वै । अ॒स्य॒ । अतः॑ । ऊ॒द्‌र्ध्वः । क्रि॒य॒ते॒ । यावान्॑ । ए॒व । य॒ज्ञ्ः । तम् । अ॒न्त॒तः । अ॒न्वारो॑ह॒तीत्य॑नु - आरो॑हति । द्वाभ्या᳚म् । प्रति॑ष्ठित्या॒ इति॒ प्रति॑ - स्थि॒त्यै॒ । एक॑या । अप्र॑स्तुत॒मित्यप्र॑-स्तु॒त॒म् । भव॑ति । अथ॑ ।  \newline




\markright{ TS 5.4.10.3  \hfill https://www.vedavms.in \hfill}

\section{ TS 5.4.10.3 }

\textbf{TS 5.4.10.3 } \newline
\textbf{Samhita Paata} \newline

-भि मृ॑श॒त्युपै॑न॒मुत्त॑रो य॒ज्ञो न॑म॒त्यथो॒ संत॑त्यै॒ प्र वा ए॒षो᳚ऽस्मान् ॅलो॒काच्च्य॑वते॒ यो᳚ऽग्निं चि॑नु॒ते न वा ए॒तस्या॑निष्ट॒क आहु॑ति॒रव॑ कल्पते॒ यां ॅवा ए॒षो॑ऽनिष्ट॒क आहु॑तिं जु॒होति॒ स्रव॑ति॒ वै सा ताꣳ स्रव॑न्तीं ॅय॒ज्ञोऽनु॒ परा॑ भवति य॒ज्ञ्ं ॅयज॑मानो॒ यत् पु॑नश्चि॒तिं चि॑नु॒त आहु॑तीनां॒ प्रति॑ष्ठित्यै॒ प्रत्याहु॑तय॒स्तिष्ठ॑न्ति॒ - [  ] \newline

\textbf{Pada Paata} \newline

अ॒भीति॑ । मृ॒श॒ति॒ । उपेति॑ । ए॒न॒म् । उत्त॑र॒ इत्युत् - त॒रः॒ । य॒ज्ञ्ः । न॒म॒ति॒ । अथो॒ इति॑ । संत॑त्या॒ इति॒ सं - त॒त्यै॒ । प्रेति॑ । वै । ए॒षः । अ॒स्मात् । लो॒कात् । च्य॒व॒ते॒ । यः । अ॒ग्निम् । चि॒नु॒ते । न । वै । ए॒तस्य॑ । अ॒नि॒ष्ट॒के । आहु॑ति॒रित्या - हु॒तिः॒ । अवेति॑ । क॒ल्प॒ते॒ । याम् । वै । ए॒षः । अ॒नि॒ष्ट॒के । आहु॑ति॒मित्या - हु॒ति॒म् । जु॒होति॑ । स्रव॑ति । वै । सा । ताम् । स्रव॑न्तीम् । य॒ज्ञ्ः । अनु॑ । परेति॑ । भ॒व॒ति॒ । य॒ज्ञ्म् । यज॑मानः । यत् । पु॒न॒श्चि॒तिमिति॑ पुनः - चि॒तिम् । चि॒नु॒ते । आहु॑तीना॒मित्या-हु॒ती॒ना॒म् । प्रति॑ष्ठित्या॒ इति॒ प्रति॑-स्थि॒त्यै॒ । प्रतीति॑ । आहु॑तय॒ इत्या - हु॒त॒यः॒ । तिष्ठ॑न्ति ।  \newline




\markright{ TS 5.4.10.4  \hfill https://www.vedavms.in \hfill}

\section{ TS 5.4.10.4 }

\textbf{TS 5.4.10.4 } \newline
\textbf{Samhita Paata} \newline

न य॒ज्ञ्ः प॑रा॒भव॑ति॒ न यज॑मानो॒ ऽष्टावुप॑ दधात्य॒ष्टाक्ष॑रा गाय॒त्री गा॑य॒त्रेणै॒वैनं॒ छन्द॑सा चिनुते॒ यदेका॑दश॒ त्रैष्टु॑भेन॒ यद् द्वाद॑श॒ जाग॑तेन॒ छन्दो॑भिरे॒वैनं॑ चिनुते नपा॒त्को वैनामै॒षो᳚ऽग्निर्यत् पु॑नश्चि॒तिर्य ए॒वं ॅवि॒द्वान् पु॑नश्चि॒तिं चि॑नु॒त आ तृ॒तीया॒त् पुरु॑षा॒दन्न॑मत्ति॒ यथा॒ वै पु॑नरा॒धेय॑ ए॒वं पु॑नश्चि॒तिर्यो᳚ऽग्न्या॒धेये॑न॒ न - [  ] \newline

\textbf{Pada Paata} \newline

न । य॒ज्ञ्ः । प॒रा॒भव॒तीति॑ परा - भव॑ति । न । यज॑मानः । अ॒ष्टौ । उपेति॑ । द॒धा॒ति॒ । अ॒ष्टाक्ष॒रेत्य॒ष्टा - अ॒क्ष॒रा॒ । गा॒य॒त्री । गा॒य॒त्रेण॑ । ए॒व । ए॒न॒म् । छन्द॑सा । चि॒नु॒ते॒ । यत् । एका॑दश । त्रैष्टु॑भेन । यत् । द्वाद॑श । जाग॑तेन । छन्दो॑भि॒रिति॒ छन्दः॑ - भिः॒ । ए॒व । ए॒न॒म् । चि॒नु॒ते॒ । न॒पा॒त्कः । वै । नाम॑ । ए॒षः । अ॒ग्निः । यत् । पु॒न॒श्चि॒तिरिति॑ पुनः - चि॒तिः । यः । ए॒वम् । वि॒द्वान् । पु॒न॒श्चि॒तिमिति॑ पुनः - चि॒तिम् । चि॒नु॒ते । एति॑ । तृ॒तीया᳚त् । पुरु॑षात् । अन्न᳚म् । अ॒त्ति॒ । यथा᳚ । वै । पु॒न॒रा॒धेय॒ इति॑ पुनः - आ॒धेयः॑ । ए॒वम् । पु॒न॒श्चि॒तिरिति॑ पुनः - चि॒तिः । यः । अ॒ग्न्या॒धेये॒नेत्य॑ग्नि-आ॒धेये॑न । न ।  \newline




\markright{ TS 5.4.10.5  \hfill https://www.vedavms.in \hfill}

\section{ TS 5.4.10.5 }

\textbf{TS 5.4.10.5 } \newline
\textbf{Samhita Paata} \newline

र्ध्नोति॒ स पु॑नरा॒धेय॒मा ध॑त्ते॒ यो᳚ऽग्निं चि॒त्वा नर्द्ध्नोति॒ स पु॑नश्चि॒तिं चि॑नुते॒ यत् पु॑नश्चि॒तिं चि॑नु॒त ऋद्ध्या॒ अथो॒ खल्वा॑हु॒र्न चे॑त॒व्येति॑ रु॒द्रो वा ए॒ष यद॒ग्निर्यथा᳚ व्या॒घ्रꣳ सु॒प्तं बो॒धय॑ति ता॒दृगे॒व तदथो॒ खल्वा॑हुश्चेत॒व्येति॒ यथा॒ वसी॑याꣳसं भाग॒धेये॑न बो॒धय॑ति ता॒दृगे॒व तन्मनु॑र॒ग्निम॑चिनुत॒ ( ) तेन॒ नाऽऽ*र्द्ध्नो॒थ्स ए॒तां पु॑नश्चि॒तिम॑पश्य॒त् ताम॑चिनुत॒ तया॒ वै स आ᳚र्द्ध्नो॒द्यत् पु॑नश्चि॒तिं चि॑नु॒त ऋद्ध्यै᳚ ॥ \newline

\textbf{Pada Paata} \newline

ऋ॒ध्नोति॑ । सः । पु॒न॒रा॒धेय॒मिति॑ पुनः-आ॒धेय᳚म् । एति॑ । ध॒त्ते॒ । यः । अ॒ग्निम् । चि॒त्वा । न । ऋ॒द्ध्नोति॑ । सः । पु॒न॒श्चि॒तिमिति॑ पुनः - चि॒तिम् । चि॒नु॒ते॒ । यत् । पु॒न॒श्चि॒तिमिति॑ पुनः - चि॒तिम् । चि॒नु॒ते । ऋद्ध्यै᳚ । अथो॒ इति॑ । खलु॑ । आ॒हुः॒ । न । चे॒त॒व्या᳚ । इति॑ । रु॒द्रः । वै । ए॒षः । यत् । अ॒ग्निः । यथा᳚ । व्या॒घ्रम् । सु॒प्तम् । बो॒धय॑ति । ता॒दृक् । ए॒व । तत् । अथो॒ इति॑ । खलु॑ । आ॒हुः॒ । चे॒त॒व्या᳚ । इति॑ । यथा᳚ । वसी॑याꣳसम् । भा॒ग॒धेये॒नेति॑ भाग - धेये॑न । बो॒धय॑ति । ता॒दृक् । ए॒व । तत् । मनुः॑ । अ॒ग्निम् । अ॒चि॒नु॒त॒ ( ) । तेन॑ । न । आ॒द्‌र्ध्नो॒त् । सः । ए॒ताम् । पु॒न॒श्चि॒तिमिति॑ पुनः - चि॒तिम् । अ॒प॒श्य॒त् । ताम् । अ॒चि॒नु॒त॒ । तया᳚ । वै । सः । आ॒द्‌र्ध्नो॒त् । यत् । पु॒न॒श्चि॒तिमिति॑ पुनः - चि॒तिम् । चि॒नु॒ते । ऋद्ध्यै᳚ ॥  \newline




\markright{ TS 5.4.11.1  \hfill https://www.vedavms.in \hfill}

\section{ TS 5.4.11.1 }

\textbf{TS 5.4.11.1 } \newline
\textbf{Samhita Paata} \newline

छ॒न्द॒श्चितं॑ चिन्वीत प॒शुका॑मः प॒शवो॒ वै छन्दाꣳ॑सि पशु॒माने॒व भ॑वति श्येन॒चितं॑ चिन्वीत सुव॒र्गका॑मः श्ये॒नो वै वय॑सां॒ पति॑ष्ठः श्ये॒न ए॒व भू॒त्वा सु॑व॒र्गं ॅलो॒कं प॑तति कङ्क॒चितं॑ चिन्वीत॒ यः का॒मये॑त शीर्.ष॒ण्वान॒मुष्मि॑न् ॅलो॒के स्या॒मिति॑ शीर्.ष॒ण्वाने॒वाऽमुष्मि॑न् ॅलो॒के भ॑वत्यलज॒चितं॑ चिन्वीत॒ चतु॑स्सीतं प्रति॒ष्ठाका॑म॒श्चत॑स्रो॒ दिशो॑ दि॒क्ष्वे॑व प्रति॑ तिष्ठति प्रौग॒चितं॑ चिन्वीत॒ भ्रातृ॑व्यवा॒न् प्रै - [  ] \newline

\textbf{Pada Paata} \newline

छ॒न्द॒श्चित॒मिति॑ छन्दः - चित᳚म् । चि॒न्वी॒त॒ । प॒शुका॑म॒ इति॑ प॒शु - का॒मः॒ । प॒शवः॑ । वै । छन्दाꣳ॑सि । प॒शु॒मानिति॑ पशु-मान् । ए॒व । भ॒व॒ति॒ । श्ये॒न॒चित॒मिति॑ श्येन - चित᳚म् । चि॒न्वी॒त॒ । सु॒व॒र्गका॑म॒ इति॑ सुव॒र्ग - का॒मः॒ । श्ये॒नः । वै । वय॑साम् । पति॑ष्ठः । श्ये॒नः । ए॒व । भू॒त्वा । सु॒व॒र्गमिति॑ सुवः - गम् । लो॒कम् । प॒त॒ति॒ । क॒ङ्क॒चित॒मिति॑ कङ्क - चित᳚म् । चि॒न्वी॒त॒ । यः । का॒मये॑त । शी॒र्.॒ष॒ण्वानिति॑ शीर्.षण् - वान् । अ॒मुष्मिन्न्॑ । लो॒के । स्या॒म् । इति॑ । शी॒र्.॒ष॒ण्वानिति॑ शीर्.षण् - वान् । ए॒व । अ॒मुष्मिन्न्॑ । लो॒के । भ॒व॒ति॒ । अ॒ल॒ज॒चित॒मित्य॑लज - चित᳚म् । चि॒न्वी॒त॒ । चतु॑स्सीत॒मिति॒ चतुः॑ - सी॒त॒म् । प्र॒ति॒ष्ठाका॑म॒ इति॑ प्रति॒ष्ठा-का॒मः॒ । चत॑स्रः । दिशः॑ । दि॒क्षु । ए॒व । प्रतीति॑ । ति॒ष्ठ॒ति॒ । प्र॒उ॒ग॒चित॒मिति॑ प्र‌उग - चित᳚म् । चि॒न्वी॒त॒ । भ्रातृ॑व्यवा॒निति॒ भ्रातृ॑व्य - वान् । प्रेति॑ ।  \newline




\markright{ TS 5.4.11.2  \hfill https://www.vedavms.in \hfill}

\section{ TS 5.4.11.2 }

\textbf{TS 5.4.11.2 } \newline
\textbf{Samhita Paata} \newline

-व भ्रातृ॑व्यान् नुदत उभ॒यतः॑ प्रौगं चिन्वीत॒यः का॒मये॑त॒ प्रजा॒तान् भ्रातृ॑व्यान् नु॒देय॒ प्रति॑ जनि॒ष्यमा॑णा॒निति॒ प्रैव जा॒तान् भ्रातृ॑व्यान् नु॒दते॒ प्रति॑ जनि॒ष्यमा॑णान् रथचक्र॒चितं॑ चिन्वीत॒ भ्रातृ॑व्यवा॒न्॒ वज्रो॒ वै रथो॒ वज्र॑मे॒व भ्रातृ॑व्येभ्यः॒ प्रह॑रति द्रोण॒चितं॑ चिन्वी॒तान्न॑कामो॒ द्रोणे॒ वा अन्नं॑ भ्रियते॒ सयो᳚न्ये॒वान्न॒मव॑ रुन्धे समू॒ह्यं॑ चिन्वीत प॒शुका॑मः पशु॒माने॒व भ॑वति - [  ] \newline

\textbf{Pada Paata} \newline

ए॒व । भ्रातृ॑व्यान् । नु॒द॒ते॒ । उ॒भ॒यतः॑ प्र‌उग॒मित्यु॑भ॒यतः॑ - प्र॒उ॒ग॒म् । चि॒न्वी॒त॒ । यः । का॒मये॑त । प्रेति॑ । जा॒तान् । भ्रातृ॑व्यान् । नु॒देय॑ । प्रतीति॑ । ज॒नि॒ष्यमा॑णान् । इति॑ । प्रेति॑ । ए॒व । जा॒तान् । भ्रातृ॑व्यान् । नु॒दते᳚ । प्रतीति॑ । ज॒नि॒ष्यमा॑णान् । र॒थ॒च॒क्र॒चित॒मिति॑ रथचक्र - चित᳚म् । चि॒न्वी॒त॒ । भ्रातृ॑व्यवा॒निति॒ भ्रातृ॑व्य - वा॒न् । वज्रः॑ । वै । रथः॑ । वज्र᳚म् । ए॒व । भ्रातृ॑व्येभ्यः । प्रेति॑ । ह॒र॒ति॒ । द्रो॒ण॒चित॒मिति॑ द्रोण - चित᳚म् । चि॒न्वी॒त॒ । अन्न॑काम॒ इत्यन्न॑-का॒मः॒ । द्रोणे᳚ । वै । अन्न᳚म् । भ्रि॒य॒ते॒ । सयो॒नीति॒ स-यो॒नि॒ । ए॒व । अन्न᳚म् । अवेति॑ । रु॒न्धे॒ । स॒मू॒ह्य॑मिति॑ सं - ऊ॒ह्य᳚म् । चि॒न्वी॒त॒ । प॒शुका॑म॒ इति॑ प॒शु - का॒मः॒ । प॒शु॒मानिति॑ पशु -मान् । ए॒व । भ॒व॒ति॒ ।  \newline




\markright{ TS 5.4.11.3  \hfill https://www.vedavms.in \hfill}

\section{ TS 5.4.11.3 }

\textbf{TS 5.4.11.3 } \newline
\textbf{Samhita Paata} \newline

परिचा॒य्यं॑ चिन्वीत॒ ग्राम॑कामो ग्रा॒म्ये॑व भ॑वति श्मशान॒चितं॑ चिन्वीत॒ यः का॒मये॑त पितृलो॒क ऋ॑द्ध्नुया॒मिति॑ पितृलो॒क ए॒वर्द्ध्नो॑ति विश्वामित्रजमद॒ग्नी वसि॑ष्ठेनाऽस्पर्द्धेताꣳ॒॒ स ए॒ता ज॒मद॑ग्निर्विह॒व्या॑ अपश्य॒त् ता उपा॑धत्त॒ ताभि॒र्वै स वसि॑ष्ठस्येन्द्रि॒यं ॅवी॒र्य॑मवृङ्क्त॒ यद्-वि॑ह॒व्या॑ उप॒दधा॑तीन्द्रि॒यमे॒व ताभि॑र्वी॒र्यं॑ ॅयज॑मानो॒ भ्रातृ॑व्यस्य वृङ्क्ते॒ होतु॒र्द्धिष्णि॑य॒ उप॑ दधाति यजमानायत॒नं ॅवै - [  ] \newline

\textbf{Pada Paata} \newline

प॒रि॒चा॒य्य॑मिति॑ परि - चा॒य्य᳚म् । चि॒न्वी॒त॒ । ग्राम॑काम॒ इति॒ ग्राम॑ - का॒मः॒ । ग्रा॒मी । ए॒व । भ॒व॒ति॒ । श्म॒शा॒न॒चित॒मिति॑ श्मशान - चित᳚म् । चि॒न्वी॒त॒ । यः । का॒मये॑त । पि॒तृ॒लो॒क इति॑ पितृ - लो॒के । ऋ॒द्ध्नु॒या॒म् । इति॑ । पि॒तृ॒लो॒क इति॑ पितृ - लो॒के । ए॒व । ऋ॒द्ध्नो॒ति॒ । वि॒श्वा॒मि॒त्र॒ज॒म॒द॒ग्नी इति॑ विश्वामित्र - ज॒म॒द॒ग्नी । वसि॑ष्ठेन । अ॒स्प॒द्‌र्धे॒ता॒म् । सः । ए॒ताः । ज॒मद॑ग्निः । वि॒ह॒व्या॑ इति॑ वि - ह॒व्याः᳚ । अ॒प॒श्य॒त् । ताः । उपेति॑ । अ॒ध॒त्त॒ । ताभिः॑ । वै । सः । वसि॑ष्ठस्य । इ॒न्द्रि॒यम् । वी॒र्य᳚म् । अ॒वृ॒ङ्क्त॒ । यत् । वि॒ह॒व्या॑ इति॑ वि - ह॒व्याः᳚ । उ॒प॒दधा॒तीत्यु॑प - दधा॑ति । इ॒न्द्रि॒यम् । ए॒व । ताभिः॑ । वी॒र्य᳚म् । यज॑मानः । भ्रातृ॑व्यस्य । वृ॒ङ्क्ते॒ । होतुः॑ । धिष्णि॑ये । उपेति॑ । द॒धा॒ति॒ । य॒ज॒मा॒ना॒य॒त॒नमिति॑ यजमान - आ॒य॒त॒नम् । वै ।  \newline




\markright{ TS 5.4.11.4  \hfill https://www.vedavms.in \hfill}

\section{ TS 5.4.11.4 }

\textbf{TS 5.4.11.4 } \newline
\textbf{Samhita Paata} \newline

होता॒ स्व ए॒वास्मा॑ आ॒यत॑न इन्द्रि॒यं ॅवी॒र्य॑मव॑ रुन्धे॒ द्वाद॒शोप॑ दधाति॒ द्वाद॑शाक्षरा॒ जग॑ती॒ जाग॑ताः प॒शवो॒ जग॑त्यै॒वास्मै॑ प॒शूनव॑ रुन्धे॒ ऽष्टाव॑ष्टाव॒न्येषु॒ धिष्णि॑ये॒षूप॑ दधात्य॒ष्टाश॑फाः प॒शवः॑ प॒शूने॒वाव॑ रुन्धे॒ षण्मा᳚र्जा॒लीये॒ षड् वा ऋ॒तव॑ ऋ॒तवः॒ खलु॒ वै दे॒वाः पि॒तर॑ ऋ॒तूने॒व दे॒वान् पि॒तॄन् प्री॑णाति ॥ \newline

\textbf{Pada Paata} \newline

होता᳚ । स्वे । ए॒व । अ॒स्मै॒ । आ॒यत॑न॒ इत्या᳚ - यत॑ने । इ॒न्द्रि॒यम् । वी॒र्य᳚म् । अवेति॑ । रु॒न्धे॒ । द्वाद॑श । उपेति॑ । द॒धा॒ति॒ । द्वाद॑शाक्ष॒रेति॒ द्वाद॑श - अ॒क्ष॒रा॒ । जग॑ती । जाग॑ताः । प॒शवः॑ । जग॑त्या । ए॒व । अ॒स्मै॒ । प॒शून् । अवेति॑ । रु॒न्धे॒ । अ॒ष्टाव॑ष्टा॒वित्य॒ष्टौ-अ॒ष्टौ॒ । अ॒न्येषु॑ । धिष्णि॑येषु । उपेति॑ । द॒धा॒ति॒ । अ॒ष्टाश॑फा॒ इत्य॒ष्टा - श॒फाः॒ । प॒शवः॑ । प॒शून् । ए॒व । अवेति॑ । रु॒न्धे॒ । षट् । मा॒र्जा॒लीये᳚ । षट् । वै । ऋ॒तवः॑ । ऋ॒तवः॑ । खलु॑ । वै । दे॒वाः । पि॒तरः॑ । ऋ॒तून् । ए॒व । दे॒वान् । पि॒तॄन् । प्री॒णा॒ति॒ ॥  \newline




\markright{ TS 5.4.12.1  \hfill https://www.vedavms.in \hfill}

\section{ TS 5.4.12.1 }

\textbf{TS 5.4.12.1 } \newline
\textbf{Samhita Paata} \newline

पव॑स्व॒ वाज॑सातय॒ इत्य॑नु॒ष्टुक् प्र॑ति॒पद्भ॑वति ति॒र्.सो॑ऽनु॒ष्टुभ॒श्चत॑स्रो गाय॒त्रियो॒ यत् ति॒स्रो॑ऽनु॒ष्टुभ॒-स्तस्मा॒-दश्व॑स्त्रि॒भिस्तिष्ठꣳ॑ स्तिष्ठति॒ यच्चत॑स्रो गाय॒त्रिय॒स्तस्मा॒थ् सर्वाꣳ॑ श्च॒तुरः॑ प॒दः प्र॑ति॒दध॒त् पला॑यते पर॒मा वा ए॒षा छन्द॑सां॒ ॅयद॑नु॒ष्टुक् प॑र॒मश्च॑तुष्टो॒मः स्तोमा॑नां पर॒मस्त्रि॑रा॒त्रो य॒ज्ञानां᳚ पर॒मोऽश्वः॑ पशू॒नां प॑र॒मेणै॒वैनं॑ पर॒मतां᳚ गमयत्येकविꣳ॒॒शमह॑र्भवति॒ - [  ] \newline

\textbf{Pada Paata} \newline

पव॑स्व । वाज॑सातय॒ इति॒ वाज॑ - सा॒त॒ये॒ । इति॑ । अ॒नु॒ष्टुगित्य॑नु - स्तुक् । प्र॒ति॒पदिति॑ प्रति - पत् । भ॒व॒ति॒ । ति॒स्रः । अ॒नु॒ष्टुभ॒ इत्य॑नु - स्तुभः॑ । चत॑स्रः । गा॒य॒त्रियः॑ । यत् । ति॒स्रः । अ॒नु॒ष्टुभ॒ इत्य॑नु - स्तुभः॑ । तस्मा᳚त् । अश्वः॑ । त्रि॒भिरिति॑ त्रि-भिः । तिष्ठन्न्॑ । ति॒ष्ठ॒ति॒ । यत् । चत॑स्रः । गा॒य॒त्रियः॑ । तस्मा᳚त् । सर्वान्॑ । च॒तुरः॑ । प॒दः । प्र॒ति॒दध॒दिति॑ प्रति - दध॑त् । पला॑यते । प॒र॒मा । वै । ए॒षा । छन्द॑साम् । यत् । अ॒नु॒ष्टुगित्य॑नु - स्तुक् । प॒र॒मः । च॒तु॒ष्टो॒म इति॑ चतुः - स्तो॒मः । स्तोमा॑नाम् । प॒र॒मः । त्रि॒रा॒त्र इति॑ त्रि-रा॒त्रः । य॒ज्ञाना᳚म् । प॒र॒मः । अश्वः॑ । प॒शू॒नाम् । प॒र॒मेण॑ । ए॒व । ए॒न॒म् । प॒र॒मता᳚म् । ग॒म॒य॒ति॒ । ए॒क॒विꣳ॒॒शमित्ये॑क - विꣳ॒॒शम् । अहः॑ । भ॒व॒ति॒ ।  \newline




\markright{ TS 5.4.12.2  \hfill https://www.vedavms.in \hfill}

\section{ TS 5.4.12.2 }

\textbf{TS 5.4.12.2 } \newline
\textbf{Samhita Paata} \newline

यस्मि॒न्नश्व॑ आल॒भ्यते॒ द्वाद॑श॒ मासाः॒ पञ्च॒र्तव॒स्त्रय॑ इ॒मे लो॒का अ॒सावा॑दि॒त्य ए॑कविꣳ॒॒श ए॒ष प्र॒जाप॑तिः प्राजाप॒त्योऽश्व॒स्तमे॒व सा॒क्षादृ॑द्ध्नोति॒ शक्व॑रयः पृ॒ष्ठं भ॑वन्त्य॒न्-यद॑न्य॒-च्छन्दो॒ऽन्ये᳚न्ये॒ वा ए॒ते प॒शव॒ आ ल॑भ्यन्त उ॒तेव॑ ग्रा॒म्या उ॒तेवा॑ऽऽ*र॒ण्या यच्छक्व॑रयः पृ॒ष्ठं भव॒न्त्यश्व॑स्य सर्व॒त्वाय॑ पार्थुर॒श्मं ब्र॑ह्मसा॒मं भ॑वति र॒श्मिना॒ वा अश्वो॑ - [  ] \newline

\textbf{Pada Paata} \newline

यस्मिन्न्॑ । अश्वः॑ । आ॒ल॒भ्यत॒ इत्या᳚ - ल॒भ्यते᳚ । द्वाद॑श । मासाः᳚ । पञ्च॑ । ऋ॒तवः॑ । त्रयः॑ । इ॒मे । लो॒काः । अ॒सौ । आ॒दि॒त्यः । ए॒क॒विꣳ॒॒श इत्ये॑क - विꣳ॒॒शः । ए॒षः । प्र॒जाप॑ति॒रिति॑ प्र॒जा-प॒तिः॒ । प्रा॒जा॒प॒त्य इति॑ प्राजा - प॒त्यः । अश्वः॑ । तम् । ए॒व । सा॒क्षादिति॑ स - अ॒क्षात् । ऋ॒द्ध्नो॒ति॒ । शक्व॑रयः । पृ॒ष्ठम् । भ॒व॒न्ति॒ । अ॒न्यद॑न्य॒दित्य॒न्यत् - अ॒न्य॒त् । छन्दः॑ । अ॒न्ये᳚ऽन्य॒ इत्य॒न्ये - अ॒न्ये॒ । वै । ए॒ते । प॒शवः॑ । एति॑ । ल॒भ्य॒न्ते॒ । उ॒त । इ॒व॒ । ग्रा॒म्याः । उ॒त । इ॒व॒ । आ॒र॒ण्याः । यत् । शक्व॑रयः । पृ॒ष्ठम् । भव॑न्ति । अश्व॑स्य । स॒र्व॒त्वायेति॑ सर्व - त्वाय॑ । पा॒र्थु॒र॒श्ममिति॑ पार्थु - र॒श्मम् । ब्र॒ह्म॒सा॒ममिति॑ ब्रह्म - सा॒मम् । भ॒व॒ति॒ । र॒श्मिना᳚ । वै । अश्वः॑ ।  \newline




\markright{ TS 5.4.12.3  \hfill https://www.vedavms.in \hfill}

\section{ TS 5.4.12.3 }

\textbf{TS 5.4.12.3 } \newline
\textbf{Samhita Paata} \newline

य॒त ई᳚श्व॒रो वा अश्वोऽय॒तोऽप्र॑तिष्ठितः॒ परां᳚ परा॒वतं॒ गन्तो॒र्यत् पा᳚र्थुर॒श्मं ब्र॑ह्मसा॒मं भव॒त्यश्व॑स्य॒ यत्यै॒ धृत्यै॒ संकृ॑त्यच्छावाकसा॒मं भ॑वत्युथ्सन्नय॒ज्ञो वा ए॒ष यद॑श्वमे॒धः कस्तद्वे॒देत्या॑हु॒र्यदि॒ सर्वो॑ वा क्रि॒यते॒ न वा॒ सर्व॒ इति॒ यथ् संकृ॑त्यच्छावाकसा॒मं भव॒त्यश्व॑स्य सर्व॒त्वाय॒ पर्या᳚प्त्या॒ अन॑न्तरायाय॒ सर्व॑स्तोमोऽतिरा॒त्र उ॑त्त॒ममह॑र्भवति॒ ( ) सर्व॒स्याऽऽ*प्त्यै॒ सर्व॑स्य॒ जित्यै॒ सर्व॑मे॒व तेना᳚ऽऽ*प्नोति॒ सर्वं॑ जयति ॥ \newline

\textbf{Pada Paata} \newline

य॒तः । ई॒श्व॒रः । वै । अश्वः॑ । अय॑तः । अप्र॑तिष्ठित॒ इत्यप्र॑ति-स्थि॒तः॒ । परा᳚म् । प॒रा॒वत॒मिति॑ परा - वत᳚म् । गन्तोः᳚ । यत् । पा॒र्थु॒र॒श्ममिति॑ पार्थु - र॒श्मम् । ब्र॒ह्म॒सा॒ममिति॑ ब्रह्म - सा॒मम् । भव॑ति । अश्व॑स्य । यत्यै᳚ । धृत्यै᳚ । संकृ॒तीति॒ सं - कृ॒ति॒ । अ॒च्छा॒वा॒क॒सा॒ममित्य॑च्छावाक - सा॒मम् । भ॒व॒ति॒ । उ॒थ्स॒न्न॒य॒ज्ञ् इत्यु॑थ्सन्न - य॒ज्ञ्ः । वै । ए॒षः । यत् । अ॒श्व॒मे॒ध इत्य॑श्व - मे॒धः । कः । तत् । वे॒द॒ । इति॑ । आ॒हुः॒ । यदि॑ । सर्वः॑ । वा॒ । क्रि॒यते᳚ । न । वा॒ । सर्वः॑ । इति॑ । यत् । संकृ॒तीति॒ सं - कृ॒ति॒ । अ॒च्छा॒वा॒क॒सा॒ममित्य॑च्छावाक - स॒मम् । भव॑ति । अश्व॑स्य । स॒र्व॒त्वायेति॑ सर्व - त्वाय॑ । पर्या᳚प्त्या॒ इति॒ परि॑ - आ॒प्त्यै॒ । अन॑न्तराया॒येत्यन॑न्तः - आ॒या॒य॒ । सर्व॑स्तोम॒ इति॒ सर्व॑ - स्तो॒मः॒ । अ॒ति॒रा॒त्र इत्य॑ति - रा॒त्रः । उ॒त्त॒ममित्यु॑त् - त॒मम् । अहः॑ । भ॒व॒ति॒ ( ) । सर्व॑स्य । आप्त्यै᳚ । सर्व॑स्य । जित्यै᳚ । सर्व᳚म् । ए॒व । तेन॑ । आ॒प्नो॒ति॒ । सर्व᳚म् । ज॒य॒ति॒ ॥  \newline






\end{document}