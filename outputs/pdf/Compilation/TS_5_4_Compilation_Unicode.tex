\documentclass[17pt]{extarticle}
\usepackage{babel}
\usepackage{fontspec}
\usepackage{polyglossia}
\usepackage{extsizes}

\usepackage{color}   %May be necessary if you want to color links
\usepackage{hyperref}
\hypersetup{
    colorlinks=true, %set true if you want colored links
    linktoc=all,     %set to all if you want both sections and subsections linked
    linkcolor=black,  %choose some color if you want links to stand out
}

\setmainlanguage{sanskrit}
\setotherlanguages{english} %% or other languages
\setlength{\parindent}{0pt}
\pagestyle{myheadings}
\newfontfamily\devanagarifont[Script=Devanagari]{AdishilaVedic}
\renewcommand{\theHsection}{\thepart.section.\thesection}

\newcommand{\VAR}[1]{}
\newcommand{\BLOCK}[1]{}




\begin{document}
\begin{titlepage}
    \begin{center}
 
\begin{sanskrit}
    { \Large
    कृष्ण यजुर्वेदीय तैत्तिरीय संहिता,पद,जटा,घन पाठः 
    }
    \\
    \vspace{2.5cm}
    \mbox{ \Large
    5.4      पञ्चमकाण्डे चतुर्थः प्रश्नः इष्टकात्रयाभिधानं   }
\end{sanskrit}
\end{center}

\end{titlepage}
\tableofcontents
\phantomsection
\pagebreak

\markright{ TS 5.4.1.1  \hfill https://www.vedavms.in \hfill}

\section{ TS 5.4.1.1 }

\textbf{TS 5.4.1.1 } \newline
\textbf{Samhita Paata} \newline

दे॒वा॒सु॒राः संॅय॑त्ता आस॒न् ते न व्य॑जयन्त॒ स ए॒ता इन्द्र॑स्त॒नूर॑पश्य॒त् ता उपा॑धत्त॒ ताभि॒र्वै स त॒नुव॑मिन्द्रि॒यं ॅवी॒र्य॑मा॒त्मन्न॑धत्त॒ ततो॑ दे॒वा अभ॑व॒न् पराऽसु॑रा॒ यदि॑न्द्रत॒नूरु॑प॒दधा॑ति त॒नुव॑मे॒व ताभि॑रिन्द्रि॒यं ॅवी॒र्यं॑ ॅयज॑मान आ॒त्मन् ध॒त्तेऽथो॒ सेन्द्र॑मे॒वाग्निꣳ सत॑नुं चिनुते॒ भव॑त्या॒त्मना॒ परा᳚ऽस्य॒ भ्रातृ॑व्यो - [  ] \newline

\textbf{Pada Paata} \newline

दे॒वा॒सु॒रा इति॑ देव - अ॒सु॒राः । संॅय॑त्ता॒ इति॒ सं-य॒त्ताः॒ । आ॒स॒न्न् । ते । न । वीति॑ । अ॒ज॒य॒न्त॒ । सः । ए॒ताः । इन्द्रः॑ । त॒नूः । अ॒प॒श्य॒त् । ताः । उपेति॑ । अ॒ध॒त्त॒ । ताभिः॑ । वै । सः । त॒नुव᳚म् । इ॒न्द्रि॒यम् । वी॒र्य᳚म् । आ॒त्मन्न् । अ॒ध॒त्त॒ । ततः॑ । दे॒वाः । अभ॑वन्न् । परेति॑ । असु॑राः । यत् । इ॒न्द्र॒त॒नूरिती᳚न्द्र - त॒नूः । उ॒प॒दधा॒तीत्यु॑प - दधा॑ति । त॒नुव᳚म् । ए॒व । ताभिः॑ । इ॒न्द्रि॒यम् । वी॒र्य᳚म् । यज॑मानः । आ॒त्मन्न् । ध॒त्ते॒ । अथो॒ इति॑ । सैन्द्र॒मिति॒ स - इ॒न्द्र॒म् । ए॒व । अ॒ग्निम् । सत॑नु॒मिति॒ स - त॒नु॒म् । चि॒नु॒ते॒ । भव॑ति । आ॒त्मना᳚ । परेति॑ । अ॒स्य॒ । भ्रातृ॑व्यः ।  \newline


\textbf{Krama Paata} \newline

दे॒वा॒सु॒राः सम्ॅय॑त्ताः । दे॒वा॒सु॒रा इति॑ देव - अ॒सु॒राः । सम्ॅय॑त्ता आसन्न् । सम्ॅय॑त्ता॒ इति॒ सम् - य॒त्ताः॒ । आ॒स॒न् ते । ते न । न वि । व्य॑जयन्त । अ॒ज॒य॒न्त॒ सः । स ए॒ताः । ए॒ता इन्द्रः॑ । इन्द्र॑स्त॒नूः । त॒नूर॑पश्यत् । अ॒प॒श्य॒त् ताः । ता उप॑ । उपा॑धत्त । अ॒ध॒त्त॒ ताभिः॑ । ताभि॒र् वै । वै सः । स त॒नुव᳚म् । त॒नुव॑मिन्द्रि॒यम् । इ॒न्द्रि॒यम् ॅवी॒र्य᳚म् । वी॒र्य॑मा॒त्मन्न् । आ॒त्मन्न॑धत्त । अ॒ध॒त्त॒ ततः॑ । ततो॑ दे॒वाः । दे॒वा अभ॑वन्न् । अभ॑व॒न् परा᳚ । पराऽसु॑राः । असु॑रा॒ यत् । यदि॑न्द्रत॒नूः । इ॒न्द्र॒त॒नूरु॑प॒दधा॑ति । इ॒न्द्र॒त॒नूरिती᳚न्द्र - त॒नूः । उ॒प॒दधा॑ति त॒नुव᳚म् । उ॒प॒दधा॒तीत्यु॑प - दधा॑ति । त॒नुव॑मे॒व । ए॒व ताभिः॑ । ताभि॑रिन्द्रि॒यम् । इ॒न्द्रि॒यम् ॅवी॒र्य᳚म् । वी॒र्य॑म् ॅयज॑मानः । यज॑मान आ॒त्मन्न् । आ॒त्मन् ध॑त्ते । ध॒त्तेऽथो᳚ । अथो॒ सेन्द्र᳚म् । अथो॒ इत्यथो᳚ । सेन्द्र॑मे॒व । सेन्द्र॒मिति॒ स - इ॒न्द्र॒म् । ए॒वाग्निम् । अ॒ग्निꣳ सत॑नुम् । सत॑नुम् चिनुते । सत॑नु॒मिति॒ स - त॒नु॒म् । चि॒नु॒ते॒ भव॑ति । भव॑त्या॒त्मना᳚ । आ॒त्मना॒ परा᳚ । परा᳚ऽस्य । अ॒स्य॒ भ्रातृ॑व्यः । भ्रातृ॑व्यो भवति \newline

\textbf{Jatai Paata} \newline

1. दे॒वा॒सु॒राः संॅय॑त्ताः॒ संॅय॑त्ता देवासु॒रा दे॑वासु॒राः संॅय॑त्ताः । \newline
2. दे॒वा॒सु॒रा इति॑ देव - अ॒सु॒राः । \newline
3. संॅय॑त्ता आसन् नास॒न् थ्संॅय॑त्ताः॒ संॅय॑त्ता आसन्न् । \newline
4. संॅय॑त्ता॒ इति॒ सं - य॒त्ताः॒ । \newline
5. आ॒स॒न् ते त आ॑सन् नास॒न् ते । \newline
6. ते न न ते ते न । \newline
7. न वि वि न न वि । \newline
8. व्य॑जयन्ता जयन्त॒ वि व्य॑जयन्त । \newline
9. अ॒ज॒य॒न्त॒ स सो॑ ऽजयन्ता जयन्त॒ सः । \newline
10. स ए॒ता ए॒ताः स स ए॒ताः । \newline
11. ए॒ता इन्द्र॒ इन्द्र॑ ए॒ता ए॒ता इन्द्रः॑ । \newline
12. इन्द्र॑ स्त॒नू स्त॒नू रिन्द्र॒ इन्द्र॑ स्त॒नूः । \newline
13. त॒नू र॑पश्य दपश्यत् त॒नू स्त॒नू र॑पश्यत् । \newline
14. अ॒प॒श्य॒त् ता स्ता अ॑पश्य दपश्य॒त् ताः । \newline
15. ता उपोप॒ ता स्ता उप॑ । \newline
16. उपा॑धत्ता ध॒त्तो पोपा॑धत्त । \newline
17. अ॒ध॒त्त॒ ताभि॒ स्ताभि॑ रधत्ता धत्त॒ ताभिः॑ । \newline
18. ताभि॒र् वै वै ताभि॒ स्ताभि॒र् वै । \newline
19. वै स स वै वै सः । \newline
20. स त॒नुव॑म् त॒नुवꣳ॒॒ स स त॒नुव᳚म् । \newline
21. त॒नुव॑ मिन्द्रि॒य मि॑न्द्रि॒यम् त॒नुव॑म् त॒नुव॑ मिन्द्रि॒यम् । \newline
22. इ॒न्द्रि॒यं ॅवी॒र्यं॑ ॅवी॒र्य॑ मिन्द्रि॒य मि॑न्द्रि॒यं ॅवी॒र्य᳚म् । \newline
23. वी॒र्य॑ मा॒त्मन् ना॒त्मन्. वी॒र्यं॑ ॅवी॒र्य॑ मा॒त्मन्न् । \newline
24. आ॒त्मन् न॑धत्ता धत्ता॒त्मन् ना॒त्मन् न॑धत्त । \newline
25. अ॒ध॒त्त॒ तत॒ स्ततो॑ ऽधत्ता धत्त॒ ततः॑ । \newline
26. ततो॑ दे॒वा दे॒वा स्तत॒ स्ततो॑ दे॒वाः । \newline
27. दे॒वा अभ॑व॒न् नभ॑वन् दे॒वा दे॒वा अभ॑वन्न् । \newline
28. अभ॑व॒न् परा॒ परा ऽभ॑व॒न् नभ॑व॒न् परा᳚ । \newline
29. परा ऽसु॑रा॒ असु॑राः॒ परा॒ परा ऽसु॑राः । \newline
30. असु॑रा॒ यद् यदसु॑रा॒ असु॑रा॒ यत् । \newline
31. यदि॑न्द्रत॒नू रि॑न्द्रत॒नूर् यद् यदि॑न्द्रत॒नूः । \newline
32. इ॒न्द्र॒त॒नू रु॑प॒दधा᳚ त्युप॒दधा॑ तीन्द्रत॒नू रि॑न्द्रत॒नू रु॑प॒दधा॑ति । \newline
33. इ॒न्द्र॒त॒नूरिती᳚न्द्र - त॒नूः । \newline
34. उ॒प॒दधा॑ति त॒नुव॑म् त॒नुव॑ मुप॒दधा᳚ त्युप॒दधा॑ति त॒नुव᳚म् । \newline
35. उ॒प॒दधा॒तीत्यु॑प - दधा॑ति । \newline
36. त॒नुव॑ मे॒वैव त॒नुव॑म् त॒नुव॑ मे॒व । \newline
37. ए॒व ताभि॒ स्ताभि॑ रे॒वैव ताभिः॑ । \newline
38. ताभि॑ रिन्द्रि॒य मि॑न्द्रि॒यम् ताभि॒ स्ताभि॑ रिन्द्रि॒यम् । \newline
39. इ॒न्द्रि॒यं ॅवी॒र्यं॑ ॅवी॒र्य॑ मिन्द्रि॒य मि॑न्द्रि॒यं ॅवी॒र्य᳚म् । \newline
40. वी॒र्यं॑ ॅयज॑मानो॒ यज॑मानो वी॒र्यं॑ ॅवी॒र्यं॑ ॅयज॑मानः । \newline
41. यज॑मान आ॒त्मन् ना॒त्मन्. यज॑मानो॒ यज॑मान आ॒त्मन्न् । \newline
42. आ॒त्मन् ध॑त्ते धत्त आ॒त्मन् ना॒त्मन् ध॑त्ते । \newline
43. ध॒त्ते ऽथो॒ अथो॑ धत्ते ध॒त्ते ऽथो᳚ । \newline
44. अथो॒ सेन्द्रꣳ॒॒ सेन्द्र॒ मथो॒ अथो॒ सेन्द्र᳚म् । \newline
45. अथो॒ इत्यथो᳚ । \newline
46. सेन्द्र॑ मे॒वैव सेन्द्रꣳ॒॒ सेन्द्र॑ मे॒व । \newline
47. सेन्द्र॒मिति॒ स - इ॒न्द्र॒म् । \newline
48. ए॒वाग्नि म॒ग्नि मे॒वै वाग्निम् । \newline
49. अ॒ग्निꣳ सत॑नुꣳ॒॒ सत॑नु म॒ग्नि म॒ग्निꣳ सत॑नुम् । \newline
50. सत॑नुम् चिनुते चिनुते॒ सत॑नुꣳ॒॒ सत॑नुम् चिनुते । \newline
51. सत॑नु॒मिति॒ स - त॒नु॒म् । \newline
52. चि॒नु॒ते॒ भव॑ति॒ भव॑ति चिनुते चिनुते॒ भव॑ति । \newline
53. भव॑ त्या॒त्मना॒ ऽऽत्मना॒ भव॑ति॒ भव॑ त्या॒त्मना᳚ । \newline
54. आ॒त्मना॒ परा॒ परा॒ ऽऽत्मना॒ ऽऽत्मना॒ परा᳚ । \newline
55. परा᳚ ऽस्यास्य॒ परा॒ परा᳚ ऽस्य । \newline
56. अ॒स्य॒ भ्रातृ॑व्यो॒ भ्रातृ॑व्यो ऽस्यास्य॒ भ्रातृ॑व्यः । \newline
57. भ्रातृ॑व्यो भवति भवति॒ भ्रातृ॑व्यो॒ भ्रातृ॑व्यो भवति । \newline

\textbf{Ghana Paata } \newline

1. दे॒वा॒सु॒राः संॅय॑त्ताः॒ संॅय॑त्ता देवासु॒रा दे॑वासु॒राः संॅय॑त्ता आसन् नास॒न् थ्संॅय॑त्ता देवासु॒रा दे॑वासु॒राः संॅय॑त्ता आसन्न् । \newline
2. दे॒वा॒सु॒रा इति॑ देव - अ॒सु॒राः । \newline
3. संॅय॑त्ता आसन् नास॒न् थ्संॅय॑त्ताः॒ संॅय॑त्ता आस॒न् ते त आ॑स॒न् थ्संॅय॑त्ताः॒ संॅय॑त्ता आस॒न् ते । \newline
4. संॅय॑त्ता॒ इति॒ सं - य॒त्ताः॒ । \newline
5. आ॒स॒न् ते त आ॑सन् नास॒न् ते न न त आ॑सन् नास॒न् ते न । \newline
6. ते न न ते ते न वि वि न ते ते न वि । \newline
7. न वि वि न न व्य॑जयन्ता जयन्त॒ वि न न व्य॑जयन्त । \newline
8. व्य॑जयन्ता जयन्त॒ वि व्य॑जयन्त॒ स सो॑ ऽजयन्त॒ वि व्य॑जयन्त॒ सः । \newline
9. अ॒ज॒य॒न्त॒ स सो॑ ऽजयन्ता जयन्त॒ स ए॒ता ए॒ताः सो॑ ऽजयन्ता जयन्त॒ स ए॒ताः । \newline
10. स ए॒ता ए॒ताः स स ए॒ता इन्द्र॒ इन्द्र॑ ए॒ताः स स ए॒ता इन्द्रः॑ । \newline
11. ए॒ता इन्द्र॒ इन्द्र॑ ए॒ता ए॒ता इन्द्र॑ स्त॒नू स्त॒नू रिन्द्र॑ ए॒ता ए॒ता इन्द्र॑ स्त॒नूः । \newline
12. इन्द्र॑ स्त॒नू स्त॒नू रिन्द्र॒ इन्द्र॑ स्त॒नू र॑पश्य दपश्यत् त॒नू रिन्द्र॒ इन्द्र॑ स्त॒नू र॑पश्यत् । \newline
13. त॒नू र॑पश्य दपश्यत् त॒नू स्त॒नू र॑पश्य॒त् ता स्ता अ॑पश्यत् त॒नू स्त॒नू र॑पश्य॒त् ताः । \newline
14. अ॒प॒श्य॒त् ता स्ता अ॑पश्य दपश्य॒त् ता उपोप॒ ता अ॑पश्य दपश्य॒त् ता उप॑ । \newline
15. ता उपोप॒ ता स्ता उपा॑धत्ता ध॒त्तोप॒ ता स्ता उपा॑धत्त । \newline
16. उपा॑ धत्ता ध॒त्तो पोपा॑ धत्त॒ ताभि॒ स्ताभि॑ रध॒त्तो पोपा॑ धत्त॒ ताभिः॑ । \newline
17. अ॒ध॒त्त॒ ताभि॒ स्ताभि॑ रधत्ता धत्त॒ ताभि॒र् वै वै ताभि॑ रधत्ता धत्त॒ ताभि॒र् वै । \newline
18. ताभि॒र् वै वै ताभि॒ स्ताभि॒र् वै स स वै ताभि॒ स्ताभि॒र् वै सः । \newline
19. वै स स वै वै स त॒नुव॑म् त॒नुवꣳ॒॒ स वै वै स त॒नुव᳚म् । \newline
20. स त॒नुव॑म् त॒नुवꣳ॒॒ स स त॒नुव॑ मिन्द्रि॒य मि॑न्द्रि॒यम् त॒नुवꣳ॒॒ स स त॒नुव॑ मिन्द्रि॒यम् । \newline
21. त॒नुव॑ मिन्द्रि॒य मि॑न्द्रि॒यम् त॒नुव॑म् त॒नुव॑ मिन्द्रि॒यं ॅवी॒र्यं॑ ॅवी॒र्य॑ मिन्द्रि॒यम् त॒नुव॑म् त॒नुव॑ मिन्द्रि॒यं ॅवी॒र्य᳚म् । \newline
22. इ॒न्द्रि॒यं ॅवी॒र्यं॑ ॅवी॒र्य॑ मिन्द्रि॒य मि॑न्द्रि॒यं ॅवी॒र्य॑ मा॒त्मन् ना॒त्मन्. वी॒र्य॑ मिन्द्रि॒य मि॑न्द्रि॒यं ॅवी॒र्य॑ मा॒त्मन्न् । \newline
23. वी॒र्य॑ मा॒त्मन् ना॒त्मन्. वी॒र्यं॑ ॅवी॒र्य॑ मा॒त्मन् न॑धत्ता धत्ता॒त्मन्. वी॒र्यं॑ ॅवी॒र्य॑ मा॒त्मन् न॑धत्त । \newline
24. आ॒त्मन् न॑धत्ता धत्ता॒त्मन् ना॒त्मन् न॑धत्त॒ तत॒ स्ततो॑ ऽधत्ता॒त्मन् ना॒त्मन् न॑धत्त॒ ततः॑ । \newline
25. अ॒ध॒त्त॒ तत॒ स्ततो॑ ऽधत्ता धत्त॒ ततो॑ दे॒वा दे॒वा स्ततो॑ ऽधत्ता धत्त॒ ततो॑ दे॒वाः । \newline
26. ततो॑ दे॒वा दे॒वा स्तत॒ स्ततो॑ दे॒वा अभ॑व॒न् नभ॑वन् दे॒वा स्तत॒ स्ततो॑ दे॒वा अभ॑वन्न् । \newline
27. दे॒वा अभ॑व॒न् नभ॑वन् दे॒वा दे॒वा अभ॑व॒न् परा॒ परा ऽभ॑वन् दे॒वा दे॒वा अभ॑व॒न् परा᳚ । \newline
28. अभ॑व॒न् परा॒ परा ऽभ॑व॒न् नभ॑व॒न् परा ऽसु॑रा॒ असु॑राः॒ परा ऽभ॑व॒न् नभ॑व॒न् परा ऽसु॑राः । \newline
29. परा ऽसु॑रा॒ असु॑राः॒ परा॒ परा ऽसु॑रा॒ यद् यदसु॑राः॒ परा॒ परा ऽसु॑रा॒ यत् । \newline
30. असु॑रा॒ यद् यदसु॑रा॒ असु॑रा॒ यदि॑न्द्रत॒नू रि॑न्द्रत॒नूर् यदसु॑रा॒ असु॑रा॒ यदि॑न्द्रत॒नूः । \newline
31. यदि॑न्द्रत॒नू रि॑न्द्रत॒नूर् यद् यदि॑न्द्रत॒नू रु॑प॒दधा᳚ त्युप॒दधा॑ती न्द्रत॒नूर् यद् यदि॑न्द्रत॒नू रु॑प॒दधा॑ति । \newline
32. इ॒न्द्र॒त॒नू रु॑प॒दधा᳚ त्युप॒दधा॑ती न्द्रत॒नू रि॑न्द्रत॒नू रु॑प॒दधा॑ति त॒नुव॑म् त॒नुव॑ मुप॒दधा॑ती न्द्रत॒नू रि॑न्द्रत॒नू रु॑प॒दधा॑ति त॒नुव᳚म् । \newline
33. इ॒न्द्र॒त॒नूरिती᳚न्द्र - त॒नूः । \newline
34. उ॒प॒दधा॑ति त॒नुव॑म् त॒नुव॑ मुप॒दधा᳚ त्युप॒दधा॑ति त॒नुव॑ मे॒वैव त॒नुव॑ मुप॒दधा᳚ त्युप॒दधा॑ति त॒नुव॑ मे॒व । \newline
35. उ॒प॒दधा॒तीत्यु॑प - दधा॑ति । \newline
36. त॒नुव॑ मे॒वैव त॒नुव॑म् त॒नुव॑ मे॒व ताभि॒ स्ताभि॑ रे॒व त॒नुव॑म् त॒नुव॑ मे॒व ताभिः॑ । \newline
37. ए॒व ताभि॒ स्ताभि॑ रे॒वैव ताभि॑ रिन्द्रि॒य मि॑न्द्रि॒यम् ताभि॑ रे॒वैव ताभि॑ रिन्द्रि॒यम् । \newline
38. ताभि॑ रिन्द्रि॒य मि॑न्द्रि॒यम् ताभि॒ स्ताभि॑ रिन्द्रि॒यं ॅवी॒र्यं॑ ॅवी॒र्य॑ मिन्द्रि॒यम् ताभि॒ स्ताभि॑ रिन्द्रि॒यं ॅवी॒र्य᳚म् । \newline
39. इ॒न्द्रि॒यं ॅवी॒र्यं॑ ॅवी॒र्य॑ मिन्द्रि॒य मि॑न्द्रि॒यं ॅवी॒र्यं॑ ॅयज॑मानो॒ यज॑मानो वी॒र्य॑ मिन्द्रि॒य मि॑न्द्रि॒यं ॅवी॒र्यं॑ ॅयज॑मानः । \newline
40. वी॒र्यं॑ ॅयज॑मानो॒ यज॑मानो वी॒र्यं॑ ॅवी॒र्यं॑ ॅयज॑मान आ॒त्मन् ना॒त्मन्. यज॑मानो वी॒र्यं॑ ॅवी॒र्यं॑ ॅयज॑मान आ॒त्मन्न् । \newline
41. यज॑मान आ॒त्मन् ना॒त्मन्. यज॑मानो॒ यज॑मान आ॒त्मन् ध॑त्ते धत्त आ॒त्मन्. यज॑मानो॒ यज॑मान आ॒त्मन् ध॑त्ते । \newline
42. आ॒त्मन् ध॑त्ते धत्त आ॒त्मन् ना॒त्मन् ध॒त्ते ऽथो॒ अथो॑ धत्त आ॒त्मन् ना॒त्मन् ध॒त्ते ऽथो᳚ । \newline
43. ध॒त्ते ऽथो॒ अथो॑ धत्ते ध॒त्ते ऽथो॒ सेन्द्रꣳ॒॒ सेन्द्र॒ मथो॑ धत्ते ध॒त्ते ऽथो॒ सेन्द्र᳚म् । \newline
44. अथो॒ सेन्द्रꣳ॒॒ सेन्द्र॒ मथो॒ अथो॒ सेन्द्र॑ मे॒वैव सेन्द्र॒ मथो॒ अथो॒ सेन्द्र॑ मे॒व । \newline
45. अथो॒ इत्यथो᳚ । \newline
46. सेन्द्र॑ मे॒वैव सेन्द्रꣳ॒॒ सेन्द्र॑ मे॒वाग्नि म॒ग्नि मे॒व सेन्द्रꣳ॒॒ सेन्द्र॑ मे॒वाग्निम् । \newline
47. सेन्द्र॒मिति॒ स - इ॒न्द्र॒म् । \newline
48. ए॒वाग्नि म॒ग्नि मे॒वै वाग्निꣳ सत॑नुꣳ॒॒ सत॑नु म॒ग्नि मे॒वै वाग्निꣳ सत॑नुम् । \newline
49. अ॒ग्निꣳ सत॑नुꣳ॒॒ सत॑नु म॒ग्नि म॒ग्निꣳ सत॑नुम् चिनुते चिनुते॒ सत॑नु म॒ग्नि म॒ग्निꣳ सत॑नुम् चिनुते । \newline
50. सत॑नुम् चिनुते चिनुते॒ सत॑नुꣳ॒॒ सत॑नुम् चिनुते॒ भव॑ति॒ भव॑ति चिनुते॒ सत॑नुꣳ॒॒ सत॑नुम् चिनुते॒ भव॑ति । \newline
51. सत॑नु॒मिति॒ स - त॒नु॒म् । \newline
52. चि॒नु॒ते॒ भव॑ति॒ भव॑ति चिनुते चिनुते॒ भव॑ त्या॒त्मना॒ ऽऽत्मना॒ भव॑ति चिनुते चिनुते॒ भव॑ त्या॒त्मना᳚ । \newline
53. भव॑ त्या॒त्मना॒ ऽऽत्मना॒ भव॑ति॒ भव॑ त्या॒त्मना॒ परा॒ परा॒ ऽऽत्मना॒ भव॑ति॒ भव॑ त्या॒त्मना॒ परा᳚ । \newline
54. आ॒त्मना॒ परा॒ परा॒ ऽऽत्मना॒ ऽऽत्मना॒ परा᳚ ऽस्यास्य॒ परा॒ ऽऽत्मना॒ ऽऽत्मना॒ परा᳚ ऽस्य । \newline
55. परा᳚ ऽस्यास्य॒ परा॒ परा᳚ ऽस्य॒ भ्रातृ॑व्यो॒ भ्रातृ॑व्यो ऽस्य॒ परा॒ परा᳚ ऽस्य॒ भ्रातृ॑व्यः । \newline
56. अ॒स्य॒ भ्रातृ॑व्यो॒ भ्रातृ॑व्यो ऽस्यास्य॒ भ्रातृ॑व्यो भवति भवति॒ भ्रातृ॑व्यो ऽस्यास्य॒ भ्रातृ॑व्यो भवति । \newline
57. भ्रातृ॑व्यो भवति भवति॒ भ्रातृ॑व्यो॒ भ्रातृ॑व्यो भवति य॒ज्ञो य॒ज्ञो भ॑वति॒ भ्रातृ॑व्यो॒ भ्रातृ॑व्यो भवति य॒ज्ञ्ः । \newline
\pagebreak
\markright{ TS 5.4.1.2  \hfill https://www.vedavms.in \hfill}

\section{ TS 5.4.1.2 }

\textbf{TS 5.4.1.2 } \newline
\textbf{Samhita Paata} \newline

भवति य॒ज्ञो दे॒वेभ्योऽपा᳚क्राम॒त् तम॑व॒रुधं॒ नाश॑क्नुव॒न्त ए॒ता य॑ज्ञ्त॒नूर॑पश्य॒न् ता उपा॑दधत॒ ताभि॒र्वै ते य॒ज्ञ्मवा॑रुन्धत॒ यद्-य॑ज्ञ्त॒नूरु॑प॒दधा॑ति य॒ज्ञ्मे॒व ताभि॒र्यज॑मा॒नोऽव॑ रुन्धे॒ त्रय॑स्त्रिꣳ शत॒मुप॑ दधाति॒ त्रय॑स्त्रिꣳश॒द्वै दे॒वता॑ दे॒वता॑ ए॒वाव॑ रु॒न्धे ऽथो॒ सात्मा॑नमे॒वाग्निꣳ सत॑नुं चिनुते॒ सात्मा॒ऽमुष्मि॑न् ॅलो॒के - [  ] \newline

\textbf{Pada Paata} \newline

भ॒व॒ति॒ । य॒ज्ञ्ः । दे॒वेभ्यः॑ । अपेति॑ । अ॒क्रा॒म॒त् । तम् । अ॒व॒रुध॒मित्य॑व - रुध᳚म् । न । अ॒श॒क्नु॒व॒न्न् । ते । ए॒ताः । य॒ज्ञ्॒त॒नूरिति॑ यज्ञ् - त॒नूः । अ॒प॒श्य॒न्न् । ताः । उपेति॑ । अ॒द॒ध॒त॒ । ताभिः॑ । वै । ते । य॒ज्ञ्म् । अवेति॑ । अ॒रु॒न्ध॒त॒ । यत् । य॒ज्ञ्॒त॒नूरिति॑ यज्ञ् - त॒नूः । उ॒प॒दधा॒तीत्यु॑प - दधा॑ति । य॒ज्ञ्म् । ए॒व । ताभिः॑ । यज॑मानः । अवेति॑ । रु॒न्धे॒ । त्रय॑स्त्रिꣳशत॒मिति॒ त्रयः॑ - त्रिꣳ॒॒श॒त॒म् । उपेति॑ । द॒धा॒ति॒ । त्रय॑स्त्रिꣳश॒दिति॒ त्रयः॑ - त्रिꣳ॒॒श॒त् । वै । दे॒वताः᳚ । दे॒वताः᳚ । ए॒व । अवेति॑ । रु॒न्धे॒ । अथो॒ इति॑ । सात्मा॑न॒मिति॒ स - आ॒त्मा॒न॒म् । ए॒व । अ॒ग्निम् । सत॑नु॒मिति॒ स-त॒नु॒म् । चि॒नु॒ते॒ । सात्मेति॒ स - आ॒त्मा॒ । अ॒मुष्मिन्॑ । लो॒के ।  \newline


\textbf{Krama Paata} \newline

भ॒व॒ति॒ य॒ज्ञ्ः । य॒ज्ञो दे॒वेभ्यः॑ । दे॒वेभ्योऽप॑ । अपा᳚क्रामत् । अ॒क्रा॒म॒त् तम् । तम॑व॒रुध᳚म् । अ॒व॒रुध॒म् न । अ॒व॒रुध॒मित्य॑व - रुध᳚म् । नाश॑क्नुवन्न् । अ॒शु॒क्नु॒व॒न् ते । त ए॒ताः । ए॒ता य॑ज्ञ्त॒नूः । य॒ज्ञ्॒त॒नूर॑पश्यन्न् । य॒ज्ञ्॒त॒नूरिति॑ यज्ञ् - त॒नूः । अ॒प॒श्य॒न् ताः । ता उप॑ । उपा॑दधत । अ॒द॒ध॒त॒ ताभिः॑ । ताभि॒र् वै । वै ते । ते य॒ज्ञ्म् । य॒ज्ञ्मव॑ । अवा॑रुन्धत । अ॒रु॒न्ध॒त॒ यत् । यद् य॑ज्ञ्त॒नूः । य॒ज्ञ्॒त॒नूरु॑प॒दधा॑ति । य॒ज्ञ्॒त॒नूरिति॑ यज्ञ् - त॒नूः । उ॒प॒दधा॑ति य॒ज्ञ्म् । उ॒प॒दधा॒तीत्यु॑प - दधा॑ति । य॒ज्ञ्मे॒व । ए॒व ताभिः॑ । ताभि॒र् यज॑मानः । यज॑मा॒नोऽव॑ । अव॑ रुन्धे । रु॒न्धे॒ त्रय॑स्त्रिꣳशतम् । त्रय॑स्त्रिꣳशत॒मुप॑ । त्रय॑स्त्रिꣳशत॒मिति॒ त्रयः॑ - त्रिꣳ॒॒श॒त॒म् । उप॑ दधाति । द॒धा॒ति॒ त्रय॑स्त्रिꣳशत् । त्रय॑स्त्रिꣳश॒द् वै । त्रय॑स्त्रिꣳश॒दिति॒ त्रयः॑ - त्रिꣳ॒॒श॒त्॒ । वै दे॒वताः᳚ । दे॒वता॑ दे॒वताः᳚ । दे॒वता॑ ए॒व । ए॒वाव॑ । अव॑ रुन्धे । रु॒न्धेऽथो᳚ । अथो॒ सात्मा॑नम् । अथो॒ इत्यथो᳚ । सात्मा॑नमे॒व । सात्मा॑न॒मिति॒ स - आ॒त्मा॒न॒म् । ए॒वाग्निम् । अ॒ग्निꣳ सत॑नुम् । सत॑नुम् चिनुते । सत॑नु॒मिति॒ स - त॒नु॒म् । चि॒नु॒ते॒ सात्मा᳚ । सात्मा॒ऽमुष्मिन्न्॑ । सात्मेति॒ स - आ॒त्मा॒ । अ॒मुष्मि॑न् ॅलो॒के । लो॒के भ॑वति \newline

\textbf{Jatai Paata} \newline

1. भ॒व॒ति॒ य॒ज्ञो य॒ज्ञो भ॑वति भवति य॒ज्ञ्ः । \newline
2. य॒ज्ञो दे॒वेभ्यो॑ दे॒वेभ्यो॑ य॒ज्ञो य॒ज्ञो दे॒वेभ्यः॑ । \newline
3. दे॒वेभ्यो ऽपाप॑ दे॒वेभ्यो॑ दे॒वेभ्यो ऽप॑ । \newline
4. अपा᳚ क्राम दक्राम॒ दपापा᳚ क्रामत् । \newline
5. अ॒क्रा॒म॒त् तम् त म॑क्राम दक्राम॒त् तम् । \newline
6. त म॑व॒रुध॑ मव॒रुध॒म् तम् त म॑व॒रुध᳚म् । \newline
7. अ॒व॒रुध॒म् न नाव॒रुध॑ मव॒रुध॒म् न । \newline
8. अ॒व॒रुध॒मित्य॑व - रुध᳚म् । \newline
9. नाश॑क्नुवन् नशक्नुव॒न् न नाश॑क्नुवन्न् । \newline
10. अ॒श॒क्नु॒व॒न् ते ते॑ ऽशक्नुवन् नशक्नुव॒न् ते । \newline
11. त ए॒ता ए॒ता स्ते त ए॒ताः । \newline
12. ए॒ता य॑ज्ञ्त॒नूर् य॑ज्ञ्त॒नू रे॒ता ए॒ता य॑ज्ञ्त॒नूः । \newline
13. य॒ज्ञ्॒त॒नू र॑पश्यन् नपश्यन्. यज्ञ्त॒नूर् य॑ज्ञ्त॒नू र॑पश्यन्न् । \newline
14. य॒ज्ञ्॒त॒नूरिति॑ यज्ञ् - त॒नूः । \newline
15. अ॒प॒श्य॒न् ता स्ता अ॑पश्यन् नपश्य॒न् ताः । \newline
16. ता उपोप॒ ता स्ता उप॑ । \newline
17. उपा॑दधता दध॒तोपोपा॑ दधत । \newline
18. अ॒द॒ध॒त॒ ताभि॒ स्ताभि॑ रदधता दधत॒ ताभिः॑ । \newline
19. ताभि॒र् वै वै ताभि॒ स्ताभि॒र् वै । \newline
20. वै ते ते वै वै ते । \newline
21. ते य॒ज्ञ्ं ॅय॒ज्ञ्म् ते ते य॒ज्ञ्म् । \newline
22. य॒ज्ञ् मवाव॑ य॒ज्ञ्ं ॅय॒ज्ञ् मव॑ । \newline
23. अवा॑ रुन्धता रुन्ध॒ता वावा॑ रुन्धत । \newline
24. अ॒रु॒न्ध॒त॒ यद् यद॑रुन्धता रुन्धत॒ यत् । \newline
25. यद् य॑ज्ञ्त॒नूर् य॑ज्ञ्त॒नूर् यद् यद् य॑ज्ञ्त॒नूः । \newline
26. य॒ज्ञ्॒त॒नू रु॑प॒दधा᳚ त्युप॒दधा॑ति यज्ञ्त॒नूर् य॑ज्ञ्त॒नू रु॑प॒दधा॑ति । \newline
27. य॒ज्ञ्॒त॒नूरिति॑ यज्ञ् - त॒नूः । \newline
28. उ॒प॒दधा॑ति य॒ज्ञ्ं ॅय॒ज्ञ् मु॑प॒दधा᳚ त्युप॒दधा॑ति य॒ज्ञ्म् । \newline
29. उ॒प॒दधा॒तीत्यु॑प - दधा॑ति । \newline
30. य॒ज्ञ् मे॒वैव य॒ज्ञ्ं ॅय॒ज्ञ् मे॒व । \newline
31. ए॒व ताभि॒ स्ताभि॑ रे॒वैव ताभिः॑ । \newline
32. ताभि॒र् यज॑मानो॒ यज॑मान॒ स्ताभि॒ स्ताभि॒र् यज॑मानः । \newline
33. यज॑मा॒नो ऽवाव॒ यज॑मानो॒ यज॑मा॒नो ऽव॑ । \newline
34. अव॑ रुन्धे रु॒न्धे ऽवाव॑ रुन्धे । \newline
35. रु॒न्धे॒ त्रय॑स्त्रिꣳशत॒म् त्रय॑स्त्रिꣳशतꣳ रुन्धे रुन्धे॒ त्रय॑स्त्रिꣳशतम् । \newline
36. त्रय॑स्त्रिꣳशत॒ मुपोप॒ त्रय॑स्त्रिꣳशत॒म् त्रय॑स्त्रिꣳशत॒ मुप॑ । \newline
37. त्रय॑स्त्रिꣳशत॒मिति॒ त्रयः॑ - त्रिꣳ॒॒श॒त॒म् । \newline
38. उप॑ दधाति दधा॒ त्युपोप॑ दधाति । \newline
39. द॒धा॒ति॒ त्रय॑स्त्रिꣳश॒त् त्रय॑स्त्रिꣳशद् दधाति दधाति॒ त्रय॑स्त्रिꣳशत् । \newline
40. त्रय॑स्त्रिꣳश॒द् वै वै त्रय॑स्त्रिꣳश॒त् त्रय॑स्त्रिꣳश॒द् वै । \newline
41. त्रय॑स्त्रिꣳश॒दिति॒ त्रयः॑ - त्रिꣳ॒॒श॒त् । \newline
42. वै दे॒वता॑ दे॒वता॒ वै वै दे॒वताः᳚ । \newline
43. दे॒वता॑ दे॒वताः᳚ । \newline
44. दे॒वता॑ ए॒वैव दे॒वता॑ दे॒वता॑ ए॒व । \newline
45. ए॒वावा वै॒वै वाव॑ । \newline
46. अव॑ रुन्धे रु॒न्धे ऽवाव॑ रुन्धे । \newline
47. रु॒न्धे ऽथो॒ अथो॑ रुन्धे रु॒न्धे ऽथो᳚ । \newline
48. अथो॒ सात्मा॑नꣳ॒॒ सात्मा॑न॒ मथो॒ अथो॒ सात्मा॑नम् । \newline
49. अथो॒ इत्यथो᳚ । \newline
50. सात्मा॑न मे॒वैव सात्मा॑नꣳ॒॒ सात्मा॑न मे॒व । \newline
51. सात्मा॑न॒मिति॒ स - आ॒त्मा॒न॒म् । \newline
52. ए॒वाग्नि म॒ग्नि मे॒वै वाग्निम् । \newline
53. अ॒ग्निꣳ सत॑नुꣳ॒॒ सत॑नु म॒ग्नि म॒ग्निꣳ सत॑नुम् । \newline
54. सत॑नुम् चिनुते चिनुते॒ सत॑नुꣳ॒॒ सत॑नुम् चिनुते । \newline
55. सत॑नु॒मिति॒ स - त॒नु॒म् । \newline
56. चि॒नु॒ते॒ सात्मा॒ सात्मा॑ चिनुते चिनुते॒ सात्मा᳚ । \newline
57. सात्मा॒ ऽमुष्मि॑न् न॒मुष्मि॒न् थ्सात्मा॒ सात्मा॒ ऽमुष्मिन्न्॑ । \newline
58. सात्मेति॒ स - आ॒त्मा॒ । \newline
59. अ॒मुष्मि॑न् ॅलो॒के लो॒के॑ ऽमुष्मि॑न् न॒मुष्मि॑न् ॅलो॒के । \newline
60. लो॒के भ॑वति भवति लो॒के लो॒के भ॑वति । \newline

\textbf{Ghana Paata } \newline

1. भ॒व॒ति॒ य॒ज्ञो य॒ज्ञो भ॑वति भवति य॒ज्ञो दे॒वेभ्यो॑ दे॒वेभ्यो॑ य॒ज्ञो भ॑वति भवति य॒ज्ञो दे॒वेभ्यः॑ । \newline
2. य॒ज्ञो दे॒वेभ्यो॑ दे॒वेभ्यो॑ य॒ज्ञो य॒ज्ञो दे॒वेभ्यो ऽपाप॑ दे॒वेभ्यो॑ य॒ज्ञो य॒ज्ञो दे॒वेभ्यो ऽप॑ । \newline
3. दे॒वेभ्यो ऽपाप॑ दे॒वेभ्यो॑ दे॒वेभ्यो ऽपा᳚क्राम दक्राम॒ दप॑ दे॒वेभ्यो॑ दे॒वेभ्यो ऽपा᳚क्रामत् । \newline
4. अपा᳚ क्राम दक्राम॒ दपापा᳚ क्राम॒त् तम् त म॑क्राम॒ दपापा᳚क्राम॒त् तम् । \newline
5. अ॒क्रा॒म॒त् तम् त म॑क्राम दक्राम॒त् त म॑व॒रुध॑ मव॒रुध॒म् त म॑क्राम दक्राम॒त् त म॑व॒रुध᳚म् । \newline
6. त म॑व॒रुध॑ मव॒रुध॒म् तम् त म॑व॒रुध॒म् न नाव॒रुध॒म् तम् त म॑व॒रुध॒म् न । \newline
7. अ॒व॒रुध॒म् न नाव॒रुध॑ मव॒रुध॒म् नाश॑क्नुवन् नशक्नुव॒न् नाव॒रुध॑ मव॒रुध॒म् नाश॑क्नुवन्न् । \newline
8. अ॒व॒रुध॒मित्य॑व - रुध᳚म् । \newline
9. नाश॑क्नुवन् नशक्नुव॒न् न नाश॑क्नुव॒न् ते ते॑ ऽशक्नुव॒न् न नाश॑क्नुव॒न् ते । \newline
10. अ॒श॒क्नु॒व॒न् ते ते॑ ऽशक्नुवन् नशक्नुव॒न् त ए॒ता ए॒ता स्ते॑ ऽशक्नुवन् नशक्नुव॒न् त ए॒ताः । \newline
11. त ए॒ता ए॒ता स्ते त ए॒ता य॑ज्ञ्त॒नूर् य॑ज्ञ्त॒नू रे॒ता स्ते त ए॒ता य॑ज्ञ्त॒नूः । \newline
12. ए॒ता य॑ज्ञ्त॒नूर् य॑ज्ञ्त॒नू रे॒ता ए॒ता य॑ज्ञ्त॒नू र॑पश्यन् नपश्यन्. यज्ञ्त॒नू रे॒ता ए॒ता य॑ज्ञ्त॒नू र॑पश्यन्न् । \newline
13. य॒ज्ञ्॒त॒नू र॑पश्यन् नपश्यन्. यज्ञ्त॒नूर् य॑ज्ञ्त॒नू र॑पश्य॒न् ता स्ता अ॑पश्यन्. यज्ञ्त॒नूर् य॑ज्ञ्त॒नू र॑पश्य॒न् ताः । \newline
14. य॒ज्ञ्॒त॒नूरिति॑ यज्ञ् - त॒नूः । \newline
15. अ॒प॒श्य॒न् ता स्ता अ॑पश्यन् नपश्य॒न् ता उपोप॒ ता अ॑पश्यन् नपश्य॒न् ता उप॑ । \newline
16. ता उपोप॒ ता स्ता उपा॑दधता दध॒तोप॒ ता स्ता उपा॑दधत । \newline
17. उपा॑ दधता दध॒तो पोपा॑ दधत॒ ताभि॒ स्ताभि॑ रदध॒तो पोपा॑ दधत॒ ताभिः॑ । \newline
18. अ॒द॒ध॒त॒ ताभि॒ स्ताभि॑ रदधता दधत॒ ताभि॒र् वै वै ताभि॑ रदधता दधत॒ ताभि॒र् वै । \newline
19. ताभि॒र् वै वै ताभि॒ स्ताभि॒र् वै ते ते वै ताभि॒ स्ताभि॒र् वै ते । \newline
20. वै ते ते वै वै ते य॒ज्ञ्ं ॅय॒ज्ञ्म् ते वै वै ते य॒ज्ञ्म् । \newline
21. ते य॒ज्ञ्ं ॅय॒ज्ञ्म् ते ते य॒ज्ञ् मवाव॑ य॒ज्ञ्म् ते ते य॒ज्ञ् मव॑ । \newline
22. य॒ज्ञ् मवाव॑ य॒ज्ञ्ं ॅय॒ज्ञ् मवा॑रुन्धता रुन्ध॒ताव॑ य॒ज्ञ्ं ॅय॒ज्ञ् मवा॑रुन्धत । \newline
23. अवा॑रुन्धता रुन्ध॒ता वावा॑ रुन्धत॒ यद् यद॑रुन्ध॒ता वावा॑ रुन्धत॒ यत् । \newline
24. अ॒रु॒न्ध॒त॒ यद् यद॑रुन्धता रुन्धत॒ यद् य॑ज्ञ्त॒नूर् य॑ज्ञ्त॒नूर् यद॑रुन्धता रुन्धत॒ यद् य॑ज्ञ्त॒नूः । \newline
25. यद् य॑ज्ञ्त॒नूर् य॑ज्ञ्त॒नूर् यद् यद् य॑ज्ञ्त॒नू रु॑प॒दधा᳚ त्युप॒दधा॑ति यज्ञ्त॒नूर् यद् यद् य॑ज्ञ्त॒नू रु॑प॒दधा॑ति । \newline
26. य॒ज्ञ्॒त॒नू रु॑प॒दधा᳚ त्युप॒दधा॑ति यज्ञ्त॒नूर् य॑ज्ञ्त॒नू रु॑प॒दधा॑ति य॒ज्ञ्ं ॅय॒ज्ञ् मु॑प॒दधा॑ति यज्ञ्त॒नूर् य॑ज्ञ्त॒नू रु॑प॒दधा॑ति य॒ज्ञ्म् । \newline
27. य॒ज्ञ्॒त॒नूरिति॑ यज्ञ् - त॒नूः । \newline
28. उ॒प॒दधा॑ति य॒ज्ञ्ं ॅय॒ज्ञ् मु॑प॒दधा᳚ त्युप॒दधा॑ति य॒ज्ञ् मे॒वैव य॒ज्ञ् मु॑प॒दधा᳚ त्युप॒दधा॑ति य॒ज्ञ् मे॒व । \newline
29. उ॒प॒दधा॒तीत्यु॑प - दधा॑ति । \newline
30. य॒ज्ञ् मे॒ वैव य॒ज्ञ्ं ॅय॒ज्ञ् मे॒व ताभि॒ स्ताभि॑ रे॒व य॒ज्ञ्ं ॅय॒ज्ञ् मे॒व ताभिः॑ । \newline
31. ए॒व ताभि॒ स्ताभि॑ रे॒वैव ताभि॒र् यज॑मानो॒ यज॑मान॒ स्ताभि॑ रे॒वैव ताभि॒र् यज॑मानः । \newline
32. ताभि॒र् यज॑मानो॒ यज॑मान॒ स्ताभि॒ स्ताभि॒र् यज॑मा॒नो ऽवाव॒ यज॑मान॒ स्ताभि॒ स्ताभि॒र् यज॑मा॒नो ऽव॑ । \newline
33. यज॑मा॒नो ऽवाव॒ यज॑मानो॒ यज॑मा॒नो ऽव॑ रुन्धे रु॒न्धे ऽव॒ यज॑मानो॒ यज॑मा॒नो ऽव॑ रुन्धे । \newline
34. अव॑ रुन्धे रु॒न्धे ऽवाव॑ रुन्धे॒ त्रय॑स्त्रिꣳशत॒म् त्रय॑स्त्रिꣳशतꣳ रु॒न्धे ऽवाव॑ रुन्धे॒ त्रय॑स्त्रिꣳशतम् । \newline
35. रु॒न्धे॒ त्रय॑स्त्रिꣳशत॒म् त्रय॑स्त्रिꣳशतꣳ रुन्धे रुन्धे॒ त्रय॑स्त्रिꣳशत॒ मुपोप॒ त्रय॑स्त्रिꣳशतꣳ रुन्धे रुन्धे॒ त्रय॑स्त्रिꣳशत॒ मुप॑ । \newline
36. त्रय॑स्त्रिꣳशत॒ मुपोप॒ त्रय॑स्त्रिꣳशत॒म् त्रय॑स्त्रिꣳशत॒ मुप॑ दधाति दधा॒ त्युप॒ त्रय॑स्त्रिꣳशत॒म् त्रय॑स्त्रिꣳशत॒ मुप॑ दधाति । \newline
37. त्रय॑स्त्रिꣳशत॒मिति॒ त्रयः॑ - त्रिꣳ॒॒श॒त॒म् । \newline
38. उप॑ दधाति दधा॒ त्युपोप॑ दधाति॒ त्रय॑स्त्रिꣳश॒त् त्रय॑स्त्रिꣳशद् दधा॒ त्युपोप॑ दधाति॒ त्रय॑स्त्रिꣳशत् । \newline
39. द॒धा॒ति॒ त्रय॑स्त्रिꣳश॒त् त्रय॑स्त्रिꣳशद् दधाति दधाति॒ त्रय॑स्त्रिꣳश॒द् वै वै त्रय॑स्त्रिꣳशद् दधाति दधाति॒ त्रय॑स्त्रिꣳश॒द् वै । \newline
40. त्रय॑स्त्रिꣳश॒द् वै वै त्रय॑स्त्रिꣳश॒त् त्रय॑स्त्रिꣳश॒द् वै दे॒वता॑ दे॒वता॒ वै त्रय॑स्त्रिꣳश॒त् त्रय॑स्त्रिꣳश॒द् वै दे॒वताः᳚ । \newline
41. त्रय॑स्त्रिꣳश॒दिति॒ त्रयः॑ - त्रिꣳ॒॒श॒त् । \newline
42. वै दे॒वता॑ दे॒वता॒ वै वै दे॒वताः᳚ । \newline
43. दे॒वता॑ दे॒वताः᳚ । \newline
44. दे॒वता॑ ए॒वैव दे॒वता॑ दे॒वता॑ ए॒वावा वै॒व दे॒वता॑ दे॒वता॑ ए॒वाव॑ । \newline
45. ए॒वावा वै॒वै वाव॑ रुन्धे रु॒न्धे ऽवै॒वै वाव॑ रुन्धे । \newline
46. अव॑ रुन्धे रु॒न्धे ऽवाव॑ रु॒न्धे ऽथो॒ अथो॑ रु॒न्धे ऽवाव॑ रु॒न्धे ऽथो᳚ । \newline
47. रु॒न्धे ऽथो॒ अथो॑ रुन्धे रु॒न्धे ऽथो॒ सात्मा॑नꣳ॒॒ सात्मा॑न॒ मथो॑ रुन्धे रु॒न्धे ऽथो॒ सात्मा॑नम् । \newline
48. अथो॒ सात्मा॑नꣳ॒॒ सात्मा॑न॒ मथो॒ अथो॒ सात्मा॑न मे॒वैव सात्मा॑न॒ मथो॒ अथो॒ सात्मा॑न मे॒व । \newline
49. अथो॒ इत्यथो᳚ । \newline
50. सात्मा॑न मे॒वैव सात्मा॑नꣳ॒॒ सात्मा॑न मे॒वाग्नि म॒ग्नि मे॒व सात्मा॑नꣳ॒॒ सात्मा॑न मे॒वाग्निम् । \newline
51. सात्मा॑न॒मिति॒ स - आ॒त्मा॒न॒म् । \newline
52. ए॒वाग्नि म॒ग्नि मे॒वैवाग्निꣳ सत॑नुꣳ॒॒ सत॑नु म॒ग्नि मे॒वै वाग्निꣳ सत॑नुम् । \newline
53. अ॒ग्निꣳ सत॑नुꣳ॒॒ सत॑नु म॒ग्नि म॒ग्निꣳ सत॑नुम् चिनुते चिनुते॒ सत॑नु म॒ग्नि म॒ग्निꣳ सत॑नुम् चिनुते । \newline
54. सत॑नुम् चिनुते चिनुते॒ सत॑नुꣳ॒॒ सत॑नुम् चिनुते॒ सात्मा॒ सात्मा॑ चिनुते॒ सत॑नुꣳ॒॒ सत॑नुम् चिनुते॒ सात्मा᳚ । \newline
55. सत॑नु॒मिति॒ स - त॒नु॒म् । \newline
56. चि॒नु॒ते॒ सात्मा॒ सात्मा॑ चिनुते चिनुते॒ सात्मा॒ ऽमुष्मि॑न् न॒मुष्मि॒न् थ्सात्मा॑ चिनुते चिनुते॒ सात्मा॒ ऽमुष्मिन्न्॑ । \newline
57. सात्मा॒ ऽमुष्मि॑न् न॒मुष्मि॒न् थ्सात्मा॒ सात्मा॒ ऽमुष्मि॑न् ॅलो॒के लो॒के॑ ऽमुष्मि॒न् थ्सात्मा॒ सात्मा॒ ऽमुष्मि॑न् ॅलो॒के । \newline
58. सात्मेति॒ स - आ॒त्मा॒ । \newline
59. अ॒मुष्मि॑न् ॅलो॒के लो॒के॑ ऽमुष्मि॑न् न॒मुष्मि॑न् ॅलो॒के भ॑वति भवति लो॒के॑ ऽमुष्मि॑न् न॒मुष्मि॑न् ॅलो॒के भ॑वति । \newline
60. लो॒के भ॑वति भवति लो॒के लो॒के भ॑वति॒ यो यो भ॑वति लो॒के लो॒के भ॑वति॒ यः । \newline
\pagebreak
\markright{ TS 5.4.1.3  \hfill https://www.vedavms.in \hfill}

\section{ TS 5.4.1.3 }

\textbf{TS 5.4.1.3 } \newline
\textbf{Samhita Paata} \newline

भ॑वति॒ य ए॒वं ॅवेद॒ ज्योति॑ष्मती॒रुप॑ दधाति॒ ज्योति॑रे॒वास्मि॑न् दधात्ये॒ताभि॒र्वा अ॒ग्निश्चि॒तो ज्व॑लति॒ ताभि॑रे॒वैनꣳ॒॒ समि॑न्ध उ॒भयो॑रस्मै लो॒कयो॒र्ज्योति॑र्भवति नक्षत्रेष्ट॒का उप॑ दधात्ये॒तानि॒ वै दि॒वो ज्योतीꣳ॑षि॒ तान्ये॒वाव॑ रुन्धे सु॒कृतां॒ ॅवा ए॒तानि॒ ज्योतीꣳ॑षि॒ यन्नक्ष॑त्राणि॒ तान्ये॒वाऽऽ*प्नो॒त्यथो॑ अनूका॒शमे॒वैतानि॒ - [  ] \newline

\textbf{Pada Paata} \newline

भ॒व॒ति॒ । यः । ए॒वम् । वेद॑ । ज्योति॑ष्मतीः । उपेति॑ । द॒धा॒ति॒ । ज्योतिः॑ । ए॒व । अ॒स्मि॒न्न् । द॒धा॒ति॒ । ए॒ताभिः॑ । वै । अ॒ग्निः । चि॒तः । ज्व॒ल॒ति॒ । ताभिः॑ । ए॒व । ए॒न॒म् । समिति॑ । इ॒न्धे॒ । उ॒भयोः᳚ । अ॒स्मै॒ । लो॒कयोः᳚ । ज्योतिः॑ । भ॒व॒ति॒ । न॒क्ष॒त्रे॒ष्ट॒का इति॑ नक्षत्र - इ॒ष्ट॒काः । उपेति॑ । द॒धा॒ति॒ । ए॒तानि॑ । वै । दि॒वः । ज्योतीꣳ॑षि । तानि॑ । ए॒व । अवेति॑ । रु॒न्धे॒ । सु॒कृता॒मिति॑ सु - कृता᳚म् । वै । ए॒तानि॑ । ज्योतीꣳ॑षि । यत् । नक्ष॑त्राणि । तानि॑ । ए॒व । आ॒प्नो॒ति॒ । अथो॒ इति॑ । अ॒नू॒का॒शमित्य॑नु - का॒शम् । ए॒व । ए॒तानि॑ ।  \newline


\textbf{Krama Paata} \newline

भ॒व॒ति॒ यः । य ए॒वम् । ए॒वम् ॅवेद॑ । वेद॒ ज्योति॑ष्मतीः । ज्योति॑ष्मती॒रुप॑ । उप॑ दधाति । द॒धा॒ति॒ ज्योतिः॑ । ज्योति॑रे॒व । ए॒वास्मिन्न्॑ । अ॒स्मि॒न् द॒धा॒ति॒ । द॒धा॒त्ये॒ताभिः॑ । ए॒ताभि॒र् वै । वा अ॒ग्निः । अ॒ग्निश्चि॒तः । चि॒तो ज्व॑लति । ज्व॒ल॒ति॒ ताभिः॑ । ताभि॑रे॒व । ए॒वैन᳚म् । ए॒नꣳ॒॒ सम् । समि॑न्धे । इ॒न्ध॒ उ॒भयोः᳚ । उ॒भयो॑रस्मै । अ॒स्मै॒ लो॒कयोः᳚ । लो॒कयो॒र् ज्योतिः॑ । ज्योति॑र् भवति । भ॒व॒ति॒ न॒क्ष॒त्रे॒ष्ट॒काः । न॒क्ष॒त्रे॒ष्ट॒का उप॑ । न॒क्ष॒त्रे॒ष्ट॒का इति॑ नक्षत्र - इ॒ष्ट॒काः । उप॑ दधाति । द॒धा॒त्ये॒तानि॑ । ए॒तानि॒ वै । वै दि॒वः । दि॒वो ज्योतीꣳ॑षि । ज्योतीꣳ॑षि॒ तानि॑ । तान्ये॒व । ए॒वाव॑ । अव॑ रुन्धे । रु॒न्धे॒ सु॒कृता᳚म् । सु॒कृता॒म् ॅवै । सु॒कृता॒मिति॑ सु - कृता᳚म् । वा ए॒तानि॑ । ए॒तानि॒ ज्योतीꣳ॑षि । ज्योतीꣳ॑षि॒ यत् । यन् नक्ष॑त्राणि । नक्ष॑त्राणि॒ तानि॑ । तान्ये॒व । ए॒वाप्नो॑ति । आ॒प्नो॒त्यथो᳚ । अथो॑ अनूका॒शम् । अथो॒ इत्यथो᳚ । अ॒नू॒का॒शमे॒व । अ॒नू॒का॒शमित्य॑नु - का॒शम् । ए॒वैतानि॑ ( ) । ए॒तानि॒ ज्योतीꣳ॑षि \newline

\textbf{Jatai Paata} \newline

1. भ॒व॒ति॒ यो यो भ॑वति भवति॒ यः । \newline
2. य ए॒व मे॒वं ॅयो य ए॒वम् । \newline
3. ए॒वं ॅवेद॒ वेदै॒व मे॒वं ॅवेद॑ । \newline
4. वेद॒ ज्योति॑ष्मती॒र् ज्योति॑ष्मती॒र् वेद॒ वेद॒ ज्योति॑ष्मतीः । \newline
5. ज्योति॑ष्मती॒ रुपोप॒ ज्योति॑ष्मती॒र् ज्योति॑ष्मती॒ रुप॑ । \newline
6. उप॑ दधाति दधा॒ त्युपोप॑ दधाति । \newline
7. द॒धा॒ति॒ ज्योति॒र् ज्योति॑र् दधाति दधाति॒ ज्योतिः॑ । \newline
8. ज्योति॑ रे॒वैव ज्योति॒र् ज्योति॑ रे॒व । \newline
9. ए॒वास्मि॑न् नस्मिन् ने॒वै वास्मिन्न्॑ । \newline
10. अ॒स्मि॒न् द॒धा॒ति॒ द॒धा॒ त्य॒स्मि॒न् न॒स्मि॒न् द॒धा॒ति॒ । \newline
11. द॒धा॒ त्ये॒ताभि॑ रे॒ताभि॑र् दधाति दधा त्ये॒ताभिः॑ । \newline
12. ए॒ताभि॒र् वै वा ए॒ताभि॑ रे॒ताभि॒र् वै । \newline
13. वा अ॒ग्नि र॒ग्निर् वै वा अ॒ग्निः । \newline
14. अ॒ग्नि श्चि॒त श्चि॒तो᳚ ऽग्नि र॒ग्नि श्चि॒तः । \newline
15. चि॒तो ज्व॑लति ज्वलति चि॒त श्चि॒तो ज्व॑लति । \newline
16. ज्व॒ल॒ति॒ ताभि॒ स्ताभि॑र् ज्वलति ज्वलति॒ ताभिः॑ । \newline
17. ताभि॑ रे॒वैव ताभि॒ स्ताभि॑ रे॒व । \newline
18. ए॒वैन॑ मेन मे॒वै वैन᳚म् । \newline
19. ए॒नꣳ॒॒ सꣳ स मे॑न मेनꣳ॒॒ सम् । \newline
20. स मि॑न्ध इन्धे॒ सꣳ स मि॑न्धे । \newline
21. इ॒न्ध॒ उ॒भयो॑ रु॒भयो॑ रिन्ध इन्ध उ॒भयोः᳚ । \newline
22. उ॒भयो॑ रस्मा अस्मा उ॒भयो॑ रु॒भयो॑ रस्मै । \newline
23. अ॒स्मै॒ लो॒कयो᳚र् लो॒कयो॑ रस्मा अस्मै लो॒कयोः᳚ । \newline
24. लो॒कयो॒र् ज्योति॒र् ज्योति॑र् लो॒कयो᳚र् लो॒कयो॒र् ज्योतिः॑ । \newline
25. ज्योति॑र् भवति भवति॒ ज्योति॒र् ज्योति॑र् भवति । \newline
26. भ॒व॒ति॒ न॒क्ष॒त्रे॒ष्ट॒का न॑क्षत्रेष्ट॒का भ॑वति भवति नक्षत्रेष्ट॒काः । \newline
27. न॒क्ष॒त्रे॒ष्ट॒का उपोप॑ नक्षत्रेष्ट॒का न॑क्षत्रेष्ट॒का उप॑ । \newline
28. न॒क्ष॒त्रे॒ष्ट॒का इति॑ नक्षत्र - इ॒ष्ट॒काः । \newline
29. उप॑ दधाति दधा॒ त्युपोप॑ दधाति । \newline
30. द॒धा॒ त्ये॒ता न्ये॒तानि॑ दधाति दधा त्ये॒तानि॑ । \newline
31. ए॒तानि॒ वै वा ए॒ता न्ये॒तानि॒ वै । \newline
32. वै दि॒वो दि॒वो वै वै दि॒वः । \newline
33. दि॒वो ज्योतीꣳ॑षि॒ ज्योतीꣳ॑षि दि॒वो दि॒वो ज्योतीꣳ॑षि । \newline
34. ज्योतीꣳ॑षि॒ तानि॒ तानि॒ ज्योतीꣳ॑षि॒ ज्योतीꣳ॑षि॒ तानि॑ । \newline
35. तान्ये॒वैव तानि॒ तान्ये॒व । \newline
36. ए॒वावा वै॒वै वाव॑ । \newline
37. अव॑ रुन्धे रु॒न्धे ऽवाव॑ रुन्धे । \newline
38. रु॒न्धे॒ सु॒कृताꣳ॑ सु॒कृताꣳ॑ रुन्धे रुन्धे सु॒कृता᳚म् । \newline
39. सु॒कृतां॒ ॅवै वै सु॒कृताꣳ॑ सु॒कृतां॒ ॅवै । \newline
40. सु॒कृता॒मिति॑ सु - कृता᳚म् । \newline
41. वा ए॒ता न्ये॒तानि॒ वै वा ए॒तानि॑ । \newline
42. ए॒तानि॒ ज्योतीꣳ॑षि॒ ज्योतीꣳ॑ ष्ये॒ता न्ये॒तानि॒ ज्योतीꣳ॑षि । \newline
43. ज्योतीꣳ॑षि॒ यद् यज् ज्योतीꣳ॑षि॒ ज्योतीꣳ॑षि॒ यत् । \newline
44. यन् नक्ष॑त्राणि॒ नक्ष॑त्राणि॒ यद् यन् नक्ष॑त्राणि । \newline
45. नक्ष॑त्राणि॒ तानि॒ तानि॒ नक्ष॑त्राणि॒ नक्ष॑त्राणि॒ तानि॑ । \newline
46. तान्ये॒ वैव तानि॒ तान्ये॒व । \newline
47. ए॒वाप्नो᳚ त्याप्नो त्ये॒वैवाप्नो॑ति । \newline
48. आ॒प्नो॒ त्यथो॒ अथो॑ आप्नो त्याप्नो॒ त्यथो᳚ । \newline
49. अथो॑ अनूका॒श म॑नूका॒श मथो॒ अथो॑ अनूका॒शम् । \newline
50. अथो॒ इत्यथो᳚ । \newline
51. अ॒नू॒का॒श मे॒वैवा नू॑का॒श म॑नूका॒श मे॒व । \newline
52. अ॒नू॒का॒शमित्य॑नु - का॒शम् । \newline
53. ए॒वैता न्ये॒ता न्ये॒वैवैतानि॑ । \newline
54. ए॒तानि॒ ज्योतीꣳ॑षि॒ ज्योतीꣳ॑ ष्ये॒ता न्ये॒तानि॒ ज्योतीꣳ॑षि । \newline

\textbf{Ghana Paata } \newline

1. भ॒व॒ति॒ यो यो भ॑वति भवति॒ य ए॒व मे॒वं ॅयो भ॑वति भवति॒ य ए॒वम् । \newline
2. य ए॒व मे॒वं ॅयो य ए॒वं ॅवेद॒ वेदै॒वं ॅयो य ए॒वं ॅवेद॑ । \newline
3. ए॒वं ॅवेद॒ वेदै॒व मे॒वं ॅवेद॒ ज्योति॑ष्मती॒र् ज्योति॑ष्मती॒र् वेदै॒व मे॒वं ॅवेद॒ ज्योति॑ष्मतीः । \newline
4. वेद॒ ज्योति॑ष्मती॒र् ज्योति॑ष्मती॒र् वेद॒ वेद॒ ज्योति॑ष्मती॒ रुपोप॒ ज्योति॑ष्मती॒र् वेद॒ वेद॒ ज्योति॑ष्मती॒ रुप॑ । \newline
5. ज्योति॑ष्मती॒ रुपोप॒ ज्योति॑ष्मती॒र् ज्योति॑ष्मती॒ रुप॑ दधाति दधा॒ त्युप॒ ज्योति॑ष्मती॒र् ज्योति॑ष्मती॒ रुप॑ दधाति । \newline
6. उप॑ दधाति दधा॒ त्युपोप॑ दधाति॒ ज्योति॒र् ज्योति॑र् दधा॒ त्युपोप॑ दधाति॒ ज्योतिः॑ । \newline
7. द॒धा॒ति॒ ज्योति॒र् ज्योति॑र् दधाति दधाति॒ ज्योति॑ रे॒वैव ज्योति॑र् दधाति दधाति॒ ज्योति॑ रे॒व । \newline
8. ज्योति॑ रे॒वैव ज्योति॒र् ज्योति॑ रे॒वास्मि॑न् नस्मिन् ने॒व ज्योति॒र् ज्योति॑ रे॒वास्मिन्न्॑ । \newline
9. ए॒वास्मि॑न् नस्मिन् ने॒वैवास्मि॑न् दधाति दधा त्यस्मिन् ने॒वै वास्मि॑न् दधाति । \newline
10. अ॒स्मि॒न् द॒धा॒ति॒ द॒धा॒ त्य॒स्मि॒न् न॒स्मि॒न् द॒धा॒ त्ये॒ताभि॑ रे॒ताभि॑र् दधा त्यस्मिन् नस्मिन् दधा त्ये॒ताभिः॑ । \newline
11. द॒धा॒ त्ये॒ताभि॑ रे॒ताभि॑र् दधाति दधा त्ये॒ताभि॒र् वै वा ए॒ताभि॑र् दधाति दधा त्ये॒ताभि॒र् वै । \newline
12. ए॒ताभि॒र् वै वा ए॒ताभि॑ रे॒ताभि॒र् वा अ॒ग्नि र॒ग्निर् वा ए॒ताभि॑ रे॒ताभि॒र् वा अ॒ग्निः । \newline
13. वा अ॒ग्नि र॒ग्निर् वै वा अ॒ग्नि श्चि॒त श्चि॒तो᳚ ऽग्निर् वै वा अ॒ग्नि श्चि॒तः । \newline
14. अ॒ग्नि श्चि॒त श्चि॒तो᳚ ऽग्नि र॒ग्नि श्चि॒तो ज्व॑लति ज्वलति चि॒तो᳚ ऽग्नि र॒ग्नि श्चि॒तो ज्व॑लति । \newline
15. चि॒तो ज्व॑लति ज्वलति चि॒त श्चि॒तो ज्व॑लति॒ ताभि॒ स्ताभि॑र् ज्वलति चि॒त श्चि॒तो ज्व॑लति॒ ताभिः॑ । \newline
16. ज्व॒ल॒ति॒ ताभि॒ स्ताभि॑र् ज्वलति ज्वलति॒ ताभि॑ रे॒वैव ताभि॑र् ज्वलति ज्वलति॒ ताभि॑ रे॒व । \newline
17. ताभि॑ रे॒वैव ताभि॒ स्ताभि॑ रे॒वैन॑ मेन मे॒व ताभि॒ स्ताभि॑ रे॒वैन᳚म् । \newline
18. ए॒वैन॑ मेन मे॒वै वैनꣳ॒॒ सꣳ स मे॑न मे॒वै वैनꣳ॒॒ सम् । \newline
19. ए॒नꣳ॒॒ सꣳ स मे॑न मेनꣳ॒॒ स मि॑न्ध इन्धे॒ स मे॑न मेनꣳ॒॒ स मि॑न्धे । \newline
20. स मि॑न्ध इन्धे॒ सꣳ स मि॑न्ध उ॒भयो॑ रु॒भयो॑ रिन्धे॒ सꣳ स मि॑न्ध उ॒भयोः᳚ । \newline
21. इ॒न्ध॒ उ॒भयो॑ रु॒भयो॑ रिन्ध इन्ध उ॒भयो॑ रस्मा अस्मा उ॒भयो॑ रिन्ध इन्ध उ॒भयो॑ रस्मै । \newline
22. उ॒भयो॑ रस्मा अस्मा उ॒भयो॑ रु॒भयो॑ रस्मै लो॒कयो᳚र् लो॒कयो॑ रस्मा उ॒भयो॑ रु॒भयो॑ रस्मै लो॒कयोः᳚ । \newline
23. अ॒स्मै॒ लो॒कयो᳚र् लो॒कयो॑ रस्मा अस्मै लो॒कयो॒र् ज्योति॒र् ज्योति॑र् लो॒कयो॑ रस्मा अस्मै लो॒कयो॒र् ज्योतिः॑ । \newline
24. लो॒कयो॒र् ज्योति॒र् ज्योति॑र् लो॒कयो᳚र् लो॒कयो॒र् ज्योति॑र् भवति भवति॒ ज्योति॑र् लो॒कयो᳚र् लो॒कयो॒र् ज्योति॑र् भवति । \newline
25. ज्योति॑र् भवति भवति॒ ज्योति॒र् ज्योति॑र् भवति नक्षत्रेष्ट॒का न॑क्षत्रेष्ट॒का भ॑वति॒ ज्योति॒र् ज्योति॑र् भवति नक्षत्रेष्ट॒काः । \newline
26. भ॒व॒ति॒ न॒क्ष॒त्रे॒ष्ट॒का न॑क्षत्रेष्ट॒का भ॑वति भवति नक्षत्रेष्ट॒का उपोप॑ नक्षत्रेष्ट॒का भ॑वति भवति नक्षत्रेष्ट॒का उप॑ । \newline
27. न॒क्ष॒त्रे॒ष्ट॒का उपोप॑ नक्षत्रेष्ट॒का न॑क्षत्रेष्ट॒का उप॑ दधाति दधा॒ त्युप॑ नक्षत्रेष्ट॒का न॑क्षत्रेष्ट॒का उप॑ दधाति । \newline
28. न॒क्ष॒त्रे॒ष्ट॒का इति॑ नक्षत्र - इ॒ष्ट॒काः । \newline
29. उप॑ दधाति दधा॒ त्युपोप॑ दधा त्ये॒ता न्ये॒तानि॑ दधा॒ त्युपोप॑ दधा त्ये॒तानि॑ । \newline
30. द॒धा॒ त्ये॒ता न्ये॒तानि॑ दधाति दधा त्ये॒तानि॒ वै वा ए॒तानि॑ दधाति दधा त्ये॒तानि॒ वै । \newline
31. ए॒तानि॒ वै वा ए॒ता न्ये॒तानि॒ वै दि॒वो दि॒वो वा ए॒ता न्ये॒तानि॒ वै दि॒वः । \newline
32. वै दि॒वो दि॒वो वै वै दि॒वो ज्योतीꣳ॑षि॒ ज्योतीꣳ॑षि दि॒वो वै वै दि॒वो ज्योतीꣳ॑षि । \newline
33. दि॒वो ज्योतीꣳ॑षि॒ ज्योतीꣳ॑षि दि॒वो दि॒वो ज्योतीꣳ॑षि॒ तानि॒ तानि॒ ज्योतीꣳ॑षि दि॒वो दि॒वो ज्योतीꣳ॑षि॒ तानि॑ । \newline
34. ज्योतीꣳ॑षि॒ तानि॒ तानि॒ ज्योतीꣳ॑षि॒ ज्योतीꣳ॑षि॒ तान्ये॒वैव तानि॒ ज्योतीꣳ॑षि॒ ज्योतीꣳ॑षि॒ तान्ये॒व । \newline
35. तान्ये॒ वैव तानि॒ तान्ये॒ वावा वै॒व तानि॒ तान्ये॒ वाव॑ । \newline
36. ए॒वावा वै॒वै वाव॑ रुन्धे रु॒न्धे ऽवै॒वै वाव॑ रुन्धे । \newline
37. अव॑ रुन्धे रु॒न्धे ऽवाव॑ रुन्धे सु॒कृताꣳ॑ सु॒कृताꣳ॑ रु॒न्धे ऽवाव॑ रुन्धे सु॒कृता᳚म् । \newline
38. रु॒न्धे॒ सु॒कृताꣳ॑ सु॒कृताꣳ॑ रुन्धे रुन्धे सु॒कृतां॒ ॅवै वै सु॒कृताꣳ॑ रुन्धे रुन्धे सु॒कृतां॒ ॅवै । \newline
39. सु॒कृतां॒ ॅवै वै सु॒कृताꣳ॑ सु॒कृतां॒ ॅवा ए॒ता न्ये॒तानि॒ वै सु॒कृताꣳ॑ सु॒कृतां॒ ॅवा ए॒तानि॑ । \newline
40. सु॒कृता॒मिति॑ सु - कृता᳚म् । \newline
41. वा ए॒ता न्ये॒तानि॒ वै वा ए॒तानि॒ ज्योतीꣳ॑षि॒ ज्योतीꣳ॑ ष्ये॒तानि॒ वै वा ए॒तानि॒ ज्योतीꣳ॑षि । \newline
42. ए॒तानि॒ ज्योतीꣳ॑षि॒ ज्योतीꣳ॑ ष्ये॒ता न्ये॒तानि॒ ज्योतीꣳ॑षि॒ यद् यज् ज्योतीꣳ॑ ष्ये॒ता न्ये॒तानि॒ ज्योतीꣳ॑षि॒ यत् । \newline
43. ज्योतीꣳ॑षि॒ यद् यज् ज्योतीꣳ॑षि॒ ज्योतीꣳ॑षि॒ यन् नक्ष॑त्राणि॒ नक्ष॑त्राणि॒ यज् ज्योतीꣳ॑षि॒ ज्योतीꣳ॑षि॒ यन् नक्ष॑त्राणि । \newline
44. यन् नक्ष॑त्राणि॒ नक्ष॑त्राणि॒ यद् यन् नक्ष॑त्राणि॒ तानि॒ तानि॒ नक्ष॑त्राणि॒ यद् यन् नक्ष॑त्राणि॒ तानि॑ । \newline
45. नक्ष॑त्राणि॒ तानि॒ तानि॒ नक्ष॑त्राणि॒ नक्ष॑त्राणि॒ तान्ये॒ वैव तानि॒ नक्ष॑त्राणि॒ नक्ष॑त्राणि॒ तान्ये॒व । \newline
46. तान्ये॒ वैव तानि॒ तान्ये॒वाप्नो᳚ त्याप्नो त्ये॒व तानि॒ तान्ये॒वा प्नो॑ति । \newline
47. ए॒वाप्नो᳚ त्याप्नो त्ये॒वैवा प्नो॒ त्यथो॒ अथो॑ आप्नो त्ये॒वैवा प्नो॒ त्यथो᳚ । \newline
48. आ॒प्नो॒ त्यथो॒ अथो॑ आप्नो त्याप्नो॒ त्यथो॑ अनूका॒श म॑नूका॒श मथो॑ आप्नो त्याप्नो॒ त्यथो॑ अनूका॒शम् । \newline
49. अथो॑ अनूका॒श म॑नूका॒श मथो॒ अथो॑ अनूका॒श मे॒वैवा नू॑का॒श मथो॒ अथो॑ अनूका॒श मे॒व । \newline
50. अथो॒ इत्यथो᳚ । \newline
51. अ॒नू॒का॒श मे॒वैवा नू॑का॒श म॑नूका॒श मे॒वै तान्ये॒ तान्ये॒ वानू॑का॒श म॑नूका॒श मे॒वैतानि॑ । \newline
52. अ॒नू॒का॒शमित्य॑नु - का॒शम् । \newline
53. ए॒वैता न्ये॒ता न्ये॒वैवै तानि॒ ज्योतीꣳ॑षि॒ ज्योतीꣳ॑ ष्ये॒ता न्ये॒वैवै तानि॒ ज्योतीꣳ॑षि । \newline
54. ए॒तानि॒ ज्योतीꣳ॑षि॒ ज्योतीꣳ॑ ष्ये॒ता न्ये॒तानि॒ ज्योतीꣳ॑षि कुरुते कुरुते॒ ज्योतीꣳ॑ ष्ये॒ता न्ये॒तानि॒ ज्योतीꣳ॑षि कुरुते । \newline
\pagebreak
\markright{ TS 5.4.1.4  \hfill https://www.vedavms.in \hfill}

\section{ TS 5.4.1.4 }

\textbf{TS 5.4.1.4 } \newline
\textbf{Samhita Paata} \newline

ज्योतीꣳ॑षि कुरुते सुव॒र्गस्य॑ लो॒कस्यानु॑ख्यात्यै॒ यथ् सꣳस्पृ॑ष्टा उपद॒द्ध्याद्-वृष्ट्यै॑ लो॒कमपि॑ दद्ध्या॒दव॑र्.षुकः प॒र्जन्यः॑ स्या॒दसꣳ॑स्पृष्टा॒ उप॑ दधाति॒ वृष्ट्या॑ ए॒व लो॒कं क॑रोति॒ वर्.षु॑कः प॒र्जन्यो॑ भवति पु॒रस्ता॑द॒न्याः प्र॒तीची॒रुप॑ दधाति प॒श्चाद॒न्याः प्राची॒स्तस्मा᳚त् प्रा॒चीना॑नि च प्रती॒चीना॑नि च॒ नक्ष॑त्रा॒ण्या व॑र्तन्ते ॥ \newline

\textbf{Pada Paata} \newline

ज्योतीꣳ॑षि । कु॒रु॒ते॒ । सु॒व॒र्गस्येति॑ सुवः - गस्य॑ । लो॒कस्य॑ । अनु॑ख्यात्या॒ इत्यनु॑ - ख्या॒त्यै॒ । यत् । सꣳस्पृ॑ष्टा॒ इति॒ सं - स्पृ॒ष्टाः॒ । उ॒प॒द॒द्ध्यादित्यु॑प-द॒द्ध्यात् । वृष्ट्यै᳚ । लो॒कम् । अपीति॑ । द॒द्ध्या॒त् । अव॑र्.षुकः । प॒र्जन्यः॑ । स्या॒त् । असꣳ॑स्पृष्टा॒ इत्यसं᳚ - स्पृ॒ष्टाः॒ । उपेति॑ । द॒धा॒ति॒ । वृष्ट्यै᳚ । ए॒व । लो॒कम् । क॒रो॒ति॒ । वर्.षु॑कः । प॒र्जन्यः॑ । भ॒व॒ति॒ । पु॒रस्ता᳚त् । अ॒न्याः । प्र॒तीचीः᳚ । उपेति॑ । द॒धा॒ति॒ । प॒श्चात् । अ॒न्याः । प्राचीः᳚ । तस्मा᳚त् । प्रा॒चीना॑नि । च॒ । प्र॒ती॒चीना॑नि । च॒ । नक्ष॑त्राणि । एति॑ । व॒र्त॒न्ते॒ ॥  \newline


\textbf{Krama Paata} \newline

ज्योतीꣳ॑षि कुरुते । कु॒रु॒ते॒ सु॒व॒र्गस्य॑ । सु॒व॒र्गस्य॑ लो॒कस्य॑ । सु॒व॒र्गस्येति॑ सुवः - गस्य॑ । लो॒कस्यानु॑ख्यात्यै । अनु॑ख्यात्यै॒ यत् । अनु॑ख्यात्या॒ इत्यनु॑ - ख्या॒त्यै॒ । यथ् सꣳस्पृ॑ष्टाः । सꣳस्पृ॑ष्टा उपद॒द्ध्यात् । सꣳस्पृ॑ष्टा॒ इति॒ सम् - स्पृ॒ष्टाः॒ । उ॒प॒द॒द्ध्याद् वृष्ट्यै᳚ । उ॒प॒द॒द्ध्यादित्यु॑प - द॒द्ध्यात् । वृष्ट्यै॑ लो॒कम् । लो॒कमपि॑ । अपि॑ दद्ध्यात् । द॒द्ध्या॒दव॑र्.षुकः । अव॑र्.षुकः प॒र्जन्यः॑ । प॒र्जन्यः॑ स्यात् । स्या॒दसꣳ॑स्पृष्टाः । असꣳ॑स्पृष्टा॒ उप॑ । असꣳ॑स्पृष्टा॒ इत्यस᳚म् - स्पृ॒ष्टाः॒ । उप॑ दधाति । द॒धा॒ति॒ वृष्ट्यै᳚ । वृष्ट्या॑ ए॒व । ए॒व लो॒कम् । लो॒कम् क॑रोति । क॒रो॒ति॒ वर्.षु॑कः । वर्.षु॑कः प॒र्जन्यः॑ । प॒र्जन्यो॑ भवति । भ॒व॒ति॒ पु॒रस्ता᳚त् । पु॒रस्ता॑द॒न्याः । अ॒न्याः प्र॒तीचीः᳚ । प्र॒तीची॒रुप॑ । उप॑ दधाति । द॒धा॒ति॒ प॒श्चात् । प॒श्चाद॒न्याः । अ॒न्याः प्राचीः᳚ । प्राची॒ स्तस्मा᳚त् । तस्मा᳚त् प्रा॒चीना॑नि । प्रा॒चीना॑नि च । च॒ प्र॒ती॒चीना॑नि । प्र॒ती॒चीना॑नि च । च॒ नक्ष॑त्राणि । नक्ष॑त्रा॒ण्या । आ व॑र्तन्ते । व॒र्त॒न्त॒ इति॑ वर्तन्ते । \newline

\textbf{Jatai Paata} \newline

1. ज्योतीꣳ॑षि कुरुते कुरुते॒ ज्योतीꣳ॑षि॒ ज्योतीꣳ॑षि कुरुते । \newline
2. कु॒रु॒ते॒ सु॒व॒र्गस्य॑ सुव॒र्गस्य॑ कुरुते कुरुते सुव॒र्गस्य॑ । \newline
3. सु॒व॒र्गस्य॑ लो॒कस्य॑ लो॒कस्य॑ सुव॒र्गस्य॑ सुव॒र्गस्य॑ लो॒कस्य॑ । \newline
4. सु॒व॒र्गस्येति॑ सुवः - गस्य॑ । \newline
5. लो॒कस्या नु॑ख्यात्या॒ अनु॑ख्यात्यै लो॒कस्य॑ लो॒कस्या नु॑ख्यात्यै । \newline
6. अनु॑ख्यात्यै॒ यद् यदनु॑ख्यात्या॒ अनु॑ख्यात्यै॒ यत् । \newline
7. अनु॑ख्यात्या॒ इत्यनु॑ - ख्या॒त्यै॒ । \newline
8. यथ् सꣳस्पृ॑ष्टाः॒ सꣳस्पृ॑ष्टा॒ यद् यथ् सꣳस्पृ॑ष्टाः । \newline
9. सꣳस्पृ॑ष्टा उपद॒द्ध्या दु॑पद॒द्ध्याथ् सꣳस्पृ॑ष्टाः॒ सꣳस्पृ॑ष्टा उपद॒द्ध्यात् । \newline
10. सꣳस्पृ॑ष्टा॒ इति॒ सं - स्पृ॒ष्टाः॒ । \newline
11. उ॒प॒द॒द्ध्याद् वृष्ट्यै॒ वृष्ट्या॑ उपद॒द्ध्या दु॑पद॒द्ध्याद् वृष्ट्यै᳚ । \newline
12. उ॒प॒द॒द्ध्यादित्यु॑प - द॒द्ध्यात् । \newline
13. वृष्ट्यै॑ लो॒कम् ॅलो॒कं ॅवृष्ट्यै॒ वृष्ट्यै॑ लो॒कम् । \newline
14. लो॒क मप्यपि॑ लो॒कम् ॅलो॒क मपि॑ । \newline
15. अपि॑ दद्ध्याद् दद्ध्या॒ दप्यपि॑ दद्ध्यात् । \newline
16. द॒द्ध्या॒ दव॑र्.षु॒को ऽव॑र्.षुको दद्ध्याद् दद्ध्या॒ दव॑र्.षुकः । \newline
17. अव॑र्.षुकः प॒र्जन्यः॑ प॒र्जन्यो ऽव॑र्.षु॒को ऽव॑र्.षुकः प॒र्जन्यः॑ । \newline
18. प॒र्जन्यः॑ स्याथ् स्यात् प॒र्जन्यः॑ प॒र्जन्यः॑ स्यात् । \newline
19. स्या॒ दसꣳ॑स्पृष्टा॒ असꣳ॑स्पृष्टाः स्याथ् स्या॒ दसꣳ॑स्पृष्टाः । \newline
20. असꣳ॑स्पृष्टा॒ उपोपा सꣳ॑स्पृष्टा॒ असꣳ॑स्पृष्टा॒ उप॑ । \newline
21. असꣳ॑स्पृष्टा॒ इत्यसं᳚ - स्पृ॒ष्टाः॒ । \newline
22. उप॑ दधाति दधा॒ त्युपोप॑ दधाति । \newline
23. द॒धा॒ति॒ वृष्ट्यै॒ वृष्ट्यै॑ दधाति दधाति॒ वृष्ट्यै᳚ । \newline
24. वृष्ट्या॑ ए॒वैव वृष्ट्यै॒ वृष्ट्या॑ ए॒व । \newline
25. ए॒व लो॒कम् ॅलो॒क मे॒वैव लो॒कम् । \newline
26. लो॒कम् क॑रोति करोति लो॒कम् ॅलो॒कम् क॑रोति । \newline
27. क॒रो॒ति॒ वर्.षु॑को॒ वर्.षु॑कः करोति करोति॒ वर्.षु॑कः । \newline
28. वर्.षु॑कः प॒र्जन्यः॑ प॒र्जन्यो॒ वर्.षु॑को॒ वर्.षु॑कः प॒र्जन्यः॑ । \newline
29. प॒र्जन्यो॑ भवति भवति प॒र्जन्यः॑ प॒र्जन्यो॑ भवति । \newline
30. भ॒व॒ति॒ पु॒रस्ता᳚त् पु॒रस्ता᳚द् भवति भवति पु॒रस्ता᳚त् । \newline
31. पु॒रस्ता॑ द॒न्या अ॒न्याः पु॒रस्ता᳚त् पु॒रस्ता॑ द॒न्याः । \newline
32. अ॒न्याः प्र॒तीचीः᳚ प्र॒तीची॑ र॒न्या अ॒न्याः प्र॒तीचीः᳚ । \newline
33. प्र॒तीची॒ रुपोप॑ प्र॒तीचीः᳚ प्र॒तीची॒ रुप॑ । \newline
34. उप॑ दधाति दधा॒ त्युपोप॑ दधाति । \newline
35. द॒धा॒ति॒ प॒श्चात् प॒श्चाद् द॑धाति दधाति प॒श्चात् । \newline
36. प॒श्चा द॒न्या अ॒न्याः प॒श्चात् प॒श्चा द॒न्याः । \newline
37. अ॒न्याः प्राचीः॒ प्राची॑ र॒न्या अ॒न्याः प्राचीः᳚ । \newline
38. प्राची॒स् तस्मा॒त् तस्मा॒त् प्राचीः॒ प्राची॒ स्तस्मा᳚त् । \newline
39. तस्मा᳚त् प्रा॒चीना॑नि प्रा॒चीना॑नि॒ तस्मा॒त् तस्मा᳚त् प्रा॒चीना॑नि । \newline
40. प्रा॒चीना॑नि च च प्रा॒चीना॑नि प्रा॒चीना॑नि च । \newline
41. च॒ प्र॒ती॒चीना॑नि प्रती॒चीना॑नि च च प्रती॒चीना॑नि । \newline
42. प्र॒ती॒चीना॑नि च च प्रती॒चीना॑नि प्रती॒चीना॑नि च । \newline
43. च॒ नक्ष॑त्राणि॒ नक्ष॑त्राणि च च॒ नक्ष॑त्राणि । \newline
44. नक्ष॑त्रा॒ण्या नक्ष॑त्राणि॒ नक्ष॑त्रा॒ण्या । \newline
45. आ व॑र्तन्ते वर्तन्त॒ आ व॑र्तन्ते । \newline
46. व॒र्त॒न्त॒ इति॑ वर्तन्ते । \newline

\textbf{Ghana Paata } \newline

1. ज्योतीꣳ॑षि कुरुते कुरुते॒ ज्योतीꣳ॑षि॒ ज्योतीꣳ॑षि कुरुते सुव॒र्गस्य॑ सुव॒र्गस्य॑ कुरुते॒ ज्योतीꣳ॑षि॒ ज्योतीꣳ॑षि कुरुते सुव॒र्गस्य॑ । \newline
2. कु॒रु॒ते॒ सु॒व॒र्गस्य॑ सुव॒र्गस्य॑ कुरुते कुरुते सुव॒र्गस्य॑ लो॒कस्य॑ लो॒कस्य॑ सुव॒र्गस्य॑ कुरुते कुरुते सुव॒र्गस्य॑ लो॒कस्य॑ । \newline
3. सु॒व॒र्गस्य॑ लो॒कस्य॑ लो॒कस्य॑ सुव॒र्गस्य॑ सुव॒र्गस्य॑ लो॒कस्या नु॑ख्यात्या॒ अनु॑ख्यात्यै लो॒कस्य॑ सुव॒र्गस्य॑ सुव॒र्गस्य॑ लो॒कस्या नु॑ख्यात्यै । \newline
4. सु॒व॒र्गस्येति॑ सुवः - गस्य॑ । \newline
5. लो॒कस्या नु॑ख्यात्या॒ अनु॑ख्यात्यै लो॒कस्य॑ लो॒कस्या नु॑ख्यात्यै॒ यद् यदनु॑ख्यात्यै लो॒कस्य॑ लो॒कस्या नु॑ख्यात्यै॒ यत् । \newline
6. अनु॑ख्यात्यै॒ यद् यदनु॑ख्यात्या॒ अनु॑ख्यात्यै॒ यथ् सꣳस्पृ॑ष्टाः॒ सꣳस्पृ॑ष्टा॒ यदनु॑ख्यात्या॒ अनु॑ख्यात्यै॒ यथ् सꣳस्पृ॑ष्टाः । \newline
7. अनु॑ख्यात्या॒ इत्यनु॑ - ख्या॒त्यै॒ । \newline
8. यथ् सꣳस्पृ॑ष्टाः॒ सꣳस्पृ॑ष्टा॒ यद् यथ् सꣳस्पृ॑ष्टा उपद॒द्ध्या दु॑पद॒द्ध्याथ् सꣳस्पृ॑ष्टा॒ यद् यथ् सꣳस्पृ॑ष्टा उपद॒द्ध्यात् । \newline
9. सꣳस्पृ॑ष्टा उपद॒द्ध्या दु॑पद॒द्ध्याथ् सꣳस्पृ॑ष्टाः॒ सꣳस्पृ॑ष्टा उपद॒द्ध्याद् वृष्ट्यै॒ वृष्ट्या॑ उपद॒द्ध्याथ् सꣳस्पृ॑ष्टाः॒ सꣳस्पृ॑ष्टा उपद॒द्ध्याद् वृष्ट्यै᳚ । \newline
10. सꣳस्पृ॑ष्टा॒ इति॒ सं - स्पृ॒ष्टाः॒ । \newline
11. उ॒प॒द॒द्ध्याद् वृष्ट्यै॒ वृष्ट्या॑ उपद॒द्ध्या दु॑पद॒द्ध्याद् वृष्ट्यै॑ लो॒कम् ॅलो॒कं ॅवृष्ट्या॑ उपद॒द्ध्या दु॑पद॒द्ध्याद् वृष्ट्यै॑ लो॒कम् । \newline
12. उ॒प॒द॒द्ध्यादित्यु॑प - द॒द्ध्यात् । \newline
13. वृष्ट्यै॑ लो॒कम् ॅलो॒कं ॅवृष्ट्यै॒ वृष्ट्यै॑ लो॒क मप्यपि॑ लो॒कं ॅवृष्ट्यै॒ वृष्ट्यै॑ लो॒क मपि॑ । \newline
14. लो॒क मप्यपि॑ लो॒कम् ॅलो॒क मपि॑ दद्ध्याद् दद्ध्या॒ दपि॑ लो॒कम् ॅलो॒क मपि॑ दद्ध्यात् । \newline
15. अपि॑ दद्ध्याद् दद्ध्या॒ दप्यपि॑ दद्ध्या॒ दव॑र्.षु॒को ऽव॑र्.षुको दद्ध्या॒ दप्यपि॑ दद्ध्या॒ दव॑र्.षुकः । \newline
16. द॒द्ध्या॒ दव॑र्.षु॒को ऽव॑र्.षुको दद्ध्याद् दद्ध्या॒ दव॑र्.षुकः प॒र्जन्यः॑ प॒र्जन्यो ऽव॑र्.षुको दद्ध्याद् दद्ध्या॒ दव॑र्.षुकः प॒र्जन्यः॑ । \newline
17. अव॑र्.षुकः प॒र्जन्यः॑ प॒र्जन्यो ऽव॑र्.षु॒को ऽव॑र्.षुकः प॒र्जन्यः॑ स्याथ् स्यात् प॒र्जन्यो ऽव॑र्.षु॒को ऽव॑र्.षुकः प॒र्जन्यः॑ स्यात् । \newline
18. प॒र्जन्यः॑ स्याथ् स्यात् प॒र्जन्यः॑ प॒र्जन्यः॑ स्या॒दसꣳ॑स्पृष्टा॒ असꣳ॑स्पृष्टाः स्यात् प॒र्जन्यः॑ प॒र्जन्यः॑ स्या॒दसꣳ॑स्पृष्टाः । \newline
19. स्या॒ दसꣳ॑स्पृष्टा॒ असꣳ॑स्पृष्टाः स्याथ् स्या॒ दसꣳ॑स्पृष्टा॒ उपोपासꣳ॑स्पृष्टाः स्याथ् स्या॒ दसꣳ॑स्पृष्टा॒ उप॑ । \newline
20. असꣳ॑स्पृष्टा॒ उपोपासꣳ॑स्पृष्टा॒ असꣳ॑स्पृष्टा॒ उप॑ दधाति दधा॒ त्युपासꣳ॑स्पृष्टा॒ असꣳ॑स्पृष्टा॒ उप॑ दधाति । \newline
21. असꣳ॑स्पृष्टा॒ इत्यसं᳚ - स्पृ॒ष्टाः॒ । \newline
22. उप॑ दधाति दधा॒ त्युपोप॑ दधाति॒ वृष्ट्यै॒ वृष्ट्यै॑ दधा॒ त्युपोप॑ दधाति॒ वृष्ट्यै᳚ । \newline
23. द॒धा॒ति॒ वृष्ट्यै॒ वृष्ट्यै॑ दधाति दधाति॒ वृष्ट्या॑ ए॒वैव वृष्ट्यै॑ दधाति दधाति॒ वृष्ट्या॑ ए॒व । \newline
24. वृष्ट्या॑ ए॒वैव वृष्ट्यै॒ वृष्ट्या॑ ए॒व लो॒कम् ॅलो॒क मे॒व वृष्ट्यै॒ वृष्ट्या॑ ए॒व लो॒कम् । \newline
25. ए॒व लो॒कम् ॅलो॒क मे॒वैव लो॒कम् क॑रोति करोति लो॒क मे॒वैव लो॒कम् क॑रोति । \newline
26. लो॒कम् क॑रोति करोति लो॒कम् ॅलो॒कम् क॑रोति॒ वर्.षु॑को॒ वर्.षु॑कः करोति लो॒कम् ॅलो॒कम् क॑रोति॒ वर्.षु॑कः । \newline
27. क॒रो॒ति॒ वर्.षु॑को॒ वर्.षु॑कः करोति करोति॒ वर्.षु॑कः प॒र्जन्यः॑ प॒र्जन्यो॒ वर्.षु॑कः करोति करोति॒ वर्.षु॑कः प॒र्जन्यः॑ । \newline
28. वर्.षु॑कः प॒र्जन्यः॑ प॒र्जन्यो॒ वर्.षु॑को॒ वर्.षु॑कः प॒र्जन्यो॑ भवति भवति प॒र्जन्यो॒ वर्.षु॑को॒ वर्.षु॑कः प॒र्जन्यो॑ भवति । \newline
29. प॒र्जन्यो॑ भवति भवति प॒र्जन्यः॑ प॒र्जन्यो॑ भवति पु॒रस्ता᳚त् पु॒रस्ता᳚द् भवति प॒र्जन्यः॑ प॒र्जन्यो॑ भवति पु॒रस्ता᳚त् । \newline
30. भ॒व॒ति॒ पु॒रस्ता᳚त् पु॒रस्ता᳚द् भवति भवति पु॒रस्ता॑ द॒न्या अ॒न्याः पु॒रस्ता᳚द् भवति भवति पु॒रस्ता॑ द॒न्याः । \newline
31. पु॒रस्ता॑ द॒न्या अ॒न्याः पु॒रस्ता᳚त् पु॒रस्ता॑ द॒न्याः प्र॒तीचीः᳚ प्र॒तीची॑ र॒न्याः पु॒रस्ता᳚त् पु॒रस्ता॑ द॒न्याः प्र॒तीचीः᳚ । \newline
32. अ॒न्याः प्र॒तीचीः᳚ प्र॒तीची॑ र॒न्या अ॒न्याः प्र॒तीची॒ रुपोप॑ प्र॒तीची॑ र॒न्या अ॒न्याः प्र॒तीची॒ रुप॑ । \newline
33. प्र॒तीची॒ रुपोप॑ प्र॒तीचीः᳚ प्र॒तीची॒ रुप॑ दधाति दधा॒ त्युप॑ प्र॒तीचीः᳚ प्र॒तीची॒ रुप॑ दधाति । \newline
34. उप॑ दधाति दधा॒ त्युपोप॑ दधाति प॒श्चात् प॒श्चाद् द॑धा॒ त्युपोप॑ दधाति प॒श्चात् । \newline
35. द॒धा॒ति॒ प॒श्चात् प॒श्चाद् द॑धाति दधाति प॒श्चा द॒न्या अ॒न्याः प॒श्चाद् द॑धाति दधाति प॒श्चा द॒न्याः । \newline
36. प॒श्चा द॒न्या अ॒न्याः प॒श्चात् प॒श्चा द॒न्याः प्राचीः॒ प्राची॑ र॒न्याः प॒श्चात् प॒श्चा द॒न्याः प्राचीः᳚ । \newline
37. अ॒न्याः प्राचीः॒ प्राची॑ र॒न्या अ॒न्याः प्राची॒ स्तस्मा॒त् तस्मा॒त् प्राची॑ र॒न्या अ॒न्याः प्राची॒ स्तस्मा᳚त् । \newline
38. प्राची॒ स्तस्मा॒त् तस्मा॒त् प्राचीः॒ प्राची॒ स्तस्मा᳚त् प्रा॒चीना॑नि प्रा॒चीना॑नि॒ तस्मा॒त् प्राचीः॒ प्राची॒ स्तस्मा᳚त् प्रा॒चीना॑नि । \newline
39. तस्मा᳚त् प्रा॒चीना॑नि प्रा॒चीना॑नि॒ तस्मा॒त् तस्मा᳚त् प्रा॒चीना॑नि च च प्रा॒चीना॑नि॒ तस्मा॒त् तस्मा᳚त् प्रा॒चीना॑नि च । \newline
40. प्रा॒चीना॑नि च च प्रा॒चीना॑नि प्रा॒चीना॑नि च प्रती॒चीना॑नि प्रती॒चीना॑नि च प्रा॒चीना॑नि प्रा॒चीना॑नि च प्रती॒चीना॑नि । \newline
41. च॒ प्र॒ती॒चीना॑नि प्रती॒चीना॑नि च च प्रती॒चीना॑नि च च प्रती॒चीना॑नि च च प्रती॒चीना॑नि च । \newline
42. प्र॒ती॒चीना॑नि च च प्रती॒चीना॑नि प्रती॒चीना॑नि च॒ नक्ष॑त्राणि॒ नक्ष॑त्राणि च प्रती॒चीना॑नि प्रती॒चीना॑नि च॒ नक्ष॑त्राणि । \newline
43. च॒ नक्ष॑त्राणि॒ नक्ष॑त्राणि च च॒ नक्ष॑त्रा॒ण्या नक्ष॑त्राणि च च॒ नक्ष॑त्रा॒ण्या । \newline
44. नक्ष॑त्रा॒ण्या नक्ष॑त्राणि॒ नक्ष॑त्रा॒ण्या व॑र्तन्ते वर्तन्त॒ आ नक्ष॑त्राणि॒ नक्ष॑त्रा॒ण्या व॑र्तन्ते । \newline
45. आ व॑र्तन्ते वर्तन्त॒ आ व॑र्तन्ते । \newline
46. व॒र्त॒न्त॒ इति॑ वर्तन्ते । \newline
\pagebreak
\markright{ TS 5.4.2.1  \hfill https://www.vedavms.in \hfill}

\section{ TS 5.4.2.1 }

\textbf{TS 5.4.2.1 } \newline
\textbf{Samhita Paata} \newline

ऋ॒त॒व्या॑ उप॑ दधात्यृतू॒नां क्लृप्त्यै᳚ द्व॒द्वंमुप॑ दधाति॒ तस्मा᳚द् द्व॒न्द्वमृ॒तवो ऽधृ॑तेव॒ वा ए॒षा यन्म॑द्ध्य॒मा चिति॑र॒न्तरि॑क्षमिव॒ वा ए॒षा द्व॒द्वंम॒न्यासु॒ चिती॒षूप॑ दधाति॒ चत॑स्रो॒ मद्ध्ये॒ धृत्या॑ अन्त॒श्श्लेष॑णं॒ ॅवा ए॒ताश्चिती॑नां॒ ॅयदृ॑त॒व्या॑ यदृ॑त॒व्या॑ उप॒दधा॑ति॒ चिती॑नां॒ ॅविधृ॑त्या॒ अव॑का॒मनूप॑ दधात्ये॒षा वा अ॒ग्नेर्योनिः॒ सयो॑नि - [  ] \newline

\textbf{Pada Paata} \newline

ऋ॒त॒व्याः᳚ । उपेति॑ । द॒धा॒ति॒ । ऋ॒तू॒नाम् । क्लृप्त्यै᳚ । द्व॒द्वंमिति॑ द्वं-द्वम् । उपेति॑ । द॒धा॒ति॒ । तस्मा᳚त् । द्व॒द्वंमिति॑ द्वं - द्वम् । ऋ॒तवः॑ । अधृ॑ता । इ॒व॒ । वै । ए॒षा । यत् । म॒द्ध्य॒मा । चितिः॑ । अ॒न्तरि॑क्षम् । इ॒व॒ । वै । ए॒षा । द्व॒द्वंमिति॑ द्वं - द्वम् । अ॒न्यासु॑ । चिती॑षु । उपेति॑ । द॒धा॒ति॒ । चत॑स्रः । मद्ध्ये᳚ । धृत्यै᳚ । अ॒न्त॒श्श्लेष॑ण॒मित्य॑न्तः - श्लेष॑णम् । वै । ए॒ताः । चिती॑नाम् । यत् । ऋ॒त॒व्याः᳚ । यत् । ऋ॒त॒व्याः᳚ । उ॒प॒दधा॒तीत्यु॑प - दधा॑ति । चिती॑नाम् । विधृ॑त्या॒ इति॒ वि - धृ॒त्यै॒ । अव॑काम् । अनु॑ । उपेति॑ । द॒धा॒ति॒ । ए॒षा । वै । अ॒ग्नेः । योनिः॑ । सयो॑नि॒मिति॒ स - यो॒नि॒म् ।  \newline


\textbf{Krama Paata} \newline

ऋ॒त॒व्या॑ उप॑ । उप॑ दधाति । द॒धा॒त्यृ॒तू॒नाम् । ऋ॒तू॒नाम् क्लृप्त्यै᳚ । क्लृप्त्यै᳚ द्व॒म्द्वम् । द्व॒म्द्वमुप॑ । द्व॒म्द्वमिति॑ द्वम् - द्वम् । उप॑ दधाति । द॒धा॒ति॒ तस्मा᳚त् । तस्मा᳚द् द्व॒म्द्वम् । द्व॒म्द्वमृ॒तवः॑ । द्व॒म्द्वमिति॑ द्वम् - द्वम् । ऋ॒तवोऽधृ॑ता । अधृ॑तेव । इ॒व॒ वै । वा ए॒षा । ए॒षा यत् । यन् म॑द्ध्य॒मा । म॒द्ध्य॒मा चितिः॑ । चिति॑र॒न्तरि॑क्षम् । अ॒न्तरि॑क्षमिव । इ॒व॒ वै । वा ए॒षा । ए॒षा द्व॒न्द्वम् । द्व॒न्द्वम॒न्यासु॑ । द्व॒न्द्वमिति॑ द्वम् - द्वम् । अ॒न्यासु॒ चिती॑षु । चिती॒षूप॑ । उप॑ दधाति । द॒धा॒ति॒ चत॑स्रः । चत॑स्रो॒ मद्ध्ये᳚ । मद्ध्ये॒ धृत्यै᳚ । धृत्या॑ अन्तः॒श्लेष॑णम् । अ॒न्तः॒श्लेष॑ण॒म् ॅवै । अ॒न्तः॒श्लेष॑ण॒मित्य॑न्तः - श्लेष॑णम् । वा ए॒ताः । ए॒ताश्चिती॑नाम् । चिती॑ना॒म् ॅयत् । यदृ॑त॒व्याः᳚ । ऋ॒त॒व्या॑ यत् । यदृ॑त॒व्याः᳚ । ऋ॒त॒व्या॑ उप॒दधा॑ति । उ॒प॒दधा॑ति॒ चिती॑नाम् । उ॒प॒दधा॒तीत्यु॑प - दधा॑ति । चिती॑ना॒म् ॅविधृ॑त्यै । विधृ॑त्या॒ अव॑काम् । विधृ॑त्या॒ इति॒ वि - धृ॒त्यै॒ । अव॑का॒मनु॑ । अनूप॑ । उप॑ दधाति । द॒धा॒त्ये॒षा । ए॒षा वै । वा अ॒ग्नेः । अ॒ग्नेर् योनिः॑ । योनिः॒ सयो॑निम् । सयो॑निमे॒व । सयो॑नि॒मिति॒ स - यो॒नि॒म् \newline

\textbf{Jatai Paata} \newline

1. ऋ॒त॒व्या॑ उपोपा᳚ र्‌त॒व्या॑ ऋत॒व्या॑ उप॑ । \newline
2. उप॑ दधाति दधा॒ त्युपोप॑ दधाति । \newline
3. द॒धा॒ त्यृ॒तू॒ना मृ॑तू॒नाम् द॑धाति दधा त्यृतू॒नाम् । \newline
4. ऋ॒तू॒नाम् क्लृप्त्यै॒ क्लृप्त्या॑ ऋतू॒ना मृ॑तू॒नाम् क्लृप्त्यै᳚ । \newline
5. क्लृप्त्यै᳚ द्व॒न्द्वम् द्व॒न्द्वम् क्लृप्त्यै॒ क्लृप्त्यै᳚ द्व॒न्द्वम् । \newline
6. द्व॒न्द्व मुपोप॑ द्व॒न्द्वम् द्व॒न्द्व मुप॑ । \newline
7. द्व॒न्द्वमिति॑ द्वं - द्वम् । \newline
8. उप॑ दधाति दधा॒ त्युपोप॑ दधाति । \newline
9. द॒धा॒ति॒ तस्मा॒त् तस्मा᳚द् दधाति दधाति॒ तस्मा᳚त् । \newline
10. तस्मा᳚द् द्व॒न्द्वम् द्व॒न्द्वम् तस्मा॒त् तस्मा᳚द् द्व॒न्द्वम् । \newline
11. द्व॒न्द्व मृ॒तव॑ ऋ॒तवो᳚ द्व॒न्द्वम् द्व॒न्द्व मृ॒तवः॑ । \newline
12. द्व॒न्द्वमिति॑ द्वं - द्वम् । \newline
13. ऋ॒तवो ऽधृ॒ता ऽधृ॑त॒ र्‌तव॑ ऋ॒तवो ऽधृ॑ता । \newline
14. अधृ॑तेवे॒ वाधृ॒ता ऽधृ॑तेव । \newline
15. इ॒व॒ वै वा इ॑वेव॒ वै । \newline
16. वा ए॒षैषा वै वा ए॒षा । \newline
17. ए॒षा यद् यदे॒षैषा यत् । \newline
18. यन् म॑द्ध्य॒मा म॑द्ध्य॒मा यद् यन् म॑द्ध्य॒मा । \newline
19. म॒द्ध्य॒मा चिति॒ श्चिति॑र् मद्ध्य॒मा म॑द्ध्य॒मा चितिः॑ । \newline
20. चिति॑ र॒न्तरि॑क्ष म॒न्तरि॑क्ष॒म् चिति॒ श्चिति॑ र॒न्तरि॑क्षम् । \newline
21. अ॒न्तरि॑क्ष मिवे वा॒न्तरि॑क्ष म॒न्तरि॑क्ष मिव । \newline
22. इ॒व॒ वै वा इ॑वेव॒ वै । \newline
23. वा ए॒षैषा वै वा ए॒षा । \newline
24. ए॒षा द्व॒न्द्वम् द्व॒न्द्व मे॒षैषा द्व॒न्द्वम् । \newline
25. द्व॒न्द्व म॒न्या स्व॒न्यासु॑ द्व॒न्द्वम् द्व॒न्द्व म॒न्यासु॑ । \newline
26. द्व॒न्द्वमिति॑ द्वं - द्वम् । \newline
27. अ॒न्यासु॒ चिती॑षु॒ चिती᳚ ष्व॒न्या स्व॒न्यासु॒ चिती॑षु । \newline
28. चिती॒षू पोप॒ चिती॑षु॒ चिती॒षूप॑ । \newline
29. उप॑ दधाति दधा॒ त्युपोप॑ दधाति । \newline
30. द॒धा॒ति॒ चत॑स्र॒ श्चत॑स्रो दधाति दधाति॒ चत॑स्रः । \newline
31. चत॑स्रो॒ मद्ध्ये॒ मद्ध्ये॒ चत॑स्र॒ श्चत॑स्रो॒ मद्ध्ये᳚ । \newline
32. मद्ध्ये॒ धृत्यै॒ धृत्यै॒ मद्ध्ये॒ मद्ध्ये॒ धृत्यै᳚ । \newline
33. धृत्या॑ अन्तः॒श्लेष॑ण मन्तः॒श्लेष॑ण॒म् धृत्यै॒ धृत्या॑ अन्तः॒श्लेष॑णम् । \newline
34. अ॒न्तः॒श्लेष॑णं॒ ॅवै वा अ॑न्तः॒श्लेष॑ण मन्तः॒श्लेष॑णं॒ ॅवै । \newline
35. अ॒न्तः॒श्लेष॑ण॒मित्य॑न्तः - श्लेष॑णम् । \newline
36. वा ए॒ता ए॒ता वै वा ए॒ताः । \newline
37. ए॒ता श्चिती॑ना॒म् चिती॑ना मे॒ता ए॒ता श्चिती॑नाम् । \newline
38. चिती॑नां॒ ॅयद् यच् चिती॑ना॒म् चिती॑नां॒ ॅयत् । \newline
39. यदृ॑त॒व्या॑ ऋत॒व्या॑ यद् यदृ॑त॒व्याः᳚ । \newline
40. ऋ॒त॒व्या॑ यद् यदृ॑त॒व्या॑ ऋत॒व्या॑ यत् । \newline
41. यदृ॑त॒व्या॑ ऋत॒व्या॑ यद् यदृ॑त॒व्याः᳚ । \newline
42. ऋ॒त॒व्या॑ उप॒दधा᳚ त्युप॒दधा᳚ त्यृत॒व्या॑ ऋत॒व्या॑ उप॒दधा॑ति । \newline
43. उ॒प॒दधा॑ति॒ चिती॑ना॒म् चिती॑ना मुप॒दधा᳚ त्युप॒दधा॑ति॒ चिती॑नाम् । \newline
44. उ॒प॒दधा॒तीत्यु॑प - दधा॑ति । \newline
45. चिती॑नां॒ ॅविधृ॑त्यै॒ विधृ॑त्यै॒ चिती॑ना॒म् चिती॑नां॒ ॅविधृ॑त्यै । \newline
46. विधृ॑त्या॒ अव॑का॒ मव॑कां॒ ॅविधृ॑त्यै॒ विधृ॑त्या॒ अव॑काम् । \newline
47. विधृ॑त्या॒ इति॒ वि - धृ॒त्यै॒ । \newline
48. अव॑का॒ मन्वन् वव॑का॒ मव॑का॒ मनु॑ । \newline
49. अनूपोपान् वनूप॑ । \newline
50. उप॑ दधाति दधा॒ त्युपोप॑ दधाति । \newline
51. द॒धा॒ त्ये॒षैषा द॑धाति दधा त्ये॒षा । \newline
52. ए॒षा वै वा ए॒षैषा वै । \newline
53. वा अ॒ग्ने र॒ग्नेर् वै वा अ॒ग्नेः । \newline
54. अ॒ग्नेर् योनि॒र् योनि॑ र॒ग्ने र॒ग्नेर् योनिः॑ । \newline
55. योनिः॒ सयो॑निꣳ॒॒ सयो॑निं॒ ॅयोनि॒र् योनिः॒ सयो॑निम् । \newline
56. सयो॑नि मे॒वैव सयो॑निꣳ॒॒ सयो॑नि मे॒व । \newline
57. सयो॑नि॒मिति॒ स - यो॒नि॒म् । \newline

\textbf{Ghana Paata } \newline

1. ऋ॒त॒व्या॑ उपोपा᳚ र्‌त॒व्या॑ ऋत॒व्या॑ उप॑ दधाति दधा॒ त्युपा᳚ र्‌त॒व्या॑ ऋत॒व्या॑ उप॑ दधाति । \newline
2. उप॑ दधाति दधा॒ त्युपोप॑ दधा त्यृतू॒ना मृ॑तू॒नाम् द॑धा॒ त्युपोप॑ दधा त्यृतू॒नाम् । \newline
3. द॒धा॒ त्यृ॒तू॒ना मृ॑तू॒नाम् द॑धाति दधा त्यृतू॒नाम् क्लृप्त्यै॒ क्लृप्त्या॑ ऋतू॒नाम् द॑धाति दधा त्यृतू॒नाम् क्लृप्त्यै᳚ । \newline
4. ऋ॒तू॒नाम् क्लृप्त्यै॒ क्लृप्त्या॑ ऋतू॒ना मृ॑तू॒नाम् क्लृप्त्यै᳚ द्व॒न्द्वम् द्व॒न्द्वम् क्लृप्त्या॑ ऋतू॒ना मृ॑तू॒नाम् क्लृप्त्यै᳚ द्व॒न्द्वम् । \newline
5. क्लृप्त्यै᳚ द्व॒न्द्वम् द्व॒न्द्वम् क्लृप्त्यै॒ क्लृप्त्यै᳚ द्व॒न्द्व मुपोप॑ द्व॒न्द्वम् क्लृप्त्यै॒ क्लृप्त्यै᳚ द्व॒न्द्व मुप॑ । \newline
6. द्व॒न्द्व मुपोप॑ द्व॒न्द्वम् द्व॒न्द्व मुप॑ दधाति दधा॒ त्युप॑ द्व॒न्द्वम् द्व॒न्द्व मुप॑ दधाति । \newline
7. द्व॒न्द्वमिति॑ द्वं - द्वम् । \newline
8. उप॑ दधाति दधा॒ त्युपोप॑ दधाति॒ तस्मा॒त् तस्मा᳚द् दधा॒ त्युपोप॑ दधाति॒ तस्मा᳚त् । \newline
9. द॒धा॒ति॒ तस्मा॒त् तस्मा᳚द् दधाति दधाति॒ तस्मा᳚द् द्व॒न्द्वम् द्व॒न्द्वम् तस्मा᳚द् दधाति दधाति॒ तस्मा᳚द् द्व॒न्द्वम् । \newline
10. तस्मा᳚द् द्व॒न्द्वम् द्व॒न्द्वम् तस्मा॒त् तस्मा᳚द् द्व॒न्द्व मृ॒तव॑ ऋ॒तवो᳚ द्व॒न्द्वम् तस्मा॒त् तस्मा᳚द् द्व॒न्द्व मृ॒तवः॑ । \newline
11. द्व॒न्द्व मृ॒तव॑ ऋ॒तवो᳚ द्व॒न्द्वम् द्व॒न्द्व मृ॒तवो ऽधृ॒ता ऽधृ॑त॒ र्‌तवो᳚ द्व॒न्द्वम् द्व॒न्द्व मृ॒तवो ऽधृ॑ता । \newline
12. द्व॒न्द्वमिति॑ द्वं - द्वम् । \newline
13. ऋ॒तवो ऽधृ॒ता ऽधृ॑त॒ र्‌तव॑ ऋ॒तवो ऽधृ॑तेवे॒ वाधृ॑त॒ र्‌तव॑ ऋ॒तवो ऽधृ॑तेव । \newline
14. अधृ॑तेवे॒ वाधृ॒ता ऽधृ॑तेव॒ वै वा इ॒वा धृ॒ता ऽधृ॑तेव॒ वै । \newline
15. इ॒व॒ वै वा इ॑वेव॒ वा ए॒षैषा वा इ॑वेव॒ वा ए॒षा । \newline
16. वा ए॒षैषा वै वा ए॒षा यद् यदे॒षा वै वा ए॒षा यत् । \newline
17. ए॒षा यद् यदे॒षैषा यन् म॑द्ध्य॒मा म॑द्ध्य॒मा यदे॒षैषा यन् म॑द्ध्य॒मा । \newline
18. यन् म॑द्ध्य॒मा म॑द्ध्य॒मा यद् यन् म॑द्ध्य॒मा चिति॒ श्चिति॑र् मद्ध्य॒मा यद् यन् म॑द्ध्य॒मा चितिः॑ । \newline
19. म॒द्ध्य॒मा चिति॒ श्चिति॑र् मद्ध्य॒मा म॑द्ध्य॒मा चिति॑ र॒न्तरि॑क्ष म॒न्तरि॑क्ष॒म् चिति॑र् मद्ध्य॒मा म॑द्ध्य॒मा चिति॑ र॒न्तरि॑क्षम् । \newline
20. चिति॑ र॒न्तरि॑क्ष म॒न्तरि॑क्ष॒म् चिति॒ श्चिति॑ र॒न्तरि॑क्ष मिवे वा॒न्तरि॑क्ष॒म् चिति॒ श्चिति॑ र॒न्तरि॑क्ष मिव । \newline
21. अ॒न्तरि॑क्ष मिवे वा॒न्तरि॑क्ष म॒न्तरि॑क्ष मिव॒ वै वा इ॑वा॒न्तरि॑क्ष म॒न्तरि॑क्ष मिव॒ वै । \newline
22. इ॒व॒ वै वा इ॑वेव॒ वा ए॒षैषा वा इ॑वेव॒ वा ए॒षा । \newline
23. वा ए॒षैषा वै वा ए॒षा द्व॒न्द्वम् द्व॒न्द्व मे॒षा वै वा ए॒षा द्व॒न्द्वम् । \newline
24. ए॒षा द्व॒न्द्वम् द्व॒न्द्व मे॒षैषा द्व॒न्द्व म॒न्या स्व॒न्यासु॑ द्व॒न्द्व मे॒षैषा द्व॒न्द्व म॒न्यासु॑ । \newline
25. द्व॒न्द्व म॒न्या स्व॒न्यासु॑ द्व॒न्द्वम् द्व॒न्द्व म॒न्यासु॒ चिती॑षु॒ चिती᳚ ष्व॒न्यासु॑ द्व॒न्द्वम् द्व॒न्द्व म॒न्यासु॒ चिती॑षु । \newline
26. द्व॒न्द्वमिति॑ द्वं - द्वम् । \newline
27. अ॒न्यासु॒ चिती॑षु॒ चिती᳚ ष्व॒न्या स्व॒न्यासु॒ चिती॒षू पोप॒ चिती᳚ ष्व॒न्या स्व॒न्यासु॒ चिती॒षूप॑ । \newline
28. चिती॒षू पोप॒ चिती॑षु॒ चिती॒षूप॑ दधाति दधा॒ त्युप॒ चिती॑षु॒ चिती॒षूप॑ दधाति । \newline
29. उप॑ दधाति दधा॒ त्युपोप॑ दधाति॒ चत॑स्र॒ श्चत॑स्रो दधा॒ त्युपोप॑ दधाति॒ चत॑स्रः । \newline
30. द॒धा॒ति॒ चत॑स्र॒ श्चत॑स्रो दधाति दधाति॒ चत॑स्रो॒ मद्ध्ये॒ मद्ध्ये॒ चत॑स्रो दधाति दधाति॒ चत॑स्रो॒ मद्ध्ये᳚ । \newline
31. चत॑स्रो॒ मद्ध्ये॒ मद्ध्ये॒ चत॑स्र॒ श्चत॑स्रो॒ मद्ध्ये॒ धृत्यै॒ धृत्यै॒ मद्ध्ये॒ चत॑स्र॒ श्चत॑स्रो॒ मद्ध्ये॒ धृत्यै᳚ । \newline
32. मद्ध्ये॒ धृत्यै॒ धृत्यै॒ मद्ध्ये॒ मद्ध्ये॒ धृत्या॑ अन्त॒श्श्लेष॑ण मन्त॒श्श्लेष॑ण॒म् धृत्यै॒ मद्ध्ये॒ मद्ध्ये॒ धृत्या॑ अन्त॒श्श्लेष॑णम् । \newline
33. धृत्या॑ अन्त॒श्श्लेष॑ण मन्त॒श्श्लेष॑ण॒म् धृत्यै॒ धृत्या॑ अन्त॒श्श्लेष॑णं॒ ॅवै वा अ॑न्त॒श्श्लेष॑ण॒म् धृत्यै॒ धृत्या॑ अन्त॒श्श्लेष॑णं॒ ॅवै । \newline
34. अ॒न्त॒श्श्लेष॑णं॒ ॅवै वा अ॑न्त॒श्श्लेष॑ण मन्त॒श्श्लेष॑णं॒ ॅवा ए॒ता ए॒ता वा अ॑न्त॒श्श्लेष॑ण मन्त॒श्श्लेष॑णं॒ ॅवा ए॒ताः । \newline
35. अ॒न्त॒श्श्लेष॑ण॒मित्य॑न्तः - श्लेष॑णम् । \newline
36. वा ए॒ता ए॒ता वै वा ए॒ता श्चिती॑ना॒म् चिती॑ना मे॒ता वै वा ए॒ता श्चिती॑नाम् । \newline
37. ए॒ता श्चिती॑ना॒म् चिती॑ना मे॒ता ए॒ता श्चिती॑नां॒ ॅयद् यच् चिती॑ना मे॒ता ए॒ता श्चिती॑नां॒ ॅयत् । \newline
38. चिती॑नां॒ ॅयद् यच् चिती॑ना॒म् चिती॑नां॒ ॅयदृ॑त॒व्या॑ ऋत॒व्या॑ यच् चिती॑ना॒म् चिती॑नां॒ ॅयदृ॑त॒व्याः᳚ । \newline
39. यदृ॑त॒व्या॑ ऋत॒व्या॑ यद् यदृ॑त॒व्या॑ यद् यदृ॑त॒व्या॑ यद् यदृ॑त॒व्या॑ यत् । \newline
40. ऋ॒त॒व्या॑ यद् यदृ॑त॒व्या॑ ऋत॒व्या॑ यदृ॑त॒व्या॑ ऋत॒व्या॑ यदृ॑त॒व्या॑ ऋत॒व्या॑ यदृ॑त॒व्याः᳚ । \newline
41. यदृ॑त॒व्या॑ ऋत॒व्या॑ यद् यदृ॑त॒व्या॑ उप॒दधा᳚ त्युप॒दधा᳚ त्यृत॒व्या॑ यद् यदृ॑त॒व्या॑ उप॒दधा॑ति । \newline
42. ऋ॒त॒व्या॑ उप॒दधा᳚ त्युप॒दधा᳚ त्यृत॒व्या॑ ऋत॒व्या॑ उप॒दधा॑ति॒ चिती॑ना॒म् चिती॑ना मुप॒दधा᳚ त्यृत॒व्या॑ ऋत॒व्या॑ उप॒दधा॑ति॒ चिती॑नाम् । \newline
43. उ॒प॒दधा॑ति॒ चिती॑ना॒म् चिती॑ना मुप॒दधा᳚ त्युप॒दधा॑ति॒ चिती॑नां॒ ॅविधृ॑त्यै॒ विधृ॑त्यै॒ चिती॑ना मुप॒दधा᳚ त्युप॒दधा॑ति॒ चिती॑नां॒ ॅविधृ॑त्यै । \newline
44. उ॒प॒दधा॒तीत्यु॑प - दधा॑ति । \newline
45. चिती॑नां॒ ॅविधृ॑त्यै॒ विधृ॑त्यै॒ चिती॑ना॒म् चिती॑नां॒ ॅविधृ॑त्या॒ अव॑का॒ मव॑कां॒ ॅविधृ॑त्यै॒ चिती॑ना॒म् चिती॑नां॒ ॅविधृ॑त्या॒ अव॑काम् । \newline
46. विधृ॑त्या॒ अव॑का॒ मव॑कां॒ ॅविधृ॑त्यै॒ विधृ॑त्या॒ अव॑का॒ मन् वन् वव॑कां॒ ॅविधृ॑त्यै॒ विधृ॑त्या॒ अव॑का॒ मनु॑ । \newline
47. विधृ॑त्या॒ इति॒ वि - धृ॒त्यै॒ । \newline
48. अव॑का॒ मन् वन् वव॑का॒ मव॑का॒ मनूपोपा न्वव॑का॒ मव॑का॒ मनूप॑ । \newline
49. अनूपोपा न्वनूप॑ दधाति दधा॒ त्युपान्व नूप॑ दधाति । \newline
50. उप॑ दधाति दधा॒ त्युपोप॑ दधा त्ये॒षैषा द॑धा॒ त्युपोप॑ दधा त्ये॒षा । \newline
51. द॒धा॒ त्ये॒षैषा द॑धाति दधा त्ये॒षा वै वा ए॒षा द॑धाति दधा त्ये॒षा वै । \newline
52. ए॒षा वै वा ए॒षैषा वा अ॒ग्ने र॒ग्नेर् वा ए॒षैषा वा अ॒ग्नेः । \newline
53. वा अ॒ग्ने र॒ग्नेर् वै वा अ॒ग्नेर् योनि॒र् योनि॑ र॒ग्नेर् वै वा अ॒ग्नेर् योनिः॑ । \newline
54. अ॒ग्नेर् योनि॒र् योनि॑ र॒ग्ने र॒ग्नेर् योनिः॒ सयो॑निꣳ॒॒ सयो॑निं॒ ॅयोनि॑ र॒ग्ने र॒ग्नेर् योनिः॒ सयो॑निम् । \newline
55. योनिः॒ सयो॑निꣳ॒॒ सयो॑निं॒ ॅयोनि॒र् योनिः॒ सयो॑नि मे॒वैव सयो॑निं॒ ॅयोनि॒र् योनिः॒ सयो॑नि मे॒व । \newline
56. सयो॑नि मे॒वैव सयो॑निꣳ॒॒ सयो॑नि मे॒वाग्नि म॒ग्नि मे॒व सयो॑निꣳ॒॒ सयो॑नि मे॒वाग्निम् । \newline
57. सयो॑नि॒मिति॒ स - यो॒नि॒म् । \newline
\pagebreak
\markright{ TS 5.4.2.2  \hfill https://www.vedavms.in \hfill}

\section{ TS 5.4.2.2 }

\textbf{TS 5.4.2.2 } \newline
\textbf{Samhita Paata} \newline

-मे॒वाग्निं चि॑नुत उ॒वाच॑ ह वि॒श्वामि॒त्रो ऽद॒दिथ् स ब्रह्म॒णाऽ*न्नं॒ ॅयस्यै॒ता उ॑पधी॒यान्तै॒ य उ॑ चैना ए॒वं ॅवेद॒दिति॑ संॅवथ्स॒रो वा ए॒तं प्र॑ति॒ष्ठायै॑ नुदते॒ यो᳚ऽग्निं चि॒त्वा न प्र॑ति॒तिष्ठ॑ति॒ पञ्च॒ पूर्वा॒श्चित॑यो भव॒न्त्यथ॑ ष॒ष्ठीं चितिं॑ चिनुते॒ षड्वा ऋ॒तवः॑ संॅवथ्स॒र ऋ॒तुष्वे॒व सं॑ॅवथ्स॒रे प्रति॑तिष्ठत्ये॒ ता वा - [  ] \newline

\textbf{Pada Paata} \newline

ए॒व । अ॒ग्निम् । चि॒नु॒ते॒ । उ॒वाच॑ । ह॒ । वि॒श्वामि॑त्र॒ इति॑ वि॒श्व-मि॒त्रः॒ । अद॑त् । इत् । सः । ब्रह्म॑णा । अन्न᳚म् । यस्य॑ । ए॒ताः । उ॒प॒धी॒यान्ता॒ इत्यु॑प - धी॒यान्तै᳚ । यः । उ॒ । च॒ । ए॒नाः॒ । ए॒वम् । वेद॑त् । इति॑ । सं॒ॅव॒थ्स॒र इति॑ सं - व॒थ्स॒रः । वै । ए॒तम् । प्र॒ति॒ष्ठाया॒ इति॑ प्रति - स्थायै᳚ । नु॒द॒ते॒ । यः । अ॒ग्निम् । चि॒त्वा । न । प्र॒ति॒तिष्ठ॒तीति॑ प्रति - तिष्ठ॑ति । पञ्च॑ । पूर्वाः᳚ । चित॑यः । भ॒व॒न्ति॒ । अथ॑ । ष॒ष्ठीम् । चिति᳚म् । चि॒नु॒ते॒ । षट् । वै । ऋ॒तवः॑ । सं॒ॅव॒थ्स॒र इति॑ सं-व॒थ्स॒रः । ऋ॒तुषु॑ । ए॒व । सं॒ॅव॒थ्स॒र इति॑ सं - व॒थ्स॒रे । प्रतीति॑ । ति॒ष्ठ॒ति॒ । ए॒ताः । वै ।  \newline


\textbf{Krama Paata} \newline

ए॒वाग्निम् । अ॒ग्निम् चि॑नुते । चि॒नु॒त॒ उ॒वाच॑ । उ॒वाच॑ ह । ह॒ वि॒श्वामि॑त्रः । वि॒श्वामि॒त्रोऽद॑त् । वि॒श्वामि॑त्र॒ इति॑ वि॒श्व - मि॒त्रः॒ । अद॒दित् । इथ् सः । स ब्रह्म॑णा । ब्रह्म॒णाऽन्न᳚म् । अन्न॒म् ॅयस्य॑ । यस्यै॒ताः । ए॒ता उ॑पधी॒यान्तै᳚ । उ॒प॒धी॒यान्तै॒ यः । उ॒प॒धी॒यान्ता॒ इत्यु॑प - धी॒यान्तै᳚ । य उ॑ । उ॒ च॒ । चै॒नाः॒ । ए॒ना॒ ए॒वम् । ए॒वम् ॅवेद॑त् । वेद॒दिति॑ । इति॑ सम्ॅवथ्स॒रः । स॒म्ॅव॒थ्स॒रो वै । स॒म्ॅव॒थ्स॒र इति॑ सम् - व॒थ्स॒रः । वा ए॒तम् । ए॒तम् प्र॑ति॒ष्ठायै᳚ । प्र॒ति॒ष्ठायै॑ नुदते । प्र॒ति॒ष्ठाया॒ इति॑ प्रति - स्थायै᳚ । नु॒द॒ते॒ यः । यो᳚ऽग्निम् । अ॒ग्निम् चि॒त्वा । चि॒त्वा न । न प्र॑ति॒तिष्ठ॑ति । प्र॒ति॒तिष्ठ॑ति॒ पञ्च॑ । प्र॒ति॒तिष्ठ॒तीति॑ प्रति - तिष्ठ॑ति । पञ्च॒ पूर्वाः᳚ । पूर्वा॒श्चित॑यः । चित॑यो भवन्ति । भ॒व॒न्त्यथ॑ । अथ॑ ष॒ष्ठीम् । ष॒ष्ठीम् चिति᳚म् । चिति॑म् चिनुते । चि॒नु॒ते॒ षट् । षड् वै । वा ऋ॒तवः॑ । ऋ॒तवः॑ सम्ॅवथ्स॒रः । स॒म्ॅव॒थ्स॒र ऋ॒तुषु॑ । स॒म्ॅव॒थ्स॒र इति॑ सम् - व॒थ्स॒रः । ऋ॒तुष्वे॒व । ए॒व स॑म्ॅवथ्स॒रे । स॒म्ॅव॒थ्स॒रे प्रति॑ । स॒म्ॅव॒थ्स॒र इति॑ सम् - व॒थ्स॒रे । प्रति॑ तिष्ठति । ति॒ष्ट॒त्ये॒ताः । ए॒ता वै । वा अधि॑पत्नीः \newline

\textbf{Jatai Paata} \newline

1. ए॒वाग्नि म॒ग्नि मे॒वै वाग्निम् । \newline
2. अ॒ग्निम् चि॑नुते चिनुते॒ ऽग्नि म॒ग्निम् चि॑नुते । \newline
3. चि॒नु॒त॒ उ॒वाचो॒ वाच॑ चिनुते चिनुत उ॒वाच॑ । \newline
4. उ॒वाच॑ ह हो॒वाचो॒ वाच॑ ह । \newline
5. ह॒ वि॒श्वामि॑त्रो वि॒श्वामि॑त्रो ह ह वि॒श्वामि॑त्रः । \newline
6. वि॒श्वामि॒त्रो ऽद॒ दद॑द् वि॒श्वामि॑त्रो वि॒श्वामि॒त्रो ऽद॑त् । \newline
7. वि॒श्वामि॑त्र॒ इति॑ वि॒श्व - मि॒त्रः॒ । \newline
8. अद॒ दिदि दद॒ दद॒ दित् । \newline
9. इथ् स सेदिथ् सः । \newline
10. स ब्रह्म॑णा॒ ब्रह्म॑णा॒ स स ब्रह्म॑णा । \newline
11. ब्रह्म॒णा ऽन्न॒ मन्न॒म् ब्रह्म॑णा॒ ब्रह्म॒णा ऽन्न᳚म् । \newline
12. अन्नं॒ ॅयस्य॒ यस्यान्न॒ मन्नं॒ ॅयस्य॑ । \newline
13. यस्यै॒ता ए॒ता यस्य॒ यस्यै॒ताः । \newline
14. ए॒ता उ॑प॒धीयान्ता॑ उप॒धीयान्ता॑ ए॒ता ए॒ता उ॑प॒धीयान्तै᳚ । \newline
15. उ॒प॒धीयान्तै॒ यो य उ॑प॒धीयान्ता॑ उप॒धीयान्तै॒ यः । \newline
16. उ॒प॒धी॒यान्ता॒ इत्यु॑प - धी॒यान्तै᳚ । \newline
17. य उ॑ वु॒ यो य उ॑ । \newline
18. उ॒ च॒ च॒ वु॒ च॒ । \newline
19. चै॒ना॒ ए॒ना॒श्च॒ चै॒नाः॒ । \newline
20. ए॒ना॒ ए॒व मे॒व मे॑ना एना ए॒वम् । \newline
21. ए॒वं ॅवेद॒द् वेद॑ दे॒व मे॒वं ॅवेद॑त् । \newline
22. वेद॒ दितीति॒ वेद॒द् वेद॒ दिति॑ । \newline
23. इति॑ संॅवथ्स॒रः सं॑ॅवथ्स॒र इतीति॑ संॅवथ्स॒रः । \newline
24. सं॒ॅव॒थ्स॒रो वै वै सं॑ॅवथ्स॒रः सं॑ॅवथ्स॒रो वै । \newline
25. सं॒ॅव॒थ्स॒र इति॑ सं - व॒थ्स॒रः । \newline
26. वा ए॒त मे॒तं ॅवै वा ए॒तम् । \newline
27. ए॒तम् प्र॑ति॒ष्ठायै᳚ प्रति॒ष्ठाया॑ ए॒त मे॒तम् प्र॑ति॒ष्ठायै᳚ । \newline
28. प्र॒ति॒ष्ठायै॑ नुदते नुदते प्रति॒ष्ठायै᳚ प्रति॒ष्ठायै॑ नुदते । \newline
29. प्र॒ति॒ष्ठाया॒ इति॑ प्रति - स्थायै᳚ । \newline
30. नु॒द॒ते॒ यो यो नु॑दते नुदते॒ यः । \newline
31. यो᳚ ऽग्नि म॒ग्निं ॅयो यो᳚ ऽग्निम् । \newline
32. अ॒ग्निम् चि॒त्वा चि॒त्वा ऽग्नि म॒ग्निम् चि॒त्वा । \newline
33. चि॒त्वा न न चि॒त्वा चि॒त्वा न । \newline
34. न प्र॑ति॒तिष्ठ॑ति प्रति॒तिष्ठ॑ति॒ न न प्र॑ति॒तिष्ठ॑ति । \newline
35. प्र॒ति॒तिष्ठ॑ति॒ पञ्च॒ पञ्च॑ प्रति॒तिष्ठ॑ति प्रति॒तिष्ठ॑ति॒ पञ्च॑ । \newline
36. प्र॒ति॒तिष्ठ॒तीति॑ प्रति - तिष्ठ॑ति । \newline
37. पञ्च॒ पूर्वाः॒ पूर्वाः॒ पञ्च॒ पञ्च॒ पूर्वाः᳚ । \newline
38. पूर्वा॒ श्चित॑य॒ श्चित॑यः॒ पूर्वाः॒ पूर्वा॒ श्चित॑यः । \newline
39. चित॑यो भवन्ति भवन्ति॒ चित॑य॒ श्चित॑यो भवन्ति । \newline
40. भ॒व॒न् त्यथाथ॑ भवन्ति भव॒न् त्यथ॑ । \newline
41. अथ॑ ष॒ष्ठीꣳ ष॒ष्ठी मथाथ॑ ष॒ष्ठीम् । \newline
42. ष॒ष्ठीम् चिति॒म् चितिꣳ॑ ष॒ष्ठीꣳ ष॒ष्ठीम् चिति᳚म् । \newline
43. चिति॑म् चिनुते चिनुते॒ चिति॒म् चिति॑म् चिनुते । \newline
44. चि॒नु॒ते॒ षट् थ्षट् चि॑नुते चिनुते॒ षट् । \newline
45. षड् वै वै षट् थ्षड् वै । \newline
46. वा ऋ॒तव॑ ऋ॒तवो॒ वै वा ऋ॒तवः॑ । \newline
47. ऋ॒तवः॑ संॅवथ्स॒रः सं॑ॅवथ्स॒र ऋ॒तव॑ ऋ॒तवः॑ संॅवथ्स॒रः । \newline
48. सं॒ॅव॒थ्स॒र ऋ॒तुष् वृ॒तुषु॑ संॅवथ्स॒रः सं॑ॅवथ्स॒र ऋ॒तुषु॑ । \newline
49. सं॒ॅव॒थ्स॒र इति॑ सं - व॒थ्स॒रः । \newline
50. ऋ॒तु ष्वे॒वैव र्‌तुष् वृ॒तु ष्वे॒व । \newline
51. ए॒व सं॑ॅवथ्स॒रे सं॑ॅवथ्स॒र ए॒वैव सं॑ॅवथ्स॒रे । \newline
52. सं॒ॅव॒थ्स॒रे प्रति॒ प्रति॑ संॅवथ्स॒रे सं॑ॅवथ्स॒रे प्रति॑ । \newline
53. सं॒ॅव॒थ्स॒र इति॑ सं - व॒थ्स॒रे । \newline
54. प्रति॑ तिष्ठति तिष्ठति॒ प्रति॒ प्रति॑ तिष्ठति । \newline
55. ति॒ष्ठ॒ त्ये॒ता ए॒ता स्ति॑ष्ठति तिष्ठ त्ये॒ताः । \newline
56. ए॒ता वै वा ए॒ता ए॒ता वै । \newline
57. वा अधि॑पत्नी॒ रधि॑पत्नी॒र् वै वा अधि॑पत्नीः । \newline

\textbf{Ghana Paata } \newline

1. ए॒वाग्नि म॒ग्नि मे॒वै वाग्निम् चि॑नुते चिनुते॒ ऽग्नि मे॒वै वाग्निम् चि॑नुते । \newline
2. अ॒ग्निम् चि॑नुते चिनुते॒ ऽग्नि म॒ग्निम् चि॑नुत उ॒वाचो॒ वाच॑ चिनुते॒ ऽग्नि म॒ग्निम् चि॑नुत उ॒वाच॑ । \newline
3. चि॒नु॒त॒ उ॒वाचो॒ वाच॑ चिनुते चिनुत उ॒वाच॑ ह हो॒वाच॑ चिनुते चिनुत उ॒वाच॑ ह । \newline
4. उ॒वाच॑ ह हो॒ वाचो॒ वाच॑ ह वि॒श्वामि॑त्रो वि॒श्वामि॑त्रो हो॒ वाचो॒ वाच॑ ह वि॒श्वामि॑त्रः । \newline
5. ह॒ वि॒श्वामि॑त्रो वि॒श्वामि॑त्रो ह ह वि॒श्वामि॒त्रो ऽद॒ दद॑द् वि॒श्वामि॑त्रो ह ह वि॒श्वामि॒त्रो ऽद॑त् । \newline
6. वि॒श्वामि॒त्रो ऽद॒ दद॑द् वि॒श्वामि॑त्रो वि॒श्वामि॒त्रो ऽद॒ दिदि दद॑द् वि॒श्वामि॑त्रो वि॒श्वामि॒त्रो ऽद॒दित् । \newline
7. वि॒श्वामि॑त्र॒ इति॑ वि॒श्व - मि॒त्रः॒ । \newline
8. अद॒ दिदिद् अद॒ दद॒ दिथ् स स इदद॒ दद॒ दिथ् सः । \newline
9. इथ् स स इदिथ् स ब्रह्म॑णा॒ ब्रह्म॑णा॒ स इदिथ् स ब्रह्म॑णा । \newline
10. स ब्रह्म॑णा॒ ब्रह्म॑णा॒ स स ब्रह्म॒णा ऽन्न॒ मन्न॒म् ब्रह्म॑णा॒ स स ब्रह्म॒णा ऽन्न᳚म् । \newline
11. ब्रह्म॒णा ऽन्न॒ मन्न॒म् ब्रह्म॑णा॒ ब्रह्म॒णा ऽन्नं॒ ॅयस्य॒ यस्यान्न॒म् ब्रह्म॑णा॒ ब्रह्म॒णा ऽन्नं॒ ॅयस्य॑ । \newline
12. अन्नं॒ ॅयस्य॒ यस्यान्न॒ मन्नं॒ ॅयस्यै॒ता ए॒ता यस्यान्न॒ मन्नं॒ ॅयस्यै॒ताः । \newline
13. यस्यै॒ता ए॒ता यस्य॒ यस्यै॒ता उ॑प॒धीयान्ता॑ उप॒धीयान्ता॑ ए॒ता यस्य॒ यस्यै॒ता उ॑प॒धीयान्तै᳚ । \newline
14. ए॒ता उ॑प॒धीयान्ता॑ उप॒धीयान्ता॑ ए॒ता ए॒ता उ॑प॒धीयान्तै॒ यो य उ॑प॒धीयान्ता॑ ए॒ता ए॒ता उ॑प॒धीयान्तै॒ यः । \newline
15. उ॒प॒धीयान्तै॒ यो य उ॑प॒धीयान्ता॑ उप॒धीयान्तै॒ य उ॑ वु॒ य उ॑प॒धीयान्ता॑ उप॒धीयान्तै॒ य उ॑ । \newline
16. उ॒प॒धी॒यान्ता॒ इत्यु॑प - धी॒यान्तै᳚ । \newline
17. य उ॑ वु॒ यो य उ॑ च च॒ चो य उ॑ च । \newline
18. उ॒ च॒ च॒ वु॒ चै॒ना॒ ए॒ना॒श्च॒ वु॒ चै॒नाः॒ । \newline
19. चै॒ना॒ ए॒ना॒श्च॒ चै॒ना॒ ए॒व मे॒व मे॑नाश्च चैना ए॒वम् । \newline
20. ए॒ना॒ ए॒व मे॒व मे॑ना एना ए॒वं ॅवेद॒द् वेद॑ दे॒व मे॑ना एना ए॒वं ॅवेद॑त् । \newline
21. ए॒वं ॅवेद॒द् वेद॑ दे॒व मे॒वं ॅवेद॒ दितीति॒ वेद॑ दे॒व मे॒वं ॅवेद॒ दिति॑ । \newline
22. वेद॒ दितीति॒ वेद॒द् वेद॒ दिति॑ संॅवथ्स॒रः सं॑ॅवथ्स॒र इति॒ वेद॒द् वेद॒ दिति॑ संॅवथ्स॒रः । \newline
23. इति॑ संॅवथ्स॒रः सं॑ॅवथ्स॒र इतीति॑ संॅवथ्स॒रो वै वै सं॑ॅवथ्स॒र इतीति॑ संॅवथ्स॒रो वै । \newline
24. सं॒ॅव॒थ्स॒रो वै वै सं॑ॅवथ्स॒रः सं॑ॅवथ्स॒रो वा ए॒त मे॒तं ॅवै सं॑ॅवथ्स॒रः सं॑ॅवथ्स॒रो वा ए॒तम् । \newline
25. सं॒ॅव॒थ्स॒र इति॑ सं - व॒थ्स॒रः । \newline
26. वा ए॒त मे॒तं ॅवै वा ए॒तम् प्र॑ति॒ष्ठायै᳚ प्रति॒ष्ठाया॑ ए॒तं ॅवै वा ए॒तम् प्र॑ति॒ष्ठायै᳚ । \newline
27. ए॒तम् प्र॑ति॒ष्ठायै᳚ प्रति॒ष्ठाया॑ ए॒त मे॒तम् प्र॑ति॒ष्ठायै॑ नुदते नुदते प्रति॒ष्ठाया॑ ए॒त मे॒तम् प्र॑ति॒ष्ठायै॑ नुदते । \newline
28. प्र॒ति॒ष्ठायै॑ नुदते नुदते प्रति॒ष्ठायै᳚ प्रति॒ष्ठायै॑ नुदते॒ यो यो नु॑दते प्रति॒ष्ठायै᳚ प्रति॒ष्ठायै॑ नुदते॒ यः । \newline
29. प्र॒ति॒ष्ठाया॒ इति॑ प्रति - स्थायै᳚ । \newline
30. नु॒द॒ते॒ यो यो नु॑दते नुदते॒ यो᳚ ऽग्नि म॒ग्निं ॅयो नु॑दते नुदते॒ यो᳚ ऽग्निम् । \newline
31. यो᳚ ऽग्नि म॒ग्निं ॅयो यो᳚ ऽग्निम् चि॒त्वा चि॒त्वा ऽग्निं ॅयो यो᳚ ऽग्निम् चि॒त्वा । \newline
32. अ॒ग्निम् चि॒त्वा चि॒त्वा ऽग्नि म॒ग्निम् चि॒त्वा न न चि॒त्वा ऽग्नि म॒ग्निम् चि॒त्वा न । \newline
33. चि॒त्वा न न चि॒त्वा चि॒त्वा न प्र॑ति॒तिष्ठ॑ति प्रति॒तिष्ठ॑ति॒ न चि॒त्वा चि॒त्वा न प्र॑ति॒तिष्ठ॑ति । \newline
34. न प्र॑ति॒तिष्ठ॑ति प्रति॒तिष्ठ॑ति॒ न न प्र॑ति॒तिष्ठ॑ति॒ पञ्च॒ पञ्च॑ प्रति॒तिष्ठ॑ति॒ न न प्र॑ति॒तिष्ठ॑ति॒ पञ्च॑ । \newline
35. प्र॒ति॒तिष्ठ॑ति॒ पञ्च॒ पञ्च॑ प्रति॒तिष्ठ॑ति प्रति॒तिष्ठ॑ति॒ पञ्च॒ पूर्वाः॒ पूर्वाः॒ पञ्च॑ प्रति॒तिष्ठ॑ति प्रति॒तिष्ठ॑ति॒ पञ्च॒ पूर्वाः᳚ । \newline
36. प्र॒ति॒तिष्ठ॒तीति॑ प्रति - तिष्ठ॑ति । \newline
37. पञ्च॒ पूर्वाः॒ पूर्वाः॒ पञ्च॒ पञ्च॒ पूर्वा॒ श्चित॑य॒ श्चित॑यः॒ पूर्वाः॒ पञ्च॒ पञ्च॒ पूर्वा॒ श्चित॑यः । \newline
38. पूर्वा॒ श्चित॑य॒ श्चित॑यः॒ पूर्वाः॒ पूर्वा॒ श्चित॑यो भवन्ति भवन्ति॒ चित॑यः॒ पूर्वाः॒ पूर्वा॒ श्चित॑यो भवन्ति । \newline
39. चित॑यो भवन्ति भवन्ति॒ चित॑य॒ श्चित॑यो भव॒न् त्यथाथ॑ भवन्ति॒ चित॑य॒ श्चित॑यो भव॒न् त्यथ॑ । \newline
40. भ॒व॒न् त्यथाथ॑ भवन्ति भव॒न् त्यथ॑ ष॒ष्ठीꣳ ष॒ष्ठी मथ॑ भवन्ति भव॒न् त्यथ॑ ष॒ष्ठीम् । \newline
41. अथ॑ ष॒ष्ठीꣳ ष॒ष्ठी मथाथ॑ ष॒ष्ठीम् चिति॒म् चितिꣳ॑ ष॒ष्ठी मथाथ॑ ष॒ष्ठीम् चिति᳚म् । \newline
42. ष॒ष्ठीम् चिति॒म् चितिꣳ॑ ष॒ष्ठीꣳ ष॒ष्ठीम् चिति॑म् चिनुते चिनुते॒ चितिꣳ॑ ष॒ष्ठीꣳ ष॒ष्ठीम् चिति॑म् चिनुते । \newline
43. चिति॑म् चिनुते चिनुते॒ चिति॒म् चिति॑म् चिनुते॒ षट् थ्षट् चि॑नुते॒ चिति॒म् चिति॑म् चिनुते॒ षट् । \newline
44. चि॒नु॒ते॒ षट् थ्षट् चि॑नुते चिनुते॒ षड् वै वै षट् चि॑नुते चिनुते॒ षड् वै । \newline
45. षड् वै वै षट् थ्षड् वा ऋ॒तव॑ ऋ॒तवो॒ वै षट् थ्षड् वा ऋ॒तवः॑ । \newline
46. वा ऋ॒तव॑ ऋ॒तवो॒ वै वा ऋ॒तवः॑ संॅवथ्स॒रः सं॑ॅवथ्स॒र ऋ॒तवो॒ वै वा ऋ॒तवः॑ संॅवथ्स॒रः । \newline
47. ऋ॒तवः॑ संॅवथ्स॒रः सं॑ॅवथ्स॒र ऋ॒तव॑ ऋ॒तवः॑ संॅवथ्स॒र ऋ॒तु ष्वृ॒तुषु॑ संॅवथ्स॒र ऋ॒तव॑ ऋ॒तवः॑ संॅवथ्स॒र ऋ॒तुषु॑ । \newline
48. सं॒ॅव॒थ्स॒र ऋ॒तु ष्वृ॒तुषु॑ संॅवथ्स॒रः सं॑ॅवथ्स॒र ऋ॒तु ष्वे॒वैव र्‌तुषु॑ संॅवथ्स॒रः सं॑ॅवथ्स॒र ऋ॒तु ष्वे॒व । \newline
49. सं॒ॅव॒थ्स॒र इति॑ सं - व॒थ्स॒रः । \newline
50. ऋ॒तु ष्वे॒वैव र्‌तु ष्वृ॒तु ष्वे॒व सं॑ॅवथ्स॒रे सं॑ॅवथ्स॒र ए॒व र्‌तुष् वृ॒तुष् वे॒व सं॑ॅवथ्स॒रे । \newline
51. ए॒व सं॑ॅवथ्स॒रे सं॑ॅवथ्स॒र ए॒वैव सं॑ॅवथ्स॒रे प्रति॒ प्रति॑ संॅवथ्स॒र ए॒वैव सं॑ॅवथ्स॒रे प्रति॑ । \newline
52. सं॒ॅव॒थ्स॒रे प्रति॒ प्रति॑ संॅवथ्स॒रे सं॑ॅवथ्स॒रे प्रति॑ तिष्ठति तिष्ठति॒ प्रति॑ संॅवथ्स॒रे सं॑ॅवथ्स॒रे प्रति॑ तिष्ठति । \newline
53. सं॒ॅव॒थ्स॒र इति॑ सं - व॒थ्स॒रे । \newline
54. प्रति॑ तिष्ठति तिष्ठति॒ प्रति॒ प्रति॑ तिष्ठ त्ये॒ता ए॒ता स्ति॑ष्ठति॒ प्रति॒ प्रति॑ तिष्ठ त्ये॒ताः । \newline
55. ति॒ष्ठ॒ त्ये॒ता ए॒ता स्ति॑ष्ठति तिष्ठ त्ये॒ता वै वा ए॒ता स्ति॑ष्ठति तिष्ठ त्ये॒ता वै । \newline
56. ए॒ता वै वा ए॒ता ए॒ता वा अधि॑पत्नी॒ रधि॑पत्नी॒र् वा ए॒ता ए॒ता वा अधि॑पत्नीः । \newline
57. वा अधि॑पत्नी॒ रधि॑पत्नी॒र् वै वा अधि॑पत्नी॒र् नाम॒ नामा धि॑पत्नी॒र् वै वा अधि॑पत्नी॒र् नाम॑ । \newline
\pagebreak
\markright{ TS 5.4.2.3  \hfill https://www.vedavms.in \hfill}

\section{ TS 5.4.2.3 }

\textbf{TS 5.4.2.3 } \newline
\textbf{Samhita Paata} \newline

अधि॑पत्नी॒र्नामेष्ट॑का॒ यस्यै॒ता उ॑पधी॒यन्तेऽधि॑पतिरे॒व स॑मा॒नानां᳚ भवति॒ यं द्वि॒ष्यात् तमु॑प॒दध॑द् ध्यायेदे॒ताभ्य॑ ए॒वैनं॑ देवता᳚भ्य॒ आ वृ॑श्चति ता॒जगार्ति॒मार्च्छ॒त्यङ्गि॑रसः सुव॒र्गं ॅलो॒कं ॅयन्तो॒ या य॒ज्ञ्स्य॒ निष्कृ॑ति॒रासी॒त् तामृषि॑भ्यः॒ प्रत्यौ॑ह॒न् तद्धिर॑ण्यमभव॒द्य-द्धि॑रण्यश॒ल्कैः प्रो॒क्षति॑ य॒ज्ञ्स्य॒ निष्कृ॑त्या॒ अथो॑ भेष॒जमे॒वास्मै॑ करो॒त्य - [  ] \newline

\textbf{Pada Paata} \newline

अधि॑पत्नी॒रित्यधि॑ - प॒त्नीः॒ । नाम॑ । इष्ट॑काः । यस्य॑ । ए॒ताः । उ॒प॒धी॒यन्त॒ इत्यु॑प - धी॒यन्ते᳚ । अधि॑पति॒रित्यधि॑ - प॒तिः॒ । ए॒व । स॒मा॒नाना᳚म् । भ॒व॒ति॒ । यम् । द्वि॒ष्यात् । तम् । उ॒प॒दध॒दित्यु॑प-दध॑त् । ध्या॒ये॒त् । ए॒ताभ्यः॑ । ए॒व । ए॒न॒म् । दे॒वता᳚भ्यः । एति॑ । वृ॒श्च॒ति॒ । ता॒जक् । आर्ति᳚म् । एति॑ । ऋ॒च्छ॒ति॒ । अङ्गि॑रसः । सु॒व॒र्गमिति॑ सुवः - गम् । लो॒कम् । यन्तः॑ । या । य॒ज्ञ्स्य॑ । निष्कृ॑ति॒रिति॒ निः-कृ॒तिः॒ । आसी᳚त् । ताम् । ऋषि॑भ्य॒ इत्यृषि॑-भ्यः॒ । प्रतीति॑ । औ॒ह॒न्न् । तत् । हिर॑ण्यम् । अ॒भ॒व॒त् । यत् । हि॒र॒ण्य॒श॒ल्कैरिति॑ हिरण्य - श॒ल्कैः । प्रो॒क्षतीति॑ प्र - उ॒क्षति॑ । य॒ज्ञ्स्य॑ । निष्कृ॑त्या॒ इति॒ निः - कृ॒त्यै॒ । अथो॒ इति॑ । भे॒ष॒जम् । ए॒व । अ॒स्मै॒ । क॒रो॒ति॒ ।  \newline


\textbf{Krama Paata} \newline

अधि॑पत्नी॒र् नाम॑ । अधि॑पत्नी॒रित्यधि॑ - प॒त्नीः॒ । नामेष्ट॑काः । इष्ट॑का॒ यस्य॑ । यस्यै॒ताः । ए॒ता उ॑पधी॒यन्ते᳚ । उ॒प॒धी॒यन्तेऽधि॑पतिः । उ॒प॒धी॒यन्त॒ इत्यु॑प - धी॒यन्ते᳚ । अधि॑पतिरे॒व । अधि॑पति॒रित्यधि॑ - प॒तिः॒ । ए॒व स॑मा॒नाना᳚म् । स॒मा॒नाना᳚म् भवति । भ॒व॒ति॒ यम् । यम् द्वि॒ष्यात् । द्वि॒ष्यात् तम् । तमु॑प॒दध॑त् । उ॒प॒दध॑द् ध्यायेत् । उ॒प॒दध॒दित्यु॑प - दध॑त् । ध्या॒ये॒दे॒ताभ्यः॑ । ए॒ताभ्य॑ ए॒व । ए॒वैन᳚म् । ए॒न॒म् दे॒वता᳚भ्यः । दे॒वता᳚भ्य॒ आ । आ वृ॑श्चति । वृ॒श्च॒ति॒ ता॒जक् । ता॒जगार्ति᳚म् । 
आर्ति॒मा । आर्च्छ॑ति । ऋ॒च्छ॒त्यङ्गि॑रसः । अङ्गि॑रसः सुव॒र्गम् । सु॒व॒र्गम् ॅलो॒कम् । सु॒व॒र्गमिति॑ सुवः - गम् । लो॒कम् ॅयन्तः॑ । यन्तो॒ या । या य॒ज्ञ्स्य॑ । य॒ज्ञ्स्य॒ निष्कृ॑तिः । निष्कृ॑ति॒रासी᳚त् । निष्कृ॑ति॒रिति॒ निः - कृ॒तिः॒ । आसी॒त् ताम् । तामृषि॑भ्यः । ऋषि॑भ्यः॒ प्रति॑ । ऋषि॑भ्य॒ इत्यृषि॑ - भ्यः॒ । प्रत्यौ॑हन्न् । औ॒ह॒न् तत् । तद्धिर॑ण्यम् । हिर॑ण्यमभवत् । अ॒भ॒व॒द् यत् । यद्धि॑रण्यश॒ल्कैः । हि॒र॒ण्य॒श॒ल्कैः प्रो॒क्षति॑ । हि॒र॒ण्य॒श॒ल्कैरिति॑ हिरण्य - श॒ल्कैः । प्रो॒क्षति॑ य॒ज्ञ्स्य॑ । प्रो॒क्षतीति॑ प्र - उ॒क्षति॑ । य॒ज्ञ्स्य॒ निष्कृ॑त्यै । निष्कृ॑त्या॒ अथो᳚ । निष्कृ॑त्या॒ इति॒ निः - कृ॒त्यै॒ । 
अथो॑ भेष॒जम् । अथो॒ इत्यथो᳚ । भे॒ष॒जमे॒व । ए॒वास्मै᳚ । 
अ॒स्मै॒ क॒रो॒ति॒ ( ) । क॒रो॒त्यथो᳚ \newline

\textbf{Jatai Paata} \newline

1. अधि॑पत्नी॒र् नाम॒ नामा धि॑पत्नी॒ रधि॑पत्नी॒र् नाम॑ । \newline
2. अधि॑पत्नी॒रित्यधि॑ - प॒त्नीः॒ । \newline
3. नामेष्ट॑का॒ इष्ट॑का॒ नाम॒ नामेष्ट॑काः । \newline
4. इष्ट॑का॒ यस्य॒ यस्येष्ट॑का॒ इष्ट॑का॒ यस्य॑ । \newline
5. यस्यै॒ता ए॒ता यस्य॒ यस्यै॒ताः । \newline
6. ए॒ता उ॑पधी॒यन्त॑ उपधी॒यन्त॑ ए॒ता ए॒ता उ॑पधी॒यन्ते᳚ । \newline
7. उ॒प॒धी॒यन्ते ऽधि॑पति॒ रधि॑पति रुपधी॒यन्त॑ उपधी॒यन्ते ऽधि॑पतिः । \newline
8. उ॒प॒धी॒यन्त॒ इत्यु॑प - धी॒यन्ते᳚ । \newline
9. अधि॑पति रे॒वैवा धि॑पति॒ रधि॑पति रे॒व । \newline
10. अधि॑पति॒रित्यधि॑ - प॒तिः॒ । \newline
11. ए॒व स॑मा॒नानाꣳ॑ समा॒नाना॑ मे॒वैव स॑मा॒नाना᳚म् । \newline
12. स॒मा॒नाना᳚म् भवति भवति समा॒नानाꣳ॑ समा॒नाना᳚म् भवति । \newline
13. भ॒व॒ति॒ यं ॅयम् भ॑वति भवति॒ यम् । \newline
14. यम् द्वि॒ष्याद् द्वि॒ष्याद् यं ॅयम् द्वि॒ष्यात् । \newline
15. द्वि॒ष्यात् तम् तम् द्वि॒ष्याद् द्वि॒ष्यात् तम् । \newline
16. त मु॑प॒दध॑ दुप॒दध॒त् तम् त मु॑प॒दध॑त् । \newline
17. उ॒प॒दध॑द् ध्यायेद् ध्याये दुप॒दध॑ दुप॒दध॑द् ध्यायेत् । \newline
18. उ॒प॒दध॒दित्यु॑प - दध॑त् । \newline
19. ध्या॒ये॒ दे॒ताभ्य॑ ए॒ताभ्यो᳚ ध्यायेद् ध्याये दे॒ताभ्यः॑ । \newline
20. ए॒ताभ्य॑ ए॒वैवैताभ्य॑ ए॒ताभ्य॑ ए॒व । \newline
21. ए॒वैन॑ मेन मे॒वै वैन᳚म् । \newline
22. ए॒न॒म् दे॒वता᳚भ्यो दे॒वता᳚भ्य एन मेनम् दे॒वता᳚भ्यः । \newline
23. दे॒वता᳚भ्य॒ आ दे॒वता᳚भ्यो दे॒वता᳚भ्य॒ आ । \newline
24. आ वृ॑श्चति वृश्च॒ त्यावृ॑श्चति । \newline
25. वृ॒श्च॒ति॒ ता॒जक् ता॒जग् वृ॑श्चति वृश्चति ता॒जक् । \newline
26. ता॒जगार्ति॒ मार्ति॑म् ता॒जक् ता॒जगार्ति᳚म् । \newline
27. आर्ति॒ मा ऽऽर्ति॒ मार्ति॒ मा । \newline
28. आर्च्छ॑ त्यृच्छ॒ त्यार्च्छ॑ति । \newline
29. ऋ॒च्छ॒ त्यङ्गि॑र॒सो ऽङ्गि॑रस ऋच्छ त्यृच्छ॒ त्यङ्गि॑रसः । \newline
30. अङ्गि॑रसः सुव॒र्गꣳ सु॑व॒र्ग मङ्गि॑र॒सो ऽङ्गि॑रसः सुव॒र्गम् । \newline
31. सु॒व॒र्गम् ॅलो॒कम् ॅलो॒कꣳ सु॑व॒र्गꣳ सु॑व॒र्गम् ॅलो॒कम् । \newline
32. सु॒व॒र्गमिति॑ सुवः - गम् । \newline
33. लो॒कं ॅयन्तो॒ यन्तो॑ लो॒कम् ॅलो॒कं ॅयन्तः॑ । \newline
34. यन्तो॒ या या यन्तो॒ यन्तो॒ या । \newline
35. या य॒ज्ञ्स्य॑ य॒ज्ञ्स्य॒ या या य॒ज्ञ्स्य॑ । \newline
36. य॒ज्ञ्स्य॒ निष्कृ॑ति॒र् निष्कृ॑तिर् य॒ज्ञ्स्य॑ य॒ज्ञ्स्य॒ निष्कृ॑तिः । \newline
37. निष्कृ॑ति॒ रासी॒ दासी॒न् निष्कृ॑ति॒र् निष्कृ॑ति॒ रासी᳚त् । \newline
38. निष्कृ॑ति॒रिति॒ निः - कृ॒तिः॒ । \newline
39. आसी॒त् ताम् ता मासी॒ दासी॒त् ताम् । \newline
40. ता मृषि॑भ्य॒ ऋषि॑भ्य॒ स्ताम् ता मृषि॑भ्यः । \newline
41. ऋषि॑भ्यः॒ प्रति॒ प्रत्यृषि॑भ्य॒ ऋषि॑भ्यः॒ प्रति॑ । \newline
42. ऋषि॑भ्य॒ इत्यृषि॑ - भ्यः॒ । \newline
43. प्रत्यौ॑हन् नौह॒न् प्रति॒ प्रत्यौ॑हन्न् । \newline
44. औ॒ह॒न् तत् तदौ॑हन् नौह॒न् तत् । \newline
45. तद्धिर॑ण्यꣳ॒॒ हिर॑ण्य॒म् तत् तद्धिर॑ण्यम् । \newline
46. हिर॑ण्य मभव दभव॒ द्धिर॑ण्यꣳ॒॒ हिर॑ण्य मभवत् । \newline
47. अ॒भ॒व॒द् यद् यद॑भव दभव॒द् यत् । \newline
48. यद्धि॑रण्यश॒ल्कैर्. हि॑रण्यश॒ल्कैर् यद् यद्धि॑रण्यश॒ल्कैः । \newline
49. हि॒र॒ण्य॒श॒ल्कैः प्रो॒क्षति॑ प्रो॒क्षति॑ हिरण्यश॒ल्कैर्. हि॑रण्यश॒ल्कैः प्रो॒क्षति॑ । \newline
50. हि॒र॒ण्य॒श॒ल्कैरिति॑ हिरण्य - श॒ल्कैः । \newline
51. प्रो॒क्षति॑ य॒ज्ञ्स्य॑ य॒ज्ञ्स्य॑ प्रो॒क्षति॑ प्रो॒क्षति॑ य॒ज्ञ्स्य॑ । \newline
52. प्रो॒क्षतीति॑ प्र - उ॒क्षति॑ । \newline
53. य॒ज्ञ्स्य॒ निष्कृ॑त्यै॒ निष्कृ॑त्यै य॒ज्ञ्स्य॑ य॒ज्ञ्स्य॒ निष्कृ॑त्यै । \newline
54. निष्कृ॑त्या॒ अथो॒ अथो॒ निष्कृ॑त्यै॒ निष्कृ॑त्या॒ अथो᳚ । \newline
55. निष्कृ॑त्या॒ इति॒ निः - कृ॒त्यै॒ । \newline
56. अथो॑ भेष॒जम् भे॑ष॒ज मथो॒ अथो॑ भेष॒जम् । \newline
57. अथो॒ इत्यथो᳚ । \newline
58. भे॒ष॒ज मे॒वैव भे॑ष॒जम् भे॑ष॒ज मे॒व । \newline
59. ए॒वास्मा॑ अस्मा ए॒वै वास्मै᳚ । \newline
60. अ॒स्मै॒ क॒रो॒ति॒ क॒रो॒ त्य॒स्मा॒ अ॒स्मै॒ क॒रो॒ति॒ । \newline
61. क॒रो॒ त्यथो॒ अथो॑ करोति करो॒ त्यथो᳚ । \newline

\textbf{Ghana Paata } \newline

1. अधि॑पत्नी॒र् नाम॒ नामा धि॑पत्नी॒ रधि॑पत्नी॒र् नामेष्ट॑का॒ इष्ट॑का॒ नामाधि॑ पत्नी॒ रधि॑पत्नी॒र् 
नामेष्ट॑काः । \newline
2. अधि॑पत्नी॒रित्यधि॑ - प॒त्नीः॒ । \newline
3. नामेष्ट॑का॒ इष्ट॑का॒ नाम॒ नामेष्ट॑का॒ यस्य॒ यस्येष्ट॑का॒ नाम॒ नामेष्ट॑का॒ यस्य॑ । \newline
4. इष्ट॑का॒ यस्य॒ यस्येष्ट॑का॒ इष्ट॑का॒ यस्यै॒ता ए॒ता यस्येष्ट॑का॒ इष्ट॑का॒ यस्यै॒ताः । \newline
5. यस्यै॒ता ए॒ता यस्य॒ यस्यै॒ता उ॑पधी॒यन्त॑ उपधी॒यन्त॑ ए॒ता यस्य॒ यस्यै॒ता उ॑पधी॒यन्ते᳚ । \newline
6. ए॒ता उ॑पधी॒यन्त॑ उपधी॒यन्त॑ ए॒ता ए॒ता उ॑पधी॒यन्ते ऽधि॑पति॒ रधि॑पति रुपधी॒यन्त॑ ए॒ता ए॒ता उ॑पधी॒यन्ते ऽधि॑पतिः । \newline
7. उ॒प॒धी॒यन्ते ऽधि॑पति॒ रधि॑पति रुपधी॒यन्त॑ उपधी॒यन्ते ऽधि॑पति रे॒वैवा धि॑पति रुपधी॒यन्त॑ उपधी॒यन्ते ऽधि॑पति रे॒व । \newline
8. उ॒प॒धी॒यन्त॒ इत्यु॑प - धी॒यन्ते᳚ । \newline
9. अधि॑पति रे॒वैवा धि॑पति॒ रधि॑पति रे॒व स॑मा॒नानाꣳ॑ समा॒नाना॑ मे॒वा धि॑पति॒ रधि॑पति रे॒व स॑मा॒नाना᳚म् । \newline
10. अधि॑पति॒रित्यधि॑ - प॒तिः॒ । \newline
11. ए॒व स॑मा॒नानाꣳ॑ समा॒नाना॑ मे॒वैव स॑मा॒नाना᳚म् भवति भवति समा॒नाना॑ मे॒वैव स॑मा॒नाना᳚म् भवति । \newline
12. स॒मा॒नाना᳚म् भवति भवति समा॒नानाꣳ॑ समा॒नाना᳚म् भवति॒ यं ॅयम् भ॑वति समा॒नानाꣳ॑ समा॒नाना᳚म् भवति॒ यम् । \newline
13. भ॒व॒ति॒ यं ॅयम् भ॑वति भवति॒ यम् द्वि॒ष्याद् द्वि॒ष्याद् यम् भ॑वति भवति॒ यम् द्वि॒ष्यात् । \newline
14. यम् द्वि॒ष्याद् द्वि॒ष्याद् यं ॅयम् द्वि॒ष्यात् तम् तम् द्वि॒ष्याद् यं ॅयम् द्वि॒ष्यात् तम् । \newline
15. द्वि॒ष्यात् तम् तम् द्वि॒ष्याद् द्वि॒ष्यात् त मु॑प॒दध॑ दुप॒दध॒त् तम् द्वि॒ष्याद् द्वि॒ष्यात् त मु॑प॒दध॑त् । \newline
16. त मु॑प॒दध॑ दुप॒दध॒त् तम् त मु॑प॒दध॑द् ध्यायेद् ध्याये दुप॒दध॒त् तम् त मु॑प॒दध॑द् ध्यायेत् । \newline
17. उ॒प॒दध॑द् ध्यायेद् ध्याये दुप॒दध॑ दुप॒दध॑द् ध्याये दे॒ताभ्य॑ ए॒ताभ्यो᳚ ध्याये दुप॒दध॑ दुप॒दध॑द् ध्याये दे॒ताभ्यः॑ । \newline
18. उ॒प॒दध॒दित्यु॑प - दध॑त् । \newline
19. ध्या॒ये॒ दे॒ताभ्य॑ ए॒ताभ्यो᳚ ध्यायेद् ध्याये दे॒ताभ्य॑ ए॒वैवै ताभ्यो᳚ ध्यायेद् ध्याये दे॒ताभ्य॑ ए॒व । \newline
20. ए॒ताभ्य॑ ए॒वैवै ताभ्य॑ ए॒ताभ्य॑ ए॒वैन॑ मेन मे॒वै ताभ्य॑ ए॒ताभ्य॑ ए॒वैन᳚म् । \newline
21. ए॒वैन॑ मेन मे॒वै वैन॑म् दे॒वता᳚भ्यो दे॒वता᳚भ्य एन मे॒वै वैन॑म् दे॒वता᳚भ्यः । \newline
22. ए॒न॒म् दे॒वता᳚भ्यो दे॒वता᳚भ्य एन मेनम् दे॒वता᳚भ्य॒ आ दे॒वता᳚भ्य एन मेनम् दे॒वता᳚भ्य॒ आ । \newline
23. दे॒वता᳚भ्य॒ आ दे॒वता᳚भ्यो दे॒वता᳚भ्य॒ आ वृ॑श्चति वृश्च॒त्या दे॒वता᳚भ्यो दे॒वता᳚भ्य॒ आ वृ॑श्चति । \newline
24. आ वृ॑श्चति वृश्च॒त्या वृ॑श्चति ता॒जक् ता॒जग् वृ॑श्च॒त्या वृ॑श्चति ता॒जक् । \newline
25. वृ॒श्च॒ति॒ ता॒जक् ता॒जग् वृ॑श्चति वृश्चति ता॒ज गार्ति॒ मार्ति॑म् ता॒जग् वृ॑श्चति वृश्चति ता॒ज गार्ति᳚म् । \newline
26. ता॒ज गार्ति॒ मार्ति॑म् ता॒जक् ता॒ज गार्ति॒ मा ऽऽर्ति॑म् ता॒जक् ता॒ज गार्ति॒ मा । \newline
27. आर्ति॒ मा ऽऽर्ति॒ मार्ति॒ मार्च्छ॑ त्यृच्छ॒ त्याऽऽर्ति॒ मार्ति॒ मार्च्छ॑ति । \newline
28. आर्च्छ॑ त्यृच्छ॒ त्यार्च्छ॒ त्यङ्गि॑र॒सो ऽङ्गि॑रस ऋच्छ॒ त्यार्च्छ॒ त्यङ्गि॑रसः । \newline
29. ऋ॒च्छ॒ त्यङ्गि॑र॒सो ऽङ्गि॑रस ऋच्छ त्यृच्छ॒ त्यङ्गि॑रसः सुव॒र्गꣳ सु॑व॒र्ग मङ्गि॑रस ऋच्छ त्यृच्छ॒ त्यङ्गि॑रसः सुव॒र्गम् । \newline
30. अङ्गि॑रसः सुव॒र्गꣳ सु॑व॒र्ग मङ्गि॑र॒सो ऽङ्गि॑रसः सुव॒र्गम् ॅलो॒कम् ॅलो॒कꣳ सु॑व॒र्ग मङ्गि॑र॒सो ऽङ्गि॑रसः सुव॒र्गम् ॅलो॒कम् । \newline
31. सु॒व॒र्गम् ॅलो॒कम् ॅलो॒कꣳ सु॑व॒र्गꣳ सु॑व॒र्गम् ॅलो॒कं ॅयन्तो॒ यन्तो॑ लो॒कꣳ सु॑व॒र्गꣳ सु॑व॒र्गम् ॅलो॒कं ॅयन्तः॑ । \newline
32. सु॒व॒र्गमिति॑ सुवः - गम् । \newline
33. लो॒कं ॅयन्तो॒ यन्तो॑ लो॒कम् ॅलो॒कं ॅयन्तो॒ या या यन्तो॑ लो॒कम् ॅलो॒कं ॅयन्तो॒ या । \newline
34. यन्तो॒ या या यन्तो॒ यन्तो॒ या य॒ज्ञ्स्य॑ य॒ज्ञ्स्य॒ या यन्तो॒ यन्तो॒ या य॒ज्ञ्स्य॑ । \newline
35. या य॒ज्ञ्स्य॑ य॒ज्ञ्स्य॒ या या य॒ज्ञ्स्य॒ निष्कृ॑ति॒र् निष्कृ॑तिर् य॒ज्ञ्स्य॒ या या य॒ज्ञ्स्य॒ निष्कृ॑तिः । \newline
36. य॒ज्ञ्स्य॒ निष्कृ॑ति॒र् निष्कृ॑तिर् य॒ज्ञ्स्य॑ य॒ज्ञ्स्य॒ निष्कृ॑ति॒ रासी॒ दासी॒न् निष्कृ॑तिर् य॒ज्ञ्स्य॑ य॒ज्ञ्स्य॒ निष्कृ॑ति॒ रासी᳚त् । \newline
37. निष्कृ॑ति॒ रासी॒ दासी॒न् निष्कृ॑ति॒र् निष्कृ॑ति॒ रासी॒त् ताम् ता मासी॒न् निष्कृ॑ति॒र् निष्कृ॑ति॒ रासी॒त् ताम् । \newline
38. निष्कृ॑ति॒रिति॒ निः - कृ॒तिः॒ । \newline
39. आसी॒त् ताम् ता मासी॒ दासी॒त् ता मृषि॑भ्य॒ ऋषि॑भ्य॒ स्ता मासी॒ दासी॒त् ता मृषि॑भ्यः । \newline
40. ता मृषि॑भ्य॒ ऋषि॑भ्य॒ स्ताम् ता मृषि॑भ्यः॒ प्रति॒ प्रत्यृषि॑भ्य॒ स्ताम् ता मृषि॑भ्यः॒ प्रति॑ । \newline
41. ऋषि॑भ्यः॒ प्रति॒ प्रत्यृषि॑भ्य॒ ऋषि॑भ्यः॒ प्रत्यौ॑हन् नौह॒न् प्रत्यृषि॑भ्य॒ ऋषि॑भ्यः॒ प्रत्यौ॑हन्न् । \newline
42. ऋषि॑भ्य॒ इत्यृषि॑ - भ्यः॒ । \newline
43. प्रत्यौ॑हन् नौह॒न् प्रति॒ प्रत्यौ॑ह॒न् तत् तदौ॑ह॒न् प्रति॒ प्रत्यौ॑ह॒न् तत् । \newline
44. औ॒ह॒न् तत् तदौ॑हन् नौह॒न् तद्धिर॑ण्यꣳ॒॒ हिर॑ण्य॒म् तदौ॑हन् नौह॒न् तद्धिर॑ण्यम् । \newline
45. तद्धिर॑ण्यꣳ॒॒ हिर॑ण्य॒म् तत् तद्धिर॑ण्य मभव दभव॒द् धिर॑ण्य॒म् तत् तद्धिर॑ण्य मभवत् । \newline
46. हिर॑ण्य मभव दभव॒द् धिर॑ण्यꣳ॒॒ हिर॑ण्य मभव॒द् यद् यद॑भव॒द् धिर॑ण्यꣳ॒॒ हिर॑ण्य मभव॒द् यत् । \newline
47. अ॒भ॒व॒द् यद् यद॑भव दभव॒द् यद्धि॑रण्यश॒ल्कैर्. हि॑रण्यश॒ल्कैर् यद॑भव दभव॒द् यद्धि॑रण्यश॒ल्कैः । \newline
48. यद्धि॑रण्यश॒ल्कैर्. हि॑रण्यश॒ल्कैर् यद् यद्धि॑रण्यश॒ल्कैः प्रो॒क्षति॑ प्रो॒क्षति॑ हिरण्यश॒ल्कैर् यद् यद्धि॑रण्यश॒ल्कैः प्रो॒क्षति॑ । \newline
49. हि॒र॒ण्य॒श॒ल्कैः प्रो॒क्षति॑ प्रो॒क्षति॑ हिरण्यश॒ल्कैर्. हि॑रण्यश॒ल्कैः प्रो॒क्षति॑ य॒ज्ञ्स्य॑ य॒ज्ञ्स्य॑ प्रो॒क्षति॑ हिरण्यश॒ल्कैर्. हि॑रण्यश॒ल्कैः प्रो॒क्षति॑ य॒ज्ञ्स्य॑ । \newline
50. हि॒र॒ण्य॒श॒ल्कैरिति॑ हिरण्य - श॒ल्कैः । \newline
51. प्रो॒क्षति॑ य॒ज्ञ्स्य॑ य॒ज्ञ्स्य॑ प्रो॒क्षति॑ प्रो॒क्षति॑ य॒ज्ञ्स्य॒ निष्कृ॑त्यै॒ निष्कृ॑त्यै य॒ज्ञ्स्य॑ प्रो॒क्षति॑ प्रो॒क्षति॑ य॒ज्ञ्स्य॒ निष्कृ॑त्यै । \newline
52. प्रो॒क्षतीति॑ प्र - उ॒क्षति॑ । \newline
53. य॒ज्ञ्स्य॒ निष्कृ॑त्यै॒ निष्कृ॑त्यै य॒ज्ञ्स्य॑ य॒ज्ञ्स्य॒ निष्कृ॑त्या॒ अथो॒ अथो॒ निष्कृ॑त्यै य॒ज्ञ्स्य॑ य॒ज्ञ्स्य॒ निष्कृ॑त्या॒ अथो᳚ । \newline
54. निष्कृ॑त्या॒ अथो॒ अथो॒ निष्कृ॑त्यै॒ निष्कृ॑त्या॒ अथो॑ भेष॒जम् भे॑ष॒ज मथो॒ निष्कृ॑त्यै॒ निष्कृ॑त्या॒ अथो॑ भेष॒जम् । \newline
55. निष्कृ॑त्या॒ इति॒ निः - कृ॒त्यै॒ । \newline
56. अथो॑ भेष॒जम् भे॑ष॒ज मथो॒ अथो॑ भेष॒ज मे॒वैव भे॑ष॒ज मथो॒ अथो॑ भेष॒ज मे॒व । \newline
57. अथो॒ इत्यथो᳚ । \newline
58. भे॒ष॒ज मे॒वैव भे॑ष॒जम् भे॑ष॒ज मे॒वास्मा॑ अस्मा ए॒व भे॑ष॒जम् भे॑ष॒ज मे॒वास्मै᳚ । \newline
59. ए॒वास्मा॑ अस्मा ए॒वै वास्मै॑ करोति करो त्यस्मा ए॒वै वास्मै॑ करोति । \newline
60. अ॒स्मै॒ क॒रो॒ति॒ क॒रो॒ त्य॒स्मा॒ अ॒स्मै॒ क॒रो॒ त्यथो॒ अथो॑ करो त्यस्मा अस्मै करो॒ त्यथो᳚ । \newline
61. क॒रो॒ त्यथो॒ अथो॑ करोति करो॒ त्यथो॑ रू॒पेण॑ रू॒पेणाथो॑ करोति करो॒ त्यथो॑ रू॒पेण॑ । \newline
\pagebreak
\markright{ TS 5.4.2.4  \hfill https://www.vedavms.in \hfill}

\section{ TS 5.4.2.4 }

\textbf{TS 5.4.2.4 } \newline
\textbf{Samhita Paata} \newline

-थो॑ रू॒पेणै॒वैनꣳ॒॒ सम॑र्द्धय॒त्यथो॒ हिर॑ण्यज्योतिषै॒व सु॑व॒र्गं ॅलो॒कमे॑ति साह॒स्रव॑ता॒ प्रोक्ष॑ति साह॒स्रः प्र॒जाप॑तिः प्र॒जाप॑ते॒राप्त्या॑ इ॒मा मे॑ अग्न॒ इष्ट॑का धे॒नवः॑ स॒न्त्वित्या॑ह धे॒नूरे॒वैनाः᳚ कुरुते॒ ता ए॑नं काम॒दुघा॑ अ॒मुत्रा॒मुष्मि॑न् ॅलो॒क उप॑ तिष्ठन्ते ॥ \newline

\textbf{Pada Paata} \newline

अथो॒ इति॑ । रू॒पेण॑ । ए॒व । ए॒न॒म् । समिति॑ । अ॒द्‌र्ध॒य॒ति॒ । अथो॒ इति॑ । हिर॑ण्यज्योति॒षेति॒ हिर॑ण्य - ज्यो॒ति॒षा॒ । ए॒व । सु॒व॒र्गमिति॑ सुवः - गम् । लो॒कम् । ए॒ति॒ । सा॒ह॒स्रव॒तेति॑ साह॒स्र - व॒ता॒ । प्रेति॑ । उ॒क्ष॒ति॒ । सा॒ह॒स्रः । प्र॒जाप॑ति॒रिति॑ प्र॒जा - प॒तिः॒ । प्र॒जाप॑ते॒रिति॑ प्र॒जा - प॒तेः॒ । आप्त्यै᳚ । इ॒माः । मे॒ । अ॒ग्ने॒ । इष्ट॑काः । धे॒नवः॑ । स॒न्तु॒ । इति॑ । आ॒ह॒ । धे॒नूः । ए॒व । ए॒नाः॒ । कु॒रु॒ते॒ । ताः । ए॒न॒म् । का॒म॒दुघा॒ इति॑ काम - दुघाः᳚ । अ॒मुत्र॑ । अ॒मुष्मिन्न्॑ । लो॒के । उपेति॑ । ति॒ष्ठ॒न्ते॒ ॥  \newline


\textbf{Krama Paata} \newline

अथो॑ रू॒पेण॑ । अथो॒ इत्यथो᳚ । रू॒पेणै॒व । ए॒वैन᳚म् । ए॒नꣳ॒॒ सम् । सम॑र्द्धयति । अ॒र्द्ध॒य॒त्यथो᳚ । अथो॒ हिर॑ण्यज्योतिषा । अथो॒ इत्यथो᳚ । हिर॑ण्यज्योतिषै॒व । हिर॑ण्यज्योति॒षेति॒ हिर॑ण्य - ज्यो॒ति॒षा॒ । ए॒व सु॑व॒र्गम् । सु॒व॒र्गम् ॅलो॒कम् । सु॒व॒र्गमिति॑ सुवः - गम् । लो॒कमे॑ति । ए॒ति॒ सा॒ह॒स्रव॑ता । सा॒ह॒स्रव॑ता॒ प्र । सा॒ह॒स्रव॒तेति॑ साह॒स्र - व॒ता॒ । प्रोक्ष॑ति । उ॒क्ष॒ति॒ सा॒ह॒स्रः । सा॒ह॒स्रः प्र॒जाप॑तिः । प्र॒जाप॑तिः प्र॒जाप॑तेः । प्र॒जाप॑ति॒रिति॑ प्र॒जा - प॒तिः॒ । प्र॒जाप॑ते॒राप्त्यै᳚ । प्र॒जाप॑ते॒रिति॑ प्र॒जा - प॒तेः॒ । आप्त्या॑ इ॒माः । इ॒मा मे᳚ । मे॒ अ॒ग्ने॒ । अ॒ग्न॒ इष्ट॑काः । इष्ट॑का धे॒नवः॑ । धे॒नवः॑ सन्तु । स॒न्त्विति॑ । इत्या॑ह । आ॒ह॒ धे॒नूः । धे॒नूरे॒व । ए॒वैनाः᳚ । ए॒नाः॒ कु॒रु॒ते॒ । कु॒रु॒ते॒ ताः । ता ए॑नम् । ए॒न॒म् का॒म॒दुघाः᳚ । का॒म॒दुघा॑ अ॒मुत्र॑ । का॒म॒दुघा॒ इति॑ काम - दुघाः᳚ । अ॒मुत्रा॒मुष्मिन्न्॑ । अ॒मुष्मि॑न् ॅलो॒के । लो॒क उप॑ । उप॑ तिष्ठन्ते । ति॒ष्ठ॒न्त॒ इति॑ तिष्ठन्ते । \newline

\textbf{Jatai Paata} \newline

1. अथो॑ रू॒पेण॑ रू॒पेणाथो॒ अथो॑ रू॒पेण॑ । \newline
2. अथो॒ इत्यथो᳚ । \newline
3. रू॒पेणै॒वैव रू॒पेण॑ रू॒पेणै॒व । \newline
4. ए॒वैन॑ मेन मे॒वै वैन᳚म् । \newline
5. ए॒नꣳ॒॒ सꣳ स मे॑न मेनꣳ॒॒ सम् । \newline
6. स म॑र्द्धय त्यर्द्धयति॒ सꣳ स म॑र्द्धयति । \newline
7. अ॒र्द्ध॒य॒ त्यथो॒ अथो॑ अर्द्धय त्यर्द्धय॒ त्यथो᳚ । \newline
8. अथो॒ हिर॑ण्यज्योतिषा॒ हिर॑ण्यज्योति॒षा ऽथो॒ अथो॒ हिर॑ण्यज्योतिषा । \newline
9. अथो॒ इत्यथो᳚ । \newline
10. हिर॑ण्यज्योति षै॒वैव हिर॑ण्यज्योतिषा॒ हिर॑ण्यज्योतिषै॒व । \newline
11. हिर॑ण्यज्योति॒षेति॒ हिर॑ण्य - ज्यो॒ति॒षा॒ । \newline
12. ए॒व सु॑व॒र्गꣳ सु॑व॒र्ग मे॒वैव सु॑व॒र्गम् । \newline
13. सु॒व॒र्गम् ॅलो॒कम् ॅलो॒कꣳ सु॑व॒र्गꣳ सु॑व॒र्गम् ॅलो॒कम् । \newline
14. सु॒व॒र्गमिति॑ सुवः - गम् । \newline
15. लो॒क मे᳚त्येति लो॒कम् ॅलो॒क मे॑ति । \newline
16. ए॒ति॒ सा॒ह॒स्रव॑ता साह॒स्रव॑ तैत्येति साह॒स्रव॑ता । \newline
17. सा॒ह॒स्रव॑ता॒ प्र प्र सा॑ह॒स्रव॑ता साह॒स्रव॑ता॒ प्र । \newline
18. सा॒ह॒स्रव॒तेति॑ साह॒स्र - व॒ता॒ । \newline
19. प्रोक्ष॑ त्युक्षति॒ प्र प्रोक्ष॑ति । \newline
20. उ॒क्ष॒ति॒ सा॒ह॒स्रः सा॑ह॒स्र उ॑क्ष त्युक्षति साह॒स्रः । \newline
21. सा॒ह॒स्रः प्र॒जाप॑तिः प्र॒जाप॑तिः साह॒स्रः सा॑ह॒स्रः प्र॒जाप॑तिः । \newline
22. प्र॒जाप॑तिः प्र॒जाप॑तेः प्र॒जाप॑तेः प्र॒जाप॑तिः प्र॒जाप॑तिः प्र॒जाप॑तेः । \newline
23. प्र॒जाप॑ति॒रिति॑ प्र॒जा - प॒तिः॒ । \newline
24. प्र॒जाप॑ते॒ राप्त्या॒ आप्त्यै᳚ प्र॒जाप॑तेः प्र॒जाप॑ते॒ राप्त्यै᳚ । \newline
25. प्र॒जाप॑ते॒रिति॑ प्र॒जा - प॒तेः॒ । \newline
26. आप्त्या॑ इ॒मा इ॒मा आप्त्या॒ आप्त्या॑ इ॒माः । \newline
27. इ॒मा मे॑ म इ॒मा इ॒मा मे᳚ । \newline
28. मे॒ अ॒ग्ने॒ ऽग्ने॒ मे॒ मे॒ अ॒ग्ने॒ । \newline
29. अ॒ग्न॒ इष्ट॑का॒ इष्ट॑का अग्ने ऽग्न॒ इष्ट॑काः । \newline
30. इष्ट॑का धे॒नवो॑ धे॒नव॒ इष्ट॑का॒ इष्ट॑का धे॒नवः॑ । \newline
31. धे॒नवः॑ सन्तु सन्तु धे॒नवो॑ धे॒नवः॑ सन्तु । \newline
32. स॒न्त्वितीति॑ सन्तु स॒न्त्विति॑ । \newline
33. इत्या॑हा॒ हेतीत्या॑ह । \newline
34. आ॒ह॒ धे॒नूर् धे॒नू रा॑हाह धे॒नूः । \newline
35. धे॒नू रे॒वैव धे॒नूर् धे॒नू रे॒व । \newline
36. ए॒वैना॑ एना ए॒वै वैनाः᳚ । \newline
37. ए॒नाः॒ कु॒रु॒ते॒ कु॒रु॒त॒ ए॒ना॒ ए॒नाः॒ कु॒रु॒ते॒ । \newline
38. कु॒रु॒ते॒ ता स्ताः कु॑रुते कुरुते॒ ताः । \newline
39. ता ए॑न मेन॒म् ता स्ता ए॑नम् । \newline
40. ए॒न॒म् का॒म॒दुघाः᳚ काम॒दुघा॑ एन मेनम् काम॒दुघाः᳚ । \newline
41. का॒म॒दुघा॑ अ॒मुत्रा॒ मुत्र॑ काम॒दुघाः᳚ काम॒दुघा॑ अ॒मुत्र॑ । \newline
42. का॒म॒दुघा॒ इति॑ काम - दुघाः᳚ । \newline
43. अ॒मुत्रा॒ मुष्मि॑न् न॒मुष्मि॑न् न॒मुत्रा॒ मुत्रा॒ मुष्मिन्न्॑ । \newline
44. अ॒मुष्मि॑न् ॅलो॒के लो॒के॑ ऽमुष्मि॑न् न॒मुष्मि॑न् ॅलो॒के । \newline
45. लो॒क उपोप॑ लो॒के लो॒क उप॑ । \newline
46. उप॑ तिष्ठन्ते तिष्ठन्त॒ उपोप॑ तिष्ठन्ते । \newline
47. ति॒ष्ठ॒न्त॒ इति॑ तिष्ठन्ते । \newline

\textbf{Ghana Paata } \newline

1. अथो॑ रू॒पेण॑ रू॒पेणाथो॒ अथो॑ रू॒पे णै॒वैव रू॒पेणाथो॒ अथो॑ रू॒पेणै॒व । \newline
2. अथो॒ इत्यथो᳚ । \newline
3. रू॒पे णै॒वैव रू॒पेण॑ रू॒पे णै॒वैन॑ मेन मे॒व रू॒पेण॑ रू॒पे णै॒वैन᳚म् । \newline
4. ए॒वैन॑ मेन मे॒वै वैनꣳ॒॒ सꣳ स मे॑न मे॒वैवैनꣳ॒॒ सम् । \newline
5. ए॒नꣳ॒॒ सꣳ स मे॑न मेनꣳ॒॒ स म॑र्द्धय त्यर्द्धयति॒ स मे॑न मेनꣳ॒॒ स म॑र्द्धयति । \newline
6. स म॑र्द्धय त्यर्द्धयति॒ सꣳ स म॑र्द्धय॒ त्यथो॒ अथो॑ अर्द्धयति॒ सꣳ स म॑र्द्धय॒ त्यथो᳚ । \newline
7. अ॒र्द्ध॒य॒ त्यथो॒ अथो॑ अर्द्धय त्यर्द्धय॒ त्यथो॒ हिर॑ण्यज्योतिषा॒ हिर॑ण्यज्योति॒षा ऽथो॑ अर्द्धय त्यर्द्धय॒ त्यथो॒ हिर॑ण्यज्योतिषा । \newline
8. अथो॒ हिर॑ण्यज्योतिषा॒ हिर॑ण्यज्योति॒षा ऽथो॒ अथो॒ हिर॑ण्यज्योति षै॒वैव हिर॑ण्यज्योति॒षा ऽथो॒ अथो॒ हिर॑ण्यज्योतिषै॒व । \newline
9. अथो॒ इत्यथो᳚ । \newline
10. हिर॑ण्यज्योति षै॒वैव हिर॑ण्यज्योतिषा॒ हिर॑ण्यज्योतिषै॒व सु॑व॒र्गꣳ सु॑व॒र्ग मे॒व हिर॑ण्यज्योतिषा॒ हिर॑ण्यज्योतिषै॒व सु॑व॒र्गम् । \newline
11. हिर॑ण्यज्योति॒षेति॒ हिर॑ण्य - ज्यो॒ति॒षा॒ । \newline
12. ए॒व सु॑व॒र्गꣳ सु॑व॒र्ग मे॒वैव सु॑व॒र्गम् ॅलो॒कम् ॅलो॒कꣳ सु॑व॒र्ग मे॒वैव सु॑व॒र्गम् ॅलो॒कम् । \newline
13. सु॒व॒र्गम् ॅलो॒कम् ॅलो॒कꣳ सु॑व॒र्गꣳ सु॑व॒र्गम् ॅलो॒क मे᳚त्येति लो॒कꣳ सु॑व॒र्गꣳ सु॑व॒र्गम् ॅलो॒क मे॑ति । \newline
14. सु॒व॒र्गमिति॑ सुवः - गम् । \newline
15. लो॒क मे᳚त्येति लो॒कम् ॅलो॒क मे॑ति साह॒स्रव॑ता साह॒स्रव॑तैति लो॒कम् ॅलो॒क मे॑ति साह॒स्रव॑ता । \newline
16. ए॒ति॒ सा॒ह॒स्रव॑ता साह॒स्रव॑तै त्येति साह॒स्रव॑ता॒ प्र प्र सा॑ह॒स्रव॑तै त्येति साह॒स्रव॑ता॒ प्र । \newline
17. सा॒ह॒स्रव॑ता॒ प्र प्र सा॑ह॒स्रव॑ता साह॒स्रव॑ता॒ प्रोक्ष॑ त्युक्षति॒ प्र सा॑ह॒स्रव॑ता साह॒स्रव॑ता॒ प्रोक्ष॑ति । \newline
18. सा॒ह॒स्रव॒तेति॑ साह॒स्र - व॒ता॒ । \newline
19. प्रोक्ष॑ त्युक्षति॒ प्र प्रोक्ष॑ति साह॒स्रः सा॑ह॒स्र उ॑क्षति॒ प्र प्रोक्ष॑ति साह॒स्रः । \newline
20. उ॒क्ष॒ति॒ सा॒ह॒स्रः सा॑ह॒स्र उ॑क्ष त्युक्षति साह॒स्रः प्र॒जाप॑तिः प्र॒जाप॑तिः साह॒स्र उ॑क्ष त्युक्षति साह॒स्रः प्र॒जाप॑तिः । \newline
21. सा॒ह॒स्रः प्र॒जाप॑तिः प्र॒जाप॑तिः साह॒स्रः सा॑ह॒स्रः प्र॒जाप॑तिः प्र॒जाप॑तेः प्र॒जाप॑तेः प्र॒जाप॑तिः साह॒स्रः सा॑ह॒स्रः प्र॒जाप॑तिः प्र॒जाप॑तेः । \newline
22. प्र॒जाप॑तिः प्र॒जाप॑तेः प्र॒जाप॑तेः प्र॒जाप॑तिः प्र॒जाप॑तिः प्र॒जाप॑ते॒ राप्त्या॒ आप्त्यै᳚ प्र॒जाप॑तेः प्र॒जाप॑तिः प्र॒जाप॑तिः प्र॒जाप॑ते॒ राप्त्यै᳚ । \newline
23. प्र॒जाप॑ति॒रिति॑ प्र॒जा - प॒तिः॒ । \newline
24. प्र॒जाप॑ते॒ राप्त्या॒ आप्त्यै᳚ प्र॒जाप॑तेः प्र॒जाप॑ते॒ राप्त्या॑ इ॒मा इ॒मा आप्त्यै᳚ प्र॒जाप॑तेः प्र॒जाप॑ते॒ राप्त्या॑ इ॒माः । \newline
25. प्र॒जाप॑ते॒रिति॑ प्र॒जा - प॒तेः॒ । \newline
26. आप्त्या॑ इ॒मा इ॒मा आप्त्या॒ आप्त्या॑ इ॒मा मे॑ म इ॒मा आप्त्या॒ आप्त्या॑ इ॒मा मे᳚ । \newline
27. इ॒मा मे॑ म इ॒मा इ॒मा मे॑ अग्ने ऽग्ने म इ॒मा इ॒मा मे॑ अग्ने । \newline
28. मे॒ अ॒ग्ने॒ ऽग्ने॒ मे॒ मे॒ अ॒ग्न॒ इष्ट॑का॒ इष्ट॑का अग्ने मे मे अग्न॒ इष्ट॑काः । \newline
29. अ॒ग्न॒ इष्ट॑का॒ इष्ट॑का अग्ने ऽग्न॒ इष्ट॑का धे॒नवो॑ धे॒नव॒ इष्ट॑का अग्ने ऽग्न॒ इष्ट॑का धे॒नवः॑ । \newline
30. इष्ट॑का धे॒नवो॑ धे॒नव॒ इष्ट॑का॒ इष्ट॑का धे॒नवः॑ सन्तु सन्तु धे॒नव॒ इष्ट॑का॒ इष्ट॑का धे॒नवः॑ सन्तु । \newline
31. धे॒नवः॑ सन्तु सन्तु धे॒नवो॑ धे॒नवः॑ स॒न्त्वितीति॑ सन्तु धे॒नवो॑ धे॒नवः॑ स॒न्त्विति॑ । \newline
32. स॒न्त् वितीति॑ सन्तु स॒न्त्वि त्या॑हा॒हेति॑ सन्तु स॒न्त्वि त्या॑ह । \newline
33. इत्या॑ हा॒हेती त्या॑ह धे॒नूर् धे॒नू रा॒हेती त्या॑ह धे॒नूः । \newline
34. आ॒ह॒ धे॒नूर् धे॒नू रा॑हाह धे॒नू रे॒वैव धे॒नू रा॑हाह धे॒नू रे॒व । \newline
35. धे॒नू रे॒वैव धे॒नूर् धे॒नू रे॒वैना॑ एना ए॒व धे॒नूर् धे॒नू रे॒वैनाः᳚ । \newline
36. ए॒वैना॑ एना ए॒वै वैनाः᳚ कुरुते कुरुत एना ए॒वै वैनाः᳚ कुरुते । \newline
37. ए॒नाः॒ कु॒रु॒ते॒ कु॒रु॒त॒ ए॒ना॒ ए॒नाः॒ कु॒रु॒ते॒ ता स्ताः कु॑रुत एना एनाः कुरुते॒ ताः । \newline
38. कु॒रु॒ते॒ ता स्ताः कु॑रुते कुरुते॒ ता ए॑न मेन॒म् ताः कु॑रुते कुरुते॒ ता ए॑नम् । \newline
39. ता ए॑न मेन॒म् ता स्ता ए॑नम् काम॒दुघाः᳚ काम॒दुघा॑ एन॒म् ता स्ता ए॑नम् काम॒दुघाः᳚ । \newline
40. ए॒न॒म् का॒म॒दुघाः᳚ काम॒दुघा॑ एन मेनम् काम॒दुघा॑ अ॒मुत्रा॒ मुत्र॑ काम॒दुघा॑ एन मेनम् काम॒दुघा॑ अ॒मुत्र॑ । \newline
41. का॒म॒दुघा॑ अ॒मुत्रा॒ मुत्र॑ काम॒दुघाः᳚ काम॒दुघा॑ अ॒मुत्रा॒ मुष्मि॑न् न॒मुष्मि॑न् न॒मुत्र॑ काम॒दुघाः᳚ काम॒दुघा॑ अ॒मुत्रा॒ मुष्मिन्न्॑ । \newline
42. का॒म॒दुघा॒ इति॑ काम - दुघाः᳚ । \newline
43. अ॒मुत्रा॒ मुष्मि॑न् न॒मुष्मि॑न् न॒मुत्रा॒ मुत्रा॒ मुष्मि॑न् ॅलो॒के लो॒के॑ ऽमुष्मि॑न् न॒मुत्रा॒ मुत्रा॒ मुष्मि॑न् ॅलो॒के । \newline
44. अ॒मुष्मि॑न् ॅलो॒के लो॒के॑ ऽमुष्मि॑न् न॒मुष्मि॑न् ॅलो॒क उपोप॑ लो॒के॑ ऽमुष्मि॑न् न॒मुष्मि॑न् ॅलो॒क उप॑ । \newline
45. लो॒क उपोप॑ लो॒के लो॒क उप॑ तिष्ठन्ते तिष्ठन्त॒ उप॑ लो॒के लो॒क उप॑ तिष्ठन्ते । \newline
46. उप॑ तिष्ठन्ते तिष्ठन्त॒ उपोप॑ तिष्ठन्ते । \newline
47. ति॒ष्ठ॒न्त॒ इति॑ तिष्ठन्ते । \newline
\pagebreak
\markright{ TS 5.4.3.1  \hfill https://www.vedavms.in \hfill}

\section{ TS 5.4.3.1 }

\textbf{TS 5.4.3.1 } \newline
\textbf{Samhita Paata} \newline

रु॒द्रो वा ए॒ष यद॒ग्निः स ए॒तर्.हि॑ जा॒तो यर्.हि॒ सर्व॑श्चि॒तः स यथा॑ व॒थ्सो जा॒तः स्तनं॑ प्रे॒फ्सत्ये॒वं ॅवा ए॒ष ए॒तर्.हि॑ भाग॒धेयं॒ प्रेफ्स॑ति॒ तस्मै॒ यदाहु॑तिं॒ न जु॑हु॒याद॑द्ध्व॒र्युं च॒ यज॑मानं च ध्यायेच्छतरु॒द्रीयं॑ जुहोति भाग॒धेये॑नै॒वैनꣳ॑ शमयति॒ नाऽऽ*र्ति॒मार्च्छ॑त्यद्ध्व॒र्युर्न यज॑मानो॒ यद् ग्रा॒म्याणां᳚ पशू॒नां - [  ] \newline

\textbf{Pada Paata} \newline

रु॒द्रः । वै । ए॒षः । यत् । अ॒ग्निः । सः । ए॒तर्.हि॑ । जा॒तः । यर्.हि॑ । सर्वः॑ । चि॒तः । सः । यथा᳚ । व॒थ्सः । जा॒तः । स्तन᳚म् । प्रे॒फ्सतीति॑ प्र - ई॒फ्सति॑ । ए॒वम् । वै । ए॒षः । ए॒तर्.हि॑ । भा॒ग॒धेय॒मिति॑ भाग-धेय᳚म् । प्रेति॑ । ई॒फ्स॒ति॒ । तस्मै᳚ । यत् । आहु॑ति॒मित्या-हु॒ति॒म् । न । जु॒हु॒यात् । अ॒द्ध्व॒र्युम् । च॒ । यज॑मानम् । च॒ । ध्या॒ये॒त् । श॒त॒रु॒द्रीय॒मिति॑ शत - रु॒द्रीय᳚म् । जु॒हो॒ति॒ । भा॒ग॒धेये॒नेति॑ भाग - धेये॑न । ए॒व । ए॒न॒म् । श॒म॒य॒ति॒ । न । आर्ति᳚म् । एति॑ । ऋ॒च्छ॒ति॒ । अ॒द्ध्व॒र्युः । न । यज॑मानः । यत् । ग्रा॒म्याणा᳚म् । प॒शू॒नाम् ।  \newline


\textbf{Krama Paata} \newline

रु॒द्रो वै । वा ए॒षः । ए॒ष यत् । यद॒ग्निः । अ॒ग्निः सः । स ए॒तर्.हि॑ । ए॒तर्.हि॑ जा॒तः । जा॒तो यर्.हि॑ । यर्.हि॒ सर्वः॑ । सर्व॑श्चि॒तः । चि॒तः सः । स यथा᳚ । यथा॑ व॒थ्सः । व॒थ्सो जा॒तः । जा॒तः स्तन᳚म् । स्तन॑म् प्रे॒फ्सति॑ । प्रे॒फ्सत्ये॒वम् । प्रे॒फ्सतीति॑ प्र - ई॒फ्सति॑ । ए॒वम् ॅवै । वा ए॒षः । ए॒ष ए॒तर्.हि॑ । ए॒तर्.हि॑ भाग॒धेय᳚म् । भा॒ग॒धेय॒म् प्र । भा॒ग॒धेय॒मिति॑ भाग - धेय᳚म् । प्रेफ्स॑ति । ई॒फ्स॒ति॒ तस्मै᳚ । तस्मै॒ यत् । यदाहु॑तिम् । आहु॑ति॒म् न । आहु॑ति॒मित्या - हु॒ति॒म् । न जु॑हु॒यात् । जु॒हु॒याद॑द्ध्व॒र्युम् । अ॒द्ध्व॒र्युम् च॑ । च॒ यज॑मानम् । यज॑मानम् च । च॒ ध्या॒ये॒त्॒ । ध्या॒ये॒च्छ॒त॒रु॒द्रीय᳚म् । श॒त॒रु॒द्रीय॑म् जुहोति । श॒त॒रु॒द्रीय॒मिति॑ शत - रु॒द्रीय᳚म् । जु॒हो॒ति॒ भा॒ग॒धेये॑न । भा॒ग॒धेये॑नै॒व । भा॒ग॒धेये॒नेति॑ भाग - धेये॑न । ए॒वैन᳚म् । ए॒नꣳ॒॒ श॒म॒य॒ति॒ । श॒म॒य॒ति॒ न । नार्ति᳚म् । आर्ति॒मा । आर्च्छ॑ति । ऋ॒च्छ॒त्य॒द्ध्व॒र्युः । अ॒द्ध्व॒र्युर् न । न यज॑मानः । यज॑मानो॒ यत् । यद् ग्रा॒म्याणा᳚म् । ग्रा॒म्याणा᳚म् पशू॒नाम् । प॒शू॒नाम् पय॑सा \newline

\textbf{Jatai Paata} \newline

1. रु॒द्रो वै वै रु॒द्रो रु॒द्रो वै । \newline
2. वा ए॒ष ए॒ष वै वा ए॒षः । \newline
3. ए॒ष यद् यदे॒ष ए॒ष यत् । \newline
4. यद॒ग्नि र॒ग्निर् यद् यद॒ग्निः । \newline
5. अ॒ग्निः स सो᳚ ऽग्नि र॒ग्निः सः । \newline
6. स ए॒तर् ह्ये॒तर्.हि॒ स स ए॒तर्.हि॑ । \newline
7. ए॒तर्.हि॑ जा॒तो जा॒त ए॒तर् ह्ये॒तर्.हि॑ जा॒तः । \newline
8. जा॒तो यर्.हि॒ यर्.हि॑ जा॒तो जा॒तो यर्.हि॑ । \newline
9. यर्.हि॒ सर्वः॒ सर्वो॒ यर्.हि॒ यर्.हि॒ सर्वः॑ । \newline
10. सर्व॑ श्चि॒त श्चि॒तः सर्वः॒ सर्व॑ श्चि॒तः । \newline
11. चि॒तः स स चि॒त श्चि॒तः सः । \newline
12. स यथा॒ यथा॒ स स यथा᳚ । \newline
13. यथा॑ व॒थ्सो व॒थ्सो यथा॒ यथा॑ व॒थ्सः । \newline
14. व॒थ्सो जा॒तो जा॒तो व॒थ्सो व॒थ्सो जा॒तः । \newline
15. जा॒तः स्तनꣳ॒॒ स्तन॑म् जा॒तो जा॒तः स्तन᳚म् । \newline
16. स्तन॑म् प्रे॒फ्सति॑ प्रे॒फ्सति॒ स्तनꣳ॒॒ स्तन॑म् प्रे॒फ्सति॑ । \newline
17. प्रे॒फ्स त्ये॒व मे॒वम् प्रे॒फ्सति॑ प्रे॒फ्स त्ये॒वम् । \newline
18. प्रे॒फ्सतीति॑ प्र - ई॒फ्सति॑ । \newline
19. ए॒वं ॅवै वा ए॒व मे॒वं ॅवै । \newline
20. वा ए॒ष ए॒ष वै वा ए॒षः । \newline
21. ए॒ष ए॒तर् ह्ये॒तर् ह्ये॒ष ए॒ष ए॒तर्.हि॑ । \newline
22. ए॒तर्.हि॑ भाग॒धेय॑म् भाग॒धेय॑ मे॒तर् ह्ये॒तर्.हि॑ भाग॒धेय᳚म् । \newline
23. भा॒ग॒धेय॒म् प्र प्र भा॑ग॒धेय॑म् भाग॒धेय॒म् प्र । \newline
24. भा॒ग॒धेय॒मिति॑ भाग - धेय᳚म् । \newline
25. प्रेफ्स॑तीफ्सति॒ प्र प्रेफ्स॑ति । \newline
26. ई॒फ्स॒ति॒ तस्मै॒ तस्मा॑ ईफ्सतीफ्सति॒ तस्मै᳚ । \newline
27. तस्मै॒ यद् यत् तस्मै॒ तस्मै॒ यत् । \newline
28. यदाहु॑ति॒ माहु॑तिं॒ ॅयद् यदाहु॑तिम् । \newline
29. आहु॑ति॒म् न नाहु॑ति॒ माहु॑ति॒म् न । \newline
30. आहु॑ति॒मित्या - हु॒ति॒म् । \newline
31. न जु॑हु॒याज् जु॑हु॒यान् न न जु॑हु॒यात् । \newline
32. जु॒हु॒या द॑द्ध्व॒र्यु म॑द्ध्व॒र्युम् जु॑हु॒याज् जु॑हु॒या द॑द्ध्व॒र्युम् । \newline
33. अ॒द्ध्व॒र्युम् च॑ चाद्ध्व॒र्यु म॑द्ध्व॒र्युम् च॑ । \newline
34. च॒ यज॑मानं॒ ॅयज॑मानम् च च॒ यज॑मानम् । \newline
35. यज॑मानम् च च॒ यज॑मानं॒ ॅयज॑मानम् च । \newline
36. च॒ ध्या॒ये॒द् ध्या॒ये॒च् च॒ च॒ ध्या॒ये॒त् । \newline
37. ध्या॒ये॒ च्छ॒त॒रु॒द्रीयꣳ॑ शतरु॒द्रीय॑म् ध्यायेद् ध्याये च्छतरु॒द्रीय᳚म् । \newline
38. श॒त॒रु॒द्रीय॑म् जुहोति जुहोति शतरु॒द्रीयꣳ॑ शतरु॒द्रीय॑म् जुहोति । \newline
39. श॒त॒रु॒द्रीय॒मिति॑ शत - रु॒द्रीय᳚म् । \newline
40. जु॒हो॒ति॒ भा॒ग॒धेये॑न भाग॒धेये॑न जुहोति जुहोति भाग॒धेये॑न । \newline
41. भा॒ग॒धेये॑ नै॒वैव भा॑ग॒धेये॑न भाग॒धेये॑नै॒व । \newline
42. भा॒ग॒धेये॒नेति॑ भाग - धेये॑न । \newline
43. ए॒वैन॑ मेन मे॒वै वैन᳚म् । \newline
44. ए॒नꣳ॒॒ श॒म॒य॒ति॒ श॒म॒य॒ त्ये॒न॒ मे॒नꣳ॒॒ श॒म॒य॒ति॒ । \newline
45. श॒म॒य॒ति॒ न न श॑मयति शमयति॒ न । \newline
46. नार्ति॒ मार्ति॒म् न नार्ति᳚म् । \newline
47. आर्ति॒ मा ऽऽर्ति॒ मार्ति॒ मा । \newline
48. आर्च्छ॑ त्यृच्छ॒ त्यार्च्छ॑ति । \newline
49. ऋ॒च्छ॒ त्य॒द्ध्व॒र्यु र॑द्ध्व॒र्युर्. ऋ॑च्छ त्यृच्छ त्यद्ध्व॒र्युः । \newline
50. अ॒द्ध्व॒र्युर् न नाद्ध्व॒र्यु र॑द्ध्व॒र्युर् न । \newline
51. न यज॑मानो॒ यज॑मानो॒ न न यज॑मानः । \newline
52. यज॑मानो॒ यद् यद् यज॑मानो॒ यज॑मानो॒ यत् । \newline
53. यद् ग्रा॒म्याणा᳚म् ग्रा॒म्याणां॒ ॅयद् यद् ग्रा॒म्याणा᳚म् । \newline
54. ग्रा॒म्याणा᳚म् पशू॒नाम् प॑शू॒नाम् ग्रा॒म्याणा᳚म् ग्रा॒म्याणा᳚म् पशू॒नाम् । \newline
55. प॒शू॒नाम् पय॑सा॒ पय॑सा पशू॒नाम् प॑शू॒नाम् पय॑सा । \newline

\textbf{Ghana Paata } \newline

1. रु॒द्रो वै वै रु॒द्रो रु॒द्रो वा ए॒ष ए॒ष वै रु॒द्रो रु॒द्रो वा ए॒षः । \newline
2. वा ए॒ष ए॒ष वै वा ए॒ष यद् यदे॒ष वै वा ए॒ष यत् । \newline
3. ए॒ष यद् यदे॒ष ए॒ष यद॒ग्नि र॒ग्निर् यदे॒ष ए॒ष यद॒ग्निः । \newline
4. यद॒ग्नि र॒ग्निर् यद् यद॒ग्निः स सो᳚ ऽग्निर् यद् यद॒ग्निः सः । \newline
5. अ॒ग्निः स सो᳚ ऽग्नि र॒ग्निः स ए॒तर् ह्ये॒तर्.हि॒ सो᳚ ऽग्नि र॒ग्निः स ए॒तर्.हि॑ । \newline
6. स ए॒तर् ह्ये॒तर्.हि॒ स स ए॒तर्.हि॑ जा॒तो जा॒त ए॒तर्.हि॒ स स ए॒तर्.हि॑ जा॒तः । \newline
7. ए॒तर्.हि॑ जा॒तो जा॒त ए॒तर् ह्ये॒तर्.हि॑ जा॒तो यर्.हि॒ यर्.हि॑ जा॒त ए॒तर् ह्ये॒तर्.हि॑ जा॒तो यर्.हि॑ । \newline
8. जा॒तो यर्.हि॒ यर्.हि॑ जा॒तो जा॒तो यर्.हि॒ सर्वः॒ सर्वो॒ यर्.हि॑ जा॒तो जा॒तो यर्.हि॒ सर्वः॑ । \newline
9. यर्.हि॒ सर्वः॒ सर्वो॒ यर्.हि॒ यर्.हि॒ सर्व॑ श्चि॒त श्चि॒तः सर्वो॒ यर्.हि॒ यर्.हि॒ सर्व॑ श्चि॒तः । \newline
10. सर्व॑ श्चि॒त श्चि॒तः सर्वः॒ सर्व॑ श्चि॒तः स स चि॒तः सर्वः॒ सर्व॑ श्चि॒तः सः । \newline
11. चि॒तः स स चि॒त श्चि॒तः स यथा॒ यथा॒ स चि॒त श्चि॒तः स यथा᳚ । \newline
12. स यथा॒ यथा॒ स स यथा॑ व॒थ्सो व॒थ्सो यथा॒ स स यथा॑ व॒थ्सः । \newline
13. यथा॑ व॒थ्सो व॒थ्सो यथा॒ यथा॑ व॒थ्सो जा॒तो जा॒तो व॒थ्सो यथा॒ यथा॑ व॒थ्सो जा॒तः । \newline
14. व॒थ्सो जा॒तो जा॒तो व॒थ्सो व॒थ्सो जा॒तः स्तनꣳ॒॒ स्तन॑म् जा॒तो व॒थ्सो व॒थ्सो जा॒तः स्तन᳚म् । \newline
15. जा॒तः स्तनꣳ॒॒ स्तन॑म् जा॒तो जा॒तः स्तन॑म् प्रे॒फ्सति॑ प्रे॒फ्सति॒ स्तन॑म् जा॒तो जा॒तः स्तन॑म् प्रे॒फ्सति॑ । \newline
16. स्तन॑म् प्रे॒फ्सति॑ प्रे॒फ्सति॒ स्तनꣳ॒॒ स्तन॑म् प्रे॒फ्स त्ये॒व मे॒वम् प्रे॒फ्सति॒ स्तनꣳ॒॒ स्तन॑म् प्रे॒फ्स त्ये॒वम् । \newline
17. प्रे॒फ्स त्ये॒व मे॒वम् प्रे॒फ्सति॑ प्रे॒फ्स त्ये॒वं ॅवै वा ए॒वम् प्रे॒फ्सति॑ प्रे॒फ्स त्ये॒वं ॅवै । \newline
18. प्रे॒फ्सतीति॑ प्र - ई॒फ्सति॑ । \newline
19. ए॒वं ॅवै वा ए॒व मे॒वं ॅवा ए॒ष ए॒ष वा ए॒व मे॒वं ॅवा ए॒षः । \newline
20. वा ए॒ष ए॒ष वै वा ए॒ष ए॒तर् ह्ये॒तर् ह्ये॒ष वै वा ए॒ष ए॒तर्.हि॑ । \newline
21. ए॒ष ए॒तर् ह्ये॒तर् ह्ये॒ष ए॒ष ए॒तर्.हि॑ भाग॒धेय॑म् भाग॒धेय॑ मे॒तर् ह्ये॒ष ए॒ष ए॒तर्.हि॑ भाग॒धेय᳚म् । \newline
22. ए॒तर्.हि॑ भाग॒धेय॑म् भाग॒धेय॑ मे॒तर् ह्ये॒तर्.हि॑ भाग॒धेय॒म् प्र प्र भा॑ग॒धेय॑ मे॒तर् ह्ये॒तर्.हि॑ भाग॒धेय॒म् प्र । \newline
23. भा॒ग॒धेय॒म् प्र प्र भा॑ग॒धेय॑म् भाग॒धेय॒म् प्रेफ्स॑ती फ्सति॒ प्र भा॑ग॒धेय॑म् भाग॒धेय॒म् प्रेफ्स॑ति । \newline
24. भा॒ग॒धेय॒मिति॑ भाग - धेय᳚म् । \newline
25. प्रेफ्स॑ती फ्सति॒ प्र प्रेफ्स॑ति॒ तस्मै॒ तस्मा॑ ईफ्सति॒ प्र प्रेफ्स॑ति॒ तस्मै᳚ । \newline
26. ई॒फ्स॒ति॒ तस्मै॒ तस्मा॑ ईफ्सतीफ्सति॒ तस्मै॒ यद् यत् तस्मा॑ ईफ्सतीफ्सति॒ तस्मै॒ यत् । \newline
27. तस्मै॒ यद् यत् तस्मै॒ तस्मै॒ यदाहु॑ति॒ माहु॑तिं॒ ॅयत् तस्मै॒ तस्मै॒ यदाहु॑तिम् । \newline
28. यदाहु॑ति॒ माहु॑तिं॒ ॅयद् यदाहु॑ति॒म् न नाहु॑तिं॒ ॅयद् यदाहु॑ति॒म् न । \newline
29. आहु॑ति॒म् न नाहु॑ति॒ माहु॑ति॒म् न जु॑हु॒याज् जु॑हु॒यान् नाहु॑ति॒ माहु॑ति॒म् न जु॑हु॒यात् । \newline
30. आहु॑ति॒मित्या - हु॒ति॒म् । \newline
31. न जु॑हु॒याज् जु॑हु॒यान् न न जु॑हु॒या द॑द्ध्व॒र्यु म॑द्ध्व॒र्युम् जु॑हु॒यान् न न जु॑हु॒या द॑द्ध्व॒र्युम् । \newline
32. जु॒हु॒या द॑द्ध्व॒र्यु म॑द्ध्व॒र्युम् जु॑हु॒याज् जु॑हु॒या द॑द्ध्व॒र्युम् च॑ चाद्ध्व॒र्युम् जु॑हु॒याज् जु॑हु॒या द॑द्ध्व॒र्युम् च॑ । \newline
33. अ॒द्ध्व॒र्युम् च॑ चाद्ध्व॒र्यु म॑द्ध्व॒र्युम् च॒ यज॑मानं॒ ॅयज॑मानम् चाद्ध्व॒र्यु म॑द्ध्व॒र्युम् च॒ यज॑मानम् । \newline
34. च॒ यज॑मानं॒ ॅयज॑मानम् च च॒ यज॑मानम् च च॒ यज॑मानम् च च॒ यज॑मानम् च । \newline
35. यज॑मानम् च च॒ यज॑मानं॒ ॅयज॑मानम् च ध्यायेद् ध्यायेच् च॒ यज॑मानं॒ ॅयज॑मानम् च ध्यायेत् । \newline
36. च॒ ध्या॒ये॒द् ध्या॒ये॒च् च॒ च॒ ध्या॒ये॒ च्छ॒त॒रु॒द्रीयꣳ॑ शतरु॒द्रीय॑म् ध्यायेच् च च ध्याये च्छतरु॒द्रीय᳚म् । \newline
37. ध्या॒ये॒ च्छ॒त॒रु॒द्रीयꣳ॑ शतरु॒द्रीय॑म् ध्यायेद् ध्याये च्छतरु॒द्रीय॑म् जुहोति जुहोति शतरु॒द्रीय॑म् ध्यायेद् ध्याये च्छतरु॒द्रीय॑म् जुहोति । \newline
38. श॒त॒रु॒द्रीय॑म् जुहोति जुहोति शतरु॒द्रीयꣳ॑ शतरु॒द्रीय॑म् जुहोति भाग॒धेये॑न भाग॒धेये॑न जुहोति शतरु॒द्रीयꣳ॑ शतरु॒द्रीय॑म् जुहोति भाग॒धेये॑न । \newline
39. श॒त॒रु॒द्रीय॒मिति॑ शत - रु॒द्रीय᳚म् । \newline
40. जु॒हो॒ति॒ भा॒ग॒धेये॑न भाग॒धेये॑न जुहोति जुहोति भाग॒धे ये॑नै॒वैव भा॑ग॒धेये॑न जुहोति जुहोति भाग॒धेये॑नै॒व । \newline
41. भा॒ग॒धे ये॑नै॒वैव भा॑ग॒धेये॑न भाग॒धेये॑ नै॒वैन॑ मेन मे॒व भा॑ग॒धेये॑न भाग॒धेये॑
नै॒वैन᳚म् । \newline
42. भा॒ग॒धेये॒नेति॑ भाग - धेये॑न । \newline
43. ए॒वैन॑ मेन मे॒वै वैनꣳ॑ शमयति शमय त्येन मे॒वै वैनꣳ॑ शमयति । \newline
44. ए॒नꣳ॒॒ श॒म॒य॒ति॒ श॒म॒य॒ त्ये॒न॒ मे॒नꣳ॒॒ श॒म॒य॒ति॒ न न श॑मय त्येन मेनꣳ शमयति॒ न । \newline
45. श॒म॒य॒ति॒ न न श॑मयति शमयति॒ नार्ति॒ मार्ति॒म् न श॑मयति शमयति॒ नार्ति᳚म् । \newline
46. नार्ति॒ मार्ति॒म् न नार्ति॒ मा ऽऽर्ति॒म् न नार्ति॒ मा । \newline
47. आर्ति॒ मा ऽऽर्ति॒ मार्ति॒ मार्च्छ॑ त्यृच्छ॒ त्याऽऽर्ति॒ मार्ति॒ मार्च्छ॑ति । \newline
48. आर्च्छ॑ त्यृच्छ॒ त्यार्च्छ॑ त्यद्ध्व॒र्यु र॑द्ध्व॒र्युर्. ऋ॑च्छ॒ त्यार्च्छ॑ त्यद्ध्व॒र्युः । \newline
49. ऋ॒च्छ॒ त्य॒द्ध्व॒र्यु र॑द्ध्व॒र्युर्. ऋ॑च्छ त्यृच्छ त्यद्ध्व॒र्युर् न नाद्ध्व॒र्युर्. ऋ॑च्छ त्यृच्छ त्यद्ध्व॒र्युर् न । \newline
50. अ॒द्ध्व॒र्युर् न नाद्ध्व॒र्यु र॑द्ध्व॒र्युर् न यज॑मानो॒ यज॑मानो॒ नाद्ध्व॒र्यु र॑द्ध्व॒र्युर् न यज॑मानः । \newline
51. न यज॑मानो॒ यज॑मानो॒ न न यज॑मानो॒ यद् यद् यज॑मानो॒ न न यज॑मानो॒ यत् । \newline
52. यज॑मानो॒ यद् यद् यज॑मानो॒ यज॑मानो॒ यद् ग्रा॒म्याणा᳚म् ग्रा॒म्याणां॒ ॅयद् यज॑मानो॒ यज॑मानो॒ यद् ग्रा॒म्याणा᳚म् । \newline
53. यद् ग्रा॒म्याणा᳚म् ग्रा॒म्याणां॒ ॅयद् यद् ग्रा॒म्याणा᳚म् पशू॒नाम् प॑शू॒नाम् ग्रा॒म्याणां॒ ॅयद् यद् ग्रा॒म्याणा᳚म् पशू॒नाम् । \newline
54. ग्रा॒म्याणा᳚म् पशू॒नाम् प॑शू॒नाम् ग्रा॒म्याणा᳚म् ग्रा॒म्याणा᳚म् पशू॒नाम् पय॑सा॒ पय॑सा पशू॒नाम् ग्रा॒म्याणा᳚म् ग्रा॒म्याणा᳚म् पशू॒नाम् पय॑सा । \newline
55. प॒शू॒नाम् पय॑सा॒ पय॑सा पशू॒नाम् प॑शू॒नाम् पय॑सा जुहु॒याज् जु॑हु॒यात् पय॑सा पशू॒नाम् प॑शू॒नाम् पय॑सा जुहु॒यात् । \newline
\pagebreak
\markright{ TS 5.4.3.2  \hfill https://www.vedavms.in \hfill}

\section{ TS 5.4.3.2 }

\textbf{TS 5.4.3.2 } \newline
\textbf{Samhita Paata} \newline

पय॑सा जुहु॒याद् ग्रा॒म्यान् प॒शूञ्छु॒चा ऽर्पये॒द्-यदा॑र॒ण्याना॑-मार॒ण्यान् ज॑र्तिलयवा॒ग्वा॑ वा जुहु॒याद् ग॑वीधुकयवा॒ग्वा॑ वा॒ न ग्रा॒म्यान् प॒शून्. हि॒नस्ति॒ नाऽऽ*र॒ण्यानथो॒ खल्वा॑हु॒रना॑हुति॒र्वै ज॒र्तिला᳚श्च ग॒वीधु॑का॒श्चेत्य॑ जक्षी॒रेण॑ जुहोत्याग्ने॒यी वा ए॒षा यद॒जाऽऽहु॑त्यै॒व जु॑होति॒ न ग्रा॒म्यान् प॒शून्. हि॒नस्ति॒ नाऽऽ*र॒ण्यानङ्गि॑रसः सुव॒र्गं ॅलो॒कं ॅयन्तो॒ - [  ] \newline

\textbf{Pada Paata} \newline

पय॑सा । जु॒हु॒यात् । ग्रा॒म्यान् । प॒शून् । शु॒चा । अ॒र्प॒ये॒त् । यत् । आ॒र॒ण्याना᳚म् । आ॒र॒ण्यान् । ज॒र्ति॒ल॒य॒वा॒ग्वेति॑ जर्तिल - य॒वा॒ग्वा᳚ । वा॒ । जु॒हु॒यात् । ग॒वी॒धु॒क॒य॒वा॒ग्वेति॑ गवीधुक - य॒वा॒ग्वा᳚ । वा॒ । न । ग्रा॒म्यान् । प॒शून् । हि॒नस्ति॑ । न । आ॒र॒ण्यान् । अथो॒ इति॑ । खलु॑ । आ॒हुः॒ । अना॑हुति॒रित्यना᳚-हु॒तिः॒ । वै । ज॒र्तिलाः᳚ । च॒ । ग॒वीधु॑काः । च॒ । इति॑ । अ॒ज॒क्षी॒रेणेत्य॑ज - क्षी॒रेण॑ । जु॒हो॒ति॒ । आ॒ग्ने॒यी । वै । ए॒षा । यत् । अ॒जा । आहु॒त्येत्या - हु॒त्या॒ । ए॒व । जु॒हो॒ति॒ । न । ग्रा॒म्यान् । प॒शून् । हि॒नस्ति॑ । न । आ॒र॒ण्यान् । अङ्गि॑रसः । सु॒व॒र्गमिति॑ सुवः - गम् । लो॒कम् । यन्तः॑ ।  \newline


\textbf{Krama Paata} \newline

पय॑सा जुहु॒यात् । ज॒हु॒याद् ग्रा॒म्यान् । ग्रा॒म्यान् प॒शून् । प॒शूञ्छु॒चा । शु॒चाऽर्प॑येत् । अ॒र्प॒ये॒द् यत् । यदा॑र॒ण्याना᳚म् । आ॒र॒ण्याना॑मार॒ण्यान् । आ॒र॒ण्यान् ज॑र्तिलयवा॒ग्वा᳚ । ज॒र्ति॒ल॒य॒वा॒ग्वा॑ वा । ज॒र्ति॒ल॒य॒वा॒ग्वेति॑ जर्तिल - य॒वा॒ग्वा᳚ । वा॒ जु॒हु॒यात् । जु॒हु॒याद् ग॑वीधुकयवा॒ग्वा᳚ । ग॒वी॒धु॒क॒य॒वा॒ग्वा॑ वा । ग॒वी॒धु॒क॒य॒वा॒ग्वेति॑ गवीधुक - य॒वा॒ग्वा᳚ । वा॒ न । न ग्रा॒म्यान् । ग्रा॒म्यान् प॒शून् । प॒शून्. हि॒नस्ति॑ । हि॒नस्ति॒ न । नार॒ण्यान् । आ॒र॒ण्यानथो᳚ । अथो॒ खलु॑ । अथो॒ इत्यथो᳚ । खल्वा॑हुः । आ॒हु॒रना॑हुतिः । अना॑हुति॒र् वै । अना॑हुति॒रित्यना᳚ - हु॒तिः॒ । वै ज॒र्तिलाः᳚ । ज॒र्तिला᳚श्च । च॒ ग॒वीधु॑काः । ग॒वीधु॑काश्च । चेति॑ । इत्य॑जक्षी॒रेण॑ । अ॒ज॒क्षी॒रेण॑ जुहोति । अ॒ज॒क्षी॒रेणेत्य॑ज - क्षी॒रेण॑ । जु॒हो॒त्या॒ग्ने॒यी । आ॒ग्ने॒यी वै । वा ए॒षा । ए॒षा यत् । यद॒जा । अ॒जाऽऽहु॑त्या । आहु॑त्यै॒व । आहु॒त्येत्या - हु॒त्या॒ । ए॒व जु॑होति । जु॒हो॒ति॒ न । न ग्रा॒म्यान् । ग्रा॒म्यान् प॒शून् । प॒शून्. हि॒नस्ति॑ । हि॒नस्ति॒ न । नार॒ण्यान् । आ॒र॒ण्यानङ्गि॑रसः । अङ्गि॑रसः सुव॒र्गम् । सु॒व॒र्गम् ॅलो॒कम् । सु॒व॒र्गमिति॑ सुवः - गम् । लो॒कम् ॅयन्तः॑ । यन्तो॒ऽजाया᳚म् \newline

\textbf{Jatai Paata} \newline

1. पय॑सा जुहु॒याज् जु॑हु॒यात् पय॑सा॒ पय॑सा जुहु॒यात् । \newline
2. जु॒हु॒याद् ग्रा॒म्यान् ग्रा॒म्यान् जु॑हु॒याज् जु॑हु॒याद् ग्रा॒म्यान् । \newline
3. ग्रा॒म्यान् प॒शून् प॒शून् ग्रा॒म्यान् ग्रा॒म्यान् प॒शून् । \newline
4. प॒शूञ् छु॒चा शु॒चा प॒शून् प॒शूञ् छु॒चा । \newline
5. शु॒चा ऽर्प॑ये दर्पये च्छु॒चा शु॒चा ऽर्प॑येत् । \newline
6. अ॒र्प॒ये॒द् यद् यद॑र्पये दर्पये॒द् यत् । \newline
7. यदा॑र॒ण्याना॑ मार॒ण्यानां॒ ॅयद् यदा॑र॒ण्याना᳚म् । \newline
8. आ॒र॒ण्याना॑ मार॒ण्या ना॑र॒ण्या ना॑र॒ण्याना॑ मार॒ण्याना॑ मार॒ण्यान् । \newline
9. आ॒र॒ण्यान् ज॑र्तिलयवा॒ग्वा॑ जर्तिलयवा॒ग्वा॑ ऽऽर॒ण्या ना॑र॒ण्यान् ज॑र्तिलयवा॒ग्वा᳚ । \newline
10. ज॒र्ति॒ल॒य॒वा॒ग्वा॑ वा वा जर्तिलयवा॒ग्वा॑ जर्तिलयवा॒ग्वा॑ वा । \newline
11. ज॒र्ति॒ल॒य॒वा॒ग्वेति॑ जर्तिल - य॒वा॒ग्वा᳚ । \newline
12. वा॒ जु॒हु॒याज् जु॑हु॒याद् वा॑ वा जुहु॒यात् । \newline
13. जु॒हु॒याद् ग॑वीधुकयवा॒ग्वा॑ गवीधुकयवा॒ग्वा॑ जुहु॒याज् जु॑हु॒याद् ग॑वीधुकयवा॒ग्वा᳚ । \newline
14. ग॒वी॒धु॒क॒य॒वा॒ग्वा॑ वा वा गवीधुकयवा॒ग्वा॑ गवीधुकयवा॒ग्वा॑ वा । \newline
15. ग॒वी॒धु॒क॒य॒वा॒ग्वेति॑ गवीधुक - य॒वा॒ग्वा᳚ । \newline
16. वा॒ न न वा॑ वा॒ न । \newline
17. न ग्रा॒म्यान् ग्रा॒म्यान् न न ग्रा॒म्यान् । \newline
18. ग्रा॒म्यान् प॒शून् प॒शून् ग्रा॒म्यान् ग्रा॒म्यान् प॒शून् । \newline
19. प॒शून्. हि॒नस्ति॑ हि॒नस्ति॑ प॒शून् प॒शून्. हि॒नस्ति॑ । \newline
20. हि॒नस्ति॒ न न हि॒नस्ति॑ हि॒नस्ति॒ न । \newline
21. नार॒ण्या ना॑र॒ण्यान् न नार॒ण्यान् । \newline
22. आ॒र॒ण्या नथो॒ अथो॑ आर॒ण्या ना॑र॒ण्या नथो᳚ । \newline
23. अथो॒ खलु॒ खल्वथो॒ अथो॒ खलु॑ । \newline
24. अथो॒ इत्यथो᳚ । \newline
25. खल्वा॑हु राहुः॒ खलु॒ खल्वा॑हुः । \newline
26. आ॒हु॒ रना॑हुति॒ रना॑हुति राहु राहु॒ रना॑हुतिः । \newline
27. अना॑हुति॒र् वै वा अना॑हुति॒ रना॑हुति॒र् वै । \newline
28. अना॑हुति॒रित्यना᳚ - हु॒तिः॒ । \newline
29. वै ज॒र्तिला॑ ज॒र्तिला॒ वै वै ज॒र्तिलाः᳚ । \newline
30. ज॒र्तिला᳚श्च च ज॒र्तिला॑ ज॒र्तिला᳚श्च । \newline
31. च॒ ग॒वीधु॑का ग॒वीधु॑काश्च च ग॒वीधु॑काः । \newline
32. ग॒वीधु॑काश्च च ग॒वीधु॑का ग॒वीधु॑काश्च । \newline
33. चेतीति॑ च॒ चेति॑ । \newline
34. इत्य॑जक्षी॒रेणा॑ जक्षी॒रे णेतीत्य॑जक्षी॒रेण॑ । \newline
35. अ॒ज॒क्षी॒रेण॑ जुहोति जुहो त्यजक्षी॒रेणा॑ जक्षी॒रेण॑ जुहोति । \newline
36. अ॒ज॒क्षी॒रेणेत्य॑ज - क्षी॒रेण॑ । \newline
37. जु॒हो॒ त्या॒ग्ने॒य्या᳚ ग्ने॒यी जु॑होति जुहो त्याग्ने॒यी । \newline
38. आ॒ग्ने॒यी वै वा आ᳚ग्ने॒ य्या᳚ग्ने॒यी वै । \newline
39. वा ए॒षैषा वै वा ए॒षा । \newline
40. ए॒षा यद् यदे॒ षैषा यत् । \newline
41. यद॒जा ऽजा यद् यद॒जा । \newline
42. अ॒जा ऽऽहु॒त्या ऽऽहु॑त्या॒ ऽजा ऽजा ऽऽहु॑त्या । \newline
43. आहु॑ त्यै॒वैवा हु॒त्या ऽऽहु॑त्यै॒व । \newline
44. आहु॒त्येत्या - हु॒त्या॒ । \newline
45. ए॒व जु॑होति जुहो त्ये॒वैव जु॑होति । \newline
46. जु॒हो॒ति॒ न न जु॑होति जुहोति॒ न । \newline
47. न ग्रा॒म्यान् ग्रा॒म्यान् न न ग्रा॒म्यान् । \newline
48. ग्रा॒म्यान् प॒शून् प॒शून् ग्रा॒म्यान् ग्रा॒म्यान् प॒शून् । \newline
49. प॒शून्. हि॒नस्ति॑ हि॒नस्ति॑ प॒शून् प॒शून्. हि॒नस्ति॑ । \newline
50. हि॒नस्ति॒ न न हि॒नस्ति॑ हि॒नस्ति॒ न । \newline
51. नार॒ण्या ना॑र॒ण्यान् न नार॒ण्यान् । \newline
52. आ॒र॒ण्या नङ्गि॑रसो ऽङ्गिरस आर॒ण्या ना॑र॒ण्या नङ्गि॑रसः । \newline
53. अङ्गि॑रसः सुव॒र्गꣳ सु॑व॒र्ग मङ्गि॑रसो ऽङ्गिरसः सुव॒र्गम् । \newline
54. सु॒व॒र्गम् ॅलो॒कम् ॅलो॒कꣳ सु॑व॒र्गꣳ सु॑व॒र्गम् ॅलो॒कम् । \newline
55. सु॒व॒र्गमिति॑ सुवः - गम् । \newline
56. लो॒कं ॅयन्तो॒ यन्तो॑ लो॒कम् ॅलो॒कं ॅयन्तः॑ । \newline
57. यन्तो॒ ऽजाया॑ म॒जायां॒ ॅयन्तो॒ यन्तो॒ ऽजाया᳚म् । \newline

\textbf{Ghana Paata } \newline

1. पय॑सा जुहु॒याज् जु॑हु॒यात् पय॑सा॒ पय॑सा जुहु॒याद् ग्रा॒म्यान् ग्रा॒म्यान् जु॑हु॒यात् पय॑सा॒ पय॑सा जुहु॒याद् ग्रा॒म्यान् । \newline
2. जु॒हु॒याद् ग्रा॒म्यान् ग्रा॒म्यान् जु॑हु॒याज् जु॑हु॒याद् ग्रा॒म्यान् प॒शून् प॒शून् ग्रा॒म्यान् जु॑हु॒याज् जु॑हु॒याद् ग्रा॒म्यान् प॒शून् । \newline
3. ग्रा॒म्यान् प॒शून् प॒शून् ग्रा॒म्यान् ग्रा॒म्यान् प॒शूञ् छु॒चा शु॒चा प॒शून् ग्रा॒म्यान् ग्रा॒म्यान् प॒शूञ् छु॒चा । \newline
4. प॒शूञ् छु॒चा शु॒चा प॒शून् प॒शूञ् छु॒चा ऽर्प॑ये दर्पये च्छु॒चा प॒शून् प॒शूञ् छु॒चा ऽर्प॑येत् । \newline
5. शु॒चा ऽर्प॑ये दर्पये च्छु॒चा शु॒चा ऽर्प॑ये॒द् यद् यद॑र्पये च्छु॒चा शु॒चा ऽर्प॑ये॒द् यत् । \newline
6. अ॒र्प॒ये॒द् यद् यद॑र्पये दर्पये॒द् यदा॑र॒ण्याना॑ मार॒ण्यानां॒ ॅयद॑र्पये दर्पये॒द् यदा॑र॒ण्याना᳚म् । \newline
7. यदा॑र॒ण्याना॑ मार॒ण्यानां॒ ॅयद् यदा॑र॒ण्याना॑ मार॒ण्या ना॑र॒ण्या ना॑र॒ण्यानां॒ ॅयद् यदा॑र॒ण्याना॑ मार॒ण्यान् । \newline
8. आ॒र॒ण्याना॑ मार॒ण्या ना॑र॒ण्या ना॑र॒ण्याना॑ मार॒ण्याना॑ मार॒ण्यान् ज॑र्तिलयवा॒ग्वा॑ जर्तिलयवा॒ग्वा॑ ऽऽर॒ण्या ना॑र॒ण्याना॑ मार॒ण्याना॑ मार॒ण्यान् ज॑र्तिलयवा॒ग्वा᳚ । \newline
9. आ॒र॒ण्यान् ज॑र्तिलयवा॒ग्वा॑ जर्तिलयवा॒ग्वा॑ ऽऽर॒ण्या ना॑र॒ण्यान् ज॑र्तिलयवा॒ग्वा॑ वा वा जर्तिलयवा॒ग्वा॑ ऽऽर॒ण्या ना॑र॒ण्यान् ज॑र्तिलयवा॒ग्वा॑ वा । \newline
10. ज॒र्ति॒ल॒य॒वा॒ग्वा॑ वा वा जर्तिलयवा॒ग्वा॑ जर्तिलयवा॒ग्वा॑ वा जुहु॒याज् जु॑हु॒याद् वा॑ जर्तिलयवा॒ग्वा॑ जर्तिलयवा॒ग्वा॑ वा जुहु॒यात् । \newline
11. ज॒र्ति॒ल॒य॒वा॒ग्वेति॑ जर्तिल - य॒वा॒ग्वा᳚ । \newline
12. वा॒ जु॒हु॒याज् जु॑हु॒याद् वा॑ वा जुहु॒याद् ग॑वीधुकयवा॒ग्वा॑ गवीधुकयवा॒ग्वा॑ जुहु॒याद् वा॑ वा जुहु॒याद् ग॑वीधुकयवा॒ग्वा᳚ । \newline
13. जु॒हु॒याद् ग॑वीधुकयवा॒ग्वा॑ गवीधुकयवा॒ग्वा॑ जुहु॒याज् जु॑हु॒याद् ग॑वीधुकयवा॒ग्वा॑ वा वा गवीधुकयवा॒ग्वा॑ जुहु॒याज् जु॑हु॒याद् ग॑वीधुकयवा॒ग्वा॑ वा । \newline
14. ग॒वी॒धु॒क॒य॒वा॒ग्वा॑ वा वा गवीधुकयवा॒ग्वा॑ गवीधुकयवा॒ग्वा॑ वा॒ न न वा॑ गवीधुकयवा॒ग्वा॑ गवीधुकयवा॒ग्वा॑ वा॒ न । \newline
15. ग॒वी॒धु॒क॒य॒वा॒ग्वेति॑ गवीधुक - य॒वा॒ग्वा᳚ । \newline
16. वा॒ न न वा॑ वा॒ न ग्रा॒म्यान् ग्रा॒म्यान् न वा॑ वा॒ न ग्रा॒म्यान् । \newline
17. न ग्रा॒म्यान् ग्रा॒म्यान् न न ग्रा॒म्यान् प॒शून् प॒शून् ग्रा॒म्यान् न न ग्रा॒म्यान् प॒शून् । \newline
18. ग्रा॒म्यान् प॒शून् प॒शून् ग्रा॒म्यान् ग्रा॒म्यान् प॒शून्. हि॒नस्ति॑ हि॒नस्ति॑ प॒शून् ग्रा॒म्यान् ग्रा॒म्यान् प॒शून्. हि॒नस्ति॑ । \newline
19. प॒शून्. हि॒नस्ति॑ हि॒नस्ति॑ प॒शून् प॒शून्. हि॒नस्ति॒ न न हि॒नस्ति॑ प॒शून् प॒शून्. हि॒नस्ति॒ न । \newline
20. हि॒नस्ति॒ न न हि॒नस्ति॑ हि॒नस्ति॒ नार॒ण्या ना॑र॒ण्यान् न हि॒नस्ति॑ हि॒नस्ति॒ नार॒ण्यान् । \newline
21. नार॒ण्या ना॑र॒ण्यान् न नार॒ण्या नथो॒ अथो॑ आर॒ण्यान् न नार॒ण्या नथो᳚ । \newline
22. आ॒र॒ण्या नथो॒ अथो॑ आर॒ण्या ना॑र॒ण्या नथो॒ खलु॒ खल्वथो॑ आर॒ण्या ना॑र॒ण्या नथो॒ खलु॑ । \newline
23. अथो॒ खलु॒ खल्वथो॒ अथो॒ खल्वा॑हु राहुः॒ खल्वथो॒ अथो॒ खल्वा॑हुः । \newline
24. अथो॒ इत्यथो᳚ । \newline
25. खल्वा॑हु राहुः॒ खलु॒ खल्वा॑हु॒ रना॑हुति॒ रना॑हुति राहुः॒ खलु॒ खल्वा॑हु॒ रना॑हुतिः । \newline
26. आ॒हु॒ रना॑हुति॒ रना॑हुति राहु राहु॒ रना॑हुति॒र् वै वा अना॑हुति राहु राहु॒ रना॑हुति॒र् वै । \newline
27. अना॑हुति॒र् वै वा अना॑हुति॒ रना॑हुति॒र् वै ज॒र्तिला॑ ज॒र्तिला॒ वा अना॑हुति॒ रना॑हुति॒र् वै ज॒र्तिलाः᳚ । \newline
28. अना॑हुति॒रित्यना᳚ - हु॒तिः॒ । \newline
29. वै ज॒र्तिला॑ ज॒र्तिला॒ वै वै ज॒र्तिला᳚श्च च ज॒र्तिला॒ वै वै ज॒र्तिला᳚श्च । \newline
30. ज॒र्तिला᳚श्च च ज॒र्तिला॑ ज॒र्तिला᳚श्च ग॒वीधु॑का ग॒वीधु॑काश्च ज॒र्तिला॑ ज॒र्तिला᳚श्च ग॒वीधु॑काः । \newline
31. च॒ ग॒वीधु॑का ग॒वीधु॑काश्च च ग॒वीधु॑काश्च च ग॒वीधु॑काश्च च ग॒वीधु॑काश्च । \newline
32. ग॒वीधु॑काश्च च ग॒वीधु॑का ग॒वीधु॑का॒श्चे तीति॑ च ग॒वीधु॑का ग॒वीधु॑का॒श्चेति॑ । \newline
33. चेतीति॑ च॒ चेत्य॑जक्षी॒रेणा॑ जक्षी॒रेणेति॑ च॒ चेत्य॑जक्षी॒रेण॑ । \newline
34. इत्य॑जक्षी॒रेणा॑ जक्षी॒रेणे तीत्य॑जक्षी॒रेण॑ जुहोति जुहो त्यजक्षी॒रेणेती त्य॑जक्षी॒रेण॑ जुहोति । \newline
35. अ॒ज॒क्षी॒रेण॑ जुहोति जुहोत्य जक्षी॒रेणा॑ जक्षी॒रेण॑ जुहो त्याग्ने॒य्या᳚ ग्ने॒यी जु॑होत्य जक्षी॒रेणा॑ जक्षी॒रेण॑ जुहो त्याग्ने॒यी । \newline
36. अ॒ज॒क्षी॒रेणेत्य॑ज - क्षी॒रेण॑ । \newline
37. जु॒हो॒ त्या॒ग्ने॒य्या᳚ ग्ने॒यी जु॑होति जुहो त्याग्ने॒यी वै वा आ᳚ग्ने॒यी जु॑होति जुहो त्याग्ने॒यी वै । \newline
38. आ॒ग्ने॒यी वै वा आ᳚ग्ने॒य्या᳚ ग्ने॒यी वा ए॒षैषा वा आ᳚ग्ने॒य्या᳚ ग्ने॒यी वा ए॒षा । \newline
39. वा ए॒षैषा वै वा ए॒षा यद् यदे॒षा वै वा ए॒षा यत् । \newline
40. ए॒षा यद् यदे॒षैषा यद॒जा ऽजा यदे॒षैषा यद॒जा । \newline
41. यद॒जा ऽजा यद् यद॒जा ऽऽहु॒त्या ऽऽहु॑त्या॒ ऽजा यद् यद॒जा ऽऽहु॑त्या । \newline
42. अ॒जा ऽऽहु॒त्या ऽऽहु॑त्या॒ ऽजा ऽजा ऽऽहु॑ त्यै॒वै वाहु॑त्या॒ ऽजा ऽजा ऽऽहु॑त्यै॒व । \newline
43. आहु॑ त्यै॒वैवाहु॒त्या ऽऽहु॑त्यै॒व जु॑होति जुहो त्ये॒वाहु॒त्या ऽऽहु॑त्यै॒व जु॑होति । \newline
44. आहु॒त्येत्या - हु॒त्या॒ । \newline
45. ए॒व जु॑होति जुहो त्ये॒वैव जु॑होति॒ न न जु॑हो त्ये॒वैव जु॑होति॒ न । \newline
46. जु॒हो॒ति॒ न न जु॑होति जुहोति॒ न ग्रा॒म्यान् ग्रा॒म्यान् न जु॑होति जुहोति॒ न ग्रा॒म्यान् । \newline
47. न ग्रा॒म्यान् ग्रा॒म्यान् न न ग्रा॒म्यान् प॒शून् प॒शून् ग्रा॒म्यान् न न ग्रा॒म्यान् प॒शून् । \newline
48. ग्रा॒म्यान् प॒शून् प॒शून् ग्रा॒म्यान् ग्रा॒म्यान् प॒शून्. हि॒नस्ति॑ हि॒नस्ति॑ प॒शून् ग्रा॒म्यान् ग्रा॒म्यान् प॒शून्. हि॒नस्ति॑ । \newline
49. प॒शून्. हि॒नस्ति॑ हि॒नस्ति॑ प॒शून् प॒शून्. हि॒नस्ति॒ न न हि॒नस्ति॑ प॒शून् प॒शून्. हि॒नस्ति॒ न । \newline
50. हि॒नस्ति॒ न न हि॒नस्ति॑ हि॒नस्ति॒ नार॒ण्या ना॑र॒ण्यान् न हि॒नस्ति॑ हि॒नस्ति॒ नार॒ण्यान् । \newline
51. नार॒ण्या ना॑र॒ण्यान् न नार॒ण्या नङ्गि॑रसो ऽग्गिरस आर॒ण्यान् न नार॒ण्या नङ्गि॑रसः । \newline
52. आ॒र॒ण्या नङ्गि॑रसो ऽङ्गिरस आर॒ण्या ना॑र॒ण्या नङ्गि॑रसः सुव॒र्गꣳ सु॑व॒र्ग मङ्गि॑रस आर॒ण्या ना॑र॒ण्या नङ्गि॑रसः सुव॒र्गम् । \newline
53. अङ्गि॑रसः सुव॒र्गꣳ सु॑व॒र्ग मङ्गि॑रसो ऽङ्गिरसः सुव॒र्गम् ॅलो॒कम् ॅलो॒कꣳ सु॑व॒र्ग 
मङ्गि॑रसो ऽङ्गिरसः सुव॒र्गम् ॅलो॒कम् । \newline
54. सु॒व॒र्गम् ॅलो॒कम् ॅलो॒कꣳ सु॑व॒र्गꣳ सु॑व॒र्गम् ॅलो॒कं ॅयन्तो॒ यन्तो॑ लो॒कꣳ सु॑व॒र्गꣳ सु॑व॒र्गम् ॅलो॒कं ॅयन्तः॑ । \newline
55. सु॒व॒र्गमिति॑ सुवः - गम् । \newline
56. लो॒कं ॅयन्तो॒ यन्तो॑ लो॒कम् ॅलो॒कं ॅयन्तो॒ ऽजाया॑ म॒जायां॒ ॅयन्तो॑ लो॒कम् ॅलो॒कं ॅयन्तो॒ ऽजाया᳚म् । \newline
57. यन्तो॒ ऽजाया॑ म॒जायां॒ ॅयन्तो॒ यन्तो॒ ऽजाया᳚म् घ॒र्मम् घ॒र्म म॒जायां॒ ॅयन्तो॒ यन्तो॒ ऽजाया᳚म् घ॒र्मम् । \newline
\pagebreak
\markright{ TS 5.4.3.3  \hfill https://www.vedavms.in \hfill}

\section{ TS 5.4.3.3 }

\textbf{TS 5.4.3.3 } \newline
\textbf{Samhita Paata} \newline

-ऽजायां᳚ घ॒र्मं प्रासि॑ञ्च॒न्थ्सा शोच॑न्ती प॒र्णं परा॑ऽजिहीत॒ सो᳚(1॒)ऽर्को॑ऽभव॒त् तद॒र्कस्या᳚-र्क॒त्वम॑र्कप॒र्णेन॑ जुहोति सयोनि॒त्वायोद॒ङ् तिष्ठ॑न् जुहोत्ये॒षा वै रु॒द्रस्य॒ दिख् स्वाया॑मे॒व दि॒शि रु॒द्रं नि॒रव॑दयते चर॒माया॒मिष्ट॑कायां जुहोत्यन्त॒त ए॒व रु॒द्रं नि॒रव॑दयते त्रेधाविभ॒क्तं जु॑होति॒ त्रय॑ इ॒मे लो॒का इ॒माने॒व लो॒कान्थ् स॒माव॑द्वीर्यान् करो॒तीय॒त्यग्रे॑ जुहो॒त्य - [  ] \newline

\textbf{Pada Paata} \newline

अ॒जाया᳚म् । घ॒र्मम् । प्रेति॑ । अ॒सि॒ञ्च॒न्न् । सा । शोच॑न्ती । प॒र्णम् । परेति॑ । अ॒जि॒ही॒त॒ । सः । अ॒र्कः । अ॒भ॒व॒त् । तत् । अ॒र्कस्य॑ । अ॒र्क॒त्वमित्य॑र्क - त्वम् । अ॒र्क॒प॒र्णेनेत्य॑र्क - प॒र्णेन॑ । जु॒हो॒ति॒ । स॒यो॒नि॒त्वायेति॑ सयोनि - त्वाय॑ । उदङ्॑ । तिष्ठन्न्॑ । जु॒हो॒ति॒ । ए॒षा । वै । रु॒द्रस्य॑ । दिक् । स्वाया᳚म् । ए॒व । दि॒शि । रु॒द्रम् । नि॒रव॑दयत॒ इति॑ निः-अव॑दयते । च॒र॒माया᳚म् । इष्ट॑कायाम् । जु॒हो॒ति॒ । अ॒न्त॒तः । ए॒व । रु॒द्रम् । नि॒रव॑दयत॒ इति॑ निः - अव॑दयते । त्रे॒धा॒वि॒भ॒क्तमिति॑ त्रेधा - वि॒भ॒क्तम् । जु॒हो॒ति॒ । त्रयः॑ । इ॒मे । लो॒काः । इ॒मान् । ए॒व । लो॒कान् । स॒माव॑द्वीर्या॒निति॑ स॒माव॑त् - वी॒र्या॒न् । क॒रो॒ति॒ । इय॑ति । अग्रे᳚ । जु॒हो॒ति॒ ।  \newline


\textbf{Krama Paata} \newline

अ॒जाया᳚म् घ॒र्मम् । घ॒र्मम् प्र । प्रासि॑ञ्चन्न् । अ॒सि॒ञ्च॒न्थ् सा । सा शोच॑न्ती । शोच॑न्ती प॒र्णम् । प॒र्णम् परा᳚ । परा॑ऽजिहीत । अ॒जि॒ही॒त॒ सः । सो᳚ऽर्कः । अ॒र्को॑ऽभवत् । अ॒भ॒व॒त् तत् । तद॒र्कस्य॑ । अ॒र्कस्या᳚र्क॒त्वम् । अ॒र्क॒त्वम॑र्कप॒र्णेन॑ । अ॒र्क॒त्वमित्य॑र्क - त्वम् । अ॒र्क॒प॒र्णेन॑ जुहोति । अ॒र्क॒प॒र्णेनेत्य॑र्क - प॒र्णेन॑ । जु॒हो॒ति॒ स॒यो॒नि॒त्वाय॑ । स॒यो॒नि॒त्वायोदङ्॑ । स॒यो॒नि॒त्वायेति॑ सयोनि - त्वाय॑ । उद॒ङ् तिष्ठन्न्॑ । तिष्ठ॑न् जुहोति । जु॒हो॒त्ये॒षा । ए॒षा वै । वै रु॒द्रस्य॑ । रु॒द्रस्य॒ दिक् । दिख् स्वाया᳚म् । स्वाया॑मे॒व । ए॒व दि॒शि । दि॒शि रु॒द्रम् । रु॒द्रम् नि॒रव॑दयते । नि॒रव॑दयते चर॒माया᳚म् । नि॒रव॑दयत॒ इति॑ निः - अव॑दयते । च॒र॒माया॒मिष्ट॑कायाम् । इष्ट॑कायाम् जुहोति । जु॒हो॒त्य॒न्त॒तः । अ॒न्त॒त ए॒व । ए॒व रु॒द्रम् । रु॒द्रम् नि॒रव॑दयते । नि॒रव॑दयते त्रेधाविभ॒क्तम् । नि॒रव॑दयत॒ इति॑ निः - अव॑दयते । त्रे॒धा॒वि॒भ॒क्तम् जु॑होति । त्रे॒धा॒वि॒भ॒क्तमिति॑ त्रेधा - वि॒भ॒क्तम् । जु॒हो॒ति॒ त्रयः॑ । त्रय॑ इ॒मे । इ॒मे लो॒काः । लो॒का इ॒मान् । इ॒माने॒व । ए॒व लो॒कान् । लो॒कान्थ् स॒माव॑द्वीर्यान् । स॒माव॑द्वीर्यान् करोति । स॒माव॑द्वीर्या॒निति॑ स॒माव॑त् - वी॒र्या॒न्॒ । क॒रो॒तीय॑ति । इय॒त्यग्रे᳚ । अग्रे॑ जुहोति । जु॒हो॒त्यथ॑ \newline

\textbf{Jatai Paata} \newline

1. अ॒जाया᳚म् घ॒र्मम् घ॒र्म म॒जाया॑ म॒जाया᳚म् घ॒र्मम् । \newline
2. घ॒र्मम् प्र प्र घ॒र्मम् घ॒र्मम् प्र । \newline
3. प्रासि॑ञ्चन् नसिञ्च॒न् प्र प्रासि॑ञ्चन्न् । \newline
4. अ॒सि॒ञ्च॒न् थ्सा सा ऽसि॑ञ्चन् नसिञ्च॒न् थ्सा । \newline
5. सा शोच॑न्ती॒ शोच॑न्ती॒ सा सा शोच॑न्ती । \newline
6. शोच॑न्ती प॒र्णम् प॒र्णꣳ शोच॑न्ती॒ शोच॑न्ती प॒र्णम् । \newline
7. प॒र्णम् परा॒ परा॑ प॒र्णम् प॒र्णम् परा᳚ । \newline
8. परा॑ ऽजिहीता जिहीत॒ परा॒ परा॑ ऽजिहीत । \newline
9. अ॒जि॒ही॒त॒ स सो॑ ऽजिहीता जिहीत॒ सः । \newline
10. सो᳚(1॒) ऽर्को᳚ ऽर्कः स सो᳚ ऽर्कः । \newline
11. अ॒र्को॑ ऽभव दभव द॒र्को᳚(1॒) ऽर्को॑ ऽभवत् । \newline
12. अ॒भ॒व॒त् तत् तद॑भव दभव॒त् तत् । \newline
13. तद॒र्कस्या॒ र्कस्य॒ तत् तद॒र्कस्य॑ । \newline
14. अ॒र्कस्या᳚ र्क॒त्व म॑र्क॒त्व म॒र्कस्या॒ र्कस्या᳚ र्क॒त्वम् । \newline
15. अ॒र्क॒त्व म॑र्कप॒र्णेना᳚ र्कप॒र्णेना᳚ र्क॒त्व म॑र्क॒त्व म॑र्कप॒र्णेन॑ । \newline
16. अ॒र्क॒त्वमित्य॑र्क - त्वम् । \newline
17. अ॒र्क॒प॒र्णेन॑ जुहोति जुहो त्यर्कप॒र्णेना᳚ र्कप॒र्णेन॑ जुहोति । \newline
18. अ॒र्क॒प॒र्णेनेत्य॑र्क - प॒र्णेन॑ । \newline
19. जु॒हो॒ति॒ स॒यो॒नि॒त्वाय॑ सयोनि॒त्वाय॑ जुहोति जुहोति सयोनि॒त्वाय॑ । \newline
20. स॒यो॒नि॒त्वा योद॒ङ् ङुदङ्᳚ ख्सयोनि॒त्वाय॑ सयोनि॒त्वा योदङ्॑ । \newline
21. स॒यो॒नि॒त्वायेति॑ सयोनि - त्वाय॑ । \newline
22. उद॒ङ् तिष्ठꣳ॒॒ स्तिष्ठ॒न् नुद॒ङ् ङुद॒ङ् तिष्ठन्न्॑ । \newline
23. तिष्ठ॑न् जुहोति जुहोति॒ तिष्ठꣳ॒॒ स्तिष्ठ॑न् जुहोति । \newline
24. जु॒हो॒ त्ये॒षैषा जु॑होति जुहो त्ये॒षा । \newline
25. ए॒षा वै वा ए॒षैषा वै । \newline
26. वै रु॒द्रस्य॑ रु॒द्रस्य॒ वै वै रु॒द्रस्य॑ । \newline
27. रु॒द्रस्य॒ दिग् दिग् रु॒द्रस्य॑ रु॒द्रस्य॒ दिक् । \newline
28. दिख् स्वायाꣳ॒॒ स्वाया॒म् दिग् दिख् स्वाया᳚म् । \newline
29. स्वाया॑ मे॒वैव स्वायाꣳ॒॒ स्वाया॑ मे॒व । \newline
30. ए॒व दि॒शि दि॒श्ये॑ वैव दि॒शि । \newline
31. दि॒शि रु॒द्रꣳ रु॒द्रम् दि॒शि दि॒शि रु॒द्रम् । \newline
32. रु॒द्रम् नि॒रव॑दयते नि॒रव॑दयते रु॒द्रꣳ रु॒द्रम् नि॒रव॑दयते । \newline
33. नि॒रव॑दयते चर॒माया᳚म् चर॒माया᳚म् नि॒रव॑दयते नि॒रव॑दयते चर॒माया᳚म् । \newline
34. नि॒रव॑दयत॒ इति॑ निः - अव॑दयते । \newline
35. च॒र॒माया॒ मिष्ट॑काया॒ मिष्ट॑कायाम् चर॒माया᳚म् चर॒माया॒ मिष्ट॑कायाम् । \newline
36. इष्ट॑कायाम् जुहोति जुहो॒तीष्ट॑काया॒ मिष्ट॑कायाम् जुहोति । \newline
37. जु॒हो॒ त्य॒न्त॒तो᳚ ऽन्त॒तो जु॑होति जुहो त्यन्त॒तः । \newline
38. अ॒न्त॒त ए॒वैवा न्त॒तो᳚ ऽन्त॒त ए॒व । \newline
39. ए॒व रु॒द्रꣳ रु॒द्र मे॒वैव रु॒द्रम् । \newline
40. रु॒द्रम् नि॒रव॑दयते नि॒रव॑दयते रु॒द्रꣳ रु॒द्रम् नि॒रव॑दयते । \newline
41. नि॒रव॑दयते त्रेधाविभ॒क्तम् त्रे॑धाविभ॒क्तम् नि॒रव॑दयते नि॒रव॑दयते त्रेधाविभ॒क्तम् । \newline
42. नि॒रव॑दयत॒ इति॑ निः - अव॑दयते । \newline
43. त्रे॒धा॒वि॒भ॒क्तम् जु॑होति जुहोति त्रेधाविभ॒क्तम् त्रे॑धाविभ॒क्तम् जु॑होति । \newline
44. त्रे॒धा॒वि॒भ॒क्तमिति॑ त्रेधा - वि॒भ॒क्तम् । \newline
45. जु॒हो॒ति॒ त्रय॒ स्त्रयो॑ जुहोति जुहोति॒ त्रयः॑ । \newline
46. त्रय॑ इ॒म इ॒मे त्रय॒ स्त्रय॑ इ॒मे । \newline
47. इ॒मे लो॒का लो॒का इ॒म इ॒मे लो॒काः । \newline
48. लो॒का इ॒मा नि॒मान् ॅलो॒का लो॒का इ॒मान् । \newline
49. इ॒मा ने॒वैवेमा नि॒मा ने॒व । \newline
50. ए॒व लो॒कान् ॅलो॒का ने॒वैव लो॒कान् । \newline
51. लो॒कान् थ्स॒माव॑द्वीर्यान् थ्स॒माव॑द्वीर्यान् ॅलो॒कान् ॅलो॒कान् थ्स॒माव॑द्वीर्यान् । \newline
52. स॒माव॑द्वीर्यान् करोति करोति स॒माव॑द्वीर्यान् थ्स॒माव॑द्वीर्यान् करोति । \newline
53. स॒माव॑द्वीर्या॒निति॑ स॒माव॑त् - वी॒र्या॒न् । \newline
54. क॒रो॒ती य॒ती य॑ति करोति करो॒ती य॑ति । \newline
55. इय॒ त्यग्रे ऽग्र॒ इय॒ती य॒त्यग्रे᳚ । \newline
56. अग्रे॑ जुहोति जुहो॒ त्यग्रे ऽग्रे॑ जुहोति । \newline
57. जु॒हो॒ त्यथाथ॑ जुहोति जुहो॒ त्यथ॑ । \newline

\textbf{Ghana Paata } \newline

1. अ॒जाया᳚म् घ॒र्मम् घ॒र्म म॒जाया॑ म॒जाया᳚म् घ॒र्मम् प्र प्र घ॒र्म म॒जाया॑ म॒जाया᳚म् घ॒र्मम् प्र । \newline
2. घ॒र्मम् प्र प्र घ॒र्मम् घ॒र्मम् प्रासि॑ञ्चन् नसिञ्च॒न् प्र घ॒र्मम् घ॒र्मम् प्रासि॑ञ्चन्न् । \newline
3. प्रासि॑ञ्चन् नसिञ्च॒न् प्र प्रासि॑ञ्च॒न् थ्सा सा ऽसि॑ञ्च॒न् प्र प्रासि॑ञ्च॒न् थ्सा । \newline
4. अ॒सि॒ञ्च॒न् थ्सा सा ऽसि॑ञ्चन् नसिञ्च॒न् थ्सा शोच॑न्ती॒ शोच॑न्ती॒ सा ऽसि॑ञ्चन् नसिञ्च॒न् थ्सा शोच॑न्ती । \newline
5. सा शोच॑न्ती॒ शोच॑न्ती॒ सा सा शोच॑न्ती प॒र्णम् प॒र्णꣳ शोच॑न्ती॒ सा सा शोच॑न्ती प॒र्णम् । \newline
6. शोच॑न्ती प॒र्णम् प॒र्णꣳ शोच॑न्ती॒ शोच॑न्ती प॒र्णम् परा॒ परा॑ प॒र्णꣳ शोच॑न्ती॒ शोच॑न्ती प॒र्णम् परा᳚ । \newline
7. प॒र्णम् परा॒ परा॑ प॒र्णम् प॒र्णम् परा॑ ऽजिहीता जिहीत॒ परा॑ प॒र्णम् प॒र्णम् परा॑ ऽजिहीत । \newline
8. परा॑ ऽजिही ताजिहीत॒ परा॒ परा॑ ऽजिहीत॒ स सो॑ ऽजिहीत॒ परा॒ परा॑ ऽजिहीत॒ सः । \newline
9. अ॒जि॒ही॒त॒ स सो॑ ऽजिहीता जिहीत॒ सो᳚(1॒) ऽर्को᳚ ऽर्कः सो॑ ऽजिहीता जिहीत॒ सो᳚ ऽर्कः । \newline
10. सो᳚(1॒) ऽर्को᳚ ऽर्कः स सो᳚(1॒) ऽर्को॑ ऽभव दभव द॒र्कः स सो᳚(1॒) ऽर्को॑ ऽभवत् । \newline
11. अ॒र्को॑ ऽभव दभव द॒र्को᳚(1॒) ऽर्को॑ ऽभव॒त् तत् तद॑भव द॒र्को᳚(1॒) ऽर्को॑ ऽभव॒त् तत् । \newline
12. अ॒भ॒व॒त् तत् तद॑भव दभव॒त् तद॒र्कस्या॒ र्कस्य॒ तद॑भव दभव॒त् तद॒र्कस्य॑ । \newline
13. तद॒र्कस्या॒ र्कस्य॒ तत् तद॒र्कस्या᳚ र्क॒त्व म॑र्क॒त्व म॒र्कस्य॒ तत् तद॒र्कस्या᳚ र्क॒त्वम् । \newline
14. अ॒र्कस्या᳚ र्क॒त्व म॑र्क॒त्व म॒र्कस्या॒ र्कस्या᳚ र्क॒त्व म॑र्कप॒र्णेना᳚ र्कप॒र्णेना᳚ र्क॒त्व म॒र्कस्या॒ र्कस्या᳚ र्क॒त्व म॑र्कप॒र्णेन॑ । \newline
15. अ॒र्क॒त्व म॑र्कप॒र्णेना᳚ र्कप॒र्णेना᳚ र्क॒त्व म॑र्क॒त्व म॑र्कप॒र्णेन॑ जुहोति जुहो त्यर्कप॒र्णेना᳚ र्क॒त्व म॑र्क॒त्व म॑र्कप॒र्णेन॑ जुहोति । \newline
16. अ॒र्क॒त्वमित्य॑र्क - त्वम् । \newline
17. अ॒र्क॒प॒र्णेन॑ जुहोति जुहो त्यर्कप॒र्णेना᳚ र्कप॒र्णेन॑ जुहोति सयोनि॒त्वाय॑ सयोनि॒त्वाय॑ जुहो त्यर्कप॒र्णेना᳚ र्कप॒र्णेन॑ जुहोति सयोनि॒त्वाय॑ । \newline
18. अ॒र्क॒प॒र्णेनेत्य॑र्क - प॒र्णेन॑ । \newline
19. जु॒हो॒ति॒ स॒यो॒नि॒त्वाय॑ सयोनि॒त्वाय॑ जुहोति जुहोति सयोनि॒त्वा योद॒ङ् ङुदङ्᳚ ख्सयोनि॒त्वाय॑ जुहोति जुहोति सयोनि॒त्वा योदङ्॑ । \newline
20. स॒यो॒नि॒त्वा योद॒ङ् ङुदङ्᳚ ख्सयोनि॒त्वाय॑ सयोनि॒त्वायोद॒ङ् तिष्ठꣳ॒॒ स्तिष्ठ॒न् नुदङ्᳚ ख्सयोनि॒त्वाय॑ सयोनि॒त्वा योद॒ङ् तिष्ठन्न्॑ । \newline
21. स॒यो॒नि॒त्वायेति॑ सयोनि - त्वाय॑ । \newline
22. उद॒ङ् तिष्ठꣳ॒॒ स्तिष्ठ॒न् नुद॒ङ् ङुद॒ङ् तिष्ठ॑न् जुहोति जुहोति॒ तिष्ठ॒न् नुद॒ङ् ङुद॒ङ् तिष्ठ॑न् जुहोति । \newline
23. तिष्ठ॑न् जुहोति जुहोति॒ तिष्ठꣳ॒॒ स्तिष्ठ॑न् जुहो त्ये॒षैषा जु॑होति॒ तिष्ठꣳ॒॒ स्तिष्ठ॑न् जुहो त्ये॒षा । \newline
24. जु॒हो॒ त्ये॒षैषा जु॑होति जुहो त्ये॒षा वै वा ए॒षा जु॑होति जुहो त्ये॒षा वै । \newline
25. ए॒षा वै वा ए॒षैषा वै रु॒द्रस्य॑ रु॒द्रस्य॒ वा ए॒षैषा वै रु॒द्रस्य॑ । \newline
26. वै रु॒द्रस्य॑ रु॒द्रस्य॒ वै वै रु॒द्रस्य॒ दिग् दिग् रु॒द्रस्य॒ वै वै रु॒द्रस्य॒ दिक् । \newline
27. रु॒द्रस्य॒ दिग् दिग् रु॒द्रस्य॑ रु॒द्रस्य॒ दिख् स्वायाꣳ॒॒ स्वाया॒म् दिग् रु॒द्रस्य॑ रु॒द्रस्य॒ दिख् स्वाया᳚म् । \newline
28. दिख् स्वायाꣳ॒॒ स्वाया॒म् दिग् दिख् स्वाया॑ मे॒वैव स्वाया॒म् दिग् दिख् स्वाया॑ मे॒व । \newline
29. स्वाया॑ मे॒वैव स्वायाꣳ॒॒ स्वाया॑ मे॒व दि॒शि दि॒श्ये॑व स्वायाꣳ॒॒ स्वाया॑ मे॒व दि॒शि । \newline
30. ए॒व दि॒शि दि॒श्ये॑वैव दि॒शि रु॒द्रꣳ रु॒द्रम् दि॒श्ये॑वैव दि॒शि रु॒द्रम् । \newline
31. दि॒शि रु॒द्रꣳ रु॒द्रम् दि॒शि दि॒शि रु॒द्रन् नि॒रव॑दयते नि॒रव॑दयते रु॒द्रम् दि॒शि दि॒शि रु॒द्रम् नि॒रव॑दयते । \newline
32. रु॒द्रन् नि॒रव॑दयते नि॒रव॑दयते रु॒द्रꣳ रु॒द्रन् नि॒रव॑दयते चर॒माया᳚म् चर॒माया᳚म् नि॒रव॑दयते रु॒द्रꣳ रु॒द्रम् नि॒रव॑दयते चर॒माया᳚म् । \newline
33. नि॒रव॑दयते चर॒माया᳚म् चर॒माया᳚म् नि॒रव॑दयते नि॒रव॑दयते चर॒माया॒ मिष्ट॑काया॒ मिष्ट॑कायाम् चर॒माया᳚म् नि॒रव॑दयते नि॒रव॑दयते चर॒माया॒ मिष्ट॑कायाम् । \newline
34. नि॒रव॑दयत॒ इति॑ निः - अव॑दयते । \newline
35. च॒र॒माया॒ मिष्ट॑काया॒ मिष्ट॑कायाम् चर॒माया᳚म् चर॒माया॒ मिष्ट॑कायाम् जुहोति जुहो॒तीष्ट॑कायाम् चर॒माया᳚म् चर॒माया॒ मिष्ट॑कायाम् जुहोति । \newline
36. इष्ट॑कायाम् जुहोति जुहो॒तीष्ट॑काया॒ मिष्ट॑कायाम् जुहो त्यन्त॒तो᳚ ऽन्त॒तो जु॑हो॒तीष्ट॑काया॒ मिष्ट॑कायाम् जुहो त्यन्त॒तः । \newline
37. जु॒हो॒ त्य॒न्त॒तो᳚ ऽन्त॒तो जु॑होति जुहो त्यन्त॒त ए॒वै वान्त॒तो जु॑होति जुहो त्यन्त॒त ए॒व । \newline
38. अ॒न्त॒त ए॒वै वान्त॒तो᳚ ऽन्त॒त ए॒व रु॒द्रꣳ रु॒द्र मे॒वा न्त॒तो᳚ ऽन्त॒त ए॒व रु॒द्रम् । \newline
39. ए॒व रु॒द्रꣳ रु॒द्र मे॒वैव रु॒द्रम् नि॒रव॑दयते नि॒रव॑दयते रु॒द्र मे॒वैव रु॒द्रम् नि॒रव॑दयते । \newline
40. रु॒द्रम् नि॒रव॑दयते नि॒रव॑दयते रु॒द्रꣳ रु॒द्रम् नि॒रव॑दयते त्रेधाविभ॒क्तम् त्रे॑धाविभ॒क्तम् नि॒रव॑दयते रु॒द्रꣳ रु॒द्रम् नि॒रव॑दयते त्रेधाविभ॒क्तम् । \newline
41. नि॒रव॑दयते त्रेधाविभ॒क्तम् त्रे॑धाविभ॒क्तम् नि॒रव॑दयते नि॒रव॑दयते त्रेधाविभ॒क्तम् जु॑होति जुहोति त्रेधाविभ॒क्तम् नि॒रव॑दयते नि॒रव॑दयते त्रेधाविभ॒क्तम् जु॑होति । \newline
42. नि॒रव॑दयत॒ इति॑ निः - अव॑दयते । \newline
43. त्रे॒धा॒वि॒भ॒क्तम् जु॑होति जुहोति त्रेधाविभ॒क्तम् त्रे॑धाविभ॒क्तम् जु॑होति॒ त्रय॒ स्त्रयो॑ जुहोति त्रेधाविभ॒क्तम् त्रे॑धाविभ॒क्तम् जु॑होति॒ त्रयः॑ । \newline
44. त्रे॒धा॒वि॒भ॒क्तमिति॑ त्रेधा - वि॒भ॒क्तम् । \newline
45. जु॒हो॒ति॒ त्रय॒ स्त्रयो॑ जुहोति जुहोति॒ त्रय॑ इ॒म इ॒मे त्रयो॑ जुहोति जुहोति॒ त्रय॑ इ॒मे । \newline
46. त्रय॑ इ॒म इ॒मे त्रय॒ स्त्रय॑ इ॒मे लो॒का लो॒का इ॒मे त्रय॒ स्त्रय॑ इ॒मे लो॒काः । \newline
47. इ॒मे लो॒का लो॒का इ॒म इ॒मे लो॒का इ॒मा नि॒मान् ॅलो॒का इ॒म इ॒मे लो॒का इ॒मान् । \newline
48. लो॒का इ॒मा नि॒मान् ॅलो॒का लो॒का इ॒मा ने॒वैवेमान् ॅलो॒का लो॒का इ॒मा ने॒व । \newline
49. इ॒मा ने॒वैवेमा नि॒मा ने॒व लो॒कान् ॅलो॒का ने॒वेमा नि॒मा ने॒व लो॒कान् । \newline
50. ए॒व लो॒कान् ॅलो॒का ने॒वैव लो॒कान् थ्स॒माव॑द्वीर्यान् थ्स॒माव॑द्वीर्यान् ॅलो॒का ने॒वैव लो॒कान् थ्स॒माव॑द्वीर्यान् । \newline
51. लो॒कान् थ्स॒माव॑द्वीर्यान् थ्स॒माव॑द्वीर्यान् ॅलो॒कान् ॅलो॒कान् थ्स॒माव॑द्वीर्यान् करोति करोति स॒माव॑द्वीर्यान् ॅलो॒कान् ॅलो॒कान् थ्स॒माव॑द्वीर्यान् करोति । \newline
52. स॒माव॑द्वीर्यान् करोति करोति स॒माव॑द्वीर्यान् थ्स॒माव॑द्वीर्यान् करो॒ती य॒तीय॑ति करोति स॒माव॑द्वीर्यान् थ्स॒माव॑द्वीर्यान् करो॒तीय॑ति । \newline
53. स॒माव॑द्वीर्या॒निति॑ स॒माव॑त् - वी॒र्या॒न् । \newline
54. क॒रो॒तीय॒ तीय॑ति करोति करो॒ती य॒त्यग्रे ऽग्र॒ इय॑ति करोति करो॒ती य॒त्यग्रे᳚ । \newline
55. इय॒त्यग्रे ऽग्र॒ इय॒ती य॒त्यग्रे॑ जुहोति जुहो॒ त्यग्र॒ इय॒ती य॒त्यग्रे॑ जुहोति । \newline
56. अग्रे॑ जुहोति जुहो॒ त्यग्रे ऽग्रे॑ जुहो॒त्य थाथ॑ जुहो॒ त्यग्रे ऽग्रे॑ जुहो॒ त्यथ॑ । \newline
57. जु॒हो॒ त्यथाथ॑ जुहोति जुहो॒ त्यथे य॒तीय॒ त्यथ॑ जुहोति जुहो॒ त्यथे य॑ति । \newline
\pagebreak
\markright{ TS 5.4.3.4  \hfill https://www.vedavms.in \hfill}

\section{ TS 5.4.3.4 }

\textbf{TS 5.4.3.4 } \newline
\textbf{Samhita Paata} \newline

-थेय॒त्यथेय॑ति॒ त्रय॑ इ॒मे लो॒का ए॒भ्य ए॒वैनं॑ ॅलो॒केभ्यः॑ शमयति ति॒स्र उत्त॑रा॒ आहु॑तीर्जुहोति॒ षट् थ्सं प॑द्यन्ते॒ षड् वा ऋ॒तव॑ ऋ॒तुभि॑रे॒वैनꣳ॑ शमयति॒ यद॑नुपरि॒क्रामं॑ जुहु॒याद॑न्तरवचा॒रिणꣳ॑ रु॒द्रं कु॑र्या॒दथो॒ खल्वा॑हुः॒ कस्यां॒ ॅवाऽह॑ दि॒शि रु॒द्रः कस्यां॒ ॅवेत्य॑नुपरि॒क्राम॑मे॒व हो॑त॒व्य॑-मप॑रिवर्गमे॒वैनꣳ॑ शमयत्ये॒ - [  ] \newline

\textbf{Pada Paata} \newline

अथ॑ । इय॑ति । अथ॑ । इय॑ति । त्रयः॑ । इ॒मे । लो॒काः । ए॒भ्यः । ए॒व । ए॒न॒म् । लो॒केभ्यः॑ । श॒म॒य॒ति॒ । ति॒स्रः । उत्त॑रा॒ इत्युत् - त॒राः॒ । आहु॑ती॒रित्या-हु॒तीः॒ । जु॒हो॒ति॒ । षट् । समिति॑ । प॒द्य॒न्ते॒ । षट् । वै । ऋ॒तवः॑ । ऋ॒तुभि॒रित्यृ॒तु - भिः॒ । ए॒व । ए॒न॒म् । श॒म॒य॒ति॒ । यत् । अ॒नु॒प॒रि॒क्राम॒मित्य॑नु - प॒रि॒क्राम᳚म् । जु॒हु॒यात् । अ॒न्त॒र॒व॒चा॒रिण॒मित्य॑न्तः - अ॒व॒चा॒रिण᳚म् । रु॒द्रम् । कु॒र्या॒त् । अथो॒ इति॑ । खलु॑ । आ॒हुः॒ । कस्या᳚म् । वा॒ । अह॑ । दि॒शि । रु॒द्रः । कस्या᳚म् । वा॒ । इति॑ । अ॒नु॒प॒रि॒क्राम॒मित्य॑नु - प॒रि॒क्राम᳚म् । ए॒व । हो॒त॒व्य᳚म् । अप॑रिवर्ग॒मित्यप॑रि - व॒र्ग॒म् । ए॒व । ए॒न॒म् । श॒म॒य॒ति॒ ।  \newline


\textbf{Krama Paata} \newline

अथेय॑ति । इय॒त्यथ॑ । अथेय॑ति । इय॑ति॒ त्रयः॑ । त्रय॑ इ॒मे । इ॒मे लो॒काः । लो॒का ए॒भ्यः । ए॒भ्य ए॒व । ए॒वैन᳚म् । ए॒न॒म् ॅलो॒केभ्यः॑ । लो॒केभ्यः॑ शमयति । श॒म॒य॒ति॒ ति॒स्रः । ति॒स्र उत्त॑राः । उत्त॑रा॒ आहु॑तीः । उत्त॑रा॒ इत्युत् - त॒राः॒ । आहु॑तीर् जुहोति । आहु॑ती॒रित्या - हु॒तीः॒ । जु॒हो॒ति॒ षट् । षट्थ् सम् । सम् प॑द्यन्ते । प॒द्य॒न्ते॒ षट् । षड् वै । वा ऋ॒तवः॑ । ऋ॒तव॑ ऋ॒तुभिः॑ । ऋ॒तुभि॑रे॒व । ऋ॒तुभि॒रित्यृ॒तु - भिः॒ । ए॒वैन᳚म् । ए॒नꣳ॒॒ श॒म॒य॒ति॒ । श॒म॒य॒ति॒ यत् । यद॑नुपरि॒क्राम᳚म् । अ॒नु॒प॒रि॒क्राम॑म् जुहु॒यात् । अ॒नु॒प॒रि॒क्राम॒मित्य॑नु - प॒रि॒क्राम᳚म् । जु॒हु॒याद॑न्तरवचा॒रिण᳚म् । अ॒न्त॒र॒व॒चा॒रिणꣳ॑ रु॒द्रम् । अ॒न्त॒र॒व॒चा॒रिण॒मित्य॑न्तः - अ॒व॒चा॒रिण᳚म् । रु॒द्रम् कु॑र्यात् । कु॒र्या॒दथो᳚ । अथो॒ खलु॑ । अथो॒ इत्यथो᳚ । खल्वा॑हुः । आ॒हुः॒ कस्या᳚म् । कस्या᳚म् ॅवा । वा ऽह॑ । अह॑ दि॒शि । दि॒शि रु॒द्रः । रु॒द्रः कस्या᳚म् । कस्या᳚म् ॅवा । वेति॑ । इत्य॑नुपरि॒क्राम᳚म् । अ॒नु॒प॒रि॒क्राम॑मे॒व । अ॒नु॒प॒रि॒क्राम॒मित्य॑नु - प॒रि॒क्राम᳚म् । ए॒व हो॑त॒व्य᳚म् । हो॒त॒व्य॑मप॑रिवर्गम् । अप॑रिवर्गमे॒व । अप॑रिवर्ग॒मित्यप॑रि - व॒र्ग॒म् । ए॒वैन᳚म् । ए॒नꣳ॒॒ श॒म॒य॒ति॒ ( ) । श॒म॒य॒त्ये॒ताः \newline

\textbf{Jatai Paata} \newline

1. अथे य॒ती य॒त्यथाथे य॑ति । \newline
2. इय॒त्यथाथे य॒ती य॒त्यथ॑ । \newline
3. अथे य॒ती य॒त्यथाथे य॑ति । \newline
4. इय॑ति॒ त्रय॒ स्त्रय॒ इय॒ती य॑ति॒ त्रयः॑ । \newline
5. त्रय॑ इ॒म इ॒मे त्रय॒ स्त्रय॑ इ॒मे । \newline
6. इ॒मे लो॒का लो॒का इ॒म इ॒मे लो॒काः । \newline
7. लो॒का ए॒भ्य ए॒भ्यो लो॒का लो॒का ए॒भ्यः । \newline
8. ए॒भ्य ए॒वै वैभ्य ए॒भ्य ए॒व । \newline
9. ए॒वैन॑ मेन मे॒वै वैन᳚म् । \newline
10. ए॒न॒म् ॅलो॒केभ्यो॑ लो॒केभ्य॑ एन मेनम् ॅलो॒केभ्यः॑ । \newline
11. लो॒केभ्यः॑ शमयति शमयति लो॒केभ्यो॑ लो॒केभ्यः॑ शमयति । \newline
12. श॒म॒य॒ति॒ ति॒स्र स्ति॒स्रः श॑मयति शमयति ति॒स्रः । \newline
13. ति॒स्र उत्त॑रा॒ उत्त॑रा स्ति॒स्र स्ति॒स्र उत्त॑राः । \newline
14. उत्त॑रा॒ आहु॑ती॒ राहु॑ती॒ रुत्त॑रा॒ उत्त॑रा॒ आहु॑तीः । \newline
15. उत्त॑रा॒ इत्युत् - त॒राः॒ । \newline
16. आहु॑तीर् जुहोति जुहो॒ त्याहु॑ती॒ राहु॑तीर् जुहोति । \newline
17. आहु॑ती॒रित्या - हु॒तीः॒ । \newline
18. जु॒हो॒ति॒ षट् थ्षड् जु॑होति जुहोति॒ षट् । \newline
19. षट् थ्सꣳ सꣳ षट् थ्षट् थ्सम् । \newline
20. सम् प॑द्यन्ते पद्यन्ते॒ सꣳ सम् प॑द्यन्ते । \newline
21. प॒द्य॒न्ते॒ षट् थ्षट् प॑द्यन्ते पद्यन्ते॒ षट् । \newline
22. षड् वै वै षट् थ्षड् वै । \newline
23. वा ऋ॒तव॑ ऋ॒तवो॒ वै वा ऋ॒तवः॑ । \newline
24. ऋ॒तव॑ ऋ॒तुभिर्॑. ऋ॒तुभिर्॑. ऋ॒तव॑ ऋ॒तव॑ ऋ॒तुभिः॑ । \newline
25. ऋ॒तुभि॑ रे॒वैव र्‌तुभिर्॑. ऋ॒तुभि॑ रे॒व । \newline
26. ऋ॒तुभि॒रित्यृ॒तु - भिः॒ । \newline
27. ए॒वैन॑ मेन मे॒वै वैन᳚म् । \newline
28. ए॒नꣳ॒॒ श॒म॒य॒ति॒ श॒म॒य॒ त्ये॒न॒ मे॒नꣳ॒॒ श॒म॒य॒ति॒ । \newline
29. श॒म॒य॒ति॒ यद् यच्छ॑मयति शमयति॒ यत् । \newline
30. यद॑नुपरि॒क्राम॑ मनुपरि॒क्रामं॒ ॅयद् यद॑नुपरि॒क्राम᳚म् । \newline
31. अ॒नु॒प॒रि॒क्राम॑म् जुहु॒याज् जु॑हु॒या द॑नुपरि॒क्राम॑ मनुपरि॒क्राम॑म् जुहु॒यात् । \newline
32. अ॒नु॒प॒रि॒क्राम॒मित्य॑नु - प॒रि॒क्राम᳚म् । \newline
33. जु॒हु॒या द॑न्तरवचा॒रिण॑ मन्तरवचा॒रिण॑म् जुहु॒याज् जु॑हु॒या द॑न्तरवचा॒रिण᳚म् । \newline
34. अ॒न्त॒र॒व॒चा॒रिणꣳ॑ रु॒द्रꣳ रु॒द्र म॑न्तरवचा॒रिण॑ मन्तरवचा॒रिणꣳ॑ रु॒द्रम् । \newline
35. अ॒न्त॒र॒व॒चा॒रिण॒मित्य॑न्तः - अ॒व॒चा॒रिण᳚म् । \newline
36. रु॒द्रम् कु॑र्यात् कुर्याद् रु॒द्रꣳ रु॒द्रम् कु॑र्यात् । \newline
37. कु॒र्या॒ दथो॒ अथो॑ कुर्यात् कुर्या॒ दथो᳚ । \newline
38. अथो॒ खलु॒ खल्वथो॒ अथो॒ खलु॑ । \newline
39. अथो॒ इत्यथो᳚ । \newline
40. खल्वा॑हु राहुः॒ खलु॒ खल्वा॑हुः । \newline
41. आ॒हुः॒ कस्या॒म् कस्या॑ माहु राहुः॒ कस्या᳚म् । \newline
42. कस्यां᳚ ॅवा वा॒ कस्या॒म् कस्यां᳚ ॅवा । \newline
43. वा ऽहाह॑ वा॒ वा ऽह॑ । \newline
44. अह॑ दि॒शि दि॒श्य हाह॑ दि॒शि । \newline
45. दि॒शि रु॒द्रो रु॒द्रो दि॒शि दि॒शि रु॒द्रः । \newline
46. रु॒द्रः कस्या॒म् कस्याꣳ॑ रु॒द्रो रु॒द्रः कस्या᳚म् । \newline
47. कस्यां᳚ ॅवा वा॒ कस्या॒म् कस्यां᳚ ॅवा । \newline
48. वेतीति॑ वा॒ वेति॑ । \newline
49. इत्य॑नुपरि॒क्राम॑ मनुपरि॒क्राम॒ मिती त्य॑नुपरि॒क्राम᳚म् । \newline
50. अ॒नु॒प॒रि॒क्राम॑ मे॒वै वानु॑ परि॒क्राम॑ मनुपरि॒क्राम॑ मे॒व । \newline
51. अ॒नु॒प॒रि॒क्राम॒मित्य॑नु - प॒रि॒क्राम᳚म् । \newline
52. ए॒व हो॑त॒व्यꣳ॑ होत॒व्य॑ मे॒वैव हो॑त॒व्य᳚म् । \newline
53. हो॒त॒व्य॑ मप॑रिवर्ग॒ मप॑रिवर्गꣳ होत॒व्यꣳ॑ होत॒व्य॑ मप॑रिवर्गम् । \newline
54. अप॑रिवर्ग मे॒वैवा प॑रिवर्ग॒ मप॑रिवर्ग मे॒व । \newline
55. अप॑रिवर्ग॒मित्यप॑रि - व॒र्ग॒म् । \newline
56. ए॒वैन॑ मेन मे॒वै वैन᳚म् । \newline
57. ए॒नꣳ॒॒ श॒म॒य॒ति॒ श॒म॒य॒ त्ये॒न॒ मे॒नꣳ॒॒ श॒म॒य॒ति॒ । \newline
58. श॒म॒य॒ त्ये॒ता ए॒ताः श॑मयति शमय त्ये॒ताः । \newline

\textbf{Ghana Paata } \newline

1. अथेय॒तीय॒ त्यथाथेय॒ त्यथाथेय॒ त्यथाथेय॒ त्यथ॑ । \newline
2. इय॒ त्यथाथे य॒तीय॒त्यथे य॒तीय॒ त्यथे य॒तीय॒ त्यथे य॑ति । \newline
3. अथे य॒ती य॒त्यथाथे य॑ति॒ त्रय॒ स्त्रय॒ इय॒ त्यथाथे य॑ति॒ त्रयः॑ । \newline
4. इय॑ति॒ त्रय॒ स्त्रय॒ इय॒ती य॑ति॒ त्रय॑ इ॒म इ॒मे त्रय॒ इय॒ती य॑ति॒ त्रय॑ इ॒मे । \newline
5. त्रय॑ इ॒म इ॒मे त्रय॒ स्त्रय॑ इ॒मे लो॒का लो॒का इ॒मे त्रय॒ स्त्रय॑ इ॒मे लो॒काः । \newline
6. इ॒मे लो॒का लो॒का इ॒म इ॒मे लो॒का ए॒भ्य ए॒भ्यो लो॒का इ॒म इ॒मे लो॒का ए॒भ्यः । \newline
7. लो॒का ए॒भ्य ए॒भ्यो लो॒का लो॒का ए॒भ्य ए॒वै वैभ्यो लो॒का लो॒का ए॒भ्य ए॒व । \newline
8. ए॒भ्य ए॒वै वैभ्य ए॒भ्य ए॒वैन॑ मेन मे॒वैभ्य ए॒भ्य ए॒वैन᳚म् । \newline
9. ए॒वैन॑ मेन मे॒वै वैन॑म् ॅलो॒केभ्यो॑ लो॒केभ्य॑ एन मे॒वै वैन॑म् ॅलो॒केभ्यः॑ । \newline
10. ए॒न॒म् ॅलो॒केभ्यो॑ लो॒केभ्य॑ एन मेनम् ॅलो॒केभ्यः॑ शमयति शमयति लो॒केभ्य॑ एन मेनम् ॅलो॒केभ्यः॑ शमयति । \newline
11. लो॒केभ्यः॑ शमयति शमयति लो॒केभ्यो॑ लो॒केभ्यः॑ शमयति ति॒स्र स्ति॒स्रः श॑मयति लो॒केभ्यो॑ लो॒केभ्यः॑ शमयति ति॒स्रः । \newline
12. श॒म॒य॒ति॒ ति॒स्र स्ति॒स्रः श॑मयति शमयति ति॒स्र उत्त॑रा॒ उत्त॑रा स्ति॒स्रः श॑मयति शमयति ति॒स्र उत्त॑राः । \newline
13. ति॒स्र उत्त॑रा॒ उत्त॑रा स्ति॒स्र स्ति॒स्र उत्त॑रा॒ आहु॑ती॒ राहु॑ती॒ रुत्त॑रा स्ति॒स्र स्ति॒स्र उत्त॑रा॒ आहु॑तीः । \newline
14. उत्त॑रा॒ आहु॑ती॒ राहु॑ती॒ रुत्त॑रा॒ उत्त॑रा॒ आहु॑तीर् जुहोति जुहो॒ त्याहु॑ती॒ रुत्त॑रा॒ उत्त॑रा॒ आहु॑तीर् जुहोति । \newline
15. उत्त॑रा॒ इत्युत् - त॒राः॒ । \newline
16. आहु॑तीर् जुहोति जुहो॒ त्याहु॑ती॒ राहु॑तीर् जुहोति॒ षट् थ्षड् जु॑हो॒ त्याहु॑ती॒ राहु॑तीर् जुहोति॒ षट् । \newline
17. आहु॑ती॒रित्या - हु॒तीः॒ । \newline
18. जु॒हो॒ति॒ षट् थ्षड् जु॑होति जुहोति॒ षट् थ्सꣳ सꣳ षड् जु॑होति जुहोति॒ षट् थ्सम् । \newline
19. षट् थ्सꣳ सꣳ षट् थ्षट् थ्सम् प॑द्यन्ते पद्यन्ते॒ सꣳ षट् थ्षट् थ्सम् प॑द्यन्ते । \newline
20. सम् प॑द्यन्ते पद्यन्ते॒ सꣳ सम् प॑द्यन्ते॒ षट् थ्षट् प॑द्यन्ते॒ सꣳ सम् प॑द्यन्ते॒ षट् । \newline
21. प॒द्य॒न्ते॒ षट् थ्षट् प॑द्यन्ते पद्यन्ते॒ षड् वै वै षट् प॑द्यन्ते पद्यन्ते॒ षड् वै । \newline
22. षड् वै वै षट् थ्षड् वा ऋ॒तव॑ ऋ॒तवो॒ वै षट् थ्षड् वा ऋ॒तवः॑ । \newline
23. वा ऋ॒तव॑ ऋ॒तवो॒ वै वा ऋ॒तव॑ ऋ॒तुभिर्॑. ऋ॒तुभिर्॑. ऋ॒तवो॒ वै वा ऋ॒तव॑ ऋ॒तुभिः॑ । \newline
24. ऋ॒तव॑ ऋ॒तुभिर्॑. ऋ॒तुभिर्॑. ऋ॒तव॑ ऋ॒तव॑ ऋ॒तुभि॑ रे॒वैव र्‌तुभिर्॑. ऋ॒तव॑ ऋ॒तव॑ ऋ॒तुभि॑ रे॒व । \newline
25. ऋ॒तुभि॑ रे॒वैव र्‌तुभिर्॑. ऋ॒तुभि॑ रे॒वैन॑ मेन मे॒व र्‌तुभिर्॑. ऋ॒तुभि॑ रे॒वैन᳚म् । \newline
26. ऋ॒तुभि॒रित्यृ॒तु - भिः॒ । \newline
27. ए॒वैन॑ मेन मे॒वै वैनꣳ॑ शमयति शमय त्येन मे॒वै वैनꣳ॑ शमयति । \newline
28. ए॒नꣳ॒॒ श॒म॒य॒ति॒ श॒म॒य॒ त्ये॒न॒ मे॒नꣳ॒॒ श॒म॒य॒ति॒ यद् यच्छ॑मय त्येन मेनꣳ शमयति॒ यत् । \newline
29. श॒म॒य॒ति॒ यद् यच्छ॑मयति शमयति॒ यद॑नुपरि॒क्राम॑ मनुपरि॒क्रामं॒ ॅयच्छ॑मयति शमयति॒ यद॑नुपरि॒क्राम᳚म् । \newline
30. यद॑नुपरि॒क्राम॑ मनुपरि॒क्रामं॒ ॅयद् यद॑नुपरि॒क्राम॑म् जुहु॒याज् जु॑हु॒या द॑नुपरि॒क्रामं॒ ॅयद् यद॑नुपरि॒क्राम॑म् जुहु॒यात् । \newline
31. अ॒नु॒प॒रि॒क्राम॑म् जुहु॒याज् जु॑हु॒या द॑नुपरि॒क्राम॑ मनुपरि॒क्राम॑म् जुहु॒या द॑न्तरवचा॒रिण॑ मन्तरवचा॒रिण॑म् जुहु॒या द॑नुपरि॒क्राम॑ मनुपरि॒क्राम॑म् जुहु॒या द॑न्तरवचा॒रिण᳚म् । \newline
32. अ॒नु॒प॒रि॒क्राम॒मित्य॑नु - प॒रि॒क्राम᳚म् । \newline
33. जु॒हु॒या द॑न्तरवचा॒रिण॑ मन्तरवचा॒रिण॑म् जुहु॒याज् जु॑हु॒या द॑न्तरवचा॒रिणꣳ॑ रु॒द्रꣳ रु॒द्र म॑न्तरवचा॒रिण॑म् जुहु॒याज् जु॑हु॒या द॑न्तरवचा॒रिणꣳ॑ रु॒द्रम् । \newline
34. अ॒न्त॒र॒व॒चा॒रिणꣳ॑ रु॒द्रꣳ रु॒द्र म॑न्तरवचा॒रिण॑ मन्तरवचा॒रिणꣳ॑ रु॒द्रम् कु॑र्यात् कुर्याद् रु॒द्र म॑न्तरवचा॒रिण॑ मन्तरवचा॒रिणꣳ॑ रु॒द्रम् कु॑र्यात् । \newline
35. अ॒न्त॒र॒व॒चा॒रिण॒मित्य॑न्तः - अ॒व॒चा॒रिण᳚म् । \newline
36. रु॒द्रम् कु॑र्यात् कुर्याद् रु॒द्रꣳ रु॒द्रम् कु॑र्या॒ दथो॒ अथो॑ कुर्याद् रु॒द्रꣳ रु॒द्रम् कु॑र्या॒ दथो᳚ । \newline
37. कु॒र्या॒ दथो॒ अथो॑ कुर्यात् कुर्या॒ दथो॒ खलु॒ खल्वथो॑ कुर्यात् कुर्या॒ दथो॒ खलु॑ । \newline
38. अथो॒ खलु॒ खल्वथो॒ अथो॒ खल्वा॑हु राहुः॒ खल्वथो॒ अथो॒ खल्वा॑हुः । \newline
39. अथो॒ इत्यथो᳚ । \newline
40. खल्वा॑हु राहुः॒ खलु॒ खल्वा॑हुः॒ कस्या॒म् कस्या॑ माहुः॒ खलु॒ खल्वा॑हुः॒ कस्या᳚म् । \newline
41. आ॒हुः॒ कस्या॒म् कस्या॑ माहु राहुः॒ कस्यां᳚ ॅवा वा॒ कस्या॑ माहु राहुः॒ कस्यां᳚ ॅवा । \newline
42. कस्यां᳚ ॅवा वा॒ कस्या॒म् कस्यां॒ ॅवा ऽहाह॑ वा॒ कस्या॒म् कस्यां॒ ॅवा ऽह॑ । \newline
43. वा ऽहाह॑ वा॒ वा ऽह॑ दि॒शि दि॒श्यह॑ वा॒ वा ऽह॑ दि॒शि । \newline
44. अह॑ दि॒शि दि॒श्यहाह॑ दि॒शि रु॒द्रो रु॒द्रो दि॒श्यहाह॑ दि॒शि रु॒द्रः । \newline
45. दि॒शि रु॒द्रो रु॒द्रो दि॒शि दि॒शि रु॒द्रः कस्या॒म् कस्याꣳ॑ रु॒द्रो दि॒शि दि॒शि रु॒द्रः कस्या᳚म् । \newline
46. रु॒द्रः कस्या॒म् कस्याꣳ॑ रु॒द्रो रु॒द्रः कस्यां᳚ ॅवा वा॒ कस्याꣳ॑ रु॒द्रो रु॒द्रः कस्यां᳚ ॅवा । \newline
47. कस्यां᳚ ॅवा वा॒ कस्या॒म् कस्यां॒ ॅवेतीति॑ वा॒ कस्या॒म् कस्यां॒ ॅवेति॑ । \newline
48. वेतीति॑ वा॒ वेत्य॑नुपरि॒क्राम॑ मनुपरि॒क्राम॒ मिति॑ वा॒ वेत्य॑नुपरि॒क्राम᳚म् । \newline
49. इत्य॑नुपरि॒क्राम॑ मनुपरि॒क्राम॒ मिती त्य॑नुपरि॒क्राम॑ मे॒वै वानु॑परि॒क्राम॒ मिती त्य॑नुपरि॒क्राम॑ मे॒व । \newline
50. अ॒नु॒प॒रि॒क्राम॑ मे॒वै वानु॑परि॒क्राम॑ मनुपरि॒क्राम॑ मे॒व हो॑त॒व्यꣳ॑ होत॒व्य॑ मे॒वा नु॑परि॒क्राम॑ मनुपरि॒क्राम॑ मे॒व हो॑त॒व्य᳚म् । \newline
51. अ॒नु॒प॒रि॒क्राम॒मित्य॑नु - प॒रि॒क्राम᳚म् । \newline
52. ए॒व हो॑त॒व्यꣳ॑ होत॒व्य॑ मे॒वैव हो॑त॒व्य॑ मप॑रिवर्ग॒ मप॑रिवर्गꣳ होत॒व्य॑ मे॒वैव हो॑त॒व्य॑ मप॑रिवर्गम् । \newline
53. हो॒त॒व्य॑ मप॑रिवर्ग॒ मप॑रिवर्गꣳ होत॒व्यꣳ॑ होत॒व्य॑ मप॑रिवर्ग मे॒वैवा प॑रिवर्गꣳ होत॒व्यꣳ॑ होत॒व्य॑ मप॑रिवर्ग मे॒व । \newline
54. अप॑रिवर्ग मे॒वैवा प॑रिवर्ग॒ मप॑रिवर्ग मे॒वैन॑ मेन मे॒वा प॑रिवर्ग॒ मप॑रिवर्ग मे॒वैन᳚म् । \newline
55. अप॑रिवर्ग॒मित्यप॑रि - व॒र्ग॒म् । \newline
56. ए॒वैन॑ मेन मे॒वै वैनꣳ॑ शमयति शमय त्येन मे॒वै वैनꣳ॑ शमयति । \newline
57. ए॒नꣳ॒॒ श॒म॒य॒ति॒ श॒म॒य॒ त्ये॒न॒ मे॒नꣳ॒॒ श॒म॒य॒ त्ये॒ता ए॒ताः श॑मय त्येन मेनꣳ शमय त्ये॒ताः । \newline
58. श॒म॒य॒ त्ये॒ता ए॒ताः श॑मयति शमय त्ये॒ता वै वा ए॒ताः श॑मयति शमय त्ये॒ता वै । \newline
\pagebreak
\markright{ TS 5.4.3.5  \hfill https://www.vedavms.in \hfill}

\section{ TS 5.4.3.5 }

\textbf{TS 5.4.3.5 } \newline
\textbf{Samhita Paata} \newline

-तावै दे॒वताः᳚ सुव॒र्ग्या॑या उ॑त्त॒मास्ता यज॑मानं ॅवाचयति॒ ताभि॑रे॒वैनꣳ॑ सुव॒र्गं ॅलो॒कं ग॑मयति॒ यं द्वि॒ष्यात् तस्य॑ सञ्च॒रे प॑शू॒नां न्य॑स्ये॒द् यः प्र॑थ॒मः प॒शुर॑भि॒तिष्ठ॑ति॒ स आर्ति॒मार्च्छ॑ति ॥ \newline

\textbf{Pada Paata} \newline

ए॒ताः । वै । दे॒वताः᳚ । सु॒व॒र्ग्या॑ इति॑ सुवः - ग्याः᳚ । याः । उ॒त्त॒मा इत्यु॑त्-त॒माः । ताः । यज॑मानम् । वा॒च॒य॒ति॒ । ताभिः॑ । ए॒व । ए॒न॒म् । सु॒व॒र्गमिति॑ सुवः-गम् । लो॒कम् । ग॒म॒य॒ति॒ । यम् । द्वि॒ष्यात् । तस्य॑ । सं॒च॒र इति॑ सं - च॒रे । प॒शू॒नाम् । नीति॑ । अ॒स्ये॒त् । यः । प्र॒थ॒मः । प॒शुः । अ॒भि॒तिष्ठ॒तीत्य॑भि - तिष्ठ॑ति । सः । आर्ति᳚म् । एति॑ । ऋ॒च्छ॒ति॒ ॥  \newline


\textbf{Krama Paata} \newline

ए॒ता वै । वै दे॒वताः᳚ । दे॒वताः᳚ सुव॒र्ग्याः᳚ । सु॒व॒र्ग्या॑ याः । सु॒व॒र्ग्या॑ इति॑ सुवः - ग्याः᳚ । या उ॑त्त॒माः । उ॒त्त॒मास्ताः । उ॒त्त॒मा इत्यु॑त् - त॒माः । ता यज॑मानम् । यज॑मानम् ॅवाचयति । वा॒च॒य॒ति॒ ताभिः॑ । ताभि॑रे॒व । ए॒वैन᳚म् । ए॒नꣳ॒॒ सु॒व॒र्गम् । सु॒व॒र्गम् ॅलो॒कम् । सु॒व॒र्गमिति॑ सुवः - गम् । लो॒कम् ग॑मयति । ग॒म॒य॒ति॒ यम् । यम् द्वि॒ष्यात् । द्वि॒ष्यात् तस्य॑ । तस्य॑ सञ्च॒रे । स॒ञ्च॒रे प॑शू॒नाम् । स॒ञ्च॒र इति॑ सम् - च॒रे । प॒शू॒नाम् नि । न्य॑स्येत् । अ॒स्ये॒द् यः । यः प्र॑थ॒मः । प्र॒थ॒मः प॒शुः । प॒शुर॑भि॒तिष्ठ॑ति । अ॒भि॒तिष्ठ॑ति॒ सः । अ॒भि॒तिष्ठ॒तीत्य॑भि - तिष्ठ॑ति । स आर्ति᳚म् । आर्ति॒मा । आर्च्छ॑ति । ऋ॒च्छ॒तीत्यृ॑च्छति । \newline

\textbf{Jatai Paata} \newline

1. ए॒ता वै वा ए॒ता ए॒ता वै । \newline
2. वै दे॒वता॑ दे॒वता॒ वै वै दे॒वताः᳚ । \newline
3. दे॒वताः᳚ सुव॒र्ग्याः᳚ सुव॒र्ग्या॑ दे॒वता॑ दे॒वताः᳚ सुव॒र्ग्याः᳚ । \newline
4. सु॒व॒र्ग्या॑ या याः सु॑व॒र्ग्याः᳚ सुव॒र्ग्या॑ याः । \newline
5. सु॒व॒र्ग्या॑ इति॑ सुवः - ग्याः᳚ । \newline
6. या उ॑त्त॒मा उ॑त्त॒मा या या उ॑त्त॒माः । \newline
7. उ॒त्त॒मा स्ता स्ता उ॑त्त॒मा उ॑त्त॒मा स्ताः । \newline
8. उ॒त्त॒मा इत्यु॑त् - त॒माः । \newline
9. ता यज॑मानं॒ ॅयज॑मान॒म् ता स्ता यज॑मानम् । \newline
10. यज॑मानं ॅवाचयति वाचयति॒ यज॑मानं॒ ॅयज॑मानं ॅवाचयति । \newline
11. वा॒च॒य॒ति॒ ताभि॒ स्ताभि॑र् वाचयति वाचयति॒ ताभिः॑ । \newline
12. ताभि॑ रे॒वैव ताभि॒ स्ताभि॑ रे॒व । \newline
13. ए॒वैन॑ मेन मे॒वै वैन᳚म् । \newline
14. ए॒नꣳ॒॒ सु॒व॒र्गꣳ सु॑व॒र्ग मे॑न मेनꣳ सुव॒र्गम् । \newline
15. सु॒व॒र्गम् ॅलो॒कम् ॅलो॒कꣳ सु॑व॒र्गꣳ सु॑व॒र्गम् ॅलो॒कम् । \newline
16. सु॒व॒र्गमिति॑ सुवः - गम् । \newline
17. लो॒कम् ग॑मयति गमयति लो॒कम् ॅलो॒कम् ग॑मयति । \newline
18. ग॒म॒य॒ति॒ यं ॅयम् ग॑मयति गमयति॒ यम् । \newline
19. यम् द्वि॒ष्याद् द्वि॒ष्याद् यं ॅयम् द्वि॒ष्यात् । \newline
20. द्वि॒ष्यात् तस्य॒ तस्य॑ द्वि॒ष्याद् द्वि॒ष्यात् तस्य॑ । \newline
21. तस्य॑ सञ्च॒रे स॑ञ्च॒रे तस्य॒ तस्य॑ सञ्च॒रे । \newline
22. स॒ञ्च॒रे प॑शू॒नाम् प॑शू॒नाꣳ स॑ञ्च॒रे स॑ञ्च॒रे प॑शू॒नाम् । \newline
23. स॒ञ्च॒र इति॑ सं - च॒रे । \newline
24. प॒शू॒नाम् नि नि प॑शू॒नाम् प॑शू॒नाम् नि । \newline
25. न्य॑स्ये दस्ये॒न् नि न्य॑स्येत् । \newline
26. अ॒स्ये॒द् यो यो᳚ ऽस्ये दस्ये॒द् यः । \newline
27. यः प्र॑थ॒मः प्र॑थ॒मो यो यः प्र॑थ॒मः । \newline
28. प्र॒थ॒मः प॒शुः प॒शुः प्र॑थ॒मः प्र॑थ॒मः प॒शुः । \newline
29. प॒शु र॑भि॒तिष्ठ॑ त्यभि॒तिष्ठ॑ति प॒शुः प॒शु र॑भि॒तिष्ठ॑ति । \newline
30. अ॒भि॒तिष्ठ॑ति॒ स सो॑ ऽभि॒तिष्ठ॑ त्यभि॒तिष्ठ॑ति॒ सः । \newline
31. अ॒भि॒तिष्ठ॒तीत्य॑भि - तिष्ठ॑ति । \newline
32. स आर्ति॒ मार्तिꣳ॒॒ स स आर्ति᳚म् । \newline
33. आर्ति॒ मा ऽऽर्ति॒ मार्ति॒ मा । \newline
34. आर्च्छ॑ त्यृच्छ॒ त्यार्च्छ॑ति । \newline
35. ऋ॒च्छ॒तीत्यृ॑च्छति । \newline

\textbf{Ghana Paata } \newline

1. ए॒ता वै वा ए॒ता ए॒ता वै दे॒वता॑ दे॒वता॒ वा ए॒ता ए॒ता वै दे॒वताः᳚ । \newline
2. वै दे॒वता॑ दे॒वता॒ वै वै दे॒वताः᳚ सुव॒र्ग्याः᳚ सुव॒र्ग्या॑ दे॒वता॒ वै वै दे॒वताः᳚ सुव॒र्ग्याः᳚ । \newline
3. दे॒वताः᳚ सुव॒र्ग्याः᳚ सुव॒र्ग्या॑ दे॒वता॑ दे॒वताः᳚ सुव॒र्ग्या॑ या याः सु॑व॒र्ग्या॑ दे॒वता॑ दे॒वताः᳚ सुव॒र्ग्या॑ याः । \newline
4. सु॒व॒र्ग्या॑ या याः सु॑व॒र्ग्याः᳚ सुव॒र्ग्या॑ या उ॑त्त॒मा उ॑त्त॒मा याः सु॑व॒र्ग्याः᳚ सुव॒र्ग्या॑ या उ॑त्त॒माः । \newline
5. सु॒व॒र्ग्या॑ इति॑ सुवः - ग्याः᳚ । \newline
6. या उ॑त्त॒मा उ॑त्त॒मा या या उ॑त्त॒मा स्ता स्ता उ॑त्त॒मा या या उ॑त्त॒मा स्ताः । \newline
7. उ॒त्त॒मा स्ता स्ता उ॑त्त॒मा उ॑त्त॒मा स्ता यज॑मानं॒ ॅयज॑मान॒म् ता उ॑त्त॒मा उ॑त्त॒मा स्ता यज॑मानम् । \newline
8. उ॒त्त॒मा इत्यु॑त् - त॒माः । \newline
9. ता यज॑मानं॒ ॅयज॑मान॒म् ता स्ता यज॑मानं ॅवाचयति वाचयति॒ यज॑मान॒म् ता स्ता यज॑मानं ॅवाचयति । \newline
10. यज॑मानं ॅवाचयति वाचयति॒ यज॑मानं॒ ॅयज॑मानं ॅवाचयति॒ ताभि॒ स्ताभि॑र् वाचयति॒ यज॑मानं॒ ॅयज॑मानं ॅवाचयति॒ ताभिः॑ । \newline
11. वा॒च॒य॒ति॒ ताभि॒ स्ताभि॑र् वाचयति वाचयति॒ ताभि॑ रे॒वैव ताभि॑र् वाचयति वाचयति॒ ताभि॑ रे॒व । \newline
12. ताभि॑ रे॒वैव ताभि॒ स्ताभि॑ रे॒वैन॑ मेन मे॒व ताभि॒ स्ताभि॑ रे॒वैन᳚म् । \newline
13. ए॒वैन॑ मेन मे॒वै वैनꣳ॑ सुव॒र्गꣳ सु॑व॒र्ग मे॑न मे॒वै वैनꣳ॑ सुव॒र्गम् । \newline
14. ए॒नꣳ॒॒ सु॒व॒र्गꣳ सु॑व॒र्ग मे॑न मेनꣳ सुव॒र्गम् ॅलो॒कम् ॅलो॒कꣳ सु॑व॒र्ग मे॑न मेनꣳ सुव॒र्गम् ॅलो॒कम् । \newline
15. सु॒व॒र्गम् ॅलो॒कम् ॅलो॒कꣳ सु॑व॒र्गꣳ सु॑व॒र्गम् ॅलो॒कम् ग॑मयति गमयति लो॒कꣳ सु॑व॒र्गꣳ सु॑व॒र्गम् ॅलो॒कम् ग॑मयति । \newline
16. सु॒व॒र्गमिति॑ सुवः - गम् । \newline
17. लो॒कम् ग॑मयति गमयति लो॒कम् ॅलो॒कम् ग॑मयति॒ यं ॅयम् ग॑मयति लो॒कम् ॅलो॒कम् ग॑मयति॒ यम् । \newline
18. ग॒म॒य॒ति॒ यं ॅयम् ग॑मयति गमयति॒ यम् द्वि॒ष्याद् द्वि॒ष्याद् यम् ग॑मयति गमयति॒ यम् द्वि॒ष्यात् । \newline
19. यम् द्वि॒ष्याद् द्वि॒ष्याद् यं ॅयम् द्वि॒ष्यात् तस्य॒ तस्य॑ द्वि॒ष्याद् यं ॅयम् द्वि॒ष्यात् तस्य॑ । \newline
20. द्वि॒ष्यात् तस्य॒ तस्य॑ द्वि॒ष्याद् द्वि॒ष्यात् तस्य॑ सञ्च॒रे स॑ञ्च॒रे तस्य॑ द्वि॒ष्याद् द्वि॒ष्यात् तस्य॑ सञ्च॒रे । \newline
21. तस्य॑ सञ्च॒रे स॑ञ्च॒रे तस्य॒ तस्य॑ सञ्च॒रे प॑शू॒नाम् प॑शू॒नाꣳ स॑ञ्च॒रे तस्य॒ तस्य॑ सञ्च॒रे प॑शू॒नाम् । \newline
22. स॒ञ्च॒रे प॑शू॒नाम् प॑शू॒नाꣳ स॑ञ्च॒रे स॑ञ्च॒रे प॑शू॒नाम् नि नि प॑शू॒नाꣳ स॑ञ्च॒रे स॑ञ्च॒रे प॑शू॒नाम् नि । \newline
23. स॒ञ्च॒र इति॑ सं - च॒रे । \newline
24. प॒शू॒नाम् नि नि प॑शू॒नाम् प॑शू॒नाम् न्य॑स्ये दस्ये॒न् नि प॑शू॒नाम् प॑शू॒नाम् न्य॑स्येत् । \newline
25. न्य॑स्ये दस्ये॒न् नि न्य॑स्ये॒द् यो यो᳚ ऽस्ये॒न् नि न्य॑स्ये॒द् यः । \newline
26. अ॒स्ये॒द् यो यो᳚ ऽस्ये दस्ये॒द् यः प्र॑थ॒मः प्र॑थ॒मो यो᳚ ऽस्ये दस्ये॒द् यः प्र॑थ॒मः । \newline
27. यः प्र॑थ॒मः प्र॑थ॒मो यो यः प्र॑थ॒मः प॒शुः प॒शुः प्र॑थ॒मो यो यः प्र॑थ॒मः प॒शुः । \newline
28. प्र॒थ॒मः प॒शुः प॒शुः प्र॑थ॒मः प्र॑थ॒मः प॒शु र॑भि॒तिष्ठ॑ त्यभि॒तिष्ठ॑ति प॒शुः प्र॑थ॒मः प्र॑थ॒मः प॒शु र॑भि॒तिष्ठ॑ति । \newline
29. प॒शु र॑भि॒तिष्ठ॑ त्यभि॒तिष्ठ॑ति प॒शुः प॒शु र॑भि॒तिष्ठ॑ति॒ स सो॑ ऽभि॒तिष्ठ॑ति प॒शुः प॒शु र॑भि॒तिष्ठ॑ति॒ सः । \newline
30. अ॒भि॒तिष्ठ॑ति॒ स सो॑ ऽभि॒तिष्ठ॑ त्यभि॒तिष्ठ॑ति॒ स आर्ति॒ मार्तिꣳ॒॒ सो॑ ऽभि॒तिष्ठ॑ त्यभि॒तिष्ठ॑ति॒ स आर्ति᳚म् । \newline
31. अ॒भि॒तिष्ठ॒तीत्य॑भि - तिष्ठ॑ति । \newline
32. स आर्ति॒ मार्तिꣳ॒॒ स स आर्ति॒ माऽऽर्तिꣳ॒॒ स स आर्ति॒ मा । \newline
33. आर्ति॒ मा ऽऽर्ति॒ मार्ति॒ मार्च्छ॑ त्यृच्छ॒ त्याऽऽर्ति॒ मार्ति॒ मार्च्छ॑ति । \newline
34. आर्च्छ॑ त्यृच्छ॒ त्यार्च्छ॑ति । \newline
35. ऋ॒च्छ॒तीत्यृ॑च्छति । \newline
\pagebreak
\markright{ TS 5.4.4.1  \hfill https://www.vedavms.in \hfill}

\section{ TS 5.4.4.1 }

\textbf{TS 5.4.4.1 } \newline
\textbf{Samhita Paata} \newline

अश्म॒न्नूर्ज॒मिति॒ परि॑ षिञ्चति मा॒र्जय॑त्ये॒वैन॒मथो॑ त॒र्पय॑त्ये॒व स ए॑नं तृ॒प्तो ऽक्षु॑द्ध्य॒-न्नशो॑च-न्न॒मुष्मि॑न् ॅलो॒क उप॑ तिष्ठते॒ तृप्य॑ति प्र॒जया॑ प॒शुभि॒र्य ए॒वं ॅवेद॒ तां न॒ इष॒मूर्जं॑ धत्त मरुतः सꣳररा॒णा इत्या॒हान्नं॒ ॅवा ऊर्गन्नं॑ म॒रुतोऽन्न॑मे॒वाव॑ रु॒न्धे ऽश्मꣳ॑स्ते॒ क्षुद॒मुं ते॒ शु - [  ] \newline

\textbf{Pada Paata} \newline

अश्मन्न्॑ । ऊर्ज᳚म् । इति॑ । परीति॑ । सि॒ञ्च॒ति॒ । मा॒र्जय॑ति । ए॒व । ए॒न॒म् । अथो॒ इति॑ । त॒र्पय॑ति । ए॒व । सः । ए॒न॒म् । तृ॒प्तः । अक्षु॑द्ध्यन्न् । अशो॑चन्न् । अ॒मुष्मिन्न्॑ । लो॒के । उपेति॑ । ति॒ष्ठ॒ते॒ । तृप्य॑ति । प्र॒जयेति॑ प्र - जया᳚ । प॒शुभि॒रिति॑ प॒शु - भिः॒ । यः । ए॒वम् । वेद॑ । ताम् । नः॒ । इष᳚म् । ऊर्ज᳚म् । ध॒त्त॒ । म॒रु॒तः॒ । सꣳ॒॒र॒रा॒णा इति॑ सं - र॒रा॒णाः । इति॑ । आ॒ह॒ । अन्न᳚म् । वै । ऊर्क् । अन्न᳚म् । म॒रुतः॑ । अन्न᳚म् । ए॒व । अवेति॑ । रु॒न्धे॒ । अश्मन्न्॑ । ते॒ । क्षुत् । अ॒मुम् । ते॒ । शुक् ।  \newline


\textbf{Krama Paata} \newline

अश्म॒न्नूर्ज᳚म् । ऊर्ज॒मिति॑ । इति॒ परि॑ । परि॑ षिञ्चति । सि॒ञ्च॒ति॒ मा॒र्जय॑ति । मा॒र्जय॑त्ये॒व । ए॒वैन᳚म् । ए॒न॒मथो᳚ । अथो॑ त॒र्पय॑ति । अथो॒ इत्यथो᳚ । त॒र्पय॑त्ये॒व । ए॒व सः । स ए॑नम् । ए॒न॒म् तृ॒प्तः । तृ॒प्तोऽक्षु॑द्ध्यन्न् । अक्षु॑द्ध्य॒न्नशो॑चन्न् । अशो॑चन्न॒मुष्मिन्न्॑ । अ॒मुष्मि॑न् ॅलो॒के । लो॒क उप॑ । उप॑ तिष्ठते । ति॒ष्ठ॒ते॒ तृप्य॑ति । तृप्य॑ति प्र॒जया᳚ । प्र॒जया॑ प॒शुभिः॑ । प्र॒जेयेति॑ प्र - जया᳚ । प॒शुभि॒र् यः । प॒शुभि॒रिति॑ प॒शु - भिः॒ । य ए॒वम् । ए॒वम् ॅवेद॑ । वेद॒ ताम् । ताम् नः॑ । न॒ इष᳚म् । इष॒ मूर्ज᳚म् । ऊज॑म् धत्त । ध॒त्त॒ म॒रु॒तः॒ । म॒रु॒तः॒ सꣳ॒॒र॒रा॒णाः । सꣳ॒॒र॒रा॒णा इति॑ । सꣳ॒॒र॒रा॒णा इति॑ सम् - र॒रा॒णाः । इत्या॑ह । आ॒हान्न᳚म् । अन्न॒म् ॅवै । वा ऊर्क् । ऊर्गन्न᳚म् । अन्न॑म् म॒रुतः॑ । म॒रुतोऽन्न᳚म् । अन्न॑मे॒व । ए॒वाव॑ । अव॑ रुन्धे । रु॒न्धेऽश्मन्न्॑ । अश्मꣳ॑स्ते । ते॒ क्षुत् । क्षुद॒मुम् । अ॒मुम् ते᳚ । ते॒ शुक् । शुगृ॑च्छतु \newline

\textbf{Jatai Paata} \newline

1. अश्म॒न् नूर्ज॒ मूर्ज॒ मश्म॒न् नश्म॒न् नूर्ज᳚म् । \newline
2. ऊर्ज॒ मिती त्यूर्ज॒ मूर्ज॒ मिति॑ । \newline
3. इति॒ परि॒ परी तीति॒ परि॑ । \newline
4. परि॑ षिञ्चति सिञ्चति॒ परि॒ परि॑ षिञ्चति । \newline
5. सि॒ञ्च॒ति॒ मा॒र्जय॑ति मा॒र्जय॑ति सिञ्चति सिञ्चति मा॒र्जय॑ति । \newline
6. मा॒र्जय॑ त्ये॒वैव मा॒र्जय॑ति मा॒र्जय॑ त्ये॒व । \newline
7. ए॒वैन॑ मेन मे॒वै वैन᳚म् । \newline
8. ए॒न॒ मथो॒ अथो॑ एन मेन॒ मथो᳚ । \newline
9. अथो॑ त॒र्पय॑ति त॒र्पय॒ त्यथो॒ अथो॑ त॒र्पय॑ति । \newline
10. अथो॒ इत्यथो᳚ । \newline
11. त॒र्पय॑ त्ये॒वैव त॒र्पय॑ति त॒र्पय॑ त्ये॒व । \newline
12. ए॒व स स ए॒वैव सः । \newline
13. स ए॑न मेनꣳ॒॒ स स ए॑नम् । \newline
14. ए॒न॒म् तृ॒प्त स्तृ॒प्त ए॑न मेनम् तृ॒प्तः । \newline
15. तृ॒प्तो ऽक्षु॑द्ध्य॒न् नक्षु॑द्ध्यन् तृ॒प्त स्तृ॒प्तो ऽक्षु॑द्ध्यन्न् । \newline
16. अक्षु॑द्ध्य॒न् नशो॑च॒न् नशो॑च॒न् नक्षु॑द्ध्य॒न् नक्षु॑द्ध्य॒न् नशो॑चन्न् । \newline
17. अशो॑चन् न॒मुष्मि॑न् न॒मुष्मि॒न् नशो॑च॒न् नशो॑चन् न॒मुष्मिन्न्॑ । \newline
18. अ॒मुष्मि॑न् ॅलो॒के लो॒के॑ ऽमुष्मि॑न् न॒मुष्मि॑न् ॅलो॒के । \newline
19. लो॒क उपोप॑ लो॒के लो॒क उप॑ । \newline
20. उप॑ तिष्ठते तिष्ठत॒ उपोप॑ तिष्ठते । \newline
21. ति॒ष्ठ॒ते॒ तृप्य॑ति॒ तृप्य॑ति तिष्ठते तिष्ठते॒ तृप्य॑ति । \newline
22. तृप्य॑ति प्र॒जया᳚ प्र॒जया॒ तृप्य॑ति॒ तृप्य॑ति प्र॒जया᳚ । \newline
23. प्र॒जया॑ प॒शुभिः॑ प॒शुभिः॑ प्र॒जया᳚ प्र॒जया॑ प॒शुभिः॑ । \newline
24. प्र॒जयेति॑ प्र - जया᳚ । \newline
25. प॒शुभि॒र् यो यः प॒शुभिः॑ प॒शुभि॒र् यः । \newline
26. प॒शुभि॒रिति॑ प॒शु - भिः॒ । \newline
27. य ए॒व मे॒वं ॅयो य ए॒वम् । \newline
28. ए॒वं ॅवेद॒ वेदै॒व मे॒वं ॅवेद॑ । \newline
29. वेद॒ ताम् तां ॅवेद॒ वेद॒ ताम् । \newline
30. ताम् नो॑ न॒ स्ताम् ताम् नः॑ । \newline
31. न॒ इष॒ मिष॑म् नो न॒ इष᳚म् । \newline
32. इष॒ मूर्ज॒ मूर्ज॒ मिष॒ मिष॒ मूर्ज᳚म् । \newline
33. ऊर्ज॑म् धत्त ध॒त्तोर्ज॒ मूर्ज॑म् धत्त । \newline
34. ध॒त्त॒ म॒रु॒तो॒ म॒रु॒तो॒ ध॒त्त॒ ध॒त्त॒ म॒रु॒तः॒ । \newline
35. म॒रु॒तः॒ सꣳ॒॒र॒रा॒णाः सꣳ॑ररा॒णा म॑रुतो मरुतः सꣳररा॒णाः । \newline
36. सꣳ॒॒र॒रा॒णा इतीति॑ सꣳररा॒णाः सꣳ॑ररा॒णा इति॑ । \newline
37. सꣳ॒॒र॒रा॒णा इति॑ सं - र॒रा॒णाः । \newline
38. इत्या॑हा॒हे तीत्या॑ह । \newline
39. आ॒हान्न॒ मन्न॑ माहा॒ हान्न᳚म् । \newline
40. अन्नं॒ ॅवै वा अन्न॒ मन्नं॒ ॅवै । \newline
41. वा ऊर् गूर्ग् वै वा ऊर्क् । \newline
42. ऊर्गन्न॒ मन्न॒ मूर् गूर् गन्न᳚म् । \newline
43. अन्न॑म् म॒रुतो॑ म॒रुतो ऽन्न॒ मन्न॑म् म॒रुतः॑ । \newline
44. म॒रुतो ऽन्न॒ मन्न॑म् म॒रुतो॑ म॒रुतो ऽन्न᳚म् । \newline
45. अन्न॑ मे॒वैवान्न॒ मन्न॑ मे॒व । \newline
46. ए॒वावा वै॒वै वाव॑ । \newline
47. अव॑ रुन्धे रु॒न्धे ऽवाव॑ रुन्धे । \newline
48. रु॒न्धे ऽश्म॒न् नश्म॑न् रुन्धे रु॒न्धे ऽश्मन्न्॑ । \newline
49. अश्मꣳ॑ स्ते ते॒ अश्म॒न् नश्मꣳ॑ स्ते । \newline
50. ते॒ क्षुत् क्षुत् ते॑ ते॒ क्षुत् । \newline
51. क्षुद॒मु म॒मुम् क्षुत् क्षुद॒मुम् । \newline
52. अ॒मुम् ते॑ ते अ॒मु म॒मुम् ते᳚ । \newline
53. ते॒ शुक् छुक् ते॑ ते॒ शुक् । \newline
54. शुगृ॑च्छत् वृच्छतु॒ शुक् छुगृ॑च्छतु । \newline

\textbf{Ghana Paata } \newline

1. अश्म॒न् नूर्ज॒ मूर्ज॒ मश्म॒न् नश्म॒न् नूर्ज॒ मिती त्यूर्ज॒ मश्म॒न् नश्म॒न् नूर्ज॒ मिति॑ । \newline
2. ऊर्ज॒ मिती त्यूर्ज॒ मूर्ज॒ मिति॒ परि॒ परी त्यूर्ज॒ मूर्ज॒ मिति॒ परि॑ । \newline
3. इति॒ परि॒ परीतीति॒ परि॑ षिञ्चति सिञ्चति॒ परीतीति॒ परि॑ षिञ्चति । \newline
4. परि॑ षिञ्चति सिञ्चति॒ परि॒ परि॑ षिञ्चति मा॒र्जय॑ति मा॒र्जय॑ति सिञ्चति॒ परि॒ परि॑ षिञ्चति मा॒र्जय॑ति । \newline
5. सि॒ञ्च॒ति॒ मा॒र्जय॑ति मा॒र्जय॑ति सिञ्चति सिञ्चति मा॒र्जय॑ त्ये॒वैव मा॒र्जय॑ति सिञ्चति सिञ्चति मा॒र्जय॑ त्ये॒व । \newline
6. मा॒र्जय॑त्ये॒ वैव मा॒र्जय॑ति मा॒र्जय॑ त्ये॒वैन॑ मेन मे॒व मा॒र्जय॑ति मा॒र्जय॑ त्ये॒वैन᳚म् । \newline
7. ए॒वैन॑ मेन मे॒वै वैन॒ मथो॒ अथो॑ एन मे॒वै वैन॒ मथो᳚ । \newline
8. ए॒न॒ मथो॒ अथो॑ एन मेन॒ मथो॑ त॒र्पय॑ति त॒र्पय॒ त्यथो॑ एन मेन॒ मथो॑ त॒र्पय॑ति । \newline
9. अथो॑ त॒र्पय॑ति त॒र्पय॒ त्यथो॒ अथो॑ त॒र्पय॑ त्ये॒वैव त॒र्पय॒ त्यथो॒ अथो॑ त॒र्पय॑ त्ये॒व । \newline
10. अथो॒ इत्यथो᳚ । \newline
11. त॒र्पय॑ त्ये॒वैव त॒र्पय॑ति त॒र्पय॑ त्ये॒व स स ए॒व त॒र्पय॑ति त॒र्पय॑ त्ये॒व सः । \newline
12. ए॒व स स ए॒वैव स ए॑न मेनꣳ॒॒ स ए॒वैव स ए॑नम् । \newline
13. स ए॑न मेनꣳ॒॒ स स ए॑नम् तृ॒प्त स्तृ॒प्त ए॑नꣳ॒॒ स स ए॑नम् तृ॒प्तः । \newline
14. ए॒न॒म् तृ॒प्त स्तृ॒प्त ए॑न मेनम् तृ॒प्तो ऽक्षु॑द्ध्य॒न् नक्षु॑द्ध्यन् तृ॒प्त ए॑न मेनम् तृ॒प्तो ऽक्षु॑द्ध्यन्न् । \newline
15. तृ॒प्तो ऽक्षु॑द्ध्य॒न् नक्षु॑द्ध्यन् तृ॒प्त स्तृ॒प्तो ऽक्षु॑द्ध्य॒न् नशो॑च॒न् नशो॑च॒न् नक्षु॑द्ध्यन् तृ॒प्त स्तृ॒प्तो ऽक्षु॑द्ध्य॒न् नशो॑चन्न् । \newline
16. अक्षु॑द्ध्य॒न् नशो॑च॒न् नशो॑च॒न् नक्षु॑द्ध्य॒न् नक्षु॑द्ध्य॒न् नशो॑चन् न॒मुष्मि॑न् न॒मुष्मि॒न् नशो॑च॒न् नक्षु॑द्ध्य॒न् नक्षु॑द्ध्य॒न् नशो॑चन् न॒मुष्मिन्न्॑ । \newline
17. अशो॑चन् न॒मुष्मि॑न् न॒मुष्मि॒न् नशो॑च॒न् नशो॑चन् न॒मुष्मि॑न् ॅलो॒के लो॒के॑ ऽमुष्मि॒न् नशो॑च॒न् नशो॑चन् न॒मुष्मि॑न् ॅलो॒के । \newline
18. अ॒मुष्मि॑न् ॅलो॒के लो॒के॑ ऽमुष्मि॑न् न॒मुष्मि॑न् ॅलो॒क उपोप॑ लो॒के॑ ऽमुष्मि॑न् न॒मुष्मि॑न् ॅलो॒क उप॑ । \newline
19. लो॒क उपोप॑ लो॒के लो॒क उप॑ तिष्ठते तिष्ठत॒ उप॑ लो॒के लो॒क उप॑ तिष्ठते । \newline
20. उप॑ तिष्ठते तिष्ठत॒ उपोप॑ तिष्ठते॒ तृप्य॑ति॒ तृप्य॑ति तिष्ठत॒ उपोप॑ तिष्ठते॒ तृप्य॑ति । \newline
21. ति॒ष्ठ॒ते॒ तृप्य॑ति॒ तृप्य॑ति तिष्ठते तिष्ठते॒ तृप्य॑ति प्र॒जया᳚ प्र॒जया॒ तृप्य॑ति तिष्ठते तिष्ठते॒ तृप्य॑ति प्र॒जया᳚ । \newline
22. तृप्य॑ति प्र॒जया᳚ प्र॒जया॒ तृप्य॑ति॒ तृप्य॑ति प्र॒जया॑ प॒शुभिः॑ प॒शुभिः॑ प्र॒जया॒ तृप्य॑ति॒ तृप्य॑ति प्र॒जया॑ प॒शुभिः॑ । \newline
23. प्र॒जया॑ प॒शुभिः॑ प॒शुभिः॑ प्र॒जया᳚ प्र॒जया॑ प॒शुभि॒र् यो यः प॒शुभिः॑ प्र॒जया᳚ प्र॒जया॑ प॒शुभि॒र् यः । \newline
24. प्र॒जयेति॑ प्र - जया᳚ । \newline
25. प॒शुभि॒र् यो यः प॒शुभिः॑ प॒शुभि॒र् य ए॒व मे॒वं ॅयः प॒शुभिः॑ प॒शुभि॒र् य ए॒वम् । \newline
26. प॒शुभि॒रिति॑ प॒शु - भिः॒ । \newline
27. य ए॒व मे॒वं ॅयो य ए॒वं ॅवेद॒ वेदै॒वं ॅयो य ए॒वं ॅवेद॑ । \newline
28. ए॒वं ॅवेद॒ वेदै॒व मे॒वं ॅवेद॒ ताम् तां ॅवेदै॒व मे॒वं ॅवेद॒ ताम् । \newline
29. वेद॒ ताम् तां ॅवेद॒ वेद॒ ताम् नो॑ न॒ स्तां ॅवेद॒ वेद॒ ताम् नः॑ । \newline
30. ताम् नो॑ न॒ स्ताम् ताम् न॒ इष॒ मिष॑म् न॒ स्ताम् ताम् न॒ इष᳚म् । \newline
31. न॒ इष॒ मिष॑म् नो न॒ इष॒ मूर्ज॒ मूर्ज॒ मिष॑म् नो न॒ इष॒ मूर्ज᳚म् । \newline
32. इष॒ मूर्ज॒ मूर्ज॒ मिष॒ मिष॒ मूर्ज॑म् धत्त ध॒त्तोर्ज॒ मिष॒ मिष॒ मूर्ज॑म् धत्त । \newline
33. ऊर्ज॑म् धत्त ध॒त्तोर्ज॒ मूर्ज॑म् धत्त मरुतो मरुतो ध॒त्तोर्ज॒ मूर्ज॑म् धत्त मरुतः । \newline
34. ध॒त्त॒ म॒रु॒तो॒ म॒रु॒तो॒ ध॒त्त॒ ध॒त्त॒ म॒रु॒तः॒ सꣳ॒॒र॒रा॒णाः सꣳ॑ररा॒णा म॑रुतो धत्त धत्त मरुतः सꣳररा॒णाः । \newline
35. म॒रु॒तः॒ सꣳ॒॒र॒रा॒णाः सꣳ॑ररा॒णा म॑रुतो मरुतः सꣳररा॒णा इतीति॑ सꣳररा॒णा म॑रुतो मरुतः सꣳररा॒णा इति॑ । \newline
36. सꣳ॒॒र॒रा॒णा इतीति॑ सꣳररा॒णाः सꣳ॑ररा॒णा इत्या॑हा॒हेति॑ सꣳररा॒णाः सꣳ॑ररा॒णा इत्या॑ह । \newline
37. सꣳ॒॒र॒रा॒णा इति॑ सं - र॒रा॒णाः । \newline
38. इत्या॑ हा॒हे तीत्या॒ हान्न॒ मन्न॑ मा॒हे तीत्या॒ हान्न᳚म् । \newline
39. आ॒हान्न॒ मन्न॑ माहा॒ हान्नं॒ ॅवै वा अन्न॑ माहा॒ हान्नं॒ ॅवै । \newline
40. अन्नं॒ ॅवै वा अन्न॒ मन्नं॒ ॅवा ऊर्गूर्ग् वा अन्न॒ मन्नं॒ ॅवा ऊर्क् । \newline
41. वा ऊर्गूर्ग् वै वा ऊर्गन्न॒ मन्न॒ मूर्ग् वै वा ऊर्गन्न᳚म् । \newline
42. ऊर्गन्न॒ मन्न॒ मूर् गूर् गन्न॑म् म॒रुतो॑ म॒रुतो ऽन्न॒ मूर् गूर् गन्न॑म् म॒रुतः॑ । \newline
43. अन्न॑म् म॒रुतो॑ म॒रुतो ऽन्न॒ मन्न॑म् म॒रुतो ऽन्न॒ मन्न॑म् म॒रुतो ऽन्न॒ मन्न॑म् म॒रुतो ऽन्न᳚म् । \newline
44. म॒रुतो ऽन्न॒ मन्न॑म् म॒रुतो॑ म॒रुतो ऽन्न॑ मे॒वै वान्न॑म् म॒रुतो॑ म॒रुतो ऽन्न॑ मे॒व । \newline
45. अन्न॑ मे॒वै वान्न॒ मन्न॑ मे॒वा वावै॒ वान्न॒ मन्न॑ मे॒वाव॑ । \newline
46. ए॒वावा वै॒वै वाव॑ रुन्धे रु॒न्धे ऽवै॒वै वाव॑ रुन्धे । \newline
47. अव॑ रुन्धे रु॒न्धे ऽवाव॑ रु॒न्धे ऽश्म॒न् नश्म॑न् रु॒न्धे ऽवाव॑ रु॒न्धे ऽश्मन्न्॑ । \newline
48. रु॒न्धे ऽश्म॒न् नश्म॑न् रुन्धे रु॒न्धे ऽश्मꣳ॑ स्ते ते॒ अश्म॑न् रुन्धे रु॒न्धे ऽश्मꣳ॑ स्ते । \newline
49. अश्मꣳ॑ स्ते ते॒ अश्म॒न् नश्मꣳ॑ स्ते॒ क्षुत् क्षुत् ते॒ अश्म॒न् नश्मꣳ॑ स्ते॒ क्षुत् । \newline
50. ते॒ क्षुत् क्षुत् ते॑ ते॒ क्षुद॒मु म॒मुम् क्षुत् ते॑ ते॒ क्षुद॒मुम् । \newline
51. क्षुद॒मु म॒मुम् क्षुत् क्षुद॒मुम् ते॑ ते अ॒मुम् क्षुत् क्षुद॒मुम् ते᳚ । \newline
52. अ॒मुम् ते॑ ते अ॒मु म॒मुम् ते॒ शुक् छुक् ते॑ अ॒मु म॒मुम् ते॒ शुक् । \newline
53. ते॒ शुक् छुक् ते॑ ते॒ शुगृ॑च्छ त्वृच्छतु॒ शुक् ते॑ ते॒ शुगृ॑च्छतु । \newline
54. शुगृ॑च्छ त्वृच्छतु॒ शुक् छुगृ॑च्छतु॒ यं ॅय मृ॑च्छतु॒ शुक् छुगृ॑च्छतु॒ यम् । \newline
\pagebreak
\markright{ TS 5.4.4.2  \hfill https://www.vedavms.in \hfill}

\section{ TS 5.4.4.2 }

\textbf{TS 5.4.4.2 } \newline
\textbf{Samhita Paata} \newline

-गृ॑च्छतु॒ यं द्वि॒ष्म इत्या॑ह॒ यमे॒व द्वेष्टि॒ तम॑स्य क्षु॒धा च॑ शु॒चा चा᳚र्पयति॒ त्रिः प॑रिषि॒ञ्चन् पर्ये॑ति त्रि॒वृद्वा अ॒ग्निर्यावा॑ने॒वा-ग्निस्तस्य॒ शुचꣳ॑ शमयति॒ त्रिः पुनः॒ पर्ये॑ति॒ षट् थ्सं प॑द्यन्ते॒ षड् वा ऋ॒तव॑ ऋ॒तुभि॑रे॒वास्य॒ शुचꣳ॑ शमयत्य॒पां ॅवा ए॒तत् पुष्पं॒ ॅयद्वे॑त॒सो॑ऽपाꣳ - [  ] \newline

\textbf{Pada Paata} \newline

ऋ॒च्छ॒तु॒ । यम् । द्वि॒ष्मः । इति॑ । आ॒ह॒ । यम् । ए॒व । द्वेष्टि॑ । तम् । अ॒स्य॒ । क्षु॒धा । च॒ । शु॒चा । च॒ । अ॒र्प॒य॒ति॒ । त्रिः । प॒रि॒षि॒ञ्चन्निति॑ परि - सि॒ञ्चन्न् । परीति॑ । ए॒ति॒ । त्रि॒वृदिति॑ त्रि-वृत् । वै । अ॒ग्निः । यावान्॑ । ए॒व । अ॒ग्निः । तस्य॑ । शुच᳚म् । श॒म॒य॒ति॒ । त्रिः । पुनः॑ । परीति॑ । ए॒ति॒ । षट् । समिति॑ । प॒द्य॒न्ते॒ । षट् । वै । ऋ॒तवः॑ । ऋ॒तुभि॒रित्यृ॒तु-भिः॒। ए॒व । अ॒स्य॒ । शुच᳚म् । श॒म॒य॒ति॒ । अ॒पाम् । वै । ए॒तत् । पुष्प᳚म् । यत् । वे॒त॒सः । अ॒पाम् ।  \newline


\textbf{Krama Paata} \newline

ऋ॒च्छ॒तु॒ यम् । यम् द्वि॒ष्मः । द्वि॒ष्म इति॑ । इत्या॑ह । आ॒ह॒ यम् । यमे॒व । ए॒व द्वेष्टि॑ । द्वेष्टि॒ तम् । तम॑स्य । अ॒स्य॒ क्षु॒धा । क्षु॒धा च॑ । च॒ शु॒चा । शु॒चा च॑ । चा॒र्प॒य॒ति॒ । अ॒र्प॒य॒ति॒ त्रिः । त्रिः प॑रिषि॒ञ्चन्न् । प॒रि॒षि॒ञ्चन् परि॑ । प॒रि॒षि॒ञ्चन्निति॑ परि - सि॒ञ्चन्न् । पर्ये॑ति । ए॒ति॒ त्रि॒वृत् । त्रि॒वृद् वै । त्रि॒वृदिति॑ त्रि - वृत् । वा अ॒ग्निः । अ॒ग्निर् यावान्॑ । यावा॑ने॒व । ए॒वाग्निः । अ॒ग्निस्तस्य॑ । तस्य॒ शुच᳚म् । शुचꣳ॑ शमयति । श॒म॒य॒ति॒ त्रिः । त्रिः पुनः॑ । पुनः॒ परि॑ । पर्ये॑ति । ए॒ति॒ षट् । षट्थ् सम् । सम् प॑द्यन्ते । प॒द्य॒न्ते॒ षट् । षड् वै । वा ऋ॒तवः॑ । ऋ॒तव॑ ऋ॒तुभिः॑ । ऋ॒तुभि॑रे॒व । ऋ॒तुभि॒रित्यृ॒तु - भिः॒ । ए॒वास्य॑ । अ॒स्य॒ शुच᳚म् । शुचꣳ॑ शमयति । श॒म॒य॒त्य॒पाम् । अ॒पाम् ॅवै । वा ए॒तत् । ए॒तत् पुष्प᳚म् । पुष्प॒म् ॅयत् । यद् वे॑त॒सः । वे॒त॒सो॑ऽपाम् । अ॒पाꣳ शरः॑ \newline

\textbf{Jatai Paata} \newline

1. ऋ॒च्छ॒तु॒ यं ॅय मृ॑च्छ त्वृच्छतु॒ यम् । \newline
2. यम् द्वि॒ष्मो द्वि॒ष्मो यं ॅयम् द्वि॒ष्मः । \newline
3. द्वि॒ष्म इतीति॑ द्वि॒ष्मो द्वि॒ष्म इति॑ । \newline
4. इत्या॑हा॒हे तीत्या॑ह । \newline
5. आ॒ह॒ यं ॅय मा॑हाह॒ यम् । \newline
6. य मे॒वैव यं ॅय मे॒व । \newline
7. ए॒व द्वेष्टि॒ द्वेष्ट्ये॒ वैव द्वेष्टि॑ । \newline
8. द्वेष्टि॒ तम् तम् द्वेष्टि॒ द्वेष्टि॒ तम् । \newline
9. त म॑स्यास्य॒ तम् त म॑स्य । \newline
10. अ॒स्य॒ क्षु॒धा क्षु॒धा ऽस्या᳚स्य क्षु॒धा । \newline
11. क्षु॒धा च॑ च क्षु॒धा क्षु॒धा च॑ । \newline
12. च॒ शु॒चा शु॒चा च॑ च शु॒चा । \newline
13. शु॒चा च॑ च शु॒चा शु॒चा च॑ । \newline
14. चा॒र्प॒य॒ त्य॒र्प॒य॒ति॒ च॒ चा॒र्प॒य॒ति॒ । \newline
15. अ॒र्प॒य॒ति॒ त्रि स्त्रि र॑र्पय त्यर्पयति॒ त्रिः । \newline
16. त्रिः प॑रिषि॒ञ्चन् प॑रिषि॒ञ्चन् त्रि स्त्रिः प॑रिषि॒ञ्चन्न् । \newline
17. प॒रि॒षि॒ञ्चन् परि॒ परि॑ परिषि॒ञ्चन् प॑रिषि॒ञ्चन् परि॑ । \newline
18. प॒रि॒षि॒ञ्चन्निति॑ परि - सि॒ञ्चन्न् । \newline
19. पर्ये᳚ त्येति॒ परि॒ पर्ये॑ति । \newline
20. ए॒ति॒ त्रि॒वृत् त्रि॒वृ दे᳚त्येति त्रि॒वृत् । \newline
21. त्रि॒वृद् वै वै त्रि॒वृत् त्रि॒वृद् वै । \newline
22. त्रि॒वृदिति॑ त्रि - वृत् । \newline
23. वा अ॒ग्नि र॒ग्निर् वै वा अ॒ग्निः । \newline
24. अ॒ग्निर् यावा॒न्॒. यावा॑ न॒ग्नि र॒ग्निर् यावान्॑ । \newline
25. यावा॑ ने॒वैव यावा॒न्॒. यावा॑ ने॒व । \newline
26. ए॒वाग्नि र॒ग्नि रे॒वै वाग्निः । \newline
27. अ॒ग्नि स्तस्य॒ तस्या॒ ग्नि र॒ग्नि स्तस्य॑ । \newline
28. तस्य॒ शुचꣳ॒॒ शुच॒म् तस्य॒ तस्य॒ शुच᳚म् । \newline
29. शुचꣳ॑ शमयति शमयति॒ शुचꣳ॒॒ शुचꣳ॑ शमयति । \newline
30. श॒म॒य॒ति॒ त्रि स्त्रिः श॑मयति शमयति॒ त्रिः । \newline
31. त्रिः पुनः॒ पुन॒ स्त्रि स्त्रिः पुनः॑ । \newline
32. पुनः॒ परि॒ परि॒ पुनः॒ पुनः॒ परि॑ । \newline
33. पर्ये᳚ त्येति॒ परि॒ पर्ये॑ति । \newline
34. ए॒ति॒ षट् थ्षडे᳚ त्येति॒ षट् । \newline
35. षट् थ्सꣳ सꣳ षट् थ्षट् थ्सम् । \newline
36. सम् प॑द्यन्ते पद्यन्ते॒ सꣳ सम् प॑द्यन्ते । \newline
37. प॒द्य॒न्ते॒ षट् थ्षट् प॑द्यन्ते पद्यन्ते॒ षट् । \newline
38. षड् वै वै षट् थ्षड् वै । \newline
39. वा ऋ॒तव॑ ऋ॒तवो॒ वै वा ऋ॒तवः॑ । \newline
40. ऋ॒तव॑ ऋ॒तुभिर्॑. ऋ॒तुभिर्॑. ऋ॒तव॑ ऋ॒तव॑ ऋ॒तुभिः॑ । \newline
41. ऋ॒तुभि॑ रे॒वैव र्‌तुभिर्॑. ऋ॒तुभि॑ रे॒व । \newline
42. ऋ॒तुभि॒रित्यृ॒तु - भिः॒ । \newline
43. ए॒वास्या᳚ स्यै॒वै वास्य॑ । \newline
44. अ॒स्य॒ शुचꣳ॒॒ शुच॑ मस्यास्य॒ शुच᳚म् । \newline
45. शुचꣳ॑ शमयति शमयति॒ शुचꣳ॒॒ शुचꣳ॑ शमयति । \newline
46. श॒म॒य॒ त्य॒पा म॒पाꣳ श॑मयति शमय त्य॒पाम् । \newline
47. अ॒पां ॅवै वा अ॒पा म॒पां ॅवै । \newline
48. वा ए॒त दे॒तद् वै वा ए॒तत् । \newline
49. ए॒तत् पुष्प॒म् पुष्प॑ मे॒त दे॒तत् पुष्प᳚म् । \newline
50. पुष्पं॒ ॅयद् यत् पुष्प॒म् पुष्पं॒ ॅयत् । \newline
51. यद् वे॑त॒सो वे॑त॒सो यद् यद् वे॑त॒सः । \newline
52. वे॒त॒सो॑ ऽपा म॒पां ॅवे॑त॒सो वे॑त॒सो॑ ऽपाम् । \newline
53. अ॒पाꣳ शरः॒ शरो॒ ऽपा म॒पाꣳ शरः॑ । \newline

\textbf{Ghana Paata } \newline

1. ऋ॒च्छ॒तु॒ यं ॅय मृ॑च्छ त्वृच्छतु॒ यम् द्वि॒ष्मो द्वि॒ष्मो य मृ॑च्छ त्वृच्छतु॒ यम् द्वि॒ष्मः । \newline
2. यम् द्वि॒ष्मो द्वि॒ष्मो यं ॅयम् द्वि॒ष्म इतीति॑ द्वि॒ष्मो यं ॅयम् द्वि॒ष्म इति॑ । \newline
3. द्वि॒ष्म इतीति॑ द्वि॒ष्मो द्वि॒ष्म इत्या॑हा॒हेति॑ द्वि॒ष्मो द्वि॒ष्म इत्या॑ह । \newline
4. इत्या॑हा॒हे तीत्या॑ह॒ यं ॅय मा॒हे तीत्या॑ह॒ यम् । \newline
5. आ॒ह॒ यं ॅय मा॑हाह॒ य मे॒वैव य मा॑हाह॒ य मे॒व । \newline
6. य मे॒वैव यं ॅय मे॒व द्वेष्टि॒ द्वेष्ट्ये॒व यं ॅय मे॒व द्वेष्टि॑ । \newline
7. ए॒व द्वेष्टि॒ द्वेष्ट्ये॒ वैव द्वेष्टि॒ तम् तम् द्वेष्ट्ये॒ वैव द्वेष्टि॒ तम् । \newline
8. द्वेष्टि॒ तम् तम् द्वेष्टि॒ द्वेष्टि॒ त म॑स्यास्य॒ तम् द्वेष्टि॒ द्वेष्टि॒ त म॑स्य । \newline
9. त म॑स्यास्य॒ तम् त म॑स्य क्षु॒धा क्षु॒धा ऽस्य॒ तम् त म॑स्य क्षु॒धा । \newline
10. अ॒स्य॒ क्षु॒धा क्षु॒धा ऽस्या᳚स्य क्षु॒धा च॑ च क्षु॒धा ऽस्या᳚स्य क्षु॒धा च॑ । \newline
11. क्षु॒धा च॑ च क्षु॒धा क्षु॒धा च॑ शु॒चा शु॒चा च॑ क्षु॒धा क्षु॒धा च॑ शु॒चा । \newline
12. च॒ शु॒चा शु॒चा च॑ च शु॒चा च॑ च शु॒चा च॑ च शु॒चा च॑ । \newline
13. शु॒चा च॑ च शु॒चा शु॒चा चा᳚र्पय त्यर्पयति च शु॒चा शु॒चा चा᳚र्पयति । \newline
14. चा॒र्प॒य॒ त्य॒र्प॒य॒ति॒ च॒ चा॒र्प॒य॒ति॒ त्रि स्त्रि र॑र्पयति च चार्पयति॒ त्रिः । \newline
15. अ॒र्प॒य॒ति॒ त्रि स्त्रि र॑र्पय त्यर्पयति॒ त्रिः प॑रिषि॒ञ्चन् प॑रिषि॒ञ्चन् त्रिर॑र्पय त्यर्पयति॒ त्रिः प॑रिषि॒ञ्चन्न् । \newline
16. त्रिः प॑रिषि॒ञ्चन् प॑रिषि॒ञ्चन् त्रि स्त्रिः प॑रिषि॒ञ्चन् परि॒ परि॑ परिषि॒ञ्चन् त्रि स्त्रिः प॑रिषि॒ञ्चन् परि॑ । \newline
17. प॒रि॒षि॒ञ्चन् परि॒ परि॑ परिषि॒ञ्चन् प॑रिषि॒ञ्चन् पर्ये᳚त्येति॒ परि॑ परिषि॒ञ्चन् प॑रिषि॒ञ्चन् पर्ये॑ति । \newline
18. प॒रि॒षि॒ञ्चन्निति॑ परि - सि॒ञ्चन्न् । \newline
19. पर्ये᳚त्येति॒ परि॒ पर्ये॑ति त्रि॒वृत् त्रि॒वृ दे॑ति॒ परि॒ पर्ये॑ति त्रि॒वृत् । \newline
20. ए॒ति॒ त्रि॒वृत् त्रि॒वृ दे᳚त्येति त्रि॒वृद् वै वै त्रि॒वृ दे᳚त्येति त्रि॒वृद् वै । \newline
21. त्रि॒वृद् वै वै त्रि॒वृत् त्रि॒वृद् वा अ॒ग्नि र॒ग्निर् वै त्रि॒वृत् त्रि॒वृद् वा अ॒ग्निः । \newline
22. त्रि॒वृदिति॑ त्रि - वृत् । \newline
23. वा अ॒ग्नि र॒ग्निर् वै वा अ॒ग्निर् यावा॒न्॒. यावा॑ न॒ग्निर् वै वा अ॒ग्निर् यावान्॑ । \newline
24. अ॒ग्निर् यावा॒न्॒. यावा॑ न॒ग्नि र॒ग्निर् यावा॑ ने॒वैव यावा॑ न॒ग्नि र॒ग्निर् यावा॑ ने॒व । \newline
25. यावा॑ ने॒वैव यावा॒न्॒. यावा॑ ने॒वाग्नि र॒ग्नि रे॒व यावा॒न्॒. यावा॑ ने॒वाग्निः । \newline
26. ए॒वाग्नि र॒ग्नि रे॒वै वाग्नि स्तस्य॒ तस्या॒ग्नि रे॒वैवाग्नि स्तस्य॑ । \newline
27. अ॒ग्नि स्तस्य॒ तस्या॒ग्नि र॒ग्नि स्तस्य॒ शुचꣳ॒॒ शुच॒म् तस्या॒ग्नि र॒ग्नि स्तस्य॒ शुच᳚म् । \newline
28. तस्य॒ शुचꣳ॒॒ शुच॒म् तस्य॒ तस्य॒ शुचꣳ॑ शमयति शमयति॒ शुच॒म् तस्य॒ तस्य॒ शुचꣳ॑ शमयति । \newline
29. शुचꣳ॑ शमयति शमयति॒ शुचꣳ॒॒ शुचꣳ॑ शमयति॒ त्रि स्त्रिः श॑मयति॒ शुचꣳ॒॒ शुचꣳ॑ शमयति॒ त्रिः । \newline
30. श॒म॒य॒ति॒ त्रि स्त्रिः श॑मयति शमयति॒ त्रिः पुनः॒ पुन॒ स्त्रिः श॑मयति शमयति॒ त्रिः पुनः॑ । \newline
31. त्रिः पुनः॒ पुन॒ स्त्रि स्त्रिः पुनः॒ परि॒ परि॒ पुन॒ स्त्रि स्त्रिः पुनः॒ परि॑ । \newline
32. पुनः॒ परि॒ परि॒ पुनः॒ पुनः॒ पर्ये᳚त्येति॒ परि॒ पुनः॒ पुनः॒ पर्ये॑ति । \newline
33. पर्ये᳚त्येति॒ परि॒ पर्ये॑ति॒ षट् थ्षडे॑ति॒ परि॒ पर्ये॑ति॒ षट् । \newline
34. ए॒ति॒ षट् थ्षडे᳚त्येति॒ षट् थ्सꣳ सꣳ षडे᳚त्येति॒ षट् थ्सम् । \newline
35. षट् थ्सꣳ सꣳ षट् थ्षट् थ्सम् प॑द्यन्ते पद्यन्ते॒ सꣳ षट् थ्षट् थ्सम् प॑द्यन्ते । \newline
36. सम् प॑द्यन्ते पद्यन्ते॒ सꣳ सम् प॑द्यन्ते॒ षट् थ्षट् प॑द्यन्ते॒ सꣳ सम् प॑द्यन्ते॒ षट् । \newline
37. प॒द्य॒न्ते॒ षट् थ्षट् प॑द्यन्ते पद्यन्ते॒ षड् वै वै षट् प॑द्यन्ते पद्यन्ते॒ षड् वै । \newline
38. षड् वै वै षट् थ्षड् वा ऋ॒तव॑ ऋ॒तवो॒ वै षट् थ्षड् वा ऋ॒तवः॑ । \newline
39. वा ऋ॒तव॑ ऋ॒तवो॒ वै वा ऋ॒तव॑ ऋ॒तुभिर्॑. ऋ॒तुभिर्॑. ऋ॒तवो॒ वै वा ऋ॒तव॑ ऋ॒तुभिः॑ । \newline
40. ऋ॒तव॑ ऋ॒तुभिर्॑. ऋ॒तुभिर्॑. ऋ॒तव॑ ऋ॒तव॑ ऋ॒तुभि॑ रे॒वैव र्‌तुभिर्॑. ऋ॒तव॑ ऋ॒तव॑ ऋ॒तुभि॑ रे॒व । \newline
41. ऋ॒तुभि॑ रे॒वैव र्‌तुभिर्॑. ऋ॒तुभि॑ रे॒वास्या᳚ स्यै॒व र्‌तुभिर्॑. ऋ॒तुभि॑ रे॒वास्य॑ । \newline
42. ऋ॒तुभि॒रित्यृ॒तु - भिः॒ । \newline
43. ए॒वास्या᳚ स्यै॒वै वास्य॒ शुचꣳ॒॒ शुच॑ मस्यै॒ वैवास्य॒ शुच᳚म् । \newline
44. अ॒स्य॒ शुचꣳ॒॒ शुच॑ मस्यास्य॒ शुचꣳ॑ शमयति शमयति॒ शुच॑ मस्यास्य॒ शुचꣳ॑ शमयति । \newline
45. शुचꣳ॑ शमयति शमयति॒ शुचꣳ॒॒ शुचꣳ॑ शमय त्य॒पा म॒पाꣳ श॑मयति॒ शुचꣳ॒॒ शुचꣳ॑ शमय त्य॒पाम् । \newline
46. श॒म॒य॒ त्य॒पा म॒पाꣳ श॑मयति शमय त्य॒पां ॅवै वा अ॒पाꣳ श॑मयति शमय त्य॒पां ॅवै । \newline
47. अ॒पां ॅवै वा अ॒पा म॒पां ॅवा ए॒त दे॒तद् वा अ॒पा म॒पां ॅवा ए॒तत् । \newline
48. वा ए॒त दे॒तद् वै वा ए॒तत् पुष्प॒म् पुष्प॑ मे॒तद् वै वा ए॒तत् पुष्प᳚म् । \newline
49. ए॒तत् पुष्प॒म् पुष्प॑ मे॒त दे॒तत् पुष्पं॒ ॅयद् यत् पुष्प॑ मे॒त दे॒तत् पुष्पं॒ ॅयत् । \newline
50. पुष्पं॒ ॅयद् यत् पुष्प॒म् पुष्पं॒ ॅयद् वे॑त॒सो वे॑त॒सो यत् पुष्प॒म् पुष्पं॒ ॅयद् वे॑त॒सः । \newline
51. यद् वे॑त॒सो वे॑त॒सो यद् यद् वे॑त॒सो॑ ऽपा म॒पां ॅवे॑त॒सो यद् यद् वे॑त॒सो॑ ऽपाम् । \newline
52. वे॒त॒सो॑ ऽपा म॒पां ॅवे॑त॒सो वे॑त॒सो॑ ऽपाꣳ शरः॒ शरो॒ ऽपां ॅवे॑त॒सो वे॑त॒सो॑ ऽपाꣳ शरः॑ । \newline
53. अ॒पाꣳ शरः॒ शरो॒ ऽपा म॒पाꣳ शरो ऽव॑का॒ अव॑काः॒ शरो॒ ऽपा म॒पाꣳ शरो ऽव॑काः । \newline
\pagebreak
\markright{ TS 5.4.4.3  \hfill https://www.vedavms.in \hfill}

\section{ TS 5.4.4.3 }

\textbf{TS 5.4.4.3 } \newline
\textbf{Samhita Paata} \newline

शरोऽव॑का वेतसशा॒खया॒ चाव॑काभिश्च॒ वि क॑र्.ष॒त्यापो॒ वै शा॒न्ताः शा॒न्ताभि॑रे॒वास्य॒ शुचꣳ॑ शमयति॒ यो वा अ॒ग्निं चि॒तं प्र॑थ॒मः प॒शुर॑धि॒क्राम॑तीश्व॒रो वै तꣳ शु॒चा प्र॒दहो॑ म॒ण्डूके॑न॒ विक॑र्.षत्ये॒ष वै प॑शू॒ना-म॑नुपजीवनी॒यो न वा ए॒ष ग्रा॒म्येषु॑ प॒शुषु॑ हि॒तो नाऽऽ*र॒ण्येषु॒ तमे॒व शु॒चाऽर्प॑यत्यष्टा॒भिर्वि क॑र्.षत्य॒ - [  ] \newline

\textbf{Pada Paata} \newline

शरः॑ । अव॑काः । वे॒त॒स॒शा॒खयेति॑ वेतस - शा॒खया᳚ । च॒ । अव॑काभिः । च॒ । वीति॑ । क॒र्.॒ष॒ति॒ । आपः॑ । वै । शा॒न्ताः । शा॒न्ताभिः॑ । ए॒व । अ॒स्य॒ । शुच᳚म् । श॒म॒य॒ति॒ । यः । वै । अ॒ग्निम् । चि॒तम् । प्र॒थ॒मः । प॒शुः । अ॒धि॒क्राम॒तीत्य॑धि-क्राम॑ति । ई॒श्व॒रः । वै । तम् । शु॒चा । प्र॒दह॒ इति॑ प्र - दहः॑ । म॒ण्डूके॑न । वीति॑ । क॒र्.॒ष॒ति॒ । ए॒षः । वै । प॒शू॒नाम् । अ॒नु॒प॒जी॒व॒नी॒य इत्य॑नुप -  जी॒व॒नी॒यः । न । वै । ए॒षः । ग्रा॒म्येषु॑ । प॒शुषु॑ । हि॒तः । न । आ॒र॒ण्येषु॑ । तम् । ए॒व । शु॒चा । अ॒र्प॒य॒ति॒ । अ॒ष्टा॒भिः । वीति॑ । क॒र्.॒ष॒ति॒ ।  \newline


\textbf{Krama Paata} \newline

शरोऽव॑काः । अव॑का वेतसशा॒खया᳚ । वे॒त॒स॒शा॒खया॑ च । वे॒त॒स॒शा॒खयेति॑ वेतस - शा॒खया᳚ । चाव॑काभिः । अव॑काभिश्च । च॒ वि । वि क॑र्.षति । क॒र्॒.ष॒त्यापः॑ । आपो॒ वै । वै शा॒न्ताः । शा॒न्ताः शा॒न्ताभिः॑ । शा॒न्ताभि॑रे॒व । ए॒वास्य॑ । अ॒स्य॒ शुच᳚म् । शुचꣳ॑ शमयति । श॒म॒य॒ति॒ यः । यो वै । वा अ॒ग्निम् । अ॒ग्निम् चि॒तम् । चि॒तम् प्र॑थ॒मः । प्र॒थ॒मः प॒शुः । प॒शुर॑धि॒क्राम॑ति । अ॒धि॒क्राम॑तीश्व॒रः । अ॒धि॒क्राम॒तीत्य॑धि - क्राम॑ति । ई॒श्व॒रो वै । वै तम् । तꣳ शु॒चा । शु॒चा प्र॒दहः॑ । प्र॒दहो॑ म॒ण्डूके॑न । प्र॒दह॒ इति॑ प्र - दहः॑ । म॒ण्डूके॑न॒ वि । वि क॑र्.षति । क॒र्॒.ष॒त्ये॒षः । ए॒ष वै । वै प॑शू॒नाम् । प॒शू॒नाम॑नुपजीवनी॒यः । अ॒नु॒प॒जी॒व॒नी॒यो न । अ॒नु॒प॒जी॒व॒नी॒य इत्य॑नुप - जी॒व॒नी॒यः । न वै । वा ए॒षः । ए॒ष ग्रा॒म्येषु॑ । ग्रा॒म्येषु॑ प॒शुषु॑ । प॒शुषु॑ हि॒तः । हि॒तो न । नार॒ण्येषु॑ । आ॒र॒ण्येषु॒ तम् । तमे॒व । ए॒व शु॒चा । शु॒चाऽर्प॑यति । अ॒र्प॒य॒त्य॒ष्टा॒भिः । अ॒ष्टा॒भिर् वि । वि क॑र्.षति । क॒र्॒.ष॒त्य॒ष्टाक्ष॑रा \newline

\textbf{Jatai Paata} \newline

1. शरो ऽव॑का॒ अव॑काः॒ शरः॒ शरो ऽव॑काः । \newline
2. अव॑का वेतसशा॒खया॑ वेतसशा॒खया ऽव॑का॒ अव॑का वेतसशा॒खया᳚ । \newline
3. वे॒त॒स॒शा॒खया॑ च च वेतसशा॒खया॑ वेतसशा॒खया॑ च । \newline
4. वे॒त॒स॒शा॒खयेति॑ वेतस - शा॒खया᳚ । \newline
5. चाव॑काभि॒ रव॑काभिश्च॒ चाव॑काभिः । \newline
6. अव॑काभिश्च॒ चाव॑काभि॒ रव॑काभिश्च । \newline
7. च॒ वि वि च॑ च॒ वि । \newline
8. वि क॑र्.षति कर्.षति॒ वि वि क॑र्.षति । \newline
9. क॒र्॒.ष॒ त्याप॒ आपः॑ कर्.षति कर्.ष॒ त्यापः॑ । \newline
10. आपो॒ वै वा आप॒ आपो॒ वै । \newline
11. वै शा॒न्ताः शा॒न्ता वै वै शा॒न्ताः । \newline
12. शा॒न्ताः शा॒न्ताभिः॑ शा॒न्ताभिः॑ शा॒न्ताः शा॒न्ताः शा॒न्ताभिः॑ । \newline
13. शा॒न्ताभि॑ रे॒वैव शा॒न्ताभिः॑ शा॒न्ताभि॑ रे॒व । \newline
14. ए॒वास्या᳚ स्यै॒वै वास्य॑ । \newline
15. अ॒स्य॒ शुचꣳ॒॒ शुच॑ मस्यास्य॒ शुच᳚म् । \newline
16. शुचꣳ॑ शमयति शमयति॒ शुचꣳ॒॒ शुचꣳ॑ शमयति । \newline
17. श॒म॒य॒ति॒ यो यः श॑मयति शमयति॒ यः । \newline
18. यो वै वै यो यो वै । \newline
19. वा अ॒ग्नि म॒ग्निं ॅवै वा अ॒ग्निम् । \newline
20. अ॒ग्निम् चि॒तम् चि॒त म॒ग्नि म॒ग्निम् चि॒तम् । \newline
21. चि॒तम् प्र॑थ॒मः प्र॑थ॒म श्चि॒तम् चि॒तम् प्र॑थ॒मः । \newline
22. प्र॒थ॒मः प॒शुः प॒शुः प्र॑थ॒मः प्र॑थ॒मः प॒शुः । \newline
23. प॒शु र॑धि॒क्राम॑ त्यधि॒क्राम॑ति प॒शुः प॒शु र॑धि॒क्राम॑ति । \newline
24. अ॒धि॒क्राम॑ती श्व॒र ई᳚श्व॒रो॑ ऽधि॒क्राम॑ त्यधि॒क्राम॑ती श्व॒रः । \newline
25. अ॒धि॒क्राम॒तीत्य॑धि - क्राम॑ति । \newline
26. ई॒श्व॒रो वै वा ई᳚श्व॒र ई᳚श्व॒रो वै । \newline
27. वै तम् तं ॅवै वै तम् । \newline
28. तꣳ शु॒चा शु॒चा तम् तꣳ शु॒चा । \newline
29. शु॒चा प्र॒दहः॑ प्र॒दहः॑ शु॒चा शु॒चा प्र॒दहः॑ । \newline
30. प्र॒दहो॑ म॒ण्डूके॑न म॒ण्डूके॑न प्र॒दहः॑ प्र॒दहो॑ म॒ण्डूके॑न । \newline
31. प्र॒दह॒ इति॑ प्र - दहः॑ । \newline
32. म॒ण्डूके॑न॒ वि वि म॒ण्डूके॑न म॒ण्डूके॑न॒ वि । \newline
33. वि क॑र्.षति कर्.षति॒ वि वि क॑र्.षति । \newline
34. क॒र्॒.ष॒ त्ये॒ष ए॒ष क॑र्.षति कर्.ष त्ये॒षः । \newline
35. ए॒ष वै वा ए॒ष ए॒ष वै । \newline
36. वै प॑शू॒नाम् प॑शू॒नां ॅवै वै प॑शू॒नाम् । \newline
37. प॒शू॒ना म॑नुपजीवनी॒यो॑ ऽनुपजीवनी॒यः प॑शू॒नाम् प॑शू॒ना म॑नुपजीवनी॒यः । \newline
38. अ॒नु॒प॒जी॒व॒नी॒यो न नानु॑पजीवनी॒यो॑ ऽनुपजीवनी॒यो न । \newline
39. अ॒नु॒प॒जी॒व॒नी॒य इत्य॑नुप - जी॒व॒नी॒यः । \newline
40. न वै वै न न वै । \newline
41. वा ए॒ष ए॒ष वै वा ए॒षः । \newline
42. ए॒ष ग्रा॒म्येषु॑ ग्रा॒म्ये ष्वे॒ष ए॒ष ग्रा॒म्येषु॑ । \newline
43. ग्रा॒म्येषु॑ प॒शुषु॑ प॒शुषु॑ ग्रा॒म्येषु॑ ग्रा॒म्येषु॑ प॒शुषु॑ । \newline
44. प॒शुषु॑ हि॒तो हि॒तः प॒शुषु॑ प॒शुषु॑ हि॒तः । \newline
45. हि॒तो न न हि॒तो हि॒तो न । \newline
46. नार॒ण्ये ष्वा॑र॒ण्येषु॒ न नार॒ण्येषु॑ । \newline
47. आ॒र॒ण्येषु॒ तम् त मा॑र॒ण्ये ष्वा॑र॒ण्येषु॒ तम् । \newline
48. त मे॒वैव तम् त मे॒व । \newline
49. ए॒व शु॒चा शु॒चै वैव शु॒चा । \newline
50. शु॒चा ऽर्प॑य त्यर्पयति शु॒चा शु॒चा ऽर्प॑यति । \newline
51. अ॒र्प॒य॒ त्य॒ष्टा॒भि र॑ष्टा॒भि र॑र्पय त्यर्पय त्यष्टा॒भिः । \newline
52. अ॒ष्टा॒भिर् वि व्य॑ष्टा॒भि र॑ष्टा॒भिर् वि । \newline
53. वि क॑र्.षति कर्.षति॒ वि वि क॑र्.षति । \newline
54. क॒र्॒.ष॒ त्य॒ष्टाक्ष॑रा॒ ऽष्टाक्ष॑रा कर्.षति कर्.ष त्य॒ष्टाक्ष॑रा । \newline

\textbf{Ghana Paata } \newline

1. शरो ऽव॑का॒ अव॑काः॒ शरः॒ शरो ऽव॑का वेतसशा॒खया॑ वेतसशा॒खया ऽव॑काः॒ शरः॒ शरो ऽव॑का वेतसशा॒खया᳚ । \newline
2. अव॑का वेतसशा॒खया॑ वेतसशा॒खया ऽव॑का॒ अव॑का वेतसशा॒खया॑ च च वेतसशा॒खया ऽव॑का॒ अव॑का वेतसशा॒खया॑ च । \newline
3. वे॒त॒स॒शा॒खया॑ च च वेतसशा॒खया॑ वेतसशा॒खया॒ चाव॑काभि॒ रव॑काभिश्च वेतसशा॒खया॑ वेतसशा॒खया॒ चाव॑काभिः । \newline
4. वे॒त॒स॒शा॒खयेति॑ वेतस - शा॒खया᳚ । \newline
5. चाव॑काभि॒ रव॑काभिश्च॒ चाव॑काभिश्च॒ चाव॑काभिश्च॒ चाव॑काभिश्च । \newline
6. अव॑काभिश्च॒ चाव॑काभि॒ रव॑काभिश्च॒ वि वि चाव॑काभि॒ रव॑काभिश्च॒ वि । \newline
7. च॒ वि वि च॑ च॒ वि क॑र्.षति कर्.षति॒ वि च॑ च॒ वि क॑र्.षति । \newline
8. वि क॑र्.षति कर्.षति॒ वि वि क॑र्.ष॒ त्याप॒ आपः॑ कर्.षति॒ वि वि क॑र्.ष॒ त्यापः॑ । \newline
9. क॒र्॒.ष॒ त्याप॒ आपः॑ कर्.षति कर्.ष॒ त्यापो॒ वै वा आपः॑ कर्.षति कर्.ष॒ त्यापो॒ वै । \newline
10. आपो॒ वै वा आप॒ आपो॒ वै शा॒न्ताः शा॒न्ता वा आप॒ आपो॒ वै शा॒न्ताः । \newline
11. वै शा॒न्ताः शा॒न्ता वै वै शा॒न्ताः शा॒न्ताभिः॑ शा॒न्ताभिः॑ शा॒न्ता वै वै शा॒न्ताः शा॒न्ताभिः॑ । \newline
12. शा॒न्ताः शा॒न्ताभिः॑ शा॒न्ताभिः॑ शा॒न्ताः शा॒न्ताः शा॒न्ताभि॑ रे॒वैव शा॒न्ताभिः॑ शा॒न्ताः शा॒न्ताः शा॒न्ताभि॑ रे॒व । \newline
13. शा॒न्ताभि॑ रे॒वैव शा॒न्ताभिः॑ शा॒न्ताभि॑ रे॒वास्या᳚ स्यै॒व शा॒न्ताभिः॑ शा॒न्ताभि॑ रे॒वास्य॑ । \newline
14. ए॒वास्या᳚ स्यै॒वै वास्य॒ शुचꣳ॒॒ शुच॑ मस्यै॒ वैवास्य॒ शुच᳚म् । \newline
15. अ॒स्य॒ शुचꣳ॒॒ शुच॑ मस्यास्य॒ शुचꣳ॑ शमयति शमयति॒ शुच॑ मस्यास्य॒ शुचꣳ॑ शमयति । \newline
16. शुचꣳ॑ शमयति शमयति॒ शुचꣳ॒॒ शुचꣳ॑ शमयति॒ यो यः श॑मयति॒ शुचꣳ॒॒ शुचꣳ॑ शमयति॒ यः । \newline
17. श॒म॒य॒ति॒ यो यः श॑मयति शमयति॒ यो वै वै यः श॑मयति शमयति॒ यो वै । \newline
18. यो वै वै यो यो वा अ॒ग्नि म॒ग्निं ॅवै यो यो वा अ॒ग्निम् । \newline
19. वा अ॒ग्नि म॒ग्निं ॅवै वा अ॒ग्निम् चि॒तम् चि॒त म॒ग्निं ॅवै वा अ॒ग्निम् चि॒तम् । \newline
20. अ॒ग्निम् चि॒तम् चि॒त म॒ग्नि म॒ग्निम् चि॒तम् प्र॑थ॒मः प्र॑थ॒म श्चि॒त म॒ग्नि म॒ग्निम् चि॒तम् प्र॑थ॒मः । \newline
21. चि॒तम् प्र॑थ॒मः प्र॑थ॒म श्चि॒तम् चि॒तम् प्र॑थ॒मः प॒शुः प॒शुः प्र॑थ॒म श्चि॒तम् चि॒तम् प्र॑थ॒मः प॒शुः । \newline
22. प्र॒थ॒मः प॒शुः प॒शुः प्र॑थ॒मः प्र॑थ॒मः प॒शु र॑धि॒क्राम॑ त्यधि॒क्राम॑ति प॒शुः प्र॑थ॒मः प्र॑थ॒मः प॒शु र॑धि॒क्राम॑ति । \newline
23. प॒शु र॑धि॒क्राम॑ त्यधि॒क्राम॑ति प॒शुः प॒शु र॑धि॒क्राम॑ तीश्व॒र ई᳚श्व॒रो॑ ऽधि॒क्राम॑ति प॒शुः प॒शु र॑धि॒क्राम॑ तीश्व॒रः । \newline
24. अ॒धि॒क्राम॑ तीश्व॒र ई᳚श्व॒रो॑ ऽधि॒क्राम॑ त्यधि॒क्राम॑ तीश्व॒रो वै वा ई᳚श्व॒रो॑ ऽधि॒क्राम॑ त्यधि॒क्राम॑ तीश्व॒रो वै । \newline
25. अ॒धि॒क्राम॒तीत्य॑धि - क्राम॑ति । \newline
26. ई॒श्व॒रो वै वा ई᳚श्व॒र ई᳚श्व॒रो वै तम् तं ॅवा ई᳚श्व॒र ई᳚श्व॒रो वै तम् । \newline
27. वै तम् तं ॅवै वै तꣳ शु॒चा शु॒चा तं ॅवै वै तꣳ शु॒चा । \newline
28. तꣳ शु॒चा शु॒चा तम् तꣳ शु॒चा प्र॒दहः॑ प्र॒दहः॑ शु॒चा तम् तꣳ शु॒चा प्र॒दहः॑ । \newline
29. शु॒चा प्र॒दहः॑ प्र॒दहः॑ शु॒चा शु॒चा प्र॒दहो॑ म॒ण्डूके॑न म॒ण्डूके॑न प्र॒दहः॑ शु॒चा शु॒चा प्र॒दहो॑ म॒ण्डूके॑न । \newline
30. प्र॒दहो॑ म॒ण्डूके॑न म॒ण्डूके॑न प्र॒दहः॑ प्र॒दहो॑ म॒ण्डूके॑न॒ वि वि म॒ण्डूके॑न प्र॒दहः॑ प्र॒दहो॑ म॒ण्डूके॑न॒ वि । \newline
31. प्र॒दह॒ इति॑ प्र - दहः॑ । \newline
32. म॒ण्डूके॑न॒ वि वि म॒ण्डूके॑न म॒ण्डूके॑न॒ वि क॑र्.षति कर्.षति॒ वि म॒ण्डूके॑न म॒ण्डूके॑न॒ वि क॑र्.षति । \newline
33. वि क॑र्.षति कर्.षति॒ वि वि क॑र्.ष त्ये॒ष ए॒ष क॑र्.षति॒ वि वि क॑र्.ष त्ये॒षः । \newline
34. क॒र्॒.ष॒ त्ये॒ष ए॒ष क॑र्.षति कर्.ष त्ये॒ष वै वा ए॒ष क॑र्.षति कर्.ष त्ये॒ष वै । \newline
35. ए॒ष वै वा ए॒ष ए॒ष वै प॑शू॒नाम् प॑शू॒नां ॅवा ए॒ष ए॒ष वै प॑शू॒नाम् । \newline
36. वै प॑शू॒नाम् प॑शू॒नां ॅवै वै प॑शू॒ना म॑नुपजीवनी॒यो॑ ऽनुपजीवनी॒यः प॑शू॒नां ॅवै वै प॑शू॒ना म॑नुपजीवनी॒यः । \newline
37. प॒शू॒ना म॑नुपजीवनी॒यो॑ ऽनुपजीवनी॒यः प॑शू॒नाम् प॑शू॒ना म॑नुपजीवनी॒यो न नानु॑पजीवनी॒यः प॑शू॒नाम् प॑शू॒ना म॑नुपजीवनी॒यो न । \newline
38. अ॒नु॒प॒जी॒व॒नी॒यो न नानु॑पजीवनी॒यो॑ ऽनुपजीवनी॒यो न वै वै नानु॑पजीवनी॒यो॑ ऽनुपजीवनी॒यो न वै । \newline
39. अ॒नु॒प॒जी॒व॒नी॒य इत्य॑नुप - जी॒व॒नी॒यः । \newline
40. न वै वै न न वा ए॒ष ए॒ष वै न न वा ए॒षः । \newline
41. वा ए॒ष ए॒ष वै वा ए॒ष ग्रा॒म्येषु॑ ग्रा॒म्ये ष्वे॒ष वै वा ए॒ष ग्रा॒म्येषु॑ । \newline
42. ए॒ष ग्रा॒म्येषु॑ ग्रा॒म्ये ष्वे॒ष ए॒ष ग्रा॒म्येषु॑ प॒शुषु॑ प॒शुषु॑ ग्रा॒म्ये ष्वे॒ष ए॒ष ग्रा॒म्येषु॑ प॒शुषु॑ । \newline
43. ग्रा॒म्येषु॑ प॒शुषु॑ प॒शुषु॑ ग्रा॒म्येषु॑ ग्रा॒म्येषु॑ प॒शुषु॑ हि॒तो हि॒तः प॒शुषु॑ ग्रा॒म्येषु॑ ग्रा॒म्येषु॑ प॒शुषु॑ हि॒तः । \newline
44. प॒शुषु॑ हि॒तो हि॒तः प॒शुषु॑ प॒शुषु॑ हि॒तो न न हि॒तः प॒शुषु॑ प॒शुषु॑ हि॒तो न । \newline
45. हि॒तो न न हि॒तो हि॒तो नार॒ण्ये ष्वा॑र॒ण्येषु॒ न हि॒तो हि॒तो नार॒ण्येषु॑ । \newline
46. नार॒ण्ये ष्वा॑र॒ण्येषु॒ न नार॒ण्येषु॒ तम् त मा॑र॒ण्येषु॒ न नार॒ण्येषु॒ तम् । \newline
47. आ॒र॒ण्येषु॒ तम् त मा॑र॒ण्ये ष्वा॑र॒ण्येषु॒ त मे॒वैव त मा॑र॒ण्ये ष्वा॑र॒ण्येषु॒ त मे॒व । \newline
48. त मे॒वैव तम् त मे॒व शु॒चा शु॒चैव तम् त मे॒व शु॒चा । \newline
49. ए॒व शु॒चा शु॒चैवैव शु॒चा ऽर्प॑य त्यर्पयति शु॒चैवैव शु॒चा ऽर्प॑यति । \newline
50. शु॒चा ऽर्प॑य त्यर्पयति शु॒चा शु॒चा ऽर्प॑य त्यष्टा॒भि र॑ष्टा॒भि र॑र्पयति शु॒चा शु॒चा ऽर्प॑य त्यष्टा॒भिः । \newline
51. अ॒र्प॒य॒ त्य॒ष्टा॒भि र॑ष्टा॒भि र॑र्पय त्यर्पय त्यष्टा॒भिर् वि व्य॑ष्टा॒भि र॑र्पय त्यर्पय त्यष्टा॒भिर् वि । \newline
52. अ॒ष्टा॒भिर् वि व्य॑ष्टा॒भि र॑ष्टा॒भिर् वि क॑र्.षति कर्.षति॒ व्य॑ष्टा॒भि र॑ष्टा॒भिर् वि क॑र्.षति । \newline
53. वि क॑र्.षति कर्.षति॒ वि वि क॑र्.ष त्य॒ष्टाक्ष॑रा॒ ऽष्टाक्ष॑रा कर्.षति॒ वि वि क॑र्.ष त्य॒ष्टाक्ष॑रा । \newline
54. क॒र्॒.ष॒ त्य॒ष्टाक्ष॑रा॒ ऽष्टाक्ष॑रा कर्.षति कर्.ष त्य॒ष्टाक्ष॑रा गाय॒त्री गा॑य॒ त्र्य॑ष्टाक्ष॑रा कर्.षति कर्.ष त्य॒ष्टाक्ष॑रा गाय॒त्री । \newline
\pagebreak
\markright{ TS 5.4.4.4  \hfill https://www.vedavms.in \hfill}

\section{ TS 5.4.4.4 }

\textbf{TS 5.4.4.4 } \newline
\textbf{Samhita Paata} \newline

-ष्टाक्ष॑रा गाय॒त्री गा॑य॒त्रो᳚ऽग्निर्यावा॑-ने॒वाऽग्निस्तस्य॒ शुचꣳ॑ शमयति पाव॒कव॑तीभि॒रन्नं॒ ॅवै पा॑व॒कोऽन्ने॑नै॒वास्य॒ शुचꣳ॑ शमयति मृ॒त्युर्वा ए॒ष यद॒ग्निर्ब्रह्म॑ण ए॒तद्रू॒पं ॅयत् कृ॑ष्णाजि॒नं कार्ष्णी॑ उपा॒नहा॒वुप॑ मुञ्चते॒ ब्रह्म॑णै॒व मृ॒त्योर॒न्तर्द्ध॑त्ते॒ ऽन्तर्मृ॒त्योर्द्ध॑त्ते॒ ऽन्तर॒न्नाद्या॒-दित्या॑हुर॒न्या-मु॑पमु॒ञ्चते॒ऽन्यां नान्त - [  ] \newline

\textbf{Pada Paata} \newline

अ॒ष्टाक्ष॒रेत्य॒ष्टा-अ॒क्ष॒रा॒ । गा॒य॒त्री । गा॒य॒त्रः । अ॒ग्निः । यावान्॑ । ए॒व । अ॒ग्निः । तस्य॑ । शुच᳚म् । श॒म॒य॒ति॒ । पा॒व॒कव॑तीभि॒रिति॑ पाव॒क - व॒ती॒भिः॒ । अन्न᳚म् । वै । पा॒व॒कः । अन्ने॑न । ए॒व । अ॒स्य॒ । शुच᳚म् । श॒म॒य॒ति॒ । मृ॒त्युः । वै ।   ए॒षः । यत् । अ॒ग्निः । ब्रह्म॑णः । ए॒तत् । रू॒पम् । यत् । कृ॒ष्णा॒जि॒नमिति॑ कृष्ण - अ॒जि॒नम् । कार्ष्णी॒ इति॑ । उ॒पा॒नहौ᳚ । उपेति॑ । मु॒ञ्च॒ते॒ । ब्रह्म॑णा । ए॒व । मृ॒त्योः । अ॒न्तः । ध॒त्ते॒ । अ॒न्तः । मृ॒त्योः । ध॒त्ते॒ । अ॒न्तः । अ॒न्नाद्या॒दित्य॑न्न - अद्या᳚त् । इति॑ । आ॒हुः॒ । अ॒न्याम् । उ॒प॒मु॒ञ्चत॒ इत्यु॑प - मु॒ञ्चते᳚ । अ॒न्याम् । न । अ॒न्तः ।  \newline


\textbf{Krama Paata} \newline

अ॒ष्टाक्ष॑रा गाय॒त्री । अ॒ष्टाक्ष॒रेत्य॒ष्टा - अ॒क्ष॒रा॒ । गा॒य॒त्री गा॑य॒त्रः । गा॒य॒त्रो᳚ऽग्निः । अ॒ग्निर् यावान्॑ । यावा॑ने॒व । ए॒वाग्निः । अ॒ग्निस्तस्य॑ । तस्य॒ शुच᳚म् । शुचꣳ॑ शमयति । श॒म॒य॒ति॒ पा॒व॒कव॑तीभिः । पा॒व॒कव॑तीभि॒रन्न᳚म् । पा॒व॒कव॑तीभि॒रिति॑ पाव॒क - व॒ती॒भिः॒ । अन्न॒म् ॅवै । वै पा॑व॒कः । पा॒व॒कोऽन्ने॑न । अन्ने॑नै॒व । ए॒वास्य॑ । अ॒स्य॒ शुच᳚म् । शुचꣳ॑ शमयति । श॒म॒य॒ति॒ मृ॒त्युः । मृ॒त्युर् वै । वा ए॒षः । ए॒ष यत् । यद॒ग्निः । अ॒ग्निर् ब्रह्म॑णः । ब्रह्म॑ण ए॒तत् । ए॒तद् रू॒पम् । रू॒पम् ॅयत् । यत् कृ॑ष्णाजि॒नम् । कृ॒ष्णा॒जि॒नम् कार्.ष्णी᳚ । कृ॒ष्णा॒जि॒नमिति॑ कृष्ण - अ॒जि॒नम् । कार्.ष्णी॑ उपा॒नहौ᳚ । कार्.ष्णी॒ इति॒ कार्.ष्णी᳚ । उ॒पा॒नहा॒वुप॑ । उप॑ मुञ्चते । मु॒ञ्च॒ते॒ ब्रह्म॑णा । ब्रह्म॑णै॒व । ए॒व मृ॒त्योः । मृ॒त्योर॒न्तः । अ॒न्तर् ध॑त्ते । ध॒त्ते॒ऽन्तः । अ॒न्तर् मृ॒त्योः । मृ॒त्योर् ध॑त्ते । ध॒त्ते॒ऽन्तः । अ॒न्तर॒न्नाद्या᳚त् । अ॒न्नाद्या॒दिति॑ । अ॒न्नाद्या॒दित्य॑न्न - अद्या᳚त् । इत्या॑हुः । आ॒हु॒र॒न्याम् । अ॒न्यामु॑पमु॒ञ्चते᳚ । उ॒प॒मु॒ञ्चते॒ऽन्याम् । उ॒प॒मु॒ञ्चत॒ इत्यु॑प - मु॒ञ्चते᳚ । अ॒न्याम् न । नान्तः ( ) । अ॒न्तरे॒व \newline

\textbf{Jatai Paata} \newline

1. अ॒ष्टाक्ष॑रा गाय॒त्री गा॑य॒ त्र्य॑ष्टाक्ष॑रा॒ ऽष्टाक्ष॑रा गाय॒त्री । \newline
2. अ॒ष्टाक्ष॒रेत्य॒ष्टा - अ॒क्ष॒रा॒ । \newline
3. गा॒य॒त्री गा॑य॒त्रो गा॑य॒त्रो गा॑य॒त्री गा॑य॒त्री गा॑य॒त्रः । \newline
4. गा॒य॒त्रो᳚ ऽग्नि र॒ग्निर् गा॑य॒त्रो गा॑य॒त्रो᳚ ऽग्निः । \newline
5. अ॒ग्निर् यावा॒न्॒. यावा॑ न॒ग्नि र॒ग्निर् यावान्॑ । \newline
6. यावा॑ ने॒वैव यावा॒न्॒. यावा॑ ने॒व । \newline
7. ए॒वाग्नि र॒ग्नि रे॒वै वाग्निः । \newline
8. अ॒ग्नि स्तस्य॒ तस्या॒ ग्नि र॒ग्नि स्तस्य॑ । \newline
9. तस्य॒ शुचꣳ॒॒ शुच॒म् तस्य॒ तस्य॒ शुच᳚म् । \newline
10. शुचꣳ॑ शमयति शमयति॒ शुचꣳ॒॒ शुचꣳ॑ शमयति । \newline
11. श॒म॒य॒ति॒ पा॒व॒कव॑तीभिः पाव॒कव॑तीभिः शमयति शमयति पाव॒कव॑तीभिः । \newline
12. पा॒व॒कव॑तीभि॒ रन्न॒ मन्न॑म् पाव॒कव॑तीभिः पाव॒कव॑तीभि॒ रन्न᳚म् । \newline
13. पा॒व॒कव॑तीभि॒रिति॑ पाव॒क - व॒ती॒भिः॒ । \newline
14. अन्नं॒ ॅवै वा अन्न॒ मन्नं॒ ॅवै । \newline
15. वै पा॑व॒कः पा॑व॒को वै वै पा॑व॒कः । \newline
16. पा॒व॒को ऽन्ने॒ना न्ने॑न पाव॒कः पा॑व॒को ऽन्ने॑न । \newline
17. अन्ने॑ नै॒वैवा न्ने॒ना न्ने॑नै॒व । \newline
18. ए॒वास्या᳚ स्यै॒वै वास्य॑ । \newline
19. अ॒स्य॒ शुचꣳ॒॒ शुच॑ मस्यास्य॒ शुच᳚म् । \newline
20. शुचꣳ॑ शमयति शमयति॒ शुचꣳ॒॒ शुचꣳ॑ शमयति । \newline
21. श॒म॒य॒ति॒ मृ॒त्युर् मृ॒त्युः श॑मयति शमयति मृ॒त्युः । \newline
22. मृ॒त्युर् वै वै मृ॒त्युर् मृ॒त्युर् वै । \newline
23. वा ए॒ष ए॒ष वै वा ए॒षः । \newline
24. ए॒ष यद् यदे॒ष ए॒ष यत् । \newline
25. यद॒ग्नि र॒ग्निर् यद् यद॒ग्निः । \newline
26. अ॒ग्निर् ब्रह्म॑णो॒ ब्रह्म॑णो॒ ऽग्नि र॒ग्निर् ब्रह्म॑णः । \newline
27. ब्रह्म॑ण ए॒त दे॒तद् ब्रह्म॑णो॒ ब्रह्म॑ण ए॒तत् । \newline
28. ए॒तद् रू॒पꣳ रू॒प मे॒त दे॒तद् रू॒पम् । \newline
29. रू॒पं ॅयद् यद् रू॒पꣳ रू॒पं ॅयत् । \newline
30. यत् कृ॑ष्णाजि॒नम् कृ॑ष्णाजि॒नं ॅयद् यत् कृ॑ष्णाजि॒नम् । \newline
31. कृ॒ष्णा॒जि॒नम् कार्ष्णी॒ कार्ष्णी॑ कृष्णाजि॒नम् कृ॑ष्णाजि॒नम् कार्ष्णी᳚ । \newline
32. कृ॒ष्णा॒जि॒नमिति॑ कृष्ण - अ॒जि॒नम् । \newline
33. कार्ष्णी॑ उपा॒नहा॑ वुपा॒नहौ॒ कार्ष्णी॒ कार्ष्णी॑ उपा॒नहौ᳚ । \newline
34. कार्ष्णी॒ इति॒ कार्ष्णी᳚ । \newline
35. उ॒पा॒नहा॒ वुपोपो॑ पा॒नहा॑ वुपा॒नहा॒ वुप॑ । \newline
36. उप॑ मुञ्चते मुञ्चत॒ उपोप॑ मुञ्चते । \newline
37. मु॒ञ्च॒ते॒ ब्रह्म॑णा॒ ब्रह्म॑णा मुञ्चते मुञ्चते॒ ब्रह्म॑णा । \newline
38. ब्रह्म॑ णै॒वैव ब्रह्म॑णा॒ ब्रह्म॑णै॒व । \newline
39. ए॒व मृ॒त्योर् मृ॒त्यो रे॒वैव मृ॒त्योः । \newline
40. मृ॒त्यो र॒न्त र॒न्तर् मृ॒त्योर् मृ॒त्यो र॒न्तः । \newline
41. अ॒न्तर् ध॑त्ते धत्ते॒ ऽन्त र॒न्तर् ध॑त्ते । \newline
42. ध॒त्ते॒ ऽन्त र॒न्तर् ध॑त्ते धत्ते॒ ऽन्तः । \newline
43. अ॒न्तर् मृ॒त्योर् मृ॒त्यो र॒न्त र॒न्तर् मृ॒त्योः । \newline
44. मृ॒त्योर् ध॑त्ते धत्ते मृ॒त्योर् मृ॒त्योर् ध॑त्ते । \newline
45. ध॒त्ते॒ ऽन्त र॒न्तर् ध॑त्ते धत्ते॒ ऽन्तः । \newline
46. अ॒न्त र॒न्नाद्या॑ द॒न्नाद्या॑ द॒न्त र॒न्त र॒न्नाद्या᳚त् । \newline
47. अ॒न्नाद्या॒ दिती त्य॒न्नाद्या॑ द॒न्नाद्या॒ दिति॑ । \newline
48. अ॒न्नाद्या॒दित्य॑न्न - अद्या᳚त् । \newline
49. इत्या॑हु राहु॒ रिती त्या॑हुः । \newline
50. आ॒हु॒ र॒न्या म॒न्या मा॑हु राहु र॒न्याम् । \newline
51. अ॒न्या मु॑पमु॒ञ्चत॑ उपमु॒ञ्चते॒ ऽन्या म॒न्या मु॑पमु॒ञ्चते᳚ । \newline
52. उ॒प॒मु॒ञ्चते॒ ऽन्या म॒न्या मु॑पमु॒ञ्चत॑ उपमु॒ञ्चते॒ ऽन्याम् । \newline
53. उ॒प॒मु॒ञ्चत॒ इत्यु॑प - मु॒ञ्चते᳚ । \newline
54. अ॒न्याम् न नान्या म॒न्याम् न । \newline
55. नान्त र॒न्तर् न नान्तः । \newline
56. अ॒न्त रे॒वै वान्त र॒न्त रे॒व । \newline

\textbf{Ghana Paata } \newline

1. अ॒ष्टाक्ष॑रा गाय॒त्री गा॑य॒ त्र्य॑ष्टाक्ष॑रा॒ ऽष्टाक्ष॑रा गाय॒त्री गा॑य॒त्रो गा॑य॒त्रो गा॑य॒
त्र्य॑ष्टाक्ष॑रा॒ ऽष्टाक्ष॑रा गाय॒त्री गा॑य॒त्रः । \newline
2. अ॒ष्टाक्ष॒रेत्य॒ष्टा - अ॒क्ष॒रा॒ । \newline
3. गा॒य॒त्री गा॑य॒त्रो गा॑य॒त्रो गा॑य॒त्री गा॑य॒त्री गा॑य॒त्रो᳚ ऽग्नि र॒ग्निर् गा॑य॒त्रो गा॑य॒त्री गा॑य॒त्री गा॑य॒त्रो᳚ ऽग्निः । \newline
4. गा॒य॒त्रो᳚ ऽग्नि र॒ग्निर् गा॑य॒त्रो गा॑य॒त्रो᳚ ऽग्निर् यावा॒न्॒. यावा॑ न॒ग्निर् गा॑य॒त्रो गा॑य॒त्रो᳚ ऽग्निर् यावान्॑ । \newline
5. अ॒ग्निर् यावा॒न्॒. यावा॑ न॒ग्नि र॒ग्निर् यावा॑ ने॒वैव यावा॑ न॒ग्नि र॒ग्निर् यावा॑ ने॒व । \newline
6. यावा॑ ने॒वैव यावा॒न्॒. यावा॑ ने॒वाग्नि र॒ग्नि रे॒व यावा॒न्॒. यावा॑ ने॒वाग्निः । \newline
7. ए॒वाग्नि र॒ग्नि रे॒वैवाग्नि स्तस्य॒ तस्या॒ ग्नि रे॒वैवाग्नि स्तस्य॑ । \newline
8. अ॒ग्नि स्तस्य॒ तस्या॒ ग्नि र॒ग्नि स्तस्य॒ शुचꣳ॒॒ शुच॒म् तस्या॒ ग्नि र॒ग्नि स्तस्य॒ शुच᳚म् । \newline
9. तस्य॒ शुचꣳ॒॒ शुच॒म् तस्य॒ तस्य॒ शुचꣳ॑ शमयति शमयति॒ शुच॒म् तस्य॒ तस्य॒ शुचꣳ॑ शमयति । \newline
10. शुचꣳ॑ शमयति शमयति॒ शुचꣳ॒॒ शुचꣳ॑ शमयति पाव॒कव॑तीभिः पाव॒कव॑तीभिः शमयति॒ शुचꣳ॒॒ शुचꣳ॑ शमयति पाव॒कव॑तीभिः । \newline
11. श॒म॒य॒ति॒ पा॒व॒कव॑तीभिः पाव॒कव॑तीभिः शमयति शमयति पाव॒कव॑तीभि॒ रन्न॒ मन्न॑म् पाव॒कव॑तीभिः शमयति शमयति पाव॒कव॑तीभि॒ रन्न᳚म् । \newline
12. पा॒व॒कव॑तीभि॒ रन्न॒ मन्न॑म् पाव॒कव॑तीभिः पाव॒कव॑तीभि॒ रन्नं॒ ॅवै वा अन्न॑म् पाव॒कव॑तीभिः पाव॒कव॑तीभि॒ रन्नं॒ ॅवै । \newline
13. पा॒व॒कव॑तीभि॒रिति॑ पाव॒क - व॒ती॒भिः॒ । \newline
14. अन्नं॒ ॅवै वा अन्न॒ मन्नं॒ ॅवै पा॑व॒कः पा॑व॒को वा अन्न॒ मन्नं॒ ॅवै पा॑व॒कः । \newline
15. वै पा॑व॒कः पा॑व॒को वै वै पा॑व॒को ऽन्ने॒ना न्ने॑न पाव॒को वै वै पा॑व॒को ऽन्ने॑न । \newline
16. पा॒व॒को ऽन्ने॒ना न्ने॑न पाव॒कः पा॑व॒को ऽन्ने॑नै॒ वैवान्ने॑न पाव॒कः पा॑व॒को ऽन्ने॑नै॒व । \newline
17. अन्ने॑नै॒ वैवान्ने॒ना न्ने॑नै॒ वास्या᳚ स्यै॒वा न्ने॒ना न्ने॑नै॒ वास्य॑ । \newline
18. ए॒वास्या᳚ स्यै॒वै वास्य॒ शुचꣳ॒॒ शुच॑ मस्यै॒ वैवास्य॒ शुच᳚म् । \newline
19. अ॒स्य॒ शुचꣳ॒॒ शुच॑ मस्यास्य॒ शुचꣳ॑ शमयति शमयति॒ शुच॑ मस्यास्य॒ शुचꣳ॑ शमयति । \newline
20. शुचꣳ॑ शमयति शमयति॒ शुचꣳ॒॒ शुचꣳ॑ शमयति मृ॒त्युर् मृ॒त्युः श॑मयति॒ शुचꣳ॒॒ शुचꣳ॑ शमयति मृ॒त्युः । \newline
21. श॒म॒य॒ति॒ मृ॒त्युर् मृ॒त्युः श॑मयति शमयति मृ॒त्युर् वै वै मृ॒त्युः श॑मयति शमयति मृ॒त्युर् वै । \newline
22. मृ॒त्युर् वै वै मृ॒त्युर् मृ॒त्युर् वा ए॒ष ए॒ष वै मृ॒त्युर् मृ॒त्युर् वा ए॒षः । \newline
23. वा ए॒ष ए॒ष वै वा ए॒ष यद् यदे॒ष वै वा ए॒ष यत् । \newline
24. ए॒ष यद् यदे॒ष ए॒ष यद॒ग्नि र॒ग्निर् यदे॒ष ए॒ष यद॒ग्निः । \newline
25. यद॒ग्नि र॒ग्निर् यद् यद॒ग्निर् ब्रह्म॑णो॒ ब्रह्म॑णो॒ ऽग्निर् यद् यद॒ग्निर् ब्रह्म॑णः । \newline
26. अ॒ग्निर् ब्रह्म॑णो॒ ब्रह्म॑णो॒ ऽग्नि र॒ग्निर् ब्रह्म॑ण ए॒त दे॒तद् ब्रह्म॑णो॒ ऽग्नि र॒ग्निर् ब्रह्म॑ण ए॒तत् । \newline
27. ब्रह्म॑ण ए॒त दे॒तद् ब्रह्म॑णो॒ ब्रह्म॑ण ए॒तद् रू॒पꣳ रू॒प मे॒तद् ब्रह्म॑णो॒ ब्रह्म॑ण ए॒तद् रू॒पम् । \newline
28. ए॒तद् रू॒पꣳ रू॒प मे॒त दे॒तद् रू॒पं ॅयद् यद् रू॒प मे॒त दे॒तद् रू॒पं ॅयत् । \newline
29. रू॒पं ॅयद् यद् रू॒पꣳ रू॒पं ॅयत् कृ॑ष्णाजि॒नम् कृ॑ष्णाजि॒नं ॅयद् रू॒पꣳ रू॒पं ॅयत् कृ॑ष्णाजि॒नम् । \newline
30. यत् कृ॑ष्णाजि॒नम् कृ॑ष्णाजि॒नं ॅयद् यत् कृ॑ष्णाजि॒नम् कार्ष्णी॒ कार्ष्णी॑ कृष्णाजि॒नं ॅयद् यत् कृ॑ष्णाजि॒नम् कार्ष्णी᳚ । \newline
31. कृ॒ष्णा॒जि॒नम् कार्ष्णी॒ कार्ष्णी॑ कृष्णाजि॒नम् कृ॑ष्णाजि॒नम् कार्ष्णी॑ उपा॒नहा॑ वुपा॒नहौ॒ कार्ष्णी॑ कृष्णाजि॒नम् कृ॑ष्णाजि॒नम् कार्ष्णी॑ उपा॒नहौ᳚ । \newline
32. कृ॒ष्णा॒जि॒नमिति॑ कृष्ण - अ॒जि॒नम् । \newline
33. कार्ष्णी॑ उपा॒नहा॑ वुपा॒नहौ॒ कार्ष्णी॒ कार्ष्णी॑ उपा॒नहा॒ वुपोपो॑ पा॒नहौ॒ कार्ष्णी॒ कार्ष्णी॑ उपा॒नहा॒ वुप॑ । \newline
34. कार्ष्णी॒ इति॒ कार्ष्णी᳚ । \newline
35. उ॒पा॒नहा॒ वुपोपो॑ पा॒नहा॑ वुपा॒नहा॒ वुप॑ मुञ्चते मुञ्चत॒ उपो॑ पा॒नहा॑ वुपा॒नहा॒ वुप॑ मुञ्चते । \newline
36. उप॑ मुञ्चते मुञ्चत॒ उपोप॑ मुञ्चते॒ ब्रह्म॑णा॒ ब्रह्म॑णा मुञ्चत॒ उपोप॑ मुञ्चते॒ ब्रह्म॑णा । \newline
37. मु॒ञ्च॒ते॒ ब्रह्म॑णा॒ ब्रह्म॑णा मुञ्चते मुञ्चते॒ ब्रह्म॑णै॒वैव ब्रह्म॑णा मुञ्चते मुञ्चते॒ ब्रह्म॑णै॒व । \newline
38. ब्रह्म॑णै॒वैव ब्रह्म॑णा॒ ब्रह्म॑णै॒व मृ॒त्योर् मृ॒त्यो रे॒व ब्रह्म॑णा॒ ब्रह्म॑णै॒व मृ॒त्योः । \newline
39. ए॒व मृ॒त्योर् मृ॒त्यो रे॒वैव मृ॒त्यो र॒न्त र॒न्तर् मृ॒त्यो रे॒वैव मृ॒त्यो र॒न्तः । \newline
40. मृ॒त्यो र॒न्त र॒न्तर् मृ॒त्योर् मृ॒त्यो र॒न्तर् ध॑त्ते धत्ते॒ ऽन्तर् मृ॒त्योर् मृ॒त्यो र॒न्तर् ध॑त्ते । \newline
41. अ॒न्तर् ध॑त्ते धत्ते॒ ऽन्त र॒न्तर् ध॑त्ते॒ ऽन्त र॒न्तर् ध॑त्ते॒ ऽन्त र॒न्तर् ध॑त्ते॒ ऽन्तः । \newline
42. ध॒त्ते॒ ऽन्त र॒न्तर् ध॑त्ते धत्ते॒ ऽन्तर् मृ॒त्योर् मृ॒त्यो र॒न्तर् ध॑त्ते धत्ते॒ ऽन्तर् मृ॒त्योः । \newline
43. अ॒न्तर् मृ॒त्योर् मृ॒त्यो र॒न्त र॒न्तर् मृ॒त्योर् ध॑त्ते धत्ते मृ॒त्यो र॒न्त र॒न्तर् मृ॒त्योर् ध॑त्ते । \newline
44. मृ॒त्योर् ध॑त्ते धत्ते मृ॒त्योर् मृ॒त्योर् ध॑त्ते॒ ऽन्त र॒न्तर् ध॑त्ते मृ॒त्योर् मृ॒त्योर् ध॑त्ते॒ ऽन्तः । \newline
45. ध॒त्ते॒ ऽन्त र॒न्तर् ध॑त्ते धत्ते॒ ऽन्त र॒न्नाद्या॑ द॒न्नाद्या॑ द॒न्तर् ध॑त्ते धत्ते॒ ऽन्त र॒न्नाद्या᳚त् । \newline
46. अ॒न्त र॒न्नाद्या॑ द॒न्नाद्या॑ द॒न्त र॒न्त र॒न्नाद्या॒ दिती त्य॒न्नाद्या॑ द॒न्त र॒न्त र॒न्नाद्या॒ दिति॑ । \newline
47. अ॒न्नाद्या॒ दिती त्य॒न्नाद्या॑ द॒न्नाद्या॒ दित्या॑हु राहु॒ रित्य॒न्नाद्या॑ द॒न्नाद्या॒ दित्या॑हुः । \newline
48. अ॒न्नाद्या॒दित्य॑न्न - अद्या᳚त् । \newline
49. इत्या॑हु राहु॒ रितीत्या॑हु र॒न्या म॒न्या मा॑हु॒ रितीत्या॑हु र॒न्याम् । \newline
50. आ॒हु॒ र॒न्या म॒न्या मा॑हु राहु र॒न्या मु॑पमु॒ञ्चत॑ उपमु॒ञ्चते॒ ऽन्या मा॑हु राहु र॒न्या मु॑पमु॒ञ्चते᳚ । \newline
51. अ॒न्या मु॑पमु॒ञ्चत॑ उपमु॒ञ्चते॒ ऽन्या म॒न्या मु॑पमु॒ञ्चते॒ ऽन्या म॒न्या मु॑पमु॒ञ्चते॒ ऽन्या म॒न्या मु॑पमु॒ञ्चते॒ ऽन्याम् । \newline
52. उ॒प॒मु॒ञ्चते॒ ऽन्या म॒न्या मु॑पमु॒ञ्चत॑ उपमु॒ञ्चते॒ ऽन्यान्न नान्या मु॑पमु॒ञ्चत॑ उपमु॒ञ्चते॒ ऽन्यान्न । \newline
53. उ॒प॒मु॒ञ्चत॒ इत्यु॑प - मु॒ञ्चते᳚ । \newline
54. अ॒न्याम् न नान्या म॒न्याम् नान्त र॒न्तर् नान्या म॒न्याम् नान्तः । \newline
55. नान्त र॒न्तर् न नान्त रे॒वै वान्तर् न नान्त रे॒व । \newline
56. अ॒न्त रे॒वै वान्त र॒न्त रे॒व मृ॒त्योर् मृ॒त्यो रे॒वान्त र॒न्त रे॒व मृ॒त्योः । \newline
\pagebreak
\markright{ TS 5.4.4.5  \hfill https://www.vedavms.in \hfill}

\section{ TS 5.4.4.5 }

\textbf{TS 5.4.4.5 } \newline
\textbf{Samhita Paata} \newline

-रे॒व मृ॒त्योर्द्ध॒त्ते ऽवा॒ऽन्नाद्यꣳ॑ रुन्धे॒ नम॑स्ते॒ हर॑से शो॒चिष॒ इत्या॑ह नम॒स्कृत्य॒ हि वसी॑याꣳ समुप॒चर॑न्त्य॒न्यं ते॑ अ॒स्मत् त॑पन्तु हे॒तय॒ इत्या॑ह॒ यमे॒व द्वेष्टि॒ तम॑स्य शु॒चाऽर्प॑यति पाव॒को अ॒स्मभ्यꣳ॑ शि॒वो भ॒वेत्या॒हान्नं॒ ॅवै पा॑व॒कोऽन्न॑मे॒वाव॑ रुन्धे॒ द्वाभ्या॒मधि॑ क्रामति॒ प्रति॑ष्ठित्या अप॒स्य॑वतीभ्याꣳ॒॒ शान्त्यै᳚ ॥ \newline

\textbf{Pada Paata} \newline

ए॒व । मृ॒त्योः । ध॒त्ते॒ । अवेति॑ । अ॒न्नाद्य॒मित्य॑न्न - अद्य᳚म् । रु॒न्धे॒ । नमः॑ । ते॒ । हर॑से । शो॒चिषे᳚ । इति॑ । आ॒ह॒ । न॒म॒स्कृत्येति॑ नमः - कृत्य॑ । हि । वसी॑याꣳसम् । उ॒प॒चर॒न्तीत्यु॑प - चर॑न्ति । अ॒न्यम् । ते॒ । अ॒स्मत् । त॒प॒न्तु॒ । हे॒तयः॑ । इति॑ । आ॒ह॒ । यम् । ए॒व । द्वेष्टि॑ । तम् । अ॒स्य॒ । शु॒चा । अ॒र्प॒य॒ति॒ । पा॒व॒कः । अ॒स्मभ्य॒मित्य॒स्म - भ्य॒म् । शि॒वः । भ॒व॒ । इति॑ । आ॒ह॒ । अन्न᳚म् । वै । पा॒व॒कः । अन्न᳚म् । ए॒व । अवेति॑ । रु॒न्धे॒ । द्वाभ्या᳚म् । अधीति॑ । क्रा॒म॒ति॒ । प्रति॑ष्ठित्या॒ इति॒ प्रति॑ - स्थि॒त्यै॒ । अ॒प॒स्य॑वतीभ्या॒मित्य॑प॒स्य॑ - व॒ती॒भ्या॒म् । शान्त्यै᳚ ॥  \newline


\textbf{Krama Paata} \newline

ए॒व मृ॒त्योः । मृ॒त्योर् ध॑त्ते । ध॒त्तेऽव॑ । अवा॒न्नाद्य᳚म् । अ॒न्नाद्यꣳ॑ रुन्धे । अ॒न्नाद्य॒मित्य॑न्न - अद्य᳚म् । रु॒न्धे॒ नमः॑ । नम॑स्ते । ते॒ हर॑से । हर॑से शो॒चिषे᳚ । शो॒चिष॒ इति॑ । इत्या॑ह । आ॒ह॒ न॒म॒स्कृत्य॑ । न॒म॒स्कृत्य॒ हि । न॒म॒स्कृत्येति॑ नमः - कृत्य॑ । हि वसी॑याꣳसम् । वसी॑याꣳसमुप॒चर॑न्ति । उ॒प॒चर॑न्त्य॒न्यम् । उ॒प॒चर॒न्तीत्यु॑प - चर॑न्ति । अ॒न्यम् ते᳚ । ते॒ अ॒स्मत् । अ॒स्मत् त॑पन्तु । त॒प॒न्तु॒ हे॒तयः॑ । हे॒तय॒ इति॑ । इत्या॑ह । आ॒ह॒ यम् । यमे॒व । ए॒व द्वेष्टि॑ । द्वेष्टि॒ तम् । तम॑स्य । अ॒स्य॒ शु॒चा । शु॒चाऽर्प॑यति । अ॒र्प॒य॒ति॒ पा॒व॒कः । पा॒व॒को अ॒स्मभ्य᳚म् । अ॒स्मभ्यꣳ॑ शि॒वः । अ॒स्मभ्य॒मित्य॒स्म - भ्य॒म् । शि॒वो भ॑व । भ॒वेति॑ । इत्या॑ह । आ॒हान्न᳚म् । अन्न॒म् ॅवै । वै पा॑व॒कः । पा॒व॒कोऽन्न᳚म् । अन्न॑मे॒व । ए॒वाव॑ । अव॑ रुन्धे । रु॒न्धे॒ द्वाभ्या᳚म् । द्वाभ्या॒मधि॑ । अधि॑ क्रामति । क्रा॒म॒ति॒ प्रति॑ष्ठित्यै । प्रति॑ष्ठित्या अप॒स्य॑वतीभ्याम् । प्रति॑ष्ठित्या॒ इति॒ प्रति॑ - स्थि॒त्यै॒ । अ॒प॒स्य॑वतीभ्याꣳ॒॒ शान्त्यै᳚ । अ॒प॒स्य॑वतीभ्या॒मित्य॑प॒स्य॑ - व॒ती॒भ्या॒म् । शान्त्या॒ इति॒ शान्त्यै᳚ । \newline

\textbf{Jatai Paata} \newline

1. ए॒व मृ॒त्योर् मृ॒त्यो रे॒वैव मृ॒त्योः । \newline
2. मृ॒त्योर् ध॑त्ते धत्ते मृ॒त्योर् मृ॒त्योर् ध॑त्ते । \newline
3. ध॒त्ते ऽवाव॑ धत्ते ध॒त्ते ऽव॑ । \newline
4. अवा॒न्नाद्य॑ म॒न्नाद्य॒ मवावा॒ न्नाद्य᳚म् । \newline
5. अ॒न्नाद्यꣳ॑ रुन्धे रुन्धे॒ ऽन्नाद्य॑ म॒न्नाद्यꣳ॑ रुन्धे । \newline
6. अ॒न्नाद्य॒मित्य॑न्न - अद्य᳚म् । \newline
7. रु॒न्धे॒ नमो॒ नमो॑ रुन्धे रुन्धे॒ नमः॑ । \newline
8. नम॑ स्ते ते॒ नमो॒ नम॑ स्ते । \newline
9. ते॒ हर॑से॒ हर॑से ते ते॒ हर॑से । \newline
10. हर॑से शो॒चिषे॑ शो॒चिषे॒ हर॑से॒ हर॑से शो॒चिषे᳚ । \newline
11. शो॒चिष॒ इतीति॑ शो॒चिषे॑ शो॒चिष॒ इति॑ । \newline
12. इत्या॑हा॒हे तीत्या॑ह । \newline
13. आ॒ह॒ न॒म॒स्कृत्य॑ नम॒स्कृ त्या॑हाह नम॒स्कृत्य॑ । \newline
14. न॒म॒स्कृत्य॒ हि हि न॑म॒स्कृत्य॑ नम॒स्कृत्य॒ हि । \newline
15. न॒म॒स्कृत्येति॑ नमः - कृत्य॑ । \newline
16. हि वसी॑याꣳसं॒ ॅवसी॑याꣳसꣳ॒॒ हि हि वसी॑याꣳसम् । \newline
17. वसी॑याꣳस मुप॒चर॑न्त्यु प॒चर॑न्ति॒ वसी॑याꣳसं॒ ॅवसी॑याꣳस मुप॒चर॑न्ति । \newline
18. उ॒प॒चर॑न् त्य॒न्य म॒न्य मु॑प॒चर॑न् त्युप॒चर॑न् त्य॒न्यम् । \newline
19. उ॒प॒चर॒न्तीत्यु॑प - चर॑न्ति । \newline
20. अ॒न्यम् ते॑ ते अ॒न्य म॒न्यम् ते᳚ । \newline
21. ते॒ अ॒स्म द॒स्मत् ते॑ ते अ॒स्मत् । \newline
22. अ॒स्मत् त॑पन्तु तपन् त्व॒स्म द॒स्मत् त॑पन्तु । \newline
23. त॒प॒न्तु॒ हे॒तयो॑ हे॒तय॑ स्तपन्तु तपन्तु हे॒तयः॑ । \newline
24. हे॒तय॒ इतीति॑ हे॒तयो॑ हे॒तय॒ इति॑ । \newline
25. इत्या॑हा॒हे तीत्या॑ह । \newline
26. आ॒ह॒ यं ॅय मा॑हाह॒ यम् । \newline
27. य मे॒वैव यं ॅय मे॒व । \newline
28. ए॒व द्वेष्टि॒ द्वेष्ट्ये॒ वैव द्वेष्टि॑ । \newline
29. द्वेष्टि॒ तम् तम् द्वेष्टि॒ द्वेष्टि॒ तम् । \newline
30. त म॑स्यास्य॒ तम् त म॑स्य । \newline
31. अ॒स्य॒ शु॒चा शु॒चा ऽस्या᳚स्य शु॒चा । \newline
32. शु॒चा ऽर्प॑य त्यर्पयति शु॒चा शु॒चा ऽर्प॑यति । \newline
33. अ॒र्प॒य॒ति॒ पा॒व॒कः पा॑व॒को᳚ ऽर्पय त्यर्पयति पाव॒कः । \newline
34. पा॒व॒को अ॒स्मभ्य॑ म॒स्मभ्य॑म् पाव॒कः पा॑व॒को अ॒स्मभ्य᳚म् । \newline
35. अ॒स्मभ्यꣳ॑ शि॒वः शि॒वो अ॒स्मभ्य॑ म॒स्मभ्यꣳ॑ शि॒वः । \newline
36. अ॒स्मभ्य॒मित्य॒स्म - भ्य॒म् । \newline
37. शि॒वो भ॑व भव शि॒वः शि॒वो भ॑व । \newline
38. भ॒वे तीति॑ भव भ॒वेति॑ । \newline
39. इत्या॑हा॒हे तीत्या॑ह । \newline
40. आ॒हान्न॒ मन्न॑ माहा॒ हान्न᳚म् । \newline
41. अन्नं॒ ॅवै वा अन्न॒ मन्नं॒ ॅवै । \newline
42. वै पा॑व॒कः पा॑व॒को वै वै पा॑व॒कः । \newline
43. पा॒व॒को ऽन्न॒ मन्न॑म् पाव॒कः पा॑व॒को ऽन्न᳚म् । \newline
44. अन्न॑ मे॒वै वान्न॒ मन्न॑ मे॒व । \newline
45. ए॒वावा वै॒वै वाव॑ । \newline
46. अव॑ रुन्धे रु॒न्धे ऽवाव॑ रुन्धे । \newline
47. रु॒न्धे॒ द्वाभ्या॒म् द्वाभ्याꣳ॑ रुन्धे रुन्धे॒ द्वाभ्या᳚म् । \newline
48. द्वाभ्या॒ मध्यधि॒ द्वाभ्या॒म् द्वाभ्या॒ मधि॑ । \newline
49. अधि॑ क्रामति क्राम॒ त्यध्यधि॑ क्रामति । \newline
50. क्रा॒म॒ति॒ प्रति॑ष्ठित्यै॒ प्रति॑ष्ठित्यै क्रामति क्रामति॒ प्रति॑ष्ठित्यै । \newline
51. प्रति॑ष्ठित्या अप॒स्य॑वतीभ्या मप॒स्य॑वतीभ्या॒म् प्रति॑ष्ठित्यै॒ प्रति॑ष्ठित्या अप॒स्य॑वतीभ्याम् । \newline
52. प्रति॑ष्ठित्या॒ इति॒ प्रति॑ - स्थि॒त्यै॒ । \newline
53. अ॒प॒स्य॑वतीभ्याꣳ॒॒ शान्त्यै॒ शान्त्या॑ अप॒स्य॑वतीभ्या मप॒स्य॑वतीभ्याꣳ॒॒ शान्त्यै᳚ । \newline
54. अ॒प॒स्य॑वतीभ्या॒मित्य॑प॒स्य॑ - व॒ती॒भ्या॒म् । \newline
55. शान्त्या॒ इति॒ शान्त्यै᳚ । \newline

\textbf{Ghana Paata } \newline

1. ए॒व मृ॒त्योर् मृ॒त्यो रे॒वैव मृ॒त्योर् ध॑त्ते धत्ते मृ॒त्यो रे॒वैव मृ॒त्योर् ध॑त्ते । \newline
2. मृ॒त्योर् ध॑त्ते धत्ते मृ॒त्योर् मृ॒त्योर् ध॒त्ते ऽवाव॑ धत्ते मृ॒त्योर् मृ॒त्योर् ध॒त्ते ऽव॑ । \newline
3. ध॒त्ते ऽवाव॑ धत्ते ध॒त्ते ऽवा॒न्नाद्य॑ म॒न्नाद्य॒ मव॑ धत्ते ध॒त्ते ऽवा॒न्नाद्य᳚म् । \newline
4. अवा॒न्नाद्य॑ म॒न्नाद्य॒ मवावा॒ न्नाद्यꣳ॑ रुन्धे रुन्धे॒ ऽन्नाद्य॒ मवावा॒ न्नाद्यꣳ॑ रुन्धे । \newline
5. अ॒न्नाद्यꣳ॑ रुन्धे रुन्धे॒ ऽन्नाद्य॑ म॒न्नाद्यꣳ॑ रुन्धे॒ नमो॒ नमो॑ रुन्धे॒ ऽन्नाद्य॑ म॒न्नाद्यꣳ॑ रुन्धे॒ नमः॑ । \newline
6. अ॒न्नाद्य॒मित्य॑न्न - अद्य᳚म् । \newline
7. रु॒न्धे॒ नमो॒ नमो॑ रुन्धे रुन्धे॒ नम॑ स्ते ते॒ नमो॑ रुन्धे रुन्धे॒ नम॑ स्ते । \newline
8. नम॑ स्ते ते॒ नमो॒ नम॑ स्ते॒ हर॑से॒ हर॑से ते॒ नमो॒ नम॑ स्ते॒ हर॑से । \newline
9. ते॒ हर॑से॒ हर॑से ते ते॒ हर॑से शो॒चिषे॑ शो॒चिषे॒ हर॑से ते ते॒ हर॑से शो॒चिषे᳚ । \newline
10. हर॑से शो॒चिषे॑ शो॒चिषे॒ हर॑से॒ हर॑से शो॒चिष॒ इतीति॑ शो॒चिषे॒ हर॑से॒ हर॑से शो॒चिष॒ इति॑ । \newline
11. शो॒चिष॒ इतीति॑ शो॒चिषे॑ शो॒चिष॒ इत्या॑हा॒हेति॑ शो॒चिषे॑ शो॒चिष॒ इत्या॑ह । \newline
12. इत्या॑हा॒ हेतीत्या॑ह नम॒स्कृत्य॑ नम॒स्कृत्या॒ हेतीत्या॑ह नम॒स्कृत्य॑ । \newline
13. आ॒ह॒ न॒म॒स्कृत्य॑ नम॒स्कृत्या॑ हाह नम॒स्कृत्य॒ हि हि न॑म॒स्कृत्या॑ हाह नम॒स्कृत्य॒ हि । \newline
14. न॒म॒स्कृत्य॒ हि हि न॑म॒स्कृत्य॑ नम॒स्कृत्य॒ हि वसी॑याꣳसं॒ ॅवसी॑याꣳसꣳ॒॒ हि न॑म॒स्कृत्य॑ नम॒स्कृत्य॒ हि वसी॑याꣳसम् । \newline
15. न॒म॒स्कृत्येति॑ नमः - कृत्य॑ । \newline
16. हि वसी॑याꣳसं॒ ॅवसी॑याꣳसꣳ॒॒ हि हि वसी॑याꣳस मुप॒चर॑न् त्युप॒चर॑न्ति॒ वसी॑याꣳसꣳ॒॒ हि हि वसी॑याꣳस मुप॒चर॑न्ति । \newline
17. वसी॑याꣳस मुप॒चर॑ न्त्युप॒चर॑न्ति॒ वसी॑याꣳसं॒ ॅवसी॑याꣳस मुप॒चर॑न् त्य॒न्य म॒न्य मु॑प॒चर॑न्ति॒ वसी॑याꣳसं॒ ॅवसी॑याꣳस मुप॒चर॑न् त्य॒न्यम् । \newline
18. उ॒प॒चर॑न् त्य॒न्य म॒न्य मु॑प॒चर॑न् त्युप॒चर॑न् त्य॒न्यम् ते॑ ते अ॒न्य मु॑प॒चर॑न् त्युप॒चर॑न् त्य॒न्यम् ते᳚ । \newline
19. उ॒प॒चर॒न्तीत्यु॑प - चर॑न्ति । \newline
20. अ॒न्यम् ते॑ ते अ॒न्य म॒न्यम् ते॑ अ॒स्म द॒स्मत् ते॑ अ॒न्य म॒न्यम् ते॑ अ॒स्मत् । \newline
21. ते॒ अ॒स्म द॒स्मत् ते॑ ते अ॒स्मत् त॑पन्तु तपन् त्व॒स्मत् ते॑ ते अ॒स्मत् त॑पन्तु । \newline
22. अ॒स्मत् त॑पन्तु तपन्त् व॒स्म द॒स्मत् त॑पन्तु हे॒तयो॑ हे॒तय॑ स्तपन् त्व॒स्म द॒स्मत् त॑पन्तु हे॒तयः॑ । \newline
23. त॒प॒न्तु॒ हे॒तयो॑ हे॒तय॑ स्तपन्तु तपन्तु हे॒तय॒ इतीति॑ हे॒तय॑ स्तपन्तु तपन्तु हे॒तय॒ इति॑ । \newline
24. हे॒तय॒ इतीति॑ हे॒तयो॑ हे॒तय॒ इत्या॑हा॒हेति॑ हे॒तयो॑ हे॒तय॒ इत्या॑ह । \newline
25. इत्या॑हा॒हे तीत्या॑ह॒ यं ॅय मा॒हे तीत्या॑ह॒ यम् । \newline
26. आ॒ह॒ यं ॅय मा॑हाह॒ य मे॒वैव य मा॑हाह॒ य मे॒व । \newline
27. य मे॒वैव यं ॅय मे॒व द्वेष्टि॒ द्वेष्ट्ये॒व यं ॅय मे॒व द्वेष्टि॑ । \newline
28. ए॒व द्वेष्टि॒ द्वेष्ट्ये॒ वैव द्वेष्टि॒ तम् तम् द्वेष्ट्ये॒ वैव द्वेष्टि॒ तम् । \newline
29. द्वेष्टि॒ तम् तम् द्वेष्टि॒ द्वेष्टि॒ त म॑स्यास्य॒ तम् द्वेष्टि॒ द्वेष्टि॒ त म॑स्य । \newline
30. त म॑स्यास्य॒ तम् त म॑स्य शु॒चा शु॒चा ऽस्य॒ तम् त म॑स्य शु॒चा । \newline
31. अ॒स्य॒ शु॒चा शु॒चा ऽस्या᳚स्य शु॒चा ऽर्प॑य त्यर्पयति शु॒चा ऽस्या᳚स्य शु॒चा ऽर्प॑यति । \newline
32. शु॒चा ऽर्प॑य त्यर्पयति शु॒चा शु॒चा ऽर्प॑यति पाव॒कः पा॑व॒को᳚ ऽर्पयति शु॒चा शु॒चा ऽर्प॑यति पाव॒कः । \newline
33. अ॒र्प॒य॒ति॒ पा॒व॒कः पा॑व॒को᳚ ऽर्पय त्यर्पयति पाव॒को अ॒स्मभ्य॑ म॒स्मभ्य॑म् पाव॒को᳚ ऽर्पय त्यर्पयति पाव॒को अ॒स्मभ्य᳚म् । \newline
34. पा॒व॒को अ॒स्मभ्य॑ म॒स्मभ्य॑म् पाव॒कः पा॑व॒को अ॒स्मभ्यꣳ॑ शि॒वः शि॒वो अ॒स्मभ्य॑म् पाव॒कः पा॑व॒को अ॒स्मभ्यꣳ॑ शि॒वः । \newline
35. अ॒स्मभ्यꣳ॑ शि॒वः शि॒वो अ॒स्मभ्य॑ म॒स्मभ्यꣳ॑ शि॒वो भ॑व भव शि॒वो अ॒स्मभ्य॑ म॒स्मभ्यꣳ॑ शि॒वो भ॑व । \newline
36. अ॒स्मभ्य॒मित्य॒स्म - भ्य॒म् । \newline
37. शि॒वो भ॑व भव शि॒वः शि॒वो भ॒वे तीति॑ भव शि॒वः शि॒वो भ॒वेति॑ । \newline
38. भ॒वे तीति॑ भव भ॒वे त्या॑हा॒हेति॑ भव भ॒वे त्या॑ह । \newline
39. इत्या॑हा॒हे तीत्या॒ हान्न॒ मन्न॑ मा॒हेती त्या॒हान्न᳚म् । \newline
40. आ॒हान्न॒ मन्न॑ माहा॒ हान्नं॒ ॅवै वा अन्न॑ माहा॒ हान्नं॒ ॅवै । \newline
41. अन्नं॒ ॅवै वा अन्न॒ मन्नं॒ ॅवै पा॑व॒कः पा॑व॒को वा अन्न॒ मन्नं॒ ॅवै पा॑व॒कः । \newline
42. वै पा॑व॒कः पा॑व॒को वै वै पा॑व॒को ऽन्न॒ मन्न॑म् पाव॒को वै वै पा॑व॒को ऽन्न᳚म् । \newline
43. पा॒व॒को ऽन्न॒ मन्न॑म् पाव॒कः पा॑व॒को ऽन्न॑ मे॒वै वान्न॑म् पाव॒कः पा॑व॒को ऽन्न॑ मे॒व । \newline
44. अन्न॑ मे॒वै वान्न॒ मन्न॑ मे॒वा वावै॒ वान्न॒ मन्न॑ मे॒वाव॑ । \newline
45. ए॒वावा वै॒वै वाव॑ रुन्धे रु॒न्धे ऽवै॒वै वाव॑ रुन्धे । \newline
46. अव॑ रुन्धे रु॒न्धे ऽवाव॑ रुन्धे॒ द्वाभ्या॒म् द्वाभ्याꣳ॑ रु॒न्धे ऽवाव॑ रुन्धे॒ द्वाभ्या᳚म् । \newline
47. रु॒न्धे॒ द्वाभ्या॒म् द्वाभ्याꣳ॑ रुन्धे रुन्धे॒ द्वाभ्या॒ मध्यधि॒ द्वाभ्याꣳ॑ रुन्धे रुन्धे॒ द्वाभ्या॒ मधि॑ । \newline
48. द्वाभ्या॒ मध्यधि॒ द्वाभ्या॒म् द्वाभ्या॒ मधि॑ क्रामति क्राम॒ त्यधि॒ द्वाभ्या॒म् द्वाभ्या॒ मधि॑ क्रामति । \newline
49. अधि॑ क्रामति क्राम॒ त्यध्यधि॑ क्रामति॒ प्रति॑ष्ठित्यै॒ प्रति॑ष्ठित्यै क्राम॒ त्यध्यधि॑ क्रामति॒ प्रति॑ष्ठित्यै । \newline
50. क्रा॒म॒ति॒ प्रति॑ष्ठित्यै॒ प्रति॑ष्ठित्यै क्रामति क्रामति॒ प्रति॑ष्ठित्या अप॒स्य॑वतीभ्या मप॒स्य॑वतीभ्या॒म् प्रति॑ष्ठित्यै क्रामति क्रामति॒ प्रति॑ष्ठित्या अप॒स्य॑वतीभ्याम् । \newline
51. प्रति॑ष्ठित्या अप॒स्य॑वतीभ्या मप॒स्य॑वतीभ्या॒म् प्रति॑ष्ठित्यै॒ प्रति॑ष्ठित्या अप॒स्य॑वतीभ्याꣳ॒॒ शान्त्यै॒ शान्त्या॑ अप॒स्य॑वतीभ्या॒म् प्रति॑ष्ठित्यै॒ प्रति॑ष्ठित्या अप॒स्य॑वतीभ्याꣳ॒॒ शान्त्यै᳚ । \newline
52. प्रति॑ष्ठित्या॒ इति॒ प्रति॑ - स्थि॒त्यै॒ । \newline
53. अ॒प॒स्य॑वतीभ्याꣳ॒॒ शान्त्यै॒ शान्त्या॑ अप॒स्य॑वतीभ्या मप॒स्य॑वतीभ्याꣳ॒॒ शान्त्यै᳚ । \newline
54. अ॒प॒स्य॑वतीभ्या॒मित्य॑प॒स्य॑ - व॒ती॒भ्या॒म् । \newline
55. शान्त्या॒ इति॒ शान्त्यै᳚ । \newline
\pagebreak
\markright{ TS 5.4.5.1  \hfill https://www.vedavms.in \hfill}

\section{ TS 5.4.5.1 }

\textbf{TS 5.4.5.1 } \newline
\textbf{Samhita Paata} \newline

नृ॒षदे॒ वडिति॒ व्याघा॑रयति प॒ङ्क्त्याऽऽहु॑त्या यज्ञ्मु॒खमा र॑भते ऽक्ष्ण॒या व्याघा॑रयति॒ तस्मा॑दक्ष्ण॒या प॒शवोऽङ्गा॑नि॒ प्रह॑रन्ति॒ प्रति॑ष्ठित्यै॒ यद्व॑षट्कु॒र्याद्-या॒तया॑माऽस्य वषट्का॒रः स्या॒द्यन्न व॑षट्कु॒र्याद्-रक्षाꣳ॑सि य॒ज्ञ्ꣳ ह॑न्यु॒र्वडित्या॑ह प॒रोक्ष॑मे॒व वष॑ट् करोति॒ नास्य॑ या॒तया॑मा वषट्का॒रो भव॑ति॒ न य॒ज्ञ्ꣳ रक्षाꣳ॑सि घ्नन्ति हु॒तादो॒ वा अ॒न्ये दे॒वा - [  ] \newline

\textbf{Pada Paata} \newline

नृ॒षद॒ इति॑ नृ - सदे᳚ । वट् । इति॑ । व्याघा॑रय॒तीति॑ वि-आघा॑रयति । प॒ङ्क्त्या । आहु॒त्येत्या-हु॒त्या॒ । य॒ज्ञ्॒मु॒खमिति॑ यज्ञ् - मु॒खम् । एति॑ । र॒भ॒ते॒ । अ॒क्ष्ण॒या । व्याघा॑रय॒तीति॑ वि - आघा॑रयति । तस्मा᳚त् । अ॒क्ष्ण॒या । प॒शवः॑ । अङ्गा॑नि । प्रेति॑ । ह॒र॒न्ति॒ । प्रति॑ष्ठित्या॒ इति॒ प्रति॑ - स्थि॒त्यै॒ । यत् । व॒ष॒ट्कु॒र्यादिति॑ वषट् - कु॒र्यात् । या॒तया॒मेति॑ या॒त - या॒मा॒ । अ॒स्य॒ । व॒ष॒ट्का॒र इति॑ वषट्-का॒रः । स्या॒त् । यत् । न । व॒ष॒ट्कु॒र्यादिति॑ वषट् - कु॒र्यात् । रक्षाꣳ॑सि । य॒ज्ञ्म् । ह॒न्युः॒ । वट् । इति॑ । आ॒ह॒ । प॒रोक्ष॒मिति॑ परः - अक्ष᳚म् । ए॒व । वष॑ट् । क॒रो॒ति॒ । न । अ॒स्य॒ । या॒तया॒मेति॑ या॒त - या॒मा॒ । व॒ष॒ट्का॒र इति॑ वषट् - का॒रः । भव॑ति । न । य॒ज्ञ्म् । रक्षाꣳ॑सि । घ्न॒न्ति॒ । हु॒ताद॒ इति॑ हुत - अदः॑ । वै । अ॒न्ये । दे॒वाः ।  \newline


\textbf{Krama Paata} \newline

नृ॒षदे॒ वट् । नृ॒षद॒ इति॑ नृ - सदे᳚ । वडिति॑ । इति॒ व्याघा॑रयति । व्याघा॑रयति प॒ङ्क्त्या । व्याघा॑रय॒तीति॑ वि - आघा॑रयति । प॒ङ्क्त्याऽऽहु॑त्या । आहु॑त्या यज्ञ्मु॒खम् । आहु॒त्येत्या - हु॒त्या॒ । य॒ज्ञ्॒मु॒खमा । य॒ज्ञ्॒मु॒खमिति॑ यज्ञ् - मु॒खम् । आ र॑भते । र॒भ॒ते॒ऽक्ष्ण॒या । अ॒क्ष्ण॒या व्याघा॑रयति । व्याघा॑रयति॒ तस्मा᳚त् । व्याघा॑रय॒तीति॑ वि - आघा॑रयति । तस्मा॑दक्ष्ण॒या । अ॒क्ष्ण॒या प॒शवः॑ । प॒शवोऽङ्गा॑नि । अङ्गा॑नि॒ प्र । प्र ह॑रन्ति । ह॒र॒न्ति॒ प्रति॑ष्ठित्यै । प्रति॑ष्ठित्यै॒ यत् । प्रति॑ष्ठित्या॒ इति॒ प्रति॑ - स्थि॒त्यै॒ । यद् व॑षट्कु॒र्यात् । व॒ष॒ट्कु॒र्याद् या॒तया॑मा । व॒ष॒ट्कु॒र्यादिति॑ वषट्कु॒र्यात् । या॒तया॑माऽस्य । या॒तया॒मेति॑ या॒त - या॒मा॒ । अ॒स्य॒ व॒ष॒ट्का॒रः । व॒ष॒ट्का॒रः स्यात् । व॒ष॒ट्का॒र इति॑ वषट् - का॒रः । स्या॒द् यत् । यन् न । न व॑षट्कु॒र्यात् । व॒ष॒ट्कु॒र्याद् रक्षाꣳ॑सि । व॒ष॒ट्कु॒र्यादिति॑ वषट् - कु॒र्यात् । रक्षाꣳ॑सि य॒ज्ञ्म् । य॒ज्ञ्ꣳ ह॑न्युः । ह॒न्यु॒र् वट् । वडिति॑ । इत्या॑ह । आ॒ह॒ प॒रोक्ष᳚म् । प॒रोक्ष॑मे॒व । प॒रोक्ष॒मिति॑ परः - अक्ष᳚म् । ए॒व वष॑ट् । वष॑ट् करोति । क॒रो॒ति॒ न । नास्य॑ । अ॒स्य॒ या॒तया॑मा । या॒तया॑मा वषट्का॒रः । या॒तया॒मेति॑ या॒त - या॒मा॒ । व॒ष॒ट्का॒रो भव॑ति । व॒ष॒ट्का॒र इति॑ वषट् - का॒रः । भव॑ति॒ न । न य॒ज्ञ्म् । य॒ज्ञ्ꣳ रक्षाꣳ॑सि । रक्षाꣳ॑सि घ्नन्ति । घ्न॒न्ति॒ हु॒तादः॑ । हु॒तादो॒ वै । हु॒ताद॒ इति॑ हुत - अदः॑ । वा अ॒न्ये । अ॒न्ये दे॒वाः । दे॒वा अ॑हु॒तादः॑ \newline

\textbf{Jatai Paata} \newline

1. नृ॒षदे॒ वड् वण् णृ॒षदे॑ नृ॒षदे॒ वट् । \newline
2. नृ॒षद॒ इति॑ नृ - सदे᳚ । \newline
3. वडितीति॒ वड् वडिति॑ । \newline
4. इति॒ व्याघा॑रयति॒ व्याघा॑रय॒ती तीति॒ व्याघा॑रयति । \newline
5. व्याघा॑रयति प॒ङ्क्त्या प॒ङ्क्त्या व्याघा॑रयति॒ व्याघा॑रयति प॒ङ्क्त्या । \newline
6. व्याघा॑रय॒तीति॑ वि - आघा॑रयति । \newline
7. प॒ङ्क्त्या ऽऽहु॒त्या ऽऽहु॑त्या प॒ङ्क्त्या प॒ङ्क्त्या ऽऽहु॑त्या । \newline
8. आहु॑त्या यज्ञ्मु॒खं ॅय॑ज्ञ्मु॒ख माहु॒त्या ऽऽहु॑त्या यज्ञ्मु॒खम् । \newline
9. आहु॒त्येत्या - हु॒त्या॒ । \newline
10. य॒ज्ञ्॒मु॒ख मा य॑ज्ञ्मु॒खं ॅय॑ज्ञ्मु॒ख मा । \newline
11. य॒ज्ञ्॒मु॒खमिति॑ यज्ञ् - मु॒खम् । \newline
12. आ र॑भते रभत॒ आ र॑भते । \newline
13. र॒भ॒ते॒ ऽक्ष्ण॒या ऽक्ष्ण॒या र॑भते रभते ऽक्ष्ण॒या । \newline
14. अ॒क्ष्ण॒या व्याघा॑रयति॒ व्याघा॑रय त्यक्ष्ण॒या ऽक्ष्ण॒या व्याघा॑रयति । \newline
15. व्याघा॑रयति॒ तस्मा॒त् तस्मा॒द् व्याघा॑रयति॒ व्याघा॑रयति॒ तस्मा᳚त् । \newline
16. व्याघा॑रय॒तीति॑ वि - आघा॑रयति । \newline
17. तस्मा॑ दक्ष्ण॒या ऽक्ष्ण॒या तस्मा॒त् तस्मा॑ दक्ष्ण॒या । \newline
18. अ॒क्ष्ण॒या प॒शवः॑ प॒शवो᳚ ऽक्ष्ण॒या ऽक्ष्ण॒या प॒शवः॑ । \newline
19. प॒शवो ऽङ्गा॒ न्यङ्गा॑नि प॒शवः॑ प॒शवो ऽङ्गा॑नि । \newline
20. अङ्गा॑नि॒ प्र प्राङ्गा॒ न्यङ्गा॑नि॒ प्र । \newline
21. प्र ह॑रन्ति हरन्ति॒ प्र प्र ह॑रन्ति । \newline
22. ह॒र॒न्ति॒ प्रति॑ष्ठित्यै॒ प्रति॑ष्ठित्यै हरन्ति हरन्ति॒ प्रति॑ष्ठित्यै । \newline
23. प्रति॑ष्ठित्यै॒ यद् यत् प्रति॑ष्ठित्यै॒ प्रति॑ष्ठित्यै॒ यत् । \newline
24. प्रति॑ष्ठित्या॒ इति॒ प्रति॑ - स्थि॒त्यै॒ । \newline
25. यद् व॑षट्कु॒र्याद् व॑षट्कु॒र्याद् यद् यद् व॑षट्कु॒र्यात् । \newline
26. व॒ष॒ट्कु॒र्याद् या॒तया॑मा या॒तया॑मा वषट्कु॒र्याद् व॑षट्कु॒र्याद् या॒तया॑मा । \newline
27. व॒ष॒ट्कु॒र्यादिति॑ वषट् - कु॒र्यात् । \newline
28. या॒तया॑मा ऽस्यास्य या॒तया॑मा या॒तया॑मा ऽस्य । \newline
29. या॒तया॒मेति॑ या॒त - या॒मा॒ । \newline
30. अ॒स्य॒ व॒ष॒ट्का॒रो व॑षट्का॒रो᳚ ऽस्यास्य वषट्का॒रः । \newline
31. व॒ष॒ट्का॒रः स्या᳚थ् स्याद् वषट्का॒रो व॑षट्का॒रः स्या᳚त् । \newline
32. व॒ष॒ट्का॒र इति॑ वषट् - का॒रः । \newline
33. स्या॒द् यद् यथ् स्या᳚थ् स्या॒द् यत् । \newline
34. यन् न न यद् यन् न । \newline
35. न व॑षट्कु॒र्याद् व॑षट्कु॒र्यान् न न व॑षट्कु॒र्यात् । \newline
36. व॒ष॒ट्कु॒र्याद् रक्षाꣳ॑सि॒ रक्षाꣳ॑सि वषट्कु॒र्याद् व॑षट्कु॒र्याद् रक्षाꣳ॑सि । \newline
37. व॒ष॒ट्कु॒र्यादिति॑ वषट् - कु॒र्यात् । \newline
38. रक्षाꣳ॑सि य॒ज्ञ्ं ॅय॒ज्ञ्ꣳ रक्षाꣳ॑सि॒ रक्षाꣳ॑सि य॒ज्ञ्म् । \newline
39. य॒ज्ञ्ꣳ ह॑न्युर्. हन्युर् य॒ज्ञ्ं ॅय॒ज्ञ्ꣳ ह॑न्युः । \newline
40. ह॒न्यु॒र् वड् वड्ढ॑न्युर्. हन्यु॒र् वट् । \newline
41. वडितीति॒ वड् वडिति॑ । \newline
42. इत्या॑हा॒हे तीत्या॑ह । \newline
43. आ॒ह॒ प॒रोक्ष॑म् प॒रोक्ष॑ माहाह प॒रोक्ष᳚म् । \newline
44. प॒रोक्ष॑ मे॒वैव प॒रोक्ष॑म् प॒रोक्ष॑ मे॒व । \newline
45. प॒रोक्ष॒मिति॑ परः - अक्ष᳚म् । \newline
46. ए॒व वष॒ड् वष॑डे॒ वैव वष॑ट् । \newline
47. वष॑ट् करोति करोति॒ वष॒ड् वष॑ट् करोति । \newline
48. क॒रो॒ति॒ न न क॑रोति करोति॒ न । \newline
49. नास्या᳚स्य॒ न नास्य॑ । \newline
50. अ॒स्य॒ या॒तया॑मा या॒तया॑मा ऽस्यास्य या॒तया॑मा । \newline
51. या॒तया॑मा वषट्का॒रो व॑षट्का॒रो या॒तया॑मा या॒तया॑मा वषट्का॒रः । \newline
52. या॒तया॒मेति॑ या॒त - या॒मा॒ । \newline
53. व॒ष॒ट्का॒रो भव॑ति॒ भव॑ति वषट्का॒रो व॑षट्का॒रो भव॑ति । \newline
54. व॒ष॒ट्का॒र इति॑ वषट् - का॒रः । \newline
55. भव॑ति॒ न न भव॑ति॒ भव॑ति॒ न । \newline
56. न य॒ज्ञ्ं ॅय॒ज्ञ्म् न न य॒ज्ञ्म् । \newline
57. य॒ज्ञ्ꣳ रक्षाꣳ॑सि॒ रक्षाꣳ॑सि य॒ज्ञ्ं ॅय॒ज्ञ्ꣳ रक्षाꣳ॑सि । \newline
58. रक्षाꣳ॑सि घ्नन्ति घ्नन्ति॒ रक्षाꣳ॑सि॒ रक्षाꣳ॑सि घ्नन्ति । \newline
59. घ्न॒न्ति॒ हु॒तादो॑ हु॒तादो᳚ घ्नन्ति घ्नन्ति हु॒तादः॑ । \newline
60. हु॒तादो॒ वै वै हु॒तादो॑ हु॒तादो॒ वै । \newline
61. हु॒ताद॒ इति॑ हुत - अदः॑ । \newline
62. वा अ॒न्ये᳚ ऽन्ये वै वा अ॒न्ये । \newline
63. अ॒न्ये दे॒वा दे॒वा अ॒न्ये᳚ ऽन्ये दे॒वाः । \newline
64. दे॒वा अ॑हु॒तादो॑ ऽहु॒तादो॑ दे॒वा दे॒वा अ॑हु॒तादः॑ । \newline

\textbf{Ghana Paata } \newline

1. नृ॒षदे॒ वड् वण् णृ॒षदे॑ नृ॒षदे॒ वडितीति॒ वण् णृ॒षदे॑ नृ॒षदे॒ वडिति॑ । \newline
2. नृ॒षद॒ इति॑ नृ - सदे᳚ । \newline
3. वडितीति॒ वड् वडिति॒ व्याघा॑रयति॒ व्याघा॑रय॒ तीति॒ वड् वडिति॒ व्याघा॑रयति । \newline
4. इति॒ व्याघा॑रयति॒ व्याघा॑रय॒ तीतीति॒ व्याघा॑रयति प॒ङ्क्त्या प॒ङ्क्त्या व्याघा॑रय॒ तीतीति॒ व्याघा॑रयति प॒ङ्क्त्या । \newline
5. व्याघा॑रयति प॒ङ्क्त्या प॒ङ्क्त्या व्याघा॑रयति॒ व्याघा॑रयति प॒ङ्क्त्या ऽऽहु॒त्या ऽऽहु॑त्या प॒ङ्क्त्या व्याघा॑रयति॒ व्याघा॑रयति प॒ङ्क्त्या ऽऽहु॑त्या । \newline
6. व्याघा॑रय॒तीति॑ वि - आघा॑रयति । \newline
7. प॒ङ्क्त्या ऽऽहु॒त्या ऽऽहु॑त्या प॒ङ्क्त्या प॒ङ्क्त्या ऽऽहु॑त्या यज्ञ्मु॒खं ॅय॑ज्ञ्मु॒ख माहु॑त्या प॒ङ्क्त्या प॒ङ्क्त्या ऽऽहु॑त्या यज्ञ्मु॒खम् । \newline
8. आहु॑त्या यज्ञ्मु॒खं ॅय॑ज्ञ्मु॒ख माहु॒त्या ऽऽहु॑त्या यज्ञ्मु॒ख मा य॑ज्ञ्मु॒ख माहु॒त्या ऽऽहु॑त्या यज्ञ्मु॒ख मा । \newline
9. आहु॒त्येत्या - हु॒त्या॒ । \newline
10. य॒ज्ञ्॒मु॒ख मा य॑ज्ञ्मु॒खं ॅय॑ज्ञ्मु॒ख मा र॑भते रभत॒ आ य॑ज्ञ्मु॒खं ॅय॑ज्ञ्मु॒ख मा र॑भते । \newline
11. य॒ज्ञ्॒मु॒खमिति॑ यज्ञ् - मु॒खम् । \newline
12. आ र॑भते रभत॒ आ र॑भते ऽक्ष्ण॒या ऽक्ष्ण॒या र॑भत॒ आ र॑भते ऽक्ष्ण॒या । \newline
13. र॒भ॒ते॒ ऽक्ष्ण॒या ऽक्ष्ण॒या र॑भते रभते ऽक्ष्ण॒या व्याघा॑रयति॒ व्याघा॑रय त्यक्ष्ण॒या र॑भते रभते ऽक्ष्ण॒या व्याघा॑रयति । \newline
14. अ॒क्ष्ण॒या व्याघा॑रयति॒ व्याघा॑रय त्यक्ष्ण॒या ऽक्ष्ण॒या व्याघा॑रयति॒ तस्मा॒त् तस्मा॒द् व्याघा॑रय त्यक्ष्ण॒या ऽक्ष्ण॒या व्याघा॑रयति॒ तस्मा᳚त् । \newline
15. व्याघा॑रयति॒ तस्मा॒त् तस्मा॒द् व्याघा॑रयति॒ व्याघा॑रयति॒ तस्मा॑ दक्ष्ण॒या ऽक्ष्ण॒या तस्मा॒द् व्याघा॑रयति॒ व्याघा॑रयति॒ तस्मा॑ दक्ष्ण॒या । \newline
16. व्याघा॑रय॒तीति॑ वि - आघा॑रयति । \newline
17. तस्मा॑ दक्ष्ण॒या ऽक्ष्ण॒या तस्मा॒त् तस्मा॑ दक्ष्ण॒या प॒शवः॑ प॒शवो᳚ ऽक्ष्ण॒या तस्मा॒त् तस्मा॑ दक्ष्ण॒या प॒शवः॑ । \newline
18. अ॒क्ष्ण॒या प॒शवः॑ प॒शवो᳚ ऽक्ष्ण॒या ऽक्ष्ण॒या प॒शवो ऽङ्गा॒ न्यङ्गा॑नि प॒शवो᳚ ऽक्ष्ण॒या ऽक्ष्ण॒या प॒शवो ऽङ्गा॑नि । \newline
19. प॒शवो ऽङ्गा॒ न्यङ्गा॑नि प॒शवः॑ प॒शवो ऽङ्गा॑नि॒ प्र प्राङ्गा॑नि प॒शवः॑ प॒शवो ऽङ्गा॑नि॒ प्र । \newline
20. अङ्गा॑नि॒ प्र प्राङ्गा॒ न्यङ्गा॑नि॒ प्र ह॑रन्ति हरन्ति॒ प्राङ्गा॒ न्यङ्गा॑नि॒ प्र ह॑रन्ति । \newline
21. प्र ह॑रन्ति हरन्ति॒ प्र प्र ह॑रन्ति॒ प्रति॑ष्ठित्यै॒ प्रति॑ष्ठित्यै हरन्ति॒ प्र प्र ह॑रन्ति॒ प्रति॑ष्ठित्यै । \newline
22. ह॒र॒न्ति॒ प्रति॑ष्ठित्यै॒ प्रति॑ष्ठित्यै हरन्ति हरन्ति॒ प्रति॑ष्ठित्यै॒ यद् यत् प्रति॑ष्ठित्यै हरन्ति हरन्ति॒ प्रति॑ष्ठित्यै॒ यत् । \newline
23. प्रति॑ष्ठित्यै॒ यद् यत् प्रति॑ष्ठित्यै॒ प्रति॑ष्ठित्यै॒ यद् व॑षट्कु॒र्याद् व॑षट्कु॒र्याद् यत् प्रति॑ष्ठित्यै॒ प्रति॑ष्ठित्यै॒ यद् व॑षट्कु॒र्यात् । \newline
24. प्रति॑ष्ठित्या॒ इति॒ प्रति॑ - स्थि॒त्यै॒ । \newline
25. यद् व॑षट्कु॒र्याद् व॑षट्कु॒र्याद् यद् यद् व॑षट्कु॒र्याद् या॒तया॑मा या॒तया॑मा वषट्कु॒र्याद् यद् यद् व॑षट्कु॒र्याद् या॒तया॑मा । \newline
26. व॒ष॒ट्कु॒र्याद् या॒तया॑मा या॒तया॑मा वषट्कु॒र्याद् व॑षट्कु॒र्याद् या॒तया॑मा ऽस्यास्य या॒तया॑मा वषट्कु॒र्याद् व॑षट्कु॒र्याद् या॒तया॑मा ऽस्य । \newline
27. व॒ष॒ट्कु॒र्यादिति॑ वषट् - कु॒र्यात् । \newline
28. या॒तया॑मा ऽस्यास्य या॒तया॑मा या॒तया॑मा ऽस्य वषट्का॒रो व॑षट्का॒रो᳚ ऽस्य या॒तया॑मा या॒तया॑मा ऽस्य वषट्का॒रः । \newline
29. या॒तया॒मेति॑ या॒त - या॒मा॒ । \newline
30. अ॒स्य॒ व॒ष॒ट्का॒रो व॑षट्का॒रो᳚ ऽस्यास्य वषट्का॒रः स्या᳚थ् स्याद् वषट्का॒रो᳚ ऽस्यास्य वषट्का॒रः स्या᳚त् । \newline
31. व॒ष॒ट्का॒रः स्या᳚थ् स्याद् वषट्का॒रो व॑षट्का॒रः स्या॒द् यद् यथ् स्या᳚द् वषट्का॒रो व॑षट्का॒रः स्या॒द् यत् । \newline
32. व॒ष॒ट्का॒र इति॑ वषट् - का॒रः । \newline
33. स्या॒द् यद् यथ् स्या᳚थ् स्या॒द् यन् न न यथ् स्या᳚थ् स्या॒द् यन् न । \newline
34. यन् न न यद् यन् न व॑षट्कु॒र्याद् व॑षट्कु॒र्यान् न यद् यन् न व॑षट्कु॒र्यात् । \newline
35. न व॑षट्कु॒र्याद् व॑षट्कु॒र्यान् न न व॑षट्कु॒र्याद् रक्षाꣳ॑सि॒ रक्षाꣳ॑सि वषट्कु॒र्यान् न न व॑षट्कु॒र्याद् रक्षाꣳ॑सि । \newline
36. व॒ष॒ट्कु॒र्याद् रक्षाꣳ॑सि॒ रक्षाꣳ॑सि वषट्कु॒र्याद् व॑षट्कु॒र्याद् रक्षाꣳ॑सि य॒ज्ञ्ं ॅय॒ज्ञ्ꣳ रक्षाꣳ॑सि वषट्कु॒र्याद् व॑षट्कु॒र्याद् रक्षाꣳ॑सि य॒ज्ञ्म् । \newline
37. व॒ष॒ट्कु॒र्यादिति॑ वषट् - कु॒र्यात् । \newline
38. रक्षाꣳ॑सि य॒ज्ञ्ं ॅय॒ज्ञ्ꣳ रक्षाꣳ॑सि॒ रक्षाꣳ॑सि य॒ज्ञ्ꣳ ह॑न्युर्. हन्युर् य॒ज्ञ्ꣳ रक्षाꣳ॑सि॒ रक्षाꣳ॑सि य॒ज्ञ्ꣳ ह॑न्युः । \newline
39. य॒ज्ञ्ꣳ ह॑न्युर्. हन्युर् य॒ज्ञ्ं ॅय॒ज्ञ्ꣳ ह॑न्यु॒र् वड् वड्ढ॑न्युर् य॒ज्ञ्ं ॅय॒ज्ञ्ꣳ ह॑न्यु॒र् वट् । \newline
40. ह॒न्यु॒र् वड् वड्ढ॑न्युर्. हन्यु॒र् वडितीति॒ वड्ढ॑न्युर्. हन्यु॒र् वडिति॑ । \newline
41. वडितीति॒ वड् वडित्या॑ हा॒हेति॒ वड् वडित्या॑ह । \newline
42. इत्या॑हा॒हे तीत्या॑ह प॒रोक्ष॑म् प॒रोक्ष॑ मा॒हे तीत्या॑ह प॒रोक्ष᳚म् । \newline
43. आ॒ह॒ प॒रोक्ष॑म् प॒रोक्ष॑ माहाह प॒रोक्ष॑ मे॒वैव प॒रोक्ष॑ माहाह प॒रोक्ष॑ मे॒व । \newline
44. प॒रोक्ष॑ मे॒वैव प॒रोक्ष॑म् प॒रोक्ष॑ मे॒व वष॒ड् वष॑ डे॒व प॒रोक्ष॑म् प॒रोक्ष॑ मे॒व वष॑ट् । \newline
45. प॒रोक्ष॒मिति॑ परः - अक्ष᳚म् । \newline
46. ए॒व वष॒ड् वष॑ डे॒वैव वष॑ट् करोति करोति॒ वष॑ डे॒वैव वष॑ट् करोति । \newline
47. वष॑ट् करोति करोति॒ वष॒ड् वष॑ट् करोति॒ न न क॑रोति॒ वष॒ड् वष॑ट् करोति॒ न । \newline
48. क॒रो॒ति॒ न न क॑रोति करोति॒ नास्या᳚स्य॒ न क॑रोति करोति॒ नास्य॑ । \newline
49. नास्या᳚स्य॒ न नास्य॑ या॒तया॑मा या॒तया॑मा ऽस्य॒ न नास्य॑ या॒तया॑मा । \newline
50. अ॒स्य॒ या॒तया॑मा या॒तया॑मा ऽस्यास्य या॒तया॑मा वषट्का॒रो व॑षट्का॒रो या॒तया॑मा ऽस्यास्य या॒तया॑मा वषट्का॒रः । \newline
51. या॒तया॑मा वषट्का॒रो व॑षट्का॒रो या॒तया॑मा या॒तया॑मा वषट्का॒रो भव॑ति॒ भव॑ति वषट्का॒रो या॒तया॑मा या॒तया॑मा वषट्का॒रो भव॑ति । \newline
52. या॒तया॒मेति॑ या॒त - या॒मा॒ । \newline
53. व॒ष॒ट्का॒रो भव॑ति॒ भव॑ति वषट्का॒रो व॑षट्का॒रो भव॑ति॒ न न भव॑ति वषट्का॒रो व॑षट्का॒रो भव॑ति॒ न । \newline
54. व॒ष॒ट्का॒र इति॑ वषट् - का॒रः । \newline
55. भव॑ति॒ न न भव॑ति॒ भव॑ति॒ न य॒ज्ञ्ं ॅय॒ज्ञ्म् न भव॑ति॒ भव॑ति॒ न य॒ज्ञ्म् । \newline
56. न य॒ज्ञ्ं ॅय॒ज्ञ्म् न न य॒ज्ञ्ꣳ रक्षाꣳ॑सि॒ रक्षाꣳ॑सि य॒ज्ञ्म् न न य॒ज्ञ्ꣳ रक्षाꣳ॑सि । \newline
57. य॒ज्ञ्ꣳ रक्षाꣳ॑सि॒ रक्षाꣳ॑सि य॒ज्ञ्ं ॅय॒ज्ञ्ꣳ रक्षाꣳ॑सि घ्नन्ति घ्नन्ति॒ रक्षाꣳ॑सि य॒ज्ञ्ं ॅय॒ज्ञ्ꣳ रक्षाꣳ॑सि घ्नन्ति । \newline
58. रक्षाꣳ॑सि घ्नन्ति घ्नन्ति॒ रक्षाꣳ॑सि॒ रक्षाꣳ॑सि घ्नन्ति हु॒तादो॑ हु॒तादो᳚ घ्नन्ति॒ रक्षाꣳ॑सि॒ रक्षाꣳ॑सि घ्नन्ति हु॒तादः॑ । \newline
59. घ्न॒न्ति॒ हु॒तादो॑ हु॒तादो᳚ घ्नन्ति घ्नन्ति हु॒तादो॒ वै वै हु॒तादो᳚ घ्नन्ति घ्नन्ति हु॒तादो॒ वै । \newline
60. हु॒तादो॒ वै वै हु॒तादो॑ हु॒तादो॒ वा अ॒न्ये᳚ ऽन्ये वै हु॒तादो॑ हु॒तादो॒ वा अ॒न्ये । \newline
61. हु॒ताद॒ इति॑ हुत - अदः॑ । \newline
62. वा अ॒न्ये᳚ ऽन्ये वै वा अ॒न्ये दे॒वा दे॒वा अ॒न्ये वै वा अ॒न्ये दे॒वाः । \newline
63. अ॒न्ये दे॒वा दे॒वा अ॒न्ये᳚ ऽन्ये दे॒वा अ॑हु॒तादो॑ ऽहु॒तादो॑ दे॒वा अ॒न्ये᳚ ऽन्ये दे॒वा अ॑हु॒तादः॑ । \newline
64. दे॒वा अ॑हु॒तादो॑ ऽहु॒तादो॑ दे॒वा दे॒वा अ॑हु॒तादो॒ ऽन्ये᳚(1॒) ऽन्ये॑ ऽहु॒तादो॑ दे॒वा दे॒वा अ॑हु॒तादो॒ ऽन्ये । \newline
\pagebreak
\markright{ TS 5.4.5.2  \hfill https://www.vedavms.in \hfill}

\section{ TS 5.4.5.2 }

\textbf{TS 5.4.5.2 } \newline
\textbf{Samhita Paata} \newline

अ॑हु॒तादो॒ऽन्ये तान॑ग्नि॒चिदे॒वोभया᳚न् प्रीणाति॒ ये दे॒वा दे॒वाना॒मिति॑ द॒द्ध्ना म॑धुमि॒श्रेणावो᳚क्षति हु॒ताद॑श्चै॒व दे॒वान॑हु॒ताद॑श्च॒ यज॑मानः प्रीणाति॒ ते यज॑मानं प्रीणन्ति द॒द्ध्नैव हु॒तादः॑ प्री॒णाति॒ मधु॑षा ऽहु॒तादो᳚ ग्रा॒म्यं ॅवा ए॒तदन्नं॒ ॅयद्दद्ध्या॑र॒ण्यं मधु॒ यद्द॒ध्ना म॑धुमि॒श्रेणा॒-वोक्ष॑त्यु॒भय॒स्याऽव॑रुद्ध्यै ग्रुमु॒ष्टिनाऽवो᳚क्षति प्राजाप॒त्यो - [  ] \newline

\textbf{Pada Paata} \newline

अ॒हु॒ताद॒ इत्य॑हुत - अदः॑ । अ॒न्ये । तान् । अ॒ग्नि॒चिदित्य॑ग्नि-चित् । ए॒व । उ॒भयान्॑ । प्री॒णा॒ति॒ । ये । दे॒वाः । दे॒वाना᳚म् । इति॑ । द॒द्ध्ना । म॒धु॒मि॒श्रेणेति॑ मधु - मि॒श्रेण॑ । अवेति॑ । उ॒क्ष॒ति॒ । हु॒ताद॒ इति॑ हुत - अदः॑ । च॒ । ए॒व । दे॒वान् । अ॒हु॒ताद॒ इत्य॑हुत - अदः॑ । च॒ । यज॑मानः । प्री॒णा॒ति॒ । ते । यज॑मानम् । प्री॒ण॒न्ति॒ । द॒द्ध्ना । ए॒व । हु॒ताद॒ इति॑ हुत - अदः॑ । प्री॒णाति॑ । मधु॑षा । अ॒हु॒ताद॒ इत्य॑हुत - अदः॑ । ग्रा॒म्यम् । वै । ए॒तत् । अन्न᳚म् । यत् । दधि॑ । आ॒र॒ण्यम् । मधु॑ । यत् । द॒द्ध्ना । म॒धु॒मि॒श्रेणेति॑ मधु - मि॒श्रेण॑ । अ॒वोक्ष॒तीत्य॑व - उक्ष॑ति । उ॒भय॑स्य । अव॑रुद्ध्या॒ इत्यव॑ - रु॒ध्यै॒ । ग्रु॒मु॒ष्टिना᳚ । अवेति॑ । उ॒क्ष॒ति॒ । प्रा॒जा॒प॒त्य इति॑ प्राजा-प॒त्यः ।  \newline


\textbf{Krama Paata} \newline

अ॒हु॒तादो॒ऽन्ये । अ॒हु॒ताद॒ इत्य॑हुत - अदः॑ । अ॒न्ये तान् । तान॑ग्नि॒चित् । अ॒ग्नि॒चिदे॒व । अ॒ग्नि॒चिदित्य॑ग्नि - चित् । ए॒वोभयान्॑ । उ॒भया᳚न् प्रीणाति । प्री॒णा॒ति॒ ये । ये दे॒वाः । दे॒वा दे॒वाना᳚म् । दे॒वाना॒मिति॑ । इति॑ द॒द्ध्ना । द॒द्ध्ना म॑धुमि॒श्रेण॑ । म॒धु॒मि॒श्रेणाव॑ । म॒धु॒मि॒श्रेणेति॑ मधु - मि॒श्रेण॑ । अवो᳚क्षति । उ॒क्ष॒ति॒ हु॒तादः॑ । हु॒ताद॑श्च । हु॒ताद॒ इति॑ हुत - अदः॑ । चै॒व । ए॒व दे॒वान् । दे॒वान॑हु॒तादः॑ । अ॒हु॒ताद॑श्च । अ॒हु॒ताद॒ इत्य॑हुत - अदः॑ । च॒ यज॑मानः । यज॑मानः प्रीणाति । प्री॒णा॒ति॒ ते । ते यज॑मानम् । यज॑मानम् प्रीणन्ति । प्री॒ण॒न्ति॒ द॒द्ध्ना । द॒द्ध्नैव । ए॒व हु॒तादः॑ । हु॒तादः॑ प्री॒णाति॑ । हु॒ताद॒ इति॑ हुत - अदः॑ । प्री॒णाति॒ मधु॑षा । मधु॑षाऽहु॒तादः॑ । अ॒हु॒तादो᳚ ग्रा॒म्यम् । अ॒हु॒ताद॒ इत्य॑हुत - अदः॑ । ग्रा॒म्यम् ॅवै । वा ए॒तत् । ए॒तदन्न᳚म् । अन्न॒म् ॅयत् । यद् दधि॑ । दद्ध्या॑र॒ण्यम् । आ॒र॒ण्यम् मधु॑ । मधु॒ यत् । यद् द॒द्ध्ना । द॒द्ध्ना म॑धुमि॒श्रेण॑ । म॒धु॒मि॒श्रेणा॒वोक्ष॑ति । म॒धु॒मि॒श्रेणेति॑ मधु - मि॒श्रेण॑ । अ॒वोक्ष॑त्यु॒भय॑स्य । अ॒वोक्ष॒तीत्य॑व - उक्ष॑ति । उ॒भय॒स्याव॑रुद्ध्यै । अव॑रुद्ध्यै ग्रुमु॒ष्टिना᳚ । अव॑रुद्ध्या॒ इत्यव॑ - रु॒द्ध्यै॒ । ग्रु॒मु॒ष्टिनाऽव॑ । अवो᳚क्षति । उ॒क्ष॒ति॒ प्रा॒जा॒प॒त्यः । प्रा॒जा॒प॒त्यो वै । प्रा॒जा॒प॒त्य इति॑ प्राजा - प॒त्यः \newline

\textbf{Jatai Paata} \newline

1. अ॒हु॒तादो॒ ऽन्ये᳚(1॒) ऽन्ये॑ ऽहु॒तादो॑ ऽहु॒तादो॒ ऽन्ये । \newline
2. अ॒हु॒ताद॒ इत्य॑हुत - अदः॑ । \newline
3. अ॒न्ये ताꣳ स्ता न॒न्ये᳚ ऽन्ये तान् । \newline
4. ता न॑ग्नि॒चि द॑ग्नि॒चित् ताꣳ स्ता न॑ग्नि॒चित् । \newline
5. अ॒ग्नि॒चि दे॒वै वाग्नि॒चि द॑ग्नि॒चि दे॒व । \newline
6. अ॒ग्नि॒चिदित्य॑ग्नि - चित् । \newline
7. ए॒वोभया॑ नु॒भया॑ ने॒वैवोभयान्॑ । \newline
8. उ॒भया᳚न् प्रीणाति प्रीणा त्यु॒भया॑ नु॒भया᳚न् प्रीणाति । \newline
9. प्री॒णा॒ति॒ ये ये प्री॑णाति प्रीणाति॒ ये । \newline
10. ये दे॒वा दे॒वा ये ये दे॒वाः । \newline
11. दे॒वा दे॒वाना᳚म् दे॒वाना᳚म् दे॒वा दे॒वा दे॒वाना᳚म् । \newline
12. दे॒वाना॒ मितीति॑ दे॒वाना᳚म् दे॒वाना॒ मिति॑ । \newline
13. इति॑ द॒द्ध्ना द॒द्ध्ने तीति॑ द॒द्ध्ना । \newline
14. द॒द्ध्ना म॑धुमि॒श्रेण॑ मधुमि॒श्रेण॑ द॒द्ध्ना द॒द्ध्ना म॑धुमि॒श्रेण॑ । \newline
15. म॒धु॒मि॒श्रेणावाव॑ मधुमि॒श्रेण॑ मधुमि॒श्रेणाव॑ । \newline
16. म॒धु॒मि॒श्रेणेति॑ मधु - मि॒श्रेण॑ । \newline
17. अवो᳚क्ष त्युक्ष॒ त्यवावो᳚क्षति । \newline
18. उ॒क्ष॒ति॒ हु॒तादो॑ हु॒ताद॑ उक्ष त्युक्षति हु॒तादः॑ । \newline
19. हु॒ताद॑श्च च हु॒तादो॑ हु॒ताद॑श्च । \newline
20. हु॒ताद॒ इति॑ हुत - अदः॑ । \newline
21. चै॒वैव च॑ चै॒व । \newline
22. ए॒व दे॒वान् दे॒वा ने॒वैव दे॒वान् । \newline
23. दे॒वा न॑हु॒तादो॑ ऽहु॒तादो॑ दे॒वान् दे॒वा न॑हु॒तादः॑ । \newline
24. अ॒हु॒ताद॑श्च चाहु॒तादो॑ ऽहु॒ताद॑श्च । \newline
25. अ॒हु॒ताद॒ इत्य॑हुत - अदः॑ । \newline
26. च॒ यज॑मानो॒ यज॑मानश्च च॒ यज॑मानः । \newline
27. यज॑मानः प्रीणाति प्रीणाति॒ यज॑मानो॒ यज॑मानः प्रीणाति । \newline
28. प्री॒णा॒ति॒ ते ते प्री॑णाति प्रीणाति॒ ते । \newline
29. ते यज॑मानं॒ ॅयज॑मान॒म् ते ते यज॑मानम् । \newline
30. यज॑मानम् प्रीणन्ति प्रीणन्ति॒ यज॑मानं॒ ॅयज॑मानम् प्रीणन्ति । \newline
31. प्री॒ण॒न्ति॒ द॒द्ध्ना द॒द्ध्ना प्री॑णन्ति प्रीणन्ति द॒द्ध्ना । \newline
32. द॒द्ध्नै वैव द॒द्ध्ना द॒द्ध्नैव । \newline
33. ए॒व हु॒तादो॑ हु॒ताद॑ ए॒वैव हु॒तादः॑ । \newline
34. हु॒तादः॑ प्री॒णाति॑ प्री॒णाति॑ हु॒तादो॑ हु॒तादः॑ प्री॒णाति॑ । \newline
35. हु॒ताद॒ इति॑ हुत - अदः॑ । \newline
36. प्री॒णाति॒ मधु॑षा॒ मधु॑षा प्री॒णाति॑ प्री॒णाति॒ मधु॑षा । \newline
37. मधु॑षा ऽहु॒तादो॑ ऽहु॒तादो॒ मधु॑षा॒ मधु॑षा ऽहु॒तादः॑ । \newline
38. अ॒हु॒तादो᳚ ग्रा॒म्यम् ग्रा॒म्य म॑हु॒तादो॑ ऽहु॒तादो᳚ ग्रा॒म्यम् । \newline
39. अ॒हु॒ताद॒ इत्य॑हुत - अदः॑ । \newline
40. ग्रा॒म्यं ॅवै वै ग्रा॒म्यम् ग्रा॒म्यं ॅवै । \newline
41. वा ए॒त दे॒तद् वै वा ए॒तत् । \newline
42. ए॒त दन्न॒ मन्न॑ मे॒त दे॒त दन्न᳚म् । \newline
43. अन्नं॒ ॅयद् यदन्न॒ मन्नं॒ ॅयत् । \newline
44. यद् दधि॒ दधि॒ यद् यद् दधि॑ । \newline
45. दध्या॑र॒ण्य मा॑र॒ण्यम् दधि॒ दध्या॑र॒ण्यम् । \newline
46. आ॒र॒ण्यम् मधु॒ मध्वा॑र॒ण्य मा॑र॒ण्यम् मधु॑ । \newline
47. मधु॒ यद् यन् मधु॒ मधु॒ यत् । \newline
48. यद् द॒द्ध्ना द॒द्ध्ना यद् यद् द॒द्ध्ना । \newline
49. द॒द्ध्ना म॑धुमि॒श्रेण॑ मधुमि॒श्रेण॑ द॒द्ध्ना द॒द्ध्ना म॑धुमि॒श्रेण॑ । \newline
50. म॒धु॒मि॒श्रेणा॒ वोक्ष॑ त्य॒वोक्ष॑ति मधुमि॒श्रेण॑ मधुमि॒श्रेणा॒ वोक्ष॑ति । \newline
51. म॒धु॒मि॒श्रेणेति॑ मधु - मि॒श्रेण॑ । \newline
52. अ॒वोक्ष॑ त्यु॒भय॑ स्यो॒भय॑स्या॒ वोक्ष॑ त्य॒वोक्ष॑ त्यु॒भय॑स्य । \newline
53. अ॒वोक्ष॒तीत्य॑व - उक्ष॑ति । \newline
54. उ॒भय॒स्या व॑रुद्ध्या॒ अव॑रुद्ध्या उ॒भय॑ स्यो॒भय॒स्या व॑रुद्ध्यै । \newline
55. अव॑रुद्ध्यै ग्रुमु॒ष्टिना᳚ ग्रुमु॒ष्टिना ऽव॑रुद्ध्या॒ अव॑रुद्ध्यै ग्रुमु॒ष्टिना᳚ । \newline
56. अव॑रुद्ध्या॒ इत्यव॑ - रु॒ध्यै॒ । \newline
57. ग्रु॒मु॒ष्टिना ऽवाव॑ ग्रुमु॒ष्टिना᳚ ग्रुमु॒ष्टिना ऽव॑ । \newline
58. अवो᳚क्ष त्युक्ष॒ त्यवावो᳚ क्षति । \newline
59. उ॒क्ष॒ति॒ प्रा॒जा॒प॒त्यः प्रा॑जाप॒त्य उ॑क्ष त्युक्षति प्राजाप॒त्यः । \newline
60. प्रा॒जा॒प॒त्यो वै वै प्रा॑जाप॒त्यः प्रा॑जाप॒त्यो वै । \newline
61. प्रा॒जा॒प॒त्य इति॑ प्राजा - प॒त्यः । \newline

\textbf{Ghana Paata } \newline

1. अ॒हु॒तादो॒ ऽन्ये᳚(1॒) ऽन्ये॑ ऽहु॒तादो॑ ऽहु॒तादो॒ ऽन्ये ताꣳ स्ता न॒न्ये॑ ऽहु॒तादो॑ ऽहु॒तादो॒ ऽन्ये तान् । \newline
2. अ॒हु॒ताद॒ इत्य॑हुत - अदः॑ । \newline
3. अ॒न्ये ताꣳ स्ता न॒न्ये᳚ ऽन्ये ता न॑ग्नि॒चि द॑ग्नि॒चित् ता न॒न्ये᳚ ऽन्ये ता न॑ग्नि॒चित् । \newline
4. ता न॑ग्नि॒चि द॑ग्नि॒चित् ताꣳ स्ता न॑ग्नि॒चि दे॒वै वाग्नि॒चित् ताꣳ स्ता न॑ग्नि॒चि दे॒व । \newline
5. अ॒ग्नि॒चि दे॒वैवाग्नि॒चि द॑ग्नि॒चिदे॒वो भया॑ नु॒भया॑ ने॒वाग्नि॒चि द॑ग्नि॒चि दे॒वो भयान्॑ । \newline
6. अ॒ग्नि॒चिदित्य॑ग्नि - चित् । \newline
7. ए॒वो भया॑ नु॒भया॑ ने॒वैवोभया᳚न् प्रीणाति प्रीणा त्यु॒भया॑ ने॒वैवोभया᳚न् प्रीणाति । \newline
8. उ॒भया᳚न् प्रीणाति प्रीणा त्यु॒भया॑ नु॒भया᳚न् प्रीणाति॒ ये ये प्री॑णा त्यु॒भया॑ नु॒भया᳚न् प्रीणाति॒ ये । \newline
9. प्री॒णा॒ति॒ ये ये प्री॑णाति प्रीणाति॒ ये दे॒वा दे॒वा ये प्री॑णाति प्रीणाति॒ ये दे॒वाः । \newline
10. ये दे॒वा दे॒वा ये ये दे॒वा दे॒वाना᳚म् दे॒वाना᳚म् दे॒वा ये ये दे॒वा दे॒वाना᳚म् । \newline
11. दे॒वा दे॒वाना᳚म् दे॒वाना᳚म् दे॒वा दे॒वा दे॒वाना॒ मितीति॑ दे॒वाना᳚म् दे॒वा दे॒वा दे॒वाना॒ मिति॑ । \newline
12. दे॒वाना॒ मितीति॑ दे॒वाना᳚म् दे॒वाना॒ मिति॑ द॒द्ध्ना द॒द्ध्नेति॑ दे॒वाना᳚म् दे॒वाना॒ मिति॑ द॒द्ध्ना । \newline
13. इति॑ द॒द्ध्ना द॒द्ध्नेतीति॑ द॒द्ध्ना म॑धुमि॒श्रेण॑ मधुमि॒श्रेण॑ द॒द्ध्नेतीति॑ द॒द्ध्ना म॑धुमि॒श्रेण॑ । \newline
14. द॒द्ध्ना म॑धुमि॒श्रेण॑ मधुमि॒श्रेण॑ द॒द्ध्ना द॒द्ध्ना म॑धुमि॒श्रेणा वाव॑ मधुमि॒श्रेण॑ द॒द्ध्ना द॒द्ध्ना म॑धुमि॒श्रेणाव॑ । \newline
15. म॒धु॒मि॒श्रेणा वाव॑ मधुमि॒श्रेण॑ मधुमि॒श्रेणा वो᳚क्ष त्युक्ष॒त्यव॑ मधुमि॒श्रेण॑ मधुमि॒श्रेणा
वो᳚क्षति । \newline
16. म॒धु॒मि॒श्रेणेति॑ मधु - मि॒श्रेण॑ । \newline
17. अवो᳚क्ष त्युक्ष॒ त्यवावो᳚ क्षति हु॒तादो॑ हु॒ताद॑ उक्ष॒ त्यवावो᳚ क्षति हु॒तादः॑ । \newline
18. उ॒क्ष॒ति॒ हु॒तादो॑ हु॒ताद॑ उक्ष त्युक्षति हु॒ताद॑श्च च हु॒ताद॑ उक्ष त्युक्षति हु॒ताद॑श्च । \newline
19. हु॒ताद॑श्च च हु॒तादो॑ हु॒ताद॑ श्चै॒वैव च॑ हु॒तादो॑ हु॒ताद॑ श्चै॒व । \newline
20. हु॒ताद॒ इति॑ हुत - अदः॑ । \newline
21. चै॒वैव च॑ चै॒व दे॒वान् दे॒वा ने॒व च॑ चै॒व दे॒वान् । \newline
22. ए॒व दे॒वान् दे॒वा ने॒वैव दे॒वा न॑हु॒तादो॑ ऽहु॒तादो॑ दे॒वा ने॒वैव दे॒वा न॑हु॒तादः॑ । \newline
23. दे॒वा न॑हु॒तादो॑ ऽहु॒तादो॑ दे॒वान् दे॒वा न॑हु॒ताद॑श्च चाहु॒तादो॑ दे॒वान् दे॒वा न॑हु॒ताद॑श्च । \newline
24. अ॒हु॒ताद॑श्च चाहु॒तादो॑ ऽहु॒ताद॑श्च॒ यज॑मानो॒ यज॑मान श्चाहु॒तादो॑ ऽहु॒ताद॑श्च॒ यज॑मानः । \newline
25. अ॒हु॒ताद॒ इत्य॑हुत - अदः॑ । \newline
26. च॒ यज॑मानो॒ यज॑मानश्च च॒ यज॑मानः प्रीणाति प्रीणाति॒ यज॑मानश्च च॒ यज॑मानः प्रीणाति । \newline
27. यज॑मानः प्रीणाति प्रीणाति॒ यज॑मानो॒ यज॑मानः प्रीणाति॒ ते ते प्री॑णाति॒ यज॑मानो॒ यज॑मानः प्रीणाति॒ ते । \newline
28. प्री॒णा॒ति॒ ते ते प्री॑णाति प्रीणाति॒ ते यज॑मानं॒ ॅयज॑मान॒म् ते प्री॑णाति प्रीणाति॒ ते यज॑मानम् । \newline
29. ते यज॑मानं॒ ॅयज॑मान॒म् ते ते यज॑मानम् प्रीणन्ति प्रीणन्ति॒ यज॑मान॒म् ते ते यज॑मानम् प्रीणन्ति । \newline
30. यज॑मानम् प्रीणन्ति प्रीणन्ति॒ यज॑मानं॒ ॅयज॑मानम् प्रीणन्ति द॒द्ध्ना द॒द्ध्ना प्री॑णन्ति॒ यज॑मानं॒ ॅयज॑मानम् प्रीणन्ति द॒द्ध्ना । \newline
31. प्री॒ण॒न्ति॒ द॒द्ध्ना द॒द्ध्ना प्री॑णन्ति प्रीणन्ति द॒द्ध्नैवैव द॒द्ध्ना प्री॑णन्ति प्रीणन्ति द॒द्ध्नैव । \newline
32. द॒द्ध्नै वैव द॒द्ध्ना द॒द्ध्नैव हु॒तादो॑ हु॒ताद॑ ए॒व द॒द्ध्ना द॒द्ध्नैव हु॒तादः॑ । \newline
33. ए॒व हु॒तादो॑ हु॒ताद॑ ए॒वैव हु॒तादः॑ प्री॒णाति॑ प्री॒णाति॑ हु॒ताद॑ ए॒वैव हु॒तादः॑ प्री॒णाति॑ । \newline
34. हु॒तादः॑ प्री॒णाति॑ प्री॒णाति॑ हु॒तादो॑ हु॒तादः॑ प्री॒णाति॒ मधु॑षा॒ मधु॑षा प्री॒णाति॑ हु॒तादो॑ हु॒तादः॑ प्री॒णाति॒ मधु॑षा । \newline
35. हु॒ताद॒ इति॑ हुत - अदः॑ । \newline
36. प्री॒णाति॒ मधु॑षा॒ मधु॑षा प्री॒णाति॑ प्री॒णाति॒ मधु॑षा ऽहु॒तादो॑ ऽहु॒तादो॒ मधु॑षा प्री॒णाति॑ प्री॒णाति॒ मधु॑षा ऽहु॒तादः॑ । \newline
37. मधु॑षा ऽहु॒तादो॑ ऽहु॒तादो॒ मधु॑षा॒ मधु॑षा ऽहु॒तादो᳚ ग्रा॒म्यम् ग्रा॒म्य म॑हु॒तादो॒ मधु॑षा॒ मधु॑षा ऽहु॒तादो᳚ ग्रा॒म्यम् । \newline
38. अ॒हु॒तादो᳚ ग्रा॒म्यम् ग्रा॒म्य म॑हु॒तादो॑ ऽहु॒तादो᳚ ग्रा॒म्यं ॅवै वै ग्रा॒म्य म॑हु॒तादो॑ ऽहु॒तादो᳚ ग्रा॒म्यं ॅवै । \newline
39. अ॒हु॒ताद॒ इत्य॑हुत - अदः॑ । \newline
40. ग्रा॒म्यं ॅवै वै ग्रा॒म्यम् ग्रा॒म्यं ॅवा ए॒त दे॒तद् वै ग्रा॒म्यम् ग्रा॒म्यं ॅवा ए॒तत् । \newline
41. वा ए॒त दे॒तद् वै वा ए॒त दन्न॒ मन्न॑ मे॒तद् वै वा ए॒त दन्न᳚म् । \newline
42. ए॒त दन्न॒ मन्न॑ मे॒त दे॒त दन्नं॒ ॅयद् यदन्न॑ मे॒त दे॒त दन्नं॒ ॅयत् । \newline
43. अन्नं॒ ॅयद् यदन्न॒ मन्नं॒ ॅयद् दधि॒ दधि॒ यदन्न॒ मन्नं॒ ॅयद् दधि॑ । \newline
44. यद् दधि॒ दधि॒ यद् यद् दध्या॑ र॒ण्य मा॑र॒ण्यम् दधि॒ यद् यद् दध्या॑ र॒ण्यम् । \newline
45. दध्या॑ र॒ण्य मा॑र॒ण्यम् दधि॒ दध्या॑ र॒ण्यम् मधु॒ मध्वा॑र॒ण्यम् दधि॒ दध्या॑ र॒ण्यम् मधु॑ । \newline
46. आ॒र॒ण्यम् मधु॒ मध्वा॑ र॒ण्य मा॑र॒ण्यम् मधु॒ यद् यन् मध्वा॑ र॒ण्य मा॑र॒ण्यम् मधु॒ यत् । \newline
47. मधु॒ यद् यन् मधु॒ मधु॒ यद् द॒द्ध्ना द॒द्ध्ना यन् मधु॒ मधु॒ यद् द॒द्ध्ना । \newline
48. यद् द॒द्ध्ना द॒द्ध्ना यद् यद् द॒द्ध्ना म॑धुमि॒श्रेण॑ मधुमि॒श्रेण॑ द॒द्ध्ना यद् यद् द॒द्ध्ना म॑धुमि॒श्रेण॑ । \newline
49. द॒द्ध्ना म॑धुमि॒श्रेण॑ मधुमि॒श्रेण॑ द॒द्ध्ना द॒द्ध्ना म॑धुमि॒श्रेणा॒ वोक्ष॑ त्य॒वोक्ष॑ति मधुमि॒श्रेण॑ द॒द्ध्ना द॒द्ध्ना म॑धुमि॒श्रेणा॒ वोक्ष॑ति । \newline
50. म॒धु॒मि॒श्रेणा॒ वोक्ष॑ त्य॒वोक्ष॑ति मधुमि॒श्रेण॑ मधुमि॒श्रेणा॒ वोक्ष॑ त्यु॒भय॑ स्यो॒भय॑स्या॒ वोक्ष॑ति मधुमि॒श्रेण॑ मधुमि॒श्रेणा॒ वोक्ष॑ त्यु॒भय॑स्य । \newline
51. म॒धु॒मि॒श्रेणेति॑ मधु - मि॒श्रेण॑ । \newline
52. अ॒वोक्ष॑ त्यु॒भय॑ स्यो॒भय॑स्या॒ वोक्ष॑ त्य॒वोक्ष॑ त्यु॒भय॒स्या व॑रुद्ध्या॒ अव॑रुद्ध्या उ॒भय॑स्या॒ वोक्ष॑ त्य॒वोक्ष॑ त्यु॒भय॒स्या व॑रुद्ध्यै । \newline
53. अ॒वोक्ष॒तीत्य॑व - उक्ष॑ति । \newline
54. उ॒भय॒स्या व॑रुद्ध्या॒ अव॑रुद्ध्या उ॒भय॑ स्यो॒भय॒स्या व॑रुद्ध्यै ग्रुमु॒ष्टिना᳚ ग्रुमु॒ष्टिना ऽव॑रुद्ध्या उ॒भय॑ स्यो॒भय॒स्या व॑रुद्ध्यै ग्रुमु॒ष्टिना᳚ । \newline
55. अव॑रुद्ध्यै ग्रुमु॒ष्टिना᳚ ग्रुमु॒ष्टिना ऽव॑रुद्ध्या॒ अव॑रुद्ध्यै ग्रुमु॒ष्टिना ऽवाव॑ ग्रुमु॒ष्टिना ऽव॑रुद्ध्या॒ अव॑रुद्ध्यै ग्रुमु॒ष्टिना ऽव॑ । \newline
56. अव॑रुद्ध्या॒ इत्यव॑ - रु॒ध्यै॒ । \newline
57. ग्रु॒मु॒ष्टिना ऽवाव॑ ग्रुमु॒ष्टिना᳚ ग्रुमु॒ष्टिना ऽवो᳚क्ष त्युक्ष॒ त्यव॑ ग्रुमु॒ष्टिना᳚ ग्रुमु॒ष्टिना ऽवो᳚क्षति । \newline
58. अवो᳚क्ष त्युक्ष॒ त्यवावो᳚क्षति प्राजाप॒त्यः प्रा॑जाप॒त्य उ॑क्ष॒ त्यवा वो᳚क्षति प्राजाप॒त्यः । \newline
59. उ॒क्ष॒ति॒ प्रा॒जा॒प॒त्यः प्रा॑जाप॒त्य उ॑क्ष त्युक्षति प्राजाप॒त्यो वै वै प्रा॑जाप॒त्य उ॑क्ष त्युक्षति प्राजाप॒त्यो वै । \newline
60. प्रा॒जा॒प॒त्यो वै वै प्रा॑जाप॒त्यः प्रा॑जाप॒त्यो वै ग्रु॑मु॒ष्टिर् ग्रु॑मु॒ष्टिर् वै प्रा॑जाप॒त्यः प्रा॑जाप॒त्यो वै ग्रु॑मु॒ष्टिः । \newline
61. प्रा॒जा॒प॒त्य इति॑ प्राजा - प॒त्यः । \newline
\pagebreak
\markright{ TS 5.4.5.3  \hfill https://www.vedavms.in \hfill}

\section{ TS 5.4.5.3 }

\textbf{TS 5.4.5.3 } \newline
\textbf{Samhita Paata} \newline

वै ग्रु॑मु॒ष्टिः स॑योनि॒त्वाय॒ द्वाभ्यां॒ प्रति॑ष्ठित्या अनुपरि॒चार॒-मवो᳚क्ष॒त्य-प॑रिवर्गमे॒वैना᳚न् प्रीणाति॒ वि वा ए॒ष प्रा॒णैः प्र॒जया॑ प॒शुभि॑र्.ऋद्ध्यते॒ यो᳚ऽग्निं चि॒न्वन्न॑धि॒क्राम॑ति प्राण॒दा अ॑पान॒दा इत्या॑ह प्रा॒णाने॒वाऽऽत्मन् ध॑त्ते वर्चो॒दा व॑रिवो॒दा इत्या॑ह प्र॒जा वै वर्चः॑ प॒शवो॒ वरि॑वः प्र॒जामे॒व प॒शूना॒त्मन् ध॑त्त॒ इन्द्रो॑ वृ॒त्रम॑ह॒न्तं ॅवृ॒त्रो - [  ] \newline

\textbf{Pada Paata} \newline

वै । ग्रु॒मु॒ष्टिः । स॒यो॒नि॒त्वायेति॑ सयोनि - त्वाय॑ । द्वाभ्या᳚म् । प्रति॑ष्ठित्या॒ इति॒ प्रति॑ - स्थि॒त्यै॒ । अ॒नु॒प॒रि॒चार॒मित्य॑नु - प॒रि॒चार᳚म् । अवेति॑ । उ॒क्ष॒ति॒ । अप॑रिवर्ग॒मित्यप॑रि - व॒र्ग॒म् । ए॒व । ए॒ना॒न् । प्री॒णा॒ति॒ । वीति॑ । वै । ए॒षः । प्रा॒णैरिति॑ प्र - अ॒नैः । प्र॒जयेति॑ प्र - जया᳚ । प॒शुभि॒रिति॑ प॒शु - भिः॒ । ऋ॒द्ध्य॒ते॒ । यः । अ॒ग्निम् । चि॒न्वन्न् । अ॒धि॒क्राम॒तीत्य॑धि - क्राम॑ति । प्रा॒ण॒दा इति॑ प्राण - दाः । अ॒पा॒न॒दा इत्य॑पान - दाः । इति॑ । आ॒ह॒ । प्रा॒णानिति॑ प्र - अ॒नान् । ए॒व । आ॒त्मन् । ध॒त्ते॒ । व॒र्चो॒दा इति॑ वर्चः - दाः । व॒रि॒वो॒दा इति॑ वरिवः - दाः । इति॑ । आ॒ह॒ । प्र॒जेति॑ प्र - जा । वै । वर्चः॑ । प॒शवः॑ । वरि॑वः । प्र॒जामिति॑ प्र - जाम् । ए॒व । प॒शून् । आ॒त्मन्न् । ध॒त्ते॒ । इन्द्रः॑ । वृ॒त्रम् । अ॒ह॒न्न् । तम् । वृ॒त्रः ।  \newline


\textbf{Krama Paata} \newline

वै ग्रु॑मु॒ष्टिः । ग्रु॒मु॒ष्टिः स॑योनि॒त्वाय॑ । स॒यो॒नि॒त्वाय॒ द्वाभ्या᳚म् । स॒यो॒नि॒त्वायेति॑ सयोनि - त्वाय॑ । द्वाभ्या॒म् प्रति॑ष्ठित्यै । प्रति॑ष्ठित्या अनुपरि॒चार᳚म् । प्रति॑ष्ठित्या॒ इति॒ प्रति॑ - स्थि॒त्यै॒ । अ॒नु॒प॒रि॒चार॒मव॑ । अ॒नु॒प॒रि॒चार॒मित्य॑नु - प॒रि॒चार᳚म् । अवो᳚क्षति । उ॒क्ष॒त्यप॑रिवर्गम् । अप॑रिवर्गमे॒व । अप॑रिवर्ग॒मित्यप॑रि - व॒र्ग॒म् । ए॒वैनान्॑ । ए॒ना॒न् प्री॒णा॒ति॒ । प्री॒णा॒ति॒ वि । वि वै । वा ए॒षः । ए॒ष प्रा॒णैः । प्रा॒णैः प्र॒जया᳚ । प्रा॒णैरिति॑ प्र - अ॒नैः । प्र॒जया॑ प॒शुभिः॑ । प्र॒जयेति॑ प्र - जया᳚ । प॒शुभि॑र्. ऋद्ध्यते । प॒शुभि॒रिति॑ प॒शु - भिः॒ । ऋ॒द्ध्य॒ते॒ यः । यो᳚ऽग्निम् । अ॒ग्निम् चि॒न्वन्न् । चि॒न्वन्न॑धि॒क्राम॑ति । अ॒धि॒क्राम॑ति प्राण॒दाः । अ॒धि॒क्राम॒तीत्य॑धि - क्राम॑ति । प्रा॒ण॒दा अ॑पान॒दाः । प्रा॒ण॒दा इति॑ प्राण - दाः । अ॒पा॒न॒दा इति॑ । अ॒पा॒न॒दा इत्य॑पान - दाः । इत्या॑ह । आ॒ह॒ प्रा॒णान् । प्रा॒णाने॒व । प्रा॒णानिति॑ प्र - अ॒नान् । ए॒वात्मन्न् । आ॒त्मन् ध॑त्ते । ध॒त्ते॒ व॒र्चो॒दाः । व॒र्चो॒दा व॑रिवो॒दाः । व॒र्चो॒दा इति॑ वर्चः - दाः । व॒रि॒वो॒दा इति॑ । व॒रि॒वो॒दा इति॑ वरिवः - दाः । इत्या॑ह । आ॒ह॒ प्र॒जा । प्र॒जा वै । प्र॒जेति॑ प्र - जा । वै वर्चः॑ । वर्चः॑ प॒शवः॑ । प॒शवो॒ वरि॑वः । वरि॑वः प्र॒जाम् । प्र॒जामे॒व । प्र॒जामिति॑ प्र - जाम् । ए॒व प॒शून् । प॒शूना॒त्मन्न् । आ॒त्मन् ध॑त्ते । ध॒त्त॒ इन्द्रः॑ । इन्द्रो॑ वृ॒त्रम् । वृ॒त्रम॑हन्न् । अ॒ह॒न् तम् । तम् ॅवृ॒त्रः । वृ॒त्रो ह॒तः \newline

\textbf{Jatai Paata} \newline

1. वै ग्रु॑मु॒ष्टिर् ग्रु॑मु॒ष्टिर् वै वै ग्रु॑मु॒ष्टिः । \newline
2. ग्रु॒मु॒ष्टिः स॑योनि॒त्वाय॑ सयोनि॒त्वाय॑ ग्रुमु॒ष्टिर् ग्रु॑मु॒ष्टिः स॑योनि॒त्वाय॑ । \newline
3. स॒यो॒नि॒त्वाय॒ द्वाभ्या॒म् द्वाभ्याꣳ॑ सयोनि॒त्वाय॑ सयोनि॒त्वाय॒ द्वाभ्या᳚म् । \newline
4. स॒यो॒नि॒त्वायेति॑ सयोनि - त्वाय॑ । \newline
5. द्वाभ्या॒म् प्रति॑ष्ठित्यै॒ प्रति॑ष्ठित्यै॒ द्वाभ्या॒म् द्वाभ्या॒म् प्रति॑ष्ठित्यै । \newline
6. प्रति॑ष्ठित्या अनुपरि॒चार॑ मनुपरि॒चार॒म् प्रति॑ष्ठित्यै॒ प्रति॑ष्ठित्या अनुपरि॒चार᳚म् । \newline
7. प्रति॑ष्ठित्या॒ इति॒ प्रति॑ - स्थि॒त्यै॒ । \newline
8. अ॒नु॒प॒रि॒चार॒ मवावा॑ नुपरि॒चार॑ मनुपरि॒चार॒ मव॑ । \newline
9. अ॒नु॒प॒रि॒चार॒मित्य॑नु - प॒रि॒चार᳚म् । \newline
10. अवो᳚क्ष त्युक्ष॒ त्यवावो᳚ क्षति । \newline
11. उ॒क्ष॒त्य प॑रिवर्ग॒ मप॑रिवर्ग मुक्ष त्युक्ष॒ त्यप॑रिवर्गम् । \newline
12. अप॑रिवर्ग मे॒वैवा प॑रिवर्ग॒ मप॑रिवर्ग मे॒व । \newline
13. अप॑रिवर्ग॒मित्यप॑रि - व॒र्ग॒म् । \newline
14. ए॒वैना॑ नेना ने॒वै वैनान्॑ । \newline
15. ए॒ना॒न् प्री॒णा॒ति॒ प्री॒णा॒ त्ये॒ना॒ ने॒ना॒न् प्री॒णा॒ति॒ । \newline
16. प्री॒णा॒ति॒ वि वि प्री॑णाति प्रीणाति॒ वि । \newline
17. वि वै वै वि वि वै । \newline
18. वा ए॒ष ए॒ष वै वा ए॒षः । \newline
19. ए॒ष प्रा॒णैः प्रा॒णै रे॒ष ए॒ष प्रा॒णैः । \newline
20. प्रा॒णैः प्र॒जया᳚ प्र॒जया᳚ प्रा॒णैः प्रा॒णैः प्र॒जया᳚ । \newline
21. प्रा॒णैरिति॑ प्र - अ॒नैः । \newline
22. प्र॒जया॑ प॒शुभिः॑ प॒शुभिः॑ प्र॒जया᳚ प्र॒जया॑ प॒शुभिः॑ । \newline
23. प्र॒जयेति॑ प्र - जया᳚ । \newline
24. प॒शुभिर्॑. ऋद्ध्यत ऋद्ध्यते प॒शुभिः॑ प॒शुभिर्॑. ऋद्ध्यते । \newline
25. प॒शुभि॒रिति॑ प॒शु - भिः॒ । \newline
26. ऋ॒द्ध्य॒ते॒ यो य ऋ॑द्ध्यत ऋद्ध्यते॒ यः । \newline
27. यो᳚ ऽग्नि म॒ग्निं ॅयो यो᳚ ऽग्निम् । \newline
28. अ॒ग्निम् चि॒न्वꣳ श्चि॒न्वन् न॒ग्नि म॒ग्निम् चि॒न्वन्न् । \newline
29. चि॒न्वन् न॑धि॒क्राम॑त्य धि॒क्राम॑ति चि॒न्वꣳ श्चि॒न्वन् न॑धि॒क्राम॑ति । \newline
30. अ॒धि॒क्राम॑ति प्राण॒दाः प्रा॑ण॒दा अ॑धि॒क्राम॑ त्यधि॒क्राम॑ति प्राण॒दाः । \newline
31. अ॒धि॒क्राम॒तीत्य॑धि - क्राम॑ति । \newline
32. प्रा॒ण॒दा अ॑पान॒दा अ॑पान॒दाः प्रा॑ण॒दाः प्रा॑ण॒दा अ॑पान॒दाः । \newline
33. प्रा॒ण॒दा इति॑ प्राण - दाः । \newline
34. अ॒पा॒न॒दा इती त्य॑पान॒दा अ॑पान॒दा इति॑ । \newline
35. अ॒पा॒न॒दा इत्य॑पान - दाः । \newline
36. इत्या॑हा॒हे तीत्या॑ह । \newline
37. आ॒ह॒ प्रा॒णान् प्रा॒णा ना॑हाह प्रा॒णान् । \newline
38. प्रा॒णा ने॒वैव प्रा॒णान् प्रा॒णा ने॒व । \newline
39. प्रा॒णानिति॑ प्र - अ॒नान् । \newline
40. ए॒वात्मन् ना॒त्मन् ने॒वै वात्मन्न् । \newline
41. आ॒त्मन् ध॑त्ते धत्त आ॒त्मन् ना॒त्मन् ध॑त्ते । \newline
42. ध॒त्ते॒ व॒र्चो॒दा व॑र्चो॒दा ध॑त्ते धत्ते वर्चो॒दाः । \newline
43. व॒र्चो॒दा व॑रिवो॒दा व॑रिवो॒दा व॑र्चो॒दा व॑र्चो॒दा व॑रिवो॒दाः । \newline
44. व॒र्चो॒दा इति॑ वर्चः - दाः । \newline
45. व॒रि॒वो॒दा इतीति॑ वरिवो॒दा व॑रिवो॒दा इति॑ । \newline
46. व॒रि॒वो॒दा इति॑ वरिवः - दाः । \newline
47. इत्या॑हा॒हे तीत्या॑ह । \newline
48. आ॒ह॒ प्र॒जा प्र॒जा ऽऽहा॑ह प्र॒जा । \newline
49. प्र॒जा वै वै प्र॒जा प्र॒जा वै । \newline
50. प्र॒जेति॑ प्र - जा । \newline
51. वै वर्चो॒ वर्चो॒ वै वै वर्चः॑ । \newline
52. वर्चः॑ प॒शवः॑ प॒शवो॒ वर्चो॒ वर्चः॑ प॒शवः॑ । \newline
53. प॒शवो॒ वरि॑वो॒ वरि॑वः प॒शवः॑ प॒शवो॒ वरि॑वः । \newline
54. वरि॑वः प्र॒जाम् प्र॒जां ॅवरि॑वो॒ वरि॑वः प्र॒जाम् । \newline
55. प्र॒जा मे॒वैव प्र॒जाम् प्र॒जा मे॒व । \newline
56. प्र॒जामिति॑ प्र - जाम् । \newline
57. ए॒व प॒शून् प॒शू ने॒वैव प॒शून् । \newline
58. प॒शू ना॒त्मन् ना॒त्मन् प॒शून् प॒शू ना॒त्मन्न् । \newline
59. आ॒त्मन् ध॑त्ते धत्त आ॒त्मन् ना॒त्मन् ध॑त्ते । \newline
60. ध॒त्त॒ इन्द्र॒ इन्द्रो॑ धत्ते धत्त॒ इन्द्रः॑ । \newline
61. इन्द्रो॑ वृ॒त्रं ॅवृ॒त्र मिन्द्र॒ इन्द्रो॑ वृ॒त्रम् । \newline
62. वृ॒त्र म॑हन् नहन् वृ॒त्रं ॅवृ॒त्र म॑हन्न् । \newline
63. अ॒ह॒न् तम् त म॑हन् नह॒न् तम् । \newline
64. तं ॅवृ॒त्रो वृ॒त्र स्तम् तं ॅवृ॒त्रः । \newline
65. वृ॒त्रो ह॒तो ह॒तो वृ॒त्रो वृ॒त्रो ह॒तः । \newline

\textbf{Ghana Paata } \newline

1. वै ग्रु॑मु॒ष्टिर् ग्रु॑मु॒ष्टिर् वै वै ग्रु॑मु॒ष्टिः स॑योनि॒त्वाय॑ सयोनि॒त्वाय॑ ग्रुमु॒ष्टिर् वै वै ग्रु॑मु॒ष्टिः स॑योनि॒त्वाय॑ । \newline
2. ग्रु॒मु॒ष्टिः स॑योनि॒त्वाय॑ सयोनि॒त्वाय॑ ग्रुमु॒ष्टिर् ग्रु॑मु॒ष्टिः स॑योनि॒त्वाय॒ द्वाभ्या॒म् द्वाभ्याꣳ॑ सयोनि॒त्वाय॑ ग्रुमु॒ष्टिर् ग्रु॑मु॒ष्टिः स॑योनि॒त्वाय॒ द्वाभ्या᳚म् । \newline
3. स॒यो॒नि॒त्वाय॒ द्वाभ्या॒म् द्वाभ्याꣳ॑ सयोनि॒त्वाय॑ सयोनि॒त्वाय॒ द्वाभ्या॒म् प्रति॑ष्ठित्यै॒ प्रति॑ष्ठित्यै॒ द्वाभ्याꣳ॑ सयोनि॒त्वाय॑ सयोनि॒त्वाय॒ द्वाभ्या॒म् प्रति॑ष्ठित्यै । \newline
4. स॒यो॒नि॒त्वायेति॑ सयोनि - त्वाय॑ । \newline
5. द्वाभ्या॒म् प्रति॑ष्ठित्यै॒ प्रति॑ष्ठित्यै॒ द्वाभ्या॒म् द्वाभ्या॒म् प्रति॑ष्ठित्या अनुपरि॒चार॑ मनुपरि॒चार॒म् प्रति॑ष्ठित्यै॒ द्वाभ्या॒म् द्वाभ्या॒म् प्रति॑ष्ठित्या अनुपरि॒चार᳚म् । \newline
6. प्रति॑ष्ठित्या अनुपरि॒चार॑ मनुपरि॒चार॒म् प्रति॑ष्ठित्यै॒ प्रति॑ष्ठित्या अनुपरि॒चार॒ मवावा॑ नुपरि॒चार॒म् प्रति॑ष्ठित्यै॒ प्रति॑ष्ठित्या अनुपरि॒चार॒ मव॑ । \newline
7. प्रति॑ष्ठित्या॒ इति॒ प्रति॑ - स्थि॒त्यै॒ । \newline
8. अ॒नु॒प॒रि॒चार॒ मवावा॑ नुपरि॒चार॑ मनुपरि॒चार॒ मवो᳚क्ष त्युक्ष॒ त्यवा॑ नुपरि॒चार॑ मनुपरि॒चार॒ मवो᳚क्षति । \newline
9. अ॒नु॒प॒रि॒चार॒मित्य॑नु - प॒रि॒चार᳚म् । \newline
10. अवो᳚क्ष त्युक्ष॒ त्यवावो᳚क्ष॒ त्यप॑रिवर्ग॒ मप॑रिवर्ग मुक्ष॒ त्यवावो᳚क्ष॒ त्यप॑रिवर्गम् । \newline
11. उ॒क्ष॒त्य प॑रिवर्ग॒ मप॑रिवर्ग मुक्ष त्युक्ष॒ त्यप॑रिवर्ग मे॒वैवा प॑रिवर्ग मुक्ष त्युक्ष॒ त्यप॑रिवर्ग मे॒व । \newline
12. अप॑रिवर्ग मे॒वैवा प॑रिवर्ग॒ मप॑रिवर्ग मे॒वैना॑ नेना ने॒वा प॑रिवर्ग॒ मप॑रिवर्ग मे॒वैनान्॑ । \newline
13. अप॑रिवर्ग॒मित्यप॑रि - व॒र्ग॒म् । \newline
14. ए॒वैना॑ नेना ने॒वै वैना᳚न् प्रीणाति प्रीणा त्येना ने॒वै वैना᳚न् प्रीणाति । \newline
15. ए॒ना॒न् प्री॒णा॒ति॒ प्री॒णा॒ त्ये॒ना॒ ने॒ना॒न् प्री॒णा॒ति॒ वि वि प्री॑णा त्येना नेनान् प्रीणाति॒ वि । \newline
16. प्री॒णा॒ति॒ वि वि प्री॑णाति प्रीणाति॒ वि वै वै वि प्री॑णाति प्रीणाति॒ वि वै । \newline
17. वि वै वै वि वि वा ए॒ष ए॒ष वै वि वि वा ए॒षः । \newline
18. वा ए॒ष ए॒ष वै वा ए॒ष प्रा॒णैः प्रा॒णै रे॒ष वै वा ए॒ष प्रा॒णैः । \newline
19. ए॒ष प्रा॒णैः प्रा॒णै रे॒ष ए॒ष प्रा॒णैः प्र॒जया᳚ प्र॒जया᳚ प्रा॒णै रे॒ष ए॒ष प्रा॒णैः प्र॒जया᳚ । \newline
20. प्रा॒णैः प्र॒जया᳚ प्र॒जया᳚ प्रा॒णैः प्रा॒णैः प्र॒जया॑ प॒शुभिः॑ प॒शुभिः॑ प्र॒जया᳚ प्रा॒णैः प्रा॒णैः प्र॒जया॑ प॒शुभिः॑ । \newline
21. प्रा॒णैरिति॑ प्र - अ॒नैः । \newline
22. प्र॒जया॑ प॒शुभिः॑ प॒शुभिः॑ प्र॒जया᳚ प्र॒जया॑ प॒शुभिर्॑. ऋद्ध्यत ऋद्ध्यते प॒शुभिः॑ प्र॒जया᳚ प्र॒जया॑ प॒शुभिर्॑. ऋद्ध्यते । \newline
23. प्र॒जयेति॑ प्र - जया᳚ । \newline
24. प॒शुभिर्॑. ऋद्ध्यत ऋद्ध्यते प॒शुभिः॑ प॒शुभिर्॑. ऋद्ध्यते॒ यो य ऋ॑द्ध्यते प॒शुभिः॑ प॒शुभिर्॑. ऋद्ध्यते॒ यः । \newline
25. प॒शुभि॒रिति॑ प॒शु - भिः॒ । \newline
26. ऋ॒द्ध्य॒ते॒ यो य ऋ॑द्ध्यत ऋद्ध्यते॒ यो᳚ ऽग्नि म॒ग्निं ॅय ऋ॑द्ध्यत ऋद्ध्यते॒ यो᳚ ऽग्निम् । \newline
27. यो᳚ ऽग्नि म॒ग्निं ॅयो यो᳚ ऽग्निम् चि॒न्वꣳ श्चि॒न्वन् न॒ग्निं ॅयो यो᳚ ऽग्निम् चि॒न्वन्न् । \newline
28. अ॒ग्निम् चि॒न्वꣳ श्चि॒न्वन् न॒ग्नि म॒ग्निम् चि॒न्वन् न॑धि॒क्राम॑ त्यधि॒क्राम॑ति चि॒न्वन् न॒ग्नि म॒ग्निम् चि॒न्वन् न॑धि॒क्राम॑ति । \newline
29. चि॒न्वन् न॑धि॒क्राम॑ त्यधि॒क्राम॑ति चि॒न्वꣳ श्चि॒न्वन् न॑धि॒क्राम॑ति प्राण॒दाः प्रा॑ण॒दा अ॑धि॒क्राम॑ति चि॒न्वꣳ श्चि॒न्वन् न॑धि॒क्राम॑ति प्राण॒दाः । \newline
30. अ॒धि॒क्राम॑ति प्राण॒दाः प्रा॑ण॒दा अ॑धि॒क्राम॑ त्यधि॒क्राम॑ति प्राण॒दा अ॑पान॒दा अ॑पान॒दाः प्रा॑ण॒दा अ॑धि॒क्राम॑ त्यधि॒क्राम॑ति प्राण॒दा अ॑पान॒दाः । \newline
31. अ॒धि॒क्राम॒तीत्य॑धि - क्राम॑ति । \newline
32. प्रा॒ण॒दा अ॑पान॒दा अ॑पान॒दाः प्रा॑ण॒दाः प्रा॑ण॒दा अ॑पान॒दा इती त्य॑पान॒दाः प्रा॑ण॒दाः प्रा॑ण॒दा अ॑पान॒दा इति॑ । \newline
33. प्रा॒ण॒दा इति॑ प्राण - दाः । \newline
34. अ॒पा॒न॒दा इती त्य॑पान॒दा अ॑पान॒दा इत्या॑हा॒हे त्य॑पान॒दा अ॑पान॒दा इत्या॑ह । \newline
35. अ॒पा॒न॒दा इत्य॑पान - दाः । \newline
36. इत्या॑हा॒हे तीत्या॑ह प्रा॒णान् प्रा॒णा ना॒हे तीत्या॑ह प्रा॒णान् । \newline
37. आ॒ह॒ प्रा॒णान् प्रा॒णा ना॑हाह प्रा॒णा ने॒वैव प्रा॒णा ना॑हाह प्रा॒णा ने॒व । \newline
38. प्रा॒णा ने॒वैव प्रा॒णान् प्रा॒णा ने॒वात्मन् ना॒त्मन् ने॒व प्रा॒णान् प्रा॒णा ने॒वात्मन्न् । \newline
39. प्रा॒णानिति॑ प्र - अ॒नान् । \newline
40. ए॒वात्मन् ना॒त्मन् ने॒वै वात्मन् ध॑त्ते धत्त आ॒त्मन् ने॒वै वात्मन् ध॑त्ते । \newline
41. आ॒त्मन् ध॑त्ते धत्त आ॒त्मन् ना॒त्मन् ध॑त्ते वर्चो॒दा व॑र्चो॒दा ध॑त्त आ॒त्मन् ना॒त्मन् ध॑त्ते वर्चो॒दाः । \newline
42. ध॒त्ते॒ व॒र्चो॒दा व॑र्चो॒दा ध॑त्ते धत्ते वर्चो॒दा व॑रिवो॒दा व॑रिवो॒दा व॑र्चो॒दा ध॑त्ते धत्ते वर्चो॒दा व॑रिवो॒दाः । \newline
43. व॒र्चो॒दा व॑रिवो॒दा व॑रिवो॒दा व॑र्चो॒दा व॑र्चो॒दा व॑रिवो॒दा इतीति॑ वरिवो॒दा व॑र्चो॒दा व॑र्चो॒दा व॑रिवो॒दा इति॑ । \newline
44. व॒र्चो॒दा इति॑ वर्चः - दाः । \newline
45. व॒रि॒वो॒दा इतीति॑ वरिवो॒दा व॑रिवो॒दा इत्या॑हा॒हेति॑ वरिवो॒दा व॑रिवो॒दा इत्या॑ह । \newline
46. व॒रि॒वो॒दा इति॑ वरिवः - दाः । \newline
47. इत्या॑हा॒हे तीत्या॑ह प्र॒जा प्र॒जा ऽऽहेती त्या॑ह प्र॒जा । \newline
48. आ॒ह॒ प्र॒जा प्र॒जा ऽऽहा॑ह प्र॒जा वै वै प्र॒जा ऽऽहा॑ह प्र॒जा वै । \newline
49. प्र॒जा वै वै प्र॒जा प्र॒जा वै वर्चो॒ वर्चो॒ वै प्र॒जा प्र॒जा वै वर्चः॑ । \newline
50. प्र॒जेति॑ प्र - जा । \newline
51. वै वर्चो॒ वर्चो॒ वै वै वर्चः॑ प॒शवः॑ प॒शवो॒ वर्चो॒ वै वै वर्चः॑ प॒शवः॑ । \newline
52. वर्चः॑ प॒शवः॑ प॒शवो॒ वर्चो॒ वर्चः॑ प॒शवो॒ वरि॑वो॒ वरि॑वः प॒शवो॒ वर्चो॒ वर्चः॑ प॒शवो॒ वरि॑वः । \newline
53. प॒शवो॒ वरि॑वो॒ वरि॑वः प॒शवः॑ प॒शवो॒ वरि॑वः प्र॒जाम् प्र॒जां ॅवरि॑वः प॒शवः॑ प॒शवो॒ वरि॑वः प्र॒जाम् । \newline
54. वरि॑वः प्र॒जाम् प्र॒जां ॅवरि॑वो॒ वरि॑वः प्र॒जा मे॒वैव प्र॒जां ॅवरि॑वो॒ वरि॑वः प्र॒जा मे॒व । \newline
55. प्र॒जा मे॒वैव प्र॒जाम् प्र॒जा मे॒व प॒शून् प॒शू ने॒व प्र॒जाम् प्र॒जा मे॒व प॒शून् । \newline
56. प्र॒जामिति॑ प्र - जाम् । \newline
57. ए॒व प॒शून् प॒शू ने॒वैव प॒शू ना॒त्मन् ना॒त्मन् प॒शू ने॒वैव प॒शू ना॒त्मन्न् । \newline
58. प॒शू ना॒त्मन् ना॒त्मन् प॒शून् प॒शू ना॒त्मन् ध॑त्ते धत्त आ॒त्मन् प॒शून् प॒शू ना॒त्मन् ध॑त्ते । \newline
59. आ॒त्मन् ध॑त्ते धत्त आ॒त्मन् ना॒त्मन् ध॑त्त॒ इन्द्र॒ इन्द्रो॑ धत्त आ॒त्मन् ना॒त्मन् ध॑त्त॒ इन्द्रः॑ । \newline
60. ध॒त्त॒ इन्द्र॒ इन्द्रो॑ धत्ते धत्त॒ इन्द्रो॑ वृ॒त्रं ॅवृ॒त्र मिन्द्रो॑ धत्ते धत्त॒ इन्द्रो॑ वृ॒त्रम् । \newline
61. इन्द्रो॑ वृ॒त्रं ॅवृ॒त्र मिन्द्र॒ इन्द्रो॑ वृ॒त्र म॑हन् नहन् वृ॒त्र मिन्द्र॒ इन्द्रो॑ वृ॒त्र म॑हन्न् । \newline
62. वृ॒त्र म॑हन् नहन् वृ॒त्रं ॅवृ॒त्र म॑ह॒न् तम् त म॑हन् वृ॒त्रं ॅवृ॒त्र म॑ह॒न् तम् । \newline
63. अ॒ह॒न् तम् त म॑हन् नह॒न् तं ॅवृ॒त्रो वृ॒त्र स्त म॑हन् नह॒न् तं ॅवृ॒त्रः । \newline
64. तं ॅवृ॒त्रो वृ॒त्र स्तम् तं ॅवृ॒त्रो ह॒तो ह॒तो वृ॒त्र स्तम् तं ॅवृ॒त्रो ह॒तः । \newline
65. वृ॒त्रो ह॒तो ह॒तो वृ॒त्रो वृ॒त्रो ह॒त ष्षो॑ड॒शभि॑ ष्षोड॒शभि॑र् ह॒तो वृ॒त्रो वृ॒त्रो ह॒त ष्षो॑ड॒शभिः॑ । \newline
\pagebreak
\markright{ TS 5.4.5.4  \hfill https://www.vedavms.in \hfill}

\section{ TS 5.4.5.4 }

\textbf{TS 5.4.5.4 } \newline
\textbf{Samhita Paata} \newline

ह॒तः षो॑ड॒शभि॑र्भो॒गैर॑सिना॒थ् स ए॒ताम॒ग्नयेऽनी॑कवत॒ आहु॑तिमपश्य॒त् ताम॑जुहो॒त् तस्या॒ग्निरनी॑ कवा॒न्थ्स्वेन॑ भाग॒धेये॑न प्री॒तः षो॑डश॒धा वृ॒त्रस्य॑ भो॒गानप्य॑दहद्-वैश्वकर्म॒णेन॑ पा॒प्मनो॒ निर॑मुच्यत॒ यद॒ग्नयेऽनी॑कवत॒ आहु॑तिं जु॒होत्य॒ग्निरे॒वा-ऽस्यानी॑कवा॒न्थ्स्वेन॑ भाग॒धेये॑न प्री॒तः पा॒प्मान॒मपि॑ दहति वैश्वकर्म॒णेन॑ पा॒प्मनो॒ निर्मु॑च्यते॒ यं का॒मये॑त चि॒रं पा॒प्मनो॒ - [  ] \newline

\textbf{Pada Paata} \newline

ह॒तः । षो॒ड॒शभि॒रिति॑ षोड॒श - भिः॒ । भो॒गैः । अ॒सि॒ना॒त् । सः । ए॒ताम् । अ॒ग्नये᳚ । अनी॑कवत॒ इत्यनी॑क - व॒ते॒ । आहु॑ति॒मित्या - हु॒ति॒म् । अ॒प॒श्य॒त् । ताम् । अ॒जु॒हो॒त् । तस्य॑ । अ॒ग्निः । अनी॑कवा॒नित्यनी॑क - वा॒न् । स्वेन॑ । भा॒ग॒धेये॒नेति॑ भाग - धेये॑न । प्री॒तः । षो॒ड॒श॒धेति॑ षोडश-धा । वृ॒त्रस्य॑ । भो॒गान् । अपीति॑ । अ॒द॒ह॒त् । वै॒श्व॒क॒र्म॒णेनेति॑ वैश्व - क॒र्म॒णेन॑ । पा॒प्मनः॑ । निरिति॑ । अ॒मु॒च्य॒त॒ । यत् । अ॒ग्नये᳚ । अनी॑कवत॒ इत्यनी॑क - व॒ते॒ । आहु॑ति॒मित्या - हु॒ति॒म् । जु॒होति॑ । अ॒ग्निः । ए॒व । अ॒स्य॒ । आनी॑कवा॒नित्यनी॑क - वा॒न् । स्वेन॑ । भा॒ग॒धेये॒नेति॑ भाग - धेये॑न । प्री॒तः । पा॒प्मान᳚म् । अपीति॑ । द॒ह॒ति॒ । वै॒श्व॒क॒र्म॒णेनेति॑ वैश्व - क॒र्म॒णेन॑ । पा॒प्मनः॑ । निरिति॑ । मु॒च्य॒ते॒ । यम् । का॒मये॑त । चि॒रम् । पा॒प्मनः॑ ।  \newline


\textbf{Krama Paata} \newline

ह॒तः षो॑ड॒शभिः॑ । षो॒ड॒शभि॑र् भो॒गैः । षो॒ड॒शभि॒रिति॑ षोड॒श - भिः॒ । भो॒गैर॑सिनात् । अ॒सि॒ना॒थ् सः । स ए॒ताम् । ए॒ताम॒ग्नये᳚ । अ॒ग्नयेऽनी॑कवते । अनी॑कवत॒ आहु॑तिम् । अनी॑कवत॒ इत्यनी॑क - व॒ते॒ । आहु॑तिमपश्यत् । आहु॑ति॒मित्या - हु॒ति॒म् । अ॒प॒श्य॒त् ताम् । ताम॑जुहोत् । अ॒जु॒हो॒त् तस्य॑ । तस्या॒ग्निः । अ॒ग्निरनी॑कवान् । अनी॑कवा॒न्थ् स्वेन॑ । अनी॑कवा॒नित्यनीक॑ - वा॒न्॒ । स्वेन॑ भाग॒धेये॑न । भा॒ग॒धेये॑न प्री॒तः । भा॒ग॒धेये॒नेति॑ भाग - धेये॑न । प्री॒तः षो॑डश॒धा । षो॒ड॒श॒धा वृ॒त्रस्य॑ । षो॒ड॒श॒धेति॑ षोडश - धा । वृ॒त्रस्य॑ भो॒गान् । भो॒गानपि॑ । अप्य॑दहत् । अ॒द॒ह॒द् वै॒श्व॒क॒र्म॒णेन॑ । वै॒श्व॒क॒र्म॒णेन॑ पा॒प्मनः॑ । वै॒श्व॒क॒र्म॒णेनेति॑ वैश्व - क॒र्म॒णेन॑ । पा॒प्मनो॒ निः । निर॑मुच्यत । अ॒मु॒च्य॒त॒ यत् । यद॒ग्नये᳚ । अ॒ग्नयेऽनी॑कवते । अनी॑कवत॒ आहु॑तिम् । अनी॑कवत॒ इत्यनी॑क - व॒ते॒ । आहु॑तिम् जु॒होति॑ । आहु॑ति॒मित्या - हु॒ति॒म् । जु॒होत्य॒ग्निः । अ॒ग्निरे॒व । ए॒वास्य॑ । अ॒स्यानी॑कवान् । अनी॑कवा॒न्थ् स्वेन॑ । अनी॑कवा॒नित्यनी॑क - वा॒न्॒ । स्वेन॑ भाग॒धेये॑न । भा॒ग॒धेये॑न प्री॒तः । भा॒ग॒धेये॒नेति॑ भाग - धेये॑न । प्री॒तः पा॒प्मान᳚म् । पा॒प्मान॒मपि॑ । अपि॑ दहति । द॒ह॒ति॒ वै॒श्व॒क॒र्म॒णेन॑ । वै॒श्व॒क॒र्म॒णेन॑ पा॒प्मनः॑ । वै॒श्व॒क॒र्म॒णेनेति॑ वैश्व - क॒र्म॒णेन॑ । पा॒प्मनो॒ निः । निर् मु॑च्यते । मु॒च्य॒ते॒ यम् । यम् का॒मये॑त । का॒मये॑त चि॒रम् । चि॒रम् पा॒प्मनः॑ ( ) । पा॒प्मनो॒ निः \newline

\textbf{Jatai Paata} \newline

1. ह॒त ष्षो॑ड॒शभि॑ ष्षोड॒शभि॑र् ह॒तो ह॒त ष्षो॑ड॒शभिः॑ । \newline
2. षो॒ड॒शभि॑र् भो॒गैर् भो॒गै ष्षो॑ड॒शभि॑ ष्षोड॒शभि॑र् भो॒गैः । \newline
3. षो॒ड॒शभि॒रिति॑ षोड॒श - भिः॒ । \newline
4. भो॒गै र॑सिना दसिनाद् भो॒गैर् भो॒गै र॑सिनात् । \newline
5. अ॒सि॒ना॒थ् स सो॑ ऽसिना दसिना॒थ् सः । \newline
6. स ए॒ता मे॒ताꣳ स स ए॒ताम् । \newline
7. ए॒ता म॒ग्नये॒ ऽग्नय॑ ए॒ता मे॒ता म॒ग्नये᳚ । \newline
8. अ॒ग्नये ऽनी॑कव॒ते ऽनी॑कवते॒ ऽग्नये॒ ऽग्नये ऽनी॑कवते । \newline
9. अनी॑कवत॒ आहु॑ति॒ माहु॑ति॒ मनी॑कव॒ते ऽनी॑कवत॒ आहु॑तिम् । \newline
10. अनी॑कवत॒ इत्यनी॑क - व॒ते॒ । \newline
11. आहु॑ति मपश्य दपश्य॒ दाहु॑ति॒ माहु॑ति मपश्यत् । \newline
12. आहु॑ति॒मित्या - हु॒ति॒म् । \newline
13. अ॒प॒श्य॒त् ताम् ता म॑पश्य दपश्य॒त् ताम् । \newline
14. ता म॑जुहो दजुहो॒त् ताम् ता म॑जुहोत् । \newline
15. अ॒जु॒हो॒त् तस्य॒ तस्या॑ जुहो दजुहो॒त् तस्य॑ । \newline
16. तस्या॒ग्नि र॒ग्नि स्तस्य॒ तस्या॒ग्निः । \newline
17. अ॒ग्नि रनी॑कवा॒ ननी॑कवा न॒ग्नि र॒ग्नि रनी॑कवान् । \newline
18. अनी॑कवा॒न् थ्स्वेन॒ स्वेना नी॑कवा॒ ननी॑कवा॒न् थ्स्वेन॑ । \newline
19. अनी॑कवा॒नित्यनी॑क - वा॒न् । \newline
20. स्वेन॑ भाग॒धेये॑न भाग॒धेये॑न॒ स्वेन॒ स्वेन॑ भाग॒धेये॑न । \newline
21. भा॒ग॒धेये॑न प्री॒तः प्री॒तो भा॑ग॒धेये॑न भाग॒धेये॑न प्री॒तः । \newline
22. भा॒ग॒धेये॒नेति॑ भाग - धेये॑न । \newline
23. प्री॒त ष्षो॑डश॒धा षो॑डश॒धा प्री॒तः प्री॒त ष्षो॑डश॒धा । \newline
24. षो॒ड॒श॒धा वृ॒त्रस्य॑ वृ॒त्रस्य॑ षोडश॒धा षो॑डश॒धा वृ॒त्रस्य॑ । \newline
25. षो॒ड॒श॒धेति॑ षोडश - धा । \newline
26. वृ॒त्रस्य॑ भो॒गान् भो॒गान् वृ॒त्रस्य॑ वृ॒त्रस्य॑ भो॒गान् । \newline
27. भो॒गा नप्यपि॑ भो॒गान् भो॒गा नपि॑ । \newline
28. अप्य॑ दह ददह॒ दप्य प्य॑ दहत् । \newline
29. अ॒द॒ह॒द् वै॒श्व॒क॒र्म॒णेन॑ वैश्वकर्म॒णेना॑ दह ददहद् वैश्वकर्म॒णेन॑ । \newline
30. वै॒श्व॒क॒र्म॒णेन॑ पा॒प्मनः॑ पा॒प्मनो॑ वैश्वकर्म॒णेन॑ वैश्वकर्म॒णेन॑ पा॒प्मनः॑ । \newline
31. वै॒श्व॒क॒र्म॒णेनेति॑ वैश्व - क॒र्म॒णेन॑ । \newline
32. पा॒प्मनो॒ निर् णिष् पा॒प्मनः॑ पा॒प्मनो॒ निः । \newline
33. निर॑मुच्यता मुच्यत॒ निर् णिर॑मुच्यत । \newline
34. अ॒मु॒च्य॒त॒ यद् यद॑मुच्यता मुच्यत॒ यत् । \newline
35. यद॒ग्नये॒ ऽग्नये॒ यद् यद॒ग्नये᳚ । \newline
36. अ॒ग्नये ऽनी॑कव॒ते ऽनी॑कवते॒ ऽग्नये॒ ऽग्नये ऽनी॑कवते । \newline
37. अनी॑कवत॒ आहु॑ति॒ माहु॑ति॒ मनी॑कव॒ते ऽनी॑कवत॒ आहु॑तिम् । \newline
38. अनी॑कवत॒ इत्यनी॑क - व॒ते॒ । \newline
39. आहु॑तिम् जु॒होति॑ जु॒हो त्याहु॑ति॒ माहु॑तिम् जु॒होति॑ । \newline
40. आहु॑ति॒मित्या - हु॒ति॒म् । \newline
41. जु॒हो त्य॒ग्नि र॒ग्निर् जु॒होति॑ जु॒हो त्य॒ग्निः । \newline
42. अ॒ग्नि रे॒वै वाग्नि र॒ग्नि रे॒व । \newline
43. ए॒वास्या᳚ स्यै॒वै वास्य॑ । \newline
44. अ॒स्या नी॑कवा॒ नानी॑कवा नस्या॒स्या नी॑कवान् । \newline
45. आनी॑कवा॒न् थ्स्वेन॒ स्वेनानी॑कवा॒ नानी॑कवा॒न् थ्स्वेन॑ । \newline
46. आनी॑कवा॒नित्यनी॑क - वा॒न् । \newline
47. स्वेन॑ भाग॒धेये॑न भाग॒धेये॑न॒ स्वेन॒ स्वेन॑ भाग॒धेये॑न । \newline
48. भा॒ग॒धेये॑न प्री॒तः प्री॒तो भा॑ग॒धेये॑न भाग॒धेये॑न प्री॒तः । \newline
49. भा॒ग॒धेये॒नेति॑ भाग - धेये॑न । \newline
50. प्री॒तः पा॒प्मान॑म् पा॒प्मान॑म् प्री॒तः प्री॒तः पा॒प्मान᳚म् । \newline
51. पा॒प्मान॒ मप्यपि॑ पा॒प्मान॑म् पा॒प्मान॒ मपि॑ । \newline
52. अपि॑ दहति दह॒ त्यप्यपि॑ दहति । \newline
53. द॒ह॒ति॒ वै॒श्व॒क॒र्म॒णेन॑ वैश्वकर्म॒णेन॑ दहति दहति वैश्वकर्म॒णेन॑ । \newline
54. वै॒श्व॒क॒र्म॒णेन॑ पा॒प्मनः॑ पा॒प्मनो॑ वैश्वकर्म॒णेन॑ वैश्वकर्म॒णेन॑ पा॒प्मनः॑ । \newline
55. वै॒श्व॒क॒र्म॒णेनेति॑ वैश्व - क॒र्म॒णेन॑ । \newline
56. पा॒प्मनो॒ निर् णिष् पा॒प्मनः॑ पा॒प्मनो॒ निः । \newline
57. निर् मु॑च्यते मुच्यते॒ निर् णिर् मु॑च्यते । \newline
58. मु॒च्य॒ते॒ यं ॅयम् मु॑च्यते मुच्यते॒ यम् । \newline
59. यम् का॒मये॑त का॒मये॑त॒ यं ॅयम् का॒मये॑त । \newline
60. का॒मये॑त चि॒रम् चि॒रम् का॒मये॑त का॒मये॑त चि॒रम् । \newline
61. चि॒रम् पा॒प्मनः॑ पा॒प्मन॑ श्चि॒रम् चि॒रम् पा॒प्मनः॑ । \newline
62. पा॒प्मनो॒ निर् णिष् पा॒प्मनः॑ पा॒प्मनो॒ निः । \newline

\textbf{Ghana Paata } \newline

1. ह॒त ष्षो॑ड॒शभि॑ ष्षोड॒शभि॑र् ह॒तो ह॒त ष्षो॑ड॒शभि॑र् भो॒गैर् भो॒गै ष्षो॑ड॒शभि॑र् ह॒तो ह॒त ष्षो॑ड॒शभि॑र् भो॒गैः । \newline
2. षो॒ड॒शभि॑र् भो॒गैर् भो॒गै ष्षो॑ड॒शभि॑ ष्षोड॒शभि॑र् भो॒गै र॑सिना दसिनाद् भो॒गै ष्षो॑ड॒शभि॑ ष्षोड॒शभि॑र् भो॒गै र॑सिनात् । \newline
3. षो॒ड॒शभि॒रिति॑ षोड॒श - भिः॒ । \newline
4. भो॒गै र॑सिना दसिनाद् भो॒गैर् भो॒गै र॑सिना॒थ् स सो॑ ऽसिनाद् भो॒गैर् भो॒गै र॑सिना॒थ् सः । \newline
5. अ॒सि॒ना॒थ् स सो॑ ऽसिना दसिना॒थ् स ए॒ता मे॒ताꣳ सो॑ ऽसिना दसिना॒थ् स ए॒ताम् । \newline
6. स ए॒ता मे॒ताꣳ स स ए॒ता म॒ग्नये॒ ऽग्नय॑ ए॒ताꣳ स स ए॒ता म॒ग्नये᳚ । \newline
7. ए॒ता म॒ग्नये॒ ऽग्नय॑ ए॒ता मे॒ता म॒ग्नये ऽनी॑कव॒ते ऽनी॑कवते॒ ऽग्नय॑ ए॒ता मे॒ता म॒ग्नये ऽनी॑कवते । \newline
8. अ॒ग्नये ऽनी॑कव॒ते ऽनी॑कवते॒ ऽग्नये॒ ऽग्नये ऽनी॑कवत॒ आहु॑ति॒ माहु॑ति॒ मनी॑कवते॒ ऽग्नये॒ ऽग्नये ऽनी॑कवत॒ आहु॑तिम् । \newline
9. अनी॑कवत॒ आहु॑ति॒ माहु॑ति॒ मनी॑कव॒ते ऽनी॑कवत॒ आहु॑ति मपश्य दपश्य॒ दाहु॑ति॒ मनी॑कव॒ते ऽनी॑कवत॒ आहु॑ति मपश्यत् । \newline
10. अनी॑कवत॒ इत्यनी॑क - व॒ते॒ । \newline
11. आहु॑ति मपश्य दपश्य॒ दाहु॑ति॒ माहु॑ति मपश्य॒त् ताम् ता म॑पश्य॒ दाहु॑ति॒ माहु॑ति मपश्य॒त् ताम् । \newline
12. आहु॑ति॒मित्या - हु॒ति॒म् । \newline
13. अ॒प॒श्य॒त् ताम् ता म॑पश्य दपश्य॒त् ता म॑जुहो दजुहो॒त् ता म॑पश्य दपश्य॒त् ता म॑जुहोत् । \newline
14. ता म॑जुहो दजुहो॒त् ताम् ता म॑जुहो॒त् तस्य॒ तस्या॑ जुहो॒त् ताम् ता म॑जुहो॒त् तस्य॑ । \newline
15. अ॒जु॒हो॒त् तस्य॒ तस्या॑ जुहो दजुहो॒त् तस्या॒ग्नि र॒ग्नि स्तस्या॑ जुहो दजुहो॒त् तस्या॒ग्निः । \newline
16. तस्या॒ग्नि र॒ग्नि स्तस्य॒ तस्या॒ग्नि रनी॑कवा॒ ननी॑कवा न॒ग्नि स्तस्य॒ तस्या॒ग्नि रनी॑कवान् । \newline
17. अ॒ग्नि रनी॑कवा॒ ननी॑कवा न॒ग्नि र॒ग्नि रनी॑कवा॒न् थ्स्वेन॒ स्वेनानी॑कवा न॒ग्नि र॒ग्नि रनी॑कवा॒न् थ्स्वेन॑ । \newline
18. अनी॑कवा॒न् थ्स्वेन॒ स्वेनानी॑कवा॒ ननी॑कवा॒न् थ्स्वेन॑ भाग॒धेये॑न भाग॒धेये॑न॒ स्वेनानी॑कवा॒ ननी॑कवा॒न् थ्स्वेन॑ भाग॒धेये॑न । \newline
19. अनी॑कवा॒नित्यनी॑क - वा॒न् । \newline
20. स्वेन॑ भाग॒धेये॑न भाग॒धेये॑न॒ स्वेन॒ स्वेन॑ भाग॒धेये॑न प्री॒तः प्री॒तो भा॑ग॒धेये॑न॒ स्वेन॒ स्वेन॑ भाग॒धेये॑न प्री॒तः । \newline
21. भा॒ग॒धेये॑न प्री॒तः प्री॒तो भा॑ग॒धेये॑न भाग॒धेये॑न प्री॒त ष्षो॑डश॒धा षो॑डश॒धा प्री॒तो भा॑ग॒धेये॑न भाग॒धेये॑न प्री॒त ष्षो॑डश॒धा । \newline
22. भा॒ग॒धेये॒नेति॑ भाग - धेये॑न । \newline
23. प्री॒त ष्षो॑डश॒धा षो॑डश॒धा प्री॒तः प्री॒त ष्षो॑डश॒धा वृ॒त्रस्य॑ वृ॒त्रस्य॑ षोडश॒धा प्री॒तः प्री॒त ष्षो॑डश॒धा वृ॒त्रस्य॑ । \newline
24. षो॒ड॒श॒धा वृ॒त्रस्य॑ वृ॒त्रस्य॑ षोडश॒धा षो॑डश॒धा वृ॒त्रस्य॑ भो॒गान् भो॒गान् वृ॒त्रस्य॑ षोडश॒धा षो॑डश॒धा वृ॒त्रस्य॑ भो॒गान् । \newline
25. षो॒ड॒श॒धेति॑ षोडश - धा । \newline
26. वृ॒त्रस्य॑ भो॒गान् भो॒गान् वृ॒त्रस्य॑ वृ॒त्रस्य॑ भो॒गा नप्यपि॑ भो॒गान् वृ॒त्रस्य॑ वृ॒त्रस्य॑ भो॒गा नपि॑ । \newline
27. भो॒गा नप्यपि॑ भो॒गान् भो॒गा नप्य॑ दहद दह॒ दपि॑ भो॒गान् भो॒गा नप्य॑दहत् । \newline
28. अप्य॑दह ददह॒ दप्य प्य॑दहद् वैश्वकर्म॒णेन॑ वैश्वकर्म॒णेना॑ दह॒ दप्य प्य॑ दहद् वैश्वकर्म॒णेन॑ । \newline
29. अ॒द॒ह॒द् वै॒श्व॒क॒र्म॒णेन॑ वैश्वकर्म॒णेना॑ दह दद हद् वैश्वकर्म॒णेन॑ पा॒प्मनः॑ पा॒प्मनो॑ वैश्वकर्म॒णेना॑ दह दद हद् वैश्वकर्म॒णेन॑ पा॒प्मनः॑ । \newline
30. वै॒श्व॒क॒र्म॒णेन॑ पा॒प्मनः॑ पा॒प्मनो॑ वैश्वकर्म॒णेन॑ वैश्वकर्म॒णेन॑ पा॒प्मनो॒ निर् णिष् पा॒प्मनो॑ वैश्वकर्म॒णेन॑ वैश्वकर्म॒णेन॑ पा॒प्मनो॒ निः । \newline
31. वै॒श्व॒क॒र्म॒णेनेति॑ वैश्व - क॒र्म॒णेन॑ । \newline
32. पा॒प्मनो॒ निर् णिष् पा॒प्मनः॑ पा॒प्मनो॒ निर॑मुच्यता मुच्यत॒ निष् पा॒प्मनः॑ पा॒प्मनो॒ निर॑मुच्यत । \newline
33. निर॑मुच्यता मुच्यत॒ निर् णिर॑मुच्यत॒ यद् यद॑मुच्यत॒ निर् णिर॑मुच्यत॒ यत् । \newline
34. अ॒मु॒च्य॒त॒ यद् यद॑मुच्यता मुच्यत॒ यद॒ग्नये॒ ऽग्नये॒ यद॑मुच्यता मुच्यत॒ यद॒ग्नये᳚ । \newline
35. यद॒ग्नये॒ ऽग्नये॒ यद् यद॒ग्नये ऽनी॑कव॒ते ऽनी॑कवते॒ ऽग्नये॒ यद् यद॒ग्नये ऽनी॑कवते । \newline
36. अ॒ग्नये ऽनी॑कव॒ते ऽनी॑कवते॒ ऽग्नये॒ ऽग्नये ऽनी॑कवत॒ आहु॑ति॒ माहु॑ति॒ मनी॑कवते॒ ऽग्नये॒ ऽग्नये ऽनी॑कवत॒ आहु॑तिम् । \newline
37. अनी॑कवत॒ आहु॑ति॒ माहु॑ति॒ मनी॑कव॒ते ऽनी॑कवत॒ आहु॑तिम् जु॒होति॑ जु॒हो त्याहु॑ति॒ मनी॑कव॒ते ऽनी॑कवत॒ आहु॑तिम् जु॒होति॑ । \newline
38. अनी॑कवत॒ इत्यनी॑क - व॒ते॒ । \newline
39. आहु॑तिम् जु॒होति॑ जु॒हो त्याहु॑ति॒ माहु॑तिम् जु॒हो त्य॒ग्नि र॒ग्निर् जु॒हो त्याहु॑ति॒ माहु॑तिम् जु॒हो त्य॒ग्निः । \newline
40. आहु॑ति॒मित्या - हु॒ति॒म् । \newline
41. जु॒हो त्य॒ग्नि र॒ग्निर् जु॒होति॑ जु॒हो त्य॒ग्नि रे॒वै वाग्निर् जु॒होति॑ जु॒हो त्य॒ग्नि रे॒व । \newline
42. अ॒ग्नि रे॒वै वाग्नि र॒ग्नि रे॒वास्या᳚ स्यै॒वाग्नि र॒ग्नि रे॒वास्य॑ । \newline
43. ए॒वास्या᳚ स्यै॒वै वास्या नी॑कवा॒ नानी॑कवा नस्यै॒ वैवास्या नी॑कवान् । \newline
44. अ॒स्यानी॑कवा॒ नानी॑कवा नस्या॒स्या नी॑कवा॒न् थ्स्वेन॒ स्वेनानी॑कवा नस्या॒स्या नी॑कवा॒न् थ्स्वेन॑ । \newline
45. आनी॑कवा॒न् थ्स्वेन॒ स्वेनानी॑कवा॒ नानी॑कवा॒न् थ्स्वेन॑ भाग॒धेये॑न भाग॒धेये॑न॒ स्वेनानी॑कवा॒ नानी॑कवा॒न् थ्स्वेन॑ भाग॒धेये॑न । \newline
46. आनी॑कवा॒नित्यनी॑क - वा॒न् । \newline
47. स्वेन॑ भाग॒धेये॑न भाग॒धेये॑न॒ स्वेन॒ स्वेन॑ भाग॒धेये॑न प्री॒तः प्री॒तो भा॑ग॒धेये॑न॒ स्वेन॒ स्वेन॑ भाग॒धेये॑न प्री॒तः । \newline
48. भा॒ग॒धेये॑न प्री॒तः प्री॒तो भा॑ग॒धेये॑न भाग॒धेये॑न प्री॒तः पा॒प्मान॑म् पा॒प्मान॑म् प्री॒तो भा॑ग॒धेये॑न भाग॒धेये॑न प्री॒तः पा॒प्मान᳚म् । \newline
49. भा॒ग॒धेये॒नेति॑ भाग - धेये॑न । \newline
50. प्री॒तः पा॒प्मान॑म् पा॒प्मान॑म् प्री॒तः प्री॒तः पा॒प्मान॒ मप्यपि॑ पा॒प्मान॑म् प्री॒तः प्री॒तः पा॒प्मान॒ मपि॑ । \newline
51. पा॒प्मान॒ मप्यपि॑ पा॒प्मान॑म् पा॒प्मान॒ मपि॑ दहति दह॒ त्यपि॑ पा॒प्मान॑म् पा॒प्मान॒ मपि॑ दहति । \newline
52. अपि॑ दहति दह॒ त्यप्यपि॑ दहति वैश्वकर्म॒णेन॑ वैश्वकर्म॒णेन॑ दह॒ त्यप्यपि॑ दहति वैश्वकर्म॒णेन॑ । \newline
53. द॒ह॒ति॒ वै॒श्व॒क॒र्म॒णेन॑ वैश्वकर्म॒णेन॑ दहति दहति वैश्वकर्म॒णेन॑ पा॒प्मनः॑ पा॒प्मनो॑ वैश्वकर्म॒णेन॑ दहति दहति वैश्वकर्म॒णेन॑ पा॒प्मनः॑ । \newline
54. वै॒श्व॒क॒र्म॒णेन॑ पा॒प्मनः॑ पा॒प्मनो॑ वैश्वकर्म॒णेन॑ वैश्वकर्म॒णेन॑ पा॒प्मनो॒ निर् णिष् पा॒प्मनो॑ वैश्वकर्म॒णेन॑ वैश्वकर्म॒णेन॑ पा॒प्मनो॒ निः । \newline
55. वै॒श्व॒क॒र्म॒णेनेति॑ वैश्व - क॒र्म॒णेन॑ । \newline
56. पा॒प्मनो॒ निर् णिष् पा॒प्मनः॑ पा॒प्मनो॒ निर् मु॑च्यते मुच्यते॒ निष् पा॒प्मनः॑ पा॒प्मनो॒ निर् मु॑च्यते । \newline
57. निर् मु॑च्यते मुच्यते॒ निर् णिर् मु॑च्यते॒ यं ॅयम् मु॑च्यते॒ निर् णिर् मु॑च्यते॒ यम् । \newline
58. मु॒च्य॒ते॒ यं ॅयम् मु॑च्यते मुच्यते॒ यम् का॒मये॑त का॒मये॑त॒ यम् मु॑च्यते मुच्यते॒ यम् का॒मये॑त । \newline
59. यम् का॒मये॑त का॒मये॑त॒ यं ॅयम् का॒मये॑त चि॒रम् चि॒रम् का॒मये॑त॒ यं ॅयम् का॒मये॑त चि॒रम् । \newline
60. का॒मये॑त चि॒रम् चि॒रम् का॒मये॑त का॒मये॑त चि॒रम् पा॒प्मनः॑ पा॒प्मन॑ श्चि॒रम् का॒मये॑त का॒मये॑त चि॒रम् पा॒प्मनः॑ । \newline
61. चि॒रम् पा॒प्मनः॑ पा॒प्मन॑ श्चि॒रम् चि॒रम् पा॒प्मनो॒ निर् णिष् पा॒प्मन॑ श्चि॒रम् चि॒रम् पा॒प्मनो॒ निः । \newline
62. पा॒प्मनो॒ निर् णिष् पा॒प्मनः॑ पा॒प्मनो॒ निर् मु॑च्येत मुच्येत॒ निष् पा॒प्मनः॑ पा॒प्मनो॒ निर् मु॑च्येत । \newline
\pagebreak
\markright{ TS 5.4.5.5  \hfill https://www.vedavms.in \hfill}

\section{ TS 5.4.5.5 }

\textbf{TS 5.4.5.5 } \newline
\textbf{Samhita Paata} \newline

निर्मु॑च्ये॒तेत्येकै॑कं॒ तस्य॑ जुहुयाच्चि॒रमे॒व पा॒प्मनो॒ निर्मु॑च्यते॒ यं का॒मये॑त ता॒जक् पा॒प्मनो॒ निर्मु॑च्ये॒तेति॒ सर्वा॑णि॒ तस्या॑नु॒द्रुत्य॑ जुहुयात् ता॒जगे॒व पा॒प्मनो॒ निर्मु॑च्य॒तेऽथो॒ खलु॒ नानै॒व सू॒क्ताभ्यां᳚ जुहोति॒ नानै॒व सू॒क्तयो᳚र्वी॒र्यं॑ दधा॒त्यथो॒ प्रति॑ष्ठित्यै ॥ \newline

\textbf{Pada Paata} \newline

निरिति॑ । मु॒च्ये॒त॒ । इति॑ । एकै॑क॒मित्येकं᳚-ए॒क॒म् । तस्य॑ । जु॒हु॒या॒त् । चि॒रम् । ए॒व । पा॒प्मनः॑ । निरिति॑ । मु॒च्य॒ते॒ । यम् । का॒मये॑त । ता॒जक् । पा॒प्मनः॑ । निरिति॑ । मु॒च्ये॒त॒ । इति॑ । सर्वा॑णि । तस्य॑ । अ॒नु॒द्रुत्येत्य॑नु-द्रुत्य॑ । जु॒हु॒या॒त् । ता॒जक् । ए॒व । पा॒प्मनः॑ । निरिति॑ । मु॒च्य॒ते॒ । अथो॒ इति॑ । खलु॑ । नाना᳚ । ए॒व । सू॒क्ताभ्या॒मिति॑ सु - उ॒क्ताभ्या᳚म् । जु॒हो॒ति॒ । नाना᳚ । ए॒व । सू॒क्तयो॒रिति॑ सु - उ॒क्तयोः᳚ । वी॒र्य᳚म् । द॒धा॒ति॒ । अथो॒ इति॑ । प्रति॑ष्ठित्या॒ इति॒ प्रति॑ - स्थि॒त्यै॒ ॥  \newline


\textbf{Krama Paata} \newline

निर् मु॑च्येत । मु॒च्ये॒तेति॑ । इत्येकै॑कम् । एकै॑क॒म् तस्य॑ । एकै॑क॒मित्येक᳚म् - ए॒क॒म् । तस्य॑ जहुयात् । जु॒हु॒या॒च्चि॒रम् । चि॒रमे॒व । ए॒व पा॒प्मनः॑ । पा॒प्मनो॒ निः । निर् मु॑च्यते । मु॒च्य॒ते॒ यम् । यम् का॒मये॑त । का॒मये॑त ता॒जक् । ता॒जक् पा॒प्मनः॑ । पा॒प्मनो॒ निः । निर् मु॑च्यत । मु॒च्य॒तेति॑ । इति॒ सर्वा॑णि । सर्वा॑णि॒ तस्य॑ । तस्या॑नु॒द्रुत्य॑ । अ॒नु॒द्रुत्य॑ जुहुयात् । अ॒नु॒द्रुत्येत्य॑नु - द्रुत्य॑ । जु॒हु॒या॒त् ता॒जक् । ता॒जगे॒व । ए॒व पा॒प्मनः॑ । पा॒प्मनो॒ निः । निर् मु॑च्यते । मु॒च्य॒तेऽथो᳚ । अथो॒ खलु॑ । अथो॒ इत्यथो᳚ । खलु॒ नाना᳚ । नानै॒व । ए॒व सू॒क्ताभ्या᳚म् । सू॒क्ताभ्या᳚म् जुहोति । सू॒क्ताभ्या॒मिति॑ सु - उ॒क्ताभ्या᳚म् । जु॒हो॒ति॒ नाना᳚ । नानै॒व । ए॒व सू॒क्तयोः᳚ । सू॒क्तयो᳚र् वी॒र्य᳚म् । सू॒क्तयो॒रिति॑ सु - उ॒क्तयोः᳚ । वी॒र्य॑म् दधाति । द॒धा॒त्यथो᳚ । अथो॒ प्रति॑ष्ठित्यै । अथो॒ इत्यथो᳚ । प्रति॑ष्ठित्या॒ इति॒ प्रति॑ - स्थि॒त्यै॒ । \newline

\textbf{Jatai Paata} \newline

1. निर् मु॑च्येत मुच्येत॒ निर् णिर् मु॑च्येत । \newline
2. मु॒च्ये॒ते तीति॑ मुच्येत मुच्ये॒तेति॑ । \newline
3. इत्येकै॑क॒ मेकै॑क॒ मिती त्येकै॑कम् । \newline
4. एकै॑क॒म् तस्य॒ तस्यैकै॑क॒ मेकै॑क॒म् तस्य॑ । \newline
5. एकै॑क॒मित्येकं᳚ - ए॒क॒म् । \newline
6. तस्य॑ जुहुयाज् जुहुया॒त् तस्य॒ तस्य॑ जुहुयात् । \newline
7. जु॒हु॒या॒च् चि॒रम् चि॒रम् जु॑हुयाज् जुहुयाच् चि॒रम् । \newline
8. चि॒र मे॒वैव चि॒रम् चि॒र मे॒व । \newline
9. ए॒व पा॒प्मनः॑ पा॒प्मन॑ ए॒वैव पा॒प्मनः॑ । \newline
10. पा॒प्मनो॒ निर् णिष् पा॒प्मनः॑ पा॒प्मनो॒ निः । \newline
11. निर् मु॑च्यते मुच्यते॒ निर् णिर् मु॑च्यते । \newline
12. मु॒च्य॒ते॒ यं ॅयम् मु॑च्यते मुच्यते॒ यम् । \newline
13. यम् का॒मये॑त का॒मये॑त॒ यं ॅयम् का॒मये॑त । \newline
14. का॒मये॑त ता॒जक् ता॒जक् का॒मये॑त का॒मये॑त ता॒जक् । \newline
15. ता॒जक् पा॒प्मनः॑ पा॒प्मन॑ स्ता॒जक् ता॒जक् पा॒प्मनः॑ । \newline
16. पा॒प्मनो॒ निर् णिष् पा॒प्मनः॑ पा॒प्मनो॒ निः । \newline
17. निर् मु॑च्येत मुच्येत॒ निर् णिर् मु॑च्येत । \newline
18. मु॒च्ये॒ते तीति॑ मुच्येत मुच्ये॒तेति॑ । \newline
19. इति॒ सर्वा॑णि॒ सर्वा॒णी तीति॒ सर्वा॑णि । \newline
20. सर्वा॑णि॒ तस्य॒ तस्य॒ सर्वा॑णि॒ सर्वा॑णि॒ तस्य॑ । \newline
21. तस्या॑ नु॒द्रुत्या॑ नु॒द्रुत्य॒ तस्य॒ तस्या॑ नु॒द्रुत्य॑ । \newline
22. अ॒नु॒द्रुत्य॑ जुहुयाज् जुहुया दनु॒द्रुत्या॑ नु॒द्रुत्य॑ जुहुयात् । \newline
23. अ॒नु॒द्रुत्येत्य॑नु - द्रुत्य॑ । \newline
24. जु॒हु॒या॒त् ता॒जक् ता॒जग् जु॑हुयाज् जुहुयात् ता॒जक् । \newline
25. ता॒जगे॒ वैव ता॒जक् ता॒जगे॒व । \newline
26. ए॒व पा॒प्मनः॑ पा॒प्मन॑ ए॒वैव पा॒प्मनः॑ । \newline
27. पा॒प्मनो॒ निर् णिष् पा॒प्मनः॑ पा॒प्मनो॒ निः । \newline
28. निर् मु॑च्यते मुच्यते॒ निर् णिर् मु॑च्यते । \newline
29. मु॒च्य॒ते ऽथो॒ अथो॑ मुच्यते मुच्य॒ते ऽथो᳚ । \newline
30. अथो॒ खलु॒ खल्वथो॒ अथो॒ खलु॑ । \newline
31. अथो॒ इत्यथो᳚ । \newline
32. खलु॒ नाना॒ नाना॒ खलु॒ खलु॒ नाना᳚ । \newline
33. नानै॒ वैव नाना॒ नानै॒व । \newline
34. ए॒व सू॒क्ताभ्याꣳ॑ सू॒क्ताभ्या॑ मे॒वैव सू॒क्ताभ्या᳚म् । \newline
35. सू॒क्ताभ्या᳚म् जुहोति जुहोति सू॒क्ताभ्याꣳ॑ सू॒क्ताभ्या᳚म् जुहोति । \newline
36. सू॒क्ताभ्या॒मिति॑ सु - उ॒क्ताभ्या᳚म् । \newline
37. जु॒हो॒ति॒ नाना॒ नाना॑ जुहोति जुहोति॒ नाना᳚ । \newline
38. नानै॒ वैव नाना॒ नानै॒व । \newline
39. ए॒व सू॒क्तयोः᳚ सू॒क्तयो॑ रे॒वैव सू॒क्तयोः᳚ । \newline
40. सू॒क्तयो᳚र् वी॒र्यं॑ ॅवी॒र्यꣳ॑ सू॒क्तयोः᳚ सू॒क्तयो᳚र् वी॒र्य᳚म् । \newline
41. सू॒क्तयो॒रिति॑ सु - उ॒क्तयोः᳚ । \newline
42. वी॒र्य॑म् दधाति दधाति वी॒र्यं॑ ॅवी॒र्य॑म् दधाति । \newline
43. द॒धा॒ त्यथो॒ अथो॑ दधाति दधा॒ त्यथो᳚ । \newline
44. अथो॒ प्रति॑ष्ठित्यै॒ प्रति॑ष्ठित्या॒ अथो॒ अथो॒ प्रति॑ष्ठित्यै । \newline
45. अथो॒ इत्यथो᳚ । \newline
46. प्रति॑ष्ठित्या॒ इति॒ प्रति॑ - स्थि॒त्यै॒ । \newline

\textbf{Ghana Paata } \newline

1. निर् मु॑च्येत मुच्येत॒ निर् णिर् मु॑च्ये॒ते तीति॑ मुच्येत॒ निर् णिर् मु॑च्ये॒तेति॑ । \newline
2. मु॒च्ये॒ते तीति॑ मुच्येत मुच्ये॒ते त्येकै॑क॒ मेकै॑क॒ मिति॑ मुच्येत मुच्ये॒ते त्येकै॑कम् । \newline
3. इत्येकै॑क॒ मेकै॑क॒ मिती त्येकै॑क॒म् तस्य॒ तस्यैकै॑क॒ मिती त्येकै॑क॒म् तस्य॑ । \newline
4. एकै॑क॒म् तस्य॒ तस्यैकै॑क॒ मेकै॑क॒म् तस्य॑ जुहुयाज् जुहुया॒त् तस्यैकै॑क॒ मेकै॑क॒म् तस्य॑ जुहुयात् । \newline
5. एकै॑क॒मित्येकं᳚ - ए॒क॒म् । \newline
6. तस्य॑ जुहुयाज् जुहुया॒त् तस्य॒ तस्य॑ जुहुयाच् चि॒रम् चि॒रम् जु॑हुया॒त् तस्य॒ तस्य॑ जुहुयाच् चि॒रम् । \newline
7. जु॒हु॒या॒च् चि॒रम् चि॒रम् जु॑हुयाज् जुहुयाच् चि॒र मे॒वैव चि॒रम् जु॑हुयाज् जुहुयाच् चि॒र मे॒व । \newline
8. चि॒र मे॒वैव चि॒रम् चि॒र मे॒व पा॒प्मनः॑ पा॒प्मन॑ ए॒व चि॒रम् चि॒र मे॒व पा॒प्मनः॑ । \newline
9. ए॒व पा॒प्मनः॑ पा॒प्मन॑ ए॒वैव पा॒प्मनो॒ निर् णिष् पा॒प्मन॑ ए॒वैव पा॒प्मनो॒ निः । \newline
10. पा॒प्मनो॒ निर् णिष् पा॒प्मनः॑ पा॒प्मनो॒ निर् मु॑च्यते मुच्यते॒ निष् पा॒प्मनः॑ पा॒प्मनो॒ निर् मु॑च्यते । \newline
11. निर् मु॑च्यते मुच्यते॒ निर् णिर् मु॑च्यते॒ यं ॅयम् मु॑च्यते॒ निर् णिर् मु॑च्यते॒ यम् । \newline
12. मु॒च्य॒ते॒ यं ॅयम् मु॑च्यते मुच्यते॒ यम् का॒मये॑त का॒मये॑त॒ यम् मु॑च्यते मुच्यते॒ यम् का॒मये॑त । \newline
13. यम् का॒मये॑त का॒मये॑त॒ यं ॅयम् का॒मये॑त ता॒जक् ता॒जक् का॒मये॑त॒ यं ॅयम् का॒मये॑त ता॒जक् । \newline
14. का॒मये॑त ता॒जक् ता॒जक् का॒मये॑त का॒मये॑त ता॒जक् पा॒प्मनः॑ पा॒प्मन॑ स्ता॒जक् का॒मये॑त का॒मये॑त ता॒जक् पा॒प्मनः॑ । \newline
15. ता॒जक् पा॒प्मनः॑ पा॒प्मन॑ स्ता॒जक् ता॒जक् पा॒प्मनो॒ निर् णिष् पा॒प्मन॑ स्ता॒जक् ता॒जक् पा॒प्मनो॒ निः । \newline
16. पा॒प्मनो॒ निर् णिष् पा॒प्मनः॑ पा॒प्मनो॒ निर् मु॑च्येत मुच्येत॒ निष् पा॒प्मनः॑ पा॒प्मनो॒ निर् मु॑च्येत । \newline
17. निर् मु॑च्येत मुच्येत॒ निर् णिर् मु॑च्ये॒ते तीति॑ मुच्येत॒ निर् णिर् मु॑च्ये॒तेति॑ । \newline
18. मु॒च्ये॒ते तीति॑ मुच्येत मुच्ये॒तेति॒ सर्वा॑णि॒ सर्वा॒णीति॑ मुच्येत मुच्ये॒तेति॒ सर्वा॑णि । \newline
19. इति॒ सर्वा॑णि॒ सर्वा॒णी तीति॒ सर्वा॑णि॒ तस्य॒ तस्य॒ सर्वा॒णी तीति॒ सर्वा॑णि॒ तस्य॑ । \newline
20. सर्वा॑णि॒ तस्य॒ तस्य॒ सर्वा॑णि॒ सर्वा॑णि॒ तस्या॑ नु॒द्रुत्या॑ नु॒द्रुत्य॒ तस्य॒ सर्वा॑णि॒ सर्वा॑णि॒ तस्या॑ नु॒द्रुत्य॑ । \newline
21. तस्या॑ नु॒द्रुत्या॑ नु॒द्रुत्य॒ तस्य॒ तस्या॑ नु॒द्रुत्य॑ जुहुयाज् जुहुया दनु॒द्रुत्य॒ तस्य॒ तस्या॑ नु॒द्रुत्य॑ जुहुयात् । \newline
22. अ॒नु॒द्रुत्य॑ जुहुयाज् जुहुया दनु॒द्रुत्या॑ नु॒द्रुत्य॑ जुहुयात् ता॒जक् ता॒जग् जु॑हुया दनु॒द्रुत्या॑ नु॒द्रुत्य॑ जुहुयात् ता॒जक् । \newline
23. अ॒नु॒द्रुत्येत्य॑नु - द्रुत्य॑ । \newline
24. जु॒हु॒या॒त् ता॒जक् ता॒जग् जु॑हुयाज् जुहुयात् ता॒जगे॒वैव ता॒जग् जु॑हुयाज् जुहुयात् ता॒जगे॒व । \newline
25. ता॒जगे॒वैव ता॒जक् ता॒जगे॒व पा॒प्मनः॑ पा॒प्मन॑ ए॒व ता॒जक् ता॒जगे॒व पा॒प्मनः॑ । \newline
26. ए॒व पा॒प्मनः॑ पा॒प्मन॑ ए॒वैव पा॒प्मनो॒ निर् णिष् पा॒प्मन॑ ए॒वैव पा॒प्मनो॒ निः । \newline
27. पा॒प्मनो॒ निर् णिष् पा॒प्मनः॑ पा॒प्मनो॒ निर् मु॑च्यते मुच्यते॒ निष् पा॒प्मनः॑ पा॒प्मनो॒ निर् मु॑च्यते । \newline
28. निर् मु॑च्यते मुच्यते॒ निर् णिर् मु॑च्य॒ते ऽथो॒ अथो॑ मुच्यते॒ निर् णिर् मु॑च्य॒ते ऽथो᳚ । \newline
29. मु॒च्य॒ते ऽथो॒ अथो॑ मुच्यते मुच्य॒ते ऽथो॒ खलु॒ खल्वथो॑ मुच्यते मुच्य॒ते ऽथो॒ खलु॑ । \newline
30. अथो॒ खलु॒ खल्वथो॒ अथो॒ खलु॒ नाना॒ नाना॒ खल्वथो॒ अथो॒ खलु॒ नाना᳚ । \newline
31. अथो॒ इत्यथो᳚ । \newline
32. खलु॒ नाना॒ नाना॒ खलु॒ खलु॒ नानै॒ वैव नाना॒ खलु॒ खलु॒ नानै॒व । \newline
33. नानै॒ वैव नाना॒ नानै॒व सू॒क्ताभ्याꣳ॑ सू॒क्ताभ्या॑ मे॒व नाना॒ नानै॒व सू॒क्ताभ्या᳚म् । \newline
34. ए॒व सू॒क्ताभ्याꣳ॑ सू॒क्ताभ्या॑ मे॒वैव सू॒क्ताभ्या᳚म् जुहोति जुहोति सू॒क्ताभ्या॑ मे॒वैव सू॒क्ताभ्या᳚म् जुहोति । \newline
35. सू॒क्ताभ्या᳚म् जुहोति जुहोति सू॒क्ताभ्याꣳ॑ सू॒क्ताभ्या᳚म् जुहोति॒ नाना॒ नाना॑ जुहोति सू॒क्ताभ्याꣳ॑ सू॒क्ताभ्या᳚म् जुहोति॒ नाना᳚ । \newline
36. सू॒क्ताभ्या॒मिति॑ सु - उ॒क्ताभ्या᳚म् । \newline
37. जु॒हो॒ति॒ नाना॒ नाना॑ जुहोति जुहोति॒ नानै॒ वैव नाना॑ जुहोति जुहोति॒ नानै॒व । \newline
38. नानै॒ वैव नाना॒ नानै॒व सू॒क्तयोः᳚ सू॒क्तयो॑ रे॒व नाना॒ नानै॒व सू॒क्तयोः᳚ । \newline
39. ए॒व सू॒क्तयोः᳚ सू॒क्तयो॑ रे॒वैव सू॒क्तयो᳚र् वी॒र्यं॑ ॅवी॒र्यꣳ॑ सू॒क्तयो॑ रे॒वैव सू॒क्तयो᳚र् वी॒र्य᳚म् । \newline
40. सू॒क्तयो᳚र् वी॒र्यं॑ ॅवी॒र्यꣳ॑ सू॒क्तयोः᳚ सू॒क्तयो᳚र् वी॒र्य॑म् दधाति दधाति वी॒र्यꣳ॑ सू॒क्तयोः᳚ सू॒क्तयो᳚र् वी॒र्य॑म् दधाति । \newline
41. सू॒क्तयो॒रिति॑ सु - उ॒क्तयोः᳚ । \newline
42. वी॒र्य॑म् दधाति दधाति वी॒र्यं॑ ॅवी॒र्य॑म् दधा॒ त्यथो॒ अथो॑ दधाति वी॒र्यं॑ ॅवी॒र्य॑म् दधा॒ त्यथो᳚ । \newline
43. द॒धा॒ त्यथो॒ अथो॑ दधाति दधा॒ त्यथो॒ प्रति॑ष्ठित्यै॒ प्रति॑ष्ठित्या॒ अथो॑ दधाति दधा॒ त्यथो॒ प्रति॑ष्ठित्यै । \newline
44. अथो॒ प्रति॑ष्ठित्यै॒ प्रति॑ष्ठित्या॒ अथो॒ अथो॒ प्रति॑ष्ठित्यै । \newline
45. अथो॒ इत्यथो᳚ । \newline
46. प्रति॑ष्ठित्या॒ इति॒ प्रति॑ - स्थि॒त्यै॒ । \newline
\pagebreak
\markright{ TS 5.4.6.1  \hfill https://www.vedavms.in \hfill}

\section{ TS 5.4.6.1 }

\textbf{TS 5.4.6.1 } \newline
\textbf{Samhita Paata} \newline

उदे॑नमुत्त॒रां न॒येति॑ स॒मिध॒ आ द॑धाति॒ यथा॒ जनं॑ ॅय॒ते॑ऽव॒सं क॒रोति॑ ता॒दृगे॒व तत् ति॒स्र आ द॑धाति त्रि॒वृद्वा अ॒ग्निर्यावा॑ने॒-वाग्निस्तस्मै॑ भाग॒धेयं॑ करो॒त्यौदु॑म्बरी-र्भव॒न्त्यूर्ग्वा उ॑दु॒म्बर॒ ऊर्ज॑मे॒वास्मा॒ अपि॑ दधा॒त्युदु॑ त्वा॒ विश्वे॑ दे॒वा इत्या॑ह प्रा॒णा वै विश्वे॑ दे॒वाः प्रा॒णै - [  ] \newline

\textbf{Pada Paata} \newline

उदिति॑ । ए॒न॒म् । उ॒त्त॒रामित्यु॑त् - त॒राम् । न॒य॒ । इति॑ । स॒मिध॒ इति॑ सं - इधः॑ । एति॑ । द॒धा॒ति॒ । यथा᳚ । जन᳚म् । य॒ते । अ॒व॒सम् । क॒रोति॑ । ता॒दृक् । ए॒व । तत् । ति॒स्रः । एति॑ । द॒धा॒ति॒ । त्रि॒वृदिति॑ त्रि - वृत् । वै । अ॒ग्निः । यावान्॑ । ए॒व । अ॒ग्निः । तस्मै᳚ । भा॒ग॒धेय॒मिति॑ भाग - धेय᳚म् । क॒रो॒ति॒ । औदु॑बंरीः । भ॒व॒न्ति॒ । ऊर्क् । वै । उ॒दु॒बंरः॑ । ऊर्ज᳚म् । ए॒व । अ॒स्मै॒ । अपीति॑ । द॒धा॒ति॒ । उदिति॑ । उ॒ । त्वा॒ । विश्वे᳚ । दे॒वाः । इति॑ । आ॒ह॒ । प्रा॒णा इति॑ प्र - अ॒नाः । वै । विश्वे᳚ । दे॒वाः । प्रा॒णैरिति॑ प्र - अ॒नैः ।  \newline


\textbf{Krama Paata} \newline

उदे॑नम् । ए॒न॒मु॒त्त॒राम् । उ॒त्त॒राम् न॑य । उ॒त्त॒रामित्यु॑त् - त॒राम् । न॒येति॑ । इति॑ स॒मिधः॑ । स॒मिध॒ आ । स॒मिध॒ इति॑ सम् - इधः॑ । आ द॑धाति । द॒धा॒ति॒ यथा᳚ । यथा॒ जन᳚म् । जन॑म् ॅय॒ते । य॒ते॑ऽव॒सम् । अ॒व॒सम् क॒रोति॑ । क॒रोति॑ ता॒दृक् । ता॒दृगे॒व । ए॒व तत् । तत् ति॒स्रः । ति॒स्र आ । आ द॑धाति । द॒धा॒ति॒ त्रि॒वृत् । त्रि॒वृद् वै । त्रि॒वृदिति॑ त्रि - वृत् । वा अ॒ग्निः । अ॒ग्निर् यावान्॑ । यावा॑ने॒व । ए॒वाग्निः । अ॒ग्निस्तस्मै᳚ । तस्मै॑ भाग॒धेय᳚म् । भा॒ग॒धेय॑म् करोति । भा॒ग॒धेय॒मिति॑ भाग - धेय᳚म् । क॒रो॒त्यौदु॑म्बरीः । औदु॑म्बरीर् भवन्ति । भ॒व॒न्त्यूर्क् । ऊर्ग् वै । वा उ॑दु॒म्बरः॑ । उ॒दु॒म्बर॒ ऊर्ज᳚म् । ऊर्ज॑मे॒व । ए॒वास्मै᳚ । अ॒स्मा॒ अपि॑ । अपि॑ दधाति । द॒धा॒त्युत् । उदु॑ । उ ॒त्वा॒ । त्वा॒ विश्वे᳚ । विश्वे॑ दे॒वाः । दे॒वा इति॑ । इत्या॑ह । आ॒ह॒ प्रा॒णाः । प्रा॒णा वै । प्रा॒णा इति॑ प्र - अ॒नाः । वै विश्वे᳚ । विश्वे॑ दे॒वाः । दे॒वाः प्रा॒णैः । प्रा॒णैरे॒व । प्रा॒णैरिति॑ प्र - अ॒नैः \newline

\textbf{Jatai Paata} \newline

1. उदे॑न मेन॒ मुदु दे॑नम् । \newline
2. ए॒न॒ मु॒त्त॒रा मु॑त्त॒रा मे॑न मेन मुत्त॒राम् । \newline
3. उ॒त्त॒राम् न॑य नयोत्त॒रा मु॑त्त॒राम् न॑य । \newline
4. उ॒त्त॒रामित्यु॑त् - त॒राम् । \newline
5. न॒येतीति॑ नय न॒येति॑ । \newline
6. इति॑ स॒मिधः॑ स॒मिध॒ इतीति॑ स॒मिधः॑ । \newline
7. स॒मिध॒ आ स॒मिधः॑ स॒मिध॒ आ । \newline
8. स॒मिध॒ इति॑ सं - इधः॑ । \newline
9. आ द॑धाति दधा॒त्या द॑धाति । \newline
10. द॒धा॒ति॒ यथा॒ यथा॑ दधाति दधाति॒ यथा᳚ । \newline
11. यथा॒ जन॒म् जनं॒ ॅयथा॒ यथा॒ जन᳚म् । \newline
12. जनं॑ ॅय॒ते य॒ते जन॒म् जनं॑ ॅय॒ते । \newline
13. य॒ते॑ ऽव॒स म॑व॒सं ॅय॒ते य॒ते॑ ऽव॒सम् । \newline
14. अ॒व॒सम् क॒रोति॑ क॒रो त्य॑व॒स म॑व॒सम् क॒रोति॑ । \newline
15. क॒रोति॑ ता॒दृक् ता॒दृक् क॒रोति॑ क॒रोति॑ ता॒दृक् । \newline
16. ता॒दृ गे॒वैव ता॒दृक् ता॒दृ गे॒व । \newline
17. ए॒व तत् तदे॒ वैव तत् । \newline
18. तत् ति॒स्र स्ति॒स्र स्तत् तत् ति॒स्रः । \newline
19. ति॒स्र आ ति॒स्र स्ति॒स्र आ । \newline
20. आ द॑धाति दधा॒त्या द॑धाति । \newline
21. द॒धा॒ति॒ त्रि॒वृत् त्रि॒वृद् द॑धाति दधाति त्रि॒वृत् । \newline
22. त्रि॒वृद् वै वै त्रि॒वृत् त्रि॒वृद् वै । \newline
23. त्रि॒वृदिति॑ त्रि - वृत् । \newline
24. वा अ॒ग्नि र॒ग्निर् वै वा अ॒ग्निः । \newline
25. अ॒ग्निर् यावा॒न्॒. यावा॑ न॒ग्नि र॒ग्निर् यावान्॑ । \newline
26. यावा॑ ने॒वैव यावा॒न्॒. यावा॑ ने॒व । \newline
27. ए॒वाग्नि र॒ग्नि रे॒वै वाग्निः । \newline
28. अ॒ग्निस् तस्मै॒ तस्मा॑ अ॒ग्नि र॒ग्नि स्तस्मै᳚ । \newline
29. तस्मै॑ भाग॒धेय॑म् भाग॒धेय॒म् तस्मै॒ तस्मै॑ भाग॒धेय᳚म् । \newline
30. भा॒ग॒धेय॑म् करोति करोति भाग॒धेय॑म् भाग॒धेय॑म् करोति । \newline
31. भा॒ग॒धेय॒मिति॑ भाग - धेय᳚म् । \newline
32. क॒रो॒ त्यौदुं॑बरी॒ रौदुं॑बरीः करोति करो॒ त्यौदुं॑बरीः । \newline
33. औदुं॑बरीर् भवन्ति भव॒न् त्यौदुं॑बरी॒ रौदुं॑बरीर् भवन्ति । \newline
34. भ॒व॒न् त्यूर् गूर्ग् भ॑वन्ति भव॒न् त्यूर्क् । \newline
35. ऊर्ग् वै वा ऊर् गूर्ग् वै । \newline
36. वा उ॑दुं॒बर॑ उदुं॒बरो॒ वै वा उ॑दुं॒बरः॑ । \newline
37. उ॒दुं॒बर॒ ऊर्ज॒ मूर्ज॑ मुदुं॒बर॑ उदुं॒बर॒ ऊर्ज᳚म् । \newline
38. ऊर्ज॑ मे॒वै वोर्ज॒ मूर्ज॑ मे॒व । \newline
39. ए॒वास्मा॑ अस्मा ए॒वै वास्मै᳚ । \newline
40. अ॒स्मा॒ अप्य प्य॑स्मा अस्मा॒ अपि॑ । \newline
41. अपि॑ दधाति दधा॒ त्यप्यपि॑ दधाति । \newline
42. द॒धा॒ त्युदुद् द॑धाति दधा॒ त्युत् । \newline
43. उदु॑ वु॒ वु दुदु॑ । \newline
44. उ॒ त्वा॒ त्व॒ वु॒ त्वा॒ । \newline
45. त्वा॒ विश्वे॒ विश्वे᳚ त्वा त्वा॒ विश्वे᳚ । \newline
46. विश्वे॑ दे॒वा दे॒वा विश्वे॒ विश्वे॑ दे॒वाः । \newline
47. दे॒वा इतीति॑ दे॒वा दे॒वा इति॑ । \newline
48. इत्या॑हा॒हे तीत्या॑ह । \newline
49. आ॒ह॒ प्रा॒णाः प्रा॒णा आ॑हाह प्रा॒णाः । \newline
50. प्रा॒णा वै वै प्रा॒णाः प्रा॒णा वै । \newline
51. प्रा॒णा इति॑ प्र - अ॒नाः । \newline
52. वै विश्वे॒ विश्वे॒ वै वै विश्वे᳚ । \newline
53. विश्वे॑ दे॒वा दे॒वा विश्वे॒ विश्वे॑ दे॒वाः । \newline
54. दे॒वाः प्रा॒णैः प्रा॒णैर् दे॒वा दे॒वाः प्रा॒णैः । \newline
55. प्रा॒णै रे॒वैव प्रा॒णैः प्रा॒णैरे॒व । \newline
56. प्रा॒णैरिति॑ प्र - अ॒नैः । \newline

\textbf{Ghana Paata } \newline

1. उदे॑न मेन॒ मुदुदे॑न मुत्त॒रा मु॑त्त॒रा मे॑न॒ मुदुदे॑न मुत्त॒राम् । \newline
2. ए॒न॒ मु॒त्त॒रा मु॑त्त॒रा मे॑न मेन मुत्त॒रान् न॑य नयोत्त॒रा मे॑न मेन मुत्त॒रान् न॑य । \newline
3. उ॒त्त॒रान् न॑य नयोत्त॒रा मु॑त्त॒रान् न॒ये तीति॑ नयोत्त॒रा मु॑त्त॒रान् न॒येति॑ । \newline
4. उ॒त्त॒रामित्यु॑त् - त॒राम् । \newline
5. न॒ये तीति॑ नय न॒येति॑ स॒मिधः॑ स॒मिध॒ इति॑ नय न॒येति॑ स॒मिधः॑ । \newline
6. इति॑ स॒मिधः॑ स॒मिध॒ इतीति॑ स॒मिध॒ आ स॒मिध॒ इतीति॑ स॒मिध॒ आ । \newline
7. स॒मिध॒ आ स॒मिधः॑ स॒मिध॒ आ द॑धाति दधा॒त्या स॒मिधः॑ स॒मिध॒ आ द॑धाति । \newline
8. स॒मिध॒ इति॑ सं - इधः॑ । \newline
9. आ द॑धाति दधा॒त्या द॑धाति॒ यथा॒ यथा॑ दधा॒त्या द॑धाति॒ यथा᳚ । \newline
10. द॒धा॒ति॒ यथा॒ यथा॑ दधाति दधाति॒ यथा॒ जन॒म् जनं॒ ॅयथा॑ दधाति दधाति॒ यथा॒ जन᳚म् । \newline
11. यथा॒ जन॒म् जनं॒ ॅयथा॒ यथा॒ जनं॑ ॅय॒ते य॒ते जनं॒ ॅयथा॒ यथा॒ जनं॑ ॅय॒ते । \newline
12. जनं॑ ॅय॒ते य॒ते जन॒म् जनं॑ ॅय॒ते॑ ऽव॒स म॑व॒सं ॅय॒ते जन॒म् जनं॑ ॅय॒ते॑ ऽव॒सम् । \newline
13. य॒ते॑ ऽव॒स म॑व॒सं ॅय॒ते य॒ते॑ ऽव॒सम् क॒रोति॑ क॒रो त्य॑व॒सं ॅय॒ते य॒ते॑ ऽव॒सम् क॒रोति॑ । \newline
14. अ॒व॒सम् क॒रोति॑ क॒रो त्य॑व॒स म॑व॒सम् क॒रोति॑ ता॒दृक् ता॒दृक् क॒रो त्य॑व॒स म॑व॒सम् क॒रोति॑ ता॒दृक् । \newline
15. क॒रोति॑ ता॒दृक् ता॒दृक् क॒रोति॑ क॒रोति॑ ता॒दृ गे॒वैव ता॒दृक् क॒रोति॑ क॒रोति॑ ता॒दृ गे॒व । \newline
16. ता॒दृ गे॒वैव ता॒दृक् ता॒दृ गे॒व तत् तदे॒व ता॒दृक् ता॒दृ गे॒व तत् । \newline
17. ए॒व तत् तदे॒ वैव तत् ति॒स्र स्ति॒स्र स्तदे॒वैव तत् ति॒स्रः । \newline
18. तत् ति॒स्र स्ति॒स्र स्तत् तत् ति॒स्र आ ति॒स्र स्तत् तत् ति॒स्र आ । \newline
19. ति॒स्र आ ति॒स्र स्ति॒स्र आ द॑धाति दधा॒त्या ति॒स्र स्ति॒स्र आ द॑धाति । \newline
20. आ द॑धाति दधा॒त्या द॑धाति त्रि॒वृत् त्रि॒वृद् द॑धा॒त्या द॑धाति त्रि॒वृत् । \newline
21. द॒धा॒ति॒ त्रि॒वृत् त्रि॒वृद् द॑धाति दधाति त्रि॒वृद् वै वै त्रि॒वृद् द॑धाति दधाति त्रि॒वृद् वै । \newline
22. त्रि॒वृद् वै वै त्रि॒वृत् त्रि॒वृद् वा अ॒ग्नि र॒ग्निर् वै त्रि॒वृत् त्रि॒वृद् वा अ॒ग्निः । \newline
23. त्रि॒वृदिति॑ त्रि - वृत् । \newline
24. वा अ॒ग्नि र॒ग्निर् वै वा अ॒ग्निर् यावा॒न्॒. यावा॑ न॒ग्निर् वै वा अ॒ग्निर् यावान्॑ । \newline
25. अ॒ग्निर् यावा॒न्॒. यावा॑ न॒ग्नि र॒ग्निर् यावा॑ ने॒वैव यावा॑ न॒ग्नि र॒ग्निर् यावा॑ ने॒व । \newline
26. यावा॑ ने॒वैव यावा॒न्॒. यावा॑ ने॒वाग्नि र॒ग्नि रे॒व यावा॒न्॒. यावा॑ ने॒वाग्निः । \newline
27. ए॒वाग्नि र॒ग्नि रे॒वै वाग्नि स्तस्मै॒ तस्मा॑ अ॒ग्नि रे॒वै वाग्नि स्तस्मै᳚ । \newline
28. अ॒ग्नि स्तस्मै॒ तस्मा॑ अ॒ग्नि र॒ग्नि स्तस्मै॑ भाग॒धेय॑म् भाग॒धेय॒म् तस्मा॑ अ॒ग्नि र॒ग्नि स्तस्मै॑ भाग॒धेय᳚म् । \newline
29. तस्मै॑ भाग॒धेय॑म् भाग॒धेय॒म् तस्मै॒ तस्मै॑ भाग॒धेय॑म् करोति करोति भाग॒धेय॒म् तस्मै॒ तस्मै॑ भाग॒धेय॑म् करोति । \newline
30. भा॒ग॒धेय॑म् करोति करोति भाग॒धेय॑म् भाग॒धेय॑म् करो॒ त्यौदुं॑बरी॒ रौदुं॑बरीः करोति भाग॒धेय॑म् भाग॒धेय॑म् करो॒ त्यौदुं॑बरीः । \newline
31. भा॒ग॒धेय॒मिति॑ भाग - धेय᳚म् । \newline
32. क॒रो॒ त्यौदुं॑ब री॒रौदुं॑बरीः करोति करो॒ त्यौदुं॑बरीर् भवन्ति भव॒न् त्यौदुं॑बरीः करोति करो॒ त्यौदुं॑बरीर् भवन्ति । \newline
33. औदुं॑बरीर् भवन्ति भव॒न् त्यौदुं॑ब री॒रौदुं॑बरीर् भव॒न् त्यूर् गूर्ग् भ॑व॒न् त्यौदुं॑बरी॒ रौदुं॑बरीर् भव॒न् त्यूर्क् । \newline
34. भ॒व॒न् त्यूर् गूर्ग् भ॑वन्ति भव॒न् त्यूर्ग् वै वा ऊर्ग् भ॑वन्ति भव॒न् त्यूर्ग् वै । \newline
35. ऊर्ग् वै वा ऊर् गूर्ग् वा उ॑दुं॒बर॑ उदुं॒बरो॒ वा ऊर् गूर्ग् वा उ॑दुं॒बरः॑ । \newline
36. वा उ॑दुं॒बर॑ उदुं॒बरो॒ वै वा उ॑दुं॒बर॒ ऊर्ज॒ मूर्ज॑ मुदुं॒बरो॒ वै वा उ॑दुं॒बर॒ ऊर्ज᳚म् । \newline
37. उ॒दुं॒बर॒ ऊर्ज॒ मूर्ज॑ मुदुं॒बर॑ उदुं॒बर॒ ऊर्ज॑ मे॒वैवोर्ज॑ मुदुं॒बर॑ उदुं॒बर॒ ऊर्ज॑ मे॒व । \newline
38. ऊर्ज॑ मे॒वै वोर्ज॒ मूर्ज॑ मे॒वास्मा॑ अस्मा ए॒वोर्ज॒ मूर्ज॑ मे॒वास्मै᳚ । \newline
39. ए॒वास्मा॑ अस्मा ए॒वैवास्मा॒ अप्य प्य॑स्मा ए॒वैवास्मा॒ अपि॑ । \newline
40. अ॒स्मा॒ अप्य प्य॑स्मा अस्मा॒ अपि॑ दधाति दधा॒ त्यप्य॑स्मा अस्मा॒ अपि॑ दधाति । \newline
41. अपि॑ दधाति दधा॒ त्यप्यपि॑ दधा॒ त्युदुद् द॑धा॒ त्यप्यपि॑ दधा॒ त्युत् । \newline
42. द॒धा॒ त्युदुद् द॑धाति दधा॒ त्युदु॑ वु॒ वुद् द॑धाति दधा॒ त्युदु॑ । \newline
43. उदु॑ वु॒ वुदुदु॑ त्वा त्व॒ वुदुदु॑ त्वा । \newline
44. उ॒ त्वा॒ त्व॒ वु॒ त्वा॒ विश्वे॒ विश्वे᳚ त्व वु त्वा॒ विश्वे᳚ । \newline
45. त्वा॒ विश्वे॒ विश्वे᳚ त्वा त्वा॒ विश्वे॑ दे॒वा दे॒वा विश्वे᳚ त्वा त्वा॒ विश्वे॑ दे॒वाः । \newline
46. विश्वे॑ दे॒वा दे॒वा विश्वे॒ विश्वे॑ दे॒वा इतीति॑ दे॒वा विश्वे॒ विश्वे॑ दे॒वा इति॑ । \newline
47. दे॒वा इतीति॑ दे॒वा दे॒वा इत्या॑हा॒हेति॑ दे॒वा दे॒वा इत्या॑ह । \newline
48. इत्या॑हा॒हे तीत्या॑ह प्रा॒णाः प्रा॒णा आ॒हे तीत्या॑ह प्रा॒णाः । \newline
49. आ॒ह॒ प्रा॒णाः प्रा॒णा आ॑हाह प्रा॒णा वै वै प्रा॒णा आ॑हाह प्रा॒णा वै । \newline
50. प्रा॒णा वै वै प्रा॒णाः प्रा॒णा वै विश्वे॒ विश्वे॒ वै प्रा॒णाः प्रा॒णा वै विश्वे᳚ । \newline
51. प्रा॒णा इति॑ प्र - अ॒नाः । \newline
52. वै विश्वे॒ विश्वे॒ वै वै विश्वे॑ दे॒वा दे॒वा विश्वे॒ वै वै विश्वे॑ दे॒वाः । \newline
53. विश्वे॑ दे॒वा दे॒वा विश्वे॒ विश्वे॑ दे॒वाः प्रा॒णैः प्रा॒णैर् दे॒वा विश्वे॒ विश्वे॑ दे॒वाः प्रा॒णैः । \newline
54. दे॒वाः प्रा॒णैः प्रा॒णैर् दे॒वा दे॒वाः प्रा॒णै रे॒वैव प्रा॒णैर् दे॒वा दे॒वाः प्रा॒णै रे॒व । \newline
55. प्रा॒णै रे॒वैव प्रा॒णैः प्रा॒णै रे॒वैन॑ मेन मे॒व प्रा॒णैः प्रा॒णै रे॒वैन᳚म् । \newline
56. प्रा॒णैरिति॑ प्र - अ॒नैः । \newline
\pagebreak
\markright{ TS 5.4.6.2  \hfill https://www.vedavms.in \hfill}

\section{ TS 5.4.6.2 }

\textbf{TS 5.4.6.2 } \newline
\textbf{Samhita Paata} \newline

-रे॒वैन॒मुद्य॑च्छ॒ते ऽग्ने॒ भर॑न्तु॒ चित्ति॑भि॒रित्या॑ह॒ यस्मा॑ ए॒वैनं॑ चि॒त्तायो॒द्यच्छ॑ते॒ तेनै॒वैनꣳ॒॒ सम॑र्द्धयति॒ पञ्च॒ दिशो॒ दैवी᳚र्य॒ज्ञ्म॑वन्तु दे॒वीरित्या॑ह॒ दिशो॒ ह्ये॑षोऽनु॑ प्र॒च्यव॒ते ऽपाम॑तिं दुर्म॒तिं बाध॑माना॒ इत्या॑ह॒ रक्ष॑सा॒मप॑हत्यै रा॒यस्पोषे॑ य॒ज्ञ्प॑ति-मा॒भज॑न्ती॒रित्या॑ह प॒शवो॒ वै रा॒यस्पोषः॑ - [  ] \newline

\textbf{Pada Paata} \newline

ए॒व । ए॒न॒म् । उदिति॑ । य॒च्छ॒ते॒ । अग्ने᳚ । भर॑न्तु । चित्ति॑भि॒रिति॒ चित्ति॑ - भिः॒ । इति॑ । आ॒ह॒ । यस्मै᳚ । ए॒व । ए॒न॒म् । चि॒त्ताय॑ । उ॒द्यच्छ॑त॒ इत्यु॑त्-यच्छ॑ते । तेन॑ । ए॒व । ए॒न॒म् । समिति॑ । अ॒र्द्ध॒य॒ति॒ । पञ्च॑ । दिशः॑ । दैवीः᳚ । य॒ज्ञ्म् । अ॒व॒न्तु॒ । दे॒वीः । इति॑ । आ॒ह॒ । दिशः॑ । हि । ए॒षः । अन्विति॑ । प्र॒च्यव॑त॒ इति॑ प्र - च्यव॑ते । अपेति॑ । अम॑तिम् । दु॒र्म॒तिमिति॑ दुः - म॒तिम् । बाध॑मानाः । इति॑ । आ॒ह॒ । रक्ष॑साम् । अप॑हत्या॒ इत्यप॑ - ह॒त्यै॒ । रा॒यः । पोषे᳚ । य॒ज्ञ्प॑ति॒मिति॑ य॒ज्ञ्-प॒ति॒म् । आ॒भज॑न्ती॒रित्या᳚ - भज॑न्तीः । इति॑ । आ॒ह॒ । प॒शवः॑ । वै । रा॒यः । पोषः॑ ।  \newline


\textbf{Krama Paata} \newline

ए॒वैन᳚म् । ए॒न॒मुत् । उद् य॑च्छते । य॒च्छ॒तेऽग्ने᳚ । अग्ने॒ भर॑न्तु । भर॑न्तु॒ चित्ति॑भिः । चित्ति॑भि॒रिति॑ । चित्ति॑भि॒रिति॒ चित्ति॑ - भिः॒ । इत्या॑ह । आ॒ह॒ यस्मै᳚ । यस्मा॑ ए॒व । ए॒वैन᳚म् । ए॒न॒म् चि॒त्ताय॑ । चि॒त्तायो॒द्यच्छ॑ते । उ॒द्यच्छ॑ते॒ तेन॑ । उ॒द्यच्छ॑त॒ इत्यु॑त् - यच्छ॑ते । तेनै॒व । 
ए॒वैन᳚म् । ए॒नꣳ॒॒ सम् । सम॑र्द्धयति । अ॒र्द्ध॒य॒ति॒ पञ्च॑ । पञ्च॒ दिशः॑ । दिशो॒ दैवीः᳚ । दैवी᳚र् य॒ज्ञ्म् । य॒ज्ञ्म॑वन्तु । अ॒व॒न्तु॒ दे॒वीः । दे॒वीरिति॑ । इत्या॑ह । आ॒ह॒ दिशः॑ । दिशो॒ हि । ह्ये॑षः । ए॒षोऽनु॑ । अनु॑ प्र॒च्यव॑ते । प्र॒च्यव॒तेऽप॑ । प्र॒च्यव॑त॒ इति॑ प्र - च्यव॑ते । अपाम॑तिम् । अम॑तिम् दुर्म॒तिम् । दु॒र्म॒तिम् बाध॑मानाः । दु॒र्म॒तिमिति॑ दुः - म॒तिम् । बाध॑माना॒ इति॑ । इत्या॑ह । आ॒ह॒ रक्ष॑साम् । रक्ष॑सा॒मप॑हत्यै । अप॑हत्यै रा॒यः । अप॑हत्या॒ इत्यप॑ - ह॒त्यै॒ । रा॒यस्पोषे᳚ । पोषे॑ य॒ज्ञ्प॑तिम् । य॒ज्ञ्प॑तिमा॒भज॑न्तीः । य॒ज्ञ्प॑ति॒मिति॑ य॒ज्ञ् - प॒ति॒म् । आ॒भज॑न्ती॒रिति॑ । आ॒भज॑न्ती॒रित्या᳚ - भज॑न्तीः । इत्या॑ह । आ॒ह॒ प॒शवः॑ । प॒शवो॒ वै । वै रा॒यः । रा॒यस्पोषः॑ । पोषः॑ प॒शून् \newline

\textbf{Jatai Paata} \newline

1. ए॒वैन॑ मेन मे॒वै वैन᳚म् । \newline
2. ए॒न॒ मुदु दे॑न मेन॒ मुत् । \newline
3. उद् य॑च्छते यच्छत॒ उदुद् य॑च्छते । \newline
4. य॒च्छ॒ते ऽग्ने ऽग्ने॑ यच्छते यच्छ॒ते ऽग्ने᳚ । \newline
5. अग्ने॒ भर॑न्तु॒ भर॒न् त्वग्ने ऽग्ने॒ भर॑न्तु । \newline
6. भर॑न्तु॒ चित्ति॑भि॒ श्चित्ति॑भि॒र् भर॑न्तु॒ भर॑न्तु॒ चित्ति॑भिः । \newline
7. चित्ति॑भि॒ रितीति॒ चित्ति॑भि॒ श्चित्ति॑भि॒ रिति॑ । \newline
8. चित्ति॑भि॒रिति॒ चित्ति॑ - भिः॒ । \newline
9. इत्या॑हा॒हे तीत्या॑ह । \newline
10. आ॒ह॒ यस्मै॒ यस्मा॑ आहाह॒ यस्मै᳚ । \newline
11. यस्मा॑ ए॒वैव यस्मै॒ यस्मा॑ ए॒व । \newline
12. ए॒वैन॑ मेन मे॒वै वैन᳚म् । \newline
13. ए॒न॒म् चि॒त्ताय॑ चि॒त्तायै॑न मेनम् चि॒त्ताय॑ । \newline
14. चि॒त्तायो॒ द्यच्छ॑त उ॒द्यच्छ॑ते चि॒त्ताय॑ चि॒त्तायो॒ द्यच्छ॑ते । \newline
15. उ॒द्यच्छ॑ते॒ तेन॒ तेनो॒ द्यच्छ॑त उ॒द्यच्छ॑ते॒ तेन॑ । \newline
16. उ॒द्यच्छ॑त॒ इत्यु॑त् - यच्छ॑ते । \newline
17. तेनै॒ वैव तेन॒ तेनै॒व । \newline
18. ए॒वैन॑ मेन मे॒वै वैन᳚म् । \newline
19. ए॒नꣳ॒॒ सꣳ स मे॑न मेनꣳ॒॒ सम् । \newline
20. स म॑र्द्धय त्यर्द्धयति॒ सꣳ स म॑र्द्धयति । \newline
21. अ॒र्द्ध॒य॒ति॒ पञ्च॒ पञ्चा᳚ र्द्धय त्यर्द्धयति॒ पञ्च॑ । \newline
22. पञ्च॒ दिशो॒ दिशः॒ पञ्च॒ पञ्च॒ दिशः॑ । \newline
23. दिशो॒ दैवी॒र् दैवी॒र् दिशो॒ दिशो॒ दैवीः᳚ । \newline
24. दैवी᳚र् य॒ज्ञ्ं ॅय॒ज्ञ्म् दैवी॒र् दैवी᳚र् य॒ज्ञ्म् । \newline
25. य॒ज्ञ् म॑वन् त्ववन्तु य॒ज्ञ्ं ॅय॒ज्ञ् म॑वन्तु । \newline
26. अ॒व॒न्तु॒ दे॒वीर् दे॒वी र॑वन् त्ववन्तु दे॒वीः । \newline
27. दे॒वी रितीति॑ दे॒वीर् दे॒वी रिति॑ । \newline
28. इत्या॑हा॒हे तीत्या॑ह । \newline
29. आ॒ह॒ दिशो॒ दिश॑ आहाह॒ दिशः॑ । \newline
30. दिशो॒ हि हि दिशो॒ दिशो॒ हि । \newline
31. ह्ये॑ष ए॒ष हि ह्ये॑षः । \newline
32. ए॒षो ऽन्वन् वे॒ष ए॒षो ऽनु॑ । \newline
33. अनु॑ प्र॒च्यव॑ते प्र॒च्यव॑ते॒ ऽन्वनु॑ प्र॒च्यव॑ते । \newline
34. प्र॒च्यव॒ते ऽपाप॑ प्र॒च्यव॑ते प्र॒च्यव॒ते ऽप॑ । \newline
35. प्र॒च्यव॑त॒ इति॑ प्र - च्यव॑ते । \newline
36. अपाम॑ति॒ मम॑ति॒ मपापा म॑तिम् । \newline
37. अम॑तिम् दुर्म॒तिम् दु॑र्म॒ति मम॑ति॒ मम॑तिम् दुर्म॒तिम् । \newline
38. दु॒र्म॒तिम् बाध॑माना॒ बाध॑माना दुर्म॒तिम् दु॑र्म॒तिम् बाध॑मानाः । \newline
39. दु॒र्म॒तिमिति॑ दुः - म॒तिम् । \newline
40. बाध॑माना॒ इतीति॒ बाध॑माना॒ बाध॑माना॒ इति॑ । \newline
41. इत्या॑हा॒हे तीत्या॑ह । \newline
42. आ॒ह॒ रक्ष॑साꣳ॒॒ रक्ष॑सा माहाह॒ रक्ष॑साम् । \newline
43. रक्ष॑सा॒ मप॑हत्या॒ अप॑हत्यै॒ रक्ष॑साꣳ॒॒ रक्ष॑सा॒ मप॑हत्यै । \newline
44. अप॑हत्यै रा॒यो रा॒यो ऽप॑हत्या॒ अप॑हत्यै रा॒यः । \newline
45. अप॑हत्या॒ इत्यप॑ - ह॒त्यै॒ । \newline
46. रा॒य स्पोषे॒ पोषे॑ रा॒यो रा॒य स्पोषे᳚ । \newline
47. पोषे॑ य॒ज्ञ्प॑तिं ॅय॒ज्ञ्प॑ति॒म् पोषे॒ पोषे॑ य॒ज्ञ्प॑तिम् । \newline
48. य॒ज्ञ्प॑ति मा॒भज॑न्ती रा॒भज॑न्तीर् य॒ज्ञ्प॑तिं ॅय॒ज्ञ्प॑ति मा॒भज॑न्तीः । \newline
49. य॒ज्ञ्प॑ति॒मिति॑ य॒ज्ञ् - प॒ति॒म् । \newline
50. आ॒भज॑न्ती॒ रिती त्या॒भज॑न्ती रा॒भज॑न्ती॒ रिति॑ । \newline
51. आ॒भज॑न्ती॒रित्या᳚ - भज॑न्तीः । \newline
52. इत्या॑हा॒हे तीत्या॑ह । \newline
53. आ॒ह॒ प॒शवः॑ प॒शव॑ आहाह प॒शवः॑ । \newline
54. प॒शवो॒ वै वै प॒शवः॑ प॒शवो॒ वै । \newline
55. वै रा॒यो रा॒यो वै वै रा॒यः । \newline
56. रा॒य स्पोषः॒ पोषो॑ रा॒यो रा॒य स्पोषः॑ । \newline
57. पोषः॑ प॒शून् प॒शून् पोषः॒ पोषः॑ प॒शून् । \newline

\textbf{Ghana Paata } \newline

1. ए॒वैन॑ मेन मे॒वै वैन॒ मुदु दे॑न मे॒वै वैन॒ मुत् । \newline
2. ए॒न॒ मुदु दे॑न मेन॒ मुद् य॑च्छते यच्छत॒ उदे॑न मेन॒ मुद् य॑च्छते । \newline
3. उद् य॑च्छते यच्छत॒ उदुद् य॑च्छ॒ते ऽग्ने ऽग्ने॑ यच्छत॒ उदुद् य॑च्छ॒ते ऽग्ने᳚ । \newline
4. य॒च्छ॒ते ऽग्ने ऽग्ने॑ यच्छते यच्छ॒ते ऽग्ने॒ भर॑न्तु॒ भर॒न् त्वग्ने॑ यच्छते यच्छ॒ते ऽग्ने॒ भर॑न्तु । \newline
5. अग्ने॒ भर॑न्तु॒ भर॒न् त्वग्ने ऽग्ने॒ भर॑न्तु॒ चित्ति॑भि॒ श्चित्ति॑भि॒र् भर॒न् त्वग्ने ऽग्ने॒ भर॑न्तु॒ चित्ति॑भिः । \newline
6. भर॑न्तु॒ चित्ति॑भि॒ श्चित्ति॑भि॒र् भर॑न्तु॒ भर॑न्तु॒ चित्ति॑भि॒ रितीति॒ चित्ति॑भि॒र् भर॑न्तु॒ भर॑न्तु॒ चित्ति॑भि॒ रिति॑ । \newline
7. चित्ति॑भि॒ रितीति॒ चित्ति॑भि॒ श्चित्ति॑भि॒रि त्या॑हा॒हेति॒ चित्ति॑भि॒ श्चित्ति॑भि॒ रित्या॑ह । \newline
8. चित्ति॑भि॒रिति॒ चित्ति॑ - भिः॒ । \newline
9. इत्या॑हा॒हे तीत्या॑ह॒ यस्मै॒ यस्मा॑ आ॒हे तीत्या॑ह॒ यस्मै᳚ । \newline
10. आ॒ह॒ यस्मै॒ यस्मा॑ आहाह॒ यस्मा॑ ए॒वैव यस्मा॑ आहाह॒ यस्मा॑ ए॒व । \newline
11. यस्मा॑ ए॒वैव यस्मै॒ यस्मा॑ ए॒वैन॑ मेन मे॒व यस्मै॒ यस्मा॑ ए॒वैन᳚म् । \newline
12. ए॒वैन॑ मेन मे॒वै वैन॑म् चि॒त्ताय॑ चि॒त्तायै॑न मे॒वै वैन॑म् चि॒त्ताय॑ । \newline
13. ए॒न॒म् चि॒त्ताय॑ चि॒त्तायै॑न मेनम् चि॒त्तायो॒ द्यच्छ॑त उ॒द्यच्छ॑ते चि॒त्तायै॑न मेनम् चि॒त्तायो॒ द्यच्छ॑ते । \newline
14. चि॒त्तायो॒ द्यच्छ॑त उ॒द्यच्छ॑ते चि॒त्ताय॑ चि॒त्तायो॒ द्यच्छ॑ते॒ तेन॒ तेनो॒द्यच्छ॑ते चि॒त्ताय॑ चि॒त्तायो॒ द्यच्छ॑ते॒ तेन॑ । \newline
15. उ॒द्यच्छ॑ते॒ तेन॒ तेनो॒ द्यच्छ॑त उ॒द्यच्छ॑ते॒ तेनै॒ वैव तेनो॒ द्यच्छ॑त उ॒द्यच्छ॑ते॒ तेनै॒व । \newline
16. उ॒द्यच्छ॑त॒ इत्यु॑त् - यच्छ॑ते । \newline
17. तेनै॒ वैव तेन॒ तेनै॒ वैन॑ मेन मे॒व तेन॒ तेनै॒ वैन᳚म् । \newline
18. ए॒वैन॑ मेन मे॒वै वैनꣳ॒॒ सꣳ स मे॑न मे॒वै वैनꣳ॒॒ सम् । \newline
19. ए॒नꣳ॒॒ सꣳ स मे॑न मेनꣳ॒॒ स म॑र्द्धय त्यर्द्धयति॒ स मे॑न मेनꣳ॒॒ स म॑र्द्धयति । \newline
20. स म॑र्द्धय त्यर्द्धयति॒ सꣳ स म॑र्द्धयति॒ पञ्च॒ पञ्चा᳚र्द्धयति॒ सꣳ स म॑र्द्धयति॒ पञ्च॑ । \newline
21. अ॒र्द्ध॒य॒ति॒ पञ्च॒ पञ्चा᳚र्द्धय त्यर्द्धयति॒ पञ्च॒ दिशो॒ दिशः॒ पञ्चा᳚र्द्धय त्यर्द्धयति॒ पञ्च॒ दिशः॑ । \newline
22. पञ्च॒ दिशो॒ दिशः॒ पञ्च॒ पञ्च॒ दिशो॒ दैवी॒र् दैवी॒र् दिशः॒ पञ्च॒ पञ्च॒ दिशो॒ दैवीः᳚ । \newline
23. दिशो॒ दैवी॒र् दैवी॒र् दिशो॒ दिशो॒ दैवी᳚र् य॒ज्ञ्ं ॅय॒ज्ञ्म् दैवी॒र् दिशो॒ दिशो॒ दैवी᳚र् य॒ज्ञ्म् । \newline
24. दैवी᳚र् य॒ज्ञ्ं ॅय॒ज्ञ्म् दैवी॒र् दैवी᳚र् य॒ज्ञ् म॑वन् त्ववन्तु य॒ज्ञ्म् दैवी॒र् दैवी᳚र् य॒ज्ञ् म॑वन्तु । \newline
25. य॒ज्ञ् म॑वन् त्ववन्तु य॒ज्ञ्ं ॅय॒ज्ञ् म॑वन्तु दे॒वीर् दे॒वी र॑वन्तु य॒ज्ञ्ं ॅय॒ज्ञ् म॑वन्तु दे॒वीः । \newline
26. अ॒व॒न्तु॒ दे॒वीर् दे॒वी र॑वन् त्ववन्तु दे॒वी रितीति॑ दे॒वी र॑वन् त्ववन्तु दे॒वी रिति॑ । \newline
27. दे॒वी रितीति॑ दे॒वीर् दे॒वी रित्या॑ हा॒हेति॑ दे॒वीर् दे॒वी रित्या॑ह । \newline
28. इत्या॑हा॒हे तीत्या॑ह॒ दिशो॒ दिश॑ आ॒हे तीत्या॑ह॒ दिशः॑ । \newline
29. आ॒ह॒ दिशो॒ दिश॑ आहाह॒ दिशो॒ हि हि दिश॑ आहाह॒ दिशो॒ हि । \newline
30. दिशो॒ हि हि दिशो॒ दिशो॒ ह्ये॑ष ए॒ष हि दिशो॒ दिशो॒ ह्ये॑षः । \newline
31. ह्ये॑ष ए॒ष हि ह्ये॑षो ऽन्वन् वे॒ष हि ह्ये॑षो ऽनु॑ । \newline
32. ए॒षो ऽन्वन्वे॒ष ए॒षो ऽनु॑ प्र॒च्यव॑ते प्र॒च्यव॑ते ऽन्वे॒ष ए॒षो ऽनु॑ प्र॒च्यव॑ते । \newline
33. अनु॑ प्र॒च्यव॑ते प्र॒च्यव॑ते॒ ऽन्वनु॑ प्र॒च्यव॒ते ऽपाप॑ प्र॒च्यव॑ते॒ ऽन्वनु॑ प्र॒च्यव॒ते ऽप॑ । \newline
34. प्र॒च्यव॒ते ऽपाप॑ प्र॒च्यव॑ते प्र॒च्यव॒ते ऽपाम॑ति॒ मम॑ति॒ मप॑ प्र॒च्यव॑ते प्र॒च्यव॒ते ऽपाम॑तिम् । \newline
35. प्र॒च्यव॑त॒ इति॑ प्र - च्यव॑ते । \newline
36. अपाम॑ति॒ मम॑ति॒ मपापा म॑तिम् दुर्म॒तिम् दु॑र्म॒ति मम॑ति॒ मपापा म॑तिम् दुर्म॒तिम् । \newline
37. अम॑तिम् दुर्म॒तिम् दु॑र्म॒ति मम॑ति॒ मम॑तिम् दुर्म॒तिम् बाध॑माना॒ बाध॑माना दुर्म॒ति मम॑ति॒ मम॑तिम् दुर्म॒तिम् बाध॑मानाः । \newline
38. दु॒र्म॒तिम् बाध॑माना॒ बाध॑माना दुर्म॒तिम् दु॑र्म॒तिम् बाध॑माना॒ इतीति॒ बाध॑माना दुर्म॒तिम् दु॑र्म॒तिम् बाध॑माना॒ इति॑ । \newline
39. दु॒र्म॒तिमिति॑ दुः - म॒तिम् । \newline
40. बाध॑माना॒ इतीति॒ बाध॑माना॒ बाध॑माना॒ इत्या॑हा॒हेति॒ बाध॑माना॒ बाध॑माना॒ इत्या॑ह । \newline
41. इत्या॑हा॒हे तीत्या॑ह॒ रक्ष॑साꣳ॒॒ रक्ष॑सा मा॒हे तीत्या॑ह॒ रक्ष॑साम् । \newline
42. आ॒ह॒ रक्ष॑साꣳ॒॒ रक्ष॑सा माहाह॒ रक्ष॑सा॒ मप॑हत्या॒ अप॑हत्यै॒ रक्ष॑सा माहाह॒ रक्ष॑सा॒ मप॑हत्यै । \newline
43. रक्ष॑सा॒ मप॑हत्या॒ अप॑हत्यै॒ रक्ष॑साꣳ॒॒ रक्ष॑सा॒ मप॑हत्यै रा॒यो रा॒यो ऽप॑हत्यै॒ रक्ष॑साꣳ॒॒ रक्ष॑सा॒ मप॑हत्यै रा॒यः । \newline
44. अप॑हत्यै रा॒यो रा॒यो ऽप॑हत्या॒ अप॑हत्यै रा॒य स्पोषे॒ पोषे॑ रा॒यो ऽप॑हत्या॒ अप॑हत्यै रा॒य स्पोषे᳚ । \newline
45. अप॑हत्या॒ इत्यप॑ - ह॒त्यै॒ । \newline
46. रा॒य स्पोषे॒ पोषे॑ रा॒यो रा॒य स्पोषे॑ य॒ज्ञ्प॑तिं ॅय॒ज्ञ्प॑ति॒म् पोषे॑ रा॒यो रा॒य स्पोषे॑ य॒ज्ञ्प॑तिम् । \newline
47. पोषे॑ य॒ज्ञ्प॑तिं ॅय॒ज्ञ्प॑ति॒म् पोषे॒ पोषे॑ य॒ज्ञ्प॑ति मा॒भज॑न्ती रा॒भज॑न्तीर् य॒ज्ञ्प॑ति॒म् पोषे॒ पोषे॑ य॒ज्ञ्प॑ति मा॒भज॑न्तीः । \newline
48. य॒ज्ञ्प॑ति मा॒भज॑न्ती रा॒भज॑न्तीर् य॒ज्ञ्प॑तिं ॅय॒ज्ञ्प॑ति मा॒भज॑न्ती॒ रिती त्या॒भज॑न्तीर् य॒ज्ञ्प॑तिं ॅय॒ज्ञ्प॑ति मा॒भज॑न्ती॒ रिति॑ । \newline
49. य॒ज्ञ्प॑ति॒मिति॑ य॒ज्ञ् - प॒ति॒म् । \newline
50. आ॒भज॑न्ती॒ रिती त्या॒भज॑न्ती रा॒भज॑न्ती॒ रित्या॑ हा॒हे त्या॒भज॑न्ती रा॒भज॑न्ती॒ रित्या॑ह । \newline
51. आ॒भज॑न्ती॒रित्या᳚ - भज॑न्तीः । \newline
52. इत्या॑हा॒हे तीत्या॑ह प॒शवः॑ प॒शव॑ आ॒हे तीत्या॑ह प॒शवः॑ । \newline
53. आ॒ह॒ प॒शवः॑ प॒शव॑ आहाह प॒शवो॒ वै वै प॒शव॑ आहाह प॒शवो॒ वै । \newline
54. प॒शवो॒ वै वै प॒शवः॑ प॒शवो॒ वै रा॒यो रा॒यो वै प॒शवः॑ प॒शवो॒ वै रा॒यः । \newline
55. वै रा॒यो रा॒यो वै वै रा॒य स्पोषः॒ पोषो॑ रा॒यो वै वै रा॒य स्पोषः॑ । \newline
56. रा॒य स्पोषः॒ पोषो॑ रा॒यो रा॒य स्पोषः॑ प॒शून् प॒शून् पोषो॑ रा॒यो रा॒य स्पोषः॑ प॒शून् । \newline
57. पोषः॑ प॒शून् प॒शून् पोषः॒ पोषः॑ प॒शू ने॒वैव प॒शून् पोषः॒ पोषः॑ प॒शू ने॒व । \newline
\pagebreak
\markright{ TS 5.4.6.3  \hfill https://www.vedavms.in \hfill}

\section{ TS 5.4.6.3 }

\textbf{TS 5.4.6.3 } \newline
\textbf{Samhita Paata} \newline

प॒शूने॒वाव॑ रुन्धे ष॒ड्भिर्.ह॑रति॒ षड् वा ऋ॒तव॑ ऋ॒तुभि॑रे॒वैनꣳ॑ हरति॒ द्वे प॑रि॒गृह्य॑वती भवतो॒ रक्ष॑सा॒मप॑हत्यै॒ सूर्य॑रश्मि॒र्॒.हरि॑केशः पु॒रस्ता॒दित्या॑ह॒ प्रसू᳚त्यै॒ ततः॑ पाव॒का आ॒शिषो॑ नो जुषन्ता॒मित्या॒हान्नं॒ ॅवै पा॑व॒कोऽन्न॑मे॒वाव॑ रुन्धे देवासु॒राः संॅय॑त्ता आस॒न् ते दे॒वा ए॒त-दप्र॑तिरथ-मपश्य॒न् तेन॒ वै ते᳚ प्र॒त्य - [  ] \newline

\textbf{Pada Paata} \newline

प॒शून् । ए॒व । अवेति॑ । रु॒न्धे॒ । ष॒ड्भिरिति॑ षट्-भिः । ह॒र॒ति॒ । षट् । वै । ऋ॒तवः॑ । ऋ॒तुभि॒रित्यृ॒तु - भिः॒ । ए॒व । ए॒न॒म् । ह॒र॒ति॒ । द्वे इति॑ । प॒रि॒गृह्य॑वती॒ इति॑ परि॒गृह्य॑ - व॒ती॒ । भ॒व॒तः॒ । रक्ष॑साम् । अप॑हत्या॒ इत्यप॑ - ह॒त्यै॒ । सूर्य॑रश्मि॒रिति॒ सूर्य॑ - र॒श्मिः॒ । हरि॑केश॒ इति॒ हरि॑-के॒शः॒ । पु॒रस्ता᳚त् । इति॑ । आ॒ह॒ । प्रसू᳚त्या॒ इति॒ प्र-सू॒त्यै॒ । ततः॑ । पा॒व॒काः । आ॒शिष॒ इत्या᳚ - शिषः॑ । नः॒ । जु॒ष॒न्ता॒म् । इति॑ । आ॒ह॒ । अन्न᳚म् । वै । पा॒व॒कः । अन्न᳚म् । ए॒व । अवेति॑ । रु॒न्धे॒ । दे॒वा॒सु॒रा इति॑ देव-अ॒सु॒राः । संॅय॑त्ता॒ इति॒ सं - य॒त्ताः॒ । आ॒स॒न्न् । ते । दे॒वाः । ए॒तत् । अप्र॑तिरथ॒मित्यप्र॑ति - र॒थ॒म् । अ॒प॒श्य॒न्न् । तेन॑ । वै । ते । अ॒प्र॒ति ।  \newline


\textbf{Krama Paata} \newline

प॒शूने॒व । ए॒वाव॑ । अव॑ रुन्धे । रु॒न्धे॒ ष॒ड्भिः । ष॒ड्भिर्. ह॑रति । ष॒ड्भिरिति॑ षट् - भिः । ह॒र॒ति॒ षट् । षड् वै । वा ऋ॒तवः॑ । ऋ॒तव॑ ऋ॒तुभिः॑ । ऋ॒तुभि॑रे॒व । ऋ॒तुभि॒रित्यृ॒तु - भिः॒ । ए॒वैन᳚म् । ए॒नꣳ॒॒ ह॒र॒ति॒ । ह॒र॒ति॒ द्वे । द्वे प॑रि॒गृह्य॑वती । द्वे इति॒ द्वे । प॒रि॒गृह्य॑वती भवतः । प॒रि॒गृह्य॑वती॒ इति॑ परि॒गृह्य॑ - व॒ती॒ । भ॒व॒तो॒ रक्ष॑साम् । रक्ष॑सा॒मप॑हत्यै । अप॑हत्यै॒ सूर्य॑रश्मिः । अप॑हत्या॒ इत्यप॑ - ह॒त्यै॒ । सूर्य॑रश्मि॒र्॒. हरि॑केशः । सूर्य॑रश्मि॒रिति॒ सूर्य॑ - र॒श्मिः॒ । हरि॑केशः पु॒रस्ता᳚त् । हरि॑केश॒ इति॒ हरि॑ - के॒शः॒ । पु॒रस्ता॒दिति॑ । इत्या॑ह । आ॒ह॒ प्रसू᳚त्यै । प्रसू᳚त्यै॒ ततः॑ । प्रसू᳚त्या॒ इति॒ प्र - सू॒त्यै॒ । ततः॑ पाव॒काः । पा॒व॒का आ॒शिषः॑ । आ॒शिषो॑ नः । आ॒शिष॒ इत्या᳚ - शिषः॑ । नो॒ जु॒ष॒न्ता॒म् । जु॒ष॒न्ता॒मिति॑ । इत्या॑ह । आ॒हान्न᳚म् । अन्न॒म् ॅवै । वै पा॑व॒कः । पा॒व॒कोऽन्न᳚म् । अन्न॑मे॒व । ए॒वाव॑ । अव॑ रुन्धे । रु॒न्धे॒ दे॒वा॒सु॒राः । दे॒वा॒सु॒राः सम्ॅय॑त्ताः । दे॒व॒सु॒रा इति॑ देव - अ॒सु॒राः । सम्ॅय॑त्ता आसन्न् । सम्ॅय॑त्ता॒ इति॒ सम् - य॒त्ताः॒ । आ॒स॒न् ते । ते दे॒वाः । दे॒वा ए॒तत् । ए॒तदप्र॑तिरथम् । अप्र॑तिरथमपश्यन्न् । अप्र॑तिरथ॒मित्यप्र॑ति - र॒थ॒म् । अ॒प॒श्य॒न् तेन॑ । तेन॒ वै । वै ते । ते᳚ऽप्र॒ति । अ॒प्र॒त्यसु॑रान् \newline

\textbf{Jatai Paata} \newline

1. प॒शू ने॒वैव प॒शून् प॒शू ने॒व । \newline
2. ए॒वावा वै॒वै वाव॑ । \newline
3. अव॑ रुन्धे रु॒न्धे ऽवाव॑ रुन्धे । \newline
4. रु॒न्धे॒ ष॒ड्भि ष्ष॒ड्भी रु॑न्धे रुन्धे ष॒ड्भिः । \newline
5. ष॒ड्भिर्. ह॑रति हरति ष॒ड्भि ष्ष॒ड्भिर्. ह॑रति । \newline
6. ष॒ड्भिरिति॑ षट् - भिः । \newline
7. ह॒र॒ति॒ षट् थ्षड्ढ॑रति हरति॒ षट् । \newline
8. षड् वै वै षट् थ्षड् वै । \newline
9. वा ऋ॒तव॑ ऋ॒तवो॒ वै वा ऋ॒तवः॑ । \newline
10. ऋ॒तव॑ ऋ॒तुभिर्॑. ऋ॒तुभिर्॑. ऋ॒तव॑ ऋ॒तव॑ ऋ॒तुभिः॑ । \newline
11. ऋ॒तुभि॑ रे॒वैव र्तुभिर्॑. ऋ॒तुभि॑ रे॒व । \newline
12. ऋ॒तुभि॒रित्यृ॒तु - भिः॒ । \newline
13. ए॒वैन॑ मेन मे॒वै वैन᳚म् । \newline
14. ए॒नꣳ॒॒ ह॒र॒ति॒ ह॒र॒ त्ये॒न॒ मे॒नꣳ॒॒ ह॒र॒ति॒ । \newline
15. ह॒र॒ति॒ द्वे द्वे ह॑रति हरति॒ द्वे । \newline
16. द्वे प॑रि॒गृह्य॑वती परि॒गृह्य॑वती॒ द्वे द्वे प॑रि॒गृह्य॑वती । \newline
17. द्वे इति॒ द्वे । \newline
18. प॒रि॒गृह्य॑वती भवतो भवतः परि॒गृह्य॑वती परि॒गृह्य॑वती भवतः । \newline
19. प॒रि॒गृह्य॑वती॒ इति॑ परि॒गृह्य॑ - व॒ती॒ । \newline
20. भ॒व॒तो॒ रक्ष॑साꣳ॒॒ रक्ष॑साम् भवतो भवतो॒ रक्ष॑साम् । \newline
21. रक्ष॑सा॒ मप॑हत्या॒ अप॑हत्यै॒ रक्ष॑साꣳ॒॒ रक्ष॑सा॒ मप॑हत्यै । \newline
22. अप॑हत्यै॒ सूर्य॑रश्मिः॒ सूर्य॑रश्मि॒ रप॑हत्या॒ अप॑हत्यै॒ सूर्य॑रश्मिः । \newline
23. अप॑हत्या॒ इत्यप॑ - ह॒त्यै॒ । \newline
24. सूर्य॑रश्मि॒र्॒. हरि॑केशो॒ हरि॑केशः॒ सूर्य॑रश्मिः॒ सूर्य॑रश्मि॒र्॒. हरि॑केशः । \newline
25. सूर्य॑रश्मि॒रिति॒ सूर्य॑ - र॒श्मिः॒ । \newline
26. हरि॑केशः पु॒रस्ता᳚त् पु॒रस्ता॒ द्धरि॑केशो॒ हरि॑केशः पु॒रस्ता᳚त् । \newline
27. हरि॑केश॒ इति॒ हरि॑ - के॒शः॒ । \newline
28. पु॒रस्ता॒ दितीति॑ पु॒रस्ता᳚त् पु॒रस्ता॒ दिति॑ । \newline
29. इत्या॑हा॒हे तीत्या॑ह । \newline
30. आ॒ह॒ प्रसू᳚त्यै॒ प्रसू᳚त्या आहाह॒ प्रसू᳚त्यै । \newline
31. प्रसू᳚त्यै॒ तत॒ स्ततः॒ प्रसू᳚त्यै॒ प्रसू᳚त्यै॒ ततः॑ । \newline
32. प्रसू᳚त्या॒ इति॒ प्र - सू॒त्यै॒ । \newline
33. ततः॑ पाव॒काः पा॑व॒का स्तत॒ स्ततः॑ पाव॒काः । \newline
34. पा॒व॒का आ॒शिष॑ आ॒शिषः॑ पाव॒काः पा॑व॒का आ॒शिषः॑ । \newline
35. आ॒शिषो॑ नो न आ॒शिष॑ आ॒शिषो॑ नः । \newline
36. आ॒शिष॒ इत्या᳚ - शिषः॑ । \newline
37. नो॒ जु॒ष॒न्ता॒म् जु॒ष॒न्ता॒म् नो॒ नो॒ जु॒ष॒न्ता॒म् । \newline
38. जु॒ष॒न्ता॒ मितीति॑ जुषन्ताम् जुषन्ता॒ मिति॑ । \newline
39. इत्या॑हा॒हे तीत्या॑ह । \newline
40. आ॒हान्न॒ मन्न॑ माहा॒ हान्न᳚म् । \newline
41. अन्नं॒ ॅवै वा अन्न॒ मन्नं॒ ॅवै । \newline
42. वै पा॑व॒कः पा॑व॒को वै वै पा॑व॒कः । \newline
43. पा॒व॒को ऽन्न॒ मन्न॑म् पाव॒कः पा॑व॒को ऽन्न᳚म् । \newline
44. अन्न॑ मे॒वै वान्न॒ मन्न॑ मे॒व । \newline
45. ए॒वावा वै॒वै वाव॑ । \newline
46. अव॑ रुन्धे रु॒न्धे ऽवाव॑ रुन्धे । \newline
47. रु॒न्धे॒ दे॒वा॒सु॒रा दे॑वासु॒रा रु॑न्धे रुन्धे देवासु॒राः । \newline
48. दे॒वा॒सु॒राः संॅय॑त्ताः॒ संॅय॑त्ता देवासु॒रा दे॑वासु॒राः संॅय॑त्ताः । \newline
49. दे॒वा॒सु॒रा इति॑ देव - अ॒सु॒राः । \newline
50. संॅय॑त्ता आसन् नास॒न् थ्संॅय॑त्ताः॒ संॅय॑त्ता आसन्न् । \newline
51. संॅय॑त्ता॒ इति॒ सं - य॒त्ताः॒ । \newline
52. आ॒स॒न् ते त आ॑सन् नास॒न् ते । \newline
53. ते दे॒वा दे॒वा स्ते ते दे॒वाः । \newline
54. दे॒वा ए॒त दे॒तद् दे॒वा दे॒वा ए॒तत् । \newline
55. ए॒त दप्र॑तिरथ॒ मप्र॑तिरथ मे॒त दे॒त दप्र॑तिरथम् । \newline
56. अप्र॑तिरथ मपश्यन् नपश्य॒न् नप्र॑तिरथ॒ मप्र॑तिरथ मपश्यन्न् । \newline
57. अप्र॑तिरथ॒मित्यप्र॑ति - र॒थ॒म् । \newline
58. अ॒प॒श्य॒न् तेन॒ तेना॑पश्यन् नपश्य॒न् तेन॑ । \newline
59. तेन॒ वै वै तेन॒ तेन॒ वै । \newline
60. वै ते ते वै वै ते । \newline
61. ते᳚ ऽप्र॒ त्य॑प्र॒ति ते ते᳚ ऽप्र॒ति । \newline
62. अ॒प्र॒ त्यसु॑रा॒ नसु॑रा नप्र॒ त्य॑प्र॒ त्यसु॑रान् । \newline

\textbf{Ghana Paata } \newline

1. प॒शू ने॒वैव प॒शून् प॒शू ने॒वावा वै॒व प॒शून् प॒शू ने॒वाव॑ । \newline
2. ए॒वावा वै॒वै वाव॑ रुन्धे रु॒न्धे ऽवै॒वै वाव॑ रुन्धे । \newline
3. अव॑ रुन्धे रु॒न्धे ऽवाव॑ रुन्धे ष॒ड्भि ष्ष॒ड्भी रु॒न्धे ऽवाव॑ रुन्धे ष॒ड्भिः । \newline
4. रु॒न्धे॒ ष॒ड्भि ष्ष॒ड्भी रु॑न्धे रुन्धे ष॒ड्भिर्. ह॑रति हरति ष॒ड्भी रु॑न्धे रुन्धे ष॒ड्भिर्. ह॑रति । \newline
5. ष॒ड्भिर्. ह॑रति हरति ष॒ड्भि ष्ष॒ड्भिर्. ह॑रति॒ षट् थ्षड्ढ॑रति ष॒ड्भि ष्ष॒ड्भिर्. ह॑रति॒ षट् । \newline
6. ष॒ड्भिरिति॑ षट् - भिः । \newline
7. ह॒र॒ति॒ षट् थ्षड्ढ॑रति हरति॒ षड् वै वै षड्ढ॑रति हरति॒ षड् वै । \newline
8. षड् वै वै षट् थ्षड् वा ऋ॒तव॑ ऋ॒तवो॒ वै षट् थ्षड् वा ऋ॒तवः॑ । \newline
9. वा ऋ॒तव॑ ऋ॒तवो॒ वै वा ऋ॒तव॑ ऋ॒तुभिर्॑. ऋ॒तुभिर्॑. ऋ॒तवो॒ वै वा ऋ॒तव॑ ऋ॒तुभिः॑ । \newline
10. ऋ॒तव॑ ऋ॒तुभिर्॑. ऋ॒तुभिर्॑. ऋ॒तव॑ ऋ॒तव॑ ऋ॒तुभि॑ रे॒वैव र्‌तुभिर्॑. ऋ॒तव॑ ऋ॒तव॑ ऋ॒तुभि॑ रे॒व । \newline
11. ऋ॒तुभि॑ रे॒वैव र्‌तुभिर्॑. ऋ॒तुभि॑ रे॒वैन॑ मेन मे॒व र्तुभिर्॑. ऋ॒तुभि॑ रे॒वैन᳚म् । \newline
12. ऋ॒तुभि॒रित्यृ॒तु - भिः॒ । \newline
13. ए॒वैन॑ मेन मे॒वै वैनꣳ॑ हरति हर त्येन मे॒वै वैनꣳ॑ हरति । \newline
14. ए॒नꣳ॒॒ ह॒र॒ति॒ ह॒र॒ त्ये॒न॒ मे॒नꣳ॒॒ ह॒र॒ति॒ द्वे द्वे ह॑र त्येन मेनꣳ हरति॒ द्वे । \newline
15. ह॒र॒ति॒ द्वे द्वे ह॑रति हरति॒ द्वे प॑रि॒गृह्य॑वती परि॒गृह्य॑वती॒ द्वे ह॑रति हरति॒ द्वे प॑रि॒गृह्य॑वती । \newline
16. द्वे प॑रि॒गृह्य॑वती परि॒गृह्य॑वती॒ द्वे द्वे प॑रि॒गृह्य॑वती भवतो भवतः परि॒गृह्य॑वती॒ द्वे द्वे प॑रि॒गृह्य॑वती भवतः । \newline
17. द्वे इति॒ द्वे । \newline
18. प॒रि॒गृह्य॑वती भवतो भवतः परि॒गृह्य॑वती परि॒गृह्य॑वती भवतो॒ रक्ष॑साꣳ॒॒ रक्ष॑साम् भवतः परि॒गृह्य॑वती परि॒गृह्य॑वती भवतो॒ रक्ष॑साम् । \newline
19. प॒रि॒गृह्य॑वती॒ इति॑ परि॒गृह्य॑ - व॒ती॒ । \newline
20. भ॒व॒तो॒ रक्ष॑साꣳ॒॒ रक्ष॑साम् भवतो भवतो॒ रक्ष॑सा॒ मप॑हत्या॒ अप॑हत्यै॒ रक्ष॑साम् भवतो भवतो॒ रक्ष॑सा॒ मप॑हत्यै । \newline
21. रक्ष॑सा॒ मप॑हत्या॒ अप॑हत्यै॒ रक्ष॑साꣳ॒॒ रक्ष॑सा॒ मप॑हत्यै॒ सूर्य॑रश्मिः॒ सूर्य॑रश्मि॒ रप॑हत्यै॒ रक्ष॑साꣳ॒॒ रक्ष॑सा॒ मप॑हत्यै॒ सूर्य॑रश्मिः । \newline
22. अप॑हत्यै॒ सूर्य॑रश्मिः॒ सूर्य॑रश्मि॒ रप॑हत्या॒ अप॑हत्यै॒ सूर्य॑रश्मि॒र्॒. हरि॑केशो॒ हरि॑केशः॒ सूर्य॑रश्मि॒ रप॑हत्या॒ अप॑हत्यै॒ सूर्य॑रश्मि॒र्॒. हरि॑केशः । \newline
23. अप॑हत्या॒ इत्यप॑ - ह॒त्यै॒ । \newline
24. सूर्य॑रश्मि॒र्॒. हरि॑केशो॒ हरि॑केशः॒ सूर्य॑रश्मिः॒ सूर्य॑रश्मि॒र्॒. हरि॑केशः पु॒रस्ता᳚त् पु॒रस्ता॒ द्धरि॑केशः॒ सूर्य॑रश्मिः॒ सूर्य॑रश्मि॒र्॒. हरि॑केशः पु॒रस्ता᳚त् । \newline
25. सूर्य॑रश्मि॒रिति॒ सूर्य॑ - र॒श्मिः॒ । \newline
26. हरि॑केशः पु॒रस्ता᳚त् पु॒रस्ता॒ द्धरि॑केशो॒ हरि॑केशः पु॒रस्ता॒ दितीति॑ पु॒रस्ता॒ द्धरि॑केशो॒ हरि॑केशः पु॒रस्ता॒ दिति॑ । \newline
27. हरि॑केश॒ इति॒ हरि॑ - के॒शः॒ । \newline
28. पु॒रस्ता॒ दितीति॑ पु॒रस्ता᳚त् पु॒रस्ता॒ दित्या॑ हा॒हेति॑ पु॒रस्ता᳚त् पु॒रस्ता॒ दित्या॑ह । \newline
29. इत्या॑हा॒हे तीत्या॑ह॒ प्रसू᳚त्यै॒ प्रसू᳚त्या आ॒हे तीत्या॑ह॒ प्रसू᳚त्यै । \newline
30. आ॒ह॒ प्रसू᳚त्यै॒ प्रसू᳚त्या आहाह॒ प्रसू᳚त्यै॒ तत॒ स्ततः॒ प्रसू᳚त्या आहाह॒ प्रसू᳚त्यै॒ ततः॑ । \newline
31. प्रसू᳚त्यै॒ तत॒ स्ततः॒ प्रसू᳚त्यै॒ प्रसू᳚त्यै॒ ततः॑ पाव॒काः पा॑व॒का स्ततः॒ प्रसू᳚त्यै॒ प्रसू᳚त्यै॒ ततः॑ पाव॒काः । \newline
32. प्रसू᳚त्या॒ इति॒ प्र - सू॒त्यै॒ । \newline
33. ततः॑ पाव॒काः पा॑व॒का स्तत॒ स्ततः॑ पाव॒का आ॒शिष॑ आ॒शिषः॑ पाव॒का स्तत॒ स्ततः॑ पाव॒का आ॒शिषः॑ । \newline
34. पा॒व॒का आ॒शिष॑ आ॒शिषः॑ पाव॒काः पा॑व॒का आ॒शिषो॑ नो न आ॒शिषः॑ पाव॒काः पा॑व॒का आ॒शिषो॑ नः । \newline
35. आ॒शिषो॑ नो न आ॒शिष॑ आ॒शिषो॑ नो जुषन्ताम् जुषन्ताम् न आ॒शिष॑ आ॒शिषो॑ नो जुषन्ताम् । \newline
36. आ॒शिष॒ इत्या᳚ - शिषः॑ । \newline
37. नो॒ जु॒ष॒न्ता॒म् जु॒ष॒न्ता॒म् नो॒ नो॒ जु॒ष॒न्ता॒ मितीति॑ जुषन्ताम् नो नो जुषन्ता॒ मिति॑ । \newline
38. जु॒ष॒न्ता॒ मितीति॑ जुषन्ताम् जुषन्ता॒ मित्या॑हा॒हेति॑ जुषन्ताम् जुषन्ता॒ मित्या॑ह । \newline
39. इत्या॑हा॒हेती त्या॒हान्न॒ मन्न॑ मा॒हेती त्या॒हान्न᳚म् । \newline
40. आ॒हान्न॒ मन्न॑ माहा॒ हान्नं॒ ॅवै वा अन्न॑ माहा॒ हान्नं॒ ॅवै । \newline
41. अन्नं॒ ॅवै वा अन्न॒ मन्नं॒ ॅवै पा॑व॒कः पा॑व॒को वा अन्न॒ मन्नं॒ ॅवै पा॑व॒कः । \newline
42. वै पा॑व॒कः पा॑व॒को वै वै पा॑व॒को ऽन्न॒ मन्न॑म् पाव॒को वै वै पा॑व॒को ऽन्न᳚म् । \newline
43. पा॒व॒को ऽन्न॒ मन्न॑म् पाव॒कः पा॑व॒को ऽन्न॑ मे॒वै वान्न॑म् पाव॒कः पा॑व॒को ऽन्न॑ मे॒व । \newline
44. अन्न॑ मे॒वै वान्न॒ मन्न॑ मे॒वावा वै॒वान्न॒ मन्न॑ मे॒वाव॑ । \newline
45. ए॒वावा वै॒वै वाव॑ रुन्धे रु॒न्धे ऽवै॒वै वाव॑ रुन्धे । \newline
46. अव॑ रुन्धे रु॒न्धे ऽवाव॑ रुन्धे देवासु॒रा दे॑वासु॒रा रु॒न्धे ऽवाव॑ रुन्धे देवासु॒राः । \newline
47. रु॒न्धे॒ दे॒वा॒सु॒रा दे॑वासु॒रा रु॑न्धे रुन्धे देवासु॒राः संॅय॑त्ताः॒ संॅय॑त्ता देवासु॒रा रु॑न्धे रुन्धे देवासु॒राः संॅय॑त्ताः । \newline
48. दे॒वा॒सु॒राः संॅय॑त्ताः॒ संॅय॑त्ता देवासु॒रा दे॑वासु॒राः संॅय॑त्ता आसन् नास॒न् थ्संॅय॑त्ता देवासु॒रा दे॑वासु॒राः संॅय॑त्ता आसन्न् । \newline
49. दे॒वा॒सु॒रा इति॑ देव - अ॒सु॒राः । \newline
50. संॅय॑त्ता आसन् नास॒न् थ्संॅय॑त्ताः॒ संॅय॑त्ता आस॒न् ते त आ॑स॒न् थ्संॅय॑त्ताः॒ संॅय॑त्ता आस॒न् ते । \newline
51. संॅय॑त्ता॒ इति॒ सं - य॒त्ताः॒ । \newline
52. आ॒स॒न् ते त आ॑सन् नास॒न् ते दे॒वा दे॒वा स्त आ॑सन् नास॒न् ते दे॒वाः । \newline
53. ते दे॒वा दे॒वा स्ते ते दे॒वा ए॒त दे॒तद् दे॒वा स्ते ते दे॒वा ए॒तत् । \newline
54. दे॒वा ए॒त दे॒तद् दे॒वा दे॒वा ए॒त दप्र॑तिरथ॒ मप्र॑तिरथ मे॒तद् दे॒वा दे॒वा ए॒त दप्र॑तिरथम् । \newline
55. ए॒त दप्र॑तिरथ॒ मप्र॑तिरथ मे॒त दे॒त दप्र॑तिरथ मपश्यन् नपश्य॒न् नप्र॑तिरथ मे॒त दे॒त दप्र॑तिरथ मपश्यन्न् । \newline
56. अप्र॑तिरथ मपश्यन् नपश्य॒न् नप्र॑तिरथ॒ मप्र॑तिरथ मपश्य॒न् तेन॒ तेना॑ पश्य॒न् नप्र॑तिरथ॒ मप्र॑तिरथ मपश्य॒न् तेन॑ । \newline
57. अप्र॑तिरथ॒मित्यप्र॑ति - र॒थ॒म् । \newline
58. अ॒प॒श्य॒न् तेन॒ तेना॑ पश्यन् नपश्य॒न् तेन॒ वै वै तेना॑ पश्यन् नपश्य॒न् तेन॒ वै । \newline
59. तेन॒ वै वै तेन॒ तेन॒ वै ते ते वै तेन॒ तेन॒ वै ते । \newline
60. वै ते ते वै वै ते᳚ ऽप्र॒ त्य॑प्र॒ति ते वै वै ते᳚ ऽप्र॒ति । \newline
61. ते᳚ ऽप्र॒त्य॑ प्र॒ति ते ते᳚ ऽप्र॒ त्यसु॑रा॒ नसु॑रा नप्र॒ति ते ते᳚ ऽप्र॒ त्यसु॑रान् । \newline
62. अ॒प्र॒ त्यसु॑रा॒ नसु॑रा नप्र॒ त्य॑प्र॒ त्यसु॑रा नजयन् नजय॒न् नसु॑रा नप्र॒ त्य॑प्र॒ त्यसु॑रा नजयन्न् । \newline
\pagebreak
\markright{ TS 5.4.6.4  \hfill https://www.vedavms.in \hfill}

\section{ TS 5.4.6.4 }

\textbf{TS 5.4.6.4 } \newline
\textbf{Samhita Paata} \newline

-सु॑रानजय॒न् तदप्र॑तिरथस्या-प्रतिरथ॒त्वं ॅयदप्र॑तिरथं द्वि॒तीयो॒ होता॒ऽन्वाहा᳚प्र॒त्ये॑व तेन॒ यज॑मानो॒ भ्रातृ॑व्यान् जय॒त्यथो॒ अन॑भिजितमे॒वाभि ज॑यति दश॒र्चं भ॑वति॒ दशा᳚क्षरा वि॒राड् वि॒राजे॒मौ लो॒कौ विधृ॑ता व॒नयो᳚र्लो॒कयो॒र्विधृ॑त्या॒ अथो॒ दशा᳚क्षरा वि॒राडन्नं॑ ॅवि॒राड् वि॒राज्ये॒वान्नाद्ये॒ प्रति॑तिष्ठ॒त्यस॑दिव॒ वा अ॒न्तरि॑क्षम॒न्तरि॑क्षमि॒वा ऽऽ*ग्नी᳚द्ध्र॒माग्नी॒द्ध्रे - [  ] \newline

\textbf{Pada Paata} \newline

असु॑रान् । अ॒ज॒य॒न्न् । तत् । अप्र॑तिरथ॒स्येत्यप्र॑ति - र॒थ॒स्य॒ । अ॒प्र॒ति॒र॒थ॒त्वमित्य॑प्रतिरथ - त्वम् । यत् । अप्र॑तिरथ॒मित्यप्र॑ति-र॒थ॒म् । द्वि॒तीयः॑ । होता᳚ । अ॒न्वाहेत्य॑नु - आह॑ । अ॒प्र॒ति । ए॒व । तेन॑ । यज॑मानः । भ्रातृ॑व्यान् । ज॒य॒ति॒ । अथो॒ इति॑ । अन॑भिजित॒मित्यन॑भि - जि॒त॒म् । ए॒व । अ॒भीति॑ । ज॒य॒ति॒ । द॒श॒र्चमिति॑ दश - ऋ॒चम् । भ॒व॒ति॒ । दशा᳚क्ष॒रेति॒ दश॑ - अ॒क्ष॒रा॒ । वि॒राडिति॑ वि - राट् । वि॒राजेति॑ वि - राजा᳚ । इ॒मौ । लो॒कौ । विधृ॑ता॒विति॒ वि - धृ॒तौ॒ । अ॒नयोः᳚ । लो॒कयोः᳚ । विधृ॑त्या॒ इति॒ वि - धृ॒त्यै॒ । अथो॒ इति॑ । दशा᳚क्ष॒रेति॒ दश॑ - अ॒क्ष॒रा॒ । वि॒राडिति॑ वि - राट् । अन्न᳚म् । वि॒राडिति॑ वि - राट् । वि॒राजीति॑ वि - राजि॑ । ए॒व । अ॒न्नाद्य॒ इत्य॑न्न - अद्ये᳚ । प्रतीति॑ । ति॒ष्ठ॒ति॒ । अस॑त् । इ॒व॒ । वै । अ॒न्तरि॑क्षम् । अ॒न्तरि॑क्षम् । इ॒व॒ । आग्नी᳚ध्र॒मित्याग्नि॑ - इ॒ध्र॒म् । आग्नी᳚द्ध्र॒ इत्याग्नि॑ - इ॒द्ध्रे॒ ।  \newline


\textbf{Krama Paata} \newline

असु॑रानजयन्न् । अ॒ज॒य॒न् तत् । तदप्र॑तिरथस्य । अप्र॑तिरथस्याप्रतिरथ॒त्वम् । अप्र॑तिरथ॒स्येत्यप्र॑ति - र॒थ॒स्य॒ । अ॒प्र॒ति॒र॒थ॒त्वम् ॅयत् । अ॒प्र॒ति॒र॒थ॒त्वमित्य॑प्रतिरथ - त्वम् । यदप्र॑तिरथम् । अप्र॑तिरथम् द्वि॒तीयः॑ । अप्र॑तिरथ॒मित्यप्र॑ति - र॒थ॒म् । द्वि॒तीयो॒ होता᳚ । होता॒ऽन्वाह॑ । अ॒न्वाहा᳚प्र॒ति । अ॒न्वाहेत्य॑नु - आह॑ । अ॒प्र॒त्ये॑व । ए॒व तेन॑ । तेन॒ यज॑मानः । यज॑मानो॒ भ्रातृ॑व्यान् । भ्रातृ॑व्यान् जयति । ज॒य॒त्यथो᳚ । अथो॒ अन॑भिजितम् । अथो॒ इत्यथो᳚ । अन॑भिजितमे॒व । अन॑भिजित॒मित्यन॑भि - जि॒त॒म् । ए॒वाभि । अ॒भि ज॑यति । ज॒य॒ति॒ द॒श॒र्चम् । द॒श॒र्चम् भ॑वति । द॒श॒र्चमिति॑ दश - ऋ॒चम् । भ॒व॒ति॒ दशा᳚क्षरा । दशा᳚क्षरा वि॒राट् । दशा᳚क्ष॒रेति॒ दश॑ - अ॒क्ष॒रा॒ । वि॒राड् वि॒राजा᳚ । वि॒राडिति॑ वि - राट् । वि॒राजे॒मौ । वि॒राजेति॑ वि - राजा᳚ । इ॒मौ लो॒कौ । लो॒कौ विधृ॑तौ । विधृ॑ताव॒नयोः᳚ । विधृ॑ता॒विति॒ वि - धृ॒तौ॒ । अ॒नयो᳚र् लो॒कयोः᳚ । लो॒कयो॒र् विधृ॑त्यै । विधृ॑त्या॒ अथो᳚ । विधृ॑त्या॒ इति॒ वि - धृ॒त्यै॒ । अथो॒ दशा᳚क्षरा । अथो॒ इत्यथो᳚ । दशा᳚क्षरा वि॒राट् । दशा᳚क्ष॒रेति॒ दश॑ - अ॒क्ष॒रा॒ । वि॒राडन्न᳚म् । वि॒राडिति॑ वि - राट् । अन्न॑म् ॅवि॒राट् । वि॒राड् वि॒राजि॑ । वि॒राडिति॑ वि - राट् । वि॒राज्ये॒व । वि॒राजीति॑ वि - राजि॑ । ए॒वान्नाद्ये᳚ । अ॒न्नाद्ये॒ प्रति॑ । अ॒न्नाद्य॒ इत्य॑न्न - अद्ये᳚ । प्रति॑ तिष्ठति । ति॒ष्ठ॒त्यस॑त् । अस॑दिव । इ॒व॒ वै । वा अ॒न्तरि॑क्षम् । अ॒न्तरि॑क्षम॒न्तरि॑क्षम् । अ॒न्तरि॑क्षमिव । इ॒वाग्नी᳚द्ध्रम् । आग्नी᳚द्ध्र॒माग्नी᳚द्ध्रे । आग्नी᳚द्ध्र॒मित्याग्नि॑ - इ॒द्ध्र॒म् । आग्नी॒द्ध्रेऽश्मा॑नम् । आग्नी᳚द्ध्र॒ इत्याग्नि॑ - इ॒द्ध्रे॒ \newline

\textbf{Jatai Paata} \newline

1. असु॑रा नजयन् नजय॒न् नसु॑रा॒ नसु॑रा नजयन्न् । \newline
2. अ॒ज॒य॒न् तत् तद॑जयन् नजय॒न् तत् । \newline
3. तदप्र॑तिरथ॒स्या प्र॑तिरथस्य॒ तत् तदप्र॑तिरथस्य । \newline
4. अप्र॑तिरथस्या प्रतिरथ॒त्व म॑प्रतिरथ॒त्व मप्र॑तिरथ॒स्या प्र॑तिरथस्या प्रतिरथ॒त्वम् । \newline
5. अप्र॑तिरथ॒स्येत्यप्र॑ति - र॒थ॒स्य॒ । \newline
6. अ॒प्र॒ति॒र॒थ॒त्वं ॅयद् यद॑प्रतिरथ॒त्व म॑प्रतिरथ॒त्वं ॅयत् । \newline
7. अ॒प्र॒ति॒र॒थ॒त्वमित्य॑प्रतिरथ - त्वम् । \newline
8. यदप्र॑तिरथ॒ मप्र॑तिरथं॒ ॅयद् यदप्र॑तिरथम् । \newline
9. अप्र॑तिरथम् द्वि॒तीयो᳚ द्वि॒तीयो ऽप्र॑तिरथ॒ मप्र॑तिरथम् द्वि॒तीयः॑ । \newline
10. अप्र॑तिरथ॒मित्यप्र॑ति - र॒थ॒म् । \newline
11. द्वि॒तीयो॒ होता॒ होता᳚ द्वि॒तीयो᳚ द्वि॒तीयो॒ होता᳚ । \newline
12. होता॒ ऽन्वाहा॒ न्वाह॒ होता॒ होता॒ ऽन्वाह॑ । \newline
13. अ॒न्वाहा᳚ प्र॒त्य॑ प्र॒त्य॑ न्वाहा॒ न्वाहा᳚ प्र॒ति । \newline
14. अ॒न्वाहेत्य॑नु - आह॑ । \newline
15. अ॒प्र॒ त्ये॑वैवा प्र॒त्य॑ प्र॒त्ये॑व । \newline
16. ए॒व तेन॒ तेनै॒ वैव तेन॑ । \newline
17. तेन॒ यज॑मानो॒ यज॑मान॒ स्तेन॒ तेन॒ यज॑मानः । \newline
18. यज॑मानो॒ भ्रातृ॑व्या॒न् भ्रातृ॑व्या॒न्॒. यज॑मानो॒ यज॑मानो॒ भ्रातृ॑व्यान् । \newline
19. भ्रातृ॑व्यान् जयति जयति॒ भ्रातृ॑व्या॒न् भ्रातृ॑व्यान् जयति । \newline
20. ज॒य॒ त्यथो॒ अथो॑ जयति जय॒ त्यथो᳚ । \newline
21. अथो॒ अन॑भिजित॒ मन॑भिजित॒ मथो॒ अथो॒ अन॑भिजितम् । \newline
22. अथो॒ इत्यथो᳚ । \newline
23. अन॑भिजित मे॒वैवा न॑भिजित॒ मन॑भिजित मे॒व । \newline
24. अन॑भिजित॒मित्यन॑भि - जि॒त॒म् । \newline
25. ए॒वाभ्या᳚(1॒) भ्ये॑वै वाभि । \newline
26. अ॒भि ज॑यति जय त्य॒भ्य॑भि ज॑यति । \newline
27. ज॒य॒ति॒ द॒श॒र्चम् द॑श॒र्चम् ज॑यति जयति दश॒र्चम् । \newline
28. द॒श॒र्चम् भ॑वति भवति दश॒र्चम् द॑श॒र्चम् भ॑वति । \newline
29. द॒श॒र्चमिति॑ दश - ऋ॒चम् । \newline
30. भ॒व॒ति॒ दशा᳚क्षरा॒ दशा᳚क्षरा भवति भवति॒ दशा᳚क्षरा । \newline
31. दशा᳚क्षरा वि॒राड् वि॒राड् दशा᳚क्षरा॒ दशा᳚क्षरा वि॒राट् । \newline
32. दशा᳚क्ष॒रेति॒ दश॑ - अ॒क्ष॒रा॒ । \newline
33. वि॒राड् वि॒राजा॑ वि॒राजा॑ वि॒राड् वि॒राड् वि॒राजा᳚ । \newline
34. वि॒राडिति॑ वि - राट् । \newline
35. वि॒राजे॒मा वि॒मौ वि॒राजा॑ वि॒राजे॒मौ । \newline
36. वि॒राजेति॑ वि - राजा᳚ । \newline
37. इ॒मौ लो॒कौ लो॒का वि॒मा वि॒मौ लो॒कौ । \newline
38. लो॒कौ विधृ॑तौ॒ विधृ॑तौ लो॒कौ लो॒कौ विधृ॑तौ । \newline
39. विधृ॑ता व॒नयो॑ र॒नयो॒र् विधृ॑तौ॒ विधृ॑ता व॒नयोः᳚ । \newline
40. विधृ॑ता॒विति॒ वि - धृ॒तौ॒ । \newline
41. अ॒नयो᳚र् लो॒कयो᳚र् लो॒कयो॑ र॒नयो॑ र॒नयो᳚र् लो॒कयोः᳚ । \newline
42. लो॒कयो॒र् विधृ॑त्यै॒ विधृ॑त्यै लो॒कयो᳚र् लो॒कयो॒र् विधृ॑त्यै । \newline
43. विधृ॑त्या॒ अथो॒ अथो॒ विधृ॑त्यै॒ विधृ॑त्या॒ अथो᳚ । \newline
44. विधृ॑त्या॒ इति॒ वि - धृ॒त्यै॒ । \newline
45. अथो॒ दशा᳚क्षरा॒ दशा᳚क्ष॒रा ऽथो॒ अथो॒ दशा᳚क्षरा । \newline
46. अथो॒ इत्यथो᳚ । \newline
47. दशा᳚क्षरा वि॒राड् वि॒राड् दशा᳚क्षरा॒ दशा᳚क्षरा वि॒राट् । \newline
48. दशा᳚क्ष॒रेति॒ दश॑ - अ॒क्ष॒रा॒ । \newline
49. वि॒राडन्न॒ मन्नं॑ ॅवि॒राड् वि॒राडन्न᳚म् । \newline
50. वि॒राडिति॑ वि - राट् । \newline
51. अन्नं॑ ॅवि॒राड् वि॒राडन्न॒ मन्नं॑ ॅवि॒राट् । \newline
52. वि॒राड् वि॒राजि॑ वि॒राजि॑ वि॒राड् वि॒राड् वि॒राजि॑ । \newline
53. वि॒राडिति॑ वि - राट् । \newline
54. वि॒राज्ये॒वैव वि॒राजि॑ वि॒राज्ये॒व । \newline
55. वि॒राजीति॑ वि - राजि॑ । \newline
56. ए॒वा न्नाद्ये॒ ऽन्नाद्य॑ ए॒वैवा न्नाद्ये᳚ । \newline
57. अ॒न्नाद्ये॒ प्रति॒ प्रत्य॒न्नाद्ये॒ ऽन्नाद्ये॒ प्रति॑ । \newline
58. अ॒न्नाद्य॒ इत्य॑न्न - अद्ये᳚ । \newline
59. प्रति॑ तिष्ठति तिष्ठति॒ प्रति॒ प्रति॑ तिष्ठति । \newline
60. ति॒ष्ठ॒ त्यस॒ दस॑त् तिष्ठति तिष्ठ॒ त्यस॑त् । \newline
61. अस॑दिवे॒वा स॒ दस॑ दिव । \newline
62. इ॒व॒ वै वा इ॑वेव॒ वै । \newline
63. वा अ॒न्तरि॑क्ष म॒न्तरि॑क्षं॒ ॅवै वा अ॒न्तरि॑क्षम् । \newline
64. अ॒न्तरि॑क्ष म॒न्तरि॑क्षम् । \newline
65. अ॒न्तरि॑क्ष मिवे वा॒न्तरि॑क्ष म॒न्तरि॑क्ष मिव । \newline
66. इ॒वाग्नी᳚ध्र॒ माग्नी᳚ध्र मिवे॒ वाग्नी᳚ध्रम् । \newline
67. आग्नी᳚ध्र॒ माग्नी᳚द्ध्र॒ आग्नी᳚द्ध्र॒ आग्नी᳚ध्र॒ माग्नी᳚ध्र॒ माग्नी᳚द्ध्रे । \newline
68. आग्नी᳚ध्र॒मित्याग्नि॑ - इ॒ध्र॒म् । \newline
69. आग्नी॒द्ध्रे ऽश्मा॑न॒ मश्मा॑न॒ माग्नी᳚द्ध्र॒ आग्नी॒द्ध्रे ऽश्मा॑नम् । \newline
70. आग्नी᳚द्ध्र॒ इत्याग्नि॑ - इ॒द्ध्रे॒ । \newline

\textbf{Ghana Paata } \newline

1. असु॑रा नजयन् नजय॒न् नसु॑रा॒ नसु॑रा नजय॒न् तत् तद॑जय॒न् नसु॑रा॒ नसु॑रा नजय॒न् तत् । \newline
2. अ॒ज॒य॒न् तत् तद॑जयन् नजय॒न् तदप्र॑तिरथ॒स्या प्र॑तिरथस्य॒ तद॑जयन् नजय॒न् तदप्र॑तिरथस्य । \newline
3. तदप्र॑तिरथ॒स्या प्र॑तिरथस्य॒ तत् तदप्र॑तिरथस्या प्रतिरथ॒त्व म॑प्रतिरथ॒त्व मप्र॑तिरथस्य॒ तत् तदप्र॑तिरथस्या प्रतिरथ॒त्वम् । \newline
4. अप्र॑तिरथस्या प्रतिरथ॒त्व म॑प्रतिरथ॒त्व मप्र॑तिरथ॒स्या प्र॑तिरथस्या प्रतिरथ॒त्वं ॅयद् यद॑प्रतिरथ॒त्व मप्र॑तिरथ॒स्या प्र॑तिरथस्या प्रतिरथ॒त्वं ॅयत् । \newline
5. अप्र॑तिरथ॒स्येत्यप्र॑ति - र॒थ॒स्य॒ । \newline
6. अ॒प्र॒ति॒र॒थ॒त्वं ॅयद् यद॑प्रतिरथ॒त्व म॑प्रतिरथ॒त्वं ॅयदप्र॑तिरथ॒ मप्र॑तिरथं॒ ॅयद॑प्रतिरथ॒त्व म॑प्रतिरथ॒त्वं ॅयदप्र॑तिरथम् । \newline
7. अ॒प्र॒ति॒र॒थ॒त्वमित्य॑प्रतिरथ - त्वम् । \newline
8. यदप्र॑तिरथ॒ मप्र॑तिरथं॒ ॅयद् यदप्र॑तिरथम् द्वि॒तीयो᳚ द्वि॒तीयो ऽप्र॑तिरथं॒ ॅयद् यदप्र॑तिरथम् द्वि॒तीयः॑ । \newline
9. अप्र॑तिरथम् द्वि॒तीयो᳚ द्वि॒तीयो ऽप्र॑तिरथ॒ मप्र॑तिरथम् द्वि॒तीयो॒ होता॒ होता᳚ द्वि॒तीयो ऽप्र॑तिरथ॒ मप्र॑तिरथम् द्वि॒तीयो॒ होता᳚ । \newline
10. अप्र॑तिरथ॒मित्यप्र॑ति - र॒थ॒म् । \newline
11. द्वि॒तीयो॒ होता॒ होता᳚ द्वि॒तीयो᳚ द्वि॒तीयो॒ होता॒ ऽन्वाहा॒ न्वाह॒ होता᳚ द्वि॒तीयो᳚ द्वि॒तीयो॒ होता॒ ऽन्वाह॑ । \newline
12. होता॒ ऽन्वाहा॒ न्वाह॒ होता॒ होता॒ ऽन्वाहा᳚ प्र॒त्य॑ प्र॒त्य॑न्वाह॒ होता॒ होता॒ ऽन्वाहा᳚ प्र॒ति । \newline
13. अ॒न्वाहा᳚ प्र॒त्य॑ प्र॒त्य॑न्वाहा॒ न्वाहा᳚ प्र॒त्ये॑वैवा प्र॒त्य॑न्वाहा॒ न्वाहा᳚ प्र॒त्ये॑व । \newline
14. अ॒न्वाहेत्य॑नु - आह॑ । \newline
15. अ॒प्र॒त्ये॑ वैवा प्र॒त्य॑ प्र॒त्ये॑व तेन॒ तेनै॒वा प्र॒त्य॑ प्र॒त्ये॑व तेन॑ । \newline
16. ए॒व तेन॒ तेनै॒ वैव तेन॒ यज॑मानो॒ यज॑मान॒ स्तेनै॒ वैव तेन॒ यज॑मानः । \newline
17. तेन॒ यज॑मानो॒ यज॑मान॒ स्तेन॒ तेन॒ यज॑मानो॒ भ्रातृ॑व्या॒न् भ्रातृ॑व्या॒न्॒. यज॑मान॒ स्तेन॒ तेन॒ यज॑मानो॒ भ्रातृ॑व्यान् । \newline
18. यज॑मानो॒ भ्रातृ॑व्या॒न् भ्रातृ॑व्या॒न्॒. यज॑मानो॒ यज॑मानो॒ भ्रातृ॑व्यान् जयति जयति॒ भ्रातृ॑व्या॒न्॒. यज॑मानो॒ यज॑मानो॒ भ्रातृ॑व्यान् जयति । \newline
19. भ्रातृ॑व्यान् जयति जयति॒ भ्रातृ॑व्या॒न् भ्रातृ॑व्यान् जय॒ त्यथो॒ अथो॑ जयति॒ भ्रातृ॑व्या॒न् भ्रातृ॑व्यान् जय॒ त्यथो᳚ । \newline
20. ज॒य॒ त्यथो॒ अथो॑ जयति जय॒ त्यथो॒ अन॑भिजित॒ मन॑भिजित॒ मथो॑ जयति जय॒ त्यथो॒ अन॑भिजितम् । \newline
21. अथो॒ अन॑भिजित॒ मन॑भिजित॒ मथो॒ अथो॒ अन॑भिजित मे॒वैवा न॑भिजित॒ मथो॒ अथो॒ अन॑भिजित मे॒व । \newline
22. अथो॒ इत्यथो᳚ । \newline
23. अन॑भिजित मे॒वैवा न॑भिजित॒ मन॑भिजित मे॒वाभ्या᳚(1॒)भ्ये॑ वान॑भिजित॒ मन॑भिजित मे॒वाभि । \newline
24. अन॑भिजित॒मित्यन॑भि - जि॒त॒म् । \newline
25. ए॒वाभ्या᳚(1॒)भ्ये॑ वैवाभि ज॑यति जय त्य॒भ्ये॑ वैवाभि ज॑यति । \newline
26. अ॒भि ज॑यति जय त्य॒भ्य॑भि ज॑यति दश॒र्चम् द॑श॒र्चम् ज॑य त्य॒भ्य॑भि ज॑यति दश॒र्चम् । \newline
27. ज॒य॒ति॒ द॒श॒र्चम् द॑श॒र्चम् ज॑यति जयति दश॒र्चम् भ॑वति भवति दश॒र्चम् ज॑यति जयति दश॒र्चम् भ॑वति । \newline
28. द॒श॒र्चम् भ॑वति भवति दश॒र्चम् द॑श॒र्चम् भ॑वति॒ दशा᳚क्षरा॒ दशा᳚क्षरा भवति दश॒र्चम् द॑श॒र्चम् भ॑वति॒ दशा᳚क्षरा । \newline
29. द॒श॒र्चमिति॑ दश - ऋ॒चम् । \newline
30. भ॒व॒ति॒ दशा᳚क्षरा॒ दशा᳚क्षरा भवति भवति॒ दशा᳚क्षरा वि॒राड् वि॒राड् दशा᳚क्षरा भवति भवति॒ दशा᳚क्षरा वि॒राट् । \newline
31. दशा᳚क्षरा वि॒राड् वि॒राड् दशा᳚क्षरा॒ दशा᳚क्षरा वि॒राड् वि॒राजा॑ वि॒राजा॑ वि॒राड् दशा᳚क्षरा॒ दशा᳚क्षरा वि॒राड् वि॒राजा᳚ । \newline
32. दशा᳚क्ष॒रेति॒ दश॑ - अ॒क्ष॒रा॒ । \newline
33. वि॒राड् वि॒राजा॑ वि॒राजा॑ वि॒राड् वि॒राड् वि॒राजे॒मा वि॒मौ वि॒राजा॑ वि॒राड् वि॒राड् वि॒राजे॒मौ । \newline
34. वि॒राडिति॑ वि - राट् । \newline
35. वि॒राजे॒मा वि॒मौ वि॒राजा॑ वि॒राजे॒मौ लो॒कौ लो॒का वि॒मौ वि॒राजा॑ वि॒राजे॒मौ लो॒कौ । \newline
36. वि॒राजेति॑ वि - राजा᳚ । \newline
37. इ॒मौ लो॒कौ लो॒का वि॒मा वि॒मौ लो॒कौ विधृ॑तौ॒ विधृ॑तौ लो॒का वि॒मा वि॒मौ लो॒कौ विधृ॑तौ । \newline
38. लो॒कौ विधृ॑तौ॒ विधृ॑तौ लो॒कौ लो॒कौ विधृ॑ता व॒नयो॑ र॒नयो॒र् विधृ॑तौ लो॒कौ लो॒कौ विधृ॑ता व॒नयोः᳚ । \newline
39. विधृ॑ता व॒नयो॑ र॒नयो॒र् विधृ॑तौ॒ विधृ॑ता व॒नयो᳚र् लो॒कयो᳚र् लो॒कयो॑ र॒नयो॒र् विधृ॑तौ॒ विधृ॑ता व॒नयो᳚र् लो॒कयोः᳚ । \newline
40. विधृ॑ता॒विति॒ वि - धृ॒तौ॒ । \newline
41. अ॒नयो᳚र् लो॒कयो᳚र् लो॒कयो॑ र॒नयो॑ र॒नयो᳚र् लो॒कयो॒र् विधृ॑त्यै॒ विधृ॑त्यै लो॒कयो॑ र॒नयो॑ र॒नयो᳚र् लो॒कयो॒र् विधृ॑त्यै । \newline
42. लो॒कयो॒र् विधृ॑त्यै॒ विधृ॑त्यै लो॒कयो᳚र् लो॒कयो॒र् विधृ॑त्या॒ अथो॒ अथो॒ विधृ॑त्यै लो॒कयो᳚र् लो॒कयो॒र् विधृ॑त्या॒ अथो᳚ । \newline
43. विधृ॑त्या॒ अथो॒ अथो॒ विधृ॑त्यै॒ विधृ॑त्या॒ अथो॒ दशा᳚क्षरा॒ दशा᳚क्ष॒रा ऽथो॒ विधृ॑त्यै॒ विधृ॑त्या॒ अथो॒ दशा᳚क्षरा । \newline
44. विधृ॑त्या॒ इति॒ वि - धृ॒त्यै॒ । \newline
45. अथो॒ दशा᳚क्षरा॒ दशा᳚क्ष॒रा ऽथो॒ अथो॒ दशा᳚क्षरा वि॒राड् वि॒राड् दशा᳚क्ष॒रा ऽथो॒ अथो॒ दशा᳚क्षरा वि॒राट् । \newline
46. अथो॒ इत्यथो᳚ । \newline
47. दशा᳚क्षरा वि॒राड् वि॒राड् दशा᳚क्षरा॒ दशा᳚क्षरा वि॒रा डन्न॒ मन्नं॑ ॅवि॒राड् दशा᳚क्षरा॒ दशा᳚क्षरा वि॒रा डन्न᳚म् । \newline
48. दशा᳚क्ष॒रेति॒ दश॑ - अ॒क्ष॒रा॒ । \newline
49. वि॒रा डन्न॒ मन्नं॑ ॅवि॒राड् वि॒रा डन्नं॑ ॅवि॒राड् वि॒रा डन्नं॑ ॅवि॒राड् वि॒रा डन्नं॑ ॅवि॒राट् । \newline
50. वि॒राडिति॑ वि - राट् । \newline
51. अन्नं॑ ॅवि॒राड् वि॒रा डन्न॒ मन्नं॑ ॅवि॒राड् वि॒राजि॑ वि॒राजि॑ वि॒रा डन्न॒ मन्नं॑ ॅवि॒राड् वि॒राजि॑ । \newline
52. वि॒राड् वि॒राजि॑ वि॒राजि॑ वि॒राड् वि॒राड् वि॒रा ज्ये॒वैव वि॒राजि॑ वि॒राड् वि॒राड् वि॒राज्ये॒व । \newline
53. वि॒राडिति॑ वि - राट् । \newline
54. वि॒रा ज्ये॒वैव वि॒राजि॑ वि॒राज्ये॒ वान्नाद्ये॒ ऽन्नाद्य॑ ए॒व वि॒राजि॑ वि॒राज्ये॒ वान्नाद्ये᳚ । \newline
55. वि॒राजीति॑ वि - राजि॑ । \newline
56. ए॒वान्नाद्ये॒ ऽन्नाद्य॑ ए॒वै वान्नाद्ये॒ प्रति॒ प्रत्य॒न्नाद्य॑ ए॒वै वान्नाद्ये॒ प्रति॑ । \newline
57. अ॒न्नाद्ये॒ प्रति॒ प्रत्य॒न्नाद्ये॒ ऽन्नाद्ये॒ प्रति॑ तिष्ठति तिष्ठति॒ प्रत्य॒न्नाद्ये॒ ऽन्नाद्ये॒ प्रति॑ तिष्ठति । \newline
58. अ॒न्नाद्य॒ इत्य॑न्न - अद्ये᳚ । \newline
59. प्रति॑ तिष्ठति तिष्ठति॒ प्रति॒ प्रति॑ तिष्ठ॒ त्यस॒ दस॑त् तिष्ठति॒ प्रति॒ प्रति॑ तिष्ठ॒ त्यस॑त् । \newline
60. ति॒ष्ठ॒ त्यस॒ दस॑त् तिष्ठति तिष्ठ॒ त्यस॑दिवे॒ वास॑त् तिष्ठति तिष्ठ॒ त्यस॑दिव । \newline
61. अस॑दिवे॒ वास॒ दस॑ दिव॒ वै वा इ॒वास॒ दस॑ दिव॒ वै । \newline
62. इ॒व॒ वै वा इ॑वेव॒ वा अ॒न्तरि॑क्ष म॒न्तरि॑क्षं॒ ॅवा इ॑वेव॒ वा अ॒न्तरि॑क्षम् । \newline
63. वा अ॒न्तरि॑क्ष म॒न्तरि॑क्षं॒ ॅवै वा अ॒न्तरि॑क्षम् । \newline
64. अ॒न्तरि॑क्ष म॒न्तरि॑क्षम् । \newline
65. अ॒न्तरि॑क्ष मिवे वा॒न्तरि॑क्ष म॒न्तरि॑क्ष मि॒वाग्नी᳚ध्र॒ माग्नी᳚ध्र मिवा॒न्तरि॑क्ष म॒न्तरि॑क्ष मि॒वाग्नी᳚ध्रम् । \newline
66. इ॒वाग्नी᳚ध्र॒ माग्नी᳚ध्र मिवे॒ वाग्नी᳚ध्र॒ माग्नी᳚द्ध्र॒ आग्नी᳚द्ध्र॒ आग्नी᳚ध्र मिवे॒ वाग्नी᳚ध्र॒ माग्नी᳚द्ध्रे । \newline
67. आग्नी᳚ध्र॒ माग्नी᳚द्ध्र॒ आग्नी᳚द्ध्र॒ आग्नी᳚ध्र॒ माग्नी᳚ध्र॒ माग्नी॒द्ध्रे ऽश्मा॑न॒ मश्मा॑न॒ माग्नी᳚द्ध्र॒ आग्नी᳚ध्र॒ माग्नी᳚ध्र॒ माग्नी॒द्ध्रे ऽश्मा॑नम् । \newline
68. आग्नी᳚ध्र॒मित्याग्नि॑ - इ॒ध्र॒म् । \newline
69. आग्नी॒द्ध्रे ऽश्मा॑न॒ मश्मा॑न॒ माग्नी᳚द्ध्र॒ आग्नी॒द्ध्रे ऽश्मा॑न॒म् नि न्यश्मा॑न॒ माग्नी᳚द्ध्र॒ आग्नी॒द्ध्रे ऽश्मा॑न॒म् नि । \newline
70. आग्नी᳚द्ध्र॒ इत्याग्नि॑ - इ॒द्ध्रे॒ । \newline
\pagebreak
\markright{ TS 5.4.6.5  \hfill https://www.vedavms.in \hfill}

\section{ TS 5.4.6.5 }

\textbf{TS 5.4.6.5 } \newline
\textbf{Samhita Paata} \newline

ऽश्मा॑नं॒ नि द॑धाति स॒त्त्वाय॒ द्वाभ्यां॒ प्रति॑ष्ठित्यै वि॒मान॑ ए॒ष दि॒वो मद्ध्य॑ आस्त॒ इत्या॑ह॒ व्ये॑वैतया॑ मिमीते॒ मद्ध्ये॑ दि॒वो निहि॑तः॒ पृश्नि॒रश्मेत्या॒हान्नं॒ ॅवै पृश्न्यन्न॑मे॒वाव॑ रुन्धे चत॒सृभि॒रा पुच्छा॑देति च॒त्वारि॒ छन्दाꣳ॑सि॒ छन्दो॑भिरे॒वेन्द्रं॒ ॅविश्वा॑ अवीवृध॒न्नित्या॑ह॒ वृद्धि॑मे॒वोपाव॑र्तते॒ वाजा॑नाꣳ॒॒ सत्प॑तिं॒ पति॒ - [  ] \newline

\textbf{Pada Paata} \newline

अश्मा॑नम् । नीति॑ । द॒धा॒ति॒ । स॒त्त्वायेति॑ सत् - त्वाय॑ । द्वाभ्या᳚म् । प्रति॑ष्ठित्या॒ इति॒ प्रति॑ - स्थि॒त्यै॒ । वि॒मान॒ इति॑ वि - मानः॑ । ए॒षः । दि॒वः । मद्ध्ये᳚ । आ॒स्ते॒ । इति॑ । आ॒ह॒ । वीति॑ । ए॒व । ए॒तया᳚ । मि॒मी॒ते॒ । मद्ध्ये᳚ । दि॒वः । निहि॑त॒ इति॒ नि - हि॒तः॒ । पृश्निः॑ । अश्मा᳚ । इति॑ । आ॒ह॒ । अन्न᳚म् । वै । पृश्नि॑ । अन्न᳚म् । ए॒व । अवेति॑ । रु॒न्धे॒ । च॒त॒सृभि॒रिति॑ चत॒सृ-भिः॒ । एति॑ । पुच्छा᳚त् । ए॒ति॒ । च॒त्वारि॑ । छन्दाꣳ॑सि । छन्दो॑भि॒रिति॒ छन्दः॑ - भिः॒ । ए॒व । इन्द्र᳚म् । विश्वाः᳚ । अ॒वी॒वृ॒ध॒न्न् । इति॑ । आ॒ह॒ । वृद्धि᳚म् । ए॒व । उ॒पाव॑र्तत॒ इत्यु॑प-आव॑र्तते । वाजा॑नाम् । सत्प॑ति॒मिति॒ सत् - प॒ति॒म् । पति᳚म् ।  \newline


\textbf{Krama Paata} \newline

अश्मा॑न॒म् नि । नि द॑धाति । द॒धा॒ति॒ स॒त्त्वाय॑ । स॒त्त्वाय॒ द्वाभ्या᳚म् । स॒त्त्वायेति॑ सत् - त्वाय॑ । द्वाभ्या॒म् प्रति॑ष्ठित्यै । प्रति॑ष्ठित्यै वि॒मानः॑ । प्रति॑ष्ठित्या॒ इति॒ प्रति॑ - स्थि॒त्यै॒ । वि॒मान॑ ए॒षः । वि॒मान॒ इति॑ वि - मानः॑ । ए॒ष दि॒वः । दि॒वो मद्ध्ये᳚ । मद्ध्य॑ आस्ते । आ॒स्त॒ इति॑ । इत्या॑ह । आ॒ह॒ वि । व्ये॑व । ए॒वैतया᳚ । ए॒तया॑ मिमीते । मि॒मी॒ते॒ मद्ध्ये᳚ । मद्ध्ये॑ दि॒वः । दि॒वो निहि॑तः । निहि॑तः॒ पृश्ञिः॑ । निहि॑त॒ इति॒ नि - हि॒तः॒ । पृश्ञि॒रश्मा᳚ । अश्मेति॑ । इत्या॑ह । आ॒हान्न᳚म् । अन्न॒म् ॅवै । वै पृश्ञि॑ । पृश्ञ्यन्न᳚म् । अन्न॑मे॒व । ए॒वाव॑ । अव॑ रुन्धे । रु॒न्धे॒ च॒त॒सृभिः॑ । च॒त॒सृभि॒रा । च॒त॒सृभि॒रिति॑ चत॒सृ - भिः॒ । आ पुच्छा᳚त् । पुच्छा॑देति । ए॒ति॒ च॒त्वारि॑ । च॒त्वारि॒ छन्दाꣳ॑सि । छन्दाꣳ॑सि॒ छन्दो॑भिः । छन्दो॑भिरे॒व । छन्दो॑भि॒रिति॒ छन्दः॑ - भिः॒ । ए॒वेन्द्र᳚म् । इन्द्र॒म् ॅविश्वाः᳚ । विश्वा॑ अवीवृधन्न् । अ॒वी॒वृ॒ध॒न्निति॑ । इत्या॑ह । आ॒ह॒ वृद्धि᳚म् । वृद्धि॑मे॒व । ए॒वोपाव॑र्तते । उ॒पाव॑र्तते॒ वाजा॑नाम् । उ॒पाव॑र्तत॒ इत्यु॑प - आव॑र्तते । वाजा॑नाꣳ॒॒ सत्प॑तिम् । सत्प॑ति॒म् पति᳚म् । सत्प॑ति॒मिति॒ सत् - प॒ति॒म् । पति॒मिति॑ \newline

\textbf{Jatai Paata} \newline

1. अश्मा॑न॒म् नि न्यश्मा॑न॒ मश्मा॑न॒म् नि । \newline
2. नि द॑धाति दधाति॒ नि नि द॑धाति । \newline
3. द॒धा॒ति॒ स॒त्त्वाय॑ स॒त्त्वाय॑ दधाति दधाति स॒त्त्वाय॑ । \newline
4. स॒त्त्वाय॒ द्वाभ्या॒म् द्वाभ्याꣳ॑ स॒त्त्वाय॑ स॒त्त्वाय॒ द्वाभ्या᳚म् । \newline
5. स॒त्त्वायेति॑ सत् - त्वाय॑ । \newline
6. द्वाभ्या॒म् प्रति॑ष्ठित्यै॒ प्रति॑ष्ठित्यै॒ द्वाभ्या॒म् द्वाभ्या॒म् प्रति॑ष्ठित्यै । \newline
7. प्रति॑ष्ठित्यै वि॒मानो॑ वि॒मानः॒ प्रति॑ष्ठित्यै॒ प्रति॑ष्ठित्यै वि॒मानः॑ । \newline
8. प्रति॑ष्ठित्या॒ इति॒ प्रति॑ - स्थि॒त्यै॒ । \newline
9. वि॒मान॑ ए॒ष ए॒ष वि॒मानो॑ वि॒मान॑ ए॒षः । \newline
10. वि॒मान॒ इति॑ वि - मानः॑ । \newline
11. ए॒ष दि॒वो दि॒व ए॒ष ए॒ष दि॒वः । \newline
12. दि॒वो मद्ध्ये॒ मद्ध्ये॑ दि॒वो दि॒वो मद्ध्ये᳚ । \newline
13. मद्ध्य॑ आस्त आस्ते॒ मद्ध्ये॒ मद्ध्य॑ आस्ते । \newline
14. आ॒स्त॒ इती त्या᳚स्त आस्त॒ इति॑ । \newline
15. इत्या॑हा॒हे तीत्या॑ह । \newline
16. आ॒ह॒ वि व्या॑हाह॒ वि । \newline
17. व्ये॑वैव वि व्ये॑व । \newline
18. ए॒वैत यै॒त यै॒वै वैतया᳚ । \newline
19. ए॒तया॑ मिमीते मिमीत ए॒त यै॒तया॑ मिमीते । \newline
20. मि॒मी॒ते॒ मद्ध्ये॒ मद्ध्ये॑ मिमीते मिमीते॒ मद्ध्ये᳚ । \newline
21. मद्ध्ये॑ दि॒वो दि॒वो मद्ध्ये॒ मद्ध्ये॑ दि॒वः । \newline
22. दि॒वो निहि॑तो॒ निहि॑तो दि॒वो दि॒वो निहि॑तः । \newline
23. निहि॑तः॒ पृश्ञिः॒ पृश्ञि॒र् निहि॑तो॒ निहि॑तः॒ पृश्ञिः॑ । \newline
24. निहि॑त॒ इति॒ नि - हि॒तः॒ । \newline
25. पृश्ञि॒ रश्मा ऽश्मा॒ पृश्ञिः॒ पृश्ञि॒ रश्मा᳚ । \newline
26. अश्मेती त्यश्मा ऽश्मेति॑ । \newline
27. इत्या॑हा॒हे तीत्या॑ह । \newline
28. आ॒हान्न॒ मन्न॑ माहा॒ हान्न᳚म् । \newline
29. अन्नं॒ ॅवै वा अन्न॒ मन्नं॒ ॅवै । \newline
30. वै पृश्ञि॒ पृश्ञि॒ वै वै पृश्ञि॑ । \newline
31. पृश्ञ्यन्न॒ मन्न॒म् पृश्ञि॒ पृश्ञ्यन्न᳚म् । \newline
32. अन्न॑ मे॒वै वान्न॒ मन्न॑ मे॒व । \newline
33. ए॒वावा वै॒वै वाव॑ । \newline
34. अव॑ रुन्धे रु॒न्धे ऽवाव॑ रुन्धे । \newline
35. रु॒न्धे॒ च॒त॒सृभि॑ श्चत॒सृभी॑ रुन्धे रुन्धे चत॒सृभिः॑ । \newline
36. च॒त॒सृभि॒रा च॑त॒सृभि॑ श्चत॒सृभि॒रा । \newline
37. च॒त॒सृभि॒रिति॑ चत॒सृ - भिः॒ । \newline
38. आ पुच्छा॒त् पुच्छा॒दा पुच्छा᳚त् । \newline
39. पुच्छा॑ देत्येति॒ पुच्छा॒त् पुच्छा॑देति । \newline
40. ए॒ति॒ च॒त्वारि॑ च॒त्वार्ये᳚ त्येति च॒त्वारि॑ । \newline
41. च॒त्वारि॒ छन्दाꣳ॑सि॒ छन्दाꣳ॑सि च॒त्वारि॑ च॒त्वारि॒ छन्दाꣳ॑सि । \newline
42. छन्दाꣳ॑सि॒ छन्दो॑भि॒ श्छन्दो॑भि॒ श्छन्दाꣳ॑सि॒ छन्दाꣳ॑सि॒ छन्दो॑भिः । \newline
43. छन्दो॑भि रे॒वैव छन्दो॑भि॒ श्छन्दो॑भि रे॒व । \newline
44. छन्दो॑भि॒रिति॒ छन्दः॑ - भिः॒ । \newline
45. ए॒वेन्द्र॒ मिन्द्र॑ मे॒वैवेन्द्र᳚म् । \newline
46. इन्द्रं॒ ॅविश्वा॒ विश्वा॒ इन्द्र॒ मिन्द्रं॒ ॅविश्वाः᳚ । \newline
47. विश्वा॑ अवीवृधन् नवीवृध॒न्॒. विश्वा॒ विश्वा॑ अवीवृधन्न् । \newline
48. अ॒वी॒वृ॒ध॒न् निती त्य॑वीवृधन् नवीवृध॒न् निति॑ । \newline
49. इत्या॑हा॒हे तीत्या॑ह । \newline
50. आ॒ह॒ वृद्धिं॒ ॅवृद्धि॑ माहाह॒ वृद्धि᳚म् । \newline
51. वृद्धि॑ मे॒वैव वृद्धिं॒ ॅवृद्धि॑ मे॒व । \newline
52. ए॒वो पाव॑र्तत उ॒पाव॑र्तत ए॒वैवो पाव॑र्तते । \newline
53. उ॒पाव॑र्तते॒ वाजा॑नां॒ ॅवाजा॑ना मु॒पाव॑र्तत उ॒पाव॑र्तते॒ वाजा॑नाम् । \newline
54. उ॒पाव॑र्तत॒ इत्यु॑प - आव॑र्तते । \newline
55. वाजा॑नाꣳ॒॒ सत्प॑तिꣳ॒॒ सत्प॑तिं॒ ॅवाजा॑नां॒ ॅवाजा॑नाꣳ॒॒ सत्प॑तिम् । \newline
56. सत्प॑ति॒म् पति॒म् पतिꣳ॒॒ सत्प॑तिꣳ॒॒ सत्प॑ति॒म् पति᳚म् । \newline
57. सत्प॑ति॒मिति॒ सत् - प॒ति॒म् । \newline
58. पति॒ मितीति॒ पति॒म् पति॒ मिति॑ । \newline

\textbf{Ghana Paata } \newline

1. अश्मा॑न॒म् नि न्यश्मा॑न॒ मश्मा॑न॒म् नि द॑धाति दधाति॒ न्यश्मा॑न॒ मश्मा॑न॒म् नि द॑धाति । \newline
2. नि द॑धाति दधाति॒ नि नि द॑धाति स॒त्त्वाय॑ स॒त्त्वाय॑ दधाति॒ नि नि द॑धाति स॒त्त्वाय॑ । \newline
3. द॒धा॒ति॒ स॒त्त्वाय॑ स॒त्त्वाय॑ दधाति दधाति स॒त्त्वाय॒ द्वाभ्या॒म् द्वाभ्याꣳ॑ स॒त्त्वाय॑ दधाति दधाति स॒त्त्वाय॒ द्वाभ्या᳚म् । \newline
4. स॒त्त्वाय॒ द्वाभ्या॒म् द्वाभ्याꣳ॑ स॒त्त्वाय॑ स॒त्त्वाय॒ द्वाभ्या॒म् प्रति॑ष्ठित्यै॒ प्रति॑ष्ठित्यै॒ द्वाभ्याꣳ॑ स॒त्त्वाय॑ स॒त्त्वाय॒ द्वाभ्या॒म् प्रति॑ष्ठित्यै । \newline
5. स॒त्त्वायेति॑ सत् - त्वाय॑ । \newline
6. द्वाभ्या॒म् प्रति॑ष्ठित्यै॒ प्रति॑ष्ठित्यै॒ द्वाभ्या॒म् द्वाभ्या॒म् प्रति॑ष्ठित्यै वि॒मानो॑ वि॒मानः॒ प्रति॑ष्ठित्यै॒ द्वाभ्या॒म् द्वाभ्या॒म् प्रति॑ष्ठित्यै वि॒मानः॑ । \newline
7. प्रति॑ष्ठित्यै वि॒मानो॑ वि॒मानः॒ प्रति॑ष्ठित्यै॒ प्रति॑ष्ठित्यै वि॒मान॑ ए॒ष ए॒ष वि॒मानः॒ प्रति॑ष्ठित्यै॒ प्रति॑ष्ठित्यै वि॒मान॑ ए॒षः । \newline
8. प्रति॑ष्ठित्या॒ इति॒ प्रति॑ - स्थि॒त्यै॒ । \newline
9. वि॒मान॑ ए॒ष ए॒ष वि॒मानो॑ वि॒मान॑ ए॒ष दि॒वो दि॒व ए॒ष वि॒मानो॑ वि॒मान॑ ए॒ष दि॒वः । \newline
10. वि॒मान॒ इति॑ वि - मानः॑ । \newline
11. ए॒ष दि॒वो दि॒व ए॒ष ए॒ष दि॒वो मद्ध्ये॒ मद्ध्ये॑ दि॒व ए॒ष ए॒ष दि॒वो मद्ध्ये᳚ । \newline
12. दि॒वो मद्ध्ये॒ मद्ध्ये॑ दि॒वो दि॒वो मद्ध्य॑ आस्त आस्ते॒ मद्ध्ये॑ दि॒वो दि॒वो मद्ध्य॑ आस्ते । \newline
13. मद्ध्य॑ आस्त आस्ते॒ मद्ध्ये॒ मद्ध्य॑ आस्त॒ इती त्या᳚स्ते॒ मद्ध्ये॒ मद्ध्य॑ आस्त॒ इति॑ । \newline
14. आ॒स्त॒ इती त्या᳚स्त आस्त॒ इत्या॑हा॒हे त्या᳚स्त आस्त॒ इत्या॑ह । \newline
15. इत्या॑हा॒हे तीत्या॑ह॒ वि व्या॑हे तीत्या॑ह॒ वि । \newline
16. आ॒ह॒ वि व्या॑हाह॒ व्ये॑वैव व्या॑हाह॒ व्ये॑व । \newline
17. व्ये॑वैव वि व्ये॑वैत यै॒त यै॒व वि व्ये॑वै तया᳚ । \newline
18. ए॒वै तयै॒ तयै॒वै वैतया॑ मिमीते मिमीत ए॒तयै॒वै वैतया॑ मिमीते । \newline
19. ए॒तया॑ मिमीते मिमीत ए॒त यै॒तया॑ मिमीते॒ मद्ध्ये॒ मद्ध्ये॑ मिमीत ए॒त यै॒तया॑ मिमीते॒ मद्ध्ये᳚ । \newline
20. मि॒मी॒ते॒ मद्ध्ये॒ मद्ध्ये॑ मिमीते मिमीते॒ मद्ध्ये॑ दि॒वो दि॒वो मद्ध्ये॑ मिमीते मिमीते॒ मद्ध्ये॑ दि॒वः । \newline
21. मद्ध्ये॑ दि॒वो दि॒वो मद्ध्ये॒ मद्ध्ये॑ दि॒वो निहि॑तो॒ निहि॑तो दि॒वो मद्ध्ये॒ मद्ध्ये॑ दि॒वो निहि॑तः । \newline
22. दि॒वो निहि॑तो॒ निहि॑तो दि॒वो दि॒वो निहि॑तः॒ पृश्ञिः॒ पृश्ञि॒र् निहि॑तो दि॒वो दि॒वो निहि॑तः॒ पृश्ञिः॑ । \newline
23. निहि॑तः॒ पृश्ञिः॒ पृश्ञि॒र् निहि॑तो॒ निहि॑तः॒ पृश्ञि॒ रश्मा ऽश्मा॒ पृश्ञि॒र् निहि॑तो॒ निहि॑तः॒ पृश्ञि॒ रश्मा᳚ । \newline
24. निहि॑त॒ इति॒ नि - हि॒तः॒ । \newline
25. पृश्ञि॒ रश्मा ऽश्मा॒ पृश्ञिः॒ पृश्ञि॒ रश्मेती त्यश्मा॒ पृश्ञिः॒ पृश्ञि॒ रश्मेति॑ । \newline
26. अश्मेती त्यश्मा ऽश्मेत्या॑ हा॒हे त्यश्मा ऽश्मेत्या॑ह । \newline
27. इत्या॑हा॒हे तीत्या॒ हान्न॒ मन्न॑ मा॒हे तीत्या॒ हान्न᳚म् । \newline
28. आ॒हान्न॒ मन्न॑ माहा॒ हान्नं॒ ॅवै वा अन्न॑ माहा॒ हान्नं॒ ॅवै । \newline
29. अन्नं॒ ॅवै वा अन्न॒ मन्नं॒ ॅवै पृश्ञि॒ पृश्ञि॒ वा अन्न॒ मन्नं॒ ॅवै पृश्ञि॑ । \newline
30. वै पृश्ञि॒ पृश्ञि॒ वै वै पृश्ञ्यन्न॒ मन्न॒म् पृश्ञि॒ वै वै पृश्ञ्यन्न᳚म् । \newline
31. पृश्ञ्यन्न॒ मन्न॒म् पृश्ञि॒ पृश्ञ्यन्न॑ मे॒वै वान्न॒म् पृश्ञि॒ पृश्ञ्यन्न॑ मे॒व । \newline
32. अन्न॑ मे॒वै वान्न॒ मन्न॑ मे॒वा वावै॒ वान्न॒ मन्न॑ मे॒वाव॑ । \newline
33. ए॒वावा वै॒वै वाव॑ रुन्धे रु॒न्धे ऽवै॒वै वाव॑ रुन्धे । \newline
34. अव॑ रुन्धे रु॒न्धे ऽवाव॑ रुन्धे चत॒सृभि॑ श्चत॒सृभी॑ रु॒न्धे ऽवाव॑ रुन्धे चत॒सृभिः॑ । \newline
35. रु॒न्धे॒ च॒त॒सृभि॑ श्चत॒सृभी॑ रुन्धे रुन्धे चत॒सृभि॒रा च॑त॒सृभी॑ रुन्धे रुन्धे चत॒सृभि॒रा । \newline
36. च॒त॒सृभि॒रा च॑त॒सृभि॑ श्चत॒सृभि॒रा पुच्छा॒त् पुच्छा॒दा च॑त॒सृभि॑ श्चत॒सृभि॒रा पुच्छा᳚त् । \newline
37. च॒त॒सृभि॒रिति॑ चत॒सृ - भिः॒ । \newline
38. आ पुच्छा॒त् पुच्छा॒दा पुच्छा॑ देत्येति॒ पुच्छा॒दा पुच्छा॑ देति । \newline
39. पुच्छा॑ देत्येति॒ पुच्छा॒त् पुच्छा॑ देति च॒त्वारि॑ च॒त्वार्ये॑ति॒ पुच्छा॒त् पुच्छा॑देति च॒त्वारि॑ । \newline
40. ए॒ति॒ च॒त्वारि॑ च॒त्वार्ये᳚ त्येति च॒त्वारि॒ छन्दाꣳ॑सि॒ छन्दाꣳ॑सि च॒त्वार्ये᳚ त्येति च॒त्वारि॒ छन्दाꣳ॑सि । \newline
41. च॒त्वारि॒ छन्दाꣳ॑सि॒ छन्दाꣳ॑सि च॒त्वारि॑ च॒त्वारि॒ छन्दाꣳ॑सि॒ छन्दो॑भि॒ श्छन्दो॑भि॒ श्छन्दाꣳ॑सि च॒त्वारि॑ च॒त्वारि॒ छन्दाꣳ॑सि॒ छन्दो॑भिः । \newline
42. छन्दाꣳ॑सि॒ छन्दो॑भि॒ श्छन्दो॑भि॒ श्छन्दाꣳ॑सि॒ छन्दाꣳ॑सि॒ छन्दो॑भि रे॒वैव छन्दो॑भि॒ श्छन्दाꣳ॑सि॒ छन्दाꣳ॑सि॒ छन्दो॑भि रे॒व । \newline
43. छन्दो॑भि रे॒वैव छन्दो॑भि॒ श्छन्दो॑भि रे॒वेन्द्र॒ मिन्द्र॑ मे॒व छन्दो॑भि॒ श्छन्दो॑भि रे॒वेन्द्र᳚म् । \newline
44. छन्दो॑भि॒रिति॒ छन्दः॑ - भिः॒ । \newline
45. ए॒वेन्द्र॒ मिन्द्र॑ मे॒वैवेन्द्रं॒ ॅविश्वा॒ विश्वा॒ इन्द्र॑ मे॒वैवेन्द्रं॒ ॅविश्वाः᳚ । \newline
46. इन्द्रं॒ ॅविश्वा॒ विश्वा॒ इन्द्र॒ मिन्द्रं॒ ॅविश्वा॑ अवीवृधन् नवीवृध॒न्॒. विश्वा॒ इन्द्र॒ मिन्द्रं॒ ॅविश्वा॑ अवीवृधन्न् । \newline
47. विश्वा॑ अवीवृधन् नवीवृध॒न्॒. विश्वा॒ विश्वा॑ अवीवृध॒न् निती त्य॑वीवृध॒न्॒. विश्वा॒ विश्वा॑ अवीवृध॒न् निति॑ । \newline
48. अ॒वी॒वृ॒ध॒न् निती त्य॑वीवृधन् नवीवृध॒न् नित्या॑हा॒हे त्य॑वीवृधन् नवीवृध॒न् नित्या॑ह । \newline
49. इत्या॑हा॒हे तीत्या॑ह॒ वृद्धिं॒ ॅवृद्धि॑ मा॒हे तीत्या॑ह॒ वृद्धि᳚म् । \newline
50. आ॒ह॒ वृद्धिं॒ ॅवृद्धि॑ माहाह॒ वृद्धि॑ मे॒वैव वृद्धि॑ माहाह॒ वृद्धि॑ मे॒व । \newline
51. वृद्धि॑ मे॒वैव वृद्धिं॒ ॅवृद्धि॑ मे॒वो पाव॑र्तत उ॒पाव॑र्तत ए॒व वृद्धिं॒ ॅवृद्धि॑ मे॒वो पाव॑र्तते । \newline
52. ए॒वो पाव॑र्तत उ॒पाव॑र्तत ए॒वैवो पाव॑र्तते॒ वाजा॑नां॒ ॅवाजा॑ना मु॒पाव॑र्तत ए॒वैवो पाव॑र्तते॒ वाजा॑नाम् । \newline
53. उ॒पाव॑र्तते॒ वाजा॑नां॒ ॅवाजा॑ना मु॒पाव॑र्तत उ॒पाव॑र्तते॒ वाजा॑नाꣳ॒॒ सत्प॑तिꣳ॒॒ सत्प॑तिं॒ ॅवाजा॑ना मु॒पाव॑र्तत उ॒पाव॑र्तते॒ वाजा॑नाꣳ॒॒ सत्प॑तिम् । \newline
54. उ॒पाव॑र्तत॒ इत्यु॑प - आव॑र्तते । \newline
55. वाजा॑नाꣳ॒॒ सत्प॑तिꣳ॒॒ सत्प॑तिं॒ ॅवाजा॑नां॒ ॅवाजा॑नाꣳ॒॒ सत्प॑ति॒म् पति॒म् पतिꣳ॒॒ सत्प॑तिं॒ ॅवाजा॑नां॒ ॅवाजा॑नाꣳ॒॒ सत्प॑ति॒म् पति᳚म् । \newline
56. सत्प॑ति॒म् पति॒म् पतिꣳ॒॒ सत्प॑तिꣳ॒॒ सत्प॑ति॒म् पति॒ मितीति॒ पतिꣳ॒॒ सत्प॑तिꣳ॒॒ सत्प॑ति॒म् पति॒ मिति॑ । \newline
57. सत्प॑ति॒मिति॒ सत् - प॒ति॒म् । \newline
58. पति॒ मितीति॒ पति॒म् पति॒ मित्या॑हा॒हेति॒ पति॒म् पति॒ मित्या॑ह । \newline
\pagebreak
\markright{ TS 5.4.6.6  \hfill https://www.vedavms.in \hfill}

\section{ TS 5.4.6.6 }

\textbf{TS 5.4.6.6 } \newline
\textbf{Samhita Paata} \newline

-मित्या॒हाऽन्नं॒ ॅवै वाजोऽन्न॑मे॒वाव॑ रुन्धे सुम्न॒हूर्य॒ज्ञो दे॒वाꣳ आ च॑ वक्ष॒दित्या॑ह प्र॒जा वै प॒शवः॑ सु॒म्नं प्र॒जामे॒व प॒शूना॒त्मन् ध॑त्ते॒ यक्ष॑द॒ग्निर्दे॒वो दे॒वाꣳ आ च॑ वक्ष॒दित्या॑ह स्व॒गाकृ॑त्यै॒ वाज॑स्य मा प्रस॒वेनो᳚द्-ग्रा॒भेणोद॑ग्रभी॒दित्या॑हा॒सौ वा आ॑दि॒त्य उ॒द्यन्नु॑द्ग्रा॒भ ए॒ष ( ) नि॒म्रोच॑न् निग्रा॒भो ब्रह्म॑णै॒वाऽऽ*त्मान॑मुद्-गृ॒ह्णाति॒ ब्रह्म॑णा॒ भ्रातृ॑व्यं॒ नि गृ॑ह्णाति ॥ \newline

\textbf{Pada Paata} \newline

इति॑ । आ॒ह॒ । अन्न᳚म् । वै । वाजः॑ । अन्न᳚म् । ए॒व । अवेति॑ । रु॒न्धे॒ । सु॒म्न॒हूरिति॑ सुम्न - हूः । य॒ज्ञ्ः । दे॒वान् । एति॑ । च॒ । व॒क्ष॒त् । इति॑ । आ॒ह॒ । प्र॒जेति॑ प्र - जा । वै । प॒शवः॑ । सु॒म्नम् । प्र॒जामिति॑ प्र - जाम् । ए॒व । प॒शून् । आ॒त्मन्न् । ध॒त्ते॒ । यक्ष॑त् । अ॒ग्निः । दे॒वः । दे॒वान् । एति॑ । च॒ । व॒क्ष॒त् । इति॑ । आ॒ह॒ । स्व॒गाकृ॑त्या॒ इति॑ स्व॒गा - कृ॒त्यै॒ । वाज॑स्य । मा॒ । प्र॒स॒वेनेति॑ प्र - स॒वेन॑ । उ॒द्ग्रा॒भेणेत्यू॑त्-ग्रा॒भेण॑ । उदिति॑ । अ॒ग्र॒भी॒त् । इति॑ । आ॒ह॒ । अ॒सौ । वै । आ॒दि॒त्यः । उ॒द्यन्नित्यु॑त् - यन्न् । उ॒द्ग्रा॒भ इत्यु॑त् - ग्रा॒भः । ए॒षः ( ) । नि॒म्रोच॒न्निति॑ नि - म्रोचन्न्॑ । नि॒ग्रा॒भ इति॑ नि - ग्रा॒भः । ब्रह्म॑णा । ए॒व । आ॒त्मान᳚म् । उ॒द्गृ॒ह्णातीत्यु॑त् - गृ॒ह्णाति॑ । ब्रह्म॑णा । भ्रातृ॑व्यम् । नीति॑ । गृ॒ह्णा॒ति॒ ॥  \newline


\textbf{Krama Paata} \newline

इत्या॑ह । आ॒हान्न᳚म् । अन्न॒म् ॅवै । वै वाजः॑ । वाजोऽन्न᳚म् । अन्न॑मे॒व । ए॒वाव॑ । अव॑ रुन्धे । रु॒न्धे॒ सु॒म्न॒हूः । सु॒म्न॒हूर् य॒ज्ञ्ः । सु॒म्न॒हूरिति॑ सुम्न - हूः । य॒ज्ञो दे॒वान् । दे॒वाꣳ आ । आ च॑ । च॒ व॒क्ष॒त्॒ । व॒क्ष॒दिति॑ । इत्या॑ह । आ॒ह॒ प्र॒जा । प्र॒जा वै । प्र॒जेति॑ प्र - जा । वै प॒शवः॑ । प॒शवः॑ सु॒म्नम् । सु॒म्नम् प्र॒जाम् । प्र॒जामे॒व । प्र॒जामिति॑ प्र - जाम् । ए॒व प॒शून् । प॒शूना॒त्मन्न् । आ॒त्मन् ध॑त्ते । ध॒त्ते॒ यक्ष॑त् । यक्ष॑द॒ग्निः । अ॒ग्निर् दे॒वः । दे॒वो दे॒वान् । दे॒वाꣳ आ । आ च॑ । च॒ व॒क्ष॒त्॒ । व॒क्ष॒दिति॑ । इत्या॑ह । आ॒ह॒ स्व॒गाकृ॑त्यै । स्व॒गाकृ॑त्यै॒ वाज॑स्य । स्व॒गाकृ॑त्या॒ इति॑ स्व॒गा - कृ॒त्यै॒ । वाज॑स्य मा । मा॒ प्र॒स॒वेन॑ । प्र॒स॒वेनो᳚द्ग्रा॒भेण॑ । प्र॒स॒वेनेति॑ प्र - स॒वेन॑ । उ॒द्ग्रा॒भेणोत् । उ॒द्ग्रा॒भेणेत्यु॑त् - ग्रा॒भेण॑ । उद॑ग्रभीत् । अ॒ग्र॒भी॒दिति॑ । इत्या॑ह । आ॒हा॒सौ । अ॒सौ वै । वा आ॑दि॒त्यः । आ॒दि॒त्य उ॒द्यन्न् । उ॒द्यन्नु॑द्ग्रा॒भः । उ॒द्यन्नित्यु॑त् - यन्न् । उ॒द्ग्रा॒भ ए॒षः ( ) । उ॒द्ग्रा॒भ इत्यु॑त् - ग्रा॒भः । ए॒ष नि॒म्रोचन्न्॑ । नि॒म्रोच॑न् निग्रा॒भः । नि॒म्रोच॒न्निति॑ नि - म्रोचन्न्॑ । नि॒ग्रा॒भो ब्रह्म॑णा । नि॒ग्रा॒भ इति॑ नि - ग्रा॒भः । ब्रह्म॑णै॒व । ए॒वात्मान᳚म् । आ॒त्मान॑मुद्गृ॒ह्णाति॑ । उ॒द्गृ॒ह्णाति॒ ब्रह्म॑णा । उ॒द्गृ॒ह्णातीत्यु॑त् - गृ॒ह्णाति॑ । ब्रह्म॑णा॒ भ्रातृ॑व्यम् । भ्रातृ॑व्य॒म् नि । नि गृ॑ह्णाति । गृ॒ह्णा॒तीति॑ गृह्णाति । \newline

\textbf{Jatai Paata} \newline

1. इत्या॑हा॒हे तीत्या॑ह । \newline
2. आ॒हान्न॒ मन्न॑ माहा॒ हान्न᳚म् । \newline
3. अन्नं॒ ॅवै वा अन्न॒ मन्नं॒ ॅवै । \newline
4. वै वाजो॒ वाजो॒ वै वै वाजः॑ । \newline
5. वाजो ऽन्न॒ मन्नं॒ ॅवाजो॒ वाजो ऽन्न᳚म् । \newline
6. अन्न॑ मे॒वै वान्न॒ मन्न॑ मे॒व । \newline
7. ए॒वावा वै॒वै वाव॑ । \newline
8. अव॑ रुन्धे रु॒न्धे ऽवाव॑ रुन्धे । \newline
9. रु॒न्धे॒ सु॒म्न॒हूः सु॑म्न॒हू रु॑न्धे रुन्धे सुम्न॒हूः । \newline
10. सु॒म्न॒हूर् य॒ज्ञो य॒ज्ञ्ः सु॑म्न॒हूः सु॑म्न॒हूर् य॒ज्ञ्ः । \newline
11. सु॒म्न॒हूरिति॑ सुम्न - हूः । \newline
12. य॒ज्ञो दे॒वान् दे॒वान्. य॒ज्ञो य॒ज्ञो दे॒वान् । \newline
13. दे॒वाꣳ आ दे॒वान् दे॒वाꣳ आ । \newline
14. आ च॒ चा च॑ । \newline
15. च॒ व॒क्ष॒द् व॒क्ष॒च् च॒ च॒ व॒क्ष॒त् । \newline
16. व॒क्ष॒ दितीति॑ वक्षद् वक्ष॒ दिति॑ । \newline
17. इत्या॑हा॒हे तीत्या॑ह । \newline
18. आ॒ह॒ प्र॒जा प्र॒जा ऽऽहा॑ह प्र॒जा । \newline
19. प्र॒जा वै वै प्र॒जा प्र॒जा वै । \newline
20. प्र॒जेति॑ प्र - जा । \newline
21. वै प॒शवः॑ प॒शवो॒ वै वै प॒शवः॑ । \newline
22. प॒शवः॑ सु॒म्नꣳ सु॒म्नम् प॒शवः॑ प॒शवः॑ सु॒म्नम् । \newline
23. सु॒म्नम् प्र॒जाम् प्र॒जाꣳ सु॒म्नꣳ सु॒म्नम् प्र॒जाम् । \newline
24. प्र॒जा मे॒वैव प्र॒जाम् प्र॒जा मे॒व । \newline
25. प्र॒जामिति॑ प्र - जाम् । \newline
26. ए॒व प॒शून् प॒शू ने॒वैव प॒शून् । \newline
27. प॒शू ना॒त्मन् ना॒त्मन् प॒शून् प॒शू ना॒त्मन्न् । \newline
28. आ॒त्मन् ध॑त्ते धत्त आ॒त्मन् ना॒त्मन् ध॑त्ते । \newline
29. ध॒त्ते॒ यक्ष॒द् यक्ष॑द् धत्ते धत्ते॒ यक्ष॑त् । \newline
30. यक्ष॑ द॒ग्नि र॒ग्निर् यक्ष॒द् यक्ष॑ द॒ग्निः । \newline
31. अ॒ग्निर् दे॒वो दे॒वो अ॒ग्नि र॒ग्निर् दे॒वः । \newline
32. दे॒वो दे॒वान् दे॒वान् दे॒वो दे॒वो दे॒वान् । \newline
33. दे॒वाꣳ आ दे॒वान् दे॒वाꣳ आ । \newline
34. आ च॒ चा च॑ । \newline
35. च॒ व॒क्ष॒द् व॒क्ष॒च् च॒ च॒ व॒क्ष॒त् । \newline
36. व॒क्ष॒ दितीति॑ वक्षद् वक्ष॒ दिति॑ । \newline
37. इत्या॑हा॒हे तीत्या॑ह । \newline
38. आ॒ह॒ स्व॒गाकृ॑त्यै स्व॒गाकृ॑त्या आहाह स्व॒गाकृ॑त्यै । \newline
39. स्व॒गाकृ॑त्यै॒ वाज॑स्य॒ वाज॑स्य स्व॒गाकृ॑त्यै स्व॒गाकृ॑त्यै॒ वाज॑स्य । \newline
40. स्व॒गाकृ॑त्या॒ इति॑ स्व॒गा - कृ॒त्यै॒ । \newline
41. वाज॑स्य मा मा॒ वाज॑स्य॒ वाज॑स्य मा । \newline
42. मा॒ प्र॒स॒वेन॑ प्रस॒वेन॑ मा मा प्रस॒वेन॑ । \newline
43. प्र॒स॒वे नो᳚द्ग्रा॒भे णो᳚द्ग्रा॒भेण॑ प्रस॒वेन॑ प्रस॒वे नो᳚द्ग्रा॒भेण॑ । \newline
44. प्र॒स॒वेनेति॑ प्र - स॒वेन॑ । \newline
45. उ॒द्ग्रा॒भे णोदुदु॑ द्ग्रा॒भे णो᳚द्ग्रा॒भेणोत् । \newline
46. उ॒द्ग्रा॒भेणेत्यू॑त् - ग्रा॒भेण॑ । \newline
47. उद॑ग्रभी दग्रभी॒ दुदु द॑ग्रभीत् । \newline
48. अ॒ग्र॒भी॒ दिती त्य॑ग्रभी दग्रभी॒ दिति॑ । \newline
49. इत्या॑हा॒हे तीत्या॑ह । \newline
50. आ॒हा॒सा व॒सा वा॑हाहा॒सौ । \newline
51. अ॒सौ वै वा अ॒सा व॒सौ वै । \newline
52. वा आ॑दि॒त्य आ॑दि॒त्यो वै वा आ॑दि॒त्यः । \newline
53. आ॒दि॒त्य उ॒द्यन् नु॒द्यन् ना॑दि॒त्य आ॑दि॒त्य उ॒द्यन्न् । \newline
54. उ॒द्यन् नु॑द्ग्रा॒भ उ॑द्ग्रा॒भ उ॒द्यन् नु॒द्यन् नु॑द्ग्रा॒भः । \newline
55. उ॒द्यन्नित्यु॑त् - यन्न् । \newline
56. उ॒द्ग्रा॒भ ए॒ष ए॒ष उ॑द्ग्रा॒भ उ॑द्ग्रा॒भ ए॒षः । \newline
57. उ॒द्ग्रा॒भ इत्यु॑त् - ग्रा॒भः । \newline
58. ए॒ष नि॒म्रोच॑न् नि॒म्रोच॑न् ने॒ष ए॒ष नि॒म्रोचन्न्॑ । \newline
59. नि॒म्रोच॑न् निग्रा॒भो नि॑ग्रा॒भो नि॒म्रोच॑न् नि॒म्रोच॑न् निग्रा॒भः । \newline
60. नि॒म्रोच॒न्निति॑ नि - म्रोचन्न्॑ । \newline
61. नि॒ग्रा॒भो ब्रह्म॑णा॒ ब्रह्म॑णा निग्रा॒भो नि॑ग्रा॒भो ब्रह्म॑णा । \newline
62. नि॒ग्रा॒भ इति॑ नि - ग्रा॒भः । \newline
63. ब्रह्म॑ णै॒वैव ब्रह्म॑णा॒ ब्रह्म॑णै॒व । \newline
64. ए॒वात्मान॑ मा॒त्मान॑ मे॒वै वात्मान᳚म् । \newline
65. आ॒त्मान॑ मुद्‍गृ॒ह्णा त्यु॑द्‍गृ॒ह्णा त्या॒त्मान॑ मा॒त्मान॑ मुद्‍गृ॒ह्णाति॑ । \newline
66. उ॒द्‍गृ॒ह्णाति॒ ब्रह्म॑णा॒ ब्रह्म॑ णोद्‍गृ॒ह्णा त्यु॑द्‍गृ॒ह्णाति॒ ब्रह्म॑णा । \newline
67. उ॒द्‍गृ॒ह्णातीत्यु॑त् - गृ॒ह्णाति॑ । \newline
68. ब्रह्म॑णा॒ भ्रातृ॑व्य॒म् भ्रातृ॑व्य॒म् ब्रह्म॑णा॒ ब्रह्म॑णा॒ भ्रातृ॑व्यम् । \newline
69. भ्रातृ॑व्य॒म् नि नि भ्रातृ॑व्य॒म् भ्रातृ॑व्य॒म् नि । \newline
70. नि गृ॑ह्णाति गृह्णाति॒ नि नि गृ॑ह्णाति । \newline
71. गृ॒ह्णा॒तीति॑ गृह्णाति । \newline

\textbf{Ghana Paata } \newline

1. इत्या॑ हा॒हेती त्या॒हान्न॒ मन्न॑ मा॒हेती त्या॒हान्न᳚म् । \newline
2. आ॒हान्न॒ मन्न॑ माहा॒ हान्नं॒ ॅवै वा अन्न॑ माहा॒ हान्नं॒ ॅवै । \newline
3. अन्नं॒ ॅवै वा अन्न॒ मन्नं॒ ॅवै वाजो॒ वाजो॒ वा अन्न॒ मन्नं॒ ॅवै वाजः॑ । \newline
4. वै वाजो॒ वाजो॒ वै वै वाजो ऽन्न॒ मन्नं॒ ॅवाजो॒ वै वै वाजो ऽन्न᳚म् । \newline
5. वाजो ऽन्न॒ मन्नं॒ ॅवाजो॒ वाजो ऽन्न॑ मे॒वै वान्नं॒ ॅवाजो॒ वाजो ऽन्न॑ मे॒व । \newline
6. अन्न॑ मे॒वै वान्न॒ मन्न॑ मे॒वावा वै॒वान्न॒ मन्न॑ मे॒वाव॑ । \newline
7. ए॒वावा वै॒वै वाव॑ रुन्धे रु॒न्धे ऽवै॒वै वाव॑ रुन्धे । \newline
8. अव॑ रुन्धे रु॒न्धे ऽवाव॑ रुन्धे सुम्न॒हूः सु॑म्न॒हू रु॒न्धे ऽवाव॑ रुन्धे सुम्न॒हूः । \newline
9. रु॒न्धे॒ सु॒म्न॒हूः सु॑म्न॒हू रु॑न्धे रुन्धे सुम्न॒हूर् य॒ज्ञो य॒ज्ञ्ः सु॑म्न॒हू रु॑न्धे रुन्धे सुम्न॒हूर् य॒ज्ञ्ः । \newline
10. सु॒म्न॒हूर् य॒ज्ञो य॒ज्ञ्ः सु॑म्न॒हूः सु॑म्न॒हूर् य॒ज्ञो दे॒वान् दे॒वान्. य॒ज्ञ्ः सु॑म्न॒हूः सु॑म्न॒हूर् य॒ज्ञो दे॒वान् । \newline
11. सु॒म्न॒हूरिति॑ सुम्न - हूः । \newline
12. य॒ज्ञो दे॒वान् दे॒वान्. य॒ज्ञो य॒ज्ञो दे॒वाꣳ आ दे॒वान्. य॒ज्ञो य॒ज्ञो दे॒वाꣳ आ । \newline
13. दे॒वाꣳ आ दे॒वान् दे॒वाꣳ आ च॒ चा दे॒वान् दे॒वाꣳ आ च॑ । \newline
14. आ च॒ चा च॑ वक्षद् वक्ष॒च् चा च॑ वक्षत् । \newline
15. च॒ व॒क्ष॒द् व॒क्ष॒च् च॒ च॒ व॒क्ष॒ दितीति॑ वक्षच् च च वक्ष॒ दिति॑ । \newline
16. व॒क्ष॒ दितीति॑ वक्षद् वक्ष॒ दित्या॑ हा॒हेति॑ वक्षद् वक्ष॒ दित्या॑ह । \newline
17. इत्या॑हा॒हे तीत्या॑ह प्र॒जा प्र॒जा ऽऽहेती त्या॑ह प्र॒जा । \newline
18. आ॒ह॒ प्र॒जा प्र॒जा ऽऽहा॑ह प्र॒जा वै वै प्र॒जा ऽऽहा॑ह प्र॒जा वै । \newline
19. प्र॒जा वै वै प्र॒जा प्र॒जा वै प॒शवः॑ प॒शवो॒ वै प्र॒जा प्र॒जा वै प॒शवः॑ । \newline
20. प्र॒जेति॑ प्र - जा । \newline
21. वै प॒शवः॑ प॒शवो॒ वै वै प॒शवः॑ सु॒म्नꣳ सु॒म्नम् प॒शवो॒ वै वै प॒शवः॑ सु॒म्नम् । \newline
22. प॒शवः॑ सु॒म्नꣳ सु॒म्नम् प॒शवः॑ प॒शवः॑ सु॒म्नम् प्र॒जाम् प्र॒जाꣳ सु॒म्नम् प॒शवः॑ प॒शवः॑ सु॒म्नम् प्र॒जाम् । \newline
23. सु॒म्नम् प्र॒जाम् प्र॒जाꣳ सु॒म्नꣳ सु॒म्नम् प्र॒जा मे॒वैव प्र॒जाꣳ सु॒म्नꣳ सु॒म्नम् प्र॒जा मे॒व । \newline
24. प्र॒जा मे॒वैव प्र॒जाम् प्र॒जा मे॒व प॒शून् प॒शू ने॒व प्र॒जाम् प्र॒जा मे॒व प॒शून् । \newline
25. प्र॒जामिति॑ प्र - जाम् । \newline
26. ए॒व प॒शून् प॒शू ने॒वैव प॒शू ना॒त्मन् ना॒त्मन् प॒शू ने॒वैव प॒शू ना॒त्मन्न् । \newline
27. प॒शू ना॒त्मन् ना॒त्मन् प॒शून् प॒शू ना॒त्मन् ध॑त्ते धत्त आ॒त्मन् प॒शून् प॒शू ना॒त्मन् ध॑त्ते । \newline
28. आ॒त्मन् ध॑त्ते धत्त आ॒त्मन् ना॒त्मन् ध॑त्ते॒ यक्ष॒द् यक्ष॑द् धत्त आ॒त्मन् ना॒त्मन् ध॑त्ते॒ यक्ष॑त् । \newline
29. ध॒त्ते॒ यक्ष॒द् यक्ष॑द् धत्ते धत्ते॒ यक्ष॑ द॒ग्नि र॒ग्निर् यक्ष॑द् धत्ते धत्ते॒ यक्ष॑ द॒ग्निः । \newline
30. यक्ष॑ द॒ग्नि र॒ग्निर् यक्ष॒द् यक्ष॑ द॒ग्निर् दे॒वो दे॒वो अ॒ग्निर् यक्ष॒द् यक्ष॑ द॒ग्निर् दे॒वः । \newline
31. अ॒ग्निर् दे॒वो दे॒वो अ॒ग्नि र॒ग्निर् दे॒वो दे॒वान् दे॒वान् दे॒वो अ॒ग्नि र॒ग्निर् दे॒वो दे॒वान् । \newline
32. दे॒वो दे॒वान् दे॒वान् दे॒वो दे॒वो दे॒वाꣳ आ दे॒वान् दे॒वो दे॒वो दे॒वाꣳ आ । \newline
33. दे॒वाꣳ आ दे॒वान् दे॒वाꣳ आ च॒ चा दे॒वान् दे॒वाꣳ आ च॑ । \newline
34. आ च॒ चा च॑ वक्षद् वक्ष॒च् चा च॑ वक्षत् । \newline
35. च॒ व॒क्ष॒द् व॒क्ष॒च् च॒ च॒ व॒क्ष॒ दितीति॑ वक्षच् च च वक्ष॒ दिति॑ । \newline
36. व॒क्ष॒ दितीति॑ वक्षद् वक्ष॒ दित्या॑ हा॒हेति॑ वक्षद् वक्ष॒ दित्या॑ह । \newline
37. इत्या॑हा॒हे तीत्या॑ह स्व॒गाकृ॑त्यै स्व॒गाकृ॑त्या आ॒हे तीत्या॑ह स्व॒गाकृ॑त्यै । \newline
38. आ॒ह॒ स्व॒गाकृ॑त्यै स्व॒गाकृ॑त्या आहाह स्व॒गाकृ॑त्यै॒ वाज॑स्य॒ वाज॑स्य स्व॒गाकृ॑त्या आहाह स्व॒गाकृ॑त्यै॒ वाज॑स्य । \newline
39. स्व॒गाकृ॑त्यै॒ वाज॑स्य॒ वाज॑स्य स्व॒गाकृ॑त्यै स्व॒गाकृ॑त्यै॒ वाज॑स्य मा मा॒ वाज॑स्य स्व॒गाकृ॑त्यै स्व॒गाकृ॑त्यै॒ वाज॑स्य मा । \newline
40. स्व॒गाकृ॑त्या॒ इति॑ स्व॒गा - कृ॒त्यै॒ । \newline
41. वाज॑स्य मा मा॒ वाज॑स्य॒ वाज॑स्य मा प्रस॒वेन॑ प्रस॒वेन॑ मा॒ वाज॑स्य॒ वाज॑स्य मा प्रस॒वेन॑ । \newline
42. मा॒ प्र॒स॒वेन॑ प्रस॒वेन॑ मा मा प्रस॒वेनो᳚ द्ग्रा॒भेणो᳚ द्ग्रा॒भेण॑ प्रस॒वेन॑ मा मा प्रस॒वेनो᳚ द्ग्रा॒भेण॑ । \newline
43. प्र॒स॒वेनो᳚ द्ग्रा॒भेणो᳚ द्ग्रा॒भेण॑ प्रस॒वेन॑ प्रस॒वेनो᳚ द्ग्रा॒भेणो दुदु॑द् ग्रा॒भेण॑ प्रस॒वेन॑ प्रस॒वेनो᳚ द्ग्रा॒भेणोत् । \newline
44. प्र॒स॒वेनेति॑ प्र - स॒वेन॑ । \newline
45. उ॒द्ग्रा॒भेणो दुदु॑द् ग्रा॒भेणो᳚ द्ग्रा॒भेणो द॑ग्रभी दग्रभी॒ दुदु॑द् ग्रा॒भेणो᳚ द्ग्रा॒भेणो द॑ग्रभीत् । \newline
46. उ॒द्ग्रा॒भेणेत्यू॑त् - ग्रा॒भेण॑ । \newline
47. उद॑ग्रभी दग्रभी॒ दुदु द॑ग्रभी॒ दिती त्य॑ग्रभी॒ दुदु द॑ग्रभी॒ दिति॑ । \newline
48. अ॒ग्र॒भी॒ दिती त्य॑ग्रभी दग्रभी॒ दित्या॑ हा॒हे त्य॑ग्रभी दग्रभी॒ दित्या॑ह । \newline
49. इत्या॑ हा॒हेती त्या॑हा॒सा व॒सा वा॒हेती त्या॑हा॒सौ । \newline
50. आ॒हा॒सा व॒सा वा॑हा हा॒सौ वै वा अ॒सा वा॑हा हा॒सौ वै । \newline
51. अ॒सौ वै वा अ॒सा व॒सौ वा आ॑दि॒त्य आ॑दि॒त्यो वा अ॒सा व॒सौ वा आ॑दि॒त्यः । \newline
52. वा आ॑दि॒त्य आ॑दि॒त्यो वै वा आ॑दि॒त्य उ॒द्यन् नु॒द्यन् ना॑दि॒त्यो वै वा आ॑दि॒त्य उ॒द्यन्न् । \newline
53. आ॒दि॒त्य उ॒द्यन् नु॒द्यन् ना॑दि॒त्य आ॑दि॒त्य उ॒द्यन् नु॑द्ग्रा॒भ उ॑द्ग्रा॒भ उ॒द्यन् ना॑दि॒त्य आ॑दि॒त्य उ॒द्यन् नु॑द्ग्रा॒भः । \newline
54. उ॒द्यन् नु॑द्ग्रा॒भ उ॑द्ग्रा॒भ उ॒द्यन् नु॒द्यन् नु॑द्ग्रा॒भ ए॒ष ए॒ष उ॑द्ग्रा॒भ उ॒द्यन् नु॒द्यन् नु॑द्ग्रा॒भ ए॒षः । \newline
55. उ॒द्यन्नित्यु॑त् - यन्न् । \newline
56. उ॒द्ग्रा॒भ ए॒ष ए॒ष उ॑द्ग्रा॒भ उ॑द्ग्रा॒भ ए॒ष नि॒म्रोच॑न् नि॒म्रोच॑न् ने॒ष उ॑द्ग्रा॒भ उ॑द्ग्रा॒भ ए॒ष नि॒म्रोचन्न्॑ । \newline
57. उ॒द्ग्रा॒भ इत्यु॑त् - ग्रा॒भः । \newline
58. ए॒ष नि॒म्रोच॑न् नि॒म्रोच॑न् ने॒ष ए॒ष नि॒म्रोच॑न् निग्रा॒भो नि॑ग्रा॒भो नि॒म्रोच॑न् ने॒ष ए॒ष नि॒म्रोच॑न् निग्रा॒भः । \newline
59. नि॒म्रोच॑न् निग्रा॒भो नि॑ग्रा॒भो नि॒म्रोच॑न् नि॒म्रोच॑न् निग्रा॒भो ब्रह्म॑णा॒ ब्रह्म॑णा निग्रा॒भो नि॒म्रोच॑न् नि॒म्रोच॑न् निग्रा॒भो ब्रह्म॑णा । \newline
60. नि॒म्रोच॒न्निति॑ नि - म्रोचन्न्॑ । \newline
61. नि॒ग्रा॒भो ब्रह्म॑णा॒ ब्रह्म॑णा निग्रा॒भो नि॑ग्रा॒भो ब्रह्म॑णै॒वैव ब्रह्म॑णा निग्रा॒भो नि॑ग्रा॒भो ब्रह्म॑णै॒व । \newline
62. नि॒ग्रा॒भ इति॑ नि - ग्रा॒भः । \newline
63. ब्रह्म॑ णै॒वैव ब्रह्म॑णा॒ ब्रह्म॑ णै॒वात्मान॑ मा॒त्मान॑ मे॒व ब्रह्म॑णा॒ ब्रह्म॑ णै॒वात्मान᳚म् । \newline
64. ए॒वात्मान॑ मा॒त्मान॑ मे॒वै वात्मान॑ मुद्‍गृ॒ह्णा त्यु॑द्‍गृ॒ह्णा त्या॒त्मान॑ मे॒वै वात्मान॑ मुद्‍गृ॒ह्णाति॑ । \newline
65. आ॒त्मान॑ मुद्‍गृ॒ह्णा त्यु॑द्‍गृ॒ह्णा त्या॒त्मान॑ मा॒त्मान॑ मुद्‍गृ॒ह्णाति॒ ब्रह्म॑णा॒ ब्रह्म॑णो द्‍गृ॒ह्णा त्या॒त्मान॑ मा॒त्मान॑ मुद्‍गृ॒ह्णाति॒ ब्रह्म॑णा । \newline
66. उ॒द्‍गृ॒ह्णाति॒ ब्रह्म॑णा॒ ब्रह्म॑ णोद्‍गृ॒ह्णा त्यु॑द्‍गृ॒ह्णाति॒ ब्रह्म॑णा॒ भ्रातृ॑व्य॒म् भ्रातृ॑व्य॒म् ब्रह्म॑ णोद्‍गृ॒ह्णा त्यु॑द्‍गृ॒ह्णाति॒ ब्रह्म॑णा॒ भ्रातृ॑व्यम् । \newline
67. उ॒द्‍गृ॒ह्णातीत्यु॑त् - गृ॒ह्णाति॑ । \newline
68. ब्रह्म॑णा॒ भ्रातृ॑व्य॒म् भ्रातृ॑व्य॒म् ब्रह्म॑णा॒ ब्रह्म॑णा॒ भ्रातृ॑व्य॒म् नि नि भ्रातृ॑व्य॒म् ब्रह्म॑णा॒ ब्रह्म॑णा॒ भ्रातृ॑व्य॒म् नि । \newline
69. भ्रातृ॑व्य॒म् नि नि भ्रातृ॑व्य॒म् भ्रातृ॑व्य॒म् नि गृ॑ह्णाति गृह्णाति॒ नि भ्रातृ॑व्य॒म् भ्रातृ॑व्य॒म् नि गृ॑ह्णाति । \newline
70. नि गृ॑ह्णाति गृह्णाति॒ नि नि गृ॑ह्णाति । \newline
71. गृ॒ह्णा॒तीति॑ गृह्णाति । \newline
\pagebreak
\markright{ TS 5.4.7.1  \hfill https://www.vedavms.in \hfill}

\section{ TS 5.4.7.1 }

\textbf{TS 5.4.7.1 } \newline
\textbf{Samhita Paata} \newline

प्राची॒मनु॑ प्र॒दिशं॒ प्रेहि॑ वि॒द्वानित्या॑ह देवलो॒क-मे॒वैतयो॒पाव॑र्तते॒ क्रम॑द्ध्वम॒ग्निना॒ नाक॒-मित्या॑हे॒माने॒वैतया॑ लो॒कान् क्र॑मते पृथि॒व्या अ॒हमुद॒न्तरि॑क्ष॒मा ऽरु॑ह॒मित्या॑हे॒माने॒वैतया॑ लो॒कान्थ् स॒मारो॑हति॒ सुव॒र्यन्तो॒ नापे᳚क्षन्त॒ इत्या॑ह सुव॒र्गमे॒वैतया॑ लो॒कमे॒त्यग्ने॒ प्रेहि॑ - [  ] \newline

\textbf{Pada Paata} \newline

प्राची᳚म् । अन्विति॑ । प्र॒दिश॒मिति॑ प्र-दिश᳚म् । प्रेति॑ । इ॒हि॒ । वि॒द्वान् । इति॑ । आ॒ह॒ । दे॒व॒लो॒कमिति॑ देव - लो॒कम् । ए॒व । ए॒तया᳚ । उ॒पाव॑र्तत॒ इत्यु॑प - आव॑र्तते । क्रम॑द्ध्वम् । अ॒ग्निना᳚ । नाक᳚म् । इति॑ । आ॒ह॒ । इ॒मान् । ए॒व । ए॒तया᳚ । लो॒कान् । क्र॒म॒ते॒ । पृ॒थि॒व्याः । अ॒हम् । उदिति॑ । अ॒न्तरि॑क्षम् । एति॑ । अ॒रु॒ह॒म् । इति॑ । आ॒ह॒ । इ॒मान् । ए॒व । ए॒तया᳚ । लो॒कान् । स॒मारो॑ह॒तीति॑ सं - आरो॑हति । सुवः॑ । यन्तः॑ । न । अपेति॑ । ई॒क्ष॒न्ते॒ । इति॑ । आ॒ह॒ । सु॒व॒र्गमिति॑ सुवः - गम् । ए॒व । ए॒तया᳚ । लो॒कम् । ए॒ति॒ । अग्ने᳚ । प्रेति॑ । इ॒हि॒ ।  \newline


\textbf{Krama Paata} \newline

प्राची॒मनु॑ । अनु॑ प्र॒दिश᳚म् । प्र॒दिश॒म् प्र । प्र॒दिश॒मिति॑ प्र - दिश᳚म् । प्रेहि॑ । इ॒हि॒ वि॒द्वान् । वि॒द्वानिति॑ । इत्या॑ह । आ॒ह॒ दे॒व॒लो॒कम् । दे॒व॒लो॒कमे॒व । दे॒व॒लो॒कमिति॑ देव - लो॒कम् । ए॒वैतया᳚ । ए॒तयो॒पाव॑र्तते । उ॒पावर्त॑ते॒ क्रम॑द्ध्वम् । उ॒पाव॑र्तत॒ इत्यु॑प - आव॑र्तते । क्रम॑द्ध्वम॒ग्निना᳚ । अ॒ग्निना॒ नाक᳚म् । नाक॒मिति॑ । इत्या॑ह । आ॒हे॒मान् । इ॒माने॒व । ए॒वैतया᳚ । ए॒तया॑ लो॒कान् । लो॒कान् क्र॑मते । क्र॒म॒ते॒ पृ॒थि॒व्याः । पृ॒थि॒व्या अ॒हम् । अ॒हमुत् । उदन्तरि॑क्षम् । अ॒न्तरि॑क्ष॒मा । आऽरु॑हम् । अ॒रु॒ह॒मिति॑ । इत्या॑ह । आ॒हे॒मान् । इ॒माने॒व । ए॒वैतया᳚ । ए॒तया॑ लो॒कान् । लो॒कान्थ् स॒मारो॑हति । स॒मारो॑हति॒ सुवः॑ । स॒मारो॑ह॒तीति॑ सम् - आरो॑हति । सुव॒र् यन्तः॑ । यन्तो॒ न । नाप॑ । अपे᳚क्षन्ते । ई॒क्ष॒न्त॒ इति॑ । इत्या॑ह । आ॒ह॒ सु॒व॒र्गम् । सु॒व॒र्गमे॒व । सु॒व॒र्गमिति॑ सुवः - गम् । ए॒वैतया᳚ । ए॒तया॑ लो॒कम् । लो॒कमे॑ति । ए॒त्यग्ने᳚ । अग्ने॒ प्र । प्रेहि॑ । इ॒हि॒ प्र॒थ॒मः \newline

\textbf{Jatai Paata} \newline

1. प्राची॒ मन्वनु॒ प्राची॒म् प्राची॒ मनु॑ । \newline
2. अनु॑ प्र॒दिश॑म् प्र॒दिश॒ मन्वनु॑ प्र॒दिश᳚म् । \newline
3. प्र॒दिश॒म् प्र प्र प्र॒दिश॑म् प्र॒दिश॒म् प्र । \newline
4. प्र॒दिश॒मिति॑ प्र - दिश᳚म् । \newline
5. प्रेही॑हि॒ प्र प्रेहि॑ । \newline
6. इ॒हि॒ वि॒द्वान्. वि॒द्वा नि॑हीहि वि॒द्वान् । \newline
7. वि॒द्वा नितीति॑ वि॒द्वान्. वि॒द्वा निति॑ । \newline
8. इत्या॑हा॒हे तीत्या॑ह । \newline
9. आ॒ह॒ दे॒व॒लो॒कम् दे॑वलो॒क मा॑हाह देवलो॒कम् । \newline
10. दे॒व॒लो॒क मे॒वैव दे॑वलो॒कम् दे॑वलो॒क मे॒व । \newline
11. दे॒व॒लो॒कमिति॑ देव - लो॒कम् । \newline
12. ए॒वैत यै॒त यै॒वै वैतया᳚ । \newline
13. ए॒तयो॒ पाव॑र्तत उ॒पाव॑र्तत ए॒त यै॒तयो॒ पाव॑र्तते । \newline
14. उ॒पाव॑र्तते॒ क्रम॑द्ध्व॒म् क्रम॑द्ध्व मु॒पाव॑र्तत उ॒पाव॑र्तते॒ क्रम॑द्ध्वम् । \newline
15. उ॒पाव॑र्तत॒ इत्यु॑प - आव॑र्तते । \newline
16. क्रम॑द्ध्व म॒ग्निना॒ ऽग्निना॒ क्रम॑द्ध्व॒म् क्रम॑द्ध्व म॒ग्निना᳚ । \newline
17. अ॒ग्निना॒ नाक॒म् नाक॑ म॒ग्निना॒ ऽग्निना॒ नाक᳚म् । \newline
18. नाक॒ मितीति॒ नाक॒म् नाक॒ मिति॑ । \newline
19. इत्या॑हा॒हे तीत्या॑ह । \newline
20. आ॒हे॒मा नि॒मा ना॑हाहे॒मान् । \newline
21. इ॒मा ने॒वैवेमा नि॒मा ने॒व । \newline
22. ए॒वैत यै॒त यै॒वै वैतया᳚ । \newline
23. ए॒तया॑ लो॒कान् ॅलो॒का ने॒त यै॒तया॑ लो॒कान् । \newline
24. लो॒कान् क्र॑मते क्रमते लो॒कान् ॅलो॒कान् क्र॑मते । \newline
25. क्र॒म॒ते॒ पृ॒थि॒व्याः पृ॑थि॒व्याः क्र॑मते क्रमते पृथि॒व्याः । \newline
26. पृ॒थि॒व्या अ॒ह म॒हम् पृ॑थि॒व्याः पृ॑थि॒व्या अ॒हम् । \newline
27. अ॒ह मुदु द॒ह म॒ह मुत् । \newline
28. उद॒न्तरि॑क्ष म॒न्तरि॑क्ष॒ मुदु द॒न्तरि॑क्षम् । \newline
29. अ॒न्तरि॑क्ष॒ मा ऽन्तरि॑क्ष म॒न्तरि॑क्ष॒ मा । \newline
30. आ ऽरु॑ह मरुह॒ मा ऽरु॑हम् । \newline
31. अ॒रु॒ह॒ मिती त्य॑रुह मरुह॒ मिति॑ । \newline
32. इत्या॑हा॒हे तीत्या॑ह । \newline
33. आ॒हे॒मा नि॒मा ना॑हाहे॒मान् । \newline
34. इ॒मा ने॒वैवेमा नि॒मा ने॒व । \newline
35. ए॒वैत यै॒त यै॒वै वैतया᳚ । \newline
36. ए॒तया॑ लो॒कान् ॅलो॒का ने॒त यै॒तया॑ लो॒कान् । \newline
37. लो॒कान् थ्स॒मारो॑हति स॒मारो॑हति लो॒कान् ॅलो॒कान् थ्स॒मारो॑हति । \newline
38. स॒मारो॑हति॒ सुवः॒ सुवः॑ स॒मारो॑हति स॒मारो॑हति॒ सुवः॑ । \newline
39. स॒मारो॑ह॒तीति॑ सं - आरो॑हति । \newline
40. सुव॒र् यन्तो॒ यन्तः॒ सुवः॒ सुव॒र् यन्तः॑ । \newline
41. यन्तो॒ न न यन्तो॒ यन्तो॒ न । \newline
42. नापाप॒ न नाप॑ । \newline
43. अपे᳚क्षन्त ईक्षन्ते॒ अपापे᳚क्षन्ते । \newline
44. ई॒क्ष॒न्त॒ इतीती᳚ क्षन्त ईक्षन्त॒ इति॑ । \newline
45. इत्या॑हा॒हे तीत्या॑ह । \newline
46. आ॒ह॒ सु॒व॒र्गꣳ सु॑व॒र्ग मा॑हाह सुव॒र्गम् । \newline
47. सु॒व॒र्ग मे॒वैव सु॑व॒र्गꣳ सु॑व॒र्ग मे॒व । \newline
48. सु॒व॒र्गमिति॑ सुवः - गम् । \newline
49. ए॒वैत यै॒त यै॒वै वैतया᳚ । \newline
50. ए॒तया॑ लो॒कम् ॅलो॒क मे॒तयै॒ तया॑ लो॒कम् । \newline
51. लो॒क मे᳚त्येति लो॒कम् ॅलो॒क मे॑ति । \newline
52. ए॒त्यग्ने ऽग्न॑ एत्ये॒ त्यग्ने᳚ । \newline
53. अग्ने॒ प्र प्राग्ने ऽग्ने॒ प्र । \newline
54. प्रेही॑हि॒ प्र प्रेहि॑ । \newline
55. इ॒हि॒ प्र॒थ॒मः प्र॑थ॒म इ॑हीहि प्रथ॒मः । \newline

\textbf{Ghana Paata } \newline

1. प्राची॒ मन्वनु॒ प्राची॒म् प्राची॒ मनु॑ प्र॒दिश॑म् प्र॒दिश॒ मनु॒ प्राची॒म् प्राची॒ मनु॑ प्र॒दिश᳚म् । \newline
2. अनु॑ प्र॒दिश॑म् प्र॒दिश॒ मन्वनु॑ प्र॒दिश॒म् प्र प्र प्र॒दिश॒ मन्वनु॑ प्र॒दिश॒म् प्र । \newline
3. प्र॒दिश॒म् प्र प्र प्र॒दिश॑म् प्र॒दिश॒म् प्रेही॑हि॒ प्र प्र॒दिश॑म् प्र॒दिश॒म् प्रेहि॑ । \newline
4. प्र॒दिश॒मिति॑ प्र - दिश᳚म् । \newline
5. प्रेही॑हि॒ प्र प्रेहि॑ वि॒द्वान्. वि॒द्वा नि॑हि॒ प्र प्रेहि॑ वि॒द्वान् । \newline
6. इ॒हि॒ वि॒द्वान्. वि॒द्वा नि॑हीहि वि॒द्वा नितीति॑ वि॒द्वा नि॑हीहि वि॒द्वा निति॑ । \newline
7. वि॒द्वा नितीति॑ वि॒द्वान्. वि॒द्वा नित्या॑ हा॒हेति॑ वि॒द्वान्. वि॒द्वा नित्या॑ह । \newline
8. इत्या॑ हा॒हे तीत्या॑ह देवलो॒कम् दे॑वलो॒क मा॒हे तीत्या॑ह देवलो॒कम् । \newline
9. आ॒ह॒ दे॒व॒लो॒कम् दे॑वलो॒क मा॑हाह देवलो॒क मे॒वैव दे॑वलो॒क मा॑हाह देवलो॒क मे॒व । \newline
10. दे॒व॒लो॒क मे॒वैव दे॑वलो॒कम् दे॑वलो॒क मे॒वैत यै॒त यै॒व दे॑वलो॒कम् दे॑वलो॒क मे॒वै तया᳚ । \newline
11. दे॒व॒लो॒कमिति॑ देव - लो॒कम् । \newline
12. ए॒वैत यै॒त यै॒वैवै तयो॒ पाव॑र्तत उ॒पाव॑र्तत ए॒त यै॒वैवै तयो॒ पाव॑र्तते । \newline
13. ए॒तयो॒ पाव॑र्तत उ॒पाव॑र्तत ए॒त यै॒त यो॒पाव॑र्तते॒ क्रम॑द्ध्व॒म् क्रम॑द्ध्व मु॒पाव॑र्तत ए॒त यै॒त
यो॒पाव॑र्तते॒ क्रम॑द्ध्वम् । \newline
14. उ॒पाव॑र्तते॒ क्रम॑द्ध्व॒म् क्रम॑द्ध्व मु॒पाव॑र्तत उ॒पाव॑र्तते॒ क्रम॑द्ध्व म॒ग्निना॒ ऽग्निना॒ क्रम॑द्ध्व मु॒पाव॑र्तत उ॒पाव॑र्तते॒ क्रम॑द्ध्व म॒ग्निना᳚ । \newline
15. उ॒पाव॑र्तत॒ इत्यु॑प - आव॑र्तते । \newline
16. क्रम॑द्ध्व म॒ग्निना॒ ऽग्निना॒ क्रम॑द्ध्व॒म् क्रम॑द्ध्व म॒ग्निना॒ नाक॒म् नाक॑ म॒ग्निना॒ क्रम॑द्ध्व॒म् क्रम॑द्ध्व म॒ग्निना॒ नाक᳚म् । \newline
17. अ॒ग्निना॒ नाक॒म् नाक॑ म॒ग्निना॒ ऽग्निना॒ नाक॒ मितीति॒ नाक॑ म॒ग्निना॒ ऽग्निना॒ नाक॒ मिति॑ । \newline
18. नाक॒ मितीति॒ नाक॒म् नाक॒ मित्या॑हा॒हेति॒ नाक॒म् नाक॒ मित्या॑ह । \newline
19. इत्या॑हा॒हे तीत्या॑हे॒मा नि॒मा ना॒हे तीत्या॑हे॒मान् । \newline
20. आ॒हे॒ मा नि॒मा ना॑हाहे॒मा ने॒वैवेमा ना॑हाहे॒ मा ने॒व । \newline
21. इ॒मा ने॒वैवेमा नि॒मा ने॒वैत यै॒त यै॒वेमा नि॒मा ने॒वैतया᳚ । \newline
22. ए॒वै तयै॒ तयै॒ वैवै तया॑ लो॒कान् ॅलो॒का ने॒त यै॒वै वैतया॑ लो॒कान् । \newline
23. ए॒तया॑ लो॒कान् ॅलो॒का ने॒त यै॒तया॑ लो॒कान् क्र॑मते क्रमते लो॒का ने॒त यै॒तया॑ लो॒कान् क्र॑मते । \newline
24. लो॒कान् क्र॑मते क्रमते लो॒कान् ॅलो॒कान् क्र॑मते पृथि॒व्याः पृ॑थि॒व्याः क्र॑मते लो॒कान् ॅलो॒कान् क्र॑मते पृथि॒व्याः । \newline
25. क्र॒म॒ते॒ पृ॒थि॒व्याः पृ॑थि॒व्याः क्र॑मते क्रमते पृथि॒व्या अ॒ह म॒हम् पृ॑थि॒व्याः क्र॑मते क्रमते पृथि॒व्या अ॒हम् । \newline
26. पृ॒थि॒व्या अ॒ह म॒हम् पृ॑थि॒व्याः पृ॑थि॒व्या अ॒ह मुदु द॒हम् पृ॑थि॒व्याः पृ॑थि॒व्या अ॒ह मुत् । \newline
27. अ॒ह मुदु द॒ह म॒ह मुद॒न्तरि॑क्ष म॒न्तरि॑क्ष॒ मुद॒ह म॒ह मुद॒न्तरि॑क्षम् । \newline
28. उद॒न्तरि॑क्ष म॒न्तरि॑क्ष॒ मुदु द॒न्तरि॑क्ष॒ मा ऽन्तरि॑क्ष॒ मुदु द॒न्तरि॑क्ष॒ मा । \newline
29. अ॒न्तरि॑क्ष॒ मा ऽन्तरि॑क्ष म॒न्तरि॑क्ष॒ मा ऽरु॑ह मरुह॒ मा ऽन्तरि॑क्ष म॒न्तरि॑क्ष॒ मा ऽरु॑हम् । \newline
30. आ ऽरु॑ह मरुह॒ मा ऽरु॑ह॒ मिती त्य॑रुह॒ मा ऽरु॑ह॒ मिति॑ । \newline
31. अ॒रु॒ह॒ मिती त्य॑रुह मरुह॒ मित्या॑ हा॒हे त्य॑रुह मरुह॒ मित्या॑ह । \newline
32. इत्या॑हा॒हे तीत्या॑हे॒ मा नि॒मा ना॒हे तीत्या॑हे॒ मान् । \newline
33. आ॒हे॒ मा नि॒मा ना॑हाहे॒ मा ने॒वैवेमा ना॑हाहे॒ मा ने॒व । \newline
34. इ॒मा ने॒वैवेमा नि॒मा ने॒वैत यै॒त यै॒वेमा नि॒मा ने॒वै तया᳚ । \newline
35. ए॒वैत यै॒त यै॒वै वैतया॑ लो॒कान् ॅलो॒का ने॒त यै॒वै वैतया॑ लो॒कान् । \newline
36. ए॒तया॑ लो॒कान् ॅलो॒का ने॒त यै॒तया॑ लो॒कान् थ्स॒मारो॑हति स॒मारो॑हति लो॒का ने॒त यै॒तया॑ लो॒कान् थ्स॒मारो॑हति । \newline
37. लो॒कान् थ्स॒मारो॑हति स॒मारो॑हति लो॒कान् ॅलो॒कान् थ्स॒मारो॑हति॒ सुवः॒ सुवः॑ स॒मारो॑हति लो॒कान् ॅलो॒कान् थ्स॒मारो॑हति॒ सुवः॑ । \newline
38. स॒मारो॑हति॒ सुवः॒ सुवः॑ स॒मारो॑हति स॒मारो॑हति॒ सुव॒र् यन्तो॒ यन्तः॒ सुवः॑ स॒मारो॑हति स॒मारो॑हति॒ सुव॒र् यन्तः॑ । \newline
39. स॒मारो॑ह॒तीति॑ सं - आरो॑हति । \newline
40. सुव॒र् यन्तो॒ यन्तः॒ सुवः॒ सुव॒र् यन्तो॒ न न यन्तः॒ सुवः॒ सुव॒र् यन्तो॒ न । \newline
41. यन्तो॒ न न यन्तो॒ यन्तो॒ नापाप॒ न यन्तो॒ यन्तो॒ नाप॑ । \newline
42. नापाप॒ न नापे᳚क्षन्त ईक्षन्ते॒ अप॒ न नापे᳚क्षन्ते । \newline
43. अपे᳚क्षन्त ईक्षन्ते॒ अपापे᳚ क्षन्त॒ इतीती᳚ क्षन्ते॒ अपापे᳚ क्षन्त॒ इति॑ । \newline
44. ई॒क्ष॒न्त॒ इतीती᳚ क्षन्त ईक्षन्त॒ इत्या॑हा॒हे ती᳚क्षन्त ईक्षन्त॒ इत्या॑ह । \newline
45. इत्या॑हा॒हे तीत्या॑ह सुव॒र्गꣳ सु॑व॒र्ग मा॒हे तीत्या॑ह सुव॒र्गम् । \newline
46. आ॒ह॒ सु॒व॒र्गꣳ सु॑व॒र्ग मा॑हाह सुव॒र्ग मे॒वैव सु॑व॒र्ग मा॑हाह सुव॒र्ग मे॒व । \newline
47. सु॒व॒र्ग मे॒वैव सु॑व॒र्गꣳ सु॑व॒र्ग मे॒वैत यै॒त यै॒व सु॑व॒र्गꣳ सु॑व॒र्ग मे॒वै तया᳚ । \newline
48. सु॒व॒र्गमिति॑ सुवः - गम् । \newline
49. ए॒वैत यै॒त यै॒वै वैतया॑ लो॒कम् ॅलो॒क मे॒त यै॒वै वैतया॑ लो॒कम् । \newline
50. ए॒तया॑ लो॒कम् ॅलो॒क मे॒त यै॒तया॑ लो॒क मे᳚त्येति लो॒क मे॒त यै॒तया॑ लो॒क मे॑ति । \newline
51. लो॒क मे᳚त्येति लो॒कम् ॅलो॒क मे॒त्यग्ने ऽग्न॑ एति लो॒कम् ॅलो॒क मे॒त्यग्ने᳚ । \newline
52. ए॒त्यग्ने ऽग्न॑ एत्ये॒ त्यग्ने॒ प्र प्राग्न॑ एत्ये॒ त्यग्ने॒ प्र । \newline
53. अग्ने॒ प्र प्राग्ने ऽग्ने॒ प्रेही॑हि॒ प्राग्ने ऽग्ने॒ प्रेहि॑ । \newline
54. प्रेही॑हि॒ प्र प्रेहि॑ प्रथ॒मः प्र॑थ॒म इ॑हि॒ प्र प्रेहि॑ प्रथ॒मः । \newline
55. इ॒हि॒ प्र॒थ॒मः प्र॑थ॒म इ॑हीहि प्रथ॒मो दे॑वय॒ताम् दे॑वय॒ताम् प्र॑थ॒म इ॑हीहि प्रथ॒मो दे॑वय॒ताम् । \newline
\pagebreak
\markright{ TS 5.4.7.2  \hfill https://www.vedavms.in \hfill}

\section{ TS 5.4.7.2 }

\textbf{TS 5.4.7.2 } \newline
\textbf{Samhita Paata} \newline

प्रथ॒मो दे॑वय॒ता-मित्या॑हो॒भये᳚ष्वे॒वैतया॑ देवमनु॒ष्येषु॒ चक्षु॑र्दधाति प॒ञ्चभि॒रधि॑ क्रामति॒ पाङ्क्तो॑ य॒ज्ञो यावा॑ने॒व य॒ज्ञ्स्तेन॑ स॒ह सु॑व॒र्गं ॅलो॒कमे॑ति॒ नक्तो॒षासेति॑ पुरोऽनुवा॒क्या॑मन्वा॑ह॒ प्रत्या॒ अग्ने॑ सहस्रा॒क्षेत्या॑ह साह॒स्रः प्र॒जाप॑तिः प्र॒जाप॑ते॒राप्त्यै॒ तस्मै॑ ते विधेम॒ वाजा॑य॒ स्वाहेत्या॒हान्नं॒ ॅवै वाजोऽन्न॑मे॒वाव॑ - [  ] \newline

\textbf{Pada Paata} \newline

प्र॒थ॒मः । दे॒व॒य॒तामिति॑ देव - य॒ताम् । इति॑ । आ॒ह॒ । उ॒भये॑षु । ए॒व । ए॒तया᳚ । दे॒व॒म॒नु॒ष्येष्विति॑ देव - म॒नु॒ष्येषु॑ । चक्षुः॑ । द॒धा॒ति॒ । प॒ञ्चभि॒रिति॑ प॒ञ्च - भिः॒ । अधीति॑ । क्रा॒म॒ति॒ । पाङ्क्तः॑ । य॒ज्ञ्ः । यावान्॑ । ए॒व । य॒ज्ञ्ः । तेन॑ । स॒ह । सु॒व॒र्गमिति॑ सुवः - गम् । लो॒कम् । ए॒ति॒ । नक्तो॒षासा᳚ । इति॑ । पु॒रो॒नु॒वा॒क्या॑मिति॑ पुरः - अ॒नु॒वा॒क्या᳚म् । अन्विति॑ । आ॒ह॒ । प्रत्यै᳚ । अग्ने᳚ । स॒ह॒स्रा॒क्षेति॑ सहस्र - अ॒क्ष॒ । इति॑ । आ॒ह॒ । सा॒ह॒स्रः । प्र॒जाप॑ति॒रिति॑ प्र॒जा - प॒तिः॒ । प्र॒जाप॑ते॒रिति॑ प्र॒जा - प॒तेः॒ । आप्त्यै᳚ । तस्मै᳚ । ते॒ । वि॒धे॒म॒ । वाजा॑य । स्वाहा᳚ । इति॑ । आ॒ह॒ । अन्न᳚म् । वै । वाजः॑ । अन्न᳚म् । ए॒व । अवेति॑ ।  \newline


\textbf{Krama Paata} \newline

प्र॒थ॒मो दे॑वय॒ताम् । दे॒व॒य॒तामिति॑ । दे॒व॒य॒तामिति॑ देव - य॒ताम् । इत्या॑ह । आ॒हो॒भये॑षु । उ॒भये᳚ष्वे॒व । ए॒वैतया᳚ । ए॒तया॑ देवमनु॒ष्येषु॑ । दे॒व॒म॒नु॒ष्येषु॒ चक्षुः॑ । दे॒व॒म॒नु॒ष्येष्विति॑ देव - म॒नु॒ष्येषु॑ । चक्षु॑र् दधाति । द॒धा॒ति॒ प॒ञ्चभिः॑ । प॒ञ्चभि॒रधि॑ । प॒ञ्चभि॒रिति॑ प॒ञ्च - भिः॒ । अधि॑ क्रामति । क्रा॒म॒ति॒ पाङ्क्तः॑ । पाङ्क्तो॑ य॒ज्ञ्ः । य॒ज्ञो यावान्॑ । यावा॑ने॒व । ए॒व य॒ज्ञ्ः । य॒ज्ञ्स्तेन॑ । तेन॑ स॒ह । स॒ह सु॑व॒र्गम् । सु॒व॒र्गम् ॅलो॒कम् । सु॒व॒र्गमिति॑ सुवः - गम् । लो॒कमे॑ति । ए॒ति॒ नक्तो॒षासा᳚ । नक्तो॒षासेति॑ । इति॑ पुरोनुवा॒क्या᳚म् । पु॒रो॒नु॒वा॒क्या॑मनु॑ । पु॒रो॒नु॒वा॒क्या॑मिति॑ पुरः - अ॒नु॒वा॒क्या᳚म् । अन्वा॑ह । आ॒ह॒ प्रत्यै᳚ । प्रत्या॒ अग्ने᳚ । अग्ने॑ सहस्राक्ष । स॒ह॒स्रा॒क्षेति॑ । स॒ह॒स्रा॒क्षेति॑ सहस्र - अ॒क्ष॒ । इत्या॑ह । आ॒ह॒ सा॒ह॒स्रः । सा॒ह॒स्रः प्र॒जाप॑तिः । प्र॒जाप॑तिः प्र॒जाप॑तेः । प्र॒जाप॑ति॒रिति॑ प्र॒जा - प॒तिः॒ । प्र॒जाप॑ते॒राप्त्यै᳚ । प्र॒जाप॑ते॒रिति॑ प्र॒जा - प॒तेः॒ । आप्त्यै॒ तस्मै᳚ । तस्मै॑ ते । ते॒ वि॒धे॒म॒ । वि॒धे॒म॒ वाजा॑य । वाजा॑य॒ स्वाहा᳚ । स्वाहेति॑ । इत्या॑ह । आ॒हान्न᳚म् । अन्न॒म् ॅवै । वै वाजः॑ । वाजोऽन्न᳚म् । अन्न॑मे॒व । ए॒वाव॑ । अव॑ रुन्धे \newline

\textbf{Jatai Paata} \newline

1. प्र॒थ॒मो दे॑वय॒ताम् दे॑वय॒ताम् प्र॑थ॒मः प्र॑थ॒मो दे॑वय॒ताम् । \newline
2. दे॒व॒य॒ता मितीति॑ देवय॒ताम् दे॑वय॒ता मिति॑ । \newline
3. दे॒व॒य॒तामिति॑ देव - य॒ताम् । \newline
4. इत्या॑हा॒हे तीत्या॑ह । \newline
5. आ॒हो॒भये॑षू॒ भये᳚ ष्वाहा हो॒भये॑षु । \newline
6. उ॒भये᳚ ष्वे॒वै वोभये॑ षू॒भये᳚ ष्वे॒व । \newline
7. ए॒वैत यै॒त यै॒वै वैतया᳚ । \newline
8. ए॒तया॑ देवमनु॒ष्येषु॑ देवमनु॒ष्ये ष्वे॒त यै॒तया॑ देवमनु॒ष्येषु॑ । \newline
9. दे॒व॒म॒नु॒ष्येषु॒ चक्षु॒ श्चक्षु॑र् देवमनु॒ष्येषु॑ देवमनु॒ष्येषु॒ चक्षुः॑ । \newline
10. दे॒व॒म॒नु॒ष्येष्विति॑ देव - म॒नु॒ष्येषु॑ । \newline
11. चक्षु॑र् दधाति दधाति॒ चक्षु॒ श्चक्षु॑र् दधाति । \newline
12. द॒धा॒ति॒ प॒ञ्चभिः॑ प॒ञ्चभि॑र् दधाति दधाति प॒ञ्चभिः॑ । \newline
13. प॒ञ्चभि॒ रध्यधि॑ प॒ञ्चभिः॑ प॒ञ्चभि॒ रधि॑ । \newline
14. प॒ञ्चभि॒रिति॑ प॒ञ्च - भिः॒ । \newline
15. अधि॑ क्रामति क्राम॒ त्यध्यधि॑ क्रामति । \newline
16. क्रा॒म॒ति॒ पाङ्क्तः॒ पाङ्क्तः॑ क्रामति क्रामति॒ पाङ्क्तः॑ । \newline
17. पाङ्क्तो॑ य॒ज्ञो य॒ज्ञ्ः पाङ्क्तः॒ पाङ्क्तो॑ य॒ज्ञ्ः । \newline
18. य॒ज्ञो यावा॒न्॒. यावान्॑. य॒ज्ञो य॒ज्ञो यावान्॑ । \newline
19. यावा॑ ने॒वैव यावा॒न्॒. यावा॑ ने॒व । \newline
20. ए॒व य॒ज्ञो य॒ज्ञ् ए॒वैव य॒ज्ञ्ः । \newline
21. य॒ज्ञ् स्तेन॒ तेन॑ य॒ज्ञो य॒ज्ञ् स्तेन॑ । \newline
22. तेन॑ स॒ह स॒ह तेन॒ तेन॑ स॒ह । \newline
23. स॒ह सु॑व॒र्गꣳ सु॑व॒र्गꣳ स॒ह स॒ह सु॑व॒र्गम् । \newline
24. सु॒व॒र्गम् ॅलो॒कम् ॅलो॒कꣳ सु॑व॒र्गꣳ सु॑व॒र्गम् ॅलो॒कम् । \newline
25. सु॒व॒र्गमिति॑ सुवः - गम् । \newline
26. लो॒क मे᳚त्येति लो॒कम् ॅलो॒क मे॑ति । \newline
27. ए॒ति॒ नक्तो॒षासा॒ नक्तो॒षा सै᳚त्येति॒ नक्तो॒षासा᳚ । \newline
28. नक्तो॒षा सेतीति॒ नक्तो॒षासा॒ नक्तो॒षा सेति॑ । \newline
29. इति॑ पुरोनुवा॒क्या᳚म् पुरोनुवा॒क्या॑ मितीति॑ पुरोनुवा॒क्या᳚म् । \newline
30. पु॒रो॒नु॒वा॒क्या॑ मन्वनु॑ पुरोनुवा॒क्या᳚म् पुरोनुवा॒क्या॑ मनु॑ । \newline
31. पु॒रो॒नु॒वा॒क्या॑मिति॑ पुरः - अ॒नु॒वा॒क्या᳚म् । \newline
32. अन्वा॑ हा॒हा न्वन् वा॑ह । \newline
33. आ॒ह॒ प्रत्यै॒ प्रत्या॑ आहाह॒ प्रत्यै᳚ । \newline
34. प्रत्या॒ अग्ने ऽग्ने॒ प्रत्यै॒ प्रत्या॒ अग्ने᳚ । \newline
35. अग्ने॑ सहस्राक्ष सहस्रा॒क्षाग्ने ऽग्ने॑ सहस्राक्ष । \newline
36. स॒ह॒स्रा॒क्षेतीति॑ सहस्राक्ष सहस्रा॒क्षेति॑ । \newline
37. स॒ह॒स्रा॒क्षेति॑ सहस्र - अ॒क्ष॒ । \newline
38. इत्या॑हा॒हे तीत्या॑ह । \newline
39. आ॒ह॒ सा॒ह॒स्रः सा॑ह॒स्र आ॑हाह साह॒स्रः । \newline
40. सा॒ह॒स्रः प्र॒जाप॑तिः प्र॒जाप॑तिः साह॒स्रः सा॑ह॒स्रः प्र॒जाप॑तिः । \newline
41. प्र॒जाप॑तिः प्र॒जाप॑तेः प्र॒जाप॑तेः प्र॒जाप॑तिः प्र॒जाप॑तिः प्र॒जाप॑तेः । \newline
42. प्र॒जाप॑ति॒रिति॑ प्र॒जा - प॒तिः॒ । \newline
43. प्र॒जाप॑ते॒ राप्त्या॒ आप्त्यै᳚ प्र॒जाप॑तेः प्र॒जाप॑ते॒ राप्त्यै᳚ । \newline
44. प्र॒जाप॑ते॒रिति॑ प्र॒जा - प॒तेः॒ । \newline
45. आप्त्यै॒ तस्मै॒ तस्मा॒ आप्त्या॒ आप्त्यै॒ तस्मै᳚ । \newline
46. तस्मै॑ ते ते॒ तस्मै॒ तस्मै॑ ते । \newline
47. ते॒ वि॒धे॒म॒ वि॒धे॒म॒ ते॒ ते॒ वि॒धे॒म॒ । \newline
48. वि॒धे॒म॒ वाजा॑य॒ वाजा॑य विधेम विधेम॒ वाजा॑य । \newline
49. वाजा॑य॒ स्वाहा॒ स्वाहा॒ वाजा॑य॒ वाजा॑य॒ स्वाहा᳚ । \newline
50. स्वाहे तीति॒ स्वाहा॒ स्वाहेति॑ । \newline
51. इत्या॑हा॒हे तीत्या॑ह । \newline
52. आ॒हान्न॒ मन्न॑ माहा॒ हान्न᳚म् । \newline
53. अन्नं॒ ॅवै वा अन्न॒ मन्नं॒ ॅवै । \newline
54. वै वाजो॒ वाजो॒ वै वै वाजः॑ । \newline
55. वाजो ऽन्न॒ मन्नं॒ ॅवाजो॒ वाजो ऽन्न᳚म् । \newline
56. अन्न॑ मे॒वै वान्न॒ मन्न॑ मे॒व । \newline
57. ए॒वावा वै॒वै वाव॑ । \newline
58. अव॑ रुन्धे रु॒न्धे ऽवाव॑ रुन्धे । \newline

\textbf{Ghana Paata } \newline

1. प्र॒थ॒मो दे॑वय॒ताम् दे॑वय॒ताम् प्र॑थ॒मः प्र॑थ॒मो दे॑वय॒ता मितीति॑ देवय॒ताम् प्र॑थ॒मः प्र॑थ॒मो दे॑वय॒ता मिति॑ । \newline
2. दे॒व॒य॒ता मितीति॑ देवय॒ताम् दे॑वय॒ता मित्या॑ हा॒हेति॑ देवय॒ताम् दे॑वय॒ता मित्या॑ह । \newline
3. दे॒व॒य॒तामिति॑ देव - य॒ताम् । \newline
4. इत्या॑हा॒हे तीत्या॑ हो॒भये॑षू॒ भये᳚ ष्वा॒हेती त्या॑हो॒भये॑षु । \newline
5. आ॒हो॒ भये॑षू॒ भये᳚ ष्वाहा हो॒भये᳚ ष्वे॒वै वोभये᳚ ष्वाहा हो॒भये᳚ ष्वे॒व । \newline
6. उ॒भये᳚ ष्वे॒वै वोभये॑षू॒ भये᳚ ष्वे॒वैत यै॒त यै॒वोभये॑षू॒ भये᳚ ष्वे॒वैतया᳚ । \newline
7. ए॒वैत यै॒त यै॒वैवै तया॑ देवमनु॒ष्येषु॑ देवमनु॒ष्ये ष्वे॒त यै॒वै वैतया॑ देवमनु॒ष्येषु॑ । \newline
8. ए॒तया॑ देवमनु॒ष्येषु॑ देवमनु॒ष्ये ष्वे॒त यै॒तया॑ देवमनु॒ष्येषु॒ चक्षु॒ श्चक्षु॑र् देवमनु॒ष्ये ष्वे॒त यै॒तया॑ देवमनु॒ष्येषु॒ चक्षुः॑ । \newline
9. दे॒व॒म॒नु॒ष्येषु॒ चक्षु॒ श्चक्षु॑र् देवमनु॒ष्येषु॑ देवमनु॒ष्येषु॒ चक्षु॑र् दधाति दधाति॒ चक्षु॑र् देवमनु॒ष्येषु॑ देवमनु॒ष्येषु॒ चक्षु॑र् दधाति । \newline
10. दे॒व॒म॒नु॒ष्येष्विति॑ देव - म॒नु॒ष्येषु॑ । \newline
11. चक्षु॑र् दधाति दधाति॒ चक्षु॒ श्चक्षु॑र् दधाति प॒ञ्चभिः॑ प॒ञ्चभि॑र् दधाति॒ चक्षु॒ श्चक्षु॑र् दधाति प॒ञ्चभिः॑ । \newline
12. द॒धा॒ति॒ प॒ञ्चभिः॑ प॒ञ्चभि॑र् दधाति दधाति प॒ञ्चभि॒ रध्यधि॑ प॒ञ्चभि॑र् दधाति दधाति प॒ञ्चभि॒ रधि॑ । \newline
13. प॒ञ्चभि॒ रध्यधि॑ प॒ञ्चभिः॑ प॒ञ्चभि॒ रधि॑ क्रामति क्राम॒ त्यधि॑ प॒ञ्चभिः॑ प॒ञ्चभि॒ रधि॑ क्रामति । \newline
14. प॒ञ्चभि॒रिति॑ प॒ञ्च - भिः॒ । \newline
15. अधि॑ क्रामति क्राम॒ त्यध्यधि॑ क्रामति॒ पाङ्क्तः॒ पाङ्क्तः॑ क्राम॒ त्यध्यधि॑ क्रामति॒ पाङ्क्तः॑ । \newline
16. क्रा॒म॒ति॒ पाङ्क्तः॒ पाङ्क्तः॑ क्रामति क्रामति॒ पाङ्क्तो॑ य॒ज्ञो य॒ज्ञ्ः पाङ्क्तः॑ क्रामति क्रामति॒ पाङ्क्तो॑ य॒ज्ञ्ः । \newline
17. पाङ्क्तो॑ य॒ज्ञो य॒ज्ञ्ः पाङ्क्तः॒ पाङ्क्तो॑ य॒ज्ञो यावा॒न्॒. यावान्॑. य॒ज्ञ्ः पाङ्क्तः॒ पाङ्क्तो॑ य॒ज्ञो यावान्॑ । \newline
18. य॒ज्ञो यावा॒न्॒. यावान्॑. य॒ज्ञो य॒ज्ञो यावा॑ ने॒वैव यावान्॑. य॒ज्ञो य॒ज्ञो यावा॑ ने॒व । \newline
19. यावा॑ ने॒वैव यावा॒न्॒. यावा॑ ने॒व य॒ज्ञो य॒ज्ञ् ए॒व यावा॒न्॒. यावा॑ ने॒व य॒ज्ञ्ः । \newline
20. ए॒व य॒ज्ञो य॒ज्ञ् ए॒वैव य॒ज्ञ् स्तेन॒ तेन॑ य॒ज्ञ् ए॒वैव य॒ज्ञ् स्तेन॑ । \newline
21. य॒ज्ञ् स्तेन॒ तेन॑ य॒ज्ञो य॒ज्ञ् स्तेन॑ स॒ह स॒ह तेन॑ य॒ज्ञो य॒ज्ञ् स्तेन॑ स॒ह । \newline
22. तेन॑ स॒ह स॒ह तेन॒ तेन॑ स॒ह सु॑व॒र्गꣳ सु॑व॒र्गꣳ स॒ह तेन॒ तेन॑ स॒ह सु॑व॒र्गम् । \newline
23. स॒ह सु॑व॒र्गꣳ सु॑व॒र्गꣳ स॒ह स॒ह सु॑व॒र्गम् ॅलो॒कम् ॅलो॒कꣳ सु॑व॒र्गꣳ स॒ह स॒ह सु॑व॒र्गम् ॅलो॒कम् । \newline
24. सु॒व॒र्गम् ॅलो॒कम् ॅलो॒कꣳ सु॑व॒र्गꣳ सु॑व॒र्गम् ॅलो॒क मे᳚त्येति लो॒कꣳ सु॑व॒र्गꣳ सु॑व॒र्गम् ॅलो॒क मे॑ति । \newline
25. सु॒व॒र्गमिति॑ सुवः - गम् । \newline
26. लो॒क मे᳚त्येति लो॒कम् ॅलो॒क मे॑ति॒ नक्तो॒षासा॒ नक्तो॒षासै॑ति लो॒कम् ॅलो॒क मे॑ति॒ नक्तो॒षासा᳚ । \newline
27. ए॒ति॒ नक्तो॒षासा॒ नक्तो॒षा सै᳚त्येति॒ नक्तो॒षा सेतीति॒ नक्तो॒षा सै᳚त्येति॒ नक्तो॒षासेति॑ । \newline
28. नक्तो॒षा सेतीति॒ नक्तो॒षासा॒ नक्तो॒षासेति॑ पुरोनुवा॒क्या᳚म् पुरोनुवा॒क्या॑ मिति॒ नक्तो॒षासा॒ नक्तो॒षासेति॑ पुरोनुवा॒क्या᳚म् । \newline
29. इति॑ पुरोनुवा॒क्या᳚म् पुरोनुवा॒क्या॑ मितीति॑ पुरोनुवा॒क्या॑ मन्वनु॑ पुरोनुवा॒क्या॑ मितीति॑ पुरोनुवा॒क्या॑ मनु॑ । \newline
30. पु॒रो॒नु॒वा॒क्या॑ मन्वनु॑ पुरोनुवा॒क्या᳚म् पुरोनुवा॒क्या॑ मन्वा॑हा॒ हानु॑ पुरोनुवा॒क्या᳚म् पुरोनुवा॒क्या॑ मन्वा॑ह । \newline
31. पु॒रो॒नु॒वा॒क्या॑मिति॑ पुरः - अ॒नु॒वा॒क्या᳚म् । \newline
32. अन्वा॑हा॒ हान्वन् वा॑ह॒ प्रत्यै॒ प्रत्या॑ आ॒हान्वन् वा॑ह॒ प्रत्यै᳚ । \newline
33. आ॒ह॒ प्रत्यै॒ प्रत्या॑ आहाह॒ प्रत्या॒ अग्ने ऽग्ने॒ प्रत्या॑ आहाह॒ प्रत्या॒ अग्ने᳚ । \newline
34. प्रत्या॒ अग्ने ऽग्ने॒ प्रत्यै॒ प्रत्या॒ अग्ने॑ सहस्राक्ष सहस्रा॒क्षाग्ने॒ प्रत्यै॒ प्रत्या॒ अग्ने॑ सहस्राक्ष । \newline
35. अग्ने॑ सहस्राक्ष सहस्रा॒क्षाग्ने ऽग्ने॑ सहस्रा॒क्षे तीति॑ सहस्रा॒क्षाग्ने ऽग्ने॑ सहस्रा॒क्षेति॑ । \newline
36. स॒ह॒स्रा॒क्षे तीति॑ सहस्राक्ष सहस्रा॒क्षे त्या॑हा॒हेति॑ सहस्राक्ष सहस्रा॒क्षे त्या॑ह । \newline
37. स॒ह॒स्रा॒क्षेति॑ सहस्र - अ॒क्ष॒ । \newline
38. इत्या॑हा॒हे तीत्या॑ह साह॒स्रः सा॑ह॒स्र आ॒हे तीत्या॑ह साह॒स्रः । \newline
39. आ॒ह॒ सा॒ह॒स्रः सा॑ह॒स्र आ॑हाह साह॒स्रः प्र॒जाप॑तिः प्र॒जाप॑तिः साह॒स्र आ॑हाह साह॒स्रः प्र॒जाप॑तिः । \newline
40. सा॒ह॒स्रः प्र॒जाप॑तिः प्र॒जाप॑तिः साह॒स्रः सा॑ह॒स्रः प्र॒जाप॑तिः प्र॒जाप॑तेः प्र॒जाप॑तेः प्र॒जाप॑तिः साह॒स्रः सा॑ह॒स्रः प्र॒जाप॑तिः प्र॒जाप॑तेः । \newline
41. प्र॒जाप॑तिः प्र॒जाप॑तेः प्र॒जाप॑तेः प्र॒जाप॑तिः प्र॒जाप॑तिः प्र॒जाप॑ते॒ राप्त्या॒ आप्त्यै᳚ प्र॒जाप॑तेः प्र॒जाप॑तिः प्र॒जाप॑तिः प्र॒जाप॑ते॒ राप्त्यै᳚ । \newline
42. प्र॒जाप॑ति॒रिति॑ प्र॒जा - प॒तिः॒ । \newline
43. प्र॒जाप॑ते॒ राप्त्या॒ आप्त्यै᳚ प्र॒जाप॑तेः प्र॒जाप॑ते॒ राप्त्यै॒ तस्मै॒ तस्मा॒ आप्त्यै᳚ प्र॒जाप॑तेः प्र॒जाप॑ते॒ राप्त्यै॒ तस्मै᳚ । \newline
44. प्र॒जाप॑ते॒रिति॑ प्र॒जा - प॒तेः॒ । \newline
45. आप्त्यै॒ तस्मै॒ तस्मा॒ आप्त्या॒ आप्त्यै॒ तस्मै॑ ते ते॒ तस्मा॒ आप्त्या॒ आप्त्यै॒ तस्मै॑ ते । \newline
46. तस्मै॑ ते ते॒ तस्मै॒ तस्मै॑ ते विधेम विधेम ते॒ तस्मै॒ तस्मै॑ ते विधेम । \newline
47. ते॒ वि॒धे॒म॒ वि॒धे॒म॒ ते॒ ते॒ वि॒धे॒म॒ वाजा॑य॒ वाजा॑य विधेम ते ते विधेम॒ वाजा॑य । \newline
48. वि॒धे॒म॒ वाजा॑य॒ वाजा॑य विधेम विधेम॒ वाजा॑य॒ स्वाहा॒ स्वाहा॒ वाजा॑य विधेम विधेम॒ वाजा॑य॒ स्वाहा᳚ । \newline
49. वाजा॑य॒ स्वाहा॒ स्वाहा॒ वाजा॑य॒ वाजा॑य॒ स्वाहे तीति॒ स्वाहा॒ वाजा॑य॒ वाजा॑य॒ स्वाहेति॑ । \newline
50. स्वाहे तीति॒ स्वाहा॒ स्वाहे त्या॑हा॒हेति॒ स्वाहा॒ स्वाहे त्या॑ह । \newline
51. इत्या॑हा॒हेती त्या॒हान्न॒ मन्न॑ मा॒हेती त्या॒हान्न᳚म् । \newline
52. आ॒हान्न॒ मन्न॑ माहा॒ हान्नं॒ ॅवै वा अन्न॑ माहा॒ हान्नं॒ ॅवै । \newline
53. अन्नं॒ ॅवै वा अन्न॒ मन्नं॒ ॅवै वाजो॒ वाजो॒ वा अन्न॒ मन्नं॒ ॅवै वाजः॑ । \newline
54. वै वाजो॒ वाजो॒ वै वै वाजो ऽन्न॒ मन्नं॒ ॅवाजो॒ वै वै वाजो ऽन्न᳚म् । \newline
55. वाजो ऽन्न॒ मन्नं॒ ॅवाजो॒ वाजो ऽन्न॑ मे॒वै वान्नं॒ ॅवाजो॒ वाजो ऽन्न॑ मे॒व । \newline
56. अन्न॑ मे॒वै वान्न॒ मन्न॑ मे॒वा वावै॒ वान्न॒ मन्न॑ मे॒वाव॑ । \newline
57. ए॒वावा वै॒वै वाव॑ रुन्धे रु॒न्धे ऽवै॒वै वाव॑ रुन्धे । \newline
58. अव॑ रुन्धे रु॒न्धे ऽवाव॑ रुन्धे द॒द्ध्नो द॒द्ध्नो रु॒न्धे ऽवाव॑ रुन्धे द॒द्ध्नः । \newline
\pagebreak
\markright{ TS 5.4.7.3  \hfill https://www.vedavms.in \hfill}

\section{ TS 5.4.7.3 }

\textbf{TS 5.4.7.3 } \newline
\textbf{Samhita Paata} \newline

रुन्धे द॒द्ध्नः पू॒र्णामौदु॑म्बरीꣳ स्वयमातृ॒ण्णायां᳚ जुहो॒त्यूर्ग्वै दद्ध्यूर्गु॑दु॒म्बरो॒ऽसौ स्व॑यमातृ॒ण्णा ऽमुष्या॑मे॒वोर्जं॑ दधाति॒ तस्मा॑द॒मुतो॒ऽर्वाची॒मूर्ज॒मुप॑ जीवामस्ति॒सृभिः॑ सादयति त्रि॒वृद्वा अ॒ग्निर्यावा॑ने॒वाग्निस्तं प्र॑ति॒ष्ठां ग॑मयति॒ प्रेद्धो॑ अग्ने दीदिहि पु॒रो न॒ इत्यौदु॑म्बरी॒मा द॑धात्ये॒षा वै सू॒र्मी कर्ण॑कावत्ये॒तया॑ ह स्म॒ - [  ] \newline

\textbf{Pada Paata} \newline

रु॒न्धे॒ । द॒द्ध्नः । पू॒र्णाम् । औदु॑बंरीम् । स्व॒य॒मा॒तृ॒ण्णाया॒मिति॑ स्वयं - आ॒तृ॒ण्णाया᳚म् । जु॒हो॒ति॒ । ऊर्क् । वै । दधि॑ । ऊर्क् । उ॒दु॒बंरः॑ । अ॒सौ । स्व॒य॒मा॒तृ॒ण्णेति॑ स्वयं - आ॒तृ॒ण्णा । अ॒मुष्या᳚म् । ए॒व । ऊर्ज᳚म् । द॒धा॒ति॒ । तस्मा᳚त् । अ॒मुतः॑ । अ॒र्वाची᳚म् । ऊर्ज᳚म् । उपेति॑ । जी॒वा॒मः॒ । ति॒सृभि॒रिति॑ ति॒सृ - भिः॒। सा॒द॒य॒ति॒ । त्रि॒वृदिति॑ त्रि - वृत् । वै । अ॒ग्निः । यावान्॑ । ए॒व । अ॒ग्निः । तम् । प्र॒ति॒ष्ठामिति॑ प्रति-स्थाम् । ग॒म॒य॒ति॒ । प्रेद्ध॒ इति॒ प्र - इ॒द्धः॒ । अ॒ग्ने॒ । दी॒दि॒हि॒ । पु॒रः । नः॒ । इति॑ । औदु॑बंरीम् । एति॑ । द॒धा॒ति॒ । ए॒षा । वै । सू॒र्मी । कर्ण॑काव॒तीति॒ कर्ण॑क - व॒ति॒ । ए॒तया᳚ । ह॒ । स्म॒ ।  \newline


\textbf{Krama Paata} \newline

रु॒न्धे॒ द॒द्ध्नः । द॒द्ध्नः पू॒र्णाम् । पू॒र्णामौदु॑म्बरीम् । औदु॑म्बरीꣳ स्वयमातृ॒ण्णाया᳚म् । स्व॒य॒मा॒तृ॒ण्णाया᳚म् जुहोति । स्व॒य॒मा॒तृ॒ण्णाया॒मिति॑ स्वयम् - आ॒तृ॒ण्णाया᳚म् । जु॒हो॒त्यूर्क् । ऊर्ग् वै । वै दधि॑ । दद्ध्यूर्क् । ऊर्गु॑दु॒म्बरः॑ । उ॒दु॒म्बरो॒ऽसौ । अ॒सौ स्व॑यमातृ॒ण्णा । स्व॒य॒मा॒तृ॒ण्णाऽमुष्या᳚म् । स्व॒य॒मा॒तृ॒ण्णेति॑ स्वयम् - आ॒तृ॒ण्णा । अ॒मुष्या॑मे॒व । ए॒वोर्ज᳚म् । ऊर्ज॑म् दधाति । द॒धा॒ति॒ तस्मा᳚त् । तस्मा॑द॒मुतः॑ । अ॒मुतो॒ऽर्वाची᳚म् । अ॒र्वाची॒मूर्ज᳚म् । ऊर्ज॒मुप॑ । उप॑ जीवामः । जी॒वा॒म॒स्ति॒सृभिः॑ । ति॒सृभिः॑ सादयति । ति॒सृभि॒रिति॑ ति॒सृ - भिः॒ । सा॒द॒य॒ति॒ त्रि॒वृत् । त्रि॒वृद् वै । त्रि॒वृदिति॑ त्रि - वृत् । वा अ॒ग्निः । अ॒ग्निर् यावान्॑ । यावा॑ने॒व । ए॒वाग्निः । अ॒ग्निस्तम् । तम् प्र॑ति॒ष्ठाम् । प्र॒ति॒ष्ठाम् ग॑मयति । प्र॒ति॒ष्ठामिति॑ प्रति - स्थाम् । ग॒म॒य॒ति॒ प्रेद्धः॑ । प्रेद्धो॑ अग्ने । प्रेद्ध॒ इति॒ प्र - इ॒द्धः॒ । अ॒ग्ने॒ दी॒दि॒हि॒ । दी॒दि॒हि॒ पु॒रः । पु॒रो नः॑ । न॒ इति॑ । इत्यौदु॑म्बरीम् । औदु॑म्बरी॒मा । आ द॑धाति । द॒धा॒त्ये॒षा । ए॒षा वै । वै सू॒र्मी । सू॒र्मी कर्ण॑कावती । कर्ण॑कावत्ये॒तया᳚ । कर्ण॑काव॒तीति॒ कर्ण॑क - व॒ती॒ । ए॒तया॑ ह । ह॒ स्म॒ । स्म॒ वै \newline

\textbf{Jatai Paata} \newline

1. रु॒न्धे॒ द॒द्ध्नो द॒द्ध्नो रु॑न्धे रुन्धे द॒द्ध्नः । \newline
2. द॒द्ध्नः पू॒र्णाम् पू॒र्णाम् द॒द्ध्नो द॒द्ध्नः पू॒र्णाम् । \newline
3. पू॒र्णा मौदुं॑बरी॒ मौदुं॑बरीम् पू॒र्णाम् पू॒र्णा मौदुं॑बरीम् । \newline
4. औदुं॑बरीꣳ स्वयमातृ॒ण्णायाꣳ॑ स्वयमातृ॒ण्णाया॒ मौदुं॑बरी॒ मौदुं॑बरीꣳ स्वयमातृ॒ण्णाया᳚म् । \newline
5. स्व॒य॒मा॒तृ॒ण्णाया᳚म् जुहोति जुहोति स्वयमातृ॒ण्णायाꣳ॑ स्वयमातृ॒ण्णाया᳚म् जुहोति । \newline
6. स्व॒य॒मा॒तृ॒ण्णाया॒मिति॑ स्वयं - आ॒तृ॒ण्णाया᳚म् । \newline
7. जु॒हो॒ त्यूर् गूर्ग् जु॑होति जुहो॒ त्यूर्क् । \newline
8. ऊर्ग् वै वा ऊर् गूर्ग् वै । \newline
9. वै दधि॒ दधि॒ वै वै दधि॑ । \newline
10. दध्यूर् गूर्ग् दधि॒ दध्यूर्क् । \newline
11. ऊर्गु॑दुं॒बर॑ उदुं॒बर॒ ऊर् गूर् गु॑दुं॒बरः॑ । \newline
12. उ॒दुं॒बरो॒ ऽसा व॒सा वु॑दुं॒बर॑ उदुं॒बरो॒ ऽसौ । \newline
13. अ॒सौ स्व॑यमातृ॒ण्णा स्व॑यमातृ॒ण्णा ऽसा व॒सौ स्व॑यमातृ॒ण्णा । \newline
14. स्व॒य॒मा॒तृ॒ण्णा ऽमुष्या॑ म॒मुष्याꣳ॑ स्वयमातृ॒ण्णा स्व॑यमातृ॒ण्णा ऽमुष्या᳚म् । \newline
15. स्व॒य॒मा॒तृ॒ण्णेति॑ स्वयं - आ॒तृ॒ण्णा । \newline
16. अ॒मुष्या॑ मे॒वैवा मुष्या॑ म॒मुष्या॑ मे॒व । \newline
17. ए॒वोर्ज॒ मूर्ज॑ मे॒वै वोर्ज᳚म् । \newline
18. ऊर्ज॑म् दधाति दधा॒ त्यूर्ज॒ मूर्ज॑म् दधाति । \newline
19. द॒धा॒ति॒ तस्मा॒त् तस्मा᳚द् दधाति दधाति॒ तस्मा᳚त् । \newline
20. तस्मा॑द॒मुतो॒ ऽमुत॒ स्तस्मा॒त् तस्मा॑ द॒मुतः॑ । \newline
21. अ॒मुतो॒ ऽर्वाची॑ म॒र्वाची॑ म॒मुतो॒ ऽमुतो॒ ऽर्वाची᳚म् । \newline
22. अ॒र्वाची॒ मूर्ज॒ मूर्ज॑ म॒र्वाची॑ म॒र्वाची॒ मूर्ज᳚म् । \newline
23. ऊर्ज॒ मुपो पोर्ज॒ मूर्ज॒ मुप॑ । \newline
24. उप॑ जीवामो जीवाम॒ उपोप॑ जीवामः । \newline
25. जी॒वा॒म॒ स्ति॒सृभि॑ स्ति॒सृभि॑र् जीवामो जीवाम स्ति॒सृभिः॑ । \newline
26. ति॒सृभिः॑ सादयति सादयति ति॒सृभि॑ स्ति॒सृभिः॑ सादयति । \newline
27. ति॒सृभि॒रिति॑ ति॒सृ - भिः॒ । \newline
28. सा॒द॒य॒ति॒ त्रि॒वृत् त्रि॒वृथ् सा॑दयति सादयति त्रि॒वृत् । \newline
29. त्रि॒वृद् वै वै त्रि॒वृत् त्रि॒वृद् वै । \newline
30. त्रि॒वृदिति॑ त्रि - वृत् । \newline
31. वा अ॒ग्नि र॒ग्निर् वै वा अ॒ग्निः । \newline
32. अ॒ग्निर् यावा॒न्॒. यावा॑ न॒ग्नि र॒ग्निर् यावान्॑ । \newline
33. यावा॑ ने॒वैव यावा॒न्॒. यावा॑ ने॒व । \newline
34. ए॒वाग्नि र॒ग्नि रे॒वै वाग्निः । \newline
35. अ॒ग्नि स्तम् त म॒ग्नि र॒ग्नि स्तम् । \newline
36. तम् प्र॑ति॒ष्ठाम् प्र॑ति॒ष्ठाम् तम् तम् प्र॑ति॒ष्ठाम् । \newline
37. प्र॒ति॒ष्ठाम् ग॑मयति गमयति प्रति॒ष्ठाम् प्र॑ति॒ष्ठाम् ग॑मयति । \newline
38. प्र॒ति॒ष्ठामिति॑ प्रति - स्थाम् । \newline
39. ग॒म॒य॒ति॒ प्रेद्धः॒ प्रेद्धो॑ गमयति गमयति॒ प्रेद्धः॑ । \newline
40. प्रेद्धो॑ अग्ने अग्ने॒ प्रेद्धः॒ प्रेद्धो॑ अग्ने । \newline
41. प्रेद्ध॒ इति॒ प्र - इ॒द्धः॒ । \newline
42. अ॒ग्ने॒ दी॒दि॒हि॒ दी॒दि॒ह्य॒ग्ने॒ अ॒ग्ने॒ दी॒दि॒हि॒ । \newline
43. दी॒दि॒हि॒ पु॒रः पु॒रो दी॑दिहि दीदिहि पु॒रः । \newline
44. पु॒रो नो॑ नः पु॒रः पु॒रो नः॑ । \newline
45. न॒ इतीति॑ नो न॒ इति॑ । \newline
46. इत्यौदुं॑बरी॒ मौदुं॑बरी॒ मिती त्यौदुं॑बरीम् । \newline
47. औदुं॑बरी॒ मौदुं॑बरी॒ मौदुं॑बरी॒ मा । \newline
48. आ द॑धाति दधा॒त्या द॑धाति । \newline
49. द॒धा॒ त्ये॒षैषा द॑धाति दधा त्ये॒षा । \newline
50. ए॒षा वै वा ए॒षैषा वै । \newline
51. वै सू॒र्मी सू॒र्मी वै वै सू॒र्मी । \newline
52. सू॒र्मी कर्ण॑कावती॒ कर्ण॑कावती सू॒र्मी सू॒र्मी कर्ण॑कावती । \newline
53. कर्ण॑काव त्ये॒त यै॒तया॒ कर्ण॑कावती॒ कर्ण॑काव त्ये॒तया᳚ । \newline
54. कर्ण॑काव॒तीति॒ कर्ण॑क - व॒ती॒ । \newline
55. ए॒तया॑ ह है॒त यै॒तया॑ ह । \newline
56. ह॒ स्म॒ स्म॒ ह॒ ह॒ स्म॒ । \newline
57. स्म॒ वै वै स्म॑ स्म॒ वै । \newline

\textbf{Ghana Paata } \newline

1. रु॒न्धे॒ द॒द्ध्नो द॒द्ध्नो रु॑न्धे रुन्धे द॒द्ध्नः पू॒र्णाम् पू॒र्णाम् द॒द्ध्नो रु॑न्धे रुन्धे द॒द्ध्नः पू॒र्णाम् । \newline
2. द॒द्ध्नः पू॒र्णाम् पू॒र्णाम् द॒द्ध्नो द॒द्ध्नः पू॒र्णा मौदुं॑बरी॒ मौदुं॑बरीम् पू॒र्णाम् द॒द्ध्नो द॒द्ध्नः पू॒र्णा मौदुं॑बरीम् । \newline
3. पू॒र्णा मौदुं॑बरी॒ मौदुं॑बरीम् पू॒र्णाम् पू॒र्णा मौदुं॑बरीꣳ स्वयमातृ॒ण्णायाꣳ॑ स्वयमातृ॒ण्णाया॒ मौदुं॑बरीम् पू॒र्णाम् पू॒र्णा मौदुं॑बरीꣳ स्वयमातृ॒ण्णाया᳚म् । \newline
4. औदुं॑बरीꣳ स्वयमातृ॒ण्णायाꣳ॑ स्वयमातृ॒ण्णाया॒ मौदुं॑बरी॒ मौदुं॑बरीꣳ स्वयमातृ॒ण्णाया᳚म् जुहोति जुहोति स्वयमातृ॒ण्णाया॒ मौदुं॑बरी॒ मौदुं॑बरीꣳ स्वयमातृ॒ण्णाया᳚म् जुहोति । \newline
5. स्व॒य॒मा॒तृ॒ण्णाया᳚म् जुहोति जुहोति स्वयमातृ॒ण्णायाꣳ॑ स्वयमातृ॒ण्णाया᳚म् जुहो॒ त्यूर् गूर्ग् जु॑होति स्वयमातृ॒ण्णायाꣳ॑ स्वयमातृ॒ण्णाया᳚म् जुहो॒ त्यूर्क् । \newline
6. स्व॒य॒मा॒तृ॒ण्णाया॒मिति॑ स्वयं - आ॒तृ॒ण्णाया᳚म् । \newline
7. जु॒हो॒ त्यूर् गूर्ग् जु॑होति जुहो॒ त्यूर्ग् वै वा ऊर्ग् जु॑होति जुहो॒ त्यूर्ग् वै । \newline
8. ऊर्ग् वै वा ऊर्गूर्ग् वै दधि॒ दधि॒ वा ऊर्गूर्ग् वै दधि॑ । \newline
9. वै दधि॒ दधि॒ वै वै दध्यूर् गूर्ग् दधि॒ वै वै दध्यूर्क् । \newline
10. दध्यूर् गूर्ग् दधि॒ दध्यूर् गु॑दुं॒बर॑ उदुं॒बर॒ ऊर्ग् दधि॒ दध्यूर् गु॑दुं॒बरः॑ । \newline
11. ऊर्गु॑दुं॒बर॑ उदुं॒बर॒ ऊर्गूर् गु॑दुं॒बरो॒ ऽसा व॒सा वु॑दुं॒बर॒ ऊर्गूर् गु॑दुं॒बरो॒ ऽसौ । \newline
12. उ॒दुं॒बरो॒ ऽसा व॒सा वु॑दुं॒बर॑ उदुं॒बरो॒ ऽसौ स्व॑यमातृ॒ण्णा स्व॑यमातृ॒ण्णा ऽसा वु॑दुं॒बर॑ उदुं॒बरो॒ ऽसौ स्व॑यमातृ॒ण्णा । \newline
13. अ॒सौ स्व॑यमातृ॒ण्णा स्व॑यमातृ॒ण्णा ऽसा व॒सौ स्व॑यमातृ॒ण्णा ऽमुष्या॑ म॒मुष्याꣳ॑ स्वयमातृ॒ण्णा ऽसा व॒सौ स्व॑यमातृ॒ण्णा ऽमुष्या᳚म् । \newline
14. स्व॒य॒मा॒तृ॒ण्णा ऽमुष्या॑ म॒मुष्याꣳ॑ स्वयमातृ॒ण्णा स्व॑यमातृ॒ण्णा ऽमुष्या॑ मे॒वैवामुष्याꣳ॑ स्वयमातृ॒ण्णा स्व॑यमातृ॒ण्णा ऽमुष्या॑ मे॒व । \newline
15. स्व॒य॒मा॒तृ॒ण्णेति॑ स्वयं - आ॒तृ॒ण्णा । \newline
16. अ॒मुष्या॑ मे॒वैवा मुष्या॑ म॒मुष्या॑ मे॒वोर्ज॒ मूर्ज॑ मे॒वा मुष्या॑ म॒मुष्या॑ मे॒वोर्ज᳚म् । \newline
17. ए॒वोर्ज॒ मूर्ज॑ मे॒वै वोर्ज॑म् दधाति दधा॒ त्यूर्ज॑ मे॒वै वोर्ज॑म् दधाति । \newline
18. ऊर्ज॑म् दधाति दधा॒ त्यूर्ज॒ मूर्ज॑म् दधाति॒ तस्मा॒त् तस्मा᳚द् दधा॒ त्यूर्ज॒ मूर्ज॑म् दधाति॒ तस्मा᳚त् । \newline
19. द॒धा॒ति॒ तस्मा॒त् तस्मा᳚द् दधाति दधाति॒ तस्मा॑ द॒मुतो॒ ऽमुत॒ स्तस्मा᳚द् दधाति दधाति॒ तस्मा॑ द॒मुतः॑ । \newline
20. तस्मा॑ द॒मुतो॒ ऽमुत॒ स्तस्मा॒त् तस्मा॑ द॒मुतो॒ ऽर्वाची॑ म॒र्वाची॑ म॒मुत॒ स्तस्मा॒त् तस्मा॑ द॒मुतो॒ ऽर्वाची᳚म् । \newline
21. अ॒मुतो॒ ऽर्वाची॑ म॒र्वाची॑ म॒मुतो॒ ऽमुतो॒ ऽर्वाची॒ मूर्ज॒ मूर्ज॑ म॒र्वाची॑ म॒मुतो॒ ऽमुतो॒ ऽर्वाची॒ मूर्ज᳚म् । \newline
22. अ॒र्वाची॒ मूर्ज॒ मूर्ज॑ म॒र्वाची॑ म॒र्वाची॒ मूर्ज॒ मुपोपोर्ज॑ म॒र्वाची॑ म॒र्वाची॒ मूर्ज॒ मुप॑ । \newline
23. ऊर्ज॒ मुपोपोर्ज॒ मूर्ज॒ मुप॑ जीवामो जीवाम॒ उपोर्ज॒ मूर्ज॒ मुप॑ जीवामः । \newline
24. उप॑ जीवामो जीवाम॒ उपोप॑ जीवाम स्ति॒सृभि॑ स्ति॒सृभि॑र् जीवाम॒ उपोप॑ जीवाम स्ति॒सृभिः॑ । \newline
25. जी॒वा॒म॒ स्ति॒सृभि॑ स्ति॒सृभि॑र् जीवामो जीवाम स्ति॒सृभिः॑ सादयति सादयति ति॒सृभि॑र् जीवामो जीवाम स्ति॒सृभिः॑ सादयति । \newline
26. ति॒सृभिः॑ सादयति सादयति ति॒सृभि॑ स्ति॒सृभिः॑ सादयति त्रि॒वृत् त्रि॒वृथ् सा॑दयति ति॒सृभि॑ स्ति॒सृभिः॑ सादयति त्रि॒वृत् । \newline
27. ति॒सृभि॒रिति॑ ति॒सृ - भिः॒ । \newline
28. सा॒द॒य॒ति॒ त्रि॒वृत् त्रि॒वृथ् सा॑दयति सादयति त्रि॒वृद् वै वै त्रि॒वृथ् सा॑दयति सादयति त्रि॒वृद् वै । \newline
29. त्रि॒वृद् वै वै त्रि॒वृत् त्रि॒वृद् वा अ॒ग्नि र॒ग्निर् वै त्रि॒वृत् त्रि॒वृद् वा अ॒ग्निः । \newline
30. त्रि॒वृदिति॑ त्रि - वृत् । \newline
31. वा अ॒ग्नि र॒ग्निर् वै वा अ॒ग्निर् यावा॒न्॒. यावा॑ न॒ग्निर् वै वा अ॒ग्निर् यावान्॑ । \newline
32. अ॒ग्निर् यावा॒न्॒. यावा॑ न॒ग्नि र॒ग्निर् यावा॑ ने॒वैव यावा॑ न॒ग्नि र॒ग्निर् यावा॑ ने॒व । \newline
33. यावा॑ ने॒वैव यावा॒न्॒. यावा॑ ने॒वाग्नि र॒ग्नि रे॒व यावा॒न्॒. यावा॑ ने॒वाग्निः । \newline
34. ए॒वाग्नि र॒ग्नि रे॒वै वाग्नि स्तम् त म॒ग्नि रे॒वै वाग्नि स्तम् । \newline
35. अ॒ग्निस्तम् त म॒ग्नि र॒ग्नि स्तम् प्र॑ति॒ष्ठाम् प्र॑ति॒ष्ठाम् त म॒ग्नि र॒ग्नि स्तम् प्र॑ति॒ष्ठाम् । \newline
36. तम् प्र॑ति॒ष्ठाम् प्र॑ति॒ष्ठाम् तम् तम् प्र॑ति॒ष्ठाम् ग॑मयति गमयति प्रति॒ष्ठाम् तम् तम् प्र॑ति॒ष्ठाम् ग॑मयति । \newline
37. प्र॒ति॒ष्ठाम् ग॑मयति गमयति प्रति॒ष्ठाम् प्र॑ति॒ष्ठाम् ग॑मयति॒ प्रेद्धः॒ प्रेद्धो॑ गमयति प्रति॒ष्ठाम् प्र॑ति॒ष्ठाम् ग॑मयति॒ प्रेद्धः॑ । \newline
38. प्र॒ति॒ष्ठामिति॑ प्रति - स्थाम् । \newline
39. ग॒म॒य॒ति॒ प्रेद्धः॒ प्रेद्धो॑ गमयति गमयति॒ प्रेद्धो॑ अग्ने अग्ने॒ प्रेद्धो॑ गमयति गमयति॒ प्रेद्धो॑ अग्ने । \newline
40. प्रेद्धो॑ अग्ने अग्ने॒ प्रेद्धः॒ प्रेद्धो॑ अग्ने दीदिहि दीदिह्यग्ने॒ प्रेद्धः॒ प्रेद्धो॑ अग्ने दीदिहि । \newline
41. प्रेद्ध॒ इति॒ प्र - इ॒द्धः॒ । \newline
42. अ॒ग्ने॒ दी॒दि॒हि॒ दी॒दि॒ह्य॒ग्ने॒ अ॒ग्ने॒ दी॒दि॒हि॒ पु॒रः पु॒रो दी॑दिह्यग्ने अग्ने दीदिहि पु॒रः । \newline
43. दी॒दि॒हि॒ पु॒रः पु॒रो दी॑दिहि दीदिहि पु॒रो नो॑ नः पु॒रो दी॑दिहि दीदिहि पु॒रो नः॑ । \newline
44. पु॒रो नो॑ नः पु॒रः पु॒रो न॒ इतीति॑ नः पु॒रः पु॒रो न॒ इति॑ । \newline
45. न॒ इतीति॑ नो न॒ इत्यौदुं॑बरी॒ मौदुं॑बरी॒ मिति॑ नो न॒ इत्यौदुं॑बरीम् । \newline
46. इत्यौदुं॑बरी॒ मौदुं॑बरी॒ मिती त्यौदुं॑बरी॒ मौदुं॑बरी॒ मिती त्यौदुं॑बरी॒ मा । \newline
47. औदुं॑बरी॒ मौदुं॑बरी॒ मौदुं॑बरी॒ मा द॑धाति दधा॒ त्यौदुं॑बरी॒ मौदुं॑बरी॒ मा द॑धाति । \newline
48. आ द॑धाति दधा॒त्या द॑धात्ये॒ षैषा द॑धा॒त्या द॑धा त्ये॒षा । \newline
49. द॒धा॒ त्ये॒षैषा द॑धाति दधा त्ये॒षा वै वा ए॒षा द॑धाति दधा त्ये॒षा वै । \newline
50. ए॒षा वै वा ए॒षैषा वै सू॒र्मी सू॒र्मी वा ए॒षैषा वै सू॒र्मी । \newline
51. वै सू॒र्मी सू॒र्मी वै वै सू॒र्मी कर्ण॑कावती॒ कर्ण॑कावती सू॒र्मी वै वै सू॒र्मी कर्ण॑कावती । \newline
52. सू॒र्मी कर्ण॑कावती॒ कर्ण॑कावती सू॒र्मी सू॒र्मी कर्ण॑काव त्ये॒त यै॒तया॒ कर्ण॑कावती सू॒र्मी सू॒र्मी कर्ण॑काव त्ये॒तया᳚ । \newline
53. कर्ण॑काव त्ये॒त यै॒तया॒ कर्ण॑कावती॒ कर्ण॑काव त्ये॒तया॑ ह है॒तया॒ कर्ण॑कावती॒ कर्ण॑काव त्ये॒तया॑ ह । \newline
54. कर्ण॑काव॒तीति॒ कर्ण॑क - व॒ती॒ । \newline
55. ए॒तया॑ ह है॒त यै॒तया॑ ह स्म स्म है॒त यै॒तया॑ ह स्म । \newline
56. ह॒ स्म॒ स्म॒ ह॒ ह॒ स्म॒ वै वै स्म॑ ह ह स्म॒ वै । \newline
57. स्म॒ वै वै स्म॑ स्म॒ वै दे॒वा दे॒वा वै स्म॑ स्म॒ वै दे॒वाः । \newline
\pagebreak
\markright{ TS 5.4.7.4  \hfill https://www.vedavms.in \hfill}

\section{ TS 5.4.7.4 }

\textbf{TS 5.4.7.4 } \newline
\textbf{Samhita Paata} \newline

वै दे॒वा असु॑राणाꣳ शतत॒र्॒.हाꣳ स्तृꣳ॑हन्ति॒ यदे॒तया॑ स॒मिध॑मा॒दधा॑ति॒ वज्र॑मे॒वैतच्छ॑त॒घ्नीं ॅयज॑मानो॒ भ्रातृ॑व्याय॒ प्रह॑रति॒ स्तृत्या॒ अच्छ॑म्बट्कारं ॅवि॒धेम॑ ते पर॒मे जन्म॑न्नग्न॒ इति॒ वैक॑ङ्कती॒मा द॑धाति॒ भा ए॒वाव॑ रुन्धे॒ ताꣳ स॑वि॒तुर्वरे᳚ण्यस्य चि॒त्रामिति॑ शमी॒मयीꣳ॒॒ शान्त्या॑ अ॒ग्निर्वा॑ ह॒ वा अ॑ग्नि॒चितं॑ दु॒हे᳚ऽग्नि॒चिद्वा॒ऽग्निं दु॑हे॒ ताꣳ - [  ] \newline

\textbf{Pada Paata} \newline

वै । दे॒वाः । असु॑राणाम् । श॒त॒त॒र्.॒हानिति॑ शत-त॒र्॒.हान् । तृꣳ॒॒ह॒न्ति॒ । यत् । ए॒तया᳚ । स॒मिध॒मिति॑ सं - इध᳚म् । आ॒दधा॒तीत्या᳚ - दधा॑ति । वज्र᳚म् । ए॒व । ए॒तत् । श॒त॒घ्नीमिति॑ शत - घ्नीम् । यज॑मानः । भ्रातृ॑व्याय । प्रेति॑ । ह॒र॒ति॒ । स्तृत्यै᳚ । अच्छ॑बंट्कार॒मित्यछ॑बंट्-का॒र॒म् । वि॒धेम॑ । ते॒ । प॒र॒मे । जन्मन्न्॑ । अ॒ग्ने॒ । इति॑ । वैक॑ङ्कतीम् । एति॑ । द॒धा॒ति॒ । भाः । ए॒व । अवेति॑ । रु॒न्धे॒ । ताम् । स॒वि॒तुः । वरे᳚ण्यस्य । चि॒त्राम् । इति॑ । श॒मी॒मयी॒मिति॑ शमी - मयी᳚म् । शान्त्यै᳚ । अ॒ग्निः । वा॒ । ह॒ । वै । अ॒ग्नि॒चित॒मित्य॑ग्नि - चित᳚म् । दु॒हे । अ॒ग्नि॒चिदित्य॑ग्नि - चित् । वा॒ । अ॒ग्निम् । दु॒हे॒ । ताम् ।  \newline


\textbf{Krama Paata} \newline

वै दे॒वाः । दे॒वा असु॑राणाम् । असु॑राणाꣳ शतत॒र्.॒हान् । श॒त॒त॒र्.॒हाꣳस्तृꣳ॑हन्ति । श॒त॒त॒र्.॒हानिति॑ शत - त॒र्.॒हान् । तृꣳ॒॒ह॒न्ति॒ यत् । यदे॒तया᳚ । ए॒तया॑ स॒मिध᳚म् । स॒मिध॑मा॒दधा॑ति । स॒मिध॒मिति॑ सम् - इध᳚म् । आ॒दधा॑ति॒ वज्र᳚म् । आ॒दधा॒तीत्या᳚ - दधा॑ति । वज्र॑मे॒व । ए॒वैतत् । ए॒तच्छ॑त॒घ्नीम् । श॒त॒घ्नीम् ॅयज॑मानः । श॒त॒घ्नीमिति॑ शत - घ्नीम् । यज॑मानो॒ भ्रातृ॑व्याय । भ्रातृ॑व्याय॒ प्र । प्र ह॑रति । ह॒र॒ति॒ स्तृत्यै᳚ । स्तृत्या॒ अछ॑म्बट्कारम् । अछ॑म्बट्कारम् ॅवि॒धेम॑ । अछ॑म्बट्कार॒मित्यछ॑म्बट् - का॒र॒म् । वि॒धेम॑ ते । ते॒ प॒र॒मे । प॒र॒मे जन्मन्न्॑ । जन्म॑न्नग्ने । अ॒ग्न॒ इति॑ । इति॒ वैक॑ङ्कतीम् । वैक॑ङ्कती॒मा । आ द॑धाति । द॒धा॒ति॒ भाः । भा ए॒व । ए॒वाव॑ । अव॑ रुन्धे । रु॒न्धे॒ ताम् । ताꣳ स॑वि॒तुः । स॒वि॒तुर् वरे᳚ण्यस्य । वरे᳚ण्यस्य चि॒त्राम् । चि॒त्रामिति॑ । इति॑ शमी॒मयी᳚म् । श॒मी॒मयीꣳ॒॒ शान्त्यै᳚ । श॒मी॒मयी॒मिति॑ शमी - मयी᳚म् । शान्त्या॑ अ॒ग्निः । अ॒ग्निर् वा᳚ । वा॒ ह॒ । ह॒ वै । वा अ॑ग्नि॒चित᳚म् । अ॒ग्नि॒चित॑म् दु॒हे । अ॒ग्नि॒चित॒मित्य॑ग्नि - चित᳚म् । दु॒हे᳚ऽग्नि॒चित् । अ॒ग्नि॒चिद् वा᳚ । अ॒ग्नि॒चिदित्य॑ग्नि - चित् । वा॒ऽग्निम् । अ॒ग्निम् दु॑हे । दु॒हे॒ ताम् । ताꣳ स॑वि॒तुः \newline

\textbf{Jatai Paata} \newline

1. वै दे॒वा दे॒वा वै वै दे॒वाः । \newline
2. दे॒वा असु॑राणा॒ मसु॑राणाम् दे॒वा दे॒वा असु॑राणाम् । \newline
3. असु॑राणाꣳ शतत॒र्॒.हाञ् छ॑तत॒र्॒.हा नसु॑राणा॒ मसु॑राणाꣳ शतत॒र्॒.हान् । \newline
4. श॒त॒त॒र्॒.हाꣳ स्तृꣳ॑हन्ति तृꣳहन्ति शतत॒र्॒.हाञ् छ॑तत॒र्॒.हाꣳ स्तृꣳ॑हन्ति । \newline
5. श॒त॒त॒र्.॒हानिति॑ शत - त॒र्॒.हान् । \newline
6. तृꣳ॒॒ह॒न्ति॒ यद् यत् तृꣳ॑हन्ति तृꣳहन्ति॒ यत् । \newline
7. यदे॒त यै॒तया॒ यद् यदे॒तया᳚ । \newline
8. ए॒तया॑ स॒मिधꣳ॑ स॒मिध॑ मे॒त यै॒तया॑ स॒मिध᳚म् । \newline
9. स॒मिध॑ मा॒दधा᳚ त्या॒दधा॑ति स॒मिधꣳ॑ स॒मिध॑ मा॒दधा॑ति । \newline
10. स॒मिध॒मिति॑ सं - इध᳚म् । \newline
11. आ॒दधा॑ति॒ वज्रं॒ ॅवज्र॑ मा॒दधा᳚ त्या॒दधा॑ति॒ वज्र᳚म् । \newline
12. आ॒दधा॒तीत्या᳚ - दधा॑ति । \newline
13. वज्र॑ मे॒वैव वज्रं॒ ॅवज्र॑ मे॒व । \newline
14. ए॒वैत दे॒त दे॒वै वैतत् । \newline
15. ए॒त च्छ॑त॒घ्नीꣳ श॑त॒घ्नी मे॒त दे॒त च्छ॑त॒घ्नीम् । \newline
16. श॒त॒घ्नीं ॅयज॑मानो॒ यज॑मानः शत॒घ्नीꣳ श॑त॒घ्नीं ॅयज॑मानः । \newline
17. श॒त॒घ्नीमिति॑ शत - घ्नीम् । \newline
18. यज॑मानो॒ भ्रातृ॑व्याय॒ भ्रातृ॑व्याय॒ यज॑मानो॒ यज॑मानो॒ भ्रातृ॑व्याय । \newline
19. भ्रातृ॑व्याय॒ प्र प्र भ्रातृ॑व्याय॒ भ्रातृ॑व्याय॒ प्र । \newline
20. प्र ह॑रति हरति॒ प्र प्र ह॑रति । \newline
21. ह॒र॒ति॒ स्तृत्यै॒ स्तृत्यै॑ हरति हरति॒ स्तृत्यै᳚ । \newline
22. स्तृत्या॒ अछं॑बट्कार॒ मछं॑बट्कारꣳ॒॒ स्तृत्यै॒ स्तृत्या॒ अछं॑बट्कारम् । \newline
23. अछं॑बट्कारं ॅवि॒धेम॑ वि॒धेमा छं॑बट्कार॒ मछं॑बट्कारं ॅवि॒धेम॑ । \newline
24. अछं॑बट्कार॒मित्यछं॑बट् - का॒र॒म् । \newline
25. वि॒धेम॑ ते ते वि॒धेम॑ वि॒धेम॑ ते । \newline
26. ते॒ प॒र॒मे प॑र॒मे ते॑ ते पर॒मे । \newline
27. प॒र॒मे जन्म॒न् जन्म॑न् पर॒मे प॑र॒मे जन्मन्न्॑ । \newline
28. जन्म॑न् नग्ने अग्ने॒ जन्म॒न् जन्म॑न् नग्ने । \newline
29. अ॒ग्न॒ इतीत्य॑ग्ने ऽग्न॒ इति॑ । \newline
30. इति॒ वैक॑ङ्कतीं॒ ॅवैक॑ङ्कती॒ मितीति॒ वैक॑ङ्कतीम् । \newline
31. वैक॑ङ्कती॒ मा वैक॑ङ्कतीं॒ ॅवैक॑ङ्कती॒ मा । \newline
32. आ द॑धाति दधा॒त्या द॑धाति । \newline
33. द॒धा॒ति॒ भा भा द॑धाति दधाति॒ भाः । \newline
34. भा ए॒वैव भा भा ए॒व । \newline
35. ए॒वावा वै॒वै वाव॑ । \newline
36. अव॑ रुन्धे रु॒न्धे ऽवाव॑ रुन्धे । \newline
37. रु॒न्धे॒ ताम् ताꣳ रु॑न्धे रुन्धे॒ ताम् । \newline
38. ताꣳ स॑वि॒तुः स॑वि॒तु स्ताम् ताꣳ स॑वि॒तुः । \newline
39. स॒वि॒तुर् वरे᳚ण्यस्य॒ वरे᳚ण्यस्य सवि॒तुः स॑वि॒तुर् वरे᳚ण्यस्य । \newline
40. वरे᳚ण्यस्य चि॒त्राम् चि॒त्रां ॅवरे᳚ण्यस्य॒ वरे᳚ण्यस्य चि॒त्राम् । \newline
41. चि॒त्रा मितीति॑ चि॒त्राम् चि॒त्रा मिति॑ । \newline
42. इति॑ शमी॒मयीꣳ॑ शमी॒मयी॒ मितीति॑ शमी॒मयी᳚म् । \newline
43. श॒मी॒मयीꣳ॒॒ शान्त्यै॒ शान्त्यै॑ शमी॒मयीꣳ॑ शमी॒मयीꣳ॒॒ शान्त्यै᳚ । \newline
44. श॒मी॒मयी॒मिति॑ शमी - मयी᳚म् । \newline
45. शान्त्या॑ अ॒ग्नि र॒ग्निः शान्त्यै॒ शान्त्या॑ अ॒ग्निः । \newline
46. अ॒ग्निर् वा॑ वा॒ ऽग्नि र॒ग्निर् वा᳚ । \newline
47. वा॒ ह॒ ह॒ वा॒ वा॒ ह॒ । \newline
48. ह॒ वै वै ह॑ ह॒ वै । \newline
49. वा अ॑ग्नि॒चित॑ मग्नि॒चितं॒ ॅवै वा अ॑ग्नि॒चित᳚म् । \newline
50. अ॒ग्नि॒चित॑म् दु॒हे दु॒हे᳚ ऽग्नि॒चित॑ मग्नि॒चित॑म् दु॒हे । \newline
51. अ॒ग्नि॒चित॒मित्य॑ग्नि - चित᳚म् । \newline
52. दु॒हे᳚ ऽग्नि॒चि द॑ग्नि॒चिद् दु॒हे दु॒हे᳚ ऽग्नि॒चित् । \newline
53. अ॒ग्नि॒चिद् वा॑ वा ऽग्नि॒चि द॑ग्नि॒चिद् वा᳚ । \newline
54. अ॒ग्नि॒चिदित्य॑ग्नि - चित् । \newline
55. वा॒ ऽग्नि म॒ग्निं ॅवा॑ वा॒ ऽग्निम् । \newline
56. अ॒ग्निम् दु॑हे दुहे॒ ऽग्नि म॒ग्निम् दु॑हे । \newline
57. दु॒हे॒ ताम् ताम् दु॑हे दुहे॒ ताम् । \newline
58. ताꣳ स॑वि॒तुः स॑वि॒तु स्ताम् ताꣳ स॑वि॒तुः । \newline

\textbf{Ghana Paata } \newline

1. वै दे॒वा दे॒वा वै वै दे॒वा असु॑राणा॒ मसु॑राणाम् दे॒वा वै वै दे॒वा असु॑राणाम् । \newline
2. दे॒वा असु॑राणा॒ मसु॑राणाम् दे॒वा दे॒वा असु॑राणाꣳ शतत॒र्॒.हाञ् छ॑तत॒र्॒.हा नसु॑राणाम् दे॒वा दे॒वा असु॑राणाꣳ शतत॒र्॒.हान् । \newline
3. असु॑राणाꣳ शतत॒र्॒.हाञ् छ॑तत॒र्॒.हा नसु॑राणा॒ मसु॑राणाꣳ शतत॒र्॒.हाꣳ स्तृꣳ॑हन्ति तृꣳहन्ति शतत॒र्॒.हा नसु॑राणा॒ मसु॑राणाꣳ शतत॒र्॒.हाꣳ स्तृꣳ॑हन्ति । \newline
4. श॒त॒त॒र्॒.हाꣳ स्तृꣳ॑हन्ति तृꣳहन्ति शतत॒र्॒.हाञ् छ॑तत॒र्॒.हाꣳ स्तृꣳ॑हन्ति॒ यद् यत् तृꣳ॑हन्ति शतत॒र्॒.हाञ् छ॑तत॒र्॒.हाꣳ स्तृꣳ॑हन्ति॒ यत् । \newline
5. श॒त॒त॒र्.॒हानिति॑ शत - त॒र्॒.हान् । \newline
6. तृꣳ॒॒ह॒न्ति॒ यद् यत् तृꣳ॑हन्ति तृꣳहन्ति॒ यदे॒त यै॒तया॒ यत् तृꣳ॑हन्ति तृꣳहन्ति॒ यदे॒ तया᳚ । \newline
7. यदे॒त यै॒तया॒ यद् यदे॒तया॑ स॒मिधꣳ॑ स॒मिध॑ मे॒तया॒ यद् यदे॒तया॑ स॒मिध᳚म् । \newline
8. ए॒तया॑ स॒मिधꣳ॑ स॒मिध॑ मे॒त यै॒तया॑ स॒मिध॑ मा॒दधा᳚ त्या॒दधा॑ति स॒मिध॑ मे॒त यै॒तया॑ स॒मिध॑ मा॒दधा॑ति । \newline
9. स॒मिध॑ मा॒दधा᳚ त्या॒दधा॑ति स॒मिधꣳ॑ स॒मिध॑ मा॒दधा॑ति॒ वज्रं॒ ॅवज्र॑ मा॒दधा॑ति स॒मिधꣳ॑ स॒मिध॑ मा॒दधा॑ति॒ वज्र᳚म् । \newline
10. स॒मिध॒मिति॑ सं - इध᳚म् । \newline
11. आ॒दधा॑ति॒ वज्रं॒ ॅवज्र॑ मा॒दधा᳚ त्या॒दधा॑ति॒ वज्र॑ मे॒वैव वज्र॑ मा॒दधा᳚ त्या॒दधा॑ति॒ वज्र॑ मे॒व । \newline
12. आ॒दधा॒तीत्या᳚ - दधा॑ति । \newline
13. वज्र॑ मे॒वैव वज्रं॒ ॅवज्र॑ मे॒वैत दे॒त दे॒व वज्रं॒ ॅवज्र॑ मे॒वैतत् । \newline
14. ए॒वैत दे॒त दे॒वैवैत च्छ॑त॒घ्नीꣳ श॑त॒घ्नी मे॒तदे॒ वैवैत च्छ॑त॒घ्नीम् । \newline
15. ए॒त च्छ॑त॒घ्नीꣳ श॑त॒घ्नी मे॒तदे॒त च्छ॑त॒घ्नीं ॅयज॑मानो॒ यज॑मानः शत॒घ्नी मे॒तदे॒त च्छ॑त॒घ्नीं ॅयज॑मानः । \newline
16. श॒त॒घ्नीं ॅयज॑मानो॒ यज॑मानः शत॒घ्नीꣳ श॑त॒घ्नीं ॅयज॑मानो॒ भ्रातृ॑व्याय॒ भ्रातृ॑व्याय॒ यज॑मानः शत॒घ्नीꣳ श॑त॒घ्नीं ॅयज॑मानो॒ भ्रातृ॑व्याय । \newline
17. श॒त॒घ्नीमिति॑ शत - घ्नीम् । \newline
18. यज॑मानो॒ भ्रातृ॑व्याय॒ भ्रातृ॑व्याय॒ यज॑मानो॒ यज॑मानो॒ भ्रातृ॑व्याय॒ प्र प्र भ्रातृ॑व्याय॒ यज॑मानो॒ यज॑मानो॒ भ्रातृ॑व्याय॒ प्र । \newline
19. भ्रातृ॑व्याय॒ प्र प्र भ्रातृ॑व्याय॒ भ्रातृ॑व्याय॒ प्र ह॑रति हरति॒ प्र भ्रातृ॑व्याय॒ भ्रातृ॑व्याय॒ प्र ह॑रति । \newline
20. प्र ह॑रति हरति॒ प्र प्र ह॑रति॒ स्तृत्यै॒ स्तृत्यै॑ हरति॒ प्र प्र ह॑रति॒ स्तृत्यै᳚ । \newline
21. ह॒र॒ति॒ स्तृत्यै॒ स्तृत्यै॑ हरति हरति॒ स्तृत्या॒ अछं॑बट्कार॒ मछं॑बट्कारꣳ॒॒ स्तृत्यै॑ हरति हरति॒ स्तृत्या॒ अछं॑बट्कारम् । \newline
22. स्तृत्या॒ अछं॑बट्कार॒ मछं॑बट्कारꣳ॒॒ स्तृत्यै॒ स्तृत्या॒ अछं॑बट्कारं ॅवि॒धेम॑ वि॒धेमा छं॑बट्कारꣳ॒॒ स्तृत्यै॒ स्तृत्या॒ अछं॑बट्कारं ॅवि॒धेम॑ । \newline
23. अछं॑बट्कारं ॅवि॒धेम॑ वि॒धेमा छं॑बट्कार॒ मछं॑बट्कारं ॅवि॒धेम॑ ते ते वि॒धेमा छं॑बट्कार॒ मछं॑बट्कारं ॅवि॒धेम॑ ते । \newline
24. अछं॑बट्कार॒मित्यछं॑बट् - का॒र॒म् । \newline
25. वि॒धेम॑ ते ते वि॒धेम॑ वि॒धेम॑ ते पर॒मे प॑र॒मे ते॑ वि॒धेम॑ वि॒धेम॑ ते पर॒मे । \newline
26. ते॒ प॒र॒मे प॑र॒मे ते॑ ते पर॒मे जन्म॒न् जन्म॑न् पर॒मे ते॑ ते पर॒मे जन्मन्न्॑ । \newline
27. प॒र॒मे जन्म॒न् जन्म॑न् पर॒मे प॑र॒मे जन्म॑न् नग्ने अग्ने॒ जन्म॑न् पर॒मे प॑र॒मे जन्म॑न् नग्ने । \newline
28. जन्म॑न् नग्ने अग्ने॒ जन्म॒न् जन्म॑न् नग्न॒ इती त्य॑ग्ने॒ जन्म॒न् जन्म॑न् नग्न॒ इति॑ । \newline
29. अ॒ग्न॒ इती त्य॑ग्ने ऽग्न॒ इति॒ वैक॑ङ्कतीं॒ ॅवैक॑ङ्कती॒ मित्य॑ग्ने ऽग्न॒ इति॒ वैक॑ङ्कतीम् । \newline
30. इति॒ वैक॑ङ्कतीं॒ ॅवैक॑ङ्कती॒ मितीति॒ वैक॑ङ्कती॒ मा वैक॑ङ्कती॒ मितीति॒ वैक॑ङ्कती॒ मा । \newline
31. वैक॑ङ्कती॒ मा वैक॑ङ्कतीं॒ ॅवैक॑ङ्कती॒ मा द॑धाति दधा॒त्या वैक॑ङ्कतीं॒ ॅवैक॑ङ्कती॒ मा द॑धाति । \newline
32. आ द॑धाति दधा॒त्या द॑धाति॒ भा भा द॑धा॒त्या द॑धाति॒ भाः । \newline
33. द॒धा॒ति॒ भा भा द॑धाति दधाति॒ भा ए॒वैव भा द॑धाति दधाति॒ भा ए॒व । \newline
34. भा ए॒वैव भा भा ए॒वावा वै॒व भा भा ए॒वाव॑ । \newline
35. ए॒वावा वै॒वै वाव॑ रुन्धे रु॒न्धे ऽवै॒वै वाव॑ रुन्धे । \newline
36. अव॑ रुन्धे रु॒न्धे ऽवाव॑ रुन्धे॒ ताम् ताꣳ रु॒न्धे ऽवाव॑ रुन्धे॒ ताम् । \newline
37. रु॒न्धे॒ ताम् ताꣳ रु॑न्धे रुन्धे॒ ताꣳ स॑वि॒तुः स॑वि॒तु स्ताꣳ रु॑न्धे रुन्धे॒ ताꣳ स॑वि॒तुः । \newline
38. ताꣳ स॑वि॒तुः स॑वि॒तु स्ताम् ताꣳ स॑वि॒तुर् वरे᳚ण्यस्य॒ वरे᳚ण्यस्य सवि॒तु स्ताम् ताꣳ स॑वि॒तुर् वरे᳚ण्यस्य । \newline
39. स॒वि॒तुर् वरे᳚ण्यस्य॒ वरे᳚ण्यस्य सवि॒तुः स॑वि॒तुर् वरे᳚ण्यस्य चि॒त्राम् चि॒त्रां ॅवरे᳚ण्यस्य सवि॒तुः स॑वि॒तुर् वरे᳚ण्यस्य चि॒त्राम् । \newline
40. वरे᳚ण्यस्य चि॒त्राम् चि॒त्रां ॅवरे᳚ण्यस्य॒ वरे᳚ण्यस्य चि॒त्रा मितीति॑ चि॒त्रां ॅवरे᳚ण्यस्य॒ वरे᳚ण्यस्य चि॒त्रा मिति॑ । \newline
41. चि॒त्रा मितीति॑ चि॒त्राम् चि॒त्रा मिति॑ शमी॒मयीꣳ॑ शमी॒मयी॒ मिति॑ चि॒त्राम् चि॒त्रा मिति॑ शमी॒मयी᳚म् । \newline
42. इति॑ शमी॒मयीꣳ॑ शमी॒मयी॒ मितीति॑ शमी॒मयीꣳ॒॒ शान्त्यै॒ शान्त्यै॑ शमी॒मयी॒ मितीति॑ शमी॒मयीꣳ॒॒ शान्त्यै᳚ । \newline
43. श॒मी॒मयीꣳ॒॒ शान्त्यै॒ शान्त्यै॑ शमी॒मयीꣳ॑ शमी॒मयीꣳ॒॒ शान्त्या॑ अ॒ग्नि र॒ग्निः शान्त्यै॑ शमी॒मयीꣳ॑ शमी॒मयीꣳ॒॒ शान्त्या॑ अ॒ग्निः । \newline
44. श॒मी॒मयी॒मिति॑ शमी - मयी᳚म् । \newline
45. शान्त्या॑ अ॒ग्नि र॒ग्निः शान्त्यै॒ शान्त्या॑ अ॒ग्निर् वा॑ वा॒ ऽग्निः शान्त्यै॒ शान्त्या॑ अ॒ग्निर् वा᳚ । \newline
46. अ॒ग्निर् वा॑ वा॒ ऽग्नि र॒ग्निर् वा॑ ह ह वा॒ ऽग्नि र॒ग्निर् वा॑ ह । \newline
47. वा॒ ह॒ ह॒ वा॒ वा॒ ह॒ वै वै ह॑ वा वा ह॒ वै । \newline
48. ह॒ वै वै ह॑ ह॒ वा अ॑ग्नि॒चित॑ मग्नि॒चितं॒ ॅवै ह॑ ह॒ वा अ॑ग्नि॒चित᳚म् । \newline
49. वा अ॑ग्नि॒चित॑ मग्नि॒चितं॒ ॅवै वा अ॑ग्नि॒चित॑म् दु॒हे दु॒हे᳚ ऽग्नि॒चितं॒ ॅवै वा अ॑ग्नि॒चित॑म् दु॒हे । \newline
50. अ॒ग्नि॒चित॑म् दु॒हे दु॒हे᳚ ऽग्नि॒चित॑ मग्नि॒चित॑म् दु॒हे᳚ ऽग्नि॒चि द॑ग्नि॒चिद् दु॒हे᳚ ऽग्नि॒चित॑ मग्नि॒चित॑म् दु॒हे᳚ ऽग्नि॒चित् । \newline
51. अ॒ग्नि॒चित॒मित्य॑ग्नि - चित᳚म् । \newline
52. दु॒हे᳚ ऽग्नि॒चि द॑ग्नि॒चिद् दु॒हे दु॒हे᳚ ऽग्नि॒चिद् वा॑ वा ऽग्नि॒चिद् दु॒हे दु॒हे᳚ ऽग्नि॒चिद् वा᳚ । \newline
53. अ॒ग्नि॒चिद् वा॑ वा ऽग्नि॒चि द॑ग्नि॒चिद् वा॒ ऽग्नि म॒ग्निं ॅवा᳚ ऽग्नि॒चि द॑ग्नि॒चिद् वा॒ ऽग्निम् । \newline
54. अ॒ग्नि॒चिदित्य॑ग्नि - चित् । \newline
55. वा॒ ऽग्नि म॒ग्निं ॅवा॑ वा॒ ऽग्निम् दु॑हे दुहे॒ ऽग्निं ॅवा॑ वा॒ ऽग्निम् दु॑हे । \newline
56. अ॒ग्निम् दु॑हे दुहे॒ ऽग्नि म॒ग्निम् दु॑हे॒ ताम् ताम् दु॑हे॒ ऽग्नि म॒ग्निम् दु॑हे॒ ताम् । \newline
57. दु॒हे॒ ताम् ताम् दु॑हे दुहे॒ ताꣳ स॑वि॒तुः स॑वि॒तु स्ताम् दु॑हे दुहे॒ ताꣳ स॑वि॒तुः । \newline
58. ताꣳ स॑वि॒तुः स॑वि॒तु स्ताम् ताꣳ स॑वि॒तुर् वरे᳚ण्यस्य॒ वरे᳚ण्यस्य सवि॒तु स्ताम् ताꣳ स॑वि॒तुर् वरे᳚ण्यस्य । \newline
\pagebreak
\markright{ TS 5.4.7.5  \hfill https://www.vedavms.in \hfill}

\section{ TS 5.4.7.5 }

\textbf{TS 5.4.7.5 } \newline
\textbf{Samhita Paata} \newline

स॑वि॒तुर्वरे᳚ण्यस्य चि॒त्रामित्या॑है॒ष वा अ॒ग्नेर्दोह॒स्तम॑स्य॒ कण्व॑ ए॒व श्रा॑य॒सो॑ऽवे॒त् तेन॑ ह स्मैनꣳ॒॒ स दु॑हे॒ यदे॒तया॑ स॒मिध॑-मा॒दधा᳚त्यग्नि॒चिदे॒व तद॒ग्निं दु॑हे स॒प्त ते॑ अग्ने स॒मिधः॑ स॒प्तजि॒ह्वा इत्या॑ह स॒प्तैवास्य॒ साप्ता॑नि प्रीणाति पू॒र्णया॑ जुहोति पू॒र्ण इ॑व॒ हि प्र॒जाप॑तिः प्र॒जाप॑ते॒ - [  ] \newline

\textbf{Pada Paata} \newline

स॒वि॒तुः । वरे᳚ण्यस्य । चि॒त्राम् । इति॑ । आ॒ह॒ । ए॒षः । वै । अ॒ग्नेः । दोहः॑ । तम् । अ॒स्य॒ । कण्वः॑ । ए॒व । श्रा॒य॒सः । अ॒वे॒त् । तेन॑ । ह॒ । स्म॒ । ए॒न॒म् । सः । दु॒हे॒ । यत् । ए॒तया᳚ । स॒मिध॒मिति॑ सं - इध᳚म् । आ॒दधा॒तीत्या᳚ - दधा॑ति । अ॒ग्नि॒चिदित्य॑ग्नि - चित् । ए॒व । तत् । अ॒ग्निम् । दु॒हे॒ । स॒प्त । ते॒ । अ॒ग्ने॒ । स॒मिध॒ इति॑ सं - इधः॑ । स॒प्त । जि॒ह्वाः । इति॑ । आ॒ह॒ । स॒प्त । ए॒व । अ॒स्य॒ । साप्ता॑नि । प्री॒णा॒ति॒ । पू॒र्णया᳚ । जु॒हो॒ति॒ । पू॒र्णः । इ॒व॒ । हि । प्र॒जाप॑ति॒रिति॑ प्र॒जा - प॒तिः॒ । प्र॒जाप॑ते॒रिति॑ प्र॒जा - प॒तेः॒ ।  \newline


\textbf{Krama Paata} \newline

स॒वि॒तुर् वरे᳚ण्यस्य । वरे᳚ण्यस्य चि॒त्राम् । चि॒त्रामिति॑ । इत्या॑ह । आ॒है॒षः । ए॒ष वै । वा अ॒ग्नेः । अ॒ग्नेर् दोहः॑ । दोह॒स्तम् । तम॑स्य । अ॒स्य॒ कण्वः॑ । कण्व॑ ए॒व । ए॒व श्रा॑य॒सः । श्रा॒य॒सो॑ऽवेत् । अ॒वे॒त् तेन॑ । तेन॑ ह । ह॒ स्म॒ । स्मै॒न॒म् । ए॒नꣳ॒॒ सः । स दु॑हे । दु॒हे॒ यत् । यदे॒तया᳚ । ए॒तया॑ स॒मिध᳚म् । स॒मिध॑मा॒दधा॑ति । स॒मिध॒मिति॑ सम् - इध᳚म् । आ॒दधा᳚त्यग्नि॒चित् । आ॒दधा॒तीत्या᳚ - दधा॑ति । अ॒ग्नि॒चिदे॒व । अ॒ग्नि॒चिदित्य॑ग्नि - चित् । ए॒व तत् । तद॒ग्निम् । अ॒ग्निम् दु॑हे । दु॒हे॒ स॒प्त । स॒प्त ते᳚ । ते॒ अ॒ग्ने॒ । अ॒ग्ने॒ स॒मिधः॑ । स॒मिधः॑ स॒प्त । स॒मिध॒ इति॑ सम् - इधः॑ । स॒प्त जि॒ह्वाः । जि॒ह्वा इति॑ । इत्या॑ह । आ॒ह॒ स॒प्त । स॒प्तैव । ए॒वास्य॑ । अ॒स्य॒ साप्ता॑नि । साप्ता॑नि प्रीणाति । प्री॒णा॒ति॒ पू॒र्णया᳚ । पू॒र्णया॑ जुहोति । जु॒हो॒ति॒ पू॒र्णः । पू॒र्ण इ॑व । इ॒व॒ हि । हि प्र॒जाप॑तिः । प्र॒जाप॑तिः प्र॒जाप॑तेः । प्र॒जाप॑ति॒रिति॑ प्र॒जा - प॒तिः॒ । प्र॒जाप॑ते॒राप्त्यै᳚ । प्र॒जाप॑ते॒रिति॑ प्र॒जा - प॒तेः॒ \newline

\textbf{Jatai Paata} \newline

1. स॒वि॒तुर् वरे᳚ण्यस्य॒ वरे᳚ण्यस्य सवि॒तुः स॑वि॒तुर् वरे᳚ण्यस्य । \newline
2. वरे᳚ण्यस्य चि॒त्राम् चि॒त्रां ॅवरे᳚ण्यस्य॒ वरे᳚ण्यस्य चि॒त्राम् । \newline
3. चि॒त्रा मितीति॑ चि॒त्राम् चि॒त्रा मिति॑ । \newline
4. इत्या॑हा॒हे तीत्या॑ह । \newline
5. आ॒है॒ष ए॒ष आ॑हा है॒षः । \newline
6. ए॒ष वै वा ए॒ष ए॒ष वै । \newline
7. वा अ॒ग्ने र॒ग्नेर् वै वा अ॒ग्नेः । \newline
8. अ॒ग्नेर् दोहो॒ दोहो॒ ऽग्ने र॒ग्नेर् दोहः॑ । \newline
9. दोह॒ स्तम् तम् दोहो॒ दोह॒ स्तम् । \newline
10. त म॑स्यास्य॒ तम् त म॑स्य । \newline
11. अ॒स्य॒ कण्वः॒ कण्वो᳚ ऽस्यास्य॒ कण्वः॑ । \newline
12. कण्व॑ ए॒वैव कण्वः॒ कण्व॑ ए॒व । \newline
13. ए॒व श्रा॑य॒सः श्रा॑य॒स ए॒वैव श्रा॑य॒सः । \newline
14. श्रा॒य॒सो॑ ऽवे दवे च्छ्राय॒सः श्रा॑य॒सो॑ ऽवेत् । \newline
15. अ॒वे॒त् तेन॒ तेना॑ वे दवे॒त् तेन॑ । \newline
16. तेन॑ ह ह॒ तेन॒ तेन॑ ह । \newline
17. ह॒ स्म॒ स्म॒ ह॒ ह॒ स्म॒ । \newline
18. स्मै॒न॒ मे॒नꣳ॒॒ स्म॒ स्मै॒न॒म् । \newline
19. ए॒नꣳ॒॒ स स ए॑न मेनꣳ॒॒ सः । \newline
20. स दु॑हे दुहे॒ स स दु॑हे । \newline
21. दु॒हे॒ यद् यद् दु॑हे दुहे॒ यत् । \newline
22. यदे॒त यै॒तया॒ यद् यदे॒तया᳚ । \newline
23. ए॒तया॑ स॒मिधꣳ॑ स॒मिध॑ मे॒त यै॒तया॑ स॒मिध᳚म् । \newline
24. स॒मिध॑ मा॒दधा᳚ त्या॒दधा॑ति स॒मिधꣳ॑ स॒मिध॑ मा॒दधा॑ति । \newline
25. स॒मिध॒मिति॑ सं - इध᳚म् । \newline
26. आ॒दधा᳚ त्यग्नि॒चि द॑ग्नि॒चि दा॒दधा᳚ त्या॒दधा᳚ त्यग्नि॒चित् । \newline
27. आ॒दधा॒तीत्या᳚ - दधा॑ति । \newline
28. अ॒ग्नि॒चि दे॒वैवा ग्नि॒चि द॑ग्नि॒चि दे॒व । \newline
29. अ॒ग्नि॒चिदित्य॑ग्नि - चित् । \newline
30. ए॒व तत् तदे॒ वैव तत् । \newline
31. तद॒ग्नि म॒ग्निम् तत् तद॒ग्निम् । \newline
32. अ॒ग्निम् दु॑हे दुहे॒ ऽग्नि म॒ग्निम् दु॑हे । \newline
33. दु॒हे॒ स॒प्त स॒प्त दु॑हे दुहे स॒प्त । \newline
34. स॒प्त ते॑ ते स॒प्त स॒प्त ते᳚ । \newline
35. ते॒ अ॒ग्ने॒ ऽग्ने॒ ते॒ ते॒ अ॒ग्ने॒ । \newline
36. अ॒ग्ने॒ स॒मिधः॑ स॒मिधो᳚ ऽग्ने ऽग्ने स॒मिधः॑ । \newline
37. स॒मिधः॑ स॒प्त स॒प्त स॒मिधः॑ स॒मिधः॑ स॒प्त । \newline
38. स॒मिध॒ इति॑ सं - इधः॑ । \newline
39. स॒प्त जि॒ह्वा जि॒ह्वाः स॒प्त स॒प्त जि॒ह्वाः । \newline
40. जि॒ह्वा इतीति॑ जि॒ह्वा जि॒ह्वा इति॑ । \newline
41. इत्या॑हा॒हे तीत्या॑ह । \newline
42. आ॒ह॒ स॒प्त स॒प्ताहा॑ह स॒प्त । \newline
43. स॒प्तैवैव स॒प्त स॒प्तैव । \newline
44. ए॒वास्या᳚ स्यै॒वै वास्य॑ । \newline
45. अ॒स्य॒ साप्ता॑नि॒ साप्ता᳚ न्यस्यास्य॒ साप्ता॑नि । \newline
46. साप्ता॑नि प्रीणाति प्रीणाति॒ साप्ता॑नि॒ साप्ता॑नि प्रीणाति । \newline
47. प्री॒णा॒ति॒ पू॒र्णया॑ पू॒र्णया᳚ प्रीणाति प्रीणाति पू॒र्णया᳚ । \newline
48. पू॒र्णया॑ जुहोति जुहोति पू॒र्णया॑ पू॒र्णया॑ जुहोति । \newline
49. जु॒हो॒ति॒ पू॒र्णः पू॒र्णो जु॑होति जुहोति पू॒र्णः । \newline
50. पू॒र्ण इ॑वेव पू॒र्णः पू॒र्ण इ॑व । \newline
51. इ॒व॒ हि हीवे॑ व॒ हि । \newline
52. हि प्र॒जाप॑तिः प्र॒जाप॑ति॒र्॒. हि हि प्र॒जाप॑तिः । \newline
53. प्र॒जाप॑तिः प्र॒जाप॑तेः प्र॒जाप॑तेः प्र॒जाप॑तिः प्र॒जाप॑तिः प्र॒जाप॑तेः । \newline
54. प्र॒जाप॑ति॒रिति॑ प्र॒जा - प॒तिः॒ । \newline
55. प्र॒जाप॑ते॒ राप्त्या॒ आप्त्यै᳚ प्र॒जाप॑तेः प्र॒जाप॑ते॒ राप्त्यै᳚ । \newline
56. प्र॒जाप॑ते॒रिति॑ प्र॒जा - प॒तेः॒ । \newline

\textbf{Ghana Paata } \newline

1. स॒वि॒तुर् वरे᳚ण्यस्य॒ वरे᳚ण्यस्य सवि॒तुः स॑वि॒तुर् वरे᳚ण्यस्य चि॒त्राम् चि॒त्रां ॅवरे᳚ण्यस्य सवि॒तुः स॑वि॒तुर् वरे᳚ण्यस्य चि॒त्राम् । \newline
2. वरे᳚ण्यस्य चि॒त्राम् चि॒त्रां ॅवरे᳚ण्यस्य॒ वरे᳚ण्यस्य चि॒त्रा मितीति॑ चि॒त्रां ॅवरे᳚ण्यस्य॒ वरे᳚ण्यस्य चि॒त्रा मिति॑ । \newline
3. चि॒त्रा मितीति॑ चि॒त्राम् चि॒त्रा मित्या॑हा॒हेति॑ चि॒त्राम् चि॒त्रा मित्या॑ह । \newline
4. इत्या॑हा॒हे तीत्या॑ है॒ष ए॒ष आ॒हे तीत्या॑ है॒षः । \newline
5. आ॒है॒ष ए॒ष आ॑हाहै॒ष वै वा ए॒ष आ॑हाहै॒ष वै । \newline
6. ए॒ष वै वा ए॒ष ए॒ष वा अ॒ग्ने र॒ग्नेर् वा ए॒ष ए॒ष वा अ॒ग्नेः । \newline
7. वा अ॒ग्ने र॒ग्नेर् वै वा अ॒ग्नेर् दोहो॒ दोहो॒ ऽग्नेर् वै वा अ॒ग्नेर् दोहः॑ । \newline
8. अ॒ग्नेर् दोहो॒ दोहो॒ ऽग्ने र॒ग्नेर् दोह॒ स्तम् तम् दोहो॒ ऽग्ने र॒ग्नेर् दोह॒ स्तम् । \newline
9. दोह॒स्तम् तम् दोहो॒ दोह॒ स्त म॑स्यास्य॒ तम् दोहो॒ दोह॒ स्त म॑स्य । \newline
10. त म॑स्यास्य॒ तम् त म॑स्य॒ कण्वः॒ कण्वो᳚ ऽस्य॒ तम् त म॑स्य॒ कण्वः॑ । \newline
11. अ॒स्य॒ कण्वः॒ कण्वो᳚ ऽस्यास्य॒ कण्व॑ ए॒वैव कण्वो᳚ ऽस्यास्य॒ कण्व॑ ए॒व । \newline
12. कण्व॑ ए॒वैव कण्वः॒ कण्व॑ ए॒व श्रा॑य॒सः श्रा॑य॒स ए॒व कण्वः॒ कण्व॑ ए॒व श्रा॑य॒सः । \newline
13. ए॒व श्रा॑य॒सः श्रा॑य॒स ए॒वैव श्रा॑य॒सो॑ ऽवे दवे च्छ्राय॒स ए॒वैव श्रा॑य॒सो॑ ऽवेत् । \newline
14. श्रा॒य॒सो॑ ऽवे दवे च्छ्राय॒सः श्रा॑य॒सो॑ ऽवे॒त् तेन॒ तेना॑वे च्छ्राय॒सः श्रा॑य॒सो॑ ऽवे॒त् तेन॑ । \newline
15. अ॒वे॒त् तेन॒ तेना॑ वे दवे॒त् तेन॑ ह ह॒ तेना॑वे दवे॒त् तेन॑ ह । \newline
16. तेन॑ ह ह॒ तेन॒ तेन॑ ह स्म स्म ह॒ तेन॒ तेन॑ ह स्म । \newline
17. ह॒ स्म॒ स्म॒ ह॒ ह॒ स्मै॒न॒ मे॒नꣳ॒॒ स्म॒ ह॒ ह॒ स्मै॒न॒म् । \newline
18. स्मै॒न॒ मे॒नꣳ॒॒ स्म॒ स्मै॒नꣳ॒॒ स स ए॑नꣳ स्म स्मैनꣳ॒॒ सः । \newline
19. ए॒नꣳ॒॒ स स ए॑न मेनꣳ॒॒ स दु॑हे दुहे॒ स ए॑न मेनꣳ॒॒ स दु॑हे । \newline
20. स दु॑हे दुहे॒ स स दु॑हे॒ यद् यद् दु॑हे॒ स स दु॑हे॒ यत् । \newline
21. दु॒हे॒ यद् यद् दु॑हे दुहे॒य दे॒त यै॒तया॒ यद् दु॑हे दुहे॒य दे॒तया᳚ । \newline
22. यदे॒त यै॒तया॒ यद् यदे॒तया॑ स॒मिधꣳ॑ स॒मिध॑ मे॒तया॒ यद् यदे॒तया॑ स॒मिध᳚म् । \newline
23. ए॒तया॑ स॒मिधꣳ॑ स॒मिध॑ मे॒त यै॒तया॑ स॒मिध॑ मा॒दधा᳚ त्या॒दधा॑ति स॒मिध॑ मे॒त यै॒तया॑ स॒मिध॑ मा॒दधा॑ति । \newline
24. स॒मिध॑ मा॒दधा᳚ त्या॒दधा॑ति स॒मिधꣳ॑ स॒मिध॑ मा॒दधा᳚ त्यग्नि॒चि द॑ग्नि॒चिदा॒ दधा॑ति स॒मिधꣳ॑ स॒मिध॑ मा॒दधा᳚ त्यग्नि॒चित् । \newline
25. स॒मिध॒मिति॑ सं - इध᳚म् । \newline
26. आ॒दधा᳚ त्यग्नि॒चि द॑ग्नि॒चिद् आ॒दधा᳚ त्या॒दधा᳚ त्यग्नि॒चि दे॒वै वाग्नि॒चि दा॒दधा᳚ त्या॒दधा᳚ त्यग्नि॒चि दे॒व । \newline
27. आ॒दधा॒तीत्या᳚ - दधा॑ति । \newline
28. अ॒ग्नि॒चि दे॒वै वाग्नि॒चि द॑ग्नि॒चि दे॒व तत् तदे॒ वाग्नि॒चि द॑ग्नि॒चि दे॒व तत् । \newline
29. अ॒ग्नि॒चिदित्य॑ग्नि - चित् । \newline
30. ए॒व तत् तदे॒ वैव तद॒ग्नि म॒ग्निम् तदे॒ वैव तद॒ग्निम् । \newline
31. तद॒ग्नि म॒ग्निम् तत् तद॒ग्निम् दु॑हे दुहे॒ ऽग्निम् तत् तद॒ग्निम् दु॑हे । \newline
32. अ॒ग्निम् दु॑हे दुहे॒ ऽग्नि म॒ग्निम् दु॑हे स॒प्त स॒प्त दु॑हे॒ ऽग्नि म॒ग्निम् दु॑हे स॒प्त । \newline
33. दु॒हे॒ स॒प्त स॒प्त दु॑हे दुहे स॒प्त ते॑ ते स॒प्त दु॑हे दुहे स॒प्त ते᳚ । \newline
34. स॒प्त ते॑ ते स॒प्त स॒प्त ते॑ अग्ने ऽग्ने ते स॒प्त स॒प्त ते॑ अग्ने । \newline
35. ते॒ अ॒ग्ने॒ ऽग्ने॒ ते॒ ते॒ अ॒ग्ने॒ स॒मिधः॑ स॒मिधो᳚ ऽग्ने ते ते अग्ने स॒मिधः॑ । \newline
36. अ॒ग्ने॒ स॒मिधः॑ स॒मिधो᳚ ऽग्ने ऽग्ने स॒मिधः॑ स॒प्त स॒प्त स॒मिधो᳚ ऽग्ने ऽग्ने स॒मिधः॑ स॒प्त । \newline
37. स॒मिधः॑ स॒प्त स॒प्त स॒मिधः॑ स॒मिधः॑ स॒प्त जि॒ह्वा जि॒ह्वाः स॒प्त स॒मिधः॑ स॒मिधः॑ स॒प्त जि॒ह्वाः । \newline
38. स॒मिध॒ इति॑ सं - इधः॑ । \newline
39. स॒प्त जि॒ह्वा जि॒ह्वाः स॒प्त स॒प्त जि॒ह्वा इतीति॑ जि॒ह्वाः स॒प्त स॒प्त जि॒ह्वा इति॑ । \newline
40. जि॒ह्वा इतीति॑ जि॒ह्वा जि॒ह्वा इत्या॑हा॒हेति॑ जि॒ह्वा जि॒ह्वा इत्या॑ह । \newline
41. इत्या॑हा॒हे तीत्या॑ह स॒प्त स॒प्ताहे तीत्या॑ह स॒प्त । \newline
42. आ॒ह॒ स॒प्त स॒प्ता हा॑ह स॒प्तै वैव स॒प्ता हा॑ह स॒प्तैव । \newline
43. स॒प्तै वैव स॒प्त स॒प्तै वास्या᳚ स्यै॒व स॒प्त स॒प्तै वास्य॑ । \newline
44. ए॒वास्या᳚ स्यै॒वै वास्य॒ साप्ता॑नि॒ साप्ता᳚ न्यस्यै॒ वैवास्य॒ साप्ता॑नि । \newline
45. अ॒स्य॒ साप्ता॑नि॒ साप्ता᳚ न्यस्यास्य॒ साप्ता॑नि प्रीणाति प्रीणाति॒ साप्ता᳚ न्यस्यास्य॒ साप्ता॑नि प्रीणाति । \newline
46. साप्ता॑नि प्रीणाति प्रीणाति॒ साप्ता॑नि॒ साप्ता॑नि प्रीणाति पू॒र्णया॑ पू॒र्णया᳚ प्रीणाति॒ साप्ता॑नि॒ साप्ता॑नि प्रीणाति पू॒र्णया᳚ । \newline
47. प्री॒णा॒ति॒ पू॒र्णया॑ पू॒र्णया᳚ प्रीणाति प्रीणाति पू॒र्णया॑ जुहोति जुहोति पू॒र्णया᳚ प्रीणाति प्रीणाति पू॒र्णया॑ जुहोति । \newline
48. पू॒र्णया॑ जुहोति जुहोति पू॒र्णया॑ पू॒र्णया॑ जुहोति पू॒र्णः पू॒र्णो जु॑होति पू॒र्णया॑ पू॒र्णया॑ जुहोति पू॒र्णः । \newline
49. जु॒हो॒ति॒ पू॒र्णः पू॒र्णो जु॑होति जुहोति पू॒र्ण इ॑वेव पू॒र्णो जु॑होति जुहोति पू॒र्ण इ॑व । \newline
50. पू॒र्ण इ॑वेव पू॒र्णः पू॒र्ण इ॑व॒ हि हीव॑ पू॒र्णः पू॒र्ण इ॑व॒ हि । \newline
51. इ॒व॒ हि हीवे॑व॒ हि प्र॒जाप॑तिः प्र॒जाप॑ति॒र्॒. हीवे॑व॒ हि प्र॒जाप॑तिः । \newline
52. हि प्र॒जाप॑तिः प्र॒जाप॑ति॒र्॒. हि हि प्र॒जाप॑तिः प्र॒जाप॑तेः प्र॒जाप॑तेः प्र॒जाप॑ति॒र्॒. हि हि प्र॒जाप॑तिः प्र॒जाप॑तेः । \newline
53. प्र॒जाप॑तिः प्र॒जाप॑तेः प्र॒जाप॑तेः प्र॒जाप॑तिः प्र॒जाप॑तिः प्र॒जाप॑ते॒ राप्त्या॒ आप्त्यै᳚ प्र॒जाप॑तेः प्र॒जाप॑तिः प्र॒जाप॑तिः प्र॒जाप॑ते॒ राप्त्यै᳚ । \newline
54. प्र॒जाप॑ति॒रिति॑ प्र॒जा - प॒तिः॒ । \newline
55. प्र॒जाप॑ते॒ राप्त्या॒ आप्त्यै᳚ प्र॒जाप॑तेः प्र॒जाप॑ते॒ राप्त्यै॒ न्यू॑नया॒ न्यू॑न॒या ऽऽप्त्यै᳚ प्र॒जाप॑तेः प्र॒जाप॑ते॒ राप्त्यै॒ न्यू॑नया । \newline
56. प्र॒जाप॑ते॒रिति॑ प्र॒जा - प॒तेः॒ । \newline
\pagebreak
\markright{ TS 5.4.7.6  \hfill https://www.vedavms.in \hfill}

\section{ TS 5.4.7.6 }

\textbf{TS 5.4.7.6 } \newline
\textbf{Samhita Paata} \newline

-राप्त्यै॒ न्यू॑नया जुहोति॒ न्यू॑ना॒द्धि प्र॒जाप॑तिः प्र॒जा असृ॑जत प्र॒जानाꣳ॒॒ सृष्ट्या॑ अ॒ग्निर्दे॒वेभ्यो॒ निला॑यत॒ स दिशोऽनु॒ प्राऽवि॑श॒ज्जुह्व॒न्मन॑सा॒ दिशो᳚ द्ध्याये द्दि॒ग्भ्य ए॒वैन॒मव॑ रुन्धे द॒द्ध्ना पु॒रस्ता᳚ज्जुहो॒त्या-ज्ये॑नो॒परि॑ष्टा॒त् तेज॑श्चै॒वास्मा॑ इन्द्रि॒यं च॑ स॒मीची॑ दधाति॒ द्वाद॑शकपालो वैश्वान॒रो भ॑वति॒ द्वाद॑श॒ मासाः᳚ संॅवथ्स॒रः सं॑ॅवथ्स॒रो᳚-ऽग्निर्वै᳚श्वान॒रः सा॒क्षा - [  ] \newline

\textbf{Pada Paata} \newline

आप्त्यै᳚ । न्यू॑न॒येति॒ नि-ऊ॒न॒या॒ । जु॒हो॒ति॒ । न्यू॑ना॒दिति॒ नि - ऊ॒ना॒त् । हि । प्र॒जाप॑ति॒रिति॑ प्र॒जा - प॒तिः॒ । प्र॒जा इति॑ प्र - जाः । असृ॑जत । प्र॒जाना॒मिति॑ प्र - जाना᳚म् । सृष्ट्यै᳚ । अ॒ग्निः । दे॒वेभ्यः॑ । निला॑यत । सः । दिशः॑ । अनु॑ । प्रेति॑ । अ॒वि॒श॒त् । जुह्व॑त् । मन॑सा । दिशः॑ । ध्या॒ये॒त् । दि॒ग्भ्य इति॑ दिक् - भ्यः । ए॒व । ए॒न॒म् । अवेति॑ । रु॒न्धे॒ । द॒द्ध्ना । पु॒रस्ता᳚त् । जु॒हो॒ति॒ । आज्ये॑न । उ॒परि॑ष्टात् । तेजः॑ । च॒ । ए॒व । अ॒स्मै॒ । इ॒न्द्रि॒यम् । च॒ । स॒मीची॒ इति॑ । द॒धा॒ति॒ । द्वाद॑शकपाल॒ इति॒ द्वाद॑श - क॒पा॒लः॒ । वै॒श्वा॒न॒रः । भ॒व॒ति॒ । द्वाद॑श । मासाः᳚ । सं॒ॅव॒थ्स॒र इति॑ सं-व॒थ्स॒रः । सं॒ॅव॒थ्स॒र इति॑ सं-व॒थ्स॒रः । अ॒ग्निः । वै॒श्वा॒न॒रः । सा॒क्षादिति॑ स - अ॒क्षात् ।  \newline


\textbf{Krama Paata} \newline

आप्त्यै॒ न्यू॑नया । न्यू॑नया जुहोति । न्यू॑न॒येति॒ नि - ऊ॒न॒या॒ । जु॒हो॒ति॒ न्यू॑नात् । न्यू॑ना॒द्धि । न्यू॑ना॒दिति॒ नि - ऊ॒ना॒त्॒ । हि प्र॒जाप॑तिः । प्र॒जाप॑तिः प्र॒जाः । प्र॒जाप॑ति॒रिति॑ प्र॒जा - प॒तिः॒ । प्र॒जा असृ॑जत । प्र॒जा इति॑ प्र - जाः । असृ॑जत प्र॒जाना᳚म् । प्र॒जानाꣳ॒॒ सृष्ट्यै᳚ । प्र॒जाना॒मिति॑ प्र - जाना᳚म् । सृष्ट्या॑ अ॒ग्निः । अ॒ग्निर् दे॒वेभ्यः॑ । दे॒वेभ्यो॒ निला॑यत । निला॑यत॒ सः । स दिशः॑ । दिशोऽनु॑ । अनु॒ प्र । प्रावि॑शत् । अ॒वि॒श॒ज् जुह्व॑त् । जुह्व॒न् मन॑सा । मन॑सा॒ दिशः॑ । दिशो᳚ ध्यायेत् । ध्या॒ये॒द् दि॒ग्भ्यः । दि॒ग्भ्य ए॒व । दि॒ग्भ्य इति॑ दिक् - भ्यः । ए॒वैन᳚म् । ए॒न॒मव॑ । अव॑ रुन्धे । रु॒न्धे॒ द॒द्ध्ना । द॒द्ध्ना पु॒रस्ता᳚त् । पु॒रस्ता᳚ज् जुहोति । जु॒हो॒त्याज्ये॑न । आज्ये॑नो॒परि॑ष्टात् । उ॒परि॑ष्टा॒त् तेजः॑ । तेज॑श्च । चै॒व । ए॒वास्मै᳚ । अ॒स्मा॒ इ॒न्द्रि॒यम् । इ॒न्द्रि॒यम् च॑ । च॒ स॒मीची᳚ । स॒मीची॑ दधाति । स॒मीची॒ इति॑ स॒मीची᳚ । द॒धा॒ति॒ द्वाद॑शकपालः । द्वाद॑शकपालो वैश्वान॒रः । द्वाद॑शकपाल॒ इति॒ द्वाद॑श - क॒पा॒लः॒ । वै॒श्वा॒न॒रो भ॑वति । भ॒व॒ति॒ द्वाद॑श । द्वाद॑श॒ मासाः᳚ । मासाः᳚ सम्ॅवथ्स॒रः । स॒म्ॅव॒थ्स॒रः स॑म्ॅवथ्स॒रः । स॒म्ॅव॒थ्स॒र इति॑ सम् - व॒थ्स॒रः । स॒म्ॅव॒थ्स॒रो᳚ऽग्निः । स॒म्ॅव॒थ्स॒र इति॑ सम् - व॒थ्स॒रः । अ॒ग्निर् वै᳚श्वान॒रः । वै॒श्वा॒न॒रः सा॒क्षात् । सा॒क्षादे॒व । सा॒क्षादिति॑ स - अ॒क्षात् \newline

\textbf{Jatai Paata} \newline

1. आप्त्यै॒ न्यू॑नया॒ न्यू॑न॒या ऽऽप्त्या॒ आप्त्यै॒ न्यू॑नया । \newline
2. न्यू॑नया जुहोति जुहोति॒ न्यू॑नया॒ न्यू॑नया जुहोति । \newline
3. न्यू॑न॒येति॒ नि - ऊ॒न॒या॒ । \newline
4. जु॒हो॒ति॒ न्यू॑ना॒न् न्यू॑नाज् जुहोति जुहोति॒ न्यू॑नात् । \newline
5. न्यू॑ना॒द्धि हि न्यू॑ना॒न् न्यू॑ना॒द्धि । \newline
6. न्यू॑ना॒दिति॒ नि - ऊ॒ना॒त् । \newline
7. हि प्र॒जाप॑तिः प्र॒जाप॑ति॒र्॒. हि हि प्र॒जाप॑तिः । \newline
8. प्र॒जाप॑तिः प्र॒जाः प्र॒जाः प्र॒जाप॑तिः प्र॒जाप॑तिः प्र॒जाः । \newline
9. प्र॒जाप॑ति॒रिति॑ प्र॒जा - प॒तिः॒ । \newline
10. प्र॒जा असृ॑ज॒ता सृ॑जत प्र॒जाः प्र॒जा असृ॑जत । \newline
11. प्र॒जा इति॑ प्र - जाः । \newline
12. असृ॑जत प्र॒जाना᳚म् प्र॒जाना॒ मसृ॑ज॒ता सृ॑जत प्र॒जाना᳚म् । \newline
13. प्र॒जानाꣳ॒॒ सृष्ट्यै॒ सृष्ट्यै᳚ प्र॒जाना᳚म् प्र॒जानाꣳ॒॒ सृष्ट्यै᳚ । \newline
14. प्र॒जाना॒मिति॑ प्र - जाना᳚म् । \newline
15. सृष्ट्या॑ अ॒ग्नि र॒ग्निः सृष्ट्यै॒ सृष्ट्या॑ अ॒ग्निः । \newline
16. अ॒ग्निर् दे॒वेभ्यो॑ दे॒वेभ्यो॒ ऽग्नि र॒ग्निर् दे॒वेभ्यः॑ । \newline
17. दे॒वेभ्यो॒ निला॑यत॒ निला॑यत दे॒वेभ्यो॑ दे॒वेभ्यो॒ निला॑यत । \newline
18. निला॑यत॒ स स निला॑यत॒ निला॑यत॒ सः । \newline
19. स दिशो॒ दिशः॒ स स दिशः॑ । \newline
20. दिशो ऽन्वनु॒ दिशो॒ दिशो ऽनु॑ । \newline
21. अनु॒ प्र प्राण्वनु॒ प्र । \newline
22. प्रावि॑श दविश॒त् प्र प्रावि॑शत् । \newline
23. अ॒वि॒श॒ज् जुह्व॒ज् जुह्व॑ दविश दविश॒ज् जुह्व॑त् । \newline
24. जुह्व॒न् मन॑सा॒ मन॑सा॒ जुह्व॒ज् जुह्व॒न् मन॑सा । \newline
25. मन॑सा॒ दिशो॒ दिशो॒ मन॑सा॒ मन॑सा॒ दिशः॑ । \newline
26. दिशो᳚ ध्यायेद् ध्याये॒द् दिशो॒ दिशो᳚ ध्यायेत् । \newline
27. ध्या॒ये॒द् दि॒ग्भ्यो दि॒ग्भ्यो ध्या॑येद् ध्यायेद् दि॒ग्भ्यः । \newline
28. दि॒ग्भ्य ए॒वैव दि॒ग्भ्यो दि॒ग्भ्य ए॒व । \newline
29. दि॒ग्भ्य इति॑ दिक् - भ्यः । \newline
30. ए॒वैन॑ मेन मे॒वै वैन᳚म् । \newline
31. ए॒न॒ मवा वै॑न मेन॒ मव॑ । \newline
32. अव॑ रुन्धे रु॒न्धे ऽवाव॑ रुन्धे । \newline
33. रु॒न्धे॒ द॒द्ध्ना द॒द्ध्ना रु॑न्धे रुन्धे द॒द्ध्ना । \newline
34. द॒द्ध्ना पु॒रस्ता᳚त् पु॒रस्ता᳚द् द॒द्ध्ना द॒द्ध्ना पु॒रस्ता᳚त् । \newline
35. पु॒रस्ता᳚ज् जुहोति जुहोति पु॒रस्ता᳚त् पु॒रस्ता᳚ज् जुहोति । \newline
36. जु॒हो॒ त्याज्ये॒ना ज्ये॑न जुहोति जुहो॒ त्याज्ये॑न । \newline
37. आज्ये॑नो॒ परि॑ष्टा दु॒परि॑ष्टा॒ दाज्ये॒ना ज्ये॑नो॒ परि॑ष्टात् । \newline
38. उ॒परि॑ष्टा॒त् तेज॒ स्तेज॑ उ॒परि॑ष्टा दु॒परि॑ष्टा॒त् तेजः॑ । \newline
39. तेज॑श्च च॒ तेज॒ स्तेज॑ श्च । \newline
40. चै॒वैव च॑ चै॒व । \newline
41. ए॒वास्मा॑ अस्मा ए॒वै वास्मै᳚ । \newline
42. अ॒स्मा॒ इ॒न्द्रि॒य मि॑न्द्रि॒य म॑स्मा अस्मा इन्द्रि॒यम् । \newline
43. इ॒न्द्रि॒यम् च॑ चेन्द्रि॒य मि॑न्द्रि॒यम् च॑ । \newline
44. च॒ स॒मीची॑ स॒मीची॑ च च स॒मीची᳚ । \newline
45. स॒मीची॑ दधाति दधाति स॒मीची॑ स॒मीची॑ दधाति । \newline
46. स॒मीची॒ इति॑ स॒मीची᳚ । \newline
47. द॒धा॒ति॒ द्वाद॑शकपालो॒ द्वाद॑शकपालो दधाति दधाति॒ द्वाद॑शकपालः । \newline
48. द्वाद॑शकपालो वैश्वान॒रो वै᳚श्वान॒रो द्वाद॑शकपालो॒ द्वाद॑शकपालो वैश्वान॒रः । \newline
49. द्वाद॑शकपाल॒ इति॒ द्वाद॑श - क॒पा॒लः॒ । \newline
50. वै॒श्वा॒न॒रो भ॑वति भवति वैश्वान॒रो वै᳚श्वान॒रो भ॑वति । \newline
51. भ॒व॒ति॒ द्वाद॑श॒ द्वाद॑श भवति भवति॒ द्वाद॑श । \newline
52. द्वाद॑श॒ मासा॒ मासा॒ द्वाद॑श॒ द्वाद॑श॒ मासाः᳚ । \newline
53. मासाः᳚ संॅवथ्स॒रः सं॑ॅवथ्स॒रो मासा॒ मासाः᳚ संॅवथ्स॒रः । \newline
54. सं॒ॅव॒थ्स॒रः सं॑ॅवथ्स॒रः । \newline
55. सं॒ॅव॒थ्स॒र इति॑ सं - व॒थ्स॒रः । \newline
56. सं॒ॅव॒थ्स॒रो᳚ ऽग्नि र॒ग्निः सं॑ॅवथ्स॒रः सं॑ॅवथ्स॒रो᳚ ऽग्निः । \newline
57. सं॒ॅव॒थ्स॒र इति॑ सं - व॒थ्स॒रः । \newline
58. अ॒ग्निर् वै᳚श्वान॒रो वै᳚श्वान॒रो᳚ ऽग्नि र॒ग्निर् वै᳚श्वान॒रः । \newline
59. वै॒श्वा॒न॒रः सा॒क्षाथ् सा॒क्षाद् वै᳚श्वान॒रो वै᳚श्वान॒रः सा॒क्षात् । \newline
60. सा॒क्षा दे॒वैव सा॒क्षाथ् सा॒क्षा दे॒व । \newline
61. सा॒क्षादिति॑ स - अ॒क्षात् । \newline

\textbf{Ghana Paata } \newline

1. आप्त्यै॒ न्यू॑नया॒ न्यू॑न॒या ऽऽप्त्या॒ आप्त्यै॒ न्यू॑नया जुहोति जुहोति॒ न्यू॑न॒या ऽऽप्त्या॒ आप्त्यै॒ न्यू॑नया जुहोति । \newline
2. न्यू॑नया जुहोति जुहोति॒ न्यू॑नया॒ न्यू॑नया जुहोति॒ न्यू॑ना॒न् न्यू॑नाज् जुहोति॒ न्यू॑नया॒ न्यू॑नया जुहोति॒ न्यू॑नात् । \newline
3. न्यू॑न॒येति॒ नि - ऊ॒न॒या॒ । \newline
4. जु॒हो॒ति॒ न्यू॑ना॒न् न्यू॑नाज् जुहोति जुहोति॒ न्यू॑ना॒द्धि हि न्यू॑नाज् जुहोति जुहोति॒ न्यू॑ना॒द्धि । \newline
5. न्यू॑ना॒द्धि हि न्यू॑ना॒न् न्यू॑ना॒द्धि प्र॒जाप॑तिः प्र॒जाप॑ति॒र्॒. हि न्यू॑ना॒न् न्यू॑ना॒द्धि प्र॒जाप॑तिः । \newline
6. न्यू॑ना॒दिति॒ नि - ऊ॒ना॒त् । \newline
7. हि प्र॒जाप॑तिः प्र॒जाप॑ति॒र्॒. हि हि प्र॒जाप॑तिः प्र॒जाः प्र॒जाः प्र॒जाप॑ति॒र्॒. हि हि प्र॒जाप॑तिः प्र॒जाः । \newline
8. प्र॒जाप॑तिः प्र॒जाः प्र॒जाः प्र॒जाप॑तिः प्र॒जाप॑तिः प्र॒जा असृ॑ज॒ता सृ॑जत प्र॒जाः प्र॒जाप॑तिः प्र॒जाप॑तिः प्र॒जा असृ॑जत । \newline
9. प्र॒जाप॑ति॒रिति॑ प्र॒जा - प॒तिः॒ । \newline
10. प्र॒जा असृ॑ज॒ता सृ॑जत प्र॒जाः प्र॒जा असृ॑जत प्र॒जाना᳚म् प्र॒जाना॒ मसृ॑जत प्र॒जाः प्र॒जा असृ॑जत प्र॒जाना᳚म् । \newline
11. प्र॒जा इति॑ प्र - जाः । \newline
12. असृ॑जत प्र॒जाना᳚म् प्र॒जाना॒ मसृ॑ज॒ता सृ॑जत प्र॒जानाꣳ॒॒ सृष्ट्यै॒ सृष्ट्यै᳚ प्र॒जाना॒ मसृ॑ज॒ता सृ॑जत प्र॒जानाꣳ॒॒ सृष्ट्यै᳚ । \newline
13. प्र॒जानाꣳ॒॒ सृष्ट्यै॒ सृष्ट्यै᳚ प्र॒जाना᳚म् प्र॒जानाꣳ॒॒ सृष्ट्या॑ अ॒ग्नि र॒ग्निः सृष्ट्यै᳚ प्र॒जाना᳚म् प्र॒जानाꣳ॒॒ सृष्ट्या॑ अ॒ग्निः । \newline
14. प्र॒जाना॒मिति॑ प्र - जाना᳚म् । \newline
15. सृष्ट्या॑ अ॒ग्नि र॒ग्निः सृष्ट्यै॒ सृष्ट्या॑ अ॒ग्निर् दे॒वेभ्यो॑ दे॒वेभ्यो॒ ऽग्निः सृष्ट्यै॒ सृष्ट्या॑ अ॒ग्निर् दे॒वेभ्यः॑ । \newline
16. अ॒ग्निर् दे॒वेभ्यो॑ दे॒वेभ्यो॒ ऽग्नि र॒ग्निर् दे॒वेभ्यो॒ निला॑यत॒ निला॑यत दे॒वेभ्यो॒ ऽग्नि र॒ग्निर् दे॒वेभ्यो॒ निला॑यत । \newline
17. दे॒वेभ्यो॒ निला॑यत॒ निला॑यत दे॒वेभ्यो॑ दे॒वेभ्यो॒ निला॑यत॒ स स निला॑यत दे॒वेभ्यो॑ दे॒वेभ्यो॒ निला॑यत॒ सः । \newline
18. निला॑यत॒ स स निला॑यत॒ निला॑यत॒ स दिशो॒ दिशः॒ स निला॑यत॒ निला॑यत॒ स दिशः॑ । \newline
19. स दिशो॒ दिशः॒ स स दिशो ऽन्वनु॒ दिशः॒ स स दिशो ऽनु॑ । \newline
20. दिशो ऽन्वनु॒ दिशो॒ दिशो ऽनु॒ प्र प्राणु॒ दिशो॒ दिशो ऽनु॒ प्र । \newline
21. अनु॒ प्र प्राण्वनु॒ प्रावि॑श दविश॒त् प्राण्वनु॒ प्रावि॑शत् । \newline
22. प्रावि॑श दविश॒त् प्र प्रावि॑श॒ज् जुह्व॒ज् जुह्व॑ दविश॒त् प्र प्रावि॑श॒ज् जुह्व॑त् । \newline
23. अ॒वि॒श॒ज् जुह्व॒ज् जुह्व॑ दविश दविश॒ज् जुह्व॒न् मन॑सा॒ मन॑सा॒ जुह्व॑ दविश दविश॒ज् जुह्व॒न् मन॑सा । \newline
24. जुह्व॒न् मन॑सा॒ मन॑सा॒ जुह्व॒ज् जुह्व॒न् मन॑सा॒ दिशो॒ दिशो॒ मन॑सा॒ जुह्व॒ज् जुह्व॒न् मन॑सा॒ दिशः॑ । \newline
25. मन॑सा॒ दिशो॒ दिशो॒ मन॑सा॒ मन॑सा॒ दिशो᳚ ध्यायेद् ध्याये॒द् दिशो॒ मन॑सा॒ मन॑सा॒ दिशो᳚ ध्यायेत् । \newline
26. दिशो᳚ ध्यायेद् ध्याये॒द् दिशो॒ दिशो᳚ ध्यायेद् दि॒ग्भ्यो दि॒ग्भ्यो ध्या॑ये॒द् दिशो॒ दिशो᳚ ध्यायेद् दि॒ग्भ्यः । \newline
27. ध्या॒ये॒द् दि॒ग्भ्यो दि॒ग्भ्यो ध्या॑येद् ध्यायेद् दि॒ग्भ्य ए॒वैव दि॒ग्भ्यो ध्या॑येद् ध्यायेद् दि॒ग्भ्य ए॒व । \newline
28. दि॒ग्भ्य ए॒वैव दि॒ग्भ्यो दि॒ग्भ्य ए॒वैन॑ मेन मे॒व दि॒ग्भ्यो दि॒ग्भ्य ए॒वैन᳚म् । \newline
29. दि॒ग्भ्य इति॑ दिक् - भ्यः । \newline
30. ए॒वैन॑ मेन मे॒वै वैन॒ मवावै॑न मे॒वै वैन॒ मव॑ । \newline
31. ए॒न॒ मवावै॑न मेन॒ मव॑ रुन्धे रु॒न्धे ऽवै॑न मेन॒ मव॑ रुन्धे । \newline
32. अव॑ रुन्धे रु॒न्धे ऽवाव॑ रुन्धे द॒द्ध्ना द॒द्ध्ना रु॒न्धे ऽवाव॑ रुन्धे द॒द्ध्ना । \newline
33. रु॒न्धे॒ द॒द्ध्ना द॒द्ध्ना रु॑न्धे रुन्धे द॒द्ध्ना पु॒रस्ता᳚त् पु॒रस्ता᳚द् द॒द्ध्ना रु॑न्धे रुन्धे द॒द्ध्ना पु॒रस्ता᳚त् । \newline
34. द॒द्ध्ना पु॒रस्ता᳚त् पु॒रस्ता᳚द् द॒द्ध्ना द॒द्ध्ना पु॒रस्ता᳚ज् जुहोति जुहोति पु॒रस्ता᳚द् द॒द्ध्ना द॒द्ध्ना पु॒रस्ता᳚ज् जुहोति । \newline
35. पु॒रस्ता᳚ज् जुहोति जुहोति पु॒रस्ता᳚त् पु॒रस्ता᳚ज् जुहो॒ त्याज्ये॒ नाज्ये॑न जुहोति पु॒रस्ता᳚त् पु॒रस्ता᳚ज् जुहो॒ त्याज्ये॑न । \newline
36. जु॒हो॒ त्याज्ये॒ना ज्ये॑न जुहोति जुहो॒ त्याज्ये॑ नो॒परि॑ष्टा दु॒परि॑ष्टा॒ दाज्ये॑न जुहोति जुहो॒ त्याज्ये॑ नो॒परि॑ष्टात् । \newline
37. आज्ये॑ नो॒परि॑ष्टा दु॒परि॑ष्टा॒ दाज्ये॒ नाज्ये॑ नो॒परि॑ष्टा॒त् तेज॒ स्तेज॑ उ॒परि॑ष्टा॒ दाज्ये॒ नाज्ये॑ नो॒परि॑ष्टा॒त् तेजः॑ । \newline
38. उ॒परि॑ष्टा॒त् तेज॒ स्तेज॑ उ॒परि॑ष्टा दु॒परि॑ष्टा॒त् तेज॑श्च च॒ तेज॑ उ॒परि॑ष्टा दु॒परि॑ष्टा॒त् तेज॑श्च । \newline
39. तेज॑श्च च॒ तेज॒ स्तेज॑ श्चै॒वैव च॒ तेज॒ स्तेज॑ श्चै॒व । \newline
40. चै॒वैव च॑ चै॒वास्मा॑ अस्मा ए॒व च॑ चै॒वास्मै᳚ । \newline
41. ए॒वास्मा॑ अस्मा ए॒वै वास्मा॑ इन्द्रि॒य मि॑न्द्रि॒य म॑स्मा ए॒वै वास्मा॑ इन्द्रि॒यम् । \newline
42. अ॒स्मा॒ इ॒न्द्रि॒य मि॑न्द्रि॒य म॑स्मा अस्मा इन्द्रि॒यम् च॑ चेन्द्रि॒य म॑स्मा अस्मा इन्द्रि॒यम् च॑ । \newline
43. इ॒न्द्रि॒यम् च॑ चेन्द्रि॒य मि॑न्द्रि॒यम् च॑ स॒मीची॑ स॒मीची॑ चेन्द्रि॒य मि॑न्द्रि॒यम् च॑ स॒मीची᳚ । \newline
44. च॒ स॒मीची॑ स॒मीची॑ च च स॒मीची॑ दधाति दधाति स॒मीची॑ च च स॒मीची॑ दधाति । \newline
45. स॒मीची॑ दधाति दधाति स॒मीची॑ स॒मीची॑ दधाति॒ द्वाद॑शकपालो॒ द्वाद॑शकपालो दधाति स॒मीची॑ स॒मीची॑ दधाति॒ द्वाद॑शकपालः । \newline
46. स॒मीची॒ इति॑ स॒मीची᳚ । \newline
47. द॒धा॒ति॒ द्वाद॑शकपालो॒ द्वाद॑शकपालो दधाति दधाति॒ द्वाद॑शकपालो वैश्वान॒रो वै᳚श्वान॒रो द्वाद॑शकपालो दधाति दधाति॒ द्वाद॑शकपालो वैश्वान॒रः । \newline
48. द्वाद॑शकपालो वैश्वान॒रो वै᳚श्वान॒रो द्वाद॑शकपालो॒ द्वाद॑शकपालो वैश्वान॒रो भ॑वति भवति वैश्वान॒रो द्वाद॑शकपालो॒ द्वाद॑शकपालो वैश्वान॒रो भ॑वति । \newline
49. द्वाद॑शकपाल॒ इति॒ द्वाद॑श - क॒पा॒लः॒ । \newline
50. वै॒श्वा॒न॒रो भ॑वति भवति वैश्वान॒रो वै᳚श्वान॒रो भ॑वति॒ द्वाद॑श॒ द्वाद॑श भवति वैश्वान॒रो वै᳚श्वान॒रो भ॑वति॒ द्वाद॑श । \newline
51. भ॒व॒ति॒ द्वाद॑श॒ द्वाद॑श भवति भवति॒ द्वाद॑श॒ मासा॒ मासा॒ द्वाद॑श भवति भवति॒ द्वाद॑श॒ मासाः᳚ । \newline
52. द्वाद॑श॒ मासा॒ मासा॒ द्वाद॑श॒ द्वाद॑श॒ मासाः᳚ संॅवथ्स॒रः सं॑ॅवथ्स॒रो मासा॒ द्वाद॑श॒ द्वाद॑श॒ मासाः᳚ संॅवथ्स॒रः । \newline
53. मासाः᳚ संॅवथ्स॒रः सं॑ॅवथ्स॒रो मासा॒ मासाः᳚ संॅवथ्स॒रः । \newline
54. सं॒ॅव॒थ्स॒रः सं॑ॅवथ्स॒रः । \newline
55. सं॒ॅव॒थ्स॒र इति॑ सं - व॒थ्स॒रः । \newline
56. सं॒ॅव॒थ्स॒रो᳚ ऽग्नि र॒ग्निः सं॑ॅवथ्स॒रः सं॑ॅवथ्स॒रो᳚ ऽग्निर् वै᳚श्वान॒रो वै᳚श्वान॒रो᳚ ऽग्निः सं॑ॅवथ्स॒रः सं॑ॅवथ्स॒रो᳚ ऽग्निर् वै᳚श्वान॒रः । \newline
57. सं॒ॅव॒थ्स॒र इति॑ सं - व॒थ्स॒रः । \newline
58. अ॒ग्निर् वै᳚श्वान॒रो वै᳚श्वान॒रो᳚ ऽग्नि र॒ग्निर् वै᳚श्वान॒रः सा॒क्षाथ् सा॒क्षाद् वै᳚श्वान॒रो᳚ ऽग्नि र॒ग्निर् वै᳚श्वान॒रः सा॒क्षात् । \newline
59. वै॒श्वा॒न॒रः सा॒क्षाथ् सा॒क्षाद् वै᳚श्वान॒रो वै᳚श्वान॒रः सा॒क्षा दे॒वैव सा॒क्षाद् वै᳚श्वान॒रो वै᳚श्वान॒रः सा॒क्षा दे॒व । \newline
60. सा॒क्षा दे॒वैव सा॒क्षाथ् सा॒क्षा दे॒व वै᳚श्वान॒रं ॅवै᳚श्वान॒र मे॒व सा॒क्षाथ् सा॒क्षा दे॒व वै᳚श्वान॒रम् । \newline
61. सा॒क्षादिति॑ स - अ॒क्षात् । \newline
\pagebreak
\markright{ TS 5.4.7.7  \hfill https://www.vedavms.in \hfill}

\section{ TS 5.4.7.7 }

\textbf{TS 5.4.7.7 } \newline
\textbf{Samhita Paata} \newline

-दे॒व वै᳚श्वान॒रमव॑ रुन्धे॒ यत् प्र॑याजानूया॒जान् कु॒र्याद्विक॑स्तिः॒ सा य॒ज्ञ्स्य॑ दर्विहो॒मं क॑रोति य॒ज्ञ्स्य॒ प्रति॑ष्ठित्यै रा॒ष्ट्रं ॅवै वै᳚श्वान॒रो विण्म॒रुतो॑ वैश्वान॒रꣳ हु॒त्वा मा॑रु॒तान् जु॑होति रा॒ष्ट्र ए॒व विश॒मनु॑ बद्ध्नात्यु॒च्चै-र्वै᳚श्वान॒रस्याऽऽ श्रा॑वयत्युपाꣳ॒॒शु मा॑रु॒तान् जु॑होति॒ तस्मा᳚द्-रा॒ष्ट्रं ॅविश॒मति॑ वदति मारु॒ता भ॑वन्ति म॒रुतो॒ वै दे॒वानां॒ ॅविशो॑ देववि॒शेनै॒वास्मै॑ मनुष्यवि॒श ( ) -मव॑ रुन्धे स॒प्त भ॑वन्ति स॒प्तग॑णा॒ वै म॒रुतो॑ गण॒श ए॒व विश॒मव॑ रुन्धे ग॒णेन॑ ग॒णम॑नु॒द्रुत्य॑ जुहोति॒ विश॑मे॒वास्मा॒ अनु॑वर्त्मानं करोति ॥ \newline

\textbf{Pada Paata} \newline

ए॒व । वै॒श्वा॒न॒रम् । अवेति॑ । रु॒न्धे॒ । यत् । प्र॒या॒जा॒नू॒या॒जानिति॑ प्रयाज - अ॒नू॒या॒जान् । कु॒र्यात् । विक॑स्ति॒रिति॒ वि - क॒स्तिः॒ । सा । य॒ज्ञ्स्य॑ । द॒र्वि॒हो॒ममिति॑ दर्वि - हो॒मम् । क॒रो॒ति॒ । य॒ज्ञ्स्य॑ । प्रति॑ष्ठित्या॒ इति॒ प्रति॑ - स्थि॒त्यै॒ । रा॒ष्ट्रम् । वै । वै॒श्वा॒न॒रः । विट् । म॒रुतः॑ । वै॒श्वा॒न॒रम् । हु॒त्वा । मा॒रु॒ताम् । जु॒हो॒ति॒ । रा॒ष्ट्रे । ए॒व । विश᳚म् । अन्विति॑ । ब॒द्ध्ना॒ति॒ । उ॒च्चैः । वै॒श्वा॒न॒रस्य॑ । एति॑ । श्रा॒व॒य॒ति॒ । उ॒पाꣳ॒॒श्वित्यु॑प - अꣳ॒॒शु । मा॒रु॒तान् । जु॒हो॒ति॒ । तस्मा᳚त् । रा॒ष्ट्रम् । विश᳚म् । अतीति॑ । व॒द॒ति॒ । मा॒रु॒ताः । भ॒व॒न्ति॒ । म॒रुतः॑ । वै । दे॒वाना᳚म् । विशः॑ । दे॒व॒वि॒शेनेति॑ देव - वि॒शेन॑ । ए॒व । अ॒स्मै॒ । म॒नु॒ष्य॒वि॒शमिति॑ मनुष्य - वि॒शम् ( ) । अवेति॑ । रु॒न्धे॒ । स॒प्त । भ॒व॒न्ति॒ । स॒प्तग॑णा॒ इति॑ स॒प्त - ग॒णाः॒ । वै । म॒रुतः॑ । ग॒ण॒श इति॑ गण - शः । ए॒व । विश᳚म् । अवेति॑ । रु॒न्धे॒ । ग॒णेन॑ । ग॒णम् । अ॒नु॒द्रुत्येत्य॑नु - द्रुत्य॑ । जु॒हो॒ति॒ । विश᳚म् । ए॒व । अ॒स्मै॒ । अनु॑वर्त्मान॒मित्यनु॑ - व॒र्त्मा॒न॒म् । क॒रो॒ति॒ ॥  \newline


\textbf{Krama Paata} \newline

ए॒व वै᳚श्वान॒रम् । वै॒श्वा॒न॒रमव॑ । अव॑ रुन्धे । रु॒न्धे॒ यत् । यत् प्र॑याजानूया॒जान् । प्र॒या॒जा॒नू॒या॒जान् कु॒र्यात् । प्र॒या॒जा॒नू॒या॒जानिति॑ प्रयाज - अ॒नू॒या॒जान् । कु॒र्याद् विक॑स्तिः । विक॑स्तिः॒ सा । विक॑स्ति॒रिति॒ वि - क॒स्तिः॒ । सा य॒ज्ञ्स्य॑ । य॒ज्ञ्स्य॑ दर्विहो॒मम् । द॒र्वि॒हो॒मम् क॑रोति । द॒र्वि॒हो॒ममिति॑ दर्वि - हो॒मम् । क॒रो॒ति॒ य॒ज्ञ्स्य॑ । य॒ज्ञ्स्य॒ प्रति॑ष्ठित्यै । प्रति॑ष्ठित्यै रा॒ष्ट्रम् । प्रति॑ष्ठित्या॒ इति॒ प्रति॑ - स्थि॒त्यै॒ । रा॒ष्ट्रम् ॅवै । वै वै᳚श्वान॒रः । वै॒श्वा॒न॒रो विट् । विण् म॒रुतः॑ । म॒रुतो॑ वैश्वान॒रम् । वै॒श्वा॒न॒रꣳ हु॒त्वा । हु॒त्वा मा॑रु॒तान् । मा॒रु॒तान् जु॑होति । जु॒हो॒ति॒ रा॒ष्ट्रे । रा॒ष्ट्र ए॒व । ए॒व विश᳚म् । विश॒मनु॑ । अनु॑ बद्ध्नाति । ब॒द्ध्ना॒त्यु॒च्चैः । उ॒च्चैर् वै᳚श्वान॒रस्य॑ । वै॒श्वा॒न॒रस्या । आ श्रा॑वयति । श्रा॒व॒य॒त्यु॒पाꣳ॒॒शु । उ॒पाꣳ॒॒शु मा॑रु॒तान् । उ॒पाꣳ॒॒श्वित्यु॑प - अꣳ॒॒शु । मा॒रु॒तान् जु॑होति । जु॒हो॒ति॒ तस्मा᳚त् । तस्मा᳚द् रा॒ष्ट्रम् । रा॒ष्ट्रम् ॅविश᳚म् । विश॒मति॑ । अति॑ वदति । व॒द॒ति॒ मा॒रु॒ताः । मा॒रु॒ता भ॑वन्ति । भ॒व॒न्ति॒ म॒रुतः॑ । म॒रुतो॒ वै । वै दे॒वाना᳚म् । दे॒वाना॒म् ॅविशः॑ । विशो॑ देववि॒शेन॑ । दे॒व॒वि॒शेनै॒व । दे॒व॒वि॒शेनेति॑ देव - वि॒शेन॑ । ए॒वास्मै᳚ । अ॒स्मै॒ म॒नु॒ष्य॒वि॒शम् ( ) । म॒नु॒ष्य॒वि॒शमव॑ । म॒नु॒ष्य॒वि॒शमिति॑ मनुष्य - वि॒शम् । अव॑ रुन्धे । रु॒न्धे॒ स॒प्त । स॒प्त भ॑वन्ति । भ॒व॒न्ति॒ स॒प्तग॑णाः । स॒प्तग॑णा॒ वै । स॒प्तग॑णा॒ इति॑ स॒प्त - ग॒णाः॒ । वै म॒रुतः॑ । म॒रुतो॑ गण॒शः । ग॒ण॒श ए॒व । ग॒ण॒श इति॑ गण - शः । ए॒व विश᳚म् । विश॒मव॑ । अव॑ रुन्धे । रु॒न्धे॒ ग॒णेन॑ । ग॒णेन॑ ग॒णम् । ग॒णम॑नु॒द्रुत्य॑ । अ॒नु॒द्रुत्य॑ जुहोति । अ॒नु॒द्रुत्येत्य॑नु - द्रुत्य॑ । जु॒हो॒ति॒ विश᳚म् । विश॑मे॒व । ए॒वास्मै᳚ । अ॒स्मा॒ अनु॑वर्त्मानम् । अनु॑वर्त्मानम् करोति । अनु॑वर्त्मान॒मित्यनु॑ - व॒र्त्मा॒न॒म् । क॒रो॒तीति॑ करोति । \newline

\textbf{Jatai Paata} \newline

1. ए॒व वै᳚श्वान॒रं ॅवै᳚श्वान॒र मे॒वैव वै᳚श्वान॒रम् । \newline
2. वै॒श्वा॒न॒र मवाव॑ वैश्वान॒रं ॅवै᳚श्वान॒र मव॑ । \newline
3. अव॑ रुन्धे रु॒न्धे ऽवाव॑ रुन्धे । \newline
4. रु॒न्धे॒ यद् यद् रु॑न्धे रुन्धे॒ यत् । \newline
5. यत् प्र॑याजानूया॒जान् प्र॑याजानूया॒जान्. यद् यत् प्र॑याजानूया॒जान्न् । \newline
6. प्र॒या॒जा॒नू॒या॒जान् कु॒र्यात् कु॒र्यात् प्र॑याजानूया॒जान् प्र॑याजानूया॒जान् कु॒र्यात् । \newline
7. प्र॒या॒जा॒नू॒या॒जानिति॑ प्रयाज - अ॒नू॒या॒जान् । \newline
8. कु॒र्याद् विक॑स्ति॒र् विक॑स्तिः कु॒र्यात् कु॒र्याद् विक॑स्तिः । \newline
9. विक॑स्तिः॒ सा सा विक॑स्ति॒र् विक॑स्तिः॒ सा । \newline
10. विक॑स्ति॒रिति॒ वि - क॒स्तिः॒ । \newline
11. सा य॒ज्ञ्स्य॑ य॒ज्ञ्स्य॒ सा सा य॒ज्ञ्स्य॑ । \newline
12. य॒ज्ञ्स्य॑ दर्विहो॒मम् द॑र्विहो॒मं ॅय॒ज्ञ्स्य॑ य॒ज्ञ्स्य॑ दर्विहो॒मम् । \newline
13. द॒र्वि॒हो॒मम् क॑रोति करोति दर्विहो॒मम् द॑र्विहो॒मम् क॑रोति । \newline
14. द॒र्वि॒हो॒ममिति॑ दर्वि - हो॒मम् । \newline
15. क॒रो॒ति॒ य॒ज्ञ्स्य॑ य॒ज्ञ्स्य॑ करोति करोति य॒ज्ञ्स्य॑ । \newline
16. य॒ज्ञ्स्य॒ प्रति॑ष्ठित्यै॒ प्रति॑ष्ठित्यै य॒ज्ञ्स्य॑ य॒ज्ञ्स्य॒ प्रति॑ष्ठित्यै । \newline
17. प्रति॑ष्ठित्यै रा॒ष्ट्रꣳ रा॒ष्ट्रम् प्रति॑ष्ठित्यै॒ प्रति॑ष्ठित्यै रा॒ष्ट्रम् । \newline
18. प्रति॑ष्ठित्या॒ इति॒ प्रति॑ - स्थि॒त्यै॒ । \newline
19. रा॒ष्ट्रं ॅवै वै रा॒ष्ट्रꣳ रा॒ष्ट्रं ॅवै । \newline
20. वै वै᳚श्वान॒रो वै᳚श्वान॒रो वै वै वै᳚श्वान॒रः । \newline
21. वै॒श्वा॒न॒रो विड् विड् वै᳚श्वान॒रो वै᳚श्वान॒रो विट् । \newline
22. विण् म॒रुतो॑ म॒रुतो॒ विड् विण् म॒रुतः॑ । \newline
23. म॒रुतो॑ वैश्वान॒रं ॅवै᳚श्वान॒रम् म॒रुतो॑ म॒रुतो॑ वैश्वान॒रम् । \newline
24. वै॒श्वा॒न॒रꣳ हु॒त्वा हु॒त्वा वै᳚श्वान॒रं ॅवै᳚श्वान॒रꣳ हु॒त्वा । \newline
25. हु॒त्वा मा॑रु॒तान् मा॑रु॒तान्. हु॒त्वा हु॒त्वा मा॑रु॒तान् । \newline
26. मा॒रु॒तान् जु॑होति जुहोति मारु॒तान् मा॑रु॒तान् जु॑होति । \newline
27. जु॒हो॒ति॒ रा॒ष्ट्रे रा॒ष्ट्रे जु॑होति जुहोति रा॒ष्ट्रे । \newline
28. रा॒ष्ट्र ए॒वैव रा॒ष्ट्रे रा॒ष्ट्र ए॒व । \newline
29. ए॒व विशं॒ ॅविश॑ मे॒वैव विश᳚म् । \newline
30. विश॒ मन्वनु॒ विशं॒ ॅविश॒ मनु॑ । \newline
31. अनु॑ बद्ध्नाति बद्ध्ना॒ त्यन्वनु॑ बद्ध्नाति । \newline
32. ब॒द्ध्ना॒ त्यु॒च्चै रु॒च्चैर् ब॑द्ध्नाति बद्ध्ना त्यु॒च्चैः । \newline
33. उ॒च्चैर् वै᳚श्वान॒रस्य॑ वैश्वान॒र स्यो॒च्चै रु॒च्चैर् वै᳚श्वान॒रस्य॑ । \newline
34. वै॒श्वा॒न॒रस्या वै᳚श्वान॒रस्य॑ वैश्वान॒रस्या । \newline
35. आ श्रा॑वयति श्रावय॒त्या श्रा॑वयति । \newline
36. श्रा॒व॒य॒ त्यु॒पाꣳ॒॒शू॑ पाꣳ॒॒शु श्रा॑वयति श्रावय त्युपाꣳ॒॒शु । \newline
37. उ॒पाꣳ॒॒शु मा॑रु॒तान् मा॑रु॒ता नु॑पाꣳ॒॒शू॑ पाꣳ॒॒शु मा॑रु॒तान् । \newline
38. उ॒पाꣳ॒॒श्वित्यु॑प - अꣳ॒॒शु । \newline
39. मा॒रु॒तान् जु॑होति जुहोति मारु॒तान् मा॑रु॒तान् जु॑होति । \newline
40. जु॒हो॒ति॒ तस्मा॒त् तस्मा᳚ज् जुहोति जुहोति॒ तस्मा᳚त् । \newline
41. तस्मा᳚द् रा॒ष्ट्रꣳ रा॒ष्ट्रम् तस्मा॒त् तस्मा᳚द् रा॒ष्ट्रम् । \newline
42. रा॒ष्ट्रं ॅविशं॒ ॅविशꣳ॑ रा॒ष्ट्रꣳ रा॒ष्ट्रं ॅविश᳚म् । \newline
43. विश॒ मत्यति॒ विशं॒ ॅविश॒ मति॑ । \newline
44. अति॑ वदति वद॒ त्य त्यति॑ वदति । \newline
45. व॒द॒ति॒ मा॒रु॒ता मा॑रु॒ता व॑दति वदति मारु॒ताः । \newline
46. मा॒रु॒ता भ॑वन्ति भवन्ति मारु॒ता मा॑रु॒ता भ॑वन्ति । \newline
47. भ॒व॒न्ति॒ म॒रुतो॑ म॒रुतो॑ भवन्ति भवन्ति म॒रुतः॑ । \newline
48. म॒रुतो॒ वै वै म॒रुतो॑ म॒रुतो॒ वै । \newline
49. वै दे॒वाना᳚म् दे॒वानां॒ ॅवै वै दे॒वाना᳚म् । \newline
50. दे॒वानां॒ ॅविशो॒ विशो॑ दे॒वाना᳚म् दे॒वानां॒ ॅविशः॑ । \newline
51. विशो॑ देववि॒शेन॑ देववि॒शेन॒ विशो॒ विशो॑ देववि॒शेन॑ । \newline
52. दे॒व॒वि॒शे नै॒वैव दे॑ववि॒शेन॑ देववि॒शेनै॒व । \newline
53. दे॒व॒वि॒शेनेति॑ देव - वि॒शेन॑ । \newline
54. ए॒वास्मा॑ अस्मा ए॒वै वास्मै᳚ । \newline
55. अ॒स्मै॒ म॒नु॒ष्य॒वि॒शम् म॑नुष्यवि॒श म॑स्मा अस्मै मनुष्यवि॒शम् । \newline
56. म॒नु॒ष्य॒वि॒श मवाव॑ मनुष्यवि॒शम् म॑नुष्यवि॒श मव॑ । \newline
57. म॒नु॒ष्य॒वि॒शमिति॑ मनुष्य - वि॒शम् । \newline
58. अव॑ रुन्धे रु॒न्धे ऽवाव॑ रुन्धे । \newline
59. रु॒न्धे॒ स॒प्त स॒प्त रु॑न्धे रुन्धे स॒प्त । \newline
60. स॒प्त भ॑वन्ति भवन्ति स॒प्त स॒प्त भ॑वन्ति । \newline
61. भ॒व॒न्ति॒ स॒प्तग॑णाः स॒प्तग॑णा भवन्ति भवन्ति स॒प्तग॑णाः । \newline
62. स॒प्तग॑णा॒ वै वै स॒प्तग॑णाः स॒प्तग॑णा॒ वै । \newline
63. स॒प्तग॑णा॒ इति॑ स॒प्त - ग॒णाः॒ । \newline
64. वै म॒रुतो॑ म॒रुतो॒ वै वै म॒रुतः॑ । \newline
65. म॒रुतो॑ गण॒शो ग॑ण॒शो म॒रुतो॑ म॒रुतो॑ गण॒शः । \newline
66. ग॒ण॒श ए॒वैव ग॑ण॒शो ग॑ण॒श ए॒व । \newline
67. ग॒ण॒श इति॑ गण - शः । \newline
68. ए॒व विशं॒ ॅविश॑ मे॒वैव विश᳚म् । \newline
69. विश॒ मवाव॒ विशं॒ ॅविश॒ मव॑ । \newline
70. अव॑ रुन्धे रु॒न्धे ऽवाव॑ रुन्धे । \newline
71. रु॒न्धे॒ ग॒णेन॑ ग॒णेन॑ रुन्धे रुन्धे ग॒णेन॑ । \newline
72. ग॒णेन॑ ग॒णम् ग॒णम् ग॒णेन॑ ग॒णेन॑ ग॒णम् । \newline
73. ग॒ण म॑नु॒द्रुत्या॑ नु॒द्रुत्य॑ ग॒णम् ग॒ण म॑नु॒द्रुत्य॑ । \newline
74. अ॒नु॒द्रुत्य॑ जुहोति जुहो त्यनु॒द्रुत्या॑ नु॒द्रुत्य॑ जुहोति । \newline
75. अ॒नु॒द्रुत्येत्य॑नु - द्रुत्य॑ । \newline
76. जु॒हो॒ति॒ विशं॒ ॅविश॑म् जुहोति जुहोति॒ विश᳚म् । \newline
77. विश॑ मे॒वैव विशं॒ ॅविश॑ मे॒व । \newline
78. ए॒वास्मा॑ अस्मा ए॒वै वास्मै᳚ । \newline
79. अ॒स्मा॒ अनु॑वर्त्मान॒ मनु॑वर्त्मान मस्मा अस्मा॒ अनु॑वर्त्मानम् । \newline
80. अनु॑वर्त्मानम् करोति करो॒ त्यनु॑वर्त्मान॒ मनु॑वर्त्मानम् करोति । \newline
81. अनु॑वर्त्मान॒मित्यनु॑ - व॒र्त्मा॒न॒म् । \newline
82. क॒रो॒तीति॑ करोति । \newline

\textbf{Ghana Paata } \newline

1. ए॒व वै᳚श्वान॒रं ॅवै᳚श्वान॒र मे॒वैव वै᳚श्वान॒र मवाव॑ वैश्वान॒र मे॒वैव वै᳚श्वान॒र मव॑ । \newline
2. वै॒श्वा॒न॒र मवाव॑ वैश्वान॒रं ॅवै᳚श्वान॒र मव॑ रुन्धे रु॒न्धे ऽव॑ वैश्वान॒रं ॅवै᳚श्वान॒र मव॑ रुन्धे । \newline
3. अव॑ रुन्धे रु॒न्धे ऽवाव॑ रुन्धे॒ यद् यद् रु॒न्धे ऽवाव॑ रुन्धे॒ यत् । \newline
4. रु॒न्धे॒ यद् यद् रु॑न्धे रुन्धे॒ यत् प्र॑याजानूया॒जान् प्र॑याजानूया॒जान्. यद् रु॑न्धे रुन्धे॒ यत् प्र॑याजानूया॒जान्न् । \newline
5. यत् प्र॑याजानूया॒जान् प्र॑याजानूया॒जान्. यद् यत् प्र॑याजानूया॒जान् कु॒र्यात् कु॒र्यात् प्र॑याजानूया॒जान्. यद् यत् प्र॑याजानूया॒जान् कु॒र्यात् । \newline
6. प्र॒या॒जा॒नू॒या॒जान् कु॒र्यात् कु॒र्यात् प्र॑याजानूया॒जान् प्र॑याजानूया॒जान् कु॒र्याद् विक॑स्ति॒र् विक॑स्तिः कु॒र्यात् प्र॑याजानूया॒जान् प्र॑याजानूया॒जान् कु॒र्याद् विक॑स्तिः । \newline
7. प्र॒या॒जा॒नू॒या॒जानिति॑ प्रयाज - अ॒नू॒या॒जान् । \newline
8. कु॒र्याद् विक॑स्ति॒र् विक॑स्तिः कु॒र्यात् कु॒र्याद् विक॑स्तिः॒ सा सा विक॑स्तिः कु॒र्यात् कु॒र्याद् विक॑स्तिः॒ सा । \newline
9. विक॑स्तिः॒ सा सा विक॑स्ति॒र् विक॑स्तिः॒ सा य॒ज्ञ्स्य॑ य॒ज्ञ्स्य॒ सा विक॑स्ति॒र् विक॑स्तिः॒ सा य॒ज्ञ्स्य॑ । \newline
10. विक॑स्ति॒रिति॒ वि - क॒स्तिः॒ । \newline
11. सा य॒ज्ञ्स्य॑ य॒ज्ञ्स्य॒ सा सा य॒ज्ञ्स्य॑ दर्विहो॒मम् द॑र्विहो॒मं ॅय॒ज्ञ्स्य॒ सा सा य॒ज्ञ्स्य॑ दर्विहो॒मम् । \newline
12. य॒ज्ञ्स्य॑ दर्विहो॒मम् द॑र्विहो॒मं ॅय॒ज्ञ्स्य॑ य॒ज्ञ्स्य॑ दर्विहो॒मम् क॑रोति करोति दर्विहो॒मं ॅय॒ज्ञ्स्य॑ य॒ज्ञ्स्य॑ दर्विहो॒मम् क॑रोति । \newline
13. द॒र्वि॒हो॒मम् क॑रोति करोति दर्विहो॒मम् द॑र्विहो॒मम् क॑रोति य॒ज्ञ्स्य॑ य॒ज्ञ्स्य॑ करोति दर्विहो॒मम् द॑र्विहो॒मम् क॑रोति य॒ज्ञ्स्य॑ । \newline
14. द॒र्वि॒हो॒ममिति॑ दर्वि - हो॒मम् । \newline
15. क॒रो॒ति॒ य॒ज्ञ्स्य॑ य॒ज्ञ्स्य॑ करोति करोति य॒ज्ञ्स्य॒ प्रति॑ष्ठित्यै॒ प्रति॑ष्ठित्यै य॒ज्ञ्स्य॑ करोति करोति य॒ज्ञ्स्य॒ प्रति॑ष्ठित्यै । \newline
16. य॒ज्ञ्स्य॒ प्रति॑ष्ठित्यै॒ प्रति॑ष्ठित्यै य॒ज्ञ्स्य॑ य॒ज्ञ्स्य॒ प्रति॑ष्ठित्यै रा॒ष्ट्रꣳ रा॒ष्ट्रम् प्रति॑ष्ठित्यै य॒ज्ञ्स्य॑ य॒ज्ञ्स्य॒ प्रति॑ष्ठित्यै रा॒ष्ट्रम् । \newline
17. प्रति॑ष्ठित्यै रा॒ष्ट्रꣳ रा॒ष्ट्रम् प्रति॑ष्ठित्यै॒ प्रति॑ष्ठित्यै रा॒ष्ट्रं ॅवै वै रा॒ष्ट्रम् प्रति॑ष्ठित्यै॒ प्रति॑ष्ठित्यै रा॒ष्ट्रं ॅवै । \newline
18. प्रति॑ष्ठित्या॒ इति॒ प्रति॑ - स्थि॒त्यै॒ । \newline
19. रा॒ष्ट्रं ॅवै वै रा॒ष्ट्रꣳ रा॒ष्ट्रं ॅवै वै᳚श्वान॒रो वै᳚श्वान॒रो वै रा॒ष्ट्रꣳ रा॒ष्ट्रं ॅवै वै᳚श्वान॒रः । \newline
20. वै वै᳚श्वान॒रो वै᳚श्वान॒रो वै वै वै᳚श्वान॒रो विड् विड् वै᳚श्वान॒रो वै वै वै᳚श्वान॒रो विट् । \newline
21. वै॒श्वा॒न॒रो विड् विड् वै᳚श्वान॒रो वै᳚श्वान॒रो विण् म॒रुतो॑ म॒रुतो॒ विड् वै᳚श्वान॒रो वै᳚श्वान॒रो विण् म॒रुतः॑ । \newline
22. विण् म॒रुतो॑ म॒रुतो॒ विड् विण् म॒रुतो॑ वैश्वान॒रं ॅवै᳚श्वान॒रम् म॒रुतो॒ विड् विण् म॒रुतो॑ वैश्वान॒रम् । \newline
23. म॒रुतो॑ वैश्वान॒रं ॅवै᳚श्वान॒रम् म॒रुतो॑ म॒रुतो॑ वैश्वान॒रꣳ हु॒त्वा हु॒त्वा वै᳚श्वान॒रम् म॒रुतो॑ म॒रुतो॑ वैश्वान॒रꣳ हु॒त्वा । \newline
24. वै॒श्वा॒न॒रꣳ हु॒त्वा हु॒त्वा वै᳚श्वान॒रं ॅवै᳚श्वान॒रꣳ हु॒त्वा मा॑रु॒तान् मा॑रु॒तान्. हु॒त्वा वै᳚श्वान॒रं ॅवै᳚श्वान॒रꣳ हु॒त्वा मा॑रु॒तान् । \newline
25. हु॒त्वा मा॑रु॒तान् मा॑रु॒तान्. हु॒त्वा हु॒त्वा मा॑रु॒तान् जु॑होति जुहोति मारु॒तान्. हु॒त्वा हु॒त्वा मा॑रु॒तान् जु॑होति । \newline
26. मा॒रु॒तान् जु॑होति जुहोति मारु॒तान् मा॑रु॒तान् जु॑होति रा॒ष्ट्रे रा॒ष्ट्रे जु॑होति मारु॒तान् मा॑रु॒तान् जु॑होति रा॒ष्ट्रे । \newline
27. जु॒हो॒ति॒ रा॒ष्ट्रे रा॒ष्ट्रे जु॑होति जुहोति रा॒ष्ट्र ए॒वैव रा॒ष्ट्रे जु॑होति जुहोति रा॒ष्ट्र ए॒व । \newline
28. रा॒ष्ट्र ए॒वैव रा॒ष्ट्रे रा॒ष्ट्र ए॒व विशं॒ ॅविश॑ मे॒व रा॒ष्ट्रे रा॒ष्ट्र ए॒व विश᳚म् । \newline
29. ए॒व विशं॒ ॅविश॑ मे॒वैव विश॒ मन्वनु॒ विश॑ मे॒वैव विश॒ मनु॑ । \newline
30. विश॒ मन्वनु॒ विशं॒ ॅविश॒ मनु॑ बद्ध्नाति बद्ध्ना॒ त्यनु॒ विशं॒ ॅविश॒ मनु॑ बद्ध्नाति । \newline
31. अनु॑ बद्ध्नाति बद्ध्ना॒ त्यन्वनु॑ बद्ध्ना त्यु॒च्चै रु॒च्चैर् ब॑द्ध्ना॒ त्यन्वनु॑ बद्ध्ना त्यु॒च्चैः । \newline
32. ब॒द्ध्ना॒ त्यु॒च्चै रु॒च्चैर् ब॑द्ध्नाति बद्ध्ना त्यु॒च्चैर् वै᳚श्वान॒रस्य॑ वैश्वान॒र स्यो॒च्चैर् ब॑द्ध्नाति बद्ध्ना त्यु॒च्चैर् वै᳚श्वान॒रस्य॑ । \newline
33. उ॒च्चैर् वै᳚श्वान॒रस्य॑ वैश्वान॒र स्यो॒च्चै रु॒च्चैर् वै᳚श्वान॒रस्या वै᳚श्वान॒र स्यो॒च्चै रु॒च्चैर् वै᳚श्वान॒रस्या । \newline
34. वै॒श्वा॒न॒रस्या वै᳚श्वान॒रस्य॑ वैश्वान॒रस्या श्रा॑वयति श्रावय॒त्या वै᳚श्वान॒रस्य॑ वैश्वान॒रस्या श्रा॑वयति । \newline
35. आ श्रा॑वयति श्रावय॒त्या श्रा॑वय त्युपाꣳ॒॒ शू॑पाꣳ॒॒शु श्रा॑वय॒त्या श्रा॑वय त्युपाꣳ॒॒शु । \newline
36. श्रा॒व॒य॒त्यु॒ पाꣳ॒॒शू॑ पाꣳ॒॒शु श्रा॑वयति श्रावयत्यु पाꣳ॒॒शु मा॑रु॒तान् मा॑रु॒ता नु॑पाꣳ॒॒शु श्रा॑वयति श्रावयत्यु पाꣳ॒॒शु मा॑रु॒तान् । \newline
37. उ॒पाꣳ॒॒शु मा॑रु॒तान् मा॑रु॒ता नु॑पाꣳ॒॒शू॑ पाꣳ॒॒शु मा॑रु॒तान् जु॑होति जुहोति मारु॒ता नु॑पाꣳ॒॒शू॑ पाꣳ॒॒शु मा॑रु॒तान् जु॑होति । \newline
38. उ॒पाꣳ॒॒श्वित्यु॑प - अꣳ॒॒शु । \newline
39. मा॒रु॒तान् जु॑होति जुहोति मारु॒तान् मा॑रु॒तान् जु॑होति॒ तस्मा॒त् तस्मा᳚ज् जुहोति मारु॒तान् मा॑रु॒तान् जु॑होति॒ तस्मा᳚त् । \newline
40. जु॒हो॒ति॒ तस्मा॒त् तस्मा᳚ज् जुहोति जुहोति॒ तस्मा᳚द् रा॒ष्ट्रꣳ रा॒ष्ट्रम् तस्मा᳚ज् जुहोति जुहोति॒ तस्मा᳚द् रा॒ष्ट्रम् । \newline
41. तस्मा᳚द् रा॒ष्ट्रꣳ रा॒ष्ट्रम् तस्मा॒त् तस्मा᳚द् रा॒ष्ट्रं ॅविशं॒ ॅविशꣳ॑ रा॒ष्ट्रम् तस्मा॒त् तस्मा᳚द् रा॒ष्ट्रं ॅविश᳚म् । \newline
42. रा॒ष्ट्रं ॅविशं॒ ॅविशꣳ॑ रा॒ष्ट्रꣳ रा॒ष्ट्रं ॅविश॒ मत्यति॒ विशꣳ॑ रा॒ष्ट्रꣳ रा॒ष्ट्रं ॅविश॒ मति॑ । \newline
43. विश॒ मत्यति॒ विशं॒ ॅविश॒ मति॑ वदति वद॒ त्यति॒ विशं॒ ॅविश॒ मति॑ वदति । \newline
44. अति॑ वदति वद॒ त्य त्यति॑ वदति मारु॒ता मा॑रु॒ता व॑द॒ त्य त्यति॑ वदति मारु॒ताः । \newline
45. व॒द॒ति॒ मा॒रु॒ता मा॑रु॒ता व॑दति वदति मारु॒ता भ॑वन्ति भवन्ति मारु॒ता व॑दति वदति मारु॒ता भ॑वन्ति । \newline
46. मा॒रु॒ता भ॑वन्ति भवन्ति मारु॒ता मा॑रु॒ता भ॑वन्ति म॒रुतो॑ म॒रुतो॑ भवन्ति मारु॒ता मा॑रु॒ता भ॑वन्ति म॒रुतः॑ । \newline
47. भ॒व॒न्ति॒ म॒रुतो॑ म॒रुतो॑ भवन्ति भवन्ति म॒रुतो॒ वै वै म॒रुतो॑ भवन्ति भवन्ति म॒रुतो॒ वै । \newline
48. म॒रुतो॒ वै वै म॒रुतो॑ म॒रुतो॒ वै दे॒वाना᳚म् दे॒वानां॒ ॅवै म॒रुतो॑ म॒रुतो॒ वै दे॒वाना᳚म् । \newline
49. वै दे॒वाना᳚म् दे॒वानां॒ ॅवै वै दे॒वानां॒ ॅविशो॒ विशो॑ दे॒वानां॒ ॅवै वै दे॒वानां॒ ॅविशः॑ । \newline
50. दे॒वानां॒ ॅविशो॒ विशो॑ दे॒वाना᳚म् दे॒वानां॒ ॅविशो॑ देववि॒शेन॑ देववि॒शेन॒ विशो॑ दे॒वाना᳚म् दे॒वानां॒ ॅविशो॑ देववि॒शेन॑ । \newline
51. विशो॑ देववि॒शेन॑ देववि॒शेन॒ विशो॒ विशो॑ देववि॒शे नै॒वैव दे॑ववि॒शेन॒ विशो॒ विशो॑ देववि॒शेनै॒व । \newline
52. दे॒व॒वि॒शे नै॒वैव दे॑ववि॒शेन॑ देववि॒शे नै॒वास्मा॑ अस्मा ए॒व दे॑ववि॒शेन॑ देववि॒शे नै॒वास्मै᳚ । \newline
53. दे॒व॒वि॒शेनेति॑ देव - वि॒शेन॑ । \newline
54. ए॒वास्मा॑ अस्मा ए॒वै वास्मै॑ मनुष्यवि॒शम् म॑नुष्यवि॒श म॑स्मा ए॒वै वास्मै॑ मनुष्यवि॒शम् । \newline
55. अ॒स्मै॒ म॒नु॒ष्य॒वि॒शम् म॑नुष्यवि॒श म॑स्मा अस्मै मनुष्यवि॒श मवाव॑ मनुष्यवि॒श म॑स्मा अस्मै मनुष्यवि॒श मव॑ । \newline
56. म॒नु॒ष्य॒वि॒श मवाव॑ मनुष्यवि॒शम् म॑नुष्यवि॒श मव॑ रुन्धे रु॒न्धे ऽव॑ मनुष्यवि॒शम् म॑नुष्यवि॒श मव॑ रुन्धे । \newline
57. म॒नु॒ष्य॒वि॒शमिति॑ मनुष्य - वि॒शम् । \newline
58. अव॑ रुन्धे रु॒न्धे ऽवाव॑ रुन्धे स॒प्त स॒प्त रु॒न्धे ऽवाव॑ रुन्धे स॒प्त । \newline
59. रु॒न्धे॒ स॒प्त स॒प्त रु॑न्धे रुन्धे स॒प्त भ॑वन्ति भवन्ति स॒प्त रु॑न्धे रुन्धे स॒प्त भ॑वन्ति । \newline
60. स॒प्त भ॑वन्ति भवन्ति स॒प्त स॒प्त भ॑वन्ति स॒प्तग॑णाः स॒प्तग॑णा भवन्ति स॒प्त स॒प्त भ॑वन्ति स॒प्तग॑णाः । \newline
61. भ॒व॒न्ति॒ स॒प्तग॑णाः स॒प्तग॑णा भवन्ति भवन्ति स॒प्तग॑णा॒ वै वै स॒प्तग॑णा भवन्ति भवन्ति स॒प्तग॑णा॒ वै । \newline
62. स॒प्तग॑णा॒ वै वै स॒प्तग॑णाः स॒प्तग॑णा॒ वै म॒रुतो॑ म॒रुतो॒ वै स॒प्तग॑णाः स॒प्तग॑णा॒ वै म॒रुतः॑ । \newline
63. स॒प्तग॑णा॒ इति॑ स॒प्त - ग॒णाः॒ । \newline
64. वै म॒रुतो॑ म॒रुतो॒ वै वै म॒रुतो॑ गण॒शो ग॑ण॒शो म॒रुतो॒ वै वै म॒रुतो॑ गण॒शः । \newline
65. म॒रुतो॑ गण॒शो ग॑ण॒शो म॒रुतो॑ म॒रुतो॑ गण॒श ए॒वैव ग॑ण॒शो म॒रुतो॑ म॒रुतो॑ गण॒श ए॒व । \newline
66. ग॒ण॒श ए॒वैव ग॑ण॒शो ग॑ण॒श ए॒व विशं॒ ॅविश॑ मे॒व ग॑ण॒शो ग॑ण॒श ए॒व विश᳚म् । \newline
67. ग॒ण॒श इति॑ गण - शः । \newline
68. ए॒व विशं॒ ॅविश॑ मे॒वैव विश॒ मवाव॒ विश॑ मे॒वैव विश॒ मव॑ । \newline
69. विश॒ मवाव॒ विशं॒ ॅविश॒ मव॑ रुन्धे रु॒न्धे ऽव॒ विशं॒ ॅविश॒ मव॑ रुन्धे । \newline
70. अव॑ रुन्धे रु॒न्धे ऽवाव॑ रुन्धे ग॒णेन॑ ग॒णेन॑ रु॒न्धे ऽवाव॑ रुन्धे ग॒णेन॑ । \newline
71. रु॒न्धे॒ ग॒णेन॑ ग॒णेन॑ रुन्धे रुन्धे ग॒णेन॑ ग॒णम् ग॒णम् ग॒णेन॑ रुन्धे रुन्धे ग॒णेन॑ ग॒णम् । \newline
72. ग॒णेन॑ ग॒णम् ग॒णम् ग॒णेन॑ ग॒णेन॑ ग॒ण म॑नु॒द्रुत्या॑ नु॒द्रुत्य॑ ग॒णम् ग॒णेन॑ ग॒णेन॑ ग॒ण म॑नु॒द्रुत्य॑ । \newline
73. ग॒ण म॑नु॒द्रुत्या॑ नु॒द्रुत्य॑ ग॒णम् ग॒ण म॑नु॒द्रुत्य॑ जुहोति जुहो त्यनु॒द्रुत्य॑ ग॒णम् ग॒ण म॑नु॒द्रुत्य॑ जुहोति । \newline
74. अ॒नु॒द्रुत्य॑ जुहोति जुहो त्यनु॒द्रुत्या॑ नु॒द्रुत्य॑ जुहोति॒ विशं॒ ॅविश॑म् जुहो त्यनु॒द्रुत्या॑ नु॒द्रुत्य॑ जुहोति॒ विश᳚म् । \newline
75. अ॒नु॒द्रुत्येत्य॑नु - द्रुत्य॑ । \newline
76. जु॒हो॒ति॒ विशं॒ ॅविश॑म् जुहोति जुहोति॒ विश॑ मे॒वैव विश॑म् जुहोति जुहोति॒ विश॑ मे॒व । \newline
77. विश॑ मे॒वैव विशं॒ ॅविश॑ मे॒वास्मा॑ अस्मा ए॒व विशं॒ ॅविश॑ मे॒वास्मै᳚ । \newline
78. ए॒वास्मा॑ अस्मा ए॒वै वास्मा॒ अनु॑वर्त्मान॒ मनु॑वर्त्मान मस्मा ए॒वै वास्मा॒ अनु॑वर्त्मानम् । \newline
79. अ॒स्मा॒ अनु॑वर्त्मान॒ मनु॑वर्त्मान मस्मा अस्मा॒ अनु॑वर्त्मानम् करोति करो॒ त्यनु॑वर्त्मान मस्मा अस्मा॒ अनु॑वर्त्मानम् करोति । \newline
80. अनु॑वर्त्मानम् करोति करो॒ त्यनु॑वर्त्मान॒ मनु॑वर्त्मानम् करोति । \newline
81. अनु॑वर्त्मान॒मित्यनु॑ - व॒र्त्मा॒न॒म् । \newline
82. क॒रो॒तीति॑ करोति । \newline
\pagebreak
\markright{ TS 5.4.8.1  \hfill https://www.vedavms.in \hfill}

\section{ TS 5.4.8.1 }

\textbf{TS 5.4.8.1 } \newline
\textbf{Samhita Paata} \newline

वसो॒र्द्धारां᳚ जुहोति॒ वसो᳚र्मे॒ धारा॑ऽस॒दिति॒ वा ए॒षा हू॑यते घृ॒तस्य॒ वा ए॑नमे॒षा धारा॒ऽमुष्मि॑न् ॅलो॒के पिन्व॑मा॒नोप॑ तिष्ठत॒ आज्ये॑न जुहोति॒ तेजो॒ वा आज्यं॒ तेजो॒ वसो॒र्द्धारा॒ तेज॑सै॒वास्मै॒ तेजोऽव॑ रु॒न्धेऽथो॒ कामा॒ वै वसो॒र्द्धारा॒ कामा॑ने॒वाव॑ रुन्धे॒ यं का॒मये॑त प्रा॒णान॑स्या॒न्नाद्यं॒ ॅवि - [  ] \newline

\textbf{Pada Paata} \newline

वसोः᳚ । धारा᳚म् । जु॒हो॒ति॒ । वसोः᳚ । मे॒ । धारा᳚ । अ॒स॒त् । इति॑ । वै । ए॒षा । हू॒य॒ते॒ । घृ॒तस्य॑ । वै । ए॒न॒म् । ए॒षा । धारा᳚ । अ॒मुष्मिन्न्॑ । लो॒के । पिन्व॑माना । उपेति॑ । ति॒ष्ठ॒ते॒ । आज्ये॑न । जु॒हो॒ति॒ । तेजः॑ । वै । आज्य᳚म् । तेजः॑ । वसोः᳚ । धारा᳚ । तेज॑सा । ए॒व । अ॒स्मै॒ । तेजः॑ । अवेति॑ । रु॒न्धे॒ । अथो॒ इति॑ । कामाः᳚ । वै । वसोः᳚ । धारा᳚ । कामान्॑ । ए॒व । अवेति॑ । रु॒न्धे॒ । यम् । का॒मये॑त । प्रा॒णानिति॑ प्र - अ॒नान् । अ॒स्य॒ । अ॒न्नाद्य॒मित्य॑न्न - अद्य᳚म् । वीति॑ ।  \newline


\textbf{Krama Paata} \newline

वसो॒र् धारा᳚म् । धारा᳚म् जुहोति । जु॒हो॒ति॒ वसोः᳚ । वसो᳚र् मे । मे॒ धारा᳚ । धारा॑ऽसत् । अ॒स॒दिति॑ । इति॒ वै । वा ए॒षा । ए॒षा हू॑यते । हू॒य॒ते॒ घृ॒तस्य॑ । घृ॒तस्य॒ वै । वा ए॑नम् । ए॒न॒मे॒षा । ए॒षा धारा᳚ । धारा॒ऽमुष्मिन्न्॑ । अ॒मुष्मि॑न् ॅलो॒के । लो॒के पिन्व॑माना । पिन्व॑मा॒नोप॑ । उप॑ तिष्ठते । ति॒ष्ठ॒त॒ आज्ये॑न । आज्ये॑न जुहोति । जु॒हो॒ति॒ तेजः॑ । तेजो॒ वै । वा आज्य᳚म् । आज्य॒म् तेजः॑ । तेजो॒ वसोः᳚ । वसो॒र् धारा᳚ । धारा॒ तेज॑सा । तेज॑सै॒व । ए॒वास्मै᳚ । अ॒स्मै॒ तेजः॑ । तेजोऽव॑ । अव॑ रुन्धे । रु॒न्धेऽथो᳚ । अथो॒ कामाः᳚ । अथो॒ इत्यथो᳚ । कामा॒ वै । वै वसोः᳚ । वसो॒र् धारा᳚ । धारा॒ कामान्॑ । कामा॑ने॒व । ए॒वाव॑ । अव॑ रुन्धे । रु॒न्धे॒ यम् । यम् का॒मये॑त । का॒मये॑त प्रा॒णान् । प्रा॒णान॑स्य । प्रा॒णानिति॑ प्र - अ॒नान् । अ॒स्या॒न्नाद्य᳚म् । अ॒न्नाद्य॒म् ॅवि । अ॒न्नाद्य॒मित्य॑न्न - अद्य᳚म् । विच्छि॑न्द्याम् । \newline

\textbf{Jatai Paata} \newline

1. वसो॒र् धारा॒म् धारां॒ ॅवसो॒र् वसो॒र् धारा᳚म् । \newline
2. धारा᳚म् जुहोति जुहोति॒ धारा॒म् धारा᳚म् जुहोति । \newline
3. जु॒हो॒ति॒ वसो॒र् वसो᳚र् जुहोति जुहोति॒ वसोः᳚ । \newline
4. वसो᳚र् मे मे॒ वसो॒र् वसो᳚र् मे । \newline
5. मे॒ धारा॒ धारा॑ मे मे॒ धारा᳚ । \newline
6. धारा॑ ऽस दस॒द् धारा॒ धारा॑ ऽसत् । \newline
7. अ॒स॒ दिती त्य॑स दस॒ दिति॑ । \newline
8. इति॒ वै वा इतीति॒ वै । \newline
9. वा ए॒षैषा वै वा ए॒षा । \newline
10. ए॒षा हू॑यते हूयत ए॒षैषा हू॑यते । \newline
11. हू॒य॒ते॒ घृ॒तस्य॑ घृ॒तस्य॑ हूयते हूयते घृ॒तस्य॑ । \newline
12. घृ॒तस्य॒ वै वै घृ॒तस्य॑ घृ॒तस्य॒ वै । \newline
13. वा ए॑न मेनं॒ ॅवै वा ए॑नम् । \newline
14. ए॒न॒ मे॒षै षैन॑ मेन मे॒षा । \newline
15. ए॒षा धारा॒ धारै॒ षैषा धारा᳚ । \newline
16. धारा॒ ऽमुष्मि॑न् न॒मुष्मि॒न् धारा॒ धारा॒ ऽमुष्मिन्न्॑ । \newline
17. अ॒मुष्मि॑न् ॅलो॒के लो॒के॑ ऽमुष्मि॑न् न॒मुष्मि॑न् ॅलो॒के । \newline
18. लो॒के पिन्व॑माना॒ पिन्व॑माना लो॒के लो॒के पिन्व॑माना । \newline
19. पिन्व॑मा॒नो पोप॒ पिन्व॑माना॒ पिन्व॑मा॒नोप॑ । \newline
20. उप॑ तिष्ठते तिष्ठत॒ उपोप॑ तिष्ठते । \newline
21. ति॒ष्ठ॒त॒ आज्ये॒ना ज्ये॑न तिष्ठते तिष्ठत॒ आज्ये॑न । \newline
22. आज्ये॑न जुहोति जुहो॒ त्याज्ये॒ना ज्ये॑न जुहोति । \newline
23. जु॒हो॒ति॒ तेज॒ स्तेजो॑ जुहोति जुहोति॒ तेजः॑ । \newline
24. तेजो॒ वै वै तेज॒ स्तेजो॒ वै । \newline
25. वा आज्य॒ माज्यं॒ ॅवै वा आज्य᳚म् । \newline
26. आज्य॒म् तेज॒ स्तेज॒ आज्य॒ माज्य॒म् तेजः॑ । \newline
27. तेजो॒ वसो॒र् वसो॒ स्तेज॒ स्तेजो॒ वसोः᳚ । \newline
28. वसो॒र् धारा॒ धारा॒ वसो॒र् वसो॒र् धारा᳚ । \newline
29. धारा॒ तेज॑सा॒ तेज॑सा॒ धारा॒ धारा॒ तेज॑सा । \newline
30. तेज॑सै॒वैव तेज॑सा॒ तेज॑सै॒व । \newline
31. ए॒वास्मा॑ अस्मा ए॒वै वास्मै᳚ । \newline
32. अ॒स्मै॒ तेज॒ स्तेजो᳚ ऽस्मा अस्मै॒ तेजः॑ । \newline
33. तेजो ऽवाव॒ तेज॒ स्तेजो ऽव॑ । \newline
34. अव॑ रुन्धे रु॒न्धे ऽवाव॑ रुन्धे । \newline
35. रु॒न्धे ऽथो॒ अथो॑ रुन्धे रु॒न्धे ऽथो᳚ । \newline
36. अथो॒ कामाः॒ कामा॒ अथो॒ अथो॒ कामाः᳚ । \newline
37. अथो॒ इत्यथो᳚ । \newline
38. कामा॒ वै वै कामाः॒ कामा॒ वै । \newline
39. वै वसो॒र् वसो॒र् वै वै वसोः᳚ । \newline
40. वसो॒र् धारा॒ धारा॒ वसो॒र् वसो॒र् धारा᳚ । \newline
41. धारा॒ कामा॒न् कामा॒न् धारा॒ धारा॒ कामान्॑ । \newline
42. कामा॑ ने॒वैव कामा॒न् कामा॑ ने॒व । \newline
43. ए॒वावा वै॒वै वाव॑ । \newline
44. अव॑ रुन्धे रु॒न्धे ऽवाव॑ रुन्धे । \newline
45. रु॒न्धे॒ यं ॅयꣳ रु॑न्धे रुन्धे॒ यम् । \newline
46. यम् का॒मये॑त का॒मये॑त॒ यं ॅयम् का॒मये॑त । \newline
47. का॒मये॑त प्रा॒णान् प्रा॒णान् का॒मये॑त का॒मये॑त प्रा॒णान् । \newline
48. प्रा॒णा न॑स्यास्य प्रा॒णान् प्रा॒णा न॑स्य । \newline
49. प्रा॒णानिति॑ प्र - अ॒नान् । \newline
50. अ॒स्या॒न्नाद्य॑ म॒न्नाद्य॑ मस्या स्या॒न्नाद्य᳚म् । \newline
51. अ॒न्नाद्यं॒ ॅवि व्य॑न्नाद्य॑ म॒न्नाद्यं॒ ॅवि । \newline
52. अ॒न्नाद्य॒मित्य॑न्न - अद्य᳚म् । \newline
53. वि च्छि॑न्द्याम् छिन्द्यां॒ ॅवि वि च्छि॑न्द्याम् । \newline

\textbf{Ghana Paata } \newline

1. वसो॒र् धारा॒म् धारां॒ ॅवसो॒र् वसो॒र् धारा᳚म् जुहोति जुहोति॒ धारां॒ ॅवसो॒र् वसो॒र् धारा᳚म् जुहोति । \newline
2. धारा᳚म् जुहोति जुहोति॒ धारा॒म् धारा᳚म् जुहोति॒ वसो॒र् वसो᳚र् जुहोति॒ धारा॒म् धारा᳚म् जुहोति॒ वसोः᳚ । \newline
3. जु॒हो॒ति॒ वसो॒र् वसो᳚र् जुहोति जुहोति॒ वसो᳚र् मे मे॒ वसो᳚र् जुहोति जुहोति॒ वसो᳚र् मे । \newline
4. वसो᳚र् मे मे॒ वसो॒र् वसो᳚र् मे॒ धारा॒ धारा॑ मे॒ वसो॒र् वसो᳚र् मे॒ धारा᳚ । \newline
5. मे॒ धारा॒ धारा॑ मे मे॒ धारा॑ ऽस दस॒द् धारा॑ मे मे॒ धारा॑ ऽसत् । \newline
6. धारा॑ ऽस दस॒द् धारा॒ धारा॑ ऽस॒दि तीत्य॑स॒द् धारा॒ धारा॑ ऽस॒दिति॑ । \newline
7. अ॒स॒ दितीत्य॑ सदस॒ दिति॒ वै वा इत्य॑स दस॒ दिति॒ वै । \newline
8. इति॒ वै वा इतीति॒ वा ए॒षैषा वा इतीति॒ वा ए॒षा । \newline
9. वा ए॒षैषा वै वा ए॒षा हू॑यते हूयत ए॒षा वै वा ए॒षा हू॑यते । \newline
10. ए॒षा हू॑यते हूयत ए॒षैषा हू॑यते घृ॒तस्य॑ घृ॒तस्य॑ हूयत ए॒षैषा हू॑यते घृ॒तस्य॑ । \newline
11. हू॒य॒ते॒ घृ॒तस्य॑ घृ॒तस्य॑ हूयते हूयते घृ॒तस्य॒ वै वै घृ॒तस्य॑ हूयते हूयते घृ॒तस्य॒ वै । \newline
12. घृ॒तस्य॒ वै वै घृ॒तस्य॑ घृ॒तस्य॒ वा ए॑न मेनं॒ ॅवै घृ॒तस्य॑ घृ॒तस्य॒ वा ए॑नम् । \newline
13. वा ए॑न मेनं॒ ॅवै वा ए॑न मे॒षैषैनं॒ ॅवै वा ए॑न मे॒षा । \newline
14. ए॒न॒ मे॒षै षैन॑ मेन मे॒षा धारा॒ धारै॒षैन॑ मेन मे॒षा धारा᳚ । \newline
15. ए॒षा धारा॒ धारै॒षैषा धारा॒ ऽमुष्मि॑न् न॒मुष्मि॒न् धारै॒षैषा धारा॒ ऽमुष्मिन्न्॑ । \newline
16. धारा॒ ऽमुष्मि॑न् न॒मुष्मि॒न् धारा॒ धारा॒ ऽमुष्मि॑न् ॅलो॒के लो॒के॑ ऽमुष्मि॒न् धारा॒ धारा॒ ऽमुष्मि॑न् ॅलो॒के । \newline
17. अ॒मुष्मि॑न् ॅलो॒के लो॒के॑ ऽमुष्मि॑न् न॒मुष्मि॑न् ॅलो॒के पिन्व॑माना॒ पिन्व॑माना लो॒के॑ ऽमुष्मि॑न् न॒मुष्मि॑न् ॅलो॒के पिन्व॑माना । \newline
18. लो॒के पिन्व॑माना॒ पिन्व॑माना लो॒के लो॒के पिन्व॑मा॒नो पोप॒ पिन्व॑माना लो॒के लो॒के पिन्व॑मा॒नोप॑ । \newline
19. पिन्व॑मा॒नो पोप॒ पिन्व॑माना॒ पिन्व॑मा॒नोप॑ तिष्ठते तिष्ठत॒ उप॒ पिन्व॑माना॒ पिन्व॑मा॒नोप॑ तिष्ठते । \newline
20. उप॑ तिष्ठते तिष्ठत॒ उपोप॑ तिष्ठत॒ आज्ये॒ना ज्ये॑न तिष्ठत॒ उपोप॑ तिष्ठत॒ आज्ये॑न । \newline
21. ति॒ष्ठ॒त॒ आज्ये॒ना ज्ये॑न तिष्ठते तिष्ठत॒ आज्ये॑न जुहोति जुहो॒ त्याज्ये॑न तिष्ठते तिष्ठत॒ आज्ये॑न जुहोति । \newline
22. आज्ये॑न जुहोति जुहो॒ त्याज्ये॒ना ज्ये॑न जुहोति॒ तेज॒ स्तेजो॑ जुहो॒ त्याज्ये॒ना ज्ये॑न जुहोति॒ तेजः॑ । \newline
23. जु॒हो॒ति॒ तेज॒ स्तेजो॑ जुहोति जुहोति॒ तेजो॒ वै वै तेजो॑ जुहोति जुहोति॒ तेजो॒ वै । \newline
24. तेजो॒ वै वै तेज॒ स्तेजो॒ वा आज्य॒ माज्यं॒ ॅवै तेज॒ स्तेजो॒ वा आज्य᳚म् । \newline
25. वा आज्य॒ माज्यं॒ ॅवै वा आज्य॒म् तेज॒ स्तेज॒ आज्यं॒ ॅवै वा आज्य॒म् तेजः॑ । \newline
26. आज्य॒म् तेज॒ स्तेज॒ आज्य॒ माज्य॒म् तेजो॒ वसो॒र् वसो॒ स्तेज॒ आज्य॒ माज्य॒म् तेजो॒ वसोः᳚ । \newline
27. तेजो॒ वसो॒र् वसो॒ स्तेज॒ स्तेजो॒ वसो॒र् धारा॒ धारा॒ वसो॒ स्तेज॒ स्तेजो॒ वसो॒र् धारा᳚ । \newline
28. वसो॒र् धारा॒ धारा॒ वसो॒र् वसो॒र् धारा॒ तेज॑सा॒ तेज॑सा॒ धारा॒ वसो॒र् वसो॒र् धारा॒ तेज॑सा । \newline
29. धारा॒ तेज॑सा॒ तेज॑सा॒ धारा॒ धारा॒ तेज॑ सै॒वैव तेज॑सा॒ धारा॒ धारा॒ तेज॑सै॒व । \newline
30. तेज॑ सै॒वैव तेज॑सा॒ तेज॑ सै॒वास्मा॑ अस्मा ए॒व तेज॑सा॒ तेज॑ सै॒वास्मै᳚ । \newline
31. ए॒वास्मा॑ अस्मा ए॒वै वास्मै॒ तेज॒ स्तेजो᳚ ऽस्मा ए॒वै वास्मै॒ तेजः॑ । \newline
32. अ॒स्मै॒ तेज॒ स्तेजो᳚ ऽस्मा अस्मै॒ तेजो ऽवाव॒ तेजो᳚ ऽस्मा अस्मै॒ तेजो ऽव॑ । \newline
33. तेजो ऽवाव॒ तेज॒ स्तेजो ऽव॑ रुन्धे रु॒न्धे ऽव॒ तेज॒ स्तेजो ऽव॑ रुन्धे । \newline
34. अव॑ रुन्धे रु॒न्धे ऽवाव॑ रु॒न्धे ऽथो॒ अथो॑ रु॒न्धे ऽवाव॑ रु॒न्धे ऽथो᳚ । \newline
35. रु॒न्धे ऽथो॒ अथो॑ रुन्धे रु॒न्धे ऽथो॒ कामाः॒ कामा॒ अथो॑ रुन्धे रु॒न्धे ऽथो॒ कामाः᳚ । \newline
36. अथो॒ कामाः॒ कामा॒ अथो॒ अथो॒ कामा॒ वै वै कामा॒ अथो॒ अथो॒ कामा॒ वै । \newline
37. अथो॒ इत्यथो᳚ । \newline
38. कामा॒ वै वै कामाः॒ कामा॒ वै वसो॒र् वसो॒र् वै कामाः॒ कामा॒ वै वसोः᳚ । \newline
39. वै वसो॒र् वसो॒र् वै वै वसो॒र् धारा॒ धारा॒ वसो॒र् वै वै वसो॒र् धारा᳚ । \newline
40. वसो॒र् धारा॒ धारा॒ वसो॒र् वसो॒र् धारा॒ कामा॒न् कामा॒न् धारा॒ वसो॒र् वसो॒र् धारा॒ कामान्॑ । \newline
41. धारा॒ कामा॒न् कामा॒न् धारा॒ धारा॒ कामा॑ ने॒वैव कामा॒न् धारा॒ धारा॒ कामा॑ ने॒व । \newline
42. कामा॑ ने॒वैव कामा॒न् कामा॑ ने॒वावा वै॒व कामा॒न् कामा॑ ने॒वाव॑ । \newline
43. ए॒वावा वै॒वै वाव॑ रुन्धे रु॒न्धे ऽवै॒वै वाव॑ रुन्धे । \newline
44. अव॑ रुन्धे रु॒न्धे ऽवाव॑ रुन्धे॒ यं ॅयꣳ रु॒न्धे ऽवाव॑ रुन्धे॒ यम् । \newline
45. रु॒न्धे॒ यं ॅयꣳ रु॑न्धे रुन्धे॒ यम् का॒मये॑त का॒मये॑त॒ यꣳ रु॑न्धे रुन्धे॒ यम् का॒मये॑त । \newline
46. यम् का॒मये॑त का॒मये॑त॒ यं ॅयम् का॒मये॑त प्रा॒णान् प्रा॒णान् का॒मये॑त॒ यं ॅयम् का॒मये॑त प्रा॒णान् । \newline
47. का॒मये॑त प्रा॒णान् प्रा॒णान् का॒मये॑त का॒मये॑त प्रा॒णा न॑स्यास्य प्रा॒णान् का॒मये॑त का॒मये॑त प्रा॒णा न॑स्य । \newline
48. प्रा॒णा न॑स्यास्य प्रा॒णान् प्रा॒णा न॑स्या॒ न्नाद्य॑ म॒न्नाद्य॑ मस्य प्रा॒णान् प्रा॒णा न॑स्या॒ न्नाद्य᳚म् । \newline
49. प्रा॒णानिति॑ प्र - अ॒नान् । \newline
50. अ॒स्या॒ न्नाद्य॑ म॒न्नाद्य॑ मस्यास्या॒ न्नाद्यं॒ ॅवि व्य॑न्नाद्य॑ मस्यास्या॒ न्नाद्यं॒ ॅवि । \newline
51. अ॒न्नाद्यं॒ ॅवि व्य॑न्नाद्य॑ म॒न्नाद्यं॒ ॅवि च्छि॑न्द्याम् छिन्द्या॒म् ॅव्य॑न्नाद्य॑ म॒न्नाद्यं॒ ॅवि च्छि॑न्द्याम् । \newline
52. अ॒न्नाद्य॒मित्य॑न्न - अद्य᳚म् । \newline
53. वि च्छि॑न्द्याम् छिन्द्या॒म् ॅवि वि च्छि॑न्द्या॒ मितीति॑ छिन्द्यां॒ ॅवि वि च्छि॑न्द्या॒ मिति॑ । \newline
\pagebreak
\markright{ TS 5.4.8.2  \hfill https://www.vedavms.in \hfill}

\section{ TS 5.4.8.2 }

\textbf{TS 5.4.8.2 } \newline
\textbf{Samhita Paata} \newline

च्छि॑न्द्या॒मिति॑ वि॒ग्राहं॒ तस्य॑ जुहुयात् प्रा॒णाने॒वास्या॒न्नाद्यं॒ ॅविच्छि॑नत्ति॒ यं का॒मये॑त प्रा॒णान॑स्या॒न्नाद्यꣳ॒॒ सं त॑नुया॒मिति॒ सं त॑तां॒ तस्य॑ जुहुयात् प्रा॒णाने॒वास्या॒न्नाद्यꣳ॒॒ सं त॑नोति॒ द्वाद॑श द्वाद॒शानि॑ जुहोति॒ द्वाद॑श॒ मासाः᳚ संॅवथ्स॒रः सं॑ॅवथ्स॒रेणै॒ वास्मा॒ अन्न॒मव॑ रु॒न्धे ऽन्नं॑ च॒ मेऽक्षु॑च्च म॒ इत्या॑है॒ तद् वा - [  ] \newline

\textbf{Pada Paata} \newline

छि॒न्द्या॒म् । इति॑ । वि॒ग्राह॒मिति॑ वि - ग्राह᳚म् । तस्य॑ । जु॒हु॒या॒त् । प्रा॒णानिति॑ प्र - अ॒नान् । ए॒व । अ॒स्य॒ । अ॒न्नाद्य॒मित्य॑न्न - अद्य᳚म् । वीति॑ । छि॒न॒त्ति॒ । यम् । का॒मये॑त । प्रा॒णानिति॑ प्र-अ॒नान् । अ॒स्य॒ । अ॒न्नाद्य॒मित्य॑न्न - अद्य᳚म् । समिति॑ । त॒नु॒या॒म् । इति॑ । संत॑ता॒मिति॒ सं - त॒ता॒म् । तस्य॑ । जु॒हु॒या॒त् । प्रा॒णानिति॑ प्र - अ॒नान् । ए॒व । अ॒स्य॒ । अ॒न्नाद्य॒मित्य॑न्न - अद्य᳚म् । समिति॑ । त॒नो॒ति॒ । द्वाद॑श । द्वा॒द॒शानि॑ । जु॒हो॒ति॒ । द्वाद॑श । मासाः᳚ । सं॒ॅव॒थ्स॒र इति॑ सं-व॒थ्स॒रः । सं॒ॅव॒थ्स॒रेणेति॑ सं - व॒थ्स॒रेण॑ । ए॒व । अ॒स्मै॒ । अन्न᳚म् । अवेति॑ । रु॒न्धे॒ । अन्न᳚म् । च॒ । मे॒ । अक्षु॑त् । च॒ । मे॒ । इति॑ । आ॒ह॒ । ए॒तत् । वै ।  \newline


\textbf{Krama Paata} \newline

छि॒न्द्या॒मिति॑ । इति॑ वि॒ग्राह᳚म् । वि॒ग्राह॒म् तस्य॑ । वि॒ग्राह॒मिति॑ वि - ग्राह᳚म् । तस्य॑ जुहुयात् । जु॒हु॒या॒त् प्रा॒णान् । प्रा॒णाने॒व । प्रा॒णानिति॑ प्र - अ॒नान् । ए॒वास्य॑ । अ॒स्या॒न्नाद्य᳚म् । अ॒न्नाद्य॒म् ॅवि । अ॒न्नाद्य॒मित्य॑न्न - अद्य᳚म् । विच्छि॑नत्ति । छि॒न॒त्ति॒ यम् । यम् का॒मये॑त । का॒मये॑त प्रा॒णान् । प्रा॒णान॑स्य । प्रा॒णानिति॑ प्र - अ॒नान् । अ॒स्या॒न्नाद्य᳚म् । अ॒न्नाद्यꣳ॒॒ सम् । अ॒न्नाद्य॒मित्य॑न्न - अद्य᳚म् । सम् त॑नुयाम् । त॒नु॒या॒मिति॑ । इति॒ सन्त॑ताम् । सन्त॑ता॒म् तस्य॑ । सन्त॑ता॒मिति॒ सम् - त॒ता॒म् । तस्य॑ जुहुयात् । जु॒हु॒या॒त् प्रा॒णान् । प्रा॒णाने॒व । प्रा॒णानिति॑ प्र - अ॒नान् । ए॒वास्य॑ । अ॒स्या॒न्नाद्य᳚म् । अ॒न्नाद्यꣳ॒॒ सम् । अ॒न्नाद्य॒मित्य॑न्न - अद्य᳚म् । सम् त॑नोति । त॒नो॒ति॒ द्वाद॑श । द्वाद॑श द्वाद॒शानि॑ । द्वा॒द॒शानि॑ जुहोति । जु॒हो॒ति॒ द्वाद॑श । द्वाद॑श॒ मासाः᳚ । मासाः᳚ सम्ॅवथ्स॒रः । स॒म्ॅव॒थ्स॒रः स॑म्ॅवथ्स॒रेण॑ । स॒म्ॅव॒थ्स॒र इति॑ सम् - व॒थ्स॒रः । स॒म्ॅव॒थ्स॒रेणै॒व । स॒म्ॅव॒थ्स॒रेणेति॑ सम् - व॒थ्स॒रेण॑ । ए॒वास्मै᳚ । अ॒स्मा॒ अन्न᳚म् । अन्न॒मव॑ । अव॑ रुन्धे । रु॒न्धेऽन्न᳚म् । अन्न॑म् च । च॒ मे॒ । मेऽक्षु॑त् । अक्षु॑च्च । च॒ मे॒ । म॒ इति॑ । इत्या॑ह । आ॒है॒तत् । ए॒तद् वै । वा अन्न॑स्य \newline

\textbf{Jatai Paata} \newline

1. छि॒न्द्या॒ मितीति॑ छिन्द्याम् छिन्द्या॒ मिति॑ । \newline
2. इति॑ वि॒ग्राहं॑ ॅवि॒ग्राह॒ मितीति॑ वि॒ग्राह᳚म् । \newline
3. वि॒ग्राह॒म् तस्य॒ तस्य॑ वि॒ग्राहं॑ ॅवि॒ग्राह॒म् तस्य॑ । \newline
4. वि॒ग्राह॒मिति॑ वि - ग्राह᳚म् । \newline
5. तस्य॑ जुहुयाज् जुहुया॒त् तस्य॒ तस्य॑ जुहुयात् । \newline
6. जु॒हु॒या॒त् प्रा॒णान् प्रा॒णान् जु॑हुयाज् जुहुयात् प्रा॒णान् । \newline
7. प्रा॒णा ने॒वैव प्रा॒णान् प्रा॒णा ने॒व । \newline
8. प्रा॒णानिति॑ प्र - अ॒नान् । \newline
9. ए॒वास्या᳚ स्यै॒वै वास्य॑ । \newline
10. अ॒स्या॒ न्नाद्य॑ म॒न्नाद्य॑ मस्यास्या॒ न्नाद्य᳚म् । \newline
11. अ॒न्नाद्यं॒ ॅवि व्य॑न्नाद्य॑ म॒न्नाद्यं॒ ॅवि । \newline
12. अ॒न्नाद्य॒मित्य॑न्न - अद्य᳚म् । \newline
13. वि च्छि॑नत्ति छिनत्ति॒द् वि वि च्छि॑नत्ति । \newline
14. छि॒न॒त्ति॒ यं ॅयम् छि॑नत्ति छिनत्ति॒ यम् । \newline
15. यम् का॒मये॑त का॒मये॑त॒ यं ॅयम् का॒मये॑त । \newline
16. का॒मये॑त प्रा॒णान् प्रा॒णान् का॒मये॑त का॒मये॑त प्रा॒णान् । \newline
17. प्रा॒णा न॑स्यास्य प्रा॒णान् प्रा॒णा न॑स्य । \newline
18. प्रा॒णानिति॑ प्र - अ॒नान् । \newline
19. अ॒स्या॒ न्नाद्य॑ म॒न्नाद्य॑ मस्यास्या॒ न्नाद्य᳚म् । \newline
20. अ॒न्नाद्यꣳ॒॒ सꣳ स म॒न्नाद्य॑ म॒न्नाद्यꣳ॒॒ सम् । \newline
21. अ॒न्नाद्य॒मित्य॑न्न - अद्य᳚म् । \newline
22. सम् त॑नुयाम् तनुयाꣳ॒॒ सꣳ सम् त॑नुयाम् । \newline
23. त॒नु॒या॒ मितीति॑ तनुयाम् तनुया॒ मिति॑ । \newline
24. इति॒ सन्त॑ताꣳ॒॒ सन्त॑ता॒ मितीति॒ सन्त॑ताम् । \newline
25. सन्त॑ता॒म् तस्य॒ तस्य॒ सन्त॑ताꣳ॒॒ सन्त॑ता॒म् तस्य॑ । \newline
26. सन्त॑ता॒मिति॒ सं - त॒ता॒म् । \newline
27. तस्य॑ जुहुयाज् जुहुया॒त् तस्य॒ तस्य॑ जुहुयात् । \newline
28. जु॒हु॒या॒त् प्रा॒णान् प्रा॒णान् जु॑हुयाज् जुहुयात् प्रा॒णान् । \newline
29. प्रा॒णा ने॒वैव प्रा॒णान् प्रा॒णा ने॒व । \newline
30. प्रा॒णानिति॑ प्र - अ॒नान् । \newline
31. ए॒वास्या᳚ स्यै॒वै वास्य॑ । \newline
32. अ॒स्या॒ न्नाद्य॑ म॒न्नाद्य॑ मस्यास्या॒ न्नाद्य᳚म् । \newline
33. अ॒न्नाद्यꣳ॒॒ सꣳ स म॒न्नाद्य॑ म॒न्नाद्यꣳ॒॒ सम् । \newline
34. अ॒न्नाद्य॒मित्य॑न्न - अद्य᳚म् । \newline
35. सम् त॑नोति तनोति॒ सꣳ सम् त॑नोति । \newline
36. त॒नो॒ति॒ द्वाद॑श॒ द्वाद॑श तनोति तनोति॒ द्वाद॑श । \newline
37. द्वाद॑श द्वाद॒शानि॑ द्वाद॒शानि॒ द्वाद॑श॒ द्वाद॑श द्वाद॒शानि॑ । \newline
38. द्वा॒द॒शानि॑ जुहोति जुहोति द्वाद॒शानि॑ द्वाद॒शानि॑ जुहोति । \newline
39. जु॒हो॒ति॒ द्वाद॑श॒ द्वाद॑श जुहोति जुहोति॒ द्वाद॑श । \newline
40. द्वाद॑श॒ मासा॒ मासा॒ द्वाद॑श॒ द्वाद॑श॒ मासाः᳚ । \newline
41. मासाः᳚ संॅवथ्स॒रः सं॑ॅवथ्स॒रो मासा॒ मासाः᳚ संॅवथ्स॒रः । \newline
42. सं॒ॅव॒थ्स॒रः सं॑ॅवथ्स॒रेण॑ संॅवथ्स॒रेण॑ संॅवथ्स॒रः सं॑ॅवथ्स॒रः सं॑ॅवथ्स॒रेण॑ । \newline
43. सं॒ॅव॒थ्स॒र इति॑ सं - व॒थ्स॒रः । \newline
44. सं॒ॅव॒थ्स॒रे णै॒वैव सं॑ॅवथ्स॒रेण॑ संॅवथ्स॒रेणै॒व । \newline
45. सं॒ॅव॒थ्स॒रेणेति॑ सं - व॒थ्स॒रेण॑ । \newline
46. ए॒वास्मा॑ अस्मा ए॒वै वास्मै᳚ । \newline
47. अ॒स्मा॒ अन्न॒ मन्न॑ मस्मा अस्मा॒ अन्न᳚म् । \newline
48. अन्न॒ मवा वान्न॒ मन्न॒ मव॑ । \newline
49. अव॑ रुन्धे रु॒न्धे ऽवाव॑ रुन्धे । \newline
50. रु॒न्धे ऽन्न॒ मन्नꣳ॑ रुन्धे रु॒न्धे ऽन्न᳚म् । \newline
51. अन्न॑म् च॒ चान्न॒ मन्न॑म् च । \newline
52. च॒ मे॒ मे॒ च॒ च॒ मे॒ । \newline
53. मे ऽक्षु॒ दक्षु॑न् मे॒ मे ऽक्षु॑त् । \newline
54. अक्षु॑च् च॒ चाक्षु॒ दक्षु॑च् च । \newline
55. च॒ मे॒ मे॒ च॒ च॒ मे॒ । \newline
56. म॒ इतीति॑ मे म॒ इति॑ । \newline
57. इत्या॑हा॒हे तीत्या॑ह । \newline
58. आ॒है॒त दे॒त दा॑हा है॒तत् । \newline
59. ए॒तद् वै वा ए॒त दे॒तद् वै । \newline
60. वा अन्न॒स्या न्न॑स्य॒ वै वा अन्न॑स्य । \newline

\textbf{Ghana Paata } \newline

1. छि॒न्द्या॒ मितीति॑ छिन्द्याम् छिन्द्या॒ मिति॑ वि॒ग्राहं॑ ॅवि॒ग्राह॒ मिति॑ छिन्द्याम् छिन्द्या॒ मिति॑ वि॒ग्राह᳚म् । \newline
2. इति॑ वि॒ग्राहं॑ ॅवि॒ग्राह॒ मितीति॑ वि॒ग्राह॒म् तस्य॒ तस्य॑ वि॒ग्राह॒ मितीति॑ वि॒ग्राह॒म् तस्य॑ । \newline
3. वि॒ग्राह॒म् तस्य॒ तस्य॑ वि॒ग्राहं॑ ॅवि॒ग्राह॒म् तस्य॑ जुहुयाज् जुहुया॒त् तस्य॑ वि॒ग्राहं॑ ॅवि॒ग्राह॒म् तस्य॑ जुहुयात् । \newline
4. वि॒ग्राह॒मिति॑ वि - ग्राह᳚म् । \newline
5. तस्य॑ जुहुयाज् जुहुया॒त् तस्य॒ तस्य॑ जुहुयात् प्रा॒णान् प्रा॒णान् जु॑हुया॒त् तस्य॒ तस्य॑ जुहुयात् प्रा॒णान् । \newline
6. जु॒हु॒या॒त् प्रा॒णान् प्रा॒णान् जु॑हुयाज् जुहुयात् प्रा॒णा ने॒वैव प्रा॒णान् जु॑हुयाज् जुहुयात् प्रा॒णा ने॒व । \newline
7. प्रा॒णा ने॒वैव प्रा॒णान् प्रा॒णा ने॒वास्या᳚ स्यै॒व प्रा॒णान् प्रा॒णा ने॒वास्य॑ । \newline
8. प्रा॒णानिति॑ प्र - अ॒नान् । \newline
9. ए॒वास्या᳚ स्यै॒वै वास्या॒ न्नाद्य॑ म॒न्नाद्य॑ मस्यै॒वै वास्या॒ न्नाद्य᳚म् । \newline
10. अ॒स्या॒ न्नाद्य॑ म॒न्नाद्य॑ मस्यास्या॒ न्नाद्यं॒ ॅवि व्य॑न्नाद्य॑ मस्यास्या॒ न्नाद्यं॒ ॅवि । \newline
11. अ॒न्नाद्यं॒ ॅवि व्य॑न्नाद्य॑ म॒न्नाद्यं॒ ॅवि च्छि॑नत्ति छिनत्ति॒ व्य॑न्नाद्य॑ म॒न्नाद्यं॒ ॅवि च्छि॑नत्ति । \newline
12. अ॒न्नाद्य॒मित्य॑न्न - अद्य᳚म् । \newline
13. वि च्छि॑नत्ति छिनत्ति॒ वि वि च्छि॑नत्ति॒ यं ॅयम् छि॑नत्ति॒ वि वि च्छि॑नत्ति॒ यम् । \newline
14. छि॒न॒त्ति॒ यं ॅयम् छि॑नत्ति छिनत्ति॒ यम् का॒मये॑त का॒मये॑त॒ यम् छि॑नत्ति छिनत्ति॒ यम् का॒मये॑त । \newline
15. यम् का॒मये॑त का॒मये॑त॒ यं ॅयम् का॒मये॑त प्रा॒णान् प्रा॒णान् का॒मये॑त॒ यं ॅयम् का॒मये॑त प्रा॒णान् । \newline
16. का॒मये॑त प्रा॒णान् प्रा॒णान् का॒मये॑त का॒मये॑त प्रा॒णा न॑स्यास्य प्रा॒णान् का॒मये॑त का॒मये॑त प्रा॒णा न॑स्य । \newline
17. प्रा॒णा न॑स्यास्य प्रा॒णान् प्रा॒णा न॑स्या॒ न्नाद्य॑ म॒न्नाद्य॑ मस्य प्रा॒णान् प्रा॒णा न॑स्या॒ न्नाद्य᳚म् । \newline
18. प्रा॒णानिति॑ प्र - अ॒नान् । \newline
19. अ॒स्या॒न्नाद्य॑ म॒न्नाद्य॑ मस्यास्या॒ न्नाद्यꣳ॒॒ सꣳ स म॒न्नाद्य॑ मस्यास्या॒ न्नाद्यꣳ॒॒ सम् । \newline
20. अ॒न्नाद्यꣳ॒॒ सꣳ स म॒न्नाद्य॑ म॒न्नाद्यꣳ॒॒ सम् त॑नुयाम् तनुयाꣳ॒॒ स म॒न्नाद्य॑ म॒न्नाद्यꣳ॒॒ सम् त॑नुयाम् । \newline
21. अ॒न्नाद्य॒मित्य॑न्न - अद्य᳚म् । \newline
22. सम् त॑नुयाम् तनुयाꣳ॒॒ सꣳ सम् त॑नुया॒ मितीति॑ तनुयाꣳ॒॒ सꣳ सम् त॑नुया॒ मिति॑ । \newline
23. त॒नु॒या॒ मितीति॑ तनुयाम् तनुया॒ मिति॒ सन्त॑ताꣳ॒॒ सन्त॑ता॒ मिति॑ तनुयाम् तनुया॒ मिति॒ सन्त॑ताम् । \newline
24. इति॒ सन्त॑ताꣳ॒॒ सन्तता॒ मितीति॒ सन्त॑ता॒म् तस्य॒ तस्य॒ सन्त॑ता॒ मितीति॒ सन्त॑ता॒म् तस्य॑ । \newline
25. सन्त॑ता॒म् तस्य॒ तस्य॒ सन्त॑ताꣳ॒॒ सन्त॑ता॒म् तस्य॑ जुहुयाज् जुहुया॒त् तस्य॒ सन्त॑ताꣳ॒॒ सन्त॑ता॒म् तस्य॑ जुहुयात् । \newline
26. सन्त॑ता॒मिति॒ सं - त॒ता॒म् । \newline
27. तस्य॑ जुहुयाज् जुहुया॒त् तस्य॒ तस्य॑ जुहुयात् प्रा॒णान् प्रा॒णान् जु॑हुया॒त् तस्य॒ तस्य॑ जुहुयात् प्रा॒णान् । \newline
28. जु॒हु॒या॒त् प्रा॒णान् प्रा॒णान् जु॑हुयाज् जुहुयात् प्रा॒णा ने॒वैव प्रा॒णान् जु॑हुयाज् जुहुयात् प्रा॒णा ने॒व । \newline
29. प्रा॒णा ने॒वैव प्रा॒णान् प्रा॒णा ने॒वास्या᳚ स्यै॒व प्रा॒णान् प्रा॒णा ने॒वास्य॑ । \newline
30. प्रा॒णानिति॑ प्र - अ॒नान् । \newline
31. ए॒वास्या᳚ स्यै॒वै वास्या॒ न्नाद्य॑ म॒न्नाद्य॑ मस्यै॒वै वास्या॒ न्नाद्य᳚म् । \newline
32. अ॒स्या॒ न्नाद्य॑ म॒न्नाद्य॑ मस्यास्या॒ न्नाद्यꣳ॒॒ सꣳ स म॒न्नाद्य॑ मस्यास्या ॒न्नाद्यꣳ॒॒ सम् । \newline
33. अ॒न्नाद्यꣳ॒॒ सꣳ स म॒न्नाद्य॑ म॒न्नाद्यꣳ॒॒ सम् त॑नोति तनोति॒ स म॒न्नाद्य॑ म॒न्नाद्यꣳ॒॒ सम् त॑नोति । \newline
34. अ॒न्नाद्य॒मित्य॑न्न - अद्य᳚म् । \newline
35. सम् त॑नोति तनोति॒ सꣳ सम् त॑नोति॒ द्वाद॑श॒ द्वाद॑श तनोति॒ सꣳ सम् त॑नोति॒ द्वाद॑श । \newline
36. त॒नो॒ति॒ द्वाद॑श॒ द्वाद॑श तनोति तनोति॒ द्वाद॑श द्वाद॒शानि॑ द्वाद॒शानि॒ द्वाद॑श तनोति तनोति॒ द्वाद॑श द्वाद॒शानि॑ । \newline
37. द्वाद॑श द्वाद॒शानि॑ द्वाद॒शानि॒ द्वाद॑श॒ द्वाद॑श द्वाद॒शानि॑ जुहोति जुहोति द्वाद॒शानि॒ द्वाद॑श॒ द्वाद॑श द्वाद॒शानि॑ जुहोति । \newline
38. द्वा॒द॒शानि॑ जुहोति जुहोति द्वाद॒शानि॑ द्वाद॒शानि॑ जुहोति॒ द्वाद॑श॒ द्वाद॑श जुहोति द्वाद॒शानि॑ द्वाद॒शानि॑ जुहोति॒ द्वाद॑श । \newline
39. जु॒हो॒ति॒ द्वाद॑श॒ द्वाद॑श जुहोति जुहोति॒ द्वाद॑श॒ मासा॒ मासा॒ द्वाद॑श जुहोति जुहोति॒ द्वाद॑श॒ मासाः᳚ । \newline
40. द्वाद॑श॒ मासा॒ मासा॒ द्वाद॑श॒ द्वाद॑श॒ मासाः᳚ संॅवथ्स॒रः सं॑ॅवथ्स॒रो मासा॒ द्वाद॑श॒ द्वाद॑श॒ मासाः᳚ संॅवथ्स॒रः । \newline
41. मासाः᳚ संॅवथ्स॒रः सं॑ॅवथ्स॒रो मासा॒ मासाः᳚ संॅवथ्स॒रः सं॑ॅवथ्स॒रेण॑ संॅवथ्स॒रेण॑ संॅवथ्स॒रो मासा॒ मासाः᳚ संॅवथ्स॒रः सं॑ॅवथ्स॒रेण॑ । \newline
42. सं॒ॅव॒थ्स॒रः सं॑ॅवथ्स॒रेण॑ संॅवथ्स॒रेण॑ संॅवथ्स॒रः सं॑ॅवथ्स॒रः सं॑ॅवथ्स॒रे णै॒वैव सं॑ॅवथ्स॒रेण॑ संॅवथ्स॒रः सं॑ॅवथ्स॒रः सं॑ॅवथ्स॒रेणै॒व । \newline
43. सं॒ॅव॒थ्स॒र इति॑ सं - व॒थ्स॒रः । \newline
44. सं॒ॅव॒थ्स॒रे णै॒वैव सं॑ॅवथ्स॒रेण॑ संॅवथ्स॒रे णै॒वास्मा॑ अस्मा ए॒व सं॑ॅवथ्स॒रेण॑ संॅवथ्स॒रे णै॒वास्मै᳚ । \newline
45. सं॒ॅव॒थ्स॒रेणेति॑ सं - व॒थ्स॒रेण॑ । \newline
46. ए॒वास्मा॑ अस्मा ए॒वै वास्मा॒ अन्न॒ मन्न॑ मस्मा ए॒वै वास्मा॒ अन्न᳚म् । \newline
47. अ॒स्मा॒ अन्न॒ मन्न॑ मस्मा अस्मा॒ अन्न॒ मवा वान्न॑ मस्मा अस्मा॒ अन्न॒ मव॑ । \newline
48. अन्न॒ मवा वान्न॒ मन्न॒ मव॑ रुन्धे रु॒न्धे ऽवान्न॒ मन्न॒ मव॑ रुन्धे । \newline
49. अव॑ रुन्धे रु॒न्धे ऽवाव॑ रु॒न्धे ऽन्न॒ मन्नꣳ॑ रु॒न्धे ऽवाव॑ रु॒न्धे ऽन्न᳚म् । \newline
50. रु॒न्धे ऽन्न॒ मन्नꣳ॑ रुन्धे रु॒न्धे ऽन्न॑म् च॒ चान्नꣳ॑ रुन्धे रु॒न्धे ऽन्न॑म् च । \newline
51. अन्न॑म् च॒ चान्न॒ मन्न॑म् च मे मे॒ चान्न॒ मन्न॑म् च मे । \newline
52. च॒ मे॒ मे॒ च॒ च॒ मे ऽक्षु॒ दक्षु॑न् मे च च॒ मे ऽक्षु॑त् । \newline
53. मे ऽक्षु॒ दक्षु॑न् मे॒ मे ऽक्षु॑च् च॒ चाक्षु॑न् मे॒ मे ऽक्षु॑च् च । \newline
54. अक्षु॑च् च॒ चाक्षु॒ दक्षु॑च् च मे मे॒ चाक्षु॒ दक्षु॑च् च मे । \newline
55. च॒ मे॒ मे॒ च॒ च॒ म॒ इतीति॑ मे च च म॒ इति॑ । \newline
56. म॒ इतीति॑ मे म॒ इत्या॑हा॒हेति॑ मे म॒ इत्या॑ह । \newline
57. इत्या॑हा॒हे तीत्या॑है॒ तदे॒ तदा॒हे तीत्या॑ है॒तत् । \newline
58. आ॒है॒ तदे॒ तदा॑हा है॒तद् वै वा ए॒त दा॑हा है॒तद् वै । \newline
59. ए॒तद् वै वा ए॒त दे॒तद् वा अन्न॒स्या न्न॑स्य॒ वा ए॒त दे॒तद् वा अन्न॑स्य । \newline
60. वा अन्न॒स्या न्न॑स्य॒ वै वा अन्न॑स्य रू॒पꣳ रू॒प मन्न॑स्य॒ वै वा अन्न॑स्य रू॒पम् । \newline
\pagebreak
\markright{ TS 5.4.8.3  \hfill https://www.vedavms.in \hfill}

\section{ TS 5.4.8.3 }

\textbf{TS 5.4.8.3 } \newline
\textbf{Samhita Paata} \newline

अन्न॑स्य रू॒पꣳ रू॒पेणै॒वान्न॒मव॑ रुन्धे॒ ऽग्निश्च॑ म॒ आप॑श्च म॒ इत्या॑है॒षा वा अन्न॑स्य॒ योनिः॒ सयो᳚न्ये॒वान्न॒मव॑ रुन्धे-ऽर्द्धे॒न्द्राणि॑ जुहोति दे॒वता॑ ए॒वाव॑ रुन्धे॒ यथ् सर्वे॑षा-म॒र्द्धमिन्द्रः॒ प्रति॒ तस्मा॒दिन्द्रो॑ दे॒वता॑नां भूयिष्ठ॒भाक्त॑म॒ इन्द्र॒मुत्त॑रमाहे-न्द्रि॒यमे॒वास्मि॑-न्नु॒परि॑ष्टाद् दधाति यज्ञायु॒धानि॑ जुहोति य॒ज्ञो - [  ] \newline

\textbf{Pada Paata} \newline

अन्न॑स्य । रू॒पम् । रू॒पेण॑ । ए॒व । अन्न᳚म् । अवेति॑ । रु॒न्धे॒ । अ॒ग्निः । च॒ । मे॒ । आपः॑ । च॒ । मे॒ । इति॑ । आ॒ह॒ । ए॒षा । वै । अन्न॑स्य । योनिः॑ । सयो॒नीति॒ स - यो॒नि॒ । ए॒व । अन्न᳚म् । अवेति॑ । रु॒न्धे॒ । अ॒द्‌र्धे॒न्द्राणीत्य॑द्‌र्ध - इ॒न्द्राणि॑ । जु॒हो॒ति॒ । दे॒वताः᳚ । ए॒व । अवेति॑ । रु॒न्धे॒ । यत् । सर्वे॑षाम् । अ॒द्‌र्धम् । इन्द्रः॑ । प्रतीति॑ । तस्मा᳚त् । इन्द्रः॑ । दे॒वता॑नाम् । भू॒यि॒ष्ठ॒भाक्त॑म॒ इति॑ भूयिष्ठ॒भाक्-त॒मः॒ । इन्द्र᳚म् । उत्त॑र॒मित्युत् -त॒र॒म् । आ॒ह॒ । इ॒न्द्रि॒यम् । ए॒व । अ॒स्मि॒न्न् । उ॒परि॑ष्टात् । द॒धा॒ति॒ । य॒ज्ञा॒यु॒धानीति॑ यज्ञ्-आ॒यु॒धानि॑ । जु॒हो॒ति॒ । य॒ज्ञ्ः ।  \newline


\textbf{Krama Paata} \newline

अन्न॑स्य रू॒पम् । रू॒पꣳ रू॒पेण॑ । रू॒पेणै॒व । ए॒वान्न᳚म् । अन्न॒मव॑ । अव॑ रुन्धे । रु॒न्धे॒ऽग्निः । अ॒ग्निश्च॑ । च॒ मे॒ । म॒ आपः॑ । आप॑श्च । च॒ मे॒ । म॒ इति॑ । इत्या॑ह । आ॒है॒षा । ए॒षा वै । वा अन्न॑स्य । अन्न॑स्य॒ योनिः॑ । योनिः॒ सयो॑नि । सयो᳚न्ये॒व । सयो॒नीति॒ स - यो॒नि॒ । ए॒वान्न᳚म् । अन्न॒मव॑ । अव॑ रुन्धे । रु॒न्धे॒ऽर्द्धे॒न्द्राणि॑ । अ॒र्द्धे॒न्द्राणि॑ जुहोति । अ॒र्द्धे॒न्द्राणीत्य॑र्द्ध - इ॒न्द्राणि॑ । जु॒हो॒ति॒ दे॒वताः᳚ । दे॒वता॑ ए॒व । ए॒वाव॑ । अव॑ रुन्धे । रु॒न्धे॒ यत् । यथ् सर्वे॑षाम् । सर्वे॑षाम॒र्द्धम् । अ॒र्द्धमिन्द्रः॑ । इन्द्रः॒ प्रति॑ । प्रति॒ तस्मा᳚त् । तस्मा॒दिन्द्रः॑ । इन्द्रो॑ दे॒वता॑नाम् । दे॒वता॑नाम् भूयिष्ठ॒भाक्त॑मः । भू॒यि॒ष्ठ॒भाक्त॑म॒ इन्द्र᳚म् । भू॒यि॒ष्ठ॒भाक्त॑म॒ इति॑ भूयिष्ठ॒भाक् - त॒मः॒ । इन्द्र॒मुत्त॑रम् । उत्त॑रमाह । उत्त॑र॒मित्युत् - त॒र॒म् । आ॒हे॒न्द्रि॒यम् । इ॒न्द्रि॒यमे॒व । ए॒वास्मिन्न्॑ । अ॒स्मि॒न्नु॒परि॑ष्टात् । उ॒परि॑ष्टाद् दधाति । द॒धा॒ति॒ य॒ज्ञा॒यु॒धानि॑ । य॒ज्ञा॒यु॒धानि॑ जुहोति । य॒ज्ञा॒यु॒धानीति॑ यज्ञ् - आ॒यु॒धानि॑ । जु॒हो॒ति॒ य॒ज्ञ्ः । य॒ज्ञो वै \newline

\textbf{Jatai Paata} \newline

1. अन्न॑स्य रू॒पꣳ रू॒प मन्न॒स्या न्न॑स्य रू॒पम् । \newline
2. रू॒पꣳ रू॒पेण॑ रू॒पेण॑ रू॒पꣳ रू॒पꣳ रू॒पेण॑ । \newline
3. रू॒पे णै॒वैव रू॒पेण॑ रू॒पेणै॒व । \newline
4. ए॒वान्न॒ मन्न॑ मे॒वै वान्न᳚म् । \newline
5. अन्न॒ मवा वान्न॒ मन्न॒ मव॑ । \newline
6. अव॑ रुन्धे रु॒न्धे ऽवाव॑ रुन्धे । \newline
7. रु॒न्धे॒ ऽग्नि र॒ग्नी रु॑न्धे रुन्धे॒ ऽग्निः । \newline
8. अ॒ग्निश्च॑ चा॒ग्नि र॒ग्नि श्च॑ । \newline
9. च॒ मे॒ मे॒ च॒ च॒ मे॒ । \newline
10. म॒ आप॒ आपो॑ मे म॒ आपः॑ । \newline
11. आप॑श्च॒ चाप॒ आप॑श्च । \newline
12. च॒ मे॒ मे॒ च॒ च॒ मे॒ । \newline
13. म॒ इतीति॑ मे म॒ इति॑ । \newline
14. इत्या॑हा॒हे तीत्या॑ह । \newline
15. आ॒है॒ षैषा ऽऽहा॑ है॒षा । \newline
16. ए॒षा वै वा ए॒षैषा वै । \newline
17. वा अन्न॒स्या न्न॑स्य॒ वै वा अन्न॑स्य । \newline
18. अन्न॑स्य॒ योनि॒र् योनि॒ रन्न॒स्या न्न॑स्य॒ योनिः॑ । \newline
19. योनिः॒ सयो॑नि॒ सयो॑नि॒ योनि॒र् योनिः॒ सयो॑नि । \newline
20. सयो᳚ न्ये॒वैव सयो॑नि॒ सयो᳚न्ये॒व । \newline
21. सयो॒नीति॒ स - यो॒नि॒ । \newline
22. ए॒वान्न॒ मन्न॑ मे॒वै वान्न᳚म् । \newline
23. अन्न॒ मवा वान्न॒ मन्न॒ मव॑ । \newline
24. अव॑ रुन्धे रु॒न्धे ऽवाव॑ रुन्धे । \newline
25. रु॒न्धे॒ ऽर्द्धे॒न्द्राण्य॑ र्द्धे॒न्द्राणि॑ रुन्धे रुन्धे ऽर्द्धे॒न्द्राणि॑ । \newline
26. अ॒र्द्धे॒न्द्राणि॑ जुहोति जुहो त्यर्द्धे॒न्द्राण्य॑ र्द्धे॒न्द्राणि॑ जुहोति । \newline
27. अ॒र्द्धे॒न्द्राणीत्य॑र्द्ध - इ॒न्द्राणि॑ । \newline
28. जु॒हो॒ति॒ दे॒वता॑ दे॒वता॑ जुहोति जुहोति दे॒वताः᳚ । \newline
29. दे॒वता॑ ए॒वैव दे॒वता॑ दे॒वता॑ ए॒व । \newline
30. ए॒वावा वै॒वै वाव॑ । \newline
31. अव॑ रुन्धे रु॒न्धे ऽवाव॑ रुन्धे । \newline
32. रु॒न्धे॒ यद् यद् रु॑न्धे रुन्धे॒ यत् । \newline
33. यथ् सर्वे॑षाꣳ॒॒ सर्वे॑षां॒ ॅयद् यथ् सर्वे॑षाम् । \newline
34. सर्वे॑षा म॒र्द्ध म॒र्द्धꣳ सर्वे॑षाꣳ॒॒ सर्वे॑षा म॒र्द्धम् । \newline
35. अ॒र्द्ध मिन्द्र॒ इन्द्रो॒ ऽर्द्ध म॒र्द्ध मिन्द्रः॑ । \newline
36. इन्द्रः॒ प्रति॒ प्रतीन्द्र॒ इन्द्रः॒ प्रति॑ । \newline
37. प्रति॒ तस्मा॒त् तस्मा॒त् प्रति॒ प्रति॒ तस्मा᳚त् । \newline
38. तस्मा॒ दिन्द्र॒ इन्द्र॒ स्तस्मा॒त् तस्मा॒ दिन्द्रः॑ । \newline
39. इन्द्रो॑ दे॒वता॑नाम् दे॒वता॑ना॒ मिन्द्र॒ इन्द्रो॑ दे॒वता॑नाम् । \newline
40. दे॒वता॑नाम् भूयिष्ठ॒भाक्त॑मो भूयिष्ठ॒भाक्त॑मो दे॒वता॑नाम् दे॒वता॑नाम् भूयिष्ठ॒भाक्त॑मः । \newline
41. भू॒यि॒ष्ठ॒भाक्त॑म॒ इन्द्र॒ मिन्द्र॑म् भूयिष्ठ॒भाक्त॑मो भूयिष्ठ॒भाक्त॑म॒ इन्द्र᳚म् । \newline
42. भू॒यि॒ष्ठ॒भाक्त॑म॒ इति॑ भूयिष्ठ॒भाक् - त॒मः॒ । \newline
43. इन्द्र॒ मुत्त॑र॒ मुत्त॑र॒ मिन्द्र॒ मिन्द्र॒ मुत्त॑रम् । \newline
44. उत्त॑र माहा॒हो त्त॑र॒ मुत्त॑र माह । \newline
45. उत्त॑र॒मित्युत् - त॒र॒म् । \newline
46. आ॒हे॒न्द्रि॒य मि॑न्द्रि॒य मा॑हाहेन्द्रि॒यम् । \newline
47. इ॒न्द्रि॒य मे॒वै वेन्द्रि॒य मि॑न्द्रि॒य मे॒व । \newline
48. ए॒वास्मि॑न् नस्मिन् ने॒वै वास्मिन्न्॑ । \newline
49. अ॒स्मि॒न् नु॒परि॑ष्टा दु॒परि॑ष्टा दस्मिन् नस्मिन् नु॒परि॑ष्टात् । \newline
50. उ॒परि॑ष्टाद् दधाति दधा त्यु॒परि॑ष्टा दु॒परि॑ष्टाद् दधाति । \newline
51. द॒धा॒ति॒ य॒ज्ञा॒यु॒धानि॑ यज्ञायु॒धानि॑ दधाति दधाति यज्ञायु॒धानि॑ । \newline
52. य॒ज्ञा॒यु॒धानि॑ जुहोति जुहोति यज्ञायु॒धानि॑ यज्ञायु॒धानि॑ जुहोति । \newline
53. य॒ज्ञा॒यु॒धानीति॑ यज्ञ् - आ॒यु॒धानि॑ । \newline
54. जु॒हो॒ति॒ य॒ज्ञो य॒ज्ञो जु॑होति जुहोति य॒ज्ञ्ः । \newline
55. य॒ज्ञो वै वै य॒ज्ञो य॒ज्ञो वै । \newline

\textbf{Ghana Paata } \newline

1. अन्न॑स्य रू॒पꣳ रू॒प मन्न॒स्या न्न॑स्य रू॒पꣳ रू॒पेण॑ रू॒पेण॑ रू॒प मन्न॒स्या न्न॑स्य रू॒पꣳ रू॒पेण॑ । \newline
2. रू॒पꣳ रू॒पेण॑ रू॒पेण॑ रू॒पꣳ रू॒पꣳ रू॒पे णै॒वैव रू॒पेण॑ रू॒पꣳ रू॒पꣳ रू॒पेणै॒व । \newline
3. रू॒पे णै॒वैव रू॒पेण॑ रू॒पे णै॒वान्न॒ मन्न॑ मे॒व रू॒पेण॑ रू॒पे णै॒वान्न᳚म् । \newline
4. ए॒वान्न॒ मन्न॑ मे॒वै वान्न॒ मवावान्न॑ मे॒वै वान्न॒ मव॑ । \newline
5. अन्न॒ मवावान्न॒ मन्न॒ मव॑ रुन्धे रु॒न्धे ऽवान्न॒ मन्न॒ मव॑ रुन्धे । \newline
6. अव॑ रुन्धे रु॒न्धे ऽवाव॑ रुन्धे॒ ऽग्नि र॒ग्नी रु॒न्धे ऽवाव॑ रुन्धे॒ ऽग्निः । \newline
7. रु॒न्धे॒ ऽग्नि र॒ग्नी रु॑न्धे रुन्धे॒ ऽग्निश्च॑ चा॒ग्नी रु॑न्धे रुन्धे॒ ऽग्निश्च॑ । \newline
8. अ॒ग्निश्च॑ चा॒ग्नि र॒ग्निश्च॑ मे मे चा॒ग्नि र॒ग्निश्च॑ मे । \newline
9. च॒ मे॒ मे॒ च॒ च॒ म॒ आप॒ आपो॑ मे च च म॒ आपः॑ । \newline
10. म॒ आप॒ आपो॑ मे म॒ आप॑श्च॒ चापो॑ मे म॒ आप॑श्च । \newline
11. आप॑श्च॒ चाप॒ आप॑श्च मे मे॒ चाप॒ आप॑श्च मे । \newline
12. च॒ मे॒ मे॒ च॒ च॒ म॒ इतीति॑ मे च च म॒ इति॑ । \newline
13. म॒ इतीति॑ मे म॒ इत्या॑हा॒हेति॑ मे म॒ इत्या॑ह । \newline
14. इत्या॑हा॒हे तीत्या॑ है॒षैषा ऽऽहेतीत्या॑ है॒षा । \newline
15. आ॒है॒ षैषा ऽऽहा॑है॒षा वै वा ए॒षा ऽऽहा॑ है॒षा वै । \newline
16. ए॒षा वै वा ए॒षैषा वा अन्न॒स्या न्न॑स्य॒ वा ए॒षैषा वा अन्न॑स्य । \newline
17. वा अन्न॒स्या न्न॑स्य॒ वै वा अन्न॑स्य॒ योनि॒र् योनि॒ रन्न॑स्य॒ वै वा अन्न॑स्य॒ योनिः॑ । \newline
18. अन्न॑स्य॒ योनि॒र् योनि॒ रन्न॒स्या न्न॑स्य॒ योनिः॒ सयो॑नि॒ सयो॑नि॒ योनि॒ रन्न॒स्या न्न॑स्य॒ योनिः॒ सयो॑नि । \newline
19. योनिः॒ सयो॑नि॒ सयो॑नि॒ योनि॒र् योनिः॒ सयो᳚न्ये॒वैव सयो॑नि॒ योनि॒र् योनिः॒ सयो᳚न्ये॒व । \newline
20. सयो᳚ न्ये॒वैव सयो॑नि॒ सयो᳚ न्ये॒वान्न॒ मन्न॑ मे॒व सयो॑नि॒ सयो᳚ न्ये॒वान्न᳚म् । \newline
21. सयो॒नीति॒ स - यो॒नि॒ । \newline
22. ए॒वान्न॒ मन्न॑ मे॒वै वान्न॒ मवावान्न॑ मे॒वै वान्न॒ मव॑ । \newline
23. अन्न॒ मवावान्न॒ मन्न॒ मव॑ रुन्धे रु॒न्धे ऽवान्न॒ मन्न॒ मव॑ रुन्धे । \newline
24. अव॑ रुन्धे रु॒न्धे ऽवाव॑ रुन्धे ऽर्द्धे॒न्द्राण्य॑ र्द्धे॒न्द्राणि॑ रु॒न्धे ऽवाव॑ रुन्धे ऽर्द्धे॒न्द्राणि॑ । \newline
25. रु॒न्धे॒ ऽर्द्धे॒न्द्राण्य॑ र्द्धे॒न्द्राणि॑ रुन्धे रुन्धे ऽर्द्धे॒न्द्राणि॑ जुहोति जुहो त्यर्द्धे॒न्द्राणि॑ रुन्धे रुन्धे ऽर्द्धे॒न्द्राणि॑ जुहोति । \newline
26. अ॒र्द्धे॒न्द्राणि॑ जुहोति जुहो त्यर्द्धे॒न्द्राण्य॑ र्द्धे॒न्द्राणि॑ जुहोति दे॒वता॑ दे॒वता॑ जुहो त्यर्द्धे॒न्द्राण्य॑ र्द्धे॒न्द्राणि॑ जुहोति दे॒वताः᳚ । \newline
27. अ॒र्द्धे॒न्द्राणीत्य॑र्द्ध - इ॒न्द्राणि॑ । \newline
28. जु॒हो॒ति॒ दे॒वता॑ दे॒वता॑ जुहोति जुहोति दे॒वता॑ ए॒वैव दे॒वता॑ जुहोति जुहोति दे॒वता॑ ए॒व । \newline
29. दे॒वता॑ ए॒वैव दे॒वता॑ दे॒वता॑ ए॒वावा वै॒व दे॒वता॑ दे॒वता॑ ए॒वाव॑ । \newline
30. ए॒वावा वै॒वै वाव॑ रुन्धे रु॒न्धे ऽवै॒वै वाव॑ रुन्धे । \newline
31. अव॑ रुन्धे रु॒न्धे ऽवाव॑ रुन्धे॒ यद् यद् रु॒न्धे ऽवाव॑ रुन्धे॒ यत् । \newline
32. रु॒न्धे॒ यद् यद् रु॑न्धे रुन्धे॒ यथ् सर्वे॑षाꣳ॒॒ सर्वे॑षां॒ ॅयद् रु॑न्धे रुन्धे॒ यथ् सर्वे॑षाम् । \newline
33. यथ् सर्वे॑षाꣳ॒॒ सर्वे॑षां॒ ॅयद् यथ् सर्वे॑षा म॒र्द्ध म॒र्द्धꣳ सर्वे॑षां॒ ॅयद् यथ् सर्वे॑षा म॒र्द्धम् । \newline
34. सर्वे॑षा म॒र्द्ध म॒र्द्धꣳ सर्वे॑षाꣳ॒॒ सर्वे॑षा म॒र्द्ध मिन्द्र॒ इन्द्रो॒ ऽर्द्धꣳ सर्वे॑षाꣳ॒॒ सर्वे॑षा म॒र्द्ध मिन्द्रः॑ । \newline
35. अ॒र्द्ध मिन्द्र॒ इन्द्रो॒ ऽर्द्ध म॒र्द्ध मिन्द्रः॒ प्रति॒ प्रतीन्द्रो॒ ऽर्द्ध म॒र्द्ध मिन्द्रः॒ प्रति॑ । \newline
36. इन्द्रः॒ प्रति॒ प्रतीन्द्र॒ इन्द्रः॒ प्रति॒ तस्मा॒त् तस्मा॒त् प्रतीन्द्र॒ इन्द्रः॒ प्रति॒ तस्मा᳚त् । \newline
37. प्रति॒ तस्मा॒त् तस्मा॒त् प्रति॒ प्रति॒ तस्मा॒ दिन्द्र॒ इन्द्र॒ स्तस्मा॒त् प्रति॒ प्रति॒ तस्मा॒ दिन्द्रः॑ । \newline
38. तस्मा॒ दिन्द्र॒ इन्द्र॒ स्तस्मा॒त् तस्मा॒ दिन्द्रो॑ दे॒वता॑नाम् दे॒वता॑ना॒ मिन्द्र॒ स्तस्मा॒त् तस्मा॒ दिन्द्रो॑ दे॒वता॑नाम् । \newline
39. इन्द्रो॑ दे॒वता॑नाम् दे॒वता॑ना॒ मिन्द्र॒ इन्द्रो॑ दे॒वता॑नाम् भूयिष्ठ॒भाक्त॑मो भूयिष्ठ॒भाक्त॑मो दे॒वता॑ना॒ मिन्द्र॒ इन्द्रो॑ दे॒वता॑नाम् भूयिष्ठ॒भाक्त॑मः । \newline
40. दे॒वता॑नाम् भूयिष्ठ॒भाक्त॑मो भूयिष्ठ॒भाक्त॑मो दे॒वता॑नाम् दे॒वता॑नाम् भूयिष्ठ॒भाक्त॑म॒ इन्द्र॒ मिन्द्र॑म् भूयिष्ठ॒भाक्त॑मो दे॒वता॑नाम् दे॒वता॑नाम् भूयिष्ठ॒भाक्त॑म॒ इन्द्र᳚म् । \newline
41. भू॒यि॒ष्ठ॒भाक्त॑म॒ इन्द्र॒ मिन्द्र॑म् भूयिष्ठ॒भाक्त॑मो भूयिष्ठ॒भाक्त॑म॒ इन्द्र॒ मुत्त॑र॒ मुत्त॑र॒ मिन्द्र॑म् भूयिष्ठ॒भाक्त॑मो भूयिष्ठ॒भाक्त॑म॒ इन्द्र॒ मुत्त॑रम् । \newline
42. भू॒यि॒ष्ठ॒भाक्त॑म॒ इति॑ भूयिष्ठ॒भाक् - त॒मः॒ । \newline
43. इन्द्र॒ मुत्त॑र॒ मुत्त॑र॒ मिन्द्र॒ मिन्द्र॒ मुत्त॑र माहा॒ होत्त॑र॒ मिन्द्र॒ मिन्द्र॒ मुत्त॑र माह । \newline
44. उत्त॑र माहा॒ होत्त॑र॒ मुत्त॑र माहे न्द्रि॒य मि॑न्द्रि॒य मा॒होत्त॑र॒ मुत्त॑र माहे न्द्रि॒यम् । \newline
45. उत्त॑र॒मित्युत् - त॒र॒म् । \newline
46. आ॒हे॒न्द्रि॒य मि॑न्द्रि॒य मा॑हाहे न्द्रि॒य मे॒वैवेन्द्रि॒य मा॑हा हेन्द्रि॒य मे॒व । \newline
47. इ॒न्द्रि॒य मे॒वै वेन्द्रि॒य मि॑न्द्रि॒य मे॒वास्मि॑न् नस्मिन् ने॒वेन्द्रि॒य मि॑न्द्रि॒य मे॒वास्मिन्न्॑ । \newline
48. ए॒वास्मि॑न् नस्मिन् ने॒वै वास्मि॑न् नु॒परि॑ष्टा दु॒परि॑ष्टा दस्मिन् ने॒वैवा स्मि॑न् नु॒परि॑ष्टात् । \newline
49. अ॒स्मि॒न् नु॒परि॑ष्टा दु॒परि॑ष्टा दस्मिन् नस्मिन् नु॒परि॑ष्टाद् दधाति दधा त्यु॒परि॑ष्टा दस्मिन् नस्मिन् नु॒परि॑ष्टाद् दधाति । \newline
50. उ॒परि॑ष्टाद् दधाति दधा त्यु॒परि॑ष्टा दु॒परि॑ष्टाद् दधाति यज्ञायु॒धानि॑ यज्ञायु॒धानि॑ दधा त्यु॒परि॑ष्टा दु॒परि॑ष्टाद् दधाति यज्ञायु॒धानि॑ । \newline
51. द॒धा॒ति॒ य॒ज्ञा॒यु॒धानि॑ यज्ञायु॒धानि॑ दधाति दधाति यज्ञायु॒धानि॑ जुहोति जुहोति यज्ञायु॒धानि॑ दधाति दधाति यज्ञायु॒धानि॑ जुहोति । \newline
52. य॒ज्ञा॒यु॒धानि॑ जुहोति जुहोति यज्ञायु॒धानि॑ यज्ञायु॒धानि॑ जुहोति य॒ज्ञो य॒ज्ञो जु॑होति यज्ञायु॒धानि॑ यज्ञायु॒धानि॑ जुहोति य॒ज्ञ्ः । \newline
53. य॒ज्ञा॒यु॒धानीति॑ यज्ञ् - आ॒यु॒धानि॑ । \newline
54. जु॒हो॒ति॒ य॒ज्ञो य॒ज्ञो जु॑होति जुहोति य॒ज्ञो वै वै य॒ज्ञो जु॑होति जुहोति य॒ज्ञो वै । \newline
55. य॒ज्ञो वै वै य॒ज्ञो य॒ज्ञो वै य॑ज्ञायु॒धानि॑ यज्ञायु॒धानि॒ वै य॒ज्ञो य॒ज्ञो वै य॑ज्ञायु॒धानि॑ । \newline
\pagebreak
\markright{ TS 5.4.8.4  \hfill https://www.vedavms.in \hfill}

\section{ TS 5.4.8.4 }

\textbf{TS 5.4.8.4 } \newline
\textbf{Samhita Paata} \newline

वै य॑ज्ञायु॒धा॑नि य॒ज्ञ्मे॒वाव॑ रु॒न्धेऽथो॑ ए॒तद्वै य॒ज्ञ्स्य॑ रू॒पꣳ रू॒पेणै॒व य॒ज्ञ्मव॑ रुन्धे ऽवभृ॒थश्च॑ मे स्वगाका॒रश्च॑ म॒ इत्या॑ह स्व॒गाकृ॑त्या अ॒ग्निश्च॑ मे घ॒र्मश्च॑ म॒ इत्या॑है॒तद् वै ब्र॑ह्मवर्च॒सस्य॑ रू॒पꣳ रू॒पेणै॒व ब्र॑ह्मवर्च॒समव॑ रुन्ध॒ ऋक्च॑ मे॒ साम॑ च म॒ इत्या॑है॒ - [  ] \newline

\textbf{Pada Paata} \newline

वै । य॒ज्ञा॒यु॒धानीति॑ यज्ञ्-आ॒यु॒धानि॑ । य॒ज्ञ्म् । ए॒व । अवेति॑ । रु॒न्धे॒ । अथो॒ इति॑ । ए॒तत् । वै । य॒ज्ञ्स्य॑ । रू॒पम् । रू॒पेण॑ । ए॒व । य॒ज्ञ्म् । अवेति॑ । रु॒न्धे॒ । अ॒व॒भृ॒थ इत्य॑व - भृ॒थः । च॒ । मे॒ । स्व॒गा॒का॒र इति॑ स्वगा - का॒रः । च॒ । मे॒ । इति॑ । आ॒ह॒ । स्व॒गाकृ॑त्या॒ इति॑ स्व॒गा - कृ॒त्यै॒ । अ॒ग्निः । च॒ । मे॒ । घ॒र्मः । च॒ । मे॒ । इति॑ । आ॒ह॒ । ए॒तत् । वै । ब्र॒ह्म॒व॒र्च॒सस्येति॑ ब्रह्म - व॒र्च॒सस्य॑ । रू॒पम् । रू॒पेण॑ । ए॒व । ब्र॒ह्म॒व॒र्च॒समिति॑ ब्रह्म - व॒र्च॒सम् । अवेति॑ । रु॒न्धे॒ । ऋक् । च॒ । मे॒ । साम॑ । च॒ । मे॒ । इति॑ । आ॒ह॒ ।  \newline


\textbf{Krama Paata} \newline

वै य॑ज्ञायु॒धानि॑ । य॒ज्ञा॒यु॒धानि॑ य॒ज्ञ्म् । य॒ज्ञा॒यु॒धानीति॑ यज्ञ् - आ॒यु॒धानि॑ । य॒ज्ञ्मे॒व । ए॒वाव॑ । अव॑ रुन्धे । रु॒न्धेऽथो᳚ । अथो॑ ए॒तत् । अथो॒ इत्यथो᳚ । ए॒तद् वै । वै य॒ज्ञ्स्य॑ । य॒ज्ञ्स्य॑ रू॒पम् । रू॒पꣳ रू॒पेण॑ । रू॒पेणै॒व । ए॒व य॒ज्ञ्म् । य॒ज्ञ्मव॑ । अव॑ रुन्धे । रु॒न्धे॒ऽव॒भृ॒थः । अ॒व॒भृ॒थश्च॑ । अ॒व॒भृ॒थ इत्य॑व - भृ॒थः । च॒ मे॒ । मे॒ स्व॒गा॒का॒रः । स्व॒गा॒का॒रश्च॑ । स्व॒गा॒का॒र इति॑ स्वगा - का॒रः । च॒ मे॒ । म॒ इति॑ । इत्या॑ह । आ॒ह॒ स्व॒गाकृ॑त्यै । स्व॒गाकृ॑त्या अ॒ग्निः । स्व॒गाकृ॑त्या॒ इति॑ स्व॒गा - कृ॒त्यै॒ । अ॒ग्निश्च॑ । च॒ मे॒ । मे॒ घ॒र्मः । घ॒र्मश्च॑ । च॒ मे॒ । म॒ इति॑ । इत्या॑ह । आ॒है॒तत् । ए॒तद् वै । वै ब्र॑ह्मवर्च॒सस्य॑ । ब्र॒ह्म॒व॒र्च॒सस्य॑ रू॒पम् । ब्र॒ह्म॒व॒र्च॒सस्येति॑ ब्रह्म - व॒र्च॒सस्य॑ । रू॒पꣳ रू॒पेण॑ । रू॒पेणै॒व । ए॒व ब्र॑ह्मवर्च॒सम् । ब्र॒ह्म॒व॒र्च॒समव॑ । ब्र॒ह्म॒व॒र्च॒समिति॑ ब्रह्म - व॒र्च॒सम् । अव॑ रुन्धे । रु॒न्ध॒ ऋक् । ऋक् च॑ । च॒ मे॒ । मे॒ साम॑ । साम॑ च । च॒ मे॒ । म॒ इति॑ । इत्या॑ह । आ॒है॒तत् \newline

\textbf{Jatai Paata} \newline

1. वै य॑ज्ञायु॒धानि॑ यज्ञायु॒धानि॒ वै वै य॑ज्ञायु॒धानि॑ । \newline
2. य॒ज्ञा॒यु॒धानि॑ य॒ज्ञ्ं ॅय॒ज्ञ्ं ॅय॑ज्ञायु॒धानि॑ यज्ञायु॒धानि॑ य॒ज्ञ्म् । \newline
3. य॒ज्ञा॒यु॒धानीति॑ यज्ञ् - आ॒यु॒धानि॑ । \newline
4. य॒ज्ञ् मे॒वैव य॒ज्ञ्ं ॅय॒ज्ञ् मे॒व । \newline
5. ए॒वावा वै॒वै वाव॑ । \newline
6. अव॑ रुन्धे रु॒न्धे ऽवाव॑ रुन्धे । \newline
7. रु॒न्धे ऽथो॒ अथो॑ रुन्धे रु॒न्धे ऽथो᳚ । \newline
8. अथो॑ ए॒त दे॒त दथो॒ अथो॑ ए॒तत् । \newline
9. अथो॒ इत्यथो᳚ । \newline
10. ए॒तद् वै वा ए॒त दे॒तद् वै । \newline
11. वै य॒ज्ञ्स्य॑ य॒ज्ञ्स्य॒ वै वै य॒ज्ञ्स्य॑ । \newline
12. य॒ज्ञ्स्य॑ रू॒पꣳ रू॒पं ॅय॒ज्ञ्स्य॑ य॒ज्ञ्स्य॑ रू॒पम् । \newline
13. रू॒पꣳ रू॒पेण॑ रू॒पेण॑ रू॒पꣳ रू॒पꣳ रू॒पेण॑ । \newline
14. रू॒पेणै॒ वैव रू॒पेण॑ रू॒पेणै॒व । \newline
15. ए॒व य॒ज्ञ्ं ॅय॒ज्ञ् मे॒वैव य॒ज्ञ्म् । \newline
16. य॒ज्ञ् मवाव॑ य॒ज्ञ्ं ॅय॒ज्ञ् मव॑ । \newline
17. अव॑ रुन्धे रु॒न्धे ऽवाव॑ रुन्धे । \newline
18. रु॒न्धे॒ ऽव॒भृ॒थो॑ ऽवभृ॒थो रु॑न्धे रुन्धे ऽवभृ॒थः । \newline
19. अ॒व॒भृ॒थ श्च॑ चावभृ॒थो॑ ऽवभृ॒थ श्च॑ । \newline
20. अ॒व॒भृ॒थ इत्य॑व - भृ॒थः । \newline
21. च॒ मे॒ मे॒ च॒ च॒ मे॒ । \newline
22. मे॒ स्व॒गा॒का॒रः स्व॑गाका॒रो मे॑ मे स्वगाका॒रः । \newline
23. स्व॒गा॒का॒रश्च॑ च स्वगाका॒रः स्व॑गाका॒रश्च॑ । \newline
24. स्व॒गा॒का॒र इति॑ स्वगा - का॒रः । \newline
25. च॒ मे॒ मे॒ च॒ च॒ मे॒ । \newline
26. म॒ इतीति॑ मे म॒ इति॑ । \newline
27. इत्या॑हा॒हे तीत्या॑ह । \newline
28. आ॒ह॒ स्व॒गाकृ॑त्यै स्व॒गाकृ॑त्या आहाह स्व॒गाकृ॑त्यै । \newline
29. स्व॒गाकृ॑त्या अ॒ग्नि र॒ग्निः स्व॒गाकृ॑त्यै स्व॒गाकृ॑त्या अ॒ग्निः । \newline
30. स्व॒गाकृ॑त्या॒ इति॑ स्व॒गा - कृ॒त्यै॒ । \newline
31. अ॒ग्निश्च॑ चा॒ग्नि र॒ग्निश्च॑ । \newline
32. च॒ मे॒ मे॒ च॒ च॒ मे॒ । \newline
33. मे॒ घ॒र्मो घ॒र्मो मे॑ मे घ॒र्मः । \newline
34. घ॒र्मश्च॑ च घ॒र्मो घ॒र्मश्च॑ । \newline
35. च॒ मे॒ मे॒ च॒ च॒ मे॒ । \newline
36. म॒ इतीति॑ मे म॒ इति॑ । \newline
37. इत्या॑हा॒हे तीत्या॑ह । \newline
38. आ॒है॒त दे॒त दा॑हा है॒तत् । \newline
39. ए॒तद् वै वा ए॒त दे॒तद् वै । \newline
40. वै ब्र॑ह्मवर्च॒सस्य॑ ब्रह्मवर्च॒सस्य॒ वै वै ब्र॑ह्मवर्च॒सस्य॑ । \newline
41. ब्र॒ह्म॒व॒र्च॒सस्य॑ रू॒पꣳ रू॒पम् ब्र॑ह्मवर्च॒सस्य॑ ब्रह्मवर्च॒सस्य॑ रू॒पम् । \newline
42. ब्र॒ह्म॒व॒र्च॒सस्येति॑ ब्रह्म - व॒र्च॒सस्य॑ । \newline
43. रू॒पꣳ रू॒पेण॑ रू॒पेण॑ रू॒पꣳ रू॒पꣳ रू॒पेण॑ । \newline
44. रू॒पे णै॒वैव रू॒पेण॑ रू॒पेणै॒व । \newline
45. ए॒व ब्र॑ह्मवर्च॒सम् ब्र॑ह्मवर्च॒स मे॒वैव ब्र॑ह्मवर्च॒सम् । \newline
46. ब्र॒ह्म॒व॒र्च॒स मवाव॑ ब्रह्मवर्च॒सम् ब्र॑ह्मवर्च॒स मव॑ । \newline
47. ब्र॒ह्म॒व॒र्च॒समिति॑ ब्रह्म - व॒र्च॒सम् । \newline
48. अव॑ रुन्धे रु॒न्धे ऽवाव॑ रुन्धे । \newline
49. रु॒न्ध॒ ऋगृग् रु॑न्धे रुन्ध॒ ऋक् । \newline
50. ऋक् च॒ च र्गृक् च॑ । \newline
51. च॒ मे॒ मे॒ च॒ च॒ मे॒ । \newline
52. मे॒ साम॒ साम॑ मे मे॒ साम॑ । \newline
53. साम॑ च च॒ साम॒ साम॑ च । \newline
54. च॒ मे॒ मे॒ च॒ च॒ मे॒ । \newline
55. म॒ इतीति॑ मे म॒ इति॑ । \newline
56. इत्या॑हा॒हे तीत्या॑ह । \newline
57. आ॒है॒त दे॒त दा॑हा है॒तत् । \newline

\textbf{Ghana Paata } \newline

1. वै य॑ज्ञायु॒धानि॑ यज्ञायु॒धानि॒ वै वै य॑ज्ञायु॒धानि॑ य॒ज्ञ्ं ॅय॒ज्ञ्ं ॅय॑ज्ञायु॒धानि॒ वै वै य॑ज्ञायु॒धानि॑ य॒ज्ञ्म् । \newline
2. य॒ज्ञा॒यु॒धानि॑ य॒ज्ञ्ं ॅय॒ज्ञ्ं ॅय॑ज्ञायु॒धानि॑ यज्ञायु॒धानि॑ य॒ज्ञ् मे॒वैव य॒ज्ञ्ं ॅय॑ज्ञायु॒धानि॑ यज्ञायु॒धानि॑ य॒ज्ञ् मे॒व । \newline
3. य॒ज्ञा॒यु॒धानीति॑ यज्ञ् - आ॒यु॒धानि॑ । \newline
4. य॒ज्ञ् मे॒वैव य॒ज्ञ्ं ॅय॒ज्ञ् मे॒वावा वै॒व य॒ज्ञ्ं ॅय॒ज्ञ् मे॒वाव॑ । \newline
5. ए॒वावा वै॒वै वाव॑ रुन्धे रु॒न्धे ऽवै॒वै वाव॑ रुन्धे । \newline
6. अव॑ रुन्धे रु॒न्धे ऽवाव॑ रु॒न्धे ऽथो॒ अथो॑ रु॒न्धे ऽवाव॑ रु॒न्धे ऽथो᳚ । \newline
7. रु॒न्धे ऽथो॒ अथो॑ रुन्धे रु॒न्धे ऽथो॑ ए॒त दे॒त दथो॑ रुन्धे रु॒न्धे ऽथो॑ ए॒तत् । \newline
8. अथो॑ ए॒त दे॒त दथो॒ अथो॑ ए॒तद् वै वा ए॒त दथो॒ अथो॑ ए॒तद् वै । \newline
9. अथो॒ इत्यथो᳚ । \newline
10. ए॒तद् वै वा ए॒त दे॒तद् वै य॒ज्ञ्स्य॑ य॒ज्ञ्स्य॒ वा ए॒त दे॒तद् वै य॒ज्ञ्स्य॑ । \newline
11. वै य॒ज्ञ्स्य॑ य॒ज्ञ्स्य॒ वै वै य॒ज्ञ्स्य॑ रू॒पꣳ रू॒पं ॅय॒ज्ञ्स्य॒ वै वै य॒ज्ञ्स्य॑ रू॒पम् । \newline
12. य॒ज्ञ्स्य॑ रू॒पꣳ रू॒पं ॅय॒ज्ञ्स्य॑ य॒ज्ञ्स्य॑ रू॒पꣳ रू॒पेण॑ रू॒पेण॑ रू॒पं ॅय॒ज्ञ्स्य॑ य॒ज्ञ्स्य॑ रू॒पꣳ रू॒पेण॑ । \newline
13. रू॒पꣳ रू॒पेण॑ रू॒पेण॑ रू॒पꣳ रू॒पꣳ रू॒पे णै॒वैव रू॒पेण॑ रू॒पꣳ रू॒पꣳ रू॒पेणै॒व । \newline
14. रू॒पे णै॒वैव रू॒पेण॑ रू॒पेणै॒व य॒ज्ञ्ं ॅय॒ज्ञ् मे॒व रू॒पेण॑ रू॒पेणै॒व य॒ज्ञ्म् । \newline
15. ए॒व य॒ज्ञ्ं ॅय॒ज्ञ् मे॒वैव य॒ज्ञ् मवाव॑ य॒ज्ञ् मे॒वैव य॒ज्ञ् मव॑ । \newline
16. य॒ज्ञ् मवाव॑ य॒ज्ञ्ं ॅय॒ज्ञ् मव॑ रुन्धे रु॒न्धे ऽव॑ य॒ज्ञ्ं ॅय॒ज्ञ् मव॑ रुन्धे । \newline
17. अव॑ रुन्धे रु॒न्धे ऽवाव॑ रुन्धे ऽवभृ॒थो॑ ऽवभृ॒थो रु॒न्धे ऽवाव॑ रुन्धे ऽवभृ॒थः । \newline
18. रु॒न्धे॒ ऽव॒भृ॒थो॑ ऽवभृ॒थो रु॑न्धे रुन्धे ऽवभृ॒थश्च॑ चावभृ॒थो रु॑न्धे रुन्धे ऽवभृ॒थश्च॑ । \newline
19. अ॒व॒भृ॒थश्च॑ चावभृ॒थो॑ ऽवभृ॒थश्च॑ मे मे चावभृ॒थो॑ ऽवभृ॒थश्च॑ मे । \newline
20. अ॒व॒भृ॒थ इत्य॑व - भृ॒थः । \newline
21. च॒ मे॒ मे॒ च॒ च॒ मे॒ स्व॒गा॒का॒रः स्व॑गाका॒रो मे॑ च च मे स्वगाका॒रः । \newline
22. मे॒ स्व॒गा॒का॒रः स्व॑गाका॒रो मे॑ मे स्वगाका॒रश्च॑ च स्वगाका॒रो मे॑ मे स्वगाका॒रश्च॑ । \newline
23. स्व॒गा॒का॒रश्च॑ च स्वगाका॒रः स्व॑गाका॒रश्च॑ मे मे च स्वगाका॒रः स्व॑गाका॒रश्च॑ मे । \newline
24. स्व॒गा॒का॒र इति॑ स्वगा - का॒रः । \newline
25. च॒ मे॒ मे॒ च॒ च॒ म॒ इतीति॑ मे च च म॒ इति॑ । \newline
26. म॒ इतीति॑ मे म॒ इत्या॑ हा॒हेति॑ मे म॒ इत्या॑ह । \newline
27. इत्या॑ हा॒हे तीत्या॑ह स्व॒गाकृ॑त्यै स्व॒गाकृ॑त्या आ॒हे तीत्या॑ह स्व॒गाकृ॑त्यै । \newline
28. आ॒ह॒ स्व॒गाकृ॑त्यै स्व॒गाकृ॑त्या आहाह स्व॒गाकृ॑त्या अ॒ग्नि र॒ग्निः स्व॒गाकृ॑त्या आहाह स्व॒गाकृ॑त्या अ॒ग्निः । \newline
29. स्व॒गाकृ॑त्या अ॒ग्नि र॒ग्निः स्व॒गाकृ॑त्यै स्व॒गाकृ॑त्या अ॒ग्निश्च॑ चा॒ग्निः स्व॒गाकृ॑त्यै स्व॒गाकृ॑त्या अ॒ग्निश्च॑ । \newline
30. स्व॒गाकृ॑त्या॒ इति॑ स्व॒गा - कृ॒त्यै॒ । \newline
31. अ॒ग्निश्च॑ चा॒ग्नि र॒ग्निश्च॑ मे मे चा॒ग्नि र॒ग्निश्च॑ मे । \newline
32. च॒ मे॒ मे॒ च॒ च॒ मे॒ घ॒र्मो घ॒र्मो मे॑ च च मे घ॒र्मः । \newline
33. मे॒ घ॒र्मो घ॒र्मो मे॑ मे घ॒र्मश्च॑ च घ॒र्मो मे॑ मे घ॒र्मश्च॑ । \newline
34. घ॒र्मश्च॑ च घ॒र्मो घ॒र्मश्च॑ मे मे च घ॒र्मो घ॒र्मश्च॑ मे । \newline
35. च॒ मे॒ मे॒ च॒ च॒ म॒ इतीति॑ मे च च म॒ इति॑ । \newline
36. म॒ इतीति॑ मे म॒ इत्या॑ हा॒हेति॑ मे म॒ इत्या॑ह । \newline
37. इत्या॑हा॒हे तीत्या॑ है॒त दे॒तदा॒हे तीत्या॑ है॒तत् । \newline
38. आ॒है॒त दे॒त दा॑हा है॒तद् वै वा ए॒त दा॑हा है॒तद् वै । \newline
39. ए॒तद् वै वा ए॒तदे॒तद् वै ब्र॑ह्मवर्च॒सस्य॑ ब्रह्मवर्च॒सस्य॒ वा ए॒त दे॒तद् वै ब्र॑ह्मवर्च॒सस्य॑ । \newline
40. वै ब्र॑ह्मवर्च॒सस्य॑ ब्रह्मवर्च॒सस्य॒ वै वै ब्र॑ह्मवर्च॒सस्य॑ रू॒पꣳ रू॒पम् ब्र॑ह्मवर्च॒सस्य॒ वै वै ब्र॑ह्मवर्च॒सस्य॑ रू॒पम् । \newline
41. ब्र॒ह्म॒व॒र्च॒सस्य॑ रू॒पꣳ रू॒पम् ब्र॑ह्मवर्च॒सस्य॑ ब्रह्मवर्च॒सस्य॑ रू॒पꣳ रू॒पेण॑ रू॒पेण॑ रू॒पम् ब्र॑ह्मवर्च॒सस्य॑ ब्रह्मवर्च॒सस्य॑ रू॒पꣳ रू॒पेण॑ । \newline
42. ब्र॒ह्म॒व॒र्च॒सस्येति॑ ब्रह्म - व॒र्च॒सस्य॑ । \newline
43. रू॒पꣳ रू॒पेण॑ रू॒पेण॑ रू॒पꣳ रू॒पꣳ रू॒पेणै॒वैव रू॒पेण॑ रू॒पꣳ रू॒पꣳ रू॒पेणै॒व । \newline
44. रू॒पेणै॒ वैव रू॒पेण॑ रू॒पेणै॒व ब्र॑ह्मवर्च॒सम् ब्र॑ह्मवर्च॒स मे॒व रू॒पेण॑ रू॒पेणै॒व ब्र॑ह्मवर्च॒सम् । \newline
45. ए॒व ब्र॑ह्मवर्च॒सम् ब्र॑ह्मवर्च॒स मे॒वैव ब्र॑ह्मवर्च॒स मवाव॑ ब्रह्मवर्च॒स मे॒वैव ब्र॑ह्मवर्च॒स मव॑ । \newline
46. ब्र॒ह्म॒व॒र्च॒स मवाव॑ ब्रह्मवर्च॒सम् ब्र॑ह्मवर्च॒स मव॑ रुन्धे रु॒न्धे ऽव॑ ब्रह्मवर्च॒सम् ब्र॑ह्मवर्च॒स मव॑ रुन्धे । \newline
47. ब्र॒ह्म॒व॒र्च॒समिति॑ ब्रह्म - व॒र्च॒सम् । \newline
48. अव॑ रुन्धे रु॒न्धे ऽवाव॑ रुन्ध॒ ऋ गृग् रु॒न्धे ऽवाव॑ रुन्ध॒ ऋक् । \newline
49. रु॒न्ध॒ ऋगृग् रु॑न्धे रुन्ध॒ ऋक् च॒ चर्ग् रु॑न्धे रुन्ध॒ ऋक् च॑ । \newline
50. ऋक् च॒ च र्गृक् च॑ मे मे॒ च र्गृक् च॑ मे । \newline
51. च॒ मे॒ मे॒ च॒ च॒ मे॒ साम॒ साम॑ मे च च मे॒ साम॑ । \newline
52. मे॒ साम॒ साम॑ मे मे॒ साम॑ च च॒ साम॑ मे मे॒ साम॑ च । \newline
53. साम॑ च च॒ साम॒ साम॑ च मे मे च॒ साम॒ साम॑ च मे । \newline
54. च॒ मे॒ मे॒ च॒ च॒ म॒ इतीति॑ मे च च म॒ इति॑ । \newline
55. म॒ इतीति॑ मे म॒ इत्या॑ हा॒हेति॑ मे म॒ इत्या॑ह । \newline
56. इत्या॑हा॒हे तीत्या॑ है॒त दे॒तदा॒हे तीत्या॑है॒तत् । \newline
57. आ॒है॒ तदे॒ तदा॑हा है॒तद् वै वा ए॒तदा॑हा है॒तद् वै । \newline
\pagebreak
\markright{ TS 5.4.8.5  \hfill https://www.vedavms.in \hfill}

\section{ TS 5.4.8.5 }

\textbf{TS 5.4.8.5 } \newline
\textbf{Samhita Paata} \newline

-तद्वै छन्द॑साꣳ रू॒पꣳ रू॒पेणै॒व छन्दाꣳ॒॒स्यव॑ रुन्धे॒ गर्भा᳚श्च मे व॒थ्साश्च॑ म॒ इत्या॑है॒तद् वै प॑शू॒नाꣳ रू॒पꣳ रू॒पेणै॒व प॒शूनव॑ रुन्धे॒ कल्पा᳚न् जुहो॒त्य क्लृ॑प्तस्य॒ क्लृप्त्यै॑ युग्मदयु॒जे जु॑होति मिथुन॒त्वायो᳚-त्त॒राव॑ती भवतो॒ऽभिक्रा᳚न्त्या॒ एका॑ च मे ति॒स्रश्च॑ म॒ इत्या॑ह देवछन्द॒सं ॅवा एका॑ च ति॒स्रश्च॑ - [  ] \newline

\textbf{Pada Paata} \newline

ए॒तत् । वै । छन्द॑साम् । रू॒पम् । रू॒पेण॑ । ए॒व । छन्दाꣳ॑सि । अवेति॑ । रु॒न्धे॒ । गर्भाः᳚ । च॒ । मे॒ । व॒थ्साः । च॒ । मे॒ । इति॑ । आ॒ह॒ । ए॒तत् । वै । प॒शू॒नाम् । रू॒पम् । रू॒पेण॑ । ए॒व । प॒शून् । अवेति॑ । रु॒न्धे॒ । कल्पान्॑ । जु॒हो॒ति॒ । अक्लृ॑प्तस्य । क्लृप्त्यै᳚ । यु॒ग्म॒द॒यु॒जे इति॑ युग्मत् - अ॒यु॒जे । जु॒हो॒ति॒ । मि॒थु॒न॒त्वायेति॑ मिथुन - त्वाय॑ । उ॒त्त॒राव॑ती॒ इत्यु॑त्त॒रा-व॒ती॒ । भ॒व॒तः॒ । अ॒भिक्रा᳚न्त्या॒ इत्य॒भि - क्रा॒न्त्यै॒ । एका᳚ । च॒ । मे॒ । ति॒स्रः । च॒ । मे॒ । इति॑ । आ॒ह॒ । दे॒व॒छ॒न्द॒समिति॑ देव - छ॒न्द॒सम् । वै । एका᳚ । च॒ । ति॒स्रः । च॒ ।  \newline


\textbf{Krama Paata} \newline

ए॒तद् वै । वै छन्द॑साम् । छन्द॑साꣳ रू॒पम् । रू॒पꣳ रू॒पेण॑ । रू॒पेणै॒व । ए॒व छन्दाꣳ॑सि । छन्दाꣳ॒॒स्यव॑ । अव॑ रुन्धे । रु॒न्धे॒ गर्भाः᳚ । गर्भा᳚श्च । च॒ मे॒ । मे॒ व॒थ्साः । व॒थ्साश्च॑ । च॒ मे॒ । म॒ इति॑ । इत्या॑ह । आ॒है॒तत् । ए॒तद् वै । वै प॑शू॒नाम् । प॒शू॒नाꣳ रू॒पम् । रू॒पꣳ रू॒पेण॑ । रू॒पेणै॒व । ए॒व प॒शून् । प॒शूनव॑ । अव॑ रुन्धे । रु॒न्धे॒ कल्पान्॑ । कल्पा᳚न् जुहोति । जु॒हो॒त्यक्लृ॑प्तस्य । अक्लृ॑प्तस्य॒ क्लृप्त्यै᳚ । क्लृप्त्यै॑ युग्मदयु॒जे । यु॒ग्म॒द॒यु॒जे जु॑होति । यु॒ग्म॒द॒यु॒जे इति॑ युग्मत् - अ॒यु॒जे । जु॒हो॒ति॒ मि॒थु॒न॒त्वाय॑ । मि॒थु॒न॒त्वायो᳚त्त॒राव॑ती । मि॒थु॒न॒त्वायेति॑ मिथुन - त्वाय॑ । उ॒त्त॒राव॑ती भवतः । उ॒त्त॒राव॑ती॒ इत्यु॑त्त॒रा - व॒ती॒ । भ॒व॒तो॒ऽभिक्रा᳚न्त्यै । अ॒भिक्रा᳚न्त्या॒ एका᳚ । अ॒भिक्रा᳚न्त्या॒ इत्य॒भि - क्रा॒न्त्यै॒ । एका॑ च । च॒ मे॒ । मे॒ ति॒स्रः । ति॒स्रश्च॑ । च॒ मे॒ । म॒ इति॑ । इत्या॑ह । आ॒ह॒ दे॒व॒छ॒न्द॒सम् । दे॒व॒छ॒न्द॒सम् ॅवै । दे॒व॒छ॒न्द॒समिति॑ देव - छ॒न्द॒सम् । वा एका᳚ । एका॑ च । च॒ ति॒स्रः ( ) । ति॒स्रश्च॑ । च॒ म॒नु॒ष्य॒छ॒न्द॒सम् \newline

\textbf{Jatai Paata} \newline

1. ए॒तद् वै वा ए॒त दे॒तद् वै । \newline
2. वै छन्द॑सा॒म् छन्द॑सां॒ ॅवै वै छन्द॑साम् । \newline
3. छन्द॑साꣳ रू॒पꣳ रू॒पम् छन्द॑सा॒म् छन्द॑साꣳ रू॒पम् । \newline
4. रू॒पꣳ रू॒पेण॑ रू॒पेण॑ रू॒पꣳ रू॒पꣳ रू॒पेण॑ । \newline
5. रू॒पे णै॒वैव रू॒पेण॑ रू॒पेणै॒व । \newline
6. ए॒व छन्दाꣳ॑सि॒ छन्दाꣳ॑ स्ये॒वैव छन्दाꣳ॑सि । \newline
7. छन्दाꣳ॒ स्यवाव॒ च्छन्दाꣳ॑सि॒ छन्दाꣳ॒स्यव॑ । \newline
8. अव॑ रुन्धे रु॒न्धे ऽवाव॑ रुन्धे । \newline
9. रु॒न्धे॒ गर्भा॒ गर्भा॑ रुन्धे रुन्धे॒ गर्भाः᳚ । \newline
10. गर्भा᳚श्च च॒ गर्भा॒ गर्भा᳚श्च । \newline
11. च॒ मे॒ मे॒ च॒ च॒ मे॒ । \newline
12. मे॒ व॒थ्सा व॒थ्सा मे॑ मे व॒थ्साः । \newline
13. व॒थ्साश्च॑ च व॒थ्सा व॒थ्साश्च॑ । \newline
14. च॒ मे॒ मे॒ च॒ च॒ मे॒ । \newline
15. म॒ इतीति॑ मे म॒ इति॑ । \newline
16. इत्या॑हा॒हे तीत्या॑ह । \newline
17. आ॒है॒त दे॒त दा॑हा है॒तत् । \newline
18. ए॒तद् वै वा ए॒त दे॒तद् वै । \newline
19. वै प॑शू॒नाम् प॑शू॒नां ॅवै वै प॑शू॒नाम् । \newline
20. प॒शू॒नाꣳ रू॒पꣳ रू॒पम् प॑शू॒नाम् प॑शू॒नाꣳ रू॒पम् । \newline
21. रू॒पꣳ रू॒पेण॑ रू॒पेण॑ रू॒पꣳ रू॒पꣳ रू॒पेण॑ । \newline
22. रू॒पे णै॒वैव रू॒पेण॑ रू॒पेणै॒व । \newline
23. ए॒व प॒शून् प॒शू ने॒वैव प॒शून् । \newline
24. प॒शू नवाव॑ प॒शून् प॒शू नव॑ । \newline
25. अव॑ रुन्धे रु॒न्धे ऽवाव॑ रुन्धे । \newline
26. रु॒न्धे॒ कल्पा॒न् कल्पा᳚न् रुन्धे रुन्धे॒ कल्पान्॑ । \newline
27. कल्पा᳚न् जुहोति जुहोति॒ कल्पा॒न् कल्पा᳚न् जुहोति । \newline
28. जु॒हो॒ त्यक्लृ॑प्त॒स्या क्लृ॑प्तस्य जुहोति जुहो॒ त्यक्लृ॑प्तस्य । \newline
29. अक्लृ॑प्तस्य॒ क्लृप्त्यै॒ क्लृप्त्या॒ अक्लृ॑प्त॒स्या क्लृ॑प्तस्य॒ क्लृप्त्यै᳚ । \newline
30. क्लृप्त्यै॑ युग्मदयु॒जे यु॑ग्मदयु॒जे क्लृप्त्यै॒ क्लृप्त्यै॑ युग्मदयु॒जे । \newline
31. यु॒ग्म॒द॒यु॒जे जु॑होति जुहोति युग्मदयु॒जे यु॑ग्मदयु॒जे जु॑होति । \newline
32. यु॒ग्म॒द॒यु॒जे इति॑ युग्मत् - अ॒यु॒जे । \newline
33. जु॒हो॒ति॒ मि॒थु॒न॒त्वाय॑ मिथुन॒त्वाय॑ जुहोति जुहोति मिथुन॒त्वाय॑ । \newline
34. मि॒थु॒न॒त्वा यो᳚त्त॒राव॑ती उत्त॒राव॑ती मिथुन॒त्वाय॑ मिथुन॒त्वा यो᳚त्त॒राव॑ती । \newline
35. मि॒थु॒न॒त्वायेति॑ मिथुन - त्वाय॑ । \newline
36. उ॒त्त॒राव॑ती भवतो भवत उत्त॒राव॑ती उत्त॒राव॑ती भवतः । \newline
37. उ॒त्त॒राव॑ती॒ इत्यु॑त्त॒रा - व॒ती॒ । \newline
38. भ॒व॒तो॒ ऽभिक्रा᳚न्त्या अ॒भिक्रा᳚न्त्यै भवतो भवतो॒ ऽभिक्रा᳚न्त्यै । \newline
39. अ॒भिक्रा᳚न्त्या॒ एकैका॒ ऽभिक्रा᳚न्त्या अ॒भिक्रा᳚न्त्या॒ एका᳚ । \newline
40. अ॒भिक्रा᳚न्त्या॒ इत्य॒भि - क्रा॒न्त्यै॒ । \newline
41. एका॑ च॒ चैकैका॑ च । \newline
42. च॒ मे॒ मे॒ च॒ च॒ मे॒ । \newline
43. मे॒ ति॒स्र स्ति॒स्रो मे॑ मे ति॒स्रः । \newline
44. ति॒स्र श्च॑ च ति॒स्र स्ति॒स्र श्च॑ । \newline
45. च॒ मे॒ मे॒ च॒ च॒ मे॒ । \newline
46. म॒ इतीति॑ मे म॒ इति॑ । \newline
47. इत्या॑हा॒हे तीत्या॑ह । \newline
48. आ॒ह॒ दे॒व॒छ॒न्द॒सम् दे॑वछन्द॒स मा॑हाह देवछन्द॒सम् । \newline
49. दे॒व॒छ॒न्द॒सं ॅवै वै दे॑वछन्द॒सम् दे॑वछन्द॒सं ॅवै । \newline
50. दे॒व॒छ॒न्द॒समिति॑ देव - छ॒न्द॒सम् । \newline
51. वा एकैका॒ वै वा एका᳚ । \newline
52. एका॑ च॒ चैकैका॑ च । \newline
53. च॒ ति॒स्र स्ति॒स्र श्च॑ च ति॒स्रः । \newline
54. ति॒स्र श्च॑ च ति॒स्र स्ति॒स्र श्च॑ । \newline
55. च॒ म॒नु॒ष्य॒छ॒न्द॒सम् म॑नुष्यछन्द॒सम् च॑ च मनुष्यछन्द॒सम् । \newline

\textbf{Ghana Paata } \newline

1. ए॒तद् वै वा ए॒त दे॒तद् वै छन्द॑सा॒म् छन्द॑सां॒ ॅवा ए॒त दे॒तद् वै छन्द॑साम् । \newline
2. वै छन्द॑सा॒म् छन्द॑सां॒ ॅवै वै छन्द॑साꣳ रू॒पꣳ रू॒पम् छन्द॑सां॒ ॅवै वै छन्द॑साꣳ रू॒पम् । \newline
3. छन्द॑साꣳ रू॒पꣳ रू॒पम् छन्द॑सा॒म् छन्द॑साꣳ रू॒पꣳ रू॒पेण॑ रू॒पेण॑ रू॒पम् छन्द॑सा॒म् छन्द॑साꣳ रू॒पꣳ रू॒पेण॑ । \newline
4. रू॒पꣳ रू॒पेण॑ रू॒पेण॑ रू॒पꣳ रू॒पꣳ रू॒पे णै॒वैव रू॒पेण॑ रू॒पꣳ रू॒पꣳ रू॒पेणै॒व । \newline
5. रू॒पेणै॒वैव रू॒पेण॑ रू॒पेणै॒व छन्दाꣳ॑सि॒ छन्दाꣳ॑स्ये॒व रू॒पेण॑ रू॒पेणै॒व छन्दाꣳ॑सि । \newline
6. ए॒व छन्दाꣳ॑सि॒ छन्दाꣳ॑ स्ये॒वैव च्छन्दाꣳ॒ स्यवाव॒ च्छन्दाꣳ॑ स्ये॒वैव च्छन्दाꣳ॑सि॒दव॑ । \newline
7. छन्दाꣳ॒स्यव वाव॒ च्छन्दाꣳ॑सि॒ छन्दाꣳ॒स्यव॑ रुन्धे रु॒न्धे ऽव॒ च्छन्दाꣳ॑सि॒ 
छन्दाꣳ॒स्यव॑ रुन्धे । \newline
8. अव॑ रुन्धे रु॒न्धे ऽवाव॑ रुन्धे॒ गर्भा॒ गर्भा॑ रु॒न्धे ऽवाव॑ रुन्धे॒ गर्भाः᳚ । \newline
9. रु॒न्धे॒ गर्भा॒ गर्भा॑ रुन्धे रुन्धे॒ गर्भा᳚श्च च॒ गर्भा॑ रुन्धे रुन्धे॒ गर्भा᳚श्च । \newline
10. गर्भा᳚श्च च॒ गर्भा॒ गर्भा᳚श्च मे मे च॒ गर्भा॒ गर्भा᳚श्च मे । \newline
11. च॒ मे॒ मे॒ च॒ च॒ मे॒ व॒थ्सा व॒थ्सा मे॑ च च मे व॒थ्साः । \newline
12. मे॒ व॒थ्सा व॒थ्सा मे॑ मे व॒थ्साश्च॑ च व॒थ्सा मे॑ मे व॒थ्साश्च॑ । \newline
13. व॒थ्साश्च॑ च व॒थ्सा व॒थ्साश्च॑ मे मे च व॒थ्सा व॒थ्साश्च॑ मे । \newline
14. च॒ मे॒ मे॒ च॒ च॒ म॒ इतीति॑ मे च च म॒ इति॑ । \newline
15. म॒ इतीति॑ मे म॒ इत्या॑ हा॒हेति॑ मे म॒ इत्या॑ह । \newline
16. इत्या॑ हा॒हे तीत्या॑ है॒त दे॒तदा॒हे तीत्या॑ है॒तत् । \newline
17. आ॒है॒ तदे॒ तदा॑हा है॒तद् वै वा ए॒त दा॑हा है॒तद् वै । \newline
18. ए॒तद् वै वा ए॒त दे॒तद् वै प॑शू॒नाम् प॑शू॒नां ॅवा ए॒त दे॒तद् वै प॑शू॒नाम् । \newline
19. वै प॑शू॒नाम् प॑शू॒नां ॅवै वै प॑शू॒नाꣳ रू॒पꣳ रू॒पम् प॑शू॒नां ॅवै वै प॑शू॒नाꣳ रू॒पम् । \newline
20. प॒शू॒नाꣳ रू॒पꣳ रू॒पम् प॑शू॒नाम् प॑शू॒नाꣳ रू॒पꣳ रू॒पेण॑ रू॒पेण॑ रू॒पम् प॑शू॒नाम् प॑शू॒नाꣳ रू॒पꣳ रू॒पेण॑ । \newline
21. रू॒पꣳ रू॒पेण॑ रू॒पेण॑ रू॒पꣳ रू॒पꣳ रू॒पे णै॒वैव रू॒पेण॑ रू॒पꣳ रू॒पꣳ रू॒पेणै॒व । \newline
22. रू॒पेणै॒ वैव रू॒पेण॑ रू॒पेणै॒व प॒शून् प॒शू ने॒व रू॒पेण॑ रू॒पेणै॒व प॒शून् । \newline
23. ए॒व प॒शून् प॒शू ने॒वैव प॒शू नवाव॑ प॒शू ने॒वैव प॒शू नव॑ । \newline
24. प॒शू नवाव॑ प॒शून् प॒शू नव॑ रुन्धे रु॒न्धे ऽव॑ प॒शून् प॒शू नव॑ रुन्धे । \newline
25. अव॑ रुन्धे रु॒न्धे ऽवाव॑ रुन्धे॒ कल्पा॒न् कल्पा᳚न् रु॒न्धे ऽवाव॑ रुन्धे॒ कल्पान्॑ । \newline
26. रु॒न्धे॒ कल्पा॒न् कल्पा᳚न् रुन्धे रुन्धे॒ कल्पा᳚न् जुहोति जुहोति॒ कल्पा᳚न् रुन्धे रुन्धे॒ कल्पा᳚न् जुहोति । \newline
27. कल्पा᳚न् जुहोति जुहोति॒ कल्पा॒न् कल्पा᳚न् जुहो॒ त्यक्लृ॑प्त॒स्या क्लृ॑प्तस्य जुहोति॒ कल्पा॒न् कल्पा᳚न् जुहो॒ त्यक्लृ॑प्तस्य । \newline
28. जु॒हो॒त्य क्लृ॑प्त॒स्या क्लृ॑प्तस्य जुहोति जुहो॒ त्यक्लृ॑प्तस्य॒ क्लृप्त्यै॒ क्लृप्त्या॒ अक्लृ॑प्तस्य जुहोति जुहो॒
त्यक्लृ॑प्तस्य॒ क्लृप्त्यै᳚ । \newline
29. अक्लृ॑प्तस्य॒ क्लृप्त्यै॒ क्लृप्त्या॒ अक्लृ॑प्त॒स्या क्लृ॑प्तस्य॒ क्लृप्त्यै॑ युग्मदयु॒जे यु॑ग्मदयु॒जे क्लृप्त्या॒ अक्लृ॑प्त॒स्या क्लृ॑प्तस्य॒ क्लृप्त्यै॑ युग्मदयु॒जे । \newline
30. क्लृप्त्यै॑ युग्मदयु॒जे यु॑ग्मदयु॒जे क्लृप्त्यै॒ क्लृप्त्यै॑ युग्मदयु॒जे जु॑होति जुहोति युग्मदयु॒जे क्लृप्त्यै॒ क्लृप्त्यै॑ युग्मदयु॒जे जु॑होति । \newline
31. यु॒ग्म॒द॒यु॒जे जु॑होति जुहोति युग्मदयु॒जे यु॑ग्मदयु॒जे जु॑होति मिथुन॒त्वाय॑ मिथुन॒त्वाय॑ जुहोति युग्मदयु॒जे यु॑ग्मदयु॒जे जु॑होति मिथुन॒त्वाय॑ । \newline
32. यु॒ग्म॒द॒यु॒जे इति॑ युग्मत् - अ॒यु॒जे । \newline
33. जु॒हो॒ति॒ मि॒थु॒न॒त्वाय॑ मिथुन॒त्वाय॑ जुहोति जुहोति मिथुन॒त्वा यो᳚त्त॒राव॑ती उत्त॒राव॑ती मिथुन॒त्वाय॑ जुहोति जुहोति मिथुन॒त्वा यो᳚त्त॒राव॑ती । \newline
34. मि॒थु॒न॒त्वा यो᳚त्त॒राव॑ती उत्त॒राव॑ती मिथुन॒त्वाय॑ मिथुन॒त्वा यो᳚त्त॒राव॑ती भवतो भवत उत्त॒राव॑ती मिथुन॒त्वाय॑ मिथुन॒त्वा यो᳚त्त॒राव॑ती भवतः । \newline
35. मि॒थु॒न॒त्वायेति॑ मिथुन - त्वाय॑ । \newline
36. उ॒त्त॒राव॑ती भवतो भवत उत्त॒राव॑ती उत्त॒राव॑ती भवतो॒ ऽभिक्रा᳚न्त्या अ॒भिक्रा᳚न्त्यै भवत उत्त॒राव॑ती उत्त॒राव॑ती भवतो॒ ऽभिक्रा᳚न्त्यै । \newline
37. उ॒त्त॒राव॑ती॒ इत्यु॑त्त॒रा - व॒ती॒ । \newline
38. भ॒व॒तो॒ ऽभिक्रा᳚न्त्या अ॒भिक्रा᳚न्त्यै भवतो भवतो॒ ऽभिक्रा᳚न्त्या॒ एकैका॒ ऽभिक्रा᳚न्त्यै भवतो भवतो॒ ऽभिक्रा᳚न्त्या॒ एका᳚ । \newline
39. अ॒भिक्रा᳚न्त्या॒ एकैका॒ ऽभिक्रा᳚न्त्या अ॒भिक्रा᳚न्त्या॒ एका॑ च॒ चैका॒ ऽभिक्रा᳚न्त्या अ॒भिक्रा᳚न्त्या॒ एका॑ च । \newline
40. अ॒भिक्रा᳚न्त्या॒ इत्य॒भि - क्रा॒न्त्यै॒ । \newline
41. एका॑ च॒ चैकैका॑ च मे मे॒ चैकैका॑ च मे । \newline
42. च॒ मे॒ मे॒ च॒ च॒ मे॒ ति॒स्र स्ति॒स्रो मे॑ च च मे ति॒स्रः । \newline
43. मे॒ ति॒स्र स्ति॒स्रो मे॑ मे ति॒स्रश्च॑ च ति॒स्रो मे॑ मे ति॒स्रश्च॑ । \newline
44. ति॒स्रश्च॑ च ति॒स्र स्ति॒स्र श्च॑ मे मे च ति॒स्र स्ति॒स्र श्च॑ मे । \newline
45. च॒ मे॒ मे॒ च॒ च॒ म॒ इतीति॑ मे च च म॒ इति॑ । \newline
46. म॒ इतीति॑ मे म॒ इत्या॑ हा॒हेति॑ मे म॒ इत्या॑ह । \newline
47. इत्या॑हा॒हे तीत्या॑ह देवछन्द॒सम् दे॑वछन्द॒स मा॒हे तीत्या॑ह देवछन्द॒सम् । \newline
48. आ॒ह॒ दे॒व॒छ॒न्द॒सम् दे॑वछन्द॒स मा॑हाह देवछन्द॒सं ॅवै वै दे॑वछन्द॒स मा॑हाह देवछन्द॒सं ॅवै । \newline
49. दे॒व॒छ॒न्द॒सं ॅवै वै दे॑वछन्द॒सम् दे॑वछन्द॒सं ॅवा एकैका॒ वै दे॑वछन्द॒सम् दे॑वछन्द॒सं ॅवा एका᳚ । \newline
50. दे॒व॒छ॒न्द॒समिति॑ देव - छ॒न्द॒सम् । \newline
51. वा एकैका॒ वै वा एका॑ च॒ चैका॒ वै वा एका॑ च । \newline
52. एका॑ च॒ चैकैका॑ च ति॒स्र स्ति॒स्र श्चैकैका॑ च ति॒स्रः । \newline
53. च॒ ति॒स्र स्ति॒स्र श्च॑ च ति॒स्रश्च॑ च ति॒स्रश्च॑ च ति॒स्रश्च॑ । \newline
54. ति॒स्र श्च॑ च ति॒स्र स्ति॒स्र श्च॑ मनुष्यछन्द॒सम् म॑नुष्यछन्द॒सम् च॑ ति॒स्र स्ति॒स्र श्च॑ मनुष्यछन्द॒सम् । \newline
55. च॒ म॒नु॒ष्य॒छ॒न्द॒सम् म॑नुष्यछन्द॒सम् च॑ च मनुष्यछन्द॒सम् चत॑स्र॒ श्चत॑स्रो मनुष्यछन्द॒सम् च॑ च मनुष्यछन्द॒सम् चत॑स्रः । \newline
\pagebreak
\markright{ TS 5.4.8.6  \hfill https://www.vedavms.in \hfill}

\section{ TS 5.4.8.6 }

\textbf{TS 5.4.8.6 } \newline
\textbf{Samhita Paata} \newline

मनुष्यछन्द॒सं चत॑स्रश्चा॒ष्टौ च॑ देवछन्द॒सं चै॒व म॑नुष्य छन्द॒सश्चाव॑ रुन्ध॒ आ त्रय॑स्त्रिꣳ शतो जुहोति॒ त्रय॑स्त्रिꣳश॒द्वै दे॒वता॑ दे॒वता॑ ए॒वाव॑ रुन्ध॒ आऽष्टाच॑त्वारिꣳशतो जुहोत्य॒ष्टाच॑त्वारिꣳ-शदक्षरा॒ जग॑ती॒ जाग॑ताः प॒शवो॒ जग॑त्यै॒वास्मै॑ प॒शूनव॑ रुन्धे॒ वाज॑श्च प्रस॒वश्चेति॑ द्वाद॒शं जु॑होति॒ द्वाद॑श॒ मासाः᳚ संॅवथ्स॒रः सं॑ॅवथ्स॒र ए॒व प्रति॑ तिष्ठति ॥ \newline

\textbf{Pada Paata} \newline

म॒नु॒ष्य॒छ॒न्द॒समिति॑ मनुष्य - छ॒न्द॒सम् । चत॑स्रः । च॒ । अ॒ष्टौ । च॒ । दे॒व॒छ॒न्द॒समिति॑ देव - छ॒न्द॒सम् । च॒ । ए॒व । म॒नु॒ष्य॒छ॒न्द॒समिति॑ मनुष्य - छ॒न्द॒सम् । च॒ । अवेति॑ । रु॒न्धे॒ । एति॑ । त्रय॑स्त्रिꣳशत॒ इति॒ त्रयः॑ - त्रिꣳ॒॒श॒तः॒ । जु॒हो॒ति॒ । त्रय॑स्त्रिꣳश॒दिति॒ त्रयः॑ - त्रिꣳ॒॒श॒त् । वै । दे॒वताः᳚ । दे॒वताः᳚ । ए॒व । अवेति॑ । रु॒न्धे॒ । एति॑ । अ॒ष्टाच॑त्वारिꣳशत॒ इत्य॒ष्टा - च॒त्वा॒रिꣳ॒॒श॒तः॒ । जु॒हो॒ति॒ । अ॒ष्टाच॑त्वारिꣳशदक्ष॒रेत्य॒ष्टाच॑त्वारिꣳशत्-अ॒क्ष॒रा॒ । जग॑ती । जाग॑ताः । प॒शवः॑ । जग॑त्या । ए॒व । अ॒स्मै॒ । प॒शून् । अवेति॑ । रु॒न्धे॒ । वाजः॑ । च॒ । प्र॒स॒व इति॑ प्र-स॒वः । च॒ । इति॑ । द्वा॒द॒शम् । जु॒हो॒ति॒ । द्वाद॑श । मासाः᳚ । सं॒ॅव॒थ्स॒र इति॑ सं - व॒थ्स॒रः । सं॒ॅव॒थ्स॒र इति॑ सं - व॒थ्स॒रे । ए॒व । प्रतीति॑ । ति॒ष्ठ॒ति॒ ॥  \newline


\textbf{Krama Paata} \newline

म॒नु॒ष्य॒छ॒न्द॒सम् चत॑स्रः । म॒नु॒ष्य॒छ॒न्द॒समिति॑ मनुष्य - छ॒न्द॒सम् । चत॑स्रश्च । चा॒ष्टौ । अ॒ष्टौ च॑ । च॒ दे॒व॒छ॒न्द॒सम् । दे॒व॒छ॒न्द॒सम् च॑ । दे॒व॒छ॒न्द॒समिति॑ देव - छ॒न्द॒सम् । चै॒व । ए॒व म॑नुष्यछन्द॒सम् । म॒नु॒ष्य॒छ॒न्द॒सम् च॑ । म॒नु॒ष्य॒छ॒न्द॒समिति॑ मनुष्य - छ॒न्द॒सम् । चाव॑ । अव॑ रुन्धे । रु॒न्ध॒ आ । आ त्रय॑स्त्रिꣳशतः । त्रय॑स्त्रिꣳशतो जुहोति । त्रय॑स्त्रिꣳशत॒ इति॒ त्रयः॑ - त्रिꣳ॒॒श॒तः॒ । जु॒हो॒ति॒ त्रय॑स्त्रिꣳशत् । त्रय॑स्त्रिꣳश॒द् वै । त्रय॑स्त्रिꣳश॒दिति॒ त्रयः॑ - त्रिꣳ॒॒श॒त्॒ । वै दे॒वताः᳚ । दे॒वता॑ दे॒वताः᳚ । दे॒वता॑ ए॒व । ए॒वाव॑ । अव॑ रुन्धे । रु॒न्ध॒ आ । आऽष्टाच॑त्वारिꣳशतः । अ॒ष्टाच॑त्वारिꣳशतो जुहोति । अ॒ष्टाच॑त्वारिꣳशत॒ इत्य॒ष्टा - च॒त्वा॒रिꣳ॒॒श॒तः॒ । जु॒हो॒त्य॒ष्टाच॑त्वारिꣳशदक्षरा । अ॒ष्टाच॑त्वारिꣳशदक्षरा॒ जग॑ती । अ॒ष्टाच॑त्वारिꣳशदक्ष॒रेत्य॒ष्टा,च॑त्वारिꣳशत् - अ॒क्ष॒रा॒ । जग॑ती॒ जाग॑ताः । जाग॑ताः प॒शवः॑ । प॒शवो॒ जग॑त्या । जग॑त्यै॒व । ए॒वास्मै᳚ । अ॒स्मै॒ प॒शून् । प॒शूनव॑ । अव॑ रुन्धे । रु॒न्धे॒ वाजः॑ । वाज॑श्च । च॒ प्र॒स॒वः । प्र॒स॒वश्च॑ । प्र॒स॒व इति॑ प्र - स॒वः । चेति॑ । इति॑ द्वाद॒शम् । द्वा॒द॒शम् जु॑होति । जु॒हो॒ति॒ द्वाद॑श । द्वाद॑श॒ मासाः᳚ । मासाः᳚ सम्ॅवथ्स॒रः । स॒म्ॅव॒थ्स॒रः स॑म्ॅवथ्स॒रे । स॒म्ॅव॒थ्स॒र इति॑ सम् - व॒थ्स॒रः । स॒म्ॅव॒थ्स॒र ए॒व । स॒म्ॅव॒थ्स॒र इति॑ सम् - व॒थ्स॒रे । ए॒व प्रति॑ । प्रति॑ तिष्ठति । ति॒ष्ठ॒तीति॑ तिष्ठति । \newline

\textbf{Jatai Paata} \newline

1. म॒नु॒ष्य॒छ॒न्द॒सम् चत॑स्र॒ श्चत॑स्रो मनुष्यछन्द॒सम् म॑नुष्यछन्द॒सम् चत॑स्रः । \newline
2. म॒नु॒ष्य॒छ॒न्द॒समिति॑ मनुष्य - छ॒न्द॒सम् । \newline
3. चत॑स्र श्च च॒ चत॑स्र॒ श्चत॑स्र श्च । \newline
4. चा॒ष्टा व॒ष्टौ च॑ चा॒ष्टौ । \newline
5. अ॒ष्टौ च॑ चा॒ष्टा व॒ष्टौ च॑ । \newline
6. च॒ दे॒व॒छ॒न्द॒सम् दे॑वछन्द॒सम् च॑ च देवछन्द॒सम् । \newline
7. दे॒व॒छ॒न्द॒सम् च॑ च देवछन्द॒सम् दे॑वछन्द॒सम् च॑ । \newline
8. दे॒व॒छ॒न्द॒समिति॑ देव - छ॒न्द॒सम् । \newline
9. चै॒वैव च॑ चै॒व । \newline
10. ए॒व म॑नुष्यछन्द॒सम् म॑नुष्यछन्द॒स मे॒वैव म॑नुष्यछन्द॒सम् । \newline
11. म॒नु॒ष्य॒छ॒न्द॒सम् च॑ च मनुष्यछन्द॒सम् म॑नुष्यछन्द॒सम् च॑ । \newline
12. म॒नु॒ष्य॒छ॒न्द॒समिति॑ मनुष्य - छ॒न्द॒सम् । \newline
13. चावाव॑ च॒ चाव॑ । \newline
14. अव॑ रुन्धे रु॒न्धे ऽवाव॑ रुन्धे । \newline
15. रु॒न्ध॒ आ रु॑न्धे रुन्ध॒ आ । \newline
16. आ त्रय॑स्त्रिꣳशत॒ स्त्रय॑स्त्रिꣳशत॒ आ त्रय॑स्त्रिꣳशतः । \newline
17. त्रय॑स्त्रिꣳशतो जुहोति जुहोति॒ त्रय॑स्त्रिꣳशत॒ स्त्रय॑स्त्रिꣳशतो जुहोति । \newline
18. त्रय॑स्त्रिꣳशत॒ इति॒ त्रयः॑ - त्रिꣳ॒॒श॒तः॒ । \newline
19. जु॒हो॒ति॒ त्रय॑स्त्रिꣳश॒त् त्रय॑स्त्रिꣳशज् जुहोति जुहोति॒ त्रय॑स्त्रिꣳशत् । \newline
20. त्रय॑स्त्रिꣳश॒द् वै वै त्रय॑स्त्रिꣳश॒त् त्रय॑स्त्रिꣳश॒द् वै । \newline
21. त्रय॑स्त्रिꣳश॒दिति॒ त्रयः॑ - त्रिꣳ॒॒श॒त् । \newline
22. वै दे॒वता॑ दे॒वता॒ वै वै दे॒वताः᳚ । \newline
23. दे॒वता॑ दे॒वताः᳚ । \newline
24. दे॒वता॑ ए॒वैव दे॒वता॑ दे॒वता॑ ए॒व । \newline
25. ए॒वावा वै॒वै वाव॑ । \newline
26. अव॑ रुन्धे रु॒न्धे ऽवाव॑ रुन्धे । \newline
27. रु॒न्ध॒ आ रु॑न्धे रुन्ध॒ आ । \newline
28. आ ऽष्टाच॑त्वारिꣳशतो॒ ऽष्टाच॑त्वारिꣳशत॒ आ ऽष्टाच॑त्वारिꣳशतः । \newline
29. अ॒ष्टाच॑त्वारिꣳशतो जुहोति जुहो त्य॒ष्टाच॑त्वारिꣳशतो॒ ऽष्टाच॑त्वारिꣳशतो जुहोति । \newline
30. अ॒ष्टाच॑त्वारिꣳशत॒ इत्य॒ष्टा - च॒त्वा॒रिꣳ॒॒श॒तः॒ । \newline
31. जु॒हो॒ त्य॒ष्टाच॑त्वारिꣳशदक्षरा॒ ऽष्टाच॑त्वारिꣳशदक्षरा जुहोति जुहो त्य॒ष्टाच॑त्वारिꣳशदक्षरा । \newline
32. अ॒ष्टाच॑त्वारिꣳशदक्षरा॒ जग॑ती॒ जग॑त्य॒ ष्टाच॑त्वारिꣳशदक्षरा॒ ऽष्टाच॑त्वारिꣳशदक्षरा॒ जग॑ती । \newline
33. अ॒ष्टाच॑त्वारिꣳशदक्ष॒रेत्य॒ष्टाच॑त्वारिꣳशत् - अ॒क्ष॒रा॒ । \newline
34. जग॑ती॒ जाग॑ता॒ जाग॑ता॒ जग॑ती॒ जग॑ती॒ जाग॑ताः । \newline
35. जाग॑ताः प॒शवः॑ प॒शवो॒ जाग॑ता॒ जाग॑ताः प॒शवः॑ । \newline
36. प॒शवो॒ जग॑त्या॒ जग॑त्या प॒शवः॑ प॒शवो॒ जग॑त्या । \newline
37. जग॑त्यै॒वैव जग॑त्या॒ जग॑त्यै॒व । \newline
38. ए॒वास्मा॑ अस्मा ए॒वै वास्मै᳚ । \newline
39. अ॒स्मै॒ प॒शून् प॒शू न॑स्मा अस्मै प॒शून् । \newline
40. प॒शू नवाव॑ प॒शून् प॒शू नव॑ । \newline
41. अव॑ रुन्धे रु॒न्धे ऽवाव॑ रुन्धे । \newline
42. रु॒न्धे॒ वाजो॒ वाजो॑ रुन्धे रुन्धे॒ वाजः॑ । \newline
43. वाज॑श्च च॒ वाजो॒ वाज॑श्च । \newline
44. च॒ प्र॒स॒वः प्र॑स॒वश्च॑ च प्रस॒वः । \newline
45. प्र॒स॒वश्च॑ च प्रस॒वः प्र॑स॒वश्च॑ । \newline
46. प्र॒स॒व इति॑ प्र - स॒वः । \newline
47. चेतीति॑ च॒ चेति॑ । \newline
48. इति॑ द्वाद॒शम् द्वा॑द॒श मितीति॑ द्वाद॒शम् । \newline
49. द्वा॒द॒शम् जु॑होति जुहोति द्वाद॒शम् द्वा॑द॒शम् जु॑होति । \newline
50. जु॒हो॒ति॒ द्वाद॑श॒ द्वाद॑श जुहोति जुहोति॒ द्वाद॑श । \newline
51. द्वाद॑श॒ मासा॒ मासा॒ द्वाद॑श॒ द्वाद॑श॒ मासाः᳚ । \newline
52. मासाः᳚ संॅवथ्स॒रः सं॑ॅवथ्स॒रो मासा॒ मासाः᳚ संॅवथ्स॒रः । \newline
53. सं॒ॅव॒थ्स॒रः सं॑ॅवथ्स॒रे सं॑ॅवथ्स॒रे सं॑ॅवथ्स॒रः सं॑ॅवथ्स॒रः सं॑ॅवथ्स॒रे । \newline
54. सं॒ॅव॒थ्स॒र इति॑ सं - व॒थ्स॒रः । \newline
55. सं॒ॅव॒थ्स॒र ए॒वैव सं॑ॅवथ्स॒रे सं॑ॅवथ्स॒र ए॒व । \newline
56. सं॒ॅव॒थ्स॒र इति॑ सं - व॒थ्स॒रे । \newline
57. ए॒व प्रति॒ प्रत्ये॒वैव प्रति॑ । \newline
58. प्रति॑ तिष्ठति तिष्ठति॒ प्रति॒ प्रति॑ तिष्ठति । \newline
59. ति॒ष्ठ॒तीति॑ तिष्ठति । \newline

\textbf{Ghana Paata } \newline

1. म॒नु॒ष्य॒छ॒न्द॒सम् चत॑स्र॒ श्चत॑स्रो मनुष्यछन्द॒सम् म॑नुष्यछन्द॒सम् चत॑स्रश्च च॒ चत॑स्रो मनुष्यछन्द॒सम् म॑नुष्यछन्द॒सम् चत॑स्रश्च । \newline
2. म॒नु॒ष्य॒छ॒न्द॒समिति॑ मनुष्य - छ॒न्द॒सम् । \newline
3. चत॑स्रश्च च॒ चत॑स्र॒ श्चत॑स्र श्चा॒ष्टा व॒ष्टौ च॒ चत॑स्र॒ श्चत॑स्र श्चा॒ष्टौ । \newline
4. चा॒ष्टा व॒ष्टौ च॑ चा॒ष्टौ च॑ चा॒ष्टौ च॑ चा॒ष्टौ च॑ । \newline
5. अ॒ष्टौ च॑ चा॒ष्टा व॒ष्टौ च॑ देवछन्द॒सम् दे॑वछन्द॒सम् चा॒ष्टा व॒ष्टौ च॑ देवछन्द॒सम् । \newline
6. च॒ दे॒व॒छ॒न्द॒सम् दे॑वछन्द॒सम् च॑ च देवछन्द॒सम् च॑ च देवछन्द॒सम् च॑ च देवछन्द॒सम् च॑ । \newline
7. दे॒व॒छ॒न्द॒सम् च॑ च देवछन्द॒सम् दे॑वछन्द॒सम् चै॒वैव च॑ देवछन्द॒सम् दे॑वछन्द॒सम् चै॒व । \newline
8. दे॒व॒छ॒न्द॒समिति॑ देव - छ॒न्द॒सम् । \newline
9. चै॒वैव च॑ चै॒व म॑नुष्यछन्द॒सम् म॑नुष्यछन्द॒स मे॒व च॑ चै॒व म॑नुष्यछन्द॒सम् । \newline
10. ए॒व म॑नुष्यछन्द॒सम् म॑नुष्यछन्द॒स मे॒वैव म॑नुष्यछन्द॒सम् च॑ च मनुष्यछन्द॒स मे॒वैव म॑नुष्यछन्द॒सम् च॑ । \newline
11. म॒नु॒ष्य॒छ॒न्द॒सम् च॑ च मनुष्यछन्द॒सम् म॑नुष्यछन्द॒सम् चावाव॑ च मनुष्यछन्द॒सम् म॑नुष्यछन्द॒सम् चाव॑ । \newline
12. म॒नु॒ष्य॒छ॒न्द॒समिति॑ मनुष्य - छ॒न्द॒सम् । \newline
13. चावाव॑ च॒ चाव॑ रुन्धे रु॒न्धे ऽव॑ च॒ चाव॑ रुन्धे । \newline
14. अव॑ रुन्धे रु॒न्धे ऽवाव॑ रुन्ध॒ आ रु॒न्धे ऽवाव॑ रुन्ध॒ आ । \newline
15. रु॒न्ध॒ आ रु॑न्धे रुन्ध॒ आ त्रय॑स्त्रिꣳशत॒ स्त्रय॑स्त्रिꣳशत॒ आ रु॑न्धे रुन्ध॒ आ त्रय॑स्त्रिꣳशतः । \newline
16. आ त्रय॑स्त्रिꣳशत॒ स्त्रय॑स्त्रिꣳशत॒ आ त्रय॑स्त्रिꣳशतो जुहोति जुहोति॒ त्रय॑स्त्रिꣳशत॒ आ त्रय॑स्त्रिꣳशतो जुहोति । \newline
17. त्रय॑स्त्रिꣳशतो जुहोति जुहोति॒ त्रय॑स्त्रिꣳशत॒ स्त्रय॑स्त्रिꣳशतो जुहोति॒ त्रय॑स्त्रिꣳश॒त् त्रय॑स्त्रिꣳशज् जुहोति॒ त्रय॑स्त्रिꣳशत॒ स्त्रय॑स्त्रिꣳशतो जुहोति॒ त्रय॑स्त्रिꣳशत् । \newline
18. त्रय॑स्त्रिꣳशत॒ इति॒ त्रयः॑ - त्रिꣳ॒॒श॒तः॒ । \newline
19. जु॒हो॒ति॒ त्रय॑स्त्रिꣳश॒त् त्रय॑स्त्रिꣳशज् जुहोति जुहोति॒ त्रय॑स्त्रिꣳश॒द् वै वै त्रय॑स्त्रिꣳशज् जुहोति जुहोति॒ त्रय॑स्त्रिꣳश॒द् वै । \newline
20. त्रय॑स्त्रिꣳश॒द् वै वै त्रय॑स्त्रिꣳश॒त् त्रय॑स्त्रिꣳश॒द् वै दे॒वता॑ दे॒वता॒ वै त्रय॑स्त्रिꣳश॒त् त्रय॑स्त्रिꣳश॒द् वै दे॒वताः᳚ । \newline
21. त्रय॑स्त्रिꣳश॒दिति॒ त्रयः॑ - त्रिꣳ॒॒श॒त् । \newline
22. वै दे॒वता॑ दे॒वता॒ वै वै दे॒वताः᳚ । \newline
23. दे॒वता॑ दे॒वताः᳚ । \newline
24. दे॒वता॑ ए॒वैव दे॒वता॑ दे॒वता॑ ए॒वावा वै॒व दे॒वता॑ दे॒वता॑ ए॒वाव॑ । \newline
25. ए॒वावा वै॒वै वाव॑ रुन्धे रु॒न्धे ऽवै॒वै वाव॑ रुन्धे । \newline
26. अव॑ रुन्धे रु॒न्धे ऽवाव॑ रुन्ध॒ आ रु॒न्धे ऽवाव॑ रुन्ध॒ आ । \newline
27. रु॒न्ध॒ आ रु॑न्धे रुन्ध॒ आ ऽष्टाच॑त्वारिꣳशतो॒ ऽष्टाच॑त्वारिꣳशत॒ आ रु॑न्धे रुन्ध॒ आ ऽष्टाच॑त्वारिꣳशतः । \newline
28. आ ऽष्टाच॑त्वारिꣳशतो॒ ऽष्टाच॑त्वारिꣳशत॒ आ ऽष्टाच॑त्वारिꣳशतो जुहोति जुहो त्य॒ष्टाच॑त्वारिꣳशत॒ आ ऽष्टाच॑त्वारिꣳशतो जुहोति । \newline
29. अ॒ष्टाच॑त्वारिꣳशतो जुहोति जुहो त्य॒ष्टाच॑त्वारिꣳशतो॒ ऽष्टाच॑त्वारिꣳशतो जुहो त्य॒ष्टाच॑त्वारिꣳशदक्षरा॒ ऽष्टाच॑त्वारिꣳशदक्षरा जुहो त्य॒ष्टाच॑त्वारिꣳशतो॒ ऽष्टाच॑त्वारिꣳशतो जुहो
त्य॒ष्टाच॑त्वारिꣳशदक्षरा । \newline
30. अ॒ष्टाच॑त्वारिꣳशत॒ इत्य॒ष्टा - च॒त्वा॒रिꣳ॒॒श॒तः॒ । \newline
31. जु॒हो॒ त्य॒ष्टाच॑त्वारिꣳशदक्षरा॒ ऽष्टाच॑त्वारिꣳशदक्षरा जुहोति जुहो त्य॒ष्टाच॑त्वारिꣳशदक्षरा॒ जग॑ती॒ जग॑ त्य॒ष्टाच॑त्वारिꣳशदक्षरा जुहोति जुहो त्य॒ष्टाच॑त्वारिꣳशदक्षरा॒ जग॑ती । \newline
32. अ॒ष्टाच॑त्वारिꣳशदक्षरा॒ जग॑ती॒ जग॑ त्य॒ष्टाच॑त्वारिꣳशदक्षरा॒ ऽष्टाच॑त्वारिꣳशदक्षरा॒ जग॑ती॒ जाग॑ता॒ जाग॑ता॒ जग॑ त्य॒ष्टाच॑त्वारिꣳशदक्षरा॒ ऽष्टाच॑त्वारिꣳशदक्षरा॒ जग॑ती॒ जाग॑ताः । \newline
33. अ॒ष्टाच॑त्वारिꣳशदक्ष॒रेत्य॒ष्टाच॑त्वारिꣳशत् - अ॒क्ष॒रा॒ । \newline
34. जग॑ती॒ जाग॑ता॒ जाग॑ता॒ जग॑ती॒ जग॑ती॒ जाग॑ताः प॒शवः॑ प॒शवो॒ जाग॑ता॒ जग॑ती॒ जग॑ती॒ जाग॑ताः प॒शवः॑ । \newline
35. जाग॑ताः प॒शवः॑ प॒शवो॒ जाग॑ता॒ जाग॑ताः प॒शवो॒ जग॑त्या॒ जग॑त्या प॒शवो॒ जाग॑ता॒ जाग॑ताः प॒शवो॒ जग॑त्या । \newline
36. प॒शवो॒ जग॑त्या॒ जग॑त्या प॒शवः॑ प॒शवो॒ जग॑त्यै॒ वैव जग॑त्या प॒शवः॑ प॒शवो॒ जग॑त्यै॒व । \newline
37. जग॑त्यै॒ वैव जग॑त्या॒ जग॑त्यै॒ वास्मा॑ अस्मा ए॒व जग॑त्या॒ जग॑त्यै॒ वास्मै᳚ । \newline
38. ए॒वास्मा॑ अस्मा ए॒वै वास्मै॑ प॒शून् प॒शू न॑स्मा ए॒वै वास्मै॑ प॒शून् । \newline
39. अ॒स्मै॒ प॒शून् प॒शू न॑स्मा अस्मै प॒शू नवाव॑ प॒शू न॑स्मा अस्मै प॒शू नव॑ । \newline
40. प॒शू नवाव॑ प॒शून् प॒शू नव॑ रुन्धे रु॒न्धे ऽव॑ प॒शून् प॒शू नव॑ रुन्धे । \newline
41. अव॑ रुन्धे रु॒न्धे ऽवाव॑ रुन्धे॒ वाजो॒ वाजो॑ रु॒न्धे ऽवाव॑ रुन्धे॒ वाजः॑ । \newline
42. रु॒न्धे॒ वाजो॒ वाजो॑ रुन्धे रुन्धे॒ वाज॑श्च च॒ वाजो॑ रुन्धे रुन्धे॒ वाज॑श्च । \newline
43. वाज॑श्च च॒ वाजो॒ वाज॑श्च प्रस॒वः प्र॑स॒वश्च॒ वाजो॒ वाज॑श्च प्रस॒वः । \newline
44. च॒ प्र॒स॒वः प्र॑स॒वश्च॑ च प्रस॒वश्च॑ च प्रस॒वश्च॑ च प्रस॒वश्च॑ । \newline
45. प्र॒स॒वश्च॑ च प्रस॒वः प्र॑स॒वश्चे तीति॑ च प्रस॒वः प्र॑स॒वश्चेति॑ । \newline
46. प्र॒स॒व इति॑ प्र - स॒वः । \newline
47. चे तीति॑ च॒ चेति॑ द्वाद॒शम् द्वा॑द॒श मिति॑ च॒ चेति॑ द्वाद॒शम् । \newline
48. इति॑ द्वाद॒शम् द्वा॑द॒श मितीति॑ द्वाद॒शम् जु॑होति जुहोति द्वाद॒श मितीति॑ द्वाद॒शम् जु॑होति । \newline
49. द्वा॒द॒शम् जु॑होति जुहोति द्वाद॒शम् द्वा॑द॒शम् जु॑होति॒ द्वाद॑श॒ द्वाद॑श जुहोति द्वाद॒शम् द्वा॑द॒शम् जु॑होति॒ द्वाद॑श । \newline
50. जु॒हो॒ति॒ द्वाद॑श॒ द्वाद॑श जुहोति जुहोति॒ द्वाद॑श॒ मासा॒ मासा॒ द्वाद॑श जुहोति जुहोति॒ द्वाद॑श॒ मासाः᳚ । \newline
51. द्वाद॑श॒ मासा॒ मासा॒ द्वाद॑श॒ द्वाद॑श॒ मासाः᳚ संॅवथ्स॒रः सं॑ॅवथ्स॒रो मासा॒ द्वाद॑श॒ द्वाद॑श॒ मासाः᳚ संॅवथ्स॒रः । \newline
52. मासाः᳚ संॅवथ्स॒रः सं॑ॅवथ्स॒रो मासा॒ मासाः᳚ संॅवथ्स॒रः सं॑ॅवथ्स॒रे सं॑ॅवथ्स॒रे सं॑ॅवथ्स॒रो मासा॒ मासाः᳚ संॅवथ्स॒रः सं॑ॅवथ्स॒रे । \newline
53. सं॒ॅव॒थ्स॒रः सं॑ॅवथ्स॒रे सं॑ॅवथ्स॒रे सं॑ॅवथ्स॒रः सं॑ॅवथ्स॒रः सं॑ॅवथ्स॒र ए॒वैव सं॑ॅवथ्स॒रे सं॑ॅवथ्स॒रः सं॑ॅवथ्स॒रः सं॑ॅवथ्स॒र ए॒व । \newline
54. सं॒ॅव॒थ्स॒र इति॑ सं - व॒थ्स॒रः । \newline
55. सं॒ॅव॒थ्स॒र ए॒वैव सं॑ॅवथ्स॒रे सं॑ॅवथ्स॒र ए॒व प्रति॒ प्रत्ये॒व सं॑ॅवथ्स॒रे सं॑ॅवथ्स॒र ए॒व प्रति॑ । \newline
56. सं॒ॅव॒थ्स॒र इति॑ सं - व॒थ्स॒रे । \newline
57. ए॒व प्रति॒ प्रत्ये॒वैव प्रति॑ तिष्ठति तिष्ठति॒ प्रत्ये॒ वैव प्रति॑ तिष्ठति । \newline
58. प्रति॑ तिष्ठति तिष्ठति॒ प्रति॒ प्रति॑ तिष्ठति । \newline
59. ति॒ष्ठ॒तीति॑ तिष्ठति । \newline
\pagebreak
\markright{ TS 5.4.9.1  \hfill https://www.vedavms.in \hfill}

\section{ TS 5.4.9.1 }

\textbf{TS 5.4.9.1 } \newline
\textbf{Samhita Paata} \newline

अ॒ग्निर्दे॒वेभ्यो ऽपा᳚क्रामद्-भाग॒धेय॑मि॒च्छमा॑न॒स्तं दे॒वा अ॑ब्रुव॒न्नुप॑ न॒ आ व॑र्तस्व ह॒व्यं नो॑ व॒हेति॒ सो᳚ऽब्रवी॒द्-वरं॑ ॅवृणै॒ मह्य॑मे॒व वा॑जप्रस॒वीयं॑ जुहव॒न्निति॒ तस्मा॑द॒ग्नये॑ वाजप्रस॒वीयं॑ जुह्वति॒ यद्-वा॑जप्रस॒वीयं॑ जु॒होत्य॒ग्निमे॒व तद्-भा॑ग॒धेये॑न॒ सम॑र्द्धय॒त्यथो॑ अभिषे॒क ए॒वास्य॒ स च॑तुर्द॒शभि॑र्जुहोति स॒प्त ग्रा॒म्या ओष॑धयः स॒प्ता - [  ] \newline

\textbf{Pada Paata} \newline

अ॒ग्निः । दे॒वेभ्यः॑ । अपेति॑ । अ॒क्रा॒म॒त् । भा॒ग॒धेय॒मिति॑ भाग-धेय᳚म् । इ॒च्छमा॑नः । तम् । दे॒वाः । अ॒ब्रु॒व॒न्न् । उपेति॑ । नः॒ । एति॑ । व॒र्त॒स्व॒ । ह॒व्यम् । नः॒ । व॒ह॒ । इति॑ । सः । अ॒ब्र॒वी॒त् । वर᳚म् । वृ॒णै॒ । मह्य᳚म् । ए॒व । वा॒ज॒प्र॒स॒वीय॒मिति॑ वाज - प्र॒स॒वीय᳚म् । जु॒ह॒व॒न्न् । इति॑ । तस्मा᳚त् । अ॒ग्नये᳚ । वा॒ज॒प्र॒स॒वीय॒मिति॑ वाज - प्र॒स॒वीय᳚म् । जु॒ह॒ति॒ । यत् । वा॒ज॒प्र॒स॒वीय॒मिति॑ वाज - प्र॒स॒वीय᳚म् । जु॒होति॑ । अ॒ग्निम् । ए॒व । तत् । भा॒ग॒धेये॒नेति॑ भाग - धेये॑न । समिति॑ । अ॒द्‌र्ध॒य॒ति॒ । अथो॒ इति॑ । अ॒भि॒षे॒क इत्य॑भि - से॒कः । ए॒व । अ॒स्य॒ । सः । च॒तु॒र्द॒शभि॒रिति॑ चतुर्द॒श - भिः॒ । जु॒हो॒ति॒ । स॒प्त । ग्रा॒म्याः । ओष॑धयः । स॒प्त ।  \newline


\textbf{Krama Paata} \newline

अ॒ग्निर् दे॒वेभ्यः॑ । दे॒वेभ्योऽप॑ । अपा᳚क्रामत् । अ॒क्रा॒म॒द् भा॒ग॒धेय᳚म् । भा॒ग॒धेय॑मि॒च्छमा॑नः । भा॒ग॒धेय॒मिति॑ भाग - धेय᳚म् । इ॒च्छमा॑न॒स्तम् । तम् दे॒वाः । दे॒वा अ॑ब्रुवन्न् । अ॒ब्रु॒व॒न्नुप॑ । उप॑ नः । न॒ आ । आ व॑र्तस्व । व॒र्त॒स्व॒ ह॒व्यम् । ह॒व्यम् नः॑ । नो॒ व॒ह॒ । व॒हेति॑ । इति॒ सः । सो᳚ऽब्रवीत् । अ॒ब्र॒वी॒द् वर᳚म् । वर॑म् ॅवृणै । वृ॒णै॒ मह्य᳚म् । मह्य॑मे॒व । ए॒व वा॑जप्रस॒वीय᳚म् । वा॒ज॒प्र॒स॒वीय॑म् जुहवन्न् । वा॒ज॒प्र॒स॒वीय॒मिति॑ वाज - प्र॒स॒वीय᳚म् । जु॒ह॒व॒न्निति॑ । इति॒ तस्मा᳚त् । तस्मा॑द॒ग्नये᳚ । अ॒ग्नये॑ वाजप्रस॒वीय᳚म् । वा॒ज॒प्र॒स॒वीय॑म् जुह्वति । वा॒ज॒प्र॒स॒वीय॒मिति॑ वाज - प्र॒स॒वीय᳚म् । जु॒ह्व॒ति॒ यत् । यद् वा॑जप्रस॒वीय᳚म् । वा॒ज॒प्र॒स॒वीय॑म् जु॒होति॑ । वा॒ज॒प्र॒स॒वीय॒मिति॑ वाज - प्र॒स॒वीय᳚म् । जु॒होत्य॒ग्निम् । अ॒ग्निमे॒व । ए॒व तत् । तद् भा॑ग॒धेये॑न । भा॒ग॒धेये॑न॒ सम् । भा॒ग॒धेये॒नेति॑ भाग - धेये॑न । सम॑र्द्धयति । अ॒र्द्ध॒य॒त्यथो᳚ । अथो॑ अभिषे॒कः । अथो॒ इत्यथो᳚ । अ॒भि॒षे॒क ए॒व । अ॒भि॒षे॒क इत्य॑भि - से॒कः । ए॒वास्य॑ । अ॒स्य॒ सः । स च॑तुर्द॒शभिः॑ । च॒तु॒र्द॒शभि॑र् जुहोति । च॒तु॒र्द॒शभि॒रिति॑ चतुर्द॒श - भिः॒ । जु॒हो॒ति॒ स॒प्त । स॒प्त ग्रा॒म्याः । ग्रा॒म्या ओष॑धयः । ओष॑धयः स॒प्त । स॒प्तार॒ण्याः \newline

\textbf{Jatai Paata} \newline

1. अ॒ग्निर् दे॒वेभ्यो॑ दे॒वेभ्यो॒ ऽग्नि र॒ग्निर् दे॒वेभ्यः॑ । \newline
2. दे॒वेभ्यो ऽपाप॑ दे॒वेभ्यो॑ दे॒वेभ्यो ऽप॑ । \newline
3. अपा᳚क्राम दक्राम॒ दपापा᳚ क्रामत् । \newline
4. अ॒क्रा॒म॒द् भा॒ग॒धेय॑म् भाग॒धेय॑ मक्राम दक्रामद् भाग॒धेय᳚म् । \newline
5. भा॒ग॒धेय॑ मि॒च्छमा॑न इ॒च्छमा॑नो भाग॒धेय॑म् भाग॒धेय॑ मि॒च्छमा॑नः । \newline
6. भा॒ग॒धेय॒मिति॑ भाग - धेय᳚म् । \newline
7. इ॒च्छमा॑न॒ स्तम् त मि॒च्छमा॑न इ॒च्छमा॑न॒ स्तम् । \newline
8. तम् दे॒वा दे॒वा स्तम् तम् दे॒वाः । \newline
9. दे॒वा अ॑ब्रुवन् नब्रुवन् दे॒वा दे॒वा अ॑ब्रुवन्न् । \newline
10. अ॒ब्रु॒व॒न् नुपोपा᳚ ब्रुवन् नब्रुव॒न् नुप॑ । \newline
11. उप॑ नो न॒ उपोप॑ नः । \newline
12. न॒ आ नो॑ न॒ आ । \newline
13. आ व॑र्तस्व वर्त॒स्वा व॑र्तस्व । \newline
14. व॒र्त॒स्व॒ ह॒व्यꣳ ह॒व्यं ॅव॑र्तस्व वर्तस्व ह॒व्यम् । \newline
15. ह॒व्यम् नो॑ नो ह॒व्यꣳ ह॒व्यम् नः॑ । \newline
16. नो॒ व॒ह॒ व॒ह॒ नो॒ नो॒ व॒ह॒ । \newline
17. व॒हे तीति॑ वह व॒हेति॑ । \newline
18. इति॒ स स इतीति॒ सः । \newline
19. सो᳚ ऽब्रवी दब्रवी॒थ् स सो᳚ ऽब्रवीत् । \newline
20. अ॒ब्र॒वी॒द् वरं॒ ॅवर॑ मब्रवी दब्रवी॒द् वर᳚म् । \newline
21. वरं॑ ॅवृणै वृणै॒ वरं॒ ॅवरं॑ ॅवृणै । \newline
22. वृ॒णै॒ मह्य॒म् मह्यं॑ ॅवृणै वृणै॒ मह्य᳚म् । \newline
23. मह्य॑ मे॒वैव मह्य॒म् मह्य॑ मे॒व । \newline
24. ए॒व वा॑जप्रस॒वीयं॑ ॅवाजप्रस॒वीय॑ मे॒वैव वा॑जप्रस॒वीय᳚म् । \newline
25. वा॒ज॒प्र॒स॒वीय॑म् जुहवन् जुहवन्. वाजप्रस॒वीयं॑ ॅवाजप्रस॒वीय॑म् जुहवन्न् । \newline
26. वा॒ज॒प्र॒स॒वीय॒मिति॑ वाज - प्र॒स॒वीय᳚म् । \newline
27. जु॒ह॒व॒न् नितीति॑ जुहवन् जुहव॒न् निति॑ । \newline
28. इति॒ तस्मा॒त् तस्मा॒ दितीति॒ तस्मा᳚त् । \newline
29. तस्मा॑ द॒ग्नये॒ ऽग्नये॒ तस्मा॒त् तस्मा॑ द॒ग्नये᳚ । \newline
30. अ॒ग्नये॑ वाजप्रस॒वीयं॑ ॅवाजप्रस॒वीय॑ म॒ग्नये॒ ऽग्नये॑ वाजप्रस॒वीय᳚म् । \newline
31. वा॒ज॒प्र॒स॒वीय॑म् जुह्वति जुह्वति वाजप्रस॒वीयं॑ ॅवाजप्रस॒वीय॑म् जुह्वति । \newline
32. वा॒ज॒प्र॒स॒वीय॒मिति॑ वाज - प्र॒स॒वीय᳚म् । \newline
33. जु॒ह्व॒ति॒ यद् यज् जु॑ह्वति जुह्वति॒ यत् । \newline
34. यद् वा॑जप्रस॒वीयं॑ ॅवाजप्रस॒वीयं॒ ॅयद् यद् वा॑जप्रस॒वीय᳚म् । \newline
35. वा॒ज॒प्र॒स॒वीय॑म् जु॒होति॑ जु॒होति॑ वाजप्रस॒वीयं॑ ॅवाजप्रस॒वीय॑म् जु॒होति॑ । \newline
36. वा॒ज॒प्र॒स॒वीय॒मिति॑ वाज - प्र॒स॒वीय᳚म् । \newline
37. जु॒हो त्य॒ग्नि म॒ग्निम् जु॒होति॑ जु॒हो त्य॒ग्निम् । \newline
38. अ॒ग्नि मे॒वै वाग्नि म॒ग्नि मे॒व । \newline
39. ए॒व तत् तदे॒ वैव तत् । \newline
40. तद् भा॑ग॒धेये॑न भाग॒धेये॑न॒ तत् तद् भा॑ग॒धेये॑न । \newline
41. भा॒ग॒धेये॑न॒ सꣳ सम् भा॑ग॒धेये॑न भाग॒धेये॑न॒ सम् । \newline
42. भा॒ग॒धेये॒नेति॑ भाग - धेये॑न । \newline
43. स म॑र्द्धय त्यर्द्धयति॒ सꣳ स म॑र्द्धयति । \newline
44. अ॒र्द्ध॒य॒ त्यथो॒ अथो॑ अर्द्धय त्यर्द्धय॒ त्यथो᳚ । \newline
45. अथो॑ अभिषे॒को॑ ऽभिषे॒को ऽथो॒ अथो॑ अभिषे॒कः । \newline
46. अथो॒ इत्यथो᳚ । \newline
47. अ॒भि॒षे॒क ए॒वैवा भि॑षे॒को॑ ऽभिषे॒क ए॒व । \newline
48. अ॒भि॒षे॒क इत्य॑भि - से॒कः । \newline
49. ए॒वास्या᳚ स्यै॒वै वास्य॑ । \newline
50. अ॒स्य॒ स सो᳚ ऽस्यास्य॒ सः । \newline
51. स च॑तुर्द॒शभि॑ श्चतुर्द॒शभिः॒ स स च॑तुर्द॒शभिः॑ । \newline
52. च॒तु॒र्द॒शभि॑र् जुहोति जुहोति चतुर्द॒शभि॑ श्चतुर्द॒शभि॑र् जुहोति । \newline
53. च॒तु॒र्द॒शभि॒रिति॑ चतुर्द॒श - भिः॒ । \newline
54. जु॒हो॒ति॒ स॒प्त स॒प्त जु॑होति जुहोति स॒प्त । \newline
55. स॒प्त ग्रा॒म्या ग्रा॒म्याः स॒प्त स॒प्त ग्रा॒म्याः । \newline
56. ग्रा॒म्या ओष॑धय॒ ओष॑धयो ग्रा॒म्या ग्रा॒म्या ओष॑धयः । \newline
57. ओष॑धयः स॒प्त स॒प्तौ ष॑धय॒ ओष॑धयः स॒प्त । \newline
58. स॒प्ता र॒ण्या आ॑र॒ण्याः स॒प्त स॒प्ता र॒ण्याः । \newline

\textbf{Ghana Paata } \newline

1. अ॒ग्निर् दे॒वेभ्यो॑ दे॒वेभ्यो॒ ऽग्नि र॒ग्निर् दे॒वेभ्यो ऽपाप॑ दे॒वेभ्यो॒ ऽग्नि र॒ग्निर् दे॒वेभ्यो ऽप॑ । \newline
2. दे॒वेभ्यो ऽपाप॑ दे॒वेभ्यो॑ दे॒वेभ्यो ऽपा᳚क्राम दक्राम॒ दप॑ दे॒वेभ्यो॑ दे॒वेभ्यो ऽपा᳚क्रामत् । \newline
3. अपा᳚क्राम दक्राम॒ दपापा᳚क्रामद् भाग॒धेय॑म् भाग॒धेय॑ मक्राम॒ दपापा᳚क्रामद् भाग॒धेय᳚म् । \newline
4. अ॒क्रा॒म॒द् भा॒ग॒धेय॑म् भाग॒धेय॑ मक्राम दक्रामद् भाग॒धेय॑ मि॒च्छमा॑न इ॒च्छमा॑नो भाग॒धेय॑ मक्राम दक्रामद् भाग॒धेय॑ मि॒च्छमा॑नः । \newline
5. भा॒ग॒धेय॑ मि॒च्छमा॑न इ॒च्छमा॑नो भाग॒धेय॑म् भाग॒धेय॑ मि॒च्छमा॑न ॒स्तम् त मि॒च्छमा॑नो भाग॒धेय॑म् भाग॒धेय॑ मि॒च्छमा॑न॒ स्तम् । \newline
6. भा॒ग॒धेय॒मिति॑ भाग - धेय᳚म् । \newline
7. इ॒च्छमा॑न॒ स्तम् त मि॒च्छमा॑न इ॒च्छमा॑न॒ स्तम् दे॒वा दे॒वा स्त मि॒च्छमा॑न इ॒च्छमा॑न॒ स्तम् दे॒वाः । \newline
8. तम् दे॒वा दे॒वा स्तम् तम् दे॒वा अ॑ब्रुवन् नब्रुवन् दे॒वा स्तम् तम् दे॒वा अ॑ब्रुवन्न् । \newline
9. दे॒वा अ॑ब्रुवन् नब्रुवन् दे॒वा दे॒वा अ॑ब्रुव॒न् नुपोपा᳚ ब्रुवन् दे॒वा दे॒वा अ॑ब्रुव॒न् नुप॑ । \newline
10. अ॒ब्रु॒व॒न् नुपोपा᳚ ब्रुवन् नब्रुव॒न् नुप॑ नो न॒ उपा᳚ ब्रुवन् नब्रुव॒न् नुप॑ नः । \newline
11. उप॑ नो न॒ उपोप॑ न॒ आ न॒ उपोप॑ न॒ आ । \newline
12. न॒ आ नो॑ न॒ आ व॑र्तस्व वर्त॒स्वा नो॑ न॒ आ व॑र्तस्व । \newline
13. आ व॑र्तस्व वर्त॒स्वा व॑र्तस्व ह॒व्यꣳ ह॒व्यं ॅव॑र्त॒स्वा व॑र्तस्व ह॒व्यम् । \newline
14. व॒र्त॒स्व॒ ह॒व्यꣳ ह॒व्यं ॅव॑र्तस्व वर्तस्व ह॒व्यम् नो॑ नो ह॒व्यं ॅव॑र्तस्व वर्तस्व ह॒व्यम् नः॑ । \newline
15. ह॒व्यम् नो॑ नो ह॒व्यꣳ ह॒व्यम् नो॑ वह वह नो ह॒व्यꣳ ह॒व्यम् नो॑ वह । \newline
16. नो॒ व॒ह॒ व॒ह॒ नो॒ नो॒ व॒हे तीति॑ वह नो नो व॒हेति॑ । \newline
17. व॒हे तीति॑ वह व॒हेति॒ स स इति॑ वह व॒हेति॒ सः । \newline
18. इति॒ स स इतीति॒ सो᳚ ऽब्रवी दब्रवी॒थ् स इतीति॒ सो᳚ ऽब्रवीत् । \newline
19. सो᳚ ऽब्रवी दब्रवी॒थ् स सो᳚ ऽब्रवी॒द् वरं॒ ॅवर॑ मब्रवी॒थ् स सो᳚ ऽब्रवी॒द् वर᳚म् । \newline
20. अ॒ब्र॒वी॒द् वरं॒ ॅवर॑ मब्रवी दब्रवी॒द् वरं॑ ॅवृणै वृणै॒ वर॑ मब्रवी दब्रवी॒द् वरं॑ ॅवृणै । \newline
21. वरं॑ ॅवृणै वृणै॒ वरं॒ ॅवरं॑ ॅवृणै॒ मह्य॒म् मह्यं॑ ॅवृणै॒ वरं॒ ॅवरं॑ ॅवृणै॒ मह्य᳚म् । \newline
22. वृ॒णै॒ मह्य॒म् मह्यं॑ ॅवृणै वृणै॒ मह्य॑ मे॒वैव मह्यं॑ ॅवृणै वृणै॒ मह्य॑ मे॒व । \newline
23. मह्य॑ मे॒वैव मह्य॒म् मह्य॑ मे॒व वा॑जप्रस॒वीयं॑ ॅवाजप्रस॒वीय॑ मे॒व मह्य॒म् मह्य॑ मे॒व वा॑जप्रस॒वीय᳚म् । \newline
24. ए॒व वा॑जप्रस॒वीयं॑ ॅवाजप्रस॒वीय॑ मे॒वैव वा॑जप्रस॒वीय॑म् जुहवन् जुहवन्. वाजप्रस॒वीय॑ मे॒वैव वा॑जप्रस॒वीय॑म् जुहवन्न् । \newline
25. वा॒ज॒प्र॒स॒वीय॑म् जुहवन् जुहवन्. वाजप्रस॒वीयं॑ ॅवाजप्रस॒वीय॑म् जुहव॒न् नितीति॑ जुहवन्. वाजप्रस॒वीयं॑ ॅवाजप्रस॒वीय॑म् जुहव॒न् निति॑ । \newline
26. वा॒ज॒प्र॒स॒वीय॒मिति॑ वाज - प्र॒स॒वीय᳚म् । \newline
27. जु॒ह॒व॒न् नितीति॑ जुहवन् जुहव॒न् निति॒ तस्मा॒त् तस्मा॒ दिति॑ जुहवन् जुहव॒न् निति॒ तस्मा᳚त् । \newline
28. इति॒ तस्मा॒त् तस्मा॒ दितीति॒ तस्मा॑ द॒ग्नये॒ ऽग्नये॒ तस्मा॒ दितीति॒ तस्मा॑ द॒ग्नये᳚ । \newline
29. तस्मा॑ द॒ग्नये॒ ऽग्नये॒ तस्मा॒त् तस्मा॑ द॒ग्नये॑ वाजप्रस॒वीयं॑ ॅवाजप्रस॒वीय॑ म॒ग्नये॒ तस्मा॒त् तस्मा॑ द॒ग्नये॑ वाजप्रस॒वीय᳚म् । \newline
30. अ॒ग्नये॑ वाजप्रस॒वीयं॑ ॅवाजप्रस॒वीय॑ म॒ग्नये॒ ऽग्नये॑ वाजप्रस॒वीय॑म् जुह्वति जुह्वति वाजप्रस॒वीय॑ म॒ग्नये॒ ऽग्नये॑ वाजप्रस॒वीय॑म् जुह्वति । \newline
31. वा॒ज॒प्र॒स॒वीय॑म् जुह्वति जुह्वति वाजप्रस॒वीयं॑ ॅवाजप्रस॒वीय॑म् जुह्वति॒ यद् यज् जु॑ह्वति वाजप्रस॒वीयं॑ ॅवाजप्रस॒वीय॑म् जुह्वति॒ यत् । \newline
32. वा॒ज॒प्र॒स॒वीय॒मिति॑ वाज - प्र॒स॒वीय᳚म् । \newline
33. जु॒ह्व॒ति॒ यद् यज् जु॑ह्वति जुह्वति॒ यद् वा॑जप्रस॒वीयं॑ ॅवाजप्रस॒वीयं॒ ॅयज् जु॑ह्वति जुह्वति॒ यद् वा॑जप्रस॒वीय᳚म् । \newline
34. यद् वा॑जप्रस॒वीयं॑ ॅवाजप्रस॒वीयं॒ ॅयद् यद् वा॑जप्रस॒वीय॑म् जु॒होति॑ जु॒होति॑ वाजप्रस॒वीयं॒ ॅयद् यद् वा॑जप्रस॒वीय॑म् जु॒होति॑ । \newline
35. वा॒ज॒प्र॒स॒वीय॑म् जु॒होति॑ जु॒होति॑ वाजप्रस॒वीयं॑ ॅवाजप्रस॒वीय॑म् जु॒हो त्य॒ग्नि म॒ग्निम् जु॒होति॑ वाजप्रस॒वीयं॑ ॅवाजप्रस॒वीय॑म् जु॒हो त्य॒ग्निम् । \newline
36. वा॒ज॒प्र॒स॒वीय॒मिति॑ वाज - प्र॒स॒वीय᳚म् । \newline
37. जु॒हो त्य॒ग्नि म॒ग्निम् जु॒होति॑ जु॒हो त्य॒ग्नि मे॒वै वाग्निम् जु॒होति॑ जु॒हो त्य॒ग्नि मे॒व । \newline
38. अ॒ग्नि मे॒वै वाग्नि म॒ग्नि मे॒व तत् तदे॒ वाग्नि म॒ग्नि मे॒व तत् । \newline
39. ए॒व तत् तदे॒ वैव तद् भा॑ग॒धेये॑न भाग॒धेये॑न॒ तदे॒ वैव तद् भा॑ग॒धेये॑न । \newline
40. तद् भा॑ग॒धेये॑न भाग॒धेये॑न॒ तत् तद् भा॑ग॒धेये॑न॒ सꣳ सम् भा॑ग॒धेये॑न॒ तत् तद् भा॑ग॒धेये॑न॒ सम् । \newline
41. भा॒ग॒धेये॑न॒ सꣳ सम् भा॑ग॒धेये॑न भाग॒धेये॑न॒ स म॑र्द्धय त्यर्द्धयति॒ सम् भा॑ग॒धेये॑न भाग॒धेये॑न॒ स म॑र्द्धयति । \newline
42. भा॒ग॒धेये॒नेति॑ भाग - धेये॑न । \newline
43. स म॑र्द्धय त्यर्द्धयति॒ सꣳ स म॑र्द्धय॒ त्यथो॒ अथो॑ अर्द्धयति॒ सꣳ स म॑र्द्धय॒ त्यथो᳚ । \newline
44. अ॒र्द्ध॒य॒ त्यथो॒ अथो॑ अर्द्धय त्यर्द्धय॒ त्यथो॑ अभिषे॒को॑ ऽभिषे॒को ऽथो॑ अर्द्धयत्य र्द्धय॒ त्यथो॑ अभिषे॒कः । \newline
45. अथो॑ अभिषे॒को॑ ऽभिषे॒को ऽथो॒ अथो॑ अभिषे॒क ए॒वै वाभि॑षे॒को ऽथो॒ अथो॑ अभिषे॒क ए॒व । \newline
46. अथो॒ इत्यथो᳚ । \newline
47. अ॒भि॒षे॒क ए॒वै वाभि॑षे॒को॑ ऽभिषे॒क ए॒वा स्या᳚स्यै॒ वाभि॑षे॒को॑ ऽभिषे॒क ए॒वास्य॑ । \newline
48. अ॒भि॒षे॒क इत्य॑भि - से॒कः । \newline
49. ए॒वास्या᳚ स्यै॒वै वास्य॒ स सो᳚ ऽस्यै॒वै वास्य॒ सः । \newline
50. अ॒स्य॒ स सो᳚ ऽस्यास्य॒ स च॑तुर्द॒शभि॑ श्चतुर्द॒शभिः॒ सो᳚ ऽस्यास्य॒ स च॑तुर्द॒शभिः॑ । \newline
51. स च॑तुर्द॒शभि॑ श्चतुर्द॒शभिः॒ स स च॑तुर्द॒शभि॑र् जुहोति जुहोति चतुर्द॒शभिः॒ स स च॑तुर्द॒शभि॑र् जुहोति । \newline
52. च॒तु॒र्द॒शभि॑र् जुहोति जुहोति चतुर्द॒शभि॑ श्चतुर्द॒शभि॑र् जुहोति स॒प्त स॒प्त जु॑होति चतुर्द॒शभि॑ श्चतुर्द॒शभि॑र् जुहोति स॒प्त । \newline
53. च॒तु॒र्द॒शभि॒रिति॑ चतुर्द॒श - भिः॒ । \newline
54. जु॒हो॒ति॒ स॒प्त स॒प्त जु॑होति जुहोति स॒प्त ग्रा॒म्या ग्रा॒म्याः स॒प्त जु॑होति जुहोति स॒प्त ग्रा॒म्याः । \newline
55. स॒प्त ग्रा॒म्या ग्रा॒म्याः स॒प्त स॒प्त ग्रा॒म्या ओष॑धय॒ ओष॑धयो ग्रा॒म्याः स॒प्त स॒प्त ग्रा॒म्या ओष॑धयः । \newline
56. ग्रा॒म्या ओष॑धय॒ ओष॑धयो ग्रा॒म्या ग्रा॒म्या ओष॑धयः स॒प्त स॒प्तौष॑धयो ग्रा॒म्या ग्रा॒म्या ओष॑धयः स॒प्त । \newline
57. ओष॑धयः स॒प्त स॒प्तौष॑धय॒ ओष॑धयः स॒प्तार॒ण्या आ॑र॒ण्याः स॒प्तौष॑धय॒ ओष॑धयः स॒प्तार॒ण्याः । \newline
58. स॒प्तार॒ण्या आ॑र॒ण्याः स॒प्त स॒प्तार॒ण्या उ॒भयी॑षा मु॒भयी॑षा मार॒ण्याः स॒प्त स॒प्तार॒ण्या उ॒भयी॑षाम् । \newline
\pagebreak
\markright{ TS 5.4.9.2  \hfill https://www.vedavms.in \hfill}

\section{ TS 5.4.9.2 }

\textbf{TS 5.4.9.2 } \newline
\textbf{Samhita Paata} \newline

-*ऽऽर॒ण्या उ॒भयी॑षा॒मव॑रुद्ध्या॒ अन्न॑स्यान्नस्य जुहो॒त्यन्न॑स्यान्न॒स्या-व॑रुद्ध्या॒ औदु॑म्बरेण स्रु॒वेण॑ जुहो॒त्यूर्ग्वा उ॑दु॒म्बर॒ ऊर्गन्न॑मू॒र्जैवास्मा॒ ऊर्ज॒मन्न॒मव॑ रुन्धे॒ ऽग्निर्वै दे॒वाना॑-म॒भिषि॑क्तोऽग्नि॒चिन्-म॑नु॒ष्या॑णां॒ तस्मा॑दग्नि॒चिद्-वर्.ष॑ति॒ न धा॑वे॒दव॑रुद्धꣳ॒॒ ह्य॑स्या-न्न॒मन्न॑मिव॒ खलु॒ वै व॒र्॒.षं ॅयद्धावे॑द॒न्नाद्-या᳚द्धावेदु॒पाव॑र्तेता॒-न्नाद्य॑मे॒वाभ्यु॒ - [  ] \newline

\textbf{Pada Paata} \newline

आ॒र॒ण्याः । उ॒भयी॑षाम् । अव॑रुद्ध्या॒ इत्यव॑ - रु॒द्ध्यै॒ । अन्न॑स्यान्न॒स्येत्यन्न॑स्य - अ॒न्न॒स्य॒ । जु॒हो॒ति॒ । अन्न॑स्यान्न॒स्येत्यन्न॑स्य - अ॒न्न॒स्य॒ । अव॑रुद्ध्या॒ इत्यव॑ - रु॒द्ध्यै॒ । औदु॑बंरेण । स्रु॒वेण॑ । जु॒हो॒ति॒ । ऊर्क् । वै । उ॒दु॒बंरः॑ । ऊर्क् । अन्न᳚म् । ऊ॒र्जा । ए॒व । अ॒स्मै॒ । ऊर्ज᳚म् । अन्न᳚म् । अवेति॑ । रु॒न्धे॒ । अ॒ग्निः । वै । दे॒वाना᳚म् । अ॒भिषि॑क्त॒ इत्य॒भि - सि॒क्तः॒ । अ॒ग्नि॒चिदित्य॑ग्नि - चित् । म॒नु॒ष्या॑णाम् । तस्मा᳚त् । अ॒ग्नि॒चिदित्य॑ग्नि - चित् । वर्.ष॑ति । न । धा॒वे॒त् । अव॑रुद्ध॒मित्यव॑ - रु॒द्ध॒म् । हि । अ॒स्य॒ । अन्न᳚म् । अन्न᳚म् । इ॒व॒ । खलु॑ । वै । व॒र्॒.षम् । यत् । धावे᳚त् । अ॒न्नाद्या॒दित्य॑न्न - अद्या᳚त् । धा॒वे॒त् । उ॒पाव॑र्ते॒तेत्य॑प-आव॑र्तेत । अ॒न्नाद्य॒मित्य॑न्न - अद्य᳚म् । ए॒व । अ॒भीति॑ ।  \newline


\textbf{Krama Paata} \newline

आ॒र॒ण्या उ॒भयी॑षाम् । उ॒भयी॑षा॒मव॑रुद्ध्यै । अव॑रुद्ध्या॒ अन्न॑स्यान्नस्य । अव॑रुद्ध्या॒ इत्यव॑ - रु॒द्ध्यै॒ । अन्न॑स्यान्नस्य जुहोति । अन्न॑स्यान्न॒स्येत्यन्न॑स्य - अ॒न्न॒स्य॒ । जु॒हो॒त्यन्न॑स्यान्नस्य । अन्न॑स्यान्न॒स्याव॑रुद्ध्यै । अन्न॑स्यान्न॒स्येत्यन्न॑स्य - अ॒न्न॒स्य॒ । अव॑रुद्ध्या॒ औदु॑म्बरेण । अव॑रुद्ध्या॒ इत्यव॑ - रु॒द्ध्यै॒ । औदु॑म्बरेण स्रु॒वेण॑ । स्रु॒वेण॑ जुहोति । जु॒हो॒त्यूर्क् । ऊर्ग् वै । वा उ॑दु॒म्बरः॑ । उ॒दु॒म्बर॒ ऊर्क् । ऊर्गन्न᳚म् । अन्न॑मू॒र्जा । ऊ॒र्जैव । ए॒वास्मै᳚ । अ॒स्मा॒ ऊर्ज᳚म् । ऊर्ज॒मन्न᳚म् । अन्न॒मव॑ । अव॑ रुन्धे । रु॒न्धे॒ऽग्निः । अ॒ग्निर् वै । वै दे॒वाना᳚म् । दे॒वाना॑म॒भिषि॑क्तः । अ॒भिषि॑क्तोऽग्नि॒चित् । अ॒भिषि॑क्त॒ इत्य॒भि - सि॒क्तः॒ । अ॒ग्नि॒चिन् म॑नु॒ष्या॑णाम् । अ॒ग्नि॒चिदित्य॑ग्नि - चित् । म॒नु॒ष्या॑णा॒म् तस्मा᳚त् । तस्मा॑दग्नि॒चित् । अ॒ग्नि॒चिद् वर्.ष॑ति । अ॒ग्नि॒चिदित्य॑ग्नि - चित् । वर्.ष॑ति॒ न । न धा॑वेत् । धा॒वे॒दव॑रुद्धम् । अव॑रुद्धꣳ॒॒ हि । अव॑रुद्ध॒मित्यव॑ - रु॒द्ध॒म् । ह्य॑स्य । अ॒स्यान्न᳚म् । अन्न॒मन्न᳚म् । अन्न॑मिव । इ॒व॒ खलु॑ । खलु॒ वै । वै व॒र्॒.षम् । व॒र्॒.षम् ॅयत् । यद् धावे᳚त् । धावे॑द॒न्नाद्या᳚त् । अ॒न्नाद्या᳚द् धावेत् । अ॒न्नाद्या॒दित्य॑न्न - अद्या᳚त् । धा॒वे॒दु॒पाव॑र्तेत । उ॒पाव॑र्तेता॒न्नाद्य᳚म् । उ॒पाव॑र्ते॒तेत्यु॑प - आव॑र्तेत । अ॒न्नाद्य॑मे॒व । अ॒न्नाद्य॒मित्य॑न्न - अद्य᳚म् । ए॒वाभि । अ॒भ्यु॑पाव॑र्तते \newline

\textbf{Jatai Paata} \newline

1. आ॒र॒ण्या उ॒भयी॑षा मु॒भयी॑षा मार॒ण्या आ॑र॒ण्या उ॒भयी॑षाम् । \newline
2. उ॒भयी॑षा॒ मव॑रुद्ध्या॒ अव॑रुद्ध्या उ॒भयी॑षा मु॒भयी॑षा॒ मव॑रुद्ध्यै । \newline
3. अव॑रुद्ध्या॒ अन्न॑स्यान्न॒स्या न्न॑स्यान्न॒स्या व॑रुद्ध्या॒ अव॑रुद्ध्या॒ अन्न॑स्यान्नस्य । \newline
4. अव॑रुद्ध्या॒ इत्यव॑ - रु॒द्ध्यै॒ । \newline
5. अन्न॑स्यान्नस्य जुहोति जुहो॒ त्यन्न॑स्यान्न॒स्या न्न॑स्यान्नस्य जुहोति । \newline
6. अन्न॑स्यान्न॒स्येत्यन्न॑स्य - अ॒न्न॒स्य॒ । \newline
7. जु॒हो॒ त्यन्न॑स्यान्न॒स्या न्न॑स्यान्नस्य जुहोति जुहो॒ त्यन्न॑स्यान्नस्य । \newline
8. अन्न॑स्यान्न॒स्या व॑रुद्ध्या॒ अव॑रुद्ध्या॒ अन्न॑स्यान्न॒स्या न्न॑स्यान्न॒स्या व॑रुद्ध्यै । \newline
9. अन्न॑स्यान्न॒स्येत्यन्न॑स्य - अ॒न्न॒स्य॒ । \newline
10. अव॑रुद्ध्या॒ औदुं॑बरे॒ णौदुं॑बरे॒णा व॑रुद्ध्या॒ अव॑रुद्ध्या॒ औदुं॑बरेण । \newline
11. अव॑रुद्ध्या॒ इत्यव॑ - रु॒द्ध्यै॒ । \newline
12. औदुं॑बरेण स्रु॒वेण॑ स्रु॒वे णौदुं॑बरे॒ णौदुं॑बरेण स्रु॒वेण॑ । \newline
13. स्रु॒वेण॑ जुहोति जुहोति स्रु॒वेण॑ स्रु॒वेण॑ जुहोति । \newline
14. जु॒हो॒ त्यूर्गूर्ग् जु॑होति जुहो॒ त्यूर्क् । \newline
15. ऊर्ग् वै वा ऊर्गूर्ग् वै । \newline
16. वा उ॑दुं॒बर॑ उदुं॒बरो॒ वै वा उ॑दुं॒बरः॑ । \newline
17. उ॒दुं॒बर॒ ऊर्गूर्गु॑ दुं॒बर॑ उदुं॒बर॒ ऊर्क् । \newline
18. ऊर्गन्न॒ मन्न॒ मूर्गूर् गन्न᳚म् । \newline
19. अन्न॑ मू॒र्जोर्जा ऽन्न॒ मन्न॑ मू॒र्जा । \newline
20. ऊ॒र्जैवै वोर्जोर् जैव । \newline
21. ए॒वास्मा॑ अस्मा ए॒वै वास्मै᳚ । \newline
22. अ॒स्मा॒ ऊर्ज॒ मूर्ज॑ मस्मा अस्मा॒ ऊर्ज᳚म् । \newline
23. ऊर्ज॒ मन्न॒ मन्न॒ मूर्ज॒ मूर्ज॒ मन्न᳚म् । \newline
24. अन्न॒ मवा वान्न॒ मन्न॒ मव॑ । \newline
25. अव॑ रुन्धे रु॒न्धे ऽवाव॑ रुन्धे । \newline
26. रु॒न्धे॒ ऽग्नि र॒ग्नी रु॑न्धे रुन्धे॒ ऽग्निः । \newline
27. अ॒ग्निर् वै वा अ॒ग्नि र॒ग्निर् वै । \newline
28. वै दे॒वाना᳚म् दे॒वानां॒ ॅवै वै दे॒वाना᳚म् । \newline
29. दे॒वाना॑ म॒भिषि॑क्तो॒ ऽभिषि॑क्तो दे॒वाना᳚म् दे॒वाना॑ म॒भिषि॑क्तः । \newline
30. अ॒भिषि॑क्तो ऽग्नि॒चि द॑ग्नि॒चि द॒भिषि॑क्तो॒ ऽभिषि॑क्तो ऽग्नि॒चित् । \newline
31. अ॒भिषि॑क्त॒ इत्य॒भि - सि॒क्तः॒ । \newline
32. अ॒ग्नि॒चिन् म॑नु॒ष्या॑णाम् मनु॒ष्या॑णा मग्नि॒चि द॑ग्नि॒चिन् म॑नु॒ष्या॑णाम् । \newline
33. अ॒ग्नि॒चिदित्य॑ग्नि - चित् । \newline
34. म॒नु॒ष्या॑णा॒म् तस्मा॒त् तस्मा᳚न् मनु॒ष्या॑णाम् मनु॒ष्या॑णा॒म् तस्मा᳚त् । \newline
35. तस्मा॑ दग्नि॒चि द॑ग्नि॒चित् तस्मा॒त् तस्मा॑ दग्नि॒चित् । \newline
36. अ॒ग्नि॒चिद् वर्.ष॑ति॒ वर्.ष॑ त्यग्नि॒चि द॑ग्नि॒चिद् वर्.ष॑ति । \newline
37. अ॒ग्नि॒चिदित्य॑ग्नि - चित् । \newline
38. वर्.ष॑ति॒ न न वर्.ष॑ति॒ वर्.ष॑ति॒ न । \newline
39. न धा॑वेद् धावे॒न् न न धा॑वेत् । \newline
40. धा॒वे॒ दव॑रुद्ध॒ मव॑रुद्धम् धावेद् धावे॒ दव॑रुद्धम् । \newline
41. अव॑रुद्धꣳ॒॒ हि ह्यव॑रुद्ध॒ मव॑रुद्धꣳ॒॒ हि । \newline
42. अव॑रुद्ध॒मित्यव॑ - रु॒द्ध॒म् । \newline
43. ह्य॑स्यास्य॒ हि ह्य॑स्य । \newline
44. अ॒स्यान्न॒ मन्न॑ मस्या॒ स्यान्न᳚म् । \newline
45. अन्न॒ मन्न᳚म् । \newline
46. अन्न॑ मिवे॒वान्न॒ मन्न॑ मिव । \newline
47. इ॒व॒ खलु॒ खल्वि॑वेव॒ खलु॑ । \newline
48. खलु॒ वै वै खलु॒ खलु॒ वै । \newline
49. वै व॒र्॒.षं ॅव॒र्॒.षं ॅवै वै व॒र्॒.षम् । \newline
50. व॒र्॒.षं ॅयद् यद् व॒र्॒.षं ॅव॒र्॒.षं ॅयत् । \newline
51. यद् धावे॒द् धावे॒द् यद् यद् धावे᳚त् । \newline
52. धावे॑ द॒न्नाद्या॑ द॒न्नाद्या॒द् धावे॒द् धावे॑ द॒न्नाद्या᳚त् । \newline
53. अ॒न्नाद्या᳚द् धावेद् धावे द॒न्नाद्या॑ द॒न्नाद्या᳚द् धावेत् । \newline
54. अ॒न्नाद्या॒दित्य॑न्न - अद्या᳚त् । \newline
55. धा॒वे॒ दु॒पाव॑र्तेतो॒ पाव॑र्तेत धावेद् धावे दु॒पाव॑र्तेत । \newline
56. उ॒पाव॑र्तेता॒ न्नाद्य॑ म॒न्नाद्य॑ मु॒पाव॑र्तेतो॒ पाव॑र्ते ता॒न्नाद्य᳚म् । \newline
57. उ॒पाव॑र्ते॒तेत्य॑प - आव॑र्तेत । \newline
58. अ॒न्नाद्य॑ मे॒वै वान्नाद्य॑ म॒न्नाद्य॑ मे॒व । \newline
59. अ॒न्नाद्य॒मित्य॑न्न - अद्य᳚म् । \newline
60. ए॒वाभ्या᳚(1॒) भ्ये॑वै वाभि । \newline
61. अ॒भ्यु॑पाव॑र्तत उ॒पाव॑र्तते॒ ऽभ्या᳚(1॒)भ्यु॑ पाव॑र्तते । \newline

\textbf{Ghana Paata } \newline

1. आ॒र॒ण्या उ॒भयी॑षा मु॒भयी॑षा मार॒ण्या आ॑र॒ण्या उ॒भयी॑षा॒ मव॑रुद्ध्या॒ अव॑रुद्ध्या उ॒भयी॑षा मार॒ण्या आ॑र॒ण्या उ॒भयी॑षा॒ मव॑रुद्ध्यै । \newline
2. उ॒भयी॑षा॒ मव॑रुद्ध्या॒ अव॑रुद्ध्या उ॒भयी॑षा मु॒भयी॑षा॒ मव॑रुद्ध्या॒ अन्न॑स्यान्न॒स्या न्न॑स्यान्न॒स्या व॑रुद्ध्या उ॒भयी॑षा मु॒भयी॑षा॒ मव॑रुद्ध्या॒ अन्न॑स्यान्नस्य । \newline
3. अव॑रुद्ध्या॒ अन्न॑स्यान्न॒स्या न्न॑स्यान्न॒स्या व॑रुद्ध्या॒ अव॑रुद्ध्या॒ अन्न॑स्यान्नस्य जुहोति जुहो॒ त्यन्न॑स्यान्न॒स्या व॑रुद्ध्या॒ अव॑रुद्ध्या॒ अन्न॑स्यान्नस्य जुहोति । \newline
4. अव॑रुद्ध्या॒ इत्यव॑ - रु॒द्ध्यै॒ । \newline
5. अन्न॑स्यान्नस्य जुहोति जुहो॒ त्यन्न॑स्यान्न॒स्या न्न॑स्यान्नस्य जुहो॒ त्यन्न॑स्यान्न॒स्या न्न॑स्यान्नस्य जुहो॒ त्यन्न॑स्यान्न॒स्या न्न॑स्यान्नस्य जुहो॒ त्यन्न॑स्यान्नस्य । \newline
6. अन्न॑स्यान्न॒स्येत्यन्न॑स्य - अ॒न्न॒स्य॒ । \newline
7. जु॒हो॒ त्यन्न॑स्यान्न॒स्या न्न॑स्यान्नस्य जुहोति जुहो॒ त्यन्न॑स्यान्न॒स्या व॑रुद्ध्या॒ अव॑रुद्ध्या॒ अन्न॑स्यान्नस्य जुहोति जुहो॒ त्यन्न॑स्यान्न॒स्या व॑रुद्ध्यै । \newline
8. अन्न॑स्यान्न॒स्या व॑रुद्ध्या॒ अव॑रुद्ध्या॒ अन्न॑स्यान्न॒स्या न्न॑स्यान्न॒स्या व॑रुद्ध्या॒ औदुं॑बरे॒
णौदुं॑बरे॒णा व॑रुद्ध्या॒ अन्न॑स्यान्न॒स्या न्न॑स्यान्न॒स्या व॑रुद्ध्या॒ औदुं॑बरेण । \newline
9. अन्न॑स्यान्न॒स्येत्यन्न॑स्य - अ॒न्न॒स्य॒ । \newline
10. अव॑रुद्ध्या॒ औदुं॑बरे॒ णौदुं॑बरे॒णा व॑रुद्ध्या॒ अव॑रुद्ध्या॒ औदुं॑बरेण स्रु॒वेण॑ स्रु॒वेणौदुं॑बरे॒णा व॑रुद्ध्या॒ अव॑रुद्ध्या॒ औदुं॑बरेण स्रु॒वेण॑ । \newline
11. अव॑रुद्ध्या॒ इत्यव॑ - रु॒द्ध्यै॒ । \newline
12. औदुं॑बरेण स्रु॒वेण॑ स्रु॒वे णौदुं॑बरे॒ णौदुं॑बरेण स्रु॒वेण॑ जुहोति जुहोति स्रु॒वे णौदुं॑बरे॒
णौदुं॑बरेण स्रु॒वेण॑ जुहोति । \newline
13. स्रु॒वेण॑ जुहोति जुहोति स्रु॒वेण॑ स्रु॒वेण॑ जुहो॒ त्यूर्गूर्ग् जु॑होति स्रु॒वेण॑ स्रु॒वेण॑ जुहो॒ त्यूर्क् । \newline
14. जु॒हो॒ त्यूर् गूर्ग् जु॑होति जुहो॒ त्यूर्ग् वै वा ऊर्ग् जु॑होति जुहो॒ त्यूर्ग् वै । \newline
15. ऊर्ग् वै वा ऊर् गूर्ग् वा उ॑दुं॒बर॑ उदुं॒बरो॒ वा ऊर् गूर्ग् वा उ॑दुं॒बरः॑ । \newline
16. वा उ॑दुं॒बर॑ उदुं॒बरो॒ वै वा उ॑दुं॒बर॒ ऊर् गूर्गु॑ दुं॒बरो॒ वै वा उ॑दुं॒बर॒ ऊर्क् । \newline
17. उ॒दुं॒बर॒ ऊर् गूर्गु॑ दुं॒बर॑ उदुं॒बर॒ ऊर्गन्न॒ मन्न॒ मूर्गु॑ दुं॒बर॑ उदुं॒बर॒ ऊर्गन्न᳚म् । \newline
18. ऊर्गन्न॒ मन्न॒ मूर्गूर् गन्न॑ मू॒र्जोर्जा ऽन्न॒ मूर्गूर् गन्न॑ मू॒र्जा । \newline
19. अन्न॑ मू॒र्जोर्जा ऽन्न॒ मन्न॑ मू॒र्जै वैवोर्जा ऽन्न॒ मन्न॑ मू॒र्जैव । \newline
20. ऊ॒र्जैवै वोर्जोर्जै वास्मा॑ अस्मा ए॒वोर् जोर्जै वास्मै᳚ । \newline
21. ए॒वास्मा॑ अस्मा ए॒वै वास्मा॒ ऊर्ज॒ मूर्ज॑ मस्मा ए॒वै वास्मा॒ ऊर्ज᳚म् । \newline
22. अ॒स्मा॒ ऊर्ज॒ मूर्ज॑ मस्मा अस्मा॒ ऊर्ज॒ मन्न॒ मन्न॒ मूर्ज॑ मस्मा अस्मा॒ ऊर्ज॒ मन्न᳚म् । \newline
23. ऊर्ज॒ मन्न॒ मन्न॒ मूर्ज॒ मूर्ज॒ मन्न॒ मवावान्न॒ मूर्ज॒ मूर्ज॒ मन्न॒ मव॑ । \newline
24. अन्न॒ मवा वान्न॒ मन्न॒ मव॑ रुन्धे रु॒न्धे ऽवान्न॒ मन्न॒ मव॑ रुन्धे । \newline
25. अव॑ रुन्धे रु॒न्धे ऽवाव॑ रुन्धे॒ ऽग्नि र॒ग्नी रु॒न्धे ऽवाव॑ रुन्धे॒ ऽग्निः । \newline
26. रु॒न्धे॒ ऽग्नि र॒ग्नी रु॑न्धे रुन्धे॒ ऽग्निर् वै वा अ॒ग्नी रु॑न्धे रुन्धे॒ ऽग्निर् वै । \newline
27. अ॒ग्निर् वै वा अ॒ग्नि र॒ग्निर् वै दे॒वाना᳚म् दे॒वानां॒ ॅवा अ॒ग्नि र॒ग्निर् वै दे॒वाना᳚म् । \newline
28. वै दे॒वाना᳚म् दे॒वानां॒ ॅवै वै दे॒वाना॑ म॒भिषि॑क्तो॒ ऽभिषि॑क्तो दे॒वानां॒ ॅवै वै दे॒वाना॑ म॒भिषि॑क्तः । \newline
29. दे॒वाना॑ म॒भिषि॑क्तो॒ ऽभिषि॑क्तो दे॒वाना᳚म् दे॒वाना॑ म॒भिषि॑क्तो ऽग्नि॒चि द॑ग्नि॒चि द॒भिषि॑क्तो दे॒वाना᳚म् दे॒वाना॑ म॒भिषि॑क्तो ऽग्नि॒चित् । \newline
30. अ॒भिषि॑क्तो ऽग्नि॒चि द॑ग्नि॒चि द॒भिषि॑क्तो॒ ऽभिषि॑क्तो ऽग्नि॒चिन् म॑नु॒ष्या॑णाम् मनु॒ष्या॑णा मग्नि॒चि द॒भिषि॑क्तो॒ ऽभिषि॑क्तो ऽग्नि॒चिन् म॑नु॒ष्या॑णाम् । \newline
31. अ॒भिषि॑क्त॒ इत्य॒भि - सि॒क्तः॒ । \newline
32. अ॒ग्नि॒चिन् म॑नु॒ष्या॑णाम् मनु॒ष्या॑णा मग्नि॒चि द॑ग्नि॒चिन् म॑नु॒ष्या॑णा॒म् तस्मा॒त् तस्मा᳚न् मनु॒ष्या॑णा मग्नि॒चि द॑ग्नि॒चिन् म॑नु॒ष्या॑णा॒म् तस्मा᳚त् । \newline
33. अ॒ग्नि॒चिदित्य॑ग्नि - चित् । \newline
34. म॒नु॒ष्या॑णा॒म् तस्मा॒त् तस्मा᳚न् मनु॒ष्या॑णाम् मनु॒ष्या॑णा॒म् तस्मा॑ दग्नि॒चि द॑ग्नि॒चित् तस्मा᳚न् मनु॒ष्या॑णाम् मनु॒ष्या॑णा॒म् तस्मा॑ दग्नि॒चित् । \newline
35. तस्मा॑ दग्नि॒चि द॑ग्नि॒चित् तस्मा॒त् तस्मा॑ दग्नि॒चिद् वर्.ष॑ति॒ वर्.ष॑ त्यग्नि॒चित् तस्मा॒त् तस्मा॑ दग्नि॒चिद् वर्.ष॑ति । \newline
36. अ॒ग्नि॒चिद् वर्.ष॑ति॒ वर्.ष॑ त्यग्नि॒चि द॑ग्नि॒चिद् वर्.ष॑ति॒ न न वर्.ष॑ त्यग्नि॒चि द॑ग्नि॒चिद् वर्.ष॑ति॒ न । \newline
37. अ॒ग्नि॒चिदित्य॑ग्नि - चित् । \newline
38. वर्.ष॑ति॒ न न वर्.ष॑ति॒ वर्.ष॑ति॒ न धा॑वेद् धावे॒न् न वर्.ष॑ति॒ वर्.ष॑ति॒ न धा॑वेत् । \newline
39. न धा॑वेद् धावे॒न् न न धा॑वे॒ दव॑रुद्ध॒ मव॑रुद्धम् धावे॒न् न न धा॑वे॒ दव॑रुद्धम् । \newline
40. धा॒वे॒ दव॑रुद्ध॒ मव॑रुद्धम् धावेद् धावे॒ दव॑रुद्धꣳ॒॒ हि ह्यव॑रुद्धम् धावेद् धावे॒ दव॑रुद्धꣳ॒॒ हि । \newline
41. अव॑रुद्धꣳ॒॒ हि ह्यव॑रुद्ध॒ मव॑रुद्धꣳ॒॒ ह्य॑स्यास्य॒ ह्यव॑रुद्ध॒ मव॑रुद्धꣳ॒॒ ह्य॑स्य । \newline
42. अव॑रुद्ध॒मित्यव॑ - रु॒द्ध॒म् । \newline
43. ह्य॑स्यास्य॒ हि ह्य॑स्यान्न॒ मन्न॑ मस्य॒ हि ह्य॑स्यान्न᳚म् । \newline
44. अ॒स्यान्न॒ मन्न॑ मस्या॒ स्यान्न᳚म् । \newline
45. अन्न॒ मन्न᳚म् । \newline
46. अन्न॑ मिवे॒ वान्न॒ मन्न॑ मिव॒ खलु॒ खल्वि॒वान्न॒ मन्न॑ मिव॒ खलु॑ । \newline
47. इ॒व॒ खलु॒ खल्वि॑वेव॒ खलु॒ वै वै खल्वि॑वेव॒ खलु॒ वै । \newline
48. खलु॒ वै वै खलु॒ खलु॒ वै व॒र्॒.षं ॅव॒र्॒.षं ॅवै खलु॒ खलु॒ वै व॒र्॒.षम् । \newline
49. वै व॒र्॒.षं ॅव॒र्॒.षं ॅवै वै व॒र्॒.षं ॅयद् यद् व॒र्॒.षं ॅवै वै व॒र्॒.षं ॅयत् । \newline
50. व॒र्॒.षं ॅयद् यद् व॒र्॒.षं ॅव॒र्॒.षं ॅयद् धावे॒द् धावे॒द् यद् व॒र्॒.षं ॅव॒र्॒.षं ॅयद् धावे᳚त् । \newline
51. यद् धावे॒द् धावे॒द् यद् यद् धावे॑ द॒न्नाद्या॑ द॒न्नाद्या॒द् धावे॒द् यद् यद् धावे॑ द॒न्नाद्या᳚त् । \newline
52. धावे॑ द॒न्नाद्या॑ द॒न्नाद्या॒द् धावे॒द् धावे॑ द॒न्नाद्या᳚द् धावेद् धावे द॒न्नाद्या॒द् धावे॒द् धावे॑ द॒न्नाद्या᳚द् धावेत् । \newline
53. अ॒न्नाद्या᳚द् धावेद् धावे द॒न्नाद्या॑ द॒न्नाद्या᳚द् धावे दु॒पाव॑र्तेतो॒ पाव॑र्तेत धावे द॒न्नाद्या॑ द॒न्नाद्या᳚द् धावे दु॒पाव॑र्तेत । \newline
54. अ॒न्नाद्या॒दित्य॑न्न - अद्या᳚त् । \newline
55. धा॒वे॒ दु॒पाव॑र्तेतो॒ पाव॑र्तेत धावेद् धावे दु॒पाव॑र्तेता॒ न्नाद्य॑ म॒न्नाद्य॑ मु॒पाव॑र्तेत धावेद् धावे दु॒पाव॑र्तेता॒ न्नाद्य᳚म् । \newline
56. उ॒पाव॑र्तेता॒ न्नाद्य॑ म॒न्नाद्य॑ मु॒पाव॑र्तेतो॒ पाव॑र्तेता॒ न्नाद्य॑ मे॒वै वान्नाद्य॑ मु॒पाव॑र्तेतो॒ पाव॑र्तेता॒ न्नाद्य॑ मे॒व । \newline
57. उ॒पाव॑र्ते॒तेत्य॑प - आव॑र्तेत । \newline
58. अ॒न्नाद्य॑ मे॒वै वान्नाद्य॑ म॒न्नाद्य॑ मे॒वाभ्या᳚(1॒)भ्ये॑ वान्नाद्य॑ म॒न्नाद्य॑ मे॒वाभि । \newline
59. अ॒न्नाद्य॒मित्य॑न्न - अद्य᳚म् । \newline
60. ए॒वाभ्या᳚(1॒)भ्ये॑वै वाभ्यु॑ पाव॑र्तत उ॒पाव॑र्तते॒ ऽभ्ये॑वै वाभ्यु॑ पाव॑र्तते । \newline
61. अ॒भ्यु॑पाव॑र्तत उ॒पाव॑र्तते॒ ऽभ्या᳚(1॒)भ्यु॑ पाव॑र्तते॒ नक्तो॒षासा॒ नक्तो॒षा सो॒पाव॑र्तते॒ ऽभ्या᳚(1॒)भ्यु॑पाव॑र्तते॒ नक्तो॒षासा᳚ । \newline
\pagebreak
\markright{ TS 5.4.9.3  \hfill https://www.vedavms.in \hfill}

\section{ TS 5.4.9.3 }

\textbf{TS 5.4.9.3 } \newline
\textbf{Samhita Paata} \newline

-पाव॑र्तते॒ नक्तो॒षासेति॑ कृ॒ष्णायै᳚ श्वे॒तव॑थ्सायै॒ पय॑सा जुहो॒त्यह्नै॒वास्मै॒ रात्रिं॒ प्रदा॑पयति॒ रात्रि॒याऽह॑रहोरा॒त्रे ए॒वास्मै॒ प्रत्ते॒ काम॑म॒न्नाद्यं॑ दुहाते राष्ट्र॒भृतो॑ जुहोति रा॒ष्ट्रमे॒वाव॑ रुन्धे ष॒ड्भिर्जु॑होति॒ षड्वा ऋ॒तव॑ ऋ॒तुष्वे॒व प्रति॑तिष्ठति॒ भुव॑नस्य पत॒ इति॑ रथमु॒खे पञ्चाऽऽ*हु॑तीर्जुहोति॒ वज्रो॒ वै रथो॒ वज्रे॑णै॒व दिशो॒ - [  ] \newline

\textbf{Pada Paata} \newline

उ॒पाव॑र्तत॒ इत्यु॑प - आव॑र्तते । नक्तो॒षासा᳚ । इति॑ । कृ॒ष्णायै᳚ । श्वे॒तव॑थ्साया॒ इति॑ श्वे॒त - व॒थ्सा॒यै॒ । पय॑सा । जु॒हो॒ति॒ । अह्ना᳚ । ए॒व । अ॒स्मै॒ । रात्रि᳚म् । प्रेति॑ । दा॒प॒य॒ति॒ । रात्रि॑या । अहः॑ । अ॒हो॒रा॒त्रे इत्य॑हः - रा॒त्रे । ए॒व । अ॒स्मै॒ । प्रत्ते॒ इति॑ । काम᳚म् । अ॒न्नाद्य॒मित्य॑न्न - अद्य᳚म् । दु॒हा॒ते॒ इति॑ । रा॒ष्ट्र॒भृत॒ इति॑ राष्ट्र - भृतः॑ । जु॒हो॒ति॒ । रा॒ष्ट्रम् । ए॒व । अवेति॑ । रु॒न्धे॒ । ष॒ड्भिरिति॑ षट् - भिः । जु॒हो॒ति॒ । षट् । वै । ऋ॒तवः॑ । ऋ॒तुषु॑ । ए॒व । प्रतीति॑ । ति॒ष्ठ॒ति॒ । भुव॑नस्य । प॒ते॒ । इति॑ । र॒थ॒मु॒ख इति॑ रथ - मु॒खे । पञ्च॑ । आहु॑ती॒रित्या - हु॒तीः॒ । जु॒हो॒ति॒ । वज्रः॑ । वै । रथः॑ । वज्रे॑ण । ए॒व । दिशः॑ ।  \newline


\textbf{Krama Paata} \newline

उ॒पाव॑र्तते॒ नक्तो॒षासा᳚ । उ॒पाव॑र्तत॒ इत्यु॑प - आव॑र्तते । नक्तो॒षासेति॑ । इति॑ कृ॒ष्णायै᳚ । कृ॒ष्णायै᳚ श्वे॒तव॑थ्सायै । श्वे॒तव॑थ्सायै॒ पय॑सा । श्वे॒तव॑थ्साया॒ इति॑ श्वे॒त - व॒थ्सा॒यै॒ । पय॑सा जुहोति । जु॒हो॒त्यह्ना᳚ । अह्नै॒व । ए॒वास्मै᳚ । अ॒स्मै॒ रात्रि᳚म् । रात्रि॒म् प्र । प्र दा॑पयति । दा॒प॒य॒ति॒ रात्रि॑या । रात्रि॒याऽहः॑ । अह॑रहोरा॒त्रे । अ॒हो॒रा॒त्रे ए॒व । अ॒हो॒रा॒त्रे इत्य॑हः - रा॒त्रे । ए॒वास्मै᳚ । अ॒स्मै॒ प्रत्ते᳚ । प्रत्ते॒ काम᳚म् । प्रत्ते॒ इति॒ प्रत्ते᳚ । काम॑म॒न्नाद्य᳚म् । अ॒न्नाद्य॑म् दुहाते । अ॒न्नाद्य॒मित्य॑न्न - अद्य᳚म् । दु॒हा॒ते॒ रा॒ष्ट्र॒भृतः॑ । दु॒हा॒ते॒ इति॑ दुहाते । रा॒ष्ट॒भृतो॑ जुहोति । रा॒ष्ट्र॒भृत॒ इति॑ राष्ट्र - भृतः॑ । जु॒हो॒ति॒ रा॒ष्ट्रम् । रा॒ष्ट्रमे॒व । ए॒वाव॑ । अव॑ रुन्धे । रु॒न्धे॒ ष॒ड्भिः । ष॒ड्भिर् जु॑होति । ष॒ड्भिरिति॑ षट् - भिः । जु॒हो॒ति॒ षट् । षड् वै । वा ऋ॒तवः॑ । ऋ॒तव॑ ऋ॒तुषु॑ । ऋ॒तुष्वे॒व । ए॒व प्रति॑ । प्रति॑ तिष्ठति । ति॒ष्ठ॒ति॒ भुव॑नस्य । भुव॑नस्य पते । प॒त॒ इति॑ । इति॑ रथमु॒खे । र॒थ॒मु॒खे पञ्च॑ । र॒थ॒मु॒ख इति॑ रथ - मु॒खे । पञ्चाहु॑तीः । आहु॑तीर् जुहोति । आहु॑ती॒रित्या - हु॒तीः॒ । जु॒हो॒ति॒ वज्रः॑ । वज्रो॒ वै । वै रथः॑ । रथो॒ वज्रे॑ण । वज्रे॑णै॒व । ए॒व दिशः॑ । दिशो॒ऽभि \newline

\textbf{Jatai Paata} \newline

1. उ॒पाव॑र्तते॒ नक्तो॒षासा॒ नक्तो॒षा सो॒पाव॑र्तत उ॒पाव॑र्तते॒ नक्तो॒षासा᳚ । \newline
2. उ॒पाव॑र्तत॒ इत्यु॑प - आव॑र्तते । \newline
3. नक्तो॒षासेतीति॒ नक्तो॒षासा॒ नक्तो॒षासेति॑ । \newline
4. इति॑ कृ॒ष्णायै॑ कृ॒ष्णाया॒ इतीति॑ कृ॒ष्णायै᳚ । \newline
5. कृ॒ष्णायै᳚ श्वे॒तव॑थ्सायै श्वे॒तव॑थ्सायै कृ॒ष्णायै॑ कृ॒ष्णायै᳚ श्वे॒तव॑थ्सायै । \newline
6. श्वे॒तव॑थ्सायै॒ पय॑सा॒ पय॑सा श्वे॒तव॑थ्सायै श्वे॒तव॑थ्सायै॒ पय॑सा । \newline
7. श्वे॒तव॑थ्साया॒ इति॑ श्वे॒त - व॒थ्सा॒यै॒ । \newline
8. पय॑सा जुहोति जुहोति॒ पय॑सा॒ पय॑सा जुहोति । \newline
9. जु॒हो॒ त्यह्ना ऽह्ना॑ जुहोति जुहो॒ त्यह्ना᳚ । \newline
10. अह्नै॒ वैवाह्ना ऽह्नै॒व । \newline
11. ए॒वास्मा॑ अस्मा ए॒वै वास्मै᳚ । \newline
12. अ॒स्मै॒ रात्रिꣳ॒॒ रात्रि॑ मस्मा अस्मै॒ रात्रि᳚म् । \newline
13. रात्रि॒म् प्र प्र रात्रिꣳ॒॒ रात्रि॒म् प्र । \newline
14. प्र दा॑पयति दापयति॒ प्र प्र दा॑पयति । \newline
15. दा॒प॒य॒ति॒ रात्रि॑या॒ रात्रि॑या दापयति दापयति॒ रात्रि॑या । \newline
16. रात्रि॒या ऽह॒ रहा॒ रात्रि॑या॒ रात्रि॒या ऽहः॑ । \newline
17. अह॑ रहोरा॒त्रे अ॑होरा॒त्रे अह॒ रह॑ रहोरा॒त्रे । \newline
18. अ॒हो॒रा॒त्रे ए॒वैवा हो॑रा॒त्रे अ॑होरा॒त्रे ए॒व । \newline
19. अ॒हो॒रा॒त्रे इत्य॑हः - रा॒त्रे । \newline
20. ए॒वास्मा॑ अस्मा ए॒वै वास्मै᳚ । \newline
21. अ॒स्मै॒ प्रत्ते॒ प्रत्ते॑ अस्मा अस्मै॒ प्रत्ते᳚ । \newline
22. प्रत्ते॒ काम॒म् काम॒म् प्रत्ते॒ प्रत्ते॒ काम᳚म् । \newline
23. प्रत्ते॒ इति॒ प्रत्ते᳚ । \newline
24. काम॑ म॒न्नाद्य॑ म॒न्नाद्य॒म् काम॒म् काम॑ म॒न्नाद्य᳚म् । \newline
25. अ॒न्नाद्य॑म् दुहाते दुहाते अ॒न्नाद्य॑ म॒न्नाद्य॑म् दुहाते । \newline
26. अ॒न्नाद्य॒मित्य॑न्न - अद्य᳚म् । \newline
27. दु॒हा॒ते॒ रा॒ष्ट्र॒भृतो॑ राष्ट्र॒भृतो॑ दुहाते दुहाते राष्ट्र॒भृतः॑ । \newline
28. दु॒हा॒ते॒ इति॑ दुहाते । \newline
29. रा॒ष्ट्र॒भृतो॑ जुहोति जुहोति राष्ट्र॒भृतो॑ राष्ट्र॒भृतो॑ जुहोति । \newline
30. रा॒ष्ट्र॒भृत॒ इति॑ राष्ट्र - भृतः॑ । \newline
31. जु॒हो॒ति॒ रा॒ष्ट्रꣳ रा॒ष्ट्रम् जु॑होति जुहोति रा॒ष्ट्रम् । \newline
32. रा॒ष्ट्र मे॒वैव रा॒ष्ट्रꣳ रा॒ष्ट्र मे॒व । \newline
33. ए॒वावा वै॒वै वाव॑ । \newline
34. अव॑ रुन्धे रु॒न्धे ऽवाव॑ रुन्धे । \newline
35. रु॒न्धे॒ ष॒ड्भि ष्ष॒ड्भी रु॑न्धे रुन्धे ष॒ड्भिः । \newline
36. ष॒ड्भिर् जु॑होति जुहोति ष॒ड्भि ष्ष॒ड्भिर् जु॑होति । \newline
37. ष॒ड्भिरिति॑ षट् - भिः । \newline
38. जु॒हो॒ति॒ षट् थ्षड् जु॑होति जुहोति॒ षट् । \newline
39. षड् वै वै षट् थ्षड् वै । \newline
40. वा ऋ॒तव॑ ऋ॒तवो॒ वै वा ऋ॒तवः॑ । \newline
41. ऋ॒तव॑ ऋ॒तुष् वृ॒तुष् वृ॒तव॑ ऋ॒तव॑ ऋ॒तुषु॑ । \newline
42. ऋ॒तु ष्वे॒वैव र्‌तुष् वृ॒तु ष्वे॒व । \newline
43. ए॒व प्रति॒ प्रत्ये॒ वैव प्रति॑ । \newline
44. प्रति॑ तिष्ठति तिष्ठति॒ प्रति॒ प्रति॑ तिष्ठति । \newline
45. ति॒ष्ठ॒ति॒ भुव॑नस्य॒ भुव॑नस्य तिष्ठति तिष्ठति॒ भुव॑नस्य । \newline
46. भुव॑नस्य पते पते॒ भुव॑नस्य॒ भुव॑नस्य पते । \newline
47. प॒त॒ इतीति॑ पते पत॒ इति॑ । \newline
48. इति॑ रथमु॒खे र॑थमु॒ख इतीति॑ रथमु॒खे । \newline
49. र॒थ॒मु॒खे पञ्च॒ पञ्च॑ रथमु॒खे र॑थमु॒खे पञ्च॑ । \newline
50. र॒थ॒मु॒ख इति॑ रथ - मु॒खे । \newline
51. पञ्चाहु॑ती॒ राहु॑तीः॒ पञ्च॒ पञ्चाहु॑तीः । \newline
52. आहु॑तीर् जुहोति जुहो॒ त्याहु॑ती॒ राहु॑तीर् जुहोति । \newline
53. आहु॑ती॒रित्या - हु॒तीः॒ । \newline
54. जु॒हो॒ति॒ वज्रो॒ वज्रो॑ जुहोति जुहोति॒ वज्रः॑ । \newline
55. वज्रो॒ वै वै वज्रो॒ वज्रो॒ वै । \newline
56. वै रथो॒ रथो॒ वै वै रथः॑ । \newline
57. रथो॒ वज्रे॑ण॒ वज्रे॑ण॒ रथो॒ रथो॒ वज्रे॑ण । \newline
58. वज्रे॑ णै॒वैव वज्रे॑ण॒ वज्रे॑णै॒व । \newline
59. ए॒व दिशो॒ दिश॑ ए॒वैव दिशः॑ । \newline
60. दिशो॒ ऽभ्य॑भि दिशो॒ दिशो॒ ऽभि । \newline

\textbf{Ghana Paata } \newline

1. उ॒पाव॑र्तते॒ नक्तो॒षासा॒ नक्तो॒षा सो॒पाव॑र्तत उ॒पाव॑र्तते॒ नक्तो॒षासे तीति॒ नक्तो॒षा सो॒पाव॑र्तत उ॒पाव॑र्तते॒ नक्तो॒षासेति॑ । \newline
2. उ॒पाव॑र्तत॒ इत्यु॑प - आव॑र्तते । \newline
3. नक्तो॒षासेतीति॒ नक्तो॒षासा॒ नक्तो॒षासेति॑ कृ॒ष्णायै॑ कृ॒ष्णाया॒ इति॒ नक्तो॒षासा॒ नक्तो॒षासेति॑ कृ॒ष्णायै᳚ । \newline
4. इति॑ कृ॒ष्णायै॑ कृ॒ष्णाया॒ इतीति॑ कृ॒ष्णायै᳚ श्वे॒तव॑थ्सायै श्वे॒तव॑थ्सायै कृ॒ष्णाया॒ इतीति॑ 
कृ॒ष्णायै᳚ श्वे॒तव॑थ्सायै । \newline
5. कृ॒ष्णायै᳚ श्वे॒तव॑थ्सायै श्वे॒तव॑थ्सायै कृ॒ष्णायै॑ कृ॒ष्णायै᳚ श्वे॒तव॑थ्सायै॒ पय॑सा॒ पय॑सा श्वे॒तव॑थ्सायै कृ॒ष्णायै॑ कृ॒ष्णायै᳚ श्वे॒तव॑थ्सायै॒ पय॑सा । \newline
6. श्वे॒तव॑थ्सायै॒ पय॑सा॒ पय॑सा श्वे॒तव॑थ्सायै श्वे॒तव॑थ्सायै॒ पय॑सा जुहोति जुहोति॒ पय॑सा श्वे॒तव॑थ्सायै श्वे॒तव॑थ्सायै॒ पय॑सा जुहोति । \newline
7. श्वे॒तव॑थ्साया॒ इति॑ श्वे॒त - व॒थ्सा॒यै॒ । \newline
8. पय॑सा जुहोति जुहोति॒ पय॑सा॒ पय॑सा जुहो॒ त्यह्ना ऽह्ना॑ जुहोति॒ पय॑सा॒ पय॑सा जुहो॒ त्यह्ना᳚ । \newline
9. जु॒हो॒ त्यह्ना ऽह्ना॑ जुहोति जुहो॒ त्यह्नै॒ वैवाह्ना॑ जुहोति जुहो॒ त्यह्नै॒व । \newline
10. अह्नै॒ वैवाह्ना ऽह्नै॒ वास्मा॑ अस्मा ए॒वाह्ना ऽह्नै॒ वास्मै᳚ । \newline
11. ए॒वास्मा॑ अस्मा ए॒वै वास्मै॒ रात्रिꣳ॒॒ रात्रि॑ मस्मा ए॒वै वास्मै॒ रात्रि᳚म् । \newline
12. अ॒स्मै॒ रात्रिꣳ॒॒ रात्रि॑ मस्मा अस्मै॒ रात्रि॒म् प्र प्र रात्रि॑ मस्मा अस्मै॒ रात्रि॒म् प्र । \newline
13. रात्रि॒म् प्र प्र रात्रिꣳ॒॒ रात्रि॒म् प्र दा॑पयति दापयति॒ प्र रात्रिꣳ॒॒ रात्रि॒म् प्र दा॑पयति । \newline
14. प्र दा॑पयति दापयति॒ प्र प्र दा॑पयति॒ रात्रि॑या॒ रात्रि॑या दापयति॒ प्र प्र दा॑पयति॒ रात्रि॑या । \newline
15. दा॒प॒य॒ति॒ रात्रि॑या॒ रात्रि॑या दापयति दापयति॒ रात्रि॒या ऽह॒ रहा॒ रात्रि॑या दापयति दापयति॒ रात्रि॒या ऽहः॑ । \newline
16. रात्रि॒या ऽह॒ रहा॒ रात्रि॑या॒ रात्रि॒या ऽह॑ रहोरा॒त्रे अ॑होरा॒त्रे अहा॒ रात्रि॑या॒ रात्रि॒या ऽह॑ रहोरा॒त्रे । \newline
17. अह॑ रहोरा॒त्रे अ॑होरा॒त्रे अह॒ रह॑ रहोरा॒त्रे ए॒वैवा हो॑रा॒त्रे अह॒ रह॑ रहोरा॒त्रे ए॒व । \newline
18. अ॒हो॒रा॒त्रे ए॒वैवा हो॑रा॒त्रे अ॑होरा॒त्रे ए॒वास्मा॑ अस्मा ए॒वाहो॑रा॒त्रे अ॑होरा॒त्रे ए॒वास्मै᳚ । \newline
19. अ॒हो॒रा॒त्रे इत्य॑हः - रा॒त्रे । \newline
20. ए॒वास्मा॑ अस्मा ए॒वै वास्मै॒ प्रत्ते॒ प्रत्ते॑ अस्मा ए॒वै वास्मै॒ प्रत्ते᳚ । \newline
21. अ॒स्मै॒ प्रत्ते॒ प्रत्ते॑ अस्मा अस्मै॒ प्रत्ते॒ काम॒म् काम॒म् प्रत्ते॑ अस्मा अस्मै॒ प्रत्ते॒ काम᳚म् । \newline
22. प्रत्ते॒ काम॒म् काम॒म् प्रत्ते॒ प्रत्ते॒ काम॑ म॒न्नाद्य॑ म॒न्नाद्य॒म् काम॒म् प्रत्ते॒ प्रत्ते॒ काम॑ म॒न्नाद्य᳚म् । \newline
23. प्रत्ते॒ इति॒ प्रत्ते᳚ । \newline
24. काम॑ म॒न्नाद्य॑ म॒न्नाद्य॒म् काम॒म् काम॑ म॒न्नाद्य॑म् दुहाते दुहाते अ॒न्नाद्य॒म् काम॒म् काम॑ म॒न्नाद्य॑म् दुहाते । \newline
25. अ॒न्नाद्य॑म् दुहाते दुहाते अ॒न्नाद्य॑ म॒न्नाद्य॑म् दुहाते राष्ट्र॒भृतो॑ राष्ट्र॒भृतो॑ दुहाते अ॒न्नाद्य॑ म॒न्नाद्य॑म् दुहाते राष्ट्र॒भृतः॑ । \newline
26. अ॒न्नाद्य॒मित्य॑न्न - अद्य᳚म् । \newline
27. दु॒हा॒ते॒ रा॒ष्ट्र॒भृतो॑ राष्ट्र॒भृतो॑ दुहाते दुहाते राष्ट्र॒भृतो॑ जुहोति जुहोति राष्ट्र॒भृतो॑ दुहाते दुहाते राष्ट्र॒भृतो॑ जुहोति । \newline
28. दु॒हा॒ते॒ इति॑ दुहाते । \newline
29. रा॒ष्ट्र॒भृतो॑ जुहोति जुहोति राष्ट्र॒भृतो॑ राष्ट्र॒भृतो॑ जुहोति रा॒ष्ट्रꣳ रा॒ष्ट्रम् जु॑होति राष्ट्र॒भृतो॑ राष्ट्र॒भृतो॑ जुहोति रा॒ष्ट्रम् । \newline
30. रा॒ष्ट्र॒भृत॒ इति॑ राष्ट्र - भृतः॑ । \newline
31. जु॒हो॒ति॒ रा॒ष्ट्रꣳ रा॒ष्ट्रम् जु॑होति जुहोति रा॒ष्ट्र मे॒वैव रा॒ष्ट्रम् जु॑होति जुहोति रा॒ष्ट्र मे॒व । \newline
32. रा॒ष्ट्र मे॒वैव रा॒ष्ट्रꣳ रा॒ष्ट्र मे॒वावा वै॒व रा॒ष्ट्रꣳ रा॒ष्ट्र मे॒वाव॑ । \newline
33. ए॒वावा वै॒वै वाव॑ रुन्धे रु॒न्धे ऽवै॒वै वाव॑ रुन्धे । \newline
34. अव॑ रुन्धे रु॒न्धे ऽवाव॑ रुन्धे ष॒ड्‌भि ष्ष॒ड्भी रु॒न्धे ऽवाव॑ रुन्धे ष॒ड्‌भिः । \newline
35. रु॒न्धे॒ ष॒ड्‌भि ष्ष॒ड्भी रु॑न्धे रुन्धे ष॒ड्‌भिर् जु॑होति जुहोति ष॒ड्भी रु॑न्धे रुन्धे ष॒ड्‌भिर् जु॑होति । \newline
36. ष॒ड्‌भिर् जु॑होति जुहोति ष॒ड्‌भि ष्ष॒ड्‌भिर् जु॑होति॒ षट् थ्षड् जु॑होति ष॒ड्‌भि ष्ष॒ड्‌भिर् जु॑होति॒ षट् । \newline
37. ष॒ड्‌भिरिति॑ षट् - भिः । \newline
38. जु॒हो॒ति॒ षट् थ्षड् जु॑होति जुहोति॒ षड् वै वै षड् जु॑होति जुहोति॒ षड् वै । \newline
39. षड् वै वै षट् थ्षड् वा ऋ॒तव॑ ऋ॒तवो॒ वै षट् थ्षड् वा ऋ॒तवः॑ । \newline
40. वा ऋ॒तव॑ ऋ॒तवो॒ वै वा ऋ॒तव॑ ऋ॒तुष् वृ॒तुष् वृ॒तवो॒ वै वा ऋ॒तव॑ ऋ॒तुषु॑ । \newline
41. ऋ॒तव॑ ऋ॒तुष् वृ॒तुष् वृ॒तव॑ ऋ॒तव॑ ऋ॒तु ष्वे॒वैव र्‌तुष् वृ॒तव॑ ऋ॒तव॑ ऋ॒तु ष्वे॒व । \newline
42. ऋ॒तु ष्वे॒वैव र्‌तु ष्वृ॒तु ष्वे॒व प्रति॒ प्रत्ये॒व र्‌तु ष्वृ॒तु ष्वे॒व प्रति॑ । \newline
43. ए॒व प्रति॒ प्रत्ये॒ वैव प्रति॑ तिष्ठति तिष्ठति॒ प्रत्ये॒ वैव प्रति॑ तिष्ठति । \newline
44. प्रति॑ तिष्ठति तिष्ठति॒ प्रति॒ प्रति॑ तिष्ठति॒ भुव॑नस्य॒ भुव॑नस्य तिष्ठति॒ प्रति॒ प्रति॑ तिष्ठति॒ भुव॑नस्य । \newline
45. ति॒ष्ठ॒ति॒ भुव॑नस्य॒ भुव॑नस्य तिष्ठति तिष्ठति॒ भुव॑नस्य पते पते॒ भुव॑नस्य तिष्ठति तिष्ठति॒ भुव॑नस्य पते । \newline
46. भुव॑नस्य पते पते॒ भुव॑नस्य॒ भुव॑नस्य पत॒ इतीति॑ पते॒ भुव॑नस्य॒ भुव॑नस्य पत॒ इति॑ । \newline
47. प॒त॒ इतीति॑ पते पत॒ इति॑ रथमु॒खे र॑थमु॒ख इति॑ पते पत॒ इति॑ रथमु॒खे । \newline
48. इति॑ रथमु॒खे र॑थमु॒ख इतीति॑ रथमु॒खे पञ्च॒ पञ्च॑ रथमु॒ख इतीति॑ रथमु॒खे पञ्च॑ । \newline
49. र॒थ॒मु॒खे पञ्च॒ पञ्च॑ रथमु॒खे र॑थमु॒खे पञ्चाहु॑ती॒ राहु॑तीः॒ पञ्च॑ रथमु॒खे र॑थमु॒खे पञ्चाहु॑तीः । \newline
50. र॒थ॒मु॒ख इति॑ रथ - मु॒खे । \newline
51. पञ्चाहु॑ती॒ राहु॑तीः॒ पञ्च॒ पञ्चाहु॑तीर् जुहोति जुहो॒ त्याहु॑तीः॒ पञ्च॒ पञ्चाहु॑तीर् जुहोति । \newline
52. आहु॑तीर् जुहोति जुहो॒ त्याहु॑ती॒ राहु॑तीर् जुहोति॒ वज्रो॒ वज्रो॑ जुहो॒ त्याहु॑ती॒ राहु॑तीर् जुहोति॒ वज्रः॑ । \newline
53. आहु॑ती॒रित्या - हु॒तीः॒ । \newline
54. जु॒हो॒ति॒ वज्रो॒ वज्रो॑ जुहोति जुहोति॒ वज्रो॒ वै वै वज्रो॑ जुहोति जुहोति॒ वज्रो॒ वै । \newline
55. वज्रो॒ वै वै वज्रो॒ वज्रो॒ वै रथो॒ रथो॒ वै वज्रो॒ वज्रो॒ वै रथः॑ । \newline
56. वै रथो॒ रथो॒ वै वै रथो॒ वज्रे॑ण॒ वज्रे॑ण॒ रथो॒ वै वै रथो॒ वज्रे॑ण । \newline
57. रथो॒ वज्रे॑ण॒ वज्रे॑ण॒ रथो॒ रथो॒ वज्रे॑णै॒वैव वज्रे॑ण॒ रथो॒ रथो॒ वज्रे॑णै॒व । \newline
58. वज्रे॑णै॒वैव वज्रे॑ण॒ वज्रे॑णै॒व दिशो॒ दिश॑ ए॒व वज्रे॑ण॒ वज्रे॑णै॒व दिशः॑ । \newline
59. ए॒व दिशो॒ दिश॑ ए॒वैव दिशो॒ ऽभ्य॑भि दिश॑ ए॒वैव दिशो॒ ऽभि । \newline
60. दिशो॒ ऽभ्य॑भि दिशो॒ दिशो॒ ऽभि ज॑यति जय त्य॒भि दिशो॒ दिशो॒ ऽभि ज॑यति । \newline
\pagebreak
\markright{ TS 5.4.9.4  \hfill https://www.vedavms.in \hfill}

\section{ TS 5.4.9.4 }

\textbf{TS 5.4.9.4 } \newline
\textbf{Samhita Paata} \newline

-ऽभि ज॑यत्यग्नि॒चितꣳ॑ ह॒ वा अ॒मुष्मि॑न् ॅलो॒के वातो॒ऽभि प॑वते वातना॒मानि॑ जुहोत्य॒भ्ये॑वैन॑-म॒मुष्मि॑न् ॅलो॒के वातः॑ पवते॒ त्रीणि॑ जुहोति॒ त्रय॑ इ॒मे लो॒का ए॒भ्य ए॒व लो॒केभ्यो॒ वात॒मव॑ रुन्धे समु॒द्रो॑ऽसि॒ नभ॑स्वा॒नित्या॑है॒तद्वै वात॑स्य रू॒पꣳ रू॒पेणै॒व वात॒मव॑ रुन्धे ऽञ्ज॒लिना॑ जुहोति॒ न ह्ये॑तेषा॑म॒न्यथा ( ) ऽऽहु॑तिरव॒कल्प॑ते ॥ \newline

\textbf{Pada Paata} \newline

अ॒भीति॑ । ज॒य॒ति॒ । अ॒ग्नि॒चित॒मित्य॑ग्नि - चित᳚म् । ह॒ । वै । अ॒मुष्मिन्न्॑ । लो॒के । वातः॑ । अ॒भीति॑ । प॒व॒ते॒ । वा॒त॒ना॒मानीति॑ वात - ना॒मानि॑ । जु॒हो॒ति॒ । अ॒भीति॑ । ए॒व । ए॒न॒म् । अ॒मुष्मिन्न्॑ । लो॒के । वातः॑ । प॒व॒ते॒ । त्रीणि॑ । जु॒हो॒ति॒ । त्रयः॑ । इ॒मे । लो॒काः । ए॒भ्यः । ए॒व । लो॒केभ्यः॑ । वात᳚म् । अवेति॑ । रु॒न्धे॒ । स॒मु॒द्रः । अ॒सि॒ । नभ॑स्वान् । इति॑ । आ॒ह॒ । ए॒तत् । वै । वात॑स्य । रू॒पम् । रू॒पेण॑ । ए॒व । वात᳚म् । अवेति॑ । रु॒न्धे॒ । अ॒ञ्ज॒लिना᳚ । जु॒हो॒ति॒ । न । हि । ए॒तेषा᳚म् । अ॒न्यथा᳚ ( ) । आहु॑ति॒रित्या - हु॒तिः॒ । अ॒व॒कल्प॑त॒ इत्य॑व - कल्प॑ते ॥  \newline


\textbf{Krama Paata} \newline

अ॒भि ज॑यति । ज॒य॒त्य॒ग्नि॒चित᳚म् । अ॒ग्नि॒चितꣳ॑ ह । अ॒ग्नि॒चित॒मित्य॑ग्नि - चित᳚म् । ह॒ वै । वा अ॒मुष्मिन्न्॑ । अ॒मुष्मि॑न् ॅलो॒के । लो॒के वातः॑ । वातो॒ऽभि । अ॒भि प॑वते । प॒व॒ते॒ वा॒त॒ना॒मानि॑ । वा॒त॒ना॒मानि॑ जुहोति । वा॒त॒ना॒मानीति॑ वात - ना॒मानि॑ । जु॒हो॒त्य॒भि । अ॒भ्ये॑व । ए॒वैन᳚म् । ए॒न॒म॒मुष्मिन्न्॑ । अ॒मुष्मि॑न् ॅलो॒के । लो॒के वातः॑ । वातः॑ पवते । प॒व॒ते॒ त्रीणि॑ । त्रीणि॑ जुहोति । जु॒हो॒ति॒ त्रयः॑ । त्रय॑ इ॒मे । इ॒मे लो॒काः । लो॒का ए॒भ्यः । ए॒भ्य ए॒व । ए॒व लो॒केभ्यः॑ । लो॒केभ्यो॒ वात᳚म् । वात॒मव॑ । अव॑ रुन्धे । रु॒न्धे॒ स॒मु॒द्रः । स॒मु॒द्रो॑ऽसि । अ॒सि॒ नभ॑स्वान् । नभ॑स्वा॒निति॑ । इत्या॑ह । आ॒है॒तत् । ए॒तद् वै । वै वात॑स्य । वात॑स्य रू॒पम् । रू॒पꣳ रू॒पेण॑ । रू॒पेणै॒व । ए॒व वात᳚म् । वात॒मव॑ । अव॑ रुन्धे । रु॒न्धे॒ऽञ्ज॒लिना᳚ । अ॒ञ्ज॒लिना॑ जुहोति । जु॒हो॒ति॒ न । न हि । ह्ये॑तेषा᳚म् । ए॒तेषा॑म॒न्यथा᳚ ( ) । अ॒न्यथाऽऽहु॑तिः । आहु॑तिरव॒कल्प॑ते । आहु॑ति॒रित्या - हु॒तिः॒ । अ॒व॒कल्प॑त॒ इत्य॑व - कल्प॑ते । \newline

\textbf{Jatai Paata} \newline

1. अ॒भि ज॑यति जय त्य॒भ्य॑भि ज॑यति । \newline
2. ज॒य॒ त्य॒ग्नि॒चित॑ मग्नि॒चित॑म् जयति जय त्यग्नि॒चित᳚म् । \newline
3. अ॒ग्नि॒चितꣳ॑ ह हाग्नि॒चित॑ मग्नि॒चितꣳ॑ ह । \newline
4. अ॒ग्नि॒चित॒मित्य॑ग्नि - चित᳚म् । \newline
5. ह॒ वै वै ह॑ ह॒ वै । \newline
6. वा अ॒मुष्मि॑न् न॒मुष्मि॒न्॒. वै वा अ॒मुष्मिन्न्॑ । \newline
7. अ॒मुष्मि॑न् ॅलो॒के लो॒के॑ ऽमुष्मि॑न् न॒मुष्मि॑न् ॅलो॒के । \newline
8. लो॒के वातो॒ वातो॑ लो॒के लो॒के वातः॑ । \newline
9. वातो॒ ऽभ्य॑भि वातो॒ वातो॒ ऽभि । \newline
10. अ॒भि प॑वते पवते॒ ऽभ्य॑भि प॑वते । \newline
11. प॒व॒ते॒ वा॒त॒ना॒मानि॑ वातना॒मानि॑ पवते पवते वातना॒मानि॑ । \newline
12. वा॒त॒ना॒मानि॑ जुहोति जुहोति वातना॒मानि॑ वातना॒मानि॑ जुहोति । \newline
13. वा॒त॒ना॒मानीति॑ वात - ना॒मानि॑ । \newline
14. जु॒हो॒त्य॒ भ्य॑भि जु॑होति जुहोत्य॒भि । \newline
15. अ॒भ्ये॑वै वाभ्या᳚(1॒) भ्ये॑व । \newline
16. ए॒वैन॑ मेन मे॒वै वैन᳚म् । \newline
17. ए॒न॒ म॒मुष्मि॑न् न॒मुष्मि॑न् नेन मेन म॒मुष्मिन्न्॑ । \newline
18. अ॒मुष्मि॑न् ॅलो॒के लो॒के॑ ऽमुष्मि॑न् न॒मुष्मि॑न् ॅलो॒के । \newline
19. लो॒के वातो॒ वातो॑ लो॒के लो॒के वातः॑ । \newline
20. वातः॑ पवते पवते॒ वातो॒ वातः॑ पवते । \newline
21. प॒व॒ते॒ त्रीणि॒ त्रीणि॑ पवते पवते॒ त्रीणि॑ । \newline
22. त्रीणि॑ जुहोति जुहोति॒ त्रीणि॒ त्रीणि॑ जुहोति । \newline
23. जु॒हो॒ति॒ त्रय॒ स्त्रयो॑ जुहोति जुहोति॒ त्रयः॑ । \newline
24. त्रय॑ इ॒म इ॒मे त्रय॒ स्त्रय॑ इ॒मे । \newline
25. इ॒मे लो॒का लो॒का इ॒म इ॒मे लो॒काः । \newline
26. लो॒का ए॒भ्य ए॒भ्यो लो॒का लो॒का ए॒भ्यः । \newline
27. ए॒भ्य ए॒वै वैभ्य ए॒भ्य ए॒व । \newline
28. ए॒व लो॒केभ्यो॑ लो॒केभ्य॑ ए॒वैव लो॒केभ्यः॑ । \newline
29. लो॒केभ्यो॒ वातं॒ ॅवात॑म् ॅलो॒केभ्यो॑ लो॒केभ्यो॒ वात᳚म् । \newline
30. वात॒ मवाव॒ वातं॒ ॅवात॒ मव॑ । \newline
31. अव॑ रुन्धे रु॒न्धे ऽवाव॑ रुन्धे । \newline
32. रु॒न्धे॒ स॒मु॒द्रः स॑मु॒द्रो रु॑न्धे रुन्धे समु॒द्रः । \newline
33. स॒मु॒द्रो᳚ ऽस्यसि समु॒द्रः स॑मु॒द्रो॑ ऽसि । \newline
34. अ॒सि॒ नभ॑स्वा॒न् नभ॑स्वा नस्यसि॒ नभ॑स्वान् । \newline
35. नभ॑स्वा॒ नितीति॒ नभ॑स्वा॒न् नभ॑स्वा॒ निति॑ । \newline
36. इत्या॑हा॒हे तीत्या॑ह । \newline
37. आ॒है॒त दे॒त दा॑हा है॒तत् । \newline
38. ए॒तद् वै वा ए॒त दे॒तद् वै । \newline
39. वै वात॑स्य॒ वात॑स्य॒ वै वै वात॑स्य । \newline
40. वात॑स्य रू॒पꣳ रू॒पं ॅवात॑स्य॒ वात॑स्य रू॒पम् । \newline
41. रू॒पꣳ रू॒पेण॑ रू॒पेण॑ रू॒पꣳ रू॒पꣳ रू॒पेण॑ । \newline
42. रू॒पे णै॒वैव रू॒पेण॑ रू॒पेणै॒व । \newline
43. ए॒व वातं॒ ॅवात॑ मे॒वैव वात᳚म् । \newline
44. वात॒ मवाव॒ वातं॒ ॅवात॒ मव॑ । \newline
45. अव॑ रुन्धे रु॒न्धे ऽवाव॑ रुन्धे । \newline
46. रु॒न्धे॒ ऽञ्ज॒लिना᳚ ऽञ्ज॒लिना॑ रुन्धे रुन्धे ऽञ्ज॒लिना᳚ । \newline
47. अ॒ञ्ज॒लिना॑ जुहोति जुहो त्यञ्ज॒लिना᳚ ऽञ्ज॒लिना॑ जुहोति । \newline
48. जु॒हो॒ति॒ न न जु॑होति जुहोति॒ न । \newline
49. न हि हि न न हि । \newline
50. ह्ये॑तेषा॑ मे॒तेषाꣳ॒॒ हि ह्ये॑तेषा᳚म् । \newline
51. ए॒तेषा॑ म॒न्यथा॒ ऽन्यथै॒तेषा॑ मे॒तेषा॑ म॒न्यथा᳚ । \newline
52. अ॒न्यथा ऽऽहु॑ती॒ राहु॑ती र॒न्यथा॒ ऽन्यथा ऽऽहु॑तीः । \newline
53. आहु॑ती रव॒कल्प॑ते ऽव॒कल्प॑त॒ आहु॑ती॒ राहु॑ती रव॒कल्प॑ते । \newline
54. आहु॑ति॒रित्या - हु॒तिः॒ । \newline
55. अ॒व॒कल्प॑त॒ इत्य॑व - कल्प॑ते । \newline

\textbf{Ghana Paata } \newline

1. अ॒भि ज॑यति जय त्य॒भ्य॑भि ज॑य त्यग्नि॒चित॑ मग्नि॒चित॑म् जय त्य॒भ्य॑भि ज॑य त्यग्नि॒चित᳚म् । \newline
2. ज॒य॒ त्य॒ग्नि॒चित॑ मग्नि॒चित॑म् जयति जय त्यग्नि॒चितꣳ॑ ह हाग्नि॒चित॑म् जयति जय त्यग्नि॒चितꣳ॑ ह । \newline
3. अ॒ग्नि॒चितꣳ॑ ह हाग्नि॒चित॑ मग्नि॒चितꣳ॑ ह॒ वै वै हा᳚ग्नि॒चित॑ मग्नि॒चितꣳ॑ ह॒ वै । \newline
4. अ॒ग्नि॒चित॒मित्य॑ग्नि - चित᳚म् । \newline
5. ह॒ वै वै ह॑ ह॒ वा अ॒मुष्मि॑न् न॒मुष्मि॒न्॒. वै ह॑ ह॒ वा अ॒मुष्मिन्न्॑ । \newline
6. वा अ॒मुष्मि॑न् न॒मुष्मि॒न्॒. वै वा अ॒मुष्मि॑न् ॅलो॒के लो॒के॑ ऽमुष्मि॒न्॒. वै वा अ॒मुष्मि॑न् ॅलो॒के । \newline
7. अ॒मुष्मि॑न् ॅलो॒के लो॒के॑ ऽमुष्मि॑न् न॒मुष्मि॑न् ॅलो॒के वातो॒ वातो॑ लो॒के॑ ऽमुष्मि॑न् न॒मुष्मि॑न् ॅलो॒के वातः॑ । \newline
8. लो॒के वातो॒ वातो॑ लो॒के लो॒के वातो॒ ऽभ्य॑भि वातो॑ लो॒के लो॒के वातो॒ ऽभि । \newline
9. वातो॒ ऽभ्य॑भि वातो॒ वातो॒ ऽभि प॑वते पवते॒ ऽभि वातो॒ वातो॒ ऽभि प॑वते । \newline
10. अ॒भि प॑वते पवते॒ ऽभ्य॑भि प॑वते वातना॒मानि॑ वातना॒मानि॑ पवते॒ ऽभ्य॑भि प॑वते वातना॒मानि॑ । \newline
11. प॒व॒ते॒ वा॒त॒ना॒मानि॑ वातना॒मानि॑ पवते पवते वातना॒मानि॑ जुहोति जुहोति वातना॒मानि॑ पवते पवते वातना॒मानि॑ जुहोति । \newline
12. वा॒त॒ना॒मानि॑ जुहोति जुहोति वातना॒मानि॑ वातना॒मानि॑ जुहो त्य॒भ्य॑भि जु॑होति वातना॒मानि॑ वातना॒मानि॑ जुहो त्य॒भि । \newline
13. वा॒त॒ना॒मानीति॑ वात - ना॒मानि॑ । \newline
14. जु॒हो॒ त्य॒भ्य॑भि जु॑होति जुहो त्य॒भ्ये॑वै वाभि जु॑होति जुहो त्य॒भ्ये॑व । \newline
15. अ॒भ्ये॑वै वाभ्या᳚(1॒)भ्ये॑ वैन॑ मेन मे॒वाभ्या᳚(1॒)भ्ये॑ वैन᳚म् । \newline
16. ए॒वैन॑ मेन मे॒वै वैन॑ म॒मुष्मि॑न् न॒मुष्मि॑न् नेन मे॒वै वैन॑ म॒मुष्मिन्न्॑ । \newline
17. ए॒न॒ म॒मुष्मि॑न् न॒मुष्मि॑न् नेन मेन म॒मुष्मि॑न् ॅलो॒के लो॒के॑ ऽमुष्मि॑न् नेन मेन म॒मुष्मि॑न् ॅलो॒के । \newline
18. अ॒मुष्मि॑न् ॅलो॒के लो॒के॑ ऽमुष्मि॑न् न॒मुष्मि॑न् ॅलो॒के वातो॒ वातो॑ लो॒के॑ ऽमुष्मि॑न् न॒मुष्मि॑न् ॅलो॒के वातः॑ । \newline
19. लो॒के वातो॒ वातो॑ लो॒के लो॒के वातः॑ पवते पवते॒ वातो॑ लो॒के लो॒के वातः॑ पवते । \newline
20. वातः॑ पवते पवते॒ वातो॒ वातः॑ पवते॒ त्रीणि॒ त्रीणि॑ पवते॒ वातो॒ वातः॑ पवते॒ त्रीणि॑ । \newline
21. प॒व॒ते॒ त्रीणि॒ त्रीणि॑ पवते पवते॒ त्रीणि॑ जुहोति जुहोति॒ त्रीणि॑ पवते पवते॒ त्रीणि॑ जुहोति । \newline
22. त्रीणि॑ जुहोति जुहोति॒ त्रीणि॒ त्रीणि॑ जुहोति॒ त्रय॒ स्त्रयो॑ जुहोति॒ त्रीणि॒ त्रीणि॑ जुहोति॒ त्रयः॑ । \newline
23. जु॒हो॒ति॒ त्रय॒ स्त्रयो॑ जुहोति जुहोति॒ त्रय॑ इ॒म इ॒मे त्रयो॑ जुहोति जुहोति॒ त्रय॑ इ॒मे । \newline
24. त्रय॑ इ॒म इ॒मे त्रय॒ स्त्रय॑ इ॒मे लो॒का लो॒का इ॒मे त्रय॒ स्त्रय॑ इ॒मे लो॒काः । \newline
25. इ॒मे लो॒का लो॒का इ॒म इ॒मे लो॒का ए॒भ्य ए॒भ्यो लो॒का इ॒म इ॒मे लो॒का ए॒भ्यः । \newline
26. लो॒का ए॒भ्य ए॒भ्यो लो॒का लो॒का ए॒भ्य ए॒वै वैभ्यो लो॒का लो॒का ए॒भ्य ए॒व । \newline
27. ए॒भ्य ए॒वै वैभ्य ए॒भ्य ए॒व लो॒केभ्यो॑ लो॒केभ्य॑ ए॒वैभ्य ए॒भ्य ए॒व लो॒केभ्यः॑ । \newline
28. ए॒व लो॒केभ्यो॑ लो॒केभ्य॑ ए॒वैव लो॒केभ्यो॒ वातं॒ ॅवात॑म् ॅलो॒केभ्य॑ ए॒वैव लो॒केभ्यो॒ वात᳚म् । \newline
29. लो॒केभ्यो॒ वातं॒ ॅवात॑म् ॅलो॒केभ्यो॑ लो॒केभ्यो॒ वात॒ मवाव॒ वात॑म् ॅलो॒केभ्यो॑ लो॒केभ्यो॒ वात॒ मव॑ । \newline
30. वात॒ मवाव॒ वातं॒ ॅवात॒ मव॑ रुन्धे रु॒न्धे ऽव॒ वातं॒ ॅवात॒ मव॑ रुन्धे । \newline
31. अव॑ रुन्धे रु॒न्धे ऽवाव॑ रुन्धे समु॒द्रः स॑मु॒द्रो रु॒न्धे ऽवाव॑ रुन्धे समु॒द्रः । \newline
32. रु॒न्धे॒ स॒मु॒द्रः स॑मु॒द्रो रु॑न्धे रुन्धे समु॒द्रो᳚ ऽस्यसि समु॒द्रो रु॑न्धे रुन्धे समु॒द्रो॑ ऽसि । \newline
33. स॒मु॒द्रो᳚ ऽस्यसि समु॒द्रः स॑मु॒द्रो॑ ऽसि॒ नभ॑स्वा॒न् नभ॑स्वा नसि समु॒द्रः स॑मु॒द्रो॑ ऽसि॒ नभ॑स्वान् । \newline
34. अ॒सि॒ नभ॑स्वा॒न् नभ॑स्वा नस्यसि॒ नभ॑स्वा॒ नितीति॒ नभ॑स्वा नस्यसि॒ नभ॑स्वा॒ निति॑ । \newline
35. नभ॑स्वा॒ नितीति॒ नभ॑स्वा॒न् नभ॑स्वा॒ नित्या॑ हा॒हेति॒ नभ॑स्वा॒न् नभ॑स्वा॒ नित्या॑ह । \newline
36. इत्या॑हा॒हे तीत्या॑है॒त दे॒त दा॒हे तीत्या॑ है॒तत् । \newline
37. आ॒है॒त दे॒त दा॑हाहै॒तद् वै वा ए॒त दा॑हा है॒तद् वै । \newline
38. ए॒तद् वै वा ए॒त दे॒तद् वै वात॑स्य॒ वात॑स्य॒ वा ए॒त दे॒तद् वै वात॑स्य । \newline
39. वै वात॑स्य॒ वात॑स्य॒ वै वै वात॑स्य रू॒पꣳ रू॒पं ॅवात॑स्य॒ वै वै वात॑स्य रू॒पम् । \newline
40. वात॑स्य रू॒पꣳ रू॒पं ॅवात॑स्य॒ वात॑स्य रू॒पꣳ रू॒पेण॑ रू॒पेण॑ रू॒पं ॅवात॑स्य॒ वात॑स्य रू॒पꣳ रू॒पेण॑ । \newline
41. रू॒पꣳ रू॒पेण॑ रू॒पेण॑ रू॒पꣳ रू॒पꣳ रू॒पेणै॒ वैव रू॒पेण॑ रू॒पꣳ रू॒पꣳ रू॒पेणै॒व । \newline
42. रू॒पेणै॒ वैव रू॒पेण॑ रू॒पेणै॒व वातं॒ ॅवात॑ मे॒व रू॒पेण॑ रू॒पेणै॒व वात᳚म् । \newline
43. ए॒व वातं॒ ॅवात॑ मे॒वैव वात॒ मवाव॒ वात॑ मे॒वैव वात॒ मव॑ । \newline
44. वात॒ मवाव॒ वातं॒ ॅवात॒ मव॑ रुन्धे रु॒न्धे ऽव॒ वातं॒ ॅवात॒ मव॑ रुन्धे । \newline
45. अव॑ रुन्धे रु॒न्धे ऽवाव॑ रुन्धे ऽञ्ज॒लिना᳚ ऽञ्ज॒लिना॑ रु॒न्धे ऽवाव॑ रुन्धे ऽञ्ज॒लिना᳚ । \newline
46. रु॒न्धे॒ ऽञ्ज॒लिना᳚ ऽञ्ज॒लिना॑ रुन्धे रुन्धे ऽञ्ज॒लिना॑ जुहोति जुहो त्यञ्ज॒लिना॑ रुन्धे रुन्धे ऽञ्ज॒लिना॑ जुहोति । \newline
47. अ॒ञ्ज॒लिना॑ जुहोति जुहो त्यञ्ज॒लिना᳚ ऽञ्ज॒लिना॑ जुहोति॒ न न जु॑हो त्यञ्ज॒लिना᳚ ऽञ्ज॒लिना॑ जुहोति॒ न । \newline
48. जु॒हो॒ति॒ न न जु॑होति जुहोति॒ न हि हि न जु॑होति जुहोति॒ न हि । \newline
49. न हि हि न न ह्ये॑तेषा॑ मे॒तेषाꣳ॒॒ हि न न ह्ये॑तेषा᳚म् । \newline
50. ह्ये॑तेषा॑ मे॒तेषाꣳ॒॒ हि ह्ये॑तेषा॑ म॒न्यथा॒ ऽन्यथै॒ तेषाꣳ॒॒ हि ह्ये॑तेषा॑ म॒न्यथा᳚ । \newline
51. ए॒तेषा॑ म॒न्यथा॒ ऽन्यथै॒तेषा॑ मे॒तेषा॑ म॒न्यथा ऽऽहु॑ती॒ राहु॑ती र॒न्य थै॒तेषा॑ मे॒तेषा॑ म॒न्यथा ऽऽहु॑तीः । \newline
52. अ॒न्यथा ऽऽहु॑ती॒ राहु॑ती र॒न्यथा॒ ऽन्यथा ऽऽहु॑ती रव॒कल्प॑ते ऽव॒कल्प॑त॒ आहु॑ती र॒न्यथा॒ ऽन्यथा ऽऽहु॑ती रव॒कल्प॑ते । \newline
53. आहु॑ती रव॒कल्प॑ते ऽव॒कल्प॑त॒ आहु॑ती॒ राहु॑ती रव॒कल्प॑ते । \newline
54. आहु॑ति॒रित्या - हु॒तिः॒ । \newline
55. अ॒व॒कल्प॑त॒ इत्य॑व - कल्प॑ते । \newline
\pagebreak
\markright{ TS 5.4.10.1  \hfill https://www.vedavms.in \hfill}

\section{ TS 5.4.10.1 }

\textbf{TS 5.4.10.1 } \newline
\textbf{Samhita Paata} \newline

सु॒व॒र्गाय॒ वै लो॒काय॑ देवर॒थो यु॑ज्यते यत्राकू॒ताय॑ मनुष्यर॒थ ए॒ष खलु॒ वै दे॑वर॒थो यद॒ग्निर॒ग्निं ॅयु॑नज्मि॒ शव॑सा घृ॒तेनेत्या॑ह यु॒नक्त्ये॒वैनꣳ॒॒ स ए॑नं ॅयु॒क्तः सु॑व॒र्गं ॅलो॒कम॒भि व॑हति॒ यथ् सर्वा॑भिः प॒ञ्चभि॑-र्यु॒ञ्ज्याद्-यु॒क्तो᳚ऽस्या॒ऽग्निः प्रच्यु॑तः स्या॒दप्र॑तिष्ठिता॒ आहु॑तयः॒ स्युरप्र॑तिष्ठिताः॒ स्तोमा॒ अप्र॑तिष्ठितान्यु॒क्थानि॑ ति॒सृभिः॑ प्रातस्सव॒ने॑ऽभि मृ॑शति त्रि॒वृ - [  ] \newline

\textbf{Pada Paata} \newline

सु॒व॒र्गायेति॑ सुवः - गाय॑ । वै । लो॒काय॑ । दे॒व॒र॒थ इति॑ देव - र॒थः । यु॒ज्य॒ते॒ । य॒त्रा॒कू॒तायेति॑ यत्र - आ॒कू॒ताय॑ । म॒नु॒ष्य॒र॒थ इति॑ मनुष्य - र॒थः । ए॒षः । खलु॑ । वै । दे॒व॒र॒थ इति॑ देव - र॒थः । यत् । अ॒ग्निः । अ॒ग्निम् । यु॒न॒ज्मि॒ । शव॑सा । घृ॒तेन॑ । इति॑ । आ॒ह॒ । यु॒नक्ति॑ । ए॒व । ए॒न॒म् । सः । ए॒न॒म् । यु॒क्तः । सु॒व॒र्गमिति॑ सुवः-गम् । लो॒कम् । अ॒भीति॑ । व॒ह॒ति॒ । यत् । सर्वा॑भिः । प॒ञ्चभि॒रिति॑ प॒ञ्च - भिः॒ । यु॒ञ्ज्यात् । यु॒क्तः । अ॒स्य॒ । अ॒ग्निः । प्रच्यु॑त॒ इति॒ प्र - च्यु॒तः॒ । स्या॒त् । अप्र॑तिष्ठिता॒ इत्यप्र॑ति - स्थि॒ताः॒ । आहु॑तय॒ इत्या - हु॒त॒यः॒ । स्युः । अप्र॑तिष्ठिता॒ इत्यप्र॑ति - स्थि॒ताः॒ । स्तोमाः᳚ । अप्र॑तिष्ठिता॒नीत्यप्र॑ति - स्थि॒ता॒नि॒ । उ॒क्थानि॑ । ति॒सृभि॒रिति॑ ति॒सृ - भिः॒ । प्रा॒त॒स्स॒व॒न इति॑ प्रातः - स॒व॒ने । अ॒भीति॑ । मृ॒श॒ति॒ । त्रि॒वृदिति॑ त्रि - वृत् ।  \newline


\textbf{Krama Paata} \newline

सु॒व॒र्गाय॒ वै । सु॒व॒र्गायेति॑ सुवः - गाय॑ । वै लो॒काय॑ । लो॒काय॑ देवर॒थः । दे॒व॒र॒थो यु॑ज्यते । दे॒व॒र॒थ इति॑ देव - र॒थः । यु॒ज्य॒ते॒ य॒त्रा॒कू॒ताय॑ । य॒त्रा॒कू॒ताय॑ मनुष्यर॒थः । य॒त्रा॒कू॒तायेति॑ यत्र - आ॒कू॒ताय॑ । म॒नु॒ष्य॒र॒थ ए॒षः । म॒नु॒ष्य॒र॒थ इति॑ मनुष्य - र॒थः । ए॒ष खलु॑ । खलु॒ वै । वै दे॑वर॒थः । दे॒व॒र॒थो यत् । दे॒व॒र॒थ इति॑ देव - र॒थः । यद॒ग्निः । अ॒ग्निर॒ग्निम् । अ॒ग्निम् ॅयु॑नज्मि । यु॒न॒ज्मि॒ शव॑सा । शव॑सा घृ॒तेन॑ । घृ॒तेनेति॑ । इत्या॑ह । आ॒ह॒ यु॒नक्ति॑ । यु॒नक्त्ये॒व । ए॒वैन᳚म् । ए॒नꣳ॒॒ सः । स ए॑नम् । ए॒न॒म् ॅयु॒क्तः । यु॒क्तः सु॑व॒र्गम् । सु॒व॒र्गम् ॅलो॒कम् । सु॒व॒र्गमिति॑ सुवः - गम् । लो॒कम॒भि । अ॒भि व॑हति । व॒ह॒ति॒ यत् । यथ् सर्वा॑भिः । सर्वा॑भिः प॒ञ्चभिः॑ । प॒ञ्चभि॑र् यु॒ञ्ज्यात् । प॒ञ्चभि॒रिति॑ प॒ञ्च - भिः॒ । यु॒ञ्ज्याद् यु॒क्तः । यु॒क्तो᳚ऽस्य । अ॒स्या॒ग्निः । अ॒ग्निः प्रच्यु॑तः । प्रच्यु॑तः स्यात् । प्रच्यु॑त॒ इति॒ प्र - च्यु॒तः॒ । स्या॒दप्र॑तिष्ठिताः । अप्र॑तिष्ठिता॒ आहु॑तयः । अप्र॑तिष्ठिता॒ इत्यप्र॑ति - स्थि॒ताः॒ । आहु॑तयः॒ स्युः । आहु॑तय॒ इत्या - हु॒त॒यः॒ । स्युरप्र॑तिष्ठिताः । अप्र॑तिष्ठिताः॒ स्तोमाः᳚ । अप्र॑तिष्ठता॒ इत्यप्र॑ति - स्थि॒ताः॒ । स्तोमा॒ अप्र॑तिष्ठितानि । अप्र॑तिष्ठितान्यु॒क्थानि॑ । अप्र॑तिष्ठिता॒नीत्यप्र॑ति - स्थि॒ता॒नि॒ । उ॒क्थानि॑ ति॒सृभिः॑ । ति॒सृभिः॑ प्रातस्सव॒ने । ति॒सृभि॒रिति॑ ति॒सृ - भिः॒ । प्रा॒त॒स्स॒व॒ने॑ऽभि । प्रा॒त॒स्स॒व॒न इति॑ प्रातः - स॒व॒ने । अ॒भि मृ॑शति । मृ॒श॒ति॒ त्रि॒वृत् । त्रि॒वृद् वै । त्रि॒वृदिति॑ त्रि - वृत् \newline

\textbf{Jatai Paata} \newline

1. सु॒व॒र्गाय॒ वै वै सु॑व॒र्गाय॑ सुव॒र्गाय॒ वै । \newline
2. सु॒व॒र्गायेति॑ सुवः - गाय॑ । \newline
3. वै लो॒काय॑ लो॒काय॒ वै वै लो॒काय॑ । \newline
4. लो॒काय॑ देवर॒थो दे॑वर॒थो लो॒काय॑ लो॒काय॑ देवर॒थः । \newline
5. दे॒व॒र॒थो यु॑ज्यते युज्यते देवर॒थो दे॑वर॒थो यु॑ज्यते । \newline
6. दे॒व॒र॒थ इति॑ देव - र॒थः । \newline
7. यु॒ज्य॒ते॒ य॒त्रा॒कू॒ताय॑ यत्राकू॒ताय॑ युज्यते युज्यते यत्राकू॒ताय॑ । \newline
8. य॒त्रा॒कू॒ताय॑ मनुष्यर॒थो म॑नुष्यर॒थो य॑त्राकू॒ताय॑ यत्राकू॒ताय॑ मनुष्यर॒थः । \newline
9. य॒त्रा॒कू॒तायेति॑ यत्र - आ॒कू॒ताय॑ । \newline
10. म॒नु॒ष्य॒र॒थ ए॒ष ए॒ष म॑नुष्यर॒थो म॑नुष्यर॒थ ए॒षः । \newline
11. म॒नु॒ष्य॒र॒थ इति॑ मनुष्य - र॒थः । \newline
12. ए॒ष खलु॒ खल्वे॒ष ए॒ष खलु॑ । \newline
13. खलु॒ वै वै खलु॒ खलु॒ वै । \newline
14. वै दे॑वर॒थो दे॑वर॒थो वै वै दे॑वर॒थः । \newline
15. दे॒व॒र॒थो यद् यद् दे॑वर॒थो दे॑वर॒थो यत् । \newline
16. दे॒व॒र॒थ इति॑ देव - र॒थः । \newline
17. यद॒ग्नि र॒ग्निर् यद् यद॒ग्निः । \newline
18. अ॒ग्नि र॒ग्नि म॒ग्नि म॒ग्नि र॒ग्नि र॒ग्निम् । \newline
19. अ॒ग्निं ॅयु॑नज्मि युनज्म्य॒ग्नि म॒ग्निं ॅयु॑नज्मि । \newline
20. यु॒न॒ज्मि॒ शव॑सा॒ शव॑सा युनज्मि युनज्मि॒ शव॑सा । \newline
21. शव॑सा घृ॒तेन॑ घृ॒तेन॒ शव॑सा॒ शव॑सा घृ॒तेन॑ । \newline
22. घृ॒तेने तीति॑ घृ॒तेन॑ घृ॒तेनेति॑ । \newline
23. इत्या॑हा॒हे तीत्या॑ह । \newline
24. आ॒ह॒ यु॒नक्ति॑ यु॒नक् त्या॑हाह यु॒नक्ति॑ । \newline
25. यु॒नक्त्ये॒ वैव यु॒नक्ति॑ यु॒नक् त्ये॒व । \newline
26. ए॒वैन॑ मेन मे॒वै वैन᳚म् । \newline
27. ए॒नꣳ॒॒ स स ए॑न मेनꣳ॒॒ सः । \newline
28. स ए॑न मेनꣳ॒॒ स स ए॑नम् । \newline
29. ए॒नं॒ ॅयु॒क्तो यु॒क्त ए॑न मेनं ॅयु॒क्तः । \newline
30. यु॒क्तः सु॑व॒र्गꣳ सु॑व॒र्गं ॅयु॒क्तो यु॒क्तः सु॑व॒र्गम् । \newline
31. सु॒व॒र्गम् ॅलो॒कम् ॅलो॒कꣳ सु॑व॒र्गꣳ सु॑व॒र्गम् ॅलो॒कम् । \newline
32. सु॒व॒र्गमिति॑ सुवः - गम् । \newline
33. लो॒क म॒भ्य॑भि लो॒कम् ॅलो॒क म॒भि । \newline
34. अ॒भि व॑हति वह त्य॒भ्य॑भि व॑हति । \newline
35. व॒ह॒ति॒ यद् यद् व॑हति वहति॒ यत् । \newline
36. यथ् सर्वा॑भिः॒ सर्वा॑भि॒र् यद् यथ् सर्वा॑भिः । \newline
37. सर्वा॑भिः प॒ञ्चभिः॑ प॒ञ्चभिः॒ सर्वा॑भिः॒ सर्वा॑भिः प॒ञ्चभिः॑ । \newline
38. प॒ञ्चभि॑र् यु॒ञ्ज्याद् यु॒ञ्ज्यात् प॒ञ्चभिः॑ प॒ञ्चभि॑र् यु॒ञ्ज्यात् । \newline
39. प॒ञ्चभि॒रिति॑ प॒ञ्च - भिः॒ । \newline
40. यु॒ञ्ज्याद् यु॒क्तो यु॒क्तो यु॒ञ्ज्याद् यु॒ञ्ज्याद् यु॒क्तः । \newline
41. यु॒क्तो᳚ ऽस्यास्य यु॒क्तो यु॒क्तो᳚ ऽस्य । \newline
42. अ॒स्या॒ग्नि र॒ग्नि र॑स्या स्या॒ग्निः । \newline
43. अ॒ग्निः प्रच्यु॑तः॒ प्रच्यु॑तो॒ ऽग्नि र॒ग्निः प्रच्यु॑तः । \newline
44. प्रच्यु॑तः स्याथ् स्या॒त् प्रच्यु॑तः॒ प्रच्यु॑तः स्यात् । \newline
45. प्रच्यु॑त॒ इति॒ प्र - च्यु॒तः॒ । \newline
46. स्या॒ दप्र॑तिष्ठिता॒ अप्र॑तिष्ठिताः स्याथ् स्या॒ दप्र॑तिष्ठिताः । \newline
47. अप्र॑तिष्ठिता॒ आहु॑तय॒ आहु॑त॒यो ऽप्र॑तिष्ठिता॒ अप्र॑तिष्ठिता॒ आहु॑तयः । \newline
48. अप्र॑तिष्ठिता॒ इत्यप्र॑ति - स्थि॒ताः॒ । \newline
49. आहु॑तयः॒ स्युः स्यु राहु॑तय॒ आहु॑तयः॒ स्युः । \newline
50. आहु॑तय॒ इत्या - हु॒त॒यः॒ । \newline
51. स्यु रप्र॑तिष्ठिता॒ अप्र॑तिष्ठिताः॒ स्युः स्यु रप्र॑तिष्ठिताः । \newline
52. अप्र॑तिष्ठिताः॒ स्तोमाः॒ स्तोमा॒ अप्र॑तिष्ठिता॒ अप्र॑तिष्ठिताः॒ स्तोमाः᳚ । \newline
53. अप्र॑तिष्ठिता॒ इत्यप्र॑ति - स्थि॒ताः॒ । \newline
54. स्तोमा॒ अप्र॑तिष्ठिता॒न्य प्र॑तिष्ठितानि॒ स्तोमाः॒ स्तोमा॒ अप्र॑तिष्ठितानि । \newline
55. अप्र॑तिष्ठिता न्यु॒क्था न्यु॒क्था न्यप्र॑तिष्ठिता॒ न्यप्र॑तिष्ठिता न्यु॒क्थानि॑ । \newline
56. अप्र॑तिष्ठिता॒नीत्यप्र॑ति - स्थि॒ता॒नि॒ । \newline
57. उ॒क्थानि॑ ति॒सृभि॑ स्ति॒सृभि॑ रु॒क्था न्यु॒क्थानि॑ ति॒सृभिः॑ । \newline
58. ति॒सृभिः॑ प्रातस्सव॒ने प्रा॑तस्सव॒ने ति॒सृभि॑ स्ति॒सृभिः॑ प्रातस्सव॒ने । \newline
59. ति॒सृभि॒रिति॑ ति॒सृ - भिः॒ । \newline
60. प्रा॒त॒स्स॒व॒ने᳚(1॒) ऽभ्य॑भि प्रा॑तस्सव॒ने प्रा॑तस्सव॒ने॑ ऽभि । \newline
61. प्रा॒त॒स्स॒व॒न इति॑ प्रातः - स॒व॒ने । \newline
62. अ॒भि मृ॑शति मृश त्य॒भ्य॑भि मृ॑शति । \newline
63. मृ॒श॒ति॒ त्रि॒वृत् त्रि॒वृन् मृ॑शति मृशति त्रि॒वृत् । \newline
64. त्रि॒वृद् वै वै त्रि॒वृत् त्रि॒वृद् वै । \newline
65. त्रि॒वृदिति॑ त्रि - वृत् । \newline

\textbf{Ghana Paata } \newline

1. सु॒व॒र्गाय॒ वै वै सु॑व॒र्गाय॑ सुव॒र्गाय॒ वै लो॒काय॑ लो॒काय॒ वै सु॑व॒र्गाय॑ सुव॒र्गाय॒ वै लो॒काय॑ । \newline
2. सु॒व॒र्गायेति॑ सुवः - गाय॑ । \newline
3. वै लो॒काय॑ लो॒काय॒ वै वै लो॒काय॑ देवर॒थो दे॑वर॒थो लो॒काय॒ वै वै लो॒काय॑ देवर॒थः । \newline
4. लो॒काय॑ देवर॒थो दे॑वर॒थो लो॒काय॑ लो॒काय॑ देवर॒थो यु॑ज्यते युज्यते देवर॒थो लो॒काय॑ लो॒काय॑ देवर॒थो यु॑ज्यते । \newline
5. दे॒व॒र॒थो यु॑ज्यते युज्यते देवर॒थो दे॑वर॒थो यु॑ज्यते यत्राकू॒ताय॑ यत्राकू॒ताय॑ युज्यते देवर॒थो दे॑वर॒थो यु॑ज्यते यत्राकू॒ताय॑ । \newline
6. दे॒व॒र॒थ इति॑ देव - र॒थः । \newline
7. यु॒ज्य॒ते॒ य॒त्रा॒कू॒ताय॑ यत्राकू॒ताय॑ युज्यते युज्यते यत्राकू॒ताय॑ मनुष्यर॒थो म॑नुष्यर॒थो य॑त्राकू॒ताय॑ युज्यते युज्यते यत्राकू॒ताय॑ मनुष्यर॒थः । \newline
8. य॒त्रा॒कू॒ताय॑ मनुष्यर॒थो म॑नुष्यर॒थो य॑त्राकू॒ताय॑ यत्राकू॒ताय॑ मनुष्यर॒थ ए॒ष ए॒ष म॑नुष्यर॒थो य॑त्राकू॒ताय॑ यत्राकू॒ताय॑ मनुष्यर॒थ ए॒षः । \newline
9. य॒त्रा॒कू॒तायेति॑ यत्र - आ॒कू॒ताय॑ । \newline
10. म॒नु॒ष्य॒र॒थ ए॒ष ए॒ष म॑नुष्यर॒थो म॑नुष्यर॒थ ए॒ष खलु॒ खल्वे॒ष म॑नुष्यर॒थो म॑नुष्यर॒थ ए॒ष खलु॑ । \newline
11. म॒नु॒ष्य॒र॒थ इति॑ मनुष्य - र॒थः । \newline
12. ए॒ष खलु॒ खल्वे॒ष ए॒ष खलु॒ वै वै खल्वे॒ष ए॒ष खलु॒ वै । \newline
13. खलु॒ वै वै खलु॒ खलु॒ वै दे॑वर॒थो दे॑वर॒थो वै खलु॒ खलु॒ वै दे॑वर॒थः । \newline
14. वै दे॑वर॒थो दे॑वर॒थो वै वै दे॑वर॒थो यद् यद् दे॑वर॒थो वै वै दे॑वर॒थो यत् । \newline
15. दे॒व॒र॒थो यद् यद् दे॑वर॒थो दे॑वर॒थो यद॒ग्नि र॒ग्निर् यद् दे॑वर॒थो दे॑वर॒थो यद॒ग्निः । \newline
16. दे॒व॒र॒थ इति॑ देव - र॒थः । \newline
17. यद॒ग्नि र॒ग्निर् यद् यद॒ग्नि र॒ग्नि म॒ग्नि म॒ग्निर् यद् यद॒ग्नि र॒ग्निम् । \newline
18. अ॒ग्नि र॒ग्नि म॒ग्नि म॒ग्नि र॒ग्नि र॒ग्निं ॅयु॑नज्मि युनज्म्य॒ग्नि म॒ग्नि र॒ग्नि र॒ग्निं ॅयु॑नज्मि । \newline
19. अ॒ग्निं ॅयु॑नज्मि युनज्म्य॒ग्नि म॒ग्निं ॅयु॑नज्मि॒ शव॑सा॒ शव॑सा युनज्म्य॒ग्नि म॒ग्निं ॅयु॑नज्मि॒ शव॑सा । \newline
20. यु॒न॒ज्मि॒ शव॑सा॒ शव॑सा युनज्मि युनज्मि॒ शव॑सा घृ॒तेन॑ घृ॒तेन॒ शव॑सा युनज्मि युनज्मि॒ शव॑सा घृ॒तेन॑ । \newline
21. शव॑सा घृ॒तेन॑ घृ॒तेन॒ शव॑सा॒ शव॑सा घृ॒तेने तीति॑ घृ॒तेन॒ शव॑सा॒ शव॑सा घृ॒तेनेति॑ । \newline
22. घृ॒तेने तीति॑ घृ॒तेन॑ घृ॒तेने त्या॑हा॒हेति॑ घृ॒तेन॑ घृ॒तेने त्या॑ह । \newline
23. इत्या॑हा॒हे तीत्या॑ह यु॒नक्ति॑ यु॒नक्त्या॒हे तीत्या॑ह यु॒नक्ति॑ । \newline
24. आ॒ह॒ यु॒नक्ति॑ यु॒नक्त्या॑ हाह यु॒नक्त्ये॒ वैव यु॒नक्त्या॑हाह यु॒नक्त्ये॒व । \newline
25. यु॒नक्त्ये॒ वैव यु॒नक्ति॑ यु॒नक्त्ये॒ वैन॑ मेन मे॒व यु॒नक्ति॑ यु॒नक्त्ये॒ वैन᳚म् । \newline
26. ए॒वैन॑ मेन मे॒वै वैनꣳ॒॒ स स ए॑न मे॒वै वैनꣳ॒॒ सः । \newline
27. ए॒नꣳ॒॒ स स ए॑न मेनꣳ॒॒ स ए॑न मेनꣳ॒॒ स ए॑न मेनꣳ॒॒ स ए॑नम् । \newline
28. स ए॑न मेनꣳ॒॒ स स ए॑नं ॅयु॒क्तो यु॒क्त ए॑नꣳ॒॒ स स ए॑नं ॅयु॒क्तः । \newline
29. ए॒नं॒ ॅयु॒क्तो यु॒क्त ए॑न मेनं ॅयु॒क्तः सु॑व॒र्गꣳ सु॑व॒र्गं ॅयु॒क्त ए॑न मेनं ॅयु॒क्तः सु॑व॒र्गम् । \newline
30. यु॒क्तः सु॑व॒र्गꣳ सु॑व॒र्गं ॅयु॒क्तो यु॒क्तः सु॑व॒र्गम् ॅलो॒कम् ॅलो॒कꣳ सु॑व॒र्गं ॅयु॒क्तो यु॒क्तः सु॑व॒र्गम् ॅलो॒कम् । \newline
31. सु॒व॒र्गम् ॅलो॒कम् ॅलो॒कꣳ सु॑व॒र्गꣳ सु॑व॒र्गम् ॅलो॒क म॒भ्य॑भि लो॒कꣳ सु॑व॒र्गꣳ सु॑व॒र्गम् ॅलो॒क म॒भि । \newline
32. सु॒व॒र्गमिति॑ सुवः - गम् । \newline
33. लो॒क म॒भ्य॑भि लो॒कम् ॅलो॒क म॒भि व॑हति वह त्य॒भि लो॒कम् ॅलो॒क म॒भि व॑हति । \newline
34. अ॒भि व॑हति वह त्य॒भ्य॑भि व॑हति॒ यद् यद् व॑ह त्य॒भ्य॑भि व॑हति॒ यत् । \newline
35. व॒ह॒ति॒ यद् यद् व॑हति वहति॒ यथ् सर्वा॑भिः॒ सर्वा॑भि॒र् यद् व॑हति वहति॒ यथ् सर्वा॑भिः । \newline
36. यथ् सर्वा॑भिः॒ सर्वा॑भि॒र् यद् यथ् सर्वा॑भिः प॒ञ्चभिः॑ प॒ञ्चभिः॒ सर्वा॑भि॒र् यद् यथ् सर्वा॑भिः प॒ञ्चभिः॑ । \newline
37. सर्वा॑भिः प॒ञ्चभिः॑ प॒ञ्चभिः॒ सर्वा॑भिः॒ सर्वा॑भिः प॒ञ्चभि॑र् यु॒ञ्ज्याद् यु॒ञ्ज्यात् प॒ञ्चभिः॒ सर्वा॑भिः॒ सर्वा॑भिः प॒ञ्चभि॑र् यु॒ञ्ज्यात् । \newline
38. प॒ञ्चभि॑र् यु॒ञ्ज्याद् यु॒ञ्ज्यात् प॒ञ्चभिः॑ प॒ञ्चभि॑र् यु॒ञ्ज्याद् यु॒क्तो यु॒क्तो यु॒ञ्ज्यात् प॒ञ्चभिः॑ प॒ञ्चभि॑र् यु॒ञ्ज्याद् यु॒क्तः । \newline
39. प॒ञ्चभि॒रिति॑ प॒ञ्च - भिः॒ । \newline
40. यु॒ञ्ज्याद् यु॒क्तो यु॒क्तो यु॒ञ्ज्याद् यु॒ञ्ज्याद् यु॒क्तो᳚ ऽस्यास्य यु॒क्तो यु॒ञ्ज्याद् यु॒ञ्ज्याद् यु॒क्तो᳚ ऽस्य । \newline
41. यु॒क्तो᳚ ऽस्यास्य यु॒क्तो यु॒क्तो᳚ ऽस्या॒ग्नि र॒ग्नि र॑स्य यु॒क्तो यु॒क्तो᳚ ऽस्या॒ग्निः । \newline
42. अ॒स्या॒ग्नि र॒ग्नि र॑स्यास्या॒ग्निः प्रच्यु॑तः॒ प्रच्यु॑तो॒ ऽग्नि र॑स्या स्या॒ग्निः प्रच्यु॑तः । \newline
43. अ॒ग्निः प्रच्यु॑तः॒ प्रच्यु॑तो॒ ऽग्नि र॒ग्निः प्रच्यु॑तः स्याथ् स्या॒त् प्रच्यु॑तो॒ ऽग्नि र॒ग्निः प्रच्यु॑तः स्यात् । \newline
44. प्रच्यु॑तः स्याथ् स्या॒त् प्रच्यु॑तः॒ प्रच्यु॑तः स्या॒ दप्र॑तिष्ठिता॒ अप्र॑तिष्ठिताः स्या॒त् प्रच्यु॑तः॒ प्रच्यु॑तः स्या॒ दप्र॑तिष्ठिताः । \newline
45. प्रच्यु॑त॒ इति॒ प्र - च्यु॒तः॒ । \newline
46. स्या॒ दप्र॑तिष्ठिता॒ अप्र॑तिष्ठिताः स्याथ् स्या॒ दप्र॑तिष्ठिता॒ आहु॑तय॒ आहु॑त॒यो ऽप्र॑तिष्ठिताः स्याथ् स्या॒ दप्र॑तिष्ठिता॒ आहु॑तयः । \newline
47. अप्र॑तिष्ठिता॒ आहु॑तय॒ आहु॑त॒यो ऽप्र॑तिष्ठिता॒ अप्र॑तिष्ठिता॒ आहु॑तयः॒ स्युः स्यु राहु॑त॒यो ऽप्र॑तिष्ठिता॒ अप्र॑तिष्ठिता॒ आहु॑तयः॒ स्युः । \newline
48. अप्र॑तिष्ठिता॒ इत्यप्र॑ति - स्थि॒ताः॒ । \newline
49. आहु॑तयः॒ स्युः स्यु राहु॑तय॒ आहु॑तयः॒ स्यु रप्र॑तिष्ठिता॒ अप्र॑तिष्ठिताः॒ स्यु राहु॑तय॒ आहु॑तयः॒ स्यु रप्र॑तिष्ठिताः । \newline
50. आहु॑तय॒ इत्या - हु॒त॒यः॒ । \newline
51. स्यु रप्र॑तिष्ठिता॒ अप्र॑तिष्ठिताः॒ स्युः स्यु रप्र॑तिष्ठिताः॒ स्तोमाः॒ स्तोमा॒ अप्र॑तिष्ठिताः॒ स्युः स्यु रप्र॑तिष्ठिताः॒ स्तोमाः᳚ । \newline
52. अप्र॑तिष्ठिताः॒ स्तोमाः॒ स्तोमा॒ अप्र॑तिष्ठिता॒ अप्र॑तिष्ठिताः॒ स्तोमा॒ अप्र॑तिष्ठिता॒ न्यप्र॑तिष्ठितानि॒ स्तोमा॒ अप्र॑तिष्ठिता॒ अप्र॑तिष्ठिताः॒ स्तोमा॒ अप्र॑तिष्ठितानि । \newline
53. अप्र॑तिष्ठिता॒ इत्यप्र॑ति - स्थि॒ताः॒ । \newline
54. स्तोमा॒ अप्र॑तिष्ठिता॒ न्यप्र॑तिष्ठितानि॒ स्तोमाः॒ स्तोमा॒ अप्र॑तिष्ठिता न्यु॒क्था न्यु॒क्था न्यप्र॑तिष्ठितानि॒ स्तोमाः॒ स्तोमा॒ अप्र॑तिष्ठिता न्यु॒क्थानि॑ । \newline
55. अप्र॑तिष्ठिता न्यु॒क्था न्यु॒क्था न्यप्र॑तिष्ठिता॒ न्यप्र॑तिष्ठिता न्यु॒क्थानि॑ ति॒सृभि॑ स्ति॒सृभि॑ रु॒क्था न्यप्र॑तिष्ठिता॒ न्यप्र॑तिष्ठिता न्यु॒क्थानि॑ ति॒सृभिः॑ । \newline
56. अप्र॑तिष्ठिता॒नीत्यप्र॑ति - स्थि॒ता॒नि॒ । \newline
57. उ॒क्थानि॑ ति॒सृभि॑ स्ति॒सृभि॑ रु॒क्थान् यु॒क्थानि॑ ति॒सृभिः॑ प्रातस्सव॒ने प्रा॑तस्सव॒ने ति॒सृभि॑ रु॒क्था न्यु॒क्थानि॑ ति॒सृभिः॑ प्रातस्सव॒ने । \newline
58. ति॒सृभिः॑ प्रातस्सव॒ने प्रा॑तस्सव॒ने ति॒सृभि॑ स्ति॒सृभिः॑ प्रातस्सव॒ने᳚(1॒) ऽभ्य॑भि प्रा॑तस्सव॒ने ति॒सृभि॑ स्ति॒सृभिः॑ प्रातस्सव॒ने॑ ऽभि । \newline
59. ति॒सृभि॒रिति॑ ति॒सृ - भिः॒ । \newline
60. प्रा॒त॒स्स॒व॒ने᳚(1॒) ऽभ्य॑भि प्रा॑तस्सव॒ने प्रा॑तस्सव॒ने॑ ऽभि मृ॑शति मृश त्य॒भि प्रा॑तस्सव॒ने प्रा॑तस्सव॒ने॑ ऽभि मृ॑शति । \newline
61. प्रा॒त॒स्स॒व॒न इति॑ प्रातः - स॒व॒ने । \newline
62. अ॒भि मृ॑शति मृश त्य॒भ्य॑भि मृ॑शति त्रि॒वृत् त्रि॒वृन् मृ॑श त्य॒भ्य॑भि मृ॑शति त्रि॒वृत् । \newline
63. मृ॒श॒ति॒ त्रि॒वृत् त्रि॒वृन् मृ॑शति मृशति त्रि॒वृद् वै वै त्रि॒वृन् मृ॑शति मृशति त्रि॒वृद् वै । \newline
64. त्रि॒वृद् वै वै त्रि॒वृत् त्रि॒वृद् वा अ॒ग्नि र॒ग्निर् वै त्रि॒वृत् त्रि॒वृद् वा अ॒ग्निः । \newline
65. त्रि॒वृदिति॑ त्रि - वृत् । \newline
\pagebreak
\markright{ TS 5.4.10.2  \hfill https://www.vedavms.in \hfill}

\section{ TS 5.4.10.2 }

\textbf{TS 5.4.10.2 } \newline
\textbf{Samhita Paata} \newline

-द्वा अ॒ग्निर्यावा॑ने॒वा-ग्निस्तं ॅयु॑नक्ति॒ यथाऽन॑सि यु॒क्त आ॑धी॒यत॑ ए॒वमे॒व तत् प्रत्याहु॑तय॒स्तिष्ठ॑न्ति॒ प्रति॒ स्तोमाः॒ प्रत्यु॒क्थानि॑ यज्ञाय॒ज्ञिय॑स्य स्तो॒त्रे द्वाभ्या॑म॒भि मृ॑शत्ये॒तावा॒न्॒ वै य॒ज्ञो यावा॑नग्निष्टो॒मो भू॒मा त्वा अ॒स्यात॑ ऊ॒र्द्ध्वः क्रि॑यते॒ यावा॑ने॒व य॒ज्ञ्स्तम॑न्त॒तो᳚ ऽन्वारो॑हति॒ द्वाभ्यां॒ प्रति॑ष्ठित्या॒ एक॒याऽप्र॑स्तुतं॒ भव॒त्यथा॒ - [  ] \newline

\textbf{Pada Paata} \newline

वै । अ॒ग्निः । यावान्॑ । ए॒व । अ॒ग्निः । तम् । यु॒न॒क्ति॒ । यथा᳚ । अन॑सि । यु॒क्ते । आ॒धी॒यत॒ इत्या᳚ - धी॒यते᳚ । ए॒वम् । ए॒व । तत् । प्रतीति॑ । आहु॑तय॒ इत्या - हु॒त॒यः॒ । तिष्ठ॑न्ति । प्रतीति॑ । स्तोमाः᳚ । प्रतीति॑ । उ॒क्थानि॑ । य॒ज्ञा॒य॒ज्ञिय॑स्य । स्तो॒त्रे । द्वाभ्या᳚म् । अ॒भीति॑ । मृ॒श॒ति॒ । ए॒तावान्॑ । वै । य॒ज्ञ्ः । यावान्॑ । अ॒ग्नि॒ष्टो॒म इत्य॑ग्नि - स्तो॒मः । भू॒मा । तु । वै । अ॒स्य॒ । अतः॑ । ऊ॒द्‌र्ध्वः । क्रि॒य॒ते॒ । यावान्॑ । ए॒व । य॒ज्ञ्ः । तम् । अ॒न्त॒तः । अ॒न्वारो॑ह॒तीत्य॑नु - आरो॑हति । द्वाभ्या᳚म् । प्रति॑ष्ठित्या॒ इति॒ प्रति॑ - स्थि॒त्यै॒ । एक॑या । अप्र॑स्तुत॒मित्यप्र॑-स्तु॒त॒म् । भव॑ति । अथ॑ ।  \newline


\textbf{Krama Paata} \newline

वा अ॒ग्निः । अ॒ग्निर् यावान्॑ । यावा॑ने॒व । ए॒वाग्निः । अ॒ग्निस्तम् । तम् ॅयु॑नक्ति । यु॒न॒क्ति॒ यथा᳚ । यथाऽन॑सि । अन॑सि यु॒क्ते । यु॒क्त आ॑धी॒यते᳚ । आ॒धी॒यत॑ ए॒वम् । आ॒धी॒यत॒ इत्या᳚ - धी॒यते᳚ । ए॒वमे॒व । ए॒व तत् । तत् प्रति॑ । प्रत्याहु॑तयः । आहु॑तय॒स्तिष्ठ॑न्ति । आहु॑तय॒ इत्या - हु॒त॒यः॒ । तिष्ठ॑न्ति॒ प्रति॑ । प्रति॒ स्तोमाः᳚ । स्तोमाः॒ प्रति॑ । प्रत्यु॒क्थानि॑ । उ॒क्थानि॑ यज्ञाय॒ज्ञिय॑स्य । य॒ज्ञा॒य॒ज्ञिय॑स्य स्तो॒त्रे । स्तो॒त्रे द्वाभ्या᳚म् । द्वाभ्या॑म॒भि । अ॒भि मृ॑शति । मृ॒श॒त्ये॒तावान्॑ । ए॒तावा॒न्॒. वै । वै य॒ज्ञ्ः । य॒ज्ञो यावान्॑ । यावा॑नग्निष्टो॒मः । अ॒ग्नि॒ष्टो॒मो भू॒मा । अ॒ग्नि॒ष्टो॒म इत्य॑ग्नि - स्तो॒मः । भू॒मा तु । त्वै । वा अ॑स्य । अ॒स्यातः॑ । अत॑ ऊ॒र्द्ध्वः । ऊ॒र्द्ध्वः क्रि॑यते । क्रि॒य॒ते॒ यावान्॑ । यावा॑ने॒व । ए॒व य॒ज्ञ्ः । य॒ज्ञ्स्तम् । तम॑न्त॒तः । अ॒न्त॒तो᳚ऽन्वारो॑हति । अ॒न्वारो॑हति॒ द्वाभ्या᳚म् । अ॒न्वारो॑ह॒तीत्य॑नु - आरो॑हति । द्वाभ्या॒म् प्रति॑ष्ठित्यै । प्रति॑ष्ठित्या॒ एक॑या । प्रति॑ष्ठित्या॒ इति॒ प्रति॑ - स्थि॒त्यै॒ । एक॒याऽप्र॑स्तुतम् । अप्र॑स्तुत॒म् भव॑ति । अप्र॑स्तुत॒मित्यप्र॑ - स्तु॒त॒म् । भव॒त्यथ॑ । अथा॒भि \newline

\textbf{Jatai Paata} \newline

1. वा अ॒ग्नि र॒ग्निर् वै वा अ॒ग्निः । \newline
2. अ॒ग्निर् यावा॒न्॒. यावा॑ न॒ग्नि र॒ग्निर् यावान्॑ । \newline
3. यावा॑ ने॒वैव यावा॒न्॒. यावा॑ ने॒व । \newline
4. ए॒वाग्नि र॒ग्नि रे॒वै वाग्निः । \newline
5. अ॒ग्नि स्तम् त म॒ग्नि र॒ग्नि स्तम् । \newline
6. तं ॅयु॑नक्ति युनक्ति॒ तम् तं ॅयु॑नक्ति । \newline
7. यु॒न॒क्ति॒ यथा॒ यथा॑ युनक्ति युनक्ति॒ यथा᳚ । \newline
8. यथा ऽन॒ स्यन॑सि॒ यथा॒ यथा ऽन॑सि । \newline
9. अन॑सि यु॒क्ते यु॒क्ते ऽन॒ स्यन॑सि यु॒क्ते । \newline
10. यु॒क्त आ॑धी॒यत॑ आधी॒यते॑ यु॒क्ते यु॒क्त आ॑धी॒यते᳚ । \newline
11. आ॒धी॒यत॑ ए॒व मे॒व मा॑धी॒यत॑ आधी॒यत॑ ए॒वम् । \newline
12. आ॒धी॒यत॒ इत्या᳚ - धी॒यते᳚ । \newline
13. ए॒व मे॒वै वैव मे॒व मे॒व । \newline
14. ए॒व तत् तदे॒ वैव तत् । \newline
15. तत् प्रति॒ प्रति॒ तत् तत् प्रति॑ । \newline
16. प्रत्याहु॑तय॒ आहु॑तयः॒ प्रति॒ प्रत्याहु॑तयः । \newline
17. आहु॑तय॒ स्तिष्ठ॑न्ति॒ तिष्ठ॒न् त्याहु॑तय॒ आहु॑तय॒ स्तिष्ठ॑न्ति । \newline
18. आहु॑तय॒ इत्या - हु॒त॒यः॒ । \newline
19. तिष्ठ॑न्ति॒ प्रति॒ प्रति॒ तिष्ठ॑न्ति॒ तिष्ठ॑न्ति॒ प्रति॑ । \newline
20. प्रति॒ स्तोमाः॒ स्तोमाः॒ प्रति॒ प्रति॒ स्तोमाः᳚ । \newline
21. स्तोमाः॒ प्रति॒ प्रति॒ स्तोमाः॒ स्तोमाः॒ प्रति॑ । \newline
22. प्रत्यु॒क्था न्यु॒क्थानि॒ प्रति॒ प्रत्यु॒क्थानि॑ । \newline
23. उ॒क्थानि॑ यज्ञाय॒ज्ञिय॑स्य यज्ञाय॒ज्ञिय॑ स्यो॒क्था न्यु॒क्थानि॑ यज्ञाय॒ज्ञिय॑स्य । \newline
24. य॒ज्ञा॒य॒ज्ञिय॑स्य स्तो॒त्रे स्तो॒त्रे य॑ज्ञाय॒ज्ञिय॑स्य यज्ञाय॒ज्ञिय॑स्य स्तो॒त्रे । \newline
25. स्तो॒त्रे द्वाभ्या॒म् द्वाभ्याꣳ॑ स्तो॒त्रे स्तो॒त्रे द्वाभ्या᳚म् । \newline
26. द्वाभ्या॑ म॒भ्य॑भि द्वाभ्या॒म् द्वाभ्या॑ म॒भि । \newline
27. अ॒भि मृ॑शति मृश त्य॒भ्य॑भि मृ॑शति । \newline
28. मृ॒श॒ त्ये॒तावा॑ ने॒तावा᳚न् मृशति मृश त्ये॒तावान्॑ । \newline
29. ए॒तावा॒न्॒. वै वा ए॒तावा॑ ने॒तावा॒न्॒. वै । \newline
30. वै य॒ज्ञो य॒ज्ञो वै वै य॒ज्ञ्ः । \newline
31. य॒ज्ञो यावा॒न्॒. यावान्॑. य॒ज्ञो य॒ज्ञो यावान्॑ । \newline
32. यावा॑ नग्निष्टो॒मो᳚ ऽग्निष्टो॒मो यावा॒न्॒. यावा॑ नग्निष्टो॒मः । \newline
33. अ॒ग्नि॒ष्टो॒मो भू॒मा भू॒मा ऽग्नि॑ष्टो॒मो᳚ ऽग्निष्टो॒मो भू॒मा । \newline
34. अ॒ग्नि॒ष्टो॒म इत्य॑ग्नि - स्तो॒मः । \newline
35. भू॒मा तु तु भू॒मा भू॒मा तु । \newline
36. त्वै वै तु त्वै । \newline
37. वा अ॑स्यास्य॒ वै वा अ॑स्य । \newline
38. अ॒स्यातो ऽतो᳚ ऽस्या॒ स्यातः॑ । \newline
39. अत॑ ऊ॒र्द्ध्व ऊ॒र्द्ध्वो ऽतो ऽत॑ ऊ॒र्द्ध्वः । \newline
40. ऊ॒र्द्ध्वः क्रि॑यते क्रियत ऊ॒र्द्ध्व ऊ॒र्द्ध्वः क्रि॑यते । \newline
41. क्रि॒य॒ते॒ यावा॒न्॒. यावा᳚न् क्रियते क्रियते॒ यावान्॑ । \newline
42. यावा॑ ने॒वैव यावा॒न्॒. यावा॑ ने॒व । \newline
43. ए॒व य॒ज्ञो य॒ज्ञ् ए॒वैव य॒ज्ञ्ः । \newline
44. य॒ज्ञ् स्तम् तं ॅय॒ज्ञो य॒ज्ञ् स्तम् । \newline
45. त म॑न्त॒तो᳚ ऽन्त॒त स्तम् त म॑न्त॒तः । \newline
46. अ॒न्त॒तो᳚ ऽन्वारो॑ह त्य॒न्वारो॑ह त्यन्त॒तो᳚ ऽन्त॒तो᳚ ऽन्वारो॑हति । \newline
47. अ॒न्वारो॑हति॒ द्वाभ्या॒म् द्वाभ्या॑ म॒न्वारो॑ह त्य॒न्वारो॑हति॒ द्वाभ्या᳚म् । \newline
48. अ॒न्वारो॑ह॒तीत्य॑नु - आरो॑हति । \newline
49. द्वाभ्या॒म् प्रति॑ष्ठित्यै॒ प्रति॑ष्ठित्यै॒ द्वाभ्या॒म् द्वाभ्या॒म् प्रति॑ष्ठित्यै । \newline
50. प्रति॑ष्ठित्या॒ एक॒यैक॑या॒ प्रति॑ष्ठित्यै॒ प्रति॑ष्ठित्या॒ एक॑या । \newline
51. प्रति॑ष्ठित्या॒ इति॒ प्रति॑ - स्थि॒त्यै॒ । \newline
52. एक॒या ऽप्र॑स्तुत॒ मप्र॑स्तुत॒ मेक॒ यैक॒या ऽप्र॑स्तुतम् । \newline
53. अप्र॑स्तुत॒म् भव॑ति॒ भव॒त्य प्र॑स्तुत॒ मप्र॑स्तुत॒म् भव॑ति । \newline
54. अप्र॑स्तुत॒मित्यप्र॑ - स्तु॒त॒म् । \newline
55. भव॒त्य थाथ॒ भव॑ति॒ भव॒ त्यथ॑ । \newline
56. अथा॒ भ्य॑भ्य थाथा॒भि । \newline

\textbf{Ghana Paata } \newline

1. वा अ॒ग्नि र॒ग्निर् वै वा अ॒ग्निर् यावा॒न्॒. यावा॑ न॒ग्निर् वै वा अ॒ग्निर् यावान्॑ । \newline
2. अ॒ग्निर् यावा॒न्॒. यावा॑ न॒ग्नि र॒ग्निर् यावा॑ ने॒वैव यावा॑ न॒ग्नि र॒ग्निर् यावा॑ ने॒व । \newline
3. यावा॑ ने॒वैव यावा॒न्॒. यावा॑ ने॒वाग्नि र॒ग्नि रे॒व यावा॒न्॒. यावा॑ ने॒वाग्निः । \newline
4. ए॒वाग्नि र॒ग्नि रे॒वै वाग्नि स्तम् त म॒ग्नि रे॒वै वाग्नि स्तम् । \newline
5. अ॒ग्नि स्तम् त म॒ग्नि र॒ग्नि स्तं ॅयु॑नक्ति युनक्ति॒ त म॒ग्नि र॒ग्नि स्तं ॅयु॑नक्ति । \newline
6. तं ॅयु॑नक्ति युनक्ति॒ तम् तं ॅयु॑नक्ति॒ यथा॒ यथा॑ युनक्ति॒ तम् तं ॅयु॑नक्ति॒ यथा᳚ । \newline
7. यु॒न॒क्ति॒ यथा॒ यथा॑ युनक्ति युनक्ति॒ यथा ऽन॒स्य न॑सि॒ यथा॑ युनक्ति युनक्ति॒ यथा ऽन॑सि । \newline
8. यथा ऽन॒स्य न॑सि॒ यथा॒ यथा ऽन॑सि यु॒क्ते यु॒क्ते ऽन॑सि॒ यथा॒ यथा ऽन॑सि यु॒क्ते । \newline
9. अन॑सि यु॒क्ते यु॒क्ते ऽन॒स्य न॑सि यु॒क्त आ॑धी॒यत॑ आधी॒यते॑ यु॒क्ते ऽन॒स्य न॑सि यु॒क्त आ॑धी॒यते᳚ । \newline
10. यु॒क्त आ॑धी॒यत॑ आधी॒यते॑ यु॒क्ते यु॒क्त आ॑धी॒यत॑ ए॒व मे॒व मा॑धी॒यते॑ यु॒क्ते यु॒क्त आ॑धी॒यत॑ ए॒वम् । \newline
11. आ॒धी॒यत॑ ए॒व मे॒व मा॑धी॒यत॑ आधी॒यत॑ ए॒व मे॒वै वैव मा॑धी॒यत॑ आधी॒यत॑ ए॒व मे॒व । \newline
12. आ॒धी॒यत॒ इत्या᳚ - धी॒यते᳚ । \newline
13. ए॒व मे॒वै वैव मे॒व मे॒व तत् तदे॒ वैव मे॒व मे॒व तत् । \newline
14. ए॒व तत् तदे॒ वैव तत् प्रति॒ प्रति॒ तदे॒ वैव तत् प्रति॑ । \newline
15. तत् प्रति॒ प्रति॒ तत् तत् प्रत्याहु॑तय॒ आहु॑तयः॒ प्रति॒ तत् तत् प्रत्याहु॑तयः । \newline
16. प्रत्याहु॑तय॒ आहु॑तयः॒ प्रति॒ प्रत्याहु॑तय॒ स्तिष्ठ॑न्ति॒ तिष्ठ॒न् त्याहु॑तयः॒ प्रति॒ प्रत्याहु॑तय॒ स्तिष्ठ॑न्ति । \newline
17. आहु॑तय॒ स्तिष्ठ॑न्ति॒ तिष्ठ॒न् त्याहु॑तय॒ आहु॑तय॒ स्तिष्ठ॑न्ति॒ प्रति॒ प्रति॒ तिष्ठ॒न् त्याहु॑तय॒ आहु॑तय॒ स्तिष्ठ॑न्ति॒ प्रति॑ । \newline
18. आहु॑तय॒ इत्या - हु॒त॒यः॒ । \newline
19. तिष्ठ॑न्ति॒ प्रति॒ प्रति॒ तिष्ठ॑न्ति॒ तिष्ठ॑न्ति॒ प्रति॒ स्तोमाः॒ स्तोमाः॒ प्रति॒ तिष्ठ॑न्ति॒ तिष्ठ॑न्ति॒ प्रति॒ स्तोमाः᳚ । \newline
20. प्रति॒ स्तोमाः॒ स्तोमाः॒ प्रति॒ प्रति॒ स्तोमाः॒ प्रति॒ प्रति॒ स्तोमाः॒ प्रति॒ प्रति॒ स्तोमाः॒ प्रति॑ । \newline
21. स्तोमाः॒ प्रति॒ प्रति॒ स्तोमाः॒ स्तोमाः॒ प्रत्यु॒क्था न्यु॒क्थानि॒ प्रति॒ स्तोमाः॒ स्तोमाः॒ प्रत्यु॒क्थानि॑ । \newline
22. प्रत्यु॒क्था न्यु॒क्थानि॒ प्रति॒ प्रत्यु॒क्थानि॑ यज्ञाय॒ज्ञिय॑स्य यज्ञाय॒ज्ञिय॑ स्यो॒क्थानि॒ प्रति॒ प्रत्यु॒क्थानि॑ यज्ञाय॒ज्ञिय॑स्य । \newline
23. उ॒क्थानि॑ यज्ञाय॒ज्ञिय॑स्य यज्ञाय॒ज्ञिय॑ स्यो॒क्था न्यु॒क्थानि॑ यज्ञाय॒ज्ञिय॑स्य स्तो॒त्रे स्तो॒त्रे य॑ज्ञाय॒ज्ञिय॑ स्यो॒क्था न्यु॒क्थानि॑ यज्ञाय॒ज्ञिय॑स्य स्तो॒त्रे । \newline
24. य॒ज्ञा॒य॒ज्ञिय॑स्य स्तो॒त्रे स्तो॒त्रे य॑ज्ञाय॒ज्ञिय॑स्य यज्ञाय॒ज्ञिय॑स्य स्तो॒त्रे द्वाभ्या॒म् द्वाभ्याꣳ॑ स्तो॒त्रे य॑ज्ञाय॒ज्ञिय॑स्य यज्ञाय॒ज्ञिय॑स्य स्तो॒त्रे द्वाभ्या᳚म् । \newline
25. स्तो॒त्रे द्वाभ्या॒म् द्वाभ्याꣳ॑ स्तो॒त्रे स्तो॒त्रे द्वाभ्या॑ म॒भ्य॑भि द्वाभ्याꣳ॑ स्तो॒त्रे स्तो॒त्रे द्वाभ्या॑ म॒भि । \newline
26. द्वाभ्या॑ म॒भ्य॑भि द्वाभ्या॒म् द्वाभ्या॑ म॒भि मृ॑शति मृश त्य॒भि द्वाभ्या॒म् द्वाभ्या॑ म॒भि मृ॑शति । \newline
27. अ॒भि मृ॑शति मृश त्य॒भ्य॑भि मृ॑श त्ये॒तावा॑ ने॒तावा᳚न् मृश त्य॒भ्य॑भि मृ॑श त्ये॒तावान्॑ । \newline
28. मृ॒श॒ त्ये॒तावा॑ ने॒तावा᳚न् मृशति मृश त्ये॒तावा॒न्॒. वै वा ए॒तावा᳚न् मृशति मृश त्ये॒तावा॒न्॒. वै । \newline
29. ए॒तावा॒न्॒. वै वा ए॒तावा॑ ने॒तावा॒न्॒. वै य॒ज्ञो य॒ज्ञो वा ए॒तावा॑ ने॒तावा॒न्॒. वै य॒ज्ञ्ः । \newline
30. वै य॒ज्ञो य॒ज्ञो वै वै य॒ज्ञो यावा॒न्॒. यावान्॑. य॒ज्ञो वै वै य॒ज्ञो यावान्॑ । \newline
31. य॒ज्ञो यावा॒न्॒. यावान्॑. य॒ज्ञो य॒ज्ञो यावा॑ नग्निष्टो॒मो᳚ ऽग्निष्टो॒मो यावान्॑. य॒ज्ञो य॒ज्ञो यावा॑ नग्निष्टो॒मः । \newline
32. यावा॑ नग्निष्टो॒मो᳚ ऽग्निष्टो॒मो यावा॒न्॒. यावा॑ नग्निष्टो॒मो भू॒मा भू॒मा ऽग्नि॑ष्टो॒मो यावा॒न्॒. यावा॑ नग्निष्टो॒मो भू॒मा । \newline
33. अ॒ग्नि॒ष्टो॒मो भू॒मा भू॒मा ऽग्नि॑ष्टो॒मो᳚ ऽग्निष्टो॒मो भू॒मा तु तु भू॒मा ऽग्नि॑ष्टो॒मो᳚ ऽग्निष्टो॒मो भू॒मा तु । \newline
34. अ॒ग्नि॒ष्टो॒म इत्य॑ग्नि - स्तो॒मः । \newline
35. भू॒मा तु तु भू॒मा भू॒मा त्वै वै तु भू॒मा भू॒मा त्वै । \newline
36. त्वै वै तु त्वा अ॑स्यास्य॒ वै तु त्वा अ॑स्य । \newline
37. वा अ॑स्यास्य॒ वै वा अ॒स्यातो ऽतो᳚ ऽस्य॒ वै वा अ॒स्यातः॑ । \newline
38. अ॒स्यातो ऽतो᳚ ऽस्या॒ स्यात॑ ऊ॒र्द्ध्व ऊ॒र्द्ध्वो ऽतो᳚ ऽस्या॒ स्यात॑ ऊ॒र्द्ध्वः । \newline
39. अत॑ ऊ॒र्द्ध्व ऊ॒र्द्ध्वो ऽतो ऽत॑ ऊ॒र्द्ध्वः क्रि॑यते क्रियत ऊ॒र्द्ध्वो ऽतो ऽत॑ ऊ॒र्द्ध्वः क्रि॑यते । \newline
40. ऊ॒र्द्ध्वः क्रि॑यते क्रियत ऊ॒र्द्ध्व ऊ॒र्द्ध्वः क्रि॑यते॒ यावा॒न्॒. यावा᳚न् क्रियत ऊ॒र्द्ध्व ऊ॒र्द्ध्वः क्रि॑यते॒ यावान्॑ । \newline
41. क्रि॒य॒ते॒ यावा॒न्॒. यावा᳚न् क्रियते क्रियते॒ यावा॑ ने॒वैव यावा᳚न् क्रियते क्रियते॒ यावा॑ ने॒व । \newline
42. यावा॑ ने॒वैव यावा॒न्॒. यावा॑ ने॒व य॒ज्ञो य॒ज्ञ् ए॒व यावा॒न्॒. यावा॑ ने॒व य॒ज्ञ्ः । \newline
43. ए॒व य॒ज्ञो य॒ज्ञ् ए॒वैव य॒ज्ञ् स्तम् तं ॅय॒ज्ञ् ए॒वैव य॒ज्ञ् स्तम् । \newline
44. य॒ज्ञ् स्तम् तं ॅय॒ज्ञो य॒ज्ञ् स्त म॑न्त॒तो᳚ ऽन्त॒त स्तं ॅय॒ज्ञो य॒ज्ञ् स्त म॑न्त॒तः । \newline
45. त म॑न्त॒तो᳚ ऽन्त॒त स्तम् त म॑न्त॒तो᳚ ऽन्वारो॑हत्य॒ न्वारो॑ह त्यन्त॒त स्तम् त म॑न्त॒तो᳚ ऽन्वारो॑हति । \newline
46. अ॒न्त॒तो᳚ ऽन्वारो॑ह त्य॒न्वारो॑ह त्यन्त॒तो᳚ ऽन्त॒तो᳚ ऽन्वारो॑हति॒ द्वाभ्या॒म् द्वाभ्या॑ म॒न्वारो॑ह त्यन्त॒तो᳚ ऽन्त॒तो᳚ ऽन्वारो॑हति॒ द्वाभ्या᳚म् । \newline
47. अ॒न्वारो॑हति॒ द्वाभ्या॒म् द्वाभ्या॑ म॒न्वारो॑ह त्य॒न्वारो॑हति॒ द्वाभ्या॒म् प्रति॑ष्ठित्यै॒ प्रति॑ष्ठित्यै॒ द्वाभ्या॑ म॒न्वारो॑ह त्य॒न्वारो॑हति॒ द्वाभ्या॒म् प्रति॑ष्ठित्यै । \newline
48. अ॒न्वारो॑ह॒तीत्य॑नु - आरो॑हति । \newline
49. द्वाभ्या॒म् प्रति॑ष्ठित्यै॒ प्रति॑ष्ठित्यै॒ द्वाभ्या॒म् द्वाभ्या॒म् प्रति॑ष्ठित्या॒ एक॒ यैक॑या॒ प्रति॑ष्ठित्यै॒ द्वाभ्या॒म् द्वाभ्या॒म् प्रति॑ष्ठित्या॒ एक॑या । \newline
50. प्रति॑ष्ठित्या॒ एक॒ यैक॑या॒ प्रति॑ष्ठित्यै॒ प्रति॑ष्ठित्या॒ एक॒या ऽप्र॑स्तुत॒ मप्र॑स्तुत॒ मेक॑या॒ प्रति॑ष्ठित्यै॒ प्रति॑ष्ठित्या॒ एक॒या ऽप्र॑स्तुतम् । \newline
51. प्रति॑ष्ठित्या॒ इति॒ प्रति॑ - स्थि॒त्यै॒ । \newline
52. एक॒या ऽप्र॑स्तुत॒ मप्र॑स्तुत॒ मेक॒ यैक॒या ऽप्र॑स्तुत॒म् भव॑ति॒ भव॒ त्यप्र॑स्तुत॒ मेक॒ यैक॒या ऽप्र॑स्तुत॒म् भव॑ति । \newline
53. अप्र॑स्तुत॒म् भव॑ति॒ भव॒ त्यप्र॑स्तुत॒ मप्र॑स्तुत॒म् भव॒ त्यथाथ॒ भव॒ त्यप्र॑स्तुत॒ मप्र॑स्तुत॒म् भव॒ त्यथ॑ । \newline
54. अप्र॑स्तुत॒मित्यप्र॑ - स्तु॒त॒म् । \newline
55. भव॒ त्यथाथ॒ भव॑ति॒ भव॒ त्यथा॒ भ्य॑भ्यथ॒ भव॑ति॒ भव॒ त्यथा॒भि । \newline
56. अथा॒ भ्य॑भ्यथा था॒भि मृ॑शति मृश त्य॒भ्यथा था॒भि मृ॑शति । \newline
\pagebreak
\markright{ TS 5.4.10.3  \hfill https://www.vedavms.in \hfill}

\section{ TS 5.4.10.3 }

\textbf{TS 5.4.10.3 } \newline
\textbf{Samhita Paata} \newline

-भि मृ॑श॒त्युपै॑न॒मुत्त॑रो य॒ज्ञो न॑म॒त्यथो॒ संत॑त्यै॒ प्र वा ए॒षो᳚ऽस्मान् ॅलो॒काच्च्य॑वते॒ यो᳚ऽग्निं चि॑नु॒ते न वा ए॒तस्या॑निष्ट॒क आहु॑ति॒रव॑ कल्पते॒ यां ॅवा ए॒षो॑ऽनिष्ट॒क आहु॑तिं जु॒होति॒ स्रव॑ति॒ वै सा ताꣳ स्रव॑न्तीं ॅय॒ज्ञोऽनु॒ परा॑ भवति य॒ज्ञ्ं ॅयज॑मानो॒ यत् पु॑नश्चि॒तिं चि॑नु॒त आहु॑तीनां॒ प्रति॑ष्ठित्यै॒ प्रत्याहु॑तय॒स्तिष्ठ॑न्ति॒ - [  ] \newline

\textbf{Pada Paata} \newline

अ॒भीति॑ । मृ॒श॒ति॒ । उपेति॑ । ए॒न॒म् । उत्त॑र॒ इत्युत् - त॒रः॒ । य॒ज्ञ्ः । न॒म॒ति॒ । अथो॒ इति॑ । संत॑त्या॒ इति॒ सं - त॒त्यै॒ । प्रेति॑ । वै । ए॒षः । अ॒स्मात् । लो॒कात् । च्य॒व॒ते॒ । यः । अ॒ग्निम् । चि॒नु॒ते । न । वै । ए॒तस्य॑ । अ॒नि॒ष्ट॒के । आहु॑ति॒रित्या - हु॒तिः॒ । अवेति॑ । क॒ल्प॒ते॒ । याम् । वै । ए॒षः । अ॒नि॒ष्ट॒के । आहु॑ति॒मित्या - हु॒ति॒म् । जु॒होति॑ । स्रव॑ति । वै । सा । ताम् । स्रव॑न्तीम् । य॒ज्ञ्ः । अनु॑ । परेति॑ । भ॒व॒ति॒ । य॒ज्ञ्म् । यज॑मानः । यत् । पु॒न॒श्चि॒तिमिति॑ पुनः - चि॒तिम् । चि॒नु॒ते । आहु॑तीना॒मित्या-हु॒ती॒ना॒म् । प्रति॑ष्ठित्या॒ इति॒ प्रति॑-स्थि॒त्यै॒ । प्रतीति॑ । आहु॑तय॒ इत्या - हु॒त॒यः॒ । तिष्ठ॑न्ति ।  \newline


\textbf{Krama Paata} \newline

अ॒भि मृ॑शति । मृ॒श॒त्युप॑ । उपै॑नम् । ए॒न॒मुत्त॑रः । उत्त॑रो य॒ज्ञ्ः । उत्त॑र॒ इत्युत् - त॒रः॒ । य॒ज्ञो न॑मति । न॒म॒त्यथो᳚ । अथो॒ सन्त॑त्यै । अथो॒ इत्यथो᳚ । सन्त॑त्यै॒ प्र । सन्त॑त्या॒ इति॒ सम् - त॒त्यै॒ । प्र वै । वा ए॒षः । ए॒षो᳚ऽस्मात् । अ॒स्माल्लो॒कात् । लो॒काच् च्य॑वते । च्य॒व॒ते॒ यः । यो᳚ऽग्निम् । अ॒ग्निम् चि॑नु॒ते । चि॒नु॒ते न । न वै । वा ए॒तस्य॑ । ए॒तस्या॑निष्ट॒के । अ॒नि॒ष्ट॒क आहु॑तिः । आहु॑ति॒रव॑ । आहु॑ति॒रित्या - हु॒तिः॒ । अव॑ कल्पते । क॒ल्प॒ते॒ याम् । याम् ॅवै । वा ए॒षः । ए॒षो॑ऽनिष्ट॒के । अ॒नि॒ष्ट॒क आहु॑तिम् । आहु॑तिम् जु॒होति॑ । आहु॑ति॒मित्या - हु॒ति॒म् । जु॒होति॒ स्रव॑ति । स्रव॑ति॒ वै । वै सा । सा ताम् । ताꣳ स्रव॑न्तीम् । स्रव॑न्तीम् ॅय॒ज्ञ्ः । य॒ज्ञोऽनु॑ । अनु॒ परा᳚ । परा॑ भवति । भ॒व॒ति॒ य॒ज्ञ्म् । य॒ज्ञ्म् ॅयज॑मानः । यज॑मानो॒ यत् । यत् पु॑नश्चि॒तिम् । पु॒न॒श्चि॒तिम् चि॑नु॒ते । पु॒न॒श्चि॒तिमिति॑ पुनः - चि॒तिम् । चि॒नु॒त आहु॑तीनाम् । आहु॑तीना॒म् प्रति॑ष्ठित्यै । आहु॑तीना॒मित्या - हु॒ती॒ना॒म् । प्रति॑ष्ठित्यै॒ प्रति॑ । प्रति॑ष्ठित्या॒ इति॒ प्रति॑ - स्थि॒त्यै॒ । प्रत्याहु॑तयः । आहु॑तय॒स्तिष्ठ॑न्ति । आहु॑तय॒ इत्या - हु॒त॒यः॒ । तिष्ठ॑न्ति॒ न \newline

\textbf{Jatai Paata} \newline

1. अ॒भि मृ॑शति मृश त्य॒भ्य॑भि मृ॑शति । \newline
2. मृ॒श॒ त्युपोप॑ मृशति मृश॒ त्युप॑ । \newline
3. उपै॑न मेन॒ मुपो पै॑नम् । \newline
4. ए॒न॒ मुत्त॑र॒ उत्त॑र एन मेन॒ मुत्त॑रः । \newline
5. उत्त॑रो य॒ज्ञो य॒ज्ञ् उत्त॑र॒ उत्त॑रो य॒ज्ञ्ः । \newline
6. उत्त॑र॒ इत्युत् - त॒रः॒ । \newline
7. य॒ज्ञो न॑मति नमति य॒ज्ञो य॒ज्ञो न॑मति । \newline
8. न॒म॒ त्यथो॒ अथो॑ नमति नम॒ त्यथो᳚ । \newline
9. अथो॒ सन्त॑त्यै॒ सन्त॑त्या॒ अथो॒ अथो॒ सन्त॑त्यै । \newline
10. अथो॒ इत्यथो᳚ । \newline
11. सन्त॑त्यै॒ प्र प्र सन्त॑त्यै॒ सन्त॑त्यै॒ प्र । \newline
12. सन्त॑त्या॒ इति॒ सं - त॒त्यै॒ । \newline
13. प्र वै वै प्र प्र वै । \newline
14. वा ए॒ष ए॒ष वै वा ए॒षः । \newline
15. ए॒षो᳚ ऽस्मा द॒स्मा दे॒ष ए॒षो᳚ ऽस्मात् । \newline
16. अ॒स्माल् लो॒काल् लो॒का द॒स्मा द॒स्माल् लो॒कात् । \newline
17. लो॒काच् च्य॑वते च्यवते लो॒काल् लो॒काच् च्य॑वते । \newline
18. च्य॒व॒ते॒ यो यश्च्य॑वते च्यवते॒ यः । \newline
19. यो᳚ ऽग्नि म॒ग्निं ॅयो यो᳚ ऽग्निम् । \newline
20. अ॒ग्निम् चि॑नु॒ते चि॑नु॒ते᳚ ऽग्नि म॒ग्निम् चि॑नु॒ते । \newline
21. चि॒नु॒ते न न चि॑नु॒ते चि॑नु॒ते न । \newline
22. न वै वै न न वै । \newline
23. वा ए॒त स्यै॒तस्य॒ वै वा ए॒तस्य॑ । \newline
24. ए॒तस्या॑ निष्ट॒के॑ ऽनिष्ट॒क ए॒त स्यै॒तस्या॑ निष्ट॒के । \newline
25. अ॒नि॒ष्ट॒क आहु॑ति॒ राहु॑ति रनिष्ट॒के॑ ऽनिष्ट॒क आहु॑तिः । \newline
26. आहु॑ति॒ रवावा हु॑ति॒ राहु॑ति॒ रव॑ । \newline
27. आहु॑ति॒रित्या - हु॒तिः॒ । \newline
28. अव॑ कल्पते कल्प॒ते ऽवाव॑ कल्पते । \newline
29. क॒ल्प॒ते॒ यां ॅयाम् क॑ल्पते कल्पते॒ याम् । \newline
30. यां ॅवै वै यां ॅयां ॅवै । \newline
31. वा ए॒ष ए॒ष वै वा ए॒षः । \newline
32. ए॒षो॑ ऽनिष्ट॒के॑ ऽनिष्ट॒क ए॒ष ए॒षो॑ ऽनिष्ट॒के । \newline
33. अ॒नि॒ष्ट॒क आहु॑ति॒ माहु॑ति मनिष्ट॒के॑ ऽनिष्ट॒क आहु॑तिम् । \newline
34. आहु॑तिम् जु॒होति॑ जु॒हो त्याहु॑ति॒ माहु॑तिम् जु॒होति॑ । \newline
35. आहु॑ति॒मित्या - हु॒ति॒म् । \newline
36. जु॒होति॒ स्रव॑ति॒ स्रव॑ति जु॒होति॑ जु॒होति॒ स्रव॑ति । \newline
37. स्रव॑ति॒ वै वै स्रव॑ति॒ स्रव॑ति॒ वै । \newline
38. वै सा सा वै वै सा । \newline
39. सा ताम् ताꣳ सा सा ताम् । \newline
40. ताꣳ स्रव॑न्तीꣳ॒॒ स्रव॑न्ती॒म् ताम् ताꣳ स्रव॑न्तीम् । \newline
41. स्रव॑न्तीं ॅय॒ज्ञो य॒ज्ञ्ः स्रव॑न्तीꣳ॒॒ स्रव॑न्तीं ॅय॒ज्ञ्ः । \newline
42. य॒ज्ञो ऽन्वनु॑ य॒ज्ञो य॒ज्ञो ऽनु॑ । \newline
43. अनु॒ परा॒ परा ऽन्वनु॒ परा᳚ । \newline
44. परा॑ भवति भवति॒ परा॒ परा॑ भवति । \newline
45. भ॒व॒ति॒ य॒ज्ञ्ं ॅय॒ज्ञ्म् भ॑वति भवति य॒ज्ञ्म् । \newline
46. य॒ज्ञ्ं ॅयज॑मानो॒ यज॑मानो य॒ज्ञ्ं ॅय॒ज्ञ्ं ॅयज॑मानः । \newline
47. यज॑मानो॒ यद् यद् यज॑मानो॒ यज॑मानो॒ यत् । \newline
48. यत् पु॑नश्चि॒तिम् पु॑नश्चि॒तिं ॅयद् यत् पु॑नश्चि॒तिम् । \newline
49. पु॒न॒श्चि॒तिम् चि॑नु॒ते चि॑नु॒ते पु॑नश्चि॒तिम् पु॑नश्चि॒तिम् चि॑नु॒ते । \newline
50. पु॒न॒श्चि॒तिमिति॑ पुनः - चि॒तिम् । \newline
51. चि॒नु॒त आहु॑तीना॒ माहु॑तीनाम् चिनु॒ते चि॑नु॒त आहु॑तीनाम् । \newline
52. आहु॑तीना॒म् प्रति॑ष्ठित्यै॒ प्रति॑ष्ठित्या॒ आहु॑तीना॒ माहु॑तीना॒म् प्रति॑ष्ठित्यै । \newline
53. आहु॑तीना॒मित्या - हु॒ती॒ना॒म् । \newline
54. प्रति॑ष्ठित्यै॒ प्रति॒ प्रति॒ प्रति॑ष्ठित्यै॒ प्रति॑ष्ठित्यै॒ प्रति॑ । \newline
55. प्रति॑ष्ठित्या॒ इति॒ प्रति॑ - स्थि॒त्यै॒ । \newline
56. प्रत्याहु॑तय॒ आहु॑तयः॒ प्रति॒ प्रत्याहु॑तयः । \newline
57. आहु॑तय॒ स्तिष्ठ॑न्ति॒ तिष्ठ॒न् त्याहु॑तय॒ आहु॑तय॒ स्तिष्ठ॑न्ति । \newline
58. आहु॑तय॒ इत्या - हु॒त॒यः॒ । \newline
59. तिष्ठ॑न्ति॒ न न तिष्ठ॑न्ति॒ तिष्ठ॑न्ति॒ न । \newline

\textbf{Ghana Paata } \newline

1. अ॒भि मृ॑शति मृश त्य॒भ्य॑भि मृ॑श॒ त्युपोप॑ मृश त्य॒भ्य॑भि मृ॑श॒ त्युप॑ । \newline
2. मृ॒श॒ त्युपोप॑ मृशति मृश॒ त्युपै॑न मेन॒ मुप॑ मृशति मृश॒ त्युपै॑नम् । \newline
3. उपै॑न मेन॒ मुपोपै॑न॒ मुत्त॑र॒ उत्त॑र एन॒ मुपो पै॑न॒ मुत्त॑रः । \newline
4. ए॒न॒ मुत्त॑र॒ उत्त॑र एन मेन॒ मुत्त॑रो य॒ज्ञो य॒ज्ञ् उत्त॑र एन मेन॒ मुत्त॑रो य॒ज्ञ्ः । \newline
5. उत्त॑रो य॒ज्ञो य॒ज्ञ् उत्त॑र॒ उत्त॑रो य॒ज्ञो न॑मति नमति य॒ज्ञ् उत्त॑र॒ उत्त॑रो य॒ज्ञो न॑मति । \newline
6. उत्त॑र॒ इत्युत् - त॒रः॒ । \newline
7. य॒ज्ञो न॑मति नमति य॒ज्ञो य॒ज्ञो न॑म॒ त्यथो॒ अथो॑ नमति य॒ज्ञो य॒ज्ञो न॑म॒ त्यथो᳚ । \newline
8. न॒म॒ त्यथो॒ अथो॑ नमति नम॒ त्यथो॒ सन्त॑त्यै॒ सन्त॑त्या॒ अथो॑ नमति नम॒ त्यथो॒ सन्त॑त्यै । \newline
9. अथो॒ सन्त॑त्यै॒ सन्त॑त्या॒ अथो॒ अथो॒ सन्त॑त्यै॒ प्र प्र सन्त॑त्या॒ अथो॒ अथो॒ सन्त॑त्यै॒ प्र । \newline
10. अथो॒ इत्यथो᳚ । \newline
11. सन्त॑त्यै॒ प्र प्र सन्त॑त्यै॒ सन्त॑त्यै॒ प्र वै वै प्र सन्त॑त्यै॒ सन्त॑त्यै॒ प्र वै । \newline
12. सन्त॑त्या॒ इति॒ सं - त॒त्यै॒ । \newline
13. प्र वै वै प्र प्र वा ए॒ष ए॒ष वै प्र प्र वा ए॒षः । \newline
14. वा ए॒ष ए॒ष वै वा ए॒षो᳚ ऽस्मा द॒स्मा दे॒ष वै वा ए॒षो᳚ ऽस्मात् । \newline
15. ए॒षो᳚ ऽस्मा द॒स्मा दे॒ष ए॒षो᳚ ऽस्माल् लो॒काल् लो॒का द॒स्मा दे॒ष ए॒षो᳚ ऽस्माल् लो॒कात् । \newline
16. अ॒स्माल् लो॒काल् लो॒का द॒स्मा द॒स्माल् लो॒काच् च्य॑वते च्यवते लो॒का द॒स्मा द॒स्माल् लो॒काच् च्य॑वते । \newline
17. लो॒काच् च्य॑वते च्यवते लो॒काल् लो॒काच् च्य॑वते॒ यो यश्च्य॑वते लो॒का ल्लो॒काच् च्य॑वते॒ यः । \newline
18. च्य॒व॒ते॒ यो यश्च्य॑वते च्यवते॒ यो᳚ ऽग्नि म॒ग्निं ॅयश्च्य॑वते च्यवते॒ यो᳚ ऽग्निम् । \newline
19. यो᳚ ऽग्नि म॒ग्निं ॅयो यो᳚ ऽग्निम् चि॑नु॒ते चि॑नु॒ते᳚ ऽग्निं ॅयो यो᳚ ऽग्निम् चि॑नु॒ते । \newline
20. अ॒ग्निम् चि॑नु॒ते चि॑नु॒ते᳚ ऽग्नि म॒ग्निम् चि॑नु॒ते न न चि॑नु॒ते᳚ ऽग्नि म॒ग्निम् चि॑नु॒ते न । \newline
21. चि॒नु॒ते न न चि॑नु॒ते चि॑नु॒ते न वै वै न चि॑नु॒ते चि॑नु॒ते न वै । \newline
22. न वै वै न न वा ए॒त स्यै॒तस्य॒ वै न न वा ए॒तस्य॑ । \newline
23. वा ए॒त स्यै॒तस्य॒ वै वा ए॒तस्या॑ निष्ट॒के॑ ऽनिष्ट॒क ए॒तस्य॒ वै वा ए॒तस्या॑ निष्ट॒के । \newline
24. ए॒तस्या॑ निष्ट॒के॑ ऽनिष्ट॒क ए॒त स्यै॒तस्या॑ निष्ट॒क आहु॑ति॒ राहु॑ति रनिष्ट॒क ए॒त स्यै॒तस्या॑ निष्ट॒क आहु॑तिः । \newline
25. अ॒नि॒ष्ट॒क आहु॑ति॒ राहु॑ति रनिष्ट॒के॑ ऽनिष्ट॒क आहु॑ति॒ रवावा हु॑ति रनिष्ट॒के॑ ऽनिष्ट॒क आहु॑ति॒ रव॑ । \newline
26. आहु॑ति॒ रवावा हु॑ति॒ राहु॑ति॒ रव॑ कल्पते कल्प॒ते ऽवाहु॑ति॒ राहु॑ति॒ रव॑ कल्पते । \newline
27. आहु॑ति॒रित्या - हु॒तिः॒ । \newline
28. अव॑ कल्पते कल्प॒ते ऽवाव॑ कल्पते॒ यां ॅयाम् क॑ल्प॒ते ऽवाव॑ कल्पते॒ याम् । \newline
29. क॒ल्प॒ते॒ यां ॅयाम् क॑ल्पते कल्पते॒ यां ॅवै वै याम् क॑ल्पते कल्पते॒ यां ॅवै । \newline
30. यां ॅवै वै यां ॅयां ॅवा ए॒ष ए॒ष वै यां ॅयां ॅवा ए॒षः । \newline
31. वा ए॒ष ए॒ष वै वा ए॒षो॑ ऽनिष्ट॒के॑ ऽनिष्ट॒क ए॒ष वै वा ए॒षो॑ ऽनिष्ट॒के । \newline
32. ए॒षो॑ ऽनिष्ट॒के॑ ऽनिष्ट॒क ए॒ष ए॒षो॑ ऽनिष्ट॒क आहु॑ति॒ माहु॑ति मनिष्ट॒क ए॒ष ए॒षो॑ ऽनिष्ट॒क आहु॑तिम् । \newline
33. अ॒नि॒ष्ट॒क आहु॑ति॒ माहु॑ति मनिष्ट॒के॑ ऽनिष्ट॒क आहु॑तिम् जु॒होति॑ जु॒हो त्याहु॑ति मनिष्ट॒के॑ ऽनिष्ट॒क आहु॑तिम् जु॒होति॑ । \newline
34. आहु॑तिम् जु॒होति॑ जु॒हो त्याहु॑ति॒ माहु॑तिम् जु॒होति॒ स्रव॑ति॒ स्रव॑ति जु॒हो त्याहु॑ति॒ माहु॑तिम् जु॒होति॒ स्रव॑ति । \newline
35. आहु॑ति॒मित्या - हु॒ति॒म् । \newline
36. जु॒होति॒ स्रव॑ति॒ स्रव॑ति जु॒होति॑ जु॒होति॒ स्रव॑ति॒ वै वै स्रव॑ति जु॒होति॑ जु॒होति॒ स्रव॑ति॒ वै । \newline
37. स्रव॑ति॒ वै वै स्रव॑ति॒ स्रव॑ति॒ वै सा सा वै स्रव॑ति॒ स्रव॑ति॒ वै सा । \newline
38. वै सा सा वै वै सा ताम् ताꣳ सा वै वै सा ताम् । \newline
39. सा ताम् ताꣳ सा सा ताꣳ स्रव॑न्तीꣳ॒॒ स्रव॑न्ती॒म् ताꣳ सा सा ताꣳ स्रव॑न्तीम् । \newline
40. ताꣳ स्रव॑न्तीꣳ॒॒ स्रव॑न्ती॒म् ताम् ताꣳ स्रव॑न्तीं ॅय॒ज्ञो य॒ज्ञ्ः स्रव॑न्ती॒म् ताम् ताꣳ स्रव॑न्तीं ॅय॒ज्ञ्ः । \newline
41. स्रव॑न्तीं ॅय॒ज्ञो य॒ज्ञ्ः स्रव॑न्तीꣳ॒॒ स्रव॑न्तीं ॅय॒ज्ञो ऽन्वनु॑ य॒ज्ञ्ः स्रव॑न्तीꣳ॒॒ स्रव॑न्तीं ॅय॒ज्ञो ऽनु॑ । \newline
42. य॒ज्ञो ऽन्वनु॑ य॒ज्ञो य॒ज्ञो ऽनु॒ परा॒ परा ऽनु॑ य॒ज्ञो य॒ज्ञो ऽनु॒ परा᳚ । \newline
43. अनु॒ परा॒ परा ऽन्वनु॒ परा॑ भवति भवति॒ परा ऽन्वनु॒ परा॑ भवति । \newline
44. परा॑ भवति भवति॒ परा॒ परा॑ भवति य॒ज्ञ्ं ॅय॒ज्ञ्म् भ॑वति॒ परा॒ परा॑ भवति य॒ज्ञ्म् । \newline
45. भ॒व॒ति॒ य॒ज्ञ्ं ॅय॒ज्ञ्म् भ॑वति भवति य॒ज्ञ्ं ॅयज॑मानो॒ यज॑मानो य॒ज्ञ्म् भ॑वति भवति य॒ज्ञ्ं ॅयज॑मानः । \newline
46. य॒ज्ञ्ं ॅयज॑मानो॒ यज॑मानो य॒ज्ञ्ं ॅय॒ज्ञ्ं ॅयज॑मानो॒ यद् यद् यज॑मानो य॒ज्ञ्ं ॅय॒ज्ञ्ं ॅयज॑मानो॒ यत् । \newline
47. यज॑मानो॒ यद् यद् यज॑मानो॒ यज॑मानो॒ यत् पु॑नश्चि॒तिम् पु॑नश्चि॒तिं ॅयद् यज॑मानो॒ यज॑मानो॒ यत् पु॑नश्चि॒तिम् । \newline
48. यत् पु॑नश्चि॒तिम् पु॑नश्चि॒तिं ॅयद् यत् पु॑नश्चि॒तिम् चि॑नु॒ते चि॑नु॒ते पु॑नश्चि॒तिं ॅयद् यत् पु॑नश्चि॒तिम् चि॑नु॒ते । \newline
49. पु॒न॒श्चि॒तिम् चि॑नु॒ते चि॑नु॒ते पु॑नश्चि॒तिम् पु॑नश्चि॒तिम् चि॑नु॒त आहु॑तीना॒ माहु॑तीनाम् चिनु॒ते पु॑नश्चि॒तिम् पु॑नश्चि॒तिम् चि॑नु॒त आहु॑तीनाम् । \newline
50. पु॒न॒श्चि॒तिमिति॑ पुनः - चि॒तिम् । \newline
51. चि॒नु॒त आहु॑तीना॒ माहु॑तीनाम् चिनु॒ते चि॑नु॒त आहु॑तीना॒म् प्रति॑ष्ठित्यै॒ प्रति॑ष्ठित्या॒ आहु॑तीनाम् चिनु॒ते चि॑नु॒त आहु॑तीना॒म् प्रति॑ष्ठित्यै । \newline
52. आहु॑तीना॒म् प्रति॑ष्ठित्यै॒ प्रति॑ष्ठित्या॒ आहु॑तीना॒ माहु॑तीना॒म् प्रति॑ष्ठित्यै॒ प्रति॒ प्रति॒ प्रति॑ष्ठित्या॒ आहु॑तीना॒ माहु॑तीना॒म् प्रति॑ष्ठित्यै॒ प्रति॑ । \newline
53. आहु॑तीना॒मित्या - हु॒ती॒ना॒म् । \newline
54. प्रति॑ष्ठित्यै॒ प्रति॒ प्रति॒ प्रति॑ष्ठित्यै॒ प्रति॑ष्ठित्यै॒ प्रत्याहु॑तय॒ आहु॑तयः॒ प्रति॒ प्रति॑ष्ठित्यै॒ प्रति॑ष्ठित्यै॒ प्रत्याहु॑तयः । \newline
55. प्रति॑ष्ठित्या॒ इति॒ प्रति॑ - स्थि॒त्यै॒ । \newline
56. प्रत्याहु॑तय॒ आहु॑तयः॒ प्रति॒ प्रत्याहु॑तय॒ स्तिष्ठ॑न्ति॒ तिष्ठ॒न् त्याहु॑तयः॒ प्रति॒ प्रत्याहु॑तय॒ स्तिष्ठ॑न्ति । \newline
57. आहु॑तय॒ स्तिष्ठ॑न्ति॒ तिष्ठ॒न् त्याहु॑तय॒ आहु॑तय॒ स्तिष्ठ॑न्ति॒ न न तिष्ठ॒न् त्याहु॑तय॒ आहु॑तय॒ स्तिष्ठ॑न्ति॒ न । \newline
58. आहु॑तय॒ इत्या - हु॒त॒यः॒ । \newline
59. तिष्ठ॑न्ति॒ न न तिष्ठ॑न्ति॒ तिष्ठ॑न्ति॒ न य॒ज्ञो य॒ज्ञो न तिष्ठ॑न्ति॒ तिष्ठ॑न्ति॒ न य॒ज्ञ्ः । \newline
\pagebreak
\markright{ TS 5.4.10.4  \hfill https://www.vedavms.in \hfill}

\section{ TS 5.4.10.4 }

\textbf{TS 5.4.10.4 } \newline
\textbf{Samhita Paata} \newline

न य॒ज्ञ्ः प॑रा॒भव॑ति॒ न यज॑मानो॒ ऽष्टावुप॑ दधात्य॒ष्टाक्ष॑रा गाय॒त्री गा॑य॒त्रेणै॒वैनं॒ छन्द॑सा चिनुते॒ यदेका॑दश॒ त्रैष्टु॑भेन॒ यद् द्वाद॑श॒ जाग॑तेन॒ छन्दो॑भिरे॒वैनं॑ चिनुते नपा॒त्को वैनामै॒षो᳚ऽग्निर्यत् पु॑नश्चि॒तिर्य ए॒वं ॅवि॒द्वान् पु॑नश्चि॒तिं चि॑नु॒त आ तृ॒तीया॒त् पुरु॑षा॒दन्न॑मत्ति॒ यथा॒ वै पु॑नरा॒धेय॑ ए॒वं पु॑नश्चि॒तिर्यो᳚ऽग्न्या॒धेये॑न॒ न - [  ] \newline

\textbf{Pada Paata} \newline

न । य॒ज्ञ्ः । प॒रा॒भव॒तीति॑ परा - भव॑ति । न । यज॑मानः । अ॒ष्टौ । उपेति॑ । द॒धा॒ति॒ । अ॒ष्टाक्ष॒रेत्य॒ष्टा - अ॒क्ष॒रा॒ । गा॒य॒त्री । गा॒य॒त्रेण॑ । ए॒व । ए॒न॒म् । छन्द॑सा । चि॒नु॒ते॒ । यत् । एका॑दश । त्रैष्टु॑भेन । यत् । द्वाद॑श । जाग॑तेन । छन्दो॑भि॒रिति॒ छन्दः॑ - भिः॒ । ए॒व । ए॒न॒म् । चि॒नु॒ते॒ । न॒पा॒त्कः । वै । नाम॑ । ए॒षः । अ॒ग्निः । यत् । पु॒न॒श्चि॒तिरिति॑ पुनः - चि॒तिः । यः । ए॒वम् । वि॒द्वान् । पु॒न॒श्चि॒तिमिति॑ पुनः - चि॒तिम् । चि॒नु॒ते । एति॑ । तृ॒तीया᳚त् । पुरु॑षात् । अन्न᳚म् । अ॒त्ति॒ । यथा᳚ । वै । पु॒न॒रा॒धेय॒ इति॑ पुनः - आ॒धेयः॑ । ए॒वम् । पु॒न॒श्चि॒तिरिति॑ पुनः - चि॒तिः । यः । अ॒ग्न्या॒धेये॒नेत्य॑ग्नि-आ॒धेये॑न । न ।  \newline


\textbf{Krama Paata} \newline

न य॒ज्ञ्ः । य॒ज्ञ्ः प॑रा॒भव॑ति । प॒रा॒भव॑ति॒ न । प॒रा॒भव॒तीति॑ परा - भव॑ति । न यज॑मानः । यज॑मानो॒ऽष्टौ । अ॒ष्टावुप॑ । उप॑ दधाति । द॒धा॒त्य॒ष्टाक्ष॑रा । अ॒ष्टाक्ष॑रा गाय॒त्री । अ॒ष्टाक्ष॒रेत्य॒ष्टा - अ॒क्ष॒रा॒ । गा॒य॒त्री गा॑य॒त्रेण॑ । गा॒य॒त्रेणै॒व । ए॒वैन᳚म् । ए॒न॒म् छन्द॑सा । छन्द॑सा चिनुते । चि॒नु॒ते॒ यत् । यदेका॑दश । एका॑दश॒ त्रैष्टु॑भेन । त्रैष्टु॑भेन॒ यत् । यद् द्वाद॑श । द्वाद॑श॒ जाग॑तेन । जाग॑तेन॒ छन्दो॑भिः । छन्दो॑भिरे॒व । छन्दो॑भि॒रिति॒ छन्दः॑ - भिः॒ । ए॒वैन᳚म् । ए॒न॒म् चि॒नु॒ते॒ । चि॒नु॒ते॒ न॒पा॒त्कः । न॒पा॒त्को वै । वै नाम॑ । नामै॒षः । ए॒षो᳚ऽग्निः । अ॒ग्निर् यत् । यत् पु॑नश्चि॒तिः । पु॒न॒श्चि॒तिर् यः । पु॒न॒श्चि॒तिरिति॑ पुनः - चि॒तिः । य ए॒वम् । ए॒वम् ॅवि॒द्वान् । वि॒द्वान् पु॑नश्चि॒तिम् । पु॒न॒श्चि॒तिम् चि॑नु॒ते । पु॒न॒श्चि॒तिमिति॑ पुनः - चि॒तिम् । चि॒नु॒त॒ आ । आ तृ॒तीया᳚त् । तृ॒तीया॒त् पुरु॑षात् । पुरु॑षा॒दन्न᳚म् । अन्न॑मत्ति । अ॒त्ति॒ यथा᳚ । यथा॒ वै । वै पु॑नरा॒धेयः॑ । पु॒न॒रा॒धेय॑ ए॒वम् । पु॒न॒रा॒धेय॒ इति॑ पुनः - आ॒धेयः॑ । ए॒वम् पु॑नश्चि॒तिः । पु॒न॒श्चि॒तिर् यः । पु॒न॒श्चि॒तिरिति॑ पुनः - चि॒तिः । यो᳚ऽग्न्या॒धेये॑न । अ॒ग्न्या॒धेये॑न॒ न । अ॒ग्न्या॒धेये॒नेत्य॑ग्नि - आ॒धेये॑न । नर्द्ध्नोति॑ \newline

\textbf{Jatai Paata} \newline

1. न य॒ज्ञो य॒ज्ञो न न य॒ज्ञ्ः । \newline
2. य॒ज्ञ्ः प॑रा॒भव॑ति परा॒भव॑ति य॒ज्ञो य॒ज्ञ्ः प॑रा॒भव॑ति । \newline
3. प॒रा॒भव॑ति॒ न न प॑रा॒भव॑ति परा॒भव॑ति॒ न । \newline
4. प॒रा॒भव॒तीति॑ परा - भव॑ति । \newline
5. न यज॑मानो॒ यज॑मानो॒ न न यज॑मानः । \newline
6. यज॑मानो॒ ऽष्टा व॒ष्टौ यज॑मानो॒ यज॑मानो॒ ऽष्टौ । \newline
7. अ॒ष्टा वुपोपा॒ष्टा व॒ष्टा वुप॑ । \newline
8. उप॑ दधाति दधा॒ त्युपोप॑ दधाति । \newline
9. द॒धा॒ त्य॒ष्टाक्ष॑रा॒ ऽष्टाक्ष॑रा दधाति दधा त्य॒ष्टाक्ष॑रा । \newline
10. अ॒ष्टाक्ष॑रा गाय॒त्री गा॑य॒ त्र्य॑ष्टाक्ष॑रा॒ ऽष्टाक्ष॑रा गाय॒त्री । \newline
11. अ॒ष्टाक्ष॒रेत्य॒ष्टा - अ॒क्ष॒रा॒ । \newline
12. गा॒य॒त्री गा॑य॒त्रेण॑ गाय॒त्रेण॑ गाय॒त्री गा॑य॒त्री गा॑य॒त्रेण॑ । \newline
13. गा॒य॒त्रेणै॒वैव गा॑य॒त्रेण॑ गाय॒त्रेणै॒व । \newline
14. ए॒वैन॑ मेन मे॒वै वैन᳚म् । \newline
15. ए॒न॒म् छन्द॑सा॒ छन्द॑सैन मेन॒म् छन्द॑सा । \newline
16. छन्द॑सा चिनुते चिनुते॒ छन्द॑सा॒ छन्द॑सा चिनुते । \newline
17. चि॒नु॒ते॒ यद् यच् चि॑नुते चिनुते॒ यत् । \newline
18. यदेका॑द॒ शैका॑दश॒ यद् यदेका॑दश । \newline
19. एका॑दश॒ त्रैष्टु॑भेन॒ त्रैष्टु॑भे॒ नैका॑द॒ शैका॑दश॒ त्रैष्टु॑भेन । \newline
20. त्रैष्टु॑भेन॒ यद् यत् त्रैष्टु॑भेन॒ त्रैष्टु॑भेन॒ यत् । \newline
21. यद् द्वाद॑श॒ द्वाद॑श॒ यद् यद् द्वाद॑श । \newline
22. द्वाद॑श॒ जाग॑तेन॒ जाग॑तेन॒ द्वाद॑श॒ द्वाद॑श॒ जाग॑तेन । \newline
23. जाग॑तेन॒ छन्दो॑भि॒ श्छन्दो॑भि॒र् जाग॑तेन॒ जाग॑तेन॒ छन्दो॑भिः । \newline
24. छन्दो॑भि रे॒वैव छन्दो॑भि॒ श्छन्दो॑भ् रे॒व । \newline
25. छन्दो॑भि॒रिति॒ छन्दः॑ - भिः॒ । \newline
26. ए॒वैन॑ मेन मे॒वै वैन᳚म् । \newline
27. ए॒न॒म् चि॒नु॒ते॒ चि॒नु॒त॒ ए॒न॒ मे॒न॒म् चि॒नु॒ते॒ । \newline
28. चि॒नु॒ते॒ न॒पा॒त्को न॑पा॒त्क श्चि॑नुते चिनुते नपा॒त्कः । \newline
29. न॒पा॒त्को वै वै न॑पा॒त्को न॑पा॒त्को वै । \newline
30. वै नाम॒ नाम॒ वै वै नाम॑ । \newline
31. नामै॒ष ए॒ष नाम॒ नामै॒षः । \newline
32. ए॒षो᳚ ऽग्नि र॒ग्नि रे॒ष ए॒षो᳚ ऽग्निः । \newline
33. अ॒ग्निर् यद् यद॒ग्नि र॒ग्निर् यत् । \newline
34. यत् पु॑नश्चि॒तिः पु॑नश्चि॒तिर् यद् यत् पु॑नश्चि॒तिः । \newline
35. पु॒न॒श्चि॒तिर् यो यः पु॑नश्चि॒तिः पु॑नश्चि॒तिर् यः । \newline
36. पु॒न॒श्चि॒तिरिति॑ पुनः - चि॒तिः । \newline
37. य ए॒व मे॒वं ॅयो य ए॒वम् । \newline
38. ए॒वं ॅवि॒द्वान्. वि॒द्वा ने॒व मे॒वं ॅवि॒द्वान् । \newline
39. वि॒द्वान् पु॑नश्चि॒तिम् पु॑नश्चि॒तिं ॅवि॒द्वान्. वि॒द्वान् पु॑नश्चि॒तिम् । \newline
40. पु॒न॒श्चि॒तिम् चि॑नु॒ते चि॑नु॒ते पु॑नश्चि॒तिम् पु॑नश्चि॒तिम् चि॑नु॒ते । \newline
41. पु॒न॒श्चि॒तिमिति॑ पुनः - चि॒तिम् । \newline
42. चि॒नु॒त आ चि॑नु॒ते चि॑नु॒त आ । \newline
43. आ तृ॒तीया᳚त् तृ॒तीया॒दा तृ॒तीया᳚त् । \newline
44. तृ॒तीया॒त् पुरु॑षा॒त् पुरु॑षात् तृ॒तीया᳚त् तृ॒तीया॒त् पुरु॑षात् । \newline
45. पुरु॑षा॒ दन्न॒ मन्न॒म् पुरु॑षा॒त् पुरु॑षा॒ दन्न᳚म् । \newline
46. अन्न॑ मत्त्य॒ त्त्यन्न॒ मन्न॑ मत्ति । \newline
47. अ॒त्ति॒ यथा॒ यथा᳚ ऽत्त्यत्ति॒ यथा᳚ । \newline
48. यथा॒ वै वै यथा॒ यथा॒ वै । \newline
49. वै पु॑नरा॒धेयः॑ पुनरा॒धेयो॒ वै वै पु॑नरा॒धेयः॑ । \newline
50. पु॒न॒रा॒धेय॑ ए॒व मे॒वम् पु॑नरा॒धेयः॑ पुनरा॒धेय॑ ए॒वम् । \newline
51. पु॒न॒रा॒धेय॒ इति॑ पुनः - आ॒धेयः॑ । \newline
52. ए॒वम् पु॑नश्चि॒तिः पु॑नश्चि॒ति रे॒व मे॒वम् पु॑नश्चि॒तिः । \newline
53. पु॒न॒श्चि॒तिर् यो यः पु॑नश्चि॒तिः पु॑नश्चि॒तिर् यः । \newline
54. पु॒न॒श्चि॒तिरिति॑ पुनः - चि॒तिः । \newline
55. यो᳚ ऽग्न्या॒धेये॑ना ग्न्या॒धेये॑न॒ यो यो᳚ ऽग्न्या॒धेये॑न । \newline
56. अ॒ग्न्या॒धेये॑न॒ न नाग्न्या॒धेये॑ नाग्न्या॒धेये॑न॒ न । \newline
57. अ॒ग्न्या॒धेये॒नेत्य॑ग्नि - आ॒धेये॑न । \newline
58. न र्द्ध्नो त्यृ॒द्ध्नोति॒ न न र्द्ध्नोति॑ । \newline

\textbf{Ghana Paata } \newline

1. न य॒ज्ञो य॒ज्ञो न न य॒ज्ञ्ः प॑रा॒भव॑ति परा॒भव॑ति य॒ज्ञो न न य॒ज्ञ्ः प॑रा॒भव॑ति । \newline
2. य॒ज्ञ्ः प॑रा॒भव॑ति परा॒भव॑ति य॒ज्ञो य॒ज्ञ्ः प॑रा॒भव॑ति॒ न न प॑रा॒भव॑ति य॒ज्ञो य॒ज्ञ्ः प॑रा॒भव॑ति॒ न । \newline
3. प॒रा॒भव॑ति॒ न न प॑रा॒भव॑ति परा॒भव॑ति॒ न यज॑मानो॒ यज॑मानो॒ न प॑रा॒भव॑ति परा॒भव॑ति॒ न यज॑मानः । \newline
4. प॒रा॒भव॒तीति॑ परा - भव॑ति । \newline
5. न यज॑मानो॒ यज॑मानो॒ न न यज॑मानो॒ ऽष्टा व॒ष्टौ यज॑मानो॒ न न यज॑मानो॒ ऽष्टौ । \newline
6. यज॑मानो॒ ऽष्टा व॒ष्टौ यज॑मानो॒ यज॑मानो॒ ऽष्टा वुपोपा॒ष्टौ यज॑मानो॒ यज॑मानो॒ ऽष्टा वुप॑ । \newline
7. अ॒ष्टा वुपोपा॒ष्टा व॒ष्टा वुप॑ दधाति दधा॒ त्युपा॒ष्टा व॒ष्टा वुप॑ दधाति । \newline
8. उप॑ दधाति दधा॒ त्युपोप॑ दधा त्य॒ष्टाक्ष॑रा॒ ऽष्टाक्ष॑रा दधा॒ त्युपोप॑ दधा त्य॒ष्टाक्ष॑रा । \newline
9. द॒धा॒ त्य॒ष्टाक्ष॑रा॒ ऽष्टाक्ष॑रा दधाति दधा त्य॒ष्टाक्ष॑रा गाय॒त्री गा॑य॒ त्र्य॑ष्टाक्ष॑रा दधाति दधा त्य॒ष्टाक्ष॑रा गाय॒त्री । \newline
10. अ॒ष्टाक्ष॑रा गाय॒त्री गा॑य॒ त्र्य॑ष्टाक्ष॑रा॒ ऽष्टाक्ष॑रा गाय॒त्री गा॑य॒त्रेण॑ गाय॒त्रेण॑ गाय॒
त्र्य॑ष्टाक्ष॑रा॒ ऽष्टाक्ष॑रा गाय॒त्री गा॑य॒त्रेण॑ । \newline
11. अ॒ष्टाक्ष॒रेत्य॒ष्टा - अ॒क्ष॒रा॒ । \newline
12. गा॒य॒त्री गा॑य॒त्रेण॑ गाय॒त्रेण॑ गाय॒त्री गा॑य॒त्री गा॑य॒त्रे णै॒वैव गा॑य॒त्रेण॑ गाय॒त्री गा॑य॒त्री गा॑य॒त्रेणै॒व । \newline
13. गा॒य॒त्रे णै॒वैव गा॑य॒त्रेण॑ गाय॒त्रे णै॒वैन॑ मेन मे॒व गा॑य॒त्रेण॑ गाय॒त्रे णै॒वैन᳚म् । \newline
14. ए॒वैन॑ मेन मे॒वै वैन॒म् छन्द॑सा॒ छन्द॑सैन मे॒वै वैन॒म् छन्द॑सा । \newline
15. ए॒न॒म् छन्द॑सा॒ छन्द॑सैन मेन॒म् छन्द॑सा चिनुते चिनुते॒ छन्द॑सैन मेन॒म् छन्द॑सा चिनुते । \newline
16. छन्द॑सा चिनुते चिनुते॒ छन्द॑सा॒ छन्द॑सा चिनुते॒ यद् यच् चि॑नुते॒ छन्द॑सा॒ छन्द॑सा चिनुते॒ यत् । \newline
17. चि॒नु॒ते॒ यद् यच् चि॑नुते चिनुते॒ यदेका॑द॒ शैका॑दश॒ यच् चि॑नुते चिनुते॒ यदेका॑दश । \newline
18. यदेका॑द॒ शैका॑दश॒ यद् यदेका॑दश॒ त्रैष्टु॑भेन॒ त्रैष्टु॑भे॒ नैका॑दश॒ यद् यदेका॑दश॒ त्रैष्टु॑भेन । \newline
19. एका॑दश॒ त्रैष्टु॑भेन॒ त्रैष्टु॑भे॒ नैका॑द॒ शैका॑दश॒ त्रैष्टु॑भेन॒ यद् यत् त्रैष्टु॑भे॒ नैका॑द॒ 
शैका॑दश॒ त्रैष्टु॑भेन॒ यत् । \newline
20. त्रैष्टु॑भेन॒ यद् यत् त्रैष्टु॑भेन॒ त्रैष्टु॑भेन॒ यद् द्वाद॑श॒ द्वाद॑श॒ यत् त्रैष्टु॑भेन॒ त्रैष्टु॑भेन॒ यद् द्वाद॑श । \newline
21. यद् द्वाद॑श॒ द्वाद॑श॒ यद् यद् द्वाद॑श॒ जाग॑तेन॒ जाग॑तेन॒ द्वाद॑श॒ यद् यद् द्वाद॑श॒ जाग॑तेन । \newline
22. द्वाद॑श॒ जाग॑तेन॒ जाग॑तेन॒ द्वाद॑श॒ द्वाद॑श॒ जाग॑तेन॒ छन्दो॑भि॒ श्छन्दो॑भि॒र् जाग॑तेन॒ द्वाद॑श॒ द्वाद॑श॒ जाग॑तेन॒ छन्दो॑भिः । \newline
23. जाग॑तेन॒ छन्दो॑भि॒ श्छन्दो॑भि॒र् जाग॑तेन॒ जाग॑तेन॒ छन्दो॑भि रे॒वैव छन्दो॑भि॒र् जाग॑तेन॒ जाग॑तेन॒ छन्दो॑भि रे॒व । \newline
24. छन्दो॑भि रे॒वैव छन्दो॑भि॒ श्छन्दो॑भि रे॒वैन॑ मेन मे॒व छन्दो॑भि॒ श्छन्दो॑भि रे॒वैन᳚म् । \newline
25. छन्दो॑भि॒रिति॒ छन्दः॑ - भिः॒ । \newline
26. ए॒वैन॑ मेन मे॒वै वैन॑म् चिनुते चिनुत एन मे॒वै वैन॑म् चिनुते । \newline
27. ए॒न॒म् चि॒नु॒ते॒ चि॒नु॒त॒ ए॒न॒ मे॒न॒म् चि॒नु॒ते॒ न॒पा॒त्को न॑पा॒त्क श्चि॑नुत एन मेनम् चिनुते नपा॒त्कः । \newline
28. चि॒नु॒ते॒ न॒पा॒त्को न॑पा॒त्क श्चि॑नुते चिनुते नपा॒त्को वै वै न॑पा॒त्क श्चि॑नुते चिनुते नपा॒त्को वै । \newline
29. न॒पा॒त्को वै वै न॑पा॒त्को न॑पा॒त्को वै नाम॒ नाम॒ वै न॑पा॒त्को न॑पा॒त्को वै नाम॑ । \newline
30. वै नाम॒ नाम॒ वै वै नामै॒ष ए॒ष नाम॒ वै वै नामै॒षः । \newline
31. नामै॒ष ए॒ष नाम॒ नामै॒षो᳚ ऽग्नि र॒ग्नि रे॒ष नाम॒ नामै॒षो᳚ ऽग्निः । \newline
32. ए॒षो᳚ ऽग्नि र॒ग्नि रे॒ष ए॒षो᳚ ऽग्निर् यद् यद॒ग्नि रे॒ष ए॒षो᳚ ऽग्निर् यत् । \newline
33. अ॒ग्निर् यद् यद॒ग्नि र॒ग्निर् यत् पु॑नश्चि॒तिः पु॑नश्चि॒तिर् यद॒ग्नि र॒ग्निर् यत् पु॑नश्चि॒तिः । \newline
34. यत् पु॑नश्चि॒तिः पु॑नश्चि॒तिर् यद् यत् पु॑नश्चि॒तिर् यो यः पु॑नश्चि॒तिर् यद् यत् पु॑नश्चि॒तिर् यः । \newline
35. पु॒न॒श्चि॒तिर् यो यः पु॑नश्चि॒तिः पु॑नश्चि॒तिर् य ए॒व मे॒वं ॅयः पु॑नश्चि॒तिः पु॑नश्चि॒तिर् य ए॒वम् । \newline
36. पु॒न॒श्चि॒तिरिति॑ पुनः - चि॒तिः । \newline
37. य ए॒व मे॒वं ॅयो य ए॒वं ॅवि॒द्वान्. वि॒द्वा ने॒वं ॅयो य ए॒वं ॅवि॒द्वान् । \newline
38. ए॒वं ॅवि॒द्वान्. वि॒द्वा ने॒व मे॒वं ॅवि॒द्वान् पु॑नश्चि॒तिम् पु॑नश्चि॒तिं ॅवि॒द्वा ने॒व मे॒वं ॅवि॒द्वान् पु॑नश्चि॒तिम् । \newline
39. वि॒द्वान् पु॑नश्चि॒तिम् पु॑नश्चि॒तिं ॅवि॒द्वान्. वि॒द्वान् पु॑नश्चि॒तिम् चि॑नु॒ते चि॑नु॒ते पु॑नश्चि॒तिं ॅवि॒द्वान्. वि॒द्वान् पु॑नश्चि॒तिम् चि॑नु॒ते । \newline
40. पु॒न॒श्चि॒तिम् चि॑नु॒ते चि॑नु॒ते पु॑नश्चि॒तिम् पु॑नश्चि॒तिम् चि॑नु॒त आ चि॑नु॒ते पु॑नश्चि॒तिम् पु॑नश्चि॒तिम् चि॑नु॒त आ । \newline
41. पु॒न॒श्चि॒तिमिति॑ पुनः - चि॒तिम् । \newline
42. चि॒नु॒त आ चि॑नु॒ते चि॑नु॒त आ तृ॒तीया᳚त् तृ॒तीया॒दा चि॑नु॒ते चि॑नु॒त आ तृ॒तीया᳚त् । \newline
43. आ तृ॒तीया᳚त् तृ॒तीया॒दा तृ॒तीया॒त् पुरु॑षा॒त् पुरु॑षात् तृ॒तीया॒दा तृ॒तीया॒त् पुरु॑षात् । \newline
44. तृ॒तीया॒त् पुरु॑षा॒त् पुरु॑षात् तृ॒तीया᳚त् तृ॒तीया॒त् पुरु॑षा॒ दन्न॒ मन्न॒म् पुरु॑षात् तृ॒तीया᳚त् तृ॒तीया॒त् पुरु॑षा॒ दन्न᳚म् । \newline
45. पुरु॑षा॒ दन्न॒ मन्न॒म् पुरु॑षा॒त् पुरु॑षा॒ दन्न॑ मत्त्य॒त् त्यन्न॒म् पुरु॑षा॒त् पुरु॑षा॒ दन्न॑ मत्ति । \newline
46. अन्न॑ मत्त्य॒त् त्यन्न॒ मन्न॑ मत्ति॒ यथा॒ यथा॒ ऽत्त्यन्न॒ मन्न॑ मत्ति॒ यथा᳚ । \newline
47. अ॒त्ति॒ यथा॒ यथा᳚ ऽत्त्यत्ति॒ यथा॒ वै वै यथा᳚ ऽत्त्यत्ति॒ यथा॒ वै । \newline
48. यथा॒ वै वै यथा॒ यथा॒ वै पु॑नरा॒धेयः॑ पुनरा॒धेयो॒ वै यथा॒ यथा॒ वै पु॑नरा॒धेयः॑ । \newline
49. वै पु॑नरा॒धेयः॑ पुनरा॒धेयो॒ वै वै पु॑नरा॒धेय॑ ए॒व मे॒वम् पु॑नरा॒धेयो॒ वै वै पु॑नरा॒धेय॑ ए॒वम् । \newline
50. पु॒न॒रा॒धेय॑ ए॒व मे॒वम् पु॑नरा॒धेयः॑ पुनरा॒धेय॑ ए॒वम् पु॑नश्चि॒तिः पु॑नश्चि॒ति रे॒वम् पु॑नरा॒धेयः॑ पुनरा॒धेय॑ ए॒वम् पु॑नश्चि॒तिः । \newline
51. पु॒न॒रा॒धेय॒ इति॑ पुनः - आ॒धेयः॑ । \newline
52. ए॒वम् पु॑नश्चि॒तिः पु॑नश्चि॒ति रे॒व मे॒वम् पु॑नश्चि॒तिर् यो यः पु॑नश्चि॒ति रे॒व मे॒वम् पु॑नश्चि॒तिर् यः । \newline
53. पु॒न॒श्चि॒तिर् यो यः पु॑नश्चि॒तिः पु॑नश्चि॒तिर् यो᳚ ऽग्न्या॒धेये॑ना ग्न्या॒धेये॑न॒ यः पु॑नश्चि॒तिः पु॑नश्चि॒तिर् यो᳚ ऽग्न्या॒धेये॑न । \newline
54. पु॒न॒श्चि॒तिरिति॑ पुनः - चि॒तिः । \newline
55. यो᳚ ऽग्न्या॒धेये॑ना ग्न्या॒धेये॑न॒ यो यो᳚ ऽग्न्या॒धेये॑न॒ न नाग्न्या॒धेये॑न॒ यो यो᳚ ऽग्न्या॒धेये॑न॒ न । \newline
56. अ॒ग्न्या॒धेये॑न॒ न नाग्न्या॒धेये॑ना ग्न्या॒धेये॑न॒ न र्द्ध्नो त्यृ॒द्ध्नोति॒ नाग्न्या॒धेये॑ना ग्न्या॒धेये॑न॒ न र्द्ध्नोति॑ । \newline
57. अ॒ग्न्या॒धेये॒नेत्य॑ग्नि - आ॒धेये॑न । \newline
58. न र्द्ध्नो त्यृ॒द्ध्नोति॒ न न र्द्ध्नोति॒ स स ऋ॒द्ध्नोति॒ न न र्द्ध्नोति॒ सः । \newline
\pagebreak
\markright{ TS 5.4.10.5  \hfill https://www.vedavms.in \hfill}

\section{ TS 5.4.10.5 }

\textbf{TS 5.4.10.5 } \newline
\textbf{Samhita Paata} \newline

र्ध्नोति॒ स पु॑नरा॒धेय॒मा ध॑त्ते॒ यो᳚ऽग्निं चि॒त्वा नर्द्ध्नोति॒ स पु॑नश्चि॒तिं चि॑नुते॒ यत् पु॑नश्चि॒तिं चि॑नु॒त ऋद्ध्या॒ अथो॒ खल्वा॑हु॒र्न चे॑त॒व्येति॑ रु॒द्रो वा ए॒ष यद॒ग्निर्यथा᳚ व्या॒घ्रꣳ सु॒प्तं बो॒धय॑ति ता॒दृगे॒व तदथो॒ खल्वा॑हुश्चेत॒व्येति॒ यथा॒ वसी॑याꣳसं भाग॒धेये॑न बो॒धय॑ति ता॒दृगे॒व तन्मनु॑र॒ग्निम॑चिनुत॒ ( ) तेन॒ नाऽऽ*र्द्ध्नो॒थ्स ए॒तां पु॑नश्चि॒तिम॑पश्य॒त् ताम॑चिनुत॒ तया॒ वै स आ᳚र्द्ध्नो॒द्यत् पु॑नश्चि॒तिं चि॑नु॒त ऋद्ध्यै᳚ ॥ \newline

\textbf{Pada Paata} \newline

ऋ॒ध्नोति॑ । सः । पु॒न॒रा॒धेय॒मिति॑ पुनः-आ॒धेय᳚म् । एति॑ । ध॒त्ते॒ । यः । अ॒ग्निम् । चि॒त्वा । न । ऋ॒द्ध्नोति॑ । सः । पु॒न॒श्चि॒तिमिति॑ पुनः - चि॒तिम् । चि॒नु॒ते॒ । यत् । पु॒न॒श्चि॒तिमिति॑ पुनः - चि॒तिम् । चि॒नु॒ते । ऋद्ध्यै᳚ । अथो॒ इति॑ । खलु॑ । आ॒हुः॒ । न । चे॒त॒व्या᳚ । इति॑ । रु॒द्रः । वै । ए॒षः । यत् । अ॒ग्निः । यथा᳚ । व्या॒घ्रम् । सु॒प्तम् । बो॒धय॑ति । ता॒दृक् । ए॒व । तत् । अथो॒ इति॑ । खलु॑ । आ॒हुः॒ । चे॒त॒व्या᳚ । इति॑ । यथा᳚ । वसी॑याꣳसम् । भा॒ग॒धेये॒नेति॑ भाग - धेये॑न । बो॒धय॑ति । ता॒दृक् । ए॒व । तत् । मनुः॑ । अ॒ग्निम् । अ॒चि॒नु॒त॒ ( ) । तेन॑ । न । आ॒द्‌र्ध्नो॒त् । सः । ए॒ताम् । पु॒न॒श्चि॒तिमिति॑ पुनः - चि॒तिम् । अ॒प॒श्य॒त् । ताम् । अ॒चि॒नु॒त॒ । तया᳚ । वै । सः । आ॒द्‌र्ध्नो॒त् । यत् । पु॒न॒श्चि॒तिमिति॑ पुनः - चि॒तिम् । चि॒नु॒ते । ऋद्ध्यै᳚ ॥  \newline


\textbf{Krama Paata} \newline

ऋ॒द्ध्नोति॒ सः । स पु॑नरा॒धेय᳚म् । पु॒न॒रा॒धेय॒मा । पु॒न॒रा॒धेय॒मिति॑ पुनः - आ॒धेय᳚म् । आ ध॑त्ते । ध॒त्ते॒ यः । यो᳚ऽग्निम् । अ॒ग्निम् चि॒त्वा । चि॒त्वा न । नर्द्ध्नोति॑ । ऋ॒द्ध्नोति॒ सः । स पु॑नश्चि॒तिम् । पु॒न॒श्चि॒तिम् चि॑नुते । पु॒न॒श्चि॒तिमिति॑ पुनः - चि॒तिम् । चि॒नु॒ते॒ यत् । यत् पु॑नश्चि॒तिम् । पु॒न॒श्चि॒तिम् चि॑नु॒ते । पु॒न॒श्चि॒तिमिति॑ पुनः - चि॒तिम् । चि॒नु॒त ऋद्ध्यै᳚ । ऋद्ध्या॒ अथो᳚ । अथो॒ खलु॑ । अथो॒ इत्यथो᳚ । खल्वा॑हुः । आ॒हु॒र् न । न चे॑त॒व्या᳚ । चे॒त॒व्येति॑ । इति॑ रु॒द्रः । रु॒द्रो वै । वा ए॒षः । ए॒ष यत् । यद॒ग्निः । अ॒ग्निर् यथा᳚ । यथा᳚ व्या॒घ्रम् । व्या॒घ्रꣳ सु॒प्तम् । सु॒प्तम् बो॒धय॑ति । बो॒धय॑ति ता॒दृक् । ता॒दृगे॒व । ए॒व तत् । तदथो᳚ । अथो॒ खलु॑ । अथो॒ इत्यथो᳚ । खल्वा॑हुः । आ॒हु॒श्चे॒त॒व्या᳚ । चे॒त॒व्येति॑ । इति॒ यथा᳚ । यथा॒ वसी॑याꣳसम् । वसी॑याꣳसम् भाग॒धेये॑न । भा॒ग॒धेये॑न बो॒धय॑ति । भा॒ग॒धेये॒नेति॑ भाग - धेये॑न । बो॒धय॑ति ता॒दृक् । ता॒दृगे॒व । ए॒व तत् । तन् मनुः॑ । मनु॑र॒ग्निम् । अ॒ग्निम॑चिनुत ( ) । अ॒चि॒नु॒त॒ तेन॑ । तेन॒ न । नार्द्ध्नो᳚त् । आ॒र्द्ध्नो॒थ् सः । स ए॒ताम् । ए॒ताम् पु॑नश्चि॒तिम् । पु॒न॒श्चि॒तिम॑पश्यत् । पु॒न॒श्चि॒तिमिति॑ पुनः - चि॒तिम् । अ॒प॒श्य॒त् ताम् । ताम॑चिनुत । अ॒चि॒नु॒त॒ तया᳚ । तया॒ वै । वै सः । स आ᳚र्द्ध्नोत् । आ॒र्द्ध्नो॒द् यत् । यत् पु॑नश्चि॒तिम् । पु॒न॒श्चि॒तिम् चि॑नु॒ते । पु॒न॒श्चि॒तिमिति॑ पुनः - चि॒तिम् । चि॒नु॒त ऋद्ध्यै᳚ । ऋद्ध्या॒ इत्यृद्ध्यै᳚ । \newline

\textbf{Jatai Paata} \newline

1. ऋ॒द्ध्नोति॒ स स ऋ॒द्ध्नो त्यृ॒द्ध्नोति॒ सः । \newline
2. स पु॑नरा॒धेय॑म् पुनरा॒धेयꣳ॒॒ स स पु॑नरा॒धेय᳚म् । \newline
3. पु॒न॒रा॒धेय॒ मा पु॑नरा॒धेय॑म् पुनरा॒धेय॒ मा । \newline
4. पु॒न॒रा॒धेय॒मिति॑ पुनः - आ॒धेय᳚म् । \newline
5. आ ध॑त्ते धत्त॒ आ ध॑त्ते । \newline
6. ध॒त्ते॒ यो यो ध॑त्ते धत्ते॒ यः । \newline
7. यो᳚ ऽग्नि म॒ग्निं ॅयो यो᳚ ऽग्निम् । \newline
8. अ॒ग्निम् चि॒त्वा चि॒त्वा ऽग्नि म॒ग्निम् चि॒त्वा । \newline
9. चि॒त्वा न न चि॒त्वा चि॒त्वा न । \newline
10. न र्द्ध्नो त्यृ॒द्ध्नोति॒ न न र्द्ध्नोति॑ । \newline
11. ऋ॒द्ध्नोति॒ स स ऋ॒द्ध्नो त्यृ॒द्ध्नोति॒ सः । \newline
12. स पु॑नश्चि॒तिम् पु॑नश्चि॒तिꣳ स स पु॑नश्चि॒तिम् । \newline
13. पु॒न॒श्चि॒तिम् चि॑नुते चिनुते पुनश्चि॒तिम् पु॑नश्चि॒तिम् चि॑नुते । \newline
14. पु॒न॒श्चि॒तिमिति॑ पुनः - चि॒तिम् । \newline
15. चि॒नु॒ते॒ यद् यच् चि॑नुते चिनुते॒ यत् । \newline
16. यत् पु॑नश्चि॒तिम् पु॑नश्चि॒तिं ॅयद् यत् पु॑नश्चि॒तिम् । \newline
17. पु॒न॒श्चि॒तिम् चि॑नु॒ते चि॑नु॒ते पु॑नश्चि॒तिम् पु॑नश्चि॒तिम् चि॑नु॒ते । \newline
18. पु॒न॒श्चि॒तिमिति॑ पुनः - चि॒तिम् । \newline
19. चि॒नु॒त ऋद्ध्या॒ ऋद्ध्यै॑ चिनु॒ते चि॑नु॒त ऋद्ध्यै᳚ । \newline
20. ऋद्ध्या॒ अथो॒ अथो॒ ऋद्ध्या॒ ऋद्ध्या॒ अथो᳚ । \newline
21. अथो॒ खलु॒ खल्वथो॒ अथो॒ खलु॑ । \newline
22. अथो॒ इत्यथो᳚ । \newline
23. खल्वा॑हु राहुः॒ खलु॒ खल्वा॑हुः । \newline
24. आ॒हु॒र् न नाहु॑ राहु॒र् न । \newline
25. न चे॑त॒व्या॑ चेत॒व्या॑ न न चे॑त॒व्या᳚ । \newline
26. चे॒त॒व्येतीति॑ चेत॒व्या॑ चेत॒व्येति॑ । \newline
27. इति॑ रु॒द्रो रु॒द्र इतीति॑ रु॒द्रः । \newline
28. रु॒द्रो वै वै रु॒द्रो रु॒द्रो वै । \newline
29. वा ए॒ष ए॒ष वै वा ए॒षः । \newline
30. ए॒ष यद् यदे॒ष ए॒ष यत् । \newline
31. यद॒ग्नि र॒ग्निर् यद् यद॒ग्निः । \newline
32. अ॒ग्निर् यथा॒ यथा॒ ऽग्नि र॒ग्निर् यथा᳚ । \newline
33. यथा᳚ व्या॒घ्रं ॅव्या॒घ्रं ॅयथा॒ यथा᳚ व्या॒घ्रम् । \newline
34. व्या॒घ्रꣳ सु॒प्तꣳ सु॒प्तं ॅव्या॒घ्रं ॅव्या॒घ्रꣳ सु॒प्तम् । \newline
35. सु॒प्तम् बो॒धय॑ति बो॒धय॑ति सु॒प्तꣳ सु॒प्तम् बो॒धय॑ति । \newline
36. बो॒धय॑ति ता॒दृक् ता॒दृग् बो॒धय॑ति बो॒धय॑ति ता॒दृक् । \newline
37. ता॒दृ गे॒वैव ता॒दृक् ता॒दृ गे॒व । \newline
38. ए॒व तत् तदे॒ वैव तत् । \newline
39. तदथो॒ अथो॒ तत् तदथो᳚ । \newline
40. अथो॒ खलु॒ खल्वथो॒ अथो॒ खलु॑ । \newline
41. अथो॒ इत्यथो᳚ । \newline
42. खल्वा॑हु राहुः॒ खलु॒ खल्वा॑हुः । \newline
43. आ॒हु॒ श्चे॒त॒व्या॑ चेत॒व्या॑ ऽऽहु राहु श्चेत॒व्या᳚ । \newline
44. चे॒त॒व्येतीति॑ चेत॒व्या॑ चेत॒व्येति॑ । \newline
45. इति॒ यथा॒ यथेतीति॒ यथा᳚ । \newline
46. यथा॒ वसी॑याꣳसं॒ ॅवसी॑याꣳसं॒ ॅयथा॒ यथा॒ वसी॑याꣳसम् । \newline
47. वसी॑याꣳसम् भाग॒धेये॑न भाग॒धेये॑न॒ वसी॑याꣳसं॒ ॅवसी॑याꣳसम् भाग॒धेये॑न । \newline
48. भा॒ग॒धेये॑न बो॒धय॑ति बो॒धय॑ति भाग॒धेये॑न भाग॒धेये॑न बो॒धय॑ति । \newline
49. भा॒ग॒धेये॒नेति॑ भाग - धेये॑न । \newline
50. बो॒धय॑ति ता॒दृक् ता॒दृग् बो॒धय॑ति बो॒धय॑ति ता॒दृक् । \newline
51. ता॒दृ गे॒वैव ता॒दृक् ता॒दृ गे॒व । \newline
52. ए॒व तत् तदे॒ वैव तत् । \newline
53. तन् मनु॒र् मनु॒ स्तत् तन् मनुः॑ । \newline
54. मनु॑ र॒ग्नि म॒ग्निम् मनु॒र् मनु॑ र॒ग्निम् । \newline
55. अ॒ग्नि म॑चिनुता चिनुता॒ग्नि म॒ग्नि म॑चिनुत । \newline
56. अ॒चि॒नु॒त॒ तेन॒ तेना॑ चिनुता चिनुत॒ तेन॑ । \newline
57. तेन॒ न न तेन॒ तेन॒ न । \newline
58. नार्द्ध्नो॑ दार्द्ध्नो॒न् न नार्द्ध्नो᳚त् । \newline
59. आ॒र्द्ध्नो॒थ् स स आ᳚र्द्ध्नो दार्द्ध्नो॒थ् सः । \newline
60. स ए॒ता मे॒ताꣳ स स ए॒ताम् । \newline
61. ए॒ताम् पु॑नश्चि॒तिम् पु॑नश्चि॒ति मे॒ता मे॒ताम् पु॑नश्चि॒तिम् । \newline
62. पु॒न॒श्चि॒ति म॑पश्य दपश्यत् पुनश्चि॒तिम् पु॑नश्चि॒ति म॑पश्यत् । \newline
63. पु॒न॒श्चि॒तिमिति॑ पुनः - चि॒तिम् । \newline
64. अ॒प॒श्य॒त् ताम् ता म॑पश्य दपश्य॒त् ताम् । \newline
65. ता म॑चिनुता चिनुत॒ ताम् ता म॑चिनुत । \newline
66. अ॒चि॒नु॒त॒ तया॒ तया॑ ऽचिनुता चिनुत॒ तया᳚ । \newline
67. तया॒ वै वै तया॒ तया॒ वै । \newline
68. वै स स वै वै सः । \newline
69. स आ᳚र्द्ध्नो दार्द्ध्नो॒थ् स स आ᳚र्द्ध्नोत् । \newline
70. आ॒र्द्ध्नो॒द् यद् यदा᳚र्द्ध्नो दार्द्ध्नो॒द् यत् । \newline
71. यत् पु॑नश्चि॒तिम् पु॑नश्चि॒तिं ॅयद् यत् पु॑नश्चि॒तिम् । \newline
72. पु॒न॒श्चि॒तिम् चि॑नु॒ते चि॑नु॒ते पु॑नश्चि॒तिम् पु॑नश्चि॒तिम् चि॑नु॒ते । \newline
73. पु॒न॒श्चि॒तिमिति॑ पुनः - चि॒तिम् । \newline
74. चि॒नु॒त ऋद्ध्या॒ ऋद्ध्यै॑ चिनु॒ते चि॑नु॒त ऋद्ध्यै᳚ । \newline
75. ऋद्ध्या॒ इत्यृद्ध्यै᳚ । \newline

\textbf{Ghana Paata } \newline

1. ऋ॒द्ध्नोति॒ स स ऋ॒द्ध्नो त्यृ॒द्ध्नोति॒ स पु॑नरा॒धेय॑म् पुनरा॒धेयꣳ॒॒ स ऋ॒द्ध्नो त्यृ॒द्ध्नोति॒ स पु॑नरा॒धेय᳚म् । \newline
2. स पु॑नरा॒धेय॑म् पुनरा॒धेयꣳ॒॒ स स पु॑नरा॒धेय॒ मा पु॑नरा॒धेयꣳ॒॒ स स पु॑नरा॒धेय॒ मा । \newline
3. पु॒न॒रा॒धेय॒ मा पु॑नरा॒धेय॑म् पुनरा॒धेय॒ मा ध॑त्ते धत्त॒ आ पु॑नरा॒धेय॑म् पुनरा॒धेय॒ मा ध॑त्ते । \newline
4. पु॒न॒रा॒धेय॒मिति॑ पुनः - आ॒धेय᳚म् । \newline
5. आ ध॑त्ते धत्त॒ आ ध॑त्ते॒ यो यो ध॑त्त॒ आ ध॑त्ते॒ यः । \newline
6. ध॒त्ते॒ यो यो ध॑त्ते धत्ते॒ यो᳚ ऽग्नि म॒ग्निं ॅयो ध॑त्ते धत्ते॒ यो᳚ ऽग्निम् । \newline
7. यो᳚ ऽग्नि म॒ग्निं ॅयो यो᳚ ऽग्निम् चि॒त्वा चि॒त्वा ऽग्निं ॅयो यो᳚ ऽग्निम् चि॒त्वा । \newline
8. अ॒ग्निम् चि॒त्वा चि॒त्वा ऽग्नि म॒ग्निम् चि॒त्वा न न चि॒त्वा ऽग्नि म॒ग्निम् चि॒त्वा न । \newline
9. चि॒त्वा न न चि॒त्वा चि॒त्वा न र्द्ध्नो त्यृ॒द्ध्नोति॒ न चि॒त्वा चि॒त्वा न र्द्ध्नोति॑ । \newline
10. न र्द्ध्नो त्यृ॒द्ध्नोति॒ न न र्द्ध्नोति॒ स स ऋ॒द्ध्नोति॒ न न र्द्ध्नोति॒ सः । \newline
11. ऋ॒द्ध्नोति॒ स स ऋ॒द्ध्नो त्यृ॒द्ध्नोति॒ स पु॑नश्चि॒तिम् पु॑नश्चि॒तिꣳ स ऋ॒द्ध्नो त्यृ॒द्ध्नोति॒ स पु॑नश्चि॒तिम् । \newline
12. स पु॑नश्चि॒तिम् पु॑नश्चि॒तिꣳ स स पु॑नश्चि॒तिम् चि॑नुते चिनुते पुनश्चि॒तिꣳ स स पु॑नश्चि॒तिम् चि॑नुते । \newline
13. पु॒न॒श्चि॒तिम् चि॑नुते चिनुते पुनश्चि॒तिम् पु॑नश्चि॒तिम् चि॑नुते॒ यद् यच् चि॑नुते पुनश्चि॒तिम् पु॑नश्चि॒तिम् चि॑नुते॒ यत् । \newline
14. पु॒न॒श्चि॒तिमिति॑ पुनः - चि॒तिम् । \newline
15. चि॒नु॒ते॒ यद् यच् चि॑नुते चिनुते॒ यत् पु॑नश्चि॒तिम् पु॑नश्चि॒तिं ॅयच् चि॑नुते चिनुते॒ यत् पु॑नश्चि॒तिम् । \newline
16. यत् पु॑नश्चि॒तिम् पु॑नश्चि॒तिं ॅयद् यत् पु॑नश्चि॒तिम् चि॑नु॒ते चि॑नु॒ते पु॑नश्चि॒तिं ॅयद् यत् पु॑नश्चि॒तिम् चि॑नु॒ते । \newline
17. पु॒न॒श्चि॒तिम् चि॑नु॒ते चि॑नु॒ते पु॑नश्चि॒तिम् पु॑नश्चि॒तिम् चि॑नु॒त ऋद्ध्या॒ ऋद्ध्यै॑ चिनु॒ते पु॑नश्चि॒तिम् पु॑नश्चि॒तिम् चि॑नु॒त ऋद्ध्यै᳚ । \newline
18. पु॒न॒श्चि॒तिमिति॑ पुनः - चि॒तिम् । \newline
19. चि॒नु॒त ऋद्ध्या॒ ऋद्ध्यै॑ चिनु॒ते चि॑नु॒त ऋद्ध्या॒ अथो॒ अथो॒ ऋद्ध्यै॑ चिनु॒ते चि॑नु॒त ऋद्ध्या॒ अथो᳚ । \newline
20. ऋद्ध्या॒ अथो॒ अथो॒ ऋद्ध्या॒ ऋद्ध्या॒ अथो॒ खलु॒ खल्वथो॒ ऋद्ध्या॒ ऋद्ध्या॒ अथो॒ खलु॑ । \newline
21. अथो॒ खलु॒ खल्वथो॒ अथो॒ खल्वा॑हु राहुः॒ खल्वथो॒ अथो॒ खल्वा॑हुः । \newline
22. अथो॒ इत्यथो᳚ । \newline
23. खल्वा॑हु राहुः॒ खलु॒ खल्वा॑हु॒र् न नाहुः॒ खलु॒ खल्वा॑हु॒र् न । \newline
24. आ॒हु॒र् न नाहु॑ राहु॒र् न चे॑त॒व्या॑ चेत॒व्या॑ नाहु॑ राहु॒र् न चे॑त॒व्या᳚ । \newline
25. न चे॑त॒व्या॑ चेत॒व्या॑ न न चे॑त॒व्ये॑ तीति॑ चेत॒व्या॑ न न चे॑त॒व्येति॑ । \newline
26. चे॒त॒व्ये॑तीति॑ चेत॒व्या॑ चेत॒व्येति॑ रु॒द्रो रु॒द्र इति॑ चेत॒व्या॑ चेत॒व्येति॑ रु॒द्रः । \newline
27. इति॑ रु॒द्रो रु॒द्र इतीति॑ रु॒द्रो वै वै रु॒द्र इतीति॑ रु॒द्रो वै । \newline
28. रु॒द्रो वै वै रु॒द्रो रु॒द्रो वा ए॒ष ए॒ष वै रु॒द्रो रु॒द्रो वा ए॒षः । \newline
29. वा ए॒ष ए॒ष वै वा ए॒ष यद् यदे॒ष वै वा ए॒ष यत् । \newline
30. ए॒ष यद् यदे॒ष ए॒ष यद॒ग्नि र॒ग्निर् यदे॒ष ए॒ष यद॒ग्निः । \newline
31. यद॒ग्नि र॒ग्निर् यद् यद॒ग्निर् यथा॒ यथा॒ ऽग्निर् यद् यद॒ग्निर् यथा᳚ । \newline
32. अ॒ग्निर् यथा॒ यथा॒ ऽग्नि र॒ग्निर् यथा᳚ व्या॒घ्रं ॅव्या॒घ्रं ॅयथा॒ ऽग्नि र॒ग्निर् यथा᳚ व्या॒घ्रम् । \newline
33. यथा᳚ व्या॒घ्रं ॅव्या॒घ्रं ॅयथा॒ यथा᳚ व्या॒घ्रꣳ सु॒प्तꣳ सु॒प्तं ॅव्या॒घ्रं ॅयथा॒ यथा᳚ व्या॒घ्रꣳ सु॒प्तम् । \newline
34. व्या॒घ्रꣳ सु॒प्तꣳ सु॒प्तं ॅव्या॒घ्रं ॅव्या॒घ्रꣳ सु॒प्तम् बो॒धय॑ति बो॒धय॑ति सु॒प्तं ॅव्या॒घ्रं ॅव्या॒घ्रꣳ सु॒प्तम् बो॒धय॑ति । \newline
35. सु॒प्तम् बो॒धय॑ति बो॒धय॑ति सु॒प्तꣳ सु॒प्तम् बो॒धय॑ति ता॒दृक् ता॒दृग् बो॒धय॑ति सु॒प्तꣳ सु॒प्तम् बो॒धय॑ति ता॒दृक् । \newline
36. बो॒धय॑ति ता॒दृक् ता॒दृग् बो॒धय॑ति बो॒धय॑ति ता॒दृ गे॒वैव ता॒दृग् बो॒धय॑ति बो॒धय॑ति ता॒दृ गे॒व । \newline
37. ता॒दृ गे॒वैव ता॒दृक् ता॒दृ गे॒व तत् तदे॒व ता॒दृक् ता॒दृ गे॒व तत् । \newline
38. ए॒व तत् तदे॒ वैव तदथो॒ अथो॒ तदे॒ वैव तदथो᳚ । \newline
39. तदथो॒ अथो॒ तत् तदथो॒ खलु॒ खल्वथो॒ तत् तदथो॒ खलु॑ । \newline
40. अथो॒ खलु॒ खल्वथो॒ अथो॒ खल्वा॑हु राहुः॒ खल्वथो॒ अथो॒ खल्वा॑हुः । \newline
41. अथो॒ इत्यथो᳚ । \newline
42. खल्वा॑हु राहुः॒ खलु॒ खल्वा॑हु श्चेत॒व्या॑ चेत॒व्या॑ ऽऽहुः॒ खलु॒ खल्वा॑हु श्चेत॒व्या᳚ । \newline
43. आ॒हु॒ श्चे॒त॒व्या॑ चेत॒व्या॑ ऽऽहु राहु श्चेत॒व्येतीति॑ चेत॒व्या॑ ऽऽहु राहु श्चेत॒व्येति॑ । \newline
44. चे॒त॒व्येतीति॑ चेत॒व्या॑ चेत॒व्येति॒ यथा॒ यथेति॑ चेत॒व्या॑ चेत॒व्येति॒ यथा᳚ । \newline
45. इति॒ यथा॒ यथेतीति॒ यथा॒ वसी॑याꣳसं॒ ॅवसी॑याꣳसं॒ ॅयथेतीति॒ यथा॒ वसी॑याꣳसम् । \newline
46. यथा॒ वसी॑याꣳसं॒ ॅवसी॑याꣳसं॒ ॅयथा॒ यथा॒ वसी॑याꣳसम् भाग॒धेये॑न भाग॒धेये॑न॒ वसी॑याꣳसं॒ ॅयथा॒ यथा॒ वसी॑याꣳसम् भाग॒धेये॑न । \newline
47. वसी॑याꣳसम् भाग॒धेये॑न भाग॒धेये॑न॒ वसी॑याꣳसं॒ ॅवसी॑याꣳसम् भाग॒धेये॑न बो॒धय॑ति बो॒धय॑ति भाग॒धेये॑न॒ वसी॑याꣳसं॒ ॅवसी॑याꣳसम् भाग॒धेये॑न बो॒धय॑ति । \newline
48. भा॒ग॒धेये॑न बो॒धय॑ति बो॒धय॑ति भाग॒धेये॑न भाग॒धेये॑न बो॒धय॑ति ता॒दृक् ता॒दृग् बो॒धय॑ति भाग॒धेये॑न भाग॒धेये॑न बो॒धय॑ति ता॒दृक् । \newline
49. भा॒ग॒धेये॒नेति॑ भाग - धेये॑न । \newline
50. बो॒धय॑ति ता॒दृक् ता॒दृग् बो॒धय॑ति बो॒धय॑ति ता॒दृ गे॒वैव ता॒दृग् बो॒धय॑ति बो॒धय॑ति ता॒दृ गे॒व । \newline
51. ता॒दृ गे॒वैव ता॒दृक् ता॒दृ गे॒व तत् तदे॒व ता॒दृक् ता॒दृ गे॒व तत् । \newline
52. ए॒व तत् तदे॒ वैव तन् मनु॒र् मनु॒ स्तदे॒ वैव तन् मनुः॑ । \newline
53. तन् मनु॒र् मनु॒ स्तत् तन् मनु॑ र॒ग्नि म॒ग्निम् मनु॒ स्तत् तन् मनु॑ र॒ग्निम् । \newline
54. मनु॑ र॒ग्नि म॒ग्निम् मनु॒र् मनु॑ र॒ग्नि म॑चिनुता चिनुता॒ग्निम् मनु॒र् मनु॑ र॒ग्नि म॑चिनुत । \newline
55. अ॒ग्नि म॑चिनुता चिनुता॒ग्नि म॒ग्नि म॑चिनुत॒ तेन॒ तेना॑ चिनुता॒ग्नि म॒ग्नि म॑चिनुत॒ तेन॑ । \newline
56. अ॒चि॒नु॒त॒ तेन॒ तेना॑ चिनुता चिनुत॒ तेन॒ न न तेना॑ चिनुता चिनुत॒ तेन॒ न । \newline
57. तेन॒ न न तेन॒ तेन॒ नार्द्ध्नो॑ दार्द्ध्नो॒न् न तेन॒ तेन॒ नार्द्ध्नो᳚त् । \newline
58. नार्द्ध्नो॑ दार्द्ध्नो॒न् न नार्द्ध्नो॒थ् स स आ᳚र्द्ध्नो॒न् न नार्द्ध्नो॒थ् सः । \newline
59. आ॒र्द्ध्नो॒थ् स स आ᳚र्द्ध्नो दार्द्ध्नो॒थ् स ए॒ता मे॒ताꣳ स आ᳚र्द्ध्नो दार्द्ध्नो॒थ् स ए॒ताम् । \newline
60. स ए॒ता मे॒ताꣳ स स ए॒ताम् पु॑नश्चि॒तिम् पु॑नश्चि॒ति मे॒ताꣳ स स ए॒ताम् पु॑नश्चि॒तिम् । \newline
61. ए॒ताम् पु॑नश्चि॒तिम् पु॑नश्चि॒ति मे॒ता मे॒ताम् पु॑नश्चि॒ति म॑पश्य दपश्यत् पुनश्चि॒ति मे॒ता मे॒ताम् पु॑नश्चि॒ति म॑पश्यत् । \newline
62. पु॒न॒श्चि॒ति म॑पश्य दपश्यत् पुनश्चि॒तिम् पु॑नश्चि॒ति म॑पश्य॒त् ताम् ता म॑पश्यत् पुनश्चि॒तिम् पु॑नश्चि॒ति म॑पश्य॒त् ताम् । \newline
63. पु॒न॒श्चि॒तिमिति॑ पुनः - चि॒तिम् । \newline
64. अ॒प॒श्य॒त् ताम् ता म॑पश्य दपश्य॒त् ता म॑चिनुता चिनुत॒ ता म॑पश्य दपश्य॒त् ता म॑चिनुत । \newline
65. ता म॑चिनुता चिनुत॒ ताम् ता म॑चिनुत॒ तया॒ तया॑ ऽचिनुत॒ ताम् ता म॑चिनुत॒ तया᳚ । \newline
66. अ॒चि॒नु॒त॒ तया॒ तया॑ ऽचिनुता चिनुत॒ तया॒ वै वै तया॑ ऽचिनुता चिनुत॒ तया॒ वै । \newline
67. तया॒ वै वै तया॒ तया॒ वै स स वै तया॒ तया॒ वै सः । \newline
68. वै स स वै वै स आ᳚र्द्ध्नो दार्द्ध्नो॒थ् स वै वै स आ᳚र्द्ध्नोत् । \newline
69. स आ᳚र्द्ध्नो दार्द्ध्नो॒थ् स स आ᳚र्द्ध्नो॒द् यद् यदा᳚र्द्ध्नो॒थ् स स आ᳚र्द्ध्नो॒द् यत् । \newline
70. आ॒र्द्ध्नो॒द् यद् यदा᳚र्द्ध्नो दार्द्ध्नो॒द् यत् पु॑नश्चि॒तिम् पु॑नश्चि॒तिं ॅयदा᳚र्द्ध्नो दार्द्ध्नो॒द् यत् पु॑नश्चि॒तिम् । \newline
71. यत् पु॑नश्चि॒तिम् पु॑नश्चि॒तिं ॅयद् यत् पु॑नश्चि॒तिम् चि॑नु॒ते चि॑नु॒ते पु॑नश्चि॒तिं ॅयद् यत् पु॑नश्चि॒तिम् चि॑नु॒ते । \newline
72. पु॒न॒श्चि॒तिम् चि॑नु॒ते चि॑नु॒ते पु॑नश्चि॒तिम् पु॑नश्चि॒तिम् चि॑नु॒त ऋद्ध्या॒ ऋद्ध्यै॑ चिनु॒ते पु॑नश्चि॒तिम् पु॑नश्चि॒तिम् चि॑नु॒त ऋद्ध्यै᳚ । \newline
73. पु॒न॒श्चि॒तिमिति॑ पुनः - चि॒तिम् । \newline
74. चि॒नु॒त ऋद्ध्या॒ ऋद्ध्यै॑ चिनु॒ते चि॑नु॒त ऋद्ध्यै᳚ । \newline
75. ऋद्ध्या॒ इत्यृद्ध्यै᳚ । \newline
\pagebreak
\markright{ TS 5.4.11.1  \hfill https://www.vedavms.in \hfill}

\section{ TS 5.4.11.1 }

\textbf{TS 5.4.11.1 } \newline
\textbf{Samhita Paata} \newline

छ॒न्द॒श्चितं॑ चिन्वीत प॒शुका॑मः प॒शवो॒ वै छन्दाꣳ॑सि पशु॒माने॒व भ॑वति श्येन॒चितं॑ चिन्वीत सुव॒र्गका॑मः श्ये॒नो वै वय॑सां॒ पति॑ष्ठः श्ये॒न ए॒व भू॒त्वा सु॑व॒र्गं ॅलो॒कं प॑तति कङ्क॒चितं॑ चिन्वीत॒ यः का॒मये॑त शीर्.ष॒ण्वान॒मुष्मि॑न् ॅलो॒के स्या॒मिति॑ शीर्.ष॒ण्वाने॒वाऽमुष्मि॑न् ॅलो॒के भ॑वत्यलज॒चितं॑ चिन्वीत॒ चतु॑स्सीतं प्रति॒ष्ठाका॑म॒श्चत॑स्रो॒ दिशो॑ दि॒क्ष्वे॑व प्रति॑ तिष्ठति प्रौग॒चितं॑ चिन्वीत॒ भ्रातृ॑व्यवा॒न् प्रै - [  ] \newline

\textbf{Pada Paata} \newline

छ॒न्द॒श्चित॒मिति॑ छन्दः - चित᳚म् । चि॒न्वी॒त॒ । प॒शुका॑म॒ इति॑ प॒शु - का॒मः॒ । प॒शवः॑ । वै । छन्दाꣳ॑सि । प॒शु॒मानिति॑ पशु-मान् । ए॒व । भ॒व॒ति॒ । श्ये॒न॒चित॒मिति॑ श्येन - चित᳚म् । चि॒न्वी॒त॒ । सु॒व॒र्गका॑म॒ इति॑ सुव॒र्ग - का॒मः॒ । श्ये॒नः । वै । वय॑साम् । पति॑ष्ठः । श्ये॒नः । ए॒व । भू॒त्वा । सु॒व॒र्गमिति॑ सुवः - गम् । लो॒कम् । प॒त॒ति॒ । क॒ङ्क॒चित॒मिति॑ कङ्क - चित᳚म् । चि॒न्वी॒त॒ । यः । का॒मये॑त । शी॒र्.॒ष॒ण्वानिति॑ शीर्.षण् - वान् । अ॒मुष्मिन्न्॑ । लो॒के । स्या॒म् । इति॑ । शी॒र्.॒ष॒ण्वानिति॑ शीर्.षण् - वान् । ए॒व । अ॒मुष्मिन्न्॑ । लो॒के । भ॒व॒ति॒ । अ॒ल॒ज॒चित॒मित्य॑लज - चित᳚म् । चि॒न्वी॒त॒ । चतु॑स्सीत॒मिति॒ चतुः॑ - सी॒त॒म् । प्र॒ति॒ष्ठाका॑म॒ इति॑ प्रति॒ष्ठा-का॒मः॒ । चत॑स्रः । दिशः॑ । दि॒क्षु । ए॒व । प्रतीति॑ । ति॒ष्ठ॒ति॒ । प्र॒उ॒ग॒चित॒मिति॑ प्र‌उग - चित᳚म् । चि॒न्वी॒त॒ । भ्रातृ॑व्यवा॒निति॒ भ्रातृ॑व्य - वान् । प्रेति॑ ।  \newline


\textbf{Krama Paata} \newline

छ॒न्द॒श्चित॑म् चिन्वीत । छ॒न्द॒श्चित॒मिति॑ छन्दः - चित᳚म् । चि॒न्वी॒त॒ प॒शुका॑मः । प॒शुका॑मः प॒शवः॑ । प॒शुका॑म॒ इति॑ प॒शु - का॒मः॒ । प॒शवो॒ वै । वै छन्दाꣳ॑सि । छन्दाꣳ॑सि पशु॒मान् । प॒शु॒माने॒व । प॒शु॒मानिति॑ पशु - मान् । ए॒व भ॑वति । भ॒व॒ति॒ श्ये॒न॒चित᳚म् । श्ये॒न॒चित॑म् चिन्वीत । श्ये॒न॒चित॒मिति॑ श्येन - चित᳚म् । चि॒न्वी॒त॒ सु॒व॒र्गका॑मः । सु॒व॒र्ग॒का॑मः श्ये॒नः । सु॒व॒र्गका॑म॒ इति॑ सुव॒र्ग - का॒मः॒ । श्ये॒नो वै । वै वय॑साम् । वय॑सा॒म् पति॑ष्ठः । पति॑ष्ठः श्ये॒नः । श्ये॒न ए॒व । ए॒व भू॒त्वा । भू॒त्वा सु॑व॒र्गम् । सु॒व॒र्गम् ॅलो॒कम् । सु॒व॒र्गमिति॑ सुवः - गम् । लो॒कम् प॑तति । प॒त॒ति॒ कङ्क॒चित᳚म् । क॒ङ्क॒चित॑म् चिन्वीत । क॒ङ्क॒चिति॒मिति॑ कङ्क - चित᳚म् । चि॒न्वी॒त॒ यः । यः का॒मये॑त । का॒मये॑त शीर्.ष॒ण्वान् । शी॒र्॒.ष॒ण्वान॒मुष्मिन्न्॑ । शी॒र्॒.ष॒ण्वानिति॑ शीर्.षण्ण् - वान् । अ॒मुष्मि॑न् ॅलो॒के । लो॒के स्या᳚म् । स्या॒मिति॑ । इति॑ शीर्.ष॒ण्वान् । शी॒र्॒.ष॒ण्वाने॒व । शी॒र्॒.ष॒ण्वानिति॑ शीर्.षण्ण् - वान् । ए॒वामुष्मिन्न्॑ । अ॒मुष्मि॑न् ॅलो॒के । लो॒के भ॑वति । भ॒व॒त्य॒ल॒ज॒चित᳚म् । अ॒ल॒ज॒चित॑म् चिन्वीत । अ॒ल॒ज॒चित॒मित्य॑लज - चित᳚म् । चि॒न्वी॒त॒ चतु॑स्सीतम् । चतु॑स्सीतम् प्रति॒ष्ठाका॑मः । चतु॑स्सीत॒मिति॒ चतुः॑ - सी॒त॒म् । प्र॒ति॒ष्ठाका॑म॒श्चत॑स्रः । प्र॒ति॒ष्ठाका॑म॒ इति॑ प्रति॒ष्ठा - का॒मः॒ । चत॑स्रो॒ दिशः॑ । दिशो॑ दि॒क्षु । दि॒क्ष्वे॑व । ए॒व प्रति॑ । प्रति॑ तिष्ठति । ति॒ष्ठ॒ति॒ प्र॒ उ॒ग॒चित᳚म् । प्र॒ उ॒ग॒चित॑म् चिन्वीत । प्र॒ उ॒ग॒चित॒मिति॑ प्र उग - चित᳚म् । चि॒न्वी॒त॒ भ्रातृ॑व्यवान् । भ्रातृ॑व्यवा॒न् प्र । भ्रातृ॑व्यवा॒निति॒ भ्रातृ॑व्य - वा॒न्॒ । प्रैव \newline

\textbf{Jatai Paata} \newline

1. छ॒न्द॒श्चित॑म् चिन्वीत चिन्वीत छन्द॒श्चित॑म् छन्द॒श्चित॑म् चिन्वीत । \newline
2. छ॒न्द॒श्चित॒मिति॑ छन्दः - चित᳚म् । \newline
3. चि॒न्वी॒त॒ प॒शुका॑मः प॒शुका॑म श्चिन्वीत चिन्वीत प॒शुका॑मः । \newline
4. प॒शुका॑मः प॒शवः॑ प॒शवः॑ प॒शुका॑मः प॒शुका॑मः प॒शवः॑ । \newline
5. प॒शुका॑म॒ इति॑ प॒शु - का॒मः॒ । \newline
6. प॒शवो॒ वै वै प॒शवः॑ प॒शवो॒ वै । \newline
7. वै छन्दाꣳ॑सि॒ छन्दाꣳ॑सि॒ वै वै छन्दाꣳ॑सि । \newline
8. छन्दाꣳ॑सि पशु॒मान् प॑शु॒मान् छन्दाꣳ॑सि॒ छन्दाꣳ॑सि पशु॒मान् । \newline
9. प॒शु॒मा ने॒वैव प॑शु॒मान् प॑शु॒मा ने॒व । \newline
10. प॒शु॒मानिति॑ पशु - मान् । \newline
11. ए॒व भ॑वति भव त्ये॒वैव भ॑वति । \newline
12. भ॒व॒ति॒ श्ये॒न॒चितꣳ॑ श्येन॒चित॑म् भवति भवति श्येन॒चित᳚म् । \newline
13. श्ये॒न॒चित॑म् चिन्वीत चिन्वीत श्येन॒चितꣳ॑ श्येन॒चित॑म् चिन्वीत । \newline
14. श्ये॒न॒चित॒मिति॑ श्येन - चित᳚म् । \newline
15. चि॒न्वी॒त॒ सु॒व॒र्गका॑मः सुव॒र्गका॑ मश्चिन्वीत चिन्वीत सुव॒र्गका॑मः । \newline
16. सु॒व॒र्गका॑मः श्ये॒नः श्ये॒नः सु॑व॒र्गका॑मः सुव॒र्गका॑मः श्ये॒नः । \newline
17. सु॒व॒र्गका॑म॒ इति॑ सुव॒र्ग - का॒मः॒ । \newline
18. श्ये॒नो वै वै श्ये॒नः श्ये॒नो वै । \newline
19. वै वय॑सां॒ ॅवय॑सां॒ ॅवै वै वय॑साम् । \newline
20. वय॑सा॒म् पति॑ष्ठः॒ पति॑ष्ठो॒ वय॑सां॒ ॅवय॑सा॒म् पति॑ष्ठः । \newline
21. पति॑ष्ठः श्ये॒नः श्ये॒नः पति॑ष्ठः॒ पति॑ष्ठः श्ये॒नः । \newline
22. श्ये॒न ए॒वैव श्ये॒नः श्ये॒न ए॒व । \newline
23. ए॒व भू॒त्वा भू॒त्वै वैव भू॒त्वा । \newline
24. भू॒त्वा सु॑व॒र्गꣳ सु॑व॒र्गम् भू॒त्वा भू॒त्वा सु॑व॒र्गम् । \newline
25. सु॒व॒र्गम् ॅलो॒कम् ॅलो॒कꣳ सु॑व॒र्गꣳ सु॑व॒र्गम् ॅलो॒कम् । \newline
26. सु॒व॒र्गमिति॑ सुवः - गम् । \newline
27. लो॒कम् प॑तति पतति लो॒कम् ॅलो॒कम् प॑तति । \newline
28. प॒त॒ति॒ क॒ङ्क॒चित॑म् कङ्क॒चित॑म् पतति पतति कङ्क॒चित᳚म् । \newline
29. क॒ङ्क॒चित॑म् चिन्वीत चिन्वीत कङ्क॒चित॑म् कङ्क॒चित॑म् चिन्वीत । \newline
30. क॒ङ्क॒चित॒मिति॑ कङ्क - चित᳚म् । \newline
31. चि॒न्वी॒त॒ यो य श्चि॑न्वीत चिन्वीत॒ यः । \newline
32. यः का॒मये॑त का॒मये॑त॒ यो यः का॒मये॑त । \newline
33. का॒मये॑त शीर्.ष॒ण्वाञ् छी॑र्.ष॒ण्वान् का॒मये॑त का॒मये॑त शीर्.ष॒ण्वान् । \newline
34. शी॒र्॒.ष॒ण्वा न॒मुष्मि॑न् न॒मुष्मि॑ञ् छीर्.ष॒ण्वाञ् छी॑र्.ष॒ण्वा न॒मुष्मिन्न्॑ । \newline
35. शी॒र्.॒ष॒ण्वानिति॑ शीर्.षण् - वान् । \newline
36. अ॒मुष्मि॑न् ॅलो॒के लो॒के॑ ऽमुष्मि॑न् न॒मुष्मि॑न् ॅलो॒के । \newline
37. लो॒के स्याꣳ॑ स्याम् ॅलो॒के लो॒के स्या᳚म् । \newline
38. स्या॒ मितीति॑ स्याꣳ स्या॒ मिति॑ । \newline
39. इति॑ शीर्.ष॒ण्वाञ् छी॑र्.ष॒ण्वा नितीति॑ शीर्.ष॒ण्वान् । \newline
40. शी॒र्॒.ष॒ण्वा ने॒वैव शी॑र्.ष॒ण्वाञ् छी॑र्.ष॒ण्वा ने॒व । \newline
41. शी॒र्.॒ष॒ण्वानिति॑ शीर्.षण् - वान् । \newline
42. ए॒वामुष्मि॑न् न॒मुष्मि॑न् ने॒वै वामुष्मिन्न्॑ । \newline
43. अ॒मुष्मि॑न् ॅलो॒के लो॒के॑ ऽमुष्मि॑न् न॒मुष्मि॑न् ॅलो॒के । \newline
44. लो॒के भ॑वति भवति लो॒के लो॒के भ॑वति । \newline
45. भ॒व॒ त्य॒ल॒ज॒चित॑ मलज॒चित॑म् भवति भव त्यलज॒चित᳚म् । \newline
46. अ॒ल॒ज॒चित॑म् चिन्वीत चिन्वीता लज॒चित॑ मलज॒चित॑म् चिन्वीत । \newline
47. अ॒ल॒ज॒चित॒मित्य॑लज - चित᳚म् । \newline
48. चि॒न्वी॒त॒ चतु॑स्सीत॒म् चतु॑स्सीतम् चिन्वीत चिन्वीत॒ चतु॑स्सीतम् । \newline
49. चतु॑स्सीतम् प्रति॒ष्ठाका॑मः प्रति॒ष्ठाका॑म॒ श्चतु॑स्सीत॒म् चतु॑स्सीतम् प्रति॒ष्ठाका॑मः । \newline
50. चतु॑स्सीत॒मिति॒ चतुः॑ - सी॒त॒म् । \newline
51. प्र॒ति॒ष्ठाका॑म॒ श्चत॑स्र॒ श्चत॑स्रः प्रति॒ष्ठाका॑मः प्रति॒ष्ठाका॑म॒ श्चत॑स्रः । \newline
52. प्र॒ति॒ष्ठाका॑म॒ इति॑ प्रति॒ष्ठा - का॒मः॒ । \newline
53. चत॑स्रो॒ दिशो॒ दिश॒ श्चत॑स्र॒ श्चत॑स्रो॒ दिशः॑ । \newline
54. दिशो॑ दि॒क्षु दि॒क्षु दिशो॒ दिशो॑ दि॒क्षु । \newline
55. दि॒क्ष्वे॑ वैव दि॒क्षु दि॒क्ष्वे॑व । \newline
56. ए॒व प्रति॒ प्रत्ये॒ वैव प्रति॑ । \newline
57. प्रति॑ तिष्ठति तिष्ठति॒ प्रति॒ प्रति॑ तिष्ठति । \newline
58. ति॒ष्ठ॒ति॒ प्र॒उ॒ग॒चित॑म् प्रौग॒चित॑म् तिष्ठति तिष्ठति प्र‌उग॒चित᳚म् । \newline
59. प्र॒उ॒ग॒चित॑म् चिन्वीत चिन्वीत प्र‌उग॒चित॑म् प्र‌उग॒चित॑म् चिन्वीत । \newline
60. प्र॒उ॒ग॒चित॒मिति॑ प्र‌उग - चित᳚म् । \newline
61. चि॒न्वी॒त॒ भ्रातृ॑व्यवा॒न् भ्रातृ॑व्यवाꣳ श्चिन्वीत चिन्वीत॒ भ्रातृ॑व्यवान् । \newline
62. भ्रातृ॑व्यवा॒न् प्र प्र भ्रातृ॑व्यवा॒न् भ्रातृ॑व्यवा॒न् प्र । \newline
63. भ्रातृ॑व्यवा॒निति॒ भ्रातृ॑व्य - वा॒न् । \newline
64. प्रैवैव प्र प्रैव । \newline

\textbf{Ghana Paata } \newline

1. छ॒न्द॒श्चित॑म् चिन्वीत चिन्वीत छन्द॒श्चित॑म् छन्द॒श्चित॑म् चिन्वीत प॒शुका॑मः प॒शुका॑म श्चिन्वीत छन्द॒श्चित॑म् छन्द॒श्चित॑म् चिन्वीत प॒शुका॑मः । \newline
2. छ॒न्द॒श्चित॒मिति॑ छन्दः - चित᳚म् । \newline
3. चि॒न्वी॒त॒ प॒शुका॑मः प॒शुका॑म श्चिन्वीत चिन्वीत प॒शुका॑मः प॒शवः॑ प॒शवः॑ प॒शुका॑म श्चिन्वीत चिन्वीत प॒शुका॑मः प॒शवः॑ । \newline
4. प॒शुका॑मः प॒शवः॑ प॒शवः॑ प॒शुका॑मः प॒शुका॑मः प॒शवो॒ वै वै प॒शवः॑ प॒शुका॑मः प॒शुका॑मः प॒शवो॒ वै । \newline
5. प॒शुका॑म॒ इति॑ प॒शु - का॒मः॒ । \newline
6. प॒शवो॒ वै वै प॒शवः॑ प॒शवो॒ वै छन्दाꣳ॑सि॒ छन्दाꣳ॑सि॒ वै प॒शवः॑ प॒शवो॒ वै छन्दाꣳ॑सि । \newline
7. वै छन्दाꣳ॑सि॒ छन्दाꣳ॑सि॒ वै वै छन्दाꣳ॑सि पशु॒मान् प॑शु॒मान् छन्दाꣳ॑सि॒ वै वै छन्दाꣳ॑सि पशु॒मान् । \newline
8. छन्दाꣳ॑सि पशु॒मान् प॑शु॒मान् छन्दाꣳ॑सि॒ छन्दाꣳ॑सि पशु॒मा ने॒वैव प॑शु॒मान् छन्दाꣳ॑सि॒ छन्दाꣳ॑सि पशु॒मा ने॒व । \newline
9. प॒शु॒मा ने॒वैव प॑शु॒मान् प॑शु॒मा ने॒व भ॑वति भवत्ये॒व प॑शु॒मान् प॑शु॒मा ने॒व भ॑वति । \newline
10. प॒शु॒मानिति॑ पशु - मान् । \newline
11. ए॒व भ॑वति भव त्ये॒वैव भ॑वति श्येन॒चितꣳ॑ श्येन॒चित॑म् भव त्ये॒वैव भ॑वति श्येन॒चित᳚म् । \newline
12. भ॒व॒ति॒ श्ये॒न॒चितꣳ॑ श्येन॒चित॑म् भवति भवति श्येन॒चित॑म् चिन्वीत चिन्वीत श्येन॒चित॑म् भवति भवति श्येन॒चित॑म् चिन्वीत । \newline
13. श्ये॒न॒चित॑म् चिन्वीत चिन्वीत श्येन॒चितꣳ॑ श्येन॒चित॑म् चिन्वीत सुव॒र्गका॑मः सुव॒र्गका॑म श्चिन्वीत श्येन॒चितꣳ॑ श्येन॒चित॑म् चिन्वीत सुव॒र्गका॑मः । \newline
14. श्ये॒न॒चित॒मिति॑ श्येन - चित᳚म् । \newline
15. चि॒न्वी॒त॒ सु॒व॒र्गका॑मः सुव॒र्गका॑म श्चिन्वीत चिन्वीत सुव॒र्गका॑मः श्ये॒नः श्ये॒नः सु॑व॒र्गका॑म श्चिन्वीत चिन्वीत सुव॒र्गका॑मः श्ये॒नः । \newline
16. सु॒व॒र्गका॑मः श्ये॒नः श्ये॒नः सु॑व॒र्गका॑मः सुव॒र्गका॑मः श्ये॒नो वै वै श्ये॒नः सु॑व॒र्गका॑मः सुव॒र्गका॑मः श्ये॒नो वै । \newline
17. सु॒व॒र्गका॑म॒ इति॑ सुव॒र्ग - का॒मः॒ । \newline
18. श्ये॒नो वै वै श्ये॒नः श्ये॒नो वै वय॑सां॒ ॅवय॑सां॒ ॅवै श्ये॒नः श्ये॒नो वै वय॑साम् । \newline
19. वै वय॑सां॒ ॅवय॑सां॒ ॅवै वै वय॑सा॒म् पति॑ष्ठः॒ पति॑ष्ठो॒ वय॑सां॒ ॅवै वै वय॑सा॒म् पति॑ष्ठः । \newline
20. वय॑सा॒म् पति॑ष्ठः॒ पति॑ष्ठो॒ वय॑सां॒ ॅवय॑सा॒म् पति॑ष्ठः श्ये॒नः श्ये॒नः पति॑ष्ठो॒ वय॑सां॒ ॅवय॑सा॒म् पति॑ष्ठः श्ये॒नः । \newline
21. पति॑ष्ठः श्ये॒नः श्ये॒नः पति॑ष्ठः॒ पति॑ष्ठः श्ये॒न ए॒वैव श्ये॒नः पति॑ष्ठः॒ पति॑ष्ठः श्ये॒न ए॒व । \newline
22. श्ये॒न ए॒वैव श्ये॒नः श्ये॒न ए॒व भू॒त्वा भू॒त्वैव श्ये॒नः श्ये॒न ए॒व भू॒त्वा । \newline
23. ए॒व भू॒त्वा भू॒त्वै वैव भू॒त्वा सु॑व॒र्गꣳ सु॑व॒र्गम् भू॒त्वै वैव भू॒त्वा सु॑व॒र्गम् । \newline
24. भू॒त्वा सु॑व॒र्गꣳ सु॑व॒र्गम् भू॒त्वा भू॒त्वा सु॑व॒र्गम् ॅलो॒कम् ॅलो॒कꣳ सु॑व॒र्गम् भू॒त्वा भू॒त्वा सु॑व॒र्गम् ॅलो॒कम् । \newline
25. सु॒व॒र्गम् ॅलो॒कम् ॅलो॒कꣳ सु॑व॒र्गꣳ सु॑व॒र्गम् ॅलो॒कम् प॑तति पतति लो॒कꣳ सु॑व॒र्गꣳ सु॑व॒र्गम् ॅलो॒कम् प॑तति । \newline
26. सु॒व॒र्गमिति॑ सुवः - गम् । \newline
27. लो॒कम् प॑तति पतति लो॒कम् ॅलो॒कम् प॑तति कङ्क॒चित॑म् कङ्क॒चित॑म् पतति लो॒कम् ॅलो॒कम् प॑तति कङ्क॒चित᳚म् । \newline
28. प॒त॒ति॒ क॒ङ्क॒चित॑म् कङ्क॒चित॑म् पतति पतति कङ्क॒चित॑म् चिन्वीत चिन्वीत कङ्क॒चित॑म् पतति पतति कङ्क॒चित॑म् चिन्वीत । \newline
29. क॒ङ्क॒चित॑म् चिन्वीत चिन्वीत कङ्क॒चित॑म् कङ्क॒चित॑म् चिन्वीत॒ यो यश्चि॑न्वीत कङ्क॒चित॑म् कङ्क॒चित॑म् चिन्वीत॒ यः । \newline
30. क॒ङ्क॒चित॒मिति॑ कङ्क - चित᳚म् । \newline
31. चि॒न्वी॒त॒ यो यश्चि॑न्वीत चिन्वीत॒ यः का॒मये॑त का॒मये॑त॒ यश्चि॑न्वीत चिन्वीत॒ यः का॒मये॑त । \newline
32. यः का॒मये॑त का॒मये॑त॒ यो यः का॒मये॑त शीर्.ष॒ण्वाञ् छी॑र्.ष॒ण्वान् का॒मये॑त॒ यो यः का॒मये॑त शीर्.ष॒ण्वान् । \newline
33. का॒मये॑त शीर्.ष॒ण्वाञ् छी॑र्.ष॒ण्वान् का॒मये॑त का॒मये॑त शीर्.ष॒ण्वा न॒मुष्मि॑न् न॒मुष्मि॑ञ् छीर्.ष॒ण्वान् का॒मये॑त का॒मये॑त शीर्.ष॒ण्वा न॒मुष्मिन्न्॑ । \newline
34. शी॒र्॒.ष॒ण्वा न॒मुष्मि॑न् न॒मुष्मि॑ञ् छीर्.ष॒ण्वाञ् छी॑र्.ष॒ण्वा न॒मुष्मि॑न् ॅलो॒के लो॒के॑ ऽमुष्मि॑ञ् छीर्.ष॒ण्वाञ् छी॑र्.ष॒ण्वा न॒मुष्मि॑न् ॅलो॒के । \newline
35. शी॒र्.॒ष॒ण्वानिति॑ शीर्.षण् - वान् । \newline
36. अ॒मुष्मि॑न् ॅलो॒के लो॒के॑ ऽमुष्मि॑न् न॒मुष्मि॑न् ॅलो॒के स्याꣳ॑ स्याम् ॅलो॒के॑ ऽमुष्मि॑न् न॒मुष्मि॑न् ॅलो॒के स्या᳚म् । \newline
37. लो॒के स्याꣳ॑ स्याम् ॅलो॒के लो॒के स्या॒ मितीति॑ स्याम् ॅलो॒के लो॒के स्या॒ मिति॑ । \newline
38. स्या॒ मितीति॑ स्याꣳ स्या॒ मिति॑ शीर्.ष॒ण्वाञ् छी॑र्.ष॒ण्वा निति॑ स्याꣳ स्या॒ मिति॑ शीर्.ष॒ण्वान् । \newline
39. इति॑ शीर्.ष॒ण्वाञ् छी॑र्.ष॒ण्वा नितीति॑ शीर्.ष॒ण्वा ने॒वैव शी॑र्.ष॒ण्वा नितीति॑ शीर्.ष॒ण्वा ने॒व । \newline
40. शी॒र्॒.ष॒ण्वा ने॒वैव शी॑र्.ष॒ण्वाञ् छी॑र्.ष॒ण्वा ने॒वामुष्मि॑न् न॒मुष्मि॑न् ने॒व शी॑र्.ष॒ण्वाञ् छी॑र्.ष॒ण्वा ने॒वामुष्मिन्न्॑ । \newline
41. शी॒र्.॒ष॒ण्वानिति॑ शीर्.षण् - वान् । \newline
42. ए॒वामुष्मि॑न् न॒मुष्मि॑न् ने॒वैवा मुष्मि॑न् ॅलो॒के लो॒के॑ ऽमुष्मि॑न् ने॒वै वामुष्मि॑न् ॅलो॒के । \newline
43. अ॒मुष्मि॑न् ॅलो॒के लो॒के॑ ऽमुष्मि॑न् न॒मुष्मि॑न् ॅलो॒के भ॑वति भवति लो॒के॑ ऽमुष्मि॑न् न॒मुष्मि॑न् ॅलो॒के भ॑वति । \newline
44. लो॒के भ॑वति भवति लो॒के लो॒के भ॑व त्यलज॒चित॑ मलज॒चित॑म् भवति लो॒के लो॒के भ॑व त्यलज॒चित᳚म् । \newline
45. भ॒व॒ त्य॒ल॒ज॒चित॑ मलज॒चित॑म् भवति भव त्यलज॒चित॑म् चिन्वीत चिन्वीता लज॒चित॑म् भवति भव त्यलज॒चित॑म् चिन्वीत । \newline
46. अ॒ल॒ज॒चित॑म् चिन्वीत चिन्वीता लज॒चित॑ मलज॒चित॑म् चिन्वीत॒ चतु॑स्सीत॒म् चतु॑स्सीतम् चिन्वीता लज॒चित॑ मलज॒चित॑म् चिन्वीत॒ चतु॑स्सीतम् । \newline
47. अ॒ल॒ज॒चित॒मित्य॑लज - चित᳚म् । \newline
48. चि॒न्वी॒त॒ चतु॑स्सीत॒म् चतु॑स्सीतम् चिन्वीत चिन्वीत॒ चतु॑स्सीतम् प्रति॒ष्ठाका॑मः प्रति॒ष्ठाका॑म॒ श्चतु॑स्सीतम् चिन्वीत चिन्वीत॒ चतु॑स्सीतम् प्रति॒ष्ठाका॑मः । \newline
49. चतु॑स्सीतम् प्रति॒ष्ठाका॑मः प्रति॒ष्ठाका॑म॒ श्चतु॑स्सीत॒म् चतु॑स्सीतम् प्रति॒ष्ठाका॑म॒ श्चत॑स्र॒ श्चत॑स्रः प्रति॒ष्ठाका॑म॒ श्चतु॑स्सीत॒म् चतु॑स्सीतम् प्रति॒ष्ठाका॑म॒ श्चत॑स्रः । \newline
50. चतु॑स्सीत॒मिति॒ चतुः॑ - सी॒त॒म् । \newline
51. प्र॒ति॒ष्ठाका॑म॒ श्चत॑स्र॒ श्चत॑स्रः प्रति॒ष्ठाका॑मः प्रति॒ष्ठाका॑म॒ श्चत॑स्रो॒ दिशो॒ दिश॒ श्चत॑स्रः प्रति॒ष्ठाका॑मः प्रति॒ष्ठाका॑म॒ श्चत॑स्रो॒ दिशः॑ । \newline
52. प्र॒ति॒ष्ठाका॑म॒ इति॑ प्रति॒ष्ठा - का॒मः॒ । \newline
53. चत॑स्रो॒ दिशो॒ दिश॒ श्चत॑स्र॒ श्चत॑स्रो॒ दिशो॑ दि॒क्षु दि॒क्षु दिश॒ श्चत॑स्र॒ श्चत॑स्रो॒ दिशो॑ दि॒क्षु । \newline
54. दिशो॑ दि॒क्षु दि॒क्षु दिशो॒ दिशो॑ दि॒क्ष्वे॑ वैव दि॒क्षु दिशो॒ दिशो॑ दि॒क्ष्वे॑व । \newline
55. दि॒क्ष्वे॑ वैव दि॒क्षु दि॒क्ष्वे॑व प्रति॒ प्रत्ये॒व दि॒क्षु दि॒क्ष्वे॑व प्रति॑ । \newline
56. ए॒व प्रति॒ प्रत्ये॒ वैव प्रति॑ तिष्ठति तिष्ठति॒ प्रत्ये॒ वैव प्रति॑ तिष्ठति । \newline
57. प्रति॑ तिष्ठति तिष्ठति॒ प्रति॒ प्रति॑ तिष्ठति प्र‌उग॒चित॑म् प्र‌उग॒चित॑म् तिष्ठति॒ प्रति॒ प्रति॑ तिष्ठति प्र‌उग॒चित᳚म् । \newline
58. ति॒ष्ठ॒ति॒ प्र॒उ॒ग॒चित॑म् प्र‌उग॒चित॑म् तिष्ठति तिष्ठति प्र‌उग॒चित॑म् चिन्वीत चिन्वीत प्र‌उग॒चित॑म् तिष्ठति तिष्ठति प्र‌उग॒चित॑म् चिन्वीत । \newline
59. प्र॒उ॒ग॒चित॑म् चिन्वीत चिन्वीत प्र‌उग॒चित॑म् प्र‌उग॒चित॑म् चिन्वीत॒ भ्रातृ॑व्यवा॒न् भ्रातृ॑व्यवाꣳ श्चिन्वीत प्र‌उग॒चित॑म् प्र‌उग॒चित॑म् चिन्वीत॒ भ्रातृ॑व्यवान् । \newline
60. प्र॒उ॒ग॒चित॒मिति॑ प्र‌उग - चित᳚म् । \newline
61. चि॒न्वी॒त॒ भ्रातृ॑व्यवा॒न् भ्रातृ॑व्यवाꣳ श्चिन्वीत चिन्वीत॒ भ्रातृ॑व्यवा॒न् प्र प्र भ्रातृ॑व्यवाꣳ श्चिन्वीत चिन्वीत॒ भ्रातृ॑व्यवा॒न् प्र । \newline
62. भ्रातृ॑व्यवा॒न् प्र प्र भ्रातृ॑व्यवा॒न् भ्रातृ॑व्यवा॒न् प्रैवैव प्र भ्रातृ॑व्यवा॒न् भ्रातृ॑व्यवा॒न् प्रैव । \newline
63. भ्रातृ॑व्यवा॒निति॒ भ्रातृ॑व्य - वा॒न् । \newline
64. प्रैवैव प्र प्रैव भ्रातृ॑व्या॒न् भ्रातृ॑व्या ने॒व प्र प्रैव भ्रातृ॑व्यान् । \newline
\pagebreak
\markright{ TS 5.4.11.2  \hfill https://www.vedavms.in \hfill}

\section{ TS 5.4.11.2 }

\textbf{TS 5.4.11.2 } \newline
\textbf{Samhita Paata} \newline

-व भ्रातृ॑व्यान् नुदत उभ॒यतः॑ प्रौगं चिन्वीत॒यः का॒मये॑त॒ प्रजा॒तान् भ्रातृ॑व्यान् नु॒देय॒ प्रति॑ जनि॒ष्यमा॑णा॒निति॒ प्रैव जा॒तान् भ्रातृ॑व्यान् नु॒दते॒ प्रति॑ जनि॒ष्यमा॑णान् रथचक्र॒चितं॑ चिन्वीत॒ भ्रातृ॑व्यवा॒न्॒ वज्रो॒ वै रथो॒ वज्र॑मे॒व भ्रातृ॑व्येभ्यः॒ प्रह॑रति द्रोण॒चितं॑ चिन्वी॒तान्न॑कामो॒ द्रोणे॒ वा अन्नं॑ भ्रियते॒ सयो᳚न्ये॒वान्न॒मव॑ रुन्धे समू॒ह्यं॑ चिन्वीत प॒शुका॑मः पशु॒माने॒व भ॑वति - [  ] \newline

\textbf{Pada Paata} \newline

ए॒व । भ्रातृ॑व्यान् । नु॒द॒ते॒ । उ॒भ॒यतः॑ प्र‌उग॒मित्यु॑भ॒यतः॑ - प्र॒उ॒ग॒म् । चि॒न्वी॒त॒ । यः । का॒मये॑त । प्रेति॑ । जा॒तान् । भ्रातृ॑व्यान् । नु॒देय॑ । प्रतीति॑ । ज॒नि॒ष्यमा॑णान् । इति॑ । प्रेति॑ । ए॒व । जा॒तान् । भ्रातृ॑व्यान् । नु॒दते᳚ । प्रतीति॑ । ज॒नि॒ष्यमा॑णान् । र॒थ॒च॒क्र॒चित॒मिति॑ रथचक्र - चित᳚म् । चि॒न्वी॒त॒ । भ्रातृ॑व्यवा॒निति॒ भ्रातृ॑व्य - वा॒न् । वज्रः॑ । वै । रथः॑ । वज्र᳚म् । ए॒व । भ्रातृ॑व्येभ्यः । प्रेति॑ । ह॒र॒ति॒ । द्रो॒ण॒चित॒मिति॑ द्रोण - चित᳚म् । चि॒न्वी॒त॒ । अन्न॑काम॒ इत्यन्न॑-का॒मः॒ । द्रोणे᳚ । वै । अन्न᳚म् । भ्रि॒य॒ते॒ । सयो॒नीति॒ स-यो॒नि॒ । ए॒व । अन्न᳚म् । अवेति॑ । रु॒न्धे॒ । स॒मू॒ह्य॑मिति॑ सं - ऊ॒ह्य᳚म् । चि॒न्वी॒त॒ । प॒शुका॑म॒ इति॑ प॒शु - का॒मः॒ । प॒शु॒मानिति॑ पशु -मान् । ए॒व । भ॒व॒ति॒ ।  \newline


\textbf{Krama Paata} \newline

ए॒व भ्रातृ॑व्यान् । भ्रातृ॑व्यान् नुदते । नु॒द॒त॒ उ॒भ॒यतः॑ प्र उगम् । उ॒भ॒यतः॑ प्र उगम् चिन्वीत । उ॒भ॒यतः॑ 
प्र उग॒मित्यु॑भ॒यतः॑ - प्र॒ उ॒ग॒म् । चि॒न्वी॒त॒ यः । यः का॒मये॑त । का॒मये॑त॒ प्र । प्र जा॒तान् । जा॒तन् भ्रातृ॑व्यान् । भ्रातृ॑व्यान् नु॒देय॑ । नु॒देय॒ प्रति॑ । प्रति॑ जनि॒ष्यमा॑णान् । ज॒नि॒ष्यमा॑णा॒निति॑ । इति॒ प्र । प्रैव । ए॒व जा॒तान् । जा॒तान् भ्रातृ॑व्यान् । भ्रातृ॑व्यान् नु॒दते᳚ । नु॒दते॒ प्रति॑ । प्रति॑ जनि॒ष्यमा॑णान् । ज॒नि॒ष्यमा॑णान् रथचक्र॒चित᳚म् । र॒थ॒च॒क्र॒चित॑म् चिन्वीत । र॒थ॒च॒क्र॒चित॒मिति॑ रथचक्र - चित᳚म् । चि॒न्वी॒त॒ भ्रातृ॑व्यवान् । भ्रातृ॑व्यवा॒न्॒. वज्रः॑ । भ्रातृ॑व्यवा॒निति॒ भ्रातृ॑व्य - वा॒न्॒ । वज्रो॒ वै । वै रथः॑ । रथो॒ वज्र᳚म् । वज्र॑मे॒व । ए॒व भ्रातृ॑व्येभ्यः । भ्रातृ॑व्येभ्यः॒ प्र । प्र ह॑रति । ह॒र॒ति॒ द्रो॒ण॒चित᳚म् । द्रो॒ण॒चित॑म् चिन्वीत । द्रो॒ण॒चित॒मिति॑ द्रोण - चित᳚म् । चि॒न्वी॒तान्न॑कामः । अन्न॑कामो॒ द्रोणे᳚ । अन्न॑काम॒ इत्यन्न॑ - का॒मः॒ । द्रोणे॒ वै । वा अन्न᳚म् । अन्न॑म् भ्रियते । भ्रि॒य॒ते॒ सयो॑नि । सयो᳚न्ये॒व । सयो॒नीति॒ स - यो॒नि॒ । ए॒वान्न᳚म् । अन्न॒मव॑ । अव॑ रुन्धे । रु॒न्धे॒ स॒मू॒ह्य᳚म् । स॒मू॒ह्य॑म् चिन्वीत । स॒मू॒ह्य॑मिति॑ सम् - ऊ॒ह्य᳚म् । चि॒न्वी॒त॒ प॒शुका॑मः । प॒शुका॑मः पशु॒मान् । प॒शुका॑म॒ इति॑ प॒शु - का॒मः॒ । प॒शु॒माने॒व । प॒शु॒मानिति॑ पशु - मान् । ए॒व भ॑वति । भ॒व॒ति॒ प॒रि॒चा॒य्य᳚म् \newline

\textbf{Jatai Paata} \newline

1. ए॒व भ्रातृ॑व्या॒न् भ्रातृ॑व्या ने॒वैव भ्रातृ॑व्यान् । \newline
2. भ्रातृ॑व्यान् नुदते नुदते॒ भ्रातृ॑व्या॒न् भ्रातृ॑व्यान् नुदते । \newline
3. नु॒द॒त॒ उ॒भ॒यतः॑प्र‌उग मुभ॒यतः॑प्र‌उगम् नुदते नुदत उभ॒यतः॑प्र‌उगम् । \newline
4. उ॒भ॒यतः॑प्र‌उगम् चिन्वीत चिन्वी तोभ॒यतः॑प्र‌उग मुभ॒यतः॑प्र‌उगम् चिन्वीत । \newline
5. उ॒भ॒यतः॑प्र‌उग॒मित्यु॑भ॒यतः॑ - प्र॒उ॒ग॒म् । \newline
6. चि॒न्वी॒त॒ यो य श्चि॑न्वीत चिन्वीत॒ यः । \newline
7. यः का॒मये॑त का॒मये॑त॒ यो यः का॒मये॑त । \newline
8. का॒मये॑त॒ प्र प्र का॒मये॑त का॒मये॑त॒ प्र । \newline
9. प्र जा॒तान् जा॒तान् प्र प्र जा॒तान् । \newline
10. जा॒तान् भ्रातृ॑व्या॒न् भ्रातृ॑व्यान् जा॒तान् जा॒तान् भ्रातृ॑व्यान् । \newline
11. भ्रातृ॑व्यान् नु॒देय॑ नु॒देय॒ भ्रातृ॑व्या॒न् भ्रातृ॑व्यान् नु॒देय॑ । \newline
12. नु॒देय॒ प्रति॒ प्रति॑ नु॒देय॑ नु॒देय॒ प्रति॑ । \newline
13. प्रति॑ जनि॒ष्यमा॑णान् जनि॒ष्यमा॑णा॒न् प्रति॒ प्रति॑ जनि॒ष्यमा॑णान् । \newline
14. ज॒नि॒ष्यमा॑णा॒ नितीति॑ जनि॒ष्यमा॑णान् जनि॒ष्यमा॑णा॒ निति॑ । \newline
15. इति॒ प्र प्रे तीति॒ प्र । \newline
16. प्रैवैव प्र प्रैव । \newline
17. ए॒व जा॒तान् जा॒ता ने॒वैव जा॒तान् । \newline
18. जा॒तान् भ्रातृ॑व्या॒न् भ्रातृ॑व्यान् जा॒तान् जा॒तान् भ्रातृ॑व्यान् । \newline
19. भ्रातृ॑व्यान् नु॒दते॑ नु॒दते॒ भ्रातृ॑व्या॒न् भ्रातृ॑व्यान् नु॒दते᳚ । \newline
20. नु॒दते॒ प्रति॒ प्रति॑ नु॒दते॑ नु॒दते॒ प्रति॑ । \newline
21. प्रति॑ जनि॒ष्यमा॑णान् जनि॒ष्यमा॑णा॒न् प्रति॒ प्रति॑ जनि॒ष्यमा॑णान् । \newline
22. ज॒नि॒ष्यमा॑णान् रथचक्र॒चितꣳ॑ रथचक्र॒चित॑म् जनि॒ष्यमा॑णान् जनि॒ष्यमा॑णान् रथचक्र॒चित᳚म् । \newline
23. र॒थ॒च॒क्र॒चित॑म् चिन्वीत चिन्वीत रथचक्र॒चितꣳ॑ रथचक्र॒चित॑म् चिन्वीत । \newline
24. र॒थ॒च॒क्र॒चित॒मिति॑ रथचक्र - चित᳚म् । \newline
25. चि॒न्वी॒त॒ भ्रातृ॑व्यवा॒न् भ्रातृ॑व्यवाꣳश् चिन्वीत चिन्वीत॒ भ्रातृ॑व्यवान् । \newline
26. भ्रातृ॑व्यवा॒न्॒. वज्रो॒ वज्रो॒ भ्रातृ॑व्यवा॒न् भ्रातृ॑व्यवा॒न्॒. वज्रः॑ । \newline
27. भ्रातृ॑व्यवा॒निति॒ भ्रातृ॑व्य - वा॒न् । \newline
28. वज्रो॒ वै वै वज्रो॒ वज्रो॒ वै । \newline
29. वै रथो॒ रथो॒ वै वै रथः॑ । \newline
30. रथो॒ वज्रं॒ ॅवज्रꣳ॒॒ रथो॒ रथो॒ वज्र᳚म् । \newline
31. वज्र॑ मे॒वैव वज्रं॒ ॅवज्र॑ मे॒व । \newline
32. ए॒व भ्रातृ॑व्येभ्यो॒ भ्रातृ॑व्येभ्य ए॒वैव भ्रातृ॑व्येभ्यः । \newline
33. भ्रातृ॑व्येभ्यः॒ प्र प्र भ्रातृ॑व्येभ्यो॒ भ्रातृ॑व्येभ्यः॒ प्र । \newline
34. प्र ह॑रति हरति॒ प्र प्र ह॑रति । \newline
35. ह॒र॒ति॒ द्रो॒ण॒चित॑म् द्रोण॒चितꣳ॑ हरति हरति द्रोण॒चित᳚म् । \newline
36. द्रो॒ण॒चित॑म् चिन्वीत चिन्वीत द्रोण॒चित॑म् द्रोण॒चित॑म् चिन्वीत । \newline
37. द्रो॒ण॒चित॒मिति॑ द्रोण - चित᳚म् । \newline
38. चि॒न्वी॒ता न्न॑का॒मो ऽन्न॑काम श्चिन्वीत चिन्वी॒ता न्न॑कामः । \newline
39. अन्न॑कामो॒ द्रोणे॒ द्रोणे ऽन्न॑का॒मो ऽन्न॑कामो॒ द्रोणे᳚ । \newline
40. अन्न॑काम॒ इत्यन्न॑ - का॒मः॒ । \newline
41. द्रोणे॒ वै वै द्रोणे॒ द्रोणे॒ वै । \newline
42. वा अन्न॒ मन्नं॒ ॅवै वा अन्न᳚म् । \newline
43. अन्न॑म् भ्रियते भ्रिय॒ते ऽन्न॒ मन्न॑म् भ्रियते । \newline
44. भ्रि॒य॒ते॒ सयो॑नि॒ सयो॑नि भ्रियते भ्रियते॒ सयो॑नि । \newline
45. सयो᳚ न्ये॒वैव सयो॑नि॒ सयो᳚ न्ये॒व । \newline
46. सयो॒नीति॒ स - यो॒नि॒ । \newline
47. ए॒वान्न॒ मन्न॑ मे॒वै वान्न᳚म् । \newline
48. अन्न॒ मवा वान्न॒ मन्न॒ मव॑ । \newline
49. अव॑ रुन्धे रु॒न्धे ऽवाव॑ रुन्धे । \newline
50. रु॒न्धे॒ स॒मू॒ह्यꣳ॑ समू॒ह्यꣳ॑ रुन्धे रुन्धे समू॒ह्य᳚म् । \newline
51. स॒मू॒ह्य॑म् चिन्वीत चिन्वीत समू॒ह्यꣳ॑ समू॒ह्य॑म् चिन्वीत । \newline
52. स॒मू॒ह्य॑मिति॑ सं - ऊ॒ह्य᳚म् । \newline
53. चि॒न्वी॒त॒ प॒शुका॑मः प॒शुका॑म श्चिन्वीत चिन्वीत प॒शुका॑मः । \newline
54. प॒शुका॑मः पशु॒मान् प॑शु॒मान् प॒शुका॑मः प॒शुका॑मः पशु॒मान् । \newline
55. प॒शुका॑म॒ इति॑ प॒शु - का॒मः॒ । \newline
56. प॒शु॒मा ने॒वैव प॑शु॒मान् प॑शु॒मा ने॒व । \newline
57. प॒शु॒मानिति॑ पशु - मान् । \newline
58. ए॒व भ॑वति भव त्ये॒वैव भ॑वति । \newline
59. भ॒व॒ति॒ प॒रि॒चा॒य्य॑म् परिचा॒य्य॑म् भवति भवति परिचा॒य्य᳚म् । \newline

\textbf{Ghana Paata } \newline

1. ए॒व भ्रातृ॑व्या॒न् भ्रातृ॑व्या ने॒वैव भ्रातृ॑व्यान् नुदते नुदते॒ भ्रातृ॑व्या ने॒वैव भ्रातृ॑व्यान् नुदते । \newline
2. भ्रातृ॑व्यान् नुदते नुदते॒ भ्रातृ॑व्या॒न् भ्रातृ॑व्यान् नुदत उभ॒यतः॑प्र‌उग मुभ॒यतः॑प्र‌उगम् नुदते॒ भ्रातृ॑व्या॒न् भ्रातृ॑व्यान् नुदत उभ॒यतः॑प्र‌उगम् । \newline
3. नु॒द॒त॒ उ॒भ॒यतः॑प्र‌उग मुभ॒यतः॑प्र‌उगम् नुदते नुदत उभ॒यतः॑प्र‌उगम् चिन्वीत चिन्वी तोभ॒यतः॑प्र‌उगम् नुदते नुदत उभ॒यतः॑प्र‌उगम् चिन्वीत । \newline
4. उ॒भ॒यतः॑प्र‌उगम् चिन्वीत चिन्वी तोभ॒यतः॑प्र‌उग मुभ॒यतः॑प्र‌उगम् चिन्वीत॒ यो य श्चि॑न्वी तोभ॒यतः॑प्र‌उग मुभ॒यतः॑प्र‌उगम् चिन्वीत॒ यः । \newline
5. उ॒भ॒यतः॑प्र‌उग॒मित्यु॑भ॒यतः॑ - प्र॒उ॒ग॒म् । \newline
6. चि॒न्वी॒त॒ यो यश्चि॑न्वीत चिन्वीत॒ यः का॒मये॑त का॒मये॑त॒ यश्चि॑न्वीत चिन्वीत॒ यः का॒मये॑त । \newline
7. यः का॒मये॑त का॒मये॑त॒ यो यः का॒मये॑त॒ प्र प्र का॒मये॑त॒ यो यः का॒मये॑त॒ प्र । \newline
8. का॒मये॑त॒ प्र प्र का॒मये॑त का॒मये॑त॒ प्र जा॒तान् जा॒तान् प्र का॒मये॑त का॒मये॑त॒ प्र जा॒तान् । \newline
9. प्र जा॒तान् जा॒तान् प्र प्र जा॒तान् भ्रातृ॑व्या॒न् भ्रातृ॑व्यान् जा॒तान् प्र प्र जा॒तान् भ्रातृ॑व्यान् । \newline
10. जा॒तान् भ्रातृ॑व्या॒न् भ्रातृ॑व्यान् जा॒तान् जा॒तान् भ्रातृ॑व्यान् नु॒देय॑ नु॒देय॒ भ्रातृ॑व्यान् जा॒तान् जा॒तान् भ्रातृ॑व्यान् नु॒देय॑ । \newline
11. भ्रातृ॑व्यान् नु॒देय॑ नु॒देय॒ भ्रातृ॑व्या॒न् भ्रातृ॑व्यान् नु॒देय॒ प्रति॒ प्रति॑ नु॒देय॒ भ्रातृ॑व्या॒न् भ्रातृ॑व्यान् नु॒देय॒ प्रति॑ । \newline
12. नु॒देय॒ प्रति॒ प्रति॑ नु॒देय॑ नु॒देय॒ प्रति॑ जनि॒ष्यमा॑णान् जनि॒ष्यमा॑णा॒न् प्रति॑ नु॒देय॑ नु॒देय॒ प्रति॑ जनि॒ष्यमा॑णान् । \newline
13. प्रति॑ जनि॒ष्यमा॑णान् जनि॒ष्यमा॑णा॒न् प्रति॒ प्रति॑ जनि॒ष्यमा॑णा॒ नितीति॑ जनि॒ष्यमा॑णा॒न् प्रति॒ प्रति॑ जनि॒ष्यमा॑णा॒ निति॑ । \newline
14. ज॒नि॒ष्यमा॑णा॒ नितीति॑ जनि॒ष्यमा॑णान् जनि॒ष्यमा॑णा॒ निति॒ प्र प्रेति॑ जनि॒ष्यमा॑णान् जनि॒ष्यमा॑णा॒ निति॒ प्र । \newline
15. इति॒ प्र प्रेतीति॒ प्रैवैव प्रेतीति॒ प्रैव । \newline
16. प्रैवैव प्र प्रैव जा॒तान् जा॒ता ने॒व प्र प्रैव जा॒तान् । \newline
17. ए॒व जा॒तान् जा॒ता ने॒वैव जा॒तान् भ्रातृ॑व्या॒न् भ्रातृ॑व्यान् जा॒ता ने॒वैव जा॒तान् भ्रातृ॑व्यान् । \newline
18. जा॒तान् भ्रातृ॑व्या॒न् भ्रातृ॑व्यान् जा॒तान् जा॒तान् भ्रातृ॑व्यान् नु॒दते॑ नु॒दते॒ भ्रातृ॑व्यान् जा॒तान् जा॒तान् भ्रातृ॑व्यान् नु॒दते᳚ । \newline
19. भ्रातृ॑व्यान् नु॒दते॑ नु॒दते॒ भ्रातृ॑व्या॒न् भ्रातृ॑व्यान् नु॒दते॒ प्रति॒ प्रति॑ नु॒दते॒ भ्रातृ॑व्या॒न् भ्रातृ॑व्यान् नु॒दते॒ प्रति॑ । \newline
20. नु॒दते॒ प्रति॒ प्रति॑ नु॒दते॑ नु॒दते॒ प्रति॑ जनि॒ष्यमा॑णान् जनि॒ष्यमा॑णा॒न् प्रति॑ नु॒दते॑ नु॒दते॒ प्रति॑ जनि॒ष्यमा॑णान् । \newline
21. प्रति॑ जनि॒ष्यमा॑णान् जनि॒ष्यमा॑णा॒न् प्रति॒ प्रति॑ जनि॒ष्यमा॑णान् रथचक्र॒चितꣳ॑ रथचक्र॒चित॑म् जनि॒ष्यमा॑णा॒न् प्रति॒ प्रति॑ जनि॒ष्यमा॑णान् रथचक्र॒चित᳚म् । \newline
22. ज॒नि॒ष्यमा॑णान् रथचक्र॒चितꣳ॑ रथचक्र॒चित॑म् जनि॒ष्यमा॑णान् जनि॒ष्यमा॑णान् रथचक्र॒चित॑म् चिन्वीत चिन्वीत रथचक्र॒चित॑म् जनि॒ष्यमा॑णान् जनि॒ष्यमा॑णान् रथचक्र॒चित॑म् चिन्वीत । \newline
23. र॒थ॒च॒क्र॒चित॑म् चिन्वीत चिन्वीत रथचक्र॒चितꣳ॑ रथचक्र॒चित॑म् चिन्वीत॒ भ्रातृ॑व्यवा॒न् भ्रातृ॑व्यवाꣳ श्चिन्वीत रथचक्र॒चितꣳ॑ रथचक्र॒चित॑म् चिन्वीत॒ भ्रातृ॑व्यवान् । \newline
24. र॒थ॒च॒क्र॒चित॒मिति॑ रथचक्र - चित᳚म् । \newline
25. चि॒न्वी॒त॒ भ्रातृ॑व्यवा॒न् भ्रातृ॑व्यवाꣳ श्चिन्वीत चिन्वीत॒ भ्रातृ॑व्यवा॒न्॒. वज्रो॒ वज्रो॒ भ्रातृ॑व्यवाꣳ श्चिन्वीत चिन्वीत॒ भ्रातृ॑व्यवा॒न्॒. वज्रः॑ । \newline
26. भ्रातृ॑व्यवा॒न्॒. वज्रो॒ वज्रो॒ भ्रातृ॑व्यवा॒न् भ्रातृ॑व्यवा॒न्॒. वज्रो॒ वै वै वज्रो॒ भ्रातृ॑व्यवा॒न् भ्रातृ॑व्यवा॒न्॒. वज्रो॒ वै । \newline
27. भ्रातृ॑व्यवा॒निति॒ भ्रातृ॑व्य - वा॒न् । \newline
28. वज्रो॒ वै वै वज्रो॒ वज्रो॒ वै रथो॒ रथो॒ वै वज्रो॒ वज्रो॒ वै रथः॑ । \newline
29. वै रथो॒ रथो॒ वै वै रथो॒ वज्रं॒ ॅवज्रꣳ॒॒ रथो॒ वै वै रथो॒ वज्र᳚म् । \newline
30. रथो॒ वज्रं॒ ॅवज्रꣳ॒॒ रथो॒ रथो॒ वज्र॑ मे॒वैव वज्रꣳ॒॒ रथो॒ रथो॒ वज्र॑ मे॒व । \newline
31. वज्र॑ मे॒वैव वज्रं॒ ॅवज्र॑ मे॒व भ्रातृ॑व्येभ्यो॒ भ्रातृ॑व्येभ्य ए॒व वज्रं॒ ॅवज्र॑ मे॒व भ्रातृ॑व्येभ्यः । \newline
32. ए॒व भ्रातृ॑व्येभ्यो॒ भ्रातृ॑व्येभ्य ए॒वैव भ्रातृ॑व्येभ्यः॒ प्र प्र भ्रातृ॑व्येभ्य ए॒वैव भ्रातृ॑व्येभ्यः॒ प्र । \newline
33. भ्रातृ॑व्येभ्यः॒ प्र प्र भ्रातृ॑व्येभ्यो॒ भ्रातृ॑व्येभ्यः॒ प्र ह॑रति हरति॒ प्र भ्रातृ॑व्येभ्यो॒ भ्रातृ॑व्येभ्यः॒ प्र ह॑रति । \newline
34. प्र ह॑रति हरति॒ प्र प्र ह॑रति द्रोण॒चित॑म् द्रोण॒चितꣳ॑ हरति॒ प्र प्र ह॑रति द्रोण॒चित᳚म् । \newline
35. ह॒र॒ति॒ द्रो॒ण॒चित॑म् द्रोण॒चितꣳ॑ हरति हरति द्रोण॒चित॑म् चिन्वीत चिन्वीत द्रोण॒चितꣳ॑ हरति हरति द्रोण॒चित॑म् चिन्वीत । \newline
36. द्रो॒ण॒चित॑म् चिन्वीत चिन्वीत द्रोण॒चित॑म् द्रोण॒चित॑म् चिन्वी॒ता न्न॑का॒मो ऽन्न॑काम श्चिन्वीत द्रोण॒चित॑म् द्रोण॒चित॑म् चिन्वी॒ता न्न॑कामः । \newline
37. द्रो॒ण॒चित॒मिति॑ द्रोण - चित᳚म् । \newline
38. चि॒न्वी॒ता न्न॑का॒मो ऽन्न॑काम श्चिन्वीत चिन्वी॒ता न्न॑कामो॒ द्रोणे॒ द्रोणे ऽन्न॑काम श्चिन्वीत चिन्वी॒ता न्न॑कामो॒ द्रोणे᳚ । \newline
39. अन्न॑कामो॒ द्रोणे॒ द्रोणे ऽन्न॑का॒मो ऽन्न॑कामो॒ द्रोणे॒ वै वै द्रोणे ऽन्न॑का॒मो ऽन्न॑कामो॒ द्रोणे॒ वै । \newline
40. अन्न॑काम॒ इत्यन्न॑ - का॒मः॒ । \newline
41. द्रोणे॒ वै वै द्रोणे॒ द्रोणे॒ वा अन्न॒ मन्नं॒ ॅवै द्रोणे॒ द्रोणे॒ वा अन्न᳚म् । \newline
42. वा अन्न॒ मन्नं॒ ॅवै वा अन्न॑म् भ्रियते भ्रिय॒ते ऽन्नं॒ ॅवै वा अन्न॑म् भ्रियते । \newline
43. अन्न॑म् भ्रियते भ्रिय॒ते ऽन्न॒ मन्न॑म् भ्रियते॒ सयो॑नि॒ सयो॑नि भ्रिय॒ते ऽन्न॒ मन्न॑म् भ्रियते॒ सयो॑नि । \newline
44. भ्रि॒य॒ते॒ सयो॑नि॒ सयो॑नि भ्रियते भ्रियते॒ सयो᳚ न्ये॒वैव सयो॑नि भ्रियते भ्रियते॒ सयो᳚न्ये॒व । \newline
45. सयो᳚ न्ये॒वैव सयो॑नि॒ सयो᳚न्ये॒ वान्न॒ मन्न॑ मे॒व सयो॑नि॒ सयो᳚ न्ये॒वान्न᳚म् । \newline
46. सयो॒नीति॒ स - यो॒नि॒ । \newline
47. ए॒वान्न॒ मन्न॑ मे॒वै वान्न॒ मवा वान्न॑ मे॒वै वान्न॒ मव॑ । \newline
48. अन्न॒ मवा वान्न॒ मन्न॒ मव॑ रुन्धे रु॒न्धे ऽवान्न॒ मन्न॒ मव॑ रुन्धे । \newline
49. अव॑ रुन्धे रु॒न्धे ऽवाव॑ रुन्धे समू॒ह्यꣳ॑ समू॒ह्यꣳ॑ रु॒न्धे ऽवाव॑ रुन्धे समू॒ह्य᳚म् । \newline
50. रु॒न्धे॒ स॒मू॒ह्यꣳ॑ समू॒ह्यꣳ॑ रुन्धे रुन्धे समू॒ह्य॑म् चिन्वीत चिन्वीत समू॒ह्यꣳ॑ रुन्धे रुन्धे समू॒ह्य॑म् चिन्वीत । \newline
51. स॒मू॒ह्य॑म् चिन्वीत चिन्वीत समू॒ह्यꣳ॑ समू॒ह्य॑म् चिन्वीत प॒शुका॑मः प॒शुका॑म श्चिन्वीत समू॒ह्यꣳ॑ समू॒ह्य॑म् चिन्वीत प॒शुका॑मः । \newline
52. स॒मू॒ह्य॑मिति॑ सं - ऊ॒ह्य᳚म् । \newline
53. चि॒न्वी॒त॒ प॒शुका॑मः प॒शुका॑म श्चिन्वीत चिन्वीत प॒शुका॑मः पशु॒मान् प॑शु॒मान् प॒शुका॑म श्चिन्वीत चिन्वीत प॒शुका॑मः पशु॒मान् । \newline
54. प॒शुका॑मः पशु॒मान् प॑शु॒मान् प॒शुका॑मः प॒शुका॑मः पशु॒मा ने॒वैव प॑शु॒मान् प॒शुका॑मः प॒शुका॑मः पशु॒मा ने॒व । \newline
55. प॒शुका॑म॒ इति॑ प॒शु - का॒मः॒ । \newline
56. प॒शु॒मा ने॒वैव प॑शु॒मान् प॑शु॒मा ने॒व भ॑वति भव त्ये॒व प॑शु॒मान् प॑शु॒मा ने॒व भ॑वति । \newline
57. प॒शु॒मानिति॑ पशु - मान् । \newline
58. ए॒व भ॑वति भव त्ये॒वैव भ॑वति परिचा॒य्य॑म् परिचा॒य्य॑म् भव त्ये॒वैव भ॑वति परिचा॒य्य᳚म् । \newline
59. भ॒व॒ति॒ प॒रि॒चा॒य्य॑म् परिचा॒य्य॑म् भवति भवति परिचा॒य्य॑म् चिन्वीत चिन्वीत परिचा॒य्य॑म् भवति भवति परिचा॒य्य॑म् चिन्वीत । \newline
\pagebreak
\markright{ TS 5.4.11.3  \hfill https://www.vedavms.in \hfill}

\section{ TS 5.4.11.3 }

\textbf{TS 5.4.11.3 } \newline
\textbf{Samhita Paata} \newline

परिचा॒य्यं॑ चिन्वीत॒ ग्राम॑कामो ग्रा॒म्ये॑व भ॑वति श्मशान॒चितं॑ चिन्वीत॒ यः का॒मये॑त पितृलो॒क ऋ॑द्ध्नुया॒मिति॑ पितृलो॒क ए॒वर्द्ध्नो॑ति विश्वामित्रजमद॒ग्नी वसि॑ष्ठेनाऽस्पर्द्धेताꣳ॒॒ स ए॒ता ज॒मद॑ग्निर्विह॒व्या॑ अपश्य॒त् ता उपा॑धत्त॒ ताभि॒र्वै स वसि॑ष्ठस्येन्द्रि॒यं ॅवी॒र्य॑मवृङ्क्त॒ यद्-वि॑ह॒व्या॑ उप॒दधा॑तीन्द्रि॒यमे॒व ताभि॑र्वी॒र्यं॑ ॅयज॑मानो॒ भ्रातृ॑व्यस्य वृङ्क्ते॒ होतु॒र्द्धिष्णि॑य॒ उप॑ दधाति यजमानायत॒नं ॅवै - [  ] \newline

\textbf{Pada Paata} \newline

प॒रि॒चा॒य्य॑मिति॑ परि - चा॒य्य᳚म् । चि॒न्वी॒त॒ । ग्राम॑काम॒ इति॒ ग्राम॑ - का॒मः॒ । ग्रा॒मी । ए॒व । भ॒व॒ति॒ । श्म॒शा॒न॒चित॒मिति॑ श्मशान - चित᳚म् । चि॒न्वी॒त॒ । यः । का॒मये॑त । पि॒तृ॒लो॒क इति॑ पितृ - लो॒के । ऋ॒द्ध्नु॒या॒म् । इति॑ । पि॒तृ॒लो॒क इति॑ पितृ - लो॒के । ए॒व । ऋ॒द्ध्नो॒ति॒ । वि॒श्वा॒मि॒त्र॒ज॒म॒द॒ग्नी इति॑ विश्वामित्र - ज॒म॒द॒ग्नी । वसि॑ष्ठेन । अ॒स्प॒द्‌र्धे॒ता॒म् । सः । ए॒ताः । ज॒मद॑ग्निः । वि॒ह॒व्या॑ इति॑ वि - ह॒व्याः᳚ । अ॒प॒श्य॒त् । ताः । उपेति॑ । अ॒ध॒त्त॒ । ताभिः॑ । वै । सः । वसि॑ष्ठस्य । इ॒न्द्रि॒यम् । वी॒र्य᳚म् । अ॒वृ॒ङ्क्त॒ । यत् । वि॒ह॒व्या॑ इति॑ वि - ह॒व्याः᳚ । उ॒प॒दधा॒तीत्यु॑प - दधा॑ति । इ॒न्द्रि॒यम् । ए॒व । ताभिः॑ । वी॒र्य᳚म् । यज॑मानः । भ्रातृ॑व्यस्य । वृ॒ङ्क्ते॒ । होतुः॑ । धिष्णि॑ये । उपेति॑ । द॒धा॒ति॒ । य॒ज॒मा॒ना॒य॒त॒नमिति॑ यजमान - आ॒य॒त॒नम् । वै ।  \newline


\textbf{Krama Paata} \newline

प॒रि॒चा॒य्य॑म् चिन्वीत । प॒रि॒चा॒य्य॑मिति॑ परि - चा॒य्य᳚म् । चि॒न्वी॒त॒ ग्राम॑कामः । ग्राम॑कामो ग्रा॒मी । ग्राम॑काम॒ इति॒ ग्राम॑ - का॒मः॒ । ग्रा॒म्ये॑व । ए॒व भ॑वति । भ॒व॒ति॒ श्म॒शा॒न॒चित᳚म् । श्म॒शा॒न॒चित॑म् चिन्वीत । श्म॒शा॒न॒चित॒मिति॑ श्मशान - चित᳚म् । चि॒न्वी॒त॒ यः । यः का॒मये॑त । का॒मये॑त पितृलो॒के । पि॒तृ॒लो॒क ऋ॑द्ध्नुयाम् । पि॒तृ॒लो॒क इति॑ पितृ - लो॒के । ऋ॒द्ध्नु॒या॒मिति॑ । इति॑ पितृलो॒के । पि॒तृ॒लो॒क ए॒व । पि॒तृ॒लो॒क इति॑ पितृ - लो॒के । ए॒वर्द्ध्नो॑ति । ऋ॒द्ध्नो॒ति॒ वि॒श्वा॒मि॒त्र॒ज॒म॒द॒ग्नी । वि॒श्वा॒मि॒त्र॒ज॒म॒द॒ग्नी वसि॑ष्ठेन । वि॒श्वा॒मि॒त्र॒ज॒म॒द॒ग्नी इति॑ विश्वामित्र - ज॒म॒द॒ग्नी । वसि॑ष्ठेना ऽस्पर्द्धेताम् । अ॒स्प॒र्द्धे॒ताꣳ॒॒ स । स ए॒ताः । ए॒ता ज॒मद॑ग्निः । ज॒मद॑ग्निर् विह॒व्याः᳚ । वि॒ह॒व्या॑ अपश्यत् । वि॒ह॒व्या॑ इति॑ वि - ह॒व्याः᳚ । अ॒प॒श्य॒त् ताः । ता उप॑ । उपा॑धत्त । अ॒ध॒त्त॒ ताभिः॑ । ताभि॒र् वै । वै सः । स वसि॑ष्ठस्य । वसि॑ष्ठस्येन्द्रि॒यम् । इ॒न्द्रि॒यम् ॅवी॒र्य᳚म् । वी॒र्य॑मवृङ्त । अ॒वृ॒ङ्त॒ यत् । यद् वि॑ह॒व्याः᳚ । वि॒ह॒व्या॑ उप॒दधा॑ति । वि॒ह॒व्या॑ इति॑ वि - ह॒व्याः᳚ । उ॒प॒दधा॑तीन्द्रि॒यम् । उ॒प॒दधा॒तीत्यु॑प - दधा॑ति । इ॒न्द्रि॒यमे॒व । ए॒व ताभिः॑ । ताभि॑र् वी॒र्य᳚म् । वी॒र्य॑म् ॅयज॑मानः । यज॑मानो॒ भ्रातृ॑व्यस्य । भ्रातृ॑व्यस्य वृङ्क्ते । वृ॒ङ्क्ते॒ होतुः॑ । होतु॒र् धिष्णि॑ये । धिष्णि॑य॒ उप॑ । उप॑ दधाति । द॒धा॒ति॒ य॒ज॒मा॒ना॒य॒त॒नम् । य॒ज॒मा॒ना॒य॒त॒नम् ॅवै ( ) । य॒ज॒मा॒ना॒य॒त॒नमिति॑ यजमान - आ॒य॒त॒नम् । वै होता᳚ \newline

\textbf{Jatai Paata} \newline

1. प॒रि॒चा॒य्य॑म् चिन्वीत चिन्वीत परिचा॒य्य॑म् परिचा॒य्य॑म् चिन्वीत । \newline
2. प॒रि॒चा॒य्य॑मिति॑ परि - चा॒य्य᳚म् । \newline
3. चि॒न्वी॒त॒ ग्राम॑कामो॒ ग्राम॑काम श्चिन्वीत चिन्वीत॒ ग्राम॑कामः । \newline
4. ग्राम॑कामो ग्रा॒मी ग्रा॒मी ग्राम॑कामो॒ ग्राम॑कामो ग्रा॒मी । \newline
5. ग्राम॑काम॒ इति॒ ग्राम॑ - का॒मः॒ । \newline
6. ग्रा॒म्ये॑ वैव ग्रा॒मी ग्रा॒म्ये॑व । \newline
7. ए॒व भ॑वति भव त्ये॒वैव भ॑वति । \newline
8. भ॒व॒ति॒ श्म॒शा॒न॒चितꣳ॑ श्मशान॒चित॑म् भवति भवति श्मशान॒चित᳚म् । \newline
9. श्म॒शा॒न॒चित॑म् चिन्वीत चिन्वीत श्मशान॒चितꣳ॑ श्मशान॒चित॑म् चिन्वीत । \newline
10. श्म॒शा॒न॒चित॒मिति॑ श्मशान - चित᳚म् । \newline
11. चि॒न्वी॒त॒ यो य श्चि॑न्वीत चिन्वीत॒ यः । \newline
12. यः का॒मये॑त का॒मये॑त॒ यो यः का॒मये॑त । \newline
13. का॒मये॑त पितृलो॒के पि॑तृलो॒के का॒मये॑त का॒मये॑त पितृलो॒के । \newline
14. पि॒तृ॒लो॒क ऋ॑द्ध्नुया मृद्ध्नुयाम् पितृलो॒के पि॑तृलो॒क ऋ॑द्ध्नुयाम् । \newline
15. पि॒तृ॒लो॒क इति॑ पितृ - लो॒के । \newline
16. ऋ॒द्ध्नु॒या॒ मिती त्यृ॑द्ध्नुया मृद्ध्नुया॒ मिति॑ । \newline
17. इति॑ पितृलो॒के पि॑तृलो॒क इतीति॑ पितृलो॒के । \newline
18. पि॒तृ॒लो॒क ए॒वैव पि॑तृलो॒के पि॑तृलो॒क ए॒व । \newline
19. पि॒तृ॒लो॒क इति॑ पितृ - लो॒के । \newline
20. ए॒व र्द्ध्नो᳚ त्यृद्ध्नो त्ये॒वैव र्द्ध्नो॑ति । \newline
21. ऋ॒द्ध्नो॒ति॒ वि॒श्वा॒मि॒त्र॒ज॒म॒द॒ग्नी वि॑श्वामित्रजमद॒ग्नी ऋ॑द्ध्नो त्यृद्ध्नोति विश्वामित्रजमद॒ग्नी । \newline
22. वि॒श्वा॒मि॒त्र॒ज॒म॒द॒ग्नी वसि॑ष्ठेन॒ वसि॑ष्ठेन विश्वामित्रजमद॒ग्नी वि॑श्वामित्रजमद॒ग्नी वसि॑ष्ठेन । \newline
23. वि॒श्वा॒मि॒त्र॒ज॒म॒द॒ग्नी इति॑ विश्वामित्र - ज॒म॒द॒ग्नी । \newline
24. वसि॑ष्ठेना ऽस्पर्द्धेता मस्पर्द्धेतां॒ ॅवसि॑ष्ठेन॒ वसि॑ष्ठेना ऽस्पर्द्धेताम् । \newline
25. अ॒स्प॒र्द्धे॒ताꣳ॒॒ स सो᳚ ऽस्पर्द्धेता मस्पर्द्धेताꣳ॒॒ सः । \newline
26. स ए॒ता ए॒ताः स स ए॒ताः । \newline
27. ए॒ता ज॒मद॑ग्निर् ज॒मद॑ग्नि रे॒ता ए॒ता ज॒मद॑ग्निः । \newline
28. ज॒मद॑ग्निर् विह॒व्या॑ विह॒व्या॑ ज॒मद॑ग्निर् ज॒मद॑ग्निर् विह॒व्याः᳚ । \newline
29. वि॒ह॒व्या॑ अपश्य दपश्यद् विह॒व्या॑ विह॒व्या॑ अपश्यत् । \newline
30. वि॒ह॒व्या॑ इति॑ वि - ह॒व्याः᳚ । \newline
31. अ॒प॒श्य॒त् ता स्ता अ॑पश्य दपश्य॒त् ताः । \newline
32. ता उपोप॒ ता स्ता उप॑ । \newline
33. उपा॑धत्ता ध॒त्तो पोपा॑ धत्त । \newline
34. अ॒ध॒त्त॒ ताभि॒ स्ताभि॑ रधत्ता धत्त॒ ताभिः॑ । \newline
35. ताभि॒र् वै वै ताभि॒ स्ताभि॒र् वै । \newline
36. वै स स वै वै सः । \newline
37. स वसि॑ष्ठस्य॒ वसि॑ष्ठस्य॒ स स वसि॑ष्ठस्य । \newline
38. वसि॑ष्ठ स्येन्द्रि॒य मि॑न्द्रि॒यं ॅवसि॑ष्ठस्य॒ वसि॑ष्ठ स्येन्द्रि॒यम् । \newline
39. इ॒न्द्रि॒यं ॅवी॒र्यं॑ ॅवी॒र्य॑ मिन्द्रि॒य मि॑न्द्रि॒यं ॅवी॒र्य᳚म् । \newline
40. वी॒र्य॑ मवृङ्क्ता वृङ्क्त वी॒र्यं॑ ॅवी॒र्य॑ मवृङ्क्त । \newline
41. अ॒वृ॒ङ्क्त॒ यद् यद॑वृङ्क्ता वृङ्क्त॒ यत् । \newline
42. यद् वि॑ह॒व्या॑ विह॒व्या॑ यद् यद् वि॑ह॒व्याः᳚ । \newline
43. वि॒ह॒व्या॑ उप॒दधा᳚ त्युप॒दधा॑ति विह॒व्या॑ विह॒व्या॑ उप॒दधा॑ति । \newline
44. वि॒ह॒व्या॑ इति॑ वि - ह॒व्याः᳚ । \newline
45. उ॒प॒दधा॑ तीन्द्रि॒य मि॑न्द्रि॒य मु॑प॒दधा᳚ त्युप॒दधा॑ तीन्द्रि॒यम् । \newline
46. उ॒प॒दधा॒तीत्यु॑प - दधा॑ति । \newline
47. इ॒न्द्रि॒य मे॒वैवेन्द्रि॒य मि॑न्द्रि॒य मे॒व । \newline
48. ए॒व ताभि॒ स्ताभि॑ रे॒वैव ताभिः॑ । \newline
49. ताभि॑र् वी॒र्यं॑ ॅवी॒र्य॑म् ताभि॒ स्ताभि॑र् वी॒र्य᳚म् । \newline
50. वी॒र्यं॑ ॅयज॑मानो॒ यज॑मानो वी॒र्यं॑ ॅवी॒र्यं॑ ॅयज॑मानः । \newline
51. यज॑मानो॒ भ्रातृ॑व्यस्य॒ भ्रातृ॑व्यस्य॒ यज॑मानो॒ यज॑मानो॒ भ्रातृ॑व्यस्य । \newline
52. भ्रातृ॑व्यस्य वृङ्क्ते वृङ्क्ते॒ भ्रातृ॑व्यस्य॒ भ्रातृ॑व्यस्य वृङ्क्ते । \newline
53. वृ॒ङ्क्ते॒ होतु॒र्॒. होतु॑र् वृङ्क्ते वृङ्क्ते॒ होतुः॑ । \newline
54. होतु॒र् धिष्णि॑ये॒ धिष्णि॑ये॒ होतु॒र्॒. होतु॒र् धिष्णि॑ये । \newline
55. धिष्णि॑य॒ उपोप॒ धिष्णि॑ये॒ धिष्णि॑य॒ उप॑ । \newline
56. उप॑ दधाति दधा॒ त्युपोप॑ दधाति । \newline
57. द॒धा॒ति॒ य॒ज॒मा॒ना॒य॒त॒नं ॅय॑जमानायत॒नम् द॑धाति दधाति यजमानायत॒नम् । \newline
58. य॒ज॒मा॒ना॒य॒त॒नं ॅवै वै य॑जमानायत॒नं ॅय॑जमानायत॒नं ॅवै । \newline
59. य॒ज॒मा॒ना॒य॒त॒नमिति॑ यजमान - आ॒य॒त॒नम् । \newline
60. वै होता॒ होता॒ वै वै होता᳚ । \newline

\textbf{Ghana Paata } \newline

1. प॒रि॒चा॒य्य॑म् चिन्वीत चिन्वीत परिचा॒य्य॑म् परिचा॒य्य॑म् चिन्वीत॒ ग्राम॑कामो॒ ग्राम॑काम श्चिन्वीत परिचा॒य्य॑म् परिचा॒य्य॑म् चिन्वीत॒ ग्राम॑कामः । \newline
2. प॒रि॒चा॒य्य॑मिति॑ परि - चा॒य्य᳚म् । \newline
3. चि॒न्वी॒त॒ ग्राम॑कामो॒ ग्राम॑काम श्चिन्वीत चिन्वीत॒ ग्राम॑कामो ग्रा॒मी ग्रा॒मी ग्राम॑काम श्चिन्वीत चिन्वीत॒ ग्राम॑कामो ग्रा॒मी । \newline
4. ग्राम॑कामो ग्रा॒मी ग्रा॒मी ग्राम॑कामो॒ ग्राम॑कामो ग्रा॒म्ये॑वैव ग्रा॒मी ग्राम॑कामो॒ ग्राम॑कामो ग्रा॒म्ये॑व । \newline
5. ग्राम॑काम॒ इति॒ ग्राम॑ - का॒मः॒ । \newline
6. ग्रा॒म्ये॑ वैव ग्रा॒मी ग्रा॒म्ये॑व भ॑वति भव त्ये॒व ग्रा॒मी ग्रा॒म्ये॑व भ॑वति । \newline
7. ए॒व भ॑वति भव त्ये॒वैव भ॑वति श्मशान॒चितꣳ॑ श्मशान॒चित॑म् भव त्ये॒वैव भ॑वति श्मशान॒चित᳚म् । \newline
8. भ॒व॒ति॒ श्म॒शा॒न॒चितꣳ॑ श्मशान॒चित॑म् भवति भवति श्मशान॒चित॑म् चिन्वीत चिन्वीत श्मशान॒चित॑म् भवति भवति श्मशान॒चित॑म् चिन्वीत । \newline
9. श्म॒शा॒न॒चित॑म् चिन्वीत चिन्वीत श्मशान॒चितꣳ॑ श्मशान॒चित॑म् चिन्वीत॒ यो यश्चि॑न्वीत श्मशान॒चितꣳ॑ श्मशान॒चित॑म् चिन्वीत॒ यः । \newline
10. श्म॒शा॒न॒चित॒मिति॑ श्मशान - चित᳚म् । \newline
11. चि॒न्वी॒त॒ यो यश्चि॑न्वीत चिन्वीत॒ यः का॒मये॑त का॒मये॑त॒ यश्चि॑न्वीत चिन्वीत॒ यः का॒मये॑त । \newline
12. यः का॒मये॑त का॒मये॑त॒ यो यः का॒मये॑त पितृलो॒के पि॑तृलो॒के का॒मये॑त॒ यो यः का॒मये॑त पितृलो॒के । \newline
13. का॒मये॑त पितृलो॒के पि॑तृलो॒के का॒मये॑त का॒मये॑त पितृलो॒क ऋ॑द्ध्नुया मृद्ध्नुयाम् पितृलो॒के का॒मये॑त का॒मये॑त पितृलो॒क ऋ॑द्ध्नुयाम् । \newline
14. पि॒तृ॒लो॒क ऋ॑द्ध्नुया मृद्ध्नुयाम् पितृलो॒के पि॑तृलो॒क ऋ॑द्ध्नुया॒ मिती त्यृ॑द्ध्नुयाम् पितृलो॒के पि॑तृलो॒क ऋ॑द्ध्नुया॒ मिति॑ । \newline
15. पि॒तृ॒लो॒क इति॑ पितृ - लो॒के । \newline
16. ऋ॒द्ध्नु॒या॒ मिती त्यृ॑द्ध्नुया मृद्ध्नुया॒ मिति॑ पितृलो॒के पि॑तृलो॒क इत्यृ॑द्ध्नुया मृद्ध्नुया॒ मिति॑ पितृलो॒के । \newline
17. इति॑ पितृलो॒के पि॑तृलो॒क इतीति॑ पितृलो॒क ए॒वैव पि॑तृलो॒क इतीति॑ पितृलो॒क ए॒व । \newline
18. पि॒तृ॒लो॒क ए॒वैव पि॑तृलो॒के पि॑तृलो॒क ए॒व र्द्ध्नो᳚ त्यृद्ध्नो त्ये॒व पि॑तृलो॒के पि॑तृलो॒क ए॒व र्द्ध्नो॑ति । \newline
19. पि॒तृ॒लो॒क इति॑ पितृ - लो॒के । \newline
20. ए॒व र्द्ध्नो᳚ त्यृद्ध्नो त्ये॒वैव र्द्ध्नो॑ति विश्वामित्रजमद॒ग्नी वि॑श्वामित्रजमद॒ग्नी ऋ॑द्ध्नो त्ये॒वैव र्द्ध्नो॑ति विश्वामित्रजमद॒ग्नी । \newline
21. ऋ॒द्ध्नो॒ति॒ वि॒श्वा॒मि॒त्र॒ज॒म॒द॒ग्नी वि॑श्वामित्रजमद॒ग्नी ऋ॑द्ध्नो त्यृद्ध्नोति विश्वामित्रजमद॒ग्नी वसि॑ष्ठेन॒ वसि॑ष्ठेन विश्वामित्रजमद॒ग्नी ऋ॑द्ध्नो त्यृद्ध्नोति विश्वामित्रजमद॒ग्नी वसि॑ष्ठेन । \newline
22. वि॒श्वा॒मि॒त्र॒ज॒म॒द॒ग्नी वसि॑ष्ठेन॒ वसि॑ष्ठेन विश्वामित्रजमद॒ग्नी वि॑श्वामित्रजमद॒ग्नी वसि॑ष्ठेना ऽस्पर्द्धेता मस्पर्द्धेतां॒ ॅवसि॑ष्ठेन विश्वामित्रजमद॒ग्नी वि॑श्वामित्रजमद॒ग्नी वसि॑ष्ठेना ऽस्पर्द्धेताम् । \newline
23. वि॒श्वा॒मि॒त्र॒ज॒म॒द॒ग्नी इति॑ विश्वामित्र - ज॒म॒द॒ग्नी । \newline
24. वसि॑ष्ठेना ऽस्पर्द्धेता मस्पर्द्धेतां॒ ॅवसि॑ष्ठेन॒ वसि॑ष्ठेना ऽस्पर्द्धेताꣳ॒॒ स सो᳚ ऽस्पर्द्धेतां॒ ॅवसि॑ष्ठेन॒ वसि॑ष्ठेना ऽस्पर्द्धेताꣳ॒॒ सः । \newline
25. अ॒स्प॒र्द्धे॒ताꣳ॒॒ स सो᳚ ऽस्पर्द्धेता मस्पर्द्धेताꣳ॒॒ स ए॒ता ए॒ताः सो᳚ ऽस्पर्द्धेता मस्पर्द्धेताꣳ॒॒ स ए॒ताः । \newline
26. स ए॒ता ए॒ताः स स ए॒ता ज॒मद॑ग्निर् ज॒मद॑ग्नि रे॒ताः स स ए॒ता ज॒मद॑ग्निः । \newline
27. ए॒ता ज॒मद॑ग्निर् ज॒मद॑ग्नि रे॒ता ए॒ता ज॒मद॑ग्निर् विह॒व्या॑ विह॒व्या॑ ज॒मद॑ग्नि रे॒ता ए॒ता ज॒मद॑ग्निर् विह॒व्याः᳚ । \newline
28. ज॒मद॑ग्निर् विह॒व्या॑ विह॒व्या॑ ज॒मद॑ग्निर् ज॒मद॑ग्निर् विह॒व्या॑ अपश्य दपश्यद् विह॒व्या॑ ज॒मद॑ग्निर् ज॒मद॑ग्निर् विह॒व्या॑ अपश्यत् । \newline
29. वि॒ह॒व्या॑ अपश्य दपश्यद् विह॒व्या॑ विह॒व्या॑ अपश्य॒त् ता स्ता अ॑पश्यद् विह॒व्या॑ विह॒व्या॑ अपश्य॒त् ताः । \newline
30. वि॒ह॒व्या॑ इति॑ वि - ह॒व्याः᳚ । \newline
31. अ॒प॒श्य॒त् ता स्ता अ॑पश्य दपश्य॒त् ता उपोप॒ ता अ॑पश्य दपश्य॒त् ता उप॑ । \newline
32. ता उपोप॒ ता स्ता उपा॑धत्ता ध॒त्तोप॒ ता स्ता उपा॑धत्त । \newline
33. उपा॑धत्ता ध॒त्तो पोपा॑ धत्त॒ ताभि॒ स्ताभि॑ रध॒त्तो पोपा॑ धत्त॒ ताभिः॑ । \newline
34. अ॒ध॒त्त॒ ताभि॒ स्ताभि॑ रधत्ता धत्त॒ ताभि॒र् वै वै ताभि॑ रधत्ता धत्त॒ ताभि॒र् वै । \newline
35. ताभि॒र् वै वै ताभि॒ स्ताभि॒र् वै स स वै ताभि॒ स्ताभि॒र् वै सः । \newline
36. वै स स वै वै स वसि॑ष्ठस्य॒ वसि॑ष्ठस्य॒ स वै वै स वसि॑ष्ठस्य । \newline
37. स वसि॑ष्ठस्य॒ वसि॑ष्ठस्य॒ स स वसि॑ष्ठ स्येन्द्रि॒य मि॑न्द्रि॒यं ॅवसि॑ष्ठस्य॒ स स वसि॑ष्ठ
स्येन्द्रि॒यम् । \newline
38. वसि॑ष्ठ स्येन्द्रि॒य मि॑न्द्रि॒यं ॅवसि॑ष्ठस्य॒ वसि॑ष्ठ स्येन्द्रि॒यं ॅवी॒र्यं॑ ॅवी॒र्य॑ मिन्द्रि॒यं ॅवसि॑ष्ठस्य॒ वसि॑ष्ठ स्येन्द्रि॒यं ॅवी॒र्य᳚म् । \newline
39. इ॒न्द्रि॒यं ॅवी॒र्यं॑ ॅवी॒र्य॑ मिन्द्रि॒य मि॑न्द्रि॒यं ॅवी॒र्य॑ मवृङ्क्ता वृङ्क्त वी॒र्य॑ मिन्द्रि॒य मि॑न्द्रि॒यं ॅवी॒र्य॑ मवृङ्क्त । \newline
40. वी॒र्य॑ मवृङ्क्ता वृङ्क्त वी॒र्यं॑ ॅवी॒र्य॑ मवृङ्क्त॒ यद् यद॑वृङ्क्त वी॒र्यं॑ ॅवी॒र्य॑ मवृङ्क्त॒ यत् । \newline
41. अ॒वृ॒ङ्क्त॒ यद् यद॑वृङ्क्ता वृङ्क्त॒ यद् वि॑ह॒व्या॑ विह॒व्या॑ यद॑वृङ्क्ता वृङ्क्त॒ यद् वि॑ह॒व्याः᳚ । \newline
42. यद् वि॑ह॒व्या॑ विह॒व्या॑ यद् यद् वि॑ह॒व्या॑ उप॒दधा᳚ त्युप॒दधा॑ति विह॒व्या॑ यद् यद् वि॑ह॒व्या॑ उप॒दधा॑ति । \newline
43. वि॒ह॒व्या॑ उप॒दधा᳚ त्युप॒दधा॑ति विह॒व्या॑ विह॒व्या॑ उप॒दधा॑ तीन्द्रि॒य मि॑न्द्रि॒य मु॑प॒दधा॑ति विह॒व्या॑ विह॒व्या॑ उप॒दधा॑ तीन्द्रि॒यम् । \newline
44. वि॒ह॒व्या॑ इति॑ वि - ह॒व्याः᳚ । \newline
45. उ॒प॒दधा॑ तीन्द्रि॒य मि॑न्द्रि॒य मु॑प॒दधा᳚ त्युप॒दधा॑ तीन्द्रि॒य मे॒वैवेन्द्रि॒य मु॑प॒दधा᳚ त्युप॒दधा॑ तीन्द्रि॒य मे॒व । \newline
46. उ॒प॒दधा॒तीत्यु॑प - दधा॑ति । \newline
47. इ॒न्द्रि॒य मे॒वैवेन्द्रि॒य मि॑न्द्रि॒य मे॒व ताभि॒ स्ताभि॑ रे॒वेन्द्रि॒य मि॑न्द्रि॒य मे॒व ताभिः॑ । \newline
48. ए॒व ताभि॒ स्ताभि॑ रे॒वैव ताभि॑र् वी॒र्यं॑ ॅवी॒र्य॑म् ताभि॑ रे॒वैव ताभि॑र् वी॒र्य᳚म् । \newline
49. ताभि॑र् वी॒र्यं॑ ॅवी॒र्य॑म् ताभि॒ स्ताभि॑र् वी॒र्यं॑ ॅयज॑मानो॒ यज॑मानो वी॒र्य॑म् ताभि॒ स्ताभि॑र् वी॒र्यं॑ ॅयज॑मानः । \newline
50. वी॒र्यं॑ ॅयज॑मानो॒ यज॑मानो वी॒र्यं॑ ॅवी॒र्यं॑ ॅयज॑मानो॒ भ्रातृ॑व्यस्य॒ भ्रातृ॑व्यस्य॒ यज॑मानो वी॒र्यं॑ ॅवी॒र्यं॑ ॅयज॑मानो॒ भ्रातृ॑व्यस्य । \newline
51. यज॑मानो॒ भ्रातृ॑व्यस्य॒ भ्रातृ॑व्यस्य॒ यज॑मानो॒ यज॑मानो॒ भ्रातृ॑व्यस्य वृङ्क्ते वृङ्क्ते॒ भ्रातृ॑व्यस्य॒ यज॑मानो॒ यज॑मानो॒ भ्रातृ॑व्यस्य वृङ्क्ते । \newline
52. भ्रातृ॑व्यस्य वृङ्क्ते वृङ्क्ते॒ भ्रातृ॑व्यस्य॒ भ्रातृ॑व्यस्य वृङ्क्ते॒ होतु॒र्॒. होतु॑र् वृङ्क्ते॒ भ्रातृ॑व्यस्य॒ भ्रातृ॑व्यस्य वृङ्क्ते॒ होतुः॑ । \newline
53. वृ॒ङ्क्ते॒ होतु॒र्॒. होतु॑र् वृङ्क्ते वृङ्क्ते॒ होतु॒र् धिष्णि॑ये॒ धिष्णि॑ये॒ होतु॑र् वृङ्क्ते वृङ्क्ते॒ होतु॒र् धिष्णि॑ये । \newline
54. होतु॒र् धिष्णि॑ये॒ धिष्णि॑ये॒ होतु॒र्॒. होतु॒र् धिष्णि॑य॒ उपोप॒ धिष्णि॑ये॒ होतु॒र्॒. होतु॒र् धिष्णि॑य॒ उप॑ । \newline
55. धिष्णि॑य॒ उपोप॒ धिष्णि॑ये॒ धिष्णि॑य॒ उप॑ दधाति दधा॒ त्युप॒ धिष्णि॑ये॒ धिष्णि॑य॒ उप॑ दधाति । \newline
56. उप॑ दधाति दधा॒ त्युपोप॑ दधाति यजमानायत॒नं ॅय॑जमानायत॒नम् द॑धा॒ त्युपोप॑ दधाति यजमानायत॒नम् । \newline
57. द॒धा॒ति॒ य॒ज॒मा॒ना॒य॒त॒नं ॅय॑जमानायत॒नम् द॑धाति दधाति यजमानायत॒नं ॅवै वै य॑जमानायत॒नम् द॑धाति दधाति यजमानायत॒नं ॅवै । \newline
58. य॒ज॒मा॒ना॒य॒त॒नं ॅवै वै य॑जमानायत॒नं ॅय॑जमानायत॒नं ॅवै होता॒ होता॒ वै य॑जमानायत॒नं ॅय॑जमानायत॒नं ॅवै होता᳚ । \newline
59. य॒ज॒मा॒ना॒य॒त॒नमिति॑ यजमान - आ॒य॒त॒नम् । \newline
60. वै होता॒ होता॒ वै वै होता॒ स्वे स्वे होता॒ वै वै होता॒ स्वे । \newline
\pagebreak
\markright{ TS 5.4.11.4  \hfill https://www.vedavms.in \hfill}

\section{ TS 5.4.11.4 }

\textbf{TS 5.4.11.4 } \newline
\textbf{Samhita Paata} \newline

होता॒ स्व ए॒वास्मा॑ आ॒यत॑न इन्द्रि॒यं ॅवी॒र्य॑मव॑ रुन्धे॒ द्वाद॒शोप॑ दधाति॒ द्वाद॑शाक्षरा॒ जग॑ती॒ जाग॑ताः प॒शवो॒ जग॑त्यै॒वास्मै॑ प॒शूनव॑ रुन्धे॒ ऽष्टाव॑ष्टाव॒न्येषु॒ धिष्णि॑ये॒षूप॑ दधात्य॒ष्टाश॑फाः प॒शवः॑ प॒शूने॒वाव॑ रुन्धे॒ षण्मा᳚र्जा॒लीये॒ षड् वा ऋ॒तव॑ ऋ॒तवः॒ खलु॒ वै दे॒वाः पि॒तर॑ ऋ॒तूने॒व दे॒वान् पि॒तॄन् प्री॑णाति ॥ \newline

\textbf{Pada Paata} \newline

होता᳚ । स्वे । ए॒व । अ॒स्मै॒ । आ॒यत॑न॒ इत्या᳚ - यत॑ने । इ॒न्द्रि॒यम् । वी॒र्य᳚म् । अवेति॑ । रु॒न्धे॒ । द्वाद॑श । उपेति॑ । द॒धा॒ति॒ । द्वाद॑शाक्ष॒रेति॒ द्वाद॑श - अ॒क्ष॒रा॒ । जग॑ती । जाग॑ताः । प॒शवः॑ । जग॑त्या । ए॒व । अ॒स्मै॒ । प॒शून् । अवेति॑ । रु॒न्धे॒ । अ॒ष्टाव॑ष्टा॒वित्य॒ष्टौ-अ॒ष्टौ॒ । अ॒न्येषु॑ । धिष्णि॑येषु । उपेति॑ । द॒धा॒ति॒ । अ॒ष्टाश॑फा॒ इत्य॒ष्टा - श॒फाः॒ । प॒शवः॑ । प॒शून् । ए॒व । अवेति॑ । रु॒न्धे॒ । षट् । मा॒र्जा॒लीये᳚ । षट् । वै । ऋ॒तवः॑ । ऋ॒तवः॑ । खलु॑ । वै । दे॒वाः । पि॒तरः॑ । ऋ॒तून् । ए॒व । दे॒वान् । पि॒तॄन् । प्री॒णा॒ति॒ ॥  \newline


\textbf{Krama Paata} \newline

होता॒ स्वे । स्व ए॒व । ए॒वास्मै᳚ । अ॒स्मा॒ आ॒यत॑ने । आ॒यत॑न इन्द्रि॒यम् । आ॒यत॑न॒ इत्या᳚ - यत॑ने । इ॒न्द्रि॒यम् ॅवी॒र्य᳚म् । वी॒र्य॑मव॑ । अव॑ रुन्धे । रु॒न्धे॒ द्वाद॑श । द्वाद॒शोप॑ । उप॑ दधाति । द॒धा॒ति॒ द्वाद॑शाक्षरा । द्वाद॑शाक्षरा॒ जग॑ती । द्वाद॑शाक्ष॒रेति॒ द्वाद॑श - अ॒क्ष॒रा॒ । जग॑ती॒ जाग॑ताः । जाग॑ताः प॒शवः॑ । प॒शवो॒ जग॑त्या । जग॑त्यै॒व । ए॒वास्मै᳚ । अ॒स्मै॒ प॒शून् । प॒शूनव॑ । अव॑ रुन्धे । रु॒न्धे॒ऽष्टाव॑ष्टौ । अ॒ष्टाव॑ष्टाव॒न्येषु॑ । अ॒ष्टाव॑ष्टा॒वित्य॒ष्टौ - अ॒ष्टौ॒ । अ॒न्येषु॒ धिष्णि॑येषु । धिष्णि॑ये॒षूप॑ । उप॑ दधाति । द॒धा॒त्य॒ष्टाश॑फाः । अ॒ष्टाश॑फाः प॒शवः॑ । अ॒ष्टाश॑फा॒ इत्य॒ष्टा - श॒फा॒ ः । प॒शवः॑ प॒शून् । प॒शूने॒व । ए॒वाव॑ । अव॑ रुन्धे । रु॒न्धे॒ षट् । षण् मा᳚र्जा॒लीये᳚ । मा॒र्जा॒लीये॒ षट् । षड् वै । वा ऋ॒तवः॑ । ऋ॒तव॑ ऋ॒तवः॑ । ऋ॒तवः॒ खलु॑ । खलु॒ वै । वै दे॒वाः । दे॒वाः पि॒तरः॑ । पि॒तर॑ ऋ॒तून् । ऋ॒तूने॒व । ए॒व दे॒वान् । दे॒वान् पि॒तॄन् । पि॒तॄन् प्री॑णाति । प्री॒णा॒तीति॑ प्रीणाति । \newline

\textbf{Jatai Paata} \newline

1. होता॒ स्वे स्वे होता॒ होता॒ स्वे । \newline
2. स्व ए॒वैव स्वे स्व ए॒व । \newline
3. ए॒वास्मा॑ अस्मा ए॒वै वास्मै᳚ । \newline
4. अ॒स्मा॒ आ॒यत॑न आ॒यत॑ने ऽस्मा अस्मा आ॒यत॑ने । \newline
5. आ॒यत॑न इन्द्रि॒य मि॑न्द्रि॒य मा॒यत॑न आ॒यत॑न इन्द्रि॒यम् । \newline
6. आ॒यत॑न॒ इत्या᳚ - यत॑ने । \newline
7. इ॒न्द्रि॒यं ॅवी॒र्यं॑ ॅवी॒र्य॑ मिन्द्रि॒य मि॑न्द्रि॒यं ॅवी॒र्य᳚म् । \newline
8. वी॒र्य॑ मवाव॑ वी॒र्यं॑ ॅवी॒र्य॑ मव॑ । \newline
9. अव॑ रुन्धे रु॒न्धे ऽवाव॑ रुन्धे । \newline
10. रु॒न्धे॒ द्वाद॑श॒ द्वाद॑श रुन्धे रुन्धे॒ द्वाद॑श । \newline
11. द्वाद॒शोपोप॒ द्वाद॑श॒ द्वाद॒शोप॑ । \newline
12. उप॑ दधाति दधा॒ त्युपोप॑ दधाति । \newline
13. द॒धा॒ति॒ द्वाद॑शाक्षरा॒ द्वाद॑शाक्षरा दधाति दधाति॒ द्वाद॑शाक्षरा । \newline
14. द्वाद॑शाक्षरा॒ जग॑ती॒ जग॑ती॒ द्वाद॑शाक्षरा॒ द्वाद॑शाक्षरा॒ जग॑ती । \newline
15. द्वाद॑शाक्ष॒रेति॒ द्वाद॑श - अ॒क्ष॒रा॒ । \newline
16. जग॑ती॒ जाग॑ता॒ जाग॑ता॒ जग॑ती॒ जग॑ती॒ जाग॑ताः । \newline
17. जाग॑ताः प॒शवः॑ प॒शवो॒ जाग॑ता॒ जाग॑ताः प॒शवः॑ । \newline
18. प॒शवो॒ जग॑त्या॒ जग॑त्या प॒शवः॑ प॒शवो॒ जग॑त्या । \newline
19. जग॑ त्यै॒वैव जग॑त्या॒ जग॑ त्यै॒व । \newline
20. ए॒वास्मा॑ अस्मा ए॒वै वास्मै᳚ । \newline
21. अ॒स्मै॒ प॒शून् प॒शू न॑स्मा अस्मै प॒शून् । \newline
22. प॒शू नवाव॑ प॒शून् प॒शू नव॑ । \newline
23. अव॑ रुन्धे रु॒न्धे ऽवाव॑ रुन्धे । \newline
24. रु॒न्धे॒ ऽष्टाव॑ष्टा व॒ष्टाव॑ष्टौ रुन्धे रुन्धे॒ ऽष्टाव॑ष्टौ । \newline
25. अ॒ष्टाव॑ष्टा व॒न्ये ष्व॒न्ये ष्व॒ष्टाव॑ष्टा व॒ष्टाव॑ष्टा व॒न्येषु॑ । \newline
26. अ॒ष्टाव॑ष्टा॒वित्य॒ष्टौ - अ॒ष्टौ॒ । \newline
27. अ॒न्येषु॒ धिष्णि॑येषु॒ धिष्णि॑ये ष्व॒न्ये ष्व॒न्येषु॒ धिष्णि॑येषु । \newline
28. धिष्णि॑ये॒ षूपोप॒ धिष्णि॑येषु॒ धिष्णि॑ये॒षूप॑ । \newline
29. उप॑ दधाति दधा॒ त्युपोप॑ दधाति । \newline
30. द॒धा॒ त्य॒ष्टाश॑फा अ॒ष्टाश॑फा दधाति दधा त्य॒ष्टाश॑फाः । \newline
31. अ॒ष्टाश॑फाः प॒शवः॑ प॒शवो॒ ऽष्टाश॑फा अ॒ष्टाश॑फाः प॒शवः॑ । \newline
32. अ॒ष्टाश॑फा॒ इत्य॒ष्टा - श॒फाः॒ । \newline
33. प॒शवः॑ प॒शून् प॒शून् प॒शवः॑ प॒शवः॑ प॒शून् । \newline
34. प॒शू ने॒वैव प॒शून् प॒शू ने॒व । \newline
35. ए॒वावा वै॒वै वाव॑ । \newline
36. अव॑ रुन्धे रु॒न्धे ऽवाव॑ रुन्धे । \newline
37. रु॒न्धे॒ षट् थ्षड् रु॑न्धे रुन्धे॒ षट् । \newline
38. षण् मा᳚र्जा॒लीये॑ मार्जा॒लीये॒ षट् थ्षण् मा᳚र्जा॒लीये᳚ । \newline
39. मा॒र्जा॒लीये॒ षट् थ्षण् मा᳚र्जा॒लीये॑ मार्जा॒लीये॒ षट् । \newline
40. षड् वै वै षट् थ्षड् वै । \newline
41. वा ऋ॒तव॑ ऋ॒तवो॒ वै वा ऋ॒तवः॑ । \newline
42. ऋ॒तव॑ ऋ॒तवः॑ । \newline
43. ऋ॒तवः॒ खलु॒ खल्वृ॒तव॑ ऋ॒तवः॒ खलु॑ । \newline
44. खलु॒ वै वै खलु॒ खलु॒ वै । \newline
45. वै दे॒वा दे॒वा वै वै दे॒वाः । \newline
46. दे॒वाः पि॒तरः॑ पि॒तरो॑ दे॒वा दे॒वाः पि॒तरः॑ । \newline
47. पि॒तर॑ ऋ॒तू नृ॒तून् पि॒तरः॑ पि॒तर॑ ऋ॒तून् । \newline
48. ऋ॒तू ने॒वैव र्‌तू नृ॒तू ने॒व । \newline
49. ए॒व दे॒वान् दे॒वा ने॒वैव दे॒वान् । \newline
50. दे॒वान् पि॒तॄन् पि॒तॄन् दे॒वान् दे॒वान् पि॒तॄन् । \newline
51. पि॒तॄन् प्री॑णाति प्रीणाति पि॒तॄन् पि॒तॄन् प्री॑णाति । \newline
52. प्री॒णा॒तीति॑ प्रीणाति । \newline

\textbf{Ghana Paata } \newline

1. होता॒ स्वे स्वे होता॒ होता॒ स्व ए॒वैव स्वे होता॒ होता॒ स्व ए॒व । \newline
2. स्व ए॒वैव स्वे स्व ए॒वास्मा॑ अस्मा ए॒व स्वे स्व ए॒वास्मै᳚ । \newline
3. ए॒वास्मा॑ अस्मा ए॒वै वास्मा॑ आ॒यत॑न आ॒यत॑ने ऽस्मा ए॒वै वास्मा॑ आ॒यत॑ने । \newline
4. अ॒स्मा॒ आ॒यत॑न आ॒यत॑ने ऽस्मा अस्मा आ॒यत॑न इन्द्रि॒य मि॑न्द्रि॒य मा॒यत॑ने ऽस्मा अस्मा आ॒यत॑न इन्द्रि॒यम् । \newline
5. आ॒यत॑न इन्द्रि॒य मि॑न्द्रि॒य मा॒यत॑न आ॒यत॑न इन्द्रि॒यं ॅवी॒र्यं॑ ॅवी॒र्य॑ मिन्द्रि॒य मा॒यत॑न आ॒यत॑न इन्द्रि॒यं ॅवी॒र्य᳚म् । \newline
6. आ॒यत॑न॒ इत्या᳚ - यत॑ने । \newline
7. इ॒न्द्रि॒यं ॅवी॒र्यं॑ ॅवी॒र्य॑ मिन्द्रि॒य मि॑न्द्रि॒यं ॅवी॒र्य॑ मवाव॑ वी॒र्य॑ मिन्द्रि॒य मि॑न्द्रि॒यं ॅवी॒र्य॑ मव॑ । \newline
8. वी॒र्य॑ मवाव॑ वी॒र्यं॑ ॅवी॒र्य॑ मव॑ रुन्धे रु॒न्धे ऽव॑ वी॒र्यं॑ ॅवी॒र्य॑ मव॑ रुन्धे । \newline
9. अव॑ रुन्धे रु॒न्धे ऽवाव॑ रुन्धे॒ द्वाद॑श॒ द्वाद॑श रु॒न्धे ऽवाव॑ रुन्धे॒ द्वाद॑श । \newline
10. रु॒न्धे॒ द्वाद॑श॒ द्वाद॑श रुन्धे रुन्धे॒ द्वाद॒शो पोप॒ द्वाद॑श रुन्धे रुन्धे॒ द्वाद॒शोप॑ । \newline
11. द्वाद॒शो पोप॒ द्वाद॑श॒ द्वाद॒शोप॑ दधाति दधा॒ त्युप॒ द्वाद॑श॒ द्वाद॒शोप॑ दधाति । \newline
12. उप॑ दधाति दधा॒ त्युपोप॑ दधाति॒ द्वाद॑शाक्षरा॒ द्वाद॑शाक्षरा दधा॒ त्युपोप॑ दधाति॒ द्वाद॑शाक्षरा । \newline
13. द॒धा॒ति॒ द्वाद॑शाक्षरा॒ द्वाद॑शाक्षरा दधाति दधाति॒ द्वाद॑शाक्षरा॒ जग॑ती॒ जग॑ती॒ द्वाद॑शाक्षरा दधाति दधाति॒ द्वाद॑शाक्षरा॒ जग॑ती । \newline
14. द्वाद॑शाक्षरा॒ जग॑ती॒ जग॑ती॒ द्वाद॑शाक्षरा॒ द्वाद॑शाक्षरा॒ जग॑ती॒ जाग॑ता॒ जाग॑ता॒ जग॑ती॒ द्वाद॑शाक्षरा॒ द्वाद॑शाक्षरा॒ जग॑ती॒ जाग॑ताः । \newline
15. द्वाद॑शाक्ष॒रेति॒ द्वाद॑श - अ॒क्ष॒रा॒ । \newline
16. जग॑ती॒ जाग॑ता॒ जाग॑ता॒ जग॑ती॒ जग॑ती॒ जाग॑ताः प॒शवः॑ प॒शवो॒ जाग॑ता॒ जग॑ती॒ जग॑ती॒ जाग॑ताः प॒शवः॑ । \newline
17. जाग॑ताः प॒शवः॑ प॒शवो॒ जाग॑ता॒ जाग॑ताः प॒शवो॒ जग॑त्या॒ जग॑त्या प॒शवो॒ जाग॑ता॒ जाग॑ताः प॒शवो॒ जग॑त्या । \newline
18. प॒शवो॒ जग॑त्या॒ जग॑त्या प॒शवः॑ प॒शवो॒ जग॑त्यै॒वैव जग॑त्या प॒शवः॑ प॒शवो॒ जग॑त्यै॒व । \newline
19. जग॑ त्यै॒वैव जग॑त्या॒ जग॑ त्यै॒वास्मा॑ अस्मा ए॒व जग॑त्या॒ जग॑ त्यै॒वास्मै᳚ । \newline
20. ए॒वास्मा॑ अस्मा ए॒वै वास्मै॑ प॒शून् प॒शू न॑स्मा ए॒वै वास्मै॑ प॒शून् । \newline
21. अ॒स्मै॒ प॒शून् प॒शू न॑स्मा अस्मै प॒शू नवाव॑ प॒शू न॑स्मा अस्मै प॒शू नव॑ । \newline
22. प॒शू नवाव॑ प॒शून् प॒शू नव॑ रुन्धे रु॒न्धे ऽव॑ प॒शून् प॒शू नव॑ रुन्धे । \newline
23. अव॑ रुन्धे रु॒न्धे ऽवाव॑ रुन्धे॒ ऽष्टाव॑ष्टा व॒ष्टाव॑ष्टौ रु॒न्धे ऽवाव॑ रुन्धे॒ ऽष्टाव॑ष्टौ । \newline
24. रु॒न्धे॒ ऽष्टाव॑ष्टा व॒ष्टाव॑ष्टौ रुन्धे रुन्धे॒ ऽष्टाव॑ष्टा व॒न्ये ष्व॒न्ये ष्व॒ष्टा व॑ष्टौ रुन्धे रुन्धे॒ ऽष्टाव॑ष्टा व॒न्येषु॑ । \newline
25. अ॒ष्टाव॑ष्टा व॒न्ये ष्व॒न्ये ष्व॒ष्टाव॑ष्टा व॒ष्टाव॑ष्टा व॒न्येषु॒ धिष्णि॑येषु॒ धिष्णि॑ये ष्व॒न्ये ष्व॒ष्टाव॑ष्टा व॒ष्टाव॑ष्टा व॒न्येषु॒ धिष्णि॑येषु । \newline
26. अ॒ष्टाव॑ष्टा॒वित्य॒ष्टौ - अ॒ष्टौ॒ । \newline
27. अ॒न्येषु॒ धिष्णि॑येषु॒ धिष्णि॑ये ष्व॒न्ये ष्व॒न्येषु॒ धिष्णि॑ये॒षू पोप॒ धिष्णि॑ये ष्व॒न्ये
ष्व॒न्येषु॒ धिष्णि॑ये॒षूप॑ । \newline
28. धिष्णि॑ये॒षू पोप॒ धिष्णि॑येषु॒ धिष्णि॑ये॒षूप॑ दधाति दधा॒ त्युप॒ धिष्णि॑येषु॒ धिष्णि॑ये॒षूप॑ दधाति । \newline
29. उप॑ दधाति दधा॒ त्युपोप॑ दधा त्य॒ष्टाश॑फा अ॒ष्टाश॑फा दधा॒ त्युपोप॑ दधा त्य॒ष्टाश॑फाः । \newline
30. द॒धा॒ त्य॒ष्टाश॑फा अ॒ष्टाश॑फा दधाति दधा त्य॒ष्टाश॑फाः प॒शवः॑ प॒शवो॒ ऽष्टाश॑फा दधाति दधा त्य॒ष्टाश॑फाः प॒शवः॑ । \newline
31. अ॒ष्टाश॑फाः प॒शवः॑ प॒शवो॒ ऽष्टाश॑फा अ॒ष्टाश॑फाः प॒शवः॑ प॒शून् प॒शून् प॒शवो॒ ऽष्टाश॑फा अ॒ष्टाश॑फाः प॒शवः॑ प॒शून् । \newline
32. अ॒ष्टाश॑फा॒ इत्य॒ष्टा - श॒फाः॒ । \newline
33. प॒शवः॑ प॒शून् प॒शून् प॒शवः॑ प॒शवः॑ प॒शू ने॒वैव प॒शून् प॒शवः॑ प॒शवः॑ प॒शू ने॒व । \newline
34. प॒शू ने॒वैव प॒शून् प॒शू ने॒वावा वै॒व प॒शून् प॒शू ने॒वाव॑ । \newline
35. ए॒वावा वै॒वै वाव॑ रुन्धे रु॒न्धे ऽवै॒वै वाव॑ रुन्धे । \newline
36. अव॑ रुन्धे रु॒न्धे ऽवाव॑ रुन्धे॒ षट् थ्षड् रु॒न्धे ऽवाव॑ रुन्धे॒ षट् । \newline
37. रु॒न्धे॒ षट् थ्षड् रु॑न्धे रुन्धे॒ षण् मा᳚र्जा॒लीये॑ मार्जा॒लीये॒ षड् रु॑न्धे रुन्धे॒ षण् मा᳚र्जा॒लीये᳚ । \newline
38. षण् मा᳚र्जा॒लीये॑ मार्जा॒लीये॒ षट् थ्षण् मा᳚र्जा॒लीये॒ षट् थ्षण् मा᳚र्जा॒लीये॒ षट् थ्षण् मा᳚र्जा॒लीये॒ षट् । \newline
39. मा॒र्जा॒लीये॒ षट् थ्षण् मा᳚र्जा॒लीये॑ मार्जा॒लीये॒ षड् वै वै षण् मा᳚र्जा॒लीये॑ मार्जा॒लीये॒ षड् वै । \newline
40. षड् वै वै षट् थ्षड् वा ऋ॒तव॑ ऋ॒तवो॒ वै षट् थ्षड् वा ऋ॒तवः॑ । \newline
41. वा ऋ॒तव॑ ऋ॒तवो॒ वै वा ऋ॒तवः॑ (ऋ॒तवः॑)  । \newline
42. ऋ॒तव॑ ऋ॒तवः॑ । \newline
43. ऋ॒तवः॒ खलु॒ खल्वृ॒तव॑ ऋ॒तवः॒ खलु॒ वै वै खल्वृ॒तव॑ ऋ॒तवः॒ खलु॒ वै । \newline
44. खलु॒ वै वै खलु॒ खलु॒ वै दे॒वा दे॒वा वै खलु॒ खलु॒ वै दे॒वाः । \newline
45. वै दे॒वा दे॒वा वै वै दे॒वाः पि॒तरः॑ पि॒तरो॑ दे॒वा वै वै दे॒वाः पि॒तरः॑ । \newline
46. दे॒वाः पि॒तरः॑ पि॒तरो॑ दे॒वा दे॒वाः पि॒तर॑ ऋ॒तू नृ॒तून् पि॒तरो॑ दे॒वा दे॒वाः पि॒तर॑ ऋ॒तून् । \newline
47. पि॒तर॑ ऋ॒तू नृ॒तून् पि॒तरः॑ पि॒तर॑ ऋ॒तू ने॒वैव र्‌तून् पि॒तरः॑ पि॒तर॑ ऋ॒तू ने॒व । \newline
48. ऋ॒तू ने॒वैव र्‌तू नृ॒तू ने॒व दे॒वान् दे॒वा ने॒व र्‌तू नृ॒तू ने॒व दे॒वान् । \newline
49. ए॒व दे॒वान् दे॒वा ने॒वैव दे॒वान् पि॒तॄन् पि॒तॄन् दे॒वा ने॒ वैव दे॒वान् पि॒तॄन् । \newline
50. दे॒वान् पि॒तॄन् पि॒तॄन् दे॒वान् दे॒वान् पि॒तॄन् प्री॑णाति प्रीणाति पि॒तॄन् दे॒वान् दे॒वान् पि॒तॄन् प्री॑णाति । \newline
51. पि॒तॄन् प्री॑णाति प्रीणाति पि॒तॄन् पि॒तॄन् प्री॑णाति । \newline
52. प्री॒णा॒तीति॑ प्रीणाति । \newline
\pagebreak
\markright{ TS 5.4.12.1  \hfill https://www.vedavms.in \hfill}

\section{ TS 5.4.12.1 }

\textbf{TS 5.4.12.1 } \newline
\textbf{Samhita Paata} \newline

पव॑स्व॒ वाज॑सातय॒ इत्य॑नु॒ष्टुक् प्र॑ति॒पद्भ॑वति ति॒र्.सो॑ऽनु॒ष्टुभ॒श्चत॑स्रो गाय॒त्रियो॒ यत् ति॒स्रो॑ऽनु॒ष्टुभ॒-स्तस्मा॒-दश्व॑स्त्रि॒भिस्तिष्ठꣳ॑ स्तिष्ठति॒ यच्चत॑स्रो गाय॒त्रिय॒स्तस्मा॒थ् सर्वाꣳ॑ श्च॒तुरः॑ प॒दः प्र॑ति॒दध॒त् पला॑यते पर॒मा वा ए॒षा छन्द॑सां॒ ॅयद॑नु॒ष्टुक् प॑र॒मश्च॑तुष्टो॒मः स्तोमा॑नां पर॒मस्त्रि॑रा॒त्रो य॒ज्ञानां᳚ पर॒मोऽश्वः॑ पशू॒नां प॑र॒मेणै॒वैनं॑ पर॒मतां᳚ गमयत्येकविꣳ॒॒शमह॑र्भवति॒ - [  ] \newline

\textbf{Pada Paata} \newline

पव॑स्व । वाज॑सातय॒ इति॒ वाज॑ - सा॒त॒ये॒ । इति॑ । अ॒नु॒ष्टुगित्य॑नु - स्तुक् । प्र॒ति॒पदिति॑ प्रति - पत् । भ॒व॒ति॒ । ति॒स्रः । अ॒नु॒ष्टुभ॒ इत्य॑नु - स्तुभः॑ । चत॑स्रः । गा॒य॒त्रियः॑ । यत् । ति॒स्रः । अ॒नु॒ष्टुभ॒ इत्य॑नु - स्तुभः॑ । तस्मा᳚त् । अश्वः॑ । त्रि॒भिरिति॑ त्रि-भिः । तिष्ठन्न्॑ । ति॒ष्ठ॒ति॒ । यत् । चत॑स्रः । गा॒य॒त्रियः॑ । तस्मा᳚त् । सर्वान्॑ । च॒तुरः॑ । प॒दः । प्र॒ति॒दध॒दिति॑ प्रति - दध॑त् । पला॑यते । प॒र॒मा । वै । ए॒षा । छन्द॑साम् । यत् । अ॒नु॒ष्टुगित्य॑नु - स्तुक् । प॒र॒मः । च॒तु॒ष्टो॒म इति॑ चतुः - स्तो॒मः । स्तोमा॑नाम् । प॒र॒मः । त्रि॒रा॒त्र इति॑ त्रि-रा॒त्रः । य॒ज्ञाना᳚म् । प॒र॒मः । अश्वः॑ । प॒शू॒नाम् । प॒र॒मेण॑ । ए॒व । ए॒न॒म् । प॒र॒मता᳚म् । ग॒म॒य॒ति॒ । ए॒क॒विꣳ॒॒शमित्ये॑क - विꣳ॒॒शम् । अहः॑ । भ॒व॒ति॒ ।  \newline


\textbf{Krama Paata} \newline

पव॑स्व॒ वाज॑सातये । वाज॑सातय॒ इति॑ । वाज॑सातय॒ इति॒ वाज॑ - सा॒त॒ये॒ । इत्य॑नु॒ष्टुक् । अ॒नु॒ष्टुक् प्र॑ति॒पत् । अ॒नु॒ष्टुगित्य॑नु - स्तुक् । प्र॒ति॒पद् भ॑वति । प्र॒ति॒पदिति॑ प्रति - पत् । भ॒व॒ति॒ ति॒स्रः । त्रि॒स्रो॑ऽनु॒ष्टुभः॑ । अ॒नु॒ष्टुभः॒ चत॑स्रः । अ॒नु॒ष्टुभ॒ इत्य॑नु - स्तुभः॑ । चत॑स्रो गाय॒त्रियः॑ । गा॒य॒त्रियो॒ यत् । यत् ति॒स्रः । ति॒स्रो॑ऽनु॒ष्टुभः॑ । अ॒नु॒ष्टुभ॒स्तस्मा᳚त् । अ॒नु॒ष्टभ॒ इत्य॑नु - स्तुभः॑ । तस्मा॒दश्वः॑ । अश्व॑स्त्रि॒भिः । त्रि॒भिस्तिष्ठन्न्॑ । त्रि॒भिरिति॑ त्रि - भिः । तिष्ठꣳ॑स्तिष्ठति । ति॒ष्ठ॒ति॒ यत् । यच् चत॑स्रः । चत॑स्रो गाय॒त्रियः॑ । गा॒य॒त्रिय॒स्तस्मा᳚त् । तस्मा॒थ् सर्वान्॑ । सर्वाꣳ॑श्च॒तुरः॑ । च॒तुरः॑ प॒दः । प॒दः प्र॑ति॒दध॑त् । प्र॒ति॒दध॒त् पला॑यते । प्र॒ति॒दध॒दिति॑ प्रति - दध॑त् । पला॑यते पर॒मा । प॒र॒मा वै । वा ए॒षा । ए॒षा छन्द॑साम् । छन्द॑सा॒म् ॅयत् । यद॑नु॒ष्टुक् । अ॒नु॒ष्टुक् प॑र॒मः । अ॒नु॒ष्टुगित्य॑नु - स्तुक् । प॒र॒मश्च॑तुष्टो॒मः । च॒तु॒ष्टो॒मः स्तोमा॑नाम् । च॒तु॒ष्टो॒म इति॑ चतुः - स्तो॒मः । स्तोमा॑नाम् पर॒मः । प॒र॒मस्त्रि॑रा॒त्रः । त्रि॒रा॒त्रो य॒ज्ञाना᳚म् । त्रि॒रा॒त्र इति॑ त्रि - रा॒त्रः । य॒ज्ञाना᳚म् पर॒मः । प॒र॒मोऽश्वः॑ । अश्वः॑ पशू॒नाम् । प॒शू॒नाम् प॑र॒मेण॑ । प॒र॒मेणै॒व । ए॒वैन᳚म् । ए॒न॒म् प॒र॒मता᳚म् । प॒र॒मता᳚म् गमयति । ग॒म॒य॒त्ये॒क॒विꣳ॒॒शम् । ए॒क॒विꣳ॒॒शमहः॑ । ए॒क॒विꣳ॒॒शमित्ये॑क - विꣳ॒॒शम् । अह॑र् भवति । भ॒व॒ति॒ यस्मिन्न्॑ \newline

\textbf{Jatai Paata} \newline

1. पव॑स्व॒ वाज॑सातये॒ वाज॑सातये॒ पव॑स्व॒ पव॑स्व॒ वाज॑सातये । \newline
2. वाज॑सातय॒ इतीति॒ वाज॑सातये॒ वाज॑सातय॒ इति॑ । \newline
3. वाज॑सातय॒ इति॒ वाज॑ - सा॒त॒ये॒ । \newline
4. इत्य॑नु॒ष्टु ग॑नु॒ष्टु गिती त्य॑नु॒ष्टुक् । \newline
5. अ॒नु॒ष्टुक् प्र॑ति॒पत् प्र॑ति॒प द॑नु॒ष्टु ग॑नु॒ष्टुक् प्र॑ति॒पत् । \newline
6. अ॒नु॒ष्टुगित्य॑नु - स्तुक् । \newline
7. प्र॒ति॒पद् भ॑वति भवति प्रति॒पत् प्र॑ति॒पद् भ॑वति । \newline
8. प्र॒ति॒पदिति॑ प्रति - पत् । \newline
9. भ॒व॒ति॒ ति॒स्र स्ति॒स्रो भ॑वति भवति ति॒स्रः । \newline
10. ति॒स्रो॑ ऽनु॒ष्टुभो॑ ऽनु॒ष्टुभ॑ स्ति॒स्र स्ति॒स्रो॑ ऽनु॒ष्टुभः॑ । \newline
11. अ॒नु॒ष्टुभ॒ श्चत॑स्र॒ श्चत॑स्रो ऽनु॒ष्टुभो॑ ऽनु॒ष्टुभ॒ श्चत॑स्रः । \newline
12. अ॒नु॒ष्टुभ॒ इत्य॑नु - स्तुभः॑ । \newline
13. चत॑स्रो गाय॒त्रियो॑ गाय॒त्रिय॒ श्चत॑स्र॒ श्चत॑स्रो गाय॒त्रियः॑ । \newline
14. गा॒य॒त्रियो॒ यद् यद् गा॑य॒त्रियो॑ गाय॒त्रियो॒ यत् । \newline
15. यत् ति॒स्र स्ति॒स्रो यद् यत् ति॒स्रः । \newline
16. ति॒स्रो॑ ऽनु॒ष्टुभो॑ ऽनु॒ष्टुभ॑ स्ति॒स्र स्ति॒स्रो॑ ऽनु॒ष्टुभः॑ । \newline
17. अ॒नु॒ष्टुभ॒ स्तस्मा॒त् तस्मा॑ दनु॒ष्टुभो॑ ऽनु॒ष्टुभ॒ स्तस्मा᳚त् । \newline
18. अ॒नु॒ष्टुभ॒ इत्य॑नु - स्तुभः॑ । \newline
19. तस्मा॒ दश्वो ऽश्व॒ स्तस्मा॒त् तस्मा॒ दश्वः॑ । \newline
20. अश्व॑ स्त्रि॒भि स्त्रि॒भि रश्वो ऽश्व॑ स्त्रि॒भिः । \newline
21. त्रि॒भि स्तिष्ठꣳ॒॒ स्तिष्ठꣳ॑ स्त्रि॒भि स्त्रि॒भि स्तिष्ठन्न्॑ । \newline
22. त्रि॒भिरिति॑ त्रि - भिः । \newline
23. तिष्ठꣳ॑ स्तिष्ठति तिष्ठति॒ तिष्ठꣳ॒॒ स्तिष्ठꣳ॑ स्तिष्ठति । \newline
24. ति॒ष्ठ॒ति॒ यद् यत् ति॑ष्ठति तिष्ठति॒ यत् । \newline
25. यच् चत॑स्र॒ श्चत॑स्रो॒ यद् यच् चत॑स्रः । \newline
26. चत॑स्रो गाय॒त्रियो॑ गाय॒त्रिय॒ श्चत॑स्र॒ श्चत॑स्रो गाय॒त्रियः॑ । \newline
27. गा॒य॒त्रिय॒ स्तस्मा॒त् तस्मा᳚द् गाय॒त्रियो॑ गाय॒त्रिय॒ स्तस्मा᳚त् । \newline
28. तस्मा॒थ् सर्वा॒न् थ्सर्वा॒न् तस्मा॒त् तस्मा॒थ् सर्वान्॑ । \newline
29. सर्वाꣳ॑ श्च॒तुर॑ श्च॒तुरः॒ सर्वा॒न् थ्सर्वाꣳ॑ श्च॒तुरः॑ । \newline
30. च॒तुरः॑ प॒दः प॒द श्च॒तुर॑ श्च॒तुरः॑ प॒दः । \newline
31. प॒दः प्र॑ति॒दध॑त् प्रति॒दध॑त् प॒दः प॒दः प्र॑ति॒दध॑त् । \newline
32. प्र॒ति॒दध॒त् पला॑यते॒ पला॑यते प्रति॒दध॑त् प्रति॒दध॒त् पला॑यते । \newline
33. प्र॒ति॒दध॒दिति॑ प्रति - दध॑त् । \newline
34. पला॑यते पर॒मा प॑र॒मा पला॑यते॒ पला॑यते पर॒मा । \newline
35. प॒र॒मा वै वै प॑र॒मा प॑र॒मा वै । \newline
36. वा ए॒षैषा वै वा ए॒षा । \newline
37. ए॒षा छन्द॑सा॒म् छन्द॑सा मे॒षैषा छन्द॑साम् । \newline
38. छन्द॑सां॒ ॅयद् यच् छन्द॑सा॒म् छन्द॑सां॒ ॅयत् । \newline
39. यद॑नु॒ष्टु ग॑नु॒ष्टुग् यद् यद॑नु॒ष्टुक् । \newline
40. अ॒नु॒ष्टुक् प॑र॒मः प॑र॒मो॑ ऽनु॒ष्टु ग॑नु॒ष्टुक् प॑र॒मः । \newline
41. अ॒नु॒ष्टुगित्य॑नु - स्तुक् । \newline
42. प॒र॒म श्च॑तुष्टो॒म श्च॑तुष्टो॒मः प॑र॒मः प॑र॒म श्च॑तुष्टो॒मः । \newline
43. च॒तु॒ष्टो॒मः स्तोमा॑नाꣳ॒॒ स्तोमा॑नाम् चतुष्टो॒म श्च॑तुष्टो॒मः स्तोमा॑नाम् । \newline
44. च॒तु॒ष्टो॒म इति॑ चतुः - स्तो॒मः । \newline
45. स्तोमा॑नाम् पर॒मः प॑र॒मः स्तोमा॑नाꣳ॒॒ स्तोमा॑नाम् पर॒मः । \newline
46. प॒र॒म स्त्रि॑रा॒त्र स्त्रि॑रा॒त्रः प॑र॒मः प॑र॒म स्त्रि॑रा॒त्रः । \newline
47. त्रि॒रा॒त्रो य॒ज्ञानां᳚ ॅय॒ज्ञाना᳚म् त्रिरा॒त्र स्त्रि॑रा॒त्रो य॒ज्ञाना᳚म् । \newline
48. त्रि॒रा॒त्र इति॑ त्रि - रा॒त्रः । \newline
49. य॒ज्ञाना᳚म् पर॒मः प॑र॒मो य॒ज्ञानां᳚ ॅय॒ज्ञाना᳚म् पर॒मः । \newline
50. प॒र॒मो ऽश्वो ऽश्वः॑ पर॒मः प॑र॒मो ऽश्वः॑ । \newline
51. अश्वः॑ पशू॒नाम् प॑शू॒ना मश्वो ऽश्वः॑ पशू॒नाम् । \newline
52. प॒शू॒नाम् प॑र॒मेण॑ पर॒मेण॑ पशू॒नाम् प॑शू॒नाम् प॑र॒मेण॑ । \newline
53. प॒र॒मे णै॒वैव प॑र॒मेण॑ पर॒मेणै॒व । \newline
54. ए॒वैन॑ मेन मे॒वै वैन᳚म् । \newline
55. ए॒न॒म् प॒र॒मता᳚म् पर॒मता॑ मेन मेनम् पर॒मता᳚म् । \newline
56. प॒र॒मता᳚म् गमयति गमयति पर॒मता᳚म् पर॒मता᳚म् गमयति । \newline
57. ग॒म॒य॒ त्ये॒क॒विꣳ॒॒श मे॑कविꣳ॒॒शम् ग॑मयति गमय त्येकविꣳ॒॒शम् । \newline
58. ए॒क॒विꣳ॒॒श मह॒ रह॑ रेकविꣳ॒॒श मे॑कविꣳ॒॒श महः॑ । \newline
59. ए॒क॒विꣳ॒॒शमित्ये॑क - विꣳ॒॒शम् । \newline
60. अह॑र् भवति भव॒ त्यह॒ रह॑र् भवति । \newline
61. भ॒व॒ति॒ यस्मि॒न्॒. यस्मि॑न् भवति भवति॒ यस्मिन्न्॑ । \newline

\textbf{Ghana Paata } \newline

1. पव॑स्व॒ वाज॑सातये॒ वाज॑सातये॒ पव॑स्व॒ पव॑स्व॒ वाज॑सातय॒ इतीति॒ वाज॑सातये॒ पव॑स्व॒ पव॑स्व॒ वाज॑सातय॒ इति॑ । \newline
2. वाज॑सातय॒ इतीति॒ वाज॑सातये॒ वाज॑सातय॒ इत्य॑नु॒ष्टु ग॑नु॒ष्टुगिति॒ वाज॑सातये॒ वाज॑सातय॒ इत्य॑नु॒ष्टुक् । \newline
3. वाज॑सातय॒ इति॒ वाज॑ - सा॒त॒ये॒ । \newline
4. इत्य॑नु॒ष्टु ग॑नु॒ष्टु गिती त्य॑नु॒ष्टुक् प्र॑ति॒पत् प्र॑ति॒प द॑नु॒ष्टु गिती त्य॑नु॒ष्टुक् प्र॑ति॒पत् । \newline
5. अ॒नु॒ष्टुक् प्र॑ति॒पत् प्र॑ति॒प द॑नु॒ष्टु ग॑नु॒ष्टुक् प्र॑ति॒पद् भ॑वति भवति प्रति॒प द॑नु॒ष्टु ग॑नु॒ष्टुक् प्र॑ति॒पद् भ॑वति । \newline
6. अ॒नु॒ष्टुगित्य॑नु - स्तुक् । \newline
7. प्र॒ति॒पद् भ॑वति भवति प्रति॒पत् प्र॑ति॒पद् भ॑वति ति॒स्र स्ति॒स्रो भ॑वति प्रति॒पत् प्र॑ति॒पद् भ॑वति ति॒स्रः । \newline
8. प्र॒ति॒पदिति॑ प्रति - पत् । \newline
9. भ॒व॒ति॒ ति॒स्र स्ति॒स्रो भ॑वति भवति ति॒स्रो॑ ऽनु॒ष्टुभो॑ ऽनु॒ष्टुभ॑ स्ति॒स्रो भ॑वति भवति ति॒स्रो॑ ऽनु॒ष्टुभः॑ । \newline
10. ति॒स्रो॑ ऽनु॒ष्टुभो॑ ऽनु॒ष्टुभ॑ स्ति॒स्र स्ति॒स्रो॑ ऽनु॒ष्टुभ॒ श्चत॑स्र॒ श्चत॑स्रो ऽनु॒ष्टुभ॑ स्ति॒स्र स्ति॒स्रो॑ ऽनु॒ष्टुभ॒ श्चत॑स्रः । \newline
11. अ॒नु॒ष्टुभ॒ श्चत॑स्र॒ श्चत॑स्रो ऽनु॒ष्टुभो॑ ऽनु॒ष्टुभ॒ श्चत॑स्रो गाय॒त्रियो॑ गाय॒त्रिय॒ श्चत॑स्रो ऽनु॒ष्टुभो॑ ऽनु॒ष्टुभ॒ श्चत॑स्रो गाय॒त्रियः॑ । \newline
12. अ॒नु॒ष्टुभ॒ इत्य॑नु - स्तुभः॑ । \newline
13. चत॑स्रो गाय॒त्रियो॑ गाय॒त्रिय॒ श्चत॑स्र॒ श्चत॑स्रो गाय॒त्रियो॒ यद् यद् गा॑य॒त्रिय॒ श्चत॑स्र॒ श्चत॑स्रो गाय॒त्रियो॒ यत् । \newline
14. गा॒य॒त्रियो॒ यद् यद् गा॑य॒त्रियो॑ गाय॒त्रियो॒ यत् ति॒स्र स्ति॒स्रो यद् गा॑य॒त्रियो॑ गाय॒त्रियो॒ यत् ति॒स्रः । \newline
15. यत् ति॒स्र स्ति॒स्रो यद् यत् ति॒स्रो॑ ऽनु॒ष्टुभो॑ ऽनु॒ष्टुभ॑ स्ति॒स्रो यद् यत् ति॒स्रो॑ ऽनु॒ष्टुभः॑ । \newline
16. ति॒स्रो॑ ऽनु॒ष्टुभो॑ ऽनु॒ष्टुभ॑ स्ति॒स्र स्ति॒स्रो॑ ऽनु॒ष्टुभ॒ स्तस्मा॒त् तस्मा॑ दनु॒ष्टुभ॑ स्ति॒स्र स्ति॒स्रो॑ ऽनु॒ष्टुभ॒ स्तस्मा᳚त् । \newline
17. अ॒नु॒ष्टुभ॒ स्तस्मा॒त् तस्मा॑ दनु॒ष्टुभो॑ ऽनु॒ष्टुभ॒ स्तस्मा॒ दश्वो ऽश्व॒ स्तस्मा॑ दनु॒ष्टुभो॑ ऽनु॒ष्टुभ॒ स्तस्मा॒ दश्वः॑ । \newline
18. अ॒नु॒ष्टुभ॒ इत्य॑नु - स्तुभः॑ । \newline
19. तस्मा॒ दश्वो ऽश्व॒ स्तस्मा॒त् तस्मा॒ दश्व॑ स्त्रि॒भि स्त्रि॒भि रश्व॒ स्तस्मा॒त् तस्मा॒ दश्व॑ स्त्रि॒भिः । \newline
20. अश्व॑ स्त्रि॒भि स्त्रि॒भि रश्वो ऽश्व॑ स्त्रि॒भि स्तिष्ठꣳ॒॒ स्तिष्ठꣳ॑ स्त्रि॒भि रश्वो ऽश्व॑ स्त्रि॒भि स्तिष्ठन्न्॑ । \newline
21. त्रि॒भि स्तिष्ठꣳ॒॒ स्तिष्ठꣳ॑ स्त्रि॒भि स्त्रि॒भि स्तिष्ठꣳ॑ स्तिष्ठति तिष्ठति॒ तिष्ठꣳ॑ स्त्रि॒भि स्त्रि॒भि स्तिष्ठꣳ॑ स्तिष्ठति । \newline
22. त्रि॒भिरिति॑ त्रि - भिः । \newline
23. तिष्ठꣳ॑ स्तिष्ठति तिष्ठति॒ तिष्ठꣳ॒॒ स्तिष्ठꣳ॑ स्तिष्ठति॒ यद् यत् ति॑ष्ठति॒ तिष्ठꣳ॒॒ स्तिष्ठꣳ॑ स्तिष्ठति॒ यत् । \newline
24. ति॒ष्ठ॒ति॒ यद् यत् ति॑ष्ठति तिष्ठति॒ यच् चत॑स्र॒ श्चत॑स्रो॒ यत् ति॑ष्ठति तिष्ठति॒ यच् चत॑स्रः । \newline
25. यच् चत॑स्र॒ श्चत॑स्रो॒ यद् यच् चत॑स्रो गाय॒त्रियो॑ गाय॒त्रिय॒ श्चत॑स्रो॒ यद् यच् चत॑स्रो गाय॒त्रियः॑ । \newline
26. चत॑स्रो गाय॒त्रियो॑ गाय॒त्रिय॒ श्चत॑स्र॒ श्चत॑स्रो गाय॒त्रिय॒ स्तस्मा॒त् तस्मा᳚द् गाय॒त्रिय॒ श्चत॑स्र॒ श्चत॑स्रो गाय॒त्रिय॒ स्तस्मा᳚त् । \newline
27. गा॒य॒त्रिय॒ स्तस्मा॒त् तस्मा᳚द् गाय॒त्रियो॑ गाय॒त्रिय॒ स्तस्मा॒थ् सर्वा॒न् थ्सर्वा॒न् तस्मा᳚द् गाय॒त्रियो॑ गाय॒त्रिय॒ स्तस्मा॒थ् सर्वान्॑ । \newline
28. तस्मा॒थ् सर्वा॒न् थ्सर्वा॒न् तस्मा॒त् तस्मा॒थ् सर्वाꣳ॑ श्च॒तुर॑ श्च॒तुरः॒ सर्वा॒न् तस्मा॒त् तस्मा॒थ् 
सर्वाꣳ॑ श्च॒तुरः॑ । \newline
29. सर्वाꣳ॑ श्च॒तुर॑ श्च॒तुरः॒ सर्वा॒न् थ्सर्वाꣳ॑ श्च॒तुरः॑ प॒दः प॒द श्च॒तुरः॒ सर्वा॒न् 
थ्सर्वाꣳ॑ श्च॒तुरः॑ प॒दः । \newline
30. च॒तुरः॑ प॒दः प॒द श्च॒तुर॑ श्च॒तुरः॑ प॒दः प्र॑ति॒दध॑त् प्रति॒दध॑त् प॒द श्च॒तुर॑ श्च॒तुरः॑ प॒दः प्र॑ति॒दध॑त् । \newline
31. प॒दः प्र॑ति॒दध॑त् प्रति॒दध॑त् प॒दः प॒दः प्र॑ति॒दध॒त् पला॑यते॒ पला॑यते प्रति॒दध॑त् प॒दः प॒दः प्र॑ति॒दध॒त् पला॑यते । \newline
32. प्र॒ति॒दध॒त् पला॑यते॒ पला॑यते प्रति॒दध॑त् प्रति॒दध॒त् पला॑यते पर॒मा प॑र॒मा पला॑यते प्रति॒दध॑त् प्रति॒दध॒त् पला॑यते पर॒मा । \newline
33. प्र॒ति॒दध॒दिति॑ प्रति - दध॑त् । \newline
34. पला॑यते पर॒मा प॑र॒मा पला॑यते॒ पला॑यते पर॒मा वै वै प॑र॒मा पला॑यते॒ पला॑यते पर॒मा वै । \newline
35. प॒र॒मा वै वै प॑र॒मा प॑र॒मा वा ए॒षैषा वै प॑र॒मा प॑र॒मा वा ए॒षा । \newline
36. वा ए॒षैषा वै वा ए॒षा छन्द॑सा॒म् छन्द॑सा मे॒षा वै वा ए॒षा छन्द॑साम् । \newline
37. ए॒षा छन्द॑सा॒म् छन्द॑सा मे॒षैषा छन्द॑सां॒ ॅयद् यच् छन्द॑सा मे॒षैषा छन्द॑सां॒ ॅयत् । \newline
38. छन्द॑सां॒ ॅयद् यच् छन्द॑सा॒म् छन्द॑सां॒ ॅयद॑नु॒ष्टु ग॑नु॒ष्टुग् यच् छन्द॑सा॒म् छन्द॑सां॒ ॅयद॑नु॒ष्टुक् । \newline
39. यद॑नु॒ष्टु ग॑नु॒ष्टुग् यद् यद॑नु॒ष्टुक् प॑र॒मः प॑र॒मो॑ ऽनु॒ष्टुग् यद् यद॑नु॒ष्टुक् प॑र॒मः । \newline
40. अ॒नु॒ष्टुक् प॑र॒मः प॑र॒मो॑ ऽनु॒ष्टु ग॑नु॒ष्टुक् प॑र॒म श्च॑तुष्टो॒म श्च॑तुष्टो॒मः प॑र॒मो॑ ऽनु॒ष्टु ग॑नु॒ष्टुक् प॑र॒म श्च॑तुष्टो॒मः । \newline
41. अ॒नु॒ष्टुगित्य॑नु - स्तुक् । \newline
42. प॒र॒म श्च॑तुष्टो॒म श्च॑तुष्टो॒मः प॑र॒मः प॑र॒म श्च॑तुष्टो॒मः स्तोमा॑नाꣳ॒॒ स्तोमा॑नाम् चतुष्टो॒मः प॑र॒मः प॑र॒म श्च॑तुष्टो॒मः स्तोमा॑नाम् । \newline
43. च॒तु॒ष्टो॒मः स्तोमा॑नाꣳ॒॒ स्तोमा॑नाम् चतुष्टो॒म श्च॑तुष्टो॒मः स्तोमा॑नाम् पर॒मः प॑र॒मः स्तोमा॑नाम् चतुष्टो॒म श्च॑तुष्टो॒मः स्तोमा॑नाम् पर॒मः । \newline
44. च॒तु॒ष्टो॒म इति॑ चतुः - स्तो॒मः । \newline
45. स्तोमा॑नाम् पर॒मः प॑र॒मः स्तोमा॑नाꣳ॒॒ स्तोमा॑नाम् पर॒म स्त्रि॑रा॒त्र स्त्रि॑रा॒त्रः प॑र॒मः स्तोमा॑नाꣳ॒॒ स्तोमा॑नाम् पर॒म स्त्रि॑रा॒त्रः । \newline
46. प॒र॒म स्त्रि॑रा॒त्र स्त्रि॑रा॒त्रः प॑र॒मः प॑र॒म स्त्रि॑रा॒त्रो य॒ज्ञानां᳚ ॅय॒ज्ञाना᳚म् त्रिरा॒त्रः प॑र॒मः प॑र॒म स्त्रि॑रा॒त्रो य॒ज्ञाना᳚म् । \newline
47. त्रि॒रा॒त्रो य॒ज्ञानां᳚ ॅय॒ज्ञाना᳚म् त्रिरा॒त्र स्त्रि॑रा॒त्रो य॒ज्ञाना᳚म् पर॒मः प॑र॒मो य॒ज्ञाना᳚म् त्रिरा॒त्र स्त्रि॑रा॒त्रो य॒ज्ञाना᳚म् पर॒मः । \newline
48. त्रि॒रा॒त्र इति॑ त्रि - रा॒त्रः । \newline
49. य॒ज्ञाना᳚म् पर॒मः प॑र॒मो य॒ज्ञानां᳚ ॅय॒ज्ञाना᳚म् पर॒मो ऽश्वो ऽश्वः॑ पर॒मो य॒ज्ञानां᳚ ॅय॒ज्ञाना᳚म् पर॒मो ऽश्वः॑ । \newline
50. प॒र॒मो ऽश्वो ऽश्वः॑ पर॒मः प॑र॒मो ऽश्वः॑ पशू॒नाम् प॑शू॒ना मश्वः॑ पर॒मः प॑र॒मो ऽश्वः॑ पशू॒नाम् । \newline
51. अश्वः॑ पशू॒नाम् प॑शू॒ना मश्वो ऽश्वः॑ पशू॒नाम् प॑र॒मेण॑ पर॒मेण॑ पशू॒ना मश्वो ऽश्वः॑ पशू॒नाम् प॑र॒मेण॑ । \newline
52. प॒शू॒नाम् प॑र॒मेण॑ पर॒मेण॑ पशू॒नाम् प॑शू॒नाम् प॑र॒मेणै॒वैव प॑र॒मेण॑ पशू॒नाम् प॑शू॒नाम् प॑र॒मेणै॒व । \newline
53. प॒र॒मेणै॒ वैव प॑र॒मेण॑ पर॒मेणै॒ वैन॑ मेन मे॒व प॑र॒मेण॑ पर॒मे णै॒वैन᳚म् । \newline
54. ए॒वैन॑ मेन मे॒वै वैन॑म् पर॒मता᳚म् पर॒मता॑ मेन मे॒वै वैन॑म् पर॒मता᳚म् । \newline
55. ए॒न॒म् प॒र॒मता᳚म् पर॒मता॑ मेन मेनम् पर॒मता᳚म् गमयति गमयति पर॒मता॑ मेन मेनम् पर॒मता᳚म् गमयति । \newline
56. प॒र॒मता᳚म् गमयति गमयति पर॒मता᳚म् पर॒मता᳚म् गमय त्येकविꣳ॒॒श मे॑कविꣳ॒॒शम् ग॑मयति पर॒मता᳚म् पर॒मता᳚म् गमय त्येकविꣳ॒॒शम् । \newline
57. ग॒म॒य॒ त्ये॒क॒विꣳ॒॒श मे॑कविꣳ॒॒शम् ग॑मयति गमय त्येकविꣳ॒॒श मह॒ रह॑ रेकविꣳ॒॒शम् ग॑मयति गमय त्येकविꣳ॒॒श महः॑ । \newline
58. ए॒क॒विꣳ॒॒श मह॒ रह॑ रेकविꣳ॒॒श मे॑कविꣳ॒॒श मह॑र् भवति भव॒ त्यह॑ रेकविꣳ॒॒श मे॑कविꣳ॒॒श मह॑र् भवति । \newline
59. ए॒क॒विꣳ॒॒शमित्ये॑क - विꣳ॒॒शम् । \newline
60. अह॑र् भवति भव॒ त्यह॒ रह॑र् भवति॒ यस्मि॒न्॒. यस्मि॑न् भव॒ त्यह॒ रह॑र् भवति॒ यस्मिन्न्॑ । \newline
61. भ॒व॒ति॒ यस्मि॒न्॒. यस्मि॑न् भवति भवति॒ यस्मि॒न् नश्वो ऽश्वो॒ यस्मि॑न् भवति भवति॒ यस्मि॒न् नश्वः॑ । \newline
\pagebreak
\markright{ TS 5.4.12.2  \hfill https://www.vedavms.in \hfill}

\section{ TS 5.4.12.2 }

\textbf{TS 5.4.12.2 } \newline
\textbf{Samhita Paata} \newline

यस्मि॒न्नश्व॑ आल॒भ्यते॒ द्वाद॑श॒ मासाः॒ पञ्च॒र्तव॒स्त्रय॑ इ॒मे लो॒का अ॒सावा॑दि॒त्य ए॑कविꣳ॒॒श ए॒ष प्र॒जाप॑तिः प्राजाप॒त्योऽश्व॒स्तमे॒व सा॒क्षादृ॑द्ध्नोति॒ शक्व॑रयः पृ॒ष्ठं भ॑वन्त्य॒न्-यद॑न्य॒-च्छन्दो॒ऽन्ये᳚न्ये॒ वा ए॒ते प॒शव॒ आ ल॑भ्यन्त उ॒तेव॑ ग्रा॒म्या उ॒तेवा॑ऽऽ*र॒ण्या यच्छक्व॑रयः पृ॒ष्ठं भव॒न्त्यश्व॑स्य सर्व॒त्वाय॑ पार्थुर॒श्मं ब्र॑ह्मसा॒मं भ॑वति र॒श्मिना॒ वा अश्वो॑ - [  ] \newline

\textbf{Pada Paata} \newline

यस्मिन्न्॑ । अश्वः॑ । आ॒ल॒भ्यत॒ इत्या᳚ - ल॒भ्यते᳚ । द्वाद॑श । मासाः᳚ । पञ्च॑ । ऋ॒तवः॑ । त्रयः॑ । इ॒मे । लो॒काः । अ॒सौ । आ॒दि॒त्यः । ए॒क॒विꣳ॒॒श इत्ये॑क - विꣳ॒॒शः । ए॒षः । प्र॒जाप॑ति॒रिति॑ प्र॒जा-प॒तिः॒ । प्रा॒जा॒प॒त्य इति॑ प्राजा - प॒त्यः । अश्वः॑ । तम् । ए॒व । सा॒क्षादिति॑ स - अ॒क्षात् । ऋ॒द्ध्नो॒ति॒ । शक्व॑रयः । पृ॒ष्ठम् । भ॒व॒न्ति॒ । अ॒न्यद॑न्य॒दित्य॒न्यत् - अ॒न्य॒त् । छन्दः॑ । अ॒न्ये᳚ऽन्य॒ इत्य॒न्ये - अ॒न्ये॒ । वै । ए॒ते । प॒शवः॑ । एति॑ । ल॒भ्य॒न्ते॒ । उ॒त । इ॒व॒ । ग्रा॒म्याः । उ॒त । इ॒व॒ । आ॒र॒ण्याः । यत् । शक्व॑रयः । पृ॒ष्ठम् । भव॑न्ति । अश्व॑स्य । स॒र्व॒त्वायेति॑ सर्व - त्वाय॑ । पा॒र्थु॒र॒श्ममिति॑ पार्थु - र॒श्मम् । ब्र॒ह्म॒सा॒ममिति॑ ब्रह्म - सा॒मम् । भ॒व॒ति॒ । र॒श्मिना᳚ । वै । अश्वः॑ ।  \newline


\textbf{Krama Paata} \newline

यस्मि॒न्नश्वः॑ । अश्व॑ आल॒भ्यते॑ । आ॒ल॒भ्यते॒ द्वाद॑श । आ॒ल॒भ्यत॒ इत्या᳚ - ल॒भ्यते᳚ । द्वाद॑श॒ मासाः᳚ । मासाः॒ पञ्च॑ । पञ्च॒र्तवः॑ । ऋ॒तव॒स्त्रयः॑ । त्रय॑ इ॒मे । इ॒मे लो॒काः । लो॒का अ॒सौ । अ॒सावा॑दि॒त्यः । आ॒दि॒त्य ए॑कविꣳ॒॒शः । ए॒क॒विꣳ॒॒श ए॒षः । ए॒क॒विꣳ॒॒श इत्ये॑क - विꣳ॒॒शः । ए॒ष प्र॒जाप॑तिः । प्र॒जाप॑तिः प्राजाप॒त्यः । प्र॒जाप॑ति॒रिति॑ प्र॒जा - प॒तिः॒ । प्रा॒जा॒प॒त्योऽश्वः॑ । प्रा॒जा॒प॒त्य इति॑ प्राजा - प॒त्यः । अश्व॒स्तम् । तमे॒व । ए॒व सा॒क्षात् । सा॒क्षादृ॑द्ध्नोति । सा॒क्षादिति॑ स - अ॒क्षात् । ऋ॒द्ध्नो॒ति॒ शक्व॑रयः । शक्व॑रयः पृ॒ष्ठम् । पृ॒ष्ठम् भ॑वन्ति । भ॒व॒न्त्य॒न्यद॑न्यत् । अ॒न्यद॑न्य॒च्छन्दः॑ । अ॒न्यद॑न्य॒दित्य॒न्यत् - अ॒न्य॒त्॒ । छन्दो॒ऽन्ये᳚न्ये । अ॒न्ये᳚न्ये॒ वै । अ॒न्ये᳚न्य॒ इत्य॒न्ये - अ॒न्ये॒ । वा ए॒ते । ए॒ते प॒शवः॑ । प॒शव॒ आ । आ ल॑भ्यन्ते । ल॒भ्य॒न्त॒ उ॒त । उ॒तेव॑ । इ॒व॒ ग्रा॒म्याः । ग्रा॒म्या उ॒त । उ॒तेव॑ । इ॒वा॒र॒ण्याः । आ॒र॒ण्या यत् । यच्छक्व॑रयः । शक्व॑रयः पृ॒ष्ठम् । पृ॒ष्ठम् भव॑न्ति । भव॒न्त्यश्व॑स्य । अश्व॑स्य सर्व॒त्वाय॑ । स॒र्व॒त्वाय॑ पार्त्थुर॒श्मम् । स॒र्व॒त्वायेति॑ सर्व - त्वाय॑ । पा॒र्त्थु॒र॒श्मम् ब्र॑ह्मसा॒मम् । पा॒र्त्थु॒र॒श्ममिति॑ पार्त्थु - र॒श्मम् । ब्र॒ह्म॒सा॒मम् भ॑वति । ब्र॒ह्म॒सा॒ममिति॑ ब्रह्म - सा॒मम् । भ॒व॒ति॒ र॒श्मिना᳚ । र॒श्मिना॒ वै । वा अश्वः॑ । अश्वो॑ य॒तः \newline

\textbf{Jatai Paata} \newline

1. यस्मि॒न् नश्वो ऽश्वो॒ यस्मि॒न्॒. यस्मि॒न् नश्वः॑ । \newline
2. अश्व॑ आल॒भ्यत॑ आल॒भ्यते ऽश्वो ऽश्व॑ आल॒भ्यते᳚ । \newline
3. आ॒ल॒भ्यते॒ द्वाद॑श॒ द्वाद॑शा ल॒भ्यत॑ आल॒भ्यते॒ द्वाद॑श । \newline
4. आ॒ल॒भ्यत॒ इत्या᳚ - ल॒भ्यते᳚ । \newline
5. द्वाद॑श॒ मासा॒ मासा॒ द्वाद॑श॒ द्वाद॑श॒ मासाः᳚ । \newline
6. मासाः॒ पञ्च॒ पञ्च॒ मासा॒ मासाः॒ पञ्च॑ । \newline
7. पञ्च॒ र्‌तव॑ ऋ॒तवः॒ पञ्च॒ पञ्च॒ र्‌तवः॑ । \newline
8. ऋ॒तव॒ स्त्रय॒ स्त्रय॑ ऋ॒तव॑ ऋ॒तव॒ स्त्रयः॑ । \newline
9. त्रय॑ इ॒म इ॒मे त्रय॒ स्त्रय॑ इ॒मे । \newline
10. इ॒मे लो॒का लो॒का इ॒म इ॒मे लो॒काः । \newline
11. लो॒का अ॒सा व॒सौ लो॒का लो॒का अ॒सौ । \newline
12. अ॒सा वा॑दि॒त्य आ॑दि॒त्यो॑ ऽसा व॒सा वा॑दि॒त्यः । \newline
13. आ॒दि॒त्य ए॑कविꣳ॒॒श ए॑कविꣳ॒॒श आ॑दि॒त्य आ॑दि॒त्य ए॑कविꣳ॒॒शः । \newline
14. ए॒क॒विꣳ॒॒श ए॒ष ए॒ष ए॑कविꣳ॒॒श ए॑कविꣳ॒॒श ए॒षः । \newline
15. ए॒क॒विꣳ॒॒श इत्ये॑क - विꣳ॒॒शः । \newline
16. ए॒ष प्र॒जाप॑तिः प्र॒जाप॑ति रे॒ष ए॒ष प्र॒जाप॑तिः । \newline
17. प्र॒जाप॑तिः प्राजाप॒त्यः प्रा॑जाप॒त्यः प्र॒जाप॑तिः प्र॒जाप॑तिः प्राजाप॒त्यः । \newline
18. प्र॒जाप॑ति॒रिति॑ प्र॒जा - प॒तिः॒ । \newline
19. प्रा॒जा॒प॒त्यो ऽश्वो ऽश्वः॑ प्राजाप॒त्यः प्रा॑जाप॒त्यो ऽश्वः॑ । \newline
20. प्रा॒जा॒प॒त्य इति॑ प्राजा - प॒त्यः । \newline
21. अश्व॒ स्तम् त मश्वो ऽश्व॒ स्तम् । \newline
22. त मे॒वैव तम् त मे॒व । \newline
23. ए॒व सा॒क्षाथ् सा॒क्षा दे॒वैव सा॒क्षात् । \newline
24. सा॒क्षा दृ॑द्ध्नो त्यृद्ध्नोति सा॒क्षाथ् सा॒क्षा दृ॑द्ध्नोति । \newline
25. सा॒क्षादिति॑ स - अ॒क्षात् । \newline
26. ऋ॒द्ध्नो॒ति॒ शक्व॑रयः॒ शक्व॑रय ऋद्ध्नो त्यृद्ध्नोति॒ शक्व॑रयः । \newline
27. शक्व॑रयः पृ॒ष्ठम् पृ॒ष्ठꣳ शक्व॑रयः॒ शक्व॑रयः पृ॒ष्ठम् । \newline
28. पृ॒ष्ठम् भ॑वन्ति भवन्ति पृ॒ष्ठम् पृ॒ष्ठम् भ॑वन्ति । \newline
29. भ॒व॒न्त्य॒ न्यद॑न्य द॒न्यद॑न्यद् भवन्ति भवन्त्य॒ न्यद॑न्यत् । \newline
30. अ॒न्यद॑न्य॒च् छन्द॒ श्छन्दो॒ ऽन्यद॑न्य द॒न्यद॑न्य॒च् छन्दः॑ । \newline
31. अ॒न्यद॑न्य॒दित्य॒न्यत् - अ॒न्य॒त् । \newline
32. छन्दो॒ ऽन्ये᳚न्ये॒ ऽन्ये᳚न्ये॒ छन्द॒ श्छन्दो॒ ऽन्ये᳚न्ये । \newline
33. अ॒न्ये᳚न्ये॒ वै वा अ॒न्ये᳚न्ये॒ ऽन्ये᳚न्ये॒ वै । \newline
34. अ॒न्ये᳚न्य॒ इत्य॒न्ये - अ॒न्ये॒ । \newline
35. वा ए॒त ए॒ते वै वा ए॒ते । \newline
36. ए॒ते प॒शवः॑ प॒शव॑ ए॒त ए॒ते प॒शवः॑ । \newline
37. प॒शव॒ आ प॒शवः॑ प॒शव॒ आ । \newline
38. आ ल॑भ्यन्ते लभ्यन्त॒ आ ल॑भ्यन्ते । \newline
39. ल॒भ्य॒न्त॒ उ॒तोत ल॑भ्यन्ते लभ्यन्त उ॒त । \newline
40. उ॒तेवे॑ वो॒तो तेव॑ । \newline
41. इ॒व॒ ग्रा॒म्या ग्रा॒म्या इ॑वेव ग्रा॒म्याः । \newline
42. ग्रा॒म्या उ॒तोत ग्रा॒म्या ग्रा॒म्या उ॒त । \newline
43. उ॒तेवे॑ वो॒तो तेव॑ । \newline
44. इ॒वा॒र॒ण्या आ॑र॒ण्या इ॑वे वार॒ण्याः । \newline
45. आ॒र॒ण्या यद् यदा॑र॒ण्या आ॑र॒ण्या यत् । \newline
46. यच्छक्व॑रयः॒ शक्व॑रयो॒ यद् यच्छक्व॑रयः । \newline
47. शक्व॑रयः पृ॒ष्ठम् पृ॒ष्ठꣳ शक्व॑रयः॒ शक्व॑रयः पृ॒ष्ठम् । \newline
48. पृ॒ष्ठम् भव॑न्ति॒ भव॑न्ति पृ॒ष्ठम् पृ॒ष्ठम् भव॑न्ति । \newline
49. भव॒न् त्यश्व॒स्या श्व॑स्य॒ भव॑न्ति॒ भव॒न् त्यश्व॑स्य । \newline
50. अश्व॑स्य सर्व॒त्वाय॑ सर्व॒त्वाया श्व॒स्या श्व॑स्य सर्व॒त्वाय॑ । \newline
51. स॒र्व॒त्वाय॑ पार्थुर॒श्मम् पा᳚र्थुर॒श्मꣳ स॑र्व॒त्वाय॑ सर्व॒त्वाय॑ पार्थुर॒श्मम् । \newline
52. स॒र्व॒त्वायेति॑ सर्व - त्वाय॑ । \newline
53. पा॒र्थु॒र॒श्मम् ब्र॑ह्मसा॒मम् ब्र॑ह्मसा॒मम् पा᳚र्थुर॒श्मम् पा᳚र्थुर॒श्मम् ब्र॑ह्मसा॒मम् । \newline
54. पा॒र्थु॒र॒श्ममिति॑ पार्थु - र॒श्मम् । \newline
55. ब्र॒ह्म॒सा॒मम् भ॑वति भवति ब्रह्मसा॒मम् ब्र॑ह्मसा॒मम् भ॑वति । \newline
56. ब्र॒ह्म॒सा॒ममिति॑ ब्रह्म - सा॒मम् । \newline
57. भ॒व॒ति॒ र॒श्मिना॑ र॒श्मिना॑ भवति भवति र॒श्मिना᳚ । \newline
58. र॒श्मिना॒ वै वै र॒श्मिना॑ र॒श्मिना॒ वै । \newline
59. वा अश्वो ऽश्वो॒ वै वा अश्वः॑ । \newline
60. अश्वो॑ य॒तो य॒तो ऽश्वो ऽश्वो॑ य॒तः । \newline

\textbf{Ghana Paata } \newline

1. यस्मि॒न् नश्वो ऽश्वो॒ यस्मि॒न्॒. यस्मि॒न् नश्व॑ आल॒भ्यत॑ आल॒भ्यते ऽश्वो॒ यस्मि॒न्॒. यस्मि॒न् नश्व॑ आल॒भ्यते᳚ । \newline
2. अश्व॑ आल॒भ्यत॑ आल॒भ्यते ऽश्वो ऽश्व॑ आल॒भ्यते॒ द्वाद॑श॒ द्वाद॑शा ल॒भ्यते ऽश्वो ऽश्व॑ आल॒भ्यते॒ द्वाद॑श । \newline
3. आ॒ल॒भ्यते॒ द्वाद॑श॒ द्वाद॑शा ल॒भ्यत॑ आल॒भ्यते॒ द्वाद॑श॒ मासा॒ मासा॒ द्वाद॑शा ल॒भ्यत॑ आल॒भ्यते॒ द्वाद॑श॒ मासाः᳚ । \newline
4. आ॒ल॒भ्यत॒ इत्या᳚ - ल॒भ्यते᳚ । \newline
5. द्वाद॑श॒ मासा॒ मासा॒ द्वाद॑श॒ द्वाद॑श॒ मासाः॒ पञ्च॒ पञ्च॒ मासा॒ द्वाद॑श॒ द्वाद॑श॒ मासाः॒ पञ्च॑ । \newline
6. मासाः॒ पञ्च॒ पञ्च॒ मासा॒ मासाः॒ पञ्च॒ र्‌तव॑ ऋ॒तवः॒ पञ्च॒ मासा॒ मासाः॒ पञ्च॒ र्‌तवः॑ । \newline
7. पञ्च॒ र्‌तव॑ ऋ॒तवः॒ पञ्च॒ पञ्च॒ र्‌तव॒ स्त्रय॒ स्त्रय॑ ऋ॒तवः॒ पञ्च॒ पञ्च॒ र्‌तव॒ स्त्रयः॑ । \newline
8. ऋ॒तव॒ स्त्रय॒ स्त्रय॑ ऋ॒तव॑ ऋ॒तव॒ स्त्रय॑ इ॒म इ॒मे त्रय॑ ऋ॒तव॑ ऋ॒तव॒ स्त्रय॑ इ॒मे । \newline
9. त्रय॑ इ॒म इ॒मे त्रय॒ स्त्रय॑ इ॒मे लो॒का लो॒का इ॒मे त्रय॒ स्त्रय॑ इ॒मे लो॒काः । \newline
10. इ॒मे लो॒का लो॒का इ॒म इ॒मे लो॒का अ॒सा व॒सौ लो॒का इ॒म इ॒मे लो॒का अ॒सौ । \newline
11. लो॒का अ॒सा व॒सौ लो॒का लो॒का अ॒सा वा॑दि॒त्य आ॑दि॒त्यो॑ ऽसौ लो॒का लो॒का अ॒सा वा॑दि॒त्यः । \newline
12. अ॒सा वा॑दि॒त्य आ॑दि॒त्यो॑ ऽसा व॒सा वा॑दि॒त्य ए॑कविꣳ॒॒श ए॑कविꣳ॒॒श आ॑दि॒त्यो॑ ऽसा व॒सा वा॑दि॒त्य ए॑कविꣳ॒॒शः । \newline
13. आ॒दि॒त्य ए॑कविꣳ॒॒श ए॑कविꣳ॒॒श आ॑दि॒त्य आ॑दि॒त्य ए॑कविꣳ॒॒श ए॒ष ए॒ष ए॑कविꣳ॒॒श आ॑दि॒त्य आ॑दि॒त्य ए॑कविꣳ॒॒श ए॒षः । \newline
14. ए॒क॒विꣳ॒॒श ए॒ष ए॒ष ए॑कविꣳ॒॒श ए॑कविꣳ॒॒श ए॒ष प्र॒जाप॑तिः प्र॒जाप॑ति रे॒ष ए॑कविꣳ॒॒श ए॑कविꣳ॒॒श ए॒ष प्र॒जाप॑तिः । \newline
15. ए॒क॒विꣳ॒॒श इत्ये॑क - विꣳ॒॒शः । \newline
16. ए॒ष प्र॒जाप॑तिः प्र॒जाप॑ति रे॒ष ए॒ष प्र॒जाप॑तिः प्राजाप॒त्यः प्रा॑जाप॒त्यः प्र॒जाप॑ति रे॒ष ए॒ष प्र॒जाप॑तिः प्राजाप॒त्यः । \newline
17. प्र॒जाप॑तिः प्राजाप॒त्यः प्रा॑जाप॒त्यः प्र॒जाप॑तिः प्र॒जाप॑तिः प्राजाप॒त्यो ऽश्वो ऽश्वः॑ प्राजाप॒त्यः प्र॒जाप॑तिः प्र॒जाप॑तिः प्राजाप॒त्यो ऽश्वः॑ । \newline
18. प्र॒जाप॑ति॒रिति॑ प्र॒जा - प॒तिः॒ । \newline
19. प्रा॒जा॒प॒त्यो ऽश्वो ऽश्वः॑ प्राजाप॒त्यः प्रा॑जाप॒त्यो ऽश्व॒ स्तम् त मश्वः॑ प्राजाप॒त्यः प्रा॑जाप॒त्यो ऽश्व॒ स्तम् । \newline
20. प्रा॒जा॒प॒त्य इति॑ प्राजा - प॒त्यः । \newline
21. अश्व॒ स्तम् त मश्वो ऽश्व॒ स्त मे॒वैव त मश्वो ऽश्व॒ स्त मे॒व । \newline
22. त मे॒वैव तम् त मे॒व सा॒क्षाथ् सा॒क्षा दे॒व तम् त मे॒व सा॒क्षात् । \newline
23. ए॒व सा॒क्षाथ् सा॒क्षा दे॒वैव सा॒क्षा दृ॑द्ध्नो त्यृद्ध्नोति सा॒क्षा दे॒वैव सा॒क्षा दृ॑द्ध्नोति । \newline
24. सा॒क्षा दृ॑द्ध्नो त्यृद्ध्नोति सा॒क्षाथ् सा॒क्षा दृ॑द्ध्नोति॒ शक्व॑रयः॒ शक्व॑रय ऋद्ध्नोति सा॒क्षाथ् सा॒क्षा दृ॑द्ध्नोति॒ शक्व॑रयः । \newline
25. सा॒क्षादिति॑ स - अ॒क्षात् । \newline
26. ऋ॒द्ध्नो॒ति॒ शक्व॑रयः॒ शक्व॑रय ऋद्ध्नो त्यृद्ध्नोति॒ शक्व॑रयः पृ॒ष्ठम् पृ॒ष्ठꣳ शक्व॑रय ऋद्ध्नो त्यृद्ध्नोति॒ शक्व॑रयः पृ॒ष्ठम् । \newline
27. शक्व॑रयः पृ॒ष्ठम् पृ॒ष्ठꣳ शक्व॑रयः॒ शक्व॑रयः पृ॒ष्ठम् भ॑वन्ति भवन्ति पृ॒ष्ठꣳ शक्व॑रयः॒ शक्व॑रयः पृ॒ष्ठम् भ॑वन्ति । \newline
28. पृ॒ष्ठम् भ॑वन्ति भवन्ति पृ॒ष्ठम् पृ॒ष्ठम् भ॑वन् त्य॒न्यद॑न्य द॒न्यद॑न्यद् भवन्ति पृ॒ष्ठम् पृ॒ष्ठम् भ॑वन्त्य॒ न्यद॑न्यत् । \newline
29. भ॒व॒न् त्य॒न्यद॑न्य द॒न्यद॑न्यद् भवन्ति भवन्त्य॒ न्यद॑न्य॒च् छन्द॒ श्छन्दो॒ ऽन्यद॑न्यद् भवन्ति भवन् त्य॒न्यद॑न्य॒च् छन्दः॑ । \newline
30. अ॒न्यद॑न्य॒च् छन्द॒ श्छन्दो॒ ऽन्यद॑न्य द॒न्यद॑न्य॒च् छन्दो॒ ऽन्ये᳚न्ये॒ ऽन्ये᳚न्ये॒ छन्दो॒ ऽन्यद॑न्य द॒न्यद॑न्य॒च् छन्दो॒ ऽन्ये᳚न्ये । \newline
31. अ॒न्यद॑न्य॒दित्य॒न्यत् - अ॒न्य॒त् । \newline
32. छन्दो॒ ऽन्ये᳚न्ये॒ ऽन्ये᳚न्ये॒ छन्द॒ श्छन्दो॒ ऽन्ये᳚न्ये॒ वै वा अ॒न्ये᳚न्ये॒ छन्द॒ श्छन्दो॒ ऽन्ये᳚न्ये॒ वै । \newline
33. अ॒न्ये᳚न्ये॒ वै वा अ॒न्ये᳚न्ये॒ ऽन्ये᳚न्ये॒ वा ए॒त ए॒ते वा अ॒न्ये᳚न्ये॒ ऽन्ये᳚न्ये॒ वा ए॒ते । \newline
34. अ॒न्ये᳚न्य॒ इत्य॒न्ये - अ॒न्ये॒ । \newline
35. वा ए॒त ए॒ते वै वा ए॒ते प॒शवः॑ प॒शव॑ ए॒ते वै वा ए॒ते प॒शवः॑ । \newline
36. ए॒ते प॒शवः॑ प॒शव॑ ए॒त ए॒ते प॒शव॒ आ प॒शव॑ ए॒त ए॒ते प॒शव॒ आ । \newline
37. प॒शव॒ आ प॒शवः॑ प॒शव॒ आ ल॑भ्यन्ते लभ्यन्त॒ आ प॒शवः॑ प॒शव॒ आ ल॑भ्यन्ते । \newline
38. आ ल॑भ्यन्ते लभ्यन्त॒ आ ल॑भ्यन्त उ॒तोत ल॑भ्यन्त॒ आ ल॑भ्यन्त उ॒त । \newline
39. ल॒भ्य॒न्त॒ उ॒तोत ल॑भ्यन्ते लभ्यन्त उ॒तेवे॑ वो॒त ल॑भ्यन्ते लभ्यन्त उ॒तेव॑ । \newline
40. उ॒तेवे॑ वो॒तोतेव॑ ग्रा॒म्या ग्रा॒म्या इ॑वो॒तोतेव॑ ग्रा॒म्याः । \newline
41. इ॒व॒ ग्रा॒म्या ग्रा॒म्या इ॑वेव ग्रा॒म्या उ॒तोत ग्रा॒म्या इ॑वेव ग्रा॒म्या उ॒त । \newline
42. ग्रा॒म्या उ॒तोत ग्रा॒म्या ग्रा॒म्या उ॒तेवे॑ वो॒त ग्रा॒म्या ग्रा॒म्या उ॒तेव॑ । \newline
43. उ॒तेवे॑ वो॒तोते वा॑र॒ण्या आ॑र॒ण्या इ॑वो॒तोते वा॑र॒ण्याः । \newline
44. इ॒वा॒र॒ण्या आ॑र॒ण्या इ॑वे वार॒ण्या यद् यदा॑र॒ण्या इ॑वे वार॒ण्या यत् । \newline
45. आ॒र॒ण्या यद् यदा॑र॒ण्या आ॑र॒ण्या यच्छक्व॑रयः॒ शक्व॑रयो॒ यदा॑र॒ण्या आ॑र॒ण्या यच्छक्व॑रयः । \newline
46. यच्छक्व॑रयः॒ शक्व॑रयो॒ यद् यच्छक्व॑रयः पृ॒ष्ठम् पृ॒ष्ठꣳ शक्व॑रयो॒ यद् यच्छक्व॑रयः पृ॒ष्ठम् । \newline
47. शक्व॑रयः पृ॒ष्ठम् पृ॒ष्ठꣳ शक्व॑रयः॒ शक्व॑रयः पृ॒ष्ठम् भव॑न्ति॒ भव॑न्ति पृ॒ष्ठꣳ शक्व॑रयः॒ शक्व॑रयः पृ॒ष्ठम् भव॑न्ति । \newline
48. पृ॒ष्ठम् भव॑न्ति॒ भव॑न्ति पृ॒ष्ठम् पृ॒ष्ठम् भव॒न् त्यश्व॒स्या श्व॑स्य॒ भव॑न्ति पृ॒ष्ठम् पृ॒ष्ठम् भव॒न् त्यश्व॑स्य । \newline
49. भव॒न् त्यश्व॒स्या श्व॑स्य॒ भव॑न्ति॒ भव॒न् त्यश्व॑स्य सर्व॒त्वाय॑ सर्व॒त्वाया श्व॑स्य॒ भव॑न्ति॒ भव॒न्
त्यश्व॑स्य सर्व॒त्वाय॑ । \newline
50. अश्व॑स्य सर्व॒त्वाय॑ सर्व॒त्वाया श्व॒स्या श्व॑स्य सर्व॒त्वाय॑ पार्थुर॒श्मम् पा᳚र्थुर॒श्मꣳ स॑र्व॒त्वाया श्व॒स्या श्व॑स्य सर्व॒त्वाय॑ पार्थुर॒श्मम् । \newline
51. स॒र्व॒त्वाय॑ पार्थुर॒श्मम् पा᳚र्थुर॒श्मꣳ स॑र्व॒त्वाय॑ सर्व॒त्वाय॑ पार्थुर॒श्मम् ब्र॑ह्मसा॒मम् ब्र॑ह्मसा॒मम् पा᳚र्थुर॒श्मꣳ स॑र्व॒त्वाय॑ सर्व॒त्वाय॑ पार्थुर॒श्मम् ब्र॑ह्मसा॒मम् । \newline
52. स॒र्व॒त्वायेति॑ सर्व - त्वाय॑ । \newline
53. पा॒र्थु॒र॒श्मम् ब्र॑ह्मसा॒मम् ब्र॑ह्मसा॒मम् पा᳚र्थुर॒श्मम् पा᳚र्थुर॒श्मम् ब्र॑ह्मसा॒मम् भ॑वति भवति ब्रह्मसा॒मम् पा᳚र्थुर॒श्मम् पा᳚र्थुर॒श्मम् ब्र॑ह्मसा॒मम् भ॑वति । \newline
54. पा॒र्थु॒र॒श्ममिति॑ पार्थु - र॒श्मम् । \newline
55. ब्र॒ह्म॒सा॒मम् भ॑वति भवति ब्रह्मसा॒मम् ब्र॑ह्मसा॒मम् भ॑वति र॒श्मिना॑ र॒श्मिना॑ भवति ब्रह्मसा॒मम् ब्र॑ह्मसा॒मम् भ॑वति र॒श्मिना᳚ । \newline
56. ब्र॒ह्म॒सा॒ममिति॑ ब्रह्म - सा॒मम् । \newline
57. भ॒व॒ति॒ र॒श्मिना॑ र॒श्मिना॑ भवति भवति र॒श्मिना॒ वै वै र॒श्मिना॑ भवति भवति र॒श्मिना॒ वै । \newline
58. र॒श्मिना॒ वै वै र॒श्मिना॑ र॒श्मिना॒ वा अश्वो ऽश्वो॒ वै र॒श्मिना॑ र॒श्मिना॒ वा अश्वः॑ । \newline
59. वा अश्वो ऽश्वो॒ वै वा अश्वो॑ य॒तो य॒तो ऽश्वो॒ वै वा अश्वो॑ य॒तः । \newline
60. अश्वो॑ य॒तो य॒तो ऽश्वो ऽश्वो॑ य॒त ई᳚श्व॒र ई᳚श्व॒रो य॒तो ऽश्वो ऽश्वो॑ य॒त ई᳚श्व॒रः । \newline
\pagebreak
\markright{ TS 5.4.12.3  \hfill https://www.vedavms.in \hfill}

\section{ TS 5.4.12.3 }

\textbf{TS 5.4.12.3 } \newline
\textbf{Samhita Paata} \newline

य॒त ई᳚श्व॒रो वा अश्वोऽय॒तोऽप्र॑तिष्ठितः॒ परां᳚ परा॒वतं॒ गन्तो॒र्यत् पा᳚र्थुर॒श्मं ब्र॑ह्मसा॒मं भव॒त्यश्व॑स्य॒ यत्यै॒ धृत्यै॒ संकृ॑त्यच्छावाकसा॒मं भ॑वत्युथ्सन्नय॒ज्ञो वा ए॒ष यद॑श्वमे॒धः कस्तद्वे॒देत्या॑हु॒र्यदि॒ सर्वो॑ वा क्रि॒यते॒ न वा॒ सर्व॒ इति॒ यथ् संकृ॑त्यच्छावाकसा॒मं भव॒त्यश्व॑स्य सर्व॒त्वाय॒ पर्या᳚प्त्या॒ अन॑न्तरायाय॒ सर्व॑स्तोमोऽतिरा॒त्र उ॑त्त॒ममह॑र्भवति॒ ( ) सर्व॒स्याऽऽ*प्त्यै॒ सर्व॑स्य॒ जित्यै॒ सर्व॑मे॒व तेना᳚ऽऽ*प्नोति॒ सर्वं॑ जयति ॥ \newline

\textbf{Pada Paata} \newline

य॒तः । ई॒श्व॒रः । वै । अश्वः॑ । अय॑तः । अप्र॑तिष्ठित॒ इत्यप्र॑ति-स्थि॒तः॒ । परा᳚म् । प॒रा॒वत॒मिति॑ परा - वत᳚म् । गन्तोः᳚ । यत् । पा॒र्थु॒र॒श्ममिति॑ पार्थु - र॒श्मम् । ब्र॒ह्म॒सा॒ममिति॑ ब्रह्म - सा॒मम् । भव॑ति । अश्व॑स्य । यत्यै᳚ । धृत्यै᳚ । संकृ॒तीति॒ सं - कृ॒ति॒ । अ॒च्छा॒वा॒क॒सा॒ममित्य॑च्छावाक - सा॒मम् । भ॒व॒ति॒ । उ॒थ्स॒न्न॒य॒ज्ञ् इत्यु॑थ्सन्न - य॒ज्ञ्ः । वै । ए॒षः । यत् । अ॒श्व॒मे॒ध इत्य॑श्व - मे॒धः । कः । तत् । वे॒द॒ । इति॑ । आ॒हुः॒ । यदि॑ । सर्वः॑ । वा॒ । क्रि॒यते᳚ । न । वा॒ । सर्वः॑ । इति॑ । यत् । संकृ॒तीति॒ सं - कृ॒ति॒ । अ॒च्छा॒वा॒क॒सा॒ममित्य॑च्छावाक - स॒मम् । भव॑ति । अश्व॑स्य । स॒र्व॒त्वायेति॑ सर्व - त्वाय॑ । पर्या᳚प्त्या॒ इति॒ परि॑ - आ॒प्त्यै॒ । अन॑न्तराया॒येत्यन॑न्तः - आ॒या॒य॒ । सर्व॑स्तोम॒ इति॒ सर्व॑ - स्तो॒मः॒ । अ॒ति॒रा॒त्र इत्य॑ति - रा॒त्रः । उ॒त्त॒ममित्यु॑त् - त॒मम् । अहः॑ । भ॒व॒ति॒ ( ) । सर्व॑स्य । आप्त्यै᳚ । सर्व॑स्य । जित्यै᳚ । सर्व᳚म् । ए॒व । तेन॑ । आ॒प्नो॒ति॒ । सर्व᳚म् । ज॒य॒ति॒ ॥  \newline


\textbf{Krama Paata} \newline

य॒त ई᳚श्व॒रः । ई॒श्व॒रो वै । वा अश्वः॑ । अश्वोऽय॑तः । अय॒तोऽप्र॑तिष्ठितः । अप्र॑तिष्ठितः॒ परा᳚म् । अप्र॑तिष्ठित॒ इत्यप्र॑ति - स्थि॒तः॒ । परा᳚म् परा॒वत᳚म् । प॒रा॒वत॒म् गन्तोः᳚ । प॒रा॒वत॒मिति॑ परा - वत᳚म् । गन्तो॒र् यत् । यत् पा᳚र्त्थुर॒श्मम् । पा॒र्त्थु॒र॒श्मम् ब्र॑ह्मसा॒मम् । पा॒र्त्थु॒र॒श्ममिति॑ पार्त्थु - र॒श्मम् । ब्र॒ह्म॒सा॒मम् भव॑ति । ब्र॒ह्म॒सा॒ममिति॑ ब्रह्म - सा॒मम् । भव॒त्यश्व॑स्य । अश्व॑स्य॒ यत्यै᳚ । यत्यै॒ धृत्यै᳚ । धृत्यै॒ सङ्कृ॑ति । सङ्कृ॑त्यच्छावाकसा॒मम् । सङ्कृ॒तीति॒ सम् - कृ॒ति॒ । अ॒च्छा॒वा॒क॒सा॒मम् भ॑वति । अ॒च्छा॒वा॒क॒सा॒ममित्य॑च्छावाक - सा॒मम् । भ॒व॒त्यु॒थ्स॒न्न॒य॒ज्ञ्ः । उ॒थ्स॒न्न॒य॒ज्ञो वै । उ॒थ्स॒न्न॒य॒ज्ञ् इत्यु॑थ्सन्न - य॒ज्ञ्ः । वा ए॒षः । ए॒ष यत् । यद॑श्वमे॒धः । अ॒श्व॒मे॒धः कः । अ॒श्म॒मे॒ध इत्य॑श्व - मे॒धः । कस्तत् । तद् वे॑द । वे॒देति॑ । इत्या॑हुः । आ॒हु॒र् यदि॑ । यदि॒ सर्वः॑ । सर्वो॑ वा । वा॒ क्रि॒यते᳚ । क्रि॒यते॒ न । न वा᳚ । वा॒ सर्वः॑ । सर्व॒ इति॑ । इति॒ यत् । यथ् सङ्कृ॑ति । सङ्कृ॑त्यच्छावाकसा॒मम् । सङ्कृ॒तीति॒ सम् - कृ॒ति॒ । अ॒च्छा॒वा॒क॒सा॒मम् भव॑ति । अ॒च्छा॒वा॒क॒सा॒ममित्य॑च्छावाक - सा॒मम् । भव॒त्यश्व॑स्य । अश्व॑स्य सर्व॒त्वाय॑ । स॒र्व॒त्वाय॒ पर्या᳚प्त्यै । स॒र्व॒त्वायेति॑ सर्व - त्वाय॑ । पर्या᳚प्त्या॒ अन॑न्तरायाय । पर्या᳚प्त्या॒ इति॒ परि॑ - आ॒प्त्यै॒ । अन॑न्तरायाय॒ सर्व॑स्तोमः । अन॑न्तराया॒येत्यन॑न्तः - आ॒या॒य॒ । सर्व॑स्तोमोऽतिरा॒त्रः । सर्व॑स्तोम॒ इति॒ सर्व॑ - स्तो॒मः॒ । अ॒ति॒रा॒त्र उ॑त्त॒मम् । अ॒ति॒रा॒त्र इत्य॑ति - रा॒त्रः । उ॒त्त॒ममहः॑ । उ॒त्त॒ममित्यु॑त् - त॒मम् । अह॑र् भवति ( ) । भ॒व॒ति॒ सर्व॑स्य । सर्व॒स्याप्त्यै᳚ । आप्त्यै॒ सर्व॑स्य । सर्व॑स्य॒ जित्यै᳚ । जित्यै॒ सर्व᳚म् । सर्व॑मे॒व । ए॒व तेन॑ । तेना᳚ऽऽप्नोति । आ॒प्नो॒ति॒ सर्व᳚म् । सर्व॑म् जयति । ज॒य॒तीति॑ जयति । \newline

\textbf{Jatai Paata} \newline

1. य॒त ई᳚श्व॒र ई᳚श्व॒रो य॒तो य॒त ई᳚श्व॒रः । \newline
2. ई॒श्व॒रो वै वा ई᳚श्व॒र ई᳚श्व॒रो वै । \newline
3. वा अश्वो ऽश्वो॒ वै वा अश्वः॑ । \newline
4. अश्वो ऽय॒तो ऽय॒तो ऽश्वो ऽश्वो ऽय॑तः । \newline
5. अय॒तो ऽप्र॑तिष्ठि॒तो ऽप्र॑तिष्ठि॒तो ऽय॒तो ऽय॒तो ऽप्र॑तिष्ठितः । \newline
6. अप्र॑तिष्ठितः॒ परा॒म् परा॒ मप्र॑तिष्ठि॒तो ऽप्र॑तिष्ठितः॒ परा᳚म् । \newline
7. अप्र॑तिष्ठित॒ इत्यप्र॑ति - स्थि॒तः॒ । \newline
8. परा᳚म् परा॒वत॑म् परा॒वत॒म् परा॒म् परा᳚म् परा॒वत᳚म् । \newline
9. प॒रा॒वत॒म् गन्तो॒र् गन्तोः᳚ परा॒वत॑म् परा॒वत॒म् गन्तोः᳚ । \newline
10. प॒रा॒वत॒मिति॑ परा - वत᳚म् । \newline
11. गन्तो॒र् यद् यद् गन्तो॒र् गन्तो॒र् यत् । \newline
12. यत् पा᳚र्थुर॒श्मम् पा᳚र्थुर॒श्मं ॅयद् यत् पा᳚र्थुर॒श्मम् । \newline
13. पा॒र्थु॒र॒श्मम् ब्र॑ह्मसा॒मम् ब्र॑ह्मसा॒मम् पा᳚र्थुर॒श्मम् पा᳚र्थुर॒श्मम् ब्र॑ह्मसा॒मम् । \newline
14. पा॒र्थु॒र॒श्ममिति॑ पार्थु - र॒श्मम् । \newline
15. ब्र॒ह्म॒सा॒मम् भव॑ति॒ भव॑ति ब्रह्मसा॒मम् ब्र॑ह्मसा॒मम् भव॑ति । \newline
16. ब्र॒ह्म॒सा॒ममिति॑ ब्रह्म - सा॒मम् । \newline
17. भव॒ त्यश्व॒स्या श्व॑स्य॒ भव॑ति॒ भव॒ त्यश्व॑स्य । \newline
18. अश्व॑स्य॒ यत्यै॒ यत्या॒ अश्व॒स्या श्व॑स्य॒ यत्यै᳚ । \newline
19. यत्यै॒ धृत्यै॒ धृत्यै॒ यत्यै॒ यत्यै॒ धृत्यै᳚ । \newline
20. धृत्यै॒ सङ्कृ॑ति॒ सङ्कृ॑ति॒ धृत्यै॒ धृत्यै॒ सङ्कृ॑ति । \newline
21. सङ्कृ॑ त्यच्छावाकसा॒म म॑च्छावाकसा॒मꣳ सङ्कृ॑ति॒ सङ्कृ॑ त्यच्छावाकसा॒मम् । \newline
22. सङ्कृ॒तीति॒ सं - कृ॒ति॒ । \newline
23. अ॒च्छा॒वा॒क॒सा॒मम् भ॑वति भव त्यच्छावाकसा॒म म॑च्छावाकसा॒मम् भ॑वति । \newline
24. अ॒च्छा॒वा॒क॒सा॒ममित्य॑च्छावाक - सा॒मम् । \newline
25. भ॒व॒ त्यु॒थ्स॒न्न॒य॒ज्ञ् उ॑थ्सन्नय॒ज्ञो भ॑वति भव त्युथ्सन्नय॒ज्ञ्ः । \newline
26. उ॒थ्स॒न्न॒य॒ज्ञो वै वा उ॑थ्सन्नय॒ज्ञ् उ॑थ्सन्नय॒ज्ञो वै । \newline
27. उ॒थ्स॒न्न॒य॒ज्ञ् इत्यु॑थ्सन्न - य॒ज्ञ्ः । \newline
28. वा ए॒ष ए॒ष वै वा ए॒षः । \newline
29. ए॒ष यद् यदे॒ष ए॒ष यत् । \newline
30. यद॑श्वमे॒धो᳚ ऽश्वमे॒धो यद् यद॑श्वमे॒धः । \newline
31. अ॒श्व॒मे॒धः कः को᳚ ऽश्वमे॒धो᳚ ऽश्वमे॒धः कः । \newline
32. अ॒श्व॒मे॒ध इत्य॑श्व - मे॒धः । \newline
33. क स्तत् तत् कः क स्तत् । \newline
34. तद् वे॑द वेद॒ तत् तद् वे॑द । \newline
35. वे॒दे तीति॑ वेद वे॒देति॑ । \newline
36. इत्या॑हु राहु॒ रिती त्या॑हुः । \newline
37. आ॒हु॒र् यदि॒ यद्या॑हु राहु॒र् यदि॑ । \newline
38. यदि॒ सर्वः॒ सर्वो॒ यदि॒ यदि॒ सर्वः॑ । \newline
39. सर्वो॑ वा वा॒ सर्वः॒ सर्वो॑ वा । \newline
40. वा॒ क्रि॒यते᳚ क्रि॒यते॑ वा वा क्रि॒यते᳚ । \newline
41. क्रि॒यते॒ न न क्रि॒यते᳚ क्रि॒यते॒ न । \newline
42. न वा॑ वा॒ न न वा᳚ । \newline
43. वा॒ सर्वः॒ सर्वो॑ वा वा॒ सर्वः॑ । \newline
44. सर्व॒ इतीति॒ सर्वः॒ सर्व॒ इति॑ । \newline
45. इति॒ यद् यदितीति॒ यत् । \newline
46. यथ् सङ्कृ॑ति॒ सङ्कृ॑ति॒ यद् यथ् सङ्कृ॑ति । \newline
47. सङ्कृ॑ त्यच्छावाकसा॒म म॑च्छावाकसा॒मꣳ सङ्कृ॑ति॒ सङ्कृ॑ त्यच्छावाकसा॒मम् । \newline
48. सङ्कृ॒तीति॒ सं - कृ॒ति॒ । \newline
49. अ॒च्छा॒वा॒क॒सा॒मम् भव॑ति॒ भव॑ त्यच्छावाकसा॒म म॑च्छावाकसा॒मम् भव॑ति । \newline
50. अ॒च्छा॒वा॒क॒सा॒ममित्य॑च्छावाक - सा॒मम् । \newline
51. भव॒ त्यश्व॒स्या श्व॑स्य॒ भव॑ति॒ भव॒ त्यश्व॑स्य । \newline
52. अश्व॑स्य सर्व॒त्वाय॑ सर्व॒त्वाया श्व॒स्या श्व॑स्य सर्व॒त्वाय॑ । \newline
53. स॒र्व॒त्वाय॒ पर्या᳚प्त्यै॒ पर्या᳚प्त्यै सर्व॒त्वाय॑ सर्व॒त्वाय॒ पर्या᳚प्त्यै । \newline
54. स॒र्व॒त्वायेति॑ सर्व - त्वाय॑ । \newline
55. पर्या᳚प्त्या॒ अन॑न्तराया॒या न॑न्तरायाय॒ पर्या᳚प्त्यै॒ पर्या᳚प्त्या॒ अन॑न्तरायाय । \newline
56. पर्या᳚प्त्या॒ इति॒ परि॑ - आ॒प्त्यै॒ । \newline
57. अन॑न्तरायाय॒ सर्व॑स्तोमः॒ सर्व॑स्तो॒मो ऽन॑न्तराया॒या न॑न्तरायाय॒ सर्व॑स्तोमः । \newline
58. अन॑न्तराया॒येत्यन॑न्तः - आ॒या॒य॒ । \newline
59. सर्व॑स्तोमो ऽतिरा॒त्रो॑ ऽतिरा॒त्रः सर्व॑स्तोमः॒ सर्व॑स्तोमो ऽतिरा॒त्रः । \newline
60. सर्व॑स्तोम॒ इति॒ सर्व॑ - स्तो॒मः॒ । \newline
61. अ॒ति॒रा॒त्र उ॑त्त॒म मु॑त्त॒म म॑तिरा॒त्रो॑ ऽतिरा॒त्र उ॑त्त॒मम् । \newline
62. अ॒ति॒रा॒त्र इत्य॑ति - रा॒त्रः । \newline
63. उ॒त्त॒म मह॒ रह॑ रुत्त॒म मु॑त्त॒म महः॑ । \newline
64. उ॒त्त॒ममित्यु॑त् - त॒मम् । \newline
65. अह॑र् भवति भव॒ त्यह॒ रह॑र् भवति । \newline
66. भ॒व॒ति॒ सर्व॑स्य॒ सर्व॑स्य भवति भवति॒ सर्व॑स्य । \newline
67. सर्व॒स्याप्त्या॒ आप्त्यै॒ सर्व॑स्य॒ सर्व॒ स्याप्त्यै᳚ । \newline
68. आप्त्यै॒ सर्व॑स्य॒ सर्व॒ स्याप्त्या॒ आप्त्यै॒ सर्व॑स्य । \newline
69. सर्व॑स्य॒ जित्यै॒ जित्यै॒ सर्व॑स्य॒ सर्व॑स्य॒ जित्यै᳚ । \newline
70. जित्यै॒ सर्वꣳ॒॒ सर्व॒म् जित्यै॒ जित्यै॒ सर्व᳚म् । \newline
71. सर्व॑ मे॒वैव सर्वꣳ॒॒ सर्व॑ मे॒व । \newline
72. ए॒व तेन॒ तेनै॒ वैव तेन॑ । \newline
73. तेना᳚प्नो त्याप्नोति॒ तेन॒ तेना᳚प्नोति । \newline
74. आ॒प्नो॒ति॒ सर्वꣳ॒॒ सर्व॑ माप्नो त्याप्नोति॒ सर्व᳚म् । \newline
75. सर्व॑म् जयति जयति॒ सर्वꣳ॒॒ सर्व॑म् जयति । \newline
76. ज॒य॒तीति॑ जयति । \newline

\textbf{Ghana Paata } \newline

1. य॒त ई᳚श्व॒र ई᳚श्व॒रो य॒तो य॒त ई᳚श्व॒रो वै वा ई᳚श्व॒रो य॒तो य॒त ई᳚श्व॒रो वै । \newline
2. ई॒श्व॒रो वै वा ई᳚श्व॒र ई᳚श्व॒रो वा अश्वो ऽश्वो॒ वा ई᳚श्व॒र ई᳚श्व॒रो वा अश्वः॑ । \newline
3. वा अश्वो ऽश्वो॒ वै वा अश्वो ऽय॒तो ऽय॒तो ऽश्वो॒ वै वा अश्वो ऽय॑तः । \newline
4. अश्वो ऽय॒तो ऽय॒तो ऽश्वो ऽश्वो ऽय॒तो ऽप्र॑तिष्ठि॒तो ऽप्र॑तिष्ठि॒तो ऽय॒तो ऽश्वो ऽश्वो ऽय॒तो ऽप्र॑तिष्ठितः । \newline
5. अय॒तो ऽप्र॑तिष्ठि॒तो ऽप्र॑तिष्ठि॒तो ऽय॒तो ऽय॒तो ऽप्र॑तिष्ठितः॒ परा॒म् परा॒ मप्र॑तिष्ठि॒तो ऽय॒तो ऽय॒तो ऽप्र॑तिष्ठितः॒ परा᳚म् । \newline
6. अप्र॑तिष्ठितः॒ परा॒म् परा॒ मप्र॑तिष्ठि॒तो ऽप्र॑तिष्ठितः॒ परा᳚म् परा॒वत॑म् परा॒वत॒म् परा॒ मप्र॑तिष्ठि॒तो ऽप्र॑तिष्ठितः॒ परा᳚म् परा॒वत᳚म् । \newline
7. अप्र॑तिष्ठित॒ इत्यप्र॑ति - स्थि॒तः॒ । \newline
8. परा᳚म् परा॒वत॑म् परा॒वत॒म् परा॒म् परा᳚म् परा॒वत॒म् गन्तो॒र् गन्तोः᳚ परा॒वत॒म् परा॒म् परा᳚म् परा॒वत॒म् गन्तोः᳚ । \newline
9. प॒रा॒वत॒म् गन्तो॒र् गन्तोः᳚ परा॒वत॑म् परा॒वत॒म् गन्तो॒र् यद् यद् गन्तोः᳚ परा॒वत॑म् परा॒वत॒म् गन्तो॒र् यत् । \newline
10. प॒रा॒वत॒मिति॑ परा - वत᳚म् । \newline
11. गन्तो॒र् यद् यद् गन्तो॒र् गन्तो॒र् यत् पा᳚र्थुर॒श्मम् पा᳚र्थुर॒श्मं ॅयद् गन्तो॒र् गन्तो॒र् यत् पा᳚र्थुर॒श्मम् । \newline
12. यत् पा᳚र्थुर॒श्मम् पा᳚र्थुर॒श्मं ॅयद् यत् पा᳚र्थुर॒श्मम् ब्र॑ह्मसा॒मम् ब्र॑ह्मसा॒मम् पा᳚र्थुर॒श्मं ॅयद् यत् पा᳚र्थुर॒श्मम् ब्र॑ह्मसा॒मम् । \newline
13. पा॒र्थु॒र॒श्मम् ब्र॑ह्मसा॒मम् ब्र॑ह्मसा॒मम् पा᳚र्थुर॒श्मम् पा᳚र्थुर॒श्मम् ब्र॑ह्मसा॒मम् भव॑ति॒ भव॑ति ब्रह्मसा॒मम् पा᳚र्थुर॒श्मम् पा᳚र्थुर॒श्मम् ब्र॑ह्मसा॒मम् भव॑ति । \newline
14. पा॒र्थु॒र॒श्ममिति॑ पार्थु - र॒श्मम् । \newline
15. ब्र॒ह्म॒सा॒मम् भव॑ति॒ भव॑ति ब्रह्मसा॒मम् ब्र॑ह्मसा॒मम् भव॒ त्यश्व॒स्या श्व॑स्य॒ भव॑ति ब्रह्मसा॒मम् ब्र॑ह्मसा॒मम् भव॒ त्यश्व॑स्य । \newline
16. ब्र॒ह्म॒सा॒ममिति॑ ब्रह्म - सा॒मम् । \newline
17. भव॒ त्यश्व॒स्या श्व॑स्य॒ भव॑ति॒ भव॒ त्यश्व॑स्य॒ यत्यै॒ यत्या॒ अश्व॑स्य॒ भव॑ति॒ भव॒ त्यश्व॑स्य॒ यत्यै᳚ । \newline
18. अश्व॑स्य॒ यत्यै॒ यत्या॒ अश्व॒स्या श्व॑स्य॒ यत्यै॒ धृत्यै॒ धृत्यै॒ यत्या॒ अश्व॒स्या श्व॑स्य॒ यत्यै॒ धृत्यै᳚ । \newline
19. यत्यै॒ धृत्यै॒ धृत्यै॒ यत्यै॒ यत्यै॒ धृत्यै॒ सङ्कृ॑ति॒ सङ्कृ॑ति॒ धृत्यै॒ यत्यै॒ यत्यै॒ धृत्यै॒ सङ्कृ॑ति । \newline
20. धृत्यै॒ सङ्कृ॑ति॒ सङ्कृ॑ति॒ धृत्यै॒ धृत्यै॒ सङ्कृ॑ त्यच्छावाकसा॒म म॑च्छावाकसा॒मꣳ सङ्कृ॑ति॒ धृत्यै॒ धृत्यै॒ सङ्कृ॑ त्यच्छावाकसा॒मम् । \newline
21. सङ्कृ॑ त्यच्छावाकसा॒म म॑च्छावाकसा॒मꣳ सङ्कृ॑ति॒ सङ्कृ॑ त्यच्छावाकसा॒मम् भ॑वति भव
त्यच्छावाकसा॒मꣳ सङ्कृ॑ति॒ सङ्कृ॑ त्यच्छावाकसा॒मम् भ॑वति । \newline
22. सङ्कृ॒तीति॒ सं - कृ॒ति॒ । \newline
23. अ॒च्छा॒वा॒क॒सा॒मम् भ॑वति भव त्यच्छावाकसा॒म म॑च्छावाकसा॒मम् भ॑व त्युथ्सन्नय॒ज्ञ् उ॑थ्सन्नय॒ज्ञो भ॑व त्यच्छावाकसा॒म म॑च्छावाकसा॒मम् भ॑व त्युथ्सन्नय॒ज्ञ्ः । \newline
24. अ॒च्छा॒वा॒क॒सा॒ममित्य॑च्छावाक - सा॒मम् । \newline
25. भ॒व॒त्यु॒ थ्स॒न्न॒य॒ज्ञ् उ॑थ्सन्नय॒ज्ञो भ॑वति भव त्युथ्सन्नय॒ज्ञो वै वा उ॑थ्सन्नय॒ज्ञो भ॑वति भव
त्युथ्सन्नय॒ज्ञो वै । \newline
26. उ॒थ्स॒न्न॒य॒ज्ञो वै वा उ॑थ्सन्नय॒ज्ञ् उ॑थ्सन्नय॒ज्ञो वा ए॒ष ए॒ष वा उ॑थ्सन्नय॒ज्ञ् उ॑थ्सन्नय॒ज्ञो वा ए॒षः । \newline
27. उ॒थ्स॒न्न॒य॒ज्ञ् इत्यु॑थ्सन्न - य॒ज्ञ्ः । \newline
28. वा ए॒ष ए॒ष वै वा ए॒ष यद् यदे॒ष वै वा ए॒ष यत् । \newline
29. ए॒ष यद् यदे॒ष ए॒ष यद॑श्वमे॒धो᳚ ऽश्वमे॒धो यदे॒ष ए॒ष यद॑श्वमे॒धः । \newline
30. यद॑श्वमे॒धो᳚ ऽश्वमे॒धो यद् यद॑श्वमे॒धः कः को᳚ ऽश्वमे॒धो यद् यद॑श्वमे॒धः कः । \newline
31. अ॒श्व॒मे॒धः कः को᳚ ऽश्वमे॒धो᳚ ऽश्वमे॒धः क स्तत् तत् को᳚ ऽश्वमे॒धो᳚ ऽश्वमे॒धः क स्तत् । \newline
32. अ॒श्व॒मे॒ध इत्य॑श्व - मे॒धः । \newline
33. क स्तत् तत् कः क स्तद् वे॑द वेद॒ तत् कः क स्तद् वे॑द । \newline
34. तद् वे॑द वेद॒ तत् तद् वे॒दे तीति॑ वेद॒ तत् तद् वे॒देति॑ । \newline
35. वे॒दे तीति॑ वेद वे॒दे त्या॑हु राहु॒ रिति॑ वेद वे॒दे त्या॑हुः । \newline
36. इत्या॑हु राहु॒ रिती त्या॑हु॒र् यदि॒ यद्या॑हु॒ रिती त्या॑हु॒र् यदि॑ । \newline
37. आ॒हु॒र् यदि॒ यद्या॑हु राहु॒र् यदि॒ सर्वः॒ सर्वो॒ यद्या॑हु राहु॒र् यदि॒ सर्वः॑ । \newline
38. यदि॒ सर्वः॒ सर्वो॒ यदि॒ यदि॒ सर्वो॑ वा वा॒ सर्वो॒ यदि॒ यदि॒ सर्वो॑ वा । \newline
39. सर्वो॑ वा वा॒ सर्वः॒ सर्वो॑ वा क्रि॒यते᳚ क्रि॒यते॑ वा॒ सर्वः॒ सर्वो॑ वा क्रि॒यते᳚ । \newline
40. वा॒ क्रि॒यते᳚ क्रि॒यते॑ वा वा क्रि॒यते॒ न न क्रि॒यते॑ वा वा क्रि॒यते॒ न । \newline
41. क्रि॒यते॒ न न क्रि॒यते᳚ क्रि॒यते॒ न वा॑ वा॒ न क्रि॒यते᳚ क्रि॒यते॒ न वा᳚ । \newline
42. न वा॑ वा॒ न न वा॒ सर्वः॒ सर्वो॑ वा॒ न न वा॒ सर्वः॑ । \newline
43. वा॒ सर्वः॒ सर्वो॑ वा वा॒ सर्व॒ इतीति॒ सर्वो॑ वा वा॒ सर्व॒ इति॑ । \newline
44. सर्व॒ इतीति॒ सर्वः॒ सर्व॒ इति॒ यद् यदिति॒ सर्वः॒ सर्व॒ इति॒ यत् । \newline
45. इति॒ यद् यदितीति॒ यथ् सङ्कृ॑ति॒ सङ्कृ॑ति॒ यदितीति॒ यथ् सङ्कृ॑ति । \newline
46. यथ् सङ्कृ॑ति॒ सङ्कृ॑ति॒ यद् यथ् सङ्कृ॑ त्यच्छावाकसा॒म म॑च्छावाकसा॒मꣳ सङ्कृ॑ति॒ यद् यथ् सङ्कृ॑ त्यच्छावाकसा॒मम् । \newline
47. सङ्कृ॑ त्यच्छावाकसा॒म म॑च्छावाकसा॒मꣳ सङ्कृ॑ति॒ सङ्कृ॑ त्यच्छावाकसा॒मम् भव॑ति॒ भव॑
त्यच्छावाकसा॒मꣳ सङ्कृ॑ति॒ सङ्कृ॑ त्यच्छावाकसा॒मम् भव॑ति । \newline
48. सङ्कृ॒तीति॒ सं - कृ॒ति॒ । \newline
49. अ॒च्छा॒वा॒क॒सा॒मम् भव॑ति॒ भव॑ त्यच्छावाकसा॒म म॑च्छावाकसा॒मम् भव॒ त्यश्व॒स्या श्व॑स्य॒ भव॑
त्यच्छावाकसा॒म म॑च्छावाकसा॒मम् भव॒ त्यश्व॑स्य । \newline
50. अ॒च्छा॒वा॒क॒सा॒ममित्य॑च्छावाक - सा॒मम् । \newline
51. भव॒त्य श्व॒स्या श्व॑स्य॒ भव॑ति॒ भव॒ त्यश्व॑स्य सर्व॒त्वाय॑ सर्व॒त्वाया श्व॑स्य॒ भव॑ति॒ भव॒
त्यश्व॑स्य सर्व॒त्वाय॑ । \newline
52. अश्व॑स्य सर्व॒त्वाय॑ सर्व॒त्वाया श्व॒स्या श्व॑स्य सर्व॒त्वाय॒ पर्या᳚प्त्यै॒ पर्या᳚प्त्यै सर्व॒त्वाया श्व॒स्या श्व॑स्य सर्व॒त्वाय॒ पर्या᳚प्त्यै । \newline
53. स॒र्व॒त्वाय॒ पर्या᳚प्त्यै॒ पर्या᳚प्त्यै सर्व॒त्वाय॑ सर्व॒त्वाय॒ पर्या᳚प्त्या॒ अन॑न्तराया॒या न॑न्तरायाय॒ पर्या᳚प्त्यै सर्व॒त्वाय॑ सर्व॒त्वाय॒ पर्या᳚प्त्या॒ अन॑न्तरायाय । \newline
54. स॒र्व॒त्वायेति॑ सर्व - त्वाय॑ । \newline
55. पर्या᳚प्त्या॒ अन॑न्तराया॒या न॑न्तरायाय॒ पर्या᳚प्त्यै॒ पर्या᳚प्त्या॒ अन॑न्तरायाय॒ सर्व॑स्तोमः॒ सर्व॑स्तो॒मो ऽन॑न्तरायाय॒ पर्या᳚प्त्यै॒ पर्या᳚प्त्या॒ अन॑न्तरायाय॒ सर्व॑स्तोमः । \newline
56. पर्या᳚प्त्या॒ इति॒ परि॑ - आ॒प्त्यै॒ । \newline
57. अन॑न्तरायाय॒ सर्व॑स्तोमः॒ सर्व॑स्तो॒मो ऽन॑न्तराया॒या न॑न्तरायाय॒ सर्व॑स्तोमो ऽतिरा॒त्रो॑ ऽतिरा॒त्रः सर्व॑स्तो॒मो ऽन॑न्तराया॒या न॑न्तरायाय॒ सर्व॑स्तोमो ऽतिरा॒त्रः । \newline
58. अन॑न्तराया॒येत्यन॑न्तः - आ॒या॒य॒ । \newline
59. सर्व॑स्तोमो ऽतिरा॒त्रो॑ ऽतिरा॒त्रः सर्व॑स्तोमः॒ सर्व॑स्तोमो ऽतिरा॒त्र उ॑त्त॒म मु॑त्त॒म म॑तिरा॒त्रः सर्व॑स्तोमः॒ सर्व॑स्तोमो ऽतिरा॒त्र उ॑त्त॒मम् । \newline
60. सर्व॑स्तोम॒ इति॒ सर्व॑ - स्तो॒मः॒ । \newline
61. अ॒ति॒रा॒त्र उ॑त्त॒म मु॑त्त॒म म॑तिरा॒त्रो॑ ऽतिरा॒त्र उ॑त्त॒म मह॒ रह॑ रुत्त॒म म॑तिरा॒त्रो॑ ऽतिरा॒त्र उ॑त्त॒म महः॑ । \newline
62. अ॒ति॒रा॒त्र इत्य॑ति - रा॒त्रः । \newline
63. उ॒त्त॒म मह॒ रह॑ रुत्त॒म मु॑त्त॒म मह॑र् भवति भव॒ त्यह॑ रुत्त॒म मु॑त्त॒म मह॑र् भवति । \newline
64. उ॒त्त॒ममित्यु॑त् - त॒मम् । \newline
65. अह॑र् भवति भव॒ त्यह॒ रह॑र् भवति॒ सर्व॑स्य॒ सर्व॑स्य भव॒ त्यह॒ रह॑र् भवति॒ सर्व॑स्य । \newline
66. भ॒व॒ति॒ सर्व॑स्य॒ सर्व॑स्य भवति भवति॒ सर्व॒स्या प्त्या॒ आप्त्यै॒ सर्व॑स्य भवति भवति॒ सर्व॒स्याप्त्यै᳚ । \newline
67. सर्व॒स्या प्त्या॒ आप्त्यै॒ सर्व॑स्य॒ सर्व॒स्याप्त्यै॒ सर्व॑स्य॒ सर्व॒स्याप्त्यै॒ सर्व॑स्य॒ सर्व॒स्याप्त्यै॒ सर्व॑स्य । \newline
68. आप्त्यै॒ सर्व॑स्य॒ सर्व॒स्या प्त्या॒ आप्त्यै॒ सर्व॑स्य॒ जित्यै॒ जित्यै॒ सर्व॒स्या प्त्या॒ आप्त्यै॒ सर्व॑स्य॒ जित्यै᳚ । \newline
69. सर्व॑स्य॒ जित्यै॒ जित्यै॒ सर्व॑स्य॒ सर्व॑स्य॒ जित्यै॒ सर्वꣳ॒॒ सर्व॒म् जित्यै॒ सर्व॑स्य॒ सर्व॑स्य॒ जित्यै॒ सर्व᳚म् । \newline
70. जित्यै॒ सर्वꣳ॒॒ सर्व॒म् जित्यै॒ जित्यै॒ सर्व॑ मे॒वैव सर्व॒म् जित्यै॒ जित्यै॒ सर्व॑ मे॒व । \newline
71. सर्व॑ मे॒वैव सर्वꣳ॒॒ सर्व॑ मे॒व तेन॒ तेनै॒व सर्वꣳ॒॒ सर्व॑ मे॒व तेन॑ । \newline
72. ए॒व तेन॒ तेनै॒ वैव तेना᳚प्नो त्याप्नोति॒ तेनै॒ वैव तेना᳚प्नोति । \newline
73. तेना᳚प्नो त्याप्नोति॒ तेन॒ तेना᳚प्नोति॒ सर्वꣳ॒॒ सर्व॑ माप्नोति॒ तेन॒ तेना᳚प्नोति॒ सर्व᳚म् । \newline
74. आ॒प्नो॒ति॒ सर्वꣳ॒॒ सर्व॑ माप्नो त्याप्नोति॒ सर्व॑म् जयति जयति॒ सर्व॑ माप्नो त्याप्नोति॒ सर्व॑म् जयति । \newline
75. सर्व॑म् जयति जयति॒ सर्वꣳ॒॒ सर्व॑म् जयति । \newline
76. ज॒य॒तीति॑ जयति । \newline
\pagebreak


\end{document}