\documentclass[17pt]{extarticle}
\usepackage{babel}
\usepackage{fontspec}
\usepackage{polyglossia}
\usepackage{extsizes}



\setmainlanguage{sanskrit}
\setotherlanguages{english} %% or other languages
\setlength{\parindent}{0pt}
\pagestyle{myheadings}
\newfontfamily\devanagarifont[Script=Devanagari]{AdishilaVedic}


\newcommand{\VAR}[1]{}
\newcommand{\BLOCK}[1]{}




\begin{document}
\begin{titlepage}
    \begin{center}
 
\begin{sanskrit}
    { \Huge
    कृष्ण यजुर्वेदीय तैत्तिरीय संहिता,पद,जटा,घन पाठः 
    }
    \\
    \vspace{2.5cm}
    \mbox{ \Huge
    1.8     प्रथमकाण्डे अषमः प्रश्नः - ( राजसूयः )   }
\end{sanskrit}
\end{center}

\end{titlepage}
\tableofcontents
\pagebreak

\markright{ TS 1.8.1.1  \hfill https://www.vedavms.in \hfill}
\addcontentsline{toc}{section}{ TS 1.8.1.1 }
\section*{ TS 1.8.1.1 }

\textbf{TS 1.8.1.1 } \newline
\textbf{Samhita Paata} \newline

अनु॑मत्यै पुरो॒डाश॑-म॒ष्टाक॑पालं॒ निर्व॑पति धे॒नुर् दक्षि॑णा॒ ये प्र॒त्यञ्चः॒ शंॅया॑या अव॒शीय॑न्ते॒ तं नैर्.ऋ॒त-मेक॑कपालं कृ॒ष्णं ॅवासः॑ कृ॒ष्णतू॑षं॒ दक्षि॑णा॒ वीहि॒ स्वाहाऽऽहु॑तिं जुषा॒ण ए॒ष ते॑ निर्.ऋते भा॒गो भूते॑ ह॒विष्म॑त्यसि मु॒ञ्चेम-मꣳह॑सः॒ स्वाहा॒ नमो॒ य इ॒दं च॒कारा॑ऽऽदि॒त्यं च॒रुं निर्व॑पति॒ वरो॒ दक्षि॑णाऽऽग्नावैष्ण॒व-मेका॑दशकपालं ॅवाम॒नो व॒ही दक्षि॑णा ऽग्नीषो॒मीय॒ - [ ] \newline

\textbf{Pada Paata} \newline

अनु॑मत्या॒ इत्यनु॑-म॒त्यै॒ । पु॒रो॒डाश᳚म् । अ॒ष्टाक॑पाल॒मित्य॒ष्टा-क॒पा॒ल॒म् । निरिति॑ । व॒प॒ति॒ । धे॒नुः । दक्षि॑णा । ये । प्र॒त्यञ्चः॑ । शम्या॑याः । अ॒व॒शीय॑न्त॒ इत्य॑व - शीय॑न्ते । तम् । नै॒र्.॒ऋ॒तमिति॑ नैः - ऋ॒तम् । एक॑कपाल॒मित्येक॑ - क॒पा॒ल॒म् । कृ॒ष्णम् । वासः॑ । कृ॒ष्णतू॑ष॒मिति॑ कृ॒ष्ण-तू॒ष॒म् । दक्षि॑णा । वीति॑ । इ॒हि॒ । स्वाहा᳚ । आहु॑ति॒मित्या-हु॒ति॒म् । जु॒षा॒णः । ए॒षः । ते॒ । नि॒र्.॒ऋ॒त॒ इति॑ निः-ऋ॒ते॒ । भा॒गः । भूते᳚ । ह॒विष्म॑ती । अ॒सि॒ । मु॒ञ्च । इ॒मम् । अꣳह॑सः । स्वाहा᳚ । नमः॑ । यः । इ॒दम् । च॒कार॑ । आ॒दि॒त्यम् । च॒रुम् । निरिति॑ । व॒प॒ति॒ । वरः॑ । दक्षि॑णा । आ॒ग्ना॒वै॒ष्ण॒वमित्या᳚ग्ना - वै॒ष्ण॒वम् । एका॑दशकपाल॒मित्येका॑दश - क॒पा॒ल॒म् । वा॒म॒नः । व॒ही । दक्षि॑णा । अ॒ग्नी॒षो॒मीय॒मित्य॑ग्नी - सो॒मीय᳚म् ।  \newline



\textbf{Jatai Paata} \newline

1. अनु॑मत्यै पुरो॒डाश॑म् पुरो॒डाश॒ मनु॑मत्या॒ अनु॑मत्यै 
पुरो॒डाश᳚म् । \newline
2. अनु॑मत्या॒ इत्यनु॑ - म॒त्यै॒ । \newline
3. पु॒रो॒डाश॑ म॒ष्टाक॑पाल म॒ष्टाक॑पालम् पुरो॒डाश॑म् पुरो॒डाश॑ म॒ष्टाक॑पालम् । \newline
4. अ॒ष्टाक॑पाल॒म् निर् णिर॒ष्टाक॑पाल म॒ष्टाक॑पाल॒म् निः । \newline
5. अ॒ष्टाक॑पाल॒मित्य॒ष्टा - क॒पा॒ल॒म् । \newline
6. निर् व॑पति वपति॒ निर् णिर् व॑पति । \newline
7. व॒प॒ति॒ धे॒नुर् धे॒नुर् व॑पति वपति धे॒नुः । \newline
8. धे॒नुर् दक्षि॑णा॒ दक्षि॑णा धे॒नुर् धे॒नुर् दक्षि॑णा । \newline
9. दक्षि॑णा॒ ये ये दक्षि॑णा॒ दक्षि॑णा॒ ये । \newline
10. ये प्र॒त्यञ्चः॑ प्र॒त्यञ्चो॒ ये ये प्र॒त्यञ्चः॑ । \newline
11. प्र॒त्यञ्चः॒ शम्या॑याः॒ शम्या॑याः प्र॒त्यञ्चः॑ प्र॒त्यञ्चः॒ शम्या॑याः । \newline
12. शम्या॑या अव॒शीय॑न्ते ऽव॒शीय॑न्ते॒ शम्या॑याः॒ शम्या॑या 
अव॒शीय॑न्ते । \newline
13. अ॒व॒शीय॑न्ते॒ तम् त म॑व॒शीय॑न्ते ऽव॒शीय॑न्ते॒ तम् । \newline
14. अ॒व॒शीय॑न्त॒ इत्य॑व - शीय॑न्ते । \newline
15. तम् नैर्॑.ऋ॒तम् नैर्॑.ऋ॒तम् तम् तम् नैर्॑.ऋ॒तम् । \newline
16. नै॒र्॒.ऋ॒त मेक॑कपाल॒ मेक॑कपालम् नैर्.ऋ॒तम् नैर्॑.ऋ॒त मेक॑कपालम् । \newline
17. नै॒र्.॒ऋ॒तमिति॑ नैः - ऋ॒तम् । \newline
18. एक॑कपालम् कृ॒ष्णम् कृ॒ष्ण मेक॑कपाल॒ मेक॑कपालम् कृ॒ष्णम् । \newline
19. एक॑कपाल॒मित्येक॑ - क॒पा॒ल॒म् । \newline
20. कृ॒ष्णं ॅवासो॒ वासः॑ कृ॒ष्णम् कृ॒ष्णं ॅवासः॑ । \newline
21. वासः॑ कृ॒ष्णतू॑षम् कृ॒ष्णतू॑षं॒ ॅवासो॒ वासः॑ कृ॒ष्णतू॑षम् । \newline
22. कृ॒ष्णतू॑ष॒म् दक्षि॑णा॒ दक्षि॑णा कृ॒ष्णतू॑षम् कृ॒ष्णतू॑ष॒म् दक्षि॑णा । \newline
23. कृ॒ष्णतू॑ष॒मिति॑ कृ॒ष्ण - तू॒ष॒म् । \newline
24. दक्षि॑णा॒ वि वि दक्षि॑णा॒ दक्षि॑णा॒ वि । \newline
25. वीही॑हि॒ वि वीहि॑ । \newline
26. इ॒हि॒ स्वाहा॒ स्वा हे॑हीहि॒ स्वाहा᳚ । \newline
27. स्वाहा ऽऽहु॑ति॒ माहु॑तिꣳ॒॒ स्वाहा॒ स्वाहा ऽऽहु॑तिम् । \newline
28. आहु॑तिम् जुषा॒णो जु॑षा॒ण आहु॑ति॒ माहु॑तिम् जुषा॒णः । \newline
29. आहु॑ति॒मित्या - हु॒ति॒म् । \newline
30. जु॒षा॒ण ए॒ष ए॒ष जु॑षा॒णो जु॑षा॒ण ए॒षः । \newline
31. ए॒ष ते॑ त ए॒ष ए॒ष ते᳚ । \newline
32. ते॒ नि॒र्॒.ऋ॒ते॒ नि॒र्॒.ऋ॒ते॒ ते॒ ते॒ नि॒र्॒.ऋ॒ते॒ । \newline
33. नि॒र्॒.ऋ॒ते॒ भा॒गो भा॒गो नि॑र्.ऋते निर्.ऋते भा॒गः । \newline
34. नि॒र्॒.ऋ॒त॒ इति॑ निः - ऋ॒ते॒ । \newline
35. भा॒गो भूते॒ भूते॑ भा॒गो भा॒गो भूते᳚ । \newline
36. भूते॑ ह॒विष्म॑ती ह॒विष्म॑ती॒ भूते॒ भूते॑ ह॒विष्म॑ती । \newline
37. ह॒विष्म॑ त्यस्यसि ह॒विष्म॑ती ह॒विष्म॑ त्यसि । \newline
38. अ॒सि॒ मु॒ञ्च मु॒ञ्चा स्य॑सि मु॒ञ्च । \newline
39. मु॒ञ्चे म मि॒मम् मु॒ञ्च मु॒ञ्चे मम् । \newline
40. इ॒म मꣳह॒सो ऽꣳह॑स इ॒म मि॒म मꣳह॑सः । \newline
41. अꣳह॑सः॒ स्वाहा॒ स्वाहा ऽꣳह॒सो ऽꣳह॑सः॒ स्वाहा᳚ । \newline
42. स्वाहा॒ नमो॒ नमः॒ स्वाहा॒ स्वाहा॒ नमः॑ । \newline
43. नमो॒ यो यो नमो॒ नमो॒ यः । \newline
44. य इ॒द मि॒दं ॅयो य इ॒दम् । \newline
45. इ॒दम् च॒कार॑ च॒कारे॒ द मि॒दम् च॒कार॑ । \newline
46. च॒का रा॑दि॒त्य मा॑दि॒त्यम् च॒कार॑ च॒का रा॑दि॒त्यम् । \newline
47. आ॒दि॒त्यम् च॒रुम् च॒रु मा॑दि॒त्य मा॑दि॒त्यम् च॒रुम् । \newline
48. च॒रुम् निर् णिश्च॒रुम् च॒रुम् निः । \newline
49. निर् व॑पति वपति॒ निर् णिर् व॑पति । \newline
50. व॒प॒ति॒ वरो॒ वरो॑ वपति वपति॒ वरः॑ । \newline
51. वरो॒ दक्षि॑णा॒ दक्षि॑णा॒ वरो॒ वरो॒ दक्षि॑णा । \newline
52. दक्षि॑णा ऽऽग्नावैष्ण॒व मा᳚ग्नावैष्ण॒वम् दक्षि॑णा॒ दक्षि॑णा ऽऽग्नावैष्ण॒वम् । \newline
53. आ॒ग्ना॒वै॒ष्ण॒व मेका॑दशकपाल॒ मेका॑दशकपाल माग्नावैष्ण॒व मा᳚ग्नावैष्ण॒व मेका॑दशकपालम् । \newline
54. आ॒ग्ना॒वै॒ष्ण॒वमित्या᳚ग्ना - वै॒ष्ण॒वम् । \newline
55. एका॑दशकपालं ॅवाम॒नो वा॑म॒न एका॑दशकपाल॒ मेका॑दशकपालं ॅवाम॒नः । \newline
56. एका॑दशकपाल॒मित्येका॑दश - क॒पा॒ल॒म् । \newline
57. वा॒म॒नो व॒ही व॒ही वा॑म॒नो वा॑म॒नो व॒ही । \newline
58. व॒ही दक्षि॑णा॒ दक्षि॑णा व॒ही व॒ही दक्षि॑णा । \newline
59. दक्षि॑णा ऽग्नीषो॒मीय॑ मग्नीषो॒मीय॒म् दक्षि॑णा॒ दक्षि॑णा ऽग्नीषो॒मीय᳚म् । \newline
60. अ॒ग्नी॒षो॒मीय॒ मेका॑दशकपाल॒ मेका॑दशकपाल मग्नीषो॒मीय॑ मग्नीषो॒मीय॒ मेका॑दशकपालम् । \newline
61. अ॒ग्नी॒षो॒मीय॒मित्य॑ग्नी - सो॒मीय᳚म् । \newline

\textbf{Ghana Paata } \newline

1. अनु॑मत्यै पुरो॒डाश॑म् पुरो॒डाश॒ मनु॑मत्या॒ अनु॑मत्यै पुरो॒डाश॑ म॒ष्टाक॑पाल म॒ष्टाक॑पालम् पुरो॒डाश॒ मनु॑मत्या॒ अनु॑मत्यै पुरो॒डाश॑ म॒ष्टाक॑पालम् । \newline
2. अनु॑मत्या॒ इत्यनु॑ - म॒त्यै॒ । \newline
3. पु॒रो॒डाश॑ म॒ष्टाक॑पाल म॒ष्टाक॑पालम् पुरो॒डाश॑म् पुरो॒डाश॑ म॒ष्टाक॑पाल॒म् निर् णिर॒ष्टाक॑पालम् पुरो॒डाश॑म् पुरो॒डाश॑ म॒ष्टाक॑पाल॒म् निः । \newline
4. अ॒ष्टाक॑पाल॒म् निर् णिर॒ष्टाक॑पाल म॒ष्टाक॑पाल॒म् निर् व॑पति वपति॒ निर॒ष्टाक॑पाल म॒ष्टाक॑पाल॒म् निर् व॑पति । \newline
5. अ॒ष्टाक॑पाल॒मित्य॒ष्टा - क॒पा॒ल॒म् । \newline
6. निर् व॑पति वपति॒ निर् णिर् व॑पति धे॒नुर् धे॒नुर् व॑पति॒ निर् णिर् व॑पति धे॒नुः । \newline
7. व॒प॒ति॒ धे॒नुर् धे॒नुर् व॑पति वपति धे॒नुर् दक्षि॑णा॒ दक्षि॑णा धे॒नुर् व॑पति वपति धे॒नुर् दक्षि॑णा । \newline
8. धे॒नुर् दक्षि॑णा॒ दक्षि॑णा धे॒नुर् धे॒नुर् दक्षि॑णा॒ ये ये दक्षि॑णा धे॒नुर् धे॒नुर् दक्षि॑णा॒ ये । \newline
9. दक्षि॑णा॒ ये ये दक्षि॑णा॒ दक्षि॑णा॒ ये प्र॒त्यञ्चः॑ प्र॒त्यञ्चो॒ ये दक्षि॑णा॒ दक्षि॑णा॒ ये प्र॒त्यञ्चः॑ । \newline
10. ये प्र॒त्यञ्चः॑ प्र॒त्यञ्चो॒ ये ये प्र॒त्यञ्चः॒ शम्या॑याः॒ शम्या॑याः प्र॒त्यञ्चो॒ ये ये प्र॒त्यञ्चः॒ शम्या॑याः । \newline
11. प्र॒त्यञ्चः॒ शम्या॑याः॒ शम्या॑याः प्र॒त्यञ्चः॑ प्र॒त्यञ्चः॒ शम्या॑या अव॒शीय॑न्ते ऽव॒शीय॑न्ते॒ शम्या॑याः प्र॒त्यञ्चः॑ प्र॒त्यञ्चः॒ शम्या॑या अव॒शीय॑न्ते । \newline
12. शम्या॑या अव॒शीय॑न्ते ऽव॒शीय॑न्ते॒ शम्या॑याः॒ शम्या॑या अव॒शीय॑न्ते॒ तम् त म॑व॒शीय॑न्ते॒ शम्या॑याः॒ शम्या॑या अव॒शीय॑न्ते॒ तम् । \newline
13. अ॒व॒शीय॑न्ते॒ तम् त म॑व॒शीय॑न्ते ऽव॒शीय॑न्ते॒ तम् नैर्॑.ऋ॒तन् नैर्॑.ऋ॒तम् त म॑व॒शीय॑न्ते ऽव॒शीय॑न्ते॒ तम् नैर्॑.ऋ॒तम् । \newline
14. अ॒व॒शीय॑न्त॒ इत्य॑व - शीय॑न्ते । \newline
15. तम् नैर्॑.ऋ॒तम् नैर्॑.ऋ॒तम् तम् तम् नै॑र्.ऋ॒त मेक॑कपाल॒ मेक॑कपालम् नैर्.ऋ॒तम् तम् तम् नैर्॑.ऋ॒त मेक॑कपालम् । \newline
16. नै॒र्॒.ऋ॒त मेक॑कपाल॒ मेक॑कपालम् नैर्.ऋ॒तम् नैर्॑.ऋ॒त मेक॑कपालम् कृ॒ष्णम् कृ॒ष्ण मेक॑कपालम् नैर्.ऋ॒तम् नैर्॑.ऋ॒त मेक॑कपालम् कृ॒ष्णम् । \newline
17. नै॒र्.॒ऋ॒तमिति॑ नैः - ऋ॒तम् । \newline
18. एक॑कपालम् कृ॒ष्णम् कृ॒ष्ण मेक॑कपाल॒ मेक॑कपालम् कृ॒ष्णं ॅवासो॒वासः॑ कृ॒ष्ण मेक॑कपाल॒ मेक॑कपालम् कृ॒ष्णं ॅवासः॑ । \newline
19. एक॑कपाल॒मित्येक॑ - क॒पा॒ल॒म् । \newline
20. कृ॒ष्णं ॅवासो॒ वासः॑ कृ॒ष्णम् कृ॒ष्णं ॅवासः॑ कृ॒ष्णतू॑षम् कृ॒ष्णतू॑षं॒ ॅवासः॑ कृ॒ष्णम् कृ॒ष्णं ॅवासः॑ कृ॒ष्णतू॑षम् । \newline
21. वासः॑ कृ॒ष्णतू॑षम् कृ॒ष्णतू॑षं॒ ॅवासो॒ वासः॑ कृ॒ष्णतू॑ष॒म् दक्षि॑णा॒ दक्षि॑णा कृ॒ष्णतू॑षं॒ ॅवासो॒ वासः॑ कृ॒ष्णतू॑ष॒म् दक्षि॑णा । \newline
22. कृ॒ष्णतू॑ष॒म् दक्षि॑णा॒ दक्षि॑णा कृ॒ष्णतू॑षम् कृ॒ष्णतू॑ष॒म् दक्षि॑णा॒ वि वि दक्षि॑णा कृ॒ष्णतू॑षम् कृ॒ष्णतू॑ष॒म् दक्षि॑णा॒ वि । \newline
23. कृ॒ष्णतू॑ष॒मिति॑ कृ॒ष्ण - तू॒ष॒म् । \newline
24. दक्षि॑णा॒ वि वि दक्षि॑णा॒ दक्षि॑णा॒ वीही॑हि॒ वि दक्षि॑णा॒ दक्षि॑णा॒ वीहि॑ । \newline
25. वीही॑हि॒ वि वीहि॒ स्वाहा॒ स्वाहे॑हि॒ वि वीहि॒ स्वाहा᳚ । \newline
26. इ॒हि॒ स्वाहा॒ स्वाहे॑हीहि॒ स्वाहा ऽऽहु॑ति॒ माहु॑तिꣳ॒॒ स्वाहे॑हीहि॒ स्वाहा ऽऽहु॑तिम् । \newline
27. स्वाहा ऽऽहु॑ति॒ माहु॑तिꣳ॒॒ स्वाहा॒ स्वाहा ऽऽहु॑तिम् जुषा॒णो जु॑षा॒ण आहु॑तिꣳ॒॒ स्वाहा॒ स्वाहा ऽऽहु॑तिम् जुषा॒णः । \newline
28. आहु॑तिम् जुषा॒णो जु॑षा॒ण आहु॑ति॒ माहु॑तिम् जुषा॒ण ए॒ष ए॒ष जु॑षा॒ण आहु॑ति॒ माहु॑तिम् जुषा॒ण ए॒षः । \newline
29. आहु॑ति॒मित्या - हु॒ति॒म् । \newline
30. जु॒षा॒ण ए॒ष ए॒ष जु॑षा॒णो जु॑षा॒ण ए॒ष ते॑ त ए॒ष जु॑षा॒णो जु॑षा॒ण ए॒ष ते᳚ । \newline
31. ए॒ष ते॑ त ए॒ष ए॒ष ते॑ निर्.ऋते निर्.ऋते त ए॒ष ए॒ष ते॑ निर्.ऋते । \newline
32. ते॒ नि॒र्॒.ऋ॒ते॒ नि॒र्॒.ऋ॒ते॒ ते॒ ते॒ नि॒र्॒.ऋ॒ते॒ भा॒गो भा॒गो नि॑र्.ऋते ते ते निर्.ऋते भा॒गः । \newline
33. नि॒र्॒.ऋ॒ते॒ भा॒गो भा॒गो नि॑र्.ऋते निर्.ऋते भा॒गो भूते॒ भूते॑ भा॒गो नि॑र्.ऋते निर्.ऋते भा॒गो भूते᳚ । \newline
34. नि॒र्॒.ऋ॒त॒ इति॑ निः - ऋ॒ते॒ । \newline
35. भा॒गो भूते॒ भूते॑ भा॒गो भा॒गो भूते॑ ह॒विष्म॑ती ह॒विष्म॑ती॒ भूते॑ भा॒गो भा॒गो भूते॑ ह॒विष्म॑ती । \newline
36. भूते॑ ह॒विष्म॑ती ह॒विष्म॑ती॒ भूते॒ भूते॑ ह॒विष्म॑ त्यस्यसि ह॒विष्म॑ती॒ भूते॒ भूते॑ ह॒विष्म॑ त्यसि । \newline
37. ह॒विष्म॑ त्यस्यसि ह॒विष्म॑ती ह॒विष्म॑ त्यसि मु॒ञ्च मु॒ञ्चासि॑ ह॒विष्म॑ती ह॒विष्म॑ त्यसि मु॒ञ्च । \newline
38. अ॒सि॒ मु॒ञ्च मु॒ञ्चा स्य॑सि मु॒ञ्चे म मि॒मम् मु॒ञ्चा स्य॑सि मु॒ञ्चे मम् । \newline
39. मु॒ञ्चे म मि॒मम् मु॒ञ्च मु॒ञ्चे म मꣳह॒सो ऽꣳह॑स इ॒मम् मु॒ञ्च मु॒ञ्चे म मꣳह॑सः । \newline
40. इ॒म मꣳह॒सो ऽꣳह॑स इ॒म मि॒म मꣳह॑सः॒ स्वाहा॒ स्वाहा ऽꣳह॑स इ॒म मि॒म मꣳह॑सः॒ स्वाहा᳚ । \newline
41. अꣳह॑सः॒ स्वाहा॒ स्वाहा ऽꣳह॒सो ऽꣳह॑सः॒ स्वाहा॒ नमो॒ नमः॒ स्वाहा ऽꣳह॒सो ऽꣳह॑सः॒ स्वाहा॒ नमः॑ । \newline
42. स्वाहा॒ नमो॒ नमः॒ स्वाहा॒ स्वाहा॒ नमो॒ यो यो नमः॒ स्वाहा॒ स्वाहा॒ नमो॒ यः । \newline
43. नमो॒ यो यो नमो॒ नमो॒ य इ॒द मि॒दं ॅयो नमो॒ नमो॒ य इ॒दम् । \newline
44. य इ॒द मि॒दं ॅयो य इ॒दम् च॒कार॑ च॒कारे॒ दं ॅयो य इ॒दम् च॒कार॑ । \newline
45. इ॒दम् च॒कार॑ च॒कारे॒ द मि॒दम् च॒कारा॑दि॒त्य मा॑दि॒त्यम् च॒कारे॒ द मि॒दम् च॒कारा॑दि॒त्यम् । \newline
46. च॒कारा॑दि॒त्य मा॑दि॒त्यम् च॒कार॑ च॒कारा॑दि॒त्यम् च॒रुम् च॒रु मा॑दि॒त्यम् च॒कार॑ च॒कारा॑दि॒त्यम् च॒रुम् । \newline
47. आ॒दि॒त्यम् च॒रुम् च॒रु मा॑दि॒त्य मा॑दि॒त्यम् च॒रुम् निर् णिश्च॒रु मा॑दि॒त्य मा॑दि॒त्यम् च॒रुम् निः । \newline
48. च॒रुम् निर् णिश्च॒रुम् च॒रुम् निर् व॑पति वपति॒ निश्च॒रुम् च॒रुम् निर् व॑पति । \newline
49. निर् व॑पति वपति॒ निर् णिर् व॑पति॒ वरो॒ वरो॑ वपति॒ निर् णिर् व॑पति॒ वरः॑ । \newline
50. व॒प॒ति॒ वरो॒ वरो॑ वपति वपति॒ वरो॒ दक्षि॑णा॒ दक्षि॑णा॒ वरो॑ वपति वपति॒ वरो॒ दक्षि॑णा । \newline
51. वरो॒ दक्षि॑णा॒ दक्षि॑णा॒ वरो॒ वरो॒ दक्षि॑णा ऽऽग्नावैष्ण॒व मा᳚ग्नावैष्ण॒वम् दक्षि॑णा॒ वरो॒ वरो॒ दक्षि॑णा ऽऽग्नावैष्ण॒वम् । \newline
52. दक्षि॑णा ऽऽग्नावैष्ण॒व मा᳚ग्नावैष्ण॒वम् दक्षि॑णा॒ दक्षि॑णा ऽऽग्नावैष्ण॒व मेका॑दशकपाल॒ मेका॑दशकपाल माग्नावैष्ण॒वम् दक्षि॑णा॒ दक्षि॑णा ऽऽग्नावैष्ण॒व मेका॑दशकपालम् । \newline
53. आ॒ग्ना॒वै॒ष्ण॒व मेका॑दशकपाल॒ मेका॑दशकपाल माग्नावैष्ण॒व मा᳚ग्नावैष्ण॒व मेका॑दशकपालं ॅवाम॒नो वा॑म॒न एका॑दशकपाल माग्नावैष्ण॒व मा᳚ग्नावैष्ण॒व मेका॑दशकपालं ॅवाम॒नः । \newline
54. आ॒ग्ना॒वै॒ष्ण॒वमित्या᳚ग्ना - वै॒ष्ण॒वम् । \newline
55. एका॑दशकपालं ॅवाम॒नो वा॑म॒न एका॑दशकपाल॒ मेका॑दशकपालं ॅवाम॒नो व॒ही व॒ही वा॑म॒न एका॑दशकपाल॒ मेका॑दशकपालं ॅवाम॒नो व॒ही । \newline
56. एका॑दशकपाल॒मित्येका॑दश - क॒पा॒ल॒म् । \newline
57. वा॒म॒नो व॒ही व॒ही वा॑म॒नो वा॑म॒नो व॒ही दक्षि॑णा॒ दक्षि॑णा व॒ही वा॑म॒नो वा॑म॒नो व॒ही दक्षि॑णा । \newline
58. व॒ही दक्षि॑णा॒ दक्षि॑णा व॒ही व॒ही दक्षि॑णा ऽग्नीषो॒मीय॑ मग्नीषो॒मीय॒म् दक्षि॑णा व॒ही व॒ही 
दक्षि॑णा ऽग्नीषो॒मीय᳚म् । \newline
59. दक्षि॑णा ऽग्नीषो॒मीय॑ मग्नीषो॒मीय॒म् दक्षि॑णा॒ दक्षि॑णा ऽग्नीषो॒मीय॒ मेका॑दशकपाल॒ मेका॑दशकपाल मग्नीषो॒मीय॒म् दक्षि॑णा॒ दक्षि॑णा ऽग्नीषो॒मीय॒ मेका॑दशकपालम् । \newline
60. अ॒ग्नी॒षो॒मीय॒ मेका॑दशकपाल॒ मेका॑दशकपाल मग्नीषो॒मीय॑ मग्नीषो॒मीय॒ मेका॑दशकपालꣳ॒॒ हिर॑ण्यꣳ॒॒ हिर॑ण्य॒ मेका॑दशकपाल मग्नीषो॒मीय॑ मग्नीषो॒मीय॒ मेका॑दशकपालꣳ॒॒ हिर॑ण्यम् । \newline
61. अ॒ग्नी॒षो॒मीय॒मित्य॑ग्नी - सो॒मीय᳚म् । \newline
\pagebreak
\markright{ TS 1.8.1.2  \hfill https://www.vedavms.in \hfill}
\addcontentsline{toc}{section}{ TS 1.8.1.2 }
\section*{ TS 1.8.1.2 }

\textbf{TS 1.8.1.2 } \newline
\textbf{Samhita Paata} \newline

मेका॑दशकपालꣳ॒॒ हिर॑ण्यं॒ दक्षि॑णै॒न्द्र-मेका॑दशकपाल-मृष॒भो व॒ही दक्षि॑णाऽऽग्ने॒य-म॒ष्टाक॑पालमै॒न्द्रं दद्ध्यृ॑ष॒भो व॒ही दक्षि॑णैन्द्रा॒ग्नं द्वाद॑शकपालं ॅवैश्वदे॒वं च॒रुं प्र॑थम॒जो व॒थ्सो दक्षि॑णा सौ॒म्यꣳ श्या॑मा॒कं च॒रुं ॅवासो॒ दक्षि॑णा॒ सर॑स्वत्यै च॒रुꣳ सर॑स्वते च॒रुं मि॑थु॒नौ गावौ॒ दक्षि॑णा ॥ \newline

\textbf{Pada Paata} \newline

एका॑दशकपाल॒मित्येका॑दश-क॒पा॒ल॒म् । हिर॑ण्यम् । दक्षि॑णा । ऐ॒न्द्रम् । एका॑दशकपाल॒मित्येका॑दश - क॒पा॒ल॒म् । ऋ॒ष॒भः । व॒ही । दक्षि॑णा । आ॒ग्ने॒यम् । अ॒ष्टाक॑पाल॒मित्य॒ष्टा-क॒पा॒ल॒म् । ऐ॒न्द्रम् । दधि॑ । ऋ॒ष॒भः । व॒ही । दक्षि॑णा । ऐ॒न्द्रा॒ग्नमित्यै᳚न्द्र -अ॒ग्नम् । द्वाद॑शकपाल॒मिति॒ द्वाद॑श -क॒पा॒ल॒म् । वै॒श्व॒दे॒वमिति॑ वैश्व - दे॒वम् । च॒रुम् । प्र॒थ॒म॒ज इति॑ प्रथम - जः । व॒थ्सः । दक्षि॑णा । सौ॒म्यम् । श्या॒मा॒कम् । च॒रुम् । वासः॑ । दक्षि॑णा । सर॑स्वत्यै । च॒रुम् । सर॑स्वते । च॒रुम् । मि॒थु॒नौ । गावौ᳚ । दक्षि॑णा ॥  \newline



\textbf{Jatai Paata} \newline

1. एका॑दशकपालꣳ॒॒ हिर॑ण्यꣳ॒॒ हिर॑ण्य॒ मेका॑दशकपाल॒ मेका॑दशकपालꣳ॒॒ हिर॑ण्यम् । \newline
2. एका॑दशकपाल॒मित्येका॑दश - क॒पा॒ल॒म् । \newline
3. हिर॑ण्य॒म् दक्षि॑णा॒ दक्षि॑णा॒ हिर॑ण्यꣳ॒॒ हिर॑ण्य॒म् दक्षि॑णा । \newline
4. दक्षि॑णै॒न्द्र मै॒न्द्रम् दक्षि॑णा॒ दक्षि॑णै॒न्द्रम् । \newline
5. ऐ॒न्द्र मेका॑दशकपाल॒ मेका॑दशकपाल मै॒न्द्र मै॒न्द्र मेका॑दशकपालम् । \newline
6. एका॑दशकपाल मृष॒भ ऋ॑ष॒भ एका॑दशकपाल॒ मेका॑दशकपाल मृष॒भः । \newline
7. एका॑दशकपाल॒मित्येका॑दश - क॒पा॒ल॒म् । \newline
8. ऋ॒ष॒भो व॒ही व॒ह्यृ॑ष॒भ ऋ॑ष॒भो व॒ही । \newline
9. व॒ही दक्षि॑णा॒ दक्षि॑णा व॒ही व॒ही दक्षि॑णा । \newline
10. दक्षि॑णा ऽऽग्ने॒य मा᳚ग्ने॒यम् दक्षि॑णा॒ दक्षि॑णा ऽऽग्ने॒यम् । \newline
11. आ॒ग्ने॒य म॒ष्टाक॑पाल म॒ष्टाक॑पाल माग्ने॒य मा᳚ग्ने॒य म॒ष्टाक॑पालम् । \newline
12. अ॒ष्टाक॑पाल मै॒न्द्र मै॒न्द्र म॒ष्टाक॑पाल म॒ष्टाक॑पाल मै॒न्द्रम् । \newline
13. अ॒ष्टाक॑पाल॒मित्य॒ष्टा - क॒पा॒ल॒म् । \newline
14. ऐ॒न्द्रम् दधि॒ दध्यै॒न्द्र मै॒न्द्रम् दधि॑ । \newline
15. दध्यृ॑ष॒भ ऋ॑ष॒भो दधि॒ दध्यृ॑ष॒भः । \newline
16. ऋ॒ष॒भो व॒ही व॒ह्यृ॑ष॒भ ऋ॑ष॒भो व॒ही । \newline
17. व॒ही दक्षि॑णा॒ दक्षि॑णा व॒ही व॒ही दक्षि॑णा । \newline
18. दक्षि॑णैन्द्रा॒ग्न मै᳚न्द्रा॒ग्नम् दक्षि॑णा॒ दक्षि॑ णैन्द्रा॒ग्नम् । \newline
19. ऐ॒न्द्रा॒ग्नम् द्वाद॑शकपाल॒म् द्वाद॑शकपाल मैन्द्रा॒ग्न मै᳚न्द्रा॒ग्नम् द्वाद॑शकपालम् । \newline
20. ऐ॒न्द्रा॒ग्नमित्यै᳚न्द्र - अ॒ग्नम् । \newline
21. द्वाद॑शकपालं ॅवैश्वदे॒वं ॅवै᳚श्वदे॒वम् द्वाद॑शकपाल॒म् द्वाद॑शकपालं ॅवैश्वदे॒वम् । \newline
22. द्वाद॑शकपाल॒मिति॒ द्वाद॑श - क॒पा॒ल॒म् । \newline
23. वै॒श्व॒दे॒वम् च॒रुम् च॒रुं ॅवै᳚श्वदे॒वं ॅवै᳚श्वदे॒वम् च॒रुम् । \newline
24. वै॒श्व॒दे॒वमिति॑ वैश्व - दे॒वम् । \newline
25. च॒रुम् प्र॑थम॒जः प्र॑थम॒ज श्च॒रुम् च॒रुम् प्र॑थम॒जः । \newline
26. प्र॒थ॒म॒जो व॒थ्सो व॒थ्सः प्र॑थम॒जः प्र॑थम॒जो व॒थ्सः । \newline
27. प्र॒थ॒म॒ज इति॑ प्रथम - जः । \newline
28. व॒थ्सो दक्षि॑णा॒ दक्षि॑णा व॒थ्सो व॒थ्सो दक्षि॑णा । \newline
29. दक्षि॑णा सौ॒म्यꣳ सौ॒म्यम् दक्षि॑णा॒ दक्षि॑णा सौ॒म्यम् । \newline
30. सौ॒म्यꣳ श्या॑मा॒कꣳ श्या॑मा॒कꣳ सौ॒म्यꣳ सौ॒म्यꣳ श्या॑मा॒कम् । \newline
31. श्या॒मा॒कम् च॒रुम् च॒रुꣳ श्या॑मा॒कꣳ श्या॑मा॒कम् च॒रुम् । \newline
32. च॒रुं ॅवासो॒ वास॑ श्च॒रुम् च॒रुं ॅवासः॑ । \newline
33. वासो॒ दक्षि॑णा॒ दक्षि॑णा॒ वासो॒ वासो॒ दक्षि॑णा । \newline
34. दक्षि॑णा॒ सर॑स्वत्यै॒ सर॑स्वत्यै॒ दक्षि॑णा॒ दक्षि॑णा॒ सर॑स्वत्यै । \newline
35. सर॑स्वत्यै च॒रुम् च॒रुꣳ सर॑स्वत्यै॒ सर॑स्वत्यै च॒रुम् । \newline
36. च॒रुꣳ सर॑स्वते॒ सर॑स्वते च॒रुम् च॒रुꣳ सर॑स्वते । \newline
37. सर॑स्वते च॒रुम् च॒रुꣳ सर॑स्वते॒ सर॑स्वते च॒रुम् । \newline
38. च॒रुम् मि॑थु॒नौ मि॑थु॒नौ च॒रुम् च॒रुम् मि॑थु॒नौ । \newline
39. मि॒थु॒नौ गावौ॒ गावौ॑ मिथु॒नौ मि॑थु॒नौ गावौ᳚ । \newline
40. गावौ॒ दक्षि॑णा॒ दक्षि॑णा॒ गावौ॒ गावौ॒ दक्षि॑णा । \newline
41. दक्षि॒णेति॒ दक्षि॑णा । \newline

\textbf{Ghana Paata } \newline

1. एका॑दशकपालꣳ॒॒ हिर॑ण्यꣳ॒॒ हिर॑ण्य॒ मेका॑दशकपाल॒ मेका॑दशकपालꣳ॒॒ हिर॑ण्य॒म् दक्षि॑णा॒ दक्षि॑णा॒ हिर॑ण्य॒ मेका॑दशकपाल॒ मेका॑दशकपालꣳ॒॒ हिर॑ण्य॒म् दक्षि॑णा । \newline
2. एका॑दशकपाल॒मित्येका॑दश - क॒पा॒ल॒म् । \newline
3. हिर॑ण्य॒म् दक्षि॑णा॒ दक्षि॑णा॒ हिर॑ण्यꣳ॒॒ हिर॑ण्य॒म् दक्षि॑णै॒न्द्र मै॒न्द्रम् दक्षि॑णा॒ हिर॑ण्यꣳ॒॒ हिर॑ण्य॒म् दक्षि॑णै॒न्द्रम् । \newline
4. दक्षि॑णै॒न्द्र मै॒न्द्रम् दक्षि॑णा॒ दक्षि॑णै॒न्द्र मेका॑दशकपाल॒ मेका॑दशकपाल मै॒न्द्रम् दक्षि॑णा॒ दक्षि॑णै॒न्द्र मेका॑दशकपालम् । \newline
5. ऐ॒न्द्र मेका॑दशकपाल॒ मेका॑दशकपाल मै॒न्द्र मै॒न्द्र मेका॑दशकपाल मृष॒भ ऋ॑ष॒भ एका॑दशकपाल मै॒न्द्र मै॒न्द्र मेका॑दशकपाल मृष॒भः । \newline
6. एका॑दशकपाल मृष॒भ ऋ॑ष॒भ एका॑दशकपाल॒ मेका॑दशकपाल मृष॒भो व॒ही व॒ह्यृ॑ष॒भ एका॑दशकपाल॒ मेका॑दशकपाल मृष॒भो व॒ही । \newline
7. एका॑दशकपाल॒मित्येका॑दश - क॒पा॒ल॒म् । \newline
8. ऋ॒ष॒भो व॒ही व॒ह्यृ॑ष॒भ ऋ॑ष॒भो व॒ही दक्षि॑णा॒ दक्षि॑णा व॒ह्यृ॑ष॒भ ऋ॑ष॒भो व॒ही दक्षि॑णा । \newline
9. व॒ही दक्षि॑णा॒ दक्षि॑णा व॒ही व॒ही दक्षि॑णा ऽऽग्ने॒य मा᳚ग्ने॒यम् दक्षि॑णा व॒ही व॒ही दक्षि॑णा ऽऽग्ने॒यम् । \newline
10. दक्षि॑णा ऽऽग्ने॒य मा᳚ग्ने॒यम् दक्षि॑णा॒ दक्षि॑णा ऽऽग्ने॒य म॒ष्टाक॑पाल म॒ष्टाक॑पाल माग्ने॒यम् दक्षि॑णा॒ दक्षि॑णा ऽऽग्ने॒य म॒ष्टाक॑पालम् । \newline
11. आ॒ग्ने॒य म॒ष्टाक॑पाल म॒ष्टाक॑पाल माग्ने॒य मा᳚ग्ने॒य म॒ष्टाक॑पाल मै॒न्द्र मै॒न्द्र म॒ष्टाक॑पाल माग्ने॒य मा᳚ग्ने॒य म॒ष्टाक॑पाल मै॒न्द्रम् । \newline
12. अ॒ष्टाक॑पाल मै॒न्द्र मै॒न्द्र म॒ष्टाक॑पाल म॒ष्टाक॑पाल मै॒न्द्रम् दधि॒ दध्यै॒न्द्र म॒ष्टाक॑पाल म॒ष्टाक॑पाल मै॒न्द्रम् दधि॑ । \newline
13. अ॒ष्टाक॑पाल॒मित्य॒ष्टा - क॒पा॒ल॒म् । \newline
14. ऐ॒न्द्रम् दधि॒ दध्यै॒न्द्र मै॒न्द्रम् दध्यृ॑ष॒भ ऋ॑ष॒भो दध्यै॒न्द्र मै॒न्द्रम् दध्यृ॑ष॒भः । \newline
15. दध्यृ॑ष॒भ ऋ॑ष॒भो दधि॒ दध्यृ॑ष॒भो व॒ही व॒ह्यृ॑ष॒भो दधि॒ दध्यृ॑ष॒भो व॒ही । \newline
16. ऋ॒ष॒भो व॒ही व॒ह्यृ॑ष॒भ ऋ॑ष॒भो व॒ही दक्षि॑णा॒ दक्षि॑णा व॒ह्यृ॑ष॒भ ऋ॑ष॒भो व॒ही दक्षि॑णा । \newline
17. व॒ही दक्षि॑णा॒ दक्षि॑णा व॒ही व॒ही दक्षि॑णैन्द्रा॒ग्न मै᳚न्द्रा॒ग्नम् दक्षि॑णा व॒ही व॒ही दक्षि॑णैन्द्रा॒ग्नम् । \newline
18. दक्षि॑णैन्द्रा॒ग्न मै᳚न्द्रा॒ग्नम् दक्षि॑णा॒ दक्षि॑णैन्द्रा॒ग्नम् द्वाद॑शकपाल॒म् द्वाद॑शकपाल मैन्द्रा॒ग्नम् दक्षि॑णा॒ दक्षि॑णैन्द्रा॒ग्नम् द्वाद॑शकपालम् । \newline
19. ऐ॒न्द्रा॒ग्नम् द्वाद॑शकपाल॒म् द्वाद॑शकपाल मैन्द्रा॒ग्न मै᳚न्द्रा॒ग्नम् द्वाद॑शकपालं ॅवैश्वदे॒वं ॅवै᳚श्वदे॒वम् द्वाद॑शकपाल मैन्द्रा॒ग्न मै᳚न्द्रा॒ग्नम् द्वाद॑शकपालं ॅवैश्वदे॒वम् । \newline
20. ऐ॒न्द्रा॒ग्नमित्यै᳚न्द्र - अ॒ग्नम् । \newline
21. द्वाद॑शकपालं ॅवैश्वदे॒वं ॅवै᳚श्वदे॒वम् द्वाद॑शकपाल॒म् द्वाद॑शकपालं ॅवैश्वदे॒वम् च॒रुम् च॒रुं ॅवै᳚श्वदे॒वम् द्वाद॑शकपाल॒म् द्वाद॑शकपालं ॅवैश्वदे॒वम् च॒रुम् । \newline
22. द्वाद॑शकपाल॒मिति॒ द्वाद॑श - क॒पा॒ल॒म् । \newline
23. वै॒श्व॒दे॒वम् च॒रुम् च॒रुं ॅवै᳚श्वदे॒वं ॅवै᳚श्वदे॒वम् च॒रुम् प्र॑थम॒जः प्र॑थम॒ज श्च॒रुं 
ॅवै᳚श्वदे॒वं ॅवै᳚श्वदे॒वम् च॒रुम् प्र॑थम॒जः । \newline
24. वै॒श्व॒दे॒वमिति॑ वैश्व - दे॒वम् । \newline
25. च॒रुम् प्र॑थम॒जः प्र॑थम॒ज श्च॒रुम् च॒रुम् प्र॑थम॒जो व॒थ्सो व॒थ्सः प्र॑थम॒ज श्च॒रुम् च॒रुम् प्र॑थम॒जो व॒थ्सः । \newline
26. प्र॒थ॒म॒जो व॒थ्सो व॒थ्सः प्र॑थम॒जः प्र॑थम॒जो व॒थ्सो दक्षि॑णा॒ दक्षि॑णा व॒थ्सः प्र॑थम॒जः प्र॑थम॒जो व॒थ्सो दक्षि॑णा । \newline
27. प्र॒थ॒म॒ज इति॑ प्रथम - जः । \newline
28. व॒थ्सो दक्षि॑णा॒ दक्षि॑णा व॒थ्सो व॒थ्सो दक्षि॑णा सौ॒म्यꣳ सौ॒म्यम् दक्षि॑णा व॒थ्सो व॒थ्सो दक्षि॑णा सौ॒म्यम् । \newline
29. दक्षि॑णा सौ॒म्यꣳ सौ॒म्यम् दक्षि॑णा॒ दक्षि॑णा सौ॒म्यꣳ श्या॑मा॒कꣳ श्या॑मा॒कꣳ सौ॒म्यम् दक्षि॑णा॒ दक्षि॑णा सौ॒म्यꣳ श्या॑मा॒कम् । \newline
30. सौ॒म्यꣳ श्या॑मा॒कꣳ श्या॑मा॒कꣳ सौ॒म्यꣳ सौ॒म्यꣳ श्या॑मा॒कम् च॒रुम् च॒रुꣳ श्या॑मा॒कꣳ सौ॒म्यꣳ सौ॒म्यꣳ श्या॑मा॒कम् च॒रुम् । \newline
31. श्या॒मा॒कम् च॒रुम् च॒रुꣳ श्या॑मा॒कꣳ श्या॑मा॒कम् च॒रुं ॅवासो॒ वास॑ श्च॒रुꣳ श्या॑मा॒कꣳ श्या॑मा॒कम् च॒रुं ॅवासः॑ । \newline
32. च॒रुं ॅवासो॒ वास॑ श्च॒रुम् च॒रुं ॅवासो॒ दक्षि॑णा॒ दक्षि॑णा॒ वास॑ श्च॒रुम् च॒रुं ॅवासो॒ दक्षि॑णा । \newline
33. वासो॒ दक्षि॑णा॒ दक्षि॑णा॒ वासो॒ वासो॒ दक्षि॑णा॒ सर॑स्वत्यै॒ सर॑स्वत्यै॒ दक्षि॑णा॒ वासो॒ वासो॒ दक्षि॑णा॒ सर॑स्वत्यै । \newline
34. दक्षि॑णा॒ सर॑स्वत्यै॒ सर॑स्वत्यै॒ दक्षि॑णा॒ दक्षि॑णा॒ सर॑स्वत्यै च॒रुम् च॒रुꣳ सर॑स्वत्यै॒ दक्षि॑णा॒ दक्षि॑णा॒ सर॑स्वत्यै च॒रुम् । \newline
35. सर॑स्वत्यै च॒रुम् च॒रुꣳ सर॑स्वत्यै॒ सर॑स्वत्यै च॒रुꣳ सर॑स्वते॒ सर॑स्वते च॒रुꣳ सर॑स्वत्यै॒ सर॑स्वत्यै च॒रुꣳ सर॑स्वते । \newline
36. च॒रुꣳ सर॑स्वते॒ सर॑स्वते च॒रुम् च॒रुꣳ सर॑स्वते च॒रुम् च॒रुꣳ सर॑स्वते च॒रुम् च॒रुꣳ सर॑स्वते च॒रुम् । \newline
37. सर॑स्वते च॒रुम् च॒रुꣳ सर॑स्वते॒ सर॑स्वते च॒रुम् मि॑थु॒नौ मि॑थु॒नौ च॒रुꣳ सर॑स्वते॒ सर॑स्वते च॒रुम् मि॑थु॒नौ । \newline
38. च॒रुम् मि॑थु॒नौ मि॑थु॒नौ च॒रुम् च॒रुम् मि॑थु॒नौ गावौ॒ गावौ॑ मिथु॒नौ च॒रुम् च॒रुम् मि॑थु॒नौ गावौ᳚ । \newline
39. मि॒थु॒नौ गावौ॒ गावौ॑ मिथु॒नौ मि॑थु॒नौ गावौ॒ दक्षि॑णा॒ दक्षि॑णा॒ गावौ॑ मिथु॒नौ मि॑थु॒नौ गावौ॒ दक्षि॑णा । \newline
40. गावौ॒ दक्षि॑णा॒ दक्षि॑णा॒ गावौ॒ गावौ॒ दक्षि॑णा । \newline
41. दक्षि॒णेति॒ दक्षि॑णा । \newline
\pagebreak
\markright{ TS 1.8.2.1  \hfill https://www.vedavms.in \hfill}
\addcontentsline{toc}{section}{ TS 1.8.2.1 }
\section*{ TS 1.8.2.1 }

\textbf{TS 1.8.2.1 } \newline
\textbf{Samhita Paata} \newline

आ॒ग्ने॒यम॒ष्टाक॑पालं॒ निर्व॑पति सौ॒म्यं च॒रुꣳ सा॑वि॒त्रं-द्वाद॑शकपालꣳ सारस्व॒तं च॒रुं पौ॒ष्णं च॒रुं मा॑रु॒तꣳ स॒प्तक॑पालं ॅवैश्वदे॒वी-मा॒मिक्षां᳚ द्यावापृथि॒व्य॑-मेक॑कपालं ॥ \newline

\textbf{Pada Paata} \newline

आ॒ग्ने॒यम् । अ॒ष्टाक॑पाल॒मित्य॒ष्टा-क॒पा॒ल॒म् । निरिति॑ । व॒प॒ति॒ । सौ॒म्यम् । च॒रुम् । सा॒वि॒त्रम् । द्वाद॑शकपाल॒मिति॒ द्वाद॑श-क॒पा॒ल॒म् । सा॒र॒स्व॒तम् । च॒रुम् । पौ॒ष्णम् । च॒रुम् । मा॒रु॒तम् । स॒प्तक॑पाल॒मिति॑ स॒प्त -क॒पा॒ल॒म् । वै॒श्व॒दे॒वीमिति॑ वैश्व - दे॒वीम् । आ॒मिक्षा᳚म् । द्या॒वा॒पृ॒थि॒व्य॑मिति॑ द्यावा-पृ॒थि॒व्य᳚म् । एक॑कपाल॒मित्येक॑-क॒पा॒ल॒म् ॥  \newline



\textbf{Jatai Paata} \newline

1. आ॒ग्ने॒य म॒ष्टाक॑पाल म॒ष्टाक॑पाल माग्ने॒य मा᳚ग्ने॒य म॒ष्टाक॑पालम् । \newline
2. अ॒ष्टाक॑पाल॒म् निर् णिर॒ष्टाक॑पाल म॒ष्टाक॑पाल॒म् निः । \newline
3. अ॒ष्टाक॑पाल॒मित्य॒ष्टा - क॒पा॒ल॒म् । \newline
4. निर् व॑पति वपति॒ निर् णिर् व॑पति । \newline
5. व॒प॒ति॒ सौ॒म्यꣳ सौ॒म्यं ॅव॑पति वपति सौ॒म्यम् । \newline
6. सौ॒म्यम् च॒रुम् च॒रुꣳ सौ॒म्यꣳ सौ॒म्यम् च॒रुम् । \newline
7. च॒रुꣳ सा॑वि॒त्रꣳ सा॑वि॒त्रम् च॒रुम् च॒रुꣳ सा॑वि॒त्रम् । \newline
8. सा॒वि॒त्रम् द्वाद॑शकपाल॒म् द्वाद॑शकपालꣳ सावि॒त्रꣳ सा॑वि॒त्रम् द्वाद॑शकपालम् । \newline
9. द्वाद॑शकपालꣳ सारस्व॒तꣳ सा॑रस्व॒तम् द्वाद॑शकपाल॒म् द्वाद॑शकपालꣳ सारस्व॒तम् । \newline
10. द्वाद॑शकपाल॒मिति॒ द्वाद॑श - क॒पा॒ल॒म् । \newline
11. सा॒र॒स्व॒तम् च॒रुम् च॒रुꣳ सा॑रस्व॒तꣳ सा॑रस्व॒तम् च॒रुम् । \newline
12. च॒रुम् पौ॒ष्णम् पौ॒ष्णम् च॒रुम् च॒रुम् पौ॒ष्णम् । \newline
13. पौ॒ष्णम् च॒रुम् च॒रुम् पौ॒ष्णम् पौ॒ष्णम् च॒रुम् । \newline
14. च॒रुम् मा॑रु॒तम् मा॑रु॒तम् च॒रुम् च॒रुम् मा॑रु॒तम् । \newline
15. मा॒रु॒तꣳ स॒प्तक॑पालꣳ स॒प्तक॑पालम् मारु॒तम् मा॑रु॒तꣳ स॒प्तक॑पालम् । \newline
16. स॒प्तक॑पालं ॅवैश्वदे॒वीं ॅवै᳚श्वदे॒वीꣳ स॒प्तक॑पालꣳ स॒प्तक॑पालं ॅवैश्वदे॒वीम् । \newline
17. स॒प्तक॑पाल॒मिति॑ स॒प्त - क॒पा॒ल॒म् । \newline
18. वै॒श्व॒दे॒वी मा॒मिक्षा॑ मा॒मिक्षां᳚ ॅवैश्वदे॒वीं ॅवै᳚श्वदे॒वी मा॒मिक्षा᳚म् । \newline
19. वै॒श्व॒दे॒वीमिति॑ वैश्व - दे॒वीम् । \newline
20. आ॒मिक्षा᳚म् द्यावापृथि॒व्य॑म् द्यावापृथि॒व्य॑ मा॒मिक्षा॑ मा॒मिक्षा᳚म् द्यावापृथि॒व्य᳚म् । \newline
21. द्या॒वा॒पृ॒थि॒व्य॑ मेक॑कपाल॒ मेक॑कपालम् द्यावापृथि॒व्य॑म् द्यावापृथि॒व्य॑ मेक॑कपालम् । \newline
22. द्या॒वा॒पृ॒थि॒व्य॑मिति॑ द्यावा - पृ॒थि॒व्य᳚म् । \newline
23. एक॑कपाल॒मित्येक॑ - क॒पा॒ल॒म् । \newline

\textbf{Ghana Paata } \newline

1. आ॒ग्ने॒य म॒ष्टाक॑पाल म॒ष्टाक॑पाल माग्ने॒य मा᳚ग्ने॒य म॒ष्टाक॑पाल॒म् निर् णिर॒ष्टाक॑पाल माग्ने॒य मा᳚ग्ने॒य म॒ष्टाक॑पाल॒म् निः । \newline
2. अ॒ष्टाक॑पाल॒म् निर् णिर॒ष्टाक॑पाल म॒ष्टाक॑पाल॒म् निर् व॑पति वपति॒ निर॒ष्टाक॑पाल म॒ष्टाक॑पाल॒म् निर् व॑पति । \newline
3. अ॒ष्टाक॑पाल॒मित्य॒ष्टा - क॒पा॒ल॒म् । \newline
4. निर् व॑पति वपति॒ निर् णिर् व॑पति सौ॒म्यꣳ सौ॒म्यं ॅव॑पति॒ निर् णिर् व॑पति सौ॒म्यम् । \newline
5. व॒प॒ति॒ सौ॒म्यꣳ सौ॒म्यं ॅव॑पति वपति सौ॒म्यम् च॒रुम् च॒रुꣳ सौ॒म्यं ॅव॑पति वपति सौ॒म्यम् च॒रुम् । \newline
6. सौ॒म्यम् च॒रुम् च॒रुꣳ सौ॒म्यꣳ सौ॒म्यम् च॒रुꣳ सा॑वि॒त्रꣳ सा॑वि॒त्रम् च॒रुꣳ सौ॒म्यꣳ सौ॒म्यम् च॒रुꣳ सा॑वि॒त्रम् । \newline
7. च॒रुꣳ सा॑वि॒त्रꣳ सा॑वि॒त्रम् च॒रुम् च॒रुꣳ सा॑वि॒त्रम् द्वाद॑शकपाल॒म् द्वाद॑शकपालꣳ सावि॒त्रम् च॒रुम् च॒रुꣳ सा॑वि॒त्रम् द्वाद॑शकपालम् । \newline
8. सा॒वि॒त्रम् द्वाद॑शकपाल॒म् द्वाद॑शकपालꣳ सावि॒त्रꣳ सा॑वि॒त्रम् द्वाद॑शकपालꣳ सारस्व॒तꣳ सा॑रस्व॒तम् द्वाद॑शकपालꣳ सावि॒त्रꣳ सा॑वि॒त्रम् द्वाद॑शकपालꣳ सारस्व॒तम् । \newline
9. द्वाद॑शकपालꣳ सारस्व॒तꣳ सा॑रस्व॒तम् द्वाद॑शकपाल॒म् द्वाद॑शकपालꣳ सारस्व॒तम् च॒रुम् च॒रुꣳ सा॑रस्व॒तम् द्वाद॑शकपाल॒म् द्वाद॑शकपालꣳ सारस्व॒तम् च॒रुम् । \newline
10. द्वाद॑शकपाल॒मिति॒ द्वाद॑श - क॒पा॒ल॒म् । \newline
11. सा॒र॒स्व॒तम् च॒रुम् च॒रुꣳ सा॑रस्व॒तꣳ सा॑रस्व॒तम् च॒रुम् पौ॒ष्णम् पौ॒ष्णम् च॒रुꣳ सा॑रस्व॒तꣳ सा॑रस्व॒तम् च॒रुम् पौ॒ष्णम् । \newline
12. च॒रुम् पौ॒ष्णम् पौ॒ष्णम् च॒रुम् च॒रुम् पौ॒ष्णम् च॒रुम् च॒रुम् पौ॒ष्णम् च॒रुम् च॒रुम् पौ॒ष्णम् च॒रुम् । \newline
13. पौ॒ष्णम् च॒रुम् च॒रुम् पौ॒ष्णम् पौ॒ष्णम् च॒रुम् मा॑रु॒तम् मा॑रु॒तम् च॒रुम् पौ॒ष्णम् पौ॒ष्णम् च॒रुम् मा॑रु॒तम् । \newline
14. च॒रुम् मा॑रु॒तम् मा॑रु॒तम् च॒रुम् च॒रुम् मा॑रु॒तꣳ स॒प्तक॑पालꣳ स॒प्तक॑पालम् मारु॒तम् च॒रुम् च॒रुम् मा॑रु॒तꣳ स॒प्तक॑पालम् । \newline
15. मा॒रु॒तꣳ स॒प्तक॑पालꣳ स॒प्तक॑पालम् मारु॒तम् मा॑रु॒तꣳ स॒प्तक॑पालं ॅवैश्वदे॒वीं ॅवै᳚श्वदे॒वीꣳ स॒प्तक॑पालम् मारु॒तम् मा॑रु॒तꣳ स॒प्तक॑पालं ॅवैश्वदे॒वीम् । \newline
16. स॒प्तक॑पालं ॅवैश्वदे॒वीं ॅवै᳚श्वदे॒वीꣳ स॒प्तक॑पालꣳ स॒प्तक॑पालं ॅवैश्वदे॒वी मा॒मिक्षा॑ मा॒मिक्षां᳚ ॅवैश्वदे॒वीꣳ स॒प्तक॑पालꣳ स॒प्तक॑पालं ॅवैश्वदे॒वी मा॒मिक्षा᳚म् । \newline
17. स॒प्तक॑पाल॒मिति॑ स॒प्त - क॒पा॒ल॒म् । \newline
18. वै॒श्व॒दे॒वी मा॒मिक्षा॑ मा॒मिक्षां᳚ ॅवैश्वदे॒वीं ॅवै᳚श्वदे॒वी मा॒मिक्षा᳚म् द्यावापृथि॒व्य॑म् द्यावापृथि॒व्य॑ मा॒मिक्षां᳚ ॅवैश्वदे॒वीं ॅवै᳚श्वदे॒वी मा॒मिक्षा᳚म् द्यावापृथि॒व्य᳚म् । \newline
19. वै॒श्व॒दे॒वीमिति॑ वैश्व - दे॒वीम् । \newline
20. आ॒मिक्षा᳚म् द्यावापृथि॒व्य॑म् द्यावापृथि॒व्य॑ मा॒मिक्षा॑ मा॒मिक्षा᳚म् द्यावापृथि॒व्य॑ मेक॑कपाल॒ मेक॑कपालम् द्यावापृथि॒व्य॑ मा॒मिक्षा॑ मा॒मिक्षा᳚म् द्यावापृथि॒व्य॑ मेक॑कपालम् । \newline
21. द्या॒वा॒पृ॒थि॒व्य॑ मेक॑कपाल॒ मेक॑कपालम् द्यावापृथि॒व्य॑म् द्यावापृथि॒व्य॑ मेक॑कपालम् । \newline
22. द्या॒वा॒पृ॒थि॒व्य॑मिति॑ द्यावा - पृ॒थि॒व्य᳚म् । \newline
23. एक॑कपाल॒मित्येक॑ - क॒पा॒ल॒म् । \newline
\pagebreak
\markright{ TS 1.8.3.1  \hfill https://www.vedavms.in \hfill}
\addcontentsline{toc}{section}{ TS 1.8.3.1 }
\section*{ TS 1.8.3.1 }

\textbf{TS 1.8.3.1 } \newline
\textbf{Samhita Paata} \newline

ऐ॒न्द्रा॒ग्न-मेका॑दशकपालं मारु॒ती-मा॒मिक्षां᳚ ॅवारु॒णी-मा॒मिक्षां᳚ का॒यमेक॑कपालं प्रघा॒स्यान्॑. हवामहे म॒रुतो॑ य॒ज्ञ्वा॑हसः करं॒भेण॑ स॒जोष॑सः ॥ मो षू ण॑ इन्द्र पृ॒थ्सु दे॒वास्तु॑ स्म ते शुष्मिन्नव॒या । म॒ही ह्य॑स्य मी॒ढुषो॑ य॒व्या । ह॒विष्म॑तो म॒रुतो॒ वन्द॑ते॒ गीः ॥ यद् ग्रामे॒ यदर॑ण्ये॒ यथ् स॒भायां॒ ॅयदि॑न्द्रि॒ये । यच्छू॒द्रे यद॒र्य॑ एन॑श्चकृ॒मा व॒यं । यदे ( ) क॒स्याधि॒ धर्म॑णि॒ तस्या॑व॒यज॑नमसि॒ स्वाहा᳚ ॥ अक्र॒न् कर्म॑ कर्म॒कृतः॑ स॒ह वा॒चा म॑योभु॒वा । दे॒वेभ्यः॒ कर्म॑ कृ॒त्वाऽस्तं॒ प्रेत॑ सुदानवः ॥ \newline

\textbf{Pada Paata} \newline

ऐ॒न्द्रा॒ग्नमित्यै᳚न्द्र - अ॒ग्नम् । एका॑दशकपाल॒मित्येका॑दश-क॒पा॒ल॒म् । मा॒रु॒तीम् । आ॒मिक्षा᳚म् । वा॒रु॒णीम् । आ॒मिक्षा᳚म् । का॒यम् । एक॑कपाल॒मित्येक॑-क॒पा॒लं॒ । प्र॒घा॒स्या॑निति॑ प्र-घा॒स्यान्॑ । ह॒वा॒म॒हे॒ । म॒रुतः॑ । य॒ज्ञ्वा॑हस॒ इति॑ य॒ज्ञ् - वा॒ह॒सः॒ । क॒र॒भेंण॑ । स॒जोष॑स॒ इति॑ स - जोष॑सः ॥ मो इति॑ । स्विति॑ । नः॒ । इ॒न्द्र॒ । पृ॒थ्स्विति॑ पृत्-सु । दे॒व॒ । अस्तु॑ । स्म॒ । ते॒ । शु॒ष्मि॒न्न् । अ॒व॒या ॥ म॒ही । हि । अ॒स्य॒ । मी॒ढुषः॑ । य॒व्या । ह॒विष्म॑तः । म॒रुतः॑ । वन्द॑ते । गीः ॥ यत् । ग्रामे᳚ । यत् । अर॑ण्ये । यत् । स॒भाया᳚म् । यत् । इ॒न्द्रि॒ये ॥ यत् । शू॒द्रे । यत् । अ॒र्ये᳚ । एनः॑ । च॒कृ॒म । व॒यम् ॥ यत् ( ) । एक॑स्य । अधीति॑ । धर्म॑णि । तस्य॑ । अ॒व॒यज॑न॒मित्य॑व-यज॑नम् । अ॒सि॒ । स्वाहा᳚ ॥ अक्रन्न्॑ । कर्म॑ । क॒र्म॒कृत॒ इति॑ कर्म-कृतः॑ । स॒ह । वा॒चा । म॒यो॒भु॒वेति॑ मयः-भु॒वा ॥ दे॒वेभ्यः॑ । कर्म॑ । कृ॒त्वा । अस्त᳚म् । प्रेति॑ । इ॒त॒ । सु॒दा॒न॒व॒ इति॑ सु-दा॒न॒वः॒ ॥  \newline



\textbf{Jatai Paata} \newline

1. ऐ॒न्द्रा॒ग्न मेका॑दशकपाल॒ मेका॑दशकपाल मैन्द्रा॒ग्न मै᳚न्द्रा॒ग्न मेका॑दशकपालम् । \newline
2. ऐ॒न्द्रा॒ग्नमित्यै᳚न्द्र - अ॒ग्नम् । \newline
3. एका॑दशकपालम् मारु॒तीम् मा॑रु॒ती मेका॑दशकपाल॒ मेका॑दशकपालम् मारु॒तीम् । \newline
4. एका॑दशकपाल॒मित्येका॑दश - क॒पा॒ल॒म् । \newline
5. मा॒रु॒ती मा॒मिक्षा॑ मा॒मिक्षा᳚म् मारु॒तीम् मा॑रु॒ती मा॒मिक्षा᳚म् । \newline
6. आ॒मिक्षां᳚ ॅवारु॒णीं ॅवा॑रु॒णी मा॒मिक्षा॑ मा॒मिक्षां᳚ ॅवारु॒णीम् । \newline
7. वा॒रु॒णी मा॒मिक्षा॑ मा॒मिक्षां᳚ ॅवारु॒णीं ॅवा॑रु॒णी मा॒मिक्षा᳚म् । \newline
8. आ॒मिक्षा᳚म् का॒यम् का॒य मा॒मिक्षा॑ मा॒मिक्षा᳚म् का॒यम् । \newline
9. का॒य मेक॑कपाल॒ मेक॑कपालम् का॒यम् का॒य मेक॑कपालम् । \newline
10. एक॑कपालम् प्रघा॒स्या᳚न् प्रघा॒स्या॒ नेक॑कपाल॒ मेक॑कपालम् प्रघा॒स्यान्॑ । \newline
11. एक॑कपाल॒मित्येक॑ - क॒पा॒ल॒म् । \newline
12. प्र॒घा॒स्यान्॑. हवामहे हवामहे प्रघा॒स्या᳚न् प्रघा॒स्यान्॑. हवामहे । \newline
13. प्र॒घा॒स्या॑निति॑ प्र - घा॒स्यान्॑ । \newline
14. ह॒वा॒म॒हे॒ म॒रुतो॑ म॒रुतो॑ हवामहे हवामहे म॒रुतः॑ । \newline
15. म॒रुतो॑ य॒ज्ञ्वा॑हसो य॒ज्ञ्वा॑हसो म॒रुतो॑ म॒रुतो॑ य॒ज्ञ्वा॑हसः । \newline
16. य॒ज्ञ्वा॑हसः करं॒भेण॑ करं॒भेण॑ य॒ज्ञ्वा॑हसो य॒ज्ञ्वा॑हसः करं॒भेण॑ । \newline
17. य॒ज्ञ्वा॑हस॒ इति॑ य॒ज्ञ् - वा॒ह॒सः॒ । \newline
18. क॒रं॒भेण॑ स॒जोष॑सः स॒जोष॑सः करं॒भेण॑ करं॒भेण॑ स॒जोष॑सः । \newline
19. स॒जोष॑स॒ इति॑ स - जोष॑सः । \newline
20. मो षू णो॑ नः॒ सु मो मो षू णः॑ । \newline
21. मो इति॒ मो । \newline
22. सु नो॑ नः॒ सु सु नः॑ । \newline
23. न॒ इ॒न्द्रे॒ न्द्र॒ नो॒ न॒ इ॒न्द्र॒ । \newline
24. इ॒न्द्र॒ पृ॒थ्सु पृ॒थ्स्वि॑न्द्रे न्द्र पृ॒थ्सु । \newline
25. पृ॒थ्सु दे॑व देव पृ॒थ्सु पृ॒थ्सु दे॑व । \newline
26. पृ॒थ्स्विति॑ पृत् - सु । \newline
27. दे॒वा स्त्व स्तु॑ देव दे॒वा स्तु॑ । \newline
28. अस्तु॑ स्म॒ स्मा स्त्व स्तु॑ स्म । \newline
29. स्म॒ ते॒ ते॒ स्म॒ स्म॒ ते॒ । \newline
30. ते॒ शु॒ष्मि॒ञ् छु॒ष्मि॒न् ते॒ ते॒ शु॒ष्मि॒न्न् । \newline
31. शु॒ष्मि॒न् न॒व॒या ऽव॒या शु॑ष्मिञ् छुष्मिन् नव॒या । \newline
32. अ॒व॒येत्य॑व॒या । \newline
33. म॒ही हि हि म॒ही म॒ही हि । \newline
34. ह्य॑स्यास्य॒ हि ह्य॑स्य । \newline
35. अ॒स्य॒ मी॒ढुषो॑ मी॒ढुषो᳚ ऽस्यास्य मी॒ढुषः॑ । \newline
36. मी॒ढुषो॑ य॒व्या य॒व्या मी॒ढुषो॑ मी॒ढुषो॑ य॒व्या । \newline
37. य॒व्येति॑ य॒व्या । \newline
38. ह॒विष्म॑तो म॒रुतो॑ म॒रुतो॑ ह॒विष्म॑तो ह॒विष्म॑तो म॒रुतः॑ । \newline
39. म॒रुतो॒ वन्द॑ते॒ वन्द॑ते म॒रुतो॑ म॒रुतो॒ वन्द॑ते । \newline
40. वन्द॑ते॒ गीर् गीर् वन्द॑ते॒ वन्द॑ते॒ गीः । \newline
41. गीरिति॒ गीः । \newline
42. यद् ग्रामे॒ ग्रामे॒ यद् यद् ग्रामे᳚ । \newline
43. ग्रामे॒ यद् यद् ग्रामे॒ ग्रामे॒ यत् । \newline
44. यदर॒ण्ये ऽर॑ण्ये॒ यद् यदर॑ण्ये । \newline
45. अर॑ण्ये॒ यद् यदर॒ण्ये ऽर॑ण्ये॒ यत् । \newline
46. यथ् स॒भायाꣳ॑ स॒भायां॒ ॅयद् यथ् स॒भाया᳚म् । \newline
47. स॒भायां॒ ॅयद् यथ् स॒भायाꣳ॑ स॒भायां॒ ॅयत् । \newline
48. यदि॑न्द्रि॒य इ॑न्द्रि॒ये यद् यदि॑न्द्रि॒ये । \newline
49. इ॒न्द्रि॒य इती᳚न्द्रि॒ये । \newline
50. यच् छू॒द्रे शू॒द्रे यद् यच् छू॒द्रे । \newline
51. शू॒द्रे यद् यच् छू॒द्रे शू॒द्रे यत् । \newline
52. यद॒र्ये᳚(1॒)ऽर्ये॑ यद् यद॒र्ये᳚ । \newline
53. अ॒र्य॑ एन॒ एनो॒ ऽर्ये᳚(1॒)ऽर्य॑ एनः॑ । \newline
54. एन॑ श्चकृ॒म च॑कृ॒मैन॒ एन॑ श्चकृ॒म । \newline
55. च॒कृ॒मा व॒यं ॅव॒यम् च॑कृ॒म च॑कृ॒मा व॒यम् । \newline
56. व॒यमिति॑ व॒यम् । \newline
57. यदेक॒ स्यैक॑स्य॒ यद् यदेक॑स्य । \newline
58. एक॒स्या ध्य ध्येक॒ स्यैक॒स्या धि॑ । \newline
59. अधि॒ धर्म॑णि॒ धर्म॒ ण्यध्य धि॒ धर्म॑णि । \newline
60. धर्म॑णि॒ तस्य॒ तस्य॒ धर्म॑णि॒ धर्म॑णि॒ तस्य॑ । \newline
61. तस्या॑ व॒यज॑न मव॒यज॑न॒म् तस्य॒ तस्या॑ व॒यज॑नम् । \newline
62. अ॒व॒यज॑न मस्यस्य व॒यज॑न मव॒यज॑न मसि । \newline
63. अ॒व॒यज॑न॒मित्य॑व - यज॑नम् । \newline
64. अ॒सि॒ स्वाहा॒ स्वाहा᳚ ऽस्यसि॒ स्वाहा᳚ । \newline
65. स्वाहेति॒ स्वाहा᳚ । \newline
66. अक्र॒न् कर्म॒ कर्माक्र॒न् नक्र॒न् कर्म॑ । \newline
67. कर्म॑ कर्म॒कृतः॑ कर्म॒कृतः॒ कर्म॒ कर्म॑ कर्म॒कृतः॑ । \newline
68. क॒र्म॒कृतः॑ स॒ह स॒ह क॑र्म॒कृतः॑ कर्म॒कृतः॑ स॒ह । \newline
69. क॒र्म॒कृत॒ इति॑ कर्म - कृतः॑ । \newline
70. स॒ह वा॒चा वा॒चा स॒ह स॒ह वा॒चा । \newline
71. वा॒चा म॑योभु॒वा म॑योभु॒वा वा॒चा वा॒चा म॑योभु॒वा । \newline
72. म॒यो॒भु॒वेति॑ मयः - भु॒वा । \newline
73. दे॒वेभ्यः॒ कर्म॒ कर्म॑ दे॒वेभ्यो॑ दे॒वेभ्यः॒ कर्म॑ । \newline
74. कर्म॑ कृ॒त्वा कृ॒त्वा कर्म॒ कर्म॑ कृ॒त्वा । \newline
75. कृ॒त्वा ऽस्त॒ मस्त॑म् कृ॒त्वा कृ॒त्वा ऽस्त᳚म् । \newline
76. अस्त॒म् प्र प्रास्त॒ मस्त॒म् प्र । \newline
77. प्रे ते॑ त॒ प्र प्रे त॑ । \newline
78. इ॒त॒ सु॒दा॒न॒वः॒ सु॒दा॒न॒व॒ इ॒ते॒ त॒ सु॒दा॒न॒वः॒ । \newline
79. सु॒दा॒न॒व॒ इति॑ सु - दा॒न॒वः॒ । \newline

\textbf{Ghana Paata } \newline

1. ऐ॒न्द्रा॒ग्न मेका॑दशकपाल॒ मेका॑दशकपाल मैन्द्रा॒ग्न मै᳚न्द्रा॒ग्न मेका॑दशकपालम् मारु॒तीम् मा॑रु॒ती मेका॑दशकपाल मैन्द्रा॒ग्न मै᳚न्द्रा॒ग्न मेका॑दशकपालम् मारु॒तीम् । \newline
2. ऐ॒न्द्रा॒ग्नमित्यै᳚न्द्र - अ॒ग्नम् । \newline
3. एका॑दशकपालम् मारु॒तीम् मा॑रु॒ती मेका॑दशकपाल॒ मेका॑दशकपालम् मारु॒ती मा॒मिक्षा॑ मा॒मिक्षा᳚म् मारु॒ती मेका॑दशकपाल॒ मेका॑दशकपालम् मारु॒ती मा॒मिक्षा᳚म् । \newline
4. एका॑दशकपाल॒मित्येका॑दश - क॒पा॒ल॒म् । \newline
5. मा॒रु॒ती मा॒मिक्षा॑ मा॒मिक्षा᳚म् मारु॒तीम् मा॑रु॒ती मा॒मिक्षां᳚ ॅवारु॒णीं ॅवा॑रु॒णी मा॒मिक्षा᳚म् मारु॒तीम् मा॑रु॒ती मा॒मिक्षां᳚ ॅवारु॒णीम् । \newline
6. आ॒मिक्षां᳚ ॅवारु॒णीं ॅवा॑रु॒णी मा॒मिक्षा॑ मा॒मिक्षां᳚ ॅवारु॒णी मा॒मिक्षा॑ मा॒मिक्षां᳚ ॅवारु॒णी मा॒मिक्षा॑ मा॒मिक्षां᳚ ॅवारु॒णी मा॒मिक्षा᳚म् । \newline
7. वा॒रु॒णी मा॒मिक्षा॑ मा॒मिक्षां᳚ ॅवारु॒णीं ॅवा॑रु॒णी मा॒मिक्षा᳚म् का॒यम् का॒य मा॒मिक्षां᳚ ॅवारु॒णीं ॅवा॑रु॒णी मा॒मिक्षा᳚म् का॒यम् । \newline
8. आ॒मिक्षा᳚म् का॒यम् का॒य मा॒मिक्षा॑ मा॒मिक्षा᳚म् का॒य मेक॑कपाल॒ मेक॑कपालम् का॒य मा॒मिक्षा॑ मा॒मिक्षा᳚म् का॒य मेक॑कपालम् । \newline
9. का॒य मेक॑कपाल॒ मेक॑कपालम् का॒यम् का॒य मेक॑कपालम् प्रघा॒स्या᳚न् प्रघा॒स्या॒ नेक॑कपालम् का॒यम् का॒य मेक॑कपालम् प्रघा॒स्यान्॑ । \newline
10. एक॑कपालम् प्रघा॒स्या᳚न् प्रघा॒स्या॒ नेक॑कपाल॒ मेक॑कपालम् प्रघा॒स्यान्॑. हवामहे हवामहे प्रघा॒स्या॒ नेक॑कपाल॒ मेक॑कपालम् प्रघा॒स्यान्॑. हवामहे । \newline
11. एक॑कपाल॒मित्येक॑ - क॒पा॒ल॒म् । \newline
12. प्र॒घा॒स्यान्॑. हवामहे हवामहे प्रघा॒स्या᳚न् प्रघा॒स्यान्॑. हवामहे म॒रुतो॑ म॒रुतो॑ हवामहे प्रघा॒स्या᳚न् प्रघा॒स्यान्॑. हवामहे म॒रुतः॑ । \newline
13. प्र॒घा॒स्या॑निति॑ प्र - घा॒स्यान्॑ । \newline
14. ह॒वा॒म॒हे॒ म॒रुतो॑ म॒रुतो॑ हवामहे हवामहे म॒रुतो॑ य॒ज्ञ्वा॑हसो य॒ज्ञ्वा॑हसो म॒रुतो॑ हवामहे हवामहे म॒रुतो॑ य॒ज्ञ्वा॑हसः । \newline
15. म॒रुतो॑ य॒ज्ञ्वा॑हसो य॒ज्ञ्वा॑हसो म॒रुतो॑ म॒रुतो॑ य॒ज्ञ्वा॑हसः करं॒भेण॑ करं॒भेण॑ य॒ज्ञ्वा॑हसो म॒रुतो॑ म॒रुतो॑ य॒ज्ञ्वा॑हसः करं॒भेण॑ । \newline
16. य॒ज्ञ्वा॑हसः करं॒भेण॑ करं॒भेण॑ य॒ज्ञ्वा॑हसो य॒ज्ञ्वा॑हसः करं॒भेण॑ स॒जोष॑सः स॒जोष॑सः करं॒भेण॑ य॒ज्ञ्वा॑हसो य॒ज्ञ्वा॑हसः करं॒भेण॑ स॒जोष॑सः । \newline
17. य॒ज्ञ्वा॑हस॒ इति॑ य॒ज्ञ् - वा॒ह॒सः॒ । \newline
18. क॒रं॒भेण॑ स॒जोष॑सः स॒जोष॑सः करं॒भेण॑ करं॒भेण॑ स॒जोष॑सः । \newline
19. स॒जोष॑स॒ इति॑ स - जोष॑सः । \newline
20. मो षू णो॑ नः॒ सु मो मो षू ण॑ इन्द्रे न्द्र नः॒ सु मो मो षू ण॑ इन्द्र । \newline
21. मो इति॒ मो । \newline
22. सु नो॑ नः॒ सु सु न॑ इन्द्रे न्द्र नः॒ सु सु न॑ इन्द्र । \newline
23. न॒ इ॒न्द्रे॒ न्द्र॒ नो॒ न॒ इ॒न्द्र॒ पृ॒थ्सु पृ॒थ्स्वि॑न्द्र नो न इन्द्र पृ॒थ्सु । \newline
24. इ॒न्द्र॒ पृ॒थ्सु पृ॒थ्स्वि॑न्द्रे न्द्र पृ॒थ्सु दे॑व देव पृ॒थ्स्वि॑न्द्रे न्द्र पृ॒थ्सु दे॑व । \newline
25. पृ॒थ्सु दे॑व देव पृ॒थ्सु पृ॒थ्सु दे॒वा स्त्वस्तु॑ देव पृ॒थ्सु पृ॒थ्सु दे॒वास्तु॑ । \newline
26. पृ॒थ्स्विति॑ पृत् - सु । \newline
27. दे॒वा स्त्वस्तु॑ देव दे॒वा स्तु॑ स्म॒ स्मास्तु॑ देव दे॒वा स्तु॑ स्म । \newline
28. अस्तु॑ स्म॒ स्मा स्त्वस्तु॑ स्म ते ते॒ स्मा स्त्वस्तु॑ स्म ते । \newline
29. स्म॒ ते॒ ते॒ स्म॒ स्म॒ ते॒ शु॒ष्मि॒ञ् छु॒ष्मि॒न् ते॒ स्म॒ स्म॒ ते॒ शु॒ष्मि॒न्न् । \newline
30. ते॒ शु॒ष्मि॒ञ् छु॒ष्मि॒न् ते॒ ते॒ शु॒ष्मि॒न् न॒व॒या ऽव॒या शु॑ष्मिन् ते ते शुष्मिन् नव॒या । \newline
31. शु॒ष्मि॒न् न॒व॒या ऽव॒या शु॑ष्मिञ् छुष्मिन् नव॒या । \newline
32. अ॒व॒येत्य॑व॒या । \newline
33. म॒ही हि हि म॒ही म॒ही ह्य॑स्यास्य॒ हि म॒ही म॒ही ह्य॑स्य । \newline
34. ह्य॑स्यास्य॒ हि ह्य॑स्य मी॒ढुषो॑ मी॒ढुषो᳚ ऽस्य॒ हि ह्य॑स्य मी॒ढुषः॑ । \newline
35. अ॒स्य॒ मी॒ढुषो॑ मी॒ढुषो᳚ ऽस्यास्य मी॒ढुषो॑ य॒व्या य॒व्या मी॒ढुषो᳚ ऽस्यास्य मी॒ढुषो॑ य॒व्या । \newline
36. मी॒ढुषो॑ य॒व्या य॒व्या मी॒ढुषो॑ मी॒ढुषो॑ य॒व्या । \newline
37. य॒व्येति॑ य॒व्या । \newline
38. ह॒विष्म॑तो म॒रुतो॑ म॒रुतो॑ ह॒विष्म॑तो ह॒विष्म॑तो म॒रुतो॒ वन्द॑ते॒ वन्द॑ते म॒रुतो॑ ह॒विष्म॑तो ह॒विष्म॑तो म॒रुतो॒ वन्द॑ते । \newline
39. म॒रुतो॒ वन्द॑ते॒ वन्द॑ते म॒रुतो॑ म॒रुतो॒ वन्द॑ते॒ गीर् गीर् वन्द॑ते म॒रुतो॑ म॒रुतो॒ वन्द॑ते॒ गीः । \newline
40. वन्द॑ते॒ गीर् गीर् वन्द॑ते॒ वन्द॑ते॒ गीः । \newline
41. गीरिति॒ गीः । \newline
42. यद् ग्रामे॒ ग्रामे॒ यद् यद् ग्रामे॒ यद् यद् ग्रामे॒ यद् यद् ग्रामे॒ यत् । \newline
43. ग्रामे॒ यद् यद् ग्रामे॒ ग्रामे॒ यदर॒ण्ये ऽर॑ण्ये॒ यद् ग्रामे॒ ग्रामे॒ यदर॑ण्ये । \newline
44. यदर॒ण्ये ऽर॑ण्ये॒ यद् यदर॑ण्ये॒ यद् यदर॑ण्ये॒ यद् यदर॑ण्ये॒ यत् । \newline
45. अर॑ण्ये॒ यद् यदर॒ण्ये ऽर॑ण्ये॒ यथ् स॒भायाꣳ॑ स॒भायां॒ ॅयदर॒ण्ये ऽर॑ण्ये॒ यथ् स॒भाया᳚म् । \newline
46. यथ् स॒भायाꣳ॑ स॒भायां॒ ॅयद् यथ् स॒भायां॒ ॅयद् यथ् स॒भायां॒ ॅयद् यथ् स॒भायां॒ ॅयत् । \newline
47. स॒भायां॒ ॅयद् यथ् स॒भायाꣳ॑ स॒भायां॒ ॅयदि॑न्द्रि॒य इ॑न्द्रि॒ये यथ् स॒भायाꣳ॑ स॒भायां॒ ॅयदि॑न्द्रि॒ये । \newline
48. यदि॑न्द्रि॒य इ॑न्द्रि॒ये यद् यदि॑न्द्रि॒ये । \newline
49. इ॒न्द्रि॒य इती᳚न्द्रि॒ये । \newline
50. यच्छू॒द्रे शू॒द्रे यद् यच्छू॒द्रे यद् यच्छू॒द्रे यद् यच्छू॒द्रे यत् । \newline
51. शू॒द्रे यद् यच्छू॒द्रे शू॒द्रे यद॒र्ये᳚(1॒) ऽर्ये॑ यच्छू॒द्रे शू॒द्रे यद॒र्ये᳚ । \newline
52. यद॒र्ये᳚(1॒)ऽर्ये॑ यद् यद॒र्य॑ एन॒ एनो॒ ऽर्ये॑ यद् यद॒र्य॑ एनः॑ । \newline
53. अ॒र्य॑ एन॒ एनो॒ ऽर्ये᳚(1॒)ऽर्य॑ एन॑ श्चकृ॒म च॑कृ॒मैनो॒ ऽर्ये᳚(1॒)ऽर्य॑ एन॑ श्चकृ॒म । \newline
54. एन॑ श्चकृ॒म च॑कृ॒मैन॒ एन॑ श्चकृ॒मा व॒यं ॅव॒यम् च॑कृ॒मैन॒ एन॑ श्चकृ॒मा व॒यम् । \newline
55. च॒कृ॒मा व॒यं ॅव॒यम् च॑कृ॒म च॑कृ॒मा व॒यम् । \newline
56. व॒यमिति॑ व॒यम् । \newline
57. यदेक॒ स्यैक॑स्य॒ यद् यदेक॒स्याध्य ध्येक॑स्य॒ यद् यदेक॒स्याधि॑ । \newline
58. एक॒स्या ध्यध्येक॒ स्यैक॒ स्याधि॒ धर्म॑णि॒ धर्म॒ण्यध्येक॒ स्यैक॒स्याधि॒ धर्म॑णि । \newline
59. अधि॒ धर्म॑णि॒ धर्म॒ ण्यध्यधि॒ धर्म॑णि॒ तस्य॒ तस्य॒ धर्म॒ ण्यध्यधि॒ धर्म॑णि॒ तस्य॑ । \newline
60. धर्म॑णि॒ तस्य॒ तस्य॒ धर्म॑णि॒ धर्म॑णि॒ तस्या॑ व॒यज॑न मव॒यज॑न॒म् तस्य॒ धर्म॑णि॒ धर्म॑णि॒ तस्या॑ व॒यज॑नम् । \newline
61. तस्या॑ व॒यज॑न मव॒यज॑न॒म् तस्य॒ तस्या॑ व॒यज॑न मस्यस्य व॒यज॑न॒म् तस्य॒ तस्या॑ व॒यज॑न मसि । \newline
62. अ॒व॒यज॑न मस्यस्य व॒यज॑न मव॒यज॑न मसि॒ स्वाहा॒ स्वाहा᳚ ऽस्यव॒यज॑न मव॒यज॑न मसि॒ स्वाहा᳚ । \newline
63. अ॒व॒यज॑न॒मित्य॑व - यज॑नम् । \newline
64. अ॒सि॒ स्वाहा॒ स्वाहा᳚ ऽस्यसि॒ स्वाहा᳚ । \newline
65. स्वाहेति॒ स्वाहा᳚ । \newline
66. अक्र॒न् कर्म॒ कर्माक्र॒न् नक्र॒न् कर्म॑ कर्म॒कृतः॑ कर्म॒कृतः॒ कर्माक्र॒न् नक्र॒न् कर्म॑ कर्म॒कृतः॑ । \newline
67. कर्म॑ कर्म॒कृतः॑ कर्म॒कृतः॒ कर्म॒ कर्म॑ कर्म॒कृतः॑ स॒ह स॒ह क॑र्म॒कृतः॒ कर्म॒ कर्म॑ कर्म॒कृतः॑ स॒ह । \newline
68. क॒र्म॒कृतः॑ स॒ह स॒ह क॑र्म॒कृतः॑ कर्म॒कृतः॑ स॒ह वा॒चा वा॒चा स॒ह क॑र्म॒कृतः॑ कर्म॒कृतः॑ स॒ह वा॒चा । \newline
69. क॒र्म॒कृत॒ इति॑ कर्म - कृतः॑ । \newline
70. स॒ह वा॒चा वा॒चा स॒ह स॒ह वा॒चा म॑योभु॒वा म॑योभु॒वा वा॒चा स॒ह स॒ह वा॒चा म॑योभु॒वा । \newline
71. वा॒चा म॑योभु॒वा म॑योभु॒वा वा॒चा वा॒चा म॑योभु॒वा । \newline
72. म॒यो॒भु॒वेति॑ मयः - भु॒वा । \newline
73. दे॒वेभ्यः॒ कर्म॒ कर्म॑ दे॒वेभ्यो॑ दे॒वेभ्यः॒ कर्म॑ कृ॒त्वा कृ॒त्वा कर्म॑ दे॒वेभ्यो॑ दे॒वेभ्यः॒ कर्म॑ कृ॒त्वा । \newline
74. कर्म॑ कृ॒त्वा कृ॒त्वा कर्म॒ कर्म॑ कृ॒त्वा ऽस्त॒ मस्त॑म् कृ॒त्वा कर्म॒ कर्म॑ कृ॒त्वा ऽस्त᳚म् । \newline
75. कृ॒त्वा ऽस्त॒ मस्त॑म् कृ॒त्वा कृ॒त्वा ऽस्त॒म् प्र प्रास्त॑म् कृ॒त्वा कृ॒त्वा ऽस्त॒म् प्र । \newline
76. अस्त॒म् प्र प्रास्त॒ मस्त॒म् प्रे ते॑ त॒ प्रास्त॒ मस्त॒म् प्रे त॑ । \newline
77. प्रे ते॑ त॒ प्र प्रे त॑ सुदानवः सुदानव इत॒ प्र प्रे त॑ सुदानवः । \newline
78. इ॒त॒ सु॒दा॒न॒वः॒ सु॒दा॒न॒व॒ इ॒ते॒ त॒ सु॒दा॒न॒वः॒ । \newline
79. सु॒दा॒न॒व॒ इति॑ सु - दा॒न॒वः॒ । \newline
\pagebreak
\markright{ TS 1.8.4.1  \hfill https://www.vedavms.in \hfill}
\addcontentsline{toc}{section}{ TS 1.8.4.1 }
\section*{ TS 1.8.4.1 }

\textbf{TS 1.8.4.1 } \newline
\textbf{Samhita Paata} \newline

अ॒ग्नयेऽनी॑कवते पुरो॒डाश॑-म॒ष्टाक॑पालं॒ निर्व॑पति सा॒कꣳ सूर्ये॑णोद्य॒ता म॒रुद्भ्यः॑ सान्तप॒नेभ्यो॑ म॒द्ध्यन्दि॑ने च॒रुं म॒रुद्भ्यो॑ गृहमे॒धिभ्यः॒ सर्वा॑सां दु॒ग्धे सा॒यं च॒रुं पू॒र्णा द॑र्वि॒ परा॑पत॒ सुपू᳚र्णा॒ पुन॒रा प॑त । व॒स्नेव॒ वि क्री॑णावहा॒ इष॒मूर्जꣳ॑ शतक्रतो ॥ दे॒हि मे॒ ददा॑मि ते॒ नि मे॑ धेहि॒ नि ते॑ दधे । नि॒हार॒मिन्नि मे॑ हरा नि॒ हारं॒- [ ] \newline

\textbf{Pada Paata} \newline

अ॒ग्नये᳚ । अनी॑कवत॒ इत्यनी॑क - व॒ते॒ । पु॒रो॒डाश᳚म् । अ॒ष्टाक॑पाल॒मित्य॒ष्टा-क॒पा॒ल॒म् । निरिति॑ । व॒प॒ति॒ । सा॒कम् । सूर्ये॑ण । उ॒द्य॒तेत्यु॑त् - य॒ता । म॒रुद्भ्य॒ इति॑ म॒रुत् - भ्यः॒ । सा॒न्त॒प॒नेभ्य॒ इति॑ सां - त॒प॒नेभ्यः॑ । म॒द्ध्यन्दि॑ने । च॒रुम् । म॒रुद्भ्य॒ इति॑ म॒रुत्-भ्यः॒ । गृ॒ह॒मे॒धिभ्य॒ इति॑ गृहमे॒धि-भ्यः॒ । सर्वा॑साम् । दु॒ग्धे । सा॒यम् । च॒रुम् । पू॒र्णा । द॒र्वि॒ । परेति॑ । प॒त॒ । सुपू॒र्णेति॒ सु - पू॒र्णा॒ । पुनः॑ । एति॑ । प॒त॒ ॥ व॒स्ना । इ॒व॒ । वीति॑ । क्री॒णा॒व॒है॒ । इष᳚म् । ऊर्ज᳚म् । श॒त॒क्र॒तो॒ इति॑ शत-क्र॒तो॒ ॥ दे॒हि । मे॒ । ददा॑मि । ते॒ । नीति॑ । मे॒ । धे॒हि॒ । नीति॑ । ते॒ । द॒धे॒ ॥ नि॒हार॒मिति॑ नि-हार᳚म् । इत् । नीति॑ । मे॒ । ह॒र॒ । नि॒हार॒मिति॑ नि-हार᳚म् ।  \newline



\textbf{Jatai Paata} \newline

1. अ॒ग्नये ऽनी॑कव॒ते ऽनी॑कवते॒ ऽग्नये॒ ऽग्नये ऽनी॑कवते । \newline
2. अनी॑कवते पुरो॒डाश॑म् पुरो॒डाश॒ मनी॑कव॒ते ऽनी॑कवते पुरो॒डाश᳚म् । \newline
3. अनी॑कवत॒ इत्यनी॑क - व॒ते॒ । \newline
4. पु॒रो॒डाश॑ म॒ष्टाक॑पाल म॒ष्टाक॑पालम् पुरो॒डाश॑म् पुरो॒डाश॑ म॒ष्टाक॑पालम् । \newline
5. अ॒ष्टाक॑पाल॒म् निर् णिर॒ष्टाक॑पाल म॒ष्टाक॑पाल॒म् निः । \newline
6. अ॒ष्टाक॑पाल॒मित्य॒ष्टा - क॒पा॒ल॒म् । \newline
7. निर् व॑पति वपति॒ निर् णिर् व॑पति । \newline
8. व॒प॒ति॒ सा॒कꣳ सा॒कं ॅव॑पति वपति सा॒कम् । \newline
9. सा॒कꣳ सूर्ये॑ण॒ सूर्ये॑ण सा॒कꣳ सा॒कꣳ सूर्ये॑ण । \newline
10. सूर्ये॑ णोद्य॒तो द्य॒ता सूर्ये॑ण॒ सूर्ये॑णोद्य॒ता । \newline
11. उ॒द्य॒ता म॒रुद्भ्यो॑ म॒रुद्भ्य॑ उद्य॒तोद्य॒ता म॒रुद्भ्यः॑ । \newline
12. उ॒द्य॒तेत्यु॑त् - य॒ता । \newline
13. म॒रुद्भ्यः॑ सान्तप॒नेभ्यः॑ सान्तप॒नेभ्यो॑ म॒रुद्भ्यो॑ म॒रुद्भ्यः॑ सान्तप॒नेभ्यः॑ । \newline
14. म॒रुद्भ्य॒ इति॑ म॒रुत् - भ्यः॒ । \newline
15. सा॒न्त॒प॒नेभ्यो॑ म॒द्ध्यन्दि॑ने म॒द्ध्यन्दि॑ने सान्तप॒नेभ्यः॑ सान्तप॒नेभ्यो॑ म॒द्ध्यन्दि॑ने । \newline
16. सा॒न्त॒प॒नेभ्य॒ इति॑ सां - त॒प॒नेभ्यः॑ । \newline
17. म॒द्ध्यन्दि॑ने च॒रुम् च॒रुम् म॒द्ध्यन्दि॑ने म॒द्ध्यन्दि॑ने च॒रुम् । \newline
18. च॒रुम् म॒रुद्भ्यो॑ म॒रुद्भ्य॑ श्च॒रुम् च॒रुम् म॒रुद्भ्यः॑ । \newline
19. म॒रुद्भ्यो॑ गृहमे॒धिभ्यो॑ गृहमे॒धिभ्यो॑ म॒रुद्भ्यो॑ म॒रुद्भ्यो॑ गृहमे॒धिभ्यः॑ । \newline
20. म॒रुद्भ्य॒ इति॑ म॒रुत् - भ्यः॒ । \newline
21. गृ॒ह॒मे॒धिभ्यः॒ सर्वा॑साꣳ॒॒ सर्वा॑साम् गृहमे॒धिभ्यो॑ गृहमे॒धिभ्यः॒ सर्वा॑साम् । \newline
22. गृ॒ह॒मे॒धिभ्य॒ इति॑ गृहमे॒धि - भ्यः॒ । \newline
23. सर्वा॑साम् दु॒ग्धे दु॒ग्धे सर्वा॑साꣳ॒॒ सर्वा॑साम् दु॒ग्धे । \newline
24. दु॒ग्धे सा॒यꣳ सा॒यम् दु॒ग्धे दु॒ग्धे सा॒यम् । \newline
25. सा॒यम् च॒रुम् च॒रुꣳ सा॒यꣳ सा॒यम् च॒रुम् । \newline
26. च॒रुम् पू॒र्णा पू॒र्णा च॒रुम् च॒रुम् पू॒र्णा । \newline
27. पू॒र्णा द॑र्वि दर्वि पू॒र्णा पू॒र्णा द॑र्वि । \newline
28. द॒र्वि॒ परा॒ परा॑ दर्वि दर्वि॒ परा᳚ । \newline
29. परा॑ पत पत॒ परा॒ परा॑ पत । \newline
30. प॒त॒ सुपू᳚र्णा॒ सुपू᳚र्णा पत पत॒ सुपू᳚र्णा । \newline
31. सुपू᳚र्णा॒ पुनः॒ पुनः॒ सुपू᳚र्णा॒ सुपू᳚र्णा॒ पुनः॑ । \newline
32. सुपू॒र्णेति॒ सु - पू॒र्णा॒ । \newline
33. पुन॒ रा पुनः॒ पुन॒ रा । \newline
34. आ प॑त प॒ता प॑त । \newline
35. प॒तेति॑ पत । \newline
36. व॒स्नेवे॑ व व॒स्ना व॒स्नेव॑ । \newline
37. इ॒व॒ वि वीवे॑ व॒ वि । \newline
38. वि क्री॑णावहै क्रीणावहै॒ वि वि क्री॑णावहै । \newline
39. क्री॒णा॒व॒हा॒ इष॒ मिष॑म् क्रीणावहै क्रीणावहा॒ इष᳚म् । \newline
40. इष॒ मूर्ज॒ मूर्ज॒ मिष॒ मिष॒ मूर्ज᳚म् । \newline
41. ऊर्जꣳ॑ शतक्रतो शतक्रतो॒ ऊर्ज॒ मूर्जꣳ॑ शतक्रतो । \newline
42. श॒त॒क्र॒तो॒ इति॑ शत - क्र॒तो॒ । \newline
43. दे॒हि मे॑ मे दे॒हि दे॒हि मे᳚ । \newline
44. मे॒ ददा॑मि॒ ददा॑मि मे मे॒ ददा॑मि । \newline
45. ददा॑मि ते ते॒ ददा॑मि॒ ददा॑मि ते । \newline
46. ते॒ नि नि ते॑ ते॒ नि । \newline
47. नि मे॑ मे॒ नि नि मे᳚ । \newline
48. मे॒ धे॒हि॒ धे॒हि॒ मे॒ मे॒ धे॒हि॒ । \newline
49. धे॒हि॒ नि नि धे॑हि धेहि॒ नि । \newline
50. नि ते॑ ते॒ नि नि ते᳚ । \newline
51. ते॒ द॒धे॒ द॒धे॒ ते॒ ते॒ द॒धे॒ । \newline
52. द॒ध॒ इति॑ दधे । \newline
53. नि॒हार॒ मिदिन् नि॒हार॑म् नि॒हार॒ मित् । \newline
54. नि॒हार॒मिति॑ नि - हार᳚म् । \newline
55. इन् नि नीदिन् नि । \newline
56. नि मे॑ मे॒ नि नि मे᳚ । \newline
57. मे॒ ह॒र॒ ह॒र॒ मे॒ मे॒ ह॒र॒ । \newline
58. ह॒रा॒ नि॒हार॑म् नि॒हारꣳ॑ हर हरा नि॒हार᳚म् । \newline
59. नि॒हार॒म् नि नि नि॒हार॑म् नि॒हार॒म् नि । \newline
60. नि॒हार॒मिति॑ नि - हार᳚म् । \newline

\textbf{Ghana Paata } \newline

1. अ॒ग्नये ऽनी॑कव॒ते ऽनी॑कवते॒ ऽग्नये॒ ऽग्नये ऽनी॑कवते पुरो॒डाश॑म् पुरो॒डाश॒ मनी॑कवते॒ ऽग्नये॒ ऽग्नये ऽनी॑कवते पुरो॒डाश᳚म् । \newline
2. अनी॑कवते पुरो॒डाश॑म् पुरो॒डाश॒ मनी॑कव॒ते ऽनी॑कवते पुरो॒डाश॑ म॒ष्टाक॑पाल म॒ष्टाक॑पालम् पुरो॒डाश॒ मनी॑कव॒ते ऽनी॑कवते पुरो॒डाश॑ म॒ष्टाक॑पालम् । \newline
3. अनी॑कवत॒ इत्यनी॑क - व॒ते॒ । \newline
4. पु॒रो॒डाश॑ म॒ष्टाक॑पाल म॒ष्टाक॑पालम् पुरो॒डाश॑म् पुरो॒डाश॑ म॒ष्टाक॑पाल॒म् निर् णिर॒ष्टाक॑पालम् पुरो॒डाश॑म् पुरो॒डाश॑ म॒ष्टाक॑पाल॒म् निः । \newline
5. अ॒ष्टाक॑पाल॒म् निर् णिर॒ष्टाक॑पाल म॒ष्टाक॑पाल॒म् निर् व॑पति वपति॒ निर॒ष्टाक॑पाल म॒ष्टाक॑पाल॒म् निर् व॑पति । \newline
6. अ॒ष्टाक॑पाल॒मित्य॒ष्टा - क॒पा॒ल॒म् । \newline
7. निर् व॑पति वपति॒ निर् णिर् व॑पति सा॒कꣳ सा॒कं ॅव॑पति॒ निर् णिर् व॑पति सा॒कम् । \newline
8. व॒प॒ति॒ सा॒कꣳ सा॒कं ॅव॑पति वपति सा॒कꣳ सूर्ये॑ण॒ सूर्ये॑ण सा॒कं ॅव॑पति वपति सा॒कꣳ सूर्ये॑ण । \newline
9. सा॒कꣳ सूर्ये॑ण॒ सूर्ये॑ण सा॒कꣳ सा॒कꣳ सूर्ये॑णो द्य॒तोद्य॒ता सूर्ये॑ण सा॒कꣳ सा॒कꣳ सूर्ये॑णो द्य॒ता । \newline
10. सूर्ये॑णो द्य॒तोद्य॒ता सूर्ये॑ण॒ सूर्ये॑णोद्य॒ता म॒रुद्भ्यो॑ म॒रुद्भ्य॑ उद्य॒ता सूर्ये॑ण॒ सूर्ये॑णोद्य॒ता म॒रुद्भ्यः॑ । \newline
11. उ॒द्य॒ता म॒रुद्भ्यो॑ म॒रुद्भ्य॑ उद्य॒तो द्य॒ता म॒रुद्भ्यः॑ सान्तप॒नेभ्यः॑ सान्तप॒नेभ्यो॑ म॒रुद्भ्य॑ उद्य॒तो द्य॒ता म॒रुद्भ्यः॑ सान्तप॒नेभ्यः॑ । \newline
12. उ॒द्य॒तेत्यु॑त् - य॒ता । \newline
13. म॒रुद्भ्यः॑ सान्तप॒नेभ्यः॑ सान्तप॒नेभ्यो॑ म॒रुद्भ्यो॑ म॒रुद्भ्यः॑ सान्तप॒नेभ्यो॑ म॒द्ध्यन्दि॑ने म॒द्ध्यन्दि॑ने सान्तप॒नेभ्यो॑ म॒रुद्भ्यो॑ म॒रुद्भ्यः॑ सान्तप॒नेभ्यो॑ म॒द्ध्यन्दि॑ने । \newline
14. म॒रुद्भ्य॒ इति॑ म॒रुत् - भ्यः॒ । \newline
15. सा॒न्त॒प॒नेभ्यो॑ म॒द्ध्यन्दि॑ने म॒द्ध्यन्दि॑ने सान्तप॒नेभ्यः॑ सान्तप॒नेभ्यो॑ म॒द्ध्यन्दि॑ने च॒रुम् च॒रुम् म॒द्ध्यन्दि॑ने सान्तप॒नेभ्यः॑ सान्तप॒नेभ्यो॑ म॒द्ध्यन्दि॑ने च॒रुम् । \newline
16. सा॒न्त॒प॒नेभ्य॒ इति॑ सां - त॒प॒नेभ्यः॑ । \newline
17. म॒द्ध्यन्दि॑ने च॒रुम् च॒रुम् म॒द्ध्यन्दि॑ने म॒द्ध्यन्दि॑ने च॒रुम् म॒रुद्भ्यो॑ म॒रुद्भ्य॑ श्च॒रुम् म॒द्ध्यन्दि॑ने म॒द्ध्यन्दि॑ने च॒रुम् म॒रुद्भ्यः॑ । \newline
18. च॒रुम् म॒रुद्भ्यो॑ म॒रुद्भ्य॑ श्च॒रुम् च॒रुम् म॒रुद्भ्यो॑ गृहमे॒धिभ्यो॑ गृहमे॒धिभ्यो॑ म॒रुद्भ्य॑ श्च॒रुम् च॒रुम् म॒रुद्भ्यो॑ गृहमे॒धिभ्यः॑ । \newline
19. म॒रुद्भ्यो॑ गृहमे॒धिभ्यो॑ गृहमे॒धिभ्यो॑ म॒रुद्भ्यो॑ म॒रुद्भ्यो॑ गृहमे॒धिभ्यः॒ सर्वा॑साꣳ॒॒ सर्वा॑साम् गृहमे॒धिभ्यो॑ म॒रुद्भ्यो॑ म॒रुद्भ्यो॑ गृहमे॒धिभ्यः॒ सर्वा॑साम् । \newline
20. म॒रुद्भ्य॒ इति॑ म॒रुत् - भ्यः॒ । \newline
21. गृ॒ह॒मे॒धिभ्यः॒ सर्वा॑साꣳ॒॒ सर्वा॑साम् गृहमे॒धिभ्यो॑ गृहमे॒धिभ्यः॒ सर्वा॑साम् दु॒ग्धे दु॒ग्धे सर्वा॑साम् गृहमे॒धिभ्यो॑ गृहमे॒धिभ्यः॒ सर्वा॑साम् दु॒ग्धे । \newline
22. गृ॒ह॒मे॒धिभ्य॒ इति॑ गृहमे॒धि - भ्यः॒ । \newline
23. सर्वा॑साम् दु॒ग्धे दु॒ग्धे सर्वा॑साꣳ॒॒ सर्वा॑साम् दु॒ग्धे सा॒यꣳ सा॒यम् दु॒ग्धे सर्वा॑साꣳ॒॒ सर्वा॑साम् दु॒ग्धे सा॒यम् । \newline
24. दु॒ग्धे सा॒यꣳ सा॒यम् दु॒ग्धे दु॒ग्धे सा॒यम् च॒रुम् च॒रुꣳ सा॒यम् दु॒ग्धे दु॒ग्धे सा॒यम् च॒रुम् । \newline
25. सा॒यम् च॒रुम् च॒रुꣳ सा॒यꣳ सा॒यम् च॒रुम् पू॒र्णा पू॒र्णा च॒रुꣳ सा॒यꣳ सा॒यम् च॒रुम् पू॒र्णा । \newline
26. च॒रुम् पू॒र्णा पू॒र्णा च॒रुम् च॒रुम् पू॒र्णा द॑र्वि दर्वि पू॒र्णा च॒रुम् च॒रुम् पू॒र्णा द॑र्वि । \newline
27. पू॒र्णा द॑र्वि दर्वि पू॒र्णा पू॒र्णा द॑र्वि॒ परा॒ परा॑ दर्वि पू॒र्णा पू॒र्णा द॑र्वि॒ परा᳚ । \newline
28. द॒र्वि॒ परा॒ परा॑ दर्वि दर्वि॒ परा॑ पत पत॒ परा॑ दर्वि दर्वि॒ परा॑ पत । \newline
29. परा॑ पत पत॒ परा॒ परा॑ पत॒ सुपू᳚र्णा॒ सुपू᳚र्णा पत॒ परा॒ परा॑ पत॒ सुपू᳚र्णा । \newline
30. प॒त॒ सुपू᳚र्णा॒ सुपू᳚र्णा पत पत॒ सुपू᳚र्णा॒ पुनः॒ पुनः॒ सुपू᳚र्णा पत पत॒ सुपू᳚र्णा॒ पुनः॑ । \newline
31. सुपू᳚र्णा॒ पुनः॒ पुनः॒ सुपू᳚र्णा॒ सुपू᳚र्णा॒ पुन॒रा पुनः॒ सुपू᳚र्णा॒ सुपू᳚र्णा॒ पुन॒रा । \newline
32. सुपू॒र्णेति॒ सु - पू॒र्णा॒ । \newline
33. पुन॒ रा पुनः॒ पुन॒ रा प॑त प॒ता पुनः॒ पुन॒ रा प॑त । \newline
34. आ प॑त प॒ता प॑त । \newline
35. प॒तेति॑ पत । \newline
36. व॒स्नेवे॑ व व॒स्ना व॒स्नेव॒ वि वीव॑ व॒स्ना व॒स्नेव॒ वि । \newline
37. इ॒व॒ वि वीवे॑ व॒ वि क्री॑णावहै क्रीणावहै॒ वीवे॑ व॒ वि क्री॑णावहै । \newline
38. वि क्री॑णावहै क्रीणावहै॒ वि वि क्री॑णावहा॒ इष॒ मिष॑म् क्रीणावहै॒ वि वि क्री॑णावहा॒ इष᳚म् । \newline
39. क्री॒णा॒व॒हा॒ इष॒ मिष॑म् क्रीणावहै क्रीणावहा॒ इष॒ मूर्ज॒ मूर्ज॒ मिष॑म् क्रीणावहै क्रीणावहा॒ इष॒ मूर्ज᳚म् । \newline
40. इष॒ मूर्ज॒ मूर्ज॒ मिष॒ मिष॒ मूर्जꣳ॑ शतक्रतो शतक्रतो॒ ऊर्ज॒ मिष॒ मिष॒ मूर्जꣳ॑ शतक्रतो । \newline
41. ऊर्जꣳ॑ शतक्रतो शतक्रतो॒ ऊर्ज॒ मूर्जꣳ॑ शतक्रतो । \newline
42. श॒त॒क्र॒तो॒ इति॑ शत - क्र॒तो॒ । \newline
43. दे॒हि मे॑ मे दे॒हि दे॒हि मे॒ ददा॑मि॒ ददा॑मि मे दे॒हि दे॒हि मे॒ ददा॑मि । \newline
44. मे॒ ददा॑मि॒ ददा॑मि मे मे॒ ददा॑मि ते ते॒ ददा॑मि मे मे॒ ददा॑मि ते । \newline
45. ददा॑मि ते ते॒ ददा॑मि॒ ददा॑मि ते॒ नि नि ते॒ ददा॑मि॒ ददा॑मि ते॒ नि । \newline
46. ते॒ नि नि ते॑ ते॒ नि मे॑ मे॒ नि ते॑ ते॒ नि मे᳚ । \newline
47. नि मे॑ मे॒ नि नि मे॑ धेहि धेहि मे॒ नि नि मे॑ धेहि । \newline
48. मे॒ धे॒हि॒ धे॒हि॒ मे॒ मे॒ धे॒हि॒ नि नि धे॑हि मे मे धेहि॒ नि । \newline
49. धे॒हि॒ नि नि धे॑हि धेहि॒ नि ते॑ ते॒ नि धे॑हि धेहि॒ नि ते᳚ । \newline
50. नि ते॑ ते॒ नि नि ते॑ दधे दधे ते॒ नि नि ते॑ दधे । \newline
51. ते॒ द॒धे॒ द॒धे॒ ते॒ ते॒ द॒धे॒ । \newline
52. द॒ध॒ इति॑ दधे । \newline
53. नि॒हार॒ मिदिन् नि॒हार॑म् नि॒हार॒ मिन् नि नीन् नि॒हार॑म् नि॒हार॒ मिन् नि । \newline
54. नि॒हार॒मिति॑ नि - हार᳚म् । \newline
55. इन् नि नीदिन् नि मे॑ मे॒ नीदिन् नि मे᳚ । \newline
56. नि मे॑ मे॒ नि नि मे॑ हर हर मे॒ नि नि मे॑ हर । \newline
57. मे॒ ह॒र॒ ह॒र॒ मे॒ मे॒ ह॒रा॒ नि॒हार॑म् नि॒हारꣳ॑ हर मे मे हरा नि॒हार᳚म् । \newline
58. ह॒रा॒ नि॒हार॑म् नि॒हारꣳ॑ हर हरा नि॒हार॒म् नि नि नि॒हारꣳ॑ हर हरा नि॒हार॒म् नि । \newline
59. नि॒हार॒म् नि नि नि॒हार॑म् नि॒हार॒म् नि ह॑रामि हरामि॒ नि नि॒हार॑म् नि॒हार॒म् नि ह॑रामि । \newline
60. नि॒हार॒मिति॑ नि - हार᳚म् । \newline
\pagebreak
\markright{ TS 1.8.4.2  \hfill https://www.vedavms.in \hfill}
\addcontentsline{toc}{section}{ TS 1.8.4.2 }
\section*{ TS 1.8.4.2 }

\textbf{TS 1.8.4.2 } \newline
\textbf{Samhita Paata} \newline

नि ह॑रामि ते ॥ म॒रुद्भ्यः॑ क्री॒डिभ्यः॑ पुरो॒डाशꣳ॑ स॒प्तक॑पालं॒ निर्व॑पति सा॒कꣳ सूर्ये॑णोद्य॒ताऽऽग्ने॒य-म॒ष्टाक॑पालं॒ निर्व॑पति सौ॒म्यं च॒रुꣳ सा॑वि॒त्रं द्वाद॑शकपालꣳ सारस्व॒तं च॒रुं पौ॒ष्णं च॒रुमै᳚न्द्रा॒ग्न-मेका॑दशकपाल-मै॒न्द्रं च॒रुं ॅवै᳚श्वकर्म॒ण-मेक॑कपालं ॥ \newline

\textbf{Pada Paata} \newline

नीति॑ । ह॒रा॒मि॒ । ते॒ ॥ म॒रुद्भ्य॒ इति॑ म॒रुत् - भ्यः॒ । क्री॒डिभ्य॒ इति॑ क्री॒डि-भ्यः॒ । पु॒रो॒डाश᳚म् । स॒प्तक॑पाल॒मिति॑ स॒प्त-क॒पा॒ल॒म् । निरिति॑ । व॒प॒ति॒ । सा॒कम् । सूर्ये॑ण । उ॒द्य॒तेत्यु॑त् - य॒ता । आ॒ग्ने॒यम् । अ॒ष्टाक॑पाल॒मित्य॒ष्टा-क॒पा॒ल॒म् । निरिति॑ । व॒प॒ति॒ । सौ॒म्यम् । च॒रुम् । सा॒वि॒त्रम् । द्वाद॑शकपाल॒मिति॒ द्वाद॑श-क॒पा॒ल॒म् । सा॒र॒स्व॒तम् । च॒रुम् । पौ॒ष्णम् । च॒रुम् । ऐ॒न्द्रा॒ग्नमित्यै᳚न्द्र - अ॒ग्नम् । एका॑दशकपाल॒मित्येका॑दश - क॒पा॒ल॒म् । ऐ॒न्द्रम् । च॒रुम् । वै॒श्व॒क॒र्म॒णमिति॑ वैश्व - क॒म॒र्णम् ॥ एक॑कपाल॒मित्येक॑-क॒पा॒ल॒म् । 6(30)  \newline



\textbf{Jatai Paata} \newline

1. नि ह॑रामि हरामि॒ नि नि ह॑रामि । \newline
2. ह॒रा॒मि॒ ते॒ ते॒ ह॒रा॒मि॒ ह॒रा॒मि॒ ते॒ । \newline
3. त॒ इति॑ ते । \newline
4. म॒रुद्भ्यः॑ क्री॒डिभ्यः॑ क्री॒डिभ्यो॑ म॒रुद्भ्यो॑ म॒रुद्भ्यः॑ क्री॒डिभ्यः॑ । \newline
5. म॒रुद्भ्य॒ इति॑ म॒रुत् - भ्यः॒ । \newline
6. क्री॒डिभ्यः॑ पुरो॒डाश॑म् पुरो॒डाश॑म् क्री॒डिभ्यः॑ क्री॒डिभ्यः॑ पुरो॒डाश᳚म् । \newline
7. क्री॒डिभ्य॒ इति॑ क्री॒डि - भ्यः॒ । \newline
8. पु॒रो॒डाशꣳ॑ स॒प्तक॑पालꣳ स॒प्तक॑पालम् पुरो॒डाश॑म् पुरो॒डाशꣳ॑ स॒प्तक॑पालम् । \newline
9. स॒प्तक॑पाल॒म् निर् णिः स॒प्तक॑पालꣳ स॒प्तक॑पाल॒म् निः । \newline
10. स॒प्तक॑पाल॒मिति॑ स॒प्त - क॒पा॒ल॒म् । \newline
11. निर् व॑पति वपति॒ निर् णिर् व॑पति । \newline
12. व॒प॒ति॒ सा॒कꣳ सा॒कं ॅव॑पति वपति सा॒कम् । \newline
13. सा॒कꣳ सूर्ये॑ण॒ सूर्ये॑ण सा॒कꣳ सा॒कꣳ सूर्ये॑ण । \newline
14. सूर्ये॑ णोद्य॒तो द्य॒ता सूर्ये॑ण॒ सूर्ये॑णोद्य॒ता । \newline
15. उ॒द्य॒ता ऽऽग्ने॒य मा᳚ग्ने॒य मु॑द्य॒तो द्य॒ता ऽऽग्ने॒यम् । \newline
16. उ॒द्य॒तेत्यु॑त् - य॒ता । \newline
17. आ॒ग्ने॒य म॒ष्टाक॑पाल म॒ष्टाक॑पाल माग्ने॒य मा᳚ग्ने॒य म॒ष्टाक॑पालम् । \newline
18. अ॒ष्टाक॑पाल॒म् निर् णिर॒ष्टाक॑पाल म॒ष्टाक॑पाल॒म् निः । \newline
19. अ॒ष्टाक॑पाल॒मित्य॒ष्टा - क॒पा॒ल॒म् । \newline
20. निर् व॑पति वपति॒ निर् णिर् व॑पति । \newline
21. व॒प॒ति॒ सौ॒म्यꣳ सौ॒म्यं ॅव॑पति वपति सौ॒म्यम् । \newline
22. सौ॒म्यम् च॒रुम् च॒रुꣳ सौ॒म्यꣳ सौ॒म्यम् च॒रुम् । \newline
23. च॒रुꣳ सा॑वि॒त्रꣳ सा॑वि॒त्रम् च॒रुम् च॒रुꣳ सा॑वि॒त्रम् । \newline
24. सा॒वि॒त्रम् द्वाद॑शकपाल॒म् द्वाद॑शकपालꣳ सावि॒त्रꣳ सा॑वि॒त्रम् द्वाद॑शकपालम् । \newline
25. द्वाद॑शकपालꣳ सारस्व॒तꣳ सा॑रस्व॒तम् द्वाद॑शकपाल॒म् द्वाद॑शकपालꣳ सारस्व॒तम् । \newline
26. द्वाद॑शकपाल॒मिति॒ द्वाद॑श - क॒पा॒ल॒म् । \newline
27. सा॒र॒स्व॒तम् च॒रुम् च॒रुꣳ सा॑रस्व॒तꣳ सा॑रस्व॒तम् च॒रुम् । \newline
28. च॒रुम् पौ॒ष्णम् पौ॒ष्णम् च॒रुम् च॒रुम् पौ॒ष्णम् । \newline
29. पौ॒ष्णम् च॒रुम् च॒रुम् पौ॒ष्णम् पौ॒ष्णम् च॒रुम् । \newline
30. च॒रु मै᳚न्द्रा॒ग्न मै᳚न्द्रा॒ग्नम् च॒रुम् च॒रु मै᳚न्द्रा॒ग्नम् । \newline
31. ऐ॒न्द्रा॒ग्न मेका॑दशकपाल॒ मेका॑दशकपाल मैन्द्रा॒ग्न मै᳚न्द्रा॒ग्न मेका॑दशकपालम् । \newline
32. ऐ॒न्द्रा॒ग्नमित्यै᳚न्द्र - अ॒ग्नम् । \newline
33. एका॑दशकपाल मै॒न्द्र मै॒न्द्र मेका॑दशकपाल॒ मेका॑दशकपाल मै॒न्द्रम् । \newline
34. एका॑दशकपाल॒मित्येका॑दश - क॒पा॒ल॒म् । \newline
35. ऐ॒न्द्रम् च॒रुम् च॒रु मै॒न्द्र मै॒न्द्रम् च॒रुम् । \newline
36. च॒रुं ॅवै᳚श्वकर्म॒णं ॅवै᳚श्वकर्म॒णम् च॒रुम् च॒रुं ॅवै᳚श्वकर्म॒णम् । \newline
37. वै॒श्व॒क॒र्म॒ण मेक॑कपाल॒ मेक॑कपालं ॅवैश्वकर्म॒णं ॅवै᳚श्वकर्म॒ण मेक॑कपालम् । \newline
38. वै॒श्व॒क॒र्म॒णमिति॑ वैश्व - क॒म॒र्णम् । \newline
39. एक॑कपाल॒मित्येक॑ - क॒पा॒ल॒म् । \newline

\textbf{Ghana Paata } \newline

1. नि ह॑रामि हरामि॒ नि नि ह॑रामि ते ते हरामि॒ नि नि ह॑रामि ते । \newline
2. ह॒रा॒मि॒ ते॒ ते॒ ह॒रा॒मि॒ ह॒रा॒मि॒ ते॒ । \newline
3. त॒ इति॑ ते । \newline
4. म॒रुद्भ्यः॑ क्री॒डिभ्यः॑ क्री॒डिभ्यो॑ म॒रुद्भ्यो॑ म॒रुद्भ्यः॑ क्री॒डिभ्यः॑ पुरो॒डाश॑म् पुरो॒डाश॑म् क्री॒डिभ्यो॑ म॒रुद्भ्यो॑ म॒रुद्भ्यः॑ क्री॒डिभ्यः॑ पुरो॒डाश᳚म् । \newline
5. म॒रुद्भ्य॒ इति॑ म॒रुत् - भ्यः॒ । \newline
6. क्री॒डिभ्यः॑ पुरो॒डाश॑म् पुरो॒डाश॑म् क्री॒डिभ्यः॑ क्री॒डिभ्यः॑ पुरो॒डाशꣳ॑ स॒प्तक॑पालꣳ स॒प्तक॑पालम् पुरो॒डाश॑म् क्री॒डिभ्यः॑ क्री॒डिभ्यः॑ पुरो॒डाशꣳ॑ स॒प्तक॑पालम् । \newline
7. क्री॒डिभ्य॒ इति॑ क्री॒डि - भ्यः॒ । \newline
8. पु॒रो॒डाशꣳ॑ स॒प्तक॑पालꣳ स॒प्तक॑पालम् पुरो॒डाश॑म् पुरो॒डाशꣳ॑ स॒प्तक॑पाल॒म् निर् णिः स॒प्तक॑पालम् पुरो॒डाश॑म् पुरो॒डाशꣳ॑ स॒प्तक॑पाल॒म् निः । \newline
9. स॒प्तक॑पाल॒म् निर् णिः स॒प्तक॑पालꣳ स॒प्तक॑पाल॒म् निर् व॑पति वपति॒ निः स॒प्तक॑पालꣳ स॒प्तक॑पाल॒म् निर् व॑पति । \newline
10. स॒प्तक॑पाल॒मिति॑ स॒प्त - क॒पा॒ल॒म् । \newline
11. निर् व॑पति वपति॒ निर् णिर् व॑पति सा॒कꣳ सा॒कं ॅव॑पति॒ निर् णिर् व॑पति सा॒कम् । \newline
12. व॒प॒ति॒ सा॒कꣳ सा॒कं ॅव॑पति वपति सा॒कꣳ सूर्ये॑ण॒ सूर्ये॑ण सा॒कं ॅव॑पति वपति सा॒कꣳ सूर्ये॑ण । \newline
13. सा॒कꣳ सूर्ये॑ण॒ सूर्ये॑ण सा॒कꣳ सा॒कꣳ सूर्ये॑णो द्य॒तोद्य॒ता सूर्ये॑ण सा॒कꣳ सा॒कꣳ सूर्ये॑णोद्य॒ता । \newline
14. सूर्ये॑णो द्य॒तोद्य॒ता सूर्ये॑ण॒ सूर्ये॑णोद्य॒ता ऽऽग्ने॒य मा᳚ग्ने॒य मु॑द्य॒ता सूर्ये॑ण॒ सूर्ये॑णोद्य॒ता ऽऽग्ने॒यम् । \newline
15. उ॒द्य॒ता ऽऽग्ने॒य मा᳚ग्ने॒य मु॑द्य॒तोद्य॒ता ऽऽग्ने॒य म॒ष्टाक॑पाल म॒ष्टाक॑पाल माग्ने॒य मु॑द्य॒तोद्य॒ता ऽऽग्ने॒य म॒ष्टाक॑पालम् । \newline
16. उ॒द्य॒तेत्यु॑त् - य॒ता । \newline
17. आ॒ग्ने॒य म॒ष्टाक॑पाल म॒ष्टाक॑पाल माग्ने॒य मा᳚ग्ने॒य म॒ष्टाक॑पाल॒म् निर् णिर॒ष्टाक॑पाल माग्ने॒य मा᳚ग्ने॒य म॒ष्टाक॑पाल॒म् निः । \newline
18. अ॒ष्टाक॑पाल॒म् निर् णिर॒ष्टाक॑पाल म॒ष्टाक॑पाल॒म् निर् व॑पति वपति॒ निर॒ष्टाक॑पाल म॒ष्टाक॑पाल॒म् निर् व॑पति । \newline
19. अ॒ष्टाक॑पाल॒मित्य॒ष्टा - क॒पा॒ल॒म् । \newline
20. निर् व॑पति वपति॒ निर् णिर् व॑पति सौ॒म्यꣳ सौ॒म्यं ॅव॑पति॒ निर् णिर् व॑पति सौ॒म्यम् । \newline
21. व॒प॒ति॒ सौ॒म्यꣳ सौ॒म्यं ॅव॑पति वपति सौ॒म्यम् च॒रुम् च॒रुꣳ सौ॒म्यं ॅव॑पति वपति सौ॒म्यम् च॒रुम् । \newline
22. सौ॒म्यम् च॒रुम् च॒रुꣳ सौ॒म्यꣳ सौ॒म्यम् च॒रुꣳ सा॑वि॒त्रꣳ सा॑वि॒त्रम् च॒रुꣳ सौ॒म्यꣳ सौ॒म्यम् च॒रुꣳ सा॑वि॒त्रम् । \newline
23. च॒रुꣳ सा॑वि॒त्रꣳ सा॑वि॒त्रम् च॒रुम् च॒रुꣳ सा॑वि॒त्रम् द्वाद॑शकपाल॒म् द्वाद॑शकपालꣳ सावि॒त्रम् च॒रुम् च॒रुꣳ सा॑वि॒त्रम् द्वाद॑शकपालम् । \newline
24. सा॒वि॒त्रम् द्वाद॑शकपाल॒म् द्वाद॑शकपालꣳ सावि॒त्रꣳ सा॑वि॒त्रम् द्वाद॑शकपालꣳ सारस्व॒तꣳ सा॑रस्व॒तम् द्वाद॑शकपालꣳ सावि॒त्रꣳ सा॑वि॒त्रम् द्वाद॑शकपालꣳ सारस्व॒तम् । \newline
25. द्वाद॑शकपालꣳ सारस्व॒तꣳ सा॑रस्व॒तम् द्वाद॑शकपाल॒म् द्वाद॑शकपालꣳ सारस्व॒तम् च॒रुम् च॒रुꣳ सा॑रस्व॒तम् द्वाद॑शकपाल॒म् द्वाद॑शकपालꣳ सारस्व॒तम् च॒रुम् । \newline
26. द्वाद॑शकपाल॒मिति॒ द्वाद॑श - क॒पा॒ल॒म् । \newline
27. सा॒र॒स्व॒तम् च॒रुम् च॒रुꣳ सा॑रस्व॒तꣳ सा॑रस्व॒तम् च॒रुम् पौ॒ष्णम् पौ॒ष्णम् च॒रुꣳ सा॑रस्व॒तꣳ सा॑रस्व॒तम् च॒रुम् पौ॒ष्णम् । \newline
28. च॒रुम् पौ॒ष्णम् पौ॒ष्णम् च॒रुम् च॒रुम् पौ॒ष्णम् च॒रुम् च॒रुम् पौ॒ष्णम् च॒रुम् च॒रुम् पौ॒ष्णम् च॒रुम् । \newline
29. पौ॒ष्णम् च॒रुम् च॒रुम् पौ॒ष्णम् पौ॒ष्णम् च॒रु मै᳚न्द्रा॒ग्न मै᳚न्द्रा॒ग्नम् च॒रुम् पौ॒ष्णम् पौ॒ष्णम् च॒रु मै᳚न्द्रा॒ग्नम् । \newline
30. च॒रु मै᳚न्द्रा॒ग्न मै᳚न्द्रा॒ग्नम् च॒रुम् च॒रु मै᳚न्द्रा॒ग्न मेका॑दशकपाल॒ मेका॑दशकपाल मैन्द्रा॒ग्नम् च॒रुम् च॒रु मै᳚न्द्रा॒ग्न मेका॑दशकपालम् । \newline
31. ऐ॒न्द्रा॒ग्न मेका॑दशकपाल॒ मेका॑दशकपाल मैन्द्रा॒ग्न मै᳚न्द्रा॒ग्न मेका॑दशकपाल मै॒न्द्र मै॒न्द्र मेका॑दशकपाल मैन्द्रा॒ग्न मै᳚न्द्रा॒ग्न मेका॑दशकपाल मै॒न्द्रम् । \newline
32. ऐ॒न्द्रा॒ग्नमित्यै᳚न्द्र - अ॒ग्नम् । \newline
33. एका॑दशकपाल मै॒न्द्र मै॒न्द्र मेका॑दशकपाल॒ मेका॑दशकपाल मै॒न्द्रम् च॒रुम् च॒रु मै॒न्द्र मेका॑दशकपाल॒ मेका॑दशकपाल मै॒न्द्रम् च॒रुम् । \newline
34. एका॑दशकपाल॒मित्येका॑दश - क॒पा॒ल॒म् । \newline
35. ऐ॒न्द्रम् च॒रुम् च॒रु मै॒न्द्र मै॒न्द्रम् च॒रुं ॅवै᳚श्वकर्म॒णं ॅवै᳚श्वकर्म॒णम् च॒रु मै॒न्द्र मै॒न्द्रम् च॒रुं ॅवै᳚श्वकर्म॒णम् । \newline
36. च॒रुं ॅवै᳚श्वकर्म॒णं ॅवै᳚श्वकर्म॒णम् च॒रुम् च॒रुं ॅवै᳚श्वकर्म॒ण मेक॑कपाल॒ मेक॑कपालं ॅवैश्वकर्म॒णम् च॒रुम् च॒रुं ॅवै᳚श्वकर्म॒ण मेक॑कपालम् । \newline
37. वै॒श्व॒क॒र्म॒ण मेक॑कपाल॒ मेक॑कपालं ॅवैश्वकर्म॒णं ॅवै᳚श्वकर्म॒ण मेक॑कपालम् । \newline
38. वै॒श्व॒क॒र्म॒णमिति॑ वैश्व - क॒म॒र्णम् । \newline
39. एक॑कपाल॒मित्येक॑ - क॒पा॒ल॒म् । \newline
\pagebreak
\markright{ TS 1.8.5.1  \hfill https://www.vedavms.in \hfill}
\addcontentsline{toc}{section}{ TS 1.8.5.1 }
\section*{ TS 1.8.5.1 }

\textbf{TS 1.8.5.1 } \newline
\textbf{Samhita Paata} \newline

सोमा॑य पितृ॒मते॑ पुरो॒डाशꣳ॒॒ षट्क॑पालं॒ निर्व॑पति पि॒तृभ्यो॑ बर्.हि॒षद्भ्यो॑ धा॒नाः पि॒तृभ्यो᳚ऽग्निष्वा॒त्तेभ्यो॑ ऽभिवा॒न्या॑यै दु॒ग्धे म॒न्थमे॒तत् ते॑ तत॒ ये च॒ त्वा-मन्वे॒तत् ते॑ पितामह प्रपितामह॒ ये च॒ त्वामन्वत्र॑ पितरो यथाभा॒गं म॑न्दद्ध्वꣳ सुस॒न्दृशं॑ त्वा व॒यं मघ॑वन् मन्दिषी॒महि॑ । प्रनू॒नं पू॒र्णव॑न्धुरः स्तु॒तो या॑सि॒ वशाꣳ॒॒ अनु॑ । योजा॒ न्वि॑न्द्र ते॒ हरी᳚ ॥ \newline

\textbf{Pada Paata} \newline

सोमा॑य । पि॒तृ॒मत॒ इति॑ पितृ - मते᳚ । पु॒रो॒डाश᳚म् । षट्क॑पाल॒मिति॒ षट् - क॒पा॒ल॒म् । निरिति॑ । व॒प॒ति॒ । पि॒तृभ्य॒ इति॑ पि॒तृ - भ्यः॒ । ब॒र्.॒हि॒षद्भ्य॒ इति॑ बर्.हि॒षद्-भ्यः॒ । धा॒नाः । पि॒तृभ्य॒ इति॑ पि॒तृ-भ्यः॒ । अ॒ग्नि॒ष्वा॒त्तेभ्य॒ इत्य॑ग्नि-स्वा॒त्तेभ्यः॑ । अ॒भि॒वा॒न्या॑या॒ इत्य॑भि-वा॒न्या॑यै । दु॒ग्धे । म॒न्थम् । ए॒तत् । ते॒ । त॒त॒ । ये । च॒ । त्वाम् । अन्विति॑ । ए॒तत् । ते॒ । पि॒ता॒म॒ह॒ । प्र॒पि॒ता॒म॒हेति॑ प्र - पि॒ता॒म॒ह॒ । ये । च॒ । त्वाम् । अन्विति॑ । अत्र॑ । पि॒त॒रः॒ । य॒था॒भा॒गमिति॑ यथा - भा॒गम् । म॒न्द॒द्ध्व॒म् । सु॒स॒दृंश॒मिति॑ सु - स॒दृंश᳚म् । त्वा॒ । व॒यम् । मघ॑व॒न्निति॒ मघ॑-व॒न्न् । म॒न्दि॒षी॒महि॑ ॥ प्रेति॑ । नू॒नम् । पू॒र्णव॑न्धुर॒ इति॑ पू॒र्ण - व॒न्धु॒रः॒ । स्तु॒तः । या॒सि॒ । वशान्॑ । अनु॑ ॥ योजा᳚ । नु । इ॒न्द्र॒ । ते॒ । हरी॒ इति॑ ॥  \newline



\textbf{Jatai Paata} \newline

1. सोमा॑य पितृ॒मते॑ पितृ॒मते॒ सोमा॑य॒ सोमा॑य पितृ॒मते᳚ । \newline
2. पि॒तृ॒मते॑ पुरो॒डाश॑म् पुरो॒डाश॑म् पितृ॒मते॑ पितृ॒मते॑ पुरो॒डाश᳚म् । \newline
3. पि॒तृ॒मत॒ इति॑ पितृ - मते᳚ । \newline
4. पु॒रो॒डाशꣳ॒॒ षट्क॑पालꣳ॒॒ षट्क॑पालम् पुरो॒डाश॑म् पुरो॒डाशꣳ॒॒ षट्क॑पालम् । \newline
5. षट्क॑पाल॒म् निर् णिष्षट्क॑पालꣳ॒॒ षट्क॑पाल॒म् निः । \newline
6. षट्क॑पाल॒मिति॒ षट् - क॒पा॒ल॒म् । \newline
7. निर् व॑पति वपति॒ निर् णिर् व॑पति । \newline
8. व॒प॒ति॒ पि॒तृभ्यः॑ पि॒तृभ्यो॑ वपति वपति पि॒तृभ्यः॑ । \newline
9. पि॒तृभ्यो॑ बर्.हि॒षद्भ्यो॑ बर्.हि॒षद्भ्यः॑ पि॒तृभ्यः॑ पि॒तृभ्यो॑ बर्.हि॒षद्भ्यः॑ । \newline
10. पि॒तृभ्य॒ इति॑ पि॒तृ - भ्यः॒ । \newline
11. ब॒र्॒.हि॒षद्भ्यो॑ धा॒ना धा॒ना ब॑र्.हि॒षद्भ्यो॑ बर्.हि॒षद्भ्यो॑ धा॒नाः । \newline
12. ब॒र्.॒हि॒षद्भ्य॒ इति॑ बर्.हि॒षद् - भ्यः॒ । \newline
13. धा॒नाः पि॒तृभ्यः॑ पि॒तृभ्यो॑ धा॒ना धा॒नाः पि॒तृभ्यः॑ । \newline
14. पि॒तृभ्यो᳚ ऽग्निष्वा॒त्तेभ्यो᳚ ऽग्निष्वा॒त्तेभ्यः॑ पि॒तृभ्यः॑ पि॒तृभ्यो᳚ ऽग्निष्वा॒त्तेभ्यः॑ । \newline
15. पि॒तृभ्य॒ इति॑ पि॒तृ - भ्यः॒ । \newline
16. अ॒ग्नि॒ष्वा॒त्तेभ्यो॑ ऽभिवा॒न्या॑या अभिवा॒न्या॑या अग्निष्वा॒त्तेभ्यो᳚ ऽग्निष्वा॒त्तेभ्यो॑ ऽभिवा॒न्या॑यै । \newline
17. अ॒ग्नि॒ष्वा॒त्तेभ्य॒ इत्य॑ग्नि - स्वा॒त्तेभ्यः॑ । \newline
18. अ॒भि॒वा॒न्या॑यै दु॒ग्धे दु॒ग्धे॑ ऽभिवा॒न्या॑या अभिवा॒न्या॑यै दु॒ग्धे । \newline
19. अ॒भि॒वा॒न्या॑या॒ इत्य॑भि - वा॒न्या॑यै । \newline
20. दु॒ग्धे म॒न्थम् म॒न्थम् दु॒ग्धे दु॒ग्धे म॒न्थम् । \newline
21. म॒न्थ मे॒त दे॒तन् म॒न्थम् म॒न्थ मे॒तत् । \newline
22. ए॒तत् ते॑ त ए॒त दे॒तत् ते᳚ । \newline
23. ते॒ त॒त॒ त॒त॒ ते॒ ते॒ त॒त॒ । \newline
24. त॒त॒ ये ये त॑त तत॒ ये । \newline
25. ये च॑ च॒ ये ये च॑ । \newline
26. च॒ त्वाम् त्वाम् च॑ च॒ त्वाम् । \newline
27. त्वा मन्वनु॒ त्वाम् त्वा मनु॑ । \newline
28. अन्वे॒त दे॒त दन्वन् वे॒तत् । \newline
29. ए॒तत् ते॑ त ए॒त दे॒तत् ते᳚ । \newline
30. ते॒ पि॒ता॒म॒ह॒ पि॒ता॒म॒ह॒ ते॒ ते॒ पि॒ता॒म॒ह॒ । \newline
31. पि॒ता॒म॒ह॒ प्र॒पि॒ता॒म॒ह॒ प्र॒पि॒ता॒म॒ह॒ पि॒ता॒म॒ह॒ पि॒ता॒म॒ह॒ प्र॒पि॒ता॒म॒ह॒ । \newline
32. प्र॒पि॒ता॒म॒ह॒ ये ये प्र॑पितामह प्रपितामह॒ ये । \newline
33. प्र॒पि॒ता॒म॒हेति॑ प्र - पि॒ता॒म॒ह॒ । \newline
34. ये च॑ च॒ ये ये च॑ । \newline
35. च॒ त्वाम् त्वाम् च॑ च॒ त्वाम् । \newline
36. त्वा मन्वनु॒ त्वाम् त्वा मनु॑ । \newline
37. अन्वत्रा त्रान्वन्वत्र॑ । \newline
38. अत्र॑ पितरः पित॒रो ऽत्रात्र॑ पितरः । \newline
39. पि॒त॒रो॒ य॒था॒भा॒गं ॅय॑थाभा॒गम् पि॑तरः पितरो यथाभा॒गम् । \newline
40. य॒था॒भा॒गम् म॑न्दद्ध्वम् मन्दद्ध्वं ॅयथाभा॒गं ॅय॑थाभा॒गम् म॑न्दद्ध्वम् । \newline
41. य॒था॒भा॒गमिति॑ यथा - भा॒गम् । \newline
42. म॒न्द॒द्ध्वꣳ॒॒ सु॒स॒न्दृशꣳ॑ सुस॒न्दृश॑म् मन्दद्ध्वम् मन्दद्ध्वꣳ सुस॒न्दृश᳚म् । \newline
43. सु॒स॒न्दृश॑म् त्वा त्वा सुस॒न्दृशꣳ॑ सुस॒न्दृश॑म् त्वा । \newline
44. सु॒स॒न्दृश॒मिति॑ सु - स॒न्दृश᳚म् । \newline
45. त्वा॒ व॒यं ॅव॒यम् त्वा᳚ त्वा व॒यम् । \newline
46. व॒यम् मघ॑व॒न् मघ॑वन्. व॒यं ॅव॒यम् मघ॑वन्न् । \newline
47. मघ॑वन् मन्दिषी॒महि॑ मन्दिषी॒महि॒ मघ॑व॒न् मघ॑वन् मन्दिषी॒महि॑ । \newline
48. मघ॑व॒न्निति॒ मघ॑ - व॒न्न् । \newline
49. म॒न्दि॒षी॒महीति॑ मन्दिषी॒महि॑ । \newline
50. प्र नू॒नन्नू॒नम् प्र प्र नू॒नम् । \newline
51. नू॒नम् पू॒र्णव॑न्धुरः पू॒र्णव॑न्धुरो नू॒नम् नू॒नम् पू॒र्णव॑न्धुरः । \newline
52. पू॒र्णव॑न्धुरः स्तु॒तः स्तु॒तः पू॒र्णव॑न्धुरः पू॒र्णव॑न्धुरः स्तु॒तः । \newline
53. पू॒र्णव॑न्धुर॒ इति॑ पू॒र्ण - व॒न्धु॒रः॒ । \newline
54. स्तु॒तो या॑सि यासि स्तु॒तः स्तु॒तो या॑सि । \newline
55. या॒सि॒ वशा॒न्॒. वशान्॑. यासि यासि॒ वशान्॑ । \newline
56. वशाꣳ॒॒ अन्वनु॒ वशा॒न्॒. वशाꣳ॒॒ अनु॑ । \newline
57. अन्वित्यनु॑ । \newline
58. योजा॒ नु नु योजा॒ योजा॒ नु । \newline
59. न्वि॑न्द्रे न्द्र॒ नु न्वि॑न्द्र । \newline
60. इ॒न्द्र॒ ते॒ त॒ इ॒न्द्रे॒ न्द्र॒ ते॒ । \newline
61. ते॒ हरी॒ हरी॑ ते ते॒ हरी᳚ । \newline
62. हरी॒ इति॒ हरी᳚ । \newline

\textbf{Ghana Paata } \newline

1. सोमा॑य पितृ॒मते॑ पितृ॒मते॒ सोमा॑य॒ सोमा॑य पितृ॒मते॑ पुरो॒डाश॑म् पुरो॒डाश॑म् पितृ॒मते॒ सोमा॑य॒ सोमा॑य पितृ॒मते॑ पुरो॒डाश᳚म् । \newline
2. पि॒तृ॒मते॑ पुरो॒डाश॑म् पुरो॒डाश॑म् पितृ॒मते॑ पितृ॒मते॑ पुरो॒डाशꣳ॒॒ षट्क॑पालꣳ॒॒ षट्क॑पालम् पुरो॒डाश॑म् पितृ॒मते॑ पितृ॒मते॑ पुरो॒डाशꣳ॒॒ षट्क॑पालम् । \newline
3. पि॒तृ॒मत॒ इति॑ पितृ - मते᳚ । \newline
4. पु॒रो॒डाशꣳ॒॒ षट्क॑पालꣳ॒॒ षट्क॑पालम् पुरो॒डाश॑म् पुरो॒डाशꣳ॒॒ षट्क॑पाल॒म् निर् णिष्षट्क॑पालम् पुरो॒डाश॑म् पुरो॒डाशꣳ॒॒ षट्क॑पाल॒म् निः । \newline
5. षट्क॑पाल॒म् निर् णिष्षट्क॑पालꣳ॒॒ षट्क॑पाल॒म् निर् व॑पति वपति॒ निष्षट्क॑पालꣳ॒॒ षट्क॑पाल॒म् निर् व॑पति । \newline
6. षट्क॑पाल॒मिति॒ षट् - क॒पा॒ल॒म् । \newline
7. निर् व॑पति वपति॒ निर् णिर् व॑पति पि॒तृभ्यः॑ पि॒तृभ्यो॑ वपति॒ निर् णिर् व॑पति पि॒तृभ्यः॑ । \newline
8. व॒प॒ति॒ पि॒तृभ्यः॑ पि॒तृभ्यो॑ वपति वपति पि॒तृभ्यो॑ बर्.हि॒षद्भ्यो॑ बर्.हि॒षद्भ्यः॑ पि॒तृभ्यो॑ वपति वपति पि॒तृभ्यो॑ बर्.हि॒षद्भ्यः॑ । \newline
9. पि॒तृभ्यो॑ बर्.हि॒षद्भ्यो॑ बर्.हि॒षद्भ्यः॑ पि॒तृभ्यः॑ पि॒तृभ्यो॑ बर्.हि॒षद्भ्यो॑ धा॒ना धा॒ना ब॑र्.हि॒षद्भ्यः॑ पि॒तृभ्यः॑ पि॒तृभ्यो॑ बर्.हि॒षद्भ्यो॑ धा॒नाः । \newline
10. पि॒तृभ्य॒ इति॑ पि॒तृ - भ्यः॒ । \newline
11. ब॒र्॒.हि॒षद्भ्यो॑ धा॒ना धा॒ना ब॑र्.हि॒षद्भ्यो॑ बर्.हि॒षद्भ्यो॑ धा॒नाः पि॒तृभ्यः॑ पि॒तृभ्यो॑ धा॒ना ब॑र्.हि॒षद्भ्यो॑ बर्.हि॒षद्भ्यो॑ धा॒नाः पि॒तृभ्यः॑ । \newline
12. ब॒र्.॒हि॒षद्भ्य॒ इति॑ बर्.हि॒षद् - भ्यः॒ । \newline
13. धा॒नाः पि॒तृभ्यः॑ पि॒तृभ्यो॑ धा॒ना धा॒नाः पि॒तृभ्यो᳚ ऽग्निष्वा॒त्तेभ्यो᳚ ऽग्निष्वा॒त्तेभ्यः॑ पि॒तृभ्यो॑ धा॒ना धा॒नाः पि॒तृभ्यो᳚ ऽग्निष्वा॒त्तेभ्यः॑ । \newline
14. पि॒तृभ्यो᳚ ऽग्निष्वा॒त्तेभ्यो᳚ ऽग्निष्वा॒त्तेभ्यः॑ पि॒तृभ्यः॑ पि॒तृभ्यो᳚ ऽग्निष्वा॒त्तेभ्यो॑ ऽभिवा॒न्या॑या अभिवा॒न्या॑या अग्निष्वा॒त्तेभ्यः॑ पि॒तृभ्यः॑ पि॒तृभ्यो᳚ ऽग्निष्वा॒त्तेभ्यो॑ ऽभिवा॒न्या॑यै । \newline
15. पि॒तृभ्य॒ इति॑ पि॒तृ - भ्यः॒ । \newline
16. अ॒ग्नि॒ष्वा॒त्तेभ्यो॑ ऽभिवा॒न्या॑या अभिवा॒न्या॑या अग्निष्वा॒त्तेभ्यो᳚ ऽग्निष्वा॒त्तेभ्यो॑ ऽभिवा॒न्या॑यै दु॒ग्धे दु॒ग्धे॑ ऽभिवा॒न्या॑या अग्निष्वा॒त्तेभ्यो᳚ ऽग्निष्वा॒त्तेभ्यो॑ ऽभिवा॒न्या॑यै दु॒ग्धे । \newline
17. अ॒ग्नि॒ष्वा॒त्तेभ्य॒ इत्य॑ग्नि - स्वा॒त्तेभ्यः॑ । \newline
18. अ॒भि॒वा॒न्या॑यै दु॒ग्धे दु॒ग्धे॑ ऽभिवा॒न्या॑या अभिवा॒न्या॑यै दु॒ग्धे म॒न्थम् म॒न्थम् दु॒ग्धे॑ ऽभिवा॒न्या॑या अभिवा॒न्या॑यै दु॒ग्धे म॒न्थम् । \newline
19. अ॒भि॒वा॒न्या॑या॒ इत्य॑भि - वा॒न्या॑यै । \newline
20. दु॒ग्धे म॒न्थम् म॒न्थम् दु॒ग्धे दु॒ग्धे म॒न्थ मे॒त दे॒तन् म॒न्थम् दु॒ग्धे दु॒ग्धे म॒न्थ मे॒तत् । \newline
21. म॒न्थ मे॒त दे॒तन् म॒न्थम् म॒न्थ मे॒तत् ते॑ त ए॒तन् म॒न्थम् म॒न्थ मे॒तत् ते᳚ । \newline
22. ए॒तत् ते॑ त ए॒त दे॒तत् ते॑ तत तत त ए॒त दे॒तत् ते॑ तत । \newline
23. ते॒ त॒त॒ त॒त॒ ते॒ ते॒ त॒त॒ ये ये त॑त ते ते तत॒ ये । \newline
24. त॒त॒ ये ये त॑त तत॒ ये च॑ च॒ ये त॑त तत॒ ये च॑ । \newline
25. ये च॑ च॒ ये ये च॒ त्वाम् त्वाम् च॒ ये ये च॒ त्वाम् । \newline
26. च॒ त्वाम् त्वाम् च॑ च॒ त्वा मन्वनु॒ त्वाम् च॑ च॒ त्वा मनु॑ । \newline
27. त्वा मन्वनु॒ त्वाम् त्वा मन्वे॒त दे॒तदनु॒ त्वाम् त्वा मन्वे॒तत् । \newline
28. अन्वे॒त दे॒तद न्वन्वे॒तत् ते॑ त ए॒त दन्वन्वे॒तत् ते᳚ । \newline
29. ए॒तत् ते॑ त ए॒तदे॒तत् ते॑ पितामह पितामह त ए॒तदे॒तत् ते॑ पितामह । \newline
30. ते॒ पि॒ता॒म॒ह॒ पि॒ता॒म॒ह॒ ते॒ ते॒ पि॒ता॒म॒ह॒ प्र॒पि॒ता॒म॒ह॒ प्र॒पि॒ता॒म॒ह॒ पि॒ता॒म॒ह॒ ते॒ ते॒ पि॒ता॒म॒ह॒ प्र॒पि॒ता॒म॒ह॒ । \newline
31. पि॒ता॒म॒ह॒ प्र॒पि॒ता॒म॒ह॒ प्र॒पि॒ता॒म॒ह॒ पि॒ता॒म॒ह॒ पि॒ता॒म॒ह॒ प्र॒पि॒ता॒म॒ह॒ ये ये प्र॑पितामह पितामह पितामह प्रपितामह॒ ये । \newline
32. प्र॒पि॒ता॒म॒ह॒ ये ये प्र॑पितामह प्रपितामह॒ ये च॑ च॒ ये प्र॑पितामह प्रपितामह॒ ये च॑ । \newline
33. प्र॒पि॒ता॒म॒हेति॑ प्र - पि॒ता॒म॒ह॒ । \newline
34. ये च॑ च॒ ये ये च॒ त्वाम् त्वाम् च॒ ये ये च॒ त्वाम् । \newline
35. च॒ त्वाम् त्वाम् च॑ च॒ त्वा मन्वनु॒ त्वाम् च॑ च॒ त्वा मनु॑ । \newline
36. त्वा मन्वनु॒ त्वाम् त्वा मन्व त्रात्रानु॒ त्वाम् त्वा मन्वत्र॑ । \newline
37. अन्व त्रात्रा न्वन्वत्र॑ पितरः पित॒रो ऽत्रा न्वन्वत्र॑ पितरः । \newline
38. अत्र॑ पितरः पित॒रो ऽत्रात्र॑ पितरो यथाभा॒गं ॅय॑थाभा॒गम् पि॑त॒रो ऽत्रात्र॑ पितरो यथाभा॒गम् । \newline
39. पि॒त॒रो॒ य॒था॒भा॒गं ॅय॑थाभा॒गम् पि॑तरः पितरो यथाभा॒गम् म॑न्दद्ध्वम् मन्दद्ध्वं ॅयथाभा॒गम् पि॑तरः पितरो यथाभा॒गम् म॑न्दद्ध्वम् । \newline
40. य॒था॒भा॒गम् म॑न्दद्ध्वम् मन्दद्ध्वं ॅयथाभा॒गं ॅय॑थाभा॒गम् म॑न्दद्ध्वꣳ सुस॒न्दृशꣳ॑ सुस॒न्दृश॑म् मन्दद्ध्वं ॅयथाभा॒गं ॅय॑थाभा॒गम् म॑न्दद्ध्वꣳ सुस॒न्दृश᳚म् । \newline
41. य॒था॒भा॒गमिति॑ यथा - भा॒गम् । \newline
42. म॒न्द॒द्ध्वꣳ॒॒ सु॒स॒न्दृशꣳ॑ सुस॒न्दृश॑म् मन्दद्ध्वम् मन्दद्ध्वꣳ सुस॒न्दृश॑म् त्वा त्वा सुस॒न्दृश॑म् मन्दद्ध्वम् मन्दद्ध्वꣳ सुस॒न्दृश॑म् त्वा । \newline
43. सु॒स॒न्दृश॑म् त्वा त्वा सुस॒न्दृशꣳ॑ सुस॒न्दृश॑म् त्वा व॒यं ॅव॒यम् त्वा॑ सुस॒न्दृशꣳ॑ सुस॒न्दृश॑म् त्वा व॒यम् । \newline
44. सु॒स॒न्दृश॒मिति॑ सु - स॒न्दृश᳚म् । \newline
45. त्वा॒ व॒यं ॅव॒यम् त्वा᳚ त्वा व॒यम् मघ॑व॒न् मघ॑वन्. व॒यम् त्वा᳚ त्वा व॒यम् मघ॑वन्न् । \newline
46. व॒यम् मघ॑व॒न् मघ॑वन्. व॒यं ॅव॒यम् मघ॑वन् मन्दिषी॒महि॑ मन्दिषी॒महि॒ मघ॑वन्. व॒यं ॅव॒यम् मघ॑वन् मन्दिषी॒महि॑ । \newline
47. मघ॑वन् मन्दिषी॒महि॑ मन्दिषी॒महि॒ मघ॑व॒न् मघ॑वन् मन्दिषी॒महि॑ । \newline
48. मघ॑व॒न्निति॒ मघ॑ - व॒न्न् । \newline
49. म॒न्दि॒षी॒महीति॑ मन्दिषी॒महि॑ । \newline
50. प्र नू॒नन्नू॒नम् प्र प्र नू॒नम् पू॒र्णव॑न्धुरः पू॒र्णव॑न्धुरो नू॒नम् प्र प्र नू॒नम् पू॒र्णव॑न्धुरः । \newline
51. नू॒नम् पू॒र्णव॑न्धुरः पू॒र्णव॑न्धुरो नू॒नम् नू॒नम् पू॒र्णव॑न्धुरः स्तु॒तः स्तु॒तः पू॒र्णव॑न्धुरो नू॒नम् नू॒नम् पू॒र्णव॑न्धुरः स्तु॒तः । \newline
52. पू॒र्णव॑न्धुरः स्तु॒तः स्तु॒तः पू॒र्णव॑न्धुरः पू॒र्णव॑न्धुरः स्तु॒तो या॑सि यासि स्तु॒तः पू॒र्णव॑न्धुरः पू॒र्णव॑न्धुरः स्तु॒तो या॑सि । \newline
53. पू॒र्णव॑न्धुर॒ इति॑ पू॒र्ण - व॒न्धु॒रः॒ । \newline
54. स्तु॒तो या॑सि यासि स्तु॒तः स्तु॒तो या॑सि॒ वशा॒न्॒. वशान्॑. यासि स्तु॒तः स्तु॒तो या॑सि॒ वशान्॑ । \newline
55. या॒सि॒ वशा॒न्॒. वशान्॑. यासि यासि॒ वशाꣳ॒॒ अन्वनु॒ वशान्॑. यासि यासि॒ वशाꣳ॒॒ अनु॑ । \newline
56. वशाꣳ॒॒ अन्वनु॒ वशा॒न्॒. वशाꣳ॒॒ अनु॑ । \newline
57. अन्वित्यनु॑ । \newline
58. योजा॒ नु नु योजा॒ योजा॒ न्वि॑न्द्रे न्द्र॒ नु योजा॒ योजा॒ न्वि॑न्द्र । \newline
59. न्वि॑न्द्रे न्द्र॒ नु न्वि॑न्द्र ते त इन्द्र॒ नु न्वि॑न्द्र ते । \newline
60. इ॒न्द्र॒ ते॒ त॒ इ॒न्द्रे॒ न्द्र॒ ते॒ हरी॒ हरी॑ त इन्द्रे न्द्र ते॒ हरी᳚ । \newline
61. ते॒ हरी॒ हरी॑ ते ते॒ हरी᳚ । \newline
62. हरी॒ इति॒ हरी᳚ । \newline
\pagebreak
\markright{ TS 1.8.5.2  \hfill https://www.vedavms.in \hfill}
\addcontentsline{toc}{section}{ TS 1.8.5.2 }
\section*{ TS 1.8.5.2 }

\textbf{TS 1.8.5.2 } \newline
\textbf{Samhita Paata} \newline

अक्ष॒न्नमी॑मदन्त॒ ह्यव॑ प्रि॒या अ॑धूषत । अस्तो॑षत॒ स्वभा॑नवो॒ विप्रा॒ नवि॑ष्ठया म॒ती । योजा॒ न्वि॑न्द्र ते॒ हरी᳚ ॥ अक्ष॑न् पि॒तरोऽमी॑मदन्त पि॒तरोऽती॑तृपन्त पि॒तरोऽमी॑मृजन्त पि॒तरः॑ ॥ परे॑त पितरः सोम्या गंभी॒रैः प॒थिभिः॑ पू॒र्व्यैः । अथा॑ पि॒तृन्थ् सु॑वि॒दत्राꣳ॒॒ अपी॑त य॒मेन॒ ये स॑ध॒मादं॒ मद॑न्ति ॥ मनो॒ न्वा हु॑वामहे नाराशꣳ॒॒सेन॒ स्तोमे॑न पितृ॒णां च॒ मन्म॑भिः ॥ आ - [ ] \newline

\textbf{Pada Paata} \newline

अक्षन्न्॑ । अमी॑मदन्त । हि । अवेति॑ । प्रि॒याः । अ॒धू॒ष॒त॒ ॥ अस्तो॑षत । स्वभा॑नव॒ इति॒ स्व-भा॒न॒वः॒ । विप्राः᳚ । नवि॑ष्ठया । म॒ती ॥ योजा᳚ । नु । इ॒न्द्र॒ । ते॒ । हरी॒ इति॑ ॥ अक्षन्न्॑ । पि॒तरः॑ । अमी॑मदन्त । पि॒तरः॑ ॥ अती॑तृपन्त । पि॒तरः॑ । अमी॑मृजन्त । पि॒तरः॑ । परेति॑ । इ॒त॒ । पि॒त॒रः॒ । सो॒म्याः॒ । ग॒भीं॒रैः । प॒थिभि॒रिति॑ प॒थि-भिः॒ । पू॒र्व्यैः ॥ अथ॑ । पि॒तॄन् । सु॒वि॒दत्रा॒निति॑ सु - वि॒दत्रान्॑ । अपीति॑ । इ॒त॒ । य॒मेन॑ । ये । स॒ध॒माद॒मिति॑ सध - माद᳚म् । मद॑न्ति ॥ मनः॑ । नु । एति॑ । हु॒वा॒म॒हे॒ । ना॒रा॒शꣳ॒॒सेन॑ । स्तोमे॑न । पि॒तृ॒णाम् । च॒ । मन्म॑भि॒रिति॒ मन्म॑ - भिः॒ ॥ एति॑ ।  \newline



\textbf{Jatai Paata} \newline

1. अक्ष॒न् नमी॑मद॒न्ता मी॑मद॒न्ताक्ष॒न् नक्ष॒न् नमी॑मदन्त । \newline
2. अमी॑मदन्त॒ हि ह्यमी॑मद॒न्ता मी॑मदन्त॒ हि । \newline
3. ह्यवाव॒ हि ह्यव॑ । \newline
4. अव॑ प्रि॒याः प्रि॒या अवाव॑ प्रि॒याः । \newline
5. प्रि॒या अ॑धूषता धूषत प्रि॒याः प्रि॒या अ॑धूषत । \newline
6. अ॒धू॒ष॒तेत्य॑धूषत । \newline
7. अस्तो॑षत॒ स्वभा॑नवः॒ स्वभा॑न॒वो ऽस्तो॑ष॒ता स्तो॑षत॒ स्वभा॑नवः । \newline
8. स्वभा॑नवो॒ विप्रा॒ विप्राः॒ स्वभा॑नवः॒ स्वभा॑नवो॒ विप्राः᳚ । \newline
9. स्वभा॑नव॒ इति॒ स्व - भा॒न॒वः॒ । \newline
10. विप्रा॒ नवि॑ष्ठया॒ नवि॑ष्ठया॒ विप्रा॒ विप्रा॒ नवि॑ष्ठया । \newline
11. नवि॑ष्ठया म॒ती म॒ती नवि॑ष्ठया॒ नवि॑ष्ठया म॒ती । \newline
12. म॒तीति॑ म॒ती । \newline
13. योजा॒ नु नु योजा॒ योजा॒ नु । \newline
14. न्वि॑न्द्रे न्द्र॒ नु न्वि॑न्द्र । \newline
15. इ॒न्द्र॒ ते॒ त॒ इ॒न्द्रे॒ न्द्र॒ ते॒ । \newline
16. ते॒ हरी॒ हरी॑ ते ते॒ हरी᳚ । \newline
17. हरी॒ इति॒ हरी᳚ । \newline
18. अक्ष॑न् पि॒तरः॑ पि॒तरो ऽक्ष॒न् नक्ष॑न् पि॒तरः॑ । \newline
19. पि॒तरो ऽमी॑मद॒न्ता मी॑मदन्त पि॒तरः॑ पि॒तरो ऽमी॑मदन्त । \newline
20. अमी॑मदन्त पि॒तरः॑ पि॒तरो ऽमी॑मद॒न्ता मी॑मदन्त पि॒तरः॑ । \newline
21. पि॒तरो ऽती॑तृप॒न्ता ती॑तृपन्त पि॒तरः॑ पि॒तरो ऽती॑तृपन्त । \newline
22. अती॑तृपन्त पि॒तरः॑ पि॒तरो ऽती॑तृप॒न्ता ती॑तृपन्त पि॒तरः॑ । \newline
23. पि॒तरो ऽमी॑मृज॒न्ता मी॑मृजन्त पि॒तरः॑ पि॒तरो ऽमी॑मृजन्त । \newline
24. अमी॑मृजन्त पि॒तरः॑ पि॒तरो ऽमी॑मृज॒न्ता मी॑मृजन्त पि॒तरः॑ । \newline
25. पि॒तर॒ इति॑ पि॒तरः॑ । \newline
26. परे॑ते त॒ परा॒ परे॑त । \newline
27. इ॒त॒ पि॒त॒रः॒ पि॒त॒र॒ इ॒ते॒ त॒ पि॒त॒रः॒ । \newline
28. पि॒त॒रः॒ सो॒म्याः॒ सो॒म्याः॒ पि॒त॒रः॒ पि॒त॒रः॒ सो॒म्याः॒ । \newline
29. सो॒म्या॒ गं॒भी॒रैर् गं॑भी॒रैः सो᳚म्याः सोम्या गंभी॒रैः । \newline
30. गं॒भी॒रैः प॒थिभिः॑ प॒थिभि॑र् गंभी॒रैर् गं॑भी॒रैः प॒थिभिः॑ । \newline
31. प॒थिभिः॑ पू॒र्व्यैः पू॒र्व्यैः प॒थिभिः॑ प॒थिभिः॑ पू॒र्व्यैः । \newline
32. प॒थिभि॒रिति॑ प॒थि - भिः॒ । \newline
33. पू॒र्व्यैरिति॑ पू॒र्व्यैः । \newline
34. अथा॑ पि॒तॄन् पि॒तॄ नथाथा॑ पि॒तॄन् । \newline
35. पि॒तॄन् थ्सु॑वि॒दत्रा᳚न् थ्सुवि॒दत्रा᳚न् पि॒तॄन् पि॒तॄन् थ्सु॑वि॒दत्रान्॑ । \newline
36. सु॒वि॒दत्राꣳ॒॒ अप्यपि॑ सुवि॒दत्रा᳚न् थ्सुवि॒दत्राꣳ॒॒ अपि॑ । \newline
37. सु॒वि॒दत्रा॒निति॑ सु - वि॒दत्रान्॑ । \newline
38. अपी॑ते॒ ताप्यपी॑त । \newline
39. इ॒त॒ य॒मेन॑ य॒मेने॑ ते त य॒मेन॑ । \newline
40. य॒मेन॒ ये ये य॒मेन॑ य॒मेन॒ ये । \newline
41. ये स॑ध॒मादꣳ॑ सध॒मादं॒ ॅये ये स॑ध॒माद᳚म् । \newline
42. स॒ध॒माद॒म् मद॑न्ति॒ मद॑न्ति सध॒मादꣳ॑ सध॒माद॒म् मद॑न्ति । \newline
43. स॒ध॒माद॒मिति॑ सध - माद᳚म् । \newline
44. मद॒न्तीति॒ मद॑न्ति । \newline
45. मनो॒ नु नु मनो॒ मनो॒ नु । \newline
46. न्वा नु न्वा । \newline
47. आ हु॑वामहे हुवामह॒ आ हु॑वामहे । \newline
48. हु॒वा॒म॒हे॒ ना॒रा॒शꣳ॒॒सेन॑ नाराशꣳ॒॒सेन॑ हुवामहे हुवामहे नाराशꣳ॒॒सेन॑ । \newline
49. ना॒रा॒शꣳ॒॒सेन॒ स्तोमे॑न॒ स्तोमे॑न नाराशꣳ॒॒सेन॑ नाराशꣳ॒॒सेन॒ स्तोमे॑न । \newline
50. स्तोमे॑न पितृ॒णाम् पि॑तृ॒णाꣳ स्तोमे॑न॒ स्तोमे॑न पितृ॒णाम् । \newline
51. पि॒तृ॒णाम् च॑ च पितृ॒णाम् पि॑तृ॒णाम् च॑ । \newline
52. च॒ मन्म॑भि॒र् मन्म॑भिश्च च॒ मन्म॑भिः । \newline
53. मन्म॑भि॒रिति॒ मन्म॑ - भिः॒ । \newline
54. आ नो॑ न॒ आ नः॑ । \newline

\textbf{Ghana Paata } \newline

1. अक्ष॒न् नमी॑मद॒न्ता मी॑मद॒न्ता क्ष॒न् नक्ष॒न् नमी॑मदन्त॒ हि ह्यमी॑मद॒न्ता क्ष॒न् नक्ष॒न् नमी॑मदन्त॒ हि । \newline
2. अमी॑मदन्त॒ हि ह्यमी॑मद॒न्ता मी॑मदन्त॒ ह्यवाव॒ ह्यमी॑मद॒न्ता मी॑मदन्त॒ ह्यव॑ । \newline
3. ह्यवाव॒ हि ह्यव॑ प्रि॒याः प्रि॒या अव॒ हि ह्यव॑ प्रि॒याः । \newline
4. अव॑ प्रि॒याः प्रि॒या अवाव॑ प्रि॒या अ॑धूषता धूषत प्रि॒या अवाव॑ प्रि॒या अ॑धूषत । \newline
5. प्रि॒या अ॑धूषता धूषत प्रि॒याः प्रि॒या अ॑धूषत । \newline
6. अ॒धू॒ष॒तेत्य॑धूषत । \newline
7. अस्तो॑षत॒ स्वभा॑नवः॒ स्वभा॑न॒वो ऽस्तो॑ष॒ता स्तो॑षत॒ स्वभा॑नवो॒ विप्रा॒ विप्राः॒ स्वभा॑न॒वो ऽस्तो॑ष॒ता स्तो॑षत॒ स्वभा॑नवो॒ विप्राः᳚ । \newline
8. स्वभा॑नवो॒ विप्रा॒ विप्राः॒ स्वभा॑नवः॒ स्वभा॑नवो॒ विप्रा॒ नवि॑ष्ठया॒ नवि॑ष्ठया॒ विप्राः॒ स्वभा॑नवः॒ स्वभा॑नवो॒ विप्रा॒ नवि॑ष्ठया । \newline
9. स्वभा॑नव॒ इति॒ स्व - भा॒न॒वः॒ । \newline
10. विप्रा॒ नवि॑ष्ठया॒ नवि॑ष्ठया॒ विप्रा॒ विप्रा॒ नवि॑ष्ठया म॒ती म॒ती नवि॑ष्ठया॒ विप्रा॒ विप्रा॒ नवि॑ष्ठया म॒ती । \newline
11. नवि॑ष्ठया म॒ती म॒ती नवि॑ष्ठया॒ नवि॑ष्ठया म॒ती । \newline
12. म॒तीति॑ म॒ती । \newline
13. योजा॒ नु नु योजा॒ योजा॒ न्वि॑न्द्रे न्द्र॒ नु योजा॒ योजा॒ न्वि॑न्द्र । \newline
14. न्वि॑न्द्रे न्द्र॒ नु न्वि॑न्द्र ते त इन्द्र॒ नु न्वि॑न्द्र ते । \newline
15. इ॒न्द्र॒ ते॒ त॒ इ॒न्द्रे॒ न्द्र॒ ते॒ हरी॒ हरी॑ त इन्द्रे न्द्र ते॒ हरी᳚ । \newline
16. ते॒ हरी॒ हरी॑ ते ते॒ हरी᳚ । \newline
17. हरी॒ इति॒ हरी᳚ । \newline
18. अक्ष॑न् पि॒तरः॑ पि॒तरो ऽक्ष॒न् नक्ष॑न् पि॒तरो ऽमी॑मद॒न्ता मी॑मदन्त पि॒तरो ऽक्ष॒न् नक्ष॑न् पि॒तरो ऽमी॑मदन्त । \newline
19. पि॒तरो ऽमी॑मद॒न्ता मी॑मदन्त पि॒तरः॑ पि॒तरो ऽमी॑मदन्त पि॒तरः॑ पि॒तरो ऽमी॑मदन्त पि॒तरः॑ पि॒तरो ऽमी॑मदन्त पि॒तरः॑ । \newline
20. अमी॑मदन्त पि॒तरः॑ पि॒तरो ऽमी॑मद॒न्ता मी॑मदन्त पि॒तरो ऽती॑तृप॒न्ता ती॑तृपन्त पि॒तरो ऽमी॑मद॒न्ता मी॑मदन्त पि॒तरो ऽती॑तृपन्त । \newline
21. पि॒तरो ऽती॑तृप॒न्ता ती॑तृपन्त पि॒तरः॑ पि॒तरो ऽती॑तृपन्त पि॒तरः॑ पि॒तरो ऽती॑तृपन्त पि॒तरः॑ 
पि॒तरो ऽती॑तृपन्त पि॒तरः॑ । \newline
22. अती॑तृपन्त पि॒तरः॑ पि॒तरो ऽती॑तृप॒न्ता ती॑तृपन्त पि॒तरो ऽमी॑मृज॒न्ता मी॑मृजन्त पि॒तरो ऽती॑तृप॒न्ता ती॑तृपन्त पि॒तरो ऽमी॑मृजन्त । \newline
23. पि॒तरो ऽमी॑मृज॒न्ता मी॑मृजन्त पि॒तरः॑ पि॒तरो ऽमी॑मृजन्त पि॒तरः॑ पि॒तरो ऽमी॑मृजन्त पि॒तरः॑ पि॒तरो ऽमी॑मृजन्त पि॒तरः॑ । \newline
24. अमी॑मृजन्त पि॒तरः॑ पि॒तरो ऽमी॑मृज॒न्ता मी॑मृजन्त पि॒तरः॑ । \newline
25. पि॒तर॒ इति॑ पि॒तरः॑ । \newline
26. परे॑ते त॒ परा॒ परे॑त पितरः पितर इत॒ परा॒ परे॑त पितरः । \newline
27. इ॒त॒ पि॒त॒रः॒ पि॒त॒र॒ इ॒ते॒ त॒ पि॒त॒रः॒ सो॒म्याः॒ सो॒म्याः॒ पि॒त॒र॒ इ॒ते॒ त॒ पि॒त॒रः॒ सो॒म्याः॒ । \newline
28. पि॒त॒रः॒ सो॒म्याः॒ सो॒म्याः॒ पि॒त॒रः॒ पि॒त॒रः॒ सो॒म्या॒ गं॒भी॒रैर् गं॑भी॒रैः सो᳚म्याः पितरः पितरः सोम्या गंभी॒रैः । \newline
29. सो॒म्या॒ गं॒भी॒रैर् गं॑भी॒रैः सो᳚म्याः सोम्या गंभी॒रैः प॒थिभिः॑ प॒थिभि॑र् गंभी॒रैः सो᳚म्याः सोम्या गंभी॒रैः प॒थिभिः॑ । \newline
30. गं॒भी॒रैः प॒थिभिः॑ प॒थिभि॑र् गंभी॒रैर् गं॑भी॒रैः प॒थिभिः॑ पू॒र्व्यैः पू॒र्व्यैः प॒थिभि॑र् गंभी॒रैर् गं॑भी॒रैः प॒थिभिः॑ पू॒र्व्यैः । \newline
31. प॒थिभिः॑ पू॒र्व्यैः पू॒र्व्यैः प॒थिभिः॑ प॒थिभिः॑ पू॒र्व्यैः । \newline
32. प॒थिभि॒रिति॑ प॒थि - भिः॒ । \newline
33. पू॒र्व्यैरिति॑ पू॒र्व्यैः । \newline
34. अथा॑ पि॒तॄन् पि॒तॄ नथाथा॑ पि॒तॄन् थ्सु॑वि॒दत्रा᳚न् थ्सुवि॒दत्रा᳚न् पि॒तॄ नथाथा॑ पि॒तॄन् थ्सु॑वि॒दत्रान्॑ । \newline
35. पि॒तॄन् थ्सु॑वि॒दत्रा᳚न् थ्सुवि॒दत्रा᳚न् पि॒तॄन् पि॒तॄन् थ्सु॑वि॒दत्राꣳ॒॒ अप्यपि॑ सुवि॒दत्रा᳚न् पि॒तॄन् पि॒तॄन् थ्सु॑वि॒दत्राꣳ॒॒ अपि॑ । \newline
36. सु॒वि॒दत्राꣳ॒॒ अप्यपि॑ सुवि॒दत्रा᳚न् थ्सुवि॒दत्राꣳ॒॒ अपी॑ते॒ तापि॑ सुवि॒दत्रा᳚न् थ्सुवि॒दत्राꣳ॒॒ अपी॑त । \newline
37. सु॒वि॒दत्रा॒निति॑ सु - वि॒दत्रान्॑ । \newline
38. अपी॑ते॒ ताप्यपी॑त य॒मेन॑ य॒मेने॒ ताप्यपी॑त य॒मेन॑ । \newline
39. इ॒त॒ य॒मेन॑ य॒मेने॑ ते त य॒मेन॒ ये ये य॒मेने॑ ते त य॒मेन॒ ये । \newline
40. य॒मेन॒ ये ये य॒मेन॑ य॒मेन॒ ये स॑ध॒मादꣳ॑ सध॒मादं॒ ॅये य॒मेन॑ य॒मेन॒ ये स॑ध॒माद᳚म् । \newline
41. ये स॑ध॒मादꣳ॑ सध॒मादं॒ ॅये ये स॑ध॒माद॒म् मद॑न्ति॒ मद॑न्ति सध॒मादं॒ ॅये ये स॑ध॒माद॒म् मद॑न्ति । \newline
42. स॒ध॒माद॒म् मद॑न्ति॒ मद॑न्ति सध॒मादꣳ॑ सध॒माद॒म् मद॑न्ति । \newline
43. स॒ध॒माद॒मिति॑ सध - माद᳚म् । \newline
44. मद॒न्तीति॒ मद॑न्ति । \newline
45. मनो॒ नु नु मनो॒ मनो॒ न्वा नु मनो॒ मनो॒ न्वा । \newline
46. न्वा नु न्वा हु॑वामहे हुवामह॒ आ नु न्वा हु॑वामहे । \newline
47. आ हु॑वामहे हुवामह॒ आ हु॑वामहे नाराशꣳ॒॒सेन॑ नाराशꣳ॒॒सेन॑ हुवामह॒ आ हु॑वामहे नाराशꣳ॒॒सेन॑ । \newline
48. हु॒वा॒म॒हे॒ ना॒रा॒शꣳ॒॒सेन॑ नाराशꣳ॒॒सेन॑ हुवामहे हुवामहे नाराशꣳ॒॒सेन॒ स्तोमे॑न॒ स्तोमे॑न नाराशꣳ॒॒सेन॑ हुवामहे हुवामहे नाराशꣳ॒॒सेन॒ स्तोमे॑न । \newline
49. ना॒रा॒शꣳ॒॒सेन॒ स्तोमे॑न॒ स्तोमे॑न नाराशꣳ॒॒सेन॑ नाराशꣳ॒॒सेन॒ स्तोमे॑न पितृ॒णाम् पि॑तृ॒णाꣳ स्तोमे॑न नाराशꣳ॒॒सेन॑ नाराशꣳ॒॒सेन॒ स्तोमे॑न पितृ॒णाम् । \newline
50. स्तोमे॑न पितृ॒णाम् पि॑तृ॒णाꣳ स्तोमे॑न॒ स्तोमे॑न पितृ॒णाम् च॑ च पितृ॒णाꣳ स्तोमे॑न॒ स्तोमे॑न पितृ॒णाम् च॑ । \newline
51. पि॒तृ॒णाम् च॑ च पितृ॒णाम् पि॑तृ॒णाम् च॒ मन्म॑भि॒र् मन्म॑भिश्च पितृ॒णाम् पि॑तृ॒णाम् च॒ मन्म॑भिः । \newline
52. च॒ मन्म॑भि॒र् मन्म॑भि श्च च॒ मन्म॑भिः । \newline
53. मन्म॑भि॒रिति॒ मन्म॑ - भिः॒ । \newline
54. आ नो॑ न॒ आ न॑ एत्वेतु न॒ आ न॑ एतु । \newline
\pagebreak
\markright{ TS 1.8.5.3  \hfill https://www.vedavms.in \hfill}
\addcontentsline{toc}{section}{ TS 1.8.5.3 }
\section*{ TS 1.8.5.3 }

\textbf{TS 1.8.5.3 } \newline
\textbf{Samhita Paata} \newline

न॑ एतु॒ मनः॒ पुनः॒ क्रत्वे॒ दक्षा॑य जी॒वसे᳚ । ज्योक् च॒ सूर्यं॑ दृ॒शे ॥ पुन॑र्नः पि॒तरो॒ मनो॒ ददा॑तु॒ दैव्यो॒ जनः॑ । जी॒वं ॅव्रातꣳ॑ सचेमहि ॥ यद॒न्तरि॑क्षं पृथि॒वीमु॒त द्यां ॅयन्मा॒तरं॑ पि॒तरं॑ ॅवा जिहिꣳसि॒म । अ॒ग्निर् मा॒ तस्मा॒देन॑सो॒ गार्.ह॑पत्यः॒ प्र मु॑ञ्चतु दुरि॒ता यानि॑ चकृ॒म क॒रोतु॒ मा-म॑ने॒नसं᳚ ॥ \newline

\textbf{Pada Paata} \newline

न॒ । ए॒तु॒ । मनः॑ । पुनः॑ । क्रत्वे᳚ । दक्षा॑य । जी॒वसे᳚ ॥ ज्योक् । च॒ । सूर्य᳚म् । दृ॒शे ॥ पुनः॑ । नः॒ । पि॒तरः॑ । मनः॑ । ददा॑तु । दैव्यः॑ । जनः॑ ॥ जी॒वम् । व्रात᳚म् । स॒चे॒म॒हि॒ ॥ यत् । अ॒न्तरि॑क्षम् । पृ॒थि॒वीम् । उ॒त । द्याम् । यत् । मा॒तर᳚म् । पि॒तर᳚म् । वा॒ । जि॒हिꣳ॒॒सि॒म ॥ अ॒ग्निः । मा॒ । तस्मा᳚त् । एन॑सः । गार्.ह॑पत्य॒ इति॒ गार्.ह॑-प॒त्यः॒ । प्रेति॑ । मु॒ञ्च॒तु॒ । दु॒रि॒तेति॑ दुः - इ॒ता । यानि॑ । च॒कृ॒म । क॒रोतु॑ । माम् । अ॒ने॒नस᳚म् ॥  \newline



\textbf{Jatai Paata} \newline

1. न॒ ए॒त्वे॒तु॒ नो॒ न॒ ए॒तु॒ । \newline
2. ए॒तु॒ मनो॒ मन॑ एत्वेतु॒ मनः॑ । \newline
3. मनः॒ पुनः॒ पुन॒र् मनो॒ मनः॒ पुनः॑ । \newline
4. पुनः॒ क्रत्वे॒ क्रत्वे॒ पुनः॒ पुनः॒ क्रत्वे᳚ । \newline
5. क्रत्वे॒ दक्षा॑य॒ दक्षा॑य॒ क्रत्वे॒ क्रत्वे॒ दक्षा॑य । \newline
6. दक्षा॑य जी॒वसे॑ जी॒वसे॒ दक्षा॑य॒ दक्षा॑य जी॒वसे᳚ । \newline
7. जी॒वस॒ इति॑ जी॒वसे᳚ । \newline
8. ज्योक् च॑ च॒ ज्योग् ज्योक् च॑ । \newline
9. च॒ सूर्यꣳ॒॒ सूर्य॑म् च च॒ सूर्य᳚म् । \newline
10. सूर्य॑म् दृ॒शे दृ॒शे सूर्यꣳ॒॒ सूर्य॑म् दृ॒शे । \newline
11. दृ॒श इति॑ दृ॒शे । \newline
12. पुन॑र् नो नः॒ पुनः॒ पुन॑र् नः । \newline
13. नः॒ पि॒तरः॑ पि॒तरो॑ नो नः पि॒तरः॑ । \newline
14. पि॒तरो॒ मनो॒ मनः॑ पि॒तरः॑ पि॒तरो॒ मनः॑ । \newline
15. मनो॒ ददा॑तु॒ ददा॑तु॒ मनो॒ मनो॒ ददा॑तु । \newline
16. ददा॑तु॒ दैव्यो॒ दैव्यो॒ ददा॑तु॒ ददा॑तु॒ दैव्यः॑ । \newline
17. दैव्यो॒ जनो॒ जनो॒ दैव्यो॒ दैव्यो॒ जनः॑ । \newline
18. जन॒ इति॒ जनः॑ । \newline
19. जी॒वं ॅव्रातं॒ ॅव्रात॑म् जी॒वम् जी॒वं ॅव्रात᳚म् । \newline
20. व्रातꣳ॑ सचेमहि सचेमहि॒ व्रातं॒ ॅव्रातꣳ॑ सचेमहि । \newline
21. स॒चे॒म॒हीति॑ सचेमहि । \newline
22. यद॒न्तरि॑क्ष म॒न्तरि॑क्षं॒ ॅयद् यद॒न्तरि॑क्षम् । \newline
23. अ॒न्तरि॑क्षम् पृथि॒वीम् पृ॑थि॒वी म॒न्तरि॑क्ष म॒न्तरि॑क्षम् पृथि॒वीम् । \newline
24. पृ॒थि॒वी मु॒तोत पृ॑थि॒वीम् पृ॑थि॒वी मु॒त । \newline
25. उ॒त द्याम् द्या मु॒तोत द्याम् । \newline
26. द्यां ॅयद् यद् द्याम् द्यां ॅयत् । \newline
27. यन् मा॒तर॑म् मा॒तरं॒ ॅयद् यन् मा॒तर᳚म् । \newline
28. मा॒तर॑म् पि॒तर॑म् पि॒तर॑म् मा॒तर॑म् मा॒तर॑म् पि॒तर᳚म् । \newline
29. पि॒तरं॑ ॅवा वा पि॒तर॑म् पि॒तरं॑ ॅवा । \newline
30. वा॒ जि॒हिꣳ॒॒सि॒म जि॑हिꣳसि॒म वा॑ वा जिहिꣳसि॒म । \newline
31. जि॒हिꣳ॒॒सि॒मेति॑ जिहिꣳसि॒म । \newline
32. अ॒ग्निर् मा॑ मा॒ ऽग्नि र॒ग्निर् मा᳚ । \newline
33. मा॒ तस्मा॒त् तस्मा᳚न् मा मा॒ तस्मा᳚त् । \newline
34. तस्मा॒ देन॑स॒ एन॑स॒ स्तस्मा॒त् तस्मा॒ देन॑सः । \newline
35. एन॑सो॒ गार्.ह॑पत्यो॒ गार्.ह॑पत्य॒ एन॑स॒ एन॑सो॒ गार्.ह॑पत्यः । \newline
36. गार्.ह॑पत्यः॒ प्र प्र गार्.ह॑पत्यो॒ गार्.ह॑पत्यः॒ प्र । \newline
37. गार्.ह॑पत्य॒ इति॒ गार्.ह॑ - प॒त्यः॒ । \newline
38. प्र मु॑ञ्चतु मुञ्चतु॒ प्र प्र मु॑ञ्चतु । \newline
39. मु॒ञ्च॒तु॒ दु॒रि॒ता दु॑रि॒ता मु॑ञ्चतु मुञ्चतु दुरि॒ता । \newline
40. दु॒रि॒ता यानि॒ यानि॑ दुरि॒ता दु॑रि॒ता यानि॑ । \newline
41. दु॒रि॒तेति॑ दुः - इ॒ता । \newline
42. यानि॑ चकृ॒म च॑कृ॒म यानि॒ यानि॑ चकृ॒म । \newline
43. च॒कृ॒म क॒रोतु॑ क॒रोतु॑ चकृ॒म च॑कृ॒म क॒रोतु॑ । \newline
44. क॒रोतु॒ माम् माम् क॒रोतु॑ क॒रोतु॒ माम् । \newline
45. मा म॑ने॒नस॑ मने॒नस॒म् माम् मा म॑ने॒नस᳚म् । \newline
46. अ॒ने॒नस॒मित्य॑ने॒नस᳚म् । \newline

\textbf{Ghana Paata } \newline

1. न॒ ए॒त्वे॒तु॒ नो॒ न॒ ए॒तु॒ मनो॒ मन॑ एतु नो न एतु॒ मनः॑ । \newline
2. ए॒तु॒ मनो॒ मन॑ एत्वेतु॒ मनः॒ पुनः॒ पुन॒र् मन॑ एत्वेतु॒ मनः॒ पुनः॑ । \newline
3. मनः॒ पुनः॒ पुन॒र् मनो॒ मनः॒ पुनः॒ क्रत्वे॒ क्रत्वे॒ पुन॒र् मनो॒ मनः॒ पुनः॒ क्रत्वे᳚ । \newline
4. पुनः॒ क्रत्वे॒ क्रत्वे॒ पुनः॒ पुनः॒ क्रत्वे॒ दक्षा॑य॒ दक्षा॑य॒ क्रत्वे॒ पुनः॒ पुनः॒ क्रत्वे॒ दक्षा॑य । \newline
5. क्रत्वे॒ दक्षा॑य॒ दक्षा॑य॒ क्रत्वे॒ क्रत्वे॒ दक्षा॑य जी॒वसे॑ जी॒वसे॒ दक्षा॑य॒ क्रत्वे॒ क्रत्वे॒ दक्षा॑य जी॒वसे᳚ । \newline
6. दक्षा॑य जी॒वसे॑ जी॒वसे॒ दक्षा॑य॒ दक्षा॑य जी॒वसे᳚ । \newline
7. जी॒वस॒ इति॑ जी॒वसे᳚ । \newline
8. ज्योक् च॑ च॒ ज्योग् ज्योक् च॒ सूर्यꣳ॒॒ सूर्य॑म् च॒ ज्योग् ज्योक् च॒ सूर्य᳚म् । \newline
9. च॒ सूर्यꣳ॒॒ सूर्य॑म् च च॒ सूर्य॑म् दृ॒शे दृ॒शे सूर्य॑म् च च॒ सूर्य॑म् दृ॒शे । \newline
10. सूर्य॑म् दृ॒शे दृ॒शे सूर्यꣳ॒॒ सूर्य॑म् दृ॒शे । \newline
11. दृ॒श इति॑ दृ॒शे । \newline
12. पुन॑र् नो नः॒ पुनः॒ पुन॑र् नः पि॒तरः॑ पि॒तरो॑ नः॒ पुनः॒ पुन॑र् नः पि॒तरः॑ । \newline
13. नः॒ पि॒तरः॑ पि॒तरो॑ नो नः पि॒तरो॒ मनो॒ मनः॑ पि॒तरो॑ नो नः पि॒तरो॒ मनः॑ । \newline
14. पि॒तरो॒ मनो॒ मनः॑ पि॒तरः॑ पि॒तरो॒ मनो॒ ददा॑तु॒ ददा॑तु॒ मनः॑ पि॒तरः॑ पि॒तरो॒ मनो॒ ददा॑तु । \newline
15. मनो॒ ददा॑तु॒ ददा॑तु॒ मनो॒ मनो॒ ददा॑तु॒ दैव्यो॒ दैव्यो॒ ददा॑तु॒ मनो॒ मनो॒ ददा॑तु॒ दैव्यः॑ । \newline
16. ददा॑तु॒ दैव्यो॒ दैव्यो॒ ददा॑तु॒ ददा॑तु॒ दैव्यो॒ जनो॒ जनो॒ दैव्यो॒ ददा॑तु॒ ददा॑तु॒ दैव्यो॒ जनः॑ । \newline
17. दैव्यो॒ जनो॒ जनो॒ दैव्यो॒ दैव्यो॒ जनः॑ । \newline
18. जन॒ इति॒ जनः॑ । \newline
19. जी॒वं ॅव्रातं॒ ॅव्रात॑म् जी॒वम् जी॒वं ॅव्रातꣳ॑ सचेमहि सचेमहि॒ व्रात॑म् जी॒वम् जी॒वं ॅव्रातꣳ॑ सचेमहि । \newline
20. व्रातꣳ॑ सचेमहि सचेमहि॒ व्रातं॒ ॅव्रातꣳ॑ सचेमहि । \newline
21. स॒चे॒म॒हीति॑ सचेमहि । \newline
22. यद॒न्तरि॑क्ष म॒न्तरि॑क्षं॒ ॅयद् यद॒न्तरि॑क्षम् पृथि॒वीम् पृ॑थि॒वी म॒न्तरि॑क्षं॒ ॅयद् यद॒न्तरि॑क्षम् पृथि॒वीम् । \newline
23. अ॒न्तरि॑क्षम् पृथि॒वीम् पृ॑थि॒वी म॒न्तरि॑क्ष म॒न्तरि॑क्षम् पृथि॒वी मु॒तोत पृ॑थि॒वी म॒न्तरि॑क्ष म॒न्तरि॑क्षम् पृथि॒वी मु॒त । \newline
24. पृ॒थि॒वी मु॒तोत पृ॑थि॒वीम् पृ॑थि॒वी मु॒त द्याम् द्या मु॒त पृ॑थि॒वीम् पृ॑थि॒वी मु॒त द्याम् । \newline
25. उ॒त द्याम् द्या मु॒तोत द्यां ॅयद् यद् द्या मु॒तोत द्यां ॅयत् । \newline
26. द्यां ॅयद् यद् द्याम् द्यां ॅयन् मा॒तर॑म् मा॒तरं॒ ॅयद् द्याम् द्यां ॅयन् मा॒तर᳚म् । \newline
27. यन् मा॒तर॑म् मा॒तरं॒ ॅयद् यन् मा॒तर॑म् पि॒तर॑म् पि॒तर॑म् मा॒तरं॒ ॅयद् यन् मा॒तर॑म् पि॒तर᳚म् । \newline
28. मा॒तर॑म् पि॒तर॑म् पि॒तर॑म् मा॒तर॑म् मा॒तर॑म् पि॒तरं॑ ॅवा वा पि॒तर॑म् मा॒तर॑म् मा॒तर॑म् पि॒तरं॑ ॅवा । \newline
29. पि॒तरं॑ ॅवा वा पि॒तर॑म् पि॒तरं॑ ॅवा जिहिꣳसि॒म जि॑हिꣳसि॒म वा॑ पि॒तर॑म् पि॒तरं॑ ॅवा जिहिꣳसि॒म । \newline
30. वा॒ जि॒हिꣳ॒॒सि॒म जि॑हिꣳसि॒म वा॑ वा जिहिꣳसि॒म । \newline
31. जि॒हिꣳ॒॒सि॒मेति॑ जिहिꣳसि॒म । \newline
32. अ॒ग्निर् मा॑ मा॒ ऽग्नि र॒ग्निर् मा॒ तस्मा॒त् तस्मा᳚न् मा॒ ऽग्नि र॒ग्निर् मा॒ तस्मा᳚त् । \newline
33. मा॒ तस्मा॒त् तस्मा᳚न् मा मा॒ तस्मा॒ देन॑स॒ एन॑स॒ स्तस्मा᳚न् मा मा॒ तस्मा॒ देन॑सः । \newline
34. तस्मा॒ देन॑स॒ एन॑स॒ स्तस्मा॒त् तस्मा॒ देन॑सो॒ गार्.ह॑पत्यो॒ गार्.ह॑पत्य॒ एन॑स॒ स्तस्मा॒त् तस्मा॒ देन॑सो॒ गार्.ह॑पत्यः । \newline
35. एन॑सो॒ गार्.ह॑पत्यो॒ गार्.ह॑पत्य॒ एन॑स॒ एन॑सो॒ गार्.ह॑पत्यः॒ प्र प्र गार्.ह॑पत्य॒ एन॑स॒ एन॑सो॒ गार्.ह॑पत्यः॒ प्र । \newline
36. गार्.ह॑पत्यः॒ प्र प्र गार्.ह॑पत्यो॒ गार्.ह॑पत्यः॒ प्र मु॑ञ्चतु मुञ्चतु॒ प्र गार्.ह॑पत्यो॒ गार्.ह॑पत्यः॒ प्र मु॑ञ्चतु । \newline
37. गार्.ह॑पत्य॒ इति॒ गार्.ह॑ - प॒त्यः॒ । \newline
38. प्र मु॑ञ्चतु मुञ्चतु॒ प्र प्र मु॑ञ्चतु दुरि॒ता दु॑रि॒ता मु॑ञ्चतु॒ प्र प्र मु॑ञ्चतु दुरि॒ता । \newline
39. मु॒ञ्च॒तु॒ दु॒रि॒ता दु॑रि॒ता मु॑ञ्चतु मुञ्चतु दुरि॒ता यानि॒ यानि॑ दुरि॒ता मु॑ञ्चतु मुञ्चतु दुरि॒ता यानि॑ । \newline
40. दु॒रि॒ता यानि॒ यानि॑ दुरि॒ता दु॑रि॒ता यानि॑ चकृ॒म च॑कृ॒म यानि॑ दुरि॒ता दु॑रि॒ता यानि॑ चकृ॒म । \newline
41. दु॒रि॒तेति॑ दुः - इ॒ता । \newline
42. यानि॑ चकृ॒म च॑कृ॒म यानि॒ यानि॑ चकृ॒म क॒रोतु॑ क॒रोतु॑ चकृ॒म यानि॒ यानि॑ चकृ॒म क॒रोतु॑ । \newline
43. च॒कृ॒म क॒रोतु॑ क॒रोतु॑ चकृ॒म च॑कृ॒म क॒रोतु॒ माम् माम् क॒रोतु॑ चकृ॒म च॑कृ॒म क॒रोतु॒ माम् । \newline
44. क॒रोतु॒ माम् माम् क॒रोतु॑ क॒रोतु॒ मा म॑ने॒नस॑ मने॒नस॒म् माम् क॒रोतु॑ क॒रोतु॒ मा म॑ने॒नस᳚म् । \newline
45. मा म॑ने॒नस॑ मने॒नस॒म् माम् मा म॑ने॒नस᳚म् । \newline
46. अ॒ने॒नस॒मित्य॑ने॒नस᳚म् । \newline
\pagebreak
\markright{ TS 1.8.6.1  \hfill https://www.vedavms.in \hfill}
\addcontentsline{toc}{section}{ TS 1.8.6.1 }
\section*{ TS 1.8.6.1 }

\textbf{TS 1.8.6.1 } \newline
\textbf{Samhita Paata} \newline

प्र॒ति॒पू॒रु॒षमेक॑कपाला॒न् निर्व॑प॒त्येक॒-मति॑रिक्तं॒ ॅयाव॑न्तो गृ॒ह्याः᳚ स्मस्तेभ्यः॒ कम॑करं पशू॒नाꣳ शर्मा॑सि॒ शर्म॒ यज॑मानस्य॒ शर्म॑ मे य॒च्छैक॑ ए॒व रु॒द्रो न द्वि॒तीया॑य तस्थ आ॒खुस्ते॑ रुद्र प॒शुस्तं जु॑षस्वै॒ष ते॑ रुद्र भा॒गः स॒ह स्वस्राऽंबि॑कया॒ तं जु॑षस्व भेष॒जं गवेऽश्वा॑य॒ पुरु॑षाय भेष॒जमथो॑ अ॒स्मभ्यं॑ भेष॒जꣳ सुभे॑षजं॒ - [ ] \newline

\textbf{Pada Paata} \newline

प्र॒ति॒पू॒रु॒षमिति॑ प्रति-पू॒रु॒षम् । एक॑कपाला॒नित्येक॑-क॒पा॒ला॒न् । निरिति॑ । व॒प॒ति॒ । एक᳚म् । अति॑रिक्त॒मित्यति॑-रि॒क्त॒म् । याव॑न्तः । गृ॒ह्याः᳚ । स्मः । तेभ्यः॑ । कम् । अ॒क॒र॒म् । प॒शू॒नाम् । शर्म॑ । अ॒सि॒ । शर्म॑ । यज॑मानस्य । शर्म॑ । मे॒ । य॒च्छ॒ । एकः॑ । ए॒व । रु॒द्रः । न । द्वि॒तीया॑य । त॒स्थे॒ । आ॒खुः । ते॒ । रु॒द्र॒ । प॒शुः । तम् । जु॒ष॒स्व॒ । ए॒षः । ते॒ । रु॒द्र॒ । भा॒गः । स॒ह । स्वस्रा᳚ । अबिं॑कया । तम् । जु॒ष॒स्व॒ । भे॒ष॒जम् । गवे᳚ । अश्वा॑य । पुरु॑षाय । भे॒ष॒जम् । अथो॒ इति॑ । अ॒स्मभ्य॒मित्य॒स्म-भ्य॒म् । भे॒ष॒जम् । सुभे॑षज॒मिति॒ सु - भे॒ष॒ज॒म् ।  \newline



\textbf{Jatai Paata} \newline

1. प्र॒ति॒पू॒रु॒ष मेक॑कपाला॒ नेक॑कपालान् प्रतिपूरु॒षम् प्र॑तिपूरु॒ष मेक॑कपालान् । \newline
2. प्र॒ति॒पू॒रु॒षमिति॑ प्रति - पू॒रु॒षम् । \newline
3. एक॑कपाला॒न् निर् णिरेक॑कपाला॒ नेक॑कपाला॒न् निः । \newline
4. एक॑कपाला॒नित्येक॑ - क॒पा॒ला॒न् । \newline
5. निर् व॑पति वपति॒ निर् णिर् व॑पति । \newline
6. व॒प॒त्येक॒ मेकं॑ ॅवपति वप॒त्येक᳚म् । \newline
7. एक॒ मति॑रिक्त॒ मति॑रिक्त॒ मेक॒ मेक॒ मति॑रिक्तम् । \newline
8. अति॑रिक्तं॒ ॅयाव॑न्तो॒ याव॒न्तो ऽति॑रिक्त॒ मति॑रिक्तं॒ ॅयाव॑न्तः । \newline
9. अति॑रिक्त॒मित्यति॑ - रि॒क्त॒म् । \newline
10. याव॑न्तो गृ॒ह्या॑ गृ॒ह्या॑ याव॑न्तो॒ याव॑न्तो गृ॒ह्याः᳚ । \newline
11. गृ॒ह्याः᳚ स्मः स्मो गृ॒ह्या॑ गृ॒ह्याः᳚ स्मः । \newline
12. स्म स्तेभ्य॒ स्तेभ्यः॒ स्मः स्म स्तेभ्यः॑ । \newline
13. तेभ्यः॒ कम् कम् तेभ्य॒ स्तेभ्यः॒ कम् । \newline
14. क म॑कर मकर॒म् कम् क म॑करम् । \newline
15. अ॒क॒र॒म् प॒शू॒नाम् प॑शू॒ना म॑कर मकरम् पशू॒नाम् । \newline
16. प॒शू॒नाꣳ शर्म॒ शर्म॑ पशू॒नाम् प॑शू॒नाꣳ शर्म॑ । \newline
17. शर्मा᳚स्यसि॒ शर्म॒ शर्मा॑सि । \newline
18. अ॒सि॒ शर्म॒ शर्मा᳚स्यसि॒ शर्म॑ । \newline
19. शर्म॒ यज॑मानस्य॒ यज॑मानस्य॒ शर्म॒ शर्म॒ यज॑मानस्य । \newline
20. यज॑मानस्य॒ शर्म॒ शर्म॒ यज॑मानस्य॒ यज॑मानस्य॒ शर्म॑ । \newline
21. शर्म॑ मे मे॒ शर्म॒ शर्म॑ मे । \newline
22. मे॒ य॒च्छ॒ य॒च्छ॒ मे॒ मे॒ य॒च्छ॒ । \newline
23. य॒च्छैक॒ एको॑ यच्छ य॒च्छैकः॑ । \newline
24. एक॑ ए॒वैवैक॒ एक॑ ए॒व । \newline
25. ए॒व रु॒द्रो रु॒द्र ए॒वैव रु॒द्रः । \newline
26. रु॒द्रो न न रु॒द्रो रु॒द्रो न । \newline
27. न द्वि॒तीया॑य द्वि॒तीया॑य॒ न न द्वि॒तीया॑य । \newline
28. द्वि॒तीया॑य तस्थे तस्थे द्वि॒तीया॑य द्वि॒तीया॑य तस्थे । \newline
29. त॒स्थ॒ आ॒खु रा॒खु स्त॑स्थे तस्थ आ॒खुः । \newline
30. आ॒खुस्ते॑ त आ॒खु रा॒खुस्ते᳚ । \newline
31. ते॒ रु॒द्र॒ रु॒द्र॒ ते॒ ते॒ रु॒द्र॒ । \newline
32. रु॒द्र॒ प॒शुः प॒शू रु॑द्र रुद्र प॒शुः । \newline
33. प॒शु स्तम् तम् प॒शुः प॒शु स्तम् । \newline
34. तम् जु॑षस्व जुषस्व॒ तम् तम् जु॑षस्व । \newline
35. जु॒ष॒स्वै॒ष ए॒ष जु॑षस्व जुषस्वै॒षः । \newline
36. ए॒ष ते॑ त ए॒ष ए॒ष ते᳚ । \newline
37. ते॒ रु॒द्र॒ रु॒द्र॒ ते॒ ते॒ रु॒द्र॒ । \newline
38. रु॒द्र॒ भा॒गो भा॒गो रु॑द्र रुद्र भा॒गः । \newline
39. भा॒गः स॒ह स॒ह भा॒गो भा॒गः स॒ह । \newline
40. स॒ह स्वस्रा॒ स्वस्रा॑ स॒ह स॒ह स्वस्रा᳚ । \newline
41. स्वस्रा म्बि॑क॒या म्बि॑कया॒ स्वस्रा॒ स्वस्रा म्बि॑कया । \newline
42. अम्बि॑कया॒ तम् त मम्बि॑क॒या म्बि॑कया॒ तम् । \newline
43. तम् जु॑षस्व जुषस्व॒ तम् तम् जु॑षस्व । \newline
44. जु॒ष॒स्व॒ भे॒ष॒जम् भे॑ष॒जम् जु॑षस्व जुषस्व भेष॒जम् । \newline
45. भे॒ष॒जम् गवे॒ गवे॑ भेष॒जम् भे॑ष॒जम् गवे᳚ । \newline
46. गवे ऽश्वा॒या श्वा॑य॒ गवे॒ गवे ऽश्वा॑य । \newline
47. अश्वा॑य॒ पुरु॑षाय॒ पुरु॑षा॒या श्वा॒या श्वा॑य॒ पुरु॑षाय । \newline
48. पुरु॑षाय भेष॒जम् भे॑ष॒जम् पुरु॑षाय॒ पुरु॑षाय भेष॒जम् । \newline
49. भे॒ष॒ज मथो॒ अथो॑ भेष॒जम् भे॑ष॒ज मथो᳚ । \newline
50. अथो॑ अ॒स्मभ्य॑ म॒स्मभ्य॒ मथो॒ अथो॑ अ॒स्मभ्य᳚म् । \newline
51. अथो॒ इत्यथो᳚ । \newline
52. अ॒स्मभ्य॑म् भेष॒जम् भे॑ष॒ज म॒स्मभ्य॑ म॒स्मभ्य॑म् भेष॒जम् । \newline
53. अ॒स्मभ्य॒मित्य॒स्म - भ्य॒म् । \newline
54. भे॒ष॒जꣳ सुभे॑षजꣳ॒॒ सुभे॑षजम् भेष॒जम् भे॑ष॒जꣳ सुभे॑षजम् । \newline
55. सुभे॑षजं॒ ॅयथा॒ यथा॒ सुभे॑षजꣳ॒॒ सुभे॑षजं॒ ॅयथा᳚ । \newline
56. सुभे॑षज॒मिति॒ सु - भे॒ष॒ज॒म् । \newline

\textbf{Ghana Paata } \newline

1. प्र॒ति॒पू॒रु॒ष मेक॑कपाला॒ नेक॑कपालान् प्रतिपूरु॒षम् प्र॑तिपूरु॒ष मेक॑कपाला॒न् निर् णिरेक॑कपालान् प्रतिपूरु॒षम् प्र॑तिपूरु॒ष मेक॑कपाला॒न् निः । \newline
2. प्र॒ति॒पू॒रु॒षमिति॑ प्रति - पू॒रु॒षम् । \newline
3. एक॑कपाला॒न् निर् णिरेक॑कपाला॒ नेक॑कपाला॒न् निर् व॑पति वपति॒ निरेक॑कपाला॒ नेक॑कपाला॒न् निर् व॑पति । \newline
4. एक॑कपाला॒नित्येक॑ - क॒पा॒ला॒न् । \newline
5. निर् व॑पति वपति॒ निर् णिर् व॑प॒ त्येक॒ मेकं॑ ॅवपति॒ निर् णिर् व॑प॒ त्येक᳚म् । \newline
6. व॒प॒ त्येक॒ मेकं॑ ॅवपति वप॒ त्येक॒ मति॑रिक्त॒ मति॑रिक्त॒ मेकं॑ ॅवपति वप॒ त्येक॒ मति॑रिक्तम् । \newline
7. एक॒ मति॑रिक्त॒ मति॑रिक्त॒ मेक॒ मेक॒ मति॑रिक्तं॒ ॅयाव॑न्तो॒ याव॒न्तो ऽति॑रिक्त॒ मेक॒ मेक॒ मति॑रिक्तं॒ ॅयाव॑न्तः । \newline
8. अति॑रिक्तं॒ ॅयाव॑न्तो॒ याव॒न्तो ऽति॑रिक्त॒ मति॑रिक्तं॒ ॅयाव॑न्तो गृ॒ह्या॑ गृ॒ह्या॑ याव॒न्तो ऽति॑रिक्त॒ मति॑रिक्तं॒ ॅयाव॑न्तो गृ॒ह्याः᳚ । \newline
9. अति॑रिक्त॒मित्यति॑ - रि॒क्त॒म् । \newline
10. याव॑न्तो गृ॒ह्या॑ गृ॒ह्या॑ याव॑न्तो॒ याव॑न्तो गृ॒ह्याः᳚ स्मः स्मो गृ॒ह्या॑ याव॑न्तो॒ याव॑न्तो गृ॒ह्याः᳚ स्मः । \newline
11. गृ॒ह्याः᳚ स्मः स्मो गृ॒ह्या॑ गृ॒ह्याः᳚ स्म स्तेभ्य॒ स्तेभ्यः॒ स्मो गृ॒ह्या॑ गृ॒ह्याः᳚ स्म स्तेभ्यः॑ । \newline
12. स्म स्तेभ्य॒ स्तेभ्यः॒ स्मः स्म स्तेभ्यः॒ कम् कम् तेभ्यः॒ स्मः स्म स्तेभ्यः॒ कम् । \newline
13. तेभ्यः॒ कम् कम् तेभ्य॒ स्तेभ्यः॒ क म॑कर मकर॒म् कम् तेभ्य॒ स्तेभ्यः॒ क म॑करम् । \newline
14. क म॑कर मकर॒म् कम् क म॑करम् पशू॒नाम् प॑शू॒ना म॑कर॒म् कम् क म॑करम् पशू॒नाम् । \newline
15. अ॒क॒र॒म् प॒शू॒नाम् प॑शू॒ना म॑कर मकरम् पशू॒नाꣳ शर्म॒ शर्म॑ पशू॒ना म॑कर मकरम् पशू॒नाꣳ शर्म॑ । \newline
16. प॒शू॒नाꣳ शर्म॒ शर्म॑ पशू॒नाम् प॑शू॒नाꣳ शर्मा᳚स्यसि॒ शर्म॑ पशू॒नाम् प॑शू॒नाꣳ शर्मा॑सि । \newline
17. शर्मा᳚स्यसि॒ शर्म॒ शर्मा॑सि॒ शर्म॒ शर्मा॑सि॒ शर्म॒ शर्मा॑सि॒ शर्म॑ । \newline
18. अ॒सि॒ शर्म॒ शर्मा᳚स्यसि॒ शर्म॒ यज॑मानस्य॒ यज॑मानस्य॒ शर्मा᳚स्यसि॒ शर्म॒ यज॑मानस्य । \newline
19. शर्म॒ यज॑मानस्य॒ यज॑मानस्य॒ शर्म॒ शर्म॒ यज॑मानस्य॒ शर्म॒ शर्म॒ यज॑मानस्य॒ शर्म॒ शर्म॒ यज॑मानस्य॒ शर्म॑ । \newline
20. यज॑मानस्य॒ शर्म॒ शर्म॒ यज॑मानस्य॒ यज॑मानस्य॒ शर्म॑ मे मे॒ शर्म॒ यज॑मानस्य॒ यज॑मानस्य॒ शर्म॑ मे । \newline
21. शर्म॑ मे मे॒ शर्म॒ शर्म॑ मे यच्छ यच्छ मे॒ शर्म॒ शर्म॑ मे यच्छ । \newline
22. मे॒ य॒च्छ॒ य॒च्छ॒ मे॒ मे॒ य॒च्छैक॒ एको॑ यच्छ मे मे य॒च्छैकः॑ । \newline
23. य॒च्छैक॒ एको॑ यच्छ य॒च्छैक॑ ए॒वै वैको॑ यच्छ य॒च्छैक॑ ए॒व । \newline
24. एक॑ ए॒वै वैक॒ एक॑ ए॒व रु॒द्रो रु॒द्र ए॒वैक॒ एक॑ ए॒व रु॒द्रः । \newline
25. ए॒व रु॒द्रो रु॒द्र ए॒वैव रु॒द्रो न न रु॒द्र ए॒वैव रु॒द्रो न । \newline
26. रु॒द्रो न न रु॒द्रो रु॒द्रो न द्वि॒तीया॑य द्वि॒तीया॑य॒ न रु॒द्रो रु॒द्रो न द्वि॒तीया॑य । \newline
27. न द्वि॒तीया॑य द्वि॒तीया॑य॒ न न द्वि॒तीया॑य तस्थे तस्थे द्वि॒तीया॑य॒ न न द्वि॒तीया॑य तस्थे । \newline
28. द्वि॒तीया॑य तस्थे तस्थे द्वि॒तीया॑य द्वि॒तीया॑य तस्थ आ॒खु रा॒खु स्त॑स्थे द्वि॒तीया॑य द्वि॒तीया॑य तस्थ आ॒खुः । \newline
29. त॒स्थ॒ आ॒खु रा॒खु स्त॑स्थे तस्थ आ॒खु स्ते॑ त आ॒खु स्त॑स्थे तस्थ आ॒खु स्ते᳚ । \newline
30. आ॒खु स्ते॑ त आ॒खु रा॒खु स्ते॑ रुद्र रुद्र त आ॒खु रा॒खु स्ते॑ रुद्र । \newline
31. ते॒ रु॒द्र॒ रु॒द्र॒ ते॒ ते॒ रु॒द्र॒ प॒शुः प॒शू रु॑द्र ते ते रुद्र प॒शुः । \newline
32. रु॒द्र॒ प॒शुः प॒शू रु॑द्र रुद्र प॒शु स्तम् तम् प॒शू रु॑द्र रुद्र प॒शु स्तम् । \newline
33. प॒शु स्तम् तम् प॒शुः प॒शु स्तम् जु॑षस्व जुषस्व॒ तम् प॒शुः प॒शु स्तम् जु॑षस्व । \newline
34. तम् जु॑षस्व जुषस्व॒ तम् तम् जु॑षस्वै॒ष ए॒ष जु॑षस्व॒ तम् तम् जु॑षस्वै॒षः । \newline
35. जु॒ष॒ स्वै॒ष ए॒ष जु॑षस्व जुष स्वै॒ष ते॑ त ए॒ष जु॑षस्व जुष स्वै॒ष ते᳚ । \newline
36. ए॒ष ते॑ त ए॒ष ए॒ष ते॑ रुद्र रुद्र त ए॒ष ए॒ष ते॑ रुद्र । \newline
37. ते॒ रु॒द्र॒ रु॒द्र॒ ते॒ ते॒ रु॒द्र॒ भा॒गो भा॒गो रु॑द्र ते ते रुद्र भा॒गः । \newline
38. रु॒द्र॒ भा॒गो भा॒गो रु॑द्र रुद्र भा॒गः स॒ह स॒ह भा॒गो रु॑द्र रुद्र भा॒गः स॒ह । \newline
39. भा॒गः स॒ह स॒ह भा॒गो भा॒गः स॒ह स्वस्रा॒ स्वस्रा॑ स॒ह भा॒गो भा॒गः स॒ह स्वस्रा᳚ । \newline
40. स॒ह स्वस्रा॒ स्वस्रा॑ स॒ह स॒ह स्वस्रा ऽम्बि॑क॒या ऽम्बि॑कया॒ स्वस्रा॑ स॒ह स॒ह स्वस्रा ऽम्बि॑कया । \newline
41. स्वस्रा ऽम्बि॑क॒या ऽम्बि॑कया॒ स्वस्रा॒ स्वस्रा ऽम्बि॑कया॒ तम् त मम्बि॑कया॒ स्वस्रा॒ स्वस्रा ऽम्बि॑कया॒ तम् । \newline
42. अम्बि॑कया॒ तम् त मम्बि॑क॒या ऽम्बि॑कया॒ तम् जु॑षस्व जुषस्व॒ त मम्बि॑क॒या ऽम्बि॑कया॒ तम् जु॑षस्व । \newline
43. तम् जु॑षस्व जुषस्व॒ तम् तम् जु॑षस्व भेष॒जम् भे॑ष॒जम् जु॑षस्व॒ तम् तम् जु॑षस्व भेष॒जम् । \newline
44. जु॒ष॒स्व॒ भे॒ष॒जम् भे॑ष॒जम् जु॑षस्व जुषस्व भेष॒जम् गवे॒ गवे॑ भेष॒जम् जु॑षस्व जुषस्व भेष॒जम् गवे᳚ । \newline
45. भे॒ष॒जम् गवे॒ गवे॑ भेष॒जम् भे॑ष॒जम् गवे ऽश्वा॒या श्वा॑य॒ गवे॑ भेष॒जम् भे॑ष॒जम् गवे ऽश्वा॑य । \newline
46. गवे ऽश्वा॒या श्वा॑य॒ गवे॒ गवे ऽश्वा॑य॒ पुरु॑षाय॒ पुरु॑षा॒ याश्वा॑य॒ गवे॒ गवे ऽश्वा॑य॒ पुरु॑षाय । \newline
47. अश्वा॑य॒ पुरु॑षाय॒ पुरु॑षा॒ याश्वा॒ याश्वा॑य॒ पुरु॑षाय भेष॒जम् भे॑ष॒जम् पुरु॑षा॒या श्वा॒या श्वा॑य॒ पुरु॑षाय भेष॒जम् । \newline
48. पुरु॑षाय भेष॒जम् भे॑ष॒जम् पुरु॑षाय॒ पुरु॑षाय भेष॒ज मथो॒ अथो॑ भेष॒जम् पुरु॑षाय॒ पुरु॑षाय भेष॒ज मथो᳚ । \newline
49. भे॒ष॒ज मथो॒ अथो॑ भेष॒जम् भे॑ष॒ज मथो॑ अ॒स्मभ्य॑ म॒स्मभ्य॒ मथो॑ भेष॒जम् भे॑ष॒ज मथो॑ अ॒स्मभ्य᳚म् । \newline
50. अथो॑ अ॒स्मभ्य॑ म॒स्मभ्य॒ मथो॒ अथो॑ अ॒स्मभ्य॑म् भेष॒जम् भे॑ष॒ज म॒स्मभ्य॒ मथो॒ अथो॑ अ॒स्मभ्य॑म् भेष॒जम् । \newline
51. अथो॒ इत्यथो᳚ । \newline
52. अ॒स्मभ्य॑म् भेष॒जम् भे॑ष॒ज म॒स्मभ्य॑ म॒स्मभ्य॑म् भेष॒जꣳ सुभे॑षजꣳ॒॒ सुभे॑षजम् भेष॒ज म॒स्मभ्य॑ म॒स्मभ्य॑म् भेष॒जꣳ सुभे॑षजम् । \newline
53. अ॒स्मभ्य॒मित्य॒स्म - भ्य॒म् । \newline
54. भे॒ष॒जꣳ सुभे॑षजꣳ॒॒ सुभे॑षजम् भेष॒जम् भे॑ष॒जꣳ सुभे॑षजं॒ ॅयथा॒ यथा॒ सुभे॑षजम् भेष॒जम् भे॑ष॒जꣳ सुभे॑षजं॒ ॅयथा᳚ । \newline
55. सुभे॑षजं॒ ॅयथा॒ यथा॒ सुभे॑षजꣳ॒॒ सुभे॑षजं॒ ॅयथा ऽस॒त्यस॑ति॒ यथा॒ सुभे॑षजꣳ॒॒ सुभे॑षजं॒ ॅयथा ऽस॑ति । \newline
56. सुभे॑षज॒मिति॒ सु - भे॒ष॒ज॒म् । \newline
\pagebreak
\markright{ TS 1.8.6.2  \hfill https://www.vedavms.in \hfill}
\addcontentsline{toc}{section}{ TS 1.8.6.2 }
\section*{ TS 1.8.6.2 }

\textbf{TS 1.8.6.2 } \newline
\textbf{Samhita Paata} \newline

ॅयथाऽस॑ति । सु॒गं मे॒षाय॑ मे॒ष्या॑ अवां᳚ब रु॒द्र-म॑दिम॒ह्यव॑ दे॒वं त्य्रं॑बकं । यथा॑ नः॒ श्रेय॑सः॒ कर॒द् यथा॑ नो॒ वस्य॑सः॒ कर॒द् यथा॑ नः पशु॒मतः॒ कर॒द् यथा॑ नो व्यवसा॒यया᳚त् ॥ त्य्रं॑बकं ॅयजामहे सुग॒न्धिं पु॑ष्टि॒वर्द्ध॑नं । उ॒र्वा॒रु॒क-मि॑व॒ बन्ध॑नान् मृ॒त्योर् मु॑क्षीय॒ माऽमृता᳚त् ॥ ए॒ष ते॑ रुद्र भा॒गस्तं जु॑षस्व॒ तेना॑व॒सेन॑ प॒रो मूज॑व॒तोऽती॒ह्य ( ) व॑ततधन्वा॒ पिना॑कहस्तः॒ कृत्ति॑वासाः ॥ \newline

\textbf{Pada Paata} \newline

यथा᳚ । अस॑ति ॥ सु॒गमिति॑ सु-गम् । मे॒षाय॑ । मे॒ष्यै᳚ । अवेति॑ । अ॒बं॒ । रु॒द्रम् । अ॒दि॒म॒हि॒ । अवेति॑ । दे॒वम् । त्र्य॑बंक॒मिति॒ त्रि-अ॒बं॒क॒म् ॥ यथा᳚ । नः॒ । श्रेय॑सः । कर॑त् । यथा᳚ । नः॒ । वस्य॑सः । कर॑त् । यथा᳚ । नः॒ । प॒शु॒मत॒ इति॑ पशु - मतः॑ । कर॑त् । यथा᳚ । नः॒ । व्य॒व॒सा॒यया॒दिति॑ वि - अ॒व॒सा॒यया᳚त् ॥ त्र्य॑बंक॒मिति॒ त्रि - अ॒बं॒क॒म् । य॒जा॒म॒हे॒ । सु॒ग॒न्धिमिति॑ सु-ग॒न्धिम् । पु॒ष्टि॒वद्‌र्ध॑न॒मिति॑ पुष्टि-वद्‌र्ध॑नम् ॥ उ॒र्वा॒रु॒कम् । इ॒व॒ । बन्ध॑नात् । मृ॒त्योः । मु॒क्षी॒य॒ । मा । अ॒मृता᳚त् ॥ ए॒षः । ते॒ । रु॒द्र॒ । भा॒गः । तम् । जु॒ष॒स्व॒ । तेन॑ । अ॒व॒सेन॑ । प॒रः । मूज॑वत॒ इति॒ मूज॑ - व॒तः॒ । अतीति॑ । इ॒हि॒ ( ) । अव॑ततध॒न्वेत्यव॑तत - ध॒न्वा॒ । पिना॑कहस्त॒ इति॒ पिना॑क - ह॒स्तः॒ । कृत्ति॑वासा॒ इति॒ कृत्ति॑ - वा॒साः॒ ॥  \newline



\textbf{Jatai Paata} \newline

1. यथा ऽस॒त्यस॑ति॒ यथा॒ यथा ऽस॑ति । \newline
2. अस॒तीत्यस॑ति । \newline
3. सु॒गम् मे॒षाय॑ मे॒षाय॑ सु॒गꣳ सु॒गम् मे॒षाय॑ । \newline
4. सु॒गमिति॑ सु - गम् । \newline
5. मे॒षाय॑ मे॒ष्यै॑ मे॒ष्यै॑ मे॒षाय॑ मे॒षाय॑ मे॒ष्यै᳚ । \newline
6. मे॒ष्या॑ अवाव॑ मे॒ष्यै॑ मे॒ष्या॑ अव॑ । \newline
7. अवा᳚म्बा॒ म्बावा वा᳚म्ब । \newline
8. अ॒म्ब॒ रु॒द्रꣳ रु॒द्र म॑म्बा म्ब रु॒द्रम् । \newline
9. रु॒द्र म॑दिम ह्यदिमहि रु॒द्रꣳ रु॒द्र म॑दिमहि । \newline
10. अ॒दि॒म॒ह्य वावा॑ दिमह्य दिम॒ह्यव॑ । \newline
11. अव॑ दे॒वम् दे॒व मवाव॑ दे॒वम् । \newline
12. दे॒वम् त्र्यं॑बक॒म् त्र्यं॑बकम् दे॒वम् दे॒वम् त्र्यं॑बकम् । \newline
13. त्र्यं॑बक॒मिति॒ त्रि - अं॒ब॒क॒म् । \newline
14. यथा॑ नो नो॒ यथा॒ यथा॑ नः । \newline
15. नः॒ श्रेय॑सः॒ श्रेय॑सो नो नः॒ श्रेय॑सः । \newline
16. श्रेय॑सः॒ कर॒त् कर॒च् छ्रेय॑सः॒ श्रेय॑सः॒ कर॑त् । \newline
17. कर॒द् यथा॒ यथा॒ कर॒त् कर॒द् यथा᳚ । \newline
18. यथा॑ नो नो॒ यथा॒ यथा॑ नः । \newline
19. नो॒ वस्य॑सो॒ वस्य॑सो नो नो॒ वस्य॑सः । \newline
20. वस्य॑सः॒ कर॒त् कर॒द् वस्य॑सो॒ वस्य॑सः॒ कर॑त् । \newline
21. कर॒द् यथा॒ यथा॒ कर॒त् कर॒द् यथा᳚ । \newline
22. यथा॑ नो नो॒ यथा॒ यथा॑ नः । \newline
23. नः॒ प॒शु॒मतः॑ पशु॒मतो॑ नो नः पशु॒मतः॑ । \newline
24. प॒शु॒मतः॒ कर॒त् कर॑त् पशु॒मतः॑ पशु॒मतः॒ कर॑त् । \newline
25. प॒शु॒मत॒ इति॑ पशु - मतः॑ । \newline
26. कर॒द् यथा॒ यथा॒ कर॒त् कर॒द् यथा᳚ । \newline
27. यथा॑ नो नो॒ यथा॒ यथा॑ नः । \newline
28. नो॒ व्य॒व॒सा॒यया᳚द् व्यवसा॒यया᳚न् नो नो व्यवसा॒यया᳚त् । \newline
29. व्य॒व॒सा॒यया॒दिति॑ वि - अ॒व॒सा॒यया᳚त् । \newline
30. त्र्य॑म्बकं ॅयजामहे यजामहे॒ त्र्य॑म्ब॒कम् त्र्य॑म्बकं ॅयजामहे । \newline
31. त्र्य॑म्बक॒मिति॒ त्रि - अ॒म्ब॒क॒म् । \newline
32. य॒जा॒म॒हे॒ सु॒ग॒न्धिꣳ सु॑ग॒न्धिं ॅय॑जामहे यजामहे सुग॒न्धिम् । \newline
33. सु॒ग॒न्धिम् पु॑ष्टि॒वर्द्ध॑नम् पुष्टि॒वर्द्ध॑नꣳ सुग॒न्धिꣳ सु॑ग॒न्धिम् पु॑ष्टि॒वर्द्ध॑नम् । \newline
34. सु॒ग॒न्धिमिति॑ सु - ग॒न्धिम् । \newline
35. पु॒ष्टि॒वर्द्ध॑न॒मिति॑ पुष्टि - वर्द्ध॑नम् । \newline
36. उ॒र्वा॒रु॒क मि॑वे वोर्वारु॒क मु॑र्वारु॒क मि॑व । \newline
37. इ॒व॒ बन्ध॑ना॒द् बन्ध॑नादिवे व॒ बन्ध॑नात् । \newline
38. बन्ध॑नान् मृ॒त्योर् मृ॒त्योर् बन्ध॑ना॒द् बन्ध॑नान् मृ॒त्योः । \newline
39. मृ॒त्योर् मु॑क्षीय मुक्षीय मृ॒त्योर् मृ॒त्योर् मु॑क्षीय । \newline
40. मु॒क्षी॒य॒ मा मा मु॑क्षीय मुक्षीय॒ मा । \newline
41. मा ऽमृता॑ द॒मृता॒न् मा मा ऽमृता᳚त् । \newline
42. अ॒मृता॒दित्य॒मृता᳚त् । \newline
43. ए॒ष ते॑ त ए॒ष ए॒ष ते᳚ । \newline
44. ते॒ रु॒द्र॒ रु॒द्र॒ ते॒ ते॒ रु॒द्र॒ । \newline
45. रु॒द्र॒ भा॒गो भा॒गो रु॑द्र रुद्र भा॒गः । \newline
46. भा॒ग स्तम् तम् भा॒गो भा॒ग स्तम् । \newline
47. तम् जु॑षस्व जुषस्व॒ तम् तम् जु॑षस्व । \newline
48. जु॒ष॒स्व॒ तेन॒ तेन॑ जुषस्व जुषस्व॒ तेन॑ । \newline
49. तेना॑ व॒सेना॑ व॒सेन॒ तेन॒ तेना॑ व॒सेन॑ । \newline
50. अ॒व॒सेन॑ प॒रः प॒रो॑ ऽव॒सेना॑ व॒सेन॑ प॒रः । \newline
51. प॒रो मूज॑वतो॒ मूज॑वतः प॒रः प॒रो मूज॑वतः । \newline
52. मूज॑व॒तो ऽत्यति॒ मूज॑वतो॒ मूज॑व॒तो ऽति॑ । \newline
53. मूज॑वत॒ इति॒ मूज॑ - व॒तः॒ । \newline
54. अती॑ही॒ ह्यत्यती॑हि । \newline
55. इ॒ह्य व॑ततध॒न्वा ऽव॑ततधन् वेही॒ ह्यव॑ततधन्वा । \newline
56. अव॑ततधन्वा॒ पिना॑कहस्तः॒ पिना॑कह॒स्तो ऽव॑ततध॒न्वा ऽव॑ततधन्वा॒ पिना॑कहस्तः । \newline
57. अव॑ततध॒न्वेत्यव॑तत - ध॒न्वा॒ । \newline
58. पिना॑कहस्तः॒ कृत्ति॑वासाः॒ कृत्ति॑वासाः॒ पिना॑कहस्तः॒ पिना॑कहस्तः॒ कृत्ति॑वासाः । \newline
59. पिना॑कहस्त॒ इति॒ पिना॑क - ह॒स्तः॒ । \newline
60. कृत्ति॑वासा॒ इति॒ कृत्ति॑ - वा॒साः॒ । \newline

\textbf{Ghana Paata } \newline

1. यथा ऽस॒त्यस॑ति॒ यथा॒ यथा ऽस॑ति । \newline
2. अस॒तीत्यस॑ति । \newline
3. सु॒गम् मे॒षाय॑ मे॒षाय॑ सु॒गꣳ सु॒गम् मे॒षाय॑ मे॒ष्यै॑ मे॒ष्यै॑ मे॒षाय॑ सु॒गꣳ सु॒गम् मे॒षाय॑ मे॒ष्यै᳚ । \newline
4. सु॒गमिति॑ सु - गम् । \newline
5. मे॒षाय॑ मे॒ष्यै॑ मे॒ष्यै॑ मे॒षाय॑ मे॒षाय॑ मे॒ष्या॑ अवाव॑ मे॒ष्यै॑ मे॒षाय॑ मे॒षाय॑ मे॒ष्या॑ अव॑ । \newline
6. मे॒ष्या॑ अवाव॑ मे॒ष्यै॑ मे॒ष्या॑ अवा᳚म्बा॒ म्बाव॑ मे॒ष्यै॑ मे॒ष्या॑ अवा᳚म्ब । \newline
7. अवा᳚म्बा॒म्बा वावा᳚म्ब रु॒द्रꣳ रु॒द्र म॒म्बा वावा᳚म्ब रु॒द्रम् । \newline
8. अ॒म्ब॒ रु॒द्रꣳ रु॒द्र म॑म्बाम्ब रु॒द्र म॑दिम ह्यदिमहि रु॒द्र म॑म्बाम्ब रु॒द्र म॑दिमहि । \newline
9. रु॒द्र म॑दिम ह्यदिमहि रु॒द्रꣳ रु॒द्र म॑दिम॒ ह्यवावा॑दिमहि रु॒द्रꣳ रु॒द्र म॑दिम॒ ह्यव॑ । \newline
10. अ॒दि॒म॒ह्य वावा॑ दिमह्य दिम॒ह्यव॑ दे॒वम् दे॒व मवा॑ दिमह्य दिम॒ह्यव॑ दे॒वम् । \newline
11. अव॑ दे॒वम् दे॒व मवाव॑ दे॒वम् त्र्यं॑बक॒म् त्र्यं॑बकम् दे॒व मवाव॑ दे॒वम् त्र्यं॑बकम् । \newline
12. दे॒वम् त्र्यं॑बक॒म् त्र्यं॑बकम् दे॒वम् दे॒वम् त्र्यं॑बकम् । \newline
13. त्र्यं॑बक॒मिति॒ त्रि - अं॒ब॒क॒म् । \newline
14. यथा॑ नो नो॒ यथा॒ यथा॑ नः॒ श्रेय॑सः॒ श्रेय॑सो नो॒ यथा॒ यथा॑ नः॒ श्रेय॑सः । \newline
15. नः॒ श्रेय॑सः॒ श्रेय॑सो नो नः॒ श्रेय॑सः॒ कर॒त् कर॒ च्छ्रेय॑सो नो नः॒ श्रेय॑सः॒ कर॑त् । \newline
16. श्रेय॑सः॒ कर॒त् कर॒च् छ्रेय॑सः॒ श्रेय॑सः॒ कर॒द् यथा॒ यथा॒ कर॒ च्छ्रेय॑सः॒ श्रेय॑सः॒ कर॒द् यथा᳚ । \newline
17. कर॒द् यथा॒ यथा॒ कर॒त् कर॒द् यथा॑ नो नो॒ यथा॒ कर॒त् कर॒द् यथा॑ नः । \newline
18. यथा॑ नो नो॒ यथा॒ यथा॑ नो॒ वस्य॑सो॒ वस्य॑सो नो॒ यथा॒ यथा॑ नो॒ वस्य॑सः । \newline
19. नो॒ वस्य॑सो॒ वस्य॑सो नो नो॒ वस्य॑सः॒ कर॒त् कर॒द् वस्य॑सो नो नो॒ वस्य॑सः॒ कर॑त् । \newline
20. वस्य॑सः॒ कर॒त् कर॒द् वस्य॑सो॒ वस्य॑सः॒ कर॒द् यथा॒ यथा॒ कर॒द् वस्य॑सो॒ वस्य॑सः॒ कर॒द् यथा᳚ । \newline
21. कर॒द् यथा॒ यथा॒ कर॒त् कर॒द् यथा॑ नो नो॒ यथा॒ कर॒त् कर॒द् यथा॑ नः । \newline
22. यथा॑ नो नो॒ यथा॒ यथा॑ नः पशु॒मतः॑ पशु॒मतो॑ नो॒ यथा॒ यथा॑ नः पशु॒मतः॑ । \newline
23. नः॒ प॒शु॒मतः॑ पशु॒मतो॑ नो नः पशु॒मतः॒ कर॒त् कर॑त् पशु॒मतो॑ नो नः पशु॒मतः॒ कर॑त् । \newline
24. प॒शु॒मतः॒ कर॒त् कर॑त् पशु॒मतः॑ पशु॒मतः॒ कर॒द् यथा॒ यथा॒ कर॑त् पशु॒मतः॑ पशु॒मतः॒ कर॒द् यथा᳚ । \newline
25. प॒शु॒मत॒ इति॑ पशु - मतः॑ । \newline
26. कर॒द् यथा॒ यथा॒ कर॒त् कर॒द् यथा॑ नो नो॒ यथा॒ कर॒त् कर॒द् यथा॑ नः । \newline
27. यथा॑ नो नो॒ यथा॒ यथा॑ नो व्यवसा॒यया᳚द् व्यवसा॒यया᳚न् नो॒ यथा॒ यथा॑ नो व्यवसा॒यया᳚त् । \newline
28. नो॒ व्य॒व॒सा॒यया᳚द् व्यवसा॒यया᳚न् नो नो व्यवसा॒यया᳚त् । \newline
29. व्य॒व॒सा॒यया॒दिति॑ वि - अ॒व॒सा॒यया᳚त् । \newline
30. त्र्यं॑बकं ॅयजामहे यजामहे॒ त्र्यं॑ब॒कम् त्र्यं॑बकं ॅयजामहे सुग॒न्धिꣳ सु॑ग॒न्धिं ॅय॑जामहे॒ त्र्यं॑ब॒कम् त्र्यं॑बकं ॅयजामहे सुग॒न्धिम् । \newline
31. त्र्यं॑बक॒मिति॒ त्रि - अं॒ब॒क॒म् । \newline
32. य॒जा॒म॒हे॒ सु॒ग॒न्धिꣳ सु॑ग॒न्धिं ॅय॑जामहे यजामहे सुग॒न्धिम् पु॑ष्टि॒वर्द्ध॑नम् पुष्टि॒वर्द्ध॑नꣳ सुग॒न्धिं ॅय॑जामहे यजामहे सुग॒न्धिम् पु॑ष्टि॒वर्द्ध॑नम् । \newline
33. सु॒ग॒न्धिम् पु॑ष्टि॒वर्द्ध॑नम् पुष्टि॒वर्द्ध॑नꣳ सुग॒न्धिꣳ सु॑ग॒न्धिम् पु॑ष्टि॒वर्द्ध॑नम् । \newline
34. सु॒ग॒न्धिमिति॑ सु - ग॒न्धिम् । \newline
35. पु॒ष्टि॒वर्द्ध॑न॒मिति॑ पुष्टि - वर्द्ध॑नम् । \newline
36. उ॒र्वा॒रु॒क मि॑वे वोर्वारु॒क मु॑र्वारु॒क मि॑व॒ बन्ध॑ना॒द् बन्ध॑ना दिवोर्वारु॒क मु॑र्वारु॒क मि॑व॒ बन्ध॑नात् । \newline
37. इ॒व॒ बन्ध॑ना॒द् बन्ध॑ना दिवेव॒ बन्ध॑नान् मृ॒त्योर् मृ॒त्योर् बन्ध॑ना दिवेव॒ बन्ध॑नान् मृ॒त्योः । \newline
38. बन्ध॑नान् मृ॒त्योर् मृ॒त्योर् बन्ध॑ना॒द् बन्ध॑नान् मृ॒त्योर् मु॑क्षीय मुक्षीय मृ॒त्योर् बन्ध॑ना॒द् बन्ध॑नान् मृ॒त्योर् मु॑क्षीय । \newline
39. मृ॒त्योर् मु॑क्षीय मुक्षीय मृ॒त्योर् मृ॒त्योर् मु॑क्षीय॒ मा मा मु॑क्षीय मृ॒त्योर् मृ॒त्योर् मु॑क्षीय॒ मा । \newline
40. मु॒क्षी॒य॒ मा मा मु॑क्षीय मुक्षीय॒ मा ऽमृता॑ द॒मृता॒न् मा मु॑क्षीय मुक्षीय॒ मा ऽमृता᳚त् । \newline
41. मा ऽमृता॑ द॒मृता॒न् मा मा ऽमृता᳚त् । \newline
42. अ॒मृता॒दित्य॒मृता᳚त् । \newline
43. ए॒ष ते॑ त ए॒ष ए॒ष ते॑ रुद्र रुद्र त ए॒ष ए॒ष ते॑ रुद्र । \newline
44. ते॒ रु॒द्र॒ रु॒द्र॒ ते॒ ते॒ रु॒द्र॒ भा॒गो भा॒गो रु॑द्र ते ते रुद्र भा॒गः । \newline
45. रु॒द्र॒ भा॒गो भा॒गो रु॑द्र रुद्र भा॒ग स्तम् तम् भा॒गो रु॑द्र रुद्र भा॒ग स्तम् । \newline
46. भा॒ग स्तम् तम् भा॒गो भा॒ग स्तम् जु॑षस्व जुषस्व॒ तम् भा॒गो भा॒ग स्तम् जु॑षस्व । \newline
47. तम् जु॑षस्व जुषस्व॒ तम् तम् जु॑षस्व॒ तेन॒ तेन॑ जुषस्व॒ तम् तम् जु॑षस्व॒ तेन॑ । \newline
48. जु॒ष॒स्व॒ तेन॒ तेन॑ जुषस्व जुषस्व॒ तेना॑ व॒सेना॑ व॒सेन॒ तेन॑ जुषस्व जुषस्व॒ तेना॑ व॒सेन॑ । \newline
49. तेना॑ व॒सेना॑ व॒सेन॒ तेन॒ तेना॑ व॒सेन॑ प॒रः प॒रो॑ ऽव॒सेन॒ तेन॒ तेना॑ व॒सेन॑ प॒रः । \newline
50. अ॒व॒सेन॑ प॒रः प॒रो॑ ऽव॒सेना॑ व॒सेन॑ प॒रो मूज॑वतो॒ मूज॑वतः प॒रो॑ ऽव॒सेना॑ व॒सेन॑ प॒रो मूज॑वतः । \newline
51. प॒रो मूज॑वतो॒ मूज॑वतः प॒रः प॒रो मूज॑व॒तो ऽत्यति॒ मूज॑वतः प॒रः प॒रो मूज॑व॒तो ऽति॑ । \newline
52. मूज॑व॒तो ऽत्यति॒ मूज॑वतो॒ मूज॑व॒तो ऽती॑ही॒ह्यति॒ मूज॑वतो॒ मूज॑व॒तो ऽती॑हि । \newline
53. मूज॑वत॒ इति॒ मूज॑ - व॒तः॒ । \newline
54. अती॑ही॒ ह्यत्यती॒ ह्यव॑ततध॒न्वा ऽव॑ततधन्वे॒ ह्यत्यती॒ ह्यव॑ततधन्वा । \newline
55. इ॒ह्यव॑ततध॒न्वा ऽव॑ततधन्वेही॒ ह्यव॑ततधन्वा॒ पिना॑कहस्तः॒ पिना॑कह॒स्तो ऽव॑ततधन्वेही॒ ह्यव॑ततधन्वा॒ पिना॑कहस्तः । \newline
56. अव॑ततधन्वा॒ पिना॑कहस्तः॒ पिना॑कह॒स्तो ऽव॑ततध॒न्वा ऽव॑ततधन्वा॒ पिना॑कहस्तः॒ कृत्ति॑वासाः॒ कृत्ति॑वासाः॒ पिना॑कह॒स्तो ऽव॑ततध॒न्वा ऽव॑ततधन्वा॒ पिना॑कहस्तः॒ कृत्ति॑वासाः । \newline
57. अव॑ततध॒न्वेत्यव॑तत - ध॒न्वा॒ । \newline
58. पिना॑कहस्तः॒ कृत्ति॑वासाः॒ कृत्ति॑वासाः॒ पिना॑कहस्तः॒ पिना॑कहस्तः॒ कृत्ति॑वासाः । \newline
59. पिना॑कहस्त॒ इति॒ पिना॑क - ह॒स्तः॒ । \newline
60. कृत्ति॑वासा॒ इति॒ कृत्ति॑ - वा॒साः॒ । \newline
\pagebreak
\markright{ TS 1.8.7.1  \hfill https://www.vedavms.in \hfill}
\addcontentsline{toc}{section}{ TS 1.8.7.1 }
\section*{ TS 1.8.7.1 }

\textbf{TS 1.8.7.1 } \newline
\textbf{Samhita Paata} \newline

ऐ॒न्द्रा॒ग्नं द्वाद॑शकपालं ॅवैश्वदे॒वं च॒रुमिन्द्रा॑य॒ शुना॒सीरा॑य पुरो॒डाशं॒ द्वाद॑शकपालं ॅवाय॒व्यं॑ पयः॑ सौ॒र्यमेक॑कपालं द्वादशग॒वꣳ सीरं॒ दक्षि॑णा- ऽऽग्ने॒य-म॒ष्टाक॑पालं॒ निर्व॑पति रौ॒द्रं गा॑वीधु॒कं च॒रुमै॒न्द्रं दधि॑ वारु॒णं ॅय॑व॒मयं॑ च॒रुं ॅव॒हिनी॑ धे॒नुर् दक्षि॑णा॒ ये दे॒वाः पु॑र॒स्सदो॒ऽग्निने᳚त्रा दक्षिण॒सदो॑ य॒मने᳚त्राः पश्चा॒थ्सदः॑ सवि॒तृने᳚त्रा उत्तर॒सदो॒ वरु॑णनेत्रा उपरि॒षदो॒ बृह॒स्पति॑नेत्रा रक्षो॒हण॒स्ते नः॑ पान्तु॒ ते नो॑ऽवन्तु॒ तेभ्यो॒ - [ ] \newline

\textbf{Pada Paata} \newline

ऐ॒न्द्रा॒ग्नमित्यै᳚न्द्र - अ॒ग्नम् । द्वाद॑शकपाल॒मिति॒ द्वाद॑श - क॒पा॒ल॒म् । वै॒श्व॒दे॒वमिति॑ वैश्व-दे॒वम् । च॒रुम् । इन्द्रा॑य । शुना॒सीरा॑य । पु॒रो॒डाश᳚म् । द्वाद॑शकपाल॒मिति॒ द्वाद॑श - क॒पा॒ल॒म् । वा॒य॒व्य᳚म् । पयः॑ । सौ॒र्यम् । एक॑कपाल॒मित्येक॑ - क॒पा॒ल॒म् । द्वा॒द॒श॒ग॒वमिति॑ द्वादश - ग॒वम् । सीर᳚म् । दक्षि॑णा । आ॒ग्ने॒यम् । अ॒ष्टाक॑पाल॒मित्य॒ष्टा - क॒पा॒ल॒म् । निरिति॑ । व॒प॒ति॒ । रौ॒द्रम् । गा॒वी॒धु॒कम् । च॒रुम् । ऐ॒न्द्रम् । दधि॑ । वा॒रु॒णम् । य॒व॒मय॒मिति॑ यव-मय᳚म् । च॒रुम् । व॒हिनी᳚ । धे॒नुः । दक्षि॑णा । ये । दे॒वाः । पु॒रः॒ सद॒ इति॑ पुरः - सदः॑ । अ॒ग्निने᳚त्रा॒ इत्य॒ग्नि - ने॒त्राः॒ । द॒क्षि॒ण॒सद॒ इति॑ दक्षिण - सदः॑ । य॒मने᳚त्रा॒ इति॑ य॒म - ने॒त्राः॒ । प॒श्चा॒थ्सद॒ इति॑ पश्चात् - सदः॑ । स॒वि॒तृने᳚त्रा॒ इति॑ सवि॒तृ - ने॒त्राः॒ । उ॒त्त॒र॒सद॒ इत्यु॑त्तर - सदः॑ । वरु॑णनेत्रा॒ इति॒ वरु॑ण - ने॒त्राः॒ । उ॒प॒रि॒षद॒ इत्यु॑परि - सदः॑ । बृह॒स्पति॑नेत्रा॒ इति॒ बृह॒स्पति॑ - ने॒त्राः॒ । र॒क्षो॒हण॒ इति॑ रक्षः - हनः॑ । ते । नः॒ । पा॒न्तु॒ । ते । नः॒ । अ॒व॒न्तु॒ । तेभ्यः॑ ।  \newline



\textbf{Jatai Paata} \newline

1. ऐ॒न्द्रा॒ग्नम् द्वाद॑शकपाल॒म् द्वाद॑शकपाल मैन्द्रा॒ग्न मै᳚न्द्रा॒ग्नम् द्वाद॑शकपालम् । \newline
2. ऐ॒न्द्रा॒ग्नमित्यै᳚न्द्र - अ॒ग्नम् । \newline
3. द्वाद॑शकपालं ॅवैश्वदे॒वं ॅवै᳚श्वदे॒वम् द्वाद॑शकपाल॒म् द्वाद॑शकपालं ॅवैश्वदे॒वम् । \newline
4. द्वाद॑शकपाल॒मिति॒ द्वाद॑श - क॒पा॒ल॒म् । \newline
5. वै॒श्व॒दे॒वम् च॒रुम् च॒रुं ॅवै᳚श्वदे॒वं ॅवै᳚श्वदे॒वम् च॒रुम् । \newline
6. वै॒श्व॒दे॒वमिति॑ वैश्व - दे॒वम् । \newline
7. च॒रु मिन्द्रा॒ये न्द्रा॑य च॒रुम् च॒रु मिन्द्रा॑य । \newline
8. इन्द्रा॑य॒ शुना॒सीरा॑य॒ शुना॒सीरा॒ये न्द्रा॒ये न्द्रा॑य॒ शुना॒सीरा॑य । \newline
9. शुना॒सीरा॑य पुरो॒डाश॑म् पुरो॒डाशꣳ॒॒ शुना॒सीरा॑य॒ शुना॒सीरा॑य पुरो॒डाश᳚म् । \newline
10. पु॒रो॒डाश॒म् द्वाद॑शकपाल॒म् द्वाद॑शकपालम् पुरो॒डाश॑म् पुरो॒डाश॒म् द्वाद॑शकपालम् । \newline
11. द्वाद॑शकपालं ॅवाय॒व्यं॑ ॅवाय॒व्य॑म् द्वाद॑शकपाल॒म् द्वाद॑शकपालं ॅवाय॒व्य᳚म् । \newline
12. द्वाद॑शकपाल॒मिति॒ द्वाद॑श - क॒पा॒ल॒म् । \newline
13. वा॒य॒व्य॑म् पयः॒ पयो॑ वाय॒व्यं॑ ॅवाय॒व्य॑म् पयः॑ । \newline
14. पयः॑ सौ॒र्यꣳ सौ॒र्यम् पयः॒ पयः॑ सौ॒र्यम् । \newline
15. सौ॒र्य मेक॑कपाल॒ मेक॑कपालꣳ सौ॒र्यꣳ सौ॒र्य मेक॑कपालम् । \newline
16. एक॑कपालम् द्वादशग॒वम् द्वा॑दशग॒व मेक॑कपाल॒ मेक॑कपालम् द्वादशग॒वम् । \newline
17. एक॑कपाल॒मित्येक॑ - क॒पा॒ल॒म् । \newline
18. द्वा॒द॒श॒ग॒वꣳ सीरꣳ॒॒ सीर॑म् द्वादशग॒वम् द्वा॑दशग॒वꣳ सीर᳚म् । \newline
19. द्वा॒द॒श॒ग॒वमिति॑ द्वादश - ग॒वम् । \newline
20. सीर॒म् दक्षि॑णा॒ दक्षि॑णा॒ सीरꣳ॒॒ सीर॒म् दक्षि॑णा । \newline
21. दक्षि॑णा ऽऽग्ने॒य मा᳚ग्ने॒यम् दक्षि॑णा॒ दक्षि॑णा ऽऽग्ने॒यम् । \newline
22. आ॒ग्ने॒य म॒ष्टाक॑पाल म॒ष्टाक॑पाल माग्ने॒य मा᳚ग्ने॒य म॒ष्टाक॑पालम् । \newline
23. अ॒ष्टाक॑पाल॒म् निर् णिर॒ष्टाक॑पाल म॒ष्टाक॑पाल॒म् निः । \newline
24. अ॒ष्टाक॑पाल॒मित्य॒ष्टा - क॒पा॒ल॒म् । \newline
25. निर् व॑पति वपति॒ निर् णिर् व॑पति । \newline
26. व॒प॒ति॒ रौ॒द्रꣳ रौ॒द्रं ॅव॑पति वपति रौ॒द्रम् । \newline
27. रौ॒द्रम् गा॑वीधु॒कम् गा॑वीधु॒कꣳ रौ॒द्रꣳ रौ॒द्रम् गा॑वीधु॒कम् । \newline
28. गा॒वी॒धु॒कम् च॒रुम् च॒रुम् गा॑वीधु॒कम् गा॑वीधु॒कम् च॒रुम् । \newline
29. च॒रु मै॒न्द्र मै॒न्द्रम् च॒रुम् च॒रु मै॒न्द्रम् । \newline
30. ऐ॒न्द्रम् दधि॒ दध्यै॒न्द्र मै॒न्द्रम् दधि॑ । \newline
31. दधि॑ वारु॒णं ॅवा॑रु॒णम् दधि॒ दधि॑ वारु॒णम् । \newline
32. वा॒रु॒णं ॅय॑व॒मयं॑ ॅयव॒मयं॑ ॅवारु॒णं ॅवा॑रु॒णं ॅय॑व॒मय᳚म् । \newline
33. य॒व॒मय॑म् च॒रुम् च॒रुं ॅय॑व॒मयं॑ ॅयव॒मय॑म् च॒रुम् । \newline
34. य॒व॒मय॒मिति॑ यव - मय᳚म् । \newline
35. च॒रुं ॅव॒हिनी॑ व॒हिनी॑ च॒रुम् च॒रुं ॅव॒हिनी᳚ । \newline
36. व॒हिनी॑ धे॒नुर् धे॒नुर् व॒हिनी॑ व॒हिनी॑ धे॒नुः । \newline
37. धे॒नुर् दक्षि॑णा॒ दक्षि॑णा धे॒नुर् धे॒नुर् दक्षि॑णा । \newline
38. दक्षि॑णा॒ ये ये दक्षि॑णा॒ दक्षि॑णा॒ ये । \newline
39. ये दे॒वा दे॒वा ये ये दे॒वाः । \newline
40. दे॒वाः पु॑र॒स्सदः॑ पुर॒स्सदो॑ दे॒वा दे॒वाः पु॑र॒स्सदः॑ । \newline
41. पु॒र॒स्सदो॒ ऽग्निने᳚त्रा अ॒ग्निने᳚त्राः पुर॒स्सदः॑ पुर॒स्सदो॒ ऽग्निने᳚त्राः । \newline
42. पु॒र॒स्सद॒ इति॑ पुरः - सदः॑ । \newline
43. अ॒ग्निने᳚त्रा दक्षिण॒सदो॑ दक्षिण॒सदो॒ ऽग्निने᳚त्रा अ॒ग्निने᳚त्रा दक्षिण॒सदः॑ । \newline
44. अ॒ग्निने᳚त्रा॒ इत्य॒ग्नि - ने॒त्राः॒ । \newline
45. द॒क्षि॒ण॒सदो॑ य॒मने᳚त्रा य॒मने᳚त्रा दक्षिण॒सदो॑ दक्षिण॒सदो॑ य॒मने᳚त्राः । \newline
46. द॒क्षि॒ण॒सद॒ इति॑ दक्षिण - सदः॑ । \newline
47. य॒मने᳚त्राः पश्चा॒थ्सदः॑ पश्चा॒थ्सदो॑ य॒मने᳚त्रा य॒मने᳚त्राः पश्चा॒थ्सदः॑ । \newline
48. य॒मने᳚त्रा॒ इति॑ य॒म - ने॒त्राः॒ । \newline
49. प॒श्चा॒थ्सदः॑ सवि॒तृने᳚त्राः सवि॒तृने᳚त्राः पश्चा॒थ्सदः॑ पश्चा॒थ्सदः॑ सवि॒तृने᳚त्राः । \newline
50. प॒श्चा॒थ्सद॒ इति॑ पश्चात् - सदः॑ । \newline
51. स॒वि॒तृने᳚त्रा उत्तर॒सद॑ उत्तर॒सदः॑ सवि॒तृने᳚त्राः सवि॒तृने᳚त्रा उत्तर॒सदः॑ । \newline
52. स॒वि॒तृने᳚त्रा॒ इति॑ सवि॒तृ - ने॒त्राः॒ । \newline
53. उ॒त्त॒र॒सदो॒ वरु॑णनेत्रा॒ वरु॑णनेत्रा उत्तर॒सद॑ उत्तर॒सदो॒ वरु॑णनेत्राः । \newline
54. उ॒त्त॒र॒सद॒ इत्यु॑त्तर - सदः॑ । \newline
55. वरु॑णनेत्रा उपरि॒षद॑ उपरि॒षदो॒ वरु॑णनेत्रा॒ वरु॑णनेत्रा उपरि॒षदः॑ । \newline
56. वरु॑णनेत्रा॒ इति॒ वरु॑ण - ने॒त्राः॒ । \newline
57. उ॒प॒रि॒षदो॒ बृह॒स्पति॑नेत्रा॒ बृह॒स्पति॑नेत्रा उपरि॒षद॑ उपरि॒षदो॒ बृह॒स्पति॑नेत्राः । \newline
58. उ॒प॒रि॒षद॒ इत्यु॑परि - सदः॑ । \newline
59. बृह॒स्पति॑नेत्रा रक्षो॒हणो॑ रक्षो॒हणो॒ बृह॒स्पति॑नेत्रा॒ बृह॒स्पति॑नेत्रा रक्षो॒हणः॑ । \newline
60. बृह॒स्पति॑नेत्रा॒ इति॒ बृह॒स्पति॑ - ने॒त्राः॒ । \newline
61. र॒क्षो॒हण॒स्ते ते र॑क्षो॒हणो॑ रक्षो॒हण॒स्ते । \newline
62. र॒क्षो॒हण॒ इति॑ रक्षः - हनः॑ । \newline
63. ते नो॑ न॒ स्ते ते नः॑ । \newline
64. नः॒ पा॒न्तु॒ पा॒न्तु॒ नो॒ नः॒ पा॒न्तु॒ । \newline
65. पा॒न्तु॒ ते ते पा᳚न्तु पान्तु॒ ते । \newline
66. ते नो॑ न॒ स्ते ते नः॑ । \newline
67. नो॒ ऽव॒ न्त्व॒ व॒न्तु॒ नो॒ नो॒ ऽव॒न्तु॒ । \newline
68. अ॒व॒न्तु॒ तेभ्य॒ स्तेभ्यो॑ ऽवन् त्ववन्तु॒ तेभ्यः॑ । \newline
69. तेभ्यो॒ नमो॒ नम॒ स्तेभ्य॒ स्तेभ्यो॒ नमः॑ । \newline

\textbf{Ghana Paata } \newline

1. ऐ॒न्द्रा॒ग्नम् द्वाद॑शकपाल॒म् द्वाद॑शकपाल मैन्द्रा॒ग्न मै᳚न्द्रा॒ग्नम् द्वाद॑शकपालं ॅवैश्वदे॒वं ॅवै᳚श्वदे॒वम् द्वाद॑शकपाल मैन्द्रा॒ग्न मै᳚न्द्रा॒ग्नम् द्वाद॑शकपालं ॅवैश्वदे॒वम् । \newline
2. ऐ॒न्द्रा॒ग्नमित्यै᳚न्द्र - अ॒ग्नम् । \newline
3. द्वाद॑शकपालं ॅवैश्वदे॒वं ॅवै᳚श्वदे॒वम् द्वाद॑शकपाल॒म् द्वाद॑शकपालं ॅवैश्वदे॒वम् च॒रुम् च॒रुं ॅवै᳚श्वदे॒वम् द्वाद॑शकपाल॒म् द्वाद॑शकपालं ॅवैश्वदे॒वम् च॒रुम् । \newline
4. द्वाद॑शकपाल॒मिति॒ द्वाद॑श - क॒पा॒ल॒म् । \newline
5. वै॒श्व॒दे॒वम् च॒रुम् च॒रुं ॅवै᳚श्वदे॒वं ॅवै᳚श्वदे॒वम् च॒रु मिन्द्रा॒ये न्द्रा॑य च॒रुं ॅवै᳚श्वदे॒वं ॅवै᳚श्वदे॒वम् च॒रु मिन्द्रा॑य । \newline
6. वै॒श्व॒दे॒वमिति॑ वैश्व - दे॒वम् । \newline
7. च॒रु मिन्द्रा॒ये न्द्रा॑य च॒रुम् च॒रु मिन्द्रा॑य॒ शुना॒सीरा॑य॒ शुना॒सीरा॒ये न्द्रा॑य च॒रुम् च॒रु मिन्द्रा॑य॒ शुना॒सीरा॑य । \newline
8. इन्द्रा॑य॒ शुना॒सीरा॑य॒ शुना॒सीरा॒ये न्द्रा॒ये न्द्रा॑य॒ शुना॒सीरा॑य पुरो॒डाश॑म् पुरो॒डाशꣳ॒॒ शुना॒सीरा॒ये न्द्रा॒ये न्द्रा॑य॒ शुना॒सीरा॑य पुरो॒डाश᳚म् । \newline
9. शुना॒सीरा॑य पुरो॒डाश॑म् पुरो॒डाशꣳ॒॒ शुना॒सीरा॑य॒ शुना॒सीरा॑य पुरो॒डाश॒म् द्वाद॑शकपाल॒म् द्वाद॑शकपालम् पुरो॒डाशꣳ॒॒ शुना॒सीरा॑य॒ शुना॒सीरा॑य पुरो॒डाश॒म् द्वाद॑शकपालम् । \newline
10. पु॒रो॒डाश॒म् द्वाद॑शकपाल॒म् द्वाद॑शकपालम् पुरो॒डाश॑म् पुरो॒डाश॒म् द्वाद॑शकपालं ॅवाय॒व्यं॑ ॅवाय॒व्य॑म् द्वाद॑शकपालम् पुरो॒डाश॑म् पुरो॒डाश॒म् द्वाद॑शकपालं ॅवाय॒व्य᳚म् । \newline
11. द्वाद॑शकपालं ॅवाय॒व्यं॑ ॅवाय॒व्य॑म् द्वाद॑शकपाल॒म् द्वाद॑शकपालं ॅवाय॒व्य॑म् पयः॒ पयो॑ वाय॒व्य॑म् द्वाद॑शकपाल॒म् द्वाद॑शकपालं ॅवाय॒व्य॑म् पयः॑ । \newline
12. द्वाद॑शकपाल॒मिति॒ द्वाद॑श - क॒पा॒ल॒म् । \newline
13. वा॒य॒व्य॑म् पयः॒ पयो॑ वाय॒व्यं॑ ॅवाय॒व्य॑म् पयः॑ सौ॒र्यꣳ सौ॒र्यम् पयो॑ वाय॒व्यं॑ ॅवाय॒व्य॑म् पयः॑ सौ॒र्यम् । \newline
14. पयः॑ सौ॒र्यꣳ सौ॒र्यम् पयः॒ पयः॑ सौ॒र्य मेक॑कपाल॒ मेक॑कपालꣳ सौ॒र्यम् पयः॒ पयः॑ सौ॒र्य मेक॑कपालम् । \newline
15. सौ॒र्य मेक॑कपाल॒ मेक॑कपालꣳ सौ॒र्यꣳ सौ॒र्य मेक॑कपालम् द्वादशग॒वम् द्वा॑दशग॒व मेक॑कपालꣳ सौ॒र्यꣳ सौ॒र्य मेक॑कपालम् द्वादशग॒वम् । \newline
16. एक॑कपालम् द्वादशग॒वम् द्वा॑दशग॒व मेक॑कपाल॒ मेक॑कपालम् द्वादशग॒वꣳ सीरꣳ॒॒ सीर॑म् द्वादशग॒व मेक॑कपाल॒ मेक॑कपालम् द्वादशग॒वꣳ सीर᳚म् । \newline
17. एक॑कपाल॒मित्येक॑ - क॒पा॒ल॒म् । \newline
18. द्वा॒द॒श॒ग॒वꣳ सीरꣳ॒॒ सीर॑म् द्वादशग॒वम् द्वा॑दशग॒वꣳ सीर॒म् दक्षि॑णा॒ दक्षि॑णा॒ सीर॑म् द्वादशग॒वम् द्वा॑दशग॒वꣳ सीर॒म् दक्षि॑णा । \newline
19. द्वा॒द॒श॒ग॒वमिति॑ द्वादश - ग॒वम् । \newline
20. सीर॒म् दक्षि॑णा॒ दक्षि॑णा॒ सीरꣳ॒॒ सीर॒म् दक्षि॑णा ऽऽग्ने॒य मा᳚ग्ने॒यम् दक्षि॑णा॒ सीरꣳ॒॒ सीर॒म् दक्षि॑णा ऽऽग्ने॒यम् । \newline
21. दक्षि॑णा ऽऽग्ने॒य मा᳚ग्ने॒यम् दक्षि॑णा॒ दक्षि॑णा ऽऽग्ने॒य म॒ष्टाक॑पाल म॒ष्टाक॑पाल माग्ने॒यम् दक्षि॑णा॒ दक्षि॑णा ऽऽग्ने॒य म॒ष्टाक॑पालम् । \newline
22. आ॒ग्ने॒य म॒ष्टाक॑पाल म॒ष्टाक॑पाल माग्ने॒य मा᳚ग्ने॒य म॒ष्टाक॑पाल॒न्निर् णिर॒ष्टाक॑पाल माग्ने॒य मा᳚ग्ने॒य म॒ष्टाक॑पाल॒न्निः । \newline
23. अ॒ष्टाक॑पाल॒न्निर् णिर॒ष्टाक॑पाल म॒ष्टाक॑पाल॒न्निर् व॑पति वपति॒ निर॒ष्टाक॑पाल म॒ष्टाक॑पाल॒न् निर् व॑पति । \newline
24. अ॒ष्टाक॑पाल॒मित्य॒ष्टा - क॒पा॒ल॒म् । \newline
25. निर् व॑पति वपति॒ निर् णिर् व॑पति रौ॒द्रꣳ रौ॒द्रं ॅव॑पति॒ निर् णिर् व॑पति रौ॒द्रम् । \newline
26. व॒प॒ति॒ रौ॒द्रꣳ रौ॒द्रं ॅव॑पति वपति रौ॒द्रम् गा॑वीधु॒कम् गा॑वीधु॒कꣳ रौ॒द्रं ॅव॑पति वपति रौ॒द्रम् गा॑वीधु॒कम् । \newline
27. रौ॒द्रम् गा॑वीधु॒कम् गा॑वीधु॒कꣳ रौ॒द्रꣳ रौ॒द्रम् गा॑वीधु॒कम् च॒रुम् च॒रुम् गा॑वीधु॒कꣳ रौ॒द्रꣳ रौ॒द्रम् गा॑वीधु॒कम् च॒रुम् । \newline
28. गा॒वी॒धु॒कम् च॒रुम् च॒रुम् गा॑वीधु॒कम् गा॑वीधु॒कम् च॒रु मै॒न्द्र मै॒न्द्रम् च॒रुम् गा॑वीधु॒कम् गा॑वीधु॒कम् च॒रु मै॒न्द्रम् । \newline
29. च॒रु मै॒न्द्र मै॒न्द्रम् च॒रुम् च॒रु मै॒न्द्रम् दधि॒ दध्यै॒न्द्रम् च॒रुम् च॒रु मै॒न्द्रम् दधि॑ । \newline
30. ऐ॒न्द्रम् दधि॒ दध्यै॒न्द्र मै॒न्द्रम् दधि॑ वारु॒णं ॅवा॑रु॒णम् दध्यै॒न्द्र मै॒न्द्रम् दधि॑ वारु॒णम् । \newline
31. दधि॑ वारु॒णं ॅवा॑रु॒णम् दधि॒ दधि॑ वारु॒णं ॅय॑व॒मयं॑ ॅयव॒मयं॑ ॅवारु॒णम् दधि॒ दधि॑ वारु॒णं ॅय॑व॒मय᳚म् । \newline
32. वा॒रु॒णं ॅय॑व॒मयं॑ ॅयव॒मयं॑ ॅवारु॒णं ॅवा॑रु॒णं ॅय॑व॒मय॑म् च॒रुम् च॒रुं ॅय॑व॒मयं॑ ॅवारु॒णं ॅवा॑रु॒णं ॅय॑व॒मय॑म् च॒रुम् । \newline
33. य॒व॒मय॑म् च॒रुम् च॒रुं ॅय॑व॒मयं॑ ॅयव॒मय॑म् च॒रुं ॅव॒हिनी॑ व॒हिनी॑ च॒रुं ॅय॑व॒मयं॑ ॅयव॒मय॑म् च॒रुं ॅव॒हिनी᳚ । \newline
34. य॒व॒मय॒मिति॑ यव - मय᳚म् । \newline
35. च॒रुं ॅव॒हिनी॑ व॒हिनी॑ च॒रुम् च॒रुं ॅव॒हिनी॑ धे॒नुर् धे॒नुर् व॒हिनी॑ च॒रुम् च॒रुं ॅव॒हिनी॑ धे॒नुः । \newline
36. व॒हिनी॑ धे॒नुर् धे॒नुर् व॒हिनी॑ व॒हिनी॑ धे॒नुर् दक्षि॑णा॒ दक्षि॑णा धे॒नुर् व॒हिनी॑ व॒हिनी॑ धे॒नुर् दक्षि॑णा । \newline
37. धे॒नुर् दक्षि॑णा॒ दक्षि॑णा धे॒नुर् धे॒नुर् दक्षि॑णा॒ ये ये दक्षि॑णा धे॒नुर् धे॒नुर् दक्षि॑णा॒ ये । \newline
38. दक्षि॑णा॒ ये ये दक्षि॑णा॒ दक्षि॑णा॒ ये दे॒वा दे॒वा ये दक्षि॑णा॒ दक्षि॑णा॒ ये दे॒वाः । \newline
39. ये दे॒वा दे॒वा ये ये दे॒वाः पु॑र॒स्सदः॑ पुर॒स्सदो॑ दे॒वा ये ये दे॒वाः पु॑र॒स्सदः॑ । \newline
40. दे॒वाः पु॑र॒स्सदः॑ पुर॒स्सदो॑ दे॒वा दे॒वाः पु॑र॒स्सदो॒ ऽग्निने᳚त्रा अ॒ग्निने᳚त्राः पुर॒स्सदो॑ दे॒वा दे॒वाः पु॑र॒स्सदो॒ ऽग्निने᳚त्राः । \newline
41. पु॒र॒स्सदो॒ ऽग्निने᳚त्रा अ॒ग्निने᳚त्राः पुर॒स्सदः॑ पुर॒स्सदो॒ ऽग्निने᳚त्रा दक्षिण॒सदो॑ दक्षिण॒सदो॒ ऽग्निने᳚त्राः पुर॒स्सदः॑ पुर॒स्सदो॒ ऽग्निने᳚त्रा दक्षिण॒सदः॑ । \newline
42. पु॒र॒स्सद॒ इति॑ पुरः - सदः॑ । \newline
43. अ॒ग्निने᳚त्रा दक्षिण॒सदो॑ दक्षिण॒सदो॒ ऽग्निने᳚त्रा अ॒ग्निने᳚त्रा दक्षिण॒सदो॑ य॒मने᳚त्रा य॒मने᳚त्रा दक्षिण॒सदो॒ ऽग्निने᳚त्रा अ॒ग्निने᳚त्रा दक्षिण॒सदो॑ य॒मने᳚त्राः । \newline
44. अ॒ग्निने᳚त्रा॒ इत्य॒ग्नि - ने॒त्राः॒ । \newline
45. द॒क्षि॒ण॒सदो॑ य॒मने᳚त्रा य॒मने᳚त्रा दक्षिण॒सदो॑ दक्षिण॒सदो॑ य॒मने᳚त्राः पश्चा॒थ्सदः॑ पश्चा॒थ्सदो॑ य॒मने᳚त्रा दक्षिण॒सदो॑ दक्षिण॒सदो॑ य॒मने᳚त्राः पश्चा॒थ्सदः॑ । \newline
46. द॒क्षि॒ण॒सद॒ इति॑ दक्षिण - सदः॑ । \newline
47. य॒मने᳚त्राः पश्चा॒थ्सदः॑ पश्चा॒थ्सदो॑ य॒मने᳚त्रा य॒मने᳚त्राः पश्चा॒थ्सदः॑ सवि॒तृने᳚त्राः सवि॒तृने᳚त्राः पश्चा॒थ्सदो॑ य॒मने᳚त्रा य॒मने᳚त्राः पश्चा॒थ्सदः॑ सवि॒तृने᳚त्राः । \newline
48. य॒मने᳚त्रा॒ इति॑ य॒म - ने॒त्राः॒ । \newline
49. प॒श्चा॒थ्सदः॑ सवि॒तृने᳚त्राः सवि॒तृने᳚त्राः पश्चा॒थ्सदः॑ पश्चा॒थ्सदः॑ सवि॒तृने᳚त्रा उत्तर॒सद॑ उत्तर॒सदः॑ सवि॒तृने᳚त्राः पश्चा॒थ्सदः॑ पश्चा॒थ्सदः॑ सवि॒तृने᳚त्रा उत्तर॒सदः॑ । \newline
50. प॒श्चा॒थ्सद॒ इति॑ पश्चात् - सदः॑ । \newline
51. स॒वि॒तृने᳚त्रा उत्तर॒सद॑ उत्तर॒सदः॑ सवि॒तृने᳚त्राः सवि॒तृने᳚त्रा उत्तर॒सदो॒ वरु॑णनेत्रा॒ वरु॑णनेत्रा उत्तर॒सदः॑ सवि॒तृने᳚त्राः सवि॒तृने᳚त्रा उत्तर॒सदो॒ वरु॑णनेत्राः । \newline
52. स॒वि॒तृने᳚त्रा॒ इति॑ सवि॒तृ - ने॒त्राः॒ । \newline
53. उ॒त्त॒र॒सदो॒ वरु॑णनेत्रा॒ वरु॑णनेत्रा उत्तर॒सद॑ उत्तर॒सदो॒ वरु॑णनेत्रा उपरि॒षद॑ उपरि॒षदो॒ वरु॑णनेत्रा उत्तर॒सद॑ उत्तर॒सदो॒ वरु॑णनेत्रा उपरि॒षदः॑ । \newline
54. उ॒त्त॒र॒सद॒ इत्यु॑त्तर - सदः॑ । \newline
55. वरु॑णनेत्रा उपरि॒षद॑ उपरि॒षदो॒ वरु॑णनेत्रा॒ वरु॑णनेत्रा उपरि॒षदो॒ बृह॒स्पति॑नेत्रा॒ बृह॒स्पति॑नेत्रा उपरि॒षदो॒ वरु॑णनेत्रा॒ वरु॑णनेत्रा उपरि॒षदो॒ बृह॒स्पति॑नेत्राः । \newline
56. वरु॑णनेत्रा॒ इति॒ वरु॑ण - ने॒त्राः॒ । \newline
57. उ॒प॒रि॒षदो॒ बृह॒स्पति॑नेत्रा॒ बृह॒स्पति॑नेत्रा उपरि॒षद॑ उपरि॒षदो॒ बृह॒स्पति॑नेत्रा रक्षो॒हणो॑ रक्षो॒हणो॒ बृह॒स्पति॑नेत्रा उपरि॒षद॑ उपरि॒षदो॒ बृह॒स्पति॑नेत्रा रक्षो॒हणः॑ । \newline
58. उ॒प॒रि॒षद॒ इत्यु॑परि - सदः॑ । \newline
59. बृह॒स्पति॑नेत्रा रक्षो॒हणो॑ रक्षो॒हणो॒ बृह॒स्पति॑नेत्रा॒ बृह॒स्पति॑नेत्रा रक्षो॒हण॒ स्ते ते र॑क्षो॒हणो॒ बृह॒स्पति॑नेत्रा॒ बृह॒स्पति॑नेत्रा रक्षो॒हण॒ स्ते । \newline
60. बृह॒स्पति॑नेत्रा॒ इति॒ बृह॒स्पति॑ - ने॒त्राः॒ । \newline
61. र॒क्षो॒हण॒ स्ते ते र॑क्षो॒हणो॑ रक्षो॒हण॒ स्ते नो॑ न॒स्ते र॑क्षो॒हणो॑ रक्षो॒हण॒ स्ते नः॑ । \newline
62. र॒क्षो॒हण॒ इति॑ रक्षः - हनः॑ । \newline
63. ते नो॑ न॒स्ते ते नः॑ पान्तु पान्तु न॒स्ते ते नः॑ पान्तु । \newline
64. नः॒ पा॒न्तु॒ पा॒न्तु॒ नो॒ नः॒ पा॒न्तु॒ ते ते पा᳚न्तु नो नः पान्तु॒ ते । \newline
65. पा॒न्तु॒ ते ते पा᳚न्तु पान्तु॒ ते नो॑ न॒स्ते पा᳚न्तु पान्तु॒ ते नः॑ । \newline
66. ते नो॑ न॒स्ते ते नो॑ ऽवन्त्ववन्तु न॒स्ते ते नो॑ ऽवन्तु । \newline
67. नो॒ ऽव॒न्त्व॒व॒न्तु॒ नो॒ नो॒ ऽव॒न्तु॒ तेभ्य॒ स्तेभ्यो॑ ऽवन्तु नो नो ऽवन्तु॒ तेभ्यः॑ । \newline
68. अ॒व॒न्तु॒ तेभ्य॒ स्तेभ्यो॑ ऽवन्त्ववन्तु॒ तेभ्यो॒ नमो॒ नम॒ स्तेभ्यो॑ ऽवन्त्ववन्तु॒ तेभ्यो॒ नमः॑ । \newline
69. तेभ्यो॒ नमो॒ नम॒ स्तेभ्य॒ स्तेभ्यो॒ नम॒ स्तेभ्य॒ स्तेभ्यो॒ नम॒ स्तेभ्य॒ स्तेभ्यो॒ नम॒ स्तेभ्यः॑ । \newline
\pagebreak
\markright{ TS 1.8.7.2  \hfill https://www.vedavms.in \hfill}
\addcontentsline{toc}{section}{ TS 1.8.7.2 }
\section*{ TS 1.8.7.2 }

\textbf{TS 1.8.7.2 } \newline
\textbf{Samhita Paata} \newline

नम॒स्तेभ्यः॒ स्वाहा॒ समू॑ढꣳ॒॒ रक्षः॒ सन्द॑॑ग्धꣳ॒॒ रक्ष॑ इ॒दम॒हꣳ रक्षो॒ऽभि सं द॑हाम्य॒ग्नये॑ रक्षो॒घ्ने स्वाहा॑ य॒माय॑ सवि॒त्रे वरु॑णाय॒ बृह॒स्पत॑ये॒ दुव॑स्वते रक्षो॒घ्ने स्वाहा᳚ प्रष्टिवा॒ही रथो॒ दक्षि॑णा दे॒वस्य॑ त्वा सवि॒तुः प्र॑स॒वे᳚ऽश्विनो᳚र् बा॒हुभ्यां᳚ पू॒ष्णो हस्ता᳚भ्याꣳ॒॒ रक्ष॑सो व॒धं जु॑होमि ह॒तꣳ रक्षोऽव॑धिष्म॒ रक्षो॒ यद् वस्ते॒ तद् दक्षि॑णा ॥ \newline

\textbf{Pada Paata} \newline

नमः॑ । तेभ्यः॑ । स्वाहा᳚ । समू॑ढ॒मिति॒ सं - ऊ॒ढ॒म् । रक्षः॑ । संद॑ग्ध॒मिति॒ सं-द॒ग्ध॒म् । रक्षः॑ । इ॒दम् । अ॒हम् । रक्षः॑ । अ॒भि । समिति॑ । द॒हा॒मि॒ । अ॒ग्नये᳚ । र॒क्षो॒घ्न इति॑ रक्षः - घ्ने । स्वाहा᳚ । य॒माय॑ । स॒वि॒त्रे । वरु॑णाय । बृह॒स्पत॑ये । दुव॑स्वते । र॒क्षो॒घ्न इति॑ रक्षः-घ्ने । स्वाहा᳚ । प्र॒ष्टि॒वा॒हीति॑ प्रष्टि-वा॒ही । रथः॑ । दक्षि॑णा । दे॒वस्य॑ । त्वा॒ । स॒वि॒तुः । प्र॒स॒व इति॑ प्र - स॒वे । अ॒श्विनोः᳚ । बा॒हुभ्या॒मिति॑ बा॒हु-भ्या॒म् । पू॒ष्णः । हस्ता᳚भ्याम् । रक्ष॑सः । व॒धम् । जु॒हो॒मि॒ । ह॒तम् । रक्षः॑ । अव॑धिष्म । रक्षः॑ । यत् । वस्ते᳚ । तत् । दक्षि॑णा ॥  \newline



\textbf{Jatai Paata} \newline

1. नम॒ स्तेभ्य॒ स्तेभ्यो॒ नमो॒ नम॒ स्तेभ्यः॑ । \newline
2. तेभ्यः॒ स्वाहा॒ स्वाहा॒ तेभ्य॒ स्तेभ्यः॒ स्वाहा᳚ । \newline
3. स्वाहा॒ समू॑ढꣳ॒॒ समू॑ढꣳ॒॒ स्वाहा॒ स्वाहा॒ समू॑ढम् । \newline
4. समू॑ढꣳ॒॒ रक्षो॒ रक्षः॒ समू॑ढꣳ॒॒ समू॑ढꣳ॒॒ रक्षः॑ । \newline
5. समू॑ढ॒मिति॒ सं - ऊ॒ढ॒म् । \newline
6. रक्षः॒ सन्द॑ग्धꣳ॒॒ सन्द॑ग्धꣳ॒॒ रक्षो॒ रक्षः॒ सन्द॑ग्धम् । \newline
7. सन्द॑ग्धꣳ॒॒ रक्षो॒ रक्षः॒ सन्द॑ग्धꣳ॒॒ सन्द॑ग्धꣳ॒॒ रक्षः॑ । \newline
8. सन्द॑ग्ध॒मिति॒ सं - द॒ग्ध॒म् । \newline
9. रक्ष॑ इ॒द मि॒दꣳ रक्षो॒ रक्ष॑ इ॒दम् । \newline
10. इ॒द म॒ह म॒ह मि॒द मि॒द म॒हम् । \newline
11. अ॒हꣳ रक्षो॒ रक्षो॒ ऽह म॒हꣳ रक्षः॑ । \newline
12. रक्षो॒ ऽभ्य॑भि रक्षो॒ रक्षो॒ ऽभि । \newline
13. अ॒भि सꣳ स म॒भ्य॑भि सम् । \newline
14. सम् द॑हामि दहामि॒ सꣳ सम् द॑हामि । \newline
15. द॒हा॒ म्य॒ग्नये॒ ऽग्नये॑ दहामि दहा म्य॒ग्नये᳚ । \newline
16. अ॒ग्नये॑ रक्षो॒घ्ने र॑क्षो॒घ्ने᳚ ऽग्नये॒ ऽग्नये॑ रक्षो॒घ्ने । \newline
17. र॒क्षो॒घ्ने स्वाहा॒ स्वाहा॑ रक्षो॒घ्ने र॑क्षो॒घ्ने स्वाहा᳚ । \newline
18. र॒क्षो॒घ्न इति॑ रक्षः - घ्ने । \newline
19. स्वाहा॑ य॒माय॑ य॒माय॒ स्वाहा॒ स्वाहा॑ य॒माय॑ । \newline
20. य॒माय॑ सवि॒त्रे स॑वि॒त्रे य॒माय॑ य॒माय॑ सवि॒त्रे । \newline
21. स॒वि॒त्रे वरु॑णाय॒ वरु॑णाय सवि॒त्रे स॑वि॒त्रे वरु॑णाय । \newline
22. वरु॑णाय॒ बृह॒स्पत॑ये॒ बृह॒स्पत॑ये॒ वरु॑णाय॒ वरु॑णाय॒ बृह॒स्पत॑ये । \newline
23. बृह॒स्पत॑ये॒ दुव॑स्वते॒ दुव॑स्वते॒ बृह॒स्पत॑ये॒ बृह॒स्पत॑ये॒ दुव॑स्वते । \newline
24. दुव॑स्वते रक्षो॒घ्ने र॑क्षो॒घ्ने दुव॑स्वते॒ दुव॑स्वते रक्षो॒घ्ने । \newline
25. र॒क्षो॒घ्ने स्वाहा॒ स्वाहा॑ रक्षो॒घ्ने र॑क्षो॒घ्ने स्वाहा᳚ । \newline
26. र॒क्षो॒घ्न इति॑ रक्षः - घ्ने । \newline
27. स्वाहा᳚ प्रष्टिवा॒ही प्र॑ष्टिवा॒ही स्वाहा॒ स्वाहा᳚ प्रष्टिवा॒ही । \newline
28. प्र॒ष्टि॒वा॒ही रथो॒ रथः॑ प्रष्टिवा॒ही प्र॑ष्टिवा॒ही रथः॑ । \newline
29. प्र॒ष्टि॒वा॒हीति॑ प्रष्टि - वा॒ही । \newline
30. रथो॒ दक्षि॑णा॒ दक्षि॑णा॒ रथो॒ रथो॒ दक्षि॑णा । \newline
31. दक्षि॑णा दे॒वस्य॑ दे॒वस्य॒ दक्षि॑णा॒ दक्षि॑णा दे॒वस्य॑ । \newline
32. दे॒वस्य॑ त्वा त्वा दे॒वस्य॑ दे॒वस्य॑ त्वा । \newline
33. त्वा॒ स॒वि॒तुः स॑वि॒तु स्त्वा᳚ त्वा सवि॒तुः । \newline
34. स॒वि॒तुः प्र॑स॒वे प्र॑स॒वे स॑वि॒तुः स॑वि॒तुः प्र॑स॒वे । \newline
35. प्र॒स॒वे᳚ ऽश्विनो॑ र॒श्विनोः᳚ प्रस॒वे प्र॑स॒वे᳚ ऽश्विनोः᳚ । \newline
36. प्र॒स॒व इति॑ प्र - स॒वे । \newline
37. अ॒श्विनो᳚र् बा॒हुभ्या᳚म् बा॒हुभ्या॑ म॒श्विनो॑ र॒श्विनो᳚र् बा॒हुभ्या᳚म् । \newline
38. बा॒हुभ्या᳚म् पू॒ष्णः पू॒ष्णो बा॒हुभ्या᳚म् बा॒हुभ्या᳚म् पू॒ष्णः । \newline
39. बा॒हुभ्या॒मिति॑ बा॒हु - भ्या॒म् । \newline
40. पू॒ष्णो हस्ता᳚भ्याꣳ॒॒ हस्ता᳚भ्याम् पू॒ष्णः पू॒ष्णो हस्ता᳚भ्याम् । \newline
41. हस्ता᳚भ्याꣳ॒॒ रक्ष॑सो॒ रक्ष॑सो॒ हस्ता᳚भ्याꣳ॒॒ हस्ता᳚भ्याꣳ॒॒ रक्ष॑सः । \newline
42. रक्ष॑सोव॒धं ॅव॒धꣳ रक्ष॑सो॒ रक्ष॑सोव॒धम् । \newline
43. व॒धम् जु॑होमि जुहोमि व॒धं ॅव॒धम् जु॑होमि । \newline
44. जु॒हो॒मि॒ ह॒तꣳ ह॒तम् जु॑होमि जुहोमि ह॒तम् । \newline
45. ह॒तꣳ रक्षो॒ रक्षो॑ ह॒तꣳ ह॒तꣳ रक्षः॑ । \newline
46. रक्षो ऽव॑धि॒ष्मा व॑धिष्म॒ रक्षो॒ रक्षो ऽव॑धिष्म । \newline
47. अव॑धिष्म॒ रक्षो॒ रक्षो ऽव॑धि॒ष्मा व॑धिष्म॒ रक्षः॑ । \newline
48. रक्षो॒ यद् यद् रक्षो॒ रक्षो॒ यत् । \newline
49. यद् वस्ते॒ वस्ते॒ यद् यद् वस्ते᳚ । \newline
50. वस्ते॒ तत् तद् वस्ते॒ वस्ते॒ तत् । \newline
51. तद् दक्षि॑णा॒ दक्षि॑णा॒ तत् तद् दक्षि॑णा । \newline
52. दक्षि॒णेति॒ दक्षि॑णा । \newline

\textbf{Ghana Paata } \newline

1. नम॒ स्तेभ्य॒ स्तेभ्यो॒ नमो॒ नम॒ स्तेभ्यः॒ स्वाहा॒ स्वाहा॒ तेभ्यो॒ नमो॒ नम॒ स्तेभ्यः॒ स्वाहा᳚ । \newline
2. तेभ्यः॒ स्वाहा॒ स्वाहा॒ तेभ्य॒ स्तेभ्यः॒ स्वाहा॒ समू॑ढꣳ॒॒ समू॑ढꣳ॒॒ स्वाहा॒ तेभ्य॒ स्तेभ्यः॒ स्वाहा॒ समू॑ढम् । \newline
3. स्वाहा॒ समू॑ढꣳ॒॒ समू॑ढꣳ॒॒ स्वाहा॒ स्वाहा॒ समू॑ढꣳ॒॒ रक्षो॒ रक्षः॒ समू॑ढꣳ॒॒ स्वाहा॒ स्वाहा॒ समू॑ढꣳ॒॒ रक्षः॑ । \newline
4. समू॑ढꣳ॒॒ रक्षो॒ रक्षः॒ समू॑ढꣳ॒॒ समू॑ढꣳ॒॒ रक्षः॒ सन्द॑ग्धꣳ॒॒ सन्द॑ग्धꣳ॒॒ रक्षः॒ समू॑ढꣳ॒॒ समू॑ढꣳ॒॒ रक्षः॒ सन्द॑ग्धम् । \newline
5. समू॑ढ॒मिति॒ सं - ऊ॒ढ॒म् । \newline
6. रक्षः॒ सन्द॑ग्धꣳ॒॒ सन्द॑ग्धꣳ॒॒ रक्षो॒ रक्षः॒ सन्द॑ग्धꣳ॒॒ रक्षो॒ रक्षः॒ सन्द॑ग्धꣳ॒॒ रक्षो॒ रक्षः॒ सन्द॑ग्धꣳ॒॒ रक्षः॑ । \newline
7. सन्द॑ग्धꣳ॒॒ रक्षो॒ रक्षः॒ सन्द॑ग्धꣳ॒॒ सन्द॑ग्धꣳ॒॒ रक्ष॑ इ॒द मि॒दꣳ रक्षः॒ सन्द॑ग्धꣳ॒॒ सन्द॑ग्धꣳ॒॒ रक्ष॑ इ॒दम् । \newline
8. सन्द॑ग्ध॒मिति॒ सं - द॒ग्ध॒म् । \newline
9. रक्ष॑ इ॒द मि॒दꣳ रक्षो॒ रक्ष॑ इ॒द म॒ह म॒ह मि॒दꣳ रक्षो॒ रक्ष॑ इ॒द म॒हम् । \newline
10. इ॒द म॒ह म॒ह मि॒द मि॒द म॒हꣳ रक्षो॒ रक्षो॒ ऽह मि॒द मि॒द म॒हꣳ रक्षः॑ । \newline
11. अ॒हꣳ रक्षो॒ रक्षो॒ ऽह म॒हꣳ रक्षो॒ ऽभ्य॑भि रक्षो॒ ऽह म॒हꣳ रक्षो॒ ऽभि । \newline
12. रक्षो॒ ऽभ्य॑भि रक्षो॒ रक्षो॒ ऽभि सꣳ स म॒भि रक्षो॒ रक्षो॒ ऽभि सम् । \newline
13. अ॒भि सꣳ स म॒भ्य॑भि सम् द॑हामि दहामि॒ स म॒भ्य॑भि सम् द॑हामि । \newline
14. सम् द॑हामि दहामि॒ सꣳ सम् द॑हा म्य॒ग्नये॒ ऽग्नये॑ दहामि॒ सꣳ सम् द॑हा म्य॒ग्नये᳚ । \newline
15. द॒हा॒ म्य॒ग्नये॒ ऽग्नये॑ दहामि दहा म्य॒ग्नये॑ रक्षो॒घ्ने र॑क्षो॒घ्ने᳚ ऽग्नये॑ दहामि दहा म्य॒ग्नये॑ रक्षो॒घ्ने । \newline
16. अ॒ग्नये॑ रक्षो॒घ्ने र॑क्षो॒घ्ने᳚ ऽग्नये॒ ऽग्नये॑ रक्षो॒घ्ने स्वाहा॒ स्वाहा॑ रक्षो॒घ्ने᳚ ऽग्नये॒ ऽग्नये॑ रक्षो॒घ्ने स्वाहा᳚ । \newline
17. र॒क्षो॒घ्ने स्वाहा॒ स्वाहा॑ रक्षो॒घ्ने र॑क्षो॒घ्ने स्वाहा॑ य॒माय॑ य॒माय॒ स्वाहा॑ रक्षो॒घ्ने र॑क्षो॒घ्ने स्वाहा॑ य॒माय॑ । \newline
18. र॒क्षो॒घ्न इति॑ रक्षः - घ्ने । \newline
19. स्वाहा॑ य॒माय॑ य॒माय॒ स्वाहा॒ स्वाहा॑ य॒माय॑ सवि॒त्रे स॑वि॒त्रे य॒माय॒ स्वाहा॒ स्वाहा॑ य॒माय॑ सवि॒त्रे । \newline
20. य॒माय॑ सवि॒त्रे स॑वि॒त्रे य॒माय॑ य॒माय॑ सवि॒त्रे वरु॑णाय॒ वरु॑णाय सवि॒त्रे य॒माय॑ य॒माय॑ सवि॒त्रे वरु॑णाय । \newline
21. स॒वि॒त्रे वरु॑णाय॒ वरु॑णाय सवि॒त्रे स॑वि॒त्रे वरु॑णाय॒ बृह॒स्पत॑ये॒ बृह॒स्पत॑ये॒ वरु॑णाय सवि॒त्रे स॑वि॒त्रे वरु॑णाय॒ बृह॒स्पत॑ये । \newline
22. वरु॑णाय॒ बृह॒स्पत॑ये॒ बृह॒स्पत॑ये॒ वरु॑णाय॒ वरु॑णाय॒ बृह॒स्पत॑ये॒ दुव॑स्वते॒ दुव॑स्वते॒ बृह॒स्पत॑ये॒ वरु॑णाय॒ वरु॑णाय॒ बृह॒स्पत॑ये॒ दुव॑स्वते । \newline
23. बृह॒स्पत॑ये॒ दुव॑स्वते॒ दुव॑स्वते॒ बृह॒स्पत॑ये॒ बृह॒स्पत॑ये॒ दुव॑स्वते रक्षो॒घ्ने र॑क्षो॒घ्ने दुव॑स्वते॒ बृह॒स्पत॑ये॒ बृह॒स्पत॑ये॒ दुव॑स्वते रक्षो॒घ्ने । \newline
24. दुव॑स्वते रक्षो॒घ्ने र॑क्षो॒घ्ने दुव॑स्वते॒ दुव॑स्वते रक्षो॒घ्ने स्वाहा॒ स्वाहा॑ रक्षो॒घ्ने दुव॑स्वते॒ दुव॑स्वते रक्षो॒घ्ने स्वाहा᳚ । \newline
25. र॒क्षो॒घ्ने स्वाहा॒ स्वाहा॑ रक्षो॒घ्ने र॑क्षो॒घ्ने स्वाहा᳚ प्रष्टिवा॒ही प्र॑ष्टिवा॒ही स्वाहा॑ रक्षो॒घ्ने र॑क्षो॒घ्ने स्वाहा᳚ प्रष्टिवा॒ही । \newline
26. र॒क्षो॒घ्न इति॑ रक्षः - घ्ने । \newline
27. स्वाहा᳚ प्रष्टिवा॒ही प्र॑ष्टिवा॒ही स्वाहा॒ स्वाहा᳚ प्रष्टिवा॒ही रथो॒ रथः॑ प्रष्टिवा॒ही स्वाहा॒ स्वाहा᳚ प्रष्टिवा॒ही रथः॑ । \newline
28. प्र॒ष्टि॒वा॒ही रथो॒ रथः॑ प्रष्टिवा॒ही प्र॑ष्टिवा॒ही रथो॒ दक्षि॑णा॒ दक्षि॑णा॒ रथः॑ प्रष्टिवा॒ही प्र॑ष्टिवा॒ही रथो॒ दक्षि॑णा । \newline
29. प्र॒ष्टि॒वा॒हीति॑ प्रष्टि - वा॒ही । \newline
30. रथो॒ दक्षि॑णा॒ दक्षि॑णा॒ रथो॒ रथो॒ दक्षि॑णा दे॒वस्य॑ दे॒वस्य॒ दक्षि॑णा॒ रथो॒ रथो॒ दक्षि॑णा दे॒वस्य॑ । \newline
31. दक्षि॑णा दे॒वस्य॑ दे॒वस्य॒ दक्षि॑णा॒ दक्षि॑णा दे॒वस्य॑ त्वा त्वा दे॒वस्य॒ दक्षि॑णा॒ दक्षि॑णा दे॒वस्य॑ त्वा । \newline
32. दे॒वस्य॑ त्वा त्वा दे॒वस्य॑ दे॒वस्य॑ त्वा सवि॒तुः स॑वि॒तु स्त्वा॑ दे॒वस्य॑ दे॒वस्य॑ त्वा सवि॒तुः । \newline
33. त्वा॒ स॒वि॒तुः स॑वि॒तु स्त्वा᳚ त्वा सवि॒तुः प्र॑स॒वे प्र॑स॒वे स॑वि॒तु स्त्वा᳚ त्वा सवि॒तुः प्र॑स॒वे । \newline
34. स॒वि॒तुः प्र॑स॒वे प्र॑स॒वे स॑वि॒तुः स॑वि॒तुः प्र॑स॒वे᳚ ऽश्विनो॑ र॒श्विनोः᳚ प्रस॒वे स॑वि॒तुः स॑वि॒तुः प्र॑स॒वे᳚ ऽश्विनोः᳚ । \newline
35. प्र॒स॒वे᳚ ऽश्विनो॑ र॒श्विनोः᳚ प्रस॒वे प्र॑स॒वे᳚ ऽश्विनो᳚र् बा॒हुभ्या᳚म् बा॒हुभ्या॑ म॒श्विनोः᳚ प्रस॒वे प्र॑स॒वे᳚ ऽश्विनो᳚र् बा॒हुभ्या᳚म् । \newline
36. प्र॒स॒व इति॑ प्र - स॒वे । \newline
37. अ॒श्विनो᳚र् बा॒हुभ्या᳚म् बा॒हुभ्या॑ म॒श्विनो॑ र॒श्विनो᳚र् बा॒हुभ्या᳚म् पू॒ष्णः पू॒ष्णो बा॒हुभ्या॑ म॒श्विनो॑ र॒श्विनो᳚र् बा॒हुभ्या᳚म् पू॒ष्णः । \newline
38. बा॒हुभ्या᳚म् पू॒ष्णः पू॒ष्णो बा॒हुभ्या᳚म् बा॒हुभ्या᳚म् पू॒ष्णो हस्ता᳚भ्याꣳ॒॒ हस्ता᳚भ्याम् पू॒ष्णो बा॒हुभ्या᳚म् बा॒हुभ्या᳚म् पू॒ष्णो हस्ता᳚भ्याम् । \newline
39. बा॒हुभ्या॒मिति॑ बा॒हु - भ्या॒म् । \newline
40. पू॒ष्णो हस्ता᳚भ्याꣳ॒॒ हस्ता᳚भ्याम् पू॒ष्णः पू॒ष्णो हस्ता᳚भ्याꣳ॒॒ रक्ष॑सो॒ रक्ष॑सो॒ हस्ता᳚भ्याम् पू॒ष्णः पू॒ष्णो हस्ता᳚भ्याꣳ॒॒ रक्ष॑सः । \newline
41. हस्ता᳚भ्याꣳ॒॒ रक्ष॑सो॒ रक्ष॑सो॒ हस्ता᳚भ्याꣳ॒॒ हस्ता᳚भ्याꣳ॒॒ रक्ष॑सो व॒धं ॅव॒धꣳ रक्ष॑सो॒ हस्ता᳚भ्याꣳ॒॒ हस्ता᳚भ्याꣳ॒॒ रक्ष॑सो व॒धम् । \newline
42. रक्ष॑सो व॒धं ॅव॒धꣳ रक्ष॑सो॒ रक्ष॑सो व॒धम् जु॑होमि जुहोमि व॒धꣳ रक्ष॑सो॒ रक्ष॑सो व॒धम् जु॑होमि । \newline
43. व॒धम् जु॑होमि जुहोमि व॒धं ॅव॒धम् जु॑होमि ह॒तꣳ ह॒तम् जु॑होमि व॒धं ॅव॒धम् जु॑होमि ह॒तम् । \newline
44. जु॒हो॒मि॒ ह॒तꣳ ह॒तम् जु॑होमि जुहोमि ह॒तꣳ रक्षो॒ रक्षो॑ ह॒तम् जु॑होमि जुहोमि ह॒तꣳ रक्षः॑ । \newline
45. ह॒तꣳ रक्षो॒ रक्षो॑ ह॒तꣳ ह॒तꣳ रक्षो ऽव॑धि॒ष्मा व॑धिष्म॒ रक्षो॑ ह॒तꣳ ह॒तꣳ रक्षो ऽव॑धिष्म । \newline
46. रक्षो ऽव॑धि॒ष्मा व॑धिष्म॒ रक्षो॒ रक्षो ऽव॑धिष्म॒ रक्षो॒ रक्षो ऽव॑धिष्म॒ रक्षो॒ रक्षो ऽव॑धिष्म॒ रक्षः॑ । \newline
47. अव॑धिष्म॒ रक्षो॒ रक्षो ऽव॑धि॒ष्मा व॑धिष्म॒ रक्षो॒ यद् यद् रक्षो ऽव॑धि॒ष्मा व॑धिष्म॒ रक्षो॒ यत् । \newline
48. रक्षो॒ यद् यद् रक्षो॒ रक्षो॒ यद् वस्ते॒ वस्ते॒ यद् रक्षो॒ रक्षो॒ यद् वस्ते᳚ । \newline
49. यद् वस्ते॒ वस्ते॒ यद् यद् वस्ते॒ तत् तद् वस्ते॒ यद् यद् वस्ते॒ तत् । \newline
50. वस्ते॒ तत् तद् वस्ते॒ वस्ते॒ तद् दक्षि॑णा॒ दक्षि॑णा॒ तद् वस्ते॒ वस्ते॒ तद् दक्षि॑णा । \newline
51. तद् दक्षि॑णा॒ दक्षि॑णा॒ तत् तद् दक्षि॑णा । \newline
52. दक्षि॒णेति॒ दक्षि॑णा । \newline
\pagebreak
\markright{ TS 1.8.8.1  \hfill https://www.vedavms.in \hfill}
\addcontentsline{toc}{section}{ TS 1.8.8.1 }
\section*{ TS 1.8.8.1 }

\textbf{TS 1.8.8.1 } \newline
\textbf{Samhita Paata} \newline

धा॒त्रे पु॑रो॒डाशं॒ द्वाद॑शकपालं॒ निर्व॑प॒त्यनु॑मत्यै च॒रुꣳ रा॒कायै॑ च॒रुꣳ सि॑नीवा॒ल्यै च॒रुं कु॒ह्वै॑ च॒रुं मि॑थु॒नौ गावौ॒ दक्षि॑णा ऽऽग्नावैष्ण॒व-मेका॑दशकपालं॒ निर्व॑पत्यैन्द्रावैष्ण॒व-मेका॑दशकपालं ॅवैष्ण॒वं त्रि॑कपा॒लं ॅवा॑म॒नो व॒ही दक्षि॑णा-ऽग्नीषो॒मीय॒-मेका॑दशकपालं॒ निर्व॑पतीन्द्रासो॒मीय॒- मेका॑दशकपालꣳ सौ॒म्यं च॒रुं ब॒भ्रुर् दक्षि॑णा सोमापौ॒ष्णं च॒रुं निर्व॑पत्यैन्द्रा पौ॒ष्णं च॒रुं पौ॒ष्णं च॒रुꣳ श्या॒मो दक्षि॑णा वैश्वान॒रं द्वाद॑शकपालं॒ निर् ( ) व॑पति॒ हिर॑ण्यं॒ दक्षि॑णा वारु॒णं ॅय॑व॒मयं॑ च॒रुमश्वो॒ दक्षि॑णा ॥ \newline

\textbf{Pada Paata} \newline

धा॒त्रे । पु॒रो॒डाश᳚म् । द्वाद॑शकपाल॒मिति॒ द्वाद॑श - क॒पा॒ल॒म् । निरिति॑ । व॒प॒ति॒ । अनु॑मत्या॒ इत्यनु॑ - म॒त्यै॒ । च॒रुम् । रा॒कायै᳚ । च॒रुम् । सि॒नी॒वा॒ल्यै । च॒रुम् । कु॒ह्वै᳚ । च॒रुम् । मि॒थु॒नौ । गावौ᳚ । दक्षि॑णा । आ॒ग्ना॒वै॒ष्ण॒वमित्या᳚ग्ना -वै॒ष्ण॒वम् । एका॑दशकपाल॒मित्येका॑दश - क॒पा॒ल॒म् । निरिति॑ । व॒प॒ति॒ । ऐ॒न्द्रा॒वै॒ष्ण॒वमित्यै᳚न्द्रा - वै॒ष्ण॒वम् । एका॑दशकपाल॒मित्येका॑दश - क॒पा॒ल॒म् । वै॒ष्ण॒वम् । त्रि॒क॒पा॒लमिति॑ त्रि - क॒पा॒लम् । वा॒म॒नः । व॒ही । दक्षि॑णा । अ॒ग्नी॒षो॒मीय॒मित्य॑ग्नी - सो॒मीय᳚म् । एका॑दशकपाल॒मित्येका॑दश - क॒पा॒ल॒म् । निरिति॑ । व॒प॒ति॒ । इ॒न्द्रा॒सो॒मीय॒मिती᳚न्द्रा - सो॒मीय᳚म् । एका॑दशकपाल॒मित्येका॑दश - क॒पा॒ल॒म् । सौ॒म्यम् । च॒रुम् । ब॒भ्रुः । दक्षि॑णा । सो॒मा॒पौ॒ष्णमिति॑ सोमा - पौ॒ष्णम् । च॒रुम् । निरिति॑ । व॒प॒ति॒ । ऐ॒न्द्रा॒पौ॒ष्णमित्यै᳚न्द्रा - पौ॒ष्णम् । च॒रुम् । पौ॒ष्णम् । च॒रुम् । श्या॒मः । दक्षि॑णा । वै॒श्वा॒न॒रम् । द्वाद॑शकपाल॒मिति॒ द्वाद॑श-क॒पा॒ल॒म् । निरिति॑ ( ) । व॒प॒ति॒ । हिर॑ण्यम् । दक्षि॑णा । वा॒रु॒णम् । य॒व॒मय॒मिति॑ यव - मय᳚म् । च॒रुम् । अश्वः॑ । दक्षि॑णा ॥  \newline



\textbf{Jatai Paata} \newline

1. धा॒त्रे पु॑रो॒डाश॑म् पुरो॒डाश॑म् धा॒त्रे धा॒त्रे पु॑रो॒डाश᳚म् । \newline
2. पु॒रो॒डाश॒म् द्वाद॑शकपाल॒म् द्वाद॑शकपालम् पुरो॒डाश॑म् पुरो॒डाश॒म् द्वाद॑शकपालम् । \newline
3. द्वाद॑शकपाल॒म् निर् णिर् द्वाद॑शकपाल॒म् द्वाद॑शकपाल॒म् निः । \newline
4. द्वाद॑शकपाल॒मिति॒ द्वाद॑श - क॒पा॒ल॒म् । \newline
5. निर् व॑पति वपति॒ निर् णिर् व॑पति । \newline
6. व॒प॒ त्यनु॑मत्या॒ अनु॑मत्यै वपति वप॒ त्यनु॑मत्यै । \newline
7. अनु॑मत्यै च॒रुम् च॒रु मनु॑मत्या॒ अनु॑मत्यै च॒रुम् । \newline
8. अनु॑मत्या॒ इत्यनु॑ - म॒त्यै॒ । \newline
9. च॒रुꣳ रा॒कायै॑ रा॒कायै॑ च॒रुम् च॒रुꣳ रा॒कायै᳚ । \newline
10. रा॒कायै॑ च॒रुम् च॒रुꣳ रा॒कायै॑ रा॒कायै॑ च॒रुम् । \newline
11. च॒रुꣳ सि॑नीवा॒ल्यै सि॑नीवा॒ल्यै च॒रुम् च॒रुꣳ सि॑नीवा॒ल्यै । \newline
12. सि॒नी॒वा॒ल्यै च॒रुम् च॒रुꣳ सि॑नीवा॒ल्यै सि॑नीवा॒ल्यै च॒रुम् । \newline
13. च॒रुम् कु॒ह्वै॑ कु॒ह्वै॑ च॒रुम् च॒रुम् कु॒ह्वै᳚ । \newline
14. कु॒ह्वै॑ च॒रुम् च॒रुम् कु॒ह्वै॑ कु॒ह्वै॑ च॒रुम् । \newline
15. च॒रुम् मि॑थु॒नौ मि॑थु॒नौ च॒रुम् च॒रुम् मि॑थु॒नौ । \newline
16. मि॒थु॒नौ गावौ॒ गावौ॑ मिथु॒नौ मि॑थु॒नौ गावौ᳚ । \newline
17. गावौ॒ दक्षि॑णा॒ दक्षि॑णा॒ गावौ॒ गावौ॒ दक्षि॑णा । \newline
18. दक्षि॑णा ऽऽग्नावैष्ण॒व मा᳚ग्नावैष्ण॒वम् दक्षि॑णा॒ दक्षि॑णा ऽऽग्नावैष्ण॒वम् । \newline
19. आ॒ग्ना॒वै॒ष्ण॒व मेका॑दशकपाल॒ मेका॑दशकपाल माग्नावैष्ण॒व मा᳚ग्नावैष्ण॒व मेका॑दशकपालम् । \newline
20. आ॒ग्ना॒वै॒ष्ण॒वमित्या᳚ग्ना - वै॒ष्ण॒वम् । \newline
21. एका॑दशकपाल॒म् निर् णिरेका॑दशकपाल॒ मेका॑दशकपाल॒म् निः । \newline
22. एका॑दशकपाल॒मित्येका॑दश - क॒पा॒ल॒म् । \newline
23. निर् व॑पति वपति॒ निर् णिर् व॑पति । \newline
24. व॒प॒ त्यै॒न्द्रा॒वै॒ष्ण॒व मै᳚न्द्रावैष्ण॒वं ॅव॑पति वप त्यैन्द्रावैष्ण॒वम् । \newline
25. ऐ॒न्द्रा॒वै॒ष्ण॒व मेका॑दशकपाल॒ मेका॑दशकपाल मैन्द्रावैष्ण॒व मै᳚न्द्रावैष्ण॒व मेका॑दशकपालम् । \newline
26. ऐ॒न्द्रा॒वै॒ष्ण॒वमित्यै᳚न्द्रा - वै॒ष्ण॒वम् । \newline
27. एका॑दशकपालं ॅवैष्ण॒वं ॅवै᳚ष्ण॒व मेका॑दशकपाल॒ मेका॑दशकपालं ॅवैष्ण॒वम् । \newline
28. एका॑दशकपाल॒मित्येका॑दश - क॒पा॒ल॒म् । \newline
29. वै॒ष्ण॒वम् त्रि॑कपा॒लम् त्रि॑कपा॒लं ॅवै᳚ष्ण॒वं ॅवै᳚ष्ण॒वम् त्रि॑कपा॒लम् । \newline
30. त्रि॒क॒पा॒लं ॅवा॑म॒नो वा॑म॒न स्त्रि॑कपा॒लम् त्रि॑कपा॒लं ॅवा॑म॒नः । \newline
31. त्रि॒क॒पा॒लमिति॑ त्रि - क॒पा॒लम् । \newline
32. वा॒म॒नो व॒ही व॒ही वा॑म॒नो वा॑म॒नो व॒ही । \newline
33. व॒ही दक्षि॑णा॒ दक्षि॑णा व॒ही व॒ही दक्षि॑णा । \newline
34. दक्षि॑णा ऽग्नीषो॒मीय॑ मग्नीषो॒मीय॒म् दक्षि॑णा॒ दक्षि॑णा ऽग्नीषो॒मीय᳚म् । \newline
35. अ॒ग्नी॒षो॒मीय॒ मेका॑दशकपाल॒ मेका॑दशकपाल मग्नीषो॒मीय॑ मग्नीषो॒मीय॒ मेका॑दशकपालम् । \newline
36. अ॒ग्नी॒षो॒मीय॒मित्य॑ग्नी - सो॒मीय᳚म् । \newline
37. एका॑दशकपाल॒म् निर् णिरेका॑दशकपाल॒ मेका॑दशकपाल॒म् निः । \newline
38. एका॑दशकपाल॒मित्येका॑दश - क॒पा॒ल॒म् । \newline
39. निर् व॑पति वपति॒ निर् णिर् व॑पति । \newline
40. व॒प॒ ती॒न्द्रा॒सो॒मीय॑ मिन्द्रासो॒मीयं॑ ॅवपति वप तीन्द्रासो॒मीय᳚म् । \newline
41. इ॒न्द्रा॒सो॒मीय॒ मेका॑दशकपाल॒ मेका॑दशकपाल मिन्द्रासो॒मीय॑ मिन्द्रासो॒मीय॒ मेका॑दशकपालम् । \newline
42. इ॒न्द्रा॒सो॒मीय॒मिती᳚न्द्रा - सो॒मीय᳚म् । \newline
43. एका॑दशकपालꣳ सौ॒म्यꣳ सौ॒म्य मेका॑दशकपाल॒ मेका॑दशकपालꣳ सौ॒म्यम् । \newline
44. एका॑दशकपाल॒मित्येका॑दश - क॒पा॒ल॒म् । \newline
45. सौ॒म्यम् च॒रुम् च॒रुꣳ सौ॒म्यꣳ सौ॒म्यम् च॒रुम् । \newline
46. च॒रुम् ब॒भ्रुर् ब॒भ्रु श्च॒रुम् च॒रुम् ब॒भ्रुः । \newline
47. ब॒भ्रुर् दक्षि॑णा॒ दक्षि॑णा ब॒भ्रुर् ब॒भ्रुर् दक्षि॑णा । \newline
48. दक्षि॑णा सोमापौ॒ष्णꣳ सो॑मापौ॒ष्णम् दक्षि॑णा॒ दक्षि॑णा सोमापौ॒ष्णम् । \newline
49. सो॒मा॒पौ॒ष्णम् च॒रुम् च॒रुꣳ सो॑मापौ॒ष्णꣳ सो॑मापौ॒ष्णम् च॒रुम् । \newline
50. सो॒मा॒पौ॒ष्णमिति॑ सोमा - पौ॒ष्णम् । \newline
51. च॒रुम् निर् णिश्च॒रुम् च॒रुम् निः । \newline
52. निर् व॑पति वपति॒ निर् णिर् व॑पति । \newline
53. व॒प॒ त्यै॒न्द्रा॒पौ॒ष्ण मै᳚न्द्रापौ॒ष्णं ॅव॑पति वप त्यैन्द्रापौ॒ष्णम् । \newline
54. ऐ॒न्द्रा॒पौ॒ष्णम् च॒रुम् च॒रु मै᳚न्द्रापौ॒ष्ण मै᳚न्द्रापौ॒ष्णम् च॒रुम् । \newline
55. ऐ॒न्द्रा॒पौ॒ष्णमित्यै᳚न्द्रा - पौ॒ष्णम् । \newline
56. च॒रुम् पौ॒ष्णम् पौ॒ष्णम् च॒रुम् च॒रुम् पौ॒ष्णम् । \newline
57. पौ॒ष्णम् च॒रुम् च॒रुम् पौ॒ष्णम् पौ॒ष्णम् च॒रुम् । \newline
58. च॒रुꣳ श्या॒मः श्या॒म श्च॒रुम् च॒रुꣳ श्या॒मः । \newline
59. श्या॒मो दक्षि॑णा॒ दक्षि॑णा श्या॒मः श्या॒मो दक्षि॑णा । \newline
60. दक्षि॑णा वैश्वान॒रं ॅवै᳚श्वान॒रम् दक्षि॑णा॒ दक्षि॑णा वैश्वान॒रम् । \newline
61. वै॒श्वा॒न॒रम् द्वाद॑शकपाल॒म् द्वाद॑शकपालं ॅवैश्वान॒रं ॅवै᳚श्वान॒रम् द्वाद॑शकपालम् । \newline
62. द्वाद॑शकपाल॒म् निर् णिर् द्वाद॑शकपाल॒म् द्वाद॑शकपाल॒म् निः । \newline
63. द्वाद॑शकपाल॒मिति॒ द्वाद॑श - क॒पा॒ल॒म् । \newline
64. निर् व॑पति वपति॒ निर् णिर् व॑पति । \newline
65. व॒प॒ति॒ हिर॑ण्यꣳ॒॒ हिर॑ण्यं ॅवपति वपति॒ हिर॑ण्यम् । \newline
66. हिर॑ण्य॒म् दक्षि॑णा॒ दक्षि॑णा॒ हिर॑ण्यꣳ॒॒ हिर॑ण्य॒म् दक्षि॑णा । \newline
67. दक्षि॑णा वारु॒णं ॅवा॑रु॒णम् दक्षि॑णा॒ दक्षि॑णा वारु॒णम् । \newline
68. वा॒रु॒णं ॅय॑व॒मयं॑ ॅयव॒मयं॑ ॅवारु॒णं ॅवा॑रु॒णं ॅय॑व॒मय᳚म् । \newline
69. य॒व॒मय॑म् च॒रुम् च॒रुं ॅय॑व॒मयं॑ ॅयव॒मय॑म् च॒रुम् । \newline
70. य॒व॒मय॒मिति॑ यव - मय᳚म् । \newline
71. च॒रु मश्वो ऽश्व॑ श्च॒रुम् च॒रु मश्वः॑ । \newline
72. अश्वो॒ दक्षि॑णा॒ दक्षि॒णा ऽश्वो ऽश्वो॒ दक्षि॑णा । \newline
73. दक्षि॒णेति॒ दक्षि॑णा । \newline

\textbf{Ghana Paata } \newline

1. धा॒त्रे पु॑रो॒डाश॑म् पुरो॒डाश॑म् धा॒त्रे धा॒त्रे पु॑रो॒डाश॒म् द्वाद॑शकपाल॒म् द्वाद॑शकपालम् पुरो॒डाश॑म् धा॒त्रे धा॒त्रे पु॑रो॒डाश॒म् द्वाद॑शकपालम् । \newline
2. पु॒रो॒डाश॒म् द्वाद॑शकपाल॒म् द्वाद॑शकपालम् पुरो॒डाश॑म् पुरो॒डाश॒म् द्वाद॑शकपाल॒न्निर् णिर् द्वाद॑शकपालम् पुरो॒डाश॑म् पुरो॒डाश॒म् द्वाद॑शकपाल॒न् निः । \newline
3. द्वाद॑शकपाल॒न्निर् णिर् द्वाद॑शकपाल॒म् द्वाद॑शकपाल॒न् निर् व॑पति वपति॒ निर् द्वाद॑शकपाल॒म् द्वाद॑शकपाल॒न् निर् व॑पति । \newline
4. द्वाद॑शकपाल॒मिति॒ द्वाद॑श - क॒पा॒ल॒म् । \newline
5. निर् व॑पति वपति॒ निर् णिर् व॑प॒ त्यनु॑मत्या॒ अनु॑मत्यै वपति॒ निर् णिर् व॑प॒ त्यनु॑मत्यै । \newline
6. व॒प॒ त्यनु॑मत्या॒ अनु॑मत्यै वपति वप॒ त्यनु॑मत्यै च॒रुम् च॒रु मनु॑मत्यै वपति वप॒ त्यनु॑मत्यै च॒रुम् । \newline
7. अनु॑मत्यै च॒रुम् च॒रु मनु॑मत्या॒ अनु॑मत्यै च॒रुꣳ रा॒कायै॑ रा॒कायै॑ च॒रु मनु॑मत्या॒ अनु॑मत्यै च॒रुꣳ रा॒कायै᳚ । \newline
8. अनु॑मत्या॒ इत्यनु॑ - म॒त्यै॒ । \newline
9. च॒रुꣳ रा॒कायै॑ रा॒कायै॑ च॒रुम् च॒रुꣳ रा॒कायै॑ च॒रुम् च॒रुꣳ रा॒कायै॑ च॒रुम् च॒रुꣳ रा॒कायै॑ च॒रुम् । \newline
10. रा॒कायै॑ च॒रुम् च॒रुꣳ रा॒कायै॑ रा॒कायै॑ च॒रुꣳ सि॑नीवा॒ल्यै सि॑नीवा॒ल्यै च॒रुꣳ रा॒कायै॑ रा॒कायै॑ च॒रुꣳ सि॑नीवा॒ल्यै । \newline
11. च॒रुꣳ सि॑नीवा॒ल्यै सि॑नीवा॒ल्यै च॒रुम् च॒रुꣳ सि॑नीवा॒ल्यै च॒रुम् च॒रुꣳ सि॑नीवा॒ल्यै च॒रुम् च॒रुꣳ सि॑नीवा॒ल्यै च॒रुम् । \newline
12. सि॒नी॒वा॒ल्यै च॒रुम् च॒रुꣳ सि॑नीवा॒ल्यै सि॑नीवा॒ल्यै च॒रुम् कु॒ह्वै॑ कु॒ह्वै॑ च॒रुꣳ सि॑नीवा॒ल्यै सि॑नीवा॒ल्यै च॒रुम् कु॒ह्वै᳚ । \newline
13. च॒रुम् कु॒ह्वै॑ कु॒ह्वै॑ च॒रुम् च॒रुम् कु॒ह्वै॑ च॒रुम् च॒रुम् कु॒ह्वै॑ च॒रुम् च॒रुम् कु॒ह्वै॑ च॒रुम् । \newline
14. कु॒ह्वै॑ च॒रुम् च॒रुम् कु॒ह्वै॑ कु॒ह्वै॑ च॒रुम् मि॑थु॒नौ मि॑थु॒नौ च॒रुम् कु॒ह्वै॑ कु॒ह्वै॑ च॒रुम् मि॑थु॒नौ । \newline
15. च॒रुम् मि॑थु॒नौ मि॑थु॒नौ च॒रुम् च॒रुम् मि॑थु॒नौ गावौ॒ गावौ॑ मिथु॒नौ च॒रुम् च॒रुम् मि॑थु॒नौ गावौ᳚ । \newline
16. मि॒थु॒नौ गावौ॒ गावौ॑ मिथु॒नौ मि॑थु॒नौ गावौ॒ दक्षि॑णा॒ दक्षि॑णा॒ गावौ॑ मिथु॒नौ मि॑थु॒नौ गावौ॒ दक्षि॑णा । \newline
17. गावौ॒ दक्षि॑णा॒ दक्षि॑णा॒ गावौ॒ गावौ॒ दक्षि॑णा ऽऽग्नावैष्ण॒व मा᳚ग्नावैष्ण॒वम् दक्षि॑णा॒ गावौ॒ गावौ॒ दक्षि॑णा ऽऽग्नावैष्ण॒वम् । \newline
18. दक्षि॑णा ऽऽग्नावैष्ण॒व मा᳚ग्नावैष्ण॒वम् दक्षि॑णा॒ दक्षि॑णा ऽऽग्नावैष्ण॒व मेका॑दशकपाल॒ मेका॑दशकपाल माग्नावैष्ण॒वम् दक्षि॑णा॒ दक्षि॑णा ऽऽग्नावैष्ण॒व मेका॑दशकपालम् । \newline
19. आ॒ग्ना॒वै॒ष्ण॒व मेका॑दशकपाल॒ मेका॑दशकपाल माग्नावैष्ण॒व मा᳚ग्नावैष्ण॒व मेका॑दशकपाल॒न्निर् णिरेका॑दशकपाल माग्नावैष्ण॒व मा᳚ग्नावैष्ण॒व मेका॑दशकपाल॒न्निः । \newline
20. आ॒ग्ना॒वै॒ष्ण॒वमित्या᳚ग्ना - वै॒ष्ण॒वम् । \newline
21. एका॑दशकपाल॒न्निर् णिरेका॑दशकपाल॒ मेका॑दशकपाल॒न्निर् व॑पति वपति॒ निरेका॑दशकपाल॒ मेका॑दशकपाल॒न्निर् व॑पति । \newline
22. एका॑दशकपाल॒मित्येका॑दश - क॒पा॒ल॒म् । \newline
23. निर् व॑पति वपति॒ निर् णिर् व॑प त्यैन्द्रावैष्ण॒व मै᳚न्द्रावैष्ण॒वं ॅव॑पति॒ निर् णिर् व॑प त्यैन्द्रावैष्ण॒वम् । \newline
24. व॒प॒ त्यै॒न्द्रा॒वै॒ष्ण॒व मै᳚न्द्रावैष्ण॒वं ॅव॑पति वप त्यैन्द्रावैष्ण॒व मेका॑दशकपाल॒ मेका॑दशकपाल मैन्द्रावैष्ण॒वं ॅव॑पति वप त्यैन्द्रावैष्ण॒व मेका॑दशकपालम् । \newline
25. ऐ॒न्द्रा॒वै॒ष्ण॒व मेका॑दशकपाल॒ मेका॑दशकपाल मैन्द्रावैष्ण॒व मै᳚न्द्रावैष्ण॒व मेका॑दशकपालं ॅवैष्ण॒वं ॅवै᳚ष्ण॒व मेका॑दशकपाल मैन्द्रावैष्ण॒व मै᳚न्द्रावैष्ण॒व मेका॑दशकपालं ॅवैष्ण॒वम् । \newline
26. ऐ॒न्द्रा॒वै॒ष्ण॒वमित्यै᳚न्द्रा - वै॒ष्ण॒वम् । \newline
27. एका॑दशकपालं ॅवैष्ण॒वं ॅवै᳚ष्ण॒व मेका॑दशकपाल॒ मेका॑दशकपालं ॅवैष्ण॒वम् त्रि॑कपा॒लम् त्रि॑कपा॒लं ॅवै᳚ष्ण॒व मेका॑दशकपाल॒ मेका॑दशकपालं ॅवैष्ण॒वम् त्रि॑कपा॒लम् । \newline
28. एका॑दशकपाल॒मित्येका॑दश - क॒पा॒ल॒म् । \newline
29. वै॒ष्ण॒वम् त्रि॑कपा॒लम् त्रि॑कपा॒लं ॅवै᳚ष्ण॒वं ॅवै᳚ष्ण॒वम् त्रि॑कपा॒लं ॅवा॑म॒नो वा॑म॒न स्त्रि॑कपा॒लं ॅवै᳚ष्ण॒वं ॅवै᳚ष्ण॒वम् त्रि॑कपा॒लं ॅवा॑म॒नः । \newline
30. त्रि॒क॒पा॒लं ॅवा॑म॒नो वा॑म॒न स्त्रि॑कपा॒लम् त्रि॑कपा॒लं ॅवा॑म॒नो व॒ही व॒ही वा॑म॒न स्त्रि॑कपा॒लम् त्रि॑कपा॒लं ॅवा॑म॒नो व॒ही । \newline
31. त्रि॒क॒पा॒लमिति॑ त्रि - क॒पा॒लम् । \newline
32. वा॒म॒नो व॒ही व॒ही वा॑म॒नो वा॑म॒नो व॒ही दक्षि॑णा॒ दक्षि॑णा व॒ही वा॑म॒नो वा॑म॒नो व॒ही दक्षि॑णा । \newline
33. व॒ही दक्षि॑णा॒ दक्षि॑णा व॒ही व॒ही दक्षि॑णा ऽग्नीषो॒मीय॑ मग्नीषो॒मीय॒म् दक्षि॑णा व॒ही व॒ही दक्षि॑णा ऽग्नीषो॒मीय᳚म् । \newline
34. दक्षि॑णा ऽग्नीषो॒मीय॑ मग्नीषो॒मीय॒म् दक्षि॑णा॒ दक्षि॑णा ऽग्नीषो॒मीय॒ मेका॑दशकपाल॒ मेका॑दशकपाल मग्नीषो॒मीय॒म् दक्षि॑णा॒ दक्षि॑णा ऽग्नीषो॒मीय॒ मेका॑दशकपालम् । \newline
35. अ॒ग्नी॒षो॒मीय॒ मेका॑दशकपाल॒ मेका॑दशकपाल मग्नीषो॒मीय॑ मग्नीषो॒मीय॒ मेका॑दशकपाल॒न् निर् णिरेका॑दशकपाल मग्नीषो॒मीय॑ मग्नीषो॒मीय॒ मेका॑दशकपाल॒न् निः । \newline
36. अ॒ग्नी॒षो॒मीय॒मित्य॑ग्नी - सो॒मीय᳚म् । \newline
37. एका॑दशकपाल॒न्निर् णिरेका॑दशकपाल॒ मेका॑दशकपाल॒न्निर् व॑पति वपति॒ निरेका॑दशकपाल॒ मेका॑दशकपाल॒न्निर् व॑पति । \newline
38. एका॑दशकपाल॒मित्येका॑दश - क॒पा॒ल॒म् । \newline
39. निर् व॑पति वपति॒ निर् णिर् व॑पतीन्द्रासो॒मीय॑ मिन्द्रासो॒मीयं॑ ॅवपति॒ निर् णिर् व॑पतीन्द्रासो॒मीय᳚म् । \newline
40. व॒प॒ती॒न्द्रा॒सो॒मीय॑ मिन्द्रासो॒मीयं॑ ॅवपति वपतीन्द्रासो॒मीय॒ मेका॑दशकपाल॒ मेका॑दशकपाल मिन्द्रासो॒मीयं॑ ॅवपति वपतीन्द्रासो॒मीय॒ मेका॑दशकपालम् । \newline
41. इ॒न्द्रा॒सो॒मीय॒ मेका॑दशकपाल॒ मेका॑दशकपाल मिन्द्रासो॒मीय॑ मिन्द्रासो॒मीय॒ मेका॑दशकपालꣳ सौ॒म्यꣳ सौ॒म्य मेका॑दशकपाल मिन्द्रासो॒मीय॑ मिन्द्रासो॒मीय॒ मेका॑दशकपालꣳ सौ॒म्यम् । \newline
42. इ॒न्द्रा॒सो॒मीय॒मिती᳚न्द्रा - सो॒मीय᳚म् । \newline
43. एका॑दशकपालꣳ सौ॒म्यꣳ सौ॒म्य मेका॑दशकपाल॒ मेका॑दशकपालꣳ सौ॒म्यम् च॒रुम् च॒रुꣳ सौ॒म्य मेका॑दशकपाल॒ मेका॑दशकपालꣳ सौ॒म्यम् च॒रुम् । \newline
44. एका॑दशकपाल॒मित्येका॑दश - क॒पा॒ल॒म् । \newline
45. सौ॒म्यम् च॒रुम् च॒रुꣳ सौ॒म्यꣳ सौ॒म्यम् च॒रुम् ब॒भ्रुर् ब॒भ्रु श्च॒रुꣳ सौ॒म्यꣳ सौ॒म्यम् च॒रुम् ब॒भ्रुः । \newline
46. च॒रुम् ब॒भ्रुर् ब॒भ्रु श्च॒रुम् च॒रुम् ब॒भ्रुर् दक्षि॑णा॒ दक्षि॑णा ब॒भ्रु श्च॒रुम् च॒रुम् ब॒भ्रुर् दक्षि॑णा । \newline
47. ब॒भ्रुर् दक्षि॑णा॒ दक्षि॑णा ब॒भ्रुर् ब॒भ्रुर् दक्षि॑णा सोमापौ॒ष्णꣳ सो॑मापौ॒ष्णम् दक्षि॑णा ब॒भ्रुर् ब॒भ्रुर् दक्षि॑णा सोमापौ॒ष्णम् । \newline
48. दक्षि॑णा सोमापौ॒ष्णꣳ सो॑मापौ॒ष्णम् दक्षि॑णा॒ दक्षि॑णा सोमापौ॒ष्णम् च॒रुम् च॒रुꣳ सो॑मापौ॒ष्णम् दक्षि॑णा॒ दक्षि॑णा सोमापौ॒ष्णम् च॒रुम् । \newline
49. सो॒मा॒पौ॒ष्णम् च॒रुम् च॒रुꣳ सो॑मापौ॒ष्णꣳ सो॑मापौ॒ष्णम् च॒रुन्निर् णिश्च॒रुꣳ सो॑मापौ॒ष्णꣳ सो॑मापौ॒ष्णम् च॒रुन्निः । \newline
50. सो॒मा॒पौ॒ष्णमिति॑ सोमा - पौ॒ष्णम् । \newline
51. च॒रुन्निर् णिश्च॒रुम् च॒रुन्निर् व॑पति वपति॒ निश्च॒रुम् च॒रुन्निर् व॑पति । \newline
52. निर् व॑पति वपति॒ निर् णिर् व॑पत्यैन्द्रापौ॒ष्ण मै᳚न्द्रापौ॒ष्णं ॅव॑पति॒ निर् णिर् व॑पत्यैन्द्रापौ॒ष्णम् । \newline
53. व॒प॒त्यै॒न्द्रा॒पौ॒ष्ण मै᳚न्द्रापौ॒ष्णं ॅव॑पति वपत्यैन्द्रापौ॒ष्णम् च॒रुम् च॒रु मै᳚न्द्रापौ॒ष्णं ॅव॑पति वपत्यैन्द्रापौ॒ष्णम् च॒रुम् । \newline
54. ऐ॒न्द्रा॒पौ॒ष्णम् च॒रुम् च॒रु मै᳚न्द्रापौ॒ष्ण मै᳚न्द्रापौ॒ष्णम् च॒रुम् पौ॒ष्णम् पौ॒ष्णम् च॒रु मै᳚न्द्रापौ॒ष्ण मै᳚न्द्रापौ॒ष्णम् च॒रुम् पौ॒ष्णम् । \newline
55. ऐ॒न्द्रा॒पौ॒ष्णमित्यै᳚न्द्रा - पौ॒ष्णम् । \newline
56. च॒रुम् पौ॒ष्णम् पौ॒ष्णम् च॒रुम् च॒रुम् पौ॒ष्णम् च॒रुम् च॒रुम् पौ॒ष्णम् च॒रुम् च॒रुम् पौ॒ष्णम् च॒रुम् । \newline
57. पौ॒ष्णम् च॒रुम् च॒रुम् पौ॒ष्णम् पौ॒ष्णम् च॒रुꣳ श्या॒मः श्या॒म श्च॒रुम् पौ॒ष्णम् पौ॒ष्णम् च॒रुꣳ श्या॒मः । \newline
58. च॒रुꣳ श्या॒मः श्या॒म श्च॒रुम् च॒रुꣳ श्या॒मो दक्षि॑णा॒ दक्षि॑णा श्या॒म श्च॒रुम् च॒रुꣳ श्या॒मो दक्षि॑णा । \newline
59. श्या॒मो दक्षि॑णा॒ दक्षि॑णा श्या॒मः श्या॒मो दक्षि॑णा वैश्वान॒रं ॅवै᳚श्वान॒रम् दक्षि॑णा श्या॒मः श्या॒मो दक्षि॑णा वैश्वान॒रम् । \newline
60. दक्षि॑णा वैश्वान॒रं ॅवै᳚श्वान॒रम् दक्षि॑णा॒ दक्षि॑णा वैश्वान॒रम् द्वाद॑शकपाल॒म् द्वाद॑शकपालं ॅवैश्वान॒रम् दक्षि॑णा॒ दक्षि॑णा वैश्वान॒रम् द्वाद॑शकपालम् । \newline
61. वै॒श्वा॒न॒रम् द्वाद॑शकपाल॒म् द्वाद॑शकपालं ॅवैश्वान॒रं ॅवै᳚श्वान॒रम् द्वाद॑शकपाल॒न्निर् णिर् द्वाद॑शकपालं ॅवैश्वान॒रं ॅवै᳚श्वान॒रम् द्वाद॑शकपाल॒न्निः । \newline
62. द्वाद॑शकपाल॒न्निर् णिर् द्वाद॑शकपाल॒म् द्वाद॑शकपाल॒न् निर् व॑पति वपति॒ निर् द्वाद॑शकपाल॒म् द्वाद॑शकपाल॒न् निर् व॑पति । \newline
63. द्वाद॑शकपाल॒मिति॒ द्वाद॑श - क॒पा॒ल॒म् । \newline
64. निर् व॑पति वपति॒ निर् णिर् व॑पति॒ हिर॑ण्यꣳ॒॒ हिर॑ण्यं ॅवपति॒ निर् णिर् व॑पति॒ हिर॑ण्यम् । \newline
65. व॒प॒ति॒ हिर॑ण्यꣳ॒॒ हिर॑ण्यं ॅवपति वपति॒ हिर॑ण्य॒म् दक्षि॑णा॒ दक्षि॑णा॒ हिर॑ण्यं ॅवपति वपति॒ हिर॑ण्य॒म् दक्षि॑णा । \newline
66. हिर॑ण्य॒म् दक्षि॑णा॒ दक्षि॑णा॒ हिर॑ण्यꣳ॒॒ हिर॑ण्य॒म् दक्षि॑णा वारु॒णं ॅवा॑रु॒णम् दक्षि॑णा॒ हिर॑ण्यꣳ॒॒ हिर॑ण्य॒म् दक्षि॑णा वारु॒णम् । \newline
67. दक्षि॑णा वारु॒णं ॅवा॑रु॒णम् दक्षि॑णा॒ दक्षि॑णा वारु॒णं ॅय॑व॒मयं॑ ॅयव॒मयं॑ ॅवारु॒णम् दक्षि॑णा॒ दक्षि॑णा वारु॒णं ॅय॑व॒मय᳚म् । \newline
68. वा॒रु॒णं ॅय॑व॒मयं॑ ॅयव॒मयं॑ ॅवारु॒णं ॅवा॑रु॒णं ॅय॑व॒मय॑म् च॒रुम् च॒रुं ॅय॑व॒मयं॑ ॅवारु॒णं ॅवा॑रु॒णं ॅय॑व॒मय॑म् च॒रुम् । \newline
69. य॒व॒मय॑म् च॒रुम् च॒रुं ॅय॑व॒मयं॑ ॅयव॒मय॑म् च॒रु मश्वो ऽश्व॑ श्च॒रुं ॅय॑व॒मयं॑ ॅयव॒मय॑म् च॒रु मश्वः॑ । \newline
70. य॒व॒मय॒मिति॑ यव - मय᳚म् । \newline
71. च॒रु मश्वो ऽश्व॑ श्च॒रुम् च॒रु मश्वो॒ दक्षि॑णा॒ दक्षि॒णा ऽश्व॑ श्च॒रुम् च॒रु मश्वो॒ दक्षि॑णा । \newline
72. अश्वो॒ दक्षि॑णा॒ दक्षि॒णा ऽश्वो ऽश्वो॒ दक्षि॑णा । \newline
73. दक्षि॒णेति॒ दक्षि॑णा । \newline
\pagebreak
\markright{ TS 1.8.9.1  \hfill https://www.vedavms.in \hfill}
\addcontentsline{toc}{section}{ TS 1.8.9.1 }
\section*{ TS 1.8.9.1 }

\textbf{TS 1.8.9.1 } \newline
\textbf{Samhita Paata} \newline

बा॒र्॒.ह॒स्प॒त्यं च॒रुं निर्व॑पति ब्र॒ह्मणो॑ गृ॒हे शि॑तिपृ॒ष्ठो दक्षि॑णै॒न्द्र-मेका॑दशकपालꣳ राज॒न्य॑स्य गृ॒ह ऋ॑ष॒भो दक्षि॑णाऽऽदि॒त्यं च॒रुं महि॑ष्यै गृ॒हे धे॒नुर् दक्षि॑णा नैर्.ऋ॒तं च॒रुं प॑रिवृ॒क्त्यै॑ गृ॒हे कृ॒ष्णानां᳚ ॅव्रीही॒णां न॒खनि॑र्भिन्नं कृ॒ष्णा कू॒टा दक्षि॑णा ऽऽग्ने॒यम॒ष्टाक॑पालꣳ सेना॒न्यो॑ गृ॒हे हिर॑ण्यं॒ दक्षि॑णा वारु॒णं दश॑कपालꣳ सू॒तस्य॑ गृ॒हे म॒हानि॑रष्टो॒ दक्षि॑णा मारु॒तꣳ स॒प्तक॑पालं ग्राम॒ण्यो॑ गृ॒हे पृश्ञ॑0079;॒र् दक्षि॑णा सावि॒त्रं द्वाद॑शकपालं - [ ] \newline

\textbf{Pada Paata} \newline

बा॒र्.॒ह॒स्प॒त्यम् । च॒रुम् । निरिति॑ । व॒प॒ति॒ । ब्र॒ह्मणः॑ । गृ॒हे । शि॒ति॒पृ॒ष्ठ इति॑ शिति - पृ॒ष्ठः । दक्षि॑णा । ऐ॒न्द्रम् । एका॑दशकपाल॒मित्येका॑दश-क॒पा॒ल॒म् । रा॒ज॒न्य॑स्य । गृ॒हे । ऋ॒ष॒भः । दक्षि॑णा । आ॒दि॒त्यम् । च॒रुम् । महि॑ष्यै । गृ॒हे । धे॒नुः । दक्षि॑णा । नै॒र॒.ऋ॒तमिति॑ नैः-ऋ॒तम् । च॒रुम् । प॒रि॒वृ॒क्त्या॑ इति॑ परि-वृ॒क्त्यै᳚ । गृ॒हे । कृ॒ष्णाना᳚म् । व्री॒ही॒णाम् । न॒खनि॑र्भिन्न॒मिति॑ न॒ख-नि॒र्भि॒न्न॒म् । कृ॒ष्णा । कू॒टा । दक्षि॑णा । आ॒ग्ने॒यम् । अ॒ष्टाक॑पाल॒मित्य॒ष्टा - क॒पा॒ल॒म् । से॒ना॒न्य॑ इति॑ सेना - न्यः॑ । गृ॒हे । हिर॑ण्यम् । दक्षि॑णा । वा॒रु॒णम् । दश॑कपाल॒मिति॒ दश॑-क॒पा॒ल॒म् । सू॒तस्य॑ । गृ॒हे । म॒हानि॑रष्ट॒ इति॑ म॒हा-नि॒र॒ष्टः॒ । दक्षि॑णा । मा॒रु॒तम् । स॒प्तक॑पाल॒मिति॑ स॒प्त - क॒पा॒ल॒म् । ग्रा॒म॒ण्य॑ इति॑ ग्राम - न्यः॑ । गृ॒हे । पृश्निः॑ । दक्षि॑णा । सा॒वि॒त्रम् । द्वाद॑शकपाल॒मिति॒ द्वाद॑श - क॒पा॒ल॒म् ।  \newline



\textbf{Jatai Paata} \newline

1. बा॒र्॒.ह॒स्प॒त्यम् च॒रुम् च॒रुम् बा॑र्.हस्प॒त्यम् बा॑र्.हस्प॒त्यम् च॒रुम् । \newline
2. च॒रुम् निर् णिश्च॒रुम् च॒रुम् निः । \newline
3. निर् व॑पति वपति॒ निर् णिर् व॑पति । \newline
4. व॒प॒ति॒ ब्र॒ह्मणो᳚ ब्र॒ह्मणो॑ वपति वपति ब्र॒ह्मणः॑ । \newline
5. ब्र॒ह्मणो॑ गृ॒हे गृ॒हे ब्र॒ह्मणो᳚ ब्र॒ह्मणो॑ गृ॒हे । \newline
6. गृ॒हे शि॑तिपृ॒ष्ठः शि॑तिपृ॒ष्ठो गृ॒हे गृ॒हे शि॑तिपृ॒ष्ठः । \newline
7. शि॒ति॒पृ॒ष्ठो दक्षि॑णा॒ दक्षि॑णा शितिपृ॒ष्ठः शि॑तिपृ॒ष्ठो दक्षि॑णा । \newline
8. शि॒ति॒पृ॒ष्ठ इति॑ शिति - पृ॒ष्ठः । \newline
9. दक्षि॑णै॒न्द्र मै॒न्द्रम् दक्षि॑णा॒ दक्षि॑णै॒न्द्रम् । \newline
10. ऐ॒न्द्र मेका॑दशकपाल॒ मेका॑दशकपाल मै॒न्द्र मै॒न्द्र मेका॑दशकपालम् । \newline
11. एका॑दशकपालꣳ राज॒न्य॑स्य राज॒न्य॑स्यैका॑दशकपाल॒ मेका॑दशकपालꣳ राज॒न्य॑स्य । \newline
12. एका॑दशकपाल॒मित्येका॑दश - क॒पा॒ल॒म् । \newline
13. रा॒ज॒न्य॑स्य गृ॒हे गृ॒हे रा॑ज॒न्य॑स्य राज॒न्य॑स्य गृ॒हे । \newline
14. गृ॒ह ऋ॑ष॒भ ऋ॑ष॒भो गृ॒हे गृ॒ह ऋ॑ष॒भः । \newline
15. ऋ॒ष॒भो दक्षि॑णा॒ दक्षि॑णर्.ष॒भ ऋ॑ष॒भो दक्षि॑णा । \newline
16. दक्षि॑णा ऽऽदि॒त्य मा॑दि॒त्यम् दक्षि॑णा॒ दक्षि॑णा ऽऽदि॒त्यम् । \newline
17. आ॒दि॒त्यम् च॒रुम् च॒रु मा॑दि॒त्य मा॑दि॒त्यम् च॒रुम् । \newline
18. च॒रुम् महि॑ष्यै॒ महि॑ष्यै च॒रुम् च॒रुम् महि॑ष्यै । \newline
19. महि॑ष्यै गृ॒हे गृ॒हे महि॑ष्यै॒ महि॑ष्यै गृ॒हे । \newline
20. गृ॒हे धे॒नुर् धे॒नुर् गृ॒हे गृ॒हे धे॒नुः । \newline
21. धे॒नुर् दक्षि॑णा॒ दक्षि॑णा धे॒नुर् धे॒नुर् दक्षि॑णा । \newline
22. दक्षि॑णा नैर्.ऋ॒तम् नैर्॑.ऋ॒तम् दक्षि॑णा॒ दक्षि॑णा नैर्.ऋ॒तम् । \newline
23. नै॒र्॒.ऋ॒तम् च॒रुम् च॒रुम् नैर्॑.ऋ॒तम् नैर्॑.ऋ॒तम् च॒रुम् । \newline
24. नै॒र्॒.ऋ॒तमिति॑ नैः - ऋ॒तम् । \newline
25. च॒रुम् प॑रिवृ॒क्त्यै॑ परिवृ॒क्त्यै॑ च॒रुम् च॒रुम् प॑रिवृ॒क्त्यै᳚ । \newline
26. प॒रि॒वृ॒क्त्यै॑ गृ॒हे गृ॒हे प॑रिवृ॒क्त्यै॑ परिवृ॒क्त्यै॑ गृ॒हे । \newline
27. प॒रि॒वृ॒क्त्या॑ इति॑ परि - वृ॒क्त्यै᳚ । \newline
28. गृ॒हे कृ॒ष्णाना᳚म् कृ॒ष्णाना᳚म् गृ॒हे गृ॒हे कृ॒ष्णाना᳚म् । \newline
29. कृ॒ष्णानां᳚ ॅव्रीही॒णां ॅव्री॑ही॒णाम् कृ॒ष्णाना᳚म् कृ॒ष्णानां᳚ ॅव्रीही॒णाम् । \newline
30. व्री॒ही॒णाम् न॒खनि॑र्भिन्नम् न॒खनि॑र्भिन्नं ॅव्रीही॒णां ॅव्री॑ही॒णाम् न॒खनि॑र्भिन्नम् । \newline
31. न॒खनि॑र्भिन्नम् कृ॒ष्णा कृ॒ष्णा न॒खनि॑र्भिन्नम् न॒खनि॑र्भिन्नम् कृ॒ष्णा । \newline
32. न॒खनि॑र्भिन्न॒मिति॑ न॒ख - नि॒र्भि॒न्न॒म् । \newline
33. कृ॒ष्णा कू॒टा कू॒टा कृ॒ष्णा कृ॒ष्णा कू॒टा । \newline
34. कू॒टा दक्षि॑णा॒ दक्षि॑णा कू॒टा कू॒टा दक्षि॑णा । \newline
35. दक्षि॑णा ऽऽग्ने॒य मा᳚ग्ने॒यम् दक्षि॑णा॒ दक्षि॑णा ऽऽग्ने॒यम् । \newline
36. आ॒ग्ने॒य म॒ष्टाक॑पाल म॒ष्टाक॑पाल माग्ने॒य मा᳚ग्ने॒य म॒ष्टाक॑पालम् । \newline
37. अ॒ष्टाक॑पालꣳ सेना॒न्यः॑ सेना॒न्यो᳚ ऽष्टाक॑पाल म॒ष्टाक॑पालꣳ सेना॒न्यः॑ । \newline
38. अ॒ष्टाक॑पाल॒मित्य॒ष्टा - क॒पा॒ल॒म् । \newline
39. से॒ना॒न्यो॑ गृ॒हे गृ॒हे से॑ना॒न्यः॑ सेना॒न्यो॑ गृ॒हे । \newline
40. से॒ना॒न्य॑ इति॑ सेना - न्यः॑ । \newline
41. गृ॒हे हिर॑ण्यꣳ॒॒ हिर॑ण्यम् गृ॒हे गृ॒हे हिर॑ण्यम् । \newline
42. हिर॑ण्य॒म् दक्षि॑णा॒ दक्षि॑णा॒ हिर॑ण्यꣳ॒॒ हिर॑ण्य॒म् दक्षि॑णा । \newline
43. दक्षि॑णा वारु॒णं ॅवा॑रु॒णम् दक्षि॑णा॒ दक्षि॑णा वारु॒णम् । \newline
44. वा॒रु॒णम् दश॑कपाल॒म् दश॑कपालं ॅवारु॒णं ॅवा॑रु॒णम् दश॑कपालम् । \newline
45. दश॑कपालꣳ सू॒तस्य॑ सू॒तस्य॒ दश॑कपाल॒म् दश॑कपालꣳ सू॒तस्य॑ । \newline
46. दश॑कपाल॒मिति॒ दश॑ - क॒पा॒ल॒म् । \newline
47. सू॒तस्य॑ गृ॒हे गृ॒हे सू॒तस्य॑ सू॒तस्य॑ गृ॒हे । \newline
48. गृ॒हे म॒हानि॑रष्टो म॒हानि॑रष्टो गृ॒हे गृ॒हे म॒हानि॑रष्टः । \newline
49. म॒हानि॑रष्टो॒ दक्षि॑णा॒ दक्षि॑णा म॒हानि॑रष्टो म॒हानि॑रष्टो॒ दक्षि॑णा । \newline
50. म॒हानि॑रष्ट॒ इति॑ म॒हा - नि॒र॒ष्टः॒ । \newline
51. दक्षि॑णा मारु॒तम् मा॑रु॒तम् दक्षि॑णा॒ दक्षि॑णा मारु॒तम् । \newline
52. मा॒रु॒तꣳ स॒प्तक॑पालꣳ स॒प्तक॑पालम् मारु॒तम् मा॑रु॒तꣳ स॒प्तक॑पालम् । \newline
53. स॒प्तक॑पालम् ग्राम॒ण्यो᳚ ग्राम॒ण्यः॑ स॒प्तक॑पालꣳ स॒प्तक॑पालम् ग्राम॒ण्यः॑ । \newline
54. स॒प्तक॑पाल॒मिति॑ स॒प्त - क॒पा॒ल॒म् । \newline
55. ग्रा॒म॒ण्यो॑ गृ॒हे गृ॒हे ग्रा॑म॒ण्यो᳚ ग्राम॒ण्यो॑ गृ॒हे । \newline
56. ग्रा॒म॒ण्य॑ इति॑ ग्राम - न्यः॑ । \newline
57. गृ॒हे पृश्ञिः॒ पृश्ञि॑र् गृ॒हे गृ॒हे पृश्ञिः॑ । \newline
58. पृश्ञि॒र् दक्षि॑णा॒ दक्षि॑णा॒ पृश्ञिः॒ पृश्ञि॒र् दक्षि॑णा । \newline
59. दक्षि॑णा सावि॒त्रꣳ सा॑वि॒त्रम् दक्षि॑णा॒ दक्षि॑णा सावि॒त्रम् । \newline
60. सा॒वि॒त्रम् द्वाद॑शकपाल॒म् द्वाद॑शकपालꣳ सावि॒त्रꣳ सा॑वि॒त्रम् द्वाद॑शकपालम् । \newline
61. द्वाद॑शकपालम् क्ष॒त्तुः क्ष॒त्तुर् द्वाद॑शकपाल॒म् द्वाद॑शकपालम् क्ष॒त्तुः । \newline
62. द्वाद॑शकपाल॒मिति॒ द्वाद॑श - क॒पा॒ल॒म् । \newline

\textbf{Ghana Paata } \newline

1. बा॒र्॒.ह॒स्प॒त्यम् च॒रुम् च॒रुम् बा॑र्.हस्प॒त्यम् बा॑र्.हस्प॒त्यम् च॒रुन् निर् णिश्च॒रुम् बा॑र्.हस्प॒त्यम् बा॑र्.हस्प॒त्यम् च॒रुन् निः । \newline
2. च॒रुन् निर् णिश्च॒रुम् च॒रुन् निर् व॑पति वपति॒ निश्च॒रुम् च॒रुन् निर् व॑पति । \newline
3. निर् व॑पति वपति॒ निर् णिर् व॑पति ब्र॒ह्मणो᳚ ब्र॒ह्मणो॑ वपति॒ निर् णिर् व॑पति ब्र॒ह्मणः॑ । \newline
4. व॒प॒ति॒ ब्र॒ह्मणो᳚ ब्र॒ह्मणो॑ वपति वपति ब्र॒ह्मणो॑ गृ॒हे गृ॒हे ब्र॒ह्मणो॑ वपति वपति ब्र॒ह्मणो॑ गृ॒हे । \newline
5. ब्र॒ह्मणो॑ गृ॒हे गृ॒हे ब्र॒ह्मणो᳚ ब्र॒ह्मणो॑ गृ॒हे शि॑तिपृ॒ष्ठः शि॑तिपृ॒ष्ठो गृ॒हे ब्र॒ह्मणो᳚ ब्र॒ह्मणो॑ गृ॒हे शि॑तिपृ॒ष्ठः । \newline
6. गृ॒हे शि॑तिपृ॒ष्ठः शि॑तिपृ॒ष्ठो गृ॒हे गृ॒हे शि॑तिपृ॒ष्ठो दक्षि॑णा॒ दक्षि॑णा शितिपृ॒ष्ठो गृ॒हे गृ॒हे शि॑तिपृ॒ष्ठो दक्षि॑णा । \newline
7. शि॒ति॒पृ॒ष्ठो दक्षि॑णा॒ दक्षि॑णा शितिपृ॒ष्ठः शि॑तिपृ॒ष्ठो दक्षि॑णै॒न्द्र मै॒न्द्रम् दक्षि॑णा शितिपृ॒ष्ठः शि॑तिपृ॒ष्ठो दक्षि॑णै॒न्द्रम् । \newline
8. शि॒ति॒पृ॒ष्ठ इति॑ शिति - पृ॒ष्ठः । \newline
9. दक्षि॑णै॒न्द्र मै॒न्द्रम् दक्षि॑णा॒ दक्षि॑णै॒न्द्र मेका॑दशकपाल॒ मेका॑दशकपाल मै॒न्द्रम् दक्षि॑णा॒ दक्षि॑णै॒न्द्र मेका॑दशकपालम् । \newline
10. ऐ॒न्द्र मेका॑दशकपाल॒ मेका॑दशकपाल मै॒न्द्र मै॒न्द्र मेका॑दशकपालꣳ राज॒न्य॑स्य राज॒न्य॑ स्यैका॑दशकपाल मै॒न्द्र मै॒न्द्र मेका॑दशकपालꣳ राज॒न्य॑स्य । \newline
11. एका॑दशकपालꣳ राज॒न्य॑स्य राज॒न्य॑ स्यैका॑दशकपाल॒ मेका॑दशकपालꣳ राज॒न्य॑स्य गृ॒हे गृ॒हे रा॑ज॒न्य॑ स्यैका॑दशकपाल॒ मेका॑दशकपालꣳ राज॒न्य॑स्य गृ॒हे । \newline
12. एका॑दशकपाल॒मित्येका॑दश - क॒पा॒ल॒म् । \newline
13. रा॒ज॒न्य॑स्य गृ॒हे गृ॒हे रा॑ज॒न्य॑स्य राज॒न्य॑स्य गृ॒ह ऋ॑ष॒भ ऋ॑ष॒भो गृ॒हे रा॑ज॒न्य॑स्य राज॒न्य॑स्य गृ॒ह ऋ॑ष॒भः । \newline
14. गृ॒ह ऋ॑ष॒भ ऋ॑ष॒भो गृ॒हे गृ॒ह ऋ॑ष॒भो दक्षि॑णा॒ दक्षि॑णर्.ष॒भो गृ॒हे गृ॒ह ऋ॑ष॒भो दक्षि॑णा । \newline
15. ऋ॒ष॒भो दक्षि॑णा॒ दक्षि॑णर्.ष॒भ ऋ॑ष॒भो दक्षि॑णा ऽऽदि॒त्य मा॑दि॒त्यम् दक्षि॑णर्.ष॒भ ऋ॑ष॒भो दक्षि॑णा ऽऽदि॒त्यम् । \newline
16. दक्षि॑णा ऽऽदि॒त्य मा॑दि॒त्यम् दक्षि॑णा॒ दक्षि॑णा ऽऽदि॒त्यम् च॒रुम् च॒रु मा॑दि॒त्यम् दक्षि॑णा॒ दक्षि॑णा ऽऽदि॒त्यम् च॒रुम् । \newline
17. आ॒दि॒त्यम् च॒रुम् च॒रु मा॑दि॒त्य मा॑दि॒त्यम् च॒रुम् महि॑ष्यै॒ महि॑ष्यै च॒रु मा॑दि॒त्य मा॑दि॒त्यम् च॒रुम् महि॑ष्यै । \newline
18. च॒रुम् महि॑ष्यै॒ महि॑ष्यै च॒रुम् च॒रुम् महि॑ष्यै गृ॒हे गृ॒हे महि॑ष्यै च॒रुम् च॒रुम् महि॑ष्यै गृ॒हे । \newline
19. महि॑ष्यै गृ॒हे गृ॒हे महि॑ष्यै॒ महि॑ष्यै गृ॒हे धे॒नुर् धे॒नुर् गृ॒हे महि॑ष्यै॒ महि॑ष्यै गृ॒हे धे॒नुः । \newline
20. गृ॒हे धे॒नुर् धे॒नुर् गृ॒हे गृ॒हे धे॒नुर् दक्षि॑णा॒ दक्षि॑णा धे॒नुर् गृ॒हे गृ॒हे धे॒नुर् दक्षि॑णा । \newline
21. धे॒नुर् दक्षि॑णा॒ दक्षि॑णा धे॒नुर् धे॒नुर् दक्षि॑णा नैर्.ऋ॒तन् नैर्॑.ऋ॒तम् दक्षि॑णा धे॒नुर् धे॒नुर् दक्षि॑णा नैर्.ऋ॒तम् । \newline
22. दक्षि॑णा नैर्.ऋ॒तन् नैर्॑.ऋ॒तम् दक्षि॑णा॒ दक्षि॑णा नैर्.ऋ॒तम् च॒रुम् च॒रुम् नैर्॑.ऋ॒तम् दक्षि॑णा॒ दक्षि॑णा नैर्.ऋ॒तम् च॒रुम् । \newline
23. नै॒र्॒.ऋ॒तम् च॒रुम् च॒रुम् नैर्॑.ऋ॒तम् नैर्॑.ऋ॒तम् च॒रुम् प॑रिवृ॒क्त्यै॑ परिवृ॒क्त्यै॑ च॒रुम् नैर्॑.ऋ॒तम् नैर्॑.ऋ॒तम् च॒रुम् प॑रिवृ॒क्त्यै᳚ । \newline
24. नै॒र्॒.ऋ॒तमिति॑ नैः - ऋ॒तम् । \newline
25. च॒रुम् प॑रिवृ॒क्त्यै॑ परिवृ॒क्त्यै॑ च॒रुम् च॒रुम् प॑रिवृ॒क्त्यै॑ गृ॒हे गृ॒हे प॑रिवृ॒क्त्यै॑ च॒रुम् च॒रुम् प॑रिवृ॒क्त्यै॑ गृ॒हे । \newline
26. प॒रि॒वृ॒क्त्यै॑ गृ॒हे गृ॒हे प॑रिवृ॒क्त्यै॑ परिवृ॒क्त्यै॑ गृ॒हे कृ॒ष्णाना᳚म् कृ॒ष्णाना᳚म् गृ॒हे प॑रिवृ॒क्त्यै॑ परिवृ॒क्त्यै॑ गृ॒हे कृ॒ष्णाना᳚म् । \newline
27. प॒रि॒वृ॒क्त्या॑ इति॑ परि - वृ॒क्त्यै᳚ । \newline
28. गृ॒हे कृ॒ष्णाना᳚म् कृ॒ष्णाना᳚म् गृ॒हे गृ॒हे कृ॒ष्णानां᳚ ॅव्रीही॒णां ॅव्री॑ही॒णाम् कृ॒ष्णाना᳚म् गृ॒हे गृ॒हे कृ॒ष्णानां᳚ ॅव्रीही॒णाम् । \newline
29. कृ॒ष्णानां᳚ ॅव्रीही॒णां ॅव्री॑ही॒णाम् कृ॒ष्णाना᳚म् कृ॒ष्णानां᳚ ॅव्रीही॒णान् न॒खनि॑र्भिन्नन् न॒खनि॑र्भिन्नं ॅव्रीही॒णाम् कृ॒ष्णाना᳚म् कृ॒ष्णानां᳚ ॅव्रीही॒णान् न॒खनि॑र्भिन्नम् । \newline
30. व्री॒ही॒णान् न॒खनि॑र्भिन्नन् न॒खनि॑र्भिन्नं ॅव्रीही॒णां ॅव्री॑ही॒णान् न॒खनि॑र्भिन्नम् कृ॒ष्णा कृ॒ष्णा न॒खनि॑र्भिन्नं ॅव्रीही॒णां ॅव्री॑ही॒णान् न॒खनि॑र्भिन्नम् कृ॒ष्णा । \newline
31. न॒खनि॑र्भिन्नम् कृ॒ष्णा कृ॒ष्णा न॒खनि॑र्भिन्नन् न॒खनि॑र्भिन्नम् कृ॒ष्णा कू॒टा कू॒टा कृ॒ष्णा न॒खनि॑र्भिन्नन् न॒खनि॑र्भिन्नम् कृ॒ष्णा कू॒टा । \newline
32. न॒खनि॑र्भिन्न॒मिति॑ न॒ख - नि॒र्भि॒न्न॒म् । \newline
33. कृ॒ष्णा कू॒टा कू॒टा कृ॒ष्णा कृ॒ष्णा कू॒टा दक्षि॑णा॒ दक्षि॑णा कू॒टा कृ॒ष्णा कृ॒ष्णा कू॒टा दक्षि॑णा । \newline
34. कू॒टा दक्षि॑णा॒ दक्षि॑णा कू॒टा कू॒टा दक्षि॑णा ऽऽग्ने॒य मा᳚ग्ने॒यम् दक्षि॑णा कू॒टा कू॒टा दक्षि॑णा ऽऽग्ने॒यम् । \newline
35. दक्षि॑णा ऽऽग्ने॒य मा᳚ग्ने॒यम् दक्षि॑णा॒ दक्षि॑णा ऽऽग्ने॒य म॒ष्टाक॑पाल म॒ष्टाक॑पाल माग्ने॒यम् दक्षि॑णा॒ दक्षि॑णा ऽऽग्ने॒य म॒ष्टाक॑पालम् । \newline
36. आ॒ग्ने॒य म॒ष्टाक॑पाल म॒ष्टाक॑पाल माग्ने॒य मा᳚ग्ने॒य म॒ष्टाक॑पालꣳ सेना॒न्यः॑ सेना॒न्यो᳚ ऽष्टाक॑पाल माग्ने॒य मा᳚ग्ने॒य म॒ष्टाक॑पालꣳ सेना॒न्यः॑ । \newline
37. अ॒ष्टाक॑पालꣳ सेना॒न्यः॑ सेना॒न्यो᳚ ऽष्टाक॑पाल म॒ष्टाक॑पालꣳ सेना॒न्यो॑ गृ॒हे गृ॒हे से॑ना॒न्यो᳚ ऽष्टाक॑पाल म॒ष्टाक॑पालꣳ सेना॒न्यो॑ गृ॒हे । \newline
38. अ॒ष्टाक॑पाल॒मित्य॒ष्टा - क॒पा॒ल॒म् । \newline
39. से॒ना॒न्यो॑ गृ॒हे गृ॒हे से॑ना॒न्यः॑ सेना॒न्यो॑ गृ॒हे हिर॑ण्यꣳ॒॒ हिर॑ण्यम् गृ॒हे से॑ना॒न्यः॑ सेना॒न्यो॑ गृ॒हे हिर॑ण्यम् । \newline
40. से॒ना॒न्य॑ इति॑ सेना - न्यः॑ । \newline
41. गृ॒हे हिर॑ण्यꣳ॒॒ हिर॑ण्यम् गृ॒हे गृ॒हे हिर॑ण्य॒म् दक्षि॑णा॒ दक्षि॑णा॒ हिर॑ण्यम् गृ॒हे गृ॒हे हिर॑ण्य॒म् दक्षि॑णा । \newline
42. हिर॑ण्य॒म् दक्षि॑णा॒ दक्षि॑णा॒ हिर॑ण्यꣳ॒॒ हिर॑ण्य॒म् दक्षि॑णा वारु॒णं ॅवा॑रु॒णम् दक्षि॑णा॒ हिर॑ण्यꣳ॒॒ हिर॑ण्य॒म् दक्षि॑णा वारु॒णम् । \newline
43. दक्षि॑णा वारु॒णं ॅवा॑रु॒णम् दक्षि॑णा॒ दक्षि॑णा वारु॒णम् दश॑कपाल॒म् दश॑कपालं ॅवारु॒णम् दक्षि॑णा॒ दक्षि॑णा वारु॒णम् दश॑कपालम् । \newline
44. वा॒रु॒णम् दश॑कपाल॒म् दश॑कपालं ॅवारु॒णं ॅवा॑रु॒णम् दश॑कपालꣳ सू॒तस्य॑ सू॒तस्य॒ दश॑कपालं ॅवारु॒णं ॅवा॑रु॒णम् दश॑कपालꣳ सू॒तस्य॑ । \newline
45. दश॑कपालꣳ सू॒तस्य॑ सू॒तस्य॒ दश॑कपाल॒म् दश॑कपालꣳ सू॒तस्य॑ गृ॒हे गृ॒हे सू॒तस्य॒ दश॑कपाल॒म् दश॑कपालꣳ सू॒तस्य॑ गृ॒हे । \newline
46. दश॑कपाल॒मिति॒ दश॑ - क॒पा॒ल॒म् । \newline
47. सू॒तस्य॑ गृ॒हे गृ॒हे सू॒तस्य॑ सू॒तस्य॑ गृ॒हे म॒हानि॑रष्टो म॒हानि॑रष्टो गृ॒हे सू॒तस्य॑ सू॒तस्य॑ गृ॒हे म॒हानि॑रष्टः । \newline
48. गृ॒हे म॒हानि॑रष्टो म॒हानि॑रष्टो गृ॒हे गृ॒हे म॒हानि॑रष्टो॒ दक्षि॑णा॒ दक्षि॑णा म॒हानि॑रष्टो गृ॒हे गृ॒हे म॒हानि॑रष्टो॒ दक्षि॑णा । \newline
49. म॒हानि॑रष्टो॒ दक्षि॑णा॒ दक्षि॑णा म॒हानि॑रष्टो म॒हानि॑रष्टो॒ दक्षि॑णा मारु॒तम् मा॑रु॒तम् दक्षि॑णा म॒हानि॑रष्टो म॒हानि॑रष्टो॒ दक्षि॑णा मारु॒तम् । \newline
50. म॒हानि॑रष्ट॒ इति॑ म॒हा - नि॒र॒ष्टः॒ । \newline
51. दक्षि॑णा मारु॒तम् मा॑रु॒तम् दक्षि॑णा॒ दक्षि॑णा मारु॒तꣳ स॒प्तक॑पालꣳ स॒प्तक॑पालम् मारु॒तम् दक्षि॑णा॒ दक्षि॑णा मारु॒तꣳ स॒प्तक॑पालम् । \newline
52. मा॒रु॒तꣳ स॒प्तक॑पालꣳ स॒प्तक॑पालम् मारु॒तम् मा॑रु॒तꣳ स॒प्तक॑पालम् ग्राम॒ण्यो᳚ ग्राम॒ण्यः॑ स॒प्तक॑पालम् मारु॒तम् मा॑रु॒तꣳ स॒प्तक॑पालम् ग्राम॒ण्यः॑ । \newline
53. स॒प्तक॑पालम् ग्राम॒ण्यो᳚ ग्राम॒ण्यः॑ स॒प्तक॑पालꣳ स॒प्तक॑पालम् ग्राम॒ण्यो॑ गृ॒हे गृ॒हे ग्रा॑म॒ण्यः॑ स॒प्तक॑पालꣳ स॒प्तक॑पालम् ग्राम॒ण्यो॑ गृ॒हे । \newline
54. स॒प्तक॑पाल॒मिति॑ स॒प्त - क॒पा॒ल॒म् । \newline
55. ग्रा॒म॒ण्यो॑ गृ॒हे गृ॒हे ग्रा॑म॒ण्यो᳚ ग्राम॒ण्यो॑ गृ॒हे पृश्ञिः॒ पृश्ञि॑र् गृ॒हे ग्रा॑म॒ण्यो᳚ ग्राम॒ण्यो॑ गृ॒हे पृश्ञिः॑ । \newline
56. ग्रा॒म॒ण्य॑ इति॑ ग्राम - न्यः॑ । \newline
57. गृ॒हे पृश्ञिः॒ पृश्ञि॑र् गृ॒हे गृ॒हे पृश्ञि॒र् दक्षि॑णा॒ दक्षि॑णा॒ पृश्ञि॑र् गृ॒हे गृ॒हे पृश्ञि॒र् दक्षि॑णा । \newline
58. पृश्ञि॒र् दक्षि॑णा॒ दक्षि॑णा॒ पृश्ञिः॒ पृश्ञि॒र् दक्षि॑णा सावि॒त्रꣳ सा॑वि॒त्रम् दक्षि॑णा॒ पृश्ञिः॒ पृश्ञि॒र् दक्षि॑णा सावि॒त्रम् । \newline
59. दक्षि॑णा सावि॒त्रꣳ सा॑वि॒त्रम् दक्षि॑णा॒ दक्षि॑णा सावि॒त्रम् द्वाद॑शकपाल॒म् द्वाद॑शकपालꣳ सावि॒त्रम् दक्षि॑णा॒ दक्षि॑णा सावि॒त्रम् द्वाद॑शकपालम् । \newline
60. सा॒वि॒त्रम् द्वाद॑शकपाल॒म् द्वाद॑शकपालꣳ सावि॒त्रꣳ सा॑वि॒त्रम् द्वाद॑शकपालम् क्ष॒त्तुः क्ष॒त्तुर् द्वाद॑शकपालꣳ सावि॒त्रꣳ सा॑वि॒त्रम् द्वाद॑शकपालम् क्ष॒त्तुः । \newline
61. द्वाद॑शकपालम् क्ष॒त्तुः क्ष॒त्तुर् द्वाद॑शकपाल॒म् द्वाद॑शकपालम् क्ष॒त्तुर् गृ॒हे गृ॒हे क्ष॒त्तुर् द्वाद॑शकपाल॒म् द्वाद॑शकपालम् क्ष॒त्तुर् गृ॒हे । \newline
62. द्वाद॑शकपाल॒मिति॒ द्वाद॑श - क॒पा॒ल॒म् । \newline
\pagebreak
\markright{ TS 1.8.9.2  \hfill https://www.vedavms.in \hfill}
\addcontentsline{toc}{section}{ TS 1.8.9.2 }
\section*{ TS 1.8.9.2 }

\textbf{TS 1.8.9.2 } \newline
\textbf{Samhita Paata} \newline

क्ष॒त्तुर् गृ॒ह उ॑पद्ध्व॒स्तो दक्षि॑णाऽऽश्वि॒नं द्वि॑कपा॒लꣳ स॑ङ्ग्रही॒तुर् गृ॒हे स॑वा॒त्यौ॑ दक्षि॑णा पौ॒ष्णं च॒रुं भा॑गदु॒घस्य॑ गृ॒हे श्या॒मो दक्षि॑णा रौ॒द्रं गा॑वीधु॒कं च॒रुम॑क्षावा॒पस्य॑ गृ॒हे श॒बल॒ उद्वा॑रो॒ दक्षि॒णेन्द्रा॑य सु॒त्रांणे॑ पुरो॒डाश॒मेका॑दशकपालं॒ प्रति॒ निर्व॑प॒तीन्द्रा॑याꣳहो॒मुचे॒ ऽयं नो॒ राजा॑ वृत्र॒हा राजा॑ भू॒त्वा वृ॒त्रं ॅव॑द्ध्यान् मैत्राबार्.हस्प॒त्यं भ॑वति श्वे॒तायै᳚ श्वे॒तव॑थ्सायै दु॒ग्धे स्व॑यंमू॒र्ते स्व॑यंमथि॒त आज्य॒ आश्व॑त्थे॒ - [ ] \newline

\textbf{Pada Paata} \newline

क्ष॒त्तुः । गृ॒हे । उ॒प॒द्ध्व॒स्त इत्यु॑प - ध्व॒स्तः । दक्षि॑णा । आ॒श्वि॒नम् । द्वि॒क॒पा॒लमिति॑ द्वि - क॒पा॒लम् । स॒ङ्ग्र॒ही॒तुरिति॑ सं-ग्र॒ही॒तुः । गृ॒हे । स॒वा॒त्या॑विति॑ स - वा॒त्यौ᳚ । दक्षि॑णा । पौ॒ष्णम् । च॒रुम् । भा॒ग॒दु॒घस्येति॑ भाग - दु॒घस्य॑ । गृ॒हे । श्या॒मः । दक्षि॑णा । रौ॒द्रम् । गा॒वी॒धु॒कम् । च॒रुम् । अ॒क्षा॒वा॒पस्येत्य॑क्ष - आ॒वा॒पस्य॑ । गृ॒हे । श॒बलः॑ । उद्वा॑र॒ इत्युत् - वा॒रः॒ । दक्षि॑णा । इन्द्रा॑य । सु॒त्रांण॒ इति॑ सु - त्रांणे᳚ । पु॒रो॒डाश᳚म् । एका॑दशकपाल॒मित्येका॑दश - क॒पा॒ल॒म् । प्रति॑ । निरिति॑ । व॒प॒ति॒ । इन्द्रा॑य । अꣳ॒॒हो॒मुच॒ इत्यꣳ॑हः-मुचे᳚ । अ॒यम् । नः॒ । राजा᳚ । वृ॒त्र॒हेति॑ वृत्र-हा । राजा᳚ । भू॒त्वा । वृ॒त्रम् । व॒द्ध्या॒त् । मै॒त्रा॒बा॒र्.॒ह॒स्प॒त्यमिति॑ मैत्रा-बा॒र्.॒ह॒स्प॒त्यम् । भ॒व॒ति॒ । श्वे॒तायै᳚ । श्वे॒तव॑थ्साया॒ इति॑ श्वे॒त-व॒थ्सा॒यै॒ । दु॒ग्धे । स्व॒य॒मूं॒र्त इति॑ स्वयम्-मू॒र्ते । स्व॒य॒मं॒थि॒त इति॑ स्वयं - म॒थि॒ते । आज्ये᳚ । आश्व॑त्थे ।  \newline



\textbf{Jatai Paata} \newline

1. क्ष॒त्तुर् गृ॒हे गृ॒हे क्ष॒त्तुः क्ष॒त्तुर् गृ॒हे । \newline
2. गृ॒ह उ॑पद्ध्व॒स्त उ॑पद्ध्व॒स्तो गृ॒हे गृ॒ह उ॑पद्ध्व॒स्तः । \newline
3. उ॒प॒द्ध्व॒स्तो दक्षि॑णा॒ दक्षि॑णोपद्ध्व॒स्त उ॑पद्ध्व॒स्तो दक्षि॑णा । \newline
4. उ॒प॒द्ध्व॒स्त इत्यु॑प - ध्व॒स्तः । \newline
5. दक्षि॑णा ऽऽश्वि॒न मा᳚श्वि॒नम् दक्षि॑णा॒ दक्षि॑णा ऽऽश्वि॒नम् । \newline
6. आ॒श्वि॒नम् द्वि॑कपा॒लम् द्वि॑कपा॒ल मा᳚श्वि॒न मा᳚श्वि॒नम् द्वि॑कपा॒लम् । \newline
7. द्वि॒क॒पा॒लꣳ स॑ङ्ग्रही॒तुः स॑ङ्ग्रही॒तुर् द्वि॑कपा॒लम् द्वि॑कपा॒लꣳ स॑ङ्ग्रही॒तुः । \newline
8. द्वि॒क॒पा॒लमिति॑ द्वि - क॒पा॒लम् । \newline
9. स॒ङ्ग्र॒ही॒तुर् गृ॒हे गृ॒हे स॑ङ्ग्रही॒तुः स॑ङ्ग्रही॒तुर् गृ॒हे । \newline
10. स॒ङ्ग्र॒ही॒तुरिति॑ सं - ग्र॒ही॒तुः । \newline
11. गृ॒हे स॑वा॒त्यौ॑ सवा॒त्यौ॑ गृ॒हे गृ॒हे स॑वा॒त्यौ᳚ । \newline
12. स॒वा॒त्यौ॑ दक्षि॑णा॒ दक्षि॑णा सवा॒त्यौ॑ सवा॒त्यौ॑ दक्षि॑णा । \newline
13. स॒वा॒त्या॑विति॑ स - वा॒त्यौ᳚ । \newline
14. दक्षि॑णा पौ॒ष्णम् पौ॒ष्णम् दक्षि॑णा॒ दक्षि॑णा पौ॒ष्णम् । \newline
15. पौ॒ष्णम् च॒रुम् च॒रुम् पौ॒ष्णम् पौ॒ष्णम् च॒रुम् । \newline
16. च॒रुम् भा॑गदु॒घस्य॑ भागदु॒घस्य॑ च॒रुम् च॒रुम् भा॑गदु॒घस्य॑ । \newline
17. भा॒ग॒दु॒घस्य॑ गृ॒हे गृ॒हे भा॑गदु॒घस्य॑ भागदु॒घस्य॑ गृ॒हे । \newline
18. भा॒ग॒दु॒घस्येति॑ भाग - दु॒घस्य॑ । \newline
19. गृ॒हे श्या॒मः श्या॒मो गृ॒हे गृ॒हे श्या॒मः । \newline
20. श्या॒मो दक्षि॑णा॒ दक्षि॑णा श्या॒मः श्या॒मो दक्षि॑णा । \newline
21. दक्षि॑णा रौ॒द्रꣳ रौ॒द्रम् दक्षि॑णा॒ दक्षि॑णा रौ॒द्रम् । \newline
22. रौ॒द्रम् गा॑वीधु॒कम् गा॑वीधु॒कꣳ रौ॒द्रꣳ रौ॒द्रम् गा॑वीधु॒कम् । \newline
23. गा॒वी॒धु॒कम् च॒रुम् च॒रुम् गा॑वीधु॒कम् गा॑वीधु॒कम् च॒रुम् । \newline
24. च॒रु म॑क्षावा॒पस्या᳚ क्षावा॒पस्य॑ च॒रुम् च॒रु म॑क्षावा॒पस्य॑ । \newline
25. अ॒क्षा॒वा॒पस्य॑ गृ॒हे गृ॒हे᳚ ऽक्षावा॒पस्या᳚ क्षावा॒पस्य॑ गृ॒हे । \newline
26. अ॒क्षा॒वा॒पस्येत्य॑क्ष - आ॒वा॒पस्य॑ । \newline
27. गृ॒हे श॒बलः॑ श॒बलो॑ गृ॒हे गृ॒हे श॒बलः॑ । \newline
28. श॒बल॒ उद्वा॑र॒ उद्वा॑रः श॒बलः॑ श॒बल॒ उद्वा॑रः । \newline
29. उद्वा॑रो॒ दक्षि॑णा॒ दक्षि॒णोद्वा॑र॒ उद्वा॑रो॒ दक्षि॑णा । \newline
30. उद्वा॑र॒ इत्युत् - वा॒रः॒ । \newline
31. दक्षि॒णेन्द्रा॒ये न्द्रा॑य॒ दक्षि॑णा॒ दक्षि॒णेन्द्रा॑य । \newline
32. इन्द्रा॑य सु॒त्रांणे॑ सु॒त्रांण॒ इन्द्रा॒ये न्द्रा॑य सु॒त्रांणे᳚ । \newline
33. सु॒त्रांणे॑ पुरो॒डाश॑म् पुरो॒डाशꣳ॑ सु॒त्रांणे॑ सु॒त्रांणे॑ पुरो॒डाश᳚म् । \newline
34. सु॒त्रांण॒ इति॑ सु - त्रांणे᳚ । \newline
35. पु॒रो॒डाश॒ मेका॑दशकपाल॒ मेका॑दशकपालम् पुरो॒डाश॑म् पुरो॒डाश॒ मेका॑दशकपालम् । \newline
36. एका॑दशकपाल॒म् प्रति॒ प्रत्येका॑दशकपाल॒ मेका॑दशकपाल॒म् प्रति॑ । \newline
37. एका॑दशकपाल॒मित्येका॑दश - क॒पा॒ल॒म् । \newline
38. प्रति॒ निर् णिष् प्रति॒ प्रति॒ निः । \newline
39. निर् व॑पति वपति॒ निर् णिर् व॑पति । \newline
40. व॒प॒तीन्द्रा॒ये न्द्रा॑य वपति वप॒तीन्द्रा॑य । \newline
41. इन्द्रा॑या ꣳहो॒मुचे ऽꣳ॑हो॒मुच॒ इन्द्रा॒ये न्द्रा॑या ꣳहो॒मुचे᳚ । \newline
42. अꣳ॒॒हो॒मुचे॒ ऽय म॒य मꣳ॑हो॒मुचे ऽꣳ॑हो॒मुचे॒ ऽयम् । \newline
43. अꣳ॒॒हो॒मुच॒ इत्यꣳ॑हः - मुचे᳚ । \newline
44. अ॒यम् नो॑ नो॒ ऽय म॒यम् नः॑ । \newline
45. नो॒ राजा॒ राजा॑ नो नो॒ राजा᳚ । \newline
46. राजा॑ वृत्र॒हा वृ॑त्र॒हा राजा॒ राजा॑ वृत्र॒हा । \newline
47. वृ॒त्र॒हा राजा॒ राजा॑ वृत्र॒हा वृ॑त्र॒हा राजा᳚ । \newline
48. वृ॒त्र॒हेति॑ वृत्र - हा । \newline
49. राजा॑ भू॒त्वा भू॒त्वा राजा॒ राजा॑ भू॒त्वा । \newline
50. भू॒त्वा वृ॒त्रं ॅवृ॒त्रम् भू॒त्वा भू॒त्वा वृ॒त्रम् । \newline
51. वृ॒त्रं ॅव॑द्ध्याद् वद्ध्याद् वृ॒त्रं ॅवृ॒त्रं ॅव॑द्ध्यात् । \newline
52. व॒द्ध्या॒न् मै॒त्रा॒बा॒र्॒.ह॒स्प॒त्यम् मै᳚त्राबार्.हस्प॒त्यं ॅव॑द्ध्याद् वद्ध्यान् मैत्राबार्.हस्प॒त्यम् । \newline
53. मै॒त्रा॒बा॒र्॒.ह॒स्प॒त्यम् भ॑वति भवति मैत्राबार्.हस्प॒त्यम् मै᳚त्राबार्.हस्प॒त्यम् भ॑वति । \newline
54. मै॒त्रा॒बा॒र्॒.ह॒स्प॒त्यमिति॑ मैत्रा - बा॒र्॒.ह॒स्प॒त्यम् । \newline
55. भ॒व॒ति॒ श्वे॒तायै᳚ श्वे॒तायै॑ भवति भवति श्वे॒तायै᳚ । \newline
56. श्वे॒तायै᳚ श्वे॒तव॑थ्सायै श्वे॒तव॑थ्सायै श्वे॒तायै᳚ श्वे॒तायै᳚ श्वे॒तव॑थ्सायै । \newline
57. श्वे॒तव॑थ्सायै दु॒ग्धे दु॒ग्धे श्वे॒तव॑थ्सायै श्वे॒तव॑थ्सायै दु॒ग्धे । \newline
58. श्वे॒तव॑थ्साया॒ इति॑ श्वे॒त - व॒थ्सा॒यै॒ । \newline
59. दु॒ग्धे स्व॑यंमू॒र्ते स्व॑यंमू॒र्ते दु॒ग्धे दु॒ग्धे स्व॑यंमू॒र्ते । \newline
60. स्व॒यं॒मू॒र्ते स्व॑यंमथि॒ते स्व॑यंमथि॒ते स्व॑यंमू॒र्ते स्व॑यंमू॒र्ते स्व॑यंमथि॒ते । \newline
61. स्व॒यं॒मू॒र्त इति॑ स्वयम् - मू॒र्ते । \newline
62. स्व॒यं॒म॒थि॒त आज्य॒ आज्ये᳚ स्वयंमथि॒ते स्व॑यंमथि॒त आज्ये᳚ । \newline
63. स्व॒यं॒म॒थि॒त इति॑ स्वयं - म॒थि॒ते । \newline
64. आज्य॒ आश्व॑त्थ॒ आश्व॑त्थ॒ आज्य॒ आज्य॒ आश्व॑त्थे । \newline
65. आश्व॑त्थे॒ पात्रे॒ पात्र॒ आश्व॑त्थ॒ आश्व॑त्थे॒ पात्रे᳚ । \newline

\textbf{Ghana Paata } \newline

1. क्ष॒त्तुर् गृ॒हे गृ॒हे क्ष॒त्तुः क्ष॒त्तुर् गृ॒ह उ॑पद्ध्व॒स्त उ॑पद्ध्व॒स्तो गृ॒हे क्ष॒त्तुः क्ष॒त्तुर् गृ॒ह उ॑पद्ध्व॒स्तः । \newline
2. गृ॒ह उ॑पद्ध्व॒स्त उ॑पद्ध्व॒स्तो गृ॒हे गृ॒ह उ॑पद्ध्व॒स्तो दक्षि॑णा॒ दक्षि॑णो पद्ध्व॒स्तो गृ॒हे गृ॒ह उ॑पद्ध्व॒स्तो दक्षि॑णा । \newline
3. उ॒प॒द्ध्व॒स्तो दक्षि॑णा॒ दक्षि॑णो पद्ध्व॒स्त उ॑पद्ध्व॒स्तो दक्षि॑णा ऽऽश्वि॒न मा᳚श्वि॒नम् दक्षि॑णो पद्ध्व॒स्त उ॑पद्ध्व॒स्तो दक्षि॑णा ऽऽश्वि॒नम् । \newline
4. उ॒प॒द्ध्व॒स्त इत्यु॑प - ध्व॒स्तः । \newline
5. दक्षि॑णा ऽऽश्वि॒न मा᳚श्वि॒नम् दक्षि॑णा॒ दक्षि॑णा ऽऽश्वि॒नम् द्वि॑कपा॒लम् द्वि॑कपा॒ल मा᳚श्वि॒नम् दक्षि॑णा॒ दक्षि॑णा ऽऽश्वि॒नम् द्वि॑कपा॒लम् । \newline
6. आ॒श्वि॒नम् द्वि॑कपा॒लम् द्वि॑कपा॒ल मा᳚श्वि॒न मा᳚श्वि॒नम् द्वि॑कपा॒लꣳ स॑ङ्ग्रही॒तुः स॑ङ्ग्रही॒तुर् द्वि॑कपा॒ल मा᳚श्वि॒न मा᳚श्वि॒नम् द्वि॑कपा॒लꣳ स॑ङ्ग्रही॒तुः । \newline
7. द्वि॒क॒पा॒लꣳ स॑ङ्ग्रही॒तुः स॑ङ्ग्रही॒तुर् द्वि॑कपा॒लम् द्वि॑कपा॒लꣳ स॑ङ्ग्रही॒तुर् गृ॒हे गृ॒हे स॑ङ्ग्रही॒तुर् द्वि॑कपा॒लम् द्वि॑कपा॒लꣳ स॑ङ्ग्रही॒तुर् गृ॒हे । \newline
8. द्वि॒क॒पा॒लमिति॑ द्वि - क॒पा॒लम् । \newline
9. स॒ङ्ग्र॒ही॒तुर् गृ॒हे गृ॒हे स॑ङ्ग्रही॒तुः स॑ङ्ग्रही॒तुर् गृ॒हे स॑वा॒त्यौ॑ सवा॒त्यौ॑ गृ॒हे स॑ङ्ग्रही॒तुः स॑ङ्ग्रही॒तुर् गृ॒हे स॑वा॒त्यौ᳚ । \newline
10. स॒ङ्ग्र॒ही॒तुरिति॑ सं - ग्र॒ही॒तुः । \newline
11. गृ॒हे स॑वा॒त्यौ॑ सवा॒त्यौ॑ गृ॒हे गृ॒हे स॑वा॒त्यौ॑ दक्षि॑णा॒ दक्षि॑णा सवा॒त्यौ॑ गृ॒हे गृ॒हे स॑वा॒त्यौ॑ दक्षि॑णा । \newline
12. स॒वा॒त्यौ॑ दक्षि॑णा॒ दक्षि॑णा सवा॒त्यौ॑ सवा॒त्यौ॑ दक्षि॑णा पौ॒ष्णम् पौ॒ष्णम् दक्षि॑णा सवा॒त्यौ॑ सवा॒त्यौ॑ दक्षि॑णा पौ॒ष्णम् । \newline
13. स॒वा॒त्या॑विति॑ स - वा॒त्यौ᳚ । \newline
14. दक्षि॑णा पौ॒ष्णम् पौ॒ष्णम् दक्षि॑णा॒ दक्षि॑णा पौ॒ष्णम् च॒रुम् च॒रुम् पौ॒ष्णम् दक्षि॑णा॒ दक्षि॑णा पौ॒ष्णम् च॒रुम् । \newline
15. पौ॒ष्णम् च॒रुम् च॒रुम् पौ॒ष्णम् पौ॒ष्णम् च॒रुम् भा॑गदु॒घस्य॑ भागदु॒घस्य॑ च॒रुम् पौ॒ष्णम् पौ॒ष्णम् च॒रुम् भा॑गदु॒घस्य॑ । \newline
16. च॒रुम् भा॑गदु॒घस्य॑ भागदु॒घस्य॑ च॒रुम् च॒रुम् भा॑गदु॒घस्य॑ गृ॒हे गृ॒हे भा॑गदु॒घस्य॑ च॒रुम् च॒रुम् भा॑गदु॒घस्य॑ गृ॒हे । \newline
17. भा॒ग॒दु॒घस्य॑ गृ॒हे गृ॒हे भा॑गदु॒घस्य॑ भागदु॒घस्य॑ गृ॒हे श्या॒मः श्या॒मो गृ॒हे भा॑गदु॒घस्य॑ भागदु॒घस्य॑ गृ॒हे श्या॒मः । \newline
18. भा॒ग॒दु॒घस्येति॑ भाग - दु॒घस्य॑ । \newline
19. गृ॒हे श्या॒मः श्या॒मो गृ॒हे गृ॒हे श्या॒मो दक्षि॑णा॒ दक्षि॑णा श्या॒मो गृ॒हे गृ॒हे श्या॒मो दक्षि॑णा । \newline
20. श्या॒मो दक्षि॑णा॒ दक्षि॑णा श्या॒मः श्या॒मो दक्षि॑णा रौ॒द्रꣳ रौ॒द्रम् दक्षि॑णा श्या॒मः श्या॒मो दक्षि॑णा रौ॒द्रम् । \newline
21. दक्षि॑णा रौ॒द्रꣳ रौ॒द्रम् दक्षि॑णा॒ दक्षि॑णा रौ॒द्रम् गा॑वीधु॒कम् गा॑वीधु॒कꣳ रौ॒द्रम् दक्षि॑णा॒ दक्षि॑णा रौ॒द्रम् गा॑वीधु॒कम् । \newline
22. रौ॒द्रम् गा॑वीधु॒कम् गा॑वीधु॒कꣳ रौ॒द्रꣳ रौ॒द्रम् गा॑वीधु॒कम् च॒रुम् च॒रुम् गा॑वीधु॒कꣳ रौ॒द्रꣳ रौ॒द्रम् गा॑वीधु॒कम् च॒रुम् । \newline
23. गा॒वी॒धु॒कम् च॒रुम् च॒रुम् गा॑वीधु॒कम् गा॑वीधु॒कम् च॒रु म॑क्षावा॒पस्या᳚ क्षावा॒पस्य॑ च॒रुम् गा॑वीधु॒कम् गा॑वीधु॒कम् च॒रु म॑क्षावा॒पस्य॑ । \newline
24. च॒रु म॑क्षावा॒पस्या᳚ क्षावा॒पस्य॑ च॒रुम् च॒रु म॑क्षावा॒पस्य॑ गृ॒हे गृ॒हे᳚ ऽक्षावा॒पस्य॑ च॒रुम् च॒रु म॑क्षावा॒पस्य॑ गृ॒हे । \newline
25. अ॒क्षा॒वा॒पस्य॑ गृ॒हे गृ॒हे᳚ ऽक्षावा॒पस्या᳚ क्षावा॒पस्य॑ गृ॒हे श॒बलः॑ श॒बलो॑ गृ॒हे᳚ ऽक्षावा॒पस्या᳚क्षावा॒पस्य॑ गृ॒हे श॒बलः॑ । \newline
26. अ॒क्षा॒वा॒पस्येत्य॑क्ष - आ॒वा॒पस्य॑ । \newline
27. गृ॒हे श॒बलः॑ श॒बलो॑ गृ॒हे गृ॒हे श॒बल॒ उद्वा॑र॒ उद्वा॑रः श॒बलो॑ गृ॒हे गृ॒हे श॒बल॒ उद्वा॑रः । \newline
28. श॒बल॒ उद्वा॑र॒ उद्वा॑रः श॒बलः॑ श॒बल॒ उद्वा॑रो॒ दक्षि॑णा॒ दक्षि॒णोद्वा॑रः श॒बलः॑ श॒बल॒ उद्वा॑रो॒ दक्षि॑णा । \newline
29. उद्वा॑रो॒ दक्षि॑णा॒ दक्षि॒णोद्वा॑र॒ उद्वा॑रो॒ दक्षि॒णेन्द्रा॒ये न्द्रा॑य॒ दक्षि॒णोद्वा॑र॒ उद्वा॑रो॒ दक्षि॒णेन्द्रा॑य । \newline
30. उद्वा॑र॒ इत्युत् - वा॒रः॒ । \newline
31. दक्षि॒णेन्द्रा॒ये न्द्रा॑य॒ दक्षि॑णा॒ दक्षि॒णेन्द्रा॑य सु॒त्रांणे॑ सु॒त्रांण॒ इन्द्रा॑य॒ दक्षि॑णा॒ दक्षि॒णेन्द्रा॑य सु॒त्रांणे᳚ । \newline
32. इन्द्रा॑य सु॒त्रांणे॑ सु॒त्रांण॒ इन्द्रा॒ये न्द्रा॑य सु॒त्रांणे॑ पुरो॒डाश॑म् पुरो॒डाशꣳ॑ सु॒त्रांण॒ इन्द्रा॒ये न्द्रा॑य सु॒त्रांणे॑ पुरो॒डाश᳚म् । \newline
33. सु॒त्रांणे॑ पुरो॒डाश॑म् पुरो॒डाशꣳ॑ सु॒त्रांणे॑ सु॒त्रांणे॑ पुरो॒डाश॒ मेका॑दशकपाल॒ मेका॑दशकपालम् पुरो॒डाशꣳ॑ सु॒त्रांणे॑ सु॒त्रांणे॑ पुरो॒डाश॒ मेका॑दशकपालम् । \newline
34. सु॒त्रांण॒ इति॑ सु - त्रांणे᳚ । \newline
35. पु॒रो॒डाश॒ मेका॑दशकपाल॒ मेका॑दशकपालम् पुरो॒डाश॑म् पुरो॒डाश॒ मेका॑दशकपाल॒म् प्रति॒ प्रत्येका॑दशकपालम् पुरो॒डाश॑म् पुरो॒डाश॒ मेका॑दशकपाल॒म् प्रति॑ । \newline
36. एका॑दशकपाल॒म् प्रति॒ प्रत्येका॑दशकपाल॒ मेका॑दशकपाल॒म् प्रति॒ निर् णिष् प्रत्येका॑दशकपाल॒ मेका॑दशकपाल॒म् प्रति॒ निः । \newline
37. एका॑दशकपाल॒मित्येका॑दश - क॒पा॒ल॒म् । \newline
38. प्रति॒ निर् णिष् प्रति॒ प्रति॒ निर् व॑पति वपति॒ निष् प्रति॒ प्रति॒ निर् व॑पति । \newline
39. निर् व॑पति वपति॒ निर् णिर् व॑प॒तीन्द्रा॒ये न्द्रा॑य वपति॒ निर् णिर् व॑प॒तीन्द्रा॑य । \newline
40. व॒प॒तीन्द्रा॒ये न्द्रा॑य वपति वप॒तीन्द्रा॑या ꣳहो॒मुचे ऽꣳ॑हो॒मुच॒ इन्द्रा॑य वपति वप॒तीन्द्रा॑या ꣳहो॒मुचे᳚ । \newline
41. इन्द्रा॑या ꣳहो॒मुचे ऽꣳ॑हो॒मुच॒ इन्द्रा॒ये न्द्रा॑याꣳहो॒मुचे॒ ऽय म॒य मꣳ॑हो॒मुच॒ इन्द्रा॒ये न्द्रा॑याꣳहो॒मुचे॒ ऽयम् । \newline
42. अꣳ॒॒हो॒मुचे॒ ऽय म॒य मꣳ॑हो॒मुचे ऽꣳ॑हो॒मुचे॒ ऽयन्नो॑ नो॒ ऽय मꣳ॑हो॒मुचे ऽꣳ॑हो॒मुचे॒ ऽयन्नः॑ । \newline
43. अꣳ॒॒हो॒मुच॒ इत्यꣳ॑हः - मुचे᳚ । \newline
44. अ॒यन्नो॑ नो॒ ऽय म॒यन्नो॒ राजा॒ राजा॑ नो॒ ऽय म॒यन्नो॒ राजा᳚ । \newline
45. नो॒ राजा॒ राजा॑ नो नो॒ राजा॑ वृत्र॒हा वृ॑त्र॒हा राजा॑ नो नो॒ राजा॑ वृत्र॒हा । \newline
46. राजा॑ वृत्र॒हा वृ॑त्र॒हा राजा॒ राजा॑ वृत्र॒हा राजा॒ राजा॑ वृत्र॒हा राजा॒ राजा॑ वृत्र॒हा राजा᳚ । \newline
47. वृ॒त्र॒हा राजा॒ राजा॑ वृत्र॒हा वृ॑त्र॒हा राजा॑ भू॒त्वा भू॒त्वा राजा॑ वृत्र॒हा वृ॑त्र॒हा राजा॑ भू॒त्वा । \newline
48. वृ॒त्र॒हेति॑ वृत्र - हा । \newline
49. राजा॑ भू॒त्वा भू॒त्वा राजा॒ राजा॑ भू॒त्वा वृ॒त्रं ॅवृ॒त्रम् भू॒त्वा राजा॒ राजा॑ भू॒त्वा वृ॒त्रम् । \newline
50. भू॒त्वा वृ॒त्रं ॅवृ॒त्रम् भू॒त्वा भू॒त्वा वृ॒त्रं ॅव॑द्ध्याद् वद्ध्याद् वृ॒त्रम् भू॒त्वा भू॒त्वा वृ॒त्रं ॅव॑द्ध्यात् । \newline
51. वृ॒त्रं ॅव॑द्ध्याद् वद्ध्याद् वृ॒त्रं ॅवृ॒त्रं ॅव॑द्ध्यान् मैत्राबार्.हस्प॒त्यम् मै᳚त्राबार्.हस्प॒त्यं ॅव॑द्ध्याद् वृ॒त्रं ॅवृ॒त्रं ॅव॑द्ध्यान् मैत्राबार्.हस्प॒त्यम् । \newline
52. व॒द्ध्या॒न् मै॒त्रा॒बा॒र्॒.ह॒स्प॒त्यम् मै᳚त्राबार्.हस्प॒त्यं ॅव॑द्ध्याद् वद्ध्यान् मैत्राबार्.हस्प॒त्यम् भ॑वति भवति मैत्राबार्.हस्प॒त्यं ॅव॑द्ध्याद् वद्ध्यान् मैत्राबार्.हस्प॒त्यम् भ॑वति । \newline
53. मै॒त्रा॒बा॒र्॒.ह॒स्प॒त्यम् भ॑वति भवति मैत्राबार्.हस्प॒त्यम् मै᳚त्राबार्.हस्प॒त्यम् भ॑वति श्वे॒तायै᳚ श्वे॒तायै॑ भवति मैत्राबार्.हस्प॒त्यम् मै᳚त्राबार्.हस्प॒त्यम् भ॑वति श्वे॒तायै᳚ । \newline
54. मै॒त्रा॒बा॒र्॒.ह॒स्प॒त्यमिति॑ मैत्रा - बा॒र्॒.ह॒स्प॒त्यम् । \newline
55. भ॒व॒ति॒ श्वे॒तायै᳚ श्वे॒तायै॑ भवति भवति श्वे॒तायै᳚ श्वे॒तव॑थ्सायै श्वे॒तव॑थ्सायै श्वे॒तायै॑ भवति भवति श्वे॒तायै᳚ श्वे॒तव॑थ्सायै । \newline
56. श्वे॒तायै᳚ श्वे॒तव॑थ्सायै श्वे॒तव॑थ्सायै श्वे॒तायै᳚ श्वे॒तायै᳚ श्वे॒तव॑थ्सायै दु॒ग्धे दु॒ग्धे श्वे॒तव॑थ्सायै श्वे॒तायै᳚ श्वे॒तायै᳚ श्वे॒तव॑थ्सायै दु॒ग्धे । \newline
57. श्वे॒तव॑थ्सायै दु॒ग्धे दु॒ग्धे श्वे॒तव॑थ्सायै श्वे॒तव॑थ्सायै दु॒ग्धे स्व॑यंमू॒र्ते स्व॑यंमू॒र्ते दु॒ग्धे श्वे॒तव॑थ्सायै श्वे॒तव॑थ्सायै दु॒ग्धे स्व॑यंमू॒र्ते । \newline
58. श्वे॒तव॑थ्साया॒ इति॑ श्वे॒त - व॒थ्सा॒यै॒ । \newline
59. दु॒ग्धे स्व॑यंमू॒र्ते स्व॑यंमू॒र्ते दु॒ग्धे दु॒ग्धे स्व॑यंमू॒र्ते स्व॑यंमथि॒ते स्व॑यंमथि॒ते स्व॑यंमू॒र्ते दु॒ग्धे दु॒ग्धे स्व॑यंमू॒र्ते स्व॑यंमथि॒ते । \newline
60. स्व॒यं॒मू॒र्ते स्व॑यंमथि॒ते स्व॑यंमथि॒ते स्व॑यंमू॒र्ते स्व॑यंमू॒र्ते स्व॑यंमथि॒त आज्य॒ आज्ये᳚ स्वयंमथि॒ते स्व॑यंमू॒र्ते स्व॑यंमू॒र्ते स्व॑यंमथि॒त आज्ये᳚ । \newline
61. स्व॒यं॒मू॒र्त इति॑ स्वयम् - मू॒र्ते । \newline
62. स्व॒यं॒म॒थि॒त आज्य॒ आज्ये᳚ स्वयंमथि॒ते स्व॑यंमथि॒त आज्य॒ आश्व॑त्थ॒ आश्व॑त्थ॒ आज्ये᳚ स्वयंमथि॒ते स्व॑यंमथि॒त आज्य॒ आश्व॑त्थे । \newline
63. स्व॒यं॒म॒थि॒त इति॑ स्वयं - म॒थि॒ते । \newline
64. आज्य॒ आश्व॑त्थ॒ आश्व॑त्थ॒ आज्य॒ आज्य॒ आश्व॑त्थे॒ पात्रे॒ पात्र॒ आश्व॑त्थ॒ आज्य॒ आज्य॒ आश्व॑त्थे॒ पात्रे᳚ । \newline
65. आश्व॑त्थे॒ पात्रे॒ पात्र॒ आश्व॑त्थ॒ आश्व॑त्थे॒ पात्रे॒ चतु॑स्स्रक्तौ॒ चतु॑स्स्रक्तौ॒ पात्र॒ आश्व॑त्थ॒ आश्व॑त्थे॒ पात्रे॒ चतु॑स्स्रक्तौ । \newline
\pagebreak
\markright{ TS 1.8.9.3  \hfill https://www.vedavms.in \hfill}
\addcontentsline{toc}{section}{ TS 1.8.9.3 }
\section*{ TS 1.8.9.3 }

\textbf{TS 1.8.9.3 } \newline
\textbf{Samhita Paata} \newline

पात्रे॒ चतुः॑स्रक्तौ स्वयमवप॒न्नायै॒ शाखा॑यै क॒र्णाꣳश्चाक॑र्णाꣳश्च तण्डु॒लान् वि चि॑नुयाद् येक॒र्णाः स प॑यसि बार्.हस्प॒त्यो येऽक॑र्णाः॒ स आज्ये॑ मै॒त्रः स्व॑यंकृ॒ता वेदि॑र् भवति स्वयन्दि॒॒नं ब॒र्॒.हिः स्व॑यंकृ॒त इ॒द्ध्मः सैव श्वे॒ता श्वे॒तव॑थ्सा॒ दक्षि॑णा ॥ \newline

\textbf{Pada Paata} \newline

पात्रे᳚ । चतुः॑स्रक्ता॒विति॒ चतुः॑ - स्र॒क्तौ॒ । स्व॒य॒म॒व॒प॒न्नाया॒ इति॑ स्वयं-अ॒व॒प॒न्नायै᳚ । शाखा॑यै । क॒र्णान् । च॒ । अक॑र्णान् । च॒ । त॒ण्डु॒लान् । वीति॑ । चि॒नु॒या॒त् । ये । क॒र्णाः । सः । पय॑सि । बा॒र्.॒ह॒स्प॒त्यः । ये । अक॑र्णाः । सः । आज्ये᳚ । मै॒त्रः । स्व॒य॒कृं॒तेति॑ स्वयम् - कृ॒ता । वेदिः॑ । भ॒व॒ति॒ । स्व॒य॒दिं॒नमिति॑ स्वयं - दि॒नम् । ब॒र्॒.हिः । स्व॒य॒कृं॒त इति॑ स्वयं - कृ॒तः । इ॒द्ध्मः । सा । ए॒व । श्वे॒ता । श्वे॒तव॒थ्सेति॑ श्वे॒त - व॒थ्सा॒ । दक्षि॑णा ॥  \newline



\textbf{Jatai Paata} \newline

1. पात्रे॒ चतु॑स्स्रक्तौ॒ चतु॑स्स्रक्तौ॒ पात्रे॒ पात्रे॒ चतु॑स्स्रक्तौ । \newline
2. चतु॑स्स्रक्तौ स्वयमवप॒न्नायै᳚ स्वयमवप॒न्नायै॒ चतु॑स्स्रक्तौ॒ चतु॑स्स्रक्तौ स्वयमवप॒न्नायै᳚ । \newline
3. चतु॑स्स्रक्ता॒विति॒ चतुः॑ - स्र॒क्तौ॒ । \newline
4. स्व॒य॒म॒व॒प॒न्नायै॒ शाखा॑यै॒ शाखा॑यै स्वयमवप॒न्नायै᳚ स्वयमवप॒न्नायै॒ शाखा॑यै । \newline
5. स्व॒य॒म॒व॒प॒न्नाया॒ इति॑ स्वयं - अ॒व॒प॒न्नायै᳚ । \newline
6. शाखा॑यै क॒र्णान् क॒र्णाञ् छाखा॑यै॒ शाखा॑यै क॒र्णान् । \newline
7. क॒र्णाꣳश् च॑ च क॒र्णान् क॒र्णाꣳश् च॑ । \newline
8. चाक॑र्णा॒ नक॑र्णाꣳश् च॒ चाक॑र्णान् । \newline
9. अक॑र्णाꣳश् च॒ चाक॑र्णा॒ नक॑र्णाꣳश् च । \newline
10. च॒ त॒ण्डु॒लान् त॑ण्डु॒लाꣳश् च॑ च तण्डु॒लान् । \newline
11. त॒ण्डु॒लान्. वि वि त॑ण्डु॒लान् त॑ण्डु॒लान्. वि । \newline
12. वि चि॑नुयाच् चिनुया॒द् वि वि चि॑नुयात् । \newline
13. चि॒नु॒या॒द् ये ये चि॑नुयाच् चिनुया॒द् ये । \newline
14. ये क॒र्णाः क॒र्णा ये ये क॒र्णाः । \newline
15. क॒र्णाः स स क॒र्णाः क॒र्णाः सः । \newline
16. स पय॑सि॒ पय॑सि॒ स स पय॑सि । \newline
17. पय॑सि बार्.हस्प॒त्यो बा॑र्.हस्प॒त्यः पय॑सि॒ पय॑सि बार्.हस्प॒त्यः । \newline
18. बा॒र्॒.ह॒स्प॒त्यो ये ये बा॑र्.हस्प॒त्यो बा॑र्.हस्प॒त्यो ये । \newline
19. ये ऽक॑र्णा॒ अक॑र्णा॒ ये ये ऽक॑र्णाः । \newline
20. अक॑र्णाः॒ स सो ऽक॑र्णा॒ अक॑र्णाः॒ सः । \newline
21. स आज्य॒ आज्ये॒ स स आज्ये᳚ । \newline
22. आज्ये॑ मै॒त्रो मै॒त्र आज्य॒ आज्ये॑ मै॒त्रः । \newline
23. मै॒त्रः स्व॑यंकृ॒ता स्व॑यंकृ॒ता मै॒त्रो मै॒त्रः स्व॑यंकृ॒ता । \newline
24. स्व॒यं॒कृ॒ता वेदि॒र् वेदिः॑ स्वयंकृ॒ता स्व॑यंकृ॒ता वेदिः॑ । \newline
25. स्व॒यं॒कृ॒तेति॑ स्वयम् - कृ॒ता । \newline
26. वेदि॑र् भवति भवति॒ वेदि॒र् वेदि॑र् भवति । \newline
27. भ॒व॒ति॒ स्व॒य॒न्दि॒नꣳ स्व॑यन्दि॒नम् भ॑वति भवति स्वयन्दि॒नम् । \newline
28. स्व॒य॒न्दि॒नम् ब॒र्॒.हिर् ब॒र्॒.हिः स्व॑यन्दि॒नꣳ स्व॑यन्दि॒नम् ब॒र्॒.हिः । \newline
29. स्व॒य॒न्दि॒नमिति॑ स्वयं - दि॒नम् । \newline
30. ब॒र्॒.हिः स्व॑यंकृ॒तः स्व॑यंकृ॒तो ब॒र्॒.हिर् ब॒र्॒.हिः स्व॑यंकृ॒तः । \newline
31. स्व॒यं॒कृ॒त इ॒द्ध्म इ॒द्ध्मः स्व॑यंकृ॒तः स्व॑यंकृ॒त इ॒द्ध्मः । \newline
32. स्व॒यं॒कृ॒त इति॑ स्वयं - कृ॒तः । \newline
33. इ॒द्ध्मः सा सेद्ध्म इ॒द्ध्मः सा । \newline
34. सैवैव सा सैव । \newline
35. ए॒व श्वे॒ता श्वे॒तैवैव श्वे॒ता । \newline
36. श्वे॒ता श्वे॒तव॑थ्सा श्वे॒तव॑थ्सा श्वे॒ता श्वे॒ता श्वे॒तव॑थ्सा । \newline
37. श्वे॒तव॑थ्सा॒ दक्षि॑णा॒ दक्षि॑णा श्वे॒तव॑थ्सा श्वे॒तव॑थ्सा॒ दक्षि॑णा । \newline
38. श्वे॒तव॒थ्सेति॑ श्वे॒त - व॒थ्सा॒ । \newline
39. दक्षि॒णेति॒ दक्षि॑णा । \newline

\textbf{Ghana Paata } \newline

1. पात्रे॒ चतु॑स्स्रक्तौ॒ चतु॑स्स्रक्तौ॒ पात्रे॒ पात्रे॒ चतु॑स्स्रक्तौ स्वयमवप॒न्नायै᳚ स्वयमवप॒न्नायै॒ चतु॑स्स्रक्तौ॒ पात्रे॒ पात्रे॒ चतु॑स्स्रक्तौ स्वयमवप॒न्नायै᳚ । \newline
2. चतु॑स्स्रक्तौ स्वयमवप॒न्नायै᳚ स्वयमवप॒न्नायै॒ चतु॑स्स्रक्तौ॒ चतु॑स्स्रक्तौ स्वयमवप॒न्नायै॒ शाखा॑यै॒ शाखा॑यै स्वयमवप॒न्नायै॒ चतु॑स्स्रक्तौ॒ चतु॑स्स्रक्तौ स्वयमवप॒न्नायै॒ शाखा॑यै । \newline
3. चतु॑स्स्रक्ता॒विति॒ चतुः॑ - स्र॒क्तौ॒ । \newline
4. स्व॒य॒म॒व॒प॒न्नायै॒ शाखा॑यै॒ शाखा॑यै स्वयमवप॒न्नायै᳚ स्वयमवप॒न्नायै॒ शाखा॑यै क॒र्णान् क॒र्णाञ् छाखा॑यै स्वयमवप॒न्नायै᳚ स्वयमवप॒न्नायै॒ शाखा॑यै क॒र्णान् । \newline
5. स्व॒य॒म॒व॒प॒न्नाया॒ इति॑ स्वयं - अ॒व॒प॒न्नायै᳚ । \newline
6. शाखा॑यै क॒र्णान् क॒र्णाञ् छाखा॑यै॒ शाखा॑यै क॒र्णाꣳ श्च॑ च क॒र्णाञ् छाखा॑यै॒ शाखा॑यै क॒र्णाꣳ श्च॑ । \newline
7. क॒र्णाꣳश्च॑ च क॒र्णान् क॒र्णाꣳ श्चाक॑र्णा॒ नक॑र्णाꣳश्च क॒र्णान् क॒र्णाꣳ श्चाक॑र्णान् । \newline
8. चाक॑र्णा॒ नक॑र्णाꣳश्च॒ चाक॑र्णाꣳश्च॒ चाक॑र्णाꣳश्च॒ चाक॑र्णाꣳश्च । \newline
9. अक॑र्णाꣳश्च॒ चाक॑र्णा॒ नक॑र्णाꣳश्च तण्डु॒लान् त॑ण्डु॒लाꣳश्चा क॑र्णा॒ नक॑र्णाꣳश्च तण्डु॒लान् । \newline
10. च॒ त॒ण्डु॒लान् त॑ण्डु॒लाꣳश्च॑ च तण्डु॒लान्. वि वि त॑ण्डु॒लाꣳश्च॑ च तण्डु॒लान्. वि । \newline
11. त॒ण्डु॒लान्. वि वि त॑ण्डु॒लान् त॑ण्डु॒लान्. वि चि॑नुयाच् चिनुया॒द् वि त॑ण्डु॒लान् त॑ण्डु॒लान्. वि चि॑नुयात् । \newline
12. वि चि॑नुयाच् चिनुया॒द् वि वि चि॑नुया॒द् ये ये चि॑नुया॒द् वि वि चि॑नुया॒द् ये । \newline
13. चि॒नु॒या॒द् ये ये चि॑नुयाच् चिनुया॒द् ये क॒र्णाः क॒र्णा ये चि॑नुयाच् चिनुया॒द् ये क॒र्णाः । \newline
14. ये क॒र्णाः क॒र्णा ये ये क॒र्णाः स स क॒र्णा ये ये क॒र्णाः सः । \newline
15. क॒र्णाः स स क॒र्णाः क॒र्णाः स पय॑सि॒ पय॑सि॒ स क॒र्णाः क॒र्णाः स पय॑सि । \newline
16. स पय॑सि॒ पय॑सि॒ स स पय॑सि बार्.हस्प॒त्यो बा॑र्.हस्प॒त्यः पय॑सि॒ स स पय॑सि बार्.हस्प॒त्यः । \newline
17. पय॑सि बार्.हस्प॒त्यो बा॑र्.हस्प॒त्यः पय॑सि॒ पय॑सि बार्.हस्प॒त्यो ये ये बा॑र्.हस्प॒त्यः पय॑सि॒ पय॑सि बार्.हस्प॒त्यो ये । \newline
18. बा॒र्॒.ह॒स्प॒त्यो ये ये बा॑र्.हस्प॒त्यो बा॑र्.हस्प॒त्यो ये ऽक॑र्णा॒ अक॑र्णा॒ ये बा॑र्.हस्प॒त्यो बा॑र्.हस्प॒त्यो ये ऽक॑र्णाः । \newline
19. ये ऽक॑र्णा॒ अक॑र्णा॒ ये ये ऽक॑र्णाः॒ स सो ऽक॑र्णा॒ ये ये ऽक॑र्णाः॒ सः । \newline
20. अक॑र्णाः॒ स सो ऽक॑र्णा॒ अक॑र्णाः॒ स आज्य॒ आज्ये॒ सो ऽक॑र्णा॒ अक॑र्णाः॒ स आज्ये᳚ । \newline
21. स आज्य॒ आज्ये॒ स स आज्ये॑ मै॒त्रो मै॒त्र आज्ये॒ स स आज्ये॑ मै॒त्रः । \newline
22. आज्ये॑ मै॒त्रो मै॒त्र आज्य॒ आज्ये॑ मै॒त्रः स्व॑यंकृ॒ता स्व॑यंकृ॒ता मै॒त्र आज्य॒ आज्ये॑ मै॒त्रः स्व॑यंकृ॒ता । \newline
23. मै॒त्रः स्व॑यंकृ॒ता स्व॑यंकृ॒ता मै॒त्रो मै॒त्रः स्व॑यंकृ॒ता वेदि॒र् वेदिः॑ स्वयंकृ॒ता मै॒त्रो मै॒त्रः स्व॑यंकृ॒ता वेदिः॑ । \newline
24. स्व॒यं॒कृ॒ता वेदि॒र् वेदिः॑ स्वयंकृ॒ता स्व॑यंकृ॒ता वेदि॑र् भवति भवति॒ वेदिः॑ स्वयंकृ॒ता स्व॑यंकृ॒ता वेदि॑र् भवति । \newline
25. स्व॒यं॒कृ॒तेति॑ स्वयम् - कृ॒ता । \newline
26. वेदि॑र् भवति भवति॒ वेदि॒र् वेदि॑र् भवति स्वयन्दि॒नꣳ स्व॑यन्दि॒नम् भ॑वति॒ वेदि॒र् वेदि॑र् भवति स्वयन्दि॒नम् । \newline
27. भ॒व॒ति॒ स्व॒य॒न्दि॒नꣳ स्व॑यन्दि॒नम् भ॑वति भवति स्वयन्दि॒नम् ब॒र्॒.हिर् ब॒र्॒.हिः स्व॑यन्दि॒नम् भ॑वति भवति स्वयन्दि॒नम् ब॒र्॒.हिः । \newline
28. स्व॒य॒न्दि॒नम् ब॒र्॒.हिर् ब॒र्॒.हिः स्व॑यन्दि॒नꣳ स्व॑यन्दि॒नम् ब॒र्॒.हिः स्व॑यंकृ॒तः स्व॑यंकृ॒तो ब॒र्॒.हिः स्व॑यन्दि॒नꣳ स्व॑यन्दि॒नम् ब॒र्॒.हिः स्व॑यंकृ॒तः । \newline
29. स्व॒य॒न्दि॒नमिति॑ स्वयं - दि॒नम् । \newline
30. ब॒र्॒.हिः स्व॑यंकृ॒तः स्व॑यंकृ॒तो ब॒र्॒.हिर् ब॒र्॒.हिः स्व॑यंकृ॒त इ॒द्ध्म इ॒द्ध्मः स्व॑यंकृ॒तो ब॒र्॒.हिर् ब॒र्॒.हिः स्व॑यंकृ॒त इ॒द्ध्मः । \newline
31. स्व॒यं॒कृ॒त इ॒द्ध्म इ॒द्ध्मः स्व॑यंकृ॒तः स्व॑यंकृ॒त इ॒द्ध्मः सा सेद्ध्मः स्व॑यंकृ॒तः स्व॑यंकृ॒त इ॒द्ध्मः सा । \newline
32. स्व॒यं॒कृ॒त इति॑ स्वयं - कृ॒तः । \newline
33. इ॒द्ध्मः सा सेद्ध्म इ॒द्ध्मः सैवैव सेद्ध्म इ॒द्ध्मः सैव । \newline
34. सैवैव सा सैव श्वे॒ता श्वे॒तैव सा सैव श्वे॒ता । \newline
35. ए॒व श्वे॒ता श्वे॒तैवैव श्वे॒ता श्वे॒तव॑थ्सा श्वे॒तव॑थ्सा श्वे॒तैवैव श्वे॒ता श्वे॒तव॑थ्सा । \newline
36. श्वे॒ता श्वे॒तव॑थ्सा श्वे॒तव॑थ्सा श्वे॒ता श्वे॒ता श्वे॒तव॑थ्सा॒ दक्षि॑णा॒ दक्षि॑णा श्वे॒तव॑थ्सा श्वे॒ता श्वे॒ता श्वे॒तव॑थ्सा॒ दक्षि॑णा । \newline
37. श्वे॒तव॑थ्सा॒ दक्षि॑णा॒ दक्षि॑णा श्वे॒तव॑थ्सा श्वे॒तव॑थ्सा॒ दक्षि॑णा । \newline
38. श्वे॒तव॒थ्सेति॑ श्वे॒त - व॒थ्सा॒ । \newline
39. दक्षि॒णेति॒ दक्षि॑णा । \newline
\pagebreak
\markright{ TS 1.8.10.1  \hfill https://www.vedavms.in \hfill}
\addcontentsline{toc}{section}{ TS 1.8.10.1 }
\section*{ TS 1.8.10.1 }

\textbf{TS 1.8.10.1 } \newline
\textbf{Samhita Paata} \newline

अ॒ग्नये॑ गृ॒हप॑तये पुरो॒डाश॑म॒ष्टाक॑पालं॒ निर्व॑पति कृ॒ष्णानां᳚ व्रीही॒णाꣳ सोमा॑य॒ वन॒स्पत॑ये श्यामा॒कं च॒रुꣳ स॑वि॒त्रे स॒त्यप्र॑सवाय पुरो॒डाशं॒ द्वाद॑शकपाल-माशू॒नां ॅव्री॑ही॒णाꣳ रु॒द्राय॑ पशु॒पत॑ये गावीधु॒कं च॒रुं बृह॒स्पत॑ये वा॒चस्पत॑ये नैवा॒रं च॒रुमिन्द्रा॑य ज्ये॒ष्ठाय॑ पुरो॒डाश॒-मेका॑दशकपालं म॒हाव्री॑हीणां मि॒त्राय॑ स॒त्याया॒ऽऽंबानां᳚ च॒रुं ॅवरु॑णाय॒ धर्म॑पतये यव॒मयं॑ च॒रुꣳ स॑वि॒ता त्वा᳚ प्रस॒वानाꣳ॑ सुवताम॒ग्निर् गृ॒हप॑तीनाꣳ॒॒ सोमो॒ वन॒स्पती॑नाꣳ रु॒द्रः प॑शू॒नां -[ ] \newline

\textbf{Pada Paata} \newline

अ॒ग्नये᳚ । गृ॒हप॑तय॒ इति॑ गृ॒ह - प॒त॒ये॒ । पु॒रो॒डाश᳚म् । अ॒ष्टाक॑पाल॒मित्य॒ष्टा - क॒पा॒ल॒म् । निरिति॑ । व॒प॒ति॒ । कृ॒ष्णाना᳚म् । व्री॒ही॒णाम् । सोमा॑य । वन॒स्पत॑ये । श्या॒मा॒कम् । च॒रुम् । स॒वि॒त्रे । स॒त्यप्र॑सवा॒येति॑ स॒त्य - प्र॒स॒वा॒य॒ । पु॒रो॒डाश᳚म् । द्वाद॑शकपाल॒मिति॒ द्वाद॑श - क॒पा॒ल॒म् । आ॒शू॒नाम् । व्री॒ही॒णाम् । रु॒द्राय॑ । प॒शु॒पत॑य॒ इति॑ पशु - पत॑ये । गा॒वी॒धु॒कम् । च॒रुम् । बृह॒स्पत॑ये । वा॒चः । पत॑ये । नै॒वा॒रम् । च॒रुम् । इन्द्रा॑य । ज्ये॒ष्ठाय॑ । पु॒रो॒डाश᳚म् । एका॑दशकपाल॒मित्येका॑दश - क॒पा॒ल॒म् । म॒हाव्री॑हीणा॒मिति॑ म॒हा-व्री॒ही॒णा॒म् । मि॒त्राय॑ । स॒त्याय॑ । आ॒बांना᳚म् । च॒रुम् । वरु॑णाय । धर्म॑पतय॒ इति॒ धर्म - प॒त॒ये॒ । य॒व॒मय॒मिति॑ यव - मय᳚म् । च॒रुम् । स॒वि॒ता । त्वा॒ । प्र॒स॒वाना॒मिति॑ प्र - स॒वाना᳚म् । सु॒व॒ता॒म् । अ॒ग्निः । गृ॒हप॑तीना॒मिति॑ गृ॒ह - प॒ती॒ना॒म् । सोमः॑ । वन॒स्पती॑नाम् । रु॒द्रः । प॒शू॒नाम् ।  \newline



\textbf{Jatai Paata} \newline

1. अ॒ग्नये॑ गृ॒हप॑तये गृ॒हप॑तये॒ ऽग्नये॒ ऽग्नये॑ गृ॒हप॑तये । \newline
2. गृ॒हप॑तये पुरो॒डाश॑म् पुरो॒डाश॑म् गृ॒हप॑तये गृ॒हप॑तये पुरो॒डाश᳚म् । \newline
3. गृ॒हप॑तय॒ इति॑ गृ॒ह - प॒त॒ये॒ । \newline
4. पु॒रो॒डाश॑ म॒ष्टाक॑पाल म॒ष्टाक॑पालम् पुरो॒डाश॑म् पुरो॒डाश॑ म॒ष्टाक॑पालम् । \newline
5. अ॒ष्टाक॑पाल॒म् निर् णिर॒ष्टाक॑पाल म॒ष्टाक॑पाल॒म् निः । \newline
6. अ॒ष्टाक॑पाल॒मित्य॒ष्टा - क॒पा॒ल॒म् । \newline
7. निर् व॑पति वपति॒ निर् णिर् व॑पति । \newline
8. व॒प॒ति॒ कृ॒ष्णाना᳚म् कृ॒ष्णानां᳚ ॅवपति वपति कृ॒ष्णाना᳚म् । \newline
9. कृ॒ष्णानां᳚ ॅव्रीही॒णां ॅव्री॑ही॒णाम् कृ॒ष्णाना᳚म् कृ॒ष्णानां᳚ ॅव्रीही॒णाम् । \newline
10. व्री॒ही॒णाꣳ सोमा॑य॒ सोमा॑य व्रीही॒णां ॅव्री॑ही॒णाꣳ सोमा॑य । \newline
11. सोमा॑य॒ वन॒स्पत॑ये॒ वन॒स्पत॑ये॒ सोमा॑य॒ सोमा॑य॒ वन॒स्पत॑ये । \newline
12. वन॒स्पत॑ये श्यामा॒कꣳ श्या॑मा॒कं ॅवन॒स्पत॑ये॒ वन॒स्पत॑ये श्यामा॒कम् । \newline
13. श्या॒मा॒कम् च॒रुम् च॒रुꣳ श्या॑मा॒कꣳ श्या॑मा॒कम् च॒रुम् । \newline
14. च॒रुꣳ स॑वि॒त्रे स॑वि॒त्रे च॒रुम् च॒रुꣳ स॑वि॒त्रे । \newline
15. स॒वि॒त्रे स॒त्यप्र॑सवाय स॒त्यप्र॑सवाय सवि॒त्रे स॑वि॒त्रे स॒त्यप्र॑सवाय । \newline
16. स॒त्यप्र॑सवाय पुरो॒डाश॑म् पुरो॒डाशꣳ॑ स॒त्यप्र॑सवाय स॒त्यप्र॑सवाय पुरो॒डाश᳚म् । \newline
17. स॒त्यप्र॑सवा॒येति॑ स॒त्य - प्र॒स॒वा॒य॒ । \newline
18. पु॒रो॒डाश॒म् द्वाद॑शकपाल॒म् द्वाद॑शकपालम् पुरो॒डाश॑म् पुरो॒डाश॒म् द्वाद॑शकपालम् । \newline
19. द्वाद॑शकपाल माशू॒ना मा॑शू॒नाम् द्वाद॑शकपाल॒म् द्वाद॑शकपाल माशू॒नाम् । \newline
20. द्वाद॑शकपाल॒मिति॒ द्वाद॑श - क॒पा॒ल॒म् । \newline
21. आ॒शू॒नां ॅव्री॑ही॒णां ॅव्री॑ही॒णा मा॑शू॒ना मा॑शू॒नां ॅव्री॑ही॒णाम् । \newline
22. व्री॒ही॒णाꣳ रु॒द्राय॑ रु॒द्राय॑ व्रीही॒णां ॅव्री॑ही॒णाꣳ रु॒द्राय॑ । \newline
23. रु॒द्राय॑ पशु॒पत॑ये पशु॒पत॑ये रु॒द्राय॑ रु॒द्राय॑ पशु॒पत॑ये । \newline
24. प॒शु॒पत॑ये गावीधु॒कम् गा॑वीधु॒कम् प॑शु॒पत॑ये पशु॒पत॑ये गावीधु॒कम् । \newline
25. प॒शु॒पत॑य॒ इति॑ पशु - पत॑ये । \newline
26. गा॒वी॒धु॒कम् च॒रुम् च॒रुम् गा॑वीधु॒कम् गा॑वीधु॒कम् च॒रुम् । \newline
27. च॒रुम् बृह॒स्पत॑ये॒ बृह॒स्पत॑ये च॒रुम् च॒रुम् बृह॒स्पत॑ये । \newline
28. बृह॒स्पत॑ये वा॒चो वा॒चो बृह॒स्पत॑ये॒ बृह॒स्पत॑ये वा॒चः । \newline
29. वा॒चस् पत॑ये॒ पत॑ये वा॒चो वा॒चस् पत॑ये । \newline
30. पत॑ये नैवा॒रन्नै॑वा॒रम् पत॑ये॒ पत॑ये नैवा॒रम् । \newline
31. नै॒वा॒रम् च॒रुम् च॒रुम् नै॑वा॒रम् नै॑वा॒रम् च॒रुम् । \newline
32. च॒रु मिन्द्रा॒ये न्द्रा॑य च॒रुम् च॒रु मिन्द्रा॑य । \newline
33. इन्द्रा॑य ज्ये॒ष्ठाय॑ ज्ये॒ष्ठाये न्द्रा॒ये न्द्रा॑य ज्ये॒ष्ठाय॑ । \newline
34. ज्ये॒ष्ठाय॑ पुरो॒डाश॑म् पुरो॒डाश॑म् ज्ये॒ष्ठाय॑ ज्ये॒ष्ठाय॑ पुरो॒डाश᳚म् । \newline
35. पु॒रो॒डाश॒ मेका॑दशकपाल॒ मेका॑दशकपालम् पुरो॒डाश॑म् पुरो॒डाश॒ मेका॑दशकपालम् । \newline
36. एका॑दशकपालम् म॒हाव्री॑हीणाम् म॒हाव्री॑हीणा॒ मेका॑दशकपाल॒ मेका॑दशकपालम् म॒हाव्री॑हीणाम् । \newline
37. एका॑दशकपाल॒मित्येका॑दश - क॒पा॒ल॒म् । \newline
38. म॒हाव्री॑हीणाम् मि॒त्राय॑ मि॒त्राय॑ म॒हाव्री॑हीणाम् म॒हाव्री॑हीणाम् मि॒त्राय॑ । \newline
39. म॒हाव्री॑हीणा॒मिति॑ म॒हा - व्री॒ही॒णा॒म् । \newline
40. मि॒त्राय॑ स॒त्याय॑ स॒त्याय॑ मि॒त्राय॑ मि॒त्राय॑ स॒त्याय॑ । \newline
41. स॒त्याया॒म्बाना॑ मा॒म्बानाꣳ॑ स॒त्याय॑ स॒त्याया॒म्बाना᳚म् । \newline
42. आ॒म्बाना᳚म् च॒रुम् च॒रु मा॒म्बाना॑ मा॒म्बाना᳚म् च॒रुम् । \newline
43. च॒रुं ॅवरु॑णाय॒ वरु॑णाय च॒रुम् च॒रुं ॅवरु॑णाय । \newline
44. वरु॑णाय॒ धर्म॑पतये॒ धर्म॑पतये॒ वरु॑णाय॒ वरु॑णाय॒ धर्म॑पतये । \newline
45. धर्म॑पतये यव॒मयं॑ ॅयव॒मय॒म् धर्म॑पतये॒ धर्म॑पतये यव॒मय᳚म् । \newline
46. धर्म॑पतय॒ इति॒ धर्म॑ - प॒त॒ये॒ । \newline
47. य॒व॒मय॑म् च॒रुम् च॒रुं ॅय॑व॒मयं॑ ॅयव॒मय॑म् च॒रुम् । \newline
48. य॒व॒मय॒मिति॑ यव - मय᳚म् । \newline
49. च॒रुꣳ स॑वि॒ता स॑वि॒ता च॒रुम् च॒रुꣳ स॑वि॒ता । \newline
50. स॒वि॒ता त्वा᳚ त्वा सवि॒ता स॑वि॒ता त्वा᳚ । \newline
51. त्वा॒ प्र॒स॒वाना᳚म् प्रस॒वाना᳚म् त्वा त्वा प्रस॒वाना᳚म् । \newline
52. प्र॒स॒वानाꣳ॑ सुवताꣳ सुवताम् प्रस॒वाना᳚म् प्रस॒वानाꣳ॑ सुवताम् । \newline
53. प्र॒स॒वाना॒मिति॑ प्र - स॒वाना᳚म् । \newline
54. सु॒व॒ता॒ म॒ग्नि र॒ग्निः सु॑वताꣳ सुवता म॒ग्निः । \newline
55. अ॒ग्निर् गृ॒हप॑तीनाम् गृ॒हप॑तीना म॒ग्निर॒ग्निर् गृ॒हप॑तीनाम् । \newline
56. गृ॒हप॑तीनाꣳ॒॒ सोमः॒ सोमो॑ गृ॒हप॑तीनाम् गृ॒हप॑तीनाꣳ॒॒ सोमः॑ । \newline
57. गृ॒हप॑तीना॒मिति॑ गृ॒ह - प॒ती॒ना॒म् । \newline
58. सोमो॒ वन॒स्पती॑नां॒ ॅवन॒स्पती॑नाꣳ॒॒ सोमः॒ सोमो॒ वन॒स्पती॑नाम् । \newline
59. वन॒स्पती॑नाꣳ रु॒द्रो रु॒द्रो वन॒स्पती॑नां॒ ॅवन॒स्पती॑नाꣳ रु॒द्रः । \newline
60. रु॒द्रः प॑शू॒नाम् प॑शू॒नाꣳ रु॒द्रो रु॒द्रः प॑शू॒नाम् । \newline
61. प॒शू॒नाम् बृह॒स्पति॒र् बृह॒स्पतिः॑ पशू॒नाम् प॑शू॒नाम् बृह॒स्पतिः॑ । \newline

\textbf{Ghana Paata } \newline

1. अ॒ग्नये॑ गृ॒हप॑तये गृ॒हप॑तये॒ ऽग्नये॒ ऽग्नये॑ गृ॒हप॑तये पुरो॒डाश॑म् पुरो॒डाश॑म् गृ॒हप॑तये॒ ऽग्नये॒ ऽग्नये॑ गृ॒हप॑तये पुरो॒डाश᳚म् । \newline
2. गृ॒हप॑तये पुरो॒डाश॑म् पुरो॒डाश॑म् गृ॒हप॑तये गृ॒हप॑तये पुरो॒डाश॑ म॒ष्टाक॑पाल म॒ष्टाक॑पालम् पुरो॒डाश॑म् गृ॒हप॑तये गृ॒हप॑तये पुरो॒डाश॑ म॒ष्टाक॑पालम् । \newline
3. गृ॒हप॑तय॒ इति॑ गृ॒ह - प॒त॒ये॒ । \newline
4. पु॒रो॒डाश॑ म॒ष्टाक॑पाल म॒ष्टाक॑पालम् पुरो॒डाश॑म् पुरो॒डाश॑ म॒ष्टाक॑पाल॒न्निर् णिर॒ष्टाक॑पालम् पुरो॒डाश॑म् पुरो॒डाश॑ म॒ष्टाक॑पाल॒न्निः । \newline
5. अ॒ष्टाक॑पाल॒न्निर् णिर॒ष्टाक॑पाल म॒ष्टाक॑पाल॒न्निर् व॑पति वपति॒ निर॒ष्टाक॑पाल म॒ष्टाक॑पाल॒न्निर् व॑पति । \newline
6. अ॒ष्टाक॑पाल॒मित्य॒ष्टा - क॒पा॒ल॒म् । \newline
7. निर् व॑पति वपति॒ निर् णिर् व॑पति कृ॒ष्णाना᳚म् कृ॒ष्णानां᳚ ॅवपति॒ निर् णिर् व॑पति कृ॒ष्णाना᳚म् । \newline
8. व॒प॒ति॒ कृ॒ष्णाना᳚म् कृ॒ष्णानां᳚ ॅवपति वपति कृ॒ष्णानां᳚ ॅव्रीही॒णां ॅव्री॑ही॒णाम् कृ॒ष्णानां᳚ ॅवपति वपति कृ॒ष्णानां᳚ ॅव्रीही॒णाम् । \newline
9. कृ॒ष्णानां᳚ ॅव्रीही॒णां ॅव्री॑ही॒णाम् कृ॒ष्णाना᳚म् कृ॒ष्णानां᳚ ॅव्रीही॒णाꣳ सोमा॑य॒ सोमा॑य व्रीही॒णाम् कृ॒ष्णाना᳚म् कृ॒ष्णानां᳚ ॅव्रीही॒णाꣳ सोमा॑य । \newline
10. व्री॒ही॒णाꣳ सोमा॑य॒ सोमा॑य व्रीही॒णां ॅव्री॑ही॒णाꣳ सोमा॑य॒ वन॒स्पत॑ये॒ वन॒स्पत॑ये॒ सोमा॑य व्रीही॒णां ॅव्री॑ही॒णाꣳ सोमा॑य॒ वन॒स्पत॑ये । \newline
11. सोमा॑य॒ वन॒स्पत॑ये॒ वन॒स्पत॑ये॒ सोमा॑य॒ सोमा॑य॒ वन॒स्पत॑ये श्यामा॒कꣳ श्या॑मा॒कं ॅवन॒स्पत॑ये॒ सोमा॑य॒ सोमा॑य॒ वन॒स्पत॑ये श्यामा॒कम् । \newline
12. वन॒स्पत॑ये श्यामा॒कꣳ श्या॑मा॒कं ॅवन॒स्पत॑ये॒ वन॒स्पत॑ये श्यामा॒कम् च॒रुम् च॒रुꣳ श्या॑मा॒कं ॅवन॒स्पत॑ये॒ वन॒स्पत॑ये श्यामा॒कम् च॒रुम् । \newline
13. श्या॒मा॒कम् च॒रुम् च॒रुꣳ श्या॑मा॒कꣳ श्या॑मा॒कम् च॒रुꣳ स॑वि॒त्रे स॑वि॒त्रे च॒रुꣳ श्या॑मा॒कꣳ श्या॑मा॒कम् च॒रुꣳ स॑वि॒त्रे । \newline
14. च॒रुꣳ स॑वि॒त्रे स॑वि॒त्रे च॒रुम् च॒रुꣳ स॑वि॒त्रे स॒त्यप्र॑सवाय स॒त्यप्र॑सवाय सवि॒त्रे च॒रुम् च॒रुꣳ स॑वि॒त्रे स॒त्यप्र॑सवाय । \newline
15. स॒वि॒त्रे स॒त्यप्र॑सवाय स॒त्यप्र॑सवाय सवि॒त्रे स॑वि॒त्रे स॒त्यप्र॑सवाय पुरो॒डाश॑म् पुरो॒डाशꣳ॑ स॒त्यप्र॑सवाय सवि॒त्रे स॑वि॒त्रे स॒त्यप्र॑सवाय पुरो॒डाश᳚म् । \newline
16. स॒त्यप्र॑सवाय पुरो॒डाश॑म् पुरो॒डाशꣳ॑ स॒त्यप्र॑सवाय स॒त्यप्र॑सवाय पुरो॒डाश॒म् द्वाद॑शकपाल॒म् द्वाद॑शकपालम् पुरो॒डाशꣳ॑ स॒त्यप्र॑सवाय स॒त्यप्र॑सवाय पुरो॒डाश॒म् द्वाद॑शकपालम् । \newline
17. स॒त्यप्र॑सवा॒येति॑ स॒त्य - प्र॒स॒वा॒य॒ । \newline
18. पु॒रो॒डाश॒म् द्वाद॑शकपाल॒म् द्वाद॑शकपालम् पुरो॒डाश॑म् पुरो॒डाश॒म् द्वाद॑शकपाल माशू॒ना मा॑शू॒नाम् द्वाद॑शकपालम् पुरो॒डाश॑म् पुरो॒डाश॒म् द्वाद॑शकपाल माशू॒नाम् । \newline
19. द्वाद॑शकपाल माशू॒ना मा॑शू॒नाम् द्वाद॑शकपाल॒म् द्वाद॑शकपाल माशू॒नां ॅव्री॑ही॒णां ॅव्री॑ही॒णा मा॑शू॒नाम् द्वाद॑शकपाल॒म् द्वाद॑शकपाल माशू॒नां ॅव्री॑ही॒णाम् । \newline
20. द्वाद॑शकपाल॒मिति॒ द्वाद॑श - क॒पा॒ल॒म् । \newline
21. आ॒शू॒नां ॅव्री॑ही॒णां ॅव्री॑ही॒णा मा॑शू॒ना मा॑शू॒नां ॅव्री॑ही॒णाꣳ रु॒द्राय॑ रु॒द्राय॑ व्रीही॒णा मा॑शू॒ना मा॑शू॒नां ॅव्री॑ही॒णाꣳ रु॒द्राय॑ । \newline
22. व्री॒ही॒णाꣳ रु॒द्राय॑ रु॒द्राय॑ व्रीही॒णां ॅव्री॑ही॒णाꣳ रु॒द्राय॑ पशु॒पत॑ये पशु॒पत॑ये रु॒द्राय॑ व्रीही॒णां ॅव्री॑ही॒णाꣳ रु॒द्राय॑ पशु॒पत॑ये । \newline
23. रु॒द्राय॑ पशु॒पत॑ये पशु॒पत॑ये रु॒द्राय॑ रु॒द्राय॑ पशु॒पत॑ये गावीधु॒कम् गा॑वीधु॒कम् प॑शु॒पत॑ये रु॒द्राय॑ रु॒द्राय॑ पशु॒पत॑ये गावीधु॒कम् । \newline
24. प॒शु॒पत॑ये गावीधु॒कम् गा॑वीधु॒कम् प॑शु॒पत॑ये पशु॒पत॑ये गावीधु॒कम् च॒रुम् च॒रुम् गा॑वीधु॒कम् प॑शु॒पत॑ये पशु॒पत॑ये गावीधु॒कम् च॒रुम् । \newline
25. प॒शु॒पत॑य॒ इति॑ पशु - पत॑ये । \newline
26. गा॒वी॒धु॒कम् च॒रुम् च॒रुम् गा॑वीधु॒कम् गा॑वीधु॒कम् च॒रुम् बृह॒स्पत॑ये॒ बृह॒स्पत॑ये च॒रुम् गा॑वीधु॒कम् गा॑वीधु॒कम् च॒रुम् बृह॒स्पत॑ये । \newline
27. च॒रुम् बृह॒स्पत॑ये॒ बृह॒स्पत॑ये च॒रुम् च॒रुम् बृह॒स्पत॑ये वा॒चो वा॒चो बृह॒स्पत॑ये च॒रुम् च॒रुम् बृह॒स्पत॑ये वा॒चः । \newline
28. बृह॒स्पत॑ये वा॒चो वा॒चो बृह॒स्पत॑ये॒ बृह॒स्पत॑ये वा॒च स्पत॑ये॒ पत॑ये वा॒चो बृह॒स्पत॑ये॒ बृह॒स्पत॑ये वा॒च स्पत॑ये । \newline
29. वा॒च स्पत॑ये॒ पत॑ये वा॒चो वा॒च स्पत॑ये नैवा॒रन् नै॑वा॒रम् पत॑ये वा॒चो वा॒च स्पत॑ये नैवा॒रम् । \newline
30. पत॑ये नैवा॒रन् नै॑वा॒रम् पत॑ये॒ पत॑ये नैवा॒रम् च॒रुम् च॒रुन् नै॑वा॒रम् पत॑ये॒ पत॑ये नैवा॒रम् च॒रुम् । \newline
31. नै॒वा॒रम् च॒रुम् च॒रुन् नै॑वा॒रन् नै॑वा॒रम् च॒रु मिन्द्रा॒ये न्द्रा॑य च॒रुन् नै॑वा॒रन् नै॑वा॒रम् च॒रु मिन्द्रा॑य । \newline
32. च॒रु मिन्द्रा॒ये न्द्रा॑य च॒रुम् च॒रु मिन्द्रा॑य ज्ये॒ष्ठाय॑ ज्ये॒ष्ठाये न्द्रा॑य च॒रुम् च॒रु मिन्द्रा॑य ज्ये॒ष्ठाय॑ । \newline
33. इन्द्रा॑य ज्ये॒ष्ठाय॑ ज्ये॒ष्ठाये न्द्रा॒ये न्द्रा॑य ज्ये॒ष्ठाय॑ पुरो॒डाश॑म् पुरो॒डाश॑म् ज्ये॒ष्ठाये न्द्रा॒ये न्द्रा॑य ज्ये॒ष्ठाय॑ पुरो॒डाश᳚म् । \newline
34. ज्ये॒ष्ठाय॑ पुरो॒डाश॑म् पुरो॒डाश॑म् ज्ये॒ष्ठाय॑ ज्ये॒ष्ठाय॑ पुरो॒डाश॒ मेका॑दशकपाल॒ मेका॑दशकपालम् पुरो॒डाश॑म् ज्ये॒ष्ठाय॑ ज्ये॒ष्ठाय॑ पुरो॒डाश॒ मेका॑दशकपालम् । \newline
35. पु॒रो॒डाश॒ मेका॑दशकपाल॒ मेका॑दशकपालम् पुरो॒डाश॑म् पुरो॒डाश॒ मेका॑दशकपालम् म॒हाव्री॑हीणाम् म॒हाव्री॑हीणा॒ मेका॑दशकपालम् पुरो॒डाश॑म् पुरो॒डाश॒ मेका॑दशकपालम् म॒हाव्री॑हीणाम् । \newline
36. एका॑दशकपालम् म॒हाव्री॑हीणाम् म॒हाव्री॑हीणा॒ मेका॑दशकपाल॒ मेका॑दशकपालम् म॒हाव्री॑हीणाम् मि॒त्राय॑ मि॒त्राय॑ म॒हाव्री॑हीणा॒ मेका॑दशकपाल॒ मेका॑दशकपालम् म॒हाव्री॑हीणाम् मि॒त्राय॑ । \newline
37. एका॑दशकपाल॒मित्येका॑दश - क॒पा॒ल॒म् । \newline
38. म॒हाव्री॑हीणाम् मि॒त्राय॑ मि॒त्राय॑ म॒हाव्री॑हीणाम् म॒हाव्री॑हीणाम् मि॒त्राय॑ स॒त्याय॑ स॒त्याय॑ मि॒त्राय॑ म॒हाव्री॑हीणाम् म॒हाव्री॑हीणाम् मि॒त्राय॑ स॒त्याय॑ । \newline
39. म॒हाव्री॑हीणा॒मिति॑ म॒हा - व्री॒ही॒णा॒म् । \newline
40. मि॒त्राय॑ स॒त्याय॑ स॒त्याय॑ मि॒त्राय॑ मि॒त्राय॑ स॒त्याया॒म्बाना॑ मा॒म्बानाꣳ॑ स॒त्याय॑ मि॒त्राय॑ मि॒त्राय॑ स॒त्याया॒म्बाना᳚म् । \newline
41. स॒त्याया॒म्बाना॑ मा॒म्बानाꣳ॑ स॒त्याय॑ स॒त्याया॒म्बाना᳚म् च॒रुम् च॒रु मां॒बानाꣳ॑ स॒त्याय॑ स॒त्याया॒म्बाना᳚म् च॒रुम् । \newline
42. आ॒म्बाना᳚म् च॒रुम् च॒रु मा॒म्बाना॑ मा॒म्बाना᳚म् च॒रुं ॅवरु॑णाय॒ वरु॑णाय च॒रु मा॒म्बाना॑ मा॒म्बाना᳚म् च॒रुं ॅवरु॑णाय । \newline
43. च॒रुं ॅवरु॑णाय॒ वरु॑णाय च॒रुम् च॒रुं ॅवरु॑णाय॒ धर्म॑पतये॒ धर्म॑पतये॒ वरु॑णाय च॒रुम् च॒रुं ॅवरु॑णाय॒ धर्म॑पतये । \newline
44. वरु॑णाय॒ धर्म॑पतये॒ धर्म॑पतये॒ वरु॑णाय॒ वरु॑णाय॒ धर्म॑पतये यव॒मयं॑ ॅयव॒मय॒म् धर्म॑पतये॒ वरु॑णाय॒ वरु॑णाय॒ धर्म॑पतये यव॒मय᳚म् । \newline
45. धर्म॑पतये यव॒मयं॑ ॅयव॒मय॒म् धर्म॑पतये॒ धर्म॑पतये यव॒मय॑म् च॒रुम् च॒रुं ॅय॑व॒मय॒म् धर्म॑पतये॒ धर्म॑पतये यव॒मय॑म् च॒रुम् । \newline
46. धर्म॑पतय॒ इति॒ धर्म॑ - प॒त॒ये॒ । \newline
47. य॒व॒मय॑म् च॒रुम् च॒रुं ॅय॑व॒मयं॑ ॅयव॒मय॑म् च॒रुꣳ स॑वि॒ता स॑वि॒ता च॒रुं ॅय॑व॒मयं॑ ॅयव॒मय॑म् च॒रुꣳ स॑वि॒ता । \newline
48. य॒व॒मय॒मिति॑ यव - मय᳚म् । \newline
49. च॒रुꣳ स॑वि॒ता स॑वि॒ता च॒रुम् च॒रुꣳ स॑वि॒ता त्वा᳚ त्वा सवि॒ता च॒रुम् च॒रुꣳ स॑वि॒ता त्वा᳚ । \newline
50. स॒वि॒ता त्वा᳚ त्वा सवि॒ता स॑वि॒ता त्वा᳚ प्रस॒वाना᳚म् प्रस॒वाना᳚म् त्वा सवि॒ता स॑वि॒ता त्वा᳚ प्रस॒वाना᳚म् । \newline
51. त्वा॒ प्र॒स॒वाना᳚म् प्रस॒वाना᳚म् त्वा त्वा प्रस॒वानाꣳ॑ सुवताꣳ सुवताम् प्रस॒वाना᳚म् त्वा त्वा प्रस॒वानाꣳ॑ सुवताम् । \newline
52. प्र॒स॒वानाꣳ॑ सुवताꣳ सुवताम् प्रस॒वाना᳚म् प्रस॒वानाꣳ॑ सुवता म॒ग्नि र॒ग्निः सु॑वताम् प्रस॒वाना᳚म् प्रस॒वानाꣳ॑ सुवता म॒ग्निः । \newline
53. प्र॒स॒वाना॒मिति॑ प्र - स॒वाना᳚म् । \newline
54. सु॒व॒ता॒ म॒ग्नि र॒ग्निः सु॑वताꣳ सुवता म॒ग्निर् गृ॒हप॑तीनाम् गृ॒हप॑तीना म॒ग्निः सु॑वताꣳ सुवता म॒ग्निर् गृ॒हप॑तीनाम् । \newline
55. अ॒ग्निर् गृ॒हप॑तीनाम् गृ॒हप॑तीना म॒ग्नि र॒ग्निर् गृ॒हप॑तीनाꣳ॒॒ सोमः॒ सोमो॑ गृ॒हप॑तीना म॒ग्नि र॒ग्निर् गृ॒हप॑तीनाꣳ॒॒ सोमः॑ । \newline
56. गृ॒हप॑तीनाꣳ॒॒ सोमः॒ सोमो॑ गृ॒हप॑तीनाम् गृ॒हप॑तीनाꣳ॒॒ सोमो॒ वन॒स्पती॑नां॒ ॅवन॒स्पती॑नाꣳ॒॒ सोमो॑ गृ॒हप॑तीनाम् गृ॒हप॑तीनाꣳ॒॒ सोमो॒ वन॒स्पती॑नाम् । \newline
57. गृ॒हप॑तीना॒मिति॑ गृ॒ह - प॒ती॒ना॒म् । \newline
58. सोमो॒ वन॒स्पती॑नां॒ ॅवन॒स्पती॑नाꣳ॒॒ सोमः॒ सोमो॒ वन॒स्पती॑नाꣳ रु॒द्रो रु॒द्रो वन॒स्पती॑नाꣳ॒॒ सोमः॒ सोमो॒ वन॒स्पती॑नाꣳ रु॒द्रः । \newline
59. वन॒स्पती॑नाꣳ रु॒द्रो रु॒द्रो वन॒स्पती॑नां॒ ॅवन॒स्पती॑नाꣳ रु॒द्रः प॑शू॒नाम् प॑शू॒नाꣳ रु॒द्रो वन॒स्पती॑नां॒ ॅवन॒स्पती॑नाꣳ रु॒द्रः प॑शू॒नाम् । \newline
60. रु॒द्रः प॑शू॒नाम् प॑शू॒नाꣳ रु॒द्रो रु॒द्रः प॑शू॒नाम् बृह॒स्पति॒र् बृह॒स्पतिः॑ पशू॒नाꣳ रु॒द्रो रु॒द्रः प॑शू॒नाम् बृह॒स्पतिः॑ । \newline
61. प॒शू॒नाम् बृह॒स्पति॒र् बृह॒स्पतिः॑ पशू॒नाम् प॑शू॒नाम् बृह॒स्पति॑र् वा॒चां ॅवा॒चाम् बृह॒स्पतिः॑ पशू॒नाम् प॑शू॒नाम् बृह॒स्पति॑र् वा॒चाम् । \newline
\pagebreak
\markright{ TS 1.8.10.2  \hfill https://www.vedavms.in \hfill}
\addcontentsline{toc}{section}{ TS 1.8.10.2 }
\section*{ TS 1.8.10.2 }

\textbf{TS 1.8.10.2 } \newline
\textbf{Samhita Paata} \newline

बृह॒स्पति॑र् वा॒चामिन्द्रो᳚ ज्ये॒ष्ठानां᳚ मि॒त्रः स॒त्यानां॒ ॅवरु॑णो॒ धर्म॑पतीनां॒ ॅये दे॑वा देव॒सुवः॒ स्थ त इ॒म-मा॑मुष्याय॒ण-म॑नमि॒त्राय॑ सुवद्ध्वं मह॒ते क्ष॒त्राय॑ मह॒त आधि॑पत्याय मह॒ते जान॑राज्यायै॒ष वो॑ भरता॒ राजा॒ सोमो॒ऽस्माकं॑ ब्राह्म॒णानाꣳ॒॒ राजा॒ प्रति॒ त्यन्नाम॑ रा॒ज्य-म॑धायि॒ स्वां त॒नुवं॒ ॅवरु॑णो अशिश्रे॒च्छुचे᳚र् मि॒त्रस्य॒ व्रत्या॑ अभू॒माम॑न्महि मह॒त ऋ॒तस्य॒ नाम॒ सर्वे॒ व्राता॒ ( ) वरु॑णस्याभूव॒न् वि मि॒त्र एवै॒-ररा॑ति-मतारी॒दसू॑षुदन्त य॒ज्ञिया॑ ऋ॒तेन॒ व्यु॑ त्रि॒तो ज॑रि॒माणं॑ न आन॒ड् विष्णोः॒ क्रमो॑ऽसि॒ विष्णोः᳚ क्रा॒न्तम॑सि॒ विष्णो॒र् विक्रा᳚न्त-मसि ॥ \newline

\textbf{Pada Paata} \newline

बृह॒स्पतिः॑ । वा॒चम् । इन्द्रः॑ । ज्ये॒ष्ठाना᳚म् । मि॒त्रः । स॒त्याना᳚म् । वरु॑णः । धर्म॑पतीना॒मिति॒ धर्म॑ - प॒ती॒ना॒म् । ये । दे॒वाः॒ । दे॒व॒सुव॒ इति॑ देव - सुवः॑ । स्थ । ते । इ॒मम् । आ॒मु॒ष्या॒य॒णम् । अ॒न॒मि॒त्राय॑ । सु॒व॒द्ध्व॒म् । म॒ह॒ते । क्ष॒त्राय॑ । म॒ह॒ते । आधि॑पत्या॒येत्याधि॑-प॒त्या॒य॒ । म॒ह॒ते । जान॑राज्या॒येति॒ जान॑-रा॒ज्या॒य॒ । ए॒षः । वः॒ । भ॒र॒ताः॒ । राजा᳚ । सोमः॑ । अ॒स्माक᳚म् । ब्रा॒ह्म॒णाना᳚म् । राजा᳚ । प्रतीति॑ । त्यत् । नाम॑ । रा॒ज्यम् । अ॒धा॒यि॒ । स्वाम् । त॒नुव᳚म् । वरु॑णः । अ॒शि॒श्रे॒त् । शुचेः᳚ । मि॒त्रस्य॑ । व्रत्याः᳚ । अ॒भू॒म॒ । अम॑न्महि । म॒ह॒तः । ऋ॒तस्य॑ । नाम॑ । सर्वे᳚ । व्राताः᳚ ( ) । वरु॑णस्य । अ॒भू॒व॒न्न् । वीति॑ । मि॒त्रः । एवैः᳚ । अरा॑तिम् । अ॒ता॒री॒त् । असू॑षुदन्त । य॒ज्ञियाः᳚ । ऋ॒तेन॑ । वीति॑ । उ॒ । त्रि॒तः । ज॒रि॒माण᳚म् । नः॒ । आ॒न॒ट् । विष्णोः᳚ । क्रमः॑ । अ॒सि॒ । विष्णोः᳚ । क्रा॒न्तम् । अ॒सि॒ । विष्णोः᳚ । विक्रा᳚न्त॒मिति॒ वि - क्रा॒न्त॒म् । अ॒सि॒ ॥(प॒शू॒नां-ॅव्राताः॒-पञ्च॑विꣳशतिश्च) (आ10 )  \newline



\textbf{Jatai Paata} \newline

1. बृह॒स्पति॑र् वा॒चां ॅवा॒चाम् बृह॒स्पति॒र् बृह॒स्पति॑र् वा॒चाम् । \newline
2. वा॒चा मिन्द्र॒ इन्द्रो॑ वा॒चां ॅवा॒चा मिन्द्रः॑ । \newline
3. इन्द्रो᳚ ज्ये॒ष्ठाना᳚म् ज्ये॒ष्ठाना॒ मिन्द्र॒ इन्द्रो᳚ ज्ये॒ष्ठाना᳚म् । \newline
4. ज्ये॒ष्ठाना᳚म् मि॒त्रो मि॒त्रो ज्ये॒ष्ठाना᳚म् ज्ये॒ष्ठाना᳚म् मि॒त्रः । \newline
5. मि॒त्रः स॒त्यानाꣳ॑ स॒त्याना᳚म् मि॒त्रो मि॒त्रः स॒त्याना᳚म् । \newline
6. स॒त्यानां॒ ॅवरु॑णो॒ वरु॑णः स॒त्यानाꣳ॑ स॒त्यानां॒ ॅवरु॑णः । \newline
7. वरु॑णो॒ धर्म॑पतीना॒म् धर्म॑पतीनां॒ ॅवरु॑णो॒ वरु॑णो॒ धर्म॑पतीनाम् । \newline
8. धर्म॑पतीनां॒ ॅये ये धर्म॑पतीना॒म् धर्म॑पतीनां॒ ॅये । \newline
9. धर्म॑पतीना॒मिति॒ धर्म॑ - प॒ती॒ना॒म् । \newline
10. ये दे॑वा देवा॒ ये ये दे॑वाः । \newline
11. दे॒वा॒ दे॒व॒सुवो॑ देव॒सुवो॑ देवा देवा देव॒सुवः॑ । \newline
12. दे॒व॒सुवः॒ स्थ स्थ दे॑व॒सुवो॑ देव॒सुवः॒ स्थ । \newline
13. दे॒व॒सुव॒ इति॑ देव - सुवः॑ । \newline
14. स्थ ते ते स्थ स्थ ते । \newline
15. त इ॒म मि॒मम् ते त इ॒मम् । \newline
16. इ॒म मा॑मुष्याय॒ण मा॑मुष्याय॒ण मि॒म मि॒म मा॑मुष्याय॒णम् । \newline
17. आ॒मु॒ष्या॒य॒ण म॑नमि॒त्राया॑ नमि॒त्राया॑ मुष्याय॒ण मा॑मुष्याय॒ण म॑नमि॒त्राय॑ । \newline
18. अ॒न॒मि॒त्राय॑ सुवद्ध्वꣳ सुवद्ध्व मनमि॒त्राया॑ नमि॒त्राय॑ सुवद्ध्वम् । \newline
19. सु॒व॒द्ध्व॒म् म॒ह॒ते म॑ह॒ते सु॑वद्ध्वꣳ सुवद्ध्वम् मह॒ते । \newline
20. म॒ह॒ते क्ष॒त्राय॑ क्ष॒त्राय॑ मह॒ते म॑ह॒ते क्ष॒त्राय॑ । \newline
21. क्ष॒त्राय॑ मह॒ते म॑ह॒ते क्ष॒त्राय॑ क्ष॒त्राय॑ मह॒ते । \newline
22. म॒ह॒त आधि॑पत्या॒या धि॑पत्याय मह॒ते म॑ह॒त आधि॑पत्याय । \newline
23. आधि॑पत्याय मह॒ते म॑ह॒त आधि॑पत्या॒या धि॑पत्याय मह॒ते । \newline
24. आधि॑पत्या॒येत्याधि॑ - प॒त्या॒य॒ । \newline
25. म॒ह॒ते जान॑राज्याय॒ जान॑राज्याय मह॒ते म॑ह॒ते जान॑राज्याय । \newline
26. जान॑राज्या यै॒ष ए॒ष जान॑राज्याय॒ जान॑राज्या यै॒षः । \newline
27. जान॑राज्या॒येति॒ जान॑ - रा॒ज्या॒य॒ । \newline
28. ए॒ष वो॑ व ए॒ष ए॒ष वः॑ । \newline
29. वो॒ भ॒र॒ता॒ भ॒र॒ता॒ वो॒ वो॒ भ॒र॒ताः॒ । \newline
30. भ॒र॒ता॒ राजा॒ राजा॑ भरता भरता॒ राजा᳚ । \newline
31. राजा॒ सोमः॒ सोमो॒ राजा॒ राजा॒ सोमः॑ । \newline
32. सोमो॒ ऽस्माक॑ म॒स्माकꣳ॒॒ सोमः॒ सोमो॒ ऽस्माक᳚म् । \newline
33. अ॒स्माक॑म् ब्राह्म॒णाना᳚म् ब्राह्म॒णाना॑ म॒स्माक॑ म॒स्माक॑म् ब्राह्म॒णाना᳚म् । \newline
34. ब्रा॒ह्म॒णानाꣳ॒॒ राजा॒ राजा᳚ ब्राह्म॒णाना᳚म् ब्राह्म॒णानाꣳ॒॒ राजा᳚ । \newline
35. राजा॒ प्रति॒ प्रति॒ राजा॒ राजा॒ प्रति॑ । \newline
36. प्रति॒ त्यत् त्यत् प्रति॒ प्रति॒ त्यत् । \newline
37. त्यन् नाम॒ नाम॒ त्यत् त्यन् नाम॑ । \newline
38. नाम॑ रा॒ज्यꣳ रा॒ज्यन्नाम॒ नाम॑ रा॒ज्यम् । \newline
39. रा॒ज्य म॑धाय्यधायि रा॒ज्यꣳ रा॒ज्य म॑धायि । \newline
40. अ॒धा॒यि॒ स्वाꣳ स्वा म॑धाय्यधायि॒ स्वाम् । \newline
41. स्वाम् त॒नुव॑म् त॒नुवꣳ॒॒ स्वाꣳ स्वाम् त॒नुव᳚म् । \newline
42. त॒नुवं॒ ॅवरु॑णो॒ वरु॑ण स्त॒नुव॑म् त॒नुवं॒ ॅवरु॑णः । \newline
43. वरु॑णो अशिश्रे दशिश्रे॒द् वरु॑णो॒ वरु॑णो अशिश्रेत् । \newline
44. अ॒शि॒श्रे॒च् छुचेः॒ शुचे॑ रशिश्रे दशिश्रे॒च् छुचेः᳚ । \newline
45. शुचे᳚र् मि॒त्रस्य॑ मि॒त्रस्य॒ शुचेः॒ शुचे᳚र् मि॒त्रस्य॑ । \newline
46. मि॒त्रस्य॒ व्रत्या॒ व्रत्या॑ मि॒त्रस्य॑ मि॒त्रस्य॒ व्रत्याः᳚ । \newline
47. व्रत्या॑ अभूमा भूम॒ व्रत्या॒ व्रत्या॑ अभूम । \newline
48. अ॒भू॒मा म॑न्म॒ह्यम॑न्म ह्यभूमा भू॒मा म॑न्महि । \newline
49. अम॑न्महि मह॒तो म॑ह॒तो ऽम॑न्म॒ ह्यम॑न्महि मह॒तः । \newline
50. म॒ह॒त ऋ॒तस्य॒ र्तस्य॑ मह॒तो म॑ह॒त ऋ॒तस्य॑ । \newline
51. ऋ॒तस्य॒ नाम॒ नाम॒ र्तस्य॒ र्तस्य॒ नाम॑ । \newline
52. नाम॒ सर्वे॒ सर्वे॒ नाम॒ नाम॒ सर्वे᳚ । \newline
53. सर्वे॒ व्राता॒ व्राताः॒ सर्वे॒ सर्वे॒ व्राताः᳚ । \newline
54. व्राता॒ वरु॑णस्य॒ वरु॑णस्य॒ व्राता॒ व्राता॒ वरु॑णस्य । \newline
55. वरु॑णस्याभूवन् नभूव॒न्॒. वरु॑णस्य॒ वरु॑णस्याभूवन्न् । \newline
56. अ॒भू॒व॒न्॒. वि व्य॑भूवन् नभूव॒न्॒. वि । \newline
57. वि मि॒त्रो मि॒त्रो वि वि मि॒त्रः । \newline
58. मि॒त्र एवै॒रेवै᳚र् मि॒त्रो मि॒त्र एवैः᳚ । \newline
59. एवै॒ ररा॑ति॒ मरा॑ति॒ मेवै॒ रेवै॒ ररा॑तिम् । \newline
60. अरा॑ति मतारी दतारी॒ दरा॑ति॒ मरा॑ति मतारीत् । \newline
61. अ॒ता॒री॒ दसू॑षुद॒न्ता सू॑षुदन्तातारी दतारी॒ दसू॑षुदन्त । \newline
62. असू॑षुदन्त य॒ज्ञिया॑ य॒ज्ञिया॒ असू॑षुद॒न्ता सू॑षुदन्त य॒ज्ञियाः᳚ । \newline
63. य॒ज्ञिया॑ ऋ॒तेन॒ र्तेन॑ य॒ज्ञिया॑ य॒ज्ञिया॑ ऋ॒तेन॑ । \newline
64. ऋ॒तेन॒ वि व्यृ॑तेन॒ र्तेन॒ वि । \newline
65. व्यु॑ वु॒ वि व्यु॑ । \newline
66. उ॒ त्रि॒त स्त्रि॒त उ॑ वु त्रि॒तः । \newline
67. त्रि॒तो ज॑रि॒माण॑म् जरि॒माण॑म् त्रि॒तस्त्रि॒तो ज॑रि॒माण᳚म् । \newline
68. ज॒रि॒माण॑म् नो नो जरि॒माण॑म् जरि॒माण॑म् नः । \newline
69. न॒ आ॒न॒ डा॒न॒ण् णो॒ न॒ आ॒न॒ट् । \newline
70. आ॒न॒ड् विष्णो॒र् विष्णो॑ रान डान॒ड् विष्णोः᳚ । \newline
71. विष्णोः॒ क्रमः॒ क्रमो॒ विष्णो॒र् विष्णोः॒ क्रमः॑ । \newline
72. क्रमो᳚ ऽस्यसि॒ क्रमः॒ क्रमो॑ ऽसि । \newline
73. अ॒सि॒ विष्णो॒र् विष्णो॑ रस्यसि॒ विष्णोः᳚ । \newline
74. विष्णोः᳚ क्रा॒न्तम् क्रा॒न्तं ॅविष्णो॒र् विष्णोः᳚ क्रा॒न्तम् । \newline
75. क्रा॒न्त म॑स्यसि क्रा॒न्तम् क्रा॒न्त म॑सि । \newline
76. अ॒सि॒ विष्णो॒र् विष्णो॑ रस्यसि॒ विष्णोः᳚ । \newline
77. विष्णो॒र् विक्रा᳚न्तं॒ ॅविक्रा᳚न्तं॒ ॅविष्णो॒र् विष्णो॒र् विक्रा᳚न्तम् । \newline
78. विक्रा᳚न्त मस्यसि॒ विक्रा᳚न्तं॒ ॅविक्रा᳚न्त मसि । \newline
79. विक्रा᳚न्त॒मिति॒ वि - क्रा॒न्त॒म् । \newline
80. अ॒सीत्य॑सि । \newline

\textbf{Ghana Paata } \newline

1. बृह॒स्पति॑र् वा॒चां ॅवा॒चाम् बृह॒स्पति॒र् बृह॒स्पति॑र् वा॒चा मिन्द्र॒ इन्द्रो॑ वा॒चाम् बृह॒स्पति॒र् बृह॒स्पति॑र् 
वा॒चा मिन्द्रः॑ । \newline
2. वा॒चा मिन्द्र॒ इन्द्रो॑ वा॒चां ॅवा॒चा मिन्द्रो᳚ ज्ये॒ष्ठाना᳚म् ज्ये॒ष्ठाना॒ मिन्द्रो॑ वा॒चां ॅवा॒चा मिन्द्रो᳚ ज्ये॒ष्ठाना᳚म् । \newline
3. इन्द्रो᳚ ज्ये॒ष्ठाना᳚म् ज्ये॒ष्ठाना॒ मिन्द्र॒ इन्द्रो᳚ ज्ये॒ष्ठाना᳚म् मि॒त्रो मि॒त्रो ज्ये॒ष्ठाना॒ मिन्द्र॒ इन्द्रो᳚ ज्ये॒ष्ठाना᳚म् मि॒त्रः । \newline
4. ज्ये॒ष्ठाना᳚म् मि॒त्रो मि॒त्रो ज्ये॒ष्ठाना᳚म् ज्ये॒ष्ठाना᳚म् मि॒त्रः स॒त्यानाꣳ॑ स॒त्याना᳚म् मि॒त्रो ज्ये॒ष्ठाना᳚म् ज्ये॒ष्ठाना᳚म् मि॒त्रः स॒त्याना᳚म् । \newline
5. मि॒त्रः स॒त्यानाꣳ॑ स॒त्याना᳚म् मि॒त्रो मि॒त्रः स॒त्यानां॒ ॅवरु॑णो॒ वरु॑णः स॒त्याना᳚म् मि॒त्रो मि॒त्रः स॒त्यानां॒ ॅवरु॑णः । \newline
6. स॒त्यानां॒ ॅवरु॑णो॒ वरु॑णः स॒त्यानाꣳ॑ स॒त्यानां॒ ॅवरु॑णो॒ धर्म॑पतीना॒म् धर्म॑पतीनां॒ ॅवरु॑णः स॒त्यानाꣳ॑ स॒त्यानां॒ ॅवरु॑णो॒ धर्म॑पतीनाम् । \newline
7. वरु॑णो॒ धर्म॑पतीना॒म् धर्म॑पतीनां॒ ॅवरु॑णो॒ वरु॑णो॒ धर्म॑पतीनां॒ ॅये ये धर्म॑पतीनां॒ ॅवरु॑णो॒ वरु॑णो॒ धर्म॑पतीनां॒ ॅये । \newline
8. धर्म॑पतीनां॒ ॅये ये धर्म॑पतीना॒म् धर्म॑पतीनां॒ ॅये दे॑वा देवा॒ ये धर्म॑पतीना॒म् धर्म॑पतीनां॒ ॅये दे॑वाः । \newline
9. धर्म॑पतीना॒मिति॒ धर्म॑ - प॒ती॒ना॒म् । \newline
10. ये दे॑वा देवा॒ ये ये दे॑वा देव॒सुवो॑ देव॒सुवो॑ देवा॒ ये ये दे॑वा देव॒सुवः॑ । \newline
11. दे॒वा॒ दे॒व॒सुवो॑ देव॒सुवो॑ देवा देवा देव॒सुवः॒ स्थ स्थ दे॑व॒सुवो॑ देवा देवा देव॒सुवः॒ स्थ । \newline
12. दे॒व॒सुवः॒ स्थ स्थ दे॑व॒सुवो॑ देव॒सुवः॒ स्थ ते ते स्थ दे॑व॒सुवो॑ देव॒सुवः॒ स्थ ते । \newline
13. दे॒व॒सुव॒ इति॑ देव - सुवः॑ । \newline
14. स्थ ते ते स्थ स्थ त इ॒म मि॒मम् ते स्थ स्थ त इ॒मम् । \newline
15. त इ॒म मि॒मम् ते त इ॒म मा॑मुष्याय॒ण मा॑मुष्याय॒ण मि॒मम् ते त इ॒म मा॑मुष्याय॒णम् । \newline
16. इ॒म मा॑मुष्याय॒ण मा॑मुष्याय॒ण मि॒म मि॒म मा॑मुष्याय॒ण म॑नमि॒त्राया॑ नमि॒त्राया॑ मुष्याय॒ण मि॒म मि॒म मा॑मुष्याय॒ण म॑नमि॒त्राय॑ । \newline
17. आ॒मु॒ष्या॒य॒ण म॑नमि॒त्राया॑ नमि॒त्राया॑ मुष्याय॒ण मा॑मुष्याय॒ण म॑नमि॒त्राय॑ सुवद्ध्वꣳ सुवद्ध्व मनमि॒त्राया॑ मुष्याय॒ण मा॑मुष्याय॒ण म॑नमि॒त्राय॑ सुवद्ध्वम् । \newline
18. अ॒न॒मि॒त्राय॑ सुवद्ध्वꣳ सुवद्ध्व मनमि॒त्राया॑ नमि॒त्राय॑ सुवद्ध्वम् मह॒ते म॑ह॒ते सु॑वद्ध्व मनमि॒त्राया॑ नमि॒त्राय॑ सुवद्ध्वम् मह॒ते । \newline
19. सु॒व॒द्ध्व॒म् म॒ह॒ते म॑ह॒ते सु॑वद्ध्वꣳ सुवद्ध्वम् मह॒ते क्ष॒त्राय॑ क्ष॒त्राय॑ मह॒ते सु॑वद्ध्वꣳ सुवद्ध्वम् मह॒ते क्ष॒त्राय॑ । \newline
20. म॒ह॒ते क्ष॒त्राय॑ क्ष॒त्राय॑ मह॒ते म॑ह॒ते क्ष॒त्राय॑ मह॒ते म॑ह॒ते क्ष॒त्राय॑ मह॒ते म॑ह॒ते क्ष॒त्राय॑ मह॒ते । \newline
21. क्ष॒त्राय॑ मह॒ते म॑ह॒ते क्ष॒त्राय॑ क्ष॒त्राय॑ मह॒त आधि॑पत्या॒या धि॑पत्याय मह॒ते क्ष॒त्राय॑ क्ष॒त्राय॑ मह॒त आधि॑पत्याय । \newline
22. म॒ह॒त आधि॑पत्या॒या धि॑पत्याय मह॒ते म॑ह॒त आधि॑पत्याय मह॒ते म॑ह॒त आधि॑पत्याय मह॒ते म॑ह॒त आधि॑पत्याय मह॒ते । \newline
23. आधि॑पत्याय मह॒ते म॑ह॒त आधि॑पत्या॒या धि॑पत्याय मह॒ते जान॑राज्याय॒ जान॑राज्याय मह॒त आधि॑पत्या॒या धि॑पत्याय मह॒ते जान॑राज्याय । \newline
24. आधि॑पत्या॒येत्याधि॑ - प॒त्या॒य॒ । \newline
25. म॒ह॒ते जान॑राज्याय॒ जान॑राज्याय मह॒ते म॑ह॒ते जान॑राज्यायै॒ष ए॒ष जान॑राज्याय मह॒ते म॑ह॒ते जान॑राज्यायै॒षः । \newline
26. जान॑राज्या यै॒ष ए॒ष जान॑राज्याय॒ जान॑राज्या यै॒ष वो॑ व ए॒ष जान॑राज्याय॒ जान॑राज्या यै॒ष वः॑ । \newline
27. जान॑राज्या॒येति॒ जान॑ - रा॒ज्या॒य॒ । \newline
28. ए॒ष वो॑ व ए॒ष ए॒ष वो॑ भरता भरता व ए॒ष ए॒ष वो॑ भरताः । \newline
29. वो॒ भ॒र॒ता॒ भ॒र॒ता॒ वो॒ वो॒ भ॒र॒ता॒ राजा॒ राजा॑ भरता वो वो भरता॒ राजा᳚ । \newline
30. भ॒र॒ता॒ राजा॒ राजा॑ भरता भरता॒ राजा॒ सोमः॒ सोमो॒ राजा॑ भरता भरता॒ राजा॒ सोमः॑ । \newline
31. राजा॒ सोमः॒ सोमो॒ राजा॒ राजा॒ सोमो॒ ऽस्माक॑ म॒स्माकꣳ॒॒ सोमो॒ राजा॒ राजा॒ सोमो॒ ऽस्माक᳚म् । \newline
32. सोमो॒ ऽस्माक॑ म॒स्माकꣳ॒॒ सोमः॒ सोमो॒ ऽस्माक॑म् ब्राह्म॒णाना᳚म् ब्राह्म॒णाना॑ म॒स्माकꣳ॒॒ सोमः॒ सोमो॒ ऽस्माक॑म् ब्राह्म॒णाना᳚म् । \newline
33. अ॒स्माक॑म् ब्राह्म॒णाना᳚म् ब्राह्म॒णाना॑ म॒स्माक॑ म॒स्माक॑म् ब्राह्म॒णानाꣳ॒॒ राजा॒ राजा᳚ ब्राह्म॒णाना॑ म॒स्माक॑ म॒स्माक॑म् ब्राह्म॒णानाꣳ॒॒ राजा᳚ । \newline
34. ब्रा॒ह्म॒णानाꣳ॒॒ राजा॒ राजा᳚ ब्राह्म॒णाना᳚म् ब्राह्म॒णानाꣳ॒॒ राजा॒ प्रति॒ प्रति॒ राजा᳚ ब्राह्म॒णाना᳚म् ब्राह्म॒णानाꣳ॒॒ राजा॒ प्रति॑ । \newline
35. राजा॒ प्रति॒ प्रति॒ राजा॒ राजा॒ प्रति॒ त्यत् त्यत् प्रति॒ राजा॒ राजा॒ प्रति॒ त्यत् । \newline
36. प्रति॒ त्यत् त्यत् प्रति॒ प्रति॒ त्यन् नाम॒ नाम॒ त्यत् प्रति॒ प्रति॒ त्यन् नाम॑ । \newline
37. त्यन् नाम॒ नाम॒ त्यत् त्यन् नाम॑ रा॒ज्यꣳ रा॒ज्यन्नाम॒ त्यत् त्यन् नाम॑ रा॒ज्यम् । \newline
38. नाम॑ रा॒ज्यꣳ रा॒ज्यन्नाम॒ नाम॑ रा॒ज्य म॑धा य्यधायि रा॒ज्यन्नाम॒ नाम॑ रा॒ज्य म॑धायि । \newline
39. रा॒ज्य म॑धा य्यधायि रा॒ज्यꣳ रा॒ज्य म॑धायि॒ स्वाꣳ स्वा म॑धायि रा॒ज्यꣳ रा॒ज्य म॑धायि॒ स्वाम् । \newline
40. अ॒धा॒यि॒ स्वाꣳ स्वा म॑धा य्यधायि॒ स्वाम् त॒नुव॑म् त॒नुवꣳ॒॒ स्वा म॑धा य्यधायि॒ स्वाम् त॒नुव᳚म् । \newline
41. स्वाम् त॒नुव॑म् त॒नुवꣳ॒॒ स्वाꣳ स्वाम् त॒नुवं॒ ॅवरु॑णो॒ वरु॑ण स्त॒नुवꣳ॒॒ स्वाꣳ स्वाम् त॒नुवं॒ ॅवरु॑णः । \newline
42. त॒नुवं॒ ॅवरु॑णो॒ वरु॑ण स्त॒नुव॑म् त॒नुवं॒ ॅवरु॑णो अशिश्रे दशिश्रे॒द् वरु॑ण स्त॒नुव॑म् त॒नुवं॒ ॅवरु॑णो अशिश्रेत् । \newline
43. वरु॑णो अशिश्रे दशिश्रे॒द् वरु॑णो॒ वरु॑णो अशिश्रे॒च् छुचेः॒ शुचे॑ रशिश्रे॒द् वरु॑णो॒ वरु॑णो अशिश्रे॒च् छुचेः᳚ । \newline
44. अ॒शि॒श्रे॒च् छुचेः॒ शुचे॑ रशिश्रे दशिश्रे॒च् छुचे᳚र् मि॒त्रस्य॑ मि॒त्रस्य॒ शुचे॑ रशिश्रे दशिश्रे॒च् छुचे᳚र् मि॒त्रस्य॑ । \newline
45. शुचे᳚र् मि॒त्रस्य॑ मि॒त्रस्य॒ शुचेः॒ शुचे᳚र् मि॒त्रस्य॒ व्रत्या॒ व्रत्या॑ मि॒त्रस्य॒ शुचेः॒ शुचे᳚र् मि॒त्रस्य॒ व्रत्याः᳚ । \newline
46. मि॒त्रस्य॒ व्रत्या॒ व्रत्या॑ मि॒त्रस्य॑ मि॒त्रस्य॒ व्रत्या॑ अभूमा भूम॒ व्रत्या॑ मि॒त्रस्य॑ मि॒त्रस्य॒ व्रत्या॑ अभूम । \newline
47. व्रत्या॑ अभूमा भूम॒ व्रत्या॒ व्रत्या॑ अभू॒मा म॑न्म॒ ह्यम॑न्म ह्यभूम॒ व्रत्या॒ व्रत्या॑ अभू॒मा म॑न्महि । \newline
48. अ॒भू॒मा म॑न्म॒ ह्यम॑न्मह्य भूमाभू॒मा म॑न्महि मह॒तो म॑ह॒तो ऽम॑न्मह्य भूमाभू॒मा म॑न्महि मह॒तः । \newline
49. अम॑न्महि मह॒तो म॑ह॒तो ऽम॑न्म॒ह्य म॑न्महि मह॒त ऋ॒तस्य॒ र्तस्य॑ मह॒तो ऽम॑न्म॒ ह्यम॑न्महि मह॒त ऋ॒तस्य॑ । \newline
50. म॒ह॒त ऋ॒तस्य॒ र्तस्य॑ मह॒तो म॑ह॒त ऋ॒तस्य॒ नाम॒ नाम॒ र्तस्य॑ मह॒तो म॑ह॒त ऋ॒तस्य॒ नाम॑ । \newline
51. ऋ॒तस्य॒ नाम॒ नाम॒ र्तस्य॒ र्तस्य॒ नाम॒ सर्वे॒ सर्वे॒ नाम॒ र्तस्य॒ र्तस्य॒ नाम॒ सर्वे᳚ । \newline
52. नाम॒ सर्वे॒ सर्वे॒ नाम॒ नाम॒ सर्वे॒ व्राता॒ व्राताः॒ सर्वे॒ नाम॒ नाम॒ सर्वे॒ व्राताः᳚ । \newline
53. सर्वे॒ व्राता॒ व्राताः॒ सर्वे॒ सर्वे॒ व्राता॒ वरु॑णस्य॒ वरु॑णस्य॒ व्राताः॒ सर्वे॒ सर्वे॒ व्राता॒ वरु॑णस्य । \newline
54. व्राता॒ वरु॑णस्य॒ वरु॑णस्य॒ व्राता॒ व्राता॒ वरु॑णस्याभूवन् नभूव॒न्॒. वरु॑णस्य॒ व्राता॒ व्राता॒ वरु॑णस्याभूवन्न् । \newline
55. वरु॑णस्याभूवन् नभूव॒न्॒. वरु॑णस्य॒ वरु॑णस्याभूव॒न्॒. वि व्य॑भूव॒न्॒. वरु॑णस्य॒ वरु॑णस्याभूव॒न्॒. वि । \newline
56. अ॒भू॒व॒न्॒. वि व्य॑भूवन् नभूव॒न्॒. वि मि॒त्रो मि॒त्रो व्य॑भूवन् नभूव॒न्॒. वि मि॒त्रः । \newline
57. वि मि॒त्रो मि॒त्रो वि वि मि॒त्र एवै॒ रेवै᳚र् मि॒त्रो वि वि मि॒त्र एवैः᳚ । \newline
58. मि॒त्र एवै॒रेवै᳚र् मि॒त्रो मि॒त्र एवै॒ ररा॑ति॒ मरा॑ति॒ मेवै᳚र् मि॒त्रो मि॒त्र एवै॒ ररा॑तिम् । \newline
59. एवै॒ ररा॑ति॒ मरा॑ति॒ मेवै॒ रेवै॒ ररा॑ति मतारी दतारी॒ दरा॑ति॒ मेवै॒ रेवै॒ ररा॑ति मतारीत् । \newline
60. अरा॑ति मतारी दतारी॒ दरा॑ति॒ मरा॑ति मतारी॒ दसू॑षुद॒न्ता सू॑षुदन्ता तारी॒ दरा॑ति॒ मरा॑ति मतारी॒ दसू॑षुदन्त । \newline
61. अ॒ता॒री॒ दसू॑षुद॒न्ता सू॑षुदन्ता तारीदतारी॒ दसू॑षुदन्त य॒ज्ञिया॑ य॒ज्ञिया॒ असू॑षुदन्ता तारी दतारी॒ दसू॑षुदन्त य॒ज्ञियाः᳚ । \newline
62. असू॑षुदन्त य॒ज्ञिया॑ य॒ज्ञिया॒ असू॑षुद॒न्ता सू॑षुदन्त य॒ज्ञिया॑ ऋ॒तेन॒ र्तेन॑ य॒ज्ञिया॒ असू॑षुद॒न्ता सू॑षुदन्त य॒ज्ञिया॑ ऋ॒तेन॑ । \newline
63. य॒ज्ञिया॑ ऋ॒तेन॒ र्तेन॑ य॒ज्ञिया॑ य॒ज्ञिया॑ ऋ॒तेन॒ वि व्यृ॑तेन॑ य॒ज्ञिया॑ य॒ज्ञिया॑ ऋ॒तेन॒ वि । \newline
64. ऋ॒तेन॒ वि व्यृ॑तेन॒ र्तेन॒ व्यु॑ वु व्यृ॒तेन॒ र्तेन॒ व्यु॑ । \newline
65. व्यु॑ वु॒ वि व्यु॑ त्रि॒त स्त्रि॒त उ॒ वि व्यु॑ त्रि॒तः । \newline
66. उ॒ त्रि॒त स्त्रि॒त उ॑ वु त्रि॒तो ज॑रि॒माण॑म् जरि॒माण॑म् त्रि॒त उ॑ वु त्रि॒तो ज॑रि॒माण᳚म् । \newline
67. त्रि॒तो ज॑रि॒माण॑म् जरि॒माण॑म् त्रि॒तस्त्रि॒तो ज॑रि॒माण॑न्नो नो जरि॒माण॑म् त्रि॒त स्त्रि॒तो ज॑रि॒माण॑न्नः । \newline
68. ज॒रि॒माण॑न्नो नो जरि॒माण॑म् जरि॒माण॑न्न आन डानण् णो जरि॒माण॑म् जरि॒माण॑न्न आनट् । \newline
69. न॒ आ॒न॒ डा॒न॒ण् णो॒ न॒ आ॒न॒ड् विष्णो॒र् विष्णो॑ रानण् णो न आन॒ड् विष्णोः᳚ । \newline
70. आ॒न॒ड् विष्णो॒र् विष्णो॑ रान डान॒ड् विष्णोः॒ क्रमः॒ क्रमो॒ विष्णो॑ रान डान॒ड् विष्णोः॒ क्रमः॑ । \newline
71. विष्णोः॒ क्रमः॒ क्रमो॒ विष्णो॒र् विष्णोः॒ क्रमो᳚ ऽस्यसि॒ क्रमो॒ विष्णो॒र् विष्णोः॒ क्रमो॑ ऽसि । \newline
72. क्रमो᳚ ऽस्यसि॒ क्रमः॒ क्रमो॑ ऽसि॒ विष्णो॒र् विष्णो॑ रसि॒ क्रमः॒ क्रमो॑ ऽसि॒ विष्णोः᳚ । \newline
73. अ॒सि॒ विष्णो॒र् विष्णो॑ रस्यसि॒ विष्णोः᳚ क्रा॒न्तम् क्रा॒न्तं ॅविष्णो॑ रस्यसि॒ विष्णोः᳚ क्रा॒न्तम् । \newline
74. विष्णोः᳚ क्रा॒न्तम् क्रा॒न्तं ॅविष्णो॒र् विष्णोः᳚ क्रा॒न्त म॑स्यसि क्रा॒न्तं ॅविष्णो॒र् विष्णोः᳚ क्रा॒न्त म॑सि । \newline
75. क्रा॒न्त म॑स्यसि क्रा॒न्तम् क्रा॒न्त म॑सि॒ विष्णो॒र् विष्णो॑रसि क्रा॒न्तम् क्रा॒न्त म॑सि॒ विष्णोः᳚ । \newline
76. अ॒सि॒ विष्णो॒र् विष्णो॑ रस्यसि॒ विष्णो॒र् विक्रा᳚न्तं॒ ॅविक्रा᳚न्तं॒ ॅविष्णो॑ रस्यसि॒ विष्णो॒र् विक्रा᳚न्तम् । \newline
77. विष्णो॒र् विक्रा᳚न्तं॒ ॅविक्रा᳚न्तं॒ ॅविष्णो॒र् विष्णो॒र् विक्रा᳚न्त मस्यसि॒ विक्रा᳚न्तं॒ ॅविष्णो॒र् विष्णो॒र् विक्रा᳚न्त मसि । \newline
78. विक्रा᳚न्त मस्यसि॒ विक्रा᳚न्तं॒ ॅविक्रा᳚न्त मसि । \newline
79. विक्रा᳚न्त॒मिति॒ वि - क्रा॒न्त॒म् । \newline
80. अ॒सीत्य॑सि । \newline
\pagebreak
\markright{ TS 1.8.11.1  \hfill https://www.vedavms.in \hfill}
\addcontentsline{toc}{section}{ TS 1.8.11.1 }
\section*{ TS 1.8.11.1 }

\textbf{TS 1.8.11.1 } \newline
\textbf{Samhita Paata} \newline

अ॒र्थेतः॑ स्था॒ऽपां पति॑रसि॒ वृषा᳚ऽस्यू॒र्मिर् वृ॑षसे॒नो॑ऽसि व्रज॒क्षितः॑ स्थ म॒रुता॒मोजः॑ स्थ॒ सूर्य॑वर्चसः स्थ॒ सूर्य॑त्वचसः स्थ॒ मान्दाः᳚ स्थ॒ वाशाः᳚ स्थ॒ शक्व॑रीः स्थ विश्व॒भृतः॑ स्थ जन॒भृतः॑ स्था॒ऽग्नेस्ते॑ज॒स्याः᳚ स्था॒ऽपामोष॑धीनाꣳ॒॒ रसः॑ स्था॒ऽपो दे॒वीर् मधु॑मतीरगृह्ण॒न्नूर्ज॑स्वती राज॒सूया॑य॒ चिता॑नाः ॥ याभि॑र् मि॒त्रावरु॑णाव॒-भ्यषि॑ञ्च॒न्॒. याभि॒-रिन्द्र॒मन॑य॒न्नत्य ( ) रा॑तीः ॥ रा॒ष्ट्र॒दाः स्थ॑ रा॒ष्ट्रं द॑त्त॒ स्वाहा॑ राष्ट्र॒दाः स्थ॑ रा॒ष्ट्रम॒मुष्मै॑ दत्त ॥ \newline

\textbf{Pada Paata} \newline

अ॒र्थेत॒ इत्य॑र्थ-इतः॑ । स्थ॒ । अ॒पाम् । पतिः॑ । अ॒सि॒ । वृषा᳚ । अ॒सि॒ । ऊ॒र्मिः । वृ॒ष॒से॒न इति॑ वृष-से॒नः । अ॒सि॒ । व्र॒ज॒क्षित॒ इति॑ व्रज-क्षितः॑ । स्थ॒ । म॒रुता᳚म् । ओजः॑ । स्थ॒ । सूर्य॑वर्चस॒ इति॒ सूर्य॑ - व॒र्च॒सः॒ । स्थ॒ । सूर्य॑त्वचस॒ इति॒ सूर्य॑ - त्व॒च॒सः॒ । स्थ॒ । मान्दाः᳚ । स्थ॒ । वाशाः᳚ । स्थ॒ । शक्व॑रीः । स्थ॒ । वि॒श्व॒भृत॒ इति॑ विश्व - भृतः॑ । स्थ॒ । ज॒न॒भृत॒ इति॑ जन- भृतः॑ । स्थ॒ । अ॒ग्नेः । ते॒ज॒स्याः᳚ । स्थ॒ । अ॒पाम् । ओष॑धीनाम् । रसः॑ । स्थ॒ । अ॒पः । दे॒वीः । मधु॑मती॒रिति॒ मधु॑ - म॒तीः॒ । अ॒गृ॒ह्ण॒न्न् । ऊर्ज॑स्वतीः । रा॒ज॒सूया॒येति॑ राज - सूया॑य । चिता॑नाः ॥ याभिः॑ । मि॒त्रावरु॑णा॒विति॑ मि॒त्रा-वरु॑णौ । अ॒भ्यषि॑ञ्च॒न्नित्य॑भि-असि॑ञ्चन्न् । याभिः॑ । इन्द्र᳚म् । अन॑यन्न् । अतीति॑ ( ) । अरा॑तीः ॥ रा॒ष्ट्र॒दा इति॑ राष्ट्र-दाः । स्थ॒ । रा॒ष्ट्रम् । द॒त्त॒ । स्वाहा᳚ । रा॒ष्ट्र॒दा इति॑ राष्ट्र - दाः । स्थ॒ । रा॒ष्ट्रम् । अ॒मुष्मै᳚ । द॒त्त॒ ॥  \newline



\textbf{Jatai Paata} \newline

1. अ॒र्थेतः॑ स्थ स्था॒र्थेतो॒ ऽर्थेतः॑ स्थ । \newline
2. अ॒र्थेत॒ इत्य॑र्थ - इतः॑ । \newline
3. स्था॒पा म॒पाꣳ स्थ॑ स्था॒पाम् । \newline
4. अ॒पाम् पति॒ष् पति॑ र॒पा म॒पाम् पतिः॑ । \newline
5. पति॑ रस्यसि॒ पति॒ष् पति॑ रसि । \newline
6. अ॒सि॒ वृषा॒ वृषा᳚ ऽस्यसि॒ वृषा᳚ । \newline
7. वृषा᳚ ऽस्यसि॒ वृषा॒ वृषा॑ ऽसि । \newline
8. अ॒स्यू॒र्मि रू॒र्मि र॑स्य स्यू॒र्मिः । \newline
9. ऊ॒र्मिर् वृ॑षसे॒नो वृ॑षसे॒न ऊ॒र्मि रू॒र्मिर् वृ॑षसे॒नः । \newline
10. वृ॒ष॒से॒नो᳚ ऽस्यसि वृषसे॒नो वृ॑षसे॒नो॑ ऽसि । \newline
11. वृ॒ष॒से॒न इति॑ वृष - से॒नः । \newline
12. अ॒सि॒ व्र॒ज॒क्षितो᳚ व्रज॒क्षितो᳚ ऽस्यसि व्रज॒क्षितः॑ । \newline
13. व्र॒ज॒क्षितः॑ स्थ स्थ व्रज॒क्षितो᳚ व्रज॒क्षितः॑ स्थ । \newline
14. व्र॒ज॒क्षित॒ इति॑ व्रज - क्षितः॑ । \newline
15. स्थ॒ म॒रुता᳚म् म॒रुताꣳ॑ स्थ स्थ म॒रुता᳚म् । \newline
16. म॒रुता॒ मोज॒ ओजो॑ म॒रुता᳚म् म॒रुता॒ मोजः॑ । \newline
17. ओजः॑ स्थ॒ स्थौज॒ ओजः॑ स्थ । \newline
18. स्थ॒ सूर्य॑वर्चसः॒ सूर्य॑वर्चसः स्थ स्थ॒ सूर्य॑वर्चसः । \newline
19. सूर्य॑वर्चसः स्थ स्थ॒ सूर्य॑वर्चसः॒ सूर्य॑वर्चसः स्थ । \newline
20. सूर्य॑वर्चस॒ इति॒ सूर्य॑ - व॒र्च॒सः॒ । \newline
21. स्थ॒ सूर्य॑त्वचसः॒ सूर्य॑त्वचसः स्थ स्थ॒ सूर्य॑त्वचसः । \newline
22. सूर्य॑त्वचसः स्थ स्थ॒ सूर्य॑त्वचसः॒ सूर्य॑त्वचसः स्थ । \newline
23. सूर्य॑त्वचस॒ इति॒ सूर्य॑ - त्व॒च॒सः॒ । \newline
24. स्थ॒ मान्दा॒ मान्दाः᳚ स्थ स्थ॒ मान्दाः᳚ । \newline
25. मान्दाः᳚ स्थ स्थ॒ मान्दा॒ मान्दाः᳚ स्थ । \newline
26. स्थ॒ वाशा॒ वाशाः᳚ स्थ स्थ॒ वाशाः᳚ । \newline
27. वाशाः᳚ स्थ स्थ॒ वाशा॒ वाशाः᳚ स्थ । \newline
28. स्थ॒ शक्व॑रीः॒ शक्व॑रीः स्थ स्थ॒ शक्व॑रीः । \newline
29. शक्व॑रीः स्थ स्थ॒ शक्व॑रीः॒ शक्व॑रीः स्थ । \newline
30. स्थ॒ वि॒श्व॒भृतो॑ विश्व॒भृतः॑ स्थ स्थ विश्व॒भृतः॑ । \newline
31. वि॒श्व॒भृतः॑ स्थ स्थ विश्व॒भृतो॑ विश्व॒भृतः॑ स्थ । \newline
32. वि॒श्व॒भृत॒ इति॑ विश्व - भृतः॑ । \newline
33. स्थ॒ ज॒न॒भृतो॑ जन॒भृतः॑ स्थ स्थ जन॒भृतः॑ । \newline
34. ज॒न॒भृतः॑ स्थ स्थ जन॒भृतो॑ जन॒भृतः॑ स्थ । \newline
35. ज॒न॒भृत॒ इति॑ जन - भृतः॑ । \newline
36. स्था॒ग्ने र॒ग्नेः स्थ॑ स्था॒ग्नेः । \newline
37. अ॒ग्ने स्ते॑ज॒स्या᳚ स्तेज॒स्या॑ अ॒ग्ने र॒ग्ने स्ते॑ज॒स्याः᳚ । \newline
38. ते॒ज॒स्याः᳚ स्थ स्थ तेज॒स्या᳚ स्तेज॒स्याः᳚ स्थ । \newline
39. स्था॒पा म॒पाꣳ स्थ॑ स्था॒पाम् । \newline
40. अ॒पा मोष॑धीना॒ मोष॑धीना म॒पा म॒पा मोष॑धीनाम् । \newline
41. ओष॑धीनाꣳ॒॒ रसो॒ रस॒ ओष॑धीना॒ मोष॑धीनाꣳ॒॒ रसः॑ । \newline
42. रसः॑ स्थ स्थ॒ रसो॒ रसः॑ स्थ । \newline
43. स्था॒पो॑ ऽपः स्थ॑ स्था॒पः । \newline
44. अ॒पो दे॒वीर् दे॒वी र॒पो॑ ऽपो दे॒वीः । \newline
45. दे॒वीर् मधु॑मती॒र् मधु॑मतीर् दे॒वीर् दे॒वीर् मधु॑मतीः । \newline
46. मधु॑मती रगृह्णन् नगृह्ण॒न् मधु॑मती॒र् मधु॑मती रगृह्णन्न् । \newline
47. मधु॑मती॒रिति॒ मधु॑ - म॒तीः॒ । \newline
48. अ॒गृ॒ह्ण॒न् नूर्ज॑स्वती॒ रूर्ज॑स्वती रगृह्णन् नगृह्ण॒न् नूर्ज॑स्वतीः । \newline
49. ऊर्ज॑स्वती राज॒सूया॑य राज॒सूया॒ योर्ज॑स्वती॒ रूर्ज॑स्वती राज॒सूया॑य । \newline
50. रा॒ज॒सूया॑य॒ चिता॑ना॒ श्चिता॑ना राज॒सूया॑य राज॒सूया॑य॒ चिता॑नाः । \newline
51. रा॒ज॒सूया॒येति॑ राज - सूया॑य । \newline
52. चिता॑ना॒ इति॒ चिता॑नाः । \newline
53. याभि॑र् मि॒त्रावरु॑णौ मि॒त्रावरु॑णौ॒ याभि॒र् याभि॑र् मि॒त्रावरु॑णौ । \newline
54. मि॒त्रावरु॑णा व॒भ्यषि॑ञ्चन् न॒भ्यषि॑ञ्चन् मि॒त्रावरु॑णौ मि॒त्रावरु॑णा व॒भ्यषि॑ञ्चन्न् । \newline
55. मि॒त्रावरु॑णा॒विति॑ मि॒त्रा - वरु॑णौ । \newline
56. अ॒भ्यषि॑ञ्च॒न्॒. याभि॒र् याभि॑र॒भ्यषि॑ञ्चन् न॒भ्यषि॑ञ्च॒न्॒. याभिः॑ । \newline
57. अ॒भ्यषि॑ञ्च॒न्नित्य॑भि - असि॑ञ्चन्न् । \newline
58. याभि॒ रिन्द्र॒ मिन्द्रं॒ ॅयाभि॒र् याभि॒ रिन्द्र᳚म् । \newline
59. इन्द्र॒ मन॑य॒न् नन॑य॒न् निन्द्र॒ मिन्द्र॒ मन॑यन्न् । \newline
60. अन॑य॒न् नत्य त्यन॑य॒न् नन॑य॒न् नति॑ । \newline
61. अत्यरा॑ती॒र रा॑ती॒ रत्य त्यरा॑तीः । \newline
62. अरा॑ती॒रित्यरा॑तीः । \newline
63. रा॒ष्ट्र॒दाः स्थ॑ स्थ राष्ट्र॒दा रा᳚ष्ट्र॒दाः स्थ॑ । \newline
64. रा॒ष्ट्र॒दा इति॑ राष्ट्र - दाः । \newline
65. स्थ॒ रा॒ष्ट्रꣳ रा॒ष्ट्रꣳ स्थ॑ स्थ रा॒ष्ट्रम् । \newline
66. रा॒ष्ट्रम् द॑त्त दत्त रा॒ष्ट्रꣳ रा॒ष्ट्रम् द॑त्त । \newline
67. द॒त्त॒ स्वाहा॒ स्वाहा॑ दत्त दत्त॒ स्वाहा᳚ । \newline
68. स्वाहा॑ राष्ट्र॒दा रा᳚ष्ट्र॒दाः स्वाहा॒ स्वाहा॑ राष्ट्र॒दाः । \newline
69. रा॒ष्ट्र॒दाः स्थ॑ स्थ राष्ट्र॒दा रा᳚ष्ट्र॒दाः स्थ॑ । \newline
70. रा॒ष्ट्र॒दा इति॑ राष्ट्र - दाः । \newline
71. स्थ॒ रा॒ष्ट्रꣳ रा॒ष्ट्रꣳ स्थ॑ स्थ रा॒ष्ट्रम् । \newline
72. रा॒ष्ट्र म॒मुष्मा॑ अ॒मुष्मै॑ रा॒ष्ट्रꣳ रा॒ष्ट्र म॒मुष्मै᳚ । \newline
73. अ॒मुष्मै॑ दत्त दत्ता॒मुष्मा॑ अ॒मुष्मै॑ दत्त । \newline
74. द॒त्तेति॑ दत्त । \newline

\textbf{Ghana Paata } \newline

1. अ॒र्थेतः॑ स्थ स्था॒र्थेतो॒ ऽर्थेतः॑ स्था॒पा म॒पाꣳ स्था॒र्थेतो॒ ऽर्थेतः॑ स्था॒पाम् । \newline
2. अ॒र्थेत॒ इत्य॑र्थ - इतः॑ । \newline
3. स्था॒पा म॒पाꣳ स्थ॑ स्था॒पाम् पति॒ष् पति॑ र॒पाꣳ स्थ॑ स्था॒पाम् पतिः॑ । \newline
4. अ॒पाम् पति॒ष् पति॑ र॒पा म॒पाम् पति॑ रस्यसि॒ पति॑ र॒पा म॒पाम् पति॑रसि । \newline
5. पति॑ रस्यसि॒ पति॒ष् पति॑ रसि॒ वृषा॒ वृषा॑ ऽसि॒ पति॒ष् पति॑ रसि॒ वृषा᳚ । \newline
6. अ॒सि॒ वृषा॒ वृषा᳚ ऽस्यसि॒ वृषा᳚ ऽस्यसि॒ वृषा᳚ ऽस्यसि॒ वृषा॑ ऽसि । \newline
7. वृषा᳚ ऽस्यसि॒ वृषा॒ वृषा᳚ ऽस्यू॒र्मि रू॒र्मि र॑सि॒ वृषा॒ वृषा᳚ ऽस्यू॒र्मिः । \newline
8. अ॒स्यू॒र्मि रू॒र्मि र॑स्य स्यू॒र्मिर् वृ॑षसे॒नो वृ॑षसे॒न ऊ॒र्मि र॑स्य स्यू॒र्मिर् वृ॑षसे॒नः । \newline
9. ऊ॒र्मिर् वृ॑षसे॒नो वृ॑षसे॒न ऊ॒र्मिर् ऊ॒र्मिर् वृ॑षसे॒नो᳚ ऽस्यसि वृषसे॒न ऊ॒र्मि रू॒र्मिर् वृ॑षसे॒नो॑ ऽसि । \newline
10. वृ॒ष॒से॒नो᳚ ऽस्यसि वृषसे॒नो वृ॑षसे॒नो॑ ऽसि व्रज॒क्षितो᳚ व्रज॒क्षितो॑ ऽसि वृषसे॒नो वृ॑षसे॒नो॑ ऽसि व्रज॒क्षितः॑ । \newline
11. वृ॒ष॒से॒न इति॑ वृष - से॒नः । \newline
12. अ॒सि॒ व्र॒ज॒क्षितो᳚ व्रज॒क्षितो᳚ ऽस्यसि व्रज॒क्षितः॑ स्थ स्थ व्रज॒क्षितो᳚ ऽस्यसि व्रज॒क्षितः॑ स्थ । \newline
13. व्र॒ज॒क्षितः॑ स्थ स्थ व्रज॒क्षितो᳚ व्रज॒क्षितः॑ स्थ म॒रुता᳚म् म॒रुताꣳ॑ स्थ व्रज॒क्षितो᳚ व्रज॒क्षितः॑ स्थ म॒रुता᳚म् । \newline
14. व्र॒ज॒क्षित॒ इति॑ व्रज - क्षितः॑ । \newline
15. स्थ॒ म॒रुता᳚म् म॒रुताꣳ॑ स्थ स्थ म॒रुता॒ मोज॒ ओजो॑ म॒रुताꣳ॑ स्थ स्थ म॒रुता॒ मोजः॑ । \newline
16. म॒रुता॒ मोज॒ ओजो॑ म॒रुता᳚म् म॒रुता॒ मोजः॑ स्थ॒ स्थौजो॑ म॒रुता᳚म् म॒रुता॒ मोजः॑ स्थ । \newline
17. ओजः॑ स्थ॒ स्थौज॒ ओजः॑ स्थ॒ सूर्य॑वर्चसः॒ सूर्य॑वर्चसः॒ स्थौज॒ ओजः॑ स्थ॒ सूर्य॑वर्चसः । \newline
18. स्थ॒ सूर्य॑वर्चसः॒ सूर्य॑वर्चसः स्थ स्थ॒ सूर्य॑वर्चसः स्थ स्थ॒ सूर्य॑वर्चसः स्थ स्थ॒ सूर्य॑वर्चसः स्थ । \newline
19. सूर्य॑वर्चसः स्थ स्थ॒ सूर्य॑वर्चसः॒ सूर्य॑वर्चसः स्थ॒ सूर्य॑त्वचसः॒ सूर्य॑त्वचसः स्थ॒ सूर्य॑वर्चसः॒ सूर्य॑वर्चसः स्थ॒ सूर्य॑त्वचसः । \newline
20. सूर्य॑वर्चस॒ इति॒ सूर्य॑ - व॒र्च॒सः॒ । \newline
21. स्थ॒ सूर्य॑त्वचसः॒ सूर्य॑त्वचसः स्थ स्थ॒ सूर्य॑त्वचसः स्थ स्थ॒ सूर्य॑त्वचसः स्थ स्थ॒ सूर्य॑त्वचसः स्थ । \newline
22. सूर्य॑त्वचसः स्थ स्थ॒ सूर्य॑त्वचसः॒ सूर्य॑त्वचसः स्थ॒ मान्दा॒ मान्दाः᳚ स्थ॒ सूर्य॑त्वचसः॒ सूर्य॑त्वचसः स्थ॒ मान्दाः᳚ । \newline
23. सूर्य॑त्वचस॒ इति॒ सूर्य॑ - त्व॒च॒सः॒ । \newline
24. स्थ॒ मान्दा॒ मान्दाः᳚ स्थ स्थ॒ मान्दाः᳚ स्थ स्थ॒ मान्दाः᳚ स्थ स्थ॒ मान्दाः᳚ स्थ । \newline
25. मान्दाः᳚ स्थ स्थ॒ मान्दा॒ मान्दाः᳚ स्थ॒ वाशा॒ वाशाः᳚ स्थ॒ मान्दा॒ मान्दाः᳚ स्थ॒ वाशाः᳚ । \newline
26. स्थ॒ वाशा॒ वाशाः᳚ स्थ स्थ॒ वाशाः᳚ स्थ स्थ॒ वाशाः᳚ स्थ स्थ॒ वाशाः᳚ स्थ । \newline
27. वाशाः᳚ स्थ स्थ॒ वाशा॒ वाशाः᳚ स्थ॒ शक्व॑रीः॒ शक्व॑रीः स्थ॒ वाशा॒ वाशाः᳚ स्थ॒ शक्व॑रीः । \newline
28. स्थ॒ शक्व॑रीः॒ शक्व॑रीः स्थ स्थ॒ शक्व॑रीः स्थ स्थ॒ शक्व॑रीः स्थ स्थ॒ शक्व॑रीः स्थ । \newline
29. शक्व॑रीः स्थ स्थ॒ शक्व॑रीः॒ शक्व॑रीः स्थ विश्व॒भृतो॑ विश्व॒भृतः॑ स्थ॒ शक्व॑रीः॒ शक्व॑रीः स्थ विश्व॒भृतः॑ । \newline
30. स्थ॒ वि॒श्व॒भृतो॑ विश्व॒भृतः॑ स्थ स्थ विश्व॒भृतः॑ स्थ स्थ विश्व॒भृतः॑ स्थ स्थ विश्व॒भृतः॑ स्थ । \newline
31. वि॒श्व॒भृतः॑ स्थ स्थ विश्व॒भृतो॑ विश्व॒भृतः॑ स्थ जन॒भृतो॑ जन॒भृतः॑ स्थ विश्व॒भृतो॑ विश्व॒भृतः॑ स्थ जन॒भृतः॑ । \newline
32. वि॒श्व॒भृत॒ इति॑ विश्व - भृतः॑ । \newline
33. स्थ॒ ज॒न॒भृतो॑ जन॒भृतः॑ स्थ स्थ जन॒भृतः॑ स्थ स्थ जन॒भृतः॑ स्थ स्थ जन॒भृतः॑ स्थ । \newline
34. ज॒न॒भृतः॑ स्थ स्थ जन॒भृतो॑ जन॒भृतः॑ स्था॒ग्ने र॒ग्नेः स्थ॑ जन॒भृतो॑ जन॒भृतः॑ स्था॒ग्नेः । \newline
35. ज॒न॒भृत॒ इति॑ जन - भृतः॑ । \newline
36. स्था॒ग्ने र॒ग्नेः स्थ॑ स्था॒ग्ने स्ते॑ज॒स्या᳚ स्तेज॒स्या॑ अ॒ग्नेः स्थ॑ स्था॒ग्ने स्ते॑ज॒स्याः᳚ । \newline
37. अ॒ग्ने स्ते॑ज॒स्या᳚ स्तेज॒स्या॑ अ॒ग्ने र॒ग्ने स्ते॑ज॒स्याः᳚ स्थ स्थ तेज॒स्या॑ अ॒ग्ने र॒ग्ने स्ते॑ज॒स्याः᳚ स्थ । \newline
38. ते॒ज॒स्याः᳚ स्थ स्थ तेज॒स्या᳚ स्तेज॒स्याः᳚ स्था॒पा म॒पाꣳ स्थ॑ तेज॒स्या᳚ स्तेज॒स्याः᳚ स्था॒पाम् । \newline
39. स्था॒पा म॒पाꣳ स्थ॑ स्था॒पा मोष॑धीना॒ मोष॑धीना म॒पाꣳ स्थ॑ स्था॒पा मोष॑धीनाम् । \newline
40. अ॒पा मोष॑धीना॒ मोष॑धीना म॒पा म॒पा मोष॑धीनाꣳ॒॒ रसो॒ रस॒ ओष॑धीना म॒पा म॒पा मोष॑धीनाꣳ॒॒ रसः॑ । \newline
41. ओष॑धीनाꣳ॒॒ रसो॒ रस॒ ओष॑धीना॒ मोष॑धीनाꣳ॒॒ रसः॑ स्थ स्थ॒ रस॒ ओष॑धीना॒ मोष॑धीनाꣳ॒॒ रसः॑ स्थ । \newline
42. रसः॑ स्थ स्थ॒ रसो॒ रसः॑ स्था॒पो॑ ऽपः स्थ॒ रसो॒ रसः॑ स्था॒पः । \newline
43. स्था॒पो॑ ऽपः स्थ॑ स्था॒पो दे॒वीर् दे॒वी र॒पः स्थ॑ स्था॒पो दे॒वीः । \newline
44. अ॒पो दे॒वीर् दे॒वीर॒पो॑ ऽपो दे॒वीर् मधु॑मती॒र् मधु॑मतीर् दे॒वीर॒पो॑ ऽपो दे॒वीर् मधु॑मतीः । \newline
45. दे॒वीर् मधु॑मती॒र् मधु॑मतीर् दे॒वीर् दे॒वीर् मधु॑मती रगृह्णन् नगृह्ण॒न् मधु॑मतीर् दे॒वीर् दे॒वीर् मधु॑मती रगृह्णन्न् । \newline
46. मधु॑मती रगृह्णन् नगृह्ण॒न् मधु॑मती॒र् मधु॑मती रगृह्ण॒न् नूर्ज॑स्वती॒ रूर्ज॑स्वती रगृह्ण॒न् मधु॑मती॒र् मधु॑मती रगृह्ण॒न् नूर्ज॑स्वतीः । \newline
47. मधु॑मती॒रिति॒ मधु॑ - म॒तीः॒ । \newline
48. अ॒गृ॒ह्ण॒न् नूर्ज॑स्वती॒ रूर्ज॑स्वती रगृह्णन् नगृह्ण॒न् नूर्ज॑स्वती राज॒सूया॑य राज॒सूया॒ योर्ज॑स्वती रगृह्णन् नगृह्ण॒न् नूर्ज॑स्वती राज॒सूया॑य । \newline
49. ऊर्ज॑स्वती राज॒सूया॑य राज॒सूया॒ योर्ज॑स्वती॒ रूर्ज॑स्वती राज॒सूया॑य॒ चिता॑ना॒ श्चिता॑ना राज॒सूया॒ योर्ज॑स्वती॒ रूर्ज॑स्वती राज॒सूया॑य॒ चिता॑नाः । \newline
50. रा॒ज॒सूया॑य॒ चिता॑ना॒ श्चिता॑ना राज॒सूया॑य राज॒सूया॑य॒ चिता॑नाः । \newline
51. रा॒ज॒सूया॒येति॑ राज - सूया॑य । \newline
52. चिता॑ना॒ इति॒ चिता॑नाः । \newline
53. याभि॑र् मि॒त्रावरु॑णौ मि॒त्रावरु॑णौ॒ याभि॒र् याभि॑र् मि॒त्रावरु॑णा व॒भ्यषि॑ञ्चन् न॒भ्यषि॑ञ्चन् मि॒त्रावरु॑णौ॒ याभि॒र् याभि॑र् मि॒त्रावरु॑णा व॒भ्यषि॑ञ्चन्न् । \newline
54. मि॒त्रावरु॑णा व॒भ्यषि॑ञ्चन् न॒भ्यषि॑ञ्चन् मि॒त्रावरु॑णौ मि॒त्रावरु॑णा व॒भ्यषि॑ञ्च॒न्॒. याभि॒र् याभि॑ र॒भ्यषि॑ञ्चन् मि॒त्रावरु॑णौ मि॒त्रावरु॑णा व॒भ्यषि॑ञ्च॒न्॒. याभिः॑ । \newline
55. मि॒त्रावरु॑णा॒विति॑ मि॒त्रा - वरु॑णौ । \newline
56. अ॒भ्यषि॑ञ्च॒न्॒. याभि॒र् याभि॑ र॒भ्यषि॑ञ्चन् न॒भ्यषि॑ञ्च॒न्॒. याभि॒रिन्द्र॒ मिन्द्रं॒  
ॅयाभि॑ र॒भ्यषि॑ञ्चन् न॒भ्यषि॑ञ्च॒न्॒. याभि॒रिन्द्र᳚म् । \newline
57. अ॒भ्यषि॑ञ्च॒न्नित्य॑भि - असि॑ञ्चन्न् । \newline
58. याभि॒ रिन्द्र॒ मिन्द्रं॒ ॅयाभि॒र् याभि॒ रिन्द्र॒ मन॑य॒न् नन॑य॒न् निन्द्रं॒ ॅयाभि॒र् याभि॒ रिन्द्र॒ मन॑यन्न् । \newline
59. इन्द्र॒ मन॑य॒न् नन॑य॒न् निन्द्र॒ मिन्द्र॒ मन॑य॒न् नत्य त्यन॑य॒न् निन्द्र॒ मिन्द्र॒ मन॑य॒न् नति॑ । \newline
60. अन॑य॒न् नत्य त्यन॑य॒न् नन॑य॒न् नत्यरा॑ती॒ ररा॑ती॒ रत्यन॑य॒न् नन॑य॒न् नत्यरा॑तीः । \newline
61. अत्य रा॑ती॒ ररा॑ती॒ रत्य त्यरा॑तीः । \newline
62. अरा॑ती॒रित्यरा॑तीः । \newline
63. रा॒ष्ट्र॒दाः स्थ॑ स्थ राष्ट्र॒दा रा᳚ष्ट्र॒दाः स्थ॑ रा॒ष्ट्रꣳ रा॒ष्ट्रꣳ स्थ॑ राष्ट्र॒दा रा᳚ष्ट्र॒दाः स्थ॑ रा॒ष्ट्रम् । \newline
64. रा॒ष्ट्र॒दा इति॑ राष्ट्र - दाः । \newline
65. स्थ॒ रा॒ष्ट्रꣳ रा॒ष्ट्रꣳ स्थ॑ स्थ रा॒ष्ट्रम् द॑त्त दत्त रा॒ष्ट्रꣳ स्थ॑ स्थ रा॒ष्ट्रम् द॑त्त । \newline
66. रा॒ष्ट्रम् द॑त्त दत्त रा॒ष्ट्रꣳ रा॒ष्ट्रम् द॑त्त॒ स्वाहा॒ स्वाहा॑ दत्त रा॒ष्ट्रꣳ रा॒ष्ट्रम् द॑त्त॒ स्वाहा᳚ । \newline
67. द॒त्त॒ स्वाहा॒ स्वाहा॑ दत्त दत्त॒ स्वाहा॑ राष्ट्र॒दा रा᳚ष्ट्र॒दाः स्वाहा॑ दत्त दत्त॒ स्वाहा॑ राष्ट्र॒दाः । \newline
68. स्वाहा॑ राष्ट्र॒दा रा᳚ष्ट्र॒दाः स्वाहा॒ स्वाहा॑ राष्ट्र॒दाः स्थ॑ स्थ राष्ट्र॒दाः स्वाहा॒ स्वाहा॑ राष्ट्र॒दाः स्थ॑ । \newline
69. रा॒ष्ट्र॒दाः स्थ॑ स्थ राष्ट्र॒दा रा᳚ष्ट्र॒दाः स्थ॑ रा॒ष्ट्रꣳ रा॒ष्ट्रꣳ स्थ॑ राष्ट्र॒दा रा᳚ष्ट्र॒दाः स्थ॑ रा॒ष्ट्रम् । \newline
70. रा॒ष्ट्र॒दा इति॑ राष्ट्र - दाः । \newline
71. स्थ॒ रा॒ष्ट्रꣳ रा॒ष्ट्रꣳ स्थ॑ स्थ रा॒ष्ट्र म॒मुष्मा॑ अ॒मुष्मै॑ रा॒ष्ट्रꣳ स्थ॑ स्थ रा॒ष्ट्र म॒मुष्मै᳚ । \newline
72. रा॒ष्ट्र म॒मुष्मा॑ अ॒मुष्मै॑ रा॒ष्ट्रꣳ रा॒ष्ट्र म॒मुष्मै॑ दत्त दत्ता॒मुष्मै॑ रा॒ष्ट्रꣳ रा॒ष्ट्र म॒मुष्मै॑ दत्त । \newline
73. अ॒मुष्मै॑ दत्त दत्ता॒मुष्मा॑ अ॒मुष्मै॑ दत्त । \newline
74. द॒त्तेति॑ दत्त । \newline
\pagebreak
\markright{ TS 1.8.12.1  \hfill https://www.vedavms.in \hfill}
\addcontentsline{toc}{section}{ TS 1.8.12.1 }
\section*{ TS 1.8.12.1 }

\textbf{TS 1.8.12.1 } \newline
\textbf{Samhita Paata} \newline

देवी॑रापः॒ सं मधु॑मती॒र् मधु॑मतीभिः सृज्यद्ध्वं॒ महि॒ वर्चः॑ क्ष॒त्रिया॑य वन्वा॒ना अना॑धृष्टाः सीद॒तोर्ज॑स्वती॒र्महि॒ वर्चः॑ क्ष॒त्रिया॑य॒ दध॑ती॒रनि॑भृष्टमसि वा॒चो बन्धु॑स्तपो॒जाः सोम॑स्य दा॒त्रम॑सि शु॒क्रा वः॑ शु॒क्रेणोत्पु॑नामि च॒न्द्राश्च॒न्द्रेणा॒मृता॑ अ॒मृते॑न॒ स्वाहा॑ राज॒सूया॑य॒ चिता॑नाः । स॒ध॒मादो᳚ द्यु॒म्निनी॒रूर्ज॑ ए॒ता अनि॑भृष्टा अप॒स्युवो॒ वसा॑नः ॥ प॒स्त्या॑सु चक्रे॒ वरु॑णः स॒धस्थ॑म॒पाꣳ शिशु॑र् - [ ] \newline

\textbf{Pada Paata} \newline

देवीः᳚ । आ॒पः॒ । समिति॑ । मधु॑मती॒रिति॒ मधु॑- म॒तीः॒ । मधु॑मतीभि॒रिति॒ मधु॑-म॒ती॒भिः॒ । सृ॒ज्य॒द्ध्व॒म् । महि॑ । वर्चः॑ । क्ष॒त्रिया॑य । व॒न्वा॒नाः । अना॑धृष्टा॒ इत्यना᳚-धृ॒ष्टाः॒ । सी॒द॒त॒ । ऊर्ज॑स्वतीः । महि॑ । वर्चः॑ । क्ष॒त्रिया॑य । दध॑तीः । अनि॑भृष्ट॒मित्यनि॑-भृ॒ष्ट॒म् । अ॒सि॒ । वा॒चः । बन्धुः॑ । त॒पो॒जा इति॑ तपः - जाः । सोम॑स्य । दा॒त्रम् । अ॒सि॒ । शु॒क्राः । वः॒ । शु॒क्रेण॑ । उदिति॑ । पु॒ना॒मि॒ । च॒न्द्राः । च॒न्द्रेण॑ । अ॒मृताः᳚ । अ॒मृते॑न । स्वाहा᳚ । रा॒ज॒सूया॒येति॑ राज - सूया॑य । चिता॑नाः ॥ स॒ध॒माद॒ इति॑ सध-मादः॑ । द्यु॒म्निनीः᳚ । ऊर्जः॑ । ए॒ताः । अनि॑भृष्टा॒ इत्यनि॑-भृ॒ष्टाः॒ । अ॒प॒स्युवः॑ । वसा॑नः ॥ प॒स्त्या॑सु । च॒क्रे॒ । वरु॑णः । स॒धस्थ॒मिति॑ स॒ध - स्थ॒म् । अ॒पाम् । शिशुः॑ ।  \newline



\textbf{Jatai Paata} \newline

1. देवी॑ राप आपो॒ देवी॒र् देवी॑ रापः । \newline
2. आ॒पः॒ सꣳ स मा॑प आपः॒ सम् । \newline
3. सम् मधु॑मती॒र् मधु॑मतीः॒ सꣳ सम् मधु॑मतीः । \newline
4. मधु॑मती॒र् मधु॑मतीभि॒र् मधु॑मतीभि॒र् मधु॑मती॒र् मधु॑मती॒र् मधु॑मतीभिः । \newline
5. मधु॑मती॒रिति॒ मधु॑ - म॒तीः॒ । \newline
6. मधु॑मतीभिः सृज्यद्ध्वꣳ सृज्यद्ध्व॒म् मधु॑मतीभि॒र् मधु॑मतीभिः सृज्यद्ध्वम् । \newline
7. मधु॑मतीभि॒रिति॒ मधु॑ - म॒ती॒भिः॒ । \newline
8. सृ॒ज्य॒द्ध्व॒म् महि॒ महि॑ सृज्यद्ध्वꣳ सृज्यद्ध्व॒म् महि॑ । \newline
9. महि॒ वर्चो॒ वर्चो॒ महि॒ महि॒ वर्चः॑ । \newline
10. वर्चः॑ क्ष॒त्रिया॑य क्ष॒त्रिया॑य॒ वर्चो॒ वर्चः॑ क्ष॒त्रिया॑य । \newline
11. क्ष॒त्रिया॑य वन्वा॒ना व॑न्वा॒नाः क्ष॒त्रिया॑य क्ष॒त्रिया॑य वन्वा॒नाः । \newline
12. व॒न्वा॒ना अना॑धृष्टा॒ अना॑धृष्टा वन्वा॒ना व॑न्वा॒ना अना॑धृष्टाः । \newline
13. अना॑धृष्टाः सीदत सीद॒ता ना॑धृष्टा॒ अना॑धृष्टाः सीदत । \newline
14. अना॑धृष्टा॒ इत्यना᳚ - धृ॒ष्टाः॒ । \newline
15. सी॒द॒तोर्ज॑स्वती॒ रूर्ज॑स्वतीः सीदत सीद॒तोर्ज॑स्वतीः । \newline
16. ऊर्ज॑स्वती॒र् महि॒ मह्यूर्ज॑स्वती॒ रूर्ज॑स्वती॒र् महि॑ । \newline
17. महि॒ वर्चो॒ वर्चो॒ महि॒ महि॒ वर्चः॑ । \newline
18. वर्चः॑ क्ष॒त्रिया॑य क्ष॒त्रिया॑य॒ वर्चो॒ वर्चः॑ क्ष॒त्रिया॑य । \newline
19. क्ष॒त्रिया॑य॒ दध॑ती॒र् दध॑तीः क्ष॒त्रिया॑य क्ष॒त्रिया॑य॒ दध॑तीः । \newline
20. दध॑ती॒ रनि॑भृष्ट॒ मनि॑भृष्ट॒म् दध॑ती॒र् दध॑ती॒ रनि॑भृष्टम् । \newline
21. अनि॑भृष्ट मस्य॒ स्यनि॑भृष्ट॒ मनि॑भृष्ट मसि । \newline
22. अनि॑भृष्ट॒मित्यनि॑ - भृ॒ष्ट॒म् । \newline
23. अ॒सि॒ वा॒चो वा॒चो᳚ ऽस्यसि वा॒चः । \newline
24. वा॒चो बन्धु॒र् बन्धु॑र् वा॒चो वा॒चो बन्धुः॑ । \newline
25. बन्धु॑ स्तपो॒जा स्त॑पो॒जा बन्धु॒र् बन्धु॑ स्तपो॒जाः । \newline
26. त॒पो॒जाः सोम॑स्य॒ सोम॑स्य तपो॒जा स्त॑पो॒जाः सोम॑स्य । \newline
27. त॒पो॒जा इति॑ तपः - जाः । \newline
28. सोम॑स्य दा॒त्रम् दा॒त्रꣳ सोम॑स्य॒ सोम॑स्य दा॒त्रम् । \newline
29. दा॒त्र म॑स्यसि दा॒त्रम् दा॒त्र म॑सि । \newline
30. अ॒सि॒ शु॒क्राः शु॒क्रा अ॑स्यसि शु॒क्राः । \newline
31. शु॒क्रा वो॑ वः शु॒क्राः शु॒क्रा वः॑ । \newline
32. वः॒ शु॒क्रेण॑ शु॒क्रेण॑ वो वः शु॒क्रेण॑ । \newline
33. शु॒क्रेणोदुच् छु॒क्रेण॑ शु॒क्रेणोत् । \newline
34. उत् पु॑नामि पुना॒ म्युदुत् पु॑नामि । \newline
35. पु॒ना॒मि॒ च॒न्द्रा श्च॒न्द्राः पु॑नामि पुनामि च॒न्द्राः । \newline
36. च॒न्द्रा श्च॒न्द्रेण॑ च॒न्द्रेण॑ च॒न्द्रा श्च॒न्द्रा श्च॒न्द्रेण॑ । \newline
37. च॒न्द्रेणा॒मृता॑ अ॒मृता᳚ श्च॒न्द्रेण॑ च॒न्द्रेणा॒मृताः᳚ । \newline
38. अ॒मृता॑ अ॒मृते॑ना॒ मृते॑ना॒ मृता॑ अ॒मृता॑ अ॒मृते॑न । \newline
39. अ॒मृते॑न॒ स्वाहा॒ स्वाहा॒ ऽमृते॑ना॒ मृते॑न॒ स्वाहा᳚ । \newline
40. स्वाहा॑ राज॒सूया॑य राज॒सूया॑य॒ स्वाहा॒ स्वाहा॑ राज॒सूया॑य । \newline
41. रा॒ज॒सूया॑य॒ चिता॑ना॒ श्चिता॑ना राज॒सूया॑य राज॒सूया॑य॒ चिता॑नाः । \newline
42. रा॒ज॒सूया॒येति॑ राज - सूया॑य । \newline
43. चिता॑ना॒ इति॒ चिता॑नाः । \newline
44. स॒ध॒मादो᳚ द्यु॒म्निनी᳚र् द्यु॒म्निनीः᳚ सध॒मादः॑ सध॒मादो᳚ द्यु॒म्निनीः᳚ । \newline
45. स॒ध॒माद॒ इति॑ सध - मादः॑ । \newline
46. द्यु॒म्निनी॒ रूर्ज॒ ऊर्जो᳚ द्यु॒म्निनी᳚र् द्यु॒म्निनी॒ रूर्जः॑ । \newline
47. ऊर्ज॑ ए॒ता ए॒ता ऊर्ज॒ ऊर्ज॑ ए॒ताः । \newline
48. ए॒ता अनि॑भृष्टा॒ अनि॑भृष्टा ए॒ता ए॒ता अनि॑भृष्टाः । \newline
49. अनि॑भृष्टा अप॒स्युवो॑ ऽप॒स्युवो ऽनि॑भृष्टा॒ अनि॑भृष्टा अप॒स्युवः॑ । \newline
50. अनि॑भृष्टा॒ इत्यनि॑ - भृ॒ष्टाः॒ । \newline
51. अ॒प॒स्युवो॒ वसा॑नो॒ वसा॑नो ऽप॒स्युवो॑ ऽप॒स्युवो॒ वसा॑नः । \newline
52. वसा॑न॒ इति॒ वसा॑नः । \newline
53. प॒स्त्या॑सु चक्रे चक्रे प॒स्त्या॑सु प॒स्त्या॑सु चक्रे । \newline
54. च॒क्रे॒ वरु॑णो॒ वरु॑ण श्चक्रे चक्रे॒ वरु॑णः । \newline
55. वरु॑णः स॒धस्थꣳ॑ स॒धस्थं॒ ॅवरु॑णो॒ वरु॑णः स॒धस्थ᳚म् । \newline
56. स॒धस्थ॑ म॒पा म॒पाꣳ स॒धस्थꣳ॑ स॒धस्थ॑ म॒पाम् । \newline
57. स॒धस्थ॒मिति॑ स॒ध - स्थ॒म् । \newline
58. अ॒पाꣳ शिशुः॒ शिशु॑र॒पा म॒पाꣳ शिशुः॑ । \newline
59. शिशु॑र् मा॒तृत॑मासु मा॒तृत॑मासु॒ शिशुः॒ शिशु॑र् मा॒तृत॑मासु । \newline

\textbf{Ghana Paata } \newline

1. देवी॑ राप आपो॒ देवी॒र् देवी॑ रापः॒ सꣳ स मा॑पो॒ देवी॒र् देवी॑ रापः॒ सम् । \newline
2. आ॒पः॒ सꣳ स मा॑प आपः॒ सम् मधु॑मती॒र् मधु॑मतीः॒ स मा॑प आपः॒ सम् मधु॑मतीः । \newline
3. सम् मधु॑मती॒र् मधु॑मतीः॒ सꣳ सम् मधु॑मती॒र् मधु॑मतीभि॒र् मधु॑मतीभि॒र् मधु॑मतीः॒ सꣳ सम् मधु॑मती॒र् मधु॑मतीभिः । \newline
4. मधु॑मती॒र् मधु॑मतीभि॒र् मधु॑मतीभि॒र् मधु॑मती॒र् मधु॑मती॒र् मधु॑मतीभिः सृज्यद्ध्वꣳ सृज्यद्ध्व॒म् मधु॑मतीभि॒र् मधु॑मती॒र् मधु॑मती॒र् मधु॑मतीभिः सृज्यद्ध्वम् । \newline
5. मधु॑मती॒रिति॒ मधु॑ - म॒तीः॒ । \newline
6. मधु॑मतीभिः सृज्यद्ध्वꣳ सृज्यद्ध्व॒म् मधु॑मतीभि॒र् मधु॑मतीभिः सृज्यद्ध्व॒म् महि॒ महि॑ सृज्यद्ध्व॒म् मधु॑मतीभि॒र् मधु॑मतीभिः सृज्यद्ध्व॒म् महि॑ । \newline
7. मधु॑मतीभि॒रिति॒ मधु॑ - म॒ती॒भिः॒ । \newline
8. सृ॒ज्य॒द्ध्व॒म् महि॒ महि॑ सृज्यद्ध्वꣳ सृज्यद्ध्व॒म् महि॒ वर्चो॒ वर्चो॒ महि॑ सृज्यद्ध्वꣳ सृज्यद्ध्व॒म् महि॒ वर्चः॑ । \newline
9. महि॒ वर्चो॒ वर्चो॒ महि॒ महि॒ वर्चः॑ क्ष॒त्रिया॑य क्ष॒त्रिया॑य॒ वर्चो॒ महि॒ महि॒ वर्चः॑ क्ष॒त्रिया॑य । \newline
10. वर्चः॑ क्ष॒त्रिया॑य क्ष॒त्रिया॑य॒ वर्चो॒ वर्चः॑ क्ष॒त्रिया॑य वन्वा॒ना व॑न्वा॒नाः क्ष॒त्रिया॑य॒ वर्चो॒ वर्चः॑ क्ष॒त्रिया॑य वन्वा॒नाः । \newline
11. क्ष॒त्रिया॑य वन्वा॒ना व॑न्वा॒नाः क्ष॒त्रिया॑य क्ष॒त्रिया॑य वन्वा॒ना अना॑धृष्टा॒ अना॑धृष्टा वन्वा॒नाः क्ष॒त्रिया॑य क्ष॒त्रिया॑य वन्वा॒ना अना॑धृष्टाः । \newline
12. व॒न्वा॒ना अना॑धृष्टा॒ अना॑धृष्टा वन्वा॒ना व॑न्वा॒ना अना॑धृष्टाः सीदत सीद॒ता ना॑धृष्टा वन्वा॒ना व॑न्वा॒ना अना॑धृष्टाः सीदत । \newline
13. अना॑धृष्टाः सीदत सीद॒ता ना॑धृष्टा॒ अना॑धृष्टाः सीद॒ तोर्ज॑स्वती॒ रूर्ज॑स्वतीः सीद॒ता ना॑धृष्टा॒ अना॑धृष्टाः सीद॒ तोर्ज॑स्वतीः । \newline
14. अना॑धृष्टा॒ इत्यना᳚ - धृ॒ष्टाः॒ । \newline
15. सी॒द॒ तोर्ज॑स्वती॒ रूर्ज॑स्वतीः सीदत सीद॒ तोर्ज॑स्वती॒र् महि॒ मह्यूर्ज॑स्वतीः सीदत सीद॒ तोर्ज॑स्वती॒र् महि॑ । \newline
16. ऊर्ज॑स्वती॒र् महि॒ मह्यूर्ज॑स्वती॒ रूर्ज॑स्वती॒र् महि॒ वर्चो॒ वर्चो॒ मह्यूर्ज॑स्वती॒ रूर्ज॑स्वती॒र् महि॒ वर्चः॑ । \newline
17. महि॒ वर्चो॒ वर्चो॒ महि॒ महि॒ वर्चः॑ क्ष॒त्रिया॑य क्ष॒त्रिया॑य॒ वर्चो॒ महि॒ महि॒ वर्चः॑ क्ष॒त्रिया॑य । \newline
18. वर्चः॑ क्ष॒त्रिया॑य क्ष॒त्रिया॑य॒ वर्चो॒ वर्चः॑ क्ष॒त्रिया॑य॒ दध॑ती॒र् दध॑तीः क्ष॒त्रिया॑य॒ वर्चो॒ वर्चः॑ क्ष॒त्रिया॑य॒ दध॑तीः । \newline
19. क्ष॒त्रिया॑य॒ दध॑ती॒र् दध॑तीः क्ष॒त्रिया॑य क्ष॒त्रिया॑य॒ दध॑ती॒ रनि॑भृष्ट॒ मनि॑भृष्ट॒म् दध॑तीः क्ष॒त्रिया॑य क्ष॒त्रिया॑य॒ दध॑ती॒ रनि॑भृष्टम् । \newline
20. दध॑ती॒ रनि॑भृष्ट॒ मनि॑भृष्ट॒म् दध॑ती॒र् दध॑ती॒ रनि॑भृष्ट मस्य॒ स्यनि॑भृष्ट॒म् दध॑ती॒र् दध॑ती॒ रनि॑भृष्ट मसि । \newline
21. अनि॑भृष्ट मस्य॒ स्यनि॑भृष्ट॒ मनि॑भृष्ट मसि वा॒चो वा॒चो᳚ ऽस्यनि॑भृष्ट॒ मनि॑भृष्ट मसि वा॒चः । \newline
22. अनि॑भृष्ट॒मित्यनि॑ - भृ॒ष्ट॒म् । \newline
23. अ॒सि॒ वा॒चो वा॒चो᳚ ऽस्यसि वा॒चो बन्धु॒र् बन्धु॑र् वा॒चो᳚ ऽस्यसि वा॒चो बन्धुः॑ । \newline
24. वा॒चो बन्धु॒र् बन्धु॑र् वा॒चो वा॒चो बन्धु॑ स्तपो॒जा स्त॑पो॒जा बन्धु॑र् वा॒चो वा॒चो बन्धु॑ स्तपो॒जाः । \newline
25. बन्धु॑ स्तपो॒जा स्त॑पो॒जा बन्धु॒र् बन्धु॑ स्तपो॒जाः सोम॑स्य॒ सोम॑स्य तपो॒जा बन्धु॒र् बन्धु॑ स्तपो॒जाः सोम॑स्य । \newline
26. त॒पो॒जाः सोम॑स्य॒ सोम॑स्य तपो॒जा स्त॑पो॒जाः सोम॑स्य दा॒त्रम् दा॒त्रꣳ सोम॑स्य तपो॒जा स्त॑पो॒जाः सोम॑स्य दा॒त्रम् । \newline
27. त॒पो॒जा इति॑ तपः - जाः । \newline
28. सोम॑स्य दा॒त्रम् दा॒त्रꣳ सोम॑स्य॒ सोम॑स्य दा॒त्र म॑स्यसि दा॒त्रꣳ सोम॑स्य॒ सोम॑स्य दा॒त्र म॑सि । \newline
29. दा॒त्र म॑स्यसि दा॒त्रम् दा॒त्र म॑सि शु॒क्राः शु॒क्रा अ॑सि दा॒त्रम् दा॒त्र म॑सि शु॒क्राः । \newline
30. अ॒सि॒ शु॒क्राः शु॒क्रा अ॑स्यसि शु॒क्रा वो॑ वः शु॒क्रा अ॑स्यसि शु॒क्रा वः॑ । \newline
31. शु॒क्रा वो॑ वः शु॒क्राः शु॒क्रा वः॑ शु॒क्रेण॑ शु॒क्रेण॑ वः शु॒क्राः शु॒क्रा वः॑ शु॒क्रेण॑ । \newline
32. वः॒ शु॒क्रेण॑ शु॒क्रेण॑ वो वः शु॒क्रेणोदुच् छु॒क्रेण॑ वो वः शु॒क्रेणोत् । \newline
33. शु॒क्रेणोदुच् छु॒क्रेण॑ शु॒क्रेणोत् पु॑नामि पुना॒म्युच् छु॒क्रेण॑ शु॒क्रेणोत् पु॑नामि । \newline
34. उत् पु॑नामि पुना॒म्युदुत् पु॑नामि च॒न्द्रा श्च॒न्द्राः पु॑ना॒म्युदुत् पु॑नामि च॒न्द्राः । \newline
35. पु॒ना॒मि॒ च॒न्द्रा श्च॒न्द्राः पु॑नामि पुनामि च॒न्द्रा श्च॒न्द्रेण॑ च॒न्द्रेण॑ च॒न्द्राः पु॑नामि पुनामि च॒न्द्रा श्च॒न्द्रेण॑ । \newline
36. च॒न्द्रा श्च॒न्द्रेण॑ च॒न्द्रेण॑ च॒न्द्रा श्च॒न्द्रा श्च॒न्द्रे णा॒मृता॑ अ॒मृता᳚ श्च॒न्द्रेण॑ च॒न्द्रा श्च॒न्द्रा श्च॒न्द्रे णा॒मृताः᳚ । \newline
37. च॒न्द्रे णा॒मृता॑ अ॒मृता᳚ श्च॒न्द्रेण॑ च॒न्द्रे णा॒मृता॑ अ॒मृते॑ ना॒मृते॑ ना॒मृता᳚ श्च॒न्द्रेण॑ च॒न्द्रे णा॒मृता॑ अ॒मृते॑न । \newline
38. अ॒मृता॑ अ॒मृते॑ना॒ मृते॑ना॒ मृता॑ अ॒मृता॑ अ॒मृते॑न॒ स्वाहा॒ स्वाहा॒ ऽमृते॑ना॒ मृता॑ अ॒मृता॑ अ॒मृते॑न॒ स्वाहा᳚ । \newline
39. अ॒मृते॑न॒ स्वाहा॒ स्वाहा॒ ऽमृते॑ना॒ मृते॑न॒ स्वाहा॑ राज॒सूया॑य राज॒सूया॑य॒ स्वाहा॒ ऽमृते॑ना॒ मृते॑न॒ स्वाहा॑ राज॒सूया॑य । \newline
40. स्वाहा॑ राज॒सूया॑य राज॒सूया॑य॒ स्वाहा॒ स्वाहा॑ राज॒सूया॑य॒ चिता॑ना॒ श्चिता॑ना राज॒सूया॑य॒ स्वाहा॒ स्वाहा॑ राज॒सूया॑य॒ चिता॑नाः । \newline
41. रा॒ज॒सूया॑य॒ चिता॑ना॒ श्चिता॑ना राज॒सूया॑य राज॒सूया॑य॒ चिता॑नाः । \newline
42. रा॒ज॒सूया॒येति॑ राज - सूया॑य । \newline
43. चिता॑ना॒ इति॒ चिता॑नाः । \newline
44. स॒ध॒मादो᳚ द्यु॒म्निनी᳚र् द्यु॒म्निनीः᳚ सध॒मादः॑ सध॒मादो᳚ द्यु॒म्निनी॒ रूर्ज॒ ऊर्जो᳚ द्यु॒म्निनीः᳚ सध॒मादः॑ सध॒मादो᳚ द्यु॒म्निनी॒ रूर्जः॑ । \newline
45. स॒ध॒माद॒ इति॑ सध - मादः॑ । \newline
46. द्यु॒म्निनी॒ रूर्ज॒ ऊर्जो᳚ द्यु॒म्निनी᳚र् द्यु॒म्निनी॒ रूर्ज॑ ए॒ता ए॒ता ऊर्जो᳚ द्यु॒म्निनी᳚र् द्यु॒म्निनी॒ रूर्ज॑ ए॒ताः । \newline
47. ऊर्ज॑ ए॒ता ए॒ता ऊर्ज॒ ऊर्ज॑ ए॒ता अनि॑भृष्टा॒ अनि॑भृष्टा ए॒ता ऊर्ज॒ ऊर्ज॑ ए॒ता अनि॑भृष्टाः । \newline
48. ए॒ता अनि॑भृष्टा॒ अनि॑भृष्टा ए॒ता ए॒ता अनि॑भृष्टा अप॒स्युवो॑ ऽप॒स्युवो ऽनि॑भृष्टा ए॒ता ए॒ता अनि॑भृष्टा अप॒स्युवः॑ । \newline
49. अनि॑भृष्टा अप॒स्युवो॑ ऽप॒स्युवो ऽनि॑भृष्टा॒ अनि॑भृष्टा अप॒स्युवो॒ वसा॑नो॒ वसा॑नो ऽप॒स्युवो ऽनि॑भृष्टा॒ अनि॑भृष्टा अप॒स्युवो॒ वसा॑नः । \newline
50. अनि॑भृष्टा॒ इत्यनि॑ - भृ॒ष्टाः॒ । \newline
51. अ॒प॒स्युवो॒ वसा॑नो॒ वसा॑नो ऽप॒स्युवो॑ ऽप॒स्युवो॒ वसा॑नः । \newline
52. वसा॑न॒ इति॒ वसा॑नः । \newline
53. प॒स्त्या॑सु चक्रे चक्रे प॒स्त्या॑सु प॒स्त्या॑सु चक्रे॒ वरु॑णो॒ वरु॑ण श्चक्रे प॒स्त्या॑सु प॒स्त्या॑सु चक्रे॒ वरु॑णः । \newline
54. च॒क्रे॒ वरु॑णो॒ वरु॑ण श्चक्रे चक्रे॒ वरु॑णः स॒धस्थꣳ॑ स॒धस्थं॒ ॅवरु॑ण श्चक्रे चक्रे॒ वरु॑णः स॒धस्थ᳚म् । \newline
55. वरु॑णः स॒धस्थꣳ॑ स॒धस्थं॒ ॅवरु॑णो॒ वरु॑णः स॒धस्थ॑ म॒पा म॒पाꣳ स॒धस्थं॒ ॅवरु॑णो॒ वरु॑णः स॒धस्थ॑ म॒पाम् । \newline
56. स॒धस्थ॑ म॒पा म॒पाꣳ स॒धस्थꣳ॑ स॒धस्थ॑ म॒पाꣳ शिशुः॒ शिशु॑र॒पाꣳ स॒धस्थꣳ॑ स॒धस्थ॑ म॒पाꣳ शिशुः॑ । \newline
57. स॒धस्थ॒मिति॑ स॒ध - स्थ॒म् । \newline
58. अ॒पाꣳ शिशुः॒ शिशु॑र॒पा म॒पाꣳ शिशु॑र् मा॒तृत॑मासु मा॒तृत॑मासु॒ शिशु॑र॒पा म॒पाꣳ शिशु॑र् मा॒तृत॑मासु । \newline
59. शिशु॑र् मा॒तृत॑मासु मा॒तृत॑मासु॒ शिशुः॒ शिशु॑र् मा॒तृत॑मा स्व॒न्त र॒न्तर् मा॒तृत॑मासु॒ शिशुः॒ शिशु॑र् मा॒तृत॑मा स्व॒न्तः । \newline
\pagebreak
\markright{ TS 1.8.12.2  \hfill https://www.vedavms.in \hfill}
\addcontentsline{toc}{section}{ TS 1.8.12.2 }
\section*{ TS 1.8.12.2 }

\textbf{TS 1.8.12.2 } \newline
\textbf{Samhita Paata} \newline

मा॒तृत॑मास्व॒न्तः ॥ क्ष॒त्रस्योल्ब॑मसि क्ष॒त्रस्य॒ योनि॑र॒स्यावि॑न्नो अ॒ग्निर् गृ॒हप॑ति॒रावि॑न्न॒ इन्द्रो॑ वृ॒द्धश्र॑वा॒ आवि॑न्नः पू॒षा वि॒श्ववे॑दा॒ आवि॑न्नौ मि॒त्रावरु॑णा वृता॒वृधा॒वावि॑न्ने॒ द्यावा॑पृथि॒वी धृ॒तव्र॑ते॒ आवि॑न्ना दे॒व्यदि॑तिर् विश्वरू॒प्यावि॑न्नो॒ ऽयम॒सावा॑मुष्याय॒णो᳚ऽस्यां ॅवि॒श्य॑स्मिन् रा॒ष्ट्रे म॑ह॒ते क्ष॒त्राय॑ मह॒त आधि॑पत्याय मह॒ते जान॑राज्यायै॒ष वो॑ भरता॒ राजा॒ सोमो॒ऽस्माकं॑ ब्राह्म॒णानाꣳ॒॒ राजेन्द्र॑स्य॒- [ ] \newline

\textbf{Pada Paata} \newline

मा॒तृत॑मा॒स्विति॑ मा॒तृ - त॒मा॒सु॒ । अ॒न्तः ॥ क्ष॒त्रस्य॑ । उल्ब᳚म् । अ॒सि॒ । क्ष॒त्रस्य॑ । योनिः॑ । अ॒सि॒ । आवि॑न्नः । अ॒ग्निः । गृ॒हप॑ति॒रिति॑ गृ॒ह-प॒तिः॒ । आवि॑न्नः । इन्द्रः॑ । वृ॒द्धश्र॑वा॒ इति॑ वृ॒द्ध-श्र॒वाः॒ । आवि॑न्नः । पू॒षा । वि॒श्ववे॑दा॒ इति॑ वि॒श्व - वे॒दाः॒ । आवि॑न्नौ । मि॒त्रावरु॑णा॒विति॑ मि॒त्रा - वरु॑णौ । ऋ॒ता॒वृधा॒वित्यृ॑त-वृधौ᳚ । आवि॑न्ने॒ इति॑ । द्यावा॑पृथि॒वी इति॒ द्यावा᳚ - पृ॒थि॒वी । धृ॒तव्र॑ते॒ इति॑ धृ॒त - व्र॒ते॒ । आवि॑न्ना । दे॒वी । अदि॑तिः । वि॒श्व॒रू॒पीति॑ विश्व-रू॒पी । आवि॑न्नः । अ॒यम् । अ॒सौ । आ॒मु॒ष्या॒य॒णः । अ॒स्याम् । वि॒शि । अ॒स्मिन्न् । रा॒ष्ट्रे । म॒ह॒ते । क्ष॒त्राय॑ । म॒ह॒ते । आधि॑पत्या॒येत्याधि॑-प॒त्या॒य॒ । म॒ह॒ते । जान॑राज्या॒येति॒ जान॑-रा॒ज्या॒य॒ । ए॒षः । वः॒ । भ॒र॒ताः॒ । राजा᳚ । सोमः॑ । अ॒स्माक᳚म् । ब्रा॒ह्म॒णाना᳚म् । राजा᳚ । इन्द्र॑स्य ।  \newline



\textbf{Jatai Paata} \newline

1. मा॒तृत॑मास्व॒न्त र॒न्तर् मा॒तृत॑मासु मा॒तृत॑मास्व॒न्तः । \newline
2. मा॒तृत॑मा॒स्विति॑ मा॒तृ - त॒मा॒सु॒ । \newline
3. अ॒न्तरित्य॒न्तः । \newline
4. क्ष॒त्रस्योल्ब॒ मुल्ब॑म् क्ष॒त्रस्य॑ क्ष॒त्रस्योल्ब᳚म् । \newline
5. उल्ब॑ मस्य॒ स्युल्ब॒ मुल्ब॑ मसि । \newline
6. अ॒सि॒ क्ष॒त्रस्य॑ क्ष॒त्रस्या᳚स्यसि क्ष॒त्रस्य॑ । \newline
7. क्ष॒त्रस्य॒ योनि॒र् योनिः॑ क्ष॒त्रस्य॑ क्ष॒त्रस्य॒ योनिः॑ । \newline
8. योनि॑ रस्यसि॒ योनि॒र् योनि॑ रसि । \newline
9. अ॒स्यावि॑न्न॒ आवि॑न्नो ऽस्य॒ स्यावि॑न्नः । \newline
10. आवि॑न्नो अ॒ग्नि र॒ग्नि रावि॑न्न॒ आवि॑न्नो अ॒ग्निः । \newline
11. अ॒ग्निर् गृ॒हप॑तिर् गृ॒हप॑ति र॒ग्नि र॒ग्निर् गृ॒हप॑तिः । \newline
12. गृ॒हप॑ति॒ रावि॑न्न॒ आवि॑न्नो गृ॒हप॑तिर् गृ॒हप॑ति॒ रावि॑न्नः । \newline
13. गृ॒हप॑ति॒रिति॑ गृ॒ह - प॒तिः॒ । \newline
14. आवि॑न्न॒ इन्द्र॒ इन्द्र॒ आवि॑न्न॒ आवि॑न्न॒ इन्द्रः॑ । \newline
15. इन्द्रो॑ वृ॒द्धश्र॑वा वृ॒द्धश्र॑वा॒ इन्द्र॒ इन्द्रो॑ वृ॒द्धश्र॑वाः । \newline
16. वृ॒द्धश्र॑वा॒ आवि॑न्न॒ आवि॑न्नो वृ॒द्धश्र॑वा वृ॒द्धश्र॑वा॒ आवि॑न्नः । \newline
17. वृ॒द्धश्र॑वा॒ इति॑ वृ॒द्ध - श्र॒वाः॒ । \newline
18. आवि॑न्नः पू॒षा पू॒षा ऽऽवि॑न्न॒ आवि॑न्नः पू॒षा । \newline
19. पू॒षा वि॒श्ववे॑दा वि॒श्ववे॑दाः पू॒षा पू॒षा वि॒श्ववे॑दाः । \newline
20. वि॒श्ववे॑दा॒ आवि॑न्ना॒ वावि॑न्नौ वि॒श्ववे॑दा वि॒श्ववे॑दा॒ आवि॑न्नौ । \newline
21. वि॒श्ववे॑दा॒ इति॑ वि॒श्व - वे॒दाः॒ । \newline
22. आवि॑न्नौ मि॒त्रावरु॑णौ मि॒त्रावरु॑णा॒ वावि॑न्ना॒ वावि॑न्नौ मि॒त्रावरु॑णौ । \newline
23. मि॒त्रावरु॑णा वृता॒वृधा॑ वृता॒वृधौ॑ मि॒त्रावरु॑णौ मि॒त्रावरु॑णा वृता॒वृधौ᳚ । \newline
24. मि॒त्रावरु॑णा॒विति॑ मि॒त्रा - वरु॑णौ । \newline
25. ऋ॒ता॒वृधा॒ वावि॑न्ने॒ आवि॑न्ने ऋता॒वृधा॑ वृता॒वृधा॒ वावि॑न्ने । \newline
26. ऋ॒ता॒वृधा॒वित्यृ॑त - वृधौ᳚ । \newline
27. आवि॑न्ने॒ द्यावा॑पृथि॒वी द्यावा॑पृथि॒वी आवि॑न्ने॒ आवि॑न्ने॒ द्यावा॑पृथि॒वी । \newline
28. आवि॑न्ने॒ इत्यावि॑न्ने । \newline
29. द्यावा॑पृथि॒वी धृ॒तव्र॑ते धृ॒तव्र॑ते॒ द्यावा॑पृथि॒वी द्यावा॑पृथि॒वी धृ॒तव्र॑ते । \newline
30. द्यावा॑पृथि॒वी इति॒ द्यावा᳚ - पृ॒थि॒वी । \newline
31. धृ॒तव्र॑ते॒ आवि॒न्ना ऽऽवि॑न्ना धृ॒तव्र॑ते धृ॒तव्र॑ते॒ आवि॑न्ना । \newline
32. धृ॒तव्र॑ते॒ इति॑ धृ॒त - व्र॒ते॒ । \newline
33. आवि॑न्ना दे॒वी दे॒व्यावि॒न्ना ऽऽवि॑न्ना दे॒वी । \newline
34. दे॒व्यदि॑ति॒रदि॑तिर् दे॒वी दे॒व्यदि॑तिः । \newline
35. अदि॑तिर् विश्वरू॒पी वि॑श्वरू॒ प्यदि॑ति॒ रदि॑तिर् विश्वरू॒पी । \newline
36. वि॒श्व॒रू॒प्यावि॑न्न॒ आवि॑न्नो विश्वरू॒पी वि॑श्वरू॒प्यावि॑न्नः । \newline
37. वि॒श्व॒रू॒पीति॑ विश्व - रू॒पी । \newline
38. आवि॑न्नो॒ ऽय म॒य मावि॑न्न॒ आवि॑न्नो॒ ऽयम् । \newline
39. अ॒य म॒सा व॒सा व॒य म॒य म॒सौ । \newline
40. अ॒सा वा॑मुष्याय॒ण आ॑मुष्याय॒णो॑ ऽसा व॒सा वा॑मुष्याय॒णः । \newline
41. आ॒मु॒ष्या॒य॒णो᳚ ऽस्या म॒स्या मा॑मुष्याय॒ण आ॑मुष्याय॒णो᳚ ऽस्याम् । \newline
42. अ॒स्यां ॅवि॒शि वि॒श्य॑स्या म॒स्यां ॅवि॒शि । \newline
43. वि॒श्य॑स्मिन् न॒स्मिन्. वि॒शि वि॒श्य॑स्मिन्न् । \newline
44. अ॒स्मिन् रा॒ष्ट्रे रा॒ष्ट्रे᳚ ऽस्मिन् न॒स्मिन् रा॒ष्ट्रे । \newline
45. रा॒ष्ट्रे म॑ह॒ते म॑ह॒ते रा॒ष्ट्रे रा॒ष्ट्रे म॑ह॒ते । \newline
46. म॒ह॒ते क्ष॒त्राय॑ क्ष॒त्राय॑ मह॒ते म॑ह॒ते क्ष॒त्राय॑ । \newline
47. क्ष॒त्राय॑ मह॒ते म॑ह॒ते क्ष॒त्राय॑ क्ष॒त्राय॑ मह॒ते । \newline
48. म॒ह॒त आधि॑पत्या॒ याधि॑पत्याय मह॒ते म॑ह॒त आधि॑पत्याय । \newline
49. आधि॑पत्याय मह॒ते म॑ह॒त आधि॑पत्या॒या धि॑पत्याय मह॒ते । \newline
50. आधि॑पत्या॒येत्याधि॑ - प॒त्या॒य॒ । \newline
51. म॒ह॒ते जान॑राज्याय॒ जान॑राज्याय मह॒ते म॑ह॒ते जान॑राज्याय । \newline
52. जान॑राज्या यै॒ष ए॒ष जान॑राज्याय॒ जान॑राज्या यै॒षः । \newline
53. जान॑राज्या॒येति॒ जान॑ - रा॒ज्या॒य॒ । \newline
54. ए॒ष वो॑ व ए॒ष ए॒ष वः॑ । \newline
55. वो॒ भ॒र॒ता॒ भ॒र॒ता॒ वो॒ वो॒ भ॒र॒ताः॒ । \newline
56. भ॒र॒ता॒ राजा॒ राजा॑ भरता भरता॒ राजा᳚ । \newline
57. राजा॒ सोमः॒ सोमो॒ राजा॒ राजा॒ सोमः॑ । \newline
58. सोमो॒ ऽस्माक॑ म॒स्माकꣳ॒॒ सोमः॒ सोमो॒ ऽस्माक᳚म् । \newline
59. अ॒स्माक॑म् ब्राह्म॒णाना᳚म् ब्राह्म॒णाना॑ म॒स्माक॑ म॒स्माक॑म् ब्राह्म॒णाना᳚म् । \newline
60. ब्रा॒ह्म॒णानाꣳ॒॒ राजा॒ राजा᳚ ब्राह्म॒णाना᳚म् ब्राह्म॒णानाꣳ॒॒ राजा᳚ । \newline
61. राजेन्द्र॒स्ये न्द्र॑स्य॒ राजा॒ राजेन्द्र॑स्य । \newline
62. इन्द्र॑स्य॒ वज्रो॒ वज्र॒ इन्द्र॒स्ये न्द्र॑स्य॒ वज्रः॑ । \newline

\textbf{Ghana Paata } \newline

1. मा॒तृत॑मा स्व॒न्त र॒न्तर् मा॒तृत॑मासु मा॒तृत॑मा स्व॒न्तः । \newline
2. मा॒तृत॑मा॒स्विति॑ मा॒तृ - त॒मा॒सु॒ । \newline
3. अ॒न्तरित्य॒न्तः । \newline
4. क्ष॒त्रस्योल्ब॒ मुल्ब॑म् क्ष॒त्रस्य॑ क्ष॒त्रस्योल्ब॑ मस्य॒ स्युल्ब॑म् क्ष॒त्रस्य॑ क्ष॒त्रस्योल्ब॑ मसि । \newline
5. उल्ब॑ मस्य॒ स्युल्ब॒ मुल्ब॑ मसि क्ष॒त्रस्य॑ क्ष॒त्रस्या॒ स्युल्ब॒ मुल्ब॑ मसि क्ष॒त्रस्य॑ । \newline
6. अ॒सि॒ क्ष॒त्रस्य॑ क्ष॒त्रस्या᳚ स्यसि क्ष॒त्रस्य॒ योनि॒र् योनिः॑ क्ष॒त्रस्या᳚ स्यसि क्ष॒त्रस्य॒ योनिः॑ । \newline
7. क्ष॒त्रस्य॒ योनि॒र् योनिः॑ क्ष॒त्रस्य॑ क्ष॒त्रस्य॒ योनि॑ रस्यसि॒ योनिः॑ क्ष॒त्रस्य॑ क्ष॒त्रस्य॒ योनि॑ रसि । \newline
8. योनि॑ रस्यसि॒ योनि॒र् योनि॑ र॒स्यावि॑न्न॒ आवि॑न्नो ऽसि॒ योनि॒र् योनि॑ र॒स्यावि॑न्नः । \newline
9. अ॒स्यावि॑न्न॒ आवि॑न्नो ऽस्य॒स्यावि॑न्नो अ॒ग्नि र॒ग्नि रावि॑न्नो ऽस्य॒ स्यावि॑न्नो अ॒ग्निः । \newline
10. आवि॑न्नो अ॒ग्नि र॒ग्नि रावि॑न्न॒ आवि॑न्नो अ॒ग्निर् गृ॒हप॑तिर् गृ॒हप॑ति र॒ग्नि रावि॑न्न॒ आवि॑न्नो अ॒ग्निर् गृ॒हप॑तिः । \newline
11. अ॒ग्निर् गृ॒हप॑तिर् गृ॒हप॑ति र॒ग्नि र॒ग्निर् गृ॒हप॑ति॒ रावि॑न्न॒ आवि॑न्नो गृ॒हप॑ति र॒ग्नि र॒ग्निर् गृ॒हप॑ति॒ रावि॑न्नः । \newline
12. गृ॒हप॑ति॒ रावि॑न्न॒ आवि॑न्नो गृ॒हप॑तिर् गृ॒हप॑ति॒ रावि॑न्न॒ इन्द्र॒ इन्द्र॒ आवि॑न्नो गृ॒हप॑तिर् गृ॒हप॑ति॒ रावि॑न्न॒ इन्द्रः॑ । \newline
13. गृ॒हप॑ति॒रिति॑ गृ॒ह - प॒तिः॒ । \newline
14. आवि॑न्न॒ इन्द्र॒ इन्द्र॒ आवि॑न्न॒ आवि॑न्न॒ इन्द्रो॑ वृ॒द्धश्र॑वा वृ॒द्धश्र॑वा॒ इन्द्र॒ आवि॑न्न॒ आवि॑न्न॒ इन्द्रो॑ वृ॒द्धश्र॑वाः । \newline
15. इन्द्रो॑ वृ॒द्धश्र॑वा वृ॒द्धश्र॑वा॒ इन्द्र॒ इन्द्रो॑ वृ॒द्धश्र॑वा॒ आवि॑न्न॒ आवि॑न्नो वृ॒द्धश्र॑वा॒ इन्द्र॒ इन्द्रो॑ वृ॒द्धश्र॑वा॒ आवि॑न्नः । \newline
16. वृ॒द्धश्र॑वा॒ आवि॑न्न॒ आवि॑न्नो वृ॒द्धश्र॑वा वृ॒द्धश्र॑वा॒ आवि॑न्नः पू॒षा पू॒षा ऽऽवि॑न्नो वृ॒द्धश्र॑वा वृ॒द्धश्र॑वा॒ आवि॑न्नः पू॒षा । \newline
17. वृ॒द्धश्र॑वा॒ इति॑ वृ॒द्ध - श्र॒वाः॒ । \newline
18. आवि॑न्नः पू॒षा पू॒षा ऽऽवि॑न्न॒ आवि॑न्नः पू॒षा वि॒श्ववे॑दा वि॒श्ववे॑दाः पू॒षा ऽऽवि॑न्न॒ आवि॑न्नः पू॒षा वि॒श्ववे॑दाः । \newline
19. पू॒षा वि॒श्ववे॑दा वि॒श्ववे॑दाः पू॒षा पू॒षा वि॒श्ववे॑दा॒ आवि॑न्ना॒ वावि॑न्नौ वि॒श्ववे॑दाः पू॒षा पू॒षा वि॒श्ववे॑दा॒ आवि॑न्नौ । \newline
20. वि॒श्ववे॑दा॒ आवि॑न्ना॒ वावि॑न्नौ वि॒श्ववे॑दा वि॒श्ववे॑दा॒ आवि॑न्नौ मि॒त्रावरु॑णौ मि॒त्रावरु॑णा॒ वावि॑न्नौ वि॒श्ववे॑दा वि॒श्ववे॑दा॒ आवि॑न्नौ मि॒त्रावरु॑णौ । \newline
21. वि॒श्ववे॑दा॒ इति॑ वि॒श्व - वे॒दाः॒ । \newline
22. आवि॑न्नौ मि॒त्रावरु॑णौ मि॒त्रावरु॑णा॒ वावि॑न्ना॒ वावि॑न्नौ मि॒त्रावरु॑णा वृता॒वृधा॑ वृता॒वृधौ॑ मि॒त्रावरु॑णा॒ वावि॑न्ना॒ वावि॑न्नौ मि॒त्रावरु॑णा वृता॒वृधौ᳚ । \newline
23. मि॒त्रावरु॑णा वृता॒वृधा॑ वृता॒वृधौ॑ मि॒त्रावरु॑णौ मि॒त्रावरु॑णा वृता॒वृधा॒ वावि॑न्ने॒ आवि॑न्ने ऋता॒वृधौ॑ मि॒त्रावरु॑णौ मि॒त्रावरु॑णा वृता॒वृधा॒ वावि॑न्ने । \newline
24. मि॒त्रावरु॑णा॒विति॑ मि॒त्रा - वरु॑णौ । \newline
25. ऋ॒ता॒वृधा॒ वावि॑न्ने॒ आवि॑न्ने ऋता॒वृधा॑ वृता॒वृधा॒ वावि॑न्ने॒ द्यावा॑पृथि॒वी द्यावा॑पृथि॒वी आवि॑न्ने ऋता॒वृधा॑ वृता॒वृधा॒ वावि॑न्ने॒ द्यावा॑पृथि॒वी । \newline
26. ऋ॒ता॒वृधा॒वित्यृ॑त - वृधौ᳚ । \newline
27. आवि॑न्ने॒ द्यावा॑पृथि॒वी द्यावा॑पृथि॒वी आवि॑न्ने॒ आवि॑न्ने॒ द्यावा॑पृथि॒वी धृ॒तव्र॑ते धृ॒तव्र॑ते॒ द्यावा॑पृथि॒वी आवि॑न्ने॒ आवि॑न्ने॒ द्यावा॑पृथि॒वी धृ॒तव्र॑ते । \newline
28. आवि॑न्ने॒ इत्यावि॑न्ने । \newline
29. द्यावा॑पृथि॒वी धृ॒तव्र॑ते धृ॒तव्र॑ते॒ द्यावा॑पृथि॒वी द्यावा॑पृथि॒वी धृ॒तव्र॑ते॒ आवि॒न्ना ऽऽवि॑न्ना धृ॒तव्र॑ते॒ द्यावा॑पृथि॒वी द्यावा॑पृथि॒वी धृ॒तव्र॑ते॒ आवि॑न्ना । \newline
30. द्यावा॑पृथि॒वी इति॒ द्यावा᳚ - पृ॒थि॒वी । \newline
31. धृ॒तव्र॑ते॒ आवि॒न्ना ऽऽवि॑न्ना धृ॒तव्र॑ते धृ॒तव्र॑ते॒ आवि॑न्ना दे॒वी दे॒व्यावि॑न्ना धृ॒तव्र॑ते धृ॒तव्र॑ते॒ आवि॑न्ना दे॒वी । \newline
32. धृ॒तव्र॑ते॒ इति॑ धृ॒त - व्र॒ते॒ । \newline
33. आवि॑न्ना दे॒वी दे॒व्यावि॒न्ना ऽऽवि॑न्ना दे॒व्यदि॑ति॒ रदि॑तिर् दे॒व्यावि॒न्ना ऽऽवि॑न्ना दे॒व्यदि॑तिः । \newline
34. दे॒व्यदि॑ति॒ रदि॑तिर् दे॒वी दे॒व्यदि॑तिर् विश्वरू॒पी वि॑श्वरू॒ प्यदि॑तिर् दे॒वी दे॒व्यदि॑तिर् विश्वरू॒पी । \newline
35. अदि॑तिर् विश्वरू॒पी वि॑श्वरू॒ प्यदि॑ति॒ रदि॑तिर् विश्वरू॒ प्यावि॑न्न॒ आवि॑न्नो विश्वरू॒ प्यदि॑ति॒ रदि॑तिर् विश्वरू॒ प्यावि॑न्नः । \newline
36. वि॒श्व॒रू॒ प्यावि॑न्न॒ आवि॑न्नो विश्वरू॒पी वि॑श्वरू॒ प्यावि॑न्नो॒ ऽय म॒य मावि॑न्नो विश्वरू॒पी वि॑श्वरू॒ प्यावि॑न्नो॒ ऽयम् । \newline
37. वि॒श्व॒रू॒पीति॑ विश्व - रू॒पी । \newline
38. आवि॑न्नो॒ ऽय म॒य मावि॑न्न॒ आवि॑न्नो॒ ऽय म॒सा व॒सा व॒य मावि॑न्न॒ आवि॑न्नो॒ ऽय म॒सौ । \newline
39. अ॒य म॒सा व॒सा व॒य म॒य म॒सा वा॑मुष्याय॒ण आ॑मुष्याय॒णो॑ ऽसा व॒य म॒य म॒सा वा॑मुष्याय॒णः । \newline
40. अ॒सा वा॑मुष्याय॒ण आ॑मुष्याय॒णो॑ ऽसा व॒सा वा॑मुष्याय॒णो᳚ ऽस्या म॒स्या मा॑मुष्याय॒णो॑ ऽसा व॒सा वा॑मुष्याय॒णो᳚ ऽस्याम् । \newline
41. आ॒मु॒ष्या॒य॒णो᳚ ऽस्या म॒स्या मा॑मुष्याय॒ण आ॑मुष्याय॒णो᳚ ऽस्यां ॅवि॒शि वि॒श्य॑स्या मा॑मुष्याय॒ण आ॑मुष्याय॒णो᳚ ऽस्यां ॅवि॒शि । \newline
42. अ॒स्यां ॅवि॒शि वि॒श्य॑स्या म॒स्यां ॅवि॒श्य॑स्मिन् न॒स्मिन्. वि॒श्य॑स्या म॒स्यां ॅवि॒श्य॑स्मिन्न् । \newline
43. वि॒श्य॑स्मिन् न॒स्मिन्. वि॒शि वि॒श्य॑स्मिन् रा॒ष्ट्रे रा॒ष्ट्रे᳚ ऽस्मिन्. वि॒शि वि॒श्य॑स्मिन् रा॒ष्ट्रे । \newline
44. अ॒स्मिन् रा॒ष्ट्रे रा॒ष्ट्रे᳚ ऽस्मिन् न॒स्मिन् रा॒ष्ट्रे म॑ह॒ते म॑ह॒ते रा॒ष्ट्रे᳚ ऽस्मिन् न॒स्मिन् रा॒ष्ट्रे म॑ह॒ते । \newline
45. रा॒ष्ट्रे म॑ह॒ते म॑ह॒ते रा॒ष्ट्रे रा॒ष्ट्रे म॑ह॒ते क्ष॒त्राय॑ क्ष॒त्राय॑ मह॒ते रा॒ष्ट्रे रा॒ष्ट्रे म॑ह॒ते क्ष॒त्राय॑ । \newline
46. म॒ह॒ते क्ष॒त्राय॑ क्ष॒त्राय॑ मह॒ते म॑ह॒ते क्ष॒त्राय॑ मह॒ते म॑ह॒ते क्ष॒त्राय॑ मह॒ते म॑ह॒ते क्ष॒त्राय॑ मह॒ते । \newline
47. क्ष॒त्राय॑ मह॒ते म॑ह॒ते क्ष॒त्राय॑ क्ष॒त्राय॑ मह॒त आधि॑पत्या॒या धि॑पत्याय मह॒ते क्ष॒त्राय॑ क्ष॒त्राय॑ मह॒त आधि॑पत्याय । \newline
48. म॒ह॒त आधि॑पत्या॒या धि॑पत्याय मह॒ते म॑ह॒त आधि॑पत्याय मह॒ते म॑ह॒त आधि॑पत्याय मह॒ते म॑ह॒त आधि॑पत्याय मह॒ते । \newline
49. आधि॑पत्याय मह॒ते म॑ह॒त आधि॑पत्या॒या धि॑पत्याय मह॒ते जान॑राज्याय॒ जान॑राज्याय मह॒त आधि॑पत्या॒या धि॑पत्याय मह॒ते जान॑राज्याय । \newline
50. आधि॑पत्या॒येत्याधि॑ - प॒त्या॒य॒ । \newline
51. म॒ह॒ते जान॑राज्याय॒ जान॑राज्याय मह॒ते म॑ह॒ते जान॑राज्यायै॒ष ए॒ष जान॑राज्याय मह॒ते म॑ह॒ते जान॑राज्यायै॒षः । \newline
52. जान॑राज्यायै॒ष ए॒ष जान॑राज्याय॒ जान॑राज्यायै॒ष वो॑ व ए॒ष जान॑राज्याय॒ जान॑राज्यायै॒ष वः॑ । \newline
53. जान॑राज्या॒येति॒ जान॑ - रा॒ज्या॒य॒ । \newline
54. ए॒ष वो॑ व ए॒ष ए॒ष वो॑ भरता भरता व ए॒ष ए॒ष वो॑ भरताः । \newline
55. वो॒ भ॒र॒ता॒ भ॒र॒ता॒ वो॒ वो॒ भ॒र॒ता॒ राजा॒ राजा॑ भरता वो वो भरता॒ राजा᳚ । \newline
56. भ॒र॒ता॒ राजा॒ राजा॑ भरता भरता॒ राजा॒ सोमः॒ सोमो॒ राजा॑ भरता भरता॒ राजा॒ सोमः॑ । \newline
57. राजा॒ सोमः॒ सोमो॒ राजा॒ राजा॒ सोमो॒ ऽस्माक॑ म॒स्माकꣳ॒॒ सोमो॒ राजा॒ राजा॒ सोमो॒ ऽस्माक᳚म् । \newline
58. सोमो॒ ऽस्माक॑ म॒स्माकꣳ॒॒ सोमः॒ सोमो॒ ऽस्माक॑म् ब्राह्म॒णाना᳚म् ब्राह्म॒णाना॑ म॒स्माकꣳ॒॒ सोमः॒ सोमो॒ ऽस्माक॑म् ब्राह्म॒णाना᳚म् । \newline
59. अ॒स्माक॑म् ब्राह्म॒णाना᳚म् ब्राह्म॒णाना॑ म॒स्माक॑ म॒स्माक॑म् ब्राह्म॒णानाꣳ॒॒ राजा॒ राजा᳚ ब्राह्म॒णाना॑ म॒स्माक॑ म॒स्माक॑म् ब्राह्म॒णानाꣳ॒॒ राजा᳚ । \newline
60. ब्रा॒ह्म॒णानाꣳ॒॒ राजा॒ राजा᳚ ब्राह्म॒णाना᳚म् ब्राह्म॒णानाꣳ॒॒ राजेन्द्र॒स्ये न्द्र॑स्य॒ राजा᳚ ब्राह्म॒णाना᳚म् ब्राह्म॒णानाꣳ॒॒ राजेन्द्र॑स्य । \newline
61. राजेन्द्र॒स्ये न्द्र॑स्य॒ राजा॒ राजेन्द्र॑स्य॒ वज्रो॒ वज्र॒ इन्द्र॑स्य॒ राजा॒ राजेन्द्र॑स्य॒ वज्रः॑ । \newline
62. इन्द्र॑स्य॒ वज्रो॒ वज्र॒ इन्द्र॒स्ये न्द्र॑स्य॒ वज्रो᳚ ऽस्यसि॒ वज्र॒ इन्द्र॒स्ये न्द्र॑स्य॒ वज्रो॑ ऽसि । \newline
\pagebreak
\markright{ TS 1.8.12.3  \hfill https://www.vedavms.in \hfill}
\addcontentsline{toc}{section}{ TS 1.8.12.3 }
\section*{ TS 1.8.12.3 }

\textbf{TS 1.8.12.3 } \newline
\textbf{Samhita Paata} \newline

वज्रो॑ऽसि॒ वार्त्र॑घ्न॒स्त्वया॒ यं ॅवृ॒त्रं ॅव॑द्ध्याच्छत्रु॒बाध॑नाः स्थ पा॒त मा᳚ प्र॒त्यञ्चं॑ पा॒त मा॑ ति॒र्यञ्च॑म॒न्वञ्चं॑ मा पात दि॒ग्भ्यो मा॑ पात॒ विश्वा᳚भ्यो मा ना॒ष्ट्राभ्यः॑ पात॒ हिर॑ण्यवर्णा-वु॒षसां᳚ ॅविरो॒केऽयः॑स्थूणा॒-वुदि॑तौ॒ सूर्य॒स्याऽऽ रो॑हतं ॅवरुण मित्र॒ गर्तं॒ तत॑श्चक्षाथा॒मदि॑तिं॒ दितिं॑ च ॥ \newline

\textbf{Pada Paata} \newline

वज्रः॑ । अ॒सि॒ । वार्त्र॑घ्न॒ इति॒ वार्त्र॑ - घ्नः॒ । त्वया᳚ । अ॒यम् । वृ॒त्रम् । व॒द्ध्या॒त् । श॒त्रु॒बाध॑ना॒ इति॑ शत्रु-बाध॑नाः । स्थ॒ । पा॒त । मा॒ । प्र॒त्यञ्च᳚म् । पा॒त । मा॒ । ति॒र्यञ्च᳚म् । अ॒न्वञ्च᳚म् । मा॒ । पा॒त॒ । दि॒ग्भ्य इति॑ दिक्-भ्यः । मा॒ । पा॒त॒ । विश्वा᳚भ्यः । मा॒ । ना॒ष्ट्राभ्यः॑ । पा॒त॒ । हिर॑ण्यवर्णा॒विति॒ हिर॑ण्य - व॒र्णौ॒ । उ॒षसा᳚म् । वि॒रो॒क इति॑ वि - रो॒के । अयः॑ स्थूणा॒विययः॑ - स्थू॒णौ॒ । उदि॑ता॒वियुत् - इ॒तौ॒ । सूर्य॑स्य । एति॑ । रो॒ह॒त॒म् । व॒रु॒ण॒ । मि॒त्र॒ । गर्त᳚म् । ततः॑ । च॒क्षा॒था॒म् । अदि॑तिम् । दिति᳚म् । च॒ ॥  \newline



\textbf{Jatai Paata} \newline

1. वज्रो᳚ ऽस्यसि॒ वज्रो॒ वज्रो॑ ऽसि । \newline
2. अ॒सि॒ वार्त्र॑घ्नो॒ वार्त्र॑घ्नो ऽस्यसि॒ वार्त्र॑घ्नः । \newline
3. वार्त्र॑घ्न॒ स्त्वया॒ त्वया॒ वार्त्र॑घ्नो॒ वार्त्र॑घ्न॒ स्त्वया᳚ । \newline
4. वार्त्र॑घ्न॒ इति॒ वार्त्र॑ - घ्नः॒ । \newline
5. त्वया॒ ऽय म॒यम् त्वया॒ त्वया॒ ऽयम् । \newline
6. अ॒यं ॅवृ॒त्रं ॅवृ॒त्र म॒य म॒यं ॅवृ॒त्रम् । \newline
7. वृ॒त्रं ॅव॑द्ध्याद् वद्ध्याद् वृ॒त्रं ॅवृ॒त्रं ॅव॑द्ध्यात् । \newline
8. व॒द्ध्या॒च् छ॒त्रु॒बाध॑नाः शत्रु॒बाध॑ना वद्ध्याद् वद्ध्याच् छत्रु॒बाध॑नाः । \newline
9. श॒त्रु॒बाध॑नाः स्थ स्थ शत्रु॒बाध॑नाः शत्रु॒बाध॑नाः स्थ । \newline
10. श॒त्रु॒बाध॑ना॒ इति॑ शत्रु - बाध॑नाः । \newline
11. स्थ॒ पा॒त पा॒त स्थ॑ स्थ पा॒त । \newline
12. पा॒त मा॑ मा पा॒त पा॒त मा᳚ । \newline
13. मा॒ प्र॒त्यञ्च॑म् प्र॒त्यञ्च॑म् मा मा प्र॒त्यञ्च᳚म् । \newline
14. प्र॒त्यञ्च॑म् पा॒त पा॒त प्र॒त्यञ्च॑म् प्र॒त्यञ्च॑म् पा॒त । \newline
15. पा॒त मा॑ मा पा॒त पा॒त मा᳚ । \newline
16. मा॒ ति॒र्यञ्च॑म् ति॒र्यञ्च॑म् मा मा ति॒र्यञ्च᳚म् । \newline
17. ति॒र्यञ्च॑ म॒न्वञ्च॑ म॒न्वञ्च॑म् ति॒र्यञ्च॑म् ति॒र्यञ्च॑ म॒न्वञ्च᳚म् । \newline
18. अ॒न्वञ्च॑म् मा मा॒ ऽन्वञ्च॑ म॒न्वञ्च॑म् मा । \newline
19. मा॒ पा॒त॒ पा॒त॒ मा॒ मा॒ पा॒त॒ । \newline
20. पा॒त॒ दि॒ग्भ्यो दि॒ग्भ्यः पा॑त पात दि॒ग्भ्यः । \newline
21. दि॒ग्भ्यो मा॑ मा दि॒ग्भ्यो दि॒ग्भ्यो मा᳚ । \newline
22. दि॒ग्भ्य इति॑ दिक् - भ्यः । \newline
23. मा॒ पा॒त॒ पा॒त॒ मा॒ मा॒ पा॒त॒ । \newline
24. पा॒त॒ विश्वा᳚भ्यो॒ विश्वा᳚भ्यः पात पात॒ विश्वा᳚भ्यः । \newline
25. विश्वा᳚भ्यो मा मा॒ विश्वा᳚भ्यो॒ विश्वा᳚भ्यो मा । \newline
26. मा॒ ना॒ष्ट्राभ्यो॑ ना॒ष्ट्राभ्यो॑ मा मा ना॒ष्ट्राभ्यः॑ । \newline
27. ना॒ष्ट्राभ्यः॑ पात पात ना॒ष्ट्राभ्यो॑ ना॒ष्ट्राभ्यः॑ पात । \newline
28. पा॒त॒ हिर॑ण्यवर्णौ॒ हिर॑ण्यवर्णौ पात पात॒ हिर॑ण्यवर्णौ । \newline
29. हिर॑ण्यवर्णा वु॒षसा॑ मु॒षसाꣳ॒॒ हिर॑ण्यवर्णौ॒ हिर॑ण्यवर्णा वु॒षसा᳚म् । \newline
30. हिर॑ण्यवर्णा॒विति॒ हिर॑ण्य - व॒र्णौ॒ । \newline
31. उ॒षसां᳚ ॅविरो॒के वि॑रो॒क उ॒षसा॑ मु॒षसां᳚ ॅविरो॒के । \newline
32. वि॒रो॒के ऽय॑स्स्थूणा॒ वय॑स्स्थूणौ विरो॒के वि॑रो॒के ऽय॑स्स्थूणौ । \newline
33. वि॒रो॒क इति॑ वि - रो॒के । \newline
34. अय॑स्स्थूणा॒ वुदि॑ता॒ वुदि॑ता॒ वय॑स्स्थूणा॒ वय॑स्स्थूणा॒ वुदि॑तौ । \newline
35. अय॑स्स्थूणा॒वित्ययः॑ - स्थू॒णौ॒ । \newline
36. उदि॑तौ॒ सूर्य॑स्य॒ सूर्य॒स्योदि॑ता॒ वुदि॑तौ॒ सूर्य॑स्य । \newline
37. उदि॑ता॒वित्युत् - इ॒तौ॒ । \newline
38. सूर्य॒स्या सूर्य॑स्य॒ सूर्य॒स्या । \newline
39. आ रो॑हतꣳ रोहत॒ मा रो॑हतम् । \newline
40. रो॒ह॒तं॒ ॅव॒रु॒ण॒ व॒रु॒ण॒ रो॒ह॒तꣳ॒॒ रो॒ह॒तं॒ ॅव॒रु॒ण॒ । \newline
41. व॒रु॒ण॒ मि॒त्र॒ मि॒त्र॒ व॒रु॒ण॒ व॒रु॒ण॒ मि॒त्र॒ । \newline
42. मि॒त्र॒ गर्त॒म् गर्त॑म् मित्र मित्र॒ गर्त᳚म् । \newline
43. गर्त॒म् तत॒ स्ततो॒ गर्त॒म् गर्त॒म् ततः॑ । \newline
44. तत॑ श्चक्षाथाम् चक्षाथा॒म् तत॒ स्तत॑ श्चक्षाथाम् । \newline
45. च॒क्षा॒था॒ मदि॑ति॒ मदि॑तिम् चक्षाथाम् चक्षाथा॒ मदि॑तिम् । \newline
46. अदि॑ति॒म् दिति॒म् दिति॒ मदि॑ति॒ मदि॑ति॒म् दिति᳚म् । \newline
47. दिति॑म् च च॒ दिति॒म् दिति॑म् च । \newline
48. चेति॑ च । \newline

\textbf{Ghana Paata } \newline

1. वज्रो᳚ ऽस्यसि॒ वज्रो॒ वज्रो॑ ऽसि॒ वार्त्र॑घ्नो॒ वार्त्र॑घ्नो ऽसि॒ वज्रो॒ वज्रो॑ ऽसि॒ वार्त्र॑घ्नः । \newline
2. अ॒सि॒ वार्त्र॑घ्नो॒ वार्त्र॑घ्नो ऽस्यसि॒ वार्त्र॑घ्न॒ स्त्वया॒ त्वया॒ वार्त्र॑घ्नो ऽस्यसि॒ वार्त्र॑घ्न॒ स्त्वया᳚ । \newline
3. वार्त्र॑घ्न॒ स्त्वया॒ त्वया॒ वार्त्र॑घ्नो॒ वार्त्र॑घ्न॒ स्त्वया॒ ऽय म॒यम् त्वया॒ वार्त्र॑घ्नो॒ वार्त्र॑घ्न॒ स्त्वया॒ ऽयम् । \newline
4. वार्त्र॑घ्न॒ इति॒ वार्त्र॑ - घ्नः॒ । \newline
5. त्वया॒ ऽय म॒यम् त्वया॒ त्वया॒ ऽयं ॅवृ॒त्रं ॅवृ॒त्र म॒यम् त्वया॒ त्वया॒ ऽयं ॅवृ॒त्रम् । \newline
6. अ॒यं ॅवृ॒त्रं ॅवृ॒त्र म॒य म॒यं ॅवृ॒त्रं ॅव॑द्ध्याद् वद्ध्याद् वृ॒त्र म॒य म॒यं ॅवृ॒त्रं ॅव॑द्ध्यात् । \newline
7. वृ॒त्रं ॅव॑द्ध्याद् वद्ध्याद् वृ॒त्रं ॅवृ॒त्रं ॅव॑द्ध्याच् छत्रु॒बाध॑नाः शत्रु॒बाध॑ना वद्ध्याद् वृ॒त्रं ॅवृ॒त्रं ॅव॑द्ध्याच् छत्रु॒बाध॑नाः । \newline
8. व॒द्ध्या॒च् छ॒त्रु॒बाध॑नाः शत्रु॒बाध॑ना वद्ध्याद् वद्ध्याच् छत्रु॒बाध॑नाः स्थ स्थ शत्रु॒बाध॑ना वद्ध्याद् वद्ध्याच् छत्रु॒बाध॑नाः स्थ । \newline
9. श॒त्रु॒बाध॑नाः स्थ स्थ शत्रु॒बाध॑नाः शत्रु॒बाध॑नाः स्थ पा॒त पा॒त स्थ॑ शत्रु॒बाध॑नाः शत्रु॒बाध॑नाः स्थ पा॒त । \newline
10. श॒त्रु॒बाध॑ना॒ इति॑ शत्रु - बाध॑नाः । \newline
11. स्थ॒ पा॒त पा॒त स्थ॑ स्थ पा॒त मा॑ मा पा॒त स्थ॑ स्थ पा॒त मा᳚ । \newline
12. पा॒त मा॑ मा पा॒त पा॒त मा᳚ प्र॒त्यञ्च॑म् प्र॒त्यञ्च॑म् मा पा॒त पा॒त मा᳚ प्र॒त्यञ्च᳚म् । \newline
13. मा॒ प्र॒त्यञ्च॑म् प्र॒त्यञ्च॑म् मा मा प्र॒त्यञ्च॑म् पा॒त पा॒त प्र॒त्यञ्च॑म् मा मा प्र॒त्यञ्च॑म् पा॒त । \newline
14. प्र॒त्यञ्च॑म् पा॒त पा॒त प्र॒त्यञ्च॑म् प्र॒त्यञ्च॑म् पा॒त मा॑ मा पा॒त प्र॒त्यञ्च॑म् प्र॒त्यञ्च॑म् पा॒त मा᳚ । \newline
15. पा॒त मा॑ मा पा॒त पा॒त मा॑ ति॒र्यञ्च॑म् ति॒र्यञ्च॑म् मा पा॒त पा॒त मा॑ ति॒र्यञ्च᳚म् । \newline
16. मा॒ ति॒र्यञ्च॑म् ति॒र्यञ्च॑म् मा मा ति॒र्यञ्च॑ म॒न्वञ्च॑ म॒न्वञ्च॑म् ति॒र्यञ्च॑म् मा मा ति॒र्यञ्च॑ म॒न्वञ्च᳚म् । \newline
17. ति॒र्यञ्च॑ म॒न्वञ्च॑ म॒न्वञ्च॑म् ति॒र्यञ्च॑म् ति॒र्यञ्च॑ म॒न्वञ्च॑म् मा मा॒ ऽन्वञ्च॑म् ति॒र्यञ्च॑म् ति॒र्यञ्च॑ म॒न्वञ्च॑म् मा । \newline
18. अ॒न्वञ्च॑म् मा मा॒ ऽन्वञ्च॑ म॒न्वञ्च॑म् मा पात पात मा॒ ऽन्वञ्च॑ म॒न्वञ्च॑म् मा पात । \newline
19. मा॒ पा॒त॒ पा॒त॒ मा॒ मा॒ पा॒त॒ दि॒ग्भ्यो दि॒ग्भ्यः पा॑त मा मा पात दि॒ग्भ्यः । \newline
20. पा॒त॒ दि॒ग्भ्यो दि॒ग्भ्यः पा॑त पात दि॒ग्भ्यो मा॑ मा दि॒ग्भ्यः पा॑त पात दि॒ग्भ्यो मा᳚ । \newline
21. दि॒ग्भ्यो मा॑ मा दि॒ग्भ्यो दि॒ग्भ्यो मा॑ पात पात मा दि॒ग्भ्यो दि॒ग्भ्यो मा॑ पात । \newline
22. दि॒ग्भ्य इति॑ दिक् - भ्यः । \newline
23. मा॒ पा॒त॒ पा॒त॒ मा॒ मा॒ पा॒त॒ विश्वा᳚भ्यो॒ विश्वा᳚भ्यः पात मा मा पात॒ विश्वा᳚भ्यः । \newline
24. पा॒त॒ विश्वा᳚भ्यो॒ विश्वा᳚भ्यः पात पात॒ विश्वा᳚भ्यो मा मा॒ विश्वा᳚भ्यः पात पात॒ विश्वा᳚भ्यो मा । \newline
25. विश्वा᳚भ्यो मा मा॒ विश्वा᳚भ्यो॒ विश्वा᳚भ्यो मा ना॒ष्ट्राभ्यो॑ ना॒ष्ट्राभ्यो॑ मा॒ विश्वा᳚भ्यो॒ विश्वा᳚भ्यो मा ना॒ष्ट्राभ्यः॑ । \newline
26. मा॒ ना॒ष्ट्राभ्यो॑ ना॒ष्ट्राभ्यो॑ मा मा ना॒ष्ट्राभ्यः॑ पात पात ना॒ष्ट्राभ्यो॑ मा मा ना॒ष्ट्राभ्यः॑ पात । \newline
27. ना॒ष्ट्राभ्यः॑ पात पात ना॒ष्ट्राभ्यो॑ ना॒ष्ट्राभ्यः॑ पात॒ हिर॑ण्यवर्णौ॒ हिर॑ण्यवर्णौ पात ना॒ष्ट्राभ्यो॑ ना॒ष्ट्राभ्यः॑ पात॒ हिर॑ण्यवर्णौ । \newline
28. पा॒त॒ हिर॑ण्यवर्णौ॒ हिर॑ण्यवर्णौ पात पात॒ हिर॑ण्यवर्णा वु॒षसा॑ मु॒षसाꣳ॒॒ हिर॑ण्यवर्णौ पात पात॒ हिर॑ण्यवर्णा वु॒षसा᳚म् । \newline
29. हिर॑ण्यवर्णा वु॒षसा॑ मु॒षसाꣳ॒॒ हिर॑ण्यवर्णौ॒ हिर॑ण्यवर्णा वु॒षसां᳚ ॅविरो॒के वि॑रो॒क उ॒षसाꣳ॒॒ हिर॑ण्यवर्णौ॒ हिर॑ण्यवर्णा वु॒षसां᳚ ॅविरो॒के । \newline
30. हिर॑ण्यवर्णा॒विति॒ हिर॑ण्य - व॒र्णौ॒ । \newline
31. उ॒षसां᳚ ॅविरो॒के वि॑रो॒क उ॒षसा॑ मु॒षसां᳚ ॅविरो॒के ऽय॑स्स्थूणा॒ वय॑स्स्थूणौ विरो॒क उ॒षसा॑ मु॒षसां᳚ ॅविरो॒के ऽय॑स्स्थूणौ । \newline
32. वि॒रो॒के ऽय॑स्स्थूणा॒ वय॑स्स्थूणौ विरो॒के वि॑रो॒के ऽय॑स्स्थूणा॒ वुदि॑ता॒ वुदि॑ता॒ वय॑स्स्थूणौ विरो॒के वि॑रो॒के ऽय॑स्स्थूणा॒ वुदि॑तौ । \newline
33. वि॒रो॒क इति॑ वि - रो॒के । \newline
34. अय॑स्स्थूणा॒ वुदि॑ता॒ वुदि॑ता॒ वय॑स्स्थूणा॒ वय॑स्स्थूणा॒ वुदि॑तौ॒ सूर्य॑स्य॒ सूर्य॒स्योदि॑ता॒ वय॑स्स्थूणा॒ वय॑स्स्थूणा॒ वुदि॑तौ॒ सूर्य॑स्य । \newline
35. अय॑स्स्थूणा॒वित्ययः॑ - स्थू॒णौ॒ । \newline
36. उदि॑तौ॒ सूर्य॑स्य॒ सूर्य॒स्योदि॑ता॒ वुदि॑तौ॒ सूर्य॒स्या सूर्य॒स्योदि॑ता॒ वुदि॑तौ॒ सूर्य॒स्या । \newline
37. उदि॑ता॒वित्युत् - इ॒तौ॒ । \newline
38. सूर्य॒स्या सूर्य॑स्य॒ सूर्य॒स्या रो॑हतꣳ रोहत॒ मा सूर्य॑स्य॒ सूर्य॒स्या रो॑हतम् । \newline
39. आ रो॑हतꣳ रोहत॒ मा रो॑हतं ॅवरुण वरुण रोहत॒ मा रो॑हतं ॅवरुण । \newline
40. रो॒ह॒तं॒ ॅव॒रु॒ण॒ व॒रु॒ण॒ रो॒ह॒तꣳ॒॒ रो॒ह॒तं॒ ॅव॒रु॒ण॒ मि॒त्र॒ मि॒त्र॒ व॒रु॒ण॒ रो॒ह॒तꣳ॒॒ रो॒ह॒तं॒ ॅव॒रु॒ण॒ मि॒त्र॒ । \newline
41. व॒रु॒ण॒ मि॒त्र॒ मि॒त्र॒ व॒रु॒ण॒ व॒रु॒ण॒ मि॒त्र॒ गर्त॒म् गर्त॑म् मित्र वरुण वरुण मित्र॒ गर्त᳚म् । \newline
42. मि॒त्र॒ गर्त॒म् गर्त॑म् मित्र मित्र॒ गर्त॒म् तत॒ स्ततो॒ गर्त॑म् मित्र मित्र॒ गर्त॒म् ततः॑ । \newline
43. गर्त॒म् तत॒ स्ततो॒ गर्त॒म् गर्त॒म् तत॑ श्चक्षाथाम् चक्षाथा॒म् ततो॒ गर्त॒म् गर्त॒म् तत॑ श्चक्षाथाम् । \newline
44. तत॑ श्चक्षाथाम् चक्षाथा॒म् तत॒ स्तत॑ श्चक्षाथा॒ मदि॑ति॒ मदि॑तिम् चक्षाथा॒म् तत॒ स्तत॑ श्चक्षाथा॒ मदि॑तिम् । \newline
45. च॒क्षा॒था॒ मदि॑ति॒ मदि॑तिम् चक्षाथाम् चक्षाथा॒ मदि॑ति॒म् दिति॒म् दिति॒ मदि॑तिम् चक्षाथाम् चक्षाथा॒ मदि॑ति॒म् दिति᳚म् । \newline
46. अदि॑ति॒म् दिति॒म् दिति॒ मदि॑ति॒ मदि॑ति॒म् दिति॑म् च च॒ दिति॒ मदि॑ति॒ मदि॑ति॒म् दिति॑म् च । \newline
47. दिति॑म् च च॒ दिति॒म् दिति॑म् च । \newline
48. चेति॑ च । \newline
\pagebreak
\markright{ TS 1.8.13.1  \hfill https://www.vedavms.in \hfill}
\addcontentsline{toc}{section}{ TS 1.8.13.1 }
\section*{ TS 1.8.13.1 }

\textbf{TS 1.8.13.1 } \newline
\textbf{Samhita Paata} \newline

स॒मिध॒मा ति॑ष्ठ गाय॒त्री त्वा॒ छन्द॑सामवतु त्रि॒वृथ्स्तोमो॑ रथन्त॒रꣳ सामा॒ग्निर् दे॒वता॒ ब्रह्म॒ द्रवि॑णमु॒ग्रामा ति॑ष्ठ त्रि॒ष्टुप् त्वा॒ छन्द॑सामवतु पञ्चद॒शः स्तोमो॑ बृ॒हथ् सामेन्द्रो॑ दे॒वता᳚ क्ष॒त्रं द्रवि॑णं ॅवि॒राज॒मा ति॑ष्ठ॒ जग॑ती त्वा॒ छन्द॑सामवतु सप्तद॒शः स्तोमो॑ वैरू॒पꣳ साम॑ म॒रुतो॑ दे॒वता॒ विड् द्रवि॑ण॒-मुदी॑ची॒मा-ति॑ष्ठानु॒ष्टुप् त्वा॒ - [ ] \newline

\textbf{Pada Paata} \newline

स॒मिध॒मिति॑ सं-इध᳚म् । एति॑ । ति॒ष्ठ॒ । गा॒य॒त्री । त्वा॒ । छन्द॑साम् । अ॒व॒तु॒ । त्रि॒वृदिति॑ त्रि-वृत् । स्तोमः॑ । र॒थ॒न्त॒रमिति॑ रथं-त॒रम् । साम॑ । अ॒ग्निः । दे॒वता᳚ । ब्रह्म॑ । द्रवि॑णम् । उ॒ग्राम् । एति॑ । ति॒ष्ठ॒ । त्रि॒ष्टुप् । त्वा॒ । छन्द॑साम् । अ॒व॒तु॒ । प॒ञ्च॒द॒श इति॑ पञ्च-द॒शः । स्तोमः॑ । बृ॒हत् । साम॑ । इन्द्रः॑ । दे॒वता᳚ । क्ष॒त्रम् । द्रवि॑णम् । वि॒राज॒मिति॑ वि-राज᳚म् । एति॑ । ति॒ष्ठ॒ । जग॑ती । त्वा॒ । छन्द॑साम् । अ॒व॒तु॒ । स॒प्त॒द॒श इति॑ सप्त-द॒शः । स्तोमः॑ । वै॒रू॒पम् । साम॑ । म॒रुतः॑ । दे॒वता᳚ । विट् । द्रवि॑णम् । उदी॑चीम् । एति॑ । ति॒ष्ठ॒ । अ॒नु॒ष्टुबित्य॑नु - स्तुप् । त्वा॒ ।  \newline



\textbf{Jatai Paata} \newline

1. स॒मिध॒ मा स॒मिधꣳ॑ स॒मिध॒ मा । \newline
2. स॒मिध॒मिति॑ सं - इध᳚म् । \newline
3. आ ति॑ष्ठ ति॒ष्ठा ति॑ष्ठ । \newline
4. ति॒ष्ठ॒ गा॒य॒त्री गा॑य॒त्री ति॑ष्ठ तिष्ठ गाय॒त्री । \newline
5. गा॒य॒त्री त्वा᳚ त्वा गाय॒त्री गा॑य॒त्री त्वा᳚ । \newline
6. त्वा॒ छन्द॑सा॒म् छन्द॑साम् त्वा त्वा॒ छन्द॑साम् । \newline
7. छन्द॑सा मवत्ववतु॒ छन्द॑सा॒म् छन्द॑सा मवतु । \newline
8. अ॒व॒तु॒ त्रि॒वृत् त्रि॒वृ द॑वत्ववतु त्रि॒वृत् । \newline
9. त्रि॒वृथ् स्तोमः॒ स्तोम॑ स्त्रि॒वृत् त्रि॒वृथ् स्तोमः॑ । \newline
10. त्रि॒वृदिति॑ त्रि - वृत् । \newline
11. स्तोमो॑ रथन्त॒रꣳ र॑थन्त॒रꣳ स्तोमः॒ स्तोमो॑ रथन्त॒रम् । \newline
12. र॒थ॒न्त॒रꣳ साम॒ साम॑ रथन्त॒रꣳ र॑थन्त॒रꣳ साम॑ । \newline
13. र॒थ॒न्त॒रमिति॑ रथं - त॒रम् । \newline
14. सामा॒ ग्नि र॒ग्निः साम॒ सामा॒ग्निः । \newline
15. अ॒ग्निर् दे॒वता॑ दे॒वता॒ ऽग्नि र॒ग्निर् दे॒वता᳚ । \newline
16. दे॒वता॒ ब्रह्म॒ ब्रह्म॑ दे॒वता॑ दे॒वता॒ ब्रह्म॑ । \newline
17. ब्रह्म॒ द्रवि॑ण॒म् द्रवि॑ण॒म् ब्रह्म॒ ब्रह्म॒ द्रवि॑णम् । \newline
18. द्रवि॑ण मु॒ग्रा मु॒ग्राम् द्रवि॑ण॒म् द्रवि॑ण मु॒ग्राम् । \newline
19. उ॒ग्रा मोग्रा मु॒ग्रा मा । \newline
20. आ ति॑ष्ठ ति॒ष्ठा ति॑ष्ठ । \newline
21. ति॒ष्ठ॒ त्रि॒ष्टुप् त्रि॒ष्टुप् ति॑ष्ठ तिष्ठ त्रि॒ष्टुप् । \newline
22. त्रि॒ष्टुप् त्वा᳚ त्वा त्रि॒ष्टुप् त्रि॒ष्टुप् त्वा᳚ । \newline
23. त्वा॒ छन्द॑सा॒म् छन्द॑साम् त्वा त्वा॒ छन्द॑साम् । \newline
24. छन्द॑सा मवत्ववतु॒ छन्द॑सा॒म् छन्द॑सा मवतु । \newline
25. अ॒व॒तु॒ प॒ञ्च॒द॒शः प॑ञ्चद॒शो॑ ऽवत्ववतु पञ्चद॒शः । \newline
26. प॒ञ्च॒द॒शः स्तोमः॒ स्तोमः॑ पञ्चद॒शः प॑ञ्चद॒शः स्तोमः॑ । \newline
27. प॒ञ्च॒द॒श इति॑ पञ्च - द॒शः । \newline
28. स्तोमो॑ बृ॒हद् बृ॒हथ् स्तोमः॒ स्तोमो॑ बृ॒हत् । \newline
29. बृ॒हथ् साम॒ साम॑ बृ॒हद् बृ॒हथ् साम॑ । \newline
30. सामे न्द्र॒ इन्द्रः॒ साम॒ सामे न्द्रः॑ । \newline
31. इन्द्रो॑ दे॒वता॑ दे॒वतेन्द्र॒ इन्द्रो॑ दे॒वता᳚ । \newline
32. दे॒वता᳚ क्ष॒त्रम् क्ष॒त्रम् दे॒वता॑ दे॒वता᳚ क्ष॒त्रम् । \newline
33. क्ष॒त्रम् द्रवि॑ण॒म् द्रवि॑णम् क्ष॒त्रम् क्ष॒त्रम् द्रवि॑णम् । \newline
34. द्रवि॑णं ॅवि॒राजं॑ ॅवि॒राज॒म् द्रवि॑ण॒म् द्रवि॑णं ॅवि॒राज᳚म् । \newline
35. वि॒राज॒ मा वि॒राजं॑ ॅवि॒राज॒ मा । \newline
36. वि॒राज॒मिति॑ वि - राज᳚म् । \newline
37. आ ति॑ष्ठ ति॒ष्ठा ति॑ष्ठ । \newline
38. ति॒ष्ठ॒ जग॑ती॒ जग॑ती तिष्ठ तिष्ठ॒ जग॑ती । \newline
39. जग॑ती त्वा त्वा॒ जग॑ती॒ जग॑ती त्वा । \newline
40. त्वा॒ छन्द॑सा॒म् छन्द॑साम् त्वा त्वा॒ छन्द॑साम् । \newline
41. छन्द॑सा मवत्ववतु॒ छन्द॑सा॒म् छन्द॑सा मवतु । \newline
42. अ॒व॒तु॒ स॒प्त॒द॒शः स॑प्तद॒शो॑ ऽवत्ववतु सप्तद॒शः । \newline
43. स॒प्त॒द॒शः स्तोमः॒ स्तोमः॑ सप्तद॒शः स॑प्तद॒शः स्तोमः॑ । \newline
44. स॒प्त॒द॒श इति॑ सप्त - द॒शः । \newline
45. स्तोमो॑ वैरू॒पं ॅवै॑रू॒पꣳ स्तोमः॒ स्तोमो॑ वैरू॒पम् । \newline
46. वै॒रू॒पꣳ साम॒ साम॑ वैरू॒पं ॅवै॑रू॒पꣳ साम॑ । \newline
47. साम॑ म॒रुतो॑ म॒रुतः॒ साम॒ साम॑ म॒रुतः॑ । \newline
48. म॒रुतो॑ दे॒वता॑ दे॒वता॑ म॒रुतो॑ म॒रुतो॑ दे॒वता᳚ । \newline
49. दे॒वता॒ विड् विड् दे॒वता॑ दे॒वता॒ विट् । \newline
50. विड् द्रवि॑ण॒म् द्रवि॑णं॒ ॅविड् विड् द्रवि॑णम् । \newline
51. द्रवि॑ण॒ मुदी॑ची॒ मुदी॑ची॒म् द्रवि॑ण॒म् द्रवि॑ण॒ मुदी॑चीम् । \newline
52. उदी॑ची॒ मोदी॑ची॒ मुदी॑ची॒ मा । \newline
53. आ ति॑ष्ठ ति॒ष्ठा ति॑ष्ठ । \newline
54. ति॒ष्ठा॒ नु॒ष्टु ब॑नु॒ष्टुप् ति॑ष्ठ तिष्ठानु॒ष्टुप् । \newline
55. अ॒नु॒ष्टुप् त्वा᳚ त्वा ऽनु॒ष्टु ब॑नु॒ष्टुप् त्वा᳚ । \newline
56. अ॒नु॒ष्टुबित्य॑नु - स्तुप् । \newline
57. त्वा॒ छन्द॑सा॒म् छन्द॑साम् त्वा त्वा॒ छन्द॑साम् । \newline

\textbf{Ghana Paata } \newline

1. स॒मिध॒ मा स॒मिधꣳ॑ स॒मिध॒ मा ति॑ष्ठ ति॒ष्ठा स॒मिधꣳ॑ स॒मिध॒ मा ति॑ष्ठ । \newline
2. स॒मिध॒मिति॑ सं - इध᳚म् । \newline
3. आ ति॑ष्ठ ति॒ष्ठा ति॑ष्ठ गाय॒त्री गा॑य॒त्री ति॒ष्ठा ति॑ष्ठ गाय॒त्री । \newline
4. ति॒ष्ठ॒ गा॒य॒त्री गा॑य॒त्री ति॑ष्ठ तिष्ठ गाय॒त्री त्वा᳚ त्वा गाय॒त्री ति॑ष्ठ तिष्ठ गाय॒त्री त्वा᳚ । \newline
5. गा॒य॒त्री त्वा᳚ त्वा गाय॒त्री गा॑य॒त्री त्वा॒ छन्द॑सा॒म् छन्द॑साम् त्वा गाय॒त्री गा॑य॒त्री त्वा॒ छन्द॑साम् । \newline
6. त्वा॒ छन्द॑सा॒म् छन्द॑साम् त्वा त्वा॒ छन्द॑सा मवत्ववतु॒ छन्द॑साम् त्वा त्वा॒ छन्द॑सा मवतु । \newline
7. छन्द॑सा मवत्ववतु॒ छन्द॑सा॒म् छन्द॑सा मवतु त्रि॒वृत् त्रि॒वृद॑वतु॒ छन्द॑सा॒म् छन्द॑सा मवतु त्रि॒वृत् । \newline
8. अ॒व॒तु॒ त्रि॒वृत् त्रि॒वृ द॑व त्ववतु त्रि॒वृथ् स्तोमः॒ स्तोम॑ स्त्रि॒वृ द॑व त्ववतु त्रि॒वृथ् स्तोमः॑ । \newline
9. त्रि॒वृथ् स्तोमः॒ स्तोम॑ स्त्रि॒वृत् त्रि॒वृथ् स्तोमो॑ रथन्त॒रꣳ र॑थन्त॒रꣳ स्तोम॑ स्त्रि॒वृत् त्रि॒वृथ् स्तोमो॑ रथन्त॒रम् । \newline
10. त्रि॒वृदिति॑ त्रि - वृत् । \newline
11. स्तोमो॑ रथन्त॒रꣳ र॑थन्त॒रꣳ स्तोमः॒ स्तोमो॑ रथन्त॒रꣳ साम॒ साम॑ रथन्त॒रꣳ स्तोमः॒ स्तोमो॑ रथन्त॒रꣳ साम॑ । \newline
12. र॒थ॒न्त॒रꣳ साम॒ साम॑ रथन्त॒रꣳ र॑थन्त॒रꣳ सामा॒ग्नि र॒ग्निः साम॑ रथन्त॒रꣳ र॑थन्त॒रꣳ सामा॒ग्निः । \newline
13. र॒थ॒न्त॒रमिति॑ रथं - त॒रम् । \newline
14. सामा॒ग्नि र॒ग्निः साम॒ सामा॒ग्निर् दे॒वता॑ दे॒वता॒ ऽग्निः साम॒ सामा॒ग्निर् दे॒वता᳚ । \newline
15. अ॒ग्निर् दे॒वता॑ दे॒वता॒ ऽग्नि र॒ग्निर् दे॒वता॒ ब्रह्म॒ ब्रह्म॑ दे॒वता॒ ऽग्नि र॒ग्निर् दे॒वता॒ ब्रह्म॑ । \newline
16. दे॒वता॒ ब्रह्म॒ ब्रह्म॑ दे॒वता॑ दे॒वता॒ ब्रह्म॒ द्रवि॑ण॒म् द्रवि॑ण॒म् ब्रह्म॑ दे॒वता॑ दे॒वता॒ ब्रह्म॒ द्रवि॑णम् । \newline
17. ब्रह्म॒ द्रवि॑ण॒म् द्रवि॑ण॒म् ब्रह्म॒ ब्रह्म॒ द्रवि॑ण मु॒ग्रा मु॒ग्राम् द्रवि॑ण॒म् ब्रह्म॒ ब्रह्म॒ द्रवि॑ण मु॒ग्राम् । \newline
18. द्रवि॑ण मु॒ग्रा मु॒ग्राम् द्रवि॑ण॒म् द्रवि॑ण मु॒ग्रा मोग्राम् द्रवि॑ण॒म् द्रवि॑ण मु॒ग्रा मा । \newline
19. उ॒ग्रा मोग्रा मु॒ग्रा मा ति॑ष्ठ ति॒ष्ठोग्रा मु॒ग्रा मा ति॑ष्ठ । \newline
20. आ ति॑ष्ठ ति॒ष्ठा ति॑ष्ठ त्रि॒ष्टुप् त्रि॒ष्टुप् ति॒ष्ठा ति॑ष्ठ त्रि॒ष्टुप् । \newline
21. ति॒ष्ठ॒ त्रि॒ष्टुप् त्रि॒ष्टुप् ति॑ष्ठ तिष्ठ त्रि॒ष्टुप् त्वा᳚ त्वा त्रि॒ष्टुप् ति॑ष्ठ तिष्ठ त्रि॒ष्टुप् त्वा᳚ । \newline
22. त्रि॒ष्टुप् त्वा᳚ त्वा त्रि॒ष्टुप् त्रि॒ष्टुप् त्वा॒ छन्द॑सा॒म् छन्द॑साम् त्वा त्रि॒ष्टुप् त्रि॒ष्टुप् त्वा॒ छन्द॑साम् । \newline
23. त्वा॒ छन्द॑सा॒म् छन्द॑साम् त्वा त्वा॒ छन्द॑सा मवत्ववतु॒ छन्द॑साम् त्वा त्वा॒ छन्द॑सा मवतु । \newline
24. छन्द॑सा मवत्ववतु॒ छन्द॑सा॒म् छन्द॑सा मवतु पञ्चद॒शः प॑ञ्चद॒शो॑ ऽवतु॒ छन्द॑सा॒म् छन्द॑सा मवतु पञ्चद॒शः । \newline
25. अ॒व॒तु॒ प॒ञ्च॒द॒शः प॑ञ्चद॒शो॑ ऽवत्ववतु पञ्चद॒शः स्तोमः॒ स्तोमः॑ पञ्चद॒शो॑ ऽवत्ववतु पञ्चद॒शः स्तोमः॑ । \newline
26. प॒ञ्च॒द॒शः स्तोमः॒ स्तोमः॑ पञ्चद॒शः प॑ञ्चद॒शः स्तोमो॑ बृ॒हद् बृ॒हथ् स्तोमः॑ पञ्चद॒शः प॑ञ्चद॒शः स्तोमो॑ बृ॒हत् । \newline
27. प॒ञ्च॒द॒श इति॑ पञ्च - द॒शः । \newline
28. स्तोमो॑ बृ॒हद् बृ॒हथ् स्तोमः॒ स्तोमो॑ बृ॒हथ् साम॒ साम॑ बृ॒हथ् स्तोमः॒ स्तोमो॑ बृ॒हथ् साम॑ । \newline
29. बृ॒हथ् साम॒ साम॑ बृ॒हद् बृ॒हथ् सामे न्द्र॒ इन्द्रः॒ साम॑ बृ॒हद् बृ॒हथ् सामे न्द्रः॑ । \newline
30. सामे न्द्र॒ इन्द्रः॒ साम॒ सामे न्द्रो॑ दे॒वता॑ दे॒वतेन्द्रः॒ साम॒ सामे न्द्रो॑ दे॒वता᳚ । \newline
31. इन्द्रो॑ दे॒वता॑ दे॒वतेन्द्र॒ इन्द्रो॑ दे॒वता᳚ क्ष॒त्रम् क्ष॒त्रम् दे॒वतेन्द्र॒ इन्द्रो॑ दे॒वता᳚ क्ष॒त्रम् । \newline
32. दे॒वता᳚ क्ष॒त्रम् क्ष॒त्रम् दे॒वता॑ दे॒वता᳚ क्ष॒त्रम् द्रवि॑ण॒म् द्रवि॑णम् क्ष॒त्रम् दे॒वता॑ दे॒वता᳚ क्ष॒त्रम् द्रवि॑णम् । \newline
33. क्ष॒त्रम् द्रवि॑ण॒म् द्रवि॑णम् क्ष॒त्रम् क्ष॒त्रम् द्रवि॑णं ॅवि॒राजं॑ ॅवि॒राज॒म् द्रवि॑णम् क्ष॒त्रम् क्ष॒त्रम् द्रवि॑णं ॅवि॒राज᳚म् । \newline
34. द्रवि॑णं ॅवि॒राजं॑ ॅवि॒राज॒म् द्रवि॑ण॒म् द्रवि॑णं ॅवि॒राज॒ मा वि॒राज॒म् द्रवि॑ण॒म् द्रवि॑णं ॅवि॒राज॒ मा । \newline
35. वि॒राज॒ मा वि॒राजं॑ ॅवि॒राज॒ मा ति॑ष्ठ ति॒ष्ठा वि॒राजं॑ ॅवि॒राज॒ मा ति॑ष्ठ । \newline
36. वि॒राज॒मिति॑ वि - राज᳚म् । \newline
37. आ ति॑ष्ठ ति॒ष्ठा ति॑ष्ठ॒ जग॑ती॒ जग॑ती ति॒ष्ठा ति॑ष्ठ॒ जग॑ती । \newline
38. ति॒ष्ठ॒ जग॑ती॒ जग॑ती तिष्ठ तिष्ठ॒ जग॑ती त्वा त्वा॒ जग॑ती तिष्ठ तिष्ठ॒ जग॑ती त्वा । \newline
39. जग॑ती त्वा त्वा॒ जग॑ती॒ जग॑ती त्वा॒ छन्द॑सा॒म् छन्द॑साम् त्वा॒ जग॑ती॒ जग॑ती त्वा॒ छन्द॑साम् । \newline
40. त्वा॒ छन्द॑सा॒म् छन्द॑साम् त्वा त्वा॒ छन्द॑सा मवत्ववतु॒ छन्द॑साम् त्वा त्वा॒ छन्द॑सा मवतु । \newline
41. छन्द॑सा मवत्ववतु॒ छन्द॑सा॒म् छन्द॑सा मवतु सप्तद॒शः स॑प्तद॒शो॑ ऽवतु॒ छन्द॑सा॒म् छन्द॑सा मवतु सप्तद॒शः । \newline
42. अ॒व॒तु॒ स॒प्त॒द॒शः स॑प्तद॒शो॑ ऽवत्ववतु सप्तद॒शः स्तोमः॒ स्तोमः॑ सप्तद॒शो॑ ऽवत्ववतु सप्तद॒शः स्तोमः॑ । \newline
43. स॒प्त॒द॒शः स्तोमः॒ स्तोमः॑ सप्तद॒शः स॑प्तद॒शः स्तोमो॑ वैरू॒पं ॅवै॑रू॒पꣳ स्तोमः॑ सप्तद॒शः स॑प्तद॒शः स्तोमो॑ वैरू॒पम् । \newline
44. स॒प्त॒द॒श इति॑ सप्त - द॒शः । \newline
45. स्तोमो॑ वैरू॒पं ॅवै॑रू॒पꣳ स्तोमः॒ स्तोमो॑ वैरू॒पꣳ साम॒ साम॑ वैरू॒पꣳ स्तोमः॒ स्तोमो॑ वैरू॒पꣳ साम॑ । \newline
46. वै॒रू॒पꣳ साम॒ साम॑ वैरू॒पं ॅवै॑रू॒पꣳ साम॑ म॒रुतो॑ म॒रुतः॒ साम॑ वैरू॒पं ॅवै॑रू॒पꣳ साम॑ म॒रुतः॑ । \newline
47. साम॑ म॒रुतो॑ म॒रुतः॒ साम॒ साम॑ म॒रुतो॑ दे॒वता॑ दे॒वता॑ म॒रुतः॒ साम॒ साम॑ म॒रुतो॑ दे॒वता᳚ । \newline
48. म॒रुतो॑ दे॒वता॑ दे॒वता॑ म॒रुतो॑ म॒रुतो॑ दे॒वता॒ विड् विड् दे॒वता॑ म॒रुतो॑ म॒रुतो॑ दे॒वता॒ विट् । \newline
49. दे॒वता॒ विड् विड् दे॒वता॑ दे॒वता॒ विड् द्रवि॑ण॒म् द्रवि॑णं॒ ॅविड् दे॒वता॑ दे॒वता॒ विड् द्रवि॑णम् । \newline
50. विड् द्रवि॑ण॒म् द्रवि॑णं॒ ॅविड् विड् द्रवि॑ण॒ मुदी॑ची॒ मुदी॑ची॒म् द्रवि॑णं॒ ॅविड् विड् द्रवि॑ण॒ मुदी॑चीम् । \newline
51. द्रवि॑ण॒ मुदी॑ची॒ मुदी॑ची॒म् द्रवि॑ण॒म् द्रवि॑ण॒ मुदी॑ची॒ मोदी॑ची॒म् द्रवि॑ण॒म् द्रवि॑ण॒ मुदी॑ची॒ मा । \newline
52. उदी॑ची॒ मोदी॑ची॒ मुदी॑ची॒ मा ति॑ष्ठ ति॒ष्ठोदी॑ची॒ मुदी॑ची॒ मा ति॑ष्ठ । \newline
53. आ ति॑ष्ठ ति॒ष्ठा ति॑ष्ठा नु॒ष्टु ब॑नु॒ष्टुप् ति॒ष्ठा ति॑ष्ठा नु॒ष्टुप् । \newline
54. ति॒ष्ठा॒ नु॒ष्टु ब॑नु॒ष्टुप् ति॑ष्ठ तिष्ठा नु॒ष्टुप् त्वा᳚ त्वा ऽनु॒ष्टुप् ति॑ष्ठ तिष्ठा नु॒ष्टुप् त्वा᳚ । \newline
55. अ॒नु॒ष्टुप् त्वा᳚ त्वा ऽनु॒ष्टु ब॑नु॒ष्टुप् त्वा॒ छन्द॑सा॒म् छन्द॑साम् त्वा ऽनु॒ष्टु ब॑नु॒ष्टुप् त्वा॒ छन्द॑साम् । \newline
56. अ॒नु॒ष्टुबित्य॑नु - स्तुप् । \newline
57. त्वा॒ छन्द॑सा॒म् छन्द॑साम् त्वा त्वा॒ छन्द॑सा मवत्ववतु॒ छन्द॑साम् त्वा त्वा॒ छन्द॑सा मवतु । \newline
\pagebreak
\markright{ TS 1.8.13.2  \hfill https://www.vedavms.in \hfill}
\addcontentsline{toc}{section}{ TS 1.8.13.2 }
\section*{ TS 1.8.13.2 }

\textbf{TS 1.8.13.2 } \newline
\textbf{Samhita Paata} \newline

छन्द॑सा-मवत्वेकविꣳ॒॒शः स्तोमो॑ वैरा॒जꣳ साम॑ मि॒त्रावरु॑णौ दे॒वता॒ बलं॒ द्रवि॑ण-मू॒र्द्ध्वामा ति॑ष्ठ प॒ङ्क्तिस्त्वा॒ छन्द॑सामवतु त्रिणवत्रयस्त्रिꣳ॒॒शौ स्तोमौ॑ शाक्वररैव॒ते साम॑नी॒ बृह॒स्पति॑र् दे॒वता॒ वर्चो॒ द्रवि॑ण-मी॒दृङ् चा᳚न्या॒दृङ् चै॑ता॒दृङ् च॑ प्रति॒दृङ् च॑ मि॒तश्च॒ संमि॑तश्च॒ सभ॑राः । शु॒क्रज्यो॑तिश्च चि॒त्रज्यो॑तिश्च स॒त्यज्यो॑तिश्च॒ ज्योति॑ष्माꣳश्च स॒त्यश्च॑र्त॒पाश्चा - [ ] \newline

\textbf{Pada Paata} \newline

छन्द॑साम् । अ॒व॒तु॒ । ए॒क॒विꣳ॒॒श इत्ये॑क-विꣳ॒॒शः । स्तोमः॑ । वै॒रा॒जम् । साम॑ । मि॒त्रावरु॑णा॒विति॑ मि॒त्रा - वरु॑णौ । दे॒वता᳚ । बल᳚म् । द्रवि॑णम् । ऊ॒द्‌र्ध्वाम् । एति॑ । ति॒ष्ठ॒ । प॒ङ्क्तिः । त्वा॒ । छन्द॑साम् । अ॒व॒तु॒ । त्रि॒ण॒व॒त्र॒य॒स्त्रिꣳ॒॒शाविति॑ त्रिणव - त्र॒य॒स्त्रिꣳ॒॒शौ । स्तोमौ᳚ । शा॒क्व॒र॒रै॒व॒ते इति॑ शाक्वर-रै॒व॒ते । साम॑नी॒ इति॑ । बृह॒स्पतिः॑ । दे॒वता᳚ । वर्चः॑ । द्रवि॑णम् । ई॒दृङ् । च॒ । अ॒न्या॒दृङ् । च॒ । ए॒ता॒दृङ् । च॒ । प्र॒ति॒दृङिति॑ प्रति - दृङ् । च॒ । मि॒तः । च॒ । संमि॑त॒ इति॒ सं - मि॒तः॒ । च॒ । सभ॑रा॒ इति॒ स - भ॒राः॒ ॥ शु॒क्रज्यो॑ति॒रिति॑ शु॒क्र - ज्यो॒तिः॒ । च॒ । चि॒त्रज्यो॑ति॒रिति॑ चि॒त्र - ज्यो॒तिः॒ । च॒ । स॒त्यज्यो॑ति॒रिति॑ स॒त्य - ज्यो॒तिः॒ । च॒ । ज्योति॑ष्मान् । च॒ । स॒त्यः । च॒ । ऋ॒त॒पा इत्यृ॑त - पाः । च॒ ।  \newline



\textbf{Jatai Paata} \newline

1. छन्द॑सा मवत्ववतु॒ छन्द॑सा॒म् छन्द॑सा मवतु । \newline
2. अ॒व॒त्वे॒क॒विꣳ॒॒श ए॑कविꣳ॒॒शो॑ ऽवत्ववत्वेकविꣳ॒॒शः । \newline
3. ए॒क॒विꣳ॒॒शः स्तोमः॒ स्तोम॑ एकविꣳ॒॒श ए॑कविꣳ॒॒शः स्तोमः॑ । \newline
4. ए॒क॒विꣳ॒॒श इत्ये॑क - विꣳ॒॒शः । \newline
5. स्तोमो॑ वैरा॒जं ॅवै॑रा॒जꣳ स्तोमः॒ स्तोमो॑ वैरा॒जम् । \newline
6. वै॒रा॒जꣳ साम॒ साम॑ वैरा॒जं ॅवै॑रा॒जꣳ साम॑ । \newline
7. साम॑ मि॒त्रावरु॑णौ मि॒त्रावरु॑णौ॒ साम॒ साम॑ मि॒त्रावरु॑णौ । \newline
8. मि॒त्रावरु॑णौ दे॒वता॑ दे॒वता॑ मि॒त्रावरु॑णौ मि॒त्रावरु॑णौ दे॒वता᳚ । \newline
9. मि॒त्रावरु॑णा॒विति॑ मि॒त्रा - वरु॑णौ । \newline
10. दे॒वता॒ बल॒म् बल॑म् दे॒वता॑ दे॒वता॒ बल᳚म् । \newline
11. बल॒म् द्रवि॑ण॒म् द्रवि॑ण॒म् बल॒म् बल॒म् द्रवि॑णम् । \newline
12. द्रवि॑ण मू॒र्द्ध्वा मू॒र्द्ध्वाम् द्रवि॑ण॒म् द्रवि॑ण मू॒र्द्ध्वाम् । \newline
13. ऊ॒र्द्ध्वा मोर्द्ध्वा मू॒र्द्ध्वा मा । \newline
14. आ ति॑ष्ठ ति॒ष्ठा ति॑ष्ठ । \newline
15. ति॒ष्ठ॒ प॒ङ्क्तिः प॒ङ्क्ति स्ति॑ष्ठ तिष्ठ प॒ङ्क्तिः । \newline
16. प॒ङ्क्ति स्त्वा᳚ त्वा प॒ङ्क्तिः प॒ङ्क्ति स्त्वा᳚ । \newline
17. त्वा॒ छन्द॑सा॒म् छन्द॑साम् त्वा त्वा॒ छन्द॑साम् । \newline
18. छन्द॑सा मवत्ववतु॒ छन्द॑सा॒म् छन्द॑सा मवतु । \newline
19. अ॒व॒तु॒ त्रि॒ण॒व॒त्र॒य॒स्त्रिꣳ॒॒शौ त्रि॑णवत्रयस्त्रिꣳ॒॒शा व॑वत्ववतु त्रिणवत्रयस्त्रिꣳ॒॒शौ । \newline
20. त्रि॒ण॒व॒त्र॒य॒स्त्रिꣳ॒॒शौ स्तोमौ॒ स्तोमौ᳚ त्रिणवत्रयस्त्रिꣳ॒॒शौ त्रि॑णवत्रयस्त्रिꣳ॒॒शौ स्तोमौ᳚ । \newline
21. त्रि॒ण॒व॒त्र॒य॒स्त्रिꣳ॒॒शाविति॑ त्रिणव - त्र॒य॒स्त्रिꣳ॒॒शौ । \newline
22. स्तोमौ॑ शाक्वररैव॒ते शा᳚क्वररैव॒ते स्तोमौ॒ स्तोमौ॑ शाक्वररैव॒ते । \newline
23. शा॒क्व॒र॒रै॒व॒ते साम॑नी॒ साम॑नी शाक्वररैव॒ते शा᳚क्वररैव॒ते साम॑नी । \newline
24. शा॒क्व॒र॒रै॒व॒ते इति॑ शाक्वर - रै॒व॒ते । \newline
25. साम॑नी॒ बृह॒स्पति॒र् बृह॒स्पतिः॒ साम॑नी॒ साम॑नी॒ बृह॒स्पतिः॑ । \newline
26. साम॑नी॒ इति॒ साम॑नी । \newline
27. बृह॒स्पति॑र् दे॒वता॑ दे॒वता॒ बृह॒स्पति॒र् बृह॒स्पति॑र् दे॒वता᳚ । \newline
28. दे॒वता॒ वर्चो॒ वर्चो॑ दे॒वता॑ दे॒वता॒ वर्चः॑ । \newline
29. वर्चो॒ द्रवि॑ण॒म् द्रवि॑णं॒ ॅवर्चो॒ वर्चो॒ द्रवि॑णम् । \newline
30. द्रवि॑ण मी॒दृङ् ङी॒दृङ् द्रवि॑ण॒म् द्रवि॑ण मी॒दृङ् । \newline
31. ई॒दृङ् च॑ चे॒दृङ् ङी॒दृङ् च॑ । \newline
32. चा॒न्या॒दृङ् ङ॑न्या॒दृङ् च॑ चान्या॒दृङ् । \newline
33. अ॒न्या॒दृङ् च॑ चान्या॒दृङ् ङ॑न्या॒दृङ् च॑ । \newline
34. चै॒ता॒दृङ् ङे॑ता॒दृङ् च॑ चैता॒दृङ् । \newline
35. ए॒ता॒दृङ् च॑ चैता॒दृङ् ङे॑ता॒दृङ् च॑ । \newline
36. च॒ प्र॒ति॒दृङ् प्र॑ति॒दृङ् च॑ च प्रति॒दृङ् । \newline
37. प्र॒ति॒दृङ् च॑ च प्रति॒दृङ् प्र॑ति॒दृङ् च॑ । \newline
38. प्र॒ति॒दृङ्ङिति॑ प्रति - दृङ् । \newline
39. च॒ मि॒तो मि॒तश्च॑ च मि॒तः । \newline
40. मि॒तश्च॑ च मि॒तो मि॒तश्च॑ । \newline
41. च॒ संमि॑तः॒ संमि॑तश्च च॒ संमि॑तः । \newline
42. संमि॑तश्च च॒ संमि॑तः॒ संमि॑तश्च । \newline
43. संमि॑त॒ इति॒ सं - मि॒तः॒ । \newline
44. च॒ सभ॑राः॒ सभ॑राश्च च॒ सभ॑राः । \newline
45. सभ॑रा॒ इति॒ स - भ॒राः॒ । \newline
46. शु॒क्रज्यो॑तिश्च च शु॒क्रज्यो॑तिः शु॒क्रज्यो॑तिश्च । \newline
47. शु॒क्रज्यो॑ति॒रिति॑ शु॒क्र - ज्यो॒तिः॒ । \newline
48. च॒ चि॒त्रज्यो॑ति श्चि॒त्रज्यो॑तिश्च च चि॒त्रज्यो॑तिः । \newline
49. चि॒त्रज्यो॑तिश्च च चि॒त्रज्यो॑ति श्चि॒त्रज्यो॑तिश्च । \newline
50. चि॒त्रज्यो॑ति॒रिति॑ चि॒त्र - ज्यो॒तिः॒ । \newline
51. च॒ स॒त्यज्यो॑तिः स॒त्यज्यो॑तिश्च च स॒त्यज्यो॑तिः । \newline
52. स॒त्यज्यो॑तिश्च च स॒त्यज्यो॑तिः स॒त्यज्यो॑तिश्च । \newline
53. स॒त्यज्यो॑ति॒रिति॑ स॒त्य - ज्यो॒तिः॒ । \newline
54. च॒ ज्योति॑ष्मा॒न् ज्योति॑ष्माꣳश्च च॒ ज्योति॑ष्मान् । \newline
55. ज्योति॑ष्माꣳश्च च॒ ज्योति॑ष्मा॒न् ज्योति॑ष्माꣳश्च । \newline
56. च॒ स॒त्यः स॒त्यश्च॑ च स॒त्यः । \newline
57. स॒त्यश्च॑ च स॒त्यः स॒त्यश्च॑ । \newline
58. च॒ र्त॒पा ऋ॑त॒पाश्च॑ च र्त॒पाः । \newline
59. ऋ॒त॒पाश्च॑ च र्त॒पा ऋ॑त॒पाश्च॑ । \newline
60. ऋ॒त॒पा इत्यृ॑त - पाः । \newline
61. चात्यꣳ॑हा॒ अत्यꣳ॑हाश्च॒ चात्यꣳ॑हाः । \newline

\textbf{Ghana Paata } \newline

1. छन्द॑सा मवत्ववतु॒ छन्द॑सा॒म् छन्द॑सा मवत्वेकविꣳ॒॒श ए॑कविꣳ॒॒शो॑ ऽवतु॒ छन्द॑सा॒म् छन्द॑सा मवत्वेकविꣳ॒॒शः । \newline
2. अ॒व॒त्वे॒क॒विꣳ॒॒श ए॑कविꣳ॒॒शो॑ ऽवत्वव त्वेकविꣳ॒॒शः स्तोमः॒ स्तोम॑ एकविꣳ॒॒शो॑ ऽवत्वव त्वेकविꣳ॒॒शः स्तोमः॑ । \newline
3. ए॒क॒विꣳ॒॒शः स्तोमः॒ स्तोम॑ एकविꣳ॒॒श ए॑कविꣳ॒॒शः स्तोमो॑ वैरा॒जं ॅवै॑रा॒जꣳ स्तोम॑ एकविꣳ॒॒श ए॑कविꣳ॒॒शः स्तोमो॑ वैरा॒जम् । \newline
4. ए॒क॒विꣳ॒॒श इत्ये॑क - विꣳ॒॒शः । \newline
5. स्तोमो॑ वैरा॒जं ॅवै॑रा॒जꣳ स्तोमः॒ स्तोमो॑ वैरा॒जꣳ साम॒ साम॑ वैरा॒जꣳ स्तोमः॒ स्तोमो॑ वैरा॒जꣳ साम॑ । \newline
6. वै॒रा॒जꣳ साम॒ साम॑ वैरा॒जं ॅवै॑रा॒जꣳ साम॑ मि॒त्रावरु॑णौ मि॒त्रावरु॑णौ॒ साम॑ वैरा॒जं ॅवै॑रा॒जꣳ साम॑ मि॒त्रावरु॑णौ । \newline
7. साम॑ मि॒त्रावरु॑णौ मि॒त्रावरु॑णौ॒ साम॒ साम॑ मि॒त्रावरु॑णौ दे॒वता॑ दे॒वता॑ मि॒त्रावरु॑णौ॒ साम॒ साम॑ मि॒त्रावरु॑णौ दे॒वता᳚ । \newline
8. मि॒त्रावरु॑णौ दे॒वता॑ दे॒वता॑ मि॒त्रावरु॑णौ मि॒त्रावरु॑णौ दे॒वता॒ बल॒म् बल॑म् दे॒वता॑ मि॒त्रावरु॑णौ मि॒त्रावरु॑णौ दे॒वता॒ बल᳚म् । \newline
9. मि॒त्रावरु॑णा॒विति॑ मि॒त्रा - वरु॑णौ । \newline
10. दे॒वता॒ बल॒म् बल॑म् दे॒वता॑ दे॒वता॒ बल॒म् द्रवि॑ण॒म् द्रवि॑ण॒म् बल॑म् दे॒वता॑ दे॒वता॒ बल॒म् द्रवि॑णम् । \newline
11. बल॒म् द्रवि॑ण॒म् द्रवि॑ण॒म् बल॒म् बल॒म् द्रवि॑ण मू॒र्द्ध्वा मू॒र्द्ध्वाम् द्रवि॑ण॒म् बल॒म् बल॒म् द्रवि॑ण मू॒र्द्ध्वाम् । \newline
12. द्रवि॑ण मू॒र्द्ध्वा मू॒र्द्ध्वाम् द्रवि॑ण॒म् द्रवि॑ण मू॒र्द्ध्वा मोर्द्ध्वाम् द्रवि॑ण॒म् द्रवि॑ण मू॒र्द्ध्वा मा । \newline
13. ऊ॒र्द्ध्वा मोर्द्ध्वा मू॒र्द्ध्वा मा ति॑ष्ठ ति॒ष्ठोर्द्ध्वा मू॒र्द्ध्वा मा ति॑ष्ठ । \newline
14. आ ति॑ष्ठ ति॒ष्ठा ति॑ष्ठ प॒ङ्क्तिः प॒ङ्क्ति स्ति॒ष्ठा ति॑ष्ठ प॒ङ्क्तिः । \newline
15. ति॒ष्ठ॒ प॒ङ्क्तिः प॒ङ्क्ति स्ति॑ष्ठ तिष्ठ प॒ङ्क्ति स्त्वा᳚ त्वा प॒ङ्क्ति स्ति॑ष्ठ तिष्ठ प॒ङ्क्ति स्त्वा᳚ । \newline
16. प॒ङ्क्ति स्त्वा᳚ त्वा प॒ङ्क्तिः प॒ङ्क्ति स्त्वा॒ छन्द॑सा॒म् छन्द॑साम् त्वा प॒ङ्क्तिः प॒ङ्क्ति स्त्वा॒ छन्द॑साम् । \newline
17. त्वा॒ छन्द॑सा॒म् छन्द॑साम् त्वा त्वा॒ छन्द॑सा मवत्ववतु॒ छन्द॑साम् त्वा त्वा॒ छन्द॑सा मवतु । \newline
18. छन्द॑सा मवत्ववतु॒ छन्द॑सा॒म् छन्द॑सा मवतु त्रिणवत्रयस्त्रिꣳ॒॒शौ त्रि॑णवत्रयस्त्रिꣳ॒॒शा व॑वतु॒ छन्द॑सा॒म् छन्द॑सा मवतु त्रिणवत्रयस्त्रिꣳ॒॒शौ । \newline
19. अ॒व॒तु॒ त्रि॒ण॒व॒त्र॒य॒स्त्रिꣳ॒॒शौ त्रि॑णवत्रयस्त्रिꣳ॒॒शा व॑वत्ववतु त्रिणवत्रयस्त्रिꣳ॒॒शौ स्तोमौ॒ स्तोमौ᳚ त्रिणवत्रयस्त्रिꣳ॒॒शा व॑वत्ववतु त्रिणवत्रयस्त्रिꣳ॒॒शौ स्तोमौ᳚ । \newline
20. त्रि॒ण॒व॒त्र॒य॒स्त्रिꣳ॒॒शौ स्तोमौ॒ स्तोमौ᳚ त्रिणवत्रयस्त्रिꣳ॒॒शौ त्रि॑णवत्रयस्त्रिꣳ॒॒शौ स्तोमौ॑ शाक्वररैव॒ते शा᳚क्वररैव॒ते स्तोमौ᳚ त्रिणवत्रयस्त्रिꣳ॒॒शौ त्रि॑णवत्रयस्त्रिꣳ॒॒शौ स्तोमौ॑ शाक्वररैव॒ते । \newline
21. त्रि॒ण॒व॒त्र॒य॒स्त्रिꣳ॒॒शाविति॑ त्रिणव - त्र॒य॒स्त्रिꣳ॒॒शौ । \newline
22. स्तोमौ॑ शाक्वररैव॒ते शा᳚क्वररैव॒ते स्तोमौ॒ स्तोमौ॑ शाक्वररैव॒ते साम॑नी॒ साम॑नी शाक्वररैव॒ते स्तोमौ॒ स्तोमौ॑ शाक्वररैव॒ते साम॑नी । \newline
23. शा॒क्व॒र॒रै॒व॒ते साम॑नी॒ साम॑नी शाक्वररैव॒ते शा᳚क्वररैव॒ते साम॑नी॒ बृह॒स्पति॒र् बृह॒स्पतिः॒ साम॑नी शाक्वररैव॒ते शा᳚क्वररैव॒ते साम॑नी॒ बृह॒स्पतिः॑ । \newline
24. शा॒क्व॒र॒रै॒व॒ते इति॑ शाक्वर - रै॒व॒ते । \newline
25. साम॑नी॒ बृह॒स्पति॒र् बृह॒स्पतिः॒ साम॑नी॒ साम॑नी॒ बृह॒स्पति॑र् दे॒वता॑ दे॒वता॒ बृह॒स्पतिः॒ साम॑नी॒ साम॑नी॒ बृह॒स्पति॑र् दे॒वता᳚ । \newline
26. साम॑नी॒ इति॒ साम॑नी । \newline
27. बृह॒स्पति॑र् दे॒वता॑ दे॒वता॒ बृह॒स्पति॒र् बृह॒स्पति॑र् दे॒वता॒ वर्चो॒ वर्चो॑ दे॒वता॒ बृह॒स्पति॒र् बृह॒स्पति॑र् दे॒वता॒ वर्चः॑ । \newline
28. दे॒वता॒ वर्चो॒ वर्चो॑ दे॒वता॑ दे॒वता॒ वर्चो॒ द्रवि॑ण॒म् द्रवि॑णं॒ ॅवर्चो॑ दे॒वता॑ दे॒वता॒ वर्चो॒ द्रवि॑णम् । \newline
29. वर्चो॒ द्रवि॑ण॒म् द्रवि॑णं॒ ॅवर्चो॒ वर्चो॒ द्रवि॑ण मी॒दृङ् ङी॒दृङ् द्रवि॑णं॒ ॅवर्चो॒ वर्चो॒ द्रवि॑ण मी॒दृङ् । \newline
30. द्रवि॑ण मी॒दृङ् ङी॒दृङ् द्रवि॑ण॒म् द्रवि॑ण मी॒दृङ् च॑ चे॒दृङ् द्रवि॑ण॒म् द्रवि॑ण मी॒दृङ् च॑ । \newline
31. ई॒दृङ् च॑ चे॒दृङ् ङी॒दृङ् चा᳚न्या॒दृङ् ङ॑न्या॒दृङ् चे॒दृङ् ङी॒दृङ् चा᳚न्या॒दृङ् । \newline
32. चा॒न्या॒दृङ् ङ॑न्या॒दृङ् च॑ चान्या॒दृङ् च॑ चान्या॒दृङ् च॑ चान्या॒दृङ् च॑ । \newline
33. अ॒न्या॒दृङ् च॑ चान्या॒दृङ् ङ॑न्या॒दृङ् चै॑ता॒दृङ् ङे॑ता॒दृङ् चा᳚न्या॒दृङ् ङ॑न्या॒दृङ् चै॑ता॒दृङ् । \newline
34. चै॒ता॒दृङ् ङे॑ता॒दृङ् च॑ चैता॒दृङ् च॑ चैता॒दृङ् च॑ चैता॒दृङ् च॑ । \newline
35. ए॒ता॒दृङ् च॑ चैता॒दृङ् ङे॑ता॒दृङ् च॑ प्रति॒दृङ् प्र॑ति॒दृङ् चै॑ता॒दृङ् ङे॑ता॒दृङ् च॑ प्रति॒दृङ् । \newline
36. च॒ प्र॒ति॒दृङ् प्र॑ति॒दृङ् च॑ च प्रति॒दृङ् च॑ च प्रति॒दृङ् च॑ च प्रति॒दृङ् च॑ । \newline
37. प्र॒ति॒दृङ् च॑ च प्रति॒दृङ् प्र॑ति॒दृङ् च॑ मि॒तो मि॒तश्च॑ प्रति॒दृङ् प्र॑ति॒दृङ् च॑ मि॒तः । \newline
38. प्र॒ति॒दृङ्ङिति॑ प्रति - दृङ् । \newline
39. च॒ मि॒तो मि॒तश्च॑ च मि॒तश्च॑ च मि॒तश्च॑ च मि॒तश्च॑ । \newline
40. मि॒तश्च॑ च मि॒तो मि॒तश्च॒ संमि॑तः॒ संमि॑तश्च मि॒तो मि॒तश्च॒ संमि॑तः । \newline
41. च॒ संमि॑तः॒ संमि॑तश्च च॒ संमि॑तश्च च॒ संमि॑तश्च च॒ संमि॑तश्च । \newline
42. संमि॑तश्च च॒ संमि॑तः॒ संमि॑तश्च॒ सभ॑राः॒ सभ॑राश्च॒ संमि॑तः॒ संमि॑तश्च॒ सभ॑राः । \newline
43. संमि॑त॒ इति॒ सं - मि॒तः॒ । \newline
44. च॒ सभ॑राः॒ सभ॑राश्च च॒ सभ॑राः । \newline
45. सभ॑रा॒ इति॒ स - भ॒राः॒ । \newline
46. शु॒क्रज्यो॑तिश्च च शु॒क्रज्यो॑तिः शु॒क्रज्यो॑तिश्च चि॒त्रज्यो॑ति श्चि॒त्रज्यो॑तिश्च शु॒क्रज्यो॑तिः शु॒क्रज्यो॑तिश्च चि॒त्रज्यो॑तिः । \newline
47. शु॒क्रज्यो॑ति॒रिति॑ शु॒क्र - ज्यो॒तिः॒ । \newline
48. च॒ चि॒त्रज्यो॑ति श्चि॒त्रज्यो॑तिश्च च चि॒त्रज्यो॑तिश्च च चि॒त्रज्यो॑तिश्च च चि॒त्रज्यो॑तिश्च । \newline
49. चि॒त्रज्यो॑तिश्च च चि॒त्रज्यो॑ति श्चि॒त्रज्यो॑तिश्च स॒त्यज्यो॑तिः स॒त्यज्यो॑तिश्च चि॒त्रज्यो॑ति श्चि॒त्रज्यो॑तिश्च स॒त्यज्यो॑तिः । \newline
50. चि॒त्रज्यो॑ति॒रिति॑ चि॒त्र - ज्यो॒तिः॒ । \newline
51. च॒ स॒त्यज्यो॑तिः स॒त्यज्यो॑तिश्च च स॒त्यज्यो॑तिश्च च स॒त्यज्यो॑तिश्च च स॒त्यज्यो॑तिश्च । \newline
52. स॒त्यज्यो॑तिश्च च स॒त्यज्यो॑तिः स॒त्यज्यो॑तिश्च॒ ज्योति॑ष्मा॒न् ज्योति॑ष्माꣳश्च स॒त्यज्यो॑तिः स॒त्यज्यो॑तिश्च॒ ज्योति॑ष्मान् । \newline
53. स॒त्यज्यो॑ति॒रिति॑ स॒त्य - ज्यो॒तिः॒ । \newline
54. च॒ ज्योति॑ष्मा॒न् ज्योति॑ष्माꣳश्च च॒ ज्योति॑ष्माꣳश्च च॒ ज्योति॑ष्माꣳश्च च॒ ज्योति॑ष्माꣳश्च । \newline
55. ज्योति॑ष्माꣳश्च च॒ ज्योति॑ष्मा॒न् ज्योति॑ष्माꣳश्च स॒त्यः स॒त्यश्च॒ ज्योति॑ष्मा॒न् ज्योति॑ष्माꣳश्च स॒त्यः । \newline
56. च॒ स॒त्यः स॒त्यश्च॑ च स॒त्यश्च॑ च स॒त्यश्च॑ च स॒त्यश्च॑ । \newline
57. स॒त्यश्च॑ च स॒त्यः स॒त्यश्च॑ र्त॒पा ऋ॑त॒पाश्च॑ स॒त्यः स॒त्यश्च॑ र्त॒पाः । \newline
58. च॒ र्त॒पा ऋ॑त॒पाश्च॑ च र्त॒पाश्च॑ च र्त॒पाश्च॑ च र्त॒पाश्च॑ । \newline
59. ऋ॒त॒पाश्च॑ च र्त॒पा ऋ॑त॒पा श्चात्यꣳ॑हा॒ अत्यꣳ॑हाश्च र्त॒पा ऋ॑त॒पा श्चात्यꣳ॑हाः । \newline
60. ऋ॒त॒पा इत्यृ॑त - पाः । \newline
61. चात्यꣳ॑हा॒ अत्यꣳ॑हाश्च॒ चात्यꣳ॑हाः । \newline
\pagebreak
\markright{ TS 1.8.13.3  \hfill https://www.vedavms.in \hfill}
\addcontentsline{toc}{section}{ TS 1.8.13.3 }
\section*{ TS 1.8.13.3 }

\textbf{TS 1.8.13.3 } \newline
\textbf{Samhita Paata} \newline

ऽत्यꣳ॑हाः । अ॒ग्नये॒ स्वाहा॒ सोमा॑य॒ स्वाहा॑ सवि॒त्रे स्वाहा॒ सर॑स्वत्यै॒ स्वाहा॑ पू॒ष्णे स्वाहा॒ बृह॒स्पत॑ये॒ स्वाहेन्द्रा॑य॒ स्वाहा॒ घोषा॑य॒ स्वाहा॒ श्लोका॑य॒ स्वाहा ऽꣳशा॑य॒ स्वाहा॒ भगा॑य॒ स्वाहा॒ क्षेत्र॑स्य॒ पत॑ये॒ स्वाहा॑ पृथि॒व्यै स्वाहा॒ ऽन्तरि॑क्षाय॒ स्वाहा॑ दि॒वे स्वाहा॒ सूर्या॑य॒ स्वाहा॑ च॒न्द्रम॑से॒ स्वाहा॒ नक्ष॑त्रेभ्यः॒ स्वाहा॒ ऽद्भ्यः स्वाहौष॑धीभ्यः॒ स्वाहा॒ वन॒स्पति॑भ्यः॒ स्वाहा॑ चराच॒रेभ्यः॒ स्वाहा॑ परिप्ल॒वेभ्यः॒ स्वाहा॑ सरीसृ॒पेभ्यः॒ स्वाहा᳚ ॥ \newline

\textbf{Pada Paata} \newline

अत्यꣳ॑हा॒ इत्यति॑-अꣳ॒॒हाः॒ ॥ अ॒ग्नये᳚ । स्वाहा᳚ । सोमा॑य । स्वाहा᳚ । स॒वि॒त्रे । स्वाहा᳚ । सर॑स्वत्यै । स्वाहा᳚ । पू॒ष्णे । स्वाहा᳚ । बृह॒स्पत॑ये । स्वाहा᳚ । इन्द्रा॑य । स्वाहा᳚ । घोषा॑य । स्वाहा᳚ । श्लोका॑य । स्वाहा᳚ । अꣳशा॑य । स्वाहा᳚ । भगा॑य । स्वाहा᳚ । क्षेत्र॑स्य । पत॑ये । स्वाहा᳚ । पृ॒थि॒व्यै । स्वाहा᳚ । अ॒न्तरि॑क्षाय । स्वाहा᳚ । दि॒वे । स्वाहा᳚ । सूर्या॑य । स्वाहा᳚ । च॒न्द्रम॑से । स्वाहा᳚ । नक्ष॑त्रेभ्यः । स्वाहा᳚ । अ॒द्भ्य इत्य॑त्-भ्यः । स्वाहा᳚ । ओष॑धीभ्य॒ इत्योष॑धि-भ्यः॒ । स्वाहा᳚ । वन॒स्पति॑भ्य॒ इति॒ वन॒स्पति॑-भ्यः॒ । स्वाहा᳚ । च॒रा॒च॒रेभ्यः॑ । स्वाहा᳚ । प॒रि॒प्ल॒वेभ्य॒ इति॑ परि - प्ल॒वेभ्यः॑ । स्वाहा᳚ । स॒री॒सृ॒पेभ्यः॑ । स्वाहा᳚ ( ) ॥  \newline



\textbf{Jatai Paata} \newline

1. अत्यꣳ॑हा॒ इत्यति॑ - अꣳ॒॒हाः॒ । \newline
2. अ॒ग्नये॒ स्वाहा॒ स्वाहा॒ ऽग्नये॒ ऽग्नये॒ स्वाहा᳚ । \newline
3. स्वाहा॒ सोमा॑य॒ सोमा॑य॒ स्वाहा॒ स्वाहा॒ सोमा॑य । \newline
4. सोमा॑य॒ स्वाहा॒ स्वाहा॒ सोमा॑य॒ सोमा॑य॒ स्वाहा᳚ । \newline
5. स्वाहा॑ सवि॒त्रे स॑वि॒त्रे स्वाहा॒ स्वाहा॑ सवि॒त्रे । \newline
6. स॒वि॒त्रे स्वाहा॒ स्वाहा॑ सवि॒त्रे स॑वि॒त्रे स्वाहा᳚ । \newline
7. स्वाहा॒ सर॑स्वत्यै॒ सर॑स्वत्यै॒ स्वाहा॒ स्वाहा॒ सर॑स्वत्यै । \newline
8. सर॑स्वत्यै॒ स्वाहा॒ स्वाहा॒ सर॑स्वत्यै॒ सर॑स्वत्यै॒ स्वाहा᳚ । \newline
9. स्वाहा॑ पू॒ष्णे पू॒ष्णे स्वाहा॒ स्वाहा॑ पू॒ष्णे । \newline
10. पू॒ष्णे स्वाहा॒ स्वाहा॑ पू॒ष्णे पू॒ष्णे स्वाहा᳚ । \newline
11. स्वाहा॒ बृह॒स्पत॑ये॒ बृह॒स्पत॑ये॒ स्वाहा॒ स्वाहा॒ बृह॒स्पत॑ये । \newline
12. बृह॒स्पत॑ये॒ स्वाहा॒ स्वाहा॒ बृह॒स्पत॑ये॒ बृह॒स्पत॑ये॒ स्वाहा᳚ । \newline
13. स्वाहेन्द्रा॒ये न्द्रा॑य॒ स्वाहा॒ स्वाहेन्द्रा॑य । \newline
14. इन्द्रा॑य॒ स्वाहा॒ स्वाहेन्द्रा॒ये न्द्रा॑य॒ स्वाहा᳚ । \newline
15. स्वाहा॒ घोषा॑य॒ घोषा॑य॒ स्वाहा॒ स्वाहा॒ घोषा॑य । \newline
16. घोषा॑य॒ स्वाहा॒ स्वाहा॒ घोषा॑य॒ घोषा॑य॒ स्वाहा᳚ । \newline
17. स्वाहा॒ श्लोका॑य॒ श्लोका॑य॒ स्वाहा॒ स्वाहा॒ श्लोका॑य । \newline
18. श्लोका॑य॒ स्वाहा॒ स्वाहा॒ श्लोका॑य॒ श्लोका॑य॒ स्वाहा᳚ । \newline
19. स्वाहाऽꣳ शा॒याꣳशा॑य॒ स्वाहा॒ स्वाहाऽꣳ शा॑य । \newline
20. अꣳशा॑य॒ स्वाहा॒ स्वाहाऽꣳ शा॒याꣳशा॑य॒ स्वाहा᳚ । \newline
21. स्वाहा॒ भगा॑य॒ भगा॑य॒ स्वाहा॒ स्वाहा॒ भगा॑य । \newline
22. भगा॑य॒ स्वाहा॒ स्वाहा॒ भगा॑य॒ भगा॑य॒ स्वाहा᳚ । \newline
23. स्वाहा॒ क्षेत्र॑स्य॒ क्षेत्र॑स्य॒ स्वाहा॒ स्वाहा॒ क्षेत्र॑स्य । \newline
24. क्षेत्र॑स्य॒ पत॑ये॒ पत॑ये॒ क्षेत्र॑स्य॒ क्षेत्र॑स्य॒ पत॑ये । \newline
25. पत॑ये॒ स्वाहा॒ स्वाहा॒ पत॑ये॒ पत॑ये॒ स्वाहा᳚ । \newline
26. स्वाहा॑ पृथि॒व्यै पृ॑थि॒व्यै स्वाहा॒ स्वाहा॑ पृथि॒व्यै । \newline
27. पृ॒थि॒व्यै स्वाहा॒ स्वाहा॑ पृथि॒व्यै पृ॑थि॒व्यै स्वाहा᳚ । \newline
28. स्वाहा॒ ऽन्तरि॑क्षाया॒ न्तरि॑क्षाय॒ स्वाहा॒ स्वाहा॒ ऽन्तरि॑क्षाय । \newline
29. अ॒न्तरि॑क्षाय॒ स्वाहा॒ स्वाहा॒ ऽन्तरि॑क्षाया॒ न्तरि॑क्षाय॒ स्वाहा᳚ । \newline
30. स्वाहा॑ दि॒वे दि॒वे स्वाहा॒ स्वाहा॑ दि॒वे । \newline
31. दि॒वे स्वाहा॒ स्वाहा॑ दि॒वे दि॒वे स्वाहा᳚ । \newline
32. स्वाहा॒ सूर्या॑य॒ सूर्या॑य॒ स्वाहा॒ स्वाहा॒ सूर्या॑य । \newline
33. सूर्या॑य॒ स्वाहा॒ स्वाहा॒ सूर्या॑य॒ सूर्या॑य॒ स्वाहा᳚ । \newline
34. स्वाहा॑ च॒न्द्रम॑से च॒न्द्रम॑से॒ स्वाहा॒ स्वाहा॑ च॒न्द्रम॑से । \newline
35. च॒न्द्रम॑से॒ स्वाहा॒ स्वाहा॑ च॒न्द्रम॑से च॒न्द्रम॑से॒ स्वाहा᳚ । \newline
36. स्वाहा॒ नक्ष॑त्रेभ्यो॒ नक्ष॑त्रेभ्यः॒ स्वाहा॒ स्वाहा॒ नक्ष॑त्रेभ्यः । \newline
37. नक्ष॑त्रेभ्यः॒ स्वाहा॒ स्वाहा॒ नक्ष॑त्रेभ्यो॒ नक्ष॑त्रेभ्यः॒ स्वाहा᳚ । \newline
38. स्वाहा॒ ऽद्भ्यो᳚ ऽद्भ्यः स्वाहा॒ स्वाहा॒ ऽद्भ्यः । \newline
39. अ॒द्भ्यः स्वाहा॒ स्वाहा॒ ऽद्भ्यो᳚ ऽद्भ्यः स्वाहा᳚ । \newline
40. अ॒द्भ्य इत्य॑त् - भ्यः । \newline
41. स्वाहौष॑धीभ्य॒ ओष॑धीभ्यः॒ स्वाहा॒ स्वाहौष॑धीभ्यः । \newline
42. ओष॑धीभ्यः॒ स्वाहा॒ स्वाहौष॑धीभ्य॒ ओष॑धीभ्यः॒ स्वाहा᳚ । \newline
43. ओष॑धीभ्य॒ इत्योष॑धि - भ्यः॒ । \newline
44. स्वाहा॒ वन॒स्पति॑भ्यो॒ वन॒स्पति॑भ्यः॒ स्वाहा॒ स्वाहा॒ वन॒स्पति॑भ्यः । \newline
45. वन॒स्पति॑भ्यः॒ स्वाहा॒ स्वाहा॒ वन॒स्पति॑भ्यो॒ वन॒स्पति॑भ्यः॒ स्वाहा᳚ । \newline
46. वन॒स्पति॑भ्य॒ इति॒ वन॒स्पति॑ - भ्यः॒ । \newline
47. स्वाहा॑ चराच॒रेभ्य॑ श्चराच॒रेभ्यः॒ स्वाहा॒ स्वाहा॑ चराच॒रेभ्यः॑ । \newline
48. च॒रा॒च॒रेभ्यः॒ स्वाहा॒ स्वाहा॑ चराच॒रेभ्य॑ श्चराच॒रेभ्यः॒ स्वाहा᳚ । \newline
49. स्वाहा॑ परिप्ल॒वेभ्यः॑ परिप्ल॒वेभ्यः॒ स्वाहा॒ स्वाहा॑ परिप्ल॒वेभ्यः॑ । \newline
50. प॒रि॒प्ल॒वेभ्यः॒ स्वाहा॒ स्वाहा॑ परिप्ल॒वेभ्यः॑ परिप्ल॒वेभ्यः॒ स्वाहा᳚ । \newline
51. प॒रि॒प्ल॒वेभ्य॒ इति॑ परि - प्ल॒वेभ्यः॑ । \newline
52. स्वाहा॑ सरीसृ॒पेभ्यः॑ सरीसृ॒पेभ्यः॒ स्वाहा॒ स्वाहा॑ सरीसृ॒पेभ्यः॑ । \newline
53. स॒री॒सृ॒पेभ्यः॒ स्वाहा॒ स्वाहा॑ सरीसृ॒पेभ्यः॑ सरीसृ॒पेभ्यः॒ स्वाहा᳚ । \newline
54. स्वाहेति॒ स्वाहा᳚ । \newline

\textbf{Ghana Paata } \newline

1. अत्यꣳ॑हा॒ इत्यति॑ - अꣳ॒॒हाः॒ । \newline
2. अ॒ग्नये॒ स्वाहा॒ स्वाहा॒ ऽग्नये॒ ऽग्नये॒ स्वाहा॒ सोमा॑य॒ सोमा॑य॒ स्वाहा॒ ऽग्नये॒ ऽग्नये॒ स्वाहा॒ सोमा॑य । \newline
3. स्वाहा॒ सोमा॑य॒ सोमा॑य॒ स्वाहा॒ स्वाहा॒ सोमा॑य॒ स्वाहा॒ स्वाहा॒ सोमा॑य॒ स्वाहा॒ स्वाहा॒ सोमा॑य॒ स्वाहा᳚ । \newline
4. सोमा॑य॒ स्वाहा॒ स्वाहा॒ सोमा॑य॒ सोमा॑य॒ स्वाहा॑ सवि॒त्रे स॑वि॒त्रे स्वाहा॒ सोमा॑य॒ सोमा॑य॒ स्वाहा॑ सवि॒त्रे । \newline
5. स्वाहा॑ सवि॒त्रे स॑वि॒त्रे स्वाहा॒ स्वाहा॑ सवि॒त्रे स्वाहा॒ स्वाहा॑ सवि॒त्रे स्वाहा॒ स्वाहा॑ सवि॒त्रे स्वाहा᳚ । \newline
6. स॒वि॒त्रे स्वाहा॒ स्वाहा॑ सवि॒त्रे स॑वि॒त्रे स्वाहा॒ सर॑स्वत्यै॒ सर॑स्वत्यै॒ स्वाहा॑ सवि॒त्रे स॑वि॒त्रे स्वाहा॒ सर॑स्वत्यै । \newline
7. स्वाहा॒ सर॑स्वत्यै॒ सर॑स्वत्यै॒ स्वाहा॒ स्वाहा॒ सर॑स्वत्यै॒ स्वाहा॒ स्वाहा॒ सर॑स्वत्यै॒ स्वाहा॒ स्वाहा॒ सर॑स्वत्यै॒ स्वाहा᳚ । \newline
8. सर॑स्वत्यै॒ स्वाहा॒ स्वाहा॒ सर॑स्वत्यै॒ सर॑स्वत्यै॒ स्वाहा॑ पू॒ष्णे पू॒ष्णे स्वाहा॒ सर॑स्वत्यै॒ सर॑स्वत्यै॒ स्वाहा॑ पू॒ष्णे । \newline
9. स्वाहा॑ पू॒ष्णे पू॒ष्णे स्वाहा॒ स्वाहा॑ पू॒ष्णे स्वाहा॒ स्वाहा॑ पू॒ष्णे स्वाहा॒ स्वाहा॑ पू॒ष्णे स्वाहा᳚ । \newline
10. पू॒ष्णे स्वाहा॒ स्वाहा॑ पू॒ष्णे पू॒ष्णे स्वाहा॒ बृह॒स्पत॑ये॒ बृह॒स्पत॑ये॒ स्वाहा॑ पू॒ष्णे पू॒ष्णे स्वाहा॒ बृह॒स्पत॑ये । \newline
11. स्वाहा॒ बृह॒स्पत॑ये॒ बृह॒स्पत॑ये॒ स्वाहा॒ स्वाहा॒ बृह॒स्पत॑ये॒ स्वाहा॒ स्वाहा॒ बृह॒स्पत॑ये॒ स्वाहा॒ स्वाहा॒ बृह॒स्पत॑ये॒ स्वाहा᳚ । \newline
12. बृह॒स्पत॑ये॒ स्वाहा॒ स्वाहा॒ बृह॒स्पत॑ये॒ बृह॒स्पत॑ये॒ स्वाहेन्द्रा॒ये न्द्रा॑य॒ स्वाहा॒ बृह॒स्पत॑ये॒ बृह॒स्पत॑ये॒ स्वाहेन्द्रा॑य । \newline
13. स्वाहेन्द्रा॒ये न्द्रा॑य॒ स्वाहा॒ स्वाहेन्द्रा॑य॒ स्वाहा॒ स्वाहेन्द्रा॑य॒ स्वाहा॒ स्वाहेन्द्रा॑य॒ स्वाहा᳚ । \newline
14. इन्द्रा॑य॒ स्वाहा॒ स्वाहेन्द्रा॒ये न्द्रा॑य॒ स्वाहा॒ घोषा॑य॒ घोषा॑य॒ स्वाहेन्द्रा॒ये न्द्रा॑य॒ स्वाहा॒ घोषा॑य । \newline
15. स्वाहा॒ घोषा॑य॒ घोषा॑य॒ स्वाहा॒ स्वाहा॒ घोषा॑य॒ स्वाहा॒ स्वाहा॒ घोषा॑य॒ स्वाहा॒ स्वाहा॒ घोषा॑य॒ स्वाहा᳚ । \newline
16. घोषा॑य॒ स्वाहा॒ स्वाहा॒ घोषा॑य॒ घोषा॑य॒ स्वाहा॒ श्लोका॑य॒ श्लोका॑य॒ स्वाहा॒ घोषा॑य॒ घोषा॑य॒ स्वाहा॒ श्लोका॑य । \newline
17. स्वाहा॒ श्लोका॑य॒ श्लोका॑य॒ स्वाहा॒ स्वाहा॒ श्लोका॑य॒ स्वाहा॒ स्वाहा॒ श्लोका॑य॒ स्वाहा॒ स्वाहा॒ श्लोका॑य॒ स्वाहा᳚ । \newline
18. श्लोका॑य॒ स्वाहा॒ स्वाहा॒ श्लोका॑य॒ श्लोका॑य॒ स्वाहा ऽꣳशा॒या ꣳशा॑य॒ स्वाहा॒ श्लोका॑य॒ श्लोका॑य॒ स्वाहा ऽꣳशा॑य । \newline
19. स्वाहा ऽꣳशा॒या ꣳशा॑य॒ स्वाहा॒ स्वाहा ऽꣳशा॑य॒ स्वाहा॒ स्वाहा ऽꣳशा॑य॒ स्वाहा॒ स्वाहा ऽꣳशा॑य॒ स्वाहा᳚ । \newline
20. अꣳशा॑य॒ स्वाहा॒ स्वाहा ऽꣳशा॒या ꣳशा॑य॒ स्वाहा॒ भगा॑य॒ भगा॑य॒ स्वाहा ऽꣳशा॒या ꣳशा॑य॒ स्वाहा॒ भगा॑य । \newline
21. स्वाहा॒ भगा॑य॒ भगा॑य॒ स्वाहा॒ स्वाहा॒ भगा॑य॒ स्वाहा॒ स्वाहा॒ भगा॑य॒ स्वाहा॒ स्वाहा॒ भगा॑य॒ स्वाहा᳚ । \newline
22. भगा॑य॒ स्वाहा॒ स्वाहा॒ भगा॑य॒ भगा॑य॒ स्वाहा॒ क्षेत्र॑स्य॒ क्षेत्र॑स्य॒ स्वाहा॒ भगा॑य॒ भगा॑य॒ स्वाहा॒ क्षेत्र॑स्य । \newline
23. स्वाहा॒ क्षेत्र॑स्य॒ क्षेत्र॑स्य॒ स्वाहा॒ स्वाहा॒ क्षेत्र॑स्य॒ पत॑ये॒ पत॑ये॒ क्षेत्र॑स्य॒ स्वाहा॒ स्वाहा॒ क्षेत्र॑स्य॒ पत॑ये । \newline
24. क्षेत्र॑स्य॒ पत॑ये॒ पत॑ये॒ क्षेत्र॑स्य॒ क्षेत्र॑स्य॒ पत॑ये॒ स्वाहा॒ स्वाहा॒ पत॑ये॒ क्षेत्र॑स्य॒ क्षेत्र॑स्य॒ पत॑ये॒ स्वाहा᳚ । \newline
25. पत॑ये॒ स्वाहा॒ स्वाहा॒ पत॑ये॒ पत॑ये॒ स्वाहा॑ पृथि॒व्यै पृ॑थि॒व्यै स्वाहा॒ पत॑ये॒ पत॑ये॒ स्वाहा॑ पृथि॒व्यै । \newline
26. स्वाहा॑ पृथि॒व्यै पृ॑थि॒व्यै स्वाहा॒ स्वाहा॑ पृथि॒व्यै स्वाहा॒ स्वाहा॑ पृथि॒व्यै स्वाहा॒ स्वाहा॑ पृथि॒व्यै स्वाहा᳚ । \newline
27. पृ॒थि॒व्यै स्वाहा॒ स्वाहा॑ पृथि॒व्यै पृ॑थि॒व्यै स्वाहा॒ ऽन्तरि॑क्षाया॒ न्तरि॑क्षाय॒ स्वाहा॑ पृथि॒व्यै पृ॑थि॒व्यै स्वाहा॒ ऽन्तरि॑क्षाय । \newline
28. स्वाहा॒ ऽन्तरि॑क्षाया॒ न्तरि॑क्षाय॒ स्वाहा॒ स्वाहा॒ ऽन्तरि॑क्षाय॒ स्वाहा॒ स्वाहा॒ ऽन्तरि॑क्षाय॒ स्वाहा॒ स्वाहा॒ ऽन्तरि॑क्षाय॒ स्वाहा᳚ । \newline
29. अ॒न्तरि॑क्षाय॒ स्वाहा॒ स्वाहा॒ ऽन्तरि॑क्षाया॒ न्तरि॑क्षाय॒ स्वाहा॑ दि॒वे दि॒वे स्वाहा॒ ऽन्तरि॑क्षाया॒ न्तरि॑क्षाय॒ स्वाहा॑ दि॒वे । \newline
30. स्वाहा॑ दि॒वे दि॒वे स्वाहा॒ स्वाहा॑ दि॒वे स्वाहा॒ स्वाहा॑ दि॒वे स्वाहा॒ स्वाहा॑ दि॒वे स्वाहा᳚ । \newline
31. दि॒वे स्वाहा॒ स्वाहा॑ दि॒वे दि॒वे स्वाहा॒ सूर्या॑य॒ सूर्या॑य॒ स्वाहा॑ दि॒वे दि॒वे स्वाहा॒ सूर्या॑य । \newline
32. स्वाहा॒ सूर्या॑य॒ सूर्या॑य॒ स्वाहा॒ स्वाहा॒ सूर्या॑य॒ स्वाहा॒ स्वाहा॒ सूर्या॑य॒ स्वाहा॒ स्वाहा॒ सूर्या॑य॒ स्वाहा᳚ । \newline
33. सूर्या॑य॒ स्वाहा॒ स्वाहा॒ सूर्या॑य॒ सूर्या॑य॒ स्वाहा॑ च॒न्द्रम॑से च॒न्द्रम॑से॒ स्वाहा॒ सूर्या॑य॒ सूर्या॑य॒ स्वाहा॑ च॒न्द्रम॑से । \newline
34. स्वाहा॑ च॒न्द्रम॑से च॒न्द्रम॑से॒ स्वाहा॒ स्वाहा॑ च॒न्द्रम॑से॒ स्वाहा॒ स्वाहा॑ च॒न्द्रम॑से॒ स्वाहा॒ स्वाहा॑ च॒न्द्रम॑से॒ स्वाहा᳚ । \newline
35. च॒न्द्रम॑से॒ स्वाहा॒ स्वाहा॑ च॒न्द्रम॑से च॒न्द्रम॑से॒ स्वाहा॒ नक्ष॑त्रेभ्यो॒ नक्ष॑त्रेभ्यः॒ स्वाहा॑ च॒न्द्रम॑से च॒न्द्रम॑से॒ स्वाहा॒ नक्ष॑त्रेभ्यः । \newline
36. स्वाहा॒ नक्ष॑त्रेभ्यो॒ नक्ष॑त्रेभ्यः॒ स्वाहा॒ स्वाहा॒ नक्ष॑त्रेभ्यः॒ स्वाहा॒ स्वाहा॒ नक्ष॑त्रेभ्यः॒ स्वाहा॒ स्वाहा॒ नक्ष॑त्रेभ्यः॒ स्वाहा᳚ । \newline
37. नक्ष॑त्रेभ्यः॒ स्वाहा॒ स्वाहा॒ नक्ष॑त्रेभ्यो॒ नक्ष॑त्रेभ्यः॒ स्वाहा॒ ऽद्भ्यो᳚ ऽद्भ्यः स्वाहा॒ नक्ष॑त्रेभ्यो॒ नक्ष॑त्रेभ्यः॒ स्वाहा॒ ऽद्भ्यः । \newline
38. स्वाहा॒ ऽद्भ्यो᳚ ऽद्भ्यः स्वाहा॒ स्वाहा॒ ऽद्भ्यः स्वाहा॒ स्वाहा॒ ऽद्भ्यः स्वाहा॒ स्वाहा॒ ऽद्भ्यः स्वाहा᳚ । \newline
39. अ॒द्भ्यः स्वाहा॒ स्वाहा॒ ऽद्भ्यो᳚ ऽद्भ्यः स्वाहौष॑धीभ्य॒ ओष॑धीभ्यः॒ स्वाहा॒ ऽद्भ्यो᳚ ऽद्भ्यः स्वाहौष॑धीभ्यः । \newline
40. अ॒द्भ्य इत्य॑त् - भ्यः । \newline
41. स्वाहौष॑धीभ्य॒ ओष॑धीभ्यः॒ स्वाहा॒ स्वाहौष॑धीभ्यः॒ स्वाहा॒ स्वाहौष॑धीभ्यः॒ स्वाहा॒ स्वाहौष॑धीभ्यः॒ स्वाहा᳚ । \newline
42. ओष॑धीभ्यः॒ स्वाहा॒ स्वाहौष॑धीभ्य॒ ओष॑धीभ्यः॒ स्वाहा॒ वन॒स्पति॑भ्यो॒ वन॒स्पति॑भ्यः॒ स्वाहौष॑धीभ्य॒ ओष॑धीभ्यः॒ स्वाहा॒ वन॒स्पति॑भ्यः । \newline
43. ओष॑धीभ्य॒ इत्योष॑धि - भ्यः॒ । \newline
44. स्वाहा॒ वन॒स्पति॑भ्यो॒ वन॒स्पति॑भ्यः॒ स्वाहा॒ स्वाहा॒ वन॒स्पति॑भ्यः॒ स्वाहा॒ स्वाहा॒ वन॒स्पति॑भ्यः॒ स्वाहा॒ स्वाहा॒ वन॒स्पति॑भ्यः॒ स्वाहा᳚ । \newline
45. वन॒स्पति॑भ्यः॒ स्वाहा॒ स्वाहा॒ वन॒स्पति॑भ्यो॒ वन॒स्पति॑भ्यः॒ स्वाहा॑ चराच॒रेभ्य॑ श्चराच॒रेभ्यः॒ स्वाहा॒ वन॒स्पति॑भ्यो॒ वन॒स्पति॑भ्यः॒ स्वाहा॑ चराच॒रेभ्यः॑ । \newline
46. वन॒स्पति॑भ्य॒ इति॒ वन॒स्पति॑ - भ्यः॒ । \newline
47. स्वाहा॑ चराच॒रेभ्य॑ श्चराच॒रेभ्यः॒ स्वाहा॒ स्वाहा॑ चराच॒रेभ्यः॒ स्वाहा॒ स्वाहा॑ चराच॒रेभ्यः॒ स्वाहा॒ स्वाहा॑ चराच॒रेभ्यः॒ स्वाहा᳚ । \newline
48. च॒रा॒च॒रेभ्यः॒ स्वाहा॒ स्वाहा॑ चराच॒रेभ्य॑ श्चराच॒रेभ्यः॒ स्वाहा॑ परिप्ल॒वेभ्यः॑ परिप्ल॒वेभ्यः॒ स्वाहा॑ चराच॒रेभ्य॑ श्चराच॒रेभ्यः॒ स्वाहा॑ परिप्ल॒वेभ्यः॑ । \newline
49. स्वाहा॑ परिप्ल॒वेभ्यः॑ परिप्ल॒वेभ्यः॒ स्वाहा॒ स्वाहा॑ परिप्ल॒वेभ्यः॒ स्वाहा॒ स्वाहा॑ परिप्ल॒वेभ्यः॒ स्वाहा॒ स्वाहा॑ परिप्ल॒वेभ्यः॒ स्वाहा᳚ । \newline
50. प॒रि॒प्ल॒वेभ्यः॒ स्वाहा॒ स्वाहा॑ परिप्ल॒वेभ्यः॑ परिप्ल॒वेभ्यः॒ स्वाहा॑ सरीसृ॒पेभ्यः॑ सरीसृ॒पेभ्यः॒ स्वाहा॑ परिप्ल॒वेभ्यः॑ परिप्ल॒वेभ्यः॒ स्वाहा॑ सरीसृ॒पेभ्यः॑ । \newline
51. प॒रि॒प्ल॒वेभ्य॒ इति॑ परि - प्ल॒वेभ्यः॑ । \newline
52. स्वाहा॑ सरीसृ॒पेभ्यः॑ सरीसृ॒पेभ्यः॒ स्वाहा॒ स्वाहा॑ सरीसृ॒पेभ्यः॒ स्वाहा॒ स्वाहा॑ सरीसृ॒पेभ्यः॒ स्वाहा॒ स्वाहा॑ सरीसृ॒पेभ्यः॒ स्वाहा᳚ । \newline
53. स॒री॒सृ॒पेभ्यः॒ स्वाहा॒ स्वाहा॑ सरीसृ॒पेभ्यः॑ सरीसृ॒पेभ्यः॒ स्वाहा᳚ । \newline
54. स्वाहेति॒ स्वाहा᳚ । \newline
\pagebreak
\markright{ TS 1.8.14.1  \hfill https://www.vedavms.in \hfill}
\addcontentsline{toc}{section}{ TS 1.8.14.1 }
\section*{ TS 1.8.14.1 }

\textbf{TS 1.8.14.1 } \newline
\textbf{Samhita Paata} \newline

सोम॑स्य॒ त्विषि॑रसि॒ तवे॑व मे॒ त्विषि॑र् भूयाद॒मृत॑मसि मृ॒त्योर् मा॑ पाहि दि॒द्योन्मा॑ पा॒ह्यवे᳚ष्टा दन्द॒शूका॒ निर॑स्तं॒ नमु॑चेः॒ शिरः॑ ॥ सोमो॒ राजा॒ वरु॑णो दे॒वा ध॑र्म॒सुव॑श्च॒ ये । ते ते॒ वाचꣳ॑ सुवन्तां॒ ते ते᳚ प्रा॒णꣳ सु॑वन्तां॒ ते ते॒ चक्षुः॑ सुवन्तां॒ ते ते॒ श्रोत्रꣳ॑ सुवन्ताꣳ॒॒ सोम॑स्य त्वा द्यु॒म्नेना॒भि षि॑ञ्चाम्य॒ग्ने - [ ] \newline

\textbf{Pada Paata} \newline

सोम॑स्य । त्विषिः॑ । अ॒सि॒ । तव॑ । इ॒व॒ । मे॒ । त्विषिः॑ । भू॒या॒त् । अ॒मृत᳚म् । अ॒सि॒ । मृ॒त्योः । मा॒ । पा॒हि॒ । दि॒द्योत् । मा॒ । पा॒हि॒ । अवे᳚ष्टा॒ इत्यव॑-इ॒ष्टाः॒ । द॒न्द॒शूकाः᳚ । निर॑स्त॒मिति॒ निः-अ॒स्त॒म् । नमु॑चेः । शिरः॑ ॥ सोमः॑ । राजा᳚ । वरु॑णः । दे॒वाः । ध॒र्म॒सुव॒ इति॑ धर्म -सुवः॑ । च॒ । ये ॥ ते । ते॒ । वाच᳚म् । सु॒व॒न्ता॒म् । ते । ते॒ । प्रा॒णमिति॑ प्र-अ॒नम् । सु॒व॒न्ता॒म् । ते । ते॒ । चक्षुः॑ । सु॒व॒न्ता॒म् । ते । ते॒ । श्रोत्र᳚म् । सु॒व॒न्ता॒म् । सोम॑स्य । त्वा॒ । द्यु॒म्नेन॑ । अ॒भीति॑ । सि॒ञ्चा॒मि॒ । अ॒ग्नेः ।  \newline



\textbf{Jatai Paata} \newline

1. सोम॑स्य॒ त्विषि॒ स्त्विषिः॒ सोम॑स्य॒ सोम॑स्य॒ त्विषिः॑ । \newline
2. त्विषि॑ रस्यसि॒ त्विषि॒ स्त्विषि॑ रसि । \newline
3. अ॒सि॒ तव॒ तवा᳚स्यसि॒ तव॑ । \newline
4. तवे॑ वे व॒ तव॒ तवे॑ व । \newline
5. इ॒व॒ मे॒ म॒ इ॒वे॒ व॒ मे॒ । \newline
6. मे॒ त्विषि॒ स्त्विषि॑र् मे मे॒ त्विषिः॑ । \newline
7. त्विषि॑र् भूयाद् भूया॒त् त्विषि॒ स्त्विषि॑र् भूयात् । \newline
8. भू॒या॒ द॒मृत॑ म॒मृत॑म् भूयाद् भूया द॒मृत᳚म् । \newline
9. अ॒मृत॑ मस्य स्य॒मृत॑ म॒मृत॑ मसि । \newline
10. अ॒सि॒ मृ॒त्योर् मृ॒त्यो र॑स्यसि मृ॒त्योः । \newline
11. मृ॒त्योर् मा॑ मा मृ॒त्योर् मृ॒त्योर् मा᳚ । \newline
12. मा॒ पा॒हि॒ पा॒हि॒ मा॒ मा॒ पा॒हि॒ । \newline
13. पा॒हि॒ दि॒द्योद् दि॒द्योत् पा॑हि पाहि दि॒द्योत् । \newline
14. दि॒द्योन् मा॑ मा दि॒द्योद् दि॒द्योन् मा᳚ । \newline
15. मा॒ पा॒हि॒ पा॒हि॒ मा॒ मा॒ पा॒हि॒ । \newline
16. पा॒ह्यवे᳚ष्टा॒ अवे᳚ष्टाः पाहि पा॒ह्यवे᳚ष्टाः । \newline
17. अवे᳚ष्टा दन्द॒शूका॑ दन्द॒शूका॒ अवे᳚ष्टा॒ अवे᳚ष्टा दन्द॒शूकाः᳚ । \newline
18. अवे᳚ष्टा॒ इत्यव॑ - इ॒ष्टाः॒ । \newline
19. द॒न्द॒शूका॒ निर॑स्त॒म् निर॑स्तम् दन्द॒शूका॑ दन्द॒शूका॒ निर॑स्तम् । \newline
20. निर॑स्त॒म् नमु॑चे॒र् नमु॑चे॒र् निर॑स्त॒म् निर॑स्त॒म् नमु॑चेः । \newline
21. निर॑स्त॒मिति॒ निः - अ॒स्त॒म् । \newline
22. नमु॑चेः॒ शिरः॒ शिरो॒ नमु॑चे॒र् नमु॑चेः॒ शिरः॑ । \newline
23. शिर॒ इति॒ शिरः॑ । \newline
24. सोमो॒ राजा॒ राजा॒ सोमः॒ सोमो॒ राजा᳚ । \newline
25. राजा॒ वरु॑णो॒ वरु॑णो॒ राजा॒ राजा॒ वरु॑णः । \newline
26. वरु॑णो दे॒वा दे॒वा वरु॑णो॒ वरु॑णो दे॒वाः । \newline
27. दे॒वा ध॑र्म॒सुवो॑ धर्म॒सुवो॑ दे॒वा दे॒वा ध॑र्म॒सुवः॑ । \newline
28. ध॒र्म॒सुव॑श्च च धर्म॒सुवो॑ धर्म॒सुव॑श्च । \newline
29. ध॒र्म॒सुव॒ इति॑ धर्म - सुवः॑ । \newline
30. च॒ ये ये च॑ च॒ ये । \newline
31. य इति॒ ये । \newline
32. ते ते॑ ते॒ ते ते ते᳚ । \newline
33. ते॒ वाचं॒ ॅवाच॑म् ते ते॒ वाच᳚म् । \newline
34. वाचꣳ॑ सुवन्ताꣳ सुवन्तां॒ ॅवाचं॒ ॅवाचꣳ॑ सुवन्ताम् । \newline
35. सु॒व॒न्ता॒म् ते ते सु॑वन्ताꣳ सुवन्ता॒म् ते । \newline
36. ते ते॑ ते॒ ते ते ते᳚ । \newline
37. ते॒ प्रा॒णम् प्रा॒णम् ते॑ ते प्रा॒णम् । \newline
38. प्रा॒णꣳ सु॑वन्ताꣳ सुवन्ताम् प्रा॒णम् प्रा॒णꣳ सु॑वन्ताम् । \newline
39. प्रा॒णमिति॑ प्र - अ॒नम् । \newline
40. सु॒व॒न्ता॒म् ते ते सु॑वन्ताꣳ सुवन्ता॒म् ते । \newline
41. ते ते॑ ते॒ ते ते ते᳚ । \newline
42. ते॒ चक्षु॒ श्चक्षु॑ स्ते ते॒ चक्षुः॑ । \newline
43. चक्षुः॑ सुवन्ताꣳ सुवन्ता॒म् चक्षु॒ श्चक्षुः॑ सुवन्ताम् । \newline
44. सु॒व॒न्ता॒म् ते ते सु॑वन्ताꣳ सुवन्ता॒म् ते । \newline
45. ते ते॑ ते॒ ते ते ते᳚ । \newline
46. ते॒ श्रोत्रꣳ॒॒ श्रोत्र॑म् ते ते॒ श्रोत्र᳚म् । \newline
47. श्रोत्रꣳ॑ सुवन्ताꣳ सुवन्ताꣳ॒॒ श्रोत्रꣳ॒॒ श्रोत्रꣳ॑ सुवन्ताम् । \newline
48. सु॒व॒न्ताꣳ॒॒ सोम॑स्य॒ सोम॑स्य सुवन्ताꣳ सुवन्ताꣳ॒॒ सोम॑स्य । \newline
49. सोम॑स्य त्वा त्वा॒ सोम॑स्य॒ सोम॑स्य त्वा । \newline
50. त्वा॒ द्यु॒म्नेन॑ द्यु॒म्नेन॑ त्वा त्वा द्यु॒म्नेन॑ । \newline
51. द्यु॒म्नेना॒ भ्य॑भि द्यु॒म्नेन॑ द्यु॒म्नेना॒भि । \newline
52. अ॒भि षि॑ञ्चामि सिञ्चा म्य॒भ्य॑भि षि॑ञ्चामि । \newline
53. सि॒ञ्चा॒ म्य॒ग्ने र॒ग्नेः सि॑ञ्चामि सिञ्चा म्य॒ग्नेः । \newline
54. अ॒ग्ने स्तेज॑सा॒ तेज॑सा॒ ऽग्ने र॒ग्ने स्तेज॑सा । \newline

\textbf{Ghana Paata } \newline

1. सोम॑स्य॒ त्विषि॒ स्त्विषिः॒ सोम॑स्य॒ सोम॑स्य॒ त्विषि॑ रस्यसि॒ त्विषिः॒ सोम॑स्य॒ सोम॑स्य॒ त्विषि॑रसि । \newline
2. त्विषि॑ रस्यसि॒ त्विषि॒ स्त्विषि॑रसि॒ तव॒ तवा॑सि॒ त्विषि॒ स्त्विषि॑रसि॒ तव॑ । \newline
3. अ॒सि॒ तव॒ तवा᳚स्यसि॒ तवे॑ वे व॒ तवा᳚स्यसि॒ तवे॑ व । \newline
4. तवे॑ वे व॒ तव॒ तवे॑ व मे म इव॒ तव॒ तवे॑ व मे । \newline
5. इ॒व॒ मे॒ म॒ इ॒वे॒ व॒ मे॒ त्विषि॒ स्त्विषि॑र् म इवे व मे॒ त्विषिः॑ । \newline
6. मे॒ त्विषि॒ स्त्विषि॑र् मे मे॒ त्विषि॑र् भूयाद् भूया॒त् त्विषि॑र् मे मे॒ त्विषि॑र् भूयात् । \newline
7. त्विषि॑र् भूयाद् भूया॒त् त्विषि॒ स्त्विषि॑र् भूयाद॒मृत॑ म॒मृत॑म् भूया॒त् त्विषि॒ स्त्विषि॑र् भूयाद॒मृत᳚म् । \newline
8. भू॒या॒ द॒मृत॑ म॒मृत॑म् भूयाद् भूयाद॒मृत॑ मस्य स्य॒मृत॑म् भूयाद् भूया द॒मृत॑ मसि । \newline
9. अ॒मृत॑ मस्य स्य॒मृत॑ म॒मृत॑ मसि मृ॒त्योर् मृ॒त्यो र॑स्य॒मृत॑ म॒मृत॑ मसि मृ॒त्योः । \newline
10. अ॒सि॒ मृ॒त्योर् मृ॒त्यो र॑स्यसि मृ॒त्योर् मा॑ मा मृ॒त्यो र॑स्यसि मृ॒त्योर् मा᳚ । \newline
11. मृ॒त्योर् मा॑ मा मृ॒त्योर् मृ॒त्योर् मा॑ पाहि पाहि मा मृ॒त्योर् मृ॒त्योर् मा॑ पाहि । \newline
12. मा॒ पा॒हि॒ पा॒हि॒ मा॒ मा॒ पा॒हि॒ दि॒द्योद् दि॒द्योत् पा॑हि मा मा पाहि दि॒द्योत् । \newline
13. पा॒हि॒ दि॒द्योद् दि॒द्योत् पा॑हि पाहि दि॒द्योन् मा॑ मा दि॒द्योत् पा॑हि पाहि दि॒द्योन् मा᳚ । \newline
14. दि॒द्योन् मा॑ मा दि॒द्योद् दि॒द्योन् मा॑ पाहि पाहि मा दि॒द्योद् दि॒द्योन् मा॑ पाहि । \newline
15. मा॒ पा॒हि॒ पा॒हि॒ मा॒ मा॒ पा॒ह्यवे᳚ष्टा॒ अवे᳚ष्टाः पाहि मा मा पा॒ह्यवे᳚ष्टाः । \newline
16. पा॒ह्यवे᳚ष्टा॒ अवे᳚ष्टाः पाहि पा॒ह्यवे᳚ष्टा दन्द॒शूका॑ दन्द॒शूका॒ अवे᳚ष्टाः पाहि पा॒ह्यवे᳚ष्टा दन्द॒शूकाः᳚ । \newline
17. अवे᳚ष्टा दन्द॒शूका॑ दन्द॒शूका॒ अवे᳚ष्टा॒ अवे᳚ष्टा दन्द॒शूका॒ निर॑स्त॒न् निर॑स्तम् दन्द॒शूका॒ अवे᳚ष्टा॒ अवे᳚ष्टा दन्द॒शूका॒ निर॑स्तम् । \newline
18. अवे᳚ष्टा॒ इत्यव॑ - इ॒ष्टाः॒ । \newline
19. द॒न्द॒शूका॒ निर॑स्त॒न् निर॑स्तम् दन्द॒शूका॑ दन्द॒शूका॒ निर॑स्त॒न् नमु॑चे॒र् नमु॑चे॒र् निर॑स्तम् दन्द॒शूका॑ दन्द॒शूका॒ निर॑स्त॒न् नमु॑चेः । \newline
20. निर॑स्त॒न् नमु॑चे॒र् नमु॑चे॒र् निर॑स्त॒न् निर॑स्त॒न् नमु॑चेः॒ शिरः॒ शिरो॒ नमु॑चे॒र् निर॑स्त॒न् निर॑स्त॒न् नमु॑चेः॒ शिरः॑ । \newline
21. निर॑स्त॒मिति॒ निः - अ॒स्त॒म् । \newline
22. नमु॑चेः॒ शिरः॒ शिरो॒ नमु॑चे॒र् नमु॑चेः॒ शिरः॑ । \newline
23. शिर॒ इति॒ शिरः॑ । \newline
24. सोमो॒ राजा॒ राजा॒ सोमः॒ सोमो॒ राजा॒ वरु॑णो॒ वरु॑णो॒ राजा॒ सोमः॒ सोमो॒ राजा॒ वरु॑णः । \newline
25. राजा॒ वरु॑णो॒ वरु॑णो॒ राजा॒ राजा॒ वरु॑णो दे॒वा दे॒वा वरु॑णो॒ राजा॒ राजा॒ वरु॑णो दे॒वाः । \newline
26. वरु॑णो दे॒वा दे॒वा वरु॑णो॒ वरु॑णो दे॒वा ध॑र्म॒सुवो॑ धर्म॒सुवो॑ दे॒वा वरु॑णो॒ वरु॑णो दे॒वा ध॑र्म॒सुवः॑ । \newline
27. दे॒वा ध॑र्म॒सुवो॑ धर्म॒सुवो॑ दे॒वा दे॒वा ध॑र्म॒सुव॑श्च च धर्म॒सुवो॑ दे॒वा दे॒वा ध॑र्म॒सुव॑श्च । \newline
28. ध॒र्म॒सुव॑श्च च धर्म॒सुवो॑ धर्म॒सुव॑श्च॒ ये ये च॑ धर्म॒सुवो॑ धर्म॒सुव॑श्च॒ ये । \newline
29. ध॒र्म॒सुव॒ इति॑ धर्म - सुवः॑ । \newline
30. च॒ ये ये च॑ च॒ ये । \newline
31. य इति॒ ये । \newline
32. ते ते॑ ते॒ ते ते ते॒ वाचं॒ ॅवाच॑म् ते॒ ते ते ते॒ वाच᳚म् । \newline
33. ते॒ वाचं॒ ॅवाच॑म् ते ते॒ वाचꣳ॑ सुवन्ताꣳ सुवन्तां॒ ॅवाच॑म् ते ते॒ वाचꣳ॑ सुवन्ताम् । \newline
34. वाचꣳ॑ सुवन्ताꣳ सुवन्तां॒ ॅवाचं॒ ॅवाचꣳ॑ सुवन्ता॒म् ते ते सु॑वन्तां॒ ॅवाचं॒ ॅवाचꣳ॑ सुवन्ता॒म् ते । \newline
35. सु॒व॒न्ता॒म् ते ते सु॑वन्ताꣳ सुवन्ता॒म् ते ते॑ ते॒ ते सु॑वन्ताꣳ सुवन्ता॒म् ते ते᳚ । \newline
36. ते ते॑ ते॒ ते ते ते᳚ प्रा॒णम् प्रा॒णम् ते॒ ते ते ते᳚ प्रा॒णम् । \newline
37. ते॒ प्रा॒णम् प्रा॒णम् ते॑ ते प्रा॒णꣳ सु॑वन्ताꣳ सुवन्ताम् प्रा॒णम् ते॑ ते प्रा॒णꣳ सु॑वन्ताम् । \newline
38. प्रा॒णꣳ सु॑वन्ताꣳ सुवन्ताम् प्रा॒णम् प्रा॒णꣳ सु॑वन्ता॒म् ते ते सु॑वन्ताम् प्रा॒णम् प्रा॒णꣳ सु॑वन्ता॒म् ते । \newline
39. प्रा॒णमिति॑ प्र - अ॒नम् । \newline
40. सु॒व॒न्ता॒म् ते ते सु॑वन्ताꣳ सुवन्ता॒म् ते ते॑ ते॒ ते सु॑वन्ताꣳ सुवन्ता॒म् ते ते᳚ । \newline
41. ते ते॑ ते॒ ते ते ते॒ चक्षु॒ श्चक्षु॑ स्ते॒ ते ते ते॒ चक्षुः॑ । \newline
42. ते॒ चक्षु॒ श्चक्षु॑ स्ते ते॒ चक्षुः॑ सुवन्ताꣳ सुवन्ता॒म् चक्षु॑ स्ते ते॒ चक्षुः॑ सुवन्ताम् । \newline
43. चक्षुः॑ सुवन्ताꣳ सुवन्ता॒म् चक्षु॒ श्चक्षुः॑ सुवन्ता॒म् ते ते सु॑वन्ता॒म् चक्षु॒ श्चक्षुः॑ सुवन्ता॒म् ते । \newline
44. सु॒व॒न्ता॒म् ते ते सु॑वन्ताꣳ सुवन्ता॒म् ते ते॑ ते॒ ते सु॑वन्ताꣳ सुवन्ता॒म् ते ते᳚ । \newline
45. ते ते॑ ते॒ ते ते ते॒ श्रोत्रꣳ॒॒ श्रोत्र॑म् ते॒ ते ते ते॒ श्रोत्र᳚म् । \newline
46. ते॒ श्रोत्रꣳ॒॒ श्रोत्र॑म् ते ते॒ श्रोत्रꣳ॑ सुवन्ताꣳ सुवन्ताꣳ॒॒ श्रोत्र॑म् ते ते॒ श्रोत्रꣳ॑ सुवन्ताम् । \newline
47. श्रोत्रꣳ॑ सुवन्ताꣳ सुवन्ताꣳ॒॒ श्रोत्रꣳ॒॒ श्रोत्रꣳ॑ सुवन्ताꣳ॒॒ सोम॑स्य॒ सोम॑स्य सुवन्ताꣳ॒॒ श्रोत्रꣳ॒॒ श्रोत्रꣳ॑ सुवन्ताꣳ॒॒ सोम॑स्य । \newline
48. सु॒व॒न्ताꣳ॒॒ सोम॑स्य॒ सोम॑स्य सुवन्ताꣳ सुवन्ताꣳ॒॒ सोम॑स्य त्वा त्वा॒ सोम॑स्य सुवन्ताꣳ सुवन्ताꣳ॒॒ सोम॑स्य त्वा । \newline
49. सोम॑स्य त्वा त्वा॒ सोम॑स्य॒ सोम॑स्य त्वा द्यु॒म्नेन॑ द्यु॒म्नेन॑ त्वा॒ सोम॑स्य॒ सोम॑स्य त्वा द्यु॒म्नेन॑ । \newline
50. त्वा॒ द्यु॒म्नेन॑ द्यु॒म्नेन॑ त्वा त्वा द्यु॒म्नेना॒ भ्य॑भि द्यु॒म्नेन॑ त्वा त्वा द्यु॒म्नेना॒भि । \newline
51. द्यु॒म्नेना॒भ्य॑भि द्यु॒म्नेन॑ द्यु॒म्नेना॒भि षि॑ञ्चामि सिञ्चाम्य॒भि द्यु॒म्नेन॑ द्यु॒म्नेना॒भि षि॑ञ्चामि । \newline
52. अ॒भि षि॑ञ्चामि सिञ्चा म्य॒भ्य॑भि षि॑ञ्चा म्य॒ग्ने र॒ग्नेः सि॑ञ्चा म्य॒भ्य॑भि षि॑ञ्चा म्य॒ग्नेः । \newline
53. सि॒ञ्चा॒ म्य॒ग्ने र॒ग्नेः सि॑ञ्चामि सिञ्चा म्य॒ग्ने स्तेज॑सा॒ तेज॑सा॒ ऽग्नेः सि॑ञ्चामि सिञ्चा म्य॒ग्ने स्तेज॑सा । \newline
54. अ॒ग्ने स्तेज॑सा॒ तेज॑सा॒ ऽग्ने र॒ग्ने स्तेज॑सा॒ सूर्य॑स्य॒ सूर्य॑स्य॒ तेज॑सा॒ ऽग्ने र॒ग्ने स्तेज॑सा॒ सूर्य॑स्य । \newline
\pagebreak
\markright{ TS 1.8.14.2  \hfill https://www.vedavms.in \hfill}
\addcontentsline{toc}{section}{ TS 1.8.14.2 }
\section*{ TS 1.8.14.2 }

\textbf{TS 1.8.14.2 } \newline
\textbf{Samhita Paata} \newline

स्तेज॑सा॒ सूर्य॑स्य॒ वर्च॒सेन्द्र॑स्येन्द्रि॒येण॑ मि॒त्रावरु॑णयोर् वी॒र्ये॑ण म॒रुता॒मोज॑सा क्ष॒त्राणां᳚ क्ष॒त्रप॑तिर॒स्यति॑ दि॒वस्पा॑हि स॒माव॑वृत्रन्न-ध॒रागुदी॑ची॒-रहिं॑ बु॒द्ध्निय॒मनु॑ स॒ञ्चर॑न्ती॒स्ताः पर्व॑तस्य वृष॒भस्य॑ पृ॒ष्ठे नाव॑श्चरन्ति स्व॒सिच॑ इया॒नाः ॥ रुद्र॒ यत्ते॒ क्रयी॒ परं॒ नाम॒ तस्मै॑ हु॒तम॑सि य॒मेष्ट॑मसि । प्रजा॑पते॒ न त्वदे॒तान्य॒न्यो विश्वा॑ जा॒तानि॒ परि॒ ता ( ) ब॑भूव । यत्का॑मास्ते जुहु॒मस्तन्नो॑ अस्तु व॒यꣳ स्या॑म॒ पत॑यो रयी॒णां ॥ \newline

\textbf{Pada Paata} \newline

तेज॑सा । सूर्य॑स्य । वर्च॑सा । इन्द्र॑स्य । इ॒न्द्रि॒येण॑ । मि॒त्रावरु॑णयो॒रिति॑ मि॒त्रा - वरु॑णयोः । वी॒र्ये॑ण । म॒रुता᳚म् । ओज॑सा । क्ष॒त्राणा᳚म् । क्ष॒त्रप॑ति॒रिति॑ क्ष॒त्र - प॒तिः॒ । अ॒सि॒ । अतीति॑ । दि॒वः । पा॒हि॒ । स॒माव॑वृत्र॒न्निति॑ सं - आव॑वृत्रन्न् । अ॒ध॒राक् । उदी॑चीः । अहि᳚म् । बु॒द्ध्निय᳚म् । अन्विति॑॑ । स॒ञ्चर॑न्ती॒रिति॑ सं-चर॑न्तीः । ताः । पर्व॑तस्य । वृ॒ष॒भस्य॑ । पृ॒ष्ठे । नावः॑ । च॒र॒न्ति॒ । स्व॒सिच॒ इति॑ स्व-सिचः॑ । इ॒या॒नाः ॥ रुद्र॑ । यत् । ते॒ । क्रयि॑ । पर᳚म् । नाम॑ । तस्मै᳚ । हु॒तम् । अ॒सि॒ । य॒मेष्ट॒मिति॑ य॒म - इ॒ष्ट॒म् । अ॒सि॒ ॥ प्रजा॑पत॒ इति॒ प्रजा᳚-प॒ते॒ । न । त्वत् । ए॒तानि॑ । अ॒न्यः । विश्वा᳚ । जा॒तानि॑ । परीति॑ । ता ( ) । ब॒भू॒व॒ ॥ यत्का॑मा॒ इति॒ यत् - का॒माः॒ । ते॒ । जु॒हु॒मः । तत् । नः॒ । अ॒स्तु॒ । व॒यम् । स्या॒म॒ । पत॑यः । र॒यी॒णाम् ॥  \newline



\textbf{Jatai Paata} \newline

1. तेज॑सा॒ सूर्य॑स्य॒ सूर्य॑स्य॒ तेज॑सा॒ तेज॑सा॒ सूर्य॑स्य । \newline
2. सूर्य॑स्य॒ वर्च॑सा॒ वर्च॑सा॒ सूर्य॑स्य॒ सूर्य॑स्य॒ वर्च॑सा । \newline
3. वर्च॒सेन्द्र॒स्ये न्द्र॑स्य॒ वर्च॑सा॒ वर्च॒सेन्द्र॑स्य । \newline
4. इन्द्र॑स्ये न्द्रि॒येणे᳚ न्द्रि॒येणे न्द्र॒स्ये न्द्र॑स्ये न्द्रि॒येण॑ । \newline
5. इ॒न्द्रि॒येण॑ मि॒त्रावरु॑णयोर् मि॒त्रावरु॑णयो रिन्द्रि॒येणे᳚ न्द्रि॒येण॑ मि॒त्रावरु॑णयोः । \newline
6. मि॒त्रावरु॑णयोर् वी॒र्ये॑ण वी॒र्ये॑ण मि॒त्रावरु॑णयोर् मि॒त्रावरु॑णयोर् वी॒र्ये॑ण । \newline
7. मि॒त्रावरु॑णयो॒रिति॑ मि॒त्रा - वरु॑णयोः । \newline
8. वी॒र्ये॑ण म॒रुता᳚म् म॒रुतां᳚ ॅवी॒र्ये॑ण वी॒र्ये॑ण म॒रुता᳚म् । \newline
9. म॒रुता॒ मोज॒ सौज॑सा म॒रुता᳚म् म॒रुता॒ मोज॑सा । \newline
10. ओज॑सा क्ष॒त्राणा᳚म् क्ष॒त्राणा॒ मोज॒ सौज॑सा क्ष॒त्राणा᳚म् । \newline
11. क्ष॒त्राणा᳚म् क्ष॒त्रप॑तिः क्ष॒त्रप॑तिः क्ष॒त्राणा᳚म् क्ष॒त्राणा᳚म् क्ष॒त्रप॑तिः । \newline
12. क्ष॒त्रप॑ति रस्यसि क्ष॒त्रप॑तिः क्ष॒त्रप॑ति रसि । \newline
13. क्ष॒त्रप॑ति॒रिति॑ क्ष॒त्र - प॒तिः॒ । \newline
14. अ॒स्य त्यत्य॑स्य॒ स्यति॑ । \newline
15. अति॑ दि॒वो दि॒वो ऽत्यति॑ दि॒वः । \newline
16. दि॒व स्पा॑हि पाहि दि॒वो दि॒व स्पा॑हि । \newline
17. पा॒हि॒ स॒माव॑वृत्रन् थ्स॒माव॑वृत्रन् पाहि पाहि स॒माव॑वृत्रन्न् । \newline
18. स॒माव॑वृत्रन् नध॒राग॑ध॒राख् स॒माव॑वृत्रन् थ्स॒माव॑वृत्रन् नध॒राक् । \newline
19. स॒माव॑वृत्र॒न्निति॑ सं - आव॑वृत्रन्न् । \newline
20. अ॒ध॒रा गुदी॑ची॒ रुदी॑ची रध॒रा ग॑ध॒रा गुदी॑चीः । \newline
21. उदी॑ची॒ रहि॒ महि॒ मुदी॑ची॒ रुदी॑ची॒ रहि᳚म् । \newline
22. अहि॑म् बु॒द्ध्निय॑म् बु॒द्ध्निय॒ महि॒ महि॑म् बु॒द्ध्निय᳚म् । \newline
23. बु॒द्ध्निय॒ मन्वनु॑ बु॒द्ध्निय॑म् बु॒द्ध्निय॒ मनु॑ । \newline
24. अनु॑ स॒ञ्चर॑न्तीः स॒ञ्चर॑न्ती॒ रन्वनु॑ स॒ञ्चर॑न्तीः । \newline
25. स॒ञ्चर॑न्ती॒ स्तास्ताः स॒ञ्चर॑न्तीः स॒ञ्चर॑न्ती॒ स्ताः । \newline
26. स॒ञ्चर॑न्ती॒रिति॑ सं - चर॑न्तीः । \newline
27. ताः पर्व॑तस्य॒ पर्व॑तस्य॒ तास्ताः पर्व॑तस्य । \newline
28. पर्व॑तस्य वृष॒भस्य॑ वृष॒भस्य॒ पर्व॑तस्य॒ पर्व॑तस्य वृष॒भस्य॑ । \newline
29. वृ॒ष॒भस्य॑ पृ॒ष्ठे पृ॒ष्ठे वृ॑ष॒भस्य॑ वृष॒भस्य॑ पृ॒ष्ठे । \newline
30. पृ॒ष्ठे नावो॒ नावः॑ पृ॒ष्ठे पृ॒ष्ठे नावः॑ । \newline
31. नाव॑ श्चरन्ति चरन्ति॒ नावो॒ नाव॑ श्चरन्ति । \newline
32. च॒र॒न्ति॒ स्व॒सिचः॑ स्व॒सिच॑ श्चरन्ति चरन्ति स्व॒सिचः॑ । \newline
33. स्व॒सिच॑ इया॒ना इ॑या॒नाः स्व॒सिचः॑ स्व॒सिच॑ इया॒नाः । \newline
34. स्व॒सिच॒ इति॑ स्व - सिचः॑ । \newline
35. इ॒या॒ना इती॑या॒नाः । \newline
36. रुद्र॒ यद् यद् रुद्र॒ रुद्र॒ यत् । \newline
37. यत् ते॑ ते॒ यद् यत् ते᳚ । \newline
38. ते॒ क्रयि॒ क्रयि॑ ते ते॒ क्रयि॑ । \newline
39. क्रयी॒ पर॒म् पर॒म् क्रयि॒ क्रयी॒ पर᳚म् । \newline
40. पर॒म् नाम॒ नाम॒ पर॒म् पर॒म् नाम॑ । \newline
41. नाम॒ तस्मै॒ तस्मै॒ नाम॒ नाम॒ तस्मै᳚ । \newline
42. तस्मै॑ हु॒तꣳ हु॒तम् तस्मै॒ तस्मै॑ हु॒तम् । \newline
43. हु॒त म॑स्यसि हु॒तꣳ हु॒त म॑सि । \newline
44. अ॒सि॒ य॒मेष्टं॑ ॅय॒मेष्ट॑ मस्यसि य॒मेष्ट᳚म् । \newline
45. य॒मेष्ट॑ मस्यसि य॒मेष्टं॑ ॅय॒मेष्ट॑ मसि । \newline
46. य॒मेष्ट॒मिति॑ य॒म - इ॒ष्ट॒म् । \newline
47. अ॒सीत्य॑सि । \newline
48. प्रजा॑पते॒ न न प्रजा॑पते॒ प्रजा॑पते॒ न । \newline
49. प्रजा॑पत॒ इति॒ प्रजा᳚ - प॒ते॒ । \newline
50. न त्वत् त्वन् न न त्वत् । \newline
51. त्वदे॒ता न्ये॒तानि॒ त्वत् त्वदे॒तानि॑ । \newline
52. ए॒तान्य॒न्यो᳚ ऽन्य ए॒ता न्ये॒ता न्य॒न्यः । \newline
53. अ॒न्यो विश्वा॒ विश्वा॒ ऽन्यो᳚ ऽन्यो विश्वा᳚ । \newline
54. विश्वा॑ जा॒तानि॑ जा॒तानि॒ विश्वा॒ विश्वा॑ जा॒तानि॑ । \newline
55. जा॒तानि॒ परि॒ परि॑ जा॒तानि॑ जा॒तानि॒ परि॑ । \newline
56. परि॒ ता ता परि॒ परि॒ ता । \newline
57. ता ब॑भूव बभूव॒ ता ता ब॑भूव । \newline
58. ब॒भू॒वेति॑ बभूव । \newline
59. यत्का॑मा स्ते ते॒ यत्का॑मा॒ यत्का॑मा स्ते । \newline
60. यत्का॑मा॒ इति॒ यत् - का॒माः॒ । \newline
61. ते॒ जु॒हु॒मो जु॑हु॒म स्ते॑ ते जुहु॒मः । \newline
62. जु॒हु॒म स्तत् तज् जु॑हु॒मो जु॑हु॒म स्तत् । \newline
63. तन् नो॑ न॒ स्तत् तन् नः॑ । \newline
64. नो॒ अ॒ स्त्व॒स्तु॒ नो॒ नो॒ अ॒स्तु॒ । \newline
65. अ॒स्तु॒ व॒यं ॅव॒य म॑ स्त्वस्तु व॒यम् । \newline
66. व॒यꣳ स्या॑म स्याम व॒यं ॅव॒यꣳ स्या॑म । \newline
67. स्या॒म॒ पत॑यः॒ पत॑यः स्याम स्याम॒ पत॑यः । \newline
68. पत॑यो रयी॒णाꣳ र॑यी॒णाम् पत॑यः॒ पत॑यो रयी॒णाम् । \newline
69. र॒यी॒णामिति॑ रयी॒णाम् । \newline

\textbf{Ghana Paata } \newline

1. तेज॑सा॒ सूर्य॑स्य॒ सूर्य॑स्य॒ तेज॑सा॒ तेज॑सा॒ सूर्य॑स्य॒ वर्च॑सा॒ वर्च॑सा॒ सूर्य॑स्य॒ तेज॑सा॒ तेज॑सा॒ सूर्य॑स्य॒ वर्च॑सा । \newline
2. सूर्य॑स्य॒ वर्च॑सा॒ वर्च॑सा॒ सूर्य॑स्य॒ सूर्य॑स्य॒ वर्च॒सेन्द्र॒स्ये न्द्र॑स्य॒ वर्च॑सा॒ सूर्य॑स्य॒ सूर्य॑स्य॒ वर्च॒सेन्द्र॑स्य । \newline
3. वर्च॒सेन्द्र॒स्ये न्द्र॑स्य॒ वर्च॑सा॒ वर्च॒सेन्द्र॑स्ये न्द्रि॒येणे᳚ न्द्रि॒येणे न्द्र॑स्य॒ वर्च॑सा॒ वर्च॒सेन्द्र॑स्ये न्द्रि॒येण॑ । \newline
4. इन्द्र॑स्ये न्द्रि॒येणे᳚ न्द्रि॒येणे न्द्र॒स्ये न्द्र॑स्ये न्द्रि॒येण॑ मि॒त्रावरु॑णयोर् मि॒त्रावरु॑णयो रिन्द्रि॒येणे न्द्र॒स्ये न्द्र॑स्ये न्द्रि॒येण॑ मि॒त्रावरु॑णयोः । \newline
5. इ॒न्द्रि॒येण॑ मि॒त्रावरु॑णयोर् मि॒त्रावरु॑णयो रिन्द्रि॒येणे᳚ न्द्रि॒येण॑ मि॒त्रावरु॑णयोर् वी॒र्ये॑ण वी॒र्ये॑ण मि॒त्रावरु॑णयो रिन्द्रि॒येणे᳚ न्द्रि॒येण॑ मि॒त्रावरु॑णयोर् वी॒र्ये॑ण । \newline
6. मि॒त्रावरु॑णयोर् वी॒र्ये॑ण वी॒र्ये॑ण मि॒त्रावरु॑णयोर् मि॒त्रावरु॑णयोर् वी॒र्ये॑ण म॒रुता᳚म् म॒रुतां᳚ ॅवी॒र्ये॑ण मि॒त्रावरु॑णयोर् मि॒त्रावरु॑णयोर् वी॒र्ये॑ण म॒रुता᳚म् । \newline
7. मि॒त्रावरु॑णयो॒रिति॑ मि॒त्रा - वरु॑णयोः । \newline
8. वी॒र्ये॑ण म॒रुता᳚म् म॒रुतां᳚ ॅवी॒र्ये॑ण वी॒र्ये॑ण म॒रुता॒ मोज॒सौज॑सा म॒रुतां᳚ ॅवी॒र्ये॑ण वी॒र्ये॑ण म॒रुता॒ मोज॑सा । \newline
9. म॒रुता॒ मोज॒सौज॑सा म॒रुता᳚म् म॒रुता॒ मोज॑सा क्ष॒त्राणा᳚म् क्ष॒त्राणा॒ मोज॑सा म॒रुता᳚म् म॒रुता॒ मोज॑सा क्ष॒त्राणा᳚म् । \newline
10. ओज॑सा क्ष॒त्राणा᳚म् क्ष॒त्राणा॒ मोज॒सौज॑सा क्ष॒त्राणा᳚म् क्ष॒त्रप॑तिः क्ष॒त्रप॑तिः क्ष॒त्राणा॒ मोज॒सौज॑सा क्ष॒त्राणा᳚म् क्ष॒त्रप॑तिः । \newline
11. क्ष॒त्राणा᳚म् क्ष॒त्रप॑तिः क्ष॒त्रप॑तिः क्ष॒त्राणा᳚म् क्ष॒त्राणा᳚म् क्ष॒त्रप॑ति रस्यसि क्ष॒त्रप॑तिः क्ष॒त्राणा᳚म् क्ष॒त्राणा᳚म् क्ष॒त्रप॑तिरसि । \newline
12. क्ष॒त्रप॑ति रस्यसि क्ष॒त्रप॑तिः क्ष॒त्रप॑ति र॒स्य त्यत्य॑सि क्ष॒त्रप॑तिः क्ष॒त्रप॑ति र॒स्यति॑ । \newline
13. क्ष॒त्रप॑ति॒रिति॑ क्ष॒त्र - प॒तिः॒ । \newline
14. अ॒स्य त्यत्य॑स्य॒ स्यति॑ दि॒वो दि॒वो ऽत्य॑ स्य॒स्यति॑ दि॒वः । \newline
15. अति॑ दि॒वो दि॒वो ऽत्यति॑ दि॒व स्पा॑हि पाहि दि॒वो ऽत्यति॑ दि॒व स्पा॑हि । \newline
16. दि॒व स्पा॑हि पाहि दि॒वो दि॒व स्पा॑हि स॒माव॑वृत्रन् थ्स॒माव॑वृत्रन् पाहि दि॒वो दि॒व स्पा॑हि स॒माव॑वृत्रन्न् । \newline
17. पा॒हि॒ स॒माव॑वृत्रन् थ्स॒माव॑वृत्रन् पाहि पाहि स॒माव॑वृत्रन् नध॒राग॑ध॒राख् स॒माव॑वृत्रन् पाहि पाहि स॒माव॑वृत्रन् नध॒राक् । \newline
18. स॒माव॑वृत्रन् नध॒राग॑ध॒राख् स॒माव॑वृत्रन् थ्स॒माव॑वृत्रन् नध॒रा गुदी॑ची॒ रुदी॑ची रध॒राख् स॒माव॑वृत्रन् थ्स॒माव॑वृत्रन् नध॒रा गुदी॑चीः । \newline
19. स॒माव॑वृत्र॒न्निति॑ सं - आव॑वृत्रन्न् । \newline
20. अ॒ध॒रा गुदी॑ची॒ रुदी॑ची रध॒रा ग॑ध॒रा गुदी॑ची॒ रहि॒ महि॒ मुदी॑ची रध॒रा ग॑ध॒रा गुदी॑ची॒ रहि᳚म् । \newline
21. उदी॑ची॒रहि॒ महि॒ मुदी॑ची॒ रुदी॑ची॒ रहि॑म् बु॒द्ध्निय॑म् बु॒द्ध्निय॒ महि॒ मुदी॑ची॒ रुदी॑ची॒ रहि॑म् बु॒द्ध्निय᳚म् । \newline
22. अहि॑म् बु॒द्ध्निय॑म् बु॒द्ध्निय॒ महि॒ महि॑म् बु॒द्ध्निय॒ मन्वनु॑ बु॒द्ध्निय॒ महि॒ महि॑म् बु॒द्ध्निय॒ मनु॑ । \newline
23. बु॒द्ध्निय॒ मन्वनु॑ बु॒द्ध्निय॑म् बु॒द्ध्निय॒ मनु॑ स॒ञ्चर॑न्तीः स॒ञ्चर॑न्ती॒ रनु॑ बु॒द्ध्निय॑म् बु॒द्ध्निय॒ मनु॑ स॒ञ्चर॑न्तीः । \newline
24. अनु॑ स॒ञ्चर॑न्तीः स॒ञ्चर॑न्ती॒ रन्वनु॑ स॒ञ्चर॑न्ती॒ स्ता स्ताः स॒ञ्चर॑न्ती॒ रन्वनु॑ स॒ञ्चर॑न्ती॒ स्ताः । \newline
25. स॒ञ्चर॑न्ती॒ स्ता स्ताः स॒ञ्चर॑न्तीः स॒ञ्चर॑न्ती॒ स्ताः पर्व॑तस्य॒ पर्व॑तस्य॒ ताः स॒ञ्चर॑न्तीः स॒ञ्चर॑न्ती॒ स्ताः पर्व॑तस्य । \newline
26. स॒ञ्चर॑न्ती॒रिति॑ सं - चर॑न्तीः । \newline
27. ताः पर्व॑तस्य॒ पर्व॑तस्य॒ तास्ताः पर्व॑तस्य वृष॒भस्य॑ वृष॒भस्य॒ पर्व॑तस्य॒ तास्ताः पर्व॑तस्य वृष॒भस्य॑ । \newline
28. पर्व॑तस्य वृष॒भस्य॑ वृष॒भस्य॒ पर्व॑तस्य॒ पर्व॑तस्य वृष॒भस्य॑ पृ॒ष्ठे पृ॒ष्ठे वृ॑ष॒भस्य॒ पर्व॑तस्य॒ पर्व॑तस्य वृष॒भस्य॑ पृ॒ष्ठे । \newline
29. वृ॒ष॒भस्य॑ पृ॒ष्ठे पृ॒ष्ठे वृ॑ष॒भस्य॑ वृष॒भस्य॑ पृ॒ष्ठे नावो॒ नावः॑ पृ॒ष्ठे वृ॑ष॒भस्य॑ वृष॒भस्य॑ पृ॒ष्ठे नावः॑ । \newline
30. पृ॒ष्ठे नावो॒ नावः॑ पृ॒ष्ठे पृ॒ष्ठे नाव॑ श्चरन्ति चरन्ति॒ नावः॑ पृ॒ष्ठे पृ॒ष्ठे नाव॑ श्चरन्ति । \newline
31. नाव॑ श्चरन्ति चरन्ति॒ नावो॒ नाव॑ श्चरन्ति स्व॒सिचः॑ स्व॒सिच॑ श्चरन्ति॒ नावो॒ नाव॑ श्चरन्ति स्व॒सिचः॑ । \newline
32. च॒र॒न्ति॒ स्व॒सिचः॑ स्व॒सिच॑ श्चरन्ति चरन्ति स्व॒सिच॑ इया॒ना इ॑या॒नाः स्व॒सिच॑ श्चरन्ति चरन्ति स्व॒सिच॑ इया॒नाः । \newline
33. स्व॒सिच॑ इया॒ना इ॑या॒नाः स्व॒सिचः॑ स्व॒सिच॑ इया॒नाः । \newline
34. स्व॒सिच॒ इति॑ स्व - सिचः॑ । \newline
35. इ॒या॒ना इती॑या॒नाः । \newline
36. रुद्र॒ यद् यद् रुद्र॒ रुद्र॒ यत् ते॑ ते॒ यद् रुद्र॒ रुद्र॒ यत् ते᳚ । \newline
37. यत् ते॑ ते॒ यद् यत् ते॒ क्रयि॒ क्रयि॑ ते॒ यद् यत् ते॒ क्रयि॑ । \newline
38. ते॒ क्रयि॒ क्रयि॑ ते ते॒ क्रयी॒ पर॒म् पर॒म् क्रयि॑ ते ते॒ क्रयी॒ पर᳚म् । \newline
39. क्रयी॒ पर॒म् पर॒म् क्रयि॒ क्रयी॒ पर॒न्नाम॒ नाम॒ पर॒म् क्रयि॒ क्रयी॒ पर॒न्नाम॑ । \newline
40. पर॒न्नाम॒ नाम॒ पर॒म् पर॒न्नाम॒ तस्मै॒ तस्मै॒ नाम॒ पर॒म् पर॒न्नाम॒ तस्मै᳚ । \newline
41. नाम॒ तस्मै॒ तस्मै॒ नाम॒ नाम॒ तस्मै॑ हु॒तꣳ हु॒तम् तस्मै॒ नाम॒ नाम॒ तस्मै॑ हु॒तम् । \newline
42. तस्मै॑ हु॒तꣳ हु॒तम् तस्मै॒ तस्मै॑ हु॒त म॑स्यसि हु॒तम् तस्मै॒ तस्मै॑ हु॒त म॑सि । \newline
43. हु॒त म॑स्यसि हु॒तꣳ हु॒त म॑सि य॒मेष्टं॑ ॅय॒मेष्ट॑ मसि हु॒तꣳ हु॒त म॑सि य॒मेष्ट᳚म् । \newline
44. अ॒सि॒ य॒मेष्टं॑ ॅय॒मेष्ट॑ मस्यसि य॒मेष्ट॑ मस्यसि य॒मेष्ट॑ मस्यसि य॒मेष्ट॑ मसि । \newline
45. य॒मेष्ट॑ मस्यसि य॒मेष्टं॑ ॅय॒मेष्ट॑ मसि । \newline
46. य॒मेष्ट॒मिति॑ य॒म - इ॒ष्ट॒म् । \newline
47. अ॒सीत्य॑सि । \newline
48. प्रजा॑पते॒ न न प्रजा॑पते॒ प्रजा॑पते॒ न त्वत् त्वन् न प्रजा॑पते॒ प्रजा॑पते॒ न त्वत् । \newline
49. प्रजा॑पत॒ इति॒ प्रजा᳚ - प॒ते॒ । \newline
50. न त्वत् त्वन् न न त्वदे॒ता न्ये॒तानि॒ त्वन् न न त्वदे॒तानि॑ । \newline
51. त्वदे॒ता न्ये॒तानि॒ त्वत् त्वदे॒ तान्य॒न्यो᳚ ऽन्य ए॒तानि॒ त्वत् त्वदे॒ता न्य॒न्यः । \newline
52. ए॒ता न्य॒न्यो᳚ ऽन्य ए॒ता न्ये॒ता न्य॒न्यो विश्वा॒ विश्वा॒ ऽन्य ए॒ता न्ये॒ता न्य॒न्यो विश्वा᳚ । \newline
53. अ॒न्यो विश्वा॒ विश्वा॒ ऽन्यो᳚ ऽन्यो विश्वा॑ जा॒तानि॑ जा॒तानि॒ विश्वा॒ ऽन्यो᳚ ऽन्यो विश्वा॑ जा॒तानि॑ । \newline
54. विश्वा॑ जा॒तानि॑ जा॒तानि॒ विश्वा॒ विश्वा॑ जा॒तानि॒ परि॒ परि॑ जा॒तानि॒ विश्वा॒ विश्वा॑ जा॒तानि॒ परि॑ । \newline
55. जा॒तानि॒ परि॒ परि॑ जा॒तानि॑ जा॒तानि॒ परि॒ ता ता परि॑ जा॒तानि॑ जा॒तानि॒ परि॒ ता । \newline
56. परि॒ ता ता परि॒ परि॒ ता ब॑भूव बभूव॒ ता परि॒ परि॒ ता ब॑भूव । \newline
57. ता ब॑भूव बभूव॒ ता ता ब॑भूव । \newline
58. ब॒भू॒वेति॑ बभूव । \newline
59. यत्का॑मा स्ते ते॒ यत्का॑मा॒ यत्का॑मा स्ते जुहु॒मो जु॑हु॒म स्ते॒ यत्का॑मा॒ यत्का॑मा स्ते जुहु॒मः । \newline
60. यत्का॑मा॒ इति॒ यत् - का॒माः॒ । \newline
61. ते॒ जु॒हु॒मो जु॑हु॒म स्ते॑ ते जुहु॒म स्तत् तज् जु॑हु॒म स्ते॑ ते जुहु॒म स्तत् । \newline
62. जु॒हु॒म स्तत् तज् जु॑हु॒मो जु॑हु॒म स्तन् नो॑ न॒ स्तज् जु॑हु॒मो जु॑हु॒ मस्तन् नः॑ । \newline
63. तन् नो॑ न॒ स्तत् तन् नो॑ अस्त्वस्तु न॒स्तत् तन् नो॑ अस्तु । \newline
64. नो॒ अ॒स्त्व॒स्तु॒ नो॒ नो॒ अ॒स्तु॒ व॒यं ॅव॒य म॑स्तु नो नो अस्तु व॒यम् । \newline
65. अ॒स्तु॒ व॒यं ॅव॒य म॑स्त्वस्तु व॒यꣳ स्या॑म स्याम व॒य म॑स्त्वस्तु व॒यꣳ स्या॑म । \newline
66. व॒यꣳ स्या॑म स्याम व॒यं ॅव॒यꣳ स्या॑म॒ पत॑यः॒ पत॑यः स्याम व॒यं ॅव॒यꣳ स्या॑म॒ पत॑यः । \newline
67. स्या॒म॒ पत॑यः॒ पत॑यः स्याम स्याम॒ पत॑यो रयी॒णाꣳ र॑यी॒णाम् पत॑यः स्याम स्याम॒ पत॑यो रयी॒णाम् । \newline
68. पत॑यो रयी॒णाꣳ र॑यी॒णाम् पत॑यः॒ पत॑यो रयी॒णाम् । \newline
69. र॒यी॒णामिति॑ रयी॒णाम् । \newline
\pagebreak
\markright{ TS 1.8.15.1  \hfill https://www.vedavms.in \hfill}
\addcontentsline{toc}{section}{ TS 1.8.15.1 }
\section*{ TS 1.8.15.1 }

\textbf{TS 1.8.15.1 } \newline
\textbf{Samhita Paata} \newline

इन्द्र॑स्य॒ वज्रो॑ऽसि॒ वार्त्र॑घ्न॒स्त्वया॒ऽयं ॅवृ॒त्रं ॅव॑द्ध्यान् मि॒त्रावरु॑णयोस्त्वा प्रशा॒स्त्रोः प्र॒शिषा॑ युनज्मि य॒ज्ञ्स्य॒ योगे॑न॒ विष्णोः॒ क्रमो॑ऽसि॒ विष्णोः᳚ क्रा॒न्तम॑सि॒ विष्णो॒र् विक्रा᳚न्तमसि म॒रुतां᳚ प्रस॒वे जे॑षमा॒प्तं मनः॒ सम॒हमि॑न्द्रि॒येण॑ वी॒र्ये॑ण पशू॒नां म॒न्युर॑सि॒ तवे॑व मे म॒न्युर् भू॑या॒न्नमो॑ मा॒त्रे पृ॑थि॒व्यै माऽहं मा॒तरं॑ पृथि॒वीꣳ हिꣳ॑सिषं॒ मा - [ ] \newline

\textbf{Pada Paata} \newline

इन्द्र॑स्य । वज्रः॑ । अ॒सि॒ । वार्त्र॑घ्न॒ इति॒ वार्त्र॑-घ्नः॒ । त्वया᳚ । अ॒यम् । वृ॒त्रम् । व॒द्ध्या॒त् । मि॒त्रावरु॑णयो॒रिति॑ मि॒त्रा - वरु॑णयोः । त्वा॒ । प्र॒शा॒स्त्रोरिति॑ प्र-शा॒स्त्रोः । प्र॒शिषेति॑ प्र -शिषा᳚ । यु॒न॒ज्मि॒ । य॒ज्ञ्स्य॑ । योगे॑न । विष्णोः᳚ । क्रमः॑ । अ॒सि॒ । विष्णोः᳚ । क्रा॒न्तम् । अ॒सि॒ । विष्णोः᳚ । विक्रा᳚न्त॒मिति॒ वि - क्रा॒न्त॒म् । अ॒सि॒ । म॒रुता᳚म् । प्र॒स॒व इति॑ प्र - स॒वे । जे॒ष॒म् । आ॒प्तम् । मनः॑ । समिति॑ । अ॒हम् । इ॒न्द्रि॒येण॑ । वी॒र्ये॑ण । प॒शू॒नाम् । म॒न्युः । अ॒सि॒ । तव॑ । इ॒व॒ । मे॒ । म॒न्युः । भू॒या॒त् । नमः॑ । मा॒त्रे । पृ॒थि॒व्यै । मा । अ॒हम् । मा॒तर᳚म् । पृ॒थि॒वीम् । हिꣳ॒॒सि॒ष॒म् । मा ।  \newline



\textbf{Jatai Paata} \newline

1. इन्द्र॑स्य॒ वज्रो॒ वज्र॒ इन्द्र॒स्ये न्द्र॑स्य॒ वज्रः॑ । \newline
2. वज्रो᳚ ऽस्यसि॒ वज्रो॒ वज्रो॑ ऽसि । \newline
3. अ॒सि॒ वार्त्र॑घ्नो॒ वार्त्र॑घ्नो ऽस्यसि॒ वार्त्र॑घ्नः । \newline
4. वार्त्र॑घ्न॒ स्त्वया॒ त्वया॒ वार्त्र॑घ्नो॒ वार्त्र॑घ्न॒ स्त्वया᳚ । \newline
5. वार्त्र॑घ्न॒ इति॒ वार्त्र॑ - घ्नः॒ । \newline
6. त्वया॒ ऽय म॒यम् त्वया॒ त्वया॒ ऽयम् । \newline
7. अ॒यं ॅवृ॒त्रं ॅवृ॒त्र म॒य म॒यं ॅवृ॒त्रम् । \newline
8. वृ॒त्रं ॅव॑द्ध्याद् वद्ध्याद् वृ॒त्रं ॅवृ॒त्रं ॅव॑द्ध्यात् । \newline
9. व॒द्ध्या॒न् मि॒त्रावरु॑णयोर् मि॒त्रावरु॑णयोर् वद्ध्याद् वद्ध्यान् मि॒त्रावरु॑णयोः । \newline
10. मि॒त्रावरु॑णयो स्त्वा त्वा मि॒त्रावरु॑णयोर् मि॒त्रावरु॑णयो स्त्वा । \newline
11. मि॒त्रावरु॑णयो॒रिति॑ मि॒त्रा - वरु॑णयोः । \newline
12. त्वा॒ प्र॒शा॒स्त्रोः प्र॑शा॒स्त्रो स्त्वा᳚ त्वा प्रशा॒स्त्रोः । \newline
13. प्र॒शा॒स्त्रोः प्र॒शिषा᳚ प्र॒शिषा᳚ प्रशा॒स्त्रोः प्र॑शा॒स्त्रोः प्र॒शिषा᳚ । \newline
14. प्र॒शा॒स्त्रोरिति॑ प्र - शा॒स्त्रोः । \newline
15. प्र॒शिषा॑ युनज्मि युनज्मि प्र॒शिषा᳚ प्र॒शिषा॑ युनज्मि । \newline
16. प्र॒शिषेति॑ प्र - शिषा᳚ । \newline
17. यु॒न॒ज्मि॒ य॒ज्ञ्स्य॑ य॒ज्ञ्स्य॑ युनज्मि युनज्मि य॒ज्ञ्स्य॑ । \newline
18. य॒ज्ञ्स्य॒ योगे॑न॒ योगे॑न य॒ज्ञ्स्य॑ य॒ज्ञ्स्य॒ योगे॑न । \newline
19. योगे॑न॒ विष्णो॒र् विष्णो॒र् योगे॑न॒ योगे॑न॒ विष्णोः᳚ । \newline
20. विष्णोः॒ क्रमः॒ क्रमो॒ विष्णो॒र् विष्णोः॒ क्रमः॑ । \newline
21. क्रमो᳚ ऽस्यसि॒ क्रमः॒ क्रमो॑ ऽसि । \newline
22. अ॒सि॒ विष्णो॒र् विष्णो॑ रस्यसि॒ विष्णोः᳚ । \newline
23. विष्णोः᳚ क्रा॒न्तम् क्रा॒न्तं ॅविष्णो॒र् विष्णोः᳚ क्रा॒न्तम् । \newline
24. क्रा॒न्त म॑स्यसि क्रा॒न्तम् क्रा॒न्त म॑सि । \newline
25. अ॒सि॒ विष्णो॒र् विष्णो॑ रस्यसि॒ विष्णोः᳚ । \newline
26. विष्णो॒र् विक्रा᳚न्तं॒ ॅविक्रा᳚न्तं॒ ॅविष्णो॒र् विष्णो॒र् विक्रा᳚न्तम् । \newline
27. विक्रा᳚न्त मस्यसि॒ विक्रा᳚न्तं॒ ॅविक्रा᳚न्त मसि । \newline
28. विक्रा᳚न्त॒मिति॒ वि - क्रा॒न्त॒म् । \newline
29. अ॒सि॒ म॒रुता᳚म् म॒रुता॑ मस्यसि म॒रुता᳚म् । \newline
30. म॒रुता᳚म् प्रस॒वे प्र॑स॒वे म॒रुता᳚म् म॒रुता᳚म् प्रस॒वे । \newline
31. प्र॒स॒वे जे॑षम् जेषम् प्रस॒वे प्र॑स॒वे जे॑षम् । \newline
32. प्र॒स॒व इति॑ प्र - स॒वे । \newline
33. जे॒ष॒ मा॒प्त मा॒प्तम् जे॑षम् जेष मा॒प्तम् । \newline
34. आ॒प्तम् मनो॒ मन॑ आ॒प्त मा॒प्तम् मनः॑ । \newline
35. मनः॒ सꣳ सम् मनो॒ मनः॒ सम् । \newline
36. स म॒ह म॒हꣳ सꣳ स म॒हम् । \newline
37. अ॒ह मि॑न्द्रि॒येणे᳚ न्द्रि॒येणा॒ह म॒ह मि॑न्द्रि॒येण॑ । \newline
38. इ॒न्द्रि॒येण॑ वी॒र्ये॑ण वी॒र्ये॑णे न्द्रि॒येणे᳚ न्द्रि॒येण॑ वी॒र्ये॑ण । \newline
39. वी॒र्ये॑ण पशू॒नाम् प॑शू॒नां ॅवी॒र्ये॑ण वी॒र्ये॑ण पशू॒नाम् । \newline
40. प॒शू॒नाम् म॒न्युर् म॒न्युः प॑शू॒नाम् प॑शू॒नाम् म॒न्युः । \newline
41. म॒न्यु र॑स्यसि म॒न्युर् म॒न्यु र॑सि । \newline
42. अ॒सि॒ तव॒ तवा᳚ स्यसि॒ तव॑ । \newline
43. तवे॑ वे व॒ तव॒ तवे॑ व । \newline
44. इ॒व॒ मे॒ म॒ इ॒वे॒ व॒ मे॒ । \newline
45. मे॒ म॒न्युर् म॒न्युर् मे॑ मे म॒न्युः । \newline
46. म॒न्युर् भू॑याद् भूयान् म॒न्युर् म॒न्युर् भू॑यात् । \newline
47. भू॒या॒न् नमो॒ नमो॑ भूयाद् भूया॒न् नमः॑ । \newline
48. नमो॑ मा॒त्रे मा॒त्रे नमो॒ नमो॑ मा॒त्रे । \newline
49. मा॒त्रे पृ॑थि॒व्यै पृ॑थि॒व्यै मा॒त्रे मा॒त्रे पृ॑थि॒व्यै । \newline
50. पृ॒थि॒व्यै मा मा पृ॑थि॒व्यै पृ॑थि॒व्यै मा । \newline
51. मा ऽह म॒हम् मा मा ऽहम् । \newline
52. अ॒हम् मा॒तर॑म् मा॒तर॑ म॒ह म॒हम् मा॒तर᳚म् । \newline
53. मा॒तर॑म् पृथि॒वीम् पृ॑थि॒वीम् मा॒तर॑म् मा॒तर॑म् पृथि॒वीम् । \newline
54. पृ॒थि॒वीꣳ हिꣳ॑सिषꣳ हिꣳसिषम् पृथि॒वीम् पृ॑थि॒वीꣳ हिꣳ॑सिषम् । \newline
55. हिꣳ॒॒सि॒ष॒म् मा मा हिꣳ॑सिषꣳ हिꣳसिष॒म् मा । \newline
56. मा माम् माम् मा मा माम् । \newline

\textbf{Ghana Paata } \newline

1. इन्द्र॑स्य॒ वज्रो॒ वज्र॒ इन्द्र॒स्ये न्द्र॑स्य॒ वज्रो᳚ ऽस्यसि॒ वज्र॒ इन्द्र॒स्ये न्द्र॑स्य॒ वज्रो॑ ऽसि । \newline
2. वज्रो᳚ ऽस्यसि॒ वज्रो॒ वज्रो॑ ऽसि॒ वार्त्र॑घ्नो॒ वार्त्र॑घ्नो ऽसि॒ वज्रो॒ वज्रो॑ ऽसि॒ वार्त्र॑घ्नः । \newline
3. अ॒सि॒ वार्त्र॑घ्नो॒ वार्त्र॑घ्नो ऽस्यसि॒ वार्त्र॑घ्न॒ स्त्वया॒ त्वया॒ वार्त्र॑घ्नो ऽस्यसि॒ वार्त्र॑घ्न॒ स्त्वया᳚ । \newline
4. वार्त्र॑घ्न॒ स्त्वया॒ त्वया॒ वार्त्र॑घ्नो॒ वार्त्र॑घ्न॒ स्त्वया॒ ऽय म॒यम् त्वया॒ वार्त्र॑घ्नो॒ वार्त्र॑घ्न॒ स्त्वया॒ ऽयम् । \newline
5. वार्त्र॑घ्न॒ इति॒ वार्त्र॑ - घ्नः॒ । \newline
6. त्वया॒ ऽय म॒यम् त्वया॒ त्वया॒ ऽयं ॅवृ॒त्रं ॅवृ॒त्र म॒यम् त्वया॒ त्वया॒ ऽयं ॅवृ॒त्रम् । \newline
7. अ॒यं ॅवृ॒त्रं ॅवृ॒त्र म॒य म॒यं ॅवृ॒त्रं ॅव॑द्ध्याद् वद्ध्याद् वृ॒त्र म॒य म॒यं ॅवृ॒त्रं ॅव॑द्ध्यात् । \newline
8. वृ॒त्रं ॅव॑द्ध्याद् वद्ध्याद् वृ॒त्रं ॅवृ॒त्रं ॅव॑द्ध्यान् मि॒त्रावरु॑णयोर् मि॒त्रावरु॑णयोर् वद्ध्याद् वृ॒त्रं ॅवृ॒त्रं ॅव॑द्ध्यान् मि॒त्रावरु॑णयोः । \newline
9. व॒द्ध्या॒न् मि॒त्रावरु॑णयोर् मि॒त्रावरु॑णयोर् वद्ध्याद् वद्ध्यान् मि॒त्रावरु॑णयो स्त्वा त्वा मि॒त्रावरु॑णयोर् वद्ध्याद् वद्ध्यान् मि॒त्रावरु॑णयो स्त्वा । \newline
10. मि॒त्रावरु॑णयो स्त्वा त्वा मि॒त्रावरु॑णयोर् मि॒त्रावरु॑णयो स्त्वा प्रशा॒स्त्रोः प्र॑शा॒स्त्रो स्त्वा॑ मि॒त्रावरु॑णयोर् मि॒त्रावरु॑णयो स्त्वा प्रशा॒स्त्रोः । \newline
11. मि॒त्रावरु॑णयो॒रिति॑ मि॒त्रा - वरु॑णयोः । \newline
12. त्वा॒ प्र॒शा॒स्त्रोः प्र॑शा॒स्त्रो स्त्वा᳚ त्वा प्रशा॒स्त्रोः प्र॒शिषा᳚ प्र॒शिषा᳚ प्रशा॒स्त्रो स्त्वा᳚ त्वा प्रशा॒स्त्रोः प्र॒शिषा᳚ । \newline
13. प्र॒शा॒स्त्रोः प्र॒शिषा᳚ प्र॒शिषा᳚ प्रशा॒स्त्रोः प्र॑शा॒स्त्रोः प्र॒शिषा॑ युनज्मि युनज्मि प्र॒शिषा᳚ प्रशा॒स्त्रोः प्र॑शा॒स्त्रोः प्र॒शिषा॑ युनज्मि । \newline
14. प्र॒शा॒स्त्रोरिति॑ प्र - शा॒स्त्रोः । \newline
15. प्र॒शिषा॑ युनज्मि युनज्मि प्र॒शिषा᳚ प्र॒शिषा॑ युनज्मि य॒ज्ञ्स्य॑ य॒ज्ञ्स्य॑ युनज्मि प्र॒शिषा᳚ प्र॒शिषा॑ युनज्मि य॒ज्ञ्स्य॑ । \newline
16. प्र॒शिषेति॑ प्र - शिषा᳚ । \newline
17. यु॒न॒ज्मि॒ य॒ज्ञ्स्य॑ य॒ज्ञ्स्य॑ युनज्मि युनज्मि य॒ज्ञ्स्य॒ योगे॑न॒ योगे॑न य॒ज्ञ्स्य॑ युनज्मि युनज्मि य॒ज्ञ्स्य॒ योगे॑न । \newline
18. य॒ज्ञ्स्य॒ योगे॑न॒ योगे॑न य॒ज्ञ्स्य॑ य॒ज्ञ्स्य॒ योगे॑न॒ विष्णो॒र् विष्णो॒र् योगे॑न य॒ज्ञ्स्य॑ य॒ज्ञ्स्य॒ योगे॑न॒ विष्णोः᳚ । \newline
19. योगे॑न॒ विष्णो॒र् विष्णो॒र् योगे॑न॒ योगे॑न॒ विष्णोः॒ क्रमः॒ क्रमो॒ विष्णो॒र् योगे॑न॒ योगे॑न॒ विष्णोः॒ क्रमः॑ । \newline
20. विष्णोः॒ क्रमः॒ क्रमो॒ विष्णो॒र् विष्णोः॒ क्रमो᳚ ऽस्यसि॒ क्रमो॒ विष्णो॒र् विष्णोः॒ क्रमो॑ ऽसि । \newline
21. क्रमो᳚ ऽस्यसि॒ क्रमः॒ क्रमो॑ ऽसि॒ विष्णो॒र् विष्णो॑ रसि॒ क्रमः॒ क्रमो॑ ऽसि॒ विष्णोः᳚ । \newline
22. अ॒सि॒ विष्णो॒र् विष्णो॑ रस्यसि॒ विष्णोः᳚ क्रा॒न्तम् क्रा॒न्तं ॅविष्णो॑ रस्यसि॒ विष्णोः᳚ क्रा॒न्तम् । \newline
23. विष्णोः᳚ क्रा॒न्तम् क्रा॒न्तं ॅविष्णो॒र् विष्णोः᳚ क्रा॒न्त म॑स्यसि क्रा॒न्तं ॅविष्णो॒र् विष्णोः᳚ क्रा॒न्त म॑सि । \newline
24. क्रा॒न्त म॑स्यसि क्रा॒न्तम् क्रा॒न्त म॑सि॒ विष्णो॒र् विष्णो॑ रसि क्रा॒न्तम् क्रा॒न्त म॑सि॒ विष्णोः᳚ । \newline
25. अ॒सि॒ विष्णो॒र् विष्णो॑ रस्यसि॒ विष्णो॒र् विक्रा᳚न्तं॒ ॅविक्रा᳚न्तं॒ ॅविष्णो॑ रस्यसि॒ विष्णो॒र् विक्रा᳚न्तम् । \newline
26. विष्णो॒र् विक्रा᳚न्तं॒ ॅविक्रा᳚न्तं॒ ॅविष्णो॒र् विष्णो॒र् विक्रा᳚न्त मस्यसि॒ विक्रा᳚न्तं॒ ॅविष्णो॒र् विष्णो॒र् विक्रा᳚न्त मसि । \newline
27. विक्रा᳚न्त मस्यसि॒ विक्रा᳚न्तं॒ ॅविक्रा᳚न्त मसि म॒रुता᳚म् म॒रुता॑ मसि॒ विक्रा᳚न्तं॒ ॅविक्रा᳚न्त मसि म॒रुता᳚म् । \newline
28. विक्रा᳚न्त॒मिति॒ वि - क्रा॒न्त॒म् । \newline
29. अ॒सि॒ म॒रुता᳚म् म॒रुता॑ मस्यसि म॒रुता᳚म् प्रस॒वे प्र॑स॒वे म॒रुता॑ मस्यसि म॒रुता᳚म् प्रस॒वे । \newline
30. म॒रुता᳚म् प्रस॒वे प्र॑स॒वे म॒रुता᳚म् म॒रुता᳚म् प्रस॒वे जे॑षम् जेषम् प्रस॒वे म॒रुता᳚म् म॒रुता᳚म् प्रस॒वे जे॑षम् । \newline
31. प्र॒स॒वे जे॑षम् जेषम् प्रस॒वे प्र॑स॒वे जे॑ष मा॒प्त मा॒प्तम् जे॑षम् प्रस॒वे प्र॑स॒वे जे॑ष मा॒प्तम् । \newline
32. प्र॒स॒व इति॑ प्र - स॒वे । \newline
33. जे॒ष॒ मा॒प्त मा॒प्तम् जे॑षम् जेष मा॒प्तम् मनो॒ मन॑ आ॒प्तम् जे॑षम् जेष मा॒प्तम् मनः॑ । \newline
34. आ॒प्तम् मनो॒ मन॑ आ॒प्त मा॒प्तम् मनः॒ सꣳ सम् मन॑ आ॒प्त मा॒प्तम् मनः॒ सम् । \newline
35. मनः॒ सꣳ सम् मनो॒ मनः॒ स म॒ह म॒हꣳ सम् मनो॒ मनः॒ स म॒हम् । \newline
36. स म॒ह म॒हꣳ सꣳ स म॒ह मि॑न्द्रि॒येणे᳚ न्द्रि॒येणा॒हꣳ सꣳ स म॒ह मि॑न्द्रि॒येण॑ । \newline
37. अ॒ह मि॑न्द्रि॒येणे᳚ न्द्रि॒येणा॒ह म॒ह मि॑न्द्रि॒येण॑ वी॒र्ये॑ण वी॒र्ये॑णे न्द्रि॒येणा॒ह म॒ह मि॑न्द्रि॒येण॑ वी॒र्ये॑ण । \newline
38. इ॒न्द्रि॒येण॑ वी॒र्ये॑ण वी॒र्ये॑णे न्द्रि॒येणे᳚ न्द्रि॒येण॑ वी॒र्ये॑ण पशू॒नाम् प॑शू॒नां ॅवी॒र्ये॑णे न्द्रि॒येणे᳚ न्द्रि॒येण॑ वी॒र्ये॑ण पशू॒नाम् । \newline
39. वी॒र्ये॑ण पशू॒नाम् प॑शू॒नां ॅवी॒र्ये॑ण वी॒र्ये॑ण पशू॒नाम् म॒न्युर् म॒न्युः प॑शू॒नां ॅवी॒र्ये॑ण वी॒र्ये॑ण पशू॒नाम् म॒न्युः । \newline
40. प॒शू॒नाम् म॒न्युर् म॒न्युः प॑शू॒नाम् प॑शू॒नाम् म॒न्यु र॑स्यसि म॒न्युः प॑शू॒नाम् प॑शू॒नाम् म॒न्यु र॑सि । \newline
41. म॒न्यु र॑स्यसि म॒न्युर् म॒न्यु र॑सि॒ तव॒ तवा॑सि म॒न्युर् म॒न्यु र॑सि॒ तव॑ । \newline
42. अ॒सि॒ तव॒ तवा᳚स्यसि॒ तवे॑ वे व॒ तवा᳚स्यसि॒ तवे॑ व । \newline
43. तवे॑ वे व॒ तव॒ तवे॑ व मे म इव॒ तव॒ तवे॑ व मे । \newline
44. इ॒व॒ मे॒ म॒ इ॒वे॒ व॒ मे॒ म॒न्युर् म॒न्युर् म॑ इवे व मे म॒न्युः । \newline
45. मे॒ म॒न्युर् म॒न्युर् मे॑ मे म॒न्युर् भू॑याद् भूयान् म॒न्युर् मे॑ मे म॒न्युर् भू॑यात् । \newline
46. म॒न्युर् भू॑याद् भूयान् म॒न्युर् म॒न्युर् भू॑या॒न् नमो॒ नमो॑ भूयान् म॒न्युर् म॒न्युर् भू॑या॒न् नमः॑ । \newline
47. भू॒या॒न् नमो॒ नमो॑ भूयाद् भूया॒न् नमो॑ मा॒त्रे मा॒त्रे नमो॑ भूयाद् भूया॒न् नमो॑ मा॒त्रे । \newline
48. नमो॑ मा॒त्रे मा॒त्रे नमो॒ नमो॑ मा॒त्रे पृ॑थि॒व्यै पृ॑थि॒व्यै मा॒त्रे नमो॒ नमो॑ मा॒त्रे पृ॑थि॒व्यै । \newline
49. मा॒त्रे पृ॑थि॒व्यै पृ॑थि॒व्यै मा॒त्रे मा॒त्रे पृ॑थि॒व्यै मा मा पृ॑थि॒व्यै मा॒त्रे मा॒त्रे पृ॑थि॒व्यै मा । \newline
50. पृ॒थि॒व्यै मा मा पृ॑थि॒व्यै पृ॑थि॒व्यै मा ऽह म॒हम् मा पृ॑थि॒व्यै पृ॑थि॒व्यै मा ऽहम् । \newline
51. मा ऽह म॒हम् मा मा ऽहम् मा॒तर॑म् मा॒तर॑ म॒हम् मा मा ऽहम् मा॒तर᳚म् । \newline
52. अ॒हम् मा॒तर॑म् मा॒तर॑ म॒ह म॒हम् मा॒तर॑म् पृथि॒वीम् पृ॑थि॒वीम् मा॒तर॑ म॒ह म॒हम् मा॒तर॑म् पृथि॒वीम् । \newline
53. मा॒तर॑म् पृथि॒वीम् पृ॑थि॒वीम् मा॒तर॑म् मा॒तर॑म् पृथि॒वीꣳ हिꣳ॑सिषꣳ हिꣳसिषम् पृथि॒वीम् मा॒तर॑म् मा॒तर॑म् पृथि॒वीꣳ हिꣳ॑सिषम् । \newline
54. पृ॒थि॒वीꣳ हिꣳ॑सिषꣳ हिꣳसिषम् पृथि॒वीम् पृ॑थि॒वीꣳ हिꣳ॑सिष॒म् मा मा हिꣳ॑सिषम् पृथि॒वीम् पृ॑थि॒वीꣳ हिꣳ॑सिष॒म् मा । \newline
55. हिꣳ॒॒सि॒ष॒म् मा मा हिꣳ॑सिषꣳ हिꣳसिष॒म् मा माम् माम् मा हिꣳ॑सिषꣳ हिꣳसिष॒म् मा माम् । \newline
56. मा माम् माम् मा मा माम् मा॒ता मा॒ता माम् मा मा माम् मा॒ता । \newline
\pagebreak
\markright{ TS 1.8.15.2  \hfill https://www.vedavms.in \hfill}
\addcontentsline{toc}{section}{ TS 1.8.15.2 }
\section*{ TS 1.8.15.2 }

\textbf{TS 1.8.15.2 } \newline
\textbf{Samhita Paata} \newline

मां मा॒ता पृ॑थि॒वी हिꣳ॑सी॒दिय॑द॒स्यायु॑-र॒स्यायु॑र् मे धे॒ह्यूर्ग॒स्यूर्जं॑ मे धेहि॒ युङ्ङ॑सि॒ वर्चो॑ऽसि॒ वर्चो॒ मयि॑ धेह्य॒ग्नये॑ गृ॒हप॑तये॒ स्वाहा॒ सोमा॑य॒ वन॒स्पत॑ये॒ स्वाहेन्द्र॑स्य॒ बला॑य॒ स्वाहा॑ म॒रुता॒मोज॑से॒ स्वाहा॑ हꣳ॒॒सः शु॑चि॒षद् वसु॑रन्तरिक्ष॒ -सद्धोता॑ वेदि॒षदति॑थिर् दुरोण॒सत् । नृ॒षद् व॑र॒सदृ॑त॒सद् व्यो॑म॒सद॒ब्जा गो॒जा ऋ॑त॒जा ( ) अ॑द्रि॒जा ऋ॒तं बृ॒हत् ॥ \newline

\textbf{Pada Paata} \newline

माम् । मा॒ता । पृ॒थि॒वी । हिꣳ॒॒सी॒त् । इय॑त् । अ॒सि॒ । आयुः॑ । अ॒सि॒ । आयुः॑ । मे॒ । धे॒हि॒ । ऊर्क् । अ॒सि॒ । ऊर्ज᳚म् । मे॒ । धे॒हि॒ । युङ् । अ॒सि॒ । वर्चः॑ । अ॒सि॒ । वर्चः॑ । मयि॑ । धे॒हि॒ । अ॒ग्नये᳚ । गृ॒हप॑तय॒ इति॑ गृ॒ह-प॒त॒ये॒ । स्वाहा᳚ । सोमा॑य । वन॒स्पत॑ये । स्वाहा᳚ । इन्द्र॑स्य । बला॑य । स्वाहा᳚ । म॒रुता᳚म् । ओज॑से । स्वाहा᳚ । हꣳ॒॒सः । शु॒चि॒षदिति॑ शुचि - सत् । वसुः॑ । अ॒न्त॒रि॒क्ष॒दित्य॑न्तरिक्ष - सत् । होता᳚ । वे॒दि॒षदिति॑ वेदि - सत् । अति॑थिः । दु॒रो॒ण॒सदिति॑ दुरोण - सत् ॥ नृ॒षदिति॑ नृ - सत् । व॒र॒सदिति॑ वर - सत् । ऋ॒त॒सदित्यृ॑त -सत् । व्यो॒म॒सदिति॑ व्योम - सत् । अ॒ब्जा इत्य॑प् - जाः । गो॒जा इति॑ गो - जाः । ऋ॒त॒जा इत्यृ॑त-जाः ( ) । अ॒द्रि॒जा इत्य॑द्रि - जाः । ऋ॒तं । बृ॒हत् ॥  \newline



\textbf{Jatai Paata} \newline

1. माम् मा॒ता मा॒ता माम् माम् मा॒ता । \newline
2. मा॒ता पृ॑थि॒वी पृ॑थि॒वी मा॒ता मा॒ता पृ॑थि॒वी । \newline
3. पृ॒थि॒वी हिꣳ॑सी द्धिꣳसीत् पृथि॒वी पृ॑थि॒वी हिꣳ॑सीत् । \newline
4. हिꣳ॒॒सी॒ दिय॒ दिय॑ द्धिꣳसीद् धिꣳसी॒ दिय॑त् । \newline
5. इय॑ दस्य॒सीय॒ दिय॑ दसि । \newline
6. अ॒स्यायु॒ रायु॑ रस्य॒ स्यायुः॑ । \newline
7. आयु॑ रस्य॒स्यायु॒ रायु॑ रसि । \newline
8. अ॒स्यायु॒ रायु॑ रस्य॒ स्यायुः॑ । \newline
9. आयु॑र् मे म॒ आयु॒ रायु॑र् मे । \newline
10. मे॒ धे॒हि॒ धे॒हि॒ मे॒ मे॒ धे॒हि॒ । \newline
11. धे॒ह्यूर् गूर्ग् धे॑हि धे॒ह्यूर्क् । \newline
12. ऊर्ग॑स्य॒ स्यूर्गूर् ग॑सि । \newline
13. अ॒स्यूर्ज॒ मूर्ज॑ मस्य॒ स्यूर्ज᳚म् । \newline
14. ऊर्ज॑म् मे म॒ ऊर्ज॒ मूर्ज॑म् मे । \newline
15. मे॒ धे॒हि॒ धे॒हि॒ मे॒ मे॒ धे॒हि॒ । \newline
16. धे॒हि॒ युङ् युङ् धे॑हि धेहि॒ युङ् । \newline
17. युङ् ङ॑स्यसि॒ युङ् युङ् ङ॑सि । \newline
18. अ॒सि॒ वर्चो॒ वर्चो᳚ ऽस्यसि॒ वर्चः॑ । \newline
19. वर्चो᳚ ऽस्यसि॒ वर्चो॒ वर्चो॑ ऽसि । \newline
20. अ॒सि॒ वर्चो॒ वर्चो᳚ ऽस्यसि॒ वर्चः॑ । \newline
21. वर्चो॒ मयि॒ मयि॒ वर्चो॒ वर्चो॒ मयि॑ । \newline
22. मयि॑ धेहि धेहि॒ मयि॒ मयि॑ धेहि । \newline
23. धे॒ह्य॒ग्नये॒ ऽग्नये॑ धेहि धेह्य॒ग्नये᳚ । \newline
24. अ॒ग्नये॑ गृ॒हप॑तये गृ॒हप॑तये॒ ऽग्नये॒ ऽग्नये॑ गृ॒हप॑तये । \newline
25. गृ॒हप॑तये॒ स्वाहा॒ स्वाहा॑ गृ॒हप॑तये गृ॒हप॑तये॒ स्वाहा᳚ । \newline
26. गृ॒हप॑तय॒ इति॑ गृ॒ह - प॒त॒ये॒ । \newline
27. स्वाहा॒ सोमा॑य॒ सोमा॑य॒ स्वाहा॒ स्वाहा॒ सोमा॑य । \newline
28. सोमा॑य॒ वन॒स्पत॑ये॒ वन॒स्पत॑ये॒ सोमा॑य॒ सोमा॑य॒ वन॒स्पत॑ये । \newline
29. वन॒स्पत॑ये॒ स्वाहा॒ स्वाहा॒ वन॒स्पत॑ये॒ वन॒स्पत॑ये॒ स्वाहा᳚ । \newline
30. स्वाहेन्द्र॒स्ये न्द्र॑स्य॒ स्वाहा॒ स्वाहेन्द्र॑स्य । \newline
31. इन्द्र॑स्य॒ बला॑य॒ बला॒ये न्द्र॒स्ये न्द्र॑स्य॒ बला॑य । \newline
32. बला॑य॒ स्वाहा॒ स्वाहा॒ बला॑य॒ बला॑य॒ स्वाहा᳚ । \newline
33. स्वाहा॑ म॒रुता᳚म् म॒रुताꣳ॒॒ स्वाहा॒ स्वाहा॑ म॒रुता᳚म् । \newline
34. म॒रुता॒ मोज॑स॒ ओज॑से म॒रुता᳚म् म॒रुता॒ मोज॑से । \newline
35. ओज॑से॒ स्वाहा॒ स्वाहौज॑स॒ ओज॑से॒ स्वाहा᳚ । \newline
36. स्वाहा॑ हꣳ॒॒सो हꣳ॒॒सः स्वाहा॒ स्वाहा॑ हꣳ॒॒सः । \newline
37. हꣳ॒॒सः शु॑चि॒षच् छु॑चि॒ष द्धꣳ॒॒सो हꣳ॒॒सः शु॑चि॒षत् । \newline
38. शु॒चि॒षद् वसु॒र् वसुः॑ शुचि॒षच् छु॑चि॒षद् वसुः॑ । \newline
39. शु॒चि॒षदिति॑ शुचि - सत् । \newline
40. वसु॑ रन्तरिक्ष॒स द॑न्तरिक्ष॒सद् वसु॒र् वसु॑ रन्तरिक्ष॒सत् । \newline
41. अ॒न्त॒रि॒क्ष॒स द्धोता॒ होता᳚ ऽन्तरिक्ष॒स द॑न्तरिक्ष॒स द्धोता᳚ । \newline
42. अ॒न्त॒रि॒क्ष॒सदित्य॑न्तरिक्ष - सत् । \newline
43. होता॑ वेदि॒षद् वे॑दि॒ष द्धोता॒ होता॑ वेदि॒षत् । \newline
44. वे॒दि॒ष दति॑थि॒ रति॑थिर् वेदि॒षद् वे॑दि॒ष दति॑थिः । \newline
45. वे॒दि॒षदिति॑ वेदि - सत् । \newline
46. अति॑थिर् दुरोण॒सद् दु॑रोण॒स दति॑थि॒ रति॑थिर् दुरोण॒सत् । \newline
47. दु॒रो॒ण॒सदिति॑ दुरोण - सत् । \newline
48. नृ॒षद् व॑र॒सद् व॑र॒सन् नृ॒षन् नृ॒षद् व॑र॒सत् । \newline
49. नृ॒षदिति॑ नृ - सत् । \newline
50. व॒र॒स दृ॑त॒स दृ॑त॒सद् व॑र॒सद् व॑र॒स दृ॑त॒सत् । \newline
51. व॒र॒सदिति॑ वर - सत् । \newline
52. ऋ॒त॒सद् व्यो॑म॒सद् व्यो॑म॒स दृ॑त॒स दृ॑त॒सद् व्यो॑म॒सत् । \newline
53. ऋ॒त॒सदित्यृ॑त - सत् । \newline
54. व्यो॒म॒स द॒ब्जा अ॒ब्जा व्यो॑म॒सद् व्यो॑म॒स द॒ब्जाः । \newline
55. व्यो॒म॒सदिति॑ व्योम - सत् । \newline
56. अ॒ब्जा गो॒जा गो॒जा अ॒ब्जा अ॒ब्जा गो॒जाः । \newline
57. अ॒ब्जा इत्य॑प् - जाः । \newline
58. गो॒जा ऋ॑त॒जा ऋ॑त॒जा गो॒जा गो॒जा ऋ॑त॒जाः । \newline
59. गो॒जा इति॑ गो - जाः । \newline
60. ऋ॒त॒जा अ॑द्रि॒जा अ॑द्रि॒जा ऋ॑त॒जा ऋ॑त॒जा अ॑द्रि॒जाः । \newline
61. ऋ॒त॒जा इत्यृ॑त - जाः । \newline
62. अ॒द्रि॒जा ऋ॒त मृ॒त म॑द्रि॒जा अ॑द्रि॒जा ऋ॒तं । \newline
63. अ॒द्रि॒जा इत्य॑द्रि - जाः । \newline
64. ऋ॒तं बृ॒हद् बृ॒हदृ॒तं ऋ॒तं बृ॒हत् । \newline
65. बृ॒हदिति॑ बृ॒हत् । \newline

\textbf{Ghana Paata } \newline

1. माम् मा॒ता मा॒ता माम् माम् मा॒ता पृ॑थि॒वी पृ॑थि॒वी मा॒ता माम् माम् मा॒ता पृ॑थि॒वी । \newline
2. मा॒ता पृ॑थि॒वी पृ॑थि॒वी मा॒ता मा॒ता पृ॑थि॒वी हिꣳ॑सी द्धिꣳसीत् पृथि॒वी मा॒ता मा॒ता पृ॑थि॒वी हिꣳ॑सीत् । \newline
3. पृ॒थि॒वी हिꣳ॑सी द्धिꣳसीत् पृथि॒वी पृ॑थि॒वी हिꣳ॑सी॒ दिय॒ दिय॑ द्धिꣳसीत् पृथि॒वी पृ॑थि॒वी हिꣳ॑सी॒ दिय॑त् । \newline
4. हिꣳ॒॒सी॒ दिय॒ दिय॑ द्धिꣳसी द्धिꣳसी॒ दिय॑ दस्य॒सी य॑द्धिꣳसी द्धिꣳसी॒ दिय॑दसि । \newline
5. इय॑ दस्य॒सी य॒दिय॑ द॒स्यायु॒ रायु॑ र॒सीय॒ दिय॑ द॒स्यायुः॑ । \newline
6. अ॒स्यायु॒ रायु॑ रस्य॒स्यायु॑ रस्य॒स्यायु॑ रस्य॒स्यायु॑ रसि । \newline
7. आयु॑ रस्य॒स्यायु॒ रायु॑ र॒स्यायु॒ रायु॑ र॒स्यायु॒ रायु॑ र॒स्यायुः॑ । \newline
8. अ॒स्यायु॒ रायु॑ रस्य॒स्यायु॑र् मे म॒ आयु॑ रस्य॒स्यायु॑र् मे । \newline
9. आयु॑र् मे म॒ आयु॒ रायु॑र् मे धेहि धेहि म॒ आयु॒ रायु॑र् मे धेहि । \newline
10. मे॒ धे॒हि॒ धे॒हि॒ मे॒ मे॒ धे॒ह्यूर्गूर्ग् धे॑हि मे मे धे॒ह्यूर्क् । \newline
11. धे॒ह्यूर्गूर्ग् धे॑हि धे॒ह्यूर्ग॑ स्य॒स्यूर्ग् धे॑हि धे॒ह्यूर्ग॑सि । \newline
12. ऊर्ग॑ स्य॒स्यूर्गूर् ग॒स्यूर्ज॒ मूर्ज॑ म॒स्यूर्गूर् ग॒स्यूर्ज᳚म् । \newline
13. अ॒स्यूर्ज॒ मूर्ज॑ मस्य॒स्यूर्ज॑म् मे म॒ ऊर्ज॑ मस्य॒स्यूर्ज॑म् मे । \newline
14. ऊर्ज॑म् मे म॒ ऊर्ज॒ मूर्ज॑म् मे धेहि धेहि म॒ ऊर्ज॒ मूर्ज॑म् मे धेहि । \newline
15. मे॒ धे॒हि॒ धे॒हि॒ मे॒ मे॒ धे॒हि॒ युङ् युङ् धे॑हि मे मे धेहि॒ युङ् । \newline
16. धे॒हि॒ युङ् युङ् धे॑हि धेहि॒ युङ् ङ॑स्यसि॒ युङ् धे॑हि धेहि॒ युङ् ङ॑सि । \newline
17. युङ् ङ॑स्यसि॒ युङ् युङ् ङ॑सि॒ वर्चो॒ वर्चो॑ ऽसि॒ युङ् युङ् ङ॑सि॒ वर्चः॑ । \newline
18. अ॒सि॒ वर्चो॒ वर्चो᳚ ऽस्यसि॒ वर्चो᳚ ऽस्यसि॒ वर्चो᳚ ऽस्यसि॒ वर्चो॑ ऽसि । \newline
19. वर्चो᳚ ऽस्यसि॒ वर्चो॒ वर्चो॑ ऽसि॒ वर्चो॒ वर्चो॑ ऽसि॒ वर्चो॒ वर्चो॑ ऽसि॒ वर्चः॑ । \newline
20. अ॒सि॒ वर्चो॒ वर्चो᳚ ऽस्यसि॒ वर्चो॒ मयि॒ मयि॒ वर्चो᳚ ऽस्यसि॒ वर्चो॒ मयि॑ । \newline
21. वर्चो॒ मयि॒ मयि॒ वर्चो॒ वर्चो॒ मयि॑ धेहि धेहि॒ मयि॒ वर्चो॒ वर्चो॒ मयि॑ धेहि । \newline
22. मयि॑ धेहि धेहि॒ मयि॒ मयि॑ धेह्य॒ग्नये॒ ऽग्नये॑ धेहि॒ मयि॒ मयि॑ धेह्य॒ग्नये᳚ । \newline
23. धे॒ह्य॒ग्नये॒ ऽग्नये॑ धेहि धेह्य॒ग्नये॑ गृ॒हप॑तये गृ॒हप॑तये॒ ऽग्नये॑ धेहि धेह्य॒ग्नये॑ गृ॒हप॑तये । \newline
24. अ॒ग्नये॑ गृ॒हप॑तये गृ॒हप॑तये॒ ऽग्नये॒ ऽग्नये॑ गृ॒हप॑तये॒ स्वाहा॒ स्वाहा॑ गृ॒हप॑तये॒ ऽग्नये॒ ऽग्नये॑ गृ॒हप॑तये॒ स्वाहा᳚ । \newline
25. गृ॒हप॑तये॒ स्वाहा॒ स्वाहा॑ गृ॒हप॑तये गृ॒हप॑तये॒ स्वाहा॒ सोमा॑य॒ सोमा॑य॒ स्वाहा॑ गृ॒हप॑तये गृ॒हप॑तये॒ स्वाहा॒ सोमा॑य । \newline
26. गृ॒हप॑तय॒ इति॑ गृ॒ह - प॒त॒ये॒ । \newline
27. स्वाहा॒ सोमा॑य॒ सोमा॑य॒ स्वाहा॒ स्वाहा॒ सोमा॑य॒ वन॒स्पत॑ये॒ वन॒स्पत॑ये॒ सोमा॑य॒ स्वाहा॒ स्वाहा॒ सोमा॑य॒ वन॒स्पत॑ये । \newline
28. सोमा॑य॒ वन॒स्पत॑ये॒ वन॒स्पत॑ये॒ सोमा॑य॒ सोमा॑य॒ वन॒स्पत॑ये॒ स्वाहा॒ स्वाहा॒ वन॒स्पत॑ये॒ सोमा॑य॒ सोमा॑य॒ वन॒स्पत॑ये॒ स्वाहा᳚ । \newline
29. वन॒स्पत॑ये॒ स्वाहा॒ स्वाहा॒ वन॒स्पत॑ये॒ वन॒स्पत॑ये॒ स्वाहेन्द्र॒स्ये न्द्र॑स्य॒ स्वाहा॒ वन॒स्पत॑ये॒ वन॒स्पत॑ये॒ स्वाहेन्द्र॑स्य । \newline
30. स्वाहेन्द्र॒स्ये न्द्र॑स्य॒ स्वाहा॒ स्वाहेन्द्र॑स्य॒ बला॑य॒ बला॒ये न्द्र॑स्य॒ स्वाहा॒ स्वाहेन्द्र॑स्य॒ बला॑य । \newline
31. इन्द्र॑स्य॒ बला॑य॒ बला॒ये न्द्र॒स्ये न्द्र॑स्य॒ बला॑य॒ स्वाहा॒ स्वाहा॒ बला॒ये न्द्र॒स्ये न्द्र॑स्य॒ बला॑य॒ स्वाहा᳚ । \newline
32. बला॑य॒ स्वाहा॒ स्वाहा॒ बला॑य॒ बला॑य॒ स्वाहा॑ म॒रुता᳚म् म॒रुताꣳ॒॒ स्वाहा॒ बला॑य॒ बला॑य॒ स्वाहा॑ म॒रुता᳚म् । \newline
33. स्वाहा॑ म॒रुता᳚म् म॒रुताꣳ॒॒ स्वाहा॒ स्वाहा॑ म॒रुता॒ मोज॑स॒ ओज॑से म॒रुताꣳ॒॒ स्वाहा॒ स्वाहा॑ म॒रुता॒ मोज॑से । \newline
34. म॒रुता॒ मोज॑स॒ ओज॑से म॒रुता᳚म् म॒रुता॒ मोज॑से॒ स्वाहा॒ स्वाहौज॑से म॒रुता᳚म् म॒रुता॒ मोज॑से॒ स्वाहा᳚ । \newline
35. ओज॑से॒ स्वाहा॒ स्वाहौज॑स॒ ओज॑से॒ स्वाहा॑ हꣳ॒॒सो हꣳ॒॒सः स्वाहौज॑स॒ ओज॑से॒ स्वाहा॑ हꣳ॒॒सः । \newline
36. स्वाहा॑ हꣳ॒॒सो हꣳ॒॒सः स्वाहा॒ स्वाहा॑ हꣳ॒॒सः शु॑चि॒षच् छु॑चि॒ष द्धꣳ॒॒सः स्वाहा॒ स्वाहा॑ हꣳ॒॒सः शु॑चि॒षत् । \newline
37. हꣳ॒॒सः शु॑चि॒षच् छु॑चि॒ष द्धꣳ॒॒सो हꣳ॒॒सः शु॑चि॒षद् वसु॒र् वसुः॑ शुचि॒ष द्धꣳ॒॒सो हꣳ॒॒सः शु॑चि॒षद् वसुः॑ । \newline
38. शु॒चि॒षद् वसु॒र् वसुः॑ शुचि॒षच् छु॑चि॒षद् वसु॑ रन्तरिक्ष॒स द॑न्तरिक्ष॒सद् वसुः॑ शुचि॒षच् छु॑चि॒षद् वसु॑ रन्तरिक्ष॒सत् । \newline
39. शु॒चि॒षदिति॑ शुचि - सत् । \newline
40. वसु॑ रन्तरिक्ष॒स द॑न्तरिक्ष॒सद् वसु॒र् वसु॑ रन्तरिक्ष॒सद् धोता॒ होता᳚ ऽन्तरिक्ष॒सद् वसु॒र् वसु॑ रन्तरिक्ष॒सद् धोता᳚ । \newline
41. अ॒न्त॒रि॒क्ष॒द् धोता॒ होता᳚ ऽन्तरिक्ष॒स द॑न्तरिक्ष॒सद् धोता॑ वेदि॒षद् वे॑दि॒षद् धोता᳚ ऽन्तरिक्ष॒स द॑न्तरिक्ष॒सद् धोता॑ वेदि॒षत् । \newline
42. अ॒न्त॒रि॒क्ष॒सदित्य॑न्तरिक्ष - सत् । \newline
43. होता॑ वेदि॒षद् वे॑दि॒षद् धोता॒ होता॑ वेदि॒ष दति॑थि॒ रति॑थिर् वेदि॒षद् धोता॒ होता॑ वेदि॒ष दति॑थिः । \newline
44. वे॒दि॒ष दति॑थि॒ रति॑थिर् वेदि॒षद् वे॑दि॒ष दति॑थिर् दुरोण॒सद् दु॑रोण॒ सदति॑थिर् वेदि॒षद् वे॑दि॒ष दति॑थिर् दुरोण॒सत् । \newline
45. वे॒दि॒षदिति॑ वेदि - सत् । \newline
46. अति॑थिर् दुरोण॒सद् दु॑रोण॒स दति॑थि॒ रति॑थिर् दुरोण॒सत् । \newline
47. दु॒रो॒ण॒सदिति॑ दुरोण - सत् । \newline
48. नृ॒षद् व॑र॒सद् व॑र॒सन् नृ॒षन् नृ॒षद् व॑र॒स दृ॑त॒स दृ॑त॒सद् व॑र॒सन् नृ॒षन् नृ॒षद् व॑र॒स दृ॑त॒सत् । \newline
49. नृ॒षदिति॑ नृ - सत् । \newline
50. व॒र॒स दृ॑त॒स दृ॑त॒सद् व॑र॒सद् व॑र॒स दृ॑त॒सद् व्यो॑म॒सद् व्यो॑म॒स दृ॑त॒सद् व॑र॒सद् व॑र॒स दृ॑त॒सद् व्यो॑म॒सत् । \newline
51. व॒र॒सदिति॑ वर - सत् । \newline
52. ऋ॒त॒सद् व्यो॑म॒सद् व्यो॑म॒स दृ॑त॒स दृ॑त॒सद् व्यो॑म॒स द॒ब्जा अ॒ब्जा व्यो॑म॒स दृ॑त॒स दृ॑त॒सद् व्यो॑म॒ सद॒ब्जाः । \newline
53. ऋ॒त॒सदित्यृ॑त - सत् । \newline
54. व्यो॒म॒स द॒ब्जा अ॒ब्जा व्यो॑म॒सद् व्यो॑म॒स द॒ब्जा गो॒जा गो॒जा अ॒ब्जा व्यो॑म॒सद् व्यो॑म॒स द॒ब्जा गो॒जाः । \newline
55. व्यो॒म॒सदिति॑ व्योम - सत् । \newline
56. अ॒ब्जा गो॒जा गो॒जा अ॒ब्जा अ॒ब्जा गो॒जा ऋ॑त॒जा ऋ॑त॒जा गो॒जा अ॒ब्जा अ॒ब्जा गो॒जा ऋ॑त॒जाः । \newline
57. अ॒ब्जा इत्य॑प् - जाः । \newline
58. गो॒जा ऋ॑त॒जा ऋ॑त॒जा गो॒जा गो॒जा ऋ॑त॒जा अ॑द्रि॒जा अ॑द्रि॒जा ऋ॑त॒जा गो॒जा गो॒जा ऋ॑त॒जा अ॑द्रि॒जाः । \newline
59. गो॒जा इति॑ गो - जाः । \newline
60. ऋ॒त॒जा अ॑द्रि॒जा अ॑द्रि॒जा ऋ॑त॒जा ऋ॑त॒जा अ॑द्रि॒जा ऋ॒त मृ॒त म॑द्रि॒जा ऋ॑त॒जा ऋ॑त॒जा अ॑द्रि॒जा ऋ॒तं । \newline
61. ऋ॒त॒जा इत्यृ॑त - जाः । \newline
62. अ॒द्रि॒जा ऋ॒त मृ॒त म॑द्रि॒जा अ॑द्रि॒जा ऋ॒तं बृ॒हद् बृ॒हदृ॒त म॑द्रि॒जा अ॑द्रि॒जा ऋ॒तं बृ॒हत् । \newline
63. अ॒द्रि॒जा इत्य॑द्रि - जाः । \newline
64. ऋ॒तं बृ॒हद् बृ॒हदृ॒त मृ॒तं बृ॒हत् । \newline
65. बृ॒हदिति॑ बृ॒हत् । \newline
\pagebreak
\markright{ TS 1.8.16.1  \hfill https://www.vedavms.in \hfill}
\addcontentsline{toc}{section}{ TS 1.8.16.1 }
\section*{ TS 1.8.16.1 }

\textbf{TS 1.8.16.1 } \newline
\textbf{Samhita Paata} \newline

मि॒त्रो॑ऽसि॒ वरु॑णोऽसि॒ सम॒हं ॅवि॒श्वै᳚र् दे॒वैः क्ष॒त्रस्य॒ नाभि॑रसि क्ष॒त्रस्य॒ योनि॑रसि स्यो॒नामा सी॑द सु॒षदा॒मा सी॑द॒ मा त्वा॑ हिꣳसी॒न्मा मा॑ हिꣳसी॒न्नि ष॑साद धृ॒तव्र॑तो॒ वरु॑णः प॒स्त्या᳚स्वा साम्रा᳚ज्याय सु॒क्रतु॒र् ब्रह्मा(3)न् त्वꣳ रा॑जन् ब्र॒ह्माऽसि॑ सवि॒ताऽसि॑ स॒त्यस॑वो॒ ब्रह्मा(3)न् त्वꣳ रा॑जन् ब्र॒ह्माऽसीन्द्रो॑ऽसि स॒त्यौजा॒ - [ ] \newline

\textbf{Pada Paata} \newline

मि॒त्रः । अ॒सि॒ । वरु॑णः । अ॒सि॒ । समिति॑ । अ॒हम् । विश्वैः᳚ । दे॒वैः । क्ष॒त्रस्य॑ । नाभिः॑ । अ॒सि॒ । क्ष॒त्रस्य॑ । योनिः॑ । अ॒सि॒ । स्यो॒नाम् । एति॑ । सी॒द॒ । सु॒षदा॒मिति॑ सु - सदा᳚म् । एति॑ । सी॒द॒ । मा । त्वा॒ । हिꣳ॒॒सी॒त् । मा । मा॒ । हिꣳ॒॒सी॒त् । नीति॑ । स॒सा॒द॒ । धृ॒तव्र॑त॒ इति॑ धृ॒त - व्र॒तः॒ । वरु॑णः । प॒स्त्या॑सु । एति॑ । साम्रा᳚ज्या॒येति॒ सां-रा॒ज्या॒य॒ । सु॒क्रतु॒रिति॑ सु- क्रतुः॑ । ब्रह्मा(3)न् । त्वम् । रा॒ज॒न्न् । ब्र॒ह्मा । अ॒सि॒ । स॒वि॒ता । अ॒सि॒ । स॒त्यस॑व॒ इति॑ स॒त्य - स॒वः॒ । ब्रह्मा(3)न् । त्वम् । रा॒ज॒न्न् । ब्र॒ह्मा । अ॒सि॒ । इन्द्रः॑ । अ॒सि॒ । स॒त्यौजा॒ इति॑ स॒त्य - ओ॒जाः॒ ।  \newline



\textbf{Jatai Paata} \newline

1. मि॒त्रो᳚ ऽस्यसि मि॒त्रो मि॒त्रो॑ ऽसि । \newline
2. अ॒सि॒ वरु॑णो॒ वरु॑णो ऽस्यसि॒ वरु॑णः । \newline
3. वरु॑णो ऽस्यसि॒ वरु॑णो॒ वरु॑णो ऽसि । \newline
4. अ॒सि॒ सꣳ स म॑स्यसि॒ सम् । \newline
5. स म॒ह म॒हꣳ सꣳ स म॒हम् । \newline
6. अ॒हं ॅविश्वै॒र् विश्वै॑ र॒ह म॒हं ॅविश्वैः᳚ । \newline
7. विश्वै᳚र् दे॒वैर् दे॒वैर् विश्वै॒र् विश्वै᳚र् दे॒वैः । \newline
8. दे॒वैः क्ष॒त्रस्य॑ क्ष॒त्रस्य॑ दे॒वैर् दे॒वैः क्ष॒त्रस्य॑ । \newline
9. क्ष॒त्रस्य॒ नाभि॒र् नाभिः॑ क्ष॒त्रस्य॑ क्ष॒त्रस्य॒ नाभिः॑ । \newline
10. नाभि॑ रस्यसि॒ नाभि॒र् नाभि॑ रसि । \newline
11. अ॒सि॒ क्ष॒त्रस्य॑ क्ष॒त्रस्या᳚स्यसि क्ष॒त्रस्य॑ । \newline
12. क्ष॒त्रस्य॒ योनि॒र् योनिः॑ क्ष॒त्रस्य॑ क्ष॒त्रस्य॒ योनिः॑ । \newline
13. योनि॑ रस्यसि॒ योनि॒र् योनि॑ रसि । \newline
14. अ॒सि॒ स्यो॒नाꣳ स्यो॒ना म॑स्यसि स्यो॒नाम् । \newline
15. स्यो॒ना मा स्यो॒नाꣳ स्यो॒ना मा । \newline
16. आ सी॑द सी॒दा सी॑द । \newline
17. सी॒द॒ सु॒षदाꣳ॑ सु॒षदाꣳ॑ सीद सीद सु॒षदा᳚म् । \newline
18. सु॒षदा॒ मा सु॒षदाꣳ॑ सु॒षदा॒ मा । \newline
19. सु॒षदा॒मिति॑ सु - सदा᳚म् । \newline
20. आ सी॑द सी॒दा सी॑द । \newline
21. सी॒द॒ मा मा सी॑द सीद॒ मा । \newline
22. मा त्वा᳚ त्वा॒ मा मा त्वा᳚ । \newline
23. त्वा॒ हिꣳ॒॒सी॒ द्धिꣳ॒॒सी॒त् त्वा॒ त्वा॒ हिꣳ॒॒सी॒त् । \newline
24. हिꣳ॒॒सी॒न् मा मा हिꣳ॑सी द्धिꣳसी॒न् मा । \newline
25. मा मा॑ मा॒ मा मा मा᳚ । \newline
26. मा॒ हिꣳ॒॒सी॒ द्धिꣳ॒॒सी॒न् मा॒ मा॒ हिꣳ॒॒सी॒त् । \newline
27. हिꣳ॒॒सी॒न् नि नि हिꣳ॑सी द्धिꣳसी॒न् नि । \newline
28. नि ष॑साद ससाद॒ नि नि ष॑साद । \newline
29. स॒सा॒द॒ धृ॒तव्र॑तो धृ॒तव्र॑तः ससाद ससाद धृ॒तव्र॑तः । \newline
30. धृ॒तव्र॑तो॒ वरु॑णो॒ वरु॑णो धृ॒तव्र॑तो धृ॒तव्र॑तो॒ वरु॑णः । \newline
31. धृ॒तव्र॑त॒ इति॑ धृ॒त - व्र॒तः॒ । \newline
32. वरु॑णः प॒स्त्या॑सु प॒स्त्या॑सु॒ वरु॑णो॒ वरु॑णः प॒स्त्या॑सु । \newline
33. प॒स्त्या᳚ स्वा प॒स्त्या॑सु प॒स्त्या᳚ स्वा । \newline
34. आ साम्रा᳚ज्याय॒ साम्रा᳚ज्या॒या साम्रा᳚ज्याय । \newline
35. साम्रा᳚ज्याय सु॒क्रतुः॑ सु॒क्रतुः॒ साम्रा᳚ज्याय॒ साम्रा᳚ज्याय सु॒क्रतुः॑ । \newline
36. साम्रा᳚ज्या॒येति॒ सां - रा॒ज्या॒य॒ । \newline
37. सु॒क्रतु॒र् ब्रह्मा(3)न् ब्रह्मा(3)न् थ्सु॒क्रतुः॑ सु॒क्रतु॒र् ब्रह्मा(3)न् । \newline
38. सु॒क्रतु॒रिति॑ सु - क्रतुः॑ । \newline
39. ब्रह्मा(3)न् त्वम् त्वम् ब्रह्मा(3)न् ब्रह्मा(3)न् त्वम् । \newline
40. त्वꣳ रा॑जन् राज॒न् त्वम् त्वꣳ रा॑जन्न् । \newline
41. रा॒ज॒न् ब्र॒ह्मा ब्र॒ह्मा रा॑जन् राजन् ब्र॒ह्मा । \newline
42. ब्र॒ह्मा ऽस्य॑सि ब्र॒ह्मा ब्र॒ह्मा ऽसि॑ । \newline
43. अ॒सि॒ स॒वि॒ता स॑वि॒ता ऽस्य॑सि सवि॒ता । \newline
44. स॒वि॒ता ऽस्य॑सि सवि॒ता स॑वि॒ता ऽसि॑ । \newline
45. अ॒सि॒ स॒त्यस॑वः स॒त्यस॑वो ऽस्यसि स॒त्यस॑वः । \newline
46. स॒त्यस॑वो॒ ब्रह्मा(3)न् ब्रह्मा(3)न् थ्स॒त्यस॑वः स॒त्यस॑वो॒ ब्रह्मा(3)न् । \newline
47. स॒त्यस॑व॒ इति॑ स॒त्य - स॒वः॒ । \newline
48. ब्रह्मा(3)न् त्वम् त्वम् ब्रह्मा(3)न् ब्रह्मा(3)न् त्वम् । \newline
49. त्वꣳ रा॑जन् राज॒न् त्वम् त्वꣳ रा॑जन्न् । \newline
50. रा॒ज॒न् ब्र॒ह्मा ब्र॒ह्मा रा॑जन् राजन् ब्र॒ह्मा । \newline
51. ब्र॒ह्मा ऽस्य॑सि ब्र॒ह्मा ब्र॒ह्मा ऽसि॑ । \newline
52. अ॒सीन्द्र॒ इन्द्रो᳚ ऽस्य॒सीन्द्रः॑ । \newline
53. इन्द्रो᳚ ऽस्य॒सीन्द्र॒ इन्द्रो॑ ऽसि । \newline
54. अ॒सि॒ स॒त्यौजाः᳚ स॒त्यौजा॑ अस्यसि स॒त्यौजाः᳚ । \newline
55. स॒त्यौजा॒ ब्रह्मा(3)न् ब्रह्मा(3)न् थ्स॒त्यौजाः᳚ स॒त्यौजा॒ ब्रह्मा(3)न् । \newline
56. स॒त्यौजा॒ इति॑ स॒त्य - ओ॒जाः॒ । \newline

\textbf{Ghana Paata } \newline

1. मि॒त्रो᳚ ऽस्यसि मि॒त्रो मि॒त्रो॑ ऽसि॒ वरु॑णो॒ वरु॑णो ऽसि मि॒त्रो मि॒त्रो॑ ऽसि॒ वरु॑णः । \newline
2. अ॒सि॒ वरु॑णो॒ वरु॑णो ऽस्यसि॒ वरु॑णो ऽस्यसि॒ वरु॑णो ऽस्यसि॒ वरु॑णो ऽसि । \newline
3. वरु॑णो ऽस्यसि॒ वरु॑णो॒ वरु॑णो ऽसि॒ सꣳ स म॑सि॒ वरु॑णो॒ वरु॑णो ऽसि॒ सम् । \newline
4. अ॒सि॒ सꣳ स म॑स्यसि॒ स म॒ह म॒हꣳ स म॑स्यसि॒ स म॒हम् । \newline
5. स म॒ह म॒हꣳ सꣳ स म॒हं ॅविश्वै॒र् विश्वै॑ र॒हꣳ सꣳ स म॒हं ॅविश्वैः᳚ । \newline
6. अ॒हं ॅविश्वै॒र् विश्वै॑ र॒ह म॒हं ॅविश्वै᳚र् दे॒वैर् दे॒वैर् विश्वै॑ र॒ह म॒हं ॅविश्वै᳚र् दे॒वैः । \newline
7. विश्वै᳚र् दे॒वैर् दे॒वैर् विश्वै॒र् विश्वै᳚र् दे॒वैः क्ष॒त्रस्य॑ क्ष॒त्रस्य॑ दे॒वैर् विश्वै॒र् विश्वै᳚र् दे॒वैः क्ष॒त्रस्य॑ । \newline
8. दे॒वैः क्ष॒त्रस्य॑ क्ष॒त्रस्य॑ दे॒वैर् दे॒वैः क्ष॒त्रस्य॒ नाभि॒र् नाभिः॑ क्ष॒त्रस्य॑ दे॒वैर् दे॒वैः क्ष॒त्रस्य॒ नाभिः॑ । \newline
9. क्ष॒त्रस्य॒ नाभि॒र् नाभिः॑ क्ष॒त्रस्य॑ क्ष॒त्रस्य॒ नाभि॑ रस्यसि॒ नाभिः॑ क्ष॒त्रस्य॑ क्ष॒त्रस्य॒ नाभि॑ रसि । \newline
10. नाभि॑रस्यसि॒ नाभि॒र् नाभि॑रसि क्ष॒त्रस्य॑ क्ष॒त्रस्या॑सि॒ नाभि॒र् नाभि॑रसि क्ष॒त्रस्य॑ । \newline
11. अ॒सि॒ क्ष॒त्रस्य॑ क्ष॒त्रस्या᳚स्यसि क्ष॒त्रस्य॒ योनि॒र् योनिः॑ क्ष॒त्रस्या᳚स्यसि क्ष॒त्रस्य॒ योनिः॑ । \newline
12. क्ष॒त्रस्य॒ योनि॒र् योनिः॑ क्ष॒त्रस्य॑ क्ष॒त्रस्य॒ योनि॑ रस्यसि॒ योनिः॑ क्ष॒त्रस्य॑ क्ष॒त्रस्य॒ योनि॑ रसि । \newline
13. योनि॑ रस्यसि॒ योनि॒र् योनि॑ रसि स्यो॒नाꣳ स्यो॒ना म॑सि॒ योनि॒र् योनि॑ रसि स्यो॒नाम् । \newline
14. अ॒सि॒ स्यो॒नाꣳ स्यो॒ना म॑स्यसि स्यो॒ना मा स्यो॒ना म॑स्यसि स्यो॒ना मा । \newline
15. स्यो॒ना मा स्यो॒नाꣳ स्यो॒ना मा सी॑द सी॒दा स्यो॒नाꣳ स्यो॒ना मा सी॑द । \newline
16. आ सी॑द सी॒दा सी॑द सु॒षदाꣳ॑ सु॒षदाꣳ॑ सी॒दा सी॑द सु॒षदा᳚म् । \newline
17. सी॒द॒ सु॒षदाꣳ॑ सु॒षदाꣳ॑ सीद सीद सु॒षदा॒ मा सु॒षदाꣳ॑ सीद सीद सु॒षदा॒ मा । \newline
18. सु॒षदा॒ मा सु॒षदाꣳ॑ सु॒षदा॒ मा सी॑द सी॒दा सु॒षदाꣳ॑ सु॒षदा॒ मा सी॑द । \newline
19. सु॒षदा॒मिति॑ सु - सदा᳚म् । \newline
20. आ सी॑द सी॒दा सी॑द॒ मा मा सी॒दा सी॑द॒ मा । \newline
21. सी॒द॒ मा मा सी॑द सीद॒ मा त्वा᳚ त्वा॒ मा सी॑द सीद॒ मा त्वा᳚ । \newline
22. मा त्वा᳚ त्वा॒ मा मा त्वा॑ हिꣳसी द्धिꣳसीत् त्वा॒ मा मा त्वा॑ हिꣳसीत् । \newline
23. त्वा॒ हिꣳ॒॒सी॒ द्धिꣳ॒॒सी॒त् त्वा॒ त्वा॒ हिꣳ॒॒सी॒न् मा मा हिꣳ॑सीत् त्वा त्वा हिꣳसी॒न् मा । \newline
24. हिꣳ॒॒सी॒न् मा मा हिꣳ॑सी द्धिꣳसी॒न् मा मा॑ मा॒ मा हिꣳ॑सी द्धिꣳसी॒न् मा मा᳚ । \newline
25. मा मा॑ मा॒ मा मा मा॑ हिꣳसी द्धिꣳसीन् मा॒ मा मा मा॑ हिꣳसीत् । \newline
26. मा॒ हिꣳ॒॒सी॒ द्धिꣳ॒॒सी॒न् मा॒ मा॒ हिꣳ॒॒सी॒न् नि नि हिꣳ॑सीन् मा मा हिꣳसी॒न् नि । \newline
27. हिꣳ॒॒सी॒न् नि नि हिꣳ॑सी द्धिꣳसी॒न् नि ष॑साद ससाद॒ नि हिꣳ॑सी द्धिꣳसी॒न् नि ष॑साद । \newline
28. नि ष॑साद ससाद॒ नि नि ष॑साद धृ॒तव्र॑तो धृ॒तव्र॑तः ससाद॒ नि नि ष॑साद धृ॒तव्र॑तः । \newline
29. स॒सा॒द॒ धृ॒तव्र॑तो धृ॒तव्र॑तः ससाद ससाद धृ॒तव्र॑तो॒ वरु॑णो॒ वरु॑णो धृ॒तव्र॑तः ससाद ससाद धृ॒तव्र॑तो॒ वरु॑णः । \newline
30. धृ॒तव्र॑तो॒ वरु॑णो॒ वरु॑णो धृ॒तव्र॑तो धृ॒तव्र॑तो॒ वरु॑णः प॒स्त्या॑सु प॒स्त्या॑सु॒ वरु॑णो धृ॒तव्र॑तो धृ॒तव्र॑तो॒ वरु॑णः प॒स्त्या॑सु । \newline
31. धृ॒तव्र॑त॒ इति॑ धृ॒त - व्र॒तः॒ । \newline
32. वरु॑णः प॒स्त्या॑सु प॒स्त्या॑सु॒ वरु॑णो॒ वरु॑णः प॒स्त्या᳚स्वा प॒स्त्या॑सु॒ वरु॑णो॒ वरु॑णः प॒स्त्या᳚स्वा । \newline
33. प॒स्त्या᳚स्वा प॒स्त्या॑सु प॒स्त्या᳚स्वा साम्रा᳚ज्याय॒ साम्रा᳚ज्या॒या प॒स्त्या॑सु प॒स्त्या᳚स्वा साम्रा᳚ज्याय । \newline
34. आ साम्रा᳚ज्याय॒ साम्रा᳚ज्या॒या साम्रा᳚ज्याय सु॒क्रतुः॑ सु॒क्रतुः॒ साम्रा᳚ज्या॒या साम्रा᳚ज्याय सु॒क्रतुः॑ । \newline
35. साम्रा᳚ज्याय सु॒क्रतुः॑ सु॒क्रतुः॒ साम्रा᳚ज्याय॒ साम्रा᳚ज्याय सु॒क्रतु॒र् ब्रह्मा(3)न् ब्रह्मा(3)न् थ्सु॒क्रतुः॒ साम्रा᳚ज्याय॒ साम्रा᳚ज्याय सु॒क्रतु॒र् ब्रह्मा(3)न् । \newline
36. साम्रा᳚ज्या॒येति॒ सां - रा॒ज्या॒य॒ । \newline
37. सु॒क्रतु॒र् ब्रह्मा(3)न् ब्रह्मा(3)न् थ्सु॒क्रतुः॑ सु॒क्रतु॒र् ब्रह्मा(3)न् त्वम् त्वम् ब्रह्मा(3)न् थ्सु॒क्रतुः॑ सु॒क्रतु॒र् ब्रह्मा(3)न् त्वम् । \newline
38. सु॒क्रतु॒रिति॑ सु - क्रतुः॑ । \newline
39. ब्रह्मा(3)न् त्वम् त्वम् ब्रह्मा(3)न् ब्रह्मा(3)न् त्वꣳ रा॑जन् राज॒न् त्वम् ब्रह्मा(3)न् ब्रह्मा(3)न् त्वꣳ रा॑जन्न् । \newline
40. त्वꣳ रा॑जन् राज॒न् त्वम् त्वꣳ रा॑जन् ब्र॒ह्मा ब्र॒ह्मा रा॑ज॒न् त्वम् त्वꣳ रा॑जन् ब्र॒ह्मा । \newline
41. रा॒ज॒न् ब्र॒ह्मा ब्र॒ह्मा रा॑जन् राजन् ब्र॒ह्मा ऽस्य॑सि ब्र॒ह्मा रा॑जन् राजन् ब्र॒ह्मा ऽसि॑ । \newline
42. ब्र॒ह्मा ऽस्य॑सि ब्र॒ह्मा ब्र॒ह्मा ऽसि॑ सवि॒ता स॑वि॒ता ऽसि॑ ब्र॒ह्मा ब्र॒ह्मा ऽसि॑ सवि॒ता । \newline
43. अ॒सि॒ स॒वि॒ता स॑वि॒ता ऽस्य॑सि सवि॒ता ऽस्य॑सि सवि॒ता ऽस्य॑सि सवि॒ता ऽसि॑ । \newline
44. स॒वि॒ता ऽस्य॑सि सवि॒ता स॑वि॒ता ऽसि॑ स॒त्यस॑वः स॒त्यस॑वो ऽसि सवि॒ता स॑वि॒ता ऽसि॑ स॒त्यस॑वः । \newline
45. अ॒सि॒ स॒त्यस॑वः स॒त्यस॑वो ऽस्यसि स॒त्यस॑वो॒ ब्रह्मा(3)न् ब्रह्मा(3)न् थ्स॒त्यस॑वो ऽस्यसि स॒त्यस॑वो॒ ब्रह्मा(3)न् । \newline
46. स॒त्यस॑वो॒ ब्रह्मा(3)न् ब्रह्मा(3)न् थ्स॒त्यस॑वः स॒त्यस॑वो॒ ब्रह्मा(3)न् त्वम् त्वम् ब्रह्मा(3)न् थ्स॒त्यस॑वः स॒त्यस॑वो॒ ब्रह्मा(3)न् त्वम् । \newline
47. स॒त्यस॑व॒ इति॑ स॒त्य - स॒वः॒ । \newline
48. ब्रह्मा(3)न् त्वम् त्वम् ब्रह्मा(3)न् ब्रह्मा(3)न् त्वꣳ रा॑जन् राज॒न् त्वम् ब्रह्मा(3)न् ब्रह्मा(3)न् त्वꣳ रा॑जन्न् । \newline
49. त्वꣳ रा॑जन् राज॒न् त्वम् त्वꣳ रा॑जन् ब्र॒ह्मा ब्र॒ह्मा रा॑ज॒न् त्वम् त्वꣳ रा॑जन् ब्र॒ह्मा । \newline
50. रा॒ज॒न् ब्र॒ह्मा ब्र॒ह्मा रा॑जन् राजन् ब्र॒ह्मा ऽस्य॑सि ब्र॒ह्मा रा॑जन् राजन् ब्र॒ह्मा ऽसि॑ । \newline
51. ब्र॒ह्मा ऽस्य॑सि ब्र॒ह्मा ब्र॒ह्मा ऽसीन्द्र॒ इन्द्रो॑ ऽसि ब्र॒ह्मा ब्र॒ह्मा ऽसीन्द्रः॑ । \newline
52. अ॒सीन्द्र॒ इन्द्रो᳚ ऽस्य॒सीन्द्रो᳚ ऽस्य॒सीन्द्रो᳚ ऽस्य॒सीन्द्रो॑ ऽसि । \newline
53. इन्द्रो᳚ ऽस्य॒सीन्द्र॒ इन्द्रो॑ ऽसि स॒त्यौजाः᳚ स॒त्यौजा॑ अ॒सीन्द्र॒ इन्द्रो॑ ऽसि स॒त्यौजाः᳚ । \newline
54. अ॒सि॒ स॒त्यौजाः᳚ स॒त्यौजा॑ अस्यसि स॒त्यौजा॒ ब्रह्मा(3)न् ब्रह्मा(3)न् थ्स॒त्यौजा॑ अस्यसि स॒त्यौजा॒ ब्रह्मा(3)न् । \newline
55. स॒त्यौजा॒ ब्रह्मा(3)न् ब्रह्मा(3)न् थ्स॒त्यौजाः᳚ स॒त्यौजा॒ ब्रह्मा(3)न् त्वम् त्वम् ब्रह्मा(3)न् थ्स॒त्यौजाः᳚ स॒त्यौजा॒ ब्रह्मा(3)न् त्वम् । \newline
56. स॒त्यौजा॒ इति॑ स॒त्य - ओ॒जाः॒ । \newline
\pagebreak
\markright{ TS 1.8.16.2  \hfill https://www.vedavms.in \hfill}
\addcontentsline{toc}{section}{ TS 1.8.16.2 }
\section*{ TS 1.8.16.2 }

\textbf{TS 1.8.16.2 } \newline
\textbf{Samhita Paata} \newline

ब्रह्मा(3)न् त्वꣳ रा॑जन् ब्र॒ह्माऽसि॑ मि॒त्रो॑ऽसि सु॒शेवो॒ ब्रह्मा(3)न् त्वꣳ रा॑जन् ब्र॒ह्माऽसि॒ वरु॑णोऽसि स॒त्यध॒र्मेन्द्र॑स्य॒ वज्रो॑ऽसि॒ वार्त्र॑घ्न॒स्तेन॑ मे रद्ध्य॒ दिशो॒ऽभ्य॑यꣳ राजा॑ऽभू॒थ् सुश्लो॒काॅ(4) सुम॑ङ्ग॒लाॅ(4) सत्य॑रा॒जा(3)न् । अ॒पां नप्त्रे॒ स्वाहो॒र्जो नप्त्रे॒ स्वाहा॒ऽग्नये॑ गृ॒हप॑तये॒ स्वाहा᳚ ॥ \newline

\textbf{Pada Paata} \newline

ब्रह्मा(3)न् । त्वम् । रा॒ज॒न्न् । ब्र॒ह्मा । अ॒सि॒ । मि॒त्रः । अ॒सि॒ । सु॒शेव॒ इति॑ सु - शेवः॑ । ब्रह्मा(3)न् । त्वम् । रा॒ज॒न्न् । ब्र॒ह्मा । अ॒सि॒ । वरु॑णः । अ॒सि॒ । स॒त्यध॒र्मेति॑ स॒त्य - ध॒र्मा॒ । इन्द्र॑स्य । वज्रः॑ । अ॒सि॒ । वार्त्र॑घ्न॒ इति॒ वार्त्र॑ - घ्नः॒ । तेन॑ । मे॒ । र॒द्ध्य॒ । दिशः॑ । अ॒भीति॑ । अ॒यम् । राजा᳚ । अ॒भू॒त् । सुश्लो॒काॅ(4) इति॒ सु - श्लो॒काॅ(4) । सुम॑ङ्ग॒लाॅ(4) इति॒ सु - म॒ङ्ग॒लाॅ(4) । सत्य॑रा॒जा(3)निति॒ सत्य॑ - रा॒जा(3)न् ॥ अ॒पाम् । नप्त्रे᳚ । स्वाहा᳚ । ऊ॒र्जः । नप्त्रे᳚ । स्वाहा᳚ । अ॒ग्नये᳚ । गृ॒हप॑तय॒ इति॑ गृ॒ह - प॒त॒ये॒ । स्वाहा᳚ ॥ 32  \newline



\textbf{Jatai Paata} \newline

1. ब्रह्मा(3)न् त्वम् त्वम् ब्रह्मा(3)न् ब्रह्मा(3)न् त्वम् । \newline
2. त्वꣳ रा॑जन् राज॒न् त्वम् त्वꣳ रा॑जन्न् । \newline
3. रा॒ज॒न् ब्र॒ह्मा ब्र॒ह्मा रा॑जन् राजन् ब्र॒ह्मा । \newline
4. ब्र॒ह्मा ऽस्य॑सि ब्र॒ह्मा ब्र॒ह्मा ऽसि॑ । \newline
5. अ॒सि॒ मि॒त्रो मि॒त्रो᳚ ऽस्यसि मि॒त्रः । \newline
6. मि॒त्रो᳚ ऽस्यसि मि॒त्रो मि॒त्रो॑ ऽसि । \newline
7. अ॒सि॒ सु॒शेवः॑ सु॒शेवो᳚ ऽस्यसि सु॒शेवः॑ । \newline
8. सु॒शेवो॒ ब्रह्मा(3)न् ब्रह्मा(3)न् थ्सु॒शेवः॑ सु॒शेवो॒ ब्रह्मा(3)न् । \newline
9. सु॒शेव॒ इति॑ सु - शेवः॑ । \newline
10. ब्रह्मा(3)न् त्वम् त्वम् ब्रह्मा(3)न् ब्रह्मा(3)न् त्वम् । \newline
11. त्वꣳ रा॑जन् राज॒न् त्वम् त्वꣳ रा॑जन्न् । \newline
12. रा॒ज॒न् ब्र॒ह्मा ब्र॒ह्मा रा॑जन् राजन् ब्र॒ह्मा । \newline
13. ब्र॒ह्मा ऽस्य॑सि ब्र॒ह्मा ब्र॒ह्मा ऽसि॑ । \newline
14. अ॒सि॒ वरु॑णो॒ वरु॑णो ऽस्यसि॒ वरु॑णः । \newline
15. वरु॑णो ऽस्यसि॒ वरु॑णो॒ वरु॑णो ऽसि । \newline
16. अ॒सि॒ स॒त्यध॑र्मा स॒त्यध॑र्मा ऽस्यसि स॒त्यध॑र्मा । \newline
17. स॒त्यध॒र्मे न्द्र॒स्ये न्द्र॑स्य स॒त्यध॑र्मा स॒त्यध॒र्मे न्द्र॑स्य । \newline
18. स॒त्यध॒र्मेति॑ स॒त्य - ध॒र्मा॒ । \newline
19. इन्द्र॑स्य॒ वज्रो॒ वज्र॒ इन्द्र॒स्ये न्द्र॑स्य॒ वज्रः॑ । \newline
20. वज्रो᳚ ऽस्यसि॒ वज्रो॒ वज्रो॑ ऽसि । \newline
21. अ॒सि॒ वार्त्र॑घ्नो॒ वार्त्र॑घ्नो ऽस्यसि॒ वार्त्र॑घ्नः । \newline
22. वार्त्र॑घ्न॒ स्तेन॒ तेन॒ वार्त्र॑घ्नो॒ वार्त्र॑घ्न॒ स्तेन॑ । \newline
23. वार्त्र॑घ्न॒ इति॒ वार्त्र॑ - घ्नः॒ । \newline
24. तेन॑ मे मे॒ तेन॒ तेन॑ मे । \newline
25. मे॒ र॒द्ध्य॒ र॒द्ध्य॒ मे॒ मे॒ र॒द्ध्य॒ । \newline
26. र॒द्ध्य॒ दिशो॒ दिशो॑ रद्ध्य रद्ध्य॒ दिशः॑ । \newline
27. दिशो॒ ऽभ्य॑भि दिशो॒ दिशो॒ ऽभि । \newline
28. अ॒भ्य॑य म॒य म॒भ्या᳚(1॒)भ्य॑यम् । \newline
29. अ॒यꣳ राजा॒ राजा॒ ऽय म॒यꣳ राजा᳚ । \newline
30. राजा॑ ऽभू दभू॒द् राजा॒ राजा॑ ऽभूत् । \newline
31. अ॒भू॒थ् सुश्लो॒काॅ(4) सुश्लो॒काॅ(4) अ॑भूदभू॒थ् सुश्लो॒काॅ(4) । \newline
32. सुश्लो॒काॅ(4) सुम॑ङ्ग॒लाॅ(4) सुम॑ङ्ग॒लाॅ(4) सुश्लो॒काॅ(4) सुश्लो॒काॅ(4) सुम॑ङ्ग॒लाॅ(4) । \newline
33. सुश्लो॒काॅ(4) इति॒ सु - श्लो॒काॅ(4) । \newline
34. सुम॑ङ्ग॒लाॅ(4) सत्य॑रा॒जा(3)न् थ्सत्य॑रा॒जा(3)न् थ्सुम॑ङ्ग॒लाॅ(4) सुम॑ङ्ग॒लाॅ(4) सत्य॑रा॒जा(3)न् । \newline
35. सुम॑ङ्ग॒लाॅ(4) इति॒ सु - म॒ङ्ग॒लाॅ(4) । \newline
36. सत्य॑रा॒जा(3)निति॒ सत्य॑ - रा॒जा(3)न् । \newline
37. अ॒पाम् नप्त्रे॒ नप्त्रे॒ ऽपा म॒पाम् नप्त्रे᳚ । \newline
38. नप्त्रे॒ स्वाहा॒ स्वाहा॒ नप्त्रे॒ नप्त्रे॒ स्वाहा᳚ । \newline
39. स्वाहो॒र्ज ऊ॒र्जः स्वाहा॒ स्वाहो॒र्जः । \newline
40. ऊ॒र्जो नप्त्रे॒ नप्त्र॑ ऊ॒र्ज ऊ॒र्जो नप्त्रे᳚ । \newline
41. नप्त्रे॒ स्वाहा॒ स्वाहा॒ नप्त्रे॒ नप्त्रे॒ स्वाहा᳚ । \newline
42. स्वाहा॒ ऽग्नये॒ ऽग्नये॒ स्वाहा॒ स्वाहा॒ ऽग्नये᳚ । \newline
43. अ॒ग्नये॑ गृ॒हप॑तये गृ॒हप॑तये॒ ऽग्नये॒ ऽग्नये॑ गृ॒हप॑तये । \newline
44. गृ॒हप॑तये॒ स्वाहा॒ स्वाहा॑ गृ॒हप॑तये गृ॒हप॑तये॒ स्वाहा᳚ । \newline
45. गृ॒हप॑तय॒ इति॑ गृ॒ह - प॒त॒ये॒ । \newline
46. स्वाहेति॒ स्वाहा᳚ । \newline

\textbf{Ghana Paata } \newline

1. ब्रह्मा(3)न् त्वम् त्वम् ब्रह्मा(3)न् ब्रह्मा(3)न् त्वꣳ रा॑जन् राज॒न् त्वम् ब्रह्मा(3)न् ब्रह्मा(3)न् त्वꣳ रा॑जन्न् । \newline
2. त्वꣳ रा॑जन् राज॒न् त्वम् त्वꣳ रा॑जन् ब्र॒ह्मा ब्र॒ह्मा रा॑ज॒न् त्वम् त्वꣳ रा॑जन् ब्र॒ह्मा । \newline
3. रा॒ज॒न् ब्र॒ह्मा ब्र॒ह्मा रा॑जन् राजन् ब्र॒ह्मा ऽस्य॑सि ब्र॒ह्मा रा॑जन् राजन् ब्र॒ह्मा ऽसि॑ । \newline
4. ब्र॒ह्मा ऽस्य॑सि ब्र॒ह्मा ब्र॒ह्मा ऽसि॑ मि॒त्रो मि॒त्रो॑ ऽसि ब्र॒ह्मा ब्र॒ह्मा ऽसि॑ मि॒त्रः । \newline
5. अ॒सि॒ मि॒त्रो मि॒त्रो᳚ ऽस्यसि मि॒त्रो᳚ ऽस्यसि मि॒त्रो᳚ ऽस्यसि मि॒त्रो॑ ऽसि । \newline
6. मि॒त्रो᳚ ऽस्यसि मि॒त्रो मि॒त्रो॑ ऽसि सु॒शेवः॑ सु॒शेवो॑ ऽसि मि॒त्रो मि॒त्रो॑ ऽसि सु॒शेवः॑ । \newline
7. अ॒सि॒ सु॒शेवः॑ सु॒शेवो᳚ ऽस्यसि सु॒शेवो॒ ब्रह्मा(3)न् ब्रह्मा(3)न् थ्सु॒शेवो᳚ ऽस्यसि सु॒शेवो॒ ब्रह्मा(3)न् । \newline
8. सु॒शेवो॒ ब्रह्मा(3)न् ब्रह्मा(3)न् थ्सु॒शेवः॑ सु॒शेवो॒ ब्रह्मा(3)न् त्वम् त्वम् ब्रह्मा(3)न् थ्सु॒शेवः॑ सु॒शेवो॒ ब्रह्मा(3)न् त्वम् । \newline
9. सु॒शेव॒ इति॑ सु - शेवः॑ । \newline
10. ब्रह्मा(3)न् त्वम् त्वम् ब्रह्मा(3)न् ब्रह्मा(3)न् त्वꣳ रा॑जन् राज॒न् त्वम् ब्रह्मा(3)न् ब्रह्मा(3)न् त्वꣳ रा॑जन्न् । \newline
11. त्वꣳ रा॑जन् राज॒न् त्वम् त्वꣳ रा॑जन् ब्र॒ह्मा ब्र॒ह्मा रा॑ज॒न् त्वम् त्वꣳ रा॑जन् ब्र॒ह्मा । \newline
12. रा॒ज॒न् ब्र॒ह्मा ब्र॒ह्मा रा॑जन् राजन् ब्र॒ह्मा ऽस्य॑सि ब्र॒ह्मा रा॑जन् राजन् ब्र॒ह्मा ऽसि॑ । \newline
13. ब्र॒ह्मा ऽस्य॑सि ब्र॒ह्मा ब्र॒ह्मा ऽसि॒ वरु॑णो॒ वरु॑णो ऽसि ब्र॒ह्मा ब्र॒ह्मा ऽसि॒ वरु॑णः । \newline
14. अ॒सि॒ वरु॑णो॒ वरु॑णो ऽस्यसि॒ वरु॑णो ऽस्यसि॒ वरु॑णो ऽस्यसि॒ वरु॑णो ऽसि । \newline
15. वरु॑णो ऽस्यसि॒ वरु॑णो॒ वरु॑णो ऽसि स॒त्यध॑र्मा स॒त्यध॑र्मा ऽसि॒ वरु॑णो॒ वरु॑णो ऽसि स॒त्यध॑र्मा । \newline
16. अ॒सि॒ स॒त्यध॑र्मा स॒त्यध॑र्मा ऽस्यसि स॒त्यध॒र्मे न्द्र॒स्ये न्द्र॑स्य स॒त्यध॑र्मा ऽस्यसि स॒त्यध॒र्मे न्द्र॑स्य । \newline
17. स॒त्यध॒र्मे न्द्र॒स्ये न्द्र॑स्य स॒त्यध॑र्मा स॒त्यध॒र्मे न्द्र॑स्य॒ वज्रो॒ वज्र॒ इन्द्र॑स्य स॒त्यध॑र्मा 
स॒त्यध॒र्मे न्द्र॑स्य॒ वज्रः॑ । \newline
18. स॒त्यध॒र्मेति॑ स॒त्य - ध॒र्मा॒ । \newline
19. इन्द्र॑स्य॒ वज्रो॒ वज्र॒ इन्द्र॒स्ये न्द्र॑स्य॒ वज्रो᳚ ऽस्यसि॒ वज्र॒ इन्द्र॒स्ये न्द्र॑स्य॒ वज्रो॑ ऽसि । \newline
20. वज्रो᳚ ऽस्यसि॒ वज्रो॒ वज्रो॑ ऽसि॒ वार्त्र॑घ्नो॒ वार्त्र॑घ्नो ऽसि॒ वज्रो॒ वज्रो॑ ऽसि॒ वार्त्र॑घ्नः । \newline
21. अ॒सि॒ वार्त्र॑घ्नो॒ वार्त्र॑घ्नो ऽस्यसि॒ वार्त्र॑घ्न॒ स्तेन॒ तेन॒ वार्त्र॑घ्नो ऽस्यसि॒ वार्त्र॑घ्न॒ स्तेन॑ । \newline
22. वार्त्र॑घ्न॒ स्तेन॒ तेन॒ वार्त्र॑घ्नो॒ वार्त्र॑घ्न॒ स्तेन॑ मे मे॒ तेन॒ वार्त्र॑घ्नो॒ वार्त्र॑घ्न॒ स्तेन॑ मे । \newline
23. वार्त्र॑घ्न॒ इति॒ वार्त्र॑ - घ्नः॒ । \newline
24. तेन॑ मे मे॒ तेन॒ तेन॑ मे रद्ध्य रद्ध्य मे॒ तेन॒ तेन॑ मे रद्ध्य । \newline
25. मे॒ र॒द्ध्य॒ र॒द्ध्य॒ मे॒ मे॒ र॒द्ध्य॒ दिशो॒ दिशो॑ रद्ध्य मे मे रद्ध्य॒ दिशः॑ । \newline
26. र॒द्ध्य॒ दिशो॒ दिशो॑ रद्ध्य रद्ध्य॒ दिशो॒ ऽभ्य॑भि दिशो॑ रद्ध्य रद्ध्य॒ दिशो॒ ऽभि । \newline
27. दिशो॒ ऽभ्य॑भि दिशो॒ दिशो॒ ऽभ्य॑य म॒य म॒भि दिशो॒ दिशो॒ ऽभ्य॑यम् । \newline
28. अ॒भ्य॑य म॒य म॒भ्या᳚(1॒)भ्य॑यꣳ राजा॒ राजा॒ ऽय म॒भ्या᳚(1॒)भ्य॑यꣳ राजा᳚ । \newline
29. अ॒यꣳ राजा॒ राजा॒ ऽय म॒यꣳ राजा॑ ऽभूदभू॒द् राजा॒ ऽय म॒यꣳ राजा॑ ऽभूत् । \newline
30. राजा॑ ऽभूदभू॒द् राजा॒ राजा॑ ऽभू॒थ् सुश्लो॒काॅ(4) सुश्लो॒काॅ(4) अ॑भू॒द् राजा॒ राजा॑ ऽभू॒थ् सुश्लो॒काॅ(4) । \newline
31. अ॒भू॒थ् सुश्लो॒काॅ(4) सुश्लो॒काॅ(4) अ॑भू दभू॒थ् सुश्लो॒काॅ(4) सुम॑ङ्ग॒लाॅ(4) सुम॑ङ्ग॒लाॅ(4) सुश्लो॒काॅ(4) 
अ॑भू दभू॒थ् सुश्लो॒काॅ(4) सुम॑ङ्ग॒लाॅ(4) । \newline
32. सुश्लो॒काॅ(4) सुम॑ङ्ग॒लाॅ(4) सुम॑ङ्ग॒लाॅ(4) सुश्लो॒काॅ(4) सुश्लो॒काॅ(4) सुम॑ङ्ग॒लाॅ(4) सत्य॑रा॒जा(3)न् थ्सत्य॑रा॒जा(3)न् थ्सुम॑ङ्ग॒लाॅ(4) सुश्लो॒काॅ(4) सुश्लो॒काॅ(4) सुम॑ङ्ग॒लाॅ(4) सत्य॑रा॒जा(3)न् । \newline
33. सुश्लो॒काॅ(4) इति॒ सु - श्लो॒काॅ(4) । \newline
34. सुम॑ङ्ग॒लाॅ(4) सत्य॑रा॒जा(3)न् थ्सत्य॑रा॒जा(3)न् थ्सुम॑ङ्ग॒लाॅ(4) सुम॑ङ्ग॒लाॅ(4) सत्य॑रा॒जा(3)न् । \newline
35. सुम॑ङ्ग॒लाॅ(4) इति॒ सु - म॒ङ्ग॒लाॅ(4) । \newline
36. सत्य॑रा॒जा(3)निति॒ सत्य॑ - रा॒जा(3)न् । \newline
37. अ॒पान्नप्त्रे॒ नप्त्रे॒ ऽपा म॒पान्नप्त्रे॒ स्वाहा॒ स्वाहा॒ नप्त्रे॒ ऽपा म॒पान्नप्त्रे॒ स्वाहा᳚ । \newline
38. नप्त्रे॒ स्वाहा॒ स्वाहा॒ नप्त्रे॒ नप्त्रे॒ स्वाहो॒र्ज ऊ॒र्जः स्वाहा॒ नप्त्रे॒ नप्त्रे॒ स्वाहो॒र्जः । \newline
39. स्वाहो॒र्ज ऊ॒र्जः स्वाहा॒ स्वाहो॒र्जो नप्त्रे॒ नप्त्र॑ ऊ॒र्जः स्वाहा॒ स्वाहो॒र्जो नप्त्रे᳚ । \newline
40. ऊ॒र्जो नप्त्रे॒ नप्त्र॑ ऊ॒र्ज ऊ॒र्जो नप्त्रे॒ स्वाहा॒ स्वाहा॒ नप्त्र॑ ऊ॒र्ज ऊ॒र्जो नप्त्रे॒ स्वाहा᳚ । \newline
41. नप्त्रे॒ स्वाहा॒ स्वाहा॒ नप्त्रे॒ नप्त्रे॒ स्वाहा॒ ऽग्नये॒ ऽग्नये॒ स्वाहा॒ नप्त्रे॒ नप्त्रे॒ स्वाहा॒ ऽग्नये᳚ । \newline
42. स्वाहा॒ ऽग्नये॒ ऽग्नये॒ स्वाहा॒ स्वाहा॒ ऽग्नये॑ गृ॒हप॑तये गृ॒हप॑तये॒ ऽग्नये॒ स्वाहा॒ स्वाहा॒ ऽग्नये॑ गृ॒हप॑तये । \newline
43. अ॒ग्नये॑ गृ॒हप॑तये गृ॒हप॑तये॒ ऽग्नये॒ ऽग्नये॑ गृ॒हप॑तये॒ स्वाहा॒ स्वाहा॑ गृ॒हप॑तये॒ ऽग्नये॒ ऽग्नये॑ गृ॒हप॑तये॒ स्वाहा᳚ । \newline
44. गृ॒हप॑तये॒ स्वाहा॒ स्वाहा॑ गृ॒हप॑तये गृ॒हप॑तये॒ स्वाहा᳚ । \newline
45. गृ॒हप॑तय॒ इति॑ गृ॒ह - प॒त॒ये॒ । \newline
46. स्वाहेति॒ स्वाहा᳚ । \newline
\pagebreak
\markright{ TS 1.8.17.1  \hfill https://www.vedavms.in \hfill}
\addcontentsline{toc}{section}{ TS 1.8.17.1 }
\section*{ TS 1.8.17.1 }

\textbf{TS 1.8.17.1 } \newline
\textbf{Samhita Paata} \newline

आ॒ग्ने॒यम॒ष्टाक॑पालं॒ निर्व॑पति॒ हिर॑ण्यं॒ दक्षि॑णा सारस्व॒तं च॒रुं ॅव॑थ्सत॒री दक्षि॑णा सावि॒त्रं द्वाद॑शकपाल-मुपद्ध्व॒स्तो दक्षि॑णा पौ॒ष्णं च॒रुꣳ श्या॒मो दक्षि॑णा बार्.हस्प॒त्यं च॒रुꣳ शि॑तिपृ॒ष्ठो दक्षि॑णै॒न्द्र-मेका॑दशकपाल-मृष॒भो दक्षि॑णा वारु॒णं दश॑कपालं म॒हानि॑रष्टो॒ दक्षि॑णा सौ॒म्यं च॒रुं ब॒भ्रुर् दक्षि॑णा त्वा॒ष्ट्रम॒ष्टाक॑पालꣳ शु॒ण्ठो दक्षि॑णा वैष्ण॒वं त्रि॑कपा॒लं वा॑म॒नो दक्षि॑णा ॥ \newline

\textbf{Pada Paata} \newline

आ॒ग्ने॒यम् । अ॒ष्टाक॑पाल॒मित्य॒ष्टा-क॒पा॒ल॒म् । निरिति॑ । व॒प॒ति॒ । हिर॑ण्यम् । दक्षि॑णा । सा॒र॒स्व॒तम् । च॒रुम् । व॒थ्स॒त॒री । दक्षि॑णा । सा॒वि॒त्रम् । द्वाद॑शकपाल॒मिति॒ द्वाद॑श - क॒पा॒ल॒म् । उ॒प॒द्ध्व॒स्त इत्यु॑प - ध्व॒स्तः । दक्षि॑णा । पौ॒ष्णम् । च॒रुम् । श्या॒मः । दक्षि॑णा । बा॒र्.॒ह॒स्प॒त्यम् । च॒रुम् । शि॒ति॒पृ॒ष्ठ इति॑ शिति - पृ॒ष्ठः । दक्षि॑णा । ऐ॒न्द्रम् । एका॑दशकपाल॒मित्येका॑दश-क॒पा॒ल॒म् । ऋ॒ष॒भः । दक्षि॑णा । वा॒रु॒णम् । दश॑कपाल॒मिति॒ दश॑ - क॒पा॒ल॒म् । म॒हानि॑रष्ट॒ इति॑ म॒हा-नि॒र॒ष्टः॒ । दक्षि॑णा । सौ॒म्यम् । च॒रुम् । ब॒भ्रुः । दक्षि॑णा । त्वा॒ष्ट्रम् । अ॒ष्टाक॑पाल॒मित्य॒ष्टा - क॒पा॒ल॒म् । शु॒ण्ठः । दक्षि॑णा । वै॒ष्ण॒वम् । त्रि॒क॒पा॒लमिति॑ त्रि - क॒पा॒ल॒म् । वा॒म॒नः । दक्षि॑णा ॥  \newline



\textbf{Jatai Paata} \newline

1. आ॒ग्ने॒य म॒ष्टाक॑पाल म॒ष्टाक॑पाल माग्ने॒य मा᳚ग्ने॒य म॒ष्टाक॑पालम् । \newline
2. अ॒ष्टाक॑पाल॒म् निर् णिर॒ष्टाक॑पाल म॒ष्टाक॑पाल॒म् निः । \newline
3. अ॒ष्टाक॑पाल॒मित्य॒ष्टा - क॒पा॒ल॒म् । \newline
4. निर् व॑पति वपति॒ निर् णिर् व॑पति । \newline
5. व॒प॒ति॒ हिर॑ण्यꣳ॒॒ हिर॑ण्यं ॅवपति वपति॒ हिर॑ण्यम् । \newline
6. हिर॑ण्य॒म् दक्षि॑णा॒ दक्षि॑णा॒ हिर॑ण्यꣳ॒॒ हिर॑ण्य॒म् दक्षि॑णा । \newline
7. दक्षि॑णा सारस्व॒तꣳ सा॑रस्व॒तम् दक्षि॑णा॒ दक्षि॑णा सारस्व॒तम् । \newline
8. सा॒र॒स्व॒तम् च॒रुम् च॒रुꣳ सा॑रस्व॒तꣳ सा॑रस्व॒तम् च॒रुम् । \newline
9. च॒रुं ॅव॑थ्सत॒री व॑थ्सत॒री च॒रुम् च॒रुं ॅव॑थ्सत॒री । \newline
10. व॒थ्स॒त॒री दक्षि॑णा॒ दक्षि॑णा वथ्सत॒री व॑थ्सत॒री दक्षि॑णा । \newline
11. दक्षि॑णा सावि॒त्रꣳ सा॑वि॒त्रम् दक्षि॑णा॒ दक्षि॑णा सावि॒त्रम् । \newline
12. सा॒वि॒त्रम् द्वाद॑शकपाल॒म् द्वाद॑शकपालꣳ सावि॒त्रꣳ सा॑वि॒त्रम् द्वाद॑शकपालम् । \newline
13. द्वाद॑शकपाल मुपद्ध्व॒स्त उ॑पद्ध्व॒स्तो द्वाद॑शकपाल॒म् द्वाद॑शकपाल मुपद्ध्व॒स्तः । \newline
14. द्वाद॑शकपाल॒मिति॒ द्वाद॑श - क॒पा॒ल॒म् । \newline
15. उ॒प॒द्ध्व॒स्तो दक्षि॑णा॒ दक्षि॑णोपद्ध्व॒स्त उ॑पद्ध्व॒स्तो दक्षि॑णा । \newline
16. उ॒प॒द्ध्व॒स्त इत्यु॑प - ध्व॒स्तः । \newline
17. दक्षि॑णा पौ॒ष्णम् पौ॒ष्णम् दक्षि॑णा॒ दक्षि॑णा पौ॒ष्णम् । \newline
18. पौ॒ष्णम् च॒रुम् च॒रुम् पौ॒ष्णम् पौ॒ष्णम् च॒रुम् । \newline
19. च॒रुꣳ श्या॒मः श्या॒म श्च॒रुम् च॒रुꣳ श्या॒मः । \newline
20. श्या॒मो दक्षि॑णा॒ दक्षि॑णा श्या॒मः श्या॒मो दक्षि॑णा । \newline
21. दक्षि॑णा बार्.हस्प॒त्यम् बा॑र्.हस्प॒त्यम् दक्षि॑णा॒ दक्षि॑णा बार्.हस्प॒त्यम् । \newline
22. बा॒र्॒.ह॒स्प॒त्यम् च॒रुम् च॒रुम् बा॑र्.हस्प॒त्यम् बा॑र्.हस्प॒त्यम् च॒रुम् । \newline
23. च॒रुꣳ शि॑तिपृ॒ष्ठः शि॑तिपृ॒ष्ठ श्च॒रुम् च॒रुꣳ शि॑तिपृ॒ष्ठः । \newline
24. शि॒ति॒पृ॒ष्ठो दक्षि॑णा॒ दक्षि॑णा शितिपृ॒ष्ठः शि॑तिपृ॒ष्ठो दक्षि॑णा । \newline
25. शि॒ति॒पृ॒ष्ठ इति॑ शिति - पृ॒ष्ठः । \newline
26. दक्षि॑ णै॒न्द्र मै॒न्द्रम् दक्षि॑णा॒ दक्षि॑ णै॒न्द्रम् । \newline
27. ऐ॒न्द्र मेका॑दशकपाल॒ मेका॑दशकपाल मै॒न्द्र मै॒न्द्र मेका॑दशकपालम् । \newline
28. एका॑दशकपाल मृष॒भ ऋ॑ष॒भ एका॑दशकपाल॒ मेका॑दशकपाल मृष॒भः । \newline
29. एका॑दशकपाल॒मित्येका॑दश - क॒पा॒ल॒म् । \newline
30. ऋ॒ष॒भो दक्षि॑णा॒ दक्षि॑णर्.ष॒भ ऋ॑ष॒भो दक्षि॑णा । \newline
31. दक्षि॑णा वारु॒णं ॅवा॑रु॒णम् दक्षि॑णा॒ दक्षि॑णा वारु॒णम् । \newline
32. वा॒रु॒णम् दश॑कपाल॒म् दश॑कपालं ॅवारु॒णं ॅवा॑रु॒णम् दश॑कपालम् । \newline
33. दश॑कपालम् म॒हानि॑रष्टो म॒हानि॑रष्टो॒ दश॑कपाल॒म् दश॑कपालम् म॒हानि॑रष्टः । \newline
34. दश॑कपाल॒मिति॒ दश॑ - क॒पा॒ल॒म् । \newline
35. म॒हानि॑रष्टो॒ दक्षि॑णा॒ दक्षि॑णा म॒हानि॑रष्टो म॒हानि॑रष्टो॒ दक्षि॑णा । \newline
36. म॒हानि॑रष्ट॒ इति॑ म॒हा - नि॒र॒ष्टः॒ । \newline
37. दक्षि॑णा सौ॒म्यꣳ सौ॒म्यम् दक्षि॑णा॒ दक्षि॑णा सौ॒म्यम् । \newline
38. सौ॒म्यम् च॒रुम् च॒रुꣳ सौ॒म्यꣳ सौ॒म्यम् च॒रुम् । \newline
39. च॒रुम् ब॒भ्रुर् ब॒भ्रु श्च॒रुम् च॒रुम् ब॒भ्रुः । \newline
40. ब॒भ्रुर् दक्षि॑णा॒ दक्षि॑णा ब॒भ्रुर् ब॒भ्रुर् दक्षि॑णा । \newline
41. दक्षि॑णा त्वा॒ष्ट्रम् त्वा॒ष्ट्रम् दक्षि॑णा॒ दक्षि॑णा त्वा॒ष्ट्रम् । \newline
42. त्वा॒ष्ट्र म॒ष्टाक॑पाल म॒ष्टाक॑पालम् त्वा॒ष्ट्रम् त्वा॒ष्ट्र म॒ष्टाक॑पालम् । \newline
43. अ॒ष्टाक॑पालꣳ शु॒ण्ठः शु॒ण्ठो᳚ ऽष्टाक॑पाल म॒ष्टाक॑पालꣳ शु॒ण्ठः । \newline
44. अ॒ष्टाक॑पाल॒मित्य॒ष्टा - क॒पा॒ल॒म् । \newline
45. शु॒ण्ठो दक्षि॑णा॒ दक्षि॑णा शु॒ण्ठः शु॒ण्ठो दक्षि॑णा । \newline
46. दक्षि॑णा वैष्ण॒वं ॅवै᳚ष्ण॒वम् दक्षि॑णा॒ दक्षि॑णा वैष्ण॒वम् । \newline
47. वै॒ष्ण॒वम् त्रि॑कपा॒लम् त्रि॑कपा॒लं ॅवै᳚ष्ण॒वं ॅवै᳚ष्ण॒वम् त्रि॑कपा॒लम् । \newline
48. त्रि॒क॒पा॒लं ॅवा॑म॒नो वा॑म॒न स्त्रि॑कपा॒लम् त्रि॑कपा॒लं ॅवा॑म॒नः । \newline
49. त्रि॒क॒पा॒लमिति॑ त्रि - क॒पा॒लम् । \newline
50. वा॒म॒नो दक्षि॑णा॒ दक्षि॑णा वाम॒नो वा॑म॒नो दक्षि॑णा । \newline
51. दक्षि॒णेति॒ दक्षि॑णा । \newline

\textbf{Ghana Paata } \newline

1. आ॒ग्ने॒य म॒ष्टाक॑पाल म॒ष्टाक॑पाल माग्ने॒य मा᳚ग्ने॒य म॒ष्टाक॑पाल॒न्निर् णिर॒ष्टाक॑पाल माग्ने॒य मा᳚ग्ने॒य म॒ष्टाक॑पाल॒न्निः । \newline
2. अ॒ष्टाक॑पाल॒न्निर् णिर॒ष्टाक॑पाल म॒ष्टाक॑पाल॒न्निर् व॑पति वपति॒ निर॒ष्टाक॑पाल म॒ष्टाक॑पाल॒न्निर् व॑पति । \newline
3. अ॒ष्टाक॑पाल॒मित्य॒ष्टा - क॒पा॒ल॒म् । \newline
4. निर् व॑पति वपति॒ निर् णिर् व॑पति॒ हिर॑ण्यꣳ॒॒ हिर॑ण्यं ॅवपति॒ निर् णिर् व॑पति॒ हिर॑ण्यम् । \newline
5. व॒प॒ति॒ हिर॑ण्यꣳ॒॒ हिर॑ण्यं ॅवपति वपति॒ हिर॑ण्य॒म् दक्षि॑णा॒ दक्षि॑णा॒ हिर॑ण्यं ॅवपति वपति॒ हिर॑ण्य॒म् दक्षि॑णा । \newline
6. हिर॑ण्य॒म् दक्षि॑णा॒ दक्षि॑णा॒ हिर॑ण्यꣳ॒॒ हिर॑ण्य॒म् दक्षि॑णा सारस्व॒तꣳ सा॑रस्व॒तम् दक्षि॑णा॒ हिर॑ण्यꣳ॒॒ हिर॑ण्य॒म् दक्षि॑णा सारस्व॒तम् । \newline
7. दक्षि॑णा सारस्व॒तꣳ सा॑रस्व॒तम् दक्षि॑णा॒ दक्षि॑णा सारस्व॒तम् च॒रुम् च॒रुꣳ सा॑रस्व॒तम् दक्षि॑णा॒ दक्षि॑णा सारस्व॒तम् च॒रुम् । \newline
8. सा॒र॒स्व॒तम् च॒रुम् च॒रुꣳ सा॑रस्व॒तꣳ सा॑रस्व॒तम् च॒रुं ॅव॑थ्सत॒री व॑थ्सत॒री च॒रुꣳ सा॑रस्व॒तꣳ सा॑रस्व॒तम् च॒रुं ॅव॑थ्सत॒री । \newline
9. च॒रुं ॅव॑थ्सत॒री व॑थ्सत॒री च॒रुम् च॒रुं ॅव॑थ्सत॒री दक्षि॑णा॒ दक्षि॑णा वथ्सत॒री च॒रुम् च॒रुं ॅव॑थ्सत॒री दक्षि॑णा । \newline
10. व॒थ्स॒त॒री दक्षि॑णा॒ दक्षि॑णा वथ्सत॒री व॑थ्सत॒री दक्षि॑णा सावि॒त्रꣳ सा॑वि॒त्रम् दक्षि॑णा वथ्सत॒री व॑थ्सत॒री दक्षि॑णा सावि॒त्रम् । \newline
11. दक्षि॑णा सावि॒त्रꣳ सा॑वि॒त्रम् दक्षि॑णा॒ दक्षि॑णा सावि॒त्रम् द्वाद॑शकपाल॒म् द्वाद॑शकपालꣳ सावि॒त्रम् दक्षि॑णा॒ दक्षि॑णा सावि॒त्रम् द्वाद॑शकपालम् । \newline
12. सा॒वि॒त्रम् द्वाद॑शकपाल॒म् द्वाद॑शकपालꣳ सावि॒त्रꣳ सा॑वि॒त्रम् द्वाद॑शकपाल मुपद्ध्व॒स्त उ॑पद्ध्व॒स्तो द्वाद॑शकपालꣳ सावि॒त्रꣳ सा॑वि॒त्रम् द्वाद॑शकपाल मुपद्ध्व॒स्तः । \newline
13. द्वाद॑शकपाल मुपद्ध्व॒स्त उ॑पद्ध्व॒स्तो द्वाद॑शकपाल॒म् द्वाद॑शकपाल मुपद्ध्व॒स्तो दक्षि॑णा॒ दक्षि॑णोपद्ध्व॒स्तो द्वाद॑शकपाल॒म् द्वाद॑शकपाल मुपद्ध्व॒स्तो दक्षि॑णा । \newline
14. द्वाद॑शकपाल॒मिति॒ द्वाद॑श - क॒पा॒ल॒म् । \newline
15. उ॒प॒द्ध्व॒स्तो दक्षि॑णा॒ दक्षि॑णोपद्ध्व॒स्त उ॑पद्ध्व॒स्तो दक्षि॑णा पौ॒ष्णम् पौ॒ष्णम् दक्षि॑णोपद्ध्व॒स्त उ॑पद्ध्व॒स्तो दक्षि॑णा पौ॒ष्णम् । \newline
16. उ॒प॒द्ध्व॒स्त इत्यु॑प - ध्व॒स्तः । \newline
17. दक्षि॑णा पौ॒ष्णम् पौ॒ष्णम् दक्षि॑णा॒ दक्षि॑णा पौ॒ष्णम् च॒रुम् च॒रुम् पौ॒ष्णम् दक्षि॑णा॒ दक्षि॑णा पौ॒ष्णम् च॒रुम् । \newline
18. पौ॒ष्णम् च॒रुम् च॒रुम् पौ॒ष्णम् पौ॒ष्णम् च॒रुꣳ श्या॒मः श्या॒मश्च॒रुम् पौ॒ष्णम् पौ॒ष्णम् च॒रुꣳ श्या॒मः । \newline
19. च॒रुꣳ श्या॒मः श्या॒म श्च॒रुम् च॒रुꣳ श्या॒मो दक्षि॑णा॒ दक्षि॑णा श्या॒म श्च॒रुम् च॒रुꣳ श्या॒मो दक्षि॑णा । \newline
20. श्या॒मो दक्षि॑णा॒ दक्षि॑णा श्या॒मः श्या॒मो दक्षि॑णा बार्.हस्प॒त्यम् बा॑र्.हस्प॒त्यम् दक्षि॑णा श्या॒मः श्या॒मो दक्षि॑णा बार्.हस्प॒त्यम् । \newline
21. दक्षि॑णा बार्.हस्प॒त्यम् बा॑र्.हस्प॒त्यम् दक्षि॑णा॒ दक्षि॑णा बार्.हस्प॒त्यम् च॒रुम् च॒रुम् बा॑र्.हस्प॒त्यम् दक्षि॑णा॒ दक्षि॑णा बार्.हस्प॒त्यम् च॒रुम् । \newline
22. बा॒र्॒.ह॒स्प॒त्यम् च॒रुम् च॒रुम् बा॑र्.हस्प॒त्यम् बा॑र्.हस्प॒त्यम् च॒रुꣳ शि॑तिपृ॒ष्ठः शि॑तिपृ॒ष्ठ श्च॒रुम् बा॑र्.हस्प॒त्यम् बा॑र्.हस्प॒त्यम् च॒रुꣳ शि॑तिपृ॒ष्ठः । \newline
23. च॒रुꣳ शि॑तिपृ॒ष्ठः शि॑तिपृ॒ष्ठ श्च॒रुम् च॒रुꣳ शि॑तिपृ॒ष्ठो दक्षि॑णा॒ दक्षि॑णा शितिपृ॒ष्ठ श्च॒रुम् च॒रुꣳ शि॑तिपृ॒ष्ठो दक्षि॑णा । \newline
24. शि॒ति॒पृ॒ष्ठो दक्षि॑णा॒ दक्षि॑णा शितिपृ॒ष्ठः शि॑तिपृ॒ष्ठो दक्षि॑णै॒न्द्र मै॒न्द्रम् दक्षि॑णा शितिपृ॒ष्ठः शि॑तिपृ॒ष्ठो दक्षि॑णै॒न्द्रम् । \newline
25. शि॒ति॒पृ॒ष्ठ इति॑ शिति - पृ॒ष्ठः । \newline
26. दक्षि॑णै॒न्द्र मै॒न्द्रम् दक्षि॑णा॒ दक्षि॑णै॒न्द्र मेका॑दशकपाल॒ मेका॑दशकपाल मै॒न्द्रम् दक्षि॑णा॒ दक्षि॑णै॒न्द्र मेका॑दशकपालम् । \newline
27. ऐ॒न्द्र मेका॑दशकपाल॒ मेका॑दशकपाल मै॒न्द्र मै॒न्द्र मेका॑दशकपाल मृष॒भ ऋ॑ष॒भ एका॑दशकपाल मै॒न्द्र मै॒न्द्र मेका॑दशकपाल मृष॒भः । \newline
28. एका॑दशकपाल मृष॒भ ऋ॑ष॒भ एका॑दशकपाल॒ मेका॑दशकपाल मृष॒भो दक्षि॑णा॒ दक्षि॑णर्.ष॒भ एका॑दशकपाल॒ मेका॑दशकपाल मृष॒भो दक्षि॑णा । \newline
29. एका॑दशकपाल॒मित्येका॑दश - क॒पा॒ल॒म् । \newline
30. ऋ॒ष॒भो दक्षि॑णा॒ दक्षि॑णर्.ष॒भ ऋ॑ष॒भो दक्षि॑णा वारु॒णं ॅवा॑रु॒णम् दक्षि॑णर्.ष॒भ ऋ॑ष॒भो दक्षि॑णा वारु॒णम् । \newline
31. दक्षि॑णा वारु॒णं ॅवा॑रु॒णम् दक्षि॑णा॒ दक्षि॑णा वारु॒णम् दश॑कपाल॒म् दश॑कपालं ॅवारु॒णम् दक्षि॑णा॒ दक्षि॑णा वारु॒णम् दश॑कपालम् । \newline
32. वा॒रु॒णम् दश॑कपाल॒म् दश॑कपालं ॅवारु॒णं ॅवा॑रु॒णम् दश॑कपालम् म॒हानि॑रष्टो म॒हानि॑रष्टो॒ दश॑कपालं ॅवारु॒णं ॅवा॑रु॒णम् दश॑कपालम् म॒हानि॑रष्टः । \newline
33. दश॑कपालम् म॒हानि॑रष्टो म॒हानि॑रष्टो॒ दश॑कपाल॒म् दश॑कपालम् म॒हानि॑रष्टो॒ दक्षि॑णा॒ दक्षि॑णा म॒हानि॑रष्टो॒ दश॑कपाल॒म् दश॑कपालम् म॒हानि॑रष्टो॒ दक्षि॑णा । \newline
34. दश॑कपाल॒मिति॒ दश॑ - क॒पा॒ल॒म् । \newline
35. म॒हानि॑रष्टो॒ दक्षि॑णा॒ दक्षि॑णा म॒हानि॑रष्टो म॒हानि॑रष्टो॒ दक्षि॑णा सौ॒म्यꣳ सौ॒म्यम् दक्षि॑णा म॒हानि॑रष्टो म॒हानि॑रष्टो॒ दक्षि॑णा सौ॒म्यम् । \newline
36. म॒हानि॑रष्ट॒ इति॑ म॒हा - नि॒र॒ष्टः॒ । \newline
37. दक्षि॑णा सौ॒म्यꣳ सौ॒म्यम् दक्षि॑णा॒ दक्षि॑णा सौ॒म्यम् च॒रुम् च॒रुꣳ सौ॒म्यम् दक्षि॑णा॒ दक्षि॑णा सौ॒म्यम् च॒रुम् । \newline
38. सौ॒म्यम् च॒रुम् च॒रुꣳ सौ॒म्यꣳ सौ॒म्यम् च॒रुम् ब॒भ्रुर् ब॒भ्रु श्च॒रुꣳ सौ॒म्यꣳ सौ॒म्यम् च॒रुम् ब॒भ्रुः । \newline
39. च॒रुम् ब॒भ्रुर् ब॒भ्रु श्च॒रुम् च॒रुम् ब॒भ्रुर् दक्षि॑णा॒ दक्षि॑णा ब॒भ्रु श्च॒रुम् च॒रुम् ब॒भ्रुर् दक्षि॑णा । \newline
40. ब॒भ्रुर् दक्षि॑णा॒ दक्षि॑णा ब॒भ्रुर् ब॒भ्रुर् दक्षि॑णा त्वा॒ष्ट्रम् त्वा॒ष्ट्रम् दक्षि॑णा ब॒भ्रुर् ब॒भ्रुर् दक्षि॑णा त्वा॒ष्ट्रम् । \newline
41. दक्षि॑णा त्वा॒ष्ट्रम् त्वा॒ष्ट्रम् दक्षि॑णा॒ दक्षि॑णा त्वा॒ष्ट्र म॒ष्टाक॑पाल म॒ष्टाक॑पालम् त्वा॒ष्ट्रम् दक्षि॑णा॒ दक्षि॑णा त्वा॒ष्ट्र म॒ष्टाक॑पालम् । \newline
42. त्वा॒ष्ट्र म॒ष्टाक॑पाल म॒ष्टाक॑पालम् त्वा॒ष्ट्रम् त्वा॒ष्ट्र म॒ष्टाक॑पालꣳ शु॒ण्ठः शु॒ण्ठो᳚ ऽष्टाक॑पालम् त्वा॒ष्ट्रम् त्वा॒ष्ट्र म॒ष्टाक॑पालꣳ शु॒ण्ठः । \newline
43. अ॒ष्टाक॑पालꣳ शु॒ण्ठः शु॒ण्ठो᳚ ऽष्टाक॑पाल म॒ष्टाक॑पालꣳ शु॒ण्ठो दक्षि॑णा॒ दक्षि॑णा शु॒ण्ठो᳚ ऽष्टाक॑पाल म॒ष्टाक॑पालꣳ शु॒ण्ठो दक्षि॑णा । \newline
44. अ॒ष्टाक॑पाल॒मित्य॒ष्टा - क॒पा॒ल॒म् । \newline
45. शु॒ण्ठो दक्षि॑णा॒ दक्षि॑णा शु॒ण्ठः शु॒ण्ठो दक्षि॑णा वैष्ण॒वं ॅवै᳚ष्ण॒वम् दक्षि॑णा शु॒ण्ठः शु॒ण्ठो दक्षि॑णा वैष्ण॒वम् । \newline
46. दक्षि॑णा वैष्ण॒वं ॅवै᳚ष्ण॒वम् दक्षि॑णा॒ दक्षि॑णा वैष्ण॒वम् त्रि॑कपा॒लम् त्रि॑कपा॒लं ॅवै᳚ष्ण॒वम् दक्षि॑णा॒ दक्षि॑णा वैष्ण॒वम् त्रि॑कपा॒लम् । \newline
47. वै॒ष्ण॒वम् त्रि॑कपा॒लम् त्रि॑कपा॒लं ॅवै᳚ष्ण॒वं ॅवै᳚ष्ण॒वम् त्रि॑कपा॒लं ॅवा॑म॒नो वा॑म॒न स्त्रि॑कपा॒लं ॅवै᳚ष्ण॒वं ॅवै᳚ष्ण॒वम् त्रि॑कपा॒लं ॅवा॑म॒नः । \newline
48. त्रि॒क॒पा॒लं ॅवा॑म॒नो वा॑म॒न स्त्रि॑कपा॒लम् त्रि॑कपा॒लं ॅवा॑म॒नो दक्षि॑णा॒ दक्षि॑णा वाम॒न स्त्रि॑कपा॒लम् त्रि॑कपा॒लं ॅवा॑म॒नो दक्षि॑णा । \newline
49. त्रि॒क॒पा॒लमिति॑ त्रि - क॒पा॒लम् । \newline
50. वा॒म॒नो दक्षि॑णा॒ दक्षि॑णा वाम॒नो वा॑म॒नो दक्षि॑णा । \newline
51. दक्षि॒णेति॒ दक्षि॑णा । \newline
\pagebreak
\markright{ TS 1.8.18.1  \hfill https://www.vedavms.in \hfill}
\addcontentsline{toc}{section}{ TS 1.8.18.1 }
\section*{ TS 1.8.18.1 }

\textbf{TS 1.8.18.1 } \newline
\textbf{Samhita Paata} \newline

स॒द्यो दी᳚क्षयन्ति स॒द्यः सोमं॑ क्रीणन्ति पुण्डरिस्र॒जां प्र य॑च्छति द॒शभि॑र् वथ्सत॒रैः सोमं॑ क्रीणाति दश॒पेयो॑ भवति श॒तं ब्रा᳚ह्म॒णाः पि॑बन्ति सप्तद॒शꣳ स्तो॒त्रं भ॑वति प्राका॒शाव॑द्ध्व॒र्यवे॑ ददाति॒ स्रज॑-मुद्गा॒त्रे रु॒क्मꣳ होत्रेऽश्वं॑ प्रस्तोतृप्रतिह॒र्तृभ्यां॒ द्वाद॑श पष्ठौ॒हीर् ब्र॒ह्मणे॑ व॒शां मै᳚त्रावरु॒णाय॑र्.ष॒भं ब्रा᳚ह्मणाच्छꣳ॒॒सिने॒ वास॑सी नेष्टापो॒तृभ्याꣳ॒॒ स्थूरि॑ यवाचि॒त-म॑च्छावा॒काया॑न॒ड्वाह॑-म॒ग्नीधे॑ भार्ग॒वो होता॑ भवति श्राय॒न्तीयं॑ ब्रह्मसा॒मं भ॑वति वारव॒न्तीय॑ ( ) मग्निष्टोमसा॒मꣳ सा॑रस्व॒ती-र॒पो गृ॑ह्णाति ॥ \newline

\textbf{Pada Paata} \newline

स॒द्यः । दी॒क्ष॒य॒न्ति॒ । स॒द्यः । सोम᳚म् । क्री॒ण॒न्ति॒ । पु॒ण्ड॒रि॒स्र॒जाम् । प्रेति॑ । य॒च्छ॒ति॒ । द॒शभि॒रिति॑ द॒श-भिः॒ । व॒थ्स॒त॒रैः । सोम᳚म् । क्री॒णा॒ति॒ । द॒श॒पेय॒ इति॑ दश-पेयः॑ । भ॒व॒ति॒ । श॒तम् । ब्रा॒ह्म॒णाः । पि॒ब॒न्ति॒ । स॒प्त॒द॒शमिति॑ सप्त - द॒शम् । स्तो॒त्रम् । भ॒व॒ति॒ । प्रा॒का॒शौ । अ॒द्ध्व॒र्यवे᳚ । द॒दा॒ति॒ । स्रज᳚म् । उ॒द्गा॒त्र इत्यु॑त्-गा॒त्रे । रु॒क्मम् । होत्रे᳚ । अश्व᳚म् । प्र॒स्तो॒तृ॒प्र॒ति॒ह॒र्तृभ्या॒मिति॑ प्रस्तोतृप्रतिह॒र्तृ-भ्यां॒ । द्वाद॑श । प॒ष्ठौ॒हीः । ब्र॒ह्मणे᳚ । व॒शाम् । मै॒त्रा॒व॒रु॒णायेति॑ मैत्रा - व॒रु॒णाय॑ । ऋ॒ष॒भम् । ब्रा॒ह्म॒णा॒च्छꣳ॒॒सिने᳚ । वास॑सी॒ इति॑ । ने॒ष्टा॒पो॒तृभ्या॒मिति॑ नेष्टापो॒तृ - भ्या॒म् । स्थूरि॑ । य॒वा॒चि॒तमिति॑ यव - आ॒चि॒तम् । अ॒च्छा॒वा॒काय॑ । अ॒न॒ड्वाह᳚म् । अ॒ग्नीध॒ इत्य॑ग्नि - इधे᳚ । भा॒र्ग॒वः । होता᳚ । भ॒व॒ति॒ । श्रा॒य॒न्तीय᳚म् । ब्र॒ह्म॒सा॒ममिति॑ ब्रह्म - सा॒मम् । भ॒व॒ति॒ । वा॒र॒व॒न्तीय॒मिति॑ वार - व॒न्तीय᳚म् ( ) । अ॒ग्नि॒ष्टो॒म॒सा॒ममित्य॑ग्निष्टोम - सा॒मम् । सा॒र॒स्व॒तीः । अ॒पः । गृ॒ह्णा॒ति॒ ॥  \newline



\textbf{Jatai Paata} \newline

1. स॒द्यो दी᳚क्षयन्ति दीक्षयन्ति स॒द्यः स॒द्यो दी᳚क्षयन्ति । \newline
2. दी॒क्ष॒य॒न्ति॒ स॒द्यः स॒द्यो दी᳚क्षयन्ति दीक्षयन्ति स॒द्यः । \newline
3. स॒द्यः सोमꣳ॒॒ सोमꣳ॑ स॒द्यः स॒द्यः सोम᳚म् । \newline
4. सोम॑म् क्रीणन्ति क्रीणन्ति॒ सोमꣳ॒॒ सोम॑म् क्रीणन्ति । \newline
5. क्री॒ण॒न्ति॒ पु॒ण्ड॒रि॒स्र॒जाम् पु॑ण्डरिस्र॒जाम् क्री॑णन्ति क्रीणन्ति पुण्डरिस्र॒जाम् । \newline
6. पु॒ण्ड॒रि॒स्र॒जाम् प्र प्र पु॑ण्डरिस्र॒जाम् पु॑ण्डरिस्र॒जाम् प्र । \newline
7. प्र य॑च्छति यच्छति॒ प्र प्र य॑च्छति । \newline
8. य॒च्छ॒ति॒ द॒शभि॑र् द॒शभि॑र् यच्छति यच्छति द॒शभिः॑ । \newline
9. द॒शभि॑र् वथ्सत॒रैर् व॑थ्सत॒रैर् द॒शभि॑र् द॒शभि॑र् वथ्सत॒रैः । \newline
10. द॒शभि॒रिति॑ द॒श - भिः॒ । \newline
11. व॒थ्स॒त॒रैः सोमꣳ॒॒ सोमं॑ ॅवथ्सत॒रैर् व॑थ्सत॒रैः सोम᳚म् । \newline
12. सोम॑म् क्रीणाति क्रीणाति॒ सोमꣳ॒॒ सोम॑म् क्रीणाति । \newline
13. क्री॒णा॒ति॒ द॒श॒पेयो॑ दश॒पेयः॑ क्रीणाति क्रीणाति दश॒पेयः॑ । \newline
14. द॒श॒पेयो॑ भवति भवति दश॒पेयो॑ दश॒पेयो॑ भवति । \newline
15. द॒श॒पेय॒ इति॑ दश - पेयः॑ । \newline
16. भ॒व॒ति॒ श॒तꣳ श॒तम् भ॑वति भवति श॒तम् । \newline
17. श॒तम् ब्रा᳚ह्म॒णा ब्रा᳚ह्म॒णाः श॒तꣳ श॒तम् ब्रा᳚ह्म॒णाः । \newline
18. ब्रा॒ह्म॒णाः पि॑बन्ति पिबन्ति ब्राह्म॒णा ब्रा᳚ह्म॒णाः पि॑बन्ति । \newline
19. पि॒ब॒न्ति॒ स॒प्त॒द॒शꣳ स॑प्तद॒शम् पि॑बन्ति पिबन्ति सप्तद॒शम् । \newline
20. स॒प्त॒द॒शꣳ स्तो॒त्रꣳ स्तो॒त्रꣳ स॑प्तद॒शꣳ स॑प्तद॒शꣳ स्तो॒त्रम् । \newline
21. स॒प्त॒द॒शमिति॑ सप्त - द॒शम् । \newline
22. स्तो॒त्रम् भ॑वति भवति स्तो॒त्रꣳ स्तो॒त्रम् भ॑वति । \newline
23. भ॒व॒ति॒ प्रा॒का॒शौ प्रा॑का॒शौ भ॑वति भवति प्राका॒शौ । \newline
24. प्रा॒का॒शा व॑द्ध्व॒र्यवे᳚ ऽद्ध्व॒र्यवे᳚ प्राका॒शौ प्रा॑का॒शा व॑द्ध्व॒र्यवे᳚ । \newline
25. अ॒द्ध्व॒र्यवे॑ ददाति ददात्यद्ध्व॒र्यवे᳚ ऽद्ध्व॒र्यवे॑ ददाति । \newline
26. द॒दा॒ति॒ स्रजꣳ॒॒ स्रज॑म् ददाति ददाति॒ स्रज᳚म् । \newline
27. स्रज॑ मुद्‍गा॒त्र उ॑द्‍गा॒त्रे स्रजꣳ॒॒ स्रज॑ मुद्‍गा॒त्रे । \newline
28. उ॒द्‍गा॒त्रे रु॒क्मꣳ रु॒क्म मु॑द्‍गा॒त्र उ॑द्‍गा॒त्रे रु॒क्मम् । \newline
29. उ॒द्‍गा॒त्र इत्यु॑त् - गा॒त्रे । \newline
30. रु॒क्मꣳ होत्रे॒ होत्रे॑ रु॒क्मꣳ रु॒क्मꣳ होत्रे᳚ । \newline
31. होत्रे ऽश्व॒ मश्वꣳ॒॒ होत्रे॒ होत्रे ऽश्व᳚म् । \newline
32. अश्व॑म् प्रस्तोतृप्रतिह॒र्तृभ्यां᳚ प्रस्तोतृप्रतिह॒र्तृभ्या॒ मश्व॒ मश्व॑म् प्रस्तोतृप्रतिह॒र्तृभ्यां᳚ । \newline
33. प्र॒स्तो॒तृ॒प्र॒ति॒ह॒र्तृभ्यां॒ द्वाद॑श॒ द्वाद॑श प्रस्तोतृप्रतिह॒र्तृभ्यां᳚ प्रस्तोतृप्रतिह॒र्तृभ्यां॒ द्वाद॑श । \newline
34. प्र॒स्तो॒तृ॒प्र॒ति॒ह॒र्तृभ्या॒मिति॑ प्रस्तोतृप्रतिह॒र्तृ - भ्यां॒ । \newline
35. द्वाद॑श पष्ठौ॒हीः प॑ष्ठौ॒हीर् द्वाद॑श॒ द्वाद॑श पष्ठौ॒हीः । \newline
36. प॒ष्ठौ॒हीर् ब्र॒ह्मणे᳚ ब्र॒ह्मणे॑ पष्ठौ॒हीः प॑ष्ठौ॒हीर् ब्र॒ह्मणे᳚ । \newline
37. ब्र॒ह्मणे॑ व॒शां ॅव॒शाम् ब्र॒ह्मणे᳚ ब्र॒ह्मणे॑ व॒शाम् । \newline
38. व॒शाम् मै᳚त्रावरु॒णाय॑ मैत्रावरु॒णाय॑ व॒शां ॅव॒शाम् मै᳚त्रावरु॒णाय॑ । \newline
39. मै॒त्रा॒व॒रु॒णाय॑ र्.ष॒भ मृ॑ष॒भम् मै᳚त्रावरु॒णाय॑ मैत्रावरु॒णाय॑ र्.ष॒भम् । \newline
40. मै॒त्रा॒व॒रु॒णायेति॑ मैत्रा - व॒रु॒णाय॑ । \newline
41. ऋ॒ष॒भम् ब्रा᳚ह्मणाच्छꣳ॒॒सिने᳚ ब्राह्मणाच्छꣳ॒॒सिन॑ ऋष॒भ मृ॑ष॒भम् ब्रा᳚ह्मणाच्छꣳ॒॒सिने᳚ । \newline
42. ब्रा॒ह्म॒णा॒च्छꣳ॒॒सिने॒ वास॑सी॒ वास॑सी ब्राह्मणाच्छꣳ॒॒सिने᳚ ब्राह्मणाच्छꣳ॒॒सिने॒ वास॑सी । \newline
43. वास॑सी नेष्टापो॒तृभ्या᳚म् नेष्टापो॒तृभ्यां॒ ॅवास॑सी॒ वास॑सी नेष्टापो॒तृभ्या᳚म् । \newline
44. वास॑सी॒ इति॒ वास॑सी । \newline
45. ने॒ष्टा॒पो॒तृभ्याꣳ॒॒ स्थूरि॒ स्थूरि॑ नेष्टापो॒तृभ्या᳚म् नेष्टापो॒तृभ्याꣳ॒॒ स्थूरि॑ । \newline
46. ने॒ष्टा॒पो॒तृभ्या॒मिति॑ नेष्टापो॒तृ - भ्या॒म् । \newline
47. स्थूरि॑ यवाचि॒तं ॅय॑वाचि॒तꣳ स्थूरि॒ स्थूरि॑ यवाचि॒तम् । \newline
48. य॒वा॒चि॒त म॑च्छावा॒काया᳚ च्छावा॒काय॑ यवाचि॒तं ॅय॑वाचि॒त म॑च्छावा॒काय॑ । \newline
49. य॒वा॒चि॒तमिति॑ यव - आ॒चि॒तम् । \newline
50. अ॒च्छा॒वा॒काया॑ न॒ड्वाह॑ मन॒ड्वाह॑ मच्छावा॒काया᳚ च्छावा॒काया॑ न॒ड्वाह᳚म् । \newline
51. अ॒न॒ड्वाह॑ म॒ग्नीधे॒ ऽग्नीधे॑ ऽन॒ड्वाह॑ मन॒ड्वाह॑ म॒ग्नीधे᳚ । \newline
52. अ॒ग्नीधे॑ भार्ग॒वो भा᳚र्ग॒वो᳚ ऽग्नीधे॒ ऽग्नीधे॑ भार्ग॒वः । \newline
53. अ॒ग्नीध॒ इत्य॑ग्नि - इधे᳚ । \newline
54. भा॒र्ग॒वो होता॒ होता॑ भार्ग॒वो भा᳚र्ग॒वो होता᳚ । \newline
55. होता॑ भवति भवति॒ होता॒ होता॑ भवति । \newline
56. भ॒व॒ति॒ श्रा॒य॒न्तीयꣳ॑ श्राय॒न्तीय॑म् भवति भवति श्राय॒न्तीय᳚म् । \newline
57. श्रा॒य॒न्तीय॑म् ब्रह्मसा॒मम् ब्र॑ह्मसा॒मꣳ श्रा॑य॒न्तीयꣳ॑ श्राय॒न्तीय॑म् ब्रह्मसा॒मम् । \newline
58. ब्र॒ह्म॒सा॒मम् भ॑वति भवति ब्रह्मसा॒मम् ब्र॑ह्मसा॒मम् भ॑वति । \newline
59. ब्र॒ह्म॒सा॒ममिति॑ ब्रह्म - सा॒मम् । \newline
60. भ॒व॒ति॒ वा॒र॒व॒न्तीयं॑ ॅवारव॒न्तीय॑म् भवति भवति वारव॒न्तीय᳚म् । \newline
61. वा॒र॒व॒न्तीय॑ मग्निष्टोमसा॒म म॑ग्निष्टोमसा॒मं ॅवा॑रव॒न्तीयं॑ ॅवारव॒न्तीय॑ मग्निष्टोमसा॒मम् । \newline
62. वा॒र॒व॒न्तीय॒मिति॑ वार - व॒न्तीय᳚म् । \newline
63. अ॒ग्नि॒ष्टो॒म॒सा॒मꣳ सा॑रस्व॒तीः सा॑रस्व॒ती र॑ग्निष्टोमसा॒म म॑ग्निष्टोमसा॒मꣳ सा॑रस्व॒तीः । \newline
64. अ॒ग्नि॒ष्टो॒म॒सा॒ममित्य॑ग्निष्टोम - सा॒मम् । \newline
65. सा॒र॒स्व॒ती र॒पो॑ ऽपः सा॑रस्व॒तीः सा॑रस्व॒ती र॒पः । \newline
66. अ॒पो गृ॑ह्णाति गृह्णात्य॒पो॑ ऽपो गृ॑ह्णाति । \newline
67. गृ॒ह्णा॒तीति॑ गृह्णाति । \newline

\textbf{Ghana Paata } \newline

1. स॒द्यो दी᳚क्षयन्ति दीक्षयन्ति स॒द्यः स॒द्यो दी᳚क्षयन्ति स॒द्यः स॒द्यो दी᳚क्षयन्ति स॒द्यः स॒द्यो दी᳚क्षयन्ति स॒द्यः । \newline
2. दी॒क्ष॒य॒न्ति॒ स॒द्यः स॒द्यो दी᳚क्षयन्ति दीक्षयन्ति स॒द्यः सोमꣳ॒॒ सोमꣳ॑ स॒द्यो दी᳚क्षयन्ति दीक्षयन्ति स॒द्यः सोम᳚म् । \newline
3. स॒द्यः सोमꣳ॒॒ सोमꣳ॑ स॒द्यः स॒द्यः सोम॑म् क्रीणन्ति क्रीणन्ति॒ सोमꣳ॑ स॒द्यः स॒द्यः सोम॑म् क्रीणन्ति । \newline
4. सोम॑म् क्रीणन्ति क्रीणन्ति॒ सोमꣳ॒॒ सोम॑म् क्रीणन्ति पुण्डरिस्र॒जाम् पु॑ण्डरिस्र॒जाम् क्री॑णन्ति॒ सोमꣳ॒॒ सोम॑म् क्रीणन्ति पुण्डरिस्र॒जाम् । \newline
5. क्री॒ण॒न्ति॒ पु॒ण्ड॒रि॒स्र॒जाम् पु॑ण्डरिस्र॒जाम् क्री॑णन्ति क्रीणन्ति पुण्डरिस्र॒जाम् प्र प्र पु॑ण्डरिस्र॒जाम् क्री॑णन्ति क्रीणन्ति पुण्डरिस्र॒जाम् प्र । \newline
6. पु॒ण्ड॒रि॒स्र॒जाम् प्र प्र पु॑ण्डरिस्र॒जाम् पु॑ण्डरिस्र॒जाम् प्र य॑च्छति यच्छति॒ प्र पु॑ण्डरिस्र॒जाम् पु॑ण्डरिस्र॒जाम् प्र य॑च्छति । \newline
7. प्र य॑च्छति यच्छति॒ प्र प्र य॑च्छति द॒शभि॑र् द॒शभि॑र् यच्छति॒ प्र प्र य॑च्छति द॒शभिः॑ । \newline
8. य॒च्छ॒ति॒ द॒शभि॑र् द॒शभि॑र् यच्छति यच्छति द॒शभि॑र् वथ्सत॒रैर् व॑थ्सत॒रैर् द॒शभि॑र् यच्छति यच्छति द॒शभि॑र् वथ्सत॒रैः । \newline
9. द॒शभि॑र् वथ्सत॒रैर् व॑थ्सत॒रैर् द॒शभि॑र् द॒शभि॑र् वथ्सत॒रैः सोमꣳ॒॒ सोमं॑ ॅवथ्सत॒रैर् द॒शभि॑र् द॒शभि॑र् वथ्सत॒रैः सोम᳚म् । \newline
10. द॒शभि॒रिति॑ द॒श - भिः॒ । \newline
11. व॒थ्स॒त॒रैः सोमꣳ॒॒ सोमं॑ ॅवथ्सत॒रैर् व॑थ्सत॒रैः सोम॑म् क्रीणाति क्रीणाति॒ सोमं॑ ॅवथ्सत॒रैर् व॑थ्सत॒रैः सोम॑म् क्रीणाति । \newline
12. सोम॑म् क्रीणाति क्रीणाति॒ सोमꣳ॒॒ सोम॑म् क्रीणाति दश॒पेयो॑ दश॒पेयः॑ क्रीणाति॒ सोमꣳ॒॒ सोम॑म् क्रीणाति दश॒पेयः॑ । \newline
13. क्री॒णा॒ति॒ द॒श॒पेयो॑ दश॒पेयः॑ क्रीणाति क्रीणाति दश॒पेयो॑ भवति भवति दश॒पेयः॑ क्रीणाति क्रीणाति दश॒पेयो॑ भवति । \newline
14. द॒श॒पेयो॑ भवति भवति दश॒पेयो॑ दश॒पेयो॑ भवति श॒तꣳ श॒तम् भ॑वति दश॒पेयो॑ दश॒पेयो॑ भवति श॒तम् । \newline
15. द॒श॒पेय॒ इति॑ दश - पेयः॑ । \newline
16. भ॒व॒ति॒ श॒तꣳ श॒तम् भ॑वति भवति श॒तम् ब्रा᳚ह्म॒णा ब्रा᳚ह्म॒णाः श॒तम् भ॑वति भवति श॒तम् ब्रा᳚ह्म॒णाः । \newline
17. श॒तम् ब्रा᳚ह्म॒णा ब्रा᳚ह्म॒णाः श॒तꣳ श॒तम् ब्रा᳚ह्म॒णाः पि॑बन्ति पिबन्ति ब्राह्म॒णाः श॒तꣳ श॒तम् ब्रा᳚ह्म॒णाः पि॑बन्ति । \newline
18. ब्रा॒ह्म॒णाः पि॑बन्ति पिबन्ति ब्राह्म॒णा ब्रा᳚ह्म॒णाः पि॑बन्ति सप्तद॒शꣳ स॑प्तद॒शम् पि॑बन्ति ब्राह्म॒णा ब्रा᳚ह्म॒णाः पि॑बन्ति सप्तद॒शम् । \newline
19. पि॒ब॒न्ति॒ स॒प्त॒द॒शꣳ स॑प्तद॒शम् पि॑बन्ति पिबन्ति सप्तद॒शꣳ स्तो॒त्रꣳ स्तो॒त्रꣳ स॑प्तद॒शम् पि॑बन्ति पिबन्ति सप्तद॒शꣳ स्तो॒त्रम् । \newline
20. स॒प्त॒द॒शꣳ स्तो॒त्रꣳ स्तो॒त्रꣳ स॑प्तद॒शꣳ स॑प्तद॒शꣳ स्तो॒त्रम् भ॑वति भवति स्तो॒त्रꣳ स॑प्तद॒शꣳ स॑प्तद॒शꣳ स्तो॒त्रम् भ॑वति । \newline
21. स॒प्त॒द॒शमिति॑ सप्त - द॒शम् । \newline
22. स्तो॒त्रम् भ॑वति भवति स्तो॒त्रꣳ स्तो॒त्रम् भ॑वति प्राका॒शौ प्रा॑का॒शौ भ॑वति स्तो॒त्रꣳ स्तो॒त्रम् भ॑वति प्राका॒शौ । \newline
23. भ॒व॒ति॒ प्रा॒का॒शौ प्रा॑का॒शौ भ॑वति भवति प्राका॒शा व॑द्ध्व॒र्यवे᳚ ऽद्ध्व॒र्यवे᳚ प्राका॒शौ भ॑वति भवति प्राका॒शा व॑द्ध्व॒र्यवे᳚ । \newline
24. प्रा॒का॒शा व॑द्ध्व॒र्यवे᳚ ऽद्ध्व॒र्यवे᳚ प्राका॒शौ प्रा॑का॒शा व॑द्ध्व॒र्यवे॑ ददाति ददात्य द्ध्व॒र्यवे᳚ प्राका॒शौ प्रा॑का॒शा व॑द्ध्व॒र्यवे॑ ददाति । \newline
25. अ॒द्ध्व॒र्यवे॑ ददाति ददात्य द्ध्व॒र्यवे᳚ ऽद्ध्व॒र्यवे॑ ददाति॒ स्रजꣳ॒॒ स्रज॑म् ददात्य द्ध्व॒र्यवे᳚ ऽद्ध्व॒र्यवे॑ ददाति॒ स्रज᳚म् । \newline
26. द॒दा॒ति॒ स्रजꣳ॒॒ स्रज॑म् ददाति ददाति॒ स्रज॑ मुद्गा॒त्र उ॑द्गा॒त्रे स्रज॑म् ददाति ददाति॒ स्रज॑ मुद्गा॒त्रे । \newline
27. स्रज॑ मुद्गा॒त्र उ॑द्गा॒त्रे स्रजꣳ॒॒ स्रज॑ मुद्गा॒त्रे रु॒क्मꣳ रु॒क्म मु॑द्गा॒त्रे स्रजꣳ॒॒ स्रज॑ मुद्गा॒त्रे रु॒क्मम् । \newline
28. उ॒द्गा॒त्रे रु॒क्मꣳ रु॒क्म मु॑द्गा॒त्र उ॑द्गा॒त्रे रु॒क्मꣳ होत्रे॒ होत्रे॑ रु॒क्म मु॑द्गा॒त्र उ॑द्गा॒त्रे रु॒क्मꣳ होत्रे᳚ । \newline
29. उ॒द्गा॒त्र इत्यु॑त् - गा॒त्रे । \newline
30. रु॒क्मꣳ होत्रे॒ होत्रे॑ रु॒क्मꣳ रु॒क्मꣳ होत्रे ऽश्व॒ मश्वꣳ॒॒ होत्रे॑ रु॒क्मꣳ रु॒क्मꣳ होत्रे ऽश्व᳚म् । \newline
31. होत्रे ऽश्व॒ मश्वꣳ॒॒ होत्रे॒ होत्रे ऽश्व॑म् प्रस्तोतृप्रतिह॒र्तृभ्यां᳚ प्रस्तोतृप्रतिह॒र्तृभ्या॒ मश्वꣳ॒॒ होत्रे॒ होत्रे ऽश्व॑म् प्रस्तोतृप्रतिह॒र्तृभ्यां᳚ । \newline
32. अश्व॑म् प्रस्तोतृप्रतिह॒र्तृभ्यां᳚ प्रस्तोतृप्रतिह॒र्तृभ्या॒ मश्व॒ मश्व॑म् प्रस्तोतृप्रतिह॒र्तृभ्यां॒ द्वाद॑श॒ द्वाद॑श प्रस्तोतृप्रतिह॒र्तृभ्या॒ मश्व॒ मश्व॑म् प्रस्तोतृप्रतिह॒र्तृभ्यां॒ द्वाद॑श । \newline
33. प्र॒स्तो॒तृ॒प्र॒ति॒ह॒र्तृभ्यां॒ द्वाद॑श॒ द्वाद॑श प्रस्तोतृप्रतिह॒र्तृभ्यां᳚ प्रस्तोतृप्रतिह॒र्तृभ्यां॒ द्वाद॑श पष्ठौ॒हीः प॑ष्ठौ॒हीर् द्वाद॑श प्रस्तोतृप्रतिह॒र्तृभ्यां᳚ प्रस्तोतृप्रतिह॒र्तृभ्यां॒ द्वाद॑श पष्ठौ॒हीः । \newline
34. प्र॒स्तो॒तृ॒प्र॒ति॒ह॒र्तृभ्या॒मिति॑ प्रस्तोतृप्रतिह॒र्तृ - भ्यां॒ । \newline
35. द्वाद॑श पष्ठौ॒हीः प॑ष्ठौ॒हीर् द्वाद॑श॒ द्वाद॑श पष्ठौ॒हीर् ब्र॒ह्मणे᳚ ब्र॒ह्मणे॑ पष्ठौ॒हीर् द्वाद॑श॒ द्वाद॑श पष्ठौ॒हीर् ब्र॒ह्मणे᳚ । \newline
36. प॒ष्ठौ॒हीर् ब्र॒ह्मणे᳚ ब्र॒ह्मणे॑ पष्ठौ॒हीः प॑ष्ठौ॒हीर् ब्र॒ह्मणे॑ व॒शां ॅव॒शाम् ब्र॒ह्मणे॑ पष्ठौ॒हीः प॑ष्ठौ॒हीर् ब्र॒ह्मणे॑ व॒शाम् । \newline
37. ब्र॒ह्मणे॑ व॒शां ॅव॒शाम् ब्र॒ह्मणे᳚ ब्र॒ह्मणे॑ व॒शाम् मै᳚त्रावरु॒णाय॑ मैत्रावरु॒णाय॑ व॒शाम् ब्र॒ह्मणे᳚ ब्र॒ह्मणे॑ व॒शाम् मै᳚त्रावरु॒णाय॑ । \newline
38. व॒शाम् मै᳚त्रावरु॒णाय॑ मैत्रावरु॒णाय॑ व॒शां ॅव॒शाम् मै᳚त्रावरु॒णाय॑ र्.ष॒भ मृ॑ष॒भम् मै᳚त्रावरु॒णाय॑ व॒शां ॅव॒शाम् मै᳚त्रावरु॒णाय॑ र्.ष॒भम् । \newline
39. मै॒त्रा॒व॒रु॒णाय॑ र्.ष॒भ मृ॑ष॒भम् मै᳚त्रावरु॒णाय॑ मैत्रावरु॒णाय॑ र्.ष॒भम् ब्रा᳚ह्मणाच्छꣳ॒॒सिने᳚ ब्राह्मणाच्छꣳ॒॒सिन॑ ऋष॒भम् मै᳚त्रावरु॒णाय॑ मैत्रावरु॒णाय॑ र्.ष॒भम् ब्रा᳚ह्मणाच्छꣳ॒॒सिने᳚ । \newline
40. मै॒त्रा॒व॒रु॒णायेति॑ मैत्रा - व॒रु॒णाय॑ । \newline
41. ऋ॒ष॒भम् ब्रा᳚ह्मणाच्छꣳ॒॒सिने᳚ ब्राह्मणाच्छꣳ॒॒सिन॑ ऋष॒भ मृ॑ष॒भम् ब्रा᳚ह्मणाच्छꣳ॒॒सिने॒ वास॑सी॒ वास॑सी ब्राह्मणाच्छꣳ॒॒सिन॑ ऋष॒भ मृ॑ष॒भम् ब्रा᳚ह्मणाच्छꣳ॒॒सिने॒ वास॑सी । \newline
42. ब्रा॒ह्म॒णा॒च्छꣳ॒॒सिने॒ वास॑सी॒ वास॑सी ब्राह्मणाच्छꣳ॒॒सिने᳚ ब्राह्मणाच्छꣳ॒॒सिने॒ वास॑सी नेष्टापो॒तृभ्या᳚न् नेष्टापो॒तृभ्यां॒ ॅवास॑सी ब्राह्मणाच्छꣳ॒॒सिने᳚ ब्राह्मणाच्छꣳ॒॒सिने॒ वास॑सी नेष्टापो॒तृभ्या᳚म् । \newline
43. वास॑सी नेष्टापो॒तृभ्या᳚न् नेष्टापो॒तृभ्यां॒ ॅवास॑सी॒ वास॑सी नेष्टापो॒तृभ्याꣳ॒॒ स्थूरि॒ स्थूरि॑ नेष्टापो॒तृभ्यां॒ ॅवास॑सी॒ वास॑सी नेष्टापो॒तृभ्याꣳ॒॒ स्थूरि॑ । \newline
44. वास॑सी॒ इति॒ वास॑सी । \newline
45. ने॒ष्टा॒पो॒तृभ्याꣳ॒॒ स्थूरि॒ स्थूरि॑ नेष्टापो॒तृभ्या᳚न् नेष्टापो॒तृभ्याꣳ॒॒ स्थूरि॑ यवाचि॒तं ॅय॑वाचि॒तꣳ स्थूरि॑ नेष्टापो॒तृभ्या᳚न् नेष्टापो॒तृभ्याꣳ॒॒ स्थूरि॑ यवाचि॒तम् । \newline
46. ने॒ष्टा॒पो॒तृभ्या॒मिति॑ नेष्टापो॒तृ - भ्या॒म् । \newline
47. स्थूरि॑ यवाचि॒तं ॅय॑वाचि॒तꣳ स्थूरि॒ स्थूरि॑ यवाचि॒त म॑च्छावा॒काया᳚ च्छावा॒काय॑ यवाचि॒तꣳ स्थूरि॒ स्थूरि॑ यवाचि॒त म॑च्छावा॒काय॑ । \newline
48. य॒वा॒चि॒त म॑च्छावा॒काया᳚ च्छावा॒काय॑ यवाचि॒तं ॅय॑वाचि॒त म॑च्छावा॒काया॑ न॒ड्वाह॑ मन॒ड्वाह॑ 
मच्छावा॒काय॑ यवाचि॒तं ॅय॑वाचि॒त म॑च्छावा॒काया॑ न॒ड्वाह᳚म् । \newline
49. य॒वा॒चि॒तमिति॑ यव - आ॒चि॒तम् । \newline
50. अ॒च्छा॒वा॒काया॑ न॒ड्वाह॑ मन॒ड्वाह॑ मच्छावा॒काया᳚ च्छावा॒काया॑ न॒ड्वाह॑ म॒ग्नीधे॒ ऽग्नीधे॑ ऽन॒ड्वाह॑ मच्छावा॒काया᳚ च्छावा॒काया॑ न॒ड्वाह॑ म॒ग्नीधे᳚ । \newline
51. अ॒न॒ड्वाह॑ म॒ग्नीधे॒ ऽग्नीधे॑ ऽन॒ड्वाह॑ मन॒ड्वाह॑ म॒ग्नीधे॑ भार्ग॒वो भा᳚र्ग॒वो᳚ ऽग्नीधे॑ ऽन॒ड्वाह॑ मन॒ड्वाह॑ म॒ग्नीधे॑ भार्ग॒वः । \newline
52. अ॒ग्नीधे॑ भार्ग॒वो भा᳚र्ग॒वो᳚ ऽग्नीधे॒ ऽग्नीधे॑ भार्ग॒वो होता॒ होता॑ भार्ग॒वो᳚ ऽग्नीधे॒ ऽग्नीधे॑ भार्ग॒वो होता᳚ । \newline
53. अ॒ग्नीध॒ इत्य॑ग्नि - इधे᳚ । \newline
54. भा॒र्ग॒वो होता॒ होता॑ भार्ग॒वो भा᳚र्ग॒वो होता॑ भवति भवति॒ होता॑ भार्ग॒वो भा᳚र्ग॒वो होता॑ भवति । \newline
55. होता॑ भवति भवति॒ होता॒ होता॑ भवति श्राय॒न्तीयꣳ॑ श्राय॒न्तीय॑म् भवति॒ होता॒ होता॑ भवति श्राय॒न्तीय᳚म् । \newline
56. भ॒व॒ति॒ श्रा॒य॒न्तीयꣳ॑ श्राय॒न्तीय॑म् भवति भवति श्राय॒न्तीय॑म् ब्रह्मसा॒मम् ब्र॑ह्मसा॒मꣳ श्रा॑य॒न्तीय॑म् भवति भवति श्राय॒न्तीय॑म् ब्रह्मसा॒मम् । \newline
57. श्रा॒य॒न्तीय॑म् ब्रह्मसा॒मम् ब्र॑ह्मसा॒मꣳ श्रा॑य॒न्तीयꣳ॑ श्राय॒न्तीय॑म् ब्रह्मसा॒मम् भ॑वति भवति ब्रह्मसा॒मꣳ श्रा॑य॒न्तीयꣳ॑ श्राय॒न्तीय॑म् ब्रह्मसा॒मम् भ॑वति । \newline
58. ब्र॒ह्म॒सा॒मम् भ॑वति भवति ब्रह्मसा॒मम् ब्र॑ह्मसा॒मम् भ॑वति वारव॒न्तीयं॑ ॅवारव॒न्तीय॑म् भवति ब्रह्मसा॒मम् ब्र॑ह्मसा॒मम् भ॑वति वारव॒न्तीय᳚म् । \newline
59. ब्र॒ह्म॒सा॒ममिति॑ ब्रह्म - सा॒मम् । \newline
60. भ॒व॒ति॒ वा॒र॒व॒न्तीयं॑ ॅवारव॒न्तीय॑म् भवति भवति वारव॒न्तीय॑ मग्निष्टोमसा॒म म॑ग्निष्टोमसा॒मं ॅवा॑रव॒न्तीय॑म् भवति भवति वारव॒न्तीय॑ मग्निष्टोमसा॒मम् । \newline
61. वा॒र॒व॒न्तीय॑ मग्निष्टोमसा॒म म॑ग्निष्टोमसा॒मं ॅवा॑रव॒न्तीयं॑ ॅवारव॒न्तीय॑ मग्निष्टोमसा॒मꣳ सा॑रस्व॒तीः सा॑रस्व॒ती र॑ग्निष्टोमसा॒मं ॅवा॑रव॒न्तीयं॑ ॅवारव॒न्तीय॑ मग्निष्टोमसा॒मꣳ सा॑रस्व॒तीः । \newline
62. वा॒र॒व॒न्तीय॒मिति॑ वार - व॒न्तीय᳚म् । \newline
63. अ॒ग्नि॒ष्टो॒म॒सा॒मꣳ सा॑रस्व॒तीः सा॑रस्व॒ती र॑ग्निष्टोमसा॒म म॑ग्निष्टोमसा॒मꣳ सा॑रस्व॒तीर॒पो॑ ऽपः सा॑रस्व॒ती र॑ग्निष्टोमसा॒म म॑ग्निष्टोमसा॒मꣳ सा॑रस्व॒तीर॒पः । \newline
64. अ॒ग्नि॒ष्टो॒म॒सा॒ममित्य॑ग्निष्टोम - सा॒मम् । \newline
65. सा॒र॒स्व॒ती र॒पो॑ ऽपः सा॑रस्व॒तीः सा॑रस्व॒ती र॒पो गृ॑ह्णाति गृह्णात्य॒पः सा॑रस्व॒तीः सा॑रस्व॒ती र॒पो गृ॑ह्णाति । \newline
66. अ॒पो गृ॑ह्णाति गृह्णात्य॒पो॑ ऽपो गृ॑ह्णाति । \newline
67. गृ॒ह्णा॒तीति॑ गृह्णाति । \newline
\pagebreak
\markright{ TS 1.8.19.1  \hfill https://www.vedavms.in \hfill}
\addcontentsline{toc}{section}{ TS 1.8.19.1 }
\section*{ TS 1.8.19.1 }

\textbf{TS 1.8.19.1 } \newline
\textbf{Samhita Paata} \newline

आ॒ग्ने॒य-म॒ष्टाक॑पालं॒ निर्व॑पति॒ हिर॑ण्यं॒ दक्षि॑णै॒न्द्र-मेका॑दशकपाल-मृष॒भो दक्षि॑णा वैश्वदे॒वं च॒रुं पि॒शङ्गी॑ पष्ठौ॒ही दक्षि॑णा मैत्रावरु॒णी-मा॒मिक्षां᳚ ॅव॒शा दक्षि॑णा बार्.हस्प॒त्यं च॒रुꣳ शि॑तिपृ॒ष्ठो दक्षि॑णाऽऽदि॒त्यां म॒ल्॒.हां ग॒र्भिणी॒मा ल॑भते मारु॒तीं पृश्ञ॑0079;ं॑ पष्ठौ॒ही-म॒श्विभ्यां᳚ पू॒ष्णे पु॑रो॒डाशं॒ द्वाद॑शकपालं॒ निर्व॑पति॒ सर॑स्वते सत्य॒वाचे॑ च॒रुꣳ स॑वि॒त्रे स॒त्यप्र॑सवाय पुरो॒डाशं॒ द्वाद॑शकपालं तिसृध॒न्वꣳ शु॑ष्कदृ॒तिर् दक्षि॑णा ॥ \newline

\textbf{Pada Paata} \newline

आ॒ग्ने॒यम् । अ॒ष्टाक॑पाल॒मित्य॒ष्टा-क॒पा॒ल॒म् । निरिति॑ । व॒प॒ति॒ । हिर॑ण्यम् । दक्षि॑णा । ऐ॒न्द्रम् । एका॑दशकपाल॒मित्येका॑दश-क॒पा॒ल॒म् । ऋ॒ष॒भः । दक्षि॑णा । वै॒श्व॒दे॒वमिति॑ वैश्व-दे॒वम् । च॒रुम् । पि॒शङ्गी᳚ । प॒ष्टौ॒ही । दक्षि॑णा । मै॒त्रा॒व॒रु॒णीमिति॑ मैत्रा-व॒रु॒णीम् । आ॒मिक्षा᳚म् । व॒शा । दक्षि॑णा । बा॒र्.॒ह॒स्प॒त्यम् । च॒रुम् । शि॒ति॒पृ॒ष्ठ इति॑ शिति-पृ॒ष्ठः । दक्षि॑णा । आ॒दि॒त्याम् । म॒ल्॒.हाम् । ग॒र्भिणी᳚म् । एति॑ । ल॒भ॒ते॒ । मा॒रु॒तीम् । पृश्नि᳚म् । प॒ष्ठौ॒हीम् । अ॒श्विभ्या॒मित्य॒श्वि-भ्या॒म् । पू॒ष्णे । पु॒रो॒डाश᳚म् । द्वाद॑शकपाल॒मिति॒ द्वाद॑श - क॒पा॒ल॒म् । निरिति॑ । व॒प॒ति॒ । सर॑स्वते । स॒त्य॒वाच॒ इति॑ सत्य - वाचे᳚ । च॒रुम् । स॒वि॒त्रे । स॒त्यप्र॑सवा॒येति॑ स॒त्य - प्र॒स॒वा॒य॒ । पु॒रो॒डाश᳚म् । द्वाद॑शकपाल॒मिति॒ द्वाद॑श - क॒पा॒ल॒म् । ति॒सृ॒ध॒न्वमिति॑ तिसृ - ध॒न्वम् । शु॒ष्क॒दृ॒तिरिति॑ शुष्क - दृ॒तिः । दक्षि॑णा ॥  \newline



\textbf{Jatai Paata} \newline

1. आ॒ग्ने॒य म॒ष्टाक॑पाल म॒ष्टाक॑पाल माग्ने॒य मा᳚ग्ने॒य म॒ष्टाक॑पालम् । \newline
2. अ॒ष्टाक॑पाल॒म् निर् णिर॒ष्टाक॑पाल म॒ष्टाक॑पाल॒म् निः । \newline
3. अ॒ष्टाक॑पाल॒मित्य॒ष्टा - क॒पा॒ल॒म् । \newline
4. निर् व॑पति वपति॒ निर् णिर् व॑पति । \newline
5. व॒प॒ति॒ हिर॑ण्यꣳ॒॒ हिर॑ण्यं ॅवपति वपति॒ हिर॑ण्यम् । \newline
6. हिर॑ण्य॒म् दक्षि॑णा॒ दक्षि॑णा॒ हिर॑ण्यꣳ॒॒ हिर॑ण्य॒म् दक्षि॑णा । \newline
7. दक्षि॑णै॒न्द्र मै॒न्द्रम् दक्षि॑णा॒ दक्षि॑णै॒न्द्रम् । \newline
8. ऐ॒न्द्र मेका॑दशकपाल॒ मेका॑दशकपाल मै॒न्द्र मै॒न्द्र मेका॑दशकपालम् । \newline
9. एका॑दशकपाल मृष॒भ ऋ॑ष॒भ एका॑दशकपाल॒ मेका॑दशकपाल मृष॒भः । \newline
10. एका॑दशकपाल॒मित्येका॑दश - क॒पा॒ल॒म् । \newline
11. ऋ॒ष॒भो दक्षि॑णा॒ दक्षि॑णर्.ष॒भ ऋ॑ष॒भो दक्षि॑णा । \newline
12. दक्षि॑णा वैश्वदे॒वं ॅवै᳚श्वदे॒वम् दक्षि॑णा॒ दक्षि॑णा वैश्वदे॒वम् । \newline
13. वै॒श्व॒दे॒वम् च॒रुम् च॒रुं ॅवै᳚श्वदे॒वं ॅवै᳚श्वदे॒वम् च॒रुम् । \newline
14. वै॒श्व॒दे॒वमिति॑ वैश्व - दे॒वम् । \newline
15. च॒रुम् पि॒शङ्गी॑ पि॒शङ्गी॑ च॒रुम् च॒रुम् पि॒शङ्गी᳚ । \newline
16. पि॒शङ्गी॑ पष्ठौ॒ही प॑ष्ठौ॒ही पि॒शङ्गी॑ पि॒शङ्गी॑ पष्ठौ॒ही । \newline
17. प॒ष्ठौ॒ही दक्षि॑णा॒ दक्षि॑णा पष्ठौ॒ही प॑ष्ठौ॒ही दक्षि॑णा । \newline
18. दक्षि॑णा मैत्रावरु॒णीम् मै᳚त्रावरु॒णीम् दक्षि॑णा॒ दक्षि॑णा मैत्रावरु॒णीम् । \newline
19. मै॒त्रा॒व॒रु॒णी मा॒मिक्षा॑ मा॒मिक्षा᳚म् मैत्रावरु॒णीम् मै᳚त्रावरु॒णी मा॒मिक्षा᳚म् । \newline
20. मै॒त्रा॒व॒रु॒णीमिति॑ मैत्रा - व॒रु॒णीम् । \newline
21. आ॒मिक्षां᳚ ॅव॒शा व॒शा ऽऽमिक्षा॑ मा॒मिक्षां᳚ ॅव॒शा । \newline
22. व॒शा दक्षि॑णा॒ दक्षि॑णा व॒शा व॒शा दक्षि॑णा । \newline
23. दक्षि॑णा बार्.हस्प॒त्यम् बा॑र्.हस्प॒त्यम् दक्षि॑णा॒ दक्षि॑णा बार्.हस्प॒त्यम् । \newline
24. बा॒र्॒.ह॒स्प॒त्यम् च॒रुम् च॒रुम् बा॑र्.हस्प॒त्यम् बा॑र्.हस्प॒त्यम् च॒रुम् । \newline
25. च॒रुꣳ शि॑तिपृ॒ष्ठः शि॑तिपृ॒ष्ठश्च॒रुम् च॒रुꣳ शि॑तिपृ॒ष्ठः । \newline
26. शि॒ति॒पृ॒ष्ठो दक्षि॑णा॒ दक्षि॑णा शितिपृ॒ष्ठः शि॑तिपृ॒ष्ठो दक्षि॑णा । \newline
27. शि॒ति॒पृ॒ष्ठ इति॑ शिति - पृ॒ष्ठः । \newline
28. दक्षि॑णा ऽऽदि॒त्या मा॑दि॒त्याम् दक्षि॑णा॒ दक्षि॑णा ऽऽदि॒त्याम् । \newline
29. आ॒दि॒त्याम् म॒ल्॒.हाम् म॒ल्॒.हा मा॑दि॒त्या मा॑दि॒त्याम् म॒ल्॒.हाम् । \newline
30. म॒ल्॒.हाम् ग॒र्भिणी᳚म् ग॒र्भिणी᳚म् म॒ल्॒.हाम् म॒ल्॒.हाम् ग॒र्भिणी᳚म् । \newline
31. ग॒र्भिणी॒ मा ग॒र्भिणी᳚म् ग॒र्भिणी॒ मा । \newline
32. आ ल॑भते लभत॒ आ ल॑भते । \newline
33. ल॒भ॒ते॒ मा॒रु॒तीम् मा॑रु॒तीम् ॅल॑भते लभते मारु॒तीम् । \newline
34. मा॒रु॒तीम् पृश्ञि॒म् पृश्ञि॑म् मारु॒तीम् मा॑रु॒तीम् पृश्ञि᳚म् । \newline
35. पृश्ञि॑म् पष्ठौ॒हीम् प॑ष्ठौ॒हीम् पृश्ञि॒म् पृश्ञि॑म् पष्ठौ॒हीम् । \newline
36. प॒ष्ठौ॒ही म॒श्विभ्या॑ म॒श्विभ्या᳚म् पष्ठौ॒हीम् प॑ष्ठौ॒ही म॒श्विभ्या᳚म् । \newline
37. अ॒श्विभ्या᳚म् पू॒ष्णे पू॒ष्णे᳚ ऽश्विभ्या॑ म॒श्विभ्या᳚म् पू॒ष्णे । \newline
38. अ॒श्विभ्या॒मित्य॒श्वि - भ्या॒म् । \newline
39. पू॒ष्णे पु॑रो॒डाश॑म् पुरो॒डाश॑म् पू॒ष्णे पू॒ष्णे पु॑रो॒डाश᳚म् । \newline
40. पु॒रो॒डाश॒म् द्वाद॑शकपाल॒म् द्वाद॑शकपालम् पुरो॒डाश॑म् पुरो॒डाश॒म् द्वाद॑शकपालम् । \newline
41. द्वाद॑शकपाल॒म् निर् णिर् द्वाद॑शकपाल॒म् द्वाद॑शकपाल॒म् निः । \newline
42. द्वाद॑शकपाल॒मिति॒ द्वाद॑श - क॒पा॒ल॒म् । \newline
43. निर् व॑पति वपति॒ निर् णिर् व॑पति । \newline
44. व॒प॒ति॒ सर॑स्वते॒ सर॑स्वते वपति वपति॒ सर॑स्वते । \newline
45. सर॑स्वते सत्य॒वाचे॑ सत्य॒वाचे॒ सर॑स्वते॒ सर॑स्वते सत्य॒वाचे᳚ । \newline
46. स॒त्य॒वाचे॑ च॒रुम् च॒रुꣳ स॑त्य॒वाचे॑ सत्य॒वाचे॑ च॒रुम् । \newline
47. स॒त्य॒वाच॒ इति॑ सत्य - वाचे᳚ । \newline
48. च॒रुꣳ स॑वि॒त्रे स॑वि॒त्रे च॒रुम् च॒रुꣳ स॑वि॒त्रे । \newline
49. स॒वि॒त्रे स॒त्यप्र॑सवाय स॒त्यप्र॑सवाय सवि॒त्रे स॑वि॒त्रे स॒त्यप्र॑सवाय । \newline
50. स॒त्यप्र॑सवाय पुरो॒डाश॑म् पुरो॒डाशꣳ॑ स॒त्यप्र॑सवाय स॒त्यप्र॑सवाय पुरो॒डाश᳚म् । \newline
51. स॒त्यप्र॑सवा॒येति॑ स॒त्य - प्र॒स॒वा॒य॒ । \newline
52. पु॒रो॒डाश॒म् द्वाद॑शकपाल॒म् द्वाद॑शकपालम् पुरो॒डाश॑म् पुरो॒डाश॒म् द्वाद॑शकपालम् । \newline
53. द्वाद॑शकपालम् तिसृध॒न्वम् ति॑सृध॒न्वम् द्वाद॑शकपाल॒म् द्वाद॑शकपालम् तिसृध॒न्वम् । \newline
54. द्वाद॑शकपाल॒मिति॒ द्वाद॑श - क॒पा॒ल॒म् । \newline
55. ति॒सृ॒ध॒न्वꣳ शु॑ष्कदृ॒तिः शु॑ष्कदृ॒ति स्ति॑सृध॒न्वम् ति॑सृध॒न्वꣳ शु॑ष्कदृ॒तिः । \newline
56. ति॒सृ॒ध॒न्वमिति॑ तिसृ - ध॒न्वम् । \newline
57. शु॒ष्क॒दृ॒तिर् दक्षि॑णा॒ दक्षि॑णा शुष्कदृ॒तिः शु॑ष्कदृ॒तिर् दक्षि॑णा । \newline
58. शु॒ष्क॒दृ॒तिरिति॑ शुष्क - दृ॒तिः । \newline
59. दक्षि॒णेति॒ दक्षि॑णा । \newline

\textbf{Ghana Paata } \newline

1. आ॒ग्ने॒य म॒ष्टाक॑पाल म॒ष्टाक॑पाल माग्ने॒य मा᳚ग्ने॒य म॒ष्टाक॑पाल॒न्निर् णिर॒ष्टाक॑पाल माग्ने॒य मा᳚ग्ने॒य म॒ष्टाक॑पाल॒न्निः । \newline
2. अ॒ष्टाक॑पाल॒न्निर् णिर॒ष्टाक॑पाल म॒ष्टाक॑पाल॒न्निर् व॑पति वपति॒ निर॒ष्टाक॑पाल म॒ष्टाक॑पाल॒न्निर् व॑पति । \newline
3. अ॒ष्टाक॑पाल॒मित्य॒ष्टा - क॒पा॒ल॒म् । \newline
4. निर् व॑पति वपति॒ निर् णिर् व॑पति॒ हिर॑ण्यꣳ॒॒ हिर॑ण्यं ॅवपति॒ निर् णिर् व॑पति॒ हिर॑ण्यम् । \newline
5. व॒प॒ति॒ हिर॑ण्यꣳ॒॒ हिर॑ण्यं ॅवपति वपति॒ हिर॑ण्य॒म् दक्षि॑णा॒ दक्षि॑णा॒ हिर॑ण्यं ॅवपति वपति॒ हिर॑ण्य॒म् दक्षि॑णा । \newline
6. हिर॑ण्य॒म् दक्षि॑णा॒ दक्षि॑णा॒ हिर॑ण्यꣳ॒॒ हिर॑ण्य॒म् दक्षि॑णै॒न्द्र मै॒न्द्रम् दक्षि॑णा॒ हिर॑ण्यꣳ॒॒ हिर॑ण्य॒म् दक्षि॑णै॒न्द्रम् । \newline
7. दक्षि॑णै॒न्द्र मै॒न्द्रम् दक्षि॑णा॒ दक्षि॑णै॒न्द्र मेका॑दशकपाल॒ मेका॑दशकपाल मै॒न्द्रम् दक्षि॑णा॒ दक्षि॑णै॒न्द्र मेका॑दशकपालम् । \newline
8. ऐ॒न्द्र मेका॑दशकपाल॒ मेका॑दशकपाल मै॒न्द्र मै॒न्द्र मेका॑दशकपाल मृष॒भ ऋ॑ष॒भ एका॑दशकपाल मै॒न्द्र मै॒न्द्र मेका॑दशकपाल मृष॒भः । \newline
9. एका॑दशकपाल मृष॒भ ऋ॑ष॒भ एका॑दशकपाल॒ मेका॑दशकपाल मृष॒भो दक्षि॑णा॒ दक्षि॑णर्.ष॒भ एका॑दशकपाल॒ मेका॑दशकपाल मृष॒भो दक्षि॑णा । \newline
10. एका॑दशकपाल॒मित्येका॑दश - क॒पा॒ल॒म् । \newline
11. ऋ॒ष॒भो दक्षि॑णा॒ दक्षि॑णर्.ष॒भ ऋ॑ष॒भो दक्षि॑णा वैश्वदे॒वं ॅवै᳚श्वदे॒वम् दक्षि॑णर्.ष॒भ ऋ॑ष॒भो दक्षि॑णा वैश्वदे॒वम् । \newline
12. दक्षि॑णा वैश्वदे॒वं ॅवै᳚श्वदे॒वम् दक्षि॑णा॒ दक्षि॑णा वैश्वदे॒वम् च॒रुम् च॒रुं ॅवै᳚श्वदे॒वम् दक्षि॑णा॒ दक्षि॑णा वैश्वदे॒वम् च॒रुम् । \newline
13. वै॒श्व॒दे॒वम् च॒रुम् च॒रुं ॅवै᳚श्वदे॒वं ॅवै᳚श्वदे॒वम् च॒रुम् पि॒शङ्गी॑ पि॒शङ्गी॑ च॒रुं ॅवै᳚श्वदे॒वं ॅवै᳚श्वदे॒वम् च॒रुम् पि॒शङ्गी᳚ । \newline
14. वै॒श्व॒दे॒वमिति॑ वैश्व - दे॒वम् । \newline
15. च॒रुम् पि॒शङ्गी॑ पि॒शङ्गी॑ च॒रुम् च॒रुम् पि॒शङ्गी॑ पष्ठौ॒ही प॑ष्ठौ॒ही पि॒शङ्गी॑ च॒रुम् च॒रुम् पि॒शङ्गी॑ पष्ठौ॒ही । \newline
16. पि॒शङ्गी॑ पष्ठौ॒ही प॑ष्ठौ॒ही पि॒शङ्गी॑ पि॒शङ्गी॑ पष्ठौ॒ही दक्षि॑णा॒ दक्षि॑णा पष्ठौ॒ही पि॒शङ्गी॑ पि॒शङ्गी॑ पष्ठौ॒ही दक्षि॑णा । \newline
17. प॒ष्ठौ॒ही दक्षि॑णा॒ दक्षि॑णा पष्ठौ॒ही प॑ष्ठौ॒ही दक्षि॑णा मैत्रावरु॒णीम् मै᳚त्रावरु॒णीम् दक्षि॑णा पष्ठौ॒ही प॑ष्ठौ॒ही दक्षि॑णा मैत्रावरु॒णीम् । \newline
18. दक्षि॑णा मैत्रावरु॒णीम् मै᳚त्रावरु॒णीम् दक्षि॑णा॒ दक्षि॑णा मैत्रावरु॒णी मा॒मिक्षा॑ मा॒मिक्षा᳚म् मैत्रावरु॒णीम् दक्षि॑णा॒ दक्षि॑णा मैत्रावरु॒णी मा॒मिक्षा᳚म् । \newline
19. मै॒त्रा॒व॒रु॒णी मा॒मिक्षा॑ मा॒मिक्षा᳚म् मैत्रावरु॒णीम् मै᳚त्रावरु॒णी मा॒मिक्षां᳚ ॅव॒शा व॒शा ऽऽमिक्षा᳚म् मैत्रावरु॒णीम् मै᳚त्रावरु॒णी मा॒मिक्षां᳚ ॅव॒शा । \newline
20. मै॒त्रा॒व॒रु॒णीमिति॑ मैत्रा - व॒रु॒णीम् । \newline
21. आ॒मिक्षां᳚ ॅव॒शा व॒शा ऽऽमिक्षा॑ मा॒मिक्षां᳚ ॅव॒शा दक्षि॑णा॒ दक्षि॑णा व॒शा ऽऽमिक्षा॑ मा॒मिक्षां᳚ ॅव॒शा दक्षि॑णा । \newline
22. व॒शा दक्षि॑णा॒ दक्षि॑णा व॒शा व॒शा दक्षि॑णा बार्.हस्प॒त्यम् बा॑र्.हस्प॒त्यम् दक्षि॑णा व॒शा व॒शा दक्षि॑णा बार्.हस्प॒त्यम् । \newline
23. दक्षि॑णा बार्.हस्प॒त्यम् बा॑र्.हस्प॒त्यम् दक्षि॑णा॒ दक्षि॑णा बार्.हस्प॒त्यम् च॒रुम् च॒रुम् बा॑र्.हस्प॒त्यम् दक्षि॑णा॒ दक्षि॑णा बार्.हस्प॒त्यम् च॒रुम् । \newline
24. बा॒र्॒.ह॒स्प॒त्यम् च॒रुम् च॒रुम् बा॑र्.हस्प॒त्यम् बा॑र्.हस्प॒त्यम् च॒रुꣳ शि॑तिपृ॒ष्ठः शि॑तिपृ॒ष्ठ श्च॒रुम् बा॑र्.हस्प॒त्यम् बा॑र्.हस्प॒त्यम् च॒रुꣳ शि॑तिपृ॒ष्ठः । \newline
25. च॒रुꣳ शि॑तिपृ॒ष्ठः शि॑तिपृ॒ष्ठ श्च॒रुम् च॒रुꣳ शि॑तिपृ॒ष्ठो दक्षि॑णा॒ दक्षि॑णा शितिपृ॒ष्ठ श्च॒रुम् च॒रुꣳ शि॑तिपृ॒ष्ठो दक्षि॑णा । \newline
26. शि॒ति॒पृ॒ष्ठो दक्षि॑णा॒ दक्षि॑णा शितिपृ॒ष्ठः शि॑तिपृ॒ष्ठो दक्षि॑णा ऽऽदि॒त्या मा॑दि॒त्याम् दक्षि॑णा शितिपृ॒ष्ठः शि॑तिपृ॒ष्ठो दक्षि॑णा ऽऽदि॒त्याम् । \newline
27. शि॒ति॒पृ॒ष्ठ इति॑ शिति - पृ॒ष्ठः । \newline
28. दक्षि॑णा ऽऽदि॒त्या मा॑दि॒त्याम् दक्षि॑णा॒ दक्षि॑णा ऽऽदि॒त्याम् म॒ल्॒.हाम् म॒ल्॒.हा मा॑दि॒त्याम् दक्षि॑णा॒ दक्षि॑णा ऽऽदि॒त्याम् म॒ल्॒.हाम् । \newline
29. आ॒दि॒त्याम् म॒ल्॒.हाम् म॒ल्॒.हा मा॑दि॒त्या मा॑दि॒त्याम् म॒ल्॒.हाम् ग॒र्भिणी᳚म् ग॒र्भिणी᳚म् म॒ल्॒.हा मा॑दि॒त्या मा॑दि॒त्याम् म॒ल्॒.हाम् ग॒र्भिणी᳚म् । \newline
30. म॒ल्॒.हाम् ग॒र्भिणी᳚म् ग॒र्भिणी᳚म् म॒ल्॒.हाम् म॒ल्॒.हाम् ग॒र्भिणी॒ मा ग॒र्भिणी᳚म् म॒ल्॒.हाम् म॒ल्॒.हाम् ग॒र्भिणी॒ मा । \newline
31. ग॒र्भिणी॒ मा ग॒र्भिणी᳚म् ग॒र्भिणी॒ मा ल॑भते लभत॒ आ ग॒र्भिणी᳚म् ग॒र्भिणी॒ मा ल॑भते । \newline
32. आ ल॑भते लभत॒ आ ल॑भते मारु॒तीम् मा॑रु॒तीम् ॅल॑भत॒ आ ल॑भते मारु॒तीम् । \newline
33. ल॒भ॒ते॒ मा॒रु॒तीम् मा॑रु॒तीम् ॅल॑भते लभते मारु॒तीम् पृश्ञि॒म् पृश्ञि॑म् मारु॒तीम् ॅल॑भते लभते मारु॒तीम् पृश्ञि᳚म् । \newline
34. मा॒रु॒तीम् पृश्ञि॒म् पृश्ञि॑म् मारु॒तीम् मा॑रु॒तीम् पृश्ञि॑म् पष्ठौ॒हीम् प॑ष्ठौ॒हीम् पृश्ञि॑म् मारु॒तीम् मा॑रु॒तीम् पृश्ञि॑म् पष्ठौ॒हीम् । \newline
35. पृश्ञि॑म् पष्ठौ॒हीम् प॑ष्ठौ॒हीम् पृश्ञि॒म् पृश्ञि॑म् पष्ठौ॒ही म॒श्विभ्या॑ म॒श्विभ्या᳚म् पष्ठौ॒हीम् पृश्ञि॒म् पृश्ञि॑म् पष्ठौ॒ही म॒श्विभ्या᳚म् । \newline
36. प॒ष्ठौ॒ही म॒श्विभ्या॑ म॒श्विभ्या᳚म् पष्ठौ॒हीम् प॑ष्ठौ॒ही म॒श्विभ्या᳚म् पू॒ष्णे पू॒ष्णे᳚ ऽश्विभ्या᳚म् पष्ठौ॒हीम् प॑ष्ठौ॒ही म॒श्विभ्या᳚म् पू॒ष्णे । \newline
37. अ॒श्विभ्या᳚म् पू॒ष्णे पू॒ष्णे᳚ ऽश्विभ्या॑ म॒श्विभ्या᳚म् पू॒ष्णे पु॑रो॒डाश॑म् पुरो॒डाश॑म् पू॒ष्णे᳚ ऽश्विभ्या॑ म॒श्विभ्या᳚म् पू॒ष्णे पु॑रो॒डाश᳚म् । \newline
38. अ॒श्विभ्या॒मित्य॒श्वि - भ्या॒म् । \newline
39. पू॒ष्णे पु॑रो॒डाश॑म् पुरो॒डाश॑म् पू॒ष्णे पू॒ष्णे पु॑रो॒डाश॒म् द्वाद॑शकपाल॒म् द्वाद॑शकपालम् पुरो॒डाश॑म् पू॒ष्णे पू॒ष्णे पु॑रो॒डाश॒म् द्वाद॑शकपालम् । \newline
40. पु॒रो॒डाश॒म् द्वाद॑शकपाल॒म् द्वाद॑शकपालम् पुरो॒डाश॑म् पुरो॒डाश॒म् द्वाद॑शकपाल॒न्निर् णिर् द्वाद॑शकपालम् पुरो॒डाश॑म् पुरो॒डाश॒म् द्वाद॑शकपाल॒न्निः । \newline
41. द्वाद॑शकपाल॒न्निर् णिर् द्वाद॑शकपाल॒म् द्वाद॑शकपाल॒न्निर् व॑पति वपति॒ निर् द्वाद॑शकपाल॒म् द्वाद॑शकपाल॒न्निर् व॑पति । \newline
42. द्वाद॑शकपाल॒मिति॒ द्वाद॑श - क॒पा॒ल॒म् । \newline
43. निर् व॑पति वपति॒ निर् णिर् व॑पति॒ सर॑स्वते॒ सर॑स्वते वपति॒ निर् णिर् व॑पति॒ सर॑स्वते । \newline
44. व॒प॒ति॒ सर॑स्वते॒ सर॑स्वते वपति वपति॒ सर॑स्वते सत्य॒वाचे॑ सत्य॒वाचे॒ सर॑स्वते वपति वपति॒ सर॑स्वते सत्य॒वाचे᳚ । \newline
45. सर॑स्वते सत्य॒वाचे॑ सत्य॒वाचे॒ सर॑स्वते॒ सर॑स्वते सत्य॒वाचे॑ च॒रुम् च॒रुꣳ स॑त्य॒वाचे॒ सर॑स्वते॒ सर॑स्वते सत्य॒वाचे॑ च॒रुम् । \newline
46. स॒त्य॒वाचे॑ च॒रुम् च॒रुꣳ स॑त्य॒वाचे॑ सत्य॒वाचे॑ च॒रुꣳ स॑वि॒त्रे स॑वि॒त्रे च॒रुꣳ स॑त्य॒वाचे॑ सत्य॒वाचे॑ च॒रुꣳ स॑वि॒त्रे । \newline
47. स॒त्य॒वाच॒ इति॑ सत्य - वाचे᳚ । \newline
48. च॒रुꣳ स॑वि॒त्रे स॑वि॒त्रे च॒रुम् च॒रुꣳ स॑वि॒त्रे स॒त्यप्र॑सवाय स॒त्यप्र॑सवाय सवि॒त्रे च॒रुम् च॒रुꣳ स॑वि॒त्रे स॒त्यप्र॑सवाय । \newline
49. स॒वि॒त्रे स॒त्यप्र॑सवाय स॒त्यप्र॑सवाय सवि॒त्रे स॑वि॒त्रे स॒त्यप्र॑सवाय पुरो॒डाश॑म् पुरो॒डाशꣳ॑ स॒त्यप्र॑सवाय सवि॒त्रे स॑वि॒त्रे स॒त्यप्र॑सवाय पुरो॒डाश᳚म् । \newline
50. स॒त्यप्र॑सवाय पुरो॒डाश॑म् पुरो॒डाशꣳ॑ स॒त्यप्र॑सवाय स॒त्यप्र॑सवाय पुरो॒डाश॒म् द्वाद॑शकपाल॒म् द्वाद॑शकपालम् पुरो॒डाशꣳ॑ स॒त्यप्र॑सवाय स॒त्यप्र॑सवाय पुरो॒डाश॒म् द्वाद॑शकपालम् । \newline
51. स॒त्यप्र॑सवा॒येति॑ स॒त्य - प्र॒स॒वा॒य॒ । \newline
52. पु॒रो॒डाश॒म् द्वाद॑शकपाल॒म् द्वाद॑शकपालम् पुरो॒डाश॑म् पुरो॒डाश॒म् द्वाद॑शकपालम् तिसृध॒न्वम् ति॑सृध॒न्वम् द्वाद॑शकपालम् पुरो॒डाश॑म् पुरो॒डाश॒म् द्वाद॑शकपालम् तिसृध॒न्वम् । \newline
53. द्वाद॑शकपालम् तिसृध॒न्वम् ति॑सृध॒न्वम् द्वाद॑शकपाल॒म् द्वाद॑शकपालम् तिसृध॒न्वꣳ शु॑ष्कदृ॒तिः शु॑ष्कदृ॒ति स्ति॑सृध॒न्वम् द्वाद॑शकपाल॒म् द्वाद॑शकपालम् तिसृध॒न्वꣳ शु॑ष्कदृ॒तिः । \newline
54. द्वाद॑शकपाल॒मिति॒ द्वाद॑श - क॒पा॒ल॒म् । \newline
55. ति॒सृ॒ध॒न्वꣳ शु॑ष्कदृ॒तिः शु॑ष्कदृ॒ति स्ति॑सृध॒न्वम् ति॑सृध॒न्वꣳ शु॑ष्कदृ॒तिर् दक्षि॑णा॒ दक्षि॑णा शुष्कदृ॒ति स्ति॑सृध॒न्वम् ति॑सृध॒न्वꣳ शु॑ष्कदृ॒तिर् दक्षि॑णा । \newline
56. ति॒सृ॒ध॒न्वमिति॑ तिसृ - ध॒न्वम् । \newline
57. शु॒ष्क॒दृ॒तिर् दक्षि॑णा॒ दक्षि॑णा शुष्कदृ॒तिः शु॑ष्कदृ॒तिर् दक्षि॑णा । \newline
58. शु॒ष्क॒दृ॒तिरिति॑ शुष्क - दृ॒तिः । \newline
59. दक्षि॒णेति॒ दक्षि॑णा । \newline
\pagebreak
\markright{ TS 1.8.20.1  \hfill https://www.vedavms.in \hfill}
\addcontentsline{toc}{section}{ TS 1.8.20.1 }
\section*{ TS 1.8.20.1 }

\textbf{TS 1.8.20.1 } \newline
\textbf{Samhita Paata} \newline

आ॒ग्ने॒य-म॒ष्टाक॑पालं॒ निर्व॑पति सौ॒म्यं च॒रुꣳ सा॑वि॒त्रं द्वाद॑शकपालं बार्.हस्प॒त्यं च॒रुं त्वा॒ष्ट्रम॒ष्टाक॑पालं ॅवैश्वान॒रं द्वाद॑शकपालं॒ दक्षि॑णो रथवाहनवा॒हो दक्षि॑णा सारस्व॒तं च॒रुं निर्व॑पति पौ॒ष्णं च॒रुं मै॒त्रं च॒रुं ॅवा॑रु॒णं च॒रुं क्षै᳚त्रप॒त्यं च॒रुमा॑दि॒त्यं च॒रुमुत्त॑रो रथवाहनवा॒हो दक्षि॑णा ॥ \newline

\textbf{Pada Paata} \newline

आ॒ग्ने॒यम् । अ॒ष्टाक॑पाल॒मित्य॒ष्टा-क॒पा॒ल॒म् । निरिति॑ । व॒प॒ति॒ । सौ॒म्यम् । च॒रुम् । सा॒वि॒त्रम् । द्वाद॑शकपाल॒मिति॒ द्वाद॑श-क॒पा॒ल॒म् । बा॒र्.॒ह॒स्प॒त्यम् । च॒रुम् । त्वा॒ष्ट्रम् । अ॒ष्टाक॑पाल॒मित्य॒ष्टा - क॒पा॒ल॒म् । वै॒श्वा॒न॒रम् । द्वाद॑शकपाल॒मिति॒ द्वाद॑श - क॒पा॒ल॒म् । दक्षि॑णः । र॒थ॒वा॒ह॒न॒वा॒ह इति॑ रथवाहन-वा॒हः । दक्षि॑णा । सा॒र॒स्व॒तम् । च॒रुम् । निरिति॑ । व॒प॒ति॒ । पौ॒ष्णम् । च॒रुम् । मै॒त्रम् । च॒रुम् । वा॒रु॒णम् । च॒रुम् । क्षै॒त्र॒प॒त्यमिति॑ क्षैत्र - प॒त्यम् । च॒रुम् । आ॒दि॒त्यम् । च॒रुम् । उत्त॑र॒ इत्युत् - त॒रः॒ । र॒थ॒वा॒ह॒न॒वा॒ह इति॑ रथवाहन - वा॒हः । दक्षि॑णा ॥  \newline



\textbf{Jatai Paata} \newline

1. आ॒ग्ने॒य म॒ष्टाक॑पाल म॒ष्टाक॑पाल माग्ने॒य मा᳚ग्ने॒य म॒ष्टाक॑पालम् । \newline
2. अ॒ष्टाक॑पाल॒म् निर् णिर॒ष्टाक॑पाल म॒ष्टाक॑पाल॒म् निः । \newline
3. अ॒ष्टाक॑पाल॒मित्य॒ष्टा - क॒पा॒ल॒म् । \newline
4. निर् व॑पति वपति॒ निर् णिर् व॑पति । \newline
5. व॒प॒ति॒ सौ॒म्यꣳ सौ॒म्यं ॅव॑पति वपति सौ॒म्यम् । \newline
6. सौ॒म्यम् च॒रुम् च॒रुꣳ सौ॒म्यꣳ सौ॒म्यम् च॒रुम् । \newline
7. च॒रुꣳ सा॑वि॒त्रꣳ सा॑वि॒त्रम् च॒रुम् च॒रुꣳ सा॑वि॒त्रम् । \newline
8. सा॒वि॒त्रम् द्वाद॑शकपाल॒म् द्वाद॑शकपालꣳ सावि॒त्रꣳ सा॑वि॒त्रम् द्वाद॑शकपालम् । \newline
9. द्वाद॑शकपालम् बार्.हस्प॒त्यम् बा॑र्.हस्प॒त्यम् द्वाद॑शकपाल॒म् द्वाद॑शकपालम् बार्.हस्प॒त्यम् । \newline
10. द्वाद॑शकपाल॒मिति॒ द्वाद॑श - क॒पा॒ल॒म् । \newline
11. बा॒र्॒.ह॒स्प॒त्यम् च॒रुम् च॒रुम् बा॑र्.हस्प॒त्यम् बा॑र्.हस्प॒त्यम् च॒रुम् । \newline
12. च॒रुम् त्वा॒ष्ट्रम् त्वा॒ष्ट्रम् च॒रुम् च॒रुम् त्वा॒ष्ट्रम् । \newline
13. त्वा॒ष्ट्र म॒ष्टाक॑पाल म॒ष्टाक॑पालम् त्वा॒ष्ट्रम् त्वा॒ष्ट्र म॒ष्टाक॑पालम् । \newline
14. अ॒ष्टाक॑पालं ॅवैश्वान॒रं ॅवै᳚श्वान॒र म॒ष्टाक॑पाल म॒ष्टाक॑पालं ॅवैश्वान॒रम् । \newline
15. अ॒ष्टाक॑पाल॒मित्य॒ष्टा - क॒पा॒ल॒म् । \newline
16. वै॒श्वा॒न॒रम् द्वाद॑शकपाल॒म् द्वाद॑शकपालं ॅवैश्वान॒रं ॅवै᳚श्वान॒रम् द्वाद॑शकपालम् । \newline
17. द्वाद॑शकपाल॒म् दक्षि॑णो॒ दक्षि॑णो॒ द्वाद॑शकपाल॒म् द्वाद॑शकपाल॒म् दक्षि॑णः । \newline
18. द्वाद॑शकपाल॒मिति॒ द्वाद॑श - क॒पा॒ल॒म् । \newline
19. दक्षि॑णो रथवाहनवा॒हो र॑थवाहनवा॒हो दक्षि॑णो॒ दक्षि॑णो रथवाहनवा॒हः । \newline
20. र॒थ॒वा॒ह॒न॒वा॒हो दक्षि॑णा॒ दक्षि॑णा रथवाहनवा॒हो र॑थवाहनवा॒हो दक्षि॑णा । \newline
21. र॒थ॒वा॒ह॒न॒वा॒ह इति॑ रथवाहन - वा॒हः । \newline
22. दक्षि॑णा सारस्व॒तꣳ सा॑रस्व॒तम् दक्षि॑णा॒ दक्षि॑णा सारस्व॒तम् । \newline
23. सा॒र॒स्व॒तम् च॒रुम् च॒रुꣳ सा॑रस्व॒तꣳ सा॑रस्व॒तम् च॒रुम् । \newline
24. च॒रुम् निर् णिश्च॒रुम् च॒रुम् निः । \newline
25. निर् व॑पति वपति॒ निर् णिर् व॑पति । \newline
26. व॒प॒ति॒ पौ॒ष्णम् पौ॒ष्णं ॅव॑पति वपति पौ॒ष्णम् । \newline
27. पौ॒ष्णम् च॒रुम् च॒रुम् पौ॒ष्णम् पौ॒ष्णम् च॒रुम् । \newline
28. च॒रुम् मै॒त्रम् मै॒त्रम् च॒रुम् च॒रुम् मै॒त्रम् । \newline
29. मै॒त्रम् च॒रुम् च॒रुम् मै॒त्रम् मै॒त्रम् च॒रुम् । \newline
30. च॒रुं ॅवा॑रु॒णं ॅवा॑रु॒णम् च॒रुम् च॒रुं ॅवा॑रु॒णम् । \newline
31. वा॒रु॒णम् च॒रुम् च॒रुं ॅवा॑रु॒णं ॅवा॑रु॒णम् च॒रुम् । \newline
32. च॒रुम् क्षै᳚त्रप॒त्यम् क्षै᳚त्रप॒त्यम् च॒रुम् च॒रुम् क्षै᳚त्रप॒त्यम् । \newline
33. क्षै॒त्र॒प॒त्यम् च॒रुम् च॒रुम् क्षै᳚त्रप॒त्यम् क्षै᳚त्रप॒त्यम् च॒रुम् । \newline
34. क्षै॒त्र॒प॒त्यमिति॑ क्षैत्र - प॒त्यम् । \newline
35. च॒रु मा॑दि॒त्य मा॑दि॒त्यम् च॒रुम् च॒रु मा॑दि॒त्यम् । \newline
36. आ॒दि॒त्यम् च॒रुम् च॒रु मा॑दि॒त्य मा॑दि॒त्यम् च॒रुम् । \newline
37. च॒रु मुत्त॑र॒ उत्त॑र श्च॒रुम् च॒रु मुत्त॑रः । \newline
38. उत्त॑रो रथवाहनवा॒हो र॑थवाहनवा॒ह उत्त॑र॒ उत्त॑रो रथवाहनवा॒हः । \newline
39. उत्त॑र॒ इत्युत् - त॒रः॒ । \newline
40. र॒थ॒वा॒ह॒न॒वा॒हो दक्षि॑णा॒ दक्षि॑णा रथवाहनवा॒हो र॑थवाहनवा॒हो दक्षि॑णा । \newline
41. र॒थ॒वा॒ह॒न॒वा॒ह इति॑ रथवाहन - वा॒हः । \newline
42. दक्षि॒णेति॒ दक्षि॑णा । \newline

\textbf{Ghana Paata } \newline

1. आ॒ग्ने॒य म॒ष्टाक॑पाल म॒ष्टाक॑पाल माग्ने॒य मा᳚ग्ने॒य म॒ष्टाक॑पाल॒न्निर् णिर॒ष्टाक॑पाल माग्ने॒य मा᳚ग्ने॒य म॒ष्टाक॑पाल॒न्निः । \newline
2. अ॒ष्टाक॑पाल॒न्निर् णिर॒ष्टाक॑पाल म॒ष्टाक॑पाल॒न्निर् व॑पति वपति॒ निर॒ष्टाक॑पाल म॒ष्टाक॑पाल॒न्निर् व॑पति । \newline
3. अ॒ष्टाक॑पाल॒मित्य॒ष्टा - क॒पा॒ल॒म् । \newline
4. निर् व॑पति वपति॒ निर् णिर् व॑पति सौ॒म्यꣳ सौ॒म्यं ॅव॑पति॒ निर् णिर् व॑पति सौ॒म्यम् । \newline
5. व॒प॒ति॒ सौ॒म्यꣳ सौ॒म्यं ॅव॑पति वपति सौ॒म्यम् च॒रुम् च॒रुꣳ सौ॒म्यं ॅव॑पति वपति सौ॒म्यम् च॒रुम् । \newline
6. सौ॒म्यम् च॒रुम् च॒रुꣳ सौ॒म्यꣳ सौ॒म्यम् च॒रुꣳ सा॑वि॒त्रꣳ सा॑वि॒त्रम् च॒रुꣳ सौ॒म्यꣳ सौ॒म्यम् च॒रुꣳ सा॑वि॒त्रम् । \newline
7. च॒रुꣳ सा॑वि॒त्रꣳ सा॑वि॒त्रम् च॒रुम् च॒रुꣳ सा॑वि॒त्रम् द्वाद॑शकपाल॒म् द्वाद॑शकपालꣳ सावि॒त्रम् च॒रुम् च॒रुꣳ सा॑वि॒त्रम् द्वाद॑शकपालम् । \newline
8. सा॒वि॒त्रम् द्वाद॑शकपाल॒म् द्वाद॑शकपालꣳ सावि॒त्रꣳ सा॑वि॒त्रम् द्वाद॑शकपालम् बार्.हस्प॒त्यम् बा॑र्.हस्प॒त्यम् द्वाद॑शकपालꣳ सावि॒त्रꣳ सा॑वि॒त्रम् द्वाद॑शकपालम् बार्.हस्प॒त्यम् । \newline
9. द्वाद॑शकपालम् बार्.हस्प॒त्यम् बा॑र्.हस्प॒त्यम् द्वाद॑शकपाल॒म् द्वाद॑शकपालम् बार्.हस्प॒त्यम् च॒रुम् च॒रुम् बा॑र्.हस्प॒त्यम् द्वाद॑शकपाल॒म् द्वाद॑शकपालम् बार्.हस्प॒त्यम् च॒रुम् । \newline
10. द्वाद॑शकपाल॒मिति॒ द्वाद॑श - क॒पा॒ल॒म् । \newline
11. बा॒र्॒.ह॒स्प॒त्यम् च॒रुम् च॒रुम् बा॑र्.हस्प॒त्यम् बा॑र्.हस्प॒त्यम् च॒रुम् त्वा॒ष्ट्रम् त्वा॒ष्ट्रम् च॒रुम् बा॑र्.हस्प॒त्यम् बा॑र्.हस्प॒त्यम् च॒रुम् त्वा॒ष्ट्रम् । \newline
12. च॒रुम् त्वा॒ष्ट्रम् त्वा॒ष्ट्रम् च॒रुम् च॒रुम् त्वा॒ष्ट्र म॒ष्टाक॑पाल म॒ष्टाक॑पालम् त्वा॒ष्ट्रम् च॒रुम् च॒रुम् त्वा॒ष्ट्र म॒ष्टाक॑पालम् । \newline
13. त्वा॒ष्ट्र म॒ष्टाक॑पाल म॒ष्टाक॑पालम् त्वा॒ष्ट्रम् त्वा॒ष्ट्र म॒ष्टाक॑पालं ॅवैश्वान॒रं ॅवै᳚श्वान॒र म॒ष्टाक॑पालम् त्वा॒ष्ट्रम् त्वा॒ष्ट्र म॒ष्टाक॑पालं ॅवैश्वान॒रम् । \newline
14. अ॒ष्टाक॑पालं ॅवैश्वान॒रं ॅवै᳚श्वान॒र म॒ष्टाक॑पाल म॒ष्टाक॑पालं ॅवैश्वान॒रम् द्वाद॑शकपाल॒म् द्वाद॑शकपालं ॅवैश्वान॒र म॒ष्टाक॑पाल म॒ष्टाक॑पालं ॅवैश्वान॒रम् द्वाद॑शकपालम् । \newline
15. अ॒ष्टाक॑पाल॒मित्य॒ष्टा - क॒पा॒ल॒म् । \newline
16. वै॒श्वा॒न॒रम् द्वाद॑शकपाल॒म् द्वाद॑शकपालं ॅवैश्वान॒रं ॅवै᳚श्वान॒रम् द्वाद॑शकपाल॒म् दक्षि॑णो॒ दक्षि॑णो॒ द्वाद॑शकपालं ॅवैश्वान॒रं ॅवै᳚श्वान॒रम् द्वाद॑शकपाल॒म् दक्षि॑णः । \newline
17. द्वाद॑शकपाल॒म् दक्षि॑णो॒ दक्षि॑णो॒ द्वाद॑शकपाल॒म् द्वाद॑शकपाल॒म् दक्षि॑णो रथवाहनवा॒हो र॑थवाहनवा॒हो दक्षि॑णो॒ द्वाद॑शकपाल॒म् द्वाद॑शकपाल॒म् दक्षि॑णो रथवाहनवा॒हः । \newline
18. द्वाद॑शकपाल॒मिति॒ द्वाद॑श - क॒पा॒ल॒म् । \newline
19. दक्षि॑णो रथवाहनवा॒हो र॑थवाहनवा॒हो दक्षि॑णो॒ दक्षि॑णो रथवाहनवा॒हो दक्षि॑णा॒ दक्षि॑णा रथवाहनवा॒हो दक्षि॑णो॒ दक्षि॑णो रथवाहनवा॒हो दक्षि॑णा । \newline
20. र॒थ॒वा॒ह॒न॒वा॒हो दक्षि॑णा॒ दक्षि॑णा रथवाहनवा॒हो र॑थवाहनवा॒हो दक्षि॑णा सारस्व॒तꣳ सा॑रस्व॒तम् दक्षि॑णा रथवाहनवा॒हो र॑थवाहनवा॒हो दक्षि॑णा सारस्व॒तम् । \newline
21. र॒थ॒वा॒ह॒न॒वा॒ह इति॑ रथवाहन - वा॒हः । \newline
22. दक्षि॑णा सारस्व॒तꣳ सा॑रस्व॒तम् दक्षि॑णा॒ दक्षि॑णा सारस्व॒तम् च॒रुम् च॒रुꣳ सा॑रस्व॒तम् दक्षि॑णा॒ दक्षि॑णा सारस्व॒तम् च॒रुम् । \newline
23. सा॒र॒स्व॒तम् च॒रुम् च॒रुꣳ सा॑रस्व॒तꣳ सा॑रस्व॒तम् च॒रुन्निर् णिश्च॒रुꣳ सा॑रस्व॒तꣳ सा॑रस्व॒तम् च॒रुन्निः । \newline
24. च॒रुन्निर् णि श्च॒रुम् च॒रुन्निर् व॑पति वपति॒ नि श्च॒रुम् च॒रुन्निर् व॑पति । \newline
25. निर् व॑पति वपति॒ निर् णिर् व॑पति पौ॒ष्णम् पौ॒ष्णं ॅव॑पति॒ निर् णिर् व॑पति पौ॒ष्णम् । \newline
26. व॒प॒ति॒ पौ॒ष्णम् पौ॒ष्णं ॅव॑पति वपति पौ॒ष्णम् च॒रुम् च॒रुम् पौ॒ष्णं ॅव॑पति वपति पौ॒ष्णम् च॒रुम् । \newline
27. पौ॒ष्णम् च॒रुम् च॒रुम् पौ॒ष्णम् पौ॒ष्णम् च॒रुम् मै॒त्रम् मै॒त्रम् च॒रुम् पौ॒ष्णम् पौ॒ष्णम् च॒रुम् मै॒त्रम् । \newline
28. च॒रुम् मै॒त्रम् मै॒त्रम् च॒रुम् च॒रुम् मै॒त्रम् च॒रुम् च॒रुम् मै॒त्रम् च॒रुम् च॒रुम् मै॒त्रम् च॒रुम् । \newline
29. मै॒त्रम् च॒रुम् च॒रुम् मै॒त्रम् मै॒त्रम् च॒रुं ॅवा॑रु॒णं ॅवा॑रु॒णम् च॒रुम् मै॒त्रम् मै॒त्रम् च॒रुं ॅवा॑रु॒णम् । \newline
30. च॒रुं ॅवा॑रु॒णं ॅवा॑रु॒णम् च॒रुम् च॒रुं ॅवा॑रु॒णम् च॒रुम् च॒रुं ॅवा॑रु॒णम् च॒रुम् च॒रुं ॅवा॑रु॒णम् च॒रुम् । \newline
31. वा॒रु॒णम् च॒रुम् च॒रुं ॅवा॑रु॒णं ॅवा॑रु॒णम् च॒रुम् क्षै᳚त्रप॒त्यम् क्षै᳚त्रप॒त्यम् च॒रुं ॅवा॑रु॒णं ॅवा॑रु॒णम् च॒रुम् क्षै᳚त्रप॒त्यम् । \newline
32. च॒रुम् क्षै᳚त्रप॒त्यम् क्षै᳚त्रप॒त्यम् च॒रुम् च॒रुम् क्षै᳚त्रप॒त्यम् च॒रुम् च॒रुम् क्षै᳚त्रप॒त्यम् च॒रुम् च॒रुम् क्षै᳚त्रप॒त्यम् च॒रुम् । \newline
33. क्षै॒त्र॒प॒त्यम् च॒रुम् च॒रुम् क्षै᳚त्रप॒त्यम् क्षै᳚त्रप॒त्यम् च॒रु मा॑दि॒त्य मा॑दि॒त्यम् च॒रुम् क्षै᳚त्रप॒त्यम् क्षै᳚त्रप॒त्यम् च॒रु मा॑दि॒त्यम् । \newline
34. क्षै॒त्र॒प॒त्यमिति॑ क्षैत्र - प॒त्यम् । \newline
35. च॒रु मा॑दि॒त्य मा॑दि॒त्यम् च॒रुम् च॒रु मा॑दि॒त्यम् च॒रुम् च॒रु मा॑दि॒त्यम् च॒रुम् च॒रु मा॑दि॒त्यम् च॒रुम् । \newline
36. आ॒दि॒त्यम् च॒रुम् च॒रु मा॑दि॒त्य मा॑दि॒त्यम् च॒रु मुत्त॑र॒ उत्त॑र श्च॒रु मा॑दि॒त्य मा॑दि॒त्यम् च॒रु मुत्त॑रः । \newline
37. च॒रु मुत्त॑र॒ उत्त॑र श्च॒रुम् च॒रु मुत्त॑रो रथवाहनवा॒हो र॑थवाहनवा॒ह उत्त॑र श्च॒रुम् च॒रु मुत्त॑रो रथवाहनवा॒हः । \newline
38. उत्त॑रो रथवाहनवा॒हो र॑थवाहनवा॒ह उत्त॑र॒ उत्त॑रो रथवाहनवा॒हो दक्षि॑णा॒ दक्षि॑णा रथवाहनवा॒ह उत्त॑र॒ उत्त॑रो रथवाहनवा॒हो दक्षि॑णा । \newline
39. उत्त॑र॒ इत्युत् - त॒रः॒ । \newline
40. र॒थ॒वा॒ह॒न॒वा॒हो दक्षि॑णा॒ दक्षि॑णा रथवाहनवा॒हो र॑थवाहनवा॒हो दक्षि॑णा । \newline
41. र॒थ॒वा॒ह॒न॒वा॒ह इति॑ रथवाहन - वा॒हः । \newline
42. दक्षि॒णेति॒ दक्षि॑णा । \newline
\pagebreak
\markright{ TS 1.8.21.1  \hfill https://www.vedavms.in \hfill}
\addcontentsline{toc}{section}{ TS 1.8.21.1 }
\section*{ TS 1.8.21.1 }

\textbf{TS 1.8.21.1 } \newline
\textbf{Samhita Paata} \newline

स्वा॒द्वीं त्वा᳚ स्वा॒दुना॑ ती॒व्रां ती॒व्रेणा॒मृता॑म॒मृते॑न सृ॒जामि॒ सꣳ सोमे॑न॒ सोमो᳚ऽस्य॒श्विभ्यां᳚ पच्यस्व॒ सर॑स्वत्यै पच्य॒स्वेन्द्रा॑य सु॒त्रांणे॑ पच्यस्व पु॒नातु॑ ते परि॒स्रुतꣳ॒॒ सोमꣳ॒॒ सूर्य॑स्य दुहि॒ता । वारे॑ण॒ शश्व॑ता॒ तना᳚ ॥ वा॒युः पू॒तः प॒वित्रे॑ण प्र॒त्यङ् सोमो॒ अति॑द्रुतः । इन्द्र॑स्य॒ युज्यः॒ सखा᳚ ॥कु॒विद॒ङ् यव॑मन्तो॒ यवं॑ चि॒द् यथा॒  दान्त्य॑नुपू॒र्वं ॅवि॒यूय॑ । इ॒हेहै॑षां कृणुत॒ भोज॑नानि॒ ( ) ये ब॒र्॒.हिषो॒ नमो॑वृक्तिं॒ न ज॒ग्मुः ॥ आ॒श्वि॒नं धू॒म्रमा ल॑भते सारस्व॒तं मे॒षमै॒न्द्रमृ॑ष॒भ-मै॒न्द्र-मेका॑दशकपालं॒ निर्व॑पति सावि॒त्रं द्वाद॑शकपालं ॅवारु॒णं दश॑कपालꣳ॒॒ सोम॑प्रतीकाः पितरस्तृप्णुत॒ वड॑बा॒ दक्षि॑णा ॥ \newline

\textbf{Pada Paata} \newline

स्वा॒द्वीम् । त्वा॒ । स्वा॒दुना᳚ । ती॒व्राम् । ती॒व्रेण॑ । अ॒मृता᳚म् । अ॒मृते॑न । सृ॒जामि॑ । समिति॑ । सोमे॑न । सोमः॑ । अ॒सि॒ । अ॒श्विभ्या॒मित्य॒श्वि-भ्या॒म् । प॒च्य॒स्व॒ । सर॑स्वत्यै । प॒च्य॒स्व॒ । इन्द्रा॑य । सु॒त्रांण॒ इति॑ सु - त्रांणे᳚ । प॒च्य॒स्व॒ । पु॒नातु॑ । ते॒ । प॒रि॒स्रुत॒मिति॑ परि - स्रुत᳚म् । सोम᳚म् । सूर्य॑स्य । दु॒हि॒ता ॥ वारे॑ण । शश्व॑ता । तना᳚ ॥ वा॒युः । पू॒तः । प॒वित्रे॑ण । प्र॒त्यङ् । सोमः॑ । अति॑द्रुत॒ इत्यति॑-द्रु॒तः॒ ॥ इन्द्र॑स्य । युज्यः॑ । सखा᳚ ॥ कु॒वित् । अ॒ङ्ग । यव॑मन्त॒ इति॒ यव॑-म॒न्तः॒ । यव᳚म् । चि॒त् । यथा᳚ । दान्ति॑ । अ॒नु॒पू॒र्वमित्य॑नु - पू॒र्वम् । वि॒यूयेति॑ वि - यूय॑ ॥ इ॒हेहेती॒ह - इ॒ह॒ । ए॒षा॒म् । कृ॒णु॒त॒ । भोज॑नानि ( ) । ये । ब॒र्॒.हिषः॑ । नमो॑वृक्ति॒मिति॒ नमः॑ - वृ॒क्ति॒म् । न । ज॒ग्मुः ॥ आ॒श्वि॒नम् । धू॒म्रम् । एति॑ । ल॒भ॒ते॒ । सा॒र॒स्व॒तम् । मे॒षम् । ऐ॒न्द्रम् । ऋ॒ष॒भम् । ऐ॒न्द्रम् । एका॑दशकपाल॒मित्येका॑दश-क॒पा॒ल॒म् । निरिति॑ । व॒प॒ति॒ । सा॒वि॒त्रम् । द्वाद॑शकपाल॒मिति॒ द्वाद॑श - क॒पा॒ल॒म् । वा॒रु॒णम् । दश॑कपाल॒मिति॒ दश॑ - क॒पा॒ल॒म् । सोम॑प्रतीका॒ इति॒ सोम॑ - प्र॒ती॒काः॒ । पि॒त॒रः॒ । तृ॒प्णु॒त॒ । वड॑बा । दक्षि॑णा ॥  \newline



\textbf{Jatai Paata} \newline

1. स्वा॒द्वीम् त्वा᳚ त्वा स्वा॒द्वीꣳ स्वा॒द्वीम् त्वा᳚ । \newline
2. त्वा॒ स्वा॒दुना᳚ स्वा॒दुना᳚ त्वा त्वा स्वा॒दुना᳚ । \newline
3. स्वा॒दुना॑ ती॒व्राम् ती॒व्राꣳ स्वा॒दुना᳚ स्वा॒दुना॑ ती॒व्राम् । \newline
4. ती॒व्राम् ती॒व्रेण॑ ती॒व्रेण॑ ती॒व्राम् ती॒व्राम् ती॒व्रेण॑ । \newline
5. ती॒व्रेणा॒मृता॑ म॒मृता᳚म् ती॒व्रेण॑ ती॒व्रेणा॒मृता᳚म् । \newline
6. अ॒मृता॑ म॒मृते॑ना॒ मृते॑ना॒मृता॑ म॒मृता॑ म॒मृते॑न । \newline
7. अ॒मृते॑न सृ॒जामि॑ सृ॒जा म्य॒मृते॑ना॒ मृते॑न सृ॒जामि॑ । \newline
8. सृ॒जामि॒ सꣳ सꣳ सृ॒जामि॑ सृ॒जामि॒ सम् । \newline
9. सꣳ सोमे॑न॒ सोमे॑न॒ सꣳ सꣳ सोमे॑न । \newline
10. सोमे॑न॒ सोमः॒ सोमः॒ सोमे॑न॒ सोमे॑न॒ सोमः॑ । \newline
11. सोमो᳚ ऽस्यसि॒ सोमः॒ सोमो॑ ऽसि । \newline
12. अ॒स्य॒श्विभ्या॑ म॒श्विभ्या॑ मस्य स्य॒श्विभ्या᳚म् । \newline
13. अ॒श्विभ्या᳚म् पच्यस्व पच्यस्वा॒ श्विभ्या॑ म॒श्विभ्या᳚म् पच्यस्व । \newline
14. अ॒श्विभ्या॒मित्य॒श्वि - भ्या॒म् । \newline
15. प॒च्य॒स्व॒ सर॑स्वत्यै॒ सर॑स्वत्यै पच्यस्व पच्यस्व॒ सर॑स्वत्यै । \newline
16. सर॑स्वत्यै पच्यस्व पच्यस्व॒ सर॑स्वत्यै॒ सर॑स्वत्यै पच्यस्व । \newline
17. प॒च्य॒स्वे न्द्रा॒ये न्द्रा॑य पच्यस्व पच्य॒स्वे न्द्रा॑य । \newline
18. इन्द्रा॑य सु॒त्रांणे॑ सु॒त्रांण॒ इन्द्रा॒ये न्द्रा॑य सु॒त्रांणे᳚ । \newline
19. सु॒त्रांणे॑ पच्यस्व पच्यस्व सु॒त्रांणे॑ सु॒त्रांणे॑ पच्यस्व । \newline
20. सु॒त्रांण॒ इति॑ सु - त्रांणे᳚ । \newline
21. प॒च्य॒स्व॒ पु॒नातु॑ पु॒नातु॑ पच्यस्व पच्यस्व पु॒नातु॑ । \newline
22. पु॒नातु॑ ते ते पु॒नातु॑ पु॒नातु॑ ते । \newline
23. ते॒ प॒रि॒स्रुत॑म् परि॒स्रुत॑म् ते ते परि॒स्रुत᳚म् । \newline
24. प॒रि॒स्रुतꣳ॒॒ सोमꣳ॒॒ सोम॑म् परि॒स्रुत॑म् परि॒स्रुतꣳ॒॒ सोम᳚म् । \newline
25. प॒रि॒स्रुत॒मिति॑ परि - स्रुत᳚म् । \newline
26. सोमꣳ॒॒ सूर्य॑स्य॒ सूर्य॑स्य॒ सोमꣳ॒॒ सोमꣳ॒॒ सूर्य॑स्य । \newline
27. सूर्य॑स्य दुहि॒ता दु॑हि॒ता सूर्य॑स्य॒ सूर्य॑स्य दुहि॒ता । \newline
28. दु॒हि॒तेति॑ दुहि॒ता । \newline
29. वारे॑ण॒ शश्व॑ता॒ शश्व॑ता॒ वारे॑ण॒ वारे॑ण॒ शश्व॑ता । \newline
30. शश्व॑ता॒ तना॒ तना॒ शश्व॑ता॒ शश्व॑ता॒ तना᳚ । \newline
31. तनेति॒ तना᳚ । \newline
32. वा॒युः पू॒तः पू॒तो वा॒युर् वा॒युः पू॒तः । \newline
33. पू॒तः प॒वित्रे॑ण प॒वित्रे॑ण पू॒तः पू॒तः प॒वित्रे॑ण । \newline
34. प॒वित्रे॑ण प्र॒त्यङ् प्र॒त्यङ् प॒वित्रे॑ण प॒वित्रे॑ण प्र॒त्यङ् । \newline
35. प्र॒त्यङ् ख्सोमः॒ सोमः॑ प्र॒त्यङ् प्र॒त्यङ् ख्सोमः॑ । \newline
36. सोमो॒ अति॑द्रुतो॒ अति॑द्रुतः॒ सोमः॒ सोमो॒ अति॑द्रुतः । \newline
37. अति॑द्रुत॒ इत्यति॑ - द्रु॒तः॒ । \newline
38. इन्द्र॑स्य॒ युज्यो॒ युज्य॒ इन्द्र॒स्ये न्द्र॑स्य॒ युज्यः॑ । \newline
39. युज्यः॒ सखा॒ सखा॒ युज्यो॒ युज्यः॒ सखा᳚ । \newline
40. सखेति॒ सखा᳚ । \newline
41. कु॒वि द॒ङ्गाङ्ग कु॒वित् कु॒वि द॒ङ्ग । \newline
42. अ॒ङ्ग यव॑मन्तो॒ यव॑मन्तो॒ ऽङ्गाङ्ग यव॑मन्तः । \newline
43. यव॑मन्तो॒ यवं॒ ॅयवं॒ ॅयव॑मन्तो॒ यव॑मन्तो॒ यव᳚म् । \newline
44. यव॑मन्त॒ इति॒ यव॑ - म॒न्तः॒ । \newline
45. यव॑म् चिच् चि॒द् यवं॒ ॅयव॑म् चित् । \newline
46. चि॒द् यथा॒ यथा॑ चिच् चि॒द् यथा᳚ । \newline
47. यथा॒ दान्ति॒ दान्ति॒ यथा॒ यथा॒ दान्ति॑ । \newline
48. दान्त्य॑नुपू॒र्व म॑नुपू॒र्वम् दान्ति॒ दान्त्य॑नुपू॒र्वम् । \newline
49. अ॒नु॒पू॒र्वं ॅवि॒यूय॑ वि॒यूया॑नुपू॒र्व म॑नुपू॒र्वं ॅवि॒यूय॑ । \newline
50. अ॒नु॒पू॒र्वमित्य॑नु - पू॒र्वम् । \newline
51. वि॒यूयेति॑ वि - यूय॑ । \newline
52. इ॒हेहै॑षा मेषा मि॒हे हे॒हे है॑षाम् । \newline
53. इ॒हेहेती॒ह - इ॒ह॒ । \newline
54. ए॒षा॒म् कृ॒णु॒त॒ कृ॒णु॒तै॒षा॒ मे॒षा॒म् कृ॒णु॒त॒ । \newline
55. कृ॒णु॒त॒ भोज॑नानि॒ भोज॑नानि कृणुत कृणुत॒ भोज॑नानि । \newline
56. भोज॑नानि॒ ये ये भोज॑नानि॒ भोज॑नानि॒ ये । \newline
57. ये ब॒र्॒.हिषो॑ ब॒र्॒.हिषो॒ ये ये ब॒र्॒.हिषः॑ । \newline
58. ब॒र्॒.हिषो॒ नमो॑वृक्ति॒म् नमो॑वृक्तिम् ब॒र्॒.हिषो॑ ब॒र्॒.हिषो॒ नमो॑वृक्तिम् । \newline
59. नमो॑वृक्ति॒म् न न नमो॑वृक्ति॒म् नमो॑वृक्ति॒म् न । \newline
60. नमो॑वृक्ति॒मिति॒ नमः॑ - वृ॒क्ति॒म् । \newline
61. न ज॒ग्मुर् ज॒ग्मुर् न न ज॒ग्मुः । \newline
62. ज॒ग्मुरिति॑ ज॒ग्मुः । \newline
63. आ॒श्वि॒नम् धू॒म्रम् धू॒म्र मा᳚श्वि॒न मा᳚श्वि॒नम् धू॒म्रम् । \newline
64. धू॒म्र मा धू॒म्रम् धू॒म्र मा । \newline
65. आ ल॑भते लभत॒ आ ल॑भते । \newline
66. ल॒भ॒ते॒ सा॒र॒स्व॒तꣳ सा॑रस्व॒तम् ॅल॑भते लभते सारस्व॒तम् । \newline
67. सा॒र॒स्व॒तम् मे॒षम् मे॒षꣳ सा॑रस्व॒तꣳ सा॑रस्व॒तम् मे॒षम् । \newline
68. मे॒ष मै॒न्द्र मै॒न्द्रम् मे॒षम् मे॒ष मै॒न्द्रम् । \newline
69. ऐ॒न्द्र मृ॑ष॒भ मृ॑ष॒भ मै॒न्द्र मै॒न्द्र मृ॑ष॒भम् । \newline
70. ऋ॒ष॒भ मै॒न्द्र मै॒न्द्र मृ॑ष॒भ मृ॑ष॒भ मै॒न्द्रम् । \newline
71. ऐ॒न्द्र मेका॑दशकपाल॒ मेका॑दशकपाल मै॒न्द्र मै॒न्द्र मेका॑दशकपालम् । \newline
72. एका॑दशकपाल॒म् निर् णिरेका॑दशकपाल॒ मेका॑दशकपाल॒म् निः । \newline
73. एका॑दशकपाल॒मित्येका॑दश - क॒पा॒ल॒म् । \newline
74. निर् व॑पति वपति॒ निर् णिर् व॑पति । \newline
75. व॒प॒ति॒ सा॒वि॒त्रꣳ सा॑वि॒त्रं ॅव॑पति वपति सावि॒त्रम् । \newline
76. सा॒वि॒त्रम् द्वाद॑शकपाल॒म् द्वाद॑शकपालꣳ सावि॒त्रꣳ सा॑वि॒त्रम् द्वाद॑शकपालम् । \newline
77. द्वाद॑शकपालं ॅवारु॒णं ॅवा॑रु॒णम् द्वाद॑शकपाल॒म् द्वाद॑शकपालं ॅवारु॒णम् । \newline
78. द्वाद॑शकपाल॒मिति॒ द्वाद॑श - क॒पा॒ल॒म् । \newline
79. वा॒रु॒णम् दश॑कपाल॒म् दश॑कपालं ॅवारु॒णं ॅवा॑रु॒णम् दश॑कपालम् । \newline
80. दश॑कपालꣳ॒॒ सोम॑प्रतीकाः॒ सोम॑प्रतीका॒ दश॑कपाल॒म् दश॑कपालꣳ॒॒ सोम॑प्रतीकाः । \newline
81. दश॑कपाल॒मिति॒ दश॑ - क॒पा॒ल॒म् । \newline
82. सोम॑प्रतीकाः पितरः पितरः॒ सोम॑प्रतीकाः॒ सोम॑प्रतीकाः पितरः । \newline
83. सोम॑प्रतीका॒ इति॒ सोम॑ - प्र॒ती॒काः॒ । \newline
84. पि॒त॒र॒ स्तृ॒प्णु॒त॒ तृ॒प्णु॒त॒ पि॒त॒रः॒ पि॒त॒र॒ स्तृ॒प्णु॒त॒ । \newline
85. तृ॒प्णु॒त॒ वड॑बा॒ वड॑बा तृप्णुत तृप्णुत॒ वड॑बा । \newline
86. वड॑बा॒ दक्षि॑णा॒ दक्षि॑णा॒ वड॑बा॒ वड॑बा॒ दक्षि॑णा । \newline
87. दक्षि॒णेति॒ दक्षि॑णा । \newline

\textbf{Ghana Paata } \newline

1. स्वा॒द्वीम् त्वा᳚ त्वा स्वा॒द्वीꣳ स्वा॒द्वीम् त्वा᳚ स्वा॒दुना᳚ स्वा॒दुना᳚ त्वा स्वा॒द्वीꣳ स्वा॒द्वीम् त्वा᳚ स्वा॒दुना᳚ । \newline
2. त्वा॒ स्वा॒दुना᳚ स्वा॒दुना᳚ त्वा त्वा स्वा॒दुना॑ ती॒व्राम् ती॒व्राꣳ स्वा॒दुना᳚ त्वा त्वा स्वा॒दुना॑ ती॒व्राम् । \newline
3. स्वा॒दुना॑ ती॒व्राम् ती॒व्राꣳ स्वा॒दुना᳚ स्वा॒दुना॑ ती॒व्राम् ती॒व्रेण॑ ती॒व्रेण॑ ती॒व्राꣳ स्वा॒दुना᳚ स्वा॒दुना॑ ती॒व्राम् ती॒व्रेण॑ । \newline
4. ती॒व्राम् ती॒व्रेण॑ ती॒व्रेण॑ ती॒व्राम् ती॒व्राम् ती॒व्रेणा॒ मृता॑ म॒मृता᳚म् ती॒व्रेण॑ ती॒व्राम् ती॒व्राम् ती॒व्रेणा॒ मृता᳚म् । \newline
5. ती॒व्रेणा॒ मृता॑ म॒मृता᳚म् ती॒व्रेण॑ ती॒व्रेणा॒ मृता॑ म॒मृते॑ना॒ मृते॑ना॒ मृता᳚म् ती॒व्रेण॑ ती॒व्रेणा॒ मृता॑ म॒मृते॑न । \newline
6. अ॒मृता॑ म॒मृते॑ ना॒मृते॑ ना॒मृता॑ म॒मृता॑ म॒मृते॑न सृ॒जामि॑ सृ॒जाम्य॒ मृते॑ना॒ मृता॑ म॒मृता॑ म॒मृते॑न सृ॒जामि॑ । \newline
7. अ॒मृते॑न सृ॒जामि॑ सृ॒जाम्य॒ मृते॑ना॒ मृते॑न सृ॒जामि॒ सꣳ सꣳ सृ॒जाम्य॒ मृते॑ना॒ मृते॑न सृ॒जामि॒ सम् । \newline
8. सृ॒जामि॒ सꣳ सꣳ सृ॒जामि॑ सृ॒जामि॒ सꣳ सोमे॑न॒ सोमे॑न॒ सꣳ सृ॒जामि॑ सृ॒जामि॒ सꣳ सोमे॑न । \newline
9. सꣳ सोमे॑न॒ सोमे॑न॒ सꣳ सꣳ सोमे॑न॒ सोमः॒ सोमः॒ सोमे॑न॒ सꣳ सꣳ सोमे॑न॒ सोमः॑ । \newline
10. सोमे॑न॒ सोमः॒ सोमः॒ सोमे॑न॒ सोमे॑न॒ सोमो᳚ ऽस्यसि॒ सोमः॒ सोमे॑न॒ सोमे॑न॒ सोमो॑ ऽसि । \newline
11. सोमो᳚ ऽस्यसि॒ सोमः॒ सोमो᳚ ऽस्य॒श्विभ्या॑ म॒श्विभ्या॑ मसि॒ सोमः॒ सोमो᳚ ऽस्य॒श्विभ्या᳚म् । \newline
12. अ॒स्य॒ श्विभ्या॑ म॒श्विभ्या॑ मस्यस्य॒ श्विभ्या᳚म् पच्यस्व पच्यस्वा॒ श्विभ्या॑ मस्यस्य॒ श्विभ्या᳚म् पच्यस्व । \newline
13. अ॒श्विभ्या᳚म् पच्यस्व पच्यस्वा॒ श्विभ्या॑ म॒श्विभ्या᳚म् पच्यस्व॒ सर॑स्वत्यै॒ सर॑स्वत्यै पच्यस्वा॒ श्विभ्या॑ म॒श्विभ्या᳚म् पच्यस्व॒ सर॑स्वत्यै । \newline
14. अ॒श्विभ्या॒मित्य॒श्वि - भ्या॒म् । \newline
15. प॒च्य॒स्व॒ सर॑स्वत्यै॒ सर॑स्वत्यै पच्यस्व पच्यस्व॒ सर॑स्वत्यै पच्यस्व पच्यस्व॒ सर॑स्वत्यै पच्यस्व पच्यस्व॒ सर॑स्वत्यै पच्यस्व । \newline
16. सर॑स्वत्यै पच्यस्व पच्यस्व॒ सर॑स्वत्यै॒ सर॑स्वत्यै पच्य॒स्वे न्द्रा॒ये न्द्रा॑य पच्यस्व॒ सर॑स्वत्यै॒ सर॑स्वत्यै पच्य॒स्वे न्द्रा॑य । \newline
17. प॒च्य॒स्वे न्द्रा॒ये न्द्रा॑य पच्यस्व पच्य॒स्वे न्द्रा॑य सु॒त्रांणे॑ सु॒त्रांण॒ इन्द्रा॑य पच्यस्व पच्य॒स्वे न्द्रा॑य सु॒त्रांणे᳚ । \newline
18. इन्द्रा॑य सु॒त्रांणे॑ सु॒त्रांण॒ इन्द्रा॒ये न्द्रा॑य सु॒त्रांणे॑ पच्यस्व पच्यस्व सु॒त्रांण॒ इन्द्रा॒ये न्द्रा॑य सु॒त्रांणे॑ पच्यस्व । \newline
19. सु॒त्रांणे॑ पच्यस्व पच्यस्व सु॒त्रांणे॑ सु॒त्रांणे॑ पच्यस्व पु॒नातु॑ पु॒नातु॑ पच्यस्व सु॒त्रांणे॑ सु॒त्रांणे॑ पच्यस्व पु॒नातु॑ । \newline
20. सु॒त्रांण॒ इति॑ सु - त्रांणे᳚ । \newline
21. प॒च्य॒स्व॒ पु॒नातु॑ पु॒नातु॑ पच्यस्व पच्यस्व पु॒नातु॑ ते ते पु॒नातु॑ पच्यस्व पच्यस्व पु॒नातु॑ ते । \newline
22. पु॒नातु॑ ते ते पु॒नातु॑ पु॒नातु॑ ते परि॒स्रुत॑म् परि॒स्रुत॑म् ते पु॒नातु॑ पु॒नातु॑ ते परि॒स्रुत᳚म् । \newline
23. ते॒ प॒रि॒स्रुत॑म् परि॒स्रुत॑म् ते ते परि॒स्रुतꣳ॒॒ सोमꣳ॒॒ सोम॑म् परि॒स्रुत॑म् ते ते परि॒स्रुतꣳ॒॒ सोम᳚म् । \newline
24. प॒रि॒स्रुतꣳ॒॒ सोमꣳ॒॒ सोम॑म् परि॒स्रुत॑म् परि॒स्रुतꣳ॒॒ सोमꣳ॒॒ सूर्य॑स्य॒ सूर्य॑स्य॒ सोम॑म् परि॒स्रुत॑म् परि॒स्रुतꣳ॒॒ सोमꣳ॒॒ सूर्य॑स्य । \newline
25. प॒रि॒स्रुत॒मिति॑ परि - स्रुत᳚म् । \newline
26. सोमꣳ॒॒ सूर्य॑स्य॒ सूर्य॑स्य॒ सोमꣳ॒॒ सोमꣳ॒॒ सूर्य॑स्य दुहि॒ता दु॑हि॒ता सूर्य॑स्य॒ सोमꣳ॒॒ सोमꣳ॒॒ सूर्य॑स्य दुहि॒ता । \newline
27. सूर्य॑स्य दुहि॒ता दु॑हि॒ता सूर्य॑स्य॒ सूर्य॑स्य दुहि॒ता । \newline
28. दु॒हि॒तेति॑ दुहि॒ता । \newline
29. वारे॑ण॒ शश्व॑ता॒ शश्व॑ता॒ वारे॑ण॒ वारे॑ण॒ शश्व॑ता॒ तना॒ तना॒ शश्व॑ता॒ वारे॑ण॒ वारे॑ण॒ शश्व॑ता॒ तना᳚ । \newline
30. शश्व॑ता॒ तना॒ तना॒ शश्व॑ता॒ शश्व॑ता॒ तना᳚ । \newline
31. तनेति॒ तना᳚ । \newline
32. वा॒युः पू॒तः पू॒तो वा॒युर् वा॒युः पू॒तः प॒वित्रे॑ण प॒वित्रे॑ण पू॒तो वा॒युर् वा॒युः पू॒तः प॒वित्रे॑ण । \newline
33. पू॒तः प॒वित्रे॑ण प॒वित्रे॑ण पू॒तः पू॒तः प॒वित्रे॑ण प्र॒त्यङ् प्र॒त्यङ् प॒वित्रे॑ण पू॒तः पू॒तः प॒वित्रे॑ण प्र॒त्यङ् । \newline
34. प॒वित्रे॑ण प्र॒त्यङ् प्र॒त्यङ् प॒वित्रे॑ण प॒वित्रे॑ण प्र॒त्यङ्ख्सोमः॒ सोमः॑ प्र॒त्यङ् प॒वित्रे॑ण प॒वित्रे॑ण 
प्र॒त्यङ्ख्सोमः॑ । \newline
35. प्र॒त्यङ्ख्सोमः॒ सोमः॑ प्र॒त्यङ् प्र॒त्यङ्ख्सोमो॒ अति॑द्रुतो॒ अति॑द्रुतः॒ सोमः॑ प्र॒त्यङ् प्र॒त्यङ्ख्सोमो॒ अति॑द्रुतः । \newline
36. सोमो॒ अति॑द्रुतो॒ अति॑द्रुतः॒ सोमः॒ सोमो॒ अति॑द्रुतः । \newline
37. अति॑द्रुत॒ इत्यति॑ - द्रु॒तः॒ । \newline
38. इन्द्र॑स्य॒ युज्यो॒ युज्य॒ इन्द्र॒स्ये न्द्र॑स्य॒ युज्यः॒ सखा॒ सखा॒ युज्य॒ इन्द्र॒स्ये न्द्र॑स्य॒ युज्यः॒ सखा᳚ । \newline
39. युज्यः॒ सखा॒ सखा॒ युज्यो॒ युज्यः॒ सखा᳚ । \newline
40. सखेति॒ सखा᳚ । \newline
41. कु॒विद॒ङ्गाङ्ग कु॒वित् कु॒विद॒ङ्ग यव॑मन्तो॒ यव॑मन्तो॒ ऽङ्ग कु॒वित् कु॒विद॒ङ्ग यव॑मन्तः । \newline
42. अ॒ङ्ग यव॑मन्तो॒ यव॑मन्तो॒ ऽङ्गाङ्ग यव॑मन्तो॒ यवं॒ ॅयवं॒ ॅयव॑मन्तो॒ ऽङ्गाङ्ग यव॑मन्तो॒ यव᳚म् । \newline
43. यव॑मन्तो॒ यवं॒ ॅयवं॒ ॅयव॑मन्तो॒ यव॑मन्तो॒ यव॑म् चिच् चि॒द् यवं॒ ॅयव॑मन्तो॒ यव॑मन्तो॒ यव॑म् चित् । \newline
44. यव॑मन्त॒ इति॒ यव॑ - म॒न्तः॒ । \newline
45. यव॑म् चिच् चि॒द् यवं॒ ॅयव॑म् चि॒द् यथा॒ यथा॑ चि॒द् यवं॒ ॅयव॑म् चि॒द् यथा᳚ । \newline
46. चि॒द् यथा॒ यथा॑ चिच् चि॒द् यथा॒ दान्ति॒ दान्ति॒ यथा॑ चिच् चि॒द् यथा॒ दान्ति॑ । \newline
47. यथा॒ दान्ति॒ दान्ति॒ यथा॒ यथा॒ दान्त्य॑नुपू॒र्व म॑नुपू॒र्वम् दान्ति॒ यथा॒ यथा॒ दान्त्य॑नुपू॒र्वम् । \newline
48. दान्त्य॑नुपू॒र्व म॑नुपू॒र्वम् दान्ति॒ दान्त्य॑नुपू॒र्वं ॅवि॒यूय॑ वि॒यूया॑नुपू॒र्वम् दान्ति॒ दान्त्य॑नुपू॒र्वं ॅवि॒यूय॑ । \newline
49. अ॒नु॒पू॒र्वं ॅवि॒यूय॑ वि॒यूया॑नुपू॒र्व म॑नुपू॒र्वं ॅवि॒यूय॑ । \newline
50. अ॒नु॒पू॒र्वमित्य॑नु - पू॒र्वम् । \newline
51. वि॒यूयेति॑ वि - यूय॑ । \newline
52. इ॒हेहै॑षा मेषा मि॒हे हे॒हे है॑षाम् कृणुत कृणुतैषा मि॒हे हे॒हे है॑षाम् कृणुत । \newline
53. इ॒हेहेती॒ह - इ॒ह॒ । \newline
54. ए॒षा॒म् कृ॒णु॒त॒ कृ॒णु॒तै॒षा॒ मे॒षा॒म् कृ॒णु॒त॒ भोज॑नानि॒ भोज॑नानि कृणुतैषा मेषाम् कृणुत॒ भोज॑नानि । \newline
55. कृ॒णु॒त॒ भोज॑नानि॒ भोज॑नानि कृणुत कृणुत॒ भोज॑नानि॒ ये ये भोज॑नानि कृणुत कृणुत॒ भोज॑नानि॒ ये । \newline
56. भोज॑नानि॒ ये ये भोज॑नानि॒ भोज॑नानि॒ ये ब॒र्॒.हिषो॑ ब॒र्॒.हिषो॒ ये भोज॑नानि॒ भोज॑नानि॒ ये ब॒र्॒.हिषः॑ । \newline
57. ये ब॒र्॒.हिषो॑ ब॒र्॒.हिषो॒ ये ये ब॒र्॒.हिषो॒ नमो॑वृक्ति॒न् नमो॑वृक्तिम् ब॒र्॒.हिषो॒ ये ये ब॒र्॒.हिषो॒ नमो॑वृक्तिम् । \newline
58. ब॒र्॒.हिषो॒ नमो॑वृक्ति॒न् नमो॑वृक्तिम् ब॒र्॒.हिषो॑ ब॒र्॒.हिषो॒ नमो॑वृक्ति॒न्न न नमो॑वृक्तिम् ब॒र्॒.हिषो॑ ब॒र्॒.हिषो॒ नमो॑वृक्ति॒न्न । \newline
59. नमो॑वृक्ति॒न्न न नमो॑वृक्ति॒न् नमो॑वृक्ति॒न्न ज॒ग्मुर् ज॒ग्मुर् न नमो॑वृक्ति॒न् नमो॑वृक्ति॒न्न ज॒ग्मुः । \newline
60. नमो॑वृक्ति॒मिति॒ नमः॑ - वृ॒क्ति॒म् । \newline
61. न ज॒ग्मुर् ज॒ग्मुर् न न ज॒ग्मुः । \newline
62. ज॒ग्मुरिति॑ ज॒ग्मुः । \newline
63. आ॒श्वि॒नम् धू॒म्रम् धू॒म्र मा᳚श्वि॒न मा᳚श्वि॒नम् धू॒म्र मा धू॒म्र मा᳚श्वि॒न मा᳚श्वि॒नम् धू॒म्र मा । \newline
64. धू॒म्र मा धू॒म्रम् धू॒म्र मा ल॑भते लभत॒ आ धू॒म्रम् धू॒म्र मा ल॑भते । \newline
65. आ ल॑भते लभत॒ आ ल॑भते सारस्व॒तꣳ सा॑रस्व॒तम् ॅल॑भत॒ आ ल॑भते सारस्व॒तम् । \newline
66. ल॒भ॒ते॒ सा॒र॒स्व॒तꣳ सा॑रस्व॒तम् ॅल॑भते लभते सारस्व॒तम् मे॒षम् मे॒षꣳ सा॑रस्व॒तम् ॅल॑भते लभते सारस्व॒तम् मे॒षम् । \newline
67. सा॒र॒स्व॒तम् मे॒षम् मे॒षꣳ सा॑रस्व॒तꣳ सा॑रस्व॒तम् मे॒ष मै॒न्द्र मै॒न्द्रम् मे॒षꣳ सा॑रस्व॒तꣳ सा॑रस्व॒तम् मे॒ष मै॒न्द्रम् । \newline
68. मे॒ष मै॒न्द्र मै॒न्द्रम् मे॒षम् मे॒ष मै॒न्द्र मृ॑ष॒भ मृ॑ष॒भ मै॒न्द्रम् मे॒षम् मे॒ष मै॒न्द्र मृ॑ष॒भम् । \newline
69. ऐ॒न्द्र मृ॑ष॒भ मृ॑ष॒भ मै॒न्द्र मै॒न्द्र मृ॑ष॒भ मै॒न्द्र मै॒न्द्र मृ॑ष॒भ मै॒न्द्र मै॒न्द्र मृ॑ष॒भ मै॒न्द्रम् । \newline
70. ऋ॒ष॒भ मै॒न्द्र मै॒न्द्र मृ॑ष॒भ मृ॑ष॒भ मै॒न्द्र मेका॑दशकपाल॒ मेका॑दशकपाल मै॒न्द्र मृ॑ष॒भ मृ॑ष॒भ मै॒न्द्र मेका॑दशकपालम् । \newline
71. ऐ॒न्द्र मेका॑दशकपाल॒ मेका॑दशकपाल मै॒न्द्र मै॒न्द्र मेका॑दशकपाल॒न्निर् णिरेका॑दशकपाल मै॒न्द्र मै॒न्द्र मेका॑दशकपाल॒न्निः । \newline
72. एका॑दशकपाल॒न्निर् णिरेका॑दशकपाल॒ मेका॑दशकपाल॒न्निर् व॑पति वपति॒ निरेका॑दशकपाल॒ मेका॑दशकपाल॒न्निर् व॑पति । \newline
73. एका॑दशकपाल॒मित्येका॑दश - क॒पा॒ल॒म् । \newline
74. निर् व॑पति वपति॒ निर् णिर् व॑पति सावि॒त्रꣳ सा॑वि॒त्रं ॅव॑पति॒ निर् णिर् व॑पति सावि॒त्रम् । \newline
75. व॒प॒ति॒ सा॒वि॒त्रꣳ सा॑वि॒त्रं ॅव॑पति वपति सावि॒त्रम् द्वाद॑शकपाल॒म् द्वाद॑शकपालꣳ सावि॒त्रं ॅव॑पति वपति सावि॒त्रम् द्वाद॑शकपालम् । \newline
76. सा॒वि॒त्रम् द्वाद॑शकपाल॒म् द्वाद॑शकपालꣳ सावि॒त्रꣳ सा॑वि॒त्रम् द्वाद॑शकपालं ॅवारु॒णं ॅवा॑रु॒णम् द्वाद॑शकपालꣳ सावि॒त्रꣳ सा॑वि॒त्रम् द्वाद॑शकपालं ॅवारु॒णम् । \newline
77. द्वाद॑शकपालं ॅवारु॒णं ॅवा॑रु॒णम् द्वाद॑शकपाल॒म् द्वाद॑शकपालं ॅवारु॒णम् दश॑कपाल॒म् दश॑कपालं ॅवारु॒णम् द्वाद॑शकपाल॒म् द्वाद॑शकपालं ॅवारु॒णम् दश॑कपालम् । \newline
78. द्वाद॑शकपाल॒मिति॒ द्वाद॑श - क॒पा॒ल॒म् । \newline
79. वा॒रु॒णम् दश॑कपाल॒म् दश॑कपालं ॅवारु॒णं ॅवा॑रु॒णम् दश॑कपालꣳ॒॒ सोम॑प्रतीकाः॒ सोम॑प्रतीका॒ दश॑कपालं ॅवारु॒णं ॅवा॑रु॒णम् दश॑कपालꣳ॒॒ सोम॑प्रतीकाः । \newline
80. दश॑कपालꣳ॒॒ सोम॑प्रतीकाः॒ सोम॑प्रतीका॒ दश॑कपाल॒म् दश॑कपालꣳ॒॒ सोम॑प्रतीकाः पितरः पितरः॒ सोम॑प्रतीका॒ दश॑कपाल॒म् दश॑कपालꣳ॒॒ सोम॑प्रतीकाः पितरः । \newline
81. दश॑कपाल॒मिति॒ दश॑ - क॒पा॒ल॒म् । \newline
82. सोम॑प्रतीकाः पितरः पितरः॒ सोम॑प्रतीकाः॒ सोम॑प्रतीकाः पितर स्तृप्णुत तृप्णुत पितरः॒ सोम॑प्रतीकाः॒ सोम॑प्रतीकाः पितर स्तृप्णुत । \newline
83. सोम॑प्रतीका॒ इति॒ सोम॑ - प्र॒ती॒काः॒ । \newline
84. पि॒त॒र॒ स्तृ॒प्णु॒त॒ तृ॒प्णु॒त॒ पि॒त॒रः॒ पि॒त॒र॒ स्तृ॒प्णु॒त॒ वड॑बा॒ वड॑बा तृप्णुत पितरः पितर स्तृप्णुत॒ वड॑बा । \newline
85. तृ॒प्णु॒त॒ वड॑बा॒ वड॑बा तृप्णुत तृप्णुत॒ वड॑बा॒ दक्षि॑णा॒ दक्षि॑णा॒ वड॑बा तृप्णुत तृप्णुत॒ वड॑बा॒ दक्षि॑णा । \newline
86. वड॑बा॒ दक्षि॑णा॒ दक्षि॑णा॒ वड॑बा॒ वड॑बा॒ दक्षि॑णा । \newline
87. दक्षि॒णेति॒ दक्षि॑णा । \newline
\pagebreak
\markright{ TS 1.8.22.1  \hfill https://www.vedavms.in \hfill}
\addcontentsline{toc}{section}{ TS 1.8.22.1 }
\section*{ TS 1.8.22.1 }

\textbf{TS 1.8.22.1 } \newline
\textbf{Samhita Paata} \newline

अग्ना॑विष्णू॒ महि॒ तद् वां᳚ महि॒त्वं ॅवी॒तं घृ॒तस्य॒ गुह्या॑नि॒ नाम॑ । दमे॑दमे स॒प्त रत्ना॒ दधा॑ना॒ प्रति॑ वां जि॒ह्वा घृ॒तमा च॑रण्येत् ॥ अग्ना॑विष्णू॒ महि॒ धाम॑ प्रि॒यं ॅवां᳚ ॅवी॒थो घृ॒तस्य॒ गुह्या॑ जुषा॒णा । दमे॑दमे सुष्टु॒तीर् वा॑वृधा॒ना प्रति॑ वां जि॒ह्वा घृ॒तमुच्च॑रण्येत् ॥ प्र णो॑ दे॒वी सर॑स्वती॒ वाजे॑भिर् वा॒जिनी॑वती । धी॒ना-म॑वि॒त्य्र॑वतु । आ नो॑ दि॒वो बृ॑ह॒तः - [ ] \newline

\textbf{Pada Paata} \newline

अग्ना॑विष्णू॒ इत्यग्ना᳚ - वि॒ष्णू॒ । महि॑ । तत् । वा॒म् । म॒हि॒त्वमिति॑ महि - त्वम् । वी॒तम् । घृ॒तस्य॑ । गुह्या॑नि । नाम॑ ॥ दमे॑दम॒ इति॒ दमे᳚-द॒मे॒ । स॒प्त । रत्ना᳚ । दधा॑ना । प्रतीति॑ । वा॒म् । जि॒ह्वा । घृ॒तम् । एति॑ । च॒र॒ण्ये॒त् ॥ अग्ना॑विष्णू॒ इत्यग्ना᳚-वि॒ष्णू॒ । महि॑ । धाम॑ । प्रि॒यम् । वा॒म् । वी॒थः । घृ॒तस्य॑ । गुह्या᳚ । जु॒षा॒णा ॥ दमे॑दम॒ इति॒ दमे᳚ - द॒मे॒ । सु॒ष्टु॒तीरिति॑ सु-स्तु॒तीः । वा॒वृ॒धा॒ना । प्रतीति॑ । वा॒म् । जि॒ह्वा । घृ॒तम् । उदिति॑ । च॒र॒ण्ये॒त् ॥ प्रेति॑ । नः॒ । दे॒वी । सर॑स्वती । वाजे॑भिः । वा॒जिनी॑व॒तीति॑ वा॒जिनी᳚ - व॒ती॒ ॥ धी॒नाम् । अ॒वि॒त्री । अ॒व॒तु॒ ॥ एति॑ । नः॒ । दि॒वः । बृ॒ह॒तः ।  \newline



\textbf{Jatai Paata} \newline

1. अग्ना॑विष्णू॒ महि॒ मह्यग्ना॑विष्णू॒ अग्ना॑विष्णू॒ महि॑ । \newline
2. अग्ना॑विष्णू॒ इत्यग्ना᳚ - वि॒ष्णू॒ । \newline
3. महि॒ तत् तन् महि॒ महि॒ तत् । \newline
4. तद् वां᳚ ॅवा॒म् तत् तद् वा᳚म् । \newline
5. वा॒म् म॒हि॒त्वम् म॑हि॒त्वं ॅवां᳚ ॅवाम् महि॒त्वम् । \newline
6. म॒हि॒त्वं ॅवी॒तं ॅवी॒तम् म॑हि॒त्वम् म॑हि॒त्वं ॅवी॒तम् । \newline
7. म॒हि॒त्वमिति॑ महि - त्वम् । \newline
8. वी॒तम् घृ॒तस्य॑ घृ॒तस्य॑ वी॒तं ॅवी॒तम् घृ॒तस्य॑ । \newline
9. घृ॒तस्य॒ गुह्या॑नि॒ गुह्या॑नि घृ॒तस्य॑ घृ॒तस्य॒ गुह्या॑नि । \newline
10. गुह्या॑नि॒ नाम॒ नाम॒ गुह्या॑नि॒ गुह्या॑नि॒ नाम॑ । \newline
11. नामेति॒ नाम॑ । \newline
12. दमे॑दमे स॒प्त स॒प्त दमे॑दमे॒ दमे॑दमे स॒प्त । \newline
13. दमे॑दम॒ इति॒ दमे᳚ - द॒मे॒ । \newline
14. स॒प्त रत्ना॒ रत्ना॑ स॒प्त स॒प्त रत्ना᳚ । \newline
15. रत्ना॒ दधा॑ना॒ दधा॑ना॒ रत्ना॒ रत्ना॒ दधा॑ना । \newline
16. दधा॑ना॒ प्रति॒ प्रति॒ दधा॑ना॒ दधा॑ना॒ प्रति॑ । \newline
17. प्रति॑ वां ॅवा॒म् प्रति॒ प्रति॑ वाम् । \newline
18. वा॒म् जि॒ह्वा जि॒ह्वा वां᳚ ॅवाम् जि॒ह्वा । \newline
19. जि॒ह्वा घृ॒तम् घृ॒तम् जि॒ह्वा जि॒ह्वा घृ॒तम् । \newline
20. घृ॒त मा घृ॒तम् घृ॒त मा । \newline
21. आ च॑रण्येच् चरण्ये॒दा च॑रण्येत् । \newline
22. च॒र॒ण्ये॒दिति॑ चरण्येत् । \newline
23. अग्ना॑विष्णू॒ महि॒ मह्यग्ना॑विष्णू॒ अग्ना॑विष्णू॒ महि॑ । \newline
24. अग्ना॑विष्णू॒ इत्यग्ना᳚ - वि॒ष्णू॒ । \newline
25. महि॒ धाम॒ धाम॒ महि॒ महि॒ धाम॑ । \newline
26. धाम॑ प्रि॒यम् प्रि॒यम् धाम॒ धाम॑ प्रि॒यम् । \newline
27. प्रि॒यं ॅवां᳚ ॅवाम् प्रि॒यम् प्रि॒यं ॅवा᳚म् । \newline
28. वां॒ ॅवी॒थो वी॒थो वां᳚ ॅवां ॅवी॒थः । \newline
29. वी॒थो घृ॒तस्य॑ घृ॒तस्य॑ वी॒थो वी॒थो घृ॒तस्य॑ । \newline
30. घृ॒तस्य॒ गुह्या॒ गुह्या॑ घृ॒तस्य॑ घृ॒तस्य॒ गुह्या᳚ । \newline
31. गुह्या॑ जुषा॒णा जु॑षा॒णा गुह्या॒ गुह्या॑ जुषा॒णा । \newline
32. जु॒षा॒णेति॑ जुषा॒णा । \newline
33. दमे॑दमे सुष्टु॒तीः सु॑ष्टु॒तीर् दमे॑दमे॒ दमे॑दमे सुष्टु॒तीः । \newline
34. दमे॑दम॒ इति॒ दमे᳚ - द॒मे॒ । \newline
35. सु॒ष्टु॒तीर् वा॑वृधा॒ना वा॑वृधा॒ना सु॑ष्टु॒तीः सु॑ष्टु॒तीर् वा॑वृधा॒ना । \newline
36. सु॒ष्टु॒तीरिति॑ सु - स्तु॒तीः । \newline
37. वा॒वृ॒धा॒ना प्रति॒ प्रति॑ वावृधा॒ना वा॑वृधा॒ना प्रति॑ । \newline
38. प्रति॑ वां ॅवा॒म् प्रति॒ प्रति॑ वाम् । \newline
39. वा॒म् जि॒ह्वा जि॒ह्वा वां᳚ ॅवाम् जि॒ह्वा । \newline
40. जि॒ह्वा घृ॒तम् घृ॒तम् जि॒ह्वा जि॒ह्वा घृ॒तम् । \newline
41. घृ॒त मुदुद् घृ॒तम् घृ॒त मुत् । \newline
42. उच् च॑रण्येच् चरण्ये॒ दुदुच् च॑रण्येत् । \newline
43. च॒र॒ण्ये॒दिति॑ चरण्येत् । \newline
44. प्र णो॑ नः॒ प्र प्र णः॑ । \newline
45. नो॒ दे॒वी दे॒वी नो॑ नो दे॒वी । \newline
46. दे॒वी सर॑स्वती॒ सर॑स्वती दे॒वी दे॒वी सर॑स्वती । \newline
47. सर॑स्वती॒ वाजे॑भि॒र् वाजे॑भिः॒ सर॑स्वती॒ सर॑स्वती॒ वाजे॑भिः । \newline
48. वाजे॑भिर् वा॒जिनी॑वती वा॒जिनी॑वती॒ वाजे॑भि॒र् वाजे॑भिर् वा॒जिनी॑वती । \newline
49. वा॒जिनी॑व॒तीति॑ वा॒जिनी᳚ - व॒ती॒ । \newline
50. धी॒ना म॑वि॒त्र्य॑वि॒त्री धी॒नाम् धी॒ना म॑वि॒त्री । \newline
51. अ॒वि॒त्र्य॑ वत्वव त्व वि॒त्र्य॑ वि॒त्र्य॑वतु । \newline
52. अ॒व॒त्वित्य॑वतु । \newline
53. आ नो॑ न॒ आ नः॑ । \newline
54. नो॒ दि॒वो दि॒वो नो॑ नो दि॒वः । \newline
55. दि॒वो बृ॑ह॒तो बृ॑ह॒तो दि॒वो दि॒वो बृ॑ह॒तः । \newline
56. बृ॒ह॒तः पर्व॑ता॒त् पर्व॑ताद् बृह॒तो बृ॑ह॒तः पर्व॑तात् । \newline

\textbf{Ghana Paata } \newline

1. अग्ना॑विष्णू॒ महि॒ मह्यग्ना॑विष्णू॒ अग्ना॑विष्णू॒ महि॒ तत् तन् मह्यग्ना॑विष्णू॒ अग्ना॑विष्णू॒ महि॒ तत् । \newline
2. अग्ना॑विष्णू॒ इत्यग्ना᳚ - वि॒ष्णू॒ । \newline
3. महि॒ तत् तन् महि॒ महि॒ तद् वां᳚ ॅवा॒म् तन् महि॒ महि॒ तद् वा᳚म् । \newline
4. तद् वां᳚ ॅवा॒म् तत् तद् वा᳚म् महि॒त्वम् म॑हि॒त्वं ॅवा॒म् तत् तद् वा᳚म् महि॒त्वम् । \newline
5. वा॒म् म॒हि॒त्वम् म॑हि॒त्वं ॅवां᳚ ॅवाम् महि॒त्वं ॅवी॒तं ॅवी॒तम् म॑हि॒त्वं ॅवां᳚ ॅवाम् महि॒त्वं ॅवी॒तम् । \newline
6. म॒हि॒त्वं ॅवी॒तं ॅवी॒तम् म॑हि॒त्वम् म॑हि॒त्वं ॅवी॒तम् घृ॒तस्य॑ घृ॒तस्य॑ वी॒तम् म॑हि॒त्वम् म॑हि॒त्वं ॅवी॒तम् घृ॒तस्य॑ । \newline
7. म॒हि॒त्वमिति॑ महि - त्वम् । \newline
8. वी॒तम् घृ॒तस्य॑ घृ॒तस्य॑ वी॒तं ॅवी॒तम् घृ॒तस्य॒ गुह्या॑नि॒ गुह्या॑नि घृ॒तस्य॑ वी॒तं ॅवी॒तम् घृ॒तस्य॒ गुह्या॑नि । \newline
9. घृ॒तस्य॒ गुह्या॑नि॒ गुह्या॑नि घृ॒तस्य॑ घृ॒तस्य॒ गुह्या॑नि॒ नाम॒ नाम॒ गुह्या॑नि घृ॒तस्य॑ घृ॒तस्य॒ गुह्या॑नि॒ नाम॑ । \newline
10. गुह्या॑नि॒ नाम॒ नाम॒ गुह्या॑नि॒ गुह्या॑नि॒ नाम॑ । \newline
11. नामेति॒ नाम॑ । \newline
12. दमे॑दमे स॒प्त स॒प्त दमे॑दमे॒ दमे॑दमे स॒प्त रत्ना॒ रत्ना॑ स॒प्त दमे॑दमे॒ दमे॑दमे स॒प्त रत्ना᳚ । \newline
13. दमे॑दम॒ इति॒ दमे᳚ - द॒मे॒ । \newline
14. स॒प्त रत्ना॒ रत्ना॑ स॒प्त स॒प्त रत्ना॒ दधा॑ना॒ दधा॑ना॒ रत्ना॑ स॒प्त स॒प्त रत्ना॒ दधा॑ना । \newline
15. रत्ना॒ दधा॑ना॒ दधा॑ना॒ रत्ना॒ रत्ना॒ दधा॑ना॒ प्रति॒ प्रति॒ दधा॑ना॒ रत्ना॒ रत्ना॒ दधा॑ना॒ प्रति॑ । \newline
16. दधा॑ना॒ प्रति॒ प्रति॒ दधा॑ना॒ दधा॑ना॒ प्रति॑ वां ॅवा॒म् प्रति॒ दधा॑ना॒ दधा॑ना॒ प्रति॑ वाम् । \newline
17. प्रति॑ वां ॅवा॒म् प्रति॒ प्रति॑ वाम् जि॒ह्वा जि॒ह्वा वा॒म् प्रति॒ प्रति॑ वाम् जि॒ह्वा । \newline
18. वा॒म् जि॒ह्वा जि॒ह्वा वां᳚ ॅवाम् जि॒ह्वा घृ॒तम् घृ॒तम् जि॒ह्वा वां᳚ ॅवाम् जि॒ह्वा घृ॒तम् । \newline
19. जि॒ह्वा घृ॒तम् घृ॒तम् जि॒ह्वा जि॒ह्वा घृ॒त मा घृ॒तम् जि॒ह्वा जि॒ह्वा घृ॒त मा । \newline
20. घृ॒त मा घृ॒तम् घृ॒त मा च॑रण्येच् चरण्ये॒दा घृ॒तम् घृ॒त मा च॑रण्येत् । \newline
21. आ च॑रण्येच् चरण्ये॒दा च॑रण्येत् । \newline
22. च॒र॒ण्ये॒दिति॑ चरण्येत् । \newline
23. अग्ना॑विष्णू॒ महि॒ मह्यग्ना॑विष्णू॒ अग्ना॑विष्णू॒ महि॒ धाम॒ धाम॒ मह्यग्ना॑विष्णू॒ अग्ना॑विष्णू॒ महि॒ धाम॑ । \newline
24. अग्ना॑विष्णू॒ इत्यग्ना᳚ - वि॒ष्णू॒ । \newline
25. महि॒ धाम॒ धाम॒ महि॒ महि॒ धाम॑ प्रि॒यम् प्रि॒यम् धाम॒ महि॒ महि॒ धाम॑ प्रि॒यम् । \newline
26. धाम॑ प्रि॒यम् प्रि॒यम् धाम॒ धाम॑ प्रि॒यं ॅवां᳚ ॅवाम् प्रि॒यम् धाम॒ धाम॑ प्रि॒यं ॅवा᳚म् । \newline
27. प्रि॒यं ॅवां᳚ ॅवाम् प्रि॒यम् प्रि॒यं ॅवां᳚ ॅवी॒थो वी॒थो वा᳚म् प्रि॒यम् प्रि॒यं ॅवां᳚ ॅवी॒थः । \newline
28. वां॒ ॅवी॒थो वी॒थो वां᳚ ॅवां ॅवी॒थो घृ॒तस्य॑ घृ॒तस्य॑ वी॒थो वां᳚ ॅवां ॅवी॒थो घृ॒तस्य॑ । \newline
29. वी॒थो घृ॒तस्य॑ घृ॒तस्य॑ वी॒थो वी॒थो घृ॒तस्य॒ गुह्या॒ गुह्या॑ घृ॒तस्य॑ वी॒थो वी॒थो घृ॒तस्य॒ गुह्या᳚ । \newline
30. घृ॒तस्य॒ गुह्या॒ गुह्या॑ घृ॒तस्य॑ घृ॒तस्य॒ गुह्या॑ जुषा॒णा जु॑षा॒णा गुह्या॑ घृ॒तस्य॑ घृ॒तस्य॒ गुह्या॑ जुषा॒णा । \newline
31. गुह्या॑ जुषा॒णा जु॑षा॒णा गुह्या॒ गुह्या॑ जुषा॒णा । \newline
32. जु॒षा॒णेति॑ जुषा॒णा । \newline
33. दमे॑दमे सुष्टु॒तीः सु॑ष्टु॒तीर् दमे॑दमे॒ दमे॑दमे सुष्टु॒तीर् वा॑वृधा॒ना वा॑वृधा॒ना सु॑ष्टु॒तीर् दमे॑दमे॒ दमे॑दमे सुष्टु॒तीर् वा॑वृधा॒ना । \newline
34. दमे॑दम॒ इति॒ दमे᳚ - द॒मे॒ । \newline
35. सु॒ष्टु॒तीर् वा॑वृधा॒ना वा॑वृधा॒ना सु॑ष्टु॒तीः सु॑ष्टु॒तीर् वा॑वृधा॒ना प्रति॒ प्रति॑ वावृधा॒ना सु॑ष्टु॒तीः सु॑ष्टु॒तीर् वा॑वृधा॒ना प्रति॑ । \newline
36. सु॒ष्टु॒तीरिति॑ सु - स्तु॒तीः । \newline
37. वा॒वृ॒धा॒ना प्रति॒ प्रति॑ वावृधा॒ना वा॑वृधा॒ना प्रति॑ वां ॅवा॒म् प्रति॑ वावृधा॒ना वा॑वृधा॒ना प्रति॑ वाम् । \newline
38. प्रति॑ वां ॅवा॒म् प्रति॒ प्रति॑ वाम् जि॒ह्वा जि॒ह्वा वा॒म् प्रति॒ प्रति॑ वाम् जि॒ह्वा । \newline
39. वा॒म् जि॒ह्वा जि॒ह्वा वां᳚ ॅवाम् जि॒ह्वा घृ॒तम् घृ॒तम् जि॒ह्वा वां᳚ ॅवाम् जि॒ह्वा घृ॒तम् । \newline
40. जि॒ह्वा घृ॒तम् घृ॒तम् जि॒ह्वा जि॒ह्वा घृ॒त मुदुद् घृ॒तम् जि॒ह्वा जि॒ह्वा घृ॒त मुत् । \newline
41. घृ॒त मुदुद् घृ॒तम् घृ॒त मुच् च॑रण्येच् चरण्ये॒दुद् घृ॒तम् घृ॒त मुच् च॑रण्येत् । \newline
42. उच् च॑रण्येच् चरण्ये॒ दुदुच् च॑रण्येत् । \newline
43. च॒र॒ण्ये॒दिति॑ चरण्येत् । \newline
44. प्र णो॑ नः॒ प्र प्र णो॑ दे॒वी दे॒वी नः॒ प्र प्र णो॑ दे॒वी । \newline
45. नो॒ दे॒वी दे॒वी नो॑ नो दे॒वी सर॑स्वती॒ सर॑स्वती दे॒वी नो॑ नो दे॒वी सर॑स्वती । \newline
46. दे॒वी सर॑स्वती॒ सर॑स्वती दे॒वी दे॒वी सर॑स्वती॒ वाजे॑भि॒र् वाजे॑भिः॒ सर॑स्वती दे॒वी दे॒वी सर॑स्वती॒ वाजे॑भिः । \newline
47. सर॑स्वती॒ वाजे॑भि॒र् वाजे॑भिः॒ सर॑स्वती॒ सर॑स्वती॒ वाजे॑भिर् वा॒जिनी॑वती वा॒जिनी॑वती॒ वाजे॑भिः॒ सर॑स्वती॒ सर॑स्वती॒ वाजे॑भिर् वा॒जिनी॑वती । \newline
48. वाजे॑भिर् वा॒जिनी॑वती वा॒जिनी॑वती॒ वाजे॑भि॒र् वाजे॑भिर् वा॒जिनी॑वती । \newline
49. वा॒जिनी॑व॒तीति॑ वा॒जिनी᳚ - व॒ती॒ । \newline
50. धी॒ना म॑वि॒त्र्य॑वि॒त्री धी॒नाम् धी॒ना म॑वि॒त्र्य॑व त्वव त्ववि॒त्री धी॒नाम् धी॒ना म॑वि॒त्र्य॑वतु । \newline
51. अ॒वि॒ त्र्य॑व त्ववत्ववि॒ त्र्य॑वि॒ त्र्य॑वतु । \newline
52. अ॒व॒त्वित्य॑वतु । \newline
53. आ नो॑ न॒ आ नो॑ दि॒वो दि॒वो न॒ आ नो॑ दि॒वः । \newline
54. नो॒ दि॒वो दि॒वो नो॑ नो दि॒वो बृ॑ह॒तो बृ॑ह॒तो दि॒वो नो॑ नो दि॒वो बृ॑ह॒तः । \newline
55. दि॒वो बृ॑ह॒तो बृ॑ह॒तो दि॒वो दि॒वो बृ॑ह॒तः पर्व॑ता॒त् पर्व॑ताद् बृह॒तो दि॒वो दि॒वो बृ॑ह॒तः पर्व॑तात् । \newline
56. बृ॒ह॒तः पर्व॑ता॒त् पर्व॑ताद् बृह॒तो बृ॑ह॒तः पर्व॑ता॒दा पर्व॑ताद् बृह॒तो बृ॑ह॒तः पर्व॑ता॒दा । \newline
\pagebreak
\markright{ TS 1.8.22.2  \hfill https://www.vedavms.in \hfill}
\addcontentsline{toc}{section}{ TS 1.8.22.2 }
\section*{ TS 1.8.22.2 }

\textbf{TS 1.8.22.2 } \newline
\textbf{Samhita Paata} \newline

पर्व॑ता॒दा सर॑स्वती यज॒ता ग॑न्तु य॒ज्ञ्ं । हवं॑ दे॒वी जु॑जुषा॒णा घृ॒ताची॑ श॒ग्मां नो॒ वाच॑मुश॒ती शृ॑णोतु ॥ बृह॑स्पते जु॒षस्व॑ नो ह॒व्यानि॑ विश्वदेव्य । रास्व॒ रत्ना॑नि दा॒शुषे᳚ ॥ ए॒वा पि॒त्रे वि॒श्वदे॑वाय॒ वृष्णे॑ य॒ज्ञिर् वि॑धेम॒ नम॑सा ह॒विर्भिः॑ । बृह॑स्पते सुप्र॒जा वी॒रव॑न्तो व॒यꣳ स्या॑म॒ पत॑यो रयी॒णां ॥ बृह॑स्पते॒ अति॒ यद॒र्यो अर्.हा᳚द् द्यु॒मद् वि॒भाति॒ क्रतु॑म॒ज्जने॑षु । यद् दी॒दय॒च्छव॑स- [ ] \newline

\textbf{Pada Paata} \newline

पर्व॑तात् । एति॑ । सर॑स्वती । य॒ज॒ता । ग॒न्तु॒ । य॒ज्ञ्म् ॥ हव᳚म् । दे॒वी । जु॒जु॒षा॒णा । घृ॒ताची᳚ । श॒ग्माम् । नः॒ । वाच᳚म् । उ॒श॒ती । शृ॒णो॒तु॒ ॥ बृह॑स्पते । जु॒षस्व॑ । नः॒ । ह॒व्यानि॑ । वि॒श्व॒दे॒व्येति॑ विश्व-दे॒व्य॒ ॥ रास्व॑ । रत्ना॑नि । दा॒शुषे᳚ ॥ ए॒वा । पि॒त्रे । वि॒श्वदे॑वा॒येति॑ वि॒श्व-दे॒वा॒य॒ । वृष्णे᳚ । य॒ज्ञिः । वि॒धे॒म॒ । नम॑सा । ह॒विर्भि॒रिति॑ ह॒विः - भिः॒ ॥ बृह॑स्पते । सु॒प्र॒जा इति॑ सु - प्र॒जाः । वी॒रव॑न्त॒ इति॑ वी॒र - व॒न्तः॒ । व॒यम् । स्या॒म॒ । पत॑यः । र॒यी॒णाम् ॥ बृह॑स्पते । अतीति॑ । यत् । अ॒र्यः । अर्.हा᳚त् । द्यु॒मदिति॑ द्यु - मत् । वि॒भातीति॑ वि - भाति॑ । क्रतु॑म॒दिति॒ क्रतु॑ - म॒त् । जने॑षु ॥ यत् । दी॒दय॑त् । शव॑सा ।  \newline



\textbf{Jatai Paata} \newline

1. पर्व॑ता॒दा पर्व॑ता॒त् पर्व॑ता॒दा । \newline
2. आ सर॑स्वती॒ सर॑स्व॒त्या सर॑स्वती । \newline
3. सर॑स्वती यज॒ता य॑ज॒ता सर॑स्वती॒ सर॑स्वती यज॒ता । \newline
4. य॒ज॒ता ग॑न्तु गन्तु यज॒ता य॑ज॒ता ग॑न्तु । \newline
5. ग॒न्तु॒ य॒ज्ञ्ं ॅय॒ज्ञ्म् ग॑न्तु गन्तु य॒ज्ञ्म् । \newline
6. य॒ज्ञ्मिति॑ य॒ज्ञ्म् । \newline
7. हव॑म् दे॒वी दे॒वी हवꣳ॒॒ हव॑म् दे॒वी । \newline
8. दे॒वी जु॑जुषा॒णा जु॑जुषा॒णा दे॒वी दे॒वी जु॑जुषा॒णा । \newline
9. जु॒जु॒षा॒णा घृ॒ताची॑ घृ॒ताची॑ जुजुषा॒णा जु॑जुषा॒णा घृ॒ताची᳚ । \newline
10. घृ॒ताची॑ श॒ग्माꣳ श॒ग्माम् घृ॒ताची॑ घृ॒ताची॑ श॒ग्माम् । \newline
11. श॒ग्माम् नो॑ नः श॒ग्माꣳ श॒ग्माम् नः॑ । \newline
12. नो॒ वाचं॒ ॅवाच॑न्नो नो॒ वाच᳚म् । \newline
13. वाच॑ मुश॒ त्यु॑श॒ती वाचं॒ ॅवाच॑ मुश॒ती । \newline
14. उ॒श॒ती शृ॑णोतु शृणोतू श॒त्यु॑श॒ती शृ॑णोतु । \newline
15. शृ॒णो॒त्विति॑ शृणोतु । \newline
16. बृह॑स्पते जु॒षस्व॑ जु॒षस्व॒ बृह॑स्पते॒ बृह॑स्पते जु॒षस्व॑ । \newline
17. जु॒षस्व॑ नो नो जु॒षस्व॑ जु॒षस्व॑ नः । \newline
18. नो॒ ह॒व्यानि॑ ह॒व्यानि॑ नो नो ह॒व्यानि॑ । \newline
19. ह॒व्यानि॑ विश्वदेव्य विश्वदेव्य ह॒व्यानि॑ ह॒व्यानि॑ विश्वदेव्य । \newline
20. वि॒श्व॒दे॒व्येति॑ विश्व - दे॒व्य॒ । \newline
21. रास्व॒ रत्ना॑नि॒ रत्ना॑नि॒ रास्व॒ रास्व॒ रत्ना॑नि । \newline
22. रत्ना॑नि दा॒शुषे॑ दा॒शुषे॒ रत्ना॑नि॒ रत्ना॑नि दा॒शुषे᳚ । \newline
23. दा॒शुष॒ इति॑ दा॒शुषे᳚ । \newline
24. ए॒वा पि॒त्रे पि॒त्र ए॒वैवा पि॒त्रे । \newline
25. पि॒त्रे वि॒श्वदे॑वाय वि॒श्वदे॑वाय पि॒त्रे पि॒त्रे वि॒श्वदे॑वाय । \newline
26. वि॒श्वदे॑वाय॒ वृष्णे॒ वृष्णे॑ वि॒श्वदे॑वाय वि॒श्वदे॑वाय॒ वृष्णे᳚ । \newline
27. वि॒श्वदे॑वा॒येति॑ वि॒श्व - दे॒वा॒य॒ । \newline
28. वृष्णे॑ य॒ज्ञिर् य॒ज्ञिर् वृष्णे॒ वृष्णे॑ य॒ज्ञिः । \newline
29. य॒ज्ञिर् वि॑धेम विधेम य॒ज्ञिर् य॒ज्ञिर् वि॑धेम । \newline
30. वि॒धे॒म॒ नम॑सा॒ नम॑सा विधेम विधेम॒ नम॑सा । \newline
31. नम॑सा ह॒विर्भि॑र्. ह॒विर्भि॒र् नम॑सा॒ नम॑सा ह॒विर्भिः॑ । \newline
32. ह॒विर्भि॒रिति॑ ह॒विः - भिः॒ । \newline
33. बृह॑स्पते सुप्र॒जाः सु॑प्र॒जा बृह॑स्पते॒ बृह॑स्पते सुप्र॒जाः । \newline
34. सु॒प्र॒जा वी॒रव॑न्तो वी॒रव॑न्तः सुप्र॒जाः सु॑प्र॒जा वी॒रव॑न्तः । \newline
35. सु॒प्र॒जा इति॑ सु - प्र॒जाः । \newline
36. वी॒रव॑न्तो व॒यं ॅव॒यं ॅवी॒रव॑न्तो वी॒रव॑न्तो व॒यम् । \newline
37. वी॒रव॑न्त॒ इति॑ वी॒र - व॒न्तः॒ । \newline
38. व॒यꣳ स्या॑म स्याम व॒यं ॅव॒यꣳ स्या॑म । \newline
39. स्या॒म॒ पत॑यः॒ पत॑यः स्याम स्याम॒ पत॑यः । \newline
40. पत॑यो रयी॒णाꣳ र॑यी॒णाम् पत॑यः॒ पत॑यो रयी॒णाम् । \newline
41. र॒यी॒णामिति॑ रयी॒णाम् । \newline
42. बृह॑स्पते॒ अत्यति॒ बृह॑स्पते॒ बृह॑स्पते॒ अति॑ । \newline
43. अति॒ यद् यदत्यति॒ यत् । \newline
44. यद॒र्यो अ॒र्यो यद् यद॒र्यः । \newline
45. अ॒र्यो अर्.हा॒ दर्.हा॑द॒र्यो अ॒र्यो अर्.हा᳚त् । \newline
46. अर्.हा᳚द् द्यु॒मद् द्यु॒म दर्.हा॒ दर्.हा᳚द् द्यु॒मत् । \newline
47. द्यु॒मद् वि॒भाति॑ वि॒भाति॑ द्यु॒मद् द्यु॒मद् वि॒भाति॑ । \newline
48. द्यु॒मदिति॑ द्यु - मत् । \newline
49. वि॒भाति॒ क्रतु॑म॒त् क्रतु॑मद् वि॒भाति॑ वि॒भाति॒ क्रतु॑मत् । \newline
50. वि॒भातीति॑ वि - भाति॑ । \newline
51. क्रतु॑म॒ज् जने॑षु॒ जने॑षु॒ क्रतु॑म॒त् क्रतु॑म॒ज् जने॑षु । \newline
52. क्रतु॑म॒दिति॒ क्रतु॑ - म॒त् । \newline
53. जने॒ष्विति॒ जने॑षु । \newline
54. यद् दी॒दय॑द् दी॒दय॒द् यद् यद् दी॒दय॑त् । \newline
55. दी॒दय॒च् छव॑सा॒ शव॑सा दी॒दय॑द् दी॒दय॒च् छव॑सा । \newline
56. शव॑सर्तप्रजात र्तप्रजात॒ शव॑सा॒ शव॑सर्तप्रजात । \newline

\textbf{Ghana Paata } \newline

1. पर्व॑ता॒दा पर्व॑ता॒त् पर्व॑ता॒दा सर॑स्वती॒ सर॑स्व॒त्या पर्व॑ता॒त् पर्व॑ता॒दा सर॑स्वती । \newline
2. आ सर॑स्वती॒ सर॑स्व॒त्या सर॑स्वती यज॒ता य॑ज॒ता सर॑स्व॒त्या सर॑स्वती यज॒ता । \newline
3. सर॑स्वती यज॒ता य॑ज॒ता सर॑स्वती॒ सर॑स्वती यज॒ता ग॑न्तु गन्तु यज॒ता सर॑स्वती॒ सर॑स्वती यज॒ता ग॑न्तु । \newline
4. य॒ज॒ता ग॑न्तु गन्तु यज॒ता य॑ज॒ता ग॑न्तु य॒ज्ञ्ं ॅय॒ज्ञ्म् ग॑न्तु यज॒ता य॑ज॒ता ग॑न्तु य॒ज्ञ्म् । \newline
5. ग॒न्तु॒ य॒ज्ञ्ं ॅय॒ज्ञ्म् ग॑न्तु गन्तु य॒ज्ञ्म् । \newline
6. य॒ज्ञ्मिति॑ य॒ज्ञ्म् । \newline
7. हव॑म् दे॒वी दे॒वी हवꣳ॒॒ हव॑म् दे॒वी जु॑जुषा॒णा जु॑जुषा॒णा दे॒वी हवꣳ॒॒ हव॑म् दे॒वी जु॑जुषा॒णा । \newline
8. दे॒वी जु॑जुषा॒णा जु॑जुषा॒णा दे॒वी दे॒वी जु॑जुषा॒णा घृ॒ताची॑ घृ॒ताची॑ जुजुषा॒णा दे॒वी दे॒वी जु॑जुषा॒णा घृ॒ताची᳚ । \newline
9. जु॒जु॒षा॒णा घृ॒ताची॑ घृ॒ताची॑ जुजुषा॒णा जु॑जुषा॒णा घृ॒ताची॑ श॒ग्माꣳ श॒ग्माम् घृ॒ताची॑ जुजुषा॒णा जु॑जुषा॒णा घृ॒ताची॑ श॒ग्माम् । \newline
10. घृ॒ताची॑ श॒ग्माꣳ श॒ग्माम् घृ॒ताची॑ घृ॒ताची॑ श॒ग्मान्नो॑ नः श॒ग्माम् घृ॒ताची॑ घृ॒ताची॑ श॒ग्मान्नः॑ । \newline
11. श॒ग्मान्नो॑ नः श॒ग्माꣳ श॒ग्मान्नो॒ वाचं॒ ॅवाच॑न्नः श॒ग्माꣳ श॒ग्मान्नो॒ वाच᳚म् । \newline
12. नो॒ वाचं॒ ॅवाच॑न्नो नो॒ वाच॑ मुश॒ त्यु॑श॒ती वाच॑न्नो नो॒ वाच॑ मुश॒ती । \newline
13. वाच॑ मुश॒ त्यु॑श॒ती वाचं॒ ॅवाच॑ मुश॒ती शृ॑णोतु शृणो तूश॒ती वाचं॒ ॅवाच॑ मुश॒ती शृ॑णोतु । \newline
14. उ॒श॒ती शृ॑णोतु शृणोतू श॒त्यु॑श॒ती शृ॑णोतु । \newline
15. शृ॒णो॒त्विति॑ शृणोतु । \newline
16. बृह॑स्पते जु॒षस्व॑ जु॒षस्व॒ बृह॑स्पते॒ बृह॑स्पते जु॒षस्व॑ नो नो जु॒षस्व॒ बृह॑स्पते॒ बृह॑स्पते जु॒षस्व॑ नः । \newline
17. जु॒षस्व॑ नो नो जु॒षस्व॑ जु॒षस्व॑ नो ह॒व्यानि॑ ह॒व्यानि॑ नो जु॒षस्व॑ जु॒षस्व॑ नो ह॒व्यानि॑ । \newline
18. नो॒ ह॒व्यानि॑ ह॒व्यानि॑ नो नो ह॒व्यानि॑ विश्वदेव्य विश्वदेव्य ह॒व्यानि॑ नो नो ह॒व्यानि॑ विश्वदेव्य । \newline
19. ह॒व्यानि॑ विश्वदेव्य विश्वदेव्य ह॒व्यानि॑ ह॒व्यानि॑ विश्वदेव्य । \newline
20. वि॒श्व॒दे॒व्येति॑ विश्व - दे॒व्य॒ । \newline
21. रास्व॒ रत्ना॑नि॒ रत्ना॑नि॒ रास्व॒ रास्व॒ रत्ना॑नि दा॒शुषे॑ दा॒शुषे॒ रत्ना॑नि॒ रास्व॒ रास्व॒ रत्ना॑नि दा॒शुषे᳚ । \newline
22. रत्ना॑नि दा॒शुषे॑ दा॒शुषे॒ रत्ना॑नि॒ रत्ना॑नि दा॒शुषे᳚ । \newline
23. दा॒शुष॒ इति॑ दा॒शुषे᳚ । \newline
24. ए॒वा पि॒त्रे पि॒त्र ए॒वैवा पि॒त्रे वि॒श्वदे॑वाय वि॒श्वदे॑वाय पि॒त्र ए॒वैवा पि॒त्रे वि॒श्वदे॑वाय । \newline
25. पि॒त्रे वि॒श्वदे॑वाय वि॒श्वदे॑वाय पि॒त्रे पि॒त्रे वि॒श्वदे॑वाय॒ वृष्णे॒ वृष्णे॑ वि॒श्वदे॑वाय पि॒त्रे पि॒त्रे वि॒श्वदे॑वाय॒ वृष्णे᳚ । \newline
26. वि॒श्वदे॑वाय॒ वृष्णे॒ वृष्णे॑ वि॒श्वदे॑वाय वि॒श्वदे॑वाय॒ वृष्णे॑ य॒ज्ञिर् य॒ज्ञिर् वृष्णे॑ वि॒श्वदे॑वाय वि॒श्वदे॑वाय॒ वृष्णे॑ य॒ज्ञिः । \newline
27. वि॒श्वदे॑वा॒येति॑ वि॒श्व - दे॒वा॒य॒ । \newline
28. वृष्णे॑ य॒ज्ञिर् य॒ज्ञिर् वृष्णे॒ वृष्णे॑ य॒ज्ञिर् वि॑धेम विधेम य॒ज्ञिर् वृष्णे॒ वृष्णे॑ य॒ज्ञिर् वि॑धेम । \newline
29. य॒ज्ञिर् वि॑धेम विधेम य॒ज्ञिर् य॒ज्ञिर् वि॑धेम॒ नम॑सा॒ नम॑सा विधेम य॒ज्ञिर् य॒ज्ञिर् वि॑धेम॒ नम॑सा । \newline
30. वि॒धे॒म॒ नम॑सा॒ नम॑सा विधेम विधेम॒ नम॑सा ह॒विर्भि॑र्. ह॒विर्भि॒र् नम॑सा विधेम विधेम॒ नम॑सा ह॒विर्भिः॑ । \newline
31. नम॑सा ह॒विर्भि॑र्. ह॒विर्भि॒र् नम॑सा॒ नम॑सा ह॒विर्भिः॑ । \newline
32. ह॒विर्भि॒रिति॑ ह॒विः - भिः॒ । \newline
33. बृह॑स्पते सुप्र॒जाः सु॑प्र॒जा बृह॑स्पते॒ बृह॑स्पते सुप्र॒जा वी॒रव॑न्तो वी॒रव॑न्तः सुप्र॒जा बृह॑स्पते॒ बृह॑स्पते सुप्र॒जा वी॒रव॑न्तः । \newline
34. सु॒प्र॒जा वी॒रव॑न्तो वी॒रव॑न्तः सुप्र॒जाः सु॑प्र॒जा वी॒रव॑न्तो व॒यं ॅव॒यं ॅवी॒रव॑न्तः सुप्र॒जाः सु॑प्र॒जा वी॒रव॑न्तो व॒यम् । \newline
35. सु॒प्र॒जा इति॑ सु - प्र॒जाः । \newline
36. वी॒रव॑न्तो व॒यं ॅव॒यं ॅवी॒रव॑न्तो वी॒रव॑न्तो व॒यꣳ स्या॑म स्याम व॒यं ॅवी॒रव॑न्तो वी॒रव॑न्तो व॒यꣳ स्या॑म । \newline
37. वी॒रव॑न्त॒ इति॑ वी॒र - व॒न्तः॒ । \newline
38. व॒यꣳ स्या॑म स्याम व॒यं ॅव॒यꣳ स्या॑म॒ पत॑यः॒ पत॑यः स्याम व॒यं ॅव॒यꣳ स्या॑म॒ पत॑यः । \newline
39. स्या॒म॒ पत॑यः॒ पत॑यः स्याम स्याम॒ पत॑यो रयी॒णाꣳ र॑यी॒णाम् पत॑यः स्याम स्याम॒ पत॑यो रयी॒णाम् । \newline
40. पत॑यो रयी॒णाꣳ र॑यी॒णाम् पत॑यः॒ पत॑यो रयी॒णाम् । \newline
41. र॒यी॒णामिति॑ रयी॒णाम् । \newline
42. बृह॑स्पते॒ अत्यति॒ बृह॑स्पते॒ बृह॑स्पते॒ अति॒ यद् यदति॒ बृह॑स्पते॒ बृह॑स्पते॒ अति॒ यत् । \newline
43. अति॒ यद् यदत्यति॒ यद॒र्यो अ॒र्यो यदत्यति॒ यद॒र्यः । \newline
44. यद॒र्यो अ॒र्यो यद् यद॒र्यो अर्.हा॒ दर्.हा॑ द॒र्यो यद् यद॒र्यो अर्.हा᳚त् । \newline
45. अ॒र्यो अर्.हा॒ दर्.हा॑ द॒र्यो अ॒र्यो अर्.हा᳚द् द्यु॒मद् द्यु॒म दर्.हा॑ द॒र्यो अ॒र्यो अर्.हा᳚द् द्यु॒मत् । \newline
46. अर्.हा᳚द् द्यु॒मद् द्यु॒म दर्.हा॒ दर्.हा᳚द् द्यु॒मद् वि॒भाति॑ वि॒भाति॑ द्यु॒म दर्.हा॒ दर्.हा᳚द् द्यु॒मद् वि॒भाति॑ । \newline
47. द्यु॒मद् वि॒भाति॑ वि॒भाति॑ द्यु॒मद् द्यु॒मद् वि॒भाति॒ क्रतु॑म॒त् क्रतु॑मद् वि॒भाति॑ द्यु॒मद् द्यु॒मद् वि॒भाति॒ क्रतु॑मत् । \newline
48. द्यु॒मदिति॑ द्यु - मत् । \newline
49. वि॒भाति॒ क्रतु॑म॒त् क्रतु॑मद् वि॒भाति॑ वि॒भाति॒ क्रतु॑म॒ज् जने॑षु॒ जने॑षु॒ क्रतु॑मद् वि॒भाति॑ वि॒भाति॒ क्रतु॑म॒ज् जने॑षु । \newline
50. वि॒भातीति॑ वि - भाति॑ । \newline
51. क्रतु॑म॒ज् जने॑षु॒ जने॑षु॒ क्रतु॑म॒त् क्रतु॑म॒ज् जने॑षु । \newline
52. क्रतु॑म॒दिति॒ क्रतु॑ - म॒त् । \newline
53. जने॒ष्विति॒ जने॑षु । \newline
54. यद् दी॒दय॑द् दी॒दय॒द् यद् यद् दी॒दय॒च् छव॑सा॒ शव॑सा दी॒दय॒द् यद् यद् दी॒दय॒च् छव॑सा । \newline
55. दी॒दय॒च् छव॑सा॒ शव॑सा दी॒दय॑द् दी॒दय॒च् छव॑सर्तप्रजात र्तप्रजात॒ शव॑सा दी॒दय॑द् दी॒दय॒च् छव॑सर्तप्रजात । \newline
56. शव॑सर्तप्रजात र्तप्रजात॒ शव॑सा॒ शव॑सर्तप्रजात॒ तत् तदृ॑तप्रजात॒ शव॑सा॒ शव॑सर्तप्रजात॒ तत् । \newline
\pagebreak
\markright{ TS 1.8.22.3  \hfill https://www.vedavms.in \hfill}
\addcontentsline{toc}{section}{ TS 1.8.22.3 }
\section*{ TS 1.8.22.3 }

\textbf{TS 1.8.22.3 } \newline
\textbf{Samhita Paata} \newline

र्तप्रजात॒ तद॒स्मासु॒ द्रवि॑णं धेहि चि॒त्रं ॥ आ नो॑ मित्रावरुणा घृ॒तैर् गव्यू॑तिमुक्षतं । मद्ध्वा॒ रजाꣳ॑सि सुक्रतू ॥ प्र बा॒हवा॑ सिसृतं जी॒वसे॑ न॒ आ नो॒ गव्यू॑ति-मुक्षतं घृ॒तेन॑ । आ नो॒ जने᳚ श्रवयतं ॅयुवाना श्रु॒तं मे॑ मित्रावरुणा॒ हवे॒मा ॥ अ॒ग्निं ॅवः॑ पू॒र्व्यं गि॒रा दे॒वमी॑डे॒ वसू॑नां । स॒प॒र्यन्तः॑ पुरुप्रि॒यं मि॒त्रं न क्षे᳚त्र॒साध॑सं ॥ म॒क्षू दे॒वव॑तो॒ रथः॒ - [ ] \newline

\textbf{Pada Paata} \newline

ऋ॒त॒प्र॒जा॒तेत्यृ॑त-प्र॒जा॒त॒ । तत् । अ॒स्मासु॑ । द्रवि॑णम् । धे॒हि॒ । चि॒त्रम् ॥ एति॑ । नः॒ । मि॒त्रा॒व॒रु॒णेति॑ मित्रा-व॒रु॒णा॒ । घृ॒तैः । गव्यू॑तिम् । उ॒क्ष॒त॒म् ॥ मद्ध्वा᳚ । रजाꣳ॑सि । सु॒क्र॒तू॒ इति॑ सु-क्र॒तू॒ ॥ प्रेति॑ । बा॒हवा᳚ । सि॒सृ॒त॒म् । जी॒वसे᳚ । नः॒ । एति॑ । नः॒ । गव्यू॑तिम् । उ॒क्ष॒त॒म् । घृ॒तेन॑ ॥ एति॑ । नः॒ । जने᳚ । श्र॒व॒य॒त॒म् । यु॒वा॒ना॒ । श्रु॒तम् । मे॒ । मि॒त्रा॒व॒रु॒णेति॑ मित्रा - व॒रु॒णा॒ । हवा᳚ । इ॒मा ॥ अ॒ग्निम् । वः॒ । पू॒र्व्यम् । गि॒रा । दे॒वम् । ई॒डे॒ । वसू॑नाम् ॥ स॒प॒र्यन्तः॑ । पु॒रु॒प्रि॒यमिति॑ पुरु-प्रि॒यम् । मि॒त्रम् । न । क्षे॒त्र॒साध॑स॒मिति॑ क्षेत्र - साध॑सम् ॥ म॒क्षु । दे॒वव॑त॒ इति॑ दे॒व - व॒तः॒ । रथः॑ ।  \newline



\textbf{Jatai Paata} \newline

1. ऋ॒त॒प्र॒जा॒त॒ तत् तदृ॑तप्रजात र्तप्रजात॒ तत् । \newline
2. ऋ॒त॒प्र॒जा॒तेत्यृ॑त - प्र॒जा॒त॒ । \newline
3. तद॒स्मा स्व॒स्मासु॒ तत् तद॒स्मासु॑ । \newline
4. अ॒स्मासु॒ द्रवि॑ण॒म् द्रवि॑ण म॒स्मा स्व॒स्मासु॒ द्रवि॑णम् । \newline
5. द्रवि॑णम् धेहि धेहि॒ द्रवि॑ण॒म् द्रवि॑णम् धेहि । \newline
6. धे॒हि॒ चि॒त्रम् चि॒त्रम् धे॑हि धेहि चि॒त्रम् । \newline
7. चि॒त्रमिति॑ चि॒त्रम् । \newline
8. आ नो॑ न॒ आ नः॑ । \newline
9. नो॒ मि॒त्रा॒व॒रु॒णा॒ मि॒त्रा॒व॒रु॒णा॒ नो॒ नो॒ मि॒त्रा॒व॒रु॒णा॒ । \newline
10. मि॒त्रा॒व॒रु॒णा॒ घृ॒तैर् घृ॒तैर् मि॑त्रावरुणा मित्रावरुणा घृ॒तैः । \newline
11. मि॒त्रा॒व॒रु॒णेति॑ मित्रा - व॒रु॒णा॒ । \newline
12. घृ॒तैर् गव्यू॑ति॒म् गव्यू॑तिम् घृ॒तैर् घृ॒तैर् गव्यू॑तिम् । \newline
13. गव्यू॑ति मुक्षत मुक्षत॒म् गव्यू॑ति॒म् गव्यू॑ति मुक्षतम् । \newline
14. उ॒क्ष॒त॒मित्यु॑क्षतम् । \newline
15. मद्ध्वा॒ रजाꣳ॑सि॒ रजाꣳ॑सि॒ मद्ध्वा॒ मद्ध्वा॒ रजाꣳ॑सि । \newline
16. रजाꣳ॑सि सुक्रतू सुक्रतू॒ रजाꣳ॑सि॒ रजाꣳ॑सि सुक्रतू । \newline
17. सु॒क्र॒तू॒ इति॑ सु - क्र॒तू॒ । \newline
18. प्र बा॒हवा॑ बा॒हवा॒ प्र प्र बा॒हवा᳚ । \newline
19. बा॒हवा॑ सिसृतꣳ सिसृतम् बा॒हवा॑ बा॒हवा॑ सिसृतम् । \newline
20. सि॒सृ॒त॒म् जी॒वसे॑ जी॒वसे॑ सिसृतꣳ सिसृतम् जी॒वसे᳚ । \newline
21. जी॒वसे॑ नो नो जी॒वसे॑ जी॒वसे॑ नः । \newline
22. न॒ आ नो॑ न॒ आ । \newline
23. आ नो॑ न॒ आ नः॑ । \newline
24. नो॒ गव्यू॑ति॒म् गव्यू॑तिम् नो नो॒ गव्यू॑तिम् । \newline
25. गव्यू॑ति मुक्षत मुक्षत॒म् गव्यू॑ति॒म् गव्यू॑ति मुक्षतम् । \newline
26. उ॒क्ष॒त॒म् घृ॒तेन॑ घृ॒तेनो᳚क्षत मुक्षतम् घृ॒तेन॑ । \newline
27. घृ॒तेनेति॑ घृ॒तेन॑ । \newline
28. आ नो॑ न॒ आ नः॑ । \newline
29. नो॒ जने॒ जने॑ नो नो॒ जने᳚ । \newline
30. जने᳚ श्रवयतꣳ श्रवयत॒म् जने॒ जने᳚ श्रवयतम् । \newline
31. श्र॒व॒य॒तं॒ ॅयु॒वा॒ना॒ यु॒वा॒ना॒ श्र॒व॒य॒तꣳ॒॒ श्र॒व॒य॒तं॒ ॅयु॒वा॒ना॒ । \newline
32. यु॒वा॒ना॒ श्रु॒तꣳ श्रु॒तं ॅयु॑वाना युवाना श्रु॒तम् । \newline
33. श्रु॒तम् मे॑ मे श्रु॒तꣳ श्रु॒तम् मे᳚ । \newline
34. मे॒ मि॒त्रा॒व॒रु॒णा॒ मि॒त्रा॒व॒रु॒णा॒ मे॒ मे॒ मि॒त्रा॒व॒रु॒णा॒ । \newline
35. मि॒त्रा॒व॒रु॒णा॒ हवा॒ हवा॑ मित्रावरुणा मित्रावरुणा॒ हवा᳚ । \newline
36. मि॒त्रा॒व॒रु॒णेति॑ मित्रा - व॒रु॒णा॒ । \newline
37. हवे॒मेमा हवा॒ हवे॒मा । \newline
38. इ॒मेती॒मा । \newline
39. अ॒ग्निं ॅवो॑ वो अ॒ग्नि म॒ग्निं ॅवः॑ । \newline
40. वः॒ पू॒र्व्यम् पू॒र्व्यं ॅवो॑ वः पू॒र्व्यम् । \newline
41. पू॒र्व्यम् गि॒रा गि॒रा पू॒र्व्यम् पू॒र्व्यम् गि॒रा । \newline
42. गि॒रा दे॒वम् दे॒वम् गि॒रा गि॒रा दे॒वम् । \newline
43. दे॒व मी॑ड ईडे दे॒वम् दे॒व मी॑डे । \newline
44. ई॒डे॒ वसू॑नां॒ ॅवसू॑ना मीड ईडे॒ वसू॑नाम् । \newline
45. वसू॑ना॒मिति॒ वसू॑नाम् । \newline
46. स॒प॒र्यन्तः॑ पुरुप्रि॒यम् पु॑रुप्रि॒यꣳ स॑प॒र्यन्तः॑ सप॒र्यन्तः॑ पुरुप्रि॒यम् । \newline
47. पु॒रु॒प्रि॒यम् मि॒त्रम् मि॒त्रम् पु॑रुप्रि॒यम् पु॑रुप्रि॒यम् मि॒त्रम् । \newline
48. पु॒रु॒प्रि॒यमिति॑ पुरु - प्रि॒यम् । \newline
49. मि॒त्रम् न न मि॒त्रम् मि॒त्रम् न । \newline
50. न क्षे᳚त्र॒साध॑सम् क्षेत्र॒साध॑स॒म् न न क्षे᳚त्र॒साध॑सम् । \newline
51. क्षे॒त्र॒साध॑स॒मिति॑ क्षेत्र - साध॑सम् । \newline
52. म॒क्षू दे॒वव॑तो दे॒वव॑तो म॒क्षु म॒क्षू दे॒वव॑तः । \newline
53. दे॒वव॑तो॒ रथो॒ रथो॑ दे॒वव॑तो दे॒वव॑तो॒ रथः॑ । \newline
54. दे॒वव॑त॒ इति॑ दे॒व - व॒तः॒ । \newline
55. रथः॒ शूरः॒ शूरो॒ रथो॒ रथः॒ शूरः॑ । \newline

\textbf{Ghana Paata } \newline

1. ऋ॒त॒प्र॒जा॒त॒ तत् तदृ॑तप्रजात र्तप्रजात॒ तद॒स्मा स्व॒स्मासु॒ तदृ॑तप्रजात र्तप्रजात॒ तद॒स्मासु॑ । \newline
2. ऋ॒त॒प्र॒जा॒तेत्यृ॑त - प्र॒जा॒त॒ । \newline
3. तद॒स्मा स्व॒स्मासु॒ तत् तद॒स्मासु॒ द्रवि॑ण॒म् द्रवि॑ण म॒स्मासु॒ तत् तद॒स्मासु॒ द्रवि॑णम् । \newline
4. अ॒स्मासु॒ द्रवि॑ण॒म् द्रवि॑ण म॒स्मा स्व॒स्मासु॒ द्रवि॑णम् धेहि धेहि॒ द्रवि॑ण म॒स्मा स्व॒स्मासु॒ द्रवि॑णम् धेहि । \newline
5. द्रवि॑णम् धेहि धेहि॒ द्रवि॑ण॒म् द्रवि॑णम् धेहि चि॒त्रम् चि॒त्रम् धे॑हि॒ द्रवि॑ण॒म् द्रवि॑णम् धेहि चि॒त्रम् । \newline
6. धे॒हि॒ चि॒त्रम् चि॒त्रम् धे॑हि धेहि चि॒त्रम् । \newline
7. चि॒त्रमिति॑ चि॒त्रम् । \newline
8. आ नो॑ न॒ आ नो॑ मित्रावरुणा मित्रावरुणा न॒ आ नो॑ मित्रावरुणा । \newline
9. नो॒ मि॒त्रा॒व॒रु॒णा॒ मि॒त्रा॒व॒रु॒णा॒ नो॒ नो॒ मि॒त्रा॒व॒रु॒णा॒ घृ॒तैर् घृ॒तैर् मि॑त्रावरुणा नो नो मित्रावरुणा घृ॒तैः । \newline
10. मि॒त्रा॒व॒रु॒णा॒ घृ॒तैर् घृ॒तैर् मि॑त्रावरुणा मित्रावरुणा घृ॒तैर् गव्यू॑ति॒म् गव्यू॑तिम् घृ॒तैर् मि॑त्रावरुणा मित्रावरुणा घृ॒तैर् गव्यू॑तिम् । \newline
11. मि॒त्रा॒व॒रु॒णेति॑ मित्रा - व॒रु॒णा॒ । \newline
12. घृ॒तैर् गव्यू॑ति॒म् गव्यू॑तिम् घृ॒तैर् घृ॒तैर् गव्यू॑ति मुक्षत मुक्षत॒म् गव्यू॑तिम् घृ॒तैर् घृ॒तैर् गव्यू॑ति मुक्षतम् । \newline
13. गव्यू॑ति मुक्षत मुक्षत॒म् गव्यू॑ति॒म् गव्यू॑ति मुक्षतम् । \newline
14. उ॒क्ष॒त॒मित्यु॑क्षतम् । \newline
15. मद्ध्वा॒ रजाꣳ॑सि॒ रजाꣳ॑सि॒ मद्ध्वा॒ मद्ध्वा॒ रजाꣳ॑सि सुक्रतू सुक्रतू॒ रजाꣳ॑सि॒ मद्ध्वा॒ मद्ध्वा॒ रजाꣳ॑सि सुक्रतू । \newline
16. रजाꣳ॑सि सुक्रतू सुक्रतू॒ रजाꣳ॑सि॒ रजाꣳ॑सि सुक्रतू । \newline
17. सु॒क्र॒तू॒ इति॑ सु - क्र॒तू॒ । \newline
18. प्र बा॒हवा॑ बा॒हवा॒ प्र प्र बा॒हवा॑ सिसृतꣳ सिसृतम् बा॒हवा॒ प्र प्र बा॒हवा॑ सिसृतम् । \newline
19. बा॒हवा॑ सिसृतꣳ सिसृतम् बा॒हवा॑ बा॒हवा॑ सिसृतम् जी॒वसे॑ जी॒वसे॑ सिसृतम् बा॒हवा॑ बा॒हवा॑ सिसृतम् जी॒वसे᳚ । \newline
20. सि॒सृ॒त॒म् जी॒वसे॑ जी॒वसे॑ सिसृतꣳ सिसृतम् जी॒वसे॑ नो नो जी॒वसे॑ सिसृतꣳ सिसृतम् जी॒वसे॑ नः । \newline
21. जी॒वसे॑ नो नो जी॒वसे॑ जी॒वसे॑ न॒ आ नो॑ जी॒वसे॑ जी॒वसे॑ न॒ आ । \newline
22. न॒ आ नो॑ न॒ आ नो॑ न॒ आ नो॑ न॒ आ नः॑ । \newline
23. आ नो॑ न॒ आ नो॒ गव्यू॑ति॒म् गव्यू॑तिन्न॒ आ नो॒ गव्यू॑तिम् । \newline
24. नो॒ गव्यू॑ति॒म् गव्यू॑तिन्नो नो॒ गव्यू॑ति मुक्षत मुक्षत॒म् गव्यू॑तिन्नो नो॒ गव्यू॑ति मुक्षतम् । \newline
25. गव्यू॑ति मुक्षत मुक्षत॒म् गव्यू॑ति॒म् गव्यू॑ति मुक्षतम् घृ॒तेन॑ घृ॒तेनो᳚क्षत॒म् गव्यू॑ति॒म् गव्यू॑ति मुक्षतम् घृ॒तेन॑ । \newline
26. उ॒क्ष॒त॒म् घृ॒तेन॑ घृ॒तेनो᳚क्षत मुक्षतम् घृ॒तेन॑ । \newline
27. घृ॒तेनेति॑ घृ॒तेन॑ । \newline
28. आ नो॑ न॒ आ नो॒ जने॒ जने॑ न॒ आ नो॒ जने᳚ । \newline
29. नो॒ जने॒ जने॑ नो नो॒ जने᳚ श्रवयतꣳ श्रवयत॒म् जने॑ नो नो॒ जने᳚ श्रवयतम् । \newline
30. जने᳚ श्रवयतꣳ श्रवयत॒म् जने॒ जने᳚ श्रवयतं ॅयुवाना युवाना श्रवयत॒म् जने॒ जने᳚ श्रवयतं ॅयुवाना । \newline
31. श्र॒व॒य॒तं॒ ॅयु॒वा॒ना॒ यु॒वा॒ना॒ श्र॒व॒य॒तꣳ॒॒ श्र॒व॒य॒तं॒ ॅयु॒वा॒ना॒ श्रु॒तꣳ श्रु॒तं ॅयु॑वाना श्रवयतꣳ श्रवयतं ॅयुवाना श्रु॒तम् । \newline
32. यु॒वा॒ना॒ श्रु॒तꣳ श्रु॒तं ॅयु॑वाना युवाना श्रु॒तम् मे॑ मे श्रु॒तं ॅयु॑वाना युवाना श्रु॒तम् मे᳚ । \newline
33. श्रु॒तम् मे॑ मे श्रु॒तꣳ श्रु॒तम् मे॑ मित्रावरुणा मित्रावरुणा मे श्रु॒तꣳ श्रु॒तम् मे॑ मित्रावरुणा । \newline
34. मे॒ मि॒त्रा॒व॒रु॒णा॒ मि॒त्रा॒व॒रु॒णा॒ मे॒ मे॒ मि॒त्रा॒व॒रु॒णा॒ हवा॒ हवा॑ मित्रावरुणा मे मे मित्रावरुणा॒ हवा᳚ । \newline
35. मि॒त्रा॒व॒रु॒णा॒ हवा॒ हवा॑ मित्रावरुणा मित्रावरुणा॒ हवे॒मेमा हवा॑ मित्रावरुणा मित्रावरुणा॒ हवे॒मा । \newline
36. मि॒त्रा॒व॒रु॒णेति॑ मित्रा - व॒रु॒णा॒ । \newline
37. हवे॒मेमा हवा॒ हवे॒मा । \newline
38. इ॒मेती॒मा । \newline
39. अ॒ग्निं ॅवो॑ वो अ॒ग्नि म॒ग्निं ॅवः॑ पू॒र्व्यम् पू॒र्व्यं ॅवो॑ अ॒ग्नि म॒ग्निं ॅवः॑ पू॒र्व्यम् । \newline
40. वः॒ पू॒र्व्यम् पू॒र्व्यं ॅवो॑ वः पू॒र्व्यम् गि॒रा गि॒रा पू॒र्व्यं ॅवो॑ वः पू॒र्व्यम् गि॒रा । \newline
41. पू॒र्व्यम् गि॒रा गि॒रा पू॒र्व्यम् पू॒र्व्यम् गि॒रा दे॒वम् दे॒वम् गि॒रा पू॒र्व्यम् पू॒र्व्यम् गि॒रा दे॒वम् । \newline
42. गि॒रा दे॒वम् दे॒वम् गि॒रा गि॒रा दे॒व मी॑ड ईडे दे॒वम् गि॒रा गि॒रा दे॒व मी॑डे । \newline
43. दे॒व मी॑ड ईडे दे॒वम् दे॒व मी॑डे॒ वसू॑नां॒ ॅवसू॑ना मीडे दे॒वम् दे॒व मी॑डे॒ वसू॑नाम् । \newline
44. ई॒डे॒ वसू॑नां॒ ॅवसू॑ना मीड ईडे॒ वसू॑नाम् । \newline
45. वसू॑ना॒मिति॒ वसू॑नाम् । \newline
46. स॒प॒र्यन्तः॑ पुरुप्रि॒यम् पु॑रुप्रि॒यꣳ स॑प॒र्यन्तः॑ सप॒र्यन्तः॑ पुरुप्रि॒यम् मि॒त्रम् मि॒त्रम् पु॑रुप्रि॒यꣳ स॑प॒र्यन्तः॑ सप॒र्यन्तः॑ पुरुप्रि॒यम् मि॒त्रम् । \newline
47. पु॒रु॒प्रि॒यम् मि॒त्रम् मि॒त्रम् पु॑रुप्रि॒यम् पु॑रुप्रि॒यम् मि॒त्रन्न न मि॒त्रम् पु॑रुप्रि॒यम् पु॑रुप्रि॒यम् मि॒त्रन्न । \newline
48. पु॒रु॒प्रि॒यमिति॑ पुरु - प्रि॒यम् । \newline
49. मि॒त्रन्न न मि॒त्रम् मि॒त्रन्न क्षे᳚त्र॒साध॑सम् क्षेत्र॒साध॑स॒न्न मि॒त्रम् मि॒त्रन्न क्षे᳚त्र॒साध॑सम् । \newline
50. न क्षे᳚त्र॒साध॑सम् क्षेत्र॒साध॑स॒न्न न क्षे᳚त्र॒साध॑सम् । \newline
51. क्षे॒त्र॒साध॑स॒मिति॑ क्षेत्र - साध॑सम् । \newline
52. म॒क्षू दे॒वव॑तो दे॒वव॑तो म॒क्षु म॒क्षू दे॒वव॑तो॒ रथो॒ रथो॑ दे॒वव॑तो म॒क्षु म॒क्षू दे॒वव॑तो॒ रथः॑ । \newline
53. दे॒वव॑तो॒ रथो॒ रथो॑ दे॒वव॑तो दे॒वव॑तो॒ रथः॒ शूरः॒ शूरो॒ रथो॑ दे॒वव॑तो दे॒वव॑तो॒ रथः॒ शूरः॑ । \newline
54. दे॒वव॑त॒ इति॑ दे॒व - व॒तः॒ । \newline
55. रथः॒ शूरः॒ शूरो॒ रथो॒ रथः॒ शूरो॑ वा वा॒ शूरो॒ रथो॒ रथः॒ शूरो॑ वा । \newline
\pagebreak
\markright{ TS 1.8.22.4  \hfill https://www.vedavms.in \hfill}
\addcontentsline{toc}{section}{ TS 1.8.22.4 }
\section*{ TS 1.8.22.4 }

\textbf{TS 1.8.22.4 } \newline
\textbf{Samhita Paata} \newline

शूरो॑ वा पृ॒थ्सु कासु॑ चित् । दे॒वानां॒ ॅय इन्मनो॒ यज॑मान॒ इय॑क्षत्य॒भीदय॑ज्वनो भुवत् ॥ न य॑जमान रिष्यसि॒ न सु॑न्वान॒ न दे॑वयो ॥ अस॒दत्र॑ सु॒वीर्य॑मु॒त त्यदा॒श्वश्वि॑यं ॥ नकि॒ष्टं कर्म॑णा नश॒न्न प्र यो॑ष॒न्न यो॑षति ॥ उप॑ क्षरन्ति॒ सिन्ध॑वो मयो॒भुव॑ ईजा॒नं च॑ य॒क्ष्यमा॑णं च धे॒नवः॑ । पृ॒णन्तं॑ च॒ पपु॑रिं च - [ ] \newline

\textbf{Pada Paata} \newline

शूरः॑ । वा॒ । पृ॒थ्स्विति॑ पृत्-सु । कासु॑ । चि॒त् ॥ दे॒वाना᳚म् । यः । इत् । मनः॑ । यज॑मानः । इय॑क्षति । अ॒भीति॑ । इत् । अय॑ज्वनः । भु॒व॒त् ॥ न । य॒ज॒मा॒न॒ । रि॒ष्य॒सि॒ । न । सु॒न्वा॒न॒ । न । दे॒व॒यो॒ इति॑ देव-यो॒ ॥ अस॑त् । अत्र॑ । सु॒वीर्य॒मिति॑ सु - वीर्य᳚म् । उ॒त । त्यत् । आ॒श्वश्वि॑य॒मित्या॑शु-अश्वि॑यम् ॥ नकिः॑ । तम् । कर्म॑णा । न॒श॒त् । न । प्रेति॑ । यो॒ष॒त् । न । यो॒ष॒ति॒ ॥ उपेति॑ । क्ष॒र॒न्ति॒ । सिन्ध॑वः । म॒यो॒भुव॒ इति॑ मयः - भुवः॑ । ई॒जा॒नम् । च॒ । य॒क्ष्यमा॑णम् । च॒ । धे॒नवः॑ ॥ पृ॒णन्त᳚म् । च॒ । पपु॑रिम् । च॒ ।  \newline



\textbf{Jatai Paata} \newline

1. शूरो॑ वा वा॒ शूरः॒ शूरो॑ वा । \newline
2. वा॒ पृ॒थ्सु पृ॒थ्सु वा॑ वा पृ॒थ्सु । \newline
3. पृ॒थ्सु कासु॒ कासु॑ पृ॒थ्सु पृ॒थ्सु कासु॑ । \newline
4. पृ॒थ्स्विति॑ पृत् - सु । \newline
5. कासु॑ चिच् चि॒त् कासु॒ कासु॑ चित् । \newline
6. चि॒दिति॑ चित् । \newline
7. दे॒वानां॒ ॅयो यो दे॒वाना᳚म् दे॒वानां॒ ॅयः । \newline
8. य इदिद् यो य इत् । \newline
9. इन् मनो॒ मन॒ इदिन् मनः॑ । \newline
10. मनो॒ यज॑मानो॒ यज॑मानो॒ मनो॒ मनो॒ यज॑मानः । \newline
11. यज॑मान॒ इय॑क्ष॒तीय॑क्षति॒ यज॑मानो॒ यज॑मान॒ इय॑क्षति । \newline
12. इय॑क्ष त्य॒भ्य॑भी य॑क्ष॒ती य॑क्षत्य॒भि । \newline
13. अ॒भी दि द॒भ्य॑भीत् । \newline
14. इदय॑ज्व॒नो ऽय॑ज्वन॒ इदि दय॑ज्वनः । \newline
15. अय॑ज्वनो भुवद् भुव॒दय॑ज्व॒नो ऽय॑ज्वनो भुवत् । \newline
16. भु॒व॒दिति॑ भुवत् । \newline
17. न य॑जमान यजमान॒ न न य॑जमान । \newline
18. य॒ज॒मा॒न॒ रि॒ष्य॒सि॒ रि॒ष्य॒सि॒ य॒ज॒मा॒न॒ य॒ज॒मा॒न॒ रि॒ष्य॒सि॒ । \newline
19. रि॒ष्य॒सि॒ न न रि॑ष्यसि रिष्यसि॒ न । \newline
20. न सु॑न्वान सुन्वान॒ न न सु॑न्वान । \newline
21. सु॒न्वा॒न॒ न न सु॑न्वान सुन्वान॒ न । \newline
22. न दे॑वयो देवयो॒ न न दे॑वयो । \newline
23. दे॒व॒यो॒ इति॑ देव - यो॒ । \newline
24. अस॒ दत्रात्रा स॒द स॒दत्र॑ । \newline
25. अत्र॑ सु॒वीर्यꣳ॑ सु॒वीर्य॒ मत्रात्र॑ सु॒वीर्य᳚म् । \newline
26. सु॒वीर्य॑ मु॒तोत सु॒वीर्यꣳ॑ सु॒वीर्य॑ मु॒त । \newline
27. सु॒वीर्य॒मिति॑ सु - वीर्य᳚म् । \newline
28. उ॒त त्यत् त्यदु॒तोत त्यत् । \newline
29. त्यदा॒श्वश्वि॑य मा॒श्वश्वि॑य॒म् त्यत् त्यदा॒श्वश्वि॑यम् । \newline
30. आ॒श्वश्वि॑य॒मित्या॑शु - अश्वि॑यम् । \newline
31. नकि॒ष् टम् तम् नकि॒र् नकि॒ष् टम् । \newline
32. तम् कर्म॑णा॒ कर्म॑णा॒ तम् तम् कर्म॑णा । \newline
33. कर्म॑णा नशन् नश॒त् कर्म॑णा॒ कर्म॑णा नशत् । \newline
34. न॒श॒न् न न न॑शन् नश॒न् न । \newline
35. न प्र प्र ण न प्र । \newline
36. प्र यो॑षद् योष॒त् प्र प्र यो॑षत् । \newline
37. यो॒ष॒न् न न यो॑षद् योष॒न् न । \newline
38. न यो॑षति योषति॒ न न यो॑षति । \newline
39. यो॒ष॒तीति॑ योषति । \newline
40. उप॑ क्षरन्ति क्षर॒न्त्युपोप॑ क्षरन्ति । \newline
41. क्ष॒र॒न्ति॒ सिन्ध॑वः॒ सिन्ध॑वः क्षरन्ति क्षरन्ति॒ सिन्ध॑वः । \newline
42. सिन्ध॑वो मयो॒भुवो॑ मयो॒भुवः॒ सिन्ध॑वः॒ सिन्ध॑वो मयो॒भुवः॑ । \newline
43. म॒यो॒भुव॑ ईजा॒न मी॑जा॒नम् म॑यो॒भुवो॑ मयो॒भुव॑ ईजा॒नम् । \newline
44. म॒यो॒भुव॒ इति॑ मयः - भुवः॑ । \newline
45. ई॒जा॒नम् च॑ चेजा॒न मी॑जा॒नम् च॑ । \newline
46. च॒ य॒क्ष्यमा॑णं ॅय॒क्ष्यमा॑णम् च च य॒क्ष्यमा॑णम् । \newline
47. य॒क्ष्यमा॑णम् च च य॒क्ष्यमा॑णं ॅय॒क्ष्यमा॑णम् च । \newline
48. च॒ धे॒नवो॑ धे॒नव॑श्च च धे॒नवः॑ । \newline
49. धे॒नव॒ इति॑ धे॒नवः॑ । \newline
50. पृ॒णन्त॑म् च च पृ॒णन्त॑म् पृ॒णन्त॑म् च । \newline
51. च॒ पपु॑रि॒म् पपु॑रिम् च च॒ पपु॑रिम् । \newline
52. पपु॑रिम् च च॒ पपु॑रि॒म् पपु॑रिम् च । \newline
53. च॒ श्र॒व॒स्यवः॑ श्रव॒स्यव॑श्च च श्रव॒स्यवः॑ । \newline

\textbf{Ghana Paata } \newline

1. शूरो॑ वा वा॒ शूरः॒ शूरो॑ वा पृ॒थ्सु पृ॒थ्सु वा॒ शूरः॒ शूरो॑ वा पृ॒थ्सु । \newline
2. वा॒ पृ॒थ्सु पृ॒थ्सु वा॑ वा पृ॒थ्सु कासु॒ कासु॑ पृ॒थ्सु वा॑ वा पृ॒थ्सु कासु॑ । \newline
3. पृ॒थ्सु कासु॒ कासु॑ पृ॒थ्सु पृ॒थ्सु कासु॑ चिच् चि॒त् कासु॑ पृ॒थ्सु पृ॒थ्सु कासु॑ चित् । \newline
4. पृ॒थ्स्विति॑ पृत् - सु । \newline
5. कासु॑ चिच् चि॒त् कासु॒ कासु॑ चित् । \newline
6. चि॒दिति॑ चित् । \newline
7. दे॒वानां॒ ॅयो यो दे॒वाना᳚म् दे॒वानां॒ ॅय इदिद् यो दे॒वाना᳚म् दे॒वानां॒ ॅय इत् । \newline
8. य इदिद् यो य इन् मनो॒ मन॒ इद् यो य इन् मनः॑ । \newline
9. इन् मनो॒ मन॒ इदिन् मनो॒ यज॑मानो॒ यज॑मानो॒ मन॒ इदिन् मनो॒ यज॑मानः । \newline
10. मनो॒ यज॑मानो॒ यज॑मानो॒ मनो॒ मनो॒ यज॑मान॒ इय॑क्ष॒ती य॑क्षति॒ यज॑मानो॒ मनो॒ मनो॒ यज॑मान॒ इय॑क्षति । \newline
11. यज॑मान॒ इय॑क्ष॒ती य॑क्षति॒ यज॑मानो॒ यज॑मान॒ इय॑क्षत्य॒भ्य॑ भीय॑क्षति॒ यज॑मानो॒ यज॑मान॒ इय॑क्षत्य॒भि । \newline
12. इय॑क्ष त्य॒भ्य॑भीय॑ क्ष॒तीय॑क्ष त्य॒भीदि द॒भीय॑ क्ष॒तीय॑ क्षत्य॒भीत् । \newline
13. अ॒भीदि द॒भ्य॑भी दय॑ज्व॒नो ऽय॑ज्वन॒ इद॒भ्य॑भी दय॑ज्वनः । \newline
14. इदय॑ज्व॒नो ऽय॑ज्वन॒ इदि दय॑ज्वनो भुवद् भुव॒ दय॑ज्वन॒ इदि दय॑ज्वनो भुवत् । \newline
15. अय॑ज्वनो भुवद् भुव॒ दय॑ज्व॒नो ऽय॑ज्वनो भुवत् । \newline
16. भु॒व॒दिति॑ भुवत् । \newline
17. न य॑जमान यजमान॒ न न य॑जमान रिष्यसि रिष्यसि यजमान॒ न न य॑जमान रिष्यसि । \newline
18. य॒ज॒मा॒न॒ रि॒ष्य॒सि॒ रि॒ष्य॒सि॒ य॒ज॒मा॒न॒ य॒ज॒मा॒न॒ रि॒ष्य॒सि॒ न न रि॑ष्यसि यजमान यजमान रिष्यसि॒ न । \newline
19. रि॒ष्य॒सि॒ न न रि॑ष्यसि रिष्यसि॒ न सु॑न्वान सुन्वान॒ न रि॑ष्यसि रिष्यसि॒ न सु॑न्वान । \newline
20. न सु॑न्वान सुन्वान॒ न न सु॑न्वान॒ न न सु॑न्वान॒ न न सु॑न्वान॒ न । \newline
21. सु॒न्वा॒न॒ न न सु॑न्वान सुन्वान॒ न दे॑वयो देवयो॒ न सु॑न्वान सुन्वान॒ न दे॑वयो । \newline
22. न दे॑वयो देवयो॒ न न दे॑वयो । \newline
23. दे॒व॒यो॒ इति॑ देव - यो॒ । \newline
24. अस॒ दत्रात्रा स॒दस॒ दत्र॑ सु॒वीर्यꣳ॑ सु॒वीर्य॒ मत्रा स॒दस॒ दत्र॑ सु॒वीर्य᳚म् । \newline
25. अत्र॑ सु॒वीर्यꣳ॑ सु॒वीर्य॒ मत्रात्र॑ सु॒वीर्य॑ मु॒तोत सु॒वीर्य॒ मत्रात्र॑ सु॒वीर्य॑ मु॒त । \newline
26. सु॒वीर्य॑ मु॒तोत सु॒वीर्यꣳ॑ सु॒वीर्य॑ मु॒त त्यत् त्यदु॒त सु॒वीर्यꣳ॑ सु॒वीर्य॑ मु॒त त्यत् । \newline
27. सु॒वीर्य॒मिति॑ सु - वीर्य᳚म् । \newline
28. उ॒त त्यत् त्यदु॒तोत त्यदा॒श्वश्वि॑य मा॒श्वश्वि॑य॒म् त्यदु॒तोत त्यदा॒ श्वश्वि॑यम् । \newline
29. त्यदा॒श्वश्वि॑य मा॒श्व श्वि॑य॒म् त्यत् त्यदा॒ श्वश्वि॑यम् । \newline
30. आ॒श्वश्वि॑य॒मित्या॑शु - अश्वि॑यम् । \newline
31. नकि॒ष् टम् तन्नकि॒र् नकि॒ष् टम् कर्म॑णा॒ कर्म॑णा॒ तन्नकि॒र् नकि॒ष् टम् कर्म॑णा । \newline
32. तम् कर्म॑णा॒ कर्म॑णा॒ तम् तम् कर्म॑णा नशन् नश॒त् कर्म॑णा॒ तम् तम् कर्म॑णा नशत् । \newline
33. कर्म॑णा नशन् नश॒त् कर्म॑णा॒ कर्म॑णा नश॒न् न न न॑श॒त् कर्म॑णा॒ कर्म॑णा नश॒न् न । \newline
34. न॒श॒न् न न न॑शन् नश॒न् न प्र प्र ण न॑शन् नश॒न् न प्र । \newline
35. न प्र प्र ण न प्र यो॑षद् योष॒त् प्र ण न प्र यो॑षत् । \newline
36. प्र यो॑षद् योष॒त् प्र प्र यो॑ष॒न् न न यो॑ष॒त् प्र प्र यो॑ष॒न् न । \newline
37. यो॒ष॒न् न न यो॑षद् योष॒न् न यो॑षति योषति॒ न यो॑षद् योष॒न् न यो॑षति । \newline
38. न यो॑षति योषति॒ न न यो॑षति । \newline
39. यो॒ष॒तीति॑ योषति । \newline
40. उप॑ क्षरन्ति क्षर॒न्त्युपोप॑ क्षरन्ति॒ सिन्ध॑वः॒ सिन्ध॑वः क्षर॒न्त्युपोप॑ क्षरन्ति॒ सिन्ध॑वः । \newline
41. क्ष॒र॒न्ति॒ सिन्ध॑वः॒ सिन्ध॑वः क्षरन्ति क्षरन्ति॒ सिन्ध॑वो मयो॒भुवो॑ मयो॒भुवः॒ सिन्ध॑वः क्षरन्ति क्षरन्ति॒ सिन्ध॑वो मयो॒भुवः॑ । \newline
42. सिन्ध॑वो मयो॒भुवो॑ मयो॒भुवः॒ सिन्ध॑वः॒ सिन्ध॑वो मयो॒भुव॑ ईजा॒न मी॑जा॒नम् म॑यो॒भुवः॒ सिन्ध॑वः॒ सिन्ध॑वो मयो॒भुव॑ ईजा॒नम् । \newline
43. म॒यो॒भुव॑ ईजा॒न मी॑जा॒नम् म॑यो॒भुवो॑ मयो॒भुव॑ ईजा॒नम् च॑ चेजा॒नम् म॑यो॒भुवो॑ मयो॒भुव॑ ईजा॒नम् च॑ । \newline
44. म॒यो॒भुव॒ इति॑ मयः - भुवः॑ । \newline
45. ई॒जा॒नम् च॑ चेजा॒न मी॑जा॒नम् च॑ य॒क्ष्यमा॑णं ॅय॒क्ष्यमा॑णम् चेजा॒न मी॑जा॒नम् च॑ य॒क्ष्यमा॑णम् । \newline
46. च॒ य॒क्ष्यमा॑णं ॅय॒क्ष्यमा॑णम् च च य॒क्ष्यमा॑णम् च च य॒क्ष्यमा॑णम् च च य॒क्ष्यमा॑णम् च । \newline
47. य॒क्ष्यमा॑णम् च च य॒क्ष्यमा॑णं ॅय॒क्ष्यमा॑णम् च धे॒नवो॑ धे॒नव॑श्च य॒क्ष्यमा॑णं ॅय॒क्ष्यमा॑णम् च धे॒नवः॑ । \newline
48. च॒ धे॒नवो॑ धे॒नव॑श्च च धे॒नवः॑ । \newline
49. धे॒नव॒ इति॑ धे॒नवः॑ । \newline
50. पृ॒णन्त॑म् च च पृ॒णन्त॑म् पृ॒णन्त॑म् च॒ पपु॑रि॒म् पपु॑रिम् च पृ॒णन्त॑म् पृ॒णन्त॑म् च॒ पपु॑रिम् । \newline
51. च॒ पपु॑रि॒म् पपु॑रिम् च च॒ पपु॑रिम् च च॒ पपु॑रिम् च च॒ पपु॑रिम् च । \newline
52. पपु॑रिम् च च॒ पपु॑रि॒म् पपु॑रिम् च श्रव॒स्यवः॑ श्रव॒स्यव॑श्च॒ पपु॑रि॒म् पपु॑रिम् च श्रव॒स्यवः॑ । \newline
53. च॒ श्र॒व॒स्यवः॑ श्रव॒स्यव॑श्च च श्रव॒स्यवो॑ घृ॒तस्य॑ घृ॒तस्य॑ श्रव॒स्यव॑श्च च श्रव॒स्यवो॑ घृ॒तस्य॑ । \newline
\pagebreak
\markright{ TS 1.8.22.5  \hfill https://www.vedavms.in \hfill}
\addcontentsline{toc}{section}{ TS 1.8.22.5 }
\section*{ TS 1.8.22.5 }

\textbf{TS 1.8.22.5 } \newline
\textbf{Samhita Paata} \newline

श्रव॒स्यवो॑ घृ॒तस्य॒ धारा॒ उप॑ यन्ति वि॒श्वतः॑ ॥सोमा॑रुद्रा॒ वि वृ॑हतं॒ ॅविषू॑ची॒ममी॑वा॒ या नो॒ गय॑-मावि॒वेश॑ । आ॒रे बा॑धेथां॒ निर्.ऋ॑तिं परा॒चैः कृ॒तं चि॒देनः॒ प्र मु॑मुक्त-म॒स्मत् ॥ सोमा॑रुद्रा यु॒व-मे॒तान्य॒स्मे विश्वा॑ त॒नूषु॑ भेष॒जानि॑ धत्तं । अव॑ स्यतं मु॒ञ्चतं॒ ॅयन्नो॒ अस्ति॑ त॒नूषु॑ ब॒द्धं कृ॒तमेनो॑ अ॒स्मत् ॥ सोमा॑पूषणा॒ जन॑ना रयी॒णां जन॑ना दि॒वो जन॑ना ( ) पृथि॒व्याः । जा॒तौ विश्व॑स्य॒ भुव॑नस्य गो॒पौ दे॒वा अ॑कृण्वन्न॒मृत॑स्य॒ नाभिं᳚ ॥ इ॒मौ दे॒वौ जाय॑मानौ जुषन्ते॒मौ तमाꣳ॑सि गूहता॒-मजु॑ष्टा । आ॒भ्यामिन्द्रः॑ प॒क्वमा॒मास्व॒न्तः सो॑मापू॒षभ्यां᳚ जनदु॒स्रिया॑सु ॥ \newline

\textbf{Pada Paata} \newline

श्र॒व॒स्यवः॑ । घृ॒तस्य॑ । धाराः᳚ । उपेति॑ । य॒न्ति॒ । वि॒श्वतः॑ ॥ सोमा॑रु॒द्रेति॒ सोमा᳚-रु॒द्रा॒ । वीति॑ । वृ॒ह॒त॒म् । विषू॑चीम् । अमी॑वा । या । नः॒ । गय᳚म् । आ॒वि॒वेशेत्या᳚ - वि॒वेश॑ ॥ आ॒रे । बा॒धे॒था॒म् । निर्.ऋ॑ति॒मिति॒ निः - ऋ॒ति॒म् । प॒रा॒चैः । कृ॒तम् । चि॒त् । एनः॑ । प्रेति॑ । मु॒मु॒क्त॒म् । अ॒स्मत् ॥ सोमा॑रु॒द्रेति॒ सोमा᳚-रु॒द्रा॒ । यु॒वम् । ए॒तानि॑ । अ॒स्मे इति॑ । विश्वा᳚ । त॒नूषु॑ । भे॒ष॒जानि॑ । ध॒त्त॒म् । अवेति॑ । स्य॒त॒म् । मु॒ञ्चत᳚म् । यत् । नः॒ । अस्ति॑ । त॒नूषु॑ । ब॒द्धम् । कृ॒तम् । एनः॑ । अ॒स्मत् । सोमा॑पूष॒णेति॒ सोमा᳚ - पू॒ष॒णा॒ । जन॑ना । र॒यी॒णाम् । जन॑ना । दि॒वः । जन॑ना ( ) । पृ॒थि॒व्याः ॥ जा॒तौ । विश्व॑स्य । भुव॑नस्य । गो॒पौ । दे॒वाः । अ॒कृ॒ण्व॒न्न् । अ॒मृत॑स्य । नाभि᳚म् ॥ इ॒मौ । दे॒वौ । जाय॑मानौ । जु॒ष॒न्त॒ । इ॒मौ । तमाꣳ॑सि । गू॒ह॒ता॒म् । अजु॑ष्टा ॥ आ॒भ्याम् । इन्द्रः॑ । प॒क्वम् । आ॒मासु॑ । अ॒न्तः । सो॒मा॒पू॒षभ्या॒मिति॑ सोमापू॒ष - भ्या॒म् । ज॒न॒त् । उ॒स्रिया॑सु ॥  \newline



\textbf{Jatai Paata} \newline

1. श्र॒व॒स्यवो॑ घृ॒तस्य॑ घृ॒तस्य॑ श्रव॒स्यवः॑ श्रव॒स्यवो॑ घृ॒तस्य॑ । \newline
2. घृ॒तस्य॒ धारा॒ धारा॑ घृ॒तस्य॑ घृ॒तस्य॒ धाराः᳚ । \newline
3. धारा॒ उपोप॒ धारा॒ धारा॒ उप॑ । \newline
4. उप॑ यन्ति य॒न्त्युपोप॑ यन्ति । \newline
5. य॒न्ति॒ वि॒श्वतो॑ वि॒श्वतो॑ यन्ति यन्ति वि॒श्वतः॑ । \newline
6. वि॒श्वत॒ इति॑ वि॒श्वतः॑ । \newline
7. सोमा॑रुद्रा॒ वि वि सोमा॑रुद्रा॒ सोमा॑रुद्रा॒ वि । \newline
8. सोमा॑रु॒द्रेति॒ सोमा᳚ - रु॒द्रा॒ । \newline
9. वि वृ॑हतं ॅवृहतं॒ ॅवि वि वृ॑हतम् । \newline
10. वृ॒ह॒तं॒ ॅविषू॑चीं॒ ॅविषू॑चीं ॅवृहतं ॅवृहतं॒ ॅविषू॑चीम् । \newline
11. विषू॑ची॒ ममी॒वा ऽमी॑वा॒ विषू॑चीं॒ ॅविषू॑ची॒ ममी॑वा । \newline
12. अमी॑वा॒ या या ऽमी॒वा ऽमी॑वा॒ या । \newline
13. या नो॑ नो॒ या या नः॑ । \newline
14. नो॒ गय॒म् गय॑म् नो नो॒ गय᳚म् । \newline
15. गय॑ मावि॒वेशा॑ वि॒वेश॒ गय॒म् गय॑ मावि॒वेश॑ । \newline
16. आ॒वि॒वेशेत्या᳚ - वि॒वेश॑ । \newline
17. आ॒रे बा॑धेथाम् बाधेथा मा॒र आ॒रे बा॑धेथाम् । \newline
18. बा॒धे॒था॒म् निर्.ऋ॑ति॒म् निर्.ऋ॑तिम् बाधेथाम् बाधेथा॒म् निर्.ऋ॑तिम् । \newline
19. निर्.ऋ॑तिम् परा॒चैः प॑रा॒चैर् निर्.ऋ॑ति॒म् निर्.ऋ॑तिम् परा॒चैः । \newline
20. निर्.ऋ॑ति॒मिति॒ निः - ऋ॒ति॒म् । \newline
21. प॒रा॒चैः कृ॒तम् कृ॒तम् प॑रा॒चैः प॑रा॒चैः कृ॒तम् । \newline
22. कृ॒तम् चि॑च् चित् कृ॒तम् कृ॒तम् चि॑त् । \newline
23. चि॒देन॒ एन॑ श्चिच् चि॒देनः॑ । \newline
24. एनः॒ प्र प्रैन॒ एनः॒ प्र । \newline
25. प्र मु॑मुक्तम् मुमुक्त॒म् प्र प्र मु॑मुक्तम् । \newline
26. मु॒मु॒क्त॒ म॒स्मद॒स्मन् मु॑मुक्तम् मुमुक्त म॒स्मत् । \newline
27. अ॒स्मदित्य॒स्मत् । \newline
28. सोमा॑रुद्रा यु॒वं ॅयु॒वꣳ सोमा॑रुद्रा॒ सोमा॑रुद्रा यु॒वम् । \newline
29. सोमा॑रु॒द्रेति॒ सोमा᳚ - रु॒द्रा॒ । \newline
30. यु॒व मे॒ता न्ये॒तानि॑ यु॒वं ॅयु॒व मे॒तानि॑ । \newline
31. ए॒ता न्य॒स्मे अ॒स्मे ए॒ता न्ये॒ता न्य॒स्मे । \newline
32. अ॒स्मे विश्वा॒ विश्वा॑ अ॒स्मे अ॒स्मे विश्वा᳚ । \newline
33. अ॒स्मे इत्य॒स्मे । \newline
34. विश्वा॑ त॒नूषु॑ त॒नूषु॒ विश्वा॒ विश्वा॑ त॒नूषु॑ । \newline
35. त॒नूषु॑ भेष॒जानि॑ भेष॒जानि॑ त॒नूषु॑ त॒नूषु॑ भेष॒जानि॑ । \newline
36. भे॒ष॒जानि॑ धत्तम् धत्तम् भेष॒जानि॑ भेष॒जानि॑ धत्तम् । \newline
37. ध॒त्त॒मिति॑ धत्तम् । \newline
38. अव॑ स्यतꣳ स्यत॒ मवाव॑ स्यतम् । \newline
39. स्य॒त॒म् मु॒ञ्चत॑म् मु॒ञ्चतꣳ॑ स्यतꣳ स्यतम् मु॒ञ्चत᳚म् । \newline
40. मु॒ञ्चतं॒ ॅयद् यन् मु॒ञ्चत॑म् मु॒ञ्चतं॒ ॅयत् । \newline
41. यन् नो॑ नो॒ यद् यन् नः॑ । \newline
42. नो॒ अस्त्यस्ति॑ नो नो॒ अस्ति॑ । \newline
43. अस्ति॑ त॒नूषु॑ त॒नू ष्वस्त्यस्ति॑ त॒नूषु॑ । \newline
44. त॒नूषु॑ ब॒द्धम् ब॒द्धम् त॒नूषु॑ त॒नूषु॑ ब॒द्धम् । \newline
45. ब॒द्धम् कृ॒तम् कृ॒तम् ब॒द्धम् ब॒द्धम् कृ॒तम् । \newline
46. कृ॒त मेन॒ एनः॑ कृ॒तम् कृ॒त मेनः॑ । \newline
47. एनो॑ अ॒स्म द॒स्म देन॒ एनो॑ अ॒स्मत् । \newline
48. अ॒स्मदित्य॒स्मत् । \newline
49. सोमा॑पूषणा॒ जन॑ना॒ जन॑ना॒ सोमा॑पूषणा॒ सोमा॑पूषणा॒ जन॑ना । \newline
50. सोमा॑पूष॒णेति॒ सोमा᳚ - पू॒ष॒णा॒ । \newline
51. जन॑ना रयी॒णाꣳ र॑यी॒णाम् जन॑ना॒ जन॑ना रयी॒णाम् । \newline
52. र॒यी॒णाम् जन॑ना॒ जन॑ना रयी॒णाꣳ र॑यी॒णाम् जन॑ना । \newline
53. जन॑ना दि॒वो दि॒वो जन॑ना॒ जन॑ना दि॒वः । \newline
54. दि॒वो जन॑ना॒ जन॑ना दि॒वो दि॒वो जन॑ना । \newline
55. जन॑ना पृथि॒व्याः पृ॑थि॒व्या जन॑ना॒ जन॑ना पृथि॒व्याः । \newline
56. पृ॒थि॒व्या इति॑ पृथि॒व्याः । \newline
57. जा॒तौ विश्व॑स्य॒ विश्व॑स्य जा॒तौ जा॒तौ विश्व॑स्य । \newline
58. विश्व॑स्य॒ भुव॑नस्य॒ भुव॑नस्य॒ विश्व॑स्य॒ विश्व॑स्य॒ भुव॑नस्य । \newline
59. भुव॑नस्य गो॒पौ गो॒पौ भुव॑नस्य॒ भुव॑नस्य गो॒पौ । \newline
60. गो॒पौ दे॒वा दे॒वा गो॒पौ गो॒पौ दे॒वाः । \newline
61. दे॒वा अ॑कृण्वन् नकृण्वन् दे॒वा दे॒वा अ॑कृण्वन्न् । \newline
62. अ॒कृ॒ण्व॒न् न॒मृत॑स्या॒ मृत॑स्या कृण्वन् नकृण्वन् न॒मृत॑स्य । \newline
63. अ॒मृत॑स्य॒ नाभि॒म् नाभि॑ म॒मृत॑स्या॒ मृत॑स्य॒ नाभि᳚म् । \newline
64. नाभि॒मिति॒ नाभि᳚म् । \newline
65. इ॒मौ दे॒वौ दे॒वा वि॒मा वि॒मौ दे॒वौ । \newline
66. दे॒वौ जाय॑मानौ॒ जाय॑मानौ दे॒वौ दे॒वौ जाय॑मानौ । \newline
67. जाय॑मानौ जुषन्त जुषन्त॒ जाय॑मानौ॒ जाय॑मानौ जुषन्त । \newline
68. जु॒ष॒न्ते॒ मा वि॒मौ जु॑षन्त जुषन्ते॒ मौ । \newline
69. इ॒मौ तमाꣳ॑सि॒ तमाꣳ॑सी॒मा वि॒मौ तमाꣳ॑सि । \newline
70. तमाꣳ॑सि गूहताम् गूहता॒म् तमाꣳ॑सि॒ तमाꣳ॑सि गूहताम् । \newline
71. गू॒ह॒ता॒ मजु॒ष्टा ऽजु॑ष्टा गूहताम् गूहता॒ मजु॑ष्टा । \newline
72. अजु॒ष्टेत्यजु॑ष्टा । \newline
73. आ॒भ्या मिन्द्र॒ इन्द्र॑ आ॒भ्या मा॒भ्या मिन्द्रः॑ । \newline
74. इन्द्रः॑ प॒क्वम् प॒क्व मिन्द्र॒ इन्द्रः॑ प॒क्वम् । \newline
75. प॒क्व मा॒मा स्वा॒मासु॑ प॒क्वम् प॒क्व मा॒मासु॑ । \newline
76. आ॒मास्व॒न्त र॒न्त रा॒मा स्वा॒मा स्व॒न्तः । \newline
77. अ॒न्तः सो॑मापू॒षभ्याꣳ॑ सोमापू॒षभ्या॑ म॒न्त र॒न्तः सो॑मापू॒षभ्या᳚म् । \newline
78. सो॒मा॒पू॒षभ्या᳚म् जनज् जनथ् सोमापू॒षभ्याꣳ॑ सोमापू॒षभ्या᳚म् जनत् । \newline
79. सो॒मा॒पू॒षभ्या॒मिति॑ सोमापू॒ष - भ्या॒म् । \newline
80. ज॒न॒ दु॒स्रिया॑सू॒ स्रिया॑सु जनज् जन दु॒स्रिया॑सु । \newline
81. उ॒स्रि॒या॒स्वित्यु॒स्रिया॑सु । \newline

\textbf{Ghana Paata } \newline

1. श्र॒व॒स्यवो॑ घृ॒तस्य॑ घृ॒तस्य॑ श्रव॒स्यवः॑ श्रव॒स्यवो॑ घृ॒तस्य॒ धारा॒ धारा॑ घृ॒तस्य॑ श्रव॒स्यवः॑ श्रव॒स्यवो॑ घृ॒तस्य॒ धाराः᳚ । \newline
2. घृ॒तस्य॒ धारा॒ धारा॑ घृ॒तस्य॑ घृ॒तस्य॒ धारा॒ उपोप॒ धारा॑ घृ॒तस्य॑ घृ॒तस्य॒ धारा॒ उप॑ । \newline
3. धारा॒ उपोप॒ धारा॒ धारा॒ उप॑ यन्ति य॒न्त्युप॒ धारा॒ धारा॒ उप॑ यन्ति । \newline
4. उप॑ यन्ति य॒न्त्युपोप॑ यन्ति वि॒श्वतो॑ वि॒श्वतो॑ य॒न्त्युपोप॑ यन्ति वि॒श्वतः॑ । \newline
5. य॒न्ति॒ वि॒श्वतो॑ वि॒श्वतो॑ यन्ति यन्ति वि॒श्वतः॑ । \newline
6. वि॒श्वत॒ इति॑ वि॒श्वतः॑ । \newline
7. सोमा॑रुद्रा॒ वि वि सोमा॑रुद्रा॒ सोमा॑रुद्रा॒ वि वृ॑हतं ॅवृहतं॒ ॅवि सोमा॑रुद्रा॒ सोमा॑रुद्रा॒ वि वृ॑हतम् । \newline
8. सोमा॑रु॒द्रेति॒ सोमा᳚ - रु॒द्रा॒ । \newline
9. वि वृ॑हतं ॅवृहतं॒ ॅवि वि वृ॑हतं॒ ॅविषू॑चीं॒ ॅविषू॑चीं ॅवृहतं॒ ॅवि वि वृ॑हतं॒ ॅविषू॑चीम् । \newline
10. वृ॒ह॒तं॒ ॅविषू॑चीं॒ ॅविषू॑चीं ॅवृहतं ॅवृहतं॒ ॅविषू॑ची॒ ममी॒वा ऽमी॑वा॒ विषू॑चीं ॅवृहतं ॅवृहतं॒ ॅविषू॑ची॒ ममी॑वा । \newline
11. विषू॑ची॒ ममी॒वा ऽमी॑वा॒ विषू॑चीं॒ ॅविषू॑ची॒ ममी॑वा॒ या या ऽमी॑वा॒ विषू॑चीं॒ ॅविषू॑ची॒ ममी॑वा॒ या । \newline
12. अमी॑वा॒ या या ऽमी॒वा ऽमी॑वा॒ या नो॑ नो॒ या ऽमी॒वा ऽमी॑वा॒ या नः॑ । \newline
13. या नो॑ नो॒ या या नो॒ गय॒म् गय॑न्नो॒ या या नो॒ गय᳚म् । \newline
14. नो॒ गय॒म् गय॑न्नो नो॒ गय॑ मावि॒वेशा॑ वि॒वेश॒ गय॑न्नो नो॒ गय॑ मावि॒वेश॑ । \newline
15. गय॑ मावि॒वेशा॑ वि॒वेश॒ गय॒म् गय॑ मावि॒वेश॑ । \newline
16. आ॒वि॒वेशेत्या᳚ - वि॒वेश॑ । \newline
17. आ॒रे बा॑धेथाम् बाधेथा मा॒र आ॒रे बा॑धेथा॒न् निर्.ऋ॑ति॒न् निर्.ऋ॑तिम् बाधेथा मा॒र आ॒रे बा॑धेथा॒न् निर्.ऋ॑तिम् । \newline
18. बा॒धे॒था॒न् निर्.ऋ॑ति॒न् निर्.ऋ॑तिम् बाधेथाम् बाधेथा॒न् निर्.ऋ॑तिम् परा॒चैः प॑रा॒चैर् निर्.ऋ॑तिम् बाधेथाम् बाधेथा॒न् निर्.ऋ॑तिम् परा॒चैः । \newline
19. निर्.ऋ॑तिम् परा॒चैः प॑रा॒चैर् निर्.ऋ॑ति॒न् निर्.ऋ॑तिम् परा॒चैः कृ॒तम् कृ॒तम् प॑रा॒चैर् निर्.ऋ॑ति॒न् निर्.ऋ॑तिम् परा॒चैः कृ॒तम् । \newline
20. निर्.ऋ॑ति॒मिति॒ निः - ऋ॒ति॒म् । \newline
21. प॒रा॒चैः कृ॒तम् कृ॒तम् प॑रा॒चैः प॑रा॒चैः कृ॒तम् चि॑च् चित् कृ॒तम् प॑रा॒चैः प॑रा॒चैः कृ॒तम् चि॑त् । \newline
22. कृ॒तम् चि॑च् चित् कृ॒तम् कृ॒तम् चि॒देन॒ एन॑श्चित् कृ॒तम् कृ॒तम् चि॒देनः॑ । \newline
23. चि॒देन॒ एन॑श्चिच् चि॒देनः॒ प्र प्रैन॑श्चिच् चि॒देनः॒ प्र । \newline
24. एनः॒ प्र प्रैन॒ एनः॒ प्र मु॑मुक्तम् मुमुक्त॒म् प्रैन॒ एनः॒ प्र मु॑मुक्तम् । \newline
25. प्र मु॑मुक्तम् मुमुक्त॒म् प्र प्र मु॑मुक्त म॒स्म द॒स्मन् मु॑मुक्त॒म् प्र प्र मु॑मुक्त म॒स्मत् । \newline
26. मु॒मु॒क्त॒ म॒स्म द॒स्मन् मु॑मुक्तम् मुमुक्त म॒स्मत् । \newline
27. अ॒स्मदित्य॒स्मत् । \newline
28. सोमा॑रुद्रा यु॒वं ॅयु॒वꣳ सोमा॑रुद्रा॒ सोमा॑रुद्रा यु॒व मे॒ता न्ये॒तानि॑ यु॒वꣳ सोमा॑रुद्रा॒ सोमा॑रुद्रा यु॒व मे॒तानि॑ । \newline
29. सोमा॑रु॒द्रेति॒ सोमा᳚ - रु॒द्रा॒ । \newline
30. यु॒व मे॒तान्ये॒तानि॑ यु॒वं ॅयु॒व मे॒ता न्य॒स्मे अ॒स्मे ए॒तानि॑ यु॒वं ॅयु॒व मे॒ता न्य॒स्मे । \newline
31. ए॒ता न्य॒स्मे अ॒स्मे ए॒ता न्ये॒ता न्य॒स्मे विश्वा॒ विश्वा॑ अ॒स्मे ए॒ता न्ये॒ता न्य॒स्मे विश्वा᳚ । \newline
32. अ॒स्मे विश्वा॒ विश्वा॑ अ॒स्मे अ॒स्मे विश्वा॑ त॒नूषु॑ त॒नूषु॒ विश्वा॑ अ॒स्मे अ॒स्मे विश्वा॑ त॒नूषु॑ । \newline
33. अ॒स्मे इत्य॒स्मे । \newline
34. विश्वा॑ त॒नूषु॑ त॒नूषु॒ विश्वा॒ विश्वा॑ त॒नूषु॑ भेष॒जानि॑ भेष॒जानि॑ त॒नूषु॒ विश्वा॒ विश्वा॑ त॒नूषु॑ भेष॒जानि॑ । \newline
35. त॒नूषु॑ भेष॒जानि॑ भेष॒जानि॑ त॒नूषु॑ त॒नूषु॑ भेष॒जानि॑ धत्तम् धत्तम् भेष॒जानि॑ त॒नूषु॑ त॒नूषु॑ भेष॒जानि॑ धत्तम् । \newline
36. भे॒ष॒जानि॑ धत्तम् धत्तम् भेष॒जानि॑ भेष॒जानि॑ धत्तम् । \newline
37. ध॒त्त॒मिति॑ धत्तम् । \newline
38. अव॑ स्यतꣳ स्यत॒ मवाव॑ स्यतम् मु॒ञ्चत॑म् मु॒ञ्चतꣳ॑ स्यत॒ मवाव॑ स्यतम् मु॒ञ्चत᳚म् । \newline
39. स्य॒त॒म् मु॒ञ्चत॑म् मु॒ञ्चतꣳ॑ स्यतꣳ स्यतम् मु॒ञ्चतं॒ ॅयद् यन् मु॒ञ्चतꣳ॑ स्यतꣳ स्यतम् मु॒ञ्चतं॒ ॅयत् । \newline
40. मु॒ञ्चतं॒ ॅयद् यन् मु॒ञ्चत॑म् मु॒ञ्चतं॒ ॅयन् नो॑ नो॒ यन् मु॒ञ्चत॑म् मु॒ञ्चतं॒ ॅयन् नः॑ । \newline
41. यन् नो॑ नो॒ यद् यन् नो॒ अस्त्यस्ति॑ नो॒ यद् यन् नो॒ अस्ति॑ । \newline
42. नो॒ अस्त्यस्ति॑ नो नो॒ अस्ति॑ त॒नूषु॑ त॒नूष्वस्ति॑ नो नो॒ अस्ति॑ त॒नूषु॑ । \newline
43. अस्ति॑ त॒नूषु॑ त॒नूष्व स्त्यस्ति॑ त॒नूषु॑ ब॒द्धम् ब॒द्धम् त॒नूष्व स्त्यस्ति॑ त॒नूषु॑ ब॒द्धम् । \newline
44. त॒नूषु॑ ब॒द्धम् ब॒द्धम् त॒नूषु॑ त॒नूषु॑ ब॒द्धम् कृ॒तम् कृ॒तम् ब॒द्धम् त॒नूषु॑ त॒नूषु॑ ब॒द्धम् कृ॒तम् । \newline
45. ब॒द्धम् कृ॒तम् कृ॒तम् ब॒द्धम् ब॒द्धम् कृ॒त मेन॒ एनः॑ कृ॒तम् ब॒द्धम् ब॒द्धम् कृ॒त मेनः॑ । \newline
46. कृ॒त मेन॒ एनः॑ कृ॒तम् कृ॒त मेनो॑ अ॒स्म द॒स्म देनः॑ कृ॒तम् कृ॒त मेनो॑ अ॒स्मत् । \newline
47. एनो॑ अ॒स्म द॒स्म देन॒ एनो॑ अ॒स्मत् । \newline
48. अ॒स्मदित्य॒स्मत् । \newline
49. सोमा॑पूषणा॒ जन॑ना॒ जन॑ना॒ सोमा॑पूषणा॒ सोमा॑पूषणा॒ जन॑ना रयी॒णाꣳ र॑यी॒णाम् जन॑ना॒ सोमा॑पूषणा॒ सोमा॑पूषणा॒ जन॑ना रयी॒णाम् । \newline
50. सोमा॑पूष॒णेति॒ सोमा᳚ - पू॒ष॒णा॒ । \newline
51. जन॑ना रयी॒णाꣳ र॑यी॒णाम् जन॑ना॒ जन॑ना रयी॒णाम् जन॑ना॒ जन॑ना रयी॒णाम् जन॑ना॒ जन॑ना रयी॒णाम् जन॑ना । \newline
52. र॒यी॒णाम् जन॑ना॒ जन॑ना रयी॒णाꣳ र॑यी॒णाम् जन॑ना दि॒वो दि॒वो जन॑ना रयी॒णाꣳ र॑यी॒णाम् जन॑ना दि॒वः । \newline
53. जन॑ना दि॒वो दि॒वो जन॑ना॒ जन॑ना दि॒वो जन॑ना॒ जन॑ना दि॒वो जन॑ना॒ जन॑ना दि॒वो जन॑ना । \newline
54. दि॒वो जन॑ना॒ जन॑ना दि॒वो दि॒वो जन॑ना पृथि॒व्याः पृ॑थि॒व्या जन॑ना दि॒वो दि॒वो जन॑ना पृथि॒व्याः । \newline
55. जन॑ना पृथि॒व्याः पृ॑थि॒व्या जन॑ना॒ जन॑ना पृथि॒व्याः । \newline
56. पृ॒थि॒व्या इति॑ पृथि॒व्याः । \newline
57. जा॒तौ विश्व॑स्य॒ विश्व॑स्य जा॒तौ जा॒तौ विश्व॑स्य॒ भुव॑नस्य॒ भुव॑नस्य॒ विश्व॑स्य जा॒तौ जा॒तौ विश्व॑स्य॒ भुव॑नस्य । \newline
58. विश्व॑स्य॒ भुव॑नस्य॒ भुव॑नस्य॒ विश्व॑स्य॒ विश्व॑स्य॒ भुव॑नस्य गो॒पौ गो॒पौ भुव॑नस्य॒ विश्व॑स्य॒ विश्व॑स्य॒ भुव॑नस्य गो॒पौ । \newline
59. भुव॑नस्य गो॒पौ गो॒पौ भुव॑नस्य॒ भुव॑नस्य गो॒पौ दे॒वा दे॒वा गो॒पौ भुव॑नस्य॒ भुव॑नस्य गो॒पौ दे॒वाः । \newline
60. गो॒पौ दे॒वा दे॒वा गो॒पौ गो॒पौ दे॒वा अ॑कृण्वन् नकृण्वन् दे॒वा गो॒पौ गो॒पौ दे॒वा अ॑कृण्वन्न् । \newline
61. दे॒वा अ॑कृण्वन् नकृण्वन् दे॒वा दे॒वा अ॑कृण्वन् न॒मृत॑स्या॒ मृत॑स्या कृण्वन् दे॒वा दे॒वा अ॑कृण्वन् न॒मृत॑स्य । \newline
62. अ॒कृ॒ण्व॒न् न॒मृत॑स्या॒ मृत॑स्या कृण्वन् नकृण्वन् न॒मृत॑स्य॒ नाभि॒न्नाभि॑ म॒मृत॑स्या कृण्वन् नकृण्वन् न॒मृत॑स्य॒ नाभि᳚म् । \newline
63. अ॒मृत॑स्य॒ नाभि॒न् नाभि॑ म॒मृत॑स्या॒ मृत॑स्य॒ नाभि᳚म् । \newline
64. नाभि॒मिति॒ नाभि᳚म् । \newline
65. इ॒मौ दे॒वौ दे॒वा वि॒मा वि॒मौ दे॒वौ जाय॑मानौ॒ जाय॑मानौ दे॒वा वि॒मा वि॒मौ दे॒वौ जाय॑मानौ । \newline
66. दे॒वौ जाय॑मानौ॒ जाय॑मानौ दे॒वौ दे॒वौ जाय॑मानौ जुषन्त जुषन्त॒ जाय॑मानौ दे॒वौ दे॒वौ जाय॑मानौ जुषन्त । \newline
67. जाय॑मानौ जुषन्त जुषन्त॒ जाय॑मानौ॒ जाय॑मानौ जुषन्ते॒ मा वि॒मौ जु॑षन्त॒ जाय॑मानौ॒ जाय॑मानौ जुषन्ते॒ मौ । \newline
68. जु॒ष॒न्ते॒ मा वि॒मौ जु॑षन्त जुषन्ते॒ मौ तमाꣳ॑सि॒ तमाꣳ॑सी॒मौ जु॑षन्त जुषन्ते॒ मौ तमाꣳ॑सि । \newline
69. इ॒मौ तमाꣳ॑सि॒ तमाꣳ॑सी॒मा वि॒मौ तमाꣳ॑सि गूहताम् गूहता॒म् तमाꣳ॑सी॒मा वि॒मौ तमाꣳ॑सि गूहताम् । \newline
70. तमाꣳ॑सि गूहताम् गूहता॒म् तमाꣳ॑सि॒ तमाꣳ॑सि गूहता॒ मजु॒ष्टा ऽजु॑ष्टा गूहता॒म् तमाꣳ॑सि॒ तमाꣳ॑सि गूहता॒ मजु॑ष्टा । \newline
71. गू॒ह॒ता॒ मजु॒ष्टा ऽजु॑ष्टा गूहताम् गूहता॒ मजु॑ष्टा । \newline
72. अजु॒ष्टेत्यजु॑ष्टा । \newline
73. आ॒भ्या मिन्द्र॒ इन्द्र॑ आ॒भ्या मा॒भ्या मिन्द्रः॑ प॒क्वम् प॒क्व मिन्द्र॑ आ॒भ्या मा॒भ्या मिन्द्रः॑ प॒क्वम् । \newline
74. इन्द्रः॑ प॒क्वम् प॒क्व मिन्द्र॒ इन्द्रः॑ प॒क्व मा॒मा स्वा॒मासु॑ प॒क्व मिन्द्र॒ इन्द्रः॑ प॒क्व मा॒मासु॑ । \newline
75. प॒क्व मा॒मा स्वा॒मासु॑ प॒क्वम् प॒क्व मा॒मा स्व॒न्त र॒न्त रा॒मासु॑ प॒क्वम् प॒क्व मा॒मा स्व॒न्तः । \newline
76. आ॒मास्व॒न्त र॒न्त रा॒मा स्वा॒मा स्व॒न्तः सो॑मापू॒षभ्याꣳ॑ सोमापू॒षभ्या॑ म॒न्त रा॒मा स्वा॒मा स्व॒न्तः सो॑मापू॒षभ्या᳚म् । \newline
77. अ॒न्तः सो॑मापू॒षभ्याꣳ॑ सोमापू॒षभ्या॑ म॒न्त र॒न्तः सो॑मापू॒षभ्या᳚म् जनज् जनथ् सोमापू॒षभ्या॑ म॒न्त र॒न्तः सो॑मापू॒षभ्या᳚म् जनत् । \newline
78. सो॒मा॒पू॒षभ्या᳚म् जनज् जनथ् सोमापू॒षभ्याꣳ॑ सोमापू॒षभ्या᳚म् जन दु॒स्रिया॑सू॒ स्रिया॑सु जनथ् सोमापू॒षभ्याꣳ॑ सोमापू॒षभ्या᳚म् जनदु॒ स्रिया॑सु । \newline
79. सो॒मा॒पू॒षभ्या॒मिति॑ सोमापू॒ष - भ्या॒म् । \newline
80. ज॒न॒ दु॒स्रिया॑सू॒ स्रिया॑सु जनज् जन दु॒स्रिया॑सु । \newline
81. उ॒स्रि॒या॒स्वित्यु॒स्रिया॑सु । \newline
\pagebreak


\end{document}