\documentclass[17pt]{extarticle}
\usepackage{babel}
\usepackage{fontspec}
\usepackage{polyglossia}
\usepackage{extsizes}

\usepackage{color}   %May be necessary if you want to color links
\usepackage{hyperref}
\hypersetup{
    colorlinks=true, %set true if you want colored links
    linktoc=all,     %set to all if you want both sections and subsections linked
    linkcolor=black,  %choose some color if you want links to stand out
}

\setmainlanguage{sanskrit}
\setotherlanguages{english} %% or other languages
\setlength{\parindent}{0pt}
\pagestyle{myheadings}
\newfontfamily\devanagarifont[Script=Devanagari]{AdishilaVedic}
\renewcommand{\theHsection}{\thepart.section.\thesection}

\newcommand{\VAR}[1]{}
\newcommand{\BLOCK}[1]{}




\begin{document}
\begin{titlepage}
    \begin{center}
 
\begin{sanskrit}
    { \Large
    कृष्ण यजुर्वेदीय तैत्तिरीय संहिता,पद,जटा,घन पाठः 
    }
    \\
    \vspace{2.5cm}
    \mbox{ \Large
    4.2      चतुर्थकाण्डे द्वीतीयः प्रश्नः - देवयजनग्रहाभिधानं   }
\end{sanskrit}
\end{center}

\end{titlepage}
\tableofcontents
\phantomsection
\pagebreak

\markright{ TS 4.2.1.1  \hfill https://www.vedavms.in \hfill}

\section{ TS 4.2.1.1 }

\textbf{TS 4.2.1.1 } \newline
\textbf{Samhita Paata} \newline

विष्णोः॒ क्रमो᳚ऽस्यभिमाति॒हा गा॑य॒त्रं छन्द॒ आ रो॑ह पृथि॒वीमनु॒ विक्र॑मस्व॒ निर्भ॑क्तः॒ स यं द्वि॒ष्मो विष्णोः॒ क्रमो᳚ऽस्यभिशस्ति॒हा त्रैष्टु॑भं॒ छन्द॒ आ रो॑हा॒न्तरि॑क्ष॒मनु॒ विक्र॑मस्व॒ निर्भ॑क्तः॒ स यं द्वि॒ष्मो विष्णोः॒ क्रमो᳚ऽस्यरातीय॒तो ह॒न्ता जाग॑तं॒ छन्द॒ आ रो॑ह॒ दिव॒मनु॒ विक्र॑मस्व॒ निर्भ॑क्तः॒ स यं द्वि॒ष्मो विष्णोः॒ - [  ] \newline

\textbf{Pada Paata} \newline

विष्णोः᳚ । क्रमः॑ । अ॒सि॒ । अ॒भि॒मा॒ति॒हेत्य॑भिमाति - हा । गा॒य॒त्रम् । छन्दः॑ । एति॑ । रो॒ह॒ । पृ॒थि॒वीम् । अनु॑ । वीति॑ । क्र॒म॒स्व॒ । निर्भ॑क्त॒ इति॒ निः - भ॒क्तः॒ । सः । यम् । द्वि॒ष्मः । विष्णोः᳚ । क्रमः॑ । अ॒सि॒ । अ॒भि॒श॒स्ति॒हेत्य॑भिशस्ति - हा । त्रैष्टु॑भम् । छन्दः॑ । एति॑ । रो॒ह॒ । अ॒न्तरि॑क्षम् । अनु॑ । वीति॑ । क्र॒म॒स्व॒ । निर्भ॑क्त॒ इति॒ निः - भ॒क्तः॒ । सः । यम् । द्वि॒ष्मः । विष्णोः᳚ । क्रमः॑ । अ॒सि॒ । अ॒रा॒ती॒य॒तः । ह॒न्ता । जाग॑तम् । छन्दः॑ । एति॑ । रो॒ह॒ । दिव᳚म् । अनु॑ । वीति॑ । क्र॒म॒स्व॒ । निर्भ॑क्त॒ इति॒ निः - भ॒क्तः॒ । सः । यम् । द्वि॒ष्मः । विष्णोः᳚ ।  \newline


\textbf{Krama Paata} \newline

विष्णोः॒ क्रमः॑ । क्रमो॑ ऽसि । अ॒स्य॒भि॒मा॒ति॒हा । अ॒भि॒मा॒ति॒हा गा॑य॒त्रम् । अ॒भि॒मा॒ति॒हेत्य॑भिमाति - हा । गा॒य॒त्रम् छन्दः॑ । छन्द॒ आ । आ रो॑ह । रो॒ह॒ पृ॒थि॒वीम् । पृ॒थि॒वीमनु॑ । अनु॒ वि । वि क्र॑मस्व । क्र॒म॒स्व॒ निर्भ॑क्तः । निर्भ॑क्तः॒ सः । निर्भ॑क्त॒ इति॒ निः - भ॒क्तः॒ । स यम् । यम् द्वि॒ष्मः । द्वि॒ष्मो विष्णोः᳚ । विष्णोः॒ क्रमः॑ । क्रमो॑ऽसि । अ॒स्य॒भि॒श॒स्ति॒हा । अ॒भि॒श॒स्ति॒हा त्रैष्टु॑भम् । अ॒भि॒श॒स्ति॒,हेत्य॑भिशस्ति - हा । त्रैष्टु॑भ॒म् छन्दः॑ । छन्द॒ आ । आ रो॑ह । रो॒हा॒न्तरि॑क्षम् । अ॒न्तरि॑क्ष॒मनु॑ । अनु॒ वि । वि क्र॑मस्व । क्र॒म॒स्व॒ निर्भ॑क्तः । निर्भ॑क्तः॒ सः । निर्भ॑क्त॒ इति॒ निः - भ॒क्तः॒ । 
स यम् । यम् द्वि॒ष्मः । द्वि॒ष्मो विष्णोः᳚ । विष्णोः॒ क्रमः॑ । क्रमो॑ऽसि । अ॒स्य॒रा॒ती॒य॒तः । अ॒रा॒ती॒य॒तो ह॒न्ता । ह॒न्ता जाग॑तम् । जाग॑त॒म् छन्दः॑ । छन्द॒ आ । आ रो॑ह । रो॒ह॒ दिव᳚म् । दिव॒मनु॑ । अनु॒ वि । वि क्र॑मस्व । क्र॒म॒स्व॒ निर्भ॑क्तः । निर्भ॑क्तः॒ सः । निर्भ॑क्त॒ इति॒ निः - भ॒क्तः॒ । स यम् । यम् द्वि॒ष्मः । द्वि॒ष्मो विष्णोः᳚ । विष्णोः॒ क्रमः॑ \newline

\textbf{Jatai Paata} \newline

1. विष्णोः॒ क्रमः॒ क्रमो॒ विष्णो॒र् विष्णोः॒ क्रमः॑ । \newline
2. क्रमो᳚ ऽस्यसि॒ क्रमः॒ क्रमो॑ ऽसि । \newline
3. अ॒स्य॒ भि॒मा॒ति॒हा ऽभि॑माति॒हा ऽस्य॑ स्यभिमाति॒हा । \newline
4. अ॒भि॒मा॒ति॒हा गा॑य॒त्रम् गा॑य॒त्र म॑भिमाति॒हा ऽभि॑माति॒हा गा॑य॒त्रम् । \newline
5. अ॒भि॒मा॒ति॒हेत्य॑भिमाति - हा । \newline
6. गा॒य॒त्रम् छन्द॒ श्छन्दो॑ गाय॒त्रम् गा॑य॒त्रम् छन्दः॑ । \newline
7. छन्द॒ आऽऽ च्छन्द॒ श्छन्द॒ आ । \newline
8. आ रो॑ह रो॒हा रो॑ह । \newline
9. रो॒ह॒ पृ॒थि॒वीम् पृ॑थि॒वीꣳ रो॑ह रोह पृथि॒वीम् । \newline
10. पृ॒थि॒वी मन्वनु॑ पृथि॒वीम् पृ॑थि॒वी मनु॑ । \newline
11. अनु॒ वि व्यन् वनु॒ वि । \newline
12. वि क्र॑मस्व क्रमस्व॒ वि वि क्र॑मस्व । \newline
13. क्र॒म॒स्व॒ निर्भ॑क्तो॒ निर्भ॑क्तः क्रमस्व क्रमस्व॒ निर्भ॑क्तः । \newline
14. निर्भ॑क्तः॒ स स निर्भ॑क्तो॒ निर्भ॑क्तः॒ सः । \newline
15. निर्भ॑क्त॒ इति॒ निः - भ॒क्तः॒ । \newline
16. स यं ॅयꣳ स स यम् । \newline
17. यम् द्वि॒ष्मो द्वि॒ष्मो यं ॅयम् द्वि॒ष्मः । \newline
18. द्वि॒ष्मो विष्णो॒र् विष्णो᳚र् द्वि॒ष्मो द्वि॒ष्मो विष्णोः᳚ । \newline
19. विष्णोः॒ क्रमः॒ क्रमो॒ विष्णो॒र् विष्णोः॒ क्रमः॑ । \newline
20. क्रमो᳚ ऽस्यसि॒ क्रमः॒ क्रमो॑ ऽसि । \newline
21. अ॒स्य॒भि॒श॒स्ति॒हा ऽभि॑शस्ति॒हा ऽस्य॑ स्यभिशस्ति॒हा । \newline
22. अ॒भि॒श॒स्ति॒हा त्रैष्टु॑भ॒म् त्रैष्टु॑भ मभिशस्ति॒हा ऽभि॑शस्ति॒हा त्रैष्टु॑भम् । \newline
23. अ॒भि॒श॒स्ति॒हेत्य॑भिशस्ति - हा । \newline
24. त्रैष्टु॑भ॒म् छन्द॒ श्छन्द॒ स्त्रैष्टु॑भ॒म् त्रैष्टु॑भ॒म् छन्दः॑ । \newline
25. छन्द॒ आऽऽ च्छन्द॒ श्छन्द॒ आ । \newline
26. आ रो॑ह रो॒हा रो॑ह । \newline
27. रो॒हा॒ न्तरि॑क्ष म॒न्तरि॑क्षꣳ रोह रोहा॒ न्तरि॑क्षम् । \newline
28. अ॒न्तरि॑क्ष॒ मन्वन् व॒न्तरि॑क्ष म॒न्तरि॑क्ष॒ मनु॑ । \newline
29. अनु॒ वि व्यन् वनु॒ वि । \newline
30. वि क्र॑मस्व क्रमस्व॒ वि वि क्र॑मस्व । \newline
31. क्र॒म॒स्व॒ निर्भ॑क्तो॒ निर्भ॑क्तः क्रमस्व क्रमस्व॒ निर्भ॑क्तः । \newline
32. निर्भ॑क्तः॒ स स निर्भ॑क्तो॒ निर्भ॑क्तः॒ सः । \newline
33. निर्भ॑क्त॒ इति॒ निः - भ॒क्तः॒ । \newline
34. स यं ॅयꣳ स स यम् । \newline
35. यम् द्वि॒ष्मो द्वि॒ष्मो यं ॅयम् द्वि॒ष्मः । \newline
36. द्वि॒ष्मो विष्णो॒र् विष्णो᳚र् द्वि॒ष्मो द्वि॒ष्मो विष्णोः᳚ । \newline
37. विष्णोः॒ क्रमः॒ क्रमो॒ विष्णो॒र् विष्णोः॒ क्रमः॑ । \newline
38. क्रमो᳚ ऽस्यसि॒ क्रमः॒ क्रमो॑ ऽसि । \newline
39. अ॒स्य॒रा॒ती॒य॒तो॑ ऽरातीय॒तो᳚ ऽस्य स्यरातीय॒तः । \newline
40. अ॒रा॒ती॒य॒तो ह॒न्ता ह॒न्ता ऽरा॑तीय॒तो॑ ऽरातीय॒तो ह॒न्ता । \newline
41. ह॒न्ता जाग॑त॒म् जाग॑तꣳ ह॒न्ता ह॒न्ता जाग॑तम् । \newline
42. जाग॑त॒म् छन्द॒ श्छन्दो॒ जाग॑त॒म् जाग॑त॒म् छन्दः॑ । \newline
43. छन्द॒ आऽऽ च्छन्द॒ श्छन्द॒ आ । \newline
44. आ रो॑ह रो॒हा रो॑ह । \newline
45. रो॒ह॒ दिव॒म् दिवꣳ॑ रोह रोह॒ दिव᳚म् । \newline
46. दिव॒ मन्वनु॒ दिव॒म् दिव॒ मनु॑ । \newline
47. अनु॒ वि व्यन् वनु॒ वि । \newline
48. वि क्र॑मस्व क्रमस्व॒ वि वि क्र॑मस्व । \newline
49. क्र॒म॒स्व॒ निर्भ॑क्तो॒ निर्भ॑क्तः क्रमस्व क्रमस्व॒ निर्भ॑क्तः । \newline
50. निर्भ॑क्तः॒ स स निर्भ॑क्तो॒ निर्भ॑क्तः॒ सः । \newline
51. निर्भ॑क्त॒ इति॒ निः - भ॒क्तः॒ । \newline
52. स यं ॅयꣳ स स यम् । \newline
53. यम् द्वि॒ष्मो द्वि॒ष्मो यं ॅयम् द्वि॒ष्मः । \newline
54. द्वि॒ष्मो विष्णो॒र् विष्णो᳚र् द्वि॒ष्मो द्वि॒ष्मो विष्णोः᳚ । \newline
55. विष्णोः॒ क्रमः॒ क्रमो॒ विष्णो॒र् विष्णोः॒ क्रमः॑ । \newline

\textbf{Ghana Paata } \newline

1. विष्णोः॒ क्रमः॒ क्रमो॒ विष्णो॒र् विष्णोः॒ क्रमो᳚ ऽस्यसि॒ क्रमो॒ विष्णो॒र् विष्णोः॒ क्रमो॑ ऽसि । \newline
2. क्रमो᳚ ऽस्यसि॒ क्रमः॒ क्रमो᳚ ऽस्यभि माति॒हा ऽभि॑माति॒हा ऽसि॒ क्रमः॒ क्रमो᳚ ऽस्यभि माति॒हा । \newline
3. अ॒स्य॒ भि॒मा॒ति॒हा ऽभि॑माति॒हा ऽस्य॑ स्यभिमाति॒हा गा॑य॒त्रम् गा॑य॒त्र म॑भिमाति॒हा ऽस्य॑ स्यभिमाति॒हा गा॑य॒त्रम् । \newline
4. अ॒भि॒मा॒ति॒हा गा॑य॒त्रम् गा॑य॒त्र म॑भिमाति॒हा ऽभि॑माति॒हा गा॑य॒त्रम् छन्द॒ श्छन्दो॑ गाय॒त्र म॑भिमाति॒हा ऽभि॑माति॒हा गा॑य॒त्रम् छन्दः॑ । \newline
5. अ॒भि॒मा॒ति॒हेत्य॑भिमाति - हा । \newline
6. गा॒य॒त्रम् छन्द॒ श्छन्दो॑ गाय॒त्रम् गा॑य॒त्रम् छन्द॒ आऽऽ च्छन्दो॑ गाय॒त्रम् गा॑य॒त्रम् छन्द॒ आ । \newline
7. छन्द॒ आऽऽ च्छन्द ॒श्छन्द॒ आ रो॑ह रो॒हाच् छन्द॒ श्छन्द॒ आ रो॑ह । \newline
8. आ रो॑ह रो॒हा रो॑ह पृथि॒वीम् पृ॑थि॒वीꣳ रो॒हा रो॑ह पृथि॒वीम् । \newline
9. रो॒ह॒ पृ॒थि॒वीम् पृ॑थि॒वीꣳ रो॑ह रोह पृथि॒वी मन्वनु॑ पृथि॒वीꣳ रो॑ह रोह पृथि॒वी मनु॑ । \newline
10. पृ॒थि॒वी मन्वनु॑ पृथि॒वीम् पृ॑थि॒वी मनु॒ वि व्यनु॑ पृथि॒वीम् पृ॑थि॒वी मनु॒ वि । \newline
11. अनु॒ वि व्यन् वनु॒ वि क्र॑मस्व क्रमस्व॒ व्यन् वनु॒ वि क्र॑मस्व । \newline
12. वि क्र॑मस्व क्रमस्व॒ वि वि क्र॑मस्व॒ निर्भ॑क्तो॒ निर्भ॑क्तः क्रमस्व॒ वि वि क्र॑मस्व॒ निर्भ॑क्तः । \newline
13. क्र॒म॒स्व॒ निर्भ॑क्तो॒ निर्भ॑क्तः क्रमस्व क्रमस्व॒ निर्भ॑क्तः॒ स स निर्भ॑क्तः क्रमस्व क्रमस्व॒ निर्भ॑क्तः॒ सः । \newline
14. निर्भ॑क्तः॒ स स निर्भ॑क्तो॒ निर्भ॑क्तः॒ स यं ॅयꣳ स निर्भ॑क्तो॒ निर्भ॑क्तः॒ स यम् । \newline
15. निर्भ॑क्त॒ इति॒ निः - भ॒क्तः॒ । \newline
16. स यं ॅयꣳ स स यम् द्वि॒ष्मो द्वि॒ष्मो यꣳ स स यम् द्वि॒ष्मः । \newline
17. यम् द्वि॒ष्मो द्वि॒ष्मो यं ॅयम् द्वि॒ष्मो विष्णो॒र् विष्णो᳚र् द्वि॒ष्मो यं ॅयम् द्वि॒ष्मो विष्णोः᳚ । \newline
18. द्वि॒ष्मो विष्णो॒र् विष्णो᳚र् द्वि॒ष्मो द्वि॒ष्मो विष्णोः॒ क्रमः॒ क्रमो॒ विष्णो᳚र् द्वि॒ष्मो द्वि॒ष्मो विष्णोः॒ क्रमः॑ । \newline
19. विष्णोः॒ क्रमः॒ क्रमो॒ विष्णो॒र् विष्णोः॒ क्रमो᳚ ऽस्यसि॒ क्रमो॒ विष्णो॒र् विष्णोः॒ क्रमो॑ ऽसि । \newline
20. क्रमो᳚ ऽस्यसि॒ क्रमः॒ क्रमो᳚ ऽस्य भिशस्ति॒हा ऽभि॑शस्ति॒हा ऽसि॒ क्रमः॒ क्रमो᳚ ऽस्य भिशस्ति॒हा । \newline
21. अ॒स्य॒भि॒श॒स्ति॒हा ऽभि॑शस्ति॒हा ऽस्य॑ स्यभिशस्ति॒हा त्रैष्टु॑भ॒म् त्रैष्टु॑भ मभिशस्ति॒हा ऽस्य॑स्य भिशस्ति॒हा त्रैष्टु॑भम् । \newline
22. अ॒भि॒श॒स्ति॒हा त्रैष्टु॑भ॒म् त्रैष्टु॑भ मभिशस्ति॒हा ऽभि॑शस्ति॒हा त्रैष्टु॑भ॒म् छन्द॒ श्छन्द॒ स्त्रैष्टु॑भ मभिशस्ति॒हा ऽभि॑शस्ति॒हा त्रैष्टु॑भ॒म् छन्दः॑ । \newline
23. अ॒भि॒श॒स्ति॒हेत्य॑भिशस्ति - हा । \newline
24. त्रैष्टु॑भ॒म् छन्द॒ श्छन्द॒ स्त्रैष्टु॑भ॒म् त्रैष्टु॑भ॒म् छन्द॒ आऽऽ च्छन्द॒ स्त्रैष्टु॑भ॒म् त्रैष्टु॑भ॒म् छन्द॒ आ । \newline
25. छन्द॒ आऽऽ च्छन्द॒ श्छन्द॒ आ रो॑ह रो॒हा च्छन्द॒ श्छन्द॒ आ रो॑ह । \newline
26. आ रो॑ह रो॒हा रो॑हा॒न्तरि॑क्ष म॒न्तरि॑क्षꣳ रो॒हा रो॑हा॒न्तरि॑क्षम् । \newline
27. रो॒हा॒न्तरि॑क्ष म॒न्तरि॑क्षꣳ रोह रोहा॒न्तरि॑क्ष॒ मन्वन् व॒न्तरि॑क्षꣳ रोह रोहा॒न्तरि॑क्ष॒ मनु॑ । \newline
28. अ॒न्तरि॑क्ष॒ मन्वन् व॒न्तरि॑क्ष म॒न्तरि॑क्ष॒ मनु॒ वि व्यन् व॒न्तरि॑क्ष म॒न्तरि॑क्ष॒ मनु॒ वि । \newline
29. अनु॒ वि व्यन् वनु॒ वि क्र॑मस्व क्रमस्व॒ व्यन् वनु॒ वि क्र॑मस्व । \newline
30. वि क्र॑मस्व क्रमस्व॒ वि वि क्र॑मस्व॒ निर्भ॑क्तो॒ निर्भ॑क्तः क्रमस्व॒ वि वि क्र॑मस्व॒ निर्भ॑क्तः । \newline
31. क्र॒म॒स्व॒ निर्भ॑क्तो॒ निर्भ॑क्तः क्रमस्व क्रमस्व॒ निर्भ॑क्तः॒ स स निर्भ॑क्तः क्रमस्व क्रमस्व॒ निर्भ॑क्तः॒ सः । \newline
32. निर्भ॑क्तः॒ स स निर्भ॑क्तो॒ निर्भ॑क्तः॒ स यं ॅयꣳ स निर्भ॑क्तो॒ निर्भ॑क्तः॒ स यम् । \newline
33. निर्भ॑क्त॒ इति॒ निः - भ॒क्तः॒ । \newline
34. स यं ॅयꣳ स स यम् द्वि॒ष्मो द्वि॒ष्मो यꣳ स स यम् द्वि॒ष्मः । \newline
35. यम् द्वि॒ष्मो द्वि॒ष्मो यं ॅयम् द्वि॒ष्मो विष्णो॒र् विष्णो᳚र् द्वि॒ष्मो यं ॅयम् द्वि॒ष्मो विष्णोः᳚ । \newline
36. द्वि॒ष्मो विष्णो॒र् विष्णो᳚र् द्वि॒ष्मो द्वि॒ष्मो विष्णोः॒ क्रमः॒ क्रमो॒ विष्णो᳚र् द्वि॒ष्मो द्वि॒ष्मो विष्णोः॒ क्रमः॑ । \newline
37. विष्णोः॒ क्रमः॒ क्रमो॒ विष्णो॒र् विष्णोः॒ क्रमो᳚ ऽस्यसि॒ क्रमो॒ विष्णो॒र् विष्णोः॒ क्रमो॑ ऽसि । \newline
38. क्रमो᳚ ऽस्यसि॒ क्रमः॒ क्रमो᳚ ऽस्यरातीय॒तो॑ ऽरातीय॒तो॑ ऽसि॒ क्रमः॒ क्रमो᳚ ऽस्यरातीय॒तः । \newline
39. अ॒स्य॒रा॒ती॒य॒तो॑ ऽरातीय॒तो᳚ ऽस्य स्यरातीय॒तो ह॒न्ता ह॒न्ता ऽरा॑तीय॒तो᳚ ऽस्य स्यरातीय॒तो ह॒न्ता । \newline
40. अ॒रा॒ती॒य॒तो ह॒न्ता ह॒न्ता ऽरा॑तीय॒तो॑ ऽरातीय॒तो ह॒न्ता जाग॑त॒म् जाग॑तꣳ ह॒न्ता ऽरा॑तीय॒तो॑ ऽरातीय॒तो ह॒न्ता जाग॑तम् । \newline
41. ह॒न्ता जाग॑त॒म् जाग॑तꣳ ह॒न्ता ह॒न्ता जाग॑त॒म् छन्द॒ श्छन्दो॒ जाग॑तꣳ ह॒न्ता ह॒न्ता जाग॑त॒म् छन्दः॑ । \newline
42. जाग॑त॒म् छन्द॒ श्छन्दो॒ जाग॑त॒म् जाग॑त॒म् छन्द॒ आऽऽ च्छन्दो॒ जाग॑त॒म् जाग॑त॒म् छन्द॒ आ । \newline
43. छन्द॒ आऽऽ च्छन्द॒ श्छन्द॒ आ रो॑ह रो॒हा च्छन्द॒ श्छन्द॒ आ रो॑ह । \newline
44. आ रो॑ह रो॒हा रो॑ह॒ दिव॒म् दिवꣳ॑ रो॒हा रो॑ह॒ दिव᳚म् । \newline
45. रो॒ह॒ दिव॒म् दिवꣳ॑ रोह रोह॒ दिव॒ मन्वनु॒ दिवꣳ॑ रोह रोह॒ दिव॒ मनु॑ । \newline
46. दिव॒ मन्वनु॒ दिव॒म् दिव॒ मनु॒ वि व्यनु॒ दिव॒म् दिव॒ मनु॒ वि । \newline
47. अनु॒ वि व्यन् वनु॒ वि क्र॑मस्व क्रमस्व॒ व्यन् वनु॒ वि क्र॑मस्व । \newline
48. वि क्र॑मस्व क्रमस्व॒ वि वि क्र॑मस्व॒ निर्भ॑क्तो॒ निर्भ॑क्तः क्रमस्व॒ वि वि क्र॑मस्व॒ निर्भ॑क्तः । \newline
49. क्र॒म॒स्व॒ निर्भ॑क्तो॒ निर्भ॑क्तः क्रमस्व क्रमस्व॒ निर्भ॑क्तः॒ स स निर्भ॑क्तः क्रमस्व क्रमस्व॒ निर्भ॑क्तः॒ सः । \newline
50. निर्भ॑क्तः॒ स स निर्भ॑क्तो॒ निर्भ॑क्तः॒ स यं ॅयꣳ स निर्भ॑क्तो॒ निर्भ॑क्तः॒ स यम् । \newline
51. निर्भ॑क्त॒ इति॒ निः - भ॒क्तः॒ । \newline
52. स यं ॅयꣳ स स यम् द्वि॒ष्मो द्वि॒ष्मो यꣳ स स यम् द्वि॒ष्मः । \newline
53. यम् द्वि॒ष्मो द्वि॒ष्मो यं ॅयम् द्वि॒ष्मो विष्णो॒र् विष्णो᳚र् द्वि॒ष्मो यं ॅयम् द्वि॒ष्मो विष्णोः᳚ । \newline
54. द्वि॒ष्मो विष्णो॒र् विष्णो᳚र् द्वि॒ष्मो द्वि॒ष्मो विष्णोः॒ क्रमः॒ क्रमो॒ विष्णो᳚र् द्वि॒ष्मो द्वि॒ष्मो विष्णोः॒ क्रमः॑ । \newline
55. विष्णोः॒ क्रमः॒ क्रमो॒ विष्णो॒र् विष्णोः॒ क्रमो᳚ ऽस्यसि॒ क्रमो॒ विष्णो॒र् विष्णोः॒ क्रमो॑ ऽसि । \newline
\pagebreak
\markright{ TS 4.2.1.2  \hfill https://www.vedavms.in \hfill}

\section{ TS 4.2.1.2 }

\textbf{TS 4.2.1.2 } \newline
\textbf{Samhita Paata} \newline

क्रमो॑ऽसि शत्रूय॒तो ह॒न्ताऽनु॑ष्टुभं॒ छन्द॒ आ रो॑ह॒ दिशोऽनु॒ विक्र॑मस्व॒ निर्भ॑क्तः॒ स यं द्वि॒ष्मः ॥ अक्र॑न्दद॒ग्निः स्त॒नय॑न्निव॒ द्यौः क्षामा॒ रेरि॑हद्-वी॒रुधः॑ सम॒ञ्जन्न् । स॒द्यो ज॑ज्ञा॒नो वि हीमि॒द्धो अख्य॒दा रोद॑सी भा॒नुना॑ भात्य॒न्तः ॥ अग्ने᳚ऽभ्यावर्तिन्न॒भि न॒ आ व॑र्त॒स्वाऽऽयु॑षा॒ वर्च॑सा स॒न्या मे॒धया᳚ प्र॒जया॒ धने॑न ॥ अग्ने॑ - [  ] \newline

\textbf{Pada Paata} \newline

क्रमः॑ । अ॒सि॒ । श॒त्रू॒य॒त इति॑ शत्रु - य॒तः । ह॒न्ता । आनु॑ष्टुभ॒मित्यानु॑ - स्तु॒भ॒म् । छन्दः॑ । एति॑ । रो॒ह॒ । दिशः॑ । अनु॑ । वीति॑ । क्र॒म॒स्व॒ । निर्भ॑क्त॒ इति॒ निः - भ॒क्तः॒ । सः । यम् । द्वि॒ष्मः ॥ अक्र॑न्दत् । अ॒ग्निः । स्त॒नयन्न्॑ । इ॒व॒ । द्यौः । क्षाम॑ । रेरि॑हत् । वी॒रुधः॑ । स॒म॒ञ्जन्निति॑ सं - अ॒ञ्जन्न् ॥ स॒द्यः । ज॒ज्ञा॒नः । वीति॑ । हि । ई॒म् । इ॒द्धः । अख्य॑त् । एति॑ । रोद॑सी॒ इति॑ । भा॒नुना᳚ । भा॒ति॒ । अ॒न्तः ॥ अग्ने᳚ । अ॒भ्या॒व॒र्ति॒न्नित्य॑भि - आ॒व॒र्ति॒न्न् । अ॒भीति॑ । नः॒ । एति॑ । व॒र्त॒स्व॒ । आयु॑षा । वर्च॑सा । स॒न्या । मे॒धया᳚ । प्र॒जयेति॑ प्र - जया᳚ । धने॑न ॥ अग्ने᳚ ।  \newline


\textbf{Krama Paata} \newline

क्रमो॑ऽसि । अ॒सि॒ श॒त्रू॒य॒तः । श॒त्रू॒य॒तो ह॒न्ता । श॒त्रू॒य॒त इति॑ शत्रु - य॒तः । ह॒न्ता ऽऽनु॑ष्टुभम् । आनु॑ष्टुभ॒म् छन्दः॑ । आनु॑ष्टभ॒मित्यानु॑ - स्तु॒भ॒म् । छन्द॒ आ । आ रो॑ह । रो॒ह॒ दिशः॑ । दिशोऽनु॑ । अनु॒ वि । वि क्र॑मस्व । क्र॒म॒स्व॒ निर्भ॑क्तः । निर्भ॑क्तः॒ सः । निर्भ॑क्त॒ इति॒ निः - भ॒क्तः॒ । स यम् । यम् द्वि॒ष्मः । द्वि॒ष्म इति॑ द्वि॒ष्मः ॥ अक्र॑न्दद॒ग्निः । अ॒ग्निः स्त॒नयन्न्॑ । स्त॒नय॑न्निव । इ॒व॒ द्यौः । द्यौः क्षाम॑ । क्षामा॒ रेरि॑हत् । रेरि॑हद् वी॒रुधः॑ । वी॒रुधः॑ सम॒ञ्जन्न् । स॒म॒ञ्जन्निति॑ सम् - अ॒ञ्जन्न् ॥ स॒द्यो ज॑ज्ञा॒नः । ज॒ज्ञा॒नो वि । वि हि । हीम् । ई॒मि॒द्धः । इ॒द्धो अख्य॑त् । 
अख्य॒दा । आ रोद॑सी । रोद॑सी भा॒नुना᳚ । रोद॑सी॒ इति॒ रोद॑सी । 
भा॒नुना॑ भाति । भा॒त्य॒न्तः । अ॒न्तरित्य॒न्तः ॥ अग्ने᳚ ऽभ्यावर्तिन्न् । अ॒भ्या॒व॒र्ति॒न्न॒भि । अ॒भ्या॒व॒र्ति॒न्नित्य॑भि - आ॒व॒र्ति॒न्न्॒ । अ॒भि नः॑ । न॒ आ । आ व॑र्तस्व । व॒र्त॒स्वायु॑षा । आयु॑षा॒ वर्च॑सा । वर्च॑सा स॒न्या । स॒न्या मे॒धया᳚ । मे॒धया᳚ प्र॒जया᳚ । प्र॒जया॒ धने॑न । प्र॒जयेति॑ प्र - जया᳚ । धने॒नेति॒ धने॑न ॥ अग्ने॑ अङ्गिरः \newline

\textbf{Jatai Paata} \newline

1. क्रमो᳚ ऽस्यसि॒ क्रमः॒ क्रमो॑ ऽसि । \newline
2. अ॒सि॒ श॒त्रू॒य॒तः श॑त्रूय॒तो᳚ ऽस्यसि शत्रूय॒तः । \newline
3. श॒त्रू॒य॒तो ह॒न्ता ह॒न्ता श॑त्रूय॒तः श॑त्रूय॒तो ह॒न्ता । \newline
4. श॒त्रू॒य॒त इति॑ शत्रु - य॒तः । \newline
5. ह॒न्ता ऽऽनु॑ष्टुभ॒ मानु॑ष्टुभꣳ ह॒न्ता ह॒न्ता ऽऽनु॑ष्टुभम् । \newline
6. आनु॑ष्टुभ॒म् छन्द॒ श्छन्द॒ आनु॑ष्टुभ॒ मानु॑ष्टुभ॒म् छन्दः॑ । \newline
7. आनु॑ष्टुभ॒मित्यानु॑ - स्तु॒भ॒म् । \newline
8. छन्द॒ आऽऽ च्छन्द॒ श्छन्द॒ आ । \newline
9. आ रो॑ह रो॒हा रो॑ह । \newline
10. रो॒ह॒ दिशो॒ दिशो॑ रोह रोह॒ दिशः॑ । \newline
11. दिशो ऽन्वनु॒ दिशो॒ दिशो ऽनु॑ । \newline
12. अनु॒ वि व्यन् वनु॒ वि । \newline
13. वि क्र॑मस्व क्रमस्व॒ वि वि क्र॑मस्व । \newline
14. क्र॒म॒स्व॒ निर्भ॑क्तो॒ निर्भ॑क्तः क्रमस्व क्रमस्व॒ निर्भ॑क्तः । \newline
15. निर्भ॑क्तः॒ स स निर्भ॑क्तो॒ निर्भ॑क्तः॒ सः । \newline
16. निर्भ॑क्त॒ इति॒ निः - भ॒क्तः॒ । \newline
17. स यं ॅयꣳ स स यम् । \newline
18. यम् द्वि॒ष्मो द्वि॒ष्मो यं ॅयम् द्वि॒ष्मः । \newline
19. द्वि॒ष्म इति॑ द्वि॒ष्मः । \newline
20. अक्र॑न्द द॒ग्नि र॒ग्नि रक्र॑न्द॒ दक्र॑न्द द॒ग्निः । \newline
21. अ॒ग्निः स्त॒नयन्᳚ थ्स्त॒नय॑न् न॒ग्नि र॒ग्निः स्त॒नयन्न्॑ । \newline
22. स्त॒नय॑न् निवे व स्त॒नयन्᳚ थ्स्त॒नय॑न् निव । \newline
23. इ॒व॒ द्यौर् द्यौ रि॑वे व॒ द्यौः । \newline
24. द्यौः क्षाम॒ क्षाम॒ द्यौर् द्यौः क्षाम॑ । \newline
25. क्षामा॒ रेरि॑ह॒द् रेरि॑ह॒त् क्षाम॒ क्षामा॒ रेरि॑हत् । \newline
26. रेरि॑हद् वी॒रुधो॑ वी॒रुधो॒ रेरि॑ह॒द् रेरि॑हद् वी॒रुधः॑ । \newline
27. वी॒रुधः॑ सम॒ञ्जन् थ्स॑म॒ञ्जन्. वी॒रुधो॑ वी॒रुधः॑ सम॒ञ्जन्न् । \newline
28. स॒म॒ञ्जन्निति॑ सं - अ॒ञ्जन्न् । \newline
29. स॒द्यो ज॑ज्ञा॒नो ज॑ज्ञा॒नः स॒द्यः स॒द्यो ज॑ज्ञा॒नः । \newline
30. ज॒ज्ञा॒नो वि वि ज॑ज्ञा॒नो ज॑ज्ञा॒नो वि । \newline
31. वि हि हि वि वि हि । \newline
32. ही मीꣳ॒॒ हि हीम् । \newline
33. ई॒ मि॒द्ध इ॒द्ध ई॑ मी मि॒द्धः । \newline
34. इ॒द्धो अख्य॒ दख्य॑ दि॒द्ध इ॒द्धो अख्य॑त् । \newline
35. अख्य॒दा ऽख्य॒ दख्य॒दा । \newline
36. आ रोद॑सी॒ रोद॑सी॒ आ रोद॑सी । \newline
37. रोद॑सी भा॒नुना॑ भा॒नुना॒ रोद॑सी॒ रोद॑सी भा॒नुना᳚ । \newline
38. रोद॑सी॒ इति॒ रोद॑सी । \newline
39. भा॒नुना॑ भाति भाति भा॒नुना॑ भा॒नुना॑ भाति । \newline
40. भा॒त्य॒न्त र॒न्तर् भा॑ति भात्य॒न्तः । \newline
41. अ॒न्तरित्य॒न्तः । \newline
42. अग्ने᳚ ऽभ्यावर्तिन् नभ्यावर्ति॒न् नग्ने ऽग्ने᳚ ऽभ्यावर्तिन्न् । \newline
43. अ॒भ्या॒व॒र्ति॒न् न॒भ्या᳚(1॒)भ्य॑भ्यावर्तिन् नभ्यावर्तिन् न॒भि । \newline
44. अ॒भ्या॒व॒र्ति॒न्नित्य॑भि - आ॒व॒र्ति॒न्न् । \newline
45. अ॒भि नो॑ नो अ॒भ्य॑भि नः॑ । \newline
46. न॒ आ नो॑ न॒ आ । \newline
47. आ व॑र्तस्व वर्त॒स्वा व॑र्तस्व । \newline
48. व॒र्त॒स्वायु॒षा ऽऽयु॑षा वर्तस्व वर्त॒स्वायु॑षा । \newline
49. आयु॑षा॒ वर्च॑सा॒ वर्च॒सा ऽऽयु॒षा ऽऽयु॑षा॒ वर्च॑सा । \newline
50. वर्च॑सा स॒न्या स॒न्या वर्च॑सा॒ वर्च॑सा स॒न्या । \newline
51. स॒न्या मे॒धया॑ मे॒धया॑ स॒न्या स॒न्या मे॒धया᳚ । \newline
52. मे॒धया᳚ प्र॒जया᳚ प्र॒जया॑ मे॒धया॑ मे॒धया᳚ प्र॒जया᳚ । \newline
53. प्र॒जया॒ धने॑न॒ धने॑न प्र॒जया᳚ प्र॒जया॒ धने॑न । \newline
54. प्र॒जयेति॑ प्र - जया᳚ । \newline
55. धने॒नेति॒ धने॑न । \newline
56. अग्ने॑ अङ्गिरो अङ्गि॒रो ऽग्ने ऽग्ने॑ अङ्गिरः । \newline

\textbf{Ghana Paata } \newline

1. क्रमो᳚ ऽस्यसि॒ क्रमः॒ क्रमो॑ ऽसि शत्रूय॒तः श॑त्रूय॒तो॑ ऽसि॒ क्रमः॒ क्रमो॑ ऽसि शत्रूय॒तः । \newline
2. अ॒सि॒ श॒त्रू॒य॒तः श॑त्रूय॒तो᳚ ऽस्यसि शत्रूय॒तो ह॒न्ता ह॒न्ता श॑त्रूय॒तो᳚ ऽस्यसि शत्रूय॒तो ह॒न्ता । \newline
3. श॒त्रू॒य॒तो ह॒न्ता ह॒न्ता श॑त्रूय॒तः श॑त्रूय॒तो ह॒न्ता ऽऽनु॑ष्टुभ॒ मानु॑ष्टुभꣳ ह॒न्ता श॑त्रूय॒तः श॑त्रूय॒तो ह॒न्ता ऽऽनु॑ष्टुभम् । \newline
4. श॒त्रू॒य॒त इति॑ शत्रु - य॒तः । \newline
5. ह॒न्ता ऽऽनु॑ष्टुभ॒ मानु॑ष्टुभꣳ ह॒न्ता ह॒न्ता ऽऽनु॑ष्टुभ॒म् छन्द॒ श्छन्द॒ आनु॑ष्टुभꣳ ह॒न्ता ह॒न्ता ऽऽनु॑ष्टुभ॒म् छन्दः॑ । \newline
6. आनु॑ष्टुभ॒म् छन्द॒ श्छन्द॒ आनु॑ष्टुभ॒ मानु॑ष्टुभ॒म् छन्द॒ आऽऽ च्छन्द॒ आनु॑ष्टुभ॒ मानु॑ष्टुभ॒म् छन्द॒ आ । \newline
7. आनु॑ष्टुभ॒मित्यानु॑ - स्तु॒भ॒म् । \newline
8. छन्द॒ आऽऽ च्छन्द॒ श्छन्द॒ आ रो॑ह रो॒हा च्छन्द॒ श्छन्द॒ आ रो॑ह । \newline
9. आ रो॑ह रो॒हा रो॑ह॒ दिशो॒ दिशो॑ रो॒हा रो॑ह॒ दिशः॑ । \newline
10. रो॒ह॒ दिशो॒ दिशो॑ रोह रोह॒ दिशो ऽन्वनु॒ दिशो॑ रोह रोह॒ दिशो ऽनु॑ । \newline
11. दिशो ऽन्वनु॒ दिशो॒ दिशो ऽनु॒ वि व्यनु॒ दिशो॒ दिशो ऽनु॒ वि । \newline
12. अनु॒ वि व्यन् वनु॒ वि क्र॑मस्व क्रमस्व॒ व्यन् वनु॒ वि क्र॑मस्व । \newline
13. वि क्र॑मस्व क्रमस्व॒ वि वि क्र॑मस्व॒ निर्भ॑क्तो॒ निर्भ॑क्तः क्रमस्व॒ वि वि क्र॑मस्व॒ निर्भ॑क्तः । \newline
14. क्र॒म॒स्व॒ निर्भ॑क्तो॒ निर्भ॑क्तः क्रमस्व क्रमस्व॒ निर्भ॑क्तः॒ स स निर्भ॑क्तः क्रमस्व क्रमस्व॒ निर्भ॑क्तः॒ सः । \newline
15. निर्भ॑क्तः॒ स स निर्भ॑क्तो॒ निर्भ॑क्तः॒ स यं ॅयꣳ स निर्भ॑क्तो॒ निर्भ॑क्तः॒ स यम् । \newline
16. निर्भ॑क्त॒ इति॒ निः - भ॒क्तः॒ । \newline
17. स यं ॅयꣳ स स यम् द्वि॒ष्मो द्वि॒ष्मो यꣳ स स यम् द्वि॒ष्मः । \newline
18. यम् द्वि॒ष्मो द्वि॒ष्मो यं ॅयम् द्वि॒ष्मः । \newline
19. द्वि॒ष्म इति॑ द्वि॒ष्मः । \newline
20. अक्र॑न्द द॒ग्नि र॒ग्नि रक्र॑न्द॒ दक्र॑न्द द॒ग्निः स्त॒नयन्᳚ थ्स्त॒नय॑न् न॒ग्नि रक्र॑न्द॒ दक्र॑न्द द॒ग्निः स्त॒नयन्न्॑ । \newline
21. अ॒ग्निः स्त॒नयन्᳚ थ्स्त॒नय॑न् न॒ग्नि र॒ग्निः स्त॒नय॑न् निवेव स्त॒नय॑न् न॒ग्नि र॒ग्निः स्त॒नय॑न् निव । \newline
22. स्त॒नय॑न् निवेव स्त॒नयन्᳚ थ्स्त॒नय॑न् निव॒ द्यौर् द्यौरि॑व स्त॒नयन्᳚ थ्स्त॒नय॑न् निव॒ द्यौः । \newline
23. इ॒व॒ द्यौर् द्यौ रि॑वेव॒ द्यौः क्षाम॒ क्षाम॒ द्यौ रि॑वेव॒ द्यौः क्षाम॑ । \newline
24. द्यौः क्षाम॒ क्षाम॒ द्यौर् द्यौः क्षामा॒ रेरि॑ह॒द् रेरि॑ह॒त् क्षाम॒ द्यौर् द्यौः क्षामा॒ रेरि॑हत् । \newline
25. क्षामा॒ रेरि॑ह॒द् रेरि॑ह॒त् क्षाम॒ क्षामा॒ रेरि॑हद् वी॒रुधो॑ वी॒रुधो॒ रेरि॑ह॒त् क्षाम॒ क्षामा॒ रेरि॑हद् वी॒रुधः॑ । \newline
26. रेरि॑हद् वी॒रुधो॑ वी॒रुधो॒ रेरि॑ह॒द् रेरि॑हद् वी॒रुधः॑ सम॒ञ्जन् थ्स॑म॒ञ्जन्. वी॒रुधो॒ रेरि॑ह॒द् रेरि॑हद् वी॒रुधः॑ सम॒ञ्जन्न् । \newline
27. वी॒रुधः॑ सम॒ञ्जन् थ्स॑म॒ञ्जन्. वी॒रुधो॑ वी॒रुधः॑ सम॒ञ्जन्न् । \newline
28. स॒म॒ञ्जन्निति॑ सं - अ॒ञ्जन्न् । \newline
29. स॒द्यो ज॑ज्ञा॒नो ज॑ज्ञा॒नः स॒द्यः स॒द्यो ज॑ज्ञा॒नो वि वि ज॑ज्ञा॒नः स॒द्यः स॒द्यो ज॑ज्ञा॒नो वि । \newline
30. ज॒ज्ञा॒नो वि वि ज॑ज्ञा॒नो ज॑ज्ञा॒नो वि हि हि वि ज॑ज्ञा॒नो ज॑ज्ञा॒नो वि हि । \newline
31. वि हि हि वि वि ही मीꣳ॒॒ हि वि वि हीम् । \newline
32. ही मीꣳ॒॒ हि ही मि॒द्ध इ॒द्ध ईꣳ॒॒ हि ही मि॒द्धः । \newline
33. ई॒मि॒द्ध इ॒द्ध ई॑मी मि॒द्धो अख्य॒ दख्य॑ दि॒द्ध ई॑मी मि॒द्धो अख्य॑त् । \newline
34. इ॒द्धो अख्य॒ दख्य॑ दि॒द्ध इ॒द्धो अख्य॒दा ऽख्य॑दि॒द्ध इ॒द्धो अख्य॒दा । \newline
35. अख्य॒दा ऽख्य॒ दख्य॒दा रोद॑सी॒ रोद॑सी॒ आ ऽख्य॒ दख्य॒दा रोद॑सी । \newline
36. आ रोद॑सी॒ रोद॑सी॒ आ रोद॑सी भा॒नुना॑ भा॒नुना॒ रोद॑सी॒ आ रोद॑सी भा॒नुना᳚ । \newline
37. रोद॑सी भा॒नुना॑ भा॒नुना॒ रोद॑सी॒ रोद॑सी भा॒नुना॑ भाति भाति भा॒नुना॒ रोद॑सी॒ रोद॑सी भा॒नुना॑ भाति । \newline
38. रोद॑सी॒ इति॒ रोद॑सी । \newline
39. भा॒नुना॑ भाति भाति भा॒नुना॑ भा॒नुना॑ भात्य॒न्त र॒न्तर् भा॑ति भा॒नुना॑ भा॒नुना॑ भात्य॒न्तः । \newline
40. भा॒त्य॒न्त र॒न्तर् भा॑ति भात्य॒न्तः । \newline
41. अ॒न्तरित्य॒न्तः । \newline
42. अग्ने᳚ ऽभ्यावर्तिन् नभ्यावर्ति॒न् नग्ने ऽग्ने᳚ ऽभ्यावर्तिन् न॒भ्या᳚(1॒)भ्य॑भ्यावर्ति॒न् नग्ने ऽग्ने᳚ ऽभ्यावर्तिन् न॒भि । \newline
43. अ॒भ्या॒व॒र्ति॒न् न॒भ्या᳚(1॒)भ्य॑भ्यावर्तिन् नभ्यावर्तिन् न॒भि नो॑ नो अ॒भ्य॑भ्यावर्तिन् नभ्यावर्तिन् न॒भि नः॑ । \newline
44. अ॒भ्या॒व॒र्ति॒न्नित्य॑भि - आ॒व॒र्ति॒न्न् । \newline
45. अ॒भि नो॑ नो अ॒भ्य॑भि न॒ आ नो॑ अ॒भ्य॑भि न॒ आ । \newline
46. न॒ आ नो॑ न॒ आ व॑र्तस्व वर्त॒स्वा नो॑ न॒ आ व॑र्तस्व । \newline
47. आ व॑र्तस्व वर्त॒स्वा व॑र्त॒ स्वायु॒षा ऽऽयु॑षा वर्त॒स्वा व॑र्त॒ स्वायु॑षा । \newline
48. व॒र्त॒ स्वायु॒षा ऽऽयु॑षा वर्तस्व वर्त॒ स्वायु॑षा॒ वर्च॑सा॒ वर्च॒सा ऽऽयु॑षा वर्तस्व वर्त॒ स्वायु॑षा॒ वर्च॑सा । \newline
49. आयु॑षा॒ वर्च॑सा॒ वर्च॒सा ऽऽयु॒षा ऽऽयु॑षा॒ वर्च॑सा स॒न्या स॒न्या वर्च॒सा ऽऽयु॒षा ऽऽयु॑षा॒ वर्च॑सा स॒न्या । \newline
50. वर्च॑सा स॒न्या स॒न्या वर्च॑सा॒ वर्च॑सा स॒न्या मे॒धया॑ मे॒धया॑ स॒न्या वर्च॑सा॒ वर्च॑सा स॒न्या मे॒धया᳚ । \newline
51. स॒न्या मे॒धया॑ मे॒धया॑ स॒न्या स॒न्या मे॒धया᳚ प्र॒जया᳚ प्र॒जया॑ मे॒धया॑ स॒न्या स॒न्या मे॒धया᳚ प्र॒जया᳚ । \newline
52. मे॒धया᳚ प्र॒जया᳚ प्र॒जया॑ मे॒धया॑ मे॒धया᳚ प्र॒जया॒ धने॑न॒ धने॑न प्र॒जया॑ मे॒धया॑ मे॒धया᳚ प्र॒जया॒ धने॑न । \newline
53. प्र॒जया॒ धने॑न॒ धने॑न प्र॒जया᳚ प्र॒जया॒ धने॑न । \newline
54. प्र॒जयेति॑ प्र - जया᳚ । \newline
55. धने॒नेति॒ धने॑न । \newline
56. अग्ने॑ अङ्गिरो अङ्गि॒रो ऽग्ने ऽग्ने॑ अङ्गिरः श॒तꣳ श॒त म॑ङ्गि॒रो ऽग्ने ऽग्ने॑ अङ्गिरः श॒तम् । \newline
\pagebreak
\markright{ TS 4.2.1.3  \hfill https://www.vedavms.in \hfill}

\section{ TS 4.2.1.3 }

\textbf{TS 4.2.1.3 } \newline
\textbf{Samhita Paata} \newline

अङ्गिरः श॒तं ते॑ सन्त्वा॒वृतः॑ स॒हस्रं॑ त उपा॒वृतः॑ । तासां॒ पोष॑स्य॒ पोषे॑ण॒ पुन॑र्नो न॒ष्टमा कृ॑धि॒ पुन॑र्नो र॒यिमा कृ॑धि ॥ पुन॑रू॒र्जा निव॑र्तस्व॒ पुन॑रग्न इ॒षाऽऽयु॑षा । पुन॑र्नः पाहि वि॒श्वतः॑ ॥ स॒ह र॒य्या नि व॑र्त॒स्वाग्ने॒ पिन्व॑स्व॒ धार॑या । वि॒श्वफ्स्नि॑या वि॒श्व त॒स्परि॑ ॥ उदु॑त्त॒मं ॅव॑रुण॒ पाश॑ म॒स्मदवा॑ध॒मं - [  ] \newline

\textbf{Pada Paata} \newline

अ॒ङ्गि॒रः॒ । श॒तम् । ते॒ । स॒न्तु॒ । आ॒वृत॒ इत्या᳚-वृतः॑ । स॒हस्र᳚म् । ते॒ । उ॒पा॒वृत॒ इत्यु॑प-आ॒वृतः॑ ॥ तासा᳚म् । पोष॑स्य । पोषे॑ण । पुनः॑ । नः॒ । न॒ष्टम् । एति॑ । कृ॒धि॒ । पुनः॑ । नः॒ । र॒यिम् । एति॑ । कृ॒धि॒ ॥ पुनः॑ । ऊ॒र्जा । नीति॑ । व॒र्त॒स्व॒ । पुनः॑ । अ॒ग्ने॒ । इ॒षा । आयु॑षा ॥ पुनः॑ । नः॒ । पा॒हि॒ । वि॒श्वतः॑ ॥ स॒ह । र॒य्या । नीति॑ । व॒र्त॒स्व॒ । अग्ने᳚ । पिन्व॑स्व । धार॑या ॥ वि॒श्वफ्स्नि॒येति॑ वि॒श्व-फ्स्नि॒या॒ । वि॒श्वतः॑ । परि॑ ॥ उदिति॑ । उ॒त्त॒ममित्यु॑त्-त॒मम् । व॒रु॒ण॒ । पाश᳚म् । अ॒स्मत् । अवेति॑ । अ॒ध॒मम् ।  \newline


\textbf{Krama Paata} \newline

अ॒ङ्गि॒रः॒ श॒तम् । श॒तम् ते᳚ । ते॒ स॒न्तु॒ । स॒न्त्वा॒वृतः॑ । आ॒वृतः॑ स॒हस्र᳚म् । आ॒वृत॒ इत्या᳚ - वृतः॑ । स॒हस्र॑म् ते । त॒ उ॒पा॒वृतः॑ । उ॒पा॒वृत॒ इत्यु॑प - आ॒वृतः॑ ॥ तासा॒म् पोष॑स्य । पोष॑स्य॒ पोषे॑ण । पोषे॑ण॒ पुनः॑ । पुन॑र् नः । नो॒ न॒ष्टम् । न॒ष्टमा । आ कृ॑धि । कृ॒धि॒ पुनः॑ । पुन॑र् नः । नो॒ र॒यिम् । र॒यिमा । आ कृ॑धि । कृ॒धीति॑ कृधि ॥ पुन॑रू॒र्जा । ऊ॒र्जा नि । नि व॑र्तस्व । व॒र्त॒स्व॒ पुनः॑ । पुन॑रग्ने । अ॒ग्न॒ इ॒षा । इ॒षाऽऽयु॑षा । आयु॒षेत्यायु॑षा ॥ पुन॑र् नः । नः॒ पा॒हि॒ । पा॒हि॒ वि॒श्वतः॑ । वि॒श्वत॒ इति॑ वि॒श्वतः॑ ॥ स॒ह र॒य्या । र॒य्या नि । नि व॑र्तस्व । व॒र्त॒स्वाग्ने᳚ । अग्ने॒ पिन्व॑स्व । पिन्व॑स्व॒ धार॑या । धार॒येति॒ धार॑या ॥ वि॒श्वफ्स्नि॑या वि॒श्वतः॑ । वि॒श्वफ्स्नि॒येति॑ वि॒श्व - फ्स्नि॒या॒ । वि॒श्वत॒स्परि॑ । परीति॒ परि॑ ॥ उदु॑त्त॒मम् । उ॒त्त॒मं ॅव॑रुण । उ॒त्त॒ममित्यु॑त् - त॒मम् । व॒रु॒ण॒ पाश᳚म् । पाश॑म॒स्मत् । अ॒स्मदव॑ । अवा॑ध॒मम् । अ॒ध॒मं ॅवि \newline

\textbf{Jatai Paata} \newline

1. अ॒ङ्गि॒रः॒ श॒तꣳ श॒त म॑ङ्गिरो अङ्गिरः श॒तम् । \newline
2. श॒तम् ते॑ ते श॒तꣳ श॒तम् ते᳚ । \newline
3. ते॒ स॒न्तु॒ स॒न्तु॒ ते॒ ते॒ स॒न्तु॒ । \newline
4. स॒न्त्वा॒वृत॑ आ॒वृतः॑ सन्तु सन्त्वा॒वृतः॑ । \newline
5. आ॒वृतः॑ स॒हस्रꣳ॑ स॒हस्र॑ मा॒वृत॑ आ॒वृतः॑ स॒हस्र᳚म् । \newline
6. आ॒वृत॒ इत्या᳚ - वृतः॑ । \newline
7. स॒हस्र॑म् ते ते स॒हस्रꣳ॑ स॒हस्र॑म् ते । \newline
8. त॒ उ॒पा॒वृत॑ उपा॒वृत॑ स्ते त उपा॒वृतः॑ । \newline
9. उ॒पा॒वृत॒ इत्यु॑प - आ॒वृतः॑ । \newline
10. तासा॒म् पोष॑स्य॒ पोष॑स्य॒ तासा॒म् तासा॒म् पोष॑स्य । \newline
11. पोष॑स्य॒ पोषे॑ण॒ पोषे॑ण॒ पोष॑स्य॒ पोष॑स्य॒ पोषे॑ण । \newline
12. पोषे॑ण॒ पुनः॒ पुनः॒ पोषे॑ण॒ पोषे॑ण॒ पुनः॑ । \newline
13. पुन॑र् नो नः॒ पुनः॒ पुन॑र् नः । \newline
14. नो॒ न॒ष्टम् न॒ष्टम् नो॑ नो न॒ष्टम् । \newline
15. न॒ष्ट मा न॒ष्टम् न॒ष्ट मा । \newline
16. आ कृ॑धि कृ॒ध्या कृ॑धि । \newline
17. कृ॒धि॒ पुनः॒ पुन॑ स्कृधि कृधि॒ पुनः॑ । \newline
18. पुन॑र् नो नः॒ पुनः॒ पुन॑र् नः । \newline
19. नो॒ र॒यिꣳ र॒यिम् नो॑ नो र॒यिम् । \newline
20. र॒यि मा र॒यिꣳ र॒यि मा । \newline
21. आ कृ॑धि कृ॒ध्या कृ॑धि । \newline
22. कृ॒धीति॑ कृधि । \newline
23. पुन॑ रू॒र्जोर्जा पुनः॒ पुन॑ रू॒र्जा । \newline
24. ऊ॒र्जा नि न्यू᳚र्जोर्जा नि । \newline
25. नि व॑र्तस्व वर्तस्व॒ नि नि व॑र्तस्व । \newline
26. व॒र्त॒स्व॒ पुनः॒ पुन॑र् वर्तस्व वर्तस्व॒ पुनः॑ । \newline
27. पुन॑ रग्ने ऽग्ने॒ पुनः॒ पुन॑ रग्ने । \newline
28. अ॒ग्न॒ इ॒षेषा ऽग्ने᳚ ऽग्न इ॒षा । \newline
29. इ॒षा ऽऽयु॒षा ऽऽयु॑षे॒ षेषा ऽऽयु॑षा । \newline
30. आयु॒षेत्यायु॑षा । \newline
31. पुन॑र् नो नः॒ पुनः॒ पुन॑र् नः । \newline
32. नः॒ पा॒हि॒ पा॒हि॒ नो॒ नः॒ पा॒हि॒ । \newline
33. पा॒हि॒ वि॒श्वतो॑ वि॒श्वत॑ स्पाहि पाहि वि॒श्वतः॑ । \newline
34. वि॒श्वत॒ इति॑ वि॒श्वतः॑ । \newline
35. स॒ह र॒य्या र॒य्या स॒ह स॒ह र॒य्या । \newline
36. र॒य्या नि नि र॒य्या र॒य्या नि । \newline
37. नि व॑र्तस्व वर्तस्व॒ नि नि व॑र्तस्व । \newline
38. व॒र्त॒स्वाग्ने ऽग्ने॑ वर्तस्व वर्त॒स्वाग्ने᳚ । \newline
39. अग्ने॒ पिन्व॑स्व॒ पिन्व॒स्वाग्ने ऽग्ने॒ पिन्व॑स्व । \newline
40. पिन्व॑स्व॒ धार॑या॒ धार॑या॒ पिन्व॑स्व॒ पिन्व॑स्व॒ धार॑या । \newline
41. धार॒येति॒ धार॑या । \newline
42. वि॒श्वफ्स्नि॑या वि॒श्वतो॑ वि॒श्वतो॑ वि॒श्वफ्स्नि॑या वि॒श्वफ्स्नि॑या वि॒श्वतः॑ । \newline
43. वि॒श्वफ्स्नि॒येति॑ वि॒श्व - फ्स्नि॒या॒ । \newline
44. वि॒श्वत॒ स्परि॒ परि॑ वि॒श्वतो॑ वि॒श्वत॒ स्परि॑ । \newline
45. परीति॒ परि॑ । \newline
46. उदु॑त्त॒म मु॑त्त॒म मुदु दु॑त्त॒मम् । \newline
47. उ॒त्त॒मं ॅव॑रुण वरुणोत्त॒म मु॑त्त॒मं ॅव॑रुण । \newline
48. उ॒त्त॒ममित्यु॑त् - त॒मम् । \newline
49. व॒रु॒ण॒ पाश॒म् पाशं॑ ॅवरुण वरुण॒ पाश᳚म् । \newline
50. पाश॑ म॒स्म द॒स्मत् पाश॒म् पाश॑ म॒स्मत् । \newline
51. अ॒स्म दवावा॒ स्म द॒स्म दव॑ । \newline
52. अवा॑ध॒म म॑ध॒म मवावा॑ ध॒मम् । \newline
53. अ॒ध॒मं ॅवि व्य॑ध॒म म॑ध॒मं ॅवि । \newline

\textbf{Ghana Paata } \newline

1. अ॒ङ्गि॒रः॒ श॒तꣳ श॒त म॑ङ्गिरो अङ्गिरः श॒तम् ते॑ ते श॒त म॑ङ्गिरो अङ्गिरः श॒तम् ते᳚ । \newline
2. श॒तम् ते॑ ते श॒तꣳ श॒तम् ते॑ सन्तु सन्तु ते श॒तꣳ श॒तम् ते॑ सन्तु । \newline
3. ते॒ स॒न्तु॒ स॒न्तु॒ ते॒ ते॒ स॒न्त्वा॒वृत॑ आ॒वृतः॑ सन्तु ते ते सन्त्वा॒वृतः॑ । \newline
4. स॒न्त्वा॒वृत॑ आ॒वृतः॑ सन्तु सन्त्वा॒वृतः॑ स॒हस्रꣳ॑ स॒हस्र॑ मा॒वृतः॑ सन्तु सन्त्वा॒वृतः॑ स॒हस्र᳚म् । \newline
5. आ॒वृतः॑ स॒हस्रꣳ॑ स॒हस्र॑ मा॒वृत॑ आ॒वृतः॑ स॒हस्र॑म् ते ते स॒हस्र॑ मा॒वृत॑ आ॒वृतः॑ स॒हस्र॑म् ते । \newline
6. आ॒वृत॒ इत्या᳚ - वृतः॑ । \newline
7. स॒हस्र॑म् ते ते स॒हस्रꣳ॑ स॒हस्र॑म् त उपा॒वृत॑ उपा॒वृत॑ स्ते स॒हस्रꣳ॑ स॒हस्र॑म् त उपा॒वृतः॑ । \newline
8. त॒ उ॒पा॒वृत॑ उपा॒वृत॑ स्ते त उपा॒वृतः॑ । \newline
9. उ॒पा॒वृत॒ इत्यु॑प - आ॒वृतः॑ । \newline
10. तासा॒म् पोष॑स्य॒ पोष॑स्य॒ तासा॒म् तासा॒म् पोष॑स्य॒ पोषे॑ण॒ पोषे॑ण॒ पोष॑स्य॒ तासा॒म् तासा॒म् पोष॑स्य॒ पोषे॑ण । \newline
11. पोष॑स्य॒ पोषे॑ण॒ पोषे॑ण॒ पोष॑स्य॒ पोष॑स्य॒ पोषे॑ण॒ पुनः॒ पुनः॒ पोषे॑ण॒ पोष॑स्य॒ पोष॑स्य॒ पोषे॑ण॒ पुनः॑ । \newline
12. पोषे॑ण॒ पुनः॒ पुनः॒ पोषे॑ण॒ पोषे॑ण॒ पुन॑र् नो नः॒ पुनः॒ पोषे॑ण॒ पोषे॑ण॒ पुन॑र् नः । \newline
13. पुन॑र् नो नः॒ पुनः॒ पुन॑र् नो न॒ष्टम् न॒ष्टम् नः॒ पुनः॒ पुन॑र् नो न॒ष्टम् । \newline
14. नो॒ न॒ष्टम् न॒ष्टम् नो॑ नो न॒ष्ट मा न॒ष्टम् नो॑ नो न॒ष्ट मा । \newline
15. न॒ष्ट मा न॒ष्टम् न॒ष्ट मा कृ॑धि कृ॒ध्या न॒ष्टम् न॒ष्ट मा कृ॑धि । \newline
16. आ कृ॑धि कृ॒ध्या कृ॑धि॒ पुनः॒ पुन॑ स्कृ॒ध्या कृ॑धि॒ पुनः॑ । \newline
17. कृ॒धि॒ पुनः॒ पुन॑ स्कृधि कृधि॒ पुन॑र् नो नः॒ पुन॑ स्कृधि कृधि॒ पुन॑र् नः । \newline
18. पुन॑र् नो नः॒ पुनः॒ पुन॑र् नो र॒यिꣳ र॒यिन्नः॒ पुनः॒ पुन॑र् नो र॒यिम् । \newline
19. नो॒ र॒यिꣳ र॒यिम् नो॑ नो र॒यि मा र॒यिम् नो॑ नो र॒यि मा । \newline
20. र॒यि मा र॒यिꣳ र॒यि मा कृ॑धि कृ॒ध्या र॒यिꣳ र॒यि मा कृ॑धि । \newline
21. आ कृ॑धि कृ॒ध्या कृ॑धि । \newline
22. कृ॒धीति॑ कृधि । \newline
23. पुन॑ रू॒र्जोर्जा पुनः॒ पुन॑ रू॒र्जा नि न्यू᳚र्जा पुनः॒ पुन॑ रू॒र्जा नि । \newline
24. ऊ॒र्जा नि न्यू᳚र्जोर्जा नि व॑र्तस्व वर्तस्व॒ न्यू᳚र्जोर्जा नि व॑र्तस्व । \newline
25. नि व॑र्तस्व वर्तस्व॒ नि नि व॑र्तस्व॒ पुनः॒ पुन॑र् वर्तस्व॒ नि नि व॑र्तस्व॒ पुनः॑ । \newline
26. व॒र्त॒स्व॒ पुनः॒ पुन॑र् वर्तस्व वर्तस्व॒ पुन॑ रग्ने ऽग्ने॒ पुन॑र् वर्तस्व वर्तस्व॒ पुन॑ रग्ने । \newline
27. पुन॑ रग्ने ऽग्ने॒ पुनः॒ पुन॑ रग्न इ॒षेषा ऽग्ने॒ पुनः॒ पुन॑ रग्न इ॒षा । \newline
28. अ॒ग्न॒ इ॒षेषा ऽग्ने᳚ ऽग्न इ॒षा ऽऽयु॒षा ऽऽयु॑षे॒षा ऽग्ने᳚ ऽग्न इ॒षा ऽऽयु॑षा । \newline
29. इ॒षा ऽऽयु॒षा ऽऽयु॑षे॒ षेषा ऽऽयु॑षा । \newline
30. आयु॒षेत्यायु॑षा । \newline
31. पुन॑र् नो नः॒ पुनः॒ पुन॑र् नः पाहि पाहि नः॒ पुनः॒ पुन॑र् नः पाहि । \newline
32. नः॒ पा॒हि॒ पा॒हि॒ नो॒ नः॒ पा॒हि॒ वि॒श्वतो॑ वि॒श्वत॑ स्पाहि नो नः पाहि वि॒श्वतः॑ । \newline
33. पा॒हि॒ वि॒श्वतो॑ वि॒श्वत॑ स्पाहि पाहि वि॒श्वतः॑ । \newline
34. वि॒श्वत॒ इति॑ वि॒श्वतः॑ । \newline
35. स॒ह र॒य्या र॒य्या स॒ह स॒ह र॒य्या नि नि र॒य्या स॒ह स॒ह र॒य्या नि । \newline
36. र॒य्या नि नि र॒य्या र॒य्या नि व॑र्तस्व वर्तस्व॒ नि र॒य्या र॒य्या नि व॑र्तस्व । \newline
37. नि व॑र्तस्व वर्तस्व॒ नि नि व॑र्त॒स्वाग्ने ऽग्ने॑ वर्तस्व॒ नि नि व॑र्त॒स्वाग्ने᳚ । \newline
38. व॒र्त॒स्वाग्ने ऽग्ने॑ वर्तस्व वर्त॒स्वाग्ने॒ पिन्व॑स्व॒ पिन्व॒स्वाग्ने॑ वर्तस्व वर्त॒स्वाग्ने॒ पिन्व॑स्व । \newline
39. अग्ने॒ पिन्व॑स्व॒ पिन्व॒स्वाग्ने ऽग्ने॒ पिन्व॑स्व॒ धार॑या॒ धार॑या॒ पिन्व॒स्वाग्ने ऽग्ने॒ पिन्व॑स्व॒ धार॑या । \newline
40. पिन्व॑स्व॒ धार॑या॒ धार॑या॒ पिन्व॑स्व॒ पिन्व॑स्व॒ धार॑या । \newline
41. धार॒येति॒ धार॑या । \newline
42. वि॒श्वफ्स्नि॑या वि॒श्वतो॑ वि॒श्वतो॑ वि॒श्वफ्स्नि॑या वि॒श्वफ्स्नि॑या वि॒श्वत॒ स्परि॒ परि॑ वि॒श्वतो॑ वि॒श्वफ्स्नि॑या वि॒श्वफ्स्नि॑या वि॒श्वत॒ स्परि॑ । \newline
43. वि॒श्वफ्स्नि॒येति॑ वि॒श्व - फ्स्नि॒या॒ । \newline
44. वि॒श्वत॒ स्परि॒ परि॑ वि॒श्वतो॑ वि॒श्वत॒ स्परि॑ । \newline
45. परीति॒ परि॑ । \newline
46. उदु॑त्त॒म मु॑त्त॒म मुदुदु॑त्त॒मं ॅव॑रुण वरुणोत्त॒म मुदुदु॑त्त॒मं ॅव॑रुण । \newline
47. उ॒त्त॒मं ॅव॑रुण वरुणोत्त॒म मु॑त्त॒मं ॅव॑रुण॒ पाश॒म् पाशं॑ ॅवरुणोत्त॒म मु॑त्त॒मं ॅव॑रुण॒ पाश᳚म् । \newline
48. उ॒त्त॒ममित्यु॑त् - त॒मम् । \newline
49. व॒रु॒ण॒ पाश॒म् पाशं॑ ॅवरुण वरुण॒ पाश॑ म॒स्म द॒स्मत् पाशं॑ ॅवरुण वरुण॒ पाश॑ म॒स्मत् । \newline
50. पाश॑ म॒स्म द॒स्मत् पाश॒म् पाश॑ म॒स्म दवा वा॒स्मत् पाश॒म् पाश॑ म॒स्मदव॑ । \newline
51. अ॒स्म दवावा॒स्म द॒स्म दवा॑ध॒म म॑ध॒म मवा॒स्म द॒स्म दवा॑ध॒मम् । \newline
52. अवा॑ध॒म म॑ध॒म मवावा॑ ध॒मं ॅवि व्य॑ध॒म मवावा॑ध॒मं ॅवि । \newline
53. अ॒ध॒मं ॅवि व्य॑ध॒म म॑ध॒मं ॅवि म॑द्ध्य॒मम् म॑द्ध्य॒मं ॅव्य॑ध॒म म॑ध॒मं ॅवि म॑द्ध्य॒मम् । \newline
\pagebreak
\markright{ TS 4.2.1.4  \hfill https://www.vedavms.in \hfill}

\section{ TS 4.2.1.4 }

\textbf{TS 4.2.1.4 } \newline
\textbf{Samhita Paata} \newline

ॅवि म॑द्ध्य॒मꣳ श्र॑थाय । अथा॑ व॒यमा॑दित्य व्र॒ते तवाना॑गसो॒ अदि॑तये स्याम ॥ आ त्वा॑ऽहार्.ष-म॒न्तर॑भूर्द्ध्रु॒वस्ति॒ष्ठा ऽवि॑चाचलिः । विश॑स्त्वा॒ सर्वा॑ वाञ्छन्त्व॒स्मिन् रा॒ष्ट्रमधि॑ श्रय ॥अग्ने॑ बृ॒हन्नु॒षसा॑मू॒र्द्ध्वो अ॑स्थान्निर्जग्मि॒वान्-तम॑सो॒ ज्योति॒षाऽऽगा᳚त् । अ॒ग्निर्भा॒नुना॒ रुश॑ता॒ स्वङ्ग॒ आ जा॒तो विश्वा॒ सद्मा᳚न्यप्राः ॥ सीद॒ त्वं मा॒तुर॒स्या - [  ] \newline

\textbf{Pada Paata} \newline

वीति॑ । म॒द्ध्य॒मम् । श्र॒था॒य॒ ॥ अथ॑ । व॒यम् । आ॒दि॒त्य॒ । व्र॒ते । तव॑ । अना॑गसः । अदि॑तये । स्या॒म॒ ॥ एति॑ । त्वा॒ । अ॒हा॒र्॒.ष॒म् । अ॒न्तः । अ॒भूः॒ । ध्रु॒वः । ति॒ष्ठ॒ । अवि॑चाचलि॒रित्यवि॑ - चा॒च॒लिः॒ ॥ विशः॑ । त्वा॒ । सर्वाः᳚ । वा॒ञ्छ॒न्तु॒ । अ॒स्मिन्न् । रा॒ष्ट्रम् । अधीति॑ । श्र॒य॒ ॥ अग्ने᳚ । बृ॒हन्न् । उ॒षसा᳚म् । ऊ॒द्‌र्ध्वः । अ॒स्था॒त् । नि॒र्ज॒ग्मि॒वानिति॑ निः - ज॒ग्मि॒वान् । तम॑सः । ज्योति॑षा । एति॑ । अ॒गा॒त् ॥ अ॒ग्निः । भा॒नुना᳚ । रुश॑ता । स्वङ्ग॒ इति॑ सु - अङ्गः॑ । एति॑ । जा॒तः । विश्वा᳚ । सद्मा॑नि । अ॒प्राः॒ ॥ सीद॑ । त्वम् । मा॒तुः । अ॒स्याः ।  \newline


\textbf{Krama Paata} \newline

वि म॑द्ध्य॒मम् । म॒द्ध्य॒मꣳ श्र॑थाय । श्र॒था॒येति॑ श्रथाय ॥ अथा॑ व॒यम् । व॒यमा॑दित्य । आ॒दि॒त्य॒ व्र॒ते । व्र॒ते तव॑ । तवाना॑गसः । अना॑गसो॒ अदि॑तये । अदि॑तये स्याम । स्या॒मेति॑ स्याम ॥ आ त्वा᳚ । त्वा॒ऽहा॒र्॒.ष॒म् । आ॒हा॒र्॒.ष॒म॒न्तः । अ॒न्तर॑भूः । अ॒भू॒र् ध्रु॒वः । ध्रु॒वस्ति॑ष्ठ । ति॒ष्ठावि॑चाचलिः । अवि॑चाचलि॒रित्यवि॑ - चा॒च॒लिः॒ ॥ विश॑स्त्वा । त्वा॒ सर्वाः᳚ । सर्वा॑ वाञ्छन्तु । वा॒ञ्छ॒न्त्व॒स्मिन्न् । अ॒स्मिन् रा॒ष्ट्रम् । रा॒ष्ट्रमधि॑ । अधि॑ श्रय । श्र॒येति॑ श्रय ॥ अग्रे॑ बृ॒हन्न् । बृ॒हन्नु॒षसा᳚म् । उ॒षसा॑मू॒र्द्ध्वः । ऊ॒र्द्ध्वो अ॑स्थात् । अ॒स्था॒न् नि॒र्ज॒ग्मि॒वान् । नि॒र्ज॒ग्मि॒वान् तम॑सः । नि॒र्ज॒ग्मि॒वानिति॑ निः - ज॒ग्मि॒वान् । तम॑सो॒ ज्योति॑षा । ज्योति॒षा । आऽगा᳚त् । अ॒गा॒दित्य॑गात् ॥ अ॒ग्निर् भा॒नुना᳚ । भा॒नुना॒ रुश॑ता । रुश॑ता॒ स्वङ्गः॑ । स्वङ्ग॒ आ । स्वङ्ग॒ इति॑ सु - अङ्गः॑ । आ जा॒तः । जा॒तो विश्वा᳚ । विश्वा॒ सद्मा॑नि । सद्मा᳚न्यप्राः । अ॒प्रा॒ इत्य॑प्राः ॥ सीद॒ त्वम् । त्वम् मा॒तुः । मा॒तुर॒स्याः । अ॒स्या उ॒पस्थे᳚ \newline

\textbf{Jatai Paata} \newline

1. वि म॑द्ध्य॒मम् म॑द्ध्य॒मं ॅवि वि म॑द्ध्य॒मम् । \newline
2. म॒द्ध्य॒मꣳ श्र॑थाय श्रथाय मद्ध्य॒मम् म॑द्ध्य॒मꣳ श्र॑थाय । \newline
3. श्र॒था॒येति॑ श्रथाय । \newline
4. अथा॑ व॒यं ॅव॒य मथाथा॑ व॒यम् । \newline
5. व॒य मा॑दित्या दित्य व॒यं ॅव॒य मा॑दित्य । \newline
6. आ॒दि॒त्य॒ व्र॒ते व्र॒त आ॑दित्या दित्य व्र॒ते । \newline
7. व्र॒ते तव॒ तव॑ व्र॒ते व्र॒ते तव॑ । \newline
8. तवा ना॑ग॒सो ऽना॑गस॒ स्तव॒ तवा ना॑गसः । \newline
9. अना॑गसो॒ अदि॑तये॒ अदि॑त॒ये ऽना॑ग॒सो ऽना॑गसो॒ अदि॑तये । \newline
10. अदि॑तये स्याम स्या॒मा दि॑तये॒ अदि॑तये स्याम । \newline
11. स्या॒मेति॑ स्याम । \newline
12. आ त्वा॒ त्वा ऽऽत्वा᳚ । \newline
13. त्वा॒ ऽहा॒र्॒.ष॒ म॒हा॒र्॒.ष॒म् त्वा॒ त्वा॒ ऽहा॒र्॒.ष॒म् । \newline
14. अ॒हा॒र्॒.ष॒ म॒न्त र॒न्त र॑हार्.ष महार्.ष म॒न्तः । \newline
15. अ॒न्त र॑भू रभू र॒न्त र॒न्त र॑भूः । \newline
16. अ॒भू॒र् ध्रु॒वो ध्रु॒वो अ॑भू रभूर् ध्रु॒वः । \newline
17. ध्रु॒व स्ति॑ष्ठ तिष्ठ ध्रु॒वो ध्रु॒व स्ति॑ष्ठ । \newline
18. ति॒ष्ठा वि॑चाचलि॒ रवि॑चाचलि स्तिष्ठ ति॒ष्ठा वि॑चाचलिः । \newline
19. अवि॑चाचलि॒रित्यवि॑ - चा॒च॒लिः॒ । \newline
20. विश॑ स्त्वा त्वा॒ विशो॒ विश॑ स्त्वा । \newline
21. त्वा॒ सर्वाः॒ सर्वा᳚ स्त्वा त्वा॒ सर्वाः᳚ । \newline
22. सर्वा॑ वाञ्छन्तु वाञ्छन्तु॒ सर्वाः॒ सर्वा॑ वाञ्छन्तु । \newline
23. वा॒ञ्छ॒ न्त्व॒स्मिन् न॒स्मिन्. वा᳚ञ्छन्तु वाञ्छ न्त्व॒स्मिन्न् । \newline
24. अ॒स्मिन् रा॒ष्ट्रꣳ रा॒ष्ट्र म॒स्मिन् न॒स्मिन् रा॒ष्ट्रम् । \newline
25. रा॒ष्ट्र मध्यधि॑ रा॒ष्ट्रꣳ रा॒ष्ट्र मधि॑ । \newline
26. अधि॑ श्रय श्र॒या ध्यधि॑ श्रय । \newline
27. श्र॒येति॑ श्रय । \newline
28. अग्रे॑ बृ॒हन् बृ॒हन् नग्रे॒ अग्रे॑ बृ॒हन्न् । \newline
29. बृ॒हन् नु॒षसा॑ मु॒षसा᳚म् बृ॒हन् बृ॒हन् नु॒षसा᳚म् । \newline
30. उ॒षसा॑ मू॒र्द्ध्व ऊ॒र्द्ध्व उ॒षसा॑ मु॒षसा॑ मू॒र्द्ध्वः । \newline
31. ऊ॒र्द्ध्वो अ॑स्था दस्था दू॒र्द्ध्व ऊ॒र्द्ध्वो अ॑स्थात् । \newline
32. अ॒स्था॒न् नि॒र्ज॒ग्मि॒वान् नि॑र्जग्मि॒वा न॑स्था दस्थान् निर्जग्मि॒वान् । \newline
33. नि॒र्ज॒ग्मि॒वान् तम॑स॒ स्तम॑सो निर्जग्मि॒वान् नि॑र्जग्मि॒वान् तम॑सः । \newline
34. नि॒र्ज॒ग्मि॒वानिति॑ निः - ज॒ग्मि॒वान् । \newline
35. तम॑सो॒ ज्योति॑षा॒ ज्योति॑षा॒ तम॑स॒ स्तम॑सो॒ ज्योति॑षा । \newline
36. ज्योति॒षा ऽऽज्योति॑षा॒ ज्योति॒षा । \newline
37. आ ऽगा॑ दगा॒दा ऽगा᳚त् । \newline
38. अ॒गा॒दित्य॑गात् । \newline
39. अ॒ग्निर् भा॒नुना॑ भा॒नुना॒ ऽग्नि र॒ग्निर् भा॒नुना᳚ । \newline
40. भा॒नुना॒ रुश॑ता॒ रुश॑ता भा॒नुना॑ भा॒नुना॒ रुश॑ता । \newline
41. रुश॑ता॒ स्वङ्गः॒ स्वङ्गो॒ रुश॑ता॒ रुश॑ता॒ स्वङ्गः॑ । \newline
42. स्वङ्ग॒ आ स्वङ्गः॒ स्वङ्ग॒ आ । \newline
43. स्वङ्ग॒ इति॑ सु - अङ्गः॑ । \newline
44. आ जा॒तो जा॒त आ जा॒तः । \newline
45. जा॒तो विश्वा॒ विश्वा॑ जा॒तो जा॒तो विश्वा᳚ । \newline
46. विश्वा॒ सद्मा॑नि॒ सद्मा॑नि॒ विश्वा॒ विश्वा॒ सद्मा॑नि । \newline
47. सद्मा᳚ न्यप्रा अप्राः॒ सद्मा॑नि॒ सद्मा᳚ न्यप्राः । \newline
48. अ॒प्रा॒ इत्य॑प्राः । \newline
49. सीद॒ त्वम् त्वꣳ सीद॒ सीद॒ त्वम् । \newline
50. त्वम् मा॒तुर् मा॒तु स्त्वम् त्वम् मा॒तुः । \newline
51. मा॒तु र॒स्या अ॒स्या मा॒तुर् मा॒तु र॒स्याः । \newline
52. अ॒स्या उ॒पस्थ॑ उ॒पस्थे॑ अ॒स्या अ॒स्या उ॒पस्थे᳚ । \newline

\textbf{Ghana Paata } \newline

1. वि म॑द्ध्य॒मम् म॑द्ध्य॒मं ॅवि वि म॑द्ध्य॒मꣳ श्र॑थाय श्रथाय मद्ध्य॒मं ॅवि वि म॑द्ध्य॒मꣳ श्र॑थाय । \newline
2. म॒द्ध्य॒मꣳ श्र॑थाय श्रथाय मद्ध्य॒मम् म॑द्ध्य॒मꣳ श्र॑थाय । \newline
3. श्र॒था॒येति॑ श्रथाय । \newline
4. अथा॑ व॒यं ॅव॒य मथाथा॑ व॒य मा॑दित्या दित्य व॒य मथाथा॑ व॒य मा॑दित्य । \newline
5. व॒य मा॑दित्या दित्य व॒यं ॅव॒य मा॑दित्य व्र॒ते व्र॒त आ॑दित्य व॒यं ॅव॒य मा॑दित्य व्र॒ते । \newline
6. आ॒दि॒त्य॒ व्र॒ते व्र॒त आ॑दित्या दित्य व्र॒ते तव॒ तव॑ व्र॒त आ॑दित्या दित्य व्र॒ते तव॑ । \newline
7. व्र॒ते तव॒ तव॑ व्र॒ते व्र॒ते तवाना॑ग॒सो ऽना॑गस॒ स्तव॑ व्र॒ते व्र॒ते तवाना॑गसः । \newline
8. तवाना॑ग॒सो ऽना॑गस॒ स्तव॒ तवाना॑गसो॒ अदि॑तये॒ अदि॑त॒ये ऽना॑गस॒ स्तव॒ तवाना॑गसो॒ अदि॑तये । \newline
9. अना॑गसो॒ अदि॑तये॒ अदि॑त॒ये ऽना॑ग॒सो ऽना॑गसो॒ अदि॑तये स्याम स्या॒मादि॑त॒ये ऽना॑ग॒सो ऽना॑गसो॒ अदि॑तये स्याम । \newline
10. अदि॑तये स्याम स्या॒मादि॑तये॒ अदि॑तये स्याम । \newline
11. स्या॒मेति॑ स्याम । \newline
12. आ त्वा॒ त्वा ऽऽत्वा॑ ऽहार्.ष महार्.ष॒म् त्वा ऽऽत्वा॑ ऽहार्.षम् । \newline
13. त्वा॒ ऽहा॒र्॒.ष॒ म॒हा॒र्॒.ष॒म् त्वा॒ त्वा॒ ऽहा॒र्॒.ष॒ म॒न्त र॒न्त र॑हार्.षम् त्वा त्वा ऽहार्.ष म॒न्तः । \newline
14. अ॒हा॒र्॒.ष॒ म॒न्त र॒न्त र॑हार्.ष महार्.ष म॒न्त र॑भू रभू र॒न्त र॑हार्.ष महार्.ष म॒न्त र॑भूः । \newline
15. अ॒न्त र॑भू रभू र॒न्त र॒न्त र॑भूर् ध्रु॒वो ध्रु॒वो अ॑भू र॒न्त र॒न्त र॑भूर् ध्रु॒वः । \newline
16. अ॒भू॒र् ध्रु॒वो ध्रु॒वो अ॑भू रभूर् ध्रु॒व स्ति॑ष्ठ तिष्ठ ध्रु॒वो अ॑भू रभूर् ध्रु॒व स्ति॑ष्ठ । \newline
17. ध्रु॒वस्ति॑ष्ठ तिष्ठ ध्रु॒वो ध्रु॒व स्ति॒ष्ठा वि॑चाचलि॒ रवि॑चाचलि स्तिष्ठ ध्रु॒वो ध्रु॒व स्ति॒ष्ठा वि॑चाचलिः । \newline
18. ति॒ष्ठा वि॑चाचलि॒ रवि॑चाचलि स्तिष्ठ ति॒ष्ठा वि॑चाचलिः । \newline
19. अवि॑चाचलि॒रित्यवि॑ - चा॒च॒लिः॒ । \newline
20. विश॑ स्त्वा त्वा॒ विशो॒ विश॑ स्त्वा॒ सर्वाः॒ सर्वा᳚ स्त्वा॒ विशो॒ विश॑ स्त्वा॒ सर्वाः᳚ । \newline
21. त्वा॒ सर्वाः॒ सर्वा᳚ स्त्वा त्वा॒ सर्वा॑ वाञ्छन्तु वाञ्छन्तु॒ सर्वा᳚ स्त्वा त्वा॒ सर्वा॑ वाञ्छन्तु । \newline
22. सर्वा॑ वाञ्छन्तु वाञ्छन्तु॒ सर्वाः॒ सर्वा॑ वाञ्छ न्त्व॒स्मिन् न॒स्मिन्. वा᳚ञ्छन्तु॒ सर्वाः॒ सर्वा॑ वाञ्छ न्त्व॒स्मिन्न् । \newline
23. वा॒ञ्छ॒ न्त्व॒स्मिन् न॒स्मिन्. वा᳚ञ्छन्तु वाञ्छ न्त्व॒स्मिन् रा॒ष्ट्रꣳ रा॒ष्ट्र म॒स्मिन्. वा᳚ञ्छन्तु वाञ्छ न्त्व॒स्मिन् रा॒ष्ट्रम् । \newline
24. अ॒स्मिन् रा॒ष्ट्रꣳ रा॒ष्ट्र म॒स्मिन् न॒स्मिन् रा॒ष्ट्र मध्यधि॑ रा॒ष्ट्र म॒स्मिन् न॒स्मिन् रा॒ष्ट्र मधि॑ । \newline
25. रा॒ष्ट्र मध्यधि॑ रा॒ष्ट्रꣳ रा॒ष्ट्र मधि॑ श्रय श्र॒याधि॑ रा॒ष्ट्रꣳ रा॒ष्ट्र मधि॑ श्रय । \newline
26. अधि॑ श्रय श्र॒याध्यधि॑ श्रय । \newline
27. श्र॒येति॑ श्रय । \newline
28. अग्रे॑ बृ॒हन् बृ॒हन् नग्रे॒ अग्रे॑ बृ॒हन् नु॒षसा॑ मु॒षसा᳚म् बृ॒हन् नग्रे॒ अग्रे॑ बृ॒हन् नु॒षसा᳚म् । \newline
29. बृ॒हन् नु॒षसा॑ मु॒षसा᳚म् बृ॒हन् बृ॒हन् नु॒षसा॑ मू॒र्द्ध्व ऊ॒र्द्ध्व उ॒षसा᳚म् बृ॒हन् बृ॒हन् नु॒षसा॑ मू॒र्द्ध्वः । \newline
30. उ॒षसा॑ मू॒र्द्ध्व ऊ॒र्द्ध्व उ॒षसा॑ मु॒षसा॑ मू॒र्द्ध्वो अ॑स्था दस्था दू॒र्द्ध्व उ॒षसा॑ मु॒षसा॑ मू॒र्द्ध्वो अ॑स्थात् । \newline
31. ऊ॒र्द्ध्वो अ॑स्था दस्था दू॒र्द्ध्व ऊ॒र्द्ध्वो अ॑स्थान् निर्जग्मि॒वान् नि॑र्जग्मि॒वा न॑स्था दू॒र्द्ध्व ऊ॒र्द्ध्वो अ॑स्थान् निर्जग्मि॒वान् । \newline
32. अ॒स्था॒न् नि॒र्ज॒ग्मि॒वान् नि॑र्जग्मि॒वा न॑स्था दस्थान् निर्जग्मि॒वान् तम॑स॒ स्तम॑सो निर्जग्मि॒वा न॑स्था दस्थान् निर्जग्मि॒वान् तम॑सः । \newline
33. नि॒र्ज॒ग्मि॒वान् तम॑स॒ स्तम॑सो निर्जग्मि॒वान् नि॑र्जग्मि॒वान् तम॑सो॒ ज्योति॑षा॒ ज्योति॑षा॒ तम॑सो निर्जग्मि॒वान् नि॑र्जग्मि॒वान् तम॑सो॒ ज्योति॑षा । \newline
34. नि॒र्ज॒ग्मि॒वानिति॑ निः - ज॒ग्मि॒वान् । \newline
35. तम॑सो॒ ज्योति॑षा॒ ज्योति॑षा॒ तम॑स॒ स्तम॑सो॒ ज्योति॒षा ऽऽज्योति॑षा॒ तम॑स॒ स्तम॑सो॒ ज्योति॒षा । \newline
36. ज्योति॒षा ऽऽज्योति॑षा॒ ज्योति॒षा ऽगा॑ दगा॒दा ज्योति॑षा॒ ज्योति॒षा ऽगा᳚त् । \newline
37. आ ऽगा॑ दगा॒दा ऽगा᳚त् । \newline
38. अ॒गा॒दित्य॑गात् । \newline
39. अ॒ग्निर् भा॒नुना॑ भा॒नुना॒ ऽग्नि र॒ग्निर् भा॒नुना॒ रुश॑ता॒ रुश॑ता भा॒नुना॒ ऽग्नि र॒ग्निर् भा॒नुना॒ रुश॑ता । \newline
40. भा॒नुना॒ रुश॑ता॒ रुश॑ता भा॒नुना॑ भा॒नुना॒ रुश॑ता॒ स्वङ्गः॒ स्वङ्गो॒ रुश॑ता भा॒नुना॑ भा॒नुना॒ रुश॑ता॒ स्वङ्गः॑ । \newline
41. रुश॑ता॒ स्वङ्गः॒ स्वङ्गो॒ रुश॑ता॒ रुश॑ता॒ स्वङ्ग॒ आ स्वङ्गो॒ रुश॑ता॒ रुश॑ता॒ स्वङ्ग॒ आ । \newline
42. स्वङ्ग॒ आ स्वङ्गः॒ स्वङ्ग॒ आ जा॒तो जा॒त आ स्वङ्गः॒ स्वङ्ग॒ आ जा॒तः । \newline
43. स्वङ्ग॒ इति॑ सु - अङ्गः॑ । \newline
44. आ जा॒तो जा॒त आ जा॒तो विश्वा॒ विश्वा॑ जा॒त आ जा॒तो विश्वा᳚ । \newline
45. जा॒तो विश्वा॒ विश्वा॑ जा॒तो जा॒तो विश्वा॒ सद्मा॑नि॒ सद्मा॑नि॒ विश्वा॑ जा॒तो जा॒तो विश्वा॒ सद्मा॑नि । \newline
46. विश्वा॒ सद्मा॑नि॒ सद्मा॑नि॒ विश्वा॒ विश्वा॒ सद्मा᳚ न्यप्रा अप्राः॒ सद्मा॑नि॒ विश्वा॒ विश्वा॒ सद्मा᳚ न्यप्राः । \newline
47. सद्मा᳚ न्यप्रा अप्राः॒ सद्मा॑नि॒ सद्मा᳚ न्यप्राः । \newline
48. अ॒प्रा॒ इत्य॑प्राः । \newline
49. सीद॒ त्वम् त्वꣳ सीद॒ सीद॒ त्वम् मा॒तुर् मा॒तु स्त्वꣳ सीद॒ सीद॒ त्वम् मा॒तुः । \newline
50. त्वम् मा॒तुर् मा॒तु स्त्वम् त्वम् मा॒तु र॒स्या अ॒स्या मा॒तु स्त्वम् त्वम् मा॒तु र॒स्याः । \newline
51. मा॒तु र॒स्या अ॒स्या मा॒तुर् मा॒तु र॒स्या उ॒पस्थ॑ उ॒पस्थे॑ अ॒स्या मा॒तुर् मा॒तु र॒स्या उ॒पस्थे᳚ । \newline
52. अ॒स्या उ॒पस्थ॑ उ॒पस्थे॑ अ॒स्या अ॒स्या उ॒पस्थे॒ विश्वा॑नि॒ विश्वा᳚ न्यु॒पस्थे॑ अ॒स्या अ॒स्या उ॒पस्थे॒ विश्वा॑नि । \newline
\pagebreak
\markright{ TS 4.2.1.5  \hfill https://www.vedavms.in \hfill}

\section{ TS 4.2.1.5 }

\textbf{TS 4.2.1.5 } \newline
\textbf{Samhita Paata} \newline

उ॒पस्थे॒ विश्वा᳚न्यग्ने व॒युना॑नि वि॒द्वान् । मैना॑म॒र्चिषा॒ मा तप॑सा॒ऽभि शू॑शुचो॒ऽन्तर॑स्याꣳ शु॒क्रज्यो॑ति॒र्वि भा॑हि ॥ अ॒न्तर॑ग्ने रु॒चा त्वमु॒खायै॒ सद॑ने॒ स्वे । तस्या॒स्त्वꣳ हर॑सा॒ तप॒ञ्जात॑वेदः शि॒वो भ॑व ॥ शि॒वो भू॒त्वा मह्य॑म॒ग्नेऽथो॑ सीद शि॒वस्त्वं । शि॒वाः कृ॒त्वा दिशः॒ सर्वाः॒ स्वां ॅयोनि॑मि॒हाऽऽ स॑दः ॥ हꣳ॒॒सः शु॑चि॒ष ( ) द्वसु॑रन्तरिक्ष॒-सद्धोता॑ वेदि॒षदति॑थि र्दुरोण॒सत् । नृ॒षद्व॑र॒सद् ऋ॑त॒सद् व्यो॑म॒सद् अ॒ब्जा गो॒जा ऋ॑त॒जा अ॑द्रि॒जा ऋ॒तं बृ॒हत् ॥ \newline

\textbf{Pada Paata} \newline

उ॒पस्थ॒ इत्यु॒प - स्थे॒ । विश्वा॑नि । अ॒ग्ने॒ । व॒युना॑नि । वि॒द्वान् ॥ मा । ए॒ना॒म् । अ॒र्चिषा᳚ । मा । तप॑सा । अ॒भीति॑ । शू॒शु॒चः॒ । अ॒न्तः । अ॒स्या॒म् । शु॒क्रज्यो॑ति॒रिति॑ शु॒क्र-ज्यो॒तिः॒ । वीति॑ । भा॒हि॒ ॥ अ॒न्तः । अ॒ग्ने॒ । रु॒चा । त्वम् । उ॒खायै᳚ । सद॑ने । स्वे ॥ तस्याः᳚ । त्वम् । हर॑सा । तपन्न्॑ । जात॑वेद॒ इति॒ जात॑-वे॒दः॒ । शि॒वः । भ॒व॒ ॥ शि॒वः । भू॒त्वा । मह्य᳚म् । अ॒ग्ने॒ । अथो॒ इति॑ । सी॒द॒ । शि॒वः । त्वम् ॥ शि॒वाः । कृ॒त्वा । दिशः॑ । सर्वाः᳚ । स्वाम् । योनि᳚म् । इ॒ह । एति॑ । अ॒स॒दः॒ ॥ हꣳ॒॒सः । शु॒चि॒षदिति॑ शुचि - सत् ( ) । वसुः॑ । अ॒न्त॒रि॒क्ष॒सदित्य॑न्तरिक्ष - सत् । होता᳚ । वे॒दि॒षदिति॑ वेदि - सत् । अति॑थिः । दु॒रो॒ण॒सदिति॑ दुरोण - सत् ॥ नृ॒षदिति॑ नृ - सत् । व॒र॒सदिति॑ वर-सत् । ऋ॒त॒सदित्यृ॑त-सत् । व्यो॒म॒सदिति॑ व्योम-सत् । अ॒ब्जा इत्य॑प्-जाः । गो॒जा इति॑ गो-जाः । ऋ॒त॒जा इत्यृ॑त - जाः । अ॒द्रि॒जा इत्य॑द्रि - जाः । ऋ॒तम् । बृ॒हत् ॥  \newline


\textbf{Krama Paata} \newline

उ॒पस्थे॒ विश्वा॑नि । उ॒पस्थ॒ इत्यु॒प - स्थे॒ । विश्वा᳚न्यग्ने । अ॒ग्ने॒ व॒युना॑नि । व॒युना॑नि वि॒द्वान् । वि॒द्वानिति॑ वि॒द्वान् ॥ मैना᳚म् । ए॒ना॒म॒र्चिषा᳚ । अ॒र्चिषा॒ मा । मा तप॑सा । तप॑सा॒ऽभि । अ॒भि शू॑शुचः । शू॒शु॒चो॒ऽन्तः । अ॒न्तर॑स्याम् । अ॒स्याꣳ॒॒ शु॒क्रज्यो॑तिः । शु॒क्रज्यो॑ति॒र् वि । शु॒क्रज्यो॑ति॒रिति॑ शु॒क्र - ज्यो॒तिः॒ । वि भा॑हि । भा॒हीति॑ भाहि ॥ अ॒न्तर॑ग्ने । अ॒ग्ने॒ रु॒चा । रु॒चा त्वम् । त्वमु॒खायै᳚ । उ॒खायै॒ सद॑ने । सद॑ने॒ स्वे । स्व इति॒ स्वे ॥ तस्या॒स्त्वम् । त्वꣳ हर॑सा । हर॑सा॒ तपन्न्॑ । तप॒न् जात॑वेदः । जात॑वेदः शि॒वः । जात॑वेद॒ इति॒ जात॑ - वे॒दः॒ । शि॒वो भ॑व । भ॒वेति॑ भव ॥ शि॒वो भू॒त्वा । भू॒त्वा मह्य᳚म् । मह्य॑मग्ने । अ॒ग्नेऽथो᳚ । अथो॑ सीद । अथो॒ इत्यथो᳚ । सी॒द॒ शि॒वः । शि॒वस्त्वम् । त्वमिति॒ त्वम् ॥ शि॒वाः कृ॒त्वा । कृ॒त्वा दिशः॑ । दिशः॒ सर्वाः᳚ । सर्वाः॒ स्वाम् । स्वां ॅयोनि᳚म् । योनि॑मि॒ह । इ॒हा । आऽस॑दः । अ॒स॒द॒ इत्य॑सदः ॥ हꣳ॒॒सः शु॑चि॒षत् ( ) । शु॒चि॒षद् वसुः॑ । शु॒चि॒षदिति॑ शुचि - सत् । वसु॑रन्तरिक्ष॒सत् । अ॒न्त॒रि॒क्ष॒सद्धोता᳚ । अ॒न्त॒रि॒क्ष॒सदित्य॑न्तरिक्ष - सत् । होता॑ वेदि॒षत् । वे॒दि॒षदति॑थिः । वे॒दि॒षदिति॑ वेदि - सत् । अति॑थिर् दुरोण॒सत् । दु॒रो॒ण॒सदिति॑ दुरोण - सत् ॥ नृ॒षद् व॑र॒सत् । नृ॒षदिति॑ नृ - सत् । व॒र॒सदृ॑त॒सत् । व॒र॒सदिति॑ वर - सत् । ऋ॒त॒सद् व्यो॑म॒सत् । ऋ॒त॒सदित्यृ॑त - सत् । व्यो॒म॒सद॒ब्जाः । व्यो॒म॒सदिति॑ व्योम - सत् । अ॒ब्जा गो॒जाः । अ॒ब्जा इत्य॑प् - जाः । गो॒जा ऋ॑त॒जाः । गो॒जा इति॑ गो - जाः । ऋ॒त॒जा अ॑द्रि॒जाः । ऋ॒त॒जा इत्यृ॑त - जाः । अ॒द्रि॒जा ऋ॒तम् । अ॒द्रि॒जा इत्य॑द्रि - जाः । ऋ॒तम् बृ॒हत् । बृ॒हदिति॑ बृ॒हत् । \newline

\textbf{Jatai Paata} \newline

1. उ॒पस्थे॒ विश्वा॑नि॒ विश्वा᳚ न्यु॒पस्थ॑ उ॒पस्थे॒ विश्वा॑नि । \newline
2. उ॒पस्थ॒ इत्यु॒प - स्थे॒ । \newline
3. विश्वा᳚ न्यग्ने अग्ने॒ विश्वा॑नि॒ विश्वा᳚ न्यग्ने । \newline
4. अ॒ग्ने॒ व॒युना॑नि व॒युना᳚ न्यग्ने अग्ने व॒युना॑नि । \newline
5. व॒युना॑नि वि॒द्वान्. वि॒द्वान्. व॒युना॑नि व॒युना॑नि वि॒द्वान् । \newline
6. वि॒द्वानिति॑ वि॒द्वान् । \newline
7. मैना॑ मेना॒म् मा मैना᳚म् । \newline
8. ए॒ना॒ म॒र्चिषा॒ ऽर्चिषै॑ना मेना म॒र्चिषा᳚ । \newline
9. अ॒र्चिषा॒ मा मा ऽर्चिषा॒ ऽर्चिषा॒ मा । \newline
10. मा तप॑सा॒ तप॑सा॒ मा मा तप॑सा । \newline
11. तप॑सा॒ ऽभ्य॑भि तप॑सा॒ तप॑सा॒ ऽभि । \newline
12. अ॒भि शू॑शुचः शूशुचो अ॒भ्य॑भि शू॑शुचः । \newline
13. शू॒शु॒चो॒ ऽन्त र॒न्तः शू॑शुचः शूशुचो॒ ऽन्तः । \newline
14. अ॒न्त र॑स्या मस्या म॒न्त र॒न्त र॑स्याम् । \newline
15. अ॒स्याꣳ॒॒ शु॒क्रज्यो॑तिः शु॒क्रज्यो॑ति रस्या मस्याꣳ शु॒क्रज्यो॑तिः । \newline
16. शु॒क्रज्यो॑ति॒र् वि वि शु॒क्रज्यो॑तिः शु॒क्रज्यो॑ति॒र् वि । \newline
17. शु॒क्रज्यो॑ति॒रिति॑ शु॒क्र - ज्यो॒तिः॒ । \newline
18. वि भा॑हि भाहि॒ वि वि भा॑हि । \newline
19. भा॒हीति॑ भाहि । \newline
20. अ॒न्त र॑ग्ने अग्ने अ॒न्त र॒न्त र॑ग्ने । \newline
21. अ॒ग्ने॒ रु॒चा रु॒चा ऽग्ने॑ अग्ने रु॒चा । \newline
22. रु॒चा त्वम् त्वꣳ रु॒चा रु॒चा त्वम् । \newline
23. त्व मु॒खाया॑ उ॒खायै॒ त्वम् त्व मु॒खायै᳚ । \newline
24. उ॒खायै॒ सद॑ने॒ सद॑न उ॒खाया॑ उ॒खायै॒ सद॑ने । \newline
25. सद॑ने॒ स्वे स्वे सद॑ने॒ सद॑ने॒ स्वे । \newline
26. स्व इति॒ स्वे । \newline
27. तस्या॒ स्त्वम् त्वम् तस्या॒ स्तस्या॒ स्त्वम् । \newline
28. त्वꣳ हर॑सा॒ हर॑सा॒ त्वम् त्वꣳ हर॑सा । \newline
29. हर॑सा॒ तप॒न् तप॒न्॒. हर॑सा॒ हर॑सा॒ तपन्न्॑ । \newline
30. तप॒न् जात॑वेदो॒ जात॑वेद॒ स्तप॒न् तप॒न् जात॑वेदः । \newline
31. जात॑वेदः शि॒वः शि॒वो जात॑वेदो॒ जात॑वेदः शि॒वः । \newline
32. जात॑वेद॒ इति॒ जात॑ - वे॒दः॒ । \newline
33. शि॒वो भ॑व भव शि॒वः शि॒वो भ॑व । \newline
34. भ॒वेति॑ भव । \newline
35. शि॒वो भू॒त्वा भू॒त्वा शि॒वः शि॒वो भू॒त्वा । \newline
36. भू॒त्वा मह्य॒म् मह्य॑म् भू॒त्वा भू॒त्वा मह्य᳚म् । \newline
37. मह्य॑ मग्ने अग्ने॒ मह्य॒म् मह्य॑ मग्ने । \newline
38. अ॒ग्ने ऽथो॒ अथो॑ अग्ने अ॒ग्ने ऽथो᳚ । \newline
39. अथो॑ सीद सी॒दाथो॒ अथो॑ सीद । \newline
40. अथो॒ इत्यथो᳚ । \newline
41. सी॒द॒ शि॒वः शि॒वः सी॑द सीद शि॒वः । \newline
42. शि॒व स्त्वम् त्वꣳ शि॒वः शि॒व स्त्वम् । \newline
43. त्वमिति॒ त्वम् । \newline
44. शि॒वाः कृ॒त्वा कृ॒त्वा शि॒वाः शि॒वाः कृ॒त्वा । \newline
45. कृ॒त्वा दिशो॒ दिशः॑ कृ॒त्वा कृ॒त्वा दिशः॑ । \newline
46. दिशः॒ सर्वाः॒ सर्वा॒ दिशो॒ दिशः॒ सर्वाः᳚ । \newline
47. सर्वाः॒ स्वाꣳ स्वाꣳ सर्वाः॒ सर्वाः॒ स्वाम् । \newline
48. स्वां ॅयोनिं॒ ॅयोनिꣳ॒॒ स्वाꣳ स्वां ॅयोनि᳚म् । \newline
49. योनि॑ मि॒हेह योनिं॒ ॅयोनि॑ मि॒ह । \newline
50. इ॒हे हेहा । \newline
51. आ ऽस॑दो असद॒ आ ऽस॑दः । \newline
52. अ॒स॒द॒ इत्य॑सदः । \newline
53. हꣳ॒॒सः शु॑चि॒ष च्छु॑चि॒ष द्धꣳ॒॒सो हꣳ॒॒सः शु॑चि॒षत् । \newline
54. शु॒चि॒षद् वसु॒र् वसुः॑ शुचि॒ष च्छु॑चि॒षद् वसुः॑ । \newline
55. शु॒चि॒षदिति॑ शुचि - सत् । \newline
56. वसु॑ रन्तरिक्ष॒स द॑न्तरिक्ष॒सद् वसु॒र् वसु॑ रन्तरिक्ष॒सत् । \newline
57. अ॒न्त॒रि॒क्ष॒स द्धोता॒ होता᳚ ऽन्तरिक्ष॒स द॑न्तरिक्ष॒स द्धोता᳚ । \newline
58. अ॒न्त॒रि॒क्ष॒सदित्य॑न्तरिक्ष - सत् । \newline
59. होता॑ वेदि॒षद् वे॑दि॒ष द्धोता॒ होता॑ वेदि॒षत् । \newline
60. वे॒दि॒ष दति॑थि॒ रति॑थिर् वेदि॒षद् वे॑दि॒ष दति॑थिः । \newline
61. वे॒दि॒षदिति॑ वेदि - सत् । \newline
62. अति॑थिर् दुरोण॒सद् दु॑रोण॒स दति॑थि॒ रति॑थिर् दुरोण॒सत् । \newline
63. दु॒रो॒ण॒सदिति॑ दुरोण - सत् । \newline
64. नृ॒षद् व॑र॒सद् व॑र॒सन् नृ॒षन् नृ॒षद् व॑र॒सत् । \newline
65. नृ॒षदिति॑ नृ - सत् । \newline
66. व॒र॒स दृ॑त॒स दृ॑त॒सद् व॑र॒सद् व॑र॒स दृ॑त॒सत् । \newline
67. व॒र॒सदिति॑ वर - सत् । \newline
68. ऋ॒त॒सद् व्यो॑म॒सद् व्यो॑म॒स दृ॑त॒स दृ॑त॒सद् व्यो॑म॒सत् । \newline
69. ऋ॒त॒सदित्यृ॑त - सत् । \newline
70. व्यो॒म॒स द॒ब्जा अ॒ब्जा व्यो॑म॒सद् व्यो॑म॒स द॒ब्जाः । \newline
71. व्यो॒म॒सदिति॑ व्योम - सत् । \newline
72. अ॒ब्जा गो॒जा गो॒जा अ॒ब्जा अ॒ब्जा गो॒जाः । \newline
73. अ॒ब्जा इत्य॑प् - जाः । \newline
74. गो॒जा ऋ॑त॒जा ऋ॑त॒जा गो॒जा गो॒जा ऋ॑त॒जाः । \newline
75. गो॒जा इति॑ गो - जाः । \newline
76. ऋ॒त॒जा अ॑द्रि॒जा अ॑द्रि॒जा ऋ॑त॒जा ऋ॑त॒जा अ॑द्रि॒जाः । \newline
77. ऋ॒त॒जा इत्यृ॑त - जाः । \newline
78. अ॒द्रि॒जा ऋ॒त मृ॒त म॑द्रि॒जा अ॑द्रि॒जा ऋ॒तम् । \newline
79. अ॒द्रि॒जा इत्य॑द्रि - जाः । \newline
80. ऋ॒तम् बृ॒हद् बृ॒हदृ॒त मृ॒तम् बृ॒हत् । \newline
81. बृ॒हदिति॑ बृ॒हत् । \newline

\textbf{Ghana Paata } \newline

1. उ॒पस्थे॒ विश्वा॑नि॒ विश्वा᳚ न्यु॒पस्थ॑ उ॒पस्थे॒ विश्वा᳚ न्यग्ने अग्ने॒ विश्वा᳚ न्यु॒पस्थ॑ उ॒पस्थे॒ विश्वा᳚ न्यग्ने । \newline
2. उ॒पस्थ॒ इत्यु॒प - स्थे॒ । \newline
3. विश्वा᳚ न्यग्ने अग्ने॒ विश्वा॑नि॒ विश्वा᳚ न्यग्ने व॒युना॑नि व॒युना᳚ न्यग्ने॒ विश्वा॑नि॒ विश्वा᳚ न्यग्ने व॒युना॑नि । \newline
4. अ॒ग्ने॒ व॒युना॑नि व॒युना᳚ न्यग्ने अग्ने व॒युना॑नि वि॒द्वान्. वि॒द्वान्. व॒युना᳚ न्यग्ने अग्ने व॒युना॑नि वि॒द्वान् । \newline
5. व॒युना॑नि वि॒द्वान्. वि॒द्वान्. व॒युना॑नि व॒युना॑नि वि॒द्वान् । \newline
6. वि॒द्वानिति॑ वि॒द्वान् । \newline
7. मैना॑ मेना॒म् मा मैना॑ म॒र्चिषा॒ ऽर्चिषै॑ना॒म् मा मैना॑ म॒र्चिषा᳚ । \newline
8. ए॒ना॒ म॒र्चिषा॒ ऽर्चिषै॑ना मेना म॒र्चिषा॒ मा मा ऽर्चिषै॑ना मेना म॒र्चिषा॒ मा । \newline
9. अ॒र्चिषा॒ मा मा ऽर्चिषा॒ ऽर्चिषा॒ मा तप॑सा॒ तप॑सा॒ मा ऽर्चिषा॒ ऽर्चिषा॒ मा तप॑सा । \newline
10. मा तप॑सा॒ तप॑सा॒ मा मा तप॑सा॒ ऽभ्य॑भि तप॑सा॒ मा मा तप॑सा॒ ऽभि । \newline
11. तप॑सा॒ ऽभ्य॑भि तप॑सा॒ तप॑सा॒ ऽभि शू॑शुचः शूशुचो अ॒भि तप॑सा॒ तप॑सा॒ ऽभि शू॑शुचः । \newline
12. अ॒भि शू॑शुचः शूशुचो अ॒भ्य॑भि शू॑शुचो॒ ऽन्त र॒न्तः शू॑शुचो अ॒भ्य॑भि शू॑शुचो॒ ऽन्तः । \newline
13. शू॒शु॒चो॒ ऽन्त र॒न्तः शू॑शुचः शूशुचो॒ ऽन्त र॑स्या मस्या म॒न्तः शू॑शुचः शूशुचो॒ ऽन्त र॑स्याम् । \newline
14. अ॒न्त र॑स्या मस्या म॒न्त र॒न्त र॑स्याꣳ शु॒क्रज्यो॑तिः शु॒क्रज्यो॑ति रस्या म॒न्त र॒न्त र॑स्याꣳ शु॒क्रज्यो॑तिः । \newline
15. अ॒स्याꣳ॒॒ शु॒क्रज्यो॑तिः शु॒क्रज्यो॑ति रस्या मस्याꣳ शु॒क्रज्यो॑ति॒र् वि वि शु॒क्रज्यो॑ति रस्या मस्याꣳ शु॒क्रज्यो॑ति॒र् वि । \newline
16. शु॒क्रज्यो॑ति॒र् वि वि शु॒क्रज्यो॑तिः शु॒क्रज्यो॑ति॒र् वि भा॑हि भाहि॒ वि शु॒क्रज्यो॑तिः शु॒क्रज्यो॑ति॒र् वि भा॑हि । \newline
17. शु॒क्रज्यो॑ति॒रिति॑ शु॒क्र - ज्यो॒तिः॒ । \newline
18. वि भा॑हि भाहि॒ वि वि भा॑हि । \newline
19. भा॒हीति॑ भाहि । \newline
20. अ॒न्त र॑ग्ने अग्ने अ॒न्त र॒न्त र॑ग्ने रु॒चा रु॒चा ऽग्ने अ॒न्त र॒न्त र॑ग्ने रु॒चा । \newline
21. अ॒ग्ने॒ रु॒चा रु॒चा ऽग्ने॑ अग्ने रु॒चा त्वम् त्वꣳ रु॒चा ऽग्ने॑ अग्ने रु॒चा त्वम् । \newline
22. रु॒चा त्वम् त्वꣳ रु॒चा रु॒चा त्व मु॒खाया॑ उ॒खायै॒ त्वꣳ रु॒चा रु॒चा त्व मु॒खायै᳚ । \newline
23. त्व मु॒खाया॑ उ॒खायै॒ त्वम् त्व मु॒खायै॒ सद॑ने॒ सद॑न उ॒खायै॒ त्वम् त्व मु॒खायै॒ सद॑ने । \newline
24. उ॒खायै॒ सद॑ने॒ सद॑न उ॒खाया॑ उ॒खायै॒ सद॑ने॒ स्वे स्वे सद॑न उ॒खाया॑ उ॒खायै॒ सद॑ने॒ स्वे । \newline
25. सद॑ने॒ स्वे स्वे सद॑ने॒ सद॑ने॒ स्वे । \newline
26. स्व इति॒ स्वे । \newline
27. तस्या॒ स्त्वम् त्वम् तस्या॒ स्तस्या॒ स्त्वꣳ हर॑सा॒ हर॑सा॒ त्वम् तस्या॒ स्तस्या॒ स्त्वꣳ हर॑सा । \newline
28. त्वꣳ हर॑सा॒ हर॑सा॒ त्वम् त्वꣳ हर॑सा॒ तप॒न् तप॒न्॒. हर॑सा॒ त्वम् त्वꣳ हर॑सा॒ तपन्न्॑ । \newline
29. हर॑सा॒ तप॒न् तप॒न्॒. हर॑सा॒ हर॑सा॒ तप॒न् जात॑वेदो॒ जात॑वेद॒ स्तप॒न्॒. हर॑सा॒ हर॑सा॒ तप॒न् जात॑वेदः । \newline
30. तप॒न् जात॑वेदो॒ जात॑वेद॒ स्तप॒न् तप॒न् जात॑वेदः शि॒वः शि॒वो जात॑वेद॒ स्तप॒न् तप॒न् जात॑वेदः शि॒वः । \newline
31. जात॑वेदः शि॒वः शि॒वो जात॑वेदो॒ जात॑वेदः शि॒वो भ॑व भव शि॒वो जात॑वेदो॒ जात॑वेदः शि॒वो भ॑व । \newline
32. जात॑वेद॒ इति॒ जात॑ - वे॒दः॒ । \newline
33. शि॒वो भ॑व भव शि॒वः शि॒वो भ॑व । \newline
34. भ॒वेति॑ भव । \newline
35. शि॒वो भू॒त्वा भू॒त्वा शि॒वः शि॒वो भू॒त्वा मह्य॒म् मह्य॑म् भू॒त्वा शि॒वः शि॒वो भू॒त्वा मह्य᳚म् । \newline
36. भू॒त्वा मह्य॒म् मह्य॑म् भू॒त्वा भू॒त्वा मह्य॑ मग्ने अग्ने॒ मह्य॑म् भू॒त्वा भू॒त्वा मह्य॑ मग्ने । \newline
37. मह्य॑ मग्ने अग्ने॒ मह्य॒म् मह्य॑ म॒ग्ने ऽथो॒ अथो॑ अग्ने॒ मह्य॒म् मह्य॑ म॒ग्ने ऽथो᳚ । \newline
38. अ॒ग्ने ऽथो॒ अथो॑ अग्ने अ॒ग्ने ऽथो॑ सीद सी॒दाथो॑ अग्ने अ॒ग्ने ऽथो॑ सीद । \newline
39. अथो॑ सीद सी॒दाथो॒ अथो॑ सीद शि॒वः शि॒वः सी॒दाथो॒ अथो॑ सीद शि॒वः । \newline
40. अथो॒ इत्यथो᳚ । \newline
41. सी॒द॒ शि॒वः शि॒वः सी॑द सीद शि॒व स्त्वम् त्वꣳ शि॒वः सी॑द सीद शि॒व स्त्वम् । \newline
42. शि॒व स्त्वम् त्वꣳ शि॒वः शि॒व स्त्वम् । \newline
43. त्वमिति॒ त्वम् । \newline
44. शि॒वाः कृ॒त्वा कृ॒त्वा शि॒वाः शि॒वाः कृ॒त्वा दिशो॒ दिशः॑ कृ॒त्वा शि॒वाः शि॒वाः कृ॒त्वा दिशः॑ । \newline
45. कृ॒त्वा दिशो॒ दिशः॑ कृ॒त्वा कृ॒त्वा दिशः॒ सर्वाः॒ सर्वा॒ दिशः॑ कृ॒त्वा कृ॒त्वा दिशः॒ सर्वाः᳚ । \newline
46. दिशः॒ सर्वाः॒ सर्वा॒ दिशो॒ दिशः॒ सर्वाः॒ स्वाꣳ स्वाꣳ सर्वा॒ दिशो॒ दिशः॒ सर्वाः॒ स्वाम् । \newline
47. सर्वाः॒ स्वाꣳ स्वाꣳ सर्वाः॒ सर्वाः॒ स्वां ॅयोनिं॒ ॅयोनिꣳ॒॒ स्वाꣳ सर्वाः॒ सर्वाः॒ स्वां ॅयोनि᳚म् । \newline
48. स्वां ॅयोनिं॒ ॅयोनिꣳ॒॒ स्वाꣳ स्वां ॅयोनि॑ मि॒हेह योनिꣳ॒॒ स्वाꣳ स्वां ॅयोनि॑ मि॒ह । \newline
49. योनि॑ मि॒हेह योनिं॒ ॅयोनि॑ मि॒हेह योनिं॒ ॅयोनि॑ मि॒हा । \newline
50. इ॒हेहे हा ऽस॑दो असद॒ एहेहा ऽस॑दः । \newline
51. आ ऽस॑दो असद॒ आ ऽस॑दः । \newline
52. अ॒स॒द॒ इत्य॑सदः । \newline
53. हꣳ॒॒सः शु॑चि॒ष च्छु॑चि॒ष द्धꣳ॒॒सो हꣳ॒॒सः शु॑चि॒षद् वसु॒र् वसुः॑ शुचि॒ष द्धꣳ॒॒सो हꣳ॒॒सः शु॑चि॒षद् वसुः॑ । \newline
54. शु॒चि॒षद् वसु॒र् वसुः॑ शुचि॒ष च्छु॑चि॒षद् वसु॑ रन्तरिक्ष॒स द॑न्तरिक्ष॒सद् वसुः॑ शुचि॒ष च्छु॑चि॒षद् वसु॑ रन्तरिक्ष॒सत् । \newline
55. शु॒चि॒षदिति॑ शुचि - सत् । \newline
56. वसु॑ रन्तरिक्ष॒स द॑न्तरिक्ष॒सद् वसु॒र् वसु॑ रन्तरिक्ष॒स द्धोता॒ होता᳚ ऽन्तरिक्ष॒सद् वसु॒र् वसु॑र् अन्तरिक्ष॒स द्धोता᳚ । \newline
57. अ॒न्त॒रि॒क्ष॒स द्धोता॒ होता᳚ ऽन्तरिक्ष॒स द॑न्तरिक्ष॒स द्धोता॑ वेदि॒षद् वे॑दि॒ष द्धोता᳚ ऽन्तरिक्ष॒स द॑न्तरिक्ष॒स द्धोता॑ वेदि॒षत् । \newline
58. अ॒न्त॒रि॒क्ष॒सदित्य॑न्तरिक्ष - सत् । \newline
59. होता॑ वेदि॒षद् वे॑दि॒ष द्धोता॒ होता॑ वेदि॒ष दति॑थि॒ रति॑थिर् वेदि॒ष द्धोता॒ होता॑ वेदि॒ष दति॑थिः । \newline
60. वे॒दि॒ष दति॑थि॒ रति॑थिर् वेदि॒षद् वे॑दि॒ष दति॑थिर् दुरोण॒सद् दु॑रोण॒स दति॑थिर् वेदि॒षद् वे॑दि॒ष दति॑थिर् दुरोण॒सत् । \newline
61. वे॒दि॒षदिति॑ वेदि - सत् । \newline
62. अति॑थिर् दुरोण॒सद् दु॑रोण॒ सदति॑थि॒ रति॑थिर् दुरोण॒सत् । \newline
63. दु॒रो॒ण॒सदिति॑ दुरोण - सत् । \newline
64. नृ॒षद् व॑र॒सद् व॑र॒सन् नृ॒षन् नृ॒षद् व॑र॒स दृ॑त॒स दृ॑त॒सद् व॑र॒सन् नृ॒षन् नृ॒षद् व॑र॒स दृ॑त॒सत् । \newline
65. नृ॒षदिति॑ नृ - सत् । \newline
66. व॒र॒स दृ॑त॒स दृ॑त॒सद् व॑र॒सद् व॑र॒स दृ॑त॒सद् व्यो॑म॒सद् व्यो॑म॒स दृ॑त॒सद् व॑र॒सद् व॑र॒स दृ॑त॒सद् व्यो॑म॒सत् । \newline
67. व॒र॒सदिति॑ वर - सत् । \newline
68. ऋ॒त॒सद् व्यो॑म॒सद् व्यो॑म॒स दृ॑त॒स दृ॑त॒सद् व्यो॑म॒स द॒ब्जा अ॒ब्जा व्यो॑म॒स दृ॑त॒स दृ॑त॒सद् व्यो॑म॒स द॒ब्जाः । \newline
69. ऋ॒त॒सदित्यृ॑त - सत् । \newline
70. व्यो॒म॒स द॒ब्जा अ॒ब्जा व्यो॑म॒सद् व्यो॑म॒स द॒ब्जा गो॒जा गो॒जा अ॒ब्जा व्यो॑म॒सद् व्यो॑म॒स द॒ब्जा गो॒जाः । \newline
71. व्यो॒म॒सदिति॑ व्योम - सत् । \newline
72. अ॒ब्जा गो॒जा गो॒जा अ॒ब्जा अ॒ब्जा गो॒जा ऋ॑त॒जा ऋ॑त॒जा गो॒जा अ॒ब्जा अ॒ब्जा गो॒जा ऋ॑त॒जाः । \newline
73. अ॒ब्जा इत्य॑प् - जाः । \newline
74. गो॒जा ऋ॑त॒जा ऋ॑त॒जा गो॒जा गो॒जा ऋ॑त॒जा अ॑द्रि॒जा अ॑द्रि॒जा ऋ॑त॒जा गो॒जा गो॒जा ऋ॑त॒जा अ॑द्रि॒जाः । \newline
75. गो॒जा इति॑ गो - जाः । \newline
76. ऋ॒त॒जा अ॑द्रि॒जा अ॑द्रि॒जा ऋ॑त॒जा ऋ॑त॒जा अ॑द्रि॒जा ऋ॒त मृ॒त म॑द्रि॒जा ऋ॑त॒जा ऋ॑त॒जा अ॑द्रि॒जा ऋ॒तम् । \newline
77. ऋ॒त॒जा इत्यृ॑त - जाः । \newline
78. अ॒द्रि॒जा ऋ॒त मृ॒त म॑द्रि॒जा अ॑द्रि॒जा ऋ॒तम् बृ॒हद् बृ॒ह दृ॒त म॑द्रि॒जा अ॑द्रि॒जा ऋ॒तम् बृ॒हत् । \newline
79. अ॒द्रि॒जा इत्य॑द्रि - जाः । \newline
80. ऋ॒तम् बृ॒हद् बृ॒ह दृ॒त मृ॒तम् बृ॒हत् । \newline
81. बृ॒हदिति॑ बृ॒हत् । \newline
\pagebreak
\markright{ TS 4.2.2.1  \hfill https://www.vedavms.in \hfill}

\section{ TS 4.2.2.1 }

\textbf{TS 4.2.2.1 } \newline
\textbf{Samhita Paata} \newline

दि॒वस्परि॑ प्रथ॒मं ज॑ज्ञे अ॒ग्निर॒स्मद् द्वि॒तीयं॒ परि॑ जा॒तवे॑दाः । तृ॒तीय॑म॒फ्सु नृ॒मणा॒ अज॑स्र॒मिन्धा॑न एनं जरते स्वा॒धीः ॥ वि॒द्मा ते॑ अग्ने त्रे॒धा त्र॒याणि॑ वि॒द्मा ते॒ सद्म॒ विभृ॑तं पुरु॒त्रा । वि॒द्मा ते॒ नाम॑ पर॒मं गुहा॒ यद्वि॒द्मा तमुथ्सं॒ ॅयत॑ आज॒गन्थ॑ ॥ स॒मु॒द्रे त्वा॑ नृ॒मणा॑ अ॒फ्स्व॑न्तर्नृ॒चक्षा॑ ईधे दि॒वो अ॑ग्न॒ ऊधन्न्॑ । तृ॒तीये᳚ त्वा॒ - [  ] \newline

\textbf{Pada Paata} \newline

दि॒वः । परीति॑ । प्र॒थ॒मम् । ज॒ज्ञे॒ । अ॒ग्निः । अ॒स्मत् । द्वि॒तीय᳚म् । परीति॑ । जा॒तवे॑दा॒ इति॑ जा॒त - वे॒दाः॒ ॥ तृ॒तीय᳚म् । अ॒फ्स्वित्य॑प्-सु । नृ॒मणा॒ इति॑ नृ - मनाः᳚ । अज॑स्रम् । इन्धा॑नः । ए॒न॒म् । ज॒र॒ते॒ । स्वा॒धीरिति॑ स्वा-धीः ॥ वि॒द्म । ते॒ । अ॒ग्ने॒ । त्रे॒धा । त्र॒याणि॑ । वि॒द्म । ते॒ । सद्म॑ । विभृ॑त॒मिति॒ वि - भृ॒त॒म् । पु॒रु॒त्रेति॑ पुरु - त्रा ॥ वि॒द्म । ते॒ । नाम॑ । प॒र॒मम् । गुहा᳚ । यत् । वि॒द्म । तम् । उथ्स᳚म् । यतः॑ । आ॒ज॒गन्थेत्या᳚ - ज॒गन्थ॑ ॥ स॒मु॒द्रे । त्वा॒ । नृ॒मणा॒ इति॑ नृ - मनाः᳚ । अ॒फ्स्वित्य॑प्- सु । अ॒न्तः । नृ॒चक्षा॒ इति॑ नृ-चक्षाः᳚ । ई॒धे॒ । दि॒वः । अ॒ग्ने॒ । ऊधन्न्॑ ॥ तृ॒तीये᳚ । त्वा॒ ।  \newline


\textbf{Krama Paata} \newline

दि॒वस्परि॑ । परि॑ प्रथ॒मम् । प्र॒थ॒मम् ज॑ज्ञे । ज॒ज्ञे॒ अ॒ग्निः । अ॒ग्निर॒स्मत् । अ॒स्मद् द्वि॒तीय᳚म् । द्वि॒तीय॒म् परि॑ । परि॑ जा॒तवे॑दाः । जा॒तवे॑दा॒ इति॑ जा॒त - वे॒दाः॒ ॥ तृ॒तीय॑म॒फ्सु । अ॒फ्सु नृ॒मणाः᳚ । अ॒फ्सित्य॑प् - सु । नृ॒मणा॒ अज॑स्रम् । नृ॒मणा॒ इति॑ नृ - मनाः᳚ । अज॑स्र॒मिन्धा॑नः । इन्धा॑न एनम् । ए॒न॒म् ज॒र॒ते॒ । ज॒र॒ते॒ स्वा॒धीः । स्वा॒धीरिति॑ स्व - धीः ॥ वि॒द्मा ते᳚ । ते॒ अ॒ग्ने॒ । अ॒ग्ने॒ त्रे॒धा । त्रे॒धा त्र॒याणि॑ । त्र॒याणि॑ वि॒द्म । वि॒द्मा ते᳚ । ते॒ सद्म॑ । सद्म॒ विभृ॑तम् । विभृ॑तम् पुरु॒त्रा । विभृ॑त॒मिति॒ वि - भृ॒त॒म् । पु॒रु॒त्रेति॑ पुरु - त्रा ॥ वि॒द्मा ते᳚ । ते॒ नाम॑ । नाम॑ पर॒मम् । प॒र॒मम् गुहा᳚ । गुहा॒ यत् । यद् वि॒द्म । वि॒द्मा तम् । तमुथ्स᳚म् । उथ्सं॒ ॅयतः॑ । यत॑ आज॒गन्थ॑ । आ॒ज॒गन्थेत्या᳚ - ज॒गन्थ॑ ॥ स॒मु॒द्रे त्वा᳚ । त्वा॒ नृ॒मणाः᳚ । नृ॒मणा॑ अ॒फ्सु । नृ॒मणा॒ इति॑ नृ - मनाः᳚ । अ॒फ्स्व॑न्तः । अ॒फ्स्वित्य॑प् - सु । अ॒न्तर् नृ॒चक्षाः᳚ । नृ॒चक्षा॑ ईधे । नृ॒चक्षा॒ इति॑ नृ - चक्षाः᳚ । ई॒धे॒ दि॒वः । दि॒वो अ॑ग्ने । अ॒ग्न॒ ऊधन्न्॑ । ऊध॒न्नित्यूधन्न्॑ ॥ तृ॒तीये᳚ त्वा । त्वा॒ रज॑सि \newline

\textbf{Jatai Paata} \newline

1. दि॒व स्परि॒ परि॑ दि॒वो दि॒व स्परि॑ । \newline
2. परि॑ प्रथ॒मम् प्र॑थ॒मम् परि॒ परि॑ प्रथ॒मम् । \newline
3. प्र॒थ॒मम् ज॑ज्ञे जज्ञे प्रथ॒मम् प्र॑थ॒मम् ज॑ज्ञे । \newline
4. ज॒ज्ञे॒ अ॒ग्नि र॒ग्निर् ज॑ज्ञे जज्ञे अ॒ग्निः । \newline
5. अ॒ग्नि र॒स्म द॒स्म द॒ग्नि र॒ग्नि र॒स्मत् । \newline
6. अ॒स्मद् द्वि॒तीय॑म् द्वि॒तीय॑ म॒स्म द॒स्मद् द्वि॒तीय᳚म् । \newline
7. द्वि॒तीय॒म् परि॒ परि॑ द्वि॒तीय॑म् द्वि॒तीय॒म् परि॑ । \newline
8. परि॑ जा॒तवे॑दा जा॒तवे॑दाः॒ परि॒ परि॑ जा॒तवे॑दाः । \newline
9. जा॒तवे॑दा॒ इति॑ जा॒त - वे॒दाः॒ । \newline
10. तृ॒तीय॑ म॒फ्स्व॑फ्सु तृ॒तीय॑म् तृ॒तीय॑ म॒फ्सु । \newline
11. अ॒फ्सु नृ॒मणा॑ नृ॒मणा॑ अ॒फ्स्व॑फ्सु नृ॒मणाः᳚ । \newline
12. अ॒फ्स्वित्य॑प् - सु । \newline
13. नृ॒मणा॒ अज॑स्र॒ मज॑स्रम् नृ॒मणा॑ नृ॒मणा॒ अज॑स्रम् । \newline
14. नृ॒मणा॒ इति॑ नृ - मनाः᳚ । \newline
15. अज॑स्र॒ मिन्धा॑न॒ इन्धा॒नो ऽज॑स्र॒ मज॑स्र॒ मिन्धा॑नः । \newline
16. इन्धा॑न एन मेन॒ मिन्धा॑न॒ इन्धा॑न एनम् । \newline
17. ए॒न॒म् ज॒र॒ते॒ ज॒र॒त॒ ए॒न॒ मे॒न॒म् ज॒र॒ते॒ । \newline
18. ज॒र॒ते॒ स्वा॒धीः स्वा॒धीर् ज॑रते जरते स्वा॒धीः । \newline
19. स्वा॒धीरिति॑ स्व - धीः । \newline
20. वि॒द्मा ते॑ ते वि॒द्म वि॒द्मा ते᳚ । \newline
21. ते॒ अ॒ग्ने॒ अ॒ग्ने॒ ते॒ ते॒ अ॒ग्ने॒ । \newline
22. अ॒ग्ने॒ त्रे॒धा त्रे॒धा ऽग्ने॑ अग्ने त्रे॒धा । \newline
23. त्रे॒धा त्र॒याणि॑ त्र॒याणि॑ त्रे॒धा त्रे॒धा त्र॒याणि॑ । \newline
24. त्र॒याणि॑ वि॒द्म वि॒द्म त्र॒याणि॑ त्र॒याणि॑ वि॒द्म । \newline
25. वि॒द्मा ते॑ ते वि॒द्म वि॒द्मा ते᳚ । \newline
26. ते॒ सद्म॒ सद्म॑ ते ते॒ सद्म॑ । \newline
27. सद्म॒ विभृ॑तं॒ ॅविभृ॑तꣳ॒॒ सद्म॒ सद्म॒ विभृ॑तम् । \newline
28. विभृ॑तम् पुरु॒त्रा पु॑रु॒त्रा विभृ॑तं॒ ॅविभृ॑तम् पुरु॒त्रा । \newline
29. विभृ॑त॒मिति॒ वि - भृ॒त॒म् । \newline
30. पु॒रु॒त्रेति॑ पुरु - त्रा । \newline
31. वि॒द्मा ते॑ ते वि॒द्म वि॒द्मा ते᳚ । \newline
32. ते॒ नाम॒ नाम॑ ते ते॒ नाम॑ । \newline
33. नाम॑ पर॒मम् प॑र॒मम् नाम॒ नाम॑ पर॒मम् । \newline
34. प॒र॒मम् गुहा॒ गुहा॑ पर॒मम् प॑र॒मम् गुहा᳚ । \newline
35. गुहा॒ यद् यद् गुहा॒ गुहा॒ यत् । \newline
36. यद् वि॒द्म वि॒द्म यद् यद् वि॒द्म । \newline
37. वि॒द्मा तम् तं ॅवि॒द्म वि॒द्मा तम् । \newline
38. त मुथ्स॒ मुथ्स॒म् तम् त मुथ्स᳚म् । \newline
39. उथ्सं॒ ॅयतो॒ यत॒ उथ्स॒ मुथ्सं॒ ॅयतः॑ । \newline
40. यत॑ आज॒गन्था॑ ज॒गन्थ॒ यतो॒ यत॑ आज॒गन्थ॑ । \newline
41. आ॒ज॒गन्थेत्या᳚ - ज॒गन्थ॑ । \newline
42. स॒मु॒द्रे त्वा᳚ त्वा समु॒द्रे स॑मु॒द्रे त्वा᳚ । \newline
43. त्वा॒ नृ॒मणा॑ नृ॒मणा᳚ स्त्वा त्वा नृ॒मणाः᳚ । \newline
44. नृ॒मणा॑ अ॒फ्स्व॑फ्सु नृ॒मणा॑ नृ॒मणा॑ अ॒फ्सु । \newline
45. नृ॒मणा॒ इति॑ नृ - मनाः᳚ । \newline
46. अ॒फ्स्व॑न्त र॒न्त र॒फ्स्वा᳚(1॒)फ्स्व॑न्तः । \newline
47. अ॒फ्स्वित्य॑प् - सु । \newline
48. अ॒न्तर् नृ॒चक्षा॑ नृ॒चक्षा॑ अ॒न्त र॒न्तर् नृ॒चक्षाः᳚ । \newline
49. नृ॒चक्षा॑ ईध ईधे नृ॒चक्षा॑ नृ॒चक्षा॑ ईधे । \newline
50. नृ॒चक्षा॒ इति॑ नृ - चक्षाः᳚ । \newline
51. ई॒धे॒ दि॒वो दि॒व ई॑ध ईधे दि॒वः । \newline
52. दि॒वो अ॑ग्ने अग्ने दि॒वो दि॒वो अ॑ग्ने । \newline
53. अ॒ग्न॒ ऊध॒न् नूध॑न् नग्ने अग्न॒ ऊधन्न्॑ । \newline
54. ऊध॒न्नित्यूधन्न्॑ । \newline
55. तृ॒तीये᳚ त्वा त्वा तृ॒तीये॑ तृ॒तीये᳚ त्वा । \newline
56. त्वा॒ रज॑सि॒ रज॑सि त्वा त्वा॒ रज॑सि । \newline

\textbf{Ghana Paata } \newline

1. दि॒व स्परि॒ परि॑ दि॒वो दि॒व स्परि॑ प्रथ॒मम् प्र॑थ॒मम् परि॑ दि॒वो दि॒व स्परि॑ प्रथ॒मम् । \newline
2. परि॑ प्रथ॒मम् प्र॑थ॒मम् परि॒ परि॑ प्रथ॒मम् ज॑ज्ञे जज्ञे प्रथ॒मम् परि॒ परि॑ प्रथ॒मम् ज॑ज्ञे । \newline
3. प्र॒थ॒मम् ज॑ज्ञे जज्ञे प्रथ॒मम् प्र॑थ॒मम् ज॑ज्ञे अ॒ग्नि र॒ग्निर् ज॑ज्ञे प्रथ॒मम् प्र॑थ॒मम् ज॑ज्ञे अ॒ग्निः । \newline
4. ज॒ज्ञे॒ अ॒ग्नि र॒ग्निर् ज॑ज्ञे जज्ञे अ॒ग्नि र॒स्म द॒स्म द॒ग्निर् ज॑ज्ञे जज्ञे अ॒ग्नि र॒स्मत् । \newline
5. अ॒ग्नि र॒स्म द॒स्म द॒ग्नि र॒ग्नि र॒स्मद् द्वि॒तीय॑म् द्वि॒तीय॑ म॒स्म द॒ग्नि र॒ग्नि र॒स्मद् द्वि॒तीय᳚म् । \newline
6. अ॒स्मद् द्वि॒तीय॑म् द्वि॒तीय॑ म॒स्म द॒स्मद् द्वि॒तीय॒म् परि॒ परि॑ द्वि॒तीय॑ म॒स्म द॒स्मद् द्वि॒तीय॒म् परि॑ । \newline
7. द्वि॒तीय॒म् परि॒ परि॑ द्वि॒तीय॑म् द्वि॒तीय॒म् परि॑ जा॒तवे॑दा जा॒तवे॑दाः॒ परि॑ द्वि॒तीय॑म् द्वि॒तीय॒म् परि॑ जा॒तवे॑दाः । \newline
8. परि॑ जा॒तवे॑दा जा॒तवे॑दाः॒ परि॒ परि॑ जा॒तवे॑दाः । \newline
9. जा॒तवे॑दा॒ इति॑ जा॒त - वे॒दाः॒ । \newline
10. तृ॒तीय॑ म॒फ्स्व॑फ्सु तृ॒तीय॑म् तृ॒तीय॑ म॒फ्सु नृ॒मणा॑ नृ॒मणा॑ अ॒फ्सु तृ॒तीय॑म् तृ॒तीय॑ म॒फ्सु नृ॒मणाः᳚ । \newline
11. अ॒फ्सु नृ॒मणा॑ नृ॒मणा॑ अ॒फ्स्व॑फ्सु नृ॒मणा॒ अज॑स्र॒ मज॑स्रम् नृ॒मणा॑ अ॒फ्स्व॑फ्सु नृ॒मणा॒ अज॑स्रम् । \newline
12. अ॒फ्स्वित्य॑प् - सु । \newline
13. नृ॒मणा॒ अज॑स्र॒ मज॑स्रम् नृ॒मणा॑ नृ॒मणा॒ अज॑स्र॒ मिन्धा॑न॒ इन्धा॒नो ऽज॑स्रम् नृ॒मणा॑ नृ॒मणा॒ अज॑स्र॒ मिन्धा॑नः । \newline
14. नृ॒मणा॒ इति॑ नृ - मनाः᳚ । \newline
15. अज॑स्र॒ मिन्धा॑न॒ इन्धा॒नो ऽज॑स्र॒ मज॑स्र॒ मिन्धा॑न एन मेन॒ मिन्धा॒नो ऽज॑स्र॒ मज॑स्र॒ मिन्धा॑न एनम् । \newline
16. इन्धा॑न एन मेन॒ मिन्धा॑न॒ इन्धा॑न एनम् जरते जरत एन॒ मिन्धा॑न॒ इन्धा॑न एनम् जरते । \newline
17. ए॒न॒म् ज॒र॒ते॒ ज॒र॒त॒ ए॒न॒ मे॒न॒म् ज॒र॒ते॒ स्वा॒धीः स्वा॒धीर् ज॑रत एन मेनम् जरते स्वा॒धीः । \newline
18. ज॒र॒ते॒ स्वा॒धीः स्वा॒धीर् ज॑रते जरते स्वा॒धीः । \newline
19. स्वा॒धीरिति॑ स्व - धीः । \newline
20. वि॒द्मा ते॑ ते वि॒द्म वि॒द्मा ते॑ अग्ने अग्ने ते वि॒द्म वि॒द्मा ते॑ अग्ने । \newline
21. ते॒ अ॒ग्ने॒ अ॒ग्ने॒ ते॒ ते॒ अ॒ग्ने॒ त्रे॒धा त्रे॒धा ऽग्ने॑ ते ते अग्ने त्रे॒धा । \newline
22. अ॒ग्ने॒ त्रे॒धा त्रे॒धा ऽग्ने॑ अग्ने त्रे॒धा त्र॒याणि॑ त्र॒याणि॑ त्रे॒धा ऽग्ने॑ अग्ने त्रे॒धा त्र॒याणि॑ । \newline
23. त्रे॒धा त्र॒याणि॑ त्र॒याणि॑ त्रे॒धा त्रे॒धा त्र॒याणि॑ वि॒द्म वि॒द्म त्र॒याणि॑ त्रे॒धा त्रे॒धा त्र॒याणि॑ वि॒द्म । \newline
24. त्र॒याणि॑ वि॒द्म वि॒द्म त्र॒याणि॑ त्र॒याणि॑ वि॒द्मा ते॑ ते वि॒द्म त्र॒याणि॑ त्र॒याणि॑ वि॒द्मा ते᳚ । \newline
25. वि॒द्मा ते॑ ते वि॒द्म वि॒द्मा ते॒ सद्म॒ सद्म॑ ते वि॒द्म वि॒द्मा ते॒ सद्म॑ । \newline
26. ते॒ सद्म॒ सद्म॑ ते ते॒ सद्म॒ विभृ॑तं॒ ॅविभृ॑तꣳ॒॒ सद्म॑ ते ते॒ सद्म॒ विभृ॑तम् । \newline
27. सद्म॒ विभृ॑तं॒ ॅविभृ॑तꣳ॒॒ सद्म॒ सद्म॒ विभृ॑तम् पुरु॒त्रा पु॑रु॒त्रा विभृ॑तꣳ॒॒ सद्म॒ सद्म॒ विभृ॑तम् पुरु॒त्रा । \newline
28. विभृ॑तम् पुरु॒त्रा पु॑रु॒त्रा विभृ॑तं॒ ॅविभृ॑तम् पुरु॒त्रा । \newline
29. विभृ॑त॒मिति॒ वि - भृ॒त॒म् । \newline
30. पु॒रु॒त्रेति॑ पुरु - त्रा । \newline
31. वि॒द्मा ते॑ ते वि॒द्म वि॒द्मा ते॒ नाम॒ नाम॑ ते वि॒द्म वि॒द्मा ते॒ नाम॑ । \newline
32. ते॒ नाम॒ नाम॑ ते ते॒ नाम॑ पर॒मम् प॑र॒मम् नाम॑ ते ते॒ नाम॑ पर॒मम् । \newline
33. नाम॑ पर॒मम् प॑र॒मम् नाम॒ नाम॑ पर॒मम् गुहा॒ गुहा॑ पर॒मम् नाम॒ नाम॑ पर॒मम् गुहा᳚ । \newline
34. प॒र॒मम् गुहा॒ गुहा॑ पर॒मम् प॑र॒मम् गुहा॒ यद् यद् गुहा॑ पर॒मम् प॑र॒मम् गुहा॒ यत् । \newline
35. गुहा॒ यद् यद् गुहा॒ गुहा॒ यद् वि॒द्म वि॒द्म यद् गुहा॒ गुहा॒ यद् वि॒द्म । \newline
36. यद् वि॒द्म वि॒द्म यद् यद् वि॒द्मा तम् तं ॅवि॒द्म यद् यद् वि॒द्मा तम् । \newline
37. वि॒द्मा तम् तं ॅवि॒द्म वि॒द्मा त मुथ्स॒ मुथ्स॒म् तं ॅवि॒द्म वि॒द्मा त मुथ्स᳚म् । \newline
38. त मुथ्स॒ मुथ्स॒म् तम् त मुथ्सं॒ ॅयतो॒ यत॒ उथ्स॒म् तम् त मुथ्सं॒ ॅयतः॑ । \newline
39. उथ्सं॒ ॅयतो॒ यत॒ उथ्स॒ मुथ्सं॒ ॅयत॑ आज॒गन्था॑ ज॒गन्थ॒ यत॒ उथ्स॒ मुथ्सं॒ ॅयत॑ आज॒गन्थ॑ । \newline
40. यत॑ आज॒गन्था॑ ज॒गन्थ॒ यतो॒ यत॑ आज॒गन्थ॑ । \newline
41. आ॒ज॒गन्थेत्या᳚ - ज॒गन्थ॑ । \newline
42. स॒मु॒द्रे त्वा᳚ त्वा समु॒द्रे स॑मु॒द्रे त्वा॑ नृ॒मणा॑ नृ॒मणा᳚ स्त्वा समु॒द्रे स॑मु॒द्रे त्वा॑ नृ॒मणाः᳚ । \newline
43. त्वा॒ नृ॒मणा॑ नृ॒मणा᳚ स्त्वा त्वा नृ॒मणा॑ अ॒फ्स्व॑फ्सु नृ॒मणा᳚ स्त्वा त्वा नृ॒मणा॑ अ॒फ्सु । \newline
44. नृ॒मणा॑ अ॒फ्स्व॑फ्सु नृ॒मणा॑ नृ॒मणा॑ अ॒फ्स्व॑न्त र॒न्त र॒फ्सु नृ॒मणा॑ नृ॒मणा॑ अ॒फ्स्व॑न्तः । \newline
45. नृ॒मणा॒ इति॑ नृ - मनाः᳚ । \newline
46. अ॒फ्स्व॑न्त र॒न्त र॒फ्स्वा᳚(1॒)फ्स्व॑न्तर् नृ॒चक्षा॑ नृ॒चक्षा॑ अ॒न्त र॒फ्स्वा᳚(1॒)फ्स्व॑न्तर् नृ॒चक्षाः᳚ । \newline
47. अ॒फ्स्वित्य॑प् - सु । \newline
48. अ॒न्तर् नृ॒चक्षा॑ नृ॒चक्षा॑ अ॒न्त र॒न्तर् नृ॒चक्षा॑ ईध ईधे नृ॒चक्षा॑ अ॒न्त र॒न्तर् नृ॒चक्षा॑ ईधे । \newline
49. नृ॒चक्षा॑ ईध ईधे नृ॒चक्षा॑ नृ॒चक्षा॑ ईधे दि॒वो दि॒व ई॑धे नृ॒चक्षा॑ नृ॒चक्षा॑ ईधे दि॒वः । \newline
50. नृ॒चक्षा॒ इति॑ नृ - चक्षाः᳚ । \newline
51. ई॒धे॒ दि॒वो दि॒व ई॑ध ईधे दि॒वो अ॑ग्ने अग्ने दि॒व ई॑ध ईधे दि॒वो अ॑ग्ने । \newline
52. दि॒वो अ॑ग्ने अग्ने दि॒वो दि॒वो अ॑ग्न॒ ऊध॒न् नूध॑न् नग्ने दि॒वो दि॒वो अ॑ग्न॒ ऊधन्न्॑ । \newline
53. अ॒ग्न॒ ऊध॒न् नूध॑न् नग्ने अग्न॒ ऊधन्न्॑ । \newline
54. ऊध॒न्नित्यूधन्न्॑ । \newline
55. तृ॒तीये᳚ त्वा त्वा तृ॒तीये॑ तृ॒तीये᳚ त्वा॒ रज॑सि॒ रज॑सि त्वा तृ॒तीये॑ तृ॒तीये᳚ त्वा॒ रज॑सि । \newline
56. त्वा॒ रज॑सि॒ रज॑सि त्वा त्वा॒ रज॑सि तस्थि॒वाꣳस॑म् तस्थि॒वाꣳसꣳ॒॒ रज॑सि त्वा त्वा॒ रज॑सि तस्थि॒वाꣳस᳚म् । \newline
\pagebreak
\markright{ TS 4.2.2.2  \hfill https://www.vedavms.in \hfill}

\section{ TS 4.2.2.2 }

\textbf{TS 4.2.2.2 } \newline
\textbf{Samhita Paata} \newline

रज॑सि तस्थि॒वाꣳ स॑मृ॒तस्य॒ योनौ॑ महि॒षा अ॑हिन्वन्न् ॥ अक्र॑न्दद॒ग्निः स्त॒नय॑न्निव॒ द्यौः क्षामा॒ रेरि॑हद्-वी॒रुधः॑ सम॒ञ्जन्न् । स॒द्यो ज॑ज्ञा॒नो वि हीमि॒द्धो अख्य॒दा रोद॑सी भा॒नुना॑ भात्य॒न्तः ॥ उ॒शिक् पा॑व॒को अ॑र॒तिः सु॑मे॒धा मर्ते᳚ष्व॒ग्निर॒मृतो॒ निधा॑यि । इय॑र्ति धू॒मम॑रु॒षं भरि॑भ्र॒दुच्छु॒क्रेण॑ शो॒चिषा॒ द्यामिन॑क्षत् ॥ विश्व॑स्य के॒तुर्भुव॑नस्य॒ गर्भ॒ आ - [  ] \newline

\textbf{Pada Paata} \newline

रज॑सि । त॒स्थि॒वाꣳस᳚म् । ऋ॒तस्य॑ । योनौ᳚ । म॒हि॒षाः । अ॒हि॒न्व॒न्न् ॥ अक्र॑न्दत् । अ॒ग्निः । स्त॒नयन्न्॑ । इ॒व॒ । द्यौः । क्षाम॑ । रेरि॑हत् । वी॒रुधः॑ । स॒म॒ञ्जन्निति॑ सं - अ॒ञ्जन्न् ॥ स॒द्यः । ज॒ज्ञा॒नः । वीति॑ । हि । ई॒म् । इ॒द्धः । अख्य॑त् । एति॑ । रोद॑सी॒ इति॑ । भा॒नुना᳚ । भा॒ति॒ । अ॒न्तः ॥ उ॒शिक् । पा॒व॒कः । अ॒र॒तिः । सु॒मे॒धा इति॑ सु - मे॒धाः । मर्ते॑षु । अ॒ग्निः । अ॒मृतः॑ । नीति॑ । धा॒यि॒ ॥ इय॑र्ति । धू॒मम् । अ॒रु॒षम् । भरि॑भ्रत् । उदिति॑ । शु॒क्रेण॑ । शो॒चिषा᳚ । द्याम् । इन॑क्षत् ॥ विश्व॑स्य । के॒तुः । भुव॑नस्य । गर्भः॑ । एति॑ ।  \newline


\textbf{Krama Paata} \newline

रज॑सि तस्थि॒वाꣳस᳚म् । त॒स्थि॒वाꣳस॑मृ॒तस्य॑ । ऋ॒तस्य॒ योनौ᳚ । योनौ॑ महि॒षाः । म॒हि॒षा अ॑हिन्वन्न् । अ॒हि॒न्व॒न्नित्य॑हिन्वन्न् ॥ अक्र॑न्दद॒ग्निः । अ॒ग्निः स्त॒नयन्न्॑ । स्त॒नय॑न्निव । इ॒व॒ द्यौः । द्यौः क्षाम॑ । क्षामा॒ रेरि॑हत् । रेरि॑हद् वी॒रुधः॑ । वी॒रुधः॑ सम॒ञ्जन्न् । स॒म॒ञ्जन्निति॑ सम् - अ॒ञ्जन्न् ॥ स॒द्यो ज॑ज्ञा॒नः । ज॒ज्ञा॒नो वि । वि हि । हीम् । ई॒मि॒द्धः । इ॒द्धो अख्य॑त् । अख्य॒दा । आ रोद॑सी । रोद॑सी भा॒नुना᳚ । रोद॑सी॒ इति॒ रोद॑सी । भा॒नुना॑ भाति । भा॒त्य॒न्तः । अ॒न्तरित्य॒न्तः ॥ उ॒शिक् पा॑व॒कः । पा॒व॒को अ॑र॒तिः । अ॒र॒तिः सु॑मे॒धाः । सु॒मे॒धा मर्ते॑षु । सु॒मे॒धा इति॑ सु - मे॒धाः । मर्ते᳚ष्व॒ग्निः । अ॒ग्निर॒मृतः॑ । अ॒मृतो॒ नि । नि धा॑यि । धा॒यीति॑ धायि ॥ इय॑र्ति धू॒मम् । धू॒मम॑रु॒षम् । अ॒रु॒षं भरि॑भ्रत् । भरि॑भ्र॒दुत् । उच्छु॒क्रेण॑ । शु॒क्रेण॑ शो॒चिषा᳚ । शो॒चिषा॒ द्याम् । द्यामिन॑क्षत् । इन॑क्ष॒दितीन॑क्षत् ॥ विश्व॑स्य के॒तुः । के॒तुर् भुव॑नस्य । भुव॑नस्य॒ गर्भः॑ । गर्भ॒ आ । आ रोद॑सी \newline

\textbf{Jatai Paata} \newline

1. रज॑सि तस्थि॒वाꣳस॑म् तस्थि॒वाꣳसꣳ॒॒ रज॑सि॒ रज॑सि तस्थि॒वाꣳस᳚म् । \newline
2. त॒स्थि॒वाꣳस॑ मृ॒तस्य॒ र्तस्य॑ तस्थि॒वाꣳस॑म् तस्थि॒वाꣳस॑ मृ॒तस्य॑ । \newline
3. ऋ॒तस्य॒ योनौ॒ योना॑ वृ॒तस्य॒ र्तस्य॒ योनौ᳚ । \newline
4. योनौ॑ महि॒षा म॑हि॒षा योनौ॒ योनौ॑ महि॒षाः । \newline
5. म॒हि॒षा अ॑हिन्वन् नहिन्वन् महि॒षा म॑हि॒षा अ॑हिन्वन्न् । \newline
6. अ॒हि॒न्व॒न्नित्य॑हिन्वन्न् । \newline
7. अक्र॑न्द द॒ग्नि र॒ग्नि रक्र॑न्द॒ दक्र॑न्द द॒ग्निः । \newline
8. अ॒ग्निः स्त॒नयन्᳚ थ्स्त॒नय॑न् न॒ग्नि र॒ग्निः स्त॒नयन्न्॑ । \newline
9. स्त॒नय॑न् निवे व स्त॒नयन्᳚ थ्स्त॒नय॑न् निव । \newline
10. इ॒व॒ द्यौर् द्यौ रि॑वेव॒ द्यौः । \newline
11. द्यौः क्षाम॒ क्षाम॒ द्यौर् द्यौः क्षाम॑ । \newline
12. क्षामा॒ रेरि॑ह॒द् रेरि॑ह॒त् क्षाम॒ क्षामा॒ रेरि॑हत् । \newline
13. रेरि॑हद् वी॒रुधो॑ वी॒रुधो॒ रेरि॑ह॒द् रेरि॑हद् वी॒रुधः॑ । \newline
14. वी॒रुधः॑ सम॒ञ्जन् थ्स॑म॒ञ्जन्. वी॒रुधो॑ वी॒रुधः॑ सम॒ञ्जन्न् । \newline
15. स॒म॒ञ्जन्निति॑ सं - अ॒ञ्जन्न् । \newline
16. स॒द्यो ज॑ज्ञा॒नो ज॑ज्ञा॒नः स॒द्यः स॒द्यो ज॑ज्ञा॒नः । \newline
17. ज॒ज्ञा॒नो वि वि ज॑ज्ञा॒नो ज॑ज्ञा॒नो वि । \newline
18. वि हि हि वि वि हि । \newline
19. ही मीꣳ॒॒ हि हीम् । \newline
20. ई॒ मि॒द्ध इ॒द्ध ई॑मी मि॒द्धः । \newline
21. इ॒द्धो अख्य॒ दख्य॑ दि॒द्ध इ॒द्धो अख्य॑त् । \newline
22. अख्य॒दा ऽख्य॒ दख्य॒दा । \newline
23. आ रोद॑सी॒ रोद॑सी॒ आ रोद॑सी । \newline
24. रोद॑सी भा॒नुना॑ भा॒नुना॒ रोद॑सी॒ रोद॑सी भा॒नुना᳚ । \newline
25. रोद॑सी॒ इति॒ रोद॑सी । \newline
26. भा॒नुना॑ भाति भाति भा॒नुना॑ भा॒नुना॑ भाति । \newline
27. भा॒त्य॒न्त र॒न्तर् भा॑ति भात्य॒न्तः । \newline
28. अ॒न्तरित्य॒न्तः । \newline
29. उ॒शिक् पा॑व॒कः पा॑व॒क उ॒शि गु॒शिक् पा॑व॒कः । \newline
30. पा॒व॒को अ॑र॒ति र॑र॒तिः पा॑व॒कः पा॑व॒को अ॑र॒तिः । \newline
31. अ॒र॒तिः सु॑मे॒धाः सु॑मे॒धा अ॑र॒ति र॑र॒तिः सु॑मे॒धाः । \newline
32. सु॒मे॒धा मर्ते॑षु॒ मर्ते॑षु सुमे॒धाः सु॑मे॒धा मर्ते॑षु । \newline
33. सु॒मे॒धा इति॑ सु - मे॒धाः । \newline
34. मर्ते᳚ष्व॒ग्नि र॒ग्निर् मर्ते॑षु॒ मर्ते᳚ष्व॒ग्निः । \newline
35. अ॒ग्नि र॒मृतो॑ अ॒मृतो॑ अ॒ग्नि र॒ग्नि र॒मृतः॑ । \newline
36. अ॒मृतो॒ नि न्य॑मृतो॑ अ॒मृतो॒ नि । \newline
37. नि धा॑यि धायि॒ नि नि धा॑यि । \newline
38. धा॒यीति॑ धायि । \newline
39. इय॑र्ति धू॒मम् धू॒म मिय॒र्ती य॑र्ति धू॒मम् । \newline
40. धू॒म म॑रु॒ष म॑रु॒षम् धू॒मम् धू॒म म॑रु॒षम् । \newline
41. अ॒रु॒षम् भरि॑भ्र॒द् भरि॑भ्र दरु॒ष म॑रु॒षम् भरि॑भ्रत् । \newline
42. भरि॑भ्र॒ दुदुद् भरि॑भ्र॒द् भरि॑भ्र॒ दुत् । \newline
43. उच्छु॒क्रेण॑ शु॒क्रेणोदु च्छु॒क्रेण॑ । \newline
44. शु॒क्रेण॑ शो॒चिषा॑ शो॒चिषा॑ शु॒क्रेण॑ शु॒क्रेण॑ शो॒चिषा᳚ । \newline
45. शो॒चिषा॒ द्याम् द्याꣳ शो॒चिषा॑ शो॒चिषा॒ द्याम् । \newline
46. द्या मिन॑क्ष॒ दिन॑क्ष॒द् द्याम् द्या मिन॑क्षत् । \newline
47. इन॑क्ष॒दितीन॑क्षत् । \newline
48. विश्व॑स्य के॒तुः के॒तुर् विश्व॑स्य॒ विश्व॑स्य के॒तुः । \newline
49. के॒तुर् भुव॑नस्य॒ भुव॑नस्य के॒तुः के॒तुर् भुव॑नस्य । \newline
50. भुव॑नस्य॒ गर्भो॒ गर्भो॒ भुव॑नस्य॒ भुव॑नस्य॒ गर्भः॑ । \newline
51. गर्भ॒ आ गर्भो॒ गर्भ॒ आ । \newline
52. आ रोद॑सी॒ रोद॑सी॒ आ रोद॑सी । \newline

\textbf{Ghana Paata } \newline

1. रज॑सि तस्थि॒वाꣳस॑म् तस्थि॒वाꣳसꣳ॒॒ रज॑सि॒ रज॑सि तस्थि॒वाꣳस॑ मृ॒तस्य॒ र्‌तस्य॑ तस्थि॒वाꣳसꣳ॒॒ रज॑सि॒ रज॑सि तस्थि॒वाꣳस॑ मृ॒तस्य॑ । \newline
2. त॒स्थि॒वाꣳस॑ मृ॒तस्य॒ र्‌तस्य॑ तस्थि॒वाꣳस॑म् तस्थि॒वाꣳस॑ मृ॒तस्य॒ योनौ॒ योना॑ वृ॒तस्य॑ तस्थि॒वाꣳस॑म् तस्थि॒वाꣳस॑ मृ॒तस्य॒ योनौ᳚ । \newline
3. ऋ॒तस्य॒ योनौ॒ योना॑ वृ॒तस्य॒ र्‌तस्य॒ योनौ॑ महि॒षा म॑हि॒षा योना॑ वृ॒तस्य॒ र्‌तस्य॒ योनौ॑ महि॒षाः । \newline
4. योनौ॑ महि॒षा म॑हि॒षा योनौ॒ योनौ॑ महि॒षा अ॑हिन्वन् नहिन्वन् महि॒षा योनौ॒ योनौ॑ महि॒षा अ॑हिन्वन्न् । \newline
5. म॒हि॒षा अ॑हिन्वन् नहिन्वन् महि॒षा म॑हि॒षा अ॑हिन्वन्न् । \newline
6. अ॒हि॒न्व॒न्नित्य॑हिन्वन्न् । \newline
7. अक्र॑न्द द॒ग्नि र॒ग्नि रक्र॑न्द॒ दक्र॑न्द द॒ग्निः स्त॒नयन्᳚ थ्स्त॒नय॑न् न॒ग्नि रक्र॑न्द॒ दक्र॑न्द द॒ग्निः स्त॒नयन्न्॑ । \newline
8. अ॒ग्निः स्त॒नयन्᳚ थ्स्त॒नय॑न् न॒ग्नि र॒ग्निः स्त॒नय॑न् निवेव स्त॒नय॑न् न॒ग्नि र॒ग्निः स्त॒नय॑न् निव । \newline
9. स्त॒नय॑न् निवेव स्त॒नयन्᳚ थ्स्त॒नय॑न् निव॒ द्यौर् द्यौरि॑व स्त॒नयन्᳚ थ्स्त॒नय॑न् निव॒ द्यौः । \newline
10. इ॒व॒ द्यौर् द्यौ रि॑वेव॒ द्यौः क्षाम॒ क्षाम॒ द्यौ रि॑वेव॒ द्यौः क्षाम॑ । \newline
11. द्यौः क्षाम॒ क्षाम॒ द्यौर् द्यौः क्षामा॒ रेरि॑ह॒द् रेरि॑ह॒त् क्षाम॒ द्यौर् द्यौः क्षामा॒ रेरि॑हत् । \newline
12. क्षामा॒ रेरि॑ह॒द् रेरि॑ह॒त् क्षाम॒ क्षामा॒ रेरि॑हद् वी॒रुधो॑ वी॒रुधो॒ रेरि॑ह॒त् क्षाम॒ क्षामा॒ रेरि॑हद् वी॒रुधः॑ । \newline
13. रेरि॑हद् वी॒रुधो॑ वी॒रुधो॒ रेरि॑ह॒द् रेरि॑हद् वी॒रुधः॑ सम॒ञ्जन् थ्स॑म॒ञ्जन्. वी॒रुधो॒ रेरि॑ह॒द् रेरि॑हद् वी॒रुधः॑ सम॒ञ्जन्न् । \newline
14. वी॒रुधः॑ सम॒ञ्जन् थ्स॑म॒ञ्जन्. वी॒रुधो॑ वी॒रुधः॑ सम॒ञ्जन्न् । \newline
15. स॒म॒ञ्जन्निति॑ सं - अ॒ञ्जन्न् । \newline
16. स॒द्यो ज॑ज्ञा॒नो ज॑ज्ञा॒नः स॒द्यः स॒द्यो ज॑ज्ञा॒नो वि वि ज॑ज्ञा॒नः स॒द्यः स॒द्यो ज॑ज्ञा॒नो वि । \newline
17. ज॒ज्ञा॒नो वि वि ज॑ज्ञा॒नो ज॑ज्ञा॒नो वि हि हि वि ज॑ज्ञा॒नो ज॑ज्ञा॒नो वि हि । \newline
18. वि हि हि वि वि ही मीꣳ॒॒ हि वि वि हीम् । \newline
19. ही मीꣳ॒॒ हि ही मि॒द्ध इ॒द्ध ईꣳ॒॒ हि ही मि॒द्धः । \newline
20. ई॒ मि॒द्ध इ॒द्ध ई॑ मी मि॒द्धो अख्य॒ दख्य॑ दि॒द्ध ई॑ मी मि॒द्धो अख्य॑त् । \newline
21. इ॒द्धो अख्य॒ दख्य॑ दि॒द्ध इ॒द्धो अख्य॒दा ऽख्य॑दि॒द्ध इ॒द्धो अख्य॒दा । \newline
22. अख्य॒दा ऽख्य॒ दख्य॒दा रोद॑सी॒ रोद॑सी॒ आ ऽख्य॒ दख्य॒दा रोद॑सी । \newline
23. आ रोद॑सी॒ रोद॑सी॒ आ रोद॑सी भा॒नुना॑ भा॒नुना॒ रोद॑सी॒ आ रोद॑सी भा॒नुना᳚ । \newline
24. रोद॑सी भा॒नुना॑ भा॒नुना॒ रोद॑सी॒ रोद॑सी भा॒नुना॑ भाति भाति भा॒नुना॒ रोद॑सी॒ रोद॑सी भा॒नुना॑ भाति । \newline
25. रोद॑सी॒ इति॒ रोद॑सी । \newline
26. भा॒नुना॑ भाति भाति भा॒नुना॑ भा॒नुना॑ भात्य॒न्त र॒न्तर् भा॑ति भा॒नुना॑ भा॒नुना॑ भात्य॒न्तः । \newline
27. भा॒त्य॒न्त र॒न्तर् भा॑ति भात्य॒न्तः । \newline
28. अ॒न्तरित्य॒न्तः । \newline
29. उ॒शिक् पा॑व॒कः पा॑व॒क उ॒शि गु॒शिक् पा॑व॒को अ॑र॒ति र॑र॒तिः पा॑व॒क उ॒शि गु॒शिक् पा॑व॒को अ॑र॒तिः । \newline
30. पा॒व॒को अ॑र॒ति र॑र॒तिः पा॑व॒कः पा॑व॒को अ॑र॒तिः सु॑मे॒धाः सु॑मे॒धा अ॑र॒तिः पा॑व॒कः पा॑व॒को अ॑र॒तिः सु॑मे॒धाः । \newline
31. अ॒र॒तिः सु॑मे॒धाः सु॑मे॒धा अ॑र॒ति र॑र॒तिः सु॑मे॒धा मर्ते॑षु॒ मर्ते॑षु सुमे॒धा अ॑र॒ति र॑र॒तिः सु॑मे॒धा मर्ते॑षु । \newline
32. सु॒मे॒धा मर्ते॑षु॒ मर्ते॑षु सुमे॒धाः सु॑मे॒धा मर्ते᳚ ष्व॒ग्नि र॒ग्निर् मर्ते॑षु सुमे॒धाः सु॑मे॒धा मर्ते᳚ष्व॒ग्निः । \newline
33. सु॒मे॒धा इति॑ सु - मे॒धाः । \newline
34. मर्ते᳚ ष्व॒ग्नि र॒ग्निर् मर्ते॑षु॒ मर्ते᳚ ष्व॒ग्नि र॒मृतो॑ अ॒मृतो॑ अ॒ग्निर् मर्ते॑षु॒ मर्ते᳚ष्व॒ग्नि र॒मृतः॑ । \newline
35. अ॒ग्नि र॒मृतो॑ अ॒मृतो॑ अ॒ग्नि र॒ग्नि र॒मृतो॒ नि न्य॑मृतो॑ अ॒ग्नि र॒ग्नि र॒मृतो॒ नि । \newline
36. अ॒मृतो॒ नि न्य॑मृतो॑ अ॒मृतो॒ नि धा॑यि धायि॒ न्य॑मृतो॑ अ॒मृतो॒ नि धा॑यि । \newline
37. नि धा॑यि धायि॒ नि नि धा॑यि । \newline
38. धा॒यीति॑ धायि । \newline
39. इय॑र्ति धू॒मम् धू॒म मिय॒र्तीय॑र्ति धू॒म म॑रु॒ष म॑रु॒षम् धू॒म मिय॒र्तीय॑र्ति धू॒म म॑रु॒षम् । \newline
40. धू॒म म॑रु॒ष म॑रु॒षम् धू॒मम् धू॒म म॑रु॒षम् भरि॑भ्र॒द् भरि॑भ्र दरु॒षम् धू॒मम् धू॒म म॑रु॒षम् भरि॑भ्रत् । \newline
41. अ॒रु॒षम् भरि॑भ्र॒द् भरि॑भ्र दरु॒ष म॑रु॒षम् भरि॑भ्र॒ दुदुद् भरि॑भ्र दरु॒ष म॑रु॒षम् भरि॑भ्र॒ दुत् । \newline
42. भरि॑भ्र॒ दुदुद् भरि॑भ्र॒द् भरि॑भ्र॒ दुच्छु॒क्रेण॑ शु॒क्रेणोद् भरि॑भ्र॒द् भरि॑भ्र॒ दुच्छु॒क्रेण॑ । \newline
43. उच्छु॒क्रेण॑ शु॒क्रेणो दुच्छु॒क्रेण॑ शो॒चिषा॑ शो॒चिषा॑ शु॒क्रेणो दुच्छु॒क्रेण॑ शो॒चिषा᳚ । \newline
44. शु॒क्रेण॑ शो॒चिषा॑ शो॒चिषा॑ शु॒क्रेण॑ शु॒क्रेण॑ शो॒चिषा॒ द्याम् द्याꣳ शो॒चिषा॑ शु॒क्रेण॑ शु॒क्रेण॑ शो॒चिषा॒ द्याम् । \newline
45. शो॒चिषा॒ द्याम् द्याꣳ शो॒चिषा॑ शो॒चिषा॒ द्या मिन॑क्ष॒ दिन॑क्ष॒द् द्याꣳ शो॒चिषा॑ शो॒चिषा॒ द्या मिन॑क्षत् । \newline
46. द्या मिन॑क्ष॒ दिन॑क्ष॒द् द्याम् द्या मिन॑क्षत् । \newline
47. इन॑क्ष॒दितीन॑क्षत् । \newline
48. विश्व॑स्य के॒तुः के॒तुर् विश्व॑स्य॒ विश्व॑स्य के॒तुर् भुव॑नस्य॒ भुव॑नस्य के॒तुर् विश्व॑स्य॒ विश्व॑स्य के॒तुर् भुव॑नस्य । \newline
49. के॒तुर् भुव॑नस्य॒ भुव॑नस्य के॒तुः के॒तुर् भुव॑नस्य॒ गर्भो॒ गर्भो॒ भुव॑नस्य के॒तुः के॒तुर् भुव॑नस्य॒ गर्भः॑ । \newline
50. भुव॑नस्य॒ गर्भो॒ गर्भो॒ भुव॑नस्य॒ भुव॑नस्य॒ गर्भ॒ आ गर्भो॒ भुव॑नस्य॒ भुव॑नस्य॒ गर्भ॒ आ । \newline
51. गर्भ॒ आ गर्भो॒ गर्भ॒ आ रोद॑सी॒ रोद॑सी॒ आ गर्भो॒ गर्भ॒ आ रोद॑सी । \newline
52. आ रोद॑सी॒ रोद॑सी॒ आ रोद॑सी अपृणा दपृणा॒द् रोद॑सी॒ आ रोद॑सी अपृणात् । \newline
\pagebreak
\markright{ TS 4.2.2.3  \hfill https://www.vedavms.in \hfill}

\section{ TS 4.2.2.3 }

\textbf{TS 4.2.2.3 } \newline
\textbf{Samhita Paata} \newline

रोद॑सी अपृणा॒ज्जाय॑मानः । वी॒डुं चि॒दद्रि॑मभिनत् परा॒यन् जना॒ यद॒ग्निमय॑जन्त॒ पञ्च॑ ॥ श्री॒णामु॑दा॒रो ध॒रुणो॑ रयी॒णां म॑नी॒षाणां॒ प्रार्प॑णः॒ सोम॑गोपाः । वसोः᳚ सू॒नुः सह॑सो अ॒फ्सु राजा॒ वि भा॒त्यग्र॑ उ॒षसा॑मिधा॒नः ॥ यस्ते॑ अ॒द्य कृ॒णव॑द्-भद्रशोचेऽपू॒पं दे॑व घृ॒तव॑न्तमग्ने । प्रतं न॑य प्रत॒रां ॅवस्यो॒ अच्छा॒भि द्यु॒म्नं दे॒वभ॑क्तं ॅयविष्ठ ॥ आ - [  ] \newline

\textbf{Pada Paata} \newline

रोद॑सी॒ इति॑ । अ॒पृ॒णा॒त् । जाय॑मानः ॥ वी॒डुम् । चि॒त् । अद्रि᳚म् । अ॒भि॒न॒त् । प॒रा॒यन्निति॑ परा - यन्न् । जनाः᳚ । यत् । अ॒ग्निम् । अय॑जन्त । पञ्च॑ ॥ श्री॒णाम् । उ॒दा॒रः । ध॒रुणः॑ । र॒यी॒णाम् । म॒नी॒षाणा᳚म् । प्रार्प॑ण॒ इति॑ प्र - अर्प॑णः । सोम॑गोपा॒ इति॒ सोम॑ - गो॒पाः॒ ॥ वसोः᳚ । सू॒नुः । सह॑सः । अ॒फ्स्वित्य॑प्- सु । राजा᳚ । वीति॑ । भा॒ति॒ । अग्रे᳚ । उ॒षसा᳚म् । इ॒धा॒नः ॥ यः । ते॒ । अ॒द्य । कृ॒णव॑त् । भ॒द्र॒शो॒च॒ इति॑ भद्र - शो॒चे॒ । अ॒पू॒पम् । दे॒व॒ । घृ॒तव॑न्त॒मिति॑ घृ॒त - व॒न्त॒म् । अ॒ग्ने॒ ॥ प्रेति॑ । तम् । न॒य॒ । प्र॒त॒रामिति॑ प्र - त॒राम् । वस्यः॑ । अच्छ॑ । अ॒भीति॑ । द्यु॒म्नम् । दे॒वभ॑क्त॒मिति॑ दे॒व - भ॒क्त॒म् । य॒वि॒ष्ठ॒ ॥ एति॑ ।  \newline


\textbf{Krama Paata} \newline

रोद॑सी अपृणात् । रोद॑सी॒ इति॒ रोद॑सी । अ॒पृ॒णा॒ज् जाय॑मानः । जाय॑मान॒ इति॒ जाय॑मानः ॥ वी॒डुम् चि॑त् । चि॒दद्रि᳚म् । अद्रि॑मभिनत् । अ॒भि॒न॒त् प॒रा॒यन्न् । प॒रा॒यन् जनाः᳚ । प॒रा॒यन्निति॑ परा - यन्न् । जना॒ यत् । यद॒ग्निम् । अ॒ग्निमय॑जन्त । अय॑जन्त॒ पञ्च॑ । पञ्चेति॒ पञ्च॑ ॥ श्री॒णामु॑दा॒रः । उ॒दा॒रो ध॒रुणः॑ । ध॒रुणो॑ रयी॒णाम् । र॒यी॒णाम् म॑नी॒षाणा᳚म् । म॒नी॒षाणा॒म् प्रार्प॑णः । प्रार्प॑णः॒ सोम॑गोपाः । प्रार्प॑ण॒ इति॑ प्र - अर्प॑णः । सोम॑गोपा॒ इति॒ सोम॑ - गो॒पाः॒ ॥ वसोः᳚ सू॒नुः । सू॒नुः सह॑सः । सह॑सो अ॒फ्सु । अ॒फ्सु राजा᳚ । अ॒फ्स्वित्य॑प् - सु । राजा॒ वि । वि भा॑ति । भा॒त्यग्रे᳚ । अग्र॑ उ॒षसा᳚म् । उ॒षसा॑मिधा॒नः । इ॒धा॒न इती॑धा॒नः ॥ यस्ते᳚ । ते॒ अ॒द्य । अ॒द्य कृ॒णव॑त् । कृ॒णव॑द् भद्रशोचे । भ॒द्र॒शो॒चे॒ऽपू॒पम् । भ॒द्र॒शो॒च॒ इति॑ भद्र - शो॒चे॒ । अ॒पू॒पम् दे॑व । दे॒व॒ घृ॒तव॑न्तम् । घृ॒तव॑न्तमग्ने । घृ॒तव॑न्त॒मिति॑ घृ॒त - व॒न्त॒म् । अ॒ग्न॒ इत्य॑ग्ने ॥ प्र तम् । तम् न॑य । न॒य॒ प्र॒त॒राम् । प्र॒त॒रां ॅवस्यः॑ । प्र॒त॒रामिति॑ प्र - त॒राम् । वस्यो॒ अच्छ॑ । अच्छा॒भि । अ॒भि द्यु॒म्नम् । द्यु॒म्नम् दे॒वभ॑क्तम् । दे॒वभ॑क्तं ॅयविष्ठ । दे॒वभ॑क्त॒मिति॑ दे॒व - भ॒क्त॒म् । य॒वि॒ष्ठेति॑ यविष्ठ ॥ आ तम् \newline

\textbf{Jatai Paata} \newline

1. रोद॑सी अपृणा दपृणा॒द् रोद॑सी॒ रोद॑सी अपृणात् । \newline
2. रोद॑सी॒ इति॒ रोद॑सी । \newline
3. अ॒पृ॒णा॒ज् जाय॑मानो॒ जाय॑मानो ऽपृणा दपृणा॒ज् जाय॑मानः । \newline
4. जाय॑मान॒ इति॒ जाय॑मानः । \newline
5. वी॒डुम् चि॑च् चिद् वी॒डुं ॅवी॒डुम् चि॑त् । \newline
6. चि॒दद्रि॒ मद्रि॑म् चिच् चि॒दद्रि᳚म् । \newline
7. अद्रि॑ मभिन दभिन॒ दद्रि॒ मद्रि॑ मभिनत् । \newline
8. अ॒भि॒न॒त् प॒रा॒यन् प॑रा॒यन् न॑भिन दभिनत् परा॒यन्न् । \newline
9. प॒रा॒यन् जना॒ जनाः᳚ परा॒यन् प॑रा॒यन् जनाः᳚ । \newline
10. प॒रा॒यन्निति॑ परा - यन्न् । \newline
11. जना॒ यद् यज् जना॒ जना॒ यत् । \newline
12. यद॒ग्नि म॒ग्निं ॅयद् यद॒ग्निम् । \newline
13. अ॒ग्नि मय॑ज॒न्ता य॑जन्ता॒ग्नि म॒ग्नि मय॑जन्त । \newline
14. अय॑जन्त॒ पञ्च॒ पञ्चा य॑ज॒न्ता य॑जन्त॒ पञ्च॑ । \newline
15. पञ्चेति॒ पञ्च॑ । \newline
16. श्री॒णा मु॑दा॒र उ॑दा॒रः श्री॒णाꣳ श्री॒णा मु॑दा॒रः । \newline
17. उ॒दा॒रो ध॒रुणो॑ ध॒रुण॑ उदा॒र उ॑दा॒रो ध॒रुणः॑ । \newline
18. ध॒रुणो॑ रयी॒णाꣳ र॑यी॒णाम् ध॒रुणो॑ ध॒रुणो॑ रयी॒णाम् । \newline
19. र॒यी॒णाम् म॑नी॒षाणा᳚म् मनी॒षाणाꣳ॑ रयी॒णाꣳ र॑यी॒णाम् म॑नी॒षाणा᳚म् । \newline
20. म॒नी॒षाणा॒म् प्रार्प॑णः॒ प्रार्प॑णो मनी॒षाणा᳚म् मनी॒षाणा॒म् प्रार्प॑णः । \newline
21. प्रार्प॑णः॒ सोम॑गोपाः॒ सोम॑गोपाः॒ प्रार्प॑णः॒ प्रार्प॑णः॒ सोम॑गोपाः । \newline
22. प्रार्प॑ण॒ इति॑ प्र - अर्प॑णः । \newline
23. सोम॑गोपा॒ इति॒ सोम॑ - गो॒पाः॒ । \newline
24. वसोः᳚ सू॒नुः सू॒नुर् वसो॒र् वसोः᳚ सू॒नुः । \newline
25. सू॒नुः सह॑सः॒ सह॑सः सू॒नुः सू॒नुः सह॑सः । \newline
26. सह॑सो अ॒फ्स्व॑फ्सु सह॑सः॒ सह॑सो अ॒फ्सु । \newline
27. अ॒फ्सु राजा॒ राजा॒ ऽफ्स्व॑फ्सु राजा᳚ । \newline
28. अ॒फ्स्वित्य॑प् - सु । \newline
29. राजा॒ वि वि राजा॒ राजा॒ वि । \newline
30. वि भा॑ति भाति॒ वि वि भा॑ति । \newline
31. भा॒त्यग्रे॒ अग्रे॑ भाति भा॒त्यग्रे᳚ । \newline
32. अग्र॑ उ॒षसा॑ मु॒षसा॒ मग्रे॒ अग्र॑ उ॒षसा᳚म् । \newline
33. उ॒षसा॑ मिधा॒न इ॑धा॒न उ॒षसा॑ मु॒षसा॑ मिधा॒नः । \newline
34. इ॒धा॒न इती॑धा॒नः । \newline
35. य स्ते॑ ते॒ यो य स्ते᳚ । \newline
36. ते॒ अ॒द्याद्य ते॑ ते अ॒द्य । \newline
37. अ॒द्य कृ॒णव॑त् कृ॒णव॑ द॒द्याद्य कृ॒णव॑त् । \newline
38. कृ॒णव॑द् भद्रशोचे भद्रशोचे कृ॒णव॑त् कृ॒णव॑द् भद्रशोचे । \newline
39. भ॒द्र॒शो॒चे॒ ऽपू॒प म॑पू॒पम् भ॑द्रशोचे भद्रशोचे ऽपू॒पम् । \newline
40. भ॒द्र॒शो॒च॒ इति॑ भद्र - शो॒चे॒ । \newline
41. अ॒पू॒पम् दे॑व देवापू॒प म॑पू॒पम् दे॑व । \newline
42. दे॒व॒ घृ॒तव॑न्तम् घृ॒तव॑न्तम् देव देव घृ॒तव॑न्तम् । \newline
43. घृ॒तव॑न्त मग्ने अग्ने घृ॒तव॑न्तम् घृ॒तव॑न्त मग्ने । \newline
44. घृ॒तव॑न्त॒मिति॑ घृ॒त - व॒न्त॒म् । \newline
45. अ॒ग्न॒ इत्य॑ग्ने । \newline
46. प्र तम् तम् प्र प्र तम् । \newline
47. तम् न॑य नय॒ तम् तम् न॑य । \newline
48. न॒य॒ प्र॒त॒राम् प्र॑त॒राम् न॑य नय प्रत॒राम् । \newline
49. प्र॒त॒रां ॅवस्यो॒ वस्यः॑ प्रत॒राम् प्र॑त॒रां ॅवस्यः॑ । \newline
50. प्र॒त॒रामिति॑ प्र - त॒राम् । \newline
51. वस्यो॒ अच्छाच्छ॒ वस्यो॒ वस्यो॒ अच्छ॑ । \newline
52. अच्छा॒ भ्य॑भ्यच्छा च्छा॒भि । \newline
53. अ॒भि द्यु॒म्नम् द्यु॒म्न म॒भ्य॑भि द्यु॒म्नम् । \newline
54. द्यु॒म्नम् दे॒वभ॑क्तम् दे॒वभ॑क्तम् द्यु॒म्नम् द्यु॒म्नम् दे॒वभ॑क्तम् । \newline
55. दे॒वभ॑क्तं ॅयविष्ठ यविष्ठ दे॒वभ॑क्तम् दे॒वभ॑क्तं ॅयविष्ठ । \newline
56. दे॒वभ॑क्त॒मिति॑ दे॒व - भ॒क्त॒म् । \newline
57. य॒वि॒ष्ठेति॑ यविष्ठ । \newline
58. आ तम् त मा तम् । \newline

\textbf{Ghana Paata } \newline

1. रोद॑सी अपृणा दपृणा॒द् रोद॑सी॒ रोद॑सी अपृणा॒ज् जाय॑मानो॒ जाय॑मानो ऽपृणा॒द् रोद॑सी॒ रोद॑सी अपृणा॒ज् जाय॑मानः । \newline
2. रोद॑सी॒ इति॒ रोद॑सी । \newline
3. अ॒पृ॒णा॒ज् जाय॑मानो॒ जाय॑मानो ऽपृणा दपृणा॒ज् जाय॑मानः । \newline
4. जाय॑मान॒ इति॒ जाय॑मानः । \newline
5. वी॒डुम् चि॑च् चिद् वी॒डुं ॅवी॒डुम् चि॒दद्रि॒ मद्रि॑म् चिद् वी॒डुं ॅवी॒डुम् चि॒दद्रि᳚म् । \newline
6. चि॒दद्रि॒ मद्रि॑म् चिच् चि॒दद्रि॑ मभिन दभिन॒ दद्रि॑म् चिच् चि॒दद्रि॑ मभिनत् । \newline
7. अद्रि॑ मभिन दभिन॒ दद्रि॒ मद्रि॑ मभिनत् परा॒यन् प॑रा॒यन् न॑भिन॒ दद्रि॒ मद्रि॑ मभिनत् परा॒यन्न् । \newline
8. अ॒भि॒न॒त् प॒रा॒यन् प॑रा॒यन् न॑भिन दभिनत् परा॒यन् जना॒ जनाः᳚ परा॒यन् न॑भिन दभिनत् परा॒यन् जनाः᳚ । \newline
9. प॒रा॒यन् जना॒ जनाः᳚ परा॒यन् प॑रा॒यन् जना॒ यद् यज् जनाः᳚ परा॒यन् प॑रा॒यन् जना॒ यत् । \newline
10. प॒रा॒यन्निति॑ परा - यन्न् । \newline
11. जना॒ यद् यज् जना॒ जना॒ यद॒ग्नि म॒ग्निं ॅयज् जना॒ जना॒ यद॒ग्निम् । \newline
12. यद॒ग्नि म॒ग्निं ॅयद् यद॒ग्नि मय॑ज॒न्ता य॑जन्ता॒ग्निं ॅयद् यद॒ग्नि मय॑जन्त । \newline
13. अ॒ग्नि मय॑ज॒न्ता य॑जन्ता॒ग्नि म॒ग्नि मय॑जन्त॒ पञ्च॒ पञ्चा य॑जन्ता॒ग्नि म॒ग्नि मय॑जन्त॒ पञ्च॑ । \newline
14. अय॑जन्त॒ पञ्च॒ पञ्चा य॑ज॒न्ता य॑जन्त॒ पञ्च॑ । \newline
15. पञ्चेति॒ पञ्च॑ । \newline
16. श्री॒णा मु॑दा॒र उ॑दा॒रः श्री॒णाꣳ श्री॒णा मु॑दा॒रो ध॒रुणो॑ ध॒रुण॑ उदा॒रः श्री॒णाꣳ श्री॒णा मु॑दा॒रो ध॒रुणः॑ । \newline
17. उ॒दा॒रो ध॒रुणो॑ ध॒रुण॑ उदा॒र उ॑दा॒रो ध॒रुणो॑ रयी॒णाꣳ र॑यी॒णाम् ध॒रुण॑ उदा॒र उ॑दा॒रो ध॒रुणो॑ रयी॒णाम् । \newline
18. ध॒रुणो॑ रयी॒णाꣳ र॑यी॒णाम् ध॒रुणो॑ ध॒रुणो॑ रयी॒णाम् म॑नी॒षाणा᳚म् मनी॒षाणाꣳ॑ रयी॒णाम् ध॒रुणो॑ ध॒रुणो॑ रयी॒णाम् म॑नी॒षाणा᳚म् । \newline
19. र॒यी॒णाम् म॑नी॒षाणा᳚म् मनी॒षाणाꣳ॑ रयी॒णाꣳ र॑यी॒णाम् म॑नी॒षाणा॒म् प्रार्प॑णः॒ प्रार्प॑णो मनी॒षाणाꣳ॑ रयी॒णाꣳ र॑यी॒णाम् म॑नी॒षाणा॒म् प्रार्प॑णः । \newline
20. म॒नी॒षाणा॒म् प्रार्प॑णः॒ प्रार्प॑णो मनी॒षाणा᳚म् मनी॒षाणा॒म् प्रार्प॑णः॒ सोम॑गोपाः॒ सोम॑गोपाः॒ प्रार्प॑णो मनी॒षाणा᳚म् मनी॒षाणा॒म् प्रार्प॑णः॒ सोम॑गोपाः । \newline
21. प्रार्प॑णः॒ सोम॑गोपाः॒ सोम॑गोपाः॒ प्रार्प॑णः॒ प्रार्प॑णः॒ सोम॑गोपाः । \newline
22. प्रार्प॑ण॒ इति॑ प्र - अर्प॑णः । \newline
23. सोम॑गोपा॒ इति॒ सोम॑ - गो॒पाः॒ । \newline
24. वसोः᳚ सू॒नुः सू॒नुर् वसो॒र् वसोः᳚ सू॒नुः सह॑सः॒ सह॑सः सू॒नुर् वसो॒र् वसोः᳚ सू॒नुः सह॑सः । \newline
25. सू॒नुः सह॑सः॒ सह॑सः सू॒नुः सू॒नुः सह॑सो अ॒फ्स्व॑फ्सु सह॑सः सू॒नुः सू॒नुः सह॑सो अ॒फ्सु । \newline
26. सह॑सो अ॒फ्स्व॑फ्सु सह॑सः॒ सह॑सो अ॒फ्सु राजा॒ राजा॒ ऽफ्सु सह॑सः॒ सह॑सो अ॒फ्सु राजा᳚ । \newline
27. अ॒फ्सु राजा॒ राजा॒ ऽफ्स्व॑फ्सु राजा॒ वि वि राजा॒ ऽफ्स्व॑फ्सु राजा॒ वि । \newline
28. अ॒फ्स्वित्य॑प् - सु । \newline
29. राजा॒ वि वि राजा॒ राजा॒ वि भा॑ति भाति॒ वि राजा॒ राजा॒ वि भा॑ति । \newline
30. वि भा॑ति भाति॒ वि वि भा॒त्यग्रे॒ अग्रे॑ भाति॒ वि वि भा॒त्यग्रे᳚ । \newline
31. भा॒त्यग्रे॒ अग्रे॑ भाति भा॒त्यग्र॑ उ॒षसा॑ मु॒षसा॒ मग्रे॑ भाति भा॒त्यग्र॑ उ॒षसा᳚म् । \newline
32. अग्र॑ उ॒षसा॑ मु॒षसा॒ मग्रे॒ अग्र॑ उ॒षसा॑ मिधा॒न इ॑धा॒न उ॒षसा॒ मग्रे॒ अग्र॑ उ॒षसा॑ मिधा॒नः । \newline
33. उ॒षसा॑ मिधा॒न इ॑धा॒न उ॒षसा॑ मु॒षसा॑ मिधा॒नः । \newline
34. इ॒धा॒न इती॑धा॒नः । \newline
35. यस्ते॑ ते॒ यो यस्ते॑ अ॒द्याद्य ते॒ यो यस्ते॑ अ॒द्य । \newline
36. ते॒ अ॒द्याद्य ते॑ ते अ॒द्य कृ॒णव॑त् कृ॒णव॑ द॒द्य ते॑ ते अ॒द्य कृ॒णव॑त् । \newline
37. अ॒द्य कृ॒णव॑त् कृ॒णव॑ द॒द्याद्य कृ॒णव॑द् भद्रशोचे भद्रशोचे कृ॒णव॑ द॒द्याद्य कृ॒णव॑द् भद्रशोचे । \newline
38. कृ॒णव॑द् भद्रशोचे भद्रशोचे कृ॒णव॑त् कृ॒णव॑द् भद्रशोचे ऽपू॒प म॑पू॒पम् भ॑द्रशोचे कृ॒णव॑त् कृ॒णव॑द् भद्रशोचे ऽपू॒पम् । \newline
39. भ॒द्र॒शो॒चे॒ ऽपू॒प म॑पू॒पम् भ॑द्रशोचे भद्रशोचे ऽपू॒पम् दे॑व देवापू॒पम् भ॑द्रशोचे भद्रशोचे ऽपू॒पम् दे॑व । \newline
40. भ॒द्र॒शो॒च॒ इति॑ भद्र - शो॒चे॒ । \newline
41. अ॒पू॒पम् दे॑व देवापू॒प म॑पू॒पम् दे॑व घृ॒तव॑न्तम् घृ॒तव॑न्तम् देवापू॒प म॑पू॒पम् दे॑व घृ॒तव॑न्तम् । \newline
42. दे॒व॒ घृ॒तव॑न्तम् घृ॒तव॑न्तम् देव देव घृ॒तव॑न्त मग्ने अग्ने घृ॒तव॑न्तम् देव देव घृ॒तव॑न्त मग्ने । \newline
43. घृ॒तव॑न्त मग्ने अग्ने घृ॒तव॑न्तम् घृ॒तव॑न्त मग्ने । \newline
44. घृ॒तव॑न्त॒मिति॑ घृ॒त - व॒न्त॒म् । \newline
45. अ॒ग्न॒ इत्य॑ग्ने । \newline
46. प्र तम् तम् प्र प्र तम् न॑य नय॒ तम् प्र प्र तम् न॑य । \newline
47. तम् न॑य नय॒ तम् तम् न॑य प्रत॒राम् प्र॑त॒राम् न॑य॒ तम् तम् न॑य प्रत॒राम् । \newline
48. न॒य॒ प्र॒त॒राम् प्र॑त॒राम् न॑य नय प्रत॒रां ॅवस्यो॒ वस्यः॑ प्रत॒राम् न॑य नय प्रत॒रां ॅवस्यः॑ । \newline
49. प्र॒त॒रां ॅवस्यो॒ वस्यः॑ प्रत॒राम् प्र॑त॒रां ॅवस्यो॒ अच्छाच्छ॒ वस्यः॑ प्रत॒राम् प्र॑त॒रां ॅवस्यो॒ अच्छ॑ । \newline
50. प्र॒त॒रामिति॑ प्र - त॒राम् । \newline
51. वस्यो॒ अच्छाच्छ॒ वस्यो॒ वस्यो॒ अच्छा॒ भ्य॑भ्यच्छ॒ वस्यो॒ वस्यो॒ अच्छा॒भि । \newline
52. अच्छा॒ भ्य॑भ्यच्छा च्छा॒भि द्यु॒म्नम् द्यु॒म्न म॒भ्यच्छा च्छा॒भि द्यु॒म्नम् । \newline
53. अ॒भि द्यु॒म्नम् द्यु॒म्न म॒भ्य॑भि द्यु॒म्नम् दे॒वभ॑क्तम् दे॒वभ॑क्तम् द्यु॒म्न म॒भ्य॑भि द्यु॒म्नम् दे॒वभ॑क्तम् । \newline
54. द्यु॒म्नम् दे॒वभ॑क्तम् दे॒वभ॑क्तम् द्यु॒म्नम् द्यु॒म्नम् दे॒वभ॑क्तं ॅयविष्ठ यविष्ठ दे॒वभ॑क्तम् द्यु॒म्नम् द्यु॒म्नम् दे॒वभ॑क्तं ॅयविष्ठ । \newline
55. दे॒वभ॑क्तं ॅयविष्ठ यविष्ठ दे॒वभ॑क्तम् दे॒वभ॑क्तं ॅयविष्ठ । \newline
56. दे॒वभ॑क्त॒मिति॑ दे॒व - भ॒क्त॒म् । \newline
57. य॒वि॒ष्ठेति॑ यविष्ठ । \newline
58. आ तम् त मा तम् भ॑ज भज॒ त मा तम् भ॑ज । \newline
\pagebreak
\markright{ TS 4.2.2.4  \hfill https://www.vedavms.in \hfill}

\section{ TS 4.2.2.4 }

\textbf{TS 4.2.2.4 } \newline
\textbf{Samhita Paata} \newline

तं भ॑ज सौश्रव॒सेष्व॑ग्न उ॒क्थ-उ॑क्थ॒ आ भ॑ज श॒स्यमा॑ने । प्रि॒यः सूर्ये᳚ प्रि॒यो अ॒ग्ना भ॑वा॒त्युज्जा॒तेन॑ भि॒नद॒दुज्जनि॑त्वैः ॥ त्वाम॑ग्ने॒ यज॑माना॒ अनु॒ द्यून्. विश्वा॒ वसू॑नि दधिरे॒ वार्या॑णि । त्वया॑ स॒ह द्रवि॑णमि॒च्छमा॑ना व्र॒जं गोम॑न्तमु॒शिजो॒ वि व॑व्रुः ॥ दृ॒शा॒नो रु॒क्म उ॒र्व्या व्य॑द्यौद्-दु॒र्मर्.ष॒मायुः॑ श्रि॒ये रु॑चा॒नः । अ॒ग्निर॒मृतो॑ अभव॒द्-वयो॑भि॒र्यदे॑- ( ) -नं॒ द्यौरज॑नयथ् सु॒रेताः᳚ ॥ \newline

\textbf{Pada Paata} \newline

तम् । भ॒ज॒ । सौ॒श्र॒व॒सेषु॑ । अ॒ग्ने॒ । उ॒क्थ उ॑क्थ॒ इत्यु॒क्थे - उ॒क्थे॒ । एति॑ । भ॒ज॒ । श॒स्यमा॑ने ॥ प्रि॒यः । सूर्ये᳚ । प्रि॒यः । अ॒ग्ना । भ॒वा॒ति॒ । उदिति॑ । जा॒तेन॑ । भि॒नद॑त् । उदिति॑ । जनि॑त्वैः ॥ त्वाम् । अ॒ग्ने॒ । यज॑मानाः । अन्विति॑ । द्यून् । विश्वा᳚ । वसू॑नि । द॒धि॒रे॒ । वार्या॑णि ॥ त्वया᳚ । स॒ह । द्रवि॑णम् । इ॒च्छमा॑नाः । व्र॒जम् । गोम॑न्त॒मिति॒ गो - म॒न्त॒म् । उ॒शिजः॑ । वीति॑ । व॒व्रुः॒ ॥ दृ॒शा॒नः । रु॒क्मः । उ॒र्व्या । वीति॑ । अ॒द्यौ॒त् । दु॒र्मर्.ष॒मिति॑ दुः - मर्.ष᳚म् । आयुः॑ । श्रि॒ये । रु॒चा॒नः ॥ अ॒ग्निः । अ॒मृतः॑ । अ॒भ॒व॒त् । वयो॑भि॒रिति॒ वयः॑ - भिः॒ । यत् ( ) । ए॒न॒म् । द्यौः । अज॑नयत् । सु॒रेता॒ इति॑ सु - रेताः᳚ ॥  \newline


\textbf{Krama Paata} \newline

तम् भ॑ज । भ॒ज॒ सौ॒श्र॒व॒सेषु॑ । सौ॒श्र॒व॒सेष्व॑ग्ने । अ॒ग्न॒ उ॒क्थौ॑क्थे । उ॒क्थ,उ॑क्थ॒ आ । उ॒क्थ,उ॑क्थ॒ इत्यु॒क्थे - उ॒क्थे॒ । आ भ॑ज । भ॒ज॒ श॒स्यमा॑ने । श॒स्यमा॑न॒ इति॑ श॒स्यमा॑ने ॥ प्रि॒यः सूर्ये᳚ । सूर्ये᳚ प्रि॒यः । प्रि॒यो अ॒ग्ना । अ॒ग्ना भ॑वाति । भ॒वा॒त्युत् । उज् जा॒तेन॑ । जा॒तेन॑ भि॒नद॑त् । भि॒नद॒दुत् । उज् जनि॑त्वैः । जनि॑त्वै॒रिति॒ जनि॑त्वैः ॥ त्वाम॑ग्ने । अ॒ग्ने॒ यज॑मानाः । यज॑माना॒ अनु॑ । अनु॒ द्यून् । द्यून्. विश्वा᳚ । विश्वा॒ वसू॑नि । वसू॑नि दधिरे । द॒धि॒रे॒ वार्या॑णि । वार्या॒णीति॒ वार्या॑णि ॥ त्वया॑ स॒ह । स॒ह द्रवि॑णम् । द्रवि॑णमि॒च्छमा॑नाः । इ॒च्छमा॑ना व्र॒जम् । व्र॒जम् गोम॑न्तम् । गोम॑न्तमु॒शिजः॑ । गोम॑न्त॒मिति॒ गो - म॒न्त॒म् । उ॒शिजो॒ वि । वि व॑व्रुः । व॒व्रु॒रिति॑ वव्रुः ॥ दृ॒शा॒नो रु॒क्मः । रु॒क्म उ॒र्व्या । उ॒र्व्या वि । व्य॑द्यौत् । अ॒द्यौ॒द् दु॒र्मर्.ष᳚म् । दु॒र्मर्.ष॒मायुः॑ । दु॒र्मर्.ष॒मिति॑ दुः - मर्.ष᳚म् । आयुः॑ श्रि॒ये । श्रि॒ये रु॑चा॒नः । रु॒चा॒न इति॑ रुचा॒नः ॥ अ॒ग्निर॒मृतः॑ । अ॒मृतो॑ अभवत् । अ॒भ॒व॒द् वयो॑भिः । वयो॑भि॒र् यत् । वयो॑भि॒रिति॒वयः॑ - भिः॒ । यदे॑नम् । ए॒न॒म् द्यौः । द्यौरज॑नयत् ( ) । अज॑नयथ् सु॒रेताः᳚ । 
सु॒रेता॒ इति॑ सु - रेताः᳚ । \newline

\textbf{Jatai Paata} \newline

1. तम् भ॑ज भज॒ तम् तम् भ॑ज । \newline
2. भ॒ज॒ सौ॒श्र॒व॒सेषु॑ सौश्रव॒सेषु॑ भज भज सौश्रव॒सेषु॑ । \newline
3. सौ॒श्र॒व॒से ष्व॑ग्ने अग्ने सौश्रव॒सेषु॑ सौश्रव॒से ष्व॑ग्ने । \newline
4. अ॒ग्न॒ उ॒क्थ‍उ॑क्थ उ॒क्थ‍उ॑क्थे अग्ने अग्न उ॒क्थ‍उ॑क्थे । \newline
5. उ॒क्थ‍उ॑क्थ॒ ओक्थ‍उ॑क्थ उ॒क्थ‍उ॑क्थ॒ आ । \newline
6. उ॒क्थ‍उ॑क्थ॒ इत्यु॒क्थे - उ॒क्थे॒ । \newline
7. आ भ॑ज भ॒जा भ॑ज । \newline
8. भ॒ज॒ श॒स्यमा॑ने श॒स्यमा॑ने भज भज श॒स्यमा॑ने । \newline
9. श॒स्यमा॑न॒ इति॑ श॒स्यमा॑ने । \newline
10. प्रि॒यः सूर्ये॒ सूर्ये᳚ प्रि॒यः प्रि॒यः सूर्ये᳚ । \newline
11. सूर्ये᳚ प्रि॒यः प्रि॒यः सूर्ये॒ सूर्ये᳚ प्रि॒यः । \newline
12. प्रि॒यो अ॒ग्ना ऽग्ना प्रि॒यः प्रि॒यो अ॒ग्ना । \newline
13. अ॒ग्ना भ॑वाति भवा त्य॒ग्ना ऽग्ना भ॑वाति । \newline
14. भ॒वा॒ त्युदुद् भ॑वाति भवा॒ त्युत् । \newline
15. उज् जा॒तेन॑ जा॒ते नोदुज् जा॒तेन॑ । \newline
16. जा॒तेन॑ भि॒नद॑द् भि॒नद॑ज् जा॒तेन॑ जा॒तेन॑ भि॒नद॑त् । \newline
17. भि॒नद॒ दुदुद् भि॒न द॑द् भि॒नद॒ दुत् । \newline
18. उज् जनि॑त्वै॒र् जनि॑त्वै॒ रुदुज् जनि॑त्वैः । \newline
19. जनि॑त्वै॒रिति॒ जनि॑त्वैः । \newline
20. त्वा म॑ग्ने अग्ने॒ त्वाम् त्वा म॑ग्ने । \newline
21. अ॒ग्ने॒ यज॑माना॒ यज॑माना अग्ने अग्ने॒ यज॑मानाः । \newline
22. यज॑माना॒ अन्वनु॒ यज॑माना॒ यज॑माना॒ अनु॑ । \newline
23. अनु॒ द्यून् द्यू नन् वनु॒ द्यून् । \newline
24. द्यून्. विश्वा॒ विश्वा॒ द्यून् द्यून्. विश्वा᳚ । \newline
25. विश्वा॒ वसू॑नि॒ वसू॑नि॒ विश्वा॒ विश्वा॒ वसू॑नि । \newline
26. वसू॑नि दधिरे दधिरे॒ वसू॑नि॒ वसू॑नि दधिरे । \newline
27. द॒धि॒रे॒ वार्या॑णि॒ वार्या॑णि दधिरे दधिरे॒ वार्या॑णि । \newline
28. वार्या॒णीति॒ वार्या॑णि । \newline
29. त्वया॑ स॒ह स॒ह त्वया॒ त्वया॑ स॒ह । \newline
30. स॒ह द्रवि॑ण॒म् द्रवि॑णꣳ स॒ह स॒ह द्रवि॑णम् । \newline
31. द्रवि॑ण मि॒च्छमा॑ना इ॒च्छमा॑ना॒ द्रवि॑ण॒म् द्रवि॑ण मि॒च्छमा॑नाः । \newline
32. इ॒च्छमा॑ना व्र॒जं ॅव्र॒ज मि॒च्छमा॑ना इ॒च्छमा॑ना व्र॒जम् । \newline
33. व्र॒जम् गोम॑न्त॒म् गोम॑न्तं ॅव्र॒जं ॅव्र॒जम् गोम॑न्तम् । \newline
34. गोम॑न्त मु॒शिज॑ उ॒शिजो॒ गोम॑न्त॒म् गोम॑न्त मु॒शिजः॑ । \newline
35. गोम॑न्त॒मिति॒ गो - म॒न्त॒म् । \newline
36. उ॒शिजो॒ वि व्यु॑शिज॑ उ॒शिजो॒ वि । \newline
37. वि व॑व्रुर् वव्रु॒र् वि वि व॑व्रुः । \newline
38. व॒व्रु॒रिति॑ वव्रुः । \newline
39. दृ॒शा॒नो रु॒क्मो रु॒क्मो दृ॑शा॒नो दृ॑शा॒नो रु॒क्मः । \newline
40. रु॒क्म उ॒र्व्योर्व्या रु॒क्मो रु॒क्म उ॒र्व्या । \newline
41. उ॒र्व्या वि व्यु॑र्व्योर्व्या वि । \newline
42. व्य॑द्यौ दद्यौ॒द् वि व्य॑द्यौत् । \newline
43. अ॒द्यौ॒द् दु॒र्मर्.ष॑म् दु॒र्मर्.ष॑ मद्यौ दद्यौद् दु॒र्मर्.ष᳚म् । \newline
44. दु॒र्मर्.ष॒ मायु॒ रायु॑र् दु॒र्मर्.ष॑म् दु॒र्मर्.ष॒ मायुः॑ । \newline
45. दु॒र्मर्.ष॒मिति॑ दुः - मर्.ष᳚म् । \newline
46. आयुः॑ श्रि॒ये श्रि॒य आयु॒ रायुः॑ श्रि॒ये । \newline
47. श्रि॒ये रु॑चा॒नो रु॑चा॒नः श्रि॒ये श्रि॒ये रु॑चा॒नः । \newline
48. रु॒चा॒न इति॑ रुचा॒नः । \newline
49. अ॒ग्नि र॒मृतो॑ अ॒मृतो॑ अ॒ग्नि र॒ग्नि र॒मृतः॑ । \newline
50. अ॒मृतो॑ अभव दभव द॒मृतो॑ अ॒मृतो॑ अभवत् । \newline
51. अ॒भ॒व॒द् वयो॑भि॒र् वयो॑भि रभव दभव॒द् वयो॑भिः । \newline
52. वयो॑भि॒र् यद् यद् वयो॑भि॒र् वयो॑भि॒र् यत् । \newline
53. वयो॑भि॒रिति॒ वयः॑ - भिः॒ । \newline
54. यदे॑न मेनं॒ ॅयद् यदे॑नम् । \newline
55. ए॒न॒म् द्यौर् द्यौ रे॑न मेन॒म् द्यौः । \newline
56. द्यौ रज॑नय॒ दज॑नय॒द् द्यौर् द्यौ रज॑नयत् । \newline
57. अज॑नयथ् सु॒रेताः᳚ सु॒रेता॒ अज॑नय॒ दज॑नयथ् सु॒रेताः᳚ । \newline
58. सु॒रेता॒ इति॑ सु - रेताः᳚ । \newline

\textbf{Ghana Paata } \newline

1. तम् भ॑ज भज॒ तम् तम् भ॑ज सौश्रव॒सेषु॑ सौश्रव॒सेषु॑ भज॒ तम् तम् भ॑ज सौश्रव॒सेषु॑ । \newline
2. भ॒ज॒ सौ॒श्र॒व॒सेषु॑ सौश्रव॒सेषु॑ भज भज सौश्रव॒से ष्व॑ग्ने अग्ने सौश्रव॒सेषु॑ भज भज सौश्रव॒से ष्व॑ग्ने । \newline
3. सौ॒श्र॒व॒से ष्व॑ग्ने अग्ने सौश्रव॒सेषु॑ सौश्रव॒से ष्व॑ग्न उ॒क्थ‌उ॑क्थ उ॒क्थ‍उ॑क्थे अग्ने सौश्रव॒सेषु॑ सौश्रव॒से ष्व॑ग्न उ॒क्थ‍उ॑क्थे । \newline
4. अ॒ग्न॒ उ॒क्थ‍उ॑क्थ उ॒क्थ‍उ॑क्थे अग्ने अग्न उ॒क्थ‍उ॑क्थ॒ ओक्थ‍उ॑क्थे अग्ने अग्न उ॒क्थ‍उ॑क्थ॒ आ । \newline
5. उ॒क्थ‍उ॑क्थ॒ ओक्थ‍उ॑क्थ उ॒क्थ‍उ॑क्थ॒ आ भ॑ज भ॒जो क्थ‍उ॑क्थ उ॒क्थ‍उ॑क्थ॒ आ भ॑ज । \newline
6. उ॒क्थ‍उ॑क्थ॒ इत्यु॒क्थे - उ॒क्थे॒ । \newline
7. आ भ॑ज भ॒जा भ॑ज श॒स्यमा॑ने श॒स्यमा॑ने भ॒जा भ॑ज श॒स्यमा॑ने । \newline
8. भ॒ज॒ श॒स्यमा॑ने श॒स्यमा॑ने भज भज श॒स्यमा॑ने । \newline
9. श॒स्यमा॑न॒ इति॑ श॒स्यमा॑ने । \newline
10. प्रि॒यः सूर्ये॒ सूर्ये᳚ प्रि॒यः प्रि॒यः सूर्ये᳚ प्रि॒यः प्रि॒यः सूर्ये᳚ प्रि॒यः प्रि॒यः सूर्ये᳚ प्रि॒यः । \newline
11. सूर्ये᳚ प्रि॒यः प्रि॒यः सूर्ये॒ सूर्ये᳚ प्रि॒यो अ॒ग्ना ऽग्ना प्रि॒यः सूर्ये॒ सूर्ये᳚ प्रि॒यो अ॒ग्ना । \newline
12. प्रि॒यो अ॒ग्ना ऽग्ना प्रि॒यः प्रि॒यो अ॒ग्ना भ॑वाति भवा त्य॒ग्ना प्रि॒यः प्रि॒यो अ॒ग्ना भ॑वाति । \newline
13. अ॒ग्ना भ॑वाति भवा त्य॒ग्ना ऽग्ना भ॑वा॒ त्युदुद् भ॑वा त्य॒ग्ना ऽग्ना भ॑वा॒ त्युत् । \newline
14. भ॒वा॒ त्युदुद् भ॑वाति भवा॒ त्युज् जा॒तेन॑ जा॒तेनोद् भ॑वाति भवा॒ त्युज् जा॒तेन॑ । \newline
15. उज् जा॒तेन॑ जा॒तेनोदुज् जा॒तेन॑ भि॒नद॑द् भि॒नद॑ज् जा॒तेनोदुज् जा॒तेन॑ भि॒नद॑त् । \newline
16. जा॒तेन॑ भि॒नद॑द् भि॒नद॑ज् जा॒तेन॑ जा॒तेन॑ भि॒नद॒ दुदुद् भि॒नद॑ज् जा॒तेन॑ जा॒तेन॑ भि॒नद॒ दुत् । \newline
17. भि॒नद॒ दुदुद् भि॒नद॑द् भि॒नद॒दुज् जनि॑त्वै॒र् जनि॑त्वै॒रुद् भि॒नद॑द् भि॒नद॒दुज् जनि॑त्वैः । \newline
18. उज् जनि॑त्वै॒र् जनि॑त्वै॒ रुदुज् जनि॑त्वैः । \newline
19. जनि॑त्वै॒रिति॒ जनि॑त्वैः । \newline
20. त्वा म॑ग्ने अग्ने॒ त्वाम् त्वा म॑ग्ने॒ यज॑माना॒ यज॑माना अग्ने॒ त्वाम् त्वा म॑ग्ने॒ यज॑मानाः । \newline
21. अ॒ग्ने॒ यज॑माना॒ यज॑माना अग्ने अग्ने॒ यज॑माना॒ अन्वनु॒ यज॑माना अग्ने अग्ने॒ यज॑माना॒ अनु॑ । \newline
22. यज॑माना॒ अन्वनु॒ यज॑माना॒ यज॑माना॒ अनु॒ द्यून् द्यूननु॒ यज॑माना॒ यज॑माना॒ अनु॒ द्यून् । \newline
23. अनु॒ द्यून् द्यूनन्वनु॒ द्यून्. विश्वा॒ विश्वा॒ द्यू नन्वनु॒ द्यून्. विश्वा᳚ । \newline
24. द्यून्. विश्वा॒ विश्वा॒ द्यून् द्यून्. विश्वा॒ वसू॑नि॒ वसू॑नि॒ विश्वा॒ द्यून् द्यून्. विश्वा॒ वसू॑नि । \newline
25. विश्वा॒ वसू॑नि॒ वसू॑नि॒ विश्वा॒ विश्वा॒ वसू॑नि दधिरे दधिरे॒ वसू॑नि॒ विश्वा॒ विश्वा॒ वसू॑नि दधिरे । \newline
26. वसू॑नि दधिरे दधिरे॒ वसू॑नि॒ वसू॑नि दधिरे॒ वार्या॑णि॒ वार्या॑णि दधिरे॒ वसू॑नि॒ वसू॑नि दधिरे॒ वार्या॑णि । \newline
27. द॒धि॒रे॒ वार्या॑णि॒ वार्या॑णि दधिरे दधिरे॒ वार्या॑णि । \newline
28. वार्या॒णीति॒ वार्या॑णि । \newline
29. त्वया॑ स॒ह स॒ह त्वया॒ त्वया॑ स॒ह द्रवि॑ण॒म् द्रवि॑णꣳ स॒ह त्वया॒ त्वया॑ स॒ह द्रवि॑णम् । \newline
30. स॒ह द्रवि॑ण॒म् द्रवि॑णꣳ स॒ह स॒ह द्रवि॑ण मि॒च्छमा॑ना इ॒च्छमा॑ना॒ द्रवि॑णꣳ स॒ह स॒ह द्रवि॑ण मि॒च्छमा॑नाः । \newline
31. द्रवि॑ण मि॒च्छमा॑ना इ॒च्छमा॑ना॒ द्रवि॑ण॒म् द्रवि॑ण मि॒च्छमा॑ना व्र॒जं ॅव्र॒ज मि॒च्छमा॑ना॒ द्रवि॑ण॒म् द्रवि॑ण मि॒च्छमा॑ना व्र॒जम् । \newline
32. इ॒च्छमा॑ना व्र॒जं ॅव्र॒ज मि॒च्छमा॑ना इ॒च्छमा॑ना व्र॒जम् गोम॑न्त॒म् गोम॑न्तं ॅव्र॒ज मि॒च्छमा॑ना इ॒च्छमा॑ना व्र॒जम् गोम॑न्तम् । \newline
33. व्र॒जम् गोम॑न्त॒म् गोम॑न्तं ॅव्र॒जं ॅव्र॒जम् गोम॑न्त मु॒शिज॑ उ॒शिजो॒ गोम॑न्तं ॅव्र॒जं ॅव्र॒जम् गोम॑न्त मु॒शिजः॑ । \newline
34. गोम॑न्त मु॒शिज॑ उ॒शिजो॒ गोम॑न्त॒म् गोम॑न्त मु॒शिजो॒ वि व्यु॑शिजो॒ गोम॑न्त॒म् गोम॑न्त मु॒शिजो॒ वि । \newline
35. गोम॑न्त॒मिति॒ गो - म॒न्त॒म् । \newline
36. उ॒शिजो॒ वि व्यु॑शिज॑ उ॒शिजो॒ वि व॑व्रुर् वव्रु॒र् व्यु॑शिज॑ उ॒शिजो॒ वि व॑व्रुः । \newline
37. वि व॑व्रुर् वव्रु॒र् वि वि व॑व्रुः । \newline
38. व॒व्रु॒रिति॑ वव्रुः । \newline
39. दृ॒शा॒नो रु॒क्मो रु॒क्मो दृ॑शा॒नो दृ॑शा॒नो रु॒क्म उ॒र्व्योर्व्या रु॒क्मो दृ॑शा॒नो दृ॑शा॒नो रु॒क्म उ॒र्व्या । \newline
40. रु॒क्म उ॒र्व्योर्व्या रु॒क्मो रु॒क्म उ॒र्व्या वि व्यु॑र्व्या रु॒क्मो रु॒क्म उ॒र्व्या वि । \newline
41. उ॒र्व्या वि व्यु॑र्व्योर्व्या व्य॑द्यौ दद्यौ॒द् व्यु॑र्व्योर्व्या व्य॑द्यौत् । \newline
42. व्य॑द्यौ दद्यौ॒द् वि व्य॑द्यौद् दु॒र्मर्.ष॑म् दु॒र्मर्.ष॑ मद्यौ॒द् वि व्य॑द्यौद् दु॒र्मर्.ष᳚म् । \newline
43. अ॒द्यौ॒द् दु॒र्मर्.ष॑म् दु॒र्मर्.ष॑ मद्यौ दद्यौद् दु॒र्मर्.ष॒ मायु॒ रायु॑र् दु॒र्मर्.ष॑ मद्यौ दद्यौद् दु॒र्मर्.ष॒ मायुः॑ । \newline
44. दु॒र्मर्.ष॒ मायु॒ रायु॑र् दु॒र्मर्.ष॑म् दु॒र्मर्.ष॒ मायुः॑ श्रि॒ये श्रि॒य आयु॑र् दु॒र्मर्.ष॑म् दु॒र्मर्.ष॒ मायुः॑ श्रि॒ये । \newline
45. दु॒र्मर्.ष॒मिति॑ दुः - मर्.ष᳚म् । \newline
46. आयुः॑ श्रि॒ये श्रि॒य आयु॒ रायुः॑ श्रि॒ये रु॑चा॒नो रु॑चा॒नः श्रि॒य आयु॒ रायुः॑ श्रि॒ये रु॑चा॒नः । \newline
47. श्रि॒ये रु॑चा॒नो रु॑चा॒नः श्रि॒ये श्रि॒ये रु॑चा॒नः । \newline
48. रु॒चा॒न इति॑ रुचा॒नः । \newline
49. अ॒ग्नि र॒मृतो॑ अ॒मृतो॑ अ॒ग्नि र॒ग्नि र॒मृतो॑ अभव दभव द॒मृतो॑ अ॒ग्नि र॒ग्नि र॒मृतो॑ अभवत् । \newline
50. अ॒मृतो॑ अभव दभव द॒मृतो॑ अ॒मृतो॑ अभव॒द् वयो॑भि॒र् वयो॑भि रभव द॒मृतो॑ अ॒मृतो॑ अभव॒द् वयो॑भिः । \newline
51. अ॒भ॒व॒द् वयो॑भि॒र् वयो॑भि रभव दभव॒द् वयो॑भि॒र् यद् यद् वयो॑भि रभव दभव॒द् वयो॑भि॒र् यत् । \newline
52. वयो॑भि॒र् यद् यद् वयो॑भि॒र् वयो॑भि॒र् यदे॑न मेनं॒ ॅयद् वयो॑भि॒र् वयो॑भि॒र् यदे॑नम् । \newline
53. वयो॑भि॒रिति॒ वयः॑ - भिः॒ । \newline
54. यदे॑न मेनं॒ ॅयद् यदे॑न॒म् द्यौर् द्यौरे॑नं॒ ॅयद् यदे॑न॒म् द्यौः । \newline
55. ए॒न॒म् द्यौर् द्यौरे॑न मेन॒म् द्यौ रज॑नय॒ दज॑नय॒द् द्यौरे॑न मेन॒म् द्यौ रज॑नयत् । \newline
56. द्यौ रज॑नय॒ दज॑नय॒द् द्यौर् द्यौ रज॑नयथ् सु॒रेताः᳚ सु॒रेता॒ अज॑नय॒द् द्यौर् द्यौ रज॑नयथ् सु॒रेताः᳚ । \newline
57. अज॑नयथ् सु॒रेताः᳚ सु॒रेता॒ अज॑नय॒ दज॑नयथ् सु॒रेताः᳚ । \newline
58. सु॒रेता॒ इति॑ सु - रेताः᳚ । \newline
\pagebreak
\markright{ TS 4.2.3.1  \hfill https://www.vedavms.in \hfill}

\section{ TS 4.2.3.1 }

\textbf{TS 4.2.3.1 } \newline
\textbf{Samhita Paata} \newline

अन्न॑प॒तेऽन्न॑स्य नो देह्यनमी॒वस्य॑ शु॒ष्मिणः॑ । प्र प्र॑दा॒तारं॑ तारिष॒ ऊर्जं॑ नो धेहि द्वि॒पदे॒ चतु॑ष्पदे ॥ उदु॑ त्वा॒ विश्वे॑ दे॒वा अग्ने॒ भर॑न्तु॒ चित्ति॑भिः । स नो॑ भव शि॒वत॑मः सु॒प्रती॑को वि॒भाव॑सुः ॥ प्रेद॑ग्ने॒ ज्योति॑ष्मान्. याहि शि॒वेभि॑र॒र्चि॑भि॒स्त्वं । बृ॒हद्भि॑-र्भा॒नुभि॒-र्भास॒न् मा हिꣳ॑सी स्त॒नुवा᳚ प्र॒जाः ॥ स॒मिधा॒ऽग्निं दु॑वस्यत घृ॒तैर्बो॑धय॒ताति॑थिं । आ - [  ] \newline

\textbf{Pada Paata} \newline

अन्न॑पत॒ इत्यन्न॑ - प॒ते॒ । अन्न॑स्य । नः॒ । दे॒हि॒ । अ॒न॒मी॒वस्य॑ । शु॒ष्मिणः॑ ॥ प्रेति॑ । प्र॒दा॒तार॒मिति॑ प्र - दा॒तार᳚म् । ता॒रि॒षः॒ । ऊर्ज᳚म् । नः॒ । धे॒हि॒ । द्वि॒पद॒ इति॑ द्वि - पदे᳚ । चतु॑ष्पद॒ इति॒ चतुः॑ - प॒दे॒ ॥ उदिति॑ । उ॒ । त्वा॒ । विश्वे᳚ । दे॒वाः । अग्ने᳚ । भर॑न्तु । चित्ति॑भि॒रिति॒ चित्ति॑ - भिः॒ ॥ सः । नः॒ । भ॒व॒ । शि॒वत॑म॒ इति॑ शि॒व - त॒मः॒ । सु॒प्रती॑क॒ इति॑ सु - प्रती॑कः । वि॒भाव॑सु॒रिति॑ वि॒भा-व॒सुः॒ ॥ प्रेति॑ । इत् । अ॒ग्ने॒ । ज्योति॑ष्मान् । या॒हि॒ । शि॒वेभिः॑ । अ॒र्चिभि॒रित्य॒र्चि - भिः॒ । त्वम् ॥ बृ॒हद्भि॒रिति॑ बृ॒हत् - भिः॒ । भा॒नुभि॒रिति॑ भा॒नु - भिः॒ । भासन्न्॑ । मा । हिꣳ॒॒सीः॒ । त॒नुवा᳚ । प्र॒जा इति॑ प्र - जाः ॥ स॒मिधेति॑ सं - इधा᳚ । अ॒ग्निम् । दु॒व॒स्य॒त॒ । घृ॒तैः । बो॒ध॒य॒त॒ । अति॑थिम् ॥ एति॑ ।  \newline


\textbf{Krama Paata} \newline

अन्न॑प॒ते ऽन्न॑स्य । अन्न॑पत॒ इत्यन्न॑ - प॒ते॒ । अन्न॑स्य नः । नो॒ दे॒हि॒ । दे॒ह्य॒न॒मी॒वस्य॑ । अ॒न॒मी॒वस्य॑ शु॒ष्मिणः॑ । शु॒ष्मिण॒ इति॑ शु॒ष्मिणः॑ ॥ प्र प्र॑दा॒तार᳚म् । प्र॒दा॒तार॑म् तारिषः । प्र॒दा॒तार॒मिति॑ प्र - दा॒तार᳚म् । ता॒रि॒ष॒ ऊर्ज᳚म् । ऊर्ज॑म् नः । नो॒ धे॒हि॒ । धे॒हि॒ द्वि॒पदे᳚ । द्वि॒पदे॒ चतु॑ष्पदे । द्वि॒पद॒ इति॑ द्वि - पदे᳚ । चतु॑ष्पद॒ इति॒ चतुः॑ - प॒दे॒ ॥ उदु॑ । उ॒ त्वा॒ । त्वा॒ विश्वे᳚ । विश्वे॑ दे॒वाः । दे॒वा अग्ने᳚ । अग्ने॒ भर॑न्तु । भर॑न्तु॒ चित्ति॑भिः । चित्ति॑भि॒रिति॒ चित्ति॑ - भिः॒ ॥ स नः॑ । नो॒ भ॒व॒ । भ॒व॒ शि॒वत॑मः । शि॒वत॑मः सु॒प्रती॑कः । शि॒वत॑म॒ इति॑ शि॒व - त॒मः॒ । सु॒प्रती॑को वि॒भाव॑सुः । सु॒प्रती॑क॒ इति॑ सु - प्रती॑कः । वि॒भाव॑सु॒रिति॑ वि॒भा - व॒सुः॒ ॥ प्रेत् । इद॑ग्ने । अ॒ग्ने॒ ज्योति॑ष्मान् । ज्योति॑ष्मान्. याहि । या॒हि॒ शि॒वेभिः॑ । शि॒वेभि॑र॒र्चिभिः॑ । अ॒र्चिभि॒स्त्वम् । अ॒र्चिभि॒रित्य॒र्चि - भिः॒ । त्वमिति॒ त्वम् ॥ बृ॒हद्भि॑र् भा॒नुभिः॑ । बृ॒हद्भि॒रिति॑ बृ॒हत् - भिः॒ । भा॒नुभि॒र् भासन्न्॑ । भा॒नुभि॒रिति॑भा॒नु - भिः॒ । भास॒न् मा । मा हिꣳ॑सीः । हिꣳ॒॒सी॒स्त॒नुवा᳚ । त॒नुवा᳚ प्र॒जाः । प्र॒जा इति॑ प्र - जाः ॥ स॒मिधा॒ऽग्निम् । स॒मिधेति॑ सम् - इधा᳚ । अ॒ग्निम् दु॑वस्यत । दु॒व॒स्य॒त॒ घृ॒तैः । घृ॒तैर् बो॑धयत । बो॒ध॒य॒ताति॑थिम् । अति॑थि॒मित्यति॑थिम् ॥ आ ऽस्मिन्न्॑ \newline

\textbf{Jatai Paata} \newline

1. अन्न॑प॒ते ऽन्न॒स्या न्न॒स्या न्न॑प॒ते ऽन्न॑प॒ते ऽन्न॑स्य । \newline
2. अन्न॑पत॒ इत्यन्न॑ - प॒ते॒ । \newline
3. अन्न॑स्य नो नो॒ अन्न॒स्या न्न॑स्य नः । \newline
4. नो॒ दे॒हि॒ दे॒हि॒ नो॒ नो॒ दे॒हि॒ । \newline
5. दे॒ह्य॒न॒मी॒वस्या॑ नमी॒वस्य॑ देहि देह्यनमी॒वस्य॑ । \newline
6. अ॒न॒मी॒वस्य॑ शु॒ष्मिणः॑ शु॒ष्मिणो॑ ऽनमी॒वस्या॑ नमी॒वस्य॑ शु॒ष्मिणः॑ । \newline
7. शु॒ष्मिण॒ इति॑ शु॒ष्मिणः॑ । \newline
8. प्र प्र॑दा॒तार॑म् प्रदा॒तार॒म् प्र प्र प्र॑दा॒तार᳚म् । \newline
9. प्र॒दा॒तार॑म् तारिष स्तारिषः प्रदा॒तार॑म् प्रदा॒तार॑म् तारिषः । \newline
10. प्र॒दा॒तार॒मिति॑ प्र - दा॒तार᳚म् । \newline
11. ता॒रि॒ष॒ ऊर्ज॒ मूर्ज॑म् तारिष स्तारिष॒ ऊर्ज᳚म् । \newline
12. ऊर्ज॑म् नो न॒ ऊर्ज॒ मूर्ज॑म् नः । \newline
13. नो॒ धे॒हि॒ धे॒हि॒ नो॒ नो॒ धे॒हि॒ । \newline
14. धे॒हि॒ द्वि॒पदे᳚ द्वि॒पदे॑ धेहि धेहि द्वि॒पदे᳚ । \newline
15. द्वि॒पदे॒ चतु॑ष्पदे॒ चतु॑ष्पदे द्वि॒पदे᳚ द्वि॒पदे॒ चतु॑ष्पदे । \newline
16. द्वि॒पद॒ इति॑ द्वि - पदे᳚ । \newline
17. चतु॑ष्पद॒ इति॒ चतुः॑ - प॒दे॒ । \newline
18. उदु॑ वु॒ वु दु दु॑ । \newline
19. उ॒ त्वा॒ त्व॒ वु॒ त्वा॒ । \newline
20. त्वा॒ विश्वे॒ विश्वे᳚ त्वा त्वा॒ विश्वे᳚ । \newline
21. विश्वे॑ दे॒वा दे॒वा विश्वे॒ विश्वे॑ दे॒वाः । \newline
22. दे॒वा अग्ने ऽग्ने॑ दे॒वा दे॒वा अग्ने᳚ । \newline
23. अग्ने॒ भर॑न्तु॒ भर॒ न्त्वग्ने ऽग्ने॒ भर॑न्तु । \newline
24. भर॑न्तु॒ चित्ति॑भि॒ श्चित्ति॑भि॒र् भर॑न्तु॒ भर॑न्तु॒ चित्ति॑भिः । \newline
25. चित्ति॑भि॒रिति॒ चित्ति॑ - भिः॒ । \newline
26. स नो॑ नः॒ स स नः॑ । \newline
27. नो॒ भ॒व॒ भ॒व॒ नो॒ नो॒ भ॒व॒ । \newline
28. भ॒व॒ शि॒वत॑मः शि॒वत॑मो भव भव शि॒वत॑मः । \newline
29. शि॒वत॑मः सु॒प्रती॑कः सु॒प्रती॑कः शि॒वत॑मः शि॒वत॑मः सु॒प्रती॑कः । \newline
30. शि॒वत॑म॒ इति॑ शि॒व - त॒मः॒ । \newline
31. सु॒प्रती॑को वि॒भाव॑सुर् वि॒भाव॑सुः सु॒प्रती॑कः सु॒प्रती॑को वि॒भाव॑सुः । \newline
32. सु॒प्रती॑क॒ इति॑ सु - प्रती॑कः । \newline
33. वि॒भाव॑सु॒रिति॑ वि॒भा - व॒सुः॒ । \newline
34. प्रे दित् प्र प्रे त् । \newline
35. इद॑ग्ने अग्न॒ इदिद॑ग्ने । \newline
36. अ॒ग्ने॒ ज्योति॑ष्मा॒न् ज्योति॑ष्मा नग्ने अग्ने॒ ज्योति॑ष्मान् । \newline
37. ज्योति॑ष्मान्. याहि याहि॒ ज्योति॑ष्मा॒न् ज्योति॑ष्मान्. याहि । \newline
38. या॒हि॒ शि॒वेभिः॑ शि॒वेभि॑र् याहि याहि शि॒वेभिः॑ । \newline
39. शि॒वेभि॑ र॒र्चिभि॑ र॒र्चिभिः॑ शि॒वेभिः॑ शि॒वेभि॑ र॒र्चिभिः॑ । \newline
40. अ॒र्चिभि॒ स्त्वम् त्व म॒र्चिभि॑ र॒र्चिभि॒ स्त्वम् । \newline
41. अ॒र्चिभि॒रित्य॒र्चि - भिः॒ । \newline
42. त्वमिति॒ त्वम् । \newline
43. बृ॒हद्भि॑र् भा॒नुभि॑र् भा॒नुभि॑र् बृ॒हद्भि॑र् बृ॒हद्भि॑र् भा॒नुभिः॑ । \newline
44. बृ॒हद्भि॒रिति॑ बृ॒हत् - भिः॒ । \newline
45. भा॒नुभि॒र् भास॒न् भास॑न् भा॒नुभि॑र् भा॒नुभि॒र् भासन्न्॑ । \newline
46. भा॒नुभि॒रिति॑ भा॒नु - भिः॒ । \newline
47. भास॒न् मा मा भास॒न् भास॒न् मा । \newline
48. मा हिꣳ॑सीर्. हिꣳसी॒र् मा मा हिꣳ॑सीः । \newline
49. हिꣳ॒॒सी॒ स्त॒नुवा॑ त॒नुवा॑ हिꣳसीर्. हिꣳसी स्त॒नुवा᳚ । \newline
50. त॒नुवा᳚ प्र॒जाः प्र॒जा स्त॒नुवा॑ त॒नुवा᳚ प्र॒जाः । \newline
51. प्र॒जा इति॑ प्र - जाः । \newline
52. स॒मिधा॒ ऽग्नि म॒ग्निꣳ स॒मिधा॑ स॒मिधा॒ ऽग्निम् । \newline
53. स॒मिधेति॑ सं - इधा᳚ । \newline
54. अ॒ग्निम् दु॑वस्यत दुवस्यता॒ग्नि म॒ग्निम् दु॑वस्यत । \newline
55. दु॒व॒स्य॒त॒ घृ॒तैर् घृ॒तैर् दु॑वस्यत दुवस्यत घृ॒तैः । \newline
56. घृ॒तैर् बो॑धयत बोधयत घृ॒तैर् घृ॒तैर् बो॑धयत । \newline
57. बो॒ध॒य॒ता ति॑थि॒ मति॑थिम् बोधयत बोधय॒ता ति॑थिम् । \newline
58. अति॑थि॒मित्यति॑थिम् । \newline
59. आ ऽस्मि॑न् नस्मि॒न् ना ऽस्मिन्न्॑ । \newline

\textbf{Ghana Paata } \newline

1. अन्न॑प॒ते ऽन्न॒स्या न्न॒स्या न्न॑प॒ते ऽन्न॑प॒ते ऽन्न॑स्य नो नो॒ अन्न॒स्या न्न॑प॒ते ऽन्न॑प॒ते ऽन्न॑स्य नः । \newline
2. अन्न॑पत॒ इत्यन्न॑ - प॒ते॒ । \newline
3. अन्न॑स्य नो नो॒ अन्न॒स्या न्न॑स्य नो देहि देहि नो॒ अन्न॒स्या न्न॑स्य नो देहि । \newline
4. नो॒ दे॒हि॒ दे॒हि॒ नो॒ नो॒ दे॒ह्य॒न॒मी॒वस्या॑ नमी॒वस्य॑ देहि नो नो देह्यनमी॒वस्य॑ । \newline
5. दे॒ह्य॒न॒मी॒वस्या॑ नमी॒वस्य॑ देहि देह्यनमी॒वस्य॑ शु॒ष्मिणः॑ शु॒ष्मिणो॑ ऽनमी॒वस्य॑ देहि देह्यनमी॒वस्य॑ शु॒ष्मिणः॑ । \newline
6. अ॒न॒मी॒वस्य॑ शु॒ष्मिणः॑ शु॒ष्मिणो॑ ऽनमी॒वस्या॑ नमी॒वस्य॑ शु॒ष्मिणः॑ । \newline
7. शु॒ष्मिण॒ इति॑ शु॒ष्मिणः॑ । \newline
8. प्र प्र॑दा॒तार॑म् प्रदा॒तार॒म् प्र प्र प्र॑दा॒तार॑म् तारिष स्तारिषः प्रदा॒तार॒म् प्र प्र प्र॑दा॒तार॑म् तारिषः । \newline
9. प्र॒दा॒तार॑म् तारिष स्तारिषः प्रदा॒तार॑म् प्रदा॒तार॑म् तारिष॒ ऊर्ज॒ मूर्ज॑म् तारिषः प्रदा॒तार॑म् प्रदा॒तार॑म् तारिष॒ ऊर्ज᳚म् । \newline
10. प्र॒दा॒तार॒मिति॑ प्र - दा॒तार᳚म् । \newline
11. ता॒रि॒ष॒ ऊर्ज॒ मूर्ज॑म् तारिष स्तारिष॒ ऊर्ज॑म् नो न॒ ऊर्ज॑म् तारिष स्तारिष॒ ऊर्ज॑म् नः । \newline
12. ऊर्ज॑म् नो न॒ ऊर्ज॒ मूर्ज॑म् नो धेहि धेहि न॒ ऊर्ज॒ मूर्ज॑म् नो धेहि । \newline
13. नो॒ धे॒हि॒ धे॒हि॒ नो॒ नो॒ धे॒हि॒ द्वि॒पदे᳚ द्वि॒पदे॑ धेहि नो नो धेहि द्वि॒पदे᳚ । \newline
14. धे॒हि॒ द्वि॒पदे᳚ द्वि॒पदे॑ धेहि धेहि द्वि॒पदे॒ चतु॑ष्पदे॒ चतु॑ष्पदे द्वि॒पदे॑ धेहि धेहि द्वि॒पदे॒ चतु॑ष्पदे । \newline
15. द्वि॒पदे॒ चतु॑ष्पदे॒ चतु॑ष्पदे द्वि॒पदे᳚ द्वि॒पदे॒ चतु॑ष्पदे । \newline
16. द्वि॒पद॒ इति॑ द्वि - पदे᳚ । \newline
17. चतु॑ष्पद॒ इति॒ चतुः॑ - प॒दे॒ । \newline
18. उदु॑ वु॒ वु दु दु॑ त्वा त्व॒ वु दु दु॑ त्वा । \newline
19. उ॒ त्वा॒ त्व॒ वु॒ त्वा॒ विश्वे॒ विश्वे᳚ त्व वु त्वा॒ विश्वे᳚ । \newline
20. त्वा॒ विश्वे॒ विश्वे᳚ त्वा त्वा॒ विश्वे॑ दे॒वा दे॒वा विश्वे᳚ त्वा त्वा॒ विश्वे॑ दे॒वाः । \newline
21. विश्वे॑ दे॒वा दे॒वा विश्वे॒ विश्वे॑ दे॒वा अग्ने ऽग्ने॑ दे॒वा विश्वे॒ विश्वे॑ दे॒वा अग्ने᳚ । \newline
22. दे॒वा अग्ने ऽग्ने॑ दे॒वा दे॒वा अग्ने॒ भर॑न्तु॒ भर॒न्त्वग्ने॑ दे॒वा दे॒वा अग्ने॒ भर॑न्तु । \newline
23. अग्ने॒ भर॑न्तु॒ भर॒न्त्वग्ने ऽग्ने॒ भर॑न्तु॒ चित्ति॑भि॒ श्चित्ति॑भि॒र् भर॒न्त्वग्ने ऽग्ने॒ भर॑न्तु॒ चित्ति॑भिः । \newline
24. भर॑न्तु॒ चित्ति॑भि॒ श्चित्ति॑भि॒र् भर॑न्तु॒ भर॑न्तु॒ चित्ति॑भिः । \newline
25. चित्ति॑भि॒रिति॒ चित्ति॑ - भिः॒ । \newline
26. स नो॑ नः॒ स स नो॑ भव भव नः॒ स स नो॑ भव । \newline
27. नो॒ भ॒व॒ भ॒व॒ नो॒ नो॒ भ॒व॒ शि॒वत॑मः शि॒वत॑मो भव नो नो भव शि॒वत॑मः । \newline
28. भ॒व॒ शि॒वत॑मः शि॒वत॑मो भव भव शि॒वत॑मः सु॒प्रती॑कः सु॒प्रती॑कः शि॒वत॑मो भव भव शि॒वत॑मः सु॒प्रती॑कः । \newline
29. शि॒वत॑मः सु॒प्रती॑कः सु॒प्रती॑कः शि॒वत॑मः शि॒वत॑मः सु॒प्रती॑को वि॒भाव॑सुर् वि॒भाव॑सुः सु॒प्रती॑कः शि॒वत॑मः शि॒वत॑मः सु॒प्रती॑को वि॒भाव॑सुः । \newline
30. शि॒वत॑म॒ इति॑ शि॒व - त॒मः॒ । \newline
31. सु॒प्रती॑को वि॒भाव॑सुर् वि॒भाव॑सुः सु॒प्रती॑कः सु॒प्रती॑को वि॒भाव॑सुः । \newline
32. सु॒प्रती॑क॒ इति॑ सु - प्रती॑कः । \newline
33. वि॒भाव॑सु॒रिति॑ वि॒भा - व॒सुः॒ । \newline
34. प्रे दित् प्र प्रे द॑ग्ने अग्न॒ इत् प्र प्रे द॑ग्ने । \newline
35. इद॑ग्ने अग्न॒ इदिद॑ग्ने॒ ज्योति॑ष्मा॒न् ज्योति॑ष्मा नग्न॒ इदिद॑ग्ने॒ ज्योति॑ष्मान् । \newline
36. अ॒ग्ने॒ ज्योति॑ष्मा॒न् ज्योति॑ष्मा नग्ने अग्ने॒ ज्योति॑ष्मान्. याहि याहि॒ ज्योति॑ष्मा नग्ने अग्ने॒ ज्योति॑ष्मान्. याहि । \newline
37. ज्योति॑ष्मान्. याहि याहि॒ ज्योति॑ष्मा॒न् ज्योति॑ष्मान्. याहि शि॒वेभिः॑ शि॒वेभि॑र् याहि॒ ज्योति॑ष्मा॒न् ज्योति॑ष्मान्. याहि शि॒वेभिः॑ । \newline
38. या॒हि॒ शि॒वेभिः॑ शि॒वेभि॑र् याहि याहि शि॒वेभि॑ र॒र्चिभि॑ र॒र्चिभिः॑ शि॒वेभि॑र् याहि याहि शि॒वेभि॑ र॒र्चिभिः॑ । \newline
39. शि॒वेभि॑ र॒र्चिभि॑ र॒र्चिभिः॑ शि॒वेभिः॑ शि॒वेभि॑ र॒र्चिभि॒ स्त्वम् त्व म॒र्चिभिः॑ शि॒वेभिः॑ शि॒वेभि॑ र॒र्चिभि॒ स्त्वम् । \newline
40. अ॒र्चिभि॒ स्त्वम् त्व म॒र्चिभि॑ र॒र्चिभि॒ स्त्वम् । \newline
41. अ॒र्चिभि॒रित्य॒र्चि - भिः॒ । \newline
42. त्वमिति॒ त्वम् । \newline
43. बृ॒हद्भि॑र् भा॒नुभि॑र् भा॒नुभि॑र् बृ॒हद्भि॑र् बृ॒हद्भि॑र् भा॒नुभि॒र् भास॒न् भास॑न् भा॒नुभि॑र् बृ॒हद्भि॑र् बृ॒हद्भि॑र् भा॒नुभि॒र् भासन्न्॑ । \newline
44. बृ॒हद्भि॒रिति॑ बृ॒हत् - भिः॒ । \newline
45. भा॒नुभि॒र् भास॒न् भास॑न् भा॒नुभि॑र् भा॒नुभि॒र् भास॒न् मा मा भास॑न् भा॒नुभि॑र् भा॒नुभि॒र् भास॒न् मा । \newline
46. भा॒नुभि॒रिति॑ भा॒नु - भिः॒ । \newline
47. भास॒न् मा मा भास॒न् भास॒न् मा हिꣳ॑सीर्. हिꣳसी॒र् मा भास॒न् भास॒न् मा हिꣳ॑सीः । \newline
48. मा हिꣳ॑सीर्. हिꣳसी॒र् मा मा हिꣳ॑सी स्त॒नुवा॑ त॒नुवा॑ हिꣳसी॒र् मा मा हिꣳ॑सी स्त॒नुवा᳚ । \newline
49. हिꣳ॒॒सी॒ स्त॒नुवा॑ त॒नुवा॑ हिꣳसीर्. हिꣳसी स्त॒नुवा᳚ प्र॒जाः प्र॒जा स्त॒नुवा॑ हिꣳसीर्. हिꣳसी स्त॒नुवा᳚ प्र॒जाः । \newline
50. त॒नुवा᳚ प्र॒जाः प्र॒जा स्त॒नुवा॑ त॒नुवा᳚ प्र॒जाः । \newline
51. प्र॒जा इति॑ प्र - जाः । \newline
52. स॒मिधा॒ ऽग्नि म॒ग्निꣳ स॒मिधा॑ स॒मिधा॒ ऽग्निम् दु॑वस्यत दुवस्यता॒ ग्निꣳ स॒मिधा॑ स॒मिधा॒ ऽग्निम् दु॑वस्यत । \newline
53. स॒मिधेति॑ सं - इधा᳚ । \newline
54. अ॒ग्निम् दु॑वस्यत दुवस्यता॒ग्नि म॒ग्निम् दु॑वस्यत घृ॒तैर् घृ॒तैर् दु॑वस्यता॒ग्नि म॒ग्निम् दु॑वस्यत घृ॒तैः । \newline
55. दु॒व॒स्य॒त॒ घृ॒तैर् घृ॒तैर् दु॑वस्यत दुवस्यत घृ॒तैर् बो॑धयत बोधयत घृ॒तैर् दु॑वस्यत दुवस्यत घृ॒तैर् बो॑धयत । \newline
56. घृ॒तैर् बो॑धयत बोधयत घृ॒तैर् घृ॒तैर् बो॑धय॒ता ति॑थि॒ मति॑थिम् बोधयत घृ॒तैर् घृ॒तैर् बो॑धय॒ता ति॑थिम् । \newline
57. बो॒ध॒य॒ता ति॑थि॒ मति॑थिम् बोधयत बोधय॒ ताति॑थिम् । \newline
58. अति॑थि॒मित्यति॑थिम् । \newline
59. आ ऽस्मि॑न् नस्मि॒न् ना ऽस्मि॑न्. ह॒व्या ह॒व्या ऽस्मि॒न् ना ऽस्मि॑न्. ह॒व्या । \newline
\pagebreak
\markright{ TS 4.2.3.2  \hfill https://www.vedavms.in \hfill}

\section{ TS 4.2.3.2 }

\textbf{TS 4.2.3.2 } \newline
\textbf{Samhita Paata} \newline

ऽस्मि॑न्. ह॒व्या जु॑होतन ॥ प्रप्रा॒यम॒ग्निर्भ॑र॒तस्य॑ शृण्वे॒ वि यथ् सूर्यो॒ न रोच॑ते बृ॒हद्भाः । अ॒भि यः पू॒रुं पृत॑नासु त॒स्थौ दी॒दाय॒ दैव्यो॒ अति॑थिः शि॒वो नः॑ ॥ आपो॑ देवीः॒ प्रति॑ गृह्णीत॒ भस्मै॒तथ् स्यो॒ने कृ॑णुद्ध्वꣳ सुर॒भावु॑ लो॒के । तस्मै॑ नमन्तां॒ जन॑यः सु॒पत्नी᳚र्मा॒तेव॑ पु॒त्रं बि॑भृ॒ता स्वे॑नं ॥ अ॒फ्स्व॑ग्ने॒ सधि॒ष्टव॒- [  ] \newline

\textbf{Pada Paata} \newline

अ॒स्मि॒न्न् । ह॒व्या । जु॒हो॒त॒न॒ ॥ प्रप्रेति॒ प्र - प्र॒ । अ॒यम् । अ॒ग्निः । भ॒र॒तस्य॑ । शृ॒ण्वे॒ । वीति॑ । यत् । सूर्यः॑ । न । रोच॑ते । बृ॒हत् । भाः ॥ अ॒भीति॑ । यः । पू॒रुम् । पृत॑नासु । त॒स्थौ । दी॒दाय॑ । दैव्यः॑ । अति॑थिः । शि॒वः । नः॒ ॥ आपः॑ । दे॒वीः॒ । प्रतीति॑ । गृ॒ह्णी॒त॒ । भस्म॑ । ए॒तत् । स्यो॒ने । कृ॒णु॒द्ध्व॒म् । सु॒र॒भौ । उ॒ । लो॒के ॥ तस्मै᳚ । न॒म॒न्ता॒म् । जन॑यः । सु॒पत्नी॒रिति॑ सु - पत्नीः᳚ । मा॒ता । इ॒व॒ । पु॒त्रम् । बि॒भृ॒त । स्विति॑ । ए॒न॒म् ॥ अ॒फ्स्वित्य॑प् - सु । अ॒ग्ने॒ । सधिः॑ । तव॑ ।  \newline


\textbf{Krama Paata} \newline

अ॒स्मि॒न्न्॒. ह॒व्या । ह॒व्या जु॑होतन । जु॒हो॒त॒नेति॑ जुहोतन ॥ प्रप्रा॒यम् । प्रप्रेति॒ प्र - प्र॒ । अ॒यम॒ग्निः । अ॒ग्निर् भ॑र॒तस्य॑ । भ॒र॒तस्य॑ शृण्वे । शृ॒ण्वे॒ वि । वि यत् । यथ् सूर्यः॑ । सूर्यो॒ न । न रोच॑ते । रोच॑ते बृ॒हत् । बृ॒हद् भाः । भा इति॒ भाः ॥ अ॒भि यः । यः पू॒रुम् । पू॒रुम् पृत॑नासु । पृत॑नासु त॒स्थौ । त॒स्थौ दी॒दाय॑ । दी॒दाय॒ दैव्यः॑ । दैव्यो॒ अति॑थिः । अति॑थिः शि॒वः । शि॒वो नः॑ । न॒ इति॑ नः ॥ आपो॑ देवीः । दे॒वीः॒ प्रति॑ । प्रति॑ गृह्णीत । गृ॒ह्णी॒त॒ भस्म॑ । भस्मै॒तत् । ए॒तथ् स्यो॒ने । स्यो॒ने कृ॑णुद्ध्वम् । कृ॒णु॒द्ध्वꣳ॒॒ सु॒र॒भौ । सु॒र॒भावु॑ । उ॒ लो॒के । लो॒क इति॑ लो॒के ॥ तस्मै॑ नमन्ताम् । न॒म॒न्ता॒म् जन॑यः । जन॑यः सु॒पत्नीः᳚ । सु॒पत्नी᳚र् मा॒ता । सु॒पत्नी॒रिति॑ सु - पत्नीः᳚ । मा॒तेव॑ । इ॒व॒ पु॒त्रम् । पु॒त्रम् बि॑भृ॒त । बि॒भृ॒ता सु । स्वे॑नम् । ए॒न॒मित्ये॑नम् ॥ अ॒फ्स्व॑ग्ने । अ॒फ्स्वित्य॑प् - सु । अ॒ग्ने॒ सधिः॑ । सधि॒ष्टव॑ । तव॒ सः \newline

\textbf{Jatai Paata} \newline

1. अ॒स्मि॒न्॒. ह॒व्या ह॒व्या ऽस्मि॑न् नस्मिन्. ह॒व्या । \newline
2. ह॒व्या जु॑होतन जुहोतन ह॒व्या ह॒व्या जु॑होतन । \newline
3. जु॒हो॒त॒नेति॑ जुहोतन । \newline
4. प्रप्रा॒य म॒यम् प्रप्र॒ प्रप्रा॒यम् । \newline
5. प्रप्रेति॒ प्र - प्र॒ । \newline
6. अ॒य म॒ग्नि र॒ग्नि र॒य म॒य म॒ग्निः । \newline
7. अ॒ग्निर् भ॑र॒तस्य॑ भर॒तस्या॒ग्नि र॒ग्निर् भ॑र॒तस्य॑ । \newline
8. भ॒र॒तस्य॑ शृण्वे शृण्वे भर॒तस्य॑ भर॒तस्य॑ शृण्वे । \newline
9. शृ॒ण्वे॒ वि वि शृ॑ण्वे शृण्वे॒ वि । \newline
10. वि यद् यद् वि वि यत् । \newline
11. यथ् सूर्यः॒ सूर्यो॒ यद् यथ् सूर्यः॑ । \newline
12. सूर्यो॒ न न सूर्यः॒ सूर्यो॒ न । \newline
13. न रोच॑ते॒ रोच॑ते॒ न न रोच॑ते । \newline
14. रोच॑ते बृ॒हद् बृ॒हद् रोच॑ते॒ रोच॑ते बृ॒हत् । \newline
15. बृ॒हद् भा भा बृ॒हद् बृ॒हद् भाः । \newline
16. भा इति॒ भाः । \newline
17. अ॒भि यो यो अ॒भ्य॑भि यः । \newline
18. यः पू॒रुम् पू॒रुं ॅयो यः पू॒रुम् । \newline
19. पू॒रुम् पृत॑नासु॒ पृत॑नासु पू॒रुम् पू॒रुम् पृत॑नासु । \newline
20. पृत॑नासु त॒स्थौ त॒स्थौ पृत॑नासु॒ पृत॑नासु त॒स्थौ । \newline
21. त॒स्थौ दी॒दाय॑ दी॒दाय॑ त॒स्थौ त॒स्थौ दी॒दाय॑ । \newline
22. दी॒दाय॒ दैव्यो॒ दैव्यो॑ दी॒दाय॑ दी॒दाय॒ दैव्यः॑ । \newline
23. दैव्यो॒ अति॑थि॒ रति॑थि॒र् दैव्यो॒ दैव्यो॒ अति॑थिः । \newline
24. अति॑थिः शि॒वः शि॒वो अति॑थि॒ रति॑थिः शि॒वः । \newline
25. शि॒वो नो॑ नः शि॒वः शि॒वो नः॑ । \newline
26. न॒ इति॑ नः । \newline
27. आपो॑ देवीर् देवी॒ राप॒ आपो॑ देवीः । \newline
28. दे॒वीः॒ प्रति॒ प्रति॑ देवीर् देवीः॒ प्रति॑ । \newline
29. प्रति॑ गृह्णीत गृह्णीत॒ प्रति॒ प्रति॑ गृह्णीत । \newline
30. गृ॒ह्णी॒त॒ भस्म॒ भस्म॑ गृह्णीत गृह्णीत॒ भस्म॑ । \newline
31. भस्मै॒त दे॒तद् भस्म॒ भस्मै॒तत् । \newline
32. ए॒तथ् स्यो॒ने स्यो॒न ए॒त दे॒तथ् स्यो॒ने । \newline
33. स्यो॒ने कृ॑णुद्ध्वम् कृणुद्ध्वꣳ स्यो॒ने स्यो॒ने कृ॑णुद्ध्वम् । \newline
34. कृ॒णु॒द्ध्वꣳ॒॒ सु॒र॒भौ सु॑र॒भौ कृ॑णुद्ध्वम् कृणुद्ध्वꣳ सुर॒भौ । \newline
35. सु॒र॒भा वु॑ वु सुर॒भौ सु॑र॒भा वु॑ । \newline
36. उ॒ लो॒के लो॒क उ॑ वु लो॒के । \newline
37. लो॒क इति॑ लो॒के । \newline
38. तस्मै॑ नमन्ताम् नमन्ता॒म् तस्मै॒ तस्मै॑ नमन्ताम् । \newline
39. न॒म॒न्ता॒म् जन॑यो॒ जन॑यो नमन्ताम् नमन्ता॒म् जन॑यः । \newline
40. जन॑यः सु॒पत्नीः᳚ सु॒पत्नी॒र् जन॑यो॒ जन॑यः सु॒पत्नीः᳚ । \newline
41. सु॒पत्नी᳚र् मा॒ता मा॒ता सु॒पत्नीः᳚ सु॒पत्नी᳚र् मा॒ता । \newline
42. सु॒पत्नी॒रिति॑ सु - पत्नीः᳚ । \newline
43. मा॒तेवे॑व मा॒ता मा॒तेव॑ । \newline
44. इ॒व॒ पु॒त्रम् पु॒त्र मि॑वेव पु॒त्रम् । \newline
45. पु॒त्रम् बि॑भृ॒त बि॑भृ॒त पु॒त्रम् पु॒त्रम् बि॑भृ॒त । \newline
46. बि॒भृ॒ता सु सु बि॑भृ॒त बि॑भृ॒ता सु । \newline
47. स्वे॑न मेनꣳ॒॒ सु स्वे॑नम् । \newline
48. ए॒न॒मित्ये॑नम् । \newline
49. अ॒फ्स्व॑ग्ने अग्ने॒ ऽफ्स्वा᳚(1॒)फ्स्व॑ग्ने । \newline
50. अ॒फ्स्वित्य॑प् - सु । \newline
51. अ॒ग्ने॒ सधिः॒ सधि॑ रग्ने अग्ने॒ सधिः॑ । \newline
52. सधि॒ ष्टव॒ तव॒ सधिः॒ सधि॒ ष्टव॑ । \newline
53. तव॒ स स तव॒ तव॒ सः । \newline

\textbf{Ghana Paata } \newline

1. अ॒स्मि॒न्॒. ह॒व्या ह॒व्या ऽस्मि॑न् नस्मिन्. ह॒व्या जु॑होतन जुहोतन ह॒व्या ऽस्मि॑न् नस्मिन्. ह॒व्या जु॑होतन । \newline
2. ह॒व्या जु॑होतन जुहोतन ह॒व्या ह॒व्या जु॑होतन । \newline
3. जु॒हो॒त॒नेति॑ जुहोतन । \newline
4. प्रप्रा॒य म॒यम् प्रप्र॒ प्रप्रा॒य म॒ग्नि र॒ग्नि र॒यम् प्रप्र॒ प्रप्रा॒य म॒ग्निः । \newline
5. प्रप्रेति॒ प्र - प्र॒ । \newline
6. अ॒य म॒ग्नि र॒ग्नि र॒य म॒य म॒ग्निर् भ॑र॒तस्य॑ भर॒तस्या॒ ग्निर॒य म॒य म॒ग्निर् भ॑र॒तस्य॑ । \newline
7. अ॒ग्निर् भ॑र॒तस्य॑ भर॒तस्या॒ ग्निर॒ग्निर् भ॑र॒तस्य॑ शृण्वे शृण्वे भर॒तस्या॒ ग्निर॒ग्निर् भ॑र॒तस्य॑ शृण्वे । \newline
8. भ॒र॒तस्य॑ शृण्वे शृण्वे भर॒तस्य॑ भर॒तस्य॑ शृण्वे॒ वि वि शृ॑ण्वे भर॒तस्य॑ भर॒तस्य॑ शृण्वे॒ वि । \newline
9. शृ॒ण्वे॒ वि वि शृ॑ण्वे शृण्वे॒ वि यद् यद् वि शृ॑ण्वे शृण्वे॒ वि यत् । \newline
10. वि यद् यद् वि वि यथ् सूर्यः॒ सूर्यो॒ यद् वि वि यथ् सूर्यः॑ । \newline
11. यथ् सूर्यः॒ सूर्यो॒ यद् यथ् सूर्यो॒ न न सूर्यो॒ यद् यथ् सूर्यो॒ न । \newline
12. सूर्यो॒ न न सूर्यः॒ सूर्यो॒ न रोच॑ते॒ रोच॑ते॒ न सूर्यः॒ सूर्यो॒ न रोच॑ते । \newline
13. न रोच॑ते॒ रोच॑ते॒ न न रोच॑ते बृ॒हद् बृ॒हद् रोच॑ते॒ न न रोच॑ते बृ॒हत् । \newline
14. रोच॑ते बृ॒हद् बृ॒हद् रोच॑ते॒ रोच॑ते बृ॒हद् भा भा बृ॒हद् रोच॑ते॒ रोच॑ते बृ॒हद् भाः । \newline
15. बृ॒हद् भा भा बृ॒हद् बृ॒हद् भाः । \newline
16. भा इति॒ भाः । \newline
17. अ॒भि यो यो अ॒भ्य॑भि यः पू॒रुम् पू॒रुं ॅयो अ॒भ्य॑भि यः पू॒रुम् । \newline
18. यः पू॒रुम् पू॒रुं ॅयो यः पू॒रुम् पृत॑नासु॒ पृत॑नासु पू॒रुं ॅयो यः पू॒रुम् पृत॑नासु । \newline
19. पू॒रुम् पृत॑नासु॒ पृत॑नासु पू॒रुम् पू॒रुम् पृत॑नासु त॒स्थौ त॒स्थौ पृत॑नासु पू॒रुम् पू॒रुम् पृत॑नासु त॒स्थौ । \newline
20. पृत॑नासु त॒स्थौ त॒स्थौ पृत॑नासु॒ पृत॑नासु त॒स्थौ दी॒दाय॑ दी॒दाय॑ त॒स्थौ पृत॑नासु॒ पृत॑नासु त॒स्थौ दी॒दाय॑ । \newline
21. त॒स्थौ दी॒दाय॑ दी॒दाय॑ त॒स्थौ त॒स्थौ दी॒दाय॒ दैव्यो॒ दैव्यो॑ दी॒दाय॑ त॒स्थौ त॒स्थौ दी॒दाय॒ दैव्यः॑ । \newline
22. दी॒दाय॒ दैव्यो॒ दैव्यो॑ दी॒दाय॑ दी॒दाय॒ दैव्यो॒ अति॑थि॒ रति॑थि॒र् दैव्यो॑ दी॒दाय॑ दी॒दाय॒ दैव्यो॒ अति॑थिः । \newline
23. दैव्यो॒ अति॑थि॒ रति॑थि॒र् दैव्यो॒ दैव्यो॒ अति॑थिः शि॒वः शि॒वो अति॑थि॒र् दैव्यो॒ दैव्यो॒ अति॑थिः शि॒वः । \newline
24. अति॑थिः शि॒वः शि॒वो अति॑थि॒ रति॑थिः शि॒वो नो॑ नः शि॒वो अति॑थि॒ रति॑थिः शि॒वो नः॑ । \newline
25. शि॒वो नो॑ नः शि॒वः शि॒वो नः॑ । \newline
26. न॒ इति॑ नः । \newline
27. आपो॑ देवीर् देवी॒ राप॒ आपो॑ देवीः॒ प्रति॒ प्रति॑ देवी॒ राप॒ आपो॑ देवीः॒ प्रति॑ । \newline
28. दे॒वीः॒ प्रति॒ प्रति॑ देवीर् देवीः॒ प्रति॑ गृह्णीत गृह्णीत॒ प्रति॑ देवीर् देवीः॒ प्रति॑ गृह्णीत । \newline
29. प्रति॑ गृह्णीत गृह्णीत॒ प्रति॒ प्रति॑ गृह्णीत॒ भस्म॒ भस्म॑ गृह्णीत॒ प्रति॒ प्रति॑ गृह्णीत॒ भस्म॑ । \newline
30. गृ॒ह्णी॒त॒ भस्म॒ भस्म॑ गृह्णीत गृह्णीत॒ भस्मै॒त दे॒तद् भस्म॑ गृह्णीत गृह्णीत॒ भस्मै॒तत् । \newline
31. भस्मै॒त दे॒तद् भस्म॒ भस्मै॒तथ् स्यो॒ने स्यो॒न ए॒तद् भस्म॒ भस्मै॒तथ् स्यो॒ने । \newline
32. ए॒तथ् स्यो॒ने स्यो॒न ए॒त दे॒तथ् स्यो॒ने कृ॑णुद्ध्वम् कृणुद्ध्वꣳ स्यो॒न ए॒त दे॒तथ् स्यो॒ने कृ॑णुद्ध्वम् । \newline
33. स्यो॒ने कृ॑णुद्ध्वम् कृणुद्ध्वꣳ स्यो॒ने स्यो॒ने कृ॑णुद्ध्वꣳ सुर॒भौ सु॑र॒भौ कृ॑णुद्ध्वꣳ स्यो॒ने स्यो॒ने कृ॑णुद्ध्वꣳ सुर॒भौ । \newline
34. कृ॒णु॒द्ध्वꣳ॒॒ सु॒र॒भौ सु॑र॒भौ कृ॑णुद्ध्वम् कृणुद्ध्वꣳ सुर॒भा वु॑ वु सुर॒भौ कृ॑णुद्ध्वम् कृणुद्ध्वꣳ सुर॒भा वु॑ । \newline
35. सु॒र॒भा वु॑ वु सुर॒भौ सु॑र॒भा वु॑ लो॒के लो॒क उ॑ सुर॒भौ सु॑र॒भा वु॑ लो॒के । \newline
36. उ॒ लो॒के लो॒क उ॑ वु लो॒के । \newline
37. लो॒क इति॑ लो॒के । \newline
38. तस्मै॑ नमन्ताम् नमन्ता॒म् तस्मै॒ तस्मै॑ नमन्ता॒म् जन॑यो॒ जन॑यो नमन्ता॒म् तस्मै॒ तस्मै॑ नमन्ता॒म् जन॑यः । \newline
39. न॒म॒न्ता॒म् जन॑यो॒ जन॑यो नमन्ताम् नमन्ता॒म् जन॑यः सु॒पत्नीः᳚ सु॒पत्नी॒र् जन॑यो नमन्ताम् नमन्ता॒म् जन॑यः सु॒पत्नीः᳚ । \newline
40. जन॑यः सु॒पत्नीः᳚ सु॒पत्नी॒र् जन॑यो॒ जन॑यः सु॒पत्नी᳚र् मा॒ता मा॒ता सु॒पत्नी॒र् जन॑यो॒ जन॑यः सु॒पत्नी᳚र् मा॒ता । \newline
41. सु॒पत्नी᳚र् मा॒ता मा॒ता सु॒पत्नीः᳚ सु॒पत्नी᳚र् मा॒तेवे॑व मा॒ता सु॒पत्नीः᳚ सु॒पत्नी᳚र् मा॒तेव॑ । \newline
42. सु॒पत्नी॒रिति॑ सु - पत्नीः᳚ । \newline
43. मा॒तेवे॑व मा॒ता मा॒तेव॑ पु॒त्रम् पु॒त्र मि॑व मा॒ता मा॒तेव॑ पु॒त्रम् । \newline
44. इ॒व॒ पु॒त्रम् पु॒त्र मि॑वेव पु॒त्रम् बि॑भृ॒त बि॑भृ॒त पु॒त्र मि॑वेव पु॒त्रम् बि॑भृ॒त । \newline
45. पु॒त्रम् बि॑भृ॒त बि॑भृ॒त पु॒त्रम् पु॒त्रम् बि॑भृ॒ता सु सु बि॑भृ॒त पु॒त्रम् पु॒त्रम् बि॑भृ॒ता सु । \newline
46. बि॒भृ॒ता सु सु बि॑भृ॒त बि॑भृ॒ता स्वे॑न मेनꣳ॒॒ सु बि॑भृ॒त बि॑भृ॒ता स्वे॑नम् । \newline
47. स्वे॑न मेनꣳ॒॒ सु स्वे॑नम् । \newline
48. ए॒न॒मित्ये॑नम् । \newline
49. अ॒फ्स्व॑ग्ने अग्ने॒ ऽफ्स्वा᳚(1॒)फ्स्व॑ग्ने॒ सधिः॒ सधि॑रग्ने॒ ऽफ्स्वा᳚(1॒)फ्स्व॑ग्ने॒ सधिः॑ । \newline
50. अ॒फ्स्वित्य॑प् - सु । \newline
51. अ॒ग्ने॒ सधिः॒ सधि॑ रग्ने अग्ने॒ सधि॒ष्टव॒ तव॒ सधि॑ रग्ने अग्ने॒ सधि॒ष्टव॑ । \newline
52. सधि॒ष् टव॒ तव॒ सधिः॒ सधि॒ष्टव॒ स स तव॒ सधिः॒ सधि॒ष्टव॒ सः । \newline
53. तव॒ स स तव॒ तव॒ सौष॑धी॒ रोष॑धीः॒ स तव॒ तव॒ सौष॑धीः । \newline
\pagebreak
\markright{ TS 4.2.3.3  \hfill https://www.vedavms.in \hfill}

\section{ TS 4.2.3.3 }

\textbf{TS 4.2.3.3 } \newline
\textbf{Samhita Paata} \newline

सौष॑धी॒रनु॑ रुद्ध्यसे । गर्भे॒ सञ्जा॑यसे॒ पुनः॑ ॥ गर्भो॑ अ॒स्योष॑धीनां॒ गर्भो॒ वन॒स्पती॑नां । गर्भो॒ विश्व॑स्य भू॒तस्याग्ने॒ गर्भो॑ अ॒पाम॑सि ॥ प्र॒सद्य॒ भस्म॑ना॒ योनि॑म॒पश्च॑ पृथि॒वीम॑ग्ने । सꣳ॒॒सृज्य॑ मा॒तृभि॒स्त्वं ज्योति॑ष्मा॒न् पुन॒राऽस॑दः ॥ पुन॑रा॒सद्य॒ सद॑नम॒पश्च॑ पृथि॒वीम॑ग्ने । शेषे॑ मा॒तुर्यथो॒पस्थे॒ ऽन्तर॒स्याꣳ शि॒वत॑मः ॥ पुन॑रू॒र्जा - [  ] \newline

\textbf{Pada Paata} \newline

सः । ओष॑धीः । अन्विति॑ । रु॒द्ध्य॒से॒ ॥ गर्भे᳚ । सन्न् । जा॒य॒से॒ । पुनः॑ ॥ गर्भः॑ । अ॒सि॒ । ओष॑धीनाम् । गर्भः॑ । वन॒स्पती॑नाम् ॥ गर्भः॑ । विश्व॑स्य । भू॒तस्य॑ । अग्ने᳚ । गर्भः॑ । अ॒पाम् । अ॒सि॒ ॥ प्र॒सद्येति॑ प्र - सद्य॑ । भस्म॑ना । योनि᳚म् । अ॒पः । च॒ । पृ॒थि॒वीम् । अ॒ग्ने॒ ॥ सꣳ॒॒सृज्येति॑ सं - सृज्य॑ । मा॒तृभि॒रिति॑ मा॒तृ - भिः॒ । त्वम् । ज्योति॑ष्मान् । पुनः॑ । एति॑ । अ॒स॒दः॒ ॥ पुनः॑ । आ॒सद्येत्या᳚-सद्य॑ । सद॑नम् । अ॒पः । च॒ । पृ॒थि॒वीम् । अ॒ग्ने॒ ॥ शेषे᳚ । मा॒तुः । यथा᳚ । उ॒पस्थ॒ इत्यु॒प - स्थे॒ । अ॒न्तः । अ॒स्याम् । शि॒वत॑म॒ इति॑ शि॒व - त॒मः॒ ॥ पुनः॑ । ऊ॒र्जा ।  \newline


\textbf{Krama Paata} \newline

सौष॑धीः । ओष॑धी॒रनु॑ । अनु॑ रुद्ध्यसे । रु॒द्ध्य॒स॒ इति॑ रुद्ध्यसे ॥ गर्भे॒ सन्न् । सन् जा॑यसे । जा॒य॒से॒ पुनः॑ । पुन॒रिति॒ पुनः॑ ॥ गर्भो॑ असि । अ॒स्योष॑धीनाम् । ओष॑धीना॒म् गर्भः॑ । गर्भो॒ वन॒स्पती॑नाम् । वन॒स्पती॑ना॒मिति॒ वन॒स्पती॑नाम् ॥ गर्भो॒ विश्व॑स्य । विश्व॑स्य भू॒तस्य॑ । भू॒तस्याग्ने᳚ । अग्ने॒ गर्भः॑ । गर्भो॑ अ॒पाम् । अ॒पाम॑सि । अ॒सीत्य॑सि ॥ प्र॒सद्य॒ भस्म॑ना । प्र॒सद्येति॑ प्र - सद्य॑ । भस्म॑ना॒ योनि᳚म् । योनि॑म॒पः । अ॒पश्च॑ । च॒ पृ॒थि॒वीम् । पृ॒थि॒वीम॑ग्ने । अ॒ग्न॒ इत्य॑ग्ने ॥ सꣳ॒॒सृज्य॑ मा॒तृभिः॑ । सꣳ॒॒सृज्येति॑ सम् - सृज्य॑ । मा॒तृभि॒स्त्वम् । मा॒तृभि॒रिति॑ मा॒तृ - भिः॒ । त्वम् ज्योति॑ष्मान् । ज्योति॑ष्मा॒न् पुनः॑ । पुन॒रा । आऽस॑दः । अ॒स॒द॒ इत्य॑सदः ॥ पुन॑रा॒सद्य॑ । आ॒सद्य॒ सद॑नम् । आ॒सद्येत्या᳚ - सद्य॑ । सद॑नम॒पः । अ॒पश्च॑ । च॒ पृ॒थि॒वीम् । पृ॒थि॒वीम॑ग्ने । अ॒ग्न॒ इत्य॑ग्ने ॥ शेषे॑ मा॒तुः । मा॒तुर् यथा᳚ । यथो॒पस्थे᳚ । उ॒पस्थे॒ऽन्तः । उ॒पस्थ॒ इत्यु॒प - स्थे॒ । अ॒न्तर॒स्याम् । अ॒स्याꣳ शि॒वत॑मः । शि॒वत॑म॒ इति॑ शि॒व - त॒मः॒ ॥ पुन॑रू॒र्जा । ऊ॒र्जा नि \newline

\textbf{Jatai Paata} \newline

1. सौष॑धी॒ रोष॑धीः॒ स सौष॑धीः । \newline
2. ओष॑धी॒ रन्वन् वोष॑धी॒ रोष॑धी॒ रनु॑ । \newline
3. अनु॑ रुद्ध्यसे रुद्ध्यसे॒ अन्वनु॑ रुद्ध्यसे । \newline
4. रु॒द्ध्य॒स॒ इति॑ रुद्ध्यसे । \newline
5. गर्भे॒ सन् थ्सन् गर्भे॒ गर्भे॒ सन्न् । \newline
6. सन् जा॑यसे जायसे॒ सन् थ्सन् जा॑यसे । \newline
7. जा॒य॒से॒ पुनः॒ पुन॑र् जायसे जायसे॒ पुनः॑ । \newline
8. पुन॒रिति॒ पुनः॑ । \newline
9. गर्भो॑ अस्यसि॒ गर्भो॒ गर्भो॑ असि । \newline
10. अ॒स्योष॑धीना॒ मोष॑धीना मस्य॒ स्योष॑धीनाम् । \newline
11. ओष॑धीना॒म् गर्भो॒ गर्भ॒ ओष॑धीना॒ मोष॑धीना॒म् गर्भः॑ । \newline
12. गर्भो॒ वन॒स्पती॑नां॒ ॅवन॒स्पती॑ना॒म् गर्भो॒ गर्भो॒ वन॒स्पती॑नाम् । \newline
13. वन॒स्पती॑ना॒मिति॒ वन॒स्पती॑नाम् । \newline
14. गर्भो॒ विश्व॑स्य॒ विश्व॑स्य॒ गर्भो॒ गर्भो॒ विश्व॑स्य । \newline
15. विश्व॑स्य भू॒तस्य॑ भू॒तस्य॒ विश्व॑स्य॒ विश्व॑स्य भू॒तस्य॑ । \newline
16. भू॒तस्याग्ने ऽग्ने॑ भू॒तस्य॑ भू॒तस्याग्ने᳚ । \newline
17. अग्ने॒ गर्भो॒ गर्भो ऽग्ने ऽग्ने॒ गर्भः॑ । \newline
18. गर्भो॑ अ॒पा म॒पाम् गर्भो॒ गर्भो॑ अ॒पाम् । \newline
19. अ॒पा म॑स्य स्य॒पा म॒पा म॑सि । \newline
20. अ॒सीत्य॑सि । \newline
21. प्र॒सद्य॒ भस्म॑ना॒ भस्म॑ना प्र॒सद्य॑ प्र॒सद्य॒ भस्म॑ना । \newline
22. प्र॒सद्येति॑ प्र - सद्य॑ । \newline
23. भस्म॑ना॒ योनिं॒ ॅयोनि॒म् भस्म॑ना॒ भस्म॑ना॒ योनि᳚म् । \newline
24. योनि॑ म॒पो अ॒पो योनिं॒ ॅयोनि॑ म॒पः । \newline
25. अ॒पश्च॑ चा॒पो अ॒पश्च॑ । \newline
26. च॒ पृ॒थि॒वीम् पृ॑थि॒वीम् च॑ च पृथि॒वीम् । \newline
27. पृ॒थि॒वी म॑ग्ने अग्ने पृथि॒वीम् पृ॑थि॒वी म॑ग्ने । \newline
28. अ॒ग्न॒ इत्य॑ग्ने । \newline
29. सꣳ॒॒सृज्य॑ मा॒तृभि॑र् मा॒तृभिः॑ सꣳ॒॒सृज्य॑ सꣳ॒॒सृज्य॑ मा॒तृभिः॑ । \newline
30. सꣳ॒॒सृज्येति॑ सं - सृज्य॑ । \newline
31. मा॒तृभि॒ स्त्वम् त्वम् मा॒तृभि॑र् मा॒तृभि॒ स्त्वम् । \newline
32. मा॒तृभि॒रिति॑ मा॒तृ - भिः॒ । \newline
33. त्वम् ज्योति॑ष्मा॒न् ज्योति॑ष्मा॒न् त्वम् त्वम् ज्योति॑ष्मान् । \newline
34. ज्योति॑ष्मा॒न् पुनः॒ पुन॒र् ज्योति॑ष्मा॒न् ज्योति॑ष्मा॒न् पुनः॑ । \newline
35. पुन॒रा पुनः॒ पुन॒रा । \newline
36. आ ऽस॑दो असद॒ आ ऽस॑दः । \newline
37. अ॒स॒द॒ इत्य॑सदः । \newline
38. पुन॑ रा॒सद्या॒ सद्य॒ पुनः॒ पुन॑ रा॒सद्य॑ । \newline
39. आ॒सद्य॒ सद॑नꣳ॒॒ सद॑न मा॒सद्या॒ सद्य॒ सद॑नम् । \newline
40. आ॒सद्येत्या᳚ - सद्य॑ । \newline
41. सद॑न म॒पो अ॒पः सद॑नꣳ॒॒ सद॑न म॒पः । \newline
42. अ॒पश्च॑ चा॒पो अ॒पश्च॑ । \newline
43. च॒ पृ॒थि॒वीम् पृ॑थि॒वीम् च॑ च पृथि॒वीम् । \newline
44. पृ॒थि॒वी म॑ग्ने अग्ने पृथि॒वीम् पृ॑थि॒वी म॑ग्ने । \newline
45. अ॒ग्न॒ इत्य॑ग्ने । \newline
46. शेषे॑ मा॒तुर् मा॒तुः शेषे॒ शेषे॑ मा॒तुः । \newline
47. मा॒तुर् यथा॒ यथा॑ मा॒तुर् मा॒तुर् यथा᳚ । \newline
48. यथो॒पस्थ॑ उ॒पस्थे॒ यथा॒ यथो॒पस्थे᳚ । \newline
49. उ॒पस्थे॒ ऽन्त र॒न्त रु॒पस्थ॑ उ॒पस्थे॒ ऽन्तः । \newline
50. उ॒पस्थ॒ इत्यु॒प - स्थे॒ । \newline
51. अ॒न्त र॒स्या म॒स्या म॒न्त र॒न्त र॒स्याम् । \newline
52. अ॒स्याꣳ शि॒वत॑मः शि॒वत॑मो अ॒स्या म॒स्याꣳ शि॒वत॑मः । \newline
53. शि॒वत॑म॒ इति॑ शि॒व - त॒मः॒ । \newline
54. पुन॑ रू॒र्जोर्जा पुनः॒ पुन॑ रू॒र्जा । \newline
55. ऊ॒र्जा नि न्यू᳚र्जोर्जा नि । \newline

\textbf{Ghana Paata } \newline

1. सौष॑धी॒ रोष॑धीः॒ स सौष॑धी॒ रन्वन् वोष॑धीः॒ स सौष॑धी॒रनु॑ । \newline
2. ओष॑धी॒ रन्वन् वोष॑धी॒ रोष॑धी॒ रनु॑ रुद्ध्यसे रुद्ध्यसे॒ अन्वोष॑धी॒ रोष॑धी॒ रनु॑ रुद्ध्यसे । \newline
3. अनु॑ रुद्ध्यसे रुद्ध्यसे॒ अन्वनु॑ रुद्ध्यसे । \newline
4. रु॒द्ध्य॒स॒ इति॑ रुद्ध्यसे । \newline
5. गर्भे॒ सन् थ्सन् गर्भे॒ गर्भे॒ सन् जा॑यसे जायसे॒ सन् गर्भे॒ गर्भे॒ सन् जा॑यसे । \newline
6. सन् जा॑यसे जायसे॒ सन् थ्सन् जा॑यसे॒ पुनः॒ पुन॑र् जायसे॒ सन् थ्सन् जा॑यसे॒ पुनः॑ । \newline
7. जा॒य॒से॒ पुनः॒ पुन॑र् जायसे जायसे॒ पुनः॑ । \newline
8. पुन॒रिति॒ पुनः॑ । \newline
9. गर्भो॑ अस्यसि॒ गर्भो॒ गर्भो॑ अ॒स्योष॑धीना॒ मोष॑धीना मसि॒ गर्भो॒ गर्भो॑ अ॒स्योष॑धीनाम् । \newline
10. अ॒स्योष॑धीना॒ मोष॑धीना मस्य॒ स्योष॑धीना॒म् गर्भो॒ गर्भ॒ ओष॑धीना मस्य॒ स्योष॑धीना॒म् गर्भः॑ । \newline
11. ओष॑धीना॒म् गर्भो॒ गर्भ॒ ओष॑धीना॒ मोष॑धीना॒म् गर्भो॒ वन॒स्पती॑नां॒ ॅवन॒स्पती॑ना॒म् गर्भ॒ ओष॑धीना॒ मोष॑धीना॒म् गर्भो॒ वन॒स्पती॑नाम् । \newline
12. गर्भो॒ वन॒स्पती॑नां॒ ॅवन॒स्पती॑ना॒म् गर्भो॒ गर्भो॒ वन॒स्पती॑नाम् । \newline
13. वन॒स्पती॑ना॒मिति॒ वन॒स्पती॑नाम् । \newline
14. गर्भो॒ विश्व॑स्य॒ विश्व॑स्य॒ गर्भो॒ गर्भो॒ विश्व॑स्य भू॒तस्य॑ भू॒तस्य॒ विश्व॑स्य॒ गर्भो॒ गर्भो॒ विश्व॑स्य भू॒तस्य॑ । \newline
15. विश्व॑स्य भू॒तस्य॑ भू॒तस्य॒ विश्व॑स्य॒ विश्व॑स्य भू॒तस्याग्ने ऽग्ने॑ भू॒तस्य॒ विश्व॑स्य॒ विश्व॑स्य भू॒तस्याग्ने᳚ । \newline
16. भू॒तस्याग्ने ऽग्ने॑ भू॒तस्य॑ भू॒तस्याग्ने॒ गर्भो॒ गर्भो ऽग्ने॑ भू॒तस्य॑ भू॒तस्याग्ने॒ गर्भः॑ । \newline
17. अग्ने॒ गर्भो॒ गर्भो ऽग्ने ऽग्ने॒ गर्भो॑ अ॒पा म॒पाम् गर्भो ऽग्ने ऽग्ने॒ गर्भो॑ अ॒पाम् । \newline
18. गर्भो॑ अ॒पा म॒पाम् गर्भो॒ गर्भो॑ अ॒पा म॑स्य स्य॒पाम् गर्भो॒ गर्भो॑ अ॒पा म॑सि । \newline
19. अ॒पा म॑स्य स्य॒पा म॒पा म॑सि । \newline
20. अ॒सीत्य॑सि । \newline
21. प्र॒सद्य॒ भस्म॑ना॒ भस्म॑ना प्र॒सद्य॑ प्र॒सद्य॒ भस्म॑ना॒ योनिं॒ ॅयोनि॒म् भस्म॑ना प्र॒सद्य॑ प्र॒सद्य॒ भस्म॑ना॒ योनि᳚म् । \newline
22. प्र॒सद्येति॑ प्र - सद्य॑ । \newline
23. भस्म॑ना॒ योनिं॒ ॅयोनि॒म् भस्म॑ना॒ भस्म॑ना॒ योनि॑ म॒पो अ॒पो योनि॒म् भस्म॑ना॒ भस्म॑ना॒ योनि॑ म॒पः । \newline
24. योनि॑ म॒पो अ॒पो योनिं॒ ॅयोनि॑ म॒पश्च॑ चा॒पो योनिं॒ ॅयोनि॑ म॒पश्च॑ । \newline
25. अ॒पश्च॑ चा॒पो अ॒पश्च॑ पृथि॒वीम् पृ॑थि॒वीम् चा॒पो अ॒पश्च॑ पृथि॒वीम् । \newline
26. च॒ पृ॒थि॒वीम् पृ॑थि॒वीम् च॑ च पृथि॒वी म॑ग्ने अग्ने पृथि॒वीम् च॑ च पृथि॒वी म॑ग्ने । \newline
27. पृ॒थि॒वी म॑ग्ने अग्ने पृथि॒वीम् पृ॑थि॒वी म॑ग्ने । \newline
28. अ॒ग्न॒ इत्य॑ग्ने । \newline
29. सꣳ॒॒सृज्य॑ मा॒तृभि॑र् मा॒तृभिः॑ सꣳ॒॒सृज्य॑ सꣳ॒॒सृज्य॑ मा॒तृभि॒ स्त्वम् त्वम् मा॒तृभिः॑ सꣳ॒॒सृज्य॑ सꣳ॒॒सृज्य॑ मा॒तृभि॒ स्त्वम् । \newline
30. सꣳ॒॒सृज्येति॑ सं - सृज्य॑ । \newline
31. मा॒तृभि॒ स्त्वम् त्वम् मा॒तृभि॑र् मा॒तृभि॒ स्त्वम् ज्योति॑ष्मा॒न् ज्योति॑ष्मा॒न् त्वम् मा॒तृभि॑र् मा॒तृभि॒ स्त्वम् ज्योति॑ष्मान् । \newline
32. मा॒तृभि॒रिति॑ मा॒तृ - भिः॒ । \newline
33. त्वम् ज्योति॑ष्मा॒न् ज्योति॑ष्मा॒न् त्वम् त्वम् ज्योति॑ष्मा॒न् पुनः॒ पुन॒र् ज्योति॑ष्मा॒न् त्वम् त्वम् ज्योति॑ष्मा॒न् पुनः॑ । \newline
34. ज्योति॑ष्मा॒न् पुनः॒ पुन॒र् ज्योति॑ष्मा॒न् ज्योति॑ष्मा॒न् पुन॒ रा पुन॒र् ज्योति॑ष्मा॒न् ज्योति॑ष्मा॒न् पुन॒ रा । \newline
35. पुन॒ रा पुनः॒ पुन॒ रा ऽस॑दो असद॒ आ पुनः॒ पुन॒ रा ऽस॑दः । \newline
36. आ ऽस॑दो असद॒ आ ऽस॑दः । \newline
37. अ॒स॒द॒ इत्य॑सदः । \newline
38. पुन॑ रा॒सद्या॒ सद्य॒ पुनः॒ पुन॑ रा॒सद्य॒ सद॑नꣳ॒॒ सद॑न मा॒सद्य॒ पुनः॒ पुन॑ रा॒सद्य॒ सद॑नम् । \newline
39. आ॒सद्य॒ सद॑नꣳ॒॒ सद॑न मा॒सद्या॒ सद्य॒ सद॑न म॒पो अ॒पः सद॑न मा॒सद्या॒ सद्य॒ सद॑न म॒पः । \newline
40. आ॒सद्येत्या᳚ - सद्य॑ । \newline
41. सद॑न म॒पो अ॒पः सद॑नꣳ॒॒ सद॑न म॒पश्च॑ चा॒पः सद॑नꣳ॒॒ सद॑न म॒पश्च॑ । \newline
42. अ॒पश्च॑ चा॒पो अ॒पश्च॑ पृथि॒वीम् पृ॑थि॒वीम् चा॒पो अ॒पश्च॑ पृथि॒वीम् । \newline
43. च॒ पृ॒थि॒वीम् पृ॑थि॒वीम् च॑ च पृथि॒वी म॑ग्ने अग्ने पृथि॒वीम् च॑ च पृथि॒वी म॑ग्ने । \newline
44. पृ॒थि॒वी म॑ग्ने अग्ने पृथि॒वीम् पृ॑थि॒वी म॑ग्ने । \newline
45. अ॒ग्न॒ इत्य॑ग्ने । \newline
46. शेषे॑ मा॒तुर् मा॒तुः शेषे॒ शेषे॑ मा॒तुर् यथा॒ यथा॑ मा॒तुः शेषे॒ शेषे॑ मा॒तुर् यथा᳚ । \newline
47. मा॒तुर् यथा॒ यथा॑ मा॒तुर् मा॒तुर् यथो॒पस्थ॑ उ॒पस्थे॒ यथा॑ मा॒तुर् मा॒तुर् यथो॒पस्थे᳚ । \newline
48. यथो॒पस्थ॑ उ॒पस्थे॒ यथा॒ यथो॒पस्थे॒ ऽन्त र॒न्त रु॒पस्थे॒ यथा॒ यथो॒पस्थे॒ ऽन्तः । \newline
49. उ॒पस्थे॒ ऽन्त र॒न्त रु॒पस्थ॑ उ॒पस्थे॒ ऽन्त र॒स्या म॒स्या म॒न्त रु॒पस्थ॑ उ॒पस्थे॒ ऽन्त र॒स्याम् । \newline
50. उ॒पस्थ॒ इत्यु॒प - स्थे॒ । \newline
51. अ॒न्त र॒स्या म॒स्या म॒न्त र॒न्त र॒स्याꣳ शि॒वत॑मः शि॒वत॑मो अ॒स्या म॒न्त र॒न्त र॒स्याꣳ शि॒वत॑मः । \newline
52. अ॒स्याꣳ शि॒वत॑मः शि॒वत॑मो अ॒स्या म॒स्याꣳ शि॒वत॑मः । \newline
53. शि॒वत॑म॒ इति॑ शि॒व - त॒मः॒ । \newline
54. पुन॑ रू॒र्जोर्जा पुनः॒ पुन॑ रू॒र्जा नि न्यू᳚र्जा पुनः॒ पुन॑ रू॒र्जा नि । \newline
55. ऊ॒र्जा नि न्यू᳚र्जोर्जा नि व॑र्तस्व वर्तस्व॒ न्यू᳚र्जोर्जा नि व॑र्तस्व । \newline
\pagebreak
\markright{ TS 4.2.3.4  \hfill https://www.vedavms.in \hfill}

\section{ TS 4.2.3.4 }

\textbf{TS 4.2.3.4 } \newline
\textbf{Samhita Paata} \newline

-नि व॑र्तस्व॒ पुन॑रग्न इ॒षा ऽऽयु॑षा । पुन॑र्नः पाहि वि॒श्वतः॑ ॥ स॒ह र॒य्या नि व॑र्त॒स्वाग्ने॒ पिन्व॑स्व॒ धार॑या । वि॒श्वफ्स्नि॑या वि॒श्वत॒स्परि॑ ॥ पुन॑स्त्वा ऽऽदि॒त्या रु॒द्रा वस॑वः॒ समि॑न्धतां॒ पुन॑र्ब्र॒ह्माणो॑ वसुनीथ य॒ज्ञिः । घृ॒तेन॒ त्वं त॒नुवो॑ वर्द्धयस्व स॒त्याः स॑न्तु॒ यज॑मानस्य॒ कामाः᳚ ॥ बोधा॑ नो अ॒स्य वच॑सो यविष्ठ॒ मꣳहि॑ष्ठस्य॒ प्रभृ॑तस्य स्वधावः । पीय॑ति त्वो॒ अनु॑ ( ) त्वो गृणाति व॒न्दारु॑स्ते त॒नुवं॑ ॅवन्दे अग्ने ॥ स बो॑धि सू॒रिर्म॒घवा॑ वसु॒दावा॒ वसु॑पतिः । यु॒यो॒द्ध्य॑स्मद् द्वेषाꣳ॑सि ॥ \newline

\textbf{Pada Paata} \newline

नीति॑ । व॒र्त॒स्व॒ । पुनः॑ । अ॒ग्ने॒ । इ॒षा । आयु॑षा ॥ पुनः॑ । नः॒ । पा॒हि॒ । वि॒श्वतः॑ ॥ स॒ह । र॒य्या । नीति॑ । व॒र्त॒स्व॒ । अग्ने᳚ । पिन्व॑स्व । धार॑या ॥ वि॒श्वफ्स्नि॒येति॑ वि॒श्व - फ्स्नि॒या॒ । वि॒श्वतः॑ । परि॑ ॥ पुनः॑ । त्वा॒ । आ॒दि॒त्याः । रु॒द्राः । वस॑वः । समिति॑ । इ॒न्ध॒ता॒म् । पुनः॑ । ब्र॒ह्माणः॑ । व॒सु॒नी॒थेति॑ वसु-नी॒थ॒ । य॒ज्ञिः ॥ घृ॒तेन॑ । त्वम् । त॒नुवः॑ । व॒द्‌र्ध॒य॒स्व॒ । स॒त्याः । स॒न्तु॒ । यज॑मानस्य । कामाः᳚ ॥ बोध॑ । नः॒ । अ॒स्य । वच॑सः । य॒वि॒ष्ठ॒ । मꣳहि॑ष्ठस्य । प्रभृ॑त॒स्येति॒ प्र - भृ॒त॒स्य॒ । स्व॒धा॒व॒ इति॑ स्वधा - वः॒ ॥ पीय॑ति । त्वः॒ । अन्विति॑ ( ) । त्वः॒ । गृ॒णा॒ति॒ । व॒न्दारुः॑ । ते॒ । त॒नुव᳚म् । व॒न्दे॒ । अ॒ग्ने॒ ॥ सः । बो॒धि॒ । सू॒रिः । म॒घवेति॑ म॒घ - वा॒ । व॒सु॒दावेति॑ वसु - दावा᳚ । वसु॑पति॒रिति॒ वसु॑ - प॒तिः॒ ॥ यु॒यो॒धि । अ॒स्मत् । द्वेषाꣳ॑सि ॥  \newline


\textbf{Krama Paata} \newline

नि व॑र्तस्व । व॒र्त॒स्व॒ पुनः॑ । पुन॑रग्ने । अ॒ग्न॒ इ॒षा । इ॒षाऽऽयु॑षा । आयु॒षेत्यायु॑षा ॥ पुन॑र् नः । नः॒ पा॒हि॒ । पा॒हि॒ वि॒श्वतः॑ । वि॒श्वत॒ इति॑ वि॒श्वतः॑ ॥ स॒ह र॒य्या । र॒य्या नि । नि व॑र्तस्व । व॒र्त॒स्वाग्ने᳚ । अग्ने॒ पिन्व॑स्व । पिन्व॑स्व॒ धार॑या । धार॒येति॒ धार॑या ॥ वि॒श्वफ्स्नि॑या वि॒श्वतः॑ । वि॒श्वफ्स्नि॒येति॑ वि॒श्व - फ्स्नि॒या॒ । वि॒श्वत॒स्परि॑ । परीति॒ परि॑ ॥ पुन॑स्त्वा । त्वा॒ ऽऽदि॒त्याः । आ॒दि॒त्या रु॒द्राः । रु॒द्रा वस॑वः । वस॑वः॒ सम् । समि॑न्धताम् । इ॒न्ध॒ता॒म् पुनः॑ । पुन॑र् ब्र॒ह्माणः॑ । ब्र॒ह्माणो॑ वसुनीथ । व॒सु॒नी॒थ॒ य॒ज्ञिः । व॒सु॒नी॒थेति॑ वसु - नी॒थ॒ । य॒ज्ञिरिति॑ य॒ज्ञिः ॥ घृ॒तेन॒ त्वम् । त्वम् त॒नुवः॑ । त॒नुवो॑ वर्द्धयस्व । व॒र्द्ध॒य॒स्व॒ स॒त्याः । स॒त्याः स॑न्तु । स॒न्तु॒ यज॑मानस्य । यज॑मानस्य॒ कामाः᳚ । कामा॒ इति॒ कामाः᳚ ॥ बोधा॑ नः । नो॒ अ॒स्य । अ॒स्य वच॑सः । वच॑सो यविष्ठ । य॒वि॒ष्ठ॒ मꣳहि॑ष्ठस्य । मꣳहि॑ष्ठस्य॒ प्रभृ॑तस्य । प्रभृ॑तस्य स्वधावः । प्रभृ॑त॒स्येति॒ प्र - भृ॒त॒स्य॒ । स्व॒धा॒व॒ इति॑ स्वधा - वः॒ ॥ पीय॑ति त्वः । त्वो॒ अनु॑ ( ) । अनु॑ त्वः । त्वो॒ गृ॒णा॒ति॒ । गृ॒णा॒ति॒ व॒न्दारुः॑ । व॒न्दारु॑स्ते । ते॒ त॒नुव᳚म् । त॒नुवं॑ ॅवन्दे । व॒न्दे॒ अ॒ग्ने॒ । अ॒ग्न॒ इत्य॑ग्ने ॥ स बो॑धि । बो॒धि॒ सू॒रिः । सू॒रिर् म॒घवा᳚ । म॒घवा॑ वसु॒दावा᳚ । म॒घवेति॑ म॒घ - वा॒ । व॒सु॒दावा॒ वसु॑पतिः । व॒सु॒दावेति॑ वसु - दावा᳚ । वसु॑पति॒रिति॒ वसु॑ - प॒तिः॒ ॥ यु॒यो॒द्ध्य॑स्मत् । अ॒स्मद् द्वेषाꣳ॑सि । द्वेषाꣳ॒॒सीति॒ द्वेषाꣳ॑सि । \newline

\textbf{Jatai Paata} \newline

1. नि व॑र्तस्व वर्तस्व॒ नि नि व॑र्तस्व । \newline
2. व॒र्त॒स्व॒ पुनः॒ पुन॑र् वर्तस्व वर्तस्व॒ पुनः॑ । \newline
3. पुन॑ रग्ने ऽग्ने॒ पुनः॒ पुन॑ रग्ने । \newline
4. अ॒ग्न॒ इ॒षेषा ऽग्ने᳚ ऽग्न इ॒षा । \newline
5. इ॒षा ऽऽयु॒षा ऽऽयु॑षे॒ षेषा ऽऽयु॑षा । \newline
6. आयु॒षेत्यायु॑षा । \newline
7. पुन॑र् नो नः॒ पुनः॒ पुन॑र् नः । \newline
8. नः॒ पा॒हि॒ पा॒हि॒ नो॒ नः॒ पा॒हि॒ । \newline
9. पा॒हि॒ वि॒श्वतो॑ वि॒श्वत॑ स्पाहि पाहि वि॒श्वतः॑ । \newline
10. वि॒श्वत॒ इति॑ वि॒श्वतः॑ । \newline
11. स॒ह र॒य्या र॒य्या स॒ह स॒ह र॒य्या । \newline
12. र॒य्या नि नि र॒य्या र॒य्या नि । \newline
13. नि व॑र्तस्व वर्तस्व॒ नि नि व॑र्तस्व । \newline
14. व॒र्त॒स्वाग्ने ऽग्ने॑ वर्तस्व वर्त॒स्वाग्ने᳚ । \newline
15. अग्ने॒ पिन्व॑स्व॒ पिन्व॒स्वाग्ने ऽग्ने॒ पिन्व॑स्व । \newline
16. पिन्व॑स्व॒ धार॑या॒ धार॑या॒ पिन्व॑स्व॒ पिन्व॑स्व॒ धार॑या । \newline
17. धार॒येति॒ धार॑या । \newline
18. वि॒श्वफ्स्नि॑या वि॒श्वतो॑ वि॒श्वतो॑ वि॒श्वफ्स्नि॑या वि॒श्वफ्स्नि॑या वि॒श्वतः॑ । \newline
19. वि॒श्वफ्स्नि॒येति॑ वि॒श्व - फ्स्नि॒या॒ । \newline
20. वि॒श्वत॒ स्परि॒ परि॑ वि॒श्वतो॑ वि॒श्वत॒ स्परि॑ । \newline
21. परीति॒ परि॑ । \newline
22. पुन॑ स्त्वा त्वा॒ पुनः॒ पुन॑ स्त्वा । \newline
23. त्वा॒ ऽऽदि॒त्या आ॑दि॒त्या स्त्वा᳚ त्वा ऽऽदि॒त्याः । \newline
24. आ॒दि॒त्या रु॒द्रा रु॒द्रा आ॑दि॒त्या आ॑दि॒त्या रु॒द्राः । \newline
25. रु॒द्रा वस॑वो॒ वस॑वो रु॒द्रा रु॒द्रा वस॑वः । \newline
26. वस॑वः॒ सꣳ सं ॅवस॑वो॒ वस॑वः॒ सम् । \newline
27. स मि॑न्धता मिन्धताꣳ॒॒ सꣳ स मि॑न्धताम् । \newline
28. इ॒न्ध॒ता॒म् पुनः॒ पुन॑ रिन्धता मिन्धता॒म् पुनः॑ । \newline
29. पुन॑र् ब्र॒ह्माणो᳚ ब्र॒ह्माणः॒ पुनः॒ पुन॑र् ब्र॒ह्माणः॑ । \newline
30. ब्र॒ह्माणो॑ वसुनीथ वसुनीथ ब्र॒ह्माणो᳚ ब्र॒ह्माणो॑ वसुनीथ । \newline
31. व॒सु॒नी॒थ॒ य॒ज्ञिर् य॒ज्ञिर् व॑सुनीथ वसुनीथ य॒ज्ञिः । \newline
32. व॒सु॒नी॒थेति॑ वसु - नी॒थ॒ । \newline
33. य॒ज्ञिरिति॑ य॒ज्ञिः । \newline
34. घृ॒तेन॒ त्वम् त्वम् घृ॒तेन॑ घृ॒तेन॒ त्वम् । \newline
35. त्वम् त॒नुव॑ स्त॒नुव॒ स्त्वम् त्वम् त॒नुवः॑ । \newline
36. त॒नुवो॑ वर्द्धयस्व वर्द्धयस्व त॒नुव॑ स्त॒नुवो॑ वर्द्धयस्व । \newline
37. व॒र्द्ध॒य॒स्व॒ स॒त्याः स॒त्या व॑र्द्धयस्व वर्द्धयस्व स॒त्याः । \newline
38. स॒त्याः स॑न्तु सन्तु स॒त्याः स॒त्याः स॑न्तु । \newline
39. स॒न्तु॒ यज॑मानस्य॒ यज॑मानस्य सन्तु सन्तु॒ यज॑मानस्य । \newline
40. यज॑मानस्य॒ कामाः॒ कामा॒ यज॑मानस्य॒ यज॑मानस्य॒ कामाः᳚ । \newline
41. कामा॒ इति॒ कामाः᳚ । \newline
42. बोधा॑ नो नो॒ बोध॒ बोधा॑ नः । \newline
43. नो॒ अ॒स्यास्य नो॑ नो अ॒स्य । \newline
44. अ॒स्य वच॑सो॒ वच॑सो अ॒स्यास्य वच॑सः । \newline
45. वच॑सो यविष्ठ यविष्ठ॒ वच॑सो॒ वच॑सो यविष्ठ । \newline
46. य॒वि॒ष्ठ॒ मꣳहि॑ष्ठस्य॒ मꣳहि॑ष्ठस्य यविष्ठ यविष्ठ॒ मꣳहि॑ष्ठस्य । \newline
47. मꣳहि॑ष्ठस्य॒ प्रभृ॑तस्य॒ प्रभृ॑तस्य॒ मꣳहि॑ष्ठस्य॒ मꣳहि॑ष्ठस्य॒ प्रभृ॑तस्य । \newline
48. प्रभृ॑तस्य स्वधावः स्वधावः॒ प्रभृ॑तस्य॒ प्रभृ॑तस्य स्वधावः । \newline
49. प्रभृ॑त॒स्येति॒ प्र - भृ॒त॒स्य॒ । \newline
50. स्व॒धा॒व॒ इति॑ स्वधा - वः॒ । \newline
51. पीय॑ति त्वस्त्वः॒ पीय॑ति॒ पीय॑ति त्वः । \newline
52. त्वो॒ अन्वनु॑ त्व स्त्वो॒ अनु॑ । \newline
53. अनु॑ त्व स्त्वो॒ अन्वनु॑ त्वः । \newline
54. त्वो॒ गृ॒णा॒ति॒ गृ॒णा॒ति॒ त्व॒ स्त्वो॒ गृ॒णा॒ति॒ । \newline
55. गृ॒णा॒ति॒ व॒न्दारु॑र् व॒न्दारु॑र् गृणाति गृणाति व॒न्दारुः॑ । \newline
56. व॒न्दारु॑ स्ते ते व॒न्दारु॑र् व॒न्दारु॑ स्ते । \newline
57. ते॒ त॒नुव॑म् त॒नुव॑म् ते ते त॒नुव᳚म् । \newline
58. त॒नुवं॑ ॅवन्दे वन्दे त॒नुव॑म् त॒नुवं॑ ॅवन्दे । \newline
59. व॒न्दे॒ अ॒ग्ने॒ अ॒ग्ने॒ व॒न्दे॒ व॒न्दे॒ अ॒ग्ने॒ । \newline
60. अ॒ग्न॒ इत्य॑ग्ने । \newline
61. स बो॑धि बोधि॒ स स बो॑धि । \newline
62. बो॒धि॒ सू॒रिः सू॒रिर् बो॑धि बोधि सू॒रिः । \newline
63. सू॒रिर् म॒घवा॑ म॒घवा॑ सू॒रिः सू॒रिर् म॒घवा᳚ । \newline
64. म॒घवा॑ वसु॒दावा॑ वसु॒दावा॑ म॒घवा॑ म॒घवा॑ वसु॒दावा᳚ । \newline
65. म॒घवेति॑ म॒घ - वा॒ । \newline
66. व॒सु॒दावा॒ वसु॑पति॒र् वसु॑पतिर् वसु॒दावा॑ वसु॒दावा॒ वसु॑पतिः । \newline
67. व॒सु॒दावेति॑ वसु - दावा᳚ । \newline
68. वसु॑पति॒रिति॒ वसु॑ - प॒तिः॒ । \newline
69. यु॒यो॒ ध्य॑स्म द॒स्मद् यु॑यो॒धि यु॑यो॒ ध्य॑स्मत् । \newline
70. अ॒स्मद् द्वेषाꣳ॑सि॒ द्वेषाꣳ॑ स्य॒स्म द॒स्मद् द्वेषाꣳ॑सि । \newline
71. द्वेषाꣳ॒॒सीति॒ द्वेषाꣳ॑सि । \newline

\textbf{Ghana Paata } \newline

1. नि व॑र्तस्व वर्तस्व॒ नि नि व॑र्तस्व॒ पुनः॒ पुन॑र् वर्तस्व॒ नि नि व॑र्तस्व॒ पुनः॑ । \newline
2. व॒र्त॒स्व॒ पुनः॒ पुन॑र् वर्तस्व वर्तस्व॒ पुन॑ रग्ने ऽग्ने॒ पुन॑र् वर्तस्व वर्तस्व॒ पुन॑ रग्ने । \newline
3. पुन॑ रग्ने ऽग्ने॒ पुनः॒ पुन॑ रग्न इ॒षेषा ऽग्ने॒ पुनः॒ पुन॑ रग्न इ॒षा । \newline
4. अ॒ग्न॒ इ॒षेषा ऽग्ने᳚ ऽग्न इ॒षा ऽऽयु॒षा ऽऽयु॑षे॒षा ऽग्ने᳚ ऽग्न इ॒षा ऽऽयु॑षा । \newline
5. इ॒षा ऽऽयु॒षा ऽऽयु॑षे॒ षेषा ऽऽयु॑षा । \newline
6. आयु॒षेत्यायु॑षा । \newline
7. पुन॑र् नो नः॒ पुनः॒ पुन॑र् नः पाहि पाहि नः॒ पुनः॒ पुन॑र् नः पाहि । \newline
8. नः॒ पा॒हि॒ पा॒हि॒ नो॒ नः॒ पा॒हि॒ वि॒श्वतो॑ वि॒श्वत॑ स्पाहि नो नः पाहि वि॒श्वतः॑ । \newline
9. पा॒हि॒ वि॒श्वतो॑ वि॒श्वत॑ स्पाहि पाहि वि॒श्वतः॑ । \newline
10. वि॒श्वत॒ इति॑ वि॒श्वतः॑ । \newline
11. स॒ह र॒य्या र॒य्या स॒ह स॒ह र॒य्या नि नि र॒य्या स॒ह स॒ह र॒य्या नि । \newline
12. र॒य्या नि नि र॒य्या र॒य्या नि व॑र्तस्व वर्तस्व॒ नि र॒य्या र॒य्या नि व॑र्तस्व । \newline
13. नि व॑र्तस्व वर्तस्व॒ नि नि व॑र्त॒स्वाग्ने ऽग्ने॑ वर्तस्व॒ नि नि व॑र्त॒स्वाग्ने᳚ । \newline
14. व॒र्त॒स्वाग्ने ऽग्ने॑ वर्तस्व वर्त॒स्वाग्ने॒ पिन्व॑स्व॒ पिन्व॒स्वाग्ने॑ वर्तस्व वर्त॒स्वाग्ने॒ पिन्व॑स्व । \newline
15. अग्ने॒ पिन्व॑स्व॒ पिन्व॒स्वाग्ने ऽग्ने॒ पिन्व॑स्व॒ धार॑या॒ धार॑या॒ पिन्व॒स्वाग्ने ऽग्ने॒ पिन्व॑स्व॒ धार॑या । \newline
16. पिन्व॑स्व॒ धार॑या॒ धार॑या॒ पिन्व॑स्व॒ पिन्व॑स्व॒ धार॑या । \newline
17. धार॒येति॒ धार॑या । \newline
18. वि॒श्वफ्स्नि॑या वि॒श्वतो॑ वि॒श्वतो॑ वि॒श्वफ्स्नि॑या वि॒श्वफ्स्नि॑या वि॒श्वत॒ स्परि॒ परि॑ वि॒श्वतो॑ वि॒श्वफ्स्नि॑या वि॒श्वफ्स्नि॑या वि॒श्वत॒ स्परि॑ । \newline
19. वि॒श्वफ्स्नि॒येति॑ वि॒श्व - फ्स्नि॒या॒ । \newline
20. वि॒श्वत॒ स्परि॒ परि॑ वि॒श्वतो॑ वि॒श्वत॒ स्परि॑ । \newline
21. परीति॒ परि॑ । \newline
22. पुन॑ स्त्वा त्वा॒ पुनः॒ पुन॑ स्त्वा ऽऽदि॒त्या आ॑दि॒त्या स्त्वा॒ पुनः॒ पुन॑ स्त्वा ऽऽदि॒त्याः । \newline
23. त्वा॒ ऽऽदि॒त्या आ॑दि॒त्या स्त्वा᳚ त्वा ऽऽदि॒त्या रु॒द्रा रु॒द्रा आ॑दि॒त्या स्त्वा᳚ त्वा ऽऽदि॒त्या रु॒द्राः । \newline
24. आ॒दि॒त्या रु॒द्रा रु॒द्रा आ॑दि॒त्या आ॑दि॒त्या रु॒द्रा वस॑वो॒ वस॑वो रु॒द्रा आ॑दि॒त्या आ॑दि॒त्या रु॒द्रा वस॑वः । \newline
25. रु॒द्रा वस॑वो॒ वस॑वो रु॒द्रा रु॒द्रा वस॑वः॒ सꣳ सं ॅवस॑वो रु॒द्रा रु॒द्रा वस॑वः॒ सम् । \newline
26. वस॑वः॒ सꣳ सं ॅवस॑वो॒ वस॑वः॒ स मि॑न्धता मिन्धताꣳ॒॒ सं ॅवस॑वो॒ वस॑वः॒ स मि॑न्धताम् । \newline
27. स मि॑न्धता मिन्धताꣳ॒॒ सꣳ स मि॑न्धता॒म् पुनः॒ पुन॑ रिन्धताꣳ॒॒ सꣳ स मि॑न्धता॒म् पुनः॑ । \newline
28. इ॒न्ध॒ता॒म् पुनः॒ पुन॑ रिन्धता मिन्धता॒म् पुन॑र् ब्र॒ह्माणो᳚ ब्र॒ह्माणः॒ पुन॑ रिन्धता मिन्धता॒म् पुन॑र् ब्र॒ह्माणः॑ । \newline
29. पुन॑र् ब्र॒ह्माणो᳚ ब्र॒ह्माणः॒ पुनः॒ पुन॑र् ब्र॒ह्माणो॑ वसुनीथ वसुनीथ ब्र॒ह्माणः॒ पुनः॒ पुन॑र् ब्र॒ह्माणो॑ वसुनीथ । \newline
30. ब्र॒ह्माणो॑ वसुनीथ वसुनीथ ब्र॒ह्माणो᳚ ब्र॒ह्माणो॑ वसुनीथ य॒ज्ञिर् य॒ज्ञिर् व॑सुनीथ ब्र॒ह्माणो᳚ ब्र॒ह्माणो॑ वसुनीथ य॒ज्ञिः । \newline
31. व॒सु॒नी॒थ॒ य॒ज्ञिर् य॒ज्ञिर् व॑सुनीथ वसुनीथ य॒ज्ञिः । \newline
32. व॒सु॒नी॒थेति॑ वसु - नी॒थ॒ । \newline
33. य॒ज्ञिरिति॑ य॒ज्ञिः । \newline
34. घृ॒तेन॒ त्वम् त्वम् घृ॒तेन॑ घृ॒तेन॒ त्वम् त॒नुव॑ स्त॒नुव॒ स्त्वम् घृ॒तेन॑ घृ॒तेन॒ त्वम् त॒नुवः॑ । \newline
35. त्वम् त॒नुव॑ स्त॒नुव॒ स्त्वम् त्वम् त॒नुवो॑ वर्द्धयस्व वर्द्धयस्व त॒नुव॒ स्त्वम् त्वम् त॒नुवो॑ वर्द्धयस्व । \newline
36. त॒नुवो॑ वर्द्धयस्व वर्द्धयस्व त॒नुव॑ स्त॒नुवो॑ वर्द्धयस्व स॒त्याः स॒त्या व॑र्द्धयस्व त॒नुव॑ स्त॒नुवो॑ वर्द्धयस्व स॒त्याः । \newline
37. व॒र्द्ध॒य॒स्व॒ स॒त्याः स॒त्या व॑र्द्धयस्व वर्द्धयस्व स॒त्याः स॑न्तु सन्तु स॒त्या व॑र्द्धयस्व वर्द्धयस्व स॒त्याः स॑न्तु । \newline
38. स॒त्याः स॑न्तु सन्तु स॒त्याः स॒त्याः स॑न्तु॒ यज॑मानस्य॒ यज॑मानस्य सन्तु स॒त्याः स॒त्याः स॑न्तु॒ यज॑मानस्य । \newline
39. स॒न्तु॒ यज॑मानस्य॒ यज॑मानस्य सन्तु सन्तु॒ यज॑मानस्य॒ कामाः॒ कामा॒ यज॑मानस्य सन्तु सन्तु॒ यज॑मानस्य॒ कामाः᳚ । \newline
40. यज॑मानस्य॒ कामाः॒ कामा॒ यज॑मानस्य॒ यज॑मानस्य॒ कामाः᳚ । \newline
41. कामा॒ इति॒ कामाः᳚ । \newline
42. बोधा॑ नो नो॒ बोध॒ बोधा॑ नो अ॒स्यास्य नो॒ बोध॒ बोधा॑ नो अ॒स्य । \newline
43. नो॒ अ॒स्यास्य नो॑ नो अ॒स्य वच॑सो॒ वच॑सो अ॒स्य नो॑ नो अ॒स्य वच॑सः । \newline
44. अ॒स्य वच॑सो॒ वच॑सो अ॒स्यास्य वच॑सो यविष्ठ यविष्ठ॒ वच॑सो अ॒स्यास्य वच॑सो यविष्ठ । \newline
45. वच॑सो यविष्ठ यविष्ठ॒ वच॑सो॒ वच॑सो यविष्ठ॒ मꣳहि॑ष्ठस्य॒ मꣳहि॑ष्ठस्य यविष्ठ॒ वच॑सो॒ वच॑सो यविष्ठ॒ मꣳहि॑ष्ठस्य । \newline
46. य॒वि॒ष्ठ॒ मꣳहि॑ष्ठस्य॒ मꣳहि॑ष्ठस्य यविष्ठ यविष्ठ॒ मꣳहि॑ष्ठस्य॒ प्रभृ॑तस्य॒ प्रभृ॑तस्य॒ मꣳहि॑ष्ठस्य यविष्ठ यविष्ठ॒ मꣳहि॑ष्ठस्य॒ प्रभृ॑तस्य । \newline
47. मꣳहि॑ष्ठस्य॒ प्रभृ॑तस्य॒ प्रभृ॑तस्य॒ मꣳहि॑ष्ठस्य॒ मꣳहि॑ष्ठस्य॒ प्रभृ॑तस्य स्वधावः स्वधावः॒ प्रभृ॑तस्य॒ मꣳहि॑ष्ठस्य॒ मꣳहि॑ष्ठस्य॒ प्रभृ॑तस्य स्वधावः । \newline
48. प्रभृ॑तस्य स्वधावः स्वधावः॒ प्रभृ॑तस्य॒ प्रभृ॑तस्य स्वधावः । \newline
49. प्रभृ॑त॒स्येति॒ प्र - भृ॒त॒स्य॒ । \newline
50. स्व॒धा॒व॒ इति॑ स्वधा - वः॒ । \newline
51. पीय॑ति त्व स्त्वः॒ पीय॑ति॒ पीय॑ति त्वो॒ अन्वनु॑ त्वः॒ पीय॑ति॒ पीय॑ति त्वो॒ अनु॑ । \newline
52. त्वो॒ अन्वनु॑ त्व स्त्वो॒ अनु॑ त्वस्त्वो॒ अनु॑ त्व स्त्वो॒ अनु॑ त्वः । \newline
53. अनु॑ त्व स्त्वो॒ अन्वनु॑ त्वो गृणाति गृणाति त्वो॒ अन्वनु॑ त्वो गृणाति । \newline
54. त्वो॒ गृ॒णा॒ति॒ गृ॒णा॒ति॒ त्व॒ स्त्वो॒ गृ॒णा॒ति॒ व॒न्दारु॑र् व॒न्दारु॑र् गृणाति त्व स्त्वो गृणाति व॒न्दारुः॑ । \newline
55. गृ॒णा॒ति॒ व॒न्दारु॑र् व॒न्दारु॑र् गृणाति गृणाति व॒न्दारु॑ स्ते ते व॒न्दारु॑र् गृणाति गृणाति व॒न्दारु॑ स्ते । \newline
56. व॒न्दारु॑ स्ते ते व॒न्दारु॑र् व॒न्दारु॑ स्ते त॒नुव॑म् त॒नुव॑म् ते व॒न्दारु॑र् व॒न्दारु॑ स्ते त॒नुव᳚म् । \newline
57. ते॒ त॒नुव॑म् त॒नुव॑म् ते ते त॒नुवं॑ ॅवन्दे वन्दे त॒नुव॑म् ते ते त॒नुवं॑ ॅवन्दे । \newline
58. त॒नुवं॑ ॅवन्दे वन्दे त॒नुव॑म् त॒नुवं॑ ॅवन्दे अग्ने अग्ने वन्दे त॒नुव॑म् त॒नुवं॑ ॅवन्दे अग्ने । \newline
59. व॒न्दे॒ अ॒ग्ने॒ अ॒ग्ने॒ व॒न्दे॒ व॒न्दे॒ अ॒ग्ने॒ । \newline
60. अ॒ग्न॒ इत्य॑ग्ने । \newline
61. स बो॑धि बोधि॒ स स बो॑धि सू॒रिः सू॒रिर् बो॑धि॒ स स बो॑धि सू॒रिः । \newline
62. बो॒धि॒ सू॒रिः सू॒रिर् बो॑धि बोधि सू॒रिर् म॒घवा॑ म॒घवा॑ सू॒रिर् बो॑धि बोधि सू॒रिर् म॒घवा᳚ । \newline
63. सू॒रिर् म॒घवा॑ म॒घवा॑ सू॒रिः सू॒रिर् म॒घवा॑ वसु॒दावा॑ वसु॒दावा॑ म॒घवा॑ सू॒रिः सू॒रिर् म॒घवा॑ वसु॒दावा᳚ । \newline
64. म॒घवा॑ वसु॒दावा॑ वसु॒दावा॑ म॒घवा॑ म॒घवा॑ वसु॒दावा॒ वसु॑पति॒र् वसु॑पतिर् वसु॒दावा॑ म॒घवा॑ म॒घवा॑ वसु॒दावा॒ वसु॑पतिः । \newline
65. म॒घवेति॑ म॒घ - वा॒ । \newline
66. व॒सु॒दावा॒ वसु॑पति॒र् वसु॑पतिर् वसु॒दावा॑ वसु॒दावा॒ वसु॑पतिः । \newline
67. व॒सु॒दावेति॑ वसु - दावा᳚ । \newline
68. वसु॑पति॒रिति॒ वसु॑ - प॒तिः॒ । \newline
69. यु॒यो॒ध्य॑स्म द॒स्मद् यु॑यो॒धि यु॑यो॒ ध्य॑स्मद् द्वेषाꣳ॑सि॒ द्वेषाꣳ॑ स्य॒स्मद् यु॑यो॒धि यु॑यो॒ ध्य॑स्मद् द्वेषाꣳ॑सि । \newline
70. अ॒स्मद् द्वेषाꣳ॑सि॒ द्वेषाꣳ॑ स्य॒स्म द॒स्मद् द्वेषाꣳ॑सि । \newline
71. द्वेषाꣳ॒॒सीति॒ द्वेषाꣳ॑सि । \newline
\pagebreak
\markright{ TS 4.2.4.1  \hfill https://www.vedavms.in \hfill}

\section{ TS 4.2.4.1 }

\textbf{TS 4.2.4.1 } \newline
\textbf{Samhita Paata} \newline

अपे॑त॒ वीत॒ वि च॑ सर्प॒तातो॒ येऽत्र॒ स्थ पु॑रा॒णा ये च॒ नूत॑नाः । अदा॑दि॒दं ॅय॒मो॑ऽव॒सानं॑ पृथि॒व्या अक्र॑न्नि॒मं पि॒तरो॑ लो॒कम॑स्मै ॥ अ॒ग्नेर्भस्मा᳚स्य॒ग्नेः पुरी॑षमसि स॒ज्ञांन॑मसि काम॒धर॑णं॒ मयि॑ ते काम॒धर॑णं भूयात् ॥ सं ॅया वः॑ प्रि॒यास्त॒नुवः॒ सं प्रि॒या हृद॑यानि वः । आ॒त्मा वो॑ अस्तु॒ - [  ] \newline

\textbf{Pada Paata} \newline

अपेति॑ । इ॒त॒ । वीति॑ । इ॒त॒ । वीति॑ । च॒ । स॒र्प॒त॒ । अतः॑ । ये । अत्र॑ । स्थ । पु॒रा॒णाः । ये । च॒ । नूत॑नाः ॥ अदा᳚त् । इ॒दम् । य॒मः । अ॒व॒सान॒मित्य॑व - सान᳚म् । पृ॒थि॒व्याः । अक्रन्न्॑ । इ॒मम् । पि॒तरः॑ । लो॒कम् । अ॒स्मै॒ ॥ अ॒ग्नेः । भस्म॑ । अ॒सि॒ । अ॒ग्नः॒ । पुरी॑षम् । अ॒सि॒ । स॒ज्ञांन॒मिति॑ सं - ज्ञान᳚म् । अ॒सि॒ । का॒म॒धर॑ण॒मिति॑ काम - धर॑णम् । मयि॑ । ते॒ । का॒म॒धर॑ण॒मिति॑ काम - धर॑णम् । भू॒या॒त् ॥ समिति॑ । याः । वः॒ । प्रि॒याः । त॒नुवः॑ । समिति॑ । प्रि॒या । हृद॑यानि । वः॒ ॥ आ॒त्मा । वः॒ । अ॒स्तु॒ ।  \newline


\textbf{Krama Paata} \newline

अपे॑त । इ॒त॒ वि । वीत॑ । इ॒त॒ वि । वि च॑ । च॒ स॒र्प॒त॒ । स॒र्प॒तातः॑ । अतो॒ ये । येऽत्र॑ । अत्र॒ स्थ । स्थ पु॑रा॒णाः । पु॒रा॒णा ये । ये च॑ । च॒ नूत॑नाः । नूत॑ना॒ इति॒ नूत॑नाः ॥ अदा॑दि॒दम् । इ॒दं ॅय॒मः । य॒मो॑ऽव॒सान᳚म् । अ॒व॒सान॑म् पृथि॒व्याः । अ॒व॒सान॒मित्य॑व - सान᳚म् । पृ॒थि॒व्या अक्रन्न्॑ । अक्र॑न्नि॒मम् । इ॒मम् पि॒तरः॑ । पि॒तरो॑ लो॒कम् । लो॒कम॑स्मै । अ॒स्मा॒ इत्य॑स्मै ॥ अ॒ग्नेर् भस्म॑ । भस्मा॑सि । अ॒स्य॒ग्नेः । अ॒ग्नेः पुरी॑षम् । पुरी॑षमसि । अ॒सि॒ स॒म्.(2)ज्ञान᳚म् । स॒म्.(2)ज्ञान॑मसि । स॒म्.(2)ज्ञान॒मिति॑ सम्. - ज्ञान᳚म् । अ॒सि॒ का॒म॒धर॑णम् । का॒म॒धर॑ण॒म् मयि॑ । का॒म॒धर॑ण॒मिति॑ काम - धर॑णम् । मयि॑ ते । ते॒ का॒म॒धर॑णम् । का॒म॒धर॑णम् भूयात् । का॒म॒धर॑ण॒मिति॑ काम - धर॑णम् । भू॒या॒दिति॑ भूयात् ॥ सं ॅयाः । या वः॑ । वः॒ प्रि॒याः । प्रि॒यास्त॒नुवः॑ । त॒नुवः॒ सम् । सम् प्रि॒या । प्रि॒या हृद॑यानि । हृद॑यानि वः । व॒ इति॑ वः ॥ आ॒त्मा वः॑ । वो॒ अ॒स्तु॒ । अ॒स्तु॒ सम्प्रि॑यः \newline

\textbf{Jatai Paata} \newline

1. अपे॑ते॒ तापा पे॑त । \newline
2. इ॒त॒ वि वीते॑त॒ वि । \newline
3. वीते॑त॒ वि वीत॑ । \newline
4. इ॒त॒ वि वीते॑त॒ वि । \newline
5. वि च॑ च॒ वि वि च॑ । \newline
6. च॒ स॒र्प॒त॒ स॒र्प॒त॒ च॒ च॒ स॒र्प॒त॒ । \newline
7. स॒र्प॒तातो॒ अतः॑ सर्पत सर्प॒तातः॑ । \newline
8. अतो॒ ये ये अतो॒ अतो॒ ये । \newline
9. ये ऽत्रात्र॒ ये ये ऽत्र॑ । \newline
10. अत्र॒ स्थ स्था त्रात्र॒ स्थ । \newline
11. स्थ पु॑रा॒णाः पु॑रा॒णाः स्थ स्थ पु॑रा॒णाः । \newline
12. पु॒रा॒णा ये ये पु॑रा॒णाः पु॑रा॒णा ये । \newline
13. ये च॑ च॒ ये ये च॑ । \newline
14. च॒ नूत॑ना॒ नूत॑नाश्च च॒ नूत॑नाः । \newline
15. नूत॑ना॒ इति॒ नूत॑नाः । \newline
16. अदा॑ दि॒द मि॒द मदा॒ ददा॑ दि॒दम् । \newline
17. इ॒दं ॅय॒मो य॒म इ॒द मि॒दं ॅय॒मः । \newline
18. य॒मो॑ ऽव॒सान॑ मव॒सानं॑ ॅय॒मो य॒मो॑ ऽव॒सान᳚म् । \newline
19. अ॒व॒सान॑म् पृथि॒व्याः पृ॑थि॒व्या अ॑व॒सान॑ मव॒सान॑म् पृथि॒व्याः । \newline
20. अ॒व॒सान॒मित्य॑व - सान᳚म् । \newline
21. पृ॒थि॒व्या अक्र॒न् नक्र॑न् पृथि॒व्याः पृ॑थि॒व्या अक्रन्न्॑ । \newline
22. अक्र॑न् नि॒म मि॒म मक्र॒न् नक्र॑न् नि॒मम् । \newline
23. इ॒मम् पि॒तरः॑ पि॒तर॑ इ॒म मि॒मम् पि॒तरः॑ । \newline
24. पि॒तरो॑ लो॒कम् ॅलो॒कम् पि॒तरः॑ पि॒तरो॑ लो॒कम् । \newline
25. लो॒क म॑स्मा अस्मै लो॒कम् ॅलो॒क म॑स्मै । \newline
26. अ॒स्मा॒ इत्य॑स्मै । \newline
27. अ॒ग्नेर् भस्म॒ भस्मा॒ग्ने र॒ग्नेर् भस्म॑ । \newline
28. भस्मा᳚ स्यसि॒ भस्म॒ भस्मा॑सि । \newline
29. अ॒स्य॒ग्ने र॒ग्ने र॑स्यस्य॒ग्नेः । \newline
30. अ॒ग्नेः पुरी॑ष॒म् पुरी॑ष म॒ग्ने र॒ग्नेः पुरी॑षम् । \newline
31. पुरी॑ष मस्यसि॒ पुरी॑ष॒म् पुरी॑ष मसि । \newline
32. अ॒सि॒ सं॒.ज्ञानꣳ॑ सं॒.ज्ञान॑ मस्यसि सं॒.ज्ञान᳚म् । \newline
33. सं॒.ज्ञान॑ मस्यसि सं॒.ज्ञानꣳ॑ सं॒.ज्ञान॑ मसि । \newline
34. सं॒.ज्ञान॒मिति॑ सं - ज्ञान᳚म् । \newline
35. अ॒सि॒ का॒म॒धर॑णम् काम॒धर॑ण मस्यसि काम॒धर॑णम् । \newline
36. का॒म॒धर॑ण॒म् मयि॒ मयि॑ काम॒धर॑णम् काम॒धर॑ण॒म् मयि॑ । \newline
37. का॒म॒धर॑ण॒मिति॑ काम - धर॑णम् । \newline
38. मयि॑ ते ते॒ मयि॒ मयि॑ ते । \newline
39. ते॒ का॒म॒धर॑णम् काम॒धर॑णम् ते ते काम॒धर॑णम् । \newline
40. का॒म॒धर॑णम् भूयाद् भूयात् काम॒धर॑णम् काम॒धर॑णम् भूयात् । \newline
41. का॒म॒धर॑ण॒मिति॑ काम - धर॑णम् । \newline
42. भू॒या॒दिति॑ भूयात् । \newline
43. सं ॅया याः सꣳ सं ॅयाः । \newline
44. या वो॑ वो॒ या या वः॑ । \newline
45. वः॒ प्रि॒याः प्रि॒या वो॑ वः प्रि॒याः । \newline
46. प्रि॒या स्त॒नुव॑ स्त॒नुवः॑ प्रि॒याः प्रि॒या स्त॒नुवः॑ । \newline
47. त॒नुवः॒ सꣳ सम् त॒नुव॑ स्त॒नुवः॒ सम् । \newline
48. सम् प्रि॒या प्रि॒या सꣳ सम् प्रि॒या । \newline
49. प्रि॒या हृद॑यानि॒ हृद॑यानि प्रि॒या प्रि॒या हृद॑यानि । \newline
50. हृद॑यानि वो वो॒ हृद॑यानि॒ हृद॑यानि वः । \newline
51. व॒ इति॑ वः । \newline
52. आ॒त्मा वो॑ व आ॒त्मा ऽऽत्मा वः॑ । \newline
53. वो॒ अ॒स्त्व॒स्तु॒ वो॒ वो॒ अ॒स्तु॒ । \newline
54. अ॒स्तु॒ संप्रि॑यः॒ संप्रि॑यो अस्त्वस्तु॒ संप्रि॑यः । \newline

\textbf{Ghana Paata } \newline

1. अपे॑ ते॒ तापापे॑ त॒ वि वीतापापे॑ त॒ वि । \newline
2. इ॒त॒ वि वीते॑ त॒ वीते॑ त॒ वीते॑ त॒ वीत॑ । \newline
3. वीते॑ त॒ वि वीत॒ वि वीत॒ वि वीत॒ वि । \newline
4. इ॒त॒ वि वीते॑ त॒ वि च॑ च॒ वीते॑ त॒ वि च॑ । \newline
5. वि च॑ च॒ वि वि च॑ सर्पत सर्पत च॒ वि वि च॑ सर्पत । \newline
6. च॒ स॒र्प॒त॒ स॒र्प॒त॒ च॒ च॒ स॒र्प॒तातो॒ अतः॑ सर्पत च च सर्प॒तातः॑ । \newline
7. स॒र्प॒तातो॒ अतः॑ सर्पत सर्प॒तातो॒ ये ये अतः॑ सर्पत सर्प॒तातो॒ ये । \newline
8. अतो॒ ये ये अतो॒ अतो॒ ये ऽत्रात्र॒ ये अतो॒ अतो॒ ये ऽत्र॑ । \newline
9. ये ऽत्रात्र॒ ये ये ऽत्र॒ स्थ स्थात्र॒ ये ये ऽत्र॒ स्थ । \newline
10. अत्र॒ स्थ स्था त्रात्र॒ स्थ पु॑रा॒णाः पु॑रा॒णाः स्था त्रात्र॒ स्थ पु॑रा॒णाः । \newline
11. स्थ पु॑रा॒णाः पु॑रा॒णाः स्थ स्थ पु॑रा॒णा ये ये पु॑रा॒णाः स्थ स्थ पु॑रा॒णा ये । \newline
12. पु॒रा॒णा ये ये पु॑रा॒णाः पु॑रा॒णा ये च॑ च॒ ये पु॑रा॒णाः पु॑रा॒णा ये च॑ । \newline
13. ये च॑ च॒ ये ये च॒ नूत॑ना॒ नूत॑नाश्च॒ ये ये च॒ नूत॑नाः । \newline
14. च॒ नूत॑ना॒ नूत॑नाश्च च॒ नूत॑नाः । \newline
15. नूत॑ना॒ इति॒ नूत॑नाः । \newline
16. अदा॑ दि॒द मि॒द मदा॒ ददा॑ दि॒दं ॅय॒मो य॒म इ॒द मदा॒ ददा॑ दि॒दं ॅय॒मः । \newline
17. इ॒दं ॅय॒मो य॒म इ॒द मि॒दं ॅय॒मो॑ ऽव॒सान॑ मव॒सानं॑ ॅय॒म इ॒द मि॒दं ॅय॒मो॑ ऽव॒सान᳚म् । \newline
18. य॒मो॑ ऽव॒सान॑ मव॒सानं॑ ॅय॒मो य॒मो॑ ऽव॒सान॑म् पृथि॒व्याः पृ॑थि॒व्या अ॑व॒सानं॑ ॅय॒मो य॒मो॑ ऽव॒सान॑म् पृथि॒व्याः । \newline
19. अ॒व॒सान॑म् पृथि॒व्याः पृ॑थि॒व्या अ॑व॒सान॑ मव॒सान॑म् पृथि॒व्या अक्र॒न् नक्र॑न् पृथि॒व्या अ॑व॒सान॑ मव॒सान॑म् पृथि॒व्या अक्रन्न्॑ । \newline
20. अ॒व॒सान॒मित्य॑व - सान᳚म् । \newline
21. पृ॒थि॒व्या अक्र॒न् नक्र॑न् पृथि॒व्याः पृ॑थि॒व्या अक्र॑न् नि॒म मि॒म मक्र॑न् पृथि॒व्याः पृ॑थि॒व्या अक्र॑न् नि॒मम् । \newline
22. अक्र॑न् नि॒म मि॒म मक्र॒न् नक्र॑न् नि॒मम् पि॒तरः॑ पि॒तर॑ इ॒म मक्र॒न् नक्र॑न् नि॒मम् पि॒तरः॑ । \newline
23. इ॒मम् पि॒तरः॑ पि॒तर॑ इ॒म मि॒मम् पि॒तरो॑ लो॒कम् ॅलो॒कम् पि॒तर॑ इ॒म मि॒मम् पि॒तरो॑ लो॒कम् । \newline
24. पि॒तरो॑ लो॒कम् ॅलो॒कम् पि॒तरः॑ पि॒तरो॑ लो॒क म॑स्मा अस्मै लो॒कम् पि॒तरः॑ पि॒तरो॑ लो॒क म॑स्मै । \newline
25. लो॒क म॑स्मा अस्मै लो॒कम् ॅलो॒क म॑स्मै । \newline
26. अ॒स्मा॒ इत्य॑स्मै । \newline
27. अ॒ग्नेर् भस्म॒ भस्मा॒ग्ने र॒ग्नेर् भस्मा᳚स्यसि॒ भस्मा॒ग्ने र॒ग्नेर् भस्मा॑सि । \newline
28. भस्मा᳚स्यसि॒ भस्म॒ भस्मा᳚ स्य॒ग्ने र॒ग्ने र॑सि॒ भस्म॒ भस्मा᳚स्य॒ग्नेः । \newline
29. अ॒स्य॒ग्ने र॒ग्ने र॑स्यस्य॒ग्नेः पुरी॑ष॒म् पुरी॑ष म॒ग्ने र॑स्यस्य॒ग्नेः पुरी॑षम् । \newline
30. अ॒ग्नेः पुरी॑ष॒म् पुरी॑ष म॒ग्ने र॒ग्नेः पुरी॑ष मस्यसि॒ पुरी॑ष म॒ग्ने र॒ग्नेः पुरी॑ष मसि । \newline
31. पुरी॑ष मस्यसि॒ पुरी॑ष॒म् पुरी॑ष मसि सं॒.ज्ञानꣳ॑ सं॒.ज्ञान॑ मसि॒ पुरी॑ष॒म् पुरी॑ष मसि सं॒.ज्ञान᳚म् । \newline
32. अ॒सि॒ सं॒.ज्ञानꣳ॑ सं॒.ज्ञान॑ मस्यसि सं॒.ज्ञान॑ मस्यसि सं॒.ज्ञान॑ मस्यसि सं॒.ज्ञान॑ मसि । \newline
33. सं॒.ज्ञान॑ मस्यसि सं॒.ज्ञानꣳ॑ सं॒.ज्ञान॑ मसि काम॒धर॑णम् काम॒धर॑ण मसि सं॒.ज्ञानꣳ॑ सं॒.ज्ञान॑ मसि काम॒धर॑णम् । \newline
34. सं॒.ज्ञान॒मिति॑ सं - ज्ञान᳚म् । \newline
35. अ॒सि॒ का॒म॒धर॑णम् काम॒धर॑ण मस्यसि काम॒धर॑ण॒म् मयि॒ मयि॑ काम॒धर॑ण मस्यसि काम॒धर॑ण॒म् मयि॑ । \newline
36. का॒म॒धर॑ण॒म् मयि॒ मयि॑ काम॒धर॑णम् काम॒धर॑ण॒म् मयि॑ ते ते॒ मयि॑ काम॒धर॑णम् काम॒धर॑ण॒म् मयि॑ ते । \newline
37. का॒म॒धर॑ण॒मिति॑ काम - धर॑णम् । \newline
38. मयि॑ ते ते॒ मयि॒ मयि॑ ते काम॒धर॑णम् काम॒धर॑णम् ते॒ मयि॒ मयि॑ ते काम॒धर॑णम् । \newline
39. ते॒ का॒म॒धर॑णम् काम॒धर॑णम् ते ते काम॒धर॑णम् भूयाद् भूयात् काम॒धर॑णम् ते ते काम॒धर॑णम् भूयात् । \newline
40. का॒म॒धर॑णम् भूयाद् भूयात् काम॒धर॑णम् काम॒धर॑णम् भूयात् । \newline
41. का॒म॒धर॑ण॒मिति॑ काम - धर॑णम् । \newline
42. भू॒या॒दिति॑ भूयात् । \newline
43. सं ॅया याः सꣳ सं ॅया वो॑ वो॒ याः सꣳ सं ॅया वः॑ । \newline
44. या वो॑ वो॒ या या वः॑ प्रि॒याः प्रि॒या वो॒ या या वः॑ प्रि॒याः । \newline
45. वः॒ प्रि॒याः प्रि॒या वो॑ वः प्रि॒या स्त॒नुव॑ स्त॒नुवः॑ प्रि॒या वो॑ वः प्रि॒या स्त॒नुवः॑ । \newline
46. प्रि॒या स्त॒नुव॑ स्त॒नुवः॑ प्रि॒याः प्रि॒या स्त॒नुवः॒ सꣳ सम् त॒नुवः॑ प्रि॒याः प्रि॒या स्त॒नुवः॒ सम् । \newline
47. त॒नुवः॒ सꣳ सम् त॒नुव॑ स्त॒नुवः॒ सम् प्रि॒या प्रि॒या सम् त॒नुव॑ स्त॒नुवः॒ सम् प्रि॒या । \newline
48. सम् प्रि॒या प्रि॒या सꣳ सम् प्रि॒या हृद॑यानि॒ हृद॑यानि प्रि॒या सꣳ सम् प्रि॒या हृद॑यानि । \newline
49. प्रि॒या हृद॑यानि॒ हृद॑यानि प्रि॒या प्रि॒या हृद॑यानि वो वो॒ हृद॑यानि प्रि॒या प्रि॒या हृद॑यानि वः । \newline
50. हृद॑यानि वो वो॒ हृद॑यानि॒ हृद॑यानि वः । \newline
51. व॒ इति॑ वः । \newline
52. आ॒त्मा वो॑ व आ॒त्मा ऽऽत्मा वो॑ अस्त्वस्तु व आ॒त्मा ऽऽत्मा वो॑ अस्तु । \newline
53. वो॒ अ॒स्त्व॒स्तु॒ वो॒ वो॒ अ॒स्तु॒ संप्रि॑यः॒ संप्रि॑यो अस्तु वो वो अस्तु॒ संप्रि॑यः । \newline
54. अ॒स्तु॒ संप्रि॑यः॒ संप्रि॑यो अस्त्वस्तु॒ संप्रि॑यः॒ संप्रि॑याः॒ संप्रि॑याः॒ संप्रि॑यो अस्त्वस्तु॒ संप्रि॑यः॒ संप्रि॑याः । \newline
\pagebreak
\markright{ TS 4.2.4.2  \hfill https://www.vedavms.in \hfill}

\section{ TS 4.2.4.2 }

\textbf{TS 4.2.4.2 } \newline
\textbf{Samhita Paata} \newline

संप्रि॑यः॒ संप्रि॑यास्त॒नुवो॒ मम॑ ॥ अ॒यꣳ सो अ॒ग्निर्यस्मि॒न्थ्-सोम॒मिन्द्रः॑ सु॒तं द॒धे ज॒ठरे॑ वावशा॒नः । स॒ह॒स्रियं॒ ॅवाज॒मत्यं॒ न सप्तिꣳ॑ सस॒वान्थ्-सन्थ्-स्तू॑यसे जातवेदः ॥ अग्ने॑ दि॒वो अर्ण॒मच्छा॑ जिगा॒स्यच्छा॑ दे॒वाꣳ ऊ॑चिषे॒ धिष्णि॑या॒ ये । याः प॒रस्ता᳚द्-रोच॒ने सूर्य॑स्य॒ याश्चा॒ वस्ता॑-दुप॒तिष्ठ॑न्त॒ आपः॑ ॥ अग्ने॒ यत् ते॑ दि॒वि वर्चः॑ पृथि॒व्यां ॅयदोष॑धीष्व॒ - [  ] \newline

\textbf{Pada Paata} \newline

संप्रि॑य॒ इति॒ सं-प्रि॒यः॒ । संप्रि॑या॒ इति॒ सं - प्रि॒याः॒ । त॒नुवः॑ । मम॑ ॥ अ॒यम् । सः । अ॒ग्निः । यस्मिन्न्॑ । सोम᳚म् । इन्द्रः॑ । सु॒तम् । द॒धे । ज॒ठरे᳚ । वा॒व॒शा॒नः ॥ स॒ह॒स्रिय᳚म् । वाज᳚म् । अत्य᳚म् । न । सप्ति᳚म् । स॒स॒वानिति॑ स - स॒वान् । सन्न् । स्तृ॒य॒से॒ । जा॒त॒वे॒द॒ इति॑ जात - वे॒दः॒ ॥ अग्ने᳚ । दि॒वः । अर्ण᳚म् । अच्छ॑ । जि॒गा॒सि॒ । अच्छ॑ । दे॒वान् । ऊ॒चि॒षे॒ । धिष्णि॑याः । ये ॥ याः । प॒रस्ता᳚त् । रो॒च॒ने । सूर्य॑स्य । याः । च॒ । अ॒वस्ता᳚त् । उ॒प॒तिष्ठ॑न्त॒ इत्यु॑प - तिष्ठ॑न्ते । आपः॑ ॥ अग्ने᳚ । यत् । ते॒ । दि॒वि । वर्चः॑ । पृ॒थि॒व्याम् । यत् । ओष॑धीषु ।  \newline


\textbf{Krama Paata} \newline

सम्प्रि॑यः॒ सम्प्रि॑याः । सम्प्रि॑य॒ इति॒ सम् - प्रि॒यः॒ । सम्प्रि॑यास्त॒नुवः॑ । सम्प्रि॑या॒ इति॒ सम् - प्रि॒याः॒ । त॒नुवो॒ मम॑ । ममेति॒ मम॑ ॥ अ॒यꣳ सः । सो अ॒ग्निः । अ॒ग्निर् यस्मिन्न्॑ । यस्मि॒न्थ् सोम᳚म् । सोम॒मिन्द्रः॑ । इन्द्रः॑ सु॒तम् । सु॒तम् द॒धे । द॒धे ज॒ठरे᳚ । ज॒ठरे॑ वावशा॒नः । वा॒व॒शा॒न इति॑ वावशा॒नः ॥ स॒ह॒स्रियं॒ ॅवाज᳚म् । वाज॒मत्य᳚म् । अत्य॒म् न । न सप्ति᳚म् । सप्तिꣳ॑ सस॒वान् । स॒स॒वान्थ् सन्न् । स॒स॒वानिति॑ स - स॒वान् । सन्थ् स्तू॑यसे । स्तू॒य॒से॒ जा॒त॒वे॒दः॒ । जा॒त॒वे॒द॒ इति॑ जात - वे॒दः॒ ॥ अग्ने॑ दि॒वः । दि॒वो अर्ण᳚म् । अर्ण॒मच्छ॑ । अच्छा॑ जिगासि । जि॒गा॒स्यच्छ॑ । अच्छा॑ दे॒वान् । दे॒वाꣳ ऊ॑चिषे । ऊ॒चि॒षे॒ धिष्णि॑याः । धिष्णि॑या॒ ये । य इति॒ ये ॥ याः प॒रस्ता᳚त् । प॒रस्ता᳚द् रोच॒ने । रो॒च॒ने सूर्य॑स्य । सूर्य॑स्य॒ याः । याश्च॑ । चा॒वस्ता᳚त् । अ॒वस्ता॑दुप॒तिष्ठ॑न्ते । उ॒प॒तिष्ठ॑न्त॒ आपः॑ । उ॒प॒तिष्ठ॑न्त॒ इत्यु॑प - तिष्ठ॑न्ते । आप॒ इत्यापः॑ ॥ अग्ने॒ यत् । यत् ते᳚ । ते॒ दि॒वि । दि॒वि वर्चः॑ । वर्चः॑ पृथि॒व्याम् । पृ॒थि॒व्यां ॅयत् । यदोष॑धीषु । ओष॑धीष्व॒फ्सु \newline

\textbf{Jatai Paata} \newline

1. संप्रि॑यः॒ संप्रि॑याः॒ संप्रि॑याः॒ संप्रि॑यः॒ संप्रि॑यः॒ संप्रि॑याः । \newline
2. संप्रि॑य॒ इति॒ सं - प्रि॒यः॒ । \newline
3. संप्रि॑या स्त॒नुव॑ स्त॒नुवः॒ संप्रि॑याः॒ संप्रि॑या स्त॒नुवः॑ । \newline
4. संप्रि॑या॒ इति॒ सं - प्रि॒याः॒ । \newline
5. त॒नुवो॒ मम॒ मम॑ त॒नुव॑ स्त॒नुवो॒ मम॑ । \newline
6. ममेति॒ मम॑ । \newline
7. अ॒यꣳ स सो॑ ऽय म॒यꣳ सः । \newline
8. सो अ॒ग्नि र॒ग्निः स सो अ॒ग्निः । \newline
9. अ॒ग्निर् यस्मि॒न्॒. यस्मि॑न् न॒ग्नि र॒ग्निर् यस्मिन्न्॑ । \newline
10. यस्मि॒न् थ्सोमꣳ॒॒ सोमं॒ ॅयस्मि॒न्॒. यस्मि॒न् थ्सोम᳚म् । \newline
11. सोम॒ मिन्द्र॒ इन्द्रः॒ सोमꣳ॒॒ सोम॒ मिन्द्रः॑ । \newline
12. इन्द्रः॑ सु॒तꣳ सु॒त मिन्द्र॒ इन्द्रः॑ सु॒तम् । \newline
13. सु॒तम् द॒धे द॒धे सु॒तꣳ सु॒तम् द॒धे । \newline
14. द॒धे ज॒ठरे॑ ज॒ठरे॑ द॒धे द॒धे ज॒ठरे᳚ । \newline
15. ज॒ठरे॑ वावशा॒नो वा॑वशा॒नो ज॒ठरे॑ ज॒ठरे॑ वावशा॒नः । \newline
16. वा॒व॒शा॒न इति॑ वावशा॒नः । \newline
17. स॒ह॒स्रियं॒ ॅवाजं॒ ॅवाजꣳ॑ सह॒स्रियꣳ॑ सह॒स्रियं॒ ॅवाज᳚म् । \newline
18. वाज॒ मत्य॒ मत्यं॒ ॅवाजं॒ ॅवाज॒ मत्य᳚म् । \newline
19. अत्य॒म् न नात्य॒ मत्य॒म् न । \newline
20. न सप्तिꣳ॒॒ सप्ति॒म् न न सप्ति᳚म् । \newline
21. सप्तिꣳ॑ सस॒वान् थ्स॑स॒वान् थ्सप्तिꣳ॒॒ सप्तिꣳ॑ सस॒वान् । \newline
22. स॒स॒वान् थ्सन् थ्सन् थ्स॑स॒वान् थ्स॑स॒वान् थ्सन्न् । \newline
23. स॒स॒वानिति॑ स - स॒वान् । \newline
24. सन् थ्स्तू॑यसे स्तूयसे॒ सन् थ्सन् थ्स्तू॑यसे । \newline
25. स्तू॒य॒से॒ जा॒त॒वे॒दो॒ जा॒त॒वे॒दः॒ स्तू॒य॒से॒ स्तू॒य॒से॒ जा॒त॒वे॒दः॒ । \newline
26. जा॒त॒वे॒द॒ इति॑ जात - वे॒दः॒ । \newline
27. अग्ने॑ दि॒वो दि॒वो ऽग्ने ऽग्ने॑ दि॒वः । \newline
28. दि॒वो अर्ण॒ मर्ण॑म् दि॒वो दि॒वो अर्ण᳚म् । \newline
29. अर्ण॒ मच्छा च्छार्ण॒ मर्ण॒ मच्छ॑ । \newline
30. अच्छा॑ जिगासि जिगा॒स्यच्छा च्छा॑ जिगासि । \newline
31. जि॒गा॒ स्यच्छा च्छ॑ जिगासि जिगा॒स्यच्छ॑ । \newline
32. अच्छा॑ दे॒वान् दे॒वाꣳ अच्छाच्छा॑ दे॒वान् । \newline
33. दे॒वाꣳ ऊ॑चिष ऊचिषे दे॒वान् दे॒वाꣳ ऊ॑चिषे । \newline
34. ऊ॒चि॒षे॒ धिष्णि॑या॒ धिष्णि॑या ऊचिष ऊचिषे॒ धिष्णि॑याः । \newline
35. धिष्णि॑या॒ ये ये धिष्णि॑या॒ धिष्णि॑या॒ ये । \newline
36. य इति॒ ये । \newline
37. याः प॒रस्ता᳚त् प॒रस्ता॒द् या याः प॒रस्ता᳚त् । \newline
38. प॒रस्ता᳚द् रोच॒ने रो॑च॒ने प॒रस्ता᳚त् प॒रस्ता᳚द् रोच॒ने । \newline
39. रो॒च॒ने सूर्य॑स्य॒ सूर्य॑स्य रोच॒ने रो॑च॒ने सूर्य॑स्य । \newline
40. सूर्य॑स्य॒ या याः सूर्य॑स्य॒ सूर्य॑स्य॒ याः । \newline
41. याश्च॑ च॒ या याश्च॑ । \newline
42. चा॒वस्ता॑ द॒वस्ता᳚च् च चा॒वस्ता᳚त् । \newline
43. अ॒वस्ता॑ दुप॒तिष्ठ॑न्त उप॒तिष्ठ॑न्ते॒ ऽवस्ता॑ द॒वस्ता॑ दुप॒तिष्ठ॑न्ते । \newline
44. उ॒प॒तिष्ठ॑न्त॒ आप॒ आप॑ उप॒तिष्ठ॑न्त उप॒तिष्ठ॑न्त॒ आपः॑ । \newline
45. उ॒प॒तिष्ठ॑न्त॒ इत्यु॑प - तिष्ठ॑न्ते । \newline
46. आप॒ इत्यापः॑ । \newline
47. अग्ने॒ यद् यदग्ने ऽग्ने॒ यत् । \newline
48. यत् ते॑ ते॒ यद् यत् ते᳚ । \newline
49. ते॒ दि॒वि दि॒वि ते॑ ते दि॒वि । \newline
50. दि॒वि वर्चो॒ वर्चो॑ दि॒वि दि॒वि वर्चः॑ । \newline
51. वर्चः॑ पृथि॒व्याम् पृ॑थि॒व्यां ॅवर्चो॒ वर्चः॑ पृथि॒व्याम् । \newline
52. पृ॒थि॒व्यां ॅयद् यत् पृ॑थि॒व्याम् पृ॑थि॒व्यां ॅयत् । \newline
53. यदोष॑धी॒ ष्वोष॑धीषु॒ यद् यदोष॑धीषु । \newline
54. ओष॑धी ष्व॒फ्स्व॑ फ्स्वोष॑धी॒ ष्वोष॑धी ष्व॒फ्सु । \newline

\textbf{Ghana Paata } \newline

1. संप्रि॑यः॒ संप्रि॑याः॒ संप्रि॑याः॒ संप्रि॑यः॒ संप्रि॑यः॒ संप्रि॑या स्त॒नुव॑ स्त॒नुवः॒ संप्रि॑याः॒ संप्रि॑यः॒ संप्रि॑यः॒ संप्रि॑या स्त॒नुवः॑ । \newline
2. संप्रि॑य॒ इति॒ सं - प्रि॒यः॒ । \newline
3. संप्रि॑या स्त॒नुव॑ स्त॒नुवः॒ संप्रि॑याः॒ संप्रि॑या स्त॒नुवो॒ मम॒ मम॑ त॒नुवः॒ संप्रि॑याः॒ संप्रि॑या स्त॒नुवो॒ मम॑ । \newline
4. संप्रि॑या॒ इति॒ सं - प्रि॒याः॒ । \newline
5. त॒नुवो॒ मम॒ मम॑ त॒नुव॑ स्त॒नुवो॒ मम॑ । \newline
6. ममेति॒ मम॑ । \newline
7. अ॒यꣳ स सो॑ ऽय म॒यꣳ सो अ॒ग्नि र॒ग्निः सो॑ ऽय म॒यꣳ सो अ॒ग्निः । \newline
8. सो अ॒ग्नि र॒ग्निः स सो अ॒ग्निर् यस्मि॒न्॒. यस्मि॑न् न॒ग्निः स सो अ॒ग्निर् यस्मिन्न्॑ । \newline
9. अ॒ग्निर् यस्मि॒न्॒. यस्मि॑न् न॒ग्नि र॒ग्निर् यस्मि॒न् थ्सोमꣳ॒॒ सोमं॒ ॅयस्मि॑न् न॒ग्नि र॒ग्निर् यस्मि॒न् थ्सोम᳚म् । \newline
10. यस्मि॒न् थ्सोमꣳ॒॒ सोमं॒ ॅयस्मि॒न्॒. यस्मि॒न् थ्सोम॒ मिन्द्र॒ इन्द्रः॒ सोमं॒ ॅयस्मि॒न्॒. यस्मि॒न् थ्सोम॒ मिन्द्रः॑ । \newline
11. सोम॒ मिन्द्र॒ इन्द्रः॒ सोमꣳ॒॒ सोम॒ मिन्द्रः॑ सु॒तꣳ सु॒त मिन्द्रः॒ सोमꣳ॒॒ सोम॒ मिन्द्रः॑ सु॒तम् । \newline
12. इन्द्रः॑ सु॒तꣳ सु॒त मिन्द्र॒ इन्द्रः॑ सु॒तम् द॒धे द॒धे सु॒त मिन्द्र॒ इन्द्रः॑ सु॒तम् द॒धे । \newline
13. सु॒तम् द॒धे द॒धे सु॒तꣳ सु॒तम् द॒धे ज॒ठरे॑ ज॒ठरे॑ द॒धे सु॒तꣳ सु॒तम् द॒धे ज॒ठरे᳚ । \newline
14. द॒धे ज॒ठरे॑ ज॒ठरे॑ द॒धे द॒धे ज॒ठरे॑ वावशा॒नो वा॑वशा॒नो ज॒ठरे॑ द॒धे द॒धे ज॒ठरे॑ वावशा॒नः । \newline
15. ज॒ठरे॑ वावशा॒नो वा॑वशा॒नो ज॒ठरे॑ ज॒ठरे॑ वावशा॒नः । \newline
16. वा॒व॒शा॒न इति॑ वावशा॒नः । \newline
17. स॒ह॒स्रियं॒ ॅवाजं॒ ॅवाजꣳ॑ सह॒स्रियꣳ॑ सह॒स्रियं॒ ॅवाज॒ मत्य॒ मत्यं॒ ॅवाजꣳ॑ सह॒स्रियꣳ॑ सह॒स्रियं॒ ॅवाज॒ मत्य᳚म् । \newline
18. वाज॒ मत्य॒ मत्यं॒ ॅवाजं॒ ॅवाज॒ मत्य॒न्न नात्यं॒ ॅवाजं॒ ॅवाज॒ मत्य॒न्न । \newline
19. अत्य॒न्न नात्य॒ मत्य॒म् न सप्तिꣳ॒॒ सप्ति॒म् नात्य॒ मत्य॒म् न सप्ति᳚म् । \newline
20. न सप्तिꣳ॒॒ सप्ति॒म् न न सप्तिꣳ॑ सस॒वान् थ्स॑स॒वान् थ्सप्ति॒म् न न सप्तिꣳ॑ सस॒वान् । \newline
21. सप्तिꣳ॑ सस॒वान् थ्स॑स॒वान् थ्सप्तिꣳ॒॒ सप्तिꣳ॑ सस॒वान् थ्सन् थ्सन् थ्स॑स॒वान् थ्सप्तिꣳ॒॒ सप्तिꣳ॑ सस॒वान् थ्सन्न् । \newline
22. स॒स॒वान् थ्सन् थ्सन् थ्स॑स॒वान् थ्स॑स॒वान् थ्सन् थ्स्तू॑यसे स्तूयसे॒ सन् थ्स॑स॒वान् थ्स॑स॒वान् थ्सन् थ्स्तू॑यसे । \newline
23. स॒स॒वानिति॑ स - स॒वान् । \newline
24. सन् थ्स्तू॑यसे स्तूयसे॒ सन् थ्सन् थ्स्तू॑यसे जातवेदो जातवेदः स्तूयसे॒ सन् थ्सन् थ्स्तू॑यसे जातवेदः । \newline
25. स्तू॒य॒से॒ जा॒त॒वे॒दो॒ जा॒त॒वे॒दः॒ स्तू॒य॒से॒ स्तू॒य॒से॒ जा॒त॒वे॒दः॒ । \newline
26. जा॒त॒वे॒द॒ इति॑ जात - वे॒दः॒ । \newline
27. अग्ने॑ दि॒वो दि॒वो ऽग्ने ऽग्ने॑ दि॒वो अर्ण॒ मर्ण॑म् दि॒वो ऽग्ने ऽग्ने॑ दि॒वो अर्ण᳚म् । \newline
28. दि॒वो अर्ण॒ मर्ण॑म् दि॒वो दि॒वो अर्ण॒ मच्छा च्छार्ण॑म् दि॒वो दि॒वो अर्ण॒ मच्छ॑ । \newline
29. अर्ण॒ मच्छा च्छार्ण॒ मर्ण॒ मच्छा॑ जिगासि जिगा॒ स्यच्छार्ण॒ मर्ण॒ मच्छा॑ जिगासि । \newline
30. अच्छा॑ जिगासि जिगा॒ स्यच्छाच्छा॑ जिगा॒ स्यच्छाच्छ॑ जिगा॒ स्यच्छाच्छा॑ जिगा॒ स्यच्छ॑ । \newline
31. जि॒गा॒ स्यच्छाच्छ॑ जिगासि जिगा॒ स्यच्छा॑ दे॒वान् दे॒वाꣳ अच्छ॑ जिगासि जिगा॒ स्यच्छा॑ दे॒वान् । \newline
32. अच्छा॑ दे॒वान् दे॒वाꣳ अच्छाच्छा॑ दे॒वाꣳ ऊ॑चिष ऊचिषे दे॒वाꣳ अच्छाच्छा॑ दे॒वाꣳ ऊ॑चिषे । \newline
33. दे॒वाꣳ ऊ॑चिष ऊचिषे दे॒वान् दे॒वाꣳ ऊ॑चिषे॒ धिष्णि॑या॒ धिष्णि॑या ऊचिषे दे॒वान् दे॒वाꣳ ऊ॑चिषे॒ धिष्णि॑याः । \newline
34. ऊ॒चि॒षे॒ धिष्णि॑या॒ धिष्णि॑या ऊचिष ऊचिषे॒ धिष्णि॑या॒ ये ये धिष्णि॑या ऊचिष ऊचिषे॒ धिष्णि॑या॒ ये । \newline
35. धिष्णि॑या॒ ये ये धिष्णि॑या॒ धिष्णि॑या॒ ये । \newline
36. य इति॒ ये । \newline
37. याः प॒रस्ता᳚त् प॒रस्ता॒द् या याः प॒रस्ता᳚द् रोच॒ने रो॑च॒ने प॒रस्ता॒द् या याः प॒रस्ता᳚द् रोच॒ने । \newline
38. प॒रस्ता᳚द् रोच॒ने रो॑च॒ने प॒रस्ता᳚त् प॒रस्ता᳚द् रोच॒ने सूर्य॑स्य॒ सूर्य॑स्य रोच॒ने प॒रस्ता᳚त् प॒रस्ता᳚द् रोच॒ने सूर्य॑स्य । \newline
39. रो॒च॒ने सूर्य॑स्य॒ सूर्य॑स्य रोच॒ने रो॑च॒ने सूर्य॑स्य॒ या याः सूर्य॑स्य रोच॒ने रो॑च॒ने सूर्य॑स्य॒ याः । \newline
40. सूर्य॑स्य॒ या याः सूर्य॑स्य॒ सूर्य॑स्य॒ याश्च॑ च॒ याः सूर्य॑स्य॒ सूर्य॑स्य॒ याश्च॑ । \newline
41. याश्च॑ च॒ या याश्चा॒ वस्ता॑ द॒वस्ता᳚च् च॒ या याश्चा॒ वस्ता᳚त् । \newline
42. चा॒वस्ता॑ द॒वस्ता᳚च् च चा॒वस्ता॑ दुप॒तिष्ठ॑न्त उप॒तिष्ठ॑न्ते॒ ऽवस्ता᳚च् च चा॒वस्ता॑ दुप॒तिष्ठ॑न्ते । \newline
43. अ॒वस्ता॑ दुप॒तिष्ठ॑न्त उप॒तिष्ठ॑न्ते॒ ऽवस्ता॑ द॒वस्ता॑ दुप॒तिष्ठ॑न्त॒ आप॒ आप॑ उप॒तिष्ठ॑न्ते॒ ऽवस्ता॑ द॒वस्ता॑ दुप॒तिष्ठ॑न्त॒ आपः॑ । \newline
44. उ॒प॒तिष्ठ॑न्त॒ आप॒ आप॑ उप॒तिष्ठ॑न्त उप॒तिष्ठ॑न्त॒ आपः॑ । \newline
45. उ॒प॒तिष्ठ॑न्त॒ इत्यु॑प - तिष्ठ॑न्ते । \newline
46. आप॒ इत्यापः॑ । \newline
47. अग्ने॒ यद् यदग्ने ऽग्ने॒ यत् ते॑ ते॒ यदग्ने ऽग्ने॒ यत् ते᳚ । \newline
48. यत् ते॑ ते॒ यद् यत् ते॑ दि॒वि दि॒वि ते॒ यद् यत् ते॑ दि॒वि । \newline
49. ते॒ दि॒वि दि॒वि ते॑ ते दि॒वि वर्चो॒ वर्चो॑ दि॒वि ते॑ ते दि॒वि वर्चः॑ । \newline
50. दि॒वि वर्चो॒ वर्चो॑ दि॒वि दि॒वि वर्चः॑ पृथि॒व्याम् पृ॑थि॒व्यां ॅवर्चो॑ दि॒वि दि॒वि वर्चः॑ पृथि॒व्याम् । \newline
51. वर्चः॑ पृथि॒व्याम् पृ॑थि॒व्यां ॅवर्चो॒ वर्चः॑ पृथि॒व्यां ॅयद् यत् पृ॑थि॒व्यां ॅवर्चो॒ वर्चः॑ पृथि॒व्यां ॅयत् । \newline
52. पृ॒थि॒व्यां ॅयद् यत् पृ॑थि॒व्याम् पृ॑थि॒व्यां ॅयदोष॑धी॒ ष्वोष॑धीषु॒ यत् पृ॑थि॒व्याम् पृ॑थि॒व्यां ॅयदोष॑धीषु । \newline
53. यदोष॑धी॒ ष्वोष॑धीषु॒ यद् यदोष॑धी ष्व॒फ्स्व॑ फ्स्वोष॑धीषु॒ यद् यदोष॑धी ष्व॒फ्सु । \newline
54. ओष॑धी ष्व॒फ्स्व॑ फ्स्वोष॑धी॒ ष्वोष॑धी ष्व॒फ्सु वा॑ वा॒ ऽफ्स्वोष॑धी॒ ष्वोष॑धी ष्व॒फ्सु वा᳚ । \newline
\pagebreak
\markright{ TS 4.2.4.3  \hfill https://www.vedavms.in \hfill}

\section{ TS 4.2.4.3 }

\textbf{TS 4.2.4.3 } \newline
\textbf{Samhita Paata} \newline

-ऽफ्सु वा॑ यजत्र । येना॒न्तरि॑क्ष-मु॒र्वा॑त॒तन्थ॑ त्वे॒षः स भा॒नुर॑र्ण॒वो नृ॒चक्षाः᳚ ॥ पु॒री॒ष्या॑सो अ॒ग्नयः॑ प्राव॒णेभिः॑ स॒जोष॑सः । जु॒षन्ताꣳ॑ ह॒व्यमाहु॑तमनमी॒वा इषो॑ म॒हीः ॥ इडा॑मग्ने पुरु॒दꣳ सꣳ॑ स॒निं गोः श॑श्वत्त॒मꣳ हव॑मानाय साध । स्यान्नः॑ सू॒नुस्तन॑यो वि॒जावाऽग्ने॒ सा ते॑ सुम॒तिर्भू᳚त्व॒स्मे ॥ अ॒यं ते॒ योनि॑र्. ऋ॒त्वियो॒ यतो॑ जा॒तो अरो॑चथाः । तं जा॒न - [  ] \newline

\textbf{Pada Paata} \newline

अ॒फ्स्वित्य॑प् - सु । वा॒ । य॒ज॒त्र॒ ॥ येन॑ । अ॒न्तरि॑क्षम् । उ॒रु । आ॒त॒तन्थेत्या᳚ - त॒तन्थ॑ । त्वे॒षः । सः । भा॒नुः । अ॒र्ण॒वः । नृ॒चक्षा॒ इति॑ नृ - चक्षाः᳚ ॥ पु॒री॒ष्या॑सः । अ॒ग्नयः॑ । प्रा॒व॒णेभि॒रिति॑ प्र - व॒नेभिः॑ । स॒जोष॑स॒ इति॑ स - जोष॑सः ॥ जु॒षन्ता᳚म् । ह॒व्यम् । आहु॑त॒मित्या - हु॒त॒म् । अ॒न॒मी॒वाः । इषः॑ । म॒हीः ॥ इडा᳚म् । अ॒ग्ने॒ । पु॒रु॒दꣳस॒मिति॑ पुरु - दꣳस᳚म् । स॒निम् । गोः । श॒श्व॒त्त॒ममिति॑ शश्वत् - त॒मम् । हव॑मानाय । सा॒ध॒ ॥ स्यात् । नः॒ । सू॒नुः । तन॑यः । वि॒जावेति॑ वि - जावा᳚ । अग्ने᳚ । सा । ते॒ । सु॒म॒तिरिति॑ सु-म॒तिः । भू॒तु॒ । अ॒स्मे इति॑ ॥ अ॒यम् । ते॒ । योनिः॑ । ऋ॒त्वियः॑ । यतः॑ । जा॒तः । अरो॑चथाः ॥ तम् । जा॒नन्न् ।  \newline


\textbf{Krama Paata} \newline

अ॒फ्सु वा᳚ । अ॒फ्स्वित्य॑प् - सु । वा॒ य॒ज॒त्र॒ । य॒ज॒त्रेति॑ यजत्र ॥ येना॒न्तरि॑क्षम् । अ॒न्तरि॑क्षमु॒रु । उ॒र्वा॑त॒तन्थ॑ । आ॒त॒तन्थ॑ त्वे॒षः । आ॒त॒तन्थेत्या᳚ - त॒तन्थ॑ । त्वे॒षः सः । स भा॒नुः । भा॒नुर॑र्ण॒वः । अ॒र्ण॒वो नृ॒चक्षाः᳚ । नृ॒चक्षा॒ इति॑ नृ - चक्षाः᳚ ॥ पु॒री॒ष्या॑सो अ॒ग्नयः॑ । अ॒ग्नयः॑ प्राव॒णेभिः॑ । प्रा॒व॒णेभिः॑ स॒जोष॑सः । प्रा॒व॒णेभि॒रिति॑ प्र - व॒नेभिः॑ । स॒जोष॑स॒ इति॑ स - जोष॑सः ॥ जु॒षन्ताꣳ॑ ह॒व्यम् । ह॒व्यमाहु॑तम् । आहु॑तमनमी॒वाः । आहु॑त॒मित्या - हु॒त॒म् । अ॒न॒मी॒वा इषः॑ । इषो॑ म॒हीः । म॒हीरिति॑ म॒हीः ॥ इडा॑मग्ने । अ॒ग्ने॒ पु॒रु॒दꣳस᳚म् । पु॒रु॒दꣳसꣳ॑ स॒निम् । पु॒रु॒दꣳस॒मिति॑ पुरु - दꣳस᳚म् । स॒निम् गोः । गोः श॑श्वत्त॒मम् । श॒श्व॒त्त॒मꣳ हव॑मानाय । श॒श्व॒त्त॒ममिति॑ शश्वत् - त॒मम् । हव॑मानाय साध । सा॒धेति॑ साध ॥ स्यान् नः॑ । नः॒ सू॒नुः । सू॒नुस्तन॑यः । तन॑यो वि॒जावा᳚ । वि॒जावाऽग्ने᳚ । वि॒जावेति॑ वि - जावा᳚ । अग्ने॒ सा । सा ते᳚ । ते॒ सु॒म॒तिः । सु॒म॒तिर् भू॑तु । सु॒म॒तिरिति॑ सु - म॒तिः । भू॒त्व॒स्मे । अ॒स्मे इत्य॒स्मे ॥ अ॒यम् ते᳚ । ते॒ योनिः॑ । योनि॑र्. ऋ॒त्वियः॑ । ऋ॒त्वियो॒ यतः॑ । यतो॑ जा॒तः । जा॒तो अरो॑चथाः । अरो॑चथा॒ इत्यरो॑चथाः ॥ तम् जा॒नन्न् । जा॒नन्न॑ग्ने \newline

\textbf{Jatai Paata} \newline

1. अ॒फ्सु वा॑ वा॒ ऽफ्स्व॑फ्सु वा᳚ । \newline
2. अ॒फ्स्वित्य॑प् - सु । \newline
3. वा॒ य॒ज॒त्र॒ य॒ज॒त्र॒ वा॒ वा॒ य॒ज॒त्र॒ । \newline
4. य॒ज॒त्रेति॑ यजत्र । \newline
5. येना॒न्तरि॑क्ष म॒न्तरि॑क्षं॒ ॅयेन॒ येना॒न्तरि॑क्षम् । \newline
6. अ॒न्तरि॑क्ष मु॒रू᳚(1॒)र्व॑न्तरि॑क्ष म॒न्तरि॑क्ष मु॒रु । \newline
7. उ॒र्वा॑त॒तन्था॑ त॒तन्थो॒ रू᳚(1॒)र्वा॑त॒तन्थ॑ । \newline
8. आ॒त॒तन्थ॑ त्वे॒षस्त्वे॒ष आ॑त॒तन्था॑ त॒तन्थ॑ त्वे॒षः । \newline
9. आ॒त॒तन्थेत्या᳚ - त॒तन्थ॑ । \newline
10. त्वे॒षः स स त्वे॒ष स्त्वे॒षः सः । \newline
11. स भा॒नुर् भा॒नुः स स भा॒नुः । \newline
12. भा॒नु र॑र्ण॒वो अ॑र्ण॒वो भा॒नुर् भा॒नु र॑र्ण॒वः । \newline
13. अ॒र्ण॒वो नृ॒चक्षा॑ नृ॒चक्षा॑ अर्ण॒वो अ॑र्ण॒वो नृ॒चक्षाः᳚ । \newline
14. नृ॒चक्षा॒ इति॑ नृ - चक्षाः᳚ । \newline
15. पु॒री॒ष्या॑सो अ॒ग्नयो॑ अ॒ग्नयः॑ पुरी॒ष्या॑सः पुरी॒ष्या॑सो अ॒ग्नयः॑ । \newline
16. अ॒ग्नयः॑ प्राव॒णेभिः॑ प्राव॒णेभि॑ र॒ग्नयो॑ अ॒ग्नयः॑ प्राव॒णेभिः॑ । \newline
17. प्रा॒व॒णेभिः॑ स॒जोष॑सः स॒जोष॑सः प्राव॒णेभिः॑ प्राव॒णेभिः॑ स॒जोष॑सः । \newline
18. प्रा॒व॒णेभि॒रिति॑ प्र - व॒नेभिः॑ । \newline
19. स॒जोष॑स॒ इति॑ स - जोष॑सः । \newline
20. जु॒षन्ताꣳ॑ ह॒व्यꣳ ह॒व्यम् जु॒षन्ता᳚म् जु॒षन्ताꣳ॑ ह॒व्यम् । \newline
21. ह॒व्य माहु॑त॒ माहु॑तꣳ ह॒व्यꣳ ह॒व्य माहु॑तम् । \newline
22. आहु॑त मनमी॒वा अ॑नमी॒वा आहु॑त॒ माहु॑त मनमी॒वाः । \newline
23. आहु॑त॒मित्या - हु॒त॒म् । \newline
24. अ॒न॒मी॒वा इष॒ इषो॑ ऽनमी॒वा अ॑नमी॒वा इषः॑ । \newline
25. इषो॑ म॒हीर् म॒ही रिष॒ इषो॑ म॒हीः । \newline
26. म॒हीरिति॑ म॒हीः । \newline
27. इडा॑ मग्ने अग्न॒ इडा॒ मिडा॑ मग्ने । \newline
28. अ॒ग्ने॒ पु॒रु॒दꣳस॑म् पुरु॒दꣳस॑ मग्ने अग्ने पुरु॒दꣳस᳚म् । \newline
29. पु॒रु॒दꣳसꣳ॑ स॒निꣳ स॒निम् पु॑रु॒दꣳस॑म् पुरु॒दꣳसꣳ॑ स॒निम् । \newline
30. पु॒रु॒दꣳस॒मिति॑ पुरु - दꣳस᳚म् । \newline
31. स॒निम् गोर् गोः स॒निꣳ स॒निम् गोः । \newline
32. गोः श॑श्वत्त॒मꣳ श॑श्वत्त॒मम् गोर् गोः श॑श्वत्त॒मम् । \newline
33. श॒श्व॒त्त॒मꣳ हव॑मानाय॒ हव॑मानाय शश्वत्त॒मꣳ श॑श्वत्त॒मꣳ हव॑मानाय । \newline
34. श॒श्व॒त्त॒ममिति॑ शश्वत् - त॒मम् । \newline
35. हव॑मानाय साध साध॒ हव॑मानाय॒ हव॑मानाय साध । \newline
36. सा॒धेति॑ साध । \newline
37. स्यान् नो॑ नः॒ स्याथ् स्यान् नः॑ । \newline
38. नः॒ सू॒नुः सू॒नुर् नो॑ नः सू॒नुः । \newline
39. सू॒नु स्तन॑य॒ स्तन॑यः सू॒नुः सू॒नु स्तन॑यः । \newline
40. तन॑यो वि॒जावा॑ वि॒जावा॒ तन॑य॒ स्तन॑यो वि॒जावा᳚ । \newline
41. वि॒जावा ऽग्ने ऽग्ने॑ वि॒जावा॑ वि॒जावा ऽग्ने᳚ । \newline
42. वि॒जावेति॑ वि - जावा᳚ । \newline
43. अग्ने॒ सा सा ऽग्ने ऽग्ने॒ सा । \newline
44. सा ते॑ ते॒ सा सा ते᳚ । \newline
45. ते॒ सु॒म॒तिः सु॑म॒तिष्टे॑ ते सुम॒तिः । \newline
46. सु॒म॒तिर् भू॑तु भूतु सुम॒तिः सु॑म॒तिर् भू॑तु । \newline
47. सु॒म॒तिरिति॑ सु - म॒तिः । \newline
48. भू॒त्व॒स्मे अ॒स्मे भू॑तु भूत्व॒स्मे । \newline
49. अ॒स्मे इत्य॒स्मे । \newline
50. अ॒यम् ते॑ ते॒ ऽय म॒यम् ते᳚ । \newline
51. ते॒ योनि॒र् योनि॑ स्ते ते॒ योनिः॑ । \newline
52. योनिर्॑. ऋ॒त्विय॑ ऋ॒त्वियो॒ योनि॒र् योनिर्॑. ऋ॒त्वियः॑ । \newline
53. ऋ॒त्वियो॒ यतो॒ यत॑ ऋ॒त्विय॑ ऋ॒त्वियो॒ यतः॑ । \newline
54. यतो॑ जा॒तो जा॒तो यतो॒ यतो॑ जा॒तः । \newline
55. जा॒तो अरो॑चथा॒ अरो॑चथा जा॒तो जा॒तो अरो॑चथाः । \newline
56. अरो॑चथा॒ इत्यरो॑चथाः । \newline
57. तम् जा॒नन् जा॒नन् तम् तम् जा॒नन्न् । \newline
58. जा॒नन् न॑ग्ने अग्ने जा॒नन् जा॒नन् न॑ग्ने । \newline

\textbf{Ghana Paata } \newline

1. अ॒फ्सु वा॑ वा॒ ऽफ्स्व॑फ्सु वा यजत्र यजत्र वा॒ ऽफ्स्व॑फ्सु वा यजत्र । \newline
2. अ॒फ्स्वित्य॑प् - सु । \newline
3. वा॒ य॒ज॒त्र॒ य॒ज॒त्र॒ वा॒ वा॒ य॒ज॒त्र॒ । \newline
4. य॒ज॒त्रेति॑ यजत्र । \newline
5. येना॒न्तरि॑क्ष म॒न्तरि॑क्षं॒ ॅयेन॒ येना॒न्तरि॑क्ष मु॒रू᳚(1॒)र्व॑न्तरि॑क्षं॒ ॅयेन॒ येना॒न्तरि॑क्ष मु॒रु । \newline
6. अ॒न्तरि॑क्ष मु॒रू᳚(1॒)र्व॑न्तरि॑क्ष म॒न्तरि॑क्ष मु॒र्वा॑ त॒तन्था॑ त॒तन्थो॒र्व॑न्तरि॑क्ष म॒न्तरि॑क्ष मु॒र्वा॑त॒तन्थ॑ । \newline
7. उ॒र्वा॑त॒तन्था॑ त॒तन्थो॒रू᳚(1॒)र्वा॑त॒तन्थ॑ त्वे॒ष स्त्वे॒ष आ॑त॒तन्थो॒ रू᳚(1॒)र्वा॑त॒तन्थ॑ त्वे॒षः । \newline
8. आ॒त॒तन्थ॑ त्वे॒ष स्त्वे॒ष आ॑त॒तन्था॑ त॒तन्थ॑ त्वे॒षः स स त्वे॒ष आ॑त॒तन्था॑ त॒तन्थ॑ त्वे॒षः सः । \newline
9. आ॒त॒तन्थेत्या᳚ - त॒तन्थ॑ । \newline
10. त्वे॒षः स स त्वे॒ष स्त्वे॒षः स भा॒नुर् भा॒नुः स त्वे॒ष स्त्वे॒षः स भा॒नुः । \newline
11. स भा॒नुर् भा॒नुः स स भा॒नु र॑र्ण॒वो अ॑र्ण॒वो भा॒नुः स स भा॒नु र॑र्ण॒वः । \newline
12. भा॒नु र॑र्ण॒वो अ॑र्ण॒वो भा॒नुर् भा॒नु र॑र्ण॒वो नृ॒चक्षा॑ नृ॒चक्षा॑ अर्ण॒वो भा॒नुर् भा॒नु र॑र्ण॒वो नृ॒चक्षाः᳚ । \newline
13. अ॒र्ण॒वो नृ॒चक्षा॑ नृ॒चक्षा॑ अर्ण॒वो अ॑र्ण॒वो नृ॒चक्षाः᳚ । \newline
14. नृ॒चक्षा॒ इति॑ नृ - चक्षाः᳚ । \newline
15. पु॒री॒ष्या॑सो अ॒ग्नयो॑ अ॒ग्नयः॑ पुरी॒ष्या॑सः पुरी॒ष्या॑सो अ॒ग्नयः॑ प्राव॒णेभिः॑ प्राव॒णेभि॑ र॒ग्नयः॑ पुरी॒ष्या॑सः पुरी॒ष्या॑सो अ॒ग्नयः॑ प्राव॒णेभिः॑ । \newline
16. अ॒ग्नयः॑ प्राव॒णेभिः॑ प्राव॒णेभि॑ र॒ग्नयो॑ अ॒ग्नयः॑ प्राव॒णेभिः॑ स॒जोष॑सः स॒जोष॑सः प्राव॒णेभि॑ र॒ग्नयो॑ अ॒ग्नयः॑ प्राव॒णेभिः॑ स॒जोष॑सः । \newline
17. प्रा॒व॒णेभिः॑ स॒जोष॑सः स॒जोष॑सः प्राव॒णेभिः॑ प्राव॒णेभिः॑ स॒जोष॑सः । \newline
18. प्रा॒व॒णेभि॒रिति॑ प्र - व॒नेभिः॑ । \newline
19. स॒जोष॑स॒ इति॑ स - जोष॑सः । \newline
20. जु॒षन्ताꣳ॑ ह॒व्यꣳ ह॒व्यम् जु॒षन्ता᳚म् जु॒षन्ताꣳ॑ ह॒व्य माहु॑त॒ माहु॑तꣳ ह॒व्यम् जु॒षन्ता᳚म् जु॒षन्ताꣳ॑ ह॒व्य माहु॑तम् । \newline
21. ह॒व्य माहु॑त॒ माहु॑तꣳ ह॒व्यꣳ ह॒व्य माहु॑त मनमी॒वा अ॑नमी॒वा आहु॑तꣳ ह॒व्यꣳ ह॒व्य माहु॑त मनमी॒वाः । \newline
22. आहु॑त मनमी॒वा अ॑नमी॒वा आहु॑त॒ माहु॑त मनमी॒वा इष॒ इषो॑ ऽनमी॒वा आहु॑त॒ माहु॑त मनमी॒वा इषः॑ । \newline
23. आहु॑त॒मित्या - हु॒त॒म् । \newline
24. अ॒न॒मी॒वा इष॒ इषो॑ ऽनमी॒वा अ॑नमी॒वा इषो॑ म॒हीर् म॒ही रिषो॑ ऽनमी॒वा अ॑नमी॒वा इषो॑ म॒हीः । \newline
25. इषो॑ म॒हीर् म॒ही रिष॒ इषो॑ म॒हीः । \newline
26. म॒हीरिति॑ म॒हीः । \newline
27. इडा॑ मग्ने अग्न॒ इडा॒ मिडा॑ मग्ने पुरु॒दꣳस॑म् पुरु॒दꣳस॑ मग्न॒ इडा॒ मिडा॑ मग्ने पुरु॒दꣳस᳚म् । \newline
28. अ॒ग्ने॒ पु॒रु॒दꣳस॑म् पुरु॒दꣳस॑ मग्ने अग्ने पुरु॒दꣳसꣳ॑ स॒निꣳ स॒निम् पु॑रु॒दꣳस॑ 
मग्ने अग्ने पुरु॒दꣳसꣳ॑ स॒निम् । \newline
29. पु॒रु॒दꣳसꣳ॑ स॒निꣳ स॒निम् पु॑रु॒दꣳस॑म् पुरु॒दꣳसꣳ॑ स॒निम् गोर् गोः स॒निम् पु॑रु॒दꣳस॑म् पुरु॒दꣳसꣳ॑ स॒निम् गोः । \newline
30. पु॒रु॒दꣳस॒मिति॑ पुरु - दꣳस᳚म् । \newline
31. स॒निम् गोर् गोः स॒निꣳ स॒निम् गोः श॑श्वत्त॒मꣳ श॑श्वत्त॒मम् गोः स॒निꣳ स॒निम् गोः श॑श्वत्त॒मम् । \newline
32. गोः श॑श्वत्त॒मꣳ श॑श्वत्त॒मम् गोर् गोः श॑श्वत्त॒मꣳ हव॑मानाय॒ हव॑मानाय शश्वत्त॒मम् गोर् गोः श॑श्वत्त॒मꣳ हव॑मानाय । \newline
33. श॒श्व॒त्त॒मꣳ हव॑मानाय॒ हव॑मानाय शश्वत्त॒मꣳ श॑श्वत्त॒मꣳ हव॑मानाय साध साध॒ हव॑मानाय शश्वत्त॒मꣳ श॑श्वत्त॒मꣳ हव॑मानाय साध । \newline
34. श॒श्व॒त्त॒ममिति॑ शश्वत् - त॒मम् । \newline
35. हव॑मानाय साध साध॒ हव॑मानाय॒ हव॑मानाय साध । \newline
36. सा॒धेति॑ साध । \newline
37. स्यान् नो॑ नः॒ स्याथ् स्यान् नः॑ सू॒नुः सू॒नुर् नः॒ स्याथ् स्यान् नः॑ सू॒नुः । \newline
38. नः॒ सू॒नुः सू॒नुर् नो॑ नः सू॒नु स्तन॑य॒ स्तन॑यः सू॒नुर् नो॑ नः सू॒नु स्तन॑यः । \newline
39. सू॒नु स्तन॑य॒ स्तन॑यः सू॒नुः सू॒नु स्तन॑यो वि॒जावा॑ वि॒जावा॒ तन॑यः सू॒नुः सू॒नु स्तन॑यो वि॒जावा᳚ । \newline
40. तन॑यो वि॒जावा॑ वि॒जावा॒ तन॑य॒ स्तन॑यो वि॒जावा ऽग्ने ऽग्ने॑ वि॒जावा॒ तन॑य॒ स्तन॑यो वि॒जावा ऽग्ने᳚ । \newline
41. वि॒जावा ऽग्ने ऽग्ने॑ वि॒जावा॑ वि॒जावा ऽग्ने॒ सा सा ऽग्ने॑ वि॒जावा॑ वि॒जावा ऽग्ने॒ सा । \newline
42. वि॒जावेति॑ वि - जावा᳚ । \newline
43. अग्ने॒ सा सा ऽग्ने ऽग्ने॒ सा ते॑ ते॒ सा ऽग्ने ऽग्ने॒ सा ते᳚ । \newline
44. सा ते॑ ते॒ सा सा ते॑ सुम॒तिः सु॑म॒तिष्टे॒ सा सा ते॑ सुम॒तिः । \newline
45. ते॒ सु॒म॒तिः सु॑म॒तिष्टे॑ ते सुम॒तिर् भू॑तु भूतु सुम॒तिष्टे॑ ते सुम॒तिर् भू॑तु । \newline
46. सु॒म॒तिर् भू॑तु भूतु सुम॒तिः सु॑म॒तिर् भू᳚त्व॒स्मे अ॒स्मे भू॑तु सुम॒तिः सु॑म॒तिर् भू᳚त्व॒स्मे । \newline
47. सु॒म॒तिरिति॑ सु - म॒तिः । \newline
48. भू॒त्व॒स्मे अ॒स्मे भू॑तु भूत्व॒स्मे । \newline
49. अ॒स्मे इत्य॒स्मे । \newline
50. अ॒यम् ते॑ ते॒ ऽय म॒यम् ते॒ योनि॒र् योनि॑ स्ते॒ ऽय म॒यम् ते॒ योनिः॑ । \newline
51. ते॒ योनि॒र् योनि॑ स्ते ते॒ योनिर्॑. ऋ॒त्विय॑ ऋ॒त्वियो॒ योनि॑ स्ते ते॒ योनिर्॑. ऋ॒त्वियः॑ । \newline
52. योनिर्॑. ऋ॒त्विय॑ ऋ॒त्वियो॒ योनि॒र् योनिर्॑. ऋ॒त्वियो॒ यतो॒ यत॑ ऋ॒त्वियो॒ योनि॒र् योनिर्॑. ऋ॒त्वियो॒ यतः॑ । \newline
53. ऋ॒त्वियो॒ यतो॒ यत॑ ऋ॒त्विय॑ ऋ॒त्वियो॒ यतो॑ जा॒तो जा॒तो यत॑ ऋ॒त्विय॑ ऋ॒त्वियो॒ यतो॑ जा॒तः । \newline
54. यतो॑ जा॒तो जा॒तो यतो॒ यतो॑ जा॒तो अरो॑चथा॒ अरो॑चथा जा॒तो यतो॒ यतो॑ जा॒तो अरो॑चथाः । \newline
55. जा॒तो अरो॑चथा॒ अरो॑चथा जा॒तो जा॒तो अरो॑चथाः । \newline
56. अरो॑चथा॒ इत्यरो॑चथाः । \newline
57. तम् जा॒नन् जा॒नन् तम् तम् जा॒नन् न॑ग्ने अग्ने जा॒नन् तम् तम् जा॒नन् न॑ग्ने । \newline
58. जा॒नन् न॑ग्ने अग्ने जा॒नन् जा॒नन् न॑ग्न॒ आ ऽग्ने॑ जा॒नन् जा॒नन् न॑ग्न॒ आ । \newline
\pagebreak
\markright{ TS 4.2.4.4  \hfill https://www.vedavms.in \hfill}

\section{ TS 4.2.4.4 }

\textbf{TS 4.2.4.4 } \newline
\textbf{Samhita Paata} \newline

न्न॑ग्न॒ आ रो॒हाथा॑ नो वर्द्धया र॒यिं ॥ चिद॑सि॒ तया॑ दे॒वत॑याऽङ्गिर॒स्वद्-ध्रु॒वा सी॑द परि॒चिद॑सि॒ तया॑ दे॒वत॑याऽङ्गिर॒स्वद् ध्रु॒वा सी॑द लो॒कं पृ॑ण छि॒द्रं पृ॒णाथो॑ सीद शि॒वा त्वं । इ॒न्द्रा॒ग्नी त्वा॒ बृह॒स्पति॑र॒स्मिन्. योना॑वसीषदन्न् ॥ ता अ॑स्य॒ सूद॑दोहसः॒ सोमꣳ॑ श्रीणन्ति॒ पृश्न॑यः । जन्म॑न् दे॒वानां॒ ॅविश॑स्त्रि॒ष्वा रो॑च॒ने दि॒वः ॥ \newline

\textbf{Pada Paata} \newline

अ॒ग्ने॒ । एति॑ । रो॒ह॒ । अथ॑ । नः॒ । व॒द्‌र्ध॒य॒ । र॒यिम् ॥ चित् । अ॒सि॒ । तया᳚ । दे॒वत॑या । अ॒ङ्गि॒र॒स्वत् । ध्रु॒वा । सी॒द॒ । प॒रि॒चिदिति॑ परि - चित् । अ॒सि॒ । तया᳚ । दे॒वत॑या । अ॒ङ्गि॒र॒स्वत् । ध्रु॒वा । सी॒द॒ । लो॒कम् । पृ॒ण॒ । छि॒द्रम् । पृ॒ण॒ । अथो॒ इति॑ । सी॒द॒ । शि॒वा । त्वम् ॥ इ॒न्द्रा॒ग्नी इती᳚न्द्र - अ॒ग्नी । त्वा॒ । बृह॒स्पतिः॑ । अ॒स्मिन्न् । योनौ᳚ । अ॒सी॒ष॒द॒न्न् ॥ ताः । अ॒स्य॒ । सूद॑दोहस॒ इति॒ सूद॑ - दो॒ह॒सः॒ । सोम᳚म् । श्री॒ण॒न्ति॒ । पृश्न॑यः ॥ जन्मन्न्॑ । दे॒वाना᳚म् । विशः॑ । त्रि॒षु । एति॑ । रो॒च॒ने । दि॒वः ॥  \newline


\textbf{Krama Paata} \newline

अ॒ग्न॒ आ । आ रो॑ह । रो॒हाथ॑ । अथा॑ नः । नो॒ व॒र्द्ध॒य॒ । व॒र्द्ध॒या॒ र॒यिम् । र॒यिमिति॑ र॒यिम् ॥ चिद॑सि । अ॒सि॒ तया᳚ । तया॑ दे॒वत॑या । दे॒वत॑या ऽङ्गिर॒स्वत् । अ॒ङ्गि॒र॒स्वद् ध्रु॒वा । ध्रु॒वा सी॑द । सी॒द॒ प॒रि॒चित् । प॒रि॒चिद॑सि । प॒रि॒चिदिति॑ परि - चित् । अ॒सि॒ तया᳚ । तया॑ दे॒वत॑या । दे॒वत॑या ऽङ्गिर॒स्वत् । अ॒ङ्गि॒र॒स्वद् ध्रु॒वा । ध्रु॒वा सी॑द । सी॒द॒ लो॒कम् । लो॒कम् पृ॑ण । पृ॒ण॒ छि॒द्रम् । छि॒द्रम् पृ॑ण । पृ॒णाथो᳚ । अथो॑ सीद । अथो॒ इत्यथो᳚ । सी॒द॒ शि॒वा । शि॒वा त्वम् । त्वमिति॒ त्वम् ॥ इ॒न्द्रा॒ग्नी त्वा᳚ । इ॒न्द्रा॒ग्नी इती᳚न्द्र - अ॒ग्नी । त्वा॒ बृह॒स्पतिः॑ । बृह॒स्पति॑र॒स्मिन्न् । अ॒स्मिन्. योनौ᳚ । योना॑वसीषदन्न् । अ॒सी॒ष॒द॒न्नित्य॑सीषदन्न् ॥ ता अ॑स्य । अ॒स्य॒ सूद॑दोहसः । सूद॑दोहसः॒ सोम᳚म् । सूद॑दोहस॒ इति॒ सूद॑ - दो॒ह॒सः॒ । सोमꣳ॑ श्रीणन्ति । श्री॒ण॒न्ति॒ पृश्ञ॑यः । पृश्ञ॑य॒ इति॒ पृश्ञ॑यः ॥ जन्म॑न् दे॒वाना᳚म् । दे॒वानां॒ ॅविशः॑ । विश॑स्त्रि॒षु । त्रि॒ष्वा । आ रो॑च॒ने । रो॒च॒ने दि॒वः । दि॒व इति॑ दि॒वः । \newline

\textbf{Jatai Paata} \newline

1. अ॒ग्न॒ आ ऽग्ने॑ अग्न॒ आ । \newline
2. आ रो॑ह रो॒हा रो॑ह । \newline
3. रो॒हा थाथ॑ रोह रो॒हाथ॑ । \newline
4. अथा॑ नो नो॒ अथाथा॑ नः । \newline
5. नो॒ व॒र्द्ध॒य॒ व॒र्द्ध॒य॒ नो॒ नो॒ व॒र्द्ध॒य॒ । \newline
6. व॒र्द्ध॒या॒ र॒यिꣳ र॒यिं ॅव॑र्द्धय वर्द्धया र॒यिम् । \newline
7. र॒यिमिति॑ र॒यिम् । \newline
8. चिद॑स्यसि॒ चिच् चिद॑सि । \newline
9. अ॒सि॒ तया॒ तया᳚ ऽस्यसि॒ तया᳚ । \newline
10. तया॑ दे॒वत॑या दे॒वत॑या॒ तया॒ तया॑ दे॒वत॑या । \newline
11. दे॒वत॑या ऽङ्गिर॒स्व द॑ङ्गिर॒स्वद् दे॒वत॑या दे॒वत॑या ऽङ्गिर॒स्वत् । \newline
12. अ॒ङ्गि॒र॒स्वद् ध्रु॒वा ध्रु॒वा ऽङ्गि॑र॒स्व द॑ङ्गिर॒स्वद् ध्रु॒वा । \newline
13. ध्रु॒वा सी॑द सीद ध्रु॒वा ध्रु॒वा सी॑द । \newline
14. सी॒द॒ प॒रि॒चित् प॑रि॒चिथ् सी॑द सीद परि॒चित् । \newline
15. प॒रि॒चि द॑स्यसि परि॒चित् प॑रि॒चि द॑सि । \newline
16. प॒रि॒चिदिति॑ परि - चित् । \newline
17. अ॒सि॒ तया॒ तया᳚ ऽस्यसि॒ तया᳚ । \newline
18. तया॑ दे॒वत॑या दे॒वत॑या॒ तया॒ तया॑ दे॒वत॑या । \newline
19. दे॒वत॑या ऽङ्गिर॒स्व द॑ङ्गिर॒स्वद् दे॒वत॑या दे॒वत॑या ऽङ्गिर॒स्वत् । \newline
20. अ॒ङ्गि॒र॒स्वद् ध्रु॒वा ध्रु॒वा ऽङ्गि॑र॒स्व द॑ङ्गिर॒स्वद् ध्रु॒वा । \newline
21. ध्रु॒वा सी॑द सीद ध्रु॒वा ध्रु॒वा सी॑द । \newline
22. सी॒द॒ लो॒कम् ॅलो॒कꣳ सी॑द सीद लो॒कम् । \newline
23. लो॒कम् पृ॑ण पृण लो॒कम् ॅलो॒कम् पृ॑ण । \newline
24. पृ॒ण॒ छि॒द्रम् छि॒द्रम् पृ॑ण पृण छि॒द्रम् । \newline
25. छि॒द्रम् पृ॑ण पृण छि॒द्रम् छि॒द्रम् पृ॑ण । \newline
26. पृ॒णाथो॒ अथो॑ पृण पृ॒णाथो᳚ । \newline
27. अथो॑ सीद सी॒दाथो॒ अथो॑ सीद । \newline
28. अथो॒ इत्यथो᳚ । \newline
29. सी॒द॒ शि॒वा शि॒वा सी॑द सीद शि॒वा । \newline
30. शि॒वा त्वम् त्वꣳ शि॒वा शि॒वा त्वम् । \newline
31. त्वमिति॒ त्वम् । \newline
32. इ॒न्द्रा॒ग्नी त्वा᳚ त्वेन्द्रा॒ग्नी इ॑न्द्रा॒ग्नी त्वा᳚ । \newline
33. इ॒न्द्रा॒ग्नी इती᳚न्द्र - अ॒ग्नी । \newline
34. त्वा॒ बृह॒स्पति॒र् बृह॒स्पति॑ स्त्वा त्वा॒ बृह॒स्पतिः॑ । \newline
35. बृह॒स्पति॑ र॒स्मिन् न॒स्मिन् बृह॒स्पति॒र् बृह॒स्पति॑ र॒स्मिन्न् । \newline
36. अ॒स्मिन्. योनौ॒ योना॑ व॒स्मिन् न॒स्मिन्. योनौ᳚ । \newline
37. योना॑ वसीषदन् नसीषद॒न्॒. योनौ॒ योना॑ वसीषदन्न् । \newline
38. अ॒सी॒ष॒द॒न् नित्य॑सीषदन्न् । \newline
39. ता अ॑स्यास्य॒ ता स्ता अ॑स्य । \newline
40. अ॒स्य॒ सूद॑दोहसः॒ सूद॑दोहसो अस्यास्य॒ सूद॑दोहसः । \newline
41. सूद॑दोहसः॒ सोमꣳ॒॒ सोमꣳ॒॒ सूद॑दोहसः॒ सूद॑दोहसः॒ सोम᳚म् । \newline
42. सूद॑दोहस॒ इति॒ सूद॑ - दो॒ह॒सः॒ । \newline
43. सोमꣳ॑ श्रीणन्ति श्रीणन्ति॒ सोमꣳ॒॒ सोमꣳ॑ श्रीणन्ति । \newline
44. श्री॒ण॒न्ति॒ पृश्ञ॑यः॒ पृश्ञ॑यः श्रीणन्ति श्रीणन्ति॒ पृश्ञ॑यः । \newline
45. पृश्ञ॑य॒ इति॒ पृश्ञ॑यः । \newline
46. जन्म॑न् दे॒वाना᳚म् दे॒वाना॒म् जन्म॒न् जन्म॑न् दे॒वाना᳚म् । \newline
47. दे॒वानां॒ ॅविशो॒ विशो॑ दे॒वाना᳚म् दे॒वानां॒ ॅविशः॑ । \newline
48. विश॑ स्त्रि॒षु त्रि॒षु विशो॒ विश॑ स्त्रि॒षु । \newline
49. त्रि॒ष्वा त्रि॒षु त्रि॒ष्वा । \newline
50. आ रो॑च॒ने रो॑च॒न आ रो॑च॒ने । \newline
51. रो॒च॒ने दि॒वो दि॒वो रो॑च॒ने रो॑च॒ने दि॒वः । \newline
52. दि॒व इति॑ दि॒वः । \newline

\textbf{Ghana Paata } \newline

1. अ॒ग्न॒ आ ऽग्ने॑ अग्न॒ आ रो॑ह रो॒हा ऽग्ने॑ अग्न॒ आ रो॑ह । \newline
2. आ रो॑ह रो॒हा रो॒हा थाथ॑ रो॒हा रो॒हाथ॑ । \newline
3. रो॒हाथाथ॑ रोह रो॒हाथा॑ नो नो॒ अथ॑ रोह रो॒हाथा॑ नः । \newline
4. अथा॑ नो नो॒ अथाथा॑ नो वर्द्धय वर्द्धय नो॒ अथाथा॑ नो वर्द्धय । \newline
5. नो॒ व॒र्द्ध॒य॒ व॒र्द्ध॒य॒ नो॒ नो॒ व॒र्द्ध॒या॒ र॒यिꣳ र॒यिं ॅव॑र्द्धय नो नो वर्द्धया र॒यिम् । \newline
6. व॒र्द्ध॒या॒ र॒यिꣳ र॒यिं ॅव॑र्द्धय वर्द्धया र॒यिम् । \newline
7. र॒यिमिति॑ र॒यिम् । \newline
8. चिद॑स्यसि॒ चिच् चिद॑सि॒ तया॒ तया॑ ऽसि॒ चिच् चिद॑सि॒ तया᳚ । \newline
9. अ॒सि॒ तया॒ तया᳚ ऽस्यसि॒ तया॑ दे॒वत॑या दे॒वत॑या॒ तया᳚ ऽस्यसि॒ तया॑ दे॒वत॑या । \newline
10. तया॑ दे॒वत॑या दे॒वत॑या॒ तया॒ तया॑ दे॒वत॑या ऽङ्गिर॒स्व द॑ङ्गिर॒स्वद् दे॒वत॑या॒ तया॒ तया॑ दे॒वत॑या ऽङ्गिर॒स्वत् । \newline
11. दे॒वत॑या ऽङ्गिर॒स्व द॑ङ्गिर॒स्वद् दे॒वत॑या दे॒वत॑या ऽङ्गिर॒स्वद् ध्रु॒वा ध्रु॒वा ऽङ्गि॑र॒स्वद् दे॒वत॑या दे॒वत॑या ऽङ्गिर॒स्वद् ध्रु॒वा । \newline
12. अ॒ङ्गि॒र॒स्वद् ध्रु॒वा ध्रु॒वा ऽङ्गि॑र॒स्व द॑ङ्गिर॒स्वद् ध्रु॒वा सी॑द सीद ध्रु॒वा ऽङ्गि॑र॒स्व द॑ङ्गिर॒स्वद् ध्रु॒वा सी॑द । \newline
13. ध्रु॒वा सी॑द सीद ध्रु॒वा ध्रु॒वा सी॑द परि॒चित् प॑रि॒चिथ् सी॑द ध्रु॒वा ध्रु॒वा सी॑द परि॒चित् । \newline
14. सी॒द॒ प॒रि॒चित् प॑रि॒चिथ् सी॑द सीद परि॒चि द॑स्यसि परि॒चिथ् सी॑द सीद परि॒चिद॑सि । \newline
15. प॒रि॒चि द॑स्यसि परि॒चित् प॑रि॒चि द॑सि॒ तया॒ तया॑ ऽसि परि॒चित् प॑रि॒चि द॑सि॒ तया᳚ । \newline
16. प॒रि॒चिदिति॑ परि - चित् । \newline
17. अ॒सि॒ तया॒ तया᳚ ऽस्यसि॒ तया॑ दे॒वत॑या दे॒वत॑या॒ तया᳚ ऽस्यसि॒ तया॑ दे॒वत॑या । \newline
18. तया॑ दे॒वत॑या दे॒वत॑या॒ तया॒ तया॑ दे॒वत॑या ऽङ्गिर॒स्व द॑ङ्गिर॒स्वद् दे॒वत॑या॒ तया॒ तया॑ दे॒वत॑या ऽङ्गिर॒स्वत् । \newline
19. दे॒वत॑या ऽङ्गिर॒स्व द॑ङ्गिर॒स्वद् दे॒वत॑या दे॒वत॑या ऽङ्गिर॒स्वद् ध्रु॒वा ध्रु॒वा ऽङ्गि॑र॒स्वद् दे॒वत॑या दे॒वत॑या ऽङ्गिर॒स्वद् ध्रु॒वा । \newline
20. अ॒ङ्गि॒र॒स्वद् ध्रु॒वा ध्रु॒वा ऽङ्गि॑र॒स्व द॑ङ्गिर॒स्वद् ध्रु॒वा सी॑द सीद ध्रु॒वा ऽङ्गि॑र॒स्व द॑ङ्गिर॒स्वद् ध्रु॒वा सी॑द । \newline
21. ध्रु॒वा सी॑द सीद ध्रु॒वा ध्रु॒वा सी॑द लो॒कम् ॅलो॒कꣳ सी॑द ध्रु॒वा ध्रु॒वा सी॑द लो॒कम् । \newline
22. सी॒द॒ लो॒कम् ॅलो॒कꣳ सी॑द सीद लो॒कम् पृ॑ण पृण लो॒कꣳ सी॑द सीद लो॒कम् पृ॑ण । \newline
23. लो॒कम् पृ॑ण पृण लो॒कम् ॅलो॒कम् पृ॑ण छि॒द्रम् छि॒द्रम् पृ॑ण लो॒कम् ॅलो॒कम् पृ॑ण छि॒द्रम् । \newline
24. पृ॒ण॒ छि॒द्रम् छि॒द्रम् पृ॑ण पृण छि॒द्रम् पृ॑ण पृण छि॒द्रम् पृ॑ण पृण छि॒द्रम् पृ॑ण । \newline
25. छि॒द्रम् पृ॑ण पृण छि॒द्रम् छि॒द्रम् पृ॒णाथो॒ अथो॑ पृण छि॒द्रम् छि॒द्रम् पृ॒णाथो᳚ । \newline
26. पृ॒णाथो॒ अथो॑ पृण पृ॒णाथो॑ सीद सी॒दाथो॑ पृण पृ॒णाथो॑ सीद । \newline
27. अथो॑ सीद सी॒दाथो॒ अथो॑ सीद शि॒वा शि॒वा सी॒दाथो॒ अथो॑ सीद शि॒वा । \newline
28. अथो॒ इत्यथो᳚ । \newline
29. सी॒द॒ शि॒वा शि॒वा सी॑द सीद शि॒वा त्वम् त्वꣳ शि॒वा सी॑द सीद शि॒वा त्वम् । \newline
30. शि॒वा त्वम् त्वꣳ शि॒वा शि॒वा त्वम् । \newline
31. त्वमिति॒ त्वम् । \newline
32. इ॒न्द्रा॒ग्नी त्वा᳚ त्वेन्द्रा॒ग्नी इ॑न्द्रा॒ग्नी त्वा॒ बृह॒स्पति॒र् बृह॒स्पति॑ स्त्वेन्द्रा॒ग्नी इ॑न्द्रा॒ग्नी त्वा॒ बृह॒स्पतिः॑ । \newline
33. इ॒न्द्रा॒ग्नी इती᳚न्द्र - अ॒ग्नी । \newline
34. त्वा॒ बृह॒स्पति॒र् बृह॒स्पति॑ स्त्वा त्वा॒ बृह॒स्पति॑ र॒स्मिन् न॒स्मिन् बृह॒स्पति॑ स्त्वा त्वा॒ बृह॒स्पति॑ र॒स्मिन्न् । \newline
35. बृह॒स्पति॑ र॒स्मिन् न॒स्मिन् बृह॒स्पति॒र् बृह॒स्पति॑ र॒स्मिन्. योनौ॒ योना॑ व॒स्मिन् बृह॒स्पति॒र् बृह॒स्पति॑ र॒स्मिन्. योनौ᳚ । \newline
36. अ॒स्मिन्. योनौ॒ योना॑ व॒स्मिन् न॒स्मिन्. योना॑ वसीषदन् नसीषद॒न्॒. योना॑ व॒स्मिन् न॒स्मिन्. योना॑ वसीषदन्न् । \newline
37. योना॑ वसीषदन् नसीषद॒न्॒. योनौ॒ योना॑ वसीषदन्न् । \newline
38. अ॒सी॒ष॒द॒न्नित्य॑सीषदन्न् । \newline
39. ता अ॑स्यास्य॒ ता स्ता अ॑स्य॒ सूद॑दोहसः॒ सूद॑दोहसो अस्य॒ तास्ता अ॑स्य॒ सूद॑दोहसः । \newline
40. अ॒स्य॒ सूद॑दोहसः॒ सूद॑दोहसो अस्यास्य॒ सूद॑दोहसः॒ सोमꣳ॒॒ सोमꣳ॒॒ सूद॑दोहसो अस्यास्य॒ सूद॑दोहसः॒ सोम᳚म् । \newline
41. सूद॑दोहसः॒ सोमꣳ॒॒ सोमꣳ॒॒ सूद॑दोहसः॒ सूद॑दोहसः॒ सोमꣳ॑ श्रीणन्ति श्रीणन्ति॒ सोमꣳ॒॒ सूद॑दोहसः॒ सूद॑दोहसः॒ सोमꣳ॑ श्रीणन्ति । \newline
42. सूद॑दोहस॒ इति॒ सूद॑ - दो॒ह॒सः॒ । \newline
43. सोमꣳ॑ श्रीणन्ति श्रीणन्ति॒ सोमꣳ॒॒ सोमꣳ॑ श्रीणन्ति॒ पृश्ञ॑यः॒ पृश्ञ॑यः श्रीणन्ति॒ सोमꣳ॒॒ सोमꣳ॑ श्रीणन्ति॒ पृश्ञ॑यः । \newline
44. श्री॒ण॒न्ति॒ पृश्ञ॑यः॒ पृश्ञ॑यः श्रीणन्ति श्रीणन्ति॒ पृश्ञ॑यः । \newline
45. पृश्ञ॑य॒ इति॒ पृश्ञ॑यः । \newline
46. जन्म॑न् दे॒वाना᳚म् दे॒वाना॒म् जन्म॒न् जन्म॑न् दे॒वानां॒ ॅविशो॒ विशो॑ दे॒वाना॒म् जन्म॒न् जन्म॑न् दे॒वानां॒ ॅविशः॑ । \newline
47. दे॒वानां॒ ॅविशो॒ विशो॑ दे॒वाना᳚म् दे॒वानां॒ ॅविश॑ स्त्रि॒षु त्रि॒षु विशो॑ दे॒वाना᳚म् दे॒वानां॒ ॅविश॑ स्त्रि॒षु । \newline
48. विश॑ स्त्रि॒षु त्रि॒षु विशो॒ विश॑ स्त्रि॒ष्वा त्रि॒षु विशो॒ विश॑ स्त्रि॒ष्वा । \newline
49. त्रि॒ष्वा त्रि॒षु त्रि॒ष्वा रो॑च॒ने रो॑च॒न आ त्रि॒षु त्रि॒ष्वा रो॑च॒ने । \newline
50. आ रो॑च॒ने रो॑च॒न आ रो॑च॒ने दि॒वो दि॒वो रो॑च॒न आ रो॑च॒ने दि॒वः । \newline
51. रो॒च॒ने दि॒वो दि॒वो रो॑च॒ने रो॑च॒ने दि॒वः । \newline
52. दि॒व इति॑ दि॒वः । \newline
\pagebreak
\markright{ TS 4.2.5.1  \hfill https://www.vedavms.in \hfill}

\section{ TS 4.2.5.1 }

\textbf{TS 4.2.5.1 } \newline
\textbf{Samhita Paata} \newline

समि॑तꣳ॒॒ सङ्क॑ल्पेथाꣳ॒॒ संप्रि॑यौ रोचि॒ष्णू सु॑मन॒स्यमा॑नौ । इष॒मूर्ज॑म॒भि सं॒ॅवसा॑नौ॒ सं ॅवां॒ मनाꣳ॑सि॒ सं ॅव्र॒ता समु॑ चि॒त्तान्याऽक॑रं ॥ अग्ने॑ पुरीष्याधि॒पा भ॑वा॒ त्वं नः॑ । इष॒मूर्जं॒ ॅयज॑मानाय धेहि ॥ पु॒री॒ष्य॑स्त्वम॑ग्ने रयि॒मान् पु॑ष्टि॒माꣳ अ॑सि । शि॒वाः कृ॒त्वा दिशः॒ सर्वाः॒ स्वां ॅयोनि॑मि॒हाऽस॑दः ॥ भव॑तं नः॒ सम॑नसौ॒ समो॑कसा - [  ] \newline

\textbf{Pada Paata} \newline

समिति॑ । इ॒त॒म् । समिति॑ । क॒ल्पे॒था॒म् । संप्रि॑या॒विति॒ सं - प्रि॒यौ॒ । रो॒चि॒ष्णू इति॑ । सु॒म॒न॒स्यमा॑ना॒विति॑ सु - म॒न॒स्यमा॑नौ ॥ इष᳚म् । ऊर्ज᳚म् । अ॒भीति॑ । सं॒ॅवसा॑ना॒विति॑ सं - वसा॑नौ । समिति॑ । वा॒म् । मनाꣳ॑सि । समिति॑ । व्र॒ता । समिति॑ । उ॒ । चि॒त्तानि॑ । एति॑ । अ॒क॒र॒म् ॥ अग्ने᳚ । पु॒री॒ष्य॒ । अ॒धि॒पा इत्य॑धि - पाः । भ॒व॒ । त्वम् । नः॒ ॥ इष᳚म् । ऊर्ज᳚म् । यज॑मानाय । धे॒हि॒ ॥ पु॒री॒ष्यः॑ । त्वम् । अ॒ग्ने॒ । र॒यि॒मानिति॑ रयि - मान् । पु॒ष्टि॒मानिति॑ पुष्टि-मान् । अ॒सि॒ ॥ शि॒वाः । कृ॒त्वा । दिशः॑ । सर्वाः᳚ । स्वाम् । योनि᳚म् । इ॒ह । एति॑ । अ॒स॒दः॒ ॥ भव॑तम् । नः॒ । सम॑नसा॒विति॒ स - म॒न॒सौ॒ । समो॑कसा॒विति॒ सं - ओ॒क॒सौ॒ ।  \newline


\textbf{Krama Paata} \newline

समि॑तम् । इ॒तꣳ॒॒ सम् । सम् क॑ल्पेथाम् । क॒ल्पे॒थाꣳ॒॒ सम्प्रि॑यौ । सम्प्रि॑यौ रोचि॒ष्णू । सम्प्रि॑या॒विति॒ सम् - प्रि॒यौ॒ । रो॒चि॒ष्णू सु॑मन॒स्यमा॑नौ । रो॒चि॒ष्णू इति॑ रोचि॒ष्णू । सु॒म॒न॒स्यमा॑ना॒विति॑ सु - म॒न॒स्यमा॑नौ ॥ इष॒मूर्ज᳚म् । ऊर्ज॑म॒भि । अ॒भि स॒म्ॅवसा॑नौ । स॒म्ॅवसा॑नौ॒ सम् । स॒म्ॅवसा॑ना॒विति॑ सम् - वसा॑नौ । सं ॅवा᳚म् । वा॒म् मनाꣳ॑सि । मनाꣳ॑सि॒ सम् । सं ॅव्र॒ता । व्र॒ता सम् । समु॑ । उ॒ चि॒त्तानि॑ । चि॒त्तान्या । आऽक॑रम् । अ॒क॒र॒मित्य॑करम् ॥ अग्ने॑ पुरीष्य । पु॒री॒ष्या॒धि॒पाः । अ॒धि॒पा भ॑व । अ॒धि॒पा इत्य॑धि - पाः । भ॒वा॒ त्वम् । त्वम् नः॑ । न॒ इति॑ नः ॥ इष॒मूर्ज᳚म् । ऊर्जं॒ ॅयज॑मानाय । यज॑मानाय धेहि । धे॒हीति॑ धेहि ॥ पु॒री॒ष्य॑स्त्वम् । त्वम॑ग्ने । अ॒ग्ने॒ र॒यि॒मान् । र॒यि॒मान् पु॑ष्टि॒मान् । र॒यि॒मानिति॑ रयि - मान् । पु॒ष्टि॒माꣳ अ॑सि । पु॒ष्टि॒मानिति॑ पुष्टि - मान् । अ॒सीत्य॑सि ॥ शि॒वाः कृ॒त्वा । कृ॒त्वा दिशः॑ । दिशः॒ सर्वाः᳚ । सर्वाः॒ स्वाम् । स्वां ॅयोनि᳚म् । योनि॑मि॒ह । इ॒हा । आऽस॑दः । अ॒स॒द॒ इत्य॑सदः ॥ भव॑तम् नः । नः॒ सम॑नसौ । सम॑नसौ॒ समो॑कसौ । सम॑नसा॒विति॒ स - म॒न॒सौ॒ । समो॑कसावरे॒पसौ᳚ । समो॑कसा॒विति॒ सम् - ओ॒क॒सौ॒ \newline

\textbf{Jatai Paata} \newline

1. स मि॑त मितꣳ॒॒ सꣳ स मि॑तम् । \newline
2. इ॒तꣳ॒॒ सꣳ स मि॑त मितꣳ॒॒ सम् । \newline
3. सम् क॑ल्पेथाम् कल्पेथाꣳ॒॒ सꣳ सम् क॑ल्पेथाम् । \newline
4. क॒ल्पे॒थाꣳ॒॒ संप्रि॑यौ॒ संप्रि॑यौ कल्पेथाम् कल्पेथाꣳ॒॒ संप्रि॑यौ । \newline
5. संप्रि॑यौ रोचि॒ष्णू रो॑चि॒ष्णू संप्रि॑यौ॒ संप्रि॑यौ रोचि॒ष्णू । \newline
6. संप्रि॑या॒विति॒ सं - प्रि॒यौ॒ । \newline
7. रो॒चि॒ष्णू सु॑मन॒स्यमा॑नौ सुमन॒स्यमा॑नौ रोचि॒ष्णू रो॑चि॒ष्णू सु॑मन॒स्यमा॑नौ । \newline
8. रो॒चि॒ष्णू इति॑ रोचि॒ष्णू । \newline
9. सु॒म॒न॒स्यमा॑ना॒विति॑ सु - म॒न॒स्यमा॑नौ । \newline
10. इष॒ मूर्ज॒ मूर्ज॒ मिष॒ मिष॒ मूर्ज᳚म् । \newline
11. ऊर्ज॑ म॒भ्य॑ भ्यूर्ज॒ मूर्ज॑ म॒भि । \newline
12. अ॒भि सं॒ॅवसा॑नौ सं॒ॅवसा॑ना व॒भ्य॑भि सं॒ॅवसा॑नौ । \newline
13. सं॒ॅवसा॑नौ॒ सꣳ सꣳ सं॒ॅवसा॑नौ सं॒ॅवसा॑नौ॒ सम् । \newline
14. सं॒ॅवसा॑ना॒विति॑ सं - वसा॑नौ । \newline
15. सं ॅवां᳚ ॅवाꣳ॒॒ सꣳ सं ॅवा᳚म् । \newline
16. वा॒म् मनाꣳ॑सि॒ मनाꣳ॑सि वां ॅवा॒म् मनाꣳ॑सि । \newline
17. मनाꣳ॑सि॒ सꣳ सम् मनाꣳ॑सि॒ मनाꣳ॑सि॒ सम् । \newline
18. सं ॅव्र॒ता व्र॒ता सꣳ सं ॅव्र॒ता । \newline
19. व्र॒ता सꣳ सं ॅव्र॒ता व्र॒ता सम् । \newline
20. समु॑ वु॒ सꣳ स मु॑ । \newline
21. उ॒ चि॒त्तानि॑ चि॒त्तान्यु॑ वु चि॒त्तानि॑ । \newline
22. चि॒त्तान्या चि॒त्तानि॑ चि॒त्तान्या । \newline
23. आ ऽक॑र मकर॒ मा ऽक॑रम् । \newline
24. अ॒क॒र॒मित्य॑करम् । \newline
25. अग्ने॑ पुरीष्य पुरी॒ष्याग्ने ऽग्ने॑ पुरीष्य । \newline
26. पु॒री॒ ष्या॒धि॒पा अ॑धि॒पाः पु॑रीष्य पुरी ष्याधि॒पाः । \newline
27. अ॒धि॒पा भ॑व भवाधि॒पा अ॑धि॒पा भ॑व । \newline
28. अ॒धि॒पा इत्य॑धि - पाः । \newline
29. भ॒वा॒ त्वम् त्वम् भ॑व भवा॒ त्वम् । \newline
30. त्वम् नो॑ न॒ स्त्वम् त्वम् नः॑ । \newline
31. न॒ इति॑ नः । \newline
32. इष॒ मूर्ज॒ मूर्ज॒ मिष॒ मिष॒ मूर्ज᳚म् । \newline
33. ऊर्जं॒ ॅयज॑मानाय॒ यज॑माना॒ योर्ज॒ मूर्जं॒ ॅयज॑मानाय । \newline
34. यज॑मानाय धेहि धेहि॒ यज॑मानाय॒ यज॑मानाय धेहि । \newline
35. धे॒हीति॑ धेहि । \newline
36. पु॒री॒ष्य॑ स्त्वम् त्वम् पु॑री॒ष्यः॑ पुरी॒ष्य॑ स्त्वम् । \newline
37. त्व म॑ग्ने अग्ने॒ त्वम् त्व म॑ग्ने । \newline
38. अ॒ग्ने॒ र॒यि॒मान् र॑यि॒मा न॑ग्ने अग्ने रयि॒मान् । \newline
39. र॒यि॒मान् पु॑ष्टि॒मान् पु॑ष्टि॒मान् र॑यि॒मान् र॑यि॒मान् पु॑ष्टि॒मान् । \newline
40. र॒यि॒मानिति॑ रयि - मान् । \newline
41. पु॒ष्टि॒माꣳ अ॑स्यसि पुष्टि॒मान् पु॑ष्टि॒माꣳ अ॑सि । \newline
42. पु॒ष्टि॒मानिति॑ पुष्टि - मान् । \newline
43. अ॒सीत्य॑सि । \newline
44. शि॒वाः कृ॒त्वा कृ॒त्वा शि॒वाः शि॒वाः कृ॒त्वा । \newline
45. कृ॒त्वा दिशो॒ दिशः॑ कृ॒त्वा कृ॒त्वा दिशः॑ । \newline
46. दिशः॒ सर्वाः॒ सर्वा॒ दिशो॒ दिशः॒ सर्वाः᳚ । \newline
47. सर्वाः॒ स्वाꣳ स्वाꣳ सर्वाः॒ सर्वाः॒ स्वाम् । \newline
48. स्वां ॅयोनिं॒ ॅयोनिꣳ॒॒ स्वाꣳ स्वां ॅयोनि᳚म् । \newline
49. योनि॑ मि॒हेह योनिं॒ ॅयोनि॑ मि॒ह । \newline
50. इ॒हेहे हा । \newline
51. आ ऽस॑दो असद॒ आ ऽस॑दः । \newline
52. अ॒स॒द॒ इत्य॑सदः । \newline
53. भव॑तम् नो नो॒ भव॑त॒म् भव॑तम् नः । \newline
54. नः॒ सम॑नसौ॒ सम॑नसौ नो नः॒ सम॑नसौ । \newline
55. सम॑नसौ॒ समो॑कसौ॒ समो॑कसौ॒ सम॑नसौ॒ सम॑नसौ॒ समो॑कसौ । \newline
56. सम॑नसा॒विति॒ स - म॒न॒सौ॒ । \newline
57. समो॑कसा वरे॒पसा॑ वरे॒पसौ॒ समो॑कसौ॒ समो॑कसा वरे॒पसौ᳚ । \newline
58. समो॑कसा॒विति॒ सं - ओ॒क॒सौ॒ । \newline

\textbf{Ghana Paata } \newline

1. समि॑त मितꣳ॒॒ सꣳ समि॑तꣳ॒॒ सꣳ समि॑तꣳ॒॒ सꣳ समि॑तꣳ॒॒ सम् । \newline
2. इ॒तꣳ॒॒ सꣳ समि॑त मितꣳ॒॒ सम् क॑ल्पेथाम् कल्पेथाꣳ॒॒ समि॑त मितꣳ॒॒ सम् क॑ल्पेथाम् । \newline
3. सम् क॑ल्पेथाम् कल्पेथाꣳ॒॒ सꣳ सम् क॑ल्पेथाꣳ॒॒ संप्रि॑यौ॒ संप्रि॑यौ कल्पेथाꣳ॒॒ सꣳ सम् क॑ल्पेथाꣳ॒॒ संप्रि॑यौ । \newline
4. क॒ल्पे॒थाꣳ॒॒ संप्रि॑यौ॒ संप्रि॑यौ कल्पेथाम् कल्पेथाꣳ॒॒ संप्रि॑यौ रोचि॒ष्णू रो॑चि॒ष्णू संप्रि॑यौ कल्पेथाम् कल्पेथाꣳ॒॒ संप्रि॑यौ रोचि॒ष्णू । \newline
5. संप्रि॑यौ रोचि॒ष्णू रो॑चि॒ष्णू संप्रि॑यौ॒ संप्रि॑यौ रोचि॒ष्णू सु॑मन॒स्यमा॑नौ सुमन॒स्यमा॑नौ रोचि॒ष्णू संप्रि॑यौ॒ संप्रि॑यौ रोचि॒ष्णू सु॑मन॒स्यमा॑नौ । \newline
6. संप्रि॑या॒विति॒ सं - प्रि॒यौ॒ । \newline
7. रो॒चि॒ष्णू सु॑मन॒स्यमा॑नौ सुमन॒स्यमा॑नौ रोचि॒ष्णू रो॑चि॒ष्णू सु॑मन॒स्यमा॑नौ । \newline
8. रो॒चि॒ष्णू इति॑ रोचि॒ष्णू । \newline
9. सु॒म॒न॒स्यमा॑ना॒विति॑ सु - म॒न॒स्यमा॑नौ । \newline
10. इष॒ मूर्ज॒ मूर्ज॒ मिष॒ मिष॒ मूर्ज॑ म॒भ्य॑ भ्यूर्ज॒ मिष॒ मिष॒ मूर्ज॑ म॒भि । \newline
11. ऊर्ज॑ म॒भ्य॑ भ्यूर्ज॒ मूर्ज॑ म॒भि सं॒ॅवसा॑नौ सं॒ॅवसा॑ना व॒भ्यूर्ज॒ मूर्ज॑ म॒भि सं॒ॅवसा॑नौ । \newline
12. अ॒भि सं॒ॅवसा॑नौ सं॒ॅवसा॑ना व॒भ्य॑भि सं॒ॅवसा॑नौ॒ सꣳ सꣳ सं॒ॅवसा॑ना व॒भ्य॑भि सं॒ॅवसा॑नौ॒ सम् । \newline
13. सं॒ॅवसा॑नौ॒ सꣳ सꣳ सं॒ॅवसा॑नौ सं॒ॅवसा॑नौ॒ सं ॅवां᳚ ॅवाꣳ॒॒ सꣳ सं॒ॅवसा॑नौ सं॒ॅवसा॑नौ॒ सं ॅवा᳚म् । \newline
14. सं॒ॅवसा॑ना॒विति॑ सं - वसा॑नौ । \newline
15. सं ॅवां᳚ ॅवाꣳ॒॒ सꣳ सं ॅवा॒म् मनाꣳ॑सि॒ मनाꣳ॑सि वाꣳ॒॒ सꣳ सं ॅवा॒म् मनाꣳ॑सि । \newline
16. वा॒म् मनाꣳ॑सि॒ मनाꣳ॑सि वां ॅवा॒म् मनाꣳ॑सि॒ सꣳ सम् मनाꣳ॑सि वां ॅवा॒म् मनाꣳ॑सि॒ सम् । \newline
17. मनाꣳ॑सि॒ सꣳ सम् मनाꣳ॑सि॒ मनाꣳ॑सि॒ सं ॅव्र॒ता व्र॒ता सम् मनाꣳ॑सि॒ मनाꣳ॑सि॒ सं ॅव्र॒ता । \newline
18. सं ॅव्र॒ता व्र॒ता सꣳ सं ॅव्र॒ता सꣳ सं ॅव्र॒ता सꣳ सं ॅव्र॒ता सम् । \newline
19. व्र॒ता सꣳ सं ॅव्र॒ता व्र॒ता समु॑ वु॒ सं ॅव्र॒ता व्र॒ता स मु॑ । \newline
20. समु॑ वु॒ सꣳ स मु॑ चि॒त्तानि॑ चि॒त्तान्यु॒ सꣳ स मु॑ चि॒त्तानि॑ । \newline
21. उ॒ चि॒त्तानि॑ चि॒त्तान्यु॑ वु चि॒त्तान्या चि॒त्तान्यु॑ वु चि॒त्तान्या । \newline
22. चि॒त्तान्या चि॒त्तानि॑ चि॒त्तान्या ऽक॑र मकर॒ मा चि॒त्तानि॑ चि॒त्तान्या ऽक॑रम् । \newline
23. आ ऽक॑र मकर॒ मा ऽक॑रम् । \newline
24. अ॒क॒र॒मित्य॑करम् । \newline
25. अग्ने॑ पुरीष्य पुरी॒ष्याग्ने ऽग्ने॑ पुरीष्या धि॒पा अ॑धि॒पाः पु॑री॒ष्याग्ने ऽग्ने॑ पुरीष्या धि॒पाः । \newline
26. पु॒री॒ष्या॒ धि॒पा अ॑धि॒पाः पु॑रीष्य पुरीष्या धि॒पा भ॑व भवा धि॒पाः पु॑रीष्य पुरीष्या धि॒पा भ॑व । \newline
27. अ॒धि॒पा भ॑व भवाधि॒पा अ॑धि॒पा भ॑वा॒ त्वम् त्वम् भ॑वाधि॒पा अ॑धि॒पा भ॑वा॒ त्वम् । \newline
28. अ॒धि॒पा इत्य॑धि - पाः । \newline
29. भ॒वा॒ त्वम् त्वम् भ॑व भवा॒ त्वम् नो॑ न॒ स्त्वम् भ॑व भवा॒ त्वम् नः॑ । \newline
30. त्वम् नो॑ न॒ स्त्वम् त्वम् नः॑ । \newline
31. न॒ इति॑ नः । \newline
32. इष॒ मूर्ज॒ मूर्ज॒ मिष॒ मिष॒ मूर्जं॒ ॅयज॑मानाय॒ यज॑माना॒योर्ज॒ मिष॒ मिष॒ मूर्जं॒ ॅयज॑मानाय । \newline
33. ऊर्जं॒ ॅयज॑मानाय॒ यज॑माना॒योर्ज॒ मूर्जं॒ ॅयज॑मानाय धेहि धेहि॒ यज॑माना॒योर्ज॒ मूर्जं॒ ॅयज॑मानाय धेहि । \newline
34. यज॑मानाय धेहि धेहि॒ यज॑मानाय॒ यज॑मानाय धेहि । \newline
35. धे॒हीति॑ धेहि । \newline
36. पु॒री॒ष्य॑ स्त्वम् त्वम् पु॑री॒ष्यः॑ पुरी॒ष्य॑ स्त्व म॑ग्ने अग्ने॒ त्वम् पु॑री॒ष्यः॑ पुरी॒ष्य॑ स्त्व म॑ग्ने । \newline
37. त्व म॑ग्ने अग्ने॒ त्वम् त्व म॑ग्ने रयि॒मान् र॑यि॒मा न॑ग्ने॒ त्वम् त्व म॑ग्ने रयि॒मान् । \newline
38. अ॒ग्ने॒ र॒यि॒मान् र॑यि॒मा न॑ग्ने अग्ने रयि॒मान् पु॑ष्टि॒मान् पु॑ष्टि॒मान् र॑यि॒मा न॑ग्ने अग्ने रयि॒मान् पु॑ष्टि॒मान् । \newline
39. र॒यि॒मान् पु॑ष्टि॒मान् पु॑ष्टि॒मान् र॑यि॒मान् र॑यि॒मान् पु॑ष्टि॒माꣳ अ॑स्यसि पुष्टि॒मान् र॑यि॒मान् र॑यि॒मान् पु॑ष्टि॒माꣳ अ॑सि । \newline
40. र॒यि॒मानिति॑ रयि - मान् । \newline
41. पु॒ष्टि॒माꣳ अ॑स्यसि पुष्टि॒मान् पु॑ष्टि॒माꣳ अ॑सि । \newline
42. पु॒ष्टि॒मानिति॑ पुष्टि - मान् । \newline
43. अ॒सीत्य॑सि । \newline
44. शि॒वाः कृ॒त्वा कृ॒त्वा शि॒वाः शि॒वाः कृ॒त्वा दिशो॒ दिशः॑ कृ॒त्वा शि॒वाः शि॒वाः कृ॒त्वा दिशः॑ । \newline
45. कृ॒त्वा दिशो॒ दिशः॑ कृ॒त्वा कृ॒त्वा दिशः॒ सर्वाः॒ सर्वा॒ दिशः॑ कृ॒त्वा कृ॒त्वा दिशः॒ सर्वाः᳚ । \newline
46. दिशः॒ सर्वाः॒ सर्वा॒ दिशो॒ दिशः॒ सर्वाः॒ स्वाꣳ स्वाꣳ सर्वा॒ दिशो॒ दिशः॒ सर्वाः॒ स्वाम् । \newline
47. सर्वाः॒ स्वाꣳ स्वाꣳ सर्वाः॒ सर्वाः॒ स्वां ॅयोनिं॒ ॅयोनिꣳ॒॒ स्वाꣳ सर्वाः॒ सर्वाः॒ स्वां ॅयोनि᳚म् । \newline
48. स्वां ॅयोनिं॒ ॅयोनिꣳ॒॒ स्वाꣳ स्वां ॅयोनि॑ मि॒हेह योनिꣳ॒॒ स्वाꣳ स्वां ॅयोनि॑ मि॒ह । \newline
49. योनि॑ मि॒हेह योनिं॒ ॅयोनि॑ मि॒हेह योनिं॒ ॅयोनि॑ मि॒हा । \newline
50. इ॒हेहे हा ऽस॑दो असद॒ एहे हा ऽस॑दः । \newline
51. आ ऽस॑दो असद॒ आ ऽस॑दः । \newline
52. अ॒स॒द॒ इत्य॑सदः । \newline
53. भव॑तम् नो नो॒ भव॑त॒म् भव॑तम् नः॒ सम॑नसौ॒ सम॑नसौ नो॒ भव॑त॒म् भव॑तम् नः॒ सम॑नसौ । \newline
54. नः॒ सम॑नसौ॒ सम॑नसौ नो नः॒ सम॑नसौ॒ समो॑कसौ॒ समो॑कसौ॒ सम॑नसौ नो नः॒ सम॑नसौ॒ समो॑कसौ । \newline
55. सम॑नसौ॒ समो॑कसौ॒ समो॑कसौ॒ सम॑नसौ॒ सम॑नसौ॒ समो॑कसा वरे॒पसा॑ वरे॒पसौ॒ समो॑कसौ॒ सम॑नसौ॒ सम॑नसौ॒ समो॑कसा वरे॒पसौ᳚ । \newline
56. सम॑नसा॒विति॒ स - म॒न॒सौ॒ । \newline
57. समो॑कसा वरे॒पसा॑ वरे॒पसौ॒ समो॑कसौ॒ समो॑कसा वरे॒पसौ᳚ । \newline
58. समो॑कसा॒विति॒ सं - ओ॒क॒सौ॒ । \newline
\pagebreak
\markright{ TS 4.2.5.2  \hfill https://www.vedavms.in \hfill}

\section{ TS 4.2.5.2 }

\textbf{TS 4.2.5.2 } \newline
\textbf{Samhita Paata} \newline

-वरे॒पसौ᳚ । मा य॒ज्ञ्ꣳ हिꣳ॑सिष्टं॒ मा य॒ज्ञ्प॑तिं जातवेदसौ शि॒वौ भ॑वतम॒द्य नः॑ ॥ मा॒तेव॑ पु॒त्रं पृ॑थि॒वी पु॑री॒ष्य॑म॒ग्निꣳ स्वे योना॑वभारु॒खा । तां ॅविश्वै᳚र्दे॒वैर्. ऋ॒तुभिः॑ संॅविदा॒नः प्र॒जाप॑तिर्वि॒श्वक॑र्मा॒ वि मु॑ञ्चतु ॥ यद॒स्य पा॒रे रज॑सः शु॒क्रं ज्योति॒रजा॑यत । तं नः॑ पर्.ष॒दति॒ द्विषोऽग्ने॑ वैश्वानर॒ स्वाहा᳚ ॥ नमः॒ सु ते॑ निर्.ऋते विश्वरूपे - [  ] \newline

\textbf{Pada Paata} \newline

अ॒रे॒पसौ᳚ ॥ मा । य॒ज्ञ्म् । हिꣳ॒॒सि॒ष्ट॒म् । मा । य॒ज्ञ्प॑ति॒मिति॑ य॒ज्ञ् - प॒ति॒म् । जा॒त॒वे॒द॒सा॒विति॑ जात - वे॒द॒सौ॒ । शि॒वौ । भ॒व॒त॒म् । अ॒द्य । नः॒ ॥ मा॒ता । इ॒व॒ । पु॒त्रम् । पृ॒थि॒वी । पु॒री॒ष्य᳚म् । अ॒ग्निम् । स्वे । योनौ᳚ । अ॒भाः॒ । उ॒खा ॥ ताम् । विश्वैः᳚ । दे॒वैः । ऋ॒तुभि॒रित्यृ॒तु - भिः॒ । सं॒ॅवि॒दा॒न इति॑ सं - वि॒दा॒नः । प्र॒जाप॑ति॒रिति॑ प्र॒जा - प॒तिः॒ । वि॒श्वक॒र्मेति॑ वि॒श्व - क॒र्मा॒ । वीति॑ । मु॒ञ्च॒तु॒ ॥ यत् । अ॒स्य । पा॒रे । रज॑सः । शु॒क्रम् । ज्योतिः॑ । अजा॑यत ॥ तम् । नः॒ । प॒र्॒.ष॒त् । अतीति॑ । द्विषः॑ । अग्ने᳚ । वै॒श्वा॒न॒र॒ । स्वाहा᳚ ॥ नमः॑ । स्विति॑ । ते॒ । नि॒र्॒.ऋ॒त॒ इति॑ निः - ऋ॒ते॒ । वि॒श्व॒रू॒प॒ इति॑ विश्व - रू॒पे॒ ।  \newline


\textbf{Krama Paata} \newline

अ॒रे॒पसा॒वित्य॑रे॒पसौ᳚ ॥ मा य॒ज्ञ्म् । य॒ज्ञ्ꣳ हिꣳ॑सिष्टम् । हिꣳ॒॒सि॒ष्ट॒म् मा । मा य॒ज्ञ्प॑तिम् । य॒ज्ञ्प॑तिम् जातवेदसौ । य॒ज्ञ्प॑ति॒मिति॑ य॒ज्ञ् - प॒ति॒म् । जा॒त॒वे॒द॒सौ॒ शि॒वौ । जा॒त॒वे॒द॒सा॒विति॑ जात - वे॒द॒सौ॒ । शि॒वौ भ॑वतम् । भ॒व॒त॒म॒द्य । अ॒द्य नः॑ । न॒ इति॑ नः ॥ मा॒तेव॑ । इ॒व॒ पु॒त्रम् । पु॒त्रम् पृ॑थि॒वी । पृ॒थि॒वी पु॑री॒ष्य᳚म् । पु॒री॒ष्य॑म॒ग्निम् । अ॒ग्निꣳ स्वे । स्वे योनौ᳚ । योना॑वभाः । अ॒भा॒रु॒खा । उ॒खेत्यु॒खा ॥ तां ॅविश्वैः᳚ । विश्वै᳚र् दे॒वैः । दे॒वैर्. ऋ॒तुभिः॑ । ऋ॒तुभिः॑ सम्ॅविदा॒नः । ऋ॒तुभि॒रित्यृ॒तु - भिः॒ । स॒म्ॅवि॒दा॒नः प्र॒जाप॑तिः । स॒म्ॅवि॒दा॒न इति॑ सम् - वि॒दा॒नः । प्र॒जाप॑तिर् वि॒श्वक॑र्मा । प्र॒जाप॑ति॒रिति॑ प्र॒जा - प॒तिः॒ । वि॒श्वक॑र्मा॒ वि । वि॒श्वक॒र्मेति॑ वि॒श्व - क॒र्मा॒ । वि मु॑ञ्चतु । मु॒ञ्च॒त्विति॑ मुञ्चतु ॥ यद॒स्य । अ॒स्य पा॒रे । पा॒रे रज॑सः । रज॑सः शु॒क्रम् । शु॒क्रम् ज्योतिः॑ । ज्योति॒रजा॑यत । अजा॑य॒तेत्यजा॑यत ॥ तम् नः॑ । नः॒ प॒र्॒.ष॒त्॒ । प॒र्॒.ष॒दति॑ । अति॒ द्विषः॑ । द्विषोऽग्ने᳚ । अग्ने॑ वैश्वानर । वै॒श्वा॒न॒र॒ स्वाहा᳚ । स्वाहेति॒ स्वाहा᳚ ॥ नमः॒ सु । सु ते᳚ । ते॒ नि॒र्॒.ऋ॒ते॒ । नि॒र्॒.ऋ॒ते॒ वि॒श्व॒रू॒पे॒ । नि॒र्॒.ऋ॒त॒ इति॑ निः - ऋ॒ते॒ । वि॒श्व॒रू॒पे॒ऽय॒स्मय᳚म् । वि॒श्व॒रू॒प॒ इति॑ विश्व - रू॒पे॒ \newline

\textbf{Jatai Paata} \newline

1. अ॒रे॒पसा॒ वित्य॑रे॒पसौ᳚ । \newline
2. मा य॒ज्ञ्ं ॅय॒ज्ञ्म् मा मा य॒ज्ञ्म् । \newline
3. य॒ज्ञ्ꣳ हिꣳ॑सिष्टꣳ हिꣳसिष्टं ॅय॒ज्ञ्ं ॅय॒ज्ञ्ꣳ हिꣳ॑सिष्टम् । \newline
4. हिꣳ॒॒सि॒ष्ट॒म् मा मा हिꣳ॑सिष्टꣳ हिꣳसिष्ट॒म् मा । \newline
5. मा य॒ज्ञ्प॑तिं ॅय॒ज्ञ्प॑ति॒म् मा मा य॒ज्ञ्प॑तिम् । \newline
6. य॒ज्ञ्प॑तिम् जातवेदसौ जातवेदसौ य॒ज्ञ्प॑तिं ॅय॒ज्ञ्प॑तिम् जातवेदसौ । \newline
7. य॒ज्ञ्प॑ति॒मिति॑ य॒ज्ञ् - प॒ति॒म् । \newline
8. जा॒त॒वे॒द॒सौ॒ शि॒वौ शि॒वौ जा॑तवेदसौ जातवेदसौ शि॒वौ । \newline
9. जा॒त॒वे॒द॒सा॒विति॑ जात - वे॒द॒सौ॒ । \newline
10. शि॒वौ भ॑वतम् भवतꣳ शि॒वौ शि॒वौ भ॑वतम् । \newline
11. भ॒व॒त॒ म॒द्याद्य भ॑वतम् भवत म॒द्य । \newline
12. अ॒द्य नो॑ नो॒ ऽद्याद्य नः॑ । \newline
13. न॒ इति॑ नः । \newline
14. मा॒तेवे॑व मा॒ता मा॒तेव॑ । \newline
15. इ॒व॒ पु॒त्रम् पु॒त्र मि॑वेव पु॒त्रम् । \newline
16. पु॒त्रम् पृ॑थि॒वी पृ॑थि॒वी पु॒त्रम् पु॒त्रम् पृ॑थि॒वी । \newline
17. पृ॒थि॒वी पु॑री॒ष्य॑म् पुरी॒ष्य॑म् पृथि॒वी पृ॑थि॒वी पु॑री॒ष्य᳚म् । \newline
18. पु॒री॒ष्य॑ म॒ग्नि म॒ग्निम् पु॑री॒ष्य॑म् पुरी॒ष्य॑ म॒ग्निम् । \newline
19. अ॒ग्निꣳ स्वे स्वे अ॒ग्नि म॒ग्निꣳ स्वे । \newline
20. स्वे योनौ॒ योनौ॒ स्वे स्वे योनौ᳚ । \newline
21. योना॑ वभा रभा॒र् योनौ॒ योना॑ वभाः । \newline
22. अ॒भा॒ रु॒खोखा ऽभा॑ रभा रु॒खा । \newline
23. उ॒खेत्यु॒खा । \newline
24. तां ॅविश्वै॒र् विश्वै॒ स्ताम् तां ॅविश्वैः᳚ । \newline
25. विश्वै᳚र् दे॒वैर् दे॒वैर् विश्वै॒र् विश्वै᳚र् दे॒वैः । \newline
26. दे॒वैर्. ऋ॒तुभिर्॑. ऋ॒तुभि॑र् दे॒वैर् दे॒वैर्. ऋ॒तुभिः॑ । \newline
27. ऋ॒तुभिः॑ संॅविदा॒नः सं॑ॅविदा॒न ऋ॒तुभिर्॑. ऋ॒तुभिः॑ संॅविदा॒नः । \newline
28. ऋ॒तुभि॒रित्यृ॒तु - भिः॒ । \newline
29. सं॒ॅवि॒दा॒नः प्र॒जाप॑तिः प्र॒जाप॑तिः संॅविदा॒नः सं॑ॅविदा॒नः प्र॒जाप॑तिः । \newline
30. सं॒ॅवि॒दा॒न इति॑ सं - वि॒दा॒नः । \newline
31. प्र॒जाप॑तिर् वि॒श्वक॑र्मा वि॒श्वक॑र्मा प्र॒जाप॑तिः प्र॒जाप॑तिर् वि॒श्वक॑र्मा । \newline
32. प्र॒जाप॑ति॒रिति॑ प्र॒जा - प॒तिः॒ । \newline
33. वि॒श्वक॑र्मा॒ वि वि वि॒श्वक॑र्मा वि॒श्वक॑र्मा॒ वि । \newline
34. वि॒श्वक॒र्मेति॑ वि॒श्व - क॒र्मा॒ । \newline
35. वि मु॑ञ्चतु मुञ्चतु॒ वि वि मु॑ञ्चतु । \newline
36. मु॒ञ्च॒त्विति॑ मुञ्चतु । \newline
37. यद॒स्या स्य यद् यद॒स्य । \newline
38. अ॒स्य पा॒रे पा॒रे अ॒स्यास्य पा॒रे । \newline
39. पा॒रे रज॑सो॒ रज॑सः पा॒रे पा॒रे रज॑सः । \newline
40. रज॑सः शु॒क्रꣳ शु॒क्रꣳ रज॑सो॒ रज॑सः शु॒क्रम् । \newline
41. शु॒क्रम् ज्योति॒र् ज्योतिः॑ शु॒क्रꣳ शु॒क्रम् ज्योतिः॑ । \newline
42. ज्योति॒ रजा॑य॒ता जा॑यत॒ ज्योति॒र् ज्योति॒ रजा॑यत । \newline
43. अजा॑य॒ते त्यजा॑यत । \newline
44. तन् नो॑ न॒ स्तत् तन् नः॑ । \newline
45. नः॒ प॒र्॒.ष॒त् प॒र्॒.ष॒न् नो॒ नः॒ प॒र्॒.ष॒त् । \newline
46. प॒र्॒.ष॒ दत्यति॑ पर्.षत् पर्.ष॒ दति॑ । \newline
47. अति॒ द्विषो॒ द्विषो॒ अत्यति॒ द्विषः॑ । \newline
48. द्विषो ऽग्ने ऽग्ने॒ द्विषो॒ द्विषो ऽग्ने᳚ । \newline
49. अग्ने॑ वैश्वानर वैश्वान॒राग्ने ऽग्ने॑ वैश्वानर । \newline
50. वै॒श्वा॒न॒र॒ स्वाहा॒ स्वाहा॑ वैश्वानर वैश्वानर॒ स्वाहा᳚ । \newline
51. स्वाहेति॒ स्वाहा᳚ । \newline
52. नमः॒ सु सु नमो॒ नमः॒ सु । \newline
53. सु ते॑ ते॒ सु सु ते᳚ । \newline
54. ते॒ नि॒र्॒.ऋ॒ते॒ नि॒र्॒.ऋ॒ते॒ ते॒ ते॒ नि॒र्॒.ऋ॒ते॒ । \newline
55. नि॒र्॒.ऋ॒ते॒ वि॒श्व॒रू॒पे॒ वि॒श्व॒रू॒पे॒ नि॒र्॒.ऋ॒ते॒ नि॒र्॒.ऋ॒ते॒ वि॒श्व॒रू॒पे॒ । \newline
56. नि॒र्॒.ऋ॒त॒ इति॑ निः - ऋ॒ते॒ । \newline
57. वि॒श्व॒रू॒पे॒ ऽय॒स्मय॑ मय॒स्मयं॑ ॅविश्वरूपे विश्वरूपे ऽय॒स्मय᳚म् । \newline
58. वि॒श्व॒रू॒प॒ इति॑ विश्व - रू॒पे॒ । \newline

\textbf{Ghana Paata } \newline

1. अ॒रे॒पसा॒वित्य॑रे॒पसौ᳚ । \newline
2. मा य॒ज्ञ्ं ॅय॒ज्ञ्म् मा मा य॒ज्ञ्ꣳ हिꣳ॑सिष्टꣳ हिꣳसिष्टं ॅय॒ज्ञ्म् मा मा य॒ज्ञ्ꣳ हिꣳ॑सिष्टम् । \newline
3. य॒ज्ञ्ꣳ हिꣳ॑सिष्टꣳ हिꣳसिष्टं ॅय॒ज्ञ्ं ॅय॒ज्ञ्ꣳ हिꣳ॑सिष्ट॒म् मा मा हिꣳ॑सिष्टं ॅय॒ज्ञ्ं ॅय॒ज्ञ्ꣳ हिꣳ॑सिष्ट॒म् मा । \newline
4. हिꣳ॒॒सि॒ष्ट॒म् मा मा हिꣳ॑सिष्टꣳ हिꣳसिष्ट॒म् मा य॒ज्ञ्प॑तिं ॅय॒ज्ञ्प॑ति॒म् मा हिꣳ॑सिष्टꣳ हिꣳसिष्ट॒म् मा य॒ज्ञ्प॑तिम् । \newline
5. मा य॒ज्ञ्प॑तिं ॅय॒ज्ञ्प॑ति॒म् मा मा य॒ज्ञ्प॑तिम् जातवेदसौ जातवेदसौ य॒ज्ञ्प॑ति॒म् मा मा य॒ज्ञ्प॑तिम् जातवेदसौ । \newline
6. य॒ज्ञ्प॑तिम् जातवेदसौ जातवेदसौ य॒ज्ञ्प॑तिं ॅय॒ज्ञ्प॑तिम् जातवेदसौ शि॒वौ शि॒वौ जा॑तवेदसौ य॒ज्ञ्प॑तिं ॅय॒ज्ञ्प॑तिम् जातवेदसौ शि॒वौ । \newline
7. य॒ज्ञ्प॑ति॒मिति॑ य॒ज्ञ् - प॒ति॒म् । \newline
8. जा॒त॒वे॒द॒सौ॒ शि॒वौ शि॒वौ जा॑तवेदसौ जातवेदसौ शि॒वौ भ॑वतम् भवतꣳ शि॒वौ जा॑तवेदसौ जातवेदसौ शि॒वौ भ॑वतम् । \newline
9. जा॒त॒वे॒द॒सा॒विति॑ जात - वे॒द॒सौ॒ । \newline
10. शि॒वौ भ॑वतम् भवतꣳ शि॒वौ शि॒वौ भ॑वत म॒द्याद्य भ॑वतꣳ शि॒वौ शि॒वौ भ॑वत म॒द्य । \newline
11. भ॒व॒त॒ म॒द्याद्य भ॑वतम् भवत म॒द्य नो॑ नो॒ ऽद्य भ॑वतम् भवत म॒द्य नः॑ । \newline
12. अ॒द्य नो॑ नो॒ ऽद्याद्य नः॑ । \newline
13. न॒ इति॑ नः । \newline
14. मा॒तेवे॑व मा॒ता मा॒तेव॑ पु॒त्रम् पु॒त्र मि॑व मा॒ता मा॒तेव॑ पु॒त्रम् । \newline
15. इ॒व॒ पु॒त्रम् पु॒त्र मि॑वे व पु॒त्रम् पृ॑थि॒वी पृ॑थि॒वी पु॒त्र मि॑वे व पु॒त्रम् पृ॑थि॒वी । \newline
16. पु॒त्रम् पृ॑थि॒वी पृ॑थि॒वी पु॒त्रम् पु॒त्रम् पृ॑थि॒वी पु॑री॒ष्य॑म् पुरी॒ष्य॑म् पृथि॒वी पु॒त्रम् पु॒त्रम् पृ॑थि॒वी पु॑री॒ष्य᳚म् । \newline
17. पृ॒थि॒वी पु॑री॒ष्य॑म् पुरी॒ष्य॑म् पृथि॒वी पृ॑थि॒वी पु॑री॒ष्य॑ म॒ग्नि म॒ग्निम् पु॑री॒ष्य॑म् पृथि॒वी पृ॑थि॒वी पु॑री॒ष्य॑ म॒ग्निम् । \newline
18. पु॒री॒ष्य॑ म॒ग्नि म॒ग्निम् पु॑री॒ष्य॑म् पुरी॒ष्य॑ म॒ग्निꣳ स्वे स्वे अ॒ग्निम् पु॑री॒ष्य॑म् पुरी॒ष्य॑ म॒ग्निꣳ स्वे । \newline
19. अ॒ग्निꣳ स्वे स्वे अ॒ग्नि म॒ग्निꣳ स्वे योनौ॒ योनौ॒ स्वे अ॒ग्नि म॒ग्निꣳ स्वे योनौ᳚ । \newline
20. स्वे योनौ॒ योनौ॒ स्वे स्वे योना॑ वभा रभा॒र् योनौ॒ स्वे स्वे योना॑ वभाः । \newline
21. योना॑ वभा रभा॒र् योनौ॒ योना॑ वभा रु॒खोखा ऽभा॒र् योनौ॒ योना॑ वभा रु॒खा । \newline
22. अ॒भा॒ रु॒खोखा ऽभा॑ रभा रु॒खा । \newline
23. उ॒खेत्यु॒खा । \newline
24. तां ॅविश्वै॒र् विश्वै॒ स्ताम् तां ॅविश्वै᳚र् दे॒वैर् दे॒वैर् विश्वै॒ स्ताम् तां ॅविश्वै᳚र् दे॒वैः । \newline
25. विश्वै᳚र् दे॒वैर् दे॒वैर् विश्वै॒र् विश्वै᳚र् दे॒वैर्. ऋ॒तुभिर्॑. ऋ॒तुभि॑र् दे॒वैर् विश्वै॒र् विश्वै᳚र् दे॒वैर्. ऋ॒तुभिः॑ । \newline
26. दे॒वैर्. ऋ॒तुभिर्॑. ऋ॒तुभि॑र् दे॒वैर् दे॒वैर्. ऋ॒तुभिः॑ संॅविदा॒नः सं॑ॅविदा॒न ऋ॒तुभि॑र् दे॒वैर् दे॒वैर्. ऋ॒तुभिः॑ संॅविदा॒नः । \newline
27. ऋ॒तुभिः॑ संॅविदा॒नः सं॑ॅविदा॒न ऋ॒तुभिर्॑. ऋ॒तुभिः॑ संॅविदा॒नः प्र॒जाप॑तिः प्र॒जाप॑तिः संॅविदा॒न ऋ॒तुभिर्॑. ऋ॒तुभिः॑ संॅविदा॒नः प्र॒जाप॑तिः । \newline
28. ऋ॒तुभि॒रित्यृ॒तु - भिः॒ । \newline
29. सं॒ॅवि॒दा॒नः प्र॒जाप॑तिः प्र॒जाप॑तिः संॅविदा॒नः सं॑ॅविदा॒नः प्र॒जाप॑तिर् वि॒श्वक॑र्मा वि॒श्वक॑र्मा प्र॒जाप॑तिः संॅविदा॒नः सं॑ॅविदा॒नः प्र॒जाप॑तिर् वि॒श्वक॑र्मा । \newline
30. सं॒ॅवि॒दा॒न इति॑ सं - वि॒दा॒नः । \newline
31. प्र॒जाप॑तिर् वि॒श्वक॑र्मा वि॒श्वक॑र्मा प्र॒जाप॑तिः प्र॒जाप॑तिर् वि॒श्वक॑र्मा॒ वि वि वि॒श्वक॑र्मा प्र॒जाप॑तिः प्र॒जाप॑तिर् वि॒श्वक॑र्मा॒ वि । \newline
32. प्र॒जाप॑ति॒रिति॑ प्र॒जा - प॒तिः॒ । \newline
33. वि॒श्वक॑र्मा॒ वि वि वि॒श्वक॑र्मा वि॒श्वक॑र्मा॒ वि मु॑ञ्चतु मुञ्चतु॒ वि वि॒श्वक॑र्मा वि॒श्वक॑र्मा॒ वि मु॑ञ्चतु । \newline
34. वि॒श्वक॒र्मेति॑ वि॒श्व - क॒र्मा॒ । \newline
35. वि मु॑ञ्चतु मुञ्चतु॒ वि वि मु॑ञ्चतु । \newline
36. मु॒ञ्च॒त्विति॑ मुञ्चतु । \newline
37. यद॒स्यास्य यद् यद॒स्य पा॒रे पा॒रे अ॒स्य यद् यद॒स्य पा॒रे । \newline
38. अ॒स्य पा॒रे पा॒रे अ॒स्यास्य पा॒रे रज॑सो॒ रज॑सः पा॒रे अ॒स्यास्य पा॒रे रज॑सः । \newline
39. पा॒रे रज॑सो॒ रज॑सः पा॒रे पा॒रे रज॑सः शु॒क्रꣳ शु॒क्रꣳ रज॑सः पा॒रे पा॒रे रज॑सः शु॒क्रम् । \newline
40. रज॑सः शु॒क्रꣳ शु॒क्रꣳ रज॑सो॒ रज॑सः शु॒क्रम् ज्योति॒र् ज्योतिः॑ शु॒क्रꣳ रज॑सो॒ रज॑सः शु॒क्रम् ज्योतिः॑ । \newline
41. शु॒क्रम् ज्योति॒र् ज्योतिः॑ शु॒क्रꣳ शु॒क्रम् ज्योति॒ रजा॑य॒ता जा॑यत॒ ज्योतिः॑ शु॒क्रꣳ शु॒क्रम् ज्योति॒ रजा॑यत । \newline
42. ज्योति॒ रजा॑य॒ता जा॑यत॒ ज्योति॒र् ज्योति॒ रजा॑यत । \newline
43. अजा॑य॒तेत्यजा॑यत । \newline
44. तन् नो॑ न॒ स्तत् तन् नः॑ पर्.षत् पर्.षन् न॒ स्तत् तन् नः॑ पर्.षत् । \newline
45. नः॒ प॒र्॒.ष॒त् प॒र्॒.ष॒न् नो॒ नः॒ प॒र्॒.ष॒ दत्यति॑ पर्.षन् नो नः पर्.ष॒ दति॑ । \newline
46. प॒र्॒.ष॒दत्यति॑ पर्.षत् पर्.ष॒दति॒ द्विषो॒ द्विषो॒ अति॑ पर्.षत् पर्.ष॒दति॒ द्विषः॑ । \newline
47. अति॒ द्विषो॒ द्विषो॒ अत्यति॒ द्विषो ऽग्ने ऽग्ने॒ द्विषो॒ अत्यति॒ द्विषो ऽग्ने᳚ । \newline
48. द्विषो ऽग्ने ऽग्ने॒ द्विषो॒ द्विषो ऽग्ने॑ वैश्वानर वैश्वान॒ राग्ने॒ द्विषो॒ द्विषो ऽग्ने॑ वैश्वानर । \newline
49. अग्ने॑ वैश्वानर वैश्वान॒ राग्ने ऽग्ने॑ वैश्वानर॒ स्वाहा॒ स्वाहा॑ वैश्वान॒ राग्ने ऽग्ने॑ वैश्वानर॒ स्वाहा᳚ । \newline
50. वै॒श्वा॒न॒र॒ स्वाहा॒ स्वाहा॑ वैश्वानर वैश्वानर॒ स्वाहा᳚ । \newline
51. स्वाहेति॒ स्वाहा᳚ । \newline
52. नमः॒ सु सु नमो॒ नमः॒ सु ते॑ ते॒ सु नमो॒ नमः॒ सु ते᳚ । \newline
53. सु ते॑ ते॒ सु सु ते॑ निर्.ऋते निर्.ऋते ते॒ सु सु ते॑ निर्.ऋते । \newline
54. ते॒ नि॒र्॒.ऋ॒ते॒ नि॒र्॒.ऋ॒ते॒ ते॒ ते॒ नि॒र्॒.ऋ॒ते॒ वि॒श्व॒रू॒पे॒ वि॒श्व॒रू॒पे॒ नि॒र्॒.ऋ॒ते॒ ते॒ ते॒ नि॒र्॒.ऋ॒ते॒ वि॒श्व॒रू॒पे॒ । \newline
55. नि॒र्॒.ऋ॒ते॒ वि॒श्व॒रू॒पे॒ वि॒श्व॒रू॒पे॒ नि॒र्॒.ऋ॒ते॒ नि॒र्॒.ऋ॒ते॒ वि॒श्व॒रू॒पे॒ ऽय॒स्मय॑ मय॒स्मयं॑ ॅविश्वरूपे निर्.ऋते निर्.ऋते विश्वरूपे ऽय॒स्मय᳚म् । \newline
56. नि॒र्॒.ऋ॒त॒ इति॑ निः - ऋ॒ते॒ । \newline
57. वि॒श्व॒रू॒पे॒ ऽय॒स्मय॑ मय॒स्मयं॑ ॅविश्वरूपे विश्वरूपे ऽय॒स्मयं॒ ॅवि व्य॑य॒स्मयं॑ ॅविश्वरूपे विश्वरूपे ऽय॒स्मयं॒ ॅवि । \newline
58. वि॒श्व॒रू॒प॒ इति॑ विश्व - रू॒पे॒ । \newline
\pagebreak
\markright{ TS 4.2.5.3  \hfill https://www.vedavms.in \hfill}

\section{ TS 4.2.5.3 }

\textbf{TS 4.2.5.3 } \newline
\textbf{Samhita Paata} \newline

ऽय॒स्मयं॒ ॅवि चृ॑ता ब॒न्धमे॒तं । य॒मेन॒ त्वं ॅय॒म्या॑ सं ॅविदा॒नोत्त॒मं नाक॒मधि॑ रोहये॒मं ॥ यत्ते॑ दे॒वी निर्.ऋ॑तिराब॒बन्ध॒ दाम॑ ग्री॒वास्व॑ विच॒र्त्यं । इ॒दं ते॒ तद् विष्यां॒ ॅयायु॑षो॒ न मद्ध्या॒दथा॑ जी॒वः पि॒तुम॑द्धि॒ प्रमु॑क्तः ॥ यस्या᳚स्ते अ॒स्याः क्रू॒र आ॒सञ्जु॒होम्ये॒षां ब॒न्धाना॑मव॒सर्ज॑नाय । भूमि॒रिति॑ त्वा॒ जना॑ वि॒दुर्निर्.ऋ॑ति॒ - [  ] \newline

\textbf{Pada Paata} \newline

अ॒य॒स्मय᳚म् । वीति॑ । चृ॒त॒ । ब॒न्धम् । ए॒तम् ॥ य॒मेन॑ । त्वम् । य॒म्या᳚ । सं॒ॅवि॒दा॒नेति॑ सं - वि॒दा॒ना । उ॒त्त॒ममित्यु॑त् - त॒मम् । नाक᳚म् । अधीति॑ । रो॒ह॒य॒ । इ॒मम् ॥ यत् । ते॒ । दे॒वी । निर्.ऋ॑ति॒रिति॒ निः - ऋ॒तिः॒ । आ॒ब॒बन्धेत्या᳚ - ब॒बन्ध॑ । दाम॑ । ग्री॒वासु॑ । अ॒वि॒च॒र्त्यमित्य॑वि - च॒र्त्यम् ॥ इ॒दम् । ते॒ । तत् । वीति॑ । स्या॒मि॒ । आयु॑षः । न । मद्ध्या᳚त् । अथ॑ । जी॒वः । पि॒तुम् । अ॒द्धि॒ । प्रमु॑क्त॒ इति॒ प्र - मु॒क्तः॒ ॥ यस्याः᳚ । ते॒ । अ॒स्याः । क्रू॒रे । आ॒सन्न् । जु॒होमि॑ । ए॒षाम् । ब॒न्धाना᳚म् । अ॒व॒सर्ज॑ना॒येत्य॑व-सर्ज॑नाय ॥ भूमिः॑ । इति॑ । त्वा॒ । जनाः᳚ । वि॒दुः । निर्.ऋ॑ति॒रिति॒ निः - ऋ॒तिः॒ ।  \newline


\textbf{Krama Paata} \newline

अ॒य॒स्मयं॒ ॅवि । वि चृ॑त । चृ॒ता॒ ब॒न्धम् । ब॒न्धमे॒तम् । ए॒तमित्ये॒तम् ॥ य॒मेन॒ त्वम् । त्वं ॅय॒म्या᳚ । य॒म्या॑ सम्ॅविदा॒ना । स॒म्ॅवि॒दा॒नोत्त॒मम् । स॒म्ॅवि॒दा॒नेति॑ सम् - वि॒दा॒ना । उ॒त्त॒मम् नाक᳚म् । उ॒त्त॒ममित्यु॑त् - त॒मम् । नाक॒मधि॑ । अधि॑ रोहय । रो॒ह॒ये॒मम् । इ॒ममिती॒मम् ॥ यत् ते᳚ । ते॒ दे॒वी । दे॒वी निर्.ऋ॑तिः । निर्.ऋ॑तिराब॒बन्ध॑ । निर्.ऋ॑ति॒रिति॒ निः - ऋ॒तिः॒ । आ॒ब॒बन्ध॒ दाम॑ । आ॒ब॒बन्धेत्या᳚ - ब॒बन्ध॑ । दाम॑ ग्री॒वासु॑ । ग्री॒वास्व॑विच॒र्त्यम् । अ॒वि॒च॒र्त्यमित्य॑वि - च॒र्त्यम् ॥ इ॒दम् ते᳚ । ते॒ तत् । तद् वि । वि ष्या॑मि । स्या॒म्यायु॑षः । आयु॑षो॒ न । न मद्ध्या᳚त् । मद्ध्या॒दथ॑ । अथा॑ जी॒वः । जी॒वः पि॒तुम् । पि॒तुम॑द्धि । अ॒द्धि॒ प्रमु॑क्तः । प्रमु॑क्त॒ इति॒ प्र - मु॒क्तः॒ ॥ यस्या᳚स्ते । ते॒ अ॒स्याः । अ॒स्याः क्रू॒रे । क्रू॒र आ॒सन्न् । आ॒सन् जु॒होमि॑ । जु॒होम्ये॒षाम् । ए॒षाम् ब॒न्धाना᳚म् । ब॒न्धाना॑मव॒सर्ज॑नाय । अ॒व॒सर्ज॑ना॒येत्य॑व - सर्ज॑नाय ॥ भूमि॒रिति॑ । इति॑ त्वा । त्वा॒ जनाः᳚ । जना॑वि॒दुः । वि॒दुर् निर्.ऋ॑तिः । निर्.ऋ॑ति॒रिति॑ । निर्.ऋ॑ति॒रिति॒ निः - ऋ॒तिः॒ \newline

\textbf{Jatai Paata} \newline

1. अ॒य॒स्मयं॒ ॅवि व्य॑य॒स्मय॑ मय॒स्मयं॒ ॅवि । \newline
2. वि चृ॑त चृत॒ वि वि चृ॑त । \newline
3. चृ॒ता॒ ब॒न्धम् ब॒न्धम् चृ॑त चृता ब॒न्धम् । \newline
4. ब॒न्ध मे॒त मे॒तम् ब॒न्धम् ब॒न्ध मे॒तम् । \newline
5. ए॒तमित्ये॒तम् । \newline
6. य॒मेन॒ त्वम् त्वं ॅय॒मेन॑ य॒मेन॒ त्वम् । \newline
7. त्वं ॅय॒म्या॑ य॒म्या᳚ त्वम् त्वं ॅय॒म्या᳚ । \newline
8. य॒म्या॑ संॅविदा॒ना सं॑ॅविदा॒ना य॒म्या॑ य॒म्या॑ संॅविदा॒ना । \newline
9. सं॒ॅवि॒दा॒ नोत्त॒म मु॑त्त॒मꣳ सं॑ॅविदा॒ना सं॑ॅविदा॒ नोत्त॒मम् । \newline
10. सं॒ॅवि॒दा॒नेति॑ सं - वि॒दा॒ना । \newline
11. उ॒त्त॒मम् नाक॒म् नाक॑ मुत्त॒म मु॑त्त॒मम् नाक᳚म् । \newline
12. उ॒त्त॒ममित्यु॑त् - त॒मम् । \newline
13. नाक॒ मध्यधि॒ नाक॒म् नाक॒ मधि॑ । \newline
14. अधि॑ रोहय रोह॒या ध्यधि॑ रोहय । \newline
15. रो॒ह॒ये॒ म मि॒मꣳ रो॑हय रोहये॒ मम् । \newline
16. इ॒ममिती॒मम् । \newline
17. यत् ते॑ ते॒ यद् यत् ते᳚ । \newline
18. ते॒ दे॒वी दे॒वी ते॑ ते दे॒वी । \newline
19. दे॒वी निर्.ऋ॑ति॒र् निर्.ऋ॑तिर् दे॒वी दे॒वी निर्.ऋ॑तिः । \newline
20. निर्.ऋ॑ति राब॒बन्धा॑ ब॒बन्ध॒ निर्.ऋ॑ति॒र् निर्.ऋ॑ति राब॒बन्ध॑ । \newline
21. निर्.ऋ॑ति॒रिति॒ निः - ऋ॒तिः॒ । \newline
22. आ॒ब॒बन्ध॒ दाम॒ दामा॑ ब॒बन्धा॑ ब॒बन्ध॒ दाम॑ । \newline
23. आ॒ब॒बन्धेत्या᳚ - ब॒बन्ध॑ । \newline
24. दाम॑ ग्री॒वासु॑ ग्री॒वासु॒ दाम॒ दाम॑ ग्री॒वासु॑ । \newline
25. ग्री॒वा स्व॑विच॒र्त्य म॑विच॒र्त्यम् ग्री॒वासु॑ ग्री॒वा स्व॑विच॒र्त्यम् । \newline
26. अ॒वि॒च॒र्त्यमित्य॑वि - च॒र्त्यम् । \newline
27. इ॒दम् ते॑ त इ॒द मि॒दम् ते᳚ । \newline
28. ते॒ तत् तत् ते॑ ते॒ तत् । \newline
29. तद् वि वि तत् तद् वि । \newline
30. वि ष्या॑मि स्यामि॒ वि वि ष्या॑मि । \newline
31. स्या॒ म्यायु॑ष॒ आयु॑षः स्यामि स्या॒ म्यायु॑षः । \newline
32. आयु॑षो॒ न नायु॑ष॒ आयु॑षो॒ न । \newline
33. न मद्ध्या॒न् मद्ध्या॒न् न न मद्ध्या᳚त् । \newline
34. मद्ध्या॒ दथाथ॒ मद्ध्या॒न् मद्ध्या॒ दथ॑ । \newline
35. अथा॑ जी॒वो जी॒वो अथाथा॑ जी॒वः । \newline
36. जी॒वः पि॒तुम् पि॒तुम् जी॒वो जी॒वः पि॒तुम् । \newline
37. पि॒तु म॑द्ध्यद्धि पि॒तुम् पि॒तु म॑द्धि । \newline
38. अ॒द्धि॒ प्रमु॑क्तः॒ प्रमु॑क्तो अद्ध्यद्धि॒ प्रमु॑क्तः । \newline
39. प्रमु॑क्त॒ इति॒ प्र - मु॒क्तः॒ । \newline
40. यस्या᳚ स्ते ते॒ यस्या॒ यस्या᳚ स्ते । \newline
41. ते॒ अ॒स्या अ॒स्या स्ते॑ ते अ॒स्याः । \newline
42. अ॒स्याः क्रू॒रे क्रू॒रे अ॒स्या अ॒स्याः क्रू॒रे । \newline
43. क्रू॒र आ॒सन् ना॒सन् क्रू॒रे क्रू॒र आ॒सन्न् । \newline
44. आ॒सन् जु॒होमि॑ जु॒हो म्या॒सन् ना॒सन् जु॒होमि॑ । \newline
45. जु॒हो म्ये॒षा मे॒षाम् जु॒होमि॑ जु॒हो म्ये॒षाम् । \newline
46. ए॒षाम् ब॒न्धाना᳚म् ब॒न्धाना॑ मे॒षा मे॒षाम् ब॒न्धाना᳚म् । \newline
47. ब॒न्धाना॑ मव॒सर्ज॑नाया व॒सर्ज॑नाय ब॒न्धाना᳚म् ब॒न्धाना॑ मव॒सर्ज॑नाय । \newline
48. अ॒व॒सर्ज॑ना॒येत्य॑व - सर्ज॑नाय । \newline
49. भूमि॒ रितीति॒ भूमि॒र् भूमि॒ रिति॑ । \newline
50. इति॑ त्वा॒ त्वेतीति॑ त्वा । \newline
51. त्वा॒ जना॒ जना᳚ स्त्वा त्वा॒ जनाः᳚ । \newline
52. जना॑ वि॒दुर् वि॒दुर् जना॒ जना॑ वि॒दुः । \newline
53. वि॒दुर् निर्.ऋ॑ति॒र् निर्.ऋ॑तिर् वि॒दुर् वि॒दुर् निर्.ऋ॑तिः । \newline
54. निर्.ऋ॑ति॒ रितीति॒ निर्.ऋ॑ति॒र् निर्.ऋ॑ति॒ रिति॑ । \newline
55. निर्.ऋ॑ति॒रिति॒ निः - ऋ॒तिः॒ । \newline

\textbf{Ghana Paata } \newline

1. अ॒य॒स्मयं॒ ॅवि व्य॑य॒स्मय॑ मय॒स्मयं॒ ॅवि चृ॑त चृत॒ व्य॑य॒स्मय॑ मय॒स्मयं॒ ॅवि चृ॑त । \newline
2. वि चृ॑त चृत॒ वि वि चृ॑ता ब॒न्धम् ब॒न्धम् चृ॑त॒ वि वि चृ॑ता ब॒न्धम् । \newline
3. चृ॒ता॒ ब॒न्धम् ब॒न्धम् चृ॑त चृता ब॒न्ध मे॒त मे॒तम् ब॒न्धम् चृ॑त चृता ब॒न्ध मे॒तम् । \newline
4. ब॒न्ध मे॒त मे॒तम् ब॒न्धम् ब॒न्ध मे॒तम् । \newline
5. ए॒तमित्ये॒तम् । \newline
6. य॒मेन॒ त्वम् त्वं ॅय॒मेन॑ य॒मेन॒ त्वं ॅय॒म्या॑ य॒म्या᳚ त्वं ॅय॒मेन॑ य॒मेन॒ त्वं ॅय॒म्या᳚ । \newline
7. त्वं ॅय॒म्या॑ य॒म्या᳚ त्वम् त्वं ॅय॒म्या॑ संॅविदा॒ना सं॑ॅविदा॒ना य॒म्या᳚ त्वम् त्वं ॅय॒म्या॑ संॅविदा॒ना । \newline
8. य॒म्या॑ संॅविदा॒ना सं॑ॅविदा॒ना य॒म्या॑ य॒म्या॑ संॅविदा॒नोत्त॒म मु॑त्त॒मꣳ सं॑ॅविदा॒ना य॒म्या॑ य॒म्या॑ संॅविदा॒नोत्त॒मम् । \newline
9. सं॒ॅवि॒दा॒नोत्त॒म मु॑त्त॒मꣳ सं॑ॅविदा॒ना सं॑ॅविदा॒नोत्त॒मम् नाक॒म् नाक॑ मुत्त॒मꣳ सं॑ॅविदा॒ना सं॑ॅविदा॒नोत्त॒मम् नाक᳚म् । \newline
10. सं॒ॅवि॒दा॒नेति॑ सं - वि॒दा॒ना । \newline
11. उ॒त्त॒मम् नाक॒म् नाक॑ मुत्त॒म मु॑त्त॒मम् नाक॒ मध्यधि॒ नाक॑ मुत्त॒म मु॑त्त॒मम् नाक॒ मधि॑ । \newline
12. उ॒त्त॒ममित्यु॑त् - त॒मम् । \newline
13. नाक॒ मध्यधि॒ नाक॒म् नाक॒ मधि॑ रोहय रोह॒याधि॒ नाक॒म् नाक॒ मधि॑ रोहय । \newline
14. अधि॑ रोहय रोह॒या ध्यधि॑ रोहये॒म मि॒मꣳ रो॑ह॒या ध्यधि॑ रोहये॒मम् । \newline
15. रो॒ह॒ये॒म मि॒मꣳ रो॑हय रोहये॒मम् । \newline
16. इ॒ममिती॒मम् । \newline
17. यत् ते॑ ते॒ यद् यत् ते॑ दे॒वी दे॒वी ते॒ यद् यत् ते॑ दे॒वी । \newline
18. ते॒ दे॒वी दे॒वी ते॑ ते दे॒वी निर्.ऋ॑ति॒र् निर्.ऋ॑तिर् दे॒वी ते॑ ते दे॒वी निर्.ऋ॑तिः । \newline
19. दे॒वी निर्.ऋ॑ति॒र् निर्.ऋ॑तिर् दे॒वी दे॒वी निर्.ऋ॑ति राब॒बन्धा॑ ब॒बन्ध॒ निर्.ऋ॑तिर् दे॒वी दे॒वी निर्.ऋ॑ति राब॒बन्ध॑ । \newline
20. निर्.ऋ॑ति राब॒बन्धा॑ ब॒बन्ध॒ निर्.ऋ॑ति॒र् निर्.ऋ॑ति राब॒बन्ध॒ दाम॒ दामा॑ ब॒बन्ध॒ निर्.ऋ॑ति॒र् निर्.ऋ॑ति राब॒बन्ध॒ दाम॑ । \newline
21. निर्.ऋ॑ति॒रिति॒ निः - ऋ॒तिः॒ । \newline
22. आ॒ब॒बन्ध॒ दाम॒ दामा॑ ब॒बन्धा॑ ब॒बन्ध॒ दाम॑ ग्री॒वासु॑ ग्री॒वासु॒ दामा॑ ब॒बन्धा॑ ब॒बन्ध॒ दाम॑ ग्री॒वासु॑ । \newline
23. आ॒ब॒बन्धेत्या᳚ - ब॒बन्ध॑ । \newline
24. दाम॑ ग्री॒वासु॑ ग्री॒वासु॒ दाम॒ दाम॑ ग्री॒वा स्व॑विच॒र्त्य म॑विच॒र्त्यम् ग्री॒वासु॒ दाम॒ दाम॑ ग्री॒वा स्व॑विच॒र्त्यम् । \newline
25. ग्री॒वा स्व॑विच॒र्त्य म॑विच॒र्त्यम् ग्री॒वासु॑ ग्री॒वा स्व॑विच॒र्त्यम् । \newline
26. अ॒वि॒च॒र्त्यमित्य॑वि - च॒र्त्यम् । \newline
27. इ॒दम् ते॑ त इ॒द मि॒दम् ते॒ तत् तत् त॑ इ॒द मि॒दम् ते॒ तत् । \newline
28. ते॒ तत् तत् ते॑ ते॒ तद् वि वि तत् ते॑ ते॒ तद् वि । \newline
29. तद् वि वि तत् तद् वि ष्या॑मि स्यामि॒ वि तत् तद् वि ष्या॑मि । \newline
30. वि ष्या॑मि स्यामि॒ वि वि ष्या॒म्यायु॑ष॒ आयु॑षः स्यामि॒ वि वि ष्या॒म्यायु॑षः । \newline
31. स्या॒म्यायु॑ष॒ आयु॑षः स्यामि स्या॒म्यायु॑षो॒ न नायु॑षः स्यामि स्या॒म्यायु॑षो॒ न । \newline
32. आयु॑षो॒ न नायु॑ष॒ आयु॑षो॒ न मद्ध्या॒न् मद्ध्या॒न् नायु॑ष॒ आयु॑षो॒ न मद्ध्या᳚त् । \newline
33. न मद्ध्या॒न् मद्ध्या॒न् न न मद्ध्या॒ दथाथ॒ मद्ध्या॒न् न न मद्ध्या॒ दथ॑ । \newline
34. मद्ध्या॒ दथाथ॒ मद्ध्या॒न् मद्ध्या॒ दथा॑ जी॒वो जी॒वो अथ॒ मद्ध्या॒न् मद्ध्या॒ दथा॑ जी॒वः । \newline
35. अथा॑ जी॒वो जी॒वो अथाथा॑ जी॒वः पि॒तुम् पि॒तुम् जी॒वो अथाथा॑ जी॒वः पि॒तुम् । \newline
36. जी॒वः पि॒तुम् पि॒तुम् जी॒वो जी॒वः पि॒तु म॑द्ध्यद्धि पि॒तुम् जी॒वो जी॒वः पि॒तु म॑द्धि । \newline
37. पि॒तु म॑द्ध्यद्धि पि॒तुम् पि॒तु म॑द्धि॒ प्रमु॑क्तः॒ प्रमु॑क्तो अद्धि पि॒तुम् पि॒तु म॑द्धि॒ प्रमु॑क्तः । \newline
38. अ॒द्धि॒ प्रमु॑क्तः॒ प्रमु॑क्तो अद्ध्यद्धि॒ प्रमु॑क्तः । \newline
39. प्रमु॑क्त॒ इति॒ प्र - मु॒क्तः॒ । \newline
40. यस्या᳚ स्ते ते॒ यस्या॒ यस्या᳚ स्ते अ॒स्या अ॒स्या स्ते॒ यस्या॒ यस्या᳚ स्ते अ॒स्याः । \newline
41. ते॒ अ॒स्या अ॒स्या स्ते॑ ते अ॒स्याः क्रू॒रे क्रू॒रे अ॒स्या स्ते॑ ते अ॒स्याः क्रू॒रे । \newline
42. अ॒स्याः क्रू॒रे क्रू॒रे अ॒स्या अ॒स्याः क्रू॒र आ॒सन् ना॒सन् क्रू॒रे अ॒स्या अ॒स्याः क्रू॒र आ॒सन्न् । \newline
43. क्रू॒र आ॒सन् ना॒सन् क्रू॒रे क्रू॒र आ॒सन् जु॒होमि॑ जु॒हो म्या॒सन् क्रू॒रे क्रू॒र आ॒सन् जु॒होमि॑ । \newline
44. आ॒सन् जु॒होमि॑ जु॒हो म्या॒सन् ना॒सन् जु॒हो म्ये॒षा मे॒षाम् जु॒हो म्या॒सन् ना॒सन् जु॒हो म्ये॒षाम् । \newline
45. जु॒हो म्ये॒षा मे॒षाम् जु॒होमि॑ जु॒हो म्ये॒षाम् ब॒न्धाना᳚म् ब॒न्धाना॑ मे॒षाम् जु॒होमि॑ जु॒हो म्ये॒षाम् ब॒न्धाना᳚म् । \newline
46. ए॒षाम् ब॒न्धाना᳚म् ब॒न्धाना॑ मे॒षा मे॒षाम् ब॒न्धाना॑ मव॒सर्ज॑नाया व॒सर्ज॑नाय ब॒न्धाना॑ मे॒षा मे॒षाम् ब॒न्धाना॑ मव॒सर्ज॑नाय । \newline
47. ब॒न्धाना॑ मव॒सर्ज॑नाया व॒सर्ज॑नाय ब॒न्धाना᳚म् ब॒न्धाना॑ मव॒सर्ज॑नाय । \newline
48. अ॒व॒सर्ज॑ना॒येत्य॑व - सर्ज॑नाय । \newline
49. भूमि॒ रितीति॒ भूमि॒र् भूमि॒ रिति॑ त्वा॒ त्वेति॒ भूमि॒र् भूमि॒ रिति॑ त्वा । \newline
50. इति॑ त्वा॒ त्वेतीति॑ त्वा॒ जना॒ जना॒ स्त्वेतीति॑ त्वा॒ जनाः᳚ । \newline
51. त्वा॒ जना॒ जना᳚ स्त्वा त्वा॒ जना॑ वि॒दुर् वि॒दुर् जना᳚ स्त्वा त्वा॒ जना॑ वि॒दुः । \newline
52. जना॑ वि॒दुर् वि॒दुर् जना॒ जना॑ वि॒दुर् निर्.ऋ॑ति॒र् निर्.ऋ॑तिर् वि॒दुर् जना॒ जना॑ वि॒दुर् निर्.ऋ॑तिः । \newline
53. वि॒दुर् निर्.ऋ॑ति॒र् निर्.ऋ॑तिर् वि॒दुर् वि॒दुर् निर्.ऋ॑ति॒ रितीति॒ निर्.ऋ॑तिर् वि॒दुर् वि॒दुर् निर्.ऋ॑ति॒ रिति॑ । \newline
54. निर्.ऋ॑ति॒ रितीति॒ निर्.ऋ॑ति॒र् निर्.ऋ॑ति॒ रिति॑ त्वा॒ त्वेति॒ निर्.ऋ॑ति॒र् निर्.ऋ॑ति॒ रिति॑ त्वा । \newline
55. निर्.ऋ॑ति॒रिति॒ निः - ऋ॒तिः॒ । \newline
\pagebreak
\markright{ TS 4.2.5.4  \hfill https://www.vedavms.in \hfill}

\section{ TS 4.2.5.4 }

\textbf{TS 4.2.5.4 } \newline
\textbf{Samhita Paata} \newline

रिति॑ त्वा॒ ऽहं परि॑ वेद वि॒श्वतः॑ ॥ असु॑न्वन्त॒म य॑जमानमिच्छ स्ते॒नस्ये॒त्यान्-तस्क॑र॒स्यान् वे॑षि । अ॒न्य म॒स्म-दि॑च्छ॒ सा त॑ इ॒त्या नमो॑ देवि निर्.ऋते॒ तुभ्य॑मस्तु ॥ दे॒वीम॒हं निर्.ऋ॑तिं॒ ॅवन्द॑मानः पि॒तेव॑ पु॒त्रं द॑सये॒ वचो॑भिः । विश्व॑स्य॒ या जाय॑मानस्य॒ वेद॒ शिरः॑ शिरः॒ प्रति॑ सू॒री वि च॑ष्टे ॥ नि॒वेश॑नः स॒ङ्गम॑नो॒ वसू॑नां॒ ॅविश्वा॑ रू॒पाऽभि च॑ष्टे॒ - [  ] \newline

\textbf{Pada Paata} \newline

इति॑ । त्वा॒ । अ॒हम् । परीति॑ । वे॒द॒ । वि॒श्वतः॑ ॥ असु॑न्वन्तम् । अय॑जमानम् । इ॒च्छ॒ । स्ते॒नस्य॑ । इ॒त्याम् । तस्क॑रस्य । अन्विति॑ । ए॒षि॒ ॥ अ॒न्यम् । अ॒स्मत् । इ॒च्छ॒ । सा । ते॒ । इ॒त्या । नमः॑ । दे॒वि॒ । नि॒र्॒.ऋ॒त॒ इति॑ निः - ऋ॒ते॒ । तुभ्य᳚म् । अ॒स्तु॒ ॥ दे॒वीम् । अ॒हम् । निर्.ऋ॑ति॒मिति॒ निः- ऋ॒ति॒म् । वन्द॑मानः । पि॒ता । इ॒व॒ । पु॒त्रम् । द॒स॒ये॒ । वचो॑भि॒रिति॒ वचः॑ - भिः॒ ॥ विश्व॑स्य । या । जाय॑मानस्य । वेद॑ । शिरः॑शिर॒ इति॒ शिरः॑-शि॒रः॒ । प्रतीति॑ । सू॒री । वीति॑ । च॒ष्टे॒ ॥ नि॒वेश॑न॒ इति॑ नि - वेश॑नः । स॒ङ्गम॑न॒ इति॑ सं - गम॑नः । वसू॑नाम् । विश्वा᳚ । रू॒पा । अ॒भीति॑ । च॒ष्टे॒ ।  \newline


\textbf{Krama Paata} \newline

इति॑ त्वा । त्वा॒ऽहम् । अ॒हम् परि॑ । परि॑ वेद । वे॒द॒ वि॒श्वतः॑ । वि॒श्वत॒ इति॑ वि॒श्वतः॑ ॥ असु॑न्वन्त॒मय॑जमानम् । अय॑जमानमिच्छ । इ॒च्छ॒ स्ते॒नस्य॑ । स्ते॒नस्ये॒त्याम् । इ॒त्याम् तस्क॑रस्य । तस्क॑र॒स्यानु॑ । अन्वे॑षि । ए॒षीत्ये॑षि ॥ अ॒न्यम॒स्मत् । अ॒स्मदि॑च्छ । इ॒च्छ॒ सा । सा ते᳚ । त॒ इ॒त्या । इ॒त्या नमः॑ । नमो॑ देवि । दे॒वि॒ नि॒र्॒.ऋ॒ते॒ । नि॒र्॒.ऋ॒ते॒ तुभ्य᳚म् । नि॒र्॒.ऋ॒त॒ इति॑ निः - ऋ॒ते॒ । तुभ्य॑मस्तु । अ॒स्त्विय॑स्तु ॥ दे॒वीम॒हम् । अ॒हम् निर्.ऋ॑तिम् । निर्.ऋ॑तिं॒ ॅवन्द॑मानः । निर्.ऋ॑ति॒मिति॒ निः - ऋ॒ति॒म् । वन्द॑मानः पि॒ता । पि॒तेव॑ । इ॒व॒ पु॒त्रम् । पु॒त्रम् द॑सये । द॒स॒ये॒ वचो॑भिः । वचो॑भि॒रिति॒ वचः॑ - भिः॒ ॥ विश्व॑स्य॒ या । या जाय॑मानस्य । जाय॑मानस्य॒ वेद॑ । वेद॒ शिर॑श्शिरः । शिर॑श्शिरः॒ प्रति॑ । शिर॑श्शिर॒ इति॒ शिरः॑ - शि॒रः॒ । प्रति॑ सू॒री । सू॒री वि । वि च॑ष्टे । च॒ष्ट॒ इति॑ चष्टे ॥ नि॒वेश॑नः स॒ङ्गम॑नः । नि॒वेश॑न॒ इति॑ नि - वेश॑नः । स॒ङ्गम॑नो॒ वसू॑नाम् । स॒ङ्गम॑न॒ इति॑ सम् - गम॑नः । वसू॑नां॒ ॅविश्वा᳚ । विश्वा॑ रू॒पा । रू॒पाऽभि । अ॒भि च॑ष्टे । च॒ष्टे॒ शची॑भिः \newline

\textbf{Jatai Paata} \newline

1. इति॑ त्वा॒ त्वेतीति॑ त्वा । \newline
2. त्वा॒ ऽह म॒हम् त्वा᳚ त्वा॒ ऽहम् । \newline
3. अ॒हम् परि॒ पर्य॒ह म॒हम् परि॑ । \newline
4. परि॑ वेद वेद॒ परि॒ परि॑ वेद । \newline
5. वे॒द॒ वि॒श्वतो॑ वि॒श्वतो॑ वेद वेद वि॒श्वतः॑ । \newline
6. वि॒श्वत॒ इति॑ वि॒श्वतः॑ । \newline
7. असु॑न्वन्त॒ मय॑जमान॒ मय॑जमान॒ मसु॑न्वन्त॒ मसु॑न्वन्त॒ मय॑जमानम् । \newline
8. अय॑जमान मिच्छे॒च्छा य॑जमान॒ मय॑जमान मिच्छ । \newline
9. इ॒च्छ॒ स्ते॒नस्य॑ स्ते॒नस्ये᳚ च्छे च्छ स्ते॒नस्य॑ । \newline
10. स्ते॒नस्ये॒ त्या मि॒त्याꣳ स्ते॒नस्य॑ स्ते॒नस्ये॒ त्याम् । \newline
11. इ॒त्याम् तस्क॑रस्य॒ तस्क॑रस्ये॒ त्या मि॒त्याम् तस्क॑रस्य । \newline
12. तस्क॑ र॒स्यान्वनु॒ तस्क॑रस्य॒ तस्क॑ र॒स्यानु॑ । \newline
13. अन्वे᳚ ष्ये॒ ष्यन्‌वन् वे॑षि । \newline
14. ए॒षीत्ये॑षि । \newline
15. अ॒न्य म॒स्म द॒स्म द॒न्य म॒न्य म॒स्मत् । \newline
16. अ॒स्म दि॑च्छेच्छा॒ स्म द॒स्म दि॑च्छ । \newline
17. इ॒च्छ॒ सा सेच्छे᳚ च्छ॒ सा । \newline
18. सा ते॑ ते॒ सा सा ते᳚ । \newline
19. त॒ इ॒त्येत्या ते॑ त इ॒त्या । \newline
20. इ॒त्या नमो॒ नम॑ इ॒त्येत्या नमः॑ । \newline
21. नमो॑ देवि देवि॒ नमो॒ नमो॑ देवि । \newline
22. दे॒वि॒ नि॒र्॒.ऋ॒ते॒ नि॒र्॒.ऋ॒ते॒ दे॒वि॒ दे॒वि॒ नि॒र्॒.ऋ॒ते॒ । \newline
23. नि॒र्॒.ऋ॒ते॒ तुभ्य॒म् तुभ्य॑म् निर्.ऋते निर्.ऋते॒ तुभ्य᳚म् । \newline
24. नि॒र्॒.ऋ॒त॒ इति॑ निः - ऋ॒ते॒ । \newline
25. तुभ्य॑ मस्त्वस्तु॒ तुभ्य॒म् तुभ्य॑ मस्तु । \newline
26. अ॒स्त्वित्य॑स्तु । \newline
27. दे॒वी म॒ह म॒हम् दे॒वीम् दे॒वी म॒हम् । \newline
28. अ॒हम् निर्.ऋ॑ति॒म् निर्.ऋ॑ति म॒ह म॒हम् निर्.ऋ॑तिम् । \newline
29. निर्.ऋ॑तिं॒ ॅवन्द॑मानो॒ वन्द॑मानो॒ निर्.ऋ॑ति॒म् निर्.ऋ॑तिं॒ ॅवन्द॑मानः । \newline
30. निर्.ऋ॑ति॒मिति॒ निः - ऋ॒ति॒म् । \newline
31. वन्द॑मानः पि॒ता पि॒ता वन्द॑मानो॒ वन्द॑मानः पि॒ता । \newline
32. पि॒तेवे॑व पि॒ता पि॒तेव॑ । \newline
33. इ॒व॒ पु॒त्रम् पु॒त्र मि॑वेव पु॒त्रम् । \newline
34. पु॒त्रम् द॑सये दसये पु॒त्रम् पु॒त्रम् द॑सये । \newline
35. द॒स॒ये॒ वचो॑भि॒र् वचो॑भिर् दसये दसये॒ वचो॑भिः । \newline
36. वचो॑भि॒रिति॒ वचः॑ - भिः॒ । \newline
37. विश्व॑स्य॒ या या विश्व॑स्य॒ विश्व॑स्य॒ या । \newline
38. या जाय॑मानस्य॒ जाय॑मानस्य॒ या या जाय॑मानस्य । \newline
39. जाय॑मानस्य॒ वेद॒ वेद॒ जाय॑मानस्य॒ जाय॑मानस्य॒ वेद॑ । \newline
40. वेद॒ शिरः॑शिरः॒ शिरः॑शिरो॒ वेद॒ वेद॒ शिरः॑शिरः । \newline
41. शिरः॑शिरः॒ प्रति॒ प्रति॒ शिरः॑शिरः॒ शिरः॑शिरः॒ प्रति॑ । \newline
42. शिरः॑शिर॒ इति॒ शिरः॑ - शि॒रः॒ । \newline
43. प्रति॑ सू॒री सू॒री प्रति॒ प्रति॑ सू॒री । \newline
44. सू॒री वि वि सू॒री सू॒री वि । \newline
45. वि च॑ष्टे चष्टे॒ वि वि च॑ष्टे । \newline
46. च॒ष्ट॒ इति॑ चष्टे । \newline
47. नि॒वेश॑नः स॒ङ्गम॑नः स॒ङ्गम॑नो नि॒वेश॑नो नि॒वेश॑नः स॒ङ्गम॑नः । \newline
48. नि॒वेश॑न॒ इति॑ नि - वेश॑नः । \newline
49. स॒ङ्गम॑नो॒ वसू॑नां॒ ॅवसू॑नाꣳ स॒ङ्गम॑नः स॒ङ्गम॑नो॒ वसू॑नाम् । \newline
50. स॒ङ्गम॑न॒ इति॑ सं - गम॑नः । \newline
51. वसू॑नां॒ ॅविश्वा॒ विश्वा॒ वसू॑नां॒ ॅवसू॑नां॒ ॅविश्वा᳚ । \newline
52. विश्वा॑ रू॒पा रू॒पा विश्वा॒ विश्वा॑ रू॒पा । \newline
53. रू॒पा ऽभ्य॑भि रू॒पा रू॒पा ऽभि । \newline
54. अ॒भि च॑ष्टे चष्टे अ॒भ्य॑भि च॑ष्टे । \newline
55. च॒ष्टे॒ शची॑भिः॒ शची॑भि श्चष्टे चष्टे॒ शची॑भिः । \newline

\textbf{Ghana Paata } \newline

1. इति॑ त्वा॒ त्वेतीति॑ त्वा॒ ऽह म॒हम् त्वेतीति॑ त्वा॒ ऽहम् । \newline
2. त्वा॒ ऽह म॒हम् त्वा᳚ त्वा॒ ऽहम् परि॒ पर्य॒हम् त्वा᳚ त्वा॒ ऽहम् परि॑ । \newline
3. अ॒हम् परि॒ पर्य॒ह म॒हम् परि॑ वेद वेद॒ पर्य॒ह म॒हम् परि॑ वेद । \newline
4. परि॑ वेद वेद॒ परि॒ परि॑ वेद वि॒श्वतो॑ वि॒श्वतो॑ वेद॒ परि॒ परि॑ वेद वि॒श्वतः॑ । \newline
5. वे॒द॒ वि॒श्वतो॑ वि॒श्वतो॑ वेद वेद वि॒श्वतः॑ । \newline
6. वि॒श्वत॒ इति॑ वि॒श्वतः॑ । \newline
7. असु॑न्वन्त॒ मय॑जमान॒ मय॑जमान॒ मसु॑न्वन्त॒ मसु॑न्वन्त॒ मय॑जमान मिच्छे॒ च्छाय॑जमान॒ मसु॑न्वन्त॒ मसु॑न्वन्त॒ मय॑जमान मिच्छ । \newline
8. अय॑जमान मिच्छे॒ च्छाय॑जमान॒ मय॑जमान मिच्छ स्ते॒नस्य॑ स्ते॒नस्ये॒ च्छाय॑जमान॒ मय॑जमान मिच्छ स्ते॒नस्य॑ । \newline
9. इ॒च्छ॒ स्ते॒नस्य॑ स्ते॒नस्ये᳚ च्छे च्छ स्ते॒नस्ये॒ त्या मि॒त्याꣳ स्ते॒नस्ये᳚ च्छे च्छ स्ते॒नस्ये॒ त्याम् । \newline
10. स्ते॒नस्ये॒ त्या मि॒त्याꣳ स्ते॒नस्य॑ स्ते॒नस्ये॒ त्याम् तस्क॑रस्य॒ तस्क॑रस्ये॒ त्याꣳ स्ते॒नस्य॑ स्ते॒नस्ये॒ त्याम् तस्क॑रस्य । \newline
11. इ॒त्याम् तस्क॑रस्य॒ तस्क॑रस्ये॒ त्या मि॒त्याम् तस्क॑र॒स्यान् वनु॒ तस्क॑रस्ये॒ त्या मि॒त्याम् तस्क॑र॒स्यानु॑ । \newline
12. तस्क॑र॒स्यान् वनु॒ तस्क॑रस्य॒ तस्क॑र॒स्यान् वे᳚ष्ये॒ष्यनु॒ तस्क॑रस्य॒ तस्क॑र॒स्यान् वे॑षि । \newline
13. अन्वे᳚ष्ये॒ ष्यन् वन् वे॑षि । \newline
14. ए॒षीत्ये॑षि । \newline
15. अ॒न्य म॒स्म द॒स्म द॒न्य म॒न्य म॒स्म दि॑च्छे च्छा॒स्म द॒न्य म॒न्य म॒स्म दि॑च्छ । \newline
16. अ॒स्म दि॑च्छे च्छा॒स्म द॒स्म दि॑च्छ॒ सा सेच्छा॒ स्म द॒स्म दि॑च्छ॒ सा । \newline
17. इ॒च्छ॒ सा सेच्छे᳚ च्छ॒ सा ते॑ ते॒ सेच्छे᳚ च्छ॒ सा ते᳚ । \newline
18. सा ते॑ ते॒ सा सा त॑ इ॒त्येत्या ते॒ सा सा त॑ इ॒त्या । \newline
19. त॒ इ॒त्येत्या ते॑ त इ॒त्या नमो॒ नम॑ इ॒त्या ते॑ त इ॒त्या नमः॑ । \newline
20. इ॒त्या नमो॒ नम॑ इ॒त्येत्या नमो॑ देवि देवि॒ नम॑ इ॒त्येत्या नमो॑ देवि । \newline
21. नमो॑ देवि देवि॒ नमो॒ नमो॑ देवि निर्.ऋते निर्.ऋते देवि॒ नमो॒ नमो॑ देवि निर्.ऋते । \newline
22. दे॒वि॒ नि॒र्॒.ऋ॒ते॒ नि॒र्॒.ऋ॒ते॒ दे॒वि॒ दे॒वि॒ नि॒र्॒.ऋ॒ते॒ तुभ्य॒म् तुभ्य॑म् निर्.ऋते देवि देवि निर्.ऋते॒ तुभ्य᳚म् । \newline
23. नि॒र्॒.ऋ॒ते॒ तुभ्य॒म् तुभ्य॑म् निर्.ऋते निर्.ऋते॒ तुभ्य॑ मस्त्वस्तु॒ तुभ्य॑म् निर्.ऋते निर्.ऋते॒ तुभ्य॑ मस्तु । \newline
24. नि॒र्॒.ऋ॒त॒ इति॑ निः - ऋ॒ते॒ । \newline
25. तुभ्य॑ मस्त्वस्तु॒ तुभ्य॒म् तुभ्य॑ मस्तु । \newline
26. अ॒स्त्वित्य॑स्तु । \newline
27. दे॒वी म॒ह म॒हम् दे॒वीम् दे॒वी म॒हम् निर्.ऋ॑ति॒म् निर्.ऋ॑ति म॒हम् दे॒वीम् दे॒वी म॒हम् निर्.ऋ॑तिम् । \newline
28. अ॒हम् निर्.ऋ॑ति॒म् निर्.ऋ॑ति म॒ह म॒हम् निर्.ऋ॑तिं॒ ॅवन्द॑मानो॒ वन्द॑मानो॒ निर्.ऋ॑ति म॒ह म॒हम् निर्.ऋ॑तिं॒ ॅवन्द॑मानः । \newline
29. निर्.ऋ॑तिं॒ ॅवन्द॑मानो॒ वन्द॑मानो॒ निर्.ऋ॑ति॒म् निर्.ऋ॑तिं॒ ॅवन्द॑मानः पि॒ता पि॒ता वन्द॑मानो॒ निर्.ऋ॑ति॒म् निर्.ऋ॑तिं॒ ॅवन्द॑मानः पि॒ता । \newline
30. निर्.ऋ॑ति॒मिति॒ निः - ऋ॒ति॒म् । \newline
31. वन्द॑मानः पि॒ता पि॒ता वन्द॑मानो॒ वन्द॑मानः पि॒तेवे॑व पि॒ता वन्द॑मानो॒ वन्द॑मानः पि॒तेव॑ । \newline
32. पि॒तेवे॑व पि॒ता पि॒तेव॑ पु॒त्रम् पु॒त्र मि॑व पि॒ता पि॒तेव॑ पु॒त्रम् । \newline
33. इ॒व॒ पु॒त्रम् पु॒त्र मि॑वेव पु॒त्रम् द॑सये दसये पु॒त्र मि॑वेव पु॒त्रम् द॑सये । \newline
34. पु॒त्रम् द॑सये दसये पु॒त्रम् पु॒त्रम् द॑सये॒ वचो॑भि॒र् वचो॑भिर् दसये पु॒त्रम् पु॒त्रम् द॑सये॒ वचो॑भिः । \newline
35. द॒स॒ये॒ वचो॑भि॒र् वचो॑भिर् दसये दसये॒ वचो॑भिः । \newline
36. वचो॑भि॒रिति॒ वचः॑ - भिः॒ । \newline
37. विश्व॑स्य॒ या या विश्व॑स्य॒ विश्व॑स्य॒ या जाय॑मानस्य॒ जाय॑मानस्य॒ या विश्व॑स्य॒ विश्व॑स्य॒ या जाय॑मानस्य । \newline
38. या जाय॑मानस्य॒ जाय॑मानस्य॒ या या जाय॑मानस्य॒ वेद॒ वेद॒ जाय॑मानस्य॒ या या जाय॑मानस्य॒ वेद॑ । \newline
39. जाय॑मानस्य॒ वेद॒ वेद॒ जाय॑मानस्य॒ जाय॑मानस्य॒ वेद॒ शिरः॑शिरः॒ शिरः॑शिरो॒ वेद॒ जाय॑मानस्य॒ जाय॑मानस्य॒ वेद॒ शिरः॑शिरः । \newline
40. वेद॒ शिरः॑शिरः॒ शिरः॑शिरो॒ वेद॒ वेद॒ शिरः॑शिरः॒ प्रति॒ प्रति॒ शिरः॑शिरो॒ वेद॒ वेद॒ शिरः॑शिरः॒ प्रति॑ । \newline
41. शिरः॑शिरः॒ प्रति॒ प्रति॒ शिरः॑शिरः॒ शिरः॑शिरः॒ प्रति॑ सू॒री सू॒री प्रति॒ शिरः॑शिरः॒ शिरः॑शिरः॒ प्रति॑ सू॒री । \newline
42. शिरः॑शिर॒ इति॒ शिरः॑ - शि॒रः॒ । \newline
43. प्रति॑ सू॒री सू॒री प्रति॒ प्रति॑ सू॒री वि वि सू॒री प्रति॒ प्रति॑ सू॒री वि । \newline
44. सू॒री वि वि सू॒री सू॒री वि च॑ष्टे चष्टे॒ वि सू॒री सू॒री वि च॑ष्टे । \newline
45. वि च॑ष्टे चष्टे॒ वि वि च॑ष्टे । \newline
46. च॒ष्ट॒ इति॑ चष्टे । \newline
47. नि॒वेश॑नः स॒ङ्गम॑नः स॒ङ्गम॑नो नि॒वेश॑नो नि॒वेश॑नः स॒ङ्गम॑नो॒ वसू॑नां॒ ॅवसू॑नाꣳ स॒ङ्गम॑नो नि॒वेश॑नो नि॒वेश॑नः स॒ङ्गम॑नो॒ वसू॑नाम् । \newline
48. नि॒वेश॑न॒ इति॑ नि - वेश॑नः । \newline
49. स॒ङ्गम॑नो॒ वसू॑नां॒ ॅवसू॑नाꣳ स॒ङ्गम॑नः स॒ङ्गम॑नो॒ वसू॑नां॒ ॅविश्वा॒ विश्वा॒ वसू॑नाꣳ स॒ङ्गम॑नः स॒ङ्गम॑नो॒ वसू॑नां॒ ॅविश्वा᳚ । \newline
50. स॒ङ्गम॑न॒ इति॑ सं - गम॑नः । \newline
51. वसू॑नां॒ ॅविश्वा॒ विश्वा॒ वसू॑नां॒ ॅवसू॑नां॒ ॅविश्वा॑ रू॒पा रू॒पा विश्वा॒ वसू॑नां॒ ॅवसू॑नां॒ ॅविश्वा॑ रू॒पा । \newline
52. विश्वा॑ रू॒पा रू॒पा विश्वा॒ विश्वा॑ रू॒पा ऽभ्य॑भि रू॒पा विश्वा॒ विश्वा॑ रू॒पा ऽभि । \newline
53. रू॒पा ऽभ्य॑भि रू॒पा रू॒पा ऽभि च॑ष्टे चष्टे अ॒भि रू॒पा रू॒पा ऽभि च॑ष्टे । \newline
54. अ॒भि च॑ष्टे चष्टे अ॒भ्य॑भि च॑ष्टे॒ शची॑भिः॒ शची॑भि श्चष्टे अ॒भ्य॑भि च॑ष्टे॒ शची॑भिः । \newline
55. च॒ष्टे॒ शची॑भिः॒ शची॑भि श्चष्टे चष्टे॒ शची॑भिः । \newline
\pagebreak
\markright{ TS 4.2.5.5  \hfill https://www.vedavms.in \hfill}

\section{ TS 4.2.5.5 }

\textbf{TS 4.2.5.5 } \newline
\textbf{Samhita Paata} \newline

शची॑भिः । दे॒व इ॑व सवि॒ता स॒त्यध॒र्मेन्द्रो॒ न त॑स्थौ सम॒रे प॑थी॒नां ॥ सं ॅव॑र॒त्रा द॑धातन॒ निरा॑हा॒वान् कृ॑णोतन । सि॒ञ्चाम॑हा अव॒टमु॒द्रिणं॑ ॅव॒यं ॅविश्वाऽहाऽद॑स्त॒मक्षि॑तं ॥ निष्कृ॑ताहा-वमव॒टꣳ सु॑वर॒त्रꣳ सु॑षेच॒नं । उ॒द्रिणꣳ॑ सिञ्चे॒ अक्षि॑तं ॥ सीरा॑ युञ्जन्ति क॒वयो॑ यु॒गा वि त॑न्वते॒ पृथ॑क् । धीरा॑ दे॒वेषु॑ सुम्न॒या ॥ यु॒नक्त॒ सीरा॒ वि यु॒गा त॑नोत कृ॒ते योनौ॑ वपते॒ह - [  ] \newline

\textbf{Pada Paata} \newline

शची॑भि॒रिति॒ शचि॑ - भिः॒ ॥ दे॒वः । इ॒व॒ । स॒वि॒ता । स॒त्यध॒र्मेति॑ स॒त्य - ध॒र्मा॒ । इन्द्रः॑ । न । त॒स्थौ॒ । स॒म॒र इति॑ सं - अ॒रे । प॒थी॒नाम् ॥ समिति॑ । व॒र॒त्राः । द॒धा॒त॒न॒ । निरिति॑ । आ॒हा॒वानित्या᳚ - हा॒वान् । कृ॒णो॒त॒न॒ ॥ सि॒ञ्चाम॑है । अ॒व॒टम् । उ॒द्रिण᳚म् । व॒यम् । विश्वा᳚ । अहा᳚ । अद॑स्तम् । अक्षि॑तम् ॥ निष्कृ॑ताहाव॒मिति॒ निष्कृ॑त - आ॒हा॒व॒म् । अ॒व॒टम् । सु॒व॒र॒त्रमिति॑ सु - व॒र॒त्रम् । सु॒षे॒च॒नमिति॑ सु - से॒च॒नम् ॥ उ॒द्रिण᳚म् । सि॒ञ्चे॒ । अक्षि॑तम् ॥ सीरा᳚ । यु॒ञ्ज॒न्ति॒ । क॒वयः॑ । यु॒गा । वीति॑ । त॒न्व॒ते॒ । पृथ॑क् ॥ धीराः᳚ । दे॒वेषु॑ । सु॒म्न॒या ॥ यु॒नक्त॑ । सीरा᳚ । वीति॑ । यु॒गा । त॒नो॒त॒ । कृ॒ते । योनौ᳚ । व॒प॒त॒ । इ॒ह ।  \newline


\textbf{Krama Paata} \newline

शची॑भि॒रिति॒ शचि॑ - भिः॒ ॥ दे॒व इ॑व । इ॒व॒ स॒वि॒ता । स॒वि॒ता स॒त्यध॑र्मा । स॒त्यध॒र्मेन्द्रः॑ । स॒त्यध॒र्मेति॑ स॒त्य - ध॒र्मा॒ । इन्द्रो॒ न । न त॑स्थौ । त॒स्थौ॒ स॒म॒रे । स॒म॒रे प॑थी॒नाम् । स॒म॒र इति॑ सम् - अ॒रे । प॒थी॒नामिति॑ पथी॒नाम् ॥ सं ॅव॑र॒त्राः । व॒र॒त्रा द॑धातन । द॒धा॒त॒न॒ निः । निरा॑हा॒वान् । आ॒हा॒वान् कृ॑णोतन । आ॒हा॒वानित्या᳚ - हा॒वान् । कृ॒णो॒त॒नेति॑ कृणोतन ॥ सि॒ञ्चाम॑हा अव॒टम् । अ॒व॒टमु॒द्रिण᳚म् । उ॒द्रिणं॑ ॅव॒यम् । व॒यं ॅविश्वा᳚ । विश्वाऽहा᳚ । अहाऽद॑स्तम् । अद॑स्त॒मक्षि॑तम् । अक्षि॑त॒मित्यक्षि॑तम् ॥ निष्कृ॑ताहावमव॒टम् । निष्कृ॑ताहाव॒मिति॒ निष्कृ॑त - आ॒हा॒व॒म् । अ॒व॒टꣳ सु॑वर॒त्रम् । सु॒व॒र॒त्रꣳ सु॑षेच॒नम् । सु॒व॒र॒त्रमिति॑ सु - व॒र॒त्रम् । सु॒षे॒च॒नमिति॑ सु - से॒च॒नम् ॥ उ॒द्रिणꣳ॑ सिञ्चे । सि॒ञ्चे॒ अक्षि॑तम् । अक्षि॑त॒मित्यक्षि॑तम् ॥ सीरा॑ युञ्जन्ति । यु॒ञ्ज॒न्ति॒ क॒वयः॑ । क॒वयो॑ यु॒गा । यु॒गा वि । वि त॑न्वते । त॒न्व॒ते॒ पृथ॑क् । पृथ॒गिति॒ पृथ॑क् ॥ धीरा॑ दे॒वेषु॑ । दे॒वेषु॑ सुम्न॒या । सु॒म्न॒येति॑ सुम्न॒या ॥ यु॒नक्त॒ सीरा᳚ । सीरा॒ वि । वि यु॒गा । यु॒गा त॑नोत । त॒नो॒त॒ कृ॒ते । कृ॒ते योनौ᳚ । योनौ॑ वपत । व॒प॒ते॒ह । 
इ॒ह बीज᳚म् \newline

\textbf{Jatai Paata} \newline

1. शची॑भि॒रिति॒ शचि॑ - भिः॒ । \newline
2. दे॒व इ॑वेव दे॒वो दे॒व इ॑व । \newline
3. इ॒व॒ स॒वि॒ता स॑वि॒तेवे॑व सवि॒ता । \newline
4. स॒वि॒ता स॒त्यध॑र्मा स॒त्यध॑र्मा सवि॒ता स॑वि॒ता स॒त्यध॑र्मा । \newline
5. स॒त्यध॒र्मेन्द्र॒ इन्द्रः॑ स॒त्यध॑र्मा स॒त्यध॒र्मेन्द्रः॑ । \newline
6. स॒त्यध॒र्मेति॑ स॒त्य - ध॒र्मा॒ । \newline
7. इन्द्रो॒ न नेन्द्र॒ इन्द्रो॒ न । \newline
8. न त॑स्थौ तस्थौ॒ न न त॑स्थौ । \newline
9. त॒स्थौ॒ स॒म॒रे स॑म॒रे त॑स्थौ तस्थौ सम॒रे । \newline
10. स॒म॒रे प॑थी॒नाम् प॑थी॒नाꣳ स॑म॒रे स॑म॒रे प॑थी॒नाम् । \newline
11. स॒म॒र इति॑ सं - अ॒रे । \newline
12. प॒थी॒नामिति॑ पथी॒नाम् । \newline
13. सं ॅव॑र॒त्रा व॑र॒त्राः सꣳ सं ॅव॑र॒त्राः । \newline
14. व॒र॒त्रा द॑धातन दधातन वर॒त्रा व॑र॒त्रा द॑धातन । \newline
15. द॒धा॒त॒न॒ निर् णिर् द॑धातन दधातन॒ निः । \newline
16. निरा॑हा॒वा ना॑हा॒वान् निर् णिरा॑हा॒वान् । \newline
17. आ॒हा॒वान् कृ॑णोतन कृणोतना हा॒वा ना॑हा॒वान् कृ॑णोतन । \newline
18. आ॒हा॒वानित्या᳚ - हा॒वान् । \newline
19. कृ॒णो॒त॒नेति॑ कृणोतन । \newline
20. सि॒ञ्चाम॑हा अव॒ट म॑व॒टꣳ सि॒ञ्चाम॑है सि॒ञ्चाम॑हा अव॒टम् । \newline
21. अ॒व॒ट मु॒द्रिण॑ मु॒द्रिण॑ मव॒ट म॑व॒ट मु॒द्रिण᳚म् । \newline
22. उ॒द्रिणं॑ ॅव॒यं ॅव॒य मु॒द्रिण॑ मु॒द्रिणं॑ ॅव॒यम् । \newline
23. व॒यं ॅविश्वा॒ विश्वा॑ व॒यं ॅव॒यं ॅविश्वा᳚ । \newline
24. विश्वा ऽहा ऽहा॒ विश्वा॒ विश्वा ऽहा᳚ । \newline
25. अहा ऽद॑स्त॒ मद॑स्त॒ महा ऽहा ऽद॑स्तम् । \newline
26. अद॑स्त॒ मक्षि॑त॒ मक्षि॑त॒ मद॑स्त॒ मद॑स्त॒ मक्षि॑तम् । \newline
27. अक्षि॑त॒मित्यक्षि॑तम् । \newline
28. निष्कृ॑ताहाव मव॒ट म॑व॒टम् निष्कृ॑ताहाव॒म् निष्कृ॑ताहाव मव॒टम् । \newline
29. निष्कृ॑ताहाव॒मिति॒ निष्कृ॑त - आ॒हा॒व॒म् । \newline
30. अ॒व॒टꣳ सु॑वर॒त्रꣳ सु॑वर॒त्र म॑व॒ट म॑व॒टꣳ सु॑वर॒त्रम् । \newline
31. सु॒व॒र॒त्रꣳ सु॑षेच॒नꣳ सु॑षेच॒नꣳ सु॑वर॒त्रꣳ सु॑वर॒त्रꣳ सु॑षेच॒नम् । \newline
32. सु॒व॒र॒त्रमिति॑ सु - व॒र॒त्रम् । \newline
33. सु॒षे॒च॒नमिति॑ सु - से॒च॒नम् । \newline
34. उ॒द्रिणꣳ॑ सिञ्चे सिञ्च उ॒द्रिण॑ मु॒द्रिणꣳ॑ सिञ्चे । \newline
35. सि॒ञ्चे॒ अक्षि॑त॒ मक्षि॑तꣳ सिञ्चे सिञ्चे॒ अक्षि॑तम् । \newline
36. अक्षि॑त॒मित्यक्षि॑तम् । \newline
37. सीरा॑ युञ्जन्ति युञ्जन्ति॒ सीरा॒ सीरा॑ युञ्जन्ति । \newline
38. यु॒ञ्ज॒न्ति॒ क॒वयः॑ क॒वयो॑ युञ्जन्ति युञ्जन्ति क॒वयः॑ । \newline
39. क॒वयो॑ यु॒गा यु॒गा क॒वयः॑ क॒वयो॑ यु॒गा । \newline
40. यु॒गा वि वि यु॒गा यु॒गा वि । \newline
41. वि त॑न्वते तन्वते॒ वि वि त॑न्वते । \newline
42. त॒न्व॒ते॒ पृथ॒क् पृथ॑क् तन्वते तन्वते॒ पृथ॑क् । \newline
43. पृथ॒गिति॒ पृथ॑क् । \newline
44. धीरा॑ दे॒वेषु॑ दे॒वेषु॒ धीरा॒ धीरा॑ दे॒वेषु॑ । \newline
45. दे॒वेषु॑ सुम्न॒या सु॑म्न॒या दे॒वेषु॑ दे॒वेषु॑ सुम्न॒या । \newline
46. सु॒म्न॒येति॑ सुम्न॒या । \newline
47. यु॒नक्त॒ सीरा॒ सीरा॑ यु॒नक्त॑ यु॒नक्त॒ सीरा᳚ । \newline
48. सीरा॒ वि वि सीरा॒ सीरा॒ वि । \newline
49. वि यु॒गा यु॒गा वि वि यु॒गा । \newline
50. यु॒गा त॑नोत तनोत यु॒गा यु॒गा त॑नोत । \newline
51. त॒नो॒त॒ कृ॒ते कृ॒ते त॑नोत तनोत कृ॒ते । \newline
52. कृ॒ते योनौ॒ योनौ॑ कृ॒ते कृ॒ते योनौ᳚ । \newline
53. योनौ॑ वपत वपत॒ योनौ॒ योनौ॑ वपत । \newline
54. व॒प॒ते॒ हेह व॑पत वपते॒ह । \newline
55. इ॒ह बीज॒म् बीज॑ मि॒हेह बीज᳚म् । \newline

\textbf{Ghana Paata } \newline

1. शची॑भि॒रिति॒ शचि॑ - भिः॒ । \newline
2. दे॒व इ॑वेव दे॒वो दे॒व इ॑व सवि॒ता स॑वि॒तेव॑ दे॒वो दे॒व इ॑व सवि॒ता । \newline
3. इ॒व॒ स॒वि॒ता स॑वि॒तेवे॑व सवि॒ता स॒त्यध॑र्मा स॒त्यध॑र्मा सवि॒तेवे॑व सवि॒ता स॒त्यध॑र्मा । \newline
4. स॒वि॒ता स॒त्यध॑र्मा स॒त्यध॑र्मा सवि॒ता स॑वि॒ता स॒त्यध॒र्मेन्द्र॒ इन्द्रः॑ स॒त्यध॑र्मा सवि॒ता स॑वि॒ता स॒त्यध॒र्मेन्द्रः॑ । \newline
5. स॒त्यध॒र्मेन्द्र॒ इन्द्रः॑ स॒त्यध॑र्मा स॒त्यध॒र्मेन्द्रो॒ न नेन्द्रः॑ स॒त्यध॑र्मा स॒त्यध॒र्मेन्द्रो॒ न । \newline
6. स॒त्यध॒र्मेति॑ स॒त्य - ध॒र्मा॒ । \newline
7. इन्द्रो॒ न नेन्द्र॒ इन्द्रो॒ न त॑स्थौ तस्थौ॒ नेन्द्र॒ इन्द्रो॒ न त॑स्थौ । \newline
8. न त॑स्थौ तस्थौ॒ न न त॑स्थौ सम॒रे स॑म॒रे त॑स्थौ॒ न न त॑स्थौ सम॒रे । \newline
9. त॒स्थौ॒ स॒म॒रे स॑म॒रे त॑स्थौ तस्थौ सम॒रे प॑थी॒नाम् प॑थी॒नाꣳ स॑म॒रे त॑स्थौ तस्थौ सम॒रे प॑थी॒नाम् । \newline
10. स॒म॒रे प॑थी॒नाम् प॑थी॒नाꣳ स॑म॒रे स॑म॒रे प॑थी॒नाम् । \newline
11. स॒म॒र इति॑ सं - अ॒रे । \newline
12. प॒थी॒नामिति॑ पथी॒नाम् । \newline
13. सं ॅव॑र॒त्रा व॑र॒त्राः सꣳ सं ॅव॑र॒त्रा द॑धातन दधातन वर॒त्राः सꣳ सं ॅव॑र॒त्रा द॑धातन । \newline
14. व॒र॒त्रा द॑धातन दधातन वर॒त्रा व॑र॒त्रा द॑धातन॒ निर् णिर् द॑धातन वर॒त्रा व॑र॒त्रा द॑धातन॒ निः । \newline
15. द॒धा॒त॒न॒ निर् णिर् द॑धातन दधातन॒ निरा॑हा॒वा ना॑हा॒वान् निर् द॑धातन दधातन॒ निरा॑हा॒वान् । \newline
16. निरा॑हा॒वा ना॑हा॒वान् निर् णिरा॑हा॒वान् कृ॑णोतन कृणोतना हा॒वान् निर् णिरा॑हा॒वान् कृ॑णोतन । \newline
17. आ॒हा॒वान् कृ॑णोतन कृणोतना हा॒वा ना॑हा॒वान् कृ॑णोतन । \newline
18. आ॒हा॒वानित्या᳚ - हा॒वान् । \newline
19. कृ॒णो॒त॒नेति॑ कृणोतन । \newline
20. सि॒ञ्चाम॑हा अव॒ट म॑व॒टꣳ सि॒ञ्चाम॑है सि॒ञ्चाम॑हा अव॒ट मु॒द्रिण॑ मु॒द्रिण॑ मव॒टꣳ सि॒ञ्चाम॑है सि॒ञ्चाम॑हा अव॒ट मु॒द्रिण᳚म् । \newline
21. अ॒व॒ट मु॒द्रिण॑ मु॒द्रिण॑ मव॒ट म॑व॒ट मु॒द्रिणं॑ ॅव॒यं ॅव॒य मु॒द्रिण॑ मव॒ट म॑व॒ट मु॒द्रिणं॑ ॅव॒यम् । \newline
22. उ॒द्रिणं॑ ॅव॒यं ॅव॒य मु॒द्रिण॑ मु॒द्रिणं॑ ॅव॒यं ॅविश्वा॒ विश्वा॑ व॒य मु॒द्रिण॑ मु॒द्रिणं॑ ॅव॒यं ॅविश्वा᳚ । \newline
23. व॒यं ॅविश्वा॒ विश्वा॑ व॒यं ॅव॒यं ॅविश्वा ऽहा ऽहा॒ विश्वा॑ व॒यं ॅव॒यं ॅविश्वा ऽहा᳚ । \newline
24. विश्वा ऽहा ऽहा॒ विश्वा॒ विश्वा ऽहा ऽद॑स्त॒ मद॑स्त॒ महा॒ विश्वा॒ विश्वा ऽहा ऽद॑स्तम् । \newline
25. अहा ऽद॑स्त॒ मद॑स्त॒ महा ऽहा ऽद॑स्त॒ मक्षि॑त॒ मक्षि॑त॒ मद॑स्त॒ महा ऽहा ऽद॑स्त॒ मक्षि॑तम् । \newline
26. अद॑स्त॒ मक्षि॑त॒ मक्षि॑त॒ मद॑स्त॒ मद॑स्त॒ मक्षि॑तम् । \newline
27. अक्षि॑त॒मित्यक्षि॑तम् । \newline
28. निष्कृ॑ताहाव मव॒ट म॑व॒टम् निष्कृ॑ताहाव॒म् निष्कृ॑ताहाव मव॒टꣳ सु॑वर॒त्रꣳ सु॑वर॒त्र म॑व॒टम् निष्कृ॑ताहाव॒म् निष्कृ॑ताहाव मव॒टꣳ सु॑वर॒त्रम् । \newline
29. निष्कृ॑ताहाव॒मिति॒ निष्कृ॑त - आ॒हा॒व॒म् । \newline
30. अ॒व॒टꣳ सु॑वर॒त्रꣳ सु॑वर॒त्र म॑व॒ट म॑व॒टꣳ सु॑वर॒त्रꣳ सु॑षेच॒नꣳ सु॑षेच॒नꣳ सु॑वर॒त्र म॑व॒ट म॑व॒टꣳ सु॑वर॒त्रꣳ सु॑षेच॒नम् । \newline
31. सु॒व॒र॒त्रꣳ सु॑षेच॒नꣳ सु॑षेच॒नꣳ सु॑वर॒त्रꣳ सु॑वर॒त्रꣳ सु॑षेच॒नम् । \newline
32. सु॒व॒र॒त्रमिति॑ सु - व॒र॒त्रम् । \newline
33. सु॒षे॒च॒नमिति॑ सु - से॒च॒नम् । \newline
34. उ॒द्रिणꣳ॑ सिञ्चे सिञ्च उ॒द्रिण॑ मु॒द्रिणꣳ॑ सिञ्चे॒ अक्षि॑त॒ मक्षि॑तꣳ सिञ्च उ॒द्रिण॑ मु॒द्रिणꣳ॑ सिञ्चे॒ अक्षि॑तम् । \newline
35. सि॒ञ्चे॒ अक्षि॑त॒ मक्षि॑तꣳ सिञ्चे सिञ्चे॒ अक्षि॑तम् । \newline
36. अक्षि॑त॒मित्यक्षि॑तम् । \newline
37. सीरा॑ युञ्जन्ति युञ्जन्ति॒ सीरा॒ सीरा॑ युञ्जन्ति क॒वयः॑ क॒वयो॑ युञ्जन्ति॒ सीरा॒ सीरा॑ युञ्जन्ति क॒वयः॑ । \newline
38. यु॒ञ्ज॒न्ति॒ क॒वयः॑ क॒वयो॑ युञ्जन्ति युञ्जन्ति क॒वयो॑ यु॒गा यु॒गा क॒वयो॑ युञ्जन्ति युञ्जन्ति क॒वयो॑ यु॒गा । \newline
39. क॒वयो॑ यु॒गा यु॒गा क॒वयः॑ क॒वयो॑ यु॒गा वि वि यु॒गा क॒वयः॑ क॒वयो॑ यु॒गा वि । \newline
40. यु॒गा वि वि यु॒गा यु॒गा वि त॑न्वते तन्वते॒ वि यु॒गा यु॒गा वि त॑न्वते । \newline
41. वि त॑न्वते तन्वते॒ वि वि त॑न्वते॒ पृथ॒क् पृथ॑क् तन्वते॒ वि वि त॑न्वते॒ पृथ॑क् । \newline
42. त॒न्व॒ते॒ पृथ॒क् पृथ॑क् तन्वते तन्वते॒ पृथ॑क् । \newline
43. पृथ॒गिति॒ पृथ॑क् । \newline
44. धीरा॑ दे॒वेषु॑ दे॒वेषु॒ धीरा॒ धीरा॑ दे॒वेषु॑ सुम्न॒या सु॑म्न॒या दे॒वेषु॒ धीरा॒ धीरा॑ दे॒वेषु॑ सुम्न॒या । \newline
45. दे॒वेषु॑ सुम्न॒या सु॑म्न॒या दे॒वेषु॑ दे॒वेषु॑ सुम्न॒या । \newline
46. सु॒म्न॒येति॑ सुम्न॒या । \newline
47. यु॒नक्त॒ सीरा॒ सीरा॑ यु॒नक्त॑ यु॒नक्त॒ सीरा॒ वि वि सीरा॑ यु॒नक्त॑ यु॒नक्त॒ सीरा॒ वि । \newline
48. सीरा॒ वि वि सीरा॒ सीरा॒ वि यु॒गा यु॒गा वि सीरा॒ सीरा॒ वि यु॒गा । \newline
49. वि यु॒गा यु॒गा वि वि यु॒गा त॑नोत तनोत यु॒गा वि वि यु॒गा त॑नोत । \newline
50. यु॒गा त॑नोत तनोत यु॒गा यु॒गा त॑नोत कृ॒ते कृ॒ते त॑नोत यु॒गा यु॒गा त॑नोत कृ॒ते । \newline
51. त॒नो॒त॒ कृ॒ते कृ॒ते त॑नोत तनोत कृ॒ते योनौ॒ योनौ॑ कृ॒ते त॑नोत तनोत कृ॒ते योनौ᳚ । \newline
52. कृ॒ते योनौ॒ योनौ॑ कृ॒ते कृ॒ते योनौ॑ वपत वपत॒ योनौ॑ कृ॒ते कृ॒ते योनौ॑ वपत । \newline
53. योनौ॑ वपत वपत॒ योनौ॒ योनौ॑ वपते॒ हेह व॑पत॒ योनौ॒ योनौ॑ वपते॒ह । \newline
54. व॒प॒ते॒ हेह व॑पत वपते॒ह बीज॒म् बीज॑ मि॒ह व॑पत वपते॒ह बीज᳚म् । \newline
55. इ॒ह बीज॒म् बीज॑ मि॒हेह बीज᳚म् । \newline
\pagebreak
\markright{ TS 4.2.5.6  \hfill https://www.vedavms.in \hfill}

\section{ TS 4.2.5.6 }

\textbf{TS 4.2.5.6 } \newline
\textbf{Samhita Paata} \newline

बीजं᳚ । गि॒रा च॑ श्रु॒ष्टिः सभ॑रा॒ अस॑न्नो॒ नेदी॑य॒ इथ् सृ॒ण्या॑ प॒क्वमा ऽय॑त् ॥ लाङ्ग॑लं॒ पवी॑रवꣳ सु॒शेवꣳ॑ सुम॒तिथ्स॑रु । उदित् कृ॑षति॒ गामविं॑ प्रफ॒र्व्यं॑ च॒ पीव॑रीं । प्र॒स्थाव॑द्-रथ॒वाह॑नं ॥ शु॒नं नः॒ फाला॒ वि तु॑दन्तु॒ भूमिꣳ॑ शु॒नं की॒नाशा॑ अ॒भि य॑न्तु वा॒हान् । शु॒नं प॒र्जन्यो॒ मधु॑ना॒ पयो॑भिः॒ शुना॑सीरा शु॒नम॒स्मासु॑ धत्तं ॥ कामं॑ कामदुघे धुक्ष्व मि॒त्राय॒ ( ) वरु॑णाय च । इन्द्रा॑या॒ग्नये॑ पू॒ष्ण ओष॑धीभ्यः प्र॒जाभ्यः॑ ॥घृ॒तेन॒ सीता॒ मधु॑ना॒ सम॑क्ता॒ विश्वै᳚र्दे॒वैरनु॑मता म॒रुद्भिः॑ । ऊर्ज॑स्वती॒ पय॑सा॒ पिन्व॑माना॒ऽस्मान्थ् सी॑ते॒ पय॑सा॒ऽभ्या-व॑वृथ्स्व ॥ \newline

\textbf{Pada Paata} \newline

बीज᳚म् ॥ गि॒रा । च॒ । श्रु॒ष्टिः । सभ॑रा॒ इति॒ स-भ॒राः॒ । अस॑त् । नः॒ । नेदी॑यः । इत् । सृ॒ण्या᳚ । प॒क्वम् । एति॑ । अ॒य॒त् ॥ लाङ्ग॑लम् । पवी॑रवम् । सु॒शेव॒मिति॑ सु - शेव᳚म् । सु॒म॒तिथ्स॒र्विति॑ सुम॒ति - थ्‌स॒रु॒ ॥ उदिति॑ । इत् । कृ॒ष॒ति॒ । गाम् । अवि᳚म् । प्र॒फ॒र्व्य॑मिति॑ प्र-फ॒र्व्य᳚म् । च॒ । पीव॑रीम् ॥ प्र॒स्थाव॒दिति॑ प्र॒स्थ-व॒त् । र॒थ॒वाह॑न॒मिति॑ रथ - वाह॑नम् ॥ शु॒नम् । नः॒ । फालाः᳚ । वीति॑ । तु॒द॒न्तु॒ । भूमि᳚म् । शु॒नम् । की॒नाशाः᳚ । अ॒भीति॑ । य॒न्तु॒ । वा॒हान् ॥ शु॒नम् । प॒र्जन्यः॑ । मधु॑ना । पयो॑भि॒रिति॒ पयः॑ - भिः॒ । शुना॑सीरा । शु॒नम् । अ॒स्मासु॑ । ध॒त्त॒म् ॥ काम᳚म् । का॒म॒दु॒घ॒ इति॑ काम - दु॒घे॒ । धु॒क्ष्व॒ । मि॒त्राय॑ ( ) । वरु॑णाय । च॒ ॥ इन्द्रा॑य । अ॒ग्नये᳚ । पू॒ष्णे । ओष॑धीभ्य॒ इत्योष॑धि - भ्यः॒ । प्र॒जाभ्य॒ इति॑ प्र - जाभ्यः॑ ॥ घृ॒तेन॑ । सीता᳚ । मधु॑ना । सम॒क्तेति॒ सम् - अ॒क्ता॒ । विश्वैः᳚ । दे॒वैः । अनु॑म॒तेत्यनु॑-म॒ता॒ । म॒रुद्भि॒रिति॑ म॒रुत् -भिः॒ ॥ ऊर्ज॑स्वती । पय॑सा । पिन्व॑माना । अ॒स्मान् । सी॒ते॒ । पय॑सा । अ॒भ्याव॑वृ॒थ्स्वेत्य॑भि - आव॑वृथ्स्व ॥  \newline


\textbf{Krama Paata} \newline

बीज॒मिति॒ बीज᳚म् ॥ गि॒रा च॑ । च॒ श्रु॒ष्टिः । श्रु॒ष्टिः सभ॑राः । सभ॑रा॒ अस॑त् । सभ॑रा॒ इति॒ स - भ॒राः॒ । अस॑न् नः । नो॒ नेदी॑यः । नेदी॑य॒ इत् । इथ् सृ॒ण्या᳚ । सृ॒ण्या॑ प॒क्वम् । प॒क्वमा । आऽय॑त् । अ॒य॒दित्य॑यत् ॥ लाङ्ग॑ल॒म् पवी॑रवम् । पवी॑रवꣳ सु॒शेव᳚म् । सु॒शेवꣳ॑ सुम॒तिथ्स॑रु । सु॒शेव॒मिति॑ सु - शेव᳚म् । सु॒म॒तिथ्स॒र्विति॑ सुम॒ति - थ्स॒रु॒ ॥ उदित् । इत् कृ॑षति । कृ॒ष॒ति॒ गाम् । गामवि᳚म् । अवि॑म् प्रफ॒र्व्य᳚म् । प्र॒फ॒र्व्य॑म् च । प्र॒फ॒र्व्य॑मिति॑ प्र - फ॒र्व्य᳚म् । च॒ पीव॑रीम् । पीव॑री॒मिति॒ पीव॑रीम् ॥ प्र॒स्थाव॑द् रथ॒वाह॑नम् । प्र॒स्थाव॒दिति॑ प्र॒स्थ - व॒त्॒ । र॒थ॒वाह॑न॒मिति॑ रथ - वाह॑नम् ॥ शु॒नम् नः॑ । नः॒ फालाः᳚ । फाला॒ वि । वि तु॑दन्तु । तु॒द॒न्तु॒ भूमि᳚म् । भूमिꣳ॑ शु॒नम् । शु॒नम् की॒नाशाः᳚ । की॒नाशा॑ अ॒भि । अ॒भि य॑न्तु । य॒न्तु॒ वा॒हान् । वा॒हानिति॑ वा॒हान् ॥ शु॒नम् प॒र्जन्यः॑ । प॒र्जन्यो॒ मधु॑ना । मधु॑ना॒ पयो॑भिः । पयो॑भिः॒ शुना॑सीरा । पयो॑भि॒रिति॒ पयः॑ - भिः॒ । शुना॑सीरा शु॒नम् । शु॒नम॒स्मासु॑ । अ॒स्मासु॑ धत्तम् । ध॒त्त॒मिति॑ धत्तम् ॥ काम॑म् कामदुघे । का॒म॒दु॒घे॒ धु॒क्ष्व॒ । का॒म॒दु॒घ॒ इति॑ काम - दु॒घे॒ । धु॒क्ष्व॒ मि॒त्राय॑ ( ) । मि॒त्राय॒ वरु॑णाय । वरु॑णाय च । चेति॑ च ॥ इन्द्रा॑या॒ग्नये᳚ । अ॒ग्नये॑ पू॒ष्णे । पू॒ष्ण ओष॑धीभ्यः । ओष॑धीभ्यः प्र॒जाभ्यः॑ । ओष॑धीभ्य॒ इत्योष॑धि - भ्यः॒ । प्र॒जाभ्य॒ इति॑ प्र - जाभ्यः॑ ॥ घृ॒तेन॒ सीता᳚ । सीता॒ मधु॑ना । मधु॑ना॒ सम॑क्ता । सम॑क्ता॒ विश्वैः᳚ । सम॒क्तेति॒ सम् - अ॒क्ता॒ । विश्वै᳚र् दे॒वैः । दे॒वैरनु॑मता । अनु॑मता म॒रुद्भिः॑ । अनु॑म॒तेत्यनु॑ - म॒ता॒ । म॒रुद्भि॒रिति॑ म॒रुत् - भिः॒ ॥ ऊर्ज॑स्वती॒ पय॑सा । पय॑सा॒ पिन्व॑माना । पिन्व॑माना॒ऽस्मान् । अ॒स्मान्थ् सी॑ते । सी॒ते॒ पय॑सा । पय॑सा॒ ऽभ्याव॑वृथ्स्व । अ॒भ्याव॑वृ॒थ्स्वेत्य॑भि - आव॑वृथ्स्व । \newline

\textbf{Jatai Paata} \newline

1. बीज॒मिति॒ बीज᳚म् । \newline
2. गि॒रा च॑ च गि॒रा गि॒रा च॑ । \newline
3. च॒ श्रु॒ष्टिः श्रु॒ष्टिश्च॑ च श्रु॒ष्टिः । \newline
4. श्रु॒ष्टिः सभ॑राः॒ सभ॑राः श्रु॒ष्टिः श्रु॒ष्टिः सभ॑राः । \newline
5. सभ॑रा॒ अस॒ दस॒थ् सभ॑राः॒ सभ॑रा॒ अस॑त् । \newline
6. सभ॑रा॒ इति॒ स - भ॒राः॒ । \newline
7. अस॑न् नो नो॒ अस॒ दस॑न् नः । \newline
8. नो॒ नेदी॑यो॒ नेदी॑यो नो नो॒ नेदी॑यः । \newline
9. नेदी॑य॒ इदिन् नेदी॑यो॒ नेदी॑य॒ इत् । \newline
10. इथ् सृ॒ण्या॑ सृ॒ण्ये॑दिथ् सृ॒ण्या᳚ । \newline
11. सृ॒ण्या॑ प॒क्वम् प॒क्वꣳ सृ॒ण्या॑ सृ॒ण्या॑ प॒क्वम् । \newline
12. प॒क्व मा प॒क्वम् प॒क्व मा । \newline
13. आ ऽय॑ दय॒दा ऽय॑त् । \newline
14. अ॒य॒दित्य॑यत् । \newline
15. लाङ्ग॑ल॒म् पवी॑रव॒म् पवी॑रव॒म् ॅलाङ्ग॑ल॒म् ॅलाङ्ग॑ल॒म् पवी॑रवम् । \newline
16. पवी॑रवꣳ सु॒शेवꣳ॑ सु॒शेव॒म् पवी॑रव॒म् पवी॑रवꣳ सु॒शेव᳚म् । \newline
17. सु॒शेवꣳ॑ सुम॒तिथ्स॑रु सुम॒तिथ्स॑रु सु॒शेवꣳ॑ सु॒शेवꣳ॑ सुम॒तिथ्स॑रु । \newline
18. सु॒शेव॒मिति॑ सु - शेव᳚म् । \newline
19. सु॒म॒तिथ्स॒र्विति॑ सुम॒ति - थ्स॒रु॒ । \newline
20. उदि दिदु दुदित् । \newline
21. इत् कृ॑षति कृष॒ती दित् कृ॑षति । \newline
22. कृ॒ष॒ति॒ गाम् गाम् कृ॑षति कृषति॒ गाम् । \newline
23. गा मवि॒ मवि॒म् गाम् गा मवि᳚म् । \newline
24. अवि॑म् प्रफ॒र्व्य॑म् प्रफ॒र्व्य॑ मवि॒ मवि॑म् प्रफ॒र्व्य᳚म् । \newline
25. प्र॒फ॒र्व्य॑म् च च प्रफ॒र्व्य॑म् प्रफ॒र्व्य॑म् च । \newline
26. प्र॒फ॒र्व्य॑मिति॑ प्र - फ॒र्व्य᳚म् । \newline
27. च॒ पीव॑री॒म् पीव॑रीम् च च॒ पीव॑रीम् । \newline
28. पीव॑री॒मिति॒ पीव॑रीम् । \newline
29. प्र॒स्थाव॑द् रथ॒वाह॑नꣳ रथ॒वाह॑नम् प्र॒स्थाव॑त् प्र॒स्थाव॑द् रथ॒वाह॑नम् । \newline
30. प्र॒स्थाव॒दिति॑ प्र॒स्थ - व॒त् । \newline
31. र॒थ॒वाह॑न॒मिति॑ रथ - वाह॑नम् । \newline
32. शु॒नम् नो॑ नः शु॒नꣳ शु॒नम् नः॑ । \newline
33. नः॒ फालाः॒ फाला॑ नो नः॒ फालाः᳚ । \newline
34. फाला॒ वि वि फालाः॒ फाला॒ वि । \newline
35. वि तु॑दन्तु तुदन्तु॒ वि वि तु॑दन्तु । \newline
36. तु॒द॒न्तु॒ भूमि॒म् भूमि॑म् तुदन्तु तुदन्तु॒ भूमि᳚म् । \newline
37. भूमिꣳ॑ शु॒नꣳ शु॒नम् भूमि॒म् भूमिꣳ॑ शु॒नम् । \newline
38. शु॒नम् की॒नाशाः᳚ की॒नाशाः᳚ शु॒नꣳ शु॒नम् की॒नाशाः᳚ । \newline
39. की॒नाशा॑ अ॒भ्य॑भि की॒नाशाः᳚ की॒नाशा॑ अ॒भि । \newline
40. अ॒भि य॑न्तु यन्त्व॒भ्य॑भि य॑न्तु । \newline
41. य॒न्तु॒ वा॒हान्. वा॒हान्. य॑न्तु यन्तु वा॒हान् । \newline
42. वा॒हानिति॑ वा॒हान् । \newline
43. शु॒नम् प॒र्जन्यः॑ प॒र्जन्यः॑ शु॒नꣳ शु॒नम् प॒र्जन्यः॑ । \newline
44. प॒र्जन्यो॒ मधु॑ना॒ मधु॑ना प॒र्जन्यः॑ प॒र्जन्यो॒ मधु॑ना । \newline
45. मधु॑ना॒ पयो॑भिः॒ पयो॑भि॒र् मधु॑ना॒ मधु॑ना॒ पयो॑भिः । \newline
46. पयो॑भिः॒ शुना॑सीरा॒ शुना॑सीरा॒ पयो॑भिः॒ पयो॑भिः॒ शुना॑सीरा । \newline
47. पयो॑भि॒रिति॒ पयः॑ - भिः॒ । \newline
48. शुना॑सीरा शु॒नꣳ शु॒नꣳ शुना॑सीरा॒ शुना॑सीरा शु॒नम् । \newline
49. शु॒न म॒स्मा स्व॒स्मासु॑ शु॒नꣳ शु॒न म॒स्मासु॑ । \newline
50. अ॒स्मासु॑ धत्तम् धत्त म॒स्मा स्व॒स्मासु॑ धत्तम् । \newline
51. ध॒त्त॒मिति॑ धत्तम् । \newline
52. काम॑म् कामदुघे कामदुघे॒ काम॒म् काम॑म् कामदुघे । \newline
53. का॒म॒दु॒घे॒ धु॒क्ष्व॒ धु॒क्ष्व॒ का॒म॒दु॒घे॒ का॒म॒दु॒घे॒ धु॒क्ष्व॒ । \newline
54. का॒म॒दु॒घ॒ इति॑ काम - दु॒घे॒ । \newline
55. धु॒क्ष्व॒ मि॒त्राय॑ मि॒त्राय॑ धुक्ष्व धुक्ष्व मि॒त्राय॑ । \newline
56. मि॒त्राय॒ वरु॑णाय॒ वरु॑णाय मि॒त्राय॑ मि॒त्राय॒ वरु॑णाय । \newline
57. वरु॑णाय च च॒ वरु॑णाय॒ वरु॑णाय च । \newline
58. चेति॑ च । \newline
59. इन्द्रा॑या॒ग्नये॑ अ॒ग्नय॒ इन्द्रा॒ येन्द्रा॑या॒ग्नये᳚ । \newline
60. अ॒ग्नये॑ पू॒ष्णे पू॒ष्णे अ॒ग्नये॑ अ॒ग्नये॑ पू॒ष्णे । \newline
61. पू॒ष्ण ओष॑धीभ्य॒ ओष॑धीभ्यः पू॒ष्णे पू॒ष्ण ओष॑धीभ्यः । \newline
62. ओष॑धीभ्यः प्र॒जाभ्यः॑ प्र॒जाभ्य॒ ओष॑धीभ्य॒ ओष॑धीभ्यः प्र॒जाभ्यः॑ । \newline
63. ओष॑धीभ्य॒ इत्योष॑धि - भ्यः॒ । \newline
64. प्र॒जाभ्य॒ इति॑ प्र - जाभ्यः॑ । \newline
65. घृ॒तेन॒ सीता॒ सीता॑ घृ॒तेन॑ घृ॒तेन॒ सीता᳚ । \newline
66. सीता॒ मधु॑ना॒ मधु॑ना॒ सीता॒ सीता॒ मधु॑ना । \newline
67. मधु॑ना॒ सम॑क्ता॒ सम॑क्ता॒ मधु॑ना॒ मधु॑ना॒ सम॑क्ता । \newline
68. सम॑क्ता॒ विश्वै॒र् विश्वैः॒ सम॑क्ता॒ सम॑क्ता॒ विश्वैः᳚ । \newline
69. सम॒क्तेति॒ सम् - अ॒क्ता॒ । \newline
70. विश्वै᳚र् दे॒वैर् दे॒वैर् विश्वै॒र् विश्वै᳚र् दे॒वैः । \newline
71. दे॒वै रनु॑म॒ता ऽनु॑मता दे॒वैर् दे॒वै रनु॑मता । \newline
72. अनु॑मता म॒रुद्भि॑र् म॒रुद्भि॒ रनु॑म॒ता ऽनु॑मता म॒रुद्भिः॑ । \newline
73. अनु॑म॒तेत्यनु॑ - म॒ता॒ । \newline
74. म॒रुद्भि॒रिति॑ म॒रुत् - भिः॒ । \newline
75. ऊर्ज॑स्वती॒ पय॑सा॒ पय॒ सोर्ज॑स्व॒ त्यूर्ज॑स्वती॒ पय॑सा । \newline
76. पय॑सा॒ पिन्व॑माना॒ पिन्व॑माना॒ पय॑सा॒ पय॑सा॒ पिन्व॑माना । \newline
77. पिन्व॑माना॒ ऽस्मा न॒स्मान् पिन्व॑माना॒ पिन्व॑माना॒ ऽस्मान् । \newline
78. अ॒स्मान् थ्सी॑ते सीते अ॒स्मा न॒स्मान् थ्सी॑ते । \newline
79. सी॒ते॒ पय॑सा॒ पय॑सा सीते सीते॒ पय॑सा । \newline
80. पय॑सा॒ ऽभ्याव॑वृथ्स्वा॒ भ्याव॑वृथ्स्व॒ पय॑सा॒ पय॑सा॒ ऽभ्याव॑वृथ्स्व । \newline
81. अ॒भ्याव॑वृ॒थ्स्वेत्य॑भि - आव॑वृथ्स्व । \newline

\textbf{Ghana Paata } \newline

1. बीज॒मिति॒ बीज᳚म् । \newline
2. गि॒रा च॑ च गि॒रा गि॒रा च॑ श्रु॒ष्टिः श्रु॒ष्टिश्च॑ गि॒रा गि॒रा च॑ श्रु॒ष्टिः । \newline
3. च॒ श्रु॒ष्टिः श्रु॒ष्टिश्च॑ च श्रु॒ष्टिः सभ॑राः॒ सभ॑राः श्रु॒ष्टिश्च॑ च श्रु॒ष्टिः सभ॑राः । \newline
4. श्रु॒ष्टिः सभ॑राः॒ सभ॑राः श्रु॒ष्टिः श्रु॒ष्टिः सभ॑रा॒ अस॒ दस॒थ् सभ॑राः श्रु॒ष्टिः श्रु॒ष्टिः सभ॑रा॒ अस॑त् । \newline
5. सभ॑रा॒ अस॒ दस॒थ् सभ॑राः॒ सभ॑रा॒ अस॑न् नो नो॒ अस॒थ् सभ॑राः॒ सभ॑रा॒ अस॑न् नः । \newline
6. सभ॑रा॒ इति॒ स - भ॒राः॒ । \newline
7. अस॑न् नो नो॒ अस॒ दस॑न् नो॒ नेदी॑यो॒ नेदी॑यो नो॒ अस॒ दस॑न् नो॒ नेदी॑यः । \newline
8. नो॒ नेदी॑यो॒ नेदी॑यो नो नो॒ नेदी॑य॒ इदिन् नेदी॑यो नो नो॒ नेदी॑य॒ इत् । \newline
9. नेदी॑य॒ इदिन् नेदी॑यो॒ नेदी॑य॒ इथ् सृ॒ण्या॑ सृ॒ण्येदिन् नेदी॑यो॒ नेदी॑य॒ इथ् सृ॒ण्या᳚ । \newline
10. इथ् सृ॒ण्या॑ सृ॒ण्ये॑दिथ् सृ॒ण्या॑ प॒क्वम् प॒क्वꣳ सृ॒ण्ये॑दिथ् सृ॒ण्या॑ प॒क्वम् । \newline
11. सृ॒ण्या॑ प॒क्वम् प॒क्वꣳ सृ॒ण्या॑ सृ॒ण्या॑ प॒क्व मा प॒क्वꣳ सृ॒ण्या॑ सृ॒ण्या॑ प॒क्व मा । \newline
12. प॒क्व मा प॒क्वम् प॒क्व मा ऽय॑ दय॒दा प॒क्वम् प॒क्व मा ऽय॑त् । \newline
13. आ ऽय॑दय॒दा ऽय॑त् । \newline
14. अ॒य॒दित्य॑यत् । \newline
15. लाङ्ग॑ल॒म् पवी॑रव॒म् पवी॑रव॒म् ॅलाङ्ग॑ल॒म् ॅलाङ्ग॑ल॒म् पवी॑रवꣳ सु॒शेवꣳ॑ सु॒शेव॒म् पवी॑रव॒म् ॅलाङ्ग॑ल॒म् ॅलाङ्ग॑ल॒म् पवी॑रवꣳ सु॒शेव᳚म् । \newline
16. पवी॑रवꣳ सु॒शेवꣳ॑ सु॒शेव॒म् पवी॑रव॒म् पवी॑रवꣳ सु॒शेवꣳ॑ सुम॒तिथ्स॑रु सुम॒तिथ्स॑रु सु॒शेव॒म् पवी॑रव॒म् पवी॑रवꣳ सु॒शेवꣳ॑ सुम॒तिथ्स॑रु । \newline
17. सु॒शेवꣳ॑ सुम॒तिथ्स॑रु सुम॒तिथ्स॑रु सु॒शेवꣳ॑ सु॒शेवꣳ॑ सुम॒तिथ्स॑रु । \newline
18. सु॒शेव॒मिति॑ सु - शेव᳚म् । \newline
19. सु॒म॒तिथ्स॒र्विति॑ सुम॒ति - थ्स॒रु॒ । \newline
20. उदि दि दु दुदित् कृ॑षति कृष॒ती दुदु दित् कृ॑षति । \newline
21. इत् कृ॑षति कृष॒तीदित् कृ॑षति॒ गाम् गाम् कृ॑ष॒तीदित् कृ॑षति॒ गाम् । \newline
22. कृ॒ष॒ति॒ गाम् गाम् कृ॑षति कृषति॒ गा मवि॒ मवि॒म् गाम् कृ॑षति कृषति॒ गा मवि᳚म् । \newline
23. गा मवि॒ मवि॒म् गाम् गा मवि॑म् प्रफ॒र्व्य॑म् प्रफ॒र्व्य॑ मवि॒म् गाम् गा मवि॑म् प्रफ॒र्व्य᳚म् । \newline
24. अवि॑म् प्रफ॒र्व्य॑म् प्रफ॒र्व्य॑ मवि॒ मवि॑म् प्रफ॒र्व्य॑म् च च प्रफ॒र्व्य॑ मवि॒ मवि॑म् प्रफ॒र्व्य॑म् च । \newline
25. प्र॒फ॒र्व्य॑म् च च प्रफ॒र्व्य॑म् प्रफ॒र्व्य॑म् च॒ पीव॑री॒म् पीव॑रीम् च प्रफ॒र्व्य॑म् प्रफ॒र्व्य॑म् च॒ पीव॑रीम् । \newline
26. प्र॒फ॒र्व्य॑मिति॑ प्र - फ॒र्व्य᳚म् । \newline
27. च॒ पीव॑री॒म् पीव॑रीम् च च॒ पीव॑रीम् । \newline
28. पीव॑री॒मिति॒ पीव॑रीम् । \newline
29. प्र॒स्थाव॑द् रथ॒वाह॑नꣳ रथ॒वाह॑नम् प्र॒स्थाव॑त् प्र॒स्थाव॑द् रथ॒वाह॑नम् । \newline
30. प्र॒स्थाव॒दिति॑ प्र॒स्थ - व॒त् । \newline
31. र॒थ॒वाह॑न॒मिति॑ रथ - वाह॑नम् । \newline
32. शु॒नम् नो॑ नः शु॒नꣳ शु॒नम् नः॒ फालाः॒ फाला॑ नः शु॒नꣳ शु॒नम् नः॒ फालाः᳚ । \newline
33. नः॒ फालाः॒ फाला॑ नो नः॒ फाला॒ वि वि फाला॑ नो नः॒ फाला॒ वि । \newline
34. फाला॒ वि वि फालाः॒ फाला॒ वि तु॑दन्तु तुदन्तु॒ वि फालाः॒ फाला॒ वि तु॑दन्तु । \newline
35. वि तु॑दन्तु तुदन्तु॒ वि वि तु॑दन्तु॒ भूमि॒म् भूमि॑म् तुदन्तु॒ वि वि तु॑दन्तु॒ भूमि᳚म् । \newline
36. तु॒द॒न्तु॒ भूमि॒म् भूमि॑म् तुदन्तु तुदन्तु॒ भूमिꣳ॑ शु॒नꣳ शु॒नम् भूमि॑म् तुदन्तु तुदन्तु॒ भूमिꣳ॑ शु॒नम् । \newline
37. भूमिꣳ॑ शु॒नꣳ शु॒नम् भूमि॒म् भूमिꣳ॑ शु॒नम् की॒नाशाः᳚ की॒नाशाः᳚ शु॒नम् भूमि॒म् भूमिꣳ॑ शु॒नम् की॒नाशाः᳚ । \newline
38. शु॒नम् की॒नाशाः᳚ की॒नाशाः᳚ शु॒नꣳ शु॒नम् की॒नाशा॑ अ॒भ्य॑भि की॒नाशाः᳚ शु॒नꣳ शु॒नम् की॒नाशा॑ अ॒भि । \newline
39. की॒नाशा॑ अ॒भ्य॑भि की॒नाशाः᳚ की॒नाशा॑ अ॒भि य॑न्तु यन्त्व॒भि की॒नाशाः᳚ की॒नाशा॑ अ॒भि य॑न्तु । \newline
40. अ॒भि य॑न्तु यन्त्व॒भ्य॑भि य॑न्तु वा॒हान्. वा॒हान्. य॑न्त्व॒भ्य॑भि य॑न्तु वा॒हान् । \newline
41. य॒न्तु॒ वा॒हान्. वा॒हान्. य॑न्तु यन्तु वा॒हान् । \newline
42. वा॒हानिति॑ वा॒हान् । \newline
43. शु॒नम् प॒र्जन्यः॑ प॒र्जन्यः॑ शु॒नꣳ शु॒नम् प॒र्जन्यो॒ मधु॑ना॒ मधु॑ना प॒र्जन्यः॑ शु॒नꣳ शु॒नम् प॒र्जन्यो॒ मधु॑ना । \newline
44. प॒र्जन्यो॒ मधु॑ना॒ मधु॑ना प॒र्जन्यः॑ प॒र्जन्यो॒ मधु॑ना॒ पयो॑भिः॒ पयो॑भि॒र् मधु॑ना प॒र्जन्यः॑ प॒र्जन्यो॒ मधु॑ना॒ पयो॑भिः । \newline
45. मधु॑ना॒ पयो॑भिः॒ पयो॑भि॒र् मधु॑ना॒ मधु॑ना॒ पयो॑भिः॒ शुना॑सीरा॒ शुना॑सीरा॒ पयो॑भि॒र् मधु॑ना॒ मधु॑ना॒ पयो॑भिः॒ शुना॑सीरा । \newline
46. पयो॑भिः॒ शुना॑सीरा॒ शुना॑सीरा॒ पयो॑भिः॒ पयो॑भिः॒ शुना॑सीरा शु॒नꣳ शु॒नꣳ शुना॑सीरा॒ पयो॑भिः॒ पयो॑भिः॒ शुना॑सीरा शु॒नम् । \newline
47. पयो॑भि॒रिति॒ पयः॑ - भिः॒ । \newline
48. शुना॑सीरा शु॒नꣳ शु॒नꣳ शुना॑सीरा॒ शुना॑सीरा शु॒न म॒स्मा स्व॒स्मासु॑ शु॒नꣳ शुना॑सीरा॒ शुना॑सीरा शु॒न म॒स्मासु॑ । \newline
49. शु॒न म॒स्मा स्व॒स्मासु॑ शु॒नꣳ शु॒न म॒स्मासु॑ धत्तम् धत्त म॒स्मासु॑ शु॒नꣳ शु॒न म॒स्मासु॑ धत्तम् । \newline
50. अ॒स्मासु॑ धत्तम् धत्त म॒स्मा स्व॒स्मासु॑ धत्तम् । \newline
51. ध॒त्त॒मिति॑ धत्तम् । \newline
52. काम॑म् कामदुघे कामदुघे॒ काम॒म् काम॑म् कामदुघे धुक्ष्व धुक्ष्व कामदुघे॒ काम॒म् काम॑म् कामदुघे धुक्ष्व । \newline
53. का॒म॒दु॒घे॒ धु॒क्ष्व॒ धु॒क्ष्व॒ का॒म॒दु॒घे॒ का॒म॒दु॒घे॒ धु॒क्ष्व॒ मि॒त्राय॑ मि॒त्राय॑ धुक्ष्व कामदुघे कामदुघे धुक्ष्व मि॒त्राय॑ । \newline
54. का॒म॒दु॒घ॒ इति॑ काम - दु॒घे॒ । \newline
55. धु॒क्ष्व॒ मि॒त्राय॑ मि॒त्राय॑ धुक्ष्व धुक्ष्व मि॒त्राय॒ वरु॑णाय॒ वरु॑णाय मि॒त्राय॑ धुक्ष्व धुक्ष्व मि॒त्राय॒ वरु॑णाय । \newline
56. मि॒त्राय॒ वरु॑णाय॒ वरु॑णाय मि॒त्राय॑ मि॒त्राय॒ वरु॑णाय च च॒ वरु॑णाय मि॒त्राय॑ मि॒त्राय॒ वरु॑णाय च । \newline
57. वरु॑णाय च च॒ वरु॑णाय॒ वरु॑णाय च । \newline
58. चेति॑ च । \newline
59. इन्द्रा॑या॒ग्नये॑ अ॒ग्नय॒ इन्द्रा॒ येन्द्रा॑या॒ग्नये॑ पू॒ष्णे पू॒ष्णे अ॒ग्नय॒ इन्द्रा॒ येन्द्रा॑या॒ग्नये॑ पू॒ष्णे । \newline
60. अ॒ग्नये॑ पू॒ष्णे पू॒ष्णे अ॒ग्नये॑ अ॒ग्नये॑ पू॒ष्ण ओष॑धीभ्य॒ ओष॑धीभ्यः पू॒ष्णे 
अ॒ग्नये॑ अ॒ग्नये॑ पू॒ष्ण ओष॑धीभ्यः । \newline
61. पू॒ष्ण ओष॑धीभ्य॒ ओष॑धीभ्यः पू॒ष्णे पू॒ष्ण ओष॑धीभ्यः प्र॒जाभ्यः॑ प्र॒जाभ्य॒ ओष॑धीभ्यः पू॒ष्णे पू॒ष्ण ओष॑धीभ्यः प्र॒जाभ्यः॑ । \newline
62. ओष॑धीभ्यः प्र॒जाभ्यः॑ प्र॒जाभ्य॒ ओष॑धीभ्य॒ ओष॑धीभ्यः प्र॒जाभ्यः॑ । \newline
63. ओष॑धीभ्य॒ इत्योष॑धि - भ्यः॒ । \newline
64. प्र॒जाभ्य॒ इति॑ प्र - जाभ्यः॑ । \newline
65. घृ॒तेन॒ सीता॒ सीता॑ घृ॒तेन॑ घृ॒तेन॒ सीता॒ मधु॑ना॒ मधु॑ना॒ सीता॑ घृ॒तेन॑ घृ॒तेन॒ सीता॒ मधु॑ना । \newline
66. सीता॒ मधु॑ना॒ मधु॑ना॒ सीता॒ सीता॒ मधु॑ना॒ सम॑क्ता॒ सम॑क्ता॒ मधु॑ना॒ सीता॒ सीता॒ मधु॑ना॒ सम॑क्ता । \newline
67. मधु॑ना॒ सम॑क्ता॒ सम॑क्ता॒ मधु॑ना॒ मधु॑ना॒ सम॑क्ता॒ विश्वै॒र् विश्वैः॒ सम॑क्ता॒ मधु॑ना॒ मधु॑ना॒ सम॑क्ता॒ विश्वैः᳚ । \newline
68. सम॑क्ता॒ विश्वै॒र् विश्वैः॒ सम॑क्ता॒ सम॑क्ता॒ विश्वै᳚र् दे॒वैर् दे॒वैर् विश्वैः॒ सम॑क्ता॒ सम॑क्ता॒ विश्वै᳚र् दे॒वैः । \newline
69. सम॒क्तेति॒ सम् - अ॒क्ता॒ । \newline
70. विश्वै᳚र् दे॒वैर् दे॒वैर् विश्वै॒र् विश्वै᳚र् दे॒वै रनु॑म॒ता ऽनु॑मता दे॒वैर् विश्वै॒र् विश्वै᳚र् दे॒वै रनु॑मता । \newline
71. दे॒वै रनु॑म॒ता ऽनु॑मता दे॒वैर् दे॒वै रनु॑मता म॒रुद्भि॑र् म॒रुद्भि॒ रनु॑मता दे॒वैर् दे॒वै रनु॑मता म॒रुद्भिः॑ । \newline
72. अनु॑मता म॒रुद्भि॑र् म॒रुद्भि॒ रनु॑म॒ता ऽनु॑मता म॒रुद्भिः॑ । \newline
73. अनु॑म॒तेत्यनु॑ - म॒ता॒ । \newline
74. म॒रुद्भि॒रिति॑ म॒रुत् - भिः॒ । \newline
75. ऊर्ज॑स्वती॒ पय॑सा॒ पय॒सोर्ज॑स्व॒ त्यूर्ज॑स्वती॒ पय॑सा॒ पिन्व॑माना॒ पिन्व॑माना॒ पय॒सोर्ज॑स्व॒ त्यूर्ज॑स्वती॒ पय॑सा॒ पिन्व॑माना । \newline
76. पय॑सा॒ पिन्व॑माना॒ पिन्व॑माना॒ पय॑सा॒ पय॑सा॒ पिन्व॑माना॒ ऽस्मा न॒स्मान् पिन्व॑माना॒ पय॑सा॒ पय॑सा॒ पिन्व॑माना॒ ऽस्मान् । \newline
77. पिन्व॑माना॒ ऽस्मा न॒स्मान् पिन्व॑माना॒ पिन्व॑माना॒ ऽस्मान् थ्सी॑ते सीते अ॒स्मान् पिन्व॑माना॒ पिन्व॑माना॒ ऽस्मान् थ्सी॑ते । \newline
78. अ॒स्मान् थ्सी॑ते सीते अ॒स्मा न॒स्मान् थ्सी॑ते॒ पय॑सा॒ पय॑सा सीते अ॒स्मा न॒स्मान् थ्सी॑ते॒ पय॑सा । \newline
79. सी॒ते॒ पय॑सा॒ पय॑सा सीते सीते॒ पय॑सा॒ ऽभ्याव॑वृथ्स्वा॒ भ्याव॑वृथ्स्व॒ पय॑सा सीते सीते॒ पय॑सा॒ ऽभ्याव॑वृथ्स्व । \newline
80. पय॑सा॒ ऽभ्याव॑वृथ्स्वा॒ भ्याव॑वृथ्स्व॒ पय॑सा॒ पय॑सा॒ ऽभ्याव॑वृथ्स्व । \newline
81. अ॒भ्याव॑वृ॒थ्स्वेत्य॑भि - आव॑वृथ्स्व । \newline
\pagebreak
\markright{ TS 4.2.6.1  \hfill https://www.vedavms.in \hfill}

\section{ TS 4.2.6.1 }

\textbf{TS 4.2.6.1 } \newline
\textbf{Samhita Paata} \newline

या जा॒ता ओष॑धयो दे॒वेभ्य॑स्त्रियु॒गं पु॒रा ।मन्दा॑मि ब॒भ्रूणा॑म॒हꣳ श॒तं धामा॑नि स॒प्त च॑ ॥ श॒तं ॅवो॑ अबं॒ धामा॑नि स॒हस्र॑मु॒त वो॒ रुहः॑ । अथा॑ शतक्रत्वो यू॒यमि॒मं मे॑ अग॒दं कृ॑त ॥ पुष्पा॑वतीः प्र॒सूव॑तीः फ॒लिनी॑रफ॒ला उ॒त । अश्वा॑ इव स॒जित्व॑री-र्वी॒रुधः॑ पारयि॒ष्णवः॑ ॥ ओष॑धी॒रिति॑ मातर॒-स्तद्वो॑ देवी॒-रुप॑ ब्रुवे । रपाꣳ॑सि विघ्न॒तीरि॑त॒ रप॑ - [  ] \newline

\textbf{Pada Paata} \newline

याः । जा॒ताः । ओष॑धयः । दे॒वेभ्यः॑ । त्रि॒यु॒गमिति॑ त्रि-यु॒गम् । पु॒रा ॥ मन्दा॑मि । ब॒भ्रूणा᳚म् । अ॒हम् । श॒तम् । धामा॑नि । स॒प्त । च॒ ॥ श॒तम् । वः॒ । अ॒बं॒ । धामा॑नि । स॒हस्र᳚म् । उ॒त । वः॒ । रुहः॑ ॥ अथ॑ । श॒त॒क्र॒त्व॒ इति॑ शत - क्र॒त्वः॒ । यू॒यम् । इ॒मम् । मे॒ । अ॒ग॒दम् । कृ॒त॒ ॥ पुष्पा॑वती॒रिति॒ पुष्प॑-व॒तीः॒ । प्र॒सूव॑ती॒रिति॑ प्र - सूव॑तीः । फ॒लिनीः᳚ । अ॒फ॒लाः । उ॒त ॥ अश्वाः᳚ । इ॒व॒ । स॒जित्व॑री॒रिति॑ स - जित्व॑रीः । वी॒रुधः॑ । पा॒र॒यि॒ष्णवः॑ ॥ ओष॑धीः । इति॑ । मा॒त॒रः॒ । तत् । वः॒ । दे॒वीः॒ । उपेति॑ । ब्रु॒वे॒ ॥ रपाꣳ॑सि । वि॒घ्न॒तीरिति॑ वि - घ्न॒तीः । इ॒त॒ । रपः॑ ।  \newline


\textbf{Krama Paata} \newline

या जा॒ताः । जा॒ता ओष॑धयः । ओष॑धयो दे॒वेभ्यः॑ । दे॒वेभ्य॑ स्त्रियु॒गम् । त्रि॒यु॒गम् पु॒रा । त्रि॒यु॒गमिति॑ त्रि - यु॒गम् । पु॒रेति॑ पु॒रा ॥ मन्दा॑मि ब॒भ्रूणा᳚म् । ब॒भ्रूणा॑म॒हम् । अ॒हꣳ श॒तम् । श॒तम् धामा॑नि । धामा॑नि स॒प्त । स॒प्त च॑ । चेति॑ च ॥ श॒तं ॅवः॑ । वो॒ अ॒म्ब॒ । अ॒म्ब॒ धामा॑नि । धामा॑नि स॒हस्र᳚म् । स॒हस्र॑मु॒त । उ॒त वः॑ । वो॒ रुहः॑ । रुह॒ इति॒ रुहः॑ ॥ अथा॑ शतक्रत्वः । श॒त॒क्र॒त्वो॒ यू॒यम् । श॒त॒क्र॒त्व॒ इति॑ शत - क्र॒त्वः॒ । यू॒यमि॒मम् । इ॒मम् मे᳚ । मे॒ अ॒ग॒दम् । अ॒ग॒दम् कृ॑त । कृ॒तेति॑ कृत ॥ पुष्पा॑वतीः प्र॒सूव॑तीः । पुष्पा॑वती॒रिति॒ पुष्प॑ - व॒तीः॒ । प्र॒सूव॑तीः फ॒लिनीः᳚ । प्र॒सूव॑ती॒रिति॑ प्र - सूव॑तीः । फ॒लिनी॑रफ॒लाः । अ॒फ॒ला उ॒त । उ॒तेत्यु॒त ॥ अश्वा॑ इव । इ॒व॒ स॒जित्व॑रीः । स॒जित्व॑रीर् वी॒रुधः॑ । स॒जित्व॑री॒रिति॑ स - जित्व॑रीः । वी॒रुधः॑ पारयि॒ष्णवः॑ । पा॒र॒यि॒ष्णव॒ इति॑ पारयि॒ष्णवः॑ ॥ ओष॑धी॒रिति॑ । इति॑ मातरः । मा॒त॒र॒स्तत् । तद् वः॑ । वो॒ दे॒वीः॒ । दे॒वी॒रुप॑ । उप॑ ब्रुवे । ब्रु॒व॒ इति॑ ब्रुवे ॥ रपाꣳ॑सि विघ्न॒तीः । वि॒घ्न॒तीरि॑त । वि॒घ्न॒तीरिति॑ वि - घ्न॒तीः । इ॒त॒ रपः॑ । रप॑श्चा॒तय॑मानाः \newline

\textbf{Jatai Paata} \newline

1. या जा॒ता जा॒ता या या जा॒ताः । \newline
2. जा॒ता ओष॑धय॒ ओष॑धयो जा॒ता जा॒ता ओष॑धयः । \newline
3. ओष॑धयो दे॒वेभ्यो॑ दे॒वेभ्य॒ ओष॑धय॒ ओष॑धयो दे॒वेभ्यः॑ । \newline
4. दे॒वेभ्य॑ स्त्रियु॒गम् त्रि॑यु॒गम् दे॒वेभ्यो॑ दे॒वेभ्य॑ स्त्रियु॒गम् । \newline
5. त्रि॒यु॒गम् पु॒रा पु॒रा त्रि॑यु॒गम् त्रि॑यु॒गम् पु॒रा । \newline
6. त्रि॒यु॒गमिति॑ त्रि - यु॒गम् । \newline
7. पु॒रेति॑ पु॒रा । \newline
8. मन्दा॑मि ब॒भ्रूणा᳚म् ब॒भ्रूणा॒म् मन्दा॑मि॒ मन्दा॑मि ब॒भ्रूणा᳚म् । \newline
9. ब॒भ्रूणा॑ म॒ह म॒हम् ब॒भ्रूणा᳚म् ब॒भ्रूणा॑ म॒हम् । \newline
10. अ॒हꣳ श॒तꣳ श॒त म॒ह म॒हꣳ श॒तम् । \newline
11. श॒तम् धामा॑नि॒ धामा॑नि श॒तꣳ श॒तम् धामा॑नि । \newline
12. धामा॑नि स॒प्त स॒प्त धामा॑नि॒ धामा॑नि स॒प्त । \newline
13. स॒प्त च॑ च स॒प्त स॒प्त च॑ । \newline
14. चेति॑ च । \newline
15. श॒तं ॅवो॑ वः श॒तꣳ श॒तं ॅवः॑ । \newline
16. वो॒ अं॒बां॒ब॒ वो॒ वो॒ अं॒ब॒ । \newline
17. अं॒ब॒ धामा॑नि॒ धामा᳚ न्यम्बाम्ब॒ धामा॑नि । \newline
18. धामा॑नि स॒हस्रꣳ॑ स॒हस्र॒म् धामा॑नि॒ धामा॑नि स॒हस्र᳚म् । \newline
19. स॒हस्र॑ मु॒तोत स॒हस्रꣳ॑ स॒हस्र॑ मु॒त । \newline
20. उ॒त वो॑ व उ॒तोत वः॑ । \newline
21. वो॒ रुहो॒ रुहो॑ वो वो॒ रुहः॑ । \newline
22. रुह॒ इति॒ रुहः॑ । \newline
23. अथा॑ शतक्रत्वः शतक्रत्वो॒ अथाथा॑ शतक्रत्वः । \newline
24. श॒त॒क्र॒त्वो॒ यू॒यं ॅयू॒यꣳ श॑तक्रत्वः शतक्रत्वो यू॒यम् । \newline
25. श॒त॒क्र॒त्व॒ इति॑ शत - क्र॒त्वः॒ । \newline
26. यू॒य मि॒म मि॒मं ॅयू॒यं ॅयू॒य मि॒मम् । \newline
27. इ॒मम् मे॑ म इ॒म मि॒मम् मे᳚ । \newline
28. मे॒ अ॒ग॒द म॑ग॒दम् मे॑ मे अग॒दम् । \newline
29. अ॒ग॒दम् कृ॑त कृता ग॒द म॑ग॒दम् कृ॑त । \newline
30. कृ॒तेति॑ कृत । \newline
31. पुष्पा॑वतीः प्र॒सूव॑तीः प्र॒सूव॑तीः॒ पुष्पा॑वतीः॒ पुष्पा॑वतीः प्र॒सूव॑तीः । \newline
32. पुष्पा॑वती॒रिति॒ पुष्प॑ - व॒तीः॒ । \newline
33. प्र॒सूव॑तीः फ॒लिनीः᳚ फ॒लिनीः᳚ प्र॒सूव॑तीः प्र॒सूव॑तीः फ॒लिनीः᳚ । \newline
34. प्र॒सूव॑ती॒रिति॑ प्र - सूव॑तीः । \newline
35. फ॒लिनी॑ रफ॒ला अ॑फ॒लाः फ॒लिनीः᳚ फ॒लिनी॑ रफ॒लाः । \newline
36. अ॒फ॒ला उ॒तोता फ॒ला अ॑फ॒ला उ॒त । \newline
37. उ॒तेत्यु॒त । \newline
38. अश्वा॑ इवे॒वा श्वा॒ अश्वा॑ इव । \newline
39. इ॒व॒ स॒जित्व॑रीः स॒जित्व॑री रिवेव स॒जित्व॑रीः । \newline
40. स॒जित्व॑रीर् वी॒रुधो॑ वी॒रुधः॑ स॒जित्व॑रीः स॒जित्व॑रीर् वी॒रुधः॑ । \newline
41. स॒जित्व॑री॒रिति॑ स - जित्व॑रीः । \newline
42. वी॒रुधः॑ पारयि॒ष्णवः॑ पारयि॒ष्णवो॑ वी॒रुधो॑ वी॒रुधः॑ पारयि॒ष्णवः॑ । \newline
43. पा॒र॒यि॒ष्णव॒ इति॑ पारयि॒ष्णवः॑ । \newline
44. ओष॑धी॒ रिती त्योष॑धी॒ रोष॑धी॒ रिति॑ । \newline
45. इति॑ मातरो मातर॒ इतीति॑ मातरः । \newline
46. मा॒त॒र॒ स्तत् तन् मा॑तरो मातर॒ स्तत् । \newline
47. तद् वो॑ व॒ स्तत् तद् वः॑ । \newline
48. वो॒ दे॒वी॒र् दे॒वी॒र् वो॒ वो॒ दे॒वीः॒ । \newline
49. दे॒वी॒ रुपोप॑ देवीर् देवी॒ रुप॑ । \newline
50. उप॑ ब्रुवे ब्रुव॒ उपोप॑ ब्रुवे । \newline
51. ब्रु॒व॒ इति॑ ब्रुवे । \newline
52. रपाꣳ॑सि विघ्न॒तीर् वि॑घ्न॒ती रपाꣳ॑सि॒ रपाꣳ॑सि विघ्न॒तीः । \newline
53. वि॒घ्न॒ती रि॑तेत विघ्न॒तीर् वि॑घ्न॒ती रि॑त । \newline
54. वि॒घ्न॒तीरिति॑ वि - घ्न॒तीः । \newline
55. इ॒त॒ रपो॒ रप॑ इतेत॒ रपः॑ । \newline
56. रप॑ श्चा॒तय॑माना श्चा॒तय॑माना॒ रपो॒ रप॑ श्चा॒तय॑मानाः । \newline

\textbf{Ghana Paata } \newline

1. या जा॒ता जा॒ता या या जा॒ता ओष॑धय॒ ओष॑धयो जा॒ता या या जा॒ता ओष॑धयः । \newline
2. जा॒ता ओष॑धय॒ ओष॑धयो जा॒ता जा॒ता ओष॑धयो दे॒वेभ्यो॑ दे॒वेभ्य॒ ओष॑धयो जा॒ता जा॒ता ओष॑धयो दे॒वेभ्यः॑ । \newline
3. ओष॑धयो दे॒वेभ्यो॑ दे॒वेभ्य॒ ओष॑धय॒ ओष॑धयो दे॒वेभ्य॑ स्त्रियु॒गम् त्रि॑यु॒गम् दे॒वेभ्य॒ ओष॑धय॒ ओष॑धयो दे॒वेभ्य॑ स्त्रियु॒गम् । \newline
4. दे॒वेभ्य॑ स्त्रियु॒गम् त्रि॑यु॒गम् दे॒वेभ्यो॑ दे॒वेभ्य॑ स्त्रियु॒गम् पु॒रा पु॒रा त्रि॑यु॒गम् दे॒वेभ्यो॑ दे॒वेभ्य॑ स्त्रियु॒गम् पु॒रा । \newline
5. त्रि॒यु॒गम् पु॒रा पु॒रा त्रि॑यु॒गम् त्रि॑यु॒गम् पु॒रा । \newline
6. त्रि॒यु॒गमिति॑ त्रि - यु॒गम् । \newline
7. पु॒रेति॑ पु॒रा । \newline
8. मन्दा॑मि ब॒भ्रूणा᳚म् ब॒भ्रूणा॒म् मन्दा॑मि॒ मन्दा॑मि ब॒भ्रूणा॑ म॒ह म॒हम् ब॒भ्रूणा॒म् मन्दा॑मि॒ मन्दा॑मि ब॒भ्रूणा॑ म॒हम् । \newline
9. ब॒भ्रूणा॑ म॒ह म॒हम् ब॒भ्रूणा᳚म् ब॒भ्रूणा॑ म॒हꣳ श॒तꣳ श॒त म॒हम् ब॒भ्रूणा᳚म् ब॒भ्रूणा॑ म॒हꣳ श॒तम् । \newline
10. अ॒हꣳ श॒तꣳ श॒त म॒ह म॒हꣳ श॒तम् धामा॑नि॒ धामा॑नि श॒त म॒ह म॒हꣳ श॒तम् धामा॑नि । \newline
11. श॒तम् धामा॑नि॒ धामा॑नि श॒तꣳ श॒तम् धामा॑नि स॒प्त स॒प्त धामा॑नि श॒तꣳ श॒तम् धामा॑नि स॒प्त । \newline
12. धामा॑नि स॒प्त स॒प्त धामा॑नि॒ धामा॑नि स॒प्त च॑ च स॒प्त धामा॑नि॒ धामा॑नि स॒प्त च॑ । \newline
13. स॒प्त च॑ च स॒प्त स॒प्त च॑ । \newline
14. चेति॑ च । \newline
15. श॒तं ॅवो॑ वः श॒तꣳ श॒तं ॅवो॑ अंबांब वः श॒तꣳ श॒तं ॅवो॑ अंब । \newline
16. वो॒ अं॒बां॒ब॒ वो॒ वो॒ अं॒ब॒ धामा॑नि॒ धामा᳚ न्यंब वो वो अंब॒ धामा॑नि । \newline
17. अं॒ब॒ धामा॑नि॒ धामा᳚ न्यंबांब॒ धामा॑नि स॒हस्रꣳ॑ स॒हस्र॒म् धामा᳚ न्यंबांब॒ धामा॑नि स॒हस्र᳚म् । \newline
18. धामा॑नि स॒हस्रꣳ॑ स॒हस्र॒म् धामा॑नि॒ धामा॑नि स॒हस्र॑ मु॒तोत स॒हस्र॒म् धामा॑नि॒ धामा॑नि स॒हस्र॑ मु॒त । \newline
19. स॒हस्र॑ मु॒तोत स॒हस्रꣳ॑ स॒हस्र॑ मु॒त वो॑ व उ॒त स॒हस्रꣳ॑ स॒हस्र॑ मु॒त वः॑ । \newline
20. उ॒त वो॑ व उ॒तोत वो॒ रुहो॒ रुहो॑ व उ॒तोत वो॒ रुहः॑ । \newline
21. वो॒ रुहो॒ रुहो॑ वो वो॒ रुहः॑ । \newline
22. रुह॒ इति॒ रुहः॑ । \newline
23. अथा॑ शतक्रत्वः शतक्रत्वो॒ अथाथा॑ शतक्रत्वो यू॒यं ॅयू॒यꣳ श॑तक्रत्वो॒ अथाथा॑ शतक्रत्वो यू॒यम् । \newline
24. श॒त॒क्र॒त्वो॒ यू॒यं ॅयू॒यꣳ श॑तक्रत्वः शतक्रत्वो यू॒य मि॒म मि॒मं ॅयू॒यꣳ श॑तक्रत्वः श॑तक्रत्वो यू॒य मि॒मम् । \newline
25. श॒त॒क्र॒त्व॒ इति॑ शत - क्र॒त्वः॒ । \newline
26. यू॒य मि॒म मि॒मं ॅयू॒यं ॅयू॒य मि॒मम् मे॑ म इ॒मं ॅयू॒यं ॅयू॒य मि॒मम् मे᳚ । \newline
27. इ॒मम् मे॑ म इ॒म मि॒मम् मे॑ अग॒द म॑ग॒दम् म॑ इ॒म मि॒मम् मे॑ अग॒दम् । \newline
28. मे॒ अ॒ग॒द म॑ग॒दम् मे॑ मे अग॒दम् कृ॑त कृता ग॒दम् मे॑ मे अग॒दम् कृ॑त । \newline
29. अ॒ग॒दम् कृ॑त कृता ग॒द म॑ग॒दम् कृ॑त । \newline
30. कृ॒तेति॑ कृत । \newline
31. पुष्पा॑वतीः प्र॒सूव॑तीः प्र॒सूव॑तीः॒ पुष्पा॑वतीः॒ पुष्पा॑वतीः प्र॒सूव॑तीः फ॒लिनीः᳚ फ॒लिनीः᳚ प्र॒सूव॑तीः॒ पुष्पा॑वतीः॒ पुष्पा॑वतीः प्र॒सूव॑तीः फ॒लिनीः᳚ । \newline
32. पुष्पा॑वती॒रिति॒ पुष्प॑ - व॒तीः॒ । \newline
33. प्र॒सूव॑तीः फ॒लिनीः᳚ फ॒लिनीः᳚ प्र॒सूव॑तीः प्र॒सूव॑तीः फ॒लिनी॑ रफ॒ला अ॑फ॒लाः फ॒लिनीः᳚ प्र॒सूव॑तीः प्र॒सूव॑तीः फ॒लिनी॑ रफ॒लाः । \newline
34. प्र॒सूव॑ती॒रिति॑ प्र - सूव॑तीः । \newline
35. फ॒लिनी॑ रफ॒ला अ॑फ॒लाः फ॒लिनीः᳚ फ॒लिनी॑ रफ॒ला उ॒तोताफ॒लाः फ॒लिनीः᳚ फ॒लिनी॑ रफ॒ला उ॒त । \newline
36. अ॒फ॒ला उ॒तोताफ॒ला अ॑फ॒ला उ॒त । \newline
37. उ॒तेत्यु॒त । \newline
38. अश्वा॑ इवे॒ वाश्वा॒ अश्वा॑ इव स॒जित्व॑रीः स॒जित्व॑री रि॒वाश्वा॒ अश्वा॑ इव स॒जित्व॑रीः । \newline
39. इ॒व॒ स॒जित्व॑रीः स॒जित्व॑री रिवेव स॒जित्व॑रीर् वी॒रुधो॑ वी॒रुधः॑ स॒जित्व॑री रिवेव स॒जित्व॑रीर् वी॒रुधः॑ । \newline
40. स॒जित्व॑रीर् वी॒रुधो॑ वी॒रुधः॑ स॒जित्व॑रीः स॒जित्व॑रीर् वी॒रुधः॑ पारयि॒ष्णवः॑ पारयि॒ष्णवो॑ वी॒रुधः॑ स॒जित्व॑रीः स॒जित्व॑रीर् वी॒रुधः॑ पारयि॒ष्णवः॑ । \newline
41. स॒जित्व॑री॒रिति॑ स - जित्व॑रीः । \newline
42. वी॒रुधः॑ पारयि॒ष्णवः॑ पारयि॒ष्णवो॑ वी॒रुधो॑ वी॒रुधः॑ पारयि॒ष्णवः॑ । \newline
43. पा॒र॒यि॒ष्णव॒ इति॑ पारयि॒ष्णवः॑ । \newline
44. ओष॑धी॒ रितीत्योष॑धी॒ रोष॑धी॒ रिति॑ मातरो मातर॒ इत्योष॑धी ॒रोष॑धी॒ रिति॑ मातरः । \newline
45. इति॑ मातरो मातर॒ इतीति॑ मातर॒ स्तत् तन् मा॑तर॒ इतीति॑ मातर॒ स्तत् । \newline
46. मा॒त॒र॒ स्तत् तन् मा॑तरो मातर॒ स्तद् वो॑ व॒स्तन् मा॑तरो मातर॒ स्तद् वः॑ । \newline
47. तद् वो॑ व॒ स्तत् तद् वो॑ देवीर् देवीर् व॒ स्तत् तद् वो॑ देवीः । \newline
48. वो॒ दे॒वी॒र् दे॒वी॒र् वो॒ वो॒ दे॒वी॒ रुपोप॑ देवीर् वो वो देवी॒ रुप॑ । \newline
49. दे॒वी॒ रुपोप॑ देवीर् देवी॒ रुप॑ ब्रुवे ब्रुव॒ उप॑ देवीर् देवी॒ रुप॑ ब्रुवे । \newline
50. उप॑ ब्रुवे ब्रुव॒ उपोप॑ ब्रुवे । \newline
51. ब्रु॒व॒ इति॑ ब्रुवे । \newline
52. रपाꣳ॑सि विघ्न॒तीर् वि॑घ्न॒ती रपाꣳ॑सि॒ रपाꣳ॑सि विघ्न॒ती रि॑तेत विघ्न॒ती रपाꣳ॑सि॒ रपाꣳ॑सि विघ्न॒ती रि॑त । \newline
53. वि॒घ्न॒ती रि॑तेत विघ्न॒तीर् वि॑घ्न॒ती रि॑त॒ रपो॒ रप॑ इत विघ्न॒तीर् वि॑घ्न॒ती रि॑त॒ रपः॑ । \newline
54. वि॒घ्न॒तीरिति॑ वि - घ्न॒तीः । \newline
55. इ॒त॒ रपो॒ रप॑ इतेत॒ रप॑ श्चा॒तय॑माना श्चा॒तय॑माना॒ रप॑ इतेत॒ रप॑ श्चा॒तय॑मानाः । \newline
56. रप॑ श्चा॒तय॑माना श्चा॒तय॑माना॒ रपो॒ रप॑ श्चा॒तय॑मानाः । \newline
\pagebreak
\markright{ TS 4.2.6.2  \hfill https://www.vedavms.in \hfill}

\section{ TS 4.2.6.2 }

\textbf{TS 4.2.6.2 } \newline
\textbf{Samhita Paata} \newline

-श्चा॒तय॑मानाः ॥ अ॒श्व॒त्थे वो॑ नि॒षद॑नं प॒र्णे वो॑ वस॒तिः कृ॒ता । गो॒भाज॒ इत् किला॑सथ॒ यथ् स॒नव॑थ॒ पूरु॑षं ॥ यद॒हं ॅवा॒जय॑-न्नि॒मा ओष॑धी॒र्॒.हस्त॑ आद॒धे । आ॒त्मा यक्ष्म॑स्य नश्यति पु॒रा जी॑व॒गृभो॑ यथा ॥ यदोष॑धयः सं॒गच्छ॑न्ते॒ राजा॑नः॒ समि॑ता विव । विप्रः॒ स उ॑च्यते भि॒षग्र॑क्षो॒हा ऽमी॑व॒ चात॑नः ॥ निष्कृ॑ति॒-र्नाम॑वो मा॒ताऽथा॑ यू॒यꣳस्थ॒ संकृ॑तीः । स॒राः प॑त॒त्रिणीः᳚ - [  ] \newline

\textbf{Pada Paata} \newline

चा॒तय॑मानाः ॥ अ॒श्व॒त्थे । वः॒ । नि॒षद॑न॒मिति॑ नि - सद॑नम् । प॒र्णे । वः॒ । व॒स॒तिः । कृ॒ता ॥ गो॒भाज॒ इति॑ गो - भाजः॑ । इत् । किल॑ । अ॒स॒थ॒ । यत् । स॒नव॑थ । पूरु॑षम् ॥ यत् । अ॒हम् । वा॒जयन्न्॑ । इ॒माः । ओष॑धीः । हस्ते᳚ । आ॒द॒ध इत्या᳚-द॒धे ॥ आ॒त्मा । यक्ष्म॑स्य । न॒श्य॒ति॒ । पु॒रा । जी॒व॒गृभ॒ इति॑ जीव - गृभः॑ । य॒था॒ ॥ यत् । ओष॑धयः । सं॒गच्छ॑न्त॒ इति॑ सं - गच्छ॑न्ते । राजा॑नः । समि॑ता॒विति॒ सं - इ॒तौ॒ । इ॒व॒ ॥ विप्रः॑ । सः । उ॒च्य॒ते॒ । भि॒षक् । र॒क्षो॒हेति॑ रक्षः - हा । अ॒मी॒व॒चात॑न॒ इत्य॑मीव - चात॑नः ॥ निष्कृ॑ति॒रिति॒ निः-कृ॒तिः॒ । नाम॑ । वः॒ । मा॒ता । अथ॑ । यू॒यम् । स्थ॒ । संकृ॑ती॒रिति॒ सं - कृ॒तीः॒ ॥ स॒राः । प॒त॒त्रिणीः᳚ ।  \newline


\textbf{Krama Paata} \newline

चा॒तय॑माना॒ इति॑ चा॒तय॑मानाः ॥ अ॒श्व॒त्थे वः॑ । वो॒ नि॒षद॑नम् । नि॒षद॑नम् प॒र्णे । नि॒षद॑न॒मिति॑ नि - सद॑नम् । प॒र्णे वः॑ । वो॒ व॒स॒तिः । व॒स॒तिः कृ॒ता । कृ॒तेति॑ कृ॒ता ॥ गो॒भाज॒ इत् । गो॒भाज॒ इति॑ गो - भाजः॑ । इत् किल॑ । किला॑सथ । अ॒स॒थ॒ यत् । यथ् स॒नव॑थ । स॒नव॑थ॒ पूरु॑षम् । पूरु॑ष॒मिति॒ पूरु॑षम् ॥ यद॒हम् । अ॒हं ॅवा॒जयन्न्॑ । वा॒जय॑न्नि॒माः । इ॒मा ओष॑धीः । ओष॑धी॒र्.॒ हस्ते᳚ । हस्त॑ आद॒धे । आ॒द॒ध इत्या᳚ - द॒धे ॥ आ॒त्मा यक्ष्म॑स्य । यक्ष्म॑स्य नश्यति । न॒श्य॒ति॒ पु॒रा । पु॒रा जी॑व॒गृभः॑ । जी॒व॒गृभो॑ यथा । जी॒व॒गृभ॒ इति॑ जीव - गृभः॑ । य॒थेति॑ यथा ॥ यदोष॑धयः । ओष॑धयः स॒ङ्गच्छ॑न्ते । स॒ङ्गच्छ॑न्ते॒ राजा॑नः । स॒ङ्गच्छ॑न्त॒ इति॑ सम् - गच्छ॑न्ते । राजा॑नः॒ समि॑तौ । समि॑ताविव । समि॑ता॒विति॒ सम् - इ॒तौ॒ । इ॒वेती॑व ॥ विप्रः॒ सः । स उ॑च्यते । उ॒च्य॒ते॒ भि॒षक् । भि॒षग् र॑क्षो॒हा । र॒क्षो॒हाऽमी॑व॒चात॑नः । र॒क्षो॒हेति॑ रक्षः - हा । अ॒मी॒व॒चात॑न॒ इत्य॑मीव - चात॑नः ॥ निष्कृ॑ति॒र् नाम॑ । निष्कृ॑ति॒रिति॒ निः - कृ॒तिः॒ । नाम॑ वः । वो॒ मा॒ता । मा॒ताऽथ॑ । अथा॑ यू॒यम् । यू॒यꣳ स्थ॑ । स्थ॒ सङ्कृ॑तीः । सङ्कृ॑ती॒रिति॒ सम् - कृ॒तीः॒ ॥ स॒राः प॑त॒त्रिणीः᳚ । प॒त॒त्रिणीः᳚ स्थन \newline

\textbf{Jatai Paata} \newline

1. चा॒तय॑माना॒ इति॑ चा॒तय॑मानाः । \newline
2. अ॒श्व॒त्थे वो॑ वो अश्व॒त्थे अ॑श्व॒त्थे वः॑ । \newline
3. वो॒ नि॒षद॑नम् नि॒षद॑नं ॅवो वो नि॒षद॑नम् । \newline
4. नि॒षद॑नम् प॒र्णे प॒र्णे नि॒षद॑नम् नि॒षद॑नम् प॒र्णे । \newline
5. नि॒षद॑न॒मिति॑ नि - सद॑नम् । \newline
6. प॒र्णे वो॑ वः प॒र्णे प॒र्णे वः॑ । \newline
7. वो॒ व॒स॒तिर् व॑स॒तिर् वो॑ वो वस॒तिः । \newline
8. व॒स॒तिः कृ॒ता कृ॒ता व॑स॒तिर् व॑स॒तिः कृ॒ता । \newline
9. कृ॒तेति॑ कृ॒ता । \newline
10. गो॒भाज॒ इदिद् गो॒भाजो॑ गो॒भाज॒ इत् । \newline
11. गो॒भाज॒ इति॑ गो - भाजः॑ । \newline
12. इत् किल॒ किले दित् किल॑ । \newline
13. किला॑ सथा सथ॒ किल॒ किला॑ सथ । \newline
14. अ॒स॒थ॒ यद् यद॑सथा सथ॒ यत् । \newline
15. यथ् स॒नव॑थ स॒नव॑थ॒ यद् यथ् स॒नव॑थ । \newline
16. स॒नव॑थ॒ पूरु॑ष॒म् पूरु॑षꣳ स॒नव॑थ स॒नव॑थ॒ पूरु॑षम् । \newline
17. पूरु॑ष॒मिति॒ पूरु॑षम् । \newline
18. यद॒ह म॒हं ॅयद् यद॒हम् । \newline
19. अ॒हं ॅवा॒जय॑न्. वा॒जय॑न् न॒ह म॒हं ॅवा॒जयन्न्॑ । \newline
20. वा॒जय॑न् नि॒मा इ॒मा वा॒जय॑न्. वा॒जय॑न् नि॒माः । \newline
21. इ॒मा ओष॑धी॒ रोष॑धी रि॒मा इ॒मा ओष॑धीः । \newline
22. ओष॑धी॒र्॒. हस्ते॒ हस्त॒ ओष॑धी॒ रोष॑धी॒र्॒. हस्ते᳚ । \newline
23. हस्त॑ आद॒ध आ॑द॒धे हस्ते॒ हस्त॑ आद॒धे । \newline
24. आ॒द॒ध इत्या᳚ - द॒धे । \newline
25. आ॒त्मा यक्ष्म॑स्य॒ यक्ष्म॑ स्या॒त्मा ऽऽत्मा यक्ष्म॑स्य । \newline
26. यक्ष्म॑स्य नश्यति नश्यति॒ यक्ष्म॑स्य॒ यक्ष्म॑स्य नश्यति । \newline
27. न॒श्य॒ति॒ पु॒रा पु॒रा न॑श्यति नश्यति पु॒रा । \newline
28. पु॒रा जी॑व॒गृभो॑ जीव॒गृभः॑ पु॒रा पु॒रा जी॑व॒गृभः॑ । \newline
29. जी॒व॒गृभो॑ यथा यथा जीव॒गृभो॑ जीव॒गृभो॑ यथा । \newline
30. जी॒व॒गृभ॒ इति॑ जीव - गृभः॑ । \newline
31. य॒थेति॑ यथा । \newline
32. यदोष॑धय॒ ओष॑धयो॒ यद् यदोष॑धयः । \newline
33. ओष॑धयः स॒ङ्गच्छ॑न्ते स॒ङ्गच्छ॑न्त॒ ओष॑धय॒ ओष॑धयः स॒ङ्गच्छ॑न्ते । \newline
34. स॒ङ्गच्छ॑न्ते॒ राजा॑नो॒ राजा॑नः स॒ङ्गच्छ॑न्ते स॒ङ्गच्छ॑न्ते॒ राजा॑नः । \newline
35. स॒ङ्गच्छ॑न्त॒ इति॑ सं - गच्छ॑न्ते । \newline
36. राजा॑नः॒ समि॑तौ॒ समि॑तौ॒ राजा॑नो॒ राजा॑नः॒ समि॑तौ । \newline
37. समि॑ता विवेव॒ समि॑तौ॒ समि॑ता विव । \newline
38. समि॑ता॒विति॒ सं - इ॒तौ॒ । \newline
39. इ॒वेती॑व । \newline
40. विप्रः॒ स स विप्रो॒ विप्रः॒ सः । \newline
41. स उ॑च्यत उच्यते॒ स स उ॑च्यते । \newline
42. उ॒च्य॒ते॒ भि॒षग् भि॒ष गु॑च्यत उच्यते भि॒षक् । \newline
43. भि॒षग् र॑क्षो॒हा र॑क्षो॒हा भि॒षग् भि॒षग् र॑क्षो॒हा । \newline
44. र॒क्षो॒हा ऽमी॑व॒चात॑नो अमीव॒चात॑नो रक्षो॒हा र॑क्षो॒हा ऽमी॑व॒चात॑नः । \newline
45. र॒क्षो॒हेति॑ रक्षः - हा । \newline
46. अ॒मी॒व॒चात॑न॒ इत्य॑मीव - चात॑नः । \newline
47. निष्कृ॑ति॒र् नाम॒ नाम॒ निष्कृ॑ति॒र् निष्कृ॑ति॒र् नाम॑ । \newline
48. निष्कृ॑ति॒रिति॒ निः - कृ॒तिः॒ । \newline
49. नाम॑ वो वो॒ नाम॒ नाम॑ वः । \newline
50. वो॒ मा॒ता मा॒ता वो॑ वो मा॒ता । \newline
51. मा॒ता ऽथाथ॑ मा॒ता मा॒ता ऽथ॑ । \newline
52. अथा॑ यू॒यं ॅयू॒य मथाथा॑ यू॒यम् । \newline
53. यू॒यꣳ स्थ॑ स्थ यू॒यं ॅयू॒यꣳ स्थ॑ । \newline
54. स्थ॒ सङ्कृ॑तीः॒ सङ्कृ॑तीः स्थ स्थ॒ सङ्कृ॑तीः । \newline
55. सङ्कृ॑ती॒रिति॒ सं - कृ॒तीः॒ । \newline
56. स॒राः प॑त॒त्रिणीः᳚ पत॒त्रिणीः᳚ स॒राः स॒राः प॑त॒त्रिणीः᳚ । \newline
57. प॒त॒त्रिणीः᳚ स्थन स्थन पत॒त्रिणीः᳚ पत॒त्रिणीः᳚ स्थन । \newline

\textbf{Ghana Paata } \newline

1. चा॒तय॑माना॒ इति॑ चा॒तय॑मानाः । \newline
2. अ॒श्व॒त्थे वो॑ वो अश्व॒त्थे अ॑श्व॒त्थे वो॑ नि॒षद॑नम् नि॒षद॑नं ॅवो अश्व॒त्थे अ॑श्व॒त्थे वो॑ नि॒षद॑नम् । \newline
3. वो॒ नि॒षद॑नम् नि॒षद॑नं ॅवो वो नि॒षद॑नम् प॒र्णे प॒र्णे नि॒षद॑नं ॅवो वो नि॒षद॑नम् प॒र्णे । \newline
4. नि॒षद॑नम् प॒र्णे प॒र्णे नि॒षद॑नम् नि॒षद॑नम् प॒र्णे वो॑ वः प॒र्णे नि॒षद॑नम् नि॒षद॑नम् प॒र्णे वः॑ । \newline
5. नि॒षद॑न॒मिति॑ नि - सद॑नम् । \newline
6. प॒र्णे वो॑ वः प॒र्णे प॒र्णे वो॑ वस॒तिर् व॑स॒तिर् वः॑ प॒र्णे प॒र्णे वो॑ वस॒तिः । \newline
7. वो॒ व॒स॒तिर् व॑स॒तिर् वो॑ वो वस॒तिः कृ॒ता कृ॒ता व॑स॒तिर् वो॑ वो वस॒तिः कृ॒ता । \newline
8. व॒स॒तिः कृ॒ता कृ॒ता व॑स॒तिर् व॑स॒तिः कृ॒ता । \newline
9. कृ॒तेति॑ कृ॒ता । \newline
10. गो॒भाज॒ इदिद् गो॒भाजो॑ गो॒भाज॒ इत् किल॒ किलेद् गो॒भाजो॑ गो॒भाज॒ इत् किल॑ । \newline
11. गो॒भाज॒ इति॑ गो - भाजः॑ । \newline
12. इत् किल॒ किले दित् किला॑ सथा सथ॒ किले दित् किला॑ सथ । \newline
13. किला॑ सथा सथ॒ किल॒ किला॑सथ॒ यद् यद॑सथ॒ किल॒ किला॑सथ॒ यत् । \newline
14. अ॒स॒थ॒ यद् यद॑सथा सथ॒ यथ् स॒नव॑थ स॒नव॑थ॒ यद॑सथा सथ॒ यथ् स॒नव॑थ । \newline
15. यथ् स॒नव॑थ स॒नव॑थ॒ यद् यथ् स॒नव॑थ॒ पूरु॑ष॒म् पूरु॑षꣳ स॒नव॑थ॒ यद् यथ् स॒नव॑थ॒ पूरु॑षम् । \newline
16. स॒नव॑थ॒ पूरु॑ष॒म् पूरु॑षꣳ स॒नव॑थ स॒नव॑थ॒ पूरु॑षम् । \newline
17. पूरु॑ष॒मिति॒ पूरु॑षम् । \newline
18. यद॒ह म॒हं ॅयद् यद॒हं ॅवा॒जय॑न्. वा॒जय॑न् न॒हं ॅयद् यद॒हं ॅवा॒जयन्न्॑ । \newline
19. अ॒हं ॅवा॒जय॑न्. वा॒जय॑न् न॒ह म॒हं ॅवा॒जय॑न् नि॒मा इ॒मा वा॒जय॑न् न॒ह म॒हं ॅवा॒जय॑न् नि॒माः । \newline
20. वा॒जय॑न् नि॒मा इ॒मा वा॒जय॑न्. वा॒जय॑न् नि॒मा ओष॑धी॒ रोष॑धी रि॒मा वा॒जय॑न्. वा॒जय॑न् नि॒मा ओष॑धीः । \newline
21. इ॒मा ओष॑धी॒ रोष॑धी रि॒मा इ॒मा ओष॑धी॒र्॒. हस्ते॒ हस्त॒ ओष॑धी रि॒मा इ॒मा ओष॑धी॒र्॒. हस्ते᳚ । \newline
22. ओष॑धी॒र्॒. हस्ते॒ हस्त॒ ओष॑धी॒ रोष॑धी॒र्॒. हस्त॑ आद॒ध आ॑द॒धे हस्त॒ ओष॑धी॒ रोष॑धी॒र्॒. हस्त॑ आद॒धे । \newline
23. हस्त॑ आद॒ध आ॑द॒धे हस्ते॒ हस्त॑ आद॒धे । \newline
24. आ॒द॒ध इत्या᳚ - द॒धे । \newline
25. आ॒त्मा यक्ष्म॑स्य॒ यक्ष्म॑स्या॒त्मा ऽऽत्मा यक्ष्म॑स्य नश्यति नश्यति॒ यक्ष्म॑स्या॒त्मा ऽऽत्मा यक्ष्म॑स्य नश्यति । \newline
26. यक्ष्म॑स्य नश्यति नश्यति॒ यक्ष्म॑स्य॒ यक्ष्म॑स्य नश्यति पु॒रा पु॒रा न॑श्यति॒ यक्ष्म॑स्य॒ यक्ष्म॑स्य नश्यति पु॒रा । \newline
27. न॒श्य॒ति॒ पु॒रा पु॒रा न॑श्यति नश्यति पु॒रा जी॑व॒गृभो॑ जीव॒गृभः॑ पु॒रा न॑श्यति नश्यति पु॒रा जी॑व॒गृभः॑ । \newline
28. पु॒रा जी॑व॒गृभो॑ जीव॒गृभः॑ पु॒रा पु॒रा जी॑व॒गृभो॑ यथा यथा जीव॒गृभः॑ पु॒रा पु॒रा जी॑व॒गृभो॑ यथा । \newline
29. जी॒व॒गृभो॑ यथा यथा जीव॒गृभो॑ जीव॒गृभो॑ यथा । \newline
30. जी॒व॒गृभ॒ इति॑ जीव - गृभः॑ । \newline
31. य॒थेति॑ यथा । \newline
32. यदोष॑धय॒ ओष॑धयो॒ यद् यदोष॑धयः स॒ङ्गच्छ॑न्ते स॒ङ्गच्छ॑न्त॒ ओष॑धयो॒ यद् यदोष॑धयः स॒ङ्गच्छ॑न्ते । \newline
33. ओष॑धयः स॒ङ्गच्छ॑न्ते स॒ङ्गच्छ॑न्त॒ ओष॑धय॒ ओष॑धयः स॒ङ्गच्छ॑न्ते॒ राजा॑नो॒ राजा॑नः स॒ङ्गच्छ॑न्त॒ ओष॑धय॒ ओष॑धयः स॒ङ्गच्छ॑न्ते॒ राजा॑नः । \newline
34. स॒ङ्गच्छ॑न्ते॒ राजा॑नो॒ राजा॑नः स॒ङ्गच्छ॑न्ते स॒ङ्गच्छ॑न्ते॒ राजा॑नः॒ समि॑तौ॒ समि॑तौ॒ राजा॑नः स॒ङ्गच्छ॑न्ते स॒ङ्गच्छ॑न्ते॒ राजा॑नः॒ समि॑तौ । \newline
35. स॒ङ्गच्छ॑न्त॒ इति॑ सं - गच्छ॑न्ते । \newline
36. राजा॑नः॒ समि॑तौ॒ समि॑तौ॒ राजा॑नो॒ राजा॑नः॒ समि॑ता विवेव॒ समि॑तौ॒ राजा॑नो॒ राजा॑नः॒ समि॑ता विव । \newline
37. समि॑ता विवेव॒ समि॑तौ॒ समि॑ता विव । \newline
38. समि॑ता॒विति॒ सं - इ॒तौ॒ । \newline
39. इ॒वेती॑व । \newline
40. विप्रः॒ स स विप्रो॒ विप्रः॒ स उ॑च्यत उच्यते॒ स विप्रो॒ विप्रः॒ स उ॑च्यते । \newline
41. स उ॑च्यत उच्यते॒ स स उ॑च्यते भि॒षग् भि॒ष गु॑च्यते॒ स स उ॑च्यते भि॒षक् । \newline
42. उ॒च्य॒ते॒ भि॒षग् भि॒ष गु॑च्यत उच्यते भि॒षग् र॑क्षो॒हा र॑क्षो॒हा भि॒ष गु॑च्यत उच्यते भि॒षग् र॑क्षो॒हा । \newline
43. भि॒षग् र॑क्षो॒हा र॑क्षो॒हा भि॒षग् भि॒षग् र॑क्षो॒हा ऽमी॑व॒चात॑नो अमीव॒चात॑नो रक्षो॒हा भि॒षग् भि॒षग् र॑क्षो॒हा ऽमी॑व॒चात॑नः । \newline
44. र॒क्षो॒हा ऽमी॑व॒चात॑नो अमीव॒चात॑नो रक्षो॒हा र॑क्षो॒हा ऽमी॑व॒चात॑नः । \newline
45. र॒क्षो॒हेति॑ रक्षः - हा । \newline
46. अ॒मी॒व॒चात॑न॒ इत्य॑मीव - चात॑नः । \newline
47. निष्कृ॑ति॒र् नाम॒ नाम॒ निष्कृ॑ति॒र् निष्कृ॑ति॒र् नाम॑ वो वो॒ नाम॒ निष्कृ॑ति॒र् निष्कृ॑ति॒र् नाम॑ वः । \newline
48. निष्कृ॑ति॒रिति॒ निः - कृ॒तिः॒ । \newline
49. नाम॑ वो वो॒ नाम॒ नाम॑ वो मा॒ता मा॒ता वो॒ नाम॒ नाम॑ वो मा॒ता । \newline
50. वो॒ मा॒ता मा॒ता वो॑ वो मा॒ता ऽथाथ॑ मा॒ता वो॑ वो मा॒ता ऽथ॑ । \newline
51. मा॒ता ऽथाथ॑ मा॒ता मा॒ता ऽथा॑ यू॒यं ॅयू॒य मथ॑ मा॒ता मा॒ता ऽथा॑ यू॒यम् । \newline
52. अथा॑ यू॒यं ॅयू॒य मथाथा॑ यू॒यꣳ स्थ॑ स्थ यू॒य मथाथा॑ यू॒यꣳ स्थ॑ । \newline
53. यू॒यꣳ स्थ॑ स्थ यू॒यं ॅयू॒यꣳ स्थ॒ सङ्कृ॑तीः॒ सङ्कृ॑तीः स्थ यू॒यं ॅयू॒यꣳ स्थ॒ सङ्कृ॑तीः । \newline
54. स्थ॒ सङ्कृ॑तीः॒ सङ्कृ॑तीः स्थ स्थ॒ सङ्कृ॑तीः । \newline
55. सङ्कृ॑ती॒रिति॒ सं - कृ॒तीः॒ । \newline
56. स॒राः प॑त॒त्रिणीः᳚ पत॒त्रिणीः᳚ स॒राः स॒राः प॑त॒त्रिणीः᳚ स्थन स्थन पत॒त्रिणीः᳚ स॒राः स॒राः प॑त॒त्रिणीः᳚ स्थन । \newline
57. प॒त॒त्रिणीः᳚ स्थन स्थन पत॒त्रिणीः᳚ पत॒त्रिणीः᳚ स्थन॒ यद् यथ् स्थ॑न पत॒त्रिणीः᳚ पत॒त्रिणीः᳚ स्थन॒ यत् । \newline
\pagebreak
\markright{ TS 4.2.6.3  \hfill https://www.vedavms.in \hfill}

\section{ TS 4.2.6.3 }

\textbf{TS 4.2.6.3 } \newline
\textbf{Samhita Paata} \newline

स्थन॒ यदा॒ मय॑ति॒ निष्कृ॑त ॥अ॒न्या वो॑ अ॒न्याम॑व-त्व॒न्याऽन्यस्या॒ उपा॑वत । ताः सर्वा॒ ओष॑धयः संॅविदा॒ना इ॒दं मे॒ प्राव॑ता॒ वचः॑ ॥ उच्छुष्मा॒ ओष॑धीनां॒ गावो॑ गो॒ष्ठा दि॑वेरते । धनꣳ॑ सनि॒ष्यन्ती॑ नामा॒त्मानं॒ तव॑ पूरुष ॥ अति॒ विश्वाः᳚ परि॒ष्ठास्ते॒न इ॑व व्र॒जम॑क्रमुः । ओष॑धयः॒ प्राचु॑च्यवु॒ र्यत् किं च॑ त॒नुवाꣳ॒॒ रपः॑ ॥ या - [  ] \newline

\textbf{Pada Paata} \newline

स्थ॒न॒ । यत् । आ॒मय॑ति । निरिति॑ । कृ॒त॒ ॥ अ॒न्या । वः॒ । अ॒न्याम् । अ॒व॒तु॒ । अ॒न्या । अ॒न्यस्याः᳚ । उपेति॑ । अ॒व॒त॒ ॥ ताः । सर्वाः᳚ । ओष॑धयः । सं॒ॅवि॒दा॒ना इति॑ सं - वि॒दा॒नाः । इ॒दम् । मे॒ । प्रेति॑ । अ॒व॒त॒ । वचः॑ ॥ उदिति॑ । शुष्माः᳚ । ओष॑धीनां । गावः॑ । गो॒ष्ठादिति॑ गो - स्थात् । इ॒व॒ । ई॒र॒ते॒ ॥ धन᳚म् । स॒नि॒ष्यन्ती॑नाम् । आ॒त्मान᳚म् । तव॑ । पू॒रु॒ष॒ ॥ अतीति॑ । विश्वाः᳚ । प॒रि॒ष्ठा इति॑ परि - स्थाः । स्ते॒नः । इ॒व॒ । व्र॒जम् । अ॒क्र॒मुः॒ ॥ ओष॑धयः । प्रेति॑ । अ॒चु॒च्य॒वुः॒ । यत् । किम् । च॒ । त॒नुवा᳚म् । रपः॑ ॥ याः ।  \newline


\textbf{Krama Paata} \newline

स्थ॒न॒ यत् । यदा॒मय॑ति । आ॒मय॑ति॒ निः । निष्कृ॑त । कृ॒तेति॑ कृत ॥ अ॒न्या वः॑ । वो॒ अ॒न्याम् । अ॒न्याम॑वतु । अ॒व॒त्व॒न्या । अ॒न्याऽन्यस्याः᳚ । अ॒न्यस्या॒ उप॑ । उपा॑वत । अ॒व॒तेत्य॑वत ॥ ताः सर्वाः᳚ । सर्वा॒ ओष॑धयः । ओष॑धयः सम्ॅविदा॒नाः । स॒म्ॅवि॒दा॒ना इ॒दम् । स॒म्ॅवि॒दा॒ना इति॑ सम् - वि॒दा॒नाः । इ॒दम् मे᳚ । मे॒ प्र । प्राव॑त । अ॒व॒ता॒ वचः॑ । वच॒ इति॒ वचः॑ ॥ उच्छुष्माः᳚ । शुष्मा॒ ओष॑धीनाम् । ओष॑धीना॒म् गावः॑ । गावो॑ गो॒ष्ठात् । गो॒ष्ठादि॑व । गो॒ष्ठादिति॑ गो - स्थात् । इ॒वे॒र॒ते॒ । ई॒र॒त॒ इती॑रते ॥ धनꣳ॑ सनि॒ष्यन्ती॑नाम् । स॒नि॒ष्यन्ती॑नामा॒त्मान᳚म् । आ॒त्मान॒म् तव॑ । तव॑ पूरुष । पू॒रु॒षेति॑ पूरुष ॥ अति॒ विश्वाः᳚ । विश्वाः᳚ परि॒ष्ठाः । प॒रि॒ष्ठाः स्ते॒नः । प॒रि॒ष्ठा इति॑ परि - स्थाः । स्ते॒न इ॑व । इ॒व॒ व्र॒जम् । व्र॒जम॑क्रमुः । अ॒क्र॒मु॒रित्य॑क्रमुः ॥ ओष॑धयः॒ प्र । प्राचु॑च्यवुः । अ॒चु॒च्य॒वु॒र् यत् । यत् किम् । किञ्च॑ । च॒ त॒नुवा᳚म् । त॒नुवाꣳ॒॒ रपः॑ । रप॒ इति॒ रपः॑ ॥ यास्ते᳚ \newline

\textbf{Jatai Paata} \newline

1. स्थ॒न॒ यद् यथ् स्थ॑न स्थन॒ यत् । \newline
2. यदा॒मय॑ त्या॒मय॑ति॒ यद् यदा॒मय॑ति । \newline
3. आ॒मय॑ति॒ निर् णिरा॒मय॑ त्या॒मय॑ति॒ निः । \newline
4. निष् कृ॑त कृत॒ निर् णिष् कृ॑त । \newline
5. कृ॒तेति॑ कृत । \newline
6. अ॒न्या वो॑ वो अ॒न्या ऽन्या वः॑ । \newline
7. वो॒ अ॒न्या म॒न्यां ॅवो॑ वो अ॒न्याम् । \newline
8. अ॒न्या म॑व त्वव त्व॒न्या म॒न्या म॑वतु । \newline
9. अ॒व॒ त्व॒न्या ऽन्या ऽव॑ त्वव त्व॒न्या । \newline
10. अ॒न्या ऽन्यस्या॑ अ॒न्यस्या॑ अ॒न्या ऽन्या ऽन्यस्याः᳚ । \newline
11. अ॒न्यस्या॒ उपो पा॒न्यस्या॑ अ॒न्यस्या॒ उप॑ । \newline
12. उपा॑वता व॒तो पोपा॑वत । \newline
13. अ॒व॒तेत्य॑वत । \newline
14. ताः सर्वाः॒ सर्वा॒ स्ता स्ताः सर्वाः᳚ । \newline
15. सर्वा॒ ओष॑धय॒ ओष॑धयः॒ सर्वाः॒ सर्वा॒ ओष॑धयः । \newline
16. ओष॑धयः संॅविदा॒नाः सं॑ॅविदा॒ना ओष॑धय॒ ओष॑धयः संॅविदा॒नाः । \newline
17. सं॒ॅवि॒दा॒ना इ॒द मि॒दꣳ सं॑ॅविदा॒नाः सं॑ॅविदा॒ना इ॒दम् । \newline
18. सं॒ॅवि॒दा॒ना इति॑ सं - वि॒दा॒नाः । \newline
19. इ॒दम् मे॑ म इ॒द मि॒दम् मे᳚ । \newline
20. मे॒ प्र प्र मे॑ मे॒ प्र । \newline
21. प्राव॑ता वत॒ प्र प्राव॑त । \newline
22. अ॒व॒ता॒ वचो॒ वचो॑ ऽवता वता॒ वचः॑ । \newline
23. वच॒ इति॒ वचः॑ । \newline
24. उच्छुष्माः॒ शुष्मा॒ उदु च्छुष्माः᳚ । \newline
25. शुष्मा॒ ओष॑धीना॒ मोष॑धीनाꣳ॒॒ शुष्माः॒ शुष्मा॒ ओष॑धीनां । \newline
26. ओष॑धीनां॒ गावो॒ गाव॒ ओष॑धीना॒ मोष॑धीनां॒ गावः॑ । \newline
27. गावो॑ गो॒ष्ठाद् गो॒ष्ठाद् गावो॒ गावो॑ गो॒ष्ठात् । \newline
28. गो॒ष्ठा दि॑वेव गो॒ष्ठाद् गो॒ष्ठा दि॑व । \newline
29. गो॒ष्ठादिति॑ गो - स्थात् । \newline
30. इ॒वे॒ र॒त॒ ई॒र॒त॒ इ॒वे॒ वे॒र॒ते॒ । \newline
31. ई॒र॒त॒ इती॑रते । \newline
32. धनꣳ॑ सनि॒ष्यन्ती॑नाꣳ सनि॒ष्यन्ती॑ना॒म् धन॒म् धनꣳ॑ सनि॒ष्यन्ती॑नाम् । \newline
33. स॒नि॒ष्यन्ती॑ना मा॒त्मान॑ मा॒त्मानꣳ॑ सनि॒ष्यन्ती॑नाꣳ सनि॒ष्यन्ती॑ना मा॒त्मान᳚म् । \newline
34. आ॒त्मान॒म् तव॒ तवा॒त्मान॑ मा॒त्मान॒म् तव॑ । \newline
35. तव॑ पूरुष पूरुष॒ तव॒ तव॑ पूरुष । \newline
36. पू॒रु॒षेति॑ पूरुष । \newline
37. अति॒ विश्वा॒ विश्वा॒ अत्यति॒ विश्वाः᳚ । \newline
38. विश्वाः᳚ परि॒ष्ठाः प॑रि॒ष्ठा विश्वा॒ विश्वाः᳚ परि॒ष्ठाः । \newline
39. प॒रि॒ष्ठाः स्ते॒नः स्ते॒नः प॑रि॒ष्ठाः प॑रि॒ष्ठाः स्ते॒नः । \newline
40. प॒रि॒ष्ठा इति॑ परि - स्थाः । \newline
41. स्ते॒न इ॑वे व स्ते॒नः स्ते॒न इ॑व । \newline
42. इ॒व॒ व्र॒जं ॅव्र॒ज मि॑वे व व्र॒जम् । \newline
43. व्र॒ज म॑क्रमु रक्रमुर् व्र॒जं ॅव्र॒ज म॑क्रमुः । \newline
44. अ॒क्र॒मु॒ रित्य॑क्रमुः । \newline
45. ओष॑धयः॒ प्र प्रौष॑धय॒ ओष॑धयः॒ प्र । \newline
46. प्राचु॑च्यवु रचुच्यवुः॒ प्र प्राचु॑च्यवुः । \newline
47. अ॒चु॒च्य॒वु॒र् यद् यद॑चुच्यवु रचुच्यवु॒र् यत् । \newline
48. यत् किम् किं ॅयद् यत् किम् । \newline
49. किम् च॑ च॒ किम् किम् च॑ । \newline
50. च॒ त॒नुवा᳚म् त॒नुवा᳚म् च च त॒नुवा᳚म् । \newline
51. त॒नुवाꣳ॒॒ रपो॒ रप॑ स्त॒नुवा᳚म् त॒नुवाꣳ॒॒ रपः॑ । \newline
52. रप॒ इति॒ रपः॑ । \newline
53. या स्ते॑ ते॒ या या स्ते᳚ । \newline

\textbf{Ghana Paata } \newline

1. स्थ॒न॒ यद् यथ् स्थ॑न स्थन॒ यदा॒मय॑ त्या॒मय॑ति॒ यथ् स्थ॑न स्थन॒ यदा॒मय॑ति । \newline
2. यदा॒मय॑ त्या॒मय॑ति॒ यद् यदा॒मय॑ति॒ निर् णिरा॒मय॑ति॒ यद् यदा॒मय॑ति॒ निः । \newline
3. आ॒मय॑ति॒ निर् णिरा॒मय॑ त्या॒मय॑ति॒ निष् कृ॑त कृत॒ निरा॒मय॑ त्या॒मय॑ति॒ निष् कृ॑त । \newline
4. निष् कृ॑त कृत॒ निर् णिष् कृ॑त । \newline
5. कृ॒तेति॑ कृत । \newline
6. अ॒न्या वो॑ वो अ॒न्या ऽन्या वो॑ अ॒न्या म॒न्यां ॅवो॑ अ॒न्या ऽन्या वो॑ अ॒न्याम् । \newline
7. वो॒ अ॒न्या म॒न्यां ॅवो॑ वो अ॒न्या म॑व त्वव त्व॒न्यां ॅवो॑ वो अ॒न्या म॑वतु । \newline
8. अ॒न्या म॑व त्वव त्व॒न्या म॒न्या म॑व त्व॒न्या ऽन्या ऽव॑ त्व॒न्या म॒न्या म॑व त्व॒न्या । \newline
9. अ॒व॒ त्व॒न्या ऽन्या ऽव॑ त्वव त्व॒न्या ऽन्यस्या॑ अ॒न्यस्या॑ अ॒न्या ऽव॑ त्वव त्व॒न्या ऽन्यस्याः᳚ । \newline
10. अ॒न्या ऽन्यस्या॑ अ॒न्यस्या॑ अ॒न्या ऽन्या ऽन्यस्या॒ उपोपा॒ न्यस्या॑ अ॒न्या ऽन्या ऽन्यस्या॒ उप॑ । \newline
11. अ॒न्यस्या॒ उपोपा॒ न्यस्या॑ अ॒न्यस्या॒ उपा॑वता व॒तोपा॒ न्यस्या॑ अ॒न्यस्या॒ उपा॑वत । \newline
12. उपा॑वता व॒तो पोपा॑वत । \newline
13. अ॒व॒तेत्य॑वत । \newline
14. ताः सर्वाः॒ सर्वा ॒स्ता स्ताः सर्वा॒ ओष॑धय॒ ओष॑धयः॒ सर्वा॒ स्ता स्ताः सर्वा॒ ओष॑धयः । \newline
15. सर्वा॒ ओष॑धय॒ ओष॑धयः॒ सर्वाः॒ सर्वा॒ ओष॑धयः संॅविदा॒नाः सं॑ॅविदा॒ना ओष॑धयः॒ सर्वाः॒ सर्वा॒ ओष॑धयः संॅविदा॒नाः । \newline
16. ओष॑धयः संॅविदा॒नाः सं॑ॅविदा॒ना ओष॑धय॒ ओष॑धयः संॅविदा॒ना इ॒द मि॒दꣳ सं॑ॅविदा॒ना ओष॑धय॒ ओष॑धयः संॅविदा॒ना इ॒दम् । \newline
17. सं॒ॅवि॒दा॒ना इ॒द मि॒दꣳ सं॑ॅविदा॒नाः सं॑ॅविदा॒ना इ॒दम् मे॑ म इ॒दꣳ सं॑ॅविदा॒नाः सं॑ॅविदा॒ना इ॒दम् मे᳚ । \newline
18. सं॒ॅवि॒दा॒ना इति॑ सं - वि॒दा॒नाः । \newline
19. इ॒दम् मे॑ म इ॒द मि॒दम् मे॒ प्र प्र म॑ इ॒द मि॒दम् मे॒ प्र । \newline
20. मे॒ प्र प्र मे॑ मे॒ प्राव॑ता वत॒ प्र मे॑ मे॒ प्राव॑त । \newline
21. प्राव॑ता वत॒ प्र प्राव॑ता॒ वचो॒ वचो॑ ऽवत॒ प्र प्राव॑ता॒ वचः॑ । \newline
22. अ॒व॒ता॒ वचो॒ वचो॑ ऽवता वता॒ वचः॑ । \newline
23. वच॒ इति॒ वचः॑ । \newline
24. उच्छुष्माः॒ शुष्मा॒ उदु च्छुष्मा॒ ओष॑धीना॒ मोष॑धीनाꣳ॒॒ शुष्मा॒ उदु च्छुष्मा॒ ओष॑धीनां । \newline
25. शुष्मा॒ ओष॑धीना॒ मोष॑धीनाꣳ॒॒ शुष्माः॒ शुष्मा॒ ओष॑धीनां॒ गावो॒ गाव॒ ओष॑धीनाꣳ॒॒ शुष्माः॒ शुष्मा॒ ओष॑धीनां॒ गावः॑ । \newline
26. ओष॑धीनां॒ गावो॒ गाव॒ ओष॑धीना॒ मोष॑धीनां॒ गावो॑ गो॒ष्ठाद् गो॒ष्ठाद् गाव॒ ओष॑धीना॒ मोष॑धीनां॒ गावो॑ गो॒ष्ठात् । \newline
27. गावो॑ गो॒ष्ठाद् गो॒ष्ठाद् गावो॒ गावो॑ गो॒ष्ठा दि॑वेव गो॒ष्ठाद् गावो॒ गावो॑ गो॒ष्ठादि॑व । \newline
28. गो॒ष्ठा दि॑वेव गो॒ष्ठाद् गो॒ष्ठा दि॑वेरत ईरत इव गो॒ष्ठाद् गो॒ष्ठा दि॑वेरते । \newline
29. गो॒ष्ठादिति॑ गो - स्थात् । \newline
30. इ॒वे॒र॒त॒ ई॒र॒त॒ इ॒वे॒ वे॒र॒ते॒ । \newline
31. ई॒र॒त॒ इती॑रते । \newline
32. धनꣳ॑ सनि॒ष्यन्ती॑नाꣳ सनि॒ष्यन्ती॑ना॒म् धन॒म् धनꣳ॑ सनि॒ष्यन्ती॑ना मा॒त्मान॑ मा॒त्मानꣳ॑ सनि॒ष्यन्ती॑ना॒म् धन॒म् धनꣳ॑ सनि॒ष्यन्ती॑ना मा॒त्मान᳚म् । \newline
33. स॒नि॒ष्यन्ती॑ना मा॒त्मान॑ मा॒त्मानꣳ॑ सनि॒ष्यन्ती॑नाꣳ सनि॒ष्यन्ती॑ना मा॒त्मान॒म् तव॒ तवा॒त्मानꣳ॑ सनि॒ष्यन्ती॑नाꣳ सनि॒ष्यन्ती॑ना मा॒त्मान॒म् तव॑ । \newline
34. आ॒त्मान॒म् तव॒ तवा॒त्मान॑ मा॒त्मान॒म् तव॑ पूरुष पूरुष॒ तवा॒त्मान॑ मा॒त्मान॒म् तव॑ पूरुष । \newline
35. तव॑ पूरुष पूरुष॒ तव॒ तव॑ पूरुष । \newline
36. पू॒रु॒षेति॑ पूरुष । \newline
37. अति॒ विश्वा॒ विश्वा॒ अत्यति॒ विश्वाः᳚ परि॒ष्ठाः प॑रि॒ष्ठा विश्वा॒ अत्यति॒ विश्वाः᳚ परि॒ष्ठाः । \newline
38. विश्वाः᳚ परि॒ष्ठाः प॑रि॒ष्ठा विश्वा॒ विश्वाः᳚ परि॒ष्ठाः स्ते॒नः स्ते॒नः प॑रि॒ष्ठा विश्वा॒ विश्वाः᳚ परि॒ष्ठाः स्ते॒नः । \newline
39. प॒रि॒ष्ठाः स्ते॒नः स्ते॒नः प॑रि॒ष्ठाः प॑रि॒ष्ठाः स्ते॒न इ॑वेव स्ते॒नः प॑रि॒ष्ठाः प॑रि॒ष्ठाः स्ते॒न इ॑व । \newline
40. प॒रि॒ष्ठा इति॑ परि - स्थाः । \newline
41. स्ते॒न इ॑वेव स्ते॒नः स्ते॒न इ॑व व्र॒जं ॅव्र॒ज मि॑व स्ते॒नः स्ते॒न इ॑व व्र॒जम् । \newline
42. इ॒व॒ व्र॒जं ॅव्र॒ज मि॑वेव व्र॒ज म॑क्रमु रक्रमुर् व्र॒ज मि॑वेव व्र॒ज म॑क्रमुः । \newline
43. व्र॒ज म॑क्रमु रक्रमुर् व्र॒जं ॅव्र॒ज म॑क्रमुः । \newline
44. अ॒क्र॒मु॒रित्य॑क्रमुः । \newline
45. ओष॑धयः॒ प्र प्रौष॑धय॒ ओष॑धयः॒ प्राचु॑च्यवु रचुच्यवुः॒ प्रौष॑धय॒ ओष॑धयः॒ प्राचु॑च्यवुः । \newline
46. प्राचु॑च्यवु रचुच्यवुः॒ प्र प्राचु॑च्यवु॒र् यद् यद॑चुच्यवुः॒ प्र प्राचु॑च्यवु॒र् यत् । \newline
47. अ॒चु॒च्य॒वु॒र् यद् यद॑चुच्यवु रचुच्यवु॒र् यत् किम् किं ॅयद॑चुच्यवु रचुच्यवु॒र् यत् किम् । \newline
48. यत् किम् किं ॅयद् यत् किम् च॑ च॒ किं ॅयद् यत् किम् च॑ । \newline
49. किम् च॑ च॒ किम् किम् च॑ त॒नुवा᳚म् त॒नुवा᳚म् च॒ किम् किम् च॑ त॒नुवा᳚म् । \newline
50. च॒ त॒नुवा᳚म् त॒नुवा᳚म् च च त॒नुवाꣳ॒॒ रपो॒ रप॑ स्त॒नुवा᳚म् च च त॒नुवाꣳ॒॒ रपः॑ । \newline
51. त॒नुवाꣳ॒॒ रपो॒ रप॑ स्त॒नुवा᳚म् त॒नुवाꣳ॒॒ रपः॑ । \newline
52. रप॒ इति॒ रपः॑ । \newline
53. या स्ते॑ ते॒ या या स्त॑ आत॒स्थु रा॑त॒स्थु स्ते॒ या या स्त॑ आत॒स्थुः । \newline
\pagebreak
\markright{ TS 4.2.6.4  \hfill https://www.vedavms.in \hfill}

\section{ TS 4.2.6.4 }

\textbf{TS 4.2.6.4 } \newline
\textbf{Samhita Paata} \newline

-स्त॑ आत॒स्थु-रा॒त्मानं॒ ॅया आ॑विवि॒शुः परुः॑ परुः ।तास्ते॒ यक्ष्मं॒ ॅविबा॑धन्ता मु॒ग्रो म॑द्ध्यम॒शीरि॑व ॥सा॒कं ॅय॑क्ष्म॒ प्र प॑त श्ये॒नेन॑ किकिदी॒विना᳚ । सा॒कं ॅवात॑स्य॒-ध्राज्या॑ सा॒कं न॑श्य नि॒हाक॑या ॥ अ॒श्वा॒व॒तीꣳ सो॑मव॒ती मू॒र्जय॑न्ती॒ मुदो॑जसं । आ वि॑थ्सि॒ सर्वा॒ ओष॑धीर॒स्मा अ॑रि॒ष्टता॑तये ॥ याः फ॒लिनी॒र्या अ॑फ॒ला अ॑पु॒ष्पा याश्च॑ पु॒ष्पिणीः᳚ । बृह॒स्पति॑ प्रसूता॒ स्तानो॑ मुञ्च॒न्त्वꣳ ह॑सः ॥ या - [  ] \newline

\textbf{Pada Paata} \newline

ते । आ॒त॒स्थुरित्या᳚ - त॒स्थुः । आ॒त्मान᳚म् । याः । आ॒वि॒वि॒शुरित्या᳚-वि॒वि॒शुः । परुः॑ परु॒रिति॒ परुः॑ - प॒रुः॒ ॥ ताः । ते॒ । यक्ष्म᳚म् । वीति॑ । बा॒ध॒न्ता॒म् । उ॒ग्रः । म॒द्ध्य॒म॒शीरिति॑ मद्ध्यम - शीः । इ॒व॒ ॥ सा॒कम् । य॒क्ष्म॒ । प्रेति॑ । प॒त॒ । श्ये॒नेन॑ । कि॒कि॒दी॒विना᳚ ॥ सा॒कम् । वात॑स्य । ध्राज्या᳚ । सा॒कम् । न॒श्य॒ । नि॒हाक॒येति॑ नि - हाक॑या ॥ अ॒श्वा॒व॒तीमित्य॑श्व - व॒तीम् । सो॒म॒व॒तीमिति॑ सोम - व॒तीम् । ऊ॒र्जय॑न्तीम् । उदो॑जस॒मित्युत् - ओ॒ज॒स॒म् ॥ एति॑ । वि॒थ्सि॒ । सर्वाः᳚ । ओष॑धीः । अ॒स्मै । अ॒रि॒ष्टता॑तय॒ इत्य॑रि॒ष्ट - ता॒त॒ये॒ ॥ याः । फ॒लिनीः᳚ । याः । अ॒फ॒लाः । अ॒पु॒ष्पाः । याः । च॒ । पु॒ष्पिणीः᳚ ॥ बृह॒स्पति॑प्रसूता॒ इति॒ बृह॒स्पति॑ - प्र॒सू॒ताः॒ । ताः । नः॒ । मु॒ञ्च॒न्तु॒ । अꣳह॑सः ॥ याः ।  \newline


\textbf{Krama Paata} \newline

त॒ आ॒त॒स्थुः । आ॒त॒स्थुरा॒त्मान᳚म् । आ॒त॒स्थुरित्या᳚ - त॒स्थुः । आ॒त्मानं॒ ॅयाः । या आ॑विवि॒शुः । आ॒वि॒वि॒शुः परुः॑परुः । आ॒वि॒वि॒शुरित्या᳚ - वि॒वि॒शुः । परुः॑परु॒रिति॒ परुः॑ - प॒रुः॒ ॥ तास्ते᳚ । ते॒ यक्ष्म᳚म् । यक्ष्मं॒ ॅवि । वि बा॑धन्ताम् । बा॒ध॒न्ता॒मु॒ग्रः । उ॒ग्रो म॑द्ध्यम॒शीः । म॒द्ध्य॒म॒शीरि॑व । म॒द्ध्य॒म॒शीरिति॑ मद्ध्यम - शीः । इ॒वेती॑व ॥ सा॒कं ॅय॑क्ष्म । य॒क्ष्म॒ प्र । प्र प॑त । प॒त॒ श्ये॒नेन॑ । श्ये॒नेन॑ किकिदी॒विना᳚ । कि॒कि॒दी॒विनेति॑ किकिदी॒विना᳚ ॥ सा॒कं ॅवात॑स्य । वात॑स्य॒ ध्राज्या᳚ । ध्राज्या॑ सा॒कम् । सा॒कम् न॑श्य । न॒श्य॒ नि॒हाक॑या । नि॒हाक॒येति॑ नि - हाक॑या ॥ अ॒श्वा॒व॒तीꣳ सो॑मव॒तीम् । अ॒श्वा॒व॒तीमित्य॑श्व - व॒तीम् । सो॒म॒व॒तीमू॒र्जय॑न्तीम् । सो॒म॒व॒तीमिति॑ सोम - व॒तीम् । ऊ॒र्जय॑न्ती॒मुदो॑जसम् । उदो॑जस॒मित्युत् - ओ॒ज॒स॒म् ॥ आ वि॑थ्सि । वि॒थ्सि॒ सर्वाः᳚ । सर्वा॒ ओष॑धीः । ओष॑धीर॒स्मै । अ॒स्मा अ॑रि॒ष्टता॑तये । अ॒रि॒ष्टता॑तय॒ इत्य॑रि॒ष्ट - ता॒त॒ये॒ ॥ याः फ॒लिनीः᳚ । फ॒लिनी॒र् याः । या अ॑फ॒लाः । अ॒फ॒ला अ॑पु॒ष्पाः । अ॒पु॒ष्पा याः । याश्च॑ । च॒ पु॒ष्पिणीः᳚ । पु॒ष्पिणी॒रिति॑ पु॒ष्पिणीः᳚ ॥ बृह॒स्पति॑प्रसूता॒स्ताः । बृह॒स्पति॑प्रसूता॒ इति॒ बृह॒स्पति॑ - प्र॒सू॒ताः॒ । ता नः॑ । नो॒ मु॒ञ्च॒न्तु॒ । मु॒ञ्च॒न्त्वꣳह॑सः । अꣳह॑स॒ इत्यꣳह॑सः ॥ या ओष॑धयः \newline

\textbf{Jatai Paata} \newline

1. त॒ आ॒त॒स्थु रा॑त॒स्थु स्ते॑ त आत॒स्थुः । \newline
2. आ॒त॒स्थु रा॒त्मान॑ मा॒त्मान॑ मात॒स्थु रा॑त॒स्थु रा॒त्मान᳚म् । \newline
3. आ॒त॒स्थुरित्या᳚ - त॒स्थुः । \newline
4. आ॒त्मानं॒ ॅया या आ॒त्मान॑ मा॒त्मानं॒ ॅयाः । \newline
5. या आ॑विवि॒शु रा॑विवि॒शुर् या या आ॑विवि॒शुः । \newline
6. आ॒वि॒वि॒शुः परुः॑परुः॒ परुः॑परु राविवि॒शु रा॑विवि॒शुः परुः॑परुः । \newline
7. आ॒वि॒वि॒शुरित्या᳚ - वि॒वि॒शुः । \newline
8. परुः॑परु॒रिति॒ परुः॑ - प॒रुः॒ । \newline
9. ता स्ते॑ ते॒ ता स्ता स्ते᳚ । \newline
10. ते॒ यक्ष्मं॒ ॅयक्ष्म॑म् ते ते॒ यक्ष्म᳚म् । \newline
11. यक्ष्मं॒ ॅवि वि यक्ष्मं॒ ॅयक्ष्मं॒ ॅवि । \newline
12. वि बा॑धन्ताम् बाधन्तां॒ ॅवि वि बा॑धन्ताम् । \newline
13. बा॒ध॒न्ता॒ मु॒ग्र उ॒ग्रो बा॑धन्ताम् बाधन्ता मु॒ग्रः । \newline
14. उ॒ग्रो म॑द्ध्यम॒शीर् म॑द्ध्यम॒शी रु॒ग्र उ॒ग्रो म॑द्ध्यम॒शीः । \newline
15. म॒द्ध्य॒म॒शी रि॑वेव मद्ध्यम॒शीर् म॑द्ध्यम॒शी रि॑व । \newline
16. म॒द्ध्य॒म॒शीरिति॑ मद्ध्यम - शीः । \newline
17. इ॒वेती॑व । \newline
18. सा॒कं ॅय॑क्ष्म यक्ष्म सा॒कꣳ सा॒कं ॅय॑क्ष्म । \newline
19. य॒क्ष्म॒ प्र प्र य॑क्ष्म यक्ष्म॒ प्र । \newline
20. प्र प॑त पत॒ प्र प्र प॑त । \newline
21. प॒त॒ श्ये॒नेन॑ श्ये॒नेन॑ पत पत श्ये॒नेन॑ । \newline
22. श्ये॒नेन॑ किकिदी॒विना॑ किकिदी॒विना᳚ श्ये॒नेन॑ श्ये॒नेन॑ किकिदी॒विना᳚ । \newline
23. कि॒कि॒दी॒विनेति॑ किकिदी॒विना᳚ । \newline
24. सा॒कं ॅवात॑स्य॒ वात॑स्य सा॒कꣳ सा॒कं ॅवात॑स्य । \newline
25. वात॑स्य॒ ध्राज्या॒ ध्राज्या॒ वात॑स्य॒ वात॑स्य॒ ध्राज्या᳚ । \newline
26. ध्राज्या॑ सा॒कꣳ सा॒कम् ध्राज्या॒ ध्राज्या॑ सा॒कम् । \newline
27. सा॒कम् न॑श्य नश्य सा॒कꣳ सा॒कम् न॑श्य । \newline
28. न॒श्य॒ नि॒हाक॑या नि॒हाक॑या नश्य नश्य नि॒हाक॑या । \newline
29. नि॒हाक॒येति॑ नि - हाक॑या । \newline
30. अ॒श्वा॒व॒तीꣳ सो॑मव॒तीꣳ सो॑मव॒ती म॑श्वाव॒ती म॑श्वाव॒तीꣳ सो॑मव॒तीम् । \newline
31. अ॒श्वा॒व॒तीमित्य॑श्व - व॒तीम् । \newline
32. सो॒म॒व॒ती मू॒र्जय॑न्ती मू॒र्जय॑न्तीꣳ सोमव॒तीꣳ सो॑मव॒ती मू॒र्जय॑न्तीम् । \newline
33. सो॒म॒व॒तीमिति॑ सोम - व॒तीम् । \newline
34. ऊ॒र्जय॑न्ती॒ मुदो॑जस॒ मुदो॑जस मू॒र्जय॑न्ती मू॒र्जय॑न्ती॒ मुदो॑जसम् । \newline
35. उदो॑जस॒मित्युत् - ओ॒ज॒स॒म् । \newline
36. आ वि॑थ्सि वि॒थ्स्या वि॑थ्सि । \newline
37. वि॒थ्सि॒ सर्वाः॒ सर्वा॑ विथ्सि विथ्सि॒ सर्वाः᳚ । \newline
38. सर्वा॒ ओष॑धी॒ रोष॑धीः॒ सर्वाः॒ सर्वा॒ ओष॑धीः । \newline
39. ओष॑धी र॒स्मा अ॒स्मा ओष॑धी॒ रोष॑धी र॒स्मै । \newline
40. अ॒स्मा अ॑रि॒ष्टता॑तये अरि॒ष्टता॑तये अ॒स्मा अ॒स्मा अ॑रि॒ष्टता॑तये । \newline
41. अ॒रि॒ष्टता॑तय॒ इत्य॑रि॒ष्ट - ता॒त॒ये॒ । \newline
42. याः फ॒लिनीः᳚ फ॒लिनी॒र् या याः फ॒लिनीः᳚ । \newline
43. फ॒लिनी॒र् या याः फ॒लिनीः᳚ फ॒लिनी॒र् याः । \newline
44. या अ॑फ॒ला अ॑फ॒ला या या अ॑फ॒लाः । \newline
45. अ॒फ॒ला अ॑पु॒ष्पा अ॑पु॒ष्पा अ॑फ॒ला अ॑फ॒ला अ॑पु॒ष्पाः । \newline
46. अ॒पु॒ष्पा या या अ॑पु॒ष्पा अ॑पु॒ष्पा याः । \newline
47. याश्च॑ च॒ या याश्च॑ । \newline
48. च॒ पु॒ष्पिणीः᳚ पु॒ष्पिणी᳚श्च च पु॒ष्पिणीः᳚ । \newline
49. पु॒ष्पिणी॒रिति॑ पु॒ष्पिणीः᳚ । \newline
50. बृह॒स्पति॑प्रसूता॒ स्ता स्ता बृह॒स्पति॑प्रसूता॒ बृह॒स्पति॑प्रसूता॒ स्ताः । \newline
51. बृह॒स्पति॑प्रसूता॒ इति॒ बृह॒स्पति॑ - प्र॒सू॒ताः॒ । \newline
52. ता नो॑ न॒ स्ता स्ता नः॑ । \newline
53. नो॒ मु॒ञ्च॒न्तु॒ मु॒ञ्च॒न्तु॒ नो॒ नो॒ मु॒ञ्च॒न्तु॒ । \newline
54. मु॒ञ्च॒ न्त्वꣳह॑सो॒ अꣳह॑सो मुञ्चन्तु मुञ्च॒ न्त्वꣳह॑सः । \newline
55. अꣳह॑स॒ इत्यꣳह॑सः । \newline
56. या ओष॑धय॒ ओष॑धयो॒ या या ओष॑धयः । \newline

\textbf{Ghana Paata } \newline

1. त॒ आ॒त॒स्थु रा॑त॒स्थु स्ते॑ त आत॒स्थु रा॒त्मान॑ मा॒त्मान॑ मात॒स्थु स्ते॑ त आत॒स्थु रा॒त्मान᳚म् । \newline
2. आ॒त॒स्थु रा॒त्मान॑ मा॒त्मान॑ मात॒स्थु रा॑त॒स्थु रा॒त्मानं॒ ॅया या आ॒त्मान॑ मात॒स्थु रा॑त॒स्थु रा॒त्मानं॒ ॅयाः । \newline
3. आ॒त॒स्थुरित्या᳚ - त॒स्थुः । \newline
4. आ॒त्मानं॒ ॅया या आ॒त्मान॑ मा॒त्मानं॒ ॅया आ॑विवि॒शु रा॑विवि॒शुर् या आ॒त्मान॑ मा॒त्मानं॒ ॅया आ॑विवि॒शुः । \newline
5. या आ॑विवि॒शु रा॑विवि॒शुर् या या आ॑विवि॒शुः परुः॑परुः॒ परुः॑परु राविवि॒शुर् या या आ॑विवि॒शुः परुः॑परुः । \newline
6. आ॒वि॒वि॒शुः परुः॑परुः॒ परुः॑परु राविवि॒शु रा॑विवि॒शुः परुः॑परुः । \newline
7. आ॒वि॒वि॒शुरित्या᳚ - वि॒वि॒शुः । \newline
8. परुः॑परु॒रिति॒ परुः॑ - प॒रुः॒ । \newline
9. ता स्ते॑ ते॒ ता स्ता स्ते॒ यक्ष्मं॒ ॅयक्ष्म॑म् ते॒ ता स्ता स्ते॒ यक्ष्म᳚म् । \newline
10. ते॒ यक्ष्मं॒ ॅयक्ष्म॑म् ते ते॒ यक्ष्मं॒ ॅवि वि यक्ष्म॑म् ते ते॒ यक्ष्मं॒ ॅवि । \newline
11. यक्ष्मं॒ ॅवि वि यक्ष्मं॒ ॅयक्ष्मं॒ ॅवि बा॑धन्ताम् बाधन्तां॒ ॅवि यक्ष्मं॒ ॅयक्ष्मं॒ ॅवि बा॑धन्ताम् । \newline
12. वि बा॑धन्ताम् बाधन्तां॒ ॅवि वि बा॑धन्ता मु॒ग्र उ॒ग्रो बा॑धन्तां॒ ॅवि वि बा॑धन्ता मु॒ग्रः । \newline
13. बा॒ध॒न्ता॒ मु॒ग्र उ॒ग्रो बा॑धन्ताम् बाधन्ता मु॒ग्रो म॑द्ध्यम॒शीर् म॑द्ध्यम॒शी रु॒ग्रो बा॑धन्ताम् बाधन्ता मु॒ग्रो म॑द्ध्यम॒शीः । \newline
14. उ॒ग्रो म॑द्ध्यम॒शीर् म॑द्ध्यम॒शी रु॒ग्र उ॒ग्रो म॑द्ध्यम॒शी रि॑वेव मद्ध्यम॒शी रु॒ग्र उ॒ग्रो म॑द्ध्यम॒शी रि॑व । \newline
15. म॒द्ध्य॒म॒शी रि॑वेव मद्ध्यम॒शीर् म॑द्ध्यम॒शी रि॑व । \newline
16. म॒द्ध्य॒म॒शीरिति॑ मद्ध्यम - शीः । \newline
17. इ॒वेती॑व । \newline
18. सा॒कं ॅय॑क्ष्म यक्ष्म सा॒कꣳ सा॒कं ॅय॑क्ष्म॒ प्र प्र य॑क्ष्म सा॒कꣳ सा॒कं ॅय॑क्ष्म॒ प्र । \newline
19. य॒क्ष्म॒ प्र प्र य॑क्ष्म यक्ष्म॒ प्र प॑त पत॒ प्र य॑क्ष्म यक्ष्म॒ प्र प॑त । \newline
20. प्र प॑त पत॒ प्र प्र प॑त श्ये॒नेन॑ श्ये॒नेन॑ पत॒ प्र प्र प॑त श्ये॒नेन॑ । \newline
21. प॒त॒ श्ये॒नेन॑ श्ये॒नेन॑ पत पत श्ये॒नेन॑ किकिदी॒विना॑ किकिदी॒विना᳚ श्ये॒नेन॑ पत पत श्ये॒नेन॑ किकिदी॒विना᳚ । \newline
22. श्ये॒नेन॑ किकिदी॒विना॑ किकिदी॒विना᳚ श्ये॒नेन॑ श्ये॒नेन॑ किकिदी॒विना᳚ । \newline
23. कि॒कि॒दी॒विनेति॑ किकिदी॒विना᳚ । \newline
24. सा॒कं ॅवात॑स्य॒ वात॑स्य सा॒कꣳ सा॒कं ॅवात॑स्य॒ ध्राज्या॒ ध्राज्या॒ वात॑स्य सा॒कꣳ सा॒कं ॅवात॑स्य॒ ध्राज्या᳚ । \newline
25. वात॑स्य॒ ध्राज्या॒ ध्राज्या॒ वात॑स्य॒ वात॑स्य॒ ध्राज्या॑ सा॒कꣳ सा॒कम् ध्राज्या॒ वात॑स्य॒ वात॑स्य॒ ध्राज्या॑ सा॒कम् । \newline
26. ध्राज्या॑ सा॒कꣳ सा॒कम् ध्राज्या॒ ध्राज्या॑ सा॒कम् न॑श्य नश्य सा॒कम् ध्राज्या॒ ध्राज्या॑ सा॒कम् न॑श्य । \newline
27. सा॒कम् न॑श्य नश्य सा॒कꣳ सा॒कम् न॑श्य नि॒हाक॑या नि॒हाक॑या नश्य सा॒कꣳ सा॒कम् न॑श्य नि॒हाक॑या । \newline
28. न॒श्य॒ नि॒हाक॑या नि॒हाक॑या नश्य नश्य नि॒हाक॑या । \newline
29. नि॒हाक॒येति॑ नि - हाक॑या । \newline
30. अ॒श्वा॒व॒तीꣳ सो॑मव॒तीꣳ सो॑मव॒ती म॑श्वाव॒ती म॑श्वाव॒तीꣳ सो॑मव॒ती मू॒र्जय॑न्ती मू॒र्जय॑न्तीꣳ सोमव॒ती म॑श्वाव॒ती म॑श्वाव॒तीꣳ सो॑मव॒ती मू॒र्जय॑न्तीम् । \newline
31. अ॒श्वा॒व॒तीमित्य॑श्व - व॒तीम् । \newline
32. सो॒म॒व॒ती मू॒र्जय॑न्ती मू॒र्जय॑न्तीꣳ सोमव॒तीꣳ सो॑मव॒ती मू॒र्जय॑न्ती॒ मुदो॑जस॒ मुदो॑जस मू॒र्जय॑न्तीꣳ सोमव॒तीꣳ सो॑मव॒ती मू॒र्जय॑न्ती॒ मुदो॑जसम् । \newline
33. सो॒म॒व॒तीमिति॑ सोम - व॒तीम् । \newline
34. ऊ॒र्जय॑न्ती॒ मुदो॑जस॒ मुदो॑जस मू॒र्जय॑न्ती मू॒र्जय॑न्ती॒ मुदो॑जसम् । \newline
35. उदो॑जस॒मित्युत् - ओ॒ज॒स॒म् । \newline
36. आ वि॑थ्सि वि॒थ्स्या वि॑थ्सि॒ सर्वाः॒ सर्वा॑ वि॒थ्स्या वि॑थ्सि॒ सर्वाः᳚ । \newline
37. वि॒थ्सि॒ सर्वाः॒ सर्वा॑ विथ्सि विथ्सि॒ सर्वा॒ ओष॑धी॒ रोष॑धीः॒ सर्वा॑ विथ्सि विथ्सि॒ सर्वा॒ ओष॑धीः । \newline
38. सर्वा॒ ओष॑धी॒ रोष॑धीः॒ सर्वाः॒ सर्वा॒ ओष॑धी र॒स्मा अ॒स्मा ओष॑धीः॒ सर्वाः॒ सर्वा॒ ओष॑धी र॒स्मै । \newline
39. ओष॑धी र॒स्मा अ॒स्मा ओष॑धी॒ रोष॑धी र॒स्मा अ॑रि॒ष्टता॑तये अरि॒ष्टता॑तये अ॒स्मा ओष॑धी॒ रोष॑धी र॒स्मा अ॑रि॒ष्टता॑तये । \newline
40. अ॒स्मा अ॑रि॒ष्टता॑तये अरि॒ष्टता॑तये अ॒स्मा अ॒स्मा अ॑रि॒ष्टता॑तये । \newline
41. अ॒रि॒ष्टता॑तय॒ इत्य॑रि॒ष्ट - ता॒त॒ये॒ । \newline
42. याः फ॒लिनीः᳚ फ॒लिनी॒र् या याः फ॒लिनी॒र् या याः फ॒लिनी॒र् या याः फ॒लिनी॒र् याः । \newline
43. फ॒लिनी॒र् या याः फ॒लिनीः᳚ फ॒लिनी॒र् या अ॑फ॒ला अ॑फ॒ला याः फ॒लिनीः᳚ फ॒लिनी॒र् या अ॑फ॒लाः । \newline
44. या अ॑फ॒ला अ॑फ॒ला या या अ॑फ॒ला अ॑पु॒ष्पा अ॑पु॒ष्पा अ॑फ॒ला या या अ॑फ॒ला अ॑पु॒ष्पाः । \newline
45. अ॒फ॒ला अ॑पु॒ष्पा अ॑पु॒ष्पा अ॑फ॒ला अ॑फ॒ला अ॑पु॒ष्पा या या अ॑पु॒ष्पा अ॑फ॒ला अ॑फ॒ला अ॑पु॒ष्पा याः । \newline
46. अ॒पु॒ष्पा या या अ॑पु॒ष्पा अ॑पु॒ष्पा याश्च॑ च॒ या अ॑पु॒ष्पा अ॑पु॒ष्पा याश्च॑ । \newline
47. याश्च॑ च॒ या याश्च॑ पु॒ष्पिणीः᳚ पु॒ष्पिणी᳚श्च॒ या याश्च॑ पु॒ष्पिणीः᳚ । \newline
48. च॒ पु॒ष्पिणीः᳚ पु॒ष्पिणी᳚श्च च पु॒ष्पिणीः᳚ । \newline
49. पु॒ष्पिणी॒रिति॑ पु॒ष्पिणीः᳚ । \newline
50. बृह॒स्पति॑प्रसूता॒ स्ता स्ता बृह॒स्पति॑प्रसूता॒ बृह॒स्पति॑प्रसूता॒ स्ता नो॑ न॒ स्ता बृह॒स्पति॑प्रसूता॒ बृह॒स्पति॑प्रसूता॒ स्ता नः॑ । \newline
51. बृह॒स्पति॑प्रसूता॒ इति॒ बृह॒स्पति॑ - प्र॒सू॒ताः॒ । \newline
52. ता नो॑ न॒ स्ता स्ता नो॑ मुञ्चन्तु मुञ्चन्तु न॒ स्ता स्ता नो॑ मुञ्चन्तु । \newline
53. नो॒ मु॒ञ्च॒न्तु॒ मु॒ञ्च॒न्तु॒ नो॒ नो॒ मु॒ञ्च॒ न्त्वꣳह॑सो॒ अꣳह॑सो मुञ्चन्तु नो नो मुञ्च॒ न्त्वꣳह॑सः । \newline
54. मु॒ञ्च॒ न्त्वꣳह॑सो॒ अꣳह॑सो मुञ्चन्तु मुञ्च॒ न्त्वꣳह॑सः । \newline
55. अꣳह॑स॒ इत्यꣳह॑सः । \newline
56. या ओष॑धय॒ ओष॑धयो॒ या या ओष॑धयः॒ सोम॑राज्ञीः॒ सोम॑राज्ञी॒ रोष॑धयो॒ या या ओष॑धयः॒ सोम॑राज्ञीः । \newline
\pagebreak
\markright{ TS 4.2.6.5  \hfill https://www.vedavms.in \hfill}

\section{ TS 4.2.6.5 }

\textbf{TS 4.2.6.5 } \newline
\textbf{Samhita Paata} \newline

ओष॑धयः॒ सोम॑राज्ञीः॒ प्रवि॑ष्टाः पृथि॒वीमनु॑ ।तासां॒ त्वम॑स्युत्त॒मा प्रणो॑ जी॒वात॑वे-सुव ॥ अ॒व॒पत॑न्तीरवदन् दि॒व ओष॑दयः॒ परि॑ । यं जी॒व म॒श्ञवा॑ महै॒ न स रि॑ष्याति॒ पूरु॑षः ॥ याश्चे॒द मु॑प-शृ॒ण्वन्ति॒ याश्च॑ दू॒रं परा॑गताः ।इ॒ह स॒ङ्गत्य॒ ताः सर्वा॑ अ॒स्मै सं द॑त्त भेष॒जं ॥मा वो॑ रिषत् खनि॒ता यस्मै॑ चा॒हं खना॑मि वः ( ) । द्वि॒प-च्चतु॑ष्प-द॒स्माकꣳ॒॒ सर्व॑-म॒स्त्वना॑तुरं ॥ ओष॑धयः॒ सं ॅव॑दन्ते॒ सोमे॑न स॒ह राज्ञा᳚ । यस्मै॑ क॒रोति॑ ब्राह्म॒णस्तꣳ रा॑जन् पारयामसि ॥ \newline

\textbf{Pada Paata} \newline

ओष॑धयः । सोम॑राज्ञी॒रिति॒ सोम॑ - रा॒ज्ञीः॒ । प्रवि॑ष्टा॒ इति॒ प्र-वि॒ष्टाः॒ । पृ॒थि॒वीम् । अनु॑ ॥ तासां᳚ । त्वम् । अ॒सि॒ । उ॒त्त॒मेत्यु॑त्- त॒मा । प्रेति॑ । नः॒ । जी॒वात॑वे । सु॒व॒ ॥ अ॒व॒पत॑न्ती॒रित्य॑व - पत॑न्तीः । अ॒व॒द॒न्न् । दि॒वः । ओष॑दयः । परि॑ ॥ यम् । जी॒वम् । अ॒श्नवा॑महै । न । सः । रि॒ष्या॒ति॒ । पूरु॑षः ॥ याः । च॒ । इ॒दम् । उ॒प॒शृ॒ण्वन्तीत्यु॑प-शृ॒ण्वन्ति॑ । याः । च॒ । दू॒रम् । परा॑गता॒ इति॒ परा᳚ - ग॒ताः॒ ॥ इ॒ह । स॒ङ्गत्येति॑ सं - गत्य॑ । ताः । सर्वाः᳚ । अ॒स्मै । समिति॑ । द॒त्त॒ । भे॒ष॒जम् ॥ मा । वः॒ । रि॒ष॒त् । ख॒नि॒ता । यस्मै᳚ । च॒ । अ॒हम् । खना॑मि । वः॒ ( ) ॥ द्वि॒पदिति॑ द्वि-पत् । चतु॑ष्प॒दिति॒ चतुः॑-प॒त् । अ॒स्माक᳚म् । सर्व᳚म् । अ॒स्तु॒ । अना॑तुर॒मित्यना᳚ - तु॒र॒म् ॥ ओष॑धयः । समिति॑ । व॒द॒न्ते॒ । सोमे॑न । स॒ह । राज्ञा᳚ ॥ यस्मै᳚ । क॒रोति॑ । ब्रा॒ह्म॒णः । तम् । रा॒ज॒न्न् । पा॒र॒या॒म॒सि॒ ॥  \newline


\textbf{Krama Paata} \newline

ओष॑धयः॒ सोम॑राज्ञीः । सोम॑राज्ञीः॒ प्रवि॑ष्टाः । सोम॑राज्ञी॒रिति॒ सोम॑ - रा॒ज्ञीः॒ । प्रवि॑ष्टाः पृथि॒वीम् । प्रवि॑ष्टा॒ इति॒ प्र - वि॒ष्टाः॒ । पृ॒थि॒वीमनु॑ । अन्वित्यनु॑ ॥ तासा॒म् त्वम् । त्वम॑सि । अ॒स्यु॒त्त॒मा । उ॒त्त॒मा प्र । उ॒त्त॒मेत्यु॑त् - त॒मा । प्र णः॑ । नो॒ जी॒वात॑वे । जी॒वात॑वे सुव । सु॒वेति॑ सुव ॥ अ॒व॒पत॑न्तीरवदन्न् । अ॒व॒पत॑न्ती॒रित्य॑व - पत॑न्तीः । अ॒व॒द॒न् दि॒वः । दि॒व ओष॑धयः । ओष॑धयः॒ परि॑ । परीति॒ परि॑ ॥ यम् जी॒वम् । जी॒वम॒श्ञवा॑महै । अ॒श्ञवा॑महै॒ न । न सः । स रि॑ष्याति । रि॒ष्या॒ति॒ पूरु॑षः । पूरु॑ष॒ इति॒ पूरु॑षः ॥ याश्च॑ । चे॒दम् । इ॒दमु॑पशृ॒ण्वन्ति॑ । उ॒प॒शृ॒ण्वन्ति॒ याः । उ॒प॒शृ॒ण्वन्तीत्यु॑प - शृ॒ण्वन्ति॑ । याश्च॑ । च॒ दू॒रम् । दू॒रम् परा॑गताः । परा॑गता॒ इति॒ परा᳚ - ग॒ताः॒ ॥ इ॒ह स॒ङ्गत्य॑ । स॒ङ्गत्य॒ ताः । स॒ङ्गत्येति॑ सम् - गत्य॑ । ताः सर्वाः᳚ । सर्वा॑ अ॒स्मै । अ॒स्मै सम् । सम् द॑त्त । द॒त्त॒ भे॒ष॒जम् । भे॒ष॒जमिति॑ भेष॒जम् ॥ मा वः॑ । वो॒ रि॒ष॒त्॒ । रि॒ष॒त् ख॒नि॒ता । ख॒नि॒ता यस्मै᳚ । यस्मै॑ च । चा॒हम् । अ॒हम् खना॑मि । खना॑मि वः ( ) । व॒ इति॑ वः ॥ द्वि॒पच्चतु॑ष्पत् । द्वि॒पदिति॑ द्वि - पत् । चतु॑ष्पद॒स्माक᳚म् । चतु॑ष्प॒दिति॒ चतुः॑ - प॒त्॒ । अ॒स्माकꣳ॒॒ सर्व᳚म् । सर्व॑म॒स्तु । अ॒स्त्वना॑तुरम् । अना॑तुर॒मित्यना᳚ - तु॒र॒म् ॥ ओष॑धयः॒ सम् । सं ॅव॑दन्ते । व॒द॒न्ते॒ सोमे॑न । सोमे॑न स॒ह । स॒ह राज्ञा᳚ । राज्ञेति॒ राज्ञा᳚ ॥ यस्मै॑ क॒रोति॑ । क॒रोति॑ ब्राह्म॒णः । ब्रा॒ह्म॒णस्तम् । तꣳ रा॑जन्न् । रा॒ज॒न् पा॒र॒या॒म॒सि॒ । पा॒र॒या॒म॒सीति॑ पारयामसि । \newline

\textbf{Jatai Paata} \newline

1. ओष॑धयः॒ सोम॑राज्ञीः॒ सोम॑राज्ञी॒ रोष॑धय॒ ओष॑धयः॒ सोम॑राज्ञीः । \newline
2. सोम॑राज्ञीः॒ प्रवि॑ष्टाः॒ प्रवि॑ष्टाः॒ सोम॑राज्ञीः॒ सोम॑राज्ञीः॒ प्रवि॑ष्टाः । \newline
3. सोम॑राज्ञी॒रिति॒ सोम॑ - रा॒ज्ञीः॒ । \newline
4. प्रवि॑ष्टाः पृथि॒वीम् पृ॑थि॒वीम् प्रवि॑ष्टाः॒ प्रवि॑ष्टाः पृथि॒वीम् । \newline
5. प्रवि॑ष्टा॒ इति॒ प्र - वि॒ष्टाः॒ । \newline
6. पृ॒थि॒वी मन्वनु॑ पृथि॒वीम् पृ॑थि॒वी मनु॑ । \newline
7. अन्वित्यनु॑ । \newline
8. तासां॒ त्वम् त्वम् तासां॒ तासां॒ त्वम् । \newline
9. त्व म॑स्यसि॒ त्वम् त्व म॑सि । \newline
10. अ॒स्यु॒त्त॒ मोत्त॒मा ऽस्य स्युत्त॒मा । \newline
11. उ॒त्त॒मा प्र प्रोत्त॒ मोत्त॒मा प्र । \newline
12. उ॒त्त॒मेत्यु॑त् - त॒मा । \newline
13. प्र णो॑ नः॒ प्र प्र णः॑ । \newline
14. नो॒ जी॒वात॑वे जी॒वात॑वे नो नो जी॒वात॑वे । \newline
15. जी॒वात॑वे सुव सुव जी॒वात॑वे जी॒वात॑वे सुव । \newline
16. सु॒वेति॑ सुव । \newline
17. अ॒व॒पत॑न्ती रवदन् नवदन् नव॒पत॑न्ती रव॒पत॑न्ती रवदन्न् । \newline
18. अ॒व॒पत॑न्ती॒रित्य॑व - पत॑न्तीः । \newline
19. अ॒व॒द॒न् दि॒वो दि॒वो॑ ऽवदन् नवदन् दि॒वः । \newline
20. दि॒व ओष॑धय॒ ओष॑धयो दि॒वो दि॒व ओष॑धयः । \newline
21. ओष॑धयः॒ परि॒ पर्योष॑धय॒ ओष॑धयः॒ परि॑ । \newline
22. परीति॒ परि॑ । \newline
23. यम् जी॒वम् जी॒वं ॅयं ॅयम् जी॒वम् । \newline
24. जी॒व म॒श्ञवा॑महा अ॒श्ञवा॑महै जी॒वम् जी॒व म॒श्ञवा॑महै । \newline
25. अ॒श्ञवा॑महै॒ न नाश्ञवा॑महा अ॒श्ञवा॑महै॒ न । \newline
26. न स स न न सः । \newline
27. स रि॑ष्याति रिष्याति॒ स स रि॑ष्याति । \newline
28. रि॒ष्या॒ति॒ पूरु॑षः॒ पूरु॑षो रिष्याति रिष्याति॒ पूरु॑षः । \newline
29. पूरु॑ष॒ इति॒ पूरु॑षः । \newline
30. याश्च॑ च॒ या याश्च॑ । \newline
31. चे॒द मि॒दम् च॑ चे॒दम् । \newline
32. इ॒द मु॑पशृ॒ण्वन् त्यु॑पशृ॒ण्वन्ती॒द मि॒द मु॑पशृ॒ण्वन्ति॑ । \newline
33. उ॒प॒शृ॒ण्वन्ति॒ या या उ॑पशृ॒ण्वन् त्यु॑पशृ॒ण्वन्ति॒ याः । \newline
34. उ॒प॒शृ॒ण्वन्तीत्यु॑प - शृ॒ण्वन्ति॑ । \newline
35. याश्च॑ च॒ या याश्च॑ । \newline
36. च॒ दू॒रम् दू॒रम् च॑ च दू॒रम् । \newline
37. दू॒रम् परा॑गताः॒ परा॑गता दू॒रम् दू॒रम् परा॑गताः । \newline
38. परा॑गता॒ इति॒ परा᳚ - ग॒ताः॒ । \newline
39. इ॒ह स॒ङ्गत्य॑ स॒ङ्गत्ये॒ हेह स॒ङ्गत्य॑ । \newline
40. स॒ङ्गत्य॒ ता स्ताः स॒ङ्गत्य॑ स॒ङ्गत्य॒ ताः । \newline
41. स॒ङ्गत्येति॑ सं - गत्य॑ । \newline
42. ताः सर्वाः॒ सर्वा॒ स्ता स्ताः सर्वाः᳚ । \newline
43. सर्वा॑ अ॒स्मा अ॒स्मै सर्वाः॒ सर्वा॑ अ॒स्मै । \newline
44. अ॒स्मै सꣳ स म॒स्मा अ॒स्मै सम् । \newline
45. सम् द॑त्त दत्त॒ सꣳ सम् द॑त्त । \newline
46. द॒त्त॒ भे॒ष॒जम् भे॑ष॒जम् द॑त्त दत्त भेष॒जम् । \newline
47. भे॒ष॒जमिति॑ भेष॒जम् । \newline
48. मा वो॑ वो॒ मा मा वः॑ । \newline
49. वो॒ रि॒ष॒द् रि॒ष॒द् वो॒ वो॒ रि॒ष॒त् । \newline
50. रि॒ष॒त् ख॒नि॒ता ख॑नि॒ता रि॑षद् रिषत् खनि॒ता । \newline
51. ख॒नि॒ता यस्मै॒ यस्मै॑ खनि॒ता ख॑नि॒ता यस्मै᳚ । \newline
52. यस्मै॑ च च॒ यस्मै॒ यस्मै॑ च । \newline
53. चा॒ह म॒हम् च॑ चा॒हम् । \newline
54. अ॒हम् खना॑मि॒ खना᳚ म्य॒ह म॒हम् खना॑मि । \newline
55. खना॑मि वो वः॒ खना॑मि॒ खना॑मि वः । \newline
56. व॒ इति॑ वः । \newline
57. द्वि॒पच् चतु॑ष्प॒च् चतु॑ष्पद् द्वि॒पद् द्वि॒पच् चतु॑ष्पत् । \newline
58. द्वि॒पदिति॑ द्वि - पत् । \newline
59. चतु॑ष्प द॒स्माक॑ म॒स्माक॒म् चतु॑ष्प॒च् चतु॑ष्प द॒स्माक᳚म् । \newline
60. चतु॑ष्प॒दिति॒ चतुः॑ - प॒त् । \newline
61. अ॒स्माकꣳ॒॒ सर्वꣳ॒॒ सर्व॑ म॒स्माक॑ म॒स्माकꣳ॒॒ सर्व᳚म् । \newline
62. सर्व॑ मस्त्वस्तु॒ सर्वꣳ॒॒ सर्व॑ मस्तु । \newline
63. अ॒स्त्वना॑तुर॒ मना॑तुर मस्त्व॒ स्त्वना॑तुरम् । \newline
64. अना॑तुर॒मित्यना᳚ - तु॒र॒म् । \newline
65. ओष॑धयः॒ सꣳ स मोष॑धय॒ ओष॑धयः॒ सम् । \newline
66. सं ॅव॑दन्ते वदन्ते॒ सꣳ सं ॅव॑दन्ते । \newline
67. व॒द॒न्ते॒ सोमे॑न॒ सोमे॑न वदन्ते वदन्ते॒ सोमे॑न । \newline
68. सोमे॑न स॒ह स॒ह सोमे॑न॒ सोमे॑न स॒ह । \newline
69. स॒ह राज्ञा॒ राज्ञा॑ स॒ह स॒ह राज्ञा᳚ । \newline
70. राज्ञेति॒ राज्ञा᳚ । \newline
71. यस्मै॑ क॒रोति॑ क॒रोति॒ यस्मै॒ यस्मै॑ क॒रोति॑ । \newline
72. क॒रोति॑ ब्राह्म॒णो ब्रा᳚ह्म॒णः क॒रोति॑ क॒रोति॑ ब्राह्म॒णः । \newline
73. ब्रा॒ह्म॒ण स्तम् तम् ब्रा᳚ह्म॒णो ब्रा᳚ह्म॒ण स्तम् । \newline
74. तꣳ रा॑जन् राज॒न् तम् तꣳ रा॑जन्न् । \newline
75. रा॒ज॒न् पा॒र॒या॒म॒सि॒ पा॒र॒या॒म॒सि॒ रा॒ज॒न् रा॒ज॒न् पा॒र॒या॒म॒सि॒ । \newline
76. पा॒र॒या॒म॒सीति॑ पारयामसि । \newline

\textbf{Ghana Paata } \newline

1. ओष॑धयः॒ सोम॑राज्ञीः॒ सोम॑राज्ञी॒ रोष॑धय॒ ओष॑धयः॒ सोम॑राज्ञीः॒ प्रवि॑ष्टाः॒ प्रवि॑ष्टाः॒ सोम॑राज्ञी॒ रोष॑धय॒ ओष॑धयः॒ सोम॑राज्ञीः॒ प्रवि॑ष्टाः । \newline
2. सोम॑राज्ञीः॒ प्रवि॑ष्टाः॒ प्रवि॑ष्टाः॒ सोम॑राज्ञीः॒ सोम॑राज्ञीः॒ प्रवि॑ष्टाः पृथि॒वीम् पृ॑थि॒वीम् प्रवि॑ष्टाः॒ सोम॑राज्ञीः॒ सोम॑राज्ञीः॒ प्रवि॑ष्टाः पृथि॒वीम् । \newline
3. सोम॑राज्ञी॒रिति॒ सोम॑ - रा॒ज्ञीः॒ । \newline
4. प्रवि॑ष्टाः पृथि॒वीम् पृ॑थि॒वीम् प्रवि॑ष्टाः॒ प्रवि॑ष्टाः पृथि॒वी मन्वनु॑ पृथि॒वीम् प्रवि॑ष्टाः॒ प्रवि॑ष्टाः पृथि॒वी मनु॑ । \newline
5. प्रवि॑ष्टा॒ इति॒ प्र - वि॒ष्टाः॒ । \newline
6. पृ॒थि॒वी मन्वनु॑ पृथि॒वीम् पृ॑थि॒वी मनु॑ । \newline
7. अन्वित्यनु॑ । \newline
8. तासां॒ त्वम् त्वम् तासां॒ तासां॒ त्व म॑स्यसि॒ त्वम् तासां॒ तासां॒ त्व म॑सि । \newline
9. त्व म॑स्यसि॒ त्वम् त्व म॑स्युत्त॒ मोत्त॒मा ऽसि॒ त्वम् त्व म॑स्युत्त॒मा । \newline
10. अ॒स्यु॒त्त॒ मोत्त॒मा ऽस्य स्युत्त॒मा प्र प्रोत्त॒मा ऽस्य स्युत्त॒मा प्र । \newline
11. उ॒त्त॒मा प्र प्रोत्त॒ मोत्त॒मा प्र णो॑ नः॒ प्रोत्त॒ मोत्त॒मा प्र णः॑ । \newline
12. उ॒त्त॒मेत्यु॑त् - त॒मा । \newline
13. प्र णो॑ नः॒ प्र प्र णो॑ जी॒वात॑वे जी॒वात॑वे नः॒ प्र प्र णो॑ जी॒वात॑वे । \newline
14. नो॒ जी॒वात॑वे जी॒वात॑वे नो नो जी॒वात॑वे सुव सुव जी॒वात॑वे नो नो जी॒वात॑वे सुव । \newline
15. जी॒वात॑वे सुव सुव जी॒वात॑वे जी॒वात॑वे सुव । \newline
16. सु॒वेति॑ सुव । \newline
17. अ॒व॒पत॑न्ती रवदन् नवदन् नव॒पत॑न्ती रव॒पत॑न्ती रवदन् दि॒वो दि॒वो॑ ऽवदन् नव॒पत॑न्ती रव॒पत॑न्ती रवदन् दि॒वः । \newline
18. अ॒व॒पत॑न्ती॒रित्य॑व - पत॑न्तीः । \newline
19. अ॒व॒द॒न् दि॒वो दि॒वो॑ ऽवदन् नवदन् दि॒व ओष॑धय॒ ओष॑धयो दि॒वो॑ ऽवदन् नवदन् दि॒व ओष॑धयः । \newline
20. दि॒व ओष॑धय॒ ओष॑धयो दि॒वो दि॒व ओष॑धयः॒ परि॒ पर्योष॑धयो दि॒वो दि॒व ओष॑धयः॒ परि॑ । \newline
21. ओष॑धयः॒ परि॒ पर्योष॑धय॒ ओष॑धयः॒ परि॑ । \newline
22. परीति॒ परि॑ । \newline
23. यम् जी॒वम् जी॒वं ॅयं ॅयम् जी॒व म॒श्ञवा॑महा अ॒श्ञवा॑महै जी॒वं ॅयं ॅयम् जी॒व म॒श्ञवा॑महै । \newline
24. जी॒व म॒श्ञवा॑महा अ॒श्ञवा॑महै जी॒वम् जी॒व म॒श्ञवा॑महै॒ न नाश्ञवा॑महै जी॒वम् जी॒व म॒श्ञवा॑महै॒ न । \newline
25. अ॒श्ञवा॑महै॒ न नाश्ञवा॑महा अ॒श्ञवा॑महै॒ न स स नाश्ञवा॑महा अ॒श्ञवा॑महै॒ न सः । \newline
26. न स स न न स रि॑ष्याति रिष्याति॒ स न न स रि॑ष्याति । \newline
27. स रि॑ष्याति रिष्याति॒ स स रि॑ष्याति॒ पूरु॑षः॒ पूरु॑षो रिष्याति॒ स स रि॑ष्याति॒ पूरु॑षः । \newline
28. रि॒ष्या॒ति॒ पूरु॑षः॒ पूरु॑षो रिष्याति रिष्याति॒ पूरु॑षः । \newline
29. पूरु॑ष॒ इति॒ पूरु॑षः । \newline
30. याश्च॑ च॒ या याश्चे॒द मि॒दम् च॒ या याश्चे॒दम् । \newline
31. चे॒द मि॒दम् च॑ चे॒द मु॑पशृ॒ण्व न्त्यु॑पशृ॒ण्वन्ती॒दम् च॑ चे॒द मु॑पशृ॒ण्वन्ति॑ । \newline
32. इ॒द मु॑पशृ॒ण्व न्त्यु॑पशृ॒ण्वन्ती॒द मि॒द मु॑पशृ॒ण्वन्ति॒ या या उ॑पशृ॒ण्वन्ती॒द मि॒द मु॑पशृ॒ण्वन्ति॒ याः । \newline
33. उ॒प॒शृ॒ण्वन्ति॒ या या उ॑पशृ॒ण्व न्त्यु॑पशृ॒ण्वन्ति॒ याश्च॑ च॒ या उ॑पशृ॒ण्व न्त्यु॑पशृ॒ण्वन्ति॒ याश्च॑ । \newline
34. उ॒प॒शृ॒ण्वन्तीत्यु॑प - शृ॒ण्वन्ति॑ । \newline
35. याश्च॑ च॒ या याश्च॑ दू॒रम् दू॒रम् च॒ या याश्च॑ दू॒रम् । \newline
36. च॒ दू॒रम् दू॒रम् च॑ च दू॒रम् परा॑गताः॒ परा॑गता दू॒रम् च॑ च दू॒रम् परा॑गताः । \newline
37. दू॒रम् परा॑गताः॒ परा॑गता दू॒रम् दू॒रम् परा॑गताः । \newline
38. परा॑गता॒ इति॒ परा᳚ - ग॒ताः॒ । \newline
39. इ॒ह स॒ङ्गत्य॑ स॒ङ्गत्ये॒ हेह स॒ङ्गत्य॒ ता स्ताः स॒ङ्गत्ये॒ हेह स॒ङ्गत्य॒ ताः । \newline
40. स॒ङ्गत्य॒ ता स्ताः स॒ङ्गत्य॑ स॒ङ्गत्य॒ ताः सर्वाः॒ सर्वा॒ स्ताः स॒ङ्गत्य॑ स॒ङ्गत्य॒ ताः सर्वाः᳚ । \newline
41. स॒ङ्गत्येति॑ सं - गत्य॑ । \newline
42. ताः सर्वाः॒ सर्वा॒ स्ता स्ताः सर्वा॑ अ॒स्मा अ॒स्मै सर्वा॒ स्ता स्ताः सर्वा॑ अ॒स्मै । \newline
43. सर्वा॑ अ॒स्मा अ॒स्मै सर्वाः॒ सर्वा॑ अ॒स्मै सꣳ स म॒स्मै सर्वाः॒ सर्वा॑ अ॒स्मै सम् । \newline
44. अ॒स्मै सꣳ स म॒स्मा अ॒स्मै सम् द॑त्त दत्त॒ स म॒स्मा अ॒स्मै सम् द॑त्त । \newline
45. सम् द॑त्त दत्त॒ सꣳ सम् द॑त्त भेष॒जम् भे॑ष॒जम् द॑त्त॒ सꣳ सम् द॑त्त भेष॒जम् । \newline
46. द॒त्त॒ भे॒ष॒जम् भे॑ष॒जम् द॑त्त दत्त भेष॒जम् । \newline
47. भे॒ष॒जमिति॑ भेष॒जम् । \newline
48. मा वो॑ वो॒ मा मा वो॑ रिषद् रिषद् वो॒ मा मा वो॑ रिषत् । \newline
49. वो॒ रि॒ष॒द् रि॒ष॒द् वो॒ वो॒ रि॒ष॒त् ख॒नि॒ता ख॑नि॒ता रि॑षद् वो वो रिषत् खनि॒ता । \newline
50. रि॒ष॒त् ख॒नि॒ता ख॑नि॒ता रि॑षद् रिषत् खनि॒ता यस्मै॒ यस्मै॑ खनि॒ता रि॑षद् रिषत् खनि॒ता यस्मै᳚ । \newline
51. ख॒नि॒ता यस्मै॒ यस्मै॑ खनि॒ता ख॑नि॒ता यस्मै॑ च च॒ यस्मै॑ खनि॒ता ख॑नि॒ता यस्मै॑ च । \newline
52. यस्मै॑ च च॒ यस्मै॒ यस्मै॑ चा॒ह म॒हम् च॒ यस्मै॒ यस्मै॑ चा॒हम् । \newline
53. चा॒ह म॒हम् च॑ चा॒हम् खना॑मि॒ खना᳚ म्य॒हम् च॑ चा॒हम् खना॑मि । \newline
54. अ॒हम् खना॑मि॒ खना᳚ म्य॒ह म॒हम् खना॑मि वो वः॒ खना᳚ म्य॒ह म॒हम् खना॑मि वः । \newline
55. खना॑मि वो वः॒ खना॑मि॒ खना॑मि वः । \newline
56. व॒ इति॑ वः । \newline
57. द्वि॒पच् चतु॑ष्प॒च् चतु॑ष्पद् द्वि॒पद् द्वि॒पच् चतु॑ष्प द॒स्माक॑ म॒स्माक॒म् चतु॑ष्पद् द्वि॒पद् द्वि॒पच् चतु॑ष्प द॒स्माक᳚म् । \newline
58. द्वि॒पदिति॑ द्वि - पत् । \newline
59. चतु॑ष्प द॒स्माक॑ म॒स्माक॒म् चतु॑ष्प॒च् चतु॑ष्प द॒स्माकꣳ॒॒ सर्वꣳ॒॒ सर्व॑ म॒स्माक॒म् चतु॑ष्प॒च् चतु॑ष्प द॒स्माकꣳ॒॒ सर्व᳚म् । \newline
60. चतु॑ष्प॒दिति॒ चतुः॑ - प॒त् । \newline
61. अ॒स्माकꣳ॒॒ सर्वꣳ॒॒ सर्व॑ म॒स्माक॑ म॒स्माकꣳ॒॒ सर्व॑ मस्त्वस्तु॒ सर्व॑ म॒स्माक॑ म॒स्माकꣳ॒॒ सर्व॑ मस्तु । \newline
62. सर्व॑ मस्त्वस्तु॒ सर्वꣳ॒॒ सर्व॑ म॒स्त्वना॑तुर॒ मना॑तुर मस्तु॒ सर्वꣳ॒॒ सर्व॑ म॒स्त्वना॑तुरम् । \newline
63. अ॒स्त्वना॑तुर॒ मना॑तुर मस्त्व॒ स्त्वना॑तुरम् । \newline
64. अना॑तुर॒मित्यना᳚ - तु॒र॒म् । \newline
65. ओष॑धयः॒ सꣳ स मोष॑धय॒ ओष॑धयः॒ सं ॅव॑दन्ते वदन्ते॒ स मोष॑धय॒ ओष॑धयः॒ सं ॅव॑दन्ते । \newline
66. सं ॅव॑दन्ते वदन्ते॒ सꣳ सं ॅव॑दन्ते॒ सोमे॑न॒ सोमे॑न वदन्ते॒ सꣳ सं ॅव॑दन्ते॒ सोमे॑न । \newline
67. व॒द॒न्ते॒ सोमे॑न॒ सोमे॑न वदन्ते वदन्ते॒ सोमे॑न स॒ह स॒ह सोमे॑न वदन्ते वदन्ते॒ सोमे॑न स॒ह । \newline
68. सोमे॑न स॒ह स॒ह सोमे॑न॒ सोमे॑न स॒ह राज्ञा॒ राज्ञा॑ स॒ह सोमे॑न॒ सोमे॑न स॒ह राज्ञा᳚ । \newline
69. स॒ह राज्ञा॒ राज्ञा॑ स॒ह स॒ह राज्ञा᳚ । \newline
70. राज्ञेति॒ राज्ञा᳚ । \newline
71. यस्मै॑ क॒रोति॑ क॒रोति॒ यस्मै॒ यस्मै॑ क॒रोति॑ ब्राह्म॒णो ब्रा᳚ह्म॒णः क॒रोति॒ यस्मै॒ यस्मै॑ क॒रोति॑ ब्राह्म॒णः । \newline
72. क॒रोति॑ ब्राह्म॒णो ब्रा᳚ह्म॒णः क॒रोति॑ क॒रोति॑ ब्राह्म॒ण स्तम् तम् ब्रा᳚ह्म॒णः क॒रोति॑ क॒रोति॑ ब्राह्म॒ण स्तम् । \newline
73. ब्रा॒ह्म॒ण स्तम् तम् ब्रा᳚ह्म॒णो ब्रा᳚ह्म॒ण स्तꣳ रा॑जन् राज॒न् तम् ब्रा᳚ह्म॒णो ब्रा᳚ह्म॒ण स्तꣳ रा॑जन्न् । \newline
74. तꣳ रा॑जन् राज॒न् तम् तꣳ रा॑जन् पारयामसि पारयामसि राज॒न् तम् तꣳ रा॑जन् पारयामसि । \newline
75. रा॒ज॒न् पा॒र॒या॒म॒सि॒ पा॒र॒या॒म॒सि॒ रा॒ज॒न् रा॒ज॒न् पा॒र॒या॒म॒सि॒ । \newline
76. पा॒र॒या॒म॒सीति॑ पारयामसि । \newline
\pagebreak
\markright{ TS 4.2.7.1  \hfill https://www.vedavms.in \hfill}

\section{ TS 4.2.7.1 }

\textbf{TS 4.2.7.1 } \newline
\textbf{Samhita Paata} \newline

मा नो॑ हिꣳसीज्जनि॒ता यः पृ॑थि॒व्या यो वा॒ दिवꣳ॑ स॒त्यध॑र्मा ज॒जान॑ । यश्चा॒पश्च॒न्द्रा बृ॑ह॒तीर्ज॒जान॒ कस्मै॑ दे॒वाय॑ ह॒विषा॑ विधेम ॥अ॒भ्याव॑र्तस्व पृथिवि य॒ज्ञेन॒ पय॑सा स॒ह । व॒पां ते॑ अ॒ग्निरि॑षि॒तोऽव॑ सर्पतु ॥ अग्ने॒ यत् ते॑ शु॒क्रं ॅयच्च॒न्द्रं ॅयत् पू॒तं ॅयद्-य॒ज्ञियं᳚ । तद्-दे॒वेभ्यो॑ भरामसि ॥ इष॒मूर्ज॑म॒हमि॒त आ - [  ] \newline

\textbf{Pada Paata} \newline

मा । नः॒ । हिꣳ॒॒सी॒त् । ज॒नि॒ता । यः । पृ॒थि॒व्याः । यः । वा॒ । दिव᳚म् । स॒त्यध॒र्मेति॑ स॒त्य - ध॒र्मा॒ । ज॒जान॑ ॥ यः । च॒ । अ॒पः । च॒न्द्राः । बृ॒ह॒तीः । ज॒जान॑ । कस्मै᳚ । दे॒वाय॑ । ह॒विषा᳚ । वि॒धे॒म॒ ॥ अ॒भ्याव॑र्त॒स्वेत्य॑भि - आव॑र्तस्व । पृ॒थि॒वि॒ । य॒ज्ञेन॑ । पय॑सा । स॒ह ॥ व॒पाम् । ते॒ । अ॒ग्निः । इ॒षि॒तः । अवेति॑ । स॒र्प॒तु॒ ॥ अग्ने᳚ । यत् । ते॒ । शु॒क्रम् । यत् । च॒न्द्रम् । यत् । पू॒तम् । यत् । य॒ज्ञिय᳚म् ॥ तत् । दे॒वेभ्यः॑ । भ॒रा॒म॒सि॒ ॥ इष᳚म् । ऊर्ज᳚म् । अ॒हम् । इ॒तः । एति॑ ।  \newline


\textbf{Krama Paata} \newline

मा नः॑ । नो॒ हिꣳ॒॒सी॒त्॒ । हिꣳ॒॒सी॒ज् ज॒नि॒ता । ज॒नि॒ता यः । यः पृ॑थि॒व्याः । पृ॒थि॒व्या यः । यो वा᳚ । वा॒ दिव᳚म् । दिवꣳ॑ स॒त्यध॑र्मा । स॒त्यध॑र्मा ज॒जान॑ । स॒त्यध॒र्मेति॑ स॒त्य - ध॒र्मा॒ । ज॒जानेति॑ ज॒जान॑ ॥ यश्च॑ । चा॒पः । अ॒पश्च॒न्द्राः । च॒न्द्रा बृ॑ह॒तीः । बृ॒ह॒तीर् ज॒जान॑ । ज॒जान॒ कस्मे᳚ । कस्मै॑ दे॒वाय॑ । दे॒वाय॑ ह॒विषा᳚ । ह॒विषा॑ विधेम । वि॒धे॒मेति॑ विधेम ॥ अ॒भ्याव॑र्तस्व पृथिवि । अ॒भ्याव॑र्त॒स्वेत्य॑भि - आव॑र्तस्व । पृ॒थि॒वि॒ य॒ज्ञेन॑ । य॒ज्ञेन॒ पय॑सा । पय॑सा स॒ह । स॒हेति॑ स॒ह ॥ व॒पाम् ते᳚ । ते॒ अ॒ग्निः । अ॒ग्निरि॑षि॒तः । इ॒षि॒तोऽव॑ । अव॑ सर्पतु । स॒र्प॒त्विति॑ सर्पतु ॥ अग्ने॒ यत् । यत् ते᳚ । ते॒ शु॒क्रम् । शु॒क्रं ॅयत् । यच् च॒न्द्रम् । च॒न्द्रम् ॅयत् । यत् पू॒तम् । पू॒तं ॅयत् । यद् य॒ज्ञियम्᳚ । य॒ज्ञिय॒मिति॑ य॒ज्ञिय᳚म् ॥ तद् दे॒वेभ्यः॑ । दे॒वेभ्यो॑ भरामसि । भ॒रा॒म॒सीति॑ भरामसि ॥ इष॒मूर्ज᳚म् । ऊर्ज॑म॒हम् । अ॒हमि॒तः । इ॒त आ । आ द॑दे \newline

\textbf{Jatai Paata} \newline

1. मा नो॑ नो॒ मा मा नः॑ । \newline
2. नो॒ हिꣳ॒॒सी॒ द्धिꣳ॒॒सी॒न् नो॒ नो॒ हिꣳ॒॒सी॒त् । \newline
3. हिꣳ॒॒सी॒ज् ज॒नि॒ता ज॑नि॒ता हिꣳ॑सी द्धिꣳसीज् जनि॒ता । \newline
4. ज॒नि॒ता यो यो ज॑नि॒ता ज॑नि॒ता यः । \newline
5. यः पृ॑थि॒व्याः पृ॑थि॒व्या यो यः पृ॑थि॒व्याः । \newline
6. पृ॒थि॒व्या यो यः पृ॑थि॒व्याः पृ॑थि॒व्या यः । \newline
7. यो वा॑ वा॒ यो यो वा᳚ । \newline
8. वा॒ दिव॒म् दिवं॑ ॅवा वा॒ दिव᳚म् । \newline
9. दिवꣳ॑ स॒त्यध॑र्मा स॒त्यध॑र्मा॒ दिव॒म् दिवꣳ॑ स॒त्यध॑र्मा । \newline
10. स॒त्यध॑र्मा ज॒जान॑ ज॒जान॑ स॒त्यध॑र्मा स॒त्यध॑र्मा ज॒जान॑ । \newline
11. स॒त्यध॒र्मेति॑ स॒त्य - ध॒र्मा॒ । \newline
12. ज॒जानेति॑ ज॒जान॑ । \newline
13. यश्च॑ च॒ यो यश्च॑ । \newline
14. चा॒पो अ॒पश्च॑ चा॒पः । \newline
15. अ॒प श्च॒न्द्रा श्च॒न्द्रा अ॒पो अ॒प श्च॒न्द्राः । \newline
16. च॒न्द्रा बृ॑ह॒तीर् बृ॑ह॒ती श्च॒न्द्रा श्च॒न्द्रा बृ॑ह॒तीः । \newline
17. बृ॒ह॒तीर् ज॒जान॑ ज॒जान॑ बृह॒तीर् बृ॑ह॒तीर् ज॒जान॑ । \newline
18. ज॒जान॒ कस्मै॒ कस्मै॑ ज॒जान॑ ज॒जान॒ कस्मै᳚ । \newline
19. कस्मै॑ दे॒वाय॑ दे॒वाय॒ कस्मै॒ कस्मै॑ दे॒वाय॑ । \newline
20. दे॒वाय॑ ह॒विषा॑ ह॒विषा॑ दे॒वाय॑ दे॒वाय॑ ह॒विषा᳚ । \newline
21. ह॒विषा॑ विधेम विधेम ह॒विषा॑ ह॒विषा॑ विधेम । \newline
22. वि॒धे॒मेति॑ विधेम । \newline
23. अ॒भ्याव॑र्तस्व पृथिवि पृथि व्य॒भ्याव॑र्तस्वा॒ भ्याव॑र्तस्व पृथिवि । \newline
24. अ॒भ्याव॑र्त॒स्वेत्य॑भि - आव॑र्तस्व । \newline
25. पृ॒थि॒वि॒ य॒ज्ञेन॑ य॒ज्ञेन॑ पृथिवि पृथिवि य॒ज्ञेन॑ । \newline
26. य॒ज्ञेन॒ पय॑सा॒ पय॑सा य॒ज्ञेन॑ य॒ज्ञेन॒ पय॑सा । \newline
27. पय॑सा स॒ह स॒ह पय॑सा॒ पय॑सा स॒ह । \newline
28. स॒हेति॑ स॒ह । \newline
29. व॒पाम् ते॑ ते व॒पां ॅव॒पाम् ते᳚ । \newline
30. ते॒ अ॒ग्नि र॒ग्नि स्ते॑ ते अ॒ग्निः । \newline
31. अ॒ग्नि रि॑षि॒त इ॑षि॒तो अ॒ग्नि र॒ग्नि रि॑षि॒तः । \newline
32. इ॒षि॒तो ऽवावे॑ षि॒त इ॑षि॒तो ऽव॑ । \newline
33. अव॑ सर्पतु सर्प॒ त्ववाव॑ सर्पतु । \newline
34. स॒र्प॒त्विति॑ सर्पतु । \newline
35. अग्ने॒ यद् यदग्ने ऽग्ने॒ यत् । \newline
36. यत् ते॑ ते॒ यद् यत् ते᳚ । \newline
37. ते॒ शु॒क्रꣳ शु॒क्रम् ते॑ ते शु॒क्रम् । \newline
38. शु॒क्रं ॅयद् यच्छु॒क्रꣳ शु॒क्रं ॅयत् । \newline
39. यच् च॒न्द्रम् च॒न्द्रं ॅयद् यच् च॒न्द्रम् । \newline
40. च॒न्द्रं ॅयद् यच् च॒न्द्रम् च॒न्द्रं ॅयत् । \newline
41. यत् पू॒तम् पू॒तं ॅयद् यत् पू॒तम् । \newline
42. पू॒तं ॅयद् यत् पू॒तम् पू॒तं ॅयत् । \newline
43. यद् य॒ज्ञियं॑ ॅय॒ज्ञियं॒ ॅयद् यद् य॒ज्ञिय᳚म् । \newline
44. य॒ज्ञिय॒मिति॑ य॒ज्ञिय᳚म् । \newline
45. तद् दे॒वेभ्यो॑ दे॒वेभ्य॒ स्तत् तद् दे॒वेभ्यः॑ । \newline
46. दे॒वेभ्यो॑ भरामसि भरामसि दे॒वेभ्यो॑ दे॒वेभ्यो॑ भरामसि । \newline
47. भ॒रा॒म॒सीति॑ भरामसि । \newline
48. इष॒ मूर्ज॒ मूर्ज॒ मिष॒ मिष॒ मूर्ज᳚म् । \newline
49. ऊर्ज॑ म॒ह म॒ह मूर्ज॒ मूर्ज॑ म॒हम् । \newline
50. अ॒ह मि॒त इ॒तो॑ ऽह म॒ह मि॒तः । \newline
51. इ॒त एत इ॒त आ । \newline
52. आ द॑दे दद॒ आ द॑दे । \newline

\textbf{Ghana Paata } \newline

1. मा नो॑ नो॒ मा मा नो॑ हिꣳसी द्धिꣳसीन् नो॒ मा मा नो॑ हिꣳसीत् । \newline
2. नो॒ हिꣳ॒॒सी॒ द्धिꣳ॒॒सी॒न् नो॒ नो॒ हिꣳ॒॒सी॒ज् ज॒नि॒ता ज॑नि॒ता हिꣳ॑सीन् नो नो हिꣳसीज् जनि॒ता । \newline
3. हिꣳ॒॒सी॒ज् ज॒नि॒ता ज॑नि॒ता हिꣳ॑सी द्धिꣳसीज् जनि॒ता यो यो ज॑नि॒ता हिꣳ॑सी द्धिꣳसीज् जनि॒ता यः । \newline
4. ज॒नि॒ता यो यो ज॑नि॒ता ज॑नि॒ता यः पृ॑थि॒व्याः पृ॑थि॒व्या यो ज॑नि॒ता ज॑नि॒ता यः पृ॑थि॒व्याः । \newline
5. यः पृ॑थि॒व्याः पृ॑थि॒व्या यो यः पृ॑थि॒व्या यो यः पृ॑थि॒व्या यो यः पृ॑थि॒व्या यः । \newline
6. पृ॒थि॒व्या यो यः पृ॑थि॒व्याः पृ॑थि॒व्या यो वा॑ वा॒ यः पृ॑थि॒व्याः पृ॑थि॒व्या यो वा᳚ । \newline
7. यो वा॑ वा॒ यो यो वा॒ दिव॒म् दिवं॑ ॅवा॒ यो यो वा॒ दिव᳚म् । \newline
8. वा॒ दिव॒म् दिवं॑ ॅवा वा॒ दिवꣳ॑ स॒त्यध॑र्मा स॒त्यध॑र्मा॒ दिवं॑ ॅवा वा॒ दिवꣳ॑ स॒त्यध॑र्मा । \newline
9. दिवꣳ॑ स॒त्यध॑र्मा स॒त्यध॑र्मा॒ दिव॒म् दिवꣳ॑ स॒त्यध॑र्मा ज॒जान॑ ज॒जान॑ स॒त्यध॑र्मा॒ दिव॒म् दिवꣳ॑ स॒त्यध॑र्मा ज॒जान॑ । \newline
10. स॒त्यध॑र्मा ज॒जान॑ ज॒जान॑ स॒त्यध॑र्मा स॒त्यध॑र्मा ज॒जान॑ । \newline
11. स॒त्यध॒र्मेति॑ स॒त्य - ध॒र्मा॒ । \newline
12. ज॒जानेति॑ ज॒जान॑ । \newline
13. यश्च॑ च॒ यो यश्चा॒पो अ॒पश्च॒ यो यश्चा॒पः । \newline
14. चा॒पो अ॒पश्च॑ चा॒प श्च॒न्द्रा श्च॒न्द्रा अ॒पश्च॑ चा॒प श्च॒न्द्राः । \newline
15. अ॒प श्च॒न्द्रा श्च॒न्द्रा अ॒पो अ॒प श्च॒न्द्रा बृ॑ह॒तीर् बृ॑ह॒ती श्च॒न्द्रा अ॒पो अ॒प श्च॒न्द्रा बृ॑ह॒तीः । \newline
16. च॒न्द्रा बृ॑ह॒तीर् बृ॑ह॒ती श्च॒न्द्रा श्च॒न्द्रा बृ॑ह॒तीर् ज॒जान॑ ज॒जान॑ बृह॒ती श्च॒न्द्रा श्च॒न्द्रा बृ॑ह॒तीर् ज॒जान॑ । \newline
17. बृ॒ह॒तीर् ज॒जान॑ ज॒जान॑ बृह॒तीर् बृ॑ह॒तीर् ज॒जान॒ कस्मै॒ कस्मै॑ ज॒जान॑ बृह॒तीर् बृ॑ह॒तीर् ज॒जान॒ कस्मै᳚ । \newline
18. ज॒जान॒ कस्मै॒ कस्मै॑ ज॒जान॑ ज॒जान॒ कस्मै॑ दे॒वाय॑ दे॒वाय॒ कस्मै॑ ज॒जान॑ ज॒जान॒ कस्मै॑ दे॒वाय॑ । \newline
19. कस्मै॑ दे॒वाय॑ दे॒वाय॒ कस्मै॒ कस्मै॑ दे॒वाय॑ ह॒विषा॑ ह॒विषा॑ दे॒वाय॒ कस्मै॒ कस्मै॑ दे॒वाय॑ ह॒विषा᳚ । \newline
20. दे॒वाय॑ ह॒विषा॑ ह॒विषा॑ दे॒वाय॑ दे॒वाय॑ ह॒विषा॑ विधेम विधेम ह॒विषा॑ दे॒वाय॑ दे॒वाय॑ ह॒विषा॑ विधेम । \newline
21. ह॒विषा॑ विधेम विधेम ह॒विषा॑ ह॒विषा॑ विधेम । \newline
22. वि॒धे॒मेति॑ विधेम । \newline
23. अ॒भ्याव॑र्तस्व पृथिवि पृथि व्य॒भ्याव॑र्तस्वा॒ भ्याव॑र्तस्व पृथिवि य॒ज्ञेन॑ य॒ज्ञेन॑ पृथि व्य॒भ्याव॑र्तस्वा॒ भ्याव॑र्तस्व पृथिवि य॒ज्ञेन॑ । \newline
24. अ॒भ्याव॑र्त॒स्वेत्य॑भि - आव॑र्तस्व । \newline
25. पृ॒थि॒वि॒ य॒ज्ञेन॑ य॒ज्ञेन॑ पृथिवि पृथिवि य॒ज्ञेन॒ पय॑सा॒ पय॑सा य॒ज्ञेन॑ पृथिवि पृथिवि य॒ज्ञेन॒ पय॑सा । \newline
26. य॒ज्ञेन॒ पय॑सा॒ पय॑सा य॒ज्ञेन॑ य॒ज्ञेन॒ पय॑सा स॒ह स॒ह पय॑सा य॒ज्ञेन॑ य॒ज्ञेन॒ पय॑सा स॒ह । \newline
27. पय॑सा स॒ह स॒ह पय॑सा॒ पय॑सा स॒ह । \newline
28. स॒हेति॑ स॒ह । \newline
29. व॒पाम् ते॑ ते व॒पां ॅव॒पाम् ते॑ अ॒ग्नि र॒ग्नि स्ते॑ व॒पां ॅव॒पाम् ते॑ अ॒ग्निः । \newline
30. ते॒ अ॒ग्नि र॒ग्नि स्ते॑ ते अ॒ग्नि रि॑षि॒त इ॑षि॒तो अ॒ग्नि स्ते॑ ते अ॒ग्नि रि॑षि॒तः । \newline
31. अ॒ग्नि रि॑षि॒त इ॑षि॒तो अ॒ग्नि र॒ग्नि रि॑षि॒तो ऽवावे॑ षि॒तो अ॒ग्नि र॒ग्नि रि॑षि॒तो ऽव॑ । \newline
32. इ॒षि॒तो ऽवावे॑ षि॒त इ॑षि॒तो ऽव॑ सर्पतु सर्प॒त्ववे॑ षि॒त इ॑षि॒तो ऽव॑ सर्पतु । \newline
33. अव॑ सर्पतु सर्प॒त्ववाव॑ सर्पतु । \newline
34. स॒र्प॒त्विति॑ सर्पतु । \newline
35. अग्ने॒ यद् यदग्ने ऽग्ने॒ यत् ते॑ ते॒ यदग्ने ऽग्ने॒ यत् ते᳚ । \newline
36. यत् ते॑ ते॒ यद् यत् ते॑ शु॒क्रꣳ शु॒क्रम् ते॒ यद् यत् ते॑ शु॒क्रम् । \newline
37. ते॒ शु॒क्रꣳ शु॒क्रम् ते॑ ते शु॒क्रं ॅयद् यच्छु॒क्रम् ते॑ ते शु॒क्रं ॅयत् । \newline
38. शु॒क्रं ॅयद् यच्छु॒क्रꣳ शु॒क्रं ॅयच् च॒न्द्रम् च॒न्द्रं ॅयच्छु॒क्रꣳ शु॒क्रं ॅयच् च॒न्द्रम् । \newline
39. यच् च॒न्द्रम् च॒न्द्रं ॅयद् यच् च॒न्द्रं ॅयद् यच् च॒न्द्रं ॅयद् यच् च॒न्द्रं ॅयत् । \newline
40. च॒न्द्रं ॅयद् यच् च॒न्द्रम् च॒न्द्रं ॅयत् पू॒तम् पू॒तं ॅयच् च॒न्द्रम् च॒न्द्रं ॅयत् पू॒तम् । \newline
41. यत् पू॒तम् पू॒तं ॅयद् यत् पू॒तं ॅयद् यत् पू॒तं ॅयद् यत् पू॒तं ॅयत् । \newline
42. पू॒तं ॅयद् यत् पू॒तम् पू॒तं ॅयद् य॒ज्ञियं॑ ॅय॒ज्ञियं॒ ॅयत् पू॒तम् पू॒तं ॅयद् य॒ज्ञिय᳚म् । \newline
43. यद् य॒ज्ञियं॑ ॅय॒ज्ञियं॒ ॅयद् यद् य॒ज्ञिय᳚म् । \newline
44. य॒ज्ञिय॒मिति॑ य॒ज्ञिय᳚म् । \newline
45. तद् दे॒वेभ्यो॑ दे॒वेभ्य॒ स्तत् तद् दे॒वेभ्यो॑ भरामसि भरामसि दे॒वेभ्य॒ स्तत् तद् दे॒वेभ्यो॑ भरामसि । \newline
46. दे॒वेभ्यो॑ भरामसि भरामसि दे॒वेभ्यो॑ दे॒वेभ्यो॑ भरामसि । \newline
47. भ॒रा॒म॒सीति॑ भरामसि । \newline
48. इष॒ मूर्ज॒ मूर्ज॒ मिष॒ मिष॒ मूर्ज॑ म॒ह म॒ह मूर्ज॒ मिष॒ मिष॒ मूर्ज॑ म॒हम् । \newline
49. ऊर्ज॑ म॒ह म॒ह मूर्ज॒ मूर्ज॑ म॒ह मि॒त इ॒तो॑ ऽह मूर्ज॒ मूर्ज॑ म॒ह मि॒तः । \newline
50. अ॒ह मि॒त इ॒तो॑ ऽह म॒ह मि॒त एतो॑ ऽह म॒ह मि॒त आ । \newline
51. इ॒त एत इ॒त आ द॑दे दद॒ एत इ॒त आ द॑दे । \newline
52. आ द॑दे दद॒ आ द॑द ऋ॒तस्य॒ र्‌तस्य॑ दद॒ आ द॑द ऋ॒तस्य॑ । \newline
\pagebreak
\markright{ TS 4.2.7.2  \hfill https://www.vedavms.in \hfill}

\section{ TS 4.2.7.2 }

\textbf{TS 4.2.7.2 } \newline
\textbf{Samhita Paata} \newline

द॑द ऋ॒तस्य॒ धाम्नो॑ अ॒मृत॑स्य॒ योनेः᳚ । आ नो॒ गोषु॑ विश॒त्वौष॑धीषु॒ जहा॑मि से॒दिमनि॑रा॒ममी॑वां ॥ अग्ने॒ तव॒ श्रवो॒ वयो॒ महि॑ भ्राजन्त्य॒र्चयो॑ विभावसो ।बृह॑द्-भानो॒ शव॑सा॒ वाज॑मु॒क्थ्यं॑ दधा॑सि दा॒शुषे॑ कवे ॥ इ॒र॒ज्यन्न॑ग्ने प्रथयस्व ज॒न्तुभि॑र॒स्मे रायो॑ अमर्त्य । स द॑र्.श॒तस्य॒ वपु॑षो॒ वि रा॑जसि पृ॒णक्षि॑ सान॒सिꣳ र॒यिं ॥ ऊर्जो॑ नपा॒ज्जात॑वेदः सुश॒स्तिभि॒-र्मन्द॑स्व - [  ] \newline

\textbf{Pada Paata} \newline

द॒दे॒ । ऋ॒तस्य॑ । धाम्नः॑ । अ॒मृत॑स्य । योनेः᳚ ॥ एति॑ । नः॒ । गोषु॑ । वि॒श॒तु॒ । एति॑ । ओष॑धीषु । जहा॑मि । से॒दिम् । अनि॑राम् । अमी॑वाम् ॥ अग्ने᳚ । तव॑ । श्रवः॑ । वयः॑ । महि॑ । भ्रा॒ज॒न्ति॒ । अ॒र्चयः॑ । वि॒भा॒व॒सो॒ इति॑ विभा - व॒सो॒ ॥ बृह॑द्भानो॒ इति॒ बृह॑त् - भा॒नो॒ । शव॑सा । वाज᳚म् । उ॒क्थ्य᳚म् । दधा॑सि । दा॒शुषे᳚ । क॒वे॒ ॥ इ॒र॒ज्यन्न् । अ॒ग्ने॒ । प्र॒थ॒य॒स्व॒ । ज॒न्तुभि॒रिति॑ ज॒न्तु-भिः॒ । अ॒स्मे इति॑ । रायः॑ । अ॒म॒र्त्य॒ ॥ सः । द॒र्॒.श॒तस्य॑ । वपु॑षः । वीति॑ । रा॒ज॒सि॒ । पृ॒णक्षि॑ । सा॒न॒सिम् । र॒यिम् ॥ ऊर्जः॑ । न॒पा॒त् । जात॑वेद॒ इति॒ जात॑ - वे॒दः॒ । सु॒श॒स्तिभि॒रिति॑ सुश॒स्ति - भिः॒ । मन्द॑स्व ।  \newline


\textbf{Krama Paata} \newline

द॒द॒ ऋ॒तस्य॑ । ऋ॒तस्य॒ धाम्नः॑ । धाम्नो॑ अ॒मृत॑स्य । अ॒मृत॑स्य॒ योनेः᳚ । योने॒रिति॒ योनेः᳚ ॥ आ नः॑ । नो॒ गोषु॑ । गोषु॑ विशतु । वि॒श॒त्वा । औष॑धीषु । ओष॑धीषु॒ जहा॑मि । जहा॑मि से॒दिम् । से॒दिमनि॑राम् । अनि॑रा॒ममी॑वाम् । अमी॑वा॒मित्यमी॑वाम् ॥ अग्ने॒ तव॑ । तव॒ श्रवः॑ । श्रवो॒ वयः॑ । वयो॒ महि॑ । महि॑ भ्राजन्ति । भ्रा॒ज॒न्त्य॒र्चयः॑ । अ॒र्चयो॑ विभावसो । वि॒भा॒व॒सो॒ इति॑ विभा - व॒सो॒ ॥ बृह॑द्भानो॒ शव॑सा । बृह॑द्भानो॒ इति॒ बृह॑त् - भा॒नो॒ । शव॑सा॒ वाज᳚म् । वाज॑मु॒क्थ्य᳚म् । उ॒क्थ्य॑म् दधा॑सि । दधा॑सि दा॒शुषे᳚ । दा॒शुषे॑ कवे । क॒व॒ इति॑ कवे ॥ इ॒र॒ज्यन्न॑ग्ने । अ॒ग्ने॒ प्र॒थ॒य॒स्व॒ । प्र॒थ॒य॒स्व॒ ज॒न्तुभिः॑ । ज॒न्तुभि॑र॒स्मे । ज॒न्तुभि॒रिति॑ ज॒न्तु - भिः॒ । अ॒स्मे रायः॑ । अ॒स्मे इत्य॒स्मे । रायो॑ अमर्त्य । अ॒म॒र्त्येत्य॑मर्त्य ॥ स द॑र्.श॒तस्य॑ । द॒र्॒.श॒तस्य॒ वपु॑षः । वपु॑षो॒ वि । वि रा॑जसि । रा॒ज॒सि॒ पृ॒णक्षि॑ । पृ॒णक्षि॑ सान॒सिम् । सा॒न॒सिꣳ र॒यिम् । र॒यिमिति॑ र॒यिम् ॥ ऊर्जो॑ नपात् । न॒पा॒ज् जात॑वेदः । जात॑वेदः सुश॒स्तिभिः॑ । जात॑वेद॒ इति॒ जात॑ - वे॒दः॒ । सु॒श॒स्तिभि॒र् मन्द॑स्व । सु॒श॒स्तिभि॒रिति॑ सुश॒स्ति - भिः॒ । मन्द॑स्व धी॒तिभिः॑ \newline

\textbf{Jatai Paata} \newline

1. द॒द॒ ऋ॒तस्य॒ र्‌तस्य॑ ददे दद ऋ॒तस्य॑ । \newline
2. ऋ॒तस्य॒ धाम्नो॒ धाम्न॑ ऋ॒तस्य॒ र्‌तस्य॒ धाम्नः॑ । \newline
3. धाम्नो॑ अ॒मृत॑स्या॒ मृत॑स्य॒ धाम्नो॒ धाम्नो॑ अ॒मृत॑स्य । \newline
4. अ॒मृत॑स्य॒ योने॒र् योने॑ र॒मृत॑स्या॒ मृत॑स्य॒ योनेः᳚ । \newline
5. योने॒रिति॒ योनेः᳚ । \newline
6. आ नो॑ न॒ आ नः॑ । \newline
7. नो॒ गोषु॒ गोषु॑ नो नो॒ गोषु॑ । \newline
8. गोषु॑ विशतु विशतु॒ गोषु॒ गोषु॑ विशतु । \newline
9. वि॒श॒ त्वा वि॑शतु विश॒ त्वा । \newline
10. औष॑धी॒ ष्वोष॑धी॒ ष्वौष॑धीषु । \newline
11. ओष॑धीषु॒ जहा॑मि॒ जहा॒ म्योष॑धी॒ ष्वोष॑धीषु॒ जहा॑मि । \newline
12. जहा॑मि से॒दिꣳ से॒दिम् जहा॑मि॒ जहा॑मि से॒दिम् । \newline
13. से॒दि मनि॑रा॒ मनि॑राꣳ से॒दिꣳ से॒दि मनि॑राम् । \newline
14. अनि॑रा॒ ममी॑वा॒ ममी॑वा॒ मनि॑रा॒ मनि॑रा॒ ममी॑वाम् । \newline
15. अमी॑वा॒मित्यमी॑वाम् । \newline
16. अग्ने॒ तव॒ तवाग्ने ऽग्ने॒ तव॑ । \newline
17. तव॒ श्रवः॒ श्रव॒ स्तव॒ तव॒ श्रवः॑ । \newline
18. श्रवो॒ वयो॒ वयः॒ श्रवः॒ श्रवो॒ वयः॑ । \newline
19. वयो॒ महि॒ महि॒ वयो॒ वयो॒ महि॑ । \newline
20. महि॑ भ्राजन्ति भ्राजन्ति॒ महि॒ महि॑ भ्राजन्ति । \newline
21. भ्रा॒ज॒ न्त्य॒र्चयो॑ अ॒र्चयो᳚ भ्राजन्ति भ्राज न्त्य॒र्चयः॑ । \newline
22. अ॒र्चयो॑ विभावसो विभावसो अ॒र्चयो॑ अ॒र्चयो॑ विभावसो । \newline
23. वि॒भा॒व॒सो॒ इति॑ विभा - व॒सो॒ । \newline
24. बृह॑द्भानो॒ शव॑सा॒ शव॑सा॒ बृह॑द्भानो॒ बृह॑द्भानो॒ शव॑सा । \newline
25. बृह॑द्भानो॒ इति॒ बृह॑त् - भा॒नो॒ । \newline
26. शव॑सा॒ वाजं॒ ॅवाजꣳ॒॒ शव॑सा॒ शव॑सा॒ वाज᳚म् । \newline
27. वाज॑ मु॒क्थ्य॑ मु॒क्थ्यं॑ ॅवाजं॒ ॅवाज॑ मु॒क्थ्य᳚म् । \newline
28. उ॒क्थ्य॑म् दधा॑सि॒ दधा᳚स्यु॒क्थ्य॑ मु॒क्थ्य॑म् दधा॑सि । \newline
29. दधा॑सि दा॒शुषे॑ दा॒शुषे॒ दधा॑सि॒ दधा॑सि दा॒शुषे᳚ । \newline
30. दा॒शुषे॑ कवे कवे दा॒शुषे॑ दा॒शुषे॑ कवे । \newline
31. क॒व॒ इति॑ कवे । \newline
32. इ॒र॒ज्यन् न॑ग्ने अग्न इर॒ज्यन् नि॑र॒ज्यन् न॑ग्ने । \newline
33. अ॒ग्ने॒ प्र॒थ॒य॒स्व॒ प्र॒थ॒य॒स्वा॒ग्ने॒ अ॒ग्ने॒ प्र॒थ॒य॒स्व॒ । \newline
34. प्र॒थ॒य॒स्व॒ ज॒न्तुभि॑र् ज॒न्तुभिः॑ प्रथयस्व प्रथयस्व ज॒न्तुभिः॑ । \newline
35. ज॒न्तुभि॑ र॒स्मे अ॒स्मे ज॒न्तुभि॑र् ज॒न्तुभि॑ र॒स्मे । \newline
36. ज॒न्तुभि॒रिति॑ ज॒न्तु - भिः॒ । \newline
37. अ॒स्मे रायो॒ रायो॑ अ॒स्मे अ॒स्मे रायः॑ । \newline
38. अ॒स्मे इत्य॒स्मे । \newline
39. रायो॑ अमर्त्या मर्त्य॒ रायो॒ रायो॑ अमर्त्य । \newline
40. अ॒म॒र्त्येत्य॑मर्त्य । \newline
41. स द॑र्.श॒तस्य॑ दर्.श॒तस्य॒ स स द॑र्.श॒तस्य॑ । \newline
42. द॒र्॒.श॒तस्य॒ वपु॑षो॒ वपु॑षो दर्.श॒तस्य॑ दर्.श॒तस्य॒ वपु॑षः । \newline
43. वपु॑षो॒ वि वि वपु॑षो॒ वपु॑षो॒ वि । \newline
44. वि रा॑जसि राजसि॒ वि वि रा॑जसि । \newline
45. रा॒ज॒सि॒ पृ॒णक्षि॑ पृ॒णक्षि॑ राजसि राजसि पृ॒णक्षि॑ । \newline
46. पृ॒णक्षि॑ सान॒सिꣳ सा॑न॒सिम् पृ॒णक्षि॑ पृ॒णक्षि॑ सान॒सिम् । \newline
47. सा॒न॒सिꣳ र॒यिꣳ र॒यिꣳ सा॑न॒सिꣳ सा॑न॒सिꣳ र॒यिम् । \newline
48. र॒यिमिति॑ र॒यिम् । \newline
49. ऊर्जो॑ नपान् नपा॒ दूर्ज॒ ऊर्जो॑ नपात् । \newline
50. न॒पा॒ज् जात॑वेदो॒ जात॑वेदो नपान् नपा॒ज् जात॑वेदः । \newline
51. जात॑वेदः सुश॒स्तिभिः॑ सुश॒स्तिभि॒र् जात॑वेदो॒ जात॑वेदः सुश॒स्तिभिः॑ । \newline
52. जात॑वेद॒ इति॒ जात॑ - वे॒दः॒ । \newline
53. सु॒श॒स्तिभि॒र् मन्द॑स्व॒ मन्द॑स्व सुश॒स्तिभिः॑ सुश॒स्तिभि॒र् मन्द॑स्व । \newline
54. सु॒श॒स्तिभि॒रिति॑ सुश॒स्ति - भिः॒ । \newline
55. मन्द॑स्व धी॒तिभि॑र् धी॒तिभि॒र् मन्द॑स्व॒ मन्द॑स्व धी॒तिभिः॑ । \newline

\textbf{Ghana Paata } \newline

1. द॒द॒ ऋ॒तस्य॒ र्‌तस्य॑ ददे दद ऋ॒तस्य॒ धाम्नो॒ धाम्न॑ ऋ॒तस्य॑ ददे दद ऋ॒तस्य॒ धाम्नः॑ । \newline
2. ऋ॒तस्य॒ धाम्नो॒ धाम्न॑ ऋ॒तस्य॒ र्‌तस्य॒ धाम्नो॑ अ॒मृत॑स्या॒ मृत॑स्य॒ धाम्न॑ ऋ॒तस्य॒ र्‌तस्य॒ धाम्नो॑ अ॒मृत॑स्य । \newline
3. धाम्नो॑ अ॒मृत॑स्या॒ मृत॑स्य॒ धाम्नो॒ धाम्नो॑ अ॒मृत॑स्य॒ योने॒र् योने॑ र॒मृत॑स्य॒ धाम्नो॒ धाम्नो॑ अ॒मृत॑स्य॒ योनेः᳚ । \newline
4. अ॒मृत॑स्य॒ योने॒र् योने॑ र॒मृत॑स्या॒ मृत॑स्य॒ योनेः᳚ । \newline
5. योने॒रिति॒ योनेः᳚ । \newline
6. आ नो॑ न॒ आ नो॒ गोषु॒ गोषु॑ न॒ आ नो॒ गोषु॑ । \newline
7. नो॒ गोषु॒ गोषु॑ नो नो॒ गोषु॑ विशतु विशतु॒ गोषु॑ नो नो॒ गोषु॑ विशतु । \newline
8. गोषु॑ विशतु विशतु॒ गोषु॒ गोषु॑ विश॒त्वा वि॑शतु॒ गोषु॒ गोषु॑ विश॒त्वा । \newline
9. वि॒श॒त्वा वि॑शतु विश॒ त्वौष॑धी॒ ष्वोष॑धी॒ष्वा वि॑शतु विश॒ त्वौष॑धीषु । \newline
10. औष॑धी॒ ष्वोष॑धी॒ ष्वौष॑धीषु॒ जहा॑मि॒ जहा॒ म्योष॑धी॒ ष्वौष॑धीषु॒ जहा॑मि । \newline
11. ओष॑धीषु॒ जहा॑मि॒ जहा॒ म्योष॑धी॒ ष्वोष॑धीषु॒ जहा॑मि से॒दिꣳ से॒दिम् जहा॒ म्योष॑धी॒ ष्वोष॑धीषु॒ जहा॑मि से॒दिम् । \newline
12. जहा॑मि से॒दिꣳ से॒दिम् जहा॑मि॒ जहा॑मि से॒दि मनि॑रा॒ मनि॑राꣳ से॒दिम् जहा॑मि॒ जहा॑मि से॒दि मनि॑राम् । \newline
13. से॒दि मनि॑रा॒ मनि॑राꣳ से॒दिꣳ से॒दि मनि॑रा॒ ममी॑वा॒ ममी॑वा॒ मनि॑राꣳ से॒दिꣳ से॒दि मनि॑रा॒ ममी॑वाम् । \newline
14. अनि॑रा॒ ममी॑वा॒ ममी॑वा॒ मनि॑रा॒ मनि॑रा॒ ममी॑वाम् । \newline
15. अमी॑वा॒मित्यमी॑वाम् । \newline
16. अग्ने॒ तव॒ तवाग्ने ऽग्ने॒ तव॒ श्रवः॒ श्रव॒ स्तवाग्ने ऽग्ने॒ तव॒ श्रवः॑ । \newline
17. तव॒ श्रवः॒ श्रव॒ स्तव॒ तव॒ श्रवो॒ वयो॒ वयः॒ श्रव॒ स्तव॒ तव॒ श्रवो॒ वयः॑ । \newline
18. श्रवो॒ वयो॒ वयः॒ श्रवः॒ श्रवो॒ वयो॒ महि॒ महि॒ वयः॒ श्रवः॒ श्रवो॒ वयो॒ महि॑ । \newline
19. वयो॒ महि॒ महि॒ वयो॒ वयो॒ महि॑ भ्राजन्ति भ्राजन्ति॒ महि॒ वयो॒ वयो॒ महि॑ भ्राजन्ति । \newline
20. महि॑ भ्राजन्ति भ्राजन्ति॒ महि॒ महि॑ भ्राज न्त्य॒र्चयो॑ अ॒र्चयो᳚ भ्राजन्ति॒ महि॒ महि॑ भ्राज न्त्य॒र्चयः॑ । \newline
21. भ्रा॒ज॒ न्त्य॒र्चयो॑ अ॒र्चयो᳚ भ्राजन्ति भ्राज न्त्य॒र्चयो॑ विभावसो विभावसो अ॒र्चयो᳚ भ्राजन्ति भ्राज न्त्य॒र्चयो॑ विभावसो । \newline
22. अ॒र्चयो॑ विभावसो विभावसो अ॒र्चयो॑ अ॒र्चयो॑ विभावसो । \newline
23. वि॒भा॒व॒सो॒ इति॑ विभा - व॒सो॒ । \newline
24. बृह॑द्भानो॒ शव॑सा॒ शव॑सा॒ बृह॑द्भानो॒ बृह॑द्भानो॒ शव॑सा॒ वाजं॒ ॅवाजꣳ॒॒ शव॑सा॒ बृह॑द्भानो॒ बृह॑द्भानो॒ शव॑सा॒ वाज᳚म् । \newline
25. बृह॑द्भानो॒ इति॒ बृह॑त् - भा॒नो॒ । \newline
26. शव॑सा॒ वाजं॒ ॅवाजꣳ॒॒ शव॑सा॒ शव॑सा॒ वाज॑ मु॒क्थ्य॑ मु॒क्थ्यं॑ ॅवाजꣳ॒॒ शव॑सा॒ शव॑सा॒ वाज॑ मु॒क्थ्य᳚म् । \newline
27. वाज॑ मु॒क्थ्य॑ मु॒क्थ्यं॑ ॅवाजं॒ ॅवाज॑ मु॒क्थ्य॑म् दधा॑सि॒ दधा᳚ स्यु॒क्थ्यं॑ ॅवाजं॒ ॅवाज॑ मु॒क्थ्य॑म् दधा॑सि । \newline
28. उ॒क्थ्य॑म् दधा॑सि॒ दधा᳚ स्यु॒क्थ्य॑ मु॒क्थ्य॑म् दधा॑सि दा॒शुषे॑ दा॒शुषे॒ दधा᳚ स्यु॒क्थ्य॑ मु॒क्थ्य॑म् दधा॑सि दा॒शुषे᳚ । \newline
29. दधा॑सि दा॒शुषे॑ दा॒शुषे॒ दधा॑सि॒ दधा॑सि दा॒शुषे॑ कवे कवे दा॒शुषे॒ दधा॑सि॒ दधा॑सि दा॒शुषे॑ कवे । \newline
30. दा॒शुषे॑ कवे कवे दा॒शुषे॑ दा॒शुषे॑ कवे । \newline
31. क॒व॒ इति॑ कवे । \newline
32. इ॒र॒ज्यन् न॑ग्ने अग्न इर॒ज्यन् नि॑र॒ज्यन् न॑ग्ने प्रथयस्व प्रथयस्वाग्न इर॒ज्यन् नि॑र॒ज्यन् न॑ग्ने प्रथयस्व । \newline
33. अ॒ग्ने॒ प्र॒थ॒य॒स्व॒ प्र॒थ॒य॒स्वा॒ग्ने॒ अ॒ग्ने॒ प्र॒थ॒य॒स्व॒ ज॒न्तुभि॑र् ज॒न्तुभिः॑ प्रथयस्वाग्ने अग्ने प्रथयस्व ज॒न्तुभिः॑ । \newline
34. प्र॒थ॒य॒स्व॒ ज॒न्तुभि॑र् ज॒न्तुभिः॑ प्रथयस्व प्रथयस्व ज॒न्तुभि॑ र॒स्मे अ॒स्मे ज॒न्तुभिः॑ प्रथयस्व प्रथयस्व ज॒न्तुभि॑ र॒स्मे । \newline
35. ज॒न्तुभि॑ र॒स्मे अ॒स्मे ज॒न्तुभि॑र् ज॒न्तुभि॑ र॒स्मे रायो॒ रायो॑ अ॒स्मे ज॒न्तुभि॑र् ज॒न्तुभि॑ र॒स्मे रायः॑ । \newline
36. ज॒न्तुभि॒रिति॑ ज॒न्तु - भिः॒ । \newline
37. अ॒स्मे रायो॒ रायो॑ अ॒स्मे अ॒स्मे रायो॑ अमर्त्या मर्त्य॒ रायो॑ अ॒स्मे अ॒स्मे रायो॑ अमर्त्य । \newline
38. अ॒स्मे इत्य॒स्मे । \newline
39. रायो॑ अमर्त्या मर्त्य॒ रायो॒ रायो॑ अमर्त्य । \newline
40. अ॒म॒र्त्येत्य॑मर्त्य । \newline
41. स दर्॑.श॒तस्य॑ दर्.श॒तस्य॒ स स दर्॑.श॒तस्य॒ वपु॑षो॒ वपु॑षो दर्.श॒तस्य॒ स स दर्॑.श॒तस्य॒ वपु॑षः । \newline
42. द॒र्॒.श॒तस्य॒ वपु॑षो॒ वपु॑षो दर्.श॒तस्य॑ दर्.श॒तस्य॒ वपु॑षो॒ वि वि वपु॑षो दर्.श॒तस्य॑ दर्.श॒तस्य॒ वपु॑षो॒ वि । \newline
43. वपु॑षो॒ वि वि वपु॑षो॒ वपु॑षो॒ वि रा॑जसि राजसि॒ वि वपु॑षो॒ वपु॑षो॒ वि रा॑जसि । \newline
44. वि रा॑जसि राजसि॒ वि वि रा॑जसि पृ॒णक्षि॑ पृ॒णक्षि॑ राजसि॒ वि वि रा॑जसि पृ॒णक्षि॑ । \newline
45. रा॒ज॒सि॒ पृ॒णक्षि॑ पृ॒णक्षि॑ राजसि राजसि पृ॒णक्षि॑ सान॒सिꣳ सा॑न॒सिम् पृ॒णक्षि॑ राजसि राजसि पृ॒णक्षि॑ सान॒सिम् । \newline
46. पृ॒णक्षि॑ सान॒सिꣳ सा॑न॒सिम् पृ॒णक्षि॑ पृ॒णक्षि॑ सान॒सिꣳ र॒यिꣳ र॒यिꣳ सा॑न॒सिम् पृ॒णक्षि॑ पृ॒णक्षि॑ सान॒सिꣳ र॒यिम् । \newline
47. सा॒न॒सिꣳ र॒यिꣳ र॒यिꣳ सा॑न॒सिꣳ सा॑न॒सिꣳ र॒यिम् । \newline
48. र॒यिमिति॑ र॒यिम् । \newline
49. ऊर्जो॑ नपान् नपा॒दूर्ज॒ ऊर्जो॑ नपा॒ज् जात॑वेदो॒ जात॑वेदो नपा॒दूर्ज॒ ऊर्जो॑ नपा॒ज् जात॑वेदः । \newline
50. न॒पा॒ज् जात॑वेदो॒ जात॑वेदो नपान् नपा॒ज् जात॑वेदः सुश॒स्तिभिः॑ सुश॒स्तिभि॒र् जात॑वेदो नपान् नपा॒ज् जात॑वेदः सुश॒स्तिभिः॑ । \newline
51. जात॑वेदः सुश॒स्तिभिः॑ सुश॒स्तिभि॒र् जात॑वेदो॒ जात॑वेदः सुश॒स्तिभि॒र् मन्द॑स्व॒ मन्द॑स्व सुश॒स्तिभि॒र् जात॑वेदो॒ जात॑वेदः सुश॒स्तिभि॒र् मन्द॑स्व । \newline
52. जात॑वेद॒ इति॒ जात॑ - वे॒दः॒ । \newline
53. सु॒श॒स्तिभि॒र् मन्द॑स्व॒ मन्द॑स्व सुश॒स्तिभिः॑ सुश॒स्तिभि॒र् मन्द॑स्व धी॒तिभि॑र् धी॒तिभि॒र् मन्द॑स्व सुश॒स्तिभिः॑ सुश॒स्तिभि॒र् मन्द॑स्व धी॒तिभिः॑ । \newline
54. सु॒श॒स्तिभि॒रिति॑ सुश॒स्ति - भिः॒ । \newline
55. मन्द॑स्व धी॒तिभि॑र् धी॒तिभि॒र् मन्द॑स्व॒ मन्द॑स्व धी॒तिभि॑र् हि॒तो हि॒तो धी॒तिभि॒र् मन्द॑स्व॒ मन्द॑स्व धी॒तिभि॑र् हि॒तः । \newline
\pagebreak
\markright{ TS 4.2.7.3  \hfill https://www.vedavms.in \hfill}

\section{ TS 4.2.7.3 }

\textbf{TS 4.2.7.3 } \newline
\textbf{Samhita Paata} \newline

धी॒तिभि॑र्.हि॒तः । त्वे इषः॒ सं द॑धु॒-र्भूरि॑रेतस-श्चि॒त्रो त॑यो वा॒मजा॑ताः ॥ पा॒व॒कव॑र्चाः शु॒क्रव॑र्चा॒ अनू॑नवर्चा॒ उदि॑यर्.षि भा॒नुना᳚ । पु॒त्रः पि॒तरा॑ वि॒चर॒न्नुपा॑वस्यु॒भे पृ॑णक्षि॒ रोद॑सी ॥ ऋ॒तावा॑नं महि॒षं ॅवि॒श्वच॑र्.षणिम॒ग्निꣳ सु॒म्नाय॑ दधिरे पु॒रो जनाः᳚ । श्रुत्क॑र्णꣳ स॒प्रथ॑स्तमं त्वा गि॒रा दैव्यं॒ मानु॑षा यु॒गा ॥ नि॒ष्क॒र्तार॑-मद्ध्व॒रस्य॒ प्रचे॑तसं॒ क्षय॑न्तꣳ॒॒ राध॑से म॒हे । रा॒तिं भृगू॑णामु॒शिजं॑ क॒विक्र॑तुं पृ॒णक्षि॑ सान॒सिꣳ - [  ] \newline

\textbf{Pada Paata} \newline

धी॒तिभि॒रिति॑ धी॒ति-भिः॒ । हि॒तः ॥ त्वे इति॑ । इषः॑ । समिति॑ । द॒धुः॒ । भूरि॑रेतस॒ इति॒ भूरि॑ - रे॒त॒सः॒ । चि॒त्रोत॑य॒ इति॑ चि॒त्र - ऊ॒त॒यः॒ । वा॒मजा॑ता॒ इति॑ वा॒म - जा॒ताः॒ ॥ पा॒व॒कव॑र्चा॒ इति॑ पाव॒क - व॒र्चाः॒ । शु॒क्रव॑र्चा॒ इति॑ शु॒क्र - व॒र्चाः॒ । अनू॑नवर्चा॒ इत्यनू॑न - व॒र्चाः॒ । उदिति॑ । इ॒य॒र्॒.षि॒ । भा॒नुना᳚ ॥ पु॒त्रः । पि॒तरा᳚ । वि॒चर॒न्निति॑ वि-चरन्न्॑ । उपेति॑ । अ॒व॒सि॒ । उ॒भे इति॑ । पृ॒ण॒क्षि॒ । रोद॑सी॒ इति॑ ॥ ऋ॒तावा॑न॒मित्यृ॒त - वा॒न॒म् । म॒हि॒षम् । वि॒श्वच॑र्.षणि॒मिति॑ वि॒श्व - च॒र्॒.ष॒णि॒म् । अ॒ग्निम् । सु॒म्नाय॑ । द॒धि॒रे॒ । पु॒रः । जनाः᳚ ॥ श्रुत्क॑र्ण॒मिति॒ श्रुत्-क॒र्ण॒म् । स॒प्रथ॑स्तम॒मिति॑ स॒प्रथः॑-त॒म॒म् । त्वा॒ । गि॒रा । दैव्य᳚म् । मानु॑षा । यु॒गा ॥ नि॒ष्क॒र्तार॒मिति॑ निः - क॒र्तार᳚म् । अ॒द्ध्व॒रस्य॑ । प्रचे॑तस॒मिति॒ प्र - चे॒त॒स॒म् । क्षय॑न्तम् । राध॑से । म॒हे ॥ रा॒तिम् । भृगू॑णाम् । उ॒शिज᳚म् । क॒विक्र॑तु॒मिति॑ क॒वि - क्र॒तु॒म् । पृ॒णक्षि॑ । सा॒न॒सिम् ।  \newline


\textbf{Krama Paata} \newline

धी॒तिभि॑र् हि॒तः । धी॒तिभि॒रिति॑ धी॒ति - भिः॒ । हि॒त इति॑ हि॒तः ॥ त्वे इषः॑ । त्वे इति॒ त्वे । इषः॒ सम् । सम् द॑धुः । द॒धु॒र् भूरि॑रेतसः । भूरि॑रेतसश्चि॒त्रोत॑यः । भूरि॑रेतस॒ इति॒ भूरि॑ - रे॒त॒सः॒ । चि॒त्रोत॑यो वा॒मजा॑ताः । चि॒त्रोत॑य॒ इति॑ चि॒त्र - ऊ॒त॒यः॒ । वा॒मजा॑ता॒ इति॑ वा॒म - जा॒ताः॒ ॥ पा॒व॒कव॑र्चाः शु॒क्रव॑र्चाः । पा॒व॒कव॑र्चा॒ इति॑ पाव॒क - व॒र्चाः॒ । शु॒क्रव॑र्चा॒ अनू॑नवर्चाः । शु॒क्रव॑र्चा॒ इति॑ शु॒क्र - व॒र्चाः॒ । अनू॑नवर्चा॒ उत् । अनू॑नवर्चा॒ इत्यनू॑न - व॒र्चाः॒ । उदि॑यर्.षि । इ॒य॒र्॒.षि॒ भा॒नुना᳚ । भा॒नुनेति॑ भा॒नुना᳚ ॥ पु॒त्रः पि॒तरा᳚ । पि॒तरा॑ वि॒चरन्न्॑ । वि॒चर॒न्नुप॑ । वि॒चर॒न्निति॑ वि - चरन्न्॑ । उपा॑वसि । अ॒व॒स्यु॒भे । उ॒भे पृ॑णक्षि । उ॒भे इत्यु॒भे । पृ॒ण॒क्षि॒ रोद॑सी । रोद॑सी॒ इति॒ रोद॑सी ॥ ऋ॒तावा॑नम् महि॒षम् । ऋ॒तावा॑न॒मित्यृ॒त - वा॒न॒म् । म॒हि॒षं ॅवि॒श्वच॑र्.षणिम् । वि॒श्वच॑र्.षणिम॒ग्निम् । वि॒श्वच॑र्.षणि॒मिति॑ वि॒श्व - च॒र्॒.ष॒णि॒म् । अ॒ग्निꣳ सु॒म्नाय॑ । सु॒म्नाय॑ दधिरे । द॒धि॒रे॒ पु॒रः । पु॒रो जनाः᳚ । जना॒ इति॒ जनाः᳚ ॥ श्रुत्क॑र्णꣳ स॒प्रथ॑स्तमम् । श्रुत्क॑र्ण॒मिति॒ श्रुत् - क॒र्ण॒म् । स॒प्रथ॑स्तमम् त्वा । स॒प्रथ॑स्तम॒मिति॑ स॒प्रथः॑ - त॒म॒म् । त्वा॒ गि॒रा । गि॒रा दैव्य᳚म् । दैव्य॒म् मानु॑षा । मानु॑षा यु॒गा । यु॒गेति॑ यु॒गा ॥ नि॒ष्क॒र्तार॑मद्ध्व॒रस्य॑ । नि॒ष्क॒र्तार॒मिति॑ निः - क॒र्तार᳚म् । अ॒द्ध्व॒रस्य॒ प्रचे॑तसम् । प्रचे॑तस॒म् क्षय॑न्तम् । प्रचे॑तस॒मिति॒ प्र - चे॒त॒स॒म् । क्षय॑न्तꣳ॒॒ राध॑से । राध॑से म॒हे । म॒ह इति॑ म॒हे ॥ रा॒तिम् भृगू॑णाम् । भृगू॑णामु॒शिज᳚म् । उ॒शिज॑म् क॒विक्र॑तुम् । क॒विक्र॑तुम् पृ॒णक्षि॑ । क॒विक्र॑तु॒मिति॑ क॒वि - क्र॒तु॒म् । पृ॒णक्षि॑ सान॒सिम् । सा॒न॒सिꣳ र॒यिम् \newline

\textbf{Jatai Paata} \newline

1. धी॒तिभिर्॑. हि॒तो हि॒तो धी॒तिभि॑र् धी॒तिभि॑र् हि॒तः । \newline
2. धी॒तिभि॒रिति॑ धी॒ति - भिः॒ । \newline
3. हि॒त इति॑ हि॒तः । \newline
4. त्वे इष॒ इष॒ स्त्वे त्वे इषः॑ । \newline
5. त्वे इति॒ त्वे । \newline
6. इषः॒ सꣳ स मिष॒ इषः॒ सम् । \newline
7. सम् द॑धुर् दधुः॒ सꣳ सम् द॑धुः । \newline
8. द॒धु॒र् भूरि॑रेतसो॒ भूरि॑रेतसो दधुर् दधु॒र् भूरि॑रेतसः । \newline
9. भूरि॑रेतस श्चि॒त्रोत॑य श्चि॒त्रोत॑यो॒ भूरि॑रेतसो॒ भूरि॑रेतस श्चि॒त्रोत॑यः । \newline
10. भूरि॑रेतस॒ इति॒ भूरि॑ - रे॒त॒सः॒ । \newline
11. चि॒त्रोत॑यो वा॒मजा॑ता वा॒मजा॑ता श्चि॒त्रोत॑य श्चि॒त्रोत॑यो वा॒मजा॑ताः । \newline
12. चि॒त्रोत॑य॒ इति॑ चि॒त्र - ऊ॒त॒यः॒ । \newline
13. वा॒मजा॑ता॒ इति॑ वा॒म - जा॒ताः॒ । \newline
14. पा॒व॒कव॑र्चाः शु॒क्रव॑र्चाः शु॒क्रव॑र्चाः पाव॒कव॑र्चाः पाव॒कव॑र्चाः शु॒क्रव॑र्चाः । \newline
15. पा॒व॒कव॑र्चा॒ इति॑ पाव॒क - व॒र्चाः॒ । \newline
16. शु॒क्रव॑र्चा॒ अनू॑नवर्चा॒ अनू॑नवर्चाः शु॒क्रव॑र्चाः शु॒क्रव॑र्चा॒ अनू॑नवर्चाः । \newline
17. शु॒क्रव॑र्चा॒ इति॑ शु॒क्र - व॒र्चाः॒ । \newline
18. अनू॑नवर्चा॒ उदु दनू॑नवर्चा॒ अनू॑नवर्चा॒ उत् । \newline
19. अनू॑नवर्चा॒ इत्यनू॑न - व॒र्चाः॒ । \newline
20. उदि॑यर्.षी य॒र्॒. ष्यु दुदि॑यर्.षि । \newline
21. इ॒य॒र्॒.षि॒ भा॒नुना॑ भा॒नुने॑ यर्.षी यर्.षि भा॒नुना᳚ । \newline
22. भा॒नुनेति॑ भा॒नुना᳚ । \newline
23. पु॒त्रः पि॒तरा॑ पि॒तरा॑ पु॒त्रः पु॒त्रः पि॒तरा᳚ । \newline
24. पि॒तरा॑ वि॒चर॑न्. वि॒चर॑न् पि॒तरा॑ पि॒तरा॑ वि॒चरन्न्॑ । \newline
25. वि॒चर॒न् नुपोप॑ वि॒चर॑न्. वि॒चर॒न् नुप॑ । \newline
26. वि॒चर॒न्निति॑ वि - चरन्न्॑ । \newline
27. उपा॑व स्यव॒ स्युपोपा॑ वसि । \newline
28. अ॒व॒स्यु॒भे उ॒भे अ॑वस्य वस्यु॒भे । \newline
29. उ॒भे पृ॑णक्षि पृणक्ष्यु॒भे उ॒भे पृ॑णक्षि । \newline
30. उ॒भे इत्यु॒भे । \newline
31. पृ॒ण॒क्षि॒ रोद॑सी॒ रोद॑सी पृणक्षि पृणक्षि॒ रोद॑सी । \newline
32. रोद॑सी॒ इति॒ रोद॑सी । \newline
33. ऋ॒तावा॑नम् महि॒षम् म॑हि॒ष मृ॒तावा॑न मृ॒तावा॑नम् महि॒षम् । \newline
34. ऋ॒तावा॑न॒मित्यृ॒त - वा॒न॒म् । \newline
35. म॒हि॒षं ॅवि॒श्वच॑र्.षणिं ॅवि॒श्वच॑र्.षणिम् महि॒षम् म॑हि॒षं ॅवि॒श्वच॑र्.षणिम् । \newline
36. वि॒श्वच॑र्.षणि म॒ग्नि म॒ग्निं ॅवि॒श्वच॑र्.षणिं ॅवि॒श्वच॑र्.षणि म॒ग्निम् । \newline
37. वि॒श्वच॑र्.षणि॒मिति॑ वि॒श्व - च॒र्॒.ष॒णि॒म् । \newline
38. अ॒ग्निꣳ सु॒म्नाय॑ सु॒म्नाया॒ग्नि म॒ग्निꣳ सु॒म्नाय॑ । \newline
39. सु॒म्नाय॑ दधिरे दधिरे सु॒म्नाय॑ सु॒म्नाय॑ दधिरे । \newline
40. द॒धि॒रे॒ पु॒रः पु॒रो द॑धिरे दधिरे पु॒रः । \newline
41. पु॒रो जना॒ जनाः᳚ पु॒रः पु॒रो जनाः᳚ । \newline
42. जना॒ इति॒ जनाः᳚ । \newline
43. श्रुत्क॑र्णꣳ स॒प्रथ॑स्तमꣳ स॒प्रथ॑स्तमꣳ॒॒ श्रुत्क॑र्णꣳ॒॒ श्रुत्क॑र्णꣳ स॒प्रथ॑स्तमम् । \newline
44. श्रुत्क॑र्ण॒मिति॒ श्रुत् - क॒र्ण॒म् । \newline
45. स॒प्रथ॑स्तमम् त्वा त्वा स॒प्रथ॑स्तमꣳ स॒प्रथ॑स्तमम् त्वा । \newline
46. स॒प्रथ॑स्तम॒मिति॑ स॒प्रथः॑ - त॒म॒म् । \newline
47. त्वा॒ गि॒रा गि॒रा त्वा᳚ त्वा गि॒रा । \newline
48. गि॒रा दैव्य॒म् दैव्य॑म् गि॒रा गि॒रा दैव्य᳚म् । \newline
49. दैव्य॒म् मानु॑षा॒ मानु॑षा॒ दैव्य॒म् दैव्य॒म् मानु॑षा । \newline
50. मानु॑षा यु॒गा यु॒गा मानु॑षा॒ मानु॑षा यु॒गा । \newline
51. य॒गेति॑ यु॒गा । \newline
52. नि॒ष्क॒र्तार॑ मद्ध्व॒रस्या᳚ द्ध्व॒रस्य॑ निष्क॒र्तार॑म् निष्क॒र्तार॑ मद्ध्व॒रस्य॑ । \newline
53. नि॒ष्क॒र्तार॒मिति॑ निः - क॒र्तार᳚म् । \newline
54. अ॒द्ध्व॒रस्य॒ प्रचे॑तस॒म् प्रचे॑तस मद्ध्व॒रस्या᳚ द्ध्व॒रस्य॒ प्रचे॑तसम् । \newline
55. प्रचे॑तस॒म् क्षय॑न्त॒म् क्षय॑न्त॒म् प्रचे॑तस॒म् प्रचे॑तस॒म् क्षय॑न्तम् । \newline
56. प्रचे॑तस॒मिति॒ प्र - चे॒त॒स॒म् । \newline
57. क्षय॑न्तꣳ॒॒ राध॑से॒ राध॑से॒ क्षय॑न्त॒म् क्षय॑न्तꣳ॒॒ राध॑से । \newline
58. राध॑से म॒हे म॒हे राध॑से॒ राध॑से म॒हे । \newline
59. म॒ह इति॑ म॒हे । \newline
60. रा॒तिम् भृगू॑णा॒म् भृगू॑णाꣳ रा॒तिꣳ रा॒तिम् भृगू॑णाम् । \newline
61. भृगू॑णा मु॒शिज॑ मु॒शिज॒म् भृगू॑णा॒म् भृगू॑णा मु॒शिज᳚म् । \newline
62. उ॒शिज॑म् क॒विक्र॑तुम् क॒विक्र॑तु मु॒शिज॑ मु॒शिज॑म् क॒विक्र॑तुम् । \newline
63. क॒विक्र॑तुम् पृ॒णक्षि॑ पृ॒णक्षि॑ क॒विक्र॑तुम् क॒विक्र॑तुम् पृ॒णक्षि॑ । \newline
64. क॒विक्र॑तु॒मिति॑ क॒वि - क्र॒तु॒म् । \newline
65. पृ॒णक्षि॑ सान॒सिꣳ सा॑न॒सिम् पृ॒णक्षि॑ पृ॒णक्षि॑ सान॒सिम् । \newline
66. सा॒न॒सिꣳ र॒यिꣳ र॒यिꣳ सा॑न॒सिꣳ सा॑न॒सिꣳ र॒यिम् । \newline

\textbf{Ghana Paata } \newline

1. धी॒तिभि॑र् हि॒तो हि॒तो धी॒तिभि॑र् धी॒तिभि॑र् हि॒तः । \newline
2. धी॒तिभि॒रिति॑ धी॒ति - भिः॒ । \newline
3. हि॒त इति॑ हि॒तः । \newline
4. त्वे इष॒ इष॒ स्त्वे त्वे इषः॒ सꣳ स मिष॒ स्त्वे त्वे इषः॒ सम् । \newline
5. त्वे इति॒ त्वे । \newline
6. इषः॒ सꣳ स मिष॒ इषः॒ सम् द॑धुर् दधुः॒ स मिष॒ इषः॒ सम् द॑धुः । \newline
7. सम् द॑धुर् दधुः॒ सꣳ सम् द॑धु॒र् भूरि॑रेतसो॒ भूरि॑रेतसो दधुः॒ सꣳ सम् द॑धु॒र् भूरि॑रेतसः । \newline
8. द॒धु॒र् भूरि॑रेतसो॒ भूरि॑रेतसो दधुर् दधु॒र् भूरि॑रेतस श्चि॒त्रोत॑य श्चि॒त्रोत॑यो॒ भूरि॑रेतसो दधुर् दधु॒र् भूरि॑रेतस श्चि॒त्रोत॑यः । \newline
9. भूरि॑रेतस श्चि॒त्रोत॑य श्चि॒त्रोत॑यो॒ भूरि॑रेतसो॒ भूरि॑रेतस श्चि॒त्रोत॑यो वा॒मजा॑ता वा॒मजा॑ता श्चि॒त्रोत॑यो॒ भूरि॑रेतसो॒ भूरि॑रेतस श्चि॒त्रोत॑यो वा॒मजा॑ताः । \newline
10. भूरि॑रेतस॒ इति॒ भूरि॑ - रे॒त॒सः॒ । \newline
11. चि॒त्रोत॑यो वा॒मजा॑ता वा॒मजा॑ता श्चि॒त्रोत॑य श्चि॒त्रोत॑यो वा॒मजा॑ताः । \newline
12. चि॒त्रोत॑य॒ इति॑ चि॒त्र - ऊ॒त॒यः॒ । \newline
13. वा॒मजा॑ता॒ इति॑ वा॒म - जा॒ताः॒ । \newline
14. पा॒व॒कव॑र्चाः शु॒क्रव॑र्चाः शु॒क्रव॑र्चाः पाव॒कव॑र्चाः पाव॒कव॑र्चाः शु॒क्रव॑र्चा॒ अनू॑नवर्चा॒ अनू॑नवर्चाः शु॒क्रव॑र्चाः पाव॒कव॑र्चाः पाव॒कव॑र्चाः शु॒क्रव॑र्चा॒ अनू॑नवर्चाः । \newline
15. पा॒व॒कव॑र्चा॒ इति॑ पाव॒क - व॒र्चाः॒ । \newline
16. शु॒क्रव॑र्चा॒ अनू॑नवर्चा॒ अनू॑नवर्चाः शु॒क्रव॑र्चाः शु॒क्रव॑र्चा॒ अनू॑नवर्चा॒ उदु दनू॑नवर्चाः शु॒क्रव॑र्चाः शु॒क्रव॑र्चा॒ अनू॑नवर्चा॒ उत् । \newline
17. शु॒क्रव॑र्चा॒ इति॑ शु॒क्र - व॒र्चाः॒ । \newline
18. अनू॑नवर्चा॒ उदु दनू॑नवर्चा॒ अनू॑नवर्चा॒ उदि॑यर्.षी य॒र्॒.ष्यु दनू॑नवर्चा॒ अनू॑नवर्चा॒ उदि॑यर्.षि । \newline
19. अनू॑नवर्चा॒ इत्यनू॑न - व॒र्चाः॒ । \newline
20. उदि॑यर्.षी य॒र्॒.ष्यु दुदि॑यर्.षि भा॒नुना॑ भा॒नुने॑ य॒र्॒.ष्यु दुदि॑यर्.षि भा॒नुना᳚ । \newline
21. इ॒य॒र्॒.षि॒ भा॒नुना॑ भा॒नुने॑ यर्.षीयर्.षि भा॒नुना᳚ । \newline
22. भा॒नुनेति॑ भा॒नुना᳚ । \newline
23. पु॒त्रः पि॒तरा॑ पि॒तरा॑ पु॒त्रः पु॒त्रः पि॒तरा॑ वि॒चर॑न्. वि॒चर॑न् पि॒तरा॑ पु॒त्रः पु॒त्रः पि॒तरा॑ वि॒चरन्न्॑ । \newline
24. पि॒तरा॑ वि॒चर॑न्. वि॒चर॑न् पि॒तरा॑ पि॒तरा॑ वि॒चर॒न् नुपोप॑ वि॒चर॑न् पि॒तरा॑ पि॒तरा॑ वि॒चर॒न् नुप॑ । \newline
25. वि॒चर॒न् नुपोप॑ वि॒चर॑न्. वि॒चर॒न् नुपा॑व स्यव॒स्युप॑ वि॒चर॑न्. वि॒चर॒न् नुपा॑वसि । \newline
26. वि॒चर॒न्निति॑ वि - चरन्न्॑ । \newline
27. उपा॑वस्य व॒स्युपोपा॑ वस्यु॒भे उ॒भे अ॑व॒ स्युपोपा॑ वस्यु॒भे । \newline
28. अ॒व॒स्यु॒भे उ॒भे अ॑वस्य वस्यु॒भे पृ॑णक्षि पृणक्ष्यु॒भे अ॑वस्य वस्यु॒भे पृ॑णक्षि । \newline
29. उ॒भे पृ॑णक्षि पृणक्ष्यु॒भे उ॒भे पृ॑णक्षि॒ रोद॑सी॒ रोद॑सी पृणक्ष्यु॒भे उ॒भे पृ॑णक्षि॒ रोद॑सी । \newline
30. उ॒भे इत्यु॒भे । \newline
31. पृ॒ण॒क्षि॒ रोद॑सी॒ रोद॑सी पृणक्षि पृणक्षि॒ रोद॑सी । \newline
32. रोद॑सी॒ इति॒ रोद॑सी । \newline
33. ऋ॒तावा॑नम् महि॒षम् म॑हि॒ष मृ॒तावा॑न मृ॒तावा॑नम् महि॒षं ॅवि॒श्वच॑र्.षणिं ॅवि॒श्वच॑र्.षणिम् महि॒ष मृ॒तावा॑न मृ॒तावा॑नम् महि॒षं ॅवि॒श्वच॑र्.षणिम् । \newline
34. ऋ॒तावा॑न॒मित्यृ॒त - वा॒न॒म् । \newline
35. म॒हि॒षं ॅवि॒श्वच॑र्.षणिं ॅवि॒श्वच॑र्.षणिम् महि॒षम् म॑हि॒षं ॅवि॒श्वच॑र्.षणि म॒ग्नि म॒ग्निं ॅवि॒श्वच॑र्.षणिम् महि॒षम् म॑हि॒षं ॅवि॒श्वच॑र्.षणि म॒ग्निम् । \newline
36. वि॒श्वच॑र्.षणि म॒ग्नि म॒ग्निं ॅवि॒श्वच॑र्.षणिं ॅवि॒श्वच॑र्.षणि म॒ग्निꣳ सु॒म्नाय॑ सु॒म्नाया॒ग्निं ॅवि॒श्वच॑र्.षणिं ॅवि॒श्वच॑र्.षणि म॒ग्निꣳ सु॒म्नाय॑ । \newline
37. वि॒श्वच॑र्.षणि॒मिति॑ वि॒श्व - च॒र्॒.ष॒णि॒म् । \newline
38. अ॒ग्निꣳ सु॒म्नाय॑ सु॒म्नाया॒ग्नि म॒ग्निꣳ सु॒म्नाय॑ दधिरे दधिरे सु॒म्नाया॒ग्नि म॒ग्निꣳ सु॒म्नाय॑ दधिरे । \newline
39. सु॒म्नाय॑ दधिरे दधिरे सु॒म्नाय॑ सु॒म्नाय॑ दधिरे पु॒रः पु॒रो द॑धिरे सु॒म्नाय॑ सु॒म्नाय॑ दधिरे पु॒रः । \newline
40. द॒धि॒रे॒ पु॒रः पु॒रो द॑धिरे दधिरे पु॒रो जना॒ जनाः᳚ पु॒रो द॑धिरे दधिरे पु॒रो जनाः᳚ । \newline
41. पु॒रो जना॒ जनाः᳚ पु॒रः पु॒रो जनाः᳚ । \newline
42. जना॒ इति॒ जनाः᳚ । \newline
43. श्रुत्क॑र्णꣳ स॒प्रथ॑स्तमꣳ स॒प्रथ॑स्तमꣳ॒॒ श्रुत्क॑र्णꣳ॒॒ श्रुत्क॑र्णꣳ स॒प्रथ॑स्तमम् त्वा त्वा स॒प्रथ॑स्तमꣳ॒॒ श्रुत्क॑र्णꣳ॒॒ श्रुत्क॑र्णꣳ स॒प्रथ॑स्तमम् त्वा । \newline
44. श्रुत्क॑र्ण॒मिति॒ श्रुत् - क॒र्ण॒म् । \newline
45. स॒प्रथ॑स्तमम् त्वा त्वा स॒प्रथ॑स्तमꣳ स॒प्रथ॑स्तमम् त्वा गि॒रा गि॒रा त्वा॑ स॒प्रथ॑स्तमꣳ स॒प्रथ॑स्तमम् त्वा गि॒रा । \newline
46. स॒प्रथ॑स्तम॒मिति॑ स॒प्रथः॑ - त॒म॒म् । \newline
47. त्वा॒ गि॒रा गि॒रा त्वा᳚ त्वा गि॒रा दैव्य॒म् दैव्य॑म् गि॒रा त्वा᳚ त्वा गि॒रा दैव्य᳚म् । \newline
48. गि॒रा दैव्य॒म् दैव्य॑म् गि॒रा गि॒रा दैव्य॒म् मानु॑षा॒ मानु॑षा॒ दैव्य॑म् गि॒रा गि॒रा दैव्य॒म् मानु॑षा । \newline
49. दैव्य॒म् मानु॑षा॒ मानु॑षा॒ दैव्य॒म् दैव्य॒म् मानु॑षा यु॒गा यु॒गा मानु॑षा॒ दैव्य॒म् दैव्य॒म् मानु॑षा यु॒गा । \newline
50. मानु॑षा यु॒गा यु॒गा मानु॑षा॒ मानु॑षा यु॒गा । \newline
51. य॒गेति॑ यु॒गा । \newline
52. नि॒ष्क॒र्तार॑ मद्ध्व॒रस्या᳚ द्ध्व॒रस्य॑ निष्क॒र्तार॑म् निष्क॒र्तार॑ मद्ध्व॒रस्य॒ प्रचे॑तस॒म् प्रचे॑तस मद्ध्व॒रस्य॑ निष्क॒र्तार॑म् निष्क॒र्तार॑ मद्ध्व॒रस्य॒ प्रचे॑तसम् । \newline
53. नि॒ष्क॒र्तार॒मिति॑ निः - क॒र्तार᳚म् । \newline
54. अ॒द्ध्व॒रस्य॒ प्रचे॑तस॒म् प्रचे॑तस मद्ध्व॒रस्या᳚ द्ध्व॒रस्य॒ प्रचे॑तस॒म् क्षय॑न्त॒म् क्षय॑न्त॒म् प्रचे॑तस मद्ध्व॒रस्या᳚ द्ध्व॒रस्य॒ प्रचे॑तस॒म् क्षय॑न्तम् । \newline
55. प्रचे॑तस॒म् क्षय॑न्त॒म् क्षय॑न्त॒म् प्रचे॑तस॒म् प्रचे॑तस॒म् क्षय॑न्तꣳ॒॒ राध॑से॒ राध॑से॒ क्षय॑न्त॒म् प्रचे॑तस॒म् प्रचे॑तस॒म् क्षय॑न्तꣳ॒॒ राध॑से । \newline
56. प्रचे॑तस॒मिति॒ प्र - चे॒त॒स॒म् । \newline
57. क्षय॑न्तꣳ॒॒ राध॑से॒ राध॑से॒ क्षय॑न्त॒म् क्षय॑न्तꣳ॒॒ राध॑से म॒हे म॒हे राध॑से॒ क्षय॑न्त॒म् क्षय॑न्तꣳ॒॒ राध॑से म॒हे । \newline
58. राध॑से म॒हे म॒हे राध॑से॒ राध॑से म॒हे । \newline
59. म॒ह इति॑ म॒हे । \newline
60. रा॒तिम् भृगू॑णा॒म् भृगू॑णाꣳ रा॒तिꣳ रा॒तिम् भृगू॑णा मु॒शिज॑ मु॒शिज॒म् भृगू॑णाꣳ रा॒तिꣳ रा॒तिम् भृगू॑णा मु॒शिज᳚म् । \newline
61. भृगू॑णा मु॒शिज॑ मु॒शिज॒म् भृगू॑णा॒म् भृगू॑णा मु॒शिज॑म् क॒विक्र॑तुम् क॒विक्र॑तु मु॒शिज॒म् भृगू॑णा॒म् भृगू॑णा मु॒शिज॑म् क॒विक्र॑तुम् । \newline
62. उ॒शिज॑म् क॒विक्र॑तुम् क॒विक्र॑तु मु॒शिज॑ मु॒शिज॑म् क॒विक्र॑तुम् पृ॒णक्षि॑ पृ॒णक्षि॑ क॒विक्र॑तु मु॒शिज॑ मु॒शिज॑म् क॒विक्र॑तुम् पृ॒णक्षि॑ । \newline
63. क॒विक्र॑तुम् पृ॒णक्षि॑ पृ॒णक्षि॑ क॒विक्र॑तुम् क॒विक्र॑तुम् पृ॒णक्षि॑ सान॒सिꣳ सा॑न॒सिम् पृ॒णक्षि॑ क॒विक्र॑तुम् क॒विक्र॑तुम् पृ॒णक्षि॑ सान॒सिम् । \newline
64. क॒विक्र॑तु॒मिति॑ क॒वि - क्र॒तु॒म् । \newline
65. पृ॒णक्षि॑ सान॒सिꣳ सा॑न॒सिम् पृ॒णक्षि॑ पृ॒णक्षि॑ सान॒सिꣳ र॒यिꣳ र॒यिꣳ सा॑न॒सिम् पृ॒णक्षि॑ पृ॒णक्षि॑ सान॒सिꣳ र॒यिम् । \newline
66. सा॒न॒सिꣳ र॒यिꣳ र॒यिꣳ सा॑न॒सिꣳ सा॑न॒सिꣳ र॒यिम् । \newline
\pagebreak
\markright{ TS 4.2.7.4  \hfill https://www.vedavms.in \hfill}

\section{ TS 4.2.7.4 }

\textbf{TS 4.2.7.4 } \newline
\textbf{Samhita Paata} \newline

र॒यिं ॥ चितः॑ स्थ परि॒चित॑ ऊर्द्ध्व॒चितः॑ श्रयद्ध्वं॒ तया॑ दे॒वत॑याऽङ्गिर॒स्वद्-ध्रु॒वाः सी॑दत ॥ आ प्या॑यस्व॒ समे॑तु ते वि॒श्वतः॑ सोम॒ वृष्णि॑यं । भवा॒ वाज॑स्य सङ्ग॒थे ॥ सं ते॒ पयाꣳ॑सि॒ समु॑ यन्तु॒ वाजाः॒ सं ॅवृष्णि॑या-न्यभिमाति॒षाहः॑ । आ॒प्याय॑मानो अ॒मृता॑य सोम दि॒वि श्रवाꣳ॑स्युत्त॒मानि॑ धिष्व ॥ \newline

\textbf{Pada Paata} \newline

र॒यिम् ॥ चितः॑ । स्थ॒ । प॒रि॒चित॒ इति॑ परि - चितः॑ । ऊ॒द्‌र्ध्व॒चित॒ इत्यू᳚द्‌र्ध्व - चितः॑ । श्र॒य॒द्ध्व॒म् । तया᳚ । दे॒वत॑या । अ॒ङ्गि॒र॒स्वत् । ध्रु॒वाः । सी॒द॒त॒ ॥ एति॑ । प्या॒य॒स्व॒ । समिति॑ । ए॒तु॒ । ते॒ । वि॒श्वतः॑ । सो॒म॒ । वृष्णि॑यम् ॥ भव॑ । वाज॑स्य । स॒ङ्ग॒थ इति॑ सं-ग॒थे ॥ समिति॑ । ते॒ । पयाꣳ॑सि । समिति॑ । उ॒ । य॒न्तु॒ । वाजाः᳚ । समिति॑ । वृष्णि॑यानि । अ॒भि॒मा॒ति॒षाह॒ इत्य॑भिमाति - साहः॑ ॥ आ॒प्याय॑मान॒ इत्या᳚ - प्याय॑मानः । अ॒मृता॑य । सो॒म॒ । दि॒वि । श्रवाꣳ॑सि । उ॒त्त॒मानीत्यु॑त् - त॒मानि॑ । धि॒ष्व॒ ॥  \newline


\textbf{Krama Paata} \newline

र॒यिमिति॑ र॒यिम् ॥ चितः॑ स्थ । स्थ॒ प॒रि॒चितः॑ । प॒रि॒चित॑ ऊर्द्ध्व॒चितः॑ । प॒रि॒चित॒ इति॑ परि - चितः॑ । ऊ॒र्द्ध्व॒चितः॑ श्रयद्ध्वम् । ऊ॒र्द्ध्व॒चित॒ इत्यू᳚र्द्ध्व - चितः॑ । श्र॒य॒द्ध्व॒म् तया᳚ । तया॑ दे॒वत॑या । दे॒वत॑या ऽङ्गिर॒स्वत् । अ॒ङ्गि॒र॒स्वद् ध्रु॒वाः । ध्रु॒वाः सी॑दत । सी॒द॒तेति॑ सीदत ॥ आ प्या॑यस्व । प्या॒य॒स्व॒ सम् । समे॑तु । ए॒तु॒ ते॒ । ते॒ वि॒श्वतः॑ । वि॒श्वतः॑ सोम । सो॒म॒ वृष्णि॑यम् । वृष्णि॑य॒मिति॒ वृष्णि॑यम् ॥ भवा॒ वाज॑स्य । वाज॑स्य सङ्ग॒थे । स॒ङ्ग॒थ इति॑ सम् - ग॒थे ॥ सम् ते᳚ । ते॒ पयाꣳ॑सि । पयाꣳ॑सि॒ सम् । समु॑ । उ॒ य॒न्तु॒ । य॒न्तु॒ वाजाः᳚ । वाजाः॒ सम् । सं ॅवृष्णि॑यानि । वृष्णि॑यान्यभिमाति॒षाहः॑ । अ॒भि॒मा॒ति॒षाह॒ इत्य॑भिमाति - साहः॑ ॥ आ॒प्याय॑मानो अ॒मृता॑य । आ॒प्याय॑मान॒ इत्या᳚ - प्याय॑मानः । अ॒मृता॑य सोम । सो॒म॒ दि॒वि । दि॒वि श्रवाꣳ॑सि । श्रवाꣳ॑स्युत्त॒मानि॑ । उ॒त्त॒मानि॑ धिष्व । उ॒त्त॒मानीत्यु॑त् - त॒मानि॑ । धि॒ष्वेति॑ धिष्व । \newline

\textbf{Jatai Paata} \newline

1. र॒यिमिति॑ र॒यिम् । \newline
2. चितः॑ स्थ स्थ॒ चित॒ श्चितः॑ स्थ । \newline
3. स्थ॒ प॒रि॒चितः॑ परि॒चितः॑ स्थ स्थ परि॒चितः॑ । \newline
4. प॒रि॒चित॑ ऊर्द्ध्व॒चित॑ ऊर्द्ध्व॒चितः॑ परि॒चितः॑ परि॒चित॑ ऊर्द्ध्व॒चितः॑ । \newline
5. प॒रि॒चित॒ इति॑ परि - चितः॑ । \newline
6. ऊ॒र्द्ध्व॒चितः॑ श्रयद्ध्वꣳ श्रयद्ध्व मूर्द्ध्व॒चित॑ ऊर्द्ध्व॒चितः॑ श्रयद्ध्वम् । \newline
7. ऊ॒र्द्ध्व॒चित॒ इत्यू᳚र्द्ध्व - चितः॑ । \newline
8. श्र॒य॒द्ध्व॒म् तया॒ तया᳚ श्रयद्ध्वꣳ श्रयद्ध्व॒म् तया᳚ । \newline
9. तया॑ दे॒वत॑या दे॒वत॑या॒ तया॒ तया॑ दे॒वत॑या । \newline
10. दे॒वत॑या ऽङ्गिर॒स्व द॑ङ्गिर॒स्वद् दे॒वत॑या दे॒वत॑या ऽङ्गिर॒स्वत् । \newline
11. अ॒ङ्गि॒र॒स्वद् ध्रु॒वा ध्रु॒वा अ॑ङ्गिर॒स्व द॑ङ्गिर॒स्वद् ध्रु॒वाः । \newline
12. ध्रु॒वाः सी॑दत सीदत ध्रु॒वा ध्रु॒वाः सी॑दत । \newline
13. सी॒द॒तेति॑ सीदत । \newline
14. आ प्या॑यस्व प्याय॒स्वा प्या॑यस्व । \newline
15. प्या॒य॒स्व॒ सꣳ सम् प्या॑यस्व प्यायस्व॒ सम् । \newline
16. स मे᳚त्वेतु॒ सꣳ स मे॑तु । \newline
17. ए॒तु॒ ते॒ त॒ ए॒त्वे॒तु॒ ते॒ । \newline
18. ते॒ वि॒श्वतो॑ वि॒श्वत॑ स्ते ते वि॒श्वतः॑ । \newline
19. वि॒श्वतः॑ सोम सोम वि॒श्वतो॑ वि॒श्वतः॑ सोम । \newline
20. सो॒म॒ वृष्णि॑यं॒ ॅवृष्णि॑यꣳ सोम सोम॒ वृष्णि॑यम् । \newline
21. वृष्णि॑य॒मिति॒ वृष्णि॑यम् । \newline
22. भवा॒ वाज॑स्य॒ वाज॑स्य॒ भव॒ भवा॒ वाज॑स्य । \newline
23. वाज॑स्य सङ्ग॒थे स॑ङ्ग॒थे वाज॑स्य॒ वाज॑स्य सङ्ग॒थे । \newline
24. स॒ङ्ग॒थ इति॑ सं - ग॒थे । \newline
25. सम् ते॑ ते॒ सꣳ सम् ते᳚ । \newline
26. ते॒ पयाꣳ॑सि॒ पयाꣳ॑सि ते ते॒ पयाꣳ॑सि । \newline
27. पयाꣳ॑सि॒ सꣳ सम् पयाꣳ॑सि॒ पयाꣳ॑सि॒ सम् । \newline
28. शमु॑ वु॒ सꣳ स मु॑ । \newline
29. उ॒ य॒न्तु॒ य॒न्तू॒ य॒न्तु॒ । \newline
30. Yअ॒न्तु॒ वाजा॒ वाजा॑ यन्तु यन्तु॒ वाजाः᳚ । \newline
31. वाजाः॒ सꣳ सं ॅवाजा॒ वाजाः॒ सम् । \newline
32. सं ॅवृष्णि॑यानि॒ वृष्णि॑यानि॒ सꣳ सं ॅवृष्णि॑यानि । \newline
33. वृष्णि॑या न्यभिमाति॒षाहो॑ अभिमाति॒षाहो॒ वृष्णि॑यानि॒ वृष्णि॑या न्यभिमाति॒षाहः॑ । \newline
34. अ॒भि॒मा॒ति॒षाह॒ इत्य॑भिमाति - साहः॑ । \newline
35. आ॒प्याय॑मानो अ॒मृता॑या॒ मृता॑या॒ प्याय॑मान आ॒प्याय॑मानो अ॒मृता॑य । \newline
36. आ॒प्याय॑मान॒ इत्या᳚ - प्याय॑मानः । \newline
37. अ॒मृता॑य सोम सोमा॒ मृता॑या॒ मृता॑य सोम । \newline
38. सो॒म॒ दि॒वि दि॒वि सो॑म सोम दि॒वि । \newline
39. दि॒वि श्रवाꣳ॑सि॒ श्रवाꣳ॑सि दि॒वि दि॒वि श्रवाꣳ॑सि । \newline
40. श्रवाꣳ॑ स्युत्त॒मा न्यु॑त्त॒मानि॒ श्रवाꣳ॑सि॒ श्रवाꣳ॑ स्युत्त॒मानि॑ । \newline
41. उ॒त्त॒मानि॑ धिष्व धिष्वोत्त॒मा न्यु॑त्त॒मानि॑ धिष्व । \newline
42. उ॒त्त॒मानीत्यु॑त् - त॒मानि॑ । \newline
43. धि॒ष्वेति॑ धिष्व । \newline

\textbf{Ghana Paata } \newline

1. र॒यिमिति॑ र॒यिम् । \newline
2. चितः॑ स्थ स्थ॒ चित॒ श्चितः॑ स्थ परि॒चितः॑ परि॒चितः॑ स्थ॒ चित॒ श्चितः॑ स्थ परि॒चितः॑ । \newline
3. स्थ॒ प॒रि॒चितः॑ परि॒चितः॑ स्थ स्थ परि॒चित॑ ऊर्द्ध्व॒चित॑ ऊर्द्ध्व॒चितः॑ परि॒चितः॑ स्थ स्थ परि॒चित॑ ऊर्द्ध्व॒चितः॑ । \newline
4. प॒रि॒चित॑ ऊर्द्ध्व॒चित॑ ऊर्द्ध्व॒चितः॑ परि॒चितः॑ परि॒चित॑ ऊर्द्ध्व॒चितः॑ श्रयद्ध्वꣳ श्रयद्ध्व मूर्द्ध्व॒चितः॑ परि॒चितः॑ परि॒चित॑ ऊर्द्ध्व॒चितः॑ श्रयद्ध्वम् । \newline
5. प॒रि॒चित॒ इति॑ परि - चितः॑ । \newline
6. ऊ॒र्द्ध्व॒चितः॑ श्रयद्ध्वꣳ श्रयद्ध्व मूर्द्ध्व॒चित॑ ऊर्द्ध्व॒चितः॑ श्रयद्ध्व॒म् तया॒ तया᳚ श्रयद्ध्व मूर्द्ध्व॒चित॑ ऊर्द्ध्व॒चितः॑ श्रयद्ध्व॒म् तया᳚ । \newline
7. ऊ॒र्द्ध्व॒चित॒ इत्यू᳚र्द्ध्व - चितः॑ । \newline
8. श्र॒य॒द्ध्व॒म् तया॒ तया᳚ श्रयद्ध्वꣳ श्रयद्ध्व॒म् तया॑ दे॒वत॑या दे॒वत॑या॒ तया᳚ श्रयद्ध्वꣳ श्रयद्ध्व॒म् तया॑ दे॒वत॑या । \newline
9. तया॑ दे॒वत॑या दे॒वत॑या॒ तया॒ तया॑ दे॒वत॑या ऽङ्गिर॒स्व द॑ङ्गिर॒स्वद् दे॒वत॑या॒ तया॒ तया॑ दे॒वत॑या ऽङ्गिर॒स्वत् । \newline
10. दे॒वत॑या ऽङ्गिर॒स्व द॑ङ्गिर॒स्वद् दे॒वत॑या दे॒वत॑या ऽङ्गिर॒स्वद् ध्रु॒वा ध्रु॒वा अ॑ङ्गिर॒स्वद् दे॒वत॑या दे॒वत॑या ऽङ्गिर॒स्वद् ध्रु॒वाः । \newline
11. अ॒ङ्गि॒र॒स्वद् ध्रु॒वा ध्रु॒वा अ॑ङ्गिर॒स्व द॑ङ्गिर॒स्वद् ध्रु॒वाः सी॑दत सीदत ध्रु॒वा अ॑ङ्गिर॒स्व द॑ङ्गिर॒स्वद् ध्रु॒वाः सी॑दत । \newline
12. ध्रु॒वाः सी॑दत सीदत ध्रु॒वा ध्रु॒वाः सी॑दत । \newline
13. सी॒द॒तेति॑ सीदत । \newline
14. आ प्या॑यस्व प्याय॒स्वा प्या॑यस्व॒ सꣳ सम् प्या॑य॒स्वा प्या॑यस्व॒ सम् । \newline
15. प्या॒य॒स्व॒ सꣳ सम् प्या॑यस्व प्यायस्व॒ स मे᳚त्वेतु॒ सम् प्या॑यस्व प्यायस्व॒ स मे॑तु । \newline
16. स मे᳚त्वेतु॒ सꣳ स मे॑तु ते त एतु॒ सꣳ स मे॑तु ते । \newline
17. ए॒तु॒ ते॒ त॒ ए॒त्वे॒तु॒ ते॒ वि॒श्वतो॑ वि॒श्वत॑ स्त एत्वेतु ते वि॒श्वतः॑ । \newline
18. ते॒ वि॒श्वतो॑ वि॒श्वत॑ स्ते ते वि॒श्वतः॑ सोम सोम वि॒श्वत॑ स्ते ते वि॒श्वतः॑ सोम । \newline
19. वि॒श्वतः॑ सोम सोम वि॒श्वतो॑ वि॒श्वतः॑ सोम॒ वृष्णि॑यं॒ ॅवृष्णि॑यꣳ सोम वि॒श्वतो॑ वि॒श्वतः॑ सोम॒ वृष्णि॑यम् । \newline
20. सो॒म॒ वृष्णि॑यं॒ ॅवृष्णि॑यꣳ सोम सोम॒ वृष्णि॑यम् । \newline
21. वृष्णि॑य॒मिति॒ वृष्णि॑यम् । \newline
22. भवा॒ वाज॑स्य॒ वाज॑स्य॒ भव॒ भवा॒ वाज॑स्य सङ्ग॒थे स॑ङ्ग॒थे वाज॑स्य॒ भव॒ भवा॒ वाज॑स्य सङ्ग॒थे । \newline
23. वाज॑स्य सङ्ग॒थे स॑ङ्ग॒थे वाज॑स्य॒ वाज॑स्य सङ्ग॒थे । \newline
24. स॒ङ्ग॒थ इति॑ सं - ग॒थे । \newline
25. सम् ते॑ ते॒ सꣳ सम् ते॒ पयाꣳ॑सि॒ पयाꣳ॑सि ते॒ सꣳ सम् ते॒ पयाꣳ॑सि । \newline
26. ते॒ पयाꣳ॑सि॒ पयाꣳ॑सि ते ते॒ पयाꣳ॑सि॒ सꣳ सम् पयाꣳ॑सि ते ते॒ पयाꣳ॑सि॒ सम् । \newline
27. पयाꣳ॑सि॒ सꣳ सम् पयाꣳ॑सि॒ पयाꣳ॑सि॒ समु॑ वु॒ सम् पयाꣳ॑सि॒ पयाꣳ॑सि॒ स मु॑ । \newline
28. समु॑ वु॒ सꣳ स मु॑ यन्तु यन्तू॒ सꣳ स मु॑ यन्तु । \newline
29. उ॒ य॒न्तु॒ य॒न्तू॒ य॒न्तु॒ वाजा॒ वाजा॑ यन्तू यन्तु॒ वाजाः᳚ । \newline
30. य॒न्तु॒ वाजा॒ वाजा॑ यन्तु यन्तु॒ वाजाः॒ सꣳ सं ॅवाजा॑ यन्तु यन्तु॒ वाजाः॒ सम् । \newline
31. वाजाः॒ सꣳ सं ॅवाजा॒ वाजाः॒ सं ॅवृष्णि॑यानि॒ वृष्णि॑यानि॒ सं ॅवाजा॒ वाजाः॒ सं ॅवृष्णि॑यानि । \newline
32. सं ॅवृष्णि॑यानि॒ वृष्णि॑यानि॒ सꣳ सं ॅवृष्णि॑या न्यभिमाति॒षाहो॑ अभिमाति॒षाहो॒ वृष्णि॑यानि॒ सꣳ सं ॅवृष्णि॑या न्यभिमाति॒षाहः॑ । \newline
33. वृष्णि॑या न्यभिमाति॒षाहो॑ अभिमाति॒षाहो॒ वृष्णि॑यानि॒ वृष्णि॑या न्यभिमाति॒षाहः॑ । \newline
34. अ॒भि॒मा॒ति॒षाह॒ इत्य॑भिमाति - साहः॑ । \newline
35. आ॒प्याय॑मानो अ॒मृता॑या॒ मृता॑या॒ प्याय॑मान आ॒प्याय॑मानो अ॒मृता॑य सोम सोमा॒मृता॑या॒ प्याय॑मान आ॒प्याय॑मानो अ॒मृता॑य सोम । \newline
36. आ॒प्याय॑मान॒ इत्या᳚ - प्याय॑मानः । \newline
37. अ॒मृता॑य सोम सोमा॒मृता॑या॒ मृता॑य सोम दि॒वि दि॒वि सो॑मा॒मृता॑या॒ मृता॑य सोम दि॒वि । \newline
38. सो॒म॒ दि॒वि दि॒वि सो॑म सोम दि॒वि श्रवाꣳ॑सि॒ श्रवाꣳ॑सि दि॒वि सो॑म सोम दि॒वि श्रवाꣳ॑सि । \newline
39. दि॒वि श्रवाꣳ॑सि॒ श्रवाꣳ॑सि दि॒वि दि॒वि श्रवाꣳ॑ स्युत्त॒मा न्यु॑त्त॒मानि॒ श्रवाꣳ॑सि दि॒वि दि॒वि श्रवाꣳ॑ स्युत्त॒मानि॑ । \newline
40. श्रवाꣳ॑ स्युत्त॒मा न्यु॑त्त॒मानि॒ श्रवाꣳ॑सि॒ श्रवाꣳ॑ स्युत्त॒मानि॑ धिष्व धिष्वोत्त॒मानि॒ श्रवाꣳ॑सि॒ श्रवाꣳ॑ स्युत्त॒मानि॑ धिष्व । \newline
41. उ॒त्त॒मानि॑ धिष्व धिष्वोत्त॒मा न्यु॑त्त॒मानि॑ धिष्व । \newline
42. उ॒त्त॒मानीत्यु॑त् - त॒मानि॑ । \newline
43. धि॒ष्वेति॑ धिष्व । \newline
\pagebreak
\markright{ TS 4.2.8.1  \hfill https://www.vedavms.in \hfill}

\section{ TS 4.2.8.1 }

\textbf{TS 4.2.8.1 } \newline
\textbf{Samhita Paata} \newline

अ॒भ्य॑स्था॒द्-विश्वाः॒ पृत॑ना॒ अरा॑ती॒स्तद॒ग्निरा॑ह॒ तदु॒ सोम॑ आह । बृह॒स्पतिः॑ सवि॒ता तन्म॑ आह पू॒षा मा॑ऽधाथ् सुकृ॒तस्य॑ लो॒के ॥ यदक्र॑न्दः प्रथ॒मं जाय॑मान उ॒द्यन्थ् स॑मु॒द्रादु॒त वा॒ पुरी॑षात् । श्ये॒नस्य॑ प॒क्षा ह॑रि॒णस्य॑ बा॒हू उप॑स्तुतं॒ जनि॑म॒ तत् ते॑ अर्वन्न् ॥ अ॒पां पृ॒ष्ठम॑सि॒ योनि॑र॒ग्नेः स॑मु॒द्रम॒भितः॒ पिन्व॑मानं । वर्द्ध॑मानं म॒ह - [  ] \newline

\textbf{Pada Paata} \newline

अ॒भीति॑ । अ॒स्था॒त् । विश्वाः᳚ । पृत॑नाः । अरा॑तीः । तत् । अ॒ग्निः । आ॒ह॒ । तत् । उ॒ । सोमः॑ । आ॒ह॒ ॥ बृह॒स्पतिः॑ । स॒वि॒ता । तत् । मे॒ । आ॒ह॒ । पू॒षा । मा॒ । अ॒धा॒त् । सु॒कृ॒तस्येति॑ सु - कृ॒तस्य॑ । लो॒के ॥ यत् । अक्र॑न्दः । प्र॒थ॒मम् । जाय॑मानः । उ॒द्यन्नित्यु॑त्- यन्न् । स॒मु॒द्रात् । उ॒त । वा॒ । पुरी॑षात् ॥ श्ये॒नस्य॑ । प॒क्षा । ह॒रि॒णस्य॑ । बा॒हू इति॑ । उप॑स्तुत॒मित्युप॑ - स्तु॒त॒म् । जनि॑म । तत् । ते॒ । अ॒र्व॒न्न् ॥ अ॒पाम् । पृ॒ष्ठम् । अ॒सि॒ । योनिः॑ । अ॒ग्नेः । स॒मु॒द्रम् । अ॒भितः॑ । पिन्व॑मानम् ॥ वद्‌र्ध॑मानम् । म॒हः ।  \newline


\textbf{Krama Paata} \newline

अ॒भ्य॑स्थात् । अ॒स्था॒द् विश्वाः᳚ । विश्वाः॒ पृत॑नाः । पृत॑ना॒ अरा॑तीः । अरा॑ती॒स्तत् । तद॒ग्निः । अ॒ग्निरा॑ह । आ॒ह॒ तत् । तदु॑ । उ॒ सोमः॑ । सोम॑ आह । आ॒हेत्या॑ह ॥ बृह॒स्पतिः॑ सवि॒ता । स॒वि॒ता तत् । तन् मे᳚ । म॒ आ॒ह॒ । आ॒ह॒ पू॒षा । पू॒षा मा᳚ । मा॒ऽधा॒त्॒ । अ॒धा॒थ् सु॒कृ॒तस्य॑ । सु॒कृ॒तस्य॑ लो॒के । सु॒कृ॒तस्येति॑ सु - कृ॒तस्य॑ । लो॒क इति॑ लो॒के ॥ यदक्र॑न्दः । अक्र॑न्दः प्रथ॒मम् । प्र॒थ॒मम् जाय॑मानः । जाय॑मान उ॒द्यन्न् । उ॒द्यन्थ् स॑मु॒द्रात् । उ॒द्यन्नित्यु॑त् - यन्न् । स॒मु॒द्रादु॒त । उ॒त वा᳚ । वा॒ पुरी॑षात् । पुरी॑षा॒दिति॒ पुरी॑षात् ॥ श्ये॒नस्य॑ प॒क्षा । प॒क्षा ह॑रि॒णस्य॑ । ह॒रि॒णस्य॑ बा॒हू । बा॒हू उप॑स्तुतम् । बा॒हू इति॑ बा॒हू । उप॑स्तुत॒म् जनि॑म । उप॑स्तुत॒मित्युप॑ - स्तु॒त॒म् । जनि॑म॒ तत् । तत् ते᳚ । ते॒ अ॒र्व॒न्न्॒ । अ॒र्व॒न्नित्य॑र्वन्न् ॥ अ॒पाम् पृ॒ष्ठम् । पृ॒ष्ठम॑सि । अ॒सि॒ योनिः॑ । योनि॑र॒ग्नेः । अ॒ग्नेः स॑मु॒द्रम् । स॒मु॒द्रम॒भि॑तः । अ॒भितः॒ पिन्व॑मानम् । पिन्व॑मान॒मिति॒ पिन्व॑मानम् ॥ वर्द्ध॑मानम् म॒हः । म॒ह आ \newline

\textbf{Jatai Paata} \newline

1. अ॒भ्य॑स्था दस्था द॒भ्या᳚(1॒)भ्य॑स्थात् । \newline
2. अ॒स्था॒द् विश्वा॒ विश्वा॑ अस्था दस्था॒द् विश्वाः᳚ । \newline
3. विश्वाः॒ पृत॑नाः॒ पृत॑ना॒ विश्वा॒ विश्वाः॒ पृत॑नाः । \newline
4. पृत॑ना॒ अरा॑ती॒ ररा॑तीः॒ पृत॑नाः॒ पृत॑ना॒ अरा॑तीः । \newline
5. अरा॑ती॒ स्तत् तदरा॑ती॒ ररा॑ती॒ स्तत् । \newline
6. तद॒ग्नि र॒ग्नि स्तत् तद॒ग्निः । \newline
7. अ॒ग्नि रा॑हाहा॒ग्नि र॒ग्नि रा॑ह । \newline
8. आ॒ह॒ तत् तदा॑हाह॒ तत् । \newline
9. तदू॒ तत् तदु॑ । \newline
10. उ॒ सोमः॒ सोम॑ उ वु॒ सोमः॑ । \newline
11. सोम॑ आहाह॒ सोमः॒ सोम॑ आह । \newline
12. आ॒हेत्या॑ह । \newline
13. बृह॒स्पतिः॑ सवि॒ता स॑वि॒ता बृह॒स्पति॒र् बृह॒स्पतिः॑ सवि॒ता । \newline
14. स॒वि॒ता तत् तथ् स॑वि॒ता स॑वि॒ता तत् । \newline
15. तन् मे॑ मे॒ तत् तन् मे᳚ । \newline
16. म॒ आ॒हा॒ह॒ मे॒ म॒ आ॒ह॒ । \newline
17. आ॒ह॒ पू॒षा पू॒षा ऽऽहा॑ह पू॒षा । \newline
18. पू॒षा मा॑ मा पू॒षा पू॒षा मा᳚ । \newline
19. मा॒ ऽधा॒ द॒धा॒न् मा॒ मा॒ ऽधा॒त् । \newline
20. अ॒धा॒थ् सु॒कृ॒तस्य॑ सुकृ॒तस्या॑ धादधाथ् सुकृ॒तस्य॑ । \newline
21. सु॒कृ॒तस्य॑ लो॒के लो॒के सु॑कृ॒तस्य॑ सुकृ॒तस्य॑ लो॒के । \newline
22. सु॒कृ॒तस्येति॑ सु - कृ॒तस्य॑ । \newline
23. लो॒क इति॑ लो॒के । \newline
24. यदक्र॑न्दो॒ अक्र॑न्दो॒ यद् यदक्र॑न्दः । \newline
25. अक्र॑न्दः प्रथ॒मम् प्र॑थ॒म मक्र॑न्दो॒ अक्र॑न्दः प्रथ॒मम् । \newline
26. प्र॒थ॒मम् जाय॑मानो॒ जाय॑मानः प्रथ॒मम् प्र॑थ॒मम् जाय॑मानः । \newline
27. जाय॑मान उ॒द्यन् नु॒द्यन् जाय॑मानो॒ जाय॑मान उ॒द्यन्न् । \newline
28. उ॒द्यन् थ्स॑मु॒द्राथ् स॑मु॒द्रा दु॒द्यन् नु॒द्यन् थ्स॑मु॒द्रात् । \newline
29. उ॒द्यन्नित्यु॑त् - यन्न् । \newline
30. स॒मु॒द्रा दु॒तोत स॑मु॒द्राथ् स॑मु॒द्रा दु॒त । \newline
31. उ॒त वा॑ वो॒तोत वा᳚ । \newline
32. वा॒ पुरी॑षा॒त् पुरी॑षाद् वा वा॒ पुरी॑षात् । \newline
33. पुरी॑षा॒दिति॒ पुरी॑षात् । \newline
34. श्ये॒नस्य॑ प॒क्षा प॒क्षा श्ये॒नस्य॑ श्ये॒नस्य॑ प॒क्षा । \newline
35. प॒क्षा ह॑रि॒णस्य॑ हरि॒णस्य॑ प॒क्षा प॒क्षा ह॑रि॒णस्य॑ । \newline
36. ह॒रि॒णस्य॑ बा॒हू बा॒हू ह॑रि॒णस्य॑ हरि॒णस्य॑ बा॒हू । \newline
37. बा॒हू उप॑स्तुत॒ मुप॑स्तुतम् बा॒हू बा॒हू उप॑स्तुतम् । \newline
38. बा॒हू इति॑ बा॒हू । \newline
39. उप॑स्तुत॒म् जनि॑म॒ जनि॒ मोप॑स्तुत॒ मुप॑स्तुत॒म् जनि॑म । \newline
40. उप॑स्तुत॒मित्युप॑ - स्तु॒त॒म् । \newline
41. जनि॑म॒ तत् तज् जनि॑म॒ जनि॑म॒ तत् । \newline
42. तत् ते॑ ते॒ तत् तत् ते᳚ । \newline
43. ते॒ अ॒र्व॒न् न॒र्व॒न् ते॒ ते॒ अ॒र्व॒न्न् । \newline
44. अ॒र्व॒न्नित्य॑र्वन्न् । \newline
45. अ॒पाम् पृ॒ष्ठम् पृ॒ष्ठ म॒पा म॒पाम् पृ॒ष्ठम् । \newline
46. पृ॒ष्ठ म॑स्यसि पृ॒ष्ठम् पृ॒ष्ठ म॑सि । \newline
47. अ॒सि॒ योनि॒र् योनि॑ रस्यसि॒ योनिः॑ । \newline
48. योनि॑ र॒ग्ने र॒ग्नेर् योनि॒र् योनि॑ र॒ग्नेः । \newline
49. अ॒ग्नेः स॑मु॒द्रꣳ स॑मु॒द्र म॒ग्ने र॒ग्नेः स॑मु॒द्रम् । \newline
50. स॒मु॒द्र म॒भितो॑ अ॒भितः॑ समु॒द्रꣳ स॑मु॒द्र म॒भितः॑ । \newline
51. अ॒भितः॒ पिन्व॑मान॒म् पिन्व॑मान म॒भितो॑ अ॒भितः॒ पिन्व॑मानम् । \newline
52. पिन्व॑मान॒मिति॒ पिन्व॑मानम् । \newline
53. वर्द्ध॑मानम् म॒हो म॒हो वर्द्ध॑मानं॒ ॅवर्द्ध॑मानम् म॒हः । \newline
54. म॒ह आ म॒हो म॒ह आ । \newline

\textbf{Ghana Paata } \newline

1. अ॒भ्य॑स्था दस्था द॒भ्या᳚(1॒)भ्य॑स्था॒द् विश्वा॒ विश्वा॑ अस्था द॒भ्या᳚(1॒)भ्य॑स्था॒द् विश्वाः᳚ । \newline
2. अ॒स्था॒द् विश्वा॒ विश्वा॑ अस्था दस्था॒द् विश्वाः॒ पृत॑नाः॒ पृत॑ना॒ विश्वा॑ अस्था दस्था॒द् विश्वाः॒ पृत॑नाः । \newline
3. विश्वाः॒ पृत॑नाः॒ पृत॑ना॒ विश्वा॒ विश्वाः॒ पृत॑ना॒ अरा॑ती॒ ररा॑तीः॒ पृत॑ना॒ विश्वा॒ विश्वाः॒ पृत॑ना॒ अरा॑तीः । \newline
4. पृत॑ना॒ अरा॑ती॒ ररा॑तीः॒ पृत॑नाः॒ पृत॑ना॒ अरा॑ती॒ स्तत् तदरा॑तीः॒ पृत॑नाः॒ पृत॑ना॒ अरा॑ती॒ स्तत् । \newline
5. अरा॑ती॒ स्तत् तदरा॑ती॒ ररा॑ती॒ स्त द॒ग्नि र॒ग्नि स्तदरा॑ती॒ ररा॑ती॒ स्तद॒ग्निः । \newline
6. तद॒ग्नि र॒ग्नि स्तत् तद॒ग्नि रा॑हा हा॒ग्नि स्तत् तद॒ग्निरा॑ह । \newline
7. अ॒ग्नि रा॑हा हा॒ग्नि र॒ग्नि रा॑ह॒ तत् तदा॑ हा॒ग्नि र॒ग्नि रा॑ह॒ तत् । \newline
8. आ॒ह॒ तत् तदा॑ हाह॒ तदू॒ तदा॑हाह॒ तदु॑ । \newline
9. तदू॒ तत् तदु॑ सोमः॒ सोम॑ उ॒ तत् तदु॒ सोमः॑ । \newline
10. उ॒ सोमः॒ सोम॑ उ वु॒ सोम॑ आहाह॒ सोम॑ उ वु॒ सोम॑ आह । \newline
11. सोम॑ आहाह॒ सोमः॒ सोम॑ आह । \newline
12. आ॒हेत्या॑ह । \newline
13. बृह॒स्पतिः॑ सवि॒ता स॑वि॒ता बृह॒स्पति॒र् बृह॒स्पतिः॑ सवि॒ता तत् तथ् स॑वि॒ता बृह॒स्पति॒र् बृह॒स्पतिः॑ सवि॒ता तत् । \newline
14. स॒वि॒ता तत् तथ् स॑वि॒ता स॑वि॒ता तन् मे॑ मे॒ तथ् स॑वि॒ता स॑वि॒ता तन् मे᳚ । \newline
15. तन् मे॑ मे॒ तत् तन् म॑ आहाह मे॒ तत् तन् म॑ आह । \newline
16. म॒ आ॒हा॒ह॒ मे॒ म॒ आ॒ह॒ पू॒षा पू॒षा ऽऽह॑ मे म आह पू॒षा । \newline
17. आ॒ह॒ पू॒षा पू॒षा ऽऽहा॑ह पू॒षा मा॑ मा पू॒षा ऽऽहा॑ह पू॒षा मा᳚ । \newline
18. पू॒षा मा॑ मा पू॒षा पू॒षा मा॑ ऽधा दधान् मा पू॒षा पू॒षा मा॑ ऽधात् । \newline
19. मा॒ ऽधा॒ द॒धा॒न् मा॒ मा॒ ऽधा॒थ् सु॒कृ॒तस्य॑ सुकृ॒तस्या॑ धान् मा मा ऽधाथ् सुकृ॒तस्य॑ । \newline
20. अ॒धा॒थ् सु॒कृ॒तस्य॑ सुकृ॒तस्या॑ धा दधाथ् सुकृ॒तस्य॑ लो॒के लो॒के सु॑कृ॒तस्या॑ धा दधाथ् सुकृ॒तस्य॑ लो॒के । \newline
21. सु॒कृ॒तस्य॑ लो॒के लो॒के सु॑कृ॒तस्य॑ सुकृ॒तस्य॑ लो॒के । \newline
22. सु॒कृ॒तस्येति॑ सु - कृ॒तस्य॑ । \newline
23. लो॒क इति॑ लो॒के । \newline
24. यदक्र॑न्दो॒ अक्र॑न्दो॒ यद् यदक्र॑न्दः प्रथ॒मम् प्र॑थ॒म मक्र॑न्दो॒ यद् यदक्र॑न्दः प्रथ॒मम् । \newline
25. अक्र॑न्दः प्रथ॒मम् प्र॑थ॒म मक्र॑न्दो॒ अक्र॑न्दः प्रथ॒मम् जाय॑मानो॒ जाय॑मानः प्रथ॒म मक्र॑न्दो॒ अक्र॑न्दः प्रथ॒मम् जाय॑मानः । \newline
26. प्र॒थ॒मम् जाय॑मानो॒ जाय॑मानः प्रथ॒मम् प्र॑थ॒मम् जाय॑मान उ॒द्यन् नु॒द्यन् जाय॑मानः प्रथ॒मम् प्र॑थ॒मम् जाय॑मान उ॒द्यन्न् । \newline
27. जाय॑मान उ॒द्यन् नु॒द्यन् जाय॑मानो॒ जाय॑मान उ॒द्यन् थ्स॑मु॒द्राथ् स॑मु॒द्रा दु॒द्यन् जाय॑मानो॒ जाय॑मान उ॒द्यन् थ्स॑मु॒द्रात् । \newline
28. उ॒द्यन् थ्स॑मु॒द्राथ् स॑मु॒द्रा दु॒द्यन् नु॒द्यन् थ्स॑मु॒द्रा दु॒तोत स॑मु॒द्रा दु॒द्यन् नु॒द्यन् थ्स॑मु॒द्रा दु॒त । \newline
29. उ॒द्यन्नित्यु॑त् - यन्न् । \newline
30. स॒मु॒द्रा दु॒तोत स॑मु॒द्राथ् स॑मु॒द्रा दु॒त वा॑ वो॒त स॑मु॒द्राथ् स॑मु॒द्रा दु॒त वा᳚ । \newline
31. उ॒त वा॑ वो॒तोत वा॒ पुरी॑षा॒त् पुरी॑षाद् वो॒तोत वा॒ पुरी॑षात् । \newline
32. वा॒ पुरी॑षा॒त् पुरी॑षाद् वा वा॒ पुरी॑षात् । \newline
33. पुरी॑षा॒दिति॒ पुरी॑षात् । \newline
34. श्ये॒नस्य॑ प॒क्षा प॒क्षा श्ये॒नस्य॑ श्ये॒नस्य॑ प॒क्षा ह॑रि॒णस्य॑ हरि॒णस्य॑ प॒क्षा श्ये॒नस्य॑ श्ये॒नस्य॑ प॒क्षा ह॑रि॒णस्य॑ । \newline
35. प॒क्षा ह॑रि॒णस्य॑ हरि॒णस्य॑ प॒क्षा प॒क्षा ह॑रि॒णस्य॑ बा॒हू बा॒हू ह॑रि॒णस्य॑ प॒क्षा प॒क्षा ह॑रि॒णस्य॑ बा॒हू । \newline
36. ह॒रि॒णस्य॑ बा॒हू बा॒हू ह॑रि॒णस्य॑ हरि॒णस्य॑ बा॒हू उप॑स्तुत॒ मुप॑स्तुतम् बा॒हू ह॑रि॒णस्य॑ हरि॒णस्य॑ बा॒हू उप॑स्तुतम् । \newline
37. बा॒हू उप॑स्तुत॒ मुप॑स्तुतम् बा॒हू बा॒हू उप॑स्तुत॒म् जनि॑म॒ जनि॒ मोप॑स्तुतम् बा॒हू बा॒हू उप॑स्तुत॒म् जनि॑म । \newline
38. बा॒हू इति॑ बा॒हू । \newline
39. उप॑स्तुत॒म् जनि॑म॒ जनि॒ मोप॑स्तुत॒ मुप॑स्तुत॒म् जनि॑म॒ तत् तज् जनि॒ मोप॑स्तुत॒ मुप॑स्तुत॒म् जनि॑म॒ तत् । \newline
40. उप॑स्तुत॒मित्युप॑ - स्तु॒त॒म् । \newline
41. जनि॑म॒ तत् तज् जनि॑म॒ जनि॑म॒ तत् ते॑ ते॒ तज् जनि॑म॒ जनि॑म॒ तत् ते᳚ । \newline
42. तत् ते॑ ते॒ तत् तत् ते॑ अर्वन् नर्वन् ते॒ तत् तत् ते॑ अर्वन्न् । \newline
43. ते॒ अ॒र्व॒न् न॒र्व॒न् ते॒ ते॒ अ॒र्व॒न्न् । \newline
44. अ॒र्व॒न्नित्य॑र्वन्न् । \newline
45. अ॒पाम् पृ॒ष्ठम् पृ॒ष्ठ म॒पा म॒पाम् पृ॒ष्ठ म॑स्यसि पृ॒ष्ठ म॒पा म॒पाम् पृ॒ष्ठ म॑सि । \newline
46. पृ॒ष्ठ म॑स्यसि पृ॒ष्ठम् पृ॒ष्ठ म॑सि॒ योनि॒र् योनि॑ रसि पृ॒ष्ठम् पृ॒ष्ठ म॑सि॒ योनिः॑ । \newline
47. अ॒सि॒ योनि॒र् योनि॑ रस्यसि॒ योनि॑ र॒ग्ने र॒ग्नेर् योनि॑ रस्यसि॒ योनि॑ र॒ग्नेः । \newline
48. योनि॑ र॒ग्ने र॒ग्नेर् योनि॒र् योनि॑ र॒ग्नेः स॑मु॒द्रꣳ स॑मु॒द्र म॒ग्नेर् योनि॒र् योनि॑ र॒ग्नेः स॑मु॒द्रम् । \newline
49. अ॒ग्नेः स॑मु॒द्रꣳ स॑मु॒द्र म॒ग्ने र॒ग्नेः स॑मु॒द्र म॒भितो॑ अ॒भितः॑ समु॒द्र म॒ग्ने र॒ग्नेः स॑मु॒द्र म॒भितः॑ । \newline
50. स॒मु॒द्र म॒भितो॑ अ॒भितः॑ समु॒द्रꣳ स॑मु॒द्र म॒भितः॒ पिन्व॑मान॒म् पिन्व॑मान म॒भितः॑ समु॒द्रꣳ स॑मु॒द्र म॒भितः॒ पिन्व॑मानम् । \newline
51. अ॒भितः॒ पिन्व॑मान॒म् पिन्व॑मान म॒भितो॑ अ॒भितः॒ पिन्व॑मानम् । \newline
52. पिन्व॑मान॒मिति॒ पिन्व॑मानम् । \newline
53. वर्द्ध॑मानम् म॒हो म॒हो वर्द्ध॑मानं॒ ॅवर्द्ध॑मानम् म॒ह आ म॒हो वर्द्ध॑मानं॒ ॅवर्द्ध॑मानम् म॒ह आ । \newline
54. म॒ह आ म॒हो म॒ह आ च॒ चा म॒हो म॒ह आ च॑ । \newline
\pagebreak
\markright{ TS 4.2.8.2  \hfill https://www.vedavms.in \hfill}

\section{ TS 4.2.8.2 }

\textbf{TS 4.2.8.2 } \newline
\textbf{Samhita Paata} \newline

आ च॒ पुष्क॑रं दि॒वो मात्र॑या वरि॒णा प्र॑थस्व ॥ ब्रह्म॑ जज्ञा॒नं प्र॑थ॒मं पु॒रस्ता॒द्वि सी॑म॒तः सु॒रुचो॑ वे॒न आ॑वः । स बु॒द्ध्निया॑ उप॒मा अ॑स्य वि॒ष्ठाः स॒तश्च॒ योनि॒मस॑तश्च॒ विवः॑ ॥ हि॒र॒ण्य॒ग॒र्भः सम॑वर्त॒ताग्रे॑ भू॒तस्य॑ जा॒तः पति॒रेक॑ आसीत् । स दा॑धार पृथि॒वीं द्यामु॒तेमां कस्मै॑ दे॒वाय॑ ह॒विषा॑ विधेम ॥ द्र॒फ्सश्च॑स्कन्द पृथि॒वीमनु॒ - [  ] \newline

\textbf{Pada Paata} \newline

एति॑ । च॒ । पुष्क॑रम् । दि॒वः । मात्र॑या । व॒रि॒णा । प्र॒थ॒स्व॒ ॥ ब्रह्म॑ । ज॒ज्ञा॒नम् । प्र॒थ॒मम् । पु॒रस्ता᳚त् । वीति॑ । सी॒म॒तः । सु॒रुच॒ इति॑ सु - रुचः॑ । वे॒नः । आ॒वः॒ ॥ सः । बु॒द्ध्नियाः᳚ । उ॒प॒मा इत्युप॑-माः । अ॒स्य॒ । वि॒ष्ठा इति॑ वि-स्थाः । स॒तः । च॒ । योनि᳚म् । अस॑तः । च॒ । विवः॑ ॥ हि॒र॒ण्य॒ग॒र्भ इति॑ हिरण्य - ग॒र्भः । समिति॑ । अ॒व॒र्त॒त । अग्रे᳚ । भू॒तस्य॑ । जा॒तः । पतिः॑ । एकः॑ । आ॒सी॒त् ॥ सः । दा॒धा॒र॒ । पृ॒थि॒वीम् । द्याम् । उ॒त । इ॒माम् । कस्मै᳚ । दे॒वाय॑ । ह॒विषा᳚ । वि॒धे॒म॒ ॥ द्र॒फ्सः । च॒स्क॒न्द॒ । पृ॒थि॒वीम् । अन्विति॑ ।  \newline


\textbf{Krama Paata} \newline

आ च॑ । च॒ पुष्क॑रम् । पुष्क॑रम् दि॒वः । दि॒वो मात्र॑या । मात्र॑या वरि॒णा । व॒रि॒णा प्र॑थस्व । प्र॒थ॒स्वेति॑ प्रथस्व ॥ ब्रह्म॑ जज्ञा॒नम् । ज॒ज्ञा॒नम् प्र॑थ॒मम् । प्र॒थ॒मम् पु॒रस्ता᳚त् । पु॒रस्ता॒द् वि । वि सी॑म॒तः । सी॒म॒तः सु॒रुचः॑ । सु॒रुचो॑ वे॒नः । सु॒रुच॒ इति॑ सु - रुचः॑ । वे॒न आ॑वः । आ॒व॒ इत्या॑वः ॥? स बु॒द्ध्नियाः᳚ । बु॒द्ध्निया॑ उप॒माः । उ॒प॒मा अ॑स्य । उ॒प॒मा इत्यु॑प - माः । अ॒स्य॒ वि॒ष्ठाः । वि॒ष्ठाः स॒तः । वि॒ष्ठा इति॑ वि - स्थाः । स॒तश्च॑ । च॒ योनि᳚म् । योनि॒मस॑तः । अस॑तश्च । च॒ विवः॑ । विव॒ इति॒ विवः॑ ॥ हि॒र॒ण्य॒ग॒र्भः सम् । हि॒र॒ण्य॒ग॒र्भ इति॑ हिरण्य - ग॒र्भः । सम॑वर्तत । अ॒व॒र्त॒ताग्रे᳚ । अग्रे॑ भू॒तस्य॑ । भू॒तस्य॑ जा॒तः । जा॒तः पतिः॑ । पति॒रेकः॑ । एक॑ आसीत् । आ॒सी॒दित्या॑सीत् ॥ स दा॑धार । दा॒धा॒र॒ पृ॒थि॒वीम् । पृ॒थि॒वीम् द्याम् । द्यामु॒त । उ॒तेमाम् । इ॒माम् कस्मै᳚ । कस्मै॑ दे॒वाय॑ । दे॒वाय॑ ह॒विषा᳚ । ह॒विषा॑ विधेम । वि॒धे॒मेति॑ विधेम ॥ द्र॒फ्सश्च॑स्कन्द । च॒स्क॒न्द॒ पृ॒थि॒वीम् । पृ॒थि॒वीमनु॑ । अनु॒ द्याम् \newline

\textbf{Jatai Paata} \newline

1. आ च॒ चा च॑ । \newline
2. च॒ पुष्क॑र॒म् पुष्क॑रम् च च॒ पुष्क॑रम् । \newline
3. पुष्क॑रम् दि॒वो दि॒वः पुष्क॑र॒म् पुष्क॑रम् दि॒वः । \newline
4. दि॒वो मात्र॑या॒ मात्र॑या दि॒वो दि॒वो मात्र॑या । \newline
5. मात्र॑या वरि॒णा व॑रि॒णा मात्र॑या॒ मात्र॑या वरि॒णा । \newline
6. व॒रि॒णा प्र॑थस्व प्रथस्व वरि॒णा व॑रि॒णा प्र॑थस्व । \newline
7. प्र॒थ॒स्वेति॑ प्रथस्व । \newline
8. ब्रह्म॑ जज्ञा॒नम् ज॑ज्ञा॒नम् ब्रह्म॒ ब्रह्म॑ जज्ञा॒नम् । \newline
9. ज॒ज्ञा॒नम् प्र॑थ॒मम् प्र॑थ॒मम् ज॑ज्ञा॒नम् ज॑ज्ञा॒नम् प्र॑थ॒मम् । \newline
10. प्र॒थ॒मम् पु॒रस्ता᳚त् पु॒रस्ता᳚त् प्रथ॒मम् प्र॑थ॒मम् पु॒रस्ता᳚त् । \newline
11. पु॒रस्ता॒द् वि वि पु॒रस्ता᳚त् पु॒रस्ता॒द् वि । \newline
12. वि सी॑म॒तः सी॑म॒तो वि वि सी॑म॒तः । \newline
13. सी॒म॒तः सु॒रुचः॑ सु॒रुचः॑ सीम॒तः सी॑म॒तः सु॒रुचः॑ । \newline
14. सु॒रुचो॑ वे॒नो वे॒नः सु॒रुचः॑ सु॒रुचो॑ वे॒नः । \newline
15. सु॒रुच॒ इति॑ सु - रुचः॑ । \newline
16. वे॒न आ॑व रावर् वे॒नो वे॒न आ॑वः । \newline
17. आ॒व॒रित्या॑वः । \newline
18. स बु॒द्ध्निया॑ बु॒द्ध्नियाः॒ स स बु॒द्ध्नियाः᳚ । \newline
19. बु॒द्ध्निया॑ उप॒मा उ॑प॒मा बु॒द्ध्निया॑ बु॒द्ध्निया॑ उप॒माः । \newline
20. उ॒प॒मा अ॑स्या स्योप॒मा उ॑प॒मा अ॑स्य । \newline
21. उ॒प॒मा इत्युप॑ - माः । \newline
22. अ॒स्य॒ वि॒ष्ठा वि॒ष्ठा अ॑स्यास्य वि॒ष्ठाः । \newline
23. वि॒ष्ठाः स॒तः स॒तो वि॒ष्ठा वि॒ष्ठाः स॒तः । \newline
24. वि॒ष्ठा इति॑ वि - स्थाः । \newline
25. स॒तश्च॑ च स॒तः स॒तश्च॑ । \newline
26. च॒ योनिं॒ ॅयोनि॑म् च च॒ योनि᳚म् । \newline
27. योनि॒ मस॑तो॒ अस॑तो॒ योनिं॒ ॅयोनि॒ मस॑तः । \newline
28. अस॑तश्च॒ चास॑तो॒ अस॑तश्च । \newline
29. च॒ विव॒र् विव॑श्च च॒ विवः॑ । \newline
30. विव॒रिति॒ विवः॑ । \newline
31. हि॒र॒ण्य॒ग॒र्भः सꣳ सꣳ हि॑रण्यग॒र्भो हि॑रण्यग॒र्भः सम् । \newline
32. हि॒र॒ण्य॒ग॒र्भ इति॑ हिरण्य - ग॒र्भः । \newline
33. स म॑वर्त॒ता व॑र्त॒त सꣳ स म॑वर्त॒त । \newline
34. अ॒व॒र्त॒ताग्रे॒ अग्रे॑ ऽवर्त॒ता व॑र्त॒ताग्रे᳚ । \newline
35. अग्रे॑ भू॒तस्य॑ भू॒तस्याग्रे॒ अग्रे॑ भू॒तस्य॑ । \newline
36. भू॒तस्य॑ जा॒तो जा॒तो भू॒तस्य॑ भू॒तस्य॑ जा॒तः । \newline
37. जा॒तः पति॒ष् पति॑र् जा॒तो जा॒तः पतिः॑ । \newline
38. पति॒ रेक॒ एक॒ स्पति॒ष् पति॒ रेकः॑ । \newline
39. एक॑ आसी दासी॒ देक॒ एक॑ आसीत् । \newline
40. आ॒सी॒दित्या॑सीत् । \newline
41. स दा॑धार दाधार॒ स स दा॑धार । \newline
42. दा॒धा॒र॒ पृ॒थि॒वीम् पृ॑थि॒वीम् दा॑धार दाधार पृथि॒वीम् । \newline
43. पृ॒थि॒वीम् द्याम् द्याम् पृ॑थि॒वीम् पृ॑थि॒वीम् द्याम् । \newline
44. द्या मु॒तोत द्याम् द्या मु॒त । \newline
45. उ॒तेमा मि॒मा मु॒तोते माम् । \newline
46. इ॒माम् कस्मै॒ कस्मा॑ इ॒मा मि॒माम् कस्मै᳚ । \newline
47. कस्मै॑ दे॒वाय॑ दे॒वाय॒ कस्मै॒ कस्मै॑ दे॒वाय॑ । \newline
48. दे॒वाय॑ ह॒विषा॑ ह॒विषा॑ दे॒वाय॑ दे॒वाय॑ ह॒विषा᳚ । \newline
49. ह॒विषा॑ विधेम विधेम ह॒विषा॑ ह॒विषा॑ विधेम । \newline
50. वि॒धे॒मेति॑ विधेम । \newline
51. द्र॒फ्स श्च॑स्कन्द चस्कन्द द्र॒फ्सो द्र॒फ्स श्च॑स्कन्द । \newline
52. च॒स्क॒न्द॒ पृ॒थि॒वीम् पृ॑थि॒वीम् च॑स्कन्द चस्कन्द पृथि॒वीम् । \newline
53. पृ॒थि॒वी मन्वनु॑ पृथि॒वीम् पृ॑थि॒वी मनु॑ । \newline
54. अनु॒ द्याम् द्या मन्वनु॒ द्याम् । \newline

\textbf{Ghana Paata } \newline

1. आ च॒ चा च॒ पुष्क॑र॒म् पुष्क॑र॒म् चा च॒ पुष्क॑रम् । \newline
2. च॒ पुष्क॑र॒म् पुष्क॑रम् च च॒ पुष्क॑रम् दि॒वो दि॒वः पुष्क॑रम् च च॒ पुष्क॑रम् दि॒वः । \newline
3. पुष्क॑रम् दि॒वो दि॒वः पुष्क॑र॒म् पुष्क॑रम् दि॒वो मात्र॑या॒ मात्र॑या दि॒वः पुष्क॑र॒म् पुष्क॑रम् दि॒वो मात्र॑या । \newline
4. दि॒वो मात्र॑या॒ मात्र॑या दि॒वो दि॒वो मात्र॑या वरि॒णा व॑रि॒णा मात्र॑या दि॒वो दि॒वो मात्र॑या वरि॒णा । \newline
5. मात्र॑या वरि॒णा व॑रि॒णा मात्र॑या॒ मात्र॑या वरि॒णा प्र॑थस्व प्रथस्व वरि॒णा मात्र॑या॒ मात्र॑या वरि॒णा प्र॑थस्व । \newline
6. व॒रि॒णा प्र॑थस्व प्रथस्व वरि॒णा व॑रि॒णा प्र॑थस्व । \newline
7. प्र॒थ॒स्वेति॑ प्रथस्व । \newline
8. ब्रह्म॑ जज्ञा॒नम् ज॑ज्ञा॒नम् ब्रह्म॒ ब्रह्म॑ जज्ञा॒नम् प्र॑थ॒मम् प्र॑थ॒मम् ज॑ज्ञा॒नम् ब्रह्म॒ ब्रह्म॑ जज्ञा॒नम् प्र॑थ॒मम् । \newline
9. ज॒ज्ञा॒नम् प्र॑थ॒मम् प्र॑थ॒मम् ज॑ज्ञा॒नम् ज॑ज्ञा॒नम् प्र॑थ॒मम् पु॒रस्ता᳚त् पु॒रस्ता᳚त् प्रथ॒मम् ज॑ज्ञा॒नम् ज॑ज्ञा॒नम् प्र॑थ॒मम् पु॒रस्ता᳚त् । \newline
10. प्र॒थ॒मम् पु॒रस्ता᳚त् पु॒रस्ता᳚त् प्रथ॒मम् प्र॑थ॒मम् पु॒रस्ता॒द् वि वि पु॒रस्ता᳚त् प्रथ॒मम् प्र॑थ॒मम् पु॒रस्ता॒द् वि । \newline
11. पु॒रस्ता॒द् वि वि पु॒रस्ता᳚त् पु॒रस्ता॒द् वि सी॑म॒तः सी॑म॒तो वि पु॒रस्ता᳚त् पु॒रस्ता॒द् वि सी॑म॒तः । \newline
12. वि सी॑म॒तः सी॑म॒तो वि वि सी॑म॒तः सु॒रुचः॑ सु॒रुचः॑ सीम॒तो वि वि सी॑म॒तः सु॒रुचः॑ । \newline
13. सी॒म॒तः सु॒रुचः॑ सु॒रुचः॑ सीम॒तः सी॑म॒तः सु॒रुचो॑ वे॒नो वे॒नः सु॒रुचः॑ सीम॒तः सी॑म॒तः सु॒रुचो॑ वे॒नः । \newline
14. सु॒रुचो॑ वे॒नो वे॒नः सु॒रुचः॑ सु॒रुचो॑ वे॒न आ॑व रावर् वे॒नः सु॒रुचः॑ सु॒रुचो॑ वे॒न आ॑वः । \newline
15. सु॒रुच॒ इति॑ सु - रुचः॑ । \newline
16. वे॒न आ॑व रावर् वे॒नो वे॒न आ॑वः । \newline
17. आ॒व॒रित्या॑वः । \newline
18. स बु॒द्ध्निया॑ बु॒द्ध्नियाः॒ स स बु॒द्ध्निया॑ उप॒मा उ॑प॒मा बु॒द्ध्नियाः॒ स स बु॒द्ध्निया॑ उप॒माः । \newline
19. बु॒द्ध्निया॑ उप॒मा उ॑प॒मा बु॒द्ध्निया॑ बु॒द्ध्निया॑ उप॒मा अ॑स्या स्योप॒मा बु॒द्ध्निया॑ बु॒द्ध्निया॑ उप॒मा अ॑स्य । \newline
20. उ॒प॒मा अ॑स्या स्योप॒मा उ॑प॒मा अ॑स्य वि॒ष्ठा वि॒ष्ठा अ॑स्योप॒मा उ॑प॒मा अ॑स्य वि॒ष्ठाः । \newline
21. उ॒प॒मा इत्युप॑ - माः । \newline
22. अ॒स्य॒ वि॒ष्ठा वि॒ष्ठा अ॑स्यास्य वि॒ष्ठाः स॒तः स॒तो वि॒ष्ठा अ॑स्यास्य वि॒ष्ठाः स॒तः । \newline
23. वि॒ष्ठाः स॒तः स॒तो वि॒ष्ठा वि॒ष्ठाः स॒तश्च॑ च स॒तो वि॒ष्ठा वि॒ष्ठाः स॒तश्च॑ । \newline
24. वि॒ष्ठा इति॑ वि - स्थाः । \newline
25. स॒तश्च॑ च स॒तः स॒तश्च॒ योनिं॒ ॅयोनि॑म् च स॒तः स॒तश्च॒ योनि᳚म् । \newline
26. च॒ योनिं॒ ॅयोनि॑म् च च॒ योनि॒ मस॑तो॒ अस॑तो॒ योनि॑म् च च॒ योनि॒ मस॑तः । \newline
27. योनि॒ मस॑तो॒ अस॑तो॒ योनिं॒ ॅयोनि॒ मस॑तश्च॒ चास॑तो॒ योनिं॒ ॅयोनि॒ मस॑तश्च । \newline
28. अस॑तश्च॒ चास॑तो॒ अस॑तश्च॒ विव॒र् विव॒ श्चास॑तो॒ अस॑तश्च॒ विवः॑ । \newline
29. च॒ विव॒र् विव॑श्च च॒ विवः॑ । \newline
30. विव॒रिति॒ विवः॑ । \newline
31. हि॒र॒ण्य॒ग॒र्भः सꣳ सꣳ हि॑रण्यग॒र्भो हि॑रण्यग॒र्भः स म॑वर्त॒ता व॑र्त॒त सꣳ हि॑रण्यग॒र्भो हि॑रण्यग॒र्भः स म॑वर्त॒त । \newline
32. हि॒र॒ण्य॒ग॒र्भ इति॑ हिरण्य - ग॒र्भः । \newline
33. स म॑वर्त॒ता व॑र्त॒त सꣳ स म॑वर्त॒ताग्रे॒ अग्रे॑ ऽवर्त॒त सꣳ स म॑वर्त॒ताग्रे᳚ । \newline
34. अ॒व॒र्त॒ताग्रे॒ अग्रे॑ ऽवर्त॒ता व॑र्त॒ताग्रे॑ भू॒तस्य॑ भू॒तस्याग्रे॑ ऽवर्त॒ता व॑र्त॒ताग्रे॑ भू॒तस्य॑ । \newline
35. अग्रे॑ भू॒तस्य॑ भू॒तस्याग्रे॒ अग्रे॑ भू॒तस्य॑ जा॒तो जा॒तो भू॒तस्याग्रे॒ अग्रे॑ भू॒तस्य॑ जा॒तः । \newline
36. भू॒तस्य॑ जा॒तो जा॒तो भू॒तस्य॑ भू॒तस्य॑ जा॒तः पति॒ष् पति॑र् जा॒तो भू॒तस्य॑ भू॒तस्य॑ जा॒तः पतिः॑ । \newline
37. जा॒तः पति॒ष् पति॑र् जा॒तो जा॒तः पति॒ रेक॒ एक॒ स्पति॑र् जा॒तो जा॒तः पति॒ रेकः॑ । \newline
38. पति॒रेक॒ एक॒ स्पति॒ष् पति॒रेक॑ आसी दासी॒ देक॒ स्पति॒ष् पति॒रेक॑ आसीत् । \newline
39. एक॑ आसी दासी॒ देक॒ एक॑ आसीत् । \newline
40. आ॒सी॒दित्या॑सीत् । \newline
41. स दा॑धार दाधार॒ स स दा॑धार पृथि॒वीम् पृ॑थि॒वीम् दा॑धार॒ स स दा॑धार पृथि॒वीम् । \newline
42. दा॒धा॒र॒ पृ॒थि॒वीम् पृ॑थि॒वीम् दा॑धार दाधार पृथि॒वीम् द्याम् द्याम् पृ॑थि॒वीम् दा॑धार दाधार पृथि॒वीम् द्याम् । \newline
43. पृ॒थि॒वीम् द्याम् द्याम् पृ॑थि॒वीम् पृ॑थि॒वीम् द्या मु॒तोत द्याम् पृ॑थि॒वीम् पृ॑थि॒वीम् द्या मु॒त । \newline
44. द्या मु॒तोत द्याम् द्या मु॒ते मा मि॒मा मु॒त द्याम् द्या मु॒ते माम् । \newline
45. उ॒ते मा मि॒मा मु॒तोते माम् कस्मै॒ कस्मा॑ इ॒मा मु॒तोते माम् कस्मै᳚ । \newline
46. इ॒माम् कस्मै॒ कस्मा॑ इ॒मा मि॒माम् कस्मै॑ दे॒वाय॑ दे॒वाय॒ कस्मा॑ इ॒मा मि॒माम् कस्मै॑ दे॒वाय॑ । \newline
47. कस्मै॑ दे॒वाय॑ दे॒वाय॒ कस्मै॒ कस्मै॑ दे॒वाय॑ ह॒विषा॑ ह॒विषा॑ दे॒वाय॒ कस्मै॒ कस्मै॑ दे॒वाय॑ ह॒विषा᳚ । \newline
48. दे॒वाय॑ ह॒विषा॑ ह॒विषा॑ दे॒वाय॑ दे॒वाय॑ ह॒विषा॑ विधेम विधेम ह॒विषा॑ दे॒वाय॑ दे॒वाय॑ ह॒विषा॑ विधेम । \newline
49. ह॒विषा॑ विधेम विधेम ह॒विषा॑ ह॒विषा॑ विधेम । \newline
50. वि॒धे॒मेति॑ विधेम । \newline
51. द्र॒फ्स श्च॑स्कन्द चस्कन्द द्र॒फ्सो द्र॒फ्स श्च॑स्कन्द पृथि॒वीम् पृ॑थि॒वीम् च॑स्कन्द द्र॒फ्सो द्र॒फ्स श्च॑स्कन्द पृथि॒वीम् । \newline
52. च॒स्क॒न्द॒ पृ॒थि॒वीम् पृ॑थि॒वीम् च॑स्कन्द चस्कन्द पृथि॒वी मन्वनु॑ पृथि॒वीम् च॑स्कन्द चस्कन्द पृथि॒वी मनु॑ । \newline
53. पृ॒थि॒वी मन्वनु॑ पृथि॒वीम् पृ॑थि॒वी मनु॒ द्याम् द्या मनु॑ पृथि॒वीम् पृ॑थि॒वी मनु॒ द्याम् । \newline
54. अनु॒ द्याम् द्या मन्वनु॒ द्या मि॒म मि॒मम् द्या मन्वनु॒ द्या मि॒मम् । \newline
\pagebreak
\markright{ TS 4.2.8.3  \hfill https://www.vedavms.in \hfill}

\section{ TS 4.2.8.3 }

\textbf{TS 4.2.8.3 } \newline
\textbf{Samhita Paata} \newline

-द्यामि॒मं च॒ योनि॒मनु॒ यश्च॒ पूर्वः॑ । तृ॒तीयं॒ ॅयोनि॒मनु॑ स॒ञ्चर॑न्तं द्र॒फ्सं जु॑हो॒म्यनु॑ स॒प्त होत्राः᳚ ॥ नमो॑ अस्तु स॒र्पेभ्यो॒ ये के च॑ पृथि॒वीमनु॑ । ये अ॒न्तरि॑क्षे॒ ये दि॒वि तेभ्यः॑ स॒र्पेभ्यो॒ नमः॑ ॥ ये॑ऽदो रो॑च॒ने दि॒वो ये वा॒ सूर्य॑स्य र॒श्मिषु॑ । येषा॑म॒फ्सु सदः॑ कृ॒तं तेभ्यः॑ स॒र्पेभ्यो॒ नमः॑ ॥ या इष॑वो यातु॒ धाना॑नां॒ ( ) ॅये॑ वा॒ वन॒स्पतीꣳ॒॒रनु॑ । ये वा॑ऽव॒टेषु॒ शेर॑ते॒ तेभ्यः॑ स॒र्पेभ्यो॒ नमः॑ ॥ \newline

\textbf{Pada Paata} \newline

द्याम् । इ॒मम् । च॒ । योनि᳚म् । अन्विति॑ । यः । च॒ । पूर्वः॑ ॥ तृ॒तीय᳚म् । योनि᳚म् । अन्विति॑ । स॒ञ्चर॑न्त॒मिति॑ सं - चर॑न्तम् । द्र॒फ्सम् । जु॒हो॒मि॒ । अन्विति॑ । स॒प्त । होत्राः᳚ ॥ नमः॑ । अ॒स्तु॒ । स॒र्पेभ्यः॑ । ये । के । च॒ । पृ॒थि॒वीम् । अनु॑ ॥ ये । अ॒न्तरि॑क्षे । ये । दि॒वि । तेभ्यः॑ । स॒र्पेभ्यः॑ । नमः॑ ॥ ये । अ॒दः । रो॒च॒ने । दि॒वः । ये । वा॒ । सूर्य॑स्य । र॒श्मिषु॑ ॥ येषा᳚म् । अ॒फ्स्वित्य॑प्- सु । सदः॑ । कृ॒तम् । तेभ्यः॑ । स॒र्पेभ्यः॑ । नमः॑ ॥ याः । इष॑वः । या॒तु॒धाना॑ना॒मिति॑ यातु - धाना॑नाम् ( ) । ये । वा॒ । वन॒स्पतीन्॑ । अनु॑ ॥ ये । वा॒ । अ॒व॒टेषु॑ । शेर॑ते । तेभ्यः॑ । स॒र्पेभ्यः॑ । नमः॑ ॥  \newline


\textbf{Krama Paata} \newline

द्यामि॒मम् । इ॒मम् च॑ । च॒ योनि᳚म् । योनि॒मनु॑ । अनु॒ यः । यश्च॑ । च॒ पूर्वः॑ । पूर्व॒ इति॒ पूर्वः॑ ॥ तृ॒तीयं॒ ॅयोनि᳚म् । योनि॒मनु॑ । अनु॑ स॒ञ्चर॑न्तम् । स॒ञ्चर॑न्तम् द्र॒फ्सम् । स॒ञ्चर॑न्त॒मिति॑ सम् - चर॑न्तम् । द्र॒फ्सम् जु॑होमि । जु॒हो॒म्यनु॑ । अनु॑ स॒प्त । स॒प्त होत्राः᳚ । होत्रा॒ इति॒ होत्राः᳚ ॥ नमो॑ अस्तु । अ॒स्तु॒ स॒र्पेभ्यः॑ । स॒र्पेभ्यो॒ ये । ये के । के च॑ । च॒ पृ॒थि॒वीम् । पृ॒थि॒वीमनु॑ । अन्वित्यनु॑ ॥ ये अ॒न्तरि॑क्षे । अ॒न्तरि॑क्षे॒ ये । ये दि॒वि । दि॒वि तेभ्यः॑ । तेभ्यः॑ स॒र्पेभ्यः॑ । स॒र्पेभ्यो॒ नमः॑ । नम॒ इति॒ नमः॑ ॥ ये॑ऽदः । अ॒दो रो॑च॒ने । रो॒च॒ने दि॒वः । दि॒वो ये । ये वा᳚ । वा॒ सूर्य॑स्य । सूर्य॑स्य र॒श्मिषु॑ । र॒श्मिष्विति॑ र॒श्मिषु॑ ॥ येषा॑म॒फ्सु । अ॒फ्सु सदः॑ । अ॒फ्स्वित्य॑प् - सु । सदः॑ कृ॒तम् । कृ॒तम् तेभ्यः॑ । तेभ्यः॑ स॒र्पेभ्यः॑ । स॒र्पेभ्यो॒ नमः॑ । नम॒ इति॒ नमः॑ ॥ या इष॑वः । इष॑वो यातु॒धाना॑नाम् । या॒तु॒धाना॑नां॒ ॅये । या॒तु॒धाना॑ना॒मिति॑ यातु - धाना॑नाम् ( ) । ये वा᳚ । वा॒ वन॒स्पतीन्॑ । वन॒स्पतीꣳ॒॒रनु॑ । अन्वित्यनु॑ ॥ ये वा᳚ । वा॒ऽव॒टेषु॑ । अ॒व॒टेषु॒ शेर॑ते । शेर॑ते॒ तेभ्यः॑ । तेभ्यः॑ स॒र्पेभ्यः॑ । स॒र्पेभ्यो॒ नमः॑ । नम॒ इति॒ नमः॑ । \newline

\textbf{Jatai Paata} \newline

1. द्या मि॒म मि॒मम् द्याम् द्या मि॒मम् । \newline
2. इ॒मम् च॑ चे॒ म मि॒मम् च॑ । \newline
3. च॒ योनिं॒ ॅयोनि॑म् च च॒ योनि᳚म् । \newline
4. योनि॒ मन्वनु॒ योनिं॒ ॅयोनि॒ मनु॑ । \newline
5. अनु॒ यो यो ऽन्वनु॒ यः । \newline
6. यश्च॑ च॒ यो यश्च॑ । \newline
7. च॒ पूर्वः॒ पूर्व॑श्च च॒ पूर्वः॑ । \newline
8. पूर्व॒ इति॒ पूर्वः॑ । \newline
9. तृ॒तीयं॒ ॅयोनिं॒ ॅयोनि॑म् तृ॒तीय॑म् तृ॒तीयं॒ ॅयोनि᳚म् । \newline
10. योनि॒ मन्वनु॒ योनिं॒ ॅयोनि॒ मनु॑ । \newline
11. अनु॑ स॒ञ्चर॑न्तꣳ स॒ञ्चर॑न्त॒ मन्वनु॑ स॒ञ्चर॑न्तम् । \newline
12. स॒ञ्चर॑न्तम् द्र॒फ्सम् द्र॒फ्सꣳ स॒ञ्चर॑न्तꣳ स॒ञ्चर॑न्तम् द्र॒फ्सम् । \newline
13. स॒ञ्चर॑न्त॒मिति॑ सं - चर॑न्तम् । \newline
14. द्र॒फ्सम् जु॑होमि जुहोमि द्र॒फ्सम् द्र॒फ्सम् जु॑होमि । \newline
15. जु॒हो॒ म्यन्वनु॑ जुहोमि जुहो॒ म्यनु॑ । \newline
16. अनु॑ स॒प्त स॒प्तान्वनु॑ स॒प्त । \newline
17. स॒प्त होत्रा॒ होत्राः᳚ स॒प्त स॒प्त होत्राः᳚ । \newline
18. होत्रा॒ इति॒ होत्राः᳚ । \newline
19. नमो॑ अस्त्वस्तु॒ नमो॒ नमो॑ अस्तु । \newline
20. अ॒स्तु॒ स॒र्पेभ्यः॑ स॒र्पेभ्यो॑ अस्त्वस्तु स॒र्पेभ्यः॑ । \newline
21. स॒र्पेभ्यो॒ ये ये स॒र्पेभ्यः॑ स॒र्पेभ्यो॒ ये । \newline
22. ये के के ये ये के । \newline
23. के च॑ च॒ के के च॑ । \newline
24. च॒ पृ॒थि॒वीम् पृ॑थि॒वीम् च॑ च पृथि॒वीम् । \newline
25. पृ॒थि॒वी मन्वनु॑ पृथि॒वीम् पृ॑थि॒वी मनु॑ । \newline
26. अन्वित्यनु॑ । \newline
27. ये अ॒न्तरि॑क्षे अ॒न्तरि॑क्षे॒ ये ये अ॒न्तरि॑क्षे । \newline
28. अ॒न्तरि॑क्षे॒ ये ये अ॒न्तरि॑क्षे अ॒न्तरि॑क्षे॒ ये । \newline
29. ये दि॒वि दि॒वि ये ये दि॒वि । \newline
30. दि॒वि तेभ्य॒ स्तेभ्यो॑ दि॒वि दि॒वि तेभ्यः॑ । \newline
31. तेभ्यः॑ स॒र्पेभ्यः॑ स॒र्पेभ्य॒ स्तेभ्य॒ स्तेभ्यः॑ स॒र्पेभ्यः॑ । \newline
32. स॒र्पेभ्यो॒ नमो॒ नमः॑ स॒र्पेभ्यः॑ स॒र्पेभ्यो॒ नमः॑ । \newline
33. नम॒ इति॒ नमः॑ । \newline
34. ये᳚(1॒) ऽदो॑ ऽदो ये ये॑ ऽदः । \newline
35. अ॒दो रो॑च॒ने रो॑च॒ने᳚(1॒) ऽदो॑ ऽदो रो॑च॒ने । \newline
36. रो॒च॒ने दि॒वो दि॒वो रो॑च॒ने रो॑च॒ने दि॒वः । \newline
37. दि॒वो ये ये दि॒वो दि॒वो ये । \newline
38. ये वा॑ वा॒ ये ये वा᳚ । \newline
39. वा॒ सूर्य॑स्य॒ सूर्य॑स्य वा वा॒ सूर्य॑स्य । \newline
40. सूर्य॑स्य र॒श्मिषु॑ र॒श्मिषु॒ सूर्य॑स्य॒ सूर्य॑स्य र॒श्मिषु॑ । \newline
41. र॒श्मिष्विति॑ र॒श्मिषु॑ । \newline
42. येषा॑ म॒फ्स्व॑फ्सु येषां॒ ॅयेषा॑ म॒फ्सु । \newline
43. अ॒फ्सु सदः॒ सदो॑ अ॒फ्स्व॑फ्सु सदः॑ । \newline
44. अ॒फ्स्वित्य॑प् - सु । \newline
45. सदः॑ कृ॒तम् कृ॒तꣳ सदः॒ सदः॑ कृ॒तम् । \newline
46. कृ॒तम् तेभ्य॒ स्तेभ्यः॑ कृ॒तम् कृ॒तम् तेभ्यः॑ । \newline
47. तेभ्यः॑ स॒र्पेभ्यः॑ स॒र्पेभ्य॒ स्तेभ्य॒ स्तेभ्यः॑ स॒र्पेभ्यः॑ । \newline
48. स॒र्पेभ्यो॒ नमो॒ नमः॑ स॒र्पेभ्यः॑ स॒र्पेभ्यो॒ नमः॑ । \newline
49. नम॒ इति॒ नमः॑ । \newline
50. या इष॑व॒ इष॑वो॒ या या इष॑वः । \newline
51. इष॑वो यातु॒धाना॑नां ॅयातु॒धाना॑ना॒ मिष॑व॒ इष॑वो यातु॒धाना॑नाम् । \newline
52. या॒तु॒धाना॑नां॒ ॅये ये या॑तु॒धाना॑नां ॅयातु॒धाना॑नां॒ ॅये । \newline
53. या॒तु॒धाना॑ना॒मिति॑ यातु - धाना॑नाम् । \newline
54. ये वा॑ वा॒ ये ये वा᳚ । \newline
55. वा॒ वन॒स्पती॒न्॒. वन॒स्पतीन्॑. वा वा॒ वन॒स्पतीन्॑ । \newline
56. वन॒स्पतीꣳ॒॒ रन्वन्. वन॒स्पती॒न्॒. वन॒स्पतीꣳ॒॒ रनु॑ । \newline
57. अन्वित्यनु॑ । \newline
58. ये वा॑ वा॒ ये ये वा᳚ । \newline
59. वा॒ ऽव॒टे ष्व॑व॒टेषु॑ वा वा ऽव॒टेषु॑ । \newline
60. अ॒व॒टेषु॒ शेर॑ते॒ शेर॑ते ऽव॒टे ष्व॑व॒टेषु॒ शेर॑ते । \newline
61. शेर॑ते॒ तेभ्य॒ स्तेभ्यः॒ शेर॑ते॒ शेर॑ते॒ तेभ्यः॑ । \newline
62. तेभ्यः॑ स॒र्पेभ्यः॑ स॒र्पेभ्य॒ स्तेभ्य॒ स्तेभ्यः॑ स॒र्पेभ्यः॑ । \newline
63. स॒र्पेभ्यो॒ नमो॒ नमः॑ स॒र्पेभ्यः॑ स॒र्पेभ्यो॒ नमः॑ । \newline
64. नम॒ इति॒ नमः॑ । \newline

\textbf{Ghana Paata } \newline

1. द्या मि॒म मि॒मम् द्याम् द्या मि॒मम् च॑ चे॒मम् द्याम् द्या मि॒मम् च॑ । \newline
2. इ॒मम् च॑ चे॒म मि॒मम् च॒ योनिं॒ ॅयोनि॑म् चे॒म मि॒मम् च॒ योनि᳚म् । \newline
3. च॒ योनिं॒ ॅयोनि॑म् च च॒ योनि॒ मन्वनु॒ योनि॑म् च च॒ योनि॒ मनु॑ । \newline
4. योनि॒ मन्वनु॒ योनिं॒ ॅयोनि॒ मनु॒ यो यो ऽनु॒ योनिं॒ ॅयोनि॒ मनु॒ यः । \newline
5. अनु॒ यो यो ऽन्वनु॒ यश्च॑ च॒ यो ऽन्वनु॒ यश्च॑ । \newline
6. यश्च॑ च॒ यो यश्च॒ पूर्वः॒ पूर्व॑श्च॒ यो यश्च॒ पूर्वः॑ । \newline
7. च॒ पूर्वः॒ पूर्व॑श्च च॒ पूर्वः॑ । \newline
8. पूर्व॒ इति॒ पूर्वः॑ । \newline
9. तृ॒तीयं॒ ॅयोनिं॒ ॅयोनि॑म् तृ॒तीय॑म् तृ॒तीयं॒ ॅयोनि॒ मन्वनु॒ योनि॑म् तृ॒तीय॑म् तृ॒तीयं॒ ॅयोनि॒ मनु॑ । \newline
10. योनि॒ मन्वनु॒ योनिं॒ ॅयोनि॒ मनु॑ स॒ञ्चर॑न्तꣳ स॒ञ्चर॑न्त॒ मनु॒ योनिं॒ ॅयोनि॒ मनु॑ स॒ञ्चर॑न्तम् । \newline
11. अनु॑ स॒ञ्चर॑न्तꣳ स॒ञ्चर॑न्त॒ मन्वनु॑ स॒ञ्चर॑न्तम् द्र॒फ्सम् द्र॒फ्सꣳ स॒ञ्चर॑न्त॒ मन्वनु॑ स॒ञ्चर॑न्तम् द्र॒फ्सम् । \newline
12. स॒ञ्चर॑न्तम् द्र॒फ्सम् द्र॒फ्सꣳ स॒ञ्चर॑न्तꣳ स॒ञ्चर॑न्तम् द्र॒फ्सम् जु॑होमि जुहोमि द्र॒फ्सꣳ स॒ञ्चर॑न्तꣳ स॒ञ्चर॑न्तम् द्र॒फ्सम् जु॑होमि । \newline
13. स॒ञ्चर॑न्त॒मिति॑ सं - चर॑न्तम् । \newline
14. द्र॒फ्सम् जु॑होमि जुहोमि द्र॒फ्सम् द्र॒फ्सम् जु॑हो॒ म्यन्वनु॑ जुहोमि द्र॒फ्सम् द्र॒फ्सम् जु॑हो॒ म्यनु॑ । \newline
15. जु॒हो॒ म्यन्वनु॑ जुहोमि जुहो॒ म्यनु॑ स॒प्त स॒प्तानु॑ जुहोमि जुहो॒ म्यनु॑ स॒प्त । \newline
16. अनु॑ स॒प्त स॒प्तान्वनु॑ स॒प्त होत्रा॒ होत्राः᳚ स॒प्तान्वनु॑ स॒प्त होत्राः᳚ । \newline
17. स॒प्त होत्रा॒ होत्राः᳚ स॒प्त स॒प्त होत्राः᳚ । \newline
18. होत्रा॒ इति॒ होत्राः᳚ । \newline
19. नमो॑ अस्त्वस्तु॒ नमो॒ नमो॑ अस्तु स॒र्पेभ्यः॑ स॒र्पेभ्यो॑ अस्तु॒ नमो॒ नमो॑ अस्तु स॒र्पेभ्यः॑ । \newline
20. अ॒स्तु॒ स॒र्पेभ्यः॑ स॒र्पेभ्यो॑ अस्त्वस्तु स॒र्पेभ्यो॒ ये ये स॒र्पेभ्यो॑ अस्त्वस्तु स॒र्पेभ्यो॒ ये । \newline
21. स॒र्पेभ्यो॒ ये ये स॒र्पेभ्यः॑ स॒र्पेभ्यो॒ ये के के ये स॒र्पेभ्यः॑ स॒र्पेभ्यो॒ ये के । \newline
22. ये के के ये ये के च॑ च॒ के ये ये के च॑ । \newline
23. के च॑ च॒ के के च॑ पृथि॒वीम् पृ॑थि॒वीम् च॒ के के च॑ पृथि॒वीम् । \newline
24. च॒ पृ॒थि॒वीम् पृ॑थि॒वीम् च॑ च पृथि॒वी मन्वनु॑ पृथि॒वीम् च॑ च पृथि॒वी मनु॑ । \newline
25. पृ॒थि॒वी मन्वनु॑ पृथि॒वीम् पृ॑थि॒वी मनु॑ । \newline
26. अन्वित्यनु॑ । \newline
27. ये अ॒न्तरि॑क्षे अ॒न्तरि॑क्षे॒ ये ये अ॒न्तरि॑क्षे॒ ये ये अ॒न्तरि॑क्षे॒ ये ये अ॒न्तरि॑क्षे॒ ये । \newline
28. अ॒न्तरि॑क्षे॒ ये ये अ॒न्तरि॑क्षे अ॒न्तरि॑क्षे॒ ये दि॒वि दि॒वि ये अ॒न्तरि॑क्षे अ॒न्तरि॑क्षे॒ ये दि॒वि । \newline
29. ये दि॒वि दि॒वि ये ये दि॒वि तेभ्य॒ स्तेभ्यो॑ दि॒वि ये ये दि॒वि तेभ्यः॑ । \newline
30. दि॒वि तेभ्य॒ स्तेभ्यो॑ दि॒वि दि॒वि तेभ्यः॑ स॒र्पेभ्यः॑ स॒र्पेभ्य॒ स्तेभ्यो॑ दि॒वि दि॒वि तेभ्यः॑ स॒र्पेभ्यः॑ । \newline
31. तेभ्यः॑ स॒र्पेभ्यः॑ स॒र्पेभ्य॒ स्तेभ्य॒ स्तेभ्यः॑ स॒र्पेभ्यो॒ नमो॒ नमः॑ स॒र्पेभ्य॒ स्तेभ्य॒ स्तेभ्यः॑ स॒र्पेभ्यो॒ नमः॑ । \newline
32. स॒र्पेभ्यो॒ नमो॒ नमः॑ स॒र्पेभ्यः॑ स॒र्पेभ्यो॒ नमः॑ । \newline
33. नम॒ इति॒ नमः॑ । \newline
34. ये᳚(1॒) ऽदो॑ ऽदो ये ये॑ ऽदो रो॑च॒ने रो॑च॒ने॑ ऽदो ये ये॑ ऽदो रो॑च॒ने । \newline
35. अ॒दो रो॑च॒ने रो॑च॒ने᳚(1॒) ऽदो॑ ऽदो रो॑च॒ने दि॒वो दि॒वो रो॑च॒ने᳚(1॒) ऽदो॑ ऽदो रो॑च॒ने दि॒वः । \newline
36. रो॒च॒ने दि॒वो दि॒वो रो॑च॒ने रो॑च॒ने दि॒वो ये ये दि॒वो रो॑च॒ने रो॑च॒ने दि॒वो ये । \newline
37. दि॒वो ये ये दि॒वो दि॒वो ये वा॑ वा॒ ये दि॒वो दि॒वो ये वा᳚ । \newline
38. ये वा॑ वा॒ ये ये वा॒ सूर्य॑स्य॒ सूर्य॑स्य वा॒ ये ये वा॒ सूर्य॑स्य । \newline
39. वा॒ सूर्य॑स्य॒ सूर्य॑स्य वा वा॒ सूर्य॑स्य र॒श्मिषु॑ र॒श्मिषु॒ सूर्य॑स्य वा वा॒ सूर्य॑स्य र॒श्मिषु॑ । \newline
40. सूर्य॑स्य र॒श्मिषु॑ र॒श्मिषु॒ सूर्य॑स्य॒ सूर्य॑स्य र॒श्मिषु॑ । \newline
41. र॒श्मिष्विति॑ र॒श्मिषु॑ । \newline
42. येषा॑ म॒फ्स्व॑फ्सु येषां॒ ॅयेषा॑ म॒फ्सु सदः॒ सदो॑ अ॒फ्सु येषां॒ ॅयेषा॑ म॒फ्सु सदः॑ । \newline
43. अ॒फ्सु सदः॒ सदो॑ अ॒फ्स्व॑फ्सु सदः॑ कृ॒तम् कृ॒तꣳ सदो॑ अ॒फ्स्व॑फ्सु सदः॑ कृ॒तम् । \newline
44. अ॒फ्स्वित्य॑प् - सु । \newline
45. सदः॑ कृ॒तम् कृ॒तꣳ सदः॒ सदः॑ कृ॒तम् तेभ्य॒ स्तेभ्यः॑ कृ॒तꣳ सदः॒ सदः॑ कृ॒तम् तेभ्यः॑ । \newline
46. कृ॒तम् तेभ्य॒ स्तेभ्यः॑ कृ॒तम् कृ॒तम् तेभ्यः॑ स॒र्पेभ्यः॑ स॒र्पेभ्य॒ स्तेभ्यः॑ कृ॒तम् कृ॒तम् तेभ्यः॑ स॒र्पेभ्यः॑ । \newline
47. तेभ्यः॑ स॒र्पेभ्यः॑ स॒र्पेभ्य॒ स्तेभ्य॒ स्तेभ्यः॑ स॒र्पेभ्यो॒ नमो॒ नमः॑ स॒र्पेभ्य॒ स्तेभ्य॒ स्तेभ्यः॑ स॒र्पेभ्यो॒ नमः॑ । \newline
48. स॒र्पेभ्यो॒ नमो॒ नमः॑ स॒र्पेभ्यः॑ स॒र्पेभ्यो॒ नमः॑ । \newline
49. नम॒ इति॒ नमः॑ । \newline
50. या इष॑व॒ इष॑वो॒ या या इष॑वो यातु॒धाना॑नां ॅयातु॒धाना॑ना॒ मिष॑वो॒ या या इष॑वो यातु॒धाना॑नाम् । \newline
51. इष॑वो यातु॒धाना॑नां ॅयातु॒धाना॑ना॒ मिष॑व॒ इष॑वो यातु॒धाना॑नां॒ ॅये ये या॑तु॒धाना॑ना॒ मिष॑व॒ इष॑वो यातु॒धाना॑नां॒ ॅये । \newline
52. या॒तु॒धाना॑नां॒ ॅये ये या॑तु॒धाना॑नां ॅयातु॒धाना॑नां॒ ॅये वा॑ वा॒ ये या॑तु॒धाना॑नां ॅयातु॒धाना॑नां॒ ॅये वा᳚ । \newline
53. या॒तु॒धाना॑ना॒मिति॑ यातु - धाना॑नाम् । \newline
54. ये वा॑ वा॒ ये ये वा॒ वन॒स्पती॒न्॒. वन॒स्पतीन्॑. वा॒ ये ये वा॒ वन॒स्पतीन्॑ । \newline
55. वा॒ वन॒स्पती॒न्॒. वन॒स्पतीन्॑. वा वा॒ वन॒स्पतीꣳ॒॒ रन्वन्. वन॒स्पतीन्॑. वा वा॒ वन॒स्पतीꣳ॒॒ रनु॑ । \newline
56. वन॒स्पतीꣳ॒॒ रन्वन्. वन॒स्पती॒न्॒. वन॒स्पतीꣳ॒॒ रनु॑ । \newline
57. अन्वित्यनु॑ । \newline
58. ये वा॑ वा॒ ये ये वा॑ ऽव॒टे ष्व॑व॒टेषु॑ वा॒ ये ये वा॑ ऽव॒टेषु॑ । \newline
59. वा॒ ऽव॒टे ष्व॑व॒टेषु॑ वा वा ऽव॒टेषु॒ शेर॑ते॒ शेर॑ते ऽव॒टेषु॑ वा वा ऽव॒टेषु॒ शेर॑ते । \newline
60. अ॒व॒टेषु॒ शेर॑ते॒ शेर॑ते ऽव॒टे ष्व॑व॒टेषु॒ शेर॑ते॒ तेभ्य॒ स्तेभ्यः॒ शेर॑ते ऽव॒टे ष्व॑व॒टेषु॒ शेर॑ते॒ तेभ्यः॑ । \newline
61. शेर॑ते॒ तेभ्य॒ स्तेभ्यः॒ शेर॑ते॒ शेर॑ते॒ तेभ्यः॑ स॒र्पेभ्यः॑ स॒र्पेभ्य॒ स्तेभ्यः॒ शेर॑ते॒ शेर॑ते॒ तेभ्यः॑ स॒र्पेभ्यः॑ । \newline
62. तेभ्यः॑ स॒र्पेभ्यः॑ स॒र्पेभ्य॒ स्तेभ्य॒ स्तेभ्यः॑ स॒र्पेभ्यो॒ नमो॒ नमः॑ स॒र्पेभ्य॒ स्तेभ्य॒ स्तेभ्यः॑ स॒र्पेभ्यो॒ नमः॑ । \newline
63. स॒र्पेभ्यो॒ नमो॒ नमः॑ स॒र्पेभ्यः॑ स॒र्पेभ्यो॒ नमः॑ । \newline
64. नम॒ इति॒ नमः॑ । \newline
\pagebreak
\markright{ TS 4.2.9.1  \hfill https://www.vedavms.in \hfill}

\section{ TS 4.2.9.1 }

\textbf{TS 4.2.9.1 } \newline
\textbf{Samhita Paata} \newline

ध्रु॒वाऽसि॑ ध॒रुणाऽस्तृ॑ता वि॒श्वक॑र्मणा॒ सुकृ॑ता । मा त्वा॑ समु॒द्र उद्व॑धी॒न्मा सु॑प॒र्णो व्य॑थमाना पृथि॒वीं दृꣳ॑ह ॥ प्र॒जाप॑तिस्त्वा सादयतु पृथि॒व्याः पृ॒ष्ठे व्यच॑स्वतीं॒ प्रथ॑स्वतीं॒ प्रथो॑ऽसि पृथि॒व्य॑सि॒ भूर॑सि॒ भूमि॑र॒स्यदि॑तिरसि वि॒श्वधा॑या॒ विश्व॑स्य॒ भुव॑नस्य ध॒र्त्री पृ॑थि॒वीं ॅय॑च्छ पृथि॒वीं दृꣳ॑ह पृथि॒वीं मा हिꣳ॑सी॒र्विश्व॑स्मै प्रा॒णाया॑पा॒नाय॑ व्या॒नायो॑दा॒नाय॑ प्रति॒ष्ठायै॑ - [  ] \newline

\textbf{Pada Paata} \newline

ध्रु॒वा । अ॒सि॒ । ध॒रुणा᳚ । अस्तृ॑ता । वि॒श्वक॑र्म॒णेति॑ वि॒श्व - क॒र्म॒णा॒ । सुकृ॒तेति॒ सु - कृ॒ता॒ ॥ मा । त्वा॒ । स॒मु॒द्रः । उदिति॑ । व॒धी॒त् । मा । सु॒प॒र्ण इति॑ सु - प॒र्णः । अव्य॑थमाना । पृ॒थि॒वीम् । दृꣳ॒॒ह॒ ॥ प्र॒जाप॑ति॒रिति॑ प्र॒जा - प॒तिः॒ । त्वा॒ । सा॒द॒य॒तु॒ । पृ॒थि॒व्याः । पृ॒ष्ठे । व्यच॑स्वतीम् । प्रथ॑स्वतीम् । प्रथः॑ । अ॒सि॒ । पृ॒थि॒वी । अ॒सि॒ । भूः । अ॒सि॒ । भूमिः॑ । अ॒सि॒ । अदि॑तिः । अ॒सि॒ । वि॒श्वधा॑या॒ इति॑ वि॒श्व - धा॒याः॒ । विश्व॑स्य । भुव॑नस्य । ध॒र्त्री । पृ॒थि॒वीम् । य॒च्छ॒ । पृ॒थि॒वीम् । दृꣳ॒॒ह॒ । पृ॒थि॒वीम् । मा । हिꣳ॒॒सीः॒ । विश्व॑स्मै । प्रा॒णायेति॑ प्रा - अ॒नाय॑ । अ॒पा॒नायेत्य॑प - अ॒नाय॑ । व्या॒नायेति॑ वि - अ॒नाय॑ । उ॒दा॒नायेत्यु॑त् - अ॒नाय॑ । प्र॒ति॒ष्ठाया॒ इति॑ प्रति - स्थायै᳚ ।  \newline


\textbf{Krama Paata} \newline

ध्रु॒वाऽसि॑ । अ॒सि॒ ध॒रुणा᳚ । ध॒रुणा ऽस्तृ॑ता । अस्तृ॑ता वि॒श्वक॑र्मणा । वि॒श्वक॑र्मणा॒ सुकृ॑ता । वि॒श्वक॑र्म॒णेति॑ वि॒श्व - क॒र्म॒णा॒ । सुकृ॒तेति॒ सु - कृ॒ता॒ ॥ मा त्वा᳚ । त्वा॒ स॒मु॒द्रः । स॒मु॒द्र उत् । उद् व॑धीत् । व॒धी॒न् मा । मा सु॑प॒र्णः । सु॒प॒र्णो ऽव्य॑थमाना । सु॒प॒र्ण इति॑ सु - प॒र्णः । अव्य॑थमाना पृथि॒वीम् । पृ॒थि॒वीम् दृꣳ॑ह । दृꣳ॒॒हेति॑ दृꣳह ॥ प्र॒जाप॑तिस्त्वा । प्र॒जाप॑ति॒रिति॑ प्र॒जा - प॒तिः॒ । त्वा॒ सा॒द॒य॒तु॒ । सा॒द॒य॒तु॒ पृ॒थि॒व्याः । पृ॒थि॒व्याः पृ॒ष्ठे । पृ॒ष्ठे व्यच॑स्वतीम् । व्यच॑स्वती॒म् प्रथ॑स्वतीम् । प्रथ॑स्वती॒म् प्रथः॑ । प्रथो॑ऽसि । अ॒सि॒ पृ॒थि॒वी । पृ॒थि॒व्य॑सि । अ॒सि॒ भूः । भूर॑सि । अ॒सि॒ भूमिः॑ । भूमि॑रसि । अ॒स्यदि॑तिः । अदि॑तिरसि । अ॒सि॒ वि॒श्वधा॑याः । वि॒श्वधा॑या॒ विश्व॑स्य । वि॒श्वधा॑या॒ इति॑ वि॒श्व - धा॒याः॒ । विश्व॑स्य॒ भुव॑नस्य । भुव॑नस्य ध॒र्त्री । ध॒र्त्री पृ॑थि॒वीम् । पृ॒थि॒वीं ॅय॑च्छ । य॒च्छ॒ पृ॒थि॒वीम् । पृ॒थि॒वीम् दृꣳ॑ह । दृꣳ॒॒ह॒ पृ॒थि॒वीम् । पृ॒थि॒वीम् मा । मा हिꣳ॑सीः । हिꣳ॒॒सी॒र् विश्व॑स्मै । विश्व॑स्मै प्रा॒णाय॑ । प्रा॒णाया॑पा॒नाय॑ । प्रा॒णायेति॑ प्र - अ॒नाय॑ । अ॒पा॒नाय॑ व्या॒नाय॑ । अ॒पा॒नायेत्य॑प - अ॒नाय॑ । व्या॒नायो॑दा॒नाय॑ । व्या॒नायेति॑ वि - अ॒नाय॑ । उ॒दा॒नाय॑ प्रति॒ष्ठायै᳚ । उ॒दा॒नायेत्यु॑त् - अ॒नाय॑ । प्र॒ति॒ष्ठायै॑ च॒रित्रा॑य । प्र॒ति॒ष्ठाया॒ इति॑ प्रति - स्थायै᳚ \newline

\textbf{Jatai Paata} \newline

1. ध्रु॒वा ऽस्य॑सि ध्रु॒वा ध्रु॒वा ऽसि॑ । \newline
2. अ॒सि॒ ध॒रुणा॑ ध॒रुणा᳚ ऽस्यसि ध॒रुणा᳚ । \newline
3. ध॒रुणा ऽस्तृ॒ता ऽस्तृ॑ता ध॒रुणा॑ ध॒रुणा ऽस्तृ॑ता । \newline
4. अस्तृ॑ता वि॒श्वक॑र्मणा वि॒श्वक॑र्म॒णा ऽस्तृ॒ता ऽस्तृ॑ता वि॒श्वक॑र्मणा । \newline
5. वि॒श्वक॑र्मणा॒ सुकृ॑ता॒ सुकृ॑ता वि॒श्वक॑र्मणा वि॒श्वक॑र्मणा॒ सुकृ॑ता । \newline
6. वि॒श्वक॑र्म॒णेति॑ वि॒श्व - क॒र्म॒णा॒ । \newline
7. सुकृ॒तेति॒ सु - कृ॒ता॒ । \newline
8. मा त्वा᳚ त्वा॒ मा मा त्वा᳚ । \newline
9. त्वा॒ स॒मु॒द्रः स॑मु॒द्र स्त्वा᳚ त्वा समु॒द्रः । \newline
10. स॒मु॒द्र उदुथ् स॑मु॒द्रः स॑मु॒द्र उत् । \newline
11. उद् व॑धीद् वधी॒ दुदुद् व॑धीत् । \newline
12. व॒धी॒न् मा मा व॑धीद् वधी॒न् मा । \newline
13. मा सु॑प॒र्णः सु॑प॒र्णो मा मा सु॑प॒र्णः । \newline
14. सु॒प॒र्णो ऽव्य॑थमा॒ना ऽव्य॑थमाना सुप॒र्णः सु॑प॒र्णो ऽव्य॑थमाना । \newline
15. सु॒प॒र्ण इति॑ सु - प॒र्णः । \newline
16. अव्य॑थमाना पृथि॒वीम् पृ॑थि॒वी मव्य॑थमा॒ना ऽव्य॑थमाना पृथि॒वीम् । \newline
17. पृ॒थि॒वीम् दृꣳ॑ह दृꣳह पृथि॒वीम् पृ॑थि॒वीम् दृꣳ॑ह । \newline
18. दृꣳ॒॒हेति॑ दृꣳह । \newline
19. प्र॒जाप॑ति स्त्वा त्वा प्र॒जाप॑तिः प्र॒जाप॑ति स्त्वा । \newline
20. प्र॒जाप॑ति॒रिति॑ प्र॒जा - प॒तिः॒ । \newline
21. त्वा॒ सा॒द॒य॒तु॒ सा॒द॒य॒तु॒ त्वा॒ त्वा॒ सा॒द॒य॒तु॒ । \newline
22. सा॒द॒य॒तु॒ पृ॒थि॒व्याः पृ॑थि॒व्याः सा॑दयतु सादयतु पृथि॒व्याः । \newline
23. पृ॒थि॒व्याः पृ॒ष्ठे पृ॒ष्ठे पृ॑थि॒व्याः पृ॑थि॒व्याः पृ॒ष्ठे । \newline
24. पृ॒ष्ठे व्यच॑स्वतीं॒ ॅव्यच॑स्वतीम् पृ॒ष्ठे पृ॒ष्ठे व्यच॑स्वतीम् । \newline
25. व्यच॑स्वती॒म् प्रथ॑स्वती॒म् प्रथ॑स्वतीं॒ ॅव्यच॑स्वतीं॒ ॅव्यच॑स्वती॒म् प्रथ॑स्वतीम् । \newline
26. प्रथ॑स्वती॒म् प्रथः॒ प्रथः॒ प्रथ॑स्वती॒म् प्रथ॑स्वती॒म् प्रथः॑ । \newline
27. प्रथो᳚ ऽस्यसि॒ प्रथः॒ प्रथो॑ ऽसि । \newline
28. अ॒सि॒ पृ॒थि॒वी पृ॑थि॒ व्य॑स्यसि पृथि॒वी । \newline
29. पृ॒थि॒ व्य॑स्यसि पृथि॒वी पृ॑थि॒ व्य॑सि । \newline
30. अ॒सि॒ भूर् भू र॑स्यसि॒ भूः । \newline
31. भू र॑स्यसि॒ भूर् भूर॑सि । \newline
32. अ॒सि॒ भूमि॒र् भूमि॑ रस्यसि॒ भूमिः॑ । \newline
33. भूमि॑ रस्यसि॒ भूमि॒र् भूमि॑ रसि । \newline
34. अ॒स्यदि॑ति॒ रदि॑ति रस्य॒ स्यदि॑तिः । \newline
35. अदि॑ति रस्य॒ स्यदि॑ति॒ रदि॑ति रसि । \newline
36. अ॒सि॒ वि॒श्वधा॑या वि॒श्वधा॑या अस्यसि वि॒श्वधा॑याः । \newline
37. वि॒श्वधा॑या॒ विश्व॑स्य॒ विश्व॑स्य वि॒श्वधा॑या वि॒श्वधा॑या॒ विश्व॑स्य । \newline
38. वि॒श्वधा॑या॒ इति॑ वि॒श्व - धा॒याः॒ । \newline
39. विश्व॑स्य॒ भुव॑नस्य॒ भुव॑नस्य॒ विश्व॑स्य॒ विश्व॑स्य॒ भुव॑नस्य । \newline
40. भुव॑नस्य ध॒र्त्री ध॒र्त्री भुव॑नस्य॒ भुव॑नस्य ध॒र्त्री । \newline
41. ध॒र्त्री पृ॑थि॒वीम् पृ॑थि॒वीम् ध॒र्त्री ध॒र्त्री पृ॑थि॒वीम् । \newline
42. पृ॒थि॒वीं ॅय॑च्छ यच्छ पृथि॒वीम् पृ॑थि॒वीं ॅय॑च्छ । \newline
43. य॒च्छ॒ पृ॒थि॒वीम् पृ॑थि॒वीं ॅय॑च्छ यच्छ पृथि॒वीम् । \newline
44. पृ॒थि॒वीम् दृꣳ॑ह दृꣳह पृथि॒वीम् पृ॑थि॒वीम् दृꣳ॑ह । \newline
45. दृꣳ॒॒ह॒ पृ॒थि॒वीम् पृ॑थि॒वीम् दृꣳ॑ह दृꣳह पृथि॒वीम् । \newline
46. पृ॒थि॒वीम् मा मा पृ॑थि॒वीम् पृ॑थि॒वीम् मा । \newline
47. मा हिꣳ॑सीर्. हिꣳसी॒र् मा मा हिꣳ॑सीः । \newline
48. हिꣳ॒॒सी॒र् विश्व॑स्मै॒ विश्व॑स्मै हिꣳसीर्. हिꣳसी॒र् विश्व॑स्मै । \newline
49. विश्व॑स्मै प्रा॒णाय॑ प्रा॒णाय॒ विश्व॑स्मै॒ विश्व॑स्मै प्रा॒णाय॑ । \newline
50. प्रा॒णाया॑ पा॒नाया॑ पा॒नाय॑ प्रा॒णाय॑ प्रा॒णाया॑ पा॒नाय॑ । \newline
51. प्रा॒णायेति॑ प्र - अ॒नाय॑ । \newline
52. अ॒पा॒नाय॑ व्या॒नाय॑ व्या॒नाया॑ पा॒नाया॑ पा॒नाय॑ व्या॒नाय॑ । \newline
53. अ॒पा॒नायेत्य॑प - अ॒नाय॑ । \newline
54. व्या॒ना यो॑दा॒ना यो॑दा॒नाय॑ व्या॒नाय॑ व्या॒ना यो॑दा॒नाय॑ । \newline
55. व्या॒नायेति॑ वि - अ॒नाय॑ । \newline
56. उ॒दा॒नाय॑ प्रति॒ष्ठायै᳚ प्रति॒ष्ठाया॑ उदा॒ना यो॑दा॒नाय॑ प्रति॒ष्ठायै᳚ । \newline
57. उ॒दा॒नायेत्यु॑त् - अ॒नाय॑ । \newline
58. प्र॒ति॒ष्ठायै॑ च॒रित्रा॑य च॒रित्रा॑य प्रति॒ष्ठायै᳚ प्रति॒ष्ठायै॑ च॒रित्रा॑य । \newline
59. प्र॒ति॒ष्ठाया॒ इति॑ प्रति - स्थायै᳚ । \newline

\textbf{Ghana Paata } \newline

1. ध्रु॒वा ऽस्य॑सि ध्रु॒वा ध्रु॒वा ऽसि॑ ध॒रुणा॑ ध॒रुणा॑ ऽसि ध्रु॒वा ध्रु॒वा ऽसि॑ ध॒रुणा᳚ । \newline
2. अ॒सि॒ ध॒रुणा॑ ध॒रुणा᳚ ऽस्यसि ध॒रुणा ऽस्तृ॒ता ऽस्तृ॑ता ध॒रुणा᳚ ऽस्यसि ध॒रुणा ऽस्तृ॑ता । \newline
3. ध॒रुणा ऽस्तृ॒ता ऽस्तृ॑ता ध॒रुणा॑ ध॒रुणा ऽस्तृ॑ता वि॒श्वक॑र्मणा वि॒श्वक॑र्म॒णा ऽस्तृ॑ता ध॒रुणा॑ ध॒रुणा ऽस्तृ॑ता वि॒श्वक॑र्मणा । \newline
4. अस्तृ॑ता वि॒श्वक॑र्मणा वि॒श्वक॑र्म॒णा ऽस्तृ॒ता ऽस्तृ॑ता वि॒श्वक॑र्मणा॒ सुकृ॑ता॒ सुकृ॑ता वि॒श्वक॑र्म॒णा ऽस्तृ॒ता ऽस्तृ॑ता वि॒श्वक॑र्मणा॒ सुकृ॑ता । \newline
5. वि॒श्वक॑र्मणा॒ सुकृ॑ता॒ सुकृ॑ता वि॒श्वक॑र्मणा वि॒श्वक॑र्मणा॒ सुकृ॑ता । \newline
6. वि॒श्वक॑र्म॒णेति॑ वि॒श्व - क॒र्म॒णा॒ । \newline
7. सुकृ॒तेति॒ सु - कृ॒ता॒ । \newline
8. मा त्वा᳚ त्वा॒ मा मा त्वा॑ समु॒द्रः स॑मु॒द्र स्त्वा॒ मा मा त्वा॑ समु॒द्रः । \newline
9. त्वा॒ स॒मु॒द्रः स॑मु॒द्र स्त्वा᳚ त्वा समु॒द्र उदुथ् स॑मु॒द्र स्त्वा᳚ त्वा समु॒द्र उत् । \newline
10. स॒मु॒द्र उदुथ् स॑मु॒द्रः स॑मु॒द्र उद् व॑धीद् वधी॒ दुथ् स॑मु॒द्रः स॑मु॒द्र उद् व॑धीत् । \newline
11. उद् व॑धीद् वधी॒ दुदुद् व॑धी॒न् मा मा व॑धी॒ दुदुद् व॑धी॒न् मा । \newline
12. व॒धी॒न् मा मा व॑धीद् वधी॒न् मा सु॑प॒र्णः सु॑प॒र्णो मा व॑धीद् वधी॒न् मा सु॑प॒र्णः । \newline
13. मा सु॑प॒र्णः सु॑प॒र्णो मा मा सु॑प॒र्णो ऽव्य॑थमा॒ना ऽव्य॑थमाना सुप॒र्णो मा मा सु॑प॒र्णो ऽव्य॑थमाना । \newline
14. सु॒प॒र्णो ऽव्य॑थमा॒ना ऽव्य॑थमाना सुप॒र्णः सु॑प॒र्णो ऽव्य॑थमाना पृथि॒वीम् पृ॑थि॒वी मव्य॑थमाना सुप॒र्णः सु॑प॒र्णो ऽव्य॑थमाना पृथि॒वीम् । \newline
15. सु॒प॒र्ण इति॑ सु - प॒र्णः । \newline
16. अव्य॑थमाना पृथि॒वीम् पृ॑थि॒वी मव्य॑थमा॒ना ऽव्य॑थमाना पृथि॒वीम् दृꣳ॑ह दृꣳह पृथि॒वी मव्य॑थमा॒ना ऽव्य॑थमाना पृथि॒वीम् दृꣳ॑ह । \newline
17. पृ॒थि॒वीम् दृꣳ॑ह दृꣳह पृथि॒वीम् पृ॑थि॒वीम् दृꣳ॑ह । \newline
18. दृꣳ॒॒हेति॑ दृꣳह । \newline
19. प्र॒जाप॑ति स्त्वा त्वा प्र॒जाप॑तिः प्र॒जाप॑ति स्त्वा सादयतु सादयतु त्वा प्र॒जाप॑तिः प्र॒जाप॑ति स्त्वा सादयतु । \newline
20. प्र॒जाप॑ति॒रिति॑ प्र॒जा - प॒तिः॒ । \newline
21. त्वा॒ सा॒द॒य॒तु॒ सा॒द॒य॒तु॒ त्वा॒ त्वा॒ सा॒द॒य॒तु॒ पृ॒थि॒व्याः पृ॑थि॒व्याः सा॑दयतु त्वा त्वा सादयतु पृथि॒व्याः । \newline
22. सा॒द॒य॒तु॒ पृ॒थि॒व्याः पृ॑थि॒व्याः सा॑दयतु सादयतु पृथि॒व्याः पृ॒ष्ठे पृ॒ष्ठे पृ॑थि॒व्याः सा॑दयतु सादयतु पृथि॒व्याः पृ॒ष्ठे । \newline
23. पृ॒थि॒व्याः पृ॒ष्ठे पृ॒ष्ठे पृ॑थि॒व्याः पृ॑थि॒व्याः पृ॒ष्ठे व्यच॑स्वतीं॒ ॅव्यच॑स्वतीम् पृ॒ष्ठे पृ॑थि॒व्याः पृ॑थि॒व्याः पृ॒ष्ठे व्यच॑स्वतीम् । \newline
24. पृ॒ष्ठे व्यच॑स्वतीं॒ ॅव्यच॑स्वतीम् पृ॒ष्ठे पृ॒ष्ठे व्यच॑स्वती॒म् प्रथ॑स्वती॒म् प्रथ॑स्वतीं॒ ॅव्यच॑स्वतीम् पृ॒ष्ठे पृ॒ष्ठे व्यच॑स्वती॒म् प्रथ॑स्वतीम् । \newline
25. व्यच॑स्वती॒म् प्रथ॑स्वती॒म् प्रथ॑स्वतीं॒ ॅव्यच॑स्वतीं॒ ॅव्यच॑स्वती॒म् प्रथ॑स्वती॒म् प्रथः॒ प्रथः॒ प्रथ॑स्वतीं॒ ॅव्यच॑स्वतीं॒ ॅव्यच॑स्वती॒म् प्रथ॑स्वती॒म् प्रथः॑ । \newline
26. प्रथ॑स्वती॒म् प्रथः॒ प्रथः॒ प्रथ॑स्वती॒म् प्रथ॑स्वती॒म् प्रथो᳚ ऽस्यसि॒ प्रथः॒ प्रथ॑स्वती॒म् प्रथ॑स्वती॒म् प्रथो॑ ऽसि । \newline
27. प्रथो᳚ ऽस्यसि॒ प्रथः॒ प्रथो॑ ऽसि पृथि॒वी पृ॑थि॒व्य॑सि॒ प्रथः॒ प्रथो॑ ऽसि पृथि॒वी । \newline
28. अ॒सि॒ पृ॒थि॒वी पृ॑थि॒व्य॑स्यसि पृथि॒व्य॑स्यसि पृथि॒व्य॑स्यसि पृथि॒व्य॑सि । \newline
29. पृ॒थि॒व्य॑स्यसि पृथि॒वी पृ॑थि॒व्य॑सि॒ भूर् भूर॑सि पृथि॒वी पृ॑थि॒व्य॑सि॒ भूः । \newline
30. अ॒सि॒ भूर् भूर॑स्यसि॒ भूर॑स्यसि॒ भूर॑स्यसि॒ भूर॑सि । \newline
31. भूर॑स्यसि॒ भूर् भूर॑सि॒ भूमि॒र् भूमि॑ रसि॒ भूर् भूर॑सि॒ भूमिः॑ । \newline
32. अ॒सि॒ भूमि॒र् भूमि॑ रस्यसि॒ भूमि॑ रस्यसि॒ भूमि॑ रस्यसि॒ भूमि॑ रसि । \newline
33. भूमि॑ रस्यसि॒ भूमि॒र् भूमि॑ र॒स्य दि॑ति॒ रदि॑ति रसि॒ भूमि॒र् भूमि॑ र॒स्य दि॑तिः । \newline
34. अ॒स्यदि॑ति॒ रदि॑ति रस्य॒स्य दि॑ति रस्य॒स्य दि॑ति रस्य॒स्य दि॑ति रसि । \newline
35. अदि॑ति रस्य॒स्य दि॑ति॒ रदि॑ति रसि वि॒श्वधा॑या वि॒श्वधा॑या अ॒स्य दि॑ति॒ रदि॑ति रसि वि॒श्वधा॑याः । \newline
36. अ॒सि॒ वि॒श्वधा॑या वि॒श्वधा॑या अस्यसि वि॒श्वधा॑या॒ विश्व॑स्य॒ विश्व॑स्य वि॒श्वधा॑या अस्यसि वि॒श्वधा॑या॒ विश्व॑स्य । \newline
37. वि॒श्वधा॑या॒ विश्व॑स्य॒ विश्व॑स्य वि॒श्वधा॑या वि॒श्वधा॑या॒ विश्व॑स्य॒ भुव॑नस्य॒ भुव॑नस्य॒ विश्व॑स्य वि॒श्वधा॑या वि॒श्वधा॑या॒ विश्व॑स्य॒ भुव॑नस्य । \newline
38. वि॒श्वधा॑या॒ इति॑ वि॒श्व - धा॒याः॒ । \newline
39. विश्व॑स्य॒ भुव॑नस्य॒ भुव॑नस्य॒ विश्व॑स्य॒ विश्व॑स्य॒ भुव॑नस्य ध॒र्त्री ध॒र्त्री भुव॑नस्य॒ विश्व॑स्य॒ विश्व॑स्य॒ भुव॑नस्य ध॒र्त्री । \newline
40. भुव॑नस्य ध॒र्त्री ध॒र्त्री भुव॑नस्य॒ भुव॑नस्य ध॒र्त्री पृ॑थि॒वीम् पृ॑थि॒वीम् ध॒र्त्री भुव॑नस्य॒ भुव॑नस्य ध॒र्त्री पृ॑थि॒वीम् । \newline
41. ध॒र्त्री पृ॑थि॒वीम् पृ॑थि॒वीम् ध॒र्त्री ध॒र्त्री पृ॑थि॒वीं ॅय॑च्छ यच्छ पृथि॒वीम् ध॒र्त्री ध॒र्त्री पृ॑थि॒वीं ॅय॑च्छ । \newline
42. पृ॒थि॒वीं ॅय॑च्छ यच्छ पृथि॒वीम् पृ॑थि॒वीं ॅय॑च्छ पृथि॒वीम् पृ॑थि॒वीं ॅय॑च्छ पृथि॒वीम् पृ॑थि॒वीं ॅय॑च्छ पृथि॒वीम् । \newline
43. य॒च्छ॒ पृ॒थि॒वीम् पृ॑थि॒वीं ॅय॑च्छ यच्छ पृथि॒वीम् दृꣳ॑ह दृꣳह पृथि॒वीं ॅय॑च्छ यच्छ पृथि॒वीम् दृꣳ॑ह । \newline
44. पृ॒थि॒वीम् दृꣳ॑ह दृꣳह पृथि॒वीम् पृ॑थि॒वीम् दृꣳ॑ह पृथि॒वीम् पृ॑थि॒वीम् दृꣳ॑ह पृथि॒वीम् पृ॑थि॒वीम् दृꣳ॑ह पृथि॒वीम् । \newline
45. दृꣳ॒॒ह॒ पृ॒थि॒वीम् पृ॑थि॒वीम् दृꣳ॑ह दृꣳह पृथि॒वीम् मा मा पृ॑थि॒वीम् दृꣳ॑ह दृꣳह पृथि॒वीम् मा । \newline
46. पृ॒थि॒वीम् मा मा पृ॑थि॒वीम् पृ॑थि॒वीम् मा हिꣳ॑सीर्. हिꣳसी॒र् मा पृ॑थि॒वीम् पृ॑थि॒वीम् मा हिꣳ॑सीः । \newline
47. मा हिꣳ॑सीर्. हिꣳसी॒र् मा मा हिꣳ॑सी॒र् विश्व॑स्मै॒ विश्व॑स्मै हिꣳसी॒र् मा मा हिꣳ॑सी॒र् विश्व॑स्मै । \newline
48. हिꣳ॒॒सी॒र् विश्व॑स्मै॒ विश्व॑स्मै हिꣳसीर्. हिꣳसी॒र् विश्व॑स्मै प्रा॒णाय॑ प्रा॒णाय॒ विश्व॑स्मै हिꣳसीर्. हिꣳसी॒र् विश्व॑स्मै प्रा॒णाय॑ । \newline
49. विश्व॑स्मै प्रा॒णाय॑ प्रा॒णाय॒ विश्व॑स्मै॒ विश्व॑स्मै प्रा॒णाया॑ पा॒नाया॑ पा॒नाय॑ प्रा॒णाय॒ विश्व॑स्मै॒ विश्व॑स्मै प्रा॒णाया॑ पा॒नाय॑ । \newline
50. प्रा॒णाया॑ पा॒नाया॑ पा॒नाय॑ प्रा॒णाय॑ प्रा॒णाया॑ पा॒नाय॑ व्या॒नाय॑ व्या॒नाया॑ पा॒नाय॑ प्रा॒णाय॑ प्रा॒णाया॑ पा॒नाय॑ व्या॒नाय॑ । \newline
51. प्रा॒णायेति॑ प्र - अ॒नाय॑ । \newline
52. अ॒पा॒नाय॑ व्या॒नाय॑ व्या॒नाया॑ पा॒नाया॑ पा॒नाय॑ व्या॒ना यो॑दा॒ना यो॑दा॒नाय॑ व्या॒नाया॑ पा॒नाया॑ पा॒नाय॑ व्या॒ना यो॑दा॒नाय॑ । \newline
53. अ॒पा॒नायेत्य॑प - अ॒नाय॑ । \newline
54. व्या॒ना यो॑दा॒ना यो॑दा॒नाय॑ व्या॒नाय॑ व्या॒ना यो॑दा॒नाय॑ प्रति॒ष्ठायै᳚ प्रति॒ष्ठाया॑ उदा॒नाय॑ व्या॒नाय॑ व्या॒ना यो॑दा॒नाय॑ प्रति॒ष्ठायै᳚ । \newline
55. व्या॒नायेति॑ वि - अ॒नाय॑ । \newline
56. उ॒दा॒नाय॑ प्रति॒ष्ठायै᳚ प्रति॒ष्ठाया॑ उदा॒ना यो॑दा॒नाय॑ प्रति॒ष्ठायै॑ च॒रित्रा॑य च॒रित्रा॑य प्रति॒ष्ठाया॑ उदा॒ना यो॑दा॒नाय॑ प्रति॒ष्ठायै॑ च॒रित्रा॑य । \newline
57. उ॒दा॒नायेत्यु॑त् - अ॒नाय॑ । \newline
58. प्र॒ति॒ष्ठायै॑ च॒रित्रा॑य च॒रित्रा॑य प्रति॒ष्ठायै᳚ प्रति॒ष्ठायै॑ च॒रित्रा॑या॒ग्नि र॒ग्नि श्च॒रित्रा॑य प्रति॒ष्ठायै᳚ प्रति॒ष्ठायै॑ च॒रित्रा॑या॒ग्निः । \newline
59. प्र॒ति॒ष्ठाया॒ इति॑ प्रति - स्थायै᳚ । \newline
\pagebreak
\markright{ TS 4.2.9.2  \hfill https://www.vedavms.in \hfill}

\section{ TS 4.2.9.2 }

\textbf{TS 4.2.9.2 } \newline
\textbf{Samhita Paata} \newline

च॒रित्रा॑या॒-ग्निस्त्वा॒ऽभि पा॑तु म॒ह्या स्व॒स्त्या छ॒र्दिषा॒ शन्त॑मेन॒ तया॑ दे॒वत॑याऽङ्गिर॒स्वद्-ध्रु॒वा सी॑द ॥ काण्डा᳚त् काण्डात् प्र॒रोह॑न्ती॒ परु॑षःपरुषः॒ परि॑ । ए॒वा नो॑ दूर्वे॒ प्र त॑नु स॒हस्रे॑ण श॒तेन॑ च ॥ या श॒तेन॑ प्रत॒नोषि॑ स॒हस्रे॑ण वि॒रोह॑सि ।तस्या᳚स्ते देवीष्टके वि॒धेम॑ ह॒विषा॑ व॒यं ॥ अषा॑ढाऽसि॒ सह॑माना॒ सह॒स्वारा॑तीः॒ सह॑स्वारातीय॒तः सह॑स्व॒ पृत॑नाः॒ सह॑स्व पृतन्य॒तः । स॒हस्र॑वीर्या - [  ] \newline

\textbf{Pada Paata} \newline

च॒रित्रा॑य । अ॒ग्निः । त्वा॒ । अ॒भीति॑ । पा॒तु॒ । म॒ह्या । स्व॒स्त्या । छ॒र्दिषा᳚ । शन्त॑मे॒नेति॒ शं - त॒मे॒न॒ । तया᳚ । दे॒वत॑या । अ॒ङ्गि॒र॒स्वत् । ध्रु॒वा । सी॒द॒ ॥ काण्डा᳚त्काण्डा॒दिति॒ काण्डात् - का॒ण्डा॒त् । प्र॒रोह॒न्तीति॑ प्र - रोह॑न्ती । परु॑षःपरुष॒ इति॒ परु॑षः-प॒रु॒षः॒ । परि॑ ॥ ए॒वा । नः॒ । दू॒र्वे॒ । प्रेति॑ । त॒नु॒ । स॒हस्रे॑ण । श॒तेन॑ । च॒ ॥ या । श॒तेन॑ । प्र॒त॒नोषीति॑ प्र - त॒नोषि॑ । स॒हस्रे॑ण । वि॒रोह॒सीति॑ वि-रोह॑सि ॥ तस्याः᳚ । ते॒ । दे॒वि॒ । इ॒ष्ट॒के॒ । वि॒धेम॑ । ह॒विषा᳚ । व॒यम् ॥ अषा॑ढा । अ॒सि॒ । सह॑माना । सह॑स्व । अरा॑तीः । सह॑स्व । अ॒रा॒ती॒य॒तः । सह॑स्व । पृत॑नाः । सह॑स्व । पृ॒त॒न्य॒तः ॥ स॒हस्र॑वी॒र्येति॑ स॒हस्र॑ - वी॒र्या॒ ।  \newline


\textbf{Krama Paata} \newline

च॒रित्रा॑या॒ग्निः । अ॒ग्निस्त्वा᳚ । त्वा॒ऽभि । अ॒भि पा॑तु । पा॒तु॒ म॒ह्या । म॒ह्या स्व॒स्त्या । स्व॒स्त्या छ॒र्दिषा᳚ । छ॒र्दिषा॒ शन्त॑मेन । शन्त॑मेन॒ तया᳚ । शन्त॑मे॒नेति॒ शम् - त॒मे॒न॒ । तया॑ दे॒वत॑या । दे॒वत॑या ऽङ्गिर॒स्वत् । अ॒ङ्गि॒र॒स्वद् ध्रु॒वा । ध्रु॒वा सी॑द । सी॒देति॑ सीद ॥ काण्डा᳚त्काण्डात् प्र॒रोह॑न्ती । काण्डा᳚त्काण्डा॒दिति॒ काण्डा᳚त् - का॒ण्डा॒त्॒ । प्र॒रोह॑न्ती॒ परु॑षःपरुषः । प्र॒रोह॒न्तीति॑ प्र - रोह॑न्ती । परु॑षःपरुषः॒ परि॑ । परु॑षःपरुष॒ इति॒ परु॑षः - प॒रु॒षः॒ । परीति॒ परि॑ ॥ ए॒वा नः॑ । नो॒ दू॒र्वे॒ । दू॒र्वे॒ प्र । प्र त॑नु । त॒नु॒ स॒हस्रे॑ण । स॒हस्रे॑ण श॒तेन॑ । श॒तेन॑ च । चेति॑ च ॥ या श॒तेन॑ । श॒तेन॑ प्रत॒नोषि॑ । प्र॒त॒नोषि॑ स॒हस्रे॑ण । प्र॒त॒नोषीति॑ प्र - त॒नोषि॑ । स॒हस्रे॑ण वि॒रोह॑सि । वि॒रोह॒सीति॑ वि - रोह॑सि ॥ तस्या᳚स्ते । ते॒ दे॒वि॒ । दे॒वी॒ष्ट॒के॒ । इ॒ष्ट॒के॒ वि॒धेम॑ । वि॒धेम॑ ह॒विषा᳚ । ह॒विषा॑ व॒यम् । व॒यमिति॑ व॒यम् ॥ अषा॑ढाऽसि । अ॒सि॒ सह॑माना । सह॑माना॒ सह॑स्व । सह॒स्वारा॑तीः । अरा॑तीः॒ सह॑स्व । सह॑स्वारातीय॒तः । अ॒रा॒ती॒य॒तः सह॑स्व । सह॑स्व॒ पृत॑नाः । पृत॑नाः॒ सह॑स्व । सह॑स्व पृतन्य॒तः । पृ॒त॒न्य॒त इति॑ पृतन्य॒तः ॥ स॒हस्र॑वीर्याऽसि । स॒हस्र॑वी॒र्येति॑ स॒हस्र॑ - वी॒र्या॒ \newline

\textbf{Jatai Paata} \newline

1. च॒रित्रा॑या॒ग्नि र॒ग्नि श्च॒रित्रा॑य च॒रित्रा॑या॒ग्निः । \newline
2. अ॒ग्नि स्त्वा᳚ त्वा॒ ऽग्नि र॒ग्नि स्त्वा᳚ । \newline
3. त्वा॒ ऽभ्य॑भि त्वा᳚ त्वा॒ ऽभि । \newline
4. अ॒भि पा॑तु पात्व॒भ्य॑भि पा॑तु । \newline
5. पा॒तु॒ म॒ह्या म॒ह्या पा॑तु पातु म॒ह्या । \newline
6. म॒ह्या स्व॒स्त्या स्व॒स्त्या म॒ह्या म॒ह्या स्व॒स्त्या । \newline
7. स्व॒स्त्या छ॒र्दिषा॑ छ॒र्दिषा᳚ स्व॒स्त्या स्व॒स्त्या छ॒र्दिषा᳚ । \newline
8. छ॒र्दिषा॒ शन्त॑मेन॒ शन्त॑मेन छ॒र्दिषा॑ छ॒र्दिषा॒ शन्त॑मेन । \newline
9. शन्त॑मेन॒ तया॒ तया॒ शन्त॑मेन॒ शन्त॑मेन॒ तया᳚ । \newline
10. शन्त॑मे॒नेति॒ शं - त॒मे॒न॒ । \newline
11. तया॑ दे॒वत॑या दे॒वत॑या॒ तया॒ तया॑ दे॒वत॑या । \newline
12. दे॒वत॑या ऽङ्गिर॒स्व द॑ङ्गिर॒स्वद् दे॒वत॑या दे॒वत॑या ऽङ्गिर॒स्वत् । \newline
13. अ॒ङ्गि॒र॒स्वद् ध्रु॒वा ध्रु॒वा ऽङ्गि॑र॒स्व द॑ङ्गिर॒स्वद् ध्रु॒वा । \newline
14. ध्रु॒वा सी॑द सीद ध्रु॒वा ध्रु॒वा सी॑द । \newline
15. सी॒देति॑ सीद । \newline
16. काण्डा᳚त्काण्डात् प्र॒रोह॑न्ती प्र॒रोह॑न्ती॒ काण्डा᳚त्काण्डा॒त् काण्डा᳚त्काण्डात् प्र॒रोह॑न्ती । \newline
17. काण्डा᳚त्काण्डा॒दिति॒ काण्डा᳚त् - का॒ण्डा॒त् । \newline
18. प्र॒रोह॑न्ती॒ परु॑षःपरुषः॒ परु॑षःपरुषः प्र॒रोह॑न्ती प्र॒रोह॑न्ती॒ परु॑षःपरुषः । \newline
19. प्र॒रोह॒न्तीति॑ प्र - रोह॑न्ती । \newline
20. परु॑षःपरुषः॒ परि॒ परि॒ परु॑षःपरुषः॒ परु॑षःपरुषः॒ परि॑ । \newline
21. परु॑षःपरुष॒ इति॒ परु॑षः - प॒रु॒षः॒ । \newline
22. परीति॒ परि॑ । \newline
23. ए॒वा नो॑ न ए॒वैवा नः॑ । \newline
24. नो॒ दू॒र्वे॒ दू॒र्वे॒ नो॒ नो॒ दू॒र्वे॒ । \newline
25. दू॒र्वे॒ प्र प्र दू᳚र्वे दूर्वे॒ प्र । \newline
26. प्र त॑नु तनु॒ प्र प्र त॑नु । \newline
27. त॒नु॒ स॒हस्रे॑ण स॒हस्रे॑ण तनु तनु स॒हस्रे॑ण । \newline
28. स॒हस्रे॑ण श॒तेन॑ श॒तेन॑ स॒हस्रे॑ण स॒हस्रे॑ण श॒तेन॑ । \newline
29. श॒तेन॑ च च श॒तेन॑ श॒तेन॑ च । \newline
30. चेति॑ च । \newline
31. या श॒तेन॑ श॒तेन॒ या या श॒तेन॑ । \newline
32. श॒तेन॑ प्रत॒नोषि॑ प्रत॒नोषि॑ श॒तेन॑ श॒तेन॑ प्रत॒नोषि॑ । \newline
33. प्र॒त॒नोषि॑ स॒हस्रे॑ण स॒हस्रे॑ण प्रत॒नोषि॑ प्रत॒नोषि॑ स॒हस्रे॑ण । \newline
34. प्र॒त॒नोषीति॑ प्र - त॒नोषि॑ । \newline
35. स॒हस्रे॑ण वि॒रोह॑सि वि॒रोह॑सि स॒हस्रे॑ण स॒हस्रे॑ण वि॒रोह॑सि । \newline
36. वि॒रोह॒सीति॑ वि - रोह॑सि । \newline
37. तस्या᳚ स्ते ते॒ तस्या॒ स्तस्या᳚ स्ते । \newline
38. ते॒ दे॒वि॒ दे॒वि॒ ते॒ ते॒ दे॒वि॒ । \newline
39. दे॒वी॒ष्ट॒क॒ इ॒ष्ट॒के॒ दे॒वि॒ दे॒वी॒ष्ट॒के॒ । \newline
40. इ॒ष्ट॒के॒ वि॒धेम॑ वि॒धेमे᳚ ष्टक इष्टके वि॒धेम॑ । \newline
41. वि॒धेम॑ ह॒विषा॑ ह॒विषा॑ वि॒धेम॑ वि॒धेम॑ ह॒विषा᳚ । \newline
42. ह॒विषा॑ व॒यं ॅव॒यꣳ ह॒विषा॑ ह॒विषा॑ व॒यम् । \newline
43. व॒यमिति॑ व॒यम् । \newline
44. अषा॑ढा ऽस्य॒ स्यषा॒ढा ऽषा॑ढा ऽसि । \newline
45. अ॒सि॒ सह॑माना॒ सह॑माना ऽस्यसि॒ सह॑माना । \newline
46. सह॑माना॒ सह॑स्व॒ सह॑स्व॒ सह॑माना॒ सह॑माना॒ सह॑स्व । \newline
47. सह॒स्वा रा॑ती॒ ररा॑तीः॒ सह॑स्व॒ सह॒स्वा रा॑तीः । \newline
48. अरा॑तीः॒ सह॑स्व॒ सह॒स्वा रा॑ती॒र रा॑तीः॒ सह॑स्व । \newline
49. सह॑स्वा रातीय॒तो॑ ऽरातीय॒तः सह॑स्व॒ सह॑स्वा रातीय॒तः । \newline
50. अ॒रा॒ती॒य॒तः सह॑स्व॒ सह॑स्वा रातीय॒तो॑ ऽरातीय॒तः सह॑स्व । \newline
51. सह॑स्व॒ पृत॑नाः॒ पृत॑नाः॒ सह॑स्व॒ सह॑स्व॒ पृत॑नाः । \newline
52. पृत॑नाः॒ सह॑स्व॒ सह॑स्व॒ पृत॑नाः॒ पृत॑नाः॒ सह॑स्व । \newline
53. सह॑स्व पृतन्य॒तः पृ॑तन्य॒तः सह॑स्व॒ सह॑स्व पृतन्य॒तः । \newline
54. पृ॒त॒न्य॒त इति॑ पृतन्य॒तः । \newline
55. स॒हस्र॑वीर्या ऽस्यसि स॒हस्र॑वीर्या स॒हस्र॑वीर्या ऽसि । \newline
56. स॒हस्र॑वी॒र्येति॑ स॒हस्र॑ - वी॒र्या॒ । \newline

\textbf{Ghana Paata } \newline

1. च॒रित्रा॑या॒ग्नि र॒ग्नि श्च॒रित्रा॑य च॒रित्रा॑या॒ग्नि स्त्वा᳚ त्वा॒ ऽग्निश्च॒रित्रा॑य च॒रित्रा॑या॒ग्नि स्त्वा᳚ । \newline
2. अ॒ग्नि स्त्वा᳚ त्वा॒ ऽग्नि र॒ग्नि स्त्वा॒ ऽभ्य॑भि त्वा॒ ऽग्नि र॒ग्नि स्त्वा॒ ऽभि । \newline
3. त्वा॒ ऽभ्य॑भि त्वा᳚ त्वा॒ ऽभि पा॑तु पात्व॒भि त्वा᳚ त्वा॒ ऽभि पा॑तु । \newline
4. अ॒भि पा॑तु पात्व॒भ्य॑भि पा॑तु म॒ह्या म॒ह्या पा᳚त्व॒भ्य॑भि पा॑तु म॒ह्या । \newline
5. पा॒तु॒ म॒ह्या म॒ह्या पा॑तु पातु म॒ह्या स्व॒स्त्या स्व॒स्त्या म॒ह्या पा॑तु पातु म॒ह्या स्व॒स्त्या । \newline
6. म॒ह्या स्व॒स्त्या स्व॒स्त्या म॒ह्या म॒ह्या स्व॒स्त्या छ॒र्दिषा॑ छ॒र्दिषा᳚ स्व॒स्त्या म॒ह्या म॒ह्या स्व॒स्त्या छ॒र्दिषा᳚ । \newline
7. स्व॒स्त्या छ॒र्दिषा॑ छ॒र्दिषा᳚ स्व॒स्त्या स्व॒स्त्या छ॒र्दिषा॒ शन्त॑मेन॒ शन्त॑मेन छ॒र्दिषा᳚ स्व॒स्त्या स्व॒स्त्या छ॒र्दिषा॒ शन्त॑मेन । \newline
8. छ॒र्दिषा॒ शन्त॑मेन॒ शन्त॑मेन छ॒र्दिषा॑ छ॒र्दिषा॒ शन्त॑मेन॒ तया॒ तया॒ शन्त॑मेन छ॒र्दिषा॑ छ॒र्दिषा॒ शन्त॑मेन॒ तया᳚ । \newline
9. शन्त॑मेन॒ तया॒ तया॒ शन्त॑मेन॒ शन्त॑मेन॒ तया॑ दे॒वत॑या दे॒वत॑या॒ तया॒ शन्त॑मेन॒ शन्त॑मेन॒ तया॑ दे॒वत॑या । \newline
10. शन्त॑मे॒नेति॒ शं - त॒मे॒न॒ । \newline
11. तया॑ दे॒वत॑या दे॒वत॑या॒ तया॒ तया॑ दे॒वत॑या ऽङ्गिर॒स्व द॑ङ्गिर॒स्वद् दे॒वत॑या॒ तया॒ तया॑ दे॒वत॑या ऽङ्गिर॒स्वत् । \newline
12. दे॒वत॑या ऽङ्गिर॒स्व द॑ङ्गिर॒स्वद् दे॒वत॑या दे॒वत॑या ऽङ्गिर॒स्वद् ध्रु॒वा ध्रु॒वा ऽङ्गि॑र॒स्वद् दे॒वत॑या दे॒वत॑या ऽङ्गिर॒स्वद् ध्रु॒वा । \newline
13. अ॒ङ्गि॒र॒स्वद् ध्रु॒वा ध्रु॒वा ऽङ्गि॑र॒स्व द॑ङ्गिर॒स्वद् ध्रु॒वा सी॑द सीद ध्रु॒वा ऽङ्गि॑र॒स्व द॑ङ्गिर॒स्वद् ध्रु॒वा सी॑द । \newline
14. ध्रु॒वा सी॑द सीद ध्रु॒वा ध्रु॒वा सी॑द । \newline
15. सी॒देति॑ सीद । \newline
16. काण्डा᳚त्काण्डात् प्र॒रोह॑न्ती प्र॒रोह॑न्ती॒ काण्डा᳚त्काण्डा॒त् काण्डा᳚त्काण्डात् प्र॒रोह॑न्ती॒ परु॑षःपरुषः॒ परु॑षःपरुषः प्र॒रोह॑न्ती॒ काण्डा᳚त्काण्डा॒त् काण्डा᳚त्काण्डात् प्र॒रोह॑न्ती॒ परु॑षःपरुषः । \newline
17. काण्डा᳚त्काण्डा॒दिति॒ काण्डा᳚त् - का॒ण्डा॒त् । \newline
18. प्र॒रोह॑न्ती॒ परु॑षःपरुषः॒ परु॑षःपरुषः प्र॒रोह॑न्ती प्र॒रोह॑न्ती॒ परु॑षःपरुषः॒ परि॒ परि॒ परु॑षःपरुषः प्र॒रोह॑न्ती प्र॒रोह॑न्ती॒ परु॑षःपरुषः॒ परि॑ । \newline
19. प्र॒रोह॒न्तीति॑ प्र - रोह॑न्ती । \newline
20. परु॑षःपरुषः॒ परि॒ परि॒ परु॑षःपरुषः॒ परु॑षःपरुषः॒ परि॑ । \newline
21. परु॑षःपरुष॒ इति॒ परु॑षः - प॒रु॒षः॒ । \newline
22. परीति॒ परि॑ । \newline
23. ए॒वा नो॑ न ए॒वैवा नो॑ दूर्वे दूर्वे न ए॒वैवा नो॑ दूर्वे । \newline
24. नो॒ दू॒र्वे॒ दू॒र्वे॒ नो॒ नो॒ दू॒र्वे॒ प्र प्र दू᳚र्वे नो नो दूर्वे॒ प्र । \newline
25. दू॒र्वे॒ प्र प्र दू᳚र्वे दूर्वे॒ प्र त॑नु तनु॒ प्र दू᳚र्वे दूर्वे॒ प्र त॑नु । \newline
26. प्र त॑नु तनु॒ प्र प्र त॑नु स॒हस्रे॑ण स॒हस्रे॑ण तनु॒ प्र प्र त॑नु स॒हस्रे॑ण । \newline
27. त॒नु॒ स॒हस्रे॑ण स॒हस्रे॑ण तनु तनु स॒हस्रे॑ण श॒तेन॑ श॒तेन॑ स॒हस्रे॑ण तनु तनु स॒हस्रे॑ण श॒तेन॑ । \newline
28. स॒हस्रे॑ण श॒तेन॑ श॒तेन॑ स॒हस्रे॑ण स॒हस्रे॑ण श॒तेन॑ च च श॒तेन॑ स॒हस्रे॑ण स॒हस्रे॑ण श॒तेन॑ च । \newline
29. श॒तेन॑ च च श॒तेन॑ श॒तेन॑ च । \newline
30. चेति॑ च । \newline
31. या श॒तेन॑ श॒तेन॒ या या श॒तेन॑ प्रत॒नोषि॑ प्रत॒नोषि॑ श॒तेन॒ या या श॒तेन॑ प्रत॒नोषि॑ । \newline
32. श॒तेन॑ प्रत॒नोषि॑ प्रत॒नोषि॑ श॒तेन॑ श॒तेन॑ प्रत॒नोषि॑ स॒हस्रे॑ण स॒हस्रे॑ण प्रत॒नोषि॑ श॒तेन॑ श॒तेन॑ प्रत॒नोषि॑ स॒हस्रे॑ण । \newline
33. प्र॒त॒नोषि॑ स॒हस्रे॑ण स॒हस्रे॑ण प्रत॒नोषि॑ प्रत॒नोषि॑ स॒हस्रे॑ण वि॒रोह॑सि वि॒रोह॑सि स॒हस्रे॑ण प्रत॒नोषि॑ प्रत॒नोषि॑ स॒हस्रे॑ण वि॒रोह॑सि । \newline
34. प्र॒त॒नोषीति॑ प्र - त॒नोषि॑ । \newline
35. स॒हस्रे॑ण वि॒रोह॑सि वि॒रोह॑सि स॒हस्रे॑ण स॒हस्रे॑ण वि॒रोह॑सि । \newline
36. वि॒रोह॒सीति॑ वि - रोह॑सि । \newline
37. तस्या᳚ स्ते ते॒ तस्या॒ स्तस्या᳚ स्ते देवि देवि ते॒ तस्या॒ स्तस्या᳚ स्ते देवि । \newline
38. ते॒ दे॒वि॒ दे॒वि॒ ते॒ ते॒ दे॒वी॒ष्ट॒क॒ इ॒ष्ट॒के॒ दे॒वि॒ ते॒ ते॒ दे॒वी॒ष्ट॒के॒ । \newline
39. दे॒वी॒ष्ट॒क॒ इ॒ष्ट॒के॒ दे॒वि॒ दे॒वी॒ष्ट॒के॒ वि॒धेम॑ वि॒धेमे᳚ ष्टके देवि देवीष्टके वि॒धेम॑ । \newline
40. इ॒ष्ट॒के॒ वि॒धेम॑ वि॒धेमे᳚ ष्टक इष्टके वि॒धेम॑ ह॒विषा॑ ह॒विषा॑ वि॒धेमे᳚ ष्टक इष्टके वि॒धेम॑ ह॒विषा᳚ । \newline
41. वि॒धेम॑ ह॒विषा॑ ह॒विषा॑ वि॒धेम॑ वि॒धेम॑ ह॒विषा॑ व॒यं ॅव॒यꣳ ह॒विषा॑ वि॒धेम॑ वि॒धेम॑ ह॒विषा॑ व॒यम् । \newline
42. ह॒विषा॑ व॒यं ॅव॒यꣳ ह॒विषा॑ ह॒विषा॑ व॒यम् । \newline
43. व॒यमिति॑ व॒यम् । \newline
44. अषा॑ढा ऽस्य॒ स्यषा॒ढा ऽषा॑ढा ऽसि॒ सह॑माना॒ सह॑माना॒ ऽस्यषा॒ढा ऽषा॑ढा ऽसि॒ सह॑माना । \newline
45. अ॒सि॒ सह॑माना॒ सह॑माना ऽस्यसि॒ सह॑माना॒ सह॑स्व॒ सह॑स्व॒ सह॑माना ऽस्यसि॒ सह॑माना॒ सह॑स्व । \newline
46. सह॑माना॒ सह॑स्व॒ सह॑स्व॒ सह॑माना॒ सह॑माना॒ सह॒स्वा रा॑ती॒ ररा॑तीः॒ सह॑स्व॒ सह॑माना॒ सह॑माना॒ सह॒स्वारा॑तीः । \newline
47. सह॒स्वारा॑ती॒ ररा॑तीः॒ सह॑स्व॒ सह॒स्वारा॑तीः॒ सह॑स्व॒ सह॒स्वारा॑तीः॒ सह॑स्व॒ सह॒स्वारा॑तीः॒ सह॑स्व । \newline
48. अरा॑तीः॒ सह॑स्व॒ सह॒स्वा रा॑ती॒ ररा॑तीः॒ सह॑स्वा रातीय॒तो॑ ऽरातीय॒तः सह॒स्वा रा॑ती॒ ररा॑तीः॒ सह॑स्वा रातीय॒तः । \newline
49. सह॑स्वा रातीय॒तो॑ ऽरातीय॒तः सह॑स्व॒ सह॑स्वा रातीय॒तः सह॑स्व॒ सह॑स्वा रातीय॒तः सह॑स्व॒ सह॑स्वा रातीय॒तः सह॑स्व । \newline
50. अ॒रा॒ती॒य॒तः सह॑स्व॒ सह॑स्वा रातीय॒तो॑ ऽरातीय॒तः सह॑स्व॒ पृत॑नाः॒ पृत॑नाः॒ सह॑स्वा रातीय॒तो॑ ऽरातीय॒तः सह॑स्व॒ पृत॑नाः । \newline
51. सह॑स्व॒ पृत॑नाः॒ पृत॑नाः॒ सह॑स्व॒ सह॑स्व॒ पृत॑नाः॒ सह॑स्व॒ सह॑स्व॒ पृत॑नाः॒ सह॑स्व॒ सह॑स्व॒ पृत॑नाः॒ सह॑स्व । \newline
52. पृत॑नाः॒ सह॑स्व॒ सह॑स्व॒ पृत॑नाः॒ पृत॑नाः॒ सह॑स्व पृतन्य॒तः पृ॑तन्य॒तः सह॑स्व॒ पृत॑नाः॒ पृत॑नाः॒ सह॑स्व पृतन्य॒तः । \newline
53. सह॑स्व पृतन्य॒तः पृ॑तन्य॒तः सह॑स्व॒ सह॑स्व पृतन्य॒तः । \newline
54. पृ॒त॒न्य॒त इति॑ पृतन्य॒तः । \newline
55. स॒हस्र॑वीर्या ऽस्यसि स॒हस्र॑वीर्या स॒हस्र॑वीर्या ऽसि॒ सा सा ऽसि॑ स॒हस्र॑वीर्या स॒हस्र॑वीर्या ऽसि॒ सा । \newline
56. स॒हस्र॑वी॒र्येति॑ स॒हस्र॑ - वी॒र्या॒ । \newline
\pagebreak
\markright{ TS 4.2.9.3  \hfill https://www.vedavms.in \hfill}

\section{ TS 4.2.9.3 }

\textbf{TS 4.2.9.3 } \newline
\textbf{Samhita Paata} \newline

-सि॒ सा मा॑ जिन्व ॥ मधु॒ वाता॑ ऋताय॒ते मधु॑ क्षरन्ति॒ सिन्ध॑वः । माद्ध्वी᳚र्नः स॒न्त्वोष॑धीः ॥ मधु॒ नक्त॑मु॒तोषसि॒ मधु॑म॒त् पार्थि॑वꣳ॒॒ रजः॑ । मधु॒ द्यौर॑स्तु नः पि॒ता ॥ मधु॑मान् नो॒ वन॒स्पति॒-र्मधु॑माꣳ अस्तु॒ सूर्यः॑ । माद्ध्वी॒र्गावो॑ भवन्तु नः ॥ म॒ही द्यौः पृ॑थि॒वी च॑ न इ॒मं ॅय॒ज्ञ्ं मि॑मिक्षतां । पि॒पृ॒तां नो॒ भरी॑मभिः ॥ तद्-विष्णोः᳚ पर॒मं - [  ] \newline

\textbf{Pada Paata} \newline

अ॒सि॒ । सा । मा॒ । जि॒न्व॒ ॥ मधु॑ । वाताः᳚ । ऋ॒ता॒य॒त इत्यृ॑त - य॒ते । मधु॑ । क्ष॒र॒न्ति॒ । सिन्ध॑वः ॥ माद्ध्वीः᳚ । नः॒ । स॒न्तु॒ । ओष॑धीः ॥ मधु॑ । नक्त᳚म् । उ॒त । उ॒षसि॑ । मधु॑म॒दिति॒ मधु॑ - म॒त् । पार्थि॑वम् । रजः॑ ॥ मधु॑ । द्यौः । अ॒स्तु॒ । नः॒ । पि॒ता ॥ मधु॑मा॒निति॒ मधु॑-मा॒न् । नः॒ । वन॒स्पतिः॑ । मधु॑मा॒निति॒ मधु॑ - मा॒न् । अ॒स्तु॒ । सूर्यः॑ ॥ माद्ध्वीः᳚ । गावः॑ । भ॒व॒न्तु॒ । नः॒ ॥ म॒ही । द्यौः । पृ॒थि॒वी । च॒ । नः॒ । इ॒मम् । य॒ज्ञ्म् । मि॒मि॒क्ष॒ता॒म् ॥ पि॒पृ॒ताम् । नः॒ । भरी॑मभि॒रिति॒ भरी॑म - भिः॒ ॥ तत् । विष्णोः᳚ । प॒र॒मम् ।  \newline


\textbf{Krama Paata} \newline

अ॒सि॒ सा । सा मा᳚ । मा॒ जि॒न्व॒ । जि॒न्वेति॑ जिन्व ॥ मधु॒ वाताः᳚ । वाता॑ ऋताय॒ते । ऋ॒ता॒य॒ते मधु॑ । ऋ॒ता॒य॒त इत्यृ॑त - य॒ते । मधु॑ क्षरन्ति । क्ष॒र॒न्ति॒ सिन्ध॑वः । सिन्ध॑व॒ इति॒ सिन्ध॑वः ॥ माद्ध्वी᳚र् नः । नः॒ स॒न्तु॒ । स॒न्त्वोष॑धीः । ओष॑धी॒रित्योष॑धीः ॥ मधु॒ नक्त᳚म् । नक्त॑मु॒त । उ॒तोषसि॑ । उ॒षसि॒ मधु॑मत् । मधु॑म॒त् पार्त्थि॑वम् । मधु॑म॒दिति॒ मधु॑ - म॒त्॒ । पार्त्थि॑वꣳ॒॒ रजः॑ । रज॒ इति॒ रजः॑ ॥ मधु॒ द्यौः । द्यौर॑स्तु । अ॒स्तु॒ नः॒ । नः॒ पि॒ता । पि॒तेति॑ पि॒ता ॥ मधु॑मान् नः । मधु॑मा॒निति॒ मधु॑ - मा॒न्॒ । नो॒ वन॒स्पतिः॑ । वन॒स्पति॒र् मधु॑मान् । मधु॑माꣳ अस्तु । मधु॑मा॒निति॒ मधु॑ - मा॒न्॒ । अ॒स्तु॒ सूर्यः॑ । सूर्य॒ इति॒ सूर्यः॑ ॥ माद्ध्वी॒र् गावः॑ । गावो॑ भवन्तु । भ॒व॒न्तु॒ नः॒ । न॒ इति॑ नः ॥ म॒ही द्यौः । द्यौः पृ॑थि॒वी । पृ॒थि॒वी च॑ । च॒ नः॒ । न॒ इ॒मम् । इ॒मम् ॅय॒ज्ञ्म् । य॒ज्ञ्म् मि॑मिक्षताम् । मि॒मि॒क्ष॒ता॒मिति॑ मिमिक्षताम् ॥ पि॒पृ॒ताम् नः॑ । नो॒ भरी॑मभिः । भरी॑मभि॒रिति॒ भरी॑म - भिः॒ ॥ तद् विष्णोः᳚ । विष्णोः᳚ पर॒मम् । प॒र॒मम् प॒दम् \newline

\textbf{Jatai Paata} \newline

1. अ॒सि॒ सा सा ऽस्य॑सि॒ सा । \newline
2. सा मा॑ मा॒ सा सा मा᳚ । \newline
3. मा॒ जि॒न्व॒ जि॒न्व॒ मा॒ मा॒ जि॒न्व॒ । \newline
4. जि॒न्वेति॑ जिन्व । \newline
5. मधु॒ वाता॒ वाता॒ मधु॒ मधु॒ वाताः᳚ । \newline
6. वाता॑ ऋताय॒त ऋ॑ताय॒ते वाता॒ वाता॑ ऋताय॒ते । \newline
7. ऋ॒ता॒य॒ते मधु॒ मध्वृ॑ताय॒त ऋ॑ताय॒ते मधु॑ । \newline
8. ऋ॒ता॒य॒त इत्यृ॑त - य॒ते । \newline
9. मधु॑ क्षरन्ति क्षरन्ति॒ मधु॒ मधु॑ क्षरन्ति । \newline
10. क्ष॒र॒न्ति॒ सिन्ध॑वः॒ सिन्ध॑वः क्षरन्ति क्षरन्ति॒ सिन्ध॑वः । \newline
11. सिन्ध॑व॒ इति॒ सिन्ध॑वः । \newline
12. माद्ध्वी᳚र् नो नो॒ माद्ध्वी॒र् माद्ध्वी᳚र् नः । \newline
13. नः॒ स॒न्तु॒ स॒न्तु॒ नो॒ नः॒ स॒न्तु॒ । \newline
14. स॒न्त्वोष॑धी॒ रोष॑धीः सन्तु स॒न्त्वोष॑धीः । \newline
15. ओष॑धी॒रित्योष॑धीः । \newline
16. मधु॒ नक्त॒म् नक्त॒म् मधु॒ मधु॒ नक्त᳚म् । \newline
17. नक्त॑ मु॒तोत नक्त॒म् नक्त॑ मु॒त । \newline
18. उ॒तोष स्यु॒ष स्यु॒तोतोषसि॑ । \newline
19. उ॒षसि॒ मधु॑म॒न् मधु॑म दु॒ष स्यु॒षसि॒ मधु॑मत् । \newline
20. मधु॑म॒त् पार्थि॑व॒म् पार्थि॑व॒म् मधु॑म॒न् मधु॑म॒त् पार्थि॑वम् । \newline
21. मधु॑म॒दिति॒ मधु॑ - म॒त् । \newline
22. पार्थि॑वꣳ॒॒ रजो॒ रजः॒ पार्थि॑व॒म् पार्थि॑वꣳ॒॒ रजः॑ । \newline
23. रज॒ इति॒ रजः॑ । \newline
24. मधु॒ द्यौर् द्यौर् मधु॒ मधु॒ द्यौः । \newline
25. द्यौ र॑स्त्वस्तु॒ द्यौर् द्यौ र॑स्तु । \newline
26. अ॒स्तु॒ नो॒ नो॒ अ॒स्त्व॒स्तु॒ नः॒ । \newline
27. नः॒ पि॒ता पि॒ता नो॑ नः पि॒ता । \newline
28. पि॒तेति॑ पि॒ता । \newline
29. मधु॑मान् नो नो॒ मधु॑मा॒न् मधु॑मान् नः । \newline
30. मधु॑मा॒निति॒ मधु॑ - मा॒न् । \newline
31. नो॒ वन॒स्पति॒र् वन॒स्पति॑र् नो नो॒ वन॒स्पतिः॑ । \newline
32. वन॒स्पति॒र् मधु॑मा॒न् मधु॑मा॒न्॒. वन॒स्पति॒र् वन॒स्पति॒र् मधु॑मान् । \newline
33. मधु॑माꣳ अस्त्वस्तु॒ मधु॑मा॒न् मधु॑माꣳ अस्तु । \newline
34. मधु॑मा॒निति॒ मधु॑ - मा॒न् । \newline
35. अ॒स्तु॒ सूर्यः॒ सूर्यो॑ अस्त्वस्तु॒ सूर्यः॑ । \newline
36. सूर्य॒ इति॒ सूर्यः॑ । \newline
37. माद्ध्वी॒र् गावो॒ गावो॒ माद्ध्वी॒र् माद्ध्वी॒र् गावः॑ । \newline
38. गावो॑ भवन्तु भवन्तु॒ गावो॒ गावो॑ भवन्तु । \newline
39. भ॒व॒न्तु॒ नो॒ नो॒ भ॒व॒न्तु॒ भ॒व॒न्तु॒ नः॒ । \newline
40. न॒ इति॑ नः । \newline
41. म॒ही द्यौर् द्यौर् म॒ही म॒ही द्यौः । \newline
42. द्यौः पृ॑थि॒वी पृ॑थि॒वी द्यौर् द्यौः पृ॑थि॒वी । \newline
43. पृ॒थि॒वी च॑ च पृथि॒वी पृ॑थि॒वी च॑ । \newline
44. च॒ नो॒ न॒श्च॒ च॒ नः॒ । \newline
45. न॒ इ॒म मि॒मम् नो॑ न इ॒मम् । \newline
46. इ॒मं ॅय॒ज्ञ्ं ॅय॒ज्ञ् मि॒म मि॒मं ॅय॒ज्ञ्म् । \newline
47. य॒ज्ञ्म् मि॑मिक्षताम् मिमिक्षतां ॅय॒ज्ञ्ं ॅय॒ज्ञ्म् मि॑मिक्षताम् । \newline
48. मि॒मि॒क्ष॒ता॒मिति॑ मिमिक्षताम् । \newline
49. पि॒पृ॒ताम् नो॑ नः पिपृ॒ताम् पि॑पृ॒ताम् नः॑ । \newline
50. नो॒ भरी॑मभि॒र् भरी॑मभिर् नो नो॒ भरी॑मभिः । \newline
51. भरी॑मभि॒रिति॒ भरी॑म - भिः॒ । \newline
52. तद् विष्णो॒र् विष्णो॒ स्तत् तद् विष्णोः᳚ । \newline
53. विष्णोः᳚ पर॒मम् प॑र॒मं ॅविष्णो॒र् विष्णोः᳚ पर॒मम् । \newline
54. प॒र॒मम् प॒दम् प॒दम् प॑र॒मम् प॑र॒मम् प॒दम् । \newline

\textbf{Ghana Paata } \newline

1. अ॒सि॒ सा सा ऽस्य॑सि॒ सा मा॑ मा॒ सा ऽस्य॑सि॒ सा मा᳚ । \newline
2. सा मा॑ मा॒ सा सा मा॑ जिन्व जिन्व मा॒ सा सा मा॑ जिन्व । \newline
3. मा॒ जि॒न्व॒ जि॒न्व॒ मा॒ मा॒ जि॒न्व॒ । \newline
4. जि॒न्वेति॑ जिन्व । \newline
5. मधु॒ वाता॒ वाता॒ मधु॒ मधु॒ वाता॑ ऋताय॒त ऋ॑ताय॒ते वाता॒ मधु॒ मधु॒ वाता॑ ऋताय॒ते । \newline
6. वाता॑ ऋताय॒त ऋ॑ताय॒ते वाता॒ वाता॑ ऋताय॒ते मधु॒ मध्वृ॑ताय॒ते वाता॒ वाता॑ ऋताय॒ते मधु॑ । \newline
7. ऋ॒ता॒य॒ते मधु॒ मध्वृ॑ताय॒त ऋ॑ताय॒ते मधु॑ क्षरन्ति क्षरन्ति॒ मध्वृ॑ताय॒त ऋ॑ताय॒ते मधु॑ क्षरन्ति । \newline
8. ऋ॒ता॒य॒त इत्यृ॑त - य॒ते । \newline
9. मधु॑ क्षरन्ति क्षरन्ति॒ मधु॒ मधु॑ क्षरन्ति॒ सिन्ध॑वः॒ सिन्ध॑वः क्षरन्ति॒ मधु॒ मधु॑ क्षरन्ति॒ सिन्ध॑वः । \newline
10. क्ष॒र॒न्ति॒ सिन्ध॑वः॒ सिन्ध॑वः क्षरन्ति क्षरन्ति॒ सिन्ध॑वः । \newline
11. सिन्ध॑व॒ इति॒ सिन्ध॑वः । \newline
12. माद्ध्वी᳚र् नो नो॒ माद्ध्वी॒र् माद्ध्वी᳚र् नः सन्तु सन्तु नो॒ माद्ध्वी॒र् माद्ध्वी᳚र् नः सन्तु । \newline
13. नः॒ स॒न्तु॒ स॒न्तु॒ नो॒ नः॒ स॒न्त्वोष॑धी॒ रोष॑धीः सन्तु नो नः स॒न्त्वोष॑धीः । \newline
14. स॒न्त्वोष॑धी॒ रोष॑धीः सन्तु स॒न्त्वोष॑धीः । \newline
15. ओष॑धी॒रित्योष॑धीः । \newline
16. मधु॒ नक्त॒म् नक्त॒म् मधु॒ मधु॒ नक्त॑ मु॒तोत नक्त॒म् मधु॒ मधु॒ नक्त॑ मु॒त । \newline
17. नक्त॑ मु॒तोत नक्त॒म् नक्त॑ मु॒तोष स्यु॒ष स्यु॒त नक्त॒म् नक्त॑ मु॒तोषसि॑ । \newline
18. उ॒तोष स्यु॒ष स्यु॒तोतोषसि॒ मधु॑म॒न् मधु॑म दु॒ष स्यु॒तोतोषसि॒ मधु॑मत् । \newline
19. उ॒षसि॒ मधु॑म॒न् मधु॑म दु॒ष स्यु॒षसि॒ मधु॑म॒त् पार्थि॑व॒म् पार्थि॑व॒म् मधु॑म दु॒ष स्यु॒षसि॒ मधु॑म॒त् पार्थि॑वम् । \newline
20. मधु॑म॒त् पार्थि॑व॒म् पार्थि॑व॒म् मधु॑म॒न् मधु॑म॒त् पार्थि॑वꣳ॒॒ रजो॒ रजः॒ पार्थि॑व॒म् मधु॑म॒न् मधु॑म॒त् पार्थि॑वꣳ॒॒ रजः॑ । \newline
21. मधु॑म॒दिति॒ मधु॑ - म॒त् । \newline
22. पार्थि॑वꣳ॒॒ रजो॒ रजः॒ पार्थि॑व॒म् पार्थि॑वꣳ॒॒ रजः॑ । \newline
23. रज॒ इति॒ रजः॑ । \newline
24. मधु॒ द्यौर् द्यौर् मधु॒ मधु॒ द्यौर॑ स्त्वस्तु॒ द्यौर् मधु॒ मधु॒ द्यौर॑स्तु । \newline
25. द्यौर॑ स्त्वस्तु॒ द्यौर् द्यौ र॑स्तु नो नो अस्तु॒ द्यौर् द्यौ र॑स्तु नः । \newline
26. अ॒स्तु॒ नो॒ नो॒ अ॒स्त्व॒स्तु॒ नः॒ पि॒ता पि॒ता नो॑ अस्त्वस्तु नः पि॒ता । \newline
27. नः॒ पि॒ता पि॒ता नो॑ नः पि॒ता । \newline
28. पि॒तेति॑ पि॒ता । \newline
29. मधु॑मान् नो नो॒ मधु॑मा॒न् मधु॑मान् नो॒ वन॒स्पति॒र् वन॒स्पति॑र् नो॒ मधु॑मा॒न् मधु॑मान् नो॒ वन॒स्पतिः॑ । \newline
30. मधु॑मा॒निति॒ मधु॑ - मा॒न् । \newline
31. नो॒ वन॒स्पति॒र् वन॒स्पति॑र् नो नो॒ वन॒स्पति॒र् मधु॑मा॒न् मधु॑मा॒न्॒. वन॒स्पति॑र् नो नो॒ वन॒स्पति॒र् मधु॑मान् । \newline
32. वन॒स्पति॒र् मधु॑मा॒न् मधु॑मा॒न्॒. वन॒स्पति॒र् वन॒स्पति॒र् मधु॑माꣳ अस्त्वस्तु॒ मधु॑मा॒न्॒. वन॒स्पति॒र् वन॒स्पति॒र् मधु॑माꣳ अस्तु । \newline
33. मधु॑माꣳ अस्त्वस्तु॒ मधु॑मा॒न् मधु॑माꣳ अस्तु॒ सूर्यः॒ सूर्यो॑ अस्तु॒ मधु॑मा॒न् मधु॑माꣳ अस्तु॒ सूर्यः॑ । \newline
34. मधु॑मा॒निति॒ मधु॑ - मा॒न् । \newline
35. अ॒स्तु॒ सूर्यः॒ सूर्यो॑ अस्त्वस्तु॒ सूर्यः॑ । \newline
36. सूर्य॒ इति॒ सूर्यः॑ । \newline
37. माद्ध्वी॒र् गावो॒ गावो॒ माद्ध्वी॒र् माद्ध्वी॒र् गावो॑ भवन्तु भवन्तु॒ गावो॒ माद्ध्वी॒र् माद्ध्वी॒र् गावो॑ भवन्तु । \newline
38. गावो॑ भवन्तु भवन्तु॒ गावो॒ गावो॑ भवन्तु नो नो भवन्तु॒ गावो॒ गावो॑ भवन्तु नः । \newline
39. भ॒व॒न्तु॒ नो॒ नो॒ भ॒व॒न्तु॒ भ॒व॒न्तु॒ नः॒ । \newline
40. न॒ इति॑ नः । \newline
41. म॒ही द्यौर् द्यौर् म॒ही म॒ही द्यौः पृ॑थि॒वी पृ॑थि॒वी द्यौर् म॒ही म॒ही द्यौः पृ॑थि॒वी । \newline
42. द्यौः पृ॑थि॒वी पृ॑थि॒वी द्यौर् द्यौः पृ॑थि॒वी च॑ च पृथि॒वी द्यौर् द्यौः पृ॑थि॒वी च॑ । \newline
43. पृ॒थि॒वी च॑ च पृथि॒वी पृ॑थि॒वी च॑ नो नश्च पृथि॒वी पृ॑थि॒वी च॑ नः । \newline
44. च॒ नो॒ न॒श्च॒ च॒ न॒ इ॒म मि॒मम् न॑श्च च न इ॒मम् । \newline
45. न॒ इ॒म मि॒मम् नो॑ न इ॒मं ॅय॒ज्ञ्ं ॅय॒ज्ञ् मि॒मम् नो॑ न इ॒मं ॅय॒ज्ञ्म् । \newline
46. इ॒मं ॅय॒ज्ञ्ं ॅय॒ज्ञ् मि॒म मि॒मं ॅय॒ज्ञ्म् मि॑मिक्षताम् मिमिक्षतां ॅय॒ज्ञ् मि॒म मि॒मं ॅय॒ज्ञ्म् मि॑मिक्षताम् । \newline
47. य॒ज्ञ्म् मि॑मिक्षताम् मिमिक्षतां ॅय॒ज्ञ्ं ॅय॒ज्ञ्म् मि॑मिक्षताम् । \newline
48. मि॒मि॒क्ष॒ता॒मिति॑ मिमिक्षताम् । \newline
49. पि॒पृ॒ताम् नो॑ नः पिपृ॒ताम् पि॑पृ॒ताम् नो॒ भरी॑मभि॒र् भरी॑मभिर् नः पिपृ॒ताम् पि॑पृ॒ताम् नो॒ भरी॑मभिः । \newline
50. नो॒ भरी॑मभि॒र् भरी॑मभिर् नो नो॒ भरी॑मभिः । \newline
51. भरी॑मभि॒रिति॒ भरी॑म - भिः॒ । \newline
52. तद् विष्णो॒र् विष्णो॒ स्तत् तद् विष्णोः᳚ पर॒मम् प॑र॒मं ॅविष्णो॒ स्तत् तद् विष्णोः᳚ पर॒मम् । \newline
53. विष्णोः᳚ पर॒मम् प॑र॒मं ॅविष्णो॒र् विष्णोः᳚ पर॒मम् प॒दम् प॒दम् प॑र॒मं ॅविष्णो॒र् विष्णोः᳚ पर॒मम् प॒दम् । \newline
54. प॒र॒मम् प॒दम् प॒दम् प॑र॒मम् प॑र॒मम् प॒दꣳ सदा॒ सदा॑ प॒दम् प॑र॒मम् प॑र॒मम् प॒दꣳ सदा᳚ । \newline
\pagebreak
\markright{ TS 4.2.9.4  \hfill https://www.vedavms.in \hfill}

\section{ TS 4.2.9.4 }

\textbf{TS 4.2.9.4 } \newline
\textbf{Samhita Paata} \newline

प॒दꣳ सदा॑ पश्यन्ति सू॒रयः॑ । दि॒वीव॒ चक्षु॒रात॑तं ॥ ध्रु॒वाऽसि॑ पृथिवि॒ सह॑स्व पृतन्य॒तः । स्यू॒ता दे॒वेभि॑र॒मृते॒ना ऽऽगाः᳚ ॥ यास्ते॑ अग्ने॒ सूर्ये॒ रुच॑ उद्य॒तो दिव॑मात॒न्वन्ति॑ र॒श्मिभिः॑ । ताभिः॒ सर्वा॑भी रु॒चे जना॑य नस्कृधि ॥ या वो॑ देवाः॒ सूर्ये॒ रुचो॒ गोष्वश्वे॑षु॒ या रुचः॑ । इन्द्रा᳚ग्नी॒ ताभिः॒ सर्वा॑भी॒ रुचं॑ नो धत्त बृहस्पते ॥ वि॒राड् - [  ] \newline

\textbf{Pada Paata} \newline

प॒दम् । सदा᳚ । प॒श्य॒न्ति॒ । सू॒रयः॑ ॥ दि॒वि । इ॒व॒ । चक्षुः॑ । आत॑त॒मित्या-त॒त॒म् ॥ ध्रु॒वा । अ॒सि॒ । पृ॒थि॒वि॒ । सह॑स्व । पृ॒त॒न्य॒तः ॥ स्यू॒ता । दे॒वेभिः॑ । अ॒मृते॑न । एति॑ । अ॒गाः॒ ॥ याः । ते॒ । अ॒ग्ने॒ । सूर्ये᳚ । रुचः॑ । उ॒द्य॒त इत्यु॑त् - य॒तः । दिव᳚म् । आ॒त॒न्वन्तीत्या᳚ - त॒न्वन्ति॑ । र॒श्मिभि॒रिति॑ र॒श्मि - भिः॒ ॥ ताभिः॑ । सर्वा॑भः । रु॒चे । जना॑य । नः॒ । कृ॒धि॒ ॥ याः । वः॒ । दे॒वाः॒ । सूर्ये᳚ । रुचः॑ । गोषु॑ । अश्वे॑षु । याः । रुचः॑ ॥ इन्द्रा᳚ग्नी॒ इतीन्द्र॑ - अ॒ग्नी॒ । ताभिः॑ । सर्वा॑भः । रुच᳚म् । नः॒ । ध॒त्त॒ । बृ॒ह॒स्प॒ते॒ ॥ वि॒राडिति॑ वि - राट् ।  \newline


\textbf{Krama Paata} \newline

प॒दꣳ सदा᳚ । सदा॑ पश्यन्ति । प॒श्य॒न्ति॒ सू॒रयः॑ । सू॒रय॒ इति॑ सू॒रयः॑ ॥ दि॒वीव॑ । इ॒व॒ चक्षुः॑ । चक्षु॒रात॑तम् । आत॑त॒मित्या - त॒त॒म् ॥ ध्रु॒वाऽसि॑ । अ॒सि॒ पृ॒थि॒वि॒ । पृ॒थि॒वि॒ सह॑स्व । सह॑स्व पृतन्य॒तः । पृ॒त॒न्य॒त इति॑ पृतन्य॒तः ॥? स्यू॒ता दे॒वेभिः॑ । दे॒वेभि॑र॒मृते॑न । अ॒मृते॒ना । आऽगाः᳚ । अ॒गा॒ इत्य॑गाः ॥ या स्ते᳚ । ते॒ अ॒ग्ने॒ । अ॒ग्ने॒ सूर्ये᳚ । सूर्ये॒ रुचः॑ । रुच॑ उद्य॒तः । उ॒द्य॒तो दिव᳚म् । उ॒द्य॒त इत्यु॑त् - य॒तः । दिव॑मात॒न्वन्ति॑ । आ॒त॒न्वन्ति॑ र॒श्मिभिः॑ । आ॒त॒न्वन्तीत्या᳚ - त॒न्वन्ति॑ । र॒श्मिभि॒रिति॑ र॒श्मि - भिः॒ ॥ ताभिः॒ सर्वा॑भिः । सर्वा॑भी रु॒चे । रु॒चे जना॑य । जना॑य नः । न॒स्कृ॒धि॒ । कृ॒धीति॑ कृधि ॥ या वः॑ । वो॒ दे॒वाः॒ । दे॒वाः॒ सूर्ये᳚ । सूर्ये॒ रुचः॑ । रुचो॒ गोषु॑ । गोष्वश्वे॑षु । अश्वे॑षु॒ याः । या रुचः॑ । रुच॒ इति॒ रुचः॑ ॥ इन्द्रा᳚ग्नी॒ ताभिः॑ । इन्द्रा᳚ग्नी॒ इतीन्द्र॑ - अ॒ग्नी॒ । ताभिः॒ सर्वा॑भिः । सर्वा॑भी॒ रुच᳚म् । रुच॑म् नः । नो॒ ध॒त्त॒ । ध॒त्त॒ बृ॒ह॒स्प॒ते॒ । बृ॒ह॒स्प॒त॒ इति॑ बृहस्पते ॥ वि॒राड् ज्योतिः॑ । वि॒राडिति॑ वि - राट् \newline

\textbf{Jatai Paata} \newline

1. प॒दꣳ सदा॒ सदा॑ प॒दम् प॒दꣳ सदा᳚ । \newline
2. सदा॑ पश्यन्ति पश्यन्ति॒ सदा॒ सदा॑ पश्यन्ति । \newline
3. प॒श्य॒न्ति॒ सू॒रयः॑ सू॒रयः॑ पश्यन्ति पश्यन्ति सू॒रयः॑ । \newline
4. सू॒रय॒ इति॑ सू॒रयः॑ । \newline
5. दि॒वीवे॑व दि॒वि दि॒वीव॑ । \newline
6. इ॒व॒ चक्षु॒ श्चक्षु॑ रिवेव॒ चक्षुः॑ । \newline
7. चक्षु॒ रात॑त॒ मात॑त॒म् चक्षु॒ श्चक्षु॒ रात॑तम् । \newline
8. आत॑त॒मित्या - त॒त॒म् । \newline
9. ध्रु॒वा ऽस्य॑सि ध्रु॒वा ध्रु॒वा ऽसि॑ । \newline
10. अ॒सि॒ पृ॒थि॒वि॒ पृ॒थि॒ व्य॒स्य॒सि॒ पृ॒थि॒वि॒ । \newline
11. पृ॒थि॒वि॒ सह॑स्व॒ सह॑स्व पृथिवि पृथिवि॒ सह॑स्व । \newline
12. सह॑स्व पृतन्य॒तः पृ॑तन्य॒तः सह॑स्व॒ सह॑स्व पृतन्य॒तः । \newline
13. पृ॒त॒न्य॒त इति॑ पृतन्य॒तः । \newline
14. स्यू॒ता दे॒वेभि॑र् दे॒वेभिः॑ स्यू॒ता स्यू॒ता दे॒वेभिः॑ । \newline
15. दे॒वेभि॑ र॒मृते॑ना॒ मृते॑न दे॒वेभि॑र् दे॒वेभि॑ र॒मृते॑न । \newline
16. अ॒मृते॒ना ऽमृते॑ना॒ मृते॒ना । \newline
17. आ ऽगा॑ अगा॒ आ ऽगाः᳚ । \newline
18. अ॒गा॒ इत्य॑गाः । \newline
19. या स्ते॑ ते॒ या या स्ते᳚ । \newline
20. ते॒ अ॒ग्ने॒ अ॒ग्ने॒ ते॒ ते॒ अ॒ग्ने॒ । \newline
21. अ॒ग्ने॒ सूर्ये॒ सूर्ये॑ अग्ने अग्ने॒ सूर्ये᳚ । \newline
22. सूर्ये॒ रुचो॒ रुचः॒ सूर्ये॒ सूर्ये॒ रुचः॑ । \newline
23. रुच॑ उद्य॒त उ॑द्य॒तो रुचो॒ रुच॑ उद्य॒तः । \newline
24. उ॒द्य॒तो दिव॒म् दिव॑ मुद्य॒त उ॑द्य॒तो दिव᳚म् । \newline
25. उ॒द्य॒त इत्यु॑त् - य॒तः । \newline
26. दिव॑ मात॒न्व न्त्या॑त॒न्वन्ति॒ दिव॒म् दिव॑ मात॒न्वन्ति॑ । \newline
27. आ॒त॒न्वन्ति॑ र॒श्मिभी॑ र॒श्मिभि॑ रात॒न्व न्त्या॑त॒न्वन्ति॑ र॒श्मिभिः॑ । \newline
28. आ॒त॒न्वन्तीत्या᳚ - त॒न्वन्ति॑ । \newline
29. र॒श्मिभि॒रिति॑ र॒श्मि - भिः॒ । \newline
30. ताभिः॒ सर्वा॑भिः॒ सर्वा॑भि॒ स्ताभि॒ स्ताभिः॒ सर्वा॑भिः । \newline
31. सर्वा॑भी रु॒चे रु॒चे सर्वा॑भिः॒ सर्वा॑भी रु॒चे । \newline
32. रु॒चे जना॑य॒ जना॑य रु॒चे रु॒चे जना॑य । \newline
33. जना॑य नो नो॒ जना॑य॒ जना॑य नः । \newline
34. न॒ स्कृ॒धि॒ कृ॒धि॒ नो॒ न॒ स्कृ॒धि॒ । \newline
35. कृ॒धीति॑ कृधि । \newline
36. या वो॑ वो॒ या या वः॑ । \newline
37. वो॒ दे॒वा॒ दे॒वा॒ वो॒ वो॒ दे॒वाः॒ । \newline
38. दे॒वाः॒ सूर्ये॒ सूर्ये॑ देवा देवाः॒ सूर्ये᳚ । \newline
39. सूर्ये॒ रुचो॒ रुचः॒ सूर्ये॒ सूर्ये॒ रुचः॑ । \newline
40. रुचो॒ गोषु॒ गोषु॒ रुचो॒ रुचो॒ गोषु॑ । \newline
41. गोष्वश्वे॒ ष्वश्वे॑षु॒ गोषु॒ गोष्वश्वे॑षु । \newline
42. अश्वे॑षु॒ या या अश्वे॒ ष्वश्वे॑षु॒ याः । \newline
43. या रुचो॒ रुचो॒ या या रुचः॑ । \newline
44. रुच॒ इति॒ रुचः॑ । \newline
45. इन्द्रा᳚ग्नी॒ ताभि॒ स्ताभि॒ रिन्द्रा᳚ग्नी॒ इन्द्रा᳚ग्नी॒ ताभिः॑ । \newline
46. इन्द्रा᳚ग्नी॒ इतीन्द्र॑ - अ॒ग्नी॒ । \newline
47. ताभिः॒ सर्वा॑भिः॒ सर्वा॑भि॒ स्ताभि॒ स्ताभिः॒ सर्वा॑भिः । \newline
48. सर्वा॑भी॒ रुचꣳ॒॒ रुचꣳ॒॒ सर्वा॑भिः॒ सर्वा॑भी॒ रुच᳚म् । \newline
49. रुच॑म् नो नो॒ रुचꣳ॒॒ रुच॑म् नः । \newline
50. नो॒ ध॒त्त॒ ध॒त्त॒ नो॒ नो॒ ध॒त्त॒ । \newline
51. ध॒त्त॒ बृ॒ह॒स्प॒ते॒ बृ॒ह॒स्प॒ते॒ ध॒त्त॒ ध॒त्त॒ बृ॒ह॒स्प॒ते॒ । \newline
52. बृ॒ह॒स्प॒त॒ इति॑ बृहस्पते । \newline
53. वि॒राड् ज्योति॒र् ज्योति॑र् वि॒राड् वि॒राड् ज्योतिः॑ । \newline
54. वि॒राडिति॑ वि - राट् । \newline

\textbf{Ghana Paata } \newline

1. प॒दꣳ सदा॒ सदा॑ प॒दम् प॒दꣳ सदा॑ पश्यन्ति पश्यन्ति॒ सदा॑ प॒दम् प॒दꣳ सदा॑ पश्यन्ति । \newline
2. सदा॑ पश्यन्ति पश्यन्ति॒ सदा॒ सदा॑ पश्यन्ति सू॒रयः॑ सू॒रयः॑ पश्यन्ति॒ सदा॒ सदा॑ पश्यन्ति सू॒रयः॑ । \newline
3. प॒श्य॒न्ति॒ सू॒रयः॑ सू॒रयः॑ पश्यन्ति पश्यन्ति सू॒रयः॑ । \newline
4. सू॒रय॒ इति॑ सू॒रयः॑ । \newline
5. दि॒वीवे॑व दि॒वि दि॒वीव॒ चक्षु॒ श्चक्षु॑ रिव दि॒वि दि॒वीव॒ चक्षुः॑ । \newline
6. इ॒व॒ चक्षु॒ श्चक्षु॑ रिवेव॒ चक्षु॒ रात॑त॒ मात॑त॒म् चक्षु॑रिवेव॒ चक्षु॒ रात॑तम् । \newline
7. चक्षु॒ रात॑त॒ मात॑त॒म् चक्षु॒ श्चक्षु॒ रात॑तम् । \newline
8. आत॑त॒मित्या - त॒त॒म् । \newline
9. ध्रु॒वा ऽस्य॑सि ध्रु॒वा ध्रु॒वा ऽसि॑ पृथिवि पृथिव्यसि ध्रु॒वा ध्रु॒वा ऽसि॑ पृथिवि । \newline
10. अ॒सि॒ पृ॒थि॒वि॒ पृ॒थि॒ व्य॒स्य॒सि॒ पृ॒थि॒वि॒ सह॑स्व॒ सह॑स्व पृथि व्यस्यसि पृथिवि॒ सह॑स्व । \newline
11. पृ॒थि॒वि॒ सह॑स्व॒ सह॑स्व पृथिवि पृथिवि॒ सह॑स्व पृतन्य॒तः पृ॑तन्य॒तः सह॑स्व पृथिवि पृथिवि॒ सह॑स्व पृतन्य॒तः । \newline
12. सह॑स्व पृतन्य॒तः पृ॑तन्य॒तः सह॑स्व॒ सह॑स्व पृतन्य॒तः । \newline
13. पृ॒त॒न्य॒त इति॑ पृतन्य॒तः । \newline
14. स्यू॒ता दे॒वेभि॑र् दे॒वेभिः॑ स्यू॒ता स्यू॒ता दे॒वेभि॑ र॒मृते॑ना॒ मृते॑न दे॒वेभिः॑ स्यू॒ता स्यू॒ता दे॒वेभि॑ र॒मृते॑न । \newline
15. दे॒वेभि॑ र॒मृते॑ना॒ मृते॑न दे॒वेभि॑र् दे॒वेभि॑ र॒मृते॒ना ऽमृते॑न दे॒वेभि॑र् दे॒वेभि॑ र॒मृते॒ना । \newline
16. अ॒मृते॒ना ऽमृते॑ना॒ मृते॒ना ऽगा॑ अगा॒ आ ऽमृते॑ना॒ मृते॒ना ऽगाः᳚ । \newline
17. आ ऽगा॑ अगा॒ आ ऽगाः᳚ । \newline
18. अ॒गा॒ इत्य॑गाः । \newline
19. या स्ते॑ ते॒ या या स्ते॑ अग्ने अग्ने ते॒ या या स्ते॑ अग्ने । \newline
20. ते॒ अ॒ग्ने॒ अ॒ग्ने॒ ते॒ ते॒ अ॒ग्ने॒ सूर्ये॒ सूर्ये॑ अग्ने ते ते अग्ने॒ सूर्ये᳚ । \newline
21. अ॒ग्ने॒ सूर्ये॒ सूर्ये॑ अग्ने अग्ने॒ सूर्ये॒ रुचो॒ रुचः॒ सूर्ये॑ अग्ने अग्ने॒ सूर्ये॒ रुचः॑ । \newline
22. सूर्ये॒ रुचो॒ रुचः॒ सूर्ये॒ सूर्ये॒ रुच॑ उद्य॒त उ॑द्य॒तो रुचः॒ सूर्ये॒ सूर्ये॒ रुच॑ उद्य॒तः । \newline
23. रुच॑ उद्य॒त उ॑द्य॒तो रुचो॒ रुच॑ उद्य॒तो दिव॒म् दिव॑ मुद्य॒तो रुचो॒ रुच॑ उद्य॒तो दिव᳚म् । \newline
24. उ॒द्य॒तो दिव॒म् दिव॑ मुद्य॒त उ॑द्य॒तो दिव॑ मात॒न्वन् त्या॑त॒न्वन्ति॒ दिव॑ मुद्य॒त उ॑द्य॒तो दिव॑ मात॒न्वन्ति॑ । \newline
25. उ॒द्य॒त इत्यु॑त् - य॒तः । \newline
26. दिव॑ मात॒न्वन् त्या॑त॒न्वन्ति॒ दिव॒म् दिव॑ मात॒न्वन्ति॑ र॒श्मिभी॑ र॒श्मिभि॑ रात॒न्वन्ति॒ दिव॒म् दिव॑ मात॒न्वन्ति॑ र॒श्मिभिः॑ । \newline
27. आ॒त॒न्वन्ति॑ र॒श्मिभी॑ र॒श्मिभि॑ रात॒न्वन् त्या॑त॒न्वन्ति॑ र॒श्मिभिः॑ । \newline
28. आ॒त॒न्वन्तीत्या᳚ - त॒न्वन्ति॑ । \newline
29. र॒श्मिभि॒रिति॑ र॒श्मि - भिः॒ । \newline
30. ताभिः॒ सर्वा॑भिः॒ सर्वा॑भि॒ स्ताभि॒ स्ताभिः॒ सर्वा॑भी रु॒चे रु॒चे सर्वा॑भि॒ स्ताभि॒ स्ताभिः॒ सर्वा॑भी रु॒चे । \newline
31. सर्वा॑भी रु॒चे रु॒चे सर्वा॑भिः॒ सर्वा॑भी रु॒चे जना॑य॒ जना॑य रु॒चे सर्वा॑भिः॒ सर्वा॑भी रु॒चे जना॑य । \newline
32. रु॒चे जना॑य॒ जना॑य रु॒चे रु॒चे जना॑य नो नो॒ जना॑य रु॒चे रु॒चे जना॑य नः । \newline
33. जना॑य नो नो॒ जना॑य॒ जना॑य न स्कृधि कृधि नो॒ जना॑य॒ जना॑य न स्कृधि । \newline
34. न॒ स्कृ॒धि॒ कृ॒धि॒ नो॒ न॒ स्कृ॒धि॒ । \newline
35. कृ॒धीति॑ कृधि । \newline
36. या वो॑ वो॒ या या वो॑ देवा देवा वो॒ या या वो॑ देवाः । \newline
37. वो॒ दे॒वा॒ दे॒वा॒ वो॒ वो॒ दे॒वाः॒ सूर्ये॒ सूर्ये॑ देवा वो वो देवाः॒ सूर्ये᳚ । \newline
38. दे॒वाः॒ सूर्ये॒ सूर्ये॑ देवा देवाः॒ सूर्ये॒ रुचो॒ रुचः॒ सूर्ये॑ देवा देवाः॒ सूर्ये॒ रुचः॑ । \newline
39. सूर्ये॒ रुचो॒ रुचः॒ सूर्ये॒ सूर्ये॒ रुचो॒ गोषु॒ गोषु॒ रुचः॒ सूर्ये॒ सूर्ये॒ रुचो॒ गोषु॑ । \newline
40. रुचो॒ गोषु॒ गोषु॒ रुचो॒ रुचो॒ गो ष्वश्वे॒ ष्वश्वे॑षु॒ गोषु॒ रुचो॒ रुचो॒ गोष्वश्वे॑षु । \newline
41. गोष्वश्वे॒ ष्वश्वे॑षु॒ गोषु॒ गोष्वश्वे॑षु॒ या या अश्वे॑षु॒ गोषु॒ गोष्वश्वे॑षु॒ याः । \newline
42. अश्वे॑षु॒ या या अश्वे॒ ष्वश्वे॑षु॒ या रुचो॒ रुचो॒ या अश्वे॒ ष्वश्वे॑षु॒ या रुचः॑ । \newline
43. या रुचो॒ रुचो॒ या या रुचः॑ । \newline
44. रुच॒ इति॒ रुचः॑ । \newline
45. इन्द्रा᳚ग्नी॒ ताभि॒ स्ताभि॒ रिन्द्रा᳚ग्नी॒ इन्द्रा᳚ग्नी॒ ताभिः॒ सर्वा॑भिः॒ सर्वा॑भि॒ स्ताभि॒ रिन्द्रा᳚ग्नी॒ इन्द्रा᳚ग्नी॒ ताभिः॒ सर्वा॑भिः । \newline
46. इन्द्रा᳚ग्नी॒ इतीन्द्र॑ - अ॒ग्नी॒ । \newline
47. ताभिः॒ सर्वा॑भिः॒ सर्वा॑भि॒ स्ताभि॒ स्ताभिः॒ सर्वा॑भी॒ रुचꣳ॒॒ रुचꣳ॒॒ सर्वा॑भि॒ स्ताभि॒ स्ताभिः॒ सर्वा॑भी॒ रुच᳚म् । \newline
48. सर्वा॑भी॒ रुचꣳ॒॒ रुचꣳ॒॒ सर्वा॑भिः॒ सर्वा॑भी॒ रुच॑म् नो नो॒ रुचꣳ॒॒ सर्वा॑भिः॒ सर्वा॑भी॒ रुच॑म् नः । \newline
49. रुच॑म् नो नो॒ रुचꣳ॒॒ रुच॑म् नो धत्त धत्त नो॒ रुचꣳ॒॒ रुच॑म् नो धत्त । \newline
50. नो॒ ध॒त्त॒ ध॒त्त॒ नो॒ नो॒ ध॒त्त॒ बृ॒ह॒स्प॒ते॒ बृ॒ह॒स्प॒ते॒ ध॒त्त॒ नो॒ नो॒ ध॒त्त॒ बृ॒ह॒स्प॒ते॒ । \newline
51. ध॒त्त॒ बृ॒ह॒स्प॒ते॒ बृ॒ह॒स्प॒ते॒ ध॒त्त॒ ध॒त्त॒ बृ॒ह॒स्प॒ते॒ । \newline
52. बृ॒ह॒स्प॒त॒ इति॑ बृहस्पते । \newline
53. वि॒राड् ज्योति॒र् ज्योति॑र् वि॒राड् वि॒राड् ज्योति॑ रधारय दधारय॒ज् ज्योति॑र् वि॒राड् वि॒राड् ज्योति॑ रधारयत् । \newline
54. वि॒राडिति॑ वि - राट् । \newline
\pagebreak
\markright{ TS 4.2.9.5  \hfill https://www.vedavms.in \hfill}

\section{ TS 4.2.9.5 }

\textbf{TS 4.2.9.5 } \newline
\textbf{Samhita Paata} \newline

ज्योति॑रधारयथ् स॒म्राड् ज्योति॑रधारयथ् स्व॒राड् ज्योति॑रधारयत् ॥ अग्ने॑ यु॒क्ष्वा हि ये तवाश्वा॑सो देव सा॒धवः॑ । अरं॒ ॅवह॑न्त्या॒शवः॑ ॥ यु॒क्ष्वा हि दे॑व॒हूत॑माꣳ॒॒ अश्वाꣳ॑ अग्ने र॒थीरि॑व । नि होता॑ पू॒र्व्यः स॑दः ॥ द्र॒फ्सश्च॑स्कन्द पृथि॒वीमनु॒ द्यामि॒मं च॒ योनि॒मनु॒ यश्च॒ पूर्वः॑ । तृ॒तीयं॒ ॅयोनि॒मनु॑ स॒ञ्चर॑न्तं द्र॒फ्सं जु॑हो॒म्यनु॑ स॒प्त - [  ] \newline

\textbf{Pada Paata} \newline

ज्योतिः॑ । अ॒धा॒र॒य॒त् । स॒म्राडिति॑ सम्-राट् । ज्योतिः॑ । अ॒धा॒र॒य॒त् । स्व॒राडिति॑ स्व - राट् । ज्योतिः॑ । अ॒धा॒र॒य॒त् ॥ अग्ने᳚ । यु॒क्ष्व । हि । ये । तव॑ । अश्वा॑सः । दे॒व॒ । सा॒धवः॑ ॥ अर᳚म् । वह॑न्ति । आ॒शवः॑ ॥ यु॒क्ष्व । हि । दे॒व॒हूत॑मा॒निति॑ देव - हूत॑मान् । अश्वान्॑ । अ॒ग्ने॒ । र॒थीः । इ॒व॒ ॥ नीति॑ । होता᳚ । पू॒र्व्यः । स॒दः॒ ॥ द्र॒फ्सः । च॒स्क॒न्द॒ । पृ॒थि॒वीम् । अन्विति॑ । द्याम् । इ॒मम् । च॒ । योनि᳚म् । अन्विति॑ । यः । च॒ । पूर्वः॑ ॥ तृ॒तीय᳚म् । योनि᳚म् । अन्विति॑ । स॒ञ्चर॑न्त॒मिति॑ सं - चर॑न्तम् । द्र॒फ्सम् । जु॒हो॒मि॒ । अन्विति॑ । स॒प्त ।  \newline


\textbf{Krama Paata} \newline

ज्योति॑रधारयत् । अ॒धा॒र॒य॒थ् स॒म्राट् । स॒म्राड् ज्योतिः॑ । स॒म्राडिति॑ सम् - राट् । ज्योति॑रधारयत् । अ॒धा॒र॒य॒थ् स्व॒राट् । स्व॒राड् ज्योतिः॑ । स्व॒राडिति॑ स्व - राट् । ज्योति॑रधारयत् । अ॒धा॒र॒य॒दित्य॑धारयत् ॥ अग्ने॑ यु॒क्ष्व । यु॒क्ष्वा हि । हि ये । ये तव॑ । तवाश्वा॑सः । अश्वा॑सो देव । दे॒व॒ सा॒धवः॑ । सा॒धव॒ इति॑ सा॒धवः॑ ॥ अरं॒ ॅवह॑न्ति । वह॑न्त्या॒शवः॑ । आ॒शव॒ इत्या॒शवः॑ ॥ यु॒क्ष्वा हि । हि दे॑व॒हूत॑मान् । दे॒व॒हूत॑माꣳ॒॒ अश्वान्॑ । दे॒व॒हूत॑मा॒निति॑ देव - हूत॑मान् । अश्वाꣳ॑ अग्ने । अ॒ग्ने॒ र॒थीः । र॒थीरि॑व । इ॒वेती॑व ॥ नि होता᳚ । होता॑ पू॒र्व्यः । पू॒र्व्यः स॑दः । स॒द॒ इति॑ सदः ॥ द्र॒फ्सश्च॑स्कन्द । च॒स्क॒न्द॒ पृ॒थि॒वीम् । पृ॒थि॒वीमनु॑ । अनु॒ द्याम् । द्यामि॒मम् । इ॒मम् च॑ । च॒ योनि᳚म् । योनि॒मनु॑ । अनु॒ यः । यश्च॑ । च॒ पूर्वः॑ । पूर्व॒ इति॒ पूर्वः॑ ॥ तृ॒तीयं॒ ॅयोनि᳚म् । योनि॒मनु॑ । अनु॑ स॒ञ्चर॑न्तम् । स॒ञ्चर॑न्तम् द्र॒फ्सम् । स॒ञ्चर॑न्त॒मिति॑ सम् - चर॑न्तम् । द्र॒फ्सम् जु॑होमि । जु॒हो॒म्यनु॑ । अनु॑ स॒प्त । स॒प्त होत्राः᳚ \newline

\textbf{Jatai Paata} \newline

1. ज्योति॑ रधारय दधारय॒ज् ज्योति॒र् ज्योति॑ रधारयत् । \newline
2. अ॒धा॒र॒य॒थ् स॒म्राट् थ्स॒म्रा ड॑धारय दधारयथ् स॒म्राट् । \newline
3. स॒म्राड् ज्योति॒र् ज्योतिः॑ स॒म्राट् थ्स॒म्राड् ज्योतिः॑ । \newline
4. स॒म्राडिति॑ सम् - राट् । \newline
5. ज्योति॑ रधारय दधारय॒ज् ज्योति॒र् ज्योति॑ रधारयत् । \newline
6. अ॒धा॒र॒य॒थ् स्व॒राट् थ् स्व॒रा ड॑धारय दधारयथ् स्व॒राट् । \newline
7. स्व॒राड् ज्योति॒र् ज्योतिः॑ स्व॒राट् थ्स्व॒राड् ज्योतिः॑ । \newline
8. स्व॒राडिति॑ स्व - राट् । \newline
9. ज्योति॑ रधारय दधारय॒ज् ज्योति॒र् ज्योति॑ रधारयत् । \newline
10. अ॒धा॒र॒य॒दित्य॑धारयत् । \newline
11. अग्ने॑ यु॒क्ष्व यु॒क्ष्वाग्ने ऽग्ने॑ यु॒क्ष्व । \newline
12. यु॒क्ष्वा हि हि यु॒क्ष्व यु॒क्ष्वा हि । \newline
13. हि ये ये हि हि ये । \newline
14. ये तव॒ तव॒ ये ये तव॑ । \newline
15. तवाश्वा॒सो ऽश्वा॑स॒ स्तव॒ तवाश्वा॑सः । \newline
16. अश्वा॑सो देव दे॒वा श्वा॒सो ऽश्वा॑सो देव । \newline
17. दे॒व॒ सा॒धवः॑ सा॒धवो॑ देव देव सा॒धवः॑ । \newline
18. सा॒धव॒ इति॑ सा॒धवः॑ । \newline
19. अरं॒ ॅवह॑न्ति॒ वह॒ न्त्यर॒ मरं॒ ॅवह॑न्ति । \newline
20. वह॑ न्त्या॒शव॑ आ॒शवो॒ वह॑न्ति॒ वह॑ न्त्या॒शवः॑ । \newline
21. आ॒शव॒ इत्या॒शवः॑ । \newline
22. यु॒क्ष्वा हि हि यु॒क्ष्व यु॒क्ष्वा हि । \newline
23. हि दे॑व॒हूत॑मान् देव॒हूत॑मा॒न्॒. हि हि दे॑व॒हूत॑मान् । \newline
24. दे॒व॒हूत॑माꣳ॒॒ अश्वाꣳ॒॒ अश्वा᳚न् देव॒हूत॑मान् देव॒हूत॑माꣳ॒॒ अश्वान्॑ । \newline
25. दे॒व॒हूत॑मा॒निति॑ देव - हूत॑मान् । \newline
26. अश्वाꣳ॑ अग्ने अ॒ग्ने ऽश्वाꣳ॒॒ अश्वाꣳ॑ अग्ने । \newline
27. अ॒ग्ने॒ र॒थी र॒थी र॑ग्ने अग्ने र॒थीः । \newline
28. र॒थी रि॑वेव र॒थी र॒थी रि॑व । \newline
29. इ॒वेती॑व । \newline
30. नि होता॒ होता॒ नि नि होता᳚ । \newline
31. होता॑ पू॒र्व्यः पू॒र्व्यो होता॒ होता॑ पू॒र्व्यः । \newline
32. पू॒र्व्यः स॑दः सदः पू॒र्व्यः पू॒र्व्यः स॑दः । \newline
33. स॒द॒ इति॑ सदः । \newline
34. द्र॒फ्स श्च॑स्कन्द चस्कन्द द्र॒फ्सो द्र॒फ्स श्च॑स्कन्द । \newline
35. च॒स्क॒न्द॒ पृ॒थि॒वीम् पृ॑थि॒वीम् च॑स्कन्द चस्कन्द पृथि॒वीम् । \newline
36. पृ॒थि॒वी मन्वनु॑ पृथि॒वीम् पृ॑थि॒वी मनु॑ । \newline
37. अनु॒ द्याम् द्या मन्वनु॒ द्याम् । \newline
38. द्या मि॒म मि॒मम् द्याम् द्या मि॒मम् । \newline
39. इ॒मम् च॑ चे॒म मि॒मम् च॑ । \newline
40. च॒ योनिं॒ ॅयोनि॑म् च च॒ योनि᳚म् । \newline
41. योनि॒ मन्वनु॒ योनिं॒ ॅयोनि॒ मनु॑ । \newline
42. अनु॒ यो यो ऽन्वनु॒ यः । \newline
43. यश्च॑ च॒ यो यश्च॑ । \newline
44. च॒ पूर्वः॒ पूर्व॑श्च च॒ पूर्वः॑ । \newline
45. पूर्व॒ इति॒ पूर्वः॑ । \newline
46. तृ॒तीयं॒ ॅयोनिं॒ ॅयोनि॑म् तृ॒तीय॑म् तृ॒तीयं॒ ॅयोनि᳚म् । \newline
47. योनि॒ मन्वनु॒ योनिं॒ ॅयोनि॒ मनु॑ । \newline
48. अनु॑ स॒ञ्चर॑न्तꣳ स॒ञ्चर॑न्त॒ मन्वनु॑ स॒ञ्चर॑न्तम् । \newline
49. स॒ञ्चर॑न्तम् द्र॒फ्सम् द्र॒फ्सꣳ स॒ञ्चर॑न्तꣳ स॒ञ्चर॑न्तम् द्र॒फ्सम् । \newline
50. स॒ञ्चर॑न्त॒मिति॑ सं - चर॑न्तम् । \newline
51. द्र॒फ्सम् जु॑होमि जुहोमि द्र॒फ्सम् द्र॒फ्सम् जु॑होमि । \newline
52. जु॒हो॒ म्यन्वनु॑ जुहोमि जुहो॒ म्यनु॑ । \newline
53. अनु॑ स॒प्त स॒प्तान्वनु॑ स॒प्त । \newline
54. स॒प्त होत्रा॒ होत्राः᳚ स॒प्त स॒प्त होत्राः᳚ । \newline

\textbf{Ghana Paata } \newline

1. ज्योति॑ रधारय दधारय॒ज् ज्योति॒र् ज्योति॑ रधारयथ् स॒म्राट् थ्स॒म्रा ड॑धारय॒ज् ज्योति॒र् ज्योति॑ रधारयथ् स॒म्राट् । \newline
2. अ॒धा॒र॒य॒थ् स॒म्राट् थ्स॒म्रा ड॑धारय दधारयथ् स॒म्राड् ज्योति॒र् ज्योतिः॑ स॒म्रा ड॑धारय दधारयथ् स॒म्राड् ज्योतिः॑ । \newline
3. स॒म्राड् ज्योति॒र् ज्योतिः॑ स॒म्राट् थ्स॒म्राड् ज्योति॑ रधारय दधारय॒ज् ज्योतिः॑ स॒म्राट् थ्स॒म्राड् ज्योति॑ रधारयत् । \newline
4. स॒म्राडिति॑ सम् - राट् । \newline
5. ज्योति॑ रधारय दधारय॒ज् ज्योति॒र् ज्योति॑ रधारयथ् स्व॒राट् थ्स्व॒रा ड॑धारय॒ज् ज्योति॒र् ज्योति॑ रधारयथ् स्व॒राट् । \newline
6. अ॒धा॒र॒य॒थ् स्व॒राट् थ्स्व॒रा ड॑धारय दधारयथ् स्व॒राड् ज्योति॒र् ज्योतिः॑ स्व॒रा ड॑धारय दधारयथ् स्व॒राड् ज्योतिः॑ । \newline
7. स्व॒राड् ज्योति॒र् ज्योतिः॑ स्व॒राट् थ्स्व॒राड् ज्योति॑ रधारय दधारय॒ज् ज्योतिः॑ स्व॒राट् थ्स्व॒राड् ज्योति॑ रधारयत् । \newline
8. स्व॒राडिति॑ स्व - राट् । \newline
9. ज्योति॑ रधारय दधारय॒ज् ज्योति॒र् ज्योति॑ रधारयत् । \newline
10. अ॒धा॒र॒य॒दित्य॑धारयत् । \newline
11. अग्ने॑ यु॒क्ष्व यु॒क्ष्वाग्ने ऽग्ने॑ यु॒क्ष्वा हि हि यु॒क्ष्वाग्ने ऽग्ने॑ यु॒क्ष्वा हि । \newline
12. यु॒क्ष्वा हि हि यु॒क्ष्व यु॒क्ष्वा हि ये ये हि यु॒क्ष्व यु॒क्ष्वा हि ये । \newline
13. हि ये ये हि हि ये तव॒ तव॒ ये हि हि ये तव॑ । \newline
14. ये तव॒ तव॒ ये ये तवाश्वा॒सो ऽश्वा॑स॒ स्तव॒ ये ये तवाश्वा॑सः । \newline
15. तवाश्वा॒सो ऽश्वा॑स॒ स्तव॒ तवाश्वा॑सो देव दे॒वाश्वा॑स॒ स्तव॒ तवाश्वा॑सो देव । \newline
16. अश्वा॑सो देव दे॒वाश्वा॒सो ऽश्वा॑सो देव सा॒धवः॑ सा॒धवो॑ दे॒वाश्वा॒सो ऽश्वा॑सो देव सा॒धवः॑ । \newline
17. दे॒व॒ सा॒धवः॑ सा॒धवो॑ देव देव सा॒धवः॑ । \newline
18. सा॒धव॒ इति॑ सा॒धवः॑ । \newline
19. अरं॒ ॅवह॑न्ति॒ वह॒न् त्यर॒ मरं॒ ॅवह॑न् त्या॒शव॑ आ॒शवो॒ वह॒न् त्यर॒ मरं॒ ॅवह॑न् त्या॒शवः॑ । \newline
20. वह॑न् त्या॒शव॑ आ॒शवो॒ वह॑न्ति॒ वह॑न् त्या॒शवः॑ । \newline
21. आ॒शव॒ इत्या॒शवः॑ । \newline
22. यु॒क्ष्वा हि हि यु॒क्ष्व यु॒क्ष्वा हि दे॑व॒हूत॑मान् देव॒हूत॑मा॒न्॒. हि यु॒क्ष्व यु॒क्ष्वा हि दे॑व॒हूत॑मान् । \newline
23. हि दे॑व॒हूत॑मान् देव॒हूत॑मा॒न्॒. हि हि दे॑व॒हूत॑माꣳ॒॒ अश्वाꣳ॒॒ अश्वा᳚न् देव॒हूत॑मा॒न्॒. हि हि दे॑व॒हूत॑माꣳ॒॒ अश्वान्॑ । \newline
24. दे॒व॒हूत॑माꣳ॒॒ अश्वाꣳ॒॒ अश्वा᳚न् देव॒हूत॑मान् देव॒हूत॑माꣳ॒॒ अश्वाꣳ॑ अग्ने अ॒ग्ने ऽश्वा᳚न् देव॒हूत॑मान् देव॒हूत॑माꣳ॒॒ अश्वाꣳ॑ अग्ने । \newline
25. दे॒व॒हूत॑मा॒निति॑ देव - हूत॑मान् । \newline
26. अश्वाꣳ॑ अग्ने अ॒ग्ने ऽश्वाꣳ॒॒ अश्वाꣳ॑ अग्ने र॒थी र॒थी र॒ग्ने ऽश्वाꣳ॒॒ अश्वाꣳ॑ अग्ने र॒थीः । \newline
27. अ॒ग्ने॒ र॒थी र॒थी र॑ग्ने अग्ने र॒थी रि॑वेव र॒थीर॑ग्ने अग्ने र॒थी रि॑व । \newline
28. र॒थी रि॑वेव र॒थी र॒थी रि॑व । \newline
29. इ॒वेती॑व । \newline
30. नि होता॒ होता॒ नि नि होता॑ पू॒र्व्यः पू॒र्व्यो होता॒ नि नि होता॑ पू॒र्व्यः । \newline
31. होता॑ पू॒र्व्यः पू॒र्व्यो होता॒ होता॑ पू॒र्व्यः स॑दः सदः पू॒र्व्यो होता॒ होता॑ पू॒र्व्यः स॑दः । \newline
32. पू॒र्व्यः स॑दः सदः पू॒र्व्यः पू॒र्व्यः स॑दः । \newline
33. स॒द॒ इति॑ सदः । \newline
34. द्र॒फ्स श्च॑स्कन्द चस्कन्द द्र॒फ्सो द्र॒फ्स श्च॑स्कन्द पृथि॒वीम् पृ॑थि॒वीम् च॑स्कन्द द्र॒फ्सो द्र॒फ्स श्च॑स्कन्द पृथि॒वीम् । \newline
35. च॒स्क॒न्द॒ पृ॒थि॒वीम् पृ॑थि॒वीम् च॑स्कन्द चस्कन्द पृथि॒वी मन्वनु॑ पृथि॒वीम् च॑स्कन्द चस्कन्द पृथि॒वी मनु॑ । \newline
36. पृ॒थि॒वी मन्वनु॑ पृथि॒वीम् पृ॑थि॒वी मनु॒ द्याम् द्या मनु॑ पृथि॒वीम् पृ॑थि॒वी मनु॒ द्याम् । \newline
37. अनु॒ द्याम् द्या मन्वनु॒ द्या मि॒म मि॒मम् द्या मन्वनु॒ द्या मि॒मम् । \newline
38. द्या मि॒म मि॒मम् द्याम् द्या मि॒मम् च॑ चे॒ मम् द्याम् द्या मि॒मम् च॑ । \newline
39. इ॒मम् च॑ चे॒ म मि॒मम् च॒ योनिं॒ ॅयोनि॑म् चे॒ म मि॒मम् च॒ योनि᳚म् । \newline
40. च॒ योनिं॒ ॅयोनि॑म् च च॒ योनि॒ मन्वनु॒ योनि॑म् च च॒ योनि॒ मनु॑ । \newline
41. योनि॒ मन्वनु॒ योनिं॒ ॅयोनि॒ मनु॒ यो यो ऽनु॒ योनिं॒ ॅयोनि॒ मनु॒ यः । \newline
42. अनु॒ यो यो ऽन्वनु॒ यश्च॑ च॒ यो ऽन्वनु॒ यश्च॑ । \newline
43. यश्च॑ च॒ यो यश्च॒ पूर्वः॒ पूर्व॑श्च॒ यो यश्च॒ पूर्वः॑ । \newline
44. च॒ पूर्वः॒ पूर्व॑श्च च॒ पूर्वः॑ । \newline
45. पूर्व॒ इति॒ पूर्वः॑ । \newline
46. तृ॒तीयं॒ ॅयोनिं॒ ॅयोनि॑म् तृ॒तीय॑म् तृ॒तीयं॒ ॅयोनि॒ मन्वनु॒ योनि॑म् तृ॒तीय॑म् तृ॒तीयं॒ ॅयोनि॒ मनु॑ । \newline
47. योनि॒ मन्वनु॒ योनिं॒ ॅयोनि॒ मनु॑ स॒ञ्चर॑न्तꣳ स॒ञ्चर॑न्त॒ मनु॒ योनिं॒ ॅयोनि॒ मनु॑ स॒ञ्चर॑न्तम् । \newline
48. अनु॑ स॒ञ्चर॑न्तꣳ स॒ञ्चर॑न्त॒ मन्वनु॑ स॒ञ्चर॑न्तम् द्र॒फ्सम् द्र॒फ्सꣳ स॒ञ्चर॑न्त॒ मन्वनु॑ स॒ञ्चर॑न्तम् द्र॒फ्सम् । \newline
49. स॒ञ्चर॑न्तम् द्र॒फ्सम् द्र॒फ्सꣳ स॒ञ्चर॑न्तꣳ स॒ञ्चर॑न्तम् द्र॒फ्सम् जु॑होमि जुहोमि द्र॒फ्सꣳ स॒ञ्चर॑न्तꣳ स॒ञ्चर॑न्तम् द्र॒फ्सम् जु॑होमि । \newline
50. स॒ञ्चर॑न्त॒मिति॑ सं - चर॑न्तम् । \newline
51. द्र॒फ्सम् जु॑होमि जुहोमि द्र॒फ्सम् द्र॒फ्सम् जु॑हो॒ म्यन्वनु॑ जुहोमि द्र॒फ्सम् द्र॒फ्सम् जु॑हो॒ म्यनु॑ । \newline
52. जु॒हो॒ म्यन्वनु॑ जुहोमि जुहो॒ म्यनु॑ स॒प्त स॒प्तानु॑ जुहोमि जुहो॒ म्यनु॑ स॒प्त । \newline
53. अनु॑ स॒प्त स॒प्तान्वनु॑ स॒प्त होत्रा॒ होत्राः᳚ स॒प्तान्वनु॑ स॒प्त होत्राः᳚ । \newline
54. स॒प्त होत्रा॒ होत्राः᳚ स॒प्त स॒प्त होत्राः᳚ । \newline
\pagebreak
\markright{ TS 4.2.9.6  \hfill https://www.vedavms.in \hfill}

\section{ TS 4.2.9.6 }

\textbf{TS 4.2.9.6 } \newline
\textbf{Samhita Paata} \newline

होत्राः᳚ ॥ अभू॑दि॒दं ॅविश्व॑स्य॒ भुव॑नस्य॒ वाजि॑नम॒ग्ने-र्वै᳚श्वान॒रस्य॑ च । अ॒ग्निर्ज्योति॑षा॒ ज्योति॑ष्मान् रु॒क्मो वर्च॑सा॒ वर्च॑स्वान् ॥ ऋ॒चे त्वा॑ रु॒चे त्वा॒ समिथ् स्र॑वन्ति स॒रितो॒ न धेनाः᳚ । अ॒न्तर्.हृ॒दा मन॑सा पू॒यमा॑नाः ॥ घृ॒तस्य॒ धारा॑ अ॒भि चा॑कशीमि । हि॒र॒ण्ययो॑ वेत॒सो मद्ध्य॑ आसां ॥ तस्मिन्᳚थ्सुप॒र्णो म॑धु॒कृत् कु॑ला॒यी भज॑न्नास्ते॒ मधु॑ दे॒वता᳚भ्यः । तस्या॑ स ते॒ हर॑यः स॒प्त तीरे᳚ ( ) स्व॒धां दुहा॑ना अ॒मृत॑स्य॒ धारां᳚ ॥ \newline

\textbf{Pada Paata} \newline

होत्राः᳚ ॥ अभू᳚त् । इ॒दम् । विश्व॑स्य । भुव॑नस्य । वाजि॑नम् । अ॒ग्नेः । वै॒श्वा॒न॒रस्य॑ । च॒ ॥ अ॒ग्निः । ज्योति॑षा । ज्योति॑ष्मान् । रु॒क्मः । वर्च॑सा । वर्च॑स्वान् ॥ ऋ॒चे । त्वा॒ । रु॒चे । त्वा॒ । समिति॑ । इत् । स्र॒व॒न्ति॒ । स॒रितः॑ । न । धेनाः᳚ ॥ अ॒न्तः । हृ॒दा । मन॑सा । पू॒यमा॑नाः ॥ घृ॒तस्य॑ । धाराः᳚ । अ॒भीति॑ । चा॒क॒शी॒मि॒ ॥ हि॒र॒ण्ययः॑ । वे॒त॒सः । मद्ध्ये᳚ । आ॒सा॒म् ॥ तस्मिन्न्॑ । सु॒प॒र्ण इति॑ सु - प॒र्णः । म॒धु॒कृदिति॑ मधु - कृत् । कु॒ला॒यी । भजन्न्॑ । आ॒स्ते॒ । मधु॑ । दे॒वता᳚भ्यः ॥ तस्य॑ । आ॒स॒ते॒ । हर॑यः । स॒प्त । तीरे᳚ ( ) । स्व॒धामिति॑ स्व-धाम् । दुहा॑नाः । अ॒मृत॑स्य । धारा᳚म् ॥  \newline


\textbf{Krama Paata} \newline

होत्रा॒ इति॒ होत्राः᳚ ॥ अभू॑दि॒दम् । इ॒दं ॅविश्व॑स्य । विश्व॑स्य॒ भुव॑नस्य । भुव॑नस्य॒ वाजि॑नम् । वाजि॑नम॒ग्नेः । अ॒ग्नेर् वै᳚श्वान॒रस्य॑ । वै॒श्वा॒न॒रस्य॑ च । चेति॑ च ॥ अ॒ग्निर् ज्योति॑षा । ज्योति॑षा॒ ज्योति॑ष्मान् । ज्योति॑ष्मान् रु॒क्मः । रु॒क्मो वर्च॑सा । वर्च॑सा॒ वर्च॑स्वान् । वर्च॑स्वा॒निति॒ वर्च॑स्वान् ॥ ऋ॒चे त्वा᳚ । त्वा॒ रु॒चे । रु॒चे त्वा᳚ । त्वा॒ सम् । समित् । इथ् स्र॑वन्ति । स्र॒व॒न्ति॒ स॒रितः॑ । स॒रितो॒ न । न धेनाः᳚ । धेना॒ इति॒ धेनाः᳚ ॥ अ॒न्तर्. हृ॒दा । हृ॒दा मन॑सा । मन॑सा पू॒यमा॑नाः । पू॒यमा॑ना॒ इति॑ पू॒यमा॑नाः ॥ घृ॒तस्य॒ धाराः᳚ । धारा॑ अ॒भि । अ॒भि चा॑कशीमि । चा॒क॒शी॒मीति॑ चाकशीमि ॥ हि॒र॒ण्ययो॑ वेत॒सः । वे॒त॒सो मद्ध्ये᳚ । मद्ध्य॑ आसाम् । आ॒सा॒मित्या॑साम् ॥ तस्मि᳚न्थ् सुप॒र्णः । सु॒प॒र्णो म॑धु॒कृत् । सु॒प॒र्ण इति॑ सु - प॒र्णः । म॒धु॒कृत् कु॑ला॒यी । म॒धु॒कृदिति॑ मधु - कृत् । कु॒ला॒यी भजन्न्॑ । भज॑न्नास्ते । आ॒स्ते॒ मधु॑ । मधु॑ दे॒वता᳚भ्यः । दे॒वता᳚भ्य॒ इति॑ दे॒वता᳚भ्यः ॥ तस्या॑सते । आ॒स॒ते॒ हर॑यः । हर॑यः स॒प्त । स॒प्त तीरे᳚ ( ) । तीरे᳚ स्व॒धाम् । स्व॒धाम् दुहा॑नाः । स्व॒धामिति॑ स्व - धाम् । दुहा॑ना अ॒मृत॑स्य । अ॒मृत॑स्य॒ धारा᳚म् । धारा॒मिति॒ धारा᳚म् । \newline

\textbf{Jatai Paata} \newline

1. होत्रा॒ इति॒ होत्राः᳚ । \newline
2. अभू॑ दि॒द मि॒द मभू॒ दभू॑ दि॒दम् । \newline
3. इ॒दं ॅविश्व॑स्य॒ विश्व॑स्ये॒द मि॒दं ॅविश्व॑स्य । \newline
4. विश्व॑स्य॒ भुव॑नस्य॒ भुव॑नस्य॒ विश्व॑स्य॒ विश्व॑स्य॒ भुव॑नस्य । \newline
5. भुव॑नस्य॒ वाजि॑नं॒ ॅवाजि॑न॒म् भुव॑नस्य॒ भुव॑नस्य॒ वाजि॑नम् । \newline
6. वाजि॑न म॒ग्ने र॒ग्नेर् वाजि॑नं॒ ॅवाजि॑न म॒ग्नेः । \newline
7. अ॒ग्नेर् वै᳚श्वान॒रस्य॑ वैश्वान॒रस्या॒ग्ने र॒ग्नेर् वै᳚श्वान॒रस्य॑ । \newline
8. वै॒श्वा॒न॒रस्य॑ च च वैश्वान॒रस्य॑ वैश्वान॒रस्य॑ च । \newline
9. चेति॑ च । \newline
10. अ॒ग्निर् ज्योति॑षा॒ ज्योति॑षा॒ ऽग्नि र॒ग्निर् ज्योति॑षा । \newline
11. ज्योति॑षा॒ ज्योति॑ष्मा॒न् ज्योति॑ष्मा॒न् ज्योति॑षा॒ ज्योति॑षा॒ ज्योति॑ष्मान् । \newline
12. ज्योति॑ष्मान् रु॒क्मो रु॒क्मो ज्योति॑ष्मा॒न् ज्योति॑ष्मान् रु॒क्मः । \newline
13. रु॒क्मो वर्च॑सा॒ वर्च॑सा रु॒क्मो रु॒क्मो वर्च॑सा । \newline
14. वर्च॑सा॒ वर्च॑स्वा॒न्॒. वर्च॑स्वा॒न्॒. वर्च॑सा॒ वर्च॑सा॒ वर्च॑स्वान् । \newline
15. वर्च॑स्वा॒निति॒ वर्च॑स्वान् । \newline
16. ऋ॒चे त्वा᳚ त्व॒ र्‌च ऋ॒चे त्वा᳚ । \newline
17. त्वा॒ रु॒चे रु॒चे त्वा᳚ त्वा रु॒चे । \newline
18. रु॒चे त्वा᳚ त्वा रु॒चे रु॒चे त्वा᳚ । \newline
19. त्वा॒ सꣳ सम् त्वा᳚ त्वा॒ सम् । \newline
20. स मिदिथ् सꣳ स मित् । \newline
21. इथ् स्र॑वन्ति स्रव॒न्तीदिथ् स्र॑वन्ति । \newline
22. स्र॒व॒न्ति॒ स॒रितः॑ स॒रितः॑ स्रवन्ति स्रवन्ति स॒रितः॑ । \newline
23. स॒रितो॒ न न स॒रितः॑ स॒रितो॒ न । \newline
24. न धेना॒ धेना॒ न न धेनाः᳚ । \newline
25. धेना॒ इति॒ धेनाः᳚ । \newline
26. अ॒न्तर्. हृ॒दा हृ॒दा ऽन्त र॒न्तर्. हृ॒दा । \newline
27. हृ॒दा मन॑सा॒ मन॑सा हृ॒दा हृ॒दा मन॑सा । \newline
28. मन॑सा पू॒यमा॑नाः पू॒यमा॑ना॒ मन॑सा॒ मन॑सा पू॒यमा॑नाः । \newline
29. पू॒यमा॑ना॒ इति॑ पू॒यमा॑नाः । \newline
30. घृ॒तस्य॒ धारा॒ धारा॑ घृ॒तस्य॑ घृ॒तस्य॒ धाराः᳚ । \newline
31. धारा॑ अ॒भ्य॑भि धारा॒ धारा॑ अ॒भि । \newline
32. अ॒भि चा॑कशीमि चाकशी म्य॒भ्य॑भि चा॑कशीमि । \newline
33. चा॒क॒शी॒मीति॑ चाकशीमि । \newline
34. हि॒र॒ण्ययो॑ वेत॒सो वे॑त॒सो हि॑र॒ण्ययो॑ हिर॒ण्ययो॑ वेत॒सः । \newline
35. वे॒त॒सो मद्ध्ये॒ मद्ध्ये॑ वेत॒सो वे॑त॒सो मद्ध्ये᳚ । \newline
36. मद्ध्य॑ आसा मासा॒म् मद्ध्ये॒ मद्ध्य॑ आसाम् । \newline
37. आ॒सा॒मित्या॑साम् । \newline
38. तस्मिन्᳚ थ्सुप॒र्णः सु॑प॒र्ण स्तस्मिꣳ॒॒ स्तस्मिन्᳚ थ्सुप॒र्णः । \newline
39. सु॒प॒र्णो म॑धु॒कृन् म॑धु॒कृथ् सु॑प॒र्णः सु॑प॒र्णो म॑धु॒कृत् । \newline
40. सु॒प॒र्ण इति॑ सु - प॒र्णः । \newline
41. म॒धु॒कृत् कु॑ला॒यी कु॑ला॒यी म॑धु॒कृन् म॑धु॒कृत् कु॑ला॒यी । \newline
42. म॒धु॒कृदिति॑ मधु - कृत् । \newline
43. कु॒ला॒यी भज॒न् भज॑न् कुला॒यी कु॑ला॒यी भजन्न्॑ । \newline
44. भज॑न् नास्त आस्ते॒ भज॒न् भज॑न् नास्ते । \newline
45. आ॒स्ते॒ मधु॒ मध्वा᳚स्त आस्ते॒ मधु॑ । \newline
46. मधु॑ दे॒वता᳚भ्यो दे॒वता᳚भ्यो॒ मधु॒ मधु॑ दे॒वता᳚भ्यः । \newline
47. दे॒वता᳚भ्य॒ इति॑ दे॒वता᳚भ्यः । \newline
48. तस्या॑सत आसते॒ तस्य॒ तस्या॑सते । \newline
49. आ॒स॒ते॒ हर॑यो॒ हर॑य आसत आसते॒ हर॑यः । \newline
50. हर॑यः स॒प्त स॒प्त हर॑यो॒ हर॑यः स॒प्त । \newline
51. स॒प्त तीरे॒ तीरे॑ स॒प्त स॒प्त तीरे᳚ । \newline
52. तीरे᳚ स्व॒धाꣳ स्व॒धाम् तीरे॒ तीरे᳚ स्व॒धाम् । \newline
53. स्व॒धाम् दुहा॑ना॒ दुहा॑नाः स्व॒धाꣳ स्व॒धाम् दुहा॑नाः । \newline
54. स्व॒धामिति॑ स्व - धाम् । \newline
55. दुहा॑ना अ॒मृत॑स्या॒ मृत॑स्य॒ दुहा॑ना॒ दुहा॑ना अ॒मृत॑स्य । \newline
56. अ॒मृत॑स्य॒ धारा॒म् धारा॑ म॒मृत॑स्या॒ मृत॑स्य॒ धारा᳚म् । \newline
57. धारा॒मिति॒ धारा᳚म् । \newline

\textbf{Ghana Paata } \newline

1. होत्रा॒ इति॒ होत्राः᳚ । \newline
2. अभू॑ दि॒द मि॒द मभू॒ दभू॑ दि॒दं ॅविश्व॑स्य॒ विश्व॑स्ये॒द मभू॒ दभू॑ दि॒दं ॅविश्व॑स्य । \newline
3. इ॒दं ॅविश्व॑स्य॒ विश्व॑स्ये॒द मि॒दं ॅविश्व॑स्य॒ भुव॑नस्य॒ भुव॑नस्य॒ विश्व॑स्ये॒द मि॒दं ॅविश्व॑स्य॒ भुव॑नस्य । \newline
4. विश्व॑स्य॒ भुव॑नस्य॒ भुव॑नस्य॒ विश्व॑स्य॒ विश्व॑स्य॒ भुव॑नस्य॒ वाजि॑नं॒ ॅवाजि॑न॒म् भुव॑नस्य॒ विश्व॑स्य॒ विश्व॑स्य॒ भुव॑नस्य॒ वाजि॑नम् । \newline
5. भुव॑नस्य॒ वाजि॑नं॒ ॅवाजि॑न॒म् भुव॑नस्य॒ भुव॑नस्य॒ वाजि॑न म॒ग्ने र॒ग्नेर् वाजि॑न॒म् भुव॑नस्य॒ भुव॑नस्य॒ वाजि॑न म॒ग्नेः । \newline
6. वाजि॑न म॒ग्ने र॒ग्नेर् वाजि॑नं॒ ॅवाजि॑न म॒ग्नेर् वै᳚श्वान॒रस्य॑ वैश्वान॒रस्या॒ग्नेर् वाजि॑नं॒ ॅवाजि॑न म॒ग्नेर् वै᳚श्वान॒रस्य॑ । \newline
7. अ॒ग्नेर् वै᳚श्वान॒रस्य॑ वैश्वान॒रस्या॒ ग्ने र॒ग्नेर् वै᳚श्वान॒रस्य॑ च च वैश्वान॒रस्या॒ ग्ने र॒ग्नेर् वै᳚श्वान॒रस्य॑ च । \newline
8. वै॒श्वा॒न॒रस्य॑ च च वैश्वान॒रस्य॑ वैश्वान॒रस्य॑ च । \newline
9. चेति॑ च । \newline
10. अ॒ग्निर् ज्योति॑षा॒ ज्योति॑षा॒ ऽग्निर॒ग्निर् ज्योति॑षा॒ ज्योति॑ष्मा॒न् ज्योति॑ष्मा॒न् ज्योति॑षा॒ ऽग्निर॒ग्निर् ज्योति॑षा॒ ज्योति॑ष्मान् । \newline
11. ज्योति॑षा॒ ज्योति॑ष्मा॒न् ज्योति॑ष्मा॒न् ज्योति॑षा॒ ज्योति॑षा॒ ज्योति॑ष्मान् रु॒क्मो रु॒क्मो ज्योति॑ष्मा॒न् ज्योति॑षा॒ ज्योति॑षा॒ ज्योति॑ष्मान् रु॒क्मः । \newline
12. ज्योति॑ष्मान् रु॒क्मो रु॒क्मो ज्योति॑ष्मा॒न् ज्योति॑ष्मान् रु॒क्मो वर्च॑सा॒ वर्च॑सा रु॒क्मो ज्योति॑ष्मा॒न् ज्योति॑ष्मान् रु॒क्मो वर्च॑सा । \newline
13. रु॒क्मो वर्च॑सा॒ वर्च॑सा रु॒क्मो रु॒क्मो वर्च॑सा॒ वर्च॑स्वा॒न्॒. वर्च॑स्वा॒न्॒. वर्च॑सा रु॒क्मो रु॒क्मो वर्च॑सा॒ वर्च॑स्वान् । \newline
14. वर्च॑सा॒ वर्च॑स्वा॒न्॒. वर्च॑स्वा॒न्॒. वर्च॑सा॒ वर्च॑सा॒ वर्च॑स्वान् । \newline
15. वर्च॑स्वा॒निति॒ वर्च॑स्वान् । \newline
16. ऋ॒चे त्वा᳚ त्व॒र्‌च ऋ॒चे त्वा॑ रु॒चे रु॒चे त्व॒र्‌च ऋ॒चे त्वा॑ रु॒चे । \newline
17. त्वा॒ रु॒चे रु॒चे त्वा᳚ त्वा रु॒चे त्वा᳚ त्वा रु॒चे त्वा᳚ त्वा रु॒चे त्वा᳚ । \newline
18. रु॒चे त्वा᳚ त्वा रु॒चे रु॒चे त्वा॒ सꣳ सम् त्वा॑ रु॒चे रु॒चे त्वा॒ सम् । \newline
19. त्वा॒ सꣳ सम् त्वा᳚ त्वा॒ स मिदिथ् सम् त्वा᳚ त्वा॒ स मित् । \newline
20. स मिदिथ् सꣳ स मिथ् स्र॑वन्ति स्रव॒न्तीथ् सꣳ स मिथ् स्र॑वन्ति । \newline
21. इथ् स्र॑वन्ति स्रव॒न्तीदिथ् स्र॑वन्ति स॒रितः॑ स॒रितः॑ स्रव॒न्तीदिथ् स्र॑वन्ति स॒रितः॑ । \newline
22. स्र॒व॒न्ति॒ स॒रितः॑ स॒रितः॑ स्रवन्ति स्रवन्ति स॒रितो॒ न न स॒रितः॑ स्रवन्ति स्रवन्ति स॒रितो॒ न । \newline
23. स॒रितो॒ न न स॒रितः॑ स॒रितो॒ न धेना॒ धेना॒ न स॒रितः॑ स॒रितो॒ न धेनाः᳚ । \newline
24. न धेना॒ धेना॒ न न धेनाः᳚ । \newline
25. धेना॒ इति॒ धेनाः᳚ । \newline
26. अ॒न्तर्. हृ॒दा हृ॒दा ऽन्त र॒न्तर्. हृ॒दा मन॑सा॒ मन॑सा हृ॒दा ऽन्त र॒न्तर्. हृ॒दा मन॑सा । \newline
27. हृ॒दा मन॑सा॒ मन॑सा हृ॒दा हृ॒दा मन॑सा पू॒यमा॑नाः पू॒यमा॑ना॒ मन॑सा हृ॒दा हृ॒दा मन॑सा पू॒यमा॑नाः । \newline
28. मन॑सा पू॒यमा॑नाः पू॒यमा॑ना॒ मन॑सा॒ मन॑सा पू॒यमा॑नाः । \newline
29. पू॒यमा॑ना॒ इति॑ पू॒यमा॑नाः । \newline
30. घृ॒तस्य॒ धारा॒ धारा॑ घृ॒तस्य॑ घृ॒तस्य॒ धारा॑ अ॒भ्य॑भि धारा॑ घृ॒तस्य॑ घृ॒तस्य॒ धारा॑ अ॒भि । \newline
31. धारा॑ अ॒भ्य॑भि धारा॒ धारा॑ अ॒भि चा॑कशीमि चाकशी म्य॒भि धारा॒ धारा॑ अ॒भि चा॑कशीमि । \newline
32. अ॒भि चा॑कशीमि चाकशी म्य॒भ्य॑भि चा॑कशीमि । \newline
33. चा॒क॒शी॒मीति॑ चाकशीमि । \newline
34. हि॒र॒ण्ययो॑ वेत॒सो वे॑त॒सो हि॑र॒ण्ययो॑ हिर॒ण्ययो॑ वेत॒सो मद्ध्ये॒ मद्ध्ये॑ वेत॒सो हि॑र॒ण्ययो॑ हिर॒ण्ययो॑ वेत॒सो मद्ध्ये᳚ । \newline
35. वे॒त॒सो मद्ध्ये॒ मद्ध्ये॑ वेत॒सो वे॑त॒सो मद्ध्य॑ आसा मासा॒म् मद्ध्ये॑ वेत॒सो वे॑त॒सो मद्ध्य॑ आसाम् । \newline
36. मद्ध्य॑ आसा मासा॒म् मद्ध्ये॒ मद्ध्य॑ आसाम् । \newline
37. आ॒सा॒मित्या॑साम् । \newline
38. तस्मिन्᳚ थ्सुप॒र्णः सु॑प॒र्ण स्तस्मिꣳ॒॒ स्तस्मिन्᳚ थ्सुप॒र्णो म॑धु॒कृन् म॑धु॒कृथ् सु॑प॒र्ण स्तस्मिꣳ॒॒
स्तस्मिन्᳚ थ्सुप॒र्णो म॑धु॒कृत् । \newline
39. सु॒प॒र्णो म॑धु॒कृन् म॑धु॒कृथ् सु॑प॒र्णः सु॑प॒र्णो म॑धु॒कृत् कु॑ला॒यी कु॑ला॒यी म॑धु॒कृथ् सु॑प॒र्णः सु॑प॒र्णो म॑धु॒कृत् कु॑ला॒यी । \newline
40. सु॒प॒र्ण इति॑ सु - प॒र्णः । \newline
41. म॒धु॒कृत् कु॑ला॒यी कु॑ला॒यी म॑धु॒कृन् म॑धु॒कृत् कु॑ला॒यी भज॒न् भज॑न् कुला॒यी म॑धु॒कृन् म॑धु॒कृत् कु॑ला॒यी भजन्न्॑ । \newline
42. म॒धु॒कृदिति॑ मधु - कृत् । \newline
43. कु॒ला॒यी भज॒न् भज॑न् कुला॒यी कु॑ला॒यी भज॑न् नास्त आस्ते॒ भज॑न् कुला॒यी कु॑ला॒यी भज॑न् नास्ते । \newline
44. भज॑न् नास्त आस्ते॒ भज॒न् भज॑न् नास्ते॒ मधु॒ मध्वा᳚स्ते॒ भज॒न् भज॑न् नास्ते॒ मधु॑ । \newline
45. आ॒स्ते॒ मधु॒ मध्वा᳚स्त आस्ते॒ मधु॑ दे॒वता᳚भ्यो दे॒वता᳚भ्यो॒ मध्वा᳚स्त आस्ते॒ मधु॑ दे॒वता᳚भ्यः । \newline
46. मधु॑ दे॒वता᳚भ्यो दे॒वता᳚भ्यो॒ मधु॒ मधु॑ दे॒वता᳚भ्यः । \newline
47. दे॒वता᳚भ्य॒ इति॑ दे॒वता᳚भ्यः । \newline
48. तस्या॑सत आसते॒ तस्य॒ तस्या॑सते॒ हर॑यो॒ हर॑य आसते॒ तस्य॒ तस्या॑सते॒ हर॑यः । \newline
49. आ॒स॒ते॒ हर॑यो॒ हर॑य आसत आसते॒ हर॑यः स॒प्त स॒प्त हर॑य आसत आसते॒ हर॑यः स॒प्त । \newline
50. हर॑यः स॒प्त स॒प्त हर॑यो॒ हर॑यः स॒प्त तीरे॒ तीरे॑ स॒प्त हर॑यो॒ हर॑यः स॒प्त तीरे᳚ । \newline
51. स॒प्त तीरे॒ तीरे॑ स॒प्त स॒प्त तीरे᳚ स्व॒धाꣳ स्व॒धाम् तीरे॑ स॒प्त स॒प्त तीरे᳚ स्व॒धाम् । \newline
52. तीरे᳚ स्व॒धाꣳ स्व॒धाम् तीरे॒ तीरे᳚ स्व॒धाम् दुहा॑ना॒ दुहा॑नाः स्व॒धाम् तीरे॒ तीरे᳚ स्व॒धाम् दुहा॑नाः । \newline
53. स्व॒धाम् दुहा॑ना॒ दुहा॑नाः स्व॒धाꣳ स्व॒धाम् दुहा॑ना अ॒मृत॑स्या॒ मृत॑स्य॒ दुहा॑नाः स्व॒धाꣳ स्व॒धाम् दुहा॑ना अ॒मृत॑स्य । \newline
54. स्व॒धामिति॑ स्व - धाम् । \newline
55. दुहा॑ना अ॒मृत॑स्या॒ मृत॑स्य॒ दुहा॑ना॒ दुहा॑ना अ॒मृत॑स्य॒ धारा॒म् धारा॑ म॒मृत॑स्य॒ दुहा॑ना॒ दुहा॑ना अ॒मृत॑स्य॒ धारा᳚म् । \newline
56. अ॒मृत॑स्य॒ धारा॒म् धारा॑ म॒मृत॑स्या॒ मृत॑स्य॒ धारा᳚म् । \newline
57. धारा॒मिति॒ धारा᳚म् । \newline
\pagebreak
\markright{ TS 4.2.10.1  \hfill https://www.vedavms.in \hfill}

\section{ TS 4.2.10.1 }

\textbf{TS 4.2.10.1 } \newline
\textbf{Samhita Paata} \newline

आ॒दि॒त्यं गर्भं॒ पय॑सा सम॒ञ्जन्थ् स॒हस्र॑स्य प्रति॒मां ॅवि॒श्वरू॑पं । परि॑ वृङ्ग्धि॒ हर॑सा॒ माऽभि मृ॑क्षः श॒तायु॑षं कृणुहि ची॒यमा॑नः ॥ इ॒मं मा हिꣳ॑सीर्द्वि॒पादं॑ पशू॒नाꣳ सह॑स्राक्ष॒ मेध॒ आ ची॒यमा॑नः । म॒युमा॑र॒ण्यमनु॑ ते दिशामि॒ तेन॑ चिन्वा॒नस्त॒नुवो॒ नि षी॑द ॥ वात॑स्य॒ ध्राजिं॒ ॅवरु॑णस्य॒ नाभि॒मश्वं॑ जज्ञा॒नꣳ स॑रि॒रस्य॒ मद्ध्ये᳚ । शिशुं॑ न॒दीनाꣳ॒॒ हरि॒मद्रि॑बुद्ध॒मग्ने॒ मा हिꣳ॑सीः - [  ] \newline

\textbf{Pada Paata} \newline

आ॒दि॒त्यम् । गर्भ᳚म् । पय॑सा । स॒म॒ञ्जन्निति॑ सं - अ॒ञ्जन्न् । स॒हस्र॑स्य । प्र॒ति॒मामिति॑ प्रति - माम् । वि॒श्वरू॑प॒मिति॑ वि॒श्व-रू॒प॒म् ॥ परीति॑ । वृ॒ङ्ग्धि॒ । हर॑सा । मा । अ॒भीति॑ । मृ॒क्षः॒ । श॒तायु॑ष॒मिति॑ श॒त - आ॒यु॒ष॒म् । कृ॒णु॒हि॒ । ची॒यमा॑नः ॥ इ॒मम् । मा । हिꣳ॒॒सीः॒ । द्वि॒पाद॒मिति॑ द्वि - पाद᳚म् । प॒शू॒नाम् । सह॑स्रा॒क्षेति॒ सह॑स्र - अ॒क्ष॒ । मेधे᳚ । एति॑ । ची॒यमा॑नः ॥ म॒युम् । आ॒र॒ण्यम् । अन्विति॑ । ते॒ । दि॒शा॒मि॒ । तेन॑ । चि॒न्वा॒नः । त॒नुवः॑ । नीति॑ । सी॒द॒ ॥ वात॑स्य । ध्राजि᳚म् । वरु॑णस्य । नाभि᳚म् । अश्व᳚म् । ज॒ज्ञा॒नम् । स॒रि॒रस्य॑ । मद्ध्ये᳚ ॥ शिशु᳚म् । न॒दीना᳚म् । हरि᳚म् । अद्रि॑बुद्ध॒मित्यद्रि॑ - बु॒द्ध॒म् । अग्ने᳚ । मा । हिꣳ॒॒सीः॒ ।  \newline


\textbf{Krama Paata} \newline

आ॒दि॒त्यम् गर्भ᳚म् । गर्भ॒म् पय॑सा । पय॑सा सम॒ञ्जन्न् । स॒म॒ञ्जन्थ् स॒हस्र॑स्य । स॒म॒ञ्जन्निति॑ सं - अ॒ञ्जन्न् । स॒हस्र॑स्य प्रति॒माम् । प्र॒ति॒मां ॅवि॒श्वरू॑पम् । प्र॒ति॒मामिति॑ प्रति - माम् । वि॒श्वरू॑प॒मिति॑ वि॒श्व - रू॒प॒म् ॥ परि॑ वृङ्ग्धि । वृ॒ङ्ग्धि॒ हर॑सा । हर॑सा॒ मा । माऽभि । अ॒भि मृ॑क्षः । मृ॒क्षः॒ श॒तायु॑षम् । श॒तायु॑षम् कृणुहि । श॒तायु॑ष॒मिति॑ श॒त - आ॒यु॒ष॒म् । कृ॒णु॒हि॒ ची॒यमा॑नः । ची॒यमा॑न॒ इति॑ ची॒यमा॑नः ॥ इ॒मम् मा । मा हिꣳ॑सीः । हिꣳ॒॒सी॒र् द्वि॒पाद᳚म् । द्वि॒पाद॑म् पशू॒नाम् । द्वि॒पाद॒मिति॑ द्वि - पाद᳚म् । प॒शू॒नाꣳ सह॑स्राक्ष । सह॑स्राक्ष॒ मेधे᳚ । सह॑स्रा॒क्षेति॒ सह॑स्र - अ॒क्ष॒ । मेध॒ आ । आ ची॒यमा॑नः । ची॒यमा॑न॒ इति॑ ची॒यमा॑नः ॥ म॒युमा॑र॒ण्यम् । आ॒र॒ण्यमनु॑ । अनु॑ ते । ते॒ दि॒शा॒मि॒ । दि॒शा॒मि॒ तेन॑ । तेन॑ चिन्वा॒नः । चि॒न्वा॒नस्त॒नुवः॑ । त॒नुवो॒ नि । नि षी॑द । सी॒देति॑ सीद ॥ वात॑स्य॒ ध्राजि᳚म् । ध्राजिं॒ ॅवरु॑णस्य । वरु॑णस्य॒ नाभि᳚म् । नाभि॒मश्व᳚म् । अश्व॑म् जज्ञा॒नम् । ज॒ज्ञा॒नꣳ स॑रि॒रस्य॑ । स॒रि॒रस्य॒ मद्ध्ये᳚ । मद्ध्य॒ इति॒ मद्ध्ये᳚ ॥ शिशु॑म् न॒दीना᳚म् । न॒दीनाꣳ॒॒ हरि᳚म् । हरि॒मद्रि॑बुद्धम् । अद्रि॑बुद्ध॒मग्ने᳚ । अद्रि॑बुद्ध॒मित्यद्रि॑ - बु॒द्ध॒म् । अग्ने॒ मा । मा हिꣳ॑सीः । हिꣳ॒॒सीः॒ प॒र॒मे \newline

\textbf{Jatai Paata} \newline

1. आ॒दि॒त्यम् गर्भ॒म् गर्भ॑ मादि॒त्य मा॑दि॒त्यम् गर्भ᳚म् । \newline
2. गर्भ॒म् पय॑सा॒ पय॑सा॒ गर्भ॒म् गर्भ॒म् पय॑सा । \newline
3. पय॑सा सम॒ञ्जन् थ्स॑म॒ञ्जन् पय॑सा॒ पय॑सा सम॒ञ्जन्न् । \newline
4. स॒म॒ञ्जन् थ्स॒हस्र॑स्य स॒हस्र॑स्य सम॒ञ्जन् थ्स॑म॒ञ्जन् थ्स॒हस्र॑स्य । \newline
5. स॒म॒ञ्जन्निति॑ सं - अ॒ञ्जन्न् । \newline
6. स॒हस्र॑स्य प्रति॒माम् प्र॑ति॒माꣳ स॒हस्र॑स्य स॒हस्र॑स्य प्रति॒माम् । \newline
7. प्र॒ति॒मां ॅवि॒श्वरू॑पं ॅवि॒श्वरू॑पम् प्रति॒माम् प्र॑ति॒मां ॅवि॒श्वरू॑पम् । \newline
8. प्र॒ति॒मामिति॑ प्रति - माम् । \newline
9. वि॒श्वरू॑प॒मिति॑ वि॒श्व - रू॒प॒म् । \newline
10. परि॑ वृङ्ग्धि वृङ्ग्धि॒ परि॒ परि॑ वृङ्ग्धि । \newline
11. वृ॒ङ्ग्धि॒ हर॑सा॒ हर॑सा वृङ्ग्धि वृङ्ग्धि॒ हर॑सा । \newline
12. हर॑सा॒ मा मा हर॑सा॒ हर॑सा॒ मा । \newline
13. मा ऽभ्य॑भि मा मा ऽभि । \newline
14. अ॒भि मृ॑क्षो मृक्षो अ॒भ्य॑भि मृ॑क्षः । \newline
15. मृ॒क्षः॒ श॒तायु॑षꣳ श॒तायु॑षम् मृक्षो मृक्षः श॒तायु॑षम् । \newline
16. श॒तायु॑षम् कृणुहि कृणुहि श॒तायु॑षꣳ श॒तायु॑षम् कृणुहि । \newline
17. श॒तायु॑ष॒मिति॑ श॒त - आ॒यु॒ष॒म् । \newline
18. कृ॒णु॒हि॒ ची॒यमा॑न श्ची॒यमा॑नः कृणुहि कृणुहि ची॒यमा॑नः । \newline
19. ची॒यमा॑न॒ इति॑ ची॒यमा॑नः । \newline
20. इ॒मम् मा मेम मि॒मम् मा । \newline
21. मा हिꣳ॑सीर्. हिꣳसी॒र् मा मा हिꣳ॑सीः । \newline
22. हिꣳ॒॒सी॒र् द्वि॒पाद॑म् द्वि॒पादꣳ॑ हिꣳसीर्. हिꣳसीर् द्वि॒पाद᳚म् । \newline
23. द्वि॒पाद॑म् पशू॒नाम् प॑शू॒नाम् द्वि॒पाद॑म् द्वि॒पाद॑म् पशू॒नाम् । \newline
24. द्वि॒पाद॒मिति॑ द्वि - पाद᳚म् । \newline
25. प॒शू॒नाꣳ सह॑स्राक्ष॒ सह॑स्राक्ष पशू॒नाम् प॑शू॒नाꣳ सह॑स्राक्ष । \newline
26. सह॑स्राक्ष॒ मेधे॒ मेधे॒ सह॑स्राक्ष॒ सह॑स्राक्ष॒ मेधे᳚ । \newline
27. सह॑स्रा॒क्षेति॒ सह॑स्र - अ॒क्ष॒ । \newline
28. मेध॒ आ मेधे॒ मेध॒ आ । \newline
29. आ ची॒यमा॑न श्ची॒यमा॑न॒ आ ची॒यमा॑नः । \newline
30. ची॒यमा॑न॒ इति॑ ची॒यमा॑नः । \newline
31. म॒यु मा॑र॒ण्य मा॑र॒ण्यम् म॒युम् म॒यु मा॑र॒ण्यम् । \newline
32. आ॒र॒ण्य मन् वन् वा॑र॒ण्य मा॑र॒ण्य मनु॑ । \newline
33. अनु॑ ते ते॒ अन्वनु॑ ते । \newline
34. ते॒ दि॒शा॒मि॒ दि॒शा॒मि॒ ते॒ ते॒ दि॒शा॒मि॒ । \newline
35. दि॒शा॒मि॒ तेन॒ तेन॑ दिशामि दिशामि॒ तेन॑ । \newline
36. तेन॑ चिन्वा॒न श्चि॑न्वा॒न स्तेन॒ तेन॑ चिन्वा॒नः । \newline
37. चि॒न्वा॒न स्त॒नुव॑ स्त॒नुव॑ श्चिन्वा॒न श्चि॑न्वा॒न स्त॒नुवः॑ । \newline
38. त॒नुवो॒ नि नि त॒नुव॑ स्त॒नुवो॒ नि । \newline
39. नि षी॑द सीद॒ नि नि षी॑द । \newline
40. सी॒देति॑ सीद । \newline
41. वात॑स्य॒ ध्राजि॒म् ध्राजिं॒ ॅवात॑स्य॒ वात॑स्य॒ ध्राजि᳚म् । \newline
42. ध्राजिं॒ ॅवरु॑णस्य॒ वरु॑णस्य॒ ध्राजि॒म् ध्राजिं॒ ॅवरु॑णस्य । \newline
43. वरु॑णस्य॒ नाभि॒म् नाभिं॒ ॅवरु॑णस्य॒ वरु॑णस्य॒ नाभि᳚म् । \newline
44. नाभि॒ मश्व॒ मश्व॒म् नाभि॒म् नाभि॒ मश्व᳚म् । \newline
45. अश्व॑म् जज्ञा॒नम् ज॑ज्ञा॒न मश्व॒ मश्व॑म् जज्ञा॒नम् । \newline
46. ज॒ज्ञा॒नꣳ स॑रि॒रस्य॑ सरि॒रस्य॑ जज्ञा॒नम् ज॑ज्ञा॒नꣳ स॑रि॒रस्य॑ । \newline
47. स॒रि॒रस्य॒ मद्ध्ये॒ मद्ध्ये॑ सरि॒रस्य॑ सरि॒रस्य॒ मद्ध्ये᳚ । \newline
48. मद्ध्य॒ इति॒ मद्ध्ये᳚ । \newline
49. शिशु॑म् न॒दीना᳚म् न॒दीनाꣳ॒॒ शिशुꣳ॒॒ शिशु॑म् न॒दीना᳚म् । \newline
50. न॒दीनाꣳ॒॒ हरिꣳ॒॒ हरि॑म् न॒दीना᳚म् न॒दीनाꣳ॒॒ हरि᳚म् । \newline
51. हरि॒ मद्रि॑बुद्ध॒ मद्रि॑बुद्धꣳ॒॒ हरिꣳ॒॒ हरि॒ मद्रि॑बुद्धम् । \newline
52. अद्रि॑बुद्ध॒ मग्ने ऽग्ने॒ अद्रि॑बुद्ध॒ मद्रि॑बुद्ध॒ मग्ने᳚ । \newline
53. अद्रि॑बुद्ध॒मित्यद्रि॑ - बु॒द्ध॒म् । \newline
54. अग्ने॒ मा मा ऽग्ने ऽग्ने॒ मा । \newline
55. मा हिꣳ॑सीर्. हिꣳसी॒र् मा मा हिꣳ॑सीः । \newline
56. हिꣳ॒॒सीः॒ प॒र॒मे प॑र॒मे हिꣳ॑सीर्. हिꣳसीः पर॒मे । \newline

\textbf{Ghana Paata } \newline

1. आ॒दि॒त्यम् गर्भ॒म् गर्भ॑ मादि॒त्य मा॑दि॒त्यम् गर्भ॒म् पय॑सा॒ पय॑सा॒ गर्भ॑ मादि॒त्य मा॑दि॒त्यम् गर्भ॒म् पय॑सा । \newline
2. गर्भ॒म् पय॑सा॒ पय॑सा॒ गर्भ॒म् गर्भ॒म् पय॑सा सम॒ञ्जन् थ्स॑म॒ञ्जन् पय॑सा॒ गर्भ॒म् गर्भ॒म् पय॑सा सम॒ञ्जन्न् । \newline
3. पय॑सा सम॒ञ्जन् थ्स॑म॒ञ्जन् पय॑सा॒ पय॑सा सम॒ञ्जन् थ्स॒हस्र॑स्य स॒हस्र॑स्य सम॒ञ्जन् पय॑सा॒ पय॑सा सम॒ञ्जन् थ्स॒हस्र॑स्य । \newline
4. स॒म॒ञ्जन् थ्स॒हस्र॑स्य स॒हस्र॑स्य सम॒ञ्जन् थ्स॑म॒ञ्जन् थ्स॒हस्र॑स्य प्रति॒माम् प्र॑ति॒माꣳ स॒हस्र॑स्य सम॒ञ्जन् थ्स॑म॒ञ्जन् थ्स॒हस्र॑स्य प्रति॒माम् । \newline
5. स॒म॒ञ्जन्निति॑ सं - अ॒ञ्जन्न् । \newline
6. स॒हस्र॑स्य प्रति॒माम् प्र॑ति॒माꣳ स॒हस्र॑स्य स॒हस्र॑स्य प्रति॒मां ॅवि॒श्वरू॑पं ॅवि॒श्वरू॑पम् प्रति॒माꣳ स॒हस्र॑स्य स॒हस्र॑स्य प्रति॒मां ॅवि॒श्वरू॑पम् । \newline
7. प्र॒ति॒मां ॅवि॒श्वरू॑पं ॅवि॒श्वरू॑पम् प्रति॒माम् प्र॑ति॒मां ॅवि॒श्वरू॑पम् । \newline
8. प्र॒ति॒मामिति॑ प्रति - माम् । \newline
9. वि॒श्वरू॑प॒मिति॑ वि॒श्व - रू॒प॒म् । \newline
10. परि॑ वृङ्ग्धि वृङ्ग्धि॒ परि॒ परि॑ वृङ्ग्धि॒ हर॑सा॒ हर॑सा वृङ्ग्धि॒ परि॒ परि॑ वृङ्ग्धि॒ हर॑सा । \newline
11. वृ॒ङ्ग्धि॒ हर॑सा॒ हर॑सा वृङ्ग्धि वृङ्ग्धि॒ हर॑सा॒ मा मा हर॑सा वृङ्ग्धि वृङ्ग्धि॒ हर॑सा॒ मा । \newline
12. हर॑सा॒ मा मा हर॑सा॒ हर॑सा॒ मा ऽभ्य॑भि मा हर॑सा॒ हर॑सा॒ मा ऽभि । \newline
13. मा ऽभ्य॑भि मा मा ऽभि मृ॑क्षो मृक्षो अ॒भि मा मा ऽभि मृ॑क्षः । \newline
14. अ॒भि मृ॑क्षो मृक्षो अ॒भ्य॑भि मृ॑क्षः श॒तायु॑षꣳ श॒तायु॑षम् मृक्षो अ॒भ्य॑भि मृ॑क्षः श॒तायु॑षम् । \newline
15. मृ॒क्षः॒ श॒तायु॑षꣳ श॒तायु॑षम् मृक्षो मृक्षः श॒तायु॑षम् कृणुहि कृणुहि श॒तायु॑षम् मृक्षो मृक्षः श॒तायु॑षम् कृणुहि । \newline
16. श॒तायु॑षम् कृणुहि कृणुहि श॒तायु॑षꣳ श॒तायु॑षम् कृणुहि ची॒यमा॑न श्ची॒यमा॑नः कृणुहि श॒तायु॑षꣳ श॒तायु॑षम् कृणुहि ची॒यमा॑नः । \newline
17. श॒तायु॑ष॒मिति॑ श॒त - आ॒यु॒ष॒म् । \newline
18. कृ॒णु॒हि॒ ची॒यमा॑न श्ची॒यमा॑नः कृणुहि कृणुहि ची॒यमा॑नः । \newline
19. ची॒यमा॑न॒ इति॑ ची॒यमा॑नः । \newline
20. इ॒मम् मा मेम मि॒मम् मा हिꣳ॑सीर्. हिꣳसी॒र् मेम मि॒मम् मा हिꣳ॑सीः । \newline
21. मा हिꣳ॑सीर्. हिꣳसी॒र् मा मा हिꣳ॑सीर् द्वि॒पाद॑म् द्वि॒पादꣳ॑ हिꣳसी॒र् मा मा हिꣳ॑सीर् द्वि॒पाद᳚म् । \newline
22. हिꣳ॒॒सी॒र् द्वि॒पाद॑म् द्वि॒पादꣳ॑ हिꣳसीर्. हिꣳसीर् द्वि॒पाद॑म् पशू॒नाम् प॑शू॒नाम् द्वि॒पादꣳ॑ हिꣳसीर्. हिꣳसीर् द्वि॒पाद॑म् पशू॒नाम् । \newline
23. द्वि॒पाद॑म् पशू॒नाम् प॑शू॒नाम् द्वि॒पाद॑म् द्वि॒पाद॑म् पशू॒नाꣳ सह॑स्राक्ष॒ सह॑स्राक्ष पशू॒नाम् द्वि॒पाद॑म् द्वि॒पाद॑म् पशू॒नाꣳ सह॑स्राक्ष । \newline
24. द्वि॒पाद॒मिति॑ द्वि - पाद᳚म् । \newline
25. प॒शू॒नाꣳ सह॑स्राक्ष॒ सह॑स्राक्ष पशू॒नाम् प॑शू॒नाꣳ सह॑स्राक्ष॒ मेधे॒ मेधे॒ सह॑स्राक्ष पशू॒नाम् प॑शू॒नाꣳ सह॑स्राक्ष॒ मेधे᳚ । \newline
26. सह॑स्राक्ष॒ मेधे॒ मेधे॒ सह॑स्राक्ष॒ सह॑स्राक्ष॒ मेध॒ आ मेधे॒ सह॑स्राक्ष॒ सह॑स्राक्ष॒ मेध॒ आ । \newline
27. सह॑स्रा॒क्षेति॒ सह॑स्र - अ॒क्ष॒ । \newline
28. मेध॒ आ मेधे॒ मेध॒ आ ची॒यमा॑न श्ची॒यमा॑न॒ आ मेधे॒ मेध॒ आ ची॒यमा॑नः । \newline
29. आ ची॒यमा॑न श्ची॒यमा॑न॒ आ ची॒यमा॑नः । \newline
30. ची॒यमा॑न॒ इति॑ ची॒यमा॑नः । \newline
31. म॒यु मा॑र॒ण्य मा॑र॒ण्यम् म॒युम् म॒यु मा॑र॒ण्य मन्वन्वा॑र॒ण्यम् म॒युम् म॒यु मा॑र॒ण्य मनु॑ । \newline
32. आ॒र॒ण्य मन्वन्वा॑र॒ण्य मा॑र॒ण्य मनु॑ ते ते॒ अन्वा॑र॒ण्य मा॑र॒ण्य मनु॑ ते । \newline
33. अनु॑ ते ते॒ अन्वनु॑ ते दिशामि दिशामि ते॒ अन्वनु॑ ते दिशामि । \newline
34. ते॒ दि॒शा॒मि॒ दि॒शा॒मि॒ ते॒ ते॒ दि॒शा॒मि॒ तेन॒ तेन॑ दिशामि ते ते दिशामि॒ तेन॑ । \newline
35. दि॒शा॒मि॒ तेन॒ तेन॑ दिशामि दिशामि॒ तेन॑ चिन्वा॒न श्चि॑न्वा॒न स्तेन॑ दिशामि दिशामि॒ तेन॑ चिन्वा॒नः । \newline
36. तेन॑ चिन्वा॒न श्चि॑न्वा॒न स्तेन॒ तेन॑ चिन्वा॒न स्त॒नुव॑ स्त॒नुव॑ श्चिन्वा॒न स्तेन॒ तेन॑ चिन्वा॒न स्त॒नुवः॑ । \newline
37. चि॒न्वा॒न स्त॒नुव॑ स्त॒नुव॑ श्चिन्वा॒न श्चि॑न्वा॒न स्त॒नुवो॒ नि नि त॒नुव॑ श्चिन्वा॒न श्चि॑न्वा॒न स्त॒नुवो॒ नि । \newline
38. त॒नुवो॒ नि नि त॒नुव॑ स्त॒नुवो॒ नि षी॑द सीद॒ नि त॒नुव॑ स्त॒नुवो॒ नि षी॑द । \newline
39. नि षी॑द सीद॒ नि नि षी॑द । \newline
40. सी॒देति॑ सीद । \newline
41. वात॑स्य॒ ध्राजि॒म् ध्राजिं॒ ॅवात॑स्य॒ वात॑स्य॒ ध्राजिं॒ ॅवरु॑णस्य॒ वरु॑णस्य॒ ध्राजिं॒ ॅवात॑स्य॒ वात॑स्य॒ ध्राजिं॒ ॅवरु॑णस्य । \newline
42. ध्राजिं॒ ॅवरु॑णस्य॒ वरु॑णस्य॒ ध्राजि॒म् ध्राजिं॒ ॅवरु॑णस्य॒ नाभि॒म् नाभिं॒ ॅवरु॑णस्य॒ ध्राजि॒म् ध्राजिं॒ ॅवरु॑णस्य॒ नाभि᳚म् । \newline
43. वरु॑णस्य॒ नाभि॒न्नाभिं॒ ॅवरु॑णस्य॒ वरु॑णस्य॒ नाभि॒ मश्व॒ मश्व॒म् नाभिं॒ ॅवरु॑णस्य॒ वरु॑णस्य॒ नाभि॒ मश्व᳚म् । \newline
44. नाभि॒ मश्व॒ मश्व॒म् नाभि॒म् नाभि॒ मश्व॑म् जज्ञा॒नम् ज॑ज्ञा॒न मश्व॒म् नाभि॒म् नाभि॒ मश्व॑म् जज्ञा॒नम् । \newline
45. अश्व॑म् जज्ञा॒नम् ज॑ज्ञा॒न मश्व॒ मश्व॑म् जज्ञा॒नꣳ स॑रि॒रस्य॑ सरि॒रस्य॑ जज्ञा॒न मश्व॒ मश्व॑म् जज्ञा॒नꣳ स॑रि॒रस्य॑ । \newline
46. ज॒ज्ञा॒नꣳ स॑रि॒रस्य॑ सरि॒रस्य॑ जज्ञा॒नम् ज॑ज्ञा॒नꣳ स॑रि॒रस्य॒ मद्ध्ये॒ मद्ध्ये॑ सरि॒रस्य॑ जज्ञा॒नम् ज॑ज्ञा॒नꣳ स॑रि॒रस्य॒ मद्ध्ये᳚ । \newline
47. स॒रि॒रस्य॒ मद्ध्ये॒ मद्ध्ये॑ सरि॒रस्य॑ सरि॒रस्य॒ मद्ध्ये᳚ । \newline
48. मद्ध्य॒ इति॒ मद्ध्ये᳚ । \newline
49. शिशु॑म् न॒दीना᳚म् न॒दीनाꣳ॒॒ शिशुꣳ॒॒ शिशु॑म् न॒दीनाꣳ॒॒ हरिꣳ॒॒ हरि॑म् न॒दीनाꣳ॒॒ शिशुꣳ॒॒ शिशु॑म् न॒दीनाꣳ॒॒ हरि᳚म् । \newline
50. न॒दीनाꣳ॒॒ हरिꣳ॒॒ हरि॑म् न॒दीना᳚म् न॒दीनाꣳ॒॒ हरि॒ मद्रि॑बुद्ध॒ मद्रि॑बुद्धꣳ॒॒ हरि॑म् न॒दीना᳚म् न॒दीनाꣳ॒॒ हरि॒ मद्रि॑बुद्धम् । \newline
51. हरि॒ मद्रि॑बुद्ध॒ मद्रि॑बुद्धꣳ॒॒ हरिꣳ॒॒ हरि॒ मद्रि॑बुद्ध॒ मग्ने ऽग्ने॒ अद्रि॑बुद्धꣳ॒॒ हरिꣳ॒॒ हरि॒ मद्रि॑बुद्ध॒ मग्ने᳚ । \newline
52. अद्रि॑बुद्ध॒ मग्ने ऽग्ने॒ अद्रि॑बुद्ध॒ मद्रि॑बुद्ध॒ मग्ने॒ मा मा ऽग्ने॒ अद्रि॑बुद्ध॒ मद्रि॑बुद्ध॒ मग्ने॒ मा । \newline
53. अद्रि॑बुद्ध॒मित्यद्रि॑ - बु॒द्ध॒म् । \newline
54. अग्ने॒ मा मा ऽग्ने ऽग्ने॒ मा हिꣳ॑सीर्. हिꣳसी॒र् मा ऽग्ने ऽग्ने॒ मा हिꣳ॑सीः । \newline
55. मा हिꣳ॑सीर्. हिꣳसी॒र् मा मा हिꣳ॑सीः पर॒मे प॑र॒मे हिꣳ॑सी॒र् मा मा हिꣳ॑सीः पर॒मे । \newline
56. हिꣳ॒॒सीः॒ प॒र॒मे प॑र॒मे हिꣳ॑सीर्. हिꣳसीः पर॒मे व्यो॑म॒न् व्यो॑मन् पर॒मे हिꣳ॑सीर्. हिꣳसीः पर॒मे व्यो॑मन्न् । \newline
\pagebreak
\markright{ TS 4.2.10.2  \hfill https://www.vedavms.in \hfill}

\section{ TS 4.2.10.2 }

\textbf{TS 4.2.10.2 } \newline
\textbf{Samhita Paata} \newline

पर॒मे व्यो॑मन्न् ॥ इ॒मं मा हिꣳ॑सी॒रेक॑शफं पशू॒नां क॑निक्र॒दं ॅवा॒जिनं॒ ॅवाजि॑नेषु । गौ॒रमा॑र॒ण्यमनु॑ ते दिशामि॒ तेन॑ चिन्वा॒नस्त॒नुवो॒ नि षी॑द ॥ अज॑स्र॒मिन्दु॑मरु॒षं भु॑र॒ण्युम॒ग्निमी॑डे पू॒र्वचि॑त्तौ॒ नमो॑भिः । स पर्व॑भिर्.ऋतु॒शः कल्प॑मानो॒ गां मा हिꣳ॑सी॒रदि॑तिं ॅवि॒राजं᳚ ॥ इ॒मꣳ स॑मु॒द्रꣳ श॒तधा॑र॒मुथ्-सं॑ ॅव्य॒च्यमा॑नं॒ भुव॑नस्य॒ मद्ध्ये᳚ । घृ॒तं दुहा॑ना॒मदि॑तिं॒ जना॒याग्ने॒ मा - [  ] \newline

\textbf{Pada Paata} \newline

प॒र॒मे । व्यो॑म॒न्निति॒ वि - ओ॒म॒न्न् ॥ इ॒मम् । मा । हिꣳ॒॒सीः॒ । एक॑शफ॒मित्येक॑ - श॒फ॒म् । प॒शू॒नाम् । क॒नि॒क्र॒दम् । वा॒जिन᳚म् । वाजि॑नेषु ॥ गौ॒रम् । आ॒र॒ण्यम् । अन्विति॑ । ते॒ । दि॒शा॒मि॒ । तेन॑ । चि॒न्वा॒नः । त॒नुवः॑ । नीति॑ । सी॒द॒ ॥ अज॑स्रम् । इन्दु᳚म् । अ॒रु॒षम् । भु॒र॒ण्युम् । अ॒ग्निम् । ई॒डे॒ । पू॒र्वचि॑त्ता॒विति॑ पू॒र्व - चि॒त्तौ॒ । नमो॑भि॒रिति॒ नमः॑ - भिः॒ ॥ सः । पर्व॑भि॒रिति॒ पर्व॑ - भिः॒ । ऋ॒तु॒श इत्यृ॑तु - शः । कल्प॑मानः । गाम् । मा । हिꣳ॒॒सीः॒ । अदि॑तिम् । वि॒राज॒मिति॑ वि - राज᳚म् ॥ इ॒मम् । स॒मु॒द्रम् । श॒तधा॑र॒मिति॑ श॒त - धा॒र॒म् । उथ्स᳚म् । व्य॒च्यमा॑न॒मिति॑ वि - अ॒च्यमा॑नम् । भुव॑नस्य । मद्ध्ये᳚ ॥ घृ॒तम् । दुहा॑नाम् । अदि॑तिम् । जना॑य । अग्ने᳚ । मा ।  \newline


\textbf{Krama Paata} \newline

प॒र॒मे व्यो॑मन्न् । व्यो॑म॒न्निति॒ वि - ओ॒म॒न्न्॒ ॥ इ॒मम् मा । मा हिꣳ॑सीः । हिꣳ॒॒सी॒रेक॑शफम् । एक॑शफम् पशू॒नाम् । एक॑शफ॒मित्येक॑ - श॒फ॒म् । प॒शू॒नाम् क॑निक्र॒दम् । क॒नि॒क्र॒दं ॅवा॒जिन᳚म् । वा॒जिनं॒ ॅवाजि॑नेषु । वाजि॑ने॒ष्विति॒ वाजि॑नेषु ॥ गौ॒रमा॑र॒ण्यम् । आ॒र॒ण्यमनु॑ । अनु॑ ते । ते॒ दि॒शा॒मि॒ । दि॒शा॒मि॒ तेन॑ । तेन॑ चिन्वा॒नः । चि॒न्वा॒न स्त॒नुवः॑ । त॒नुवो॒ नि । नि षी॑द । सी॒देति॑ सीद ॥ अज॑स्र॒मिन्दु᳚म् । इन्दु॑मरु॒षम् । अ॒रु॒षम् भु॑र॒ण्युम् । भु॒र॒ण्युम॒ग्निम् । अ॒ग्निमी॑डे । ई॒डे॒ पू॒र्वचि॑त्तौ । पू॒र्वचि॑त्तौ॒ नमो॑भिः । पू॒र्वचि॑त्ता॒विति॑ पू॒र्व - चि॒त्तौ॒ । नमो॑भि॒रिति॒ नमः॑ - भिः॒ ॥ स पर्व॑भिः । पर्व॑भिर्. ऋतु॒शः । पर्व॑भि॒रिति॒ पर्व॑ - भिः॒ । ऋ॒तु॒शः कल्प॑मानः । ऋ॒तु॒श इत्यृ॑तु - शः । कल्प॑मानो॒ गाम् । गाम् मा । मा हिꣳ॑सीः । हिꣳ॒॒सी॒रदि॑तिम् । अदि॑तिं ॅवि॒राज᳚म् । वि॒राज॒मिति॑ वि - राज᳚म् ॥ इ॒मꣳ स॑मु॒द्रम् । स॒मु॒द्रꣳ श॒तधा॑रम् । श॒तधा॑र॒मुथ्स᳚म् । श॒तधा॑र॒मिति॑ श॒त - धा॒र॒म् । उथ्सं॑ ॅव्य॒च्यमा॑नम् । व्य॒च्यमा॑न॒म् भुव॑नस्य । व्य॒च्यमा॑न॒मिति॑ वि - अ॒च्यमा॑नम् । भुव॑नस्य॒ मद्ध्ये᳚ । मद्ध्य॒ इति॒ मद्ध्ये᳚ ॥ घृ॒तम् दुहा॑नाम् । दुहा॑ना॒मदि॑तिम् । अदि॑ति॒म् जना॑य । जना॒याग्ने᳚ । अग्ने॒ मा । मा हिꣳ॑सीः \newline

\textbf{Jatai Paata} \newline

1. प॒र॒मे व्यो॑म॒न् व्यो॑मन् पर॒मे प॑र॒मे व्यो॑मन्न् । \newline
2. व्यो॑म॒न्निति॒ वि - ओ॒म॒न्न् । \newline
3. इ॒मम् मा मेम मि॒मम् मा । \newline
4. मा हिꣳ॑सीर्. हिꣳसी॒र् मा मा हिꣳ॑सीः । \newline
5. हिꣳ॒॒सी॒ रेक॑शफ॒ मेक॑शफꣳ हिꣳसीर्. हिꣳसी॒ रेक॑शफम् । \newline
6. एक॑शफम् पशू॒नाम् प॑शू॒ना मेक॑शफ॒ मेक॑शफम् पशू॒नाम् । \newline
7. एक॑शफ॒मित्येक॑ - श॒फ॒म् । \newline
8. प॒शू॒नाम् क॑निक्र॒दम् क॑निक्र॒दम् प॑शू॒नाम् प॑शू॒नाम् क॑निक्र॒दम् । \newline
9. क॒नि॒क्र॒दं ॅवा॒जिनं॑ ॅवा॒जिन॑म् कनिक्र॒दम् क॑निक्र॒दं ॅवा॒जिन᳚म् । \newline
10. वा॒जिनं॒ ॅवाजि॑नेषु॒ वाजि॑नेषु वा॒जिनं॑ ॅवा॒जिनं॒ ॅवाजि॑नेषु । \newline
11. वाजि॑ने॒ष्विति॒ वाजि॑नेषु । \newline
12. गौ॒र मा॑र॒ण्य मा॑र॒ण्यम् गौ॒रम् गौ॒र मा॑र॒ण्यम् । \newline
13. आ॒र॒ण्य मन् वन् वा॑र॒ण्य मा॑र॒ण्य मनु॑ । \newline
14. अनु॑ ते ते॒ अन्वनु॑ ते । \newline
15. ते॒ दि॒शा॒मि॒ दि॒शा॒मि॒ ते॒ ते॒ दि॒शा॒मि॒ । \newline
16. दि॒शा॒मि॒ तेन॒ तेन॑ दिशामि दिशामि॒ तेन॑ । \newline
17. तेन॑ चिन्वा॒न श्चि॑न्वा॒न स्तेन॒ तेन॑ चिन्वा॒नः । \newline
18. चि॒न्वा॒न स्त॒नुव॑ स्त॒नुव॑ श्चिन्वा॒न श्चि॑न्वा॒न स्त॒नुवः॑ । \newline
19. त॒नुवो॒ नि नि त॒नुव॑ स्त॒नुवो॒ नि । \newline
20. नि षी॑द सीद॒ नि नि षी॑द । \newline
21. सी॒देति॑ सीद । \newline
22. अज॑स्र॒ मिन्दु॒ मिन्दु॒ मज॑स्र॒ मज॑स्र॒ मिन्दु᳚म् । \newline
23. इन्दु॑ मरु॒ष म॑रु॒ष मिन्दु॒ मिन्दु॑ मरु॒षम् । \newline
24. अ॒रु॒षम् भु॑र॒ण्युम् भु॑र॒ण्यु म॑रु॒ष म॑रु॒षम् भु॑र॒ण्युम् । \newline
25. भु॒र॒ण्यु म॒ग्नि म॒ग्निम् भु॑र॒ण्युम् भु॑र॒ण्यु म॒ग्निम् । \newline
26. अ॒ग्नि मी॑ड ईडे अ॒ग्नि म॒ग्नि मी॑डे । \newline
27. ई॒डे॒ पू॒र्वचि॑त्तौ पू॒र्वचि॑त्ता वीड ईडे पू॒र्वचि॑त्तौ । \newline
28. पू॒र्वचि॑त्तौ॒ नमो॑भि॒र् नमो॑भिः पू॒र्वचि॑त्तौ पू॒र्वचि॑त्तौ॒ नमो॑भिः । \newline
29. पू॒र्वचि॑त्ता॒विति॑ पू॒र्व - चि॒त्तौ॒ । \newline
30. नमो॑भि॒रिति॒ नमः॑ - भिः॒ । \newline
31. स पर्व॑भिः॒ पर्व॑भिः॒ स स पर्व॑भिः । \newline
32. पर्व॑भिर्. ऋतु॒श ऋ॑तु॒शः पर्व॑भिः॒ पर्व॑भिर्. ऋतु॒शः । \newline
33. पर्व॑भि॒रिति॒ पर्व॑ - भिः॒ । \newline
34. ऋ॒तु॒शः कल्प॑मानः॒ कल्प॑मान ऋतु॒श ऋ॑तु॒शः कल्प॑मानः । \newline
35. ऋ॒तु॒श इत्यृ॑तु - शः । \newline
36. कल्प॑मानो॒ गाम् गाम् कल्प॑मानः॒ कल्प॑मानो॒ गाम् । \newline
37. गाम् मा मा गाम् गाम् मा । \newline
38. मा हिꣳ॑सीर्. हिꣳसी॒र् मा मा हिꣳ॑सीः । \newline
39. हिꣳ॒॒सी॒ रदि॑ति॒ मदि॑तिꣳ हिꣳसीर्. हिꣳसी॒ रदि॑तिम् । \newline
40. अदि॑तिं ॅवि॒राजं॑ ॅवि॒राज॒ मदि॑ति॒ मदि॑तिं ॅवि॒राज᳚म् । \newline
41. वि॒राज॒मिति॑ वि - राज᳚म् । \newline
42. इ॒मꣳ स॑मु॒द्रꣳ स॑मु॒द्र मि॒म मि॒मꣳ स॑मु॒द्रम् । \newline
43. स॒मु॒द्रꣳ श॒तधा॑रꣳ श॒तधा॑रꣳ समु॒द्रꣳ स॑मु॒द्रꣳ श॒तधा॑रम् । \newline
44. श॒तधा॑र॒ मुथ्स॒ मुथ्सꣳ॑ श॒तधा॑रꣳ श॒तधा॑र॒ मुथ्स᳚म् । \newline
45. श॒तधा॑र॒मिति॑ श॒त - धा॒र॒म् । \newline
46. उथ्सं॑ ॅव्य॒च्यमा॑नं ॅव्य॒च्यमा॑न॒ मुथ्स॒ मुथ्सं॑ ॅव्य॒च्यमा॑नम् । \newline
47. व्य॒च्यमा॑न॒म् भुव॑नस्य॒ भुव॑नस्य व्य॒च्यमा॑नं ॅव्य॒च्यमा॑न॒म् भुव॑नस्य । \newline
48. व्य॒च्यमा॑न॒मिति॑ वि - अ॒च्यमा॑नम् । \newline
49. भुव॑नस्य॒ मद्ध्ये॒ मद्ध्ये॒ भुव॑नस्य॒ भुव॑नस्य॒ मद्ध्ये᳚ । \newline
50. मद्ध्य॒ इति॒ मद्ध्ये᳚ । \newline
51. घृ॒तम् दुहा॑ना॒म् दुहा॑नाम् घृ॒तम् घृ॒तम् दुहा॑नाम् । \newline
52. दुहा॑ना॒ मदि॑ति॒ मदि॑ति॒म् दुहा॑ना॒म् दुहा॑ना॒ मदि॑तिम् । \newline
53. अदि॑ति॒म् जना॑य॒ जना॒या दि॑ति॒ मदि॑ति॒म् जना॑य । \newline
54. जना॒याग्ने ऽग्ने॒ जना॑य॒ जना॒याग्ने᳚ । \newline
55. अग्ने॒ मा मा ऽग्ने ऽग्ने॒ मा । \newline
56. मा हिꣳ॑सीर्. हिꣳसी॒र् मा मा हिꣳ॑सीः । \newline

\textbf{Ghana Paata } \newline

1. प॒र॒मे व्यो॑म॒न् व्यो॑मन् पर॒मे प॑र॒मे व्यो॑मन्न् । \newline
2. व्यो॑म॒न्निति॒ वि - ओ॒म॒न्न् । \newline
3. इ॒मम् मा मेम मि॒मम् मा हिꣳ॑सीर्. हिꣳसी॒र् मेम मि॒मम् मा हिꣳ॑सीः । \newline
4. मा हिꣳ॑सीर्. हिꣳसी॒र् मा मा हिꣳ॑सी॒ रेक॑शफ॒ मेक॑शफꣳ हिꣳसी॒र् मा मा हिꣳ॑सी॒ रेक॑शफम् । \newline
5. हिꣳ॒॒सी॒ रेक॑शफ॒ मेक॑शफꣳ हिꣳसीर्. हिꣳसी॒ रेक॑शफम् पशू॒नाम् प॑शू॒ना मेक॑शफꣳ हिꣳसीर्. हिꣳसी॒ रेक॑शफम् पशू॒नाम् । \newline
6. एक॑शफम् पशू॒नाम् प॑शू॒ना मेक॑शफ॒ मेक॑शफम् पशू॒नाम् क॑निक्र॒दम् क॑निक्र॒दम् प॑शू॒ना मेक॑शफ॒ मेक॑शफम् पशू॒नाम् क॑निक्र॒दम् । \newline
7. एक॑शफ॒मित्येक॑ - श॒फ॒म् । \newline
8. प॒शू॒नाम् क॑निक्र॒दम् क॑निक्र॒दम् प॑शू॒नाम् प॑शू॒नाम् क॑निक्र॒दं ॅवा॒जिनं॑ ॅवा॒जिन॑म् कनिक्र॒दम् प॑शू॒नाम् प॑शू॒नाम् क॑निक्र॒दं ॅवा॒जिन᳚म् । \newline
9. क॒नि॒क्र॒दं ॅवा॒जिनं॑ ॅवा॒जिन॑म् कनिक्र॒दम् क॑निक्र॒दं ॅवा॒जिनं॒ ॅवाजि॑नेषु॒ वाजि॑नेषु वा॒जिन॑म् कनिक्र॒दम् क॑निक्र॒दं ॅवा॒जिनं॒ ॅवाजि॑नेषु । \newline
10. वा॒जिनं॒ ॅवाजि॑नेषु॒ वाजि॑नेषु वा॒जिनं॑ ॅवा॒जिनं॒ ॅवाजि॑नेषु । \newline
11. वाजि॑ने॒ष्विति॒ वाजि॑नेषु । \newline
12. गौ॒र मा॑र॒ण्य मा॑र॒ण्यम् गौ॒रम् गौ॒र मा॑र॒ण्य मन्वन्वा॑र॒ण्यम् गौ॒रम् गौ॒र मा॑र॒ण्य मनु॑ । \newline
13. आ॒र॒ण्य मन्वन्वा॑र॒ण्य मा॑र॒ण्य मनु॑ ते ते॒ अन्वा॑र॒ण्य मा॑र॒ण्य मनु॑ ते । \newline
14. अनु॑ ते ते॒ अन्वनु॑ ते दिशामि दिशामि ते॒ अन्वनु॑ ते दिशामि । \newline
15. ते॒ दि॒शा॒मि॒ दि॒शा॒मि॒ ते॒ ते॒ दि॒शा॒मि॒ तेन॒ तेन॑ दिशामि ते ते दिशामि॒ तेन॑ । \newline
16. दि॒शा॒मि॒ तेन॒ तेन॑ दिशामि दिशामि॒ तेन॑ चिन्वा॒न श्चि॑न्वा॒न स्तेन॑ दिशामि दिशामि॒ तेन॑ चिन्वा॒नः । \newline
17. तेन॑ चिन्वा॒न श्चि॑न्वा॒न स्तेन॒ तेन॑ चिन्वा॒न स्त॒नुव॑ स्त॒नुव॑ श्चिन्वा॒न स्तेन॒ तेन॑ चिन्वा॒न स्त॒नुवः॑ । \newline
18. चि॒न्वा॒न स्त॒नुव॑ स्त॒नुव॑ श्चिन्वा॒न श्चि॑न्वा॒न स्त॒नुवो॒ नि नि त॒नुव॑ श्चिन्वा॒न श्चि॑न्वा॒न स्त॒नुवो॒ नि । \newline
19. त॒नुवो॒ नि नि त॒नुव॑ स्त॒नुवो॒ नि षी॑द सीद॒ नि त॒नुव॑ स्त॒नुवो॒ नि षी॑द । \newline
20. नि षी॑द सीद॒ नि नि षी॑द । \newline
21. सी॒देति॑ सीद । \newline
22. अज॑स्र॒ मिन्दु॒ मिन्दु॒ मज॑स्र॒ मज॑स्र॒ मिन्दु॑ मरु॒ष म॑रु॒ष मिन्दु॒ मज॑स्र॒ मज॑स्र॒ मिन्दु॑ मरु॒षम् । \newline
23. इन्दु॑ मरु॒ष म॑रु॒ष मिन्दु॒ मिन्दु॑ मरु॒षम् भु॑र॒ण्युम् भु॑र॒ण्यु म॑रु॒ष मिन्दु॒ मिन्दु॑ मरु॒षम् भु॑र॒ण्युम् । \newline
24. अ॒रु॒षम् भु॑र॒ण्युम् भु॑र॒ण्यु म॑रु॒ष म॑रु॒षम् भु॑र॒ण्यु म॒ग्नि म॒ग्निम् भु॑र॒ण्यु म॑रु॒ष म॑रु॒षम् भु॑र॒ण्यु म॒ग्निम् । \newline
25. भु॒र॒ण्यु म॒ग्नि म॒ग्निम् भु॑र॒ण्युम् भु॑र॒ण्यु म॒ग्नि मी॑ड ईडे अ॒ग्निम् भु॑र॒ण्युम् भु॑र॒ण्यु म॒ग्नि मी॑डे । \newline
26. अ॒ग्नि मी॑ड ईडे अ॒ग्नि म॒ग्नि मी॑डे पू॒र्वचि॑त्तौ पू॒र्वचि॑त्ता वीडे अ॒ग्नि म॒ग्नि मी॑डे पू॒र्वचि॑त्तौ । \newline
27. ई॒डे॒ पू॒र्वचि॑त्तौ पू॒र्वचि॑त्ता वीड ईडे पू॒र्वचि॑त्तौ॒ नमो॑भि॒र् नमो॑भिः पू॒र्वचि॑त्ता वीड ईडे पू॒र्वचि॑त्तौ॒ नमो॑भिः । \newline
28. पू॒र्वचि॑त्तौ॒ नमो॑भि॒र् नमो॑भिः पू॒र्वचि॑त्तौ पू॒र्वचि॑त्तौ॒ नमो॑भिः । \newline
29. पू॒र्वचि॑त्ता॒विति॑ पू॒र्व - चि॒त्तौ॒ । \newline
30. नमो॑भि॒रिति॒ नमः॑ - भिः॒ । \newline
31. स पर्व॑भिः॒ पर्व॑भिः॒ स स पर्व॑भिर्. ऋतु॒श ऋ॑तु॒शः पर्व॑भिः॒ स स पर्व॑भिर्. ऋतु॒शः । \newline
32. पर्व॑भिर्. ऋतु॒श ऋ॑तु॒शः पर्व॑भिः॒ पर्व॑भिर्. ऋतु॒शः कल्प॑मानः॒ कल्प॑मान ऋतु॒शः पर्व॑भिः॒ पर्व॑भिर्. ऋतु॒शः कल्प॑मानः । \newline
33. पर्व॑भि॒रिति॒ पर्व॑ - भिः॒ । \newline
34. ऋ॒तु॒शः कल्प॑मानः॒ कल्प॑मान ऋतु॒श ऋ॑तु॒शः कल्प॑मानो॒ गाम् गाम् कल्प॑मान ऋतु॒श ऋ॑तु॒शः कल्प॑मानो॒ गाम् । \newline
35. ऋ॒तु॒श इत्यृ॑तु - शः । \newline
36. कल्प॑मानो॒ गाम् गाम् कल्प॑मानः॒ कल्प॑मानो॒ गाम् मा मा गाम् कल्प॑मानः॒ कल्प॑मानो॒ गाम् मा । \newline
37. गाम् मा मा गाम् गाम् मा हिꣳ॑सीर्. हिꣳसी॒र् मा गाम् गाम् मा हिꣳ॑सीः । \newline
38. मा हिꣳ॑सीर्. हिꣳसी॒र् मा मा हिꣳ॑सी॒ रदि॑ति॒ मदि॑तिꣳ हिꣳसी॒र् मा मा हिꣳ॑सी॒ रदि॑तिम् । \newline
39. हिꣳ॒॒सी॒ रदि॑ति॒ मदि॑तिꣳ हिꣳसीर्. हिꣳसी॒ रदि॑तिं ॅवि॒राजं॑ ॅवि॒राज॒ मदि॑तिꣳ हिꣳसीर्. हिꣳसी॒ रदि॑तिं ॅवि॒राज᳚म् । \newline
40. अदि॑तिं ॅवि॒राजं॑ ॅवि॒राज॒ मदि॑ति॒ मदि॑तिं ॅवि॒राज᳚म् । \newline
41. वि॒राज॒मिति॑ वि - राज᳚म् । \newline
42. इ॒मꣳ स॑मु॒द्रꣳ स॑मु॒द्र मि॒म मि॒मꣳ स॑मु॒द्रꣳ श॒तधा॑रꣳ श॒तधा॑रꣳ समु॒द्र मि॒म मि॒मꣳ स॑मु॒द्रꣳ श॒तधा॑रम् । \newline
43. स॒मु॒द्रꣳ श॒तधा॑रꣳ श॒तधा॑रꣳ समु॒द्रꣳ स॑मु॒द्रꣳ श॒तधा॑र॒ मुथ्स॒ मुथ्सꣳ॑ श॒तधा॑रꣳ समु॒द्रꣳ स॑मु॒द्रꣳ श॒तधा॑र॒ मुथ्स᳚म् । \newline
44. श॒तधा॑र॒ मुथ्स॒ मुथ्सꣳ॑ श॒तधा॑रꣳ श॒तधा॑र॒ मुथ्सं॑ ॅव्य॒च्यमा॑नं ॅव्य॒च्यमा॑न॒ मुथ्सꣳ॑ श॒तधा॑रꣳ श॒तधा॑र॒ मुथ्सं॑ ॅव्य॒च्यमा॑नम् । \newline
45. श॒तधा॑र॒मिति॑ श॒त - धा॒र॒म् । \newline
46. उथ्सं॑ ॅव्य॒च्यमा॑नं ॅव्य॒च्यमा॑न॒ मुथ्स॒ मुथ्सं॑ ॅव्य॒च्यमा॑न॒म् भुव॑नस्य॒ भुव॑नस्य व्य॒च्यमा॑न॒ मुथ्स॒ मुथ्सं॑ ॅव्य॒च्यमा॑न॒म् भुव॑नस्य । \newline
47. व्य॒च्यमा॑न॒म् भुव॑नस्य॒ भुव॑नस्य व्य॒च्यमा॑नं ॅव्य॒च्यमा॑न॒म् भुव॑नस्य॒ मद्ध्ये॒ मद्ध्ये॒ भुव॑नस्य व्य॒च्यमा॑नं ॅव्य॒च्यमा॑न॒म् भुव॑नस्य॒ मद्ध्ये᳚ । \newline
48. व्य॒च्यमा॑न॒मिति॑ वि - अ॒च्यमा॑नम् । \newline
49. भुव॑नस्य॒ मद्ध्ये॒ मद्ध्ये॒ भुव॑नस्य॒ भुव॑नस्य॒ मद्ध्ये᳚ । \newline
50. मद्ध्य॒ इति॒ मद्ध्ये᳚ । \newline
51. घृ॒तम् दुहा॑ना॒म् दुहा॑नाम् घृ॒तम् घृ॒तम् दुहा॑ना॒ मदि॑ति॒ मदि॑ति॒म् दुहा॑नाम् घृ॒तम् घृ॒तम् दुहा॑ना॒ मदि॑तिम् । \newline
52. दुहा॑ना॒ मदि॑ति॒ मदि॑ति॒म् दुहा॑ना॒म् दुहा॑ना॒ मदि॑ति॒म् जना॑य॒ जना॒या दि॑ति॒म् दुहा॑ना॒म् दुहा॑ना॒ मदि॑ति॒म् जना॑य । \newline
53. अदि॑ति॒म् जना॑य॒ जना॒या दि॑ति॒ मदि॑ति॒म् जना॒याग्ने ऽग्ने॒ जना॒या दि॑ति॒ मदि॑ति॒म् जना॒याग्ने᳚ । \newline
54. जना॒याग्ने ऽग्ने॒ जना॑य॒ जना॒याग्ने॒ मा मा ऽग्ने॒ जना॑य॒ जना॒याग्ने॒ मा । \newline
55. अग्ने॒ मा मा ऽग्ने ऽग्ने॒ मा हिꣳ॑सीर्. हिꣳसी॒र् मा ऽग्ने ऽग्ने॒ मा हिꣳ॑सीः । \newline
56. मा हिꣳ॑सीर्. हिꣳसी॒र् मा मा हिꣳ॑सीः पर॒मे प॑र॒मे हिꣳ॑सी॒र् मा मा हिꣳ॑सीः पर॒मे । \newline
\pagebreak
\markright{ TS 4.2.10.3  \hfill https://www.vedavms.in \hfill}

\section{ TS 4.2.10.3 }

\textbf{TS 4.2.10.3 } \newline
\textbf{Samhita Paata} \newline

हिꣳ॑सीः पर॒मे व्यो॑मन्न् । ग॒व॒यमा॑र॒ण्यमनु॑ ते दिशामि॒ तेन॑ चिन्वा॒नस्त॒नुवो॒ नि षी॑द ॥ वरू᳚त्रिं॒ त्वष्टु॒र्वरु॑णस्य॒ नाभि॒मविं॑ जज्ञा॒नाꣳ रज॑सः॒ पर॑स्मात् । म॒हीꣳ सा॑ह॒स्रीमसु॑रस्य मा॒यामग्ने॒ मा हिꣳ॑सीः पर॒मे व्यो॑मन्न् ॥ इ॒मामू᳚र्णा॒युं ॅवरु॑णस्य मा॒यां त्वचं॑ पशू॒नां द्वि॒पदां॒ चतु॑ष्पदां । त्वष्टुः॑ प्र॒जानां᳚ प्रथ॒मं ज॒नित्र॒मग्ने॒ मा हिꣳ॑सीः पर॒मे व्यो॑मन्न् । उष्ट्र॑मार॒ण्यमनु॑ - [  ] \newline

\textbf{Pada Paata} \newline

हिꣳ॒॒सीः॒ । प॒र॒मे । व्यो॑म॒न्निति॒ वि - ओ॒म॒न्न् ॥ ग॒व॒यम् । आ॒र॒ण्यम् । अन्विति॑ । ते॒ । दि॒शा॒मि॒ । तेन॑ । चि॒न्वा॒नः । त॒नुवः॑ । नीति॑ । सी॒द॒ ॥ वरू᳚त्रिम् । त्वष्टुः॑ । वरु॑णस्य । नाभि᳚म् । अवि᳚म् । ज॒ज्ञा॒नाम् । रज॑सः । पर॑स्मात् ॥ म॒हीम् । सा॒ह॒स्रीम् । असु॑रस्य । मा॒याम् । अग्ने᳚ । मा । हिꣳ॒॒सीः॒ । प॒र॒मे । व्यो॑म॒न्निति॒ वि - ओ॒म॒न्न् ॥ इ॒माम् । ऊ॒र्णा॒युम् । वरु॑णस्य । मा॒याम् । त्वच᳚म् । प॒शू॒नाम् । द्वि॒पदा॒मिति॑ द्वि - पदा᳚म् । चतु॑ष्पदा॒मिति॒ चतुः॑ - प॒दा॒म् ॥ त्वष्टुः॑ । प्र॒जाना॒मिति॑ प्र - जाना᳚म् । प्र॒थ॒मम् । ज॒नित्र᳚म् । अग्ने᳚ । मा । हिꣳ॒॒सीः॒ । प॒र॒मे । व्यो॑म॒न्निति॒ वि - ओ॒म॒न्न् ॥ उष्ट्र᳚म् । आ॒र॒ण्यम् । अन्विति॑ ।  \newline


\textbf{Krama Paata} \newline

हिꣳ॒॒सीः॒ प॒र॒मे । प॒र॒मे व्यो॑मन्न् । व्यो॑म॒न्निति॒ वि - ओ॒म॒न्न्॒ ॥ ग॒व॒यमा॑र॒ण्यम् । आ॒र॒ण्यमनु॑ । अनु॑ ते । ते॒ दि॒शा॒मि॒ । दि॒शा॒मि॒ तेन॑ । तेन॑ चिन्वा॒नः । चि॒न्वा॒नस्त॒नुवः॑ । त॒नुवो॒ नि । नि षी॑द । सी॒देति॑ सीद ॥ वरू᳚त्रि॒म् त्वष्टुः॑ । त्वष्टु॒र् वरु॑णस्य । वरु॑णस्य॒ नाभि᳚म् । नाभि॒मवि᳚म् । अवि॑म् जज्ञा॒नाम् । ज॒ज्ञा॒नाꣳ रज॑सः । रज॑सः॒ पर॑स्मात् । पर॑स्मा॒दिति॒ पर॑स्मात् ॥ म॒हीꣳ सा॑ह॒स्रीम् । सा॒ह॒स्रीमसु॑रस्य । असु॑रस्य मा॒याम् । मा॒यामग्ने᳚ । अग्ने॒ मा । मा हिꣳ॑सीः । हिꣳ॒॒सीः॒ प॒र॒मे । प॒र॒मे व्यो॑मन्न् । व्यो॑म॒न्निति॒ वि - ओ॒म॒न्न्॒ ॥ इ॒मामू᳚र्णा॒युम् । ऊ॒र्णा॒युं ॅवरु॑णस्य । वरु॑णस्य मा॒याम् । मा॒याम् त्वच᳚म् । त्वच॑म् पशू॒नाम् । प॒शू॒नाम् द्वि॒पदा᳚म् । द्वि॒पदा॒म् चतु॑ष्पदाम् । द्वि॒पदा॒मिति॑ द्वि - पदा᳚म् । चतु॑ष्पदा॒मिति॒ चतुः॑ - प॒दा॒म् ॥ त्वष्टुः॑ प्र॒जाना᳚म् । प्र॒जाना᳚म् प्रथ॒मम् । प्र॒जाना॒मिति॑ प्र - जाना᳚म् । प्र॒थ॒मम् ज॒नित्र᳚म् । ज॒नित्र॒मग्ने᳚ । अग्ने॒ मा । मा हिꣳ॑सीः । हिꣳ॒॒सीः॒ प॒र॒मे । प॒र॒मे व्यो॑मन्न् । व्यो॑म॒न्निति॒ वि - ओ॒म॒न्न्॒ ॥ उष्ट्र॑मार॒ण्यम् । आ॒र॒ण्यमनु॑ । अनु॑ ते \newline

\textbf{Jatai Paata} \newline

1. हिꣳ॒॒सीः॒ प॒र॒मे प॑र॒मे हिꣳ॑सीर्. हिꣳसीः पर॒मे । \newline
2. प॒र॒मे व्यो॑म॒न् व्यो॑मन् पर॒मे प॑र॒मे व्यो॑मन्न् । \newline
3. व्यो॑म॒न्निति॒ वि - ओ॒म॒न्न् । \newline
4. ग॒व॒य मा॑र॒ण्य मा॑र॒ण्यम् ग॑व॒यम् ग॑व॒य मा॑र॒ण्यम् । \newline
5. आ॒र॒ण्य मन् वन् वा॑र॒ण्य मा॑र॒ण्य मनु॑ । \newline
6. अनु॑ ते ते॒ अन्वनु॑ ते । \newline
7. ते॒ दि॒शा॒मि॒ दि॒शा॒मि॒ ते॒ ते॒ दि॒शा॒मि॒ । \newline
8. दि॒शा॒मि॒ तेन॒ तेन॑ दिशामि दिशामि॒ तेन॑ । \newline
9. तेन॑ चिन्वा॒न श्चि॑न्वा॒न स्तेन॒ तेन॑ चिन्वा॒नः । \newline
10. चि॒न्वा॒न स्त॒नुव॑ स्त॒नुव॑ श्चिन्वा॒न श्चि॑न्वा॒न स्त॒नुवः॑ । \newline
11. त॒नुवो॒ नि नि त॒नुव॑ स्त॒नुवो॒ नि । \newline
12. नि षी॑द सीद॒ नि नि षी॑द । \newline
13. सी॒देति॑ सीद । \newline
14. वरू᳚त्रि॒म् त्वष्टु॒ स्त्वष्टु॒र् वरू᳚त्रिं॒ ॅवरू᳚त्रि॒म् त्वष्टुः॑ । \newline
15. त्वष्टु॒र् वरु॑णस्य॒ वरु॑णस्य॒ त्वष्टु॒ स्त्वष्टु॒र् वरु॑णस्य । \newline
16. वरु॑णस्य॒ नाभि॒म् नाभिं॒ ॅवरु॑णस्य॒ वरु॑णस्य॒ नाभि᳚म् । \newline
17. नाभि॒ मवि॒ मवि॒म् नाभि॒म् नाभि॒ मवि᳚म् । \newline
18. अवि॑म् जज्ञा॒नाम् ज॑ज्ञा॒ना मवि॒ मवि॑म् जज्ञा॒नाम् । \newline
19. ज॒ज्ञा॒नाꣳ रज॑सो॒ रज॑सो जज्ञा॒नाम् ज॑ज्ञा॒नाꣳ रज॑सः । \newline
20. रज॑सः॒ पर॑स्मा॒त् पर॑स्मा॒द् रज॑सो॒ रज॑सः॒ पर॑स्मात् । \newline
21. पर॑स्मा॒दिति॒ पर॑स्मात् । \newline
22. म॒हीꣳ सा॑ह॒स्रीꣳ सा॑ह॒स्रीम् म॒हीम् म॒हीꣳ सा॑ह॒स्रीम् । \newline
23. सा॒ह॒स्री मसु॑र॒स्या सु॑रस्य साह॒स्रीꣳ सा॑ह॒स्री मसु॑रस्य । \newline
24. असु॑रस्य मा॒याम् मा॒या मसु॑र॒स्या सु॑रस्य मा॒याम् । \newline
25. मा॒या मग्ने ऽग्ने॑ मा॒याम् मा॒या मग्ने᳚ । \newline
26. अग्ने॒ मा मा ऽग्ने ऽग्ने॒ मा । \newline
27. मा हिꣳ॑सीर्. हिꣳसी॒र् मा मा हिꣳ॑सीः । \newline
28. हिꣳ॒॒सीः॒ प॒र॒मे प॑र॒मे हिꣳ॑सीर्. हिꣳसीः पर॒मे । \newline
29. प॒र॒मे व्यो॑म॒न् व्यो॑मन् पर॒मे प॑र॒मे व्यो॑मन्न् । \newline
30. व्यो॑म॒न्निति॒ वि - ओ॒म॒न्न् । \newline
31. इ॒मा मू᳚र्णा॒यु मू᳚र्णा॒यु मि॒मा मि॒मा मू᳚र्णा॒युम् । \newline
32. ऊ॒र्णा॒युं ॅवरु॑णस्य॒ वरु॑ण स्योर्णा॒यु मू᳚र्णा॒युं ॅवरु॑णस्य । \newline
33. वरु॑णस्य मा॒याम् मा॒यां ॅवरु॑णस्य॒ वरु॑णस्य मा॒याम् । \newline
34. मा॒याम् त्वच॒म् त्वच॑म् मा॒याम् मा॒याम् त्वच᳚म् । \newline
35. त्वच॑म् पशू॒नाम् प॑शू॒नाम् त्वच॒म् त्वच॑म् पशू॒नाम् । \newline
36. प॒शू॒नाम् द्वि॒पदा᳚म् द्वि॒पदा᳚म् पशू॒नाम् प॑शू॒नाम् द्वि॒पदा᳚म् । \newline
37. द्वि॒पदा॒म् चतु॑ष्पदा॒म् चतु॑ष्पदाम् द्वि॒पदा᳚म् द्वि॒पदा॒म् चतु॑ष्पदाम् । \newline
38. द्वि॒पदा॒मिति॑ द्वि - पदा᳚म् । \newline
39. चतु॑ष्पदा॒मिति॒ चतुः॑ - प॒दा॒म् । \newline
40. त्वष्टुः॑ प्र॒जाना᳚म् प्र॒जाना॒म् त्वष्टु॒ स्त्वष्टुः॑ प्र॒जाना᳚म् । \newline
41. प्र॒जाना᳚म् प्रथ॒मम् प्र॑थ॒मम् प्र॒जाना᳚म् प्र॒जाना᳚म् प्रथ॒मम् । \newline
42. प्र॒जाना॒मिति॑ प्र - जाना᳚म् । \newline
43. प्र॒थ॒मम् ज॒नित्र॑म् ज॒नित्र॑म् प्रथ॒मम् प्र॑थ॒मम् ज॒नित्र᳚म् । \newline
44. ज॒नित्र॒ मग्ने ऽग्ने॑ ज॒नित्र॑म् ज॒नित्र॒ मग्ने᳚ । \newline
45. अग्ने॒ मा मा ऽग्ने ऽग्ने॒ मा । \newline
46. मा हिꣳ॑सीर्. हिꣳसी॒र् मा मा हिꣳ॑सीः । \newline
47. हिꣳ॒॒सीः॒ प॒र॒मे प॑र॒मे हिꣳ॑सीर्. हिꣳसीः पर॒मे । \newline
48. प॒र॒मे व्यो॑म॒न् व्यो॑मन् पर॒मे प॑र॒मे व्यो॑मन्न् । \newline
49. व्यो॑म॒न्निति॒ वि - ओ॒म॒न्न् । \newline
50. उष्ट्र॑ मार॒ण्य मा॑र॒ण्य मुष्ट्र॒ मुष्ट्र॑ मार॒ण्यम् । \newline
51. आ॒र॒ण्य मन् वन् वा॑र॒ण्य मा॑र॒ण्य मनु॑ । \newline
52. अनु॑ ते ते॒ अन्वनु॑ ते । \newline

\textbf{Ghana Paata } \newline

1. हिꣳ॒॒सीः॒ प॒र॒मे प॑र॒मे हिꣳ॑सीर्. हिꣳसीः पर॒मे व्यो॑म॒न् व्यो॑मन् पर॒मे हिꣳ॑सीर्. हिꣳसीः पर॒मे व्यो॑मन्न् । \newline
2. प॒र॒मे व्यो॑म॒न् व्यो॑मन् पर॒मे प॑र॒मे व्यो॑मन्न् । \newline
3. व्यो॑म॒न्निति॒ वि - ओ॒म॒न्न् । \newline
4. ग॒व॒य मा॑र॒ण्य मा॑र॒ण्यम् ग॑व॒यम् ग॑व॒य मा॑र॒ण्य मन्वन्वा॑र॒ण्यम् ग॑व॒यम् ग॑व॒य मा॑र॒ण्य मनु॑ । \newline
5. आ॒र॒ण्य मन्वन्वा॑र॒ण्य मा॑र॒ण्य मनु॑ ते ते॒ अन्वा॑र॒ण्य मा॑र॒ण्य मनु॑ ते । \newline
6. अनु॑ ते ते॒ अन्वनु॑ ते दिशामि दिशामि ते॒ अन्वनु॑ ते दिशामि । \newline
7. ते॒ दि॒शा॒मि॒ दि॒शा॒मि॒ ते॒ ते॒ दि॒शा॒मि॒ तेन॒ तेन॑ दिशामि ते ते दिशामि॒ तेन॑ । \newline
8. दि॒शा॒मि॒ तेन॒ तेन॑ दिशामि दिशामि॒ तेन॑ चिन्वा॒न श्चि॑न्वा॒न स्तेन॑ दिशामि दिशामि॒ तेन॑ चिन्वा॒नः । \newline
9. तेन॑ चिन्वा॒न श्चि॑न्वा॒न स्तेन॒ तेन॑ चिन्वा॒न स्त॒नुव॑ स्त॒नुव॑ श्चिन्वा॒न स्तेन॒ तेन॑ चिन्वा॒न स्त॒नुवः॑ । \newline
10. चि॒न्वा॒न स्त॒नुव॑ स्त॒नुव॑ श्चिन्वा॒न श्चि॑न्वा॒न स्त॒नुवो॒ नि नि त॒नुव॑ श्चिन्वा॒न श्चि॑न्वा॒न स्त॒नुवो॒ नि । \newline
11. त॒नुवो॒ नि नि त॒नुव॑ स्त॒नुवो॒ नि षी॑द सीद॒ नि त॒नुव॑ स्त॒नुवो॒ नि षी॑द । \newline
12. नि षी॑द सीद॒ नि नि षी॑द । \newline
13. सी॒देति॑ सीद । \newline
14. वरू᳚त्रि॒म् त्वष्टु॒ स्त्वष्टु॒र् वरू᳚त्रिं॒ ॅवरू᳚त्रि॒म् त्वष्टु॒र् वरु॑णस्य॒ वरु॑णस्य॒ त्वष्टु॒र् वरू᳚त्रिं॒ ॅवरू᳚त्रि॒म् त्वष्टु॒र् वरु॑णस्य । \newline
15. त्वष्टु॒र् वरु॑णस्य॒ वरु॑णस्य॒ त्वष्टु॒ स्त्वष्टु॒र् वरु॑णस्य॒ नाभि॒म् नाभिं॒ ॅवरु॑णस्य॒ त्वष्टु॒ स्त्वष्टु॒र् वरु॑णस्य॒ नाभि᳚म् । \newline
16. वरु॑णस्य॒ नाभि॒म् नाभिं॒ ॅवरु॑णस्य॒ वरु॑णस्य॒ नाभि॒ मवि॒ मवि॒म् नाभिं॒ ॅवरु॑णस्य॒ वरु॑णस्य॒ नाभि॒ मवि᳚म् । \newline
17. नाभि॒ मवि॒ मवि॒म् नाभि॒म् नाभि॒ मवि॑म् जज्ञा॒नाम् ज॑ज्ञा॒ना मवि॒म् नाभि॒म् नाभि॒ मवि॑म् जज्ञा॒नाम् । \newline
18. अवि॑म् जज्ञा॒नाम् ज॑ज्ञा॒ना मवि॒ मवि॑म् जज्ञा॒नाꣳ रज॑सो॒ रज॑सो जज्ञा॒ना मवि॒ मवि॑म् जज्ञा॒नाꣳ रज॑सः । \newline
19. ज॒ज्ञा॒नाꣳ रज॑सो॒ रज॑सो जज्ञा॒नाम् ज॑ज्ञा॒नाꣳ रज॑सः॒ पर॑स्मा॒त् पर॑स्मा॒द् रज॑सो जज्ञा॒नाम् ज॑ज्ञा॒नाꣳ रज॑सः॒ पर॑स्मात् । \newline
20. रज॑सः॒ पर॑स्मा॒त् पर॑स्मा॒द् रज॑सो॒ रज॑सः॒ पर॑स्मात् । \newline
21. पर॑स्मा॒दिति॒ पर॑स्मात् । \newline
22. म॒हीꣳ सा॑ह॒स्रीꣳ सा॑ह॒स्रीम् म॒हीम् म॒हीꣳ सा॑ह॒स्री मसु॑र॒स्या सु॑रस्य साह॒स्रीम् म॒हीम् म॒हीꣳ सा॑ह॒स्री मसु॑रस्य । \newline
23. सा॒ह॒स्री मसु॑र॒स्या सु॑रस्य साह॒स्रीꣳ सा॑ह॒स्री मसु॑रस्य मा॒याम् मा॒या मसु॑रस्य साह॒स्रीꣳ सा॑ह॒स्री मसु॑रस्य मा॒याम् । \newline
24. असु॑रस्य मा॒याम् मा॒या मसु॑र॒स्या सु॑रस्य मा॒या मग्ने ऽग्ने॑ मा॒या मसु॑र॒स्या सु॑रस्य मा॒या मग्ने᳚ । \newline
25. मा॒या मग्ने ऽग्ने॑ मा॒याम् मा॒या मग्ने॒ मा मा ऽग्ने॑ मा॒याम् मा॒या मग्ने॒ मा । \newline
26. अग्ने॒ मा मा ऽग्ने ऽग्ने॒ मा हिꣳ॑सीर्. हिꣳसी॒र् मा ऽग्ने ऽग्ने॒ मा हिꣳ॑सीः । \newline
27. मा हिꣳ॑सीर्. हिꣳसी॒र् मा मा हिꣳ॑सीः पर॒मे प॑र॒मे हिꣳ॑सी॒र् मा मा हिꣳ॑सीः पर॒मे । \newline
28. हिꣳ॒॒सीः॒ प॒र॒मे प॑र॒मे हिꣳ॑सीर्. हिꣳसीः पर॒मे व्यो॑म॒न् व्यो॑मन् पर॒मे हिꣳ॑सीर्. हिꣳसीः पर॒मे व्यो॑मन्न् । \newline
29. प॒र॒मे व्यो॑म॒न् व्यो॑मन् पर॒मे प॑र॒मे व्यो॑मन्न् । \newline
30. व्यो॑म॒न्निति॒ वि - ओ॒म॒न्न् । \newline
31. इ॒मा मू᳚र्णा॒यु मू᳚र्णा॒यु मि॒मा मि॒मा मू᳚र्णा॒युं ॅवरु॑णस्य॒ वरु॑ण स्योर्णा॒यु मि॒मा मि॒मा मू᳚र्णा॒युं ॅवरु॑णस्य । \newline
32. ऊ॒र्णा॒युं ॅवरु॑णस्य॒ वरु॑ण स्योर्णा॒यु मू᳚र्णा॒युं ॅवरु॑णस्य मा॒याम् मा॒यां ॅवरु॑ण स्योर्णा॒यु मू᳚र्णा॒युं ॅवरु॑णस्य मा॒याम् । \newline
33. वरु॑णस्य मा॒याम् मा॒यां ॅवरु॑णस्य॒ वरु॑णस्य मा॒याम् त्वच॒म् त्वच॑म् मा॒यां ॅवरु॑णस्य॒ वरु॑णस्य मा॒याम् त्वच᳚म् । \newline
34. मा॒याम् त्वच॒म् त्वच॑म् मा॒याम् मा॒याम् त्वच॑म् पशू॒नाम् प॑शू॒नाम् त्वच॑म् मा॒याम् मा॒याम् त्वच॑म् पशू॒नाम् । \newline
35. त्वच॑म् पशू॒नाम् प॑शू॒नाम् त्वच॒म् त्वच॑म् पशू॒नाम् द्वि॒पदा᳚म् द्वि॒पदा᳚म् पशू॒नाम् त्वच॒म् त्वच॑म् पशू॒नाम् द्वि॒पदा᳚म् । \newline
36. प॒शू॒नाम् द्वि॒पदा᳚म् द्वि॒पदा᳚म् पशू॒नाम् प॑शू॒नाम् द्वि॒पदा॒म् चतु॑ष्पदा॒म् चतु॑ष्पदाम् द्वि॒पदा᳚म् पशू॒नाम् प॑शू॒नाम् द्वि॒पदा॒म् चतु॑ष्पदाम् । \newline
37. द्वि॒पदा॒म् चतु॑ष्पदा॒म् चतु॑ष्पदाम् द्वि॒पदा᳚म् द्वि॒पदा॒म् चतु॑ष्पदाम् । \newline
38. द्वि॒पदा॒मिति॑ द्वि - पदा᳚म् । \newline
39. चतु॑ष्पदा॒मिति॒ चतुः॑ - प॒दा॒म् । \newline
40. त्वष्टुः॑ प्र॒जाना᳚म् प्र॒जाना॒म् त्वष्टु॒ स्त्वष्टुः॑ प्र॒जाना᳚म् प्रथ॒मम् प्र॑थ॒मम् प्र॒जाना॒म् त्वष्टु॒ स्त्वष्टुः॑ प्र॒जाना᳚म् प्रथ॒मम् । \newline
41. प्र॒जाना᳚म् प्रथ॒मम् प्र॑थ॒मम् प्र॒जाना᳚म् प्र॒जाना᳚म् प्रथ॒मम् ज॒नित्र॑म् ज॒नित्र॑म् प्रथ॒मम् प्र॒जाना᳚म् प्र॒जाना᳚म् प्रथ॒मम् ज॒नित्र᳚म् । \newline
42. प्र॒जाना॒मिति॑ प्र - जाना᳚म् । \newline
43. प्र॒थ॒मम् ज॒नित्र॑म् ज॒नित्र॑म् प्रथ॒मम् प्र॑थ॒मम् ज॒नित्र॒ मग्ने ऽग्ने॑ ज॒नित्र॑म् प्रथ॒मम् प्र॑थ॒मम् ज॒नित्र॒ मग्ने᳚ । \newline
44. ज॒नित्र॒ मग्ने ऽग्ने॑ ज॒नित्र॑म् ज॒नित्र॒ मग्ने॒ मा मा ऽग्ने॑ ज॒नित्र॑म् ज॒नित्र॒ मग्ने॒ मा । \newline
45. अग्ने॒ मा मा ऽग्ने ऽग्ने॒ मा हिꣳ॑सीर्. हिꣳसी॒र् मा ऽग्ने ऽग्ने॒ मा हिꣳ॑सीः । \newline
46. मा हिꣳ॑सीर्. हिꣳसी॒र् मा मा हिꣳ॑सीः पर॒मे प॑र॒मे हिꣳ॑सी॒र् मा मा हिꣳ॑सीः पर॒मे । \newline
47. हिꣳ॒॒सीः॒ प॒र॒मे प॑र॒मे हिꣳ॑सीर्. हिꣳसीः पर॒मे व्यो॑म॒न् व्यो॑मन् पर॒मे हिꣳ॑सीर्. हिꣳसीः पर॒मे व्यो॑मन्न् । \newline
48. प॒र॒मे व्यो॑म॒न् व्यो॑मन् पर॒मे प॑र॒मे व्यो॑मन्न् । \newline
49. व्यो॑म॒न्निति॒ वि - ओ॒म॒न्न् । \newline
50. उष्ट्र॑ मार॒ण्य मा॑र॒ण्य मुष्ट्र॒ मुष्ट्र॑ मार॒ण्य मन्वन् वा॑र॒ण्य मुष्ट्र॒ मुष्ट्र॑ मार॒ण्य मनु॑ । \newline
51. आ॒र॒ण्य मन्वन् वा॑र॒ण्य मा॑र॒ण्य मनु॑ ते ते॒ अन्वा॑र॒ण्य मा॑र॒ण्य मनु॑ ते । \newline
52. अनु॑ ते ते॒ अन्वनु॑ ते दिशामि दिशामि ते॒ अन्वनु॑ ते दिशामि । \newline
\pagebreak
\markright{ TS 4.2.10.4  \hfill https://www.vedavms.in \hfill}

\section{ TS 4.2.10.4 }

\textbf{TS 4.2.10.4 } \newline
\textbf{Samhita Paata} \newline

ते दिशामि॒ तेन॑ चिन्वा॒नस्त॒नुवो॒ नि षी॑द ॥ यो अ॒ग्निर॒ग्नेस्त-प॒सोऽधि॑ जा॒तः शोचा᳚त् पृथि॒व्या उ॒त वा॑ दि॒वस्परि॑ । येन॑ प्र॒जा वि॒श्वक॑र्मा॒ व्यान॒ट् तम॑ग्ने॒ हेडः॒ परि॑ ते वृणक्तु ॥ अ॒जा ह्य॑ग्नेरज॑निष्ट॒ गर्भा॒थ् सा वा अ॑पश्यज्जनि॒तार॒मग्रे᳚ । तया॒ रोह॑माय॒न्नुप॒ मेद्ध्या॑स॒स्तया॑ दे॒वा दे॒वता॒मग्र॑ आयन्न् । श॒र॒भ-( )-मा॑र॒ण्यमनु॑ ते दिशामि॒ तेन॑ चिन्वा॒नस्त॒नुवो॒ निषी॑द ॥ \newline

\textbf{Pada Paata} \newline

ते॒ । दि॒शा॒मि॒ । तेन॑ । चि॒न्वा॒नः । त॒नुवः॑ । नीति॑ । सी॒द॒ ॥ यः । अ॒ग्निः । अ॒ग्नेः । तप॑सः । अधीति॑ । जा॒तः । शोचा᳚त् । पृ॒थि॒व्याः । उ॒त । वा॒ । दि॒वः । परि॑ ॥ येन॑ । प्र॒जा इति॑ प्र - जाः । वि॒श्वक॒र्मेति॑ वि॒श्व - क॒र्मा॒ । व्यान॒डिति॑ वि - आन॑ट् । तम् । अ॒ग्ने॒ । हेडः॑ । परीति॑ । ते॒ । वृ॒ण॒क्तु॒ ॥ अ॒जा । हि । अ॒ग्नेः । अज॑निष्ट । गर्भा᳚त् । सा । वै । अ॒प॒श्य॒त् । ज॒नि॒तार᳚म् । अग्रे᳚ ॥ तया᳚ । रोह᳚म् । आ॒य॒न्न् । उपेति॑ । मेद्ध्या॑सः । तया᳚ । दे॒वाः । दे॒वता᳚म् । अग्रे᳚ । आ॒य॒न्न् ॥ श॒र॒भम् ( ) । आ॒र॒ण्यम् । अन्विति॑ । ते॒ । दि॒शा॒मि॒ । तेन॑ । चि॒न्वा॒नः । त॒नुवः॑ । नीति॑ । सी॒द॒ ॥  \newline


\textbf{Krama Paata} \newline

ते॒ दि॒शा॒मि॒ । दि॒शा॒मि॒ तेन॑ । तेन॑ चिन्वा॒नः । चि॒न्वा॒नस्त॒नुवः॑ । त॒नुवो॒ नि । नि षी॑द । सी॒देति॑ सीद ॥ यो अ॒ग्निः । अ॒ग्निर॒ग्नेः । अ॒ग्ने स्तप॑सः । तप॒सोऽधि॑ । अधि॑ जा॒तः । जा॒तः शोचा᳚त् । शोचा᳚त् पृथि॒व्याः । पृ॒थि॒व्या उ॒त । उ॒त वा᳚ । वा॒ दि॒वः । दि॒वस्परि॑ । परीति॒ परि॑ ॥ येन॑ प्र॒जाः । प्र॒जा वि॒श्वक॑र्मा । प्र॒जा इति॑ प्र - जाः । वि॒श्वक॑र्मा॒ व्यान॑ट् । वि॒श्वक॒र्मेति॑ वि॒श्व - क॒र्मा॒ । व्यान॒ट् तम् । व्यान॒डिति॑ वि - आन॑ट् । तम॑ग्ने । अ॒ग्ने॒ हेडः॑ । हेडः॒ परि॑ । परि॑ ते । ते॒ वृ॒ण॒क्तु॒ । वृ॒ण॒क्त्विति॑ वृणक्तु ॥ अ॒जा हि । ह्य॑ग्नेः । अ॒ग्नेरज॑निष्ट । अज॑निष्ट॒ गर्भा᳚त् । गर्भा॒थ् सा । सा वै । वा अ॑पश्यत् । अ॒प॒श्य॒ज् ज॒नि॒तार᳚म् । ज॒नि॒तार॒मग्रे᳚ । अग्र॒ इत्यग्रे᳚ ॥ तया॒ रोह᳚म् । रोह॑मायन्न् । आ॒य॒न्नुप॑ । उप॒ मेद्ध्या॑सः । मेद्ध्या॑स॒स्तया᳚ । तया॑ दे॒वाः । दे॒वा दे॒वता᳚म् । दे॒वता॒मग्रे᳚ । अग्र॑ आयन्न् । आ॒य॒न्नित्या॑यन्न् ॥ श॒र॒भमा॑र॒ण्यम् ( ) । आ॒र॒ण्यमनु॑ । अनु॑ ते । ते॒ दि॒शा॒मि॒ । दि॒शा॒मि॒ तेन॑ । तेन॑ चिन्वा॒नः । चि॒न्वा॒न स्त॒नुवः॑ । त॒नुवो॒ नि । नि षी॑द । सी॒देति॑ सीद । \newline

\textbf{Jatai Paata} \newline

1. ते॒ दि॒शा॒मि॒ दि॒शा॒मि॒ ते॒ ते॒ दि॒शा॒मि॒ । \newline
2. दि॒शा॒मि॒ तेन॒ तेन॑ दिशामि दिशामि॒ तेन॑ । \newline
3. तेन॑ चिन्वा॒न श्चि॑न्वा॒न स्तेन॒ तेन॑ चिन्वा॒नः । \newline
4. चि॒न्वा॒न स्त॒नुव॑ स्त॒नुव॑ श्चिन्वा॒न श्चि॑न्वा॒न स्त॒नुवः॑ । \newline
5. त॒नुवो॒ नि नि त॒नुव॑ स्त॒नुवो॒ नि । \newline
6. नि षी॑द सीद॒ नि नि षी॑द । \newline
7. सी॒देति॑ सीद । \newline
8. यो अ॒ग्नि र॒ग्निर् यो यो अ॒ग्निः । \newline
9. अ॒ग्नि र॒ग्ने र॒ग्ने र॒ग्नि र॒ग्नि र॒ग्नेः । \newline
10. अ॒ग्ने स्तप॑स॒ स्तप॑सो॒ ऽग्ने र॒ग्ने स्तप॑सः । \newline
11. तप॒सो ऽध्यधि॒ तप॑स॒ स्तप॒सो ऽधि॑ । \newline
12. अधि॑ जा॒तो जा॒तो अध्यधि॑ जा॒तः । \newline
13. जा॒तः शोचा॒ च्छोचा᳚ज् जा॒तो जा॒तः शोचा᳚त् । \newline
14. शोचा᳚त् पृथि॒व्याः पृ॑थि॒व्याः शोचा॒ च्छोचा᳚त् पृथि॒व्याः । \newline
15. पृ॒थि॒व्या उ॒तोत पृ॑थि॒व्याः पृ॑थि॒व्या उ॒त । \newline
16. उ॒त वा॑ वो॒तोत वा᳚ । \newline
17. वा॒ दि॒वो दि॒वो वा॑ वा दि॒वः । \newline
18. दि॒व स्परि॒ परि॑ दि॒वो दि॒व स्परि॑ । \newline
19. परीति॒ परि॑ । \newline
20. येन॑ प्र॒जाः प्र॒जा येन॒ येन॑ प्र॒जाः । \newline
21. प्र॒जा वि॒श्वक॑र्मा वि॒श्वक॑र्मा प्र॒जाः प्र॒जा वि॒श्वक॑र्मा । \newline
22. प्र॒जा इति॑ प्र - जाः । \newline
23. वि॒श्वक॑र्मा॒ व्यान॒ड् व्यान॑ड् वि॒श्वक॑र्मा वि॒श्वक॑र्मा॒ व्यान॑ट् । \newline
24. वि॒श्वक॒र्मेति॑ वि॒श्व - क॒र्मा॒ । \newline
25. व्यान॒ट् तम् तं ॅव्यान॒ड् व्यान॒ट् तम् । \newline
26. व्यान॒डिति॑ वि - आन॑ट् । \newline
27. त म॑ग्ने अग्ने॒ तम् त म॑ग्ने । \newline
28. अ॒ग्ने॒ हेडो॒ हेडो॑ अग्ने अग्ने॒ हेडः॑ । \newline
29. हेडः॒ परि॒ परि॒ हेडो॒ हेडः॒ परि॑ । \newline
30. परि॑ ते ते॒ परि॒ परि॑ ते । \newline
31. ते॒ वृ॒ण॒क्तु वृ॒ण॒क्तु॒ ते॒ ते॒ वृ॒ण॒क्तु॒ । \newline
32. वृ॒ण॒क्त्विति॑ वृणक्तु । \newline
33. अ॒जा हि ह्य॑जा ऽजा हि । \newline
34. ह्य॑ग्ने र॒ग्नेर्. हि ह्य॑ग्नेः । \newline
35. अ॒ग्ने रज॑नि॒ष्टा ज॑निष्टा॒ग्ने र॒ग्ने रज॑निष्ट । \newline
36. अज॑निष्ट॒ गर्भा॒द् गर्भा॒ दज॑नि॒ष्टा ज॑निष्ट॒ गर्भा᳚त् । \newline
37. गर्भा॒थ् सा सा गर्भा॒द् गर्भा॒थ् सा । \newline
38. सा वै वै सा सा वै । \newline
39. वा अ॑पश्य दपश्य॒द् वै वा अ॑पश्यत् । \newline
40. अ॒प॒श्य॒ज् ज॒नि॒तार॑म् जनि॒तार॑ मपश्य दपश्यज् जनि॒तार᳚म् । \newline
41. ज॒नि॒तार॒ मग्रे॒ अग्रे॑ जनि॒तार॑म् जनि॒तार॒ मग्रे᳚ । \newline
42. अग्र॒ इत्यग्रे᳚ । \newline
43. तया॒ रोहꣳ॒॒ रोह॒म् तया॒ तया॒ रोह᳚म् । \newline
44. रोह॑ मायन् नाय॒न् रोहꣳ॒॒ रोह॑ मायन्न् । \newline
45. आ॒य॒न् नुपोपा॑यन् नाय॒न् नुप॑ । \newline
46. उप॒ मेद्ध्या॑सो॒ मेद्ध्या॑स॒ उपोप॒ मेद्ध्या॑सः । \newline
47. मेद्ध्या॑स॒ स्तया॒ तया॒ मेद्ध्या॑सो॒ मेद्ध्या॑स॒ स्तया᳚ । \newline
48. तया॑ दे॒वा दे॒वा स्तया॒ तया॑ दे॒वाः । \newline
49. दे॒वा दे॒वता᳚म् दे॒वता᳚म् दे॒वा दे॒वा दे॒वता᳚म् । \newline
50. दे॒वता॒ मग्रे॒ अग्रे॑ दे॒वता᳚म् दे॒वता॒ मग्रे᳚ । \newline
51. अग्र॑ आयन् नाय॒न् नग्रे॒ अग्र॑ आयन्न् । \newline
52. आ॒य॒न्नित्या॑यन्न् । \newline
53. श॒र॒भ मा॑र॒ण्य मा॑र॒ण्यꣳ श॑र॒भꣳ श॑र॒भ मा॑र॒ण्यम् । \newline
54. आ॒र॒ण्य मन् वन् वा॑र॒ण्य मा॑र॒ण्य मनु॑ । \newline
55. अनु॑ ते ते॒ अन्वनु॑ ते । \newline
56. ते॒ दि॒शा॒मि॒ दि॒शा॒मि॒ ते॒ ते॒ दि॒शा॒मि॒ । \newline
57. दि॒शा॒मि॒ तेन॒ तेन॑ दिशामि दिशामि॒ तेन॑ । \newline
58. तेन॑ चिन्वा॒न श्चि॑न्वा॒न स्तेन॒ तेन॑ चिन्वा॒नः । \newline
59. चि॒न्वा॒न स्त॒नुव॑ स्त॒नुव॑ श्चिन्वा॒न श्चि॑न्वा॒न स्त॒नुवः॑ । \newline
60. त॒नुवो॒ नि नि त॒नुव॑ स्त॒नुवो॒ नि । \newline
61. नि षी॑द सीद॒ नि नि षी॑द । \newline
62. सी॒देति॑ सीद । \newline

\textbf{Ghana Paata } \newline

1. ते॒ दि॒शा॒मि॒ दि॒शा॒मि॒ ते॒ ते॒ दि॒शा॒मि॒ तेन॒ तेन॑ दिशामि ते ते दिशामि॒ तेन॑ । \newline
2. दि॒शा॒मि॒ तेन॒ तेन॑ दिशामि दिशामि॒ तेन॑ चिन्वा॒न श्चि॑न्वा॒न स्तेन॑ दिशामि दिशामि॒ तेन॑ चिन्वा॒नः । \newline
3. तेन॑ चिन्वा॒न श्चि॑न्वा॒न स्तेन॒ तेन॑ चिन्वा॒न स्त॒नुव॑ स्त॒नुव॑ श्चिन्वा॒न स्तेन॒ तेन॑ चिन्वा॒न स्त॒नुवः॑ । \newline
4. चि॒न्वा॒न स्त॒नुव॑ स्त॒नुव॑ श्चिन्वा॒न श्चि॑न्वा॒न स्त॒नुवो॒ नि नि त॒नुव॑ श्चिन्वा॒न श्चि॑न्वा॒न स्त॒नुवो॒ नि । \newline
5. त॒नुवो॒ नि नि त॒नुव॑ स्त॒नुवो॒ नि षी॑द सीद॒ नि त॒नुव॑ स्त॒नुवो॒ नि षी॑द । \newline
6. नि षी॑द सीद॒ नि नि षी॑द । \newline
7. सी॒देति॑ सीद । \newline
8. यो अ॒ग्नि र॒ग्निर् यो यो अ॒ग्नि र॒ग्ने र॒ग्ने र॒ग्निर् यो यो अ॒ग्नि र॒ग्नेः । \newline
9. अ॒ग्नि र॒ग्ने र॒ग्ने र॒ग्नि र॒ग्नि र॒ग्ने स्तप॑स॒ स्तप॑सो॒ ऽग्ने र॒ग्नि र॒ग्नि र॒ग्ने स्तप॑सः । \newline
10. अ॒ग्ने स्तप॑स॒ स्तप॑सो॒ ऽग्ने र॒ग्ने स्तप॒सो ऽध्यधि॒ तप॑सो॒ ऽग्ने र॒ग्ने स्तप॒सो ऽधि॑ । \newline
11. तप॒सो ऽध्यधि॒ तप॑स॒ स्तप॒सो ऽधि॑ जा॒तो जा॒तो अधि॒ तप॑स॒ स्तप॒सो ऽधि॑ जा॒तः । \newline
12. अधि॑ जा॒तो जा॒तो अध्यधि॑ जा॒तः शोचा॒ च्छोचा᳚ज् जा॒तो अध्यधि॑ जा॒तः शोचा᳚त् । \newline
13. जा॒तः शोचा॒ च्छोचा᳚ज् जा॒तो जा॒तः शोचा᳚त् पृथि॒व्याः पृ॑थि॒व्याः शोचा᳚ज् जा॒तो जा॒तः शोचा᳚त् पृथि॒व्याः । \newline
14. शोचा᳚त् पृथि॒व्याः पृ॑थि॒व्याः शोचा॒ च्छोचा᳚त् पृथि॒व्या उ॒तोत पृ॑थि॒व्याः शोचा॒ च्छोचा᳚त् पृथि॒व्या उ॒त । \newline
15. पृ॒थि॒व्या उ॒तोत पृ॑थि॒व्याः पृ॑थि॒व्या उ॒त वा॑ वो॒त पृ॑थि॒व्याः पृ॑थि॒व्या उ॒त वा᳚ । \newline
16. उ॒त वा॑ वो॒तोत वा॑ दि॒वो दि॒वो वो॒तोत वा॑ दि॒वः । \newline
17. वा॒ दि॒वो दि॒वो वा॑ वा दि॒व स्परि॒ परि॑ दि॒वो वा॑ वा दि॒व स्परि॑ । \newline
18. दि॒व स्परि॒ परि॑ दि॒वो दि॒व स्परि॑ । \newline
19. परीति॒ परि॑ । \newline
20. येन॑ प्र॒जाः प्र॒जा येन॒ येन॑ प्र॒जा वि॒श्वक॑र्मा वि॒श्वक॑र्मा प्र॒जा येन॒ येन॑ प्र॒जा वि॒श्वक॑र्मा । \newline
21. प्र॒जा वि॒श्वक॑र्मा वि॒श्वक॑र्मा प्र॒जाः प्र॒जा वि॒श्वक॑र्मा॒ व्यान॒ड् व्यान॑ड् वि॒श्वक॑र्मा प्र॒जाः प्र॒जा वि॒श्वक॑र्मा॒ व्यान॑ट् । \newline
22. प्र॒जा इति॑ प्र - जाः । \newline
23. वि॒श्वक॑र्मा॒ व्यान॒ड् व्यान॑ड् वि॒श्वक॑र्मा वि॒श्वक॑र्मा॒ व्यान॒ट् तम् तं ॅव्यान॑ड् वि॒श्वक॑र्मा वि॒श्वक॑र्मा॒ व्यान॒ट् तम् । \newline
24. वि॒श्वक॒र्मेति॑ वि॒श्व - क॒र्मा॒ । \newline
25. व्यान॒ट् तम् तं ॅव्यान॒ड् व्यान॒ट् त म॑ग्ने अग्ने॒ तं ॅव्यान॒ड् व्यान॒ट् त म॑ग्ने । \newline
26. व्यान॒डिति॑ वि - आन॑ट् । \newline
27. त म॑ग्ने अग्ने॒ तम् त म॑ग्ने॒ हेडो॒ हेडो॑ अग्ने॒ तम् त म॑ग्ने॒ हेडः॑ । \newline
28. अ॒ग्ने॒ हेडो॒ हेडो॑ अग्ने अग्ने॒ हेडः॒ परि॒ परि॒ हेडो॑ अग्ने अग्ने॒ हेडः॒ परि॑ । \newline
29. हेडः॒ परि॒ परि॒ हेडो॒ हेडः॒ परि॑ ते ते॒ परि॒ हेडो॒ हेडः॒ परि॑ ते । \newline
30. परि॑ ते ते॒ परि॒ परि॑ ते वृणक्तु वृणक्तु ते॒ परि॒ परि॑ ते वृणक्तु । \newline
31. ते॒ वृ॒ण॒क्तु वृ॒ण॒क्तु॒ ते॒ ते॒ वृ॒ण॒क्तु॒ । \newline
32. वृ॒ण॒क्त्विति॑ वृणक्तु । \newline
33. अ॒जा हि ह्य॑जा ऽजा ह्य॑ग्ने र॒ग्नेर् ह्य॑जा ऽजा ह्य॑ग्नेः । \newline
34. ह्य॑ग्ने र॒ग्नेर्. हि ह्य॑ग्ने रज॑नि॒ष्टा ज॑निष्टा॒ग्नेर्. हि ह्य॑ग्ने रज॑निष्ट । \newline
35. अ॒ग्ने रज॑नि॒ष्टा ज॑निष्टा॒ग्ने र॒ग्ने रज॑निष्ट॒ गर्भा॒द् गर्भा॒ दज॑निष्टा॒ग्ने र॒ग्ने रज॑निष्ट॒ गर्भा᳚त् । \newline
36. अज॑निष्ट॒ गर्भा॒द् गर्भा॒ दज॑नि॒ष्टा ज॑निष्ट॒ गर्भा॒थ् सा सा गर्भा॒ दज॑नि॒ष्टा ज॑निष्ट॒ गर्भा॒थ् सा । \newline
37. गर्भा॒थ् सा सा गर्भा॒द् गर्भा॒थ् सा वै वै सा गर्भा॒द् गर्भा॒थ् सा वै । \newline
38. सा वै वै सा सा वा अ॑पश्य दपश्य॒द् वै सा सा वा अ॑पश्यत् । \newline
39. वा अ॑पश्य दपश्य॒द् वै वा अ॑पश्यज् जनि॒तार॑म् जनि॒तार॑ मपश्य॒द् वै वा अ॑पश्यज् जनि॒तार᳚म् । \newline
40. अ॒प॒श्य॒ज् ज॒नि॒तार॑म् जनि॒तार॑ मपश्य दपश्यज् जनि॒तार॒ मग्रे॒ अग्रे॑ जनि॒तार॑ मपश्य दपश्यज् जनि॒तार॒ मग्रे᳚ । \newline
41. ज॒नि॒तार॒ मग्रे॒ अग्रे॑ जनि॒तार॑म् जनि॒तार॒ मग्रे᳚ । \newline
42. अग्र॒ इत्यग्रे᳚ । \newline
43. तया॒ रोहꣳ॒॒ रोह॒म् तया॒ तया॒ रोह॑ मायन् नाय॒न् रोह॒म् तया॒ तया॒ रोह॑ मायन्न् । \newline
44. रोह॑ मायन् नाय॒न् रोहꣳ॒॒ रोह॑ माय॒न् नुपोपा॑य॒न् रोहꣳ॒॒ रोह॑ माय॒न् नुप॑ । \newline
45. आ॒य॒न् नुपोपा॑यन् नाय॒न् नुप॒ मेद्ध्या॑सो॒ मेद्ध्या॑स॒ उपा॑यन् नाय॒न् नुप॒ मेद्ध्या॑सः । \newline
46. उप॒ मेद्ध्या॑सो॒ मेद्ध्या॑स॒ उपोप॒ मेद्ध्या॑स॒ स्तया॒ तया॒ मेद्ध्या॑स॒ उपोप॒ मेद्ध्या॑स॒ स्तया᳚ । \newline
47. मेद्ध्या॑स॒ स्तया॒ तया॒ मेद्ध्या॑सो॒ मेद्ध्या॑स॒ स्तया॑ दे॒वा दे॒वा स्तया॒ मेद्ध्या॑सो॒ मेद्ध्या॑स॒ स्तया॑ दे॒वाः । \newline
48. तया॑ दे॒वा दे॒वा स्तया॒ तया॑ दे॒वा दे॒वता᳚म् दे॒वता᳚म् दे॒वा स्तया॒ तया॑ दे॒वा दे॒वता᳚म् । \newline
49. दे॒वा दे॒वता᳚म् दे॒वता᳚म् दे॒वा दे॒वा दे॒वता॒ मग्रे॒ अग्रे॑ दे॒वता᳚म् दे॒वा दे॒वा दे॒वता॒ मग्रे᳚ । \newline
50. दे॒वता॒ मग्रे॒ अग्रे॑ दे॒वता᳚म् दे॒वता॒ मग्र॑ आयन् नाय॒न् नग्रे॑ दे॒वता᳚म् दे॒वता॒ मग्र॑ आयन्न् । \newline
51. अग्र॑ आयन् नाय॒न् नग्रे॒ अग्र॑ आयन्न् । \newline
52. आ॒य॒न्नित्या॑यन्न् । \newline
53. श॒र॒भ मा॑र॒ण्य मा॑र॒ण्यꣳ श॑र॒भꣳ श॑र॒भ मा॑र॒ण्य मन्वन्वा॑र॒ण्यꣳ श॑र॒भꣳ श॑र॒भ मा॑र॒ण्य मनु॑ । \newline
54. आ॒र॒ण्य मन्वन्वा॑र॒ण्य मा॑र॒ण्य मनु॑ ते ते॒ अन्वा॑र॒ण्य मा॑र॒ण्य मनु॑ ते । \newline
55. अनु॑ ते ते॒ अन्वनु॑ ते दिशामि दिशामि ते॒ अन्वनु॑ ते दिशामि । \newline
56. ते॒ दि॒शा॒मि॒ दि॒शा॒मि॒ ते॒ ते॒ दि॒शा॒मि॒ तेन॒ तेन॑ दिशामि ते ते दिशामि॒ तेन॑ । \newline
57. दि॒शा॒मि॒ तेन॒ तेन॑ दिशामि दिशामि॒ तेन॑ चिन्वा॒न श्चि॑न्वा॒न स्तेन॑ दिशामि दिशामि॒ तेन॑ चिन्वा॒नः । \newline
58. तेन॑ चिन्वा॒न श्चि॑न्वा॒न स्तेन॒ तेन॑ चिन्वा॒न स्त॒नुव॑ स्त॒नुव॑ श्चिन्वा॒न स्तेन॒ तेन॑ चिन्वा॒न स्त॒नुवः॑ । \newline
59. चि॒न्वा॒न स्त॒नुव॑ स्त॒नुव॑ श्चिन्वा॒न श्चि॑न्वा॒न स्त॒नुवो॒ नि नि त॒नुव॑ श्चिन्वा॒न श्चि॑न्वा॒न स्त॒नुवो॒ नि । \newline
60. त॒नुवो॒ नि नि त॒नुव॑ स्त॒नुवो॒ नि षी॑द सीद॒ नि त॒नुव॑ स्त॒नुवो॒ नि षी॑द । \newline
61. नि षी॑द सीद॒ नि नि षी॑द । \newline
62. सी॒देति॑ सीद । \newline
\pagebreak
\markright{ TS 4.2.11.1  \hfill https://www.vedavms.in \hfill}

\section{ TS 4.2.11.1 }

\textbf{TS 4.2.11.1 } \newline
\textbf{Samhita Paata} \newline

इन्द्रा᳚ग्नी रोच॒ना दि॒वः परि॒ वाजे॑षु भूषथः । तद्वां᳚ चेति॒ प्रवी॒र्यं᳚ ॥ श्नथ॑द्-वृ॒त्रमु॒त स॑नोति॒ वाज॒मिन्द्रा॒ यो अ॒ग्नी सहु॑री सप॒र्यात् । इ॒र॒ज्यन्ता॑ वस॒व्य॑स्य॒ भूरेः॒ सह॑स्तमा॒ सह॑सा वाज॒यन्ता᳚ ॥ प्र च॑र्.ष॒णिभ्यः॑ पृतना॒ हवे॑षु॒ प्र पृ॑थि॒व्या रि॑रिचाथे दि॒वश्च॑ । प्र सिन्धु॑भ्यः॒ प्रगि॒रिभ्यो॑ महि॒त्वा प्रेन्द्रा᳚ग्नी॒ विश्वा॒ भुव॒नाऽत्य॒न्या ॥ मरु॑तो॒ यस्य॒ हि - [  ] \newline

\textbf{Pada Paata} \newline

इन्द्रा᳚ग्नी॒ इतीन्द्र॑ - अ॒ग्नी॒ । रो॒च॒ना । दि॒वः । परीति॑ । वाजे॑षु । भू॒ष॒थः॒ ॥ तत् । वा॒म् । चे॒ति॒ । प्रेति॑ । वी॒र्य᳚म् ॥ श्नथ॑त् । वृ॒त्रम् । उ॒त । स॒नो॒ति॒ । वाज᳚म् । इन्द्रा᳚ । यः । अ॒ग्नी इति॑ । सहु॑री॒ इति॑ स - हु॒री॒ । स॒प॒र्यात् ॥ इ॒र॒ज्यन्ता᳚ । व॒स॒व्य॑स्य । भूरेः᳚ । सह॑स्त॒मेति॒ सहः॑ - त॒मा॒ । सह॑सा । वा॒ज॒यन्तेति॑ वाज - यन्ता᳚ ॥ प्रेति॑ । च॒र्॒.ष॒णिभ्य॒ इति॑ चर्.ष॒णि-भ्यः॒ । पृ॒त॒ना॒ । हवे॑षु । प्रेति॑ । पृ॒थि॒व्याः । रि॒रि॒चा॒थे॒ इति॑ । दि॒वः । च॒ ॥ प्रेति॑ । सिन्धु॑भ्य॒ इति॒ सिन्धु॑ - भ्यः॒ । प्रेति॑ । गि॒रिभ्य॒ इति॑ गि॒रि - भ्यः॒ । म॒हि॒त्वेति॑ महि - त्वा । प्रेति॑ । इ॒न्द्रा॒ग्नी॒ इती᳚न्द्र - अ॒ग्नी॒ । विश्वा᳚ । भुव॑ना । अतीति॑ । अ॒न्या ॥ मरु॑तः । यस्य॑ । हि ।  \newline


\textbf{Krama Paata} \newline

इन्द्रा᳚ग्नी रोच॒ना । इन्द्रा᳚ग्नी॒ इतीन्द्र॑ - अ॒ग्नी॒ । रो॒च॒ना दि॒वः । दि॒वः परि॑ । परि॒ वाजे॑षु । वाजे॑षु भूषथः । भू॒ष॒थ॒ इति॑ भूषथः ॥ तद् वा᳚म् । वां॒ चे॒ति॒ । चे॒ति॒ प्र । प्र वी॒र्य᳚म् । वी॒र्य॑मिति॑ वी॒र्य᳚म् ॥ श्ञथ॑द् वृ॒त्रम् । वृ॒त्रमु॒त । उ॒त स॑नोति । स॒नो॒ति॒ वाज᳚म् । वाज॒मिन्द्रा᳚ । इन्द्रा॒ यः । यो अ॒ग्नी । अ॒ग्नी सहु॑री । अ॒ग्नी इत्य॒ग्नी । सहु॑री सप॒र्यात् । सहु॑री॒ इति॒ स - हु॒री॒ । स॒प॒र्यादिति॑ सप॒र्यात् ॥ इ॒र॒ज्यन्ता॑ वस॒व्य॑स्य । व॒स॒व्य॑स्य॒ भूरेः᳚ । भूरेः॒ सह॑स्तमा । सह॑स्तमा॒ सह॑सा । सह॑स्त॒मेति॒ सहः॑ - त॒मा॒ । सह॑सा वाज॒यन्ता᳚ । वा॒ज॒यन्तेति॑ वाज - यन्ता᳚ ॥ प्र च॑र्॒.ष॒णिभ्यः॑ । च॒र्.ष॒णिभ्यः॑ पृतना । च॒र्॒.ष॒णिभ्य॒ इति॑ चर्.ष॒णि - भ्यः॒ । पृ॒त॒ना॒ हवे॑षु । हवे॑षु॒ प्र । प्र पृ॑थि॒व्याः । पृ॒थि॒व्या रि॑रिचाथे । रि॒रि॒चा॒थे॒ दि॒वः । रि॒रि॒चा॒थे॒ इति॑ रिरिचाथे । दि॒वश्च॑ । चेति॑ च ॥ प्र सिन्धु॑भ्यः । सिन्धु॑भ्यः॒ प्र । सिन्धु॑भ्य॒ इति॒ सिन्धु॑ - भ्यः॒ । प्र गि॒रिभ्यः॑ । गि॒रिभ्यो॑ महि॒त्वा । गि॒रिभ्य॒ इति॑ गि॒रि - भ्यः॒ । म॒हि॒त्वा प्र । म॒हि॒त्वेति॑ महि - त्वा । प्रेन्द्रा᳚ग्नी । इ॒न्द्रा॒ग्नी॒ विश्वा᳚ । इ॒न्द्रा॒ग्नी॒ इती᳚न्द्र - अ॒ग्नी॒ । विश्वा॒ भुव॑ना । भुव॒नाऽति॑ । अत्य॒न्या । अ॒न्येत्य॒न्या ॥ मरु॑तो॒ यस्य॑ । यस्य॒ हि । हि क्षये᳚ \newline

\textbf{Jatai Paata} \newline

1. इन्द्रा᳚ग्नी रोच॒ना रो॑च॒ नेन्द्रा᳚ग्नी॒ इन्द्रा᳚ग्नी रोच॒ना । \newline
2. इन्द्रा᳚ग्नी॒ इतीन्द्र॑ - अ॒ग्नी॒ । \newline
3. रो॒च॒ना दि॒वो दि॒वो रो॑च॒ना रो॑च॒ना दि॒वः । \newline
4. दि॒वः परि॒ परि॑ दि॒वो दि॒वः परि॑ । \newline
5. परि॒ वाजे॑षु॒ वाजे॑षु॒ परि॒ परि॒ वाजे॑षु । \newline
6. वाजे॑षु भूषथो भूषथो॒ वाजे॑षु॒ वाजे॑षु भूषथः । \newline
7. भू॒ष॒थ॒ इति॑ भूषथः । \newline
8. तद् वां᳚ ॅवा॒म् तत् तद् वा᳚म् । \newline
9. वा॒म् चे॒ति॒ चे॒ति॒ वां॒ ॅवा॒म् चे॒ति॒ । \newline
10. चे॒ति॒ प्र प्र चे॑ति चेति॒ प्र । \newline
11. प्र वी॒र्यं॑ ॅवी॒र्य॑म् प्र प्र वी॒र्य᳚म् । \newline
12. वी॒र्य॑मिति॑ वी॒र्य᳚म् । \newline
13. श्ञथ॑द् वृ॒त्रं ॅवृ॒त्रꣳ श्ञथ॒ च्छ्ञथ॑द् वृ॒त्रम् । \newline
14. वृ॒त्र मु॒तोत वृ॒त्रं ॅवृ॒त्र मु॒त । \newline
15. उ॒त स॑नोति सनो त्यु॒तोत स॑नोति । \newline
16. स॒नो॒ति॒ वाजं॒ ॅवाजꣳ॑ सनोति सनोति॒ वाज᳚म् । \newline
17. वाज॒ मिन्द्रेन्द्रा॒ वाजं॒ ॅवाज॒ मिन्द्रा᳚ । \newline
18. इन्द्रा॒ यो य इन्द्रेन्द्रा॒ यः । \newline
19. यो अ॒ग्नी अ॒ग्नी यो यो अ॒ग्नी । \newline
20. अ॒ग्नी सहु॑री॒ सहु॑री अ॒ग्नी अ॒ग्नी सहु॑री । \newline
21. अ॒ग्नी इत्य॒ग्नी । \newline
22. सहु॑री सप॒र्याथ् स॑प॒र्याथ् सहु॑री॒ सहु॑री सप॒र्यात् । \newline
23. सहु॑री॒ इति॒ स - हु॒री॒ । \newline
24. स॒प॒र्यादिति॑ सप॒र्यात् । \newline
25. इ॒र॒ज्यन्ता॑ वस॒व्य॑स्य वस॒व्य॑ स्येर॒ज्यन्ते॑ र॒ज्यन्ता॑ वस॒व्य॑स्य । \newline
26. व॒स॒व्य॑स्य॒ भूरे॒र् भूरे᳚र् वस॒व्य॑स्य वस॒व्य॑स्य॒ भूरेः᳚ । \newline
27. भूरेः॒ सह॑स्तमा॒ सह॑स्तमा॒ भूरे॒र् भूरेः॒ सह॑स्तमा । \newline
28. सह॑स्तमा॒ सह॑सा॒ सह॑सा॒ सह॑स्तमा॒ सह॑स्तमा॒ सह॑सा । \newline
29. सह॑स्त॒मेति॒ सहः॑ - त॒मा॒ । \newline
30. सह॑सा वाज॒यन्ता॑ वाज॒यन्ता॒ सह॑सा॒ सह॑सा वाज॒यन्ता᳚ । \newline
31. वा॒ज॒यन्तेति॑ वाज - यन्ता᳚ । \newline
32. प्र च॑र्.ष॒णिभ्य॑ श्चर्.ष॒णिभ्यः॒ प्र प्र च॑र्.ष॒णिभ्यः॑ । \newline
33. च॒र्॒.ष॒णिभ्यः॑ पृतना पृतना चर्.ष॒णिभ्य॑ श्चर्.ष॒णिभ्यः॑ पृतना । \newline
34. च॒र्॒.ष॒णिभ्य॒ इति॑ चर्.ष॒णि - भ्यः॒ । \newline
35. पृ॒त॒ना॒ हवे॑षु॒ हवे॑षु पृतना पृतना॒ हवे॑षु । \newline
36. हवे॑षु॒ प्र प्र हवे॑षु॒ हवे॑षु॒ प्र । \newline
37. प्र पृ॑थि॒व्याः पृ॑थि॒व्याः प्र प्र पृ॑थि॒व्याः । \newline
38. पृ॒थि॒व्या रि॑रिचाथे रिरिचाथे पृथि॒व्याः पृ॑थि॒व्या रि॑रिचाथे । \newline
39. रि॒रि॒चा॒थे॒ दि॒वो दि॒वो रि॑रिचाथे रिरिचाथे दि॒वः । \newline
40. रि॒रि॒चा॒थे॒ इति॑ रिरिचाथे । \newline
41. दि॒वश्च॑ च दि॒वो दि॒वश्च॑ । \newline
42. चेति॑ च । \newline
43. प्र सिन्धु॑भ्यः॒ सिन्धु॑भ्यः॒ प्र प्र सिन्धु॑भ्यः । \newline
44. सिन्धु॑भ्यः॒ प्र प्र सिन्धु॑भ्यः॒ सिन्धु॑भ्यः॒ प्र । \newline
45. सिन्धु॑भ्य॒ इति॒ सिन्धु॑ - भ्यः॒ । \newline
46. प्र गि॒रिभ्यो॑ गि॒रिभ्यः॒ प्र प्र गि॒रिभ्यः॑ । \newline
47. गि॒रिभ्यो॑ महि॒त्वा म॑हि॒त्वा गि॒रिभ्यो॑ गि॒रिभ्यो॑ महि॒त्वा । \newline
48. गि॒रिभ्य॒ इति॑ गि॒रि - भ्यः॒ । \newline
49. म॒हि॒त्वा प्र प्र म॑हि॒त्वा म॑हि॒त्वा प्र । \newline
50. म॒हि॒त्वेति॑ महि - त्वा । \newline
51. प्रेन्द्रा᳚ग्नी इन्द्रानी॒ प्र प्रेन्द्रा᳚ग्नी । \newline
52. इ॒न्द्रा॒ग्नी॒ विश्वा॒ विश्वे᳚न्द्राग्नी इन्द्राग्नी॒ विश्वा᳚ । \newline
53. इ॒न्द्रा॒ग्नी॒ इती᳚न्द्र - अ॒ग्नी॒ । \newline
54. विश्वा॒ भुव॑ना॒ भुव॑ना॒ विश्वा॒ विश्वा॒ भुव॑ना । \newline
55. भुव॒ना ऽत्यति॒ भुव॑ना॒ भुव॒ना ऽति॑ । \newline
56. अत्य॒न्या ऽन्या ऽत्य त्य॒न्या । \newline
57. अ॒न्येत्य॒न्या । \newline
58. मरु॑तो॒ यस्य॒ यस्य॒ मरु॑तो॒ मरु॑तो॒ यस्य॑ । \newline
59. यस्य॒ हि हि यस्य॒ यस्य॒ हि । \newline
60. हि क्षये॒ क्षये॒ हि हि क्षये᳚ । \newline

\textbf{Ghana Paata } \newline

1. इन्द्रा᳚ग्नी रोच॒ना रो॑च॒नेन्द्रा᳚ग्नी॒ इन्द्रा᳚ग्नी रोच॒ना दि॒वो दि॒वो रो॑च॒नेन्द्रा᳚ग्नी॒ इन्द्रा᳚ग्नी रोच॒ना दि॒वः । \newline
2. इन्द्रा᳚ग्नी॒ इतीन्द्र॑ - अ॒ग्नी॒ । \newline
3. रो॒च॒ना दि॒वो दि॒वो रो॑च॒ना रो॑च॒ना दि॒वः परि॒ परि॑ दि॒वो रो॑च॒ना रो॑च॒ना दि॒वः परि॑ । \newline
4. दि॒वः परि॒ परि॑ दि॒वो दि॒वः परि॒ वाजे॑षु॒ वाजे॑षु॒ परि॑ दि॒वो दि॒वः परि॒ वाजे॑षु । \newline
5. परि॒ वाजे॑षु॒ वाजे॑षु॒ परि॒ परि॒ वाजे॑षु भूषथो भूषथो॒ वाजे॑षु॒ परि॒ परि॒ वाजे॑षु भूषथः । \newline
6. वाजे॑षु भूषथो भूषथो॒ वाजे॑षु॒ वाजे॑षु भूषथः । \newline
7. भू॒ष॒थ॒ इति॑ भूषथः । \newline
8. तद् वां᳚ ॅवा॒म् तत् तद् वा᳚म् चेति चेति वा॒म् तत् तद् वा᳚म् चेति । \newline
9. वा॒म् चे॒ति॒ चे॒ति॒ वां॒ ॅवा॒म् चे॒ति॒ प्र प्र चे॑ति वां ॅवाम् चेति॒ प्र । \newline
10. चे॒ति॒ प्र प्र चे॑ति चेति॒ प्र वी॒र्यं॑ ॅवी॒र्य॑म् प्र चे॑ति चेति॒ प्र वी॒र्य᳚म् । \newline
11. प्र वी॒र्यं॑ ॅवी॒र्य॑म् प्र प्र वी॒र्य᳚म् । \newline
12. वी॒र्य॑मिति॑ वी॒र्य᳚म् । \newline
13. श्ञथ॑द् वृ॒त्रं ॅवृ॒त्रꣳ श्ञथ॒ च्छ्ञथ॑द् वृ॒त्र मु॒तोत वृ॒त्रꣳ श्ञथ॒ च्छ्ञथ॑द् वृ॒त्र मु॒त । \newline
14. वृ॒त्र मु॒तोत वृ॒त्रं ॅवृ॒त्र मु॒त स॑नोति सनो त्यु॒त वृ॒त्रं ॅवृ॒त्र मु॒त स॑नोति । \newline
15. उ॒त स॑नोति सनो त्यु॒तोत स॑नोति॒ वाजं॒ ॅवाजꣳ॑ सनो त्यु॒तोत स॑नोति॒ वाज᳚म् । \newline
16. स॒नो॒ति॒ वाजं॒ ॅवाजꣳ॑ सनोति सनोति॒ वाज॒ मिन्द्रेन्द्रा॒ वाजꣳ॑ सनोति सनोति॒ वाज॒ मिन्द्रा᳚ । \newline
17. वाज॒ मिन्द्रेन्द्रा॒ वाजं॒ ॅवाज॒ मिन्द्रा॒ यो य इन्द्रा॒ वाजं॒ ॅवाज॒ मिन्द्रा॒ यः । \newline
18. इन्द्रा॒ यो य इन्द्रेन्द्रा॒ यो अ॒ग्नी अ॒ग्नी य इन्द्रेन्द्रा॒ यो अ॒ग्नी । \newline
19. यो अ॒ग्नी अ॒ग्नी यो यो अ॒ग्नी सहु॑री॒ सहु॑री अ॒ग्नी यो यो अ॒ग्नी सहु॑री । \newline
20. अ॒ग्नी सहु॑री॒ सहु॑री अ॒ग्नी अ॒ग्नी सहु॑री सप॒र्याथ् स॑प॒र्याथ् सहु॑री अ॒ग्नी अ॒ग्नी सहु॑री सप॒र्यात् । \newline
21. अ॒ग्नी इत्य॒ग्नी । \newline
22. सहु॑री सप॒र्याथ् स॑प॒र्याथ् सहु॑री॒ सहु॑री सप॒र्यात् । \newline
23. सहु॑री॒ इति॒ स - हु॒री॒ । \newline
24. स॒प॒र्यादिति॑ सप॒र्यात् । \newline
25. इ॒र॒ज्यन्ता॑ वस॒व्य॑स्य वस॒व्य॑स्ये र॒ज्यन्ते॑ र॒ज्यन्ता॑ वस॒व्य॑स्य॒ भूरे॒र् भूरे᳚र् वस॒व्य॑स्ये र॒ज्यन्ते॑ र॒ज्यन्ता॑ वस॒व्य॑स्य॒ भूरेः᳚ । \newline
26. व॒स॒व्य॑स्य॒ भूरे॒र् भूरे᳚र् वस॒व्य॑स्य वस॒व्य॑स्य॒ भूरेः॒ सह॑स्तमा॒ सह॑स्तमा॒ भूरे᳚र् वस॒व्य॑स्य वस॒व्य॑स्य॒ भूरेः॒ सह॑स्तमा । \newline
27. भूरेः॒ सह॑स्तमा॒ सह॑स्तमा॒ भूरे॒र् भूरेः॒ सह॑स्तमा॒ सह॑सा॒ सह॑सा॒ सह॑स्तमा॒ भूरे॒र् भूरेः॒ सह॑स्तमा॒ सह॑सा । \newline
28. सह॑स्तमा॒ सह॑सा॒ सह॑सा॒ सह॑स्तमा॒ सह॑स्तमा॒ सह॑सा वाज॒यन्ता॑ वाज॒यन्ता॒ सह॑सा॒ सह॑स्तमा॒ सह॑स्तमा॒ सह॑सा वाज॒यन्ता᳚ । \newline
29. सह॑स्त॒मेति॒ सहः॑ - त॒मा॒ । \newline
30. सह॑सा वाज॒यन्ता॑ वाज॒यन्ता॒ सह॑सा॒ सह॑सा वाज॒यन्ता᳚ । \newline
31. वा॒ज॒यन्तेति॑ वाज - यन्ता᳚ । \newline
32. प्र च॑र्.ष॒णिभ्य॑ श्चर्.ष॒णिभ्यः॒ प्र प्र च॑र्.ष॒णिभ्यः॑ पृतना पृतना चर्.ष॒णिभ्यः॒ प्र प्र च॑र्.ष॒णिभ्यः॑ पृतना । \newline
33. च॒र्॒.ष॒णिभ्यः॑ पृतना पृतना चर्.ष॒णिभ्य॑ श्चर्.ष॒णिभ्यः॑ पृतना॒ हवे॑षु॒ हवे॑षु पृतना चर्.ष॒णिभ्य॑ श्चर्.ष॒णिभ्यः॑ पृतना॒ हवे॑षु । \newline
34. च॒र्॒.ष॒णिभ्य॒ इति॑ चर्.ष॒णि - भ्यः॒ । \newline
35. पृ॒त॒ना॒ हवे॑षु॒ हवे॑षु पृतना पृतना॒ हवे॑षु॒ प्र प्र हवे॑षु पृतना पृतना॒ हवे॑षु॒ प्र । \newline
36. हवे॑षु॒ प्र प्र हवे॑षु॒ हवे॑षु॒ प्र पृ॑थि॒व्याः पृ॑थि॒व्याः प्र हवे॑षु॒ हवे॑षु॒ प्र पृ॑थि॒व्याः । \newline
37. प्र पृ॑थि॒व्याः पृ॑थि॒व्याः प्र प्र पृ॑थि॒व्या रि॑रिचाथे रिरिचाथे पृथि॒व्याः प्र प्र पृ॑थि॒व्या रि॑रिचाथे । \newline
38. पृ॒थि॒व्या रि॑रिचाथे रिरिचाथे पृथि॒व्याः पृ॑थि॒व्या रि॑रिचाथे दि॒वो दि॒वो रि॑रिचाथे पृथि॒व्याः पृ॑थि॒व्या रि॑रिचाथे दि॒वः । \newline
39. रि॒रि॒चा॒थे॒ दि॒वो दि॒वो रि॑रिचाथे रिरिचाथे दि॒वश्च॑ च दि॒वो रि॑रिचाथे रिरिचाथे दि॒वश्च॑ । \newline
40. रि॒रि॒चा॒थे॒ इति॑ रिरिचाथे । \newline
41. दि॒वश्च॑ च दि॒वो दि॒वश्च॑ । \newline
42. चेति॑ च । \newline
43. प्र सिन्धु॑भ्यः॒ सिन्धु॑भ्यः॒ प्र प्र सिन्धु॑भ्यः॒ प्र प्र सिन्धु॑भ्यः॒ प्र प्र सिन्धु॑भ्यः॒ प्र । \newline
44. सिन्धु॑भ्यः॒ प्र प्र सिन्धु॑भ्यः॒ सिन्धु॑भ्यः॒ प्र गि॒रिभ्यो॑ गि॒रिभ्यः॒ प्र सिन्धु॑भ्यः॒ सिन्धु॑भ्यः॒ प्र गि॒रिभ्यः॑ । \newline
45. सिन्धु॑भ्य॒ इति॒ सिन्धु॑ - भ्यः॒ । \newline
46. प्र गि॒रिभ्यो॑ गि॒रिभ्यः॒ प्र प्र गि॒रिभ्यो॑ महि॒त्वा म॑हि॒त्वा गि॒रिभ्यः॒ प्र प्र गि॒रिभ्यो॑ महि॒त्वा । \newline
47. गि॒रिभ्यो॑ महि॒त्वा म॑हि॒त्वा गि॒रिभ्यो॑ गि॒रिभ्यो॑ महि॒त्वा प्र प्र म॑हि॒त्वा गि॒रिभ्यो॑ गि॒रिभ्यो॑ महि॒त्वा प्र । \newline
48. गि॒रिभ्य॒ इति॑ गि॒रि - भ्यः॒ । \newline
49. म॒हि॒त्वा प्र प्र म॑हि॒त्वा म॑हि॒त्वा प्रेन्द्रा᳚ग्नी इन्द्राग्नी॒ प्र म॑हि॒त्वा म॑हि॒त्वा प्रेन्द्रा᳚ग्नी । \newline
50. म॒हि॒त्वेति॑ महि - त्वा । \newline
51. प्रेन्द्रा᳚ग्नी इन्द्रानी॒ प्र प्रेन्द्रा᳚ग्नी॒ विश्वा॒ विश्वे᳚न्द्राग्नी॒ प्र प्रेन्द्रा᳚ग्नी॒ विश्वा᳚ । \newline
52. इ॒न्द्रा॒ग्नी॒ विश्वा॒ विश्वे᳚न्द्राग्नी इन्द्राग्नी॒ विश्वा॒ भुव॑ना॒ भुव॑ना॒ विश्वे᳚न्द्राग्नी इन्द्राग्नी॒ विश्वा॒ भुव॑ना । \newline
53. इ॒न्द्रा॒ग्नी॒ इती᳚न्द्र - अ॒ग्नी॒ । \newline
54. विश्वा॒ भुव॑ना॒ भुव॑ना॒ विश्वा॒ विश्वा॒ भुव॒ना ऽत्यति॒ भुव॑ना॒ विश्वा॒ विश्वा॒ भुव॒ना ऽति॑ । \newline
55. भुव॒ना ऽत्यति॒ भुव॑ना॒ भुव॒ना ऽत्य॒न्या ऽन्या ऽति॒ भुव॑ना॒ भुव॒ना ऽत्य॒न्या । \newline
56. अत्य॒न्या ऽन्या ऽत्यत्य॒न्या । \newline
57. अ॒न्येत्य॒न्या । \newline
58. मरु॑तो॒ यस्य॒ यस्य॒ मरु॑तो॒ मरु॑तो॒ यस्य॒ हि हि यस्य॒ मरु॑तो॒ मरु॑तो॒ यस्य॒ हि । \newline
59. यस्य॒ हि हि यस्य॒ यस्य॒ हि क्षये॒ क्षये॒ हि यस्य॒ यस्य॒ हि क्षये᳚ । \newline
60. हि क्षये॒ क्षये॒ हि हि क्षये॑ पा॒थ पा॒थ क्षये॒ हि हि क्षये॑ पा॒थ । \newline
\pagebreak
\markright{ TS 4.2.11.2  \hfill https://www.vedavms.in \hfill}

\section{ TS 4.2.11.2 }

\textbf{TS 4.2.11.2 } \newline
\textbf{Samhita Paata} \newline

क्षये॑ पा॒था दि॒वो वि॑महसः । स सु॑गो॒पात॑मो॒ जनः॑ ॥ य॒ज्ञिर्वा॑ यज्ञ्वाहसो॒ विप्र॑स्य वा मती॒नां । मरु॑तः शृणु॒ता हवं᳚ ॥ श्रि॒यसे॒ कं भा॒नुभिः॒ सं मि॑मिक्षिरे॒ ते र॒श्मिभि॒स्त ऋक्व॑भिः सुखा॒दयः॑ । ते वाशी॑मन्त इ॒ष्मिणो॒ अभी॑रवो वि॒द्रे प्रि॒यस्य॒ मारु॑तस्य॒ धाम्नः॑ ॥ अव॑ ते॒ हेड॒> 1, उदु॑त्त॒मं >2 ॥ कया॑ नश्चि॒त्र आ भु॑वदू॒ती स॒दा वृ॑धः॒ सखा᳚ । कया॒ शचि॑ष्ठया वृ॒ता ॥ \newline

\textbf{Pada Paata} \newline

क्षये᳚ । पा॒थ । दि॒वः । वि॒म॒ह॒स॒ इति॑ वि - म॒ह॒सः॒ ॥ सः । सु॒गो॒पात॑म॒ इति॑ सुगो॒प - त॒मः॒ । जनः॑ ॥ य॒ज्ञिः । वा॒ । य॒ज्ञ्॒वा॒ह॒स॒ इति॑ यज्ञ् - वा॒ह॒सः॒ । विप्र॑स्य । वा॒ । म॒ती॒नाम् ॥ मरु॑तः । शृ॒णु॒त । हव᳚म् ॥ श्रि॒यसे᳚ । कम् । भा॒नुभि॒रिति॑ भा॒नु - भिः॒ । समिति॑ । मि॒मि॒क्षि॒रे॒ । ते । र॒श्मिभि॒रिति॑ र॒श्मि - भिः॒ । ते । ऋक्व॑भि॒रित्यृक्व॑ - भिः॒ । सु॒खा॒दय॒ इति॑ सु - खा॒दयः॑ ॥ ते । वाशी॑मन्त॒ इति॒ वाशि॑ - म॒न्तः॒ । इ॒ष्मिणः॑ । अभी॑रवः । वि॒द्रे । प्रि॒यस्य॑ । मारु॑तस्य । धाम्नः॑ ॥ अवेति॑ । ते॒ । हेडः॑ । उदिति॑ । उ॒त्त॒ममित्यु॑त् - त॒मम् ॥ कया᳚ । नः॒ । चि॒त्रः । एति॑ । भु॒व॒त् । ऊ॒ती । स॒दावृ॑ध॒ इति॑ स॒दा - वृ॒धः॒ । सखा᳚ ॥ कया᳚ । शचि॑ष्ठया । वृ॒ता ॥  \newline


\textbf{Krama Paata} \newline

क्षये॑ पा॒थ । पा॒था दि॒वः । दि॒वो वि॑महसः । वि॒म॒ह॒स॒ इति॑ वि - म॒ह॒सः॒ ॥ स सु॑गो॒पात॑मः । सु॒गो॒पात॑मो॒ जनः॑ । सु॒गो॒पात॑म॒ इति॑ सुगो॒प - त॒मः॒ । जन॒ इति॒ जनः॑ ॥ य॒ज्ञिर् वा᳚ । वा॒ य॒ज्ञ्॒वा॒ह॒सः॒ । य॒ज्ञ्॒वा॒ह॒सो॒ विप्र॑स्य । य॒ज्ञ्॒वा॒ह॒स॒ इति॑ यज्ञ् - वा॒ह॒सः॒ । विप्र॑स्य वा । वा॒ म॒ती॒नाम् । म॒ती॒नामिति॑ मती॒नाम् ॥ मरु॑तः शृणु॒त । शृ॒णु॒ता हव᳚म् । हव॒मिति॒ हव᳚म् ॥ श्रि॒यसे॒ कम् । कम् भा॒नुभिः॑ । भा॒नुभिः॒ सम् । भा॒नुभि॒रिति॑ भा॒नु - भिः॒ । सम् मि॑मिक्षिरे । मि॒मि॒क्ष॒रे॒ ते । ते र॒श्मिभिः॑ । र॒श्मिभि॒स्ते । र॒श्मिभि॒रिति॑ र॒श्मि - भिः॒ । त ऋक्व॑भिः । ऋक्व॑भिः सुखा॒दयः॑ । ऋक्व॑भि॒रित्यृक्व॑ - भिः॒ । सु॒खा॒दय॒ इति॑ सु - खा॒दयः॑ ॥ ते वाशी॑मन्तः । वाशी॑मन्त इ॒ष्मिणः॑ । वाशी॑मन्त॒ इति॒ वाशि॑ - म॒न्तः॒ । इ॒ष्मिणो॒ अभी॑रवः । अभी॑रवो वि॒द्रे । वि॒द्रे प्रि॒यस्य॑ । प्रि॒यस्य॒ मारु॑तस्य । मारु॑तस्य॒ धाम्नः॑ । धाम्न॒ इति॒ धाम्नः॑ ॥ अव॑ ते । ते॒ हेडः॑ । हेड॒ उत् । उदु॑त्त॒मम् । उ॒त्त॒ममित्यु॑त् - त॒मम् ॥ कया॑ नः । न॒श्चि॒त्रः । चि॒त्र आ । आ भु॑वत् । भु॒व॒दू॒ती । ऊ॒ती स॒दावृ॑धः । स॒दावृ॑धः॒ सखा᳚ । स॒दावृ॑ध॒ इति॑ स॒दा - वृ॒धः॒ । सखेति॒ सखा᳚ ॥? कया॒ शचि॑ष्ठया । शचि॑ष्ठया वृ॒ता । वृ॒तेति॑ वृ॒ता । \newline

\textbf{Jatai Paata} \newline

1. क्षये॑ पा॒थ पा॒थ क्षये॒ क्षये॑ पा॒थ । \newline
2. पा॒था दि॒वो दि॒वः पा॒थ पा॒था दि॒वः । \newline
3. दि॒वो वि॑महसो विमहसो दि॒वो दि॒वो वि॑महसः । \newline
4. वि॒म॒ह॒स॒ इति॑ वि - म॒ह॒सः॒ । \newline
5. स सु॑गो॒पात॑मः सुगो॒पात॑मः॒ स स सु॑गो॒पात॑मः । \newline
6. सु॒गो॒पात॑मो॒ जनो॒ जनः॑ सुगो॒पात॑मः सुगो॒पात॑मो॒ जनः॑ । \newline
7. सु॒गो॒पात॑म॒ इति॑ सुगो॒प - त॒मः॒ । \newline
8. जन॒ इति॒ जनः॑ । \newline
9. य॒ज्ञिर् वा॑ वा य॒ज्ञिर् य॒ज्ञिर् वा᳚ । \newline
10. वा॒ य॒ज्ञ्॒वा॒ह॒सो॒ य॒ज्ञ्॒वा॒ह॒सो॒ वा॒ वा॒ य॒ज्ञ्॒वा॒ह॒सः॒ । \newline
11. य॒ज्ञ्॒वा॒ह॒सो॒ विप्र॑स्य॒ विप्र॑स्य यज्ञ्वाहसो यज्ञ्वाहसो॒ विप्र॑स्य । \newline
12. य॒ज्ञ्॒वा॒ह॒स॒ इति॑ यज्ञ् - वा॒ह॒सः॒ । \newline
13. विप्र॑स्य वा वा॒ विप्र॑स्य॒ विप्र॑स्य वा । \newline
14. वा॒ म॒ती॒नाम् म॑ती॒नां ॅवा॑ वा मती॒नाम् । \newline
15. म॒ती॒नामिति॑ मती॒नाम् । \newline
16. मरु॑तः शृणु॒त शृ॑णु॒त मरु॑तो॒ मरु॑तः शृणु॒त । \newline
17. शृ॒णु॒ता हवꣳ॒॒ हवꣳ॑ शृणु॒त शृ॑णु॒ता हव᳚म् । \newline
18. हव॒मिति॒ हव᳚म् । \newline
19. श्रि॒यसे॒ कम् कꣳ श्रि॒यसे᳚ श्रि॒यसे॒ कम् । \newline
20. कम् भा॒नुभि॑र् भा॒नुभिः॒ कम् कम् भा॒नुभिः॑ । \newline
21. भा॒नुभिः॒ सꣳ सम् भा॒नुभि॑र् भा॒नुभिः॒ सम् । \newline
22. भा॒नुभि॒रिति॑ भा॒नु - भिः॒ । \newline
23. सम् मि॑मिक्षिरे मिमिक्षिरे॒ सꣳ सम् मि॑मिक्षिरे । \newline
24. मि॒मि॒क्षि॒रे॒ ते ते मि॑मिक्षिरे मिमिक्षिरे॒ ते । \newline
25. ते र॒श्मिभी॑ र॒श्मिभि॒ स्ते ते र॒श्मिभिः॑ । \newline
26. र॒श्मिभि॒ स्ते ते र॒श्मिभी॑ र॒श्मिभि॒ स्ते । \newline
27. र॒श्मिभि॒रिति॑ र॒श्मि - भिः॒ । \newline
28. त ऋक्व॑भि॒र्॒. ऋक्व॑भि॒ स्ते त ऋक्व॑भिः । \newline
29. ऋक्व॑भिः सुखा॒दयः॑ सुखा॒दय॒ ऋक्व॑भि॒र्॒. ऋक्व॑भिः सुखा॒दयः॑ । \newline
30. ऋक्व॑भि॒रित्यृक्व॑ - भिः॒ । \newline
31. सु॒खा॒दय॒ इति॑ सु - खा॒दयः॑ । \newline
32. ते वाशी॑मन्तो॒ वाशी॑मन्त॒ स्ते ते वाशी॑मन्तः । \newline
33. वाशी॑मन्त इ॒ष्मिण॑ इ॒ष्मिणो॒ वाशी॑मन्तो॒ वाशी॑मन्त इ॒ष्मिणः॑ । \newline
34. वाशी॑मन्त॒ इति॒ वाशि॑ - म॒न्तः॒ । \newline
35. इ॒ष्मिणो॒ अभी॑रवो॒ अभी॑रव इ॒ष्मिण॑ इ॒ष्मिणो॒ अभी॑रवः । \newline
36. अभी॑रवो वि॒द्रे वि॒द्रे अभी॑रवो॒ अभी॑रवो वि॒द्रे । \newline
37. वि॒द्रे प्रि॒यस्य॑ प्रि॒यस्य॑ वि॒द्रे वि॒द्रे प्रि॒यस्य॑ । \newline
38. प्रि॒यस्य॒ मारु॑तस्य॒ मारु॑तस्य प्रि॒यस्य॑ प्रि॒यस्य॒ मारु॑तस्य । \newline
39. मारु॑तस्य॒ धाम्नो॒ धाम्नो॒ मारु॑तस्य॒ मारु॑तस्य॒ धाम्नः॑ । \newline
40. धाम्न॒ इति॒ धाम्नः॑ । \newline
41. अव॑ ते॒ ते ऽवाव॑ ते । \newline
42. ते॒ हेडो॒ हेड॑ स्ते ते॒ हेडः॑ । \newline
43. हेड॒ उदु द्धेडो॒ हेड॒ उत् । \newline
44. उदु॑त्त॒म मु॑त्त॒म मुदु दु॑त्त॒मम् । \newline
45. उ॒त्त॒ममित्यु॑त् - त॒मम् । \newline
46. कया॑ नो नः॒ कया॒ कया॑ नः । \newline
47. न॒ श्चि॒त्र श्चि॒त्रो नो॑ न श्चि॒त्रः । \newline
48. चि॒त्र आ चि॒त्र श्चि॒त्र आ । \newline
49. आ भु॑वद् भुव॒दा भु॑वत् । \newline
50. भु॒व॒ दू॒त्यू॑ती भु॑वद् भुव दू॒ती । \newline
51. ऊ॒ती स॒दावृ॑धः स॒दावृ॑ध ऊ॒त्यू॑ती स॒दावृ॑धः । \newline
52. स॒दावृ॑धः॒ सखा॒ सखा॑ स॒दावृ॑धः स॒दावृ॑धः॒ सखा᳚ । \newline
53. स॒दावृ॑ध॒ इति॑ स॒दा - वृ॒धः॒ । \newline
54. सखेति॒ सखा᳚ । \newline
55. कया॒ शचि॑ष्ठया॒ शचि॑ष्ठया॒ कया॒ कया॒ शचि॑ष्ठया । \newline
56. शचि॑ष्ठया वृ॒ता वृ॒ता शचि॑ष्ठया॒ शचि॑ष्ठया वृ॒ता । \newline
57. वृ॒तेति॑ वृ॒ता । \newline

\textbf{Ghana Paata } \newline

1. क्षये॑ पा॒थ पा॒थ क्षये॒ क्षये॑ पा॒था दि॒वो दि॒वः पा॒थ क्षये॒ क्षये॑ पा॒था दि॒वः । \newline
2. पा॒था दि॒वो दि॒वः पा॒थ पा॒था दि॒वो वि॑महसो विमहसो दि॒वः पा॒थ पा॒था दि॒वो वि॑महसः । \newline
3. दि॒वो वि॑महसो विमहसो दि॒वो दि॒वो वि॑महसः । \newline
4. वि॒म॒ह॒स॒ इति॑ वि - म॒ह॒सः॒ । \newline
5. स सु॑गो॒पात॑मः सुगो॒पात॑मः॒ स स सु॑गो॒पात॑मो॒ जनो॒ जनः॑ सुगो॒पात॑मः॒ स स सु॑गो॒पात॑मो॒ जनः॑ । \newline
6. सु॒गो॒पात॑मो॒ जनो॒ जनः॑ सुगो॒पात॑मः सुगो॒पात॑मो॒ जनः॑ । \newline
7. सु॒गो॒पात॑म॒ इति॑ सुगो॒प - त॒मः॒ । \newline
8. जन॒ इति॒ जनः॑ । \newline
9. य॒ज्ञिर् वा॑ वा य॒ज्ञिर् य॒ज्ञिर् वा॑ यज्ञ्वाहसो यज्ञ्वाहसो वा य॒ज्ञिर् य॒ज्ञिर् वा॑ यज्ञ्वाहसः । \newline
10. वा॒ य॒ज्ञ्॒वा॒ह॒सो॒ य॒ज्ञ्॒वा॒ह॒सो॒ वा॒ वा॒ य॒ज्ञ्॒वा॒ह॒सो॒ विप्र॑स्य॒ विप्र॑स्य यज्ञ्वाहसो वा वा यज्ञ्वाहसो॒ विप्र॑स्य । \newline
11. य॒ज्ञ्॒वा॒ह॒सो॒ विप्र॑स्य॒ विप्र॑स्य यज्ञ्वाहसो यज्ञ्वाहसो॒ विप्र॑स्य वा वा॒ विप्र॑स्य यज्ञ्वाहसो यज्ञ्वाहसो॒ विप्र॑स्य वा । \newline
12. य॒ज्ञ्॒वा॒ह॒स॒ इति॑ यज्ञ् - वा॒ह॒सः॒ । \newline
13. विप्र॑स्य वा वा॒ विप्र॑स्य॒ विप्र॑स्य वा मती॒नाम् म॑ती॒नां ॅवा॒ विप्र॑स्य॒ विप्र॑स्य वा मती॒नाम् । \newline
14. वा॒ म॒ती॒नाम् म॑ती॒नां ॅवा॑ वा मती॒नाम् । \newline
15. म॒ती॒नामिति॑ मती॒नाम् । \newline
16. मरु॑तः शृणु॒त शृ॑णु॒त मरु॑तो॒ मरु॑तः शृणु॒ता हवꣳ॒॒ हवꣳ॑ शृणु॒त मरु॑तो॒ मरु॑तः शृणु॒ता हव᳚म् । \newline
17. शृ॒णु॒ता हवꣳ॒॒ हवꣳ॑ शृणु॒त शृ॑णु॒ता हव᳚म् । \newline
18. हव॒मिति॒ हव᳚म् । \newline
19. श्रि॒यसे॒ कम् कꣳ श्रि॒यसे᳚ श्रि॒यसे॒ कम् भा॒नुभि॑र् भा॒नुभिः॒ कꣳ श्रि॒यसे᳚ श्रि॒यसे॒ कम् भा॒नुभिः॑ । \newline
20. कम् भा॒नुभि॑र् भा॒नुभिः॒ कम् कम् भा॒नुभिः॒ सꣳ सम् भा॒नुभिः॒ कम् कम् भा॒नुभिः॒ सम् । \newline
21. भा॒नुभिः॒ सꣳ सम् भा॒नुभि॑र् भा॒नुभिः॒ सम् मि॑मिक्षिरे मिमिक्षिरे॒ सम् भा॒नुभि॑र् भा॒नुभिः॒ सम् मि॑मिक्षिरे । \newline
22. भा॒नुभि॒रिति॑ भा॒नु - भिः॒ । \newline
23. सम् मि॑मिक्षिरे मिमिक्षिरे॒ सꣳ सम् मि॑मिक्षिरे॒ ते ते मि॑मिक्षिरे॒ सꣳ सम् मि॑मिक्षिरे॒ ते । \newline
24. मि॒मि॒क्षि॒रे॒ ते ते मि॑मिक्षिरे मिमिक्षिरे॒ ते र॒श्मिभी॑ र॒श्मिभि॒ स्ते मि॑मिक्षिरे मिमिक्षिरे॒ ते र॒श्मिभिः॑ । \newline
25. ते र॒श्मिभी॑ र॒श्मिभि॒ स्ते ते र॒श्मिभि॒ स्ते ते र॒श्मिभि॒ स्ते ते र॒श्मिभि॒ स्ते । \newline
26. र॒श्मिभि॒ स्ते ते र॒श्मिभी॑ र॒श्मिभि॒ स्त ऋक्व॑भि॒र्॒. ऋक्व॑भि॒ स्ते र॒श्मिभी॑ र॒श्मिभि॒ स्त ऋक्व॑भिः । \newline
27. र॒श्मिभि॒रिति॑ र॒श्मि - भिः॒ । \newline
28. त ऋक्व॑भि॒र्॒. ऋक्व॑भि॒स्ते त ऋक्व॑भिः सुखा॒दयः॑ सुखा॒दय॒ ऋक्व॑भि॒ स्ते त ऋक्व॑भिः सुखा॒दयः॑ । \newline
29. ऋक्व॑भिः सुखा॒दयः॑ सुखा॒दय॒ ऋक्व॑भि॒र्॒. ऋक्व॑भिः सुखा॒दयः॑ । \newline
30. ऋक्व॑भि॒रित्यृक्व॑ - भिः॒ । \newline
31. सु॒खा॒दय॒ इति॑ सु - खा॒दयः॑ । \newline
32. ते वाशी॑मन्तो॒ वाशी॑मन्त॒ स्ते ते वाशी॑मन्त इ॒ष्मिण॑ इ॒ष्मिणो॒ वाशी॑मन्त॒ स्ते ते वाशी॑मन्त इ॒ष्मिणः॑ । \newline
33. वाशी॑मन्त इ॒ष्मिण॑ इ॒ष्मिणो॒ वाशी॑मन्तो॒ वाशी॑मन्त इ॒ष्मिणो॒ अभी॑रवो॒ अभी॑रव इ॒ष्मिणो॒ वाशी॑मन्तो॒ वाशी॑मन्त इ॒ष्मिणो॒ अभी॑रवः । \newline
34. वाशी॑मन्त॒ इति॒ वाशि॑ - म॒न्तः॒ । \newline
35. इ॒ष्मिणो॒ अभी॑रवो॒ अभी॑रव इ॒ष्मिण॑ इ॒ष्मिणो॒ अभी॑रवो वि॒द्रे वि॒द्रे अभी॑रव इ॒ष्मिण॑ इ॒ष्मिणो॒ अभी॑रवो वि॒द्रे । \newline
36. अभी॑रवो वि॒द्रे वि॒द्रे अभी॑रवो॒ अभी॑रवो वि॒द्रे प्रि॒यस्य॑ प्रि॒यस्य॑ वि॒द्रे अभी॑रवो॒ अभी॑रवो वि॒द्रे प्रि॒यस्य॑ । \newline
37. वि॒द्रे प्रि॒यस्य॑ प्रि॒यस्य॑ वि॒द्रे वि॒द्रे प्रि॒यस्य॒ मारु॑तस्य॒ मारु॑तस्य प्रि॒यस्य॑ वि॒द्रे वि॒द्रे प्रि॒यस्य॒ मारु॑तस्य । \newline
38. प्रि॒यस्य॒ मारु॑तस्य॒ मारु॑तस्य प्रि॒यस्य॑ प्रि॒यस्य॒ मारु॑तस्य॒ धाम्नो॒ धाम्नो॒ मारु॑तस्य प्रि॒यस्य॑ प्रि॒यस्य॒ मारु॑तस्य॒ धाम्नः॑ । \newline
39. मारु॑तस्य॒ धाम्नो॒ धाम्नो॒ मारु॑तस्य॒ मारु॑तस्य॒ धाम्नः॑ । \newline
40. धाम्न॒ इति॒ धाम्नः॑ । \newline
41. अव॑ ते॒ ते ऽवाव॑ ते॒ हेडो॒ हेड॒ स्ते ऽवाव॑ ते॒ हेडः॑ । \newline
42. ते॒ हेडो॒ हेड॑ स्ते ते॒ हेड॒ उदु द्धेड॑ स्ते ते॒ हेड॒ उत् । \newline
43. हेड॒ उदु द्धेडो॒ हेड॒ उदु॑त्त॒म मु॑त्त॒म मुद्धेडो॒ हेड॒ उदु॑त्त॒मम् । \newline
44. उदु॑त्त॒म मु॑त्त॒म मुदु दु॑त्त॒मम् । \newline
45. उ॒त्त॒ममित्यु॑त् - त॒मम् । \newline
46. कया॑ नो नः॒ कया॒ कया॑ नश्चि॒त्र श्चि॒त्रो नः॒ कया॒ कया॑ नश्चि॒त्रः । \newline
47. न॒श्चि॒त्र श्चि॒त्रो नो॑ नश्चि॒त्र आ चि॒त्रो नो॑ नश्चि॒त्र आ । \newline
48. चि॒त्र आ चि॒त्र श्चि॒त्र आ भु॑वद् भुव॒दा चि॒त्र श्चि॒त्र आ भु॑वत् । \newline
49. आ भु॑वद् भुव॒दा भु॑व दू॒त्यू॑ती भु॑व॒दा भु॑वदू॒ती । \newline
50. भु॒व॒ दू॒त्यू॑ती भु॑वद् भुवदू॒ती स॒दावृ॑धः स॒दावृ॑ध ऊ॒ती भु॑वद् भुव दू॒ती स॒दावृ॑धः । \newline
51. ऊ॒ती स॒दावृ॑धः स॒दावृ॑ध ऊ॒त्यू॑ती स॒दावृ॑धः॒ सखा॒ सखा॑ स॒दावृ॑ध ऊ॒त्यू॑ती स॒दावृ॑धः॒ सखा᳚ । \newline
52. स॒दावृ॑धः॒ सखा॒ सखा॑ स॒दावृ॑धः स॒दावृ॑धः॒ सखा᳚ । \newline
53. स॒दावृ॑ध॒ इति॑ स॒दा - वृ॒धः॒ । \newline
54. सखेति॒ सखा᳚ । \newline
55. कया॒ शचि॑ष्ठया॒ शचि॑ष्ठया॒ कया॒ कया॒ शचि॑ष्ठया वृ॒ता वृ॒ता शचि॑ष्ठया॒ कया॒ कया॒ शचि॑ष्ठया वृ॒ता । \newline
56. शचि॑ष्ठया वृ॒ता वृ॒ता शचि॑ष्ठया॒ शचि॑ष्ठया वृ॒ता । \newline
57. वृ॒तेति॑ वृ॒ता । \newline
\pagebreak
\markright{ TS 4.2.11.3  \hfill https://www.vedavms.in \hfill}

\section{ TS 4.2.11.3 }

\textbf{TS 4.2.11.3 } \newline
\textbf{Samhita Paata} \newline

को अ॒द्य यु॑ङ्क्ते धु॒रि गा ऋ॒तस्य॒ शिमी॑वतो भा॒मिनो॑ दुर्.हृणा॒यून् । आ॒सन्नि॑षून्. हृ॒थ्स्वसो॑ मयो॒भून्. य ए॑षां भृ॒त्यामृ॒णध॒थ् स जी॑वात् ॥ अग्ने॒ नया > 3, ऽऽदे॒वानाꣳ॒॒ >4, शन्नो॑ भवन्तु॒ > 5, वाजे॑वाजे>6 । अ॒फ्स्व॑ग्ने॒ सधि॒ष्टव॒ सौष॑धी॒रनु॑ रुद्ध्यसे । गर्भे॒ सञ्जा॑यसे॒ पुनः॑ ॥ वृषा॑ सोम द्यु॒माꣳ अ॑सि॒ वृषा॑ देव॒ वृष॑व्रतः । वृषा॒ धर्मा॑णि दधिषे ॥ इ॒मं मे॑ ( ) वरुण॒ > 7, तत्त्वा॑ यामि॒>8, त्वं नो॑ अग्ने॒>9, स त्वं नो॑ अग्ने> 10 ॥ \newline

\textbf{Pada Paata} \newline

कः । अ॒द्य । यु॒ङ्क्ते॒ । धु॒रि । गाः । ऋ॒तस्य॑ । शिमी॑वतः । भा॒मिनः॑ । दु॒र्॒.हृ॒णा॒यूनिति॑ दुः - हृ॒णा॒यून् ॥ आ॒सन्नि॑षू॒नित्या॒सन्न् - इ॒षू॒न् । हृ॒थ्स्वस॒ इति॑ हृथ्सु - असः॑ । म॒यो॒भूनिति॑ मयः - भून् । यः । ए॒षा॒म् । भृ॒त्याम् । ऋ॒णध॑त् । सः । जी॒वा॒त् ॥ अग्ने᳚ । नय॑ । एति॑ । दे॒वाना᳚म् । शम् । नः॒ । भ॒व॒न्तु॒ । वाजे॑वाज॒ इति॒ वाजे᳚ - वा॒जे॒ ॥ अ॒फ्स्वित्य॑प्- सु । अ॒ग्ने॒ । सधिः॑ । तव॑ । सः । ओष॑धीः । अन्विति॑ । रु॒द्ध्य॒से॒ ॥ गर्भे᳚ । सन्न् । जा॒य॒से॒ । पुनः॑ ॥ वृषा᳚ । सो॒म॒ । द्यु॒मानिति॑ द्यु-मान् । अ॒सि॒ । वृषा᳚ । दे॒व॒ । वृष॑व्रत॒ इति॒ वृष॑-व्र॒तः॒ ॥ वृषा᳚ । धर्मा॑णि । द॒धि॒षे॒ ॥ इ॒मम् । मे॒ ( ) । व॒रु॒ण॒ । तत् । त्वा॒ । या॒मि॒ । त्वम् । नः॒ । अ॒ग्ने॒ । सः । त्वम् । नः॒ । अ॒ग्ने॒ ॥  \newline


\textbf{Krama Paata} \newline

को अ॒द्य । अ॒द्य यु॑ङ्क्ते । यु॒ङ्क्ते॒ धु॒रि । धु॒रि गाः । गा ऋ॒तस्य॑ । ऋ॒तस्य॒ शिमी॑वतः । शिमी॑वतो भा॒मिनः॑ । भा॒मिनो॑ दुर्.हृणा॒यून्न् । दु॒र्॒.हृ॒णा॒यूनिति॑ दुः - हृ॒णा॒यून्न् ॥ आ॒सन्नि॑षून् हृ॒थ्स्वसः॑ । आ॒सन्नि॑षू॒नित्या॒सन्न् - इ॒षू॒न्॒ । हृ॒थ्स्वसो॑ मयो॒भून् । हृ॒थ्स्वस॒ इति॑ हृथ्सु - असः॑ । म॒यो॒भून्. यः । म॒यो॒भूनिति॑ मयः - भून् । य ए॑षाम् । ए॒षा॒म् भृ॒त्याम् । भृ॒त्यामृ॒णध॑त् । ऋ॒णध॒थ् सः । स जी॑वात् । जी॒वा॒दिति॑ जीवात् ॥ अग्ने॒ नय॑ । नया । आ दे॒वाना᳚म् । दे॒वानाꣳ॒॒ शम् । शम् नः॑ । नो॒ भ॒व॒न्तु॒ । भ॒व॒न्तु॒ वाजे॑वाजे । वाजे॑वाज॒ इति॒ वाजे᳚ - वा॒जे॒ ॥ अ॒फ्स्व॑ग्ने । अ॒फ्स्वित्य॑प् - सु । अ॒ग्ने॒ सधिः॑ । सधि॒ष्टव॑ । तव॒ सः । सौष॑धीः । ओष॑धी॒रनु॑ । अनु॑ रुद्ध्यसे । रु॒द्ध्य॒स॒ इति॑ रुद्ध्यसे ॥ गर्भे॒ सन्न् । सन् जा॑यसे । जा॒य॒से॒ पुनः॑ । पुन॒रिति॒ पुनः॑ ॥ वृषा॑ सोम । सो॒म॒ द्यु॒मान् । द्यु॒माꣳ अ॑सि । द्यु॒मानिति॑ द्यु - मान् । अ॒सि॒ वृषा᳚ । वृषा॑ देव । दे॒व॒ वृष॑व्रतः । वृष॑व्रत॒ इति॒ वृष॑ - व्र॒तः॒ ॥ वृषा॒ धर्मा॑णि । धर्मा॑णि दधिषे । द॒धि॒ष॒ इति॑ दधिषे ॥ इ॒मम् मे᳚ ( ) । मे॒ व॒रु॒ण॒ । व॒रु॒ण॒ तत् । तत् त्वा᳚ । त्वा॒ या॒मि॒ । या॒मि॒ त्वम् । त्वम् नः॑ । नो॒ अ॒ग्ने॒ । अ॒ग्ने॒ सः । स त्वम् । त्वम् नः॑ । नो॒ अ॒ग्ने॒ । अ॒ग्ने॒ इत्य॑ग्ने । \newline

\textbf{Jatai Paata} \newline

1. को अ॒द्याद्य कः को अ॒द्य । \newline
2. अ॒द्य यु॑ङ्क्ते युङ्क्ते अ॒द्याद्य यु॑ङ्क्ते । \newline
3. यु॒ङ्क्ते॒ धु॒रि धु॒रि यु॑ङ्क्ते युङ्क्ते धु॒रि । \newline
4. धु॒रि गा गा धु॒रि धु॒रि गाः । \newline
5. गा ऋ॒तस्य॒ र्‌तस्य॒ गा गा ऋ॒तस्य॑ । \newline
6. ऋ॒तस्य॒ शिमी॑वतः॒ शिमी॑वत ऋ॒तस्य॒ र्‌तस्य॒ शिमी॑वतः । \newline
7. शिमी॑वतो भा॒मिनो॑ भा॒मिनः॒ शिमी॑वतः॒ शिमी॑वतो भा॒मिनः॑ । \newline
8. भा॒मिनो॑ दुर्.हृणा॒यून् दु॑र्.हृणा॒यून् भा॒मिनो॑ भा॒मिनो॑ दुर्.हृणा॒यून् । \newline
9. दु॒र्॒.हृ॒णा॒यूनिति॑ दुः - हृ॒णा॒यून् । \newline
10. आ॒सन्नि॑षून्. हृ॒थ्स्वसो॑ हृ॒थ्स्वस॑ आ॒सन्नि॑षू ना॒सन्नि॑षून्. हृ॒थ्स्वसः॑ । \newline
11. आ॒सन्नि॑षू॒नित्या॒सन्न् - इ॒षू॒न् । \newline
12. हृ॒थ्स्वसो॑ मयो॒भून् म॑यो॒भून्. हृ॒थ्स्वसो॑ हृ॒थ्स्वसो॑ मयो॒भून् । \newline
13. हृ॒थ्स्वस॒ इति॑ हृथ्सु - असः॑ । \newline
14. म॒यो॒भून्. यो यो म॑यो॒भून् म॑यो॒भून्. यः । \newline
15. म॒यो॒भूनिति॑ मयः - भून् । \newline
16. य ए॑षा मेषां॒ ॅयो य ए॑षाम् । \newline
17. ए॒षा॒म् भृ॒त्याम् भृ॒त्या मे॑षा मेषाम् भृ॒त्याम् । \newline
18. भृ॒त्या मृ॒णध॑ दृ॒णध॑द् भृ॒त्याम् भृ॒त्या मृ॒णध॑त् । \newline
19. ऋ॒णध॒थ् स स ऋ॒णध॑ दृ॒णध॒थ् सः । \newline
20. स जी॑वाज् जीवा॒थ् स स जी॑वात् । \newline
21. जी॒वा॒दिति॑ जीवात् । \newline
22. अग्ने॒ नय॒ नयाग्ने ऽग्ने॒ नय॑ । \newline
23. नया नय॒ नया । \newline
24. आ दे॒वाना᳚म् दे॒वाना॒ मा दे॒वाना᳚म् । \newline
25. दे॒वानाꣳ॒॒ शꣳ शम् दे॒वाना᳚म् दे॒वानाꣳ॒॒ शम् । \newline
26. शन्नो॑ नः॒ शꣳ शन्नः॑ । \newline
27. नो॒ भ॒व॒न्तु॒ भ॒व॒न्तु॒ नो॒ नो॒ भ॒व॒न्तु॒ । \newline
28. भ॒व॒न्तु॒ वाजे॑वाजे॒ वाजे॑वाजे भवन्तु भवन्तु॒ वाजे॑वाजे । \newline
29. वाजे॑वाज॒ इति॒ वाजे᳚ - वा॒जे॒ । \newline
30. अ॒फ्स्व॑ग्ने अग्ने अ॒फ्स्वा᳚(1॒)फ्स्व॑ग्ने । \newline
31. अ॒फ्स्वित्य॑प् - सु । \newline
32. अ॒ग्ने॒ सधिः॒ सधि॑ रग्ने अग्ने॒ सधिः॑ । \newline
33. सधि॒ष्टव॒ तव॒ सधिः॒ सधि॒ष्टव॑ । \newline
34. तव॒ स स तव॒ तव॒ सः । \newline
35. सौष॑धी॒ रोष॑धीः॒ स सौष॑धीः । \newline
36. ओष॑धी॒ रन्वन् वोष॑धी॒ रोष॑धी॒ रनु॑ । \newline
37. अनु॑ रुद्ध्यसे रुद्ध्यसे॒ अन्वनु॑ रुद्ध्यसे । \newline
38. रु॒ध्य॒स॒ इति॑ रुध्यसे । \newline
39. गर्भे॒ सन् थ्सन् गर्भे॒ गर्भे॒ सन्न् । \newline
40. सन् जा॑यसे जायसे॒ सन् थ्सन् जा॑यसे । \newline
41. जा॒य॒से॒ पुनः॒ पुन॑र् जायसे जायसे॒ पुनः॑ । \newline
42. पुन॒रिति॒ पुनः॑ । \newline
43. वृषा॑ सोम सोम॒ वृषा॒ वृषा॑ सोम । \newline
44. सो॒म॒ द्यु॒मान् द्यु॒मान् थ्सो॑म सोम द्यु॒मान् । \newline
45. द्यु॒माꣳ अ॑स्यसि द्यु॒मान् द्यु॒माꣳ अ॑सि । \newline
46. द्यु॒मानिति॑ द्यु - मान् । \newline
47. अ॒सि॒ वृषा॒ वृषा᳚ ऽस्यसि॒ वृषा᳚ । \newline
48. वृषा॑ देव देव॒ वृषा॒ वृषा॑ देव । \newline
49. दे॒व॒ वृष॑व्रतो॒ वृष॑व्रतो देव देव॒ वृष॑व्रतः । \newline
50. वृष॑व्रत॒ इति॒ वृष॑ - व्र॒तः॒ । \newline
51. वृषा॒ धर्मा॑णि॒ धर्मा॑णि॒ वृषा॒ वृषा॒ धर्मा॑णि । \newline
52. धर्मा॑णि दधिषे दधिषे॒ धर्मा॑णि॒ धर्मा॑णि दधिषे । \newline
53. द॒धि॒ष॒ इति॑ दधिषे । \newline
54. इ॒मम् मे॑ म इ॒म मि॒मम् मे᳚ । \newline
55. मे॒ व॒रु॒ण॒ व॒रु॒ण॒ मे॒ मे॒ व॒रु॒ण॒ । \newline
56. व॒रु॒ण॒ तत् तद् व॑रुण वरुण॒ तत् । \newline
57. तत् त्वा᳚ त्वा॒ तत् तत् त्वा᳚ । \newline
58. त्वा॒ या॒मि॒ या॒मि॒ त्वा॒ त्वा॒ या॒मि॒ । \newline
59. या॒मि॒ त्वम् त्वं ॅया॑मि यामि॒ त्वम् । \newline
60. त्वम् नो॑ न॒ स्त्वम् त्वम् नः॑ । \newline
61. नो॒ अ॒ग्ने॒ अ॒ग्ने॒ नो॒ नो॒ अ॒ग्ने॒ । \newline
62. अ॒ग्ने॒ स सो अ॑ग्ने अग्ने॒ सः । \newline
63. स त्वम् त्वꣳ स स त्वम् । \newline
64. त्वम् नो॑ न॒ स्त्वम् त्वम् नः॑ । \newline
65. नो॒ अ॒ग्ने॒ अ॒ग्ने॒ नो॒ नो॒ अ॒ग्ने॒ । \newline
66. अ॒ग्ने॒ इत्य॑ग्ने । \newline

\textbf{Ghana Paata } \newline

1. को अ॒द्याद्य कः को अ॒द्य यु॑ङ्क्ते युङ्क्ते अ॒द्य कः को अ॒द्य यु॑ङ्क्ते । \newline
2. अ॒द्य यु॑ङ्क्ते युङ्क्ते अ॒द्याद्य यु॑ङ्क्ते धु॒रि धु॒रि यु॑ङ्क्ते अ॒द्याद्य यु॑ङ्क्ते धु॒रि । \newline
3. यु॒ङ्क्ते॒ धु॒रि धु॒रि यु॑ङ्क्ते युङ्क्ते धु॒रि गा गा धु॒रि यु॑ङ्क्ते युङ्क्ते धु॒रि गाः । \newline
4. धु॒रि गा गा धु॒रि धु॒रि गा ऋ॒तस्य॒ र्‌तस्य॒ गा धु॒रि धु॒रि गा ऋ॒तस्य॑ । \newline
5. गा ऋ॒तस्य॒ र्‌तस्य॒ गा गा ऋ॒तस्य॒ शिमी॑वतः॒ शिमी॑वत ऋ॒तस्य॒ गा गा ऋ॒तस्य॒ शिमी॑वतः । \newline
6. ऋ॒तस्य॒ शिमी॑वतः॒ शिमी॑वत ऋ॒तस्य॒ र्‌तस्य॒ शिमी॑वतो भा॒मिनो॑ भा॒मिनः॒ शिमी॑वत ऋ॒तस्य॒ र्‌तस्य॒ शिमी॑वतो भा॒मिनः॑ । \newline
7. शिमी॑वतो भा॒मिनो॑ भा॒मिनः॒ शिमी॑वतः॒ शिमी॑वतो भा॒मिनो॑ दुर्.हृणा॒यून् दु॑र्.हृणा॒यून् भा॒मिनः॒ शिमी॑वतः॒ शिमी॑वतो भा॒मिनो॑ दुर्.हृणा॒यून् । \newline
8. भा॒मिनो॑ दुर्.हृणा॒यून् दु॑र्.हृणा॒यून् भा॒मिनो॑ भा॒मिनो॑ दुर्.हृणा॒यून् । \newline
9. दु॒र्॒.हृ॒णा॒यूनिति॑ दुः - हृ॒णा॒यून् । \newline
10. आ॒सन्नि॑षून्. हृ॒थ्स्वसो॑ हृ॒थ्स्वस॑ आ॒सन्नि॑षू ना॒सन्नि॑षून्. हृ॒थ्स्वसो॑ मयो॒भून् म॑यो॒भून्. हृ॒थ्स्वस॑ आ॒सन्नि॑षू ना॒सन्नि॑षून्. हृ॒थ्स्वसो॑ मयो॒भून् । \newline
11. आ॒सन्नि॑षू॒नित्या॒सन्न् - इ॒षू॒न् । \newline
12. हृ॒थ्स्वसो॑ मयो॒भून् म॑यो॒भून्. हृ॒थ्स्वसो॑ हृ॒थ्स्वसो॑ मयो॒भून्. यो यो म॑यो॒भून्. हृ॒थ्स्वसो॑ हृ॒थ्स्वसो॑ मयो॒भून्. यः । \newline
13. हृ॒थ्स्वस॒ इति॑ हृथ्सु - असः॑ । \newline
14. म॒यो॒भून्. यो यो म॑यो॒भून् म॑यो॒भून्. य ए॑षा मेषां॒ ॅयो म॑यो॒भून् म॑यो॒भून्. य ए॑षाम् । \newline
15. म॒यो॒भूनिति॑ मयः - भून् । \newline
16. य ए॑षा मेषां॒ ॅयो य ए॑षाम् भृ॒त्याम् भृ॒त्या मे॑षां॒ ॅयो य ए॑षाम् भृ॒त्याम् । \newline
17. ए॒षा॒म् भृ॒त्याम् भृ॒त्या मे॑षा मेषाम् भृ॒त्या मृ॒णध॑दृ॒णध॑द् भृ॒त्या मे॑षा मेषाम् भृ॒त्या मृ॒णध॑त् । \newline
18. भृ॒त्या मृ॒णध॑दृ॒णध॑द् भृ॒त्याम् भृ॒त्या मृ॒णध॒थ् स स ऋ॒णध॑द् भृ॒त्याम् भृ॒त्या मृ॒णध॒थ् सः । \newline
19. ऋ॒णध॒थ् स स ऋ॒णध॑दृ॒णध॒थ् स जी॑वाज् जीवा॒थ् स ऋ॒णध॑दृ॒णध॒थ् स जी॑वात् । \newline
20. स जी॑वाज् जीवा॒थ् स स जी॑वात् । \newline
21. जी॒वा॒दिति॑ जीवात् । \newline
22. अग्ने॒ नय॒ नयाग्ने ऽग्ने॒ नया नयाग्ने ऽग्ने॒ नया । \newline
23. नया नय॒ नया दे॒वाना᳚म् दे॒वाना॒ मा नय॒ नया दे॒वाना᳚म् । \newline
24. आ दे॒वाना᳚म् दे॒वाना॒ मा दे॒वानाꣳ॒॒ शꣳ शम् दे॒वाना॒ मा दे॒वानाꣳ॒॒ शम् । \newline
25. दे॒वानाꣳ॒॒ शꣳ शम् दे॒वाना᳚म् दे॒वानाꣳ॒॒ शन्नो॑ नः॒ शम् दे॒वाना᳚म् दे॒वानाꣳ॒॒ शन्नः॑ । \newline
26. शन्नो॑ नः॒ शꣳ शन्नो॑ भवन्तु भवन्तु नः॒ शꣳ शन्नो॑ भवन्तु । \newline
27. नो॒ भ॒व॒न्तु॒ भ॒व॒न्तु॒ नो॒ नो॒ भ॒व॒न्तु॒ वाजे॑वाजे॒ वाजे॑वाजे भवन्तु नो नो भवन्तु॒ वाजे॑वाजे । \newline
28. भ॒व॒न्तु॒ वाजे॑वाजे॒ वाजे॑वाजे भवन्तु भवन्तु॒ वाजे॑वाजे । \newline
29. वाजे॑वाज॒ इति॒ वाजे᳚ - वा॒जे॒ । \newline
30. अ॒फ्स्व॑ग्ने अग्ने अ॒फ्स्वा᳚(1॒)फ्स्व॑ग्ने॒ सधिः॒ सधि॑रग्ने अ॒फ्स्वा᳚(1॒)फ्स्व॑ग्ने॒ सधिः॑ । \newline
31. अ॒फ्स्वित्य॑प् - सु । \newline
32. अ॒ग्ने॒ सधिः॒ सधि॑ रग्ने अग्ने॒ सधि॒ष् टव॒ तव॒ सधि॑ रग्ने अग्ने॒ सधि॒ष् टव॑ । \newline
33. सधि॒ष् टव॒ तव॒ सधिः॒ सधि॒ष् टव॒ स स तव॒ सधिः॒ सधि॒ष् टव॒ सः । \newline
34. तव॒ स स तव॒ तव॒ सौष॑धी॒ रोष॑धीः॒ स तव॒ तव॒ सौष॑धीः । \newline
35. सौष॑धी॒ रोष॑धीः॒ स सौष॑धी॒ रन्वन् वोष॑धीः॒ स सौष॑धी॒ रनु॑ । \newline
36. ओष॑धी॒ रन्वन्वोष॑धी॒ रोष॑धी॒ रनु॑ रुद्ध्यसे रुद्ध्यसे॒ अन्वोष॑धी॒ रोष॑धी॒ रनु॑ रुद्ध्यसे । \newline
37. अनु॑ रुद्ध्यसे रुद्ध्यसे॒ अन्वनु॑ रुद्ध्यसे । \newline
38. रु॒ध्य॒स॒ इति॑ रुध्यसे । \newline
39. गर्भे॒ सन् थ्सन् गर्भे॒ गर्भे॒ सन् जा॑यसे जायसे॒ सन् गर्भे॒ गर्भे॒ सन् जा॑यसे । \newline
40. सन् जा॑यसे जायसे॒ सन् थ्सन् जा॑यसे॒ पुनः॒ पुन॑र् जायसे॒ सन् थ्सन् जा॑यसे॒ पुनः॑ । \newline
41. जा॒य॒से॒ पुनः॒ पुन॑र् जायसे जायसे॒ पुनः॑ । \newline
42. पुन॒रिति॒ पुनः॑ । \newline
43. वृषा॑ सोम सोम॒ वृषा॒ वृषा॑ सोम द्यु॒मान् द्यु॒मान् थ्सो॑म॒ वृषा॒ वृषा॑ सोम द्यु॒मान् । \newline
44. सो॒म॒ द्यु॒मान् द्यु॒मान् थ्सो॑म सोम द्यु॒माꣳ अ॑स्यसि द्यु॒मान् थ्सो॑म सोम द्यु॒माꣳ अ॑सि । \newline
45. द्यु॒माꣳ अ॑स्यसि द्यु॒मान् द्यु॒माꣳ अ॑सि॒ वृषा॒ वृषा॑ ऽसि द्यु॒मान् द्यु॒माꣳ अ॑सि॒ वृषा᳚ । \newline
46. द्यु॒मानिति॑ द्यु - मान् । \newline
47. अ॒सि॒ वृषा॒ वृषा᳚ ऽस्यसि॒ वृषा॑ देव देव॒ वृषा᳚ ऽस्यसि॒ वृषा॑ देव । \newline
48. वृषा॑ देव देव॒ वृषा॒ वृषा॑ देव॒ वृष॑व्रतो॒ वृष॑व्रतो देव॒ वृषा॒ वृषा॑ देव॒ वृष॑व्रतः । \newline
49. दे॒व॒ वृष॑व्रतो॒ वृष॑व्रतो देव देव॒ वृष॑व्रतः । \newline
50. वृष॑व्रत॒ इति॒ वृष॑ - व्र॒तः॒ । \newline
51. वृषा॒ धर्मा॑णि॒ धर्मा॑णि॒ वृषा॒ वृषा॒ धर्मा॑णि दधिषे दधिषे॒ धर्मा॑णि॒ वृषा॒ वृषा॒ धर्मा॑णि दधिषे । \newline
52. धर्मा॑णि दधिषे दधिषे॒ धर्मा॑णि॒ धर्मा॑णि दधिषे । \newline
53. द॒धि॒ष॒ इति॑ दधिषे । \newline
54. इ॒मम् मे॑ म इ॒म मि॒मम् मे॑ वरुण वरुण म इ॒म मि॒मम् मे॑ वरुण । \newline
55. मे॒ व॒रु॒ण॒ व॒रु॒ण॒ मे॒ मे॒ व॒रु॒ण॒ तत् तद् व॑रुण मे मे वरुण॒ तत् । \newline
56. व॒रु॒ण॒ तत् तद् व॑रुण वरुण॒ तत् त्वा᳚ त्वा॒ तद् व॑रुण वरुण॒ तत् त्वा᳚ । \newline
57. तत् त्वा᳚ त्वा॒ तत् तत् त्वा॑ यामि यामि त्वा॒ तत् तत् त्वा॑ यामि । \newline
58. त्वा॒ या॒मि॒ या॒मि॒ त्वा॒ त्वा॒ या॒मि॒ त्वम् त्वं ॅया॑मि त्वा त्वा यामि॒ त्वम् । \newline
59. या॒मि॒ त्वम् त्वं ॅया॑मि यामि॒ त्वन्नो॑ न॒स्त्वं ॅया॑मि यामि॒ त्वन्नः॑ । \newline
60. त्वन्नो॑ न॒ स्त्वम् त्वन्नो॑ अग्ने अग्ने न॒स्त्वम् त्वन्नो॑ अग्ने । \newline
61. नो॒ अ॒ग्ने॒ अ॒ग्ने॒ नो॒ नो॒ अ॒ग्ने॒ स सो अ॑ग्ने नो नो अग्ने॒ सः । \newline
62. अ॒ग्ने॒ स सो अ॑ग्ने अग्ने॒ स त्वम् त्वꣳ सो अ॑ग्ने अग्ने॒ स त्वम् । \newline
63. स त्वम् त्वꣳ स स त्वन्नो॑ न॒ स्त्वꣳ स स त्वन्नः॑ । \newline
64. त्वन्नो॑ न॒ स्त्वम् त्वन्नो॑ अग्ने अग्ने न॒ स्त्वम् त्वन्नो॑ अग्ने । \newline
65. नो॒ अ॒ग्ने॒ अ॒ग्ने॒ नो॒ नो॒ अ॒ग्ने॒ । \newline
66. अ॒ग्ने॒ इत्य॑ग्ने । \newline
\pagebreak


\end{document}