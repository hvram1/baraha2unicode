\documentclass[17pt]{extarticle}
\usepackage{babel}
\usepackage{fontspec}
\usepackage{polyglossia}
\usepackage{extsizes}



\setmainlanguage{sanskrit}
\setotherlanguages{english} %% or other languages
\setlength{\parindent}{0pt}
\pagestyle{myheadings}
\newfontfamily\devanagarifont[Script=Devanagari]{AdishilaVedic}


\newcommand{\VAR}[1]{}
\newcommand{\BLOCK}[1]{}




\begin{document}
\begin{titlepage}
    \begin{center}
 
\begin{sanskrit}
    { \Huge
    कृष्ण यजुर्वेदीय तैत्तिरीय संहिता,पद,जटा,घन पाठः 
    }
    \\
    \vspace{2.5cm}
    \mbox{ \Huge
    4.2      चतुर्थकाण्डे द्वीतीयः प्रश्नः - देवयजनग्रहाभिधानं   }
\end{sanskrit}
\end{center}

\end{titlepage}
\tableofcontents
\pagebreak

\markright{ TS 4.2.1.1  \hfill https://www.vedavms.in \hfill}
\addcontentsline{toc}{section}{ TS 4.2.1.1 }
\section*{ TS 4.2.1.1 }

\textbf{TS 4.2.1.1 } \newline
\textbf{Samhita Paata} \newline

विष्णोः॒ क्रमो᳚ऽस्यभिमाति॒हा गा॑य॒त्रं छन्द॒ आ रो॑ह पृथि॒वीमनु॒ विक्र॑मस्व॒ निर्भ॑क्तः॒ स यं द्वि॒ष्मो विष्णोः॒ क्रमो᳚ऽस्यभिशस्ति॒हा त्रैष्टु॑भं॒ छन्द॒ आ रो॑हा॒न्तरि॑क्ष॒मनु॒ विक्र॑मस्व॒ निर्भ॑क्तः॒ स यं द्वि॒ष्मो विष्णोः॒ क्रमो᳚ऽस्यरातीय॒तो ह॒न्ता जाग॑तं॒ छन्द॒ आ रो॑ह॒ दिव॒मनु॒ विक्र॑मस्व॒ निर्भ॑क्तः॒ स यं द्वि॒ष्मो विष्णोः॒ - [  ] \newline

\textbf{Pada Paata} \newline

विष्णोः᳚ । क्रमः॑ । अ॒सि॒ । अ॒भि॒मा॒ति॒हेत्य॑भिमाति - हा । गा॒य॒त्रम् । छन्दः॑ । एति॑ । रो॒ह॒ । पृ॒थि॒वीम् । अनु॑ । वीति॑ । क्र॒म॒स्व॒ । निर्भ॑क्त॒ इति॒ निः - भ॒क्तः॒ । सः । यम् । द्वि॒ष्मः । विष्णोः᳚ । क्रमः॑ । अ॒सि॒ । अ॒भि॒श॒स्ति॒हेत्य॑भिशस्ति - हा । त्रैष्टु॑भम् । छन्दः॑ । एति॑ । रो॒ह॒ । अ॒न्तरि॑क्षम् । अनु॑ । वीति॑ । क्र॒म॒स्व॒ । निर्भ॑क्त॒ इति॒ निः - भ॒क्तः॒ । सः । यम् । द्वि॒ष्मः । विष्णोः᳚ । क्रमः॑ । अ॒सि॒ । अ॒रा॒ती॒य॒तः । ह॒न्ता । जाग॑तम् । छन्दः॑ । एति॑ । रो॒ह॒ । दिव᳚म् । अनु॑ । वीति॑ । क्र॒म॒स्व॒ । निर्भ॑क्त॒ इति॒ निः - भ॒क्तः॒ । सः । यम् । द्वि॒ष्मः । विष्णोः᳚ ।  \newline




\markright{ TS 4.2.1.2  \hfill https://www.vedavms.in \hfill}
\addcontentsline{toc}{section}{ TS 4.2.1.2 }
\section*{ TS 4.2.1.2 }

\textbf{TS 4.2.1.2 } \newline
\textbf{Samhita Paata} \newline

क्रमो॑ऽसि शत्रूय॒तो ह॒न्ताऽनु॑ष्टुभं॒ छन्द॒ आ रो॑ह॒ दिशोऽनु॒ विक्र॑मस्व॒ निर्भ॑क्तः॒ स यं द्वि॒ष्मः ॥ अक्र॑न्दद॒ग्निः स्त॒नय॑न्निव॒ द्यौः क्षामा॒ रेरि॑हद्-वी॒रुधः॑ सम॒ञ्जन्न् । स॒द्यो ज॑ज्ञा॒नो वि हीमि॒द्धो अख्य॒दा रोद॑सी भा॒नुना॑ भात्य॒न्तः ॥ अग्ने᳚ऽभ्यावर्तिन्न॒भि न॒ आ व॑र्त॒स्वाऽऽयु॑षा॒ वर्च॑सा स॒न्या मे॒धया᳚ प्र॒जया॒ धने॑न ॥ अग्ने॑ - [  ] \newline

\textbf{Pada Paata} \newline

क्रमः॑ । अ॒सि॒ । श॒त्रू॒य॒त इति॑ शत्रु - य॒तः । ह॒न्ता । आनु॑ष्टुभ॒मित्यानु॑ - स्तु॒भ॒म् । छन्दः॑ । एति॑ । रो॒ह॒ । दिशः॑ । अनु॑ । वीति॑ । क्र॒म॒स्व॒ । निर्भ॑क्त॒ इति॒ निः - भ॒क्तः॒ । सः । यम् । द्वि॒ष्मः ॥ अक्र॑न्दत् । अ॒ग्निः । स्त॒नयन्न्॑ । इ॒व॒ । द्यौः । क्षाम॑ । रेरि॑हत् । वी॒रुधः॑ । स॒म॒ञ्जन्निति॑ सं - अ॒ञ्जन्न् ॥ स॒द्यः । ज॒ज्ञा॒नः । वीति॑ । हि । ई॒म् । इ॒द्धः । अख्य॑त् । एति॑ । रोद॑सी॒ इति॑ । भा॒नुना᳚ । भा॒ति॒ । अ॒न्तः ॥ अग्ने᳚ । अ॒भ्या॒व॒र्ति॒न्नित्य॑भि - आ॒व॒र्ति॒न्न् । अ॒भीति॑ । नः॒ । एति॑ । व॒र्त॒स्व॒ । आयु॑षा । वर्च॑सा । स॒न्या । मे॒धया᳚ । प्र॒जयेति॑ प्र - जया᳚ । धने॑न ॥ अग्ने᳚ ।  \newline




\markright{ TS 4.2.1.3  \hfill https://www.vedavms.in \hfill}
\addcontentsline{toc}{section}{ TS 4.2.1.3 }
\section*{ TS 4.2.1.3 }

\textbf{TS 4.2.1.3 } \newline
\textbf{Samhita Paata} \newline

अङ्गिरः श॒तं ते॑ सन्त्वा॒वृतः॑ स॒हस्रं॑ त उपा॒वृतः॑ । तासां॒ पोष॑स्य॒ पोषे॑ण॒ पुन॑र्नो न॒ष्टमा कृ॑धि॒ पुन॑र्नो र॒यिमा कृ॑धि ॥ पुन॑रू॒र्जा निव॑र्तस्व॒ पुन॑रग्न इ॒षाऽऽयु॑षा । पुन॑र्नः पाहि वि॒श्वतः॑ ॥ स॒ह र॒य्या नि व॑र्त॒स्वाग्ने॒ पिन्व॑स्व॒ धार॑या । वि॒श्वफ्स्नि॑या वि॒श्व त॒स्परि॑ ॥ उदु॑त्त॒मं ॅव॑रुण॒ पाश॑ म॒स्मदवा॑ध॒मं - [  ] \newline

\textbf{Pada Paata} \newline

अ॒ङ्गि॒रः॒ । श॒तम् । ते॒ । स॒न्तु॒ । आ॒वृत॒ इत्या᳚-वृतः॑ । स॒हस्र᳚म् । ते॒ । उ॒पा॒वृत॒ इत्यु॑प-आ॒वृतः॑ ॥ तासा᳚म् । पोष॑स्य । पोषे॑ण । पुनः॑ । नः॒ । न॒ष्टम् । एति॑ । कृ॒धि॒ । पुनः॑ । नः॒ । र॒यिम् । एति॑ । कृ॒धि॒ ॥ पुनः॑ । ऊ॒र्जा । नीति॑ । व॒र्त॒स्व॒ । पुनः॑ । अ॒ग्ने॒ । इ॒षा । आयु॑षा ॥ पुनः॑ । नः॒ । पा॒हि॒ । वि॒श्वतः॑ ॥ स॒ह । र॒य्या । नीति॑ । व॒र्त॒स्व॒ । अग्ने᳚ । पिन्व॑स्व । धार॑या ॥ वि॒श्वफ्स्नि॒येति॑ वि॒श्व-फ्स्नि॒या॒ । वि॒श्वतः॑ । परि॑ ॥ उदिति॑ । उ॒त्त॒ममित्यु॑त्-त॒मम् । व॒रु॒ण॒ । पाश᳚म् । अ॒स्मत् । अवेति॑ । अ॒ध॒मम् ।  \newline




\markright{ TS 4.2.1.4  \hfill https://www.vedavms.in \hfill}
\addcontentsline{toc}{section}{ TS 4.2.1.4 }
\section*{ TS 4.2.1.4 }

\textbf{TS 4.2.1.4 } \newline
\textbf{Samhita Paata} \newline

ॅवि म॑द्ध्य॒मꣳ श्र॑थाय । अथा॑ व॒यमा॑दित्य व्र॒ते तवाना॑गसो॒ अदि॑तये स्याम ॥ आ त्वा॑ऽहार्.ष-म॒न्तर॑भूर्द्ध्रु॒वस्ति॒ष्ठा ऽवि॑चाचलिः । विश॑स्त्वा॒ सर्वा॑ वाञ्छन्त्व॒स्मिन् रा॒ष्ट्रमधि॑ श्रय ॥अग्ने॑ बृ॒हन्नु॒षसा॑मू॒र्द्ध्वो अ॑स्थान्निर्जग्मि॒वान्-तम॑सो॒ ज्योति॒षाऽऽगा᳚त् । अ॒ग्निर्भा॒नुना॒ रुश॑ता॒ स्वङ्ग॒ आ जा॒तो विश्वा॒ सद्मा᳚न्यप्राः ॥ सीद॒ त्वं मा॒तुर॒स्या - [  ] \newline

\textbf{Pada Paata} \newline

वीति॑ । म॒द्ध्य॒मम् । श्र॒था॒य॒ ॥ अथ॑ । व॒यम् । आ॒दि॒त्य॒ । व्र॒ते । तव॑ । अना॑गसः । अदि॑तये । स्या॒म॒ ॥ एति॑ । त्वा॒ । अ॒हा॒र्॒.ष॒म् । अ॒न्तः । अ॒भूः॒ । ध्रु॒वः । ति॒ष्ठ॒ । अवि॑चाचलि॒रित्यवि॑ - चा॒च॒लिः॒ ॥ विशः॑ । त्वा॒ । सर्वाः᳚ । वा॒ञ्छ॒न्तु॒ । अ॒स्मिन्न् । रा॒ष्ट्रम् । अधीति॑ । श्र॒य॒ ॥ अग्ने᳚ । बृ॒हन्न् । उ॒षसा᳚म् । ऊ॒द्‌र्ध्वः । अ॒स्था॒त् । नि॒र्ज॒ग्मि॒वानिति॑ निः - ज॒ग्मि॒वान् । तम॑सः । ज्योति॑षा । एति॑ । अ॒गा॒त् ॥ अ॒ग्निः । भा॒नुना᳚ । रुश॑ता । स्वङ्ग॒ इति॑ सु - अङ्गः॑ । एति॑ । जा॒तः । विश्वा᳚ । सद्मा॑नि । अ॒प्राः॒ ॥ सीद॑ । त्वम् । मा॒तुः । अ॒स्याः ।  \newline




\markright{ TS 4.2.1.5  \hfill https://www.vedavms.in \hfill}
\addcontentsline{toc}{section}{ TS 4.2.1.5 }
\section*{ TS 4.2.1.5 }

\textbf{TS 4.2.1.5 } \newline
\textbf{Samhita Paata} \newline

उ॒पस्थे॒ विश्वा᳚न्यग्ने व॒युना॑नि वि॒द्वान् । मैना॑म॒र्चिषा॒ मा तप॑सा॒ऽभि शू॑शुचो॒ऽन्तर॑स्याꣳ शु॒क्रज्यो॑ति॒र्वि भा॑हि ॥ अ॒न्तर॑ग्ने रु॒चा त्वमु॒खायै॒ सद॑ने॒ स्वे । तस्या॒स्त्वꣳ हर॑सा॒ तप॒ञ्जात॑वेदः शि॒वो भ॑व ॥ शि॒वो भू॒त्वा मह्य॑म॒ग्नेऽथो॑ सीद शि॒वस्त्वं । शि॒वाः कृ॒त्वा दिशः॒ सर्वाः॒ स्वां ॅयोनि॑मि॒हाऽऽ स॑दः ॥ हꣳ॒॒सः शु॑चि॒ष ( ) द्वसु॑रन्तरिक्ष॒-सद्धोता॑ वेदि॒षदति॑थि र्दुरोण॒सत् । नृ॒षद्व॑र॒सद् ऋ॑त॒सद् व्यो॑म॒सद् अ॒ब्जा गो॒जा ऋ॑त॒जा अ॑द्रि॒जा ऋ॒तं बृ॒हत् ॥ \newline

\textbf{Pada Paata} \newline

उ॒पस्थ॒ इत्यु॒प - स्थे॒ । विश्वा॑नि । अ॒ग्ने॒ । व॒युना॑नि । वि॒द्वान् ॥ मा । ए॒ना॒म् । अ॒र्चिषा᳚ । मा । तप॑सा । अ॒भीति॑ । शू॒शु॒चः॒ । अ॒न्तः । अ॒स्या॒म् । शु॒क्रज्यो॑ति॒रिति॑ शु॒क्र-ज्यो॒तिः॒ । वीति॑ । भा॒हि॒ ॥ अ॒न्तः । अ॒ग्ने॒ । रु॒चा । त्वम् । उ॒खायै᳚ । सद॑ने । स्वे ॥ तस्याः᳚ । त्वम् । हर॑सा । तपन्न्॑ । जात॑वेद॒ इति॒ जात॑-वे॒दः॒ । शि॒वः । भ॒व॒ ॥ शि॒वः । भू॒त्वा । मह्य᳚म् । अ॒ग्ने॒ । अथो॒ इति॑ । सी॒द॒ । शि॒वः । त्वम् ॥ शि॒वाः । कृ॒त्वा । दिशः॑ । सर्वाः᳚ । स्वाम् । योनि᳚म् । इ॒ह । एति॑ । अ॒स॒दः॒ ॥ हꣳ॒॒सः । शु॒चि॒षदिति॑ शुचि - सत् ( ) । वसुः॑ । अ॒न्त॒रि॒क्ष॒सदित्य॑न्तरिक्ष - सत् । होता᳚ । वे॒दि॒षदिति॑ वेदि - सत् । अति॑थिः । दु॒रो॒ण॒सदिति॑ दुरोण - सत् ॥ नृ॒षदिति॑ नृ - सत् । व॒र॒सदिति॑ वर-सत् । ऋ॒त॒सदित्यृ॑त-सत् । व्यो॒म॒सदिति॑ व्योम-सत् । अ॒ब्जा इत्य॑प्-जाः । गो॒जा इति॑ गो-जाः । ऋ॒त॒जा इत्यृ॑त - जाः । अ॒द्रि॒जा इत्य॑द्रि - जाः । ऋ॒तम् । बृ॒हत् ॥  \newline




\markright{ TS 4.2.2.1  \hfill https://www.vedavms.in \hfill}
\addcontentsline{toc}{section}{ TS 4.2.2.1 }
\section*{ TS 4.2.2.1 }

\textbf{TS 4.2.2.1 } \newline
\textbf{Samhita Paata} \newline

दि॒वस्परि॑ प्रथ॒मं ज॑ज्ञे अ॒ग्निर॒स्मद् द्वि॒तीयं॒ परि॑ जा॒तवे॑दाः । तृ॒तीय॑म॒फ्सु नृ॒मणा॒ अज॑स्र॒मिन्धा॑न एनं जरते स्वा॒धीः ॥ वि॒द्मा ते॑ अग्ने त्रे॒धा त्र॒याणि॑ वि॒द्मा ते॒ सद्म॒ विभृ॑तं पुरु॒त्रा । वि॒द्मा ते॒ नाम॑ पर॒मं गुहा॒ यद्वि॒द्मा तमुथ्सं॒ ॅयत॑ आज॒गन्थ॑ ॥ स॒मु॒द्रे त्वा॑ नृ॒मणा॑ अ॒फ्स्व॑न्तर्नृ॒चक्षा॑ ईधे दि॒वो अ॑ग्न॒ ऊधन्न्॑ । तृ॒तीये᳚ त्वा॒ - [  ] \newline

\textbf{Pada Paata} \newline

दि॒वः । परीति॑ । प्र॒थ॒मम् । ज॒ज्ञे॒ । अ॒ग्निः । अ॒स्मत् । द्वि॒तीय᳚म् । परीति॑ । जा॒तवे॑दा॒ इति॑ जा॒त - वे॒दाः॒ ॥ तृ॒तीय᳚म् । अ॒फ्स्वित्य॑प्-सु । नृ॒मणा॒ इति॑ नृ - मनाः᳚ । अज॑स्रम् । इन्धा॑नः । ए॒न॒म् । ज॒र॒ते॒ । स्वा॒धीरिति॑ स्वा-धीः ॥ वि॒द्म । ते॒ । अ॒ग्ने॒ । त्रे॒धा । त्र॒याणि॑ । वि॒द्म । ते॒ । सद्म॑ । विभृ॑त॒मिति॒ वि - भृ॒त॒म् । पु॒रु॒त्रेति॑ पुरु - त्रा ॥ वि॒द्म । ते॒ । नाम॑ । प॒र॒मम् । गुहा᳚ । यत् । वि॒द्म । तम् । उथ्स᳚म् । यतः॑ । आ॒ज॒गन्थेत्या᳚ - ज॒गन्थ॑ ॥ स॒मु॒द्रे । त्वा॒ । नृ॒मणा॒ इति॑ नृ - मनाः᳚ । अ॒फ्स्वित्य॑प्- सु । अ॒न्तः । नृ॒चक्षा॒ इति॑ नृ-चक्षाः᳚ । ई॒धे॒ । दि॒वः । अ॒ग्ने॒ । ऊधन्न्॑ ॥ तृ॒तीये᳚ । त्वा॒ ।  \newline




\markright{ TS 4.2.2.2  \hfill https://www.vedavms.in \hfill}
\addcontentsline{toc}{section}{ TS 4.2.2.2 }
\section*{ TS 4.2.2.2 }

\textbf{TS 4.2.2.2 } \newline
\textbf{Samhita Paata} \newline

रज॑सि तस्थि॒वाꣳ स॑मृ॒तस्य॒ योनौ॑ महि॒षा अ॑हिन्वन्न् ॥ अक्र॑न्दद॒ग्निः स्त॒नय॑न्निव॒ द्यौः क्षामा॒ रेरि॑हद्-वी॒रुधः॑ सम॒ञ्जन्न् । स॒द्यो ज॑ज्ञा॒नो वि हीमि॒द्धो अख्य॒दा रोद॑सी भा॒नुना॑ भात्य॒न्तः ॥ उ॒शिक् पा॑व॒को अ॑र॒तिः सु॑मे॒धा मर्ते᳚ष्व॒ग्निर॒मृतो॒ निधा॑यि । इय॑र्ति धू॒मम॑रु॒षं भरि॑भ्र॒दुच्छु॒क्रेण॑ शो॒चिषा॒ द्यामिन॑क्षत् ॥ विश्व॑स्य के॒तुर्भुव॑नस्य॒ गर्भ॒ आ - [  ] \newline

\textbf{Pada Paata} \newline

रज॑सि । त॒स्थि॒वाꣳस᳚म् । ऋ॒तस्य॑ । योनौ᳚ । म॒हि॒षाः । अ॒हि॒न्व॒न्न् ॥ अक्र॑न्दत् । अ॒ग्निः । स्त॒नयन्न्॑ । इ॒व॒ । द्यौः । क्षाम॑ । रेरि॑हत् । वी॒रुधः॑ । स॒म॒ञ्जन्निति॑ सं - अ॒ञ्जन्न् ॥ स॒द्यः । ज॒ज्ञा॒नः । वीति॑ । हि । ई॒म् । इ॒द्धः । अख्य॑त् । एति॑ । रोद॑सी॒ इति॑ । भा॒नुना᳚ । भा॒ति॒ । अ॒न्तः ॥ उ॒शिक् । पा॒व॒कः । अ॒र॒तिः । सु॒मे॒धा इति॑ सु - मे॒धाः । मर्ते॑षु । अ॒ग्निः । अ॒मृतः॑ । नीति॑ । धा॒यि॒ ॥ इय॑र्ति । धू॒मम् । अ॒रु॒षम् । भरि॑भ्रत् । उदिति॑ । शु॒क्रेण॑ । शो॒चिषा᳚ । द्याम् । इन॑क्षत् ॥ विश्व॑स्य । के॒तुः । भुव॑नस्य । गर्भः॑ । एति॑ ।  \newline




\markright{ TS 4.2.2.3  \hfill https://www.vedavms.in \hfill}
\addcontentsline{toc}{section}{ TS 4.2.2.3 }
\section*{ TS 4.2.2.3 }

\textbf{TS 4.2.2.3 } \newline
\textbf{Samhita Paata} \newline

रोद॑सी अपृणा॒ज्जाय॑मानः । वी॒डुं चि॒दद्रि॑मभिनत् परा॒यन् जना॒ यद॒ग्निमय॑जन्त॒ पञ्च॑ ॥ श्री॒णामु॑दा॒रो ध॒रुणो॑ रयी॒णां म॑नी॒षाणां॒ प्रार्प॑णः॒ सोम॑गोपाः । वसोः᳚ सू॒नुः सह॑सो अ॒फ्सु राजा॒ वि भा॒त्यग्र॑ उ॒षसा॑मिधा॒नः ॥ यस्ते॑ अ॒द्य कृ॒णव॑द्-भद्रशोचेऽपू॒पं दे॑व घृ॒तव॑न्तमग्ने । प्रतं न॑य प्रत॒रां ॅवस्यो॒ अच्छा॒भि द्यु॒म्नं दे॒वभ॑क्तं ॅयविष्ठ ॥ आ - [  ] \newline

\textbf{Pada Paata} \newline

रोद॑सी॒ इति॑ । अ॒पृ॒णा॒त् । जाय॑मानः ॥ वी॒डुम् । चि॒त् । अद्रि᳚म् । अ॒भि॒न॒त् । प॒रा॒यन्निति॑ परा - यन्न् । जनाः᳚ । यत् । अ॒ग्निम् । अय॑जन्त । पञ्च॑ ॥ श्री॒णाम् । उ॒दा॒रः । ध॒रुणः॑ । र॒यी॒णाम् । म॒नी॒षाणा᳚म् । प्रार्प॑ण॒ इति॑ प्र - अर्प॑णः । सोम॑गोपा॒ इति॒ सोम॑ - गो॒पाः॒ ॥ वसोः᳚ । सू॒नुः । सह॑सः । अ॒फ्स्वित्य॑प्- सु । राजा᳚ । वीति॑ । भा॒ति॒ । अग्रे᳚ । उ॒षसा᳚म् । इ॒धा॒नः ॥ यः । ते॒ । अ॒द्य । कृ॒णव॑त् । भ॒द्र॒शो॒च॒ इति॑ भद्र - शो॒चे॒ । अ॒पू॒पम् । दे॒व॒ । घृ॒तव॑न्त॒मिति॑ घृ॒त - व॒न्त॒म् । अ॒ग्ने॒ ॥ प्रेति॑ । तम् । न॒य॒ । प्र॒त॒रामिति॑ प्र - त॒राम् । वस्यः॑ । अच्छ॑ । अ॒भीति॑ । द्यु॒म्नम् । दे॒वभ॑क्त॒मिति॑ दे॒व - भ॒क्त॒म् । य॒वि॒ष्ठ॒ ॥ एति॑ ।  \newline




\markright{ TS 4.2.2.4  \hfill https://www.vedavms.in \hfill}
\addcontentsline{toc}{section}{ TS 4.2.2.4 }
\section*{ TS 4.2.2.4 }

\textbf{TS 4.2.2.4 } \newline
\textbf{Samhita Paata} \newline

तं भ॑ज सौश्रव॒सेष्व॑ग्न उ॒क्थ-उ॑क्थ॒ आ भ॑ज श॒स्यमा॑ने । प्रि॒यः सूर्ये᳚ प्रि॒यो अ॒ग्ना भ॑वा॒त्युज्जा॒तेन॑ भि॒नद॒दुज्जनि॑त्वैः ॥ त्वाम॑ग्ने॒ यज॑माना॒ अनु॒ द्यून्. विश्वा॒ वसू॑नि दधिरे॒ वार्या॑णि । त्वया॑ स॒ह द्रवि॑णमि॒च्छमा॑ना व्र॒जं गोम॑न्तमु॒शिजो॒ वि व॑व्रुः ॥ दृ॒शा॒नो रु॒क्म उ॒र्व्या व्य॑द्यौद्-दु॒र्मर्.ष॒मायुः॑ श्रि॒ये रु॑चा॒नः । अ॒ग्निर॒मृतो॑ अभव॒द्-वयो॑भि॒र्यदे॑- ( ) -नं॒ द्यौरज॑नयथ् सु॒रेताः᳚ ॥ \newline

\textbf{Pada Paata} \newline

तम् । भ॒ज॒ । सौ॒श्र॒व॒सेषु॑ । अ॒ग्ने॒ । उ॒क्थ उ॑क्थ॒ इत्यु॒क्थे - उ॒क्थे॒ । एति॑ । भ॒ज॒ । श॒स्यमा॑ने ॥ प्रि॒यः । सूर्ये᳚ । प्रि॒यः । अ॒ग्ना । भ॒वा॒ति॒ । उदिति॑ । जा॒तेन॑ । भि॒नद॑त् । उदिति॑ । जनि॑त्वैः ॥ त्वाम् । अ॒ग्ने॒ । यज॑मानाः । अन्विति॑ । द्यून् । विश्वा᳚ । वसू॑नि । द॒धि॒रे॒ । वार्या॑णि ॥ त्वया᳚ । स॒ह । द्रवि॑णम् । इ॒च्छमा॑नाः । व्र॒जम् । गोम॑न्त॒मिति॒ गो - म॒न्त॒म् । उ॒शिजः॑ । वीति॑ । व॒व्रुः॒ ॥ दृ॒शा॒नः । रु॒क्मः । उ॒र्व्या । वीति॑ । अ॒द्यौ॒त् । दु॒र्मर्.ष॒मिति॑ दुः - मर्.ष᳚म् । आयुः॑ । श्रि॒ये । रु॒चा॒नः ॥ अ॒ग्निः । अ॒मृतः॑ । अ॒भ॒व॒त् । वयो॑भि॒रिति॒ वयः॑ - भिः॒ । यत् ( ) । ए॒न॒म् । द्यौः । अज॑नयत् । सु॒रेता॒ इति॑ सु - रेताः᳚ ॥  \newline




\markright{ TS 4.2.3.1  \hfill https://www.vedavms.in \hfill}
\addcontentsline{toc}{section}{ TS 4.2.3.1 }
\section*{ TS 4.2.3.1 }

\textbf{TS 4.2.3.1 } \newline
\textbf{Samhita Paata} \newline

अन्न॑प॒तेऽन्न॑स्य नो देह्यनमी॒वस्य॑ शु॒ष्मिणः॑ । प्र प्र॑दा॒तारं॑ तारिष॒ ऊर्जं॑ नो धेहि द्वि॒पदे॒ चतु॑ष्पदे ॥ उदु॑ त्वा॒ विश्वे॑ दे॒वा अग्ने॒ भर॑न्तु॒ चित्ति॑भिः । स नो॑ भव शि॒वत॑मः सु॒प्रती॑को वि॒भाव॑सुः ॥ प्रेद॑ग्ने॒ ज्योति॑ष्मान्. याहि शि॒वेभि॑र॒र्चि॑भि॒स्त्वं । बृ॒हद्भि॑-र्भा॒नुभि॒-र्भास॒न् मा हिꣳ॑सी स्त॒नुवा᳚ प्र॒जाः ॥ स॒मिधा॒ऽग्निं दु॑वस्यत घृ॒तैर्बो॑धय॒ताति॑थिं । आ - [  ] \newline

\textbf{Pada Paata} \newline

अन्न॑पत॒ इत्यन्न॑ - प॒ते॒ । अन्न॑स्य । नः॒ । दे॒हि॒ । अ॒न॒मी॒वस्य॑ । शु॒ष्मिणः॑ ॥ प्रेति॑ । प्र॒दा॒तार॒मिति॑ प्र - दा॒तार᳚म् । ता॒रि॒षः॒ । ऊर्ज᳚म् । नः॒ । धे॒हि॒ । द्वि॒पद॒ इति॑ द्वि - पदे᳚ । चतु॑ष्पद॒ इति॒ चतुः॑ - प॒दे॒ ॥ उदिति॑ । उ॒ । त्वा॒ । विश्वे᳚ । दे॒वाः । अग्ने᳚ । भर॑न्तु । चित्ति॑भि॒रिति॒ चित्ति॑ - भिः॒ ॥ सः । नः॒ । भ॒व॒ । शि॒वत॑म॒ इति॑ शि॒व - त॒मः॒ । सु॒प्रती॑क॒ इति॑ सु - प्रती॑कः । वि॒भाव॑सु॒रिति॑ वि॒भा-व॒सुः॒ ॥ प्रेति॑ । इत् । अ॒ग्ने॒ । ज्योति॑ष्मान् । या॒हि॒ । शि॒वेभिः॑ । अ॒र्चिभि॒रित्य॒र्चि - भिः॒ । त्वम् ॥ बृ॒हद्भि॒रिति॑ बृ॒हत् - भिः॒ । भा॒नुभि॒रिति॑ भा॒नु - भिः॒ । भासन्न्॑ । मा । हिꣳ॒॒सीः॒ । त॒नुवा᳚ । प्र॒जा इति॑ प्र - जाः ॥ स॒मिधेति॑ सं - इधा᳚ । अ॒ग्निम् । दु॒व॒स्य॒त॒ । घृ॒तैः । बो॒ध॒य॒त॒ । अति॑थिम् ॥ एति॑ ।  \newline




\markright{ TS 4.2.3.2  \hfill https://www.vedavms.in \hfill}
\addcontentsline{toc}{section}{ TS 4.2.3.2 }
\section*{ TS 4.2.3.2 }

\textbf{TS 4.2.3.2 } \newline
\textbf{Samhita Paata} \newline

ऽस्मि॑न्. ह॒व्या जु॑होतन ॥ प्रप्रा॒यम॒ग्निर्भ॑र॒तस्य॑ शृण्वे॒ वि यथ् सूर्यो॒ न रोच॑ते बृ॒हद्भाः । अ॒भि यः पू॒रुं पृत॑नासु त॒स्थौ दी॒दाय॒ दैव्यो॒ अति॑थिः शि॒वो नः॑ ॥ आपो॑ देवीः॒ प्रति॑ गृह्णीत॒ भस्मै॒तथ् स्यो॒ने कृ॑णुद्ध्वꣳ सुर॒भावु॑ लो॒के । तस्मै॑ नमन्तां॒ जन॑यः सु॒पत्नी᳚र्मा॒तेव॑ पु॒त्रं बि॑भृ॒ता स्वे॑नं ॥ अ॒फ्स्व॑ग्ने॒ सधि॒ष्टव॒- [  ] \newline

\textbf{Pada Paata} \newline

अ॒स्मि॒न्न् । ह॒व्या । जु॒हो॒त॒न॒ ॥ प्रप्रेति॒ प्र - प्र॒ । अ॒यम् । अ॒ग्निः । भ॒र॒तस्य॑ । शृ॒ण्वे॒ । वीति॑ । यत् । सूर्यः॑ । न । रोच॑ते । बृ॒हत् । भाः ॥ अ॒भीति॑ । यः । पू॒रुम् । पृत॑नासु । त॒स्थौ । दी॒दाय॑ । दैव्यः॑ । अति॑थिः । शि॒वः । नः॒ ॥ आपः॑ । दे॒वीः॒ । प्रतीति॑ । गृ॒ह्णी॒त॒ । भस्म॑ । ए॒तत् । स्यो॒ने । कृ॒णु॒द्ध्व॒म् । सु॒र॒भौ । उ॒ । लो॒के ॥ तस्मै᳚ । न॒म॒न्ता॒म् । जन॑यः । सु॒पत्नी॒रिति॑ सु - पत्नीः᳚ । मा॒ता । इ॒व॒ । पु॒त्रम् । बि॒भृ॒त । स्विति॑ । ए॒न॒म् ॥ अ॒फ्स्वित्य॑प् - सु । अ॒ग्ने॒ । सधिः॑ । तव॑ ।  \newline




\markright{ TS 4.2.3.3  \hfill https://www.vedavms.in \hfill}
\addcontentsline{toc}{section}{ TS 4.2.3.3 }
\section*{ TS 4.2.3.3 }

\textbf{TS 4.2.3.3 } \newline
\textbf{Samhita Paata} \newline

सौष॑धी॒रनु॑ रुद्ध्यसे । गर्भे॒ सञ्जा॑यसे॒ पुनः॑ ॥ गर्भो॑ अ॒स्योष॑धीनां॒ गर्भो॒ वन॒स्पती॑नां । गर्भो॒ विश्व॑स्य भू॒तस्याग्ने॒ गर्भो॑ अ॒पाम॑सि ॥ प्र॒सद्य॒ भस्म॑ना॒ योनि॑म॒पश्च॑ पृथि॒वीम॑ग्ने । सꣳ॒॒सृज्य॑ मा॒तृभि॒स्त्वं ज्योति॑ष्मा॒न् पुन॒राऽस॑दः ॥ पुन॑रा॒सद्य॒ सद॑नम॒पश्च॑ पृथि॒वीम॑ग्ने । शेषे॑ मा॒तुर्यथो॒पस्थे॒ ऽन्तर॒स्याꣳ शि॒वत॑मः ॥ पुन॑रू॒र्जा - [  ] \newline

\textbf{Pada Paata} \newline

सः । ओष॑धीः । अन्विति॑ । रु॒द्ध्य॒से॒ ॥ गर्भे᳚ । सन्न् । जा॒य॒से॒ । पुनः॑ ॥ गर्भः॑ । अ॒सि॒ । ओष॑धीनाम् । गर्भः॑ । वन॒स्पती॑नाम् ॥ गर्भः॑ । विश्व॑स्य । भू॒तस्य॑ । अग्ने᳚ । गर्भः॑ । अ॒पाम् । अ॒सि॒ ॥ प्र॒सद्येति॑ प्र - सद्य॑ । भस्म॑ना । योनि᳚म् । अ॒पः । च॒ । पृ॒थि॒वीम् । अ॒ग्ने॒ ॥ सꣳ॒॒सृज्येति॑ सं - सृज्य॑ । मा॒तृभि॒रिति॑ मा॒तृ - भिः॒ । त्वम् । ज्योति॑ष्मान् । पुनः॑ । एति॑ । अ॒स॒दः॒ ॥ पुनः॑ । आ॒सद्येत्या᳚-सद्य॑ । सद॑नम् । अ॒पः । च॒ । पृ॒थि॒वीम् । अ॒ग्ने॒ ॥ शेषे᳚ । मा॒तुः । यथा᳚ । उ॒पस्थ॒ इत्यु॒प - स्थे॒ । अ॒न्तः । अ॒स्याम् । शि॒वत॑म॒ इति॑ शि॒व - त॒मः॒ ॥ पुनः॑ । ऊ॒र्जा ।  \newline




\markright{ TS 4.2.3.4  \hfill https://www.vedavms.in \hfill}
\addcontentsline{toc}{section}{ TS 4.2.3.4 }
\section*{ TS 4.2.3.4 }

\textbf{TS 4.2.3.4 } \newline
\textbf{Samhita Paata} \newline

-नि व॑र्तस्व॒ पुन॑रग्न इ॒षा ऽऽयु॑षा । पुन॑र्नः पाहि वि॒श्वतः॑ ॥ स॒ह र॒य्या नि व॑र्त॒स्वाग्ने॒ पिन्व॑स्व॒ धार॑या । वि॒श्वफ्स्नि॑या वि॒श्वत॒स्परि॑ ॥ पुन॑स्त्वा ऽऽदि॒त्या रु॒द्रा वस॑वः॒ समि॑न्धतां॒ पुन॑र्ब्र॒ह्माणो॑ वसुनीथ य॒ज्ञिः । घृ॒तेन॒ त्वं त॒नुवो॑ वर्द्धयस्व स॒त्याः स॑न्तु॒ यज॑मानस्य॒ कामाः᳚ ॥ बोधा॑ नो अ॒स्य वच॑सो यविष्ठ॒ मꣳहि॑ष्ठस्य॒ प्रभृ॑तस्य स्वधावः । पीय॑ति त्वो॒ अनु॑ ( ) त्वो गृणाति व॒न्दारु॑स्ते त॒नुवं॑ ॅवन्दे अग्ने ॥ स बो॑धि सू॒रिर्म॒घवा॑ वसु॒दावा॒ वसु॑पतिः । यु॒यो॒द्ध्य॑स्मद् द्वेषाꣳ॑सि ॥ \newline

\textbf{Pada Paata} \newline

नीति॑ । व॒र्त॒स्व॒ । पुनः॑ । अ॒ग्ने॒ । इ॒षा । आयु॑षा ॥ पुनः॑ । नः॒ । पा॒हि॒ । वि॒श्वतः॑ ॥ स॒ह । र॒य्या । नीति॑ । व॒र्त॒स्व॒ । अग्ने᳚ । पिन्व॑स्व । धार॑या ॥ वि॒श्वफ्स्नि॒येति॑ वि॒श्व - फ्स्नि॒या॒ । वि॒श्वतः॑ । परि॑ ॥ पुनः॑ । त्वा॒ । आ॒दि॒त्याः । रु॒द्राः । वस॑वः । समिति॑ । इ॒न्ध॒ता॒म् । पुनः॑ । ब्र॒ह्माणः॑ । व॒सु॒नी॒थेति॑ वसु-नी॒थ॒ । य॒ज्ञिः ॥ घृ॒तेन॑ । त्वम् । त॒नुवः॑ । व॒द्‌र्ध॒य॒स्व॒ । स॒त्याः । स॒न्तु॒ । यज॑मानस्य । कामाः᳚ ॥ बोध॑ । नः॒ । अ॒स्य । वच॑सः । य॒वि॒ष्ठ॒ । मꣳहि॑ष्ठस्य । प्रभृ॑त॒स्येति॒ प्र - भृ॒त॒स्य॒ । स्व॒धा॒व॒ इति॑ स्वधा - वः॒ ॥ पीय॑ति । त्वः॒ । अन्विति॑ ( ) । त्वः॒ । गृ॒णा॒ति॒ । व॒न्दारुः॑ । ते॒ । त॒नुव᳚म् । व॒न्दे॒ । अ॒ग्ने॒ ॥ सः । बो॒धि॒ । सू॒रिः । म॒घवेति॑ म॒घ - वा॒ । व॒सु॒दावेति॑ वसु - दावा᳚ । वसु॑पति॒रिति॒ वसु॑ - प॒तिः॒ ॥ यु॒यो॒धि । अ॒स्मत् । द्वेषाꣳ॑सि ॥  \newline




\markright{ TS 4.2.4.1  \hfill https://www.vedavms.in \hfill}
\addcontentsline{toc}{section}{ TS 4.2.4.1 }
\section*{ TS 4.2.4.1 }

\textbf{TS 4.2.4.1 } \newline
\textbf{Samhita Paata} \newline

अपे॑त॒ वीत॒ वि च॑ सर्प॒तातो॒ येऽत्र॒ स्थ पु॑रा॒णा ये च॒ नूत॑नाः । अदा॑दि॒दं ॅय॒मो॑ऽव॒सानं॑ पृथि॒व्या अक्र॑न्नि॒मं पि॒तरो॑ लो॒कम॑स्मै ॥ अ॒ग्नेर्भस्मा᳚स्य॒ग्नेः पुरी॑षमसि स॒ज्ञांन॑मसि काम॒धर॑णं॒ मयि॑ ते काम॒धर॑णं भूयात् ॥ सं ॅया वः॑ प्रि॒यास्त॒नुवः॒ सं प्रि॒या हृद॑यानि वः । आ॒त्मा वो॑ अस्तु॒ - [  ] \newline

\textbf{Pada Paata} \newline

अपेति॑ । इ॒त॒ । वीति॑ । इ॒त॒ । वीति॑ । च॒ । स॒र्प॒त॒ । अतः॑ । ये । अत्र॑ । स्थ । पु॒रा॒णाः । ये । च॒ । नूत॑नाः ॥ अदा᳚त् । इ॒दम् । य॒मः । अ॒व॒सान॒मित्य॑व - सान᳚म् । पृ॒थि॒व्याः । अक्रन्न्॑ । इ॒मम् । पि॒तरः॑ । लो॒कम् । अ॒स्मै॒ ॥ अ॒ग्नेः । भस्म॑ । अ॒सि॒ । अ॒ग्नः॒ । पुरी॑षम् । अ॒सि॒ । स॒ज्ञांन॒मिति॑ सं - ज्ञान᳚म् । अ॒सि॒ । का॒म॒धर॑ण॒मिति॑ काम - धर॑णम् । मयि॑ । ते॒ । का॒म॒धर॑ण॒मिति॑ काम - धर॑णम् । भू॒या॒त् ॥ समिति॑ । याः । वः॒ । प्रि॒याः । त॒नुवः॑ । समिति॑ । प्रि॒या । हृद॑यानि । वः॒ ॥ आ॒त्मा । वः॒ । अ॒स्तु॒ ।  \newline




\markright{ TS 4.2.4.2  \hfill https://www.vedavms.in \hfill}
\addcontentsline{toc}{section}{ TS 4.2.4.2 }
\section*{ TS 4.2.4.2 }

\textbf{TS 4.2.4.2 } \newline
\textbf{Samhita Paata} \newline

संप्रि॑यः॒ संप्रि॑यास्त॒नुवो॒ मम॑ ॥ अ॒यꣳ सो अ॒ग्निर्यस्मि॒न्थ्-सोम॒मिन्द्रः॑ सु॒तं द॒धे ज॒ठरे॑ वावशा॒नः । स॒ह॒स्रियं॒ ॅवाज॒मत्यं॒ न सप्तिꣳ॑ सस॒वान्थ्-सन्थ्-स्तू॑यसे जातवेदः ॥ अग्ने॑ दि॒वो अर्ण॒मच्छा॑ जिगा॒स्यच्छा॑ दे॒वाꣳ ऊ॑चिषे॒ धिष्णि॑या॒ ये । याः प॒रस्ता᳚द्-रोच॒ने सूर्य॑स्य॒ याश्चा॒ वस्ता॑-दुप॒तिष्ठ॑न्त॒ आपः॑ ॥ अग्ने॒ यत् ते॑ दि॒वि वर्चः॑ पृथि॒व्यां ॅयदोष॑धीष्व॒ - [  ] \newline

\textbf{Pada Paata} \newline

संप्रि॑य॒ इति॒ सं-प्रि॒यः॒ । संप्रि॑या॒ इति॒ सं - प्रि॒याः॒ । त॒नुवः॑ । मम॑ ॥ अ॒यम् । सः । अ॒ग्निः । यस्मिन्न्॑ । सोम᳚म् । इन्द्रः॑ । सु॒तम् । द॒धे । ज॒ठरे᳚ । वा॒व॒शा॒नः ॥ स॒ह॒स्रिय᳚म् । वाज᳚म् । अत्य᳚म् । न । सप्ति᳚म् । स॒स॒वानिति॑ स - स॒वान् । सन्न् । स्तृ॒य॒से॒ । जा॒त॒वे॒द॒ इति॑ जात - वे॒दः॒ ॥ अग्ने᳚ । दि॒वः । अर्ण᳚म् । अच्छ॑ । जि॒गा॒सि॒ । अच्छ॑ । दे॒वान् । ऊ॒चि॒षे॒ । धिष्णि॑याः । ये ॥ याः । प॒रस्ता᳚त् । रो॒च॒ने । सूर्य॑स्य । याः । च॒ । अ॒वस्ता᳚त् । उ॒प॒तिष्ठ॑न्त॒ इत्यु॑प - तिष्ठ॑न्ते । आपः॑ ॥ अग्ने᳚ । यत् । ते॒ । दि॒वि । वर्चः॑ । पृ॒थि॒व्याम् । यत् । ओष॑धीषु ।  \newline




\markright{ TS 4.2.4.3  \hfill https://www.vedavms.in \hfill}
\addcontentsline{toc}{section}{ TS 4.2.4.3 }
\section*{ TS 4.2.4.3 }

\textbf{TS 4.2.4.3 } \newline
\textbf{Samhita Paata} \newline

-ऽफ्सु वा॑ यजत्र । येना॒न्तरि॑क्ष-मु॒र्वा॑त॒तन्थ॑ त्वे॒षः स भा॒नुर॑र्ण॒वो नृ॒चक्षाः᳚ ॥ पु॒री॒ष्या॑सो अ॒ग्नयः॑ प्राव॒णेभिः॑ स॒जोष॑सः । जु॒षन्ताꣳ॑ ह॒व्यमाहु॑तमनमी॒वा इषो॑ म॒हीः ॥ इडा॑मग्ने पुरु॒दꣳ सꣳ॑ स॒निं गोः श॑श्वत्त॒मꣳ हव॑मानाय साध । स्यान्नः॑ सू॒नुस्तन॑यो वि॒जावाऽग्ने॒ सा ते॑ सुम॒तिर्भू᳚त्व॒स्मे ॥ अ॒यं ते॒ योनि॑र्. ऋ॒त्वियो॒ यतो॑ जा॒तो अरो॑चथाः । तं जा॒न - [  ] \newline

\textbf{Pada Paata} \newline

अ॒फ्स्वित्य॑प् - सु । वा॒ । य॒ज॒त्र॒ ॥ येन॑ । अ॒न्तरि॑क्षम् । उ॒रु । आ॒त॒तन्थेत्या᳚ - त॒तन्थ॑ । त्वे॒षः । सः । भा॒नुः । अ॒र्ण॒वः । नृ॒चक्षा॒ इति॑ नृ - चक्षाः᳚ ॥ पु॒री॒ष्या॑सः । अ॒ग्नयः॑ । प्रा॒व॒णेभि॒रिति॑ प्र - व॒नेभिः॑ । स॒जोष॑स॒ इति॑ स - जोष॑सः ॥ जु॒षन्ता᳚म् । ह॒व्यम् । आहु॑त॒मित्या - हु॒त॒म् । अ॒न॒मी॒वाः । इषः॑ । म॒हीः ॥ इडा᳚म् । अ॒ग्ने॒ । पु॒रु॒दꣳस॒मिति॑ पुरु - दꣳस᳚म् । स॒निम् । गोः । श॒श्व॒त्त॒ममिति॑ शश्वत् - त॒मम् । हव॑मानाय । सा॒ध॒ ॥ स्यात् । नः॒ । सू॒नुः । तन॑यः । वि॒जावेति॑ वि - जावा᳚ । अग्ने᳚ । सा । ते॒ । सु॒म॒तिरिति॑ सु-म॒तिः । भू॒तु॒ । अ॒स्मे इति॑ ॥ अ॒यम् । ते॒ । योनिः॑ । ऋ॒त्वियः॑ । यतः॑ । जा॒तः । अरो॑चथाः ॥ तम् । जा॒नन्न् ।  \newline




\markright{ TS 4.2.4.4  \hfill https://www.vedavms.in \hfill}
\addcontentsline{toc}{section}{ TS 4.2.4.4 }
\section*{ TS 4.2.4.4 }

\textbf{TS 4.2.4.4 } \newline
\textbf{Samhita Paata} \newline

न्न॑ग्न॒ आ रो॒हाथा॑ नो वर्द्धया र॒यिं ॥ चिद॑सि॒ तया॑ दे॒वत॑याऽङ्गिर॒स्वद्-ध्रु॒वा सी॑द परि॒चिद॑सि॒ तया॑ दे॒वत॑याऽङ्गिर॒स्वद् ध्रु॒वा सी॑द लो॒कं पृ॑ण छि॒द्रं पृ॒णाथो॑ सीद शि॒वा त्वं । इ॒न्द्रा॒ग्नी त्वा॒ बृह॒स्पति॑र॒स्मिन्. योना॑वसीषदन्न् ॥ ता अ॑स्य॒ सूद॑दोहसः॒ सोमꣳ॑ श्रीणन्ति॒ पृश्न॑यः । जन्म॑न् दे॒वानां॒ ॅविश॑स्त्रि॒ष्वा रो॑च॒ने दि॒वः ॥ \newline

\textbf{Pada Paata} \newline

अ॒ग्ने॒ । एति॑ । रो॒ह॒ । अथ॑ । नः॒ । व॒द्‌र्ध॒य॒ । र॒यिम् ॥ चित् । अ॒सि॒ । तया᳚ । दे॒वत॑या । अ॒ङ्गि॒र॒स्वत् । ध्रु॒वा । सी॒द॒ । प॒रि॒चिदिति॑ परि - चित् । अ॒सि॒ । तया᳚ । दे॒वत॑या । अ॒ङ्गि॒र॒स्वत् । ध्रु॒वा । सी॒द॒ । लो॒कम् । पृ॒ण॒ । छि॒द्रम् । पृ॒ण॒ । अथो॒ इति॑ । सी॒द॒ । शि॒वा । त्वम् ॥ इ॒न्द्रा॒ग्नी इती᳚न्द्र - अ॒ग्नी । त्वा॒ । बृह॒स्पतिः॑ । अ॒स्मिन्न् । योनौ᳚ । अ॒सी॒ष॒द॒न्न् ॥ ताः । अ॒स्य॒ । सूद॑दोहस॒ इति॒ सूद॑ - दो॒ह॒सः॒ । सोम᳚म् । श्री॒ण॒न्ति॒ । पृश्न॑यः ॥ जन्मन्न्॑ । दे॒वाना᳚म् । विशः॑ । त्रि॒षु । एति॑ । रो॒च॒ने । दि॒वः ॥  \newline




\markright{ TS 4.2.5.1  \hfill https://www.vedavms.in \hfill}
\addcontentsline{toc}{section}{ TS 4.2.5.1 }
\section*{ TS 4.2.5.1 }

\textbf{TS 4.2.5.1 } \newline
\textbf{Samhita Paata} \newline

समि॑तꣳ॒॒ सङ्क॑ल्पेथाꣳ॒॒ संप्रि॑यौ रोचि॒ष्णू सु॑मन॒स्यमा॑नौ । इष॒मूर्ज॑म॒भि सं॒ॅवसा॑नौ॒ सं ॅवां॒ मनाꣳ॑सि॒ सं ॅव्र॒ता समु॑ चि॒त्तान्याऽक॑रं ॥ अग्ने॑ पुरीष्याधि॒पा भ॑वा॒ त्वं नः॑ । इष॒मूर्जं॒ ॅयज॑मानाय धेहि ॥ पु॒री॒ष्य॑स्त्वम॑ग्ने रयि॒मान् पु॑ष्टि॒माꣳ अ॑सि । शि॒वाः कृ॒त्वा दिशः॒ सर्वाः॒ स्वां ॅयोनि॑मि॒हाऽस॑दः ॥ भव॑तं नः॒ सम॑नसौ॒ समो॑कसा - [  ] \newline

\textbf{Pada Paata} \newline

समिति॑ । इ॒त॒म् । समिति॑ । क॒ल्पे॒था॒म् । संप्रि॑या॒विति॒ सं - प्रि॒यौ॒ । रो॒चि॒ष्णू इति॑ । सु॒म॒न॒स्यमा॑ना॒विति॑ सु - म॒न॒स्यमा॑नौ ॥ इष᳚म् । ऊर्ज᳚म् । अ॒भीति॑ । सं॒ॅवसा॑ना॒विति॑ सं - वसा॑नौ । समिति॑ । वा॒म् । मनाꣳ॑सि । समिति॑ । व्र॒ता । समिति॑ । उ॒ । चि॒त्तानि॑ । एति॑ । अ॒क॒र॒म् ॥ अग्ने᳚ । पु॒री॒ष्य॒ । अ॒धि॒पा इत्य॑धि - पाः । भ॒व॒ । त्वम् । नः॒ ॥ इष᳚म् । ऊर्ज᳚म् । यज॑मानाय । धे॒हि॒ ॥ पु॒री॒ष्यः॑ । त्वम् । अ॒ग्ने॒ । र॒यि॒मानिति॑ रयि - मान् । पु॒ष्टि॒मानिति॑ पुष्टि-मान् । अ॒सि॒ ॥ शि॒वाः । कृ॒त्वा । दिशः॑ । सर्वाः᳚ । स्वाम् । योनि᳚म् । इ॒ह । एति॑ । अ॒स॒दः॒ ॥ भव॑तम् । नः॒ । सम॑नसा॒विति॒ स - म॒न॒सौ॒ । समो॑कसा॒विति॒ सं - ओ॒क॒सौ॒ ।  \newline




\markright{ TS 4.2.5.2  \hfill https://www.vedavms.in \hfill}
\addcontentsline{toc}{section}{ TS 4.2.5.2 }
\section*{ TS 4.2.5.2 }

\textbf{TS 4.2.5.2 } \newline
\textbf{Samhita Paata} \newline

-वरे॒पसौ᳚ । मा य॒ज्ञ्ꣳ हिꣳ॑सिष्टं॒ मा य॒ज्ञ्प॑तिं जातवेदसौ शि॒वौ भ॑वतम॒द्य नः॑ ॥ मा॒तेव॑ पु॒त्रं पृ॑थि॒वी पु॑री॒ष्य॑म॒ग्निꣳ स्वे योना॑वभारु॒खा । तां ॅविश्वै᳚र्दे॒वैर्. ऋ॒तुभिः॑ संॅविदा॒नः प्र॒जाप॑तिर्वि॒श्वक॑र्मा॒ वि मु॑ञ्चतु ॥ यद॒स्य पा॒रे रज॑सः शु॒क्रं ज्योति॒रजा॑यत । तं नः॑ पर्.ष॒दति॒ द्विषोऽग्ने॑ वैश्वानर॒ स्वाहा᳚ ॥ नमः॒ सु ते॑ निर्.ऋते विश्वरूपे - [  ] \newline

\textbf{Pada Paata} \newline

अ॒रे॒पसौ᳚ ॥ मा । य॒ज्ञ्म् । हिꣳ॒॒सि॒ष्ट॒म् । मा । य॒ज्ञ्प॑ति॒मिति॑ य॒ज्ञ् - प॒ति॒म् । जा॒त॒वे॒द॒सा॒विति॑ जात - वे॒द॒सौ॒ । शि॒वौ । भ॒व॒त॒म् । अ॒द्य । नः॒ ॥ मा॒ता । इ॒व॒ । पु॒त्रम् । पृ॒थि॒वी । पु॒री॒ष्य᳚म् । अ॒ग्निम् । स्वे । योनौ᳚ । अ॒भाः॒ । उ॒खा ॥ ताम् । विश्वैः᳚ । दे॒वैः । ऋ॒तुभि॒रित्यृ॒तु - भिः॒ । सं॒ॅवि॒दा॒न इति॑ सं - वि॒दा॒नः । प्र॒जाप॑ति॒रिति॑ प्र॒जा - प॒तिः॒ । वि॒श्वक॒र्मेति॑ वि॒श्व - क॒र्मा॒ । वीति॑ । मु॒ञ्च॒तु॒ ॥ यत् । अ॒स्य । पा॒रे । रज॑सः । शु॒क्रम् । ज्योतिः॑ । अजा॑यत ॥ तम् । नः॒ । प॒र्॒.ष॒त् । अतीति॑ । द्विषः॑ । अग्ने᳚ । वै॒श्वा॒न॒र॒ । स्वाहा᳚ ॥ नमः॑ । स्विति॑ । ते॒ । नि॒र्॒.ऋ॒त॒ इति॑ निः - ऋ॒ते॒ । वि॒श्व॒रू॒प॒ इति॑ विश्व - रू॒पे॒ ।  \newline




\markright{ TS 4.2.5.3  \hfill https://www.vedavms.in \hfill}
\addcontentsline{toc}{section}{ TS 4.2.5.3 }
\section*{ TS 4.2.5.3 }

\textbf{TS 4.2.5.3 } \newline
\textbf{Samhita Paata} \newline

ऽय॒स्मयं॒ ॅवि चृ॑ता ब॒न्धमे॒तं । य॒मेन॒ त्वं ॅय॒म्या॑ सं ॅविदा॒नोत्त॒मं नाक॒मधि॑ रोहये॒मं ॥ यत्ते॑ दे॒वी निर्.ऋ॑तिराब॒बन्ध॒ दाम॑ ग्री॒वास्व॑ विच॒र्त्यं । इ॒दं ते॒ तद् विष्यां॒ ॅयायु॑षो॒ न मद्ध्या॒दथा॑ जी॒वः पि॒तुम॑द्धि॒ प्रमु॑क्तः ॥ यस्या᳚स्ते अ॒स्याः क्रू॒र आ॒सञ्जु॒होम्ये॒षां ब॒न्धाना॑मव॒सर्ज॑नाय । भूमि॒रिति॑ त्वा॒ जना॑ वि॒दुर्निर्.ऋ॑ति॒ - [  ] \newline

\textbf{Pada Paata} \newline

अ॒य॒स्मय᳚म् । वीति॑ । चृ॒त॒ । ब॒न्धम् । ए॒तम् ॥ य॒मेन॑ । त्वम् । य॒म्या᳚ । सं॒ॅवि॒दा॒नेति॑ सं - वि॒दा॒ना । उ॒त्त॒ममित्यु॑त् - त॒मम् । नाक᳚म् । अधीति॑ । रो॒ह॒य॒ । इ॒मम् ॥ यत् । ते॒ । दे॒वी । निर्.ऋ॑ति॒रिति॒ निः - ऋ॒तिः॒ । आ॒ब॒बन्धेत्या᳚ - ब॒बन्ध॑ । दाम॑ । ग्री॒वासु॑ । अ॒वि॒च॒र्त्यमित्य॑वि - च॒र्त्यम् ॥ इ॒दम् । ते॒ । तत् । वीति॑ । स्या॒मि॒ । आयु॑षः । न । मद्ध्या᳚त् । अथ॑ । जी॒वः । पि॒तुम् । अ॒द्धि॒ । प्रमु॑क्त॒ इति॒ प्र - मु॒क्तः॒ ॥ यस्याः᳚ । ते॒ । अ॒स्याः । क्रू॒रे । आ॒सन्न् । जु॒होमि॑ । ए॒षाम् । ब॒न्धाना᳚म् । अ॒व॒सर्ज॑ना॒येत्य॑व-सर्ज॑नाय ॥ भूमिः॑ । इति॑ । त्वा॒ । जनाः᳚ । वि॒दुः । निर्.ऋ॑ति॒रिति॒ निः - ऋ॒तिः॒ ।  \newline




\markright{ TS 4.2.5.4  \hfill https://www.vedavms.in \hfill}
\addcontentsline{toc}{section}{ TS 4.2.5.4 }
\section*{ TS 4.2.5.4 }

\textbf{TS 4.2.5.4 } \newline
\textbf{Samhita Paata} \newline

रिति॑ त्वा॒ ऽहं परि॑ वेद वि॒श्वतः॑ ॥ असु॑न्वन्त॒म य॑जमानमिच्छ स्ते॒नस्ये॒त्यान्-तस्क॑र॒स्यान् वे॑षि । अ॒न्य म॒स्म-दि॑च्छ॒ सा त॑ इ॒त्या नमो॑ देवि निर्.ऋते॒ तुभ्य॑मस्तु ॥ दे॒वीम॒हं निर्.ऋ॑तिं॒ ॅवन्द॑मानः पि॒तेव॑ पु॒त्रं द॑सये॒ वचो॑भिः । विश्व॑स्य॒ या जाय॑मानस्य॒ वेद॒ शिरः॑ शिरः॒ प्रति॑ सू॒री वि च॑ष्टे ॥ नि॒वेश॑नः स॒ङ्गम॑नो॒ वसू॑नां॒ ॅविश्वा॑ रू॒पाऽभि च॑ष्टे॒ - [  ] \newline

\textbf{Pada Paata} \newline

इति॑ । त्वा॒ । अ॒हम् । परीति॑ । वे॒द॒ । वि॒श्वतः॑ ॥ असु॑न्वन्तम् । अय॑जमानम् । इ॒च्छ॒ । स्ते॒नस्य॑ । इ॒त्याम् । तस्क॑रस्य । अन्विति॑ । ए॒षि॒ ॥ अ॒न्यम् । अ॒स्मत् । इ॒च्छ॒ । सा । ते॒ । इ॒त्या । नमः॑ । दे॒वि॒ । नि॒र्॒.ऋ॒त॒ इति॑ निः - ऋ॒ते॒ । तुभ्य᳚म् । अ॒स्तु॒ ॥ दे॒वीम् । अ॒हम् । निर्.ऋ॑ति॒मिति॒ निः- ऋ॒ति॒म् । वन्द॑मानः । पि॒ता । इ॒व॒ । पु॒त्रम् । द॒स॒ये॒ । वचो॑भि॒रिति॒ वचः॑ - भिः॒ ॥ विश्व॑स्य । या । जाय॑मानस्य । वेद॑ । शिरः॑शिर॒ इति॒ शिरः॑-शि॒रः॒ । प्रतीति॑ । सू॒री । वीति॑ । च॒ष्टे॒ ॥ नि॒वेश॑न॒ इति॑ नि - वेश॑नः । स॒ङ्गम॑न॒ इति॑ सं - गम॑नः । वसू॑नाम् । विश्वा᳚ । रू॒पा । अ॒भीति॑ । च॒ष्टे॒ ।  \newline




\markright{ TS 4.2.5.5  \hfill https://www.vedavms.in \hfill}
\addcontentsline{toc}{section}{ TS 4.2.5.5 }
\section*{ TS 4.2.5.5 }

\textbf{TS 4.2.5.5 } \newline
\textbf{Samhita Paata} \newline

शची॑भिः । दे॒व इ॑व सवि॒ता स॒त्यध॒र्मेन्द्रो॒ न त॑स्थौ सम॒रे प॑थी॒नां ॥ सं ॅव॑र॒त्रा द॑धातन॒ निरा॑हा॒वान् कृ॑णोतन । सि॒ञ्चाम॑हा अव॒टमु॒द्रिणं॑ ॅव॒यं ॅविश्वाऽहाऽद॑स्त॒मक्षि॑तं ॥ निष्कृ॑ताहा-वमव॒टꣳ सु॑वर॒त्रꣳ सु॑षेच॒नं । उ॒द्रिणꣳ॑ सिञ्चे॒ अक्षि॑तं ॥ सीरा॑ युञ्जन्ति क॒वयो॑ यु॒गा वि त॑न्वते॒ पृथ॑क् । धीरा॑ दे॒वेषु॑ सुम्न॒या ॥ यु॒नक्त॒ सीरा॒ वि यु॒गा त॑नोत कृ॒ते योनौ॑ वपते॒ह - [  ] \newline

\textbf{Pada Paata} \newline

शची॑भि॒रिति॒ शचि॑ - भिः॒ ॥ दे॒वः । इ॒व॒ । स॒वि॒ता । स॒त्यध॒र्मेति॑ स॒त्य - ध॒र्मा॒ । इन्द्रः॑ । न । त॒स्थौ॒ । स॒म॒र इति॑ सं - अ॒रे । प॒थी॒नाम् ॥ समिति॑ । व॒र॒त्राः । द॒धा॒त॒न॒ । निरिति॑ । आ॒हा॒वानित्या᳚ - हा॒वान् । कृ॒णो॒त॒न॒ ॥ सि॒ञ्चाम॑है । अ॒व॒टम् । उ॒द्रिण᳚म् । व॒यम् । विश्वा᳚ । अहा᳚ । अद॑स्तम् । अक्षि॑तम् ॥ निष्कृ॑ताहाव॒मिति॒ निष्कृ॑त - आ॒हा॒व॒म् । अ॒व॒टम् । सु॒व॒र॒त्रमिति॑ सु - व॒र॒त्रम् । सु॒षे॒च॒नमिति॑ सु - से॒च॒नम् ॥ उ॒द्रिण᳚म् । सि॒ञ्चे॒ । अक्षि॑तम् ॥ सीरा᳚ । यु॒ञ्ज॒न्ति॒ । क॒वयः॑ । यु॒गा । वीति॑ । त॒न्व॒ते॒ । पृथ॑क् ॥ धीराः᳚ । दे॒वेषु॑ । सु॒म्न॒या ॥ यु॒नक्त॑ । सीरा᳚ । वीति॑ । यु॒गा । त॒नो॒त॒ । कृ॒ते । योनौ᳚ । व॒प॒त॒ । इ॒ह ।  \newline




\markright{ TS 4.2.5.6  \hfill https://www.vedavms.in \hfill}
\addcontentsline{toc}{section}{ TS 4.2.5.6 }
\section*{ TS 4.2.5.6 }

\textbf{TS 4.2.5.6 } \newline
\textbf{Samhita Paata} \newline

बीजं᳚ । गि॒रा च॑ श्रु॒ष्टिः सभ॑रा॒ अस॑न्नो॒ नेदी॑य॒ इथ् सृ॒ण्या॑ प॒क्वमा ऽय॑त् ॥ लाङ्ग॑लं॒ पवी॑रवꣳ सु॒शेवꣳ॑ सुम॒तिथ्स॑रु । उदित् कृ॑षति॒ गामविं॑ प्रफ॒र्व्यं॑ च॒ पीव॑रीं । प्र॒स्थाव॑द्-रथ॒वाह॑नं ॥ शु॒नं नः॒ फाला॒ वि तु॑दन्तु॒ भूमिꣳ॑ शु॒नं की॒नाशा॑ अ॒भि य॑न्तु वा॒हान् । शु॒नं प॒र्जन्यो॒ मधु॑ना॒ पयो॑भिः॒ शुना॑सीरा शु॒नम॒स्मासु॑ धत्तं ॥ कामं॑ कामदुघे धुक्ष्व मि॒त्राय॒ ( ) वरु॑णाय च । इन्द्रा॑या॒ग्नये॑ पू॒ष्ण ओष॑धीभ्यः प्र॒जाभ्यः॑ ॥घृ॒तेन॒ सीता॒ मधु॑ना॒ सम॑क्ता॒ विश्वै᳚र्दे॒वैरनु॑मता म॒रुद्भिः॑ । ऊर्ज॑स्वती॒ पय॑सा॒ पिन्व॑माना॒ऽस्मान्थ् सी॑ते॒ पय॑सा॒ऽभ्या-व॑वृथ्स्व ॥ \newline

\textbf{Pada Paata} \newline

बीज᳚म् ॥ गि॒रा । च॒ । श्रु॒ष्टिः । सभ॑रा॒ इति॒ स-भ॒राः॒ । अस॑त् । नः॒ । नेदी॑यः । इत् । सृ॒ण्या᳚ । प॒क्वम् । एति॑ । अ॒य॒त् ॥ लाङ्ग॑लम् । पवी॑रवम् । सु॒शेव॒मिति॑ सु - शेव᳚म् । सु॒म॒तिथ्स॒र्विति॑ सुम॒ति - थ्‌स॒रु॒ ॥ उदिति॑ । इत् । कृ॒ष॒ति॒ । गाम् । अवि᳚म् । प्र॒फ॒र्व्य॑मिति॑ प्र-फ॒र्व्य᳚म् । च॒ । पीव॑रीम् ॥ प्र॒स्थाव॒दिति॑ प्र॒स्थ-व॒त् । र॒थ॒वाह॑न॒मिति॑ रथ - वाह॑नम् ॥ शु॒नम् । नः॒ । फालाः᳚ । वीति॑ । तु॒द॒न्तु॒ । भूमि᳚म् । शु॒नम् । की॒नाशाः᳚ । अ॒भीति॑ । य॒न्तु॒ । वा॒हान् ॥ शु॒नम् । प॒र्जन्यः॑ । मधु॑ना । पयो॑भि॒रिति॒ पयः॑ - भिः॒ । शुना॑सीरा । शु॒नम् । अ॒स्मासु॑ । ध॒त्त॒म् ॥ काम᳚म् । का॒म॒दु॒घ॒ इति॑ काम - दु॒घे॒ । धु॒क्ष्व॒ । मि॒त्राय॑ ( ) । वरु॑णाय । च॒ ॥ इन्द्रा॑य । अ॒ग्नये᳚ । पू॒ष्णे । ओष॑धीभ्य॒ इत्योष॑धि - भ्यः॒ । प्र॒जाभ्य॒ इति॑ प्र - जाभ्यः॑ ॥ घृ॒तेन॑ । सीता᳚ । मधु॑ना । सम॒क्तेति॒ सम् - अ॒क्ता॒ । विश्वैः᳚ । दे॒वैः । अनु॑म॒तेत्यनु॑-म॒ता॒ । म॒रुद्भि॒रिति॑ म॒रुत् -भिः॒ ॥ ऊर्ज॑स्वती । पय॑सा । पिन्व॑माना । अ॒स्मान् । सी॒ते॒ । पय॑सा । अ॒भ्याव॑वृ॒थ्स्वेत्य॑भि - आव॑वृथ्स्व ॥  \newline




\markright{ TS 4.2.6.1  \hfill https://www.vedavms.in \hfill}
\addcontentsline{toc}{section}{ TS 4.2.6.1 }
\section*{ TS 4.2.6.1 }

\textbf{TS 4.2.6.1 } \newline
\textbf{Samhita Paata} \newline

या जा॒ता ओष॑धयो दे॒वेभ्य॑स्त्रियु॒गं पु॒रा ।मन्दा॑मि ब॒भ्रूणा॑म॒हꣳ श॒तं धामा॑नि स॒प्त च॑ ॥ श॒तं ॅवो॑ अबं॒ धामा॑नि स॒हस्र॑मु॒त वो॒ रुहः॑ । अथा॑ शतक्रत्वो यू॒यमि॒मं मे॑ अग॒दं कृ॑त ॥ पुष्पा॑वतीः प्र॒सूव॑तीः फ॒लिनी॑रफ॒ला उ॒त । अश्वा॑ इव स॒जित्व॑री-र्वी॒रुधः॑ पारयि॒ष्णवः॑ ॥ ओष॑धी॒रिति॑ मातर॒-स्तद्वो॑ देवी॒-रुप॑ ब्रुवे । रपाꣳ॑सि विघ्न॒तीरि॑त॒ रप॑ - [  ] \newline

\textbf{Pada Paata} \newline

याः । जा॒ताः । ओष॑धयः । दे॒वेभ्यः॑ । त्रि॒यु॒गमिति॑ त्रि-यु॒गम् । पु॒रा ॥ मन्दा॑मि । ब॒भ्रूणा᳚म् । अ॒हम् । श॒तम् । धामा॑नि । स॒प्त । च॒ ॥ श॒तम् । वः॒ । अ॒बं॒ । धामा॑नि । स॒हस्र᳚म् । उ॒त । वः॒ । रुहः॑ ॥ अथ॑ । श॒त॒क्र॒त्व॒ इति॑ शत - क्र॒त्वः॒ । यू॒यम् । इ॒मम् । मे॒ । अ॒ग॒दम् । कृ॒त॒ ॥ पुष्पा॑वती॒रिति॒ पुष्प॑-व॒तीः॒ । प्र॒सूव॑ती॒रिति॑ प्र - सूव॑तीः । फ॒लिनीः᳚ । अ॒फ॒लाः । उ॒त ॥ अश्वाः᳚ । इ॒व॒ । स॒जित्व॑री॒रिति॑ स - जित्व॑रीः । वी॒रुधः॑ । पा॒र॒यि॒ष्णवः॑ ॥ ओष॑धीः । इति॑ । मा॒त॒रः॒ । तत् । वः॒ । दे॒वीः॒ । उपेति॑ । ब्रु॒वे॒ ॥ रपाꣳ॑सि । वि॒घ्न॒तीरिति॑ वि - घ्न॒तीः । इ॒त॒ । रपः॑ ।  \newline




\markright{ TS 4.2.6.2  \hfill https://www.vedavms.in \hfill}
\addcontentsline{toc}{section}{ TS 4.2.6.2 }
\section*{ TS 4.2.6.2 }

\textbf{TS 4.2.6.2 } \newline
\textbf{Samhita Paata} \newline

-श्चा॒तय॑मानाः ॥ अ॒श्व॒त्थे वो॑ नि॒षद॑नं प॒र्णे वो॑ वस॒तिः कृ॒ता । गो॒भाज॒ इत् किला॑सथ॒ यथ् स॒नव॑थ॒ पूरु॑षं ॥ यद॒हं ॅवा॒जय॑-न्नि॒मा ओष॑धी॒र्॒.हस्त॑ आद॒धे । आ॒त्मा यक्ष्म॑स्य नश्यति पु॒रा जी॑व॒गृभो॑ यथा ॥ यदोष॑धयः सं॒गच्छ॑न्ते॒ राजा॑नः॒ समि॑ता विव । विप्रः॒ स उ॑च्यते भि॒षग्र॑क्षो॒हा ऽमी॑व॒ चात॑नः ॥ निष्कृ॑ति॒-र्नाम॑वो मा॒ताऽथा॑ यू॒यꣳस्थ॒ संकृ॑तीः । स॒राः प॑त॒त्रिणीः᳚ - [  ] \newline

\textbf{Pada Paata} \newline

चा॒तय॑मानाः ॥ अ॒श्व॒त्थे । वः॒ । नि॒षद॑न॒मिति॑ नि - सद॑नम् । प॒र्णे । वः॒ । व॒स॒तिः । कृ॒ता ॥ गो॒भाज॒ इति॑ गो - भाजः॑ । इत् । किल॑ । अ॒स॒थ॒ । यत् । स॒नव॑थ । पूरु॑षम् ॥ यत् । अ॒हम् । वा॒जयन्न्॑ । इ॒माः । ओष॑धीः । हस्ते᳚ । आ॒द॒ध इत्या᳚-द॒धे ॥ आ॒त्मा । यक्ष्म॑स्य । न॒श्य॒ति॒ । पु॒रा । जी॒व॒गृभ॒ इति॑ जीव - गृभः॑ । य॒था॒ ॥ यत् । ओष॑धयः । सं॒गच्छ॑न्त॒ इति॑ सं - गच्छ॑न्ते । राजा॑नः । समि॑ता॒विति॒ सं - इ॒तौ॒ । इ॒व॒ ॥ विप्रः॑ । सः । उ॒च्य॒ते॒ । भि॒षक् । र॒क्षो॒हेति॑ रक्षः - हा । अ॒मी॒व॒चात॑न॒ इत्य॑मीव - चात॑नः ॥ निष्कृ॑ति॒रिति॒ निः-कृ॒तिः॒ । नाम॑ । वः॒ । मा॒ता । अथ॑ । यू॒यम् । स्थ॒ । संकृ॑ती॒रिति॒ सं - कृ॒तीः॒ ॥ स॒राः । प॒त॒त्रिणीः᳚ ।  \newline




\markright{ TS 4.2.6.3  \hfill https://www.vedavms.in \hfill}
\addcontentsline{toc}{section}{ TS 4.2.6.3 }
\section*{ TS 4.2.6.3 }

\textbf{TS 4.2.6.3 } \newline
\textbf{Samhita Paata} \newline

स्थन॒ यदा॒ मय॑ति॒ निष्कृ॑त ॥अ॒न्या वो॑ अ॒न्याम॑व-त्व॒न्याऽन्यस्या॒ उपा॑वत । ताः सर्वा॒ ओष॑धयः संॅविदा॒ना इ॒दं मे॒ प्राव॑ता॒ वचः॑ ॥ उच्छुष्मा॒ ओष॑धीनां॒ गावो॑ गो॒ष्ठा दि॑वेरते । धनꣳ॑ सनि॒ष्यन्ती॑ नामा॒त्मानं॒ तव॑ पूरुष ॥ अति॒ विश्वाः᳚ परि॒ष्ठास्ते॒न इ॑व व्र॒जम॑क्रमुः । ओष॑धयः॒ प्राचु॑च्यवु॒ र्यत् किं च॑ त॒नुवाꣳ॒॒ रपः॑ ॥ या - [  ] \newline

\textbf{Pada Paata} \newline

स्थ॒न॒ । यत् । आ॒मय॑ति । निरिति॑ । कृ॒त॒ ॥ अ॒न्या । वः॒ । अ॒न्याम् । अ॒व॒तु॒ । अ॒न्या । अ॒न्यस्याः᳚ । उपेति॑ । अ॒व॒त॒ ॥ ताः । सर्वाः᳚ । ओष॑धयः । सं॒ॅवि॒दा॒ना इति॑ सं - वि॒दा॒नाः । इ॒दम् । मे॒ । प्रेति॑ । अ॒व॒त॒ । वचः॑ ॥ उदिति॑ । शुष्माः᳚ । ओष॑धीनां । गावः॑ । गो॒ष्ठादिति॑ गो - स्थात् । इ॒व॒ । ई॒र॒ते॒ ॥ धन᳚म् । स॒नि॒ष्यन्ती॑नाम् । आ॒त्मान᳚म् । तव॑ । पू॒रु॒ष॒ ॥ अतीति॑ । विश्वाः᳚ । प॒रि॒ष्ठा इति॑ परि - स्थाः । स्ते॒नः । इ॒व॒ । व्र॒जम् । अ॒क्र॒मुः॒ ॥ ओष॑धयः । प्रेति॑ । अ॒चु॒च्य॒वुः॒ । यत् । किम् । च॒ । त॒नुवा᳚म् । रपः॑ ॥ याः ।  \newline




\markright{ TS 4.2.6.4  \hfill https://www.vedavms.in \hfill}
\addcontentsline{toc}{section}{ TS 4.2.6.4 }
\section*{ TS 4.2.6.4 }

\textbf{TS 4.2.6.4 } \newline
\textbf{Samhita Paata} \newline

-स्त॑ आत॒स्थु-रा॒त्मानं॒ ॅया आ॑विवि॒शुः परुः॑ परुः ।तास्ते॒ यक्ष्मं॒ ॅविबा॑धन्ता मु॒ग्रो म॑द्ध्यम॒शीरि॑व ॥सा॒कं ॅय॑क्ष्म॒ प्र प॑त श्ये॒नेन॑ किकिदी॒विना᳚ । सा॒कं ॅवात॑स्य॒-ध्राज्या॑ सा॒कं न॑श्य नि॒हाक॑या ॥ अ॒श्वा॒व॒तीꣳ सो॑मव॒ती मू॒र्जय॑न्ती॒ मुदो॑जसं । आ वि॑थ्सि॒ सर्वा॒ ओष॑धीर॒स्मा अ॑रि॒ष्टता॑तये ॥ याः फ॒लिनी॒र्या अ॑फ॒ला अ॑पु॒ष्पा याश्च॑ पु॒ष्पिणीः᳚ । बृह॒स्पति॑ प्रसूता॒ स्तानो॑ मुञ्च॒न्त्वꣳ ह॑सः ॥ या - [  ] \newline

\textbf{Pada Paata} \newline

ते । आ॒त॒स्थुरित्या᳚ - त॒स्थुः । आ॒त्मान᳚म् । याः । आ॒वि॒वि॒शुरित्या᳚-वि॒वि॒शुः । परुः॑ परु॒रिति॒ परुः॑ - प॒रुः॒ ॥ ताः । ते॒ । यक्ष्म᳚म् । वीति॑ । बा॒ध॒न्ता॒म् । उ॒ग्रः । म॒द्ध्य॒म॒शीरिति॑ मद्ध्यम - शीः । इ॒व॒ ॥ सा॒कम् । य॒क्ष्म॒ । प्रेति॑ । प॒त॒ । श्ये॒नेन॑ । कि॒कि॒दी॒विना᳚ ॥ सा॒कम् । वात॑स्य । ध्राज्या᳚ । सा॒कम् । न॒श्य॒ । नि॒हाक॒येति॑ नि - हाक॑या ॥ अ॒श्वा॒व॒तीमित्य॑श्व - व॒तीम् । सो॒म॒व॒तीमिति॑ सोम - व॒तीम् । ऊ॒र्जय॑न्तीम् । उदो॑जस॒मित्युत् - ओ॒ज॒स॒म् ॥ एति॑ । वि॒थ्सि॒ । सर्वाः᳚ । ओष॑धीः । अ॒स्मै । अ॒रि॒ष्टता॑तय॒ इत्य॑रि॒ष्ट - ता॒त॒ये॒ ॥ याः । फ॒लिनीः᳚ । याः । अ॒फ॒लाः । अ॒पु॒ष्पाः । याः । च॒ । पु॒ष्पिणीः᳚ ॥ बृह॒स्पति॑प्रसूता॒ इति॒ बृह॒स्पति॑ - प्र॒सू॒ताः॒ । ताः । नः॒ । मु॒ञ्च॒न्तु॒ । अꣳह॑सः ॥ याः ।  \newline




\markright{ TS 4.2.6.5  \hfill https://www.vedavms.in \hfill}
\addcontentsline{toc}{section}{ TS 4.2.6.5 }
\section*{ TS 4.2.6.5 }

\textbf{TS 4.2.6.5 } \newline
\textbf{Samhita Paata} \newline

ओष॑धयः॒ सोम॑राज्ञीः॒ प्रवि॑ष्टाः पृथि॒वीमनु॑ ।तासां॒ त्वम॑स्युत्त॒मा प्रणो॑ जी॒वात॑वे-सुव ॥ अ॒व॒पत॑न्तीरवदन् दि॒व ओष॑दयः॒ परि॑ । यं जी॒व म॒श्ञवा॑ महै॒ न स रि॑ष्याति॒ पूरु॑षः ॥ याश्चे॒द मु॑प-शृ॒ण्वन्ति॒ याश्च॑ दू॒रं परा॑गताः ।इ॒ह स॒ङ्गत्य॒ ताः सर्वा॑ अ॒स्मै सं द॑त्त भेष॒जं ॥मा वो॑ रिषत् खनि॒ता यस्मै॑ चा॒हं खना॑मि वः ( ) । द्वि॒प-च्चतु॑ष्प-द॒स्माकꣳ॒॒ सर्व॑-म॒स्त्वना॑तुरं ॥ ओष॑धयः॒ सं ॅव॑दन्ते॒ सोमे॑न स॒ह राज्ञा᳚ । यस्मै॑ क॒रोति॑ ब्राह्म॒णस्तꣳ रा॑जन् पारयामसि ॥ \newline

\textbf{Pada Paata} \newline

ओष॑धयः । सोम॑राज्ञी॒रिति॒ सोम॑ - रा॒ज्ञीः॒ । प्रवि॑ष्टा॒ इति॒ प्र-वि॒ष्टाः॒ । पृ॒थि॒वीम् । अनु॑ ॥ तासां᳚ । त्वम् । अ॒सि॒ । उ॒त्त॒मेत्यु॑त्- त॒मा । प्रेति॑ । नः॒ । जी॒वात॑वे । सु॒व॒ ॥ अ॒व॒पत॑न्ती॒रित्य॑व - पत॑न्तीः । अ॒व॒द॒न्न् । दि॒वः । ओष॑दयः । परि॑ ॥ यम् । जी॒वम् । अ॒श्नवा॑महै । न । सः । रि॒ष्या॒ति॒ । पूरु॑षः ॥ याः । च॒ । इ॒दम् । उ॒प॒शृ॒ण्वन्तीत्यु॑प-शृ॒ण्वन्ति॑ । याः । च॒ । दू॒रम् । परा॑गता॒ इति॒ परा᳚ - ग॒ताः॒ ॥ इ॒ह । स॒ङ्गत्येति॑ सं - गत्य॑ । ताः । सर्वाः᳚ । अ॒स्मै । समिति॑ । द॒त्त॒ । भे॒ष॒जम् ॥ मा । वः॒ । रि॒ष॒त् । ख॒नि॒ता । यस्मै᳚ । च॒ । अ॒हम् । खना॑मि । वः॒ ( ) ॥ द्वि॒पदिति॑ द्वि-पत् । चतु॑ष्प॒दिति॒ चतुः॑-प॒त् । अ॒स्माक᳚म् । सर्व᳚म् । अ॒स्तु॒ । अना॑तुर॒मित्यना᳚ - तु॒र॒म् ॥ ओष॑धयः । समिति॑ । व॒द॒न्ते॒ । सोमे॑न । स॒ह । राज्ञा᳚ ॥ यस्मै᳚ । क॒रोति॑ । ब्रा॒ह्म॒णः । तम् । रा॒ज॒न्न् । पा॒र॒या॒म॒सि॒ ॥  \newline




\markright{ TS 4.2.7.1  \hfill https://www.vedavms.in \hfill}
\addcontentsline{toc}{section}{ TS 4.2.7.1 }
\section*{ TS 4.2.7.1 }

\textbf{TS 4.2.7.1 } \newline
\textbf{Samhita Paata} \newline

मा नो॑ हिꣳसीज्जनि॒ता यः पृ॑थि॒व्या यो वा॒ दिवꣳ॑ स॒त्यध॑र्मा ज॒जान॑ । यश्चा॒पश्च॒न्द्रा बृ॑ह॒तीर्ज॒जान॒ कस्मै॑ दे॒वाय॑ ह॒विषा॑ विधेम ॥अ॒भ्याव॑र्तस्व पृथिवि य॒ज्ञेन॒ पय॑सा स॒ह । व॒पां ते॑ अ॒ग्निरि॑षि॒तोऽव॑ सर्पतु ॥ अग्ने॒ यत् ते॑ शु॒क्रं ॅयच्च॒न्द्रं ॅयत् पू॒तं ॅयद्-य॒ज्ञियं᳚ । तद्-दे॒वेभ्यो॑ भरामसि ॥ इष॒मूर्ज॑म॒हमि॒त आ - [  ] \newline

\textbf{Pada Paata} \newline

मा । नः॒ । हिꣳ॒॒सी॒त् । ज॒नि॒ता । यः । पृ॒थि॒व्याः । यः । वा॒ । दिव᳚म् । स॒त्यध॒र्मेति॑ स॒त्य - ध॒र्मा॒ । ज॒जान॑ ॥ यः । च॒ । अ॒पः । च॒न्द्राः । बृ॒ह॒तीः । ज॒जान॑ । कस्मै᳚ । दे॒वाय॑ । ह॒विषा᳚ । वि॒धे॒म॒ ॥ अ॒भ्याव॑र्त॒स्वेत्य॑भि - आव॑र्तस्व । पृ॒थि॒वि॒ । य॒ज्ञेन॑ । पय॑सा । स॒ह ॥ व॒पाम् । ते॒ । अ॒ग्निः । इ॒षि॒तः । अवेति॑ । स॒र्प॒तु॒ ॥ अग्ने᳚ । यत् । ते॒ । शु॒क्रम् । यत् । च॒न्द्रम् । यत् । पू॒तम् । यत् । य॒ज्ञिय᳚म् ॥ तत् । दे॒वेभ्यः॑ । भ॒रा॒म॒सि॒ ॥ इष᳚म् । ऊर्ज᳚म् । अ॒हम् । इ॒तः । एति॑ ।  \newline




\markright{ TS 4.2.7.2  \hfill https://www.vedavms.in \hfill}
\addcontentsline{toc}{section}{ TS 4.2.7.2 }
\section*{ TS 4.2.7.2 }

\textbf{TS 4.2.7.2 } \newline
\textbf{Samhita Paata} \newline

द॑द ऋ॒तस्य॒ धाम्नो॑ अ॒मृत॑स्य॒ योनेः᳚ । आ नो॒ गोषु॑ विश॒त्वौष॑धीषु॒ जहा॑मि से॒दिमनि॑रा॒ममी॑वां ॥ अग्ने॒ तव॒ श्रवो॒ वयो॒ महि॑ भ्राजन्त्य॒र्चयो॑ विभावसो ।बृह॑द्-भानो॒ शव॑सा॒ वाज॑मु॒क्थ्यं॑ दधा॑सि दा॒शुषे॑ कवे ॥ इ॒र॒ज्यन्न॑ग्ने प्रथयस्व ज॒न्तुभि॑र॒स्मे रायो॑ अमर्त्य । स द॑र्.श॒तस्य॒ वपु॑षो॒ वि रा॑जसि पृ॒णक्षि॑ सान॒सिꣳ र॒यिं ॥ ऊर्जो॑ नपा॒ज्जात॑वेदः सुश॒स्तिभि॒-र्मन्द॑स्व - [  ] \newline

\textbf{Pada Paata} \newline

द॒दे॒ । ऋ॒तस्य॑ । धाम्नः॑ । अ॒मृत॑स्य । योनेः᳚ ॥ एति॑ । नः॒ । गोषु॑ । वि॒श॒तु॒ । एति॑ । ओष॑धीषु । जहा॑मि । से॒दिम् । अनि॑राम् । अमी॑वाम् ॥ अग्ने᳚ । तव॑ । श्रवः॑ । वयः॑ । महि॑ । भ्रा॒ज॒न्ति॒ । अ॒र्चयः॑ । वि॒भा॒व॒सो॒ इति॑ विभा - व॒सो॒ ॥ बृह॑द्भानो॒ इति॒ बृह॑त् - भा॒नो॒ । शव॑सा । वाज᳚म् । उ॒क्थ्य᳚म् । दधा॑सि । दा॒शुषे᳚ । क॒वे॒ ॥ इ॒र॒ज्यन्न् । अ॒ग्ने॒ । प्र॒थ॒य॒स्व॒ । ज॒न्तुभि॒रिति॑ ज॒न्तु-भिः॒ । अ॒स्मे इति॑ । रायः॑ । अ॒म॒र्त्य॒ ॥ सः । द॒र्॒.श॒तस्य॑ । वपु॑षः । वीति॑ । रा॒ज॒सि॒ । पृ॒णक्षि॑ । सा॒न॒सिम् । र॒यिम् ॥ ऊर्जः॑ । न॒पा॒त् । जात॑वेद॒ इति॒ जात॑ - वे॒दः॒ । सु॒श॒स्तिभि॒रिति॑ सुश॒स्ति - भिः॒ । मन्द॑स्व ।  \newline




\markright{ TS 4.2.7.3  \hfill https://www.vedavms.in \hfill}
\addcontentsline{toc}{section}{ TS 4.2.7.3 }
\section*{ TS 4.2.7.3 }

\textbf{TS 4.2.7.3 } \newline
\textbf{Samhita Paata} \newline

धी॒तिभि॑र्.हि॒तः । त्वे इषः॒ सं द॑धु॒-र्भूरि॑रेतस-श्चि॒त्रो त॑यो वा॒मजा॑ताः ॥ पा॒व॒कव॑र्चाः शु॒क्रव॑र्चा॒ अनू॑नवर्चा॒ उदि॑यर्.षि भा॒नुना᳚ । पु॒त्रः पि॒तरा॑ वि॒चर॒न्नुपा॑वस्यु॒भे पृ॑णक्षि॒ रोद॑सी ॥ ऋ॒तावा॑नं महि॒षं ॅवि॒श्वच॑र्.षणिम॒ग्निꣳ सु॒म्नाय॑ दधिरे पु॒रो जनाः᳚ । श्रुत्क॑र्णꣳ स॒प्रथ॑स्तमं त्वा गि॒रा दैव्यं॒ मानु॑षा यु॒गा ॥ नि॒ष्क॒र्तार॑-मद्ध्व॒रस्य॒ प्रचे॑तसं॒ क्षय॑न्तꣳ॒॒ राध॑से म॒हे । रा॒तिं भृगू॑णामु॒शिजं॑ क॒विक्र॑तुं पृ॒णक्षि॑ सान॒सिꣳ - [  ] \newline

\textbf{Pada Paata} \newline

धी॒तिभि॒रिति॑ धी॒ति-भिः॒ । हि॒तः ॥ त्वे इति॑ । इषः॑ । समिति॑ । द॒धुः॒ । भूरि॑रेतस॒ इति॒ भूरि॑ - रे॒त॒सः॒ । चि॒त्रोत॑य॒ इति॑ चि॒त्र - ऊ॒त॒यः॒ । वा॒मजा॑ता॒ इति॑ वा॒म - जा॒ताः॒ ॥ पा॒व॒कव॑र्चा॒ इति॑ पाव॒क - व॒र्चाः॒ । शु॒क्रव॑र्चा॒ इति॑ शु॒क्र - व॒र्चाः॒ । अनू॑नवर्चा॒ इत्यनू॑न - व॒र्चाः॒ । उदिति॑ । इ॒य॒र्॒.षि॒ । भा॒नुना᳚ ॥ पु॒त्रः । पि॒तरा᳚ । वि॒चर॒न्निति॑ वि-चरन्न्॑ । उपेति॑ । अ॒व॒सि॒ । उ॒भे इति॑ । पृ॒ण॒क्षि॒ । रोद॑सी॒ इति॑ ॥ ऋ॒तावा॑न॒मित्यृ॒त - वा॒न॒म् । म॒हि॒षम् । वि॒श्वच॑र्.षणि॒मिति॑ वि॒श्व - च॒र्॒.ष॒णि॒म् । अ॒ग्निम् । सु॒म्नाय॑ । द॒धि॒रे॒ । पु॒रः । जनाः᳚ ॥ श्रुत्क॑र्ण॒मिति॒ श्रुत्-क॒र्ण॒म् । स॒प्रथ॑स्तम॒मिति॑ स॒प्रथः॑-त॒म॒म् । त्वा॒ । गि॒रा । दैव्य᳚म् । मानु॑षा । यु॒गा ॥ नि॒ष्क॒र्तार॒मिति॑ निः - क॒र्तार᳚म् । अ॒द्ध्व॒रस्य॑ । प्रचे॑तस॒मिति॒ प्र - चे॒त॒स॒म् । क्षय॑न्तम् । राध॑से । म॒हे ॥ रा॒तिम् । भृगू॑णाम् । उ॒शिज᳚म् । क॒विक्र॑तु॒मिति॑ क॒वि - क्र॒तु॒म् । पृ॒णक्षि॑ । सा॒न॒सिम् ।  \newline




\markright{ TS 4.2.7.4  \hfill https://www.vedavms.in \hfill}
\addcontentsline{toc}{section}{ TS 4.2.7.4 }
\section*{ TS 4.2.7.4 }

\textbf{TS 4.2.7.4 } \newline
\textbf{Samhita Paata} \newline

र॒यिं ॥ चितः॑ स्थ परि॒चित॑ ऊर्द्ध्व॒चितः॑ श्रयद्ध्वं॒ तया॑ दे॒वत॑याऽङ्गिर॒स्वद्-ध्रु॒वाः सी॑दत ॥ आ प्या॑यस्व॒ समे॑तु ते वि॒श्वतः॑ सोम॒ वृष्णि॑यं । भवा॒ वाज॑स्य सङ्ग॒थे ॥ सं ते॒ पयाꣳ॑सि॒ समु॑ यन्तु॒ वाजाः॒ सं ॅवृष्णि॑या-न्यभिमाति॒षाहः॑ । आ॒प्याय॑मानो अ॒मृता॑य सोम दि॒वि श्रवाꣳ॑स्युत्त॒मानि॑ धिष्व ॥ \newline

\textbf{Pada Paata} \newline

र॒यिम् ॥ चितः॑ । स्थ॒ । प॒रि॒चित॒ इति॑ परि - चितः॑ । ऊ॒द्‌र्ध्व॒चित॒ इत्यू᳚द्‌र्ध्व - चितः॑ । श्र॒य॒द्ध्व॒म् । तया᳚ । दे॒वत॑या । अ॒ङ्गि॒र॒स्वत् । ध्रु॒वाः । सी॒द॒त॒ ॥ एति॑ । प्या॒य॒स्व॒ । समिति॑ । ए॒तु॒ । ते॒ । वि॒श्वतः॑ । सो॒म॒ । वृष्णि॑यम् ॥ भव॑ । वाज॑स्य । स॒ङ्ग॒थ इति॑ सं-ग॒थे ॥ समिति॑ । ते॒ । पयाꣳ॑सि । समिति॑ । उ॒ । य॒न्तु॒ । वाजाः᳚ । समिति॑ । वृष्णि॑यानि । अ॒भि॒मा॒ति॒षाह॒ इत्य॑भिमाति - साहः॑ ॥ आ॒प्याय॑मान॒ इत्या᳚ - प्याय॑मानः । अ॒मृता॑य । सो॒म॒ । दि॒वि । श्रवाꣳ॑सि । उ॒त्त॒मानीत्यु॑त् - त॒मानि॑ । धि॒ष्व॒ ॥  \newline




\markright{ TS 4.2.8.1  \hfill https://www.vedavms.in \hfill}
\addcontentsline{toc}{section}{ TS 4.2.8.1 }
\section*{ TS 4.2.8.1 }

\textbf{TS 4.2.8.1 } \newline
\textbf{Samhita Paata} \newline

अ॒भ्य॑स्था॒द्-विश्वाः॒ पृत॑ना॒ अरा॑ती॒स्तद॒ग्निरा॑ह॒ तदु॒ सोम॑ आह । बृह॒स्पतिः॑ सवि॒ता तन्म॑ आह पू॒षा मा॑ऽधाथ् सुकृ॒तस्य॑ लो॒के ॥ यदक्र॑न्दः प्रथ॒मं जाय॑मान उ॒द्यन्थ् स॑मु॒द्रादु॒त वा॒ पुरी॑षात् । श्ये॒नस्य॑ प॒क्षा ह॑रि॒णस्य॑ बा॒हू उप॑स्तुतं॒ जनि॑म॒ तत् ते॑ अर्वन्न् ॥ अ॒पां पृ॒ष्ठम॑सि॒ योनि॑र॒ग्नेः स॑मु॒द्रम॒भितः॒ पिन्व॑मानं । वर्द्ध॑मानं म॒ह - [  ] \newline

\textbf{Pada Paata} \newline

अ॒भीति॑ । अ॒स्था॒त् । विश्वाः᳚ । पृत॑नाः । अरा॑तीः । तत् । अ॒ग्निः । आ॒ह॒ । तत् । उ॒ । सोमः॑ । आ॒ह॒ ॥ बृह॒स्पतिः॑ । स॒वि॒ता । तत् । मे॒ । आ॒ह॒ । पू॒षा । मा॒ । अ॒धा॒त् । सु॒कृ॒तस्येति॑ सु - कृ॒तस्य॑ । लो॒के ॥ यत् । अक्र॑न्दः । प्र॒थ॒मम् । जाय॑मानः । उ॒द्यन्नित्यु॑त्- यन्न् । स॒मु॒द्रात् । उ॒त । वा॒ । पुरी॑षात् ॥ श्ये॒नस्य॑ । प॒क्षा । ह॒रि॒णस्य॑ । बा॒हू इति॑ । उप॑स्तुत॒मित्युप॑ - स्तु॒त॒म् । जनि॑म । तत् । ते॒ । अ॒र्व॒न्न् ॥ अ॒पाम् । पृ॒ष्ठम् । अ॒सि॒ । योनिः॑ । अ॒ग्नेः । स॒मु॒द्रम् । अ॒भितः॑ । पिन्व॑मानम् ॥ वद्‌र्ध॑मानम् । म॒हः ।  \newline




\markright{ TS 4.2.8.2  \hfill https://www.vedavms.in \hfill}
\addcontentsline{toc}{section}{ TS 4.2.8.2 }
\section*{ TS 4.2.8.2 }

\textbf{TS 4.2.8.2 } \newline
\textbf{Samhita Paata} \newline

आ च॒ पुष्क॑रं दि॒वो मात्र॑या वरि॒णा प्र॑थस्व ॥ ब्रह्म॑ जज्ञा॒नं प्र॑थ॒मं पु॒रस्ता॒द्वि सी॑म॒तः सु॒रुचो॑ वे॒न आ॑वः । स बु॒द्ध्निया॑ उप॒मा अ॑स्य वि॒ष्ठाः स॒तश्च॒ योनि॒मस॑तश्च॒ विवः॑ ॥ हि॒र॒ण्य॒ग॒र्भः सम॑वर्त॒ताग्रे॑ भू॒तस्य॑ जा॒तः पति॒रेक॑ आसीत् । स दा॑धार पृथि॒वीं द्यामु॒तेमां कस्मै॑ दे॒वाय॑ ह॒विषा॑ विधेम ॥ द्र॒फ्सश्च॑स्कन्द पृथि॒वीमनु॒ - [  ] \newline

\textbf{Pada Paata} \newline

एति॑ । च॒ । पुष्क॑रम् । दि॒वः । मात्र॑या । व॒रि॒णा । प्र॒थ॒स्व॒ ॥ ब्रह्म॑ । ज॒ज्ञा॒नम् । प्र॒थ॒मम् । पु॒रस्ता᳚त् । वीति॑ । सी॒म॒तः । सु॒रुच॒ इति॑ सु - रुचः॑ । वे॒नः । आ॒वः॒ ॥ सः । बु॒द्ध्नियाः᳚ । उ॒प॒मा इत्युप॑-माः । अ॒स्य॒ । वि॒ष्ठा इति॑ वि-स्थाः । स॒तः । च॒ । योनि᳚म् । अस॑तः । च॒ । विवः॑ ॥ हि॒र॒ण्य॒ग॒र्भ इति॑ हिरण्य - ग॒र्भः । समिति॑ । अ॒व॒र्त॒त । अग्रे᳚ । भू॒तस्य॑ । जा॒तः । पतिः॑ । एकः॑ । आ॒सी॒त् ॥ सः । दा॒धा॒र॒ । पृ॒थि॒वीम् । द्याम् । उ॒त । इ॒माम् । कस्मै᳚ । दे॒वाय॑ । ह॒विषा᳚ । वि॒धे॒म॒ ॥ द्र॒फ्सः । च॒स्क॒न्द॒ । पृ॒थि॒वीम् । अन्विति॑ ।  \newline




\markright{ TS 4.2.8.3  \hfill https://www.vedavms.in \hfill}
\addcontentsline{toc}{section}{ TS 4.2.8.3 }
\section*{ TS 4.2.8.3 }

\textbf{TS 4.2.8.3 } \newline
\textbf{Samhita Paata} \newline

-द्यामि॒मं च॒ योनि॒मनु॒ यश्च॒ पूर्वः॑ । तृ॒तीयं॒ ॅयोनि॒मनु॑ स॒ञ्चर॑न्तं द्र॒फ्सं जु॑हो॒म्यनु॑ स॒प्त होत्राः᳚ ॥ नमो॑ अस्तु स॒र्पेभ्यो॒ ये के च॑ पृथि॒वीमनु॑ । ये अ॒न्तरि॑क्षे॒ ये दि॒वि तेभ्यः॑ स॒र्पेभ्यो॒ नमः॑ ॥ ये॑ऽदो रो॑च॒ने दि॒वो ये वा॒ सूर्य॑स्य र॒श्मिषु॑ । येषा॑म॒फ्सु सदः॑ कृ॒तं तेभ्यः॑ स॒र्पेभ्यो॒ नमः॑ ॥ या इष॑वो यातु॒ धाना॑नां॒ ( ) ॅये॑ वा॒ वन॒स्पतीꣳ॒॒रनु॑ । ये वा॑ऽव॒टेषु॒ शेर॑ते॒ तेभ्यः॑ स॒र्पेभ्यो॒ नमः॑ ॥ \newline

\textbf{Pada Paata} \newline

द्याम् । इ॒मम् । च॒ । योनि᳚म् । अन्विति॑ । यः । च॒ । पूर्वः॑ ॥ तृ॒तीय᳚म् । योनि᳚म् । अन्विति॑ । स॒ञ्चर॑न्त॒मिति॑ सं - चर॑न्तम् । द्र॒फ्सम् । जु॒हो॒मि॒ । अन्विति॑ । स॒प्त । होत्राः᳚ ॥ नमः॑ । अ॒स्तु॒ । स॒र्पेभ्यः॑ । ये । के । च॒ । पृ॒थि॒वीम् । अनु॑ ॥ ये । अ॒न्तरि॑क्षे । ये । दि॒वि । तेभ्यः॑ । स॒र्पेभ्यः॑ । नमः॑ ॥ ये । अ॒दः । रो॒च॒ने । दि॒वः । ये । वा॒ । सूर्य॑स्य । र॒श्मिषु॑ ॥ येषा᳚म् । अ॒फ्स्वित्य॑प्- सु । सदः॑ । कृ॒तम् । तेभ्यः॑ । स॒र्पेभ्यः॑ । नमः॑ ॥ याः । इष॑वः । या॒तु॒धाना॑ना॒मिति॑ यातु - धाना॑नाम् ( ) । ये । वा॒ । वन॒स्पतीन्॑ । अनु॑ ॥ ये । वा॒ । अ॒व॒टेषु॑ । शेर॑ते । तेभ्यः॑ । स॒र्पेभ्यः॑ । नमः॑ ॥  \newline




\markright{ TS 4.2.9.1  \hfill https://www.vedavms.in \hfill}
\addcontentsline{toc}{section}{ TS 4.2.9.1 }
\section*{ TS 4.2.9.1 }

\textbf{TS 4.2.9.1 } \newline
\textbf{Samhita Paata} \newline

ध्रु॒वाऽसि॑ ध॒रुणाऽस्तृ॑ता वि॒श्वक॑र्मणा॒ सुकृ॑ता । मा त्वा॑ समु॒द्र उद्व॑धी॒न्मा सु॑प॒र्णो व्य॑थमाना पृथि॒वीं दृꣳ॑ह ॥ प्र॒जाप॑तिस्त्वा सादयतु पृथि॒व्याः पृ॒ष्ठे व्यच॑स्वतीं॒ प्रथ॑स्वतीं॒ प्रथो॑ऽसि पृथि॒व्य॑सि॒ भूर॑सि॒ भूमि॑र॒स्यदि॑तिरसि वि॒श्वधा॑या॒ विश्व॑स्य॒ भुव॑नस्य ध॒र्त्री पृ॑थि॒वीं ॅय॑च्छ पृथि॒वीं दृꣳ॑ह पृथि॒वीं मा हिꣳ॑सी॒र्विश्व॑स्मै प्रा॒णाया॑पा॒नाय॑ व्या॒नायो॑दा॒नाय॑ प्रति॒ष्ठायै॑ - [  ] \newline

\textbf{Pada Paata} \newline

ध्रु॒वा । अ॒सि॒ । ध॒रुणा᳚ । अस्तृ॑ता । वि॒श्वक॑र्म॒णेति॑ वि॒श्व - क॒र्म॒णा॒ । सुकृ॒तेति॒ सु - कृ॒ता॒ ॥ मा । त्वा॒ । स॒मु॒द्रः । उदिति॑ । व॒धी॒त् । मा । सु॒प॒र्ण इति॑ सु - प॒र्णः । अव्य॑थमाना । पृ॒थि॒वीम् । दृꣳ॒॒ह॒ ॥ प्र॒जाप॑ति॒रिति॑ प्र॒जा - प॒तिः॒ । त्वा॒ । सा॒द॒य॒तु॒ । पृ॒थि॒व्याः । पृ॒ष्ठे । व्यच॑स्वतीम् । प्रथ॑स्वतीम् । प्रथः॑ । अ॒सि॒ । पृ॒थि॒वी । अ॒सि॒ । भूः । अ॒सि॒ । भूमिः॑ । अ॒सि॒ । अदि॑तिः । अ॒सि॒ । वि॒श्वधा॑या॒ इति॑ वि॒श्व - धा॒याः॒ । विश्व॑स्य । भुव॑नस्य । ध॒र्त्री । पृ॒थि॒वीम् । य॒च्छ॒ । पृ॒थि॒वीम् । दृꣳ॒॒ह॒ । पृ॒थि॒वीम् । मा । हिꣳ॒॒सीः॒ । विश्व॑स्मै । प्रा॒णायेति॑ प्रा - अ॒नाय॑ । अ॒पा॒नायेत्य॑प - अ॒नाय॑ । व्या॒नायेति॑ वि - अ॒नाय॑ । उ॒दा॒नायेत्यु॑त् - अ॒नाय॑ । प्र॒ति॒ष्ठाया॒ इति॑ प्रति - स्थायै᳚ ।  \newline




\markright{ TS 4.2.9.2  \hfill https://www.vedavms.in \hfill}
\addcontentsline{toc}{section}{ TS 4.2.9.2 }
\section*{ TS 4.2.9.2 }

\textbf{TS 4.2.9.2 } \newline
\textbf{Samhita Paata} \newline

च॒रित्रा॑या॒-ग्निस्त्वा॒ऽभि पा॑तु म॒ह्या स्व॒स्त्या छ॒र्दिषा॒ शन्त॑मेन॒ तया॑ दे॒वत॑याऽङ्गिर॒स्वद्-ध्रु॒वा सी॑द ॥ काण्डा᳚त् काण्डात् प्र॒रोह॑न्ती॒ परु॑षःपरुषः॒ परि॑ । ए॒वा नो॑ दूर्वे॒ प्र त॑नु स॒हस्रे॑ण श॒तेन॑ च ॥ या श॒तेन॑ प्रत॒नोषि॑ स॒हस्रे॑ण वि॒रोह॑सि ।तस्या᳚स्ते देवीष्टके वि॒धेम॑ ह॒विषा॑ व॒यं ॥ अषा॑ढाऽसि॒ सह॑माना॒ सह॒स्वारा॑तीः॒ सह॑स्वारातीय॒तः सह॑स्व॒ पृत॑नाः॒ सह॑स्व पृतन्य॒तः । स॒हस्र॑वीर्या - [  ] \newline

\textbf{Pada Paata} \newline

च॒रित्रा॑य । अ॒ग्निः । त्वा॒ । अ॒भीति॑ । पा॒तु॒ । म॒ह्या । स्व॒स्त्या । छ॒र्दिषा᳚ । शन्त॑मे॒नेति॒ शं - त॒मे॒न॒ । तया᳚ । दे॒वत॑या । अ॒ङ्गि॒र॒स्वत् । ध्रु॒वा । सी॒द॒ ॥ काण्डा᳚त्काण्डा॒दिति॒ काण्डात् - का॒ण्डा॒त् । प्र॒रोह॒न्तीति॑ प्र - रोह॑न्ती । परु॑षःपरुष॒ इति॒ परु॑षः-प॒रु॒षः॒ । परि॑ ॥ ए॒वा । नः॒ । दू॒र्वे॒ । प्रेति॑ । त॒नु॒ । स॒हस्रे॑ण । श॒तेन॑ । च॒ ॥ या । श॒तेन॑ । प्र॒त॒नोषीति॑ प्र - त॒नोषि॑ । स॒हस्रे॑ण । वि॒रोह॒सीति॑ वि-रोह॑सि ॥ तस्याः᳚ । ते॒ । दे॒वि॒ । इ॒ष्ट॒के॒ । वि॒धेम॑ । ह॒विषा᳚ । व॒यम् ॥ अषा॑ढा । अ॒सि॒ । सह॑माना । सह॑स्व । अरा॑तीः । सह॑स्व । अ॒रा॒ती॒य॒तः । सह॑स्व । पृत॑नाः । सह॑स्व । पृ॒त॒न्य॒तः ॥ स॒हस्र॑वी॒र्येति॑ स॒हस्र॑ - वी॒र्या॒ ।  \newline




\markright{ TS 4.2.9.3  \hfill https://www.vedavms.in \hfill}
\addcontentsline{toc}{section}{ TS 4.2.9.3 }
\section*{ TS 4.2.9.3 }

\textbf{TS 4.2.9.3 } \newline
\textbf{Samhita Paata} \newline

-सि॒ सा मा॑ जिन्व ॥ मधु॒ वाता॑ ऋताय॒ते मधु॑ क्षरन्ति॒ सिन्ध॑वः । माद्ध्वी᳚र्नः स॒न्त्वोष॑धीः ॥ मधु॒ नक्त॑मु॒तोषसि॒ मधु॑म॒त् पार्थि॑वꣳ॒॒ रजः॑ । मधु॒ द्यौर॑स्तु नः पि॒ता ॥ मधु॑मान् नो॒ वन॒स्पति॒-र्मधु॑माꣳ अस्तु॒ सूर्यः॑ । माद्ध्वी॒र्गावो॑ भवन्तु नः ॥ म॒ही द्यौः पृ॑थि॒वी च॑ न इ॒मं ॅय॒ज्ञ्ं मि॑मिक्षतां । पि॒पृ॒तां नो॒ भरी॑मभिः ॥ तद्-विष्णोः᳚ पर॒मं - [  ] \newline

\textbf{Pada Paata} \newline

अ॒सि॒ । सा । मा॒ । जि॒न्व॒ ॥ मधु॑ । वाताः᳚ । ऋ॒ता॒य॒त इत्यृ॑त - य॒ते । मधु॑ । क्ष॒र॒न्ति॒ । सिन्ध॑वः ॥ माद्ध्वीः᳚ । नः॒ । स॒न्तु॒ । ओष॑धीः ॥ मधु॑ । नक्त᳚म् । उ॒त । उ॒षसि॑ । मधु॑म॒दिति॒ मधु॑ - म॒त् । पार्थि॑वम् । रजः॑ ॥ मधु॑ । द्यौः । अ॒स्तु॒ । नः॒ । पि॒ता ॥ मधु॑मा॒निति॒ मधु॑-मा॒न् । नः॒ । वन॒स्पतिः॑ । मधु॑मा॒निति॒ मधु॑ - मा॒न् । अ॒स्तु॒ । सूर्यः॑ ॥ माद्ध्वीः᳚ । गावः॑ । भ॒व॒न्तु॒ । नः॒ ॥ म॒ही । द्यौः । पृ॒थि॒वी । च॒ । नः॒ । इ॒मम् । य॒ज्ञ्म् । मि॒मि॒क्ष॒ता॒म् ॥ पि॒पृ॒ताम् । नः॒ । भरी॑मभि॒रिति॒ भरी॑म - भिः॒ ॥ तत् । विष्णोः᳚ । प॒र॒मम् ।  \newline




\markright{ TS 4.2.9.4  \hfill https://www.vedavms.in \hfill}
\addcontentsline{toc}{section}{ TS 4.2.9.4 }
\section*{ TS 4.2.9.4 }

\textbf{TS 4.2.9.4 } \newline
\textbf{Samhita Paata} \newline

प॒दꣳ सदा॑ पश्यन्ति सू॒रयः॑ । दि॒वीव॒ चक्षु॒रात॑तं ॥ ध्रु॒वाऽसि॑ पृथिवि॒ सह॑स्व पृतन्य॒तः । स्यू॒ता दे॒वेभि॑र॒मृते॒ना ऽऽगाः᳚ ॥ यास्ते॑ अग्ने॒ सूर्ये॒ रुच॑ उद्य॒तो दिव॑मात॒न्वन्ति॑ र॒श्मिभिः॑ । ताभिः॒ सर्वा॑भी रु॒चे जना॑य नस्कृधि ॥ या वो॑ देवाः॒ सूर्ये॒ रुचो॒ गोष्वश्वे॑षु॒ या रुचः॑ । इन्द्रा᳚ग्नी॒ ताभिः॒ सर्वा॑भी॒ रुचं॑ नो धत्त बृहस्पते ॥ वि॒राड् - [  ] \newline

\textbf{Pada Paata} \newline

प॒दम् । सदा᳚ । प॒श्य॒न्ति॒ । सू॒रयः॑ ॥ दि॒वि । इ॒व॒ । चक्षुः॑ । आत॑त॒मित्या-त॒त॒म् ॥ ध्रु॒वा । अ॒सि॒ । पृ॒थि॒वि॒ । सह॑स्व । पृ॒त॒न्य॒तः ॥ स्यू॒ता । दे॒वेभिः॑ । अ॒मृते॑न । एति॑ । अ॒गाः॒ ॥ याः । ते॒ । अ॒ग्ने॒ । सूर्ये᳚ । रुचः॑ । उ॒द्य॒त इत्यु॑त् - य॒तः । दिव᳚म् । आ॒त॒न्वन्तीत्या᳚ - त॒न्वन्ति॑ । र॒श्मिभि॒रिति॑ र॒श्मि - भिः॒ ॥ ताभिः॑ । सर्वा॑भः । रु॒चे । जना॑य । नः॒ । कृ॒धि॒ ॥ याः । वः॒ । दे॒वाः॒ । सूर्ये᳚ । रुचः॑ । गोषु॑ । अश्वे॑षु । याः । रुचः॑ ॥ इन्द्रा᳚ग्नी॒ इतीन्द्र॑ - अ॒ग्नी॒ । ताभिः॑ । सर्वा॑भः । रुच᳚म् । नः॒ । ध॒त्त॒ । बृ॒ह॒स्प॒ते॒ ॥ वि॒राडिति॑ वि - राट् ।  \newline




\markright{ TS 4.2.9.5  \hfill https://www.vedavms.in \hfill}
\addcontentsline{toc}{section}{ TS 4.2.9.5 }
\section*{ TS 4.2.9.5 }

\textbf{TS 4.2.9.5 } \newline
\textbf{Samhita Paata} \newline

ज्योति॑रधारयथ् स॒म्राड् ज्योति॑रधारयथ् स्व॒राड् ज्योति॑रधारयत् ॥ अग्ने॑ यु॒क्ष्वा हि ये तवाश्वा॑सो देव सा॒धवः॑ । अरं॒ ॅवह॑न्त्या॒शवः॑ ॥ यु॒क्ष्वा हि दे॑व॒हूत॑माꣳ॒॒ अश्वाꣳ॑ अग्ने र॒थीरि॑व । नि होता॑ पू॒र्व्यः स॑दः ॥ द्र॒फ्सश्च॑स्कन्द पृथि॒वीमनु॒ द्यामि॒मं च॒ योनि॒मनु॒ यश्च॒ पूर्वः॑ । तृ॒तीयं॒ ॅयोनि॒मनु॑ स॒ञ्चर॑न्तं द्र॒फ्सं जु॑हो॒म्यनु॑ स॒प्त - [  ] \newline

\textbf{Pada Paata} \newline

ज्योतिः॑ । अ॒धा॒र॒य॒त् । स॒म्राडिति॑ सम्-राट् । ज्योतिः॑ । अ॒धा॒र॒य॒त् । स्व॒राडिति॑ स्व - राट् । ज्योतिः॑ । अ॒धा॒र॒य॒त् ॥ अग्ने᳚ । यु॒क्ष्व । हि । ये । तव॑ । अश्वा॑सः । दे॒व॒ । सा॒धवः॑ ॥ अर᳚म् । वह॑न्ति । आ॒शवः॑ ॥ यु॒क्ष्व । हि । दे॒व॒हूत॑मा॒निति॑ देव - हूत॑मान् । अश्वान्॑ । अ॒ग्ने॒ । र॒थीः । इ॒व॒ ॥ नीति॑ । होता᳚ । पू॒र्व्यः । स॒दः॒ ॥ द्र॒फ्सः । च॒स्क॒न्द॒ । पृ॒थि॒वीम् । अन्विति॑ । द्याम् । इ॒मम् । च॒ । योनि᳚म् । अन्विति॑ । यः । च॒ । पूर्वः॑ ॥ तृ॒तीय᳚म् । योनि᳚म् । अन्विति॑ । स॒ञ्चर॑न्त॒मिति॑ सं - चर॑न्तम् । द्र॒फ्सम् । जु॒हो॒मि॒ । अन्विति॑ । स॒प्त ।  \newline




\markright{ TS 4.2.9.6  \hfill https://www.vedavms.in \hfill}
\addcontentsline{toc}{section}{ TS 4.2.9.6 }
\section*{ TS 4.2.9.6 }

\textbf{TS 4.2.9.6 } \newline
\textbf{Samhita Paata} \newline

होत्राः᳚ ॥ अभू॑दि॒दं ॅविश्व॑स्य॒ भुव॑नस्य॒ वाजि॑नम॒ग्ने-र्वै᳚श्वान॒रस्य॑ च । अ॒ग्निर्ज्योति॑षा॒ ज्योति॑ष्मान् रु॒क्मो वर्च॑सा॒ वर्च॑स्वान् ॥ ऋ॒चे त्वा॑ रु॒चे त्वा॒ समिथ् स्र॑वन्ति स॒रितो॒ न धेनाः᳚ । अ॒न्तर्.हृ॒दा मन॑सा पू॒यमा॑नाः ॥ घृ॒तस्य॒ धारा॑ अ॒भि चा॑कशीमि । हि॒र॒ण्ययो॑ वेत॒सो मद्ध्य॑ आसां ॥ तस्मिन्᳚थ्सुप॒र्णो म॑धु॒कृत् कु॑ला॒यी भज॑न्नास्ते॒ मधु॑ दे॒वता᳚भ्यः । तस्या॑ स ते॒ हर॑यः स॒प्त तीरे᳚ ( ) स्व॒धां दुहा॑ना अ॒मृत॑स्य॒ धारां᳚ ॥ \newline

\textbf{Pada Paata} \newline

होत्राः᳚ ॥ अभू᳚त् । इ॒दम् । विश्व॑स्य । भुव॑नस्य । वाजि॑नम् । अ॒ग्नेः । वै॒श्वा॒न॒रस्य॑ । च॒ ॥ अ॒ग्निः । ज्योति॑षा । ज्योति॑ष्मान् । रु॒क्मः । वर्च॑सा । वर्च॑स्वान् ॥ ऋ॒चे । त्वा॒ । रु॒चे । त्वा॒ । समिति॑ । इत् । स्र॒व॒न्ति॒ । स॒रितः॑ । न । धेनाः᳚ ॥ अ॒न्तः । हृ॒दा । मन॑सा । पू॒यमा॑नाः ॥ घृ॒तस्य॑ । धाराः᳚ । अ॒भीति॑ । चा॒क॒शी॒मि॒ ॥ हि॒र॒ण्ययः॑ । वे॒त॒सः । मद्ध्ये᳚ । आ॒सा॒म् ॥ तस्मिन्न्॑ । सु॒प॒र्ण इति॑ सु - प॒र्णः । म॒धु॒कृदिति॑ मधु - कृत् । कु॒ला॒यी । भजन्न्॑ । आ॒स्ते॒ । मधु॑ । दे॒वता᳚भ्यः ॥ तस्य॑ । आ॒स॒ते॒ । हर॑यः । स॒प्त । तीरे᳚ ( ) । स्व॒धामिति॑ स्व-धाम् । दुहा॑नाः । अ॒मृत॑स्य । धारा᳚म् ॥  \newline




\markright{ TS 4.2.10.1  \hfill https://www.vedavms.in \hfill}
\addcontentsline{toc}{section}{ TS 4.2.10.1 }
\section*{ TS 4.2.10.1 }

\textbf{TS 4.2.10.1 } \newline
\textbf{Samhita Paata} \newline

आ॒दि॒त्यं गर्भं॒ पय॑सा सम॒ञ्जन्थ् स॒हस्र॑स्य प्रति॒मां ॅवि॒श्वरू॑पं । परि॑ वृङ्ग्धि॒ हर॑सा॒ माऽभि मृ॑क्षः श॒तायु॑षं कृणुहि ची॒यमा॑नः ॥ इ॒मं मा हिꣳ॑सीर्द्वि॒पादं॑ पशू॒नाꣳ सह॑स्राक्ष॒ मेध॒ आ ची॒यमा॑नः । म॒युमा॑र॒ण्यमनु॑ ते दिशामि॒ तेन॑ चिन्वा॒नस्त॒नुवो॒ नि षी॑द ॥ वात॑स्य॒ ध्राजिं॒ ॅवरु॑णस्य॒ नाभि॒मश्वं॑ जज्ञा॒नꣳ स॑रि॒रस्य॒ मद्ध्ये᳚ । शिशुं॑ न॒दीनाꣳ॒॒ हरि॒मद्रि॑बुद्ध॒मग्ने॒ मा हिꣳ॑सीः - [  ] \newline

\textbf{Pada Paata} \newline

आ॒दि॒त्यम् । गर्भ᳚म् । पय॑सा । स॒म॒ञ्जन्निति॑ सं - अ॒ञ्जन्न् । स॒हस्र॑स्य । प्र॒ति॒मामिति॑ प्रति - माम् । वि॒श्वरू॑प॒मिति॑ वि॒श्व-रू॒प॒म् ॥ परीति॑ । वृ॒ङ्ग्धि॒ । हर॑सा । मा । अ॒भीति॑ । मृ॒क्षः॒ । श॒तायु॑ष॒मिति॑ श॒त - आ॒यु॒ष॒म् । कृ॒णु॒हि॒ । ची॒यमा॑नः ॥ इ॒मम् । मा । हिꣳ॒॒सीः॒ । द्वि॒पाद॒मिति॑ द्वि - पाद᳚म् । प॒शू॒नाम् । सह॑स्रा॒क्षेति॒ सह॑स्र - अ॒क्ष॒ । मेधे᳚ । एति॑ । ची॒यमा॑नः ॥ म॒युम् । आ॒र॒ण्यम् । अन्विति॑ । ते॒ । दि॒शा॒मि॒ । तेन॑ । चि॒न्वा॒नः । त॒नुवः॑ । नीति॑ । सी॒द॒ ॥ वात॑स्य । ध्राजि᳚म् । वरु॑णस्य । नाभि᳚म् । अश्व᳚म् । ज॒ज्ञा॒नम् । स॒रि॒रस्य॑ । मद्ध्ये᳚ ॥ शिशु᳚म् । न॒दीना᳚म् । हरि᳚म् । अद्रि॑बुद्ध॒मित्यद्रि॑ - बु॒द्ध॒म् । अग्ने᳚ । मा । हिꣳ॒॒सीः॒ ।  \newline




\markright{ TS 4.2.10.2  \hfill https://www.vedavms.in \hfill}
\addcontentsline{toc}{section}{ TS 4.2.10.2 }
\section*{ TS 4.2.10.2 }

\textbf{TS 4.2.10.2 } \newline
\textbf{Samhita Paata} \newline

पर॒मे व्यो॑मन्न् ॥ इ॒मं मा हिꣳ॑सी॒रेक॑शफं पशू॒नां क॑निक्र॒दं ॅवा॒जिनं॒ ॅवाजि॑नेषु । गौ॒रमा॑र॒ण्यमनु॑ ते दिशामि॒ तेन॑ चिन्वा॒नस्त॒नुवो॒ नि षी॑द ॥ अज॑स्र॒मिन्दु॑मरु॒षं भु॑र॒ण्युम॒ग्निमी॑डे पू॒र्वचि॑त्तौ॒ नमो॑भिः । स पर्व॑भिर्.ऋतु॒शः कल्प॑मानो॒ गां मा हिꣳ॑सी॒रदि॑तिं ॅवि॒राजं᳚ ॥ इ॒मꣳ स॑मु॒द्रꣳ श॒तधा॑र॒मुथ्-सं॑ ॅव्य॒च्यमा॑नं॒ भुव॑नस्य॒ मद्ध्ये᳚ । घृ॒तं दुहा॑ना॒मदि॑तिं॒ जना॒याग्ने॒ मा - [  ] \newline

\textbf{Pada Paata} \newline

प॒र॒मे । व्यो॑म॒न्निति॒ वि - ओ॒म॒न्न् ॥ इ॒मम् । मा । हिꣳ॒॒सीः॒ । एक॑शफ॒मित्येक॑ - श॒फ॒म् । प॒शू॒नाम् । क॒नि॒क्र॒दम् । वा॒जिन᳚म् । वाजि॑नेषु ॥ गौ॒रम् । आ॒र॒ण्यम् । अन्विति॑ । ते॒ । दि॒शा॒मि॒ । तेन॑ । चि॒न्वा॒नः । त॒नुवः॑ । नीति॑ । सी॒द॒ ॥ अज॑स्रम् । इन्दु᳚म् । अ॒रु॒षम् । भु॒र॒ण्युम् । अ॒ग्निम् । ई॒डे॒ । पू॒र्वचि॑त्ता॒विति॑ पू॒र्व - चि॒त्तौ॒ । नमो॑भि॒रिति॒ नमः॑ - भिः॒ ॥ सः । पर्व॑भि॒रिति॒ पर्व॑ - भिः॒ । ऋ॒तु॒श इत्यृ॑तु - शः । कल्प॑मानः । गाम् । मा । हिꣳ॒॒सीः॒ । अदि॑तिम् । वि॒राज॒मिति॑ वि - राज᳚म् ॥ इ॒मम् । स॒मु॒द्रम् । श॒तधा॑र॒मिति॑ श॒त - धा॒र॒म् । उथ्स᳚म् । व्य॒च्यमा॑न॒मिति॑ वि - अ॒च्यमा॑नम् । भुव॑नस्य । मद्ध्ये᳚ ॥ घृ॒तम् । दुहा॑नाम् । अदि॑तिम् । जना॑य । अग्ने᳚ । मा ।  \newline




\markright{ TS 4.2.10.3  \hfill https://www.vedavms.in \hfill}
\addcontentsline{toc}{section}{ TS 4.2.10.3 }
\section*{ TS 4.2.10.3 }

\textbf{TS 4.2.10.3 } \newline
\textbf{Samhita Paata} \newline

हिꣳ॑सीः पर॒मे व्यो॑मन्न् । ग॒व॒यमा॑र॒ण्यमनु॑ ते दिशामि॒ तेन॑ चिन्वा॒नस्त॒नुवो॒ नि षी॑द ॥ वरू᳚त्रिं॒ त्वष्टु॒र्वरु॑णस्य॒ नाभि॒मविं॑ जज्ञा॒नाꣳ रज॑सः॒ पर॑स्मात् । म॒हीꣳ सा॑ह॒स्रीमसु॑रस्य मा॒यामग्ने॒ मा हिꣳ॑सीः पर॒मे व्यो॑मन्न् ॥ इ॒मामू᳚र्णा॒युं ॅवरु॑णस्य मा॒यां त्वचं॑ पशू॒नां द्वि॒पदां॒ चतु॑ष्पदां । त्वष्टुः॑ प्र॒जानां᳚ प्रथ॒मं ज॒नित्र॒मग्ने॒ मा हिꣳ॑सीः पर॒मे व्यो॑मन्न् । उष्ट्र॑मार॒ण्यमनु॑ - [  ] \newline

\textbf{Pada Paata} \newline

हिꣳ॒॒सीः॒ । प॒र॒मे । व्यो॑म॒न्निति॒ वि - ओ॒म॒न्न् ॥ ग॒व॒यम् । आ॒र॒ण्यम् । अन्विति॑ । ते॒ । दि॒शा॒मि॒ । तेन॑ । चि॒न्वा॒नः । त॒नुवः॑ । नीति॑ । सी॒द॒ ॥ वरू᳚त्रिम् । त्वष्टुः॑ । वरु॑णस्य । नाभि᳚म् । अवि᳚म् । ज॒ज्ञा॒नाम् । रज॑सः । पर॑स्मात् ॥ म॒हीम् । सा॒ह॒स्रीम् । असु॑रस्य । मा॒याम् । अग्ने᳚ । मा । हिꣳ॒॒सीः॒ । प॒र॒मे । व्यो॑म॒न्निति॒ वि - ओ॒म॒न्न् ॥ इ॒माम् । ऊ॒र्णा॒युम् । वरु॑णस्य । मा॒याम् । त्वच᳚म् । प॒शू॒नाम् । द्वि॒पदा॒मिति॑ द्वि - पदा᳚म् । चतु॑ष्पदा॒मिति॒ चतुः॑ - प॒दा॒म् ॥ त्वष्टुः॑ । प्र॒जाना॒मिति॑ प्र - जाना᳚म् । प्र॒थ॒मम् । ज॒नित्र᳚म् । अग्ने᳚ । मा । हिꣳ॒॒सीः॒ । प॒र॒मे । व्यो॑म॒न्निति॒ वि - ओ॒म॒न्न् ॥ उष्ट्र᳚म् । आ॒र॒ण्यम् । अन्विति॑ ।  \newline




\markright{ TS 4.2.10.4  \hfill https://www.vedavms.in \hfill}
\addcontentsline{toc}{section}{ TS 4.2.10.4 }
\section*{ TS 4.2.10.4 }

\textbf{TS 4.2.10.4 } \newline
\textbf{Samhita Paata} \newline

ते दिशामि॒ तेन॑ चिन्वा॒नस्त॒नुवो॒ नि षी॑द ॥ यो अ॒ग्निर॒ग्नेस्त-प॒सोऽधि॑ जा॒तः शोचा᳚त् पृथि॒व्या उ॒त वा॑ दि॒वस्परि॑ । येन॑ प्र॒जा वि॒श्वक॑र्मा॒ व्यान॒ट् तम॑ग्ने॒ हेडः॒ परि॑ ते वृणक्तु ॥ अ॒जा ह्य॑ग्नेरज॑निष्ट॒ गर्भा॒थ् सा वा अ॑पश्यज्जनि॒तार॒मग्रे᳚ । तया॒ रोह॑माय॒न्नुप॒ मेद्ध्या॑स॒स्तया॑ दे॒वा दे॒वता॒मग्र॑ आयन्न् । श॒र॒भ-( )-मा॑र॒ण्यमनु॑ ते दिशामि॒ तेन॑ चिन्वा॒नस्त॒नुवो॒ निषी॑द ॥ \newline

\textbf{Pada Paata} \newline

ते॒ । दि॒शा॒मि॒ । तेन॑ । चि॒न्वा॒नः । त॒नुवः॑ । नीति॑ । सी॒द॒ ॥ यः । अ॒ग्निः । अ॒ग्नेः । तप॑सः । अधीति॑ । जा॒तः । शोचा᳚त् । पृ॒थि॒व्याः । उ॒त । वा॒ । दि॒वः । परि॑ ॥ येन॑ । प्र॒जा इति॑ प्र - जाः । वि॒श्वक॒र्मेति॑ वि॒श्व - क॒र्मा॒ । व्यान॒डिति॑ वि - आन॑ट् । तम् । अ॒ग्ने॒ । हेडः॑ । परीति॑ । ते॒ । वृ॒ण॒क्तु॒ ॥ अ॒जा । हि । अ॒ग्नेः । अज॑निष्ट । गर्भा᳚त् । सा । वै । अ॒प॒श्य॒त् । ज॒नि॒तार᳚म् । अग्रे᳚ ॥ तया᳚ । रोह᳚म् । आ॒य॒न्न् । उपेति॑ । मेद्ध्या॑सः । तया᳚ । दे॒वाः । दे॒वता᳚म् । अग्रे᳚ । आ॒य॒न्न् ॥ श॒र॒भम् ( ) । आ॒र॒ण्यम् । अन्विति॑ । ते॒ । दि॒शा॒मि॒ । तेन॑ । चि॒न्वा॒नः । त॒नुवः॑ । नीति॑ । सी॒द॒ ॥  \newline




\markright{ TS 4.2.11.1  \hfill https://www.vedavms.in \hfill}
\addcontentsline{toc}{section}{ TS 4.2.11.1 }
\section*{ TS 4.2.11.1 }

\textbf{TS 4.2.11.1 } \newline
\textbf{Samhita Paata} \newline

इन्द्रा᳚ग्नी रोच॒ना दि॒वः परि॒ वाजे॑षु भूषथः । तद्वां᳚ चेति॒ प्रवी॒र्यं᳚ ॥ श्नथ॑द्-वृ॒त्रमु॒त स॑नोति॒ वाज॒मिन्द्रा॒ यो अ॒ग्नी सहु॑री सप॒र्यात् । इ॒र॒ज्यन्ता॑ वस॒व्य॑स्य॒ भूरेः॒ सह॑स्तमा॒ सह॑सा वाज॒यन्ता᳚ ॥ प्र च॑र्.ष॒णिभ्यः॑ पृतना॒ हवे॑षु॒ प्र पृ॑थि॒व्या रि॑रिचाथे दि॒वश्च॑ । प्र सिन्धु॑भ्यः॒ प्रगि॒रिभ्यो॑ महि॒त्वा प्रेन्द्रा᳚ग्नी॒ विश्वा॒ भुव॒नाऽत्य॒न्या ॥ मरु॑तो॒ यस्य॒ हि - [  ] \newline

\textbf{Pada Paata} \newline

इन्द्रा᳚ग्नी॒ इतीन्द्र॑ - अ॒ग्नी॒ । रो॒च॒ना । दि॒वः । परीति॑ । वाजे॑षु । भू॒ष॒थः॒ ॥ तत् । वा॒म् । चे॒ति॒ । प्रेति॑ । वी॒र्य᳚म् ॥ श्नथ॑त् । वृ॒त्रम् । उ॒त । स॒नो॒ति॒ । वाज᳚म् । इन्द्रा᳚ । यः । अ॒ग्नी इति॑ । सहु॑री॒ इति॑ स - हु॒री॒ । स॒प॒र्यात् ॥ इ॒र॒ज्यन्ता᳚ । व॒स॒व्य॑स्य । भूरेः᳚ । सह॑स्त॒मेति॒ सहः॑ - त॒मा॒ । सह॑सा । वा॒ज॒यन्तेति॑ वाज - यन्ता᳚ ॥ प्रेति॑ । च॒र्॒.ष॒णिभ्य॒ इति॑ चर्.ष॒णि-भ्यः॒ । पृ॒त॒ना॒ । हवे॑षु । प्रेति॑ । पृ॒थि॒व्याः । रि॒रि॒चा॒थे॒ इति॑ । दि॒वः । च॒ ॥ प्रेति॑ । सिन्धु॑भ्य॒ इति॒ सिन्धु॑ - भ्यः॒ । प्रेति॑ । गि॒रिभ्य॒ इति॑ गि॒रि - भ्यः॒ । म॒हि॒त्वेति॑ महि - त्वा । प्रेति॑ । इ॒न्द्रा॒ग्नी॒ इती᳚न्द्र - अ॒ग्नी॒ । विश्वा᳚ । भुव॑ना । अतीति॑ । अ॒न्या ॥ मरु॑तः । यस्य॑ । हि ।  \newline




\markright{ TS 4.2.11.2  \hfill https://www.vedavms.in \hfill}
\addcontentsline{toc}{section}{ TS 4.2.11.2 }
\section*{ TS 4.2.11.2 }

\textbf{TS 4.2.11.2 } \newline
\textbf{Samhita Paata} \newline

क्षये॑ पा॒था दि॒वो वि॑महसः । स सु॑गो॒पात॑मो॒ जनः॑ ॥ य॒ज्ञिर्वा॑ यज्ञ्वाहसो॒ विप्र॑स्य वा मती॒नां । मरु॑तः शृणु॒ता हवं᳚ ॥ श्रि॒यसे॒ कं भा॒नुभिः॒ सं मि॑मिक्षिरे॒ ते र॒श्मिभि॒स्त ऋक्व॑भिः सुखा॒दयः॑ । ते वाशी॑मन्त इ॒ष्मिणो॒ अभी॑रवो वि॒द्रे प्रि॒यस्य॒ मारु॑तस्य॒ धाम्नः॑ ॥ अव॑ ते॒ हेड॒> 1, उदु॑त्त॒मं >2 ॥ कया॑ नश्चि॒त्र आ भु॑वदू॒ती स॒दा वृ॑धः॒ सखा᳚ । कया॒ शचि॑ष्ठया वृ॒ता ॥ \newline

\textbf{Pada Paata} \newline

क्षये᳚ । पा॒थ । दि॒वः । वि॒म॒ह॒स॒ इति॑ वि - म॒ह॒सः॒ ॥ सः । सु॒गो॒पात॑म॒ इति॑ सुगो॒प - त॒मः॒ । जनः॑ ॥ य॒ज्ञिः । वा॒ । य॒ज्ञ्॒वा॒ह॒स॒ इति॑ यज्ञ् - वा॒ह॒सः॒ । विप्र॑स्य । वा॒ । म॒ती॒नाम् ॥ मरु॑तः । शृ॒णु॒त । हव᳚म् ॥ श्रि॒यसे᳚ । कम् । भा॒नुभि॒रिति॑ भा॒नु - भिः॒ । समिति॑ । मि॒मि॒क्षि॒रे॒ । ते । र॒श्मिभि॒रिति॑ र॒श्मि - भिः॒ । ते । ऋक्व॑भि॒रित्यृक्व॑ - भिः॒ । सु॒खा॒दय॒ इति॑ सु - खा॒दयः॑ ॥ ते । वाशी॑मन्त॒ इति॒ वाशि॑ - म॒न्तः॒ । इ॒ष्मिणः॑ । अभी॑रवः । वि॒द्रे । प्रि॒यस्य॑ । मारु॑तस्य । धाम्नः॑ ॥ अवेति॑ । ते॒ । हेडः॑ । उदिति॑ । उ॒त्त॒ममित्यु॑त् - त॒मम् ॥ कया᳚ । नः॒ । चि॒त्रः । एति॑ । भु॒व॒त् । ऊ॒ती । स॒दावृ॑ध॒ इति॑ स॒दा - वृ॒धः॒ । सखा᳚ ॥ कया᳚ । शचि॑ष्ठया । वृ॒ता ॥  \newline




\markright{ TS 4.2.11.3  \hfill https://www.vedavms.in \hfill}
\addcontentsline{toc}{section}{ TS 4.2.11.3 }
\section*{ TS 4.2.11.3 }

\textbf{TS 4.2.11.3 } \newline
\textbf{Samhita Paata} \newline

को अ॒द्य यु॑ङ्क्ते धु॒रि गा ऋ॒तस्य॒ शिमी॑वतो भा॒मिनो॑ दुर्.हृणा॒यून् । आ॒सन्नि॑षून्. हृ॒थ्स्वसो॑ मयो॒भून्. य ए॑षां भृ॒त्यामृ॒णध॒थ् स जी॑वात् ॥ अग्ने॒ नया > 3, ऽऽदे॒वानाꣳ॒॒ >4, शन्नो॑ भवन्तु॒ > 5, वाजे॑वाजे>6 । अ॒फ्स्व॑ग्ने॒ सधि॒ष्टव॒ सौष॑धी॒रनु॑ रुद्ध्यसे । गर्भे॒ सञ्जा॑यसे॒ पुनः॑ ॥ वृषा॑ सोम द्यु॒माꣳ अ॑सि॒ वृषा॑ देव॒ वृष॑व्रतः । वृषा॒ धर्मा॑णि दधिषे ॥ इ॒मं मे॑ ( ) वरुण॒ > 7, तत्त्वा॑ यामि॒>8, त्वं नो॑ अग्ने॒>9, स त्वं नो॑ अग्ने> 10 ॥ \newline

\textbf{Pada Paata} \newline

कः । अ॒द्य । यु॒ङ्क्ते॒ । धु॒रि । गाः । ऋ॒तस्य॑ । शिमी॑वतः । भा॒मिनः॑ । दु॒र्॒.हृ॒णा॒यूनिति॑ दुः - हृ॒णा॒यून् ॥ आ॒सन्नि॑षू॒नित्या॒सन्न् - इ॒षू॒न् । हृ॒थ्स्वस॒ इति॑ हृथ्सु - असः॑ । म॒यो॒भूनिति॑ मयः - भून् । यः । ए॒षा॒म् । भृ॒त्याम् । ऋ॒णध॑त् । सः । जी॒वा॒त् ॥ अग्ने᳚ । नय॑ । एति॑ । दे॒वाना᳚म् । शम् । नः॒ । भ॒व॒न्तु॒ । वाजे॑वाज॒ इति॒ वाजे᳚ - वा॒जे॒ ॥ अ॒फ्स्वित्य॑प्- सु । अ॒ग्ने॒ । सधिः॑ । तव॑ । सः । ओष॑धीः । अन्विति॑ । रु॒द्ध्य॒से॒ ॥ गर्भे᳚ । सन्न् । जा॒य॒से॒ । पुनः॑ ॥ वृषा᳚ । सो॒म॒ । द्यु॒मानिति॑ द्यु-मान् । अ॒सि॒ । वृषा᳚ । दे॒व॒ । वृष॑व्रत॒ इति॒ वृष॑-व्र॒तः॒ ॥ वृषा᳚ । धर्मा॑णि । द॒धि॒षे॒ ॥ इ॒मम् । मे॒ ( ) । व॒रु॒ण॒ । तत् । त्वा॒ । या॒मि॒ । त्वम् । नः॒ । अ॒ग्ने॒ । सः । त्वम् । नः॒ । अ॒ग्ने॒ ॥  \newline






\end{document}