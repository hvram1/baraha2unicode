\documentclass[17pt]{extarticle}
\usepackage{babel}
\usepackage{fontspec}
\usepackage{polyglossia}
\usepackage{extsizes}

\usepackage{color}   %May be necessary if you want to color links
\usepackage{hyperref}
\hypersetup{
    colorlinks=true, %set true if you want colored links
    linktoc=all,     %set to all if you want both sections and subsections linked
    linkcolor=black,  %choose some color if you want links to stand out
}

\setmainlanguage{sanskrit}
\setotherlanguages{english} %% or other languages
\setlength{\parindent}{0pt}
\pagestyle{myheadings}
\newfontfamily\devanagarifont[Script=Devanagari]{AdishilaVedic}
\renewcommand{\theHsection}{\thepart.section.\thesection}

\newcommand{\VAR}[1]{}
\newcommand{\BLOCK}[1]{}




\begin{document}
\begin{titlepage}
    \begin{center}
 
\begin{sanskrit}
    { \Large
    कृष्ण यजुर्वेदीय तैत्तिरीय संहिता,पद,जटा,घन पाठः 
    }
    \\
    \vspace{2.5cm}
    \mbox{ \Large
    2.4      द्वितीयकाण्डे चतुर्थः प्रश्नः - इष्टिविधानं   }
\end{sanskrit}
\end{center}

\end{titlepage}
\tableofcontents
\phantomsection
\pagebreak

\markright{ TS 2.4.1.1  \hfill https://www.vedavms.in \hfill}

\section{ TS 2.4.1.1 }

\textbf{TS 2.4.1.1 } \newline
\textbf{Samhita Paata} \newline

दे॒वा म॑नु॒ष्याः᳚ पि॒तर॒स्ते᳚ऽन्यत॑ आस॒न्नसु॑रा॒ रक्षाꣳ॑सि पिशा॒चास्ते᳚ ऽन्यत॒स्तेषां᳚ दे॒वाना॑मु॒त यदल्पं॒ ॅलोहि॑त॒मकु॑र्व॒न् तद्-रक्षाꣳ॑सि॒ रात्री॑भिरसुभ्न॒न् तान्थ् सु॒ब्धान् मृ॒तान॒भि व्यौ᳚च्छ॒त् ते दे॒वा अ॑विदु॒र्यो वै नो॒ऽयं म्रि॒यते॒ रक्षाꣳ॑सि॒ वा इ॒मं घ्न॒न्तीति॒ ते रक्षाꣳ॒॒स्युपा॑मन्त्रयन्त॒ तान्य॑ब्रुव॒न्. वरं॑ ॅवृणामहै॒ य - [  ] \newline

\textbf{Pada Paata} \newline

दे॒वाः । म॒नु॒ष्याः᳚ । पि॒तरः॑ । ते । अ॒न्यतः॑ । आ॒स॒न्न् । असु॑राः । रक्षाꣳ॑सि । पि॒शा॒चाः । ते । अ॒न्यतः॑ । तेषा᳚म् । दे॒वाना᳚म् । उ॒त । यत् । अल्प᳚म् । लोहि॑तम् । अकु॑र्वन्न् । तत् । रक्षाꣳ॑सि । रात्री॑भि॒रिति॒ रात्रि॑-भिः॒ । अ॒सु॒भ्न॒न्न् । तान् । सु॒ब्धान् । मृ॒तान् । अ॒भि । वीति॑ । औ॒च्छ॒त् । ते । दे॒वाः । अ॒वि॒दुः॒ । यः । वै । नः॒ । अ॒यम् । म्रि॒यते᳚ । रक्षाꣳ॑सि । वै । इ॒मम् । घ्न॒न्ति॒ । इति॑ । ते । रक्षाꣳ॑सि । उपेति॑ । अ॒म॒न्त्र॒य॒न्त॒ । तानि॑ । अ॒ब्रु॒व॒न्न् । वर᳚म् । वृ॒णा॒म॒है॒ । यत् ।  \newline


\textbf{Krama Paata} \newline

दे॒वा म॑नु॒ष्याः᳚ । म॒नु॒ष्याः᳚ पि॒तरः॑ । पि॒तर॒स्ते । ते᳚ ऽन्यतः॑ । अ॒न्यत॑ आसन्न् । आ॒स॒न्नसु॑राः । असु॑रा॒ रक्षाꣳ॑सि । रक्षाꣳ॑सि पिशा॒चाः । पि॒शा॒चास्ते । ते᳚ ऽन्यतः॑ । अ॒न्यत॒ स्तेषा᳚म् । तेषा᳚म् दे॒वाना᳚म् । दे॒वाना॑मु॒त । उ॒त यत् । यदल्प᳚म् । अल्पं॒ ॅलोहि॑तम् । लोहि॑त॒मकु॑र्वन्न् । अकु॑र्व॒न् तत् । तद् रक्षाꣳ॑सि । रक्षाꣳ॑सि॒ रात्री॑भिः । रात्री॑भिरसुभ्नन्न् । रात्री॑भि॒रिति॒ रात्रि॑ - भिः॒ । अ॒सु॒भ्न॒न् तान् । तान्थ् सु॒ब्धान् । सु॒ब्धान् मृ॒तान् । मृ॒तान॒भि । 
अ॒भि वि । व्यौ᳚च्छत् । औ॒च्छ॒त् ते । ते दे॒वाः । दे॒वा अ॑विदुः । अ॒वि॒दु॒र् यः । यो वै । वै नः॑ । नो॒ ऽयम् । अ॒यम् म्रि॒यते᳚ । म्रि॒यते॒ रक्षाꣳ॑सि । रक्षाꣳ॑सि॒ वै । वा इ॒मम् । इ॒मम्(2) घ्न॑न्ति । घ्न॒न्तीति॑ । इति॒ ते । ते रक्षाꣳ॑सि । रक्षाꣳ॒॒स्युप॑ । उपा॑मन्त्रयन्त । अ॒म॒न्त्र॒य॒न्त॒ तानि॑ । तान्य॑ब्रुवन्न् । अ॒ब्रु॒व॒न् वर᳚म् । वरं॑ ॅवृणामहै । वृ॒णा॒म॒है॒ यत् । यदसु॑रान् \newline

\textbf{Jatai Paata} \newline

1. दे॒वा म॑नु॒ष्या॑ मनु॒ष्या॑ दे॒वा दे॒वा म॑नु॒ष्याः᳚ । \newline
2. म॒नु॒ष्याः᳚ पि॒तरः॑ पि॒तरो॑ मनु॒ष्या॑ मनु॒ष्याः᳚ पि॒तरः॑ । \newline
3. पि॒तर॒स्ते ते पि॒तरः॑ पि॒तर॒स्ते । \newline
4. ते᳚ ऽन्यतो॒ ऽन्यत॒स्ते ते᳚ ऽन्यतः॑ । \newline
5. अ॒न्यत॑ आसन् नासन् न॒न्यतो॒ ऽन्यत॑ आसन्न् । \newline
6. आ॒स॒न् नसु॑रा॒ असु॑रा आसन् नास॒न् नसु॑राः । \newline
7. असु॑रा॒ रक्षाꣳ॑सि॒ रक्षाꣳ॒॒स्यसु॑रा॒ असु॑रा॒ रक्षाꣳ॑सि । \newline
8. रक्षाꣳ॑सि पिशा॒चाः पि॑शा॒चा रक्षाꣳ॑सि॒ रक्षाꣳ॑सि पिशा॒चाः । \newline
9. पि॒शा॒चा स्ते ते पि॑शा॒चाः पि॑शा॒चा स्ते । \newline
10. ते᳚ ऽन्यतो॒ ऽन्यत॒ स्ते ते᳚ ऽन्यतः॑ । \newline
11. अ॒न्यत॒ स्तेषा॒म् तेषा॑ म॒न्यतो॒ ऽन्यत॒ स्तेषा᳚म् । \newline
12. तेषा᳚म् दे॒वाना᳚म् दे॒वाना॒म् तेषा॒म् तेषा᳚म् दे॒वाना᳚म् । \newline
13. दे॒वाना॑ मु॒तोत दे॒वाना᳚म् दे॒वाना॑ मु॒त । \newline
14. उ॒त यद् यदु॒तोत यत् । \newline
15. यदल्प॒ मल्पं॒ ॅयद् यदल्प᳚म् । \newline
16. अल्प॒म् ॅलोहि॑त॒म् ॅलोहि॑त॒ मल्प॒ मल्प॒म् ॅलोहि॑तम् । \newline
17. लोहि॑त॒ मकु॑र्व॒न् नकु॑र्व॒न् ॅलोहि॑त॒म् ॅलोहि॑त॒ मकु॑र्वन्न् । \newline
18. अकु॑र्व॒न् तत् तदकु॑र्व॒न् नकु॑र्व॒न् तत् । \newline
19. तद् रक्षाꣳ॑सि॒ रक्षाꣳ॑सि॒ तत् तद् रक्षाꣳ॑सि । \newline
20. रक्षाꣳ॑सि॒ रात्री॑भी॒ रात्री॑भी॒ रक्षाꣳ॑सि॒ रक्षाꣳ॑सि॒ रात्री॑भिः । \newline
21. रात्री॑भि रसुभ्नन् नसुभ्न॒न् रात्री॑भी॒ रात्री॑भि रसुभ्नन्न् । \newline
22. रात्री॑भि॒रिति॒ रात्रि॑ - भिः॒ । \newline
23. अ॒सु॒भ्न॒न् ताꣳ स्ता न॑सुभ्नन् नसुभ्न॒न् तान् । \newline
24. तान् थ्सु॒ब्धान् थ्सु॒ब्धान् ताꣳ स्तान् थ्सु॒ब्धान् । \newline
25. सु॒ब्धान् मृ॒तान् मृ॒तान् थ्सु॒ब्धान् थ्सु॒ब्धान् मृ॒तान् । \newline
26. मृ॒ता न॒भ्य॑भि मृ॒तान् मृ॒ता न॒भि । \newline
27. अ॒भि वि व्या᳚(1॒)भ्य॑भि वि । \newline
28. व्यौ᳚च्छ दौच्छ॒द् वि व्यौ᳚च्छत् । \newline
29. औ॒च्छ॒त् ते त औ᳚च्छ दौच्छ॒त् ते । \newline
30. ते दे॒वा दे॒वा स्ते ते दे॒वाः । \newline
31. दे॒वा अ॑विदु रविदुर् दे॒वा दे॒वा अ॑विदुः । \newline
32. अ॒वि॒दु॒र् यो यो॑ ऽविदु रविदु॒र् यः । \newline
33. यो वै वै यो यो वै । \newline
34. वै नो॑ नो॒ वै वै नः॑ । \newline
35. नो॒ ऽय म॒यम् नो॑ नो॒ ऽयम् । \newline
36. अ॒यम् म्रि॒यते᳚ म्रि॒यते॒ ऽय म॒यम् म्रि॒यते᳚ । \newline
37. म्रि॒यते॒ रक्षाꣳ॑सि॒ रक्षाꣳ॑सि म्रि॒यते᳚ म्रि॒यते॒ रक्षाꣳ॑सि । \newline
38. रक्षाꣳ॑सि॒ वै वै रक्षाꣳ॑सि॒ रक्षाꣳ॑सि॒ वै । \newline
39. वा इ॒म मि॒मं ॅवै वा इ॒मम् । \newline
40. ई॒मम्(2) घ्न॑न्ति घ्नन्ती॒म मि॒मम्(2) घ्न॑न्ति । \newline
41. घ्न॒न्तीतीति॑ घ्नन्ति घ्न॒न्तीति॑ । \newline
42. इति॒ ते त इतीति॒ ते । \newline
43. ते रक्षाꣳ॑सि॒ रक्षाꣳ॑सि॒ ते ते रक्षाꣳ॑सि । \newline
44. रक्षाꣳ॒॒ स्युपोप॒ रक्षाꣳ॑सि॒ रक्षाꣳ॒॒ स्युप॑ । \newline
45. उपा॑मन्त्रयन्ता मन्त्रय॒न्तो पोपा॑ मन्त्रयन्त । \newline
46. अ॒म॒न्त्र॒य॒न्त॒ तानि॒ तान्य॑मन्त्रयन्ता मन्त्रयन्त॒ तानि॑ । \newline
47. तान्य॑ब्रुवन् नब्रुव॒न् तानि॒ तान्य॑ब्रुवन्न् । \newline
48. अ॒ब्रु॒व॒न्॒. वरं॒ ॅवर॑ मब्रुवन् नब्रुव॒न्॒. वर᳚म् । \newline
49. वरं॑ ॅवृणामहै वृणामहै॒ वरं॒ ॅवरं॑ ॅवृणामहै । \newline
50. वृ॒णा॒म॒है॒ यद् यद् वृ॑णामहै वृणामहै॒ यत् । \newline
51. यदसु॑रा॒ नसु॑रा॒न्॒. यद् यदसु॑रान् । \newline

\textbf{Ghana Paata } \newline

1. दे॒वा म॑नु॒ष्या॑ मनु॒ष्या॑ दे॒वा दे॒वा म॑नु॒ष्याः᳚ पि॒तरः॑ पि॒तरो॑ मनु॒ष्या॑ दे॒वा दे॒वा म॑नु॒ष्याः᳚ पि॒तरः॑ । \newline
2. म॒नु॒ष्याः᳚ पि॒तरः॑ पि॒तरो॑ मनु॒ष्या॑ मनु॒ष्याः᳚ पि॒तर॒ स्ते ते पि॒तरो॑ मनु॒ष्या॑ मनु॒ष्याः᳚ पि॒तर॒ स्ते । \newline
3. पि॒तर॒ स्ते ते पि॒तरः॑ पि॒तर॒ स्ते᳚ ऽन्यतो॒ ऽन्यत॒ स्ते पि॒तरः॑ पि॒तर॒ स्ते᳚ ऽन्यतः॑ । \newline
4. ते᳚ ऽन्यतो॒ ऽन्यत॒ स्ते ते᳚ ऽन्यत॑ आसन् नासन् न॒न्यत॒ स्ते ते᳚ ऽन्यत॑ आसन्न् । \newline
5. अ॒न्यत॑ आसन् नासन् न॒न्यतो॒ ऽन्यत॑ आस॒न् नसु॑रा॒ असु॑रा आसन् न॒न्यतो॒ ऽन्यत॑ आस॒न् नसु॑राः । \newline
6. आ॒स॒न् नसु॑रा॒ असु॑रा आसन् नास॒न् नसु॑रा॒ रक्षाꣳ॑सि॒ रक्षाꣳ॒॒ स्यसु॑रा आसन् नास॒न् नसु॑रा॒ रक्षाꣳ॑सि । \newline
7. असु॑रा॒ रक्षाꣳ॑सि॒ रक्षाꣳ॒॒ स्यसु॑रा॒ असु॑रा॒ रक्षाꣳ॑सि पिशा॒चाः पि॑शा॒चा रक्षाꣳ॒॒ स्यसु॑रा॒ असु॑रा॒ रक्षाꣳ॑सि पिशा॒चाः । \newline
8. रक्षाꣳ॑सि पिशा॒चाः पि॑शा॒चा रक्षाꣳ॑सि॒ रक्षाꣳ॑सि पिशा॒चा स्ते ते पि॑शा॒चा रक्षाꣳ॑सि॒ रक्षाꣳ॑सि पिशा॒चा स्ते । \newline
9. पि॒शा॒चा स्ते ते पि॑शा॒चाः पि॑शा॒चा स्ते᳚ ऽन्यतो॒ ऽन्यत॒ स्ते पि॑शा॒चाः पि॑शा॒चा स्ते᳚ ऽन्यतः॑ । \newline
10. ते᳚ ऽन्यतो॒ ऽन्यत॒ स्ते ते᳚ ऽन्यत॒ स्तेषा॒म् तेषा॑ म॒न्यत॒ स्ते ते᳚ ऽन्यत॒ स्तेषा᳚म् । \newline
11. अ॒न्यत॒ स्तेषा॒म् तेषा॑ म॒न्यतो॒ ऽन्यत॒ स्तेषा᳚म् दे॒वाना᳚म् दे॒वाना॒म् तेषा॑ म॒न्यतो॒ ऽन्यत॒ स्तेषा᳚म् दे॒वाना᳚म् । \newline
12. तेषा᳚म् दे॒वाना᳚म् दे॒वाना॒म् तेषा॒म् तेषा᳚म् दे॒वाना॑ मु॒तोत दे॒वाना॒म् तेषा॒म् तेषा᳚म् दे॒वाना॑ मु॒त । \newline
13. दे॒वाना॑ मु॒तोत दे॒वाना᳚म् दे॒वाना॑ मु॒त यद् यदु॒त दे॒वाना᳚म् दे॒वाना॑ मु॒त यत् । \newline
14. उ॒त यद् यदु॒तोत यदल्प॒ मल्पं॒ ॅयदु॒तोत यदल्प᳚म् । \newline
15. यदल्प॒ मल्पं॒ ॅयद् यदल्प॒म् ॅलोहि॑त॒म् ॅलोहि॑त॒ मल्पं॒ ॅयद् यदल्प॒म् ॅलोहि॑तम् । \newline
16. अल्प॒म् ॅलोहि॑त॒म् ॅलोहि॑त॒ मल्प॒ मल्प॒म् ॅलोहि॑त॒ मकु॑र्व॒न् नकु॑र्व॒न् ॅलोहि॑त॒ मल्प॒ मल्प॒म् ॅलोहि॑त॒ मकु॑र्वन्न् । \newline
17. लोहि॑त॒ मकु॑र्व॒न् नकु॑र्व॒न् ॅलोहि॑त॒म् ॅलोहि॑त॒ मकु॑र्व॒न् तत् तदकु॑र्व॒न् ॅलोहि॑त॒म् ॅलोहि॑त॒ मकु॑र्व॒न् तत् । \newline
18. अकु॑र्व॒न् तत् तदकु॑र्व॒न् नकु॑र्व॒न् तद् रक्षाꣳ॑सि॒ रक्षाꣳ॑सि॒ तदकु॑र्व॒न् नकु॑र्व॒न् तद् रक्षाꣳ॑सि । \newline
19. तद् रक्षाꣳ॑सि॒ रक्षाꣳ॑सि॒ तत् तद् रक्षाꣳ॑सि॒ रात्री॑भी॒ रात्री॑भी॒ रक्षाꣳ॑सि॒ तत् तद् रक्षाꣳ॑सि॒ रात्री॑भिः । \newline
20. रक्षाꣳ॑सि॒ रात्री॑भी॒ रात्री॑भी॒ रक्षाꣳ॑सि॒ रक्षाꣳ॑सि॒ रात्री॑भि रसुभ्नन् नसुभ्न॒न् रात्री॑भी॒ रक्षाꣳ॑सि॒ रक्षाꣳ॑सि॒ रात्री॑भि रसुभ्नन्न् । \newline
21. रात्री॑भि रसुभ्नन् नसुभ्न॒न् रात्री॑भी॒ रात्री॑भि रसुभ्न॒न् ताꣳ स्ता न॑सुभ्न॒न् रात्री॑भी॒ रात्री॑भि रसुभ्न॒न् तान् । \newline
22. रात्री॑भि॒रिति॒ रात्रि॑ - भिः॒ । \newline
23. अ॒सु॒भ्न॒न् ताꣳ स्ता न॑सुभ्नन् नसुभ्न॒न् तान् थ्सु॒ब्धान् थ्सु॒ब्धान् ता न॑सुभ्नन् नसुभ्न॒न् तान् थ्सु॒ब्धान् । \newline
24. तान् थ्सु॒ब्धान् थ्सु॒ब्धान् ताꣳ स्तान् थ्सु॒ब्धान् मृ॒तान् मृ॒तान् थ्सु॒ब्धान् ताꣳ स्तान् थ्सु॒ब्धान् मृ॒तान् । \newline
25. सु॒ब्धान् मृ॒तान् मृ॒तान् थ्सु॒ब्धान् थ्सु॒ब्धान् मृ॒ता न॒भ्य॑भि मृ॒तान् थ्सु॒ब्धान् थ्सु॒ब्धान् मृ॒ता न॒भि । \newline
26. मृ॒ता न॒भ्य॑भि मृ॒तान् मृ॒ता न॒भि वि व्य॑भि मृ॒तान् मृ॒ता न॒भि वि । \newline
27. अ॒भि वि व्या᳚(1॒)भ्य॑भि व्यौ᳚च्छ दौच्छ॒द् व्या᳚(1॒)भ्य॑भि व्यौ᳚च्छत् । \newline
28. व्यौ᳚च्छ दौच्छ॒द् वि व्यौ᳚च्छ॒त् ते त औ᳚च्छ॒द् वि व्यौ᳚च्छ॒त् ते । \newline
29. औ॒च्छ॒त् ते त औ᳚च्छ दौच्छ॒त् ते दे॒वा दे॒वा स्त औ᳚च्छ दौच्छ॒त् ते दे॒वाः । \newline
30. ते दे॒वा दे॒वा स्ते ते दे॒वा अ॑विदु रविदुर् दे॒वा स्ते ते दे॒वा अ॑विदुः । \newline
31. दे॒वा अ॑विदु रविदुर् दे॒वा दे॒वा अ॑विदु॒र् यो यो॑ ऽविदुर् दे॒वा दे॒वा अ॑विदु॒र् यः । \newline
32. अ॒वि॒दु॒र् यो यो॑ ऽविदु रविदु॒र् यो वै वै यो॑ ऽविदु रविदु॒र् यो वै । \newline
33. यो वै वै यो यो वै नो॑ नो॒ वै यो यो वै नः॑ । \newline
34. वै नो॑ नो॒ वै वै नो॒ ऽय म॒यम् नो॒ वै वै नो॒ ऽयम् । \newline
35. नो॒ ऽय म॒यम् नो॑ नो॒ ऽयम् म्रि॒यते᳚ म्रि॒यते॒ ऽयम् नो॑ नो॒ ऽयम् म्रि॒यते᳚ । \newline
36. अ॒यम् म्रि॒यते᳚ म्रि॒यते॒ ऽय म॒यम् म्रि॒यते॒ रक्षाꣳ॑सि॒ रक्षाꣳ॑सि म्रि॒यते॒ ऽय म॒यम् म्रि॒यते॒ रक्षाꣳ॑सि । \newline
37. म्रि॒यते॒ रक्षाꣳ॑सि॒ रक्षाꣳ॑सि म्रि॒यते᳚ म्रि॒यते॒ रक्षाꣳ॑सि॒ वै वै रक्षाꣳ॑सि म्रि॒यते᳚ म्रि॒यते॒ रक्षाꣳ॑सि॒ वै । \newline
38. रक्षाꣳ॑सि॒ वै वै रक्षाꣳ॑सि॒ रक्षाꣳ॑सि॒ वा इ॒म मि॒मं ॅवै रक्षाꣳ॑सि॒ रक्षाꣳ॑सि॒ वा इ॒मम् । \newline
39. वा इ॒म मि॒मं ॅवै वा इ॒मम्(2) घ्न॑न्ति घ्नन्ती॒मं ॅवै वा इ॒मम्(2) घ्न॑न्ति । \newline
40. इ॒मम्(2( घ्न॑न्ति घ्नन्ती॒म मि॒मम्(2) घ्न॒न्तीतीति॑ घ्नन्ती॒म मि॒मम्(2) घ्न॒न्तीति॑ । \newline
41. घ्न॒न्तीतीति॑ घ्नन्ति घ्न॒न्तीति॒ ते त इति॑ घ्नन्ति घ्न॒न्तीति॒ ते । \newline
42. इति॒ ते त इतीति॒ ते रक्षाꣳ॑सि॒ रक्षाꣳ॑सि॒ त इतीति॒ ते रक्षाꣳ॑सि । \newline
43. ते रक्षाꣳ॑सि॒ रक्षाꣳ॑सि॒ ते ते रक्षाꣳ॒॒ स्युपोप॒ रक्षाꣳ॑सि॒ ते ते रक्षाꣳ॒॒ स्युप॑ । \newline
44. रक्षाꣳ॒॒ स्युपोप॒ रक्षाꣳ॑सि॒ रक्षाꣳ॒॒ स्युपा॑मन्त्रयन्ता मन्त्रय॒न्तोप॒ रक्षाꣳ॑सि॒ रक्षाꣳ॒॒स्युपा॑ मन्त्रयन्त । \newline
45. उपा॑मन्त्रयन्ता मन्त्रय॒न्तोपोपा॑ मन्त्रयन्त॒ तानि॒ तान्य॑मन्त्रय॒न्तो पोपा॑ मन्त्रयन्त॒ तानि॑ । \newline
46. अ॒म॒न्त्र॒य॒न्त॒ तानि॒ तान्य॑ मन्त्रयन्तामन्त्रयन्त॒ तान्य॑ब्रुवन् नब्रुव॒न् तान्य॑मन्त्रयन्ता मन्त्रयन्त॒ तान्य॑ब्रुवन्न् । \newline
47. तान्य॑ब्रुवन् नब्रुव॒न् तानि॒ तान्य॑ब्रुव॒न्॒. वरं॒ ॅवर॑ मब्रुव॒न् तानि॒ तान्य॑ब्रुव॒न्॒. वर᳚म् । \newline
48. अ॒ब्रु॒व॒न्॒. वरं॒ ॅवर॑ मब्रुवन् नब्रुव॒न्॒. वरं॑ ॅवृणामहै वृणामहै॒ वर॑ मब्रुवन् नब्रुव॒न्॒. वरं॑ ॅवृणामहै । \newline
49. वरं॑ ॅवृणामहै वृणामहै॒ वरं॒ ॅवरं॑ ॅवृणामहै॒ यद् यद् वृ॑णामहै॒ वरं॒ ॅवरं॑ ॅवृणामहै॒ यत् । \newline
50. वृ॒णा॒म॒है॒ यद् यद् वृ॑णामहै वृणामहै॒ यदसु॑रा॒ नसु॑रा॒न्॒. यद् वृ॑णामहै वृणामहै॒ यदसु॑रान् । \newline
51. यदसु॑रा॒ नसु॑रा॒न्॒. यद् यदसु॑रा॒न् जया॑म॒ जया॒मासु॑रा॒न्॒. यद् यदसु॑रा॒न् जया॑म । \newline
\pagebreak
\markright{ TS 2.4.1.2  \hfill https://www.vedavms.in \hfill}

\section{ TS 2.4.1.2 }

\textbf{TS 2.4.1.2 } \newline
\textbf{Samhita Paata} \newline

-दसु॑रा॒न् जया॑म॒ तन्नः॑ स॒हास॒दिति॒ ततो॒ वै दे॒वा असु॑रानजय॒न् तेऽसु॑रान् जि॒त्वारक्षाꣳ॒॒स्यपा॑नुदन्त॒ तानि॒ रक्षाꣳ॒॒स्यनृ॑तम क॒र्तेति॑ सम॒न्तं दे॒वान् पर्य॑विश॒न् ते दे॒वा अ॒ग्नाव॑नाथन्त॒ ते᳚ऽग्नये॒ प्रव॑ते पुरो॒डाश॑म॒ष्टाक॑पालं॒ निर॑वपन्न॒ग्नये॑ विबा॒धव॑ते॒ऽग्नये॒ प्रती॑कवते॒ यद॒ग्नये॒ प्रव॑ते नि॒रव॑प॒न्॒. यान्ये॒व पु॒रस्ता॒द्-रक्षाꣳ॒॒स्या - [  ] \newline

\textbf{Pada Paata} \newline

असु॑रान् । जया॑म । तत् । नः॒ । स॒ह । अ॒स॒त् । इति॑ । ततः॑ । वै । दे॒वाः । असु॑रान् । अ॒ज॒य॒न्न् । ते । असु॑रान् । जि॒त्वा । रक्षाꣳ॑सि । अपेति॑ । अ॒नु॒द॒न्त॒ । तानि॑ । रक्षाꣳ॑सि । अनृ॑तम् । अ॒क॒र्त॒ । इति॑ । स॒म॒न्तमिति॑ सं - अ॒न्तम् । दे॒वान् । परीति॑ । अ॒वि॒श॒न्न् । ते । दे॒वाः । अ॒ग्नौ । अ॒ना॒थ॒न्त॒ । ते । अ॒ग्नये᳚ । प्रव॑त॒ इति॒ प्र - व॒ते॒ । पु॒रो॒डाश᳚म् । अ॒ष्टाक॑पाल॒मित्य॒ष्टा - क॒पा॒ल॒म् । निरिति॑ । अ॒व॒प॒न्न् । अ॒ग्नये᳚ । वि॒बा॒धव॑त॒ इति॑ विबा॒ध - व॒ते॒ । अ॒ग्नये᳚ । प्रती॑कवत॒ इति॒ प्रती॑क - व॒ते॒ । यत् । अ॒ग्नये᳚ । प्रव॑त॒ इति॒ प्र - व॒ते॒ । नि॒रव॑प॒न्निति॑ निः - अव॑पन्न् । यानि॑ । ए॒व । पु॒रस्ता᳚त् । रक्षाꣳ॑सि ।  \newline


\textbf{Krama Paata} \newline

असु॑रा॒न् जया॑म । जया॑म॒ तत् । तन्नः॑ । नः॒ स॒ह । स॒हास॑त् । अ॒स॒दिति॑ । इति॒ ततः॑ । ततो॒ वै । वै दे॒वाः । दे॒वा असु॑रान् । असु॑रानजयन्न् । अ॒ज॒य॒न् ते । तेऽसु॑रान् । असु॑रान् जि॒त्वा । जि॒त्वा रक्षाꣳ॑सि । रक्षाꣳ॒॒स्यप॑ । अपा॑नुदन्त । अ॒नु॒द॒न्त॒ तानि॑ । तानि॒ रक्षाꣳ॑सि । रक्षाꣳ॒॒स्यनृ॑तम् । अनृ॑तमकर्त । अ॒क॒र्तेति॑ । इति॑ सम॒न्तम् । स॒म॒न्तम् दे॒वान् । स॒म॒न्तमिति॑ सम् - अ॒न्तम् । दे॒वान् परि॑ । पर्य॑विशन्न् । अ॒वि॒श॒न् ते । ते दे॒वाः । दे॒वा अ॒ग्नौ । अ॒ग्नाव॑नाथन्त । अ॒ना॒थ॒न्त॒ ते । ते᳚ ऽग्नये᳚ । अ॒ग्नये॒ प्रव॑ते । प्रव॑ते पुरो॒डाश᳚म् । प्रव॑त॒ इति॒ प्र - व॒ते॒ । पु॒रो॒डाश॑म॒ष्टाक॑पालम् । अ॒ष्टक॑पाल॒म् निः । अ॒ष्टक॑पाल॒मित्य॒ष्टा - क॒पा॒ल॒म् । निर॑वपन्न् । अ॒व॒प॒न्न॒ग्नये᳚ । अ॒ग्नये॑ विबा॒धव॑ते । वि॒बा॒धव॑ते॒ ऽग्नये᳚ । वि॒बा॒धव॑त॒ इति॑ विबा॒ध - व॒ते॒ । अ॒ग्नये॒ प्रती॑कवते । प्रती॑कवते॒ यत् । प्रती॑कवत॒ इति॒ प्रती॑क - व॒ते॒ । यद॒ग्नये᳚ । अ॒ग्नये॒ प्रव॑ते । प्रव॑ते नि॒रव॑पन्न् । प्रव॑त॒ इति॒ प्र - व॒ते॒ । नि॒रव॑प॒न्॒. यानि॑ । नि॒रव॑प॒न्निति॑ निः - अव॑पन्न् । यान्ये॒व । ए॒व पु॒रस्ता᳚त् । पु॒रस्ता॒द् रक्षाꣳ॑सि । रक्षाꣳ॒॒स्यासन्न्॑ \newline

\textbf{Jatai Paata} \newline

1. असु॑रा॒न् जया॑म॒ जया॒मासु॑रा॒ नसु॑रा॒न् जया॑म । \newline
2. जया॑म॒ तत् तज् जया॑म॒ जया॑म॒ तत् । \newline
3. तन् नो॑ न॒ स्तत् तन् नः॑ । \newline
4. नः॒ स॒ह स॒ह नो॑ नः स॒ह । \newline
5. स॒हा स॑दसथ् स॒ह स॒हास॑त् । \newline
6. अ॒स॒ दिती त्य॑स दस॒दिति॑ । \newline
7. इति॒ तत॒ स्तत॒ इतीति॒ ततः॑ । \newline
8. ततो॒ वै वै तत॒ स्ततो॒ वै । \newline
9. वै दे॒वा दे॒वा वै वै दे॒वाः । \newline
10. दे॒वा असु॑रा॒ नसु॑रान् दे॒वा दे॒वा असु॑रान् । \newline
11. असु॑रा नजयन् नजय॒न् नसु॑रा॒ नसु॑रा नजयन्न् । \newline
12. अ॒ज॒य॒न् ते ते॑ ऽजयन् नजय॒न् ते । \newline
13. ते ऽसु॑रा॒ नसु॑रा॒न् ते ते ऽसु॑रान् । \newline
14. असु॑रान् जि॒त्वा जि॒त्वा ऽसु॑रा॒ नसु॑रान् जि॒त्वा । \newline
15. जि॒त्वा रक्षाꣳ॑सि॒ रक्षाꣳ॑सि जि॒त्वा जि॒त्वा रक्षाꣳ॑सि । \newline
16. रक्षाꣳ॒॒ स्यपाप॒ रक्षाꣳ॑सि॒ रक्षाꣳ॒॒ स्यप॑ । \newline
17. अपा॑नुदन्ता नुद॒न्ता पापा॑ नुदन्त । \newline
18. अ॒नु॒द॒न्त॒ तानि॒ तान्य॑नुदन्ता नुदन्त॒ तानि॑ । \newline
19. तानि॒ रक्षाꣳ॑सि॒ रक्षाꣳ॑सि॒ तानि॒ तानि॒ रक्षाꣳ॑सि । \newline
20. रक्षाꣳ॒॒ स्यनृ॑त॒ मनृ॑तꣳ॒॒ रक्षाꣳ॑सि॒ रक्षाꣳ॒॒ स्यनृ॑तम् । \newline
21. अनृ॑त मकर्ता क॒र्तानृ॑त॒ मनृ॑त मकर्त । \newline
22. अ॒क॒र्ते तीत्य॑कर्ता क॒र्ते ति॑ । \newline
23. इति॑ सम॒न्तꣳ स॑म॒न्त मितीति॑ सम॒न्तम् । \newline
24. स॒म॒न्तम् दे॒वान् दे॒वान् थ्स॑म॒न्तꣳ स॑म॒न्तम् दे॒वान् । \newline
25. स॒म॒न्तमिति॑ सं - अ॒न्तम् । \newline
26. दे॒वान् परि॒ परि॑ दे॒वान् दे॒वान् परि॑ । \newline
27. पर्य॑विशन् नविश॒न् परि॒ पर्य॑विशन्न् । \newline
28. अ॒वि॒श॒न् ते ते॑ ऽविशन् नविश॒न् ते । \newline
29. ते दे॒वा दे॒वा स्ते ते दे॒वाः । \newline
30. दे॒वा अ॒ग्ना व॒ग्नौ दे॒वा दे॒वा अ॒ग्नौ । \newline
31. अ॒ग्ना व॑नाथन्ता नाथन्ता॒ग्ना व॒ग्ना व॑नाथन्त । \newline
32. अ॒ना॒थ॒न्त॒ ते ते॑ ऽनाथन्ता नाथन्त॒ ते । \newline
33. ते᳚ ऽग्नये॒ ऽग्नये॒ ते ते᳚ ऽग्नये᳚ । \newline
34. अ॒ग्नये॒ प्रव॑ते॒ प्रव॑ते॒ ऽग्नये॒ ऽग्नये॒ प्रव॑ते । \newline
35. प्रव॑ते पुरो॒डाश॑म् पुरो॒डाश॒म् प्रव॑ते॒ प्रव॑ते पुरो॒डाश᳚म् । \newline
36. प्रव॑त॒ इति॒ प्र - व॒ते॒ । \newline
37. पु॒रो॒डाश॑ म॒ष्टाक॑पाल म॒ष्टाक॑पालम् पुरो॒डाश॑म् पुरो॒डाश॑ म॒ष्टाक॑पालम् । \newline
38. अ॒ष्टाक॑पाल॒म् निर् णिर॒ष्टाक॑पाल म॒ष्टाक॑पाल॒म् निः । \newline
39. अ॒ष्टाक॑पाल॒मित्य॒ष्टा - क॒पा॒ल॒म् । \newline
40. निर॑वपन् नवप॒न् निर् णिर॑वपन्न् । \newline
41. अ॒व॒प॒न् न॒ग्नये॒ ऽग्नये॑ ऽवपन् नवपन् न॒ग्नये᳚ । \newline
42. अ॒ग्नये॑ विबा॒धव॑ते विबा॒धव॑ते॒ ऽग्नये॒ ऽग्नये॑ विबा॒धव॑ते । \newline
43. वि॒बा॒धव॑ते॒ ऽग्नये॒ ऽग्नये॑ विबा॒धव॑ते विबा॒धव॑ते॒ ऽग्नये᳚ । \newline
44. वि॒बा॒धव॑त॒ इति॑ विबा॒ध - व॒ते॒ । \newline
45. अ॒ग्नये॒ प्रती॑कवते॒ प्रती॑कवते॒ ऽग्नये॒ ऽग्नये॒ प्रती॑कवते । \newline
46. प्रती॑कवते॒ यद् यत् प्रती॑कवते॒ प्रती॑कवते॒ यत् । \newline
47. प्रती॑कवत॒ इति॒ प्रती॑क - व॒ते॒ । \newline
48. यद॒ग्नये॒ ऽग्नये॒ यद् यद॒ग्नये᳚ । \newline
49. अ॒ग्नये॒ प्रव॑ते॒ प्रव॑ते॒ ऽग्नये॒ ऽग्नये॒ प्रव॑ते । \newline
50. प्रव॑ते नि॒रव॑पन् नि॒रव॑प॒न् प्रव॑ते॒ प्रव॑ते नि॒रव॑पन्न् । \newline
51. प्रव॑त॒ इति॒ प्र - व॒ते॒ । \newline
52. नि॒रव॑प॒न्॒. यानि॒ यानि॑ नि॒रव॑पन् नि॒रव॑प॒न्॒. यानि॑ । \newline
53. नि॒रव॑प॒न्निति॑ निः - अव॑पन्न् । \newline
54. यान्ये॒वैव यानि॒ यान्ये॒व । \newline
55. ए॒व पु॒रस्ता᳚त् पु॒रस्ता॑ दे॒वैव पु॒रस्ता᳚त् । \newline
56. पु॒रस्ता॒द् रक्षाꣳ॑सि॒ रक्षाꣳ॑सि पु॒रस्ता᳚त् पु॒रस्ता॒द् रक्षाꣳ॑सि । \newline
57. रक्षाꣳ॒॒ स्यास॒न् नास॒न् रक्षाꣳ॑सि॒ रक्षाꣳ॒॒ स्यासन्न्॑ । \newline

\textbf{Ghana Paata } \newline

1. असु॑रा॒न् जया॑म॒ जया॒मासु॑रा॒ नसु॑रा॒न् जया॑म॒ तत् तज् जया॒मासु॑रा॒ नसु॑रा॒न् जया॑म॒ तत् । \newline
2. जया॑म॒ तत् तज् जया॑म॒ जया॑म॒ तन् नो॑ न॒ स्तज् जया॑म॒ जया॑म॒ तन् नः॑ । \newline
3. तन् नो॑ न॒ स्तत् तन् नः॑ स॒ह स॒ह न॒ स्तत् तन् नः॑ स॒ह । \newline
4. नः॒ स॒ह स॒ह नो॑ नः स॒हास॑ दसथ् स॒ह नो॑ नः स॒हास॑त् । \newline
5. स॒हास॑ दसथ् स॒ह स॒हा स॒दिती त्य॑सथ् स॒ह स॒हास॒दिति॑ । \newline
6. अ॒स॒ दिती त्य॑स दस॒ दिति॒ तत॒ स्तत॒ इत्य॑स दस॒दिति॒ ततः॑ । \newline
7. इति॒ तत॒ स्तत॒ इतीति॒ ततो॒ वै वै तत॒ इतीति॒ ततो॒ वै । \newline
8. ततो॒ वै वै तत॒ स्ततो॒ वै दे॒वा दे॒वा वै तत॒ स्ततो॒ वै दे॒वाः । \newline
9. वै दे॒वा दे॒वा वै वै दे॒वा असु॑रा॒ नसु॑रान् दे॒वा वै वै दे॒वा असु॑रान् । \newline
10. दे॒वा असु॑रा॒ नसु॑रान् दे॒वा दे॒वा असु॑रा नजयन् नजय॒न् नसु॑रान् दे॒वा दे॒वा असु॑रा नजयन्न् । \newline
11. असु॑रा नजयन् नजय॒न् नसु॑रा॒ नसु॑रा नजय॒न् ते ते॑ ऽजय॒न् नसु॑रा॒ नसु॑रा नजय॒न् ते । \newline
12. अ॒ज॒य॒न् ते ते॑ ऽजयन् नजय॒न् ते ऽसु॑रा॒ नसु॑रा॒न् ते॑ ऽजयन् नजय॒न् ते ऽसु॑रान् । \newline
13. ते ऽसु॑रा॒ नसु॑रा॒न् ते ते ऽसु॑रान् जि॒त्वा जि॒त्वा ऽसु॑रा॒न् ते ते ऽसु॑रान् जि॒त्वा । \newline
14. असु॑रान् जि॒त्वा जि॒त्वा ऽसु॑रा॒ नसु॑रान् जि॒त्वा रक्षाꣳ॑सि॒ रक्षाꣳ॑सि जि॒त्वा ऽसु॑रा॒ नसु॑रान् जि॒त्वा रक्षाꣳ॑सि । \newline
15. जि॒त्वा रक्षाꣳ॑सि॒ रक्षाꣳ॑सि जि॒त्वा जि॒त्वा रक्षाꣳ॒॒ स्यपाप॒ रक्षाꣳ॑सि जि॒त्वा जि॒त्वा रक्षाꣳ॒॒ स्यप॑ । \newline
16. रक्षाꣳ॒॒ स्यपाप॒ रक्षाꣳ॑सि॒ रक्षाꣳ॒॒ स्यपा॑नुदन्ता नुद॒न्ताप॒ रक्षाꣳ॑सि॒ रक्षाꣳ॒॒ स्यपा॑नुदन्त । \newline
17. अपा॑नुदन्ता नुद॒न्ता पापा॑नुदन्त॒ तानि॒ तान्य॑नुद॒न्ता पापा॑नुदन्त॒ तानि॑ । \newline
18. अ॒नु॒द॒न्त॒ तानि॒ तान्य॑नुदन्ता नुदन्त॒ तानि॒ रक्षाꣳ॑सि॒ रक्षाꣳ॑सि॒ तान्य॑नुदन्ता नुदन्त॒ तानि॒ रक्षाꣳ॑सि । \newline
19. तानि॒ रक्षाꣳ॑सि॒ रक्षाꣳ॑सि॒ तानि॒ तानि॒ रक्षाꣳ॒॒ स्यनृ॑त॒ मनृ॑तꣳ॒॒ रक्षाꣳ॑सि॒ तानि॒ तानि॒ 
रक्षाꣳ॒॒ स्यनृ॑तम् । \newline
20. रक्षाꣳ॒॒ स्यनृ॑त॒ मनृ॑तꣳ॒॒ रक्षाꣳ॑सि॒ रक्षाꣳ॒॒ स्यनृ॑त मकर्ता क॒र्ता नृ॑तꣳ॒॒ रक्षाꣳ॑सि॒ रक्षाꣳ॒॒ स्यनृ॑त मकर्त । \newline
21. अनृ॑त मकर्ता क॒र्ता नृ॑त॒ मनृ॑त मक॒र्ते तीत्य॑क॒र्ता नृ॑त॒ मनृ॑त मक॒र्ते ति॑ । \newline
22. अ॒क॒र्ते तीत्य॑कर्ता क॒र्ते ति॑ सम॒न्तꣳ स॑म॒न्त मित्य॑कर्ता क॒र्ते ति॑ सम॒न्तम् । \newline
23. इति॑ सम॒न्तꣳ स॑म॒न्त मितीति॑ सम॒न्तम् दे॒वान् दे॒वान् थ्स॑म॒न्त मितीति॑ सम॒न्तम् दे॒वान् । \newline
24. स॒म॒न्तम् दे॒वान् दे॒वान् थ्स॑म॒न्तꣳ स॑म॒न्तम् दे॒वान् परि॒ परि॑ दे॒वान् थ्स॑म॒न्तꣳ स॑म॒न्तम् दे॒वान् परि॑ । \newline
25. स॒म॒न्तमिति॑ सं - अ॒न्तम् । \newline
26. दे॒वान् परि॒ परि॑ दे॒वान् दे॒वान् पर्य॑विशन् नविश॒न् परि॑ दे॒वान् दे॒वान् पर्य॑विशन्न् । \newline
27. पर्य॑विशन् नविश॒न् परि॒ पर्य॑विश॒न् ते ते॑ ऽविश॒न् परि॒ पर्य॑विश॒न् ते । \newline
28. अ॒वि॒श॒न् ते ते॑ ऽविशन् नविश॒न् ते दे॒वा दे॒वास्ते॑ ऽविशन् नविश॒न् ते दे॒वाः । \newline
29. ते दे॒वा दे॒वा स्ते ते दे॒वा अ॒ग्ना व॒ग्नौ दे॒वा स्ते ते दे॒वा अ॒ग्नौ । \newline
30. दे॒वा अ॒ग्ना व॒ग्नौ दे॒वा दे॒वा अ॒ग्ना व॑नाथन्ता नाथन्ता॒ग्नौ दे॒वा दे॒वा अ॒ग्ना व॑नाथन्त । \newline
31. अ॒ग्ना व॑नाथन्ता नाथन्ता॒ग्ना व॒ग्ना व॑नाथन्त॒ ते ते॑ ऽनाथन्ता॒ग्ना व॒ग्ना व॑नाथन्त॒ ते । \newline
32. अ॒ना॒थ॒न्त॒ ते ते॑ ऽनाथन्ता नाथन्त॒ ते᳚ ऽग्नये॒ ऽग्नये॒ ते॑ ऽनाथन्ता नाथन्त॒ ते᳚ ऽग्नये᳚ । \newline
33. ते᳚ ऽग्नये॒ ऽग्नये॒ ते ते᳚ ऽग्नये॒ प्रव॑ते॒ प्रव॑ते॒ ऽग्नये॒ ते ते᳚ ऽग्नये॒ प्रव॑ते । \newline
34. अ॒ग्नये॒ प्रव॑ते॒ प्रव॑ते॒ ऽग्नये॒ ऽग्नये॒ प्रव॑ते पुरो॒डाश॑म् पुरो॒डाश॒म् प्रव॑ते॒ ऽग्नये॒ ऽग्नये॒ प्रव॑ते पुरो॒डाश᳚म् । \newline
35. प्रव॑ते पुरो॒डाश॑म् पुरो॒डाश॒म् प्रव॑ते॒ प्रव॑ते पुरो॒डाश॑ म॒ष्टाक॑पाल म॒ष्टाक॑पालम् पुरो॒डाश॒म् प्रव॑ते॒ प्रव॑ते पुरो॒डाश॑ म॒ष्टाक॑पालम् । \newline
36. प्रव॑त॒ इति॒ प्र - व॒ते॒ । \newline
37. पु॒रो॒डाश॑ म॒ष्टाक॑पाल म॒ष्टाक॑पालम् पुरो॒डाश॑म् पुरो॒डाश॑ म॒ष्टाक॑पाल॒म् निर् णिर॒ष्टाक॑पालम् पुरो॒डाश॑म् पुरो॒डाश॑ म॒ष्टाक॑पाल॒म् निः । \newline
38. अ॒ष्टाक॑पाल॒म् निर् णिर॒ष्टाक॑पाल म॒ष्टाक॑पाल॒म् निर॑वपन् नवप॒न् निर॒ष्टाक॑पाल म॒ष्टाक॑पाल॒म् निर॑वपन्न् । \newline
39. अ॒ष्टाक॑पाल॒मित्य॒ष्टा - क॒पा॒ल॒म् । \newline
40. निर॑वपन् नवप॒न् निर् णिर॑वपन् न॒ग्नये॒ ऽग्नये॑ ऽवप॒न् निर् णिर॑वपन् न॒ग्नये᳚ । \newline
41. अ॒व॒प॒न् न॒ग्नये॒ ऽग्नये॑ ऽवपन् नवपन् न॒ग्नये॑ विबा॒धव॑ते विबा॒धव॑ते॒ ऽग्नये॑ ऽवपन् नवपन् न॒ग्नये॑ विबा॒धव॑ते । \newline
42. अ॒ग्नये॑ विबा॒धव॑ते विबा॒धव॑ते॒ ऽग्नये॒ ऽग्नये॑ विबा॒धव॑ते॒ ऽग्नये॒ ऽग्नये॑ विबा॒धव॑ते॒ ऽग्नये॒ ऽग्नये॑ विबा॒धव॑ते॒ ऽग्नये᳚ । \newline
43. वि॒बा॒धव॑ते॒ ऽग्नये॒ ऽग्नये॑ विबा॒धव॑ते विबा॒धव॑ते॒ ऽग्नये॒ प्रती॑कवते॒ प्रती॑कवते॒ ऽग्नये॑ विबा॒धव॑ते विबा॒धव॑ते॒ ऽग्नये॒ प्रती॑कवते । \newline
44. वि॒बा॒धव॑त॒ इति॑ विबा॒ध - व॒ते॒ । \newline
45. अ॒ग्नये॒ प्रती॑कवते॒ प्रती॑कवते॒ ऽग्नये॒ ऽग्नये॒ प्रती॑कवते॒ यद् यत् प्रती॑कवते॒ ऽग्नये॒ ऽग्नये॒ प्रती॑कवते॒ यत् । \newline
46. प्रती॑कवते॒ यद् यत् प्रती॑कवते॒ प्रती॑कवते॒ यद॒ग्नये॒ ऽग्नये॒ यत् प्रती॑कवते॒ प्रती॑कवते॒ यद॒ग्नये᳚ । \newline
47. प्रती॑कवत॒ इति॒ प्रती॑क - व॒ते॒ । \newline
48. यद॒ग्नये॒ ऽग्नये॒ यद् यद॒ग्नये॒ प्रव॑ते॒ प्रव॑ते॒ ऽग्नये॒ यद् यद॒ग्नये॒ प्रव॑ते । \newline
49. अ॒ग्नये॒ प्रव॑ते॒ प्रव॑ते॒ ऽग्नये॒ ऽग्नये॒ प्रव॑ते नि॒रव॑पन् नि॒रव॑प॒न् प्रव॑ते॒ ऽग्नये॒ ऽग्नये॒ प्रव॑ते नि॒रव॑पन्न् । \newline
50. प्रव॑ते नि॒रव॑पन् नि॒रव॑प॒न् प्रव॑ते॒ प्रव॑ते नि॒रव॑प॒न्॒. यानि॒ यानि॑ नि॒रव॑प॒न् प्रव॑ते॒ प्रव॑ते नि॒रव॑प॒न्॒. यानि॑ । \newline
51. प्रव॑त॒ इति॒ प्र - व॒ते॒ । \newline
52. नि॒रव॑प॒न्॒. यानि॒ यानि॑ नि॒रव॑पन् नि॒रव॑प॒न्॒. यान्ये॒वैव यानि॑ नि॒रव॑पन् नि॒रव॑प॒न्॒. यान्ये॒व । \newline
53. नि॒रव॑प॒न्निति॑ निः - अव॑पन्न् । \newline
54. यान्ये॒वैव यानि॒ यान्ये॒व पु॒रस्ता᳚त् पु॒रस्ता॑दे॒व यानि॒ यान्ये॒व पु॒रस्ता᳚त् । \newline
55. ए॒व पु॒रस्ता᳚त् पु॒रस्ता॑ दे॒वैव पु॒रस्ता॒द् रक्षाꣳ॑सि॒ रक्षाꣳ॑सि पु॒रस्ता॑ दे॒वैव पु॒रस्ता॒द् रक्षाꣳ॑सि । \newline
56. पु॒रस्ता॒द् रक्षाꣳ॑सि॒ रक्षाꣳ॑सि पु॒रस्ता᳚त् पु॒रस्ता॒द् रक्षाꣳ॒॒ स्यास॒न् नास॒न् रक्षाꣳ॑सि पु॒रस्ता᳚त् पु॒रस्ता॒द् रक्षाꣳ॒॒ स्यासन्न्॑ । \newline
57. रक्षाꣳ॒॒ स्यास॒न् नास॒न् रक्षाꣳ॑सि॒ रक्षाꣳ॒॒ स्यास॒न् तानि॒ तान्यास॒न् रक्षाꣳ॑सि॒ रक्षाꣳ॒॒ स्यास॒न् तानि॑ । \newline
\pagebreak
\markright{ TS 2.4.1.3  \hfill https://www.vedavms.in \hfill}

\section{ TS 2.4.1.3 }

\textbf{TS 2.4.1.3 } \newline
\textbf{Samhita Paata} \newline

-स॒न्तानि॒ तेन॒ प्राणु॑दन्त॒ यद॒ग्नये॑ विबा॒धव॑ते॒ यान्ये॒वाभितो॒ रक्षाꣳ॒॒स्यास॒न् तानि॒ तेन॒ व्य॑बाधन्त॒ यद॒ग्नये॒ प्रती॑कवते॒ यान्ये॒व प॒श्चाद्-रक्षाꣳ॒॒स्यास॒न् तानि॒ तेनापा॑नुदन्त॒ ततो॑ दे॒वा अभ॑व॒न् परासु॑रा॒ यो भ्रातृ॑व्यवा॒न्थ् स्याथ् स स्पर्द्ध॑मान ए॒तयेष्‌ट्या॑ यजेता॒ग्नये॒ प्रव॑ते पुरो॒डाश॑म॒ष्टाक॑पालं॒ निर्व॑पेद॒ग्नये॑ विबा॒धव॑ते॒ - [  ] \newline

\textbf{Pada Paata} \newline

आसन्न्॑ । तानि॑ । तेन॑ । प्रेति॑ । अ॒नु॒द॒न्त॒ । यत् । अ॒ग्नये᳚ । वि॒बा॒धव॑त॒ इति॑ विबा॒ध - व॒ते॒ । यानि॑ । ए॒व । अ॒भितः॑ । रक्षाꣳ॑सि । आसन्न्॑ । तानि॑ । तेन॑ । वीति॑ । अ॒बा॒ध॒न्त॒ । यत् । अ॒ग्नये᳚ । प्रती॑कवत॒ इति॒ प्रती॑क - व॒ते॒ । यानि॑ । ए॒व । प॒श्चात् । रक्षाꣳ॑सि । आसन्न्॑ । तानि॑ । तेन॑ ।   अपेति॑ । अ॒नु॒द॒न्त॒ । ततः॑ । दे॒वाः । अभ॑वन्न् । परेति॑ । असु॑राः । यः । भ्रातृ॑व्यवा॒निति॒ भ्रातृ॑व्य - वा॒न् । स्यात् । सः । स्पर्द्ध॑मानः । ए॒तया᳚ । इष्ट्या᳚ । य॒जे॒त॒ । अ॒ग्नये᳚ । प्रव॑त॒ इति॒ प्र - व॒ते॒ । पु॒रो॒डाश᳚म् । अ॒ष्टाक॑पाल॒मित्य॒ष्टा - क॒पा॒ल॒म् । निरिति॑ । व॒पे॒त् । अ॒ग्नये᳚ । वि॒बा॒धव॑त॒ इति॑ विबा॒ध - व॒ते॒ ।  \newline


\textbf{Krama Paata} \newline

आस॒न् तानि॑ । तानि॒ तेन॑ । तेन॒ प्र । प्राणु॑दन्त । अ॒नु॒द॒न्त॒ यत् । यद॒ग्नये᳚ । अ॒ग्नये॑ विबा॒धव॑ते । वि॒बा॒धव॑ते॒ यानि॑ । वि॒बा॒धव॑त॒ इति॑ विबा॒ध - व॒ते॒ । यान्ये॒व । ए॒वाभितः॑ । अ॒भितो॒ रक्षाꣳ॑सि । रक्षाꣳ॒॒स्यासन्न्॑ । आस॒न् तानि॑ । तानि॒ तेन॑ । तेन॒ वि । व्य॑बाधन्त । अ॒बा॒ध॒न्त॒ यत् । यद॒ग्नये᳚ । अ॒ग्नये॒ प्रती॑कवते । प्रती॑कवते॒ यानि॑ । प्रती॑कवत॒ इति॒ प्रती॑क - व॒ते॒ । यान्ये॒व । ए॒व प॒श्चात् । प॒श्चाद् रक्षाꣳ॑सि । रक्षाꣳ॒॒स्यासन्न्॑ । आस॒न् तानि॑ । तानि॒ तेन॑ । तेनाप॑ । अपा॑नुदन्त । अ॒नु॒द॒न्त॒ ततः॑ । ततो॑ दे॒वाः । दे॒वा अभ॑वन्न् । अभ॑व॒न् परा᳚ । परा ऽसु॑राः । असु॑रा॒ यः । यो भ्रातृ॑व्यवान् । भ्रातृ॑व्यवा॒न्थ् स्यात् । भ्रातृ॑व्यवा॒निति॒ भ्रातृ॑व्य - वा॒न्॒ । स्याथ् सः । स स्पर्द्ध॑मानः । स्पर्द्ध॑मान ए॒तया᳚ । ए॒तयेष्ट्या᳚ । इष्ट्या॑ यजेत । य॒जे॒ता॒ग्नये᳚ । अ॒ग्नये॒ प्रव॑ते । प्रव॑ते पुरो॒डाश᳚म् । प्रव॑त॒ इति॒ प्र - व॒ते॒ । पु॒रो॒डाश॑म॒ष्टाक॑पालम् । अ॒ष्टाक॑पाल॒म् निः । अ॒ष्टाक॑पाल॒मित्य॒ष्टा - क॒पा॒ल॒म् । निर् व॑पेत् । व॒पे॒द॒ग्नये᳚ । अ॒ग्नये॑ विबा॒धव॑ते । वि॒बा॒धव॑ते॒ ऽग्नये᳚ । वि॒बा॒धव॑त॒ इति॑ विबा॒ध - व॒ते॒ \newline

\textbf{Jatai Paata} \newline

1. आस॒न् तानि॒ तान्यास॒न् नास॒न् तानि॑ । \newline
2. तानि॒ तेन॒ तेन॒ तानि॒ तानि॒ तेन॑ । \newline
3. तेन॒ प्र प्र तेन॒ तेन॒ प्र । \newline
4. प्राणु॑दन्ता नुदन्त॒ प्र प्राणु॑दन्त । \newline
5. अ॒नु॒द॒न्त॒ यद् यद॑नुदन्ता नुदन्त॒ यत् । \newline
6. यद॒ग्नये॒ ऽग्नये॒ यद् यद॒ग्नये᳚ । \newline
7. अ॒ग्नये॑ विबा॒धव॑ते विबा॒धव॑ते॒ ऽग्नये॒ ऽग्नये॑ विबा॒धव॑ते । \newline
8. वि॒बा॒धव॑ते॒ यानि॒ यानि॑ विबा॒धव॑ते विबा॒धव॑ते॒ यानि॑ । \newline
9. वि॒बा॒धव॑त॒ इति॑ विबा॒ध - व॒ते॒ । \newline
10. यान्ये॒वैव यानि॒ यान्ये॒व । \newline
11. ए॒वाभितो॒ ऽभित॑ ए॒वैवाभितः॑ । \newline
12. अ॒भितो॒ रक्षाꣳ॑सि॒ रक्षाꣳ॑ स्य॒भितो॒ ऽभितो॒ रक्षाꣳ॑सि । \newline
13. रक्षाꣳ॒॒ स्यास॒न् नास॒न् रक्षाꣳ॑सि॒ रक्षाꣳ॒॒ स्यासन्न्॑ । \newline
14. आस॒न् तानि॒ तान्यास॒न् नास॒न् तानि॑ । \newline
15. तानि॒ तेन॒ तेन॒ तानि॒ तानि॒ तेन॑ । \newline
16. तेन॒ वि वि तेन॒ तेन॒ वि । \newline
17. व्य॑बाधन्ता बाधन्त॒ वि व्य॑बाधन्त । \newline
18. अ॒बा॒ध॒न्त॒ यद् यद॑बाधन्ता बाधन्त॒ यत् । \newline
19. यद॒ग्नये॒ ऽग्नये॒ यद् यद॒ग्नये᳚ । \newline
20. अ॒ग्नये॒ प्रती॑कवते॒ प्रती॑कवते॒ ऽग्नये॒ ऽग्नये॒ प्रती॑कवते । \newline
21. प्रती॑कवते॒ यानि॒ यानि॒ प्रती॑कवते॒ प्रती॑कवते॒ यानि॑ । \newline
22. प्रती॑कवत॒ इति॒ प्रती॑क - व॒ते॒ । \newline
23. यान्ये॒वैव यानि॒ यान्ये॒व । \newline
24. ए॒व प॒श्चात् प॒श्चा दे॒वैव प॒श्चात् । \newline
25. प॒श्चाद् रक्षाꣳ॑सि॒ रक्षाꣳ॑सि प॒श्चात् प॒श्चाद् रक्षाꣳ॑सि । \newline
26. रक्षाꣳ॒॒ स्यास॒न् नास॒न् रक्षाꣳ॑सि॒ रक्षाꣳ॒॒स्यासन्न्॑ । \newline
27. आस॒न् तानि॒ तान्यास॒न् नास॒न् तानि॑ । \newline
28. तानि॒ तेन॒ तेन॒ तानि॒ तानि॒ तेन॑ । \newline
29. तेनापाप॒ तेन॒ तेनाप॑ । \newline
30. अपा॑नुदन्ता नुद॒न्ता पापा॑ नुदन्त । \newline
31. अ॒नु॒द॒न्त॒ तत॒ स्ततो॑ ऽनुदन्ता नुदन्त॒ ततः॑ । \newline
32. ततो॑ दे॒वा दे॒वा स्तत॒ स्ततो॑ दे॒वाः । \newline
33. दे॒वा अभ॑व॒न् नभ॑वन् दे॒वा दे॒वा अभ॑वन्न् । \newline
34. अभ॑व॒न् परा॒ परा ऽभ॑व॒न् नभ॑व॒न् परा᳚ । \newline
35. परा ऽसु॑रा॒ असु॑राः॒ परा॒ परा ऽसु॑राः । \newline
36. असु॑रा॒ यो यो ऽसु॑रा॒ असु॑रा॒ यः । \newline
37. यो भ्रातृ॑व्यवा॒न् भ्रातृ॑व्यवा॒न्॒. यो यो भ्रातृ॑व्यवान् । \newline
38. भ्रातृ॑व्यवा॒न् थ्स्याथ् स्याद् भ्रातृ॑व्यवा॒न् भ्रातृ॑व्यवा॒न् थ्स्यात् । \newline
39. भ्रातृ॑व्यवा॒निति॒ भ्रातृ॑व्य - वा॒न् । \newline
40. स्याथ् स स स्याथ् स्याथ् सः । \newline
41. स स्पर्द्ध॑मानः॒ स्पर्द्ध॑मानः॒ स स स्पर्द्ध॑मानः । \newline
42. स्पर्द्ध॑मान ए॒तयै॒तया॒ स्पर्द्ध॑मानः॒ स्पर्द्ध॑मान ए॒तया᳚ । \newline
43. ए॒तयेष्ट्ये ष्ट्यै॒त यै॒तयेष्ट्या᳚ । \newline
44. इष्ट्या॑ यजेत यजे॒ते ष्ट्येष्ट्या॑ यजेत । \newline
45. य॒जे॒ता॒ग्नये॒ ऽग्नये॑ यजेत यजेता॒ग्नये᳚ । \newline
46. अ॒ग्नये॒ प्रव॑ते॒ प्रव॑ते॒ ऽग्नये॒ ऽग्नये॒ प्रव॑ते । \newline
47. प्रव॑ते पुरो॒डाश॑म् पुरो॒डाश॒म् प्रव॑ते॒ प्रव॑ते पुरो॒डाश᳚म् । \newline
48. प्रव॑त॒ इति॒ प्र - व॒ते॒ । \newline
49. पु॒रो॒डाश॑ म॒ष्टाक॑पाल म॒ष्टाक॑पालम् पुरो॒डाश॑म् पुरो॒डाश॑ म॒ष्टाक॑पालम् । \newline
50. अ॒ष्टाक॑पाल॒म् निर् णिर॒ष्टाक॑पाल म॒ष्टाक॑पाल॒म् निः । \newline
51. अ॒ष्टाक॑पाल॒मित्य॒ष्टा - क॒पा॒ल॒म् । \newline
52. निर् व॑पेद् वपे॒न् निर् णिर् व॑पेत् । \newline
53. व॒पे॒द॒ग्नये॒ ऽग्नये॑ वपेद् वपेद॒ग्नये᳚ । \newline
54. अ॒ग्नये॑ विबा॒धव॑ते विबा॒धव॑ते॒ ऽग्नये॒ ऽग्नये॑ विबा॒धव॑ते । \newline
55. वि॒बा॒धव॑ते॒ ऽग्नये॒ ऽग्नये॑ विबा॒धव॑ते विबा॒धव॑ते॒ ऽग्नये᳚ । \newline
56. वि॒बा॒धव॑त॒ इति॑ विबा॒ध - व॒ते॒ । \newline

\textbf{Ghana Paata } \newline

1. आस॒न् तानि॒ तान्यास॒न् नास॒न् तानि॒ तेन॒ तेन॒ तान्यास॒न् नास॒न् तानि॒ तेन॑ । \newline
2. तानि॒ तेन॒ तेन॒ तानि॒ तानि॒ तेन॒ प्र प्र तेन॒ तानि॒ तानि॒ तेन॒ प्र । \newline
3. तेन॒ प्र प्र तेन॒ तेन॒ प्राणु॑दन्ता नुदन्त॒ प्र तेन॒ तेन॒ प्राणु॑दन्त । \newline
4. प्राणु॑दन्ता नुदन्त॒ प्र प्राणु॑दन्त॒ यद् यद॑नुदन्त॒ प्र प्राणु॑दन्त॒ यत् । \newline
5. अ॒नु॒द॒न्त॒ यद् यद॑नुदन्ता नुदन्त॒ यद॒ग्नये॒ ऽग्नये॒ यद॑नुदन्ता नुदन्त॒ यद॒ग्नये᳚ । \newline
6. यद॒ग्नये॒ ऽग्नये॒ यद् यद॒ग्नये॑ विबा॒धव॑ते विबा॒धव॑ते॒ ऽग्नये॒ यद् यद॒ग्नये॑ विबा॒धव॑ते । \newline
7. अ॒ग्नये॑ विबा॒धव॑ते विबा॒धव॑ते॒ ऽग्नये॒ ऽग्नये॑ विबा॒धव॑ते॒ यानि॒ यानि॑ विबा॒धव॑ते॒ ऽग्नये॒ ऽग्नये॑ विबा॒धव॑ते॒ यानि॑ । \newline
8. वि॒बा॒धव॑ते॒ यानि॒ यानि॑ विबा॒धव॑ते विबा॒धव॑ते॒ यान्ये॒वैव यानि॑ विबा॒धव॑ते विबा॒धव॑ते॒ यान्ये॒व । \newline
9. वि॒बा॒धव॑त॒ इति॑ विबा॒ध - व॒ते॒ । \newline
10. यान्ये॒वैव यानि॒ यान्ये॒वा भितो॒ ऽभित॑ ए॒व यानि॒ यान्ये॒वाभितः॑ । \newline
11. ए॒वाभितो॒ ऽभित॑ ए॒वैवाभितो॒ रक्षाꣳ॑सि॒ रक्षाꣳ॑ स्य॒भित॑ ए॒वैवाभितो॒ रक्षाꣳ॑सि । \newline
12. अ॒भितो॒ रक्षाꣳ॑सि॒ रक्षाꣳ॑ स्य॒भितो॒ ऽभितो॒ रक्षाꣳ॒॒ स्यास॒न् नास॒न् रक्षाꣳ॑ स्य॒भितो॒ ऽभितो॒ 
रक्षाꣳ॒॒ स्यासन्न्॑ । \newline
13. रक्षाꣳ॒॒ स्यास॒न् नास॒न् रक्षाꣳ॑सि॒ रक्षाꣳ॒॒ स्यास॒न् तानि॒ तान्यास॒न् रक्षाꣳ॑सि॒ रक्षाꣳ॒॒ स्यास॒न् तानि॑ । \newline
14. आस॒न् तानि॒ तान्यास॒न् नास॒न् तानि॒ तेन॒ तेन॒ तान्यास॒न् नास॒न् तानि॒ तेन॑ । \newline
15. तानि॒ तेन॒ तेन॒ तानि॒ तानि॒ तेन॒ वि वि तेन॒ तानि॒ तानि॒ तेन॒ वि । \newline
16. तेन॒ वि वि तेन॒ तेन॒ व्य॑बाधन्ता बाधन्त॒ वि तेन॒ तेन॒ व्य॑बाधन्त । \newline
17. व्य॑बाधन्ता बाधन्त॒ वि व्य॑बाधन्त॒ यद् यद॑बाधन्त॒ वि व्य॑बाधन्त॒ यत् । \newline
18. अ॒बा॒ध॒न्त॒ यद् यद॑बाधन्ता बाधन्त॒ यद॒ग्नये॒ ऽग्नये॒ यद॑बाधन्ता बाधन्त॒ यद॒ग्नये᳚ । \newline
19. यद॒ग्नये॒ ऽग्नये॒ यद् यद॒ग्नये॒ प्रती॑कवते॒ प्रती॑कवते॒ ऽग्नये॒ यद् यद॒ग्नये॒ प्रती॑कवते । \newline
20. अ॒ग्नये॒ प्रती॑कवते॒ प्रती॑कवते॒ ऽग्नये॒ ऽग्नये॒ प्रती॑कवते॒ यानि॒ यानि॒ प्रती॑कवते॒ ऽग्नये॒ ऽग्नये॒ प्रती॑कवते॒ यानि॑ । \newline
21. प्रती॑कवते॒ यानि॒ यानि॒ प्रती॑कवते॒ प्रती॑कवते॒ यान्ये॒वैव यानि॒ प्रती॑कवते॒ प्रती॑कवते॒ यान्ये॒व । \newline
22. प्रती॑कवत॒ इति॒ प्रती॑क - व॒ते॒ । \newline
23. यान्ये॒वैव यानि॒ यान्ये॒व प॒श्चात् प॒श्चादे॒व यानि॒ यान्ये॒व प॒श्चात् । \newline
24. ए॒व प॒श्चात् प॒श्चा दे॒वैव प॒श्चाद् रक्षाꣳ॑सि॒ रक्षाꣳ॑सि प॒श्चा दे॒वैव प॒श्चाद् रक्षाꣳ॑सि । \newline
25. प॒श्चाद् रक्षाꣳ॑सि॒ रक्षाꣳ॑सि प॒श्चात् प॒श्चाद् रक्षाꣳ॒॒ स्यास॒न् नास॒न् रक्षाꣳ॑सि प॒श्चात् प॒श्चाद् रक्षाꣳ॒॒ स्यासन्न्॑ । \newline
26. रक्षाꣳ॒॒ स्यास॒न् नास॒न् रक्षाꣳ॑सि॒ रक्षाꣳ॒॒ स्यास॒न् तानि॒ तान्यास॒न् रक्षाꣳ॑सि॒ रक्षाꣳ॒॒ स्यास॒न् तानि॑ । \newline
27. आस॒न् तानि॒ तान्यास॒न् नास॒न् तानि॒ तेन॒ तेन॒ तान्यास॒न् नास॒न् तानि॒ तेन॑ । \newline
28. तानि॒ तेन॒ तेन॒ तानि॒ तानि॒ तेनापाप॒ तेन॒ तानि॒ तानि॒ तेनाप॑ । \newline
29. तेनापाप॒ तेन॒ तेनापा॑नुदन्ता नुद॒न्ताप॒ तेन॒ तेनापा॑नुदन्त । \newline
30. अपा॑नुदन्ता नुद॒न्ता पापा॑नुदन्त॒ तत॒ स्ततो॑ ऽनुद॒न्ता पापा॑नुदन्त॒ ततः॑ । \newline
31. अ॒नु॒द॒न्त॒ तत॒ स्ततो॑ ऽनुदन्ता नुदन्त॒ ततो॑ दे॒वा दे॒वा स्ततो॑ ऽनुदन्ता नुदन्त॒ ततो॑ दे॒वाः । \newline
32. ततो॑ दे॒वा दे॒वा स्तत॒ स्ततो॑ दे॒वा अभ॑व॒न् नभ॑वन् दे॒वा स्तत॒ स्ततो॑ दे॒वा अभ॑वन्न् । \newline
33. दे॒वा अभ॑व॒न् नभ॑वन् दे॒वा दे॒वा अभ॑व॒न् परा॒ परा ऽभ॑वन् दे॒वा दे॒वा अभ॑व॒न् परा᳚ । \newline
34. अभ॑व॒न् परा॒ परा ऽभ॑व॒न् नभ॑व॒न् परा ऽसु॑रा॒ असु॑राः॒ परा ऽभ॑व॒न् नभ॑व॒न् परा ऽसु॑राः । \newline
35. परा ऽसु॑रा॒ असु॑राः॒ परा॒ परा ऽसु॑रा॒ यो यो ऽसु॑राः॒ परा॒ परा ऽसु॑रा॒ यः । \newline
36. असु॑रा॒ यो यो ऽसु॑रा॒ असु॑रा॒ यो भ्रातृ॑व्यवा॒न् भ्रातृ॑व्यवा॒न्॒. यो ऽसु॑रा॒ असु॑रा॒ यो भ्रातृ॑व्यवान् । \newline
37. यो भ्रातृ॑व्यवा॒न् भ्रातृ॑व्यवा॒न्॒. यो यो भ्रातृ॑व्यवा॒न् थ्स्याथ् स्याद् भ्रातृ॑व्यवा॒न्॒. यो यो भ्रातृ॑व्यवा॒न् थ्स्यात् । \newline
38. भ्रातृ॑व्यवा॒न् थ्स्याथ् स्याद् भ्रातृ॑व्यवा॒न् भ्रातृ॑व्यवा॒न् थ्स्याथ् स स स्याद् भ्रातृ॑व्यवा॒न् भ्रातृ॑व्यवा॒न् थ्स्याथ् सः । \newline
39. भ्रातृ॑व्यवा॒निति॒ भ्रातृ॑व्य - वा॒न् । \newline
40. स्याथ् स स स्याथ् स्याथ् स स्पर्द्ध॑मानः॒ स्पर्द्ध॑मानः॒ स स्याथ् स्याथ् स स्पर्द्ध॑मानः । \newline
41. स स्पर्द्ध॑मानः॒ स्पर्द्ध॑मानः॒ स स स्पर्द्ध॑मान ए॒त यै॒तया॒ स्पर्द्ध॑मानः॒ स स स्पर्द्ध॑मान ए॒तया᳚ । \newline
42. स्पर्द्ध॑मान ए॒त यै॒तया॒ स्पर्द्ध॑मानः॒ स्पर्द्ध॑मान ए॒त येष्ट्येष्ट् यै॒तया॒ स्पर्द्ध॑मानः॒ स्पर्द्ध॑मान ए॒तयेष्ट्या᳚ । \newline
43. ए॒त येष्ट्येष्ट्यै॒त यै॒तयेष्ट्या॑ यजेत यजे॒ते ष्ट्यै॒त यै॒तयेष्ट्या॑ यजेत । \newline
44. इष्ट्या॑ यजेत यजे॒ते ष्ट्येष्ट्या॑ यजेता॒ग्नये॒ ऽग्नये॑ यजे॒ते ष्ट्येष्ट्या॑ यजेता॒ग्नये᳚ । \newline
45. य॒जे॒ता॒ग्नये॒ ऽग्नये॑ यजेत यजेता॒ग्नये॒ प्रव॑ते॒ प्रव॑ते॒ ऽग्नये॑ यजेत यजेता॒ग्नये॒ प्रव॑ते । \newline
46. अ॒ग्नये॒ प्रव॑ते॒ प्रव॑ते॒ ऽग्नये॒ ऽग्नये॒ प्रव॑ते पुरो॒डाश॑म् पुरो॒डाश॒म् प्रव॑ते॒ ऽग्नये॒ ऽग्नये॒ प्रव॑ते पुरो॒डाश᳚म् । \newline
47. प्रव॑ते पुरो॒डाश॑म् पुरो॒डाश॒म् प्रव॑ते॒ प्रव॑ते पुरो॒डाश॑ म॒ष्टाक॑पाल म॒ष्टाक॑पालम् पुरो॒डाश॒म् प्रव॑ते॒ प्रव॑ते पुरो॒डाश॑ म॒ष्टाक॑पालम् । \newline
48. प्रव॑त॒ इति॒ प्र - व॒ते॒ । \newline
49. पु॒रो॒डाश॑ म॒ष्टाक॑पाल म॒ष्टाक॑पालम् पुरो॒डाश॑म् पुरो॒डाश॑ म॒ष्टाक॑पाल॒म् निर् णिर॒ष्टाक॑पालम् पुरो॒डाश॑म् पुरो॒डाश॑ म॒ष्टाक॑पाल॒म् निः । \newline
50. अ॒ष्टाक॑पाल॒म् निर् णिर॒ष्टाक॑पाल म॒ष्टाक॑पाल॒म् निर् व॑पेद् वपे॒न् निर॒ष्टाक॑पाल म॒ष्टाक॑पाल॒म् निर् व॑पेत् । \newline
51. अ॒ष्टाक॑पाल॒मित्य॒ष्टा - क॒पा॒ल॒म् । \newline
52. निर् व॑पेद् वपे॒न् निर् णिर् व॑पे द॒ग्नये॒ ऽग्नये॑ वपे॒न् निर् णिर् व॑पे द॒ग्नये᳚ । \newline
53. व॒पे॒द॒ग्नये॒ ऽग्नये॑ वपेद् वपे द॒ग्नये॑ विबा॒धव॑ते विबा॒धव॑ते॒ ऽग्नये॑ वपेद् वपे द॒ग्नये॑ विबा॒धव॑ते । \newline
54. अ॒ग्नये॑ विबा॒धव॑ते विबा॒धव॑ते॒ ऽग्नये॒ ऽग्नये॑ विबा॒धव॑ते॒ ऽग्नये॒ ऽग्नये॑ विबा॒धव॑ते॒ ऽग्नये॒ ऽग्नये॑ विबा॒धव॑ते॒ ऽग्नये᳚ । \newline
55. वि॒बा॒धव॑ते॒ ऽग्नये॒ ऽग्नये॑ विबा॒धव॑ते विबा॒धव॑ते॒ ऽग्नये॒ प्रती॑कवते॒ प्रती॑कवते॒ ऽग्नये॑ विबा॒धव॑ते विबा॒धव॑ते॒ ऽग्नये॒ प्रती॑कवते । \newline
56. वि॒बा॒धव॑त॒ इति॑ विबा॒ध - व॒ते॒ । \newline
\pagebreak
\markright{ TS 2.4.1.4  \hfill https://www.vedavms.in \hfill}

\section{ TS 2.4.1.4 }

\textbf{TS 2.4.1.4 } \newline
\textbf{Samhita Paata} \newline

ऽग्नये॒ प्रती॑कवते॒ यद॒ग्नये॒ प्रव॑ते नि॒र्वप॑ति॒ य ए॒वास्मा॒च्छ्रेया॒न्-भ्रातृ॑व्य॒स्तं तेन॒ प्रणु॑दते॒ यद॒ग्नये॑ विबा॒धव॑ते॒ य ए॒वैने॑न स॒दृङ्तं तेन॒ वि बा॑धते॒ यद॒ग्नये॒ प्रती॑कवते॒ य ए॒वास्मा॒त् पापी॑या॒न् तं तेनाप॑ नुदते॒ प्र श्रेयाꣳ॑सं॒ भ्रातृ॑व्यं नुद॒तेति॑ स॒दृशं॑ क्रामति॒ नैनं॒ पापी॑यानाप्नोति॒ य ए॒वं ( ) ॅवि॒द्वाने॒तयेष्‌ट्या॒ यज॑ते ॥ \newline

\textbf{Pada Paata} \newline

अ॒ग्नय᳚ । प्रती॑कवत॒ इति॒ प्रती॑क - व॒ते॒ । यत् । अ॒ग्नये᳚ । प्रव॑त॒ इति॒ प्र - व॒ते॒ । नि॒र्वप॒तीति॑ निः - वप॑ति । यः । ए॒व । अ॒स्मा॒त् । श्रेयान्॑ । भ्रातृ॑व्यः । तम् । तेन॑ । प्रेति॑ । नु॒द॒ते॒ । यत् । अ॒ग्नये᳚ । वि॒बा॒धव॑त॒ इति॑ विबा॒ध - व॒ते॒ । यः । ए॒व । ए॒ने॒न॒ । स॒दृङ्ङिति॑ स - दृङ् । तम् । तेन॑ । वीति॑ । बा॒ध॒ते॒ । यत् । अ॒ग्नये᳚ । प्रती॑कवत॒ इति॒ प्रती॑क - व॒ते॒ । यः । ए॒व । अ॒स्मा॒त् । पापी॑यान् । तम् । तेन॑ । अपेति॑ । नु॒द॒ते॒ । प्रेति॑ । श्रेयाꣳ॑सम् । भ्रातृ॑व्यम् । नु॒द॒ते॒ । अतीति॑ । स॒दृश᳚म् । क्रा॒म॒ति॒ । न । ए॒न॒म् । पापी॑यान् । आ॒प्नो॒ति॒ । यः । ए॒वम् ( ) । वि॒द्वान् । ए॒तया᳚ । इष्ट्या᳚ । यज॑ते ॥  \newline


\textbf{Krama Paata} \newline

अ॒ग्नये॒ प्रती॑कवते । प्रती॑कवते॒ यत् । प्रती॑कवत॒ इति॒ प्रती॑क - व॒ते॒ । यद॒ग्नये᳚ । अ॒ग्नये॒ प्रव॑ते । प्रव॑ते नि॒र्वप॑ति । प्रव॑त॒ इति॒ प्र - व॒ते॒ । नि॒र्वप॑ति॒ यः । नि॒र्वप॒तीति॑ निः - वप॑ति । य ए॒व । ए॒वास्मा᳚त् । अ॒स्मा॒च्छ्रेयान्॑ । श्रेया॒न् भ्रातृ॑व्यः । भ्रातृ॑व्य॒स्तम् । तम् तेन॑ । तेन॒ प्र । प्र णु॑दते । नु॒द॒ते॒ यत् । यद॒ग्नये᳚ । अ॒ग्नये॑ विबा॒धव॑ते । वि॒बा॒धव॑ते॒ यः । वि॒बा॒धव॑त॒ इति॑ विबा॒ध - 
व॒ते॒ । य ए॒व । ए॒वैने॑न । ए॒ने॒न॒ स॒दृङ्ङ् । स॒दृङ् तम् । स॒दृङ्ङिति॑ स - दृङ्ङ् । तम् तेन॑ । तेन॒ वि । वि बा॑धते । बा॒ध॒ते॒ यत् । यद॒ग्नये᳚ । अ॒ग्नये॒ प्रती॑कवते । प्रती॑कवते॒ यः । प्रती॑कवत॒ इति॒ प्रती॑क - व॒ते॒ । य ए॒व । ए॒वास्मा᳚त् । अ॒स्मा॒त् पापी॑यान् । पापी॑या॒न् तम् । तम् तेन॑ । तेनाप॑ । अप॑ नुदते । नु॒द॒ते॒ प्र । प्र श्रेयाꣳ॑सम् । श्रेयाꣳ॑स॒म् भ्रातृ॑व्यम् । भ्रातृ॑व्यम् नुदते । नु॒द॒ते ऽति॑ । अति॑ स॒दृश᳚म् । स॒दृश॑म् क्रामति । क्रा॒म॒ति॒ न । नैन᳚म् । ए॒न॒म् पापी॑यान् । पापी॑यानाप्नोति । आ॒प्नो॒ति॒ यः । य ए॒वम् ( ) । ए॒वं ॅवि॒द्वान् । वि॒द्वाने॒तया᳚ । ए॒तयेष्ट्या᳚ । इष्ट्या॒ यज॑ते । यज॑त॒ इति॒ यज॑ते । \newline

\textbf{Jatai Paata} \newline

1. अ॒ग्नये॒ प्रती॑कवते॒ प्रती॑कवते॒ ऽग्नये॒ ऽग्नये॒ प्रती॑कवते । \newline
2. प्रती॑कवते॒ यद् यत् प्रती॑कवते॒ प्रती॑कवते॒ यत् । \newline
3. प्रती॑कवत॒ इति॒ प्रती॑क - व॒ते॒ । \newline
4. यद॒ग्नये॒ ऽग्नये॒ यद् यद॒ग्नये᳚ । \newline
5. अ॒ग्नये॒ प्रव॑ते॒ प्रव॑ते॒ ऽग्नये॒ ऽग्नये॒ प्रव॑ते । \newline
6. प्रव॑ते नि॒र्वप॑ति नि॒र्वप॑ति॒ प्रव॑ते॒ प्रव॑ते नि॒र्वप॑ति । \newline
7. प्रव॑त॒ इति॒ प्र - व॒ते॒ । \newline
8. नि॒र्वप॑ति॒ यो यो नि॒र्वप॑ति नि॒र्वप॑ति॒ यः । \newline
9. नि॒र्वप॒तीति॑ निः - वप॑ति । \newline
10. य ए॒वैव यो य ए॒व । \newline
11. ए॒वास्मा॑ दस्मा दे॒वै वास्मा᳚त् । \newline
12. अ॒स्मा॒च् छ्रेया॒ञ् छ्रेया॑ नस्मा दस्मा॒च् छ्रेयान्॑ । \newline
13. श्रेया॒न् भ्रातृ॑व्यो॒ भ्रातृ॑व्यः॒ श्रेया॒ञ् छ्रेया॒न् भ्रातृ॑व्यः । \newline
14. भ्रातृ॑व्य॒ स्तम् तम् भ्रातृ॑व्यो॒ भ्रातृ॑व्य॒ स्तम् । \newline
15. तम् तेन॒ तेन॒ तम् तम् तेन॑ । \newline
16. तेन॒ प्र प्र तेन॒ तेन॒ प्र । \newline
17. प्र णु॑दते नुदते॒ प्र प्र णु॑दते । \newline
18. नु॒द॒ते॒ यद् यन् नु॑दते नुदते॒ यत् । \newline
19. यद॒ग्नये॒ ऽग्नये॒ यद् यद॒ग्नये᳚ । \newline
20. अ॒ग्नये॑ विबा॒धव॑ते विबा॒धव॑ते॒ ऽग्नये॒ ऽग्नये॑ विबा॒धव॑ते । \newline
21. वि॒बा॒धव॑ते॒ यो यो वि॑बा॒धव॑ते विबा॒धव॑ते॒ यः । \newline
22. वि॒बा॒धव॑त॒ इति॑ विबा॒ध - व॒ते॒ । \newline
23. य ए॒वैव यो य ए॒व । \newline
24. ए॒ वैने॑नैनेनै॒ वैवैने॑न । \newline
25. ए॒ने॒न॒ स॒दृङ् ख्स॒दृङ् ङे॑नेनैनेन स॒दृङ् । \newline
26. स॒दृङ् तम् तꣳ स॒दृङ् ख्स॒दृङ् तम् । \newline
27. स॒दृङ्ङिति॑ स - दृङ् । \newline
28. तम् तेन॒ तेन॒ तम् तम् तेन॑ । \newline
29. तेन॒ वि वि तेन॒ तेन॒ वि । \newline
30. वि बा॑धते बाधते॒ वि वि बा॑धते । \newline
31. बा॒ध॒ते॒ यद् यद् बा॑धते बाधते॒ यत् । \newline
32. यद॒ग्नये॒ ऽग्नये॒ यद् यद॒ग्नये᳚ । \newline
33. अ॒ग्नये॒ प्रती॑कवते॒ प्रती॑कवते॒ ऽग्नये॒ ऽग्नये॒ प्रती॑कवते । \newline
34. प्रती॑कवते॒ यो यः प्रती॑कवते॒ प्रती॑कवते॒ यः । \newline
35. प्रती॑कवत॒ इति॒ प्रती॑क - व॒ते॒ । \newline
36. य ए॒वैव यो य ए॒व । \newline
37. ए॒वास्मा॑ दस्मा दे॒वै वास्मा᳚त् । \newline
38. अ॒स्मा॒त् पापी॑या॒न् पापी॑या नस्मा दस्मा॒त् पापी॑यान् । \newline
39. पापी॑या॒न् तम् तम् पापी॑या॒न् पापी॑या॒न् तम् । \newline
40. तम् तेन॒ तेन॒ तम् तम् तेन॑ । \newline
41. तेनापाप॒ तेन॒ तेनाप॑ । \newline
42. अप॑ नुदते नुद॒ते ऽपाप॑ नुदते । \newline
43. नु॒द॒ते॒ प्र प्र णु॑दते नुदते॒ प्र । \newline
44. प्र श्रेयाꣳ॑सꣳ॒॒ श्रेयाꣳ॑स॒म् प्र प्र श्रेयाꣳ॑सम् । \newline
45. श्रेयाꣳ॑स॒म् भ्रातृ॑व्य॒म् भ्रातृ॑व्यꣳ॒॒ श्रेयाꣳ॑सꣳ॒॒ श्रेयाꣳ॑स॒म् भ्रातृ॑व्यम् । \newline
46. भ्रातृ॑व्यम् नुदते नुदते॒ भ्रातृ॑व्य॒म् भ्रातृ॑व्यम् नुदते । \newline
47. नु॒द॒ते ऽत्यति॑ नुदते नुद॒ते ऽति॑ । \newline
48. अति॑ स॒दृशꣳ॑ स॒दृश॒ मत्यति॑ स॒दृश᳚म् । \newline
49. स॒दृश॑म् क्रामति क्रामति स॒दृशꣳ॑ स॒दृश॑म् क्रामति । \newline
50. क्रा॒म॒ति॒ न न क्रा॑मति क्रामति॒ न । \newline
51. नैन॑ मेन॒म् न नैन᳚म् । \newline
52. ए॒न॒म् पापी॑या॒न् पापी॑या नेन मेन॒म् पापी॑यान् । \newline
53. पापी॑या नाप्नो त्याप्नोति॒ पापी॑या॒न् पापी॑या नाप्नोति । \newline
54. आ॒प्नो॒ति॒ यो य आ᳚प्नो त्याप्नोति॒ यः । \newline
55. य ए॒व मे॒वं ॅयो य ए॒वम् । \newline
56. ए॒वं ॅवि॒द्वान्. वि॒द्वा ने॒व मे॒वं ॅवि॒द्वान् । \newline
57. वि॒द्वा ने॒त यै॒तया॑ वि॒द्वान्. वि॒द्वा ने॒तया᳚ । \newline
58. ए॒तयेष्ट्ये ष्ट्यै॒त यै॒त येष्ट्या᳚ । \newline
59. इष्ट्या॒ यज॑ते॒ यज॑त॒ इष्ट्येष्ट्या॒ यज॑ते । \newline
60. यज॑त॒ इति॒ यज॑ते । \newline

\textbf{Ghana Paata } \newline

1. अ॒ग्नये॒ प्रती॑कवते॒ प्रती॑कवते॒ ऽग्नये॒ ऽग्नये॒ प्रती॑कवते॒ यद् यत् प्रती॑कवते॒ ऽग्नये॒ ऽग्नये॒ प्रती॑कवते॒ यत् । \newline
2. प्रती॑कवते॒ यद् यत् प्रती॑कवते॒ प्रती॑कवते॒ यद॒ग्नये॒ ऽग्नये॒ यत् प्रती॑कवते॒ प्रती॑कवते॒ यद॒ग्नये᳚ । \newline
3. प्रती॑कवत॒ इति॒ प्रती॑क - व॒ते॒ । \newline
4. यद॒ग्नये॒ ऽग्नये॒ यद् यद॒ग्नये॒ प्रव॑ते॒ प्रव॑ते॒ ऽग्नये॒ यद् यद॒ग्नये॒ प्रव॑ते । \newline
5. अ॒ग्नये॒ प्रव॑ते॒ प्रव॑ते॒ ऽग्नये॒ ऽग्नये॒ प्रव॑ते नि॒र्वप॑ति नि॒र्वप॑ति॒ प्रव॑ते॒ ऽग्नये॒ ऽग्नये॒ प्रव॑ते नि॒र्वप॑ति । \newline
6. प्रव॑ते नि॒र्वप॑ति नि॒र्वप॑ति॒ प्रव॑ते॒ प्रव॑ते नि॒र्वप॑ति॒ यो यो नि॒र्वप॑ति॒ प्रव॑ते॒ प्रव॑ते नि॒र्वप॑ति॒ यः । \newline
7. प्रव॑त॒ इति॒ प्र - व॒ते॒ । \newline
8. नि॒र्वप॑ति॒ यो यो नि॒र्वप॑ति नि॒र्वप॑ति॒ य ए॒वैव यो नि॒र्वप॑ति नि॒र्वप॑ति॒ य ए॒व । \newline
9. नि॒र्वप॒तीति॑ निः - वप॑ति । \newline
10. य ए॒वैव यो य ए॒वास्मा॑ दस्मादे॒व यो य ए॒वास्मा᳚त् । \newline
11. ए॒वास्मा॑ दस्मा दे॒वैवास्मा॒च् छ्रेया॒ञ् छ्रेया॑ नस्मा दे॒वैवास्मा॒च् छ्रेयान्॑ । \newline
12. अ॒स्मा॒च् छ्रेया॒ञ् छ्रेया॑ नस्मा दस्मा॒च् छ्रेया॒न् भ्रातृ॑व्यो॒ भ्रातृ॑व्यः॒ श्रेया॑ नस्मा दस्मा॒च् छ्रेया॒न् भ्रातृ॑व्यः । \newline
13. श्रेया॒न् भ्रातृ॑व्यो॒ भ्रातृ॑व्यः॒ श्रेया॒ञ् छ्रेया॒न् भ्रातृ॑व्य॒ स्तम् तम् भ्रातृ॑व्यः॒ श्रेया॒ञ् छ्रेया॒न् भ्रातृ॑व्य॒ स्तम् । \newline
14. भ्रातृ॑व्य॒ स्तम् तम् भ्रातृ॑व्यो॒ भ्रातृ॑व्य॒ स्तम् तेन॒ तेन॒ तम् भ्रातृ॑व्यो॒ भ्रातृ॑व्य॒ स्तम् तेन॑ । \newline
15. तम् तेन॒ तेन॒ तम् तम् तेन॒ प्र प्र तेन॒ तम् तम् तेन॒ प्र । \newline
16. तेन॒ प्र प्र तेन॒ तेन॒ प्र णु॑दते नुदते॒ प्र तेन॒ तेन॒ प्र णु॑दते । \newline
17. प्र णु॑दते नुदते॒ प्र प्र णु॑दते॒ यद् यन् नु॑दते॒ प्र प्र णु॑दते॒ यत् । \newline
18. नु॒द॒ते॒ यद् यन् नु॑दते नुदते॒ यद॒ग्नये॒ ऽग्नये॒ यन् नु॑दते नुदते॒ यद॒ग्नये᳚ । \newline
19. यद॒ग्नये॒ ऽग्नये॒ यद् यद॒ग्नये॑ विबा॒धव॑ते विबा॒धव॑ते॒ ऽग्नये॒ यद् यद॒ग्नये॑ विबा॒धव॑ते । \newline
20. अ॒ग्नये॑ विबा॒धव॑ते विबा॒धव॑ते॒ ऽग्नये॒ ऽग्नये॑ विबा॒धव॑ते॒ यो यो वि॑बा॒धव॑ते॒ ऽग्नये॒ ऽग्नये॑ विबा॒धव॑ते॒ यः । \newline
21. वि॒बा॒धव॑ते॒ यो यो वि॑बा॒धव॑ते विबा॒धव॑ते॒ य ए॒वैव यो वि॑बा॒धव॑ते विबा॒धव॑ते॒ य ए॒व । \newline
22. वि॒बा॒धव॑त॒ इति॑ विबा॒ध - व॒ते॒ । \newline
23. य ए॒वैव यो य ए॒वै ने॑नै नेनै॒व यो य ए॒वैने॑न । \newline
24. ए॒वै ने॑नैने नै॒वैवैने॑न स॒दृङ् ख्स॒दृ ङे॑नेनै॒ वैवैने॑न स॒दृङ् । \newline
25. ए॒ने॒न॒ स॒दृङ् ख्स॒दृङ् ङे॑नेनैनेन स॒दृङ् तम् तꣳ स॒दृङ् ङे॑नेनैनेन स॒दृङ् तम् । \newline
26. स॒दृङ् तम् तꣳ स॒दृङ् ख्स॒दृङ् तम् तेन॒ तेन॒ तꣳ स॒दृङ् ख्स॒दृङ् तम् तेन॑ । \newline
27. स॒दृङ्ङिति॑ स - दृङ् । \newline
28. तम् तेन॒ तेन॒ तम् तम् तेन॒ वि वि तेन॒ तम् तम् तेन॒ वि । \newline
29. तेन॒ वि वि तेन॒ तेन॒ वि बा॑धते बाधते॒ वि तेन॒ तेन॒ वि बा॑धते । \newline
30. वि बा॑धते बाधते॒ वि वि बा॑धते॒ यद् यद् बा॑धते॒ वि वि बा॑धते॒ यत् । \newline
31. बा॒ध॒ते॒ यद् यद् बा॑धते बाधते॒ यद॒ग्नये॒ ऽग्नये॒ यद् बा॑धते बाधते॒ यद॒ग्नये᳚ । \newline
32. यद॒ग्नये॒ ऽग्नये॒ यद् यद॒ग्नये॒ प्रती॑कवते॒ प्रती॑कवते॒ ऽग्नये॒ यद् यद॒ग्नये॒ प्रती॑कवते । \newline
33. अ॒ग्नये॒ प्रती॑कवते॒ प्रती॑कवते॒ ऽग्नये॒ ऽग्नये॒ प्रती॑कवते॒ यो यः प्रती॑कवते॒ ऽग्नये॒ ऽग्नये॒ प्रती॑कवते॒ यः । \newline
34. प्रती॑कवते॒ यो यः प्रती॑कवते॒ प्रती॑कवते॒ य ए॒वैव यः प्रती॑कवते॒ प्रती॑कवते॒ य ए॒व । \newline
35. प्रती॑कवत॒ इति॒ प्रती॑क - व॒ते॒ । \newline
36. य ए॒वैव यो य ए॒वास्मा॑ दस्मा दे॒व यो य ए॒वास्मा᳚त् । \newline
37. ए॒वास्मा॑ दस्मा दे॒वैवास्मा॒त् पापी॑या॒न् पापी॑या नस्मा दे॒वैवास्मा॒त् पापी॑यान् । \newline
38. अ॒स्मा॒त् पापी॑या॒न् पापी॑या नस्मा दस्मा॒त् पापी॑या॒न् तम् तम् पापी॑या नस्मा दस्मा॒त् पापी॑या॒न् तम् । \newline
39. पापी॑या॒न् तम् तम् पापी॑या॒न् पापी॑या॒न् तम् तेन॒ तेन॒ तम् पापी॑या॒न् पापी॑या॒न् तम् तेन॑ । \newline
40. तम् तेन॒ तेन॒ तम् तम् तेनापाप॒ तेन॒ तम् तम् तेनाप॑ । \newline
41. तेनापाप॒ तेन॒ तेनाप॑ नुदते नुद॒ते ऽप॒ तेन॒ तेनाप॑ नुदते । \newline
42. अप॑ नुदते नुद॒ते ऽपाप॑ नुदते॒ प्र प्र णु॑द॒ते ऽपाप॑ नुदते॒ प्र । \newline
43. नु॒द॒ते॒ प्र प्र णु॑दते नुदते॒ प्र श्रेयाꣳ॑सꣳ॒॒ श्रेयाꣳ॑स॒म् प्र णु॑दते नुदते॒ प्र श्रेयाꣳ॑सम् । \newline
44. प्र श्रेयाꣳ॑सꣳ॒॒ श्रेयाꣳ॑स॒म् प्र प्र श्रेयाꣳ॑स॒म् भ्रातृ॑व्य॒म् भ्रातृ॑व्यꣳ॒॒ श्रेयाꣳ॑स॒म् प्र प्र श्रेयाꣳ॑स॒म् भ्रातृ॑व्यम् । \newline
45. श्रेयाꣳ॑स॒म् भ्रातृ॑व्य॒म् भ्रातृ॑व्यꣳ॒॒ श्रेयाꣳ॑सꣳ॒॒ श्रेयाꣳ॑स॒म् भ्रातृ॑व्यम् नुदते नुदते॒ भ्रातृ॑व्यꣳ॒॒ श्रेयाꣳ॑सꣳ॒॒ श्रेयाꣳ॑स॒म् भ्रातृ॑व्यम् नुदते । \newline
46. भ्रातृ॑व्यम् नुदते नुदते॒ भ्रातृ॑व्य॒म् भ्रातृ॑व्यम् नुद॒ते ऽत्यति॑ नुदते॒ भ्रातृ॑व्य॒म् भ्रातृ॑व्यम् नुद॒ते ऽति॑ । \newline
47. नु॒द॒ते ऽत्यति॑ नुदते नुद॒ते ऽति॑ स॒दृशꣳ॑ स॒दृश॒ मति॑ नुदते नुद॒ते ऽति॑ स॒दृश᳚म् । \newline
48. अति॑ स॒दृशꣳ॑ स॒दृश॒ मत्यति॑ स॒दृश॑म् क्रामति क्रामति स॒दृश॒ मत्यति॑ स॒दृश॑म् क्रामति । \newline
49. स॒दृश॑म् क्रामति क्रामति स॒दृशꣳ॑ स॒दृश॑म् क्रामति॒ न न क्रा॑मति स॒दृशꣳ॑ स॒दृश॑म् क्रामति॒ न । \newline
50. क्रा॒म॒ति॒ न न क्रा॑मति क्रामति॒ नैन॑ मेन॒न्न क्रा॑मति क्रामति॒ नैन᳚म् । \newline
51. नैन॑ मेन॒म् न नैन॒म् पापी॑या॒न् पापी॑या नेन॒म् न नैन॒म् पापी॑यान् । \newline
52. ए॒न॒म् पापी॑या॒न् पापी॑या नेन मेन॒म् पापी॑या नाप्नो त्याप्नोति॒ पापी॑या नेन मेन॒म् पापी॑या नाप्नोति । \newline
53. पापी॑या नाप्नो त्याप्नोति॒ पापी॑या॒न् पापी॑या नाप्नोति॒ यो य आ᳚प्नोति॒ पापी॑या॒न् पापी॑या नाप्नोति॒ यः । \newline
54. आ॒प्नो॒ति॒ यो य आ᳚प्नो त्याप्नोति॒ य ए॒व मे॒वं ॅय आ᳚प्नो त्याप्नोति॒ य ए॒वम् । \newline
55. य ए॒व मे॒वं ॅयो य ए॒वं ॅवि॒द्वान्. वि॒द्वा ने॒वं ॅयो य ए॒वं ॅवि॒द्वान् । \newline
56. ए॒वं ॅवि॒द्वान्. वि॒द्वा ने॒व मे॒वं ॅवि॒द्वा ने॒तयै॒तया॑ वि॒द्वा ने॒व मे॒वं ॅवि॒द्वा ने॒तया᳚ । \newline
57. वि॒द्वा ने॒तयै॒तया॑ वि॒द्वान्. वि॒द्वा ने॒तयेष्ट्ये ष्ट्यै॒तया॑ वि॒द्वान्. वि॒द्वा ने॒तयेष्ट्या᳚ । \newline
58. ए॒तयेष्ट्ये ष्ट्यै॒त यै॒त येष्ट्या॒ यज॑ते॒ यज॑त॒ इष्ट्यै॒त यै॒त येष्ट्या॒ यज॑ते । \newline
59. इष्ट्या॒ यज॑ते॒ यज॑त॒ इष्ट्येष्ट्या॒ यज॑ते । \newline
60. यज॑त॒ इति॒ यज॑ते । \newline
\pagebreak
\markright{ TS 2.4.2.1  \hfill https://www.vedavms.in \hfill}

\section{ TS 2.4.2.1 }

\textbf{TS 2.4.2.1 } \newline
\textbf{Samhita Paata} \newline

दे॒वा॒सु॒राः संॅय॑त्ता आस॒न् ते दे॒वा अ॑ब्रुव॒न्॒. यो नो॑ वी॒र्या॑वत्तम॒स्तमनु॑ स॒मार॑भामहा॒ इति॒ त इन्द्र॑मब्रुव॒न् त्वं ॅवै नो॑ वी॒र्या॑वत्तमोऽसि॒ त्वामनु॑ स॒मार॑भामहा॒ इति॒ सो᳚ऽब्रवीत् ति॒स्रो म॑ इ॒मास्त॒नुवो॑ वी॒र्या॑वती॒स्ताः प्री॑णी॒ताथा-सु॑रान॒भि भ॑विष्य॒थेति॒ ता वै ब्रू॒हीत्य॑ब्रुवन्नि॒यमꣳ॑ हो॒मुगि॒यं ॅवि॑मृ॒धेयमि॑न्द्रि॒याव॒ती - [  ] \newline

\textbf{Pada Paata} \newline

दे॒वा॒सु॒रा इति॑ देव - अ॒सु॒राः । संॅय॑त्ता॒ इति॒ सं - य॒त्ताः॒ । आ॒स॒न्न् । ते । दे॒वाः । अ॒ब्रु॒व॒न्न् । यः । नः॒ । वी॒र्या॑वत्तम॒ इति॑ वी॒र्या॑वत् - त॒मः॒ । तम् । अन्विति॑ । स॒मार॑भामहा॒ इति॑ सं - आर॑भामहै । इति॑ । ते । इन्द्र᳚म् । अ॒ब्रु॒व॒न्न् । त्वम् । वै । नः॒ । वी॒र्या॑वत्तम॒ इति॑ वी॒र्या॑वत् - त॒मः॒ । अ॒सि॒ । त्वाम् । अन्विति॑ । स॒मार॑भामहा॒ इति॑ सं-आर॑भामहै । इति॑ । सः । अ॒ब्र॒वी॒त् । ति॒स्रः । मे॒ । इ॒माः । त॒नुवः॑ । वी॒र्या॑वती॒रिति॑ वी॒र्य॑ - व॒तीः॒ । ताः । प्री॒णी॒त॒ । अथ॑ । असु॑रान् । अ॒भीति॑ । भ॒वि॒ष्य॒थ॒ । इति॑ । ताः । वै । ब्रू॒हि॒ । इति॑ । अ॒ब्रु॒व॒न्न् । इ॒यम् । अꣳ॒॒हो॒मुगित्यꣳ॑हः -मुक् । इ॒यम् । वि॒मृ॒धेति॑ वि - मृ॒धा । इ॒यम् । इ॒न्द्रि॒याव॒तीती᳚न्द्रि॒य - व॒ती॒ ।  \newline


\textbf{Krama Paata} \newline

दे॒वा॒सु॒राः सम्ॅय॑त्ताः । दे॒वा॒सु॒रा इति॑ देव - अ॒सु॒राः । सम्ॅय॑त्ता आसन्न् । सम्ॅय॑त्ता॒ इति॒ सं - य॒त्ताः॒ । आ॒स॒न् ते । 
ते दे॒वाः । दे॒वा अ॑ब्रुवन्न् । अ॒बु॒॒वन्॒. यः । यो नः॑ । नो॒ वी॒र्या॑वत्तमः । वी॒र्या॑वत्तम॒स्तम् । वी॒र्या॑वत्तम॒ इति॑ वी॒र्या॑वत् - त॒मः॒ । तमनु॑ । अनु॑ स॒मार॑भामहै । स॒मार॑भामहा॒ इति॑ । स॒मार॑भामहा॒ इति॑ सम् - आर॑भामहै । इति॒ ते । त इन्द्र᳚म् । इन्द्र॑मब्रुवन्न् । अ॒ब्रु॒व॒न् त्वम् । त्वं ॅवै । वै नः॑ । नो॒ वी॒र्या॑वत्तमः । वी॒र्या॑वत्तमोऽसि । वी॒र्या॑वत्तम॒ इति॑ वी॒र्या॑वत् - त॒मः॒ । अ॒सि॒ त्वाम् । त्वामनु॑ । अनु॑ स॒मार॑भामहै । स॒मार॑भामहा॒ इति॑ । स॒मार॑भामहा॒ इति॑ सम् - आर॑भामहै । इति॒ सः । सो᳚ ऽब्रवीत् । अ॒ब्र॒वी॒त् ति॒स्रः । ति॒स्रो मे᳚ । म॒ इ॒माः । इ॒मा स्त॒नुवः॑ । त॒नुवो॑ वी॒र्या॑वतीः । वी॒र्या॑वती॒स्ताः । वी॒र्या॑वती॒रिति॑ वी॒र्य॑ - व॒तीः॒ । ताः प्री॑णीत । प्री॒णी॒ताथ॑ । अथासु॑रान् । असु॑रान॒भि । अ॒भि भ॑विष्यथ । भ॒वि॒ष्य॒थेति॑ । इति॒ ताः । ता वै । वै ब्रू॑हि । ब्रू॒हीति॑ । इत्य॑ब्रुवन्न् । अ॒ब्रु॒व॒न्नि॒यम् । इ॒यमꣳ॑हो॒मुक् । अꣳ॒॒हो॒मुगि॒यम् । अꣳ॒॒हो॒मुगित्यꣳ॑हः - मुक् । इ॒यं ॅवि॑मृ॒धा । वि॒मृ॒धेयम् । वि॒मृ॒धेति॑ वि - मृ॒धा । इ॒यमि॑न्द्रि॒याव॑ती । इ॒न्द्रि॒याव॒तीति॑ । इ॒न्द्रि॒याव॒तीती᳚न्द्रि॒य - व॒ती॒ \newline

\textbf{Jatai Paata} \newline

1. दे॒वा॒सु॒राः संॅय॑त्ताः॒ संॅय॑त्ता देवासु॒रा दे॑वासु॒राः संॅय॑त्ताः । \newline
2. दे॒वा॒सु॒रा इति॑ देव - अ॒सु॒राः । \newline
3. संॅय॑त्ता आसन् नास॒न् थ्संॅय॑त्ताः॒ संॅय॑त्ता आसन्न् । \newline
4. संॅय॑त्ता॒ इति॒ सं - य॒त्ताः॒ । \newline
5. आ॒स॒न् ते त आ॑सन् नास॒न् ते । \newline
6. ते दे॒वा दे॒वा स्ते ते दे॒वाः । \newline
7. दे॒वा अ॑ब्रुवन् नब्रुवन् दे॒वा दे॒वा अ॑ब्रुवन्न् । \newline
8. अ॒ब्रु॒व॒न्॒. यो यो᳚ ऽब्रुवन् नब्रुव॒न्॒. यः । \newline
9. यो नो॑ नो॒ यो यो नः॑ । \newline
10. नो॒ वी॒र्या॑वत्तमो वी॒र्या॑वत्तमो नो नो वी॒र्या॑वत्तमः । \newline
11. वी॒र्या॑वत्तम॒ स्तम् तं ॅवी॒र्या॑वत्तमो वी॒र्या॑वत्तम॒ स्तम् । \newline
12. वी॒र्या॑वत्तम॒ इति॑ वी॒र्या॑वत् - त॒मः॒ । \newline
13. त मन्वनु॒ तम् त मनु॑ । \newline
14. अनु॑ स॒मार॑भामहै स॒मार॑भामहा॒ अन्वनु॑ स॒मार॑भामहै । \newline
15. स॒मार॑भामहा॒ इतीति॑ स॒मार॑भामहै स॒मार॑भामहा॒ इति॑ । \newline
16. स॒मार॑भामहा॒ इति॑ सं - आर॑भामहै । \newline
17. इति॒ ते त इतीति॒ ते । \newline
18. त इन्द्र॒ मिन्द्र॒म् ते त इन्द्र᳚म् । \newline
19. इन्द्र॑ मब्रुवन् नब्रुव॒न् निन्द्र॒ मिन्द्र॑ मब्रुवन्न् । \newline
20. अ॒ब्रु॒व॒न् त्वम् त्व म॑ब्रुवन् नब्रुव॒न् त्वम् । \newline
21. त्वं ॅवै वै त्वम् त्वं ॅवै । \newline
22. वै नो॑ नो॒ वै वै नः॑ । \newline
23. नो॒ वी॒र्या॑वत्तमो वी॒र्या॑वत्तमो नो नो वी॒र्या॑वत्तमः । \newline
24. वी॒र्या॑वत्तमो ऽस्यसि वी॒र्या॑वत्तमो वी॒र्या॑वत्तमो ऽसि । \newline
25. वी॒र्या॑वत्तम॒ इति॑ वी॒र्या॑वत् - त॒मः॒ । \newline
26. अ॒सि॒ त्वाम् त्वा म॑स्यसि॒ त्वाम् । \newline
27. त्वा मन्वनु॒ त्वाम् त्वा मनु॑ । \newline
28. अनु॑ स॒मार॑भामहै स॒मार॑भामहा॒ अन्वनु॑ स॒मार॑भामहै । \newline
29. स॒मार॑भामहा॒ इतीति॑ स॒मार॑भामहै स॒मार॑भामहा॒ इति॑ । \newline
30. स॒मार॑भामहा॒ इति॑ सं - आर॑भामहै । \newline
31. इति॒ स स इतीति॒ सः । \newline
32. सो᳚ ऽब्रवी दब्रवी॒थ् स सो᳚ ऽब्रवीत् । \newline
33. अ॒ब्र॒वी॒त् ति॒स्र स्ति॒स्रो᳚ ऽब्रवी दब्रवीत् ति॒स्रः । \newline
34. ति॒स्रो मे॑ मे ति॒स्र स्ति॒स्रो मे᳚ । \newline
35. म॒ इ॒मा इ॒मा मे॑ म इ॒माः । \newline
36. इ॒मा स्त॒नुव॑ स्त॒नुव॑ इ॒मा इ॒मा स्त॒नुवः॑ । \newline
37. त॒नुवो॑ वी॒र्या॑वतीर् वी॒र्या॑वती स्त॒नुव॑ स्त॒नुवो॑ वी॒र्या॑वतीः । \newline
38. वी॒र्या॑वती॒स्ता स्ता वी॒र्या॑वतीर् वी॒र्या॑वती॒ स्ताः । \newline
39. वी॒र्या॑वती॒रिति॑ वी॒र्य॑ - व॒तीः॒ । \newline
40. ताः प्री॑णीत प्रीणीत॒ ता स्ताः प्री॑णीत । \newline
41. प्री॒णी॒ताथाथ॑ प्रीणीत प्रीणी॒ताथ॑ । \newline
42. अथासु॑रा॒ नसु॑रा॒ नथाथा सु॑रान् । \newline
43. असु॑रा न॒भ्य॑भ्यसु॑रा॒ नसु॑रा न॒भि । \newline
44. अ॒भि भ॑विष्यथ भविष्यथा॒ भ्य॑भि भ॑विष्यथ । \newline
45. भ॒वि॒ष्य॒थे तीति॑ भविष्यथ भविष्य॒थे ति॑ । \newline
46. इति॒ ता स्ता इतीति॒ ताः । \newline
47. ता वै वै ता स्ता वै । \newline
48. वै ब्रू॑हि ब्रूहि॒ वै वै ब्रू॑हि । \newline
49. ब्रू॒हीतीति॑ ब्रूहि ब्रू॒हीति॑ । \newline
50. इत्य॑ ब्रुवन् नब्रुव॒न् निती त्य॑ब्रुवन्न् । \newline
51. अ॒ब्रु॒व॒न् नि॒य मि॒य म॑ब्रुवन् नब्रुवन् नि॒यम् । \newline
52. इ॒य मꣳ॑हो॒मु गꣳ॑हो॒मु गि॒य मि॒य मꣳ॑हो॒मुक् । \newline
53. अꣳ॒॒हो॒मु गि॒य मि॒य मꣳ॑हो॒मु गꣳ॑हो॒मु गि॒यम् । \newline
54. अꣳ॒॒हो॒मुगित्यꣳ॑हः - मुक् । \newline
55. इ॒यं ॅवि॑मृ॒धा वि॑मृ॒धेय मि॒यं ॅवि॑मृ॒धा । \newline
56. वि॒मृ॒धेय मि॒यं ॅवि॑मृ॒धा वि॑मृ॒धेयम् । \newline
57. वि॒मृ॒धेति॑ वि - मृ॒धा । \newline
58. इ॒य मि॑न्द्रि॒याव॑ती न्द्रि॒याव॑ती॒य मि॒य मि॑न्द्रि॒याव॑ती । \newline
59. इ॒न्द्रि॒याव॒ती तीती᳚न्द्रि॒याव॑ती न्द्रि॒याव॒तीति॑ । \newline
60. इ॒न्द्रि॒याव॒तीती᳚न्द्रि॒य - व॒ती॒ । \newline

\textbf{Ghana Paata } \newline

1. दे॒वा॒सु॒राः संॅय॑त्ताः॒ संॅय॑त्ता देवासु॒रा दे॑वासु॒राः संॅय॑त्ता आसन् नास॒न् थ्संॅय॑त्ता देवासु॒रा दे॑वासु॒राः संॅय॑त्ता आसन्न् । \newline
2. दे॒वा॒सु॒रा इति॑ देव - अ॒सु॒राः । \newline
3. संॅय॑त्ता आसन् नास॒न् थ्संॅय॑त्ताः॒ संॅय॑त्ता आस॒न् ते त आ॑स॒न् थ्संॅय॑त्ताः॒ संॅय॑त्ता आस॒न् ते । \newline
4. संॅय॑त्ता॒ इति॒ सं - य॒त्ताः॒ । \newline
5. आ॒स॒न् ते त आ॑सन् नास॒न् ते दे॒वा दे॒वा स्त आ॑सन् नास॒न् ते दे॒वाः । \newline
6. ते दे॒वा दे॒वा स्ते ते दे॒वा अ॑ब्रुवन् नब्रुवन् दे॒वा स्ते ते दे॒वा अ॑ब्रुवन्न् । \newline
7. दे॒वा अ॑ब्रुवन् नब्रुवन् दे॒वा दे॒वा अ॑ब्रुव॒न्॒. यो यो᳚ ऽब्रुवन् दे॒वा दे॒वा अ॑ब्रुव॒न्॒. यः । \newline
8. अ॒ब्रु॒व॒न्॒. यो यो᳚ ऽब्रुवन् नब्रुव॒न्॒. यो नो॑ नो॒ यो᳚ ऽब्रुवन् नब्रुव॒न्॒. यो नः॑ । \newline
9. यो नो॑ नो॒ यो यो नो॑ वी॒र्या॑वत्तमो वी॒र्या॑वत्तमो नो॒ यो यो नो॑ वी॒र्या॑वत्तमः । \newline
10. नो॒ वी॒र्या॑वत्तमो वी॒र्या॑वत्तमो नो नो वी॒र्या॑वत्तम॒ स्तम् तं ॅवी॒र्या॑वत्तमो नो नो वी॒र्या॑वत्तम॒ स्तम् । \newline
11. वी॒र्या॑वत्तम॒ स्तम् तं ॅवी॒र्या॑वत्तमो वी॒र्या॑वत्तम॒ स्त मन्वनु॒ तं ॅवी॒र्या॑वत्तमो वी॒र्या॑वत्तम॒ स्त मनु॑ । \newline
12. वी॒र्या॑वत्तम॒ इति॑ वी॒र्या॑वत् - त॒मः॒ । \newline
13. त मन्वनु॒ तम् त मनु॑ स॒मार॑भामहै स॒मार॑भामहा॒ अनु॒ तम् त मनु॑ स॒मार॑भामहै । \newline
14. अनु॑ स॒मार॑भामहै स॒मार॑भामहा॒ अन्वनु॑ स॒मार॑भामहा॒ इतीति॑ स॒मार॑भामहा॒ अन्वनु॑ स॒मार॑भामहा॒ इति॑ । \newline
15. स॒मार॑भामहा॒ इतीति॑ स॒मार॑भामहै स॒मार॑भामहा॒ इति॒ ते त इति॑ स॒मार॑भामहै स॒मार॑भामहा॒ इति॒ ते । \newline
16. स॒मार॑भामहा॒ इति॑ सं - आर॑भामहै । \newline
17. इति॒ ते त इतीति॒ त इन्द्र॒ मिन्द्र॒म् त इतीति॒ त इन्द्र᳚म् । \newline
18. त इन्द्र॒ मिन्द्र॒म् ते त इन्द्र॑ मब्रुवन् नब्रुव॒न् निन्द्र॒म् ते त इन्द्र॑ मब्रुवन्न् । \newline
19. इन्द्र॑ मब्रुवन् नब्रुव॒न् निन्द्र॒ मिन्द्र॑ मब्रुव॒न् त्वम् त्व म॑ब्रुव॒न् निन्द्र॒ मिन्द्र॑ मब्रुव॒न् त्वम् । \newline
20. अ॒ब्रु॒व॒न् त्वम् त्व म॑ब्रुवन् नब्रुव॒न् त्वं ॅवै वै त्व म॑ब्रुवन् नब्रुव॒न् त्वं ॅवै । \newline
21. त्वं ॅवै वै त्वम् त्वं ॅवै नो॑ नो॒ वै त्वम् त्वं ॅवै नः॑ । \newline
22. वै नो॑ नो॒ वै वै नो॑ वी॒र्या॑वत्तमो वी॒र्या॑वत्तमो नो॒ वै वै नो॑ वी॒र्या॑वत्तमः । \newline
23. नो॒ वी॒र्या॑वत्तमो वी॒र्या॑वत्तमो नो नो वी॒र्या॑वत्तमो ऽस्यसि वी॒र्या॑वत्तमो नो नो वी॒र्या॑वत्तमो ऽसि । \newline
24. वी॒र्या॑वत्तमो ऽस्यसि वी॒र्या॑वत्तमो वी॒र्या॑वत्तमो ऽसि॒ त्वाम् त्वा म॑सि वी॒र्या॑वत्तमो वी॒र्या॑वत्तमो ऽसि॒ त्वाम् । \newline
25. वी॒र्या॑वत्तम॒ इति॑ वी॒र्या॑वत् - त॒मः॒ । \newline
26. अ॒सि॒ त्वाम् त्वा म॑स्यसि॒ त्वा मन्वनु॒ त्वा म॑स्यसि॒ त्वा मनु॑ । \newline
27. त्वा मन्वनु॒ त्वाम् त्वा मनु॑ स॒मार॑भामहै स॒मार॑भामहा॒ अनु॒ त्वाम् त्वा मनु॑ स॒मार॑भामहै । \newline
28. अनु॑ स॒मार॑भामहै स॒मार॑भामहा॒ अन्वनु॑ स॒मार॑भामहा॒ इतीति॑ स॒मार॑भामहा॒ अन्वनु॑ स॒मार॑भामहा॒ इति॑ । \newline
29. स॒मार॑भामहा॒ इतीति॑ स॒मार॑भामहै स॒मार॑भामहा॒ इति॒ स स इति॑ स॒मार॑भामहै स॒मार॑भामहा॒ इति॒ सः । \newline
30. स॒मार॑भामहा॒ इति॑ सं - आर॑भामहै । \newline
31. इति॒ स स इतीति॒ सो᳚ ऽब्रवी दब्रवी॒थ् स इतीति॒ सो᳚ ऽब्रवीत् । \newline
32. सो᳚ ऽब्रवी दब्रवी॒थ् स सो᳚ ऽब्रवीत् ति॒स्र स्ति॒स्रो᳚ ऽब्रवी॒थ् स सो᳚ ऽब्रवीत् ति॒स्रः । \newline
33. अ॒ब्र॒वी॒त् ति॒स्र स्ति॒स्रो᳚ ऽब्रवी दब्रवीत् ति॒स्रो मे॑ मे ति॒स्रो᳚ ऽब्रवी दब्रवीत् ति॒स्रो मे᳚ । \newline
34. ति॒स्रो मे॑ मे ति॒स्र स्ति॒स्रो म॑ इ॒मा इ॒मा मे॑ ति॒स्र स्ति॒स्रो म॑ इ॒माः । \newline
35. म॒ इ॒मा इ॒मा मे॑ म इ॒मा स्त॒नुव॑ स्त॒नुव॑ इ॒मा मे॑ म इ॒मा स्त॒नुवः॑ । \newline
36. इ॒मा स्त॒नुव॑ स्त॒नुव॑ इ॒मा इ॒मा स्त॒नुवो॑ वी॒र्या॑वतीर् वी॒र्या॑वती स्त॒नुव॑ इ॒मा इ॒मा स्त॒नुवो॑ वी॒र्या॑वतीः । \newline
37. त॒नुवो॑ वी॒र्या॑वतीर् वी॒र्या॑वती स्त॒नुव॑ स्त॒नुवो॑ वी॒र्या॑वती॒ स्ता स्ता वी॒र्या॑वती स्त॒नुव॑ स्त॒नुवो॑ वी॒र्या॑वती॒ स्ताः । \newline
38. वी॒र्या॑वती॒ स्ता स्ता वी॒र्या॑वतीर् वी॒र्या॑वती॒ स्ताः प्री॑णीत प्रीणीत॒ ता वी॒र्या॑वतीर् वी॒र्या॑वती॒ स्ताः प्री॑णीत । \newline
39. वी॒र्या॑वती॒रिति॑ वी॒र्य॑ - व॒तीः॒ । \newline
40. ताः प्री॑णीत प्रीणीत॒ ता स्ताः प्री॑णी॒ताथाथ॑ प्रीणीत॒ ता स्ताः प्री॑णी॒ताथ॑ । \newline
41. प्री॒णी॒ताथाथ॑ प्रीणीत प्रीणी॒ताथासु॑रा॒ नसु॑रा॒ नथ॑ प्रीणीत प्रीणी॒ताथासु॑रान् । \newline
42. अथासु॑रा॒ नसु॑रा॒ नथाथासु॑रा न॒भ्य॑भ्यसु॑रा॒ नथाथासु॑रा न॒भि । \newline
43. असु॑रा न॒भ्य॑भ्यसु॑रा॒ नसु॑रा न॒भि भ॑विष्यथ भविष्यथा॒ भ्यसु॑रा॒ नसु॑रा न॒भि भ॑विष्यथ । \newline
44. अ॒भि भ॑विष्यथ भविष्यथा॒ भ्य॑भि भ॑विष्य॒थे तीति॑ भविष्यथा॒ भ्य॑भि भ॑विष्य॒थे ति॑ । \newline
45. भ॒वि॒ष्य॒थे तीति॑ भविष्यथ भविष्य॒थे ति॒ ता स्ता इति॑ भविष्यथ भविष्य॒थे ति॒ ताः । \newline
46. इति॒ ता स्ता इतीति॒ ता वै वै ता इतीति॒ ता वै । \newline
47. ता वै वै ता स्ता वै ब्रू॑हि ब्रूहि॒ वै ता स्ता वै ब्रू॑हि । \newline
48. वै ब्रू॑हि ब्रूहि॒ वै वै ब्रू॒हीतीति॑ ब्रूहि॒ वै वै ब्रू॒हीति॑ । \newline
49. ब्रू॒हीतीति॑ ब्रूहि ब्रू॒ही त्य॑ब्रुवन् नब्रुव॒न् निति॑ ब्रूहि ब्रू॒ही त्य॑ब्रुवन्न् । \newline
50. इत्य॑ब्रुवन् नब्रुव॒न् निती त्य॑ब्रुवन् नि॒य मि॒य म॑ब्रुव॒न् निती त्य॑ब्रुवन् नि॒यम् । \newline
51. अ॒ब्रु॒व॒न् नि॒य मि॒य म॑ब्रुवन् नब्रुवन् नि॒य मꣳ॑हो॒मु गꣳ॑हो॒मुगि॒य म॑ब्रुवन् नब्रुवन् नि॒य मꣳ॑हो॒मुक् । \newline
52. इ॒य मꣳ॑हो॒मु गꣳ॑हो॒मु गि॒य मि॒य मꣳ॑हो॒मु गि॒य मि॒य मꣳ॑हो॒मु गि॒य मि॒य मꣳ॑हो॒मुगि॒यम् । \newline
53. अꣳ॒॒हो॒मुगि॒य मि॒य मꣳ॑हो॒मु गꣳ॑हो॒मुगि॒यं ॅवि॑मृ॒धा वि॑मृ॒धेय मꣳ॑हो॒मु गꣳ॑हो॒मुगि॒यं ॅवि॑मृ॒धा । \newline
54. अꣳ॒॒हो॒मुगित्यꣳ॑हः - मुक् । \newline
55. इ॒यं ॅवि॑मृ॒धा वि॑मृ॒धेय मि॒यं ॅवि॑मृ॒धेय मि॒यं ॅवि॑मृ॒धेय मि॒यं ॅवि॑मृ॒धेयम् । \newline
56. वि॒मृ॒धेय मि॒यं ॅवि॑मृ॒धा वि॑मृ॒धेय मि॑न्द्रि॒याव॑ती न्द्रि॒याव॑ती॒यं ॅवि॑मृ॒धा वि॑मृ॒धेय मि॑न्द्रि॒याव॑ती । \newline
57. वि॒मृ॒धेति॑ वि - मृ॒धा । \newline
58. इ॒य मि॑न्द्रि॒याव॑ती न्द्रि॒याव॑ती॒य मि॒य मि॑न्द्रि॒याव॒तीती ती᳚न्द्रि॒याव॑ती॒य मि॒य मि॑न्द्रि॒याव॒तीति॑ । \newline
59. इ॒न्द्रि॒याव॒ती तीती᳚न्द्रि॒याव॑ती न्द्रि॒याव॒ती त्य॑ब्रवी दब्रवी॒दिती᳚ न्द्रि॒याव॑ती न्द्रि॒याव॒ती त्य॑ब्रवीत् । \newline
60. इ॒न्द्रि॒याव॒तीती᳚न्द्रि॒य - व॒ती॒ । \newline
\pagebreak
\markright{ TS 2.4.2.2  \hfill https://www.vedavms.in \hfill}

\section{ TS 2.4.2.2 }

\textbf{TS 2.4.2.2 } \newline
\textbf{Samhita Paata} \newline

-त्य॑ब्रवी॒त्त इन्द्रा॑याꣳ हो॒मुचे॑ पुरो॒डाश॒मेका॑दशकपालं॒ निर॑वप॒न्निन्द्रा॑य वैमृ॒धाये-न्द्रा॑येन्द्रि॒याव॑ते॒ यदिन्द्रा॑याꣳ हो॒मुचे॑ नि॒रव॑प॒न्नꣳह॑स ए॒व तेना॑मुच्यन्त॒ यदिन्द्रा॑य वै मृ॒धाय॒ मृध॑ ए॒व तेनापा᳚घ्नत॒यदिन्द्रा॑येन्द्रि॒याव॑त इन्द्रि॒यमे॒व तेना॒*ऽऽत्मन्न॑दधत॒ त्रय॑स्त्रिꣳशत्कपालं पुरो॒डाशं॒ निर॑वप॒न् त्रय॑स्त्रिꣳश॒द्वै दे॒वता॒स्ता इन्द्र॑ आ॒त्मन्ननु॑ स॒मार॑भंयत॒ भूत्यै॒ - [  ] \newline

\textbf{Pada Paata} \newline

इति॑ । अ॒ब्र॒वी॒त् । ते । इन्द्रा॑य । अꣳ॒॒हो॒मुच॒ इत्यꣳ॑हः - मुचे᳚ । पु॒रो॒डाश᳚म् । एका॑दशकपाल॒मित्येका॑दश - क॒पा॒ल॒म् । निरिति॑ । अ॒व॒प॒न्न् । इन्द्रा॑य । वै॒मृ॒धाय॑ । इन्द्रा॑य । इ॒न्द्रि॒याव॑त॒ इती᳚न्द्रि॒य - व॒ते॒ । यत् । इन्द्रा॑य । अꣳ॒॒हो॒मुच॒ इत्यꣳ॑हः - मुचे᳚ । नि॒रव॑प॒न्निति॑ निः - अव॑पन्न् । अꣳह॑सः । ए॒व । तेन॑ । अ॒मु॒च्य॒न्त॒ । यत् । इन्द्रा॑य । वै॒मृ॒धाय॑ । मृधः॑ । ए॒व । तेन॑ । अपेति॑ । अ॒घ्न॒त॒ । यत् । इन्द्रा॑य । इ॒न्द्रि॒याव॑त॒ इती᳚न्द्रि॒य - व॒ते॒ । इ॒न्द्रि॒यम् । ए॒व । तेन॑ । आ॒त्मन्न् । अ॒द॒ध॒त॒ । त्रय॑स्त्रिꣳशत्कपाल॒मिति॒ त्रय॑स्त्रिꣳशत् - क॒पा॒ल॒म् । पु॒रो॒डाश᳚म् । निरिति॑ । अ॒व॒प॒न्न् । त्रय॑स्त्रिꣳश॒दिति॒ त्रयः॑ - त्रिꣳ॒॒श॒त् । वै । दे॒वताः᳚ । ताः । इन्द्रः॑ । आ॒त्मन्न् । अन्विति॑ । स॒मार॑भंय॒तेति॑ सं - आर॑भंयत । भूत्यै᳚ ।  \newline


\textbf{Krama Paata} \newline

इत्य॑ब्रवीत् । अ॒ब्र॒वी॒त् ते । त इन्द्रा॑य । इन्द्रा॑याꣳहो॒मुचे᳚ । अꣳ॒॒हो॒मुचे॑ पुरो॒डाश᳚म् । अꣳ॒॒हो॒मुच॒ इत्यꣳ॑हः - मुचे᳚ । पु॒रो॒डाश॒मेका॑दशकपालम् । एका॑दशकपाल॒म् निः । एका॑दशकपाल॒मित्येका॑दश - क॒पा॒ल॒म् । निर॑वपन्न् । अ॒व॒प॒न्निन्द्रा॑य । इन्द्रा॑य वैमृ॒धाय॑ । वै॒मृ॒धायेन्द्रा॑य । इन्द्रा॑येन्द्रि॒याव॑ते । इ॒न्द्रि॒याव॑ते॒ यत् । इ॒न्द्रि॒याव॑त॒ इती᳚न्द्रि॒य - व॒ते॒ । यदिन्द्रा॑य । इन्द्रा॑याꣳहो॒मुचे᳚ । अꣳ॒॒हो॒मुचे॑ नि॒रव॑पन्न् । अꣳ॒॒हो॒मुच॒ इत्यꣳ॑हः - मुचे᳚ । नि॒रव॑प॒न्नꣳह॑सः । नि॒रव॑प॒न्निति॑ निः - अव॑पन्न् । अꣳह॑स ए॒व । ए॒व तेन॑ । तेना॑मुच्यन्त । अ॒मु॒च्य॒न्त॒ यत् । यदिन्द्रा॑य । इन्द्रा॑य वैमृ॒धाय॑ । वै॒मृ॒धाय॒ मृधः॑ । मृध॑ ए॒व । ए॒व तेन॑ । तेनाप॑ । अपा᳚घ्नत । अ॒घ्न॒त॒ यत् । यदिन्द्रा॑य । इन्द्रा॑येन्द्रि॒याव॑ते । इ॒न्द्रि॒याव॑त इन्द्रि॒यम् । इ॒न्द्रि॒याव॑त॒ इती᳚न्द्रि॒य - व॒ते॒ । इ॒न्द्रि॒यमे॒व । ए॒व तेन॑ । तेना॒त्मन्न् । आ॒त्मन्न॑दधत । अ॒द॒ध॒त॒ त्रय॑स्त्रिꣳशत्कपालम् । त्रय॑स्त्रिꣳशत्कपालम् पुरो॒डाश᳚म् । त्रय॑स्त्रिꣳशत्कपाल॒मिति॒ त्रय॑स्त्रिꣳशत् - क॒पा॒ल॒म् । पु॒रो॒डाश॒म् निः । निर॑वपन्न् । अ॒व॒प॒न् त्रय॑स्त्रिꣳशत् । त्रय॑स्त्रिꣳश॒द् वै । त्रय॑स्त्रिꣳश॒दिति॒ त्रयः॑ - त्रिꣳ॒॒श॒त्॒ । वै दे॒वताः᳚ । दे॒वता॒स्ताः । ता इन्द्रः॑ । इन्द्र॑ आ॒त्मन्न् । आ॒त्मन्ननु॑ । अनु॑ स॒मार॑म्भयत । स॒मार॑म्भयत॒ भूत्यै᳚ । स॒मार॑म्भय॒तेति॑ सं - आर॑म्भयत । भूत्यै॒ ताम् \newline

\textbf{Jatai Paata} \newline

1. इत्य॑ब्रवी दब्रवी॒ दिती त्य॑ब्रवीत् । \newline
2. अ॒ब्र॒वी॒त् ते ते᳚ ऽब्रवी दब्रवी॒त् ते । \newline
3. त इन्द्रा॒ये न्द्रा॑य॒ ते त इन्द्रा॑य । \newline
4. इन्द्रा॑या ꣳहो॒मुचे॑ ऽꣳहो॒मुच॒ इन्द्रा॒ये न्द्रा॑या ꣳहो॒मुचे᳚ । \newline
5. अꣳ॒॒हो॒मुचे॑ पुरो॒डाश॑म् पुरो॒डाश॑ मꣳहो॒मुचे॑ ऽꣳहो॒मुचे॑ पुरो॒डाश᳚म् । \newline
6. अꣳ॒॒हो॒मुच॒ इत्यꣳ॑हः - मुचे᳚ । \newline
7. पु॒रो॒डाश॒ मेका॑दशकपाल॒ मेका॑दशकपालम् पुरो॒डाश॑म् पुरो॒डाश॒ मेका॑दशकपालम् । \newline
8. एका॑दशकपाल॒म् निर् णिरेका॑दशकपाल॒ मेका॑दशकपाल॒म् निः । \newline
9. एका॑दशकपाल॒मित्येका॑दश - क॒पा॒ल॒म् । \newline
10. निर॑वपन् नवप॒न् निर् णिर॑वपन्न् । \newline
11. अ॒व॒प॒न् निन्द्रा॒ये न्द्रा॑या वपन् नवप॒न् निन्द्रा॑य । \newline
12. इन्द्रा॑य वैमृ॒धाय॑ वैमृ॒धाये न्द्रा॒ये न्द्रा॑य वैमृ॒धाय॑ । \newline
13. वै॒मृ॒धाये न्द्रा॒ये न्द्रा॑य वैमृ॒धाय॑ वैमृ॒धाये न्द्रा॑य । \newline
14. इन्द्रा॑ये न्द्रि॒याव॑त इन्द्रि॒याव॑त॒ इन्द्रा॒ये न्द्रा॑ये न्द्रि॒याव॑ते । \newline
15. इ॒न्द्रि॒याव॑ते॒ यद् यदि॑न्द्रि॒याव॑त इन्द्रि॒याव॑ते॒ यत् । \newline
16. इ॒न्द्रि॒याव॑त॒ इती᳚न्द्रि॒य - व॒ते॒ । \newline
17. यदिन्द्रा॒ये न्द्रा॑य॒ यद् यदिन्द्रा॑य । \newline
18. इन्द्रा॑या ꣳहो॒मुचे ऽꣳ॑हो॒मुच॒ इन्द्रा॒ये न्द्रा॑या ꣳहो॒मुचे᳚ । \newline
19. अꣳ॒॒हो॒मुचे॑ नि॒रव॑पन् नि॒रव॑पन् नꣳहो॒मुचे ऽꣳ॑हो॒मुचे॑ नि॒रव॑पन्न् । \newline
20. अꣳ॒॒हो॒मुच॒ इत्यꣳ॑हः - मुचे᳚ । \newline
21. नि॒रव॑प॒न् नꣳह॒सो ऽꣳह॑सो नि॒रव॑पन् नि॒रव॑प॒न् नꣳह॑सः । \newline
22. नि॒रव॑प॒न्निति॑ निः - अव॑पन्न् । \newline
23. अꣳह॑स ए॒वैवा ꣳह॒सो ऽꣳह॑स ए॒व । \newline
24. ए॒व तेन॒ तेनै॒वैव तेन॑ । \newline
25. तेना॑ मुच्यन्ता मुच्यन्त॒ तेन॒ तेना॑ मुच्यन्त । \newline
26. अ॒मु॒च्य॒न्त॒ यद् यद॑मुच्यन्ता मुच्यन्त॒ यत् । \newline
27. यदिन्द्रा॒ये न्द्रा॑य॒ यद् यदिन्द्रा॑य । \newline
28. इन्द्रा॑य वैमृ॒धाय॑ वैमृ॒धाये न्द्रा॒ये न्द्रा॑य वैमृ॒धाय॑ । \newline
29. वै॒मृ॒धाय॒ मृधो॒ मृधो॑ वैमृ॒धाय॑ वैमृ॒धाय॒ मृधः॑ । \newline
30. मृध॑ ए॒वैव मृधो॒ मृध॑ ए॒व । \newline
31. ए॒व तेन॒ ते नै॒वैव तेन॑ । \newline
32. तेनापाप॒ तेन॒ तेनाप॑ । \newline
33. अपा᳚घ्नता घ्न॒ता पापा᳚ घ्नत । \newline
34. अ॒घ्न॒त॒ यद् यद॑घ्नता घ्नत॒ यत् । \newline
35. यदिन्द्रा॒ये न्द्रा॑य॒ यद् यदिन्द्रा॑य । \newline
36. इन्द्रा॑ये न्द्रि॒याव॑त इन्द्रि॒याव॑त॒ इन्द्रा॒ये न्द्रा॑ये न्द्रि॒याव॑ते । \newline
37. इ॒न्द्रि॒याव॑त इन्द्रि॒य मि॑न्द्रि॒य मि॑न्द्रि॒याव॑त इन्द्रि॒याव॑त इन्द्रि॒यम् । \newline
38. इ॒न्द्रि॒याव॑त॒ इती᳚न्द्रि॒य - व॒ते॒ । \newline
39. इ॒न्द्रि॒य मे॒वैवे न्द्रि॒य मि॑न्द्रि॒य मे॒व । \newline
40. ए॒व तेन॒ तेनै॒वैव तेन॑ । \newline
41. तेना॒त्मन् ना॒त्मन् तेन॒ तेना॒त्मन्न् । \newline
42. आ॒त्मन् न॑दधता दधता॒त्मन् ना॒त्मन् न॑दधत । \newline
43. अ॒द॒ध॒त॒ त्रय॑स्त्रिꣳशत्कपाल॒म् त्रय॑स्त्रिꣳशत्कपाल मदधतादधत॒ त्रय॑स्त्रिꣳशत्कपालम् । \newline
44. त्रय॑स्त्रिꣳशत्कपालम् पुरो॒डाश॑म् पुरो॒डाश॒म् त्रय॑स्त्रिꣳशत्कपाल॒म् त्रय॑स्त्रिꣳशत्कपालम् पुरो॒डाश᳚म् । \newline
45. त्रय॑स्त्रिꣳशत्कपाल॒मिति॒ त्रय॑स्त्रिꣳशत् - क॒पा॒ल॒म् । \newline
46. पु॒रो॒डाश॒म् निर् णिष् पु॑रो॒डाश॑म् पुरो॒डाश॒म् निः । \newline
47. निर॑वपन् नवप॒न् निर् णिर॑वपन्न् । \newline
48. अ॒व॒प॒न् त्रय॑स्त्रिꣳश॒त् त्रय॑स्त्रिꣳश दवपन् नवप॒न् त्रय॑स्त्रिꣳशत् । \newline
49. त्रय॑स्त्रिꣳश॒द् वै वै त्रय॑स्त्रिꣳश॒त् त्रय॑स्त्रिꣳश॒द् वै । \newline
50. त्रय॑स्त्रिꣳश॒दिति॒ त्रयः॑ - त्रिꣳ॒॒श॒त् । \newline
51. वै दे॒वता॑ दे॒वता॒ वै वै दे॒वताः᳚ । \newline
52. दे॒वता॒ स्ता स्ता दे॒वता॑ दे॒वता॒ स्ताः । \newline
53. ता इन्द्र॒ इन्द्र॒ स्ता स्ता इन्द्रः॑ । \newline
54. इन्द्र॑ आ॒त्मन् ना॒त्मन् निन्द्र॒ इन्द्र॑ आ॒त्मन्न् । \newline
55. आ॒त्मन् नन्वन्वा॒ त्मन् ना॒त्मन् ननु॑ । \newline
56. अनु॑ स॒मारं॑भयत स॒मारं॑भय॒ता न्वनु॑ स॒मारं॑भयत । \newline
57. स॒मारं॑भयत॒ भूत्यै॒ भूत्यै॑ स॒मारं॑भयत स॒मारं॑भयत॒ भूत्यै᳚ । \newline
58. स॒मारं॑भय॒तेति॑ सं - आरं॑भयत । \newline
59. भूत्यै॒ ताम् ताम् भूत्यै॒ भूत्यै॒ ताम् । \newline

\textbf{Ghana Paata } \newline

1. इत्य॑ब्रवी दब्रवी॒ दिती त्य॑ब्रवी॒त् ते ते᳚ ऽब्रवी॒ दिती त्य॑ब्रवी॒त् ते । \newline
2. अ॒ब्र॒वी॒त् ते ते᳚ ऽब्रवी दब्रवी॒त् त इन्द्रा॒ये न्द्रा॑य॒ ते᳚ ऽब्रवी दब्रवी॒त् त इन्द्रा॑य । \newline
3. त इन्द्रा॒ये न्द्रा॑य॒ ते त इन्द्रा॑याꣳहो॒मुचे॑ ऽꣳहो॒मुच॒ इन्द्रा॑य॒ ते त इन्द्रा॑याꣳहो॒मुचे᳚ । \newline
4. इन्द्रा॑याꣳहो॒मुचे॑ ऽꣳहो॒मुच॒ इन्द्रा॒ये न्द्रा॑याꣳहो॒मुचे॑ पुरो॒डाश॑म् पुरो॒डाश॑ मꣳहो॒मुच॒ इन्द्रा॒ये न्द्रा॑याꣳहो॒मुचे॑ पुरो॒डाश᳚म् । \newline
5. अꣳ॒॒हो॒मुचे॑ पुरो॒डाश॑म् पुरो॒डाश॑ मꣳहो॒मुचे॑ ऽꣳहो॒मुचे॑ पुरो॒डाश॒ मेका॑दशकपाल॒ मेका॑दशकपालम् पुरो॒डाश॑ मꣳहो॒मुचे॑ ऽꣳहो॒मुचे॑ पुरो॒डाश॒ मेका॑दशकपालम् । \newline
6. अꣳ॒॒हो॒मुच॒ इत्यꣳ॑हः - मुचे᳚ । \newline
7. पु॒रो॒डाश॒ मेका॑दशकपाल॒ मेका॑दशकपालम् पुरो॒डाश॑म् पुरो॒डाश॒ मेका॑दशकपाल॒म् निर् णिरेका॑दशकपालम् पुरो॒डाश॑म् पुरो॒डाश॒ मेका॑दशकपाल॒म् निः । \newline
8. एका॑दशकपाल॒म् निर् णिरेका॑दशकपाल॒ मेका॑दशकपाल॒म् निर॑वपन् नवप॒न् निरेका॑दशकपाल॒ मेका॑दशकपाल॒म् निर॑वपन्न् । \newline
9. एका॑दशकपाल॒मित्येका॑दश - क॒पा॒ल॒म् । \newline
10. निर॑वपन् नवप॒न् निर् णिर॑वप॒न् निन्द्रा॒ये न्द्रा॑यावप॒न् निर् णिर॑वप॒न् निन्द्रा॑य । \newline
11. अ॒व॒प॒न् निन्द्रा॒ये न्द्रा॑यावपन् नवप॒न् निन्द्रा॑य वैमृ॒धाय॑ वैमृ॒धाये न्द्रा॑यावपन् नवप॒न् निन्द्रा॑य वैमृ॒धाय॑ । \newline
12. इन्द्रा॑य वैमृ॒धाय॑ वैमृ॒धाये न्द्रा॒ये न्द्रा॑य वैमृ॒धाये न्द्रा॒ये न्द्रा॑य वैमृ॒धाये न्द्रा॒ये न्द्रा॑य वैमृ॒धाये न्द्रा॑य । \newline
13. वै॒मृ॒धाये न्द्रा॒ये न्द्रा॑य वैमृ॒धाय॑ वैमृ॒धाये न्द्रा॑ये न्द्रि॒याव॑त इन्द्रि॒याव॑त॒ इन्द्रा॑य वैमृ॒धाय॑ वैमृ॒धाये न्द्रा॑ये न्द्रि॒याव॑ते । \newline
14. इन्द्रा॑ये न्द्रि॒याव॑त इन्द्रि॒याव॑त॒ इन्द्रा॒ये न्द्रा॑ये न्द्रि॒याव॑ते॒ यद् यदि॑न्द्रि॒याव॑त॒ इन्द्रा॒ये न्द्रा॑ये न्द्रि॒याव॑ते॒ यत् । \newline
15. इ॒न्द्रि॒याव॑ते॒ यद् यदि॑न्द्रि॒याव॑त इन्द्रि॒याव॑ते॒ यदिन्द्रा॒ये न्द्रा॑य॒ यदि॑न्द्रि॒याव॑त इन्द्रि॒याव॑ते॒ यदिन्द्रा॑य । \newline
16. इ॒न्द्रि॒याव॑त॒ इती᳚न्द्रि॒य - व॒ते॒ । \newline
17. यदिन्द्रा॒ये न्द्रा॑य॒ यद् यदिन्द्रा॑या ꣳहो॒मुचे ऽꣳ॑हो॒मुच॒ इन्द्रा॑य॒ यद् यदिन्द्रा॑या ꣳहो॒मुचे᳚ । \newline
18. इन्द्रा॑या ꣳहो॒मुचे ऽꣳ॑हो॒मुच॒ इन्द्रा॒ये न्द्रा॑याꣳहो॒मुचे॑ नि॒रव॑पन् नि॒रव॑पन् नꣳहो॒मुच॒ इन्द्रा॒ये न्द्रा॑या ꣳहो॒मुचे॑ नि॒रव॑पन्न् । \newline
19. अꣳ॒॒हो॒मुचे॑ नि॒रव॑पन् नि॒रव॑पन् नꣳहो॒मुचे ऽꣳ॑हो॒मुचे॑ नि॒रव॑प॒न् नꣳह॒सो ऽꣳह॑सो नि॒रव॑पन् नꣳहो॒मुचे ऽꣳ॑हो॒मुचे॑ नि॒रव॑प॒न् नꣳह॑सः । \newline
20. अꣳ॒॒हो॒मुच॒ इत्यꣳ॑हः - मुचे᳚ । \newline
21. नि॒रव॑प॒न् नꣳह॒सो ऽꣳह॑सो नि॒रव॑पन् नि॒रव॑प॒न् नꣳह॑स ए॒वैवाꣳह॑सो नि॒रव॑पन् नि॒रव॑प॒न् नꣳह॑स ए॒व । \newline
22. नि॒रव॑प॒न्निति॑ निः - अव॑पन्न् । \newline
23. अꣳह॑स ए॒वैवाꣳह॒सो ऽꣳह॑स ए॒व तेन॒ तेनै॒वाꣳह॒सो ऽꣳह॑स ए॒व तेन॑ । \newline
24. ए॒व तेन॒ तेनै॒वैव तेना॑ मुच्यन्ता मुच्यन्त॒ तेनै॒वैव तेना॑ मुच्यन्त । \newline
25. तेना॑मुच्यन्ता मुच्यन्त॒ तेन॒ तेना॑ मुच्यन्त॒ यद् यद॑मुच्यन्त॒ तेन॒ तेना॑ मुच्यन्त॒ यत् । \newline
26. अ॒मु॒च्य॒न्त॒ यद् यद॑मुच्यन्ता मुच्यन्त॒ यदिन्द्रा॒ये न्द्रा॑य॒ यद॑मुच्यन्ता मुच्यन्त॒ यदिन्द्रा॑य । \newline
27. यदिन्द्रा॒ये न्द्रा॑य॒ यद् यदिन्द्रा॑य वैमृ॒धाय॑ वैमृ॒धाये न्द्रा॑य॒ यद् यदिन्द्रा॑य वैमृ॒धाय॑ । \newline
28. इन्द्रा॑य वैमृ॒धाय॑ वैमृ॒धाये न्द्रा॒ये न्द्रा॑य वैमृ॒धाय॒ मृधो॒ मृधो॑ वैमृ॒धाये न्द्रा॒ये न्द्रा॑य वैमृ॒धाय॒ मृधः॑ । \newline
29. वै॒मृ॒धाय॒ मृधो॒ मृधो॑ वैमृ॒धाय॑ वैमृ॒धाय॒ मृध॑ ए॒वैव मृधो॑ वैमृ॒धाय॑ वैमृ॒धाय॒ मृध॑ ए॒व । \newline
30. मृध॑ ए॒वैव मृधो॒ मृध॑ ए॒व तेन॒ तेनै॒व मृधो॒ मृध॑ ए॒व तेन॑ । \newline
31. ए॒व तेन॒ तेनै॒वैव तेनापाप॒ तेनै॒वैव तेनाप॑ । \newline
32. तेनापाप॒ तेन॒ तेनापा᳚ घ्नता घ्न॒ताप॒ तेन॒ तेनापा᳚घ्नत । \newline
33. अपा᳚घ्नता घ्न॒ता पापा᳚घ्नत॒ यद् यद॑घ्न॒ता पापा᳚घ्नत॒ यत् । \newline
34. अ॒घ्न॒त॒ यद् यद॑घ्नता घ्नत॒ यदिन्द्रा॒ये न्द्रा॑य॒ यद॑घ्नता घ्नत॒ यदिन्द्रा॑य । \newline
35. यदिन्द्रा॒ये न्द्रा॑य॒ यद् यदिन्द्रा॑ये न्द्रि॒याव॑त इन्द्रि॒याव॑त॒ इन्द्रा॑य॒ यद् यदिन्द्रा॑ये न्द्रि॒याव॑ते । \newline
36. इन्द्रा॑ये न्द्रि॒याव॑त इन्द्रि॒याव॑त॒ इन्द्रा॒ये न्द्रा॑ये न्द्रि॒याव॑त इन्द्रि॒य मि॑न्द्रि॒य मि॑न्द्रि॒याव॑त॒ इन्द्रा॒ये न्द्रा॑ये न्द्रि॒याव॑त इन्द्रि॒यम् । \newline
37. इ॒न्द्रि॒याव॑त इन्द्रि॒य मि॑न्द्रि॒य मि॑न्द्रि॒याव॑त इन्द्रि॒याव॑त इन्द्रि॒य मे॒वैवे न्द्रि॒य मि॑न्द्रि॒याव॑त इन्द्रि॒याव॑त इन्द्रि॒य मे॒व । \newline
38. इ॒न्द्रि॒याव॑त॒ इती᳚न्द्रि॒य - व॒ते॒ । \newline
39. इ॒न्द्रि॒य मे॒वैवे न्द्रि॒य मि॑न्द्रि॒य मे॒व तेन॒ तेनै॒वे न्द्रि॒य मि॑न्द्रि॒य मे॒व तेन॑ । \newline
40. ए॒व तेन॒ तेनै॒वैव तेना॒त्मन् ना॒त्मन् तेनै॒वैव तेना॒त्मन्न् । \newline
41. तेना॒त्मन् ना॒त्मन् तेन॒ तेना॒त्मन् न॑दधता दधता॒त्मन् तेन॒ तेना॒त्मन् न॑दधत । \newline
42. आ॒त्मन् न॑दधता दधता॒त्मन् ना॒त्मन् न॑दधत॒ त्रय॑स्त्रिꣳशत्कपाल॒म् त्रय॑स्त्रिꣳशत्कपाल मदधता॒त्मन् ना॒त्मन् न॑दधत॒ त्रय॑स्त्रिꣳशत्कपालम् । \newline
43. अ॒द॒ध॒त॒ त्रय॑स्त्रिꣳशत्कपाल॒म् त्रय॑स्त्रिꣳशत्कपाल मदधता दधत॒ त्रय॑स्त्रिꣳशत्कपालम् पुरो॒डाश॑म् पुरो॒डाश॒म् त्रय॑स्त्रिꣳशत्कपाल मदधता दधत॒ त्रय॑स्त्रिꣳशत्कपालम् पुरो॒डाश᳚म् । \newline
44. त्रय॑स्त्रिꣳशत्कपालम् पुरो॒डाश॑म् पुरो॒डाश॒म् त्रय॑स्त्रिꣳशत्कपाल॒म् त्रय॑स्त्रिꣳशत्कपालम् पुरो॒डाश॒म् निर् णिष् पु॑रो॒डाश॒म् त्रय॑स्त्रिꣳशत्कपाल॒म् त्रय॑स्त्रिꣳशत्कपालम् पुरो॒डाश॒म् निः । \newline
45. त्रय॑स्त्रिꣳशत्कपाल॒मिति॒ त्रय॑स्त्रिꣳशत् - क॒पा॒ल॒म् । \newline
46. पु॒रो॒डाश॒म् निर् णिष् पु॑रो॒डाश॑म् पुरो॒डाश॒म् निर॑वपन् नवप॒न् निष् पु॑रो॒डाश॑म् पुरो॒डाश॒म् निर॑वपन्न् । \newline
47. निर॑वपन् नवप॒न् निर् णिर॑वप॒न् त्रय॑स्त्रिꣳश॒त् त्रय॑स्त्रिꣳश दवप॒न् निर् णिर॑वप॒न् त्रय॑स्त्रिꣳशत् । \newline
48. अ॒व॒प॒न् त्रय॑स्त्रिꣳश॒त् त्रय॑स्त्रिꣳश दवपन् नवप॒न् त्रय॑स्त्रिꣳश॒द् वै वै त्रय॑स्त्रिꣳश दवपन् नवप॒न् त्रय॑स्त्रिꣳश॒द् वै । \newline
49. त्रय॑स्त्रिꣳश॒द् वै वै त्रय॑स्त्रिꣳश॒त् त्रय॑स्त्रिꣳश॒द् वै दे॒वता॑ दे॒वता॒ वै त्रय॑स्त्रिꣳश॒त् त्रय॑स्त्रिꣳश॒द् वै दे॒वताः᳚ । \newline
50. त्रय॑स्त्रिꣳश॒दिति॒ त्रयः॑ - त्रिꣳ॒॒श॒त् । \newline
51. वै दे॒वता॑ दे॒वता॒ वै वै दे॒वता॒ स्ता स्ता दे॒वता॒ वै वै दे॒वता॒ स्ताः । \newline
52. दे॒वता॒ स्ता स्ता दे॒वता॑ दे॒वता॒ स्ता इन्द्र॒ इन्द्र॒ स्ता दे॒वता॑ दे॒वता॒ स्ता इन्द्रः॑ । \newline
53. ता इन्द्र॒ इन्द्र॒ स्ता स्ता इन्द्र॑ आ॒त्मन् ना॒त्मन् निन्द्र॒ स्ता स्ता इन्द्र॑ आ॒त्मन्न् । \newline
54. इन्द्र॑ आ॒त्मन् ना॒त्मन् निन्द्र॒ इन्द्र॑ आ॒त्मन् नन्वन्वा॒त्मन् निन्द्र॒ इन्द्र॑ आ॒त्मन् ननु॑ । \newline
55. आ॒त्मन् नन्वन्वा॒त्मन् ना॒त्मन् ननु॑ स॒मारं॑भयत स॒मारं॑भय॒ता न्वा॒त्मन् ना॒त्मन् ननु॑ स॒मारं॑भयत । \newline
56. अनु॑ स॒मारं॑भयत स॒मारं॑भय॒ता न्वनु॑ स॒मारं॑भयत॒ भूत्यै॒ भूत्यै॑ स॒मारं॑भय॒ता न्वनु॑ स॒मारं॑भयत॒ भूत्यै᳚ । \newline
57. स॒मारं॑भयत॒ भूत्यै॒ भूत्यै॑ स॒मारं॑भयत स॒मारं॑भयत॒ भूत्यै॒ ताम् ताम् भूत्यै॑ स॒मारं॑भयत स॒मारं॑भयत॒ भूत्यै॒ ताम् । \newline
58. स॒मारं॑भय॒तेति॑ सं - आरं॑भयत । \newline
59. भूत्यै॒ ताम् ताम् भूत्यै॒ भूत्यै॒ तां ॅवाव वाव ताम् भूत्यै॒ भूत्यै॒ तां ॅवाव । \newline
\pagebreak
\markright{ TS 2.4.2.3  \hfill https://www.vedavms.in \hfill}

\section{ TS 2.4.2.3 }

\textbf{TS 2.4.2.3 } \newline
\textbf{Samhita Paata} \newline

तां ॅवाव दे॒वा विजि॑ति-मुत्त॒मा-मसु॑रै॒-र्व्य॑जयन्त॒यो भ्रातृ॑व्यवा॒न्थ्- स्याथ् स स्पर्द्ध॑मान ए॒तयेष्‌ट्या॑ यजे॒तेन्द्रा॑याꣳ हो॒मुचे॑ पुरो॒डाश॒मेका॑दशकपालं॒ निर्व॑पे॒दिन्द्रा॑य वैमृ॒धायेन्द्रा॑येन्द्रि॒याव॒तेऽꣳ ह॑सा॒ वा ए॒ष गृ॑ही॒तो यस्मा॒च्छ्रेया॒न् भ्रातृ॑व्यो॒यदिन्द्रा॑याꣳ हो॒मुचे॑ नि॒र्वप॒त्यꣳह॑स ए॒व तेन॑ मुच्यतेमृ॒धा वा ए॒षो॑ऽभिष॑ण्णो॒ यस्मा᳚थ् समा॒नेष्व॒न्यः श्रेया॑नु॒ता - [  ] \newline

\textbf{Pada Paata} \newline

ताम् । वाव । दे॒वाः । विजि॑ति॒मिति॒ वि - जि॒ति॒म् । उ॒त्त॒मामित्यु॑त् - त॒माम् । असु॑रैः । वीति॑ । अ॒ज॒य॒न्त॒ । यः । भ्रातृ॑व्यवा॒निति॒ भ्रातृ॑व्य - वा॒न् । स्यात् । सः । स्पर्द्ध॑मानः । ए॒तया᳚ । इष्ट्या᳚ । य॒जे॒त॒ । इन्द्रा॑य । अꣳ॒॒हो॒मुच॒ इत्यꣳ॑हः - मुचे᳚ । पु॒रो॒डाश᳚म् । एका॑दशकपाल॒मित्येका॑दश - क॒पा॒ल॒म् । निरिति॑ । व॒पे॒त् । इन्द्रा॑य । वै॒मृ॒धाय॑ । इन्द्रा॑य । इ॒न्द्रि॒याव॑त॒ इती᳚न्द्रि॒य - व॒ते॒ । अꣳह॑सा । वै । ए॒षः । गृ॒ही॒तः । यस्मा᳚त् । श्रेयान्॑ । भ्रातृ॑व्यः । यत् । इन्द्रा॑य । अꣳ॒॒हो॒मुच॒ इत्यꣳ॑हः - मुचे᳚ । नि॒र्वप॒तीति॑ निः - वप॑ति । अꣳह॑सः । ए॒व । तेन॑ । मु॒च्य॒ते॒ । मृ॒धा । वै । ए॒षः । अ॒भिष॑ण्ण॒ इत्य॒भि - स॒न्नः॒ । यस्मा᳚त् । स॒मा॒नेषु॑ । अ॒न्यः । श्रेयान॑ । उ॒त ।  \newline


\textbf{Krama Paata} \newline

तां ॅवाव । वाव दे॒वाः । दे॒वा विजि॑तिम् । विजि॑तिमुत्त॒माम् । विजि॑ति॒मिति॒ वि - जि॒ति॒म् । उ॒त्त॒मामसु॑रैः । उ॒त्त॒मामित्यु॑त् - त॒माम् । असु॑रै॒र् वि । व्य॑जयन्त । अ॒ज॒य॒न्त॒ यः । यो भ्रातृ॑व्यवान् । भ्रातृ॑व्यवा॒न्थ् स्यात् । भ्रातृ॑व्यवा॒निति॒ भ्रातृ॑व्य - वा॒न्॒ । स्याथ् सः । स स्पर्द्ध॑मानः । स्पर्द्ध॑मान ए॒तया᳚ । ए॒तयेष्ट्या᳚ । इष्ट्या॑ यजेत । य॒जे॒तेन्द्रा॑य । इन्द्रा॑याꣳहो॒मुचे᳚ । अꣳ॒॒हो॒मुचे॑ पुरो॒डाश᳚म् । अꣳ॒॒हो॒मुच॒ इत्यꣳ॑हः - मुचे᳚ । पु॒रो॒डाश॒मेका॑दशकपालम् । एका॑दशकपाल॒म् निः । एका॑दशकपाल॒मित्येका॑दश - क॒पा॒ल॒म् । निर् व॑पेत् । व॒पे॒दिन्द्रा॑य । इन्द्रा॑य वैमृ॒धाय॑ । वै॒मृ॒धायेन्द्रा॑य । इन्द्रा॑येन्द्रि॒याव॑ते । इ॒न्द्रि॒याव॒ते ऽꣳह॑सा । इ॒न्द्रि॒याव॑त॒ इती᳚न्द्रि॒य - व॒ते॒ । अꣳह॑सा॒ वै । वा ए॒षः । ए॒ष गृ॑ही॒तः । गृ॒ही॒तो यस्मा᳚त् । यस्मा॒च्छ्रेयान्॑ । श्रेया॒न् भ्रातृ॑व्यः । भ्रातृ॑व्यो॒ यत् । यदिन्द्रा॑य । इन्द्रा॑याꣳहो॒मुचे᳚ । अꣳ॒॒हो॒मुचे॑ नि॒र्वप॑ति । अꣳ॒॒हो॒मुच॒ इत्यꣳ॑हः - मुचे᳚ । नि॒र्वप॒त्यꣳह॑सः । नि॒र्वप॒तीति॑ निः - वप॑ति । अꣳह॑स ए॒व । ए॒व तेन॑ । तेन॑ मुच्यते । मु॒च्य॒ते॒ मृ॒धा । मृ॒धा वै । वा ए॒षः । ए॒षो॑ ऽभिष॑ण्णः । अ॒भिष॑ण्णो॒ यस्मा᳚त् । अ॒भिष॑ण्ण॒ इत्य॒भि - स॒न्नः॒ । यस्मा᳚थ् समा॒नेषु॑ । स॒मा॒नेष्व॒न्यः । अ॒न्यः श्रेयान्॑ । श्रेया॑नु॒त ( ) । उ॒ताभ्रा॑तृव्यः \newline

\textbf{Jatai Paata} \newline

1. तां ॅवाव वाव ताम् तां ॅवाव । \newline
2. वाव दे॒वा दे॒वा वाव वाव दे॒वाः । \newline
3. दे॒वा विजि॑तिं॒ ॅविजि॑तिम् दे॒वा दे॒वा विजि॑तिम् । \newline
4. विजि॑ति मुत्त॒मा मु॑त्त॒मां ॅविजि॑तिं॒ ॅविजि॑ति मुत्त॒माम् । \newline
5. विजि॑ति॒मिति॒ वि - जि॒ति॒म् । \newline
6. उ॒त्त॒मा मसु॑ रै॒रसु॑रै रुत्त॒मा मु॑त्त॒मा मसु॑रैः । \newline
7. उ॒त्त॒मामित्यु॑त् - त॒माम् । \newline
8. असु॑रै॒र् वि व्यसु॑रै॒ रसु॑रै॒र् वि । \newline
9. व्य॑जयन्ता जयन्त॒ वि व्य॑जयन्त । \newline
10. अ॒ज॒य॒न्त॒ यो यो॑ ऽजयन्ता जयन्त॒ यः । \newline
11. यो भ्रातृ॑व्यवा॒न् भ्रातृ॑व्यवा॒न्॒. यो यो भ्रातृ॑व्यवान् । \newline
12. भ्रातृ॑व्यवा॒न् थ्स्याथ् स्याद् भ्रातृ॑व्यवा॒न् भ्रातृ॑व्यवा॒न् थ्स्यात् । \newline
13. भ्रातृ॑व्यवा॒निति॒ भ्रातृ॑व्य - वा॒न् । \newline
14. स्याथ् स स स्याथ् स्याथ् सः । \newline
15. स स्पर्द्ध॑मानः॒ स्पर्द्ध॑मानः॒ स स स्पर्द्ध॑मानः । \newline
16. स्पर्द्ध॑मान ए॒त यै॒तया॒ स्पर्द्ध॑मानः॒ स्पर्द्ध॑मान ए॒तया᳚ । \newline
17. ए॒त येष्ट्ये ष्ट्यै॒त यै॒त येष्ट्या᳚ । \newline
18. इष्ट्या॑ यजेत यजे॒ते ष्ट्येष्ट्या॑ यजेत । \newline
19. य॒जे॒ते न्द्रा॒ये न्द्रा॑य यजेत यजे॒ते न्द्रा॑य । \newline
20. इन्द्रा॑या ꣳहो॒मुचे ऽꣳ॑हो॒मुच॒ इन्द्रा॒ये न्द्रा॑या ꣳहो॒मुचे᳚ । \newline
21. अꣳ॒॒हो॒मुचे॑ पुरो॒डाश॑म् पुरो॒डाश॑ मꣳहो॒मुचे ऽꣳ॑हो॒मुचे॑ पुरो॒डाश᳚म् । \newline
22. अꣳ॒॒हो॒मुच॒ इत्यꣳ॑हः - मुचे᳚ । \newline
23. पु॒रो॒डाश॒ मेका॑दशकपाल॒ मेका॑दशकपालम् पुरो॒डाश॑म् पुरो॒डाश॒ मेका॑दशकपालम् । \newline
24. एका॑दशकपाल॒म् निर् णिरेका॑दशकपाल॒ मेका॑दशकपाल॒म् निः । \newline
25. एका॑दशकपाल॒मित्येका॑दश - क॒पा॒ल॒म् । \newline
26. निर् व॑पेद् वपे॒न् निर् णिर् व॑पेत् । \newline
27. व॒पे॒ दिन्द्रा॒ये न्द्रा॑य वपेद् वपे॒ दिन्द्रा॑य । \newline
28. इन्द्रा॑य वैमृ॒धाय॑ वैमृ॒धाये न्द्रा॒ये न्द्रा॑य वैमृ॒धाय॑ । \newline
29. वै॒मृ॒धाये न्द्रा॒ये न्द्रा॑य वैमृ॒धाय॑ वैमृ॒धाये न्द्रा॑य । \newline
30. इन्द्रा॑ये न्द्रि॒याव॑त इन्द्रि॒याव॑त॒ इन्द्रा॒ये न्द्रा॑ये न्द्रि॒याव॑ते । \newline
31. इ॒न्द्रि॒याव॒ते ऽꣳह॒सा ऽꣳह॑से न्द्रि॒याव॑त इन्द्रि॒याव॒ते ऽꣳह॑सा । \newline
32. इ॒न्द्रि॒याव॑त॒ इती᳚न्द्रि॒य - व॒ते॒ । \newline
33. अꣳह॑सा॒ वै वा अꣳह॒सा ऽꣳह॑सा॒ वै । \newline
34. वा ए॒ष ए॒ष वै वा ए॒षः । \newline
35. ए॒ष गृ॑ही॒तो गृ॑ही॒त ए॒ष ए॒ष गृ॑ही॒तः । \newline
36. गृ॒ही॒तो यस्मा॒द् यस्मा᳚द् गृही॒तो गृ॑ही॒तो यस्मा᳚त् । \newline
37. यस्मा॒च् छ्रेया॒ञ् छ्रेया॒न्॒. यस्मा॒द् यस्मा॒च् छ्रेयान्॑ । \newline
38. श्रेया॒न् भ्रातृ॑व्यो॒ भ्रातृ॑व्यः॒ श्रेया॒ञ् छ्रेया॒न् भ्रातृ॑व्यः । \newline
39. भ्रातृ॑व्यो॒ यद् यद् भ्रातृ॑व्यो॒ भ्रातृ॑व्यो॒ यत् । \newline
40. यदिन्द्रा॒ये न्द्रा॑य॒ यद् यदिन्द्रा॑य । \newline
41. इन्द्रा॑या ꣳहो॒मुचे ऽꣳ॑हो॒मुच॒ इन्द्रा॒ये न्द्रा॑या ꣳहो॒मुचे᳚ । \newline
42. अꣳ॒॒हो॒मुचे॑ नि॒र्वप॑ति नि॒र्वप॑ त्यꣳहो॒मुचे ऽꣳ॑हो॒मुचे॑ नि॒र्वप॑ति । \newline
43. अꣳ॒॒हो॒मुच॒ इत्यꣳ॑हः - मुचे᳚ । \newline
44. नि॒र्वप॒ त्यꣳह॒सो ऽꣳह॑सो नि॒र्वप॑ति नि॒र्वप॒ त्यꣳह॑सः । \newline
45. नि॒र्वप॒तीति॑ निः - वप॑ति । \newline
46. अꣳह॑स ए॒वैवा ꣳह॒सो ऽꣳह॑स ए॒व । \newline
47. ए॒व तेन॒ तेनै॒वैव तेन॑ । \newline
48. तेन॑ मुच्यते मुच्यते॒ तेन॒ तेन॑ मुच्यते । \newline
49. मु॒च्य॒ते॒ मृ॒धा मृ॒धा मु॑च्यते मुच्यते मृ॒धा । \newline
50. मृ॒धा वै वै मृ॒धा मृ॒धा वै । \newline
51. वा ए॒ष ए॒ष वै वा ए॒षः । \newline
52. ए॒षो॑ ऽभिष॑ण्णो॒ ऽभिष॑ण्ण ए॒ष ए॒षो॑ ऽभिष॑ण्णः । \newline
53. अ॒भिष॑ण्णो॒ यस्मा॒द् यस्मा॑ द॒भिष॑ण्णो॒ ऽभिष॑ण्णो॒ यस्मा᳚त् । \newline
54. अ॒भिष॑ण्ण॒ इत्य॒भि - स॒न्नः॒ । \newline
55. यस्मा᳚थ् समा॒नेषु॑ समा॒नेषु॒ यस्मा॒द् यस्मा᳚थ् समा॒नेषु॑ । \newline
56. स॒मा॒ने ष्व॒न्यो᳚ ऽन्यः स॑मा॒नेषु॑ समा॒ने ष्व॒न्यः । \newline
57. अ॒न्यः श्रेया॒ञ् छ्रेया॑ न॒न्यो᳚ ऽन्यः श्रेयान्॑ । \newline
58. श्रेया॑ नु॒तोत श्रेया॒ञ् छ्रेया॑ नु॒त । \newline
59. उ॒ता भ्रा॑तृ॒व्यो ऽभ्रा॑तृव्य उ॒तोता भ्रा॑तृव्यः । \newline

\textbf{Ghana Paata } \newline

1. तां ॅवाव वाव ताम् तां ॅवाव दे॒वा दे॒वा वाव ताम् तां ॅवाव दे॒वाः । \newline
2. वाव दे॒वा दे॒वा वाव वाव दे॒वा विजि॑तिं॒ ॅविजि॑तिम् दे॒वा वाव वाव दे॒वा विजि॑तिम् । \newline
3. दे॒वा विजि॑तिं॒ ॅविजि॑तिम् दे॒वा दे॒वा विजि॑ति मुत्त॒मा मु॑त्त॒मां ॅविजि॑तिम् दे॒वा दे॒वा विजि॑ति मुत्त॒माम् । \newline
4. विजि॑ति मुत्त॒मा मु॑त्त॒मां ॅविजि॑तिं॒ ॅविजि॑ति मुत्त॒मा मसु॑ रै॒रसु॑रै रुत्त॒मां ॅविजि॑तिं॒ ॅविजि॑ति मुत्त॒मा मसु॑रैः । \newline
5. विजि॑ति॒मिति॒ वि - जि॒ति॒म् । \newline
6. उ॒त्त॒मा मसु॑रै॒ रसु॑रै रुत्त॒मा मु॑त्त॒मा मसु॑रै॒र् वि व्यसु॑रै रुत्त॒मा मु॑त्त॒मा मसु॑रै॒र् वि । \newline
7. उ॒त्त॒मामित्यु॑त् - त॒माम् । \newline
8. असु॑रै॒र् वि व्यसु॑रै॒ रसु॑रै॒र् व्य॑जयन्ता जयन्त॒ व्यसु॑रै॒ रसु॑रै॒र् व्य॑जयन्त । \newline
9. व्य॑जयन्ता जयन्त॒ वि व्य॑जयन्त॒ यो यो॑ ऽजयन्त॒ वि व्य॑जयन्त॒ यः । \newline
10. अ॒ज॒य॒न्त॒ यो यो॑ ऽजयन्ता जयन्त॒ यो भ्रातृ॑व्यवा॒न् भ्रातृ॑व्यवा॒न्॒. यो॑ ऽजयन्ता जयन्त॒ यो भ्रातृ॑व्यवान् । \newline
11. यो भ्रातृ॑व्यवा॒न् भ्रातृ॑व्यवा॒न्॒. यो यो भ्रातृ॑व्यवा॒न् थ्स्याथ् स्याद् भ्रातृ॑व्यवा॒न्॒. यो यो भ्रातृ॑व्यवा॒न् थ्स्यात् । \newline
12. भ्रातृ॑व्यवा॒न् थ्स्याथ् स्याद् भ्रातृ॑व्यवा॒न् भ्रातृ॑व्यवा॒न् थ्स्याथ् स स स्याद् भ्रातृ॑व्यवा॒न् भ्रातृ॑व्यवा॒न् थ्स्याथ् सः । \newline
13. भ्रातृ॑व्यवा॒निति॒ भ्रातृ॑व्य - वा॒न् । \newline
14. स्याथ् स स स्याथ् स्याथ् स स्पर्द्ध॑मानः॒ स्पर्द्ध॑मानः॒ स स्याथ् स्याथ् स स्पर्द्ध॑मानः । \newline
15. स स्पर्द्ध॑मानः॒ स्पर्द्ध॑मानः॒ स स स्पर्द्ध॑मान ए॒तयै॒तया॒ स्पर्द्ध॑मानः॒ स स स्पर्द्ध॑मान ए॒तया᳚ । \newline
16. स्पर्द्ध॑मान ए॒तयै॒तया॒ स्पर्द्ध॑मानः॒ स्पर्द्ध॑मान ए॒त येष्ट्ये ष्ट्यै॒तया॒ स्पर्द्ध॑मानः॒ स्पर्द्ध॑मान ए॒तयेष्ट्या᳚ । \newline
17. ए॒तयेष्ट्ये ष्ट्यै॒त यै॒तयेष्ट्या॑ यजेत यजे॒ते ष्ट्यै॒त यै॒तयेष्ट्या॑ यजेत । \newline
18. इष्ट्या॑ यजेत यजे॒ते ष्ट्येष्ट्या॑ यजे॒ते न्द्रा॒ये न्द्रा॑य यजे॒ते ष्ट्येष्ट्या॑ यजे॒ते न्द्रा॑य । \newline
19. य॒जे॒ते न्द्रा॒ये न्द्रा॑य यजेत यजे॒ते न्द्रा॑याꣳहो॒मुचे ऽꣳ॑हो॒मुच॒ इन्द्रा॑य यजेत यजे॒ते न्द्रा॑याꣳहो॒मुचे᳚ । \newline
20. इन्द्रा॑याꣳहो॒मुचे ऽꣳ॑हो॒मुच॒ इन्द्रा॒ये न्द्रा॑याꣳहो॒मुचे॑ पुरो॒डाश॑म् पुरो॒डाश॑ मꣳहो॒मुच॒ इन्द्रा॒ये न्द्रा॑याꣳहो॒मुचे॑ पुरो॒डाश᳚म् । \newline
21. अꣳ॒॒हो॒मुचे॑ पुरो॒डाश॑म् पुरो॒डाश॑ मꣳहो॒मुचे ऽꣳ॑हो॒मुचे॑ पुरो॒डाश॒ मेका॑दशकपाल॒ मेका॑दशकपालम् पुरो॒डाश॑ मꣳहो॒मुचे ऽꣳ॑हो॒मुचे॑ पुरो॒डाश॒ मेका॑दशकपालम् । \newline
22. अꣳ॒॒हो॒मुच॒ इत्यꣳ॑हः - मुचे᳚ । \newline
23. पु॒रो॒डाश॒ मेका॑दशकपाल॒ मेका॑दशकपालम् पुरो॒डाश॑म् पुरो॒डाश॒ मेका॑दशकपाल॒म् निर् णिरेका॑दशकपालम् पुरो॒डाश॑म् पुरो॒डाश॒ मेका॑दशकपाल॒म् निः । \newline
24. एका॑दशकपाल॒म् निर् णिरेका॑दशकपाल॒ मेका॑दशकपाल॒म् निर् व॑पेद् वपे॒न् निरेका॑दशकपाल॒ मेका॑दशकपाल॒म् निर् व॑पेत् । \newline
25. एका॑दशकपाल॒मित्येका॑दश - क॒पा॒ल॒म् । \newline
26. निर् व॑पेद् वपे॒न् निर् णिर् व॑पे॒ दिन्द्रा॒ये न्द्रा॑य वपे॒न् निर् णिर् व॑पे॒ दिन्द्रा॑य । \newline
27. व॒पे॒दिन्द्रा॒ये न्द्रा॑य वपेद् वपे॒दिन्द्रा॑य वैमृ॒धाय॑ वैमृ॒धाये न्द्रा॑य वपेद् वपे॒दिन्द्रा॑य वैमृ॒धाय॑ । \newline
28. इन्द्रा॑य वैमृ॒धाय॑ वैमृ॒धाये न्द्रा॒ये न्द्रा॑य वैमृ॒धाये न्द्रा॒ये न्द्रा॑य वैमृ॒धाये न्द्रा॒ये न्द्रा॑य वैमृ॒धाये न्द्रा॑य । \newline
29. वै॒मृ॒धाये न्द्रा॒ये न्द्रा॑य वैमृ॒धाय॑ वैमृ॒धाये न्द्रा॑ये न्द्रि॒याव॑त इन्द्रि॒याव॑त॒ इन्द्रा॑य वैमृ॒धाय॑ वैमृ॒धाये न्द्रा॑ये न्द्रि॒याव॑ते । \newline
30. इन्द्रा॑ये न्द्रि॒याव॑त इन्द्रि॒याव॑त॒ इन्द्रा॒ये न्द्रा॑ये न्द्रि॒याव॒ते ऽꣳह॒सा ऽꣳह॑से न्द्रि॒याव॑त॒ इन्द्रा॒ये न्द्रा॑ये न्द्रि॒याव॒ते ऽꣳह॑सा । \newline
31. इ॒न्द्रि॒याव॒ते ऽꣳह॒सा ऽꣳह॑से न्द्रि॒याव॑त इन्द्रि॒याव॒ते ऽꣳह॑सा॒ वै वा अꣳह॑से न्द्रि॒याव॑त इन्द्रि॒याव॒ते ऽꣳह॑सा॒ वै । \newline
32. इ॒न्द्रि॒याव॑त॒ इती᳚न्द्रि॒य - व॒ते॒ । \newline
33. अꣳह॑सा॒ वै वा अꣳह॒सा ऽꣳह॑सा॒ वा ए॒ष ए॒ष वा अꣳह॒सा ऽꣳह॑सा॒ वा ए॒षः । \newline
34. वा ए॒ष ए॒ष वै वा ए॒ष गृ॑ही॒तो गृ॑ही॒त ए॒ष वै वा ए॒ष गृ॑ही॒तः । \newline
35. ए॒ष गृ॑ही॒तो गृ॑ही॒त ए॒ष ए॒ष गृ॑ही॒तो यस्मा॒द् यस्मा᳚द् गृही॒त ए॒ष ए॒ष गृ॑ही॒तो यस्मा᳚त् । \newline
36. गृ॒ही॒तो यस्मा॒द् यस्मा᳚द् गृही॒तो गृ॑ही॒तो यस्मा॒च् छ्रेया॒ञ् छ्रेया॒न्॒. यस्मा᳚द् गृही॒तो गृ॑ही॒तो यस्मा॒च् छ्रेयान्॑ । \newline
37. यस्मा॒च् छ्रेया॒ञ् छ्रेया॒न्॒. यस्मा॒द् यस्मा॒च् छ्रेया॒न् भ्रातृ॑व्यो॒ भ्रातृ॑व्यः॒ श्रेया॒न्॒. यस्मा॒द् यस्मा॒च् छ्रेया॒न् भ्रातृ॑व्यः । \newline
38. श्रेया॒न् भ्रातृ॑व्यो॒ भ्रातृ॑व्यः॒ श्रेया॒ञ् छ्रेया॒न् भ्रातृ॑व्यो॒ यद् यद् भ्रातृ॑व्यः॒ श्रेया॒ञ् छ्रेया॒न् भ्रातृ॑व्यो॒ यत् । \newline
39. भ्रातृ॑व्यो॒ यद् यद् भ्रातृ॑व्यो॒ भ्रातृ॑व्यो॒ यदिन्द्रा॒ये न्द्रा॑य॒ यद् भ्रातृ॑व्यो॒ भ्रातृ॑व्यो॒ यदिन्द्रा॑य । \newline
40. यदिन्द्रा॒ये न्द्रा॑य॒ यद् यदिन्द्रा॑या ꣳहो॒मुचे ऽꣳ॑हो॒मुच॒ इन्द्रा॑य॒ यद् यदिन्द्रा॑या ꣳहो॒मुचे᳚ । \newline
41. इन्द्रा॑या ꣳहो॒मुचे ऽꣳहो॒॑मुच॒ इन्द्रा॒ये न्द्रा॑या ꣳहो॒मुचे॑ नि॒र्वप॑ति नि॒र्वप॑ त्यꣳहो॒मुच॒ इन्द्रा॒ये न्द्रा॑याꣳहो॒मुचे॑ नि॒र्वप॑ति । \newline
42. अꣳ॒॒हो॒मुचे॑ नि॒र्वप॑ति नि॒र्वप॑ त्यꣳहो॒मुचे ऽꣳ॑हो॒मुचे॑ नि॒र्वप॒ त्यꣳह॒सो ऽꣳह॑सो नि॒र्वप॑ 
त्यꣳहो॒मुचे ऽꣳ॑हो॒मुचे॑ नि॒र्वप॒ त्यꣳह॑सः । \newline
43. अꣳ॒॒हो॒मुच॒ इत्यꣳ॑हः - मुचे᳚ । \newline
44. नि॒र्वप॒ त्यꣳह॒सो ऽꣳह॑सो नि॒र्वप॑ति नि॒र्वप॒ त्यꣳह॑स ए॒वैवाꣳह॑सो नि॒र्वप॑ति नि॒र्वप॒ त्यꣳह॑स ए॒व । \newline
45. नि॒र्वप॒तीति॑ निः - वप॑ति । \newline
46. अꣳह॑स ए॒वैवाꣳह॒सो ऽꣳह॑स ए॒व तेन॒ तेनै॒वाꣳह॒सो ऽꣳह॑स ए॒व तेन॑ । \newline
47. ए॒व तेन॒ तेनै॒वैव तेन॑ मुच्यते मुच्यते॒ तेनै॒वैव तेन॑ मुच्यते । \newline
48. तेन॑ मुच्यते मुच्यते॒ तेन॒ तेन॑ मुच्यते मृ॒धा मृ॒धा मु॑च्यते॒ तेन॒ तेन॑ मुच्यते मृ॒धा । \newline
49. मु॒च्य॒ते॒ मृ॒धा मृ॒धा मु॑च्यते मुच्यते मृ॒धा वै वै मृ॒धा मु॑च्यते मुच्यते मृ॒धा वै । \newline
50. मृ॒धा वै वै मृ॒धा मृ॒धा वा ए॒ष ए॒ष वै मृ॒धा मृ॒धा वा ए॒षः । \newline
51. वा ए॒ष ए॒ष वै वा ए॒षो॑ ऽभिष॑ण्णो॒ ऽभिष॑ण्ण ए॒ष वै वा ए॒षो॑ ऽभिष॑ण्णः । \newline
52. ए॒षो॑ ऽभिष॑ण्णो॒ ऽभिष॑ण्ण ए॒ष ए॒षो॑ ऽभिष॑ण्णो॒ यस्मा॒द् यस्मा॑ द॒भिष॑ण्ण ए॒ष ए॒षो॑ ऽभिष॑ण्णो॒ यस्मा᳚त् । \newline
53. अ॒भिष॑ण्णो॒ यस्मा॒द् यस्मा॑ द॒भिष॑ण्णो॒ ऽभिष॑ण्णो॒ यस्मा᳚थ् समा॒नेषु॑ समा॒नेषु॒ यस्मा॑ द॒भिष॑ण्णो॒ ऽभिष॑ण्णो॒ यस्मा᳚थ् समा॒नेषु॑ । \newline
54. अ॒भिष॑ण्ण॒ इत्य॒भि - स॒न्नः॒ । \newline
55. यस्मा᳚थ् समा॒नेषु॑ समा॒नेषु॒ यस्मा॒द् यस्मा᳚थ् समा॒नेष्व॒न्यो᳚ ऽन्यः स॑मा॒नेषु॒ यस्मा॒द् यस्मा᳚थ् समा॒नेष्व॒न्यः । \newline
56. स॒मा॒नेष्व॒न्यो᳚ ऽन्यः स॑मा॒नेषु॑ समा॒नेष्व॒न्यः श्रेया॒ञ् छ्रेया॑ न॒न्यः स॑मा॒नेषु॑ समा॒नेष्व॒न्यः श्रेयान्॑ । \newline
57. अ॒न्यः श्रेया॒ञ् छ्रेया॑ न॒न्यो᳚ ऽन्यः श्रेया॑ नु॒तोत श्रेया॑ न॒न्यो᳚ ऽन्यः श्रेया॑ नु॒त । \newline
58. श्रेया॑ नु॒तोत श्रेया॒ञ् छ्रेया॑ नु॒ताभ्रा॑तृ॒व्यो ऽभ्रा॑तृव्य उ॒त श्रेया॒ञ् छ्रेया॑ नु॒ताभ्रा॑तृव्यः । \newline
59. उ॒ताभ्रा॑तृ॒व्यो ऽभ्रा॑तृव्य उ॒तोता भ्रा॑तृव्यो॒ यद् यदभ्रा॑तृव्य उ॒तोता भ्रा॑तृव्यो॒ यत् । \newline
\pagebreak
\markright{ TS 2.4.2.4  \hfill https://www.vedavms.in \hfill}

\section{ TS 2.4.2.4 }

\textbf{TS 2.4.2.4 } \newline
\textbf{Samhita Paata} \newline

-ऽभ्रा॑तृव्यो॒ यदिन्द्रा॑य वैमृ॒धाय॒ मृध॑ ए॒व तेनाप॑ हते॒यदिन्द्रा॑येन्द्रि॒याव॑त इन्द्रि॒यमे॒व तेना॒त्मन् ध॑त्ते॒ त्रय॑स्त्रिꣳशत्कपालं पुरो॒डाशं॒ निर्व॑पति॒ त्रय॑स्त्रिꣳश॒द्वै दे॒वता॒स्ता ए॒व यज॑मान आ॒त्मन्ननु॑ स॒मार॑भंयते॒ भूत्यै॒ सा वा ए॒षा विजि॑ति॒र्नामेष्टि॒र्य ए॒वं ॅवि॒द्वाने॒तयेष्‌ट्या॒ यज॑त उत्त॒मामे॒व विजि॑तिं॒ भ्रातृ॑व्येण॒ वि ज॑यते ॥(इ॒न्द्रि॒याव॑ती॒ - भूत्या॑ - उ॒तै - का॒न्न प॑ञ्चा॒शच्च॑) \newline

\textbf{Pada Paata} \newline

अभ्रा॑तृव्यः । यत् । इन्द्रा॑य । वै॒मृ॒धाय॑ । मृधः॑ । ए॒व । तेन॑ । अपेति॑ । ह॒ते॒ । यत् । इन्द्रा॑य । इ॒न्द्रि॒याव॑त॒ इती᳚न्द्रि॒य - व॒ते॒ । इ॒न्द्रि॒यम् । ए॒व । तेन॑ । आ॒त्मन्न् । ध॒त्ते॒ । त्रय॑स्त्रिꣳशत्कपाल॒मिति॒ त्रय॑स्त्रिꣳशत् - क॒पा॒ल॒म् । पु॒रो॒डाश᳚म् । निरिति॑ । व॒प॒ति॒ । त्रय॑स्त्रिꣳश॒दिति॒ त्रयः॑ - त्रिꣳ॒॒श॒त् । वै । दे॒वताः᳚ । ताः । ए॒व । यज॑मानः । आ॒त्मन्न् । अन्विति॑ । स॒मार॑भंयत॒ इति॑ सं - आर॑भंयते । भूत्यै᳚ । सा । वै । ए॒षा । विजि॑ति॒रिति॒ वि - जि॒तिः॒ । नाम॑ । इष्टिः॑ । यः । ए॒वम् । वि॒द्वान् । ए॒तया᳚ । इष्ट्या᳚ । यज॑ते । उ॒त्त॒मामित्यु॑त् - त॒माम् । ए॒व । विजि॑ति॒मिति॒ वि - जि॒ति॒म् । भ्रातृ॑व्येण । वीति॑ । ज॒य॒ते॒ ॥  \newline


\textbf{Krama Paata} \newline

अभ्रा॑तृव्यो॒ यत् । यदिन्द्रा॑य । इन्द्रा॑य वैमृ॒धाय॑ । वै॒मृ॒धाय॒ मृधः॑ । मृध॑ ए॒व । ए॒व तेन॑ । तेनाप॑ । अप॑ हते । ह॒ते॒ यत् । यदिन्द्रा॑य । इन्द्रा॑येन्द्रि॒याव॑ते । इ॒न्द्रि॒याव॑त इन्द्रि॒यम् । इ॒न्द्रि॒याव॑त॒ इती᳚न्द्रि॒य - व॒ते॒ । इ॒न्द्रि॒यमे॒व । ए॒व तेन॑ । तेना॒त्मन्न् । आ॒त्मन् ध॑त्ते । ध॒त्ते॒ त्रय॑स्त्रिꣳशत्कपालम् । त्रय॑स्त्रिꣳशत्कपालम् पुरो॒डाश᳚म् । त्रय॑स्त्रिꣳशत्कपाल॒मिति॒ त्रय॑स्त्रिꣳशत् - क॒पा॒ल॒म् । पु॒रो॒डाश॒म् निः । निर् व॑पति । व॒प॒ति॒ त्रय॑स्त्रिꣳशत् । त्रय॑स्त्रिꣳश॒द् वै । त्रय॑स्त्रिꣳश॒दिति॒ त्रयः॑ - त्रिꣳ॒॒श॒त्॒ । वै दे॒वताः᳚ । दे॒वता॒स्ताः । ता ए॒व । ए॒व यज॑मानः । यज॑मान आ॒त्मन्न् । आ॒त्मन्ननु॑ । अनु॑ स॒मार॑म्भयते । स॒मार॑म्भयते॒ भूत्यै᳚ । स॒मार॑म्भयत॒ इति॑ सम् - आर॑म्भयते । भूत्यै॒ सा । सा वै । वा ए॒षा । ए॒षा विजि॑तिः । विजि॑ति॒र् नाम॑ । विजि॑ति॒रिति॒ वि - जि॒तिः॒ । नामेष्टिः॑ । इष्टि॒र् यः । य ए॒वम् । ए॒वं ॅवि॒द्वान् । वि॒द्वाने॒तया᳚ । ए॒तयेष्ट्या᳚ । इष्ट्या॒ यज॑ते । यज॑त उत्त॒माम् । उ॒त्त॒मामे॒व । उ॒त्त॒मामित्यु॑त् - त॒माम् । ए॒व विजि॑तम् । विजि॑ति॒म् भ्रातृ॑व्येण । विजि॑ति॒मिति॒ वि - जि॒ति॒म् । भ्रातृ॑व्येण॒ वि । वि ज॑यते । ज॒य॒त॒ इति॑ जयते । \newline

\textbf{Jatai Paata} \newline

1. अभ्रा॑तृव्यो॒ यद् यदभ्रा॑तृ॒व्यो ऽभ्रा॑तृव्यो॒ यत् । \newline
2. यदिन्द्रा॒ये न्द्रा॑य॒ यद् यदिन्द्रा॑य । \newline
3. इन्द्रा॑य वैमृ॒धाय॑ वैमृ॒धाये न्द्रा॒ये न्द्रा॑य वैमृ॒धाय॑ । \newline
4. वै॒मृ॒धाय॒ मृधो॒ मृधो॑ वैमृ॒धाय॑ वैमृ॒धाय॒ मृधः॑ । \newline
5. मृध॑ ए॒वैव मृधो॒ मृध॑ ए॒व । \newline
6. ए॒व तेन॒ ते नै॒वैव तेन॑ । \newline
7. तेनापाप॒ तेन॒ तेनाप॑ । \newline
8. अप॑ हते ह॒ते ऽपाप॑ हते । \newline
9. ह॒ते॒ यद् य द्ध॑ते हते॒ यत् । \newline
10. यदिन्द्रा॒ये न्द्रा॑य॒ यद् यदिन्द्रा॑य । \newline
11. इन्द्रा॑ये न्द्रि॒याव॑त इन्द्रि॒याव॑त॒ इन्द्रा॒ये न्द्रा॑ये न्द्रि॒याव॑ते । \newline
12. इ॒न्द्रि॒याव॑त इन्द्रि॒य मि॑न्द्रि॒य मि॑न्द्रि॒याव॑त इन्द्रि॒याव॑त इन्द्रि॒यम् । \newline
13. इ॒न्द्रि॒याव॑त॒ इती᳚न्द्रि॒य - व॒ते॒ । \newline
14. इ॒न्द्रि॒य मे॒वैवे न्द्रि॒य मि॑न्द्रि॒य मे॒व । \newline
15. ए॒व तेन॒ ते नै॒वैव तेन॑ । \newline
16. तेना॒ त्मन् ना॒त्मन् तेन॒ तेना॒ त्मन्न् । \newline
17. आ॒त्मन् ध॑त्ते धत्त आ॒त्मन् ना॒त्मन् ध॑त्ते । \newline
18. ध॒त्ते॒ त्रय॑स्त्रिꣳशत्कपाल॒म् त्रय॑स्त्रिꣳशत्कपालम् धत्ते धत्ते॒ त्रय॑स्त्रिꣳशत्कपालम् । \newline
19. त्रय॑स्त्रिꣳशत्कपालम् पुरो॒डाश॑म् पुरो॒डाश॒म् त्रय॑स्त्रिꣳशत्कपाल॒म् त्रय॑स्त्रिꣳशत्कपालम् पुरो॒डाश᳚म् । \newline
20. त्रय॑स्त्रिꣳशत्कपाल॒मिति॒ त्रय॑स्त्रिꣳशत् - क॒पा॒ल॒म् । \newline
21. पु॒रो॒डाश॒म् निर् णिष् पु॑रो॒डाश॑म् पुरो॒डाश॒म् निः । \newline
22. निर् व॑पति वपति॒ निर् णिर् व॑पति । \newline
23. व॒प॒ति॒ त्रय॑स्त्रिꣳश॒त् त्रय॑स्त्रिꣳशद् वपति वपति॒ त्रय॑स्त्रिꣳशत् । \newline
24. त्रय॑स्त्रिꣳश॒द् वै वै त्रय॑स्त्रिꣳश॒त् त्रय॑स्त्रिꣳश॒द् वै । \newline
25. त्रय॑स्त्रिꣳश॒दिति॒ त्रयः॑ - त्रिꣳ॒॒श॒त् । \newline
26. वै दे॒वता॑ दे॒वता॒ वै वै दे॒वताः᳚ । \newline
27. दे॒वता॒ स्ता स्ता दे॒वता॑ दे॒वता॒ स्ताः । \newline
28. ता ए॒वैव ता स्ता ए॒व । \newline
29. ए॒व यज॑मानो॒ यज॑मान ए॒वैव यज॑मानः । \newline
30. यज॑मान आ॒त्मन् ना॒त्मन्. यज॑मानो॒ यज॑मान आ॒त्मन्न् । \newline
31. आ॒त्मन् नन्वन्वा॒ त्मन् ना॒त्मन् ननु॑ । \newline
32. अनु॑ स॒मारं॑भयते स॒मारं॑भय॒ते ऽन्वनु॑ स॒मारं॑भयते । \newline
33. स॒मारं॑भयते॒ भूत्यै॒ भूत्यै॑ स॒मारं॑भयते स॒मारं॑भयते॒ भूत्यै᳚ । \newline
34. स॒मारं॑भयत॒ इति॑ सं - आरं॑भयते । \newline
35. भूत्यै॒ सा सा भूत्यै॒ भूत्यै॒ सा । \newline
36. सा वै वै सा सा वै । \newline
37. वा ए॒षैषा वै वा ए॒षा । \newline
38. ए॒षा विजि॑ति॒र् विजि॑ति रे॒षैषा विजि॑तिः । \newline
39. विजि॑ति॒र् नाम॒ नाम॒ विजि॑ति॒र् विजि॑ति॒र् नाम॑ । \newline
40. विजि॑ति॒रिति॒ वि - जि॒तिः॒ । \newline
41. नामे ष्टि॒ रिष्टि॒र् नाम॒ नामे ष्टिः॑ । \newline
42. इष्टि॒र् यो य इष्टि॒ रिष्टि॒र् यः । \newline
43. य ए॒व मे॒वं ॅयो य ए॒वम् । \newline
44. ए॒वं ॅवि॒द्वान्. वि॒द्वा ने॒व मे॒वं ॅवि॒द्वान् । \newline
45. वि॒द्वा ने॒त यै॒तया॑ वि॒द्वान्. वि॒द्वा ने॒तया᳚ । \newline
46. ए॒त येष्ट्ये ष्ट्यै॒त यै॒त येष्ट्या᳚ । \newline
47. इष्ट्या॒ यज॑ते॒ यज॑त॒ इष्ट्येष्ट्या॒ यज॑ते । \newline
48. यज॑त उत्त॒मा मु॑त्त॒मां ॅयज॑ते॒ यज॑त उत्त॒माम् । \newline
49. उ॒त्त॒मा मे॒वैवो त्त॒मा मु॑त्त॒मा मे॒व । \newline
50. उ॒त्त॒मामित्यु॑त् - त॒माम् । \newline
51. ए॒व विजि॑तिं॒ ॅविजि॑ति मे॒वैव विजि॑तिम् । \newline
52. विजि॑ति॒म् भ्रातृ॑व्येण॒ भ्रातृ॑व्येण॒ विजि॑तिं॒ ॅविजि॑ति॒म् भ्रातृ॑व्येण । \newline
53. विजि॑ति॒मिति॒ वि - जि॒ति॒म् । \newline
54. भ्रातृ॑व्येण॒ वि वि भ्रातृ॑व्येण॒ भ्रातृ॑व्येण॒ वि । \newline
55. वि ज॑यते जयते॒ वि वि ज॑यते । \newline
56. ज॒य॒त॒ इति॑ जयते । \newline

\textbf{Ghana Paata } \newline

1. अभ्रा॑तृव्यो॒ यद् यदभ्रा॑तृ॒व्यो ऽभ्रा॑तृव्यो॒ यदिन्द्रा॒ये न्द्रा॑य॒ यदभ्रा॑तृ॒व्यो ऽभ्रा॑तृव्यो॒ यदिन्द्रा॑य । \newline
2. यदिन्द्रा॒ये न्द्रा॑य॒ यद् यदिन्द्रा॑य वैमृ॒धाय॑ वैमृ॒धाये न्द्रा॑य॒ यद् यदिन्द्रा॑य वैमृ॒धाय॑ । \newline
3. इन्द्रा॑य वैमृ॒धाय॑ वैमृ॒धाये न्द्रा॒ये न्द्रा॑य वैमृ॒धाय॒ मृधो॒ मृधो॑ वैमृ॒धाये न्द्रा॒ये न्द्रा॑य वैमृ॒धाय॒ मृधः॑ । \newline
4. वै॒मृ॒धाय॒ मृधो॒ मृधो॑ वैमृ॒धाय॑ वैमृ॒धाय॒ मृध॑ ए॒वैव मृधो॑ वैमृ॒धाय॑ वैमृ॒धाय॒ मृध॑ ए॒व । \newline
5. मृध॑ ए॒वैव मृधो॒ मृध॑ ए॒व तेन॒ तेनै॒व मृधो॒ मृध॑ ए॒व तेन॑ । \newline
6. ए॒व तेन॒ तेनै॒वैव तेनापाप॒ तेनै॒वैव तेनाप॑ । \newline
7. तेनापाप॒ तेन॒ तेनाप॑ हते ह॒ते ऽप॒ तेन॒ तेनाप॑ हते । \newline
8. अप॑ हते ह॒ते ऽपाप॑ हते॒ यद् यद्ध॒ते ऽपाप॑ हते॒ यत् । \newline
9. ह॒ते॒ यद् यद्ध॑ते हते॒ यदिन्द्रा॒ये न्द्रा॑य॒ यद्ध॑ते हते॒ यदिन्द्रा॑य । \newline
10. यदिन्द्रा॒ये न्द्रा॑य॒ यद् यदिन्द्रा॑ये न्द्रि॒याव॑त इन्द्रि॒याव॑त॒ इन्द्रा॑य॒ यद् यदिन्द्रा॑ये न्द्रि॒याव॑ते । \newline
11. इन्द्रा॑ये न्द्रि॒याव॑त इन्द्रि॒याव॑त॒ इन्द्रा॒ये न्द्रा॑ये न्द्रि॒याव॑त इन्द्रि॒य मि॑न्द्रि॒य मि॑न्द्रि॒याव॑त॒ इन्द्रा॒ये न्द्रा॑ये न्द्रि॒याव॑त इन्द्रि॒यम् । \newline
12. इ॒न्द्रि॒याव॑त इन्द्रि॒य मि॑न्द्रि॒य मि॑न्द्रि॒याव॑त इन्द्रि॒याव॑त इन्द्रि॒य मे॒वैवे न्द्रि॒य मि॑न्द्रि॒याव॑त इन्द्रि॒याव॑त इन्द्रि॒य मे॒व । \newline
13. इ॒न्द्रि॒याव॑त॒ इती᳚न्द्रि॒य - व॒ते॒ । \newline
14. इ॒न्द्रि॒य मे॒वैवे न्द्रि॒य मि॑न्द्रि॒य मे॒व तेन॒ तेनै॒वे न्द्रि॒य मि॑न्द्रि॒य मे॒व तेन॑ । \newline
15. ए॒व तेन॒ तेनै॒वैव तेना॒त्मन् ना॒त्मन् तेनै॒वैव तेना॒त्मन्न् । \newline
16. तेना॒त्मन् ना॒त्मन् तेन॒ तेना॒त्मन् ध॑त्ते धत्त आ॒त्मन् तेन॒ तेना॒त्मन् ध॑त्ते । \newline
17. आ॒त्मन् ध॑त्ते धत्त आ॒त्मन् ना॒त्मन् ध॑त्ते॒ त्रय॑स्त्रिꣳशत्कपाल॒म् त्रय॑स्त्रिꣳशत्कपालम् धत्त आ॒त्मन् ना॒त्मन् ध॑त्ते॒ त्रय॑स्त्रिꣳशत्कपालम् । \newline
18. ध॒त्ते॒ त्रय॑स्त्रिꣳशत्कपाल॒म् त्रय॑स्त्रिꣳशत्कपालम् धत्ते धत्ते॒ त्रय॑स्त्रिꣳशत्कपालम् पुरो॒डाश॑म् पुरो॒डाश॒म् त्रय॑स्त्रिꣳशत्कपालम् धत्ते धत्ते॒ त्रय॑स्त्रिꣳशत्कपालम् पुरो॒डाश᳚म् । \newline
19. त्रय॑स्त्रिꣳशत्कपालम् पुरो॒डाश॑म् पुरो॒डाश॒म् त्रय॑स्त्रिꣳशत्कपाल॒म् त्रय॑स्त्रिꣳशत्कपालम् पुरो॒डाश॒म् निर् णिष् पु॑रो॒डाश॒म् त्रय॑स्त्रिꣳशत्कपाल॒म् त्रय॑स्त्रिꣳशत्कपालम् पुरो॒डाश॒म् निः । \newline
20. त्रय॑स्त्रिꣳशत्कपाल॒मिति॒ त्रय॑स्त्रिꣳशत् - क॒पा॒ल॒म् । \newline
21. पु॒रो॒डाश॒म् निर् णिष् पु॑रो॒डाश॑म् पुरो॒डाश॒म् निर् व॑पति वपति॒ निष् पु॑रो॒डाश॑म् पुरो॒डाश॒म् निर् व॑पति । \newline
22. निर् व॑पति वपति॒ निर् णिर् व॑पति॒ त्रय॑स्त्रिꣳश॒त् त्रय॑स्त्रिꣳशद् वपति॒ निर् णिर् व॑पति॒ त्रय॑स्त्रिꣳशत् । \newline
23. व॒प॒ति॒ त्रय॑स्त्रिꣳश॒त् त्रय॑स्त्रिꣳशद् वपति वपति॒ त्रय॑स्त्रिꣳश॒द् वै वै त्रय॑स्त्रिꣳशद् वपति वपति॒ त्रय॑स्त्रिꣳश॒द् वै । \newline
24. त्रय॑स्त्रिꣳश॒द् वै वै त्रय॑स्त्रिꣳश॒त् त्रय॑स्त्रिꣳश॒द् वै दे॒वता॑ दे॒वता॒ वै त्रय॑स्त्रिꣳश॒त् त्रय॑स्त्रिꣳश॒द् वै दे॒वताः᳚ । \newline
25. त्रय॑स्त्रिꣳश॒दिति॒ त्रयः॑ - त्रिꣳ॒॒श॒त् । \newline
26. वै दे॒वता॑ दे॒वता॒ वै वै दे॒वता॒ स्ता स्ता दे॒वता॒ वै वै दे॒वता॒ स्ताः । \newline
27. दे॒वता॒ स्ता स्ता दे॒वता॑ दे॒वता॒ स्ता ए॒वैव ता दे॒वता॑ दे॒वता॒ स्ता ए॒व । \newline
28. ता ए॒वैव ता स्ता ए॒व यज॑मानो॒ यज॑मान ए॒व ता स्ता ए॒व यज॑मानः । \newline
29. ए॒व यज॑मानो॒ यज॑मान ए॒वैव यज॑मान आ॒त्मन् ना॒त्मन्. यज॑मान ए॒वैव यज॑मान आ॒त्मन्न् । \newline
30. यज॑मान आ॒त्मन् ना॒त्मन्. यज॑मानो॒ यज॑मान आ॒त्मन् नन्वन्वा॒त्मन्. यज॑मानो॒ यज॑मान आ॒त्मन् ननु॑ । \newline
31. आ॒त्मन् नन्वन्वा॒त्मन् ना॒त्मन् ननु॑ स॒मारं॑भयते स॒मारं॑भय॒ते ऽन्वा॒त्मन् ना॒त्मन् ननु॑ स॒मारं॑भयते । \newline
32. अनु॑ स॒मारं॑भयते स॒मारं॑भय॒ते ऽन्वनु॑ स॒मारं॑भयते॒ भूत्यै॒ भूत्यै॑ स॒मारं॑भय॒ते ऽन्वनु॑ स॒मारं॑भयते॒ भूत्यै᳚ । \newline
33. स॒मारं॑भयते॒ भूत्यै॒ भूत्यै॑ स॒मारं॑भयते स॒मारं॑भयते॒ भूत्यै॒ सा सा भूत्यै॑ स॒मारं॑भयते स॒मारं॑भयते॒ भूत्यै॒ सा । \newline
34. स॒मारं॑भयत॒ इति॑ सं - आरं॑भयते । \newline
35. भूत्यै॒ सा सा भूत्यै॒ भूत्यै॒ सा वै वै सा भूत्यै॒ भूत्यै॒ सा वै । \newline
36. सा वै वै सा सा वा ए॒षैषा वै सा सा वा ए॒षा । \newline
37. वा ए॒षैषा वै वा ए॒षा विजि॑ति॒र् विजि॑ति रे॒षा वै वा ए॒षा विजि॑तिः । \newline
38. ए॒षा विजि॑ति॒र् विजि॑ति रे॒षैषा विजि॑ति॒र् नाम॒ नाम॒ विजि॑ति रे॒षैषा विजि॑ति॒र् नाम॑ । \newline
39. विजि॑ति॒र् नाम॒ नाम॒ विजि॑ति॒र् विजि॑ति॒र् नामे ष्टि॒रिष्टि॒र् नाम॒ विजि॑ति॒र् विजि॑ति॒र् नामे ष्टिः॑ । \newline
40. विजि॑ति॒रिति॒ वि - जि॒तिः॒ । \newline
41. नामे ष्टि॒रिष्टि॒र् नाम॒ नामे ष्टि॒र् यो य इष्टि॒र् नाम॒ नामे ष्टि॒र् यः । \newline
42. इष्टि॒र् यो य इष्टि॒ रिष्टि॒र् य ए॒व मे॒वं ॅय इष्टि॒ रिष्टि॒र् य ए॒वम् । \newline
43. य ए॒व मे॒वं ॅयो य ए॒वं ॅवि॒द्वान्. वि॒द्वा ने॒वं ॅयो य ए॒वं ॅवि॒द्वान् । \newline
44. ए॒वं ॅवि॒द्वान्. वि॒द्वा ने॒व मे॒वं ॅवि॒द्वा ने॒तयै॒तया॑ वि॒द्वा ने॒व मे॒वं ॅवि॒द्वा ने॒तया᳚ । \newline
45. वि॒द्वा ने॒तयै॒तया॑ वि॒द्वान्. वि॒द्वा ने॒त येष्ट्ये ष्ट्यै॒तया॑ वि॒द्वान्. वि॒द्वा ने॒तयेष्ट्या᳚ । \newline
46. ए॒त येष्ट्ये ष्ट्यै॒त यै॒तयेष्ट्या॒ यज॑ते॒ यज॑त॒ इष्ट्यै॒त यै॒तयेष्ट्या॒ यज॑ते । \newline
47. इष्ट्या॒ यज॑ते॒ यज॑त॒ इष्ट्येष्ट्या॒ यज॑त उत्त॒मा मु॑त्त॒मां ॅयज॑त॒ इष्ट्येष्ट्या॒ यज॑त उत्त॒माम् । \newline
48. यज॑त उत्त॒मा मु॑त्त॒मां ॅयज॑ते॒ यज॑त उत्त॒मा मे॒वैवोत्त॒मां ॅयज॑ते॒ यज॑त उत्त॒मा मे॒व । \newline
49. उ॒त्त॒मा मे॒वैवोत्त॒मा मु॑त्त॒मा मे॒व विजि॑तिं॒ ॅविजि॑ति मे॒वोत्त॒मा मु॑त्त॒मा मे॒व विजि॑तिम् । \newline
50. उ॒त्त॒मामित्यु॑त् - त॒माम् । \newline
51. ए॒व विजि॑तिं॒ ॅविजि॑ति मे॒वैव विजि॑ति॒म् भ्रातृ॑व्येण॒ भ्रातृ॑व्येण॒ विजि॑ति मे॒वैव विजि॑ति॒म् भ्रातृ॑व्येण । \newline
52. विजि॑ति॒म् भ्रातृ॑व्येण॒ भ्रातृ॑व्येण॒ विजि॑तिं॒ ॅविजि॑ति॒म् भ्रातृ॑व्येण॒ वि वि भ्रातृ॑व्येण॒ विजि॑तिं॒ ॅविजि॑ति॒म् भ्रातृ॑व्येण॒ वि । \newline
53. विजि॑ति॒मिति॒ वि - जि॒ति॒म् । \newline
54. भ्रातृ॑व्येण॒ वि वि भ्रातृ॑व्येण॒ भ्रातृ॑व्येण॒ वि ज॑यते जयते॒ वि भ्रातृ॑व्येण॒ भ्रातृ॑व्येण॒ वि ज॑यते । \newline
55. वि ज॑यते जयते॒ वि वि ज॑यते । \newline
56. ज॒य॒त॒ इति॑ जयते । \newline
\pagebreak
\markright{ TS 2.4.3.1  \hfill https://www.vedavms.in \hfill}

\section{ TS 2.4.3.1 }

\textbf{TS 2.4.3.1 } \newline
\textbf{Samhita Paata} \newline

दे॒वा॒सु॒राः संॅय॑त्ता आस॒न् तेषां᳚ गाय॒त्र्योजो॒ बल॑मिन्द्रि॒यं ॅवी॒र्यं॑ प्र॒जां प॒शून्थ् स॒गृंह्या॒ *ऽऽदाया॑-प॒क्रम्या॑तिष्ठ॒त् ते॑ऽमन्यन्त यत॒रान्. वा इ॒यमु॑पाव॒र्थ्स्यति॒ त इ॒दं भ॑विष्य॒न्तीति॒ तां ॅव्य॑ह्वयन्त॒ विश्व॑कर्म॒न्निति॑ दे॒वा दाभीत्यसु॑राः॒ सा नान्य॑त॒राꣳश्च॒-नोपाव॑र्तत॒ ते दे॒वा ए॒तद्-यजु॑रपश्य॒न्नोजो॑ऽसि॒ सहो॑ऽसि॒ बल॑मसि॒ - [  ] \newline

\textbf{Pada Paata} \newline

दे॒वा॒सु॒रा इति॑ देव - अ॒सु॒राः । संॅय॑त्ता॒ इति॒ सं - य॒त्ताः॒ । आ॒स॒न्न् । तेषा᳚म् । गा॒य॒त्री । ओजः॑ । बल᳚म् । इ॒न्द्रि॒यम् । वी॒र्य᳚म् । प्र॒जामिति॑ प्र - जाम् । प॒शून् । स॒गृंह्येति॑ सं - गृह्य॑ । आ॒दायेत्या᳚ - दाय॑ । अ॒प॒क्रम्येत्य॑प - क्रम्य॑ । अ॒ति॒ष्ठ॒त् । ते । अ॒म॒न्य॒न्त॒ । य॒त॒रान् । वै । इ॒यम् । उ॒पा॒व॒र्थ्स्यतीत्यु॑प - आ॒व॒र्थ्स्यति॑ । ते । इ॒दम् । भ॒वि॒ष्य॒न्ति॒ । इति॑ । ताम् । वीति॑ । अ॒ह्व॒य॒न्त॒ । विश्व॑कर्म॒न्निति॒ विश्व॑ - क॒र्म॒न्न् । इति॑ । दे॒वाः । दाभि॑ । इति॑ । असु॑राः । सा । न । अ॒न्य॒त॒रान् । च॒न । उ॒पाव॑र्त॒तेत्यु॑प - आव॑र्तत । ते । दे॒वाः । ए॒तत् । यजुः॑ । अ॒प॒श्य॒न्न् । ओजः॑ । अ॒सि॒ । सहः॑ । अ॒सि॒ । बल᳚म् । अ॒सि॒ ।  \newline


\textbf{Krama Paata} \newline

दे॒वा॒सु॒राः सम्ॅय॑त्ताः । दे॒वा॒सु॒रा इति॑ देव - अ॒सु॒राः । सम्ॅय॑त्ता आसन्न् । सम्ॅय॑त्ता॒ इति॒ सम् - य॒त्ताः॒ । आ॒स॒न् तेषा᳚म् । तेषा᳚म् गाय॒त्री । गा॒य॒त्र्योजः॑ । ओजो॒ बल᳚म् । बल॑मिन्द्रि॒यम् । इ॒न्द्रि॒यं ॅवी॒र्य᳚म् । वी॒र्य॑म् प्र॒जाम् । प्र॒जाम् प॒शून् । प्र॒जामिति॑ प्र - जाम् । प॒शून्थ् स॒ङ्गृह्य॑ । स॒ङ्गृह्या॒दाय॑ । स॒ङ्गृह्येति॑ सम् - गृह्य॑ । आ॒दाया॑प॒क्रम्य॑ । आ॒दायेत्या᳚ - दाय॑ । अ॒प॒क्रम्या॑तिष्ठत् । अ॒प॒क्रम्येत्य॑प - क्रम्य॑ । अ॒ति॒ष्ठ॒त् ते । ते॑ ऽमन्यन्त । अ॒म॒न्य॒न्त॒ य॒त॒रान् । य॒त॒रान्. वै । वा इ॒यम् । इ॒यमु॑पाव॒र्त्थ्स्यति॑ । उ॒पा॒व॒र्त्थ्स्यति॒ ते । उ॒पा॒व॒र्त्थ्स्यतीत्यु॑प - आ॒व॒र्त्थ्स्यति॑ । त इ॒दम् । इ॒दं भ॑विष्यन्ति । भ॒वि॒ष्य॒न्तीति॑ । इति॒ ताम् । तां ॅवि । व्य॑ह्वयन्त । अ॒ह्व॒य॒न्त॒ विश्व॑कर्मन्न् । विश्व॑कर्म॒न्निति॑ । विश्व॑कर्म॒न्निति॒ विश्व॑ - क॒र्म॒न्न्॒ । इति॑ दे॒वाः । दे॒वा दाभि॑ । दाभीति॑ । इत्यसु॑राः । असु॑राः॒ सा । सा न । नान्य॑त॒रान् । अ॒न्य॒त॒राꣳ श्च॒न । च॒नोपाव॑र्तत । उ॒पाव॑र्तत॒ ते । उ॒पाव॑र्त॒तेत्यु॑प - आव॑र्तत । ते दे॒वाः । दे॒वा ए॒तत् । ए॒तद् यजुः॑ । यजु॑रपश्यन्न् । अ॒प॒श्य॒न्नोजः॑ । ओजो॑ऽसि । अ॒सि॒ सहः॑ । सहो॑ऽसि । अ॒सि॒ बल᳚म् । बल॑मसि । अ॒सि॒ भ्राजः॑ \newline

\textbf{Jatai Paata} \newline

1. दे॒वा॒सु॒राः संॅय॑त्ताः॒ संॅय॑त्ता देवासु॒रा दे॑वासु॒राः संॅय॑त्ताः । \newline
2. दे॒वा॒सु॒रा इति॑ देव - अ॒सु॒राः । \newline
3. संॅय॑त्ता आसन् नास॒न् थ्संॅय॑त्ताः॒ संॅय॑त्ता आसन्न् । \newline
4. संॅय॑त्ता॒ इति॒ सं - य॒त्ताः॒ । \newline
5. आ॒स॒न् तेषा॒म् तेषा॑ मासन् नास॒न् तेषा᳚म् । \newline
6. तेषा᳚म् गाय॒त्री गा॑य॒त्री तेषा॒म् तेषा᳚म् गाय॒त्री । \newline
7. गा॒य॒त्र्योज॒ ओजो॑ गाय॒त्री गा॑य॒त्र्योजः॑ । \newline
8. ओजो॒ बल॒म् बल॒ मोज॒ ओजो॒ बल᳚म् । \newline
9. बल॑ मिन्द्रि॒य मि॑न्द्रि॒यम् बल॒म् बल॑ मिन्द्रि॒यम् । \newline
10. इ॒न्द्रि॒यं ॅवी॒र्यं॑ ॅवी॒र्य॑ मिन्द्रि॒य मि॑न्द्रि॒यं ॅवी॒र्य᳚म् । \newline
11. वी॒र्य॑म् प्र॒जाम् प्र॒जां ॅवी॒र्यं॑ ॅवी॒र्य॑म् प्र॒जाम् । \newline
12. प्र॒जाम् प॒शून् प॒शून् प्र॒जाम् प्र॒जाम् प॒शून् । \newline
13. प्र॒जामिति॑ प्र - जाम् । \newline
14. प॒शून् थ्स॒ङ्गृह्य॑ स॒ङ्गृह्य॑ प॒शून् प॒शून् थ्स॒ङ्गृह्य॑ । \newline
15. स॒ङ्गृह्या॒ दाया॒ दाय॑ स॒ङ्गृह्य॑ स॒ङ्गृह्या॒ दाय॑ । \newline
16. स॒ङ्गृह्येति॑ सं - गृह्य॑ । \newline
17. आ॒दाया॑ प॒क्रम्या॑ प॒क्रम्या॒ दाया॒ दाया॑ प॒क्रम्य॑ । \newline
18. आ॒दायेत्या᳚ - दाय॑ । \newline
19. अ॒प॒क्रम्या॑ तिष्ठ दतिष्ठ दप॒क्रम्या॑ प॒क्रम्या॑ तिष्ठत् । \newline
20. अ॒प॒क्रम्येत्य॑प - क्रम्य॑ । \newline
21. अ॒ति॒ष्ठ॒त् ते ते॑ ऽतिष्ठ दतिष्ठ॒त् ते । \newline
22. ते॑ ऽमन्यन्ता मन्यन्त॒ ते ते॑ ऽमन्यन्त । \newline
23. अ॒म॒न्य॒न्त॒ य॒त॒रान्. य॑त॒रा न॑मन्यन्ता मन्यन्त यत॒रान् । \newline
24. य॒त॒रान्. वै वै य॑त॒रान्. य॑त॒रान्. वै । \newline
25. वा इ॒य मि॒यं ॅवै वा इ॒यम् । \newline
26. इ॒य मु॑पाव॒र्थ्स्य त्यु॑पाव॒र्थ्स्यती॒य मि॒य मु॑पाव॒र्थ्स्यति॑ । \newline
27. उ॒पा॒व॒र्थ्स्यति॒ ते त उ॑पाव॒र्थ्स्य त्यु॑पाव॒र्थ्स्यति॒ ते । \newline
28. उ॒पा॒व॒र्थ्स्यतीत्यु॑प - आ॒व॒र्थ्स्यति॑ । \newline
29. त इ॒द मि॒दम् ते त इ॒दम् । \newline
30. इ॒दम् भ॑विष्यन्ति भविष्यन्ती॒द मि॒दम् भ॑विष्यन्ति । \newline
31. भ॒वि॒ष्य॒न्तीतीति॑ भविष्यन्ति भविष्य॒न्तीति॑ । \newline
32. इति॒ ताम् ता मितीति॒ ताम् । \newline
33. तां ॅवि वि ताम् तां ॅवि । \newline
34. व्य॑ह्वयन्ता ह्वयन्त॒ वि व्य॑ह्वयन्त । \newline
35. अ॒ह्व॒य॒न्त॒ विश्व॑कर्म॒न्॒. विश्व॑कर्मन् नह्वयन्ता ह्वयन्त॒ विश्व॑कर्मन्न् । \newline
36. विश्व॑कर्म॒न् नितीति॒ विश्व॑कर्म॒न्॒. विश्व॑कर्म॒न् निति॑ । \newline
37. विश्व॑कर्म॒न्निति॒ विश्व॑ - क॒र्म॒न्न् । \newline
38. इति॑ दे॒वा दे॒वा इतीति॑ दे॒वाः । \newline
39. दे॒वा दाभि॒ दाभि॑ दे॒वा दे॒वा दाभि॑ । \newline
40. दाभीतीति॒ दाभि॒ दाभीति॑ । \newline
41. इत्यसु॑रा॒ असु॑रा॒ इती त्यसु॑राः । \newline
42. असु॑राः॒ सा सा ऽसु॑रा॒ असु॑राः॒ सा । \newline
43. सा न न सा सा न । \newline
44. नान्य॑त॒रा न॑न्यत॒रान् न नान्य॑त॒रान् । \newline
45. अ॒न्य॒त॒राꣳ श्च॒न च॒नान्य॑त॒रा न॑न्यत॒राꣳ श्च॒न । \newline
46. च॒नोपाव॑र्ततो॒ पाव॑र्तत च॒न च॒नोपाव॑र्तत । \newline
47. उ॒पाव॑र्तत॒ ते त उ॒पाव॑र्ततो॒ पाव॑र्तत॒ ते । \newline
48. उ॒पाव॑र्त॒तेत्यु॑प - आव॑र्तत । \newline
49. ते दे॒वा दे॒वा स्ते ते दे॒वाः । \newline
50. दे॒वा ए॒त दे॒तद् दे॒वा दे॒वा ए॒तत् । \newline
51. ए॒तद् यजु॒र् यजु॑ रे॒त दे॒तद् यजुः॑ । \newline
52. यजु॑ रपश्यन् नपश्य॒न्॒. यजु॒र् यजु॑ रपश्यन्न् । \newline
53. अ॒प॒श्य॒न् नोज॒ ओजो॑ ऽपश्यन् नपश्य॒न् नोजः॑ । \newline
54. ओजो᳚ ऽस्य॒स्योज॒ ओजो॑ ऽसि । \newline
55. अ॒सि॒ सहः॒ सहो᳚ ऽस्यसि॒ सहः॑ । \newline
56. सहो᳚ ऽस्यसि॒ सहः॒ सहो॑ ऽसि । \newline
57. अ॒सि॒ बल॒म् बल॑ मस्यसि॒ बल᳚म् । \newline
58. बल॑ मस्यसि॒ बल॒म् बल॑ मसि । \newline
59. अ॒सि॒ भ्राजो॒ भ्राजो᳚ ऽस्यसि॒ भ्राजः॑ । \newline

\textbf{Ghana Paata } \newline

1. दे॒वा॒सु॒राः संॅय॑त्ताः॒ संॅय॑त्ता देवासु॒रा दे॑वासु॒राः संॅय॑त्ता आसन् नास॒न् थ्संॅय॑त्ता देवासु॒रा दे॑वासु॒राः संॅय॑त्ता आसन्न् । \newline
2. दे॒वा॒सु॒रा इति॑ देव - अ॒सु॒राः । \newline
3. संॅय॑त्ता आसन् नास॒न् थ्संॅय॑त्ताः॒ संॅय॑त्ता आस॒न् तेषा॒म् तेषा॑ मास॒न् थ्संॅय॑त्ताः॒ संॅय॑त्ता आस॒न् तेषा᳚म् । \newline
4. संॅय॑त्ता॒ इति॒ सं - य॒त्ताः॒ । \newline
5. आ॒स॒न् तेषा॒म् तेषा॑ मासन् नास॒न् तेषा᳚म् गाय॒त्री गा॑य॒त्री तेषा॑ मासन् नास॒न् तेषा᳚म् गाय॒त्री । \newline
6. तेषा᳚म् गाय॒त्री गा॑य॒त्री तेषा॒म् तेषा᳚म् गाय॒त्र्योज॒ ओजो॑ गाय॒त्री तेषा॒म् तेषा᳚म् गाय॒त्र्योजः॑ । \newline
7. गा॒य॒त्र्योज॒ ओजो॑ गाय॒त्री गा॑य॒त्र्योजो॒ बल॒म् बल॒ मोजो॑ गाय॒त्री गा॑य॒त्र्योजो॒ बल᳚म् । \newline
8. ओजो॒ बल॒म् बल॒ मोज॒ ओजो॒ बल॑ मिन्द्रि॒य मि॑न्द्रि॒यम् बल॒ मोज॒ ओजो॒ बल॑ मिन्द्रि॒यम् । \newline
9. बल॑ मिन्द्रि॒य मि॑न्द्रि॒यम् बल॒म् बल॑ मिन्द्रि॒यं ॅवी॒र्यं॑ ॅवी॒र्य॑ मिन्द्रि॒यम् बल॒म् बल॑ मिन्द्रि॒यं ॅवी॒र्य᳚म् । \newline
10. इ॒न्द्रि॒यं ॅवी॒र्यं॑ ॅवी॒र्य॑ मिन्द्रि॒य मि॑न्द्रि॒यं ॅवी॒र्य॑म् प्र॒जाम् प्र॒जां ॅवी॒र्य॑ मिन्द्रि॒य मि॑न्द्रि॒यं ॅवी॒र्य॑म् प्र॒जाम् । \newline
11. वी॒र्य॑म् प्र॒जाम् प्र॒जां ॅवी॒र्यं॑ ॅवी॒र्य॑म् प्र॒जाम् प॒शून् प॒शून् प्र॒जां ॅवी॒र्यं॑ ॅवी॒र्य॑म् प्र॒जाम् प॒शून् । \newline
12. प्र॒जाम् प॒शून् प॒शून् प्र॒जाम् प्र॒जाम् प॒शून् थ्स॒ङ्गृह्य॑ स॒ङ्गृह्य॑ प॒शून् प्र॒जाम् प्र॒जाम् प॒शून् थ्स॒ङ्गृह्य॑ । \newline
13. प्र॒जामिति॑ प्र - जाम् । \newline
14. प॒शून् थ्स॒ङ्गृह्य॑ स॒ङ्गृह्य॑ प॒शून् प॒शून् थ्स॒ङ्गृह्या॒ दाया॒दाय॑ स॒ङ्गृह्य॑ प॒शून् प॒शून् थ्स॒ङ्गृह्या॒दाय॑ । \newline
15. स॒ङ्गृह्या॒ दाया॒दाय॑ स॒ङ्गृह्य॑ स॒ङ्गृह्या॒ दाया॑ प॒क्रम्या॑ प॒क्रम्या॒ दाय॑ स॒ङ्गृह्य॑ स॒ङ्गृह्या॒ दाया॑ प॒क्रम्य॑ । \newline
16. स॒ङ्गृह्येति॑ सं - गृह्य॑ । \newline
17. आ॒दाया॑ प॒क्रम्या॑ प॒क्रम्या॒ दाया॒ दाया॑ प॒क्रम्या॑ तिष्ठ दतिष्ठ दप॒क्रम्या॒ दाया॒ दाया॑ प॒क्रम्या॑तिष्ठत् । \newline
18. आ॒दायेत्या᳚ - दाय॑ । \newline
19. अ॒प॒क्रम्या॑ तिष्ठ दतिष्ठ दप॒क्रम्या॑ प॒क्रम्या॑ तिष्ठ॒त् ते ते॑ ऽतिष्ठ दप॒क्रम्या॑ प॒क्रम्या॑ तिष्ठ॒त् ते । \newline
20. अ॒प॒क्रम्येत्य॑प - क्रम्य॑ । \newline
21. अ॒ति॒ष्ठ॒त् ते ते॑ ऽतिष्ठ दतिष्ठ॒त् ते॑ ऽमन्यन्ता मन्यन्त॒ ते॑ ऽतिष्ठ दतिष्ठ॒त् ते॑ ऽमन्यन्त । \newline
22. ते॑ ऽमन्यन्ता मन्यन्त॒ ते ते॑ ऽमन्यन्त यत॒रान्. य॑त॒रा न॑मन्यन्त॒ ते ते॑ ऽमन्यन्त यत॒रान् । \newline
23. अ॒म॒न्य॒न्त॒ य॒त॒रान्. य॑त॒रा न॑मन्यन्ता मन्यन्त यत॒रान्. वै वै य॑त॒रा न॑मन्यन्ता मन्यन्त यत॒रान्. वै । \newline
24. य॒त॒रान्. वै वै य॑त॒रान्. य॑त॒रान्. वा इ॒य मि॒यं ॅवै य॑त॒रान्. य॑त॒रान्. वा इ॒यम् । \newline
25. वा इ॒य मि॒यं ॅवै वा इ॒य मु॑पाव॒र्थ्स्य त्यु॑पाव॒र्थ्स्यती॒यं ॅवै वा इ॒य मु॑पाव॒र्थ्स्यति॑ । \newline
26. इ॒य मु॑पाव॒र्थ्स्य त्यु॑पाव॒र्थ्स्यती॒य मि॒य मु॑पाव॒र्थ्स्यति॒ ते त उ॑पाव॒र्थ्स्यती॒य मि॒य मु॑पाव॒र्थ्स्यति॒ ते । \newline
27. उ॒पा॒व॒र्थ्स्यति॒ ते त उ॑पाव॒र्थ्स्य त्यु॑पाव॒र्थ्स्यति॒ त इ॒द मि॒दम् त उ॑पाव॒र्थ्स्य त्यु॑पाव॒र्थ्स्यति॒ त इ॒दम् । \newline
28. उ॒पा॒व॒र्थ्स्यतीत्यु॑प - आ॒व॒र्थ्स्यति॑ । \newline
29. त इ॒द मि॒दम् ते त इ॒दम् भ॑विष्यन्ति भविष्यन्ती॒दम् ते त इ॒दम् भ॑विष्यन्ति । \newline
30. इ॒दम् भ॑विष्यन्ति भविष्यन्ती॒द मि॒दम् भ॑विष्य॒न्तीतीति॑ भविष्यन्ती॒द मि॒दम् भ॑विष्य॒न्तीति॑ । \newline
31. भ॒वि॒ष्य॒न्तीतीति॑ भविष्यन्ति भविष्य॒न्तीति॒ ताम् ता मिति॑ भविष्यन्ति भविष्य॒न्तीति॒ ताम् । \newline
32. इति॒ ताम् ता मितीति॒ तां ॅवि वि ता मितीति॒ तां ॅवि । \newline
33. तां ॅवि वि ताम् तां ॅव्य॑ह्वयन्ता ह्वयन्त॒ वि ताम् तां ॅव्य॑ह्वयन्त । \newline
34. व्य॑ह्वयन्ता ह्वयन्त॒ वि व्य॑ह्वयन्त॒ विश्व॑कर्म॒न्॒. विश्व॑कर्मन् नह्वयन्त॒ वि व्य॑ह्वयन्त॒ विश्व॑कर्मन्न् । \newline
35. अ॒ह्व॒य॒न्त॒ विश्व॑कर्म॒न्॒. विश्व॑कर्मन् नह्वयन्ता ह्वयन्त॒ विश्व॑कर्म॒न् नितीति॒ विश्व॑कर्मन् नह्वयन्ता ह्वयन्त॒ विश्व॑कर्म॒न् निति॑ । \newline
36. विश्व॑कर्म॒न् नितीति॒ विश्व॑कर्म॒न्॒. विश्व॑कर्म॒न् निति॑ दे॒वा दे॒वा इति॒ विश्व॑कर्म॒न्॒. विश्व॑कर्म॒न् निति॑ दे॒वाः । \newline
37. विश्व॑कर्म॒न्निति॒ विश्व॑ - क॒र्म॒न्न् । \newline
38. इति॑ दे॒वा दे॒वा इतीति॑ दे॒वा दाभि॒ दाभि॑ दे॒वा इतीति॑ दे॒वा दाभि॑ । \newline
39. दे॒वा दाभि॒ दाभि॑ दे॒वा दे॒वा दाभीतीति॒ दाभि॑ दे॒वा दे॒वा दाभीति॑ । \newline
40. दाभीतीति॒ दाभि॒ दाभी त्यसु॑रा॒ असु॑रा॒ इति॒ दाभि॒ दाभी त्यसु॑राः । \newline
41. इत्यसु॑रा॒ असु॑रा॒ इती त्यसु॑राः॒ सा सा ऽसु॑रा॒ इती त्यसु॑राः॒ सा । \newline
42. असु॑राः॒ सा सा ऽसु॑रा॒ असु॑राः॒ सा न न सा ऽसु॑रा॒ असु॑राः॒ सा न । \newline
43. सा न न सा सा नान्य॑त॒रा न॑न्यत॒रान् न सा सा नान्य॑त॒रान् । \newline
4. नान्य॑त॒रा न॑न्यत॒रान् न नान्य॑त॒राꣳ श्च॒न च॒नान्य॑त॒रान् न नान्य॑त॒राꣳ श्च॒न । \newline
45. अ॒न्य॒त॒राꣳ श्च॒न च॒नान्य॑त॒रा न॑न्यत॒राꣳ श्च॒नो पाव॑र्ततो॒ पाव॑र्तत च॒नान्य॑त॒रा न॑न्यत॒राꣳ श्च॒नोपाव॑र्तत । \newline
46. च॒नोपाव॑र्ततो॒ पाव॑र्तत च॒न च॒नोपाव॑र्तत॒ ते त उ॒पाव॑र्तत च॒न च॒नोपाव॑र्तत॒ ते । \newline
47. उ॒पाव॑र्तत॒ ते त उ॒पाव॑र्ततो॒ पाव॑र्तत॒ ते दे॒वा दे॒वा स्त उ॒पाव॑र्ततो॒ पाव॑र्तत॒ ते दे॒वाः । \newline
48. उ॒पाव॑र्त॒तेत्यु॑प - आव॑र्तत । \newline
49. ते दे॒वा दे॒वा स्ते ते दे॒वा ए॒त दे॒तद् दे॒वा स्ते ते दे॒वा ए॒तत् । \newline
50. दे॒वा ए॒त दे॒तद् दे॒वा दे॒वा ए॒तद् यजु॒र् यजु॑ रे॒तद् दे॒वा दे॒वा ए॒तद् यजुः॑ । \newline
51. ए॒तद् यजु॒र् यजु॑ रे॒त दे॒तद् यजु॑ रपश्यन् नपश्य॒न्॒. यजु॑ रे॒त दे॒तद् यजु॑ रपश्यन्न् । \newline
52. यजु॑ रपश्यन् नपश्य॒न्॒. यजु॒र् यजु॑ रपश्य॒न् नोज॒ ओजो॑ ऽपश्य॒न्॒. यजु॒र् यजु॑ रपश्य॒न् नोजः॑ । \newline
53. अ॒प॒श्य॒न् नोज॒ ओजो॑ ऽपश्यन् नपश्य॒न् नोजो᳚ ऽस्य॒स्योजो॑ ऽपश्यन् नपश्य॒न् नोजो॑ ऽसि । \newline
54. ओजो᳚ ऽस्य॒स्योज॒ ओजो॑ ऽसि॒ सहः॒ सहो॒ ऽस्योज॒ ओजो॑ ऽसि॒ सहः॑ । \newline
55. अ॒सि॒ सहः॒ सहो᳚ ऽस्यसि॒ सहो᳚ ऽस्यसि॒ सहो᳚ ऽस्यसि॒ सहो॑ ऽसि । \newline
56. सहो᳚ ऽस्यसि॒ सहः॒ सहो॑ ऽसि॒ बल॒म् बल॑ मसि॒ सहः॒ सहो॑ ऽसि॒ बल᳚म् । \newline
57. अ॒सि॒ बल॒म् बल॑ मस्यसि॒ बल॑ मस्यसि॒ बल॑ मस्यसि॒ बल॑ मसि । \newline
58. बल॑ मस्यसि॒ बल॒म् बल॑ मसि॒ भ्राजो॒ भ्राजो॑ ऽसि॒ बल॒म् बल॑ मसि॒ भ्राजः॑ । \newline
59. अ॒सि॒ भ्राजो॒ भ्राजो᳚ ऽस्यसि॒ भ्राजो᳚ ऽस्यसि॒ भ्राजो᳚ ऽस्यसि॒ भ्राजो॑ ऽसि । \newline
\pagebreak
\markright{ TS 2.4.3.2  \hfill https://www.vedavms.in \hfill}

\section{ TS 2.4.3.2 }

\textbf{TS 2.4.3.2 } \newline
\textbf{Samhita Paata} \newline

भ्राजो॑ऽसि दे॒वानां॒ धाम॒ नामा॑ऽसि॒ विश्व॑मसि वि॒श्वायुः॒ सर्व॑मसि स॒र्वायु॑रभि॒भूरिति॒ वाव दे॒वा असु॑राणा॒मोजो॒ बल॑मिन्द्रि॒यं ॅवी॒र्यं॑ प्र॒जां प॒शून॑वृञ्जत॒ यद्-गा॑य॒त्र्य॑प॒क्रम्याति॑ष्ठ॒त् तस्मा॑दे॒तां गा॑य॒त्रीतीष्टि॑माहुः संॅवथ्स॒रो वै गा॑य॒त्री सं॑ॅवथ्स॒रो वै तद॑प॒क्रम्या॑तिष्ठ॒द्-यदे॒तया॑ दे॒वा असु॑राणा॒मोजो॒ बल॑मिन्द्रि॒यं ॅवी॒र्यं॑ - [  ] \newline

\textbf{Pada Paata} \newline

भ्राजः॑ । अ॒सि॒ । दे॒वाना᳚म् । धाम॑ । नाम॑ । अ॒सि॒ । विश्व᳚म् । अ॒सि॒ । वि॒श्वायु॒रिति॑ वि॒श्व - आ॒युः॒ । सर्व᳚म् । अ॒सि॒ । स॒र्वायु॒रिति॑ स॒र्व - आ॒युः॒ । अ॒भि॒भूरित्य॑भि - भूः । इति॑ । वाव । दे॒वाः । असु॑राणाम् । ओजः॑ । बल᳚म् । इ॒न्द्रि॒यम् । वी॒र्य᳚म् । प्र॒जामिति॑ प्र-जाम् । प॒शून् । अ॒वृ॒ञ्ज॒त॒ । यत् । गा॒य॒त्री । अ॒प॒क्रम्येत्य॑प - क्रम्य॑ । अति॑ष्ठत् । तस्मा᳚त् । ए॒ताम् । गा॒य॒त्री । इति॑ । इष्टि᳚म् । आ॒हुः॒ । सं॒ॅव॒थ्स॒र इति॑ सं - व॒थ्स॒रः । वै । गा॒य॒त्री । सं॒ॅव॒थ्स॒र इति॑ सं - व॒थ्स॒रः । वै । तत् । अ॒प॒क्रम्येत्य॑प - क्रम्य॑ । अ॒ति॒ष्ठ॒त् । यत् । ए॒तया᳚ । दे॒वाः । असु॑राणाम् । ओजः॑ । बल᳚म् । इ॒न्द्रि॒यम् । वी॒र्य᳚म् ।  \newline


\textbf{Krama Paata} \newline

भ्राजो॑ऽसि । अ॒सि॒ दे॒वाना᳚म् । दे॒वानां॒ धाम॑ । धाम॒ नाम॑ । नामा॑ऽसि । अ॒सि॒ विश्व᳚म् । विश्व॑मसि । अ॒सि॒ वि॒श्वायुः॑ । वि॒श्वायुः॒ सर्व᳚म् । वि॒श्वायु॒रिति॑ वि॒श्व - आ॒युः॒ । सर्व॑मसि । अ॒सि॒ स॒र्वायुः॑ । स॒र्वायु॑रभि॒भूः । स॒र्वायु॒रिति॑ स॒र्व - आ॒युः॒ । अ॒भि॒भूरिति॑ । अ॒भि॒भूरित्य॑भि - भूः । इति॒ वाव । वाव दे॒वाः । दे॒वा असु॑राणाम् । असु॑राणा॒मोजः॑ । ओजो॒ बल᳚म् । बल॑मिन्द्रि॒यम् । इ॒न्द्रि॒यं ॅवी॒र्य᳚म् । वी॒र्य॑म् प्र॒जाम् । प्र॒जाम् प॒शून् । प्र॒जामिति॑ प्र - जाम् । प॒शून॑वृञ्जत । अ॒वृ॒ञ्ज॒त॒ यत् । यद् गा॑य॒त्री । गा॒य॒त्र्य॑प॒क्रम्य॑ । अ॒प॒क्रम्याति॑ष्ठत् । अ॒प॒क्रम्येत्य॑प - क्रम्य॑ । अति॑ष्ठ॒त् तस्मा᳚त् । तस्मा॑दे॒ताम् । ए॒ताम् गा॑य॒त्री । गा॒य॒त्रीति॑ । इतीष्टि᳚म् । इष्टि॑माहुः । आ॒हुः॒ स॒म्ॅव॒थ्स॒रः । स॒म्ॅव॒थ्स॒रो वै । स॒म्ॅव॒थ्स॒र इति॑ सं - व॒थ्स॒रः । वै गा॑य॒त्री । गा॒य॒त्री स॑म्ॅवथ्स॒रः । स॒म्ॅव॒थ्स॒रो वै । स॒म्ॅव॒थ्स॒र इति॑ सम् - व॒थ्स॒रः । वै तत् । तद॑प॒क्रम्य॑ । अ॒प॒क्रम्या॑तिष्ठत् । अ॒प॒क्रम्येत्य॑प - कम्य॑ । अ॒ति॒ष्ठ॒द् यत् । यदे॒तया᳚ । ए॒तया॑ दे॒वाः । दे॒वा असु॑राणाम् । असु॑राणा॒मोजः॑ । ओजो॒ बल᳚म् । बल॑मिन्द्रि॒यम् । इ॒न्द्रि॒यं ॅवी॒र्य᳚म् ( ) । 
वी॒र्य॑म् प्र॒जाम् \newline

\textbf{Jatai Paata} \newline

1. भ्राजो᳚ ऽस्यसि॒ भ्राजो॒ भ्राजो॑ ऽसि । \newline
2. अ॒सि॒ दे॒वाना᳚म् दे॒वाना॑ मस्यसि दे॒वाना᳚म् । \newline
3. दे॒वाना॒म् धाम॒ धाम॑ दे॒वाना᳚म् दे॒वाना॒म् धाम॑ । \newline
4. धाम॒ नाम॒ नाम॒ धाम॒ धाम॒ नाम॑ । \newline
5. नामा᳚स्यसि॒ नाम॒ नामा॑सि । \newline
6. अ॒सि॒ विश्वं॒ ॅविश्व॑ मस्यसि॒ विश्व᳚म् । \newline
7. विश्व॑ मस्यसि॒ विश्वं॒ ॅविश्व॑ मसि । \newline
8. अ॒सि॒ वि॒श्वायु॑र् वि॒श्वायु॑ रस्यसि वि॒श्वायुः॑ । \newline
9. वि॒श्वायुः॒ सर्वꣳ॒॒ सर्वं॑ ॅवि॒श्वायु॑र् वि॒श्वायुः॒ सर्व᳚म् । \newline
10. वि॒श्वायु॒रिति॑ वि॒श्व - आ॒युः॒ । \newline
11. सर्व॑ मस्यसि॒ सर्वꣳ॒॒ सर्व॑ मसि । \newline
12. अ॒सि॒ स॒र्वायुः॑ स॒र्वायु॑ रस्यसि स॒र्वायुः॑ । \newline
13. स॒र्वायु॑ रभि॒भू र॑भि॒भूः स॒र्वायुः॑ स॒र्वायु॑ रभि॒भूः । \newline
14. स॒र्वायु॒रिति॑ स॒र्व - आ॒युः॒ । \newline
15. अ॒भि॒भू रिती त्य॑भि॒भू र॑भि॒भू रिति॑ । \newline
16. अ॒भि॒भूरित्य॑भि - भूः । \newline
17. इति॒ वाव वावे तीति॒ वाव । \newline
18. वाव दे॒वा दे॒वा वाव वाव दे॒वाः । \newline
19. दे॒वा असु॑राणा॒ मसु॑राणाम् दे॒वा दे॒वा असु॑राणाम् । \newline
20. असु॑राणा॒ मोज॒ ओजो ऽसु॑राणा॒ मसु॑राणा॒ मोजः॑ । \newline
21. ओजो॒ बल॒म् बल॒ मोज॒ ओजो॒ बल᳚म् । \newline
22. बल॑ मिन्द्रि॒य मि॑न्द्रि॒यम् बल॒म् बल॑ मिन्द्रि॒यम् । \newline
23. इ॒न्द्रि॒यं ॅवी॒र्यं॑ ॅवी॒र्य॑ मिन्द्रि॒य मि॑न्द्रि॒यं ॅवी॒र्य᳚म् । \newline
24. वी॒र्य॑म् प्र॒जाम् प्र॒जां ॅवी॒र्यं॑ ॅवी॒र्य॑म् प्र॒जाम् । \newline
25. प्र॒जाम् प॒शून् प॒शून् प्र॒जाम् प्र॒जाम् प॒शून् । \newline
26. प्र॒जामिति॑ प्र - जाम् । \newline
27. प॒शू न॑वृञ्जता वृञ्जत प॒शून् प॒शू न॑वृञ्जत । \newline
28. अ॒वृ॒ञ्ज॒त॒ यद् यद॑वृञ्जता वृञ्जत॒ यत् । \newline
29. यद् गा॑य॒त्री गा॑य॒त्री यद् यद् गा॑य॒त्री । \newline
30. गा॒य॒ त्र्य॑प॒क्रम्या॑ प॒क्रम्य॑ गाय॒त्री गा॑य॒ त्र्य॑प॒क्रम्य॑ । \newline
31. अ॒प॒क्रम्या ति॑ष्ठ॒ दति॑ष्ठ दप॒क्रम्या॑ प॒क्रम्या ति॑ष्ठत् । \newline
32. अ॒प॒क्रम्येत्य॑प - क्रम्य॑ । \newline
33. अति॑ष्ठ॒त् तस्मा॒त् तस्मा॒ दति॑ष्ठ॒ दति॑ष्ठ॒त् तस्मा᳚त् । \newline
34. तस्मा॑ दे॒ता मे॒ताम् तस्मा॒त् तस्मा॑ दे॒ताम् । \newline
35. ए॒ताम् गा॑य॒त्री गा॑य॒ त्र्ये॑ता मे॒ताम् गा॑य॒त्री । \newline
36. गा॒य॒त्रीतीति॑ गाय॒त्री गा॑य॒त्रीति॑ । \newline
37. इतीष्टि॒ मिष्टि॒ मितीतीष्टि᳚म् । \newline
38. इष्टि॑ माहु राहु॒ रिष्टि॒ मिष्टि॑ माहुः । \newline
39. आ॒हुः॒ सं॒ॅव॒थ्स॒रः सं॑ॅवथ्स॒र आ॑हु राहुः संॅवथ्स॒रः । \newline
40. सं॒ॅव॒थ्स॒रो वै वै सं॑ॅवथ्स॒रः सं॑ॅवथ्स॒रो वै । \newline
41. सं॒ॅव॒थ्स॒र इति॑ सं - व॒थ्स॒रः । \newline
42. वै गा॑य॒त्री गा॑य॒त्री वै वै गा॑य॒त्री । \newline
43. गा॒य॒त्री सं॑ॅवथ्स॒रः सं॑ॅवथ्स॒रो गा॑य॒त्री गा॑य॒त्री सं॑ॅवथ्स॒रः । \newline
44. सं॒ॅव॒थ्स॒रो वै वै सं॑ॅवथ्स॒रः सं॑ॅवथ्स॒रो वै । \newline
45. सं॒ॅव॒थ्स॒र इति॑ सं - व॒थ्स॒रः । \newline
46. वै तत् तद् वै वै तत् । \newline
47. तद॑प॒क्रम्या॑ प॒क्रम्य॒ तत् तद॑प॒क्रम्य॑ । \newline
48. अ॒प॒क्रम्या॑ तिष्ठ दतिष्ठ दप॒क्रम्या॑ प॒क्रम्या॑ तिष्ठत् । \newline
49. अ॒प॒क्रम्येत्य॑प - क्रम्य॑ । \newline
50. अ॒ति॒ष्ठ॒द् यद् यद॑तिष्ठ दतिष्ठ॒द् यत् । \newline
51. यदे॒त यै॒तया॒ यद् यदे॒तया᳚ । \newline
52. ए॒तया॑ दे॒वा दे॒वा ए॒त यै॒तया॑ दे॒वाः । \newline
53. दे॒वा असु॑राणा॒ मसु॑राणाम् दे॒वा दे॒वा असु॑राणाम् । \newline
54. असु॑राणा॒ मोज॒ ओजो ऽसु॑राणा॒ मसु॑राणा॒ मोजः॑ । \newline
55. ओजो॒ बल॒म् बल॒ मोज॒ ओजो॒ बल᳚म् । \newline
56. बल॑ मिन्द्रि॒य मि॑न्द्रि॒यम् बल॒म् बल॑ मिन्द्रि॒यम् । \newline
57. इ॒न्द्रि॒यं ॅवी॒र्यं॑ ॅवी॒र्य॑ मिन्द्रि॒य मि॑न्द्रि॒यं ॅवी॒र्य᳚म् । \newline
58. वी॒र्य॑म् प्र॒जाम् प्र॒जां ॅवी॒र्यं॑ ॅवी॒र्य॑म् प्र॒जाम् । \newline

\textbf{Ghana Paata } \newline

1. भ्राजो᳚ ऽस्यसि॒ भ्राजो॒ भ्राजो॑ ऽसि दे॒वाना᳚म् दे॒वाना॑ मसि॒ भ्राजो॒ भ्राजो॑ ऽसि दे॒वाना᳚म् । \newline
2. अ॒सि॒ दे॒वाना᳚म् दे॒वाना॑ मस्यसि दे॒वाना॒म् धाम॒ धाम॑ दे॒वाना॑ मस्यसि दे॒वाना॒म् धाम॑ । \newline
3. दे॒वाना॒म् धाम॒ धाम॑ दे॒वाना᳚म् दे॒वाना॒म् धाम॒ नाम॒ नाम॒ धाम॑ दे॒वाना᳚म् दे॒वाना॒म् धाम॒ नाम॑ । \newline
4. धाम॒ नाम॒ नाम॒ धाम॒ धाम॒ नामा᳚स्यसि॒ नाम॒ धाम॒ धाम॒ नामा॑सि । \newline
5. नामा᳚स्यसि॒ नाम॒ नामा॑सि॒ विश्वं॒ ॅविश्व॑ मसि॒ नाम॒ नामा॑सि॒ विश्व᳚म् । \newline
6. अ॒सि॒ विश्वं॒ ॅविश्व॑ मस्यसि॒ विश्व॑ मस्यसि॒ विश्व॑ मस्यसि॒ विश्व॑ मसि । \newline
7. विश्व॑ मस्यसि॒ विश्वं॒ ॅविश्व॑ मसि वि॒श्वायु॑र् वि॒श्वायु॑रसि॒ विश्वं॒ ॅविश्व॑ मसि वि॒श्वायुः॑ । \newline
8. अ॒सि॒ वि॒श्वायु॑र् वि॒श्वायु॑ रस्यसि वि॒श्वायुः॒ सर्वꣳ॒॒ सर्वं॑ ॅवि॒श्वायु॑ रस्यसि वि॒श्वायुः॒ सर्व᳚म् । \newline
9. वि॒श्वायुः॒ सर्वꣳ॒॒ सर्वं॑ ॅवि॒श्वायु॑र् वि॒श्वायुः॒ सर्व॑ मस्यसि॒ सर्वं॑ ॅवि॒श्वायु॑र् वि॒श्वायुः॒ सर्व॑ मसि । \newline
10. वि॒श्वायु॒रिति॑ वि॒श्व - आ॒युः॒ । \newline
11. सर्व॑ मस्यसि॒ सर्वꣳ॒॒ सर्व॑ मसि स॒र्वायुः॑ स॒र्वायु॑रसि॒ सर्वꣳ॒॒ सर्व॑ मसि स॒र्वायुः॑ । \newline
12. अ॒सि॒ स॒र्वायुः॑ स॒र्वायु॑ रस्यसि स॒र्वायु॑ रभि॒भू र॑भि॒भूः स॒र्वायु॑ रस्यसि स॒र्वायु॑ रभि॒भूः । \newline
13. स॒र्वायु॑ रभि॒भू र॑भि॒भूः स॒र्वायुः॑ स॒र्वायु॑ रभि॒भूरिती त्य॑भि॒भूः स॒र्वायुः॑ स॒र्वायु॑ रभि॒भूरिति॑ । \newline
14. स॒र्वायु॒रिति॑ स॒र्व - आ॒युः॒ । \newline
15. अ॒भि॒भूरिती त्य॑भि॒भू र॑भि॒भूरिति॒ वाव वावे त्य॑भि॒भू र॑भि॒भूरिति॒ वाव । \newline
16. अ॒भि॒भूरित्य॑भि - भूः । \newline
17. इति॒ वाव वावे तीति॒ वाव दे॒वा दे॒वा वावे तीति॒ वाव दे॒वाः । \newline
18. वाव दे॒वा दे॒वा वाव वाव दे॒वा असु॑राणा॒ मसु॑राणाम् दे॒वा वाव वाव दे॒वा असु॑राणाम् । \newline
19. दे॒वा असु॑राणा॒ मसु॑राणाम् दे॒वा दे॒वा असु॑राणा॒ मोज॒ ओजो ऽसु॑राणाम् दे॒वा दे॒वा असु॑राणा॒ मोजः॑ । \newline
20. असु॑राणा॒ मोज॒ ओजो ऽसु॑राणा॒ मसु॑राणा॒ मोजो॒ बल॒म् बल॒ मोजो ऽसु॑राणा॒ मसु॑राणा॒ मोजो॒ बल᳚म् । \newline
21. ओजो॒ बल॒म् बल॒ मोज॒ ओजो॒ बल॑ मिन्द्रि॒य मि॑न्द्रि॒यम् बल॒ मोज॒ ओजो॒ बल॑ मिन्द्रि॒यम् । \newline
22. बल॑ मिन्द्रि॒य मि॑न्द्रि॒यम् बल॒म् बल॑ मिन्द्रि॒यं ॅवी॒र्यं॑ ॅवी॒र्य॑ मिन्द्रि॒यम् बल॒म् बल॑ मिन्द्रि॒यं ॅवी॒र्य᳚म् । \newline
23. इ॒न्द्रि॒यं ॅवी॒र्यं॑ ॅवी॒र्य॑ मिन्द्रि॒य मि॑न्द्रि॒यं ॅवी॒र्य॑म् प्र॒जाम् प्र॒जां ॅवी॒र्य॑ मिन्द्रि॒य मि॑न्द्रि॒यं ॅवी॒र्य॑म् प्र॒जाम् । \newline
24. वी॒र्य॑म् प्र॒जाम् प्र॒जां ॅवी॒र्यं॑ ॅवी॒र्य॑म् प्र॒जाम् प॒शून् प॒शून् प्र॒जां ॅवी॒र्यं॑ ॅवी॒र्य॑म् प्र॒जाम् प॒शून् । \newline
25. प्र॒जाम् प॒शून् प॒शून् प्र॒जाम् प्र॒जाम् प॒शू न॑वृञ्जता वृञ्जत प॒शून् प्र॒जाम् प्र॒जाम् प॒शू न॑वृञ्जत । \newline
26. प्र॒जामिति॑ प्र - जाम् । \newline
27. प॒शू न॑वृञ्जता वृञ्जत प॒शून् प॒शू न॑वृञ्जत॒ यद् यद॑वृञ्जत प॒शून् प॒शू न॑वृञ्जत॒ यत् । \newline
28. अ॒वृ॒ञ्ज॒त॒ यद् यद॑वृञ्जता वृञ्जत॒ यद् गा॑य॒त्री गा॑य॒त्री यद॑वृञ्जता वृञ्जत॒ यद् गा॑य॒त्री । \newline
29. यद् गा॑य॒त्री गा॑य॒त्री यद् यद् गा॑य॒त्र्य॑ प॒क्रम्या॑ प॒क्रम्य॑ गाय॒त्री यद् यद् गा॑य॒त्र्य॑ प॒क्रम्य॑ । \newline
30. गा॒य॒ त्र्य॑प॒क्रम्या॑ प॒क्रम्य॑ गाय॒त्री गा॑य॒त्र्य॑ प॒क्रम्या ति॑ष्ठ॒दति॑ष्ठ दप॒क्रम्य॑ गाय॒त्री गा॑य॒ त्र्य॑प॒क्रम्याति॑ष्ठत् । \newline
31. अ॒प॒क्रम्याति॑ष्ठ॒ दति॑ष्ठ दप॒क्रम्या॑ प॒क्रम्याति॑ष्ठ॒त् तस्मा॒त् तस्मा॒ दति॑ष्ठ दप॒क्रम्या॑ प॒क्रम्या ति॑ष्ठ॒त् तस्मा᳚त् । \newline
32. अ॒प॒क्रम्येत्य॑प - क्रम्य॑ । \newline
33. अति॑ष्ठ॒त् तस्मा॒त् तस्मा॒ दति॑ष्ठ॒ दति॑ष्ठ॒त् तस्मा॑ दे॒ता मे॒ताम् तस्मा॒ दति॑ष्ठ॒ दति॑ष्ठ॒त् तस्मा॑ दे॒ताम् । \newline
34. तस्मा॑दे॒ता मे॒ताम् तस्मा॒त् तस्मा॑ दे॒ताम् गा॑य॒त्री गा॑य॒त्र्ये॑ताम् तस्मा॒त् तस्मा॑ दे॒ताम् गा॑य॒त्री । \newline
35. ए॒ताम् गा॑य॒त्री गा॑य॒त्र्ये॑ता मे॒ताम् गा॑य॒त्रीतीति॑ गाय॒त्र्ये॑ता मे॒ताम् गा॑य॒त्रीति॑ । \newline
36. गा॒य॒त्रीतीति॑ गाय॒त्री गा॑य॒त्रीतीष्टि॒ मिष्टि॒ मिति॑ गाय॒त्री गा॑य॒त्रीतीष्टि᳚म् । \newline
37. इतीष्टि॒ मिष्टि॒ मितीतीष्टि॑ माहु राहु॒रिष्टि॒ मितीतीष्टि॑ माहुः । \newline
38. इष्टि॑ माहु राहु॒रिष्टि॒ मिष्टि॑ माहुः संॅवथ्स॒रः सं॑ॅवथ्स॒र आ॑हु॒ रिष्टि॒ मिष्टि॑ माहुः संॅवथ्स॒रः । \newline
39. आ॒हुः॒ सं॒ॅव॒थ्स॒रः सं॑ॅवथ्स॒र आ॑हु राहुः संॅवथ्स॒रो वै वै सं॑ॅवथ्स॒र आ॑हु राहुः संॅवथ्स॒रो वै । \newline
40. सं॒ॅव॒थ्स॒रो वै वै सं॑ॅवथ्स॒रः सं॑ॅवथ्स॒रो वै गा॑य॒त्री गा॑य॒त्री वै सं॑ॅवथ्स॒रः सं॑ॅवथ्स॒रो वै गा॑य॒त्री । \newline
41. सं॒ॅव॒थ्स॒र इति॑ सं - व॒थ्स॒रः । \newline
42. वै गा॑य॒त्री गा॑य॒त्री वै वै गा॑य॒त्री सं॑ॅवथ्स॒रः सं॑ॅवथ्स॒रो गा॑य॒त्री वै वै गा॑य॒त्री सं॑ॅवथ्स॒रः । \newline
43. गा॒य॒त्री सं॑ॅवथ्स॒रः सं॑ॅवथ्स॒रो गा॑य॒त्री गा॑य॒त्री सं॑ॅवथ्स॒रो वै वै सं॑ॅवथ्स॒रो गा॑य॒त्री गा॑य॒त्री सं॑ॅवथ्स॒रो वै । \newline
44. सं॒ॅव॒थ्स॒रो वै वै सं॑ॅवथ्स॒रः सं॑ॅवथ्स॒रो वै तत् तद् वै सं॑ॅवथ्स॒रः सं॑ॅवथ्स॒रो वै तत् । \newline
45. सं॒ॅव॒थ्स॒र इति॑ सं - व॒थ्स॒रः । \newline
46. वै तत् तद् वै वै तद॑प॒क्रम्या॑ प॒क्रम्य॒ तद् वै वै तद॑प॒क्रम्य॑ । \newline
47. तद॑प॒क्रम्या॑ प॒क्रम्य॒ तत् तद॑प॒क्रम्या॑ तिष्ठदतिष्ठ दप॒क्रम्य॒ तत् तद॑प॒क्रम्या॑तिष्ठत् । \newline
48. अ॒प॒क्रम्या॑ तिष्ठदतिष्ठ दप॒क्रम्या॑ प॒क्रम्या॑ तिष्ठ॒द् यद् यद॑तिष्ठ दप॒क्रम्या॑ प॒क्रम्या॑ तिष्ठ॒द् यत् । \newline
49. अ॒प॒क्रम्येत्य॑प - क्रम्य॑ । \newline
50. अ॒ति॒ष्ठ॒द् यद् यद॑तिष्ठ दतिष्ठ॒द् यदे॒त यै॒तया॒ यद॑तिष्ठ दतिष्ठ॒द् यदे॒तया᳚ । \newline
51. यदे॒त यै॒तया॒ यद् यदे॒तया॑ दे॒वा दे॒वा ए॒तया॒ यद् यदे॒तया॑ दे॒वाः । \newline
52. ए॒तया॑ दे॒वा दे॒वा ए॒त यै॒तया॑ दे॒वा असु॑राणा॒ मसु॑राणाम् दे॒वा ए॒तयै॒तया॑ दे॒वा असु॑राणाम् । \newline
53. दे॒वा असु॑राणा॒ मसु॑राणाम् दे॒वा दे॒वा असु॑राणा॒ मोज॒ ओजो ऽसु॑राणाम् दे॒वा दे॒वा असु॑राणा॒ मोजः॑ । \newline
54. असु॑राणा॒ मोज॒ ओजो ऽसु॑राणा॒ मसु॑राणा॒ मोजो॒ बल॒म् बल॒ मोजो ऽसु॑राणा॒ मसु॑राणा॒ मोजो॒ बल᳚म् । \newline
55. ओजो॒ बल॒म् बल॒ मोज॒ ओजो॒ बल॑ मिन्द्रि॒य मि॑न्द्रि॒यम् बल॒ मोज॒ ओजो॒ बल॑ मिन्द्रि॒यम् । \newline
56. बल॑ मिन्द्रि॒य मि॑न्द्रि॒यम् बल॒म् बल॑ मिन्द्रि॒यं ॅवी॒र्यं॑ ॅवी॒र्य॑ मिन्द्रि॒यम् बल॒म् बल॑ मिन्द्रि॒यं ॅवी॒र्य᳚म् । \newline
57. इ॒न्द्रि॒यं ॅवी॒र्यं॑ ॅवी॒र्य॑ मिन्द्रि॒य मि॑न्द्रि॒यं ॅवी॒र्य॑म् प्र॒जाम् प्र॒जां ॅवी॒र्य॑ मिन्द्रि॒य मि॑न्द्रि॒यं ॅवी॒र्य॑म् प्र॒जाम् । \newline
58. वी॒र्य॑म् प्र॒जाम् प्र॒जां ॅवी॒र्यं॑ ॅवी॒र्य॑म् प्र॒जाम् प॒शून् प॒शून् प्र॒जां ॅवी॒र्यं॑ ॅवी॒र्य॑म् प्र॒जाम् प॒शून् । \newline
\pagebreak
\markright{ TS 2.4.3.3  \hfill https://www.vedavms.in \hfill}

\section{ TS 2.4.3.3 }

\textbf{TS 2.4.3.3 } \newline

\textbf{Pada Paata} \newline

प्र॒जामिति॑ प्र - जाम् । प॒शून् । अवृ॑ञ्जत । तस्मा᳚त् । ए॒ताम् । सं॒ॅव॒र्ग इति॑ सं - व॒र्गः । इति॑ । इष्टि᳚म् । आ॒हुः॒ । यः । भ्रातृ॑व्यवा॒निति॒ भ्रातृ॑व्य - वा॒न् । स्यात् । सः । स्पर्द्ध॑मानः । ए॒तया᳚ । इष्ट्या᳚ । य॒जे॒त॒ । अ॒ग्नये᳚ । सं॒ॅव॒र्गायेति॑ सं - व॒र्गाय॑ । पु॒रो॒डाश᳚म् । अ॒ष्टाक॑पाल॒मित्य॒ष्टा - क॒पा॒ल॒म् । निरिति॑ । व॒पे॒त् । तम् । शृ॒तम् । आस॑न्न॒मित्या - स॒न्न॒म् । ए॒तेन॑ । यजु॑षा । अ॒भीति॑ । मृ॒शे॒त् । ओजः॑ । ए॒व । बल᳚म् । इ॒न्द्रि॒यम् । वी॒र्य᳚म् । प्र॒जामिति॑ प्र - जाम् । प॒शून् । भ्रातृ॑व्यस्य । वृ॒ङ्क्ते॒ । भव॑ति । आ॒त्मना᳚ । परेति॑ । अ॒स्य॒ । भ्रातृ॑व्यः । भ॒व॒ति॒ ॥  \newline


\textbf{Krama Paata} \newline

प्र॒जाम् प॒शून् । प्र॒जामिति॑ प्र - जाम् । प॒शूनवृ॑ञ्जत । अवृ॑ञ्जत॒ तस्मा᳚त् । तस्मा॑दे॒ताम् । ए॒ताꣳ स॑म्ॅव॒र्गः । स॒म्ॅव॒र्ग इति॑ । स॒म्ॅव॒र्ग इति॑ सम् - व॒र्गः । इतीष्टि᳚म् । इष्टि॑माहुः । आ॒हु॒र् यः । यो भ्रातृ॑व्यवान् । भ्रातृ॑व्यवा॒न्थ् स्यात् । भ्रातृ॑व्यवा॒निति॒ भ्रातृ॑व्य - वा॒न्॒ । स्याथ् सः । स स्पर्द्ध॑मानः । स्पर्द्ध॑मान ए॒तया᳚ । ए॒तयेष्ट्या᳚ । इष्ट्या॑ यजेत । य॒जे॒ता॒ग्नये᳚ । अ॒ग्नये॑ सम्ॅव॒र्गाय॑ । स॒म्ॅव॒र्गाय॑ पुरो॒डाश᳚म् । स॒म्ॅव॒र्गायेति॑ सम् - व॒र्गाय॑ । पु॒रो॒डाश॑म॒ष्टाक॑पालम् । अ॒ष्टाक॑पाल॒म् निः । अ॒ष्टाक॑पाल॒मित्य॒ष्टा - क॒पा॒ल॒म् । निर् व॑पेत् । व॒पे॒त् तम् । तꣳ शृ॒तम् । शृ॒तमास॑न्नम् । आस॑न्नमे॒तेन॑ । आस॑न्न॒मित्या - स॒न्न॒म् । ए॒तेन॒ यजु॑षा । यजु॑षा॒ऽभि । अ॒भि मृ॑शेत् । मृ॒शे॒दोजः॑ । ओज॑ ए॒व । ए॒व बल᳚म् । बल॑मिन्द्रि॒यम् । इ॒न्द्रि॒यं ॅवी॒र्य᳚म् । वी॒र्यं॑ प्र॒जाम् । प्र॒जाम् प॒शून् । प्र॒जामिति॑ प्र - जाम् । प॒शून् भ्रातृ॑व्यस्य । भ्रातृ॑व्यस्य वृङ्क्ते । वृ॒ङ्क्ते॒ भव॑ति । भव॑त्या॒त्मना᳚ । आ॒त्मना॒ परा᳚ । परा᳚ ऽस्य । अ॒स्य॒ भ्रातृ॑व्यः । भ्रातृ॑व्यो भवति । भ॒व॒तीति॑ भवति । \newline

\textbf{Jatai Paata} \newline

1. प्र॒जाम् प॒शून् प॒शून् प्र॒जाम् प्र॒जाम् प॒शून् । \newline
2. प्र॒जामिति॑ प्र - जाम् । \newline
3. प॒शू नवृ॑ञ्ज॒ता वृ॑ञ्जत प॒शून् प॒शू नवृ॑ञ्जत । \newline
4. अवृ॑ञ्जत॒ तस्मा॒त् तस्मा॒ दवृ॑ञ्ज॒ता वृ॑ञ्जत॒ तस्मा᳚त् । \newline
5. तस्मा॑ दे॒ता मे॒ताम् तस्मा॒त् तस्मा॑ दे॒ताम् । \newline
6. ए॒ताꣳ सं॑ॅव॒र्गः सं॑ॅव॒र्ग ए॒ता मे॒ताꣳ सं॑ॅव॒र्गः । \newline
7. सं॒ॅव॒र्ग इतीति॑ संॅव॒र्गः सं॑ॅव॒र्ग इति॑ । \newline
8. सं॒ॅव॒र्ग इति॑ सं - व॒र्गः । \newline
9. इतीष्टि॒ मिष्टि॒ मितीतीष्टि᳚म् । \newline
10. इष्टि॑ माहु राहु॒ रिष्टि॒ मिष्टि॑ माहुः । \newline
11. आ॒हु॒र् यो य आ॑हु राहु॒र् यः । \newline
12. यो भ्रातृ॑व्यवा॒न् भ्रातृ॑व्यवा॒न्॒. यो यो भ्रातृ॑व्यवान् । \newline
13. भ्रातृ॑व्यवा॒न् थ्स्याथ् स्याद् भ्रातृ॑व्यवा॒न् भ्रातृ॑व्यवा॒न् थ्स्यात् । \newline
14. भ्रातृ॑व्यवा॒निति॒ भ्रातृ॑व्य - वा॒न् । \newline
15. स्याथ् स स स्याथ् स्याथ् सः । \newline
16. स स्पर्द्ध॑मानः॒ स्पर्द्ध॑मानः॒ स स स्पर्द्ध॑मानः । \newline
17. स्पर्द्ध॑मान ए॒तयै॒तया॒ स्पर्द्ध॑मानः॒ स्पर्द्ध॑मान ए॒तया᳚ । \newline
18. ए॒त येष्ट्येष्ट् यै॒त यै॒त येष्ट्या᳚ । \newline
19. इष्ट्या॑ यजेत यजे॒ते ष्ट्येष्ट्या॑ यजेत । \newline
20. य॒जे॒ता॒ग्नये॒ ऽग्नये॑ यजेत यजेता॒ग्नये᳚ । \newline
21. अ॒ग्नये॑ संॅव॒र्गाय॑ संॅव॒र्गाया॒ग्नये॒ ऽग्नये॑ संॅव॒र्गाय॑ । \newline
22. सं॒ॅव॒र्गाय॑ पुरो॒डाश॑म् पुरो॒डाशꣳ॑ संॅव॒र्गाय॑ संॅव॒र्गाय॑ पुरो॒डाश᳚म् । \newline
23. सं॒ॅव॒र्गायेति॑ सं - व॒र्गाय॑ । \newline
24. पु॒रो॒डाश॑ म॒ष्टाक॑पाल म॒ष्टाक॑पालम् पुरो॒डाश॑म् पुरो॒डाश॑ म॒ष्टाक॑पालम् । \newline
25. अ॒ष्टाक॑पाल॒म् निर् णिर॒ष्टाक॑पाल म॒ष्टाक॑पाल॒म् निः । \newline
26. अ॒ष्टाक॑पाल॒मित्य॒ष्टा - क॒पा॒ल॒म् । \newline
27. निर् व॑पेद् वपे॒न् निर् णिर् व॑पेत् । \newline
28. व॒पे॒त् तम् तं ॅव॑पेद् वपे॒त् तम् । \newline
29. तꣳ शृ॒तꣳ शृ॒तम् तम् तꣳ शृ॒तम् । \newline
30. शृ॒त मास॑न्न॒ मास॑न्नꣳ शृ॒तꣳ शृ॒त मास॑न्नम् । \newline
31. आस॑न्न मे॒ते नै॒ते नास॑न्न॒ मास॑न्न मे॒तेन॑ । \newline
32. आस॑न्न॒मित्या - स॒न्न॒म् । \newline
33. ए॒तेन॒ यजु॑षा॒ यजु॑षै॒ते नै॒तेन॒ यजु॑षा । \newline
34. यजु॑षा॒ ऽभ्य॑भि यजु॑षा॒ यजु॑षा॒ ऽभि । \newline
35. अ॒भि मृ॑शेन् मृशे द॒भ्य॑भि मृ॑शेत् । \newline
36. मृ॒शे॒ दोज॒ ओजो॑ मृशेन् मृशे॒ दोजः॑ । \newline
37. ओज॑ ए॒वैवौज॒ ओज॑ ए॒व । \newline
38. ए॒व बल॒म् बल॑ मे॒वैव बल᳚म् । \newline
39. बल॑ मिन्द्रि॒य मि॑न्द्रि॒यम् बल॒म् बल॑ मिन्द्रि॒यम् । \newline
40. इ॒न्द्रि॒यं ॅवी॒र्यं॑ ॅवी॒र्य॑ मिन्द्रि॒य मि॑न्द्रि॒यं ॅवी॒र्य᳚म् । \newline
41. वी॒र्य॑म् प्र॒जाम् प्र॒जां ॅवी॒र्यं॑ ॅवी॒र्य॑म् प्र॒जाम् । \newline
42. प्र॒जाम् प॒शून् प॒शून् प्र॒जाम् प्र॒जाम् प॒शून् । \newline
43. प्र॒जामिति॑ प्र - जाम् । \newline
44. प॒शून् भ्रातृ॑व्यस्य॒ भ्रातृ॑व्यस्य प॒शून् प॒शून् भ्रातृ॑व्यस्य । \newline
45. भ्रातृ॑व्यस्य वृङ्क्ते वृङ्क्ते॒ भ्रातृ॑व्यस्य॒ भ्रातृ॑व्यस्य वृङ्क्ते । \newline
46. वृ॒ङ्क्ते॒ भव॑ति॒ भव॑ति वृङ्क्ते वृङ्क्ते॒ भव॑ति । \newline
47. भव॑ त्या॒त्मना॒ ऽऽत्मना॒ भव॑ति॒ भव॑ त्या॒त्मना᳚ । \newline
48. आ॒त्मना॒ परा॒ परा॒ ऽऽत्मना॒ ऽऽत्मना॒ परा᳚ । \newline
49. परा᳚ ऽस्यास्य॒ परा॒ परा᳚ ऽस्य । \newline
50. अ॒स्य॒ भ्रातृ॑व्यो॒ भ्रातृ॑व्यो ऽस्यास्य॒ भ्रातृ॑व्यः । \newline
51. भ्रातृ॑व्यो भवति भवति॒ भ्रातृ॑व्यो॒ भ्रातृ॑व्यो भवति । \newline
52. भ॒व॒तीति॑ भवति । \newline

\textbf{Ghana Paata } \newline

1. प्र॒जाम् प॒शून् प॒शून् प्र॒जाम् प्र॒जाम् प॒शू नवृ॑ञ्ज॒ता वृ॑ञ्जत प॒शून् प्र॒जाम् प्र॒जाम् प॒शू नवृ॑ञ्जत । \newline
2. प्र॒जामिति॑ प्र - जाम् । \newline
3. प॒शू नवृ॑ञ्ज॒ता वृ॑ञ्जत प॒शून् प॒शू नवृ॑ञ्जत॒ तस्मा॒त् तस्मा॒ दवृ॑ञ्जत प॒शून् प॒शू नवृ॑ञ्जत॒ तस्मा᳚त् । \newline
4. अवृ॑ञ्जत॒ तस्मा॒त् तस्मा॒ दवृ॑ञ्ज॒ता वृ॑ञ्जत॒ तस्मा॑दे॒ता मे॒ताम् तस्मा॒ दवृ॑ञ्ज॒ता वृ॑ञ्जत॒ तस्मा॑ दे॒ताम् । \newline
5. तस्मा॑ दे॒ता मे॒ताम् तस्मा॒त् तस्मा॑ दे॒ताꣳ सं॑ॅव॒र्गः सं॑ॅव॒र्ग ए॒ताम् तस्मा॒त् तस्मा॑ दे॒ताꣳ सं॑ॅव॒र्गः । \newline
6. ए॒ताꣳ सं॑ॅव॒र्गः सं॑ॅव॒र्ग ए॒ता मे॒ताꣳ सं॑ॅव॒र्ग इतीति॑ संॅव॒र्ग ए॒ता मे॒ताꣳ सं॑ॅव॒र्ग इति॑ । \newline
7. सं॒ॅव॒र्ग इतीति॑ संॅव॒र्गः सं॑ॅव॒र्ग इतीष्टि॒ मिष्टि॒ मिति॑ संॅव॒र्गः सं॑ॅव॒र्ग इतीष्टि᳚म् । \newline
8. सं॒ॅव॒र्ग इति॑ सं - व॒र्गः । \newline
9. इतीष्टि॒ मिष्टि॒ मितीतीष्टि॑ माहु राहु॒ रिष्टि॒ मितीतीष्टि॑ माहुः । \newline
10. इष्टि॑ माहु राहु॒ रिष्टि॒ मिष्टि॑ माहु॒र् यो य आ॑हु॒ रिष्टि॒ मिष्टि॑ माहु॒र् यः । \newline
11. आ॒हु॒र् यो य आ॑हु राहु॒र् यो भ्रातृ॑व्यवा॒न् भ्रातृ॑व्यवा॒न्॒. य आ॑हु राहु॒र् यो भ्रातृ॑व्यवान् । \newline
12. यो भ्रातृ॑व्यवा॒न् भ्रातृ॑व्यवा॒न्॒. यो यो भ्रातृ॑व्यवा॒न् थ्स्याथ् स्याद् भ्रातृ॑व्यवा॒न्॒. यो यो भ्रातृ॑व्यवा॒न् थ्स्यात् । \newline
13. भ्रातृ॑व्यवा॒न् थ्स्याथ् स्याद् भ्रातृ॑व्यवा॒न् भ्रातृ॑व्यवा॒न् थ्स्याथ् स स स्याद् भ्रातृ॑व्यवा॒न् भ्रातृ॑व्यवा॒न् थ्स्याथ् सः । \newline
14. भ्रातृ॑व्यवा॒निति॒ भ्रातृ॑व्य - वा॒न् । \newline
15. स्याथ् स स स्याथ् स्याथ् स स्पर्द्ध॑मानः॒ स्पर्द्ध॑मानः॒ स स्याथ् स्याथ् स स्पर्द्ध॑मानः । \newline
16. स स्पर्द्ध॑मानः॒ स्पर्द्ध॑मानः॒ स स स्पर्द्ध॑मान ए॒तयै॒तया॒ स्पर्द्ध॑मानः॒ स स स्पर्द्ध॑मान ए॒तया᳚ । \newline
17. स्पर्द्ध॑मान ए॒तयै॒तया॒ स्पर्द्ध॑मानः॒ स्पर्द्ध॑मान ए॒तयेष्ट्ये ष्ट्यै॒तया॒ स्पर्द्ध॑मानः॒ स्पर्द्ध॑मान ए॒तयेष्ट्या᳚ । \newline
18. ए॒तयेष्ट्ये ष्ट्यै॒तयै॒त येष्ट्या॑ यजेत यजे॒ते ष्ट्यै॒त यै॒तयेष्ट्या॑ यजेत । \newline
19. इष्ट्या॑ यजेत यजे॒ते ष्ट्येष्ट्या॑ यजेता॒ग्नये॒ ऽग्नये॑ यजे॒ते ष्ट्येष्ट्या॑ यजेता॒ग्नये᳚ । \newline
20. य॒जे॒ता॒ग्नये॒ ऽग्नये॑ यजेत यजेता॒ग्नये॑ संॅव॒र्गाय॑ संॅव॒र्गाया॒ग्नये॑ यजेत यजेता॒ग्नये॑ संॅव॒र्गाय॑ । \newline
21. अ॒ग्नये॑ संॅव॒र्गाय॑ संॅव॒र्गाया॒ग्नये॒ ऽग्नये॑ संॅव॒र्गाय॑ पुरो॒डाश॑म् पुरो॒डाशꣳ॑ संॅव॒र्गाया॒ग्नये॒ ऽग्नये॑ संॅव॒र्गाय॑ पुरो॒डाश᳚म् । \newline
22. सं॒ॅव॒र्गाय॑ पुरो॒डाश॑म् पुरो॒डाशꣳ॑ संॅव॒र्गाय॑ संॅव॒र्गाय॑ पुरो॒डाश॑ म॒ष्टाक॑पाल म॒ष्टाक॑पालम् पुरो॒डाशꣳ॑ संॅव॒र्गाय॑ संॅव॒र्गाय॑ पुरो॒डाश॑ म॒ष्टाक॑पालम् । \newline
23. सं॒ॅव॒र्गायेति॑ सं - व॒र्गाय॑ । \newline
24. पु॒रो॒डाश॑ म॒ष्टाक॑पाल म॒ष्टाक॑पालम् पुरो॒डाश॑म् पुरो॒डाश॑ म॒ष्टाक॑पाल॒म् निर् णिर॒ष्टाक॑पालम् पुरो॒डाश॑म् पुरो॒डाश॑ म॒ष्टाक॑पाल॒म् निः । \newline
25. अ॒ष्टाक॑पाल॒म् निर् णिर॒ष्टाक॑पाल म॒ष्टाक॑पाल॒म् निर् व॑पेद् वपे॒न् निर॒ष्टाक॑पाल म॒ष्टाक॑पाल॒म् निर् व॑पेत् । \newline
26. अ॒ष्टाक॑पाल॒मित्य॒ष्टा - क॒पा॒ल॒म् । \newline
27. निर् व॑पेद् वपे॒न् निर् णिर् व॑पे॒त् तम् तं ॅव॑पे॒न् निर् णिर् व॑पे॒त् तम् । \newline
28. व॒पे॒त् तम् तं ॅव॑पेद् वपे॒त् तꣳ शृ॒तꣳ शृ॒तम् तं ॅव॑पेद् वपे॒त् तꣳ शृ॒तम् । \newline
29. तꣳ शृ॒तꣳ शृ॒तम् तम् तꣳ शृ॒त मास॑न्न॒ मास॑न्नꣳ शृ॒तम् तम् तꣳ शृ॒त मास॑न्नम् । \newline
30. शृ॒त मास॑न्न॒ मास॑न्नꣳ शृ॒तꣳ शृ॒त मास॑न्न मे॒तेनै॒ते नास॑न्नꣳ शृ॒तꣳ शृ॒त मास॑न्न मे॒तेन॑ । \newline
31. आस॑न्न मे॒तेनै॒ते नास॑न्न॒ मास॑न्न मे॒तेन॒ यजु॑षा॒ यजु॑षै॒ते नास॑न्न॒ मास॑न्न मे॒तेन॒ यजु॑षा । \newline
32. आस॑न्न॒मित्या - स॒न्न॒म् । \newline
33. ए॒तेन॒ यजु॑षा॒ यजु॑षै॒ते नै॒तेन॒ यजु॑षा॒ ऽभ्य॑भि यजु॑षै॒ते नै॒तेन॒ यजु॑षा॒ ऽभि । \newline
34. यजु॑षा॒ ऽभ्य॑भि यजु॑षा॒ यजु॑षा॒ ऽभि मृ॑शेन् मृशेद॒भि यजु॑षा॒ यजु॑षा॒ ऽभि मृ॑शेत् । \newline
35. अ॒भि मृ॑शेन् मृशे द॒भ्य॑भि मृ॑शे॒ दोज॒ ओजो॑ मृशे द॒भ्य॑भि मृ॑शे॒ दोजः॑ । \newline
36. मृ॒शे॒ दोज॒ ओजो॑ मृशेन् मृशे॒ दोज॑ ए॒वैवौजो॑ मृशेन् मृशे॒ दोज॑ ए॒व । \newline
37. ओज॑ ए॒वैवौज॒ ओज॑ ए॒व बल॒म् बल॑ मे॒वौज॒ ओज॑ ए॒व बल᳚म् । \newline
38. ए॒व बल॒म् बल॑ मे॒वैव बल॑ मिन्द्रि॒य मि॑न्द्रि॒यम् बल॑ मे॒वैव बल॑ मिन्द्रि॒यम् । \newline
39. बल॑ मिन्द्रि॒य मि॑न्द्रि॒यम् बल॒म् बल॑ मिन्द्रि॒यं ॅवी॒र्यं॑ ॅवी॒र्य॑ मिन्द्रि॒यम् बल॒म् बल॑ मिन्द्रि॒यं ॅवी॒र्य᳚म् । \newline
40. इ॒न्द्रि॒यं ॅवी॒र्यं॑ ॅवी॒र्य॑ मिन्द्रि॒य मि॑न्द्रि॒यं ॅवी॒र्य॑म् प्र॒जाम् प्र॒जां ॅवी॒र्य॑ मिन्द्रि॒य मि॑न्द्रि॒यं ॅवी॒र्य॑म् प्र॒जाम् । \newline
41. वी॒र्य॑म् प्र॒जाम् प्र॒जां ॅवी॒र्यं॑ ॅवी॒र्य॑म् प्र॒जाम् प॒शून् प॒शून् प्र॒जां ॅवी॒र्यं॑ ॅवी॒र्य॑म् प्र॒जाम् प॒शून् । \newline
42. प्र॒जाम् प॒शून् प॒शून् प्र॒जाम् प्र॒जाम् प॒शून् भ्रातृ॑व्यस्य॒ भ्रातृ॑व्यस्य प॒शून् प्र॒जाम् प्र॒जाम् प॒शून् भ्रातृ॑व्यस्य । \newline
43. प्र॒जामिति॑ प्र - जाम् । \newline
44. प॒शून् भ्रातृ॑व्यस्य॒ भ्रातृ॑व्यस्य प॒शून् प॒शून् भ्रातृ॑व्यस्य वृङ्क्ते वृङ्क्ते॒ भ्रातृ॑व्यस्य प॒शून् प॒शून् भ्रातृ॑व्यस्य वृङ्क्ते । \newline
45. भ्रातृ॑व्यस्य वृङ्क्ते वृङ्क्ते॒ भ्रातृ॑व्यस्य॒ भ्रातृ॑व्यस्य वृङ्क्ते॒ भव॑ति॒ भव॑ति वृङ्क्ते॒ भ्रातृ॑व्यस्य॒ भ्रातृ॑व्यस्य वृङ्क्ते॒ भव॑ति । \newline
46. वृ॒ङ्क्ते॒ भव॑ति॒ भव॑ति वृङ्क्ते वृङ्क्ते॒ भव॑ त्या॒त्मना॒ ऽऽत्मना॒ भव॑ति वृङ्क्ते वृङ्क्ते॒ भव॑ त्या॒त्मना᳚ । \newline
47. भव॑ त्या॒त्मना॒ ऽऽत्मना॒ भव॑ति॒ भव॑ त्या॒त्मना॒ परा॒ परा॒ ऽऽत्मना॒ भव॑ति॒ भव॑ त्या॒त्मना॒ परा᳚ । \newline
48. आ॒त्मना॒ परा॒ परा॒ ऽऽत्मना॒ ऽऽत्मना॒ परा᳚ ऽस्यास्य॒ परा॒ ऽऽत्मना॒ ऽऽत्मना॒ परा᳚ ऽस्य । \newline
49. परा᳚ ऽस्यास्य॒ परा॒ परा᳚ ऽस्य॒ भ्रातृ॑व्यो॒ भ्रातृ॑व्यो ऽस्य॒ परा॒ परा᳚ ऽस्य॒ भ्रातृ॑व्यः । \newline
50. अ॒स्य॒ भ्रातृ॑व्यो॒ भ्रातृ॑व्यो ऽस्यास्य॒ भ्रातृ॑व्यो भवति भवति॒ भ्रातृ॑व्यो ऽस्यास्य॒ भ्रातृ॑व्यो भवति । \newline
51. भ्रातृ॑व्यो भवति भवति॒ भ्रातृ॑व्यो॒ भ्रातृ॑व्यो भवति । \newline
52. भ॒व॒तीति॑ भवति । \newline
\pagebreak
\markright{ TS 2.4.4.1  \hfill https://www.vedavms.in \hfill}

\section{ TS 2.4.4.1 }

\textbf{TS 2.4.4.1 } \newline
\textbf{Samhita Paata} \newline

प्र॒जाप॑तिः प्र॒जा अ॑सृजत॒ ता अ॑स्माथ् सृ॒ष्टाः परा॑चीराय॒न् ता यत्राव॑स॒न् ततो॑ ग॒र्मुदुद॑तिष्ठ॒त् ता बृह॒स्पति॑श्चा॒न्ववै॑ताꣳ॒॒ सो᳚ऽब्रवी॒द् बृह॒स्पति॑र॒नया᳚ त्वा॒ प्रति॑ष्ठा॒न्यथ॑ त्वा प्र॒जा उ॒पाव॑र्थ्स्य॒न्तीति॒ तं प्राति॑ष्ठ॒त् ततो॒ वै प्र॒जाप॑तिं प्र॒जा उ॒पाव॑र्तन्त॒ यः प्र॒जाका॑मः॒ स्यात् तस्मा॑ ए॒तं प्रा॑जाप॒त्यं गा᳚र्मु॒तं च॒रुं निर्व॑पेत् प्र॒जाप॑ति - [  ] \newline

\textbf{Pada Paata} \newline

प्र॒जाप॑ति॒रिति॑ प्र॒जा - प॒तिः॒ । प्र॒जा इति॑ प्र-जाः । अ॒सृ॒ज॒त॒ । ताः । अ॒स्मा॒त् । सृ॒ष्टाः । परा॑चीः । आ॒य॒न्न् । ताः । यत्र॑ । अव॑सन्न् । ततः॑ । ग॒र्मुत् । उदिति॑ । अ॒ति॒ष्ठ॒त् । ताः । बृह॒स्पतिः॑ । च॒ । अ॒न्ववै॑ता॒मित्य॑नु - अवै॑ताम् । सः । अ॒ब्र॒वी॒त् । बृह॒स्पतिः॑ । अ॒नया᳚ । त्वा॒ । प्रेति॑ । ति॒ष्ठा॒नि॒ । अथ॑ । त्वा॒ । प्र॒जा इति॑ प्र - जाः । उ॒पाव॑र्थ्स्य॒न्तीत्यु॑प - आव॑र्थ्स्यन्ति । इति॑ । तम् । प्रेति॑ । अ॒ति॒ष्ठ॒त् । ततः॑ । वै । प्र॒जाप॑ति॒मिति॑ प्र॒जा - प॒ति॒म् । प्र॒जा इति॑ प्र - जाः । उ॒पाव॑र्त॒न्तेत्यु॑प - आव॑र्तन्त । यः । प्र॒जाका॑म॒ इति॑ प्र॒जा - का॒मः॒ । स्यात् । तस्मै᳚ । ए॒तम् । प्रा॒जा॒प॒त्यमिति॑ प्राजा - प॒त्यम् । गा॒र्मु॒तम् । च॒रुम् । निरिति॑ । व॒पे॒त् । प्र॒जाप॑ति॒मिति॑ प्र॒जा - प॒ति॒म् ।  \newline


\textbf{Krama Paata} \newline

प्र॒जाप॑तिः प्र॒जाः । प्र॒जाप॑ति॒रिति॑ प्र॒जा - प॒तिः॒ । प्र॒जा अ॑सृजत । प्र॒जा इति॑ प्र - जाः । अ॒सृ॒ज॒त॒ ताः । ता अ॑स्मात् । अ॒स्मा॒थ् सृ॒ष्टाः । सृ॒ष्टाः परा॑चीः । परा॑चीरायन्न् । आ॒य॒न् ताः । ता यत्र॑ । यत्राव॑सन्न् । अव॑स॒न् ततः॑ । ततो॑ ग॒र्मुत् । ग॒र्मुदुत् । उद॑तिष्ठत् । 
अ॒ति॒ष्ठ॒त् ताः । ता बृह॒स्पतिः॑ । बृह॒स्पति॑श्च । चा॒न्ववै॑ताम् । अ॒न्ववै॑ताꣳ॒॒ सः । अ॒न्ववै॑ता॒मित्य॑नु - अवै॑ताम् । सो᳚ऽब्रवीत् । अ॒ब्र॒वी॒द् बृह॒स्पतिः॑ । बृह॒स्पति॑र॒नया᳚ । अ॒नया᳚ त्वा । त्वा॒ प्र । प्र ति॑ष्ठानि । ति॒ष्ठा॒न्यथ॑ । अथ॑ त्वा । त्वा॒ प्र॒जाः । प्र॒जा उ॒पाव॑र्त्थ्स्यन्ति । प्र॒जा इति॑ प्र - जाः । उ॒पाव॑र्त्थ्स्य॒न्तीति॑ । उ॒पाव॑र्त्थ्स्य॒न्तीत्यु॑प - आव॑र्त्थ्स्यन्ति । इति॒ तम् । तम् प्र । प्राति॑ष्ठत् । अ॒ति॒ष्ठ॒त् ततः॑ । ततो॒ वै । वै प्र॒जाप॑तिम् । प्र॒जाप॑तिम् प्र॒जाः । प्र॒जाप॑ति॒मिति॑ प्र॒जा - प॒ति॒म् । प्र॒जा उ॒पाव॑र्तन्त । प्र॒जा इति॑ प्र - जाः । उ॒पाव॑र्तन्त॒ यः । उ॒पाव॑र्त॒न्तेत्यु॑प - आव॑र्तन्त । यः प्र॒जाका॑मः । प्र॒जाका॑मः॒ स्यात् । प्र॒जाका॑म॒ इति॑ प्र॒जा - का॒मः॒ । स्यात् तस्मै᳚ । तस्मा॑ ए॒तम् । ए॒तम् प्रा॑जाप॒त्यम् । प्रा॒जा॒प॒त्यम् गा᳚र्मु॒तम् । प्रा॒जा॒प॒त्यमिति॑ प्राजा - प॒त्यम् । गा॒र्मु॒तम् च॒रुम् । च॒रुम् निः । निर् व॑पेत् । व॒पे॒त् प्र॒जाप॑तिम् । प्र॒जाप॑तिमे॒व । 
प्र॒जाप॑ति॒मिति॑ प्र॒जा - प॒ति॒म् \newline

\textbf{Jatai Paata} \newline

1. प्र॒जाप॑तिः प्र॒जाः प्र॒जाः प्र॒जाप॑तिः प्र॒जाप॑तिः प्र॒जाः । \newline
2. प्र॒जाप॑ति॒रिति॑ प्र॒जा - प॒तिः॒ । \newline
3. प्र॒जा अ॑सृजता सृजत प्र॒जाः प्र॒जा अ॑सृजत । \newline
4. प्र॒जा इति॑ प्र - जाः । \newline
5. अ॒सृ॒ज॒त॒ ता स्ता अ॑सृजता सृजत॒ ताः । \newline
6. ता अ॑स्मा दस्मा॒त् ता स्ता अ॑स्मात् । \newline
7. अ॒स्मा॒थ् सृ॒ष्टाः सृ॒ष्टा अ॑स्मा दस्माथ् सृ॒ष्टाः । \newline
8. सृ॒ष्टाः परा॑चीः॒ परा॑चीः सृ॒ष्टाः सृ॒ष्टाः परा॑चीः । \newline
9. परा॑ची रायन् नाय॒न् परा॑चीः॒ परा॑ची रायन्न् । \newline
10. आ॒य॒न् ता स्ता आ॑यन् नाय॒न् ताः । \newline
11. ता यत्र॒ यत्र॒ ता स्ता यत्र॑ । \newline
12. यत्रा व॑स॒न् नव॑स॒न्॒. यत्र॒ यत्रा व॑सन्न् । \newline
13. अव॑स॒न् तत॒ स्ततो ऽव॑स॒न् नव॑स॒न् ततः॑ । \newline
14. ततो॑ ग॒र्मुद् ग॒र्मुत् तत॒ स्ततो॑ ग॒र्मुत् । \newline
15. ग॒र्मु दुदुद् ग॒र्मुद् ग॒र्मु दुत् । \newline
16. उद॑ तिष्ठ दतिष्ठ॒ दुदु द॑तिष्ठत् । \newline
17. अ॒ति॒ष्ठ॒त् ता स्ता अ॑तिष्ठ दतिष्ठ॒त् ताः । \newline
18. ता बृह॒स्पति॒र् बृह॒स्पति॒ स्ता स्ता बृह॒स्पतिः॑ । \newline
19. बृह॒स्पति॑श्च च॒ बृह॒स्पति॒र् बृह॒स्पति॑श्च । \newline
20. चा॒न्ववै॑ता म॒न्ववै॑ताम् च चा॒न्ववै॑ताम् । \newline
21. अ॒न्ववै॑ताꣳ॒॒ स सो᳚ ऽन्ववै॑ता म॒न्ववै॑ताꣳ॒॒ सः । \newline
22. अ॒न्ववै॑ता॒मित्य॑नु - अवै॑ताम् । \newline
23. सो᳚ ऽब्रवी दब्रवी॒थ् स सो᳚ ऽब्रवीत् । \newline
24. अ॒ब्र॒वी॒द् बृह॒स्पति॒र् बृह॒स्पति॑ रब्रवी दब्रवी॒द् बृह॒स्पतिः॑ । \newline
25. बृह॒स्पति॑ र॒नया॒ ऽनया॒ बृह॒स्पति॒र् बृह॒स्पति॑ र॒नया᳚ । \newline
26. अ॒नया᳚ त्वा त्वा॒ ऽनया॒ ऽनया᳚ त्वा । \newline
27. त्वा॒ प्र प्र त्वा᳚ त्वा॒ प्र । \newline
28. प्र ति॑ष्ठानि तिष्ठानि॒ प्र प्र ति॑ष्ठानि । \newline
29. ति॒ष्ठा॒ न्यथाथ॑ तिष्ठानि तिष्ठा॒ न्यथ॑ । \newline
30. अथ॑ त्वा॒ त्वा ऽथाथ॑ त्वा । \newline
31. त्वा॒ प्र॒जाः प्र॒जा स्त्वा᳚ त्वा प्र॒जाः । \newline
32. प्र॒जा उ॒पाव॑र्थ्स्य न्त्यु॒पाव॑र्थ्स्यन्ति प्र॒जाः प्र॒जा उ॒पाव॑र्थ्स्यन्ति । \newline
33. प्र॒जा इति॑ प्र - जाः । \newline
34. उ॒पाव॑र्थ्स्य॒न्तीती त्यु॒पाव॑र्थ्स्य न्त्यु॒पाव॑र्थ्स्य॒न्तीति॑ । \newline
35. उ॒पाव॑र्थ्स्य॒न्तीत्यु॑प - आव॑र्थ्स्यन्ति । \newline
36. इति॒ तम् त मितीति॒ तम् । \newline
37. तम् प्र प्र तम् तम् प्र । \newline
38. प्राति॑ष्ठ दतिष्ठ॒त् प्र प्राति॑ष्ठत् । \newline
39. अ॒ति॒ष्ठ॒त् तत॒ स्ततो॑ ऽतिष्ठ दतिष्ठ॒त् ततः॑ । \newline
40. ततो॒ वै वै तत॒ स्ततो॒ वै । \newline
41. वै प्र॒जाप॑तिम् प्र॒जाप॑तिं॒ ॅवै वै प्र॒जाप॑तिम् । \newline
42. प्र॒जाप॑तिम् प्र॒जाः प्र॒जाः प्र॒जाप॑तिम् प्र॒जाप॑तिम् प्र॒जाः । \newline
43. प्र॒जाप॑ति॒मिति॑ प्र॒जा - प॒ति॒म् । \newline
44. प्र॒जा उ॒पाव॑र्तन्तो॒ पाव॑र्तन्त प्र॒जाः प्र॒जा उ॒पाव॑र्तन्त । \newline
45. प्र॒जा इति॑ प्र - जाः । \newline
46. उ॒पाव॑र्तन्त॒ यो य उ॒पाव॑र्तन्तो॒ पाव॑र्तन्त॒ यः । \newline
47. उ॒पाव॑र्त॒न्तेत्यु॑प - आव॑र्तन्त । \newline
48. यः प्र॒जाका॑मः प्र॒जाका॑मो॒ यो यः प्र॒जाका॑मः । \newline
49. प्र॒जाका॑मः॒ स्याथ् स्यात् प्र॒जाका॑मः प्र॒जाका॑मः॒ स्यात् । \newline
50. प्र॒जाका॑म॒ इति॑ प्र॒जा - का॒मः॒ । \newline
51. स्यात् तस्मै॒ तस्मै॒ स्याथ् स्यात् तस्मै᳚ । \newline
52. तस्मा॑ ए॒त मे॒तम् तस्मै॒ तस्मा॑ ए॒तम् । \newline
53. ए॒तम् प्रा॑जाप॒त्यम् प्रा॑जाप॒त्य मे॒त मे॒तम् प्रा॑जाप॒त्यम् । \newline
54. प्रा॒जा॒प॒त्यम् गा᳚र्मु॒तम् गा᳚र्मु॒तम् प्रा॑जाप॒त्यम् प्रा॑जाप॒त्यम् गा᳚र्मु॒तम् । \newline
55. प्रा॒जा॒प॒त्यमिति॑ प्राजा - प॒त्यम् । \newline
56. गा॒र्मु॒तम् च॒रुम् च॒रुम् गा᳚र्मु॒तम् गा᳚र्मु॒तम् च॒रुम् । \newline
57. च॒रुम् निर् णि श्च॒रुम् च॒रुम् निः । \newline
58. निर् व॑पेद् वपे॒न् निर् णिर् व॑पेत् । \newline
59. व॒पे॒त् प्र॒जाप॑तिम् प्र॒जाप॑तिं ॅवपेद् वपेत् प्र॒जाप॑तिम् । \newline
60. प्र॒जाप॑ति मे॒वैव प्र॒जाप॑तिम् प्र॒जाप॑ति मे॒व । \newline
61. प्र॒जाप॑ति॒मिति॑ प्र॒जा - प॒ति॒म् । \newline

\textbf{Ghana Paata } \newline

1. प्र॒जाप॑तिः प्र॒जाः प्र॒जाः प्र॒जाप॑तिः प्र॒जाप॑तिः प्र॒जा अ॑सृजता सृजत प्र॒जाः प्र॒जाप॑तिः प्र॒जाप॑तिः प्र॒जा अ॑सृजत । \newline
2. प्र॒जाप॑ति॒रिति॑ प्र॒जा - प॒तिः॒ । \newline
3. प्र॒जा अ॑सृजता सृजत प्र॒जाः प्र॒जा अ॑सृजत॒ ता स्ता अ॑सृजत प्र॒जाः प्र॒जा अ॑सृजत॒ ताः । \newline
4. प्र॒जा इति॑ प्र - जाः । \newline
5. अ॒सृ॒ज॒त॒ ता स्ता अ॑सृजता सृजत॒ ता अ॑स्मा दस्मा॒त् ता अ॑सृजता सृजत॒ ता अ॑स्मात् । \newline
6. ता अ॑स्मा दस्मा॒त् ता स्ता अ॑स्माथ् सृ॒ष्टाः सृ॒ष्टा अ॑स्मा॒त् ता स्ता अ॑स्माथ् सृ॒ष्टाः । \newline
7. अ॒स्मा॒थ् सृ॒ष्टाः सृ॒ष्टा अ॑स्मा दस्माथ् सृ॒ष्टाः परा॑चीः॒ परा॑चीः सृ॒ष्टा अ॑स्मा दस्माथ् सृ॒ष्टाः परा॑चीः । \newline
8. सृ॒ष्टाः परा॑चीः॒ परा॑चीः सृ॒ष्टाः सृ॒ष्टाः परा॑चीरायन् नाय॒न् परा॑चीः सृ॒ष्टाः सृ॒ष्टाः परा॑चीरायन्न् । \newline
9. परा॑ची रायन् नाय॒न् परा॑चीः॒ परा॑ची राय॒न् ता स्ता आ॑य॒न् परा॑चीः॒ परा॑ची राय॒न् ताः । \newline
10. आ॒य॒न् ता स्ता आ॑यन् नाय॒न् ता यत्र॒ यत्र॒ ता आ॑यन् नाय॒न् ता यत्र॑ । \newline
11. ता यत्र॒ यत्र॒ ता स्ता यत्राव॑स॒न् नव॑स॒न्॒. यत्र॒ ता स्ता यत्राव॑सन्न् । \newline
12. यत्राव॑स॒न् नव॑स॒न्॒. यत्र॒ यत्राव॑स॒न् तत॒ स्ततो ऽव॑स॒न्॒. यत्र॒ यत्राव॑स॒न् ततः॑ । \newline
13. अव॑स॒न् तत॒ स्ततो ऽव॑स॒न् नव॑स॒न् ततो॑ ग॒र्मुद् ग॒र्मुत् ततो ऽव॑स॒न् नव॑स॒न् ततो॑ ग॒र्मुत् । \newline
14. ततो॑ ग॒र्मुद् ग॒र्मुत् तत॒ स्ततो॑ ग॒र्मु दुदुद् ग॒र्मुत् तत॒ स्ततो॑ ग॒र्मुदुत् । \newline
15. ग॒र्मु दुदुद् ग॒र्मुद् ग॒र्मुदुद॑ तिष्ठ दतिष्ठ॒दुद् ग॒र्मुद् ग॒र्मु दुद॑तिष्ठत् । \newline
16. उद॑तिष्ठ दतिष्ठ॒ दुदु द॑तिष्ठ॒त् ता स्ता अ॑तिष्ठ॒ दुदु द॑तिष्ठ॒त् ताः । \newline
17. अ॒ति॒ष्ठ॒त् ता स्ता अ॑तिष्ठ दतिष्ठ॒त् ता बृह॒स्पति॒र् बृह॒स्पति॒ स्ता अ॑तिष्ठ दतिष्ठ॒त् ता बृह॒स्पतिः॑ । \newline
18. ता बृह॒स्पति॒र् बृह॒स्पति॒ स्ता स्ता बृह॒स्पति॑श्च च॒ बृह॒स्पति॒ स्ता स्ता बृह॒स्पति॑श्च । \newline
19. बृह॒स्पति॑श्च च॒ बृह॒स्पति॒र् बृह॒स्पति॑ श्चा॒न्ववै॑ता म॒न्ववै॑ताम् च॒ बृह॒स्पति॒र् बृह॒स्पति॑ श्चा॒न्ववै॑ताम् । \newline
20. चा॒न्ववै॑ता म॒न्ववै॑ताम् च चा॒न्ववै॑ताꣳ॒॒ स सो᳚ ऽन्ववै॑ताम् च चा॒न्ववै॑ताꣳ॒॒ सः । \newline
21. अ॒न्ववै॑ताꣳ॒॒ स सो᳚ ऽन्ववै॑ता म॒न्ववै॑ताꣳ॒॒ सो᳚ ऽब्रवी दब्रवी॒थ् सो᳚ ऽन्ववै॑ता म॒न्ववै॑ताꣳ॒॒ सो᳚ ऽब्रवीत् । \newline
22. अ॒न्ववै॑ता॒मित्य॑नु - अवै॑ताम् । \newline
23. सो᳚ ऽब्रवी दब्रवी॒थ् स सो᳚ ऽब्रवी॒द् बृह॒स्पति॒र् बृह॒स्पति॑ रब्रवी॒थ् स सो᳚ ऽब्रवी॒द् बृह॒स्पतिः॑ । \newline
24. अ॒ब्र॒वी॒द् बृह॒स्पति॒र् बृह॒स्पति॑ रब्रवी दब्रवी॒द् बृह॒स्पति॑ र॒नया॒ ऽनया॒ बृह॒स्पति॑ रब्रवी दब्रवी॒द् बृह॒स्पति॑ र॒नया᳚ । \newline
25. बृह॒स्पति॑ र॒नया॒ ऽनया॒ बृह॒स्पति॒र् बृह॒स्पति॑ र॒नया᳚ त्वा त्वा॒ ऽनया॒ बृह॒स्पति॒र् बृह॒स्पति॑ र॒नया᳚ त्वा । \newline
26. अ॒नया᳚ त्वा त्वा॒ ऽनया॒ ऽनया᳚ त्वा॒ प्र प्र त्वा॒ ऽनया॒ ऽनया᳚ त्वा॒ प्र । \newline
27. त्वा॒ प्र प्र त्वा᳚ त्वा॒ प्र ति॑ष्ठानि तिष्ठानि॒ प्र त्वा᳚ त्वा॒ प्र ति॑ष्ठानि । \newline
28. प्र ति॑ष्ठानि तिष्ठानि॒ प्र प्र ति॑ष्ठा॒न्यथाथ॑ तिष्ठानि॒ प्र प्र ति॑ष्ठा॒न्यथ॑ । \newline
29. ति॒ष्ठा॒न्यथाथ॑ तिष्ठानि तिष्ठा॒न्यथ॑ त्वा॒ त्वा ऽथ॑ तिष्ठानि तिष्ठा॒न्यथ॑ त्वा । \newline
30. अथ॑ त्वा॒ त्वा ऽथाथ॑ त्वा प्र॒जाः प्र॒जा स्त्वा ऽथाथ॑ त्वा प्र॒जाः । \newline
31. त्वा॒ प्र॒जाः प्र॒जा स्त्वा᳚ त्वा प्र॒जा उ॒पाव॑र्थ्स्य न्त्यु॒पाव॑र्थ्स्यन्ति प्र॒जा स्त्वा᳚ त्वा प्र॒जा उ॒पाव॑र्थ्स्यन्ति । \newline
32. प्र॒जा उ॒पाव॑र्थ्स्यन् त्यु॒पाव॑र्थ्स्यन्ति प्र॒जाः प्र॒जा उ॒पाव॑र्थ्स्य॒न्तीती त्यु॒पाव॑र्थ्स्यन्ति प्र॒जाः प्र॒जा उ॒पाव॑र्थ्स्य॒न्तीति॑ । \newline
33. प्र॒जा इति॑ प्र - जाः । \newline
34. उ॒पाव॑र्थ्स्य॒न्तीती त्यु॒पाव॑र्थ्स्य न्त्यु॒पाव॑र्थ्स्य॒न्तीति॒ तम् त मित्यु॒पाव॑र्थ्स्य न्त्यु॒पाव॑र्थ्स्य॒न्तीति॒ तम् । \newline
35. उ॒पाव॑र्थ्स्य॒न्तीत्यु॑प - आव॑र्थ्स्यन्ति । \newline
36. इति॒ तम् त मितीति॒ तम् प्र प्र त मितीति॒ तम् प्र । \newline
37. तम् प्र प्र तम् तम् प्राति॑ष्ठ दतिष्ठ॒त् प्र तम् तम् प्राति॑ष्ठत् । \newline
38. प्राति॑ष्ठ दतिष्ठ॒त् प्र प्राति॑ष्ठ॒त् तत॒ स्ततो॑ ऽतिष्ठ॒त् प्र प्राति॑ष्ठ॒त् ततः॑ । \newline
39. अ॒ति॒ष्ठ॒त् तत॒ स्ततो॑ ऽतिष्ठ दतिष्ठ॒त् ततो॒ वै वै ततो॑ ऽतिष्ठ दतिष्ठ॒त् ततो॒ वै । \newline
40. ततो॒ वै वै तत॒ स्ततो॒ वै प्र॒जाप॑तिम् प्र॒जाप॑तिं॒ ॅवै तत॒ स्ततो॒ वै प्र॒जाप॑तिम् । \newline
41. वै प्र॒जाप॑तिम् प्र॒जाप॑तिं॒ ॅवै वै प्र॒जाप॑तिम् प्र॒जाः प्र॒जाः प्र॒जाप॑तिं॒ ॅवै वै प्र॒जाप॑तिम् प्र॒जाः । \newline
42. प्र॒जाप॑तिम् प्र॒जाः प्र॒जाः प्र॒जाप॑तिम् प्र॒जाप॑तिम् प्र॒जा उ॒पाव॑र्तन्तो॒ पाव॑र्तन्त प्र॒जाः प्र॒जाप॑तिम् प्र॒जाप॑तिम् प्र॒जा उ॒पाव॑र्तन्त । \newline
43. प्र॒जाप॑ति॒मिति॑ प्र॒जा - प॒ति॒म् । \newline
44. प्र॒जा उ॒पाव॑र्तन्तो॒ पाव॑र्तन्त प्र॒जाः प्र॒जा उ॒पाव॑र्तन्त॒ यो य उ॒पाव॑र्तन्त प्र॒जाः प्र॒जा उ॒पाव॑र्तन्त॒ यः । \newline
45. प्र॒जा इति॑ प्र - जाः । \newline
46. उ॒पाव॑र्तन्त॒ यो य उ॒पाव॑र्तन्तो॒ पाव॑र्तन्त॒ यः प्र॒जाका॑मः प्र॒जाका॑मो॒ य उ॒पाव॑र्तन्तो॒ पाव॑र्तन्त॒ यः प्र॒जाका॑मः । \newline
47. उ॒पाव॑र्त॒न्तेत्यु॑प - आव॑र्तन्त । \newline
48. यः प्र॒जाका॑मः प्र॒जाका॑मो॒ यो यः प्र॒जाका॑मः॒ स्याथ् स्यात् प्र॒जाका॑मो॒ यो यः प्र॒जाका॑मः॒ स्यात् । \newline
49. प्र॒जाका॑मः॒ स्याथ् स्यात् प्र॒जाका॑मः प्र॒जाका॑मः॒ स्यात् तस्मै॒ तस्मै॒ स्यात् प्र॒जाका॑मः प्र॒जाका॑मः॒ स्यात् तस्मै᳚ । \newline
50. प्र॒जाका॑म॒ इति॑ प्र॒जा - का॒मः॒ । \newline
51. स्यात् तस्मै॒ तस्मै॒ स्याथ् स्यात् तस्मा॑ ए॒त मे॒तम् तस्मै॒ स्याथ् स्यात् तस्मा॑ ए॒तम् । \newline
52. तस्मा॑ ए॒त मे॒तम् तस्मै॒ तस्मा॑ ए॒तम् प्रा॑जाप॒त्यम् प्रा॑जाप॒त्य मे॒तम् तस्मै॒ तस्मा॑ ए॒तम् प्रा॑जाप॒त्यम् । \newline
53. ए॒तम् प्रा॑जाप॒त्यम् प्रा॑जाप॒त्य मे॒त मे॒तम् प्रा॑जाप॒त्यम् गा᳚र्मु॒तम् गा᳚र्मु॒तम् प्रा॑जाप॒त्य मे॒त मे॒तम् प्रा॑जाप॒त्यम् गा᳚र्मु॒तम् । \newline
54. प्रा॒जा॒प॒त्यम् गा᳚र्मु॒तम् गा᳚र्मु॒तम् प्रा॑जाप॒त्यम् प्रा॑जाप॒त्यम् गा᳚र्मु॒तम् च॒रुम् च॒रुम् गा᳚र्मु॒तम् प्रा॑जाप॒त्यम् प्रा॑जाप॒त्यम् गा᳚र्मु॒तम् च॒रुम् । \newline
55. प्रा॒जा॒प॒त्यमिति॑ प्राजा - प॒त्यम् । \newline
56. गा॒र्मु॒तम् च॒रुम् च॒रुम् गा᳚र्मु॒तम् गा᳚र्मु॒तम् च॒रुम् निर् णि श्च॒रुम् गा᳚र्मु॒तम् गा᳚र्मु॒तम् च॒रुम् निः । \newline
57. च॒रुम् निर् णि श्च॒रुम् च॒रुम् निर् व॑पेद् वपे॒न् नि श्च॒रुम् च॒रुम् निर् व॑पेत् । \newline
58. निर् व॑पेद् वपे॒न् निर् णिर् व॑पेत् प्र॒जाप॑तिम् प्र॒जाप॑तिं ॅवपे॒न् निर् णिर् व॑पेत् प्र॒जाप॑तिम् । \newline
59. व॒पे॒त् प्र॒जाप॑तिम् प्र॒जाप॑तिं ॅवपेद् वपेत् प्र॒जाप॑ति मे॒वैव प्र॒जाप॑तिं ॅवपेद् वपेत् प्र॒जाप॑ति मे॒व । \newline
60. प्र॒जाप॑ति मे॒वैव प्र॒जाप॑तिम् प्र॒जाप॑ति मे॒व स्वेन॒ स्वेनै॒व प्र॒जाप॑तिम् प्र॒जाप॑ति मे॒व स्वेन॑ । \newline
61. प्र॒जाप॑ति॒मिति॑ प्र॒जा - प॒ति॒म् । \newline
\pagebreak
\markright{ TS 2.4.4.2  \hfill https://www.vedavms.in \hfill}

\section{ TS 2.4.4.2 }

\textbf{TS 2.4.4.2 } \newline
\textbf{Samhita Paata} \newline

-मे॒व स्वेन॑ भाग॒धेये॒नोप॑ धावति॒ स ए॒वास्मै᳚ प्र॒जां प्रज॑नयतिप्र॒जाप॑तिः प॒शून॑सृजत॒ ते᳚ऽस्माथ् सृ॒ष्टाः परा᳚ञ्च आय॒न् ते यत्राव॑स॒न् ततो॑ ग॒र्मुदुद॑तिष्ठ॒त् तान् पू॒षा चा॒न्ववै॑ताꣳ॒॒ सो᳚ऽब्रवीत् पू॒षाऽनया॑ मा॒ प्रति॒ष्ठाथ॑ त्वा प॒शव॑ उ॒पाव॑र्थ्स्य॒न्तीति॒ मां प्रति॒ष्ठेति॒ सोमो᳚ऽब्रवी॒न् मम॒ वा - [  ] \newline

\textbf{Pada Paata} \newline

ए॒व । स्वेन॑ । भा॒ग॒धेये॒नेति॑ भाग - धेये॑न । उपेति॑ । धा॒व॒ति॒ । सः । ए॒व । अ॒स्मै॒ । प्र॒जामिति॑ प्र - जाम् । प्रेति॑ । ज॒न॒य॒ति॒ । प्र॒जाप॑ति॒रिति॑ प्र॒जा - प॒तिः॒ । प॒शून् । अ॒सृ॒ज॒त॒ । ते । अ॒स्मा॒त् । सृ॒ष्टाः । परा᳚ञ्चः । आ॒य॒न्न् । ते । यत्र॑ । अव॑सन्न् । ततः॑ । ग॒र्मुत् । उदिति॑ । अ॒ति॒ष्ठ॒त् । तान् । पू॒षा । च॒ । अ॒न्ववै॑ता॒मित्य॑नु - अवै॑ताम् । सः । अ॒ब्र॒वी॒त् । पू॒षा । अ॒नया᳚ । मा॒ । प्रेति॑ । ति॒ष्ठ॒ । अथ॑ । त्वा॒ । प॒शवः॑ । उ॒पाव॑र्थ्स्य॒न्तीत्यु॑प - आव॑र्थ्स्यन्ति । इति॑ । माम् । प्रेति॑ । ति॒ष्ठ॒ । इति॑ । सोमः॑ । अ॒ब्र॒वी॒त् । मम॑ । वै ।  \newline


\textbf{Krama Paata} \newline

ए॒व स्वेन॑ । स्वेन॑ भाग॒धेये॑न । भा॒ग॒धेये॒नोप॑ । भा॒ग॒धेये॒नेति॑ भाग - धेये॑न । उप॑ धावति । धा॒व॒ति॒ सः । 
स ए॒व । ए॒वास्मै᳚ । अ॒स्मै॒ प्र॒जाम् । प्र॒जाम् प्र । प्र॒जामिति॑ प्र - जाम् । प्र ज॑नयति । ज॒न॒य॒ति॒ प्र॒जाप॑तिः । प्र॒जाप॑तिः प॒शून् । प्र॒जाप॑ति॒रिति॑ प्र॒जा - प॒तिः॒ । प॒शून॑सृजत । अ॒सृ॒ज॒त॒ ते । ते᳚ ऽस्मात् । अ॒स्मा॒थ् सृ॒ष्टाः । सृ॒ष्टाः परा᳚ञ्चः । परा᳚ञ्च आयन्न् । आ॒य॒न् ते । ते यत्र॑ । यत्राव॑सन्न् । अव॑स॒न् ततः॑ । ततो॑ ग॒र्मुत् । ग॒र्मुदुत् । उद॑तिष्ठत् । अ॒ति॒ष्ठ॒त् तान् । तान् पू॒षा । पू॒षा च॑ । चा॒न्ववै॑ताम् । अ॒न्ववै॑ताꣳ॒॒ सः । अ॒न्ववै॑ता॒मित्य॑नु - अवै॑ताम् । सो᳚ ऽब्रवीत् । अ॒ब्र॒वी॒त् पू॒षा । पू॒षा ऽनया᳚ । अ॒नया॑ मा । मा॒ प्र । प्र ति॑ष्ठ । ति॒ष्ठाथ॑ । अथ॑ त्वा । त्वा॒ प॒शवः॑ । प॒शव॑ उ॒पाव॑र्त्थ्स्यन्ति । उ॒पाव॑र्त्थ्स्य॒न्तीति॑ । उ॒पाव॑र्त्थ्स्य॒न्तीत्यु॑प - आव॑र्त्थ्स्यन्ति । इति॒ माम् । माम् प्र । प्र ति॑ष्ठ । ति॒ष्ठेति॑ । इति॒ सोमः॑ । सोमो᳚ ऽब्रवीत् । अ॒ब्र॒वी॒न् मम॑ । मम॒ वै । वा अ॑कृष्टप॒च्यम् \newline

\textbf{Jatai Paata} \newline

1. ए॒व स्वेन॒ स्वेनै॒वैव स्वेन॑ । \newline
2. स्वेन॑ भाग॒धेये॑न भाग॒धेये॑न॒ स्वेन॒ स्वेन॑ भाग॒धेये॑न । \newline
3. भा॒ग॒धेये॒नो पोप॑ भाग॒धेये॑न भाग॒धेये॒नोप॑ । \newline
4. भा॒ग॒धेये॒नेति॑ भाग - धेये॑न । \newline
5. उप॑ धावति धाव॒ त्युपोप॑ धावति । \newline
6. धा॒व॒ति॒ स स धा॑वति धावति॒ सः । \newline
7. स ए॒वैव स स ए॒व । \newline
8. ए॒वास्मा॑ अस्मा ए॒वैवास्मै᳚ । \newline
9. अ॒स्मै॒ प्र॒जाम् प्र॒जा म॑स्मा अस्मै प्र॒जाम् । \newline
10. प्र॒जाम् प्र प्र प्र॒जाम् प्र॒जाम् प्र । \newline
11. प्र॒जामिति॑ प्र - जाम् । \newline
12. प्र ज॑नयति जनयति॒ प्र प्र ज॑नयति । \newline
13. ज॒न॒य॒ति॒ प्र॒जाप॑तिः प्र॒जाप॑तिर् जनयति जनयति प्र॒जाप॑तिः । \newline
14. प्र॒जाप॑तिः प॒शून् प॒शून् प्र॒जाप॑तिः प्र॒जाप॑तिः प॒शून् । \newline
15. प्र॒जाप॑ति॒रिति॑ प्र॒जा - प॒तिः॒ । \newline
16. प॒शू न॑सृजता सृजत प॒शून् प॒शू न॑सृजत । \newline
17. अ॒सृ॒ज॒त॒ ते ते॑ ऽसृजता सृजत॒ ते । \newline
18. ते᳚ ऽस्मा दस्मा॒त् ते ते᳚ ऽस्मात् । \newline
19. अ॒स्मा॒थ् सृ॒ष्टाः सृ॒ष्टा अ॑स्मा दस्माथ् सृ॒ष्टाः । \newline
20. सृ॒ष्टाः परा᳚ञ्चः॒ परा᳚ञ्चः सृ॒ष्टाः सृ॒ष्टाः परा᳚ञ्चः । \newline
21. परा᳚ञ्च आयन् नाय॒न् परा᳚ञ्चः॒ परा᳚ञ्च आयन्न् । \newline
22. आ॒य॒न् ते त आ॑यन् नाय॒न् ते । \newline
23. ते यत्र॒ यत्र॒ ते ते यत्र॑ । \newline
24. यत्रा व॑स॒न् नव॑स॒न्॒. यत्र॒ यत्रा व॑सन्न् । \newline
25. अव॑स॒न् तत॒ स्ततो ऽव॑स॒न् नव॑स॒न् ततः॑ । \newline
26. ततो॑ ग॒र्मुद् ग॒र्मुत् तत॒ स्ततो॑ ग॒र्मुत् । \newline
27. ग॒र्मु दुदुद् ग॒र्मुद् ग॒र्मु दुत् । \newline
28. उद॑तिष्ठ दतिष्ठ॒ दुदु द॑तिष्ठत् । \newline
29. अ॒ति॒ष्ठ॒त् ताꣳ स्ता न॑तिष्ठ दतिष्ठ॒त् तान् । \newline
30. तान् पू॒षा पू॒षा ताꣳ स्तान् पू॒षा । \newline
31. पू॒षा च॑ च पू॒षा पू॒षा च॑ । \newline
32. चा॒न्ववै॑ता म॒न्ववै॑ताम् च चा॒न्ववै॑ताम् । \newline
33. अ॒न्ववै॑ताꣳ॒॒ स सो᳚ ऽन्ववै॑ता म॒न्ववै॑ताꣳ॒॒ सः । \newline
34. अ॒न्ववै॑ता॒मित्य॑नु - अवै॑ताम् । \newline
35. सो᳚ ऽब्रवी दब्रवी॒थ् स सो᳚ ऽब्रवीत् । \newline
36. अ॒ब्र॒वी॒त् पू॒षा पू॒षा ऽब्र॑वी दब्रवीत् पू॒षा । \newline
37. पू॒षा ऽनया॒ ऽनया॑ पू॒षा पू॒षा ऽनया᳚ । \newline
38. अ॒नया॑ मा मा॒ ऽनया॒ ऽनया॑ मा । \newline
39. मा॒ प्र प्र मा॑ मा॒ प्र । \newline
40. प्र ति॑ष्ठ तिष्ठ॒ प्र प्र ति॑ष्ठ । \newline
41. ति॒ष्ठाथाथ॑ तिष्ठ ति॒ष्ठाथ॑ । \newline
42. अथ॑ त्वा॒ त्वा ऽथाथ॑ त्वा । \newline
43. त्वा॒ प॒शवः॑ प॒शव॑ स्त्वा त्वा प॒शवः॑ । \newline
44. प॒शव॑ उ॒पाव॑र्थ्स्य न्त्यु॒पाव॑र्थ्स्यन्ति प॒शवः॑ प॒शव॑ उ॒पाव॑र्थ्स्यन्ति । \newline
45. उ॒पाव॑र्थ्स्य॒न्तीती त्यु॒पाव॑र्थ्स्य न्त्यु॒पाव॑र्थ्स्य॒न्तीति॑ । \newline
46. उ॒पाव॑र्थ्स्य॒न्तीत्यु॑प - आव॑र्थ्स्यन्ति । \newline
47. इति॒ माम् मा मितीति॒ माम् । \newline
48. माम् प्र प्र माम् माम् प्र । \newline
49. प्र ति॑ष्ठ तिष्ठ॒ प्र प्र ति॑ष्ठ । \newline
50. ति॒ष्ठे तीति॑ तिष्ठ ति॒ष्ठे ति॑ । \newline
51. इति॒ सोमः॒ सोम॒ इतीति॒ सोमः॑ । \newline
52. सोमो᳚ ऽब्रवी दब्रवी॒थ् सोमः॒ सोमो᳚ ऽब्रवीत् । \newline
53. अ॒ब्र॒वी॒न् मम॒ ममा᳚ब्रवी दब्रवी॒न् मम॑ । \newline
54. मम॒ वै वै मम॒ मम॒ वै । \newline
55. वा अ॑कृष्टप॒च्य म॑कृष्टप॒च्यं ॅवै वा अ॑कृष्टप॒च्यम् । \newline

\textbf{Ghana Paata } \newline

1. ए॒व स्वेन॒ स्वेनै॒वैव स्वेन॑ भाग॒धेये॑न भाग॒धेये॑न॒ स्वेनै॒वैव स्वेन॑ भाग॒धेये॑न । \newline
2. स्वेन॑ भाग॒धेये॑न भाग॒धेये॑न॒ स्वेन॒ स्वेन॑ भाग॒धेये॒नोपोप॑ भाग॒धेये॑न॒ स्वेन॒ स्वेन॑ भाग॒धेये॒नोप॑ । \newline
3. भा॒ग॒धेये॒नोपोप॑ भाग॒धेये॑न भाग॒धेये॒नोप॑ धावति धाव॒त्युप॑ भाग॒धेये॑न भाग॒धेये॒नोप॑ धावति । \newline
4. भा॒ग॒धेये॒नेति॑ भाग - धेये॑न । \newline
5. उप॑ धावति धाव॒ त्युपोप॑ धावति॒ स स धा॑व॒ त्युपोप॑ धावति॒ सः । \newline
6. धा॒व॒ति॒ स स धा॑वति धावति॒ स ए॒वैव स धा॑वति धावति॒ स ए॒व । \newline
7. स ए॒वैव स स ए॒वास्मा॑ अस्मा ए॒व स स ए॒वास्मै᳚ । \newline
8. ए॒वास्मा॑ अस्मा ए॒वैवास्मै᳚ प्र॒जाम् प्र॒जा म॑स्मा ए॒वैवास्मै᳚ प्र॒जाम् । \newline
9. अ॒स्मै॒ प्र॒जाम् प्र॒जा म॑स्मा अस्मै प्र॒जाम् प्र प्र प्र॒जा म॑स्मा अस्मै प्र॒जाम् प्र । \newline
10. प्र॒जाम् प्र प्र प्र॒जाम् प्र॒जाम् प्र ज॑नयति जनयति॒ प्र प्र॒जाम् प्र॒जाम् प्र ज॑नयति । \newline
11. प्र॒जामिति॑ प्र - जाम् । \newline
12. प्र ज॑नयति जनयति॒ प्र प्र ज॑नयति प्र॒जाप॑तिः प्र॒जाप॑तिर् जनयति॒ प्र प्र ज॑नयति प्र॒जाप॑तिः । \newline
13. ज॒न॒य॒ति॒ प्र॒जाप॑तिः प्र॒जाप॑तिर् जनयति जनयति प्र॒जाप॑तिः प॒शून् प॒शून् प्र॒जाप॑तिर् जनयति जनयति प्र॒जाप॑तिः प॒शून् । \newline
14. प्र॒जाप॑तिः प॒शून् प॒शून् प्र॒जाप॑तिः प्र॒जाप॑तिः प॒शू न॑सृजता सृजत प॒शून् प्र॒जाप॑तिः प्र॒जाप॑तिः प॒शू न॑सृजत । \newline
15. प्र॒जाप॑ति॒रिति॑ प्र॒जा - प॒तिः॒ । \newline
16. प॒शू न॑सृजता सृजत प॒शून् प॒शू न॑सृजत॒ ते ते॑ ऽसृजत प॒शून् प॒शू न॑सृजत॒ ते । \newline
17. अ॒सृ॒ज॒त॒ ते ते॑ ऽसृजता सृजत॒ ते᳚ ऽस्मा दस्मा॒त् ते॑ ऽसृजता सृजत॒ ते᳚ ऽस्मात् । \newline
18. ते᳚ ऽस्मा दस्मा॒त् ते ते᳚ ऽस्माथ् सृ॒ष्टाः सृ॒ष्टा अ॑स्मा॒त् ते ते᳚ ऽस्माथ् सृ॒ष्टाः । \newline
19. अ॒स्मा॒थ् सृ॒ष्टाः सृ॒ष्टा अ॑स्मा दस्माथ् सृ॒ष्टाः परा᳚ञ्चः॒ परा᳚ञ्चः सृ॒ष्टा अ॑स्मा दस्माथ् सृ॒ष्टाः परा᳚ञ्चः । \newline
20. सृ॒ष्टाः परा᳚ञ्चः॒ परा᳚ञ्चः सृ॒ष्टाः सृ॒ष्टाः परा᳚ञ्च आयन् नाय॒न् परा᳚ञ्चः सृ॒ष्टाः सृ॒ष्टाः परा᳚ञ्च आयन्न् । \newline
21. परा᳚ञ्च आयन् नाय॒न् परा᳚ञ्चः॒ परा᳚ञ्च आय॒न् ते त आ॑य॒न् परा᳚ञ्चः॒ परा᳚ञ्च आय॒न् ते । \newline
22. आ॒य॒न् ते त आ॑यन् नाय॒न् ते यत्र॒ यत्र॒ त आ॑यन् नाय॒न् ते यत्र॑ । \newline
23. ते यत्र॒ यत्र॒ ते ते यत्राव॑स॒न् नव॑स॒न्॒. यत्र॒ ते ते यत्राव॑सन्न् । \newline
24. यत्राव॑स॒न् नव॑स॒न्॒. यत्र॒ यत्राव॑स॒न् तत॒ स्ततो ऽव॑स॒न्॒. यत्र॒ यत्राव॑स॒न् ततः॑ । \newline
25. अव॑स॒न् तत॒ स्ततो ऽव॑स॒न् नव॑स॒न् ततो॑ ग॒र्मुद् ग॒र्मुत् ततो ऽव॑स॒न् नव॑स॒न् ततो॑ ग॒र्मुत् । \newline
26. ततो॑ ग॒र्मुद् ग॒र्मुत् तत॒ स्ततो॑ ग॒र्मुदुदुद् ग॒र्मुत् तत॒ स्ततो॑ ग॒र्मुदुत् । \newline
27. ग॒र्मु दुदुद् ग॒र्मुद् ग॒र्मु दुद॑तिष्ठ दतिष्ठ॒ दुद् ग॒र्मुद् ग॒र्मु दुद॑तिष्ठत् । \newline
28. उद॑तिष्ठ दतिष्ठ॒ दुदुद॑तिष्ठ॒त् ताꣳ स्ता न॑तिष्ठ॒ दुदुद॑तिष्ठ॒त् तान् । \newline
29. अ॒ति॒ष्ठ॒त् ताꣳ स्ता न॑तिष्ठ दतिष्ठ॒त् तान् पू॒षा पू॒षा ता न॑तिष्ठ दतिष्ठ॒त् तान् पू॒षा । \newline
30. तान् पू॒षा पू॒षा ताꣳ स्तान् पू॒षा च॑ च पू॒षा ताꣳ स्तान् पू॒षा च॑ । \newline
31. पू॒षा च॑ च पू॒षा पू॒षा चा॒न्ववै॑ता म॒न्ववै॑ताम् च पू॒षा पू॒षा चा॒न्ववै॑ताम् । \newline
32. चा॒न्ववै॑ता म॒न्ववै॑ताम् च चा॒न्ववै॑ताꣳ॒॒ स सो᳚ ऽन्ववै॑ताम् च चा॒न्ववै॑ताꣳ॒॒ सः । \newline
33. अ॒न्ववै॑ताꣳ॒॒ स सो᳚ ऽन्ववै॑ता म॒न्ववै॑ताꣳ॒॒ सो᳚ ऽब्रवी दब्रवी॒थ् सो᳚ ऽन्ववै॑ता म॒न्ववै॑ताꣳ॒॒ सो᳚ ऽब्रवीत् । \newline
34. अ॒न्ववै॑ता॒मित्य॑नु - अवै॑ताम् । \newline
35. सो᳚ ऽब्रवी दब्रवी॒थ् स सो᳚ ऽब्रवीत् पू॒षा पू॒षा ऽब्र॑वी॒थ् स सो᳚ ऽब्रवीत् पू॒षा । \newline
36. अ॒ब्र॒वी॒त् पू॒षा पू॒षा ऽब्र॑वी दब्रवीत् पू॒षा ऽनया॒ ऽनया॑ पू॒षा ऽब्र॑वी दब्रवीत् पू॒षा ऽनया᳚ । \newline
37. पू॒षा ऽनया॒ ऽनया॑ पू॒षा पू॒षा ऽनया॑ मा मा॒ ऽनया॑ पू॒षा पू॒षा ऽनया॑ मा । \newline
38. अ॒नया॑ मा मा॒ ऽनया॒ ऽनया॑ मा॒ प्र प्र मा॒ ऽनया॒ ऽनया॑ मा॒ प्र । \newline
39. मा॒ प्र प्र मा॑ मा॒ प्र ति॑ष्ठ तिष्ठ॒ प्र मा॑ मा॒ प्र ति॑ष्ठ । \newline
40. प्र ति॑ष्ठ तिष्ठ॒ प्र प्र ति॒ष्ठाथाथ॑ तिष्ठ॒ प्र प्र ति॒ष्ठाथ॑ । \newline
41. ति॒ष्ठाथाथ॑ तिष्ठ ति॒ष्ठाथ॑ त्वा॒ त्वा ऽथ॑ तिष्ठ ति॒ष्ठाथ॑ त्वा । \newline
42. अथ॑ त्वा॒ त्वा ऽथाथ॑ त्वा प॒शवः॑ प॒शव॒ स्त्वा ऽथाथ॑ त्वा प॒शवः॑ । \newline
43. त्वा॒ प॒शवः॑ प॒शव॑ स्त्वा त्वा प॒शव॑ उ॒पाव॑र्थ्स्य न्त्यु॒पाव॑र्थ्स्यन्ति प॒शव॑ स्त्वा त्वा प॒शव॑ उ॒पाव॑र्थ्स्यन्ति । \newline
44. प॒शव॑ उ॒पाव॑र्थ्स्य न्त्यु॒पाव॑र्थ्स्यन्ति प॒शवः॑ प॒शव॑ उ॒पाव॑र्थ्स्य॒न्तीती त्यु॒पाव॑र्थ्स्यन्ति प॒शवः॑ प॒शव॑ उ॒पाव॑र्थ्स्य॒न्तीति॑ । \newline
45. उ॒पाव॑र्थ्स्य॒न्तीती त्यु॒पाव॑र्थ्स्य न्त्यु॒पाव॑र्थ्स्य॒न्तीति॒ माम् मा मित्यु॒पाव॑र्थ्स्य न्त्यु॒पाव॑र्थ्स्य॒न्तीति॒ माम् । \newline
46. उ॒पाव॑र्थ्स्य॒न्तीत्यु॑प - आव॑र्थ्स्यन्ति । \newline
47. इति॒ माम् मा मितीति॒ माम् प्र प्र मा मितीति॒ माम् प्र । \newline
48. माम् प्र प्र माम् माम् प्र ति॑ष्ठ तिष्ठ॒ प्र माम् माम् प्र ति॑ष्ठ । \newline
49. प्र ति॑ष्ठ तिष्ठ॒ प्र प्र ति॒ष्ठे तीति॑ तिष्ठ॒ प्र प्र ति॒ष्ठे ति॑ । \newline
50. ति॒ष्ठे तीति॑ तिष्ठ ति॒ष्ठे ति॒ सोमः॒ सोम॒ इति॑ तिष्ठ ति॒ष्ठे ति॒ सोमः॑ । \newline
51. इति॒ सोमः॒ सोम॒ इतीति॒ सोमो᳚ ऽब्रवी दब्रवी॒थ् सोम॒ इतीति॒ सोमो᳚ ऽब्रवीत् । \newline
52. सोमो᳚ ऽब्रवी दब्रवी॒थ् सोमः॒ सोमो᳚ ऽब्रवी॒न् मम॒ ममा᳚ब्रवी॒थ् सोमः॒ सोमो᳚ ऽब्रवी॒न् मम॑ । \newline
53. अ॒ब्र॒वी॒न् मम॒ ममा᳚ब्रवी दब्रवी॒न् मम॒ वै वै ममा᳚ब्रवी दब्रवी॒न् मम॒ वै । \newline
54. मम॒ वै वै मम॒ मम॒ वा अ॑कृष्टप॒च्य म॑कृष्टप॒च्यं ॅवै मम॒ मम॒ वा अ॑कृष्टप॒च्यम् । \newline
55. वा अ॑कृष्टप॒च्य म॑कृष्टप॒च्यं ॅवै वा अ॑कृष्टप॒च्य मितीत्य॑कृष्टप॒च्यं ॅवै वा अ॑कृष्टप॒च्य मिति॑ । \newline
\pagebreak
\markright{ TS 2.4.4.3  \hfill https://www.vedavms.in \hfill}

\section{ TS 2.4.4.3 }

\textbf{TS 2.4.4.3 } \newline
\textbf{Samhita Paata} \newline

अ॑कृष्टप॒च्यमित्यु॒भौ वां॒ प्रति॑ष्ठा॒नीत्य॑ब्रवी॒त् तौ प्राति॑ष्ठ॒त् ततो॒ वै प्र॒जाप॑तिं प॒शव॑ उ॒पाव॑र्तन्त॒ यः प॒शुका॑मः॒ स्यात् तस्मा॑ ए॒तꣳ सो॑मापौ॒ष्णं गा᳚र्मु॒तं च॒रुं निर्व॑पेथ् सोमापू॒षणा॑वे॒व स्वेन॑ भाग॒धेये॒नोप॑ धावति॒ तावे॒वास्मै॑ प॒शून् प्रज॑नयतः॒ सोमो॒ वै रे॑तो॒धाः पू॒षा प॑शू॒नां प्र॑जनयि॒ता सोम॑ ए॒वास्मै॒ रेतो॒ दधा॑ति पू॒षा ( ) प॒शून् प्रज॑नयति ॥ \newline

\textbf{Pada Paata} \newline

अ॒कृ॒ष्ट॒प॒च्यमित्य॑कृष्ट - प॒च्यम् । इति॑ । उ॒भौ । वा॒म् । प्रेति॑ । ति॒ष्ठा॒नि॒ । इति॑ । अ॒ब्र॒वी॒त् । तौ । प्रेति॑ । अ॒ति॒ष्ठ॒त् । ततः॑ । वै । प्र॒जाप॑ति॒मिति॑ प्र॒जा - प॒ति॒म् । प॒शवः॑ । उ॒पाव॑र्त॒न्तेत्यु॑प-आव॑र्तन्त । यः । प॒शुका॑म॒ इति॑ प॒शु - का॒मः॒ । स्यात् । तस्मै᳚ । ए॒तम् । सो॒मा॒पौ॒ष्णमिति॑ सोमा - पौ॒ष्णम् । गा॒र्मु॒तम् । च॒रुम् । निरिति॑ । व॒पे॒त् । सो॒मा॒पू॒षणा॒विति॑ सोमा - पू॒षणौ᳚ । ए॒व । स्वेन॑ । भा॒ग॒धेये॒नेति॑ भाग- धेये॑न । उपेति॑ । धा॒व॒ति॒ । तौ । ए॒व । अ॒स्मै॒ । प॒शून् । प्रेति॑ । ज॒न॒य॒तः॒ । सोमः॑ । वै । रे॒तो॒धा इति॑ रेतः - धाः । पू॒षा । प॒शू॒नाम् । प्र॒ज॒न॒यि॒तेति॑ प्र - ज॒न॒यि॒ता । सोमः॑ । ए॒व । अ॒स्मै॒ । रेतः॑ । दधा॑ति । पू॒षा ( ) । प॒शून् । प्रेति॑ । ज॒न॒य॒ति॒ ॥  \newline


\textbf{Krama Paata} \newline

अ॒कृ॒ष्ट॒प॒च्यमिति॑ । अ॒कृ॒ष्ट॒प॒च्यमित्य॑कृष्ट - प॒च्यम् । इत्यु॒भौ । उ॒भौ वा᳚म् । वा॒म् प्र । प्र ति॑ष्ठानि । ति॒ष्ठा॒नीति॑ । 
इत्य॑ब्रवीत् । अ॒ब्र॒वी॒त् तौ । तौ प्र । प्राति॑ष्ठत् । अ॒ति॒ष्ठ॒त् ततः॑ । ततो॒ वै । वै प्र॒जाप॑तिम् । प्र॒जाप॑तिम् प॒शवः॑ । प्र॒जाप॑ति॒मिति॑ प्र॒जा - प॒ति॒म् । प॒शव॑ उ॒पाव॑र्तन्त । उ॒पाव॑र्तन्त॒ यः । उ॒पाव॑र्त॒न्तेत्यु॑प - आव॑र्तन्त । यः प॒शुका॑मः । प॒शुका॑मः॒ 
स्यात् । प॒शुका॑म॒ इति॑ प॒शु - का॒मः॒ । स्यात् तस्मै᳚ । तस्मा॑ ए॒तम् । ए॒तꣳ सो॑मापौ॒ष्णम् । सो॒मा॒पौ॒ष्णम् गा᳚र्मु॒तम् । सो॒मा॒पौ॒ष्णमिति॑ सोमा - पौ॒ष्णम् । गा॒र्मु॒तम् च॒रुम् । च॒रुम् निः । निर् व॑पेत् । व॒पे॒थ् सो॒मा॒पू॒षणौ᳚ । सो॒मा॒पू॒षणा॑वे॒व । सो॒मा॒पू॒षणा॒विति॑ सोमा - पू॒षणौ᳚ । ए॒व स्वेन॑ । स्वेन॑ भाग॒धेये॑न । भा॒ग॒धेये॒नोप॑ । भा॒ग॒धेये॒नेति॑ भाग - धेये॑न । उप॑ धावति । धा॒व॒ति॒ तौ । तावे॒व । ए॒वास्मै᳚ । अ॒स्मै॒ प॒शून् । प॒शून् प्र । प्र ज॑नयतः । ज॒न॒य॒तः॒ सोमः॑ । सोमो॒ वै । वै रे॑तो॒धाः । रे॒तो॒धाः पू॒षा । रे॒तो॒धा इति॑ रेतः - धाः । पू॒षा प॑शू॒नाम् । प॒शू॒नाम् प्र॑जनयि॒ता । प्र॒ज॒न॒यि॒ता सोमः॑ । प्र॒ज॒न॒यि॒तेति॑ प्र - ज॒न॒यि॒ता । सोम॑ ए॒व । ए॒वास्मै᳚ । अ॒स्मै॒ रेतः॑ । रेतो॒ दधा॑ति । दधा॑ति पू॒षा ( ) । पू॒षा प॒शून् । प॒शून् प्र । प्र ज॑नयति । ज॒न॒य॒तीति॑ जनयति । \newline

\textbf{Jatai Paata} \newline

1. अ॒कृ॒ष्ट॒प॒च्य मिती त्य॑कृष्टप॒च्य म॑कृष्टप॒च्य मिति॑ । \newline
2. अ॒कृ॒ष्ट॒प॒च्यमित्य॑कृष्ट - प॒च्यम् । \newline
3. इत्यु॒भा वु॒भा विती त्यु॒भौ । \newline
4. उ॒भौ वां᳚ ॅवा मु॒भा वु॒भौ वा᳚म् । \newline
5. वा॒म् प्र प्र वां᳚ ॅवा॒म् प्र । \newline
6. प्र ति॑ष्ठानि तिष्ठानि॒ प्र प्र ति॑ष्ठानि । \newline
7. ति॒ष्ठा॒नीतीति॑ तिष्ठानि तिष्ठा॒नीति॑ । \newline
8. इत्य॑ब्रवी दब्रवी॒ दिती त्य॑ब्रवीत् । \newline
9. अ॒ब्र॒वी॒त् तौ ता व॑ब्रवी दब्रवी॒त् तौ । \newline
10. तौ प्र प्र तौ तौ प्र । \newline
11. प्राति॑ष्ठ दतिष्ठ॒त् प्र प्राति॑ष्ठत् । \newline
12. अ॒ति॒ष्ठ॒त् तत॒ स्ततो॑ ऽतिष्ठ दतिष्ठ॒त् ततः॑ । \newline
13. ततो॒ वै वै तत॒ स्ततो॒ वै । \newline
14. वै प्र॒जाप॑तिम् प्र॒जाप॑तिं॒ ॅवै वै प्र॒जाप॑तिम् । \newline
15. प्र॒जाप॑तिम् प॒शवः॑ प॒शवः॑ प्र॒जाप॑तिम् प्र॒जाप॑तिम् प॒शवः॑ । \newline
16. प्र॒जाप॑ति॒मिति॑ प्र॒जा - प॒ति॒म् । \newline
17. प॒शव॑ उ॒पाव॑र्तन्तो॒ पाव॑र्तन्त प॒शवः॑ प॒शव॑ उ॒पाव॑र्तन्त । \newline
18. उ॒पाव॑र्तन्त॒ यो य उ॒पाव॑र्तन्तो॒ पाव॑र्तन्त॒ यः । \newline
19. उ॒पाव॑र्त॒न्तेत्यु॑प - आव॑र्तन्त । \newline
20. यः प॒शुका॑मः प॒शुका॑मो॒ यो यः प॒शुका॑मः । \newline
21. प॒शुका॑मः॒ स्याथ् स्यात् प॒शुका॑मः प॒शुका॑मः॒ स्यात् । \newline
22. प॒शुका॑म॒ इति॑ प॒शु - का॒मः॒ । \newline
23. स्यात् तस्मै॒ तस्मै॒ स्याथ् स्यात् तस्मै᳚ । \newline
24. तस्मा॑ ए॒त मे॒तम् तस्मै॒ तस्मा॑ ए॒तम् । \newline
25. ए॒तꣳ सो॑मापौ॒ष्णꣳ सो॑मापौ॒ष्ण मे॒त मे॒तꣳ सो॑मापौ॒ष्णम् । \newline
26. सो॒मा॒पौ॒ष्णम् गा᳚र्मु॒तम् गा᳚र्मु॒तꣳ सो॑मापौ॒ष्णꣳ सो॑मापौ॒ष्णम् गा᳚र्मु॒तम् । \newline
27. सो॒मा॒पौ॒ष्णमिति॑ सोमा - पौ॒ष्णम् । \newline
28. गा॒र्मु॒तम् च॒रुम् च॒रुम् गा᳚र्मु॒तम् गा᳚र्मु॒तम् च॒रुम् । \newline
29. च॒रुम् निर् णि श्च॒रुम् च॒रुम् निः । \newline
30. निर् व॑पेद् वपे॒न् निर् णिर् व॑पेत् । \newline
31. व॒पे॒थ् सो॒मा॒पू॒षणौ॑ सोमापू॒षणौ॑ वपेद् वपेथ् सोमापू॒षणौ᳚ । \newline
32. सो॒मा॒पू॒षणा॑ वे॒वैव सो॑मापू॒षणौ॑ सोमापू॒षणा॑ वे॒व । \newline
33. सो॒मा॒पू॒षणा॒विति॑ सोमा - पू॒षणौ᳚ । \newline
34. ए॒व स्वेन॒ स्वेनै॒वैव स्वेन॑ । \newline
35. स्वेन॑ भाग॒धेये॑न भाग॒धेये॑न॒ स्वेन॒ स्वेन॑ भाग॒धेये॑न । \newline
36. भा॒ग॒धेये॒नोपोप॑ भाग॒धेये॑न भाग॒धेये॒नोप॑ । \newline
37. भा॒ग॒धेये॒नेति॑ भाग - धेये॑न । \newline
38. उप॑ धावति धाव॒ त्युपोप॑ धावति । \newline
39. धा॒व॒ति॒ तौ तौ धा॑वति धावति॒ तौ । \newline
40. ता वे॒वैव तौ ता वे॒व । \newline
41. ए॒वास्मा॑ अस्मा ए॒वैवास्मै᳚ । \newline
42. अ॒स्मै॒ प॒शून् प॒शू न॑स्मा अस्मै प॒शून् । \newline
43. प॒शून् प्र प्र प॒शून् प॒शून् प्र । \newline
44. प्र ज॑नयतो जनयतः॒ प्र प्र ज॑नयतः । \newline
45. ज॒न॒य॒तः॒ सोमः॒ सोमो॑ जनयतो जनयतः॒ सोमः॑ । \newline
46. सोमो॒ वै वै सोमः॒ सोमो॒ वै । \newline
47. वै रे॑तो॒धा रे॑तो॒धा वै वै रे॑तो॒धाः । \newline
48. रे॒तो॒धाः पू॒षा पू॒षा रे॑तो॒धा रे॑तो॒धाः पू॒षा । \newline
49. रे॒तो॒धा इति॑ रेतः - धाः । \newline
50. पू॒षा प॑शू॒नाम् प॑शू॒नाम् पू॒षा पू॒षा प॑शू॒नाम् । \newline
51. प॒शू॒नाम् प्र॑जनयि॒ता प्र॑जनयि॒ता प॑शू॒नाम् प॑शू॒नाम् प्र॑जनयि॒ता । \newline
52. प्र॒ज॒न॒यि॒ता सोमः॒ सोमः॑ प्रजनयि॒ता प्र॑जनयि॒ता सोमः॑ । \newline
53. प्र॒ज॒न॒यि॒तेति॑ प्र - ज॒न॒यि॒ता । \newline
54. सोम॑ ए॒वैव सोमः॒ सोम॑ ए॒व । \newline
55. ए॒वास्मा॑ अस्मा ए॒वैवास्मै᳚ । \newline
56. अ॒स्मै॒ रेतो॒ रेतो᳚ ऽस्मा अस्मै॒ रेतः॑ । \newline
57. रेतो॒ दधा॑ति॒ दधा॑ति॒ रेतो॒ रेतो॒ दधा॑ति । \newline
58. दधा॑ति पू॒षा पू॒षा दधा॑ति॒ दधा॑ति पू॒षा । \newline
59. पू॒षा प॒शून् प॒शून् पू॒षा पू॒षा प॒शून् । \newline
60. प॒शून् प्र प्र प॒शून् प॒शून् प्र । \newline
61. प्र ज॑नयति जनयति॒ प्र प्र ज॑नयति । \newline
62. ज॒न॒य॒तीति॑ जनयति । \newline

\textbf{Ghana Paata } \newline

1. अ॒कृ॒ष्ट॒प॒च्य मिती त्य॑कृष्टप॒च्य म॑कृष्टप॒च्य मित्यु॒भा वु॒भा वित्य॑कृष्टप॒च्य म॑कृष्टप॒च्य मित्यु॒भौ । \newline
2. अ॒कृ॒ष्ट॒प॒च्यमित्य॑कृष्ट - प॒च्यम् । \newline
3. इत्यु॒भा वु॒भा विती त्यु॒भौ वां᳚ ॅवा मु॒भा विती त्यु॒भौ वा᳚म् । \newline
4. उ॒भौ वां᳚ ॅवा मु॒भा वु॒भौ वा॒म् प्र प्र वा॑ मु॒भा वु॒भौ वा॒म् प्र । \newline
5. वा॒म् प्र प्र वां᳚ ॅवा॒म् प्र ति॑ष्ठानि तिष्ठानि॒ प्र वां᳚ ॅवा॒म् प्र ति॑ष्ठानि । \newline
6. प्र ति॑ष्ठानि तिष्ठानि॒ प्र प्र ति॑ष्ठा॒नीतीति॑ तिष्ठानि॒ प्र प्र ति॑ष्ठा॒नीति॑ । \newline
7. ति॒ष्ठा॒नीतीति॑ तिष्ठानि तिष्ठा॒नी त्य॑ब्रवी दब्रवी॒ दिति॑ तिष्ठानि तिष्ठा॒नी त्य॑ब्रवीत् । \newline
8. इत्य॑ब्रवी दब्रवी॒ दितीत्य॑ब्रवी॒त् तौ ता व॑ब्रवी॒ दितीत्य॑ब्रवी॒त् तौ । \newline
9. अ॒ब्र॒वी॒त् तौ ता व॑ब्रवी दब्रवी॒त् तौ प्र प्र ता व॑ब्रवी दब्रवी॒त् तौ प्र । \newline
10. तौ प्र प्र तौ तौ प्राति॑ष्ठ दतिष्ठ॒त् प्र तौ तौ प्राति॑ष्ठत् । \newline
11. प्राति॑ष्ठ दतिष्ठ॒त् प्र प्राति॑ष्ठ॒त् तत॒ स्ततो॑ ऽतिष्ठ॒त् प्र प्राति॑ष्ठ॒त् ततः॑ । \newline
12. अ॒ति॒ष्ठ॒त् तत॒ स्ततो॑ ऽतिष्ठ दतिष्ठ॒त् ततो॒ वै वै ततो॑ ऽतिष्ठ दतिष्ठ॒त् ततो॒ वै । \newline
13. ततो॒ वै वै तत॒ स्ततो॒ वै प्र॒जाप॑तिम् प्र॒जाप॑तिं॒ ॅवै तत॒ स्ततो॒ वै प्र॒जाप॑तिम् । \newline
14. वै प्र॒जाप॑तिम् प्र॒जाप॑तिं॒ ॅवै वै प्र॒जाप॑तिम् प॒शवः॑ प॒शवः॑ प्र॒जाप॑तिं॒ ॅवै वै प्र॒जाप॑तिम् प॒शवः॑ । \newline
15. प्र॒जाप॑तिम् प॒शवः॑ प॒शवः॑ प्र॒जाप॑तिम् प्र॒जाप॑तिम् प॒शव॑ उ॒पाव॑र्तन्तो॒ पाव॑र्तन्त प॒शवः॑ प्र॒जाप॑तिम् प्र॒जाप॑तिम् प॒शव॑ उ॒पाव॑र्तन्त । \newline
16. प्र॒जाप॑ति॒मिति॑ प्र॒जा - प॒ति॒म् । \newline
17. प॒शव॑ उ॒पाव॑र्तन्तो॒ पाव॑र्तन्त प॒शवः॑ प॒शव॑ उ॒पाव॑र्तन्त॒ यो य उ॒पाव॑र्तन्त प॒शवः॑ प॒शव॑ उ॒पाव॑र्तन्त॒ यः । \newline
18. उ॒पाव॑र्तन्त॒ यो य उ॒पाव॑र्तन्तो॒ पाव॑र्तन्त॒ यः प॒शुका॑मः प॒शुका॑मो॒ य उ॒पाव॑र्तन्तो॒ पाव॑र्तन्त॒ यः प॒शुका॑मः । \newline
19. उ॒पाव॑र्त॒न्तेत्यु॑प - आव॑र्तन्त । \newline
20. यः प॒शुका॑मः प॒शुका॑मो॒ यो यः प॒शुका॑मः॒ स्याथ् स्यात् प॒शुका॑मो॒ यो यः प॒शुका॑मः॒ स्यात् । \newline
21. प॒शुका॑मः॒ स्याथ् स्यात् प॒शुका॑मः प॒शुका॑मः॒ स्यात् तस्मै॒ तस्मै॒ स्यात् प॒शुका॑मः प॒शुका॑मः॒ स्यात् तस्मै᳚ । \newline
22. प॒शुका॑म॒ इति॑ प॒शु - का॒मः॒ । \newline
23. स्यात् तस्मै॒ तस्मै॒ स्याथ् स्यात् तस्मा॑ ए॒त मे॒तम् तस्मै॒ स्याथ् स्यात् तस्मा॑ ए॒तम् । \newline
24. तस्मा॑ ए॒त मे॒तम् तस्मै॒ तस्मा॑ ए॒तꣳ सो॑मापौ॒ष्णꣳ सो॑मापौ॒ष्ण मे॒तम् तस्मै॒ तस्मा॑ ए॒तꣳ सो॑मापौ॒ष्णम् । \newline
25. ए॒तꣳ सो॑मापौ॒ष्णꣳ सो॑मापौ॒ष्ण मे॒त मे॒तꣳ सो॑मापौ॒ष्णम् गा᳚र्मु॒तम् गा᳚र्मु॒तꣳ सो॑मापौ॒ष्ण मे॒त मे॒तꣳ सो॑मापौ॒ष्णम् गा᳚र्मु॒तम् । \newline
26. सो॒मा॒पौ॒ष्णम् गा᳚र्मु॒तम् गा᳚र्मु॒तꣳ सो॑मापौ॒ष्णꣳ सो॑मापौ॒ष्णम् गा᳚र्मु॒तम् च॒रुम् च॒रुम् गा᳚र्मु॒तꣳ सो॑मापौ॒ष्णꣳ सो॑मापौ॒ष्णम् गा᳚र्मु॒तम् च॒रुम् । \newline
27. सो॒मा॒पौ॒ष्णमिति॑ सोमा - पौ॒ष्णम् । \newline
28. गा॒र्मु॒तम् च॒रुम् च॒रुम् गा᳚र्मु॒तम् गा᳚र्मु॒तम् च॒रुम् निर् णि श्च॒रुम् गा᳚र्मु॒तम् गा᳚र्मु॒तम् च॒रुम् निः । \newline
29. च॒रुम् निर् णि श्च॒रुम् च॒रुम् निर् व॑पेद् वपे॒न् नि श्च॒रुम् च॒रुम् निर् व॑पेत् । \newline
30. निर् व॑पेद् वपे॒न् निर् णिर् व॑पेथ् सोमापू॒षणौ॑ सोमापू॒षणौ॑ वपे॒न् निर् णिर् व॑पेथ् सोमापू॒षणौ᳚ । \newline
31. व॒पे॒थ् सो॒मा॒पू॒षणौ॑ सोमापू॒षणौ॑ वपेद् वपेथ् सोमापू॒षणा॑ वे॒वैव सो॑मापू॒षणौ॑ वपेद् वपेथ् सोमापू॒षणा॑ वे॒व । \newline
32. सो॒मा॒पू॒षणा॑ वे॒वैव सो॑मापू॒षणौ॑ सोमापू॒षणा॑ वे॒व स्वेन॒ स्वेनै॒व सो॑मापू॒षणौ॑ सोमापू॒षणा॑ वे॒व स्वेन॑ । \newline
33. सो॒मा॒पू॒षणा॒विति॑ सोमा - पू॒षणौ᳚ । \newline
34. ए॒व स्वेन॒ स्वेनै॒वैव स्वेन॑ भाग॒धेये॑न भाग॒धेये॑न॒ स्वेनै॒वैव स्वेन॑ भाग॒धेये॑न । \newline
35. स्वेन॑ भाग॒धेये॑न भाग॒धेये॑न॒ स्वेन॒ स्वेन॑ भाग॒धेये॒नोपोप॑ भाग॒धेये॑न॒ स्वेन॒ स्वेन॑ भाग॒धेये॒नोप॑ । \newline
36. भा॒ग॒धेये॒नोपोप॑ भाग॒धेये॑न भाग॒धेये॒नोप॑ धावति धाव॒त्युप॑ भाग॒धेये॑न भाग॒धेये॒नोप॑ धावति । \newline
37. भा॒ग॒धेये॒नेति॑ भाग - धेये॑न । \newline
38. उप॑ धावति धाव॒ त्युपोप॑ धावति॒ तौ तौ धा॑व॒ त्युपोप॑ धावति॒ तौ । \newline
39. धा॒व॒ति॒ तौ तौ धा॑वति धावति॒ ता वे॒वैव तौ धा॑वति धावति॒ ता वे॒व । \newline
40. ता वे॒वैव तौ ता वे॒वास्मा॑ अस्मा ए॒व तौ ता वे॒वास्मै᳚ । \newline
41. ए॒वास्मा॑ अस्मा ए॒वैवास्मै॑ प॒शून् प॒शू न॑स्मा ए॒वैवास्मै॑ प॒शून् । \newline
42. अ॒स्मै॒ प॒शून् प॒शू न॑स्मा अस्मै प॒शून् प्र प्र प॒शू न॑स्मा अस्मै प॒शून् प्र । \newline
43. प॒शून् प्र प्र प॒शून् प॒शून् प्र ज॑नयतो जनयतः॒ प्र प॒शून् प॒शून् प्र ज॑नयतः । \newline
44. प्र ज॑नयतो जनयतः॒ प्र प्र ज॑नयतः॒ सोमः॒ सोमो॑ जनयतः॒ प्र प्र ज॑नयतः॒ सोमः॑ । \newline
45. ज॒न॒य॒तः॒ सोमः॒ सोमो॑ जनयतो जनयतः॒ सोमो॒ वै वै सोमो॑ जनयतो जनयतः॒ सोमो॒ वै । \newline
46. सोमो॒ वै वै सोमः॒ सोमो॒ वै रे॑तो॒धा रे॑तो॒धा वै सोमः॒ सोमो॒ वै रे॑तो॒धाः । \newline
47. वै रे॑तो॒धा रे॑तो॒धा वै वै रे॑तो॒धाः पू॒षा पू॒षा रे॑तो॒धा वै वै रे॑तो॒धाः पू॒षा । \newline
48. रे॒तो॒धाः पू॒षा पू॒षा रे॑तो॒धा रे॑तो॒धाः पू॒षा प॑शू॒नाम् प॑शू॒नाम् पू॒षा रे॑तो॒धा रे॑तो॒धाः पू॒षा प॑शू॒नाम् । \newline
49. रे॒तो॒धा इति॑ रेतः - धाः । \newline
50. पू॒षा प॑शू॒नाम् प॑शू॒नाम् पू॒षा पू॒षा प॑शू॒नाम् प्र॑जनयि॒ता प्र॑जनयि॒ता प॑शू॒नाम् पू॒षा पू॒षा प॑शू॒नाम् प्र॑जनयि॒ता । \newline
51. प॒शू॒नाम् प्र॑जनयि॒ता प्र॑जनयि॒ता प॑शू॒नाम् प॑शू॒नाम् प्र॑जनयि॒ता सोमः॒ सोमः॑ प्रजनयि॒ता प॑शू॒नाम् प॑शू॒नाम् प्र॑जनयि॒ता सोमः॑ । \newline
52. प्र॒ज॒न॒यि॒ता सोमः॒ सोमः॑ प्रजनयि॒ता प्र॑जनयि॒ता सोम॑ ए॒वैव सोमः॑ प्रजनयि॒ता प्र॑जनयि॒ता सोम॑ ए॒व । \newline
53. प्र॒ज॒न॒यि॒तेति॑ प्र - ज॒न॒यि॒ता । \newline
54. सोम॑ ए॒वैव सोमः॒ सोम॑ ए॒वास्मा॑ अस्मा ए॒व सोमः॒ सोम॑ ए॒वास्मै᳚ । \newline
55. ए॒वास्मा॑ अस्मा ए॒वैवास्मै॒ रेतो॒ रेतो᳚ ऽस्मा ए॒वैवास्मै॒ रेतः॑ । \newline
56. अ॒स्मै॒ रेतो॒ रेतो᳚ ऽस्मा अस्मै॒ रेतो॒ दधा॑ति॒ दधा॑ति॒ रेतो᳚ ऽस्मा अस्मै॒ रेतो॒ दधा॑ति । \newline
57. रेतो॒ दधा॑ति॒ दधा॑ति॒ रेतो॒ रेतो॒ दधा॑ति पू॒षा पू॒षा दधा॑ति॒ रेतो॒ रेतो॒ दधा॑ति पू॒षा । \newline
58. दधा॑ति पू॒षा पू॒षा दधा॑ति॒ दधा॑ति पू॒षा प॒शून् प॒शून् पू॒षा दधा॑ति॒ दधा॑ति पू॒षा प॒शून् । \newline
59. पू॒षा प॒शून् प॒शून् पू॒षा पू॒षा प॒शून् प्र प्र प॒शून् पू॒षा पू॒षा प॒शून् प्र । \newline
60. प॒शून् प्र प्र प॒शून् प॒शून् प्र ज॑नयति जनयति॒ प्र प॒शून् प॒शून् प्र ज॑नयति । \newline
61. प्र ज॑नयति जनयति॒ प्र प्र ज॑नयति । \newline
62. ज॒न॒य॒तीति॑ जनयति । \newline
\pagebreak
\markright{ TS 2.4.5.1  \hfill https://www.vedavms.in \hfill}

\section{ TS 2.4.5.1 }

\textbf{TS 2.4.5.1 } \newline
\textbf{Samhita Paata} \newline

अग्ने॒ गोभि॑र्न॒ आ ग॒हीन्दो॑ पु॒ष्‌ट्या जु॑षस्व नः । इन्द्रो॑ ध॒र्ता गृ॒हेषु॑ नः ॥ स॒वि॒ता यः स॑ह॒स्रियः॒ स नो॑ गृ॒हेषु॑ रारणत् । आ पू॒षा ए॒त्वा वसु॑ ॥ धा॒ता द॑दातु नो र॒यिमीशा॑नो॒ जग॑त॒स्पतिः॑ । स नः॑ पू॒र्णेन॑ वावनत् ॥ त्वष्टा॒ यो वृ॑ष॒भो वृषा॒ स नो॑ गृ॒हेषु॑ रारणत् । स॒हस्रे॑णा॒युते॑न च ॥ येन॑ दे॒वा अ॒मृतं॑ - [  ] \newline

\textbf{Pada Paata} \newline

अग्ने᳚ । गोभिः॑ । नः॒ । एति॑ । ग॒हि॒ । इन्दो॒ इति॑ । पु॒ष्ट्या । जु॒ष॒स्व॒ । नः॒ ॥ इन्द्रः॑ । ध॒र्ता । गृ॒हेषु॑ । नः॒ ॥ स॒वि॒ता । यः । स॒ह॒स्रियः॑ । सः । नः॒ । गृ॒हेषु॑ । रा॒र॒ण॒त् ॥ एति॑ । पू॒षा । ए॒तु॒ । एति॑ । वसु॑ ॥ धा॒ता । द॒दा॒तु॒ । नः॒ । र॒यिम् । ईशा॑नः । जग॑तः । पतिः॑ ॥ सः । नः॒ । पू॒र्णेन॑ । वा॒व॒न॒त् ॥ त्वष्टा᳚ । यः । वृ॒ष॒भः । वृषा᳚ । सः । नः॒ । गृ॒हेषु॑ । रा॒र॒ण॒त् ॥ स॒हस्रे॑ण । अ॒युते॑न । च॒ । येन॑ । दे॒वाः । अ॒मृत᳚म् ।  \newline


\textbf{Krama Paata} \newline

अग्ने॒ गोभिः॑ । गोभि॑र् नः । न॒ आ । आ ग॑हि । ग॒हीन्दो᳚ । इन्दो॑ पु॒ष्ट्या । इन्दो॒ इतीन्दो᳚ । पु॒ष्ट्या जु॑षस्व । जु॒ष॒स्व॒ नः॒ । न॒ इति॑ नः ॥ इन्द्रो॑ ध॒र्ता । ध॒र्ता गृ॒हेषु॑ । गृ॒हेषु॑ नः । न॒ इति॑ नः ॥ स॒वि॒ता यः । यः स॑ह॒स्रियः॑ । स॒ह॒स्रियः॒ सः । स नः॑ । नो॒ गृ॒हेषु॑ । गृ॒हेषु॑ रारणत् । रा॒र॒ण॒दिति॑ रारणत् ॥ आ पू॒षा । पू॒षा ए॑तु । ए॒त्वा । आ वसु॑ । वस्विति॒ वसु॑ ॥ धा॒ता द॑दातु । द॒दा॒तु॒ नः॒ । नो॒ र॒यिम् । र॒यिमीशा॑नः । ईशा॑नो॒ जग॑तः । जग॑त॒स्पतिः॑ । पति॒रिति॒ पतिः॑ ॥ स नः॑ । नः॒ पू॒र्णेन॑ । पू॒र्णेन॑ वावनत् । वा॒व॒न॒दिति॑ वावनत् ॥ त्वष्टा॒ यः । यो वृ॑ष॒भः । वृ॒ष॒भो वृषा᳚ । वृषा॒ सः । स नः॑ । नो॒ गृ॒हेषु॑ । गृ॒हेषु॑ रारणत् । रा॒र॒ण॒दिति॑ रारणत् ॥ स॒हस्रे॑णा॒युते॑न । अ॒युते॑न च । चेति॑ च ॥ येन॑ दे॒वाः । दे॒वा अ॒मृत᳚म् ( ) । अ॒मृतं॑ दी॒र्घम् \newline

\textbf{Jatai Paata} \newline

1. अग्ने॒ गोभि॒र् गोभि॒ रग्ने ऽग्ने॒ गोभिः॑ । \newline
2. गोभि॑र् नो नो॒ गोभि॒र् गोभि॑र् नः । \newline
3. न॒ आ नो॑ न॒ आ । \newline
4. आ ग॑हि ग॒ह्या ग॑हि । \newline
5. ग॒हीन्दो॒ इन्दो॑ गहि ग॒हीन्दो᳚ । \newline
6. इन्दो॑ पु॒ष्ट्या पु॒ष्ट्येन्दो॒ इन्दो॑ पु॒ष्ट्या । \newline
7. इन्दो॒ इतीन्दो᳚ । \newline
8. पु॒ष्ट्या जु॑षस्व जुषस्व पु॒ष्ट्या पु॒ष्ट्या जु॑षस्व । \newline
9. जु॒ष॒स्व॒ नो॒ नो॒ जु॒ष॒स्व॒ जु॒ष॒स्व॒ नः॒ । \newline
10. न॒ इति॑ नः । \newline
11. इन्द्रो॑ ध॒र्ता ध॒र्तेन्द्र॒ इन्द्रो॑ ध॒र्ता । \newline
12. ध॒र्ता गृ॒हेषु॑ गृ॒हेषु॑ ध॒र्ता ध॒र्ता गृ॒हेषु॑ । \newline
13. गृ॒हेषु॑ नो नो गृ॒हेषु॑ गृ॒हेषु॑ नः । \newline
14. न॒ इति॑ नः । \newline
15. स॒वि॒ता यो यः स॑वि॒ता स॑वि॒ता यः । \newline
16. यः स॑ह॒स्रियः॑ सह॒स्रियो॒ यो यः स॑ह॒स्रियः॑ । \newline
17. स॒ह॒स्रियः॒ स स स॑ह॒स्रियः॑ सह॒स्रियः॒ सः । \newline
18. स नो॑ नः॒ स स नः॑ । \newline
19. नो॒ गृ॒हेषु॑ गृ॒हेषु॑ नो नो गृ॒हेषु॑ । \newline
20. गृ॒हेषु॑ रारणद् रारणद् गृ॒हेषु॑ गृ॒हेषु॑ रारणत् । \newline
21. रा॒र॒ण॒दिति॑ रारणत् । \newline
22. आ पू॒षा पू॒षा ऽऽपू॒षा । \newline
23. पू॒षा ए᳚त्वेतु पू॒षा पू॒षा ए॑तु । \newline
24. ए॒त्वैत्वे॒त्वा । \newline
25. आ वसु॒ वस्वा वसु॑ । \newline
26. वस्विति॒ वसु॑ । \newline
27. धा॒ता द॑दातु ददातु धा॒ता धा॒ता द॑दातु । \newline
28. द॒दा॒तु॒ नो॒ नो॒ द॒दा॒तु॒ द॒दा॒तु॒ नः॒ । \newline
29. नो॒ र॒यिꣳ र॒यिम् नो॑ नो र॒यिम् । \newline
30. र॒यि मीशा॑न॒ ईशा॑नो र॒यिꣳ र॒यि मीशा॑नः । \newline
31. ईशा॑नो॒ जग॑तो॒ जग॑त॒ ईशा॑न॒ ईशा॑नो॒ जग॑तः । \newline
32. जग॑त॒ स्पति॒ष् पति॒र् जग॑तो॒ जग॑त॒ स्पतिः॑ । \newline
33. पति॒रिति॒ पतिः॑ । \newline
34. स नो॑ नः॒ स स नः॑ । \newline
35. नः॒ पू॒र्णेन॑ पू॒र्णेन॑ नो नः पू॒र्णेन॑ । \newline
36. पू॒र्णेन॑ वावनद् वावनत् पू॒र्णेन॑ पू॒र्णेन॑ वावनत् । \newline
37. वा॒व॒न॒दिति॑ वावनत् । \newline
38. त्वष्टा॒ यो यस्त्वष्टा॒ त्वष्टा॒ यः । \newline
39. यो वृ॑ष॒भो वृ॑ष॒भो यो यो वृ॑ष॒भः । \newline
40. वृ॒ष॒भो वृषा॒ वृषा॑ वृष॒भो वृ॑ष॒भो वृषा᳚ । \newline
41. वृषा॒ स स वृषा॒ वृषा॒ सः । \newline
42. स नो॑ नः॒ स स नः॑ । \newline
43. नो॒ गृ॒हेषु॑ गृ॒हेषु॑ नो नो गृ॒हेषु॑ । \newline
44. गृ॒हेषु॑ रारणद् रारणद् गृ॒हेषु॑ गृ॒हेषु॑ रारणत् । \newline
45. रा॒र॒ण॒दिति॑ रारणत् । \newline
46. स॒हस्रे॑णा॒ युते॑ना॒ युते॑न स॒हस्रे॑ण स॒हस्रे॑णा॒ युते॑न । \newline
47. अ॒युते॑न च चा॒युते॑ना॒ युते॑न च । \newline
48. चेति॑ च । \newline
49. येन॑ दे॒वा दे॒वा येन॒ येन॑ दे॒वाः । \newline
50. दे॒वा अ॒मृत॑ म॒मृत॑म् दे॒वा दे॒वा अ॒मृत᳚म् । \newline
51. अ॒मृत॑म् दी॒र्घम् दी॒र्घ म॒मृत॑ म॒मृत॑म् दी॒र्घम् । \newline

\textbf{Ghana Paata } \newline

1. अग्ने॒ गोभि॒र् गोभि॒ रग्ने ऽग्ने॒ गोभि॑र् नो नो॒ गोभि॒ रग्ने ऽग्ने॒ गोभि॑र् नः । \newline
2. गोभि॑र् नो नो॒ गोभि॒र् गोभि॑र् न॒ आ नो॒ गोभि॒र् गोभि॑र् न॒ आ । \newline
3. न॒ आ नो॑ न॒ आ ग॑हि ग॒ह्या नो॑ न॒ आ ग॑हि । \newline
4. आ ग॑हि ग॒ह्या ग॒हीन्दो॒ इन्दो॑ ग॒ह्या ग॒हीन्दो᳚ । \newline
5. ग॒हीन्दो॒ इन्दो॑ गहि ग॒हीन्दो॑ पु॒ष्ट्या पु॒ष्ट्येन्दो॑ गहि ग॒हीन्दो॑ पु॒ष्ट्या । \newline
6. इन्दो॑ पु॒ष्ट्या पु॒ष्ट्येन्दो॒ इन्दो॑ पु॒ष्ट्या जु॑षस्व जुषस्व पु॒ष्ट्येन्दो॒ इन्दो॑ पु॒ष्ट्या जु॑षस्व । \newline
7. इन्दो॒ इतीन्दो᳚ । \newline
8. पु॒ष्ट्या जु॑षस्व जुषस्व पु॒ष्ट्या पु॒ष्ट्या जु॑षस्व नो नो जुषस्व पु॒ष्ट्या पु॒ष्ट्या जु॑षस्व नः । \newline
9. जु॒ष॒स्व॒ नो॒ नो॒ जु॒ष॒स्व॒ जु॒ष॒स्व॒ नः॒ । \newline
10. न॒ इति॑ नः । \newline
11. इन्द्रो॑ ध॒र्ता ध॒र्तेन्द्र॒ इन्द्रो॑ ध॒र्ता गृ॒हेषु॑ गृ॒हेषु॑ ध॒र्तेन्द्र॒ इन्द्रो॑ ध॒र्ता गृ॒हेषु॑ । \newline
12. ध॒र्ता गृ॒हेषु॑ गृ॒हेषु॑ ध॒र्ता ध॒र्ता गृ॒हेषु॑ नो नो गृ॒हेषु॑ ध॒र्ता ध॒र्ता गृ॒हेषु॑ नः । \newline
13. गृ॒हेषु॑ नो नो गृ॒हेषु॑ गृ॒हेषु॑ नः । \newline
14. न॒ इति॑ नः । \newline
15. स॒वि॒ता यो यः स॑वि॒ता स॑वि॒ता यः स॑ह॒स्रियः॑ सह॒स्रियो॒ यः स॑वि॒ता स॑वि॒ता यः स॑ह॒स्रियः॑ । \newline
16. यः स॑ह॒स्रियः॑ सह॒स्रियो॒ यो यः स॑ह॒स्रियः॒ स स स॑ह॒स्रियो॒ यो यः स॑ह॒स्रियः॒ सः । \newline
17. स॒ह॒स्रियः॒ स स स॑ह॒स्रियः॑ सह॒स्रियः॒ स नो॑ नः॒ स स॑ह॒स्रियः॑ सह॒स्रियः॒ स नः॑ । \newline
18. स नो॑ नः॒ स स नो॑ गृ॒हेषु॑ गृ॒हेषु॑ नः॒ स स नो॑ गृ॒हेषु॑ । \newline
19. नो॒ गृ॒हेषु॑ गृ॒हेषु॑ नो नो गृ॒हेषु॑ रारणद् रारणद् गृ॒हेषु॑ नो नो गृ॒हेषु॑ रारणत् । \newline
20. गृ॒हेषु॑ रारणद् रारणद् गृ॒हेषु॑ गृ॒हेषु॑ रारणत् । \newline
21. रा॒र॒ण॒दिति॑ रारणत् । \newline
22. आ पू॒षा पू॒षा ऽऽपू॒षा ए᳚त्वेतु पू॒षा ऽऽपू॒षा ए॑तु । \newline
23. पू॒षा ए᳚त्वेतु पू॒षा पू॒षा ए॒त्वैतु॑ पू॒षा पू॒षा ए॒त्वा । \newline
24. ए॒त्वै त्वे॒त्वा वसु॒ वस्वै त्वे॒त्वा वसु॑ । \newline
25. आ वसु॒ वस्वा वसु॑ । \newline
26. वस्विति॒ वसु॑ । \newline
27. धा॒ता द॑दातु ददातु धा॒ता धा॒ता द॑दातु नो नो ददातु धा॒ता धा॒ता द॑दातु नः । \newline
28. द॒दा॒तु॒ नो॒ नो॒ द॒दा॒तु॒ द॒दा॒तु॒ नो॒ र॒यिꣳ र॒यिम् नो॑ ददातु ददातु नो र॒यिम् । \newline
29. नो॒ र॒यिꣳ र॒यिम् नो॑ नो र॒यि मीशा॑न॒ ईशा॑नो र॒यिम् नो॑ नो र॒यि मीशा॑नः । \newline
30. र॒यि मीशा॑न॒ ईशा॑नो र॒यिꣳ र॒यि मीशा॑नो॒ जग॑तो॒ जग॑त॒ ईशा॑नो र॒यिꣳ र॒यि मीशा॑नो॒ जग॑तः । \newline
31. ईशा॑नो॒ जग॑तो॒ जग॑त॒ ईशा॑न॒ ईशा॑नो॒ जग॑त॒ स्पति॒ष् पति॒र् जग॑त॒ ईशा॑न॒ ईशा॑नो॒ जग॑त॒ स्पतिः॑ । \newline
32. जग॑त॒ स्पति॒ष् पति॒र् जग॑तो॒ जग॑त॒ स्पतिः॑ । \newline
33. पति॒रिति॒ पतिः॑ । \newline
34. स नो॑ नः॒ स स नः॑ पू॒र्णेन॑ पू॒र्णेन॑ नः॒ स स नः॑ पू॒र्णेन॑ । \newline
35. नः॒ पू॒र्णेन॑ पू॒र्णेन॑ नो नः पू॒र्णेन॑ वावनद् वावनत् पू॒र्णेन॑ नो नः पू॒र्णेन॑ वावनत् । \newline
36. पू॒र्णेन॑ वावनद् वावनत् पू॒र्णेन॑ पू॒र्णेन॑ वावनत् । \newline
37. वा॒व॒न॒दिति॑ वावनत् । \newline
38. त्वष्टा॒ यो य स्त्वष्टा॒ त्वष्टा॒ यो वृ॑ष॒भो वृ॑ष॒भो य स्त्वष्टा॒ त्वष्टा॒ यो वृ॑ष॒भः । \newline
39. यो वृ॑ष॒भो वृ॑ष॒भो यो यो वृ॑ष॒भो वृषा॒ वृषा॑ वृष॒भो यो यो वृ॑ष॒भो वृषा᳚ । \newline
40. वृ॒ष॒भो वृषा॒ वृषा॑ वृष॒भो वृ॑ष॒भो वृषा॒ स स वृषा॑ वृष॒भो वृ॑ष॒भो वृषा॒ सः । \newline
41. वृषा॒ स स वृषा॒ वृषा॒ स नो॑ नः॒ स वृषा॒ वृषा॒ स नः॑ । \newline
42. स नो॑ नः॒ स स नो॑ गृ॒हेषु॑ गृ॒हेषु॑ नः॒ स स नो॑ गृ॒हेषु॑ । \newline
43. नो॒ गृ॒हेषु॑ गृ॒हेषु॑ नो नो गृ॒हेषु॑ रारणद् रारणद् गृ॒हेषु॑ नो नो गृ॒हेषु॑ रारणत् । \newline
44. गृ॒हेषु॑ रारणद् रारणद् गृ॒हेषु॑ गृ॒हेषु॑ रारणत् । \newline
45. रा॒र॒ण॒दिति॑ रारणत् । \newline
46. स॒हस्रे॑णा॒ युते॑ना॒ युते॑न स॒हस्रे॑ण स॒हस्रे॑णा॒ युते॑न च चा॒युते॑न स॒हस्रे॑ण स॒हस्रे॑णा॒ युते॑न च । \newline
47. अ॒युते॑न च चा॒युते॑ना॒ युते॑न च । \newline
48. चेति॑ च । \newline
49. येन॑ दे॒वा दे॒वा येन॒ येन॑ दे॒वा अ॒मृत॑ म॒मृत॑म् दे॒वा येन॒ येन॑ दे॒वा अ॒मृत᳚म् । \newline
50. दे॒वा अ॒मृत॑ म॒मृत॑म् दे॒वा दे॒वा अ॒मृत॑म् दी॒र्घम् दी॒र्घ म॒मृत॑म् दे॒वा दे॒वा अ॒मृत॑म् दी॒र्घम् । \newline
51. अ॒मृत॑म् दी॒र्घम् दी॒र्घ म॒मृत॑ म॒मृत॑म् दी॒र्घꣳ श्रवः॒ श्रवो॑ दी॒र्घ म॒मृत॑ म॒मृत॑म् दी॒र्घꣳ श्रवः॑ । \newline
\pagebreak
\markright{ TS 2.4.5.2  \hfill https://www.vedavms.in \hfill}

\section{ TS 2.4.5.2 }

\textbf{TS 2.4.5.2 } \newline
\textbf{Samhita Paata} \newline

दी॒र्घꣳ श्रवो॑ दि॒व्यैर॑यन्त । राय॑स्पोष॒ त्वम॒स्मभ्यं॒ गवां᳚ कु॒ल्मिं जी॒वस॒ आ यु॑वस्व ॥ अ॒ग्नि र्गृ॒हप॑तिः॒ सोमो॑ विश्व॒वनिः॑ सवि॒ता सु॑मे॒धाः स्वाहा᳚ ॥ अग्ने॑ गृहपते॒ यस्ते॒ घृत्यो॑ भा॒गस्तेन॒ सह॒ ओज॑ आ॒क्रम॑माणाय धेहि॒ श्रैष्‌ठ्या᳚त्प॒थो मा यो॑षं मू॒र्द्धा भू॑यासꣳ॒॒ स्वाहा᳚ ॥ \newline

\textbf{Pada Paata} \newline

दी॒र्घम् । श्रवः॑ । दि॒वि । ऐर॑यन्त ॥ रायः॑ । पो॒ष॒ । त्वम् । अ॒स्मभ्य॒मित्य॒स्म - भ्य॒म् । गवा᳚म् । कु॒ल्मिम् । जी॒वसे᳚ । एति॑ । यु॒व॒स्व॒ ॥ अ॒ग्निः । गृ॒हप॑ति॒रिति॑ गृ॒ह - प॒तिः॒ । सोमः॑ । वि॒श्व॒वनि॒रिति॑ विश्व - वनिः॑ । स॒वि॒ता । सु॒मे॒धा इति॑ सु-मे॒धाः । स्वाहा᳚ ॥ अग्ने᳚ । गृ॒ह॒प॒त॒ इति॑ गृह - प॒ते॒ । यः । ते॒ । घृत्यः॑ । भा॒गः । तेन॑ । सहः॑ । ओजः॑ । आ॒क्रम॑माणा॒येत्या᳚-क्रम॑माणाय । धे॒हि॒ । श्रष्ठ्या᳚त् । प॒थः । मा । यो॒ष॒म् । मू॒र्द्धा । भू॒या॒स॒म् । स्वाहा᳚ ॥  \newline


\textbf{Krama Paata} \newline

दी॒र्घꣳ श्रवः॑ । श्रवो॑ दि॒वि । दि॒व्यैर॑यन्त । ऐर॑य॒न्तेत्यैर॑यन्त ॥ राय॑स्पोष । पो॒ष॒ त्वम् । त्वम॒स्मभ्य᳚म् । अ॒स्मभ्य॒म् गवा᳚म् । अ॒स्मभ्य॒मित्य॒स्म - भ्य॒म् । गवा᳚म् कु॒ल्मिम् । कु॒ल्मिम् जी॒वसे᳚ । जी॒वस॒ आ । आ यु॑वस्व । यु॒व॒स्वेति॑ युवस्व ॥ अ॒ग्निर् गृ॒हप॑तिः । गृ॒हप॑तिः॒ सोमः॑ । गृ॒हप॑ति॒रिति॑ गृ॒ह - प॒तिः॒ । सोमो॑ विश्व॒वनिः॑ । वि॒श्व॒वनिः॑ सवि॒ता । वि॒श्व॒वनि॒रिति॑ विश्व - वनिः॑ । स॒वि॒ता सु॑मे॒धाः । सु॒मे॒धाः स्वाहा᳚ । सु॒मे॒धा इति॑ सु - मे॒धाः । स्वाहेति॒ स्वाहा᳚ ॥ अग्ने॑ गृहपते । गृ॒ह॒प॒ते॒ यः । गृ॒ह॒प॒त॒ इति॑ गृह - प॒ते॒ । यस्ते᳚ । ते॒ घृत्यः॑ । घृत्यो॑ भा॒गः । 
भा॒गस्तेन॑ । तेन॒ सहः॑ । सह॒ ओजः॑ । ओज॑ आ॒क्रम॑माणाय । आ॒क्रम॑माणाय धेहि । आ॒क्रम॑माणा॒येत्या᳚ - क्रम॑माणाय । धे॒हि॒ श्रैष्ठ्या᳚त् । श्रैष्ठ्या᳚त् प॒थः । प॒थो मा । मा यो॑षम् । यो॒ष॒म् मू॒र्द्धा । मू॒र्द्धा भू॑यासम् । भू॒या॒सꣳ॒॒ स्वाहा᳚ । 
स्वाहेति॒ स्वाहा᳚ । \newline

\textbf{Jatai Paata} \newline

1. दी॒र्घꣳ श्रवः॒ श्रवो॑ दी॒र्घम् दी॒र्घꣳ श्रवः॑ । \newline
2. श्रवो॑ दि॒वि दि॒वि श्रवः॒ श्रवो॑ दि॒वि । \newline
3. दि॒व्यैर॑य॒ न्तैर॑यन्त दि॒वि दि॒व्यैर॑यन्त । \newline
4. ऐर॑य॒न्तेत्यैर॑यन्त । \newline
5. राय॑ स्पोष पोष॒ रायो॒ राय॑ स्पोष । \newline
6. पो॒ष॒ त्वम् त्वम् पो॑ष पोष॒ त्वम् । \newline
7. त्व म॒स्मभ्य॑ म॒स्मभ्य॒म् त्वम् त्व म॒स्मभ्य᳚म् । \newline
8. अ॒स्मभ्य॒म् गवा॒म् गवा॑ म॒स्मभ्य॑ म॒स्मभ्य॒म् गवा᳚म् । \newline
9. अ॒स्मभ्य॒मित्य॒स्म - भ्य॒म् । \newline
10. गवा᳚म् कु॒ल्मिम् कु॒ल्मिम् गवा॒म् गवा᳚म् कु॒ल्मिम् । \newline
11. कु॒ल्मिम् जी॒वसे॑ जी॒वसे॑ कु॒ल्मिम् कु॒ल्मिम् जी॒वसे᳚ । \newline
12. जी॒वस॒ आ जी॒वसे॑ जी॒वस॒ आ । \newline
13. आ यु॑वस्व युव॒स्वा यु॑वस्व । \newline
14. यु॒व॒स्वेति॑ युवस्व । \newline
15. अ॒ग्निर् गृ॒हप॑तिर् गृ॒हप॑ति र॒ग्नि र॒ग्निर् गृ॒हप॑तिः । \newline
16. गृ॒हप॑तिः॒ सोमः॒ सोमो॑ गृ॒हप॑तिर् गृ॒हप॑तिः॒ सोमः॑ । \newline
17. गृ॒हप॑ति॒रिति॑ गृ॒ह - प॒तिः॒ । \newline
18. सोमो॑ विश्व॒वनि॑र् विश्व॒वनिः॒ सोमः॒ सोमो॑ विश्व॒वनिः॑ । \newline
19. वि॒श्व॒वनिः॑ सवि॒ता स॑वि॒ता वि॑श्व॒वनि॑र् विश्व॒वनिः॑ सवि॒ता । \newline
20. वि॒श्व॒वनि॒रिति॑ विश्व - वनिः॑ । \newline
21. स॒वि॒ता सु॑मे॒धाः सु॑मे॒धाः स॑वि॒ता स॑वि॒ता सु॑मे॒धाः । \newline
22. सु॒मे॒धाः स्वाहा॒ स्वाहा॑ सुमे॒धाः सु॑मे॒धाः स्वाहा᳚ । \newline
23. सु॒मे॒धा इति॑ सु - मे॒धाः । \newline
24. स्वाहेति॒ स्वाहा᳚ । \newline
25. अग्ने॑ गृहपते गृहप॒ते ऽग्ने ऽग्ने॑ गृहपते । \newline
26. गृ॒ह॒प॒ते॒ यो यो गृ॑हपते गृहपते॒ यः । \newline
27. गृ॒ह॒प॒त॒ इति॑ गृह - प॒ते॒ । \newline
28. य स्ते॑ ते॒ यो य स्ते᳚ । \newline
29. ते॒ घृत्यो॒ घृत्य॑ स्ते ते॒ घृत्यः॑ । \newline
30. घृत्यो॑ भा॒गो भा॒गो घृत्यो॒ घृत्यो॑ भा॒गः । \newline
31. भा॒ग स्तेन॒ तेन॑ भा॒गो भा॒ग स्तेन॑ । \newline
32. तेन॒ सहः॒ सह॒ स्तेन॒ तेन॒ सहः॑ । \newline
33. सह॒ ओज॒ ओजः॒ सहः॒ सह॒ ओजः॑ । \newline
34. ओज॑ आ॒क्रम॑माणाया॒ क्रम॑माणा॒यौज॒ ओज॑ आ॒क्रम॑माणाय । \newline
35. आ॒क्रम॑माणाय धेहि धेह्या॒ क्रम॑माणाया॒ क्रम॑माणाय धेहि । \newline
36. आ॒क्रम॑माणा॒येत्या᳚ - क्रम॑माणाय । \newline
37. धे॒हि॒ श्रैष्ठ्‍या॒च् छ्रैष्ठ्‍या᳚द् धेहि धेहि॒ श्रैष्ठ्‍या᳚त् । \newline
38. श्रैष्ठ्‍या᳚त् प॒थः प॒थः श्रैष्ठ्‍या॒च् छ्रैष्ठ्‍या᳚त् प॒थः । \newline
39. प॒थो मा मा प॒थः प॒थो मा । \newline
40. मा यो॑षं ॅयोष॒म् मा मा यो॑षम् । \newline
41. यो॒ष॒म् मू॒र्द्धा मू॒र्द्धा यो॑षं ॅयोषम् मू॒र्द्धा । \newline
42. मू॒र्द्धा भू॑यासम् भूयासम् मू॒र्द्धा मू॒र्द्धा भू॑यासम् । \newline
43. भू॒या॒सꣳ॒॒ स्वाहा॒ स्वाहा॑ भूयासम् भूयासꣳ॒॒ स्वाहा᳚ । \newline
44. स्वाहेति॒ स्वाहा᳚ । \newline

\textbf{Ghana Paata } \newline

1. दी॒र्घꣳ श्रवः॒ श्रवो॑ दी॒र्घम् दी॒र्घꣳ श्रवो॑ दि॒वि दि॒वि श्रवो॑ दी॒र्घम् दी॒र्घꣳ श्रवो॑ दि॒वि । \newline
2. श्रवो॑ दि॒वि दि॒वि श्रवः॒ श्रवो॑ दि॒व्यैर॑य॒ न्तैर॑यन्त दि॒वि श्रवः॒ श्रवो॑ दि॒व्यैर॑यन्त । \newline
3. दि॒व्यैर॑य॒ न्तैर॑यन्त दि॒वि दि॒व्यैर॑यन्त । \newline
4. ऐर॑य॒न्तेत्यैर॑यन्त । \newline
5. राय॑ स्पोष पोष॒ रायो॒ राय॑ स्पोष॒ त्वम् त्वम् पो॑ष॒ रायो॒ राय॑ स्पोष॒ त्वम् । \newline
6. पो॒ष॒ त्वम् त्वम् पो॑ष पोष॒ त्व म॒स्मभ्य॑ म॒स्मभ्य॒म् त्वम् पो॑ष पोष॒ त्व म॒स्मभ्य᳚म् । \newline
7. त्व म॒स्मभ्य॑ म॒स्मभ्य॒म् त्वम् त्व म॒स्मभ्य॒म् गवा॒म् गवा॑ म॒स्मभ्य॒म् त्वम् त्व म॒स्मभ्य॒म् गवा᳚म् । \newline
8. अ॒स्मभ्य॒म् गवा॒म् गवा॑ म॒स्मभ्य॑ म॒स्मभ्य॒म् गवा᳚म् कु॒ल्मिम् कु॒ल्मिम् गवा॑ म॒स्मभ्य॑ म॒स्मभ्य॒म् गवा᳚म् कु॒ल्मिम् । \newline
9. अ॒स्मभ्य॒मित्य॒स्म - भ्य॒म् । \newline
10. गवा᳚म् कु॒ल्मिम् कु॒ल्मिम् गवा॒म् गवा᳚म् कु॒ल्मिम् जी॒वसे॑ जी॒वसे॑ कु॒ल्मिम् गवा॒म् गवा᳚म् कु॒ल्मिम् जी॒वसे᳚ । \newline
11. कु॒ल्मिम् जी॒वसे॑ जी॒वसे॑ कु॒ल्मिम् कु॒ल्मिम् जी॒वस॒ आ जी॒वसे॑ कु॒ल्मिम् कु॒ल्मिम् जी॒वस॒ आ । \newline
12. जी॒वस॒ आ जी॒वसे॑ जी॒वस॒ आ यु॑वस्व युव॒स्वा जी॒वसे॑ जी॒वस॒ आ यु॑वस्व । \newline
13. आ यु॑वस्व युव॒स्वा यु॑वस्व । \newline
14. यु॒व॒स्वेति॑ युवस्व । \newline
15. अ॒ग्निर् गृ॒हप॑तिर् गृ॒हप॑ति र॒ग्नि र॒ग्निर् गृ॒हप॑तिः॒ सोमः॒ सोमो॑ गृ॒हप॑ति र॒ग्नि र॒ग्निर् गृ॒हप॑तिः॒ सोमः॑ । \newline
16. गृ॒हप॑तिः॒ सोमः॒ सोमो॑ गृ॒हप॑तिर् गृ॒हप॑तिः॒ सोमो॑ विश्व॒वनि॑र् विश्व॒वनिः॒ सोमो॑ गृ॒हप॑तिर् गृ॒हप॑तिः॒ सोमो॑ विश्व॒वनिः॑ । \newline
17. गृ॒हप॑ति॒रिति॑ गृ॒ह - प॒तिः॒ । \newline
18. सोमो॑ विश्व॒वनि॑र् विश्व॒वनिः॒ सोमः॒ सोमो॑ विश्व॒वनिः॑ सवि॒ता स॑वि॒ता वि॑श्व॒वनिः॒ सोमः॒ सोमो॑ विश्व॒वनिः॑ सवि॒ता । \newline
19. वि॒श्व॒वनिः॑ सवि॒ता स॑वि॒ता वि॑श्व॒वनि॑र् विश्व॒वनिः॑ सवि॒ता सु॑मे॒धाः सु॑मे॒धाः स॑वि॒ता वि॑श्व॒वनि॑र् विश्व॒वनिः॑ सवि॒ता सु॑मे॒धाः । \newline
20. वि॒श्व॒वनि॒रिति॑ विश्व - वनिः॑ । \newline
21. स॒वि॒ता सु॑मे॒धाः सु॑मे॒धाः स॑वि॒ता स॑वि॒ता सु॑मे॒धाः स्वाहा॒ स्वाहा॑ सुमे॒धाः स॑वि॒ता स॑वि॒ता सु॑मे॒धाः स्वाहा᳚ । \newline
22. सु॒मे॒धाः स्वाहा॒ स्वाहा॑ सुमे॒धाः सु॑मे॒धाः स्वाहा᳚ । \newline
23. सु॒मे॒धा इति॑ सु - मे॒धाः । \newline
24. स्वाहेति॒ स्वाहा᳚ । \newline
25. अग्ने॑ गृहपते गृहप॒ते ऽग्ने ऽग्ने॑ गृहपते॒ यो यो गृ॑हप॒ते ऽग्ने ऽग्ने॑ गृहपते॒ यः । \newline
26. गृ॒ह॒प॒ते॒ यो यो गृ॑हपते गृहपते॒ यस्ते॑ ते॒ यो गृ॑हपते गृहपते॒ यस्ते᳚ । \newline
27. गृ॒ह॒प॒त॒ इति॑ गृह - प॒ते॒ । \newline
28. यस्ते॑ ते॒ यो यस्ते॒ घृत्यो॒ घृत्य॑ स्ते॒ यो यस्ते॒ घृत्यः॑ । \newline
29. ते॒ घृत्यो॒ घृत्य॑ स्ते ते॒ घृत्यो॑ भा॒गो भा॒गो घृत्य॑ स्ते ते॒ घृत्यो॑ भा॒गः । \newline
30. घृत्यो॑ भा॒गो भा॒गो घृत्यो॒ घृत्यो॑ भा॒ग स्तेन॒ तेन॑ भा॒गो घृत्यो॒ घृत्यो॑ भा॒ग स्तेन॑ । \newline
31. भा॒ग स्तेन॒ तेन॑ भा॒गो भा॒ग स्तेन॒ सहः॒ सह॒स्तेन॑ भा॒गो भा॒ग स्तेन॒ सहः॑ । \newline
32. तेन॒ सहः॒ सह॒ स्तेन॒ तेन॒ सह॒ ओज॒ ओजः॒ सह॒ स्तेन॒ तेन॒ सह॒ ओजः॑ । \newline
33. सह॒ ओज॒ ओजः॒ सहः॒ सह॒ ओज॑ आ॒क्रम॑माणाया॒ क्रम॑माणा॒यौजः॒ सहः॒ सह॒ ओज॑ आ॒क्रम॑माणाय । \newline
34. ओज॑ आ॒क्रम॑माणाया॒ क्रम॑माणा॒यौज॒ ओज॑ आ॒क्रम॑माणाय धेहि धेह्या॒क्रम॑माणा॒यौज॒ ओज॑ आ॒क्रम॑माणाय धेहि । \newline
35. आ॒क्रम॑माणाय धेहि धेह्या॒क्रम॑माणाया॒ क्रम॑माणाय धेहि॒ श्रैष्ठ्‍या॒च् छ्रैष्ठ्‍या᳚द् धेह्या॒क्रम॑माणाया॒ क्रम॑माणाय धेहि॒ श्रैष्ठ्‍या᳚त् । \newline
36. आ॒क्रम॑माणा॒येत्या᳚ - क्रम॑माणाय । \newline
37. धे॒हि॒ श्रैष्ठ्‍या॒च् छ्रैष्ठ्‍या᳚द् धेहि धेहि॒ श्रैष्ठ्‍या᳚त् प॒थः प॒थः श्रैष्ठ्‍या᳚द् धेहि धेहि॒ श्रैष्ठ्‍या᳚त् प॒थः । \newline
38. श्रैष्ठ्‍या᳚त् प॒थः प॒थः श्रैष्ठ्‍या॒च् छ्रैष्ठ्‍या᳚त् प॒थो मा मा प॒थः श्रैष्ठ्‍या॒च् छ्रैष्ठ्‍या᳚त् प॒थो मा । \newline
39. प॒थो मा मा प॒थः प॒थो मा यो॑षं ॅयोष॒म् मा प॒थः प॒थो मा यो॑षम् । \newline
40. मा यो॑षं ॅयोष॒म् मा मा यो॑षम् मू॒र्द्धा मू॒र्द्धा यो॑ष॒म् मा मा यो॑षम् मू॒र्द्धा । \newline
41. यो॒ष॒म् मू॒र्द्धा मू॒र्द्धा यो॑षं ॅयोषम् मू॒र्द्धा भू॑यासम् भूयासम् मू॒र्द्धा यो॑षं ॅयोषम् मू॒र्द्धा भू॑यासम् । \newline
42. मू॒र्द्धा भू॑यासम् भूयासम् मू॒र्द्धा मू॒र्द्धा भू॑यासꣳ॒॒ स्वाहा॒ स्वाहा॑ भूयासम् मू॒र्द्धा मू॒र्द्धा भू॑यासꣳ॒॒ स्वाहा᳚ । \newline
43. भू॒या॒सꣳ॒॒ स्वाहा॒ स्वाहा॑ भूयासम् भूयासꣳ॒॒ स्वाहा᳚ । \newline
44. स्वाहेति॒ स्वाहा᳚ । \newline
\pagebreak
\markright{ TS 2.4.6.1  \hfill https://www.vedavms.in \hfill}

\section{ TS 2.4.6.1 }

\textbf{TS 2.4.6.1 } \newline
\textbf{Samhita Paata} \newline

चि॒त्रया॑ यजेत प॒शुका॑म इ॒यं ॅवै चि॒त्रा यद्वा अ॒स्यां ॅविश्वं॑ भू॒तमधि॑ प्र॒जाय॑ते॒ ते ने॒यञ्चि॒त्रा य ए॒वं ॅवि॒द्वाꣳ श्चि॒त्रया॑ प॒शुका॑मो॒ यज॑ते॒ प्र प्र॒जया॑ प॒शुभि॑ र्मिथु॒नै र्जा॑यते॒ प्रैवाग्ने॒येन॑ वापयति॒ रेतः॑ सौ॒म्येन॑ दधाति॒ रेत॑ ए॒व हि॒तं त्वष्टा॑ रू॒पाणि॒ वि क॑रोतिसारस्व॒तौ भ॑वत ए॒तद्वै दैव्यं॑ मिथु॒नं दैव्य॑मे॒वास्मै॑ - [  ] \newline

\textbf{Pada Paata} \newline

चि॒त्रया᳚ । य॒जे॒त॒ । प॒शुका॑म॒ इति॑ प॒शु - का॒मः॒ । इ॒यम् । वै । चि॒त्रा । यत् । वै । अ॒स्याम् । विश्व᳚म् । भू॒तम् । अधीति॑ । प्र॒जाय॑त॒ इति॑ प्र - जाय॑ते । तेन॑ । इ॒यम् । चि॒त्रा । यः । ए॒वम् । वि॒द्वान् । चि॒त्रया᳚ । प॒शुका॑म॒ इति॑ प॒शु - का॒मः॒ । यज॑ते । प्रेति॑ । प्र॒जयेति॑ प्र - जया᳚ । प॒शुभि॒रिति॑ प॒शु - भिः॒ । मि॒थु॒नैः । जा॒य॒ते॒ । प्रेति॑ । ए॒व । आ॒ग्ने॒येन॑ । वा॒प॒य॒ति॒ । रेतः॑ । सौ॒म्येन॑ । द॒धा॒ति॒ । रेतः॑ । ए॒व । हि॒तम् । त्वष्टा᳚ । रू॒पाणि॑ । वीति॑ । क॒रो॒ति॒ । सा॒र॒स्व॒तौ । भ॒व॒तः॒ । ए॒तत् । वै । दैव्य᳚म् । मि॒थु॒नम् । दैव्य᳚म् । ए॒व । अ॒स्मै॒ ।  \newline


\textbf{Krama Paata} \newline

चि॒त्रया॑ यजेत । य॒जे॒त॒ प॒शुका॑मः । प॒शुका॑म इ॒यम् । प॒शुका॑म॒ इति॑ प॒शु - का॒मः॒ । इ॒यं ॅवै । वै चि॒त्रा । चि॒त्रा यत् । यद् वै । वा अ॒स्याम् । अ॒स्यां ॅविश्व᳚म् । विश्व॑म् भू॒तम् । भू॒तमधि॑ । अधि॑ प्र॒जाय॑ते । प्र॒जाय॑ते॒ तेन॑ । प्र॒जाय॑त॒ इति॑ प्र - जाय॑ते । तेने॒यम् । इ॒यम् चि॒त्रा । चि॒त्रा यः । य ए॒वम् । ए॒वं ॅवि॒द्वान् । वि॒द्वाꣳश्चि॒त्रया᳚ । चि॒त्रया॑ प॒शुका॑मः । प॒शुका॑मो॒ यज॑ते । प॒शुका॑म॒ इति॑ प॒शु - का॒मः॒ । यज॑ते॒ प्र । प्र प्र॒जया᳚ । प्र॒जया॑ प॒शुभिः॑ । 
प्र॒जयेति॑ प्र - जया᳚ । प॒शुभि॑र् मिथु॒नैः । प॒शुभि॒रिति॑ प॒शु - भिः॒ । मि॒थु॒नैर् जा॑यते । जा॒य॒ते॒ प्र । प्रैव । ए॒वाग्ने॒येन॑ । आ॒ग्ने॒येन॑ वापयति । वा॒प॒य॒ति॒ रेतः॑ । रेतः॑ सौ॒म्येन॑ । सौ॒म्येन॑ दधाति । द॒धा॒ति॒ रेतः॑ । रेत॑ ए॒व । ए॒व हि॒तम् । हि॒तम् त्वष्टा᳚ । त्वष्टा॑ रू॒पाणि॑ । रू॒पाणि॒ वि । वि क॑रोति । क॒रो॒ति॒ सा॒र॒स्व॒तौ । सा॒र॒स्व॒तौ भ॑वतः । भ॒व॒त॒ ए॒तत् । ए॒तद् वै । वै दैव्य᳚म् । दैव्य॑म् मिथु॒नम् । मि॒थु॒नम् दैव्य᳚म् । दैव्य॑मे॒व । ए॒वास्मै᳚ । अ॒स्मै॒ मि॒थु॒नम् \newline

\textbf{Jatai Paata} \newline

1. चि॒त्रया॑ यजेत यजेत चि॒त्रया॑ चि॒त्रया॑ यजेत । \newline
2. य॒जे॒त॒ प॒शुका॑मः प॒शुका॑मो यजेत यजेत प॒शुका॑मः । \newline
3. प॒शुका॑म इ॒य मि॒यम् प॒शुका॑मः प॒शुका॑म इ॒यम् । \newline
4. प॒शुका॑म॒ इति॑ प॒शु - का॒मः॒ । \newline
5. इ॒यं ॅवै वा इ॒य मि॒यं ॅवै । \newline
6. वै चि॒त्रा चि॒त्रा वै वै चि॒त्रा । \newline
7. चि॒त्रा यद् यच् चि॒त्रा चि॒त्रा यत् । \newline
8. यद् वै वै यद् यद् वै । \newline
9. वा अ॒स्या म॒स्यां ॅवै वा अ॒स्याम् । \newline
10. अ॒स्यां ॅविश्वं॒ ॅविश्व॑ म॒स्या म॒स्यां ॅविश्व᳚म् । \newline
11. विश्व॑म् भू॒तम् भू॒तं ॅविश्वं॒ ॅविश्व॑म् भू॒तम् । \newline
12. भू॒त मध्यधि॑ भू॒तम् भू॒त मधि॑ । \newline
13. अधि॑ प्र॒जाय॑ते प्र॒जाय॒ते ऽध्यधि॑ प्र॒जाय॑ते । \newline
14. प्र॒जाय॑ते॒ तेन॒ तेन॑ प्र॒जाय॑ते प्र॒जाय॑ते॒ तेन॑ । \newline
15. प्र॒जाय॑त॒ इति॑ प्र - जाय॑ते । \newline
16. तेने॒ य मि॒यम् तेन॒ तेने॒ यम् । \newline
17. इ॒यम् चि॒त्रा चि॒त्रेय मि॒यम् चि॒त्रा । \newline
18. चि॒त्रा यो यश्चि॒त्रा चि॒त्रा यः । \newline
19. य ए॒व मे॒वं ॅयो य ए॒वम् । \newline
20. ए॒वं ॅवि॒द्वान्. वि॒द्वा ने॒व मे॒वं ॅवि॒द्वान् । \newline
21. वि॒द्वाꣳ श्चि॒त्रया॑ चि॒त्रया॑ वि॒द्वान्. वि॒द्वाꣳ श्चि॒त्रया᳚ । \newline
22. चि॒त्रया॑ प॒शुका॑मः प॒शुका॑म श्चि॒त्रया॑ चि॒त्रया॑ प॒शुका॑मः । \newline
23. प॒शुका॑मो॒ यज॑ते॒ यज॑ते प॒शुका॑मः प॒शुका॑मो॒ यज॑ते । \newline
24. प॒शुका॑म॒ इति॑ प॒शु - का॒मः॒ । \newline
25. यज॑ते॒ प्र प्र यज॑ते॒ यज॑ते॒ प्र । \newline
26. प्र प्र॒जया᳚ प्र॒जया॒ प्र प्र प्र॒जया᳚ । \newline
27. प्र॒जया॑ प॒शुभिः॑ प॒शुभिः॑ प्र॒जया᳚ प्र॒जया॑ प॒शुभिः॑ । \newline
28. प्र॒जयेति॑ प्र - जया᳚ । \newline
29. प॒शुभि॑र् मिथु॒नैर् मि॑थु॒नैः प॒शुभिः॑ प॒शुभि॑र् मिथु॒नैः । \newline
30. प॒शुभि॒रिति॑ प॒शु - भिः॒ । \newline
31. मि॒थु॒नैर् जा॑यते जायते मिथु॒नैर् मि॑थु॒नैर् जा॑यते । \newline
32. जा॒य॒ते॒ प्र प्र जा॑यते जायते॒ प्र । \newline
33. प्रैवैव प्र प्रैव । \newline
34. ए॒वाग्ने॒येना᳚ ग्ने॒ये नै॒वैवा ग्ने॒येन॑ । \newline
35. आ॒ग्ने॒येन॑ वापयति वापय त्याग्ने॒येना᳚ ग्ने॒येन॑ वापयति । \newline
36. वा॒प॒य॒ति॒ रेतो॒ रेतो॑ वापयति वापयति॒ रेतः॑ । \newline
37. रेतः॑ सौ॒म्येन॑ सौ॒म्येन॒ रेतो॒ रेतः॑ सौ॒म्येन॑ । \newline
38. सौ॒म्येन॑ दधाति दधाति सौ॒म्येन॑ सौ॒म्येन॑ दधाति । \newline
39. द॒धा॒ति॒ रेतो॒ रेतो॑ दधाति दधाति॒ रेतः॑ । \newline
40. रेत॑ ए॒वैव रेतो॒ रेत॑ ए॒व । \newline
41. ए॒व हि॒तꣳ हि॒त मे॒वैव हि॒तम् । \newline
42. हि॒तम् त्वष्टा॒ त्वष्टा॑ हि॒तꣳ हि॒तम् त्वष्टा᳚ । \newline
43. त्वष्टा॑ रू॒पाणि॑ रू॒पाणि॒ त्वष्टा॒ त्वष्टा॑ रू॒पाणि॑ । \newline
44. रू॒पाणि॒ वि वि रू॒पाणि॑ रू॒पाणि॒ वि । \newline
45. वि क॑रोति करोति॒ वि वि क॑रोति । \newline
46. क॒रो॒ति॒ सा॒र॒स्व॒तौ सा॑रस्व॒तौ क॑रोति करोति सारस्व॒तौ । \newline
47. सा॒र॒स्व॒तौ भ॑वतो भवतः सारस्व॒तौ सा॑रस्व॒तौ भ॑वतः । \newline
48. भ॒व॒त॒ ए॒तदे॒तद् भ॑वतो भवत ए॒तत् । \newline
49. ए॒तद् वै वा ए॒त दे॒तद् वै । \newline
50. वै दैव्य॒म् दैव्यं॒ ॅवै वै दैव्य᳚म् । \newline
51. दैव्य॑म् मिथु॒नम् मि॑थु॒नम् दैव्य॒म् दैव्य॑म् मिथु॒नम् । \newline
52. मि॒थु॒नम् दैव्य॒म् दैव्य॑म् मिथु॒नम् मि॑थु॒नम् दैव्य᳚म् । \newline
53. दैव्य॑ मे॒वैव दैव्य॒म् दैव्य॑ मे॒व । \newline
54. ए॒वास्मा॑ अस्मा ए॒वैवास्मै᳚ । \newline
55. अ॒स्मै॒ मि॒थु॒नम् मि॑थु॒न म॑स्मा अस्मै मिथु॒नम् । \newline

\textbf{Ghana Paata } \newline

1. चि॒त्रया॑ यजेत यजेत चि॒त्रया॑ चि॒त्रया॑ यजेत प॒शुका॑मः प॒शुका॑मो यजेत चि॒त्रया॑ चि॒त्रया॑ यजेत प॒शुका॑मः । \newline
2. य॒जे॒त॒ प॒शुका॑मः प॒शुका॑मो यजेत यजेत प॒शुका॑म इ॒य मि॒यम् प॒शुका॑मो यजेत यजेत प॒शुका॑म इ॒यम् । \newline
3. प॒शुका॑म इ॒य मि॒यम् प॒शुका॑मः प॒शुका॑म इ॒यं ॅवै वा इ॒यम् प॒शुका॑मः प॒शुका॑म इ॒यं ॅवै । \newline
4. प॒शुका॑म॒ इति॑ प॒शु - का॒मः॒ । \newline
5. इ॒यं ॅवै वा इ॒य मि॒यं ॅवै चि॒त्रा चि॒त्रा वा इ॒य मि॒यं ॅवै चि॒त्रा । \newline
6. वै चि॒त्रा चि॒त्रा वै वै चि॒त्रा यद् यच् चि॒त्रा वै वै चि॒त्रा यत् । \newline
7. चि॒त्रा यद् यच् चि॒त्रा चि॒त्रा यद् वै वै यच् चि॒त्रा चि॒त्रा यद् वै । \newline
8. यद् वै वै यद् यद् वा अ॒स्या म॒स्यां ॅवै यद् यद् वा अ॒स्याम् । \newline
9. वा अ॒स्या म॒स्यां ॅवै वा अ॒स्यां ॅविश्वं॒ ॅविश्व॑ म॒स्यां ॅवै वा अ॒स्यां ॅविश्व᳚म् । \newline
10. अ॒स्यां ॅविश्वं॒ ॅविश्व॑ म॒स्या म॒स्यां ॅविश्व॑म् भू॒तम् भू॒तं ॅविश्व॑ म॒स्या म॒स्यां ॅविश्व॑म् भू॒तम् । \newline
11. विश्व॑म् भू॒तम् भू॒तं ॅविश्वं॒ ॅविश्व॑म् भू॒त मध्यधि॑ भू॒तं ॅविश्वं॒ ॅविश्व॑म् भू॒त मधि॑ । \newline
12. भू॒त मध्यधि॑ भू॒तम् भू॒त मधि॑ प्र॒जाय॑ते प्र॒जाय॒ते ऽधि॑ भू॒तम् भू॒त मधि॑ प्र॒जाय॑ते । \newline
13. अधि॑ प्र॒जाय॑ते प्र॒जाय॒ते ऽध्यधि॑ प्र॒जाय॑ते॒ तेन॒ तेन॑ प्र॒जाय॒ते ऽध्यधि॑ प्र॒जाय॑ते॒ तेन॑ । \newline
14. प्र॒जाय॑ते॒ तेन॒ तेन॑ प्र॒जाय॑ते प्र॒जाय॑ते॒ तेने॒ य मि॒यम् तेन॑ प्र॒जाय॑ते प्र॒जाय॑ते॒ तेने॒ यम् । \newline
15. प्र॒जाय॑त॒ इति॑ प्र - जाय॑ते । \newline
16. तेने॒ य मि॒यम् तेन॒ तेने॒ यम् चि॒त्रा चि॒त्रेयम् तेन॒ तेने॒ यम् चि॒त्रा । \newline
17. इ॒यम् चि॒त्रा चि॒त्रेय मि॒यम् चि॒त्रा यो यश्चि॒त्रेय मि॒यम् चि॒त्रा यः । \newline
18. चि॒त्रा यो यश्चि॒त्रा चि॒त्रा य ए॒व मे॒वं ॅयश्चि॒त्रा चि॒त्रा य ए॒वम् । \newline
19. य ए॒व मे॒वं ॅयो य ए॒वं ॅवि॒द्वान्. वि॒द्वा ने॒वं ॅयो य ए॒वं ॅवि॒द्वान् । \newline
20. ए॒वं ॅवि॒द्वान्. वि॒द्वा ने॒व मे॒वं ॅवि॒द्वाꣳ श्चि॒त्रया॑ चि॒त्रया॑ वि॒द्वा ने॒व मे॒वं ॅवि॒द्वाꣳ श्चि॒त्रया᳚ । \newline
21. वि॒द्वाꣳ श्चि॒त्रया॑ चि॒त्रया॑ वि॒द्वान्. वि॒द्वाꣳ श्चि॒त्रया॑ प॒शुका॑मः प॒शुका॑म श्चि॒त्रया॑ वि॒द्वान्. वि॒द्वाꣳ श्चि॒त्रया॑ प॒शुका॑मः । \newline
22. चि॒त्रया॑ प॒शुका॑मः प॒शुका॑मश्चि॒त्रया॑ चि॒त्रया॑ प॒शुका॑मो॒ यज॑ते॒ यज॑ते प॒शुका॑म श्चि॒त्रया॑ चि॒त्रया॑ प॒शुका॑मो॒ यज॑ते । \newline
23. प॒शुका॑मो॒ यज॑ते॒ यज॑ते प॒शुका॑मः प॒शुका॑मो॒ यज॑ते॒ प्र प्र यज॑ते प॒शुका॑मः प॒शुका॑मो॒ यज॑ते॒ प्र । \newline
24. प॒शुका॑म॒ इति॑ प॒शु - का॒मः॒ । \newline
25. यज॑ते॒ प्र प्र यज॑ते॒ यज॑ते॒ प्र प्र॒जया᳚ प्र॒जया॒ प्र यज॑ते॒ यज॑ते॒ प्र प्र॒जया᳚ । \newline
26. प्र प्र॒जया᳚ प्र॒जया॒ प्र प्र प्र॒जया॑ प॒शुभिः॑ प॒शुभिः॑ प्र॒जया॒ प्र प्र प्र॒जया॑ प॒शुभिः॑ । \newline
27. प्र॒जया॑ प॒शुभिः॑ प॒शुभिः॑ प्र॒जया᳚ प्र॒जया॑ प॒शुभि॑र् मिथु॒नैर् मि॑थु॒नैः प॒शुभिः॑ प्र॒जया᳚ प्र॒जया॑ प॒शुभि॑र् मिथु॒नैः । \newline
28. प्र॒जयेति॑ प्र - जया᳚ । \newline
29. प॒शुभि॑र् मिथु॒नैर् मि॑थु॒नैः प॒शुभिः॑ प॒शुभि॑र् मिथु॒नैर् जा॑यते जायते मिथु॒नैः प॒शुभिः॑ प॒शुभि॑र् मिथु॒नैर् जा॑यते । \newline
30. प॒शुभि॒रिति॑ प॒शु - भिः॒ । \newline
31. मि॒थु॒नैर् जा॑यते जायते मिथु॒नैर् मि॑थु॒नैर् जा॑यते॒ प्र प्र जा॑यते मिथु॒नैर् मि॑थु॒नैर् जा॑यते॒ प्र । \newline
32. जा॒य॒ते॒ प्र प्र जा॑यते जायते॒ प्रैवैव प्र जा॑यते जायते॒ प्रैव । \newline
33. प्रैवैव प्र प्रैवाग्ने॒येना᳚ ग्ने॒येनै॒व प्र प्रैवाग्ने॒येन॑ । \newline
34. ए॒वाग्ने॒येना᳚ ग्ने॒येनै॒ वैवाग्ने॒येन॑ वापयति वापय त्याग्ने॒येनै॒ वैवाग्ने॒येन॑ वापयति । \newline
35. आ॒ग्ने॒येन॑ वापयति वापय त्याग्ने॒येना᳚ ग्ने॒येन॑ वापयति॒ रेतो॒ रेतो॑ वापय त्याग्ने॒येना᳚ ग्ने॒येन॑ वापयति॒ रेतः॑ । \newline
36. वा॒प॒य॒ति॒ रेतो॒ रेतो॑ वापयति वापयति॒ रेतः॑ सौ॒म्येन॑ सौ॒म्येन॒ रेतो॑ वापयति वापयति॒ रेतः॑ सौ॒म्येन॑ । \newline
37. रेतः॑ सौ॒म्येन॑ सौ॒म्येन॒ रेतो॒ रेतः॑ सौ॒म्येन॑ दधाति दधाति सौ॒म्येन॒ रेतो॒ रेतः॑ सौ॒म्येन॑ दधाति । \newline
38. सौ॒म्येन॑ दधाति दधाति सौ॒म्येन॑ सौ॒म्येन॑ दधाति॒ रेतो॒ रेतो॑ दधाति सौ॒म्येन॑ सौ॒म्येन॑ दधाति॒ रेतः॑ । \newline
39. द॒धा॒ति॒ रेतो॒ रेतो॑ दधाति दधाति॒ रेत॑ ए॒वैव रेतो॑ दधाति दधाति॒ रेत॑ ए॒व । \newline
40. रेत॑ ए॒वैव रेतो॒ रेत॑ ए॒व हि॒तꣳ हि॒त मे॒व रेतो॒ रेत॑ ए॒व हि॒तम् । \newline
41. ए॒व हि॒तꣳ हि॒त मे॒वैव हि॒तम् त्वष्टा॒ त्वष्टा॑ हि॒त मे॒वैव हि॒तम् त्वष्टा᳚ । \newline
42. हि॒तम् त्वष्टा॒ त्वष्टा॑ हि॒तꣳ हि॒तम् त्वष्टा॑ रू॒पाणि॑ रू॒पाणि॒ त्वष्टा॑ हि॒तꣳ हि॒तम् त्वष्टा॑ रू॒पाणि॑ । \newline
43. त्वष्टा॑ रू॒पाणि॑ रू॒पाणि॒ त्वष्टा॒ त्वष्टा॑ रू॒पाणि॒ वि वि रू॒पाणि॒ त्वष्टा॒ त्वष्टा॑ रू॒पाणि॒ वि । \newline
44. रू॒पाणि॒ वि वि रू॒पाणि॑ रू॒पाणि॒ वि क॑रोति करोति॒ वि रू॒पाणि॑ रू॒पाणि॒ वि क॑रोति । \newline
45. वि क॑रोति करोति॒ वि वि क॑रोति सारस्व॒तौ सा॑रस्व॒तौ क॑रोति॒ वि वि क॑रोति सारस्व॒तौ । \newline
46. क॒रो॒ति॒ सा॒र॒स्व॒तौ सा॑रस्व॒तौ क॑रोति करोति सारस्व॒तौ भ॑वतो भवतः सारस्व॒तौ क॑रोति करोति सारस्व॒तौ भ॑वतः । \newline
47. सा॒र॒स्व॒तौ भ॑वतो भवतः सारस्व॒तौ सा॑रस्व॒तौ भ॑वत ए॒त दे॒तद् भ॑वतः सारस्व॒तौ सा॑रस्व॒तौ भ॑वत ए॒तत् । \newline
48. भ॒व॒त॒ ए॒त दे॒तद् भ॑वतो भवत ए॒तद् वै वा ए॒तद् भ॑वतो भवत ए॒तद् वै । \newline
49. ए॒तद् वै वा ए॒त दे॒तद् वै दैव्य॒म् दैव्यं॒ ॅवा ए॒त दे॒तद् वै दैव्य᳚म् । \newline
50. वै दैव्य॒म् दैव्यं॒ ॅवै वै दैव्य॑म् मिथु॒नम् मि॑थु॒नम् दैव्यं॒ ॅवै वै दैव्य॑म् मिथु॒नम् । \newline
51. दैव्य॑म् मिथु॒नम् मि॑थु॒नम् दैव्य॒म् दैव्य॑म् मिथु॒नम् दैव्य॒म् दैव्य॑म् मिथु॒नम् दैव्य॒म् दैव्य॑म् मिथु॒नम् दैव्य᳚म् । \newline
52. मि॒थु॒नम् दैव्य॒म् दैव्य॑म् मिथु॒नम् मि॑थु॒नम् दैव्य॑ मे॒वैव दैव्य॑म् मिथु॒नम् मि॑थु॒नम् दैव्य॑ मे॒व । \newline
53. दैव्य॑ मे॒वैव दैव्य॒म् दैव्य॑ मे॒वास्मा॑ अस्मा ए॒व दैव्य॒म् दैव्य॑ मे॒वास्मै᳚ । \newline
54. ए॒वास्मा॑ अस्मा ए॒वैवास्मै॑ मिथु॒नम् मि॑थु॒न म॑स्मा ए॒वैवास्मै॑ मिथु॒नम् । \newline
55. अ॒स्मै॒ मि॒थु॒नम् मि॑थु॒न म॑स्मा अस्मै मिथु॒नम् म॑द्ध्य॒तो म॑द्ध्य॒तो मि॑थु॒न म॑स्मा अस्मै मिथु॒नम् म॑द्ध्य॒तः । \newline
\pagebreak
\markright{ TS 2.4.6.2  \hfill https://www.vedavms.in \hfill}

\section{ TS 2.4.6.2 }

\textbf{TS 2.4.6.2 } \newline
\textbf{Samhita Paata} \newline

मिथु॒नं म॑द्ध्य॒तो द॑धाति॒ पुष्‌ट्यै᳚ प्र॒जन॑नाय सिनीवा॒ल्यै च॒रुर्भ॑वति॒ वाग्वै सि॑नीवा॒ली पुष्टिः॒ खलु॒ वै वाक्पुष्टि॑मे॒व वाच॒मुपै᳚त्यै॒न्द्र उ॑त्त॒मो भ॑वति॒ तेनै॒व तन्मि॑थु॒नꣳ स॒प्तैतानि॑ ह॒वीꣳषि॑ भवन्ति स॒प्त ग्रा॒म्याः प॒शवः॑ स॒प्तार॒ण्याः स॒प्त छन्दाꣳ॑स्यु॒-भय॒स्या-व॑रुद्ध्या॒ अथै॒ता आहु॑ती र्जुहोत्ये॒ते वै दे॒वाः पुष्टि॑पतय॒स्त ए॒वा ( ) स्मि॒न् पुष्टिं॑ दधति॒ पुष्य॑ति प्र॒जया॑ प॒शुभि॒रथो॒ यदे॒ता आहु॑ती र्जु॒होति॒ प्रति॑ष्ठित्यै ॥ \newline

\textbf{Pada Paata} \newline

मि॒थु॒नम् । म॒द्ध्य॒तः । द॒धा॒ति॒ । पुष्ट्यै᳚ । प्र॒जन॑ना॒येति॑ प्र - जन॑नाय । सि॒नी॒वा॒ल्यै । च॒रुः । भ॒व॒ति॒ । वाक् । वै । सि॒नी॒वा॒ली । पुष्टिः॑ । खलु॑ । वै । वाक् । पुष्टि᳚म् । ए॒व । वाच᳚म् । उपेति॑ । ए॒ति॒ । ऐ॒न्द्रः । उ॒त्त॒म इत्यु॑त् - त॒मः । भ॒व॒ति॒ । तेन॑ । ए॒व । तत् । मि॒थु॒नम् । स॒प्त । ए॒तानि॑ । ह॒वीꣳषि॑ । भ॒व॒न्ति॒ । स॒प्त । ग्रा॒म्याः । प॒शवः॑ । स॒प्त । आ॒र॒ण्याः । स॒प्त । छन्दाꣳ॑सि । उ॒भय॑स्य । अव॑रुद्ध्या॒ इत्यव॑ - रु॒द्ध्यै॒ । अथ॑ । ए॒ताः । आहु॑ती॒रित्या - हु॒तीः॒ । जु॒हो॒ति॒ । ए॒ते । वै । दे॒वाः । पुष्टि॑पतय॒ इति॒ पुष्टि॑ - प॒त॒यः॒ । ते । ए॒व ( ) । अ॒स्मि॒न्न् । पुष्टि᳚म् । द॒ध॒ति॒ । पुष्य॑ति । प्र॒जयेति॑ प्र - जया᳚ । प॒शुभि॒रिति॑ प॒शु - भिः॒ । अथो॒ इति॑ । यत् । ए॒ताः । आहु॑ती॒रित्या - हु॒तीः॒ । जु॒होति॑ । प्रति॑ष्ठित्या॒ इति॒ प्रति॑ - स्थि॒त्यै॒ ॥  \newline


\textbf{Krama Paata} \newline

मि॒थु॒नम् म॑द्ध्य॒तः । म॒द्ध्य॒तो द॑धाति । द॒धा॒ति॒ पुष्ट्यै᳚ । पुष्ट्यै᳚ प्र॒जन॑नाय । प्र॒जन॑नाय सिनीवा॒ल्यै । प्र॒जन॑ना॒येति॑ प्र - जन॑नाय । सि॒नी॒वा॒ल्यै च॒रुः । च॒रुर् भ॑वति । भ॒व॒ति॒ वाक् । वाग् वै । वै सि॑नीवा॒ली । सि॒नी॒वा॒ली पुष्टिः॑ । पुष्टिः॒ खलु॑ । खलु॒ वै । वै वाक् । वाक् पुष्टि᳚म् । पुष्टि॑मे॒व । ए॒व वाच᳚म् । वाच॒मुप॑ । उपै॑ति । ए॒त्यै॒न्द्रः । ऐ॒न्द्र उ॑त्त॒मः । उ॒त्त॒मो भ॑वति । उ॒त्त॒म इत्यु॑त् - त॒मः । भ॒व॒ति॒ तेन॑ । तेनै॒व । ए॒व तत् । तन्मि॑थु॒नम् । मि॒थु॒नꣳ स॒प्त । स॒प्तैतानि॑ । ए॒तानि॑ ह॒वीꣳषि॑ । ह॒वीꣳषि॑ भवन्ति । भ॒व॒न्ति॒ स॒प्त । स॒प्त ग्रा॒म्याः । ग्रा॒म्याः प॒शवः॑ । प॒शवः॑ स॒प्त । स॒प्तार॒ण्याः । आ॒र॒ण्याः स॒प्त । स॒प्त छन्दाꣳ॑सि । छन्दाꣳ॑स्यु॒भय॑स्य । उ॒भय॒स्याव॑रुद्ध्यै । अव॑रुद्ध्या॒ अथ॑ । अव॑रुद्ध्या॒ इत्यव॑ - रु॒द्ध्यै॒ । अथै॒ताः । ए॒ता आहु॑तीः । अहु॑तीर् जुहोति । आहु॑ती॒रित्या - हु॒तीः॒ । जु॒हो॒त्ये॒ते । ए॒ते वै । वै दे॒वाः । दे॒वाः पुष्टि॑पतयः । पुष्टि॑पतय॒स्ते । पुष्टि॑पतय॒ इति॒ पुष्टि॑ - प॒त॒यः॒ । त ए॒व ( ) । ए॒वास्मिन्न्॑ । अ॒स्मि॒न् पुष्टि᳚म् । पुष्टि॑म् दधति । द॒ध॒ति॒ पुष्य॑ति । पुष्य॑ति प्र॒जया᳚ । प्र॒जया॑ प॒शुभिः॑ । प्र॒जयेति॑ प्र - जया᳚ । प॒शुभि॒रथो᳚ । प॒शुभि॒रिति॑ प॒शु - भिः॒ । अथो॒ यत् । अथो॒ इत्यथो᳚ । यदे॒ताः । ए॒ता आहु॑तीः । आहु॑तीर् जु॒होति॑ । आहु॑ती॒रित्या - हु॒तीः॒ । जु॒होति॒ प्रति॑ष्ठित्यै । प्रति॑ष्ठित्या॒ इति॒ प्रति॑ - स्थि॒त्यै॒ । \newline

\textbf{Jatai Paata} \newline

1. मि॒थु॒नम् म॑द्ध्य॒तो म॑द्ध्य॒तो मि॑थु॒नम् मि॑थु॒नम् म॑द्ध्य॒तः । \newline
2. म॒द्ध्य॒तो द॑धाति दधाति मद्ध्य॒तो म॑द्ध्य॒तो द॑धाति । \newline
3. द॒धा॒ति॒ पुष्ट्यै॒ पुष्ट्यै॑ दधाति दधाति॒ पुष्ट्यै᳚ । \newline
4. पुष्ट्यै᳚ प्र॒जन॑नाय प्र॒जन॑नाय॒ पुष्ट्यै॒ पुष्ट्यै᳚ प्र॒जन॑नाय । \newline
5. प्र॒जन॑नाय सिनीवा॒ल्यै सि॑नीवा॒ल्यै प्र॒जन॑नाय प्र॒जन॑नाय सिनीवा॒ल्यै । \newline
6. प्र॒जन॑ना॒येति॑ प्र - जन॑नाय । \newline
7. सि॒नी॒वा॒ल्यै च॒रु श्च॒रुः सि॑नीवा॒ल्यै सि॑नीवा॒ल्यै च॒रुः । \newline
8. च॒रुर् भ॑वति भवति च॒रु श्च॒रुर् भ॑वति । \newline
9. भ॒व॒ति॒ वाग् वाग् भ॑वति भवति॒ वाक् । \newline
10. वाग् वै वै वाग् वाग् वै । \newline
11. वै सि॑नीवा॒ली सि॑नीवा॒ली वै वै सि॑नीवा॒ली । \newline
12. सि॒नी॒वा॒ली पुष्टिः॒ पुष्टिः॑ सिनीवा॒ली सि॑नीवा॒ली पुष्टिः॑ । \newline
13. पुष्टिः॒ खलु॒ खलु॒ पुष्टिः॒ पुष्टिः॒ खलु॑ । \newline
14. खलु॒ वै वै खलु॒ खलु॒ वै । \newline
15. वै वाग् वाग् वै वै वाक् । \newline
16. वाक् पुष्टि॒म् पुष्टिं॒ ॅवाग् वाक् पुष्टि᳚म् । \newline
17. पुष्टि॑ मे॒वैव पुष्टि॒म् पुष्टि॑ मे॒व । \newline
18. ए॒व वाचं॒ ॅवाच॑ मे॒वैव वाच᳚म् । \newline
19. वाच॒ मुपोप॒ वाचं॒ ॅवाच॒ मुप॑ । \newline
20. उपै᳚त्ये॒त्युपोपै॑ति । \newline
21. ए॒त्यै॒न्द्र ऐ॒न्द्र ए᳚त्ये त्यै॒न्द्रः । \newline
22. ऐ॒न्द्र उ॑त्त॒म उ॑त्त॒म ऐ॒न्द्र ऐ॒न्द्र उ॑त्त॒मः । \newline
23. उ॒त्त॒मो भ॑वति भव त्युत्त॒म उ॑त्त॒मो भ॑वति । \newline
24. उ॒त्त॒म इत्यु॑त् - त॒मः । \newline
25. भ॒व॒ति॒ तेन॒ तेन॑ भवति भवति॒ तेन॑ । \newline
26. तेनै॒वैव तेन॒ तेनै॒व । \newline
27. ए॒व तत् तदे॒वैव तत् । \newline
28. तन् मि॑थु॒नम् मि॑थु॒नम् तत् तन् मि॑थु॒नम् । \newline
29. मि॒थु॒नꣳ स॒प्त स॒प्त मि॑थु॒नम् मि॑थु॒नꣳ स॒प्त । \newline
30. स॒प्तैता न्ये॒तानि॑ स॒प्त स॒प्तैतानि॑ । \newline
31. ए॒तानि॑ ह॒वीꣳषि॑ ह॒वीꣳ ष्ये॒ता न्ये॒तानि॑ ह॒वीꣳषि॑ । \newline
32. ह॒वीꣳषि॑ भवन्ति भवन्ति ह॒वीꣳषि॑ ह॒वीꣳषि॑ भवन्ति । \newline
33. भ॒व॒न्ति॒ स॒प्त स॒प्त भ॑वन्ति भवन्ति स॒प्त । \newline
34. स॒प्त ग्रा॒म्या ग्रा॒म्याः स॒प्त स॒प्त ग्रा॒म्याः । \newline
35. ग्रा॒म्याः प॒शवः॑ प॒शवो᳚ ग्रा॒म्या ग्रा॒म्याः प॒शवः॑ । \newline
36. प॒शवः॑ स॒प्त स॒प्त प॒शवः॑ प॒शवः॑ स॒प्त । \newline
37. स॒प्ता र॒ण्या आ॑र॒ण्याः स॒प्त स॒प्ता र॒ण्याः । \newline
38. आ॒र॒ण्याः स॒प्त स॒प्ता र॒ण्या आ॑र॒ण्याः स॒प्त । \newline
39. स॒प्त छन्दाꣳ॑सि॒ छन्दाꣳ॑सि स॒प्त स॒प्त छन्दाꣳ॑सि । \newline
40. छन्दाꣳ॑ स्यु॒भय॑स्यो॒ भय॑स्य॒ छन्दाꣳ॑सि॒ छन्दाꣳ॑ स्यु॒भय॑स्य । \newline
41. उ॒भय॒स्या व॑रुद्ध्या॒ अव॑रुद्ध्या उ॒भय॑स्यो॒ भय॒स्या व॑रुद्ध्यै । \newline
42. अव॑रुद्ध्या॒ अथाथा व॑रुद्ध्या॒ अव॑रुद्ध्या॒ अथ॑ । \newline
43. अव॑रुद्ध्या॒ इत्यव॑ - रु॒द्ध्यै॒ । \newline
44. अथै॒ता ए॒ता अथाथै॒ताः । \newline
45. ए॒ता आहु॑ती॒ राहु॑ती रे॒ता ए॒ता आहु॑तीः । \newline
46. आहु॑तीर् जुहोति जुहो॒ त्याहु॑ती॒ राहु॑तीर् जुहोति । \newline
47. आहु॑ती॒रित्या - हु॒तीः॒ । \newline
48. जु॒हो॒ त्ये॒त ए॒ते जु॑होति जुहो त्ये॒ते । \newline
49. ए॒ते वै वा ए॒त ए॒ते वै । \newline
50. वै दे॒वा दे॒वा वै वै दे॒वाः । \newline
51. दे॒वाः पुष्टि॑पतयः॒ पुष्टि॑पतयो दे॒वा दे॒वाः पुष्टि॑पतयः । \newline
52. पुष्टि॑पतय॒ स्ते ते पुष्टि॑पतयः॒ पुष्टि॑पतय॒ स्ते । \newline
53. पुष्टि॑पतय॒ इति॒ पुष्टि॑ - प॒त॒यः॒ । \newline
54. त ए॒वैव ते त ए॒व । \newline
55. ए॒वास्मि॑न् नस्मिन् ने॒वैवास्मिन्न्॑ । \newline
56. अ॒स्मि॒न् पुष्टि॒म् पुष्टि॑ मस्मिन् नस्मि॒न् पुष्टि᳚म् । \newline
57. पुष्टि॑म् दधति दधति॒ पुष्टि॒म् पुष्टि॑म् दधति । \newline
58. द॒ध॒ति॒ पुष्य॑ति॒ पुष्य॑ति दधति दधति॒ पुष्य॑ति । \newline
59. पुष्य॑ति प्र॒जया᳚ प्र॒जया॒ पुष्य॑ति॒ पुष्य॑ति प्र॒जया᳚ । \newline
60. प्र॒जया॑ प॒शुभिः॑ प॒शुभिः॑ प्र॒जया᳚ प्र॒जया॑ प॒शुभिः॑ । \newline
61. प्र॒जयेति॑ प्र - जया᳚ । \newline
62. प॒शुभि॒ रथो॒ अथो॑ प॒शुभिः॑ प॒शुभि॒ रथो᳚ । \newline
63. प॒शुभि॒रिति॑ प॒शु - भिः॒ । \newline
64. अथो॒ यद् यदथो॒ अथो॒ यत् । \newline
65. अथो॒ इत्यथो᳚ । \newline
66. यदे॒ता ए॒ता यद् यदे॒ताः । \newline
67. ए॒ता आहु॑ती॒ राहु॑ती रे॒ता ए॒ता आहु॑तीः । \newline
68. आहु॑तीर् जु॒होति॑ जु॒हो त्याहु॑ती॒ राहु॑तीर् जु॒होति॑ । \newline
69. आहु॑ती॒रित्या - हु॒तीः॒ । \newline
70. जु॒होति॒ प्रति॑ष्ठित्यै॒ प्रति॑ष्ठित्यै जु॒होति॑ जु॒होति॒ प्रति॑ष्ठित्यै । \newline
71. प्रति॑ष्ठित्या॒ इति॒ प्रति॑ - स्थि॒त्यै॒ । \newline

\textbf{Ghana Paata } \newline

1. मि॒थु॒नम् म॑द्ध्य॒तो म॑द्ध्य॒तो मि॑थु॒नम् मि॑थु॒नम् म॑द्ध्य॒तो द॑धाति दधाति मद्ध्य॒तो मि॑थु॒नम् मि॑थु॒नम् म॑द्ध्य॒तो द॑धाति । \newline
2. म॒द्ध्य॒तो द॑धाति दधाति मद्ध्य॒तो म॑द्ध्य॒तो द॑धाति॒ पुष्ट्यै॒ पुष्ट्यै॑ दधाति मद्ध्य॒तो म॑द्ध्य॒तो द॑धाति॒ पुष्ट्यै᳚ । \newline
3. द॒धा॒ति॒ पुष्ट्यै॒ पुष्ट्यै॑ दधाति दधाति॒ पुष्ट्यै᳚ प्र॒जन॑नाय प्र॒जन॑नाय॒ पुष्ट्यै॑ दधाति दधाति॒ पुष्ट्यै᳚ प्र॒जन॑नाय । \newline
4. पुष्ट्यै᳚ प्र॒जन॑नाय प्र॒जन॑नाय॒ पुष्ट्यै॒ पुष्ट्यै᳚ प्र॒जन॑नाय सिनीवा॒ल्यै सि॑नीवा॒ल्यै प्र॒जन॑नाय॒ पुष्ट्यै॒ पुष्ट्यै᳚ प्र॒जन॑नाय सिनीवा॒ल्यै । \newline
5. प्र॒जन॑नाय सिनीवा॒ल्यै सि॑नीवा॒ल्यै प्र॒जन॑नाय प्र॒जन॑नाय सिनीवा॒ल्यै च॒रु श्च॒रुः सि॑नीवा॒ल्यै प्र॒जन॑नाय प्र॒जन॑नाय सिनीवा॒ल्यै च॒रुः । \newline
6. प्र॒जन॑ना॒येति॑ प्र - जन॑नाय । \newline
7. सि॒नी॒वा॒ल्यै च॒रु श्च॒रुः सि॑नीवा॒ल्यै सि॑नीवा॒ल्यै च॒रुर् भ॑वति भवति च॒रुः सि॑नीवा॒ल्यै सि॑नीवा॒ल्यै च॒रुर् भ॑वति । \newline
8. च॒रुर् भ॑वति भवति च॒रु श्च॒रुर् भ॑वति॒ वाग् वाग् भ॑वति च॒रु श्च॒रुर् भ॑वति॒ वाक् । \newline
9. भ॒व॒ति॒ वाग् वाग् भ॑वति भवति॒ वाग् वै वै वाग् भ॑वति भवति॒ वाग् वै । \newline
10. वाग् वै वै वाग् वाग् वै सि॑नीवा॒ली सि॑नीवा॒ली वै वाग् वाग् वै सि॑नीवा॒ली । \newline
11. वै सि॑नीवा॒ली सि॑नीवा॒ली वै वै सि॑नीवा॒ली पुष्टिः॒ पुष्टिः॑ सिनीवा॒ली वै वै सि॑नीवा॒ली पुष्टिः॑ । \newline
12. सि॒नी॒वा॒ली पुष्टिः॒ पुष्टिः॑ सिनीवा॒ली सि॑नीवा॒ली पुष्टिः॒ खलु॒ खलु॒ पुष्टिः॑ सिनीवा॒ली सि॑नीवा॒ली पुष्टिः॒ खलु॑ । \newline
13. पुष्टिः॒ खलु॒ खलु॒ पुष्टिः॒ पुष्टिः॒ खलु॒ वै वै खलु॒ पुष्टिः॒ पुष्टिः॒ खलु॒ वै । \newline
14. खलु॒ वै वै खलु॒ खलु॒ वै वाग् वाग् वै खलु॒ खलु॒ वै वाक् । \newline
15. वै वाग् वाग् वै वै वाक् पुष्टि॒म् पुष्टिं॒ ॅवाग् वै वै वाक् पुष्टि᳚म् । \newline
16. वाक् पुष्टि॒म् पुष्टिं॒ ॅवाग् वाक् पुष्टि॑ मे॒वैव पुष्टिं॒ ॅवाग् वाक् पुष्टि॑ मे॒व । \newline
17. पुष्टि॑ मे॒वैव पुष्टि॒म् पुष्टि॑ मे॒व वाचं॒ ॅवाच॑ मे॒व पुष्टि॒म् पुष्टि॑ मे॒व वाच᳚म् । \newline
18. ए॒व वाचं॒ ॅवाच॑ मे॒वैव वाच॒ मुपोप॒ वाच॑ मे॒वैव वाच॒ मुप॑ । \newline
19. वाच॒ मुपोप॒ वाचं॒ ॅवाच॒ मुपै᳚त्ये॒त्युप॒ वाचं॒ ॅवाच॒ मुपै॑ति । \newline
20. उपै᳚त्ये॒ त्युपोपै᳚त्यै॒न्द्र ऐ॒न्द्र ए॒त्युपोपै᳚त्यै॒न्द्रः । \newline
21. ए॒त्यै॒न्द्र ऐ॒न्द्र ए᳚त्येत्यै॒न्द्र उ॑त्त॒म उ॑त्त॒म ऐ॒न्द्र ए᳚त्येत्यै॒न्द्र उ॑त्त॒मः । \newline
22. ऐ॒न्द्र उ॑त्त॒म उ॑त्त॒म ऐ॒न्द्र ऐ॒न्द्र उ॑त्त॒मो भ॑वति भव त्युत्त॒म ऐ॒न्द्र ऐ॒न्द्र उ॑त्त॒मो भ॑वति । \newline
23. उ॒त्त॒मो भ॑वति भव त्युत्त॒म उ॑त्त॒मो भ॑वति॒ तेन॒ तेन॑ भव त्युत्त॒म उ॑त्त॒मो भ॑वति॒ तेन॑ । \newline
24. उ॒त्त॒म इत्यु॑त् - त॒मः । \newline
25. भ॒व॒ति॒ तेन॒ तेन॑ भवति भवति॒ तेनै॒वैव तेन॑ भवति भवति॒ तेनै॒व । \newline
26. तेनै॒वैव तेन॒ तेनै॒व तत् तदे॒व तेन॒ तेनै॒व तत् । \newline
27. ए॒व तत् तदे॒वैव तन् मि॑थु॒नम् मि॑थु॒नम् तदे॒वैव तन् मि॑थु॒नम् । \newline
28. तन् मि॑थु॒नम् मि॑थु॒नम् तत् तन् मि॑थु॒नꣳ स॒प्त स॒प्त मि॑थु॒नम् तत् तन् मि॑थु॒नꣳ स॒प्त । \newline
29. मि॒थु॒नꣳ स॒प्त स॒प्त मि॑थु॒नम् मि॑थु॒नꣳ स॒प्तैता न्ये॒तानि॑ स॒प्त मि॑थु॒नम् मि॑थु॒नꣳ स॒प्तैतानि॑ । \newline
30. स॒प्तैता न्ये॒तानि॑ स॒प्त स॒प्तैतानि॑ ह॒वीꣳषि॑ ह॒वीꣳ ष्ये॒तानि॑ स॒प्त स॒प्तैतानि॑ ह॒वीꣳषि॑ । \newline
31. ए॒तानि॑ ह॒वीꣳषि॑ ह॒वीꣳ ष्ये॒ता न्ये॒तानि॑ ह॒वीꣳषि॑ भवन्ति भवन्ति ह॒वीꣳ ष्ये॒ता न्ये॒तानि॑ ह॒वीꣳषि॑ भवन्ति । \newline
32. ह॒वीꣳषि॑ भवन्ति भवन्ति ह॒वीꣳषि॑ ह॒वीꣳषि॑ भवन्ति स॒प्त स॒प्त भ॑वन्ति ह॒वीꣳषि॑ ह॒वीꣳषि॑ भवन्ति स॒प्त । \newline
33. भ॒व॒न्ति॒ स॒प्त स॒प्त भ॑वन्ति भवन्ति स॒प्त ग्रा॒म्या ग्रा॒म्याः स॒प्त भ॑वन्ति भवन्ति स॒प्त ग्रा॒म्याः । \newline
34. स॒प्त ग्रा॒म्या ग्रा॒म्याः स॒प्त स॒प्त ग्रा॒म्याः प॒शवः॑ प॒शवो᳚ ग्रा॒म्याः स॒प्त स॒प्त ग्रा॒म्याः प॒शवः॑ । \newline
35. ग्रा॒म्याः प॒शवः॑ प॒शवो᳚ ग्रा॒म्या ग्रा॒म्याः प॒शवः॑ स॒प्त स॒प्त प॒शवो᳚ ग्रा॒म्या ग्रा॒म्याः प॒शवः॑ स॒प्त । \newline
36. प॒शवः॑ स॒प्त स॒प्त प॒शवः॑ प॒शवः॑ स॒प्तार॒ण्या आ॑र॒ण्याः स॒प्त प॒शवः॑ प॒शवः॑ स॒प्तार॒ण्याः । \newline
37. स॒प्तार॒ण्या आ॑र॒ण्याः स॒प्त स॒प्तार॒ण्याः स॒प्त स॒प्तार॒ण्याः स॒प्त स॒प्तार॒ण्याः स॒प्त । \newline
38. आ॒र॒ण्याः स॒प्त स॒प्तार॒ण्या आ॑र॒ण्याः स॒प्त छन्दाꣳ॑सि॒ छन्दाꣳ॑सि स॒प्तार॒ण्या आ॑र॒ण्याः स॒प्त छन्दाꣳ॑सि । \newline
39. स॒प्त छन्दाꣳ॑सि॒ छन्दाꣳ॑सि स॒प्त स॒प्त छन्दाꣳ॑ स्यु॒भय॑स्यो॒ भय॑स्य॒ छन्दाꣳ॑सि स॒प्त स॒प्त छन्दाꣳ॑ स्यु॒भय॑स्य । \newline
40. छन्दाꣳ॑ स्यु॒भय॑स्यो॒ भय॑स्य॒ छन्दाꣳ॑सि॒ छन्दाꣳ॑ स्यु॒भय॒स्या व॑रुद्ध्या॒ अव॑रुद्ध्या उ॒भय॑स्य॒ छन्दाꣳ॑सि॒ छन्दाꣳ॑ स्यु॒भय॒स्या व॑रुद्ध्यै । \newline
41. उ॒भय॒स्या व॑रुद्ध्या॒ अव॑रुद्ध्या उ॒भय॑स्यो॒ भय॒स्या व॑रुद्ध्या॒ अथाथा व॑रुद्ध्या उ॒भय॑स्यो॒ भय॒स्या व॑रुद्ध्या॒ अथ॑ । \newline
42. अव॑रुद्ध्या॒ अथाथा व॑रुद्ध्या॒ अव॑रुद्ध्या॒ अथै॒ता ए॒ता अथाव॑रुद्ध्या॒ अव॑रुद्ध्या॒ अथै॒ताः । \newline
43. अव॑रुद्ध्या॒ इत्यव॑ - रु॒द्ध्यै॒ । \newline
44. अथै॒ता ए॒ता अथाथै॒ता आहु॑ती॒ राहु॑ती रे॒ता अथाथै॒ता आहु॑तीः । \newline
45. ए॒ता आहु॑ती॒ राहु॑ती रे॒ता ए॒ता आहु॑तीर् जुहोति जुहो॒ त्याहु॑ती रे॒ता ए॒ता आहु॑तीर् जुहोति । \newline
46. आहु॑तीर् जुहोति जुहो॒ त्याहु॑ती॒ राहु॑तीर् जुहो त्ये॒त ए॒ते जु॑हो॒ त्याहु॑ती॒ राहु॑तीर् जुहो त्ये॒ते । \newline
47. आहु॑ती॒रित्या - हु॒तीः॒ । \newline
48. जु॒हो॒त्ये॒त ए॒ते जु॑होति जुहोत्ये॒ते वै वा ए॒ते जु॑होति जुहोत्ये॒ते वै । \newline
49. ए॒ते वै वा ए॒त ए॒ते वै दे॒वा दे॒वा वा ए॒त ए॒ते वै दे॒वाः । \newline
50. वै दे॒वा दे॒वा वै वै दे॒वाः पुष्टि॑पतयः॒ पुष्टि॑पतयो दे॒वा वै वै दे॒वाः पुष्टि॑पतयः । \newline
51. दे॒वाः पुष्टि॑पतयः॒ पुष्टि॑पतयो दे॒वा दे॒वाः पुष्टि॑पतय॒ स्ते ते पुष्टि॑पतयो दे॒वा दे॒वाः पुष्टि॑पतय॒ स्ते । \newline
52. पुष्टि॑पतय॒ स्ते ते पुष्टि॑पतयः॒ पुष्टि॑पतय॒ स्त ए॒वैव ते पुष्टि॑पतयः॒ पुष्टि॑पतय॒ स्त ए॒व । \newline
53. पुष्टि॑पतय॒ इति॒ पुष्टि॑ - प॒त॒यः॒ । \newline
54. त ए॒वैव ते त ए॒वास्मि॑न् नस्मिन् ने॒व ते त ए॒वास्मिन्न्॑ । \newline
55. ए॒वास्मि॑न् नस्मिन् ने॒वैवास्मि॒न् पुष्टि॒म् पुष्टि॑ मस्मिन् ने॒वैवास्मि॒न् पुष्टि᳚म् । \newline
56. अ॒स्मि॒न् पुष्टि॒म् पुष्टि॑ मस्मिन् नस्मि॒न् पुष्टि॑म् दधति दधति॒ पुष्टि॑ मस्मिन् नस्मि॒न् पुष्टि॑म् दधति । \newline
57. पुष्टि॑म् दधति दधति॒ पुष्टि॒म् पुष्टि॑म् दधति॒ पुष्य॑ति॒ पुष्य॑ति दधति॒ पुष्टि॒म् पुष्टि॑म् दधति॒ पुष्य॑ति । \newline
58. द॒ध॒ति॒ पुष्य॑ति॒ पुष्य॑ति दधति दधति॒ पुष्य॑ति प्र॒जया᳚ प्र॒जया॒ पुष्य॑ति दधति दधति॒ पुष्य॑ति प्र॒जया᳚ । \newline
59. पुष्य॑ति प्र॒जया᳚ प्र॒जया॒ पुष्य॑ति॒ पुष्य॑ति प्र॒जया॑ प॒शुभिः॑ प॒शुभिः॑ प्र॒जया॒ पुष्य॑ति॒ पुष्य॑ति प्र॒जया॑ प॒शुभिः॑ । \newline
60. प्र॒जया॑ प॒शुभिः॑ प॒शुभिः॑ प्र॒जया᳚ प्र॒जया॑ प॒शुभि॒ रथो॒ अथो॑ प॒शुभिः॑ प्र॒जया᳚ प्र॒जया॑ प॒शुभि॒ रथो᳚ । \newline
61. प्र॒जयेति॑ प्र - जया᳚ । \newline
62. प॒शुभि॒ रथो॒ अथो॑ प॒शुभिः॑ प॒शुभि॒ रथो॒ यद् यदथो॑ प॒शुभिः॑ प॒शुभि॒ रथो॒ यत् । \newline
63. प॒शुभि॒रिति॑ प॒शु - भिः॒ । \newline
64. अथो॒ यद् यदथो॒ अथो॒ यदे॒ता ए॒ता यदथो॒ अथो॒ यदे॒ताः । \newline
65. अथो॒ इत्यथो᳚ । \newline
66. यदे॒ता ए॒ता यद् यदे॒ता आहु॑ती॒ राहु॑ती रे॒ता यद् यदे॒ता आहु॑तीः । \newline
67. ए॒ता आहु॑ती॒ राहु॑ती रे॒ता ए॒ता आहु॑तीर् जु॒होति॑ जु॒हो त्याहु॑ती रे॒ता ए॒ता आहु॑तीर् जु॒होति॑ । \newline
68. आहु॑तीर् जु॒होति॑ जु॒हो त्याहु॑ती॒ राहु॑तीर् जु॒होति॒ प्रति॑ष्ठित्यै॒ प्रति॑ष्ठित्यै जु॒हो त्याहु॑ती॒ राहु॑तीर् जु॒होति॒ प्रति॑ष्ठित्यै । \newline
69. आहु॑ती॒रित्या - हु॒तीः॒ । \newline
70. जु॒होति॒ प्रति॑ष्ठित्यै॒ प्रति॑ष्ठित्यै जु॒होति॑ जु॒होति॒ प्रति॑ष्ठित्यै । \newline
71. प्रति॑ष्ठित्या॒ इति॒ प्रति॑ - स्थि॒त्यै॒ । \newline
\pagebreak
\markright{ TS 2.4.7.1  \hfill https://www.vedavms.in \hfill}

\section{ TS 2.4.7.1 }

\textbf{TS 2.4.7.1 } \newline
\textbf{Samhita Paata} \newline

मा॒रु॒तम॑सि म॒रुता॒मोजो॒ऽपां धारां᳚ भिन्धि र॒मय॑त मरुतः श्ये॒नमा॒यिनं॒ मनो॑जव सं॒ ॅवृष॑णꣳ सुवृ॒क्तिं ॥ येन॒ शर्द्ध॑ उ॒ग्रमव॑-सृष्ट॒मेति॒ तद॑श्विना॒ परि॑ धत्तꣳ स्व॒स्ति । पु॒रो॒वा॒तो वर्.ष॑ञ्जि॒न्वरा॒वृथ्-स्वाहा॑ वा॒तावद्- वर्.ष॑न्नु॒ग्ररा॒वृथ् स्वाहा᳚ स्त॒नय॒न् वर्.ष॑न् भी॒मरा॒वथ्‌स्वाहा॑ ऽनश॒न्य॑व॒स्फूर्ज॑न्-दि॒द्युद्-वर्.ष॑न्-त्वे॒षरा॒वृथ् स्वाहा॑ ऽतिरा॒त्रं॒ ॅवर्.ष॑न् पू॒र्तिरा॒वृथ् - [  ] \newline

\textbf{Pada Paata} \newline

मा॒रु॒तम् । अ॒सि॒ । म॒रुता᳚म् । ओजः॑ । अ॒पाम् । धारा᳚म् । भि॒न्धि॒ । र॒मय॑त । म॒रु॒तः॒ । श्ये॒नम् । आ॒यिन᳚म् । मनो॑जवस॒मिति॒ मनः॑ - ज॒व॒स॒म् । वृष॑णम् । सु॒वृ॒क्तिमिति॑ सु - वृ॒क्तिम् ॥ येन॑ । शर्द्धः॑ । उ॒ग्रम् । अव॑सृष्ट॒मित्यव॑ - सृ॒ष्ट॒म् । एति॑ । तत् । अ॒श्वि॒ना॒ । परीति॑ । ध॒त्त॒म् । स्व॒स्ति ॥ पु॒रो॒वा॒त इति॑ पुरः - वा॒तः । वर्.षन्न्॑ । जि॒न्वः । आ॒वृदित्या᳚ - वृत् । स्वाहा᳚ । वा॒ताव॒दिति॑ वा॒त - व॒त् । वर्.षन्न्॑ । उ॒ग्रः । आ॒वृदित्या᳚ - वृत् । स्वाहा᳚ । स्त॒नयन्न्॑ । वर्.षन्न्॑ । भी॒मः । आ॒वृदित्या᳚ - वृत् । स्वाहा᳚ । अ॒न॒श॒नि । अ॒व॒स्फूर्ज॒न्नित्य॑व-स्फूर्जन्न्॑ । दि॒द्युत् । वर्.षन्न्॑ । त्वे॒षः । आ॒वृदित्या᳚ - वृत् । स्वाहा᳚ । अ॒ति॒रा॒त्रमित्य॑ति - रा॒त्रम् । वर्.षन्न्॑ । पू॒र्तिः । आ॒वृदित्या᳚ - वृत् ।  \newline


\textbf{Krama Paata} \newline

मा॒रु॒तम॑सि । अ॒सि॒ म॒रुता᳚म् । म॒रुता॒मोजः॑ । ओजो॒ ऽपाम् । अ॒पाम् धारा᳚म् । धारा᳚म् भिन्धि । भि॒न्धि॒ र॒मय॑त । र॒मय॑त मरुतः । म॒रु॒तः॒ श्ये॒नम् । श्ये॒नमा॒यिन᳚म् । आ॒यिन॒म् मनो॑जवसम् । मनो॑जवसं॒ ॅवृष॑णम् । मनो॑जवस॒मिति॒ मनः॑ - 
ज॒व॒स॒म् । वृष॑णꣳ सुवृ॒क्तिम् । सु॒वृ॒क्तिमिति॑ सु - वृ॒क्तिम् ॥ येन॒ शर्द्धः॑ । शर्द्ध॑ उ॒ग्रम् । उ॒ग्रमव॑सृष्टम् । अव॑सृष्ट॒मेति॑ । अव॑सृष्ट॒मित्यव॑ - सृ॒ष्ट॒म् । एति॒ तत् । तद॑श्विना । अ॒श्वि॒ना॒ परि॑ । परि॑ धत्तम् । ध॒त्तꣳ॒॒ स्व॒स्ति । स्व॒स्तीति॑ स्व॒स्ति ॥ पु॒रो॒वा॒तो वर्.षन्न्॑ । पु॒रो॒वा॒त इति॑ पुरः - वा॒तः । 
वर्.ष॑न् जि॒न्वः । जि॒न्वरा॒वृत् । आ॒वृथ् स्वाहा᳚ । आ॒वृदित्या᳚ - वृत् । स्वाहा॑ वा॒ताव॑त् । वा॒ताव॒द् वर्.षन्न्॑ । वा॒ताव॒दिति॑ वा॒त - व॒त्॒ । वर्.ष॑न्नु॒ग्रः । उ॒ग्ररा॒वृत् । आ॒वृथ् स्वाहा᳚ । आ॒वृदित्या᳚ - वृत् । स्वाहा᳚ स्त॒नयन्न्॑ । स्त॒नय॒न् वर्.षन्न्॑ । वर्.ष॑न् भी॒मः । भी॒मरा॒वृत् । आ॒वृथ् स्वाहा᳚ । आ॒वृदित्या᳚ - वृत् । स्वाहा॑ ऽनश॒नि । अ॒न॒श॒न्य॑व॒स्फूर्जन्न्॑ । अ॒व॒स्फूर्ज॑न् दि॒द्युत् । अ॒व॒स्फूर्ज॒न्नित्य॑व - स्फूर्जन्न्॑ । दि॒द्युद् वर्.षन्न्॑ । वर्.ष॑न् त्वे॒षः । त्वे॒षरा॒वृत् । आ॒वृथ् स्वाहा᳚ । आ॒वृदित्या᳚ - वृत् । स्वाहा॑ ऽतिरा॒त्रम् । अ॒ति॒रा॒त्रं ॅवर्.षन्न्॑ । अ॒ति॒रा॒त्रमित्य॑ति - रा॒त्रम् । वर्.ष॑न् पू॒र्तिः । पू॒र्तिरा॒वृत् ( ) । आ॒वृथ् स्वाहा᳚ । आ॒वृदित्या᳚ - वृत् \newline

\textbf{Jatai Paata} \newline

1. मा॒रु॒त म॑स्यसि मारु॒तम् मा॑रु॒त म॑सि । \newline
2. अ॒सि॒ म॒रुता᳚म् म॒रुता॑ मस्यसि म॒रुता᳚म् । \newline
3. म॒रुता॒ मोज॒ ओजो॑ म॒रुता᳚म् म॒रुता॒ मोजः॑ । \newline
4. ओजो॒ ऽपा म॒पा मोज॒ ओजो॒ ऽपाम् । \newline
5. अ॒पाम् धारा॒म् धारा॑ म॒पा म॒पाम् धारा᳚म् । \newline
6. धारा᳚म् भिन्धि भिन्धि॒ धारा॒म् धारा᳚म् भिन्धि । \newline
7. भि॒न्धि॒ र॒मय॑त र॒मय॑त भिन्धि भिन्धि र॒मय॑त । \newline
8. र॒मय॑त मरुतो मरुतो र॒मय॑त र॒मय॑त मरुतः । \newline
9. म॒रु॒तः॒ श्ये॒नꣳ श्ये॒नम् म॑रुतो मरुतः श्ये॒नम् । \newline
10. श्ये॒न मा॒यिन॑ मा॒यिनꣳ॑ श्ये॒नꣳ श्ये॒न मा॒यिन᳚म् । \newline
11. आ॒यिन॒म् मनो॑जवस॒म् मनो॑जवस मा॒यिन॑ मा॒यिन॒म् मनो॑जवसम् । \newline
12. मनो॑जवसं॒ ॅवृष॑णं॒ ॅवृष॑ण॒म् मनो॑जवस॒म् मनो॑जवसं॒ ॅवृष॑णम् । \newline
13. मनो॑जवस॒मिति॒ मनः॑ - ज॒व॒स॒म् । \newline
14. वृष॑णꣳ सुवृ॒क्तिꣳ सु॑वृ॒क्तिं ॅवृष॑णं॒ ॅवृष॑णꣳ सुवृ॒क्तिम् । \newline
15. सु॒वृ॒क्तिमिति॑ सु - वृ॒क्तिम् । \newline
16. येन॒ शर्द्धः॒ शर्द्धो॒ येन॒ येन॒ शर्द्धः॑ । \newline
17. शर्द्ध॑ उ॒ग्र मु॒ग्रꣳ शर्द्धः॒ शर्द्ध॑ उ॒ग्रम् । \newline
18. उ॒ग्र मव॑सृष्ट॒ मव॑सृष्ट मु॒ग्र मु॒ग्र मव॑सृष्टम् । \newline
19. अव॑सृष्ट॒ मेत्ये त्यव॑सृष्ट॒ मव॑सृष्ट॒ मेति॑ । \newline
20. अव॑सृष्ट॒मित्यव॑ - सृ॒ष्ट॒म् । \newline
21. एति॒ तत् तदे त्येति॒ तत् । \newline
22. तद॑श्विना ऽश्विना॒ तत् तद॑श्विना । \newline
23. अ॒श्वि॒ना॒ परि॒ पर्य॑श्विना ऽश्विना॒ परि॑ । \newline
24. परि॑ धत्तम् धत्त॒म् परि॒ परि॑ धत्तम् । \newline
25. ध॒त्तꣳ॒॒ स्व॒स्ति स्व॒स्ति ध॑त्तम् धत्तꣳ स्व॒स्ति । \newline
26. स्व॒स्तीति॑ स्व॒स्ति । \newline
27. पु॒रो॒वा॒तो वर्.ष॒न्॒. वर्.ष॑न् पुरोवा॒तः पु॑रोवा॒तो वर्.षन्न्॑ । \newline
28. पु॒रो॒वा॒त इति॑ पुरः - वा॒तः । \newline
29. वर्.ष॑न् जि॒न्वो जि॒न्वो वर्.ष॒न्॒. वर्.ष॑न् जि॒न्वः । \newline
30. जि॒न्व रा॒वृ दा॒वृज् जि॒न्वो जि॒न्व रा॒वृत् । \newline
31. आ॒वृथ् स्वाहा॒ स्वाहा॒ ऽऽवृ दा॒वृथ् स्वाहा᳚ । \newline
32. आ॒वृदित्या᳚ - वृत् । \newline
33. स्वाहा॑ वा॒ताव॑द् वा॒ताव॒थ् स्वाहा॒ स्वाहा॑ वा॒ताव॑त् । \newline
34. वा॒ताव॒द् वर्.ष॒न्॒. वर्.ष॑न्. वा॒ताव॑द् वा॒ताव॒द् वर्.षन्न्॑ । \newline
35. वा॒ताव॒दिति॑ वा॒त - व॒त् । \newline
36. वर्.ष॑न् नु॒ग्र उ॒ग्रो वर्.ष॒न्॒. वर्.ष॑न् नु॒ग्रः । \newline
37. उ॒ग्र रा॒वृ दा॒वृ दु॒ग्र उ॒ग्र रा॒वृत् । \newline
38. आ॒वृथ् स्वाहा॒ स्वाहा॒ ऽऽवृ दा॒वृथ् स्वाहा᳚ । \newline
39. आ॒वृदित्या᳚ - वृत् । \newline
40. स्वाहा᳚ स्त॒नयन्᳚ थ्स्त॒नय॒न् थ्स्वाहा॒ स्वाहा᳚ स्त॒नयन्न्॑ । \newline
41. स्त॒नय॒न्॒. वर्.ष॒न्॒. वर्.षन्᳚ थ्स्त॒नयन्᳚ थ्स्त॒नय॒न्॒. वर्.षन्न्॑ । \newline
42. वर्.ष॑न् भी॒मो भी॒मो वर्.ष॒न्॒. वर्.ष॑न् भी॒मः । \newline
43. भी॒म रा॒वृ दा॒वृद् भी॒मो भी॒म रा॒वृत् । \newline
44. आ॒वृथ् स्वाहा॒ स्वाहा॒ ऽऽवृ दा॒वृथ् स्वाहा᳚ । \newline
45. आ॒वृदित्या᳚ - वृत् । \newline
46. स्वाहा॑ ऽनश॒ न्य॑नश॒नि स्वाहा॒ स्वाहा॑ ऽनश॒नि । \newline
47. अ॒न॒श॒ न्य॑व॒स्फूर्ज॑न् नव॒स्फूर्ज॑न् ननश॒ न्य॑नश॒ न्य॑व॒स्फूर्जन्न्॑ । \newline
48. अ॒व॒स्फूर्ज॑न् दि॒द्युद् दि॒द्यु द॑व॒स्फूर्ज॑न् नव॒स्फूर्ज॑न् दि॒द्युत् । \newline
49. अ॒व॒स्फूर्ज॒न्नित्य॑व - स्फूर्जन्न्॑ । \newline
50. दि॒द्युद् वर्.ष॒न्॒. वर्.ष॑न् दि॒द्युद् दि॒द्युद् वर्.षन्न्॑ । \newline
51. वर्.ष॑न् त्वे॒ष स्त्वे॒षो वर्.ष॒न्॒. वर्.ष॑न् त्वे॒षः । \newline
52. त्वे॒ष रा॒वृ दा॒वृत् त्वे॒ष स्त्वे॒ष रा॒वृत् । \newline
53. आ॒वृथ् स्वाहा॒ स्वाहा॒ ऽऽवृ दा॒वृथ् स्वाहा᳚ । \newline
54. आ॒वृदित्या᳚ - वृत् । \newline
55. स्वाहा॑ ऽतिरा॒त्र म॑तिरा॒त्रꣳ स्वाहा॒ स्वाहा॑ ऽतिरा॒त्रम् । \newline
56. अ॒ति॒रा॒त्रं ॅवर्.ष॒न्॒. वर्.ष॑न् नतिरा॒त्र म॑तिरा॒त्रं ॅवर्.षन्न्॑ । \newline
57. अ॒ति॒रा॒त्रमित्य॑ति - रा॒त्रम् । \newline
58. वर्.ष॑न् पू॒र्तिः पू॒र्तिर् वर्.ष॒न्॒. वर्.ष॑न् पू॒र्तिः । \newline
59. पू॒र्ति रा॒वृ दा॒वृत् पू॒र्तिः पू॒र्ति रा॒वृत् । \newline
60. आ॒वृथ् स्वाहा॒ स्वाहा॒ ऽऽवृ दा॒वृथ् स्वाहा᳚ । \newline
61. आ॒वृदित्या᳚ - वृत् । \newline

\textbf{Ghana Paata } \newline

1. मा॒रु॒त म॑स्यसि मारु॒तम् मा॑रु॒त म॑सि म॒रुता᳚म् म॒रुता॑ मसि मारु॒तम् मा॑रु॒त म॑सि म॒रुता᳚म् । \newline
2. अ॒सि॒ म॒रुता᳚म् म॒रुता॑ मस्यसि म॒रुता॒ मोज॒ ओजो॑ म॒रुता॑ मस्यसि म॒रुता॒ मोजः॑ । \newline
3. म॒रुता॒ मोज॒ ओजो॑ म॒रुता᳚म् म॒रुता॒ मोजो॒ ऽपा म॒पा मोजो॑ म॒रुता᳚म् म॒रुता॒ मोजो॒ ऽपाम् । \newline
4. ओजो॒ ऽपा म॒पा मोज॒ ओजो॒ ऽपाम् धारा॒म् धारा॑ म॒पा मोज॒ ओजो॒ ऽपाम् धारा᳚म् । \newline
5. अ॒पाम् धारा॒म् धारा॑ म॒पा म॒पाम् धारा᳚म् भिन्धि भिन्धि॒ धारा॑ म॒पा म॒पाम् धारा᳚म् भिन्धि । \newline
6. धारा᳚म् भिन्धि भिन्धि॒ धारा॒म् धारा᳚म् भिन्धि र॒मय॑त र॒मय॑त भिन्धि॒ धारा॒म् धारा᳚म् भिन्धि र॒मय॑त । \newline
7. भि॒न्धि॒ र॒मय॑त र॒मय॑त भिन्धि भिन्धि र॒मय॑त मरुतो मरुतो र॒मय॑त भिन्धि भिन्धि र॒मय॑त मरुतः । \newline
8. र॒मय॑त मरुतो मरुतो र॒मय॑त र॒मय॑त मरुतः श्ये॒नꣳ श्ये॒नम् म॑रुतो र॒मय॑त र॒मय॑त मरुतः श्ये॒नम् । \newline
9. म॒रु॒तः॒ श्ये॒नꣳ श्ये॒नम् म॑रुतो मरुतः श्ये॒न मा॒यिन॑ मा॒यिनꣳ॑ श्ये॒नम् म॑रुतो मरुतः श्ये॒न मा॒यिन᳚म् । \newline
10. श्ये॒न मा॒यिन॑ मा॒यिनꣳ॑ श्ये॒नꣳ श्ये॒न मा॒यिन॒म् मनो॑जवस॒म् मनो॑जवस मा॒यिनꣳ॑ श्ये॒नꣳ श्ये॒न मा॒यिन॒म् मनो॑जवसम् । \newline
11. आ॒यिन॒म् मनो॑जवस॒म् मनो॑जवस मा॒यिन॑ मा॒यिन॒म् मनो॑जवसं॒ ॅवृष॑णं॒ ॅवृष॑ण॒म् मनो॑जवस मा॒यिन॑ मा॒यिन॒म् मनो॑जवसं॒ ॅवृष॑णम् । \newline
12. मनो॑जवसं॒ ॅवृष॑णं॒ ॅवृष॑ण॒म् मनो॑जवस॒म् मनो॑जवसं॒ ॅवृष॑णꣳ सुवृ॒क्तिꣳ सु॑वृ॒क्तिं ॅवृष॑ण॒म् मनो॑जवस॒म् मनो॑जवसं॒ ॅवृष॑णꣳ सुवृ॒क्तिम् । \newline
13. मनो॑जवस॒मिति॒ मनः॑ - ज॒व॒स॒म् । \newline
14. वृष॑णꣳ सुवृ॒क्तिꣳ सु॑वृ॒क्तिं ॅवृष॑णं॒ ॅवृष॑णꣳ सुवृ॒क्तिम् । \newline
15. सु॒वृ॒क्तिमिति॑ सु - वृ॒क्तिम् । \newline
16. येन॒ शर्द्धः॒ शर्द्धो॒ येन॒ येन॒ शर्द्ध॑ उ॒ग्र मु॒ग्रꣳ शर्द्धो॒ येन॒ येन॒ शर्द्ध॑ उ॒ग्रम् । \newline
17. शर्द्ध॑ उ॒ग्र मु॒ग्रꣳ शर्द्धः॒ शर्द्ध॑ उ॒ग्र मव॑सृष्ट॒ मव॑सृष्ट मु॒ग्रꣳ शर्द्धः॒ शर्द्ध॑ उ॒ग्र मव॑सृष्टम् । \newline
18. उ॒ग्र मव॑सृष्ट॒ मव॑सृष्ट मु॒ग्र मु॒ग्र मव॑सृष्ट॒ मेत्ये त्यव॑सृष्ट मु॒ग्र मु॒ग्र मव॑सृष्ट॒ मेति॑ । \newline
19. अव॑सृष्ट॒ मेत्ये त्यव॑सृष्ट॒ मव॑सृष्ट॒ मेति॒ तत् तदेत्यव॑सृष्ट॒ मव॑सृष्ट॒ मेति॒ तत् । \newline
20. अव॑सृष्ट॒मित्यव॑ - सृ॒ष्ट॒म् । \newline
21. एति॒ तत् तदेत्येति॒ तद॑श्विना ऽश्विना॒ तदेत्येति॒ तद॑श्विना । \newline
22. तद॑श्विना ऽश्विना॒ तत् तद॑श्विना॒ परि॒ पर्य॑श्विना॒ तत् तद॑श्विना॒ परि॑ । \newline
23. अ॒श्वि॒ना॒ परि॒ पर्य॑श्विना ऽश्विना॒ परि॑ धत्तम् धत्त॒म् पर्य॑श्विना ऽश्विना॒ परि॑ धत्तम् । \newline
24. परि॑ धत्तम् धत्त॒म् परि॒ परि॑ धत्तꣳ स्व॒स्ति स्व॒स्ति ध॑त्त॒म् परि॒ परि॑ धत्तꣳ स्व॒स्ति । \newline
25. ध॒त्तꣳ॒॒ स्व॒स्ति स्व॒स्ति ध॑त्तम् धत्तꣳ स्व॒स्ति । \newline
26. स्व॒स्तीति॑ स्व॒स्ति । \newline
27. पु॒रो॒वा॒तो वर्.ष॒न्॒. वर्.ष॑न् पुरोवा॒तः पु॑रोवा॒तो वर्.ष॑न् जि॒न्वो जि॒न्वो वर्.ष॑न् पुरोवा॒तः पु॑रोवा॒तो वर्.ष॑न् जि॒न्वः । \newline
28. पु॒रो॒वा॒त इति॑ पुरः - वा॒तः । \newline
29. वर्.ष॑न् जि॒न्वो जि॒न्वो वर्.ष॒न्॒. वर्.ष॑न् जि॒न्व रा॒वृ दा॒वृज् जि॒न्वो वर्.ष॒न्॒. वर्.ष॑न् जि॒न्व रा॒वृत् । \newline
30. जि॒न्व रा॒वृ दा॒वृज् जि॒न्वो जि॒न्व रा॒वृथ् स्वाहा॒ स्वाहा॒ ऽऽवृज् जि॒न्वो जि॒न्व रा॒वृथ् स्वाहा᳚ । \newline
31. आ॒वृथ् स्वाहा॒ स्वाहा॒ ऽऽवृदा॒वृथ् स्वाहा॑ वा॒ताव॑द् वा॒ताव॒थ् स्वाहा॒ ऽऽवृदा॒वृथ् स्वाहा॑ वा॒ताव॑त् । \newline
32. आ॒वृदित्या᳚ - वृत् । \newline
33. स्वाहा॑ वा॒ताव॑द् वा॒ताव॒थ् स्वाहा॒ स्वाहा॑ वा॒ताव॒द् वर्.ष॒न्॒. वर्.ष॑न्. वा॒ताव॒थ् स्वाहा॒ स्वाहा॑ वा॒ताव॒द् वर्.षन्न्॑ । \newline
34. वा॒ताव॒द् वर्.ष॒न्॒. वर्.ष॑न्. वा॒ताव॑द् वा॒ताव॒द् वर्.ष॑न् नु॒ग्र उ॒ग्रो वर्.ष॑न्. वा॒ताव॑द् वा॒ताव॒द् 
वर्.ष॑न् नु॒ग्रः । \newline
35. वा॒ताव॒दिति॑ वा॒त - व॒त् । \newline
36. वर्.ष॑न् नु॒ग्र उ॒ग्रो वर्.ष॒न्॒. वर्.ष॑न् नु॒ग्र रा॒वृ दा॒वृ दु॒ग्रो वर्.ष॒न्॒. वर्.ष॑न् नु॒ग्र रा॒वृत् । \newline
37. उ॒ग्र रा॒वृ दा॒वृ दु॒ग्र उ॒ग्र रा॒वृथ् स्वाहा॒ स्वाहा॒ ऽऽवृदु॒ग्र उ॒ग्र रा॒वृथ् स्वाहा᳚ । \newline
38. आ॒वृथ् स्वाहा॒ स्वाहा॒ ऽऽवृदा॒वृथ् स्वाहा᳚ स्त॒नयन्᳚ थ्स्त॒नय॒न् थ्स्वाहा॒ ऽऽवृदा॒वृथ् स्वाहा᳚ स्त॒नयन्न्॑ । \newline
39. आ॒वृदित्या᳚ - वृत् । \newline
40. स्वाहा᳚ स्त॒नयन्᳚ थ्स्त॒नय॒न् थ्स्वाहा॒ स्वाहा᳚ स्त॒नय॒न्॒. वर्.ष॒न्॒. वर्.षन्᳚ थ्स्त॒नय॒न् थ्स्वाहा॒ स्वाहा᳚ स्त॒नय॒न्॒. वर्.षन्न्॑ । \newline
41. स्त॒नय॒न्॒. वर्.ष॒न्॒. वर्.षन्᳚ थ्स्त॒नयन्᳚ थ्स्त॒नय॒न्॒. वर्.ष॑न् भी॒मो भी॒मो वर्.षन्᳚ थ्स्त॒नयन्᳚ थ्स्त॒नय॒न्॒. वर्.ष॑न् भी॒मः । \newline
42. वर्.ष॑न् भी॒मो भी॒मो वर्.ष॒न्॒. वर्.ष॑न् भी॒म रा॒वृ दा॒वृद् भी॒मो वर्.ष॒न्॒. वर्.ष॑न् भी॒म रा॒वृत् । \newline
43. भी॒म रा॒वृ दा॒वृद् भी॒मो भी॒म रा॒वृथ् स्वाहा॒ स्वाहा॒ ऽऽवृद् भी॒मो भी॒म रा॒वृथ् स्वाहा᳚ । \newline
44. आ॒वृथ् स्वाहा॒ स्वाहा॒ ऽऽवृदा॒वृथ् स्वाहा॑ ऽनश॒ न्य॑नश॒नि स्वाहा॒ ऽऽवृदा॒वृथ् स्वाहा॑ ऽनश॒नि । \newline
45. आ॒वृदित्या᳚ - वृत् । \newline
46. स्वाहा॑ ऽनश॒ न्य॑नश॒नि स्वाहा॒ स्वाहा॑ ऽनश॒ न्य॑व॒स्फूर्ज॑न् नव॒स्फूर्ज॑न् ननश॒नि स्वाहा॒ स्वाहा॑ ऽनश॒ न्य॑व॒स्फूर्जन्न्॑ । \newline
47. अ॒न॒श॒ न्य॑व॒स्फूर्ज॑न् नव॒स्फूर्ज॑न् ननश॒ न्य॑नश॒ न्य॑व॒स्फूर्ज॑न् दि॒द्युद् दि॒द्यु द॑व॒स्फूर्ज॑न् ननश॒ न्य॑नश॒ न्य॑व॒स्फूर्ज॑न् दि॒द्युत् । \newline
48. अ॒व॒स्फूर्ज॑न् दि॒द्युद् दि॒द्यु द॑व॒स्फूर्ज॑न् नव॒स्फूर्ज॑न् दि॒द्युद् वर्.ष॒न्॒. वर्.ष॑न् दि॒द्यु द॑व॒स्फूर्ज॑न् नव॒स्फूर्ज॑न् दि॒द्युद् वर्.षन्न्॑ । \newline
49. अ॒व॒स्फूर्ज॒न्नित्य॑व - स्फूर्जन्न्॑ । \newline
50. दि॒द्युद् वर्.ष॒न्॒. वर्.ष॑न् दि॒द्युद् दि॒द्युद् वर्.ष॑न् त्वे॒ष स्त्वे॒षो वर्.ष॑न् दि॒द्युद् दि॒द्युद् वर्.ष॑न् त्वे॒षः । \newline
51. वर्.ष॑न् त्वे॒ष स्त्वे॒षो वर्.ष॒न्॒. वर्.ष॑न् त्वे॒ष रा॒वृ दा॒वृत् त्वे॒षो वर्.ष॒न्॒. वर्.ष॑न् त्वे॒ष रा॒वृत् । \newline
52. त्वे॒ष रा॒वृ दा॒वृत् त्वे॒ष स्त्वे॒ष रा॒वृथ् स्वाहा॒ स्वाहा॒ ऽऽवृत् त्वे॒ष स्त्वे॒ष रा॒वृथ् स्वाहा᳚ । \newline
53. आ॒वृथ् स्वाहा॒ स्वाहा॒ ऽऽवृदा॒वृथ् स्वाहा॑ ऽतिरा॒त्र म॑तिरा॒त्रꣳ स्वाहा॒ ऽऽवृदा॒वृथ् स्वाहा॑ ऽतिरा॒त्रम् । \newline
54. आ॒वृदित्या᳚ - वृत् । \newline
55. स्वाहा॑ ऽतिरा॒त्र म॑तिरा॒त्रꣳ स्वाहा॒ स्वाहा॑ ऽतिरा॒त्रं ॅवर्.ष॒न्॒. वर्.ष॑न् नतिरा॒त्रꣳ स्वाहा॒ स्वाहा॑ ऽतिरा॒त्रं ॅवर्.षन्न्॑ । \newline
56. अ॒ति॒रा॒त्रं ॅवर्.ष॒न्॒. वर्.ष॑न् नतिरा॒त्र म॑तिरा॒त्रं ॅवर्.ष॑न् पू॒र्तिः पू॒र्तिर् वर्.ष॑न् नतिरा॒त्र म॑तिरा॒त्रं ॅवर्.ष॑न् पू॒र्तिः । \newline
57. अ॒ति॒रा॒त्रमित्य॑ति - रा॒त्रम् । \newline
58. वर्.ष॑न् पू॒र्तिः पू॒र्तिर् वर्.ष॒न्॒. वर्.ष॑न् पू॒र्ति रा॒वृ दा॒वृत् पू॒र्तिर् वर्.ष॒न्॒. वर्.ष॑न् पू॒र्ति रा॒वृत् । \newline
59. पू॒र्ति रा॒वृ दा॒वृत् पू॒र्तिः पू॒र्ति रा॒वृथ् स्वाहा॒ स्वाहा॒ ऽऽवृत् पू॒र्तिः पू॒र्ति रा॒वृथ् स्वाहा᳚ । \newline
60. आ॒वृथ् स्वाहा॒ स्वाहा॒ ऽऽवृदा॒वृथ् स्वाहा॑ ब॒हु ब॒हु स्वाहा॒ ऽऽवृदा॒वृथ् स्वाहा॑ ब॒हु । \newline
61. आ॒वृदित्या᳚ - वृत् । \newline
\pagebreak
\markright{ TS 2.4.7.2  \hfill https://www.vedavms.in \hfill}

\section{ TS 2.4.7.2 }

\textbf{TS 2.4.7.2 } \newline
\textbf{Samhita Paata} \newline

स्वाहा॑ ब॒हु हा॒यम॑वृषा॒दिति॑ श्रु॒तरा॒वृथ् स्वाहा॒ ऽऽतप॑ति॒ वर्.ष॑न्-वि॒राडा॒वृथ् स्वाहा॑ ऽव॒स्फूर्ज॑न्-दि॒द्युद्-वर्.ष॑न् भू॒तरा॒वृथ् स्वाहा॒मान्दा॒ वाशाः॒ शुन्ध्यू॒रजि॑राः । ज्योति॑ष्मती॒-स्तम॑स्वरी॒-रुन्द॑तीः॒ सुफे॑नाः । मित्र॑भृतः॒ क्षत्र॑भृतः॒ सुरा᳚ष्ट्रा इ॒ह मा॑ऽवत ॥वृष्णो॒ अश्व॑स्य स॒न्दान॑मसि॒ वृष्‌ट्यै॒ त्वोप॑ नह्यामि ॥ \newline

\textbf{Pada Paata} \newline

स्वाहा᳚ । ब॒हु । ह॒ । अ॒यम् । अ॒वृ॒षा॒त् । इति॑ । श्रु॒तः । आ॒वृदित्या᳚ - वृत् । स्वाहा᳚ । आ॒तप॒तीत्या᳚ - तप॑ति । वर्.षन्न्॑ । वि॒राडिति॑ वि - राट् । आ॒वृदित्या᳚ - वृत् । स्वाहा᳚ । अ॒व॒स्फूर्ज॒न्नित्य॑व - स्फूर्जन्न्॑ । दि॒द्युद् । वर्.षन्न्॑ । भू॒तः । आ॒वृदित्या᳚ - वृत् । स्वाहा᳚ । मान्दाः᳚ । वाशाः᳚ । शुन्ध्यूः᳚ । अजि॑राः ॥ ज्योति॑ष्मतीः । तम॑स्वरीः । उन्द॑तीः । सुफे॑ना॒ इति॒ सु - फे॒नाः॒ ॥ मित्र॑भृत॒ इति॒ मित्र॑ - भृ॒तः॒ । क्षत्र॑भृत॒ इति॒ क्षत्र॑ - भृ॒तः॒ । सुरा᳚ष्ट्रा॒ इति॒ सु - रा॒ष्ट्राः॒ । इ॒ह । मा॒ । अ॒व॒त॒ ॥ वृष्णः॑ । अश्व॑स्य । स॒दांन॒मिति॑ सं - दान᳚म् । अ॒सि॒ । वृष्ट्यै᳚ । त्वा॒ । उपेति॑ । न॒ह्या॒मि॒ ॥  \newline


\textbf{Krama Paata} \newline

स्वाहा॑ ब॒हु । ब॒हु ह॑ । हा॒यम् । अ॒यम॑वृषात् । अ॒वृ॒षा॒दिति॑ । इति॑ श्रु॒तः । श्रु॒तरा॒वृत् । आ॒वृथ् स्वाहा᳚ । आ॒वृदित्या᳚ - वृत् । स्वाहा॒ ऽऽतप॑ति । आ॒तप॑ति॒ वर्.षन्न्॑ । आ॒तप॒तीत्या᳚ - तप॑ति । वर्.ष॑न्. वि॒राट् । वि॒राडा॒वृत् । वि॒राडिति॑ वि - राट् । आ॒वृथ् स्वाहा᳚ । आ॒वृदित्या᳚ - वृत् । स्वाहा॑ ऽव॒स्फूर्जन्न्॑ । अ॒व॒स्फूर्ज॑न् दि॒द्युत् । अ॒व॒स्फूर्ज॒न्नित्य॑व - स्फूर्जन्न्॑ । दि॒द्युद् वर्.षन्न्॑ । वर्.ष॑न् भू॒तः । भू॒तरा॒वृत् । 
आ॒वृथ् स्वाहा᳚ । आ॒वृदित्या᳚ - वृत् । स्वाहा॒ मान्दाः᳚ । मान्दा॒ वाशाः᳚ । वाशाः॒ शुन्ध्यूः᳚ । शुन्ध्यू॒रजि॑राः । अजि॑रा॒ इत्यजि॑राः ॥ ज्योति॑ष्मती॒स्तम॑स्वरीः । तमस्व॑री॒रुन्द॑तीः । उन्द॑तीः॒ सुफे॑नाः । सुफे॑ना॒ इति॒ सु - फे॒नाः॒ ॥ मित्र॑भृतः॒ क्षत्र॑भृतः । मित्र॑भृत॒ इति॒ मित्र॑ - भृ॒तः॒ । क्षत्र॑भृतः॒ सुरा᳚ष्ट्राः । क्षत्र॑भृत॒ इति॒ क्षत्र॑ - भृ॒तः॒ । सुरा᳚ष्ट्रा इ॒ह । सुरा᳚ष्ट्रा॒ इति॒ सु - रा॒ष्ट्राः॒ । इ॒हमा᳚ । मा॒ऽव॒त॒ । अ॒व॒तेत्य॑वत ॥ वृष्णो॒ अश्व॑स्य । अश्व॑स्य स॒न्दान᳚म् । स॒न्दान॑मसि । स॒न्दान॒मिति॑ सं - दान᳚म् । अ॒सि॒ वृष्ट्यै᳚ । वृष्ट्यै᳚ त्वा । त्वोप॑ । उप॑ नह्यामि । 
न॒ह्या॒मीति॑ नह्यामि । \newline

\textbf{Jatai Paata} \newline

1. स्वाहा॑ ब॒हु ब॒हु स्वाहा॒ स्वाहा॑ ब॒हु । \newline
2. ब॒हु ह॑ ह ब॒हु ब॒हु ह॑ । \newline
3. हा॒य म॒यꣳ ह॑ हा॒यम् । \newline
4. अ॒य म॑वृषा दवृषा द॒य म॒य म॑वृषात् । \newline
5. अ॒वृ॒षा॒ दिती त्य॑वृषा दवृषा॒ दिति॑ । \newline
6. इति॑ श्रु॒तः श्रु॒त रितीति॑ श्रु॒तः । \newline
7. श्रु॒त रा॒वृ दा॒वृच् छ्रु॒तः श्रु॒त रा॒वृत् । \newline
8. आ॒वृथ् स्वाहा॒ स्वाहा॒ ऽऽवृ दा॒वृथ् स्वाहा᳚ । \newline
9. आ॒वृदित्या᳚ - वृत् । \newline
10. स्वाहा॒ ऽऽतप॑ त्या॒तप॑ति॒ स्वाहा॒ स्वाहा॒ ऽऽतप॑ति । \newline
11. आ॒तप॑ति॒ वर्.ष॒न्॒. वर्.ष॑न् ना॒तप॑ त्या॒तप॑ति॒ वर्.षन्न्॑ । \newline
12. आ॒तप॒तीत्या᳚ - तप॑ति । \newline
13. वर्.ष॑न् वि॒राड् वि॒राड् वर्.ष॒न् वर्.ष॑न् वि॒राट् । \newline
14. वि॒रा डा॒वृ दा॒वृद् वि॒राड् वि॒रा डा॒वृत् । \newline
15. वि॒राडिति॑ वि - राट् । \newline
16. आ॒वृथ् स्वाहा॒ स्वाहा॒ ऽऽवृ दा॒वृथ् स्वाहा᳚ । \newline
17. आ॒वृदित्या᳚ - वृत् । \newline
18. स्वाहा॑ ऽव॒स्फूर्ज॑न् नव॒स्फूर्ज॒न् थ्स्वाहा॒ स्वाहा॑ ऽव॒स्फूर्जन्न्॑ । \newline
19. अ॒व॒स्फूर्ज॑न् दि॒द्युद् दि॒द्यु द॑व॒स्फूर्ज॑न् नव॒स्फूर्ज॑न् दि॒द्युद् । \newline
20. अ॒व॒स्फूर्ज॒न्नित्य॑व - स्फूर्जन्न्॑ । \newline
21. दि॒द्युद् वर्.ष॒न्॒. वर्.ष॑न् दि॒द्युद् दि॒द्युद् वर्.षन्न्॑ । \newline
22. वर्.ष॑न् भू॒तो भू॒तो वर्.ष॒न्॒. वर्.ष॑न् भू॒तः । \newline
23. भू॒त रा॒वृ दा॒वृद् भू॒तो भू॒त रा॒वृत् । \newline
24. आ॒वृथ् स्वाहा॒ स्वाहा॒ ऽऽवृ दा॒वृथ् स्वाहा᳚ । \newline
25. आ॒वृदित्या᳚ - वृत् । \newline
26. स्वाहा॒ मान्दा॒ मान्दाः॒ स्वाहा॒ स्वाहा॒ मान्दाः᳚ । \newline
27. मान्दा॒ वाशा॒ वाशा॒ मान्दा॒ मान्दा॒ वाशाः᳚ । \newline
28. वाशाः॒ शुन्ध्यूः॒ शुन्ध्यू॒र् वाशा॒ वाशाः॒ शुन्ध्यूः᳚ । \newline
29. शुन्ध्यू॒ रजि॑रा॒ अजि॑राः॒ शुन्ध्यूः॒ शुन्ध्यू॒ रजि॑राः । \newline
30. अजि॑रा॒ इत्यजि॑राः । \newline
31. ज्योति॑ष्मती॒ स्तम॑स्वरी॒ स्तम॑स्वरी॒र् ज्योति॑ष्मती॒र् ज्योति॑ष्मती॒ स्तम॑स्वरीः । \newline
32. तम॑स्वरी॒ रुन्द॑ती॒ रुन्द॑ती॒ स्तम॑स्वरी॒ स्तम॑स्वरी॒ रुन्द॑तीः । \newline
33. उन्द॑तीः॒ सुफे॑नाः॒ सुफे॑ना॒ उन्द॑ती॒ रुन्द॑तीः॒ सुफे॑नाः । \newline
34. सुफे॑ना॒ इति॒ सु - फे॒नाः॒ । \newline
35. मित्र॑भृतः॒ क्षत्र॑भृतः॒ क्षत्र॑भृतो॒ मित्र॑भृतो॒ मित्र॑भृतः॒ क्षत्र॑भृतः । \newline
36. मित्र॑भृत॒ इति॒ मित्र॑ - भृ॒तः॒ । \newline
37. क्षत्र॑भृतः॒ सुरा᳚ष्ट्राः॒ सुरा᳚ष्ट्राः॒ क्षत्र॑भृतः॒ क्षत्र॑भृतः॒ सुरा᳚ष्ट्राः । \newline
38. क्षत्र॑भृत॒ इति॒ क्षत्र॑ - भृ॒तः॒ । \newline
39. सुरा᳚ष्ट्रा इ॒हे ह सुरा᳚ष्ट्राः॒ सुरा᳚ष्ट्रा इ॒ह । \newline
40. सुरा᳚ष्ट्रा॒ इति॒ सु - रा॒ष्ट्राः॒ । \newline
41. इ॒ह मा॑ मे॒हे ह मा᳚ । \newline
42. मा॒ ऽव॒ता॒व॒त॒ मा॒ मा॒ ऽव॒त॒ । \newline
43. अ॒व॒तेत्य॑वत । \newline
44. वृष्णो॒ अश्व॒स्या श्व॑स्य॒ वृष्णो॒ वृष्णो॒ अश्व॑स्य । \newline
45. अश्व॑स्य स॒न्दानꣳ॑ स॒न्दान॒ मश्व॒स्या श्व॑स्य स॒न्दान᳚म् । \newline
46. स॒न्दान॑ मस्यसि स॒न्दानꣳ॑ स॒न्दान॑ मसि । \newline
47. स॒न्दान॒मिति॑ सं - दान᳚म् । \newline
48. अ॒सि॒ वृष्ट्यै॒ वृष्ट्या॑ अस्यसि॒ वृष्ट्यै᳚ । \newline
49. वृष्ट्यै᳚ त्वा त्वा॒ वृष्ट्यै॒ वृष्ट्यै᳚ त्वा । \newline
50. त्वोपोप॑ त्वा॒ त्वोप॑ । \newline
51. उप॑ नह्यामि नह्या॒ म्युपोप॑ नह्यामि । \newline
52. न॒ह्या॒मीति॑ नह्यामि । \newline

\textbf{Ghana Paata } \newline

1. स्वाहा॑ ब॒हु ब॒हु स्वाहा॒ स्वाहा॑ ब॒हु ह॑ ह ब॒हु स्वाहा॒ स्वाहा॑ ब॒हु ह॑ । \newline
2. ब॒हु ह॑ ह ब॒हु ब॒हु हा॒य म॒यꣳ ह॑ ब॒हु ब॒हु हा॒यम् । \newline
3. हा॒य म॒यꣳ ह॑ हा॒य म॑वृषा दवृषा द॒यꣳ ह॑ हा॒य म॑वृषात् । \newline
4. अ॒य म॑वृषा दवृषाद॒य म॒य म॑वृषा॒ दितीत्य॑वृषाद॒य म॒य म॑वृषा॒दिति॑ । \newline
5. अ॒वृ॒षा॒ दितीत्य॑वृषा दवृषा॒ दिति॑ श्रु॒तः श्रु॒त रित्य॑वृषा दवृषा॒ दिति॑ श्रु॒तः । \newline
6. इति॑ श्रु॒तः श्रु॒त रितीति॑ श्रु॒त रा॒वृ दा॒वृच् छ्रु॒त रितीति॑ श्रु॒त रा॒वृत् । \newline
7. श्रु॒त रा॒वृ दा॒वृच् छ्रु॒तः श्रु॒त रा॒वृथ् स्वाहा॒ स्वाहा॒ ऽऽवृच्छ्रु॒तः श्रु॒त रा॒वृथ् स्वाहा᳚ । \newline
8. आ॒वृथ् स्वाहा॒ स्वाहा॒ ऽऽवृदा॒वृथ् स्वाहा॒ ऽऽतप॑ त्या॒तप॑ति॒ स्वाहा॒ ऽऽवृदा॒वृथ् स्वाहा॒ ऽऽतप॑ति । \newline
9. आ॒वृदित्या᳚ - वृत् । \newline
10. स्वाहा॒ ऽऽतप॑ त्या॒तप॑ति॒ स्वाहा॒ स्वाहा॒ ऽऽतप॑ति॒ वर्.ष॒न्॒. वर्.ष॑न् ना॒तप॑ति॒ स्वाहा॒ स्वाहा॒ ऽऽतप॑ति॒ वर्.षन्न्॑ । \newline
11. आ॒तप॑ति॒ वर्.ष॒न्॒. वर्.ष॑न् ना॒तप॑ त्या॒तप॑ति॒ वर्.ष॑न् वि॒राड् वि॒राड् वर्.ष॑न् ना॒तप॑ त्या॒तप॑ति॒ वर्.ष॑न् वि॒राट् । \newline
12. आ॒तप॒तीत्या᳚ - तप॑ति । \newline
13. वर्.ष॑न् वि॒राड् वि॒राड् वर्.ष॒न् वर्.ष॑न् वि॒रा डा॒वृ दा॒वृद् वि॒राड् वर्.ष॒न् वर्.ष॑न् वि॒रा डा॒वृत् । \newline
14. वि॒रा डा॒वृ दा॒वृद् वि॒राड् वि॒रा डा॒वृथ् स्वाहा॒ स्वाहा॒ ऽऽवृद् वि॒राड् वि॒रा डा॒वृथ् स्वाहा᳚ । \newline
15. वि॒राडिति॑ वि - राट् । \newline
16. आ॒वृथ् स्वाहा॒ स्वाहा॒ ऽऽवृदा॒वृथ् स्वाहा॑ ऽव॒स्फूर्ज॑न् नव॒स्फूर्ज॒न् थ्स्वाहा॒ ऽऽवृदा॒वृथ् स्वाहा॑ ऽव॒स्फूर्जन्न्॑ । \newline
17. आ॒वृदित्या᳚ - वृत् । \newline
18. स्वाहा॑ ऽव॒स्फूर्ज॑न् नव॒स्फूर्ज॒न् थ्स्वाहा॒ स्वाहा॑ ऽव॒स्फूर्ज॑न् दि॒द्युद् दि॒द्यु द॑व॒स्फूर्ज॒न् थ्स्वाहा॒ स्वाहा॑ ऽव॒स्फूर्ज॑न् दि॒द्युद् । \newline
19. अ॒व॒स्फूर्ज॑न् दि॒द्युद् दि॒द्यु द॑व॒स्फूर्ज॑न् नव॒स्फूर्ज॑न् दि॒द्युद् वर्.ष॒न्॒. वर्.ष॑न् दि॒द्यु द॑व॒स्फूर्ज॑न् नव॒स्फूर्ज॑न् दि॒द्युद् वर्.षन्न्॑ । \newline
20. अ॒व॒स्फूर्ज॒न्नित्य॑व - स्फूर्जन्न्॑ । \newline
21. दि॒द्युद् वर्.ष॒न्॒. वर्.ष॑न् दि॒द्युद् दि॒द्युद् वर्.ष॑न् भू॒तो भू॒तो वर्.ष॑न् दि॒द्युद् दि॒द्युद् वर्.ष॑न् भू॒तः । \newline
22. वर्.ष॑न् भू॒तो भू॒तो वर्.ष॒न्॒. वर्.ष॑न् भू॒त रा॒वृ दा॒वृद् भू॒तो वर्.ष॒न्॒. वर्.ष॑न् भू॒त रा॒वृत् । \newline
23. भू॒त रा॒वृ दा॒वृद् भू॒तो भू॒त रा॒वृथ् स्वाहा॒ स्वाहा॒ ऽऽवृद् भू॒तो भू॒त रा॒वृथ् स्वाहा᳚ । \newline
24. आ॒वृथ् स्वाहा॒ स्वाहा॒ ऽऽवृदा॒वृथ् स्वाहा॒ मान्दा॒ मान्दाः॒ स्वाहा॒ ऽऽवृदा॒वृथ् स्वाहा॒ मान्दाः᳚ । \newline
25. आ॒वृदित्या᳚ - वृत् । \newline
26. स्वाहा॒ मान्दा॒ मान्दाः॒ स्वाहा॒ स्वाहा॒ मान्दा॒ वाशा॒ वाशा॒ मान्दाः॒ स्वाहा॒ स्वाहा॒ मान्दा॒ वाशाः᳚ । \newline
27. मान्दा॒ वाशा॒ वाशा॒ मान्दा॒ मान्दा॒ वाशाः॒ शुन्ध्यूः॒ शुन्ध्यू॒र् वाशा॒ मान्दा॒ मान्दा॒ वाशाः॒ शुन्ध्यूः᳚ । \newline
28. वाशाः॒ शुन्ध्यूः॒ शुन्ध्यू॒र् वाशा॒ वाशाः॒ शुन्ध्यू॒ रजि॑रा॒ अजि॑राः॒ शुन्ध्यू॒र् वाशा॒ वाशाः॒ शुन्ध्यू॒ रजि॑राः । \newline
29. शुन्ध्यू॒ रजि॑रा॒ अजि॑राः॒ शुन्ध्यूः॒ शुन्ध्यू॒ रजि॑राः । \newline
30. अजि॑रा॒ इत्यजि॑राः । \newline
31. ज्योति॑ष्मती॒ स्तम॑स्वरी॒ स्तम॑स्वरी॒र् ज्योति॑ष्मती॒र् ज्योति॑ष्मती॒ स्तम॑स्वरी॒ रुन्द॑ती॒ रुन्द॑ती॒ स्तम॑स्वरी॒र् ज्योति॑ष्मती॒र् ज्योति॑ष्मती॒ स्तम॑स्वरी॒ रुन्द॑तीः । \newline
32. तम॑स्वरी॒ रुन्द॑ती॒ रुन्द॑ती॒ स्तम॑स्वरी॒ स्तम॑स्वरी॒ रुन्द॑तीः॒ सुफे॑नाः॒ सुफे॑ना॒ उन्द॑ती॒ स्तम॑स्वरी॒ स्तम॑स्वरी॒ रुन्द॑तीः॒ सुफे॑नाः । \newline
33. उन्द॑तीः॒ सुफे॑नाः॒ सुफे॑ना॒ उन्द॑ती॒ रुन्द॑तीः॒ सुफे॑नाः । \newline
34. सुफे॑ना॒ इति॒ सु - फे॒नाः॒ । \newline
35. मित्र॑भृतः॒ क्षत्र॑भृतः॒ क्षत्र॑भृतो॒ मित्र॑भृतो॒ मित्र॑भृतः॒ क्षत्र॑भृतः॒ सुरा᳚ष्ट्राः॒ सुरा᳚ष्ट्राः॒ क्षत्र॑भृतो॒ मित्र॑भृतो॒ मित्र॑भृतः॒ क्षत्र॑भृतः॒ सुरा᳚ष्ट्राः । \newline
36. मित्र॑भृत॒ इति॒ मित्र॑ - भृ॒तः॒ । \newline
37. क्षत्र॑भृतः॒ सुरा᳚ष्ट्राः॒ सुरा᳚ष्ट्राः॒ क्षत्र॑भृतः॒ क्षत्र॑भृतः॒ सुरा᳚ष्ट्रा इ॒हे ह सुरा᳚ष्ट्राः॒ क्षत्र॑भृतः॒ क्षत्र॑भृतः॒ सुरा᳚ष्ट्रा इ॒ह । \newline
38. क्षत्र॑भृत॒ इति॒ क्षत्र॑ - भृ॒तः॒ । \newline
39. सुरा᳚ष्ट्रा इ॒हे ह सुरा᳚ष्ट्राः॒ सुरा᳚ष्ट्रा इ॒ह मा॑ मे॒ह सुरा᳚ष्ट्राः॒ सुरा᳚ष्ट्रा इ॒ह मा᳚ । \newline
40. सुरा᳚ष्ट्रा॒ इति॒ सु - रा॒ष्ट्राः॒ । \newline
41. इ॒ह मा॑ मे॒हे ह मा॑ ऽवतावत मे॒हे ह मा॑ ऽवत । \newline
42. मा॒ ऽव॒ता॒व॒त॒ मा॒ मा॒ ऽव॒त॒ । \newline
43. अ॒व॒तेत्य॑वत । \newline
44. वृष्णो॒ अश्व॒स्याश्व॑स्य॒ वृष्णो॒ वृष्णो॒ अश्व॑स्य स॒न्दानꣳ॑ स॒न्दान॒ मश्व॑स्य॒ वृष्णो॒ वृष्णो॒ अश्व॑स्य स॒न्दान᳚म् । \newline
45. अश्व॑स्य स॒न्दानꣳ॑ स॒न्दान॒ मश्व॒स्याश्व॑स्य स॒न्दान॑ मस्यसि स॒न्दान॒ मश्व॒स्याश्व॑स्य स॒न्दान॑ मसि । \newline
46. स॒न्दान॑ मस्यसि स॒न्दानꣳ॑ स॒न्दान॑ मसि॒ वृष्ट्यै॒ वृष्ट्या॑ असि स॒न्दानꣳ॑ स॒न्दान॑ मसि॒ वृष्ट्यै᳚ । \newline
47. स॒न्दान॒मिति॑ सं - दान᳚म् । \newline
48. अ॒सि॒ वृष्ट्यै॒ वृष्ट्या॑ अस्यसि॒ वृष्ट्यै᳚ त्वा त्वा॒ वृष्ट्या॑ अस्यसि॒ वृष्ट्यै᳚ त्वा । \newline
49. वृष्ट्यै᳚ त्वा त्वा॒ वृष्ट्यै॒ वृष्ट्यै॒ त्वोपोप॑ त्वा॒ वृष्ट्यै॒ वृष्ट्यै॒ त्वोप॑ । \newline
50. त्वोपोप॑ त्वा॒ त्वोप॑ नह्यामि नह्या॒म्युप॑ त्वा॒ त्वोप॑ नह्यामि । \newline
51. उप॑ नह्यामि नह्या॒ म्युपोप॑ नह्यामि । \newline
52. न॒ह्या॒मीति॑ नह्यामि । \newline
\pagebreak
\markright{ TS 2.4.8.1  \hfill https://www.vedavms.in \hfill}

\section{ TS 2.4.8.1 }

\textbf{TS 2.4.8.1 } \newline
\textbf{Samhita Paata} \newline

देवा॑ वसव्या॒ अग्ने॑ सोम सूर्य ॥ देवाः᳚ शर्मण्या॒ मित्रा॑वरुणाऽर्यमन्न् ॥ देवाः᳚ सपीत॒यो ऽपा᳚न्-नपादाशुहेमन्न् । उ॒द्नो द॑त्तोद॒धिं भि॑न्त्त दि॒वः प॒र्जन्या॑द॒न्तरि॑क्षात्-पृथि॒व्यास्ततो॑ नो॒ वृष्‌ट्या॑ऽवत ॥ दिवा॑ चि॒त्तमः॑ कृण्वन्ति प॒र्जन्ये॑नो-दवा॒हेन॑ । पृ॒थि॒वीं ॅयद् व्यु॒न्दन्ति॑ ॥आयन् नरः॑ सु॒दान॑वो ददा॒शुषे॑ दि॒वः कोश॒मचु॑च्यवुः । वि प॒र्जन्याः᳚ सृजन्ति॒ रोद॑सी॒ अनु॒ धन्व॑ना यन्ति - [  ] \newline

\textbf{Pada Paata} \newline

देवाः᳚ । व॒स॒व्याः॒ । अग्ने᳚ । सो॒म॒ । सू॒र्य॒ ॥ देवाः᳚ । श॒र्म॒ण्याः॒ । मित्रा॑वरु॒णेति॒ मित्रा᳚ - व॒रु॒णा॒ । अ॒र्य॒म॒न्न् ॥ देवाः᳚ । स॒पी॒त॒य॒ इति॑ स - पी॒त॒यः॒ । अपा᳚म् । न॒पा॒त् । आ॒शु॒हे॒म॒न्नित्या॑शु - हे॒म॒न्न् ॥ उ॒द्नः । द॒त्त । उ॒द॒धिमित्यु॑द - धिम् । भि॒न्त॒ । दि॒वः ।   प॒र्जन्या᳚त् । अ॒न्तरि॑क्षात् । पृ॒थि॒व्याः । ततः॑ । नः॒ । वृष्ट्या᳚ । अ॒व॒त॒ ॥ दिवा᳚ । चि॒त् । तमः॑ । कृ॒ण्व॒न्ति॒ । प॒र्जन्ये॑न । उ॒द॒वा॒हेनेत्यु॑द - वा॒हेन॑ ॥ पृ॒थि॒वीम् । यत् । व्यु॒न्दन्तीति॑ वि-उ॒न्दन्ति॑ ॥ एति॑ । यम् । नरः॑ । सु॒दान॑व॒ इति॑ सु-दान॑वः । द॒दा॒शुषे᳚ । दि॒वः । कोश᳚म् । अचु॑च्यवुः ॥ वीति॑ । प॒र्जन्याः᳚ । सृ॒ज॒न्ति॒ । रोद॑सी॒ इति॑ । अन्विति॑ । धन्व॑ना । य॒न्ति॒ ।  \newline


\textbf{Krama Paata} \newline

देवा॑ वसव्याः । व॒स॒व्या॒ अग्ने᳚ । अग्ने॑ सोम । सो॒म॒ सू॒र्य॒ । सू॒र्येति॑ सूर्य ॥ देवाः᳚ शर्मण्याः । श॒र्म॒ण्या॒ मित्रा॑वरुणा । मित्रा॑वरुणा ऽर्यमन्न् । मित्रा॑वरु॒णेति॒ मित्रा᳚ - व॒रु॒णा॒ । अ॒र्य॒म॒नित्य॑र्यमन्न् ॥ देवाः᳚ सपीतयः । स॒पी॒त॒योऽपा᳚म् । स॒पी॒त॒य॒ इति॑ स - पी॒त॒यः॒ । अपा᳚म् नपात् । न॒पा॒दा॒शु॒हे॒म॒न्न्॒ । आ॒शु॒हे॒म॒न्नित्या॑शु - हे॒म॒न्न्॒ ॥ उ॒द्नो द॑त्त । द॒त्तो॒द॒धिम् । 
उ॒द॒धिम् भि॑न्त । उ॒द॒धि मित्यु॑द - धिम् । भि॒न्त॒ दि॒वः । दि॒वः प॒र्जन्या᳚त् । प॒र्जन्या॑द॒न्तरि॑क्षात् । अ॒न्तरि॑क्षात् पृथि॒व्याः । पृ॒थि॒व्यास्ततः॑ । ततो॑ नः । नो॒ वृष्ट्या᳚ । वृष्ट्या॑ ऽवत । अ॒व॒तेत्य॑वत ॥ दिवा॑ चित् । चि॒त् तमः॑ । तमः॑ कृण्वन्ति । कृ॒ण्व॒न्ति॒ प॒र्जन्ये॑न । प॒र्जन्ये॑नोदवा॒हेन॑ । उ॒द॒वा॒हेनेत्यु॑द - वा॒हेन॑ ॥ पृ॒थि॒वीं ॅयत् । यद् व्यु॒न्दन्ति॑ । व्यु॒न्दन्तीति॑ वि - उ॒न्दन्ति॑ ॥ आ यम् । यम् नरः॑ । नरः॑ सु॒दान॑वः । सु॒दान॑वो ददा॒शुषे᳚ । सु॒दान॑व॒ इति॑ सु - दान॑वः । द॒दा॒शुषे॑ दि॒वः । दि॒वः कोश᳚म् । कोश॒मचु॑च्यवुः । अचु॑च्यवु॒,रित्यचु॑च्यवुः ॥ वि प॒र्जन्याः᳚ । प॒र्जन्याः᳚ सृजन्ति । सृ॒ज॒न्ति॒ रोद॑सी । रोद॑सी॒ अनु॑ । रोद॑सी॒ इति॒ रोद॑सी । अनु॒ धन्व॑ना । धन्व॑ना यन्ति । य॒न्ति॒ वृ॒ष्टयः॑ \newline

\textbf{Jatai Paata} \newline

1. देवा॑ वसव्या वसव्या॒ देवा॒ देवा॑ वसव्याः । \newline
2. व॒स॒व्या॒ अग्ने ऽग्ने॑ वसव्या वसव्या॒ अग्ने᳚ । \newline
3. अग्ने॑ सोम सो॒माग्ने ऽग्ने॑ सोम । \newline
4. सो॒म॒ सू॒र्य॒ सू॒र्य॒ सो॒म॒ सो॒म॒ सू॒र्य॒ । \newline
5. सू॒र्येति॑ सूर्य । \newline
6. देवाः᳚ शर्मण्याः शर्मण्या॒ देवा॒ देवाः᳚ शर्मण्याः । \newline
7. श॒र्म॒ण्या॒ मित्रा॑वरुणा॒ मित्रा॑वरुणा शर्मण्याः शर्मण्या॒ मित्रा॑वरुणा । \newline
8. मित्रा॑वरुणा ऽर्यमन् नर्यम॒न् मित्रा॑वरुणा॒ मित्रा॑वरुणा ऽर्यमन्न् । \newline
9. मित्रा॑वरु॒णेति॒ मित्रा᳚ - व॒रु॒णा॒ । \newline
10. अ॒र्य॒म॒नित्य॑र्यमन्न् । \newline
11. देवाः᳚ सपीतयः सपीतयो॒ देवा॒ देवाः᳚ सपीतयः । \newline
12. स॒पी॒त॒यो ऽपा॒ मपाꣳ॑ सपीतयः सपीत॒यो ऽपा᳚म् । \newline
13. स॒पी॒त॒य॒ इति॑ स - पी॒त॒यः॒ । \newline
14. अपा᳚म् नपान् नपा॒दपा॒ मपा᳚म् नपात् । \newline
15. न॒पा॒ दा॒शु॒हे॒म॒न् ना॒शु॒हे॒म॒न् न॒पा॒न् न॒पा॒ दा॒शु॒हे॒म॒न्न् । \newline
16. आ॒शु॒हे॒म॒न्नित्या॑शु - हे॒म॒न्न् । \newline
17. उ॒द्नो द॒त्त द॒त्तोद्न उ॒द्नो द॒त्त । \newline
18. द॒त्तोद॒धि मु॑द॒धिम् द॒त्त द॒त्तोद॒धिम् । \newline
19. उ॒द॒धिम् भि॑न्त भिन्तोद॒धि मु॑द॒धिम् भि॑न्त । \newline
20. उ॒द॒धिमित्यु॑द - धिम् । \newline
21. भि॒न्त॒ दि॒वो दि॒वो भि॑न्त भिन्त दि॒वः । \newline
22. दि॒वः प॒र्जन्या᳚त् प॒र्जन्या᳚द् दि॒वो दि॒वः प॒र्जन्या᳚त् । \newline
23. प॒र्जन्या॑ द॒न्तरि॑क्षा द॒न्तरि॑क्षात् प॒र्जन्या᳚त् प॒र्जन्या॑ द॒न्तरि॑क्षात् । \newline
24. अ॒न्तरि॑क्षात् पृथि॒व्याः पृ॑थि॒व्या अ॒न्तरि॑क्षा द॒न्तरि॑क्षात् पृथि॒व्याः । \newline
25. पृ॒थि॒व्या स्तत॒ स्ततः॑ पृथि॒व्याः पृ॑थि॒व्या स्ततः॑ । \newline
26. ततो॑ नो न॒ स्तत॒ स्ततो॑ नः । \newline
27. नो॒ वृष्ट्या॒ वृष्ट्या॑ नो नो॒ वृष्ट्या᳚ । \newline
28. वृष्ट्या॑ ऽवता वत॒ वृष्ट्या॒ वृष्ट्या॑ ऽवत । \newline
29. अ॒व॒तेत्य॑वत । \newline
30. दिवा॑ चिच् चि॒द् दिवा॒ दिवा॑ चित् । \newline
31. चि॒त् तम॒ स्तम॑ श्चिच् चि॒त् तमः॑ । \newline
32. तमः॑ कृण्वन्ति कृण्वन्ति॒ तम॒ स्तमः॑ कृण्वन्ति । \newline
33. कृ॒ण्व॒न्ति॒ प॒र्जन्ये॑न प॒र्जन्ये॑न कृण्वन्ति कृण्वन्ति प॒र्जन्ये॑न । \newline
34. प॒र्जन्ये॑नो दवा॒हेनो॑ दवा॒हेन॑ प॒र्जन्ये॑न प॒र्जन्ये॑नो दवा॒हेन॑ । \newline
35. उ॒द॒वा॒हेनेत्यु॑द - वा॒हेन॑ । \newline
36. पृ॒थि॒वीं ॅयद् यत् पृ॑थि॒वीम् पृ॑थि॒वीं ॅयत् । \newline
37. यद् व्यु॒न्दन्ति॑ व्यु॒न्दन्ति॒ यद् यद् व्यु॒न्दन्ति॑ । \newline
38. व्यु॒न्दन्तीति॑ वि - उ॒न्दन्ति॑ । \newline
39. आ यं ॅय मा यम् । \newline
40. यम् नरो॒ नरो॒ यं ॅयम् नरः॑ । \newline
41. नरः॑ सु॒दान॑वः सु॒दान॑वो॒ नरो॒ नरः॑ सु॒दान॑वः । \newline
42. सु॒दान॑वो ददा॒शुषे॑ ददा॒शुषे॑ सु॒दान॑वः सु॒दान॑वो ददा॒शुषे᳚ । \newline
43. सु॒दान॑व॒ इति॑ सु - दान॑वः । \newline
44. द॒दा॒शुषे॑ दि॒वो दि॒वो द॑दा॒शुषे॑ ददा॒शुषे॑ दि॒वः । \newline
45. दि॒वः कोश॒म् कोश॑म् दि॒वो दि॒वः कोश᳚म् । \newline
46. कोश॒ मचु॑च्यवु॒ रचु॑च्यवुः॒ कोश॒म् कोश॒ मचु॑च्यवुः । \newline
47. अचु॑च्यवु॒रित्यचु॑च्यवुः । \newline
48. वि प॒र्जन्याः᳚ प॒र्जन्या॒ वि वि प॒र्जन्याः᳚ । \newline
49. प॒र्जन्याः᳚ सृजन्ति सृजन्ति प॒र्जन्याः᳚ प॒र्जन्याः᳚ सृजन्ति । \newline
50. सृ॒ज॒न्ति॒ रोद॑सी॒ रोद॑सी सृजन्ति सृजन्ति॒ रोद॑सी । \newline
51. रोद॑सी॒ अन्वनु॒ रोद॑सी॒ रोद॑सी॒ अनु॑ । \newline
52. रोद॑सी॒ इति॒ रोद॑सी । \newline
53. अनु॒ धन्व॑ना॒ धन्व॒ना ऽन्वनु॒ धन्व॑ना । \newline
54. धन्व॑ना यन्ति यन्ति॒ धन्व॑ना॒ धन्व॑ना यन्ति । \newline
55. य॒न्ति॒ वृ॒ष्टयो॑ वृ॒ष्टयो॑ यन्ति यन्ति वृ॒ष्टयः॑ । \newline

\textbf{Ghana Paata } \newline

1. देवा॑ वसव्या वसव्या॒ देवा॒ देवा॑ वसव्या॒ अग्ने ऽग्ने॑ वसव्या॒ देवा॒ देवा॑ वसव्या॒ अग्ने᳚ । \newline
2. व॒स॒व्या॒ अग्ने ऽग्ने॑ वसव्या वसव्या॒ अग्ने॑ सोम सो॒माग्ने॑ वसव्या वसव्या॒ अग्ने॑ सोम । \newline
3. अग्ने॑ सोम सो॒माग्ने ऽग्ने॑ सोम सूर्य सूर्य सो॒माग्ने ऽग्ने॑ सोम सूर्य । \newline
4. सो॒म॒ सू॒र्य॒ सू॒र्य॒ सो॒म॒ सो॒म॒ सू॒र्य॒ । \newline
5. सू॒र्येति॑ सूर्य । \newline
6. देवाः᳚ शर्मण्याः शर्मण्या॒ देवा॒ देवाः᳚ शर्मण्या॒ मित्रा॑वरुणा॒ मित्रा॑वरुणा शर्मण्या॒ देवा॒ देवाः᳚ शर्मण्या॒ मित्रा॑वरुणा । \newline
7. श॒र्म॒ण्या॒ मित्रा॑वरुणा॒ मित्रा॑वरुणा शर्मण्याः शर्मण्या॒ मित्रा॑वरुणा ऽर्यमन् नर्यम॒न् मित्रा॑वरुणा शर्मण्याः शर्मण्या॒ मित्रा॑वरुणा ऽर्यमन्न् । \newline
8. मित्रा॑वरुणा ऽर्यमन् नर्यम॒न् मित्रा॑वरुणा॒ मित्रा॑वरुणा ऽर्यमन्न् । \newline
9. मित्रा॑वरु॒णेति॒ मित्रा᳚ - व॒रु॒णा॒ । \newline
10. अ॒र्य॒म॒नित्य॑र्यमन्न् । \newline
11. देवाः᳚ सपीतयः सपीतयो॒ देवा॒ देवाः᳚ सपीत॒यो ऽपा॒ मपाꣳ॑ सपीतयो॒ देवा॒ देवाः᳚ सपीत॒यो ऽपा᳚म् । \newline
12. स॒पी॒त॒यो ऽपा॒ मपाꣳ॑ सपीतयः सपीत॒यो ऽपा᳚म् नपान् नपा॒ दपाꣳ॑ सपीतयः सपीत॒यो ऽपा᳚म् नपात् । \newline
13. स॒पी॒त॒य॒ इति॑ स - पी॒त॒यः॒ । \newline
14. अपा᳚म् नपान् नपा॒दपा॒ मपा᳚म् नपा दाशुहेमन् नाशुहेमन् नपा॒दपा॒ मपा᳚म् नपा दाशुहेमन्न् । \newline
15. न॒पा॒ दा॒शु॒हे॒म॒न् ना॒शु॒हे॒म॒न् न॒पा॒न् न॒पा॒ दा॒शु॒हे॒म॒न्न् । \newline
16. आ॒शु॒हे॒म॒न्नित्या॑शु - हे॒म॒न्न् । \newline
17. उ॒द्नो द॒त्त द॒त्तोद्न उ॒द्नो द॒त्तोद॒धि मु॑द॒धिम् द॒त्तोद्न उ॒द्नो द॒त्तोद॒धिम् । \newline
18. द॒त्तोद॒धि मु॑द॒धिम् द॒त्त द॒त्तोद॒धिम् भि॑न्त भिन्तोद॒धिम् द॒त्त द॒त्तोद॒धिम् भि॑न्त । \newline
19. उ॒द॒धिम् भि॑न्त भिन्तोद॒धि मु॑द॒धिम् भि॑न्त दि॒वो दि॒वो भि॑न्तोद॒धि मु॑द॒धिम् भि॑न्त दि॒वः । \newline
20. उ॒द॒धिमित्यु॑द - धिम् । \newline
21. भि॒न्त॒ दि॒वो दि॒वो भि॑न्त भिन्त दि॒वः प॒र्जन्या᳚त् प॒र्जन्या᳚द् दि॒वो भि॑न्त भिन्त दि॒वः प॒र्जन्या᳚त् । \newline
22. दि॒वः प॒र्जन्या᳚त् प॒र्जन्या᳚द् दि॒वो दि॒वः प॒र्जन्या॑ द॒न्तरि॑क्षा द॒न्तरि॑क्षात् प॒र्जन्या᳚द् दि॒वो दि॒वः प॒र्जन्या॑ द॒न्तरि॑क्षात् । \newline
23. प॒र्जन्या॑ द॒न्तरि॑क्षा द॒न्तरि॑क्षात् प॒र्जन्या᳚त् प॒र्जन्या॑ द॒न्तरि॑क्षात् पृथि॒व्याः पृ॑थि॒व्या अ॒न्तरि॑क्षात् प॒र्जन्या᳚त् प॒र्जन्या॑ द॒न्तरि॑क्षात् पृथि॒व्याः । \newline
24. अ॒न्तरि॑क्षात् पृथि॒व्याः पृ॑थि॒व्या अ॒न्तरि॑क्षा द॒न्तरि॑क्षात् पृथि॒व्या स्तत॒ स्ततः॑ पृथि॒व्या अ॒न्तरि॑क्षा द॒न्तरि॑क्षात् पृथि॒व्या स्ततः॑ । \newline
25. पृ॒थि॒व्या स्तत॒ स्ततः॑ पृथि॒व्याः पृ॑थि॒व्या स्ततो॑ नो न॒ स्ततः॑ पृथि॒व्याः पृ॑थि॒व्या स्ततो॑ नः । \newline
26. ततो॑ नो न॒ स्तत॒ स्ततो॑ नो॒ वृष्ट्या॒ वृष्ट्या॑ न॒ स्तत॒ स्ततो॑ नो॒ वृष्ट्या᳚ । \newline
27. नो॒ वृष्ट्या॒ वृष्ट्या॑ नो नो॒ वृष्ट्या॑ ऽवतावत॒ वृष्ट्या॑ नो नो॒ वृष्ट्या॑ ऽवत । \newline
28. वृष्ट्या॑ ऽवतावत॒ वृष्ट्या॒ वृष्ट्या॑ ऽवत । \newline
29. अ॒व॒तेत्य॑वत । \newline
30. दिवा॑ चिच् चि॒द् दिवा॒ दिवा॑ चि॒त् तम॒ स्तम॑ श्चि॒द् दिवा॒ दिवा॑ चि॒त् तमः॑ । \newline
31. चि॒त् तम॒ स्तम॑ श्चिच् चि॒त् तमः॑ कृण्वन्ति कृण्वन्ति॒ तम॑ श्चिच् चि॒त् तमः॑ कृण्वन्ति । \newline
32. तमः॑ कृण्वन्ति कृण्वन्ति॒ तम॒ स्तमः॑ कृण्वन्ति प॒र्जन्ये॑न प॒र्जन्ये॑न कृण्वन्ति॒ तम॒ स्तमः॑ कृण्वन्ति प॒र्जन्ये॑न । \newline
33. कृ॒ण्व॒न्ति॒ प॒र्जन्ये॑न प॒र्जन्ये॑न कृण्वन्ति कृण्वन्ति प॒र्जन्ये॑ नोदवा॒हे नो॑दवा॒हेन॑ प॒र्जन्ये॑न कृण्वन्ति कृण्वन्ति प॒र्जन्ये॑ नोदवा॒हेन॑ । \newline
34. प॒र्जन्ये॑ नोदवा॒हे नो॑दवा॒हेन॑ प॒र्जन्ये॑न प॒र्जन्ये॑ नोदवा॒हेन॑ । \newline
35. उ॒द॒वा॒हेनेत्यु॑द - वा॒हेन॑ । \newline
36. पृ॒थि॒वीं ॅयद् यत् पृ॑थि॒वीम् पृ॑थि॒वीं ॅयद् व्यु॒न्दन्ति॑ व्यु॒न्दन्ति॒ यत् पृ॑थि॒वीम् पृ॑थि॒वीं ॅयद् व्यु॒न्दन्ति॑ । \newline
37. यद् व्यु॒न्दन्ति॑ व्यु॒न्दन्ति॒ यद् यद् व्यु॒न्दन्ति॑ । \newline
38. व्यु॒न्दन्तीति॑ वि - उ॒न्दन्ति॑ । \newline
39. आ यं ॅय मा यम् नरो॒ नरो॒ य मा यम् नरः॑ । \newline
40. यम् नरो॒ नरो॒ यं ॅयम् नरः॑ सु॒दान॑वः सु॒दान॑वो॒ नरो॒ यं ॅयम् नरः॑ सु॒दान॑वः । \newline
41. नरः॑ सु॒दान॑वः सु॒दान॑वो॒ नरो॒ नरः॑ सु॒दान॑वो ददा॒शुषे॑ ददा॒शुषे॑ सु॒दान॑वो॒ नरो॒ नरः॑ सु॒दान॑वो ददा॒शुषे᳚ । \newline
42. सु॒दान॑वो ददा॒शुषे॑ ददा॒शुषे॑ सु॒दान॑वः सु॒दान॑वो ददा॒शुषे॑ दि॒वो दि॒वो द॑दा॒शुषे॑ सु॒दान॑वः सु॒दान॑वो ददा॒शुषे॑ दि॒वः । \newline
43. सु॒दान॑व॒ इति॑ सु - दान॑वः । \newline
44. द॒दा॒शुषे॑ दि॒वो दि॒वो द॑दा॒शुषे॑ ददा॒शुषे॑ दि॒वः कोश॒म् कोश॑म् दि॒वो द॑दा॒शुषे॑ ददा॒शुषे॑ दि॒वः कोश᳚म् । \newline
45. दि॒वः कोश॒म् कोश॑म् दि॒वो दि॒वः कोश॒ मचु॑च्यवु॒ रचु॑च्यवुः॒ कोश॑म् दि॒वो दि॒वः कोश॒ मचु॑च्यवुः । \newline
46. कोश॒ मचु॑च्यवु॒ रचु॑च्यवुः॒ कोश॒म् कोश॒ मचु॑च्यवुः । \newline
47. अचु॑च्यवु॒रित्यचु॑च्यवुः । \newline
48. वि प॒र्जन्याः᳚ प॒र्जन्या॒ वि वि प॒र्जन्याः᳚ सृजन्ति सृजन्ति प॒र्जन्या॒ वि वि प॒र्जन्याः᳚ सृजन्ति । \newline
49. प॒र्जन्याः᳚ सृजन्ति सृजन्ति प॒र्जन्याः᳚ प॒र्जन्याः᳚ सृजन्ति॒ रोद॑सी॒ रोद॑सी सृजन्ति प॒र्जन्याः᳚ प॒र्जन्याः᳚ सृजन्ति॒ रोद॑सी । \newline
50. सृ॒ज॒न्ति॒ रोद॑सी॒ रोद॑सी सृजन्ति सृजन्ति॒ रोद॑सी॒ अन्वनु॒ रोद॑सी सृजन्ति सृजन्ति॒ रोद॑सी॒ अनु॑ । \newline
51. रोद॑सी॒ अन्वनु॒ रोद॑सी॒ रोद॑सी॒ अनु॒ धन्व॑ना॒ धन्व॒ना ऽनु॒ रोद॑सी॒ रोद॑सी॒ अनु॒ धन्व॑ना । \newline
52. रोद॑सी॒ इति॒ रोद॑सी । \newline
53. अनु॒ धन्व॑ना॒ धन्व॒ना ऽन्वनु॒ धन्व॑ना यन्ति यन्ति॒ धन्व॒ना ऽन्वनु॒ धन्व॑ना यन्ति । \newline
54. धन्व॑ना यन्ति यन्ति॒ धन्व॑ना॒ धन्व॑ना यन्ति वृ॒ष्टयो॑ वृ॒ष्टयो॑ यन्ति॒ धन्व॑ना॒ धन्व॑ना यन्ति वृ॒ष्टयः॑ । \newline
55. य॒न्ति॒ वृ॒ष्टयो॑ वृ॒ष्टयो॑ यन्ति यन्ति वृ॒ष्टयः॑ । \newline
\pagebreak
\markright{ TS 2.4.8.2  \hfill https://www.vedavms.in \hfill}

\section{ TS 2.4.8.2 }

\textbf{TS 2.4.8.2 } \newline
\textbf{Samhita Paata} \newline

वृ॒ष्टयः॑ ॥ उदी॑रयथा मरुतः समुद्र॒तो यू॒यं ॅवृ॒ष्टिं ॅव॑र्.षयथा पुरीषिणः । न वो॑ दस्रा॒ उप॑ दस्यन्ति धे॒नवः॒ शुभं॑ ॅया॒तामनु॒ रथा॑ अवृथ्सत ॥ सृ॒जा वृ॒ष्टिं दि॒व आऽद्भिः स॑मु॒द्रं पृ॑ण ॥ अ॒ब्जा अ॑सि प्रथम॒जा बल॑मसि समु॒द्रियं᳚ ॥ उन्न॑म्भय पृथि॒वीं भि॒न्धीदं दि॒व्यं नभः॑ । उ॒द्नो दि॒व्यस्य॑ नो दे॒हीशा॑नो॒ विसृ॑जा॒ दृतिं᳚ ॥ ये दे॒वा ( ) दि॒विभा॑गा॒ ये᳚ऽन्तरि॑क्ष भागा॒ ये पृ॑थि॒वि भा॑गाः । त इ॒मं ॅय॒ज्ञ्म॑वन्तु॒ त इ॒दं क्षेत्र॒मा वि॑शन्तु॒ त इ॒दं क्षेत्र॒मनु॒ वि वि॑शन्तु ॥ \newline

\textbf{Pada Paata} \newline

वृ॒ष्टयः॑ ॥ उदिति॑ । ई॒र॒य॒थ॒ । म॒रु॒तः॒ । स॒मु॒द्र॒तः । यू॒यम् । वृ॒ष्टिम् । व॒र्॒.ष॒य॒थ॒ । पु॒री॒षि॒णः॒ ॥ न । वः॒ । द॒स्राः॒ । उपेति॑ । द॒स्य॒न्ति॒ । धे॒नवः॑ ।   शुभ᳚म् । या॒ताम् । अन्विति॑ । रथाः᳚ । अ॒वृ॒थ्स॒त॒ ॥ सृ॒ज ।  वृ॒ष्टिम् । दि॒वः । एति॑ । अ॒द्भिरित्य॑त् - भिः । स॒मु॒द्रम् । पृ॒ण॒ ॥ अ॒ब्जा इत्य॑प् - जाः । अ॒सि॒ । प्र॒थ॒म॒जा इति॑ प्रथम - जाः । बल᳚म् । अ॒सि॒ । स॒मु॒द्रिय᳚म् ॥ उदिति॑ । न॒भं॒य॒ । पृ॒थि॒वीम् । भि॒न्धि । इ॒दम् । दि॒व्यम् । नभः॑ ॥ उ॒द्नः । दि॒व्यस्य॑ । नः॒ । दे॒हि॒ । ईशा॑नः । वीति॑ । सृ॒ज॒ । दृति᳚म् ॥ ये । दे॒वाः ( ) । दि॒विभा॑गा॒ इति॑ दि॒वि - भा॒गाः॒ । ये । अ॒न्तरि॑क्षभागा॒ इत्य॒न्तरि॑क्ष - भा॒गाः॒ । ये । पृ॒थि॒विभा॑गा॒ इति॑ पृथि॒वि - भा॒गाः॒ ॥ ते । इ॒मम् । य॒ज्ञ्म् । अ॒व॒न्तु॒ । ते । इ॒दम् । क्षेत्र᳚म् । एति॑ । वि॒श॒न्तु॒ । ते । इ॒दम् । क्षेत्र᳚म् । अनु॑ । वीति॑ । वि॒श॒न्तु॒ ॥  \newline


\textbf{Krama Paata} \newline

वृ॒ष्टय॒ इति॑ वृ॒ष्टयः॑ ॥ उदी॑रयथ । ई॒र॒य॒था॒ म॒रु॒तः॒ । म॒रु॒तः॒ स॒मु॒द्र॒तः । स॒मु॒द्र॒तो यू॒यम् । यू॒यं ॅवृ॒ष्टिम् । वृ॒ष्टिं ॅव॑र्.षयथ । व॒र्॒.ष॒य॒था॒ पु॒री॒षि॒णः॒ । पु॒री॒ष॒ण॒ इति॑ पुरीषणः ॥ न वः॑ । वो॒ द॒स्राः॒ । द॒स्रा॒ उप॑ । उप॑ दस्यन्ति । द॒स्य॒न्ति॒ धे॒नवः॑ । धे॒नवः॒ शुभ᳚म् । शुभं॑ ॅया॒ताम् । या॒तामनु॑ । अनु॒ रथाः᳚ । रथा॑ अवृथ्सत । अ॒वृ॒थ्स॒तेत्य॑वृथ्सत ॥ सृ॒जा वृ॒ष्टिम् । वृ॒ष्टिम् दि॒वः । दि॒व आ । आ ऽद्भिः । अ॒द्भिः स॑मु॒द्रम् । अ॒द्भिरित्य॑त् - भिः । स॒मु॒द्रम् पृ॑ण । पृ॒णेति॑पृण ॥ अ॒ब्जा अ॑सि । अ॒ब्जा इत्य॑प् - जाः । अ॒सि॒ प्र॒थ॒म॒जाः । प्र॒थ॒म॒जा बल᳚म् । प्र॒थ॒म॒जा इति॑ प्रथम - जाः । बल॑मसि । अ॒सि॒ स॒मु॒द्रिय᳚म् । स॒मु॒द्रिय॒मिति॑ समु॒द्रिय᳚म् ॥ उन्न॑म्भय । न॒म्भ॒य॒ पृ॒थि॒वीम् । पृ॒थि॒वीम् भि॒न्धि । भि॒न्धीदम् । इ॒दम् दि॒व्यम् । दि॒व्यम् नभः॑ । नभ॒ इति॒ नभः॑ ॥ उ॒द्नो दि॒व्यस्य॑ । दि॒व्यस्य॑ नः । नो॒ दे॒हि॒ । दे॒हीशा॑नः । ईशा॑नो॒ वि । वि सृ॑ज । सृ॒जा॒ दृति᳚म् । दृति॒मिति॒ दृति᳚म् ॥ ये दे॒वाः ( ) । दे॒वा दि॒विभा॑गाः । दि॒विभा॑गा॒ ये । दि॒विभा॑गा॒ इति॑ दि॒वि - भा॒गाः॒ । ये᳚ऽन्तरि॑क्षभागाः । अ॒न्तरि॑क्षभागा॒ ये । अ॒न्तरि॑क्षभागा॒ इत्य॒न्तरि॑क्ष - भा॒गाः॒ । ये पृ॑थि॒विभा॑गाः । पृ॒थि॒विभा॑गा॒ इति॑ पृथि॒वि - भा॒गाः॒ ॥ त इ॒मम् । इ॒मं ॅय॒ज्ञ्म् । य॒ज्ञ्म॑वन्तु । अ॒व॒न्तु॒ ते । त इ॒दम् । इ॒दम् क्षेत्र᳚म् । क्षेत्र॒मा । आ वि॑शन्तु । वि॒श॒न्तु॒ ते । त इ॒दम् । इ॒दम् क्षेत्र᳚म् । क्षेत्र॒मनु॑ । अनु॒ वि । वि वि॑शन्तु । वि॒श॒न्त्विति॑ विशन्तु । \newline

\textbf{Jatai Paata} \newline

1. वृ॒ष्टय॒ इति॑ वृ॒ष्टयः॑ । \newline
2. उदी॑रयथे रय॒थो दुदी॑रयथ । \newline
3. ई॒र॒य॒था॒ म॒रु॒तो॒ म॒रु॒त॒ ई॒र॒य॒थे॒ र॒य॒था॒ म॒रु॒तः॒ । \newline
4. म॒रु॒तः॒ स॒मु॒द्र॒तः स॑मुद्र॒तो म॑रुतो मरुतः समुद्र॒तः । \newline
5. स॒मु॒द्र॒तो यू॒यं ॅयू॒यꣳ स॑मुद्र॒तः स॑मुद्र॒तो यू॒यम् । \newline
6. यू॒यं ॅवृ॒ष्टिं ॅवृ॒ष्टिं ॅयू॒यं ॅयू॒यं ॅवृ॒ष्टिम् । \newline
7. वृ॒ष्टिं ॅव॑र्.षयथ वर्.षयथ वृ॒ष्टिं ॅवृ॒ष्टिं ॅव॑र्.षयथ । \newline
8. व॒र्॒.ष॒य॒था॒ पु॒री॒षि॒णः॒ पु॒री॒षि॒णो॒ व॒र्॒.ष॒य॒थ॒ व॒र्॒.ष॒य॒था॒ पु॒री॒षि॒णः॒ । \newline
9. पु॒री॒ष॒ण॒ इति॑ पुरीषणः । \newline
10. न वो॑ वो॒ न न वः॑ । \newline
11. वो॒ द॒स्रा॒ द॒स्रा॒ वो॒ वो॒ द॒स्राः॒ । \newline
12. द॒स्रा॒ उपोप॑ दस्रा दस्रा॒ उप॑ । \newline
13. उप॑ दस्यन्ति दस्य॒ न्त्युपोप॑ दस्यन्ति । \newline
14. द॒स्य॒न्ति॒ धे॒नवो॑ धे॒नवो॑ दस्यन्ति दस्यन्ति धे॒नवः॑ । \newline
15. धे॒नवः॒ शुभꣳ॒॒ शुभ॑म् धे॒नवो॑ धे॒नवः॒ शुभ᳚म् । \newline
16. शुभं॑ ॅया॒तां ॅया॒ताꣳ शुभꣳ॒॒ शुभं॑ ॅया॒ताम् । \newline
17. या॒ता मन्वनु॑ या॒तां ॅया॒ता मनु॑ । \newline
18. अनु॒ रथा॒ रथा॒ अन्वनु॒ रथाः᳚ । \newline
19. रथा॑ अवृथ्सता वृथ्सत॒ रथा॒ रथा॑ अवृथ्सत । \newline
20. अ॒वृ॒थ्स॒तेत्य॑वृथ्सत । \newline
21. सृ॒जा वृ॒ष्टिं ॅवृ॒ष्टिꣳ सृ॒ज सृ॒जा वृ॒ष्टिम् । \newline
22. वृ॒ष्टिम् दि॒वो दि॒वो वृ॒ष्टिं ॅवृ॒ष्टिम् दि॒वः । \newline
23. दि॒व आ दि॒वो दि॒व आ । \newline
24. आ ऽद्भि र॒द्भिरा ऽद्भिः । \newline
25. अ॒द्भिः स॑मु॒द्रꣳ स॑मु॒द्र म॒द्भि र॒द्भिः स॑मु॒द्रम् । \newline
26. अ॒द्भिरित्य॑त् - भिः । \newline
27. स॒मु॒द्रम् पृ॑ण पृण समु॒द्रꣳ स॑मु॒द्रम् पृ॑ण । \newline
28. पृ॒णेति॑ पृण । \newline
29. अ॒ब्जा अ॑स्यस्य॒ब्जा अ॒ब्जा अ॑सि । \newline
30. अ॒ब्जा इत्य॑प् - जाः । \newline
31. अ॒सि॒ प्र॒थ॒म॒जाः प्र॑थम॒जा अ॑स्यसि प्रथम॒जाः । \newline
32. प्र॒थ॒म॒जा बल॒म् बल॑म् प्रथम॒जाः प्र॑थम॒जा बल᳚म् । \newline
33. प्र॒थ॒म॒जा इति॑ प्रथम - जाः । \newline
34. बल॑ मस्यसि॒ बल॒म् बल॑ मसि । \newline
35. अ॒सि॒ स॒मु॒द्रियꣳ॑ समु॒द्रिय॑ मस्यसि समु॒द्रिय᳚म् । \newline
36. स॒मु॒द्रिय॒मिति॑ समु॒द्रिय᳚म् । \newline
37. उन् नं॑भय नंभ॒योदुन् नं॑भय । \newline
38. नं॒भ॒य॒ पृ॒थि॒वीम् पृ॑थि॒वीम् नं॑भय नंभय पृथि॒वीम् । \newline
39. पृ॒थि॒वीम् भि॒न्धि भि॒न्धि पृ॑थि॒वीम् पृ॑थि॒वीम् भि॒न्धि । \newline
40. भि॒न्धीद मि॒दम् भि॒न्धि भि॒न्धीदम् । \newline
41. इ॒दम् दि॒व्यम् दि॒व्य मि॒द मि॒दम् दि॒व्यम् । \newline
42. दि॒व्यम् नभो॒ नभो॑ दि॒व्यम् दि॒व्यम् नभः॑ । \newline
43. नभ॒ इति॒ नभः॑ । \newline
44. उ॒द्नो दि॒व्यस्य॑ दि॒व्यस्यो॒द्न उ॒द्नो दि॒व्यस्य॑ । \newline
45. दि॒व्यस्य॑ नो नो दि॒व्यस्य॑ दि॒व्यस्य॑ नः । \newline
46. नो॒ दे॒हि॒ दे॒हि॒ नो॒ नो॒ दे॒हि॒ । \newline
47. दे॒हीशा॑न॒ ईशा॑नो देहि दे॒हीशा॑नः । \newline
48. ईशा॑नो॒ वि वीशा॑न॒ ईशा॑नो॒ वि । \newline
49. वि सृ॑ज सृज॒ वि वि सृ॑ज । \newline
50. सृ॒जा॒ दृति॒म् दृतिꣳ॑ सृज सृजा॒ दृति᳚म् । \newline
51. दृति॒मिति॒ दृति᳚म् । \newline
52. ये दे॒वा दे॒वा ये ये दे॒वाः । \newline
53. दे॒वा दि॒विभा॑गा दि॒विभा॑गा दे॒वा दे॒वा दि॒विभा॑गाः । \newline
54. दि॒विभा॑गा॒ ये ये दि॒विभा॑गा दि॒विभा॑गा॒ ये । \newline
55. दि॒विभा॑गा॒ इति॑ दि॒वि - भा॒गाः॒ । \newline
56. ये᳚ ऽन्तरि॑क्षभागा अ॒न्तरि॑क्षभागा॒ ये ये᳚ ऽन्तरि॑क्षभागाः । \newline
57. अ॒न्तरि॑क्षभागा॒ ये ये᳚ ऽन्तरि॑क्षभागा अ॒न्तरि॑क्षभागा॒ ये । \newline
58. अ॒न्तरि॑क्षभागा॒ इत्य॒न्तरि॑क्ष - भा॒गाः॒ । \newline
59. ये पृ॑थि॒विभा॑गाः पृथि॒विभा॑गा॒ ये ये पृ॑थि॒विभा॑गाः । \newline
60. पृ॒थि॒विभा॑गा॒ इति॑ पृथि॒वि - भा॒गाः॒ । \newline
61. त इ॒म मि॒मम् ते त इ॒मम् । \newline
62. इ॒मं ॅय॒ज्ञ्ं ॅय॒ज्ञ् मि॒म मि॒मं ॅय॒ज्ञ्म् । \newline
63. य॒ज्ञ् म॑व न्त्ववन्तु य॒ज्ञ्ं ॅय॒ज्ञ् म॑वन्तु । \newline
64. अ॒व॒न्तु॒ ते ते॑ ऽवन्त्ववन्तु॒ ते । \newline
65. त इ॒द मि॒दम् ते त इ॒दम् । \newline
66. इ॒दम् क्षेत्र॒म् क्षेत्र॑ मि॒द मि॒दम् क्षेत्र᳚म् । \newline
67. क्षेत्र॒ मा क्षेत्र॒म् क्षेत्र॒ मा । \newline
68. आ वि॑शन्तु विश॒न्त्वा वि॑शन्तु । \newline
69. वि॒श॒न्तु॒ ते ते वि॑शन्तु विशन्तु॒ ते । \newline
70. त इ॒द मि॒दम् ते त इ॒दम् । \newline
71. इ॒दम् क्षेत्र॒म् क्षेत्र॑ मि॒द मि॒दम् क्षेत्र᳚म् । \newline
72. क्षेत्र॒ मन्वनु॒ क्षेत्र॒म् क्षेत्र॒ मनु॑ । \newline
73. अनु॒ वि व्यन्वनु॒ वि । \newline
74. वि वि॑शन्तु विशन्तु॒ वि वि वि॑शन्तु । \newline
75. वि॒श॒न्त्विति॑ विशन्तु । \newline

\textbf{Ghana Paata } \newline

1. वृ॒ष्टय॒ इति॑ वृ॒ष्टयः॑ । \newline
2. उदी॑रयथे रय॒थो दुदी॑रयथा मरुतो मरुत ईरय॒थो दुदी॑रयथा मरुतः । \newline
3. ई॒र॒य॒था॒ म॒रु॒तो॒ म॒रु॒त॒ ई॒र॒य॒थे॒र॒य॒था॒ म॒रु॒तः॒ स॒मु॒द्र॒तः स॑मुद्र॒तो म॑रुत ईरयथेरयथा मरुतः समुद्र॒तः । \newline
4. म॒रु॒तः॒ स॒मु॒द्र॒तः स॑मुद्र॒तो म॑रुतो मरुतः समुद्र॒तो यू॒यं ॅयू॒यꣳ स॑मुद्र॒तो म॑रुतो मरुतः समुद्र॒तो यू॒यम् । \newline
5. स॒मु॒द्र॒तो यू॒यं ॅयू॒यꣳ स॑मुद्र॒तः स॑मुद्र॒तो यू॒यं ॅवृ॒ष्टिं ॅवृ॒ष्टिं ॅयू॒यꣳ स॑मुद्र॒तः स॑मुद्र॒तो यू॒यं ॅवृ॒ष्टिम् । \newline
6. यू॒यं ॅवृ॒ष्टिं ॅवृ॒ष्टिं ॅयू॒यं ॅयू॒यं ॅवृ॒ष्टिं ॅव॑र्.षयथ वर्.षयथ वृ॒ष्टिं ॅयू॒यं ॅयू॒यं ॅवृ॒ष्टिं ॅव॑र्.षयथ । \newline
7. वृ॒ष्टिं ॅव॑र्.षयथ वर्.षयथ वृ॒ष्टिं ॅवृ॒ष्टिं ॅव॑र्.षयथा पुरीषिणः पुरीषिणो वर्.षयथ वृ॒ष्टिं ॅवृ॒ष्टिं ॅव॑र्.षयथा पुरीषिणः । \newline
8. व॒र्॒.ष॒य॒था॒ पु॒री॒षि॒णः॒ पु॒री॒षि॒णो॒ व॒र्॒.ष॒य॒थ॒ व॒र्॒.ष॒य॒था॒ पु॒री॒षि॒णः॒ । \newline
9. पु॒री॒ष॒ण॒ इति॑ पुरीषणः । \newline
10. न वो॑ वो॒ न न वो॑ दस्रा दस्रा वो॒ न न वो॑ दस्राः । \newline
11. वो॒ द॒स्रा॒ द॒स्रा॒ वो॒ वो॒ द॒स्रा॒ उपोप॑ दस्रा वो वो दस्रा॒ उप॑ । \newline
12. द॒स्रा॒ उपोप॑ दस्रा दस्रा॒ उप॑ दस्यन्ति दस्य॒ न्त्युप॑ दस्रा दस्रा॒ उप॑ दस्यन्ति । \newline
13. उप॑ दस्यन्ति दस्य॒ न्त्युपोप॑ दस्यन्ति धे॒नवो॑ धे॒नवो॑ दस्य॒ न्त्युपोप॑ दस्यन्ति धे॒नवः॑ । \newline
14. द॒स्य॒न्ति॒ धे॒नवो॑ धे॒नवो॑ दस्यन्ति दस्यन्ति धे॒नवः॒ शुभꣳ॒॒ शुभ॑म् धे॒नवो॑ दस्यन्ति दस्यन्ति धे॒नवः॒ शुभ᳚म् । \newline
15. धे॒नवः॒ शुभꣳ॒॒ शुभ॑म् धे॒नवो॑ धे॒नवः॒ शुभं॑ ॅया॒तां ॅया॒ताꣳ शुभ॑म् धे॒नवो॑ धे॒नवः॒ शुभं॑ ॅया॒ताम् । \newline
16. शुभं॑ ॅया॒तां ॅया॒ताꣳ शुभꣳ॒॒ शुभं॑ ॅया॒ता मन्वनु॑ या॒ताꣳ शुभꣳ॒॒ शुभं॑ ॅया॒ता मनु॑ । \newline
17. या॒ता मन्वनु॑ या॒तां ॅया॒ता मनु॒ रथा॒ रथा॒ अनु॑ या॒तां ॅया॒ता मनु॒ रथाः᳚ । \newline
18. अनु॒ रथा॒ रथा॒ अन्वनु॒ रथा॑ अवृथ्सता वृथ्सत॒ रथा॒ अन्वनु॒ रथा॑ अवृथ्सत । \newline
19. रथा॑ अवृथ्सता वृथ्सत॒ रथा॒ रथा॑ अवृथ्सत । \newline
20. अ॒वृ॒थ्स॒तेत्य॑वृथ्सत । \newline
21. सृ॒जा वृ॒ष्टिं ॅवृ॒ष्टिꣳ सृ॒ज सृ॒जा वृ॒ष्टिम् दि॒वो दि॒वो वृ॒ष्टिꣳ सृ॒ज सृ॒जा वृ॒ष्टिम् दि॒वः । \newline
22. वृ॒ष्टिम् दि॒वो दि॒वो वृ॒ष्टिं ॅवृ॒ष्टिम् दि॒व आ दि॒वो वृ॒ष्टिं ॅवृ॒ष्टिम् दि॒व आ । \newline
23. दि॒व आ दि॒वो दि॒व आ ऽद्भि र॒द्भिरा दि॒वो दि॒व आ ऽद्भिः । \newline
24. आ ऽद्भिर॒द्भिरा ऽद्भिः स॑मु॒द्रꣳ स॑मु॒द्र म॒द्भिरा ऽद्भिः स॑मु॒द्रम् । \newline
25. अ॒द्भिः स॑मु॒द्रꣳ स॑मु॒द्र म॒द्भि र॒द्भिः स॑मु॒द्रम् पृ॑ण पृण समु॒द्र म॒द्भि र॒द्भिः स॑मु॒द्रम् पृ॑ण । \newline
26. अ॒द्भिरित्य॑त् - भिः । \newline
27. स॒मु॒द्रम् पृ॑ण पृण समु॒द्रꣳ स॑मु॒द्रम् पृ॑ण । \newline
28. पृ॒णेति॑ पृण । \newline
29. अ॒ब्जा अ॑स्यस्य॒ब्जा अ॒ब्जा अ॑सि प्रथम॒जाः प्र॑थम॒जा अ॑स्य॒ब्जा अ॒ब्जा अ॑सि प्रथम॒जाः । \newline
30. अ॒ब्जा इत्य॑प् - जाः । \newline
31. अ॒सि॒ प्र॒थ॒म॒जाः प्र॑थम॒जा अ॑स्यसि प्रथम॒जा बल॒म् बल॑म् प्रथम॒जा अ॑स्यसि प्रथम॒जा बल᳚म् । \newline
32. प्र॒थ॒म॒जा बल॒म् बल॑म् प्रथम॒जाः प्र॑थम॒जा बल॑ मस्यसि॒ बल॑म् प्रथम॒जाः प्र॑थम॒जा बल॑ मसि । \newline
33. प्र॒थ॒म॒जा इति॑ प्रथम - जाः । \newline
34. बल॑ मस्यसि॒ बल॒म् बल॑ मसि समु॒द्रियꣳ॑ समु॒द्रिय॑ मसि॒ बल॒म् बल॑ मसि समु॒द्रिय᳚म् । \newline
35. अ॒सि॒ स॒मु॒द्रियꣳ॑ समु॒द्रिय॑ मस्यसि समु॒द्रिय᳚म् । \newline
36. स॒मु॒द्रिय॒मिति॑ समु॒द्रिय᳚म् । \newline
37. उन् नं॑भय नंभ॒योदुन् नं॑भय पृथि॒वीम् पृ॑थि॒वीम् नं॑भ॒योदुन् नं॑भय पृथि॒वीम् । \newline
38. नं॒भ॒य॒ पृ॒थि॒वीम् पृ॑थि॒वीम् नं॑भय नंभय पृथि॒वीम् भि॒न्धि भि॒न्धि पृ॑थि॒वीम् नं॑भय नंभय पृथि॒वीम् भि॒न्धि । \newline
39. पृ॒थि॒वीम् भि॒न्धि भि॒न्धि पृ॑थि॒वीम् पृ॑थि॒वीम् भि॒न्धीद मि॒दम् भि॒न्धि पृ॑थि॒वीम् पृ॑थि॒वीम् भि॒न्धीदम् । \newline
40. भि॒न्धीद मि॒दम् भि॒न्धि भि॒न्धीदम् दि॒व्यम् दि॒व्य मि॒दम् भि॒न्धि भि॒न्धीदम् दि॒व्यम् । \newline
41. इ॒दम् दि॒व्यम् दि॒व्य मि॒द मि॒दम् दि॒व्यम् नभो॒ नभो॑ दि॒व्य मि॒द मि॒दम् दि॒व्यम् नभः॑ । \newline
42. दि॒व्यम् नभो॒ नभो॑ दि॒व्यम् दि॒व्यम् नभः॑ । \newline
43. नभ॒ इति॒ नभः॑ । \newline
44. उ॒द्नो दि॒व्यस्य॑ दि॒व्यस्यो॒द्न उ॒द्नो दि॒व्यस्य॑ नो नो दि॒व्यस्यो॒द्न उ॒द्नो दि॒व्यस्य॑ नः । \newline
45. दि॒व्यस्य॑ नो नो दि॒व्यस्य॑ दि॒व्यस्य॑ नो देहि देहि नो दि॒व्यस्य॑ दि॒व्यस्य॑ नो देहि । \newline
46. नो॒ दे॒हि॒ दे॒हि॒ नो॒ नो॒ दे॒हीशा॑न॒ ईशा॑नो देहि नो नो दे॒हीशा॑नः । \newline
47. दे॒हीशा॑न॒ ईशा॑नो देहि दे॒हीशा॑नो॒ वि वीशा॑नो देहि दे॒हीशा॑नो॒ वि । \newline
48. ईशा॑नो॒ वि वीशा॑न॒ ईशा॑नो॒ वि सृ॑ज सृज॒ वीशा॑न॒ ईशा॑नो॒ वि सृ॑ज । \newline
49. वि सृ॑ज सृज॒ वि वि सृ॑जा॒ दृति॒म् दृतिꣳ॑ सृज॒ वि वि सृ॑जा॒ दृति᳚म् । \newline
50. सृ॒जा॒ दृति॒म् दृतिꣳ॑ सृज सृजा॒ दृति᳚म् । \newline
51. दृति॒मिति॒ दृति᳚म् । \newline
52. ये दे॒वा दे॒वा ये ये दे॒वा दि॒विभा॑गा दि॒विभा॑गा दे॒वा ये ये दे॒वा दि॒विभा॑गाः । \newline
53. दे॒वा दि॒विभा॑गा दि॒विभा॑गा दे॒वा दे॒वा दि॒विभा॑गा॒ ये ये दि॒विभा॑गा दे॒वा दे॒वा दि॒विभा॑गा॒ ये । \newline
54. दि॒विभा॑गा॒ ये ये दि॒विभा॑गा दि॒विभा॑गा॒ ये᳚ ऽन्तरि॑क्षभागा अ॒न्तरि॑क्षभागा॒ ये दि॒विभा॑गा दि॒विभा॑गा॒ ये᳚ ऽन्तरि॑क्षभागाः । \newline
55. दि॒विभा॑गा॒ इति॑ दि॒वि - भा॒गाः॒ । \newline
56. ये᳚ ऽन्तरि॑क्षभागा अ॒न्तरि॑क्षभागा॒ ये ये᳚ ऽन्तरि॑क्षभागा॒ ये ये᳚ ऽन्तरि॑क्षभागा॒ ये ये᳚ ऽन्तरि॑क्षभागा॒ ये । \newline
57. अ॒न्तरि॑क्षभागा॒ ये ये᳚ ऽन्तरि॑क्षभागा अ॒न्तरि॑क्षभागा॒ ये पृ॑थि॒विभा॑गाः पृथि॒विभा॑गा॒ ये᳚ ऽन्तरि॑क्षभागा अ॒न्तरि॑क्षभागा॒ ये पृ॑थि॒विभा॑गाः । \newline
58. अ॒न्तरि॑क्षभागा॒ इत्य॒न्तरि॑क्ष - भा॒गाः॒ । \newline
59. ये पृ॑थि॒विभा॑गाः पृथि॒विभा॑गा॒ ये ये पृ॑थि॒विभा॑गाः । \newline
60. पृ॒थि॒विभा॑गा॒ इति॑ पृथि॒वि - भा॒गाः॒ । \newline
61. त इ॒म मि॒मम् ते त इ॒मं ॅय॒ज्ञ्ं ॅय॒ज्ञ् मि॒मम् ते त इ॒मं ॅय॒ज्ञ्म् । \newline
62. इ॒मं ॅय॒ज्ञ्ं ॅय॒ज्ञ् मि॒म मि॒मं ॅय॒ज्ञ् म॑व न्त्ववन्तु य॒ज्ञ् मि॒म मि॒मं ॅय॒ज्ञ् म॑वन्तु । \newline
63. य॒ज्ञ् म॑व न्त्ववन्तु य॒ज्ञ्ं ॅय॒ज्ञ् म॑वन्तु॒ ते ते॑ ऽवन्तु य॒ज्ञ्ं ॅय॒ज्ञ् म॑वन्तु॒ ते । \newline
64. अ॒व॒न्तु॒ ते ते॑ ऽवन्त्ववन्तु॒ त इ॒द मि॒दम् ते॑ ऽवन्त्ववन्तु॒ त इ॒दम् । \newline
65. त इ॒द मि॒दम् ते त इ॒दम् क्षेत्र॒म् क्षेत्र॑ मि॒दम् ते त इ॒दम् क्षेत्र᳚म् । \newline
66. इ॒दम् क्षेत्र॒म् क्षेत्र॑ मि॒द मि॒दम् क्षेत्र॒ मा क्षेत्र॑ मि॒द मि॒दम् क्षेत्र॒ मा । \newline
67. क्षेत्र॒ मा क्षेत्र॒म् क्षेत्र॒ मा वि॑शन्तु विश॒न्त्वा क्षेत्र॒म् क्षेत्र॒ मा वि॑शन्तु । \newline
68. आ वि॑शन्तु विश॒न्त्वा वि॑शन्तु॒ ते ते वि॑श॒न्त्वा वि॑शन्तु॒ ते । \newline
69. वि॒श॒न्तु॒ ते ते वि॑शन्तु विशन्तु॒ त इ॒द मि॒दम् ते वि॑शन्तु विशन्तु॒ त इ॒दम् । \newline
70. त इ॒द मि॒दम् ते त इ॒दम् क्षेत्र॒म् क्षेत्र॑ मि॒दम् ते त इ॒दम् क्षेत्र᳚म् । \newline
71. इ॒दम् क्षेत्र॒म् क्षेत्र॑ मि॒द मि॒दम् क्षेत्र॒ मन्वनु॒ क्षेत्र॑ मि॒द मि॒दम् क्षेत्र॒ मनु॑ । \newline
72. क्षेत्र॒ मन्वनु॒ क्षेत्र॒म् क्षेत्र॒ मनु॒ वि व्यनु॒ क्षेत्र॒म् क्षेत्र॒ मनु॒ वि । \newline
73. अनु॒ वि व्यन्वनु॒ वि वि॑शन्तु विशन्तु॒ व्यन्वनु॒ वि वि॑शन्तु । \newline
74. वि वि॑शन्तु विशन्तु॒ वि वि वि॑शन्तु । \newline
75. वि॒श॒न्त्विति॑ विशन्तु । \newline
\pagebreak
\markright{ TS 2.4.9.1  \hfill https://www.vedavms.in \hfill}

\section{ TS 2.4.9.1 }

\textbf{TS 2.4.9.1 } \newline
\textbf{Samhita Paata} \newline

मा॒रु॒तम॑सि म॒रुता॒मोज॒ इति॑ कृ॒ष्णं ॅवासः॑ कृ॒ष्णतू॑षं॒ परि॑ धत्त ए॒तद्वै वृष्‌ट्यै॑ रू॒पꣳ सरू॑प ए॒व भू॒त्वा प॒र्जन्यं॑ ॅवर्.षयतिर॒मय॑त मरुतः श्ये॒नमा॒यिन॒मिति॑ पश्चाद्वा॒तं प्रति॑ मीवति पुरोवा॒तमे॒व ज॑नयति व॒र्॒.षस्या व॑रुद्ध्यै वातना॒मानि॑ जुहोति वा॒युर्वै वृष्‌ट्या॑ ईशे वा॒युमे॒व स्वेन॑ भाग॒धेये॒नोप॑ धावति॒ स ए॒वास्मै॑ प॒र्जन्यं॑ ॅवर्.षयत्य॒ष्टौ - [  ] \newline

\textbf{Pada Paata} \newline

मा॒रु॒तम् । अ॒सि॒ । म॒रुता᳚म् । ओजः॑ । इति॑ । कृ॒ष्णम् । वासः॑ । कृ॒ष्णतू॑ष॒मिति॑ कृ॒ष्ण - तू॒ष॒म् । परीति॑ । ध॒त्ते॒ । ए॒तत् । वै । वृष्ट्यै᳚ । रू॒पम् । सरू॑प॒ इति॒ स - रू॒पः॒ ।  ए॒व । भू॒त्वा । प॒र्जन्य᳚म् । व॒र्.॒ष॒य॒ति॒ । र॒मय॑त । म॒रु॒तः॒ । श्ये॒नम् । आ॒यिन᳚म् । इति॑ । प॒श्चा॒द्वा॒तमिति॑ पश्चात् - वा॒तम् । प्रतीति॑ । मी॒व॒ति॒ । पु॒रो॒वा॒तमिति॑ पुरः - वा॒तम् । ए॒व । ज॒न॒य॒ति॒ । व॒र्॒.षस्य॑ । अव॑रुद्ध्या॒ इत्यव॑ - रु॒द्ध्यै॒ । वा॒त॒ना॒मानीति॑ वात - ना॒मानि॑ । जु॒हो॒ति॒ । वा॒युः । वै । वृष्ट्याः᳚ । ई॒शे॒ ।   वा॒युम् । ए॒व । स्वेन॑ । भा॒ग॒धेये॒नेति॑ भाग- धेये॑न । उपेति॑ । धा॒व॒ति॒ । सः । ए॒व । अ॒स्मै॒ । प॒र्जन्य᳚म् । व॒र्.॒ष॒य॒ति॒ । अ॒ष्टौ ।  \newline


\textbf{Krama Paata} \newline

मा॒रु॒तम॑सि । अ॒सि॒ म॒रुता᳚म् । म॒रुता॒मोजः॑ । ओज॒ इति॑ । इति॑ कृ॒ष्णम् । कृ॒ष्णं ॅवासः॑ । वासः॑ कृ॒ष्णतू॑षम् । कृ॒ष्णतू॑ष॒म् परि॑ । कृ॒ष्णतू॑ष॒मिति॑ कृ॒ष्ण - तू॒ष॒म् । परि॑ धत्ते । ध॒त्त॒ ए॒तत् । ए॒तद् वै । वै वृष्ट्यै᳚ । वृष्ट्यै॑ रू॒पम् । रू॒पꣳ सरू॑पः । सरू॑प ए॒व । सरू॑प॒ इति॒ स - रू॒पः॒ । ए॒व भू॒त्वा । भू॒त्वा प॒र्जन्य᳚म् । प॒र्जन्यं॑ ॅवर्.षयति । व॒र्॒.ष॒य॒ति॒ र॒मय॑त । र॒मय॑त मरुतः । म॒रु॒तः॒ श्ये॒नम् । श्ये॒नमा॒यिन᳚म् । आ॒यिन॒मिति॑ । इति॑ पश्चाद्वा॒तम् । प॒श्चा॒द्वा॒तम् प्रति॑ । प॒श्चा॒द्वा॒तमिति॑ पश्चात् - वा॒तम् । प्रति॑ मीवति । मी॒व॒ति॒ पु॒रो॒वा॒तम् । पु॒रो॒वा॒तमे॒व । पु॒रो॒वा॒तमिति॑ पुरः - वा॒तम् । ए॒व ज॑नयति । ज॒न॒य॒ति॒ व॒र्.॒षस्य॑ । व॒र्॒.षस्याव॑रुद्ध्यै । अव॑रुद्ध्यै वातना॒मानि॑ । अव॑रुद्ध्या॒ इत्यव॑ - रु॒द्ध्यै॒ । वा॒त॒ना॒मानि॑ जुहोति । वा॒त॒ना॒मानीति॑ वात - ना॒मानि॑ । जु॒हो॒ति॒ वा॒युः । वा॒युर् वै । वै वृष्ट्याः᳚ । वृष्ट्या॑ ईशे । ई॒शे॒ वा॒युम् । वा॒युमे॒व । ए॒व स्वेन॑ । स्वेन॑ भाग॒धेये॑न । भा॒ग॒धेये॒नोप॑ । भा॒ग॒धेये॒नेति॑ भाग - धेये॑न । उप॑ धावति । धा॒व॒ति॒ सः । स ए॒व । ए॒वास्मै᳚ । अ॒स्मै॒ प॒र्जन्य᳚म् । प॒र्जन्यं॑ ॅवर्.षयति । व॒र्॒.ष॒य॒त्य॒ष्टौ । अ॒ष्टौ जु॑होति \newline

\textbf{Jatai Paata} \newline

1. मा॒रु॒त म॑स्यसि मारु॒तम् मा॑रु॒त म॑सि । \newline
2. अ॒सि॒ म॒रुता᳚म् म॒रुता॑ मस्यसि म॒रुता᳚म् । \newline
3. म॒रुता॒ मोज॒ ओजो॑ म॒रुता᳚म् म॒रुता॒ मोजः॑ । \newline
4. ओज॒ इती त्योज॒ ओज॒ इति॑ । \newline
5. इति॑ कृ॒ष्णम् कृ॒ष्ण मितीति॑ कृ॒ष्णम् । \newline
6. कृ॒ष्णं ॅवासो॒ वासः॑ कृ॒ष्णम् कृ॒ष्णं ॅवासः॑ । \newline
7. वासः॑ कृ॒ष्णतू॑षम् कृ॒ष्णतू॑षं॒ ॅवासो॒ वासः॑ कृ॒ष्णतू॑षम् । \newline
8. कृ॒ष्णतू॑ष॒म् परि॒ परि॑ कृ॒ष्णतू॑षम् कृ॒ष्णतू॑ष॒म् परि॑ । \newline
9. कृ॒ष्णतू॑ष॒मिति॑ कृ॒ष्ण - तू॒ष॒म् । \newline
10. परि॑ धत्ते धत्ते॒ परि॒ परि॑ धत्ते । \newline
11. ध॒त्त॒ ए॒त दे॒तद् ध॑त्ते धत्त ए॒तत् । \newline
12. ए॒तद् वै वा ए॒त दे॒तद् वै । \newline
13. वै वृष्ट्यै॒ वृष्ट्यै॒ वै वै वृष्ट्यै᳚ । \newline
14. वृष्ट्यै॑ रू॒पꣳ रू॒पं ॅवृष्ट्यै॒ वृष्ट्यै॑ रू॒पम् । \newline
15. रू॒पꣳ सरू॑पः॒ सरू॑पो रू॒पꣳ रू॒पꣳ सरू॑पः । \newline
16. सरू॑प ए॒वैव सरू॑पः॒ सरू॑प ए॒व । \newline
17. सरू॑प॒ इति॒ स - रू॒पः॒ । \newline
18. ए॒व भू॒त्वा भू॒त्वैवैव भू॒त्वा । \newline
19. भू॒त्वा प॒र्जन्य॑म् प॒र्जन्य॑म् भू॒त्वा भू॒त्वा प॒र्जन्य᳚म् । \newline
20. प॒र्जन्यं॑ ॅवर्.षयति वर्.षयति प॒र्जन्य॑म् प॒र्जन्यं॑ ॅवर्.षयति । \newline
21. व॒र्॒.ष॒य॒ति॒ र॒मय॑त र॒मय॑त वर्.षयति वर्.षयति र॒मय॑त । \newline
22. र॒मय॑त मरुतो मरुतो र॒मय॑त र॒मय॑त मरुतः । \newline
23. म॒रु॒तः॒ श्ये॒नꣳ श्ये॒नम् म॑रुतो मरुतः श्ये॒नम् । \newline
24. श्ये॒न मा॒यिन॑ मा॒यिनꣳ॑ श्ये॒नꣳ श्ये॒न मा॒यिन᳚म् । \newline
25. आ॒यिन॒ मिती त्या॒यिन॑ मा॒यिन॒ मिति॑ । \newline
26. इति॑ पश्चाद्वा॒तम् प॑श्चाद्वा॒त मितीति॑ पश्चाद्वा॒तम् । \newline
27. प॒श्चा॒द्वा॒तम् प्रति॒ प्रति॑ पश्चाद्वा॒तम् प॑श्चाद्वा॒तम् प्रति॑ । \newline
28. प॒श्चा॒द्वा॒तमिति॑ पश्चात् - वा॒तम् । \newline
29. प्रति॑ मीवति मीवति॒ प्रति॒ प्रति॑ मीवति । \newline
30. मी॒व॒ति॒ पु॒रो॒वा॒तम् पु॑रोवा॒तम् मी॑वति मीवति पुरोवा॒तम् । \newline
31. पु॒रो॒वा॒त मे॒वैव पु॑रोवा॒तम् पु॑रोवा॒त मे॒व । \newline
32. पु॒रो॒वा॒तमिति॑ पुरः - वा॒तम् । \newline
33. ए॒व ज॑नयति जनय त्ये॒वैव ज॑नयति । \newline
34. ज॒न॒य॒ति॒ व॒र्॒.षस्य॑ व॒र्॒.षस्य॑ जनयति जनयति व॒र्॒.षस्य॑ । \newline
35. व॒र्॒.षस्या व॑रुद्ध्या॒ अव॑रुद्ध्यै व॒र्॒.षस्य॑ व॒र्॒.षस्या व॑रुद्ध्यै । \newline
36. अव॑रुद्ध्यै वातना॒मानि॑ वातना॒मान्य व॑रुद्ध्या॒ अव॑रुद्ध्यै वातना॒मानि॑ । \newline
37. अव॑रुद्ध्या॒ इत्यव॑ - रु॒द्ध्यै॒ । \newline
38. वा॒त॒ना॒मानि॑ जुहोति जुहोति वातना॒मानि॑ वातना॒मानि॑ जुहोति । \newline
39. वा॒त॒ना॒मानीति॑ वात - ना॒मानि॑ । \newline
40. जु॒हो॒ति॒ वा॒युर् वा॒युर् जु॑होति जुहोति वा॒युः । \newline
41. वा॒युर् वै वै वा॒युर् वा॒युर् वै । \newline
42. वै वृष्ट्या॒ वृष्ट्या॒ वै वै वृष्ट्याः᳚ । \newline
43. वृष्ट्या॑ ईश ईशे॒ वृष्ट्या॒ वृष्ट्या॑ ईशे । \newline
44. ई॒शे॒ वा॒युं ॅवा॒यु मी॑श ईशे वा॒युम् । \newline
45. वा॒यु मे॒वैव वा॒युं ॅवा॒यु मे॒व । \newline
46. ए॒व स्वेन॒ स्वेनै॒वैव स्वेन॑ । \newline
47. स्वेन॑ भाग॒धेये॑न भाग॒धेये॑न॒ स्वेन॒ स्वेन॑ भाग॒धेये॑न । \newline
48. भा॒ग॒धेये॒नोपोप॑ भाग॒धेये॑न भाग॒धेये॒नोप॑ । \newline
49. भा॒ग॒धेये॒नेति॑ भाग - धेये॑न । \newline
50. उप॑ धावति धाव॒ त्युपोप॑ धावति । \newline
51. धा॒व॒ति॒ स स धा॑वति धावति॒ सः । \newline
52. स ए॒वैव स स ए॒व । \newline
53. ए॒वास्मा॑ अस्मा ए॒वैवास्मै᳚ । \newline
54. अ॒स्मै॒ प॒र्जन्य॑म् प॒र्जन्य॑ मस्मा अस्मै प॒र्जन्य᳚म् । \newline
55. प॒र्जन्यं॑ ॅवर्.षयति वर्.षयति प॒र्जन्य॑म् प॒र्जन्यं॑ ॅवर्.षयति । \newline
56. व॒र्॒.ष॒य॒ त्य॒ष्टा व॒ष्टौ व॑र्.षयति वर्.षय त्य॒ष्टौ । \newline
57. अ॒ष्टौ जु॑होति जुहो त्य॒ष्टा व॒ष्टौ जु॑होति । \newline

\textbf{Ghana Paata } \newline

1. मा॒रु॒त म॑स्यसि मारु॒तम् मा॑रु॒त म॑सि म॒रुता᳚म् म॒रुता॑ मसि मारु॒तम् मा॑रु॒त म॑सि म॒रुता᳚म् । \newline
2. अ॒सि॒ म॒रुता᳚म् म॒रुता॑ मस्यसि म॒रुता॒ मोज॒ ओजो॑ म॒रुता॑ मस्यसि म॒रुता॒ मोजः॑ । \newline
3. म॒रुता॒ मोज॒ ओजो॑ म॒रुता᳚म् म॒रुता॒ मोज॒ इतीत्योजो॑ म॒रुता᳚म् म॒रुता॒ मोज॒ इति॑ । \newline
4. ओज॒ इतीत्योज॒ ओज॒ इति॑ कृ॒ष्णम् कृ॒ष्ण मित्योज॒ ओज॒ इति॑ कृ॒ष्णम् । \newline
5. इति॑ कृ॒ष्णम् कृ॒ष्ण मितीति॑ कृ॒ष्णं ॅवासो॒ वासः॑ कृ॒ष्ण मितीति॑ कृ॒ष्णं ॅवासः॑ । \newline
6. कृ॒ष्णं ॅवासो॒ वासः॑ कृ॒ष्णम् कृ॒ष्णं ॅवासः॑ कृ॒ष्णतू॑षम् कृ॒ष्णतू॑षं॒ ॅवासः॑ कृ॒ष्णम् कृ॒ष्णं ॅवासः॑ कृ॒ष्णतू॑षम् । \newline
7. वासः॑ कृ॒ष्णतू॑षम् कृ॒ष्णतू॑षं॒ ॅवासो॒ वासः॑ कृ॒ष्णतू॑ष॒म् परि॒ परि॑ कृ॒ष्णतू॑षं॒ ॅवासो॒ वासः॑ कृ॒ष्णतू॑ष॒म् परि॑ । \newline
8. कृ॒ष्णतू॑ष॒म् परि॒ परि॑ कृ॒ष्णतू॑षम् कृ॒ष्णतू॑ष॒म् परि॑ धत्ते धत्ते॒ परि॑ कृ॒ष्णतू॑षम् कृ॒ष्णतू॑ष॒म् परि॑ धत्ते । \newline
9. कृ॒ष्णतू॑ष॒मिति॑ कृ॒ष्ण - तू॒ष॒म् । \newline
10. परि॑ धत्ते धत्ते॒ परि॒ परि॑ धत्त ए॒त दे॒तद् ध॑त्ते॒ परि॒ परि॑ धत्त ए॒तत् । \newline
11. ध॒त्त॒ ए॒त दे॒तद् ध॑त्ते धत्त ए॒तद् वै वा ए॒तद् ध॑त्ते धत्त ए॒तद् वै । \newline
12. ए॒तद् वै वा ए॒त दे॒तद् वै वृष्ट्यै॒ वृष्ट्यै॒ वा ए॒त दे॒तद् वै वृष्ट्यै᳚ । \newline
13. वै वृष्ट्यै॒ वृष्ट्यै॒ वै वै वृष्ट्यै॑ रू॒पꣳ रू॒पं ॅवृष्ट्यै॒ वै वै वृष्ट्यै॑ रू॒पम् । \newline
14. वृष्ट्यै॑ रू॒पꣳ रू॒पं ॅवृष्ट्यै॒ वृष्ट्यै॑ रू॒पꣳ सरू॑पः॒ सरू॑पो रू॒पं ॅवृष्ट्यै॒ वृष्ट्यै॑ रू॒पꣳ सरू॑पः । \newline
15. रू॒पꣳ सरू॑पः॒ सरू॑पो रू॒पꣳ रू॒पꣳ सरू॑प ए॒वैव सरू॑पो रू॒पꣳ रू॒पꣳ सरू॑प ए॒व । \newline
16. सरू॑प ए॒वैव सरू॑पः॒ सरू॑प ए॒व भू॒त्वा भू॒त्वैव सरू॑पः॒ सरू॑प ए॒व भू॒त्वा । \newline
17. सरू॑प॒ इति॒ स - रू॒पः॒ । \newline
18. ए॒व भू॒त्वा भू॒त्वैवैव भू॒त्वा प॒र्जन्य॑म् प॒र्जन्य॑म् भू॒त्वैवैव भू॒त्वा प॒र्जन्य᳚म् । \newline
19. भू॒त्वा प॒र्जन्य॑म् प॒र्जन्य॑म् भू॒त्वा भू॒त्वा प॒र्जन्यं॑ ॅवर्.षयति वर्.षयति प॒र्जन्य॑म् भू॒त्वा भू॒त्वा प॒र्जन्यं॑ ॅवर्.षयति । \newline
20. प॒र्जन्यं॑ ॅवर्.षयति वर्.षयति प॒र्जन्य॑म् प॒र्जन्यं॑ ॅवर्.षयति र॒मय॑त र॒मय॑त वर्.षयति प॒र्जन्य॑म् प॒र्जन्यं॑ ॅवर्.षयति र॒मय॑त । \newline
21. व॒र्॒.ष॒य॒ति॒ र॒मय॑त र॒मय॑त वर्.षयति वर्.षयति र॒मय॑त मरुतो मरुतो र॒मय॑त वर्.षयति वर्.षयति र॒मय॑त मरुतः । \newline
22. र॒मय॑त मरुतो मरुतो र॒मय॑त र॒मय॑त मरुतः श्ये॒नꣳ श्ये॒नम् म॑रुतो र॒मय॑त र॒मय॑त मरुतः श्ये॒नम् । \newline
23. म॒रु॒तः॒ श्ये॒नꣳ श्ये॒नम् म॑रुतो मरुतः श्ये॒न मा॒यिन॑ मा॒यिनꣳ॑ श्ये॒नम् म॑रुतो मरुतः श्ये॒न मा॒यिन᳚म् । \newline
24. श्ये॒न मा॒यिन॑ मा॒यिनꣳ॑ श्ये॒नꣳ श्ये॒न मा॒यिन॒ मिती त्या॒यिनꣳ॑ श्ये॒नꣳ श्ये॒न मा॒यिन॒ मिति॑ । \newline
25. आ॒यिन॒ मिती त्या॒यिन॑ मा॒यिन॒ मिति॑ पश्चाद्वा॒तम् प॑श्चाद्वा॒त मित्या॒यिन॑ मा॒यिन॒ मिति॑ पश्चाद्वा॒तम् । \newline
26. इति॑ पश्चाद्वा॒तम् प॑श्चाद्वा॒त मितीति॑ पश्चाद्वा॒तम् प्रति॒ प्रति॑ पश्चाद्वा॒त मितीति॑ पश्चाद्वा॒तम् प्रति॑ । \newline
27. प॒श्चा॒द्वा॒तम् प्रति॒ प्रति॑ पश्चाद्वा॒तम् प॑श्चाद्वा॒तम् प्रति॑ मीवति मीवति॒ प्रति॑ पश्चाद्वा॒तम् प॑श्चाद्वा॒तम् प्रति॑ मीवति । \newline
28. प॒श्चा॒द्वा॒तमिति॑ पश्चात् - वा॒तम् । \newline
29. प्रति॑ मीवति मीवति॒ प्रति॒ प्रति॑ मीवति पुरोवा॒तम् पु॑रोवा॒तम् मी॑वति॒ प्रति॒ प्रति॑ मीवति पुरोवा॒तम् । \newline
30. मी॒व॒ति॒ पु॒रो॒वा॒तम् पु॑रोवा॒तम् मी॑वति मीवति पुरोवा॒त मे॒वैव पु॑रोवा॒तम् मी॑वति मीवति पुरोवा॒त मे॒व । \newline
31. पु॒रो॒वा॒त मे॒वैव पु॑रोवा॒तम् पु॑रोवा॒त मे॒व ज॑नयति जनयत्ये॒व पु॑रोवा॒तम् पु॑रोवा॒त मे॒व ज॑नयति । \newline
32. पु॒रो॒वा॒तमिति॑ पुरः - वा॒तम् । \newline
33. ए॒व ज॑नयति जनय त्ये॒वैव ज॑नयति व॒र्॒.षस्य॑ व॒र्॒.षस्य॑ जनय त्ये॒वैव ज॑नयति व॒र्॒.षस्य॑ । \newline
34. ज॒न॒य॒ति॒ व॒र्॒.षस्य॑ व॒र्॒.षस्य॑ जनयति जनयति व॒र्॒.षस्या व॑रुद्ध्या॒ अव॑रुद्ध्यै व॒र्॒.षस्य॑ जनयति जनयति व॒र्॒.षस्या व॑रुद्ध्यै । \newline
35. व॒र्॒.षस्या व॑रुद्ध्या॒ अव॑रुद्ध्यै व॒र्॒.षस्य॑ व॒र्॒.षस्या व॑रुद्ध्यै वातना॒मानि॑ वातना॒मा न्यव॑रुद्ध्यै व॒र्॒.षस्य॑ व॒र्॒.षस्या व॑रुद्ध्यै वातना॒मानि॑ । \newline
36. अव॑रुद्ध्यै वातना॒मानि॑ वातना॒मा न्यव॑रुद्ध्या॒ अव॑रुद्ध्यै वातना॒मानि॑ जुहोति जुहोति वातना॒मा न्यव॑रुद्ध्या॒ अव॑रुद्ध्यै वातना॒मानि॑ जुहोति । \newline
37. अव॑रुद्ध्या॒ इत्यव॑ - रु॒द्ध्यै॒ । \newline
38. वा॒त॒ना॒मानि॑ जुहोति जुहोति वातना॒मानि॑ वातना॒मानि॑ जुहोति वा॒युर् वा॒युर् जु॑होति वातना॒मानि॑ वातना॒मानि॑ जुहोति वा॒युः । \newline
39. वा॒त॒ना॒मानीति॑ वात - ना॒मानि॑ । \newline
40. जु॒हो॒ति॒ वा॒युर् वा॒युर् जु॑होति जुहोति वा॒युर् वै वै वा॒युर् जु॑होति जुहोति वा॒युर् वै । \newline
41. वा॒युर् वै वै वा॒युर् वा॒युर् वै वृष्ट्या॒ वृष्ट्या॒ वै वा॒युर् वा॒युर् वै वृष्ट्याः᳚ । \newline
42. वै वृष्ट्या॒ वृष्ट्या॒ वै वै वृष्ट्या॑ ईश ईशे॒ वृष्ट्या॒ वै वै वृष्ट्या॑ ईशे । \newline
43. वृष्ट्या॑ ईश ईशे॒ वृष्ट्या॒ वृष्ट्या॑ ईशे वा॒युं ॅवा॒यु मी॑शे॒ वृष्ट्या॒ वृष्ट्या॑ ईशे वा॒युम् । \newline
44. ई॒शे॒ वा॒युं ॅवा॒यु मी॑श ईशे वा॒यु मे॒वैव वा॒यु मी॑श ईशे वा॒यु मे॒व । \newline
45. वा॒यु मे॒वैव वा॒युं ॅवा॒यु मे॒व स्वेन॒ स्वेनै॒व वा॒युं ॅवा॒यु मे॒व स्वेन॑ । \newline
46. ए॒व स्वेन॒ स्वेनै॒वैव स्वेन॑ भाग॒धेये॑न भाग॒धेये॑न॒ स्वेनै॒वैव स्वेन॑ भाग॒धेये॑न । \newline
47. स्वेन॑ भाग॒धेये॑न भाग॒धेये॑न॒ स्वेन॒ स्वेन॑ भाग॒धेये॒नोपोप॑ भाग॒धेये॑न॒ स्वेन॒ स्वेन॑ भाग॒धेये॒नोप॑ । \newline
48. भा॒ग॒धेये॒नोपोप॑ भाग॒धेये॑न भाग॒धेये॒नोप॑ धावति धाव॒त्युप॑ भाग॒धेये॑न भाग॒धेये॒नोप॑ धावति । \newline
49. भा॒ग॒धेये॒नेति॑ भाग - धेये॑न । \newline
50. उप॑ धावति धाव॒त्युपोप॑ धावति॒ स स धा॑व॒त्युपोप॑ धावति॒ सः । \newline
51. धा॒व॒ति॒ स स धा॑वति धावति॒ स ए॒वैव स धा॑वति धावति॒ स ए॒व । \newline
52. स ए॒वैव स स ए॒वास्मा॑ अस्मा ए॒व स स ए॒वास्मै᳚ । \newline
53. ए॒वास्मा॑ अस्मा ए॒वैवास्मै॑ प॒र्जन्य॑म् प॒र्जन्य॑ मस्मा ए॒वैवास्मै॑ प॒र्जन्य᳚म् । \newline
54. अ॒स्मै॒ प॒र्जन्य॑म् प॒र्जन्य॑ मस्मा अस्मै प॒र्जन्यं॑ ॅवर्.षयति वर्.षयति प॒र्जन्य॑ मस्मा अस्मै प॒र्जन्यं॑ ॅवर्.षयति । \newline
55. प॒र्जन्यं॑ ॅवर्.षयति वर्.षयति प॒र्जन्य॑म् प॒र्जन्यं॑ ॅवर्.षय त्य॒ष्टा व॒ष्टौ व॑र्.षयति प॒र्जन्य॑म् प॒र्जन्यं॑ ॅवर्.षय त्य॒ष्टौ । \newline
56. व॒र्॒.ष॒य॒ त्य॒ष्टा व॒ष्टौ व॑र्.षयति वर्.षय त्य॒ष्टौ जु॑होति जुहोत्य॒ष्टौ व॑र्.षयति वर्.षय त्य॒ष्टौ जु॑होति । \newline
57. अ॒ष्टौ जु॑होति जुहो त्य॒ष्टा व॒ष्टौ जु॑होति॒ चत॑स्र॒ श्चत॑स्रो जुहो त्य॒ष्टा व॒ष्टौ जु॑होति॒ चत॑स्रः । \newline
\pagebreak
\markright{ TS 2.4.9.2  \hfill https://www.vedavms.in \hfill}

\section{ TS 2.4.9.2 }

\textbf{TS 2.4.9.2 } \newline
\textbf{Samhita Paata} \newline

जु॑होति॒ चत॑स्रो॒ वै दिश॒श्चत॑स्रोऽवान्तरदि॒शा दि॒ग्भ्य ए॒व वृष्टिꣳ॒॒ सं प्र च्या॑वयति कृष्णाजि॒ने संॅयौ॑ति ह॒विरे॒वाक॑रन्तर्वे॒दि संॅयौ॒त्य व॑रुद्ध्यै॒ यती॑नाम॒द्यमा॑नानाꣳ शी॒र्॒.षाणि॒ परा॑ऽपत॒न्ते ख॒र्जूरा॑ अभव॒न्-तेषाꣳ॒॒ रस॑ ऊ॒र्द्ध्वो॑ऽपत॒त्-तानि॑ क॒रीरा᳚ण्य-भवन्थ् सौ॒म्यानि॒ वै क॒रीरा॑णि सौ॒म्या खलु॒ वा आहु॑ति र्दि॒वो वृष्टिं॑ च्यावयति॒ यत्क॒रीरा॑णि॒ भव॑न्ति - [  ] \newline

\textbf{Pada Paata} \newline

जु॒हो॒ति॒ । चत॑स्रः । वै । दिशः॑ । चत॑स्रः । अ॒वा॒न्त॒र॒दि॒शा इत्य॑वान्तर - दि॒शाः । दि॒ग्भ्य इति॑ दिक् - भ्यः । ए॒व । वृष्टि᳚म् । सम् । प्रेति॑ । च्या॒व॒य॒ति॒ । कृ॒ष्णा॒जि॒न इति॑ कृष्ण-अ॒जि॒ने । समिति॑ । यौ॒ति॒ । ह॒विः । ए॒व । अ॒कः॒ । अ॒न्त॒र्वे॒दीत्य॑न्तः - वे॒दि । समिति॑ । यौ॒ति॒ । अव॑रुद्ध्या॒ इत्यव॑ - रु॒द्ध्यै॒ । यती॑नाम् । अ॒द्यमा॑नानाम् । शी॒र्॒.षाणि॑ । परेति॑ । अ॒प॒त॒न्न् । ते । ख॒र्जूराः᳚ । अ॒भ॒व॒न्न् । तेषा᳚म् । रसः॑ । ऊ॒र्द्ध्वः । अ॒प॒त॒त् । तानि॑ । क॒रीरा॑णि । अ॒भ॒व॒न्न् ।  सौ॒म्यानि॑ । वै । क॒रीरा॑णि । सौ॒म्या । खलु॑ । वै । आहु॑ति॒रित्या - हु॒तिः॒ । दि॒वः । वृष्टि᳚म् । च्या॒व॒य॒ति॒ । यत् । क॒रीरा॑णि । भव॑न्ति ।  \newline


\textbf{Krama Paata} \newline

जु॒हो॒ति॒ चत॑स्रः । चत॑स्रो॒ वै । वै दिशः॑ । दिश॒श्चत॑स्रः । चत॑स्रो ऽवान्तरदि॒शाः । अ॒वा॒न्त॒र॒दि॒शा दि॒ग्भ्यः । अ॒वा॒न्त॒र॒दि॒शा इत्य॑वान्तर - दि॒शाः । दि॒ग्भ्य ए॒व । दि॒ग्भ्य इति॑ दिक् - भ्यः । ए॒व वृष्टि᳚म् । वृष्टिꣳ॒॒ सम् । सम् प्र । प्र च्या॑वयति । च्या॒व॒य॒ति॒ कृ॒ष्णा॒जि॒ने । कृ॒ष्णा॒जि॒ने सम् । कृ॒ष्णा॒जि॒न इति॑ कृष्ण - अ॒जि॒ने । सं ॅयौ॑ति । यौ॒ति॒ ह॒विः । ह॒विरे॒व । ए॒वाकः॑ । अ॒क॒र॒न्त॒र्वे॒दि । अ॒न्त॒र्वे॒दि सम् । अ॒न्त॒र्वे॒दीत्य॑न्तः - वे॒दि । सं ॅयौ॑ति । यौ॒त्यव॑रुद्ध्यै । अव॑रुद्ध्यै॒ यती॑नाम् । अव॑रुद्ध्या॒ इत्यव॑ - रु॒द्ध्यै॒ । यती॑नाम॒द्यमा॑नानाम् । अ॒द्यमा॑नानाꣳ शी॒र्.॒षाणि॑ । शी॒र्.॒षाणि॒ परा᳚ । परा॑ ऽपतन्न् । अ॒प॒त॒न् ते । ते ख॒र्जूराः᳚ । ख॒र्जूरा॑ अभवन्न् । अ॒भ॒व॒न् तेषा᳚म् । तेषाꣳ॒॒ रसः॑ । रस॑ ऊ॒र्द्ध्वः । ऊ॒र्द्ध्वो॑ऽ पतत् । अ॒प॒त॒त् तानि॑ । तानि॑ क॒रीरा॑णि । क॒रीरा᳚ण्यभवन्न् । अ॒भ॒व॒न्थ् सौ॒म्यानि॑ । सौ॒म्यानि॒ वै । वै क॒रीरा॑णि । क॒रीरा॑णि सौ॒म्या । सौ॒म्या खलु॑ । खलु॒ वै । वा आहु॑तिः । आहु॑तिर् दि॒वः । आहु॑ति॒रित्या - हु॒तिः॒ । दि॒वो वृष्टि᳚म् । वृष्टि॑म् च्यावयति । च्या॒व॒य॒ति॒ यत् । यत् क॒रीरा॑णि । क॒रीरा॑णि॒ भव॑न्ति । भव॑न्ति सौ॒म्यया᳚ \newline

\textbf{Jatai Paata} \newline

1. जु॒हो॒ति॒ चत॑स्र॒ श्चत॑स्रो जुहोति जुहोति॒ चत॑स्रः । \newline
2. चत॑स्रो॒ वै वै चत॑स्र॒ श्चत॑स्रो॒ वै । \newline
3. वै दिशो॒ दिशो॒ वै वै दिशः॑ । \newline
4. दिश॒ श्चत॑स्र॒ श्चत॑स्रो॒ दिशो॒ दिश॒श्चत॑स्रः । \newline
5. चत॑स्रो ऽवान्तरदि॒शा अ॑वान्तरदि॒शा श्चत॑स्र॒ श्चत॑स्रो ऽवान्तरदि॒शाः । \newline
6. अ॒वा॒न्त॒र॒दि॒शा दि॒ग्भ्यो दि॒ग्भ्यो॑ ऽवान्तरदि॒शा अ॑वान्तरदि॒शा दि॒ग्भ्यः । \newline
7. अ॒वा॒न्त॒र॒दि॒शा इत्य॑वान्तर - दि॒शाः । \newline
8. दि॒ग्भ्य ए॒वैव दि॒ग्भ्यो दि॒ग्भ्य ए॒व । \newline
9. दि॒ग्भ्य इति॑ दिक् - भ्यः । \newline
10. ए॒व वृष्टिं॒ ॅवृष्टि॑ मे॒वैव वृष्टि᳚म् । \newline
11. वृष्टिꣳ॒॒ सꣳ सं ॅवृष्टिं॒ ॅवृष्टिꣳ॒॒ सम् । \newline
12. सम् प्र प्र सꣳ सम् प्र । \newline
13. प्र च्या॑वयति च्यावयति॒ प्र प्र च्या॑वयति । \newline
14. च्या॒व॒य॒ति॒ कृ॒ष्णा॒जि॒ने कृ॑ष्णाजि॒ने च्या॑वयति च्यावयति कृष्णाजि॒ने । \newline
15. कृ॒ष्णा॒जि॒ने सꣳ सम् कृ॑ष्णाजि॒ने कृ॑ष्णाजि॒ने सम् । \newline
16. कृ॒ष्णा॒जि॒न इति॑ कृष्ण - अ॒जि॒ने । \newline
17. सं ॅयौ॑ति यौति॒ सꣳ सं ॅयौ॑ति । \newline
18. यौ॒ति॒ ह॒विर्. ह॒विर् यौ॑ति यौति ह॒विः । \newline
19. ह॒वि रे॒वैव ह॒विर्. ह॒वि रे॒व । \newline
20. ए॒वाक॑ रक रे॒वैवाकः॑ । \newline
21. अ॒क॒ र॒न्त॒र्वे॒ द्य॑न्तर्वे॒द्य॑क रक रन्तर्वे॒दि । \newline
22. अ॒न्त॒र्वे॒दि सꣳ स म॑न्तर्वे॒ द्य॑न्तर्वे॒दि सम् । \newline
23. अ॒न्त॒र्वे॒दीत्य॑न्तः - वे॒दि । \newline
24. सं ॅयौ॑ति यौति॒ सꣳ सं ॅयौ॑ति । \newline
25. यौ॒त्यव॑रुद्ध्या॒ अव॑रुद्ध्यै यौति यौ॒त्यव॑रुद्ध्यै । \newline
26. अव॑रुद्ध्यै॒ यती॑नां॒ ॅयती॑ना॒ मव॑रुद्ध्या॒ अव॑रुद्ध्यै॒ यती॑नाम् । \newline
27. अव॑रुद्ध्या॒ इत्यव॑ - रु॒द्ध्यै॒ । \newline
28. यती॑ना म॒द्यमा॑नाना म॒द्यमा॑नानां॒ ॅयती॑नां॒ ॅयती॑ना म॒द्यमा॑नानाम् । \newline
29. अ॒द्यमा॑नानाꣳ शी॒र्॒.षाणि॑ शी॒र्॒.षा ण्य॒द्यमा॑नाना म॒द्यमा॑नानाꣳ शी॒र्॒.षाणि॑ । \newline
30. शी॒र्॒.षाणि॒ परा॒ परा॑ शी॒र्॒.षाणि॑ शी॒र्॒.षाणि॒ परा᳚ । \newline
31. परा॑ ऽपतन् नपत॒न् परा॒ परा॑ ऽपतन्न् । \newline
32. अ॒प॒त॒न् ते ते॑ ऽपतन् नपत॒न् ते । \newline
33. ते ख॒र्जूराः᳚ ख॒र्जूरा॒ स्ते ते ख॒र्जूराः᳚ । \newline
34. ख॒र्जूरा॑ अभवन् नभवन् ख॒र्जूराः᳚ ख॒र्जूरा॑ अभवन्न् । \newline
35. अ॒भ॒व॒न् तेषा॒म् तेषा॑ मभवन् नभव॒न् तेषा᳚म् । \newline
36. तेषाꣳ॒॒ रसो॒ रस॒ स्तेषा॒म् तेषाꣳ॒॒ रसः॑ । \newline
37. रस॑ ऊ॒र्द्ध्व ऊ॒र्द्ध्वो रसो॒ रस॑ ऊ॒र्द्ध्वः । \newline
38. ऊ॒र्द्ध्वो॑ ऽपत दपत दू॒र्द्ध्व ऊ॒र्द्ध्वो॑ ऽपतत् । \newline
39. अ॒प॒त॒त् तानि॒ तान्य॑पत दपत॒त् तानि॑ । \newline
40. तानि॑ क॒रीरा॑णि क॒रीरा॑णि॒ तानि॒ तानि॑ क॒रीरा॑णि । \newline
41. क॒रीरा᳚ ण्यभवन् नभवन् क॒रीरा॑णि क॒रीरा᳚ ण्यभवन्न् । \newline
42. अ॒भ॒व॒न् थ्सौ॒म्यानि॑ सौ॒म्या न्य॑भवन् नभवन् थ्सौ॒म्यानि॑ । \newline
43. सौ॒म्यानि॒ वै वै सौ॒म्यानि॑ सौ॒म्यानि॒ वै । \newline
44. वै क॒रीरा॑णि क॒रीरा॑णि॒ वै वै क॒रीरा॑णि । \newline
45. क॒रीरा॑णि सौ॒म्या सौ॒म्या क॒रीरा॑णि क॒रीरा॑णि सौ॒म्या । \newline
46. सौ॒म्या खलु॒ खलु॑ सौ॒म्या सौ॒म्या खलु॑ । \newline
47. खलु॒ वै वै खलु॒ खलु॒ वै । \newline
48. वा आहु॑ति॒ राहु॑ति॒र् वै वा आहु॑तिः । \newline
49. आहु॑तिर् दि॒वो दि॒व आहु॑ति॒ राहु॑तिर् दि॒वः । \newline
50. आहु॑ति॒रित्या - हु॒तिः॒ । \newline
51. दि॒वो वृष्टिं॒ ॅवृष्टि॑म् दि॒वो दि॒वो वृष्टि᳚म् । \newline
52. वृष्टि॑म् च्यावयति च्यावयति॒ वृष्टिं॒ ॅवृष्टि॑म् च्यावयति । \newline
53. च्या॒व॒य॒ति॒ यद् यच् च्या॑वयति च्यावयति॒ यत् । \newline
54. यत् क॒रीरा॑णि क॒रीरा॑णि॒ यद् यत् क॒रीरा॑णि । \newline
55. क॒रीरा॑णि॒ भव॑न्ति॒ भव॑न्ति क॒रीरा॑णि क॒रीरा॑णि॒ भव॑न्ति । \newline
56. भव॑न्ति सौ॒म्यया॑ सौ॒म्यया॒ भव॑न्ति॒ भव॑न्ति सौ॒म्यया᳚ । \newline

\textbf{Ghana Paata } \newline

1. जु॒हो॒ति॒ चत॑स्र॒ श्चत॑स्रो जुहोति जुहोति॒ चत॑स्रो॒ वै वै चत॑स्रो जुहोति जुहोति॒ चत॑स्रो॒ वै । \newline
2. चत॑स्रो॒ वै वै चत॑स्र॒ श्चत॑स्रो॒ वै दिशो॒ दिशो॒ वै चत॑स्र॒ श्चत॑स्रो॒ वै दिशः॑ । \newline
3. वै दिशो॒ दिशो॒ वै वै दिश॒ श्चत॑स्र॒ श्चत॑स्रो॒ दिशो॒ वै वै दिश॒ श्चत॑स्रः । \newline
4. दिश॒ श्चत॑स्र॒ श्चत॑स्रो॒ दिशो॒ दिश॒ श्चत॑स्रो ऽवान्तरदि॒शा अ॑वान्तरदि॒शा श्चत॑स्रो॒ दिशो॒ दिश॒ श्चत॑स्रो ऽवान्तरदि॒शाः । \newline
5. चत॑स्रो ऽवान्तरदि॒शा अ॑वान्तरदि॒शा श्चत॑स्र॒ श्चत॑स्रो ऽवान्तरदि॒शा दि॒ग्भ्यो दि॒ग्भ्यो॑ ऽवान्तरदि॒शा श्चत॑स्र॒ श्चत॑स्रो ऽवान्तरदि॒शा दि॒ग्भ्यः । \newline
6. अ॒वा॒न्त॒र॒दि॒शा दि॒ग्भ्यो दि॒ग्भ्यो॑ ऽवान्तरदि॒शा अ॑वान्तरदि॒शा दि॒ग्भ्य ए॒वैव दि॒ग्भ्यो॑ ऽवान्तरदि॒शा अ॑वान्तरदि॒शा दि॒ग्भ्य ए॒व । \newline
7. अ॒वा॒न्त॒र॒दि॒शा इत्य॑वान्तर - दि॒शाः । \newline
8. दि॒ग्भ्य ए॒वैव दि॒ग्भ्यो दि॒ग्भ्य ए॒व वृष्टिं॒ ॅवृष्टि॑ मे॒व दि॒ग्भ्यो दि॒ग्भ्य ए॒व वृष्टि᳚म् । \newline
9. दि॒ग्भ्य इति॑ दिक् - भ्यः । \newline
10. ए॒व वृष्टिं॒ ॅवृष्टि॑ मे॒वैव वृष्टिꣳ॒॒ सꣳ सं ॅवृष्टि॑ मे॒वैव वृष्टिꣳ॒॒ सम् । \newline
11. वृष्टिꣳ॒॒ सꣳ सं ॅवृष्टिं॒ ॅवृष्टिꣳ॒॒ सम् प्र प्र सं ॅवृष्टिं॒ ॅवृष्टिꣳ॒॒ सम् प्र । \newline
12. सम् प्र प्र सꣳ सम् प्र च्या॑वयति च्यावयति॒ प्र सꣳ सम् प्र च्या॑वयति । \newline
13. प्र च्या॑वयति च्यावयति॒ प्र प्र च्या॑वयति कृष्णाजि॒ने कृ॑ष्णाजि॒ने च्या॑वयति॒ प्र प्र च्या॑वयति कृष्णाजि॒ने । \newline
14. च्या॒व॒य॒ति॒ कृ॒ष्णा॒जि॒ने कृ॑ष्णाजि॒ने च्या॑वयति च्यावयति कृष्णाजि॒ने सꣳ सम् कृ॑ष्णाजि॒ने च्या॑वयति च्यावयति कृष्णाजि॒ने सम् । \newline
15. कृ॒ष्णा॒जि॒ने सꣳ सम् कृ॑ष्णाजि॒ने कृ॑ष्णाजि॒ने सं ॅयौ॑ति यौति॒ सम् कृ॑ष्णाजि॒ने कृ॑ष्णाजि॒ने सं ॅयौ॑ति । \newline
16. कृ॒ष्णा॒जि॒न इति॑ कृष्ण - अ॒जि॒ने । \newline
17. सं ॅयौ॑ति यौति॒ सꣳ सं ॅयौ॑ति ह॒विर्. ह॒विर् यौ॑ति॒ सꣳ सं ॅयौ॑ति ह॒विः । \newline
18. यौ॒ति॒ ह॒विर्. ह॒विर् यौ॑ति यौति ह॒वि रे॒वैव ह॒विर् यौ॑ति यौति ह॒विरे॒व । \newline
19. ह॒वि रे॒वैव ह॒विर्. ह॒वि रे॒वाक॑ रक रे॒व ह॒विर्. ह॒वि रे॒वाकः॑ । \newline
20. ए॒वाक॑ रक रे॒वैवाक॑ रन्तर्वे॒ द्य॑न्तर्वे॒ द्य॑क रे॒वैवाक॑ रन्तर्वे॒दि । \newline
21. अ॒क॒ र॒न्त॒र्वे॒ द्य॑न्तर्वे॒ द्य॑क रक रन्तर्वे॒दि सꣳ स म॑न्तर्वे॒ द्य॑क रक रन्तर्वे॒दि सम् । \newline
22. अ॒न्त॒र्वे॒दि सꣳ स म॑न्तर्वे॒ द्य॑न्तर्वे॒दि सं ॅयौ॑ति यौति॒ स म॑न्तर्वे॒ द्य॑न्तर्वे॒दि सं ॅयौ॑ति । \newline
23. अ॒न्त॒र्वे॒दीत्य॑न्तः - वे॒दि । \newline
24. सं ॅयौ॑ति यौति॒ सꣳ सं ॅयौ॒त्यव॑रुद्ध्या॒ अव॑रुद्ध्यै यौति॒ सꣳ सं ॅयौ॒त्यव॑रुद्ध्यै । \newline
25. यौ॒त्यव॑रुद्ध्या॒ अव॑रुद्ध्यै यौति यौ॒त्यव॑रुद्ध्यै॒ यती॑नां॒ ॅयती॑ना॒ मव॑रुद्ध्यै यौति यौ॒त्यव॑रुद्ध्यै॒ यती॑नाम् । \newline
26. अव॑रुद्ध्यै॒ यती॑नां॒ ॅयती॑ना॒ मव॑रुद्ध्या॒ अव॑रुद्ध्यै॒ यती॑ना म॒द्यमा॑नाना म॒द्यमा॑नानां॒ ॅयती॑ना॒ मव॑रुद्ध्या॒ अव॑रुद्ध्यै॒ यती॑ना म॒द्यमा॑नानाम् । \newline
27. अव॑रुद्ध्या॒ इत्यव॑ - रु॒द्ध्यै॒ । \newline
28. यती॑ना म॒द्यमा॑नाना म॒द्यमा॑नानां॒ ॅयती॑नां॒ ॅयती॑ना म॒द्यमा॑नानाꣳ शी॒र्॒.षाणि॑ शी॒र्॒.षा ण्य॒ द्यमा॑नानां॒ ॅयती॑नां॒ ॅयती॑ना म॒द्यमा॑नानाꣳ शी॒र्॒.षाणि॑ । \newline
29. अ॒द्यमा॑नानाꣳ शी॒र्॒.षाणि॑ शी॒र्॒.षा ण्य॒द्यमा॑नाना म॒द्यमा॑नानाꣳ शी॒र्॒.षाणि॒ परा॒ परा॑ शी॒र्॒.षा ण्य॒द्यमा॑नाना म॒द्यमा॑नानाꣳ शी॒र्॒.षाणि॒ परा᳚ । \newline
30. शी॒र्॒.षाणि॒ परा॒ परा॑ शी॒र्॒.षाणि॑ शी॒र्॒.षाणि॒ परा॑ ऽपतन् नपत॒न् परा॑ शी॒र्॒.षाणि॑ शी॒र्॒.षाणि॒ परा॑ ऽपतन्न् । \newline
31. परा॑ ऽपतन् नपत॒न् परा॒ परा॑ ऽपत॒न् ते ते॑ ऽपत॒न् परा॒ परा॑ ऽपत॒न् ते । \newline
32. अ॒प॒त॒न् ते ते॑ ऽपतन् नपत॒न् ते ख॒र्जूराः᳚ ख॒र्जूरा॒ स्ते॑ ऽपतन् नपत॒न् ते ख॒र्जूराः᳚ । \newline
33. ते ख॒र्जूराः᳚ ख॒र्जूरा॒ स्ते ते ख॒र्जूरा॑ अभवन् नभवन् ख॒र्जूरा॒ स्ते ते ख॒र्जूरा॑ अभवन्न् । \newline
34. ख॒र्जूरा॑ अभवन् नभवन् ख॒र्जूराः᳚ ख॒र्जूरा॑ अभव॒न् तेषा॒म् तेषा॑ मभवन् ख॒र्जूराः᳚ ख॒र्जूरा॑ अभव॒न् तेषा᳚म् । \newline
35. अ॒भ॒व॒न् तेषा॒म् तेषा॑ मभवन् नभव॒न् तेषाꣳ॒॒ रसो॒ रस॒ स्तेषा॑ मभवन् नभव॒न् तेषाꣳ॒॒ रसः॑ । \newline
36. तेषाꣳ॒॒ रसो॒ रस॒ स्तेषा॒म् तेषाꣳ॒॒ रस॑ ऊ॒र्द्ध्व ऊ॒र्द्ध्वो रस॒ स्तेषा॒म् तेषाꣳ॒॒ रस॑ ऊ॒र्द्ध्वः । \newline
37. रस॑ ऊ॒र्द्ध्व ऊ॒र्द्ध्वो रसो॒ रस॑ ऊ॒र्द्ध्वो॑ ऽपत दपत दू॒र्द्ध्वो रसो॒ रस॑ ऊ॒र्द्ध्वो॑ ऽपतत् । \newline
38. ऊ॒र्द्ध्वो॑ ऽपत दपत दू॒र्द्ध्व ऊ॒र्द्ध्वो॑ ऽपत॒त् तानि॒ तान्य॑पत दू॒र्द्ध्व ऊ॒र्द्ध्वो॑ ऽपत॒त् तानि॑ । \newline
39. अ॒प॒त॒त् तानि॒ तान्य॑पत दपत॒त् तानि॑ क॒रीरा॑णि क॒रीरा॑णि॒ तान्य॑पत दपत॒त् तानि॑ क॒रीरा॑णि । \newline
40. तानि॑ क॒रीरा॑णि क॒रीरा॑णि॒ तानि॒ तानि॑ क॒रीरा᳚ ण्यभवन् नभवन् क॒रीरा॑णि॒ तानि॒ तानि॑ क॒रीरा᳚ ण्यभवन्न् । \newline
41. क॒रीरा᳚ ण्यभवन् नभवन् क॒रीरा॑णि क॒रीरा᳚ ण्यभवन् थ्सौ॒म्यानि॑ सौ॒म्या न्य॑भवन् क॒रीरा॑णि क॒री रा᳚ण्यभवन् थ्सौ॒म्यानि॑ । \newline
42. अ॒भ॒व॒न् थ्सौ॒म्यानि॑ सौ॒म्या न्य॑भवन् नभवन् थ्सौ॒म्यानि॒ वै वै सौ॒म्या न्य॑भवन् नभवन् थ्सौ॒म्यानि॒ वै । \newline
43. सौ॒म्यानि॒ वै वै सौ॒म्यानि॑ सौ॒म्यानि॒ वै क॒रीरा॑णि क॒रीरा॑णि॒ वै सौ॒म्यानि॑ सौ॒म्यानि॒ वै क॒रीरा॑णि । \newline
44. वै क॒रीरा॑णि क॒रीरा॑णि॒ वै वै क॒रीरा॑णि सौ॒म्या सौ॒म्या क॒रीरा॑णि॒ वै वै क॒रीरा॑णि सौ॒म्या । \newline
45. क॒रीरा॑णि सौ॒म्या सौ॒म्या क॒रीरा॑णि क॒रीरा॑णि सौ॒म्या खलु॒ खलु॑ सौ॒म्या क॒रीरा॑णि क॒रीरा॑णि सौ॒म्या खलु॑ । \newline
46. सौ॒म्या खलु॒ खलु॑ सौ॒म्या सौ॒म्या खलु॒ वै वै खलु॑ सौ॒म्या सौ॒म्या खलु॒ वै । \newline
47. खलु॒ वै वै खलु॒ खलु॒ वा आहु॑ति॒ राहु॑ति॒र् वै खलु॒ खलु॒ वा आहु॑तिः । \newline
48. वा आहु॑ति॒ राहु॑ति॒र् वै वा आहु॑तिर् दि॒वो दि॒व आहु॑ति॒र् वै वा आहु॑तिर् दि॒वः । \newline
49. आहु॑तिर् दि॒वो दि॒व आहु॑ति॒ राहु॑तिर् दि॒वो वृष्टिं॒ ॅवृष्टि॑म् दि॒व आहु॑ति॒ राहु॑तिर् दि॒वो वृष्टि᳚म् । \newline
50. आहु॑ति॒रित्या - हु॒तिः॒ । \newline
51. दि॒वो वृष्टिं॒ ॅवृष्टि॑म् दि॒वो दि॒वो वृष्टि॑म् च्यावयति च्यावयति॒ वृष्टि॑म् दि॒वो दि॒वो वृष्टि॑म् च्यावयति । \newline
52. वृष्टि॑म् च्यावयति च्यावयति॒ वृष्टिं॒ ॅवृष्टि॑म् च्यावयति॒ यद् यच् च्या॑वयति॒ वृष्टिं॒ ॅवृष्टि॑म् च्यावयति॒ यत् । \newline
53. च्या॒व॒य॒ति॒ यद् यच् च्या॑वयति च्यावयति॒ यत् क॒रीरा॑णि क॒रीरा॑णि॒ यच् च्या॑वयति च्यावयति॒ यत् क॒रीरा॑णि । \newline
54. यत् क॒रीरा॑णि क॒रीरा॑णि॒ यद् यत् क॒रीरा॑णि॒ भव॑न्ति॒ भव॑न्ति क॒रीरा॑णि॒ यद् यत् क॒रीरा॑णि॒ भव॑न्ति । \newline
55. क॒रीरा॑णि॒ भव॑न्ति॒ भव॑न्ति क॒रीरा॑णि क॒रीरा॑णि॒ भव॑न्ति सौ॒म्यया॑ सौ॒म्यया॒ भव॑न्ति क॒रीरा॑णि क॒रीरा॑णि॒ भव॑न्ति सौ॒म्यया᳚ । \newline
56. भव॑न्ति सौ॒म्यया॑ सौ॒म्यया॒ भव॑न्ति॒ भव॑न्ति सौ॒म्ययै॒वैव सौ॒म्यया॒ भव॑न्ति॒ भव॑न्ति सौ॒म्ययै॒व । \newline
\pagebreak
\markright{ TS 2.4.9.3  \hfill https://www.vedavms.in \hfill}

\section{ TS 2.4.9.3 }

\textbf{TS 2.4.9.3 } \newline
\textbf{Samhita Paata} \newline

सौ॒म्ययै॒वाऽऽ*हु॑त्या दि॒वो वृष्टि॒मव॑ रुन्धे॒ मधु॑षा॒ सं ॅयौ᳚त्य॒पां ॅवा ए॒ष ओष॑धीनाꣳ॒॒ रसो॒ यन्मद्ध्व॒द्भ्य ए॒वौष॑धीभ्यो वर्.ष॒त्यथो॑ अ॒द्भ्य ए॒वौष॑धीभ्यो॒ वृष्टि॒निं न॑यति॒ मान्दा॒ वाशा॒ इति॒ संॅयौ॑ति नाम॒धेयै॑रे॒वैना॒ अच्छै॒त्यथो॒ यथा᳚ ब्रू॒यादसा॒ वेहीत्ये॒वमे॒वैना॑ नाम॒धेयै॒रा - [  ] \newline

\textbf{Pada Paata} \newline

सौ॒म्यया᳚ । ए॒व । आहु॒त्येत्या - हु॒त्या॒ । दि॒वः । वृष्टि᳚म् । अवेति॑ । रु॒न्धे॒ । मधु॑षा । समिति॑ । यौ॒ति॒ । अ॒पाम् । वै । ए॒षः । ओष॑धीनाम् । रसः॑ । यत् । मधु॑ । अ॒द्भ्य इत्य॑त् - भ्यः । ए॒व । ओष॑धीभ्य॒ इत्योष॑धि - भ्यः॒ । व॒र्.॒ष॒ति॒ । अथो॒ इति॑ । अ॒द्भ्य इत्य॑त् - भ्यः । ए॒व । ओष॑धीभ्य॒ इत्योष॑धि - भ्यः॒ । वृष्टि᳚म् । नीति॑ । न॒य॒ति॒ । मान्दाः᳚ । वाशाः᳚ । इति॑ । समिति॑ । यौ॒ति॒ । ना॒म॒धेयै॒रिति॑ नाम-धेयैः᳚ । ए॒व । ए॒नाः॒ । अच्छ॑ । ए॒ति॒ । अथो॒ इति॑ । यथा᳚ । ब्रू॒यात् । असौ᳚ ।   एति॑ । इ॒हि॒ । इति॑ । ए॒वम् । ए॒व । ए॒नाः॒ । ना॒म॒धेयै॒रिति॑ नाम-धेयैः᳚ । एति॑ ।  \newline


\textbf{Krama Paata} \newline

सौ॒म्ययै॒व । ए॒वाहु॑त्या । आहु॑त्या दि॒वः । आहु॒त्येत्या - हु॒त्या॒ । दि॒वो वृष्टि᳚म् । वृष्टि॒मव॑ । अव॑ रुन्धे । रु॒न्धे॒ मधु॑षा । मधु॑षा॒ सम् । सं ॅयौ॑ति । यौ॒त्य॒पाम् । अ॒पां ॅवै । वा ए॒षः । ए॒ष ओष॑धीनाम् । ओष॑धीनाꣳ॒॒ रसः॑ । रसो॒ यत् । यन् मधु॑ । मध्व॒द्भ्यः । अ॒द्भ्य ए॒व । अ॒द्भ्य इत्य॑त् - भ्यः । ए॒वौष॑धीभ्यः । ओष॑धीभ्यो वर्.षति । ओष॑धीभ्य॒ इत्योष॑धि - भ्यः॒ । व॒र्॒.ष॒त्यथो᳚ । अथो॑ अ॒द्भ्यः । अथो॒ इत्यथो᳚ । अ॒द्भ्य ए॒व । अ॒द्भ्य इत्य॑त् - भ्यः । ए॒वौष॑धीभ्यः । ओष॑धीभ्यो॒ वृष्टि᳚म् । ओष॑धीभ्य॒ इत्योष॑धि - भ्यः॒ । वृष्टि॒म् नि । नि न॑यति । न॒य॒ति॒ मान्दाः᳚ । मान्दा॒ वाशाः᳚ । वाशा॒ इति॑ । इति॒ सम् । सं ॅयौ॑ति । यौ॒ति॒ ना॒म॒धेयैः᳚ । ना॒म॒धेयै॑रे॒व । ना॒म॒धेयै॒रिति॑ नाम - धेयैः᳚ । ए॒वैनाः᳚ । ए॒ना॒ अच्छ॑ । अच्छै॑ति । ए॒त्यथो᳚ । अथो॒ यथा᳚ । अथो॒ इत्यथो᳚ । यथा᳚ ब्रू॒यात् । ब्रू॒यादसौ᳚ । असा॒वा । एहि॑ । इ॒हीति॑ । इत्ये॒वम् । ए॒वमे॒व । ए॒वैनाः᳚ । ए॒ना॒ ना॒म॒धेयैः᳚ ( ) । ना॒म॒धेयै॒रा । ना॒म॒धेयै॒रिति॑ नाम - धेयैः᳚ । आ च्या॑वयति \newline

\textbf{Jatai Paata} \newline

1. सौ॒म्ययै॒वैव सौ॒म्यया॑ सौ॒म्ययै॒व । \newline
2. ए॒वाहु॒त्या ऽऽहु॑त्यै॒वै वाहु॑त्या । \newline
3. आहु॑त्या दि॒वो दि॒व आहु॒त्या ऽऽहु॑त्या दि॒वः । \newline
4. आहु॒त्येत्या - हु॒त्या॒ । \newline
5. दि॒वो वृष्टिं॒ ॅवृष्टि॑म् दि॒वो दि॒वो वृष्टि᳚म् । \newline
6. वृष्टि॒ मवाव॒ वृष्टिं॒ ॅवृष्टि॒ मव॑ । \newline
7. अव॑ रुन्धे रु॒न्धे ऽवाव॑ रुन्धे । \newline
8. रु॒न्धे॒ मधु॑षा॒ मधु॑षा रुन्धे रुन्धे॒ मधु॑षा । \newline
9. मधु॑षा॒ सꣳ सम् मधु॑षा॒ मधु॑षा॒ सम् । \newline
10. सं ॅयौ॑ति यौति॒ सꣳ सं ॅयौ॑ति । \newline
11. यौ॒त्य॒पा म॒पां ॅयौ॑ति यौत्य॒पाम् । \newline
12. अ॒पां ॅवै वा अ॒पा म॒पां ॅवै । \newline
13. वा ए॒ष ए॒ष वै वा ए॒षः । \newline
14. ए॒ष ओष॑धीना॒ मोष॑धीना मे॒ष ए॒ष ओष॑धीनाम् । \newline
15. ओष॑धीनाꣳ॒॒ रसो॒ रस॒ ओष॑धीना॒ मोष॑धीनाꣳ॒॒ रसः॑ । \newline
16. रसो॒ यद् यद् रसो॒ रसो॒ यत् । \newline
17. यन् मधु॒ मधु॒ यद् यन् मधु॑ । \newline
18. मध्व॒द्भ्यो᳚ ऽद्भ्यो मधु॒ मध्व॒द्भ्यः । \newline
19. अ॒द्भ्य ए॒वैवाद्भ्यो᳚ ऽद्भ्य ए॒व । \newline
20. अ॒द्भ्य इत्य॑त् - भ्यः । \newline
21. ए॒वौष॑धीभ्य॒ ओष॑धीभ्य ए॒वैवौष॑धीभ्यः । \newline
22. ओष॑धीभ्यो वर्.षति वर्.ष॒ त्योष॑धीभ्य॒ ओष॑धीभ्यो वर्.षति । \newline
23. ओष॑धीभ्य॒ इत्योष॑धि - भ्यः॒ । \newline
24. व॒र्॒.ष॒ त्यथो॒ अथो॑ वर्.षति वर्.ष॒ त्यथो᳚ । \newline
25. अथो॑ अ॒द्भ्यो᳚ ऽद्भ्यो ऽथो॒ अथो॑ अ॒द्भ्यः । \newline
26. अथो॒ इत्यथो᳚ । \newline
27. अ॒द्भ्य ए॒वैवाद्भ्यो᳚ ऽद्भ्य ए॒व । \newline
28. अ॒द्भ्य इत्य॑त् - भ्यः । \newline
29. ए॒वौष॑धीभ्य॒ ओष॑धीभ्य ए॒वैवौष॑धीभ्यः । \newline
30. ओष॑धीभ्यो॒ वृष्टिं॒ ॅवृष्टि॒ मोष॑धीभ्य॒ ओष॑धीभ्यो॒ वृष्टि᳚म् । \newline
31. ओष॑धीभ्य॒ इत्योष॑धि - भ्यः॒ । \newline
32. वृष्टि॒म् नि नि वृष्टिं॒ ॅवृष्टि॒म् नि । \newline
33. नि न॑यति नयति॒ नि नि न॑यति । \newline
34. न॒य॒ति॒ मान्दा॒ मान्दा॑ नयति नयति॒ मान्दाः᳚ । \newline
35. मान्दा॒ वाशा॒ वाशा॒ मान्दा॒ मान्दा॒ वाशाः᳚ । \newline
36. वाशा॒ इतीति॒ वाशा॒ वाशा॒ इति॑ । \newline
37. इति॒ सꣳ स मितीति॒ सम् । \newline
38. सं ॅयौ॑ति यौति॒ सꣳ सं ॅयौ॑ति । \newline
39. यौ॒ति॒ ना॒म॒धेयै᳚र् नाम॒धेयै᳚र् यौति यौति नाम॒धेयैः᳚ । \newline
40. ना॒म॒धेयै॑ रे॒वैव ना॑म॒धेयै᳚र् नाम॒धेयै॑ रे॒व । \newline
41. ना॒म॒धेयै॒रिति॑ नाम - धेयैः᳚ । \newline
42. ए॒वैना॑ एना ए॒वैवैनाः᳚ । \newline
43. ए॒ना॒ अच्छाच्छै॑ना एना॒ अच्छ॑ । \newline
44. अच्छै᳚ त्ये॒त्यच्छा च्छै॑ति । \newline
45. ए॒त्यथो॒ अथो॑ एत्ये॒त्यथो᳚ । \newline
46. अथो॒ यथा॒ यथा ऽथो॒ अथो॒ यथा᳚ । \newline
47. अथो॒ इत्यथो᳚ । \newline
48. यथा᳚ ब्रू॒याद् ब्रू॒याद् यथा॒ यथा᳚ ब्रू॒यात् । \newline
49. ब्रू॒या दसा॒ वसौ᳚ ब्रू॒याद् ब्रू॒या दसौ᳚ । \newline
50. असा॒ वा ऽसा॒ वसा॒ वा । \newline
51. एही॒ह्येहि॑ । \newline
52. इ॒हीतीती॑ही॒हीति॑ । \newline
53. इत्ये॒व मे॒व मिती त्ये॒वम् । \newline
54. ए॒व मे॒वैवैव मे॒व मे॒व । \newline
55. ए॒वैना॑ एना ए॒वैवैनाः᳚ । \newline
56. ए॒ना॒ ना॒म॒धेयै᳚र् नाम॒धेयै॑ रेना एना नाम॒धेयैः᳚ । \newline
57. ना॒म॒धेयै॒रा ना॑म॒धेयै᳚र् नाम॒धेयै॒रा । \newline
58. ना॒म॒धेयै॒रिति॑ नाम - धेयैः᳚ । \newline
59. आ च्या॑वयति च्यावय॒त्या च्या॑वयति । \newline

\textbf{Ghana Paata } \newline

1. सौ॒म्ययै॒वैव सौ॒म्यया॑ सौ॒म्ययै॒ वाहु॒त्या ऽऽहु॑त्यै॒व सौ॒म्यया॑ सौ॒म्ययै॒ वाहु॑त्या । \newline
2. ए॒वाहु॒त्या ऽऽहु॑त्यै॒वै वाहु॑त्या दि॒वो दि॒व आहु॑त्यै॒वै वाहु॑त्या दि॒वः । \newline
3. आहु॑त्या दि॒वो दि॒व आहु॒त्या ऽऽहु॑त्या दि॒वो वृष्टिं॒ ॅवृष्टि॑म् दि॒व आहु॒त्या ऽऽहु॑त्या दि॒वो वृष्टि᳚म् । \newline
4. आहु॒त्येत्या - हु॒त्या॒ । \newline
5. दि॒वो वृष्टिं॒ ॅवृष्टि॑म् दि॒वो दि॒वो वृष्टि॒ मवाव॒ वृष्टि॑म् दि॒वो दि॒वो वृष्टि॒ मव॑ । \newline
6. वृष्टि॒ मवाव॒ वृष्टिं॒ ॅवृष्टि॒ मव॑ रुन्धे रु॒न्धे ऽव॒ वृष्टिं॒ ॅवृष्टि॒ मव॑ रुन्धे । \newline
7. अव॑ रुन्धे रु॒न्धे ऽवाव॑ रुन्धे॒ मधु॑षा॒ मधु॑षा रु॒न्धे ऽवाव॑ रुन्धे॒ मधु॑षा । \newline
8. रु॒न्धे॒ मधु॑षा॒ मधु॑षा रुन्धे रुन्धे॒ मधु॑षा॒ सꣳ सम् मधु॑षा रुन्धे रुन्धे॒ मधु॑षा॒ सम् । \newline
9. मधु॑षा॒ सꣳ सम् मधु॑षा॒ मधु॑षा॒ सं ॅयौ॑ति यौति॒ सम् मधु॑षा॒ मधु॑षा॒ सं ॅयौ॑ति । \newline
10. सं ॅयौ॑ति यौति॒ सꣳ सं ॅयौ᳚त्य॒पा म॒पां ॅयौ॑ति॒ सꣳ सं ॅयौ᳚त्य॒पाम् । \newline
11. यौ॒त्य॒पा म॒पां ॅयौ॑ति यौत्य॒पां ॅवै वा अ॒पां ॅयौ॑ति यौत्य॒पां ॅवै । \newline
12. अ॒पां ॅवै वा अ॒पा म॒पां ॅवा ए॒ष ए॒ष वा अ॒पा म॒पां ॅवा ए॒षः । \newline
13. वा ए॒ष ए॒ष वै वा ए॒ष ओष॑धीना॒ मोष॑धीना मे॒ष वै वा ए॒ष ओष॑धीनाम् । \newline
14. ए॒ष ओष॑धीना॒ मोष॑धीना मे॒ष ए॒ष ओष॑धीनाꣳ॒॒ रसो॒ रस॒ ओष॑धीना मे॒ष ए॒ष ओष॑धीनाꣳ॒॒ रसः॑ । \newline
15. ओष॑धीनाꣳ॒॒ रसो॒ रस॒ ओष॑धीना॒ मोष॑धीनाꣳ॒॒ रसो॒ यद् यद् रस॒ ओष॑धीना॒ मोष॑धीनाꣳ॒॒ रसो॒ यत् । \newline
16. रसो॒ यद् यद् रसो॒ रसो॒ यन् मधु॒ मधु॒ यद् रसो॒ रसो॒ यन् मधु॑ । \newline
17. यन् मधु॒ मधु॒ यद् यन् मध्व॒द्भ्यो᳚ ऽद्भ्यो मधु॒ यद् यन् मध्व॒द्भ्यः । \newline
18. मध्व॒द्भ्यो᳚ ऽद्भ्यो मधु॒ मध्व॒द्भ्य ए॒वैवाद्भ्यो मधु॒ मध्व॒द्भ्य ए॒व । \newline
19. अ॒द्भ्य ए॒वैवाद्भ्यो᳚ ऽद्भ्य ए॒वौष॑धीभ्य॒ ओष॑धीभ्य ए॒वाद्भ्यो᳚ ऽद्भ्य ए॒वौष॑धीभ्यः । \newline
20. अ॒द्भ्य इत्य॑त् - भ्यः । \newline
21. ए॒वौष॑धीभ्य॒ ओष॑धीभ्य ए॒वै वौष॑धीभ्यो वर्.षति वर्.ष॒ त्योष॑धीभ्य ए॒वै वौष॑धीभ्यो वर्.षति । \newline
22. ओष॑धीभ्यो वर्.षति वर्.ष॒ त्योष॑धीभ्य॒ ओष॑धीभ्यो वर्.ष॒ त्यथो॒ अथो॑ वर्.ष॒ त्योष॑धीभ्य॒ ओष॑धीभ्यो वर्.ष॒ त्यथो᳚ । \newline
23. ओष॑धीभ्य॒ इत्योष॑धि - भ्यः॒ । \newline
24. व॒र्॒.ष॒ त्यथो॒ अथो॑ वर्.षति वर्.ष॒ त्यथो॑ अ॒द्भ्यो᳚ ऽद्भ्यो ऽथो॑ वर्.षति वर्.ष॒ त्यथो॑ अ॒द्भ्यः । \newline
25. अथो॑ अ॒द्भ्यो᳚ ऽद्भ्यो ऽथो॒ अथो॑ अ॒द्भ्य ए॒वैवाद्भ्यो ऽथो॒ अथो॑ अ॒द्भ्य ए॒व । \newline
26. अथो॒ इत्यथो᳚ । \newline
27. अ॒द्भ्य ए॒वैवाद्भ्यो᳚ ऽद्भ्य ए॒वौष॑धीभ्य॒ ओष॑धीभ्य ए॒वाद्भ्यो᳚ ऽद्भ्य ए॒वौष॑धीभ्यः । \newline
28. अ॒द्भ्य इत्य॑त् - भ्यः । \newline
29. ए॒वौष॑धीभ्य॒ ओष॑धीभ्य ए॒वैवौष॑धीभ्यो॒ वृष्टिं॒ ॅवृष्टि॒ मोष॑धीभ्य ए॒वैवौष॑धीभ्यो॒ वृष्टि᳚म् । \newline
30. ओष॑धीभ्यो॒ वृष्टिं॒ ॅवृष्टि॒ मोष॑धीभ्य॒ ओष॑धीभ्यो॒ वृष्टि॒म् नि नि वृष्टि॒ मोष॑धीभ्य॒ ओष॑धीभ्यो॒ वृष्टि॒म् नि । \newline
31. ओष॑धीभ्य॒ इत्योष॑धि - भ्यः॒ । \newline
32. वृष्टि॒म् नि नि वृष्टिं॒ ॅवृष्टि॒म् नि न॑यति नयति॒ नि वृष्टिं॒ ॅवृष्टि॒म् नि न॑यति । \newline
33. नि न॑यति नयति॒ नि नि न॑यति॒ मान्दा॒ मान्दा॑ नयति॒ नि नि न॑यति॒ मान्दाः᳚ । \newline
34. न॒य॒ति॒ मान्दा॒ मान्दा॑ नयति नयति॒ मान्दा॒ वाशा॒ वाशा॒ मान्दा॑ नयति नयति॒ मान्दा॒ वाशाः᳚ । \newline
35. मान्दा॒ वाशा॒ वाशा॒ मान्दा॒ मान्दा॒ वाशा॒ इतीति॒ वाशा॒ मान्दा॒ मान्दा॒ वाशा॒ इति॑ । \newline
36. वाशा॒ इतीति॒ वाशा॒ वाशा॒ इति॒ सꣳ स मिति॒ वाशा॒ वाशा॒ इति॒ सम् । \newline
37. इति॒ सꣳ स मितीति॒ सं ॅयौ॑ति यौति॒ स मितीति॒ सं ॅयौ॑ति । \newline
38. सं ॅयौ॑ति यौति॒ सꣳ सं ॅयौ॑ति नाम॒धेयै᳚र् नाम॒धेयै᳚र् यौति॒ सꣳ सं ॅयौ॑ति नाम॒धेयैः᳚ । \newline
39. यौ॒ति॒ ना॒म॒धेयै᳚र् नाम॒धेयै᳚र् यौति यौति नाम॒धेयै॑ रे॒वैव ना॑म॒धेयै᳚र् यौति यौति नाम॒धेयै॑ रे॒व । \newline
40. ना॒म॒धेयै॑ रे॒वैव ना॑म॒धेयै᳚र् नाम॒धेयै॑ रे॒वैना॑ एना ए॒व ना॑म॒धेयै᳚र् नाम॒धेयै॑ रे॒वैनाः᳚ । \newline
41. ना॒म॒धेयै॒रिति॑ नाम - धेयैः᳚ । \newline
42. ए॒वैना॑ एना ए॒वैवैना॒ अच्छाच्छै॑ना ए॒वैवैना॒ अच्छ॑ । \newline
43. ए॒ना॒ अच्छाच्छै॑ना एना॒ अच्छै᳚ त्ये॒ त्यच्छै॑ना एना॒ अच्छै॑ति । \newline
44. अच्छै᳚ त्ये॒ त्यच्छा च्छै॒त्यथो॒ अथो॑ ए॒त्यच्छा च्छै॒त्यथो᳚ । \newline
45. ए॒त्यथो॒ अथो॑ एत्ये॒ त्यथो॒ यथा॒ यथा ऽथो॑ एत्ये॒ त्यथो॒ यथा᳚ । \newline
46. अथो॒ यथा॒ यथा ऽथो॒ अथो॒ यथा᳚ ब्रू॒याद् ब्रू॒याद् यथा ऽथो॒ अथो॒ यथा᳚ ब्रू॒यात् । \newline
47. अथो॒ इत्यथो᳚ । \newline
48. यथा᳚ ब्रू॒याद् ब्रू॒याद् यथा॒ यथा᳚ ब्रू॒या दसा॒ वसौ᳚ ब्रू॒याद् यथा॒ यथा᳚ ब्रू॒या दसौ᳚ । \newline
49. ब्रू॒या दसा॒ वसौ᳚ ब्रू॒याद् ब्रू॒या दसा॒ वा ऽसौ᳚ ब्रू॒याद् ब्रू॒या दसा॒ वा । \newline
50. असा॒ वा ऽसा॒ वसा॒ वेही॒ह्या ऽसा॒ वसा॒ वेहि॑ । \newline
51. एही॒ ह्येहीतीती॒ ह्येहीति॑ । \newline
52. इ॒हीतीती॑ ही॒ही त्ये॒व मे॒व मिती॑ही॒ही त्ये॒वम् । \newline
53. इत्ये॒व मे॒व मिती त्ये॒व मे॒वैवैव मिती त्ये॒व मे॒व । \newline
54. ए॒व मे॒वैवैव मे॒व मे॒वैना॑ एना ए॒वैव मे॒व मे॒वैनाः᳚ । \newline
55. ए॒वैना॑ एना ए॒वैवैना॑ नाम॒धेयै᳚र् नाम॒धेयै॑ रेना ए॒वैवैना॑ नाम॒धेयैः᳚ । \newline
56. ए॒ना॒ ना॒म॒धेयै᳚र् नाम॒धेयै॑ रेना एना नाम॒धेयै॒रा ना॑म॒धेयै॑ रेना एना नाम॒धेयै॒रा । \newline
57. ना॒म॒धेयै॒रा ना॑म॒धेयै᳚र् नाम॒धेयै॒रा च्या॑वयति च्यावय॒त्या ना॑म॒धेयै᳚र् नाम॒धेयै॒रा च्या॑वयति । \newline
58. ना॒म॒धेयै॒रिति॑ नाम - धेयैः᳚ । \newline
59. आ च्या॑वयति च्यावय॒त्या च्या॑वयति॒ वृष्णो॒ वृष्ण॑ श्च्यावय॒त्या च्या॑वयति॒ वृष्णः॑ । \newline
\pagebreak
\markright{ TS 2.4.9.4  \hfill https://www.vedavms.in \hfill}

\section{ TS 2.4.9.4 }

\textbf{TS 2.4.9.4 } \newline
\textbf{Samhita Paata} \newline

च्या॑वयति॒ वृष्णो॒ अश्व॑स्य स॒न्दान॑मसि॒ वृष्‌ट्यै॒ त्वोप॑ नह्या॒मीत्या॑ह॒ वृषा॒ वा अश्वो॒ वृषा॑ प॒र्जन्यः॑ कृ॒ष्ण इ॑व॒ खलु॒ वै भू॒त्वा व॑र्.षति रू॒पेणै॒वैनꣳ॒॒ सम॑र्द्धयति व॒र्॒.षस्या व॑रुद्ध्यै ॥ \newline

\textbf{Pada Paata} \newline

च्या॒व॒य॒ति॒ । वृष्णः॑ । अश्व॑स्य । स॒दांन॒मिति॑ सं - दान᳚म् । अ॒सि॒ । वृष्ट्यै᳚ । त्वा॒ । उपेति॑ । न॒ह्या॒मि॒ । इति॑ । आ॒ह॒ । वृषा᳚ । वै । अश्वः॑ । वृषा᳚ । प॒र्जन्यः॑ । कृ॒ष्णः । इ॒व॒ । खलु॑ । वै । भू॒त्वा । व॒र्.॒ष॒ति॒ । रू॒पेण॑ । ए॒व । ए॒न॒म् । समिति॑ । अ॒र्द्ध॒य॒ति॒ । व॒र्॒.षस्य॑ । अव॑रुद्ध्य॒ इत्यव॑ - रु॒द्ध्यै॒ ॥  \newline


\textbf{Krama Paata} \newline

च्या॒व॒य॒ति॒ वृष्णः॑ । वृष्णो॒ अश्व॑स्य । अश्व॑स्य स॒न्दान᳚म् । स॒न्दान॑मसि । स॒न्दान॒मिति॑ सं - दान᳚म् ॥ अ॒सि॒ वृष्ट्यै᳚ । वृष्ट्यै᳚ त्वा । त्वोप॑ । उप॑ नह्यामि । न॒ह्या॒मीति॑ । इत्या॑ह । आ॒ह॒ वृषा᳚ । वृषा॒ वै । वा अश्वः॑ । अश्वो॒ वृषा᳚ । वृषा॑ प॒र्जन्यः॑ । प॒र्जन्यः॑ कृ॒ष्णः । कृ॒ष्ण इ॑व । इ॒व॒ खलु॑ । खलु॒ वै । वै भू॒त्वा । भू॒त्वा व॑र्.षति । व॒र्॒.ष॒ति॒ रू॒पेण॑ । रू॒पेणै॒व । ए॒वैन᳚म् । ए॒नꣳ॒॒ सम् । सम॑र्द्धयति । अ॒र्द्ध॒य॒ति॒ व॒र्॒.षस्य॑ । व॒र्॒.षस्याव॑रुद्ध्यै । अव॑रुद्ध्या॒ इत्यव॑ - रु॒द्ध्यै॒ । \newline

\textbf{Jatai Paata} \newline

1. च्या॒व॒य॒ति॒ वृष्णो॒ वृष्ण॑ श्च्यावयति च्यावयति॒ वृष्णः॑ । \newline
2. वृष्णो॒ अश्व॒स्या श्व॑स्य॒ वृष्णो॒ वृष्णो॒ अश्व॑स्य । \newline
3. अश्व॑स्य स॒न्दानꣳ॑ स॒न्दान॒ मश्व॒स्या श्व॑स्य स॒न्दान᳚म् । \newline
4. स॒न्दान॑ मस्यसि स॒न्दानꣳ॑ स॒न्दान॑ मसि । \newline
5. स॒न्दान॒मिति॑ सं - दान᳚म् । \newline
6. अ॒सि॒ वृष्ट्यै॒ वृष्ट्या॑ अस्यसि॒ वृष्ट्यै᳚ । \newline
7. वृष्ट्यै᳚ त्वा त्वा॒ वृष्ट्यै॒ वृष्ट्यै᳚ त्वा । \newline
8. त्वोपोप॑ त्वा॒ त्वोप॑ । \newline
9. उप॑ नह्यामि नह्या॒ म्युपोप॑ नह्यामि । \newline
10. न॒ह्या॒मीतीति॑ नह्यामि नह्या॒मीति॑ । \newline
11. इत्या॑हा॒हे तीत्या॑ह । \newline
12. आ॒ह॒ वृषा॒ वृषा॑ ऽऽहाह॒ वृषा᳚ । \newline
13. वृषा॒ वै वै वृषा॒ वृषा॒ वै । \newline
14. वा अश्वो ऽश्वो॒ वै वा अश्वः॑ । \newline
15. अश्वो॒ वृषा॒ वृषा ऽश्वो ऽश्वो॒ वृषा᳚ । \newline
16. वृषा॑ प॒र्जन्यः॑ प॒र्जन्यो॒ वृषा॒ वृषा॑ प॒र्जन्यः॑ । \newline
17. प॒र्जन्यः॑ कृ॒ष्णः कृ॒ष्णः प॒र्जन्यः॑ प॒र्जन्यः॑ कृ॒ष्णः । \newline
18. कृ॒ष्ण इ॑वे व कृ॒ष्णः कृ॒ष्ण इ॑व । \newline
19. इ॒व॒ खलु॒ खल्वि॑वे व॒ खलु॑ । \newline
20. खलु॒ वै वै खलु॒ खलु॒ वै । \newline
21. वै भू॒त्वा भू॒त्वा वै वै भू॒त्वा । \newline
22. भू॒त्वा व॑र्.षति वर्.षति भू॒त्वा भू॒त्वा व॑र्.षति । \newline
23. व॒र्॒.ष॒ति॒ रू॒पेण॑ रू॒पेण॑ वर्.षति वर्.षति रू॒पेण॑ । \newline
24. रू॒पेणै॒वैव रू॒पेण॑ रू॒पेणै॒व । \newline
25. ए॒वैन॑ मेन मे॒वैवैन᳚म् । \newline
26. ए॒नꣳ॒॒ सꣳ स मे॑न मेनꣳ॒॒ सम् । \newline
27. स म॑र्द्धय त्यर्द्धयति॒ सꣳ स म॑र्द्धयति । \newline
28. अ॒र्द्ध॒य॒ति॒ व॒र्॒.षस्य॑ व॒र्॒.षस्या᳚ र्द्धय त्यर्द्धयति व॒र्॒.षस्य॑ । \newline
29. व॒र्॒.षस्या व॑रुद्ध्या॒ अव॑रुद्ध्यै व॒र्॒.षस्य॑ व॒र्॒.षस्या व॑रुद्ध्यै । \newline
30. अव॑रुद्ध्या॒ इत्यव॑ - रु॒द्ध्यै॒ । \newline

\textbf{Ghana Paata } \newline

1. च्या॒व॒य॒ति॒ वृष्णो॒ वृष्ण॑ श्च्यावयति च्यावयति॒ वृष्णो॒ अश्व॒स्याश्व॑स्य॒ वृष्ण॑ श्च्यावयति च्यावयति॒ वृष्णो॒ अश्व॑स्य । \newline
2. वृष्णो॒ अश्व॒स्याश्व॑स्य॒ वृष्णो॒ वृष्णो॒ अश्व॑स्य स॒न्दानꣳ॑ स॒न्दान॒ मश्व॑स्य॒ वृष्णो॒ वृष्णो॒ अश्व॑स्य स॒न्दान᳚म् । \newline
3. अश्व॑स्य स॒न्दानꣳ॑ स॒न्दान॒ मश्व॒स्याश्व॑स्य स॒न्दान॑ मस्यसि स॒न्दान॒ मश्व॒स्याश्व॑स्य स॒न्दान॑ मसि । \newline
4. स॒न्दान॑ मस्यसि स॒न्दानꣳ॑ स॒न्दान॑ मसि॒ वृष्ट्यै॒ वृष्ट्या॑ असि स॒न्दानꣳ॑ स॒न्दान॑ मसि॒ वृष्ट्यै᳚ । \newline
5. स॒न्दान॒मिति॑ सं - दान᳚म् । \newline
6. अ॒सि॒ वृष्ट्यै॒ वृष्ट्या॑ अस्यसि॒ वृष्ट्यै᳚ त्वा त्वा॒ वृष्ट्या॑ अस्यसि॒ वृष्ट्यै᳚ त्वा । \newline
7. वृष्ट्यै᳚ त्वा त्वा॒ वृष्ट्यै॒ वृष्ट्यै॒ त्वोपोप॑ त्वा॒ वृष्ट्यै॒ वृष्ट्यै॒ त्वोप॑ । \newline
8. त्वोपोप॑ त्वा॒ त्वोप॑ नह्यामि नह्या॒ म्युप॑ त्वा॒ त्वोप॑ नह्यामि । \newline
9. उप॑ नह्यामि नह्या॒ म्युपोप॑ नह्या॒मीतीति॑ नह्या॒ म्युपोप॑ नह्या॒मीति॑ । \newline
10. न॒ह्या॒मीतीति॑ नह्यामि नह्या॒ मीत्या॑हा॒हे ति॑ नह्यामि नह्या॒ मीत्या॑ह । \newline
11. इत्या॑हा॒हे तीत्या॑ह॒ वृषा॒ वृषा॒ ऽऽहे तीत्या॑ह॒ वृषा᳚ । \newline
12. आ॒ह॒ वृषा॒ वृषा॑ ऽऽहाह॒ वृषा॒ वै वै वृषा॑ ऽऽहाह॒ वृषा॒ वै । \newline
13. वृषा॒ वै वै वृषा॒ वृषा॒ वा अश्वो ऽश्वो॒ वै वृषा॒ वृषा॒ वा अश्वः॑ । \newline
14. वा अश्वो ऽश्वो॒ वै वा अश्वो॒ वृषा॒ वृषा ऽश्वो॒ वै वा अश्वो॒ वृषा᳚ । \newline
15. अश्वो॒ वृषा॒ वृषा ऽश्वो ऽश्वो॒ वृषा॑ प॒र्जन्यः॑ प॒र्जन्यो॒ वृषा ऽश्वो ऽश्वो॒ वृषा॑ प॒र्जन्यः॑ । \newline
16. वृषा॑ प॒र्जन्यः॑ प॒र्जन्यो॒ वृषा॒ वृषा॑ प॒र्जन्यः॑ कृ॒ष्णः कृ॒ष्णः प॒र्जन्यो॒ वृषा॒ वृषा॑ प॒र्जन्यः॑ कृ॒ष्णः । \newline
17. प॒र्जन्यः॑ कृ॒ष्णः कृ॒ष्णः प॒र्जन्यः॑ प॒र्जन्यः॑ कृ॒ष्ण इ॑वे व कृ॒ष्णः प॒र्जन्यः॑ प॒र्जन्यः॑ कृ॒ष्ण इ॑व । \newline
18. कृ॒ष्ण इ॑वे व कृ॒ष्णः कृ॒ष्ण इ॑व॒ खलु॒ खल्वि॑व कृ॒ष्णः कृ॒ष्ण इ॑व॒ खलु॑ । \newline
19. इ॒व॒ खलु॒ खल्वि॑वे व॒ खलु॒ वै वै खल्वि॑वे व॒ खलु॒ वै । \newline
20. खलु॒ वै वै खलु॒ खलु॒ वै भू॒त्वा भू॒त्वा वै खलु॒ खलु॒ वै भू॒त्वा । \newline
21. वै भू॒त्वा भू॒त्वा वै वै भू॒त्वा व॑र्.षति वर्.षति भू॒त्वा वै वै भू॒त्वा व॑र्.षति । \newline
22. भू॒त्वा व॑र्.षति वर्.षति भू॒त्वा भू॒त्वा व॑र्.षति रू॒पेण॑ रू॒पेण॑ वर्.षति भू॒त्वा भू॒त्वा व॑र्.षति रू॒पेण॑ । \newline
23. व॒र्॒.ष॒ति॒ रू॒पेण॑ रू॒पेण॑ वर्.षति वर्.षति रू॒पेणै॒वैव रू॒पेण॑ वर्.षति वर्.षति रू॒पेणै॒व । \newline
24. रू॒पेणै॒वैव रू॒पेण॑ रू॒पेणै॒वैन॑ मेन मे॒व रू॒पेण॑ रू॒पेणै॒वैन᳚म् । \newline
25. ए॒वैन॑ मेन मे॒वैवैनꣳ॒॒ सꣳ स मे॑न मे॒वैवैनꣳ॒॒ सम् । \newline
26. ए॒नꣳ॒॒ सꣳ स मे॑न मेनꣳ॒॒ स म॑र्द्धय त्यर्द्धयति॒ स मे॑न मेनꣳ॒॒ स म॑र्द्धयति । \newline
27. स म॑र्द्धय त्यर्द्धयति॒ सꣳ स म॑र्द्धयति व॒र्॒.षस्य॑ व॒र्॒.षस्या᳚र्द्धयति॒ सꣳ स म॑र्द्धयति व॒र्॒.षस्य॑ । \newline
28. अ॒र्द्ध॒य॒ति॒ व॒र्॒.षस्य॑ व॒र्॒.षस्या᳚र्द्धय त्यर्द्धयति व॒र्॒.षस्या व॑रुद्ध्या॒ अव॑रुद्ध्यै व॒र्॒.षस्या᳚र्द्धय त्यर्द्धयति व॒र्॒.षस्या व॑रुद्ध्यै । \newline
29. व॒र्॒.षस्या व॑रुद्ध्या॒ अव॑रुद्ध्यै व॒र्॒.षस्य॑ व॒र्॒.षस्या व॑रुद्ध्यै । \newline
30. अव॑रुद्ध्या॒ इत्यव॑ - रु॒द्ध्यै॒ । \newline
\pagebreak
\markright{ TS 2.4.10.1  \hfill https://www.vedavms.in \hfill}

\section{ TS 2.4.10.1 }

\textbf{TS 2.4.10.1 } \newline
\textbf{Samhita Paata} \newline

देवा॑ वसव्या॒ देवाः᳚ शर्मण्या॒ देवाः᳚ सपीतय॒ इत्या ब॑द्ध्नाति दे॒वता॑भिरे॒वान्व॒हं ॅवृष्टि॑मिच्छति॒ यदि॒ वर्.षे॒त्-ताव॑त्ये॒व हो॑त॒व्यं॑ ॅयदि॒ न वर्.षे॒च्छ्वो भू॒ते ह॒विर्निर्व॑पेदहोरा॒त्रे वै मि॒त्रावरु॑णावहोरा॒त्राभ्यां॒ खलु॒ वै प॒र्जन्यो॑ वर्.षति॒ नक्तं॑ ॅवा॒ हि दिवा॑ वा॒ वर्.ष॑ति मि॒त्रावरु॑णावे॒व स्वेन॑ भाग॒धेये॒नोप॑ धावति॒ तावे॒वास्मा॑ - [  ] \newline

\textbf{Pada Paata} \newline

देवाः᳚ । व॒स॒व्याः॒ । देवाः᳚ । श॒र्म॒ण्याः॒ । देवाः᳚ । स॒पी॒त॒य॒ इति॑ स - पी॒त॒यः॒ । इति॑ । एति॑ । ब॒द्ध्ना॒ति॒ । दे॒वता॑भिः । ए॒व । अ॒न्व॒हमित्य॑नु - अ॒हम् । वृष्टि᳚म् । इ॒च्छ॒ति॒ । यदि॑ । वर्.षे᳚त् । ताव॑ति । ए॒व । हो॒त॒व्य᳚म् । यदि॑ । न । वर्.षे᳚त् । श्वः । भू॒ते । ह॒विः । निरिति॑ । व॒पे॒त् । अ॒हो॒रा॒त्रे इत्य॑हः - रा॒त्रे । वै । मि॒त्रावरु॑णा॒विति॑ मि॒त्रा - वरु॑णौ । अ॒हो॒रा॒त्राभ्या॒मित्य॑हः - रा॒त्राभ्या᳚म् । खलु॑ । वै । प॒र्जन्यः॑ । व॒र्.॒ष॒ति॒ । नक्त᳚म् । वा॒ । हि । दिवा᳚ । वा॒ । वर्.ष॑ति । मि॒त्रावरु॑णा॒विति॑ मि॒त्रा - वरु॑णौ । ए॒व । स्वेन॑ । भा॒ग॒धेये॒नेति॑ भाग - धेये॑न । उपेति॑ । धा॒व॒ति॒ । तौ । ए॒व । अ॒स्मै॒ ।  \newline


\textbf{Krama Paata} \newline

देवा॑ वसव्याः । व॒स॒व्या॒ देवाः᳚ । देवाः᳚ शर्मण्याः । श॒र्म॒ण्या॒ देवाः᳚ । देवाः᳚ सपीतयः । स॒पी॒त॒य॒ इति॑ । स॒पी॒त॒य॒ इति॑ स - पी॒त॒यः॒ । इत्या । आ ब॑ध्नाति । ब॒ध्ना॒ति॒ दे॒वता॑भिः । दे॒वता॑भिरे॒व । ए॒वान्व॒हम् । अ॒न्व॒हं ॅवृष्टि᳚म् । अ॒न्व॒हमित्य॑नु - अ॒हम् । वृष्टि॑मिच्छति । इ॒च्छ॒ति॒ यदि॑ । यदि॒ वर्.षे᳚त् । वर्.षे॒त् ताव॑ति । ताव॑त्ये॒व । ए॒व हो॑त॒व्य᳚म् । हो॒त॒व्यं॑ ॅयदि॑ । यदि॒ न । न वर्.षे᳚त् । वर्.षे॒च्छ्वः । श्वो भू॒ते । भू॒ते ह॒विः । ह॒विर् निः । निर् व॑पेत् । व॒पे॒द॒हो॒रा॒त्रे । अ॒हो॒रा॒त्रे वै । अ॒हो॒रा॒त्रे इत्य॑हः - रा॒त्रे । वै मि॒त्रावरु॑णौ । मि॒त्रावरु॑णावहोरा॒त्राभ्या᳚म् । मि॒त्रावरु॑णा॒विति॑ मि॒त्रा - वरु॑णौ । अ॒हो॒रा॒त्राभ्या॒म् खलु॑ । अ॒हो॒रा॒त्राभ्या॒मित्य॑हः - रा॒त्राभ्या᳚म् । खलु॒ वै । वै प॒र्जन्यः॑ । प॒र्जन्यो॑ वर्.षति । व॒र्॒.ष॒ति॒ नक्त᳚म् । नक्तं॑ ॅवा । वा॒ हि । हि दिवा᳚ । दिवा॑ वा । वा॒ वर्.ष॑ति । वर्.ष॑ति मि॒त्रावरु॑णौ । मि॒त्रावरु॑णावे॒व । मि॒त्रावरु॑णा॒विति॑ मि॒त्रा - वरु॑णौ । ए॒व स्वेन॑ । स्वेन॑ भाग॒धेये॑न । भा॒ग॒धेये॒नोप॑ । भा॒ग॒धेये॒नेति॑ भाग - धेये॑न । उप॑ धावति । धा॒व॒ति॒ तौ । तावे॒व । ए॒वास्मै᳚ । 
अ॒स्मा॒ अ॒हो॒रा॒त्राभ्या᳚म् \newline

\textbf{Jatai Paata} \newline

1. देवा॑ वसव्या वसव्या॒ देवा॒ देवा॑ वसव्याः । \newline
2. व॒स॒व्या॒ देवा॒ देवा॑ वसव्या वसव्या॒ देवाः᳚ । \newline
3. देवाः᳚ शर्मण्याः शर्मण्या॒ देवा॒ देवाः᳚ शर्मण्याः । \newline
4. श॒र्म॒ण्या॒ देवा॒ देवाः᳚ शर्मण्याः शर्मण्या॒ देवाः᳚ । \newline
5. देवाः᳚ सपीतयः सपीतयो॒ देवा॒ देवाः᳚ सपीतयः । \newline
6. स॒पी॒त॒य॒ इतीति॑ सपीतयः सपीतय॒ इति॑ । \newline
7. स॒पी॒त॒य॒ इति॑ स - पी॒त॒यः॒ । \newline
8. इत्येतीत्या । \newline
9. आ ब॑द्ध्नाति बद्ध्ना॒ त्या ब॑द्ध्नाति । \newline
10. ब॒द्ध्ना॒ति॒ दे॒वता॑भिर् दे॒वता॑भिर् बद्ध्नाति बद्ध्नाति दे॒वता॑भिः । \newline
11. दे॒वता॑भि रे॒वैव दे॒वता॑भिर् दे॒वता॑भि रे॒व । \newline
12. ए॒वा न्व॒ह म॑न्व॒ह मे॒वैवा न्व॒हम् । \newline
13. अ॒न्व॒हं ॅवृष्टिं॒ ॅवृष्टि॑ मन्व॒ह म॑न्व॒हं ॅवृष्टि᳚म् । \newline
14. अ॒न्व॒हमित्य॑नु - अ॒हम् । \newline
15. वृष्टि॑ मिच्छतीच्छति॒ वृष्टिं॒ ॅवृष्टि॑ मिच्छति । \newline
16. इ॒च्छ॒ति॒ यदि॒ यदी᳚ च्छती च्छति॒ यदि॑ । \newline
17. यदि॒ वर्.षे॒द् वर्.षे॒द् यदि॒ यदि॒ वर्.षे᳚त् । \newline
18. वर्.षे॒त् ताव॑ति॒ ताव॑ति॒ वर्.षे॒द् वर्.षे॒त् ताव॑ति । \newline
19. ताव॑ त्ये॒वैव ताव॑ति॒ ताव॑ त्ये॒व । \newline
20. ए॒व हो॑त॒व्यꣳ॑ होत॒व्य॑ मे॒वैव हो॑त॒व्य᳚म् । \newline
21. हो॒त॒व्यं॑ ॅयदि॒ यदि॑ होत॒व्यꣳ॑ होत॒व्यं॑ ॅयदि॑ । \newline
22. यदि॒ न न यदि॒ यदि॒ न । \newline
23. न वर्.षे॒द् वर्.षे॒न् न न वर्.षे᳚त् । \newline
24. वर्.षे॒च् छ्वः श्वो वर्.षे॒द् वर्.षे॒च् छ्वः । \newline
25. श्वो भू॒ते भू॒ते श्वः श्वो भू॒ते । \newline
26. भू॒ते ह॒विर्. ह॒विर् भू॒ते भू॒ते ह॒विः । \newline
27. ह॒विर् निर् णिर्. ह॒विर्. ह॒विर् निः । \newline
28. निर् व॑पेद् वपे॒न् निर् णिर् व॑पेत् । \newline
29. व॒पे॒ द॒हो॒रा॒त्रे अ॑होरा॒त्रे व॑पेद् वपे दहोरा॒त्रे । \newline
30. अ॒हो॒रा॒त्रे वै वा अ॑होरा॒त्रे अ॑होरा॒त्रे वै । \newline
31. अ॒हो॒रा॒त्रे इत्य॑हः - रा॒त्रे । \newline
32. वै मि॒त्रावरु॑णौ मि॒त्रावरु॑णौ॒ वै वै मि॒त्रावरु॑णौ । \newline
33. मि॒त्रावरु॑णा वहोरा॒त्राभ्या॑ महोरा॒त्राभ्या᳚म् मि॒त्रावरु॑णौ मि॒त्रावरु॑णा वहोरा॒त्राभ्या᳚म् । \newline
34. मि॒त्रावरु॑णा॒विति॑ मि॒त्रा - वरु॑णौ । \newline
35. अ॒हो॒रा॒त्राभ्या॒म् खलु॒ खल्व॑होरा॒त्राभ्या॑ महोरा॒त्राभ्या॒म् खलु॑ । \newline
36. अ॒हो॒रा॒त्राभ्या॒मित्य॑हः - रा॒त्राभ्या᳚म् । \newline
37. खलु॒ वै वै खलु॒ खलु॒ वै । \newline
38. वै प॒र्जन्यः॑ प॒र्जन्यो॒ वै वै प॒र्जन्यः॑ । \newline
39. प॒र्जन्यो॑ वर्.षति वर्.षति प॒र्जन्यः॑ प॒र्जन्यो॑ वर्.षति । \newline
40. व॒र्॒.ष॒ति॒ नक्त॒म् नक्तं॑ ॅवर्.षति वर्.षति॒ नक्त᳚म् । \newline
41. नक्तं॑ ॅवा वा॒ नक्त॒म् नक्तं॑ ॅवा । \newline
42. वा॒ हि हि वा॑ वा॒ हि । \newline
43. हि दिवा॒ दिवा॒ हि हि दिवा᳚ । \newline
44. दिवा॑ वा वा॒ दिवा॒ दिवा॑ वा । \newline
45. वा॒ वर्.ष॑ति॒ वर्.ष॑ति वा वा॒ वर्.ष॑ति । \newline
46. वर्.ष॑ति मि॒त्रावरु॑णौ मि॒त्रावरु॑णौ॒ वर्.ष॑ति॒ वर्.ष॑ति मि॒त्रावरु॑णौ । \newline
47. मि॒त्रावरु॑णा वे॒वैव मि॒त्रावरु॑णौ मि॒त्रावरु॑णा वे॒व । \newline
48. मि॒त्रावरु॑णा॒विति॑ मि॒त्रा - वरु॑णौ । \newline
49. ए॒व स्वेन॒ स्वेनै॒वैव स्वेन॑ । \newline
50. स्वेन॑ भाग॒धेये॑न भाग॒धेये॑न॒ स्वेन॒ स्वेन॑ भाग॒धेये॑न । \newline
51. भा॒ग॒धेये॒नोपोप॑ भाग॒धेये॑न भाग॒धेये॒नोप॑ । \newline
52. भा॒ग॒धेये॒नेति॑ भाग - धेये॑न । \newline
53. उप॑ धावति धाव॒ त्युपोप॑ धावति । \newline
54. धा॒व॒ति॒ तौ तौ धा॑वति धावति॒ तौ । \newline
55. ता वे॒वैव तौ ता वे॒व । \newline
56. ए॒वास्मा॑ अस्मा ए॒वैवास्मै᳚ । \newline
57. अ॒स्मा॒ अ॒हो॒रा॒त्राभ्या॑ महोरा॒त्राभ्या॑ मस्मा अस्मा अहोरा॒त्राभ्या᳚म् । \newline

\textbf{Ghana Paata } \newline

1. देवा॑ वसव्या वसव्या॒ देवा॒ देवा॑ वसव्या॒ देवा॒ देवा॑ वसव्या॒ देवा॒ देवा॑ वसव्या॒ देवाः᳚ । \newline
2. व॒स॒व्या॒ देवा॒ देवा॑ वसव्या वसव्या॒ देवाः᳚ शर्मण्याः शर्मण्या॒ देवा॑ वसव्या वसव्या॒ देवाः᳚ शर्मण्याः । \newline
3. देवाः᳚ शर्मण्याः शर्मण्या॒ देवा॒ देवाः᳚ शर्मण्या॒ देवा॒ देवाः᳚ शर्मण्या॒ देवा॒ देवाः᳚ शर्मण्या॒ देवाः᳚ । \newline
4. श॒र्म॒ण्या॒ देवा॒ देवाः᳚ शर्मण्याः शर्मण्या॒ देवाः᳚ सपीतयः सपीतयो॒ देवाः᳚ शर्मण्याः शर्मण्या॒ देवाः᳚ सपीतयः । \newline
5. देवाः᳚ सपीतयः सपीतयो॒ देवा॒ देवाः᳚ सपीतय॒ इतीति॑ सपीतयो॒ देवा॒ देवाः᳚ सपीतय॒ इति॑ । \newline
6. स॒पी॒त॒य॒ इतीति॑ सपीतयः सपीतय॒ इत्येति॑ सपीतयः सपीतय॒ इत्या । \newline
7. स॒पी॒त॒य॒ इति॑ स - पी॒त॒यः॒ । \newline
8. इत्येतीत्या ब॑द्ध्नाति बद्ध्ना॒ त्येतीत्या ब॑द्ध्नाति । \newline
9. आ ब॑द्ध्नाति बद्ध्ना॒त्या ब॑द्ध्नाति दे॒वता॑भिर् दे॒वता॑भिर् बद्ध्ना॒त्या ब॑द्ध्नाति दे॒वता॑भिः । \newline
10. ब॒द्ध्ना॒ति॒ दे॒वता॑भिर् दे॒वता॑भिर् बद्ध्नाति बद्ध्नाति दे॒वता॑भि रे॒वैव दे॒वता॑भिर् बद्ध्नाति बद्ध्नाति दे॒वता॑भिरे॒व । \newline
11. दे॒वता॑भि रे॒वैव दे॒वता॑भिर् दे॒वता॑भि रे॒वान्व॒ह म॑न्व॒ह मे॒व दे॒वता॑भिर् दे॒वता॑भि रे॒वान्व॒हम् । \newline
12. ए॒वान्व॒ह म॑न्व॒ह मे॒वैवान्व॒हं ॅवृष्टिं॒ ॅवृष्टि॑ मन्व॒ह मे॒वैवान्व॒हं ॅवृष्टि᳚म् । \newline
13. अ॒न्व॒हं ॅवृष्टिं॒ ॅवृष्टि॑ मन्व॒ह म॑न्व॒हं ॅवृष्टि॑ मिच्छतीच्छति॒ वृष्टि॑ मन्व॒ह म॑न्व॒हं ॅवृष्टि॑ मिच्छति । \newline
14. अ॒न्व॒हमित्य॑नु - अ॒हम् । \newline
15. वृष्टि॑ मिच्छतीच्छति॒ वृष्टिं॒ ॅवृष्टि॑ मिच्छति॒ यदि॒ यदी᳚च्छति॒ वृष्टिं॒ ॅवृष्टि॑ मिच्छति॒ यदि॑ । \newline
16. इ॒च्छ॒ति॒ यदि॒ यदी᳚च्छती च्छति॒ यदि॒ वर्.षे॒द् वर्.षे॒द् यदी᳚च्छती च्छति॒ यदि॒ वर्.षे᳚त् । \newline
17. यदि॒ वर्.षे॒द् वर्.षे॒द् यदि॒ यदि॒ वर्.षे॒त् ताव॑ति॒ ताव॑ति॒ वर्.षे॒द् यदि॒ यदि॒ वर्.षे॒त् ताव॑ति । \newline
18. वर्.षे॒त् ताव॑ति॒ ताव॑ति॒ वर्.षे॒द् वर्.षे॒त् ताव॑ त्ये॒वैव ताव॑ति॒ वर्.षे॒द् वर्.षे॒त् ताव॑त्ये॒व । \newline
19. ताव॑त्ये॒वैव ताव॑ति॒ ताव॑त्ये॒व हो॑त॒व्यꣳ॑ होत॒व्य॑ मे॒व ताव॑ति॒ ताव॑त्ये॒व हो॑त॒व्य᳚म् । \newline
20. ए॒व हो॑त॒व्यꣳ॑ होत॒व्य॑ मे॒वैव हो॑त॒व्यं॑ ॅयदि॒ यदि॑ होत॒व्य॑ मे॒वैव हो॑त॒व्यं॑ ॅयदि॑ । \newline
21. हो॒त॒व्यं॑ ॅयदि॒ यदि॑ होत॒व्यꣳ॑ होत॒व्यं॑ ॅयदि॒ न न यदि॑ होत॒व्यꣳ॑ होत॒व्यं॑ ॅयदि॒ न । \newline
22. यदि॒ न न यदि॒ यदि॒ न वर्.षे॒द् वर्.षे॒न् न यदि॒ यदि॒ न वर्.षे᳚त् । \newline
23. न वर्.षे॒द् वर्.षे॒न् न न वर्.षे॒च् छ्वः श्वो वर्.षे॒न् न न वर्.षे॒च् छ्वः । \newline
24. वर्.षे॒च् छ्वः श्वो वर्.षे॒द् वर्.षे॒च् छ्वो भू॒ते भू॒ते श्वो वर्.षे॒द् वर्.षे॒च् छ्वो भू॒ते । \newline
25. श्वो भू॒ते भू॒ते श्वः श्वो भू॒ते ह॒विर्. ह॒विर् भू॒ते श्वः श्वो भू॒ते ह॒विः । \newline
26. भू॒ते ह॒विर्. ह॒विर् भू॒ते भू॒ते ह॒विर् निर् णिर्. ह॒विर् भू॒ते भू॒ते ह॒विर् निः । \newline
27. ह॒विर् निर् णिर्. ह॒विर्. ह॒विर् निर् व॑पेद् वपे॒न् निर्. ह॒विर्. ह॒विर् निर् व॑पेत् । \newline
28. निर् व॑पेद् वपे॒न् निर् णिर् व॑पे दहोरा॒त्रे अ॑होरा॒त्रे व॑पे॒न् निर् णिर् व॑पे दहोरा॒त्रे । \newline
29. व॒पे॒ द॒हो॒रा॒त्रे अ॑होरा॒त्रे व॑पेद् वपे दहोरा॒त्रे वै वा अ॑होरा॒त्रे व॑पेद् वपे दहोरा॒त्रे वै । \newline
30. अ॒हो॒रा॒त्रे वै वा अ॑होरा॒त्रे अ॑होरा॒त्रे वै मि॒त्रावरु॑णौ मि॒त्रावरु॑णौ॒ वा अ॑होरा॒त्रे अ॑होरा॒त्रे वै मि॒त्रावरु॑णौ । \newline
31. अ॒हो॒रा॒त्रे इत्य॑हः - रा॒त्रे । \newline
32. वै मि॒त्रावरु॑णौ मि॒त्रावरु॑णौ॒ वै वै मि॒त्रावरु॑णा वहोरा॒त्राभ्या॑ महोरा॒त्राभ्या᳚म् मि॒त्रावरु॑णौ॒ वै वै मि॒त्रावरु॑णा वहोरा॒त्राभ्या᳚म् । \newline
33. मि॒त्रावरु॑णा वहोरा॒त्राभ्या॑ महोरा॒त्राभ्या᳚म् मि॒त्रावरु॑णौ मि॒त्रावरु॑णा वहोरा॒त्राभ्या॒म् खलु॒ खल्व॑होरा॒त्राभ्या᳚म् मि॒त्रावरु॑णौ मि॒त्रावरु॑णा वहोरा॒त्राभ्या॒म् खलु॑ । \newline
34. मि॒त्रावरु॑णा॒विति॑ मि॒त्रा - वरु॑णौ । \newline
35. अ॒हो॒रा॒त्राभ्या॒म् खलु॒ खल्व॑होरा॒त्राभ्या॑ महोरा॒त्राभ्या॒म् खलु॒ वै वै खल्व॑होरा॒त्राभ्या॑ महोरा॒त्राभ्या॒म् खलु॒ वै । \newline
36. अ॒हो॒रा॒त्राभ्या॒मित्य॑हः - रा॒त्राभ्या᳚म् । \newline
37. खलु॒ वै वै खलु॒ खलु॒ वै प॒र्जन्यः॑ प॒र्जन्यो॒ वै खलु॒ खलु॒ वै प॒र्जन्यः॑ । \newline
38. वै प॒र्जन्यः॑ प॒र्जन्यो॒ वै वै प॒र्जन्यो॑ वर्.षति वर्.षति प॒र्जन्यो॒ वै वै प॒र्जन्यो॑ वर्.षति । \newline
39. प॒र्जन्यो॑ वर्.षति वर्.षति प॒र्जन्यः॑ प॒र्जन्यो॑ वर्.षति॒ नक्त॒म् नक्तं॑ ॅवर्.षति प॒र्जन्यः॑ प॒र्जन्यो॑ वर्.षति॒ नक्त᳚म् । \newline
40. व॒र्॒.ष॒ति॒ नक्त॒म् नक्तं॑ ॅवर्.षति वर्.षति॒ नक्तं॑ ॅवा वा॒ नक्तं॑ ॅवर्.षति वर्.षति॒ नक्तं॑ ॅवा । \newline
41. नक्तं॑ ॅवा वा॒ नक्त॒म् नक्तं॑ ॅवा॒ हि हि वा॒ नक्त॒म् नक्तं॑ ॅवा॒ हि । \newline
42. वा॒ हि हि वा॑ वा॒ हि दिवा॒ दिवा॒ हि वा॑ वा॒ हि दिवा᳚ । \newline
43. हि दिवा॒ दिवा॒ हि हि दिवा॑ वा वा॒ दिवा॒ हि हि दिवा॑ वा । \newline
44. दिवा॑ वा वा॒ दिवा॒ दिवा॑ वा॒ वर्.ष॑ति॒ वर्.ष॑ति वा॒ दिवा॒ दिवा॑ वा॒ वर्.ष॑ति । \newline
45. वा॒ वर्.ष॑ति॒ वर्.ष॑ति वा वा॒ वर्.ष॑ति मि॒त्रावरु॑णौ मि॒त्रावरु॑णौ॒ वर्.ष॑ति वा वा॒ वर्.ष॑ति मि॒त्रावरु॑णौ । \newline
46. वर्.ष॑ति मि॒त्रावरु॑णौ मि॒त्रावरु॑णौ॒ वर्.ष॑ति॒ वर्.ष॑ति मि॒त्रावरु॑णा वे॒वैव मि॒त्रावरु॑णौ॒ वर्.ष॑ति॒ वर्.ष॑ति मि॒त्रावरु॑णा वे॒व । \newline
47. मि॒त्रावरु॑णा वे॒वैव मि॒त्रावरु॑णौ मि॒त्रावरु॑णा वे॒व स्वेन॒ स्वेनै॒व मि॒त्रावरु॑णौ मि॒त्रावरु॑णा वे॒व स्वेन॑ । \newline
48. मि॒त्रावरु॑णा॒विति॑ मि॒त्रा - वरु॑णौ । \newline
49. ए॒व स्वेन॒ स्वेनै॒वैव स्वेन॑ भाग॒धेये॑न भाग॒धेये॑न॒ स्वेनै॒वैव स्वेन॑ भाग॒धेये॑न । \newline
50. स्वेन॑ भाग॒धेये॑न भाग॒धेये॑न॒ स्वेन॒ स्वेन॑ भाग॒धेये॒नोपोप॑ भाग॒धेये॑न॒ स्वेन॒ स्वेन॑ भाग॒धेये॒नोप॑ । \newline
51. भा॒ग॒धेये॒नोपोप॑ भाग॒धेये॑न भाग॒धेये॒नोप॑ धावति धाव॒त्युप॑ भाग॒धेये॑न भाग॒धेये॒नोप॑ धावति । \newline
52. भा॒ग॒धेये॒नेति॑ भाग - धेये॑न । \newline
53. उप॑ धावति धाव॒ त्युपोप॑ धावति॒ तौ तौ धा॑व॒ त्युपोप॑ धावति॒ तौ । \newline
54. धा॒व॒ति॒ तौ तौ धा॑वति धावति॒ ता वे॒वैव तौ धा॑वति धावति॒ ता वे॒व । \newline
55. ता वे॒वैव तौ ता वे॒वास्मा॑ अस्मा ए॒व तौ ता वे॒वास्मै᳚ । \newline
56. ए॒वास्मा॑ अस्मा ए॒वैवास्मा॑ अहोरा॒त्राभ्या॑ महोरा॒त्राभ्या॑ मस्मा ए॒वैवास्मा॑ अहोरा॒त्राभ्या᳚म् । \newline
57. अ॒स्मा॒ अ॒हो॒रा॒त्राभ्या॑ महोरा॒त्राभ्या॑ मस्मा अस्मा अहोरा॒त्राभ्या᳚म् प॒र्जन्य॑म् प॒र्जन्य॑ महोरा॒त्राभ्या॑ मस्मा अस्मा अहोरा॒त्राभ्या᳚म् प॒र्जन्य᳚म् । \newline
\pagebreak
\markright{ TS 2.4.10.2  \hfill https://www.vedavms.in \hfill}

\section{ TS 2.4.10.2 }

\textbf{TS 2.4.10.2 } \newline
\textbf{Samhita Paata} \newline

अहोरा॒त्राभ्यां᳚ प॒र्जन्यं॑ ॅवर्.षयतो॒ऽग्नये॑ धाम॒च्छदे॑ पुरो॒डाश॑म॒ष्टाक॑पालं॒ निर्व॑पेन्मारु॒तꣳ स॒प्तक॑पालꣳ सौ॒र्यमेक॑कपालम॒ग्निर्वा इ॒तो वृष्टि॒मुदी॑रयति म॒रुतः॑ सृ॒ष्टां न॑यन्ति य॒दा खलु॒ वा अ॒सावा॑दि॒त्यो न्य॑ङ्-र॒श्मिभिः॑ पर्या॒वर्त॒तेऽथ॑वर्.षतिधाम॒च्छदि॑व॒ खलु॒ वै भू॒त्वा व॑र्.षत्ये॒ता वै दे॒वता॒ वृष्‌ट्या॑ ईशते॒ ता ए॒व स्वेन॑ भाग॒धेये॒नोप॑ धावति॒ ता - [  ] \newline

\textbf{Pada Paata} \newline

अ॒हो॒रा॒त्राभ्या॒मित्य॑हः-रा॒त्राभ्या᳚म् ।  प॒र्जन्य᳚म् । व॒र्.॒ष॒य॒तः॒ । अ॒ग्नये᳚ । धा॒म॒च्छद॒ इति॑ धाम - छदे᳚ । पु॒रो॒डाश᳚म् । अ॒ष्टाक॑पाल॒मित्य॒ष्टा - क॒पा॒ल॒म् । निरिति॑ । व॒पे॒त् । मा॒रु॒तम् । स॒प्तक॑पाल॒मिति॑ स॒प्त - क॒पा॒ल॒म् । सौ॒र्यम् । एक॑कपाल॒मित्येक॑ - क॒पा॒ल॒म् । अ॒ग्निः । वै । इ॒तः । वृष्टि᳚म् । उदिति॑ । ई॒र॒य॒ति॒ । म॒रुतः॑ । सृ॒ष्टाम् । न॒य॒न्ति॒ । य॒दा । खलु॑ । वै । अ॒सौ । आ॒दि॒त्यः । न्यङ्॑ । र॒श्मिभि॒रिति॑ र॒श्मि - भिः॒ । प॒र्या॒वर्त॑त॒ इति॑ परि - आ॒वर्त॑ते । अथ॑ । व॒र्.॒ष॒ति॒ । धा॒म॒च्छदिति॑ धाम - छत् । इ॒व॒ । खलु॑ । वै । भू॒त्वा ।   व॒र्.॒ष॒ति॒ । ए॒ताः । वै । दे॒वताः᳚ । वृष्ट्याः᳚ । ई॒श॒ते॒ । ताः । ए॒व । स्वेन॑ । भा॒ग॒धेये॒नेति॑ भाग - धेये॑न । उपेति॑ । धा॒व॒ति॒ । ताः ।  \newline


\textbf{Krama Paata} \newline

अ॒हो॒रा॒त्राभ्या᳚म् प॒र्जन्य᳚म् । अ॒हो॒रा॒त्राभ्या॒मित्य॑हः - रा॒त्राभ्या᳚म् । प॒र्जन्यं॑ ॅवर्.षयतः । व॒र्॒.ष॒य॒तो॒ऽग्नये᳚ । अ॒ग्नये॑ धाम॒च्छदे᳚ । धा॒म॒च्छदे॑ पुरो॒डाश᳚म् । धा॒म॒च्छद॒ इति॑ धाम - छदे᳚ । पु॒रो॒डाश॑म॒ष्टाक॑पालम् । अ॒ष्टाक॑पाल॒म् निः । अ॒ष्टाक॑पाल॒मित्य॒ष्टा - क॒पा॒ल॒म् । निर् व॑पेत् । व॒पे॒न् मा॒रु॒तम् । मा॒रु॒तꣳ स॒प्तक॑पालम् । स॒प्तक॑पालꣳ सौ॒र्यम् । स॒प्तक॑पाल॒मिति॑ स॒प्त - क॒पा॒ल॒म् । सौ॒र्यमेक॑कपालम् । एक॑कपालम॒ग्निः । एक॑कपाल॒मित्येक॑ - क॒पा॒ल॒म् । अ॒ग्निर् वै । वा इ॒तः । इ॒तो वृष्टि᳚म् । वृष्टि॒मुत् । उदी॑रयति । ई॒र॒य॒ति॒ म॒रुतः॑ । म॒रुतः॑ सृ॒ष्टाम् । सृ॒ष्टाम् न॑यन्ति । न॒य॒न्ति॒ य॒दा । य॒दा खलु॑ । खलु॒ वै । वा अ॒सौ । अ॒सावा॑दि॒त्यः । आ॒दि॒त्यो न्यङ्ङ्॑ । न्य॑ङ् र॒श्मिभिः॑ । र॒श्मिभिः॑ पर्या॒वर्त॑ते । र॒श्मिभि॒रिति॑ र॒श्मि - भिः॒ । प॒र्या॒वर्त॒ते ऽथ॑ । प॒र्या॒वर्त॑त॒ इति॑ परि - आ॒वर्त॑ते । अथ॑ वर्.षति । व॒र्॒.ष॒ति॒ धा॒म॒च्छत् । धा॒म॒च्छदि॑व । धा॒म॒च्छदिति॑ धाम - छत् । इ॒व॒ खलु॑ । खलु॒ वै । वै भू॒त्वा । भू॒त्वा व॑र्.षति । व॒र्॒.ष॒त्ये॒ताः । ए॒ता वै । वै दे॒वताः᳚ । दे॒वता॒ वृष्ट्याः᳚ । वृष्ट्या॑ ईशते । ई॒श॒ते॒ ताः । ता ए॒व । ए॒व स्वेन॑ । स्वेन॑ भाग॒धेये॑न । भा॒ग॒धेये॒नोप॑ । भा॒ग॒धेये॒नेति॑ भाग - धेये॑न । उप॑ धावति । धा॒व॒ति॒ ताः । ता ए॒व \newline

\textbf{Jatai Paata} \newline

1. अ॒हो॒रा॒त्राभ्या᳚म् प॒र्जन्य॑म् प॒र्जन्य॑ महोरा॒त्राभ्या॑ महोरा॒त्राभ्या᳚म् प॒र्जन्य᳚म् । \newline
2. अ॒हो॒रा॒त्राभ्या॒मित्य॑हः - रा॒त्राभ्या᳚म् । \newline
3. प॒र्जन्यं॑ ॅवर्.षयतो वर्.षयतः प॒र्जन्य॑म् प॒र्जन्यं॑ ॅवर्.षयतः । \newline
4. व॒र्॒.ष॒य॒तो॒ ऽग्नये॒ ऽग्नये॑ वर्.षयतो वर्.षयतो॒ ऽग्नये᳚ । \newline
5. अ॒ग्नये॑ धाम॒च्छदे॑ धाम॒च्छदे॒ ऽग्नये॒ ऽग्नये॑ धाम॒च्छदे᳚ । \newline
6. धा॒म॒च्छदे॑ पुरो॒डाश॑म् पुरो॒डाश॑म् धाम॒च्छदे॑ धाम॒च्छदे॑ पुरो॒डाश᳚म् । \newline
7. धा॒म॒च्छद॒ इति॑ धाम - छदे᳚ । \newline
8. पु॒रो॒डाश॑ म॒ष्टाक॑पाल म॒ष्टाक॑पालम् पुरो॒डाश॑म् पुरो॒डाश॑ म॒ष्टाक॑पालम् । \newline
9. अ॒ष्टाक॑पाल॒म् निर् णिर॒ष्टाक॑पाल म॒ष्टाक॑पाल॒म् निः । \newline
10. अ॒ष्टाक॑पाल॒मित्य॒ष्टा - क॒पा॒ल॒म् । \newline
11. निर् व॑पेद् वपे॒न् निर् णिर् व॑पेत् । \newline
12. व॒पे॒न् मा॒रु॒तम् मा॑रु॒तं ॅव॑पेद् वपेन् मारु॒तम् । \newline
13. मा॒रु॒तꣳ स॒प्तक॑पालꣳ स॒प्तक॑पालम् मारु॒तम् मा॑रु॒तꣳ स॒प्तक॑पालम् । \newline
14. स॒प्तक॑पालꣳ सौ॒र्यꣳ सौ॒र्यꣳ स॒प्तक॑पालꣳ स॒प्तक॑पालꣳ सौ॒र्यम् । \newline
15. स॒प्तक॑पाल॒मिति॑ स॒प्त - क॒पा॒ल॒म् । \newline
16. सौ॒र्य मेक॑कपाल॒ मेक॑कपालꣳ सौ॒र्यꣳ सौ॒र्य मेक॑कपालम् । \newline
17. एक॑कपाल म॒ग्निर॒ग्निरेक॑कपाल॒ मेक॑कपाल म॒ग्निः । \newline
18. एक॑कपाल॒मित्येक॑ - क॒पा॒ल॒म् । \newline
19. अ॒ग्निर् वै वा अ॒ग्नि र॒ग्निर् वै । \newline
20. वा इ॒त इ॒तो वै वा इ॒तः । \newline
21. इ॒तो वृष्टिं॒ ॅवृष्टि॑ मि॒त इ॒तो वृष्टि᳚म् । \newline
22. वृष्टि॒ मुदुद् वृष्टिं॒ ॅवृष्टि॒ मुत् । \newline
23. उदी॑रय तीरय॒ त्युदुदी॑रयति । \newline
24. ई॒र॒य॒ति॒ म॒रुतो॑ म॒रुत॑ ईरयती रयति म॒रुतः॑ । \newline
25. म॒रुतः॑ सृ॒ष्टाꣳ सृ॒ष्टाम् म॒रुतो॑ म॒रुतः॑ सृ॒ष्टाम् । \newline
26. सृ॒ष्टाम् न॑यन्ति नयन्ति सृ॒ष्टाꣳ सृ॒ष्टाम् न॑यन्ति । \newline
27. न॒य॒न्ति॒ य॒दा य॒दा न॑यन्ति नयन्ति य॒दा । \newline
28. य॒दा खलु॒ खलु॑ य॒दा य॒दा खलु॑ । \newline
29. खलु॒ वै वै खलु॒ खलु॒ वै । \newline
30. वा अ॒सा व॒सौ वै वा अ॒सौ । \newline
31. अ॒सा वा॑दि॒त्य आ॑दि॒त्यो॑ ऽसा व॒सा वा॑दि॒त्यः । \newline
32. आ॒दि॒त्यो न्या᳚(1॒अ)ङ् न्य॑ङ् ङादि॒त्य आ॑दि॒त्यो न्यङ्॑। \newline
33. न्य॑ङ् र॒श्मिभी॑ र॒श्मिभि॒र् न्या᳚(1॒अ)ङ् न्य॑ङ् र॒श्मिभिः॑ । \newline
34. र॒श्मिभिः॑ पर्या॒वर्त॑ते पर्या॒वर्त॑ते र॒श्मिभी॑ र॒श्मिभिः॑ पर्या॒वर्त॑ते । \newline
35. र॒श्मिभि॒रिति॑ र॒श्मि - भिः॒ । \newline
36. प॒र्या॒वर्त॒ते ऽथाथ॑ पर्या॒वर्त॑ते पर्या॒वर्त॒ते ऽथ॑ । \newline
37. प॒र्या॒वर्त॑त॒ इति॑ परि - आ॒वर्त॑ते । \newline
38. अथ॑ वर्.षति वर्.ष॒ त्यथाथ॑ वर्.षति । \newline
39. व॒र्॒.ष॒ति॒ धा॒म॒च्छद् धा॑म॒च्छद् व॑र्.षति वर्.षति धाम॒च्छत् । \newline
40. धा॒म॒च्छ दि॑वे व धाम॒च्छद् धा॑म॒च्छ दि॑व । \newline
41. धा॒म॒च्छदिति॑ धाम - छत् । \newline
42. इ॒व॒ खलु॒ खल्वि॑वे व॒ खलु॑ । \newline
43. खलु॒ वै वै खलु॒ खलु॒ वै । \newline
44. वै भू॒त्वा भू॒त्वा वै वै भू॒त्वा । \newline
45. भू॒त्वा व॑र्.षति वर्.षति भू॒त्वा भू॒त्वा व॑र्.षति । \newline
46. व॒र्॒.ष॒ त्ये॒ता ए॒ता व॑र्.षति वर्.ष त्ये॒ताः । \newline
47. ए॒ता वै वा ए॒ता ए॒ता वै । \newline
48. वै दे॒वता॑ दे॒वता॒ वै वै दे॒वताः᳚ । \newline
49. दे॒वता॒ वृष्ट्या॒ वृष्ट्या॑ दे॒वता॑ दे॒वता॒ वृष्ट्याः᳚ । \newline
50. वृष्ट्या॑ ईशत ईशते॒ वृष्ट्या॒ वृष्ट्या॑ ईशते । \newline
51. ई॒श॒ते॒ ता स्ता ई॑शत ईशते॒ ताः । \newline
52. ता ए॒वैव ता स्ता ए॒व । \newline
53. ए॒व स्वेन॒ स्वेनै॒वैव स्वेन॑ । \newline
54. स्वेन॑ भाग॒धेये॑न भाग॒धेये॑न॒ स्वेन॒ स्वेन॑ भाग॒धेये॑न । \newline
55. भा॒ग॒धेये॒नोपोप॑ भाग॒धेये॑न भाग॒धेये॒नोप॑ । \newline
56. भा॒ग॒धेये॒नेति॑ भाग - धेये॑न । \newline
57. उप॑ धावति धाव॒ त्युपोप॑ धावति । \newline
58. धा॒व॒ति॒ ता स्ता धा॑वति धावति॒ ताः । \newline
59. ता ए॒वैव ता स्ता ए॒व । \newline

\textbf{Ghana Paata } \newline

1. अ॒हो॒रा॒त्राभ्या᳚म् प॒र्जन्य॑म् प॒र्जन्य॑ महोरा॒त्राभ्या॑ महोरा॒त्राभ्या᳚म् प॒र्जन्यं॑ ॅवर्.षयतो वर्.षयतः प॒र्जन्य॑ महोरा॒त्राभ्या॑ महोरा॒त्राभ्या᳚म् प॒र्जन्यं॑ ॅवर्.षयतः । \newline
2. अ॒हो॒रा॒त्राभ्या॒मित्य॑हः - रा॒त्राभ्या᳚म् । \newline
3. प॒र्जन्यं॑ ॅवर्.षयतो वर्.षयतः प॒र्जन्य॑म् प॒र्जन्यं॑ ॅवर्.षयतो॒ ऽग्नये॒ ऽग्नये॑ वर्.षयतः प॒र्जन्य॑म् प॒र्जन्यं॑ ॅवर्.षयतो॒ ऽग्नये᳚ । \newline
4. व॒र्॒.ष॒य॒तो॒ ऽग्नये॒ ऽग्नये॑ वर्.षयतो वर्.षयतो॒ ऽग्नये॑ धाम॒च्छदे॑ धाम॒च्छदे॒ ऽग्नये॑ वर्.षयतो वर्.षयतो॒ ऽग्नये॑ धाम॒च्छदे᳚ । \newline
5. अ॒ग्नये॑ धाम॒च्छदे॑ धाम॒च्छदे॒ ऽग्नये॒ ऽग्नये॑ धाम॒च्छदे॑ पुरो॒डाश॑म् पुरो॒डाश॑म् धाम॒च्छदे॒ ऽग्नये॒ ऽग्नये॑ धाम॒च्छदे॑ पुरो॒डाश᳚म् । \newline
6. धा॒म॒च्छदे॑ पुरो॒डाश॑म् पुरो॒डाश॑म् धाम॒च्छदे॑ धाम॒च्छदे॑ पुरो॒डाश॑ म॒ष्टाक॑पाल म॒ष्टाक॑पालम् पुरो॒डाश॑म् धाम॒च्छदे॑ धाम॒च्छदे॑ पुरो॒डाश॑ म॒ष्टाक॑पालम् । \newline
7. धा॒म॒च्छद॒ इति॑ धाम - छदे᳚ । \newline
8. पु॒रो॒डाश॑ म॒ष्टाक॑पाल म॒ष्टाक॑पालम् पुरो॒डाश॑म् पुरो॒डाश॑ म॒ष्टाक॑पाल॒म् निर् णिर॒ष्टाक॑पालम् पुरो॒डाश॑म् पुरो॒डाश॑ म॒ष्टाक॑पाल॒म् निः । \newline
9. अ॒ष्टाक॑पाल॒म् निर् णिर॒ष्टाक॑पाल म॒ष्टाक॑पाल॒म् निर् व॑पेद् वपे॒न् निर॒ष्टाक॑पाल म॒ष्टाक॑पाल॒म् निर् व॑पेत् । \newline
10. अ॒ष्टाक॑पाल॒मित्य॒ष्टा - क॒पा॒ल॒म् । \newline
11. निर् व॑पेद् वपे॒न् निर् णिर् व॑पेन् मारु॒तम् मा॑रु॒तं ॅव॑पे॒न् निर् णिर् व॑पेन् मारु॒तम् । \newline
12. व॒पे॒न् मा॒रु॒तम् मा॑रु॒तं ॅव॑पेद् वपेन् मारु॒तꣳ स॒प्तक॑पालꣳ स॒प्तक॑पालम् मारु॒तं ॅव॑पेद् वपेन् मारु॒तꣳ स॒प्तक॑पालम् । \newline
13. मा॒रु॒तꣳ स॒प्तक॑पालꣳ स॒प्तक॑पालम् मारु॒तम् मा॑रु॒तꣳ स॒प्तक॑पालꣳ सौ॒र्यꣳ सौ॒र्यꣳ स॒प्तक॑पालम् मारु॒तम् मा॑रु॒तꣳ स॒प्तक॑पालꣳ सौ॒र्यम् । \newline
14. स॒प्तक॑पालꣳ सौ॒र्यꣳ सौ॒र्यꣳ स॒प्तक॑पालꣳ स॒प्तक॑पालꣳ सौ॒र्य मेक॑कपाल॒ मेक॑कपालꣳ सौ॒र्यꣳ स॒प्तक॑पालꣳ स॒प्तक॑पालꣳ सौ॒र्य मेक॑कपालम् । \newline
15. स॒प्तक॑पाल॒मिति॑ स॒प्त - क॒पा॒ल॒म् । \newline
16. सौ॒र्य मेक॑कपाल॒ मेक॑कपालꣳ सौ॒र्यꣳ सौ॒र्य मेक॑कपाल म॒ग्नि र॒ग्नि रेक॑कपालꣳ सौ॒र्यꣳ सौ॒र्य मेक॑कपाल म॒ग्निः । \newline
17. एक॑कपाल म॒ग्नि र॒ग्नि रेक॑कपाल॒ मेक॑कपाल म॒ग्निर् वै वा अ॒ग्नि रेक॑कपाल॒ मेक॑कपाल म॒ग्निर् वै । \newline
18. एक॑कपाल॒मित्येक॑ - क॒पा॒ल॒म् । \newline
19. अ॒ग्निर् वै वा अ॒ग्नि र॒ग्निर् वा इ॒त इ॒तो वा अ॒ग्नि र॒ग्निर् वा इ॒तः । \newline
20. वा इ॒त इ॒तो वै वा इ॒तो वृष्टिं॒ ॅवृष्टि॑ मि॒तो वै वा इ॒तो वृष्टि᳚म् । \newline
21. इ॒तो वृष्टिं॒ ॅवृष्टि॑ मि॒त इ॒तो वृष्टि॒ मुदुद् वृष्टि॑ मि॒त इ॒तो वृष्टि॒ मुत् । \newline
22. वृष्टि॒ मुदुद् वृष्टिं॒ ॅवृष्टि॒ मुदी॑रयतीरय॒ त्युद् वृष्टिं॒ ॅवृष्टि॒ मुदी॑रयति । \newline
23. उदी॑रयतीरय॒ त्युदुदी॑रयति म॒रुतो॑ म॒रुत॑ ईरय॒ त्युदुदी॑रयति म॒रुतः॑ । \newline
24. ई॒र॒य॒ति॒ म॒रुतो॑ म॒रुत॑ ईरयतीरयति म॒रुतः॑ सृ॒ष्टाꣳ सृ॒ष्टाम् म॒रुत॑ ईरयतीरयति म॒रुतः॑ सृ॒ष्टाम् । \newline
25. म॒रुतः॑ सृ॒ष्टाꣳ सृ॒ष्टाम् म॒रुतो॑ म॒रुतः॑ सृ॒ष्टाम् न॑यन्ति नयन्ति सृ॒ष्टाम् म॒रुतो॑ म॒रुतः॑ सृ॒ष्टाम् न॑यन्ति । \newline
26. सृ॒ष्टाम् न॑यन्ति नयन्ति सृ॒ष्टाꣳ सृ॒ष्टाम् न॑यन्ति य॒दा य॒दा न॑यन्ति सृ॒ष्टाꣳ सृ॒ष्टाम् न॑यन्ति य॒दा । \newline
27. न॒य॒न्ति॒ य॒दा य॒दा न॑यन्ति नयन्ति य॒दा खलु॒ खलु॑ य॒दा न॑यन्ति नयन्ति य॒दा खलु॑ । \newline
28. य॒दा खलु॒ खलु॑ य॒दा य॒दा खलु॒ वै वै खलु॑ य॒दा य॒दा खलु॒ वै । \newline
29. खलु॒ वै वै खलु॒ खलु॒ वा अ॒सा व॒सौ वै खलु॒ खलु॒ वा अ॒सौ । \newline
30. वा अ॒सा व॒सौ वै वा अ॒सा वा॑दि॒त्य आ॑दि॒त्यो॑ ऽसौ वै वा अ॒सा वा॑दि॒त्यः । \newline
31. अ॒सा वा॑दि॒त्य आ॑दि॒तो॑ ऽसा व॒सा वा॑दि॒त्यो न्या᳚(1॒अ)ङ् न्य॑ङ् ङादि॒त्यो॑ ऽसा व॒सा वा॑दि॒त्यो न्यङ्॑ । \newline
32. आ॒दि॒त्यो न्या᳚(1॒अ)ङ् न्य॑ङ् ङादि॒त्य आ॑दि॒त्यो न्य॑ङ् र॒श्मिभी॑ र॒श्मिभि॒र् न्य॑ङ् ङादि॒त्य आ॑दि॒त्यो न्य॑ङ् र॒श्मिभिः॑ । \newline
33. न्य॑ङ् र॒श्मिभी॑ र॒श्मिभि॒र् न्या᳚(1॒अ)ङ् न्य॑ङ् र॒श्मिभिः॑ पर्या॒वर्त॑ते पर्या॒वर्त॑ते र॒श्मिभि॒र् 
न्या᳚(1॒अ)ङ् न्य॑ङ् र॒श्मिभिः॑ पर्या॒वर्त॑ते । \newline
34. र॒श्मिभिः॑ पर्या॒वर्त॑ते पर्या॒वर्त॑ते र॒श्मिभी॑ र॒श्मिभिः॑ पर्या॒वर्त॒ते ऽथाथ॑ पर्या॒वर्त॑ते र॒श्मिभी॑ र॒श्मिभिः॑ पर्या॒वर्त॒ते ऽथ॑ । \newline
35. र॒श्मिभि॒रिति॑ र॒श्मि - भिः॒ । \newline
36. प॒र्या॒वर्त॒ते ऽथाथ॑ पर्या॒वर्त॑ते पर्या॒वर्त॒ते ऽथ॑ वर्.षति वर्.ष॒त्यथ॑ पर्या॒वर्त॑ते पर्या॒वर्त॒ते ऽथ॑ वर्.षति । \newline
37. प॒र्या॒वर्त॑त॒ इति॑ परि - आ॒वर्त॑ते । \newline
38. अथ॑ वर्.षति वर्.ष॒ त्यथाथ॑ वर्.षति धाम॒च्छद् धा॑म॒च्छद् व॑र्.ष॒ त्यथाथ॑ वर्.षति धाम॒च्छत् । \newline
39. व॒र्॒.ष॒ति॒ धा॒म॒च्छद् धा॑म॒च्छद् व॑र्.षति वर्.षति धाम॒च्छदि॑वे व धाम॒च्छद् व॑र्.षति वर्.षति धाम॒च्छदि॑व । \newline
40. धा॒म॒च्छदि॑वे व धाम॒च्छद् धा॑म॒च्छदि॑व॒ खलु॒ खल्वि॑व धाम॒च्छद् धा॑म॒च्छदि॑व॒ खलु॑ । \newline
41. धा॒म॒च्छदिति॑ धाम - छत् । \newline
42. इ॒व॒ खलु॒ खल्वि॑वे व॒ खलु॒ वै वै खल्वि॑वे व॒ खलु॒ वै । \newline
43. खलु॒ वै वै खलु॒ खलु॒ वै भू॒त्वा भू॒त्वा वै खलु॒ खलु॒ वै भू॒त्वा । \newline
44. वै भू॒त्वा भू॒त्वा वै वै भू॒त्वा व॑र्.षति वर्.षति भू॒त्वा वै वै भू॒त्वा व॑र्.षति । \newline
45. भू॒त्वा व॑र्.षति वर्.षति भू॒त्वा भू॒त्वा व॑र्.षत्ये॒ता ए॒ता व॑र्.षति भू॒त्वा भू॒त्वा व॑र्.षत्ये॒ताः । \newline
46. व॒र्॒.ष॒त्ये॒ता ए॒ता व॑र्.षति वर्.षत्ये॒ता वै वा ए॒ता व॑र्.षति वर्.षत्ये॒ता वै । \newline
47. ए॒ता वै वा ए॒ता ए॒ता वै दे॒वता॑ दे॒वता॒ वा ए॒ता ए॒ता वै दे॒वताः᳚ । \newline
48. वै दे॒वता॑ दे॒वता॒ वै वै दे॒वता॒ वृष्ट्या॒ वृष्ट्या॑ दे॒वता॒ वै वै दे॒वता॒ वृष्ट्याः᳚ । \newline
49. दे॒वता॒ वृष्ट्या॒ वृष्ट्या॑ दे॒वता॑ दे॒वता॒ वृष्ट्या॑ ईशत ईशते॒ वृष्ट्या॑ दे॒वता॑ दे॒वता॒ वृष्ट्या॑ ईशते । \newline
50. वृष्ट्या॑ ईशत ईशते॒ वृष्ट्या॒ वृष्ट्या॑ ईशते॒ ता स्ता ई॑शते॒ वृष्ट्या॒ वृष्ट्या॑ ईशते॒ ताः । \newline
51. ई॒श॒ते॒ ता स्ता ई॑शत ईशते॒ ता ए॒वैव ता ई॑शत ईशते॒ ता ए॒व । \newline
52. ता ए॒वैव ता स्ता ए॒व स्वेन॒ स्वेनै॒व ता स्ता ए॒व स्वेन॑ । \newline
53. ए॒व स्वेन॒ स्वेनै॒वैव स्वेन॑ भाग॒धेये॑न भाग॒धेये॑न॒ स्वेनै॒वैव स्वेन॑ भाग॒धेये॑न । \newline
54. स्वेन॑ भाग॒धेये॑न भाग॒धेये॑न॒ स्वेन॒ स्वेन॑ भाग॒धेये॒नोपोप॑ भाग॒धेये॑न॒ स्वेन॒ स्वेन॑ भाग॒धेये॒नोप॑ । \newline
55. भा॒ग॒धेये॒नोपोप॑ भाग॒धेये॑न भाग॒धेये॒नोप॑ धावति धाव॒त्युप॑ भाग॒धेये॑न भाग॒धेये॒नोप॑ धावति । \newline
56. भा॒ग॒धेये॒नेति॑ भाग - धेये॑न । \newline
57. उप॑ धावति धाव॒त्युपोप॑ धावति॒ ता स्ता धा॑व॒त्युपोप॑ धावति॒ ताः । \newline
58. धा॒व॒ति॒ ता स्ता धा॑वति धावति॒ ता ए॒वैव ता धा॑वति धावति॒ ता ए॒व । \newline
59. ता ए॒वैव ता स्ता ए॒वास्मा॑ अस्मा ए॒व ता स्ता ए॒वास्मै᳚ । \newline
\pagebreak
\markright{ TS 2.4.10.3  \hfill https://www.vedavms.in \hfill}

\section{ TS 2.4.10.3 }

\textbf{TS 2.4.10.3 } \newline
\textbf{Samhita Paata} \newline

ए॒वास्मै॑ प॒र्जन्यं॑ ॅवर्.षयन्त्यु॒ता व॑र्.षिष्य॒न् वर्.ष॑त्ये॒व सृ॒जा वृ॒ष्टिं दि॒व आ*ऽद्भिः स॑मु॒द्रं पृ॒णेत्या॑हे॒माश्चै॒वा-मूश्चा॒पः सम॑र्द्धय॒त्यथो॑ आ॒भिरे॒वा-मूरच्छै᳚त्य॒ब्जा अ॑सि प्रथम॒जा बल॑मसि समु॒द्रिय॒मित्या॑ह यथाय॒जुरे॒वैत-दुन्नं॑ भय पृथि॒वीमिति॑ वर्.षा॒ह्वां जु॑होत्ये॒षा वा ( ) ओष॑धीनां ॅवृष्टि॒वनि॒स्तयै॒व वृष्टि॒मा च्या॑वयति॒ ये दे॒वा दि॒विभा॑गा॒ इति॑ कृष्णाजि॒नमव॑ धूनोती॒म ए॒वास्मै॑ लो॒काः प्री॒ता अ॒भीष्टा॑ भवन्ति ॥ \newline

\textbf{Pada Paata} \newline

ए॒व । अ॒स्मै॒ । प॒र्जन्य᳚म् । व॒र्.॒ष॒य॒न्ति॒ । उ॒त । अव॑र्.षिष्यन्न् । वर्.ष॑ति । ए॒व । सृ॒ज ।  वृ॒ष्टिम् । दि॒वः । एति॑ । अ॒द्भिरित्य॑त् - भिः । स॒मु॒द्रम् । पृ॒ण॒ । इति॑ । आ॒ह॒ । इ॒माः । च॒ । ए॒व । अ॒मूः । च॒ । अ॒पः । समिति॑ । अ॒र्द्ध॒य॒ति॒ । अथो॒ इति॑ । आ॒भिः । ए॒व । अ॒मूः । अच्छ॑ । ए॒ति॒ । अ॒ब्जा इत्य॑प् - जाः । अ॒सि॒ । प्र॒थ॒म॒जा इति॑ प्रथम - जाः । बल᳚म् । अ॒सि॒ । स॒मु॒द्रिय᳚म् । इति॑ । आ॒ह॒ । य॒था॒य॒जुरिति॑ यथा - य॒जुः । ए॒व । ए॒तत् । उदिति॑ । न॒भं॒य॒ । पृ॒थि॒वीम् । इति॑ । व॒र्.॒षा॒ह्वामिति॑ वर्.ष - ह्वाम् । जु॒हो॒ति॒ । ए॒षा । वै ( ) । ओष॑धीनाम् । वृ॒ष्टि॒वनि॒रिति॑ वृष्टि - वनिः॑ । तया᳚ । ए॒व । वृष्टि᳚म् । एति॑ । च्या॒व॒य॒ति॒ । ये । दे॒वाः । दि॒विभा॑गा॒ इति॑ दि॒वि - भा॒गाः॒ । इति॑ । कृ॒ष्णा॒जि॒नमिति॑ कृष्ण - अ॒जि॒नम् । अवेति॑ । धू॒नो॒ति॒ । इ॒मे । ए॒व । अ॒स्मै॒ । लो॒काः । प्री॒ताः । अ॒भीष्टा॒ इत्य॒भि - इ॒ष्टाः॒ । भ॒व॒न्ति॒ ॥  \newline


\textbf{Krama Paata} \newline

ए॒वास्मै᳚ । अ॒स्मै॒ प॒र्जन्य᳚म् । प॒र्जन्यं॑ ॅवर्.षयन्ति । व॒र्॒.ष॒य॒न्त्यु॒त । उ॒ताव॑र्.षिष्यन्न् । अव॑र्.षिष्य॒न् वर्.ष॑ति । वर्.ष॑त्ये॒व । ए॒व सृ॒ज । सृ॒जा वृ॒ष्टिम् । वृ॒ष्टिम् दि॒वः । दि॒व आ । आऽद्भिः । अ॒द्भिः स॑मु॒द्रम् । अ॒द्भिरित्य॑त् - भिः । स॒मु॒द्रम् पृ॑ण । पृ॒णेति॑ । इत्या॑ह । आ॒हे॒माः । इ॒माश्च॑ । चै॒व । ए॒वामूः । अ॒मूश्च॑ । चा॒पः । अ॒पः सम् । सम॑र्द्धयति । अ॒र्द्ध॒य॒त्यथो᳚ । अथो॑ आ॒भिः । अथो॒ इत्यथो᳚ । आ॒भि रे॒व । ए॒वामूः । अ॒मूरच्छ॑ । अच्छै॑ति । ए॒त्य॒ब्जाः । अ॒ब्जा अ॑सि । अ॒ब्जा इत्य॑प् - जाः । अ॒सि॒ प्र॒थ॒म॒जाः । प्र॒थ॒म॒जा बल᳚म् । प्र॒थ॒म॒जा इति॑ प्रथम - जाः । बल॑मसि । अ॒सि॒ स॒मु॒द्रिय᳚म् । स॒मु॒द्रिय॒मिति॑ । इत्या॑ह । आ॒ह॒ य॒था॒य॒जुः । य॒था॒य॒जुरे॒व । य॒था॒य॒जुरिति॑ यथा - य॒जुः । ए॒वैतत् । ए॒तदुत् । उन्न॑म्भय । न॒म्भ॒य॒ पृ॒थि॒वीम् । पृ॒थि॒वीमिति॑ । इति॑ वर्.षा॒ह्वाम् । व॒र्॒.षा॒ह्वाम् जु॑होति । व॒र्॒.षा॒ह्वामिति॑ वर्.ष - ह्वाम् । जु॒हो॒त्ये॒षा । ए॒षा वै ( ) । वा ओष॑धीनाम् । ओष॑धीनां ॅवृष्टि॒वनिः॑ । वृ॒ष्टि॒वनि॒स्तया᳚ । वृ॒ष्टि॒वनि॒रिति॑ वृष्टि - वनिः॑ । तयै॒व । ए॒व वृष्टि᳚म् । वृष्टि॒मा । आ च्या॑वयति । च्या॒व॒य॒ति॒ ये । ये दे॒वाः । दे॒वा दि॒विभा॑गाः । दि॒विभा॑गा॒ इति॑ । दि॒विभा॑गा॒ इति॑ दि॒वि - भा॒गाः॒ । इति॑ कृष्णाजि॒नम् । कृ॒ष्णा॒जि॒नमव॑ । कृ॒ष्णा॒जि॒नमिति॑ कृष्ण - अ॒जि॒नम् । अव॑ धूनोति । धू॒नो॒ती॒मे । इ॒म ए॒व । ए॒वास्मै᳚ । अ॒स्मै॒ लो॒काः । लो॒काः प्री॒ताः । प्री॒ता अ॒भीष्टाः᳚ । अ॒भीष्टा॑ भवन्ति । अ॒भीष्टा॒ इत्य॒भि - इ॒ष्टाः॒ । भ॒व॒न्तीति॑ भवन्ति । \newline

\textbf{Jatai Paata} \newline

1. ए॒वास्मा॑ अस्मा ए॒वैवास्मै᳚ । \newline
2. अ॒स्मै॒ प॒र्जन्य॑म् प॒र्जन्य॑ मस्मा अस्मै प॒र्जन्य᳚म् । \newline
3. प॒र्जन्यं॑ ॅवर्.षयन्ति वर्.षयन्ति प॒र्जन्य॑म् प॒र्जन्यं॑ ॅवर्.षयन्ति । \newline
4. व॒र्॒.ष॒य॒ न्त्यु॒तोत व॑र्.षयन्ति वर्.षय न्त्यु॒त । \newline
5. उ॒ता व॑र्.षिष्य॒न् नव॑र्.षिष्यन् नु॒तोता व॑र्.षिष्यन्न् । \newline
6. अव॑र्.षिष्य॒न्॒. वर्.ष॑ति॒ वर्.ष॒ त्यव॑र्.षिष्य॒न् नव॑र्.षिष्य॒न्॒. वर्.ष॑ति । \newline
7. वर्.ष॑ त्ये॒वैव वर्.ष॑ति॒ वर्.ष॑ त्ये॒व । \newline
8. ए॒व सृ॒ज सृ॒जैवैव सृ॒ज । \newline
9. सृ॒जा वृ॒ष्टिं ॅवृ॒ष्टिꣳ सृ॒ज सृ॒जा वृ॒ष्टिम् । \newline
10. वृ॒ष्टिम् दि॒वो दि॒वो वृ॒ष्टिं ॅवृ॒ष्टिम् दि॒वः । \newline
11. दि॒व आ दि॒वो दि॒व आ । \newline
12. आ ऽद्भि र॒द्भिरा ऽद्भिः । \newline
13. अ॒द्भिः स॑मु॒द्रꣳ स॑मु॒द्र म॒द्भि र॒द्भिः स॑मु॒द्रम् । \newline
14. अ॒द्भिरित्य॑त् - भिः । \newline
15. स॒मु॒द्रम् पृ॑ण पृण समु॒द्रꣳ स॑मु॒द्रम् पृ॑ण । \newline
16. पृ॒णे तीति॑ पृण पृ॒णे ति॑ । \newline
17. इत्या॑हा॒हे तीत्या॑ह । \newline
18. आ॒हे॒ मा इ॒मा आ॑हाहे॒ माः । \newline
19. इ॒माश्च॑ चे॒ मा इ॒माश्च॑ । \newline
20. चै॒वैव च॑ चै॒व । \newline
21. ए॒वामू र॒मू रे॒वैवामूः । \newline
22. अ॒मूश्च॑ चा॒मू र॒मूश्च॑ । \newline
23. चा॒पो॑ ऽपश्च॑ चा॒पः । \newline
24. अ॒पः सꣳ स म॒पो॑ ऽपः सम् । \newline
25. स म॑र्द्धय त्यर्द्धयति॒ सꣳ स म॑र्द्धयति । \newline
26. अ॒र्द्ध॒य॒ त्यथो॒ अथो॑ अर्द्धय त्यर्द्धय॒ त्यथो᳚ । \newline
27. अथो॑ आ॒भि रा॒भि रथो॒ अथो॑ आ॒भिः । \newline
28. अथो॒ इत्यथो᳚ । \newline
29. आ॒भि रे॒वैवा भिरा॒भि रे॒व । \newline
30. ए॒वामू र॒मू रे॒वैवा मूः । \newline
31. अ॒मू रच्छाच्छा॒ मू र॒मू रच्छ॑ । \newline
32. अच्छै᳚ त्ये॒त्यच्छा च्छै॑ति । \newline
33. ए॒त्य॒ब्जा अ॒ब्जा ए᳚त्येत्य॒ब्जाः । \newline
34. अ॒ब्जा अ॑स्यस्य॒ब्जा अ॒ब्जा अ॑सि । \newline
35. अ॒ब्जा इत्य॑प् - जाः । \newline
36. अ॒सि॒ प्र॒थ॒म॒जाः प्र॑थम॒जा अ॑स्यसि प्रथम॒जाः । \newline
37. प्र॒थ॒म॒जा बल॒म् बल॑म् प्रथम॒जाः प्र॑थम॒जा बल᳚म् । \newline
38. प्र॒थ॒म॒जा इति॑ प्रथम - जाः । \newline
39. बल॑ मस्यसि॒ बल॒म् बल॑ मसि । \newline
40. अ॒सि॒ स॒मु॒द्रियꣳ॑ समु॒द्रिय॑ मस्यसि समु॒द्रिय᳚म् । \newline
41. स॒मु॒द्रिय॒ मितीति॑ समु॒द्रियꣳ॑ समु॒द्रिय॒ मिति॑ । \newline
42. इत्या॑हा॒हे तीत्या॑ह । \newline
43. आ॒ह॒ य॒था॒य॒जुर् य॑थाय॒जु रा॑हाह यथाय॒जुः । \newline
44. य॒था॒य॒जु रे॒वैव य॑थाय॒जुर् य॑थाय॒जु रे॒व । \newline
45. य॒था॒य॒जुरिति॑ यथा - य॒जुः । \newline
46. ए॒वैत दे॒त दे॒वैवैतत् । \newline
47. ए॒त दुदु दे॒त दे॒त दुत् । \newline
48. उन् नं॑भय नंभ॒यो दुन् नं॑भय । \newline
49. नं॒भ॒य॒ पृ॒थि॒वीम् पृ॑थि॒वीम् नं॑भय नंभय पृथि॒वीम् । \newline
50. पृ॒थि॒वी मितीति॑ पृथि॒वीम् पृ॑थि॒वी मिति॑ । \newline
51. इति॑ वर्.षा॒ह्वां ॅव॑र्.षा॒ह्वा मितीति॑ वर्.षा॒ह्वाम् । \newline
52. व॒र्॒.षा॒ह्वाम् जु॑होति जुहोति वर्.षा॒ह्वां ॅव॑र्.षा॒ह्वाम् जु॑होति । \newline
53. व॒र्.॒षा॒ह्वामिति॑ वर्.ष - ह्वाम् । \newline
54. जु॒हो॒ त्ये॒षैषा जु॑होति जुहो त्ये॒षा । \newline
55. ए॒षा वै वा ए॒षैषा वै । \newline
56. वा ओष॑धीना॒ मोष॑धीनां॒ ॅवै वा ओष॑धीनाम् । \newline
57. ओष॑धीनां ॅवृष्टि॒वनि॑र् वृष्टि॒वनि॒ रोष॑धीना॒ मोष॑धीनां ॅवृष्टि॒वनिः॑ । \newline
58. वृ॒ष्टि॒वनि॒ स्तया॒ तया॑ वृष्टि॒वनि॑र् वृष्टि॒वनि॒ स्तया᳚ । \newline
59. वृ॒ष्टि॒वनि॒रिति॑ वृष्टि - वनिः॑ । \newline
60. तयै॒वैव तया॒ तयै॒व । \newline
61. ए॒व वृष्टिं॒ ॅवृष्टि॑ मे॒वैव वृष्टि᳚म् । \newline
62. वृष्टि॒ मा वृष्टिं॒ ॅवृष्टि॒ मा । \newline
63. आ च्या॑वयति च्यावय॒ त्या च्या॑वयति । \newline
64. च्या॒व॒य॒ति॒ ये ये च्या॑वयति च्यावयति॒ ये । \newline
65. ये दे॒वा दे॒वा ये ये दे॒वाः । \newline
66. दे॒वा दि॒विभा॑गा दि॒विभा॑गा दे॒वा दे॒वा दि॒विभा॑गाः । \newline
67. दि॒विभा॑गा॒ इतीति॑ दि॒विभा॑गा दि॒विभा॑गा॒ इति॑ । \newline
68. दि॒विभा॑गा॒ इति॑ दि॒वि - भा॒गाः॒ । \newline
69. इति॑ कृष्णाजि॒नम् कृ॑ष्णाजि॒न मितीति॑ कृष्णाजि॒नम् । \newline
70. कृ॒ष्णा॒जि॒न मवाव॑ कृष्णाजि॒नम् कृ॑ष्णाजि॒न मव॑ । \newline
71. कृ॒ष्णा॒जि॒नमिति॑ कृष्ण - अ॒जि॒नम् । \newline
72. अव॑ धूनोति धूनो॒ त्यवाव॑ धूनोति । \newline
73. धू॒नो॒ती॒म इ॒मे धू॑नोति धूनोती॒मे । \newline
74. इ॒म ए॒वैवे म इ॒म ए॒व । \newline
75. ए॒वास्मा॑ अस्मा ए॒वैवास्मै᳚ । \newline
76. अ॒स्मै॒ लो॒का लो॒का अ॑स्मा अस्मै लो॒काः । \newline
77. लो॒काः प्री॒ताः प्री॒ता लो॒का लो॒काः प्री॒ताः । \newline
78. प्री॒ता अ॒भीष्टा॑ अ॒भीष्टाः᳚ प्री॒ताः प्री॒ता अ॒भीष्टाः᳚ । \newline
79. अ॒भीष्टा॑ भवन्ति भव न्त्य॒भीष्टा॑ अ॒भीष्टा॑ भवन्ति । \newline
80. अ॒भीष्टा॒ इत्य॒भि - इ॒ष्टाः॒ । \newline
81. भ॒व॒न्तीति॑ भवन्ति । \newline

\textbf{Ghana Paata } \newline

1. ए॒वास्मा॑ अस्मा ए॒वैवास्मै॑ प॒र्जन्य॑म् प॒र्जन्य॑ मस्मा ए॒वैवास्मै॑ प॒र्जन्य᳚म् । \newline
2. अ॒स्मै॒ प॒र्जन्य॑म् प॒र्जन्य॑ मस्मा अस्मै प॒र्जन्यं॑ ॅवर्.षयन्ति वर्.षयन्ति प॒र्जन्य॑ मस्मा अस्मै प॒र्जन्यं॑ ॅवर्.षयन्ति । \newline
3. प॒र्जन्यं॑ ॅवर्.षयन्ति वर्.षयन्ति प॒र्जन्य॑म् प॒र्जन्यं॑ ॅवर्.षय न्त्यु॒तोत व॑र्.षयन्ति प॒र्जन्य॑म् प॒र्जन्यं॑ ॅवर्.षय न्त्यु॒त । \newline
4. व॒र्॒.ष॒य॒ न्त्यु॒तोत व॑र्.षयन्ति वर्.षयन्त्यु॒ता व॑र्.षिष्य॒न् नव॑र्.षिष्यन् नु॒त व॑र्.षयन्ति वर्.षय न्त्यु॒ता व॑र्.षिष्यन्न् । \newline
5. उ॒ताव॑र्.षिष्य॒न् नव॑र्.षिष्यन् नु॒तोता व॑र्.षिष्य॒न्॒. वर्.ष॑ति॒ वर्.ष॒ त्यव॑र्.षिष्यन् नु॒तोता व॑र्.षिष्य॒न्॒. वर्.ष॑ति । \newline
6. अव॑र्.षिष्य॒न्॒. वर्.ष॑ति॒ वर्.ष॒ त्यव॑र्.षिष्य॒न् नव॑र्.षिष्य॒न्॒. वर्.ष॑ त्ये॒वैव वर्.ष॒ त्यव॑र्.षिष्य॒न् नव॑र्.षिष्य॒न्॒. वर्.ष॑त्ये॒व । \newline
7. वर्.ष॑ त्ये॒वैव वर्.ष॑ति॒ वर्.ष॑ त्ये॒व सृ॒ज सृ॒जैव वर्.ष॑ति॒ वर्.ष॑ त्ये॒व सृ॒ज । \newline
8. ए॒व सृ॒ज सृ॒जैवैव सृ॒जा वृ॒ष्टिं ॅवृ॒ष्टिꣳ सृ॒जैवैव सृ॒जा वृ॒ष्टिम् । \newline
9. सृ॒जा वृ॒ष्टिं ॅवृ॒ष्टिꣳ सृ॒ज सृ॒जा वृ॒ष्टिम् दि॒वो दि॒वो वृ॒ष्टिꣳ सृ॒ज सृ॒जा वृ॒ष्टिम् दि॒वः । \newline
10. वृ॒ष्टिम् दि॒वो दि॒वो वृ॒ष्टिं ॅवृ॒ष्टिम् दि॒व आ दि॒वो वृ॒ष्टिं ॅवृ॒ष्टिम् दि॒व आ । \newline
11. दि॒व आ दि॒वो दि॒व आ ऽद्भि र॒द्भिरा दि॒वो दि॒व आ ऽद्भिः । \newline
12. आ ऽद्भि र॒द्भिरा ऽद्भिः स॑मु॒द्रꣳ स॑मु॒द्र म॒द्भिरा ऽद्भिः स॑मु॒द्रम् । \newline
13. अ॒द्भिः स॑मु॒द्रꣳ स॑मु॒द्र म॒द्भि र॒द्भिः स॑मु॒द्रम् पृ॑ण पृण समु॒द्र म॒द्भि र॒द्भिः स॑मु॒द्रम् पृ॑ण । \newline
14. अ॒द्भिरित्य॑त् - भिः । \newline
15. स॒मु॒द्रम् पृ॑ण पृण समु॒द्रꣳ स॑मु॒द्रम् पृ॒णे तीति॑ पृण समु॒द्रꣳ स॑मु॒द्रम् पृ॒णे ति॑ । \newline
16. पृ॒णे तीति॑ पृण पृ॒णे त्या॑हा॒हे ति॑ पृण पृ॒णे त्या॑ह । \newline
17. इत्या॑हा॒हे तीत्या॑हे॒ मा इ॒मा आ॒हे तीत्या॑हे॒ माः । \newline
18. आ॒हे॒ मा इ॒मा आ॑हाहे॒ माश्च॑ चे॒ मा आ॑हाहे॒ माश्च॑ । \newline
19. इ॒माश्च॑ चे॒ मा इ॒माश्चै॒वैव चे॒ मा इ॒माश्चै॒व । \newline
20. चै॒वैव च॑ चै॒वामू र॒मू रे॒व च॑ चै॒वामूः । \newline
21. ए॒वामू र॒मू रे॒वैवामूश्च॑ चा॒मू रे॒वैवामूश्च॑ । \newline
22. अ॒मूश्च॑ चा॒मू र॒मूश्चा॒पो॑ ऽपश्चा॒मू र॒मूश्चा॒पः । \newline
23. चा॒पो॑ ऽपश्च॑ चा॒पः सꣳ स म॒पश्च॑ चा॒पः सम् । \newline
24. अ॒पः सꣳ स म॒पो॑ ऽपः स म॑र्द्धय त्यर्द्धयति॒ स म॒पो॑ ऽपः स म॑र्द्धयति । \newline
25. स म॑र्द्धय त्यर्द्धयति॒ सꣳ स म॑र्द्धय॒त्यथो॒ अथो॑ अर्द्धयति॒ सꣳ स म॑र्द्धय॒त्यथो᳚ । \newline
26. अ॒र्द्ध॒य॒ त्यथो॒ अथो॑ अर्द्धय त्यर्द्धय॒ त्यथो॑ आ॒भि रा॒भि रथो॑ अर्द्धय त्यर्द्धय॒ त्यथो॑ आ॒भिः । \newline
27. अथो॑ आ॒भिरा॒भि रथो॒ अथो॑ आ॒भि रे॒वैवा भिरथो॒ अथो॑ आ॒भिरे॒व । \newline
28. अथो॒ इत्यथो᳚ । \newline
29. आ॒भि रे॒वैवाभि रा॒भि रे॒वामू र॒मू रे॒वाभि रा॒भि रे॒वामूः । \newline
30. ए॒वामू र॒मू रे॒वैवामू रच्छाच्छा॒ मूरे॒वैवामू रच्छ॑ । \newline
31. अ॒मू रच्छाच्छा॒ मूर॒मू रच्छै᳚त्ये॒ त्यच्छा॒मू र॒मू रच्छै॑ति । \newline
32. अच्छै᳚ त्ये॒त्यच्छाच्छै᳚ त्य॒ब्जा अ॒ब्जा ए॒त्यच्छाच्छै᳚ त्य॒ब्जाः । \newline
33. ए॒त्य॒ब्जा अ॒ब्जा ए᳚त्ये त्य॒ब्जा अ॑स्यस्य॒ब्जा ए᳚त्ये त्य॒ब्जा अ॑सि । \newline
34. अ॒ब्जा अ॑स्यस्य॒ब्जा अ॒ब्जा अ॑सि प्रथम॒जाः प्र॑थम॒जा अ॑स्य॒ब्जा अ॒ब्जा अ॑सि प्रथम॒जाः । \newline
35. अ॒ब्जा इत्य॑प् - जाः । \newline
36. अ॒सि॒ प्र॒थ॒म॒जाः प्र॑थम॒जा अ॑स्यसि प्रथम॒जा बल॒म् बल॑म् प्रथम॒जा अ॑स्यसि प्रथम॒जा बल᳚म् । \newline
37. प्र॒थ॒म॒जा बल॒म् बल॑म् प्रथम॒जाः प्र॑थम॒जा बल॑ मस्यसि॒ बल॑म् प्रथम॒जाः प्र॑थम॒जा बल॑ मसि । \newline
38. प्र॒थ॒म॒जा इति॑ प्रथम - जाः । \newline
39. बल॑ मस्यसि॒ बल॒म् बल॑ मसि समु॒द्रियꣳ॑ समु॒द्रिय॑ मसि॒ बल॒म् बल॑ मसि समु॒द्रिय᳚म् । \newline
40. अ॒सि॒ स॒मु॒द्रियꣳ॑ समु॒द्रिय॑ मस्यसि समु॒द्रिय॒ मितीति॑ समु॒द्रिय॑ मस्यसि समु॒द्रिय॒ मिति॑ । \newline
41. स॒मु॒द्रिय॒ मितीति॑ समु॒द्रियꣳ॑ समु॒द्रिय॒ मित्या॑हा॒हे ति॑ समु॒द्रियꣳ॑ समु॒द्रिय॒ मित्या॑ह । \newline
42. इत्या॑हा॒हे तीत्या॑ह यथाय॒जुर् य॑थाय॒जु रा॒हे तीत्या॑ह यथाय॒जुः । \newline
43. आ॒ह॒ य॒था॒य॒जुर् य॑थाय॒जु रा॑हाह यथाय॒जु रे॒वैव य॑थाय॒जु रा॑हाह यथाय॒जु रे॒व । \newline
44. य॒था॒य॒जु रे॒वैव य॑थाय॒जुर् य॑थाय॒जु रे॒वैत दे॒तदे॒व य॑थाय॒जुर् य॑थाय॒जु रे॒वैतत् । \newline
45. य॒था॒य॒जुरिति॑ यथा - य॒जुः । \newline
46. ए॒वैत दे॒त दे॒वैवैत दुदु दे॒त दे॒वैवैत दुत् । \newline
47. ए॒त दुदु दे॒त दे॒तदुन् नं॑भय नंभ॒यो दे॒त दे॒तदुन् नं॑भय । \newline
48. उन् नं॑भय नंभ॒योदुन् नं॑भय पृथि॒वीम् पृ॑थि॒वीम् नं॑भ॒योदुन् नं॑भय पृथि॒वीम् । \newline
49. नं॒भ॒य॒ पृ॒थि॒वीम् पृ॑थि॒वीम् नं॑भय नंभय पृथि॒वी मितीति॑ पृथि॒वीम् नं॑भय नंभय पृथि॒वी मिति॑ । \newline
50. पृ॒थि॒वी मितीति॑ पृथि॒वीम् पृ॑थि॒वी मिति॑ वर्.षा॒ह्वां ॅव॑र्.षा॒ह्वा मिति॑ पृथि॒वीम् पृ॑थि॒वी मिति॑ वर्.षा॒ह्वाम् । \newline
51. इति॑ वर्.षा॒ह्वां ॅव॑र्.षा॒ह्वा मितीति॑ वर्.षा॒ह्वाम् जु॑होति जुहोति वर्.षा॒ह्वा मितीति॑ वर्.षा॒ह्वाम् जु॑होति । \newline
52. व॒र्॒.षा॒ह्वाम् जु॑होति जुहोति वर्.षा॒ह्वां ॅव॑र्.षा॒ह्वाम् जु॑होत्ये॒षैषा जु॑होति वर्.षा॒ह्वां ॅव॑र्.षा॒ह्वाम् जु॑होत्ये॒षा । \newline
53. व॒र्.॒षा॒ह्वामिति॑ वर्.ष - ह्वाम् । \newline
54. जु॒हो॒त्ये॒षैषा जु॑होति जुहोत्ये॒षा वै वा ए॒षा जु॑होति जुहोत्ये॒षा वै । \newline
55. ए॒षा वै वा ए॒षैषा वा ओष॑धीना॒ मोष॑धीनां॒ ॅवा ए॒षैषा वा ओष॑धीनाम् । \newline
56. वा ओष॑धीना॒ मोष॑धीनां॒ ॅवै वा ओष॑धीनां ॅवृष्टि॒वनि॑र् वृष्टि॒वनि॒ रोष॑धीनां॒ ॅवै वा ओष॑धीनां ॅवृष्टि॒वनिः॑ । \newline
57. ओष॑धीनां ॅवृष्टि॒वनि॑र् वृष्टि॒वनि॒ रोष॑धीना॒ मोष॑धीनां ॅवृष्टि॒वनि॒ स्तया॒ तया॑ वृष्टि॒वनि॒ रोष॑धीना॒ मोष॑धीनां ॅवृष्टि॒वनि॒ स्तया᳚ । \newline
58. वृ॒ष्टि॒वनि॒ स्तया॒ तया॑ वृष्टि॒वनि॑र् वृष्टि॒वनि॒ स्तयै॒वैव तया॑ वृष्टि॒वनि॑र् वृष्टि॒वनि॒ स्तयै॒व । \newline
59. वृ॒ष्टि॒वनि॒रिति॑ वृष्टि - वनिः॑ । \newline
60. तयै॒वैव तया॒ तयै॒व वृष्टिं॒ ॅवृष्टि॑ मे॒व तया॒ तयै॒व वृष्टि᳚म् । \newline
61. ए॒व वृष्टिं॒ ॅवृष्टि॑ मे॒वैव वृष्टि॒ मा वृष्टि॑ मे॒वैव वृष्टि॒ मा । \newline
62. वृष्टि॒ मा वृष्टिं॒ ॅवृष्टि॒ मा च्या॑वयति च्यावय॒त्या वृष्टिं॒ ॅवृष्टि॒ मा च्या॑वयति । \newline
63. आ च्या॑वयति च्यावय॒त्या च्या॑वयति॒ ये ये च्या॑वय॒त्या च्या॑वयति॒ ये । \newline
64. च्या॒व॒य॒ति॒ ये ये च्या॑वयति च्यावयति॒ ये दे॒वा दे॒वा ये च्या॑वयति च्यावयति॒ ये दे॒वाः । \newline
65. ये दे॒वा दे॒वा ये ये दे॒वा दि॒विभा॑गा दि॒विभा॑गा दे॒वा ये ये दे॒वा दि॒विभा॑गाः । \newline
66. दे॒वा दि॒विभा॑गा दि॒विभा॑गा दे॒वा दे॒वा दि॒विभा॑गा॒ इतीति॑ दि॒विभा॑गा दे॒वा दे॒वा दि॒विभा॑गा॒ इति॑ । \newline
67. दि॒विभा॑गा॒ इतीति॑ दि॒विभा॑गा दि॒विभा॑गा॒ इति॑ कृष्णाजि॒नम् कृ॑ष्णाजि॒न मिति॑ दि॒विभा॑गा दि॒विभा॑गा॒ इति॑ कृष्णाजि॒नम् । \newline
68. दि॒विभा॑गा॒ इति॑ दि॒वि - भा॒गाः॒ । \newline
69. इति॑ कृष्णाजि॒नम् कृ॑ष्णाजि॒न मितीति॑ कृष्णाजि॒न मवाव॑ कृष्णाजि॒न मितीति॑ कृष्णाजि॒न मव॑ । \newline
70. कृ॒ष्णा॒जि॒न मवाव॑ कृष्णाजि॒नम् कृ॑ष्णाजि॒न मव॑ धूनोति धूनो॒त्यव॑ कृष्णाजि॒नम् कृ॑ष्णाजि॒न मव॑ धूनोति । \newline
71. कृ॒ष्णा॒जि॒नमिति॑ कृष्ण - अ॒जि॒नम् । \newline
72. अव॑ धूनोति धूनो॒ त्यवाव॑ धूनोती॒म इ॒मे धू॑नो॒ त्यवाव॑ धूनोती॒मे । \newline
73. धू॒नो॒ती॒म इ॒मे धू॑नोति धूनोती॒म ए॒वैवे मे धू॑नोति धूनोती॒म ए॒व । \newline
74. इ॒म ए॒वैवे म इ॒म ए॒वास्मा॑ अस्मा ए॒वे म इ॒म ए॒वास्मै᳚ । \newline
75. ए॒वास्मा॑ अस्मा ए॒वैवास्मै॑ लो॒का लो॒का अ॑स्मा ए॒वैवास्मै॑ लो॒काः । \newline
76. अ॒स्मै॒ लो॒का लो॒का अ॑स्मा अस्मै लो॒काः प्री॒ताः प्री॒ता लो॒का अ॑स्मा अस्मै लो॒काः प्री॒ताः । \newline
77. लो॒काः प्री॒ताः प्री॒ता लो॒का लो॒काः प्री॒ता अ॒भीष्टा॑ अ॒भीष्टाः᳚ प्री॒ता लो॒का लो॒काः प्री॒ता अ॒भीष्टाः᳚ । \newline
78. प्री॒ता अ॒भीष्टा॑ अ॒भीष्टाः᳚ प्री॒ताः प्री॒ता अ॒भीष्टा॑ भवन्ति भव न्त्य॒भीष्टाः᳚ प्री॒ताः प्री॒ता अ॒भीष्टा॑ भवन्ति । \newline
79. अ॒भीष्टा॑ भवन्ति भव न्त्य॒भीष्टा॑ अ॒भीष्टा॑ भवन्ति । \newline
80. अ॒भीष्टा॒ इत्य॒भि - इ॒ष्टाः॒ । \newline
81. भ॒व॒न्तीति॑ भवन्ति । \newline
\pagebreak
\markright{ TS 2.4.11.1  \hfill https://www.vedavms.in \hfill}

\section{ TS 2.4.11.1 }

\textbf{TS 2.4.11.1 } \newline
\textbf{Samhita Paata} \newline

सर्वा॑णि॒ छन्दाꣳ॑स्ये॒तस्या॒-मिष्ट्या॑-म॒नूच्या॒नीत्या॑हु-स्त्रि॒ष्टुभो॒ वा ए॒तद्वी॒र्यं॑ ॅयत् क॒कुदु॒ष्णिहा॒ जग॑त्यै॒ यदु॑ष्णिह-क॒कुभा॑व॒न्वाह॒ तेनै॒व सर्वा॑णि॒ छन्दाꣳ॒॒स्यव॑ रुन्धे गाय॒त्री वा ए॒षा यदु॒ष्णिहा॒ यानि॑ च॒त्वार्यद्ध्य॒क्षरा॑णि॒ चतु॑ष्पाद ए॒व ते प॒शवो॒यथा॑ पुरो॒डाशे॑ पुरो॒डाशोऽद्ध्ये॒वमे॒व तद्-यद्-ऋ॒च्यद्ध्य॒क्षरा॑णि॒ यज्जग॑त्या - [  ] \newline

\textbf{Pada Paata} \newline

सर्वा॑णि । छन्दाꣳ॑सि । ए॒तस्या᳚म् । इष्ट्या᳚म् । अ॒नूच्या॒नीत्य॑नु - उच्या॑नि । इति॑ । आ॒हुः॒ । त्रि॒ष्टुभः॑ । वै । ए॒तत् । वी॒र्य᳚म् । यत् । क॒कुत् । उ॒ष्णिहा᳚ । जग॑त्यै । यत् । उ॒ष्णि॒ह॒क॒कुभा॒वित्यु॑ष्णिह - क॒कुभौ᳚ । अ॒न्वाहेत्य॑नु - आह॑ । तेन॑ । ए॒व । सर्वा॑णि । छन्दाꣳ॑सि । अवेति॑ । रु॒न्धे॒ । गा॒य॒त्री । वै । ए॒षा । यत् । उ॒ष्णिहा᳚ । यानि॑ । च॒त्वारि॑ । अधीति॑ । अ॒क्षरा॑णि । चतु॑ष्पाद॒ इति॒ चतुः॑ - पा॒दः॒ । ए॒व । ते । प॒शवः॑ । यथा᳚ । पु॒रो॒डाशे᳚ । पु॒रो॒डाशः॑ । अधीति॑ । ए॒वम् । ए॒व । तत् । यत् । ऋ॒चि । अधीति॑ । अ॒क्षरा॑णि । यत् । जग॑त्या ।  \newline


\textbf{Krama Paata} \newline

सर्वा॑णि॒ छन्दाꣳ॑सि । छन्दाꣳ॑स्ये॒तस्या᳚म् । ए॒तस्या॒मिष्ट्या᳚म् । इष्ट्या॑म॒नूच्या॑नि । अ॒नूच्या॒नीति॑ । अ॒नूच्या॒नीत्य॑नु - उच्या॑नि । इत्या॑हुः । आ॒हु॒स्त्रि॒ष्टुभः॑ । त्रि॒ष्टुभो॒ वै । वा ए॒तत् । ए॒तद् वी॒र्य᳚म् । वी॒र्यं॑ ॅयत् । यत् क॒कुत् । क॒कुदु॒ष्णिहा᳚ । उ॒ष्णिहा॒ जग॑त्यै । जग॑त्यै॒ यत् । यदु॑ष्णिहक॒कुभौ᳚ । उ॒ष्णि॒ह॒क॒कुभा॑व॒न्वाह॑ । उ॒ष्णि॒ह॒क॒कुभा॒वित्यु॑ष्णिह - क॒कुभौ᳚ । अ॒न्वाह॒ तेन॑ । अ॒न्वाहेत्य॑नु - आह॑ । तेनै॒व । ए॒व सर्वा॑णि । सर्वा॑णि॒ छन्दाꣳ॑सि । छन्दाꣳ॒॒स्यव॑ । अव॑ रुन्धे । रु॒न्धे॒ गा॒य॒त्री । गा॒य॒त्री वै । वा ए॒षा । ए॒षा यत् । यदु॒ष्णिहा᳚ । उ॒ष्णिहा॒ यानि॑ । यानि॑ च॒त्वारि॑ । च॒त्वार्यधि॑ । अद्ध्य॒क्षरा॑णि । अ॒क्षरा॑णि॒ चतु॑ष्पादः । चतु॑ष्पाद ए॒व । चतु॑ष्पाद॒ इति॒ चतुः॑ - पा॒दः॒ । ए॒व ते । ते प॒शवः॑ । प॒शवो॒ यथा᳚ । यथा॑ पुरो॒डाशे᳚ । पु॒रो॒डाशे॑ पुरो॒डाशः॑ । पु॒रो॒डाशो॑ऽधि । अद्ध्ये॒वम् । ए॒वमे॒व । ए॒व तत् । तद् यत् । यदृ॒चि । ऋ॒च्यधि॑ । अद्ध्य॒क्षरा॑णि । अ॒क्षरा॑णि॒ यत् । यज्जग॑त्या । जग॑त्या परिद॒द्ध्यात् \newline

\textbf{Jatai Paata} \newline

1. सर्वा॑णि॒ छन्दाꣳ॑सि॒ छन्दाꣳ॑सि॒ सर्वा॑णि॒ सर्वा॑णि॒ छन्दाꣳ॑सि । \newline
2. छन्दाꣳ॑ स्ये॒तस्या॑ मे॒तस्या॒म् छन्दाꣳ॑सि॒ छन्दाꣳ॑ स्ये॒तस्या᳚म् । \newline
3. ए॒तस्या॒ मिष्ट्या॒ मिष्ट्या॑ मे॒तस्या॑ मे॒तस्या॒ मिष्ट्या᳚म् । \newline
4. इष्ट्या॑ म॒नूच्या᳚ न्य॒नूच्या॒नीष्ट्या॒ मिष्ट्या॑ म॒नूच्या॑नि । \newline
5. अ॒नूच्या॒नीती त्य॒नूच्या᳚ न्य॒नूच्या॒नीति॑ । \newline
6. अ॒नूच्या॒नीत्य॑नु - उच्या॑नि । \newline
7. इत्या॑हु राहु॒ रिती त्या॑हुः । \newline
8. आ॒हु॒ स्त्रि॒ष्टुभ॑ स्त्रि॒ष्टुभ॑ आहु राहु स्त्रि॒ष्टुभः॑ । \newline
9. त्रि॒ष्टुभो॒ वै वै त्रि॒ष्टुभ॑ स्त्रि॒ष्टुभो॒ वै । \newline
10. वा ए॒त दे॒तद् वै वा ए॒तत् । \newline
11. ए॒तद् वी॒र्यं॑ ॅवी॒र्य॑ मे॒त दे॒तद् वी॒र्य᳚म् । \newline
12. वी॒र्यं॑ ॅयद् यद् वी॒र्यं॑ ॅवी॒र्यं॑ ॅयत् । \newline
13. यत् क॒कुत् क॒कुद् यद् यत् क॒कुत् । \newline
14. क॒कु दु॒ष्णिहो॒ ष्णिहा॑ क॒कुत् क॒कु दु॒ष्णिहा᳚ । \newline
15. उ॒ष्णिहा॒ जग॑त्यै॒ जग॑त्या उ॒ष्णिहो॒ ष्णिहा॒ जग॑त्यै । \newline
16. जग॑त्यै॒ यद् यज् जग॑त्यै॒ जग॑त्यै॒ यत् । \newline
17. यदु॑ष्णिहक॒कुभा॑ वुष्णिहक॒कुभौ॒ यद् यदु॑ष्णिहक॒कुभौ᳚ । \newline
18. उ॒ष्णि॒ह॒क॒कुभा॑ व॒न्वाहा॒ न्वाहो᳚ष्णिहक॒कुभा॑ वुष्णिहक॒कुभा॑ व॒न्वाह॑ । \newline
19. उ॒ष्णि॒ह॒क॒कुभा॒वित्यु॑ष्णिह - क॒कुभौ᳚ । \newline
20. अ॒न्वाह॒ तेन॒ तेना॒ न्वाहा॒ न्वाह॒ तेन॑ । \newline
21. अ॒न्वाहेत्य॑नु - आह॑ । \newline
22. तेनै॒वैव तेन॒ तेनै॒व । \newline
23. ए॒व सर्वा॑णि॒ सर्वा᳚ ण्ये॒वैव सर्वा॑णि । \newline
24. सर्वा॑णि॒ छन्दाꣳ॑सि॒ छन्दाꣳ॑सि॒ सर्वा॑णि॒ सर्वा॑णि॒ छन्दाꣳ॑सि । \newline
25. छन्दाꣳ॒॒ स्यवाव॒च् छन्दाꣳ॑सि॒ छन्दाꣳ॒॒ स्यव॑ । \newline
26. अव॑ रुन्धे रु॒न्धे ऽवाव॑ रुन्धे । \newline
27. रु॒न्धे॒ गा॒य॒त्री गा॑य॒त्री रु॑न्धे रुन्धे गाय॒त्री । \newline
28. गा॒य॒त्री वै वै गा॑य॒त्री गा॑य॒त्री वै । \newline
29. वा ए॒षैषा वै वा ए॒षा । \newline
30. ए॒षा यद् यदे॒षैषा यत् । \newline
31. यदु॒ष्णिहो॒ ष्णिहा॒ यद् यदु॒ष्णिहा᳚ । \newline
32. उ॒ष्णिहा॒ यानि॒ यान्यु॒ष्णिहो॒ ष्णिहा॒ यानि॑ । \newline
33. यानि॑ च॒त्वारि॑ च॒त्वारि॒ यानि॒ यानि॑ च॒त्वारि॑ । \newline
34. च॒त्वार्यध्यधि॑ च॒त्वारि॑ च॒त्वार्यधि॑ । \newline
35. अध्य॒क्षरा᳚ ण्य॒क्षरा॒ ण्यध्य ध्य॒क्षरा॑णि । \newline
36. अ॒क्षरा॑णि॒ चतु॑ष्पाद॒ श्चतु॑ष्पादो॒ ऽक्षरा᳚ ण्य॒क्षरा॑णि॒ चतु॑ष्पादः । \newline
37. चतु॑ष्पाद ए॒वैव चतु॑ष्पाद॒ श्चतु॑ष्पाद ए॒व । \newline
38. चतु॑ष्पाद॒ इति॒ चतुः॑ - पा॒दः॒ । \newline
39. ए॒व ते त ए॒वैव ते । \newline
40. ते प॒शवः॑ प॒शव॒ स्ते ते प॒शवः॑ । \newline
41. प॒शवो॒ यथा॒ यथा॑ प॒शवः॑ प॒शवो॒ यथा᳚ । \newline
42. यथा॑ पुरो॒डाशे॑ पुरो॒डाशे॒ यथा॒ यथा॑ पुरो॒डाशे᳚ । \newline
43. पु॒रो॒डाशे॑ पुरो॒डाशः॑ पुरो॒डाशः॑ पुरो॒डाशे॑ पुरो॒डाशे॑ पुरो॒डाशः॑ । \newline
44. पु॒रो॒डाशो ऽध्यधि॑ पुरो॒डाशः॑ पुरो॒डाशो ऽधि॑ । \newline
45. अध्ये॒व मे॒व मध्य ध्ये॒वम् । \newline
46. ए॒व मे॒वैवैव मे॒व मे॒व । \newline
47. ए॒व तत् तदे॒वैव तत् । \newline
48. तद् यद् यत् तत् तद् यत् । \newline
49. यदृ॒च्यृ॑चि यद् यदृ॒चि । \newline
50. ऋ॒च्यध्य ध्यृ॒च्यृ॑ च्यधि॑ । \newline
51. अध्य॒क्षरा᳚ ण्य॒क्षरा॒ ण्यध्य ध्य॒क्षरा॑णि । \newline
52. अ॒क्षरा॑णि॒ यद् यद॒क्षरा᳚ ण्य॒क्षरा॑णि॒ यत् । \newline
53. यज् जग॑त्या॒ जग॑त्या॒ यद् यज् जग॑त्या । \newline
54. जग॑त्या परिद॒द्ध्यात् प॑रिद॒द्ध्याज् जग॑त्या॒ जग॑त्या परिद॒द्ध्यात् । \newline

\textbf{Ghana Paata } \newline

1. सर्वा॑णि॒ छन्दाꣳ॑सि॒ छन्दाꣳ॑सि॒ सर्वा॑णि॒ सर्वा॑णि॒ छन्दाꣳ॑ स्ये॒तस्या॑ मे॒तस्या॒म् छन्दाꣳ॑सि॒ सर्वा॑णि॒ सर्वा॑णि॒ छन्दाꣳ॑ स्ये॒तस्या᳚म् । \newline
2. छन्दाꣳ॑ स्ये॒तस्या॑ मे॒तस्या॒म् छन्दाꣳ॑सि॒ छन्दाꣳ॑ स्ये॒तस्या॒ मिष्ट्या॒ मिष्ट्या॑ मे॒तस्या॒म् छन्दाꣳ॑सि॒ छन्दाꣳ॑ स्ये॒तस्या॒ मिष्ट्या᳚म् । \newline
3. ए॒तस्या॒ मिष्ट्या॒ मिष्ट्या॑ मे॒तस्या॑ मे॒तस्या॒ मिष्ट्या॑ म॒नूच्या᳚ न्य॒नूच्या॒नीष्ट्या॑ मे॒तस्या॑ मे॒तस्या॒ मिष्ट्या॑ म॒नूच्या॑नि । \newline
4. इष्ट्या॑ म॒नूच्या᳚ न्य॒नूच्या॒नीष्ट्या॒ मिष्ट्या॑ म॒नूच्या॒नीती त्य॒नूच्या॒नीष्ट्या॒ मिष्ट्या॑ म॒नूच्या॒नीति॑ । \newline
5. अ॒नूच्या॒नीती त्य॒नूच्या᳚ न्य॒नूच्या॒नी त्या॑हु राहु॒ रित्य॒नूच्या᳚ न्य॒नूच्या॒नी त्या॑हुः । \newline
6. अ॒नूच्या॒नीत्य॑नु - उच्या॑नि । \newline
7. इत्या॑हु राहु॒ रिती त्या॑हु स्त्रि॒ष्टुभ॑ स्त्रि॒ष्टुभ॑ आहु॒ रिती त्या॑हु स्त्रि॒ष्टुभः॑ । \newline
8. आ॒हु॒ स्त्रि॒ष्टुभ॑ स्त्रि॒ष्टुभ॑ आहु राहु स्त्रि॒ष्टुभो॒ वै वै त्रि॒ष्टुभ॑ आहु राहु स्त्रि॒ष्टुभो॒ वै । \newline
9. त्रि॒ष्टुभो॒ वै वै त्रि॒ष्टुभ॑ स्त्रि॒ष्टुभो॒ वा ए॒तदे॒तद् वै त्रि॒ष्टुभ॑ स्त्रि॒ष्टुभो॒ वा ए॒तत् । \newline
10. वा ए॒तदे॒तद् वै वा ए॒तद् वी॒र्यं॑ ॅवी॒र्य॑ मे॒तद् वै वा ए॒तद् वी॒र्य᳚म् । \newline
11. ए॒तद् वी॒र्यं॑ ॅवी॒र्य॑ मे॒त दे॒तद् वी॒र्यं॑ ॅयद् यद् वी॒र्य॑ मे॒त दे॒तद् वी॒र्यं॑ ॅयत् । \newline
12. वी॒र्यं॑ ॅयद् यद् वी॒र्यं॑ ॅवी॒र्यं॑ ॅयत् क॒कुत् क॒कुद् यद् वी॒र्यं॑ ॅवी॒र्यं॑ ॅयत् क॒कुत् । \newline
13. यत् क॒कुत् क॒कुद् यद् यत् क॒कु दु॒ष्णिहो॒ष्णिहा॑ क॒कुद् यद् यत् क॒कु दु॒ष्णिहा᳚ । \newline
14. क॒कु दु॒ष्णिहो॒ष्णिहा॑ क॒कुत् क॒कु दु॒ष्णिहा॒ जग॑त्यै॒ जग॑त्या उ॒ष्णिहा॑ क॒कुत् क॒कु दु॒ष्णिहा॒ जग॑त्यै । \newline
15. उ॒ष्णिहा॒ जग॑त्यै॒ जग॑त्या उ॒ष्णिहो॒ष्णिहा॒ जग॑त्यै॒ यद् यज् जग॑त्या उ॒ष्णिहो॒ष्णिहा॒ जग॑त्यै॒ यत् । \newline
16. जग॑त्यै॒ यद् यज् जग॑त्यै॒ जग॑त्यै॒ यदु॑ष्णिहक॒कुभा॑ वुष्णिहक॒कुभौ॒ यज् जग॑त्यै॒ जग॑त्यै॒ यदु॑ष्णिहक॒कुभौ᳚ । \newline
17. यदु॑ष्णिहक॒कुभा॑ वुष्णिहक॒कुभौ॒ यद् यदु॑ष्णिहक॒कुभा॑ व॒न्वाहा॒न्वा हो᳚ष्णिहक॒कुभौ॒ यद् 
यदु॑ष्णिहक॒कुभा॑ व॒न्वाह॑ । \newline
18. उ॒ष्णि॒ह॒क॒कुभा॑ व॒न्वाहा॒ न्वाहो᳚ष्णिहक॒कुभा॑ वुष्णिहक॒कुभा॑ व॒न्वाह॒ तेन॒ तेना॒न्वा 
हो᳚ष्णिहक॒कुभा॑ वुष्णिहक॒कुभा॑ व॒न्वाह॒ तेन॑ । \newline
19. उ॒ष्णि॒ह॒क॒कुभा॒वित्यु॑ष्णिह - क॒कुभौ᳚ । \newline
20. अ॒न्वाह॒ तेन॒ तेना॒न्वाहा॒ न्वाह॒ तेनै॒वैव तेना॒न्वाहा॒ न्वाह॒ तेनै॒व । \newline
21. अ॒न्वाहेत्य॑नु - आह॑ । \newline
22. तेनै॒वैव तेन॒ तेनै॒व सर्वा॑णि॒ सर्वा᳚ण्ये॒व तेन॒ तेनै॒व सर्वा॑णि । \newline
23. ए॒व सर्वा॑णि॒ सर्वा᳚ण्ये॒वैव सर्वा॑णि॒ छन्दाꣳ॑सि॒ छन्दाꣳ॑सि॒ सर्वा᳚ण्ये॒वैव सर्वा॑णि॒ छन्दाꣳ॑सि । \newline
24. सर्वा॑णि॒ छन्दाꣳ॑सि॒ छन्दाꣳ॑सि॒ सर्वा॑णि॒ सर्वा॑णि॒ छन्दाꣳ॒॒ स्यवाव॒च् छन्दाꣳ॑सि॒ सर्वा॑णि॒ सर्वा॑णि॒ छन्दाꣳ॒॒स्यव॑ । \newline
25. छन्दाꣳ॒॒ स्यवाव॒च् छन्दाꣳ॑सि॒ छन्दाꣳ॒॒स्यव॑ रुन्धे रु॒न्धे ऽव॒च् छन्दाꣳ॑सि॒ छन्दाꣳ॒॒स्यव॑ रुन्धे । \newline
26. अव॑ रुन्धे रु॒न्धे ऽवाव॑ रुन्धे गाय॒त्री गा॑य॒त्री रु॒न्धे ऽवाव॑ रुन्धे गाय॒त्री । \newline
27. रु॒न्धे॒ गा॒य॒त्री गा॑य॒त्री रु॑न्धे रुन्धे गाय॒त्री वै वै गा॑य॒त्री रु॑न्धे रुन्धे गाय॒त्री वै । \newline
28. गा॒य॒त्री वै वै गा॑य॒त्री गा॑य॒त्री वा ए॒षैषा वै गा॑य॒त्री गा॑य॒त्री वा ए॒षा । \newline
29. वा ए॒षैषा वै वा ए॒षा यद् यदे॒षा वै वा ए॒षा यत् । \newline
30. ए॒षा यद् यदे॒षैषा यदु॒ष्णि हो॒ष्णिहा॒ यदे॒षैषा यदु॒ष्णिहा᳚ । \newline
31. यदु॒ष्णि हो॒ष्णिहा॒ यद् यदु॒ष्णिहा॒ यानि॒ यान्यु॒ष्णिहा॒ यद् यदु॒ष्णिहा॒ यानि॑ । \newline
32. उ॒ष्णिहा॒ यानि॒ यान्यु॒ष्णि हो॒ष्णिहा॒ यानि॑ च॒त्वारि॑ च॒त्वारि॒ यान्यु॒ष्णि हो॒ष्णिहा॒ यानि॑ च॒त्वारि॑ । \newline
33. यानि॑ च॒त्वारि॑ च॒त्वारि॒ यानि॒ यानि॑ च॒त्वार्यध्यधि॑ च॒त्वारि॒ यानि॒ यानि॑ च॒त्वार्यधि॑ । \newline
34. च॒त्वार्यध्यधि॑ च॒त्वारि॑ च॒त्वार्यध्य॒क्षरा᳚ ण्य॒क्षरा॒ण्यधि॑ च॒त्वारि॑ च॒त्वार्यध्य॒क्षरा॑णि । \newline
35. अध्य॒क्षरा᳚ ण्य॒क्षरा॒ ण्यध्यध्य॒क्षरा॑णि॒ चतु॑ष्पाद॒ श्चतु॑ष्पादो॒ ऽक्षरा॒ ण्यध्यध्य॒क्षरा॑णि॒ चतु॑ष्पादः । \newline
36. अ॒क्षरा॑णि॒ चतु॑ष्पाद॒ श्चतु॑ष्पादो॒ ऽक्षरा᳚ण्य॒क्षरा॑णि॒ चतु॑ष्पाद ए॒वैव चतु॑ष्पादो॒ ऽक्षरा᳚ ण्य॒क्षरा॑णि॒ चतु॑ष्पाद ए॒व । \newline
37. चतु॑ष्पाद ए॒वैव चतु॑ष्पाद॒ श्चतु॑ष्पाद ए॒व ते त ए॒व चतु॑ष्पाद॒ श्चतु॑ष्पाद ए॒व ते । \newline
38. चतु॑ष्पाद॒ इति॒ चतुः॑ - पा॒दः॒ । \newline
39. ए॒व ते त ए॒वैव ते प॒शवः॑ प॒शव॒ स्त ए॒वैव ते प॒शवः॑ । \newline
40. ते प॒शवः॑ प॒शव॒ स्ते ते प॒शवो॒ यथा॒ यथा॑ प॒शव॒ स्ते ते प॒शवो॒ यथा᳚ । \newline
41. प॒शवो॒ यथा॒ यथा॑ प॒शवः॑ प॒शवो॒ यथा॑ पुरो॒डाशे॑ पुरो॒डाशे॒ यथा॑ प॒शवः॑ प॒शवो॒ यथा॑ पुरो॒डाशे᳚ । \newline
42. यथा॑ पुरो॒डाशे॑ पुरो॒डाशे॒ यथा॒ यथा॑ पुरो॒डाशे॑ पुरो॒डाशः॑ पुरो॒डाशः॑ पुरो॒डाशे॒ यथा॒ यथा॑ पुरो॒डाशे॑ पुरो॒डाशः॑ । \newline
43. पु॒रो॒डाशे॑ पुरो॒डाशः॑ पुरो॒डाशः॑ पुरो॒डाशे॑ पुरो॒डाशे॑ पुरो॒डाशो ऽध्यधि॑ पुरो॒डाशः॑ पुरो॒डाशे॑ पुरो॒डाशे॑ पुरो॒डाशो ऽधि॑ । \newline
44. पु॒रो॒डाशो ऽध्यधि॑ पुरो॒डाशः॑ पुरो॒डाशो ऽध्ये॒व मे॒व मधि॑ पुरो॒डाशः॑ पुरो॒डाशो ऽध्ये॒वम् । \newline
45. अध्ये॒व मे॒व मध्यध्ये॒व मे॒वैवैव मध्यध्ये॒व मे॒व । \newline
46. ए॒व मे॒वैवैव मे॒व मे॒व तत् तदे॒वैव मे॒व मे॒व तत् । \newline
47. ए॒व तत् तदे॒वैव तद् यद् यत् तदे॒वैव तद् यत् । \newline
48. तद् यद् यत् तत् तद् यदृ॒च्यृ॑चि यत् तत् तद् यदृ॒चि । \newline
49. यदृ॒च्यृ॑चि यद् यदृ॒च्य ध्यध्यृ॒चि यद् यदृ॒च्यधि॑ । \newline
50. ऋ॒च्य ध्यध्यृ॒च्यृ॑च्य ध्य॒क्षरा᳚ ण्य॒क्षरा॒ ण्यध्यृ॒च्यृ॑च्य ध्य॒क्षरा॑णि । \newline
51. अध्य॒क्षरा᳚ ण्य॒क्षरा॒ ण्यध्यध्य॒ क्षरा॑णि॒ यद् यद॒क्षरा॒ ण्यध्यध्य॒ क्षरा॑णि॒ यत् । \newline
52. अ॒क्षरा॑णि॒ यद् यद॒क्षरा᳚ ण्य॒क्षरा॑णि॒ यज् जग॑त्या॒ जग॑त्या॒ यद॒क्षरा᳚ ण्य॒क्षरा॑णि॒ यज् जग॑त्या । \newline
53. यज् जग॑त्या॒ जग॑त्या॒ यद् यज् जग॑त्या परिद॒द्ध्यात् प॑रिद॒द्ध्याज् जग॑त्या॒ यद् यज् जग॑त्या परिद॒द्ध्यात् । \newline
54. जग॑त्या परिद॒द्ध्यात् प॑रिद॒द्ध्याज् जग॑त्या॒ जग॑त्या परिद॒द्ध्या दन्त॒ मन्त॑म् परिद॒द्ध्याज् जग॑त्या॒ जग॑त्या परिद॒द्ध्या दन्त᳚म् । \newline
\pagebreak
\markright{ TS 2.4.11.2  \hfill https://www.vedavms.in \hfill}

\section{ TS 2.4.11.2 }

\textbf{TS 2.4.11.2 } \newline
\textbf{Samhita Paata} \newline

परिद॒द्ध्यादन्तं॑ ॅय॒ज्ञ्ं ग॑मयेत् त्रि॒ष्टुभा॒ परि॑ दधातीन्द्रि॒यं ॅवै वी॒र्यं॑ त्रि॒ष्टुगि॑न्द्रि॒य ए॒व वी॒र्ये॑ य॒ज्ञ्ं प्रति॑ष्ठापयति॒ नान्तं॑ गमय॒त्यग्ने॒ त्री ते॒ वाजि॑ना॒ त्री ष॒धस्थेति॒ त्रिव॑त्या॒ परि॑ दधाति सरूप॒त्वाय॒ सर्वो॒ वा ए॒ष य॒ज्ञो यत् त्रै॑धात॒वीयं॒ कामा॑य-कामाय॒ प्रयु॑ज्यते॒ सर्वे᳚भ्यो॒ हि कामे᳚भ्यो य॒ज्ञ्ः प्र॑यु॒ज्यते᳚ त्रैधात॒वीये॑न यजेताभि॒चर॒न्थ् सर्वो॒ वा- [  ] \newline

\textbf{Pada Paata} \newline

प॒रि॒द॒द्ध्यादिति॑ परि-द॒द्ध्यात् । अन्त᳚म् । य॒ज्ञ्म् । ग॒म॒ये॒त् । त्रि॒ष्टुभा᳚ । परीति॑ । द॒धा॒ति॒ । इ॒न्द्रि॒यम् । वै । वी॒र्य᳚म् । त्रि॒ष्टुक् । इ॒न्द्रि॒ये । ए॒व । वी॒र्ये᳚ । य॒ज्ञ्म् । प्रतीति॑ । स्था॒प॒य॒ति॒ । न । अन्त᳚म् । ग॒म॒य॒ति॒ । अग्ने᳚ । त्री । ते॒ । वाजि॑ना । त्री । स॒धस्थेति॑ स॒ध - स्था॒ । इति॑ । त्रिव॒त्येति॒ त्रि - व॒त्या॒ । परीति॑ । द॒धा॒ति॒ । स॒रू॒प॒त्वायेति॑ सरूप - त्वाय॑ । सर्वः॑ । वै । ए॒षः । य॒ज्ञ्ः । यत् । त्रै॒धा॒त॒वीय᳚म् । कामा॑य कामा॒येति॒ कामा॑य - का॒मा॒य॒ । प्रेति॑ । यु॒ज्य॒ते॒ । सर्वे᳚भ्यः । हि । कामे᳚भ्यः । य॒ज्ञ्ः । प्र॒यु॒ज्यत॒ इति॑ प्र - यु॒ज्यते᳚ । त्रै॒धा॒त॒वीये॑न । य॒जे॒त॒ । अ॒भि॒चर॒न्नित्य॑भि - चरन्न्॑ । सर्वः॑ । वै । 31(50)  \newline


\textbf{Krama Paata} \newline

प॒रि॒द॒द्ध्यादन्त᳚म् । प॒रि॒द॒द्ध्यादिति॑ परि - द॒द्ध्यात् । अन्तं॑ ॅय॒ज्ञ्म् । य॒ज्ञ्म् ग॑मयेत् । ग॒म॒ये॒त् त्रि॒ष्टुभा᳚ । त्रि॒ष्टुभा॒ परि॑ । परि॑ दधाति । द॒धा॒ती॒न्द्रि॒यम् । इ॒न्द्रि॒यं ॅवै । वै वी॒र्य᳚म् । वी॒र्यं॑ त्रि॒ष्टुक् । त्रि॒ष्टुगि॑न्द्रि॒ये । इ॒न्द्रि॒य ए॒व । ए॒व वी॒र्ये᳚ । वी॒र्ये॑ य॒ज्ञ्म् । य॒ज्ञ्म् प्रति॑ । प्रति॑ ष्ठापयति । स्था॒प॒य॒ति॒ न । नान्त᳚म् । अन्त॑म् गमयति । ग॒म॒य॒त्यग्ने᳚ । अग्ने॒ त्री । त्री ते᳚ । ते॒ वाजि॑ना । वाजि॑ना॒ त्री । त्री ष॒धस्था᳚ । स॒धस्थेति॑ । स॒धस्थेति॑ स॒ध - स्था॒ । इति॒ त्रिव॑त्या । त्रिव॑त्या॒ परि॑ । त्रिव॒त्येति॒ त्रि - व॒त्या॒ । परि॑ दधाति । द॒धा॒ति॒ स॒रू॒प॒त्वाय॑ । स॒रू॒प॒त्वाय॒ सर्वः॑ । स॒रू॒प॒त्वायेति॑ सरूप - त्वाय॑ । सर्वो॒ वै । वा ए॒षः । ए॒ष य॒ज्ञ्ः । य॒ज्ञो यत् । यत् त्रै॑धात॒वीय᳚म् । त्रै॒धा॒त॒वीय॒म् कामा॑यकामाय । कामा॑यकामाय॒ प्र । कामा॑यकामा॒येति॒ कामा॑य - का॒मा॒य॒ । प्र यु॑ज्यते । यु॒ज्य॒ते॒ सर्वे᳚भ्यः । सर्वे᳚भ्यो॒ हि । हि कामे᳚भ्यः । कामे᳚भ्यो य॒ज्ञ्ः । य॒ज्ञ्ः प्र॑यु॒ज्यते᳚ । प्र॒यु॒ज्यते᳚ त्रैधात॒वीये॑न । प्र॒यु॒ज्यत॒ इति॑ प्र - यु॒ज्यते᳚ । त्रै॒धा॒त॒वीये॑न यजेत । य॒जे॒ता॒भि॒चरन्न्॑ । अ॒भि॒चर॒न्थ् सर्वः॑ । अ॒भि॒चर॒न्नित्य॑भि - चरन्न्॑ । सर्वो॒ वै । वा ए॒षः \newline

\textbf{Jatai Paata} \newline

1. प॒रि॒द॒द्ध्या दन्त॒ मन्त॑म् परिद॒द्ध्यात् प॑रिद॒द्ध्या दन्त᳚म् । \newline
2. प॒रि॒द॒द्ध्यादिति॑ परि - द॒द्ध्यात् । \newline
3. अन्तं॑ ॅय॒ज्ञ्ं ॅय॒ज्ञ् मन्त॒ मन्तं॑ ॅय॒ज्ञ्म् । \newline
4. य॒ज्ञ्म् ग॑मयेद् गमयेद् य॒ज्ञ्ं ॅय॒ज्ञ्म् ग॑मयेत् । \newline
5. ग॒म॒ये॒त् त्रि॒ष्टुभा᳚ त्रि॒ष्टुभा॑ गमयेद् गमयेत् त्रि॒ष्टुभा᳚ । \newline
6. त्रि॒ष्टुभा॒ परि॒ परि॑ त्रि॒ष्टुभा᳚ त्रि॒ष्टुभा॒ परि॑ । \newline
7. परि॑ दधाति दधाति॒ परि॒ परि॑ दधाति । \newline
8. द॒धा॒ती॒न्द्रि॒य मि॑न्द्रि॒यम् द॑धाति दधातीन्द्रि॒यम् । \newline
9. इ॒न्द्रि॒यं ॅवै वा इ॑न्द्रि॒य मि॑न्द्रि॒यं ॅवै । \newline
10. वै वी॒र्यं॑ ॅवी॒र्यं॑ ॅवै वै वी॒र्य᳚म् । \newline
11. वी॒र्य॑म् त्रि॒ष्टुक् त्रि॒ष्टुग् वी॒र्यं॑ ॅवी॒र्य॑म् त्रि॒ष्टुक् । \newline
12. त्रि॒ष्टु गि॑न्द्रि॒य इ॑न्द्रि॒ये त्रि॒ष्टुक् त्रि॒ष्टु गि॑न्द्रि॒ये । \newline
13. इ॒न्द्रि॒य ए॒वैवे न्द्रि॒य इ॑न्द्रि॒य ए॒व । \newline
14. ए॒व वी॒र्ये॑ वी॒र्य॑ ए॒वैव वी॒र्ये᳚ । \newline
15. वी॒र्ये॑ य॒ज्ञ्ं ॅय॒ज्ञ्ं ॅवी॒र्ये॑ वी॒र्ये॑ य॒ज्ञ्म् । \newline
16. य॒ज्ञ्म् प्रति॒ प्रति॑ य॒ज्ञ्ं ॅय॒ज्ञ्म् प्रति॑ । \newline
17. प्रति॑ ष्ठापयति स्थापयति॒ प्रति॒ प्रति॑ ष्ठापयति । \newline
18. स्था॒प॒य॒ति॒ न न स्था॑पयति स्थापयति॒ न । \newline
19. नान्त॒ मन्त॒म् न नान्त᳚म् । \newline
20. अन्त॑म् गमयति गमय॒ त्यन्त॒ मन्त॑म् गमयति । \newline
21. ग॒म॒य॒ त्यग्ने ऽग्ने॑ गमयति गमय॒ त्यग्ने᳚ । \newline
22. अग्ने॒ त्री त्र्यग्ने ऽग्ने॒ त्री । \newline
23. त्री ते॑ ते॒ त्री त्री ते᳚ । \newline
24. ते॒ वाजि॑ना॒ वाजि॑ना ते ते॒ वाजि॑ना । \newline
25. वाजि॑ना॒ त्री त्री वाजि॑ना॒ वाजि॑ना॒ त्री । \newline
26. त्री ष॒धस्था॑ स॒धस्था॒ त्री त्री ष॒धस्था᳚ । \newline
27. स॒धस्थेतीति॑ स॒धस्था॑ स॒धस्थेति॑ । \newline
28. स॒धस्थेति॑ स॒ध - स्था॒ । \newline
29. इति॒ त्रिव॑त्या॒ त्रिव॒ त्येतीति॒ त्रिव॑त्या । \newline
30. त्रिव॑त्या॒ परि॒ परि॒ त्रिव॑त्या॒ त्रिव॑त्या॒ परि॑ । \newline
31. त्रिव॒त्येति॒ त्रि - व॒त्या॒ । \newline
32. परि॑ दधाति दधाति॒ परि॒ परि॑ दधाति । \newline
33. द॒धा॒ति॒ स॒रू॒प॒त्वाय॑ सरूप॒त्वाय॑ दधाति दधाति सरूप॒त्वाय॑ । \newline
34. स॒रू॒प॒त्वाय॒ सर्वः॒ सर्वः॑ सरूप॒त्वाय॑ सरूप॒त्वाय॒ सर्वः॑ । \newline
35. स॒रू॒प॒त्वायेति॑ सरूप - त्वाय॑ । \newline
36. सर्वो॒ वै वै सर्वः॒ सर्वो॒ वै । \newline
37. वा ए॒ष ए॒ष वै वा ए॒षः । \newline
38. ए॒ष य॒ज्ञो य॒ज्ञ् ए॒ष ए॒ष य॒ज्ञ्ः । \newline
39. य॒ज्ञो यद् यद् य॒ज्ञो य॒ज्ञो यत् । \newline
40. यत् त्रै॑धात॒वीय॑म् त्रैधात॒वीयं॒ ॅयद् यत् त्रै॑धात॒वीय᳚म् । \newline
41. त्रै॒धा॒त॒वीय॒म् कामा॑यकामाय॒ कामा॑यकामाय त्रैधात॒वीय॑म् त्रैधात॒वीय॒म् कामा॑यकामाय । \newline
42. कामा॑यकामाय॒ प्र प्र कामा॑यकामाय॒ कामा॑यकामाय॒ प्र । \newline
43. कामा॑यकामा॒येति॒ कामा॑य - का॒मा॒य॒ । \newline
44. प्र यु॑ज्यते युज्यते॒ प्र प्र यु॑ज्यते । \newline
45. यु॒ज्य॒ते॒ सर्वे᳚भ्यः॒ सर्वे᳚भ्यो युज्यते युज्यते॒ सर्वे᳚भ्यः । \newline
46. सर्वे᳚भ्यो॒ हि हि सर्वे᳚भ्यः॒ सर्वे᳚भ्यो॒ हि । \newline
47. हि कामे᳚भ्यः॒ कामे᳚भ्यो॒ हि हि कामे᳚भ्यः । \newline
48. कामे᳚भ्यो य॒ज्ञो य॒ज्ञ्ः कामे᳚भ्यः॒ कामे᳚भ्यो य॒ज्ञ्ः । \newline
49. य॒ज्ञ्ः प्र॑यु॒ज्यते᳚ प्रयु॒ज्यते॑ य॒ज्ञो य॒ज्ञ्ः प्र॑यु॒ज्यते᳚ । \newline
50. प्र॒यु॒ज्यते᳚ त्रैधात॒वीये॑न त्रैधात॒वीये॑न प्रयु॒ज्यते᳚ प्रयु॒ज्यते᳚ त्रैधात॒वीये॑न । \newline
51. प्र॒यु॒ज्यत॒ इति॑ प्र - यु॒ज्यते᳚ । \newline
52. त्रै॒धा॒त॒वीये॑न यजेत यजेत त्रैधात॒वीये॑न त्रैधात॒वीये॑न यजेत । \newline
53. य॒जे॒ता॒ भि॒चर॑न् नभि॒चर॑न्. यजेत यजेता भि॒चरन्न्॑ । \newline
54. अ॒भि॒चर॒न् थ्सर्वः॒ सर्वो॑ ऽभि॒चर॑न् नभि॒चर॒न् थ्सर्वः॑ । \newline
55. अ॒भि॒चर॒न्नित्य॑भि - चरन्न्॑ । \newline
56. सर्वो॒ वै वै सर्वः॒ सर्वो॒ वै । \newline
57. वा ए॒ष ए॒ष वै वा ए॒षः । \newline

\textbf{Ghana Paata } \newline

1. प॒रि॒द॒द्ध्या दन्त॒ मन्त॑म् परिद॒द्ध्यात् प॑रिद॒द्ध्या दन्तं॑ ॅय॒ज्ञ्ं ॅय॒ज्ञ् मन्त॑म् परिद॒द्ध्यात् प॑रिद॒द्ध्या दन्तं॑ ॅय॒ज्ञ्म् । \newline
2. प॒रि॒द॒द्ध्यादिति॑ परि - द॒द्ध्यात् । \newline
3. अन्तं॑ ॅय॒ज्ञ्ं ॅय॒ज्ञ् मन्त॒ मन्तं॑ ॅय॒ज्ञ्म् ग॑मयेद् गमयेद् य॒ज्ञ् मन्त॒ मन्तं॑ ॅय॒ज्ञ्म् ग॑मयेत् । \newline
4. य॒ज्ञ्म् ग॑मयेद् गमयेद् य॒ज्ञ्ं ॅय॒ज्ञ्म् ग॑मयेत् त्रि॒ष्टुभा᳚ त्रि॒ष्टुभा॑ गमयेद् य॒ज्ञ्ं ॅय॒ज्ञ्म् ग॑मयेत् त्रि॒ष्टुभा᳚ । \newline
5. ग॒म॒ये॒त् त्रि॒ष्टुभा᳚ त्रि॒ष्टुभा॑ गमयेद् गमयेत् त्रि॒ष्टुभा॒ परि॒ परि॑ त्रि॒ष्टुभा॑ गमयेद् गमयेत् त्रि॒ष्टुभा॒ परि॑ । \newline
6. त्रि॒ष्टुभा॒ परि॒ परि॑ त्रि॒ष्टुभा᳚ त्रि॒ष्टुभा॒ परि॑ दधाति दधाति॒ परि॑ त्रि॒ष्टुभा᳚ त्रि॒ष्टुभा॒ परि॑ दधाति । \newline
7. परि॑ दधाति दधाति॒ परि॒ परि॑ दधातीन्द्रि॒य मि॑न्द्रि॒यम् द॑धाति॒ परि॒ परि॑ दधातीन्द्रि॒यम् । \newline
8. द॒धा॒ती॒न्द्रि॒य मि॑न्द्रि॒यम् द॑धाति दधातीन्द्रि॒यं ॅवै वा इ॑न्द्रि॒यम् द॑धाति दधातीन्द्रि॒यं ॅवै । \newline
9. इ॒न्द्रि॒यं ॅवै वा इ॑न्द्रि॒य मि॑न्द्रि॒यं ॅवै वी॒र्यं॑ ॅवी॒र्यं॑ ॅवा इ॑न्द्रि॒य मि॑न्द्रि॒यं ॅवै वी॒र्य᳚म् । \newline
10. वै वी॒र्यं॑ ॅवी॒र्यं॑ ॅवै वै वी॒र्य॑म् त्रि॒ष्टुक् त्रि॒ष्टुग् वी॒र्यं॑ ॅवै वै वी॒र्य॑म् त्रि॒ष्टुक् । \newline
11. वी॒र्य॑म् त्रि॒ष्टुक् त्रि॒ष्टुग् वी॒र्यं॑ ॅवी॒र्य॑म् त्रि॒ष्टु गि॑न्द्रि॒य इ॑न्द्रि॒ये त्रि॒ष्टुग् वी॒र्यं॑ ॅवी॒र्य॑म् त्रि॒ष्टु गि॑न्द्रि॒ये । \newline
12. त्रि॒ष्टु गि॑न्द्रि॒य इ॑न्द्रि॒ये त्रि॒ष्टुक् त्रि॒ष्टु गि॑न्द्रि॒य ए॒वैवे न्द्रि॒ये त्रि॒ष्टुक् त्रि॒ष्टु गि॑न्द्रि॒य ए॒व । \newline
13. इ॒न्द्रि॒य ए॒वैवे न्द्रि॒य इ॑न्द्रि॒य ए॒व वी॒र्ये॑ वी॒र्य॑ ए॒वे न्द्रि॒य इ॑न्द्रि॒य ए॒व वी॒र्ये᳚ । \newline
14. ए॒व वी॒र्ये॑ वी॒र्य॑ ए॒वैव वी॒र्ये॑ य॒ज्ञ्ं ॅय॒ज्ञ्ं ॅवी॒र्य॑ ए॒वैव वी॒र्ये॑ य॒ज्ञ्म् । \newline
15. वी॒र्ये॑ य॒ज्ञ्ं ॅय॒ज्ञ्ं ॅवी॒र्ये॑ वी॒र्ये॑ य॒ज्ञ्म् प्रति॒ प्रति॑ य॒ज्ञ्ं ॅवी॒र्ये॑ वी॒र्ये॑ य॒ज्ञ्म् प्रति॑ । \newline
16. य॒ज्ञ्म् प्रति॒ प्रति॑ य॒ज्ञ्ं ॅय॒ज्ञ्म् प्रति॑ ष्ठापयति स्थापयति॒ प्रति॑ य॒ज्ञ्ं ॅय॒ज्ञ्म् प्रति॑ ष्ठापयति । \newline
17. प्रति॑ ष्ठापयति स्थापयति॒ प्रति॒ प्रति॑ ष्ठापयति॒ न न स्था॑पयति॒ प्रति॒ प्रति॑ ष्ठापयति॒ न । \newline
18. स्था॒प॒य॒ति॒ न न स्था॑पयति स्थापयति॒ नान्त॒ मन्त॒म् न स्था॑पयति स्थापयति॒ नान्त᳚म् । \newline
19. नान्त॒ मन्त॒म् न नान्त॑म् गमयति गमय॒ त्यन्त॒म् न नान्त॑म् गमयति । \newline
20. अन्त॑म् गमयति गमय॒ त्यन्त॒ मन्त॑म् गमय॒ त्यग्ने ऽग्ने॑ गमय॒ त्यन्त॒ मन्त॑म् गमय॒ त्यग्ने᳚ । \newline
21. ग॒म॒य॒ त्यग्ने ऽग्ने॑ गमयति गमय॒ त्यग्ने॒ त्री त्र्यग्ने॑ गमयति गमय॒ त्यग्ने॒ त्री । \newline
22. अग्ने॒ त्री त्र्यग्ने ऽग्ने॒ त्री ते॑ ते॒ त्र्यग्ने ऽग्ने॒ त्री ते᳚ । \newline
23. त्री ते॑ ते॒ त्री त्री ते॒ वाजि॑ना॒ वाजि॑ना ते॒ त्री त्री ते॒ वाजि॑ना । \newline
24. ते॒ वाजि॑ना॒ वाजि॑ना ते ते॒ वाजि॑ना॒ त्री त्री वाजि॑ना ते ते॒ वाजि॑ना॒ त्री । \newline
25. वाजि॑ना॒ त्री त्री वाजि॑ना॒ वाजि॑ना॒ त्री ष॒धस्था॑ स॒धस्था॒ त्री वाजि॑ना॒ वाजि॑ना॒ त्री ष॒धस्था᳚ । \newline
26. त्री ष॒धस्था॑ स॒धस्था॒ त्री त्री ष॒धस्थेतीति॑ स॒धस्था॒ त्री त्री ष॒धस्थेति॑ । \newline
27. स॒धस्थेतीति॑ स॒धस्था॑ स॒धस्थेति॒ त्रिव॑त्या॒ त्रिव॒त्येति॑ स॒धस्था॑ स॒धस्थेति॒ त्रिव॑त्या । \newline
28. स॒धस्थेति॑ स॒ध - स्था॒ । \newline
29. इति॒ त्रिव॑त्या॒ त्रिव॒त्येतीति॒ त्रिव॑त्या॒ परि॒ परि॒ त्रिव॒त्येतीति॒ त्रिव॑त्या॒ परि॑ । \newline
30. त्रिव॑त्या॒ परि॒ परि॒ त्रिव॑त्या॒ त्रिव॑त्या॒ परि॑ दधाति दधाति॒ परि॒ त्रिव॑त्या॒ त्रिव॑त्या॒ परि॑ दधाति । \newline
31. त्रिव॒त्येति॒ त्रि - व॒त्या॒ । \newline
32. परि॑ दधाति दधाति॒ परि॒ परि॑ दधाति सरूप॒त्वाय॑ सरूप॒त्वाय॑ दधाति॒ परि॒ परि॑ दधाति सरूप॒त्वाय॑ । \newline
33. द॒धा॒ति॒ स॒रू॒प॒त्वाय॑ सरूप॒त्वाय॑ दधाति दधाति सरूप॒त्वाय॒ सर्वः॒ सर्वः॑ सरूप॒त्वाय॑ दधाति दधाति सरूप॒त्वाय॒ सर्वः॑ । \newline
34. स॒रू॒प॒त्वाय॒ सर्वः॒ सर्वः॑ सरूप॒त्वाय॑ सरूप॒त्वाय॒ सर्वो॒ वै वै सर्वः॑ सरूप॒त्वाय॑ सरूप॒त्वाय॒ सर्वो॒ वै । \newline
35. स॒रू॒प॒त्वायेति॑ सरूप - त्वाय॑ । \newline
36. सर्वो॒ वै वै सर्वः॒ सर्वो॒ वा ए॒ष ए॒ष वै सर्वः॒ सर्वो॒ वा ए॒षः । \newline
37. वा ए॒ष ए॒ष वै वा ए॒ष य॒ज्ञो य॒ज्ञ् ए॒ष वै वा ए॒ष य॒ज्ञ्ः । \newline
38. ए॒ष य॒ज्ञो य॒ज्ञ् ए॒ष ए॒ष य॒ज्ञो यद् यद् य॒ज्ञ् ए॒ष ए॒ष य॒ज्ञो यत् । \newline
39. य॒ज्ञो यद् यद् य॒ज्ञो य॒ज्ञो यत् त्रै॑धात॒वीय॑म् त्रैधात॒वीयं॒ ॅयद् य॒ज्ञो य॒ज्ञो यत् त्रै॑धात॒वीय᳚म् । \newline
40. यत् त्रै॑धात॒वीय॑म् त्रैधात॒वीयं॒ ॅयद् यत् त्रै॑धात॒वीय॒म् कामा॑यकामाय॒ कामा॑यकामाय त्रैधात॒वीयं॒ ॅयद् यत् त्रै॑धात॒वीय॒म् कामा॑यकामाय । \newline
41. त्रै॒धा॒त॒वीय॒म् कामा॑यकामाय॒ कामा॑यकामाय त्रैधात॒वीय॑म् त्रैधात॒वीय॒म् कामा॑यकामाय॒ प्र प्र कामा॑यकामाय त्रैधात॒वीय॑म् त्रैधात॒वीय॒म् कामा॑यकामाय॒ प्र । \newline
42. कामा॑यकामाय॒ प्र प्र कामा॑यकामाय॒ कामा॑यकामाय॒ प्र यु॑ज्यते युज्यते॒ प्र कामा॑यकामाय॒ कामा॑यकामाय॒ प्र यु॑ज्यते । \newline
43. कामा॑यकामा॒येति॒ कामा॑य - का॒मा॒य॒ । \newline
44. प्र यु॑ज्यते युज्यते॒ प्र प्र यु॑ज्यते॒ सर्वे᳚भ्यः॒ सर्वे᳚भ्यो युज्यते॒ प्र प्र यु॑ज्यते॒ सर्वे᳚भ्यः । \newline
45. यु॒ज्य॒ते॒ सर्वे᳚भ्यः॒ सर्वे᳚भ्यो युज्यते युज्यते॒ सर्वे᳚भ्यो॒ हि हि सर्वे᳚भ्यो युज्यते युज्यते॒ सर्वे᳚भ्यो॒ हि । \newline
46. सर्वे᳚भ्यो॒ हि हि सर्वे᳚भ्यः॒ सर्वे᳚भ्यो॒ हि कामे᳚भ्यः॒ कामे᳚भ्यो॒ हि सर्वे᳚भ्यः॒ सर्वे᳚भ्यो॒ हि कामे᳚भ्यः । \newline
47. हि कामे᳚भ्यः॒ कामे᳚भ्यो॒ हि हि कामे᳚भ्यो य॒ज्ञो य॒ज्ञ्ः कामे᳚भ्यो॒ हि हि कामे᳚भ्यो य॒ज्ञ्ः । \newline
48. कामे᳚भ्यो य॒ज्ञो य॒ज्ञ्ः कामे᳚भ्यः॒ कामे᳚भ्यो य॒ज्ञ्ः प्र॑यु॒ज्यते᳚ प्रयु॒ज्यते॑ य॒ज्ञ्ः कामे᳚भ्यः॒ कामे᳚भ्यो य॒ज्ञ्ः प्र॑यु॒ज्यते᳚ । \newline
49. य॒ज्ञ्ः प्र॑यु॒ज्यते᳚ प्रयु॒ज्यते॑ य॒ज्ञो य॒ज्ञ्ः प्र॑यु॒ज्यते᳚ त्रैधात॒वीये॑न त्रैधात॒वीये॑न प्रयु॒ज्यते॑ य॒ज्ञो य॒ज्ञ्ः प्र॑यु॒ज्यते᳚ त्रैधात॒वीये॑न । \newline
50. प्र॒यु॒ज्यते᳚ त्रैधात॒वीये॑न त्रैधात॒वीये॑न प्रयु॒ज्यते᳚ प्रयु॒ज्यते᳚ त्रैधात॒वीये॑न यजेत यजेत त्रैधात॒वीये॑न प्रयु॒ज्यते᳚ प्रयु॒ज्यते᳚ त्रैधात॒वीये॑न यजेत । \newline
51. प्र॒यु॒ज्यत॒ इति॑ प्र - यु॒ज्यते᳚ । \newline
52. त्रै॒धा॒त॒वीये॑न यजेत यजेत त्रैधात॒वीये॑न त्रैधात॒वीये॑न यजेता भि॒चर॑न् नभि॒चर॑न्. यजेत त्रैधात॒वीये॑न त्रैधात॒वीये॑न यजेता भि॒चरन्न्॑ । \newline
53. य॒जे॒ता॒ भि॒चर॑न् नभि॒चर॑न्. यजेत यजेता भि॒चर॒न् थ्सर्वः॒ सर्वो॑ ऽभि॒चर॑न्. यजेत यजेता भि॒चर॒न् थ्सर्वः॑ । \newline
54. अ॒भि॒चर॒न् थ्सर्वः॒ सर्वो॑ ऽभि॒चर॑न् नभि॒चर॒न् थ्सर्वो॒ वै वै सर्वो॑ ऽभि॒चर॑न् नभि॒चर॒न् थ्सर्वो॒ वै । \newline
55. अ॒भि॒चर॒न्नित्य॑भि - चरन्न्॑ । \newline
56. सर्वो॒ वै वै सर्वः॒ सर्वो॒ वा ए॒ष ए॒ष वै सर्वः॒ सर्वो॒ वा ए॒षः । \newline
57. वा ए॒ष ए॒ष वै वा ए॒ष य॒ज्ञो य॒ज्ञ् ए॒ष वै वा ए॒ष य॒ज्ञ्ः । \newline
\pagebreak
\markright{ TS 2.4.11.3  \hfill https://www.vedavms.in \hfill}

\section{ TS 2.4.11.3 }

\textbf{TS 2.4.11.3 } \newline
\textbf{Samhita Paata} \newline

ए॒ष य॒ज्ञो यत् त्रै॑धात॒वीयꣳ॒॒ सर्वे॑णै॒वैनं॑ ॅय॒ज्ञेना॒भि च॑रति स्तृणु॒त ए॒वैन॑मे॒तयै॒व य॑जेताभिच॒र्यमा॑णः॒ सर्वो॒ वा ए॒ष य॒ज्ञो यत् त्रै॑धात॒वीयꣳ॒॒ सर्वे॑णै॒व य॒ज्ञेन॑ यजते॒ नैन॑मभि॒चर॑006छ्;᳚थ् स्तृणुत ए॒तयै॒व य॑जेत स॒हस्रे॑ण य॒क्ष्यमा॑णः॒ प्रजा॑तमे॒वैन॑द्-ददात्ये॒तयै॒व य॑जेत स॒हस्रे॑णेजा॒नोऽन्तं॒ ॅवा ए॒ष प॑शू॒नां ग॑च्छति॒ - [  ] \newline

\textbf{Pada Paata} \newline

ए॒षः । य॒ज्ञ्ः । यत् । त्रै॒धा॒त॒वीय᳚म् ।   सर्वे॑ण । ए॒व । ए॒न॒म् । य॒ज्ञेन॑ । अ॒भीति॑ । च॒र॒ति॒ । स्तृ॒णु॒ते । ए॒व । ए॒न॒म् । ए॒तया᳚ ।   ए॒व । य॒जे॒त॒ । अ॒भि॒च॒र्यमा॑ण॒ इत्य॑भि-च॒र्यमा॑णः । सर्वः॑ । वै । ए॒षः । य॒ज्ञ्ः । यत् । त्रै॒धा॒त॒वीय᳚म् । सर्वे॑ण । ए॒व । य॒ज्ञेन॑ । य॒ज॒ते॒ । न । ए॒न॒म् । अ॒भि॒चर॒न्नित्य॑भि - चरन्न्॑ । स्तृ॒णु॒ते॒ । ए॒तया᳚ । ए॒व । य॒जे॒त॒ । स॒हस्रे॑ण । य॒क्ष्यमा॑णः । प्रजा॑त॒मिति॒ प्र - जा॒त॒म् । ए॒व । ए॒न॒त् । द॒दा॒ति॒ । ए॒तया᳚ । ए॒व । य॒जे॒त॒ । स॒हस्रे॑ण । ई॒जा॒नः । अन्त᳚म् । वै । ए॒षः । प॒शू॒नाम् । ग॒च्छ॒ति॒ ।  \newline


\textbf{Krama Paata} \newline

ए॒ष य॒ज्ञ्ः । य॒ज्ञो यत् । यत् त्रै॑धात॒वीय᳚म् । त्रै॒धा॒त॒वीयꣳ॒॒ सर्वे॑ण । सर्वे॑णै॒व । ए॒वैन᳚म् । ए॒नं॒ ॅय॒ज्ञेन॑ । य॒ज्ञेना॒भि । अ॒भि च॑रति । च॒र॒ति॒ स्तृ॒णु॒ते । स्तृ॒णु॒त ए॒व । ए॒वैन᳚म् । ए॒न॒मे॒तया᳚ । ए॒तयै॒व । ए॒व य॑जेत । य॒जे॒ता॒भि॒च॒र्यमा॑णः । अ॒भि॒च॒र्यमा॑णः॒ सर्वः॑ । अ॒भि॒च॒र्यमा॑ण॒ इत्य॑भि - च॒र्यमा॑णः । सर्वो॒ वै । वा ए॒षः । ए॒ष य॒ज्ञ्ः । य॒ज्ञो यत् । यत् त्रै॑धात॒वीय᳚म् । त्रै॒धा॒त॒वीयꣳ॒॒ सर्वे॑ण । सर्वे॑णै॒व । ए॒व य॒ज्ञेन॑ । य॒ज्ञेन॑ यजते । य॒ज॒ते॒ न । नैन᳚म् । ए॒न॒म॒भि॒चरन्न्॑ । अ॒भि॒चर᳚न्थ् स्तृणुते । अ॒भि॒चर॒न्नित्य॑भि - चरन्न्॑ । स्तृ॒णु॒त॒ ए॒तया᳚ । ए॒तयै॒व । ए॒व य॑जेत । य॒जे॒त॒ स॒हस्रे॑ण । स॒हस्रे॑ण य॒क्ष्यमा॑णः । य॒क्ष्यमा॑णः॒ प्रजा॑तम् । प्रजा॑तमे॒व । प्रजा॑त॒मिति॒ प्र - जा॒त॒म् । ए॒वैन॑त् । ए॒न॒द् द॒दा॒ति॒ । द॒दा॒त्ये॒तया᳚ । ए॒तयै॒व । ए॒व य॑जेत । य॒जे॒त॒ स॒हस्रे॑ण । स॒हस्रे॑णेजा॒नः । ई॒जा॒नोऽन्त᳚म् । अन्तं॒ ॅवै । वा ए॒षः । ए॒ष प॑शू॒नाम् । प॒शू॒नाम् ग॑च्छति । ग॒च्छ॒ति॒ यः \newline

\textbf{Jatai Paata} \newline

1. ए॒ष य॒ज्ञो य॒ज्ञ् ए॒ष ए॒ष य॒ज्ञ्ः । \newline
2. य॒ज्ञो यद् यद् य॒ज्ञो य॒ज्ञो यत् । \newline
3. यत् त्रै॑धात॒वीय॑म् त्रैधात॒वीयं॒ ॅयद् यत् त्रै॑धात॒वीय᳚म् । \newline
4. त्रै॒धा॒त॒वीयꣳ॒॒ सर्वे॑ण॒ सर्वे॑ण त्रैधात॒वीय॑म् त्रैधात॒वीयꣳ॒॒ सर्वे॑ण । \newline
5. सर्वे॑णै॒वैव सर्वे॑ण॒ सर्वे॑णै॒व । \newline
6. ए॒वैन॑ मेन मे॒वैवैन᳚म् । \newline
7. ए॒नं॒ ॅय॒ज्ञेन॑ य॒ज्ञेनै॑न मेनं ॅय॒ज्ञेन॑ । \newline
8. य॒ज्ञेना॒भ्य॑भि य॒ज्ञेन॑ य॒ज्ञेना॒भि । \newline
9. अ॒भि च॑रति चर त्य॒भ्य॑भि च॑रति । \newline
10. च॒र॒ति॒ स्तृ॒णु॒ते स्तृ॑णु॒ते च॑रति चरति स्तृणु॒ते । \newline
11. स्तृ॒णु॒त ए॒वैव स्तृ॑णु॒ते स्तृ॑णु॒त ए॒व । \newline
12. ए॒वैन॑ मेन मे॒वैवैन᳚म् । \newline
13. ए॒न॒ मे॒त यै॒तयै॑न मेन मे॒तया᳚ । \newline
14. ए॒त यै॒ वैवैत यै॒त यै॒व । \newline
15. ए॒व य॑जेत यजेतै॒वैव य॑जेत । \newline
16. य॒जे॒ता॒ भि॒च॒र्यमा॑णो ऽभिच॒र्यमा॑णो यजेत यजेता भिच॒र्यमा॑णः । \newline
17. अ॒भि॒च॒र्यमा॑णः॒ सर्वः॒ सर्वो॑ ऽभिच॒र्यमा॑णो ऽभिच॒र्यमा॑णः॒ सर्वः॑ । \newline
18. अ॒भि॒च॒र्यमा॑ण॒ इत्य॑भि - च॒र्यमा॑णः । \newline
19. सर्वो॒ वै वै सर्वः॒ सर्वो॒ वै । \newline
20. वा ए॒ष ए॒ष वै वा ए॒षः । \newline
21. ए॒ष य॒ज्ञो य॒ज्ञ् ए॒ष ए॒ष य॒ज्ञ्ः । \newline
22. य॒ज्ञो यद् यद् य॒ज्ञो य॒ज्ञो यत् । \newline
23. यत् त्रै॑धात॒वीय॑म् त्रैधात॒वीयं॒ ॅयद् यत् त्रै॑धात॒वीय᳚म् । \newline
24. त्रै॒धा॒त॒वीयꣳ॒॒ सर्वे॑ण॒ सर्वे॑ण त्रैधात॒वीय॑म् त्रैधात॒वीयꣳ॒॒ सर्वे॑ण । \newline
25. सर्वे॑णै॒वैव सर्वे॑ण॒ सर्वे॑णै॒व । \newline
26. ए॒व य॒ज्ञेन॑ य॒ज्ञेनै॒वैव य॒ज्ञेन॑ । \newline
27. य॒ज्ञेन॑ यजते यजते य॒ज्ञेन॑ य॒ज्ञेन॑ यजते । \newline
28. य॒ज॒ते॒ न न य॑जते यजते॒ न । \newline
29. नैन॑ मेन॒म् न नैन᳚म् । \newline
30. ए॒न॒ म॒भि॒चर॑न् नभि॒चर॑न् नेन मेन मभि॒चरन्न्॑ । \newline
31. अ॒भि॒चरन्᳚ थ्स्तृणुते स्तृणुते ऽभि॒चर॑न् नभि॒चरन्᳚ थ्स्तृणुते । \newline
32. अ॒भि॒चर॒न्नित्य॑भि - चरन्न्॑ । \newline
33. स्तृ॒णु॒त॒ ए॒त यै॒तया᳚ स्तृणुते स्तृणुत ए॒तया᳚ । \newline
34. ए॒त यै॒वैवैत यै॒तयै॒व । \newline
35. ए॒व य॑जेत यजेतै॒वैव य॑जेत । \newline
36. य॒जे॒त॒ स॒हस्रे॑ण स॒हस्रे॑ण यजेत यजेत स॒हस्रे॑ण । \newline
37. स॒हस्रे॑ण य॒क्ष्यमा॑णो य॒क्ष्यमा॑णः स॒हस्रे॑ण स॒हस्रे॑ण य॒क्ष्यमा॑णः । \newline
38. य॒क्ष्यमा॑णः॒ प्रजा॑त॒म् प्रजा॑तं ॅय॒क्ष्यमा॑णो य॒क्ष्यमा॑णः॒ प्रजा॑तम् । \newline
39. प्रजा॑त मे॒वैव प्रजा॑त॒म् प्रजा॑त मे॒व । \newline
40. प्रजा॑त॒मिति॒ प्र - जा॒त॒म् । \newline
41. ए॒वैन॑ देन दे॒वै वैन॑त् । \newline
42. ए॒न॒द् द॒दा॒ति॒ द॒दा॒ त्ये॒न॒ दे॒न॒द् द॒दा॒ति॒ । \newline
43. द॒दा॒ त्ये॒त यै॒तया॑ ददाति ददा त्ये॒तया᳚ । \newline
44. ए॒त यै॒वैवैत यै॒तयै॒व । \newline
45. ए॒व य॑जेत यजेतै॒वैव य॑जेत । \newline
46. य॒जे॒त॒ स॒हस्रे॑ण स॒हस्रे॑ण यजेत यजेत स॒हस्रे॑ण । \newline
47. स॒हस्रे॑णेजा॒न ई॑जा॒नः स॒हस्रे॑ण स॒हस्रे॑णेजा॒नः । \newline
48. ई॒जा॒नो ऽन्त॒ मन्त॑ मीजा॒न ई॑जा॒नो ऽन्त᳚म् । \newline
49. अन्तं॒ ॅवै वा अन्त॒ मन्तं॒ ॅवै । \newline
50. वा ए॒ष ए॒ष वै वा ए॒षः । \newline
51. ए॒ष प॑शू॒नाम् प॑शू॒ना मे॒ष ए॒ष प॑शू॒नाम् । \newline
52. प॒शू॒नाम् ग॑च्छति गच्छति पशू॒नाम् प॑शू॒नाम् ग॑च्छति । \newline
53. ग॒च्छ॒ति॒ यो यो ग॑च्छति गच्छति॒ यः । \newline

\textbf{Ghana Paata } \newline

1. ए॒ष य॒ज्ञो य॒ज्ञ् ए॒ष ए॒ष य॒ज्ञो यद् यद् य॒ज्ञ् ए॒ष ए॒ष य॒ज्ञो यत् । \newline
2. य॒ज्ञो यद् यद् य॒ज्ञो य॒ज्ञो यत् त्रै॑धात॒वीय॑म् त्रैधात॒वीयं॒ ॅयद् य॒ज्ञो य॒ज्ञो यत् त्रै॑धात॒वीय᳚म् । \newline
3. यत् त्रै॑धात॒वीय॑म् त्रैधात॒वीयं॒ ॅयद् यत् त्रै॑धात॒वीयꣳ॒॒ सर्वे॑ण॒ सर्वे॑ण त्रैधात॒वीयं॒ ॅयद् यत् त्रै॑धात॒वीयꣳ॒॒ सर्वे॑ण । \newline
4. त्रै॒धा॒त॒वीयꣳ॒॒ सर्वे॑ण॒ सर्वे॑ण त्रैधात॒वीय॑म् त्रैधात॒वीयꣳ॒॒ सर्वे॑णै॒वैव सर्वे॑ण त्रैधात॒वीय॑म् त्रैधात॒वीयꣳ॒॒ सर्वे॑णै॒व । \newline
5. सर्वे॑णै॒वैव सर्वे॑ण॒ सर्वे॑णै॒वैन॑ मेन मे॒व सर्वे॑ण॒ सर्वे॑णै॒वैन᳚म् । \newline
6. ए॒वैन॑ मेन मे॒वैवैनं॑ ॅय॒ज्ञेन॑ य॒ज्ञेनै॑न मे॒वैवैनं॑ ॅय॒ज्ञेन॑ । \newline
7. ए॒नं॒ ॅय॒ज्ञेन॑ य॒ज्ञेनै॑न मेनं ॅय॒ज्ञेना॒भ्य॑भि य॒ज्ञेनै॑न मेनं ॅय॒ज्ञेना॒भि । \newline
8. य॒ज्ञेना॒भ्य॑भि य॒ज्ञेन॑ य॒ज्ञेना॒भि च॑रति चरत्य॒भि य॒ज्ञेन॑ य॒ज्ञेना॒भि च॑रति । \newline
9. अ॒भि च॑रति चरत्य॒भ्य॑भि च॑रति स्तृणु॒ते स्तृ॑णु॒ते च॑रत्य॒भ्य॑भि च॑रति स्तृणु॒ते । \newline
10. च॒र॒ति॒ स्तृ॒णु॒ते स्तृ॑णु॒ते च॑रति चरति स्तृणु॒त ए॒वैव स्तृ॑णु॒ते च॑रति चरति स्तृणु॒त ए॒व । \newline
11. स्तृ॒णु॒त ए॒वैव स्तृ॑णु॒ते स्तृ॑णु॒त ए॒वैन॑ मेन मे॒व स्तृ॑णु॒ते स्तृ॑णु॒त ए॒वैन᳚म् । \newline
12. ए॒वैन॑ मेन मे॒वैवैन॑ मे॒तयै॒तयै॑न मे॒वैवैन॑ मे॒तया᳚ । \newline
13. ए॒न॒ मे॒तयै॒तयै॑न मेन मे॒त यै॒वैवैतयै॑न मेन मे॒तयै॒व । \newline
14. ए॒तयै॒वै वैत यै॒तयै॒व य॑जेत यजेतै॒वैत यै॒तयै॒व य॑जेत । \newline
15. ए॒व य॑जेत यजेतै॒वैव य॑जेता भिच॒र्यमा॑णो ऽभिच॒र्यमा॑णो यजेतै॒वैव य॑जेता भिच॒र्यमा॑णः । \newline
16. य॒जे॒ता॒ भि॒च॒र्यमा॑णो ऽभिच॒र्यमा॑णो यजेत यजेता भिच॒र्यमा॑णः॒ सर्वः॒ सर्वो॑ ऽभिच॒र्यमा॑णो यजेत यजेता भिच॒र्यमा॑णः॒ सर्वः॑ । \newline
17. अ॒भि॒च॒र्यमा॑णः॒ सर्वः॒ सर्वो॑ ऽभिच॒र्यमा॑णो ऽभिच॒र्यमा॑णः॒ सर्वो॒ वै वै सर्वो॑ ऽभिच॒र्यमा॑णो ऽभिच॒र्यमा॑णः॒ सर्वो॒ वै । \newline
18. अ॒भि॒च॒र्यमा॑ण॒ इत्य॑भि - च॒र्यमा॑णः । \newline
19. सर्वो॒ वै वै सर्वः॒ सर्वो॒ वा ए॒ष ए॒ष वै सर्वः॒ सर्वो॒ वा ए॒षः । \newline
20. वा ए॒ष ए॒ष वै वा ए॒ष य॒ज्ञो य॒ज्ञ् ए॒ष वै वा ए॒ष य॒ज्ञ्ः । \newline
21. ए॒ष य॒ज्ञो य॒ज्ञ् ए॒ष ए॒ष य॒ज्ञो यद् यद् य॒ज्ञ् ए॒ष ए॒ष य॒ज्ञो यत् । \newline
22. य॒ज्ञो यद् यद् य॒ज्ञो य॒ज्ञो यत् त्रै॑धात॒वीय॑म् त्रैधात॒वीयं॒ ॅयद् य॒ज्ञो य॒ज्ञो यत् त्रै॑धात॒वीय᳚म् । \newline
23. यत् त्रै॑धात॒वीय॑म् त्रैधात॒वीयं॒ ॅयद् यत् त्रै॑धात॒वीयꣳ॒॒ सर्वे॑ण॒ सर्वे॑ण त्रैधात॒वीयं॒ ॅयद् यत् त्रै॑धात॒वीयꣳ॒॒ सर्वे॑ण । \newline
24. त्रै॒धा॒त॒वीयꣳ॒॒ सर्वे॑ण॒ सर्वे॑ण त्रैधात॒वीय॑म् त्रैधात॒वीयꣳ॒॒ सर्वे॑णै॒वैव सर्वे॑ण त्रैधात॒वीय॑म् त्रैधात॒वीयꣳ॒॒ सर्वे॑णै॒व । \newline
25. सर्वे॑णै॒वैव सर्वे॑ण॒ सर्वे॑णै॒व य॒ज्ञेन॑ य॒ज्ञेनै॒व सर्वे॑ण॒ सर्वे॑णै॒व य॒ज्ञेन॑ । \newline
26. ए॒व य॒ज्ञेन॑ य॒ज्ञेनै॒वैव य॒ज्ञेन॑ यजते यजते य॒ज्ञेनै॒वैव य॒ज्ञेन॑ यजते । \newline
27. य॒ज्ञेन॑ यजते यजते य॒ज्ञेन॑ य॒ज्ञेन॑ यजते॒ न न य॑जते य॒ज्ञेन॑ य॒ज्ञेन॑ यजते॒ न । \newline
28. य॒ज॒ते॒ न न य॑जते यजते॒ नैन॑ मेन॒म् न य॑जते यजते॒ नैन᳚म् । \newline
29. नैन॑ मेन॒म् न नैन॑ मभि॒चर॑न् नभि॒चर॑न् नेन॒म् न नैन॑ मभि॒चरन्न्॑ । \newline
30. ए॒न॒ म॒भि॒चर॑न् नभि॒चर॑न् नेन मेन मभि॒चरन्᳚ थ्स्तृणुते स्तृणुते ऽभि॒चर॑न् नेन मेन मभि॒चरन्᳚ थ्स्तृणुते । \newline
31. अ॒भि॒चरन्᳚ थ्स्तृणुते स्तृणुते ऽभि॒चर॑न् नभि॒चरन्᳚ थ्स्तृणुत ए॒तयै॒तया᳚ स्तृणुते ऽभि॒चर॑न् नभि॒चरन्᳚ थ्स्तृणुत ए॒तया᳚ । \newline
32. अ॒भि॒चर॒न्नित्य॑भि - चरन्न्॑ । \newline
33. स्तृ॒णु॒त॒ ए॒तयै॒तया᳚ स्तृणुते स्तृणुत ए॒तयै॒वै वैतया᳚ स्तृणुते स्तृणुत ए॒तयै॒व । \newline
34. ए॒तयै॒वै वैत यै॒तयै॒व य॑जेत यजेतै॒वैत यै॒तयै॒व य॑जेत । \newline
35. ए॒व य॑जेत यजेतै॒वैव य॑जेत स॒हस्रे॑ण स॒हस्रे॑ण यजेतै॒वैव य॑जेत स॒हस्रे॑ण । \newline
36. य॒जे॒त॒ स॒हस्रे॑ण स॒हस्रे॑ण यजेत यजेत स॒हस्रे॑ण य॒क्ष्यमा॑णो य॒क्ष्यमा॑णः स॒हस्रे॑ण यजेत यजेत स॒हस्रे॑ण य॒क्ष्यमा॑णः । \newline
37. स॒हस्रे॑ण य॒क्ष्यमा॑णो य॒क्ष्यमा॑णः स॒हस्रे॑ण स॒हस्रे॑ण य॒क्ष्यमा॑णः॒ प्रजा॑त॒म् प्रजा॑तं ॅय॒क्ष्यमा॑णः स॒हस्रे॑ण स॒हस्रे॑ण य॒क्ष्यमा॑णः॒ प्रजा॑तम् । \newline
38. य॒क्ष्यमा॑णः॒ प्रजा॑त॒म् प्रजा॑तं ॅय॒क्ष्यमा॑णो य॒क्ष्यमा॑णः॒ प्रजा॑त मे॒वैव प्रजा॑तं ॅय॒क्ष्यमा॑णो य॒क्ष्यमा॑णः॒ प्रजा॑त मे॒व । \newline
39. प्रजा॑त मे॒वैव प्रजा॑त॒म् प्रजा॑त मे॒वैन॑ देन दे॒व प्रजा॑त॒म् प्रजा॑त मे॒वैन॑त् । \newline
40. प्रजा॑त॒मिति॒ प्र - जा॒त॒म् । \newline
41. ए॒वैन॑ देन दे॒वैवैन॑द् ददाति ददा त्येन दे॒वैवैन॑द् ददाति । \newline
42. ए॒न॒द् द॒दा॒ति॒ द॒दा॒ त्ये॒न॒ दे॒न॒द् द॒दा॒ त्ये॒तयै॒तया॑ ददा त्येन देनद् ददा त्ये॒तया᳚ । \newline
43. द॒दा॒ त्ये॒तयै॒तया॑ ददाति ददा त्ये॒तयै॒वै वैतया॑ ददाति ददा त्ये॒तयै॒व । \newline
44. ए॒तयै॒वै वैत यै॒तयै॒व य॑जेत यजेतै॒वैत यै॒तयै॒व य॑जेत । \newline
45. ए॒व य॑जेत यजेतै॒वैव य॑जेत स॒हस्रे॑ण स॒हस्रे॑ण यजेतै॒वैव य॑जेत स॒हस्रे॑ण । \newline
46. य॒जे॒त॒ स॒हस्रे॑ण स॒हस्रे॑ण यजेत यजेत स॒हस्रे॑णेजा॒न ई॑जा॒नः स॒हस्रे॑ण यजेत यजेत स॒हस्रे॑णेजा॒नः । \newline
47. स॒हस्रे॑णेजा॒न ई॑जा॒नः स॒हस्रे॑ण स॒हस्रे॑णेजा॒नो ऽन्त॒ मन्त॑ मीजा॒नः स॒हस्रे॑ण स॒हस्रे॑णेजा॒नो ऽन्त᳚म् । \newline
48. ई॒जा॒नो ऽन्त॒ मन्त॑ मीजा॒न ई॑जा॒नो ऽन्तं॒ ॅवै वा अन्त॑ मीजा॒न ई॑जा॒नो ऽन्तं॒ ॅवै । \newline
49. अन्तं॒ ॅवै वा अन्त॒ मन्तं॒ ॅवा ए॒ष ए॒ष वा अन्त॒ मन्तं॒ ॅवा ए॒षः । \newline
50. वा ए॒ष ए॒ष वै वा ए॒ष प॑शू॒नाम् प॑शू॒ना मे॒ष वै वा ए॒ष प॑शू॒नाम् । \newline
51. ए॒ष प॑शू॒नाम् प॑शू॒ना मे॒ष ए॒ष प॑शू॒नाम् ग॑च्छति गच्छति पशू॒ना मे॒ष ए॒ष प॑शू॒नाम् ग॑च्छति । \newline
52. प॒शू॒नाम् ग॑च्छति गच्छति पशू॒नाम् प॑शू॒नाम् ग॑च्छति॒ यो यो ग॑च्छति पशू॒नाम् प॑शू॒नाम् ग॑च्छति॒ यः । \newline
53. ग॒च्छ॒ति॒ यो यो ग॑च्छति गच्छति॒ यः स॒हस्रे॑ण स॒हस्रे॑ण॒ यो ग॑च्छति गच्छति॒ यः स॒हस्रे॑ण । \newline
\pagebreak
\markright{ TS 2.4.11.4  \hfill https://www.vedavms.in \hfill}

\section{ TS 2.4.11.4 }

\textbf{TS 2.4.11.4 } \newline
\textbf{Samhita Paata} \newline

यः स॒हस्रे॑ण॒ यज॑ते प्र॒जाप॑तिः॒ खलु॒ वै प॒शून॑सृजत॒ ताꣳस्त्रै॑धात॒ वीये॑-नै॒वासृ॑जत॒ य ए॒वं ॅवि॒द्वाꣳ स्त्रै॑धात॒वीये॑नप॒शुका॑मो॒ यज॑ते॒ यस्मा॑दे॒व योनेः᳚ प्र॒जाप॑तिः प॒शूनसृ॑जत॒ तस्मा॑दे॒वैना᳚न्थ् सृजत॒ उपै॑न॒मुत्त॑रꣳ स॒हस्रं॑ नमति दे॒वता᳚भ्यो॒ वा ए॒ष आ वृ॑श्च्यते॒ यो य॒क्ष्य इत्यु॒क्त्वा न यज॑ते त्रैधात॒वीये॑न यजेत॒ सर्वो॒ वा ए॒ष य॒ज्ञो - [  ] \newline

\textbf{Pada Paata} \newline

यः । स॒हस्रे॑ण । यज॑ते । प्र॒जाप॑ति॒रिति॑ प्र॒जा - प॒तिः॒ । खलु॑ । वै । प॒शून् । अ॒सृ॒ज॒त॒ । तान् । त्रै॒धा॒त॒वीये॑न । ए॒व । अ॒सृ॒ज॒त॒ । यः ।   ए॒वम् । वि॒द्वान् । त्रै॒धा॒त॒वीये॑न । प॒शुका॑म॒ इति॑ प॒शु - का॒मः॒ । यज॑ते । यस्मा᳚त् । ए॒व । योनेः᳚ । प्र॒जाप॑ति॒रिति॑ प्र॒जा-प॒तिः॒ । प॒शून् । असृ॑जत । तस्मा᳚त् । ए॒व । ए॒ना॒न् । सृ॒ज॒ते॒ । उपेति॑ । ए॒न॒म् । उत्त॑र॒मित्युत् - त॒र॒म् । स॒हस्र᳚म् । न॒म॒ति॒ । दे॒वता᳚भ्यः । वै । ए॒षः । एति॑ । वृ॒श्च्य॒ते॒ । यः । य॒क्ष्ये । इति॑ । उ॒क्त्वा । न । यज॑ते । त्रै॒धा॒त॒वीये॑न । य॒जे॒त॒ । सर्वः॑ । वै । ए॒षः । य॒ज्ञ्ः ।  \newline


\textbf{Krama Paata} \newline

यः स॒हस्रे॑ण । स॒हस्रे॑ण॒ यज॑ते । यज॑ते प्र॒जाप॑तिः । प्र॒जाप॑तिः॒ खलु॑ । प्र॒जाप॑ति॒रिति॑ प्र॒जा - प॒तिः॒ । खलु॒ वै । वै प॒शून् । प॒शून॑सृजत । अ॒सृ॒ज॒त॒ तान् । ताꣳ स्त्रै॑धात॒वीये॑न । त्रै॒धा॒त॒वीये॑नै॒व । ए॒वासृ॑जत । अ॒सृ॒ज॒त॒ यः । य ए॒वम् । ए॒वं ॅवि॒द्वान् । वि॒द्वाꣳ स्त्रै॑धात॒वीये॑न । त्रै॒धा॒त॒वीये॑न प॒शुका॑मः । प॒शुका॑मो॒ यज॑ते । प॒शुका॑म॒ इति॑ प॒शु - का॒मः॒ । यज॑ते॒ यस्मा᳚त् । यस्मा॑दे॒व । ए॒व योनेः᳚ । योनेः᳚ प्र॒जाप॑तिः । प्र॒जाप॑तिः प॒शून् । प्र॒जाप॑ति॒रिति॑ प्र॒जा - प॒तिः॒ । प॒शूनसृ॑जत । असृ॑जत॒ तस्मा᳚त् । तस्मा॑दे॒व । ए॒वैनान्॑ । ए॒ना॒न्थ् सृ॒ज॒ते॒ । सृ॒ज॒त॒ उप॑ । उपै॑नम् । ए॒न॒मुत्त॑रम् । उत्त॑रꣳ स॒हस्र᳚म् । उत्त॑र॒मित्युत् - त॒र॒म् । स॒हस्र॑म् नमति । न॒म॒ति॒ दे॒वता᳚भ्यः । दे॒वता᳚भ्यो॒ वै । वा ए॒षः । ए॒ष आ । आ वृ॑श्च्यते । वृ॒श्च्य॒ते॒ यः । यो य॒क्ष्ये । य॒क्ष्य इति॑ । इत्यु॒क्त्वा । उ॒क्त्वा न । न यज॑ते । यज॑ते त्रैधात॒वीये॑न । त्रै॒धा॒त॒वीये॑न यजेत । य॒जे॒त॒ सर्वः॑ । सर्वो॒ वै । वा ए॒षः । ए॒ष य॒ज्ञ्ः । य॒ज्ञो यत् \newline

\textbf{Jatai Paata} \newline

1. यः स॒हस्रे॑ण स॒हस्रे॑ण॒ यो यः स॒हस्रे॑ण । \newline
2. स॒हस्रे॑ण॒ यज॑ते॒ यज॑ते स॒हस्रे॑ण स॒हस्रे॑ण॒ यज॑ते । \newline
3. यज॑ते प्र॒जाप॑तिः प्र॒जाप॑ति॒र् यज॑ते॒ यज॑ते प्र॒जाप॑तिः । \newline
4. प्र॒जाप॑तिः॒ खलु॒ खलु॑ प्र॒जाप॑तिः प्र॒जाप॑तिः॒ खलु॑ । \newline
5. प्र॒जाप॑ति॒रिति॑ प्र॒जा - प॒तिः॒ । \newline
6. खलु॒ वै वै खलु॒ खलु॒ वै । \newline
7. वै प॒शून् प॒शून्. वै वै प॒शून् । \newline
8. प॒शू न॑सृजता सृजत प॒शून् प॒शू न॑सृजत । \newline
9. अ॒सृ॒ज॒त॒ ताꣳ स्ता न॑सृजता सृजत॒ तान् । \newline
10. ताꣳ स्त्रै॑धात॒वीये॑न त्रैधात॒वीये॑न॒ ताꣳ स्ताꣳ स्त्रै॑धात॒वीये॑न । \newline
11. त्रै॒धा॒त॒वीये॑ नै॒वैव त्रै॑धात॒वीये॑न त्रैधात॒वीये॑ नै॒व । \newline
12. ए॒वासृ॑जता सृजतै॒वैवा सृ॑जत । \newline
13. अ॒सृ॒ज॒त॒ यो यो॑ ऽसृजता सृजत॒ यः । \newline
14. य ए॒व मे॒वं ॅयो य ए॒वम् । \newline
15. ए॒वं ॅवि॒द्वान्. वि॒द्वा ने॒व मे॒वं ॅवि॒द्वान् । \newline
16. वि॒द्वाꣳ स्त्रै॑धात॒वीये॑न त्रैधात॒वीये॑न वि॒द्वान्. वि॒द्वाꣳ स्त्रै॑धात॒वीये॑न । \newline
17. त्रै॒धा॒त॒वीये॑न प॒शुका॑मः प॒शुका॑म स्त्रैधात॒वीये॑न त्रैधात॒वीये॑न प॒शुका॑मः । \newline
18. प॒शुका॑मो॒ यज॑ते॒ यज॑ते प॒शुका॑मः प॒शुका॑मो॒ यज॑ते । \newline
19. प॒शुका॑म॒ इति॑ प॒शु - का॒मः॒ । \newline
20. यज॑ते॒ यस्मा॒द् यस्मा॒द् यज॑ते॒ यज॑ते॒ यस्मा᳚त् । \newline
21. यस्मा॑ दे॒वैव यस्मा॒द् यस्मा॑ दे॒व । \newline
22. ए॒व योने॒र् योने॑ रे॒वैव योनेः᳚ । \newline
23. योनेः᳚ प्र॒जाप॑तिः प्र॒जाप॑ति॒र् योने॒र् योनेः᳚ प्र॒जाप॑तिः । \newline
24. प्र॒जाप॑तिः प॒शून् प॒शून् प्र॒जाप॑तिः प्र॒जाप॑तिः प॒शून् । \newline
25. प्र॒जाप॑ति॒रिति॑ प्र॒जा - प॒तिः॒ । \newline
26. प॒शू नसृ॑ज॒ता सृ॑जत प॒शून् प॒शू नसृ॑जत । \newline
27. असृ॑जत॒ तस्मा॒त् तस्मा॒ दसृ॑ज॒ता सृ॑जत॒ तस्मा᳚त् । \newline
28. तस्मा॑ दे॒वैव तस्मा॒त् तस्मा॑ दे॒व । \newline
29. ए॒वैना॑ नेना ने॒वैवैनान्॑ । \newline
30. ए॒ना॒न् थ्सृ॒ज॒ते॒ सृ॒ज॒त॒ ए॒ना॒ ने॒ना॒न् थ्सृ॒ज॒ते॒ । \newline
31. सृ॒ज॒त॒ उपोप॑ सृजते सृजत॒ उप॑ । \newline
32. उपै॑न मेन॒ मुपोपै॑नम् । \newline
33. ए॒न॒ मुत्त॑र॒ मुत्त॑र मेन मेन॒ मुत्त॑रम् । \newline
34. उत्त॑रꣳ स॒हस्रꣳ॑ स॒हस्र॒ मुत्त॑र॒ मुत्त॑रꣳ स॒हस्र᳚म् । \newline
35. उत्त॑र॒मित्युत् - त॒र॒म् । \newline
36. स॒हस्र॑म् नमति नमति स॒हस्रꣳ॑ स॒हस्र॑म् नमति । \newline
37. न॒म॒ति॒ दे॒वता᳚भ्यो दे॒वता᳚भ्यो नमति नमति दे॒वता᳚भ्यः । \newline
38. दे॒वता᳚भ्यो॒ वै वै दे॒वता᳚भ्यो दे॒वता᳚भ्यो॒ वै । \newline
39. वा ए॒ष ए॒ष वै वा ए॒षः । \newline
40. ए॒ष ऐष ए॒ष आ । \newline
41. आ वृ॑श्च्यते वृश्च्यत॒ आ वृ॑श्च्यते । \newline
42. वृ॒श्च्य॒ते॒ यो यो वृ॑श्च्यते वृश्च्यते॒ यः । \newline
43. यो य॒क्ष्ये य॒क्ष्ये यो यो य॒क्ष्ये । \newline
44. य॒क्ष्य इतीति॑ य॒क्ष्ये य॒क्ष्य इति॑ । \newline
45. इत्यु॒क्त्वो क्त्वेती त्यु॒क्त्वा । \newline
46. उ॒क्त्वा न नोक्त्वो क्त्वा न । \newline
47. न यज॑ते॒ यज॑ते॒ न न यज॑ते । \newline
48. यज॑ते त्रैधात॒वीये॑न त्रैधात॒वीये॑न॒ यज॑ते॒ यज॑ते त्रैधात॒वीये॑न । \newline
49. त्रै॒धा॒त॒वीये॑न यजेत यजेत त्रैधात॒वीये॑न त्रैधात॒वीये॑न यजेत । \newline
50. य॒जे॒त॒ सर्वः॒ सर्वो॑ यजेत यजेत॒ सर्वः॑ । \newline
51. सर्वो॒ वै वै सर्वः॒ सर्वो॒ वै । \newline
52. वा ए॒ष ए॒ष वै वा ए॒षः । \newline
53. ए॒ष य॒ज्ञो य॒ज्ञ् ए॒ष ए॒ष य॒ज्ञ्ः । \newline
54. य॒ज्ञो यद् यद् य॒ज्ञो य॒ज्ञो यत् । \newline

\textbf{Ghana Paata } \newline

1. यः स॒हस्रे॑ण स॒हस्रे॑ण॒ यो यः स॒हस्रे॑ण॒ यज॑ते॒ यज॑ते स॒हस्रे॑ण॒ यो यः स॒हस्रे॑ण॒ यज॑ते । \newline
2. स॒हस्रे॑ण॒ यज॑ते॒ यज॑ते स॒हस्रे॑ण स॒हस्रे॑ण॒ यज॑ते प्र॒जाप॑तिः प्र॒जाप॑ति॒र् यज॑ते स॒हस्रे॑ण स॒हस्रे॑ण॒ यज॑ते प्र॒जाप॑तिः । \newline
3. यज॑ते प्र॒जाप॑तिः प्र॒जाप॑ति॒र् यज॑ते॒ यज॑ते प्र॒जाप॑तिः॒ खलु॒ खलु॑ प्र॒जाप॑ति॒र् यज॑ते॒ यज॑ते प्र॒जाप॑तिः॒ खलु॑ । \newline
4. प्र॒जाप॑तिः॒ खलु॒ खलु॑ प्र॒जाप॑तिः प्र॒जाप॑तिः॒ खलु॒ वै वै खलु॑ प्र॒जाप॑तिः प्र॒जाप॑तिः॒ खलु॒ वै । \newline
5. प्र॒जाप॑ति॒रिति॑ प्र॒जा - प॒तिः॒ । \newline
6. खलु॒ वै वै खलु॒ खलु॒ वै प॒शून् प॒शून्. वै खलु॒ खलु॒ वै प॒शून् । \newline
7. वै प॒शून् प॒शून्. वै वै प॒शू न॑सृजता सृजत प॒शून्. वै वै प॒शू न॑सृजत । \newline
8. प॒शू न॑सृजता सृजत प॒शून् प॒शू न॑सृजत॒ ताꣳ स्ता न॑सृजत प॒शून् प॒शू न॑सृजत॒ तान् । \newline
9. अ॒सृ॒ज॒त॒ ताꣳ स्ता न॑सृजता सृजत॒ ताꣳ स्त्रै॑धात॒वीये॑न त्रैधात॒वीये॑न॒ ता न॑सृजता सृजत॒ 
ताꣳ स्त्रै॑धात॒वीये॑न । \newline
10. ताꣳ स्त्रै॑धात॒वीये॑न त्रैधात॒वीये॑न॒ ताꣳ स्ताꣳ स्त्रै॑धात॒वीये॑ नै॒वैव त्रै॑धात॒वीये॑न॒ ताꣳ 
स्ताꣳ स्त्रै॑धात॒वीये॑ नै॒व । \newline
11. त्रै॒धा॒त॒वीये॑ नै॒वैव त्रै॑धात॒वीये॑न त्रैधात॒वीये॑ नै॒वा सृ॑जता सृजतै॒व त्रै॑धात॒वीये॑न त्रैधात॒वीये॑ नै॒वा सृ॑जत । \newline
12. ए॒वासृ॑जता सृजतै॒वैवा सृ॑जत॒ यो यो॑ ऽसृजतै॒ वैवा सृ॑जत॒ यः । \newline
13. अ॒सृ॒ज॒त॒ यो यो॑ ऽसृजता सृजत॒ य ए॒व मे॒वं ॅयो॑ ऽसृजता सृजत॒ य ए॒वम् । \newline
14. य ए॒व मे॒वं ॅयो य ए॒वं ॅवि॒द्वान्. वि॒द्वा ने॒वं ॅयो य ए॒वं ॅवि॒द्वान् । \newline
15. ए॒वं ॅवि॒द्वान्. वि॒द्वा ने॒व मे॒वं ॅवि॒द्वाꣳ स्त्रै॑धात॒वीये॑न त्रैधात॒वीये॑न वि॒द्वा ने॒व मे॒वं 
ॅवि॒द्वाꣳ स्त्रै॑धात॒वीये॑न । \newline
16. वि॒द्वाꣳ स्त्रै॑धात॒वीये॑न त्रैधात॒वीये॑न वि॒द्वान्. वि॒द्वाꣳ स्त्रै॑धात॒वीये॑न प॒शुका॑मः प॒शुका॑म स्त्रैधात॒वीये॑न वि॒द्वान्. वि॒द्वाꣳ स्त्रै॑धात॒वीये॑न प॒शुका॑मः । \newline
17. त्रै॒धा॒त॒वीये॑न प॒शुका॑मः प॒शुका॑म स्त्रैधात॒वीये॑न त्रैधात॒वीये॑न प॒शुका॑मो॒ यज॑ते॒ यज॑ते प॒शुका॑म स्त्रैधात॒वीये॑न त्रैधात॒वीये॑न प॒शुका॑मो॒ यज॑ते । \newline
18. प॒शुका॑मो॒ यज॑ते॒ यज॑ते प॒शुका॑मः प॒शुका॑मो॒ यज॑ते॒ यस्मा॒द् यस्मा॒द् यज॑ते प॒शुका॑मः प॒शुका॑मो॒ यज॑ते॒ यस्मा᳚त् । \newline
19. प॒शुका॑म॒ इति॑ प॒शु - का॒मः॒ । \newline
20. यज॑ते॒ यस्मा॒द् यस्मा॒द् यज॑ते॒ यज॑ते॒ यस्मा॑ दे॒वैव यस्मा॒द् यज॑ते॒ यज॑ते॒ यस्मा॑दे॒व । \newline
21. यस्मा॑ दे॒वैव यस्मा॒द् यस्मा॑ दे॒व योने॒र् योने॑रे॒व यस्मा॒द् यस्मा॑दे॒व योनेः᳚ । \newline
22. ए॒व योने॒र् योने॑ रे॒वैव योनेः᳚ प्र॒जाप॑तिः प्र॒जाप॑ति॒र् योने॑ रे॒वैव योनेः᳚ प्र॒जाप॑तिः । \newline
23. योनेः᳚ प्र॒जाप॑तिः प्र॒जाप॑ति॒र् योने॒र् योनेः᳚ प्र॒जाप॑तिः प॒शून् प॒शून् प्र॒जाप॑ति॒र् योने॒र् योनेः᳚ प्र॒जाप॑तिः प॒शून् । \newline
24. प्र॒जाप॑तिः प॒शून् प॒शून् प्र॒जाप॑तिः प्र॒जाप॑तिः प॒शू नसृ॑ज॒ता सृ॑जत प॒शून् प्र॒जाप॑तिः प्र॒जाप॑तिः प॒शू नसृ॑जत । \newline
25. प्र॒जाप॑ति॒रिति॑ प्र॒जा - प॒तिः॒ । \newline
26. प॒शू नसृ॑ज॒ता सृ॑जत प॒शून् प॒शू नसृ॑जत॒ तस्मा॒त् तस्मा॒द सृ॑जत प॒शून् प॒शू नसृ॑जत॒ तस्मा᳚त् । \newline
27. असृ॑जत॒ तस्मा॒त् तस्मा॒ दसृ॑ज॒ता सृ॑जत॒ तस्मा॑ दे॒वैव तस्मा॒ दसृ॑ज॒ता सृ॑जत॒ तस्मा॑ दे॒व । \newline
28. तस्मा॑ दे॒वैव तस्मा॒त् तस्मा॑ दे॒वैना॑ नेना ने॒व तस्मा॒त् तस्मा॑ दे॒वैनान्॑ । \newline
29. ए॒वैना॑ नेना ने॒वैवैनान्᳚ थ्सृजते सृजत एना ने॒वैवैनान्᳚ थ्सृजते । \newline
30. ए॒ना॒न् थ्सृ॒ज॒ते॒ सृ॒ज॒त॒ ए॒ना॒ ने॒ना॒न् थ्सृ॒ज॒त॒ उपोप॑ सृजत एना नेनान् थ्सृजत॒ उप॑ । \newline
31. सृ॒ज॒त॒ उपोप॑ सृजते सृजत॒ उपै॑न मेन॒ मुप॑ सृजते सृजत॒ उपै॑नम् । \newline
32. उपै॑न मेन॒ मुपोपै॑न॒ मुत्त॑र॒ मुत्त॑र मेन॒ मुपोपै॑न॒ मुत्त॑रम् । \newline
33. ए॒न॒ मुत्त॑र॒ मुत्त॑र मेन मेन॒ मुत्त॑रꣳ स॒हस्रꣳ॑ स॒हस्र॒ मुत्त॑र मेन मेन॒ मुत्त॑रꣳ स॒हस्र᳚म् । \newline
34. उत्त॑रꣳ स॒हस्रꣳ॑ स॒हस्र॒ मुत्त॑र॒ मुत्त॑रꣳ स॒हस्र॑म् नमति नमति स॒हस्र॒ मुत्त॑र॒ मुत्त॑रꣳ स॒हस्र॑म् नमति । \newline
35. उत्त॑र॒मित्युत् - त॒र॒म् । \newline
36. स॒हस्र॑म् नमति नमति स॒हस्रꣳ॑ स॒हस्र॑म् नमति दे॒वता᳚भ्यो दे॒वता᳚भ्यो नमति स॒हस्रꣳ॑ स॒हस्र॑म् नमति दे॒वता᳚भ्यः । \newline
37. न॒म॒ति॒ दे॒वता᳚भ्यो दे॒वता᳚भ्यो नमति नमति दे॒वता᳚भ्यो॒ वै वै दे॒वता᳚भ्यो नमति नमति दे॒वता᳚भ्यो॒ वै । \newline
38. दे॒वता᳚भ्यो॒ वै वै दे॒वता᳚भ्यो दे॒वता᳚भ्यो॒ वा ए॒ष ए॒ष वै दे॒वता᳚भ्यो दे॒वता᳚भ्यो॒ वा ए॒षः । \newline
39. वा ए॒ष ए॒ष वै वा ए॒ष ऐष वै वा ए॒ष आ । \newline
40. ए॒ष ऐष ए॒ष आ वृ॑श्च्यते वृश्च्यत॒ ऐष ए॒ष आ वृ॑श्च्यते । \newline
41. आ वृ॑श्च्यते वृश्च्यत॒ आ वृ॑श्च्यते॒ यो यो वृ॑श्च्यत॒ आ वृ॑श्च्यते॒ यः । \newline
42. वृ॒श्च्य॒ते॒ यो यो वृ॑श्च्यते वृश्च्यते॒ यो य॒क्ष्ये य॒क्ष्ये यो वृ॑श्च्यते वृश्च्यते॒ यो य॒क्ष्ये । \newline
43. यो य॒क्ष्ये य॒क्ष्ये यो यो य॒क्ष्य इतीति॑ य॒क्ष्ये यो यो य॒क्ष्य इति॑ । \newline
44. य॒क्ष्य इतीति॑ य॒क्ष्ये य॒क्ष्य इत्यु॒क्त्वो क्त्वेति॑ य॒क्ष्ये य॒क्ष्य इत्यु॒क्त्वा । \newline
45. इत्यु॒क्त्वो क्त्वेती त्यु॒क्त्वा न नोक्त्वेती त्यु॒क्त्वा न । \newline
46. उ॒क्त्वा न नोक्त्वो क्त्वा न यज॑ते॒ यज॑ते॒ नोक्त्वो क्त्वा न यज॑ते । \newline
47. न यज॑ते॒ यज॑ते॒ न न यज॑ते त्रैधात॒वीये॑न त्रैधात॒वीये॑न॒ यज॑ते॒ न न यज॑ते त्रैधात॒वीये॑न । \newline
48. यज॑ते त्रैधात॒वीये॑न त्रैधात॒वीये॑न॒ यज॑ते॒ यज॑ते त्रैधात॒वीये॑न यजेत यजेत त्रैधात॒वीये॑न॒ यज॑ते॒ यज॑ते त्रैधात॒वीये॑न यजेत । \newline
49. त्रै॒धा॒त॒वीये॑न यजेत यजेत त्रैधात॒वीये॑न त्रैधात॒वीये॑न यजेत॒ सर्वः॒ सर्वो॑ यजेत त्रैधात॒वीये॑न त्रैधात॒वीये॑न यजेत॒ सर्वः॑ । \newline
50. य॒जे॒त॒ सर्वः॒ सर्वो॑ यजेत यजेत॒ सर्वो॒ वै वै सर्वो॑ यजेत यजेत॒ सर्वो॒ वै । \newline
51. सर्वो॒ वै वै सर्वः॒ सर्वो॒ वा ए॒ष ए॒ष वै सर्वः॒ सर्वो॒ वा ए॒षः । \newline
52. वा ए॒ष ए॒ष वै वा ए॒ष य॒ज्ञो य॒ज्ञ् ए॒ष वै वा ए॒ष य॒ज्ञ्ः । \newline
53. ए॒ष य॒ज्ञो य॒ज्ञ् ए॒ष ए॒ष य॒ज्ञो यद् यद् य॒ज्ञ् ए॒ष ए॒ष य॒ज्ञो यत् । \newline
54. य॒ज्ञो यद् यद् य॒ज्ञो य॒ज्ञो यत् त्रै॑धात॒वीय॑म् त्रैधात॒वीयं॒ ॅयद् य॒ज्ञो य॒ज्ञो यत् त्रै॑धात॒वीय᳚म् । \newline
\pagebreak
\markright{ TS 2.4.11.5  \hfill https://www.vedavms.in \hfill}

\section{ TS 2.4.11.5 }

\textbf{TS 2.4.11.5 } \newline
\textbf{Samhita Paata} \newline

यत् त्रै॑धात॒वीयꣳ॒॒ सर्वे॑णै॒व य॒ज्ञेन॑ यजते॒ न दे॒वता᳚भ्य॒ आ वृ॑श्च्यते॒ द्वाद॑शकपालः पुरो॒डाशो॑ भवति॒ ते त्रय॒श्चतु॑ष्कपाला-स्त्रिष्षमृद्ध॒त्वाय॒ त्रयः॑ पुरो॒डाशा॑ भवन्ति॒ त्रय॑ इ॒मे लो॒का ए॒षां ॅलो॒काना॒माप्त्या॒ उत्त॑र-उत्तरो॒ ज्याया᳚न् भवत्ये॒वमि॑व॒ हीमे लो॒का य॑व॒मयो॒ मद्ध्य॑ ए॒तद्वा अ॒न्तरि॑क्षस्य रू॒पꣳ समृ॑द्ध्यै॒ सर्वे॑षामभिग॒मय॒न्नव॑ द्य॒त्यछ॑बंट्कारꣳ॒॒ हिर॑ण्यं ददाति॒ तेज॑ ए॒वा - [  ] \newline

\textbf{Pada Paata} \newline

यत् । त्रै॒धा॒त॒वीय᳚म् । सर्वे॑ण । ए॒व । य॒ज्ञेन॑ । य॒ज॒ते॒ । न । दे॒वता᳚भ्यः । एति॑ । वृ॒श्च्य॒ते॒ । द्वाद॑शकपाल॒ इति॒ द्वाद॑श-क॒पा॒लः॒ । पु॒रो॒डाशः॑ । भ॒व॒ति॒ । ते । त्रयः॑ । चतु॑ष्कपाला॒ इति॒ चतुः॑-क॒पा॒लाः॒ । त्रि॒ष्ष॒मृ॒द्ध॒त्वायेति॑ त्रिष्षमृद्ध - त्वाय॑ । त्रयः॑ । पु॒रो॒डाशाः᳚ । भ॒व॒न्ति॒ ।   त्रयः॑ । इ॒मे । लो॒काः । ए॒षाम् । लो॒काना᳚म् ।  आप्त्यै᳚ । उत्त॑र उत्तर॒ इत्युत्त॑रः -उ॒त्त॒रः॒ ।  ज्यायान्॑ । भ॒व॒ति॒ । ए॒वम् । इ॒व॒ । हि । इ॒मे । लो॒काः । य॒व॒मय॒ इति॑ यव - मयः॑ । मद्ध्ये᳚ । ए॒तत् ।   वै । अ॒न्तरि॑क्षस्य । रू॒पम् । समृ॑द्ध्या॒ इति॒ सं - ऋ॒द्ध्यै॒ । सर्वे॑षाम् । अ॒भि॒ग॒मय॒न्नित्य॑भि - ग॒मयन्न्॑ । अवेति॑ । द्य॒ति॒ । अछ॑बंट्कार॒मित्यछ॑बंट् - का॒र॒म् । हिर॑ण्यम् । द॒दा॒ति॒ । तेजः॑ । ए॒व ।  \newline


\textbf{Krama Paata} \newline

यत् त्रै॑धात॒वीय᳚म् । त्रै॒धा॒त॒वीयꣳ॒॒ सर्वे॑ण । सर्वे॑णै॒व । ए॒व य॒ज्ञेन॑ । य॒ज्ञेन॑ यजते । य॒ज॒ते॒ न । न दे॒वता᳚भ्यः । दे॒वता᳚भ्य॒ आ । आ वृ॑श्च्यते । वृ॒श्च्य॒ते॒ द्वाद॑शकपालः । द्वाद॑शकपालः पुरो॒डाशः॑ । द्वाद॑शकपाल॒ इति॒ द्वाद॑श - क॒पा॒लः॒ । पु॒रो॒डाशो॑ भवति । भ॒व॒ति॒ ते । ते त्रयः॑ । त्रय॒ श्चतु॑ष्कपालाः । चतु॑ष्कपालास्त्रिष्षमृद्ध॒त्वाय॑ । चतु॑ष्कपाला॒ इति॒ चतुः॑ - क॒पा॒लाः॒ । त्रि॒ष्ष॒मृ॒द्ध॒त्वाय॒ त्रयः॑ । त्रि॒ष्ष॒मृ॒द्ध॒त्वायेति॑ त्रिष्षमृद्ध - त्वाय॑ । त्रयः॑ पुरो॒डाशाः᳚ । पु॒रो॒डाशा॑ भवन्ति । भ॒व॒न्ति॒ त्रयः॑ । त्रय॑ इ॒मे । इ॒मे लो॒काः । लो॒का ए॒षाम् । ए॒षां ॅलो॒काना᳚म् । लो॒काना॒माप्तै᳚ । आप्त्या॒ उत्त॑र उत्तरः । उत्त॑र उत्तरो॒ ज्यायान्॑ । उत्त॑र उत्तर॒ इत्युत्त॑रः - उ॒त्त॒रः॒ । ज्याया᳚न् भवति । भ॒व॒त्ये॒वम् । ए॒वमि॑व । इ॒व॒ हि । हीमे । इ॒मे लो॒काः । लो॒का य॑व॒मयः॑ । य॒व॒मयो॒ मद्ध्ये᳚ । य॒व॒मय॒ इति॑ यव - मयः॑ । मद्ध्य॑ ए॒तत् । ए॒तद् वै । वा अ॒न्तरि॑क्षस्य । अ॒न्तरि॑क्षस्य रू॒पम् । रू॒पꣳ समृ॑द्ध्यै । समृ॑द्ध्यै॒ सर्वे॑षाम् । समृ॑द्ध्या॒ इति॒ सम् - ऋ॒द्ध्ये॒ । सर्वे॑षामभिग॒मयन्न्॑ । अ॒भि॒ग॒मय॒न्नव॑ । अ॒भि॒ग॒मय॒न्नित्य॑भि - ग॒मयन्न्॑ । अव॑ द्यति । द्य॒त्यछ॑म्बट्कारम् । 
अछ॑म्बट्कारꣳ॒॒ हिर॑ण्यम् । अछ॑म्बट्कार॒मित्यछ॑म्बट् - का॒र॒म् । हिर॑ण्यम् ददाति । द॒दा॒ति॒ तेजः॑ । तेज॑ ए॒व । ए॒वाव॑ ( ) \newline

\textbf{Jatai Paata} \newline

1. यत् त्रै॑धात॒वीय॑म् त्रैधात॒वीयं॒ ॅयद् यत् त्रै॑धात॒वीय᳚म् । \newline
2. त्रै॒धा॒त॒वीयꣳ॒॒ सर्वे॑ण॒ सर्वे॑ण त्रैधात॒वीय॑म् त्रैधात॒वीयꣳ॒॒ सर्वे॑ण । \newline
3. सर्वे॑णै॒वैव सर्वे॑ण॒ सर्वे॑णै॒व । \newline
4. ए॒व य॒ज्ञेन॑ य॒ज्ञे नै॒वैव य॒ज्ञेन॑ । \newline
5. य॒ज्ञेन॑ यजते यजते य॒ज्ञेन॑ य॒ज्ञेन॑ यजते । \newline
6. य॒ज॒ते॒ न न य॑जते यजते॒ न । \newline
7. न दे॒वता᳚भ्यो दे॒वता᳚भ्यो॒ न न दे॒वता᳚भ्यः । \newline
8. दे॒वता᳚भ्य॒ आ दे॒वता᳚भ्यो दे॒वता᳚भ्य॒ आ । \newline
9. आ वृ॑श्च्यते वृश्च्यत॒ आ वृ॑श्च्यते । \newline
10. वृ॒श्च्य॒ते॒ द्वाद॑शकपालो॒ द्वाद॑शकपालो वृश्च्यते वृश्च्यते॒ द्वाद॑शकपालः । \newline
11. द्वाद॑शकपालः पुरो॒डाशः॑ पुरो॒डाशो॒ द्वाद॑शकपालो॒ द्वाद॑शकपालः पुरो॒डाशः॑ । \newline
12. द्वाद॑शकपाल॒ इति॒ द्वाद॑श - क॒पा॒लः॒ । \newline
13. पु॒रो॒डाशो॑ भवति भवति पुरो॒डाशः॑ पुरो॒डाशो॑ भवति । \newline
14. भ॒व॒ति॒ ते ते भ॑वति भवति॒ ते । \newline
15. ते त्रय॒ स्त्रय॒ स्ते ते त्रयः॑ । \newline
16. त्रय॒ श्चतु॑ष्कपाला॒ श्चतु॑ष्कपाला॒ स्त्रय॒ स्त्रय॒ श्चतु॑ष्कपालाः । \newline
17. चतु॑ष्कपाला स्त्रिष्षमृद्ध॒त्वाय॑ त्रिष्षमृद्ध॒त्वाय॒ चतु॑ष्कपाला॒ श्चतु॑ष्कपाला स्त्रिष्षमृद्ध॒त्वाय॑ । \newline
18. चतु॑ष्कपाला॒ इति॒ चतुः॑ - क॒पा॒लाः॒ । \newline
19. त्रि॒ष्ष॒मृ॒द्ध॒त्वाय॒ त्रय॒ स्त्रय॑ स्त्रिष्षमृद्ध॒त्वाय॑ त्रिष्षमृद्ध॒त्वाय॒ त्रयः॑ । \newline
20. त्रि॒ष्ष॒मृ॒द्ध॒त्वायेति॑ त्रिष्षमृद्ध - त्वाय॑ । \newline
21. त्रयः॑ पुरो॒डाशाः᳚ पुरो॒डाशा॒ स्त्रय॒ स्त्रयः॑ पुरो॒डाशाः᳚ । \newline
22. पु॒रो॒डाशा॑ भवन्ति भवन्ति पुरो॒डाशाः᳚ पुरो॒डाशा॑ भवन्ति । \newline
23. भ॒व॒न्ति॒ त्रय॒ स्त्रयो॑ भवन्ति भवन्ति॒ त्रयः॑ । \newline
24. त्रय॑ इ॒म इ॒मे त्रय॒ स्त्रय॑ इ॒मे । \newline
25. इ॒मे लो॒का लो॒का इ॒म इ॒मे लो॒काः । \newline
26. लो॒का ए॒षा मे॒षाम् ॅलो॒का लो॒का ए॒षाम् । \newline
27. ए॒षाम् ॅलो॒काना᳚म् ॅलो॒काना॑ मे॒षा मे॒षाम् ॅलो॒काना᳚म् । \newline
28. लो॒काना॒ माप्त्या॒ आप्त्यै॑ लो॒काना᳚म् ॅलो॒काना॒ माप्त्यै᳚ । \newline
29. आप्त्या॒ उत्त॑र‍उत्तर॒ उत्त॑र‍उत्तर॒ आप्त्या॒ आप्त्या॒ उत्त॑र‍उत्तरः । \newline
30. उत्त॑र‍उत्तरो॒ ज्याया॒न् ज्याया॒ नुत्त॑र‍उत्तर॒ उत्त॑र‍उत्तरो॒ ज्यायान्॑ । \newline
31. उत्त॑र‌उत्तर॒ इत्युत्त॑रः - उ॒त्त॒रः॒ । \newline
32. ज्याया᳚न् भवति भवति॒ ज्याया॒न् ज्याया᳚न् भवति । \newline
33. भ॒व॒त्ये॒व मे॒वम् भ॑वति भवत्ये॒वम् । \newline
34. ए॒व मि॑वे वै॒व मे॒व मि॑व । \newline
35. इ॒व॒ हि हीवे॑ व॒ हि । \newline
36. हीम इ॒मे हि हीमे । \newline
37. इ॒मे लो॒का लो॒का इ॒म इ॒मे लो॒काः । \newline
38. लो॒का य॑व॒मयो॑ यव॒मयो॑ लो॒का लो॒का य॑व॒मयः॑ । \newline
39. य॒व॒मयो॒ मद्ध्ये॒ मद्ध्ये॑ यव॒मयो॑ यव॒मयो॒ मद्ध्ये᳚ । \newline
40. य॒व॒मय॒ इति॑ यव - मयः॑ । \newline
41. मद्ध्य॑ ए॒त दे॒तन् मद्ध्ये॒ मद्ध्य॑ ए॒तत् । \newline
42. ए॒तद् वै वा ए॒त दे॒तद् वै । \newline
43. वा अ॒न्तरि॑क्षस्या॒ न्तरि॑क्षस्य॒ वै वा अ॒न्तरि॑क्षस्य । \newline
44. अ॒न्तरि॑क्षस्य रू॒पꣳ रू॒प म॒न्तरि॑क्षस्या॒ न्तरि॑क्षस्य रू॒पम् । \newline
45. रू॒पꣳ समृ॑द्ध्यै॒ समृ॑द्ध्यै रू॒पꣳ रू॒पꣳ समृ॑द्ध्यै । \newline
46. समृ॑द्ध्यै॒ सर्वे॑षाꣳ॒॒ सर्वे॑षाꣳ॒॒ समृ॑द्ध्यै॒ समृ॑द्ध्यै॒ सर्वे॑षाम् । \newline
47. समृ॑द्ध्या॒ इति॒ सं - ऋ॒द्ध्यै॒ । \newline
48. सर्वे॑षा मभिग॒मय॑न् नभिग॒मय॒न् थ्सर्वे॑षाꣳ॒॒ सर्वे॑षा मभिग॒मयन्न्॑ । \newline
49. अ॒भि॒ग॒मय॒न् नवावा॑ भिग॒मय॑न् नभिग॒मय॒न् नव॑ । \newline
50. अ॒भि॒ग॒मय॒न्नित्य॑भि - ग॒मयन्न्॑ । \newline
51. अव॑ द्यति द्य॒त्यवाव॑ द्यति । \newline
52. द्य॒त्यछं॑बट्कार॒ मछं॑बट्कारम् द्यति द्य॒त्यछं॑बट्कारम् । \newline
53. अछं॑बट्कारꣳ॒॒ हिर॑ण्यꣳ॒॒ हिर॑ण्य॒ मछं॑बट्कार॒ मछं॑बट्कारꣳ॒॒ हिर॑ण्यम् । \newline
54. अछं॑बट्कार॒मित्यछं॑बट् - का॒र॒म् । \newline
55. हिर॑ण्यम् ददाति ददाति॒ हिर॑ण्यꣳ॒॒ हिर॑ण्यम् ददाति । \newline
56. द॒दा॒ति॒ तेज॒ स्तेजो॑ ददाति ददाति॒ तेजः॑ । \newline
57. तेज॑ ए॒वैव तेज॒ स्तेज॑ ए॒व । \newline
58. ए॒वावावै॒वैवाव॑ । \newline

\textbf{Ghana Paata } \newline

1. यत् त्रै॑धात॒वीय॑म् त्रैधात॒वीयं॒ ॅयद् यत् त्रै॑धात॒वीयꣳ॒॒ सर्वे॑ण॒ सर्वे॑ण त्रैधात॒वीयं॒ ॅयद् यत् त्रै॑धात॒वीयꣳ॒॒ सर्वे॑ण । \newline
2. त्रै॒धा॒त॒वीयꣳ॒॒ सर्वे॑ण॒ सर्वे॑ण त्रैधात॒वीय॑म् त्रैधात॒वीयꣳ॒॒ सर्वे॑णै॒वैव सर्वे॑ण त्रैधात॒वीय॑म् त्रैधात॒वीयꣳ॒॒ सर्वे॑णै॒व । \newline
3. सर्वे॑णै॒वैव सर्वे॑ण॒ सर्वे॑णै॒व य॒ज्ञेन॑ य॒ज्ञेनै॒व सर्वे॑ण॒ सर्वे॑णै॒व य॒ज्ञेन॑ । \newline
4. ए॒व य॒ज्ञेन॑ य॒ज्ञेनै॒वैव य॒ज्ञेन॑ यजते यजते य॒ज्ञेनै॒वैव य॒ज्ञेन॑ यजते । \newline
5. य॒ज्ञेन॑ यजते यजते य॒ज्ञेन॑ य॒ज्ञेन॑ यजते॒ न न य॑जते य॒ज्ञेन॑ य॒ज्ञेन॑ यजते॒ न । \newline
6. य॒ज॒ते॒ न न य॑जते यजते॒ न दे॒वता᳚भ्यो दे॒वता᳚भ्यो॒ न य॑जते यजते॒ न दे॒वता᳚भ्यः । \newline
7. न दे॒वता᳚भ्यो दे॒वता᳚भ्यो॒ न न दे॒वता᳚भ्य॒ आ दे॒वता᳚भ्यो॒ न न दे॒वता᳚भ्य॒ आ । \newline
8. दे॒वता᳚भ्य॒ आ दे॒वता᳚भ्यो दे॒वता᳚भ्य॒ आ वृ॑श्च्यते वृश्च्यत॒ आ दे॒वता᳚भ्यो दे॒वता᳚भ्य॒ आ वृ॑श्च्यते । \newline
9. आ वृ॑श्च्यते वृश्च्यत॒ आ वृ॑श्च्यते॒ द्वाद॑शकपालो॒ द्वाद॑शकपालो वृश्च्यत॒ आ वृ॑श्च्यते॒ द्वाद॑शकपालः । \newline
10. वृ॒श्च्य॒ते॒ द्वाद॑शकपालो॒ द्वाद॑शकपालो वृश्च्यते वृश्च्यते॒ द्वाद॑शकपालः पुरो॒डाशः॑ पुरो॒डाशो॒ द्वाद॑शकपालो वृश्च्यते वृश्च्यते॒ द्वाद॑शकपालः पुरो॒डाशः॑ । \newline
11. द्वाद॑शकपालः पुरो॒डाशः॑ पुरो॒डाशो॒ द्वाद॑शकपालो॒ द्वाद॑शकपालः पुरो॒डाशो॑ भवति भवति पुरो॒डाशो॒ द्वाद॑शकपालो॒ द्वाद॑शकपालः पुरो॒डाशो॑ भवति । \newline
12. द्वाद॑शकपाल॒ इति॒ द्वाद॑श - क॒पा॒लः॒ । \newline
13. पु॒रो॒डाशो॑ भवति भवति पुरो॒डाशः॑ पुरो॒डाशो॑ भवति॒ ते ते भ॑वति पुरो॒डाशः॑ पुरो॒डाशो॑ भवति॒ ते । \newline
14. भ॒व॒ति॒ ते ते भ॑वति भवति॒ ते त्रय॒ स्त्रय॒ स्ते भ॑वति भवति॒ ते त्रयः॑ । \newline
15. ते त्रय॒ स्त्रय॒ स्ते ते त्रय॒ श्चतु॑ष्कपाला॒ श्चतु॑ष्कपाला॒ स्त्रय॒ स्ते ते त्रय॒ श्चतु॑ष्कपालाः । \newline
16. त्रय॒ श्चतु॑ष्कपाला॒ श्चतु॑ष्कपाला॒ स्त्रय॒ स्त्रय॒ श्चतु॑ष्कपाला स्त्रिष्षमृद्ध॒त्वाय॑ त्रिष्षमृद्ध॒त्वाय॒ चतु॑ष्कपाला॒ स्त्रय॒ स्त्रय॒ श्चतु॑ष्कपाला स्त्रिष्षमृद्ध॒त्वाय॑ । \newline
17. चतु॑ष्कपाला स्त्रिष्षमृद्ध॒त्वाय॑ त्रिष्षमृद्ध॒त्वाय॒ चतु॑ष्कपाला॒ श्चतु॑ष्कपाला स्त्रिष्षमृद्ध॒त्वाय॒ त्रय॒ स्त्रय॑ स्त्रिष्षमृद्ध॒त्वाय॒ चतु॑ष्कपाला॒ श्चतु॑ष्कपाला स्त्रिष्षमृद्ध॒त्वाय॒ त्रयः॑ । \newline
18. चतु॑ष्कपाला॒ इति॒ चतुः॑ - क॒पा॒लाः॒ । \newline
19. त्रि॒ष्ष॒मृ॒द्ध॒त्वाय॒ त्रय॒ स्त्रय॑ स्त्रिष्षमृद्ध॒त्वाय॑ त्रिष्षमृद्ध॒त्वाय॒ त्रयः॑ पुरो॒डाशाः᳚ पुरो॒डाशा॒ स्त्रय॑ स्त्रिष्षमृद्ध॒त्वाय॑ त्रिष्षमृद्ध॒त्वाय॒ त्रयः॑ पुरो॒डाशाः᳚ । \newline
20. त्रि॒ष्ष॒मृ॒द्ध॒त्वायेति॑ त्रिष्षमृद्ध - त्वाय॑ । \newline
21. त्रयः॑ पुरो॒डाशाः᳚ पुरो॒डाशा॒ स्त्रय॒ स्त्रयः॑ पुरो॒डाशा॑ भवन्ति भवन्ति पुरो॒डाशा॒ स्त्रय॒ स्त्रयः॑ पुरो॒डाशा॑ भवन्ति । \newline
22. पु॒रो॒डाशा॑ भवन्ति भवन्ति पुरो॒डाशाः᳚ पुरो॒डाशा॑ भवन्ति॒ त्रय॒ स्त्रयो॑ भवन्ति पुरो॒डाशाः᳚ पुरो॒डाशा॑ भवन्ति॒ त्रयः॑ । \newline
23. भ॒व॒न्ति॒ त्रय॒ स्त्रयो॑ भवन्ति भवन्ति॒ त्रय॑ इ॒म इ॒मे त्रयो॑ भवन्ति भवन्ति॒ त्रय॑ इ॒मे । \newline
24. त्रय॑ इ॒म इ॒मे त्रय॒ स्त्रय॑ इ॒मे लो॒का लो॒का इ॒मे त्रय॒ स्त्रय॑ इ॒मे लो॒काः । \newline
25. इ॒मे लो॒का लो॒का इ॒म इ॒मे लो॒का ए॒षा मे॒षाम् ॅलो॒का इ॒म इ॒मे लो॒का ए॒षाम् । \newline
26. लो॒का ए॒षा मे॒षाम् ॅलो॒का लो॒का ए॒षाम् ॅलो॒काना᳚म् ॅलो॒काना॑ मे॒षाम् ॅलो॒का लो॒का ए॒षाम् ॅलो॒काना᳚म् । \newline
27. ए॒षाम् ॅलो॒काना᳚म् ॅलो॒काना॑ मे॒षा मे॒षाम् ॅलो॒काना॒ माप्त्या॒ आप्त्यै॑ लो॒काना॑ मे॒षा मे॒षाम् ॅलो॒काना॒ माप्त्यै᳚ । \newline
28. लो॒काना॒ माप्त्या॒ आप्त्यै॑ लो॒काना᳚म् ॅलो॒काना॒ माप्त्या॒ उत्त॑र‍उत्तर॒ उत्त॑र‍उत्तर॒ आप्त्यै॑ लो॒काना᳚म् ॅलो॒काना॒ माप्त्या॒ उत्त॑र‍उत्तरः । \newline
29. आप्त्या॒ उत्त॑र‍उत्तर॒ उत्त॑र‍उत्तर॒ आप्त्या॒ आप्त्या॒ उत्त॑र‍उत्तरो॒ ज्याया॒न् ज्याया॒ नुत्त॑र‍उत्तर॒ आप्त्या॒ आप्त्या॒ उत्त॑र‍उत्तरो॒ ज्यायान्॑ । \newline
30. उत्त॑र‍उत्तरो॒ ज्याया॒न् ज्याया॒ नुत्त॑र‍उत्तर॒ उत्त॑र‍उत्तरो॒ ज्याया᳚न् भवति भवति॒ ज्याया॒ नुत्त॑र‍उत्तर॒ उत्त॑र‍उत्तरो॒ ज्याया᳚न् भवति । \newline
31. उत्त॑र‌उत्तर॒ इत्युत्त॑रः - उ॒त्त॒रः॒ । \newline
32. ज्याया᳚न् भवति भवति॒ ज्याया॒न् ज्याया᳚न् भव त्ये॒व मे॒वम् भ॑वति॒ ज्याया॒न् ज्याया᳚न् भव त्ये॒वम् । \newline
33. भ॒व॒ त्ये॒व मे॒वम् भ॑वति भव त्ये॒व मि॑वे वै॒वम् भ॑वति भव त्ये॒व मि॑व । \newline
34. ए॒व मि॑वे वै॒व मे॒व मि॑व॒ हि हीवै॒व मे॒व मि॑व॒ हि । \newline
35. इ॒व॒ हि हीवे॑ व॒ हीम इ॒मे हीवे॑ व॒ हीमे । \newline
36. हीम इ॒मे हि हीमे लो॒का लो॒का इ॒मे हि हीमे लो॒काः । \newline
37. इ॒मे लो॒का लो॒का इ॒म इ॒मे लो॒का य॑व॒मयो॑ यव॒मयो॑ लो॒का इ॒म इ॒मे लो॒का य॑व॒मयः॑ । \newline
38. लो॒का य॑व॒मयो॑ यव॒मयो॑ लो॒का लो॒का य॑व॒मयो॒ मद्ध्ये॒ मद्ध्ये॑ यव॒मयो॑ लो॒का लो॒का य॑व॒मयो॒ मद्ध्ये᳚ । \newline
39. य॒व॒मयो॒ मद्ध्ये॒ मद्ध्ये॑ यव॒मयो॑ यव॒मयो॒ मद्ध्य॑ ए॒त दे॒तन् मद्ध्ये॑ यव॒मयो॑ यव॒मयो॒ मद्ध्य॑ ए॒तत् । \newline
40. य॒व॒मय॒ इति॑ यव - मयः॑ । \newline
41. मद्ध्य॑ ए॒त दे॒तन् मद्ध्ये॒ मद्ध्य॑ ए॒तद् वै वा ए॒तन् मद्ध्ये॒ मद्ध्य॑ ए॒तद् वै । \newline
42. ए॒तद् वै वा ए॒त दे॒तद् वा अ॒न्तरि॑क्षस्या॒ न्तरि॑क्षस्य॒ वा ए॒तदे॒तद् वा अ॒न्तरि॑क्षस्य । \newline
43. वा अ॒न्तरि॑क्षस्या॒ न्तरि॑क्षस्य॒ वै वा अ॒न्तरि॑क्षस्य रू॒पꣳ रू॒प म॒न्तरि॑क्षस्य॒ वै वा अ॒न्तरि॑क्षस्य रू॒पम् । \newline
44. अ॒न्तरि॑क्षस्य रू॒पꣳ रू॒प म॒न्तरि॑क्षस्या॒ न्तरि॑क्षस्य रू॒पꣳ समृ॑द्ध्यै॒ समृ॑द्ध्यै रू॒प म॒न्तरि॑क्षस्या॒ न्तरि॑क्षस्य रू॒पꣳ समृ॑द्ध्यै । \newline
45. रू॒पꣳ समृ॑द्ध्यै॒ समृ॑द्ध्यै रू॒पꣳ रू॒पꣳ समृ॑द्ध्यै॒ सर्वे॑षाꣳ॒॒ सर्वे॑षाꣳ॒॒ समृ॑द्ध्यै रू॒पꣳ रू॒पꣳ समृ॑द्ध्यै॒ सर्वे॑षाम् । \newline
46. समृ॑द्ध्यै॒ सर्वे॑षाꣳ॒॒ सर्वे॑षाꣳ॒॒ समृ॑द्ध्यै॒ समृ॑द्ध्यै॒ सर्वे॑षा मभिग॒मय॑न् नभिग॒मय॒न् थ्सर्वे॑षाꣳ॒॒ समृ॑द्ध्यै॒ समृ॑द्ध्यै॒ सर्वे॑षा मभिग॒मयन्न्॑ । \newline
47. समृ॑द्ध्या॒ इति॒ सं - ऋ॒द्ध्यै॒ । \newline
48. सर्वे॑षा मभिग॒मय॑न् नभिग॒मय॒न् थ्सर्वे॑षाꣳ॒॒ सर्वे॑षा मभिग॒मय॒न् नवावा॑भिग॒मय॒न् थ्सर्वे॑षाꣳ॒॒ सर्वे॑षा मभिग॒मय॒न् नव॑ । \newline
49. अ॒भि॒ग॒मय॒न् नवावा॑ भिग॒मय॑न् नभिग॒मय॒न् नव॑ द्यति द्य॒त्यवा॑ भिग॒मय॑न् नभिग॒मय॒न् नव॑ द्यति । \newline
50. अ॒भि॒ग॒मय॒न्नित्य॑भि - ग॒मयन्न्॑ । \newline
51. अव॑ द्यति द्य॒त्यवाव॑ द्य॒ त्यछं॑बट्कार॒ मछं॑बट्कारम् द्य॒त्यवाव॑ द्य॒ त्यछं॑बट्कारम् । \newline
52. द्य॒त्यछं॑बट्कार॒ मछं॑बट्कारम् द्यति द्य॒त्यछं॑बट्कारꣳ॒॒ हिर॑ण्यꣳ॒॒ हिर॑ण्य॒ मछं॑बट्कारम् द्यति द्य॒त्यछं॑बट्कारꣳ॒॒ हिर॑ण्यम् । \newline
53. अछं॑बट्कारꣳ॒॒ हिर॑ण्यꣳ॒॒ हिर॑ण्य॒ मछं॑बट्कार॒ मछं॑बट्कारꣳ॒॒ हिर॑ण्यम् ददाति ददाति॒ हिर॑ण्य॒ मछं॑बट्कार॒ मछं॑बट्कारꣳ॒॒ हिर॑ण्यम् ददाति । \newline
54. अछं॑बट्कार॒मित्यछं॑बट् - का॒र॒म् । \newline
55. हिर॑ण्यम् ददाति ददाति॒ हिर॑ण्यꣳ॒॒ हिर॑ण्यम् ददाति॒ तेज॒ स्तेजो॑ ददाति॒ हिर॑ण्यꣳ॒॒ हिर॑ण्यम् ददाति॒ तेजः॑ । \newline
56. द॒दा॒ति॒ तेज॒ स्तेजो॑ ददाति ददाति॒ तेज॑ ए॒वैव तेजो॑ ददाति ददाति॒ तेज॑ ए॒व । \newline
57. तेज॑ ए॒वैव तेज॒ स्तेज॑ ए॒वावावै॒व तेज॒ स्तेज॑ ए॒वाव॑ । \newline
58. ए॒वावा वै॒वैवाव॑ रुन्धे रु॒न्धे ऽवै॒वैवाव॑ रुन्धे । \newline
\pagebreak
\markright{ TS 2.4.11.6  \hfill https://www.vedavms.in \hfill}

\section{ TS 2.4.11.6 }

\textbf{TS 2.4.11.6 } \newline
\textbf{Samhita Paata} \newline

ऽव॑ रुन्धे ता॒र्प्यं द॑दाति प॒शूने॒वाव॑ रुन्धे धे॒नुं द॑दात्या॒शिष॑ ए॒वाव॑ रुन्धे॒ साम्नो॒ वा ए॒ष वर्णो॒ यद्धिर॑ण्यं॒ ॅयजु॑षां ता॒र्प्यमु॑क्थाम॒दानां᳚ धे॒नुरे॒ताने॒व सर्वा॒न्॒. वर्णा॒नव॑ रुन्धे ॥ \newline

\textbf{Pada Paata} \newline

अवेति॑ । रु॒न्धे॒ । ता॒र्प्यम् । द॒दा॒ति॒ । प॒शून् । ए॒व । अवेति॑ । रु॒न्धे॒ । धे॒नुम् । द॒दा॒ति॒ । आ॒शिष॒ इत्या᳚ - शिषः॑ । ए॒व । अवेति॑ । रु॒न्धे॒ । साम्नः॑ । वै । ए॒षः । वर्णः॑ । यत् । हिर॑ण्यम् । यजु॑षाम् । ता॒र्प्यम् । उ॒क्था॒म॒दाना॒मित्यु॑क्थ - म॒दाना᳚म् । धे॒नुः । ए॒तान् । ए॒व । सर्वान्॑ । वर्णान्॑ । अवेति॑ । रु॒न्धे॒ ॥  \newline


\textbf{Krama Paata} \newline

अव॑ रुन्धे । रु॒न्धे॒ ता॒र्प्यम् । ता॒र्प्यम् द॑दाति । द॒दा॒ति॒ प॒शून् । प॒शूने॒व । ए॒वाव॑ । अव॑ रुन्धे । रु॒न्धे॒ धे॒नुम् । धे॒नुम् द॑दाति । द॒दा॒त्या॒शिषः॑ । आ॒शिष॑ ए॒व । आ॒शिष॒ इत्या᳚ - शिषः॑ । ए॒वाव॑ । अव॑ रुन्धे । रु॒न्धे॒ साम्नः॑ । साम्नो॒ वै । वा ए॒षः । ए॒ष वर्णः॑ । वर्णो॒ यत् । यद्धिर॑ण्यम् । हिर॑ण्यं॒ ॅयजु॑षाम् । यजु॑षाम् ता॒र्प्यम् । ता॒र्प्यमु॑क्थाम॒दाना᳚म् । उ॒क्था॒म॒दाना᳚म् धे॒नुः । उ॒क्था॒म॒दाना॒मित्यु॑क्थ - म॒दाना᳚म् । धे॒नुरे॒तान् । ए॒ताने॒व । ए॒व सर्वान्॑ । सर्वा॒न्॒. वर्णान्॑ । वर्णा॒नव॑ । अव॑ रुन्धे । रु॒न्ध॒ इति॑ रुन्धे । \newline

\textbf{Jatai Paata} \newline

1. अव॑ रुन्धे रु॒न्धे ऽवाव॑ रुन्धे । \newline
2. रु॒न्धे॒ ता॒र्प्यम् ता॒र्प्यꣳ रु॑न्धे रुन्धे ता॒र्प्यम् । \newline
3. ता॒र्प्यम् द॑दाति ददाति ता॒र्प्यम् ता॒र्प्यम् द॑दाति । \newline
4. द॒दा॒ति॒ प॒शून् प॒शून् द॑दाति ददाति प॒शून् । \newline
5. प॒शू ने॒वैव प॒शून् प॒शू ने॒व । \newline
6. ए॒वावा वै॒वै वाव॑ । \newline
7. अव॑ रुन्धे रु॒न्धे ऽवाव॑ रुन्धे । \newline
8. रु॒न्धे॒ धे॒नुम् धे॒नुꣳ रु॑न्धे रुन्धे धे॒नुम् । \newline
9. धे॒नुम् द॑दाति ददाति धे॒नुम् धे॒नुम् द॑दाति । \newline
10. द॒दा॒ त्या॒शिष॑ आ॒शिषो॑ ददाति ददा त्या॒शिषः॑ । \newline
11. आ॒शिष॑ ए॒वै वाशिष॑ आ॒शिष॑ ए॒व । \newline
12. आ॒शिष॒ इत्या᳚ - शिषः॑ । \newline
13. ए॒ वावा वै॒वै वाव॑ । \newline
14. अव॑ रुन्धे रु॒न्धे ऽवाव॑ रुन्धे । \newline
15. रु॒न्धे॒ साम्नः॒ साम्नो॑ रुन्धे रुन्धे॒ साम्नः॑ । \newline
16. साम्नो॒ वै वै साम्नः॒ साम्नो॒ वै । \newline
17. वा ए॒ष ए॒ष वै वा ए॒षः । \newline
18. ए॒ष वर्णो॒ वर्ण॑ ए॒ष ए॒ष वर्णः॑ । \newline
19. वर्णो॒ यद् यद् वर्णो॒ वर्णो॒ यत् । \newline
20. य द्धिर॑ण्यꣳ॒॒ हिर॑ण्यं॒ ॅयद् य द्धिर॑ण्यम् । \newline
21. हिर॑ण्यं॒ ॅयजु॑षां॒ ॅयजु॑षाꣳ॒॒ हिर॑ण्यꣳ॒॒ हिर॑ण्यं॒ ॅयजु॑षाम् । \newline
22. यजु॑षाम् ता॒र्प्यम् ता॒र्प्यं ॅयजु॑षां॒ ॅयजु॑षाम् ता॒र्प्यम् । \newline
23. ता॒र्प्य मु॑क्थाम॒दाना॑ मुक्थाम॒दाना᳚म् ता॒र्प्यम् ता॒र्प्य मु॑क्थाम॒दाना᳚म् । \newline
24. उ॒क्था॒म॒दाना᳚म् धे॒नुर् धे॒नुरु॑क्थाम॒दाना॑ मुक्थाम॒दाना᳚म् धे॒नुः । \newline
25. उ॒क्था॒म॒दाना॒मित्यु॑क्थ - म॒दाना᳚म् । \newline
26. धे॒नु रे॒ता ने॒तान् धे॒नुर् धे॒नु रे॒तान् । \newline
27. ए॒ता ने॒वैवैता ने॒ता ने॒व । \newline
28. ए॒व सर्वा॒न् थ्सर्वा॑ ने॒वैव सर्वान्॑ । \newline
29. सर्वा॒न्॒. वर्णा॒न्॒. वर्णा॒न् थ्सर्वा॒न् थ्सर्वा॒न्॒. वर्णान्॑ । \newline
30. वर्णा॒ नवाव॒ वर्णा॒न्॒. वर्णा॒ नव॑ । \newline
31. अव॑ रुन्धे रु॒न्धे ऽवाव॑ रुन्धे । \newline
32. रु॒न्ध॒ इति॑ रुन्धे । \newline

\textbf{Ghana Paata } \newline

1. अव॑ रुन्धे रु॒न्धे ऽवाव॑ रुन्धे ता॒र्प्यम् ता॒र्प्यꣳ रु॒न्धे ऽवाव॑ रुन्धे ता॒र्प्यम् । \newline
2. रु॒न्धे॒ ता॒र्प्यम् ता॒र्प्यꣳ रु॑न्धे रुन्धे ता॒र्प्यम् द॑दाति ददाति ता॒र्प्यꣳ रु॑न्धे रुन्धे ता॒र्प्यम् द॑दाति । \newline
3. ता॒र्प्यम् द॑दाति ददाति ता॒र्प्यम् ता॒र्प्यम् द॑दाति प॒शून् प॒शून् द॑दाति ता॒र्प्यम् ता॒र्प्यम् द॑दाति प॒शून् । \newline
4. द॒दा॒ति॒ प॒शून् प॒शून् द॑दाति ददाति प॒शू ने॒वैव प॒शून् द॑दाति ददाति प॒शू ने॒व । \newline
5. प॒शू ने॒वैव प॒शून् प॒शू ने॒वावा वै॒व प॒शून् प॒शू ने॒वाव॑ । \newline
6. ए॒वावा वै॒वैवाव॑ रुन्धे रु॒न्धे ऽवै॒वैवाव॑ रुन्धे । \newline
7. अव॑ रुन्धे रु॒न्धे ऽवाव॑ रुन्धे धे॒नुम् धे॒नुꣳ रु॒न्धे ऽवाव॑ रुन्धे धे॒नुम् । \newline
8. रु॒न्धे॒ धे॒नुम् धे॒नुꣳ रु॑न्धे रुन्धे धे॒नुम् द॑दाति ददाति धे॒नुꣳ रु॑न्धे रुन्धे धे॒नुम् द॑दाति । \newline
9. धे॒नुम् द॑दाति ददाति धे॒नुम् धे॒नुम् द॑दा त्या॒शिष॑ आ॒शिषो॑ ददाति धे॒नुम् धे॒नुम् द॑दा त्या॒शिषः॑ । \newline
10. द॒दा॒ त्या॒शिष॑ आ॒शिषो॑ ददाति ददा त्या॒शिष॑ ए॒वैवाशिषो॑ ददाति ददा त्या॒शिष॑ ए॒व । \newline
11. आ॒शिष॑ ए॒वैवाशिष॑ आ॒शिष॑ ए॒वावा वै॒वाशिष॑ आ॒शिष॑ ए॒वाव॑ । \newline
12. आ॒शिष॒ इत्या᳚ - शिषः॑ । \newline
13. ए॒वावा वै॒वैवाव॑ रुन्धे रु॒न्धे ऽवै॒वैवाव॑ रुन्धे । \newline
14. अव॑ रुन्धे रु॒न्धे ऽवाव॑ रुन्धे॒ साम्नः॒ साम्नो॑ रु॒न्धे ऽवाव॑ रुन्धे॒ साम्नः॑ । \newline
15. रु॒न्धे॒ साम्नः॒ साम्नो॑ रुन्धे रुन्धे॒ साम्नो॒ वै वै साम्नो॑ रुन्धे रुन्धे॒ साम्नो॒ वै । \newline
16. साम्नो॒ वै वै साम्नः॒ साम्नो॒ वा ए॒ष ए॒ष वै साम्नः॒ साम्नो॒ वा ए॒षः । \newline
17. वा ए॒ष ए॒ष वै वा ए॒ष वर्णो॒ वर्ण॑ ए॒ष वै वा ए॒ष वर्णः॑ । \newline
18. ए॒ष वर्णो॒ वर्ण॑ ए॒ष ए॒ष वर्णो॒ यद् यद् वर्ण॑ ए॒ष ए॒ष वर्णो॒ यत् । \newline
19. वर्णो॒ यद् यद् वर्णो॒ वर्णो॒ यद्धिर॑ण्यꣳ॒॒ हिर॑ण्यं॒ ॅयद् वर्णो॒ वर्णो॒ यद्धिर॑ण्यम् । \newline
20. यद्धिर॑ण्यꣳ॒॒ हिर॑ण्यं॒ ॅयद् यद्धिर॑ण्यं॒ ॅयजु॑षां॒ ॅयजु॑षाꣳ॒॒ हिर॑ण्यं॒ ॅयद् यद्धिर॑ण्यं॒ ॅयजु॑षाम् । \newline
21. हिर॑ण्यं॒ ॅयजु॑षां॒ ॅयजु॑षाꣳ॒॒ हिर॑ण्यꣳ॒॒ हिर॑ण्यं॒ ॅयजु॑षाम् ता॒र्प्यम् ता॒र्प्यं ॅयजु॑षाꣳ॒॒ हिर॑ण्यꣳ॒॒ हिर॑ण्यं॒ ॅयजु॑षाम् ता॒र्प्यम् । \newline
22. यजु॑षाम् ता॒र्प्यम् ता॒र्प्यं ॅयजु॑षां॒ ॅयजु॑षाम् ता॒र्प्य मु॑क्थाम॒दाना॑ मुक्थाम॒दाना᳚म् ता॒र्प्यं ॅयजु॑षां॒ ॅयजु॑षाम् ता॒र्प्य मु॑क्थाम॒दाना᳚म् । \newline
23. ता॒र्प्य मु॑क्थाम॒दाना॑ मुक्थाम॒दाना᳚म् ता॒र्प्यम् ता॒र्प्य मु॑क्थाम॒दाना᳚म् धे॒नुर् धे॒नु रु॑क्थाम॒दाना᳚म् ता॒र्प्यम् ता॒र्प्य मु॑क्थाम॒दाना᳚म् धे॒नुः । \newline
24. उ॒क्था॒म॒दाना᳚म् धे॒नुर् धे॒नु रु॑क्थाम॒दाना॑ मुक्थाम॒दाना᳚म् धे॒नुरे॒ता ने॒तान् धे॒नु रु॑क्थाम॒दाना॑ मुक्थाम॒दाना᳚म् धे॒नु रे॒तान् । \newline
25. उ॒क्था॒म॒दाना॒मित्यु॑क्थ - म॒दाना᳚म् । \newline
26. धे॒नुरे॒ता ने॒तान् धे॒नुर् धे॒नुरे॒ता ने॒वैवैतान् धे॒नुर् धे॒नुरे॒ता ने॒व । \newline
27. ए॒ता ने॒वैवैता ने॒ता ने॒व सर्वा॒न् थ्सर्वा॑ ने॒वैता ने॒ता ने॒व सर्वान्॑ । \newline
28. ए॒व सर्वा॒न् थ्सर्वा॑ ने॒वैव सर्वा॒न्॒. वर्णा॒न्॒. वर्णा॒न् थ्सर्वा॑ ने॒वैव सर्वा॒न्॒. वर्णान्॑ । \newline
29. सर्वा॒न्॒. वर्णा॒न्॒. वर्णा॒न् थ्सर्वा॒न् थ्सर्वा॒न्॒. वर्णा॒ नवाव॒ वर्णा॒न् थ्सर्वा॒न् थ्सर्वा॒न्॒. वर्णा॒ नव॑ । \newline
30. वर्णा॒ नवाव॒ वर्णा॒न्॒. वर्णा॒ नव॑ रुन्धे रु॒न्धे ऽव॒ वर्णा॒न्॒. वर्णा॒ नव॑ रुन्धे । \newline
31. अव॑ रुन्धे रु॒न्धे ऽवाव॑ रुन्धे । \newline
32. रु॒न्ध॒ इति॑ रुन्धे । \newline
\pagebreak
\markright{ TS 2.4.12.1  \hfill https://www.vedavms.in \hfill}

\section{ TS 2.4.12.1 }

\textbf{TS 2.4.12.1 } \newline
\textbf{Samhita Paata} \newline

त्वष्टा॑ ह॒तपु॑त्रो॒ वीन्द्रꣳ॒॒ सोम॒माऽह॑र॒त् तस्मि॒न्निन्द्र॑ उपह॒वमै᳚च्छत॒ तं नोपा᳚ह्वयत पु॒त्रं मे॑ऽवधी॒रिति॒ स य॑ज्ञ्वेश॒सं कृ॒त्वा प्रा॒सहा॒ सोम॑मपिब॒त् तस्य॒ यद॒त्यशि॑ष्यत॒ तत् त्वष्टा॑ऽऽहव॒नीय॒मुप॒ प्राव॑र्तय॒थ् स्वाहेन्द्र॑शत्रुर्वर्द्ध॒स्वेति॒ स याव॑दू॒र्द्ध्वः प॑रा॒विद्ध्य॑ति॒ ताव॑ति स्व॒यमे॒व व्य॑रमत॒ यदि॑ वा॒ ताव॑त् प्रव॒ण - [  ] \newline

\textbf{Pada Paata} \newline

त्वष्टा᳚ । ह॒तपु॑त्र॒ इति॑ ह॒त - पु॒त्रः॒ । वीन्द्र॒मिति॒ वि - इ॒न्द्र॒म् । सोम᳚म् । एति॑ । अ॒ह॒र॒त् । तस्मिन्न्॑ । इन्द्रः॑ । उ॒प॒ह॒वमित्यु॑प-ह॒वम् । ऐ॒च्छ॒त॒ । तम् । न । उपेति॑ । अ॒ह्व॒य॒त॒ । पु॒त्रम् । मे॒ । अ॒व॒धीः॒ । इति॑ । सः । य॒ज्ञ्॒वे॒श॒समिति॑ यज्ञ् - वे॒श॒सम् । कृ॒त्वा । प्रा॒सहेति॑ प्र - सहा᳚ । सोम᳚म् । अ॒पि॒ब॒त् । तस्य॑ । यत् । अ॒त्यशि॑ष्य॒तेत्य॑ति - अशि॑ष्यत । तत् । त्वष्टा᳚ । आ॒ह॒व॒नीय॒मित्या᳚ - ह॒व॒नीय᳚म् । उप॑ । प्रेति॑ । अ॒व॒र्त॒य॒त् । स्वाहा᳚ । इन्द्र॑शत्रु॒रितीन्द्र॑ -  श॒त्रुः॒ । व॒र्द्ध॒स्व॒ । इति॑ । सः । याव॑त् । ऊ॒र्द्ध्वः । प॒रा॒विद्ध्य॒तीति॑ परा - विद्ध्य॑ति । ताव॑ति । स्व॒यम् । ए॒व । वीति॑ । अ॒र॒म॒त॒ । यदि॑ । वा॒ । ताव॑त् । प्र॒व॒णमिति॑ प्र - व॒नम् ।  \newline


\textbf{Krama Paata} \newline

त्वष्टा॑ ह॒तपु॑त्रः । ह॒तपु॑त्रो॒ वीन्द्र᳚म् । ह॒तपु॑त्र॒ इति॑ ह॒त - पु॒त्रः॒ । वीन्द्रꣳ॒॒ सोम᳚म् । वीन्द्र॒मिति॒ वि - इ॒न्द्र॒म् । सोम॒मा । आ ऽह॑रत् । अ॒ह॒र॒त् तस्मिन्न्॑ । तस्मि॒न्निन्द्रः॑ । इन्द्र॑ उपह॒वम् । उ॒प॒ह॒वमै᳚च्छत । उ॒प॒ह॒वमित्यु॑प - ह॒वम् । ऐ॒च्छ॒त॒ तम् । तम् न । नोप॑ । उपा᳚ह्वयत । अ॒ह्व॒य॒त॒ पु॒त्रम् । पु॒त्रम् मे᳚ । मे॒ ऽव॒धीः॒ । अ॒व॒धी॒रिति॑ । इति॒ सः । स य॑ज्ञ्वेश॒सम् । य॒ज्ञ्॒वे॒श॒सम् कृ॒त्वा । य॒ज्ञ्॒वे॒श॒समिति॑ यज्ञ् - वे॒श॒सम् । कृ॒त्वा प्रा॒सहा᳚ । प्रा॒सहा॒ सोम᳚म् । प्रा॒सहेति॑ प्र - सहा᳚ । सोम॑मपिबत् । अ॒पि॒ब॒त् तस्य॑ । तस्य॒ यत् । यद॒त्यशि॑ष्यत । अ॒त्यशि॑ष्यत॒ तत् । अ॒त्यशि॑ष्य॒तेत्य॑ति - अशि॑ष्यत । तत् त्वष्टा᳚ । त्वष्टा॑ ऽऽहव॒नीय᳚म् । आ॒ह॒व॒नीय॒मुप॑ । आ॒ह॒व॒नीय॒मित्या᳚ - ह॒व॒नीय᳚म् । उप॒ प्र । प्राव॑र्तयत् । अ॒व॒र्त॒य॒थ् स्वाहा᳚ । स्वाहेन्द्र॑शत्रुः । इन्द्र॑शत्रुर् वर्द्धस्व । इन्द्र॑शत्रु॒रितीन्द्र॑ - श॒त्रुः॒ । व॒र्द्ध॒स्वेति॑ । इति॒ सः । स याव॑त् । याव॑दू॒र्द्ध्वः । ऊ॒र्द्ध्वः प॑रा॒विद्ध्य॑ति । प॒रा॒विद्ध्य॑ति॒ ताव॑ति । प॒रा॒विद्ध्य॒तीति॑ परा - विद्ध्य॑ति । ताव॑ति स्व॒यम् । स्व॒यमे॒व । ए॒व वि । व्य॑रमत । अ॒र॒म॒त॒ यदि॑ । यदि॑ वा । वा॒ ताव॑त् । ताव॑त् प्रव॒णम् । प्र॒व॒णमासी᳚त् । प्र॒व॒णमिति॑ प्र - व॒नम् \newline

\textbf{Jatai Paata} \newline

1. त्वष्टा॑ ह॒तपु॑त्रो ह॒तपु॑त्र॒ स्त्वष्टा॒ त्वष्टा॑ ह॒तपु॑त्रः । \newline
2. ह॒तपु॑त्रो॒ वीन्द्रं॒ ॅवीन्द्रꣳ॑ ह॒तपु॑त्रो ह॒तपु॑त्रो॒ वीन्द्र᳚म् । \newline
3. ह॒तपु॑त्र॒ इति॑ ह॒त - पु॒त्रः॒ । \newline
4. वीन्द्रꣳ॒॒ सोमꣳ॒॒ सोमं॒ ॅवीन्द्रं॒ ॅवीन्द्रꣳ॒॒ सोम᳚म् । \newline
5. वीन्द्र॒मिति॒ वि - इ॒न्द्र॒म् । \newline
6. सोम॒ मा सोमꣳ॒॒ सोम॒ मा । \newline
7. आ ऽह॑र दहर॒दा ऽह॑रत् । \newline
8. अ॒ह॒र॒त् तस्मिꣳ॒॒ स्तस्मि॑न् नहर दहर॒त् तस्मिन्न्॑ । \newline
9. तस्मि॒न् निन्द्र॒ इन्द्र॒ स्तस्मिꣳ॒॒ स्तस्मि॒न् निन्द्रः॑ । \newline
10. इन्द्र॑ उपह॒व मु॑पह॒व मिन्द्र॒ इन्द्र॑ उपह॒वम् । \newline
11. उ॒प॒ह॒व मै᳚च्छ तैच्छतो पह॒व मु॑पह॒व मै᳚च्छत । \newline
12. उ॒प॒ह॒वमित्यु॑प - ह॒वम् । \newline
13. ऐ॒च्छ॒त॒ तम् त मै᳚च्छतैच्छत॒ तम् । \newline
14. तम् न न तम् तम् न । \newline
15. नोपोप॒ न नोप॑ । \newline
16. उपा᳚ह्वयता ह्वय॒तो पोपा᳚ ह्वयत । \newline
17. अ॒ह्व॒य॒त॒ पु॒त्रम् पु॒त्र म॑ह्वयता ह्वयत पु॒त्रम् । \newline
18. पु॒त्रम् मे॑ मे पु॒त्रम् पु॒त्रम् मे᳚ । \newline
19. मे॒ ऽव॒धी॒ र॒व॒धी॒र् मे॒ मे॒ ऽव॒धीः॒ । \newline
20. अ॒व॒धी॒ रिती त्य॑वधी रवधी॒ रिति॑ । \newline
21. इति॒ स स इतीति॒ सः । \newline
22. स य॑ज्ञ्वेश॒सं ॅय॑ज्ञ्वेश॒सꣳ स स य॑ज्ञ्वेश॒सम् । \newline
23. य॒ज्ञ्॒वे॒श॒सम् कृ॒त्वा कृ॒त्वा य॑ज्ञ्वेश॒सं ॅय॑ज्ञ्वेश॒सम् कृ॒त्वा । \newline
24. य॒ज्ञ्॒वे॒श॒समिति॑ यज्ञ् - वे॒श॒सम् । \newline
25. कृ॒त्वा प्रा॒सहा᳚ प्रा॒सहा॑ कृ॒त्वा कृ॒त्वा प्रा॒सहा᳚ । \newline
26. प्रा॒सहा॒ सोमꣳ॒॒ सोम॑म् प्रा॒सहा᳚ प्रा॒सहा॒ सोम᳚म् । \newline
27. प्रा॒सहेति॑ प्र - सहा᳚ । \newline
28. सोम॑ मपिबदपिब॒थ् सोमꣳ॒॒ सोम॑ मपिबत् । \newline
29. अ॒पि॒ब॒त् तस्य॒ तस्या॑ पिब दपिब॒त् तस्य॑ । \newline
30. तस्य॒ यद् यत् तस्य॒ तस्य॒ यत् । \newline
31. यद॒त्यशि॑ष्यता॒ त्यशि॑ष्यत॒ यद् यद॒त्यशि॑ष्यत । \newline
32. अ॒त्यशि॑ष्यत॒ तत् तद॒त्यशि॑ष्यता॒ त्यशि॑ष्यत॒ तत् । \newline
33. अ॒त्यशि॑ष्य॒तेत्य॑ति - अशि॑ष्यत । \newline
34. तत् त्वष्टा॒ त्वष्टा॒ तत् तत् त्वष्टा᳚ । \newline
35. त्वष्टा॑ ऽऽहव॒नीय॑ माहव॒नीय॒म् त्वष्टा॒ त्वष्टा॑ ऽऽहव॒नीय᳚म् । \newline
36. आ॒ह॒व॒नीय॒ मुपोपा॑ हव॒नीय॑ माहव॒नीय॒ मुप॑ । \newline
37. आ॒ह॒व॒नीय॒मित्या᳚ - ह॒व॒नीय᳚म् । \newline
38. उप॒ प्र प्रोपोप॒ प्र । \newline
39. प्राव॑र्तय दवर्तय॒त् प्र प्राव॑र्तयत् । \newline
40. अ॒व॒र्त॒य॒थ् स्वाहा॒ स्वाहा॑ ऽवर्तय दवर्तय॒थ् स्वाहा᳚ । \newline
41. स्वाहेन्द्र॑शत्रु॒ रिन्द्र॑शत्रुः॒ स्वाहा॒ स्वाहेन्द्र॑शत्रुः । \newline
42. इन्द्र॑शत्रुर् वर्द्धस्व वर्द्ध॒स्वे न्द्र॑शत्रु॒ रिन्द्र॑शत्रुर् वर्द्धस्व । \newline
43. इन्द्र॑शत्रु॒रितीन्द्र॑ - श॒त्रुः॒ । \newline
44. व॒र्द्ध॒स्वे तीति॑ वर्द्धस्व वर्द्ध॒स्वे ति॑ । \newline
45. इति॒ स स इतीति॒ सः । \newline
46. स याव॒द् याव॒थ् स स याव॑त् । \newline
47. याव॑ दू॒र्द्ध्व ऊ॒र्द्ध्वो याव॒द् याव॑ दू॒र्द्ध्वः । \newline
48. ऊ॒र्द्ध्वः प॑रा॒विद्ध्य॑ति परा॒विद्ध्य॑ त्यू॒र्द्ध्व ऊ॒र्द्ध्वः प॑रा॒विद्ध्य॑ति । \newline
49. प॒रा॒विद्ध्य॑ति॒ ताव॑ति॒ ताव॑ति परा॒विद्ध्य॑ति परा॒विद्ध्य॑ति॒ ताव॑ति । \newline
50. प॒रा॒विद्ध्य॒तीति॑ परा - विद्ध्य॑ति । \newline
51. ताव॑ति स्व॒यꣳ स्व॒यम् ताव॑ति॒ ताव॑ति स्व॒यम् । \newline
52. स्व॒य मे॒वैव स्व॒यꣳ स्व॒य मे॒व । \newline
53. ए॒व वि व्ये॑वैव वि । \newline
54. व्य॑रमता रमत॒ वि व्य॑रमत । \newline
55. अ॒र॒म॒त॒ यदि॒ यद्य॑रमता रमत॒ यदि॑ । \newline
56. यदि॑ वा वा॒ यदि॒ यदि॑ वा । \newline
57. वा॒ ताव॒त् ताव॑द् वा वा॒ ताव॑त् । \newline
58. ताव॑त् प्रव॒णम् प्र॑व॒णम् ताव॒त् ताव॑त् प्रव॒णम् । \newline
59. प्र॒व॒ण मासी॒ दासी᳚त् प्रव॒णम् प्र॑व॒ण मासी᳚त् । \newline
60. प्र॒व॒णमिति॑ प्र - व॒नम् । \newline

\textbf{Ghana Paata } \newline

1. त्वष्टा॑ ह॒तपु॑त्रो ह॒तपु॑त्र॒ स्त्वष्टा॒ त्वष्टा॑ ह॒तपु॑त्रो॒ वीन्द्रं॒ ॅवीन्द्रꣳ॑ ह॒तपु॑त्र॒ स्त्वष्टा॒ त्वष्टा॑ ह॒तपु॑त्रो॒ वीन्द्र᳚म् । \newline
2. ह॒तपु॑त्रो॒ वीन्द्रं॒ ॅवीन्द्रꣳ॑ ह॒तपु॑त्रो ह॒तपु॑त्रो॒ वीन्द्रꣳ॒॒ सोमꣳ॒॒ सोमं॒ ॅवीन्द्रꣳ॑ ह॒तपु॑त्रो ह॒तपु॑त्रो॒ वीन्द्रꣳ॒॒ सोम᳚म् । \newline
3. ह॒तपु॑त्र॒ इति॑ ह॒त - पु॒त्रः॒ । \newline
4. वीन्द्रꣳ॒॒ सोमꣳ॒॒ सोमं॒ ॅवीन्द्रं॒ ॅवीन्द्रꣳ॒॒ सोम॒ मा सोमं॒ ॅवीन्द्रं॒ ॅवीन्द्रꣳ॒॒ सोम॒ मा । \newline
5. वीन्द्र॒मिति॒ वि - इ॒न्द्र॒म् । \newline
6. सोम॒ मा सोमꣳ॒॒ सोम॒ मा ऽह॑र दहर॒दा सोमꣳ॒॒ सोम॒ मा ऽह॑रत् । \newline
7. आ ऽह॑र दहर॒दा ऽह॑र॒त् तस्मिꣳ॒॒ स्तस्मि॑न् नहर॒दा ऽह॑र॒त् तस्मिन्न्॑ । \newline
8. अ॒ह॒र॒त् तस्मिꣳ॒॒ स्तस्मि॑न् नहर दहर॒त् तस्मि॒न् निन्द्र॒ इन्द्र॒ स्तस्मि॑न् नहर दहर॒त् तस्मि॒न् निन्द्रः॑ । \newline
9. तस्मि॒न् निन्द्र॒ इन्द्र॒ स्तस्मिꣳ॒॒ स्तस्मि॒न् निन्द्र॑ उपह॒व मु॑पह॒व मिन्द्र॒ स्तस्मिꣳ॒॒ स्तस्मि॒न् निन्द्र॑ उपह॒वम् । \newline
10. इन्द्र॑ उपह॒व मु॑पह॒व मिन्द्र॒ इन्द्र॑ उपह॒व मै᳚च्छतैच्छतो पह॒व मिन्द्र॒ इन्द्र॑ उपह॒व मै᳚च्छत । \newline
11. उ॒प॒ह॒व मै᳚च्छतैच्छतो पह॒व मु॑पह॒व मै᳚च्छत॒ तम् त मै᳚च्छतो पह॒व मु॑पह॒व मै᳚च्छत॒ तम् । \newline
12. उ॒प॒ह॒वमित्यु॑प - ह॒वम् । \newline
13. ऐ॒च्छ॒त॒ तम् त मै᳚च्छतैच्छत॒ तम् न न त मै᳚च्छतैच्छत॒ तम् न । \newline
14. तम् न न तम् तन्नोपोप॒ न तम् तम् नोप॑ । \newline
15. नोपोप॒ न नोपा᳚ ह्वयता ह्वय॒तोप॒ न नोपा᳚ह्वयत । \newline
16. उपा᳚ह्वयता ह्वय॒तो पोपा᳚ह्वयत पु॒त्रम् पु॒त्र म॑ह्वय॒तो पोपा᳚ह्वयत पु॒त्रम् । \newline
17. अ॒ह्व॒य॒त॒ पु॒त्रम् पु॒त्र म॑ह्वयता ह्वयत पु॒त्रम् मे॑ मे पु॒त्र म॑ह्वयता ह्वयत पु॒त्रम् मे᳚ । \newline
18. पु॒त्रम् मे॑ मे पु॒त्रम् पु॒त्रम् मे॑ ऽवधी रवधीर् मे पु॒त्रम् पु॒त्रम् मे॑ ऽवधीः । \newline
19. मे॒ ऽव॒धी॒ र॒व॒धी॒र् मे॒ मे॒ ऽव॒धी॒ रिती त्य॑वधीर् मे मे ऽवधी॒रिति॑ । \newline
20. अ॒व॒धी॒ रिती त्य॑वधी रवधी॒ रिति॒ स स इत्य॑वधी रवधी॒रिति॒ सः । \newline
21. इति॒ स स इतीति॒ स य॑ज्ञ्वेश॒सं ॅय॑ज्ञ्वेश॒सꣳ स इतीति॒ स य॑ज्ञ्वेश॒सम् । \newline
22. स य॑ज्ञ्वेश॒सं ॅय॑ज्ञ्वेश॒सꣳ स स य॑ज्ञ्वेश॒सम् कृ॒त्वा कृ॒त्वा य॑ज्ञ्वेश॒सꣳ स स य॑ज्ञ्वेश॒सम् कृ॒त्वा । \newline
23. य॒ज्ञ्॒वे॒श॒सम् कृ॒त्वा कृ॒त्वा य॑ज्ञ्वेश॒सं ॅय॑ज्ञ्वेश॒सम् कृ॒त्वा प्रा॒सहा᳚ प्रा॒सहा॑ कृ॒त्वा य॑ज्ञ्वेश॒सं ॅय॑ज्ञ्वेश॒सम् कृ॒त्वा प्रा॒सहा᳚ । \newline
24. य॒ज्ञ्॒वे॒श॒समिति॑ यज्ञ् - वे॒श॒सम् । \newline
25. कृ॒त्वा प्रा॒सहा᳚ प्रा॒सहा॑ कृ॒त्वा कृ॒त्वा प्रा॒सहा॒ सोमꣳ॒॒ सोम॑म् प्रा॒सहा॑ कृ॒त्वा कृ॒त्वा प्रा॒सहा॒ सोम᳚म् । \newline
26. प्रा॒सहा॒ सोमꣳ॒॒ सोम॑म् प्रा॒सहा᳚ प्रा॒सहा॒ सोम॑ मपिब दपिब॒थ् सोम॑म् प्रा॒सहा᳚ प्रा॒सहा॒ सोम॑ मपिबत् । \newline
27. प्रा॒सहेति॑ प्र - सहा᳚ । \newline
28. सोम॑ मपिब दपिब॒थ् सोमꣳ॒॒ सोम॑ मपिब॒त् तस्य॒ तस्या॑पिब॒थ् सोमꣳ॒॒ सोम॑ मपिब॒त् तस्य॑ । \newline
29. अ॒पि॒ब॒त् तस्य॒ तस्या॑पिब दपिब॒त् तस्य॒ यद् यत् तस्या॑पिब दपिब॒त् तस्य॒ यत् । \newline
30. तस्य॒ यद् यत् तस्य॒ तस्य॒ यद॒त्यशि॑ष्यता॒ त्यशि॑ष्यत॒ यत् तस्य॒ तस्य॒ यद॒त्यशि॑ष्यत । \newline
31. यद॒त्यशि॑ष्यता॒ त्यशि॑ष्यत॒ यद् यद॒त्यशि॑ष्यत॒ तत् तद॒त्यशि॑ष्यत॒ यद् यद॒त्यशि॑ष्यत॒ तत् । \newline
32. अ॒त्यशि॑ष्यत॒ तत् तद॒त्यशि॑ष्यता॒ त्यशि॑ष्यत॒ तत् त्वष्टा॒ त्वष्टा॒ तद॒त्यशि॑ष्यता॒ त्यशि॑ष्यत॒ तत् त्वष्टा᳚ । \newline
33. अ॒त्यशि॑ष्य॒तेत्य॑ति - अशि॑ष्यत । \newline
34. तत् त्वष्टा॒ त्वष्टा॒ तत् तत् त्वष्टा॑ ऽऽहव॒नीय॑ माहव॒नीय॒म् त्वष्टा॒ तत् तत् त्वष्टा॑ ऽऽहव॒नीय᳚म् । \newline
35. त्वष्टा॑ ऽऽहव॒नीय॑ माहव॒नीय॒म् त्वष्टा॒ त्वष्टा॑ ऽऽहव॒नीय॒ मुपोपा॑हव॒नीय॒म् त्वष्टा॒ त्वष्टा॑ ऽऽहव॒नीय॒ मुप॑ । \newline
36. आ॒ह॒व॒नीय॒ मुपोपा॑हव॒नीय॑ माहव॒नीय॒ मुप॒ प्र प्रोपा॑हव॒नीय॑ माहव॒नीय॒ मुप॒ प्र । \newline
37. आ॒ह॒व॒नीय॒मित्या᳚ - ह॒व॒नीय᳚म् । \newline
38. उप॒ प्र प्रोपोप॒ प्राव॑र्तय दवर्तय॒त् प्रोपोप॒ प्राव॑र्तयत् । \newline
39. प्राव॑र्तय दवर्तय॒त् प्र प्राव॑र्तय॒थ् स्वाहा॒ स्वाहा॑ ऽवर्तय॒त् प्र प्राव॑र्तय॒थ् स्वाहा᳚ । \newline
40. अ॒व॒र्त॒य॒थ् स्वाहा॒ स्वाहा॑ ऽवर्तय दवर्तय॒थ् स्वाहेन्द्र॑शत्रु॒ रिन्द्र॑शत्रुः॒ स्वाहा॑ ऽवर्तय दवर्तय॒थ् स्वाहेन्द्र॑शत्रुः । \newline
41. स्वाहेन्द्र॑शत्रु॒ रिन्द्र॑शत्रुः॒ स्वाहा॒ स्वाहेन्द्र॑शत्रुर् वर्द्धस्व वर्द्ध॒स्वे न्द्र॑शत्रुः॒ स्वाहा॒ स्वाहेन्द्र॑शत्रुर् वर्द्धस्व । \newline
42. इन्द्र॑शत्रुर् वर्द्धस्व वर्द्ध॒स्वे न्द्र॑शत्रु॒ रिन्द्र॑शत्रुर् वर्द्ध॒स्वे तीति॑ वर्द्ध॒स्वे न्द्र॑शत्रु॒ रिन्द्र॑शत्रुर् वर्द्ध॒स्वे ति॑ । \newline
43. इन्द्र॑शत्रु॒रितीन्द्र॑ - श॒त्रुः॒ । \newline
44. व॒र्द्ध॒स्वे तीति॑ वर्द्धस्व वर्द्ध॒स्वे ति॒ स स इति॑ वर्द्धस्व वर्द्ध॒स्वे ति॒ सः । \newline
45. इति॒ स स इतीति॒ स याव॒द् याव॒थ् स इतीति॒ स याव॑त् । \newline
46. स याव॒द् याव॒थ् स स याव॑दू॒र्द्ध्व ऊ॒र्द्ध्वो याव॒थ् स स याव॑दू॒र्द्ध्वः । \newline
47. याव॑दू॒र्द्ध्व ऊ॒र्द्ध्वो याव॒द् याव॑दू॒र्द्ध्वः प॑रा॒विद्ध्य॑ति परा॒विद्ध्य॑ त्यू॒र्द्ध्वो याव॒द् याव॑दू॒र्द्ध्वः प॑रा॒विद्ध्य॑ति । \newline
48. ऊ॒र्द्ध्वः प॑रा॒विद्ध्य॑ति परा॒विद्ध्य॑ त्यू॒र्द्ध्व ऊ॒र्द्ध्वः प॑रा॒विद्ध्य॑ति॒ ताव॑ति॒ ताव॑ति परा॒विद्ध्य॑ त्यू॒र्द्ध्व ऊ॒र्द्ध्वः प॑रा॒विद्ध्य॑ति॒ ताव॑ति । \newline
49. प॒रा॒विद्ध्य॑ति॒ ताव॑ति॒ ताव॑ति परा॒विद्ध्य॑ति परा॒विद्ध्य॑ति॒ ताव॑ति स्व॒यꣳ स्व॒यम् ताव॑ति परा॒विद्ध्य॑ति परा॒विद्ध्य॑ति॒ ताव॑ति स्व॒यम् । \newline
50. प॒रा॒विद्ध्य॒तीति॑ परा - विद्ध्य॑ति । \newline
51. ताव॑ति स्व॒यꣳ स्व॒यम् ताव॑ति॒ ताव॑ति स्व॒य मे॒वैव स्व॒यम् ताव॑ति॒ ताव॑ति स्व॒य मे॒व । \newline
52. स्व॒य मे॒वैव स्व॒यꣳ स्व॒य मे॒व वि व्ये॑व स्व॒यꣳ स्व॒य मे॒व वि । \newline
53. ए॒व वि व्ये॑वैव व्य॑रमता रमत॒ व्ये॑वैव व्य॑रमत । \newline
54. व्य॑रमता रमत॒ वि व्य॑रमत॒ यदि॒ यद्य॑रमत॒ वि व्य॑रमत॒ यदि॑ । \newline
55. अ॒र॒म॒त॒ यदि॒ यद्य॑रमता रमत॒ यदि॑ वा वा॒ यद्य॑रमता रमत॒ यदि॑ वा । \newline
56. यदि॑ वा वा॒ यदि॒ यदि॑ वा॒ ताव॒त् ताव॑द् वा॒ यदि॒ यदि॑ वा॒ ताव॑त् । \newline
57. वा॒ ताव॒त् ताव॑द् वा वा॒ ताव॑त् प्रव॒णम् प्र॑व॒णम् ताव॑द् वा वा॒ ताव॑त् प्रव॒णम् । \newline
58. ताव॑त् प्रव॒णम् प्र॑व॒णम् ताव॒त् ताव॑त् प्रव॒ण मासी॒ दासी᳚त् प्रव॒णम् ताव॒त् ताव॑त् प्रव॒ण मासी᳚त् । \newline
59. प्र॒व॒ण मासी॒ दासी᳚त् प्रव॒णम् प्र॑व॒ण मासी॒द् यदि॒ यद्यासी᳚त् प्रव॒णम् प्र॑व॒ण मासी॒द् यदि॑ । \newline
60. प्र॒व॒णमिति॑ प्र - व॒नम् । \newline
\pagebreak
\markright{ TS 2.4.12.2  \hfill https://www.vedavms.in \hfill}

\section{ TS 2.4.12.2 }

\textbf{TS 2.4.12.2 } \newline
\textbf{Samhita Paata} \newline

-मासी॒द्यदि॑ वा॒ ताव॒दद्ध्य॒ग्नेरासी॒थ् स सं॒भव॑न्न॒ग्नीषोमा॑व॒भि सम॑भव॒थ् स इ॑षुमा॒त्रमि॑षुमात्रं॒ ॅविष्व॑ङ्ङवर्द्धत॒ स इ॒मां ॅलो॒कान॑वृणो॒द्य-दि॒मां ॅलो॒कानवृ॑णो॒त् तद्-वृ॒त्रस्य॑ वृत्र॒त्वं तस्मा॒दिन्द्रो॑ऽबिभे॒दपि॒ त्वष्टा॒ तस्मै॒ त्वष्टा॒ वज्र॑मसिञ्च॒त् तपो॒ वै स वज्र॑ आसी॒त् तमुद्य॑न्तुं॒ नाश॑क्नो॒दथ॒ वै तर्.हि॒ विष्णु॑ - [  ] \newline

\textbf{Pada Paata} \newline

आसी᳚त् । यदि॑ । वा॒ । ताव॑त् । अधीति॑ । अ॒ग्नेः । आसी᳚त् ।   सः । सं॒भव॒न्निति॑ सं - भवन्न्॑ । अ॒ग्नीषोमा॒वित्य॒ग्नी - सोमौ᳚ । अ॒भि । समिति॑ । अ॒भ॒व॒त् । सः । इ॒षु॒मा॒त्रमि॑षुमात्र॒मिती॑षुमा॒त्रं - इ॒षु॒मा॒त्र॒म् । विष्वङ्॑ । अ॒व॒र्द्ध॒त॒ । सः । इ॒मान् । लो॒कान् । अ॒वृ॒णो॒त् । यत् । इ॒मान् । लो॒कान् । अवृ॑णोत् । तत् । वृ॒त्रस्य॑ । वृ॒त्र॒त्वमिति॑ वृत्र - त्वम् । तस्मा᳚त् । इन्द्रः॑ । अ॒बि॒भे॒त् । अपीति॑ । त्वष्टा᳚ । तस्मै᳚ । त्वष्टा᳚ । वज्र᳚म् । अ॒सि॒ञ्च॒त् । तपः॑ । वै । सः । वज्रः॑ । आ॒सी॒त् । तम् । उद्य॑न्तु॒मित्युत् -   य॒न्तु॒म् । न । अ॒श॒क्नो॒त् । अथ॑ । वै । तर्.हि॑ । विष्णुः॑ ।  \newline


\textbf{Krama Paata} \newline

आसी॒द् यदि॑ । यदि॑ वा । वा॒ ताव॑त् । ताव॒दधि॑ । अद्ध्य॒ग्नेः । अ॒ग्नेरासी᳚त् । आसी॒थ् सः । स स॒म्भवन्न्॑ । स॒म्भव॑न्न॒ग्नीषोमौ᳚ । स॒म्भव॒न्निति॑ सम् - भवन्न्॑ । अ॒ग्नीषोमा॑व॒भि । अ॒ग्नीषोमा॒वित्य॒ग्नी - सोमौ᳚ । अ॒भि सम् । सम॑भवत् । अ॒भ॒व॒थ् सः । स इ॑षुमा॒त्रमि॑षुमात्रम् । इ॒षु॒मा॒त्रमि॑षुमात्रं॒ ॅविष्वङ्ङ्॑ । इ॒षु॒मा॒त्रमि॑षुमात्र॒मिती॑षुमा॒त्रम् - इ॒षु॒मा॒त्र॒म् । विष्व॑ङ्ङवर्द्धत । अ॒व॒र्द्ध॒त॒ सः । स इ॒मान् । इ॒मान् ॅलो॒कान् । लो॒कान॑वृणोत् । अ॒वृ॒णो॒द् यत् । यदि॒मान् । इ॒मान् ॅलो॒कान् । लो॒कानवृ॑णोत् । अवृ॑णो॒त् तत् । तद् वृ॒त्रस्य॑ । वृ॒त्रस्य॑ वृत्र॒त्वम् । वृ॒त्र॒त्वम् तस्मा᳚त् । वृ॒त्र॒त्वमिति॑ वृत्र - त्वम् । तस्मा॒दिन्द्रः॑ । इन्द्रो॑ ऽबिभेत् । अ॒बि॒भे॒दपि॑ । अपि॒ त्वष्टा᳚ । त्वष्टा॒ तस्मै᳚ । तस्मै॒ त्वष्टा᳚ । त्वष्टा॒ वज्र᳚म् । वज्र॑मसिञ्चत् । अ॒सि॒ञ्च॒त् तपः॑ । तपो॒ वै । वै सः । स वज्रः॑ । वज्र॑ आसीत् । आ॒सी॒त् तम् । तमुद्य॑न्तुम् । उद्य॑न्तु॒म् न । उद्य॑न्तु॒मित्युत् - य॒न्तु॒म् । नाश॑क्नोत् । अ॒श॒क्नो॒दथ॑ । अथ॒ वै । वै तर्.हि॑ । तर्.हि॒ विष्णुः॑ । विष्णु॑र॒न्या \newline

\textbf{Jatai Paata} \newline

1. आसी॒द् यदि॒ यद्यासी॒ दासी॒द् यदि॑ । \newline
2. यदि॑ वा वा॒ यदि॒ यदि॑ वा । \newline
3. वा॒ ताव॒त् ताव॑द् वा वा॒ ताव॑त् । \newline
4. ताव॒ दध्यधि॒ ताव॒त् ताव॒ दधि॑ । \newline
5. अध्य॒ग्ने र॒ग्ने रध्यध्य॒ग्नेः । \newline
6. अ॒ग्ने रासी॒ दासी॑ द॒ग्ने र॒ग्ने रासी᳚त् । \newline
7. आसी॒थ् स स आसी॒ दासी॒थ् सः । \newline
8. स सं॒भवन्᳚ थ्सं॒भव॒न् थ्स स सं॒भवन्न्॑ । \newline
9. सं॒भव॑न् न॒ग्नीषोमा॑ व॒ग्नीषोमौ॑ सं॒भवन्᳚ थ्सं॒भव॑न् न॒ग्नीषोमौ᳚ । \newline
10. सं॒भव॒न्निति॑ सं - भवन्न्॑ । \newline
11. अ॒ग्नीषोमा॑ व॒भ्या᳚(1॒)भ्य॑ग्नीषोमा॑ व॒ग्नीषोमा॑ व॒भि । \newline
12. अ॒ग्नीषोमा॒वित्य॒ग्नी - सोमौ᳚ । \newline
13. अ॒भि सꣳ स म॒भ्य॑भि सम् । \newline
14. स म॑भव दभव॒थ् सꣳ स म॑भवत् । \newline
15. अ॒भ॒व॒थ् स सो॑ ऽभव दभव॒थ् सः । \newline
16. स इ॑षुमा॒त्रमि॑षुमात्र मिषुमा॒त्रमि॑षुमात्रꣳ॒॒ स स इ॑षुमा॒त्रमि॑षुमात्रम् । \newline
17. इ॒षु॒मा॒त्रमि॑षुमात्रं॒ ॅविष्व॒ङ्॒. विष्वङ् ङि॑षुमा॒त्रमि॑षुमात्र मिषुमा॒त्रमि॑षुमात्रं॒ ॅविष्वङ्॑ । \newline
18. इ॒षु॒मा॒त्रमि॑षुमात्र॒मिती॑षुमा॒त्रं - इ॒षु॒मा॒त्र॒म् । \newline
19. विष्व॑ङ् ङवर्द्धता वर्द्धत॒ विष्व॒ङ्॒. विष्व॑ङ् ङवर्द्धत । \newline
20. अ॒व॒र्द्ध॒त॒ स सो॑ ऽवर्द्धता वर्द्धत॒ सः । \newline
21. स इ॒मा नि॒मान् थ्स स इ॒मान् । \newline
22. इ॒मान् ॅलो॒कान् ॅलो॒का नि॒मा नि॒मान् ॅलो॒कान् । \newline
23. लो॒का न॑वृणो दवृणो ल्लो॒कान् ॅलो॒का न॑वृणोत् । \newline
24. अ॒वृ॒णो॒द् यद् यद॑वृणो दवृणो॒द् यत् । \newline
25. यदि॒मा नि॒मान्. यद् यदि॒मान् । \newline
26. इ॒मान् ॅलो॒कान् ॅलो॒का नि॒मा नि॒मान् ॅलो॒कान् । \newline
27. लो॒का नवृ॑णो॒ दवृ॑णो ल्लो॒कान् ॅलो॒का नवृ॑णोत् । \newline
28. अवृ॑णो॒त् तत् तदवृ॑णो॒ दवृ॑णो॒त् तत् । \newline
29. तद् वृ॒त्रस्य॑ वृ॒त्रस्य॒ तत् तद् वृ॒त्रस्य॑ । \newline
30. वृ॒त्रस्य॑ वृत्र॒त्वं ॅवृ॑त्र॒त्वं ॅवृ॒त्रस्य॑ वृ॒त्रस्य॑ वृत्र॒त्वम् । \newline
31. वृ॒त्र॒त्वम् तस्मा॒त् तस्मा᳚द् वृत्र॒त्वं ॅवृ॑त्र॒त्वम् तस्मा᳚त् । \newline
32. वृ॒त्र॒त्वमिति॑ वृत्र - त्वम् । \newline
33. तस्मा॒ दिन्द्र॒ इन्द्र॒ स्तस्मा॒त् तस्मा॒ दिन्द्रः॑ । \newline
34. इन्द्रो॑ ऽबिभे दबिभे॒ दिन्द्र॒ इन्द्रो॑ ऽबिभेत् । \newline
35. अ॒बि॒भे॒ दप्य प्य॑बिभे दबिभे॒दपि॑ । \newline
36. अपि॒ त्वष्टा॒ त्वष्टा ऽप्यपि॒ त्वष्टा᳚ । \newline
37. त्वष्टा॒ तस्मै॒ तस्मै॒ त्वष्टा॒ त्वष्टा॒ तस्मै᳚ । \newline
38. तस्मै॒ त्वष्टा॒ त्वष्टा॒ तस्मै॒ तस्मै॒ त्वष्टा᳚ । \newline
39. त्वष्टा॒ वज्रं॒ ॅवज्र॒म् त्वष्टा॒ त्वष्टा॒ वज्र᳚म् । \newline
40. वज्र॑ मसिञ्च दसिञ्च॒द् वज्रं॒ ॅवज्र॑ मसिञ्चत् । \newline
41. अ॒सि॒ञ्च॒त् तप॒ स्तपो॑ ऽसिञ्च दसिञ्च॒त् तपः॑ । \newline
42. तपो॒ वै वै तप॒ स्तपो॒ वै । \newline
43. वै स स वै वै सः । \newline
44. स वज्रो॒ वज्रः॒ स स वज्रः॑ । \newline
45. वज्र॑ आसी दासी॒द् वज्रो॒ वज्र॑ आसीत् । \newline
46. आ॒सी॒त् तम् त मा॑सी दासी॒त् तम् । \newline
47. त मुद्य॑न्तु॒ मुद्य॑न्तु॒म् तम् त मुद्य॑न्तुम् । \newline
48. उद्य॑न्तु॒म् न नोद्य॑न्तु॒ मुद्य॑न्तु॒म् न । \newline
49. उद्य॑न्तु॒मित्युत् - य॒न्तु॒म् । \newline
50. नाश॑क्नो दशक्नो॒न् न नाश॑क्नोत् । \newline
51. अ॒श॒क्नो॒ दथाथा॑ शक्नो दशक्नो॒ दथ॑ । \newline
52. अथ॒ वै वा अथाथ॒ वै । \newline
53. वै तर्.हि॒ तर्.हि॒ वै वै तर्.हि॑ । \newline
54. तर्.हि॒ विष्णु॒र् विष्णु॒ स्तर्.हि॒ तर्.हि॒ विष्णुः॑ । \newline
55. विष्णु॑ र॒न्या ऽन्या विष्णु॒र् विष्णु॑ र॒न्या । \newline

\textbf{Ghana Paata } \newline

1. आसी॒द् यदि॒ यद्यासी॒ दासी॒द् यदि॑ वा वा॒ यद्यासी॒ दासी॒द् यदि॑ वा । \newline
2. यदि॑ वा वा॒ यदि॒ यदि॑ वा॒ ताव॒त् ताव॑द् वा॒ यदि॒ यदि॑ वा॒ ताव॑त् । \newline
3. वा॒ ताव॒त् ताव॑द् वा वा॒ ताव॒ दध्यधि॒ ताव॑द् वा वा॒ ताव॒दधि॑ । \newline
4. ताव॒ दध्यधि॒ ताव॒त् ताव॒ दध्य॒ग्ने र॒ग्ने रधि॒ ताव॒त् ताव॒ दध्य॒ग्नेः । \newline
5. अध्य॒ग्ने र॒ग्ने रध्यध्य॒ग्ने रासी॒ दासी॑ द॒ग्ने रध्यध्य॒ग्ने रासी᳚त् । \newline
6. अ॒ग्नेरासी॒ दासी॑ द॒ग्ने र॒ग्ने रासी॒थ् स स आसी॑ द॒ग्ने र॒ग्ने रासी॒थ् सः । \newline
7. आसी॒थ् स स आसी॒ दासी॒थ् स सं॒भवन्᳚ थ्सं॒भव॒न् थ्स आसी॒ दासी॒थ् स सं॒भवन्न्॑ । \newline
8. स सं॒भवन्᳚ थ्सं॒भव॒न् थ्स स सं॒भव॑न् न॒ग्नीषोमा॑ व॒ग्नीषोमौ॑ सं॒भव॒न् थ्स स सं॒भव॑न् न॒ग्नीषोमौ᳚ । \newline
9. सं॒भव॑न् न॒ग्नीषोमा॑ व॒ग्नीषोमौ॑ सं॒भवन्᳚ थ्सं॒भव॑न् न॒ग्नीषोमा॑ व॒भ्या᳚(1॒)भ्य॑ग्नीषोमौ॑ सं॒भवन्᳚ थ्सं॒भव॑न् न॒ग्नीषोमा॑ व॒भि । \newline
10. सं॒भव॒न्निति॑ सं - भवन्न्॑ । \newline
11. अ॒ग्नीषोमा॑ व॒भ्या᳚(1॒)भ्य॑ग्नीषोमा॑ व॒ग्नीषोमा॑ व॒भि सꣳ स म॒भ्य॑ग्नीषोमा॑ व॒ग्नीषोमा॑ व॒भि सम् । \newline
12. अ॒ग्नीषोमा॒वित्य॒ग्नी - सोमौ᳚ । \newline
13. अ॒भि सꣳ स म॒भ्य॑भि स म॑भव दभव॒थ् स म॒भ्य॑भि स म॑भवत् । \newline
14. स म॑भव दभव॒थ् सꣳ स म॑भव॒थ् स सो॑ ऽभव॒थ् सꣳ स म॑भव॒थ् सः । \newline
15. अ॒भ॒व॒थ् स सो॑ ऽभव दभव॒थ् स इ॑षुमा॒त्रमि॑षुमात्र मिषुमा॒त्रमि॑षुमात्रꣳ॒॒ सो॑ ऽभव दभव॒थ् स इ॑षुमा॒त्रमि॑षुमात्रम् । \newline
16. स इ॑षुमा॒त्रमि॑षुमात्र मिषुमा॒त्रमि॑षुमात्रꣳ॒॒ स स इ॑षुमा॒त्रमि॑षुमात्रं॒ ॅविष्व॒ङ्॒. 
विष्वङ् ङि॑षुमा॒त्रमि॑षुमात्रꣳ॒॒ स स इ॑षुमा॒त्रमि॑षुमात्रं॒ ॅविष्वङ्॑ । \newline
17. इ॒षु॒मा॒त्रमि॑षुमात्रं॒ ॅविष्व॒ङ्॒. विष्वङ् ङि॑षुमा॒त्रमि॑षुमात्र मिषुमा॒त्रमि॑षुमात्रं॒ ॅविष्व॑ङ् ङवर्द्धता वर्द्धत॒ विष्वङ् ङि॑षुमा॒त्रमि॑षुमात्र मिषुमा॒त्रमि॑षुमात्रं॒ ॅविष्व॑ङ् ङवर्द्धत । \newline
18. इ॒षु॒मा॒त्रमि॑षुमात्र॒मिती॑षुमा॒त्रं - इ॒षु॒मा॒त्र॒म् । \newline
19. विष्व॑ङ् ङवर्द्धता वर्द्धत॒ विष्व॒ङ्॒. विष्व॑ङ् ङवर्द्धत॒ स सो॑ ऽवर्द्धत॒ विष्व॒ङ्॒. विष्व॑ङ् ङवर्द्धत॒ सः । \newline
20. अ॒व॒र्द्ध॒त॒ स सो॑ ऽवर्द्धता वर्द्धत॒ स इ॒मा नि॒मान् थ्सो॑ ऽवर्द्धता वर्द्धत॒ स इ॒मान् । \newline
21. स इ॒मा नि॒मान् थ्स स इ॒मान् ॅलो॒कान् ॅलो॒का नि॒मान् थ्स स इ॒मान् ॅलो॒कान् । \newline
22. इ॒मान् ॅलो॒कान् ॅलो॒का नि॒मा नि॒मान् ॅलो॒का न॑वृणो दवृणो ल्लो॒का नि॒मा नि॒मान् ॅलो॒का न॑वृणोत् । \newline
23. लो॒का न॑वृणो दवृणो ल्लो॒कान् ॅलो॒का न॑वृणो॒द् यद् यद॑वृणो ल्लो॒कान् ॅलो॒का न॑वृणो॒द् यत् । \newline
24. अ॒वृ॒णो॒द् यद् यद॑वृणो दवृणो॒द् यदि॒मा नि॒मान्. यद॑वृणो दवृणो॒द् यदि॒मान् । \newline
25. यदि॒मा नि॒मान्. यद् यदि॒मान् ॅलो॒कान् ॅलो॒का नि॒मान्. यद् यदि॒मान् ॅलो॒कान् । \newline
26. इ॒मान् ॅलो॒कान् ॅलो॒का नि॒मा नि॒मान् ॅलो॒का नवृ॑णो॒ दवृ॑णो ल्लो॒का नि॒मा नि॒मान् ॅलो॒का नवृ॑णोत् । \newline
27. लो॒का नवृ॑णो॒ दवृ॑णो ल्लो॒कान् ॅलो॒का नवृ॑णो॒त् तत् तदवृ॑णो ल्लो॒कान् ॅलो॒का नवृ॑णो॒त् तत् । \newline
28. अवृ॑णो॒त् तत् तदवृ॑णो॒ दवृ॑णो॒त् तद् वृ॒त्रस्य॑ वृ॒त्रस्य॒ तदवृ॑णो॒ दवृ॑णो॒त् तद् वृ॒त्रस्य॑ । \newline
29. तद् वृ॒त्रस्य॑ वृ॒त्रस्य॒ तत् तद् वृ॒त्रस्य॑ वृत्र॒त्वं ॅवृ॑त्र॒त्वं ॅवृ॒त्रस्य॒ तत् तद् वृ॒त्रस्य॑ वृत्र॒त्वम् । \newline
30. वृ॒त्रस्य॑ वृत्र॒त्वं ॅवृ॑त्र॒त्वं ॅवृ॒त्रस्य॑ वृ॒त्रस्य॑ वृत्र॒त्वम् तस्मा॒त् तस्मा᳚द् वृत्र॒त्वं ॅवृ॒त्रस्य॑ वृ॒त्रस्य॑ वृत्र॒त्वम् तस्मा᳚त् । \newline
31. वृ॒त्र॒त्वम् तस्मा॒त् तस्मा᳚द् वृत्र॒त्वं ॅवृ॑त्र॒त्वम् तस्मा॒ दिन्द्र॒ इन्द्र॒ स्तस्मा᳚द् वृत्र॒त्वं ॅवृ॑त्र॒त्वम् तस्मा॒ दिन्द्रः॑ । \newline
32. वृ॒त्र॒त्वमिति॑ वृत्र - त्वम् । \newline
33. तस्मा॒ दिन्द्र॒ इन्द्र॒ स्तस्मा॒त् तस्मा॒ दिन्द्रो॑ ऽबिभे दबिभे॒ दिन्द्र॒ स्तस्मा॒त् तस्मा॒ दिन्द्रो॑ ऽबिभेत् । \newline
34. इन्द्रो॑ ऽबिभे दबिभे॒ दिन्द्र॒ इन्द्रो॑ ऽबिभे॒ दप्यप्य॑ बिभे॒ दिन्द्र॒ इन्द्रो॑ ऽबिभे॒दपि॑ । \newline
35. अ॒बि॒भे॒दप्यप्य॑ बिभे दबिभे॒ दपि॒ त्वष्टा॒ त्वष्टा ऽप्य॑ बिभे दबिभे॒ दपि॒ त्वष्टा᳚ । \newline
36. अपि॒ त्वष्टा॒ त्वष्टा ऽप्यपि॒ त्वष्टा॒ तस्मै॒ तस्मै॒ त्वष्टा ऽप्यपि॒ त्वष्टा॒ तस्मै᳚ । \newline
37. त्वष्टा॒ तस्मै॒ तस्मै॒ त्वष्टा॒ त्वष्टा॒ तस्मै॒ त्वष्टा॒ त्वष्टा॒ तस्मै॒ त्वष्टा॒ त्वष्टा॒ तस्मै॒ त्वष्टा᳚ । \newline
38. तस्मै॒ त्वष्टा॒ त्वष्टा॒ तस्मै॒ तस्मै॒ त्वष्टा॒ वज्रं॒ ॅवज्र॒म् त्वष्टा॒ तस्मै॒ तस्मै॒ त्वष्टा॒ वज्र᳚म् । \newline
39. त्वष्टा॒ वज्रं॒ ॅवज्र॒म् त्वष्टा॒ त्वष्टा॒ वज्र॑ मसिञ्च दसिञ्च॒द् वज्र॒म् त्वष्टा॒ त्वष्टा॒ वज्र॑ मसिञ्चत् । \newline
40. वज्र॑ मसिञ्च दसिञ्च॒द् वज्रं॒ ॅवज्र॑ मसिञ्च॒त् तप॒ स्तपो॑ ऽसिञ्च॒द् वज्रं॒ ॅवज्र॑ मसिञ्च॒त् तपः॑ । \newline
41. अ॒सि॒ञ्च॒त् तप॒ स्तपो॑ ऽसिञ्च दसिञ्च॒त् तपो॒ वै वै तपो॑ ऽसिञ्च दसिञ्च॒त् तपो॒ वै । \newline
42. तपो॒ वै वै तप॒ स्तपो॒ वै स स वै तप॒ स्तपो॒ वै सः । \newline
43. वै स स वै वै स वज्रो॒ वज्रः॒ स वै वै स वज्रः॑ । \newline
44. स वज्रो॒ वज्रः॒ स स वज्र॑ आसी दासी॒द् वज्रः॒ स स वज्र॑ आसीत् । \newline
45. वज्र॑ आसी दासी॒द् वज्रो॒ वज्र॑ आसी॒त् तम् त मा॑सी॒द् वज्रो॒ वज्र॑ आसी॒त् तम् । \newline
46. आ॒सी॒त् तम् त मा॑सी दासी॒त् त मुद्य॑न्तु॒ मुद्य॑न्तु॒म् त मा॑सी दासी॒त् त मुद्य॑न्तुम् । \newline
47. त मुद्य॑न्तु॒ मुद्य॑न्तु॒म् तम् त मुद्य॑न्तु॒म् न नोद्य॑न्तु॒म् तम् त मुद्य॑न्तु॒म् न । \newline
48. उद्य॑न्तु॒म् न नोद्य॑न्तु॒ मुद्य॑न्तु॒म् नाश॑क्नो दशक्नो॒न् नोद्य॑न्तु॒ मुद्य॑न्तु॒म् नाश॑क्नोत् । \newline
49. उद्य॑न्तु॒मित्युत् - य॒न्तु॒म् । \newline
50. नाश॑क्नो दशक्नो॒न् न नाश॑क्नो॒ दथाथा॑ शक्नो॒न् न नाश॑क्नो॒ दथ॑ । \newline
51. अ॒श॒क्नो॒ दथाथा॑ शक्नो दशक्नो॒ दथ॒ वै वा अथा॑शक्नो दशक्नो॒ दथ॒ वै । \newline
52. अथ॒ वै वा अथाथ॒ वै तर्.हि॒ तर्.हि॒ वा अथाथ॒ वै तर्.हि॑ । \newline
53. वै तर्.हि॒ तर्.हि॒ वै वै तर्.हि॒ विष्णु॒र् विष्णु॒ स्तर्.हि॒ वै वै तर्.हि॒ विष्णुः॑ । \newline
54. तर्.हि॒ विष्णु॒र् विष्णु॒ स्तर्.हि॒ तर्.हि॒ विष्णु॑ र॒न्या ऽन्या विष्णु॒ स्तर्.हि॒ तर्.हि॒ विष्णु॑ र॒न्या । \newline
55. विष्णु॑ र॒न्या ऽन्या विष्णु॒र् विष्णु॑ र॒न्या दे॒वता॑ दे॒वता॒ ऽन्या विष्णु॒र् विष्णु॑ र॒न्या दे॒वता᳚ । \newline
\pagebreak
\markright{ TS 2.4.12.3  \hfill https://www.vedavms.in \hfill}

\section{ TS 2.4.12.3 }

\textbf{TS 2.4.12.3 } \newline
\textbf{Samhita Paata} \newline

-र॒न्या दे॒वता॑ ऽऽसी॒थ् सो᳚ऽब्रवी॒द्-विष्ण॒वेही॒दमा ह॑रिष्यावो॒ येना॒यमि॒दमिति॒स विष्णु॑स्त्रे॒धाऽऽत्मानं॒ ॅविन्य॑धत्त पृथि॒व्यां तृती॑यम॒न्तरि॑क्षे॒ तृती॑यं दि॒वि तृती॑य-मभिपर्याव॒र्ताद्-ध्यबि॑भे॒द्यत्-पृ॑थि॒व्यां तृती॑य॒मासी॒त् तेनेन्द्रो॒ वज्र॒मुद॑यच्छ॒द्-विष्ण्व॑नुस्थितः॒ सो᳚ऽब्रवी॒न्मा मे॒ प्र हा॒रस्ति॒ वा इ॒दं - [  ] \newline

\textbf{Pada Paata} \newline

अ॒न्या । दे॒वता᳚ । आ॒सी॒त् । सः । अ॒ब्र॒वी॒त् । विष्णो᳚ । एति॑ । इ॒हि॒ । इ॒दम् । एति॑ । ह॒रि॒ष्या॒वः॒ । येन॑ । अ॒यम् । इ॒दम् । इति॑ । सः । विष्णुः॑ । त्रे॒धा । आ॒त्मान᳚म् । वि । नीति॑ । अ॒ध॒त्त॒ । पृ॒थि॒व्याम् । तृती॑यम् । अ॒न्तरि॑क्षे । तृती॑यम् । दि॒वि । तृती॑यम् । अ॒भि॒प॒र्या॒व॒र्तादित्य॑भि - प॒र्या॒व॒र्तात् । हि । अबि॑भेत् । यत् । पृ॒थि॒व्याम् । तृती॑यम् । आसी᳚त् । तेन॑ । इन्द्रः॑ । वज्र᳚म् । उदिति॑ । अ॒य॒च्छ॒त् । विष्ण्व॑नुस्थित॒ इति॒ विष्णु॑ - अ॒नु॒स्थि॒तः॒ । सः । अ॒ब्र॒वी॒त् । मा । मे॒ । प्रेति॑ । हाः॒ । अस्ति॑ । वै । इ॒दम् ।  \newline


\textbf{Krama Paata} \newline

अ॒न्या दे॒वता᳚ । दे॒वता॑ऽऽसीत् । आ॒सी॒थ् सः । सो᳚ऽब्रवीत् । अ॒ब्र॒वी॒द् विष्णो᳚ । विष्ण॒वा । एहि॑ । इ॒ही॒दम् । इ॒दमा । आ ह॑रिष्यावः । ह॒रि॒ष्या॒वो॒ येन॑ । येना॒यम् । अ॒यमि॒दम् । इ॒दमिति॑ । इति॒ सः । स विष्णुः॑ । विष्णु॑स्त्रे॒धा । त्रे॒धा ऽऽत्मान᳚म् । आ॒त्मानं॒ ॅवि । वि नि । न्य॑धत्त । अ॒ध॒त्त॒ पृ॒थि॒व्याम् । पृ॒थि॒व्याम् तृती॑यम् । तृती॑यम॒न्तरि॑क्षे । अ॒न्तरि॑क्षे॒ तृती॑यम् । तृती॑यम् दि॒वि । दि॒वि तृती॑यम् । तृती॑यमभिपर्याव॒र्तात् । अ॒भि॒प॒र्या॒व॒र्ताद्धि । अ॒भि॒प॒र्या॒व॒र्तादित्य॑भि - प॒र्या॒व॒र्तात् । ह्यबि॑भेत् । अबि॑भे॒द् यत् । यत् पृ॑थि॒व्याम् । पृ॒थि॒व्याम् तृती॑यम् । तृती॑य॒मासी᳚त् । आसी॒त् तेन॑ । तेनेन्द्रः॑ । इन्द्रो॒ वज्र᳚म् । वज्र॒मुत् । उद॑यच्छत् । अ॒य॒च्छ॒द् विष्ण्व॑नुस्थितः । विष्ण्व॑नुस्थितः॒ सः । विष्ण्व॑नुस्थित॒ इति॒ विष्णु॑ - अ॒नु॒स्थि॒तः॒ । सो᳚ऽब्रवीत् । अ॒ब्र॒वी॒न् मा । मा मे᳚ । मे॒ प्र । प्र हाः᳚ । हा॒रस्ति॑ । अस्ति॒ वै । वा इ॒दम् । इ॒दं मयि॑ \newline

\textbf{Jatai Paata} \newline

1. अ॒न्या दे॒वता॑ दे॒वता॒ ऽन्या ऽन्या दे॒वता᳚ । \newline
2. दे॒वता॑ ऽऽसी दासीद् दे॒वता॑ दे॒वता॑ ऽऽसीत् । \newline
3. आ॒सी॒थ् स स आ॑सी दासी॒थ् सः । \newline
4. सो᳚ ऽब्रवी दब्रवी॒थ् स सो᳚ ऽब्रवीत् । \newline
5. अ॒ब्र॒वी॒द् विष्णो॒ विष्णो᳚ ऽब्रवी दब्रवी॒द् विष्णो᳚ । \newline
6. विष्ण॒वा विष्णो॒ विष्ण॒वा । \newline
7. एही॒ह्येहि॑ । \newline
8. इ॒ही॒द मि॒द मि॑हीही॒दम् । \newline
9. इ॒द मेद मि॒द मा । \newline
10. आ ह॑रिष्यावो हरिष्याव॒ आ ह॑रिष्यावः । \newline
11. ह॒रि॒ष्या॒वो॒ येन॒ येन॑ हरिष्यावो हरिष्यावो॒ येन॑ । \newline
12. येना॒य म॒यं ॅयेन॒ येना॒यम् । \newline
13. अ॒य मि॒द मि॒द म॒य म॒य मि॒दम् । \newline
14. इ॒द मितीती॒द मि॒द मिति॑ । \newline
15. इति॒ स स इतीति॒ सः । \newline
16. स विष्णु॒र् विष्णुः॒ स स विष्णुः॑ । \newline
17. विष्णु॑ स्त्रे॒धा त्रे॒धा विष्णु॒र् विष्णु॑ स्त्रे॒धा । \newline
18. त्रे॒धा ऽऽत्मान॑ मा॒त्मान॑म् त्रे॒धा त्रे॒धा ऽऽत्मान᳚म् । \newline
19. आ॒त्मानं॒ ॅवि व्या᳚त्मान॑ मा॒त्मानं॒ ॅवि । \newline
20. वि नि नि वि वि नि । \newline
21. न्य॑धत्ता धत्त॒ नि न्य॑धत्त । \newline
22. अ॒ध॒त्त॒ पृ॒थि॒व्याम् पृ॑थि॒व्या म॑धत्ता धत्त पृथि॒व्याम् । \newline
23. पृ॒थि॒व्याम् तृती॑य॒म् तृती॑यम् पृथि॒व्याम् पृ॑थि॒व्याम् तृती॑यम् । \newline
24. तृती॑य म॒न्तरि॑क्षे॒ ऽन्तरि॑क्षे॒ तृती॑य॒म् तृती॑य म॒न्तरि॑क्षे । \newline
25. अ॒न्तरि॑क्षे॒ तृती॑य॒म् तृती॑य म॒न्तरि॑क्षे॒ ऽन्तरि॑क्षे॒ तृती॑यम् । \newline
26. तृती॑यम् दि॒वि दि॒वि तृती॑य॒म् तृती॑यम् दि॒वि । \newline
27. दि॒वि तृती॑य॒म् तृती॑यम् दि॒वि दि॒वि तृती॑यम् । \newline
28. तृती॑य मभिपर्याव॒र्ता द॑भिपर्याव॒र्तात् तृती॑य॒म् तृती॑य मभिपर्याव॒र्तात् । \newline
29. अ॒भि॒प॒र्या॒व॒र्ता द्धि ह्य॑भिपर्याव॒र्ता द॑भिपर्याव॒र्ता द्धि । \newline
30. अ॒भि॒प॒र्या॒व॒र्तादित्य॑भि - प॒र्या॒व॒र्तात् । \newline
31. ह्यबि॑भे॒ दबि॑भे॒ द्धि ह्यबि॑भेत् । \newline
32. अबि॑भे॒द् यद् यदबि॑भे॒ दबि॑भे॒द् यत् । \newline
33. यत् पृ॑थि॒व्याम् पृ॑थि॒व्यां ॅयद् यत् पृ॑थि॒व्याम् । \newline
34. पृ॒थि॒व्याम् तृती॑य॒म् तृती॑यम् पृथि॒व्याम् पृ॑थि॒व्याम् तृती॑यम् । \newline
35. तृती॑य॒ मासी॒ दासी॒त् तृती॑य॒म् तृती॑य॒ मासी᳚त् । \newline
36. आसी॒त् तेन॒ तेनासी॒ दासी॒त् तेन॑ । \newline
37. तेने न्द्र॒ इन्द्र॒ स्तेन॒ तेने न्द्रः॑ । \newline
38. इन्द्रो॒ वज्रं॒ ॅवज्र॒ मिन्द्र॒ इन्द्रो॒ वज्र᳚म् । \newline
39. वज्र॒ मुदुद् वज्रं॒ ॅवज्र॒ मुत् । \newline
40. उद॑यच्छ दयच्छ॒ दुदु द॑यच्छत् । \newline
41. अ॒य॒च्छ॒द् विष्ण्व॑नुस्थितो॒ विष्ण्व॑नुस्थितो ऽयच्छ दयच्छ॒द् विष्ण्व॑नुस्थितः । \newline
42. विष्ण्व॑नुस्थितः॒ स स विष्ण्व॑नुस्थितो॒ विष्ण्व॑नुस्थितः॒ सः । \newline
43. विष्ण्व॑नुस्थित॒ इति॒ विष्णु॑ - अ॒नु॒स्थि॒तः॒ । \newline
44. सो᳚ ऽब्रवी दब्रवी॒थ् स सो᳚ ऽब्रवीत् । \newline
45. अ॒ब्र॒वी॒न् मा मा ऽब्र॑वी दब्रवी॒न् मा । \newline
46. मा मे॑ मे॒ मा मा मे᳚ । \newline
47. मे॒ प्र प्र मे॑ मे॒ प्र । \newline
48. प्र हार्॑. हा॒र् प्र प्र हाः᳚ । \newline
49. हा॒ रस्त्यस्ति॑ हार्. हा॒ रस्ति॑ । \newline
50. अस्ति॒ वै वा अस्त्यस्ति॒ वै । \newline
51. वा इ॒द मि॒दं ॅवै वा इ॒दम् । \newline
52. इ॒दम् मयि॒ मयी॒द मि॒दम् मयि॑ । \newline

\textbf{Ghana Paata } \newline

1. अ॒न्या दे॒वता॑ दे॒वता॒ ऽन्या ऽन्या दे॒वता॑ ऽऽसी दासीद् दे॒वता॒ ऽन्या ऽन्या दे॒वता॑ ऽऽसीत् । \newline
2. दे॒वता॑ ऽऽसी दासीद् दे॒वता॑ दे॒वता॑ ऽऽसी॒थ् स स आ॑सीद् दे॒वता॑ दे॒वता॑ ऽऽसी॒थ् सः । \newline
3. आ॒सी॒थ् स स आ॑सी दासी॒थ् सो᳚ ऽब्रवी दब्रवी॒थ् स आ॑सी दासी॒थ् सो᳚ ऽब्रवीत् । \newline
4. सो᳚ ऽब्रवी दब्रवी॒थ् स सो᳚ ऽब्रवी॒द् विष्णो॒ विष्णो᳚ ऽब्रवी॒थ् स सो᳚ ऽब्रवी॒द् विष्णो᳚ । \newline
5. अ॒ब्र॒वी॒द् विष्णो॒ विष्णो᳚ ऽब्रवी दब्रवी॒द् विष्ण॒वा विष्णो᳚ऽब्रवी दब्रवी॒द् विष्ण॒वा । \newline
6. विष्ण॒वा विष्णो॒ विष्ण॒ वेही॒ह्या विष्णो॒ विष्ण॒वेहि॑ । \newline
7. एही॒ह्येही॒द मि॒द मि॒ह्येही॒दम् । \newline
8. इ॒ही॒ द मि॒द मि॑ही ही॒द मेद मि॑ही ही॒द मा । \newline
9. इ॒द मेद मि॒द मा ह॑रिष्यावो हरिष्याव॒ एद मि॒द मा ह॑रिष्यावः । \newline
10. आ ह॑रिष्यावो हरिष्याव॒ आ ह॑रिष्यावो॒ येन॒ येन॑ हरिष्याव॒ आ ह॑रिष्यावो॒ येन॑ । \newline
11. ह॒रि॒ष्या॒वो॒ येन॒ येन॑ हरिष्यावो हरिष्यावो॒ येना॒य म॒यं ॅयेन॑ हरिष्यावो हरिष्यावो॒ येना॒यम् । \newline
12. येना॒य म॒यं ॅयेन॒ येना॒य मि॒द मि॒द म॒यं ॅयेन॒ येना॒य मि॒दम् । \newline
13. अ॒य मि॒द मि॒द म॒य म॒य मि॒द मितीती॒द म॒य म॒य मि॒द मिति॑ । \newline
14. इ॒द मितीती॒द मि॒द मिति॒ स स इती॒द मि॒द मिति॒ सः । \newline
15. इति॒ स स इतीति॒ स विष्णु॒र् विष्णुः॒ स इतीति॒ स विष्णुः॑ । \newline
16. स विष्णु॒र् विष्णुः॒ स स विष्णु॑ स्त्रे॒धा त्रे॒धा विष्णुः॒ स स विष्णु॑ स्त्रे॒धा । \newline
17. विष्णु॑ स्त्रे॒धा त्रे॒धा विष्णु॒र् विष्णु॑ स्त्रे॒धा ऽऽत्मान॑ मा॒त्मान॑म् त्रे॒धा विष्णु॒र् विष्णु॑ स्त्रे॒धा ऽऽत्मान᳚म् । \newline
18. त्रे॒धा ऽऽत्मान॑ मा॒त्मान॑म् त्रे॒धा त्रे॒धा ऽऽत्मानं॒ ॅवि व्या᳚त्मान॑म् त्रे॒धा त्रे॒धा ऽऽत्मानं॒ ॅवि । \newline
19. आ॒त्मानं॒ ॅवि व्या᳚त्मान॑ मा॒त्मानं॒ ॅवि नि नि व्या᳚त्मान॑ मा॒त्मानं॒ ॅवि नि । \newline
20. वि नि नि वि वि न्य॑धत्ता धत्त॒ नि वि वि न्य॑धत्त । \newline
21. न्य॑धत्ताधत्त॒ नि न्य॑धत्त पृथि॒व्याम् पृ॑थि॒व्या म॑धत्त॒ नि न्य॑धत्त पृथि॒व्याम् । \newline
22. अ॒ध॒त्त॒ पृ॒थि॒व्याम् पृ॑थि॒व्या म॑धत्ताधत्त पृथि॒व्याम् तृती॑य॒म् तृती॑यम् पृथि॒व्या म॑धत्ताधत्त पृथि॒व्याम् तृती॑यम् । \newline
23. पृ॒थि॒व्याम् तृती॑य॒म् तृती॑यम् पृथि॒व्याम् पृ॑थि॒व्याम् तृती॑य म॒न्तरि॑क्षे॒ ऽन्तरि॑क्षे॒ तृती॑यम् पृथि॒व्याम् पृ॑थि॒व्याम् तृती॑य म॒न्तरि॑क्षे । \newline
24. तृती॑य म॒न्तरि॑क्षे॒ ऽन्तरि॑क्षे॒ तृती॑य॒म् तृती॑य म॒न्तरि॑क्षे॒ तृती॑य॒म् तृती॑य म॒न्तरि॑क्षे॒ तृती॑य॒म् तृती॑य म॒न्तरि॑क्षे॒ तृती॑यम् । \newline
25. अ॒न्तरि॑क्षे॒ तृती॑य॒म् तृती॑य म॒न्तरि॑क्षे॒ ऽन्तरि॑क्षे॒ तृती॑यम् दि॒वि दि॒वि तृती॑य म॒न्तरि॑क्षे॒ ऽन्तरि॑क्षे॒ तृती॑यम् दि॒वि । \newline
26. तृती॑यम् दि॒वि दि॒वि तृती॑य॒म् तृती॑यम् दि॒वि तृती॑य॒म् तृती॑यम् दि॒वि तृती॑य॒म् तृती॑यम् दि॒वि तृती॑यम् । \newline
27. दि॒वि तृती॑य॒म् तृती॑यम् दि॒वि दि॒वि तृती॑य मभिपर्याव॒र्ता द॑भिपर्याव॒र्तात् तृती॑यम् दि॒वि दि॒वि तृती॑य मभिपर्याव॒र्तात् । \newline
28. तृती॑य मभिपर्याव॒र्ता द॑भिपर्याव॒र्तात् तृती॑य॒म् तृती॑य मभिपर्याव॒र्ताद्धि ह्य॑भिपर्याव॒र्तात् तृती॑य॒म् तृती॑य मभिपर्याव॒र्ताद्धि । \newline
29. अ॒भि॒प॒र्या॒व॒र्ता द्धि ह्य॑भिपर्याव॒र्ता द॑भिपर्याव॒र्ता द्ध्यबि॑भे॒ दबि॑भे॒ द्ध्य॑भिपर्याव॒र्ता द॑भिपर्याव॒र्ता द्ध्यबि॑भेत् । \newline
30. अ॒भि॒प॒र्या॒व॒र्तादित्य॑भि - प॒र्या॒व॒र्तात् । \newline
31. ह्यबि॑भे॒ दबि॑भे॒द्धि ह्यबि॑भे॒द् यद् यदबि॑भे॒द्धि ह्यबि॑भे॒द् यत् । \newline
32. अबि॑भे॒द् यद् यदबि॑भे॒ दबि॑भे॒द् यत् पृ॑थि॒व्याम् पृ॑थि॒व्यां ॅयदबि॑भे॒ दबि॑भे॒द् यत् पृ॑थि॒व्याम् । \newline
33. यत् पृ॑थि॒व्याम् पृ॑थि॒व्यां ॅयद् यत् पृ॑थि॒व्याम् तृती॑य॒म् तृती॑यम् पृथि॒व्यां ॅयद् यत् पृ॑थि॒व्याम् तृती॑यम् । \newline
34. पृ॒थि॒व्याम् तृती॑य॒म् तृती॑यम् पृथि॒व्याम् पृ॑थि॒व्याम् तृती॑य॒ मासी॒ दासी॒त् तृती॑यम् पृथि॒व्याम् पृ॑थि॒व्याम् तृती॑य॒ मासी᳚त् । \newline
35. तृती॑य॒ मासी॒ दासी॒त् तृती॑य॒म् तृती॑य॒ मासी॒त् तेन॒ तेनासी॒त् तृती॑य॒म् तृती॑य॒ मासी॒त् तेन॑ । \newline
36. आसी॒त् तेन॒ तेनासी॒ दासी॒त् तेने न्द्र॒ इन्द्र॒ स्तेनासी॒ दासी॒त् तेने न्द्रः॑ । \newline
37. तेने न्द्र॒ इन्द्र॒ स्तेन॒ तेने न्द्रो॒ वज्रं॒ ॅवज्र॒ मिन्द्र॒ स्तेन॒ तेने न्द्रो॒ वज्र᳚म् । \newline
38. इन्द्रो॒ वज्रं॒ ॅवज्र॒ मिन्द्र॒ इन्द्रो॒ वज्र॒ मुदुद् वज्र॒ मिन्द्र॒ इन्द्रो॒ वज्र॒ मुत् । \newline
39. वज्र॒ मुदुद् वज्रं॒ ॅवज्र॒ मुद॑यच्छ दयच्छ॒ दुद् वज्रं॒ ॅवज्र॒ मुद॑यच्छत् । \newline
40. उद॑यच्छ दयच्छ॒ दुदु द॑यच्छ॒द् विष्ण्व॑नुस्थितो॒ विष्ण्व॑नुस्थितो ऽयच्छ॒दु दुद॑यच्छ॒द् विष्ण्व॑नुस्थितः । \newline
41. अ॒य॒च्छ॒द् विष्ण्व॑नुस्थितो॒ विष्ण्व॑नुस्थितो ऽयच्छ दयच्छ॒द् विष्ण्व॑नुस्थितः॒ स स विष्ण्व॑नुस्थितो ऽयच्छ दयच्छ॒द् विष्ण्व॑नुस्थितः॒ सः । \newline
42. विष्ण्व॑नुस्थितः॒ स स विष्ण्व॑नुस्थितो॒ विष्ण्व॑नुस्थितः॒ सो᳚ ऽब्रवी दब्रवी॒थ् स विष्ण्व॑नुस्थितो॒ विष्ण्व॑नुस्थितः॒ सो᳚ ऽब्रवीत् । \newline
43. विष्ण्व॑नुस्थित॒ इति॒ विष्णु॑ - अ॒नु॒स्थि॒तः॒ । \newline
44. सो᳚ ऽब्रवी दब्रवी॒थ् स सो᳚ ऽब्रवी॒न् मा मा ऽब्र॑वी॒थ् स सो᳚ ऽब्रवी॒न् मा । \newline
45. अ॒ब्र॒वी॒न् मा मा ऽब्र॑वी दब्रवी॒न् मा मे॑ मे॒ मा ऽब्र॑वी दब्रवी॒न् मा मे᳚ । \newline
46. मा मे॑ मे॒ मा मा मे॒ प्र प्र मे॒ मा मा मे॒ प्र । \newline
47. मे॒ प्र प्र मे॑ मे॒ प्र हार्॑. हा॒र् प्र मे॑ मे॒ प्र हाः᳚ । \newline
48. प्र हार्॑. हा॒र् प्र प्र हा॒ रस्त्यस्ति॑ हा॒र् प्र प्र हा॒ रस्ति॑ । \newline
49. हा॒ रस्त्यस्ति॑ हार्. हा॒ रस्ति॒ वै वा अस्ति॑ हार्. हा॒ रस्ति॒ वै । \newline
50. अस्ति॒ वै वा अस्त्यस्ति॒ वा इ॒द मि॒दं ॅवा अस्त्यस्ति॒ वा इ॒दम् । \newline
51. वा इ॒द मि॒दं ॅवै वा इ॒दम् मयि॒ मयी॒दं ॅवै वा इ॒दम् मयि॑ । \newline
52. इ॒दम् मयि॒ मयी॒द मि॒दम् मयि॑ वी॒र्यं॑ ॅवी॒र्य॑म् मयी॒द मि॒दम् मयि॑ वी॒र्य᳚म् । \newline
\pagebreak
\markright{ TS 2.4.12.4  \hfill https://www.vedavms.in \hfill}

\section{ TS 2.4.12.4 }

\textbf{TS 2.4.12.4 } \newline
\textbf{Samhita Paata} \newline

मयि॑ वी॒र्यं॑ तत् ते॒ प्रदा᳚स्या॒मीति॒ तद॑स्मै॒ प्राय॑च्छ॒त् तत् प्रत्य॑गृह्णा॒दधा॒ मेति॒ तद्-विष्ण॒वेऽति॒ प्राय॑च्छ॒त् तद्-विष्णुः॒ प्रत्य॑गृह्णा-द॒स्मास्विन्द्र॑ इन्द्रि॒यं द॑धा॒त्विति॒ यद॒न्तरि॑क्षे॒ तृती॑य॒मासी॒त् तेनेन्द्रो॒ वज्र॒मुद॑यच्छ॒द्-विष्ण्व॑नुस्थितः॒ सो᳚ऽब्रवी॒न्मा मे॒ प्रहा॒रस्ति॒ वा इ॒दं - [  ] \newline

\textbf{Pada Paata} \newline

मयि॑ । वी॒र्य᳚म् । तत् । ते॒ । प्रेति॑ । दा॒स्या॒मि॒ । इति॑ ।   तत् । अ॒स्मै॒ । प्रेति॑ । अ॒य॒च्छ॒त् । तत् । प्रतीति॑ । अ॒गृ॒ह्णा॒त् । अधाः᳚ । मा॒ । इति॑ । तत् । विष्ण॑वे । अति॑ । प्रेति॑ । अ॒य॒च्छ॒त् । तत् । विष्णुः॑ । प्रतीति॑ । अ॒गृ॒ह्णा॒त् । अ॒स्मासु॑ । इन्द्रः॑ । इ॒न्द्रि॒यम् । द॒धा॒तु॒ । इति॑ । यत् । अ॒न्तरि॑क्षे । तृती॑यम् । आसी᳚त् । तेन॑ । इन्द्रः॑ । वज्र᳚म् । उदिति॑ । अ॒य॒च्छ॒त् । विष्ण्व॑नुस्थित॒ इति॒ विष्णु॑ - अ॒नु॒स्थि॒तः॒ । सः । अ॒ब्र॒वी॒त् । मा । मे॒ । प्रेति॑ । हाः॒ । अस्ति॑ । वै । इ॒दम् ।  \newline


\textbf{Krama Paata} \newline

मयि॑ वी॒र्य᳚म् । वी॒र्य॑म् तत् । तत् ते᳚ । ते॒ प्र । प्र दा᳚स्यामि । दा॒स्या॒मीति॑ । इति॒ तत् । तद॑स्मै । अ॒स्मै॒ प्र । प्राय॑च्छत् । अ॒य॒च्छ॒त् तत् । तत् प्रति॑ । प्रत्य॑गृह्णात् । अ॒गृ॒ह्णा॒दधाः᳚ । अधा॒ मा । मेति॑ । इति॒ तत् । तद् विष्ण॑वे । विष्ण॒वेऽति॑ । अति॒ प्र । प्राय॑च्छत् । अ॒य॒च्छ॒त् तत् । तद् विष्णुः॑ । विष्णुः॒ प्रति॑ । प्रत्य॑गृह्णात् । अ॒गृ॒ह्णा॒द॒स्मासु॑ । अ॒स्मास्विन्द्रः॑ । इन्द्र॑ इन्द्रि॒यम् । इ॒न्द्रि॒यम् द॑धातु । द॒धा॒त्विति॑ । इति॒ यत् । यद॒न्तरि॑क्षे । अ॒न्तरि॑क्षे॒ तृती॑यम् । तृती॑य॒मासी᳚त् । आसी॒त् तेन॑ । तेनेन्द्रः॑ । इन्द्रो॒ वज्र᳚म् । वज्र॒मुत् । उद॑यच्छत् । अ॒य॒च्छ॒द् विष्ण्व॑नुस्थितः । विष्ण्व॑नुस्थितः॒ सः । विष्ण्व॑नुस्थित॒ इति॒ विष्णु॑ - अ॒नु॒स्थि॒तः॒ । सो᳚ऽब्रवीत् । अ॒ब्र॒वी॒न् मा । मा मे᳚ । मे॒ प्र । प्र हाः᳚ । हा॒रस्ति॑ । अस्ति॒ वै । वा इ॒दम् । इ॒दं मयि॑ \newline

\textbf{Jatai Paata} \newline

1. मयि॑ वी॒र्यं॑ ॅवी॒र्य॑म् मयि॒ मयि॑ वी॒र्य᳚म् । \newline
2. वी॒र्य॑म् तत् तद् वी॒र्यं॑ ॅवी॒र्य॑म् तत् । \newline
3. तत् ते॑ ते॒ तत् तत् ते᳚ । \newline
4. ते॒ प्र प्र ते॑ ते॒ प्र । \newline
5. प्र दा᳚स्यामि दास्यामि॒ प्र प्र दा᳚स्यामि । \newline
6. दा॒स्या॒मीतीति॑ दास्यामि दास्या॒मीति॑ । \newline
7. इति॒ तत् तदितीति॒ तत् । \newline
8. तद॑स्मा अस्मै॒ तत् तद॑स्मै । \newline
9. अ॒स्मै॒ प्र प्रास्मा॑ अस्मै॒ प्र । \newline
10. प्राय॑च्छ दयच्छ॒त् प्र प्राय॑च्छत् । \newline
11. अ॒य॒च्छ॒त् तत् तद॑यच्छ दयच्छ॒त् तत् । \newline
12. तत् प्रति॒ प्रति॒ तत् तत् प्रति॑ । \newline
13. प्रत्य॑गृह्णा दगृह्णा॒त् प्रति॒ प्रत्य॑गृह्णात् । \newline
14. अ॒गृ॒ह्णा॒ दधा॒ अधा॑ अगृह्णा दगृह्णा॒ दधाः᳚ । \newline
15. अधा॑ मा॒ मा ऽधा॒ अधा॑ मा । \newline
16. मेतीति॑ मा॒ मेति॑ । \newline
17. इति॒ तत् तदितीति॒ तत् । \newline
18. तद् विष्ण॑वे॒ विष्ण॑वे॒ तत् तद् विष्ण॑वे । \newline
19. विष्ण॒वे ऽत्यति॒ विष्ण॑वे॒ विष्ण॒वे ऽति॑ । \newline
20. अति॒ प्र प्रात्यति॒ प्र । \newline
21. प्राय॑च्छ दयच्छ॒त् प्र प्राय॑च्छत् । \newline
22. अ॒य॒च्छ॒त् तत् तद॑यच्छ दयच्छ॒त् तत् । \newline
23. तद् विष्णु॒र् विष्णु॒ स्तत् तद् विष्णुः॑ । \newline
24. विष्णुः॒ प्रति॒ प्रति॒ विष्णु॒र् विष्णुः॒ प्रति॑ । \newline
25. प्रत्य॑गृह्णा दगृह्णा॒त् प्रति॒ प्रत्य॑गृह्णात् । \newline
26. अ॒गृ॒ह्णा॒ द॒स्मा स्व॒स्मा स्व॑गृह्णा दगृह्णा द॒स्मासु॑ । \newline
27. अ॒स्मा स्विन्द्र॒ इन्द्रो॒ ऽस्मास्व॒स्मा स्विन्द्रः॑ । \newline
28. इन्द्र॑ इन्द्रि॒य मि॑न्द्रि॒य मिन्द्र॒ इन्द्र॑ इन्द्रि॒यम् । \newline
29. इ॒न्द्रि॒यम् द॑धातु दधा त्विन्द्रि॒य मि॑न्द्रि॒यम् द॑धातु । \newline
30. द॒धा॒ त्वितीति॑ दधातु दधा॒ त्विति॑ । \newline
31. इति॒ यद् यदितीति॒ यत् । \newline
32. यद॒न्तरि॑क्षे॒ ऽन्तरि॑क्षे॒ यद् यद॒न्तरि॑क्षे । \newline
33. अ॒न्तरि॑क्षे॒ तृती॑य॒म् तृती॑य म॒न्तरि॑क्षे॒ ऽन्तरि॑क्षे॒ तृती॑यम् । \newline
34. तृती॑य॒ मासी॒ दासी॒त् तृती॑य॒म् तृती॑य॒ मासी᳚त् । \newline
35. आसी॒त् तेन॒ तेनासी॒ दासी॒त् तेन॑ । \newline
36. तेने न्द्र॒ इन्द्र॒ स्तेन॒ तेने न्द्रः॑ । \newline
37. इन्द्रो॒ वज्रं॒ ॅवज्र॒ मिन्द्र॒ इन्द्रो॒ वज्र᳚म् । \newline
38. वज्र॒ मुदुद् वज्रं॒ ॅवज्र॒ मुत् । \newline
39. उद॑यच्छ दयच्छ॒ दुदु द॑यच्छत् । \newline
40. अ॒य॒च्छ॒द् विष्ण्व॑नुस्थितो॒ विष्ण्व॑नुस्थितो ऽयच्छ दयच्छ॒द् विष्ण्व॑नुस्थितः । \newline
41. विष्ण्व॑नुस्थितः॒ स स विष्ण्व॑नुस्थितो॒ विष्ण्व॑नुस्थितः॒ सः । \newline
42. विष्ण्व॑नुस्थित॒ इति॒ विष्णु॑ - अ॒नु॒स्थि॒तः॒ । \newline
43. सो᳚ ऽब्रवी दब्रवी॒थ् स सो᳚ ऽब्रवीत् । \newline
44. अ॒ब्र॒वी॒न् मा मा ऽब्र॑वी दब्रवी॒न् मा । \newline
45. मा मे॑ मे॒ मा मा मे᳚ । \newline
46. मे॒ प्र प्र मे॑ मे॒ प्र । \newline
47. प्र हार्॑. हा॒र् प्र प्र हाः᳚ । \newline
48. हा॒ रस्त्यस्ति॑ हार्. हा॒ रस्ति॑ । \newline
49. अस्ति॒ वै वा अस्त्यस्ति॒ वै । \newline
50. वा इ॒द मि॒दं ॅवै वा इ॒दम् । \newline
51. इ॒दम् मयि॒ मयी॒द मि॒दम् मयि॑ । \newline

\textbf{Ghana Paata } \newline

1. मयि॑ वी॒र्यं॑ ॅवी॒र्य॑म् मयि॒ मयि॑ वी॒र्य॑म् तत् तद् वी॒र्य॑म् मयि॒ मयि॑ वी॒र्य॑म् तत् । \newline
2. वी॒र्य॑म् तत् तद् वी॒र्यं॑ ॅवी॒र्य॑म् तत् ते॑ ते॒ तद् वी॒र्यं॑ ॅवी॒र्य॑म् तत् ते᳚ । \newline
3. तत् ते॑ ते॒ तत् तत् ते॒ प्र प्र ते॒ तत् तत् ते॒ प्र । \newline
4. ते॒ प्र प्र ते॑ ते॒ प्र दा᳚स्यामि दास्यामि॒ प्र ते॑ ते॒ प्र दा᳚स्यामि । \newline
5. प्र दा᳚स्यामि दास्यामि॒ प्र प्र दा᳚स्या॒मीतीति॑ दास्यामि॒ प्र प्र दा᳚स्या॒मीति॑ । \newline
6. दा॒स्या॒मीतीति॑ दास्यामि दास्या॒मीति॒ तत् तदिति॑ दास्यामि दास्या॒मीति॒ तत् । \newline
7. इति॒ तत् तदितीति॒ तद॑स्मा अस्मै॒ तदितीति॒ तद॑स्मै । \newline
8. तद॑स्मा अस्मै॒ तत् तद॑स्मै॒ प्र प्रास्मै॒ तत् तद॑स्मै॒ प्र । \newline
9. अ॒स्मै॒ प्र प्रास्मा॑ अस्मै॒ प्राय॑च्छ दयच्छ॒त् प्रास्मा॑ अस्मै॒ प्राय॑च्छत् । \newline
10. प्राय॑च्छ दयच्छ॒त् प्र प्राय॑च्छ॒त् तत् तद॑यच्छ॒त् प्र प्राय॑च्छ॒त् तत् । \newline
11. अ॒य॒च्छ॒त् तत् तद॑यच्छ दयच्छ॒त् तत् प्रति॒ प्रति॒ तद॑यच्छ दयच्छ॒त् तत् प्रति॑ । \newline
12. तत् प्रति॒ प्रति॒ तत् तत् प्रत्य॑गृह्णा दगृह्णा॒त् प्रति॒ तत् तत् प्रत्य॑गृह्णात् । \newline
13. प्रत्य॑गृह्णा दगृह्णा॒त् प्रति॒ प्रत्य॑गृह्णा॒ दधा॒ अधा॑ अगृह्णा॒त् प्रति॒ प्रत्य॑गृह्णा॒ दधाः᳚ । \newline
14. अ॒गृ॒ह्णा॒ दधा॒ अधा॑ अगृह्णा दगृह्णा॒ दधा॑ मा॒ मा ऽधा॑ अगृह्णा दगृह्णा॒ दधा॑ मा । \newline
15. अधा॑ मा॒ मा ऽधा॒ अधा॒ मेतीति॒ मा ऽधा॒ अधा॒ मेति॑ । \newline
16. मेतीति॑ मा॒ मेति॒ तत् तदिति॑ मा॒ मेति॒ तत् । \newline
17. इति॒ तत् तदितीति॒ तद् विष्ण॑वे॒ विष्ण॑वे॒ तदितीति॒ तद् विष्ण॑वे । \newline
18. तद् विष्ण॑वे॒ विष्ण॑वे॒ तत् तद् विष्ण॒वे ऽत्यति॒ विष्ण॑वे॒ तत् तद् विष्ण॒वे ऽति॑ । \newline
19. विष्ण॒वे ऽत्यति॒ विष्ण॑वे॒ विष्ण॒वे ऽति॒ प्र प्राति॒ विष्ण॑वे॒ विष्ण॒वे ऽति॒ प्र । \newline
20. अति॒ प्र प्रात्यति॒ प्राय॑च्छ दयच्छ॒त् प्रात्यति॒ प्राय॑च्छत् । \newline
21. प्राय॑च्छ दयच्छ॒त् प्र प्राय॑च्छ॒त् तत् तद॑यच्छ॒त् प्र प्राय॑च्छ॒त् तत् । \newline
22. अ॒य॒च्छ॒त् तत् तद॑यच्छ दयच्छ॒त् तद् विष्णु॒र् विष्णु॒ स्तद॑यच्छ दयच्छ॒त् तद् विष्णुः॑ । \newline
23. तद् विष्णु॒र् विष्णु॒ स्तत् तद् विष्णुः॒ प्रति॒ प्रति॒ विष्णु॒ स्तत् तद् विष्णुः॒ प्रति॑ । \newline
24. विष्णुः॒ प्रति॒ प्रति॒ विष्णु॒र् विष्णुः॒ प्रत्य॑गृह्णा दगृह्णा॒त् प्रति॒ विष्णु॒र् विष्णुः॒ प्रत्य॑गृह्णात् । \newline
25. प्रत्य॑गृह्णा दगृह्णा॒त् प्रति॒ प्रत्य॑गृह्णा द॒स्मा स्व॒स्मा स्व॑गृह्णा॒त् प्रति॒ प्रत्य॑गृह्णा द॒स्मासु॑ । \newline
26. अ॒गृ॒ह्णा॒ द॒स्मा स्व॒स्मा स्व॑गृह्णा दगृह्णा द॒स्मा स्विन्द्र॒ इन्द्रो॒ ऽस्मास्व॑गृह्णा दगृह्णा द॒स्मा स्विन्द्रः॑ । \newline
27. अ॒स्मा स्विन्द्र॒ इन्द्रो॒ ऽस्मास्व॒स्मा स्विन्द्र॑ इन्द्रि॒य मि॑न्द्रि॒य मिन्द्रो॒ ऽस्मास्व॒स्मा स्विन्द्र॑ इन्द्रि॒यम् । \newline
28. इन्द्र॑ इन्द्रि॒य मि॑न्द्रि॒य मिन्द्र॒ इन्द्र॑ इन्द्रि॒यम् द॑धातु दधा त्विन्द्रि॒य मिन्द्र॒ इन्द्र॑ इन्द्रि॒यम् द॑धातु । \newline
29. इ॒न्द्रि॒यम् द॑धातु दधा त्विन्द्रि॒य मि॑न्द्रि॒यम् द॑धा॒ त्वितीति॑ दधा त्विन्द्रि॒य मि॑न्द्रि॒यम् द॑धा॒ त्विति॑ । \newline
30. द॒धा॒ त्वितीति॑ दधातु दधा॒ त्विति॒ यद् यदिति॑ दधातु दधा॒ त्विति॒ यत् । \newline
31. इति॒ यद् यदितीति॒ यद॒न्तरि॑क्षे॒ ऽन्तरि॑क्षे॒ यदितीति॒ यद॒न्तरि॑क्षे । \newline
32. यद॒न्तरि॑क्षे॒ ऽन्तरि॑क्षे॒ यद् यद॒न्तरि॑क्षे॒ तृती॑य॒म् तृती॑य म॒न्तरि॑क्षे॒ यद् यद॒न्तरि॑क्षे॒ तृती॑यम् । \newline
33. अ॒न्तरि॑क्षे॒ तृती॑य॒म् तृती॑य म॒न्तरि॑क्षे॒ ऽन्तरि॑क्षे॒ तृती॑य॒ मासी॒ दासी॒त् तृती॑य म॒न्तरि॑क्षे॒ ऽन्तरि॑क्षे॒ तृती॑य॒ मासी᳚त् । \newline
34. तृती॑य॒ मासी॒ दासी॒त् तृती॑य॒म् तृती॑य॒ मासी॒त् तेन॒ तेनासी॒त् तृती॑य॒म् तृती॑य॒ मासी॒त् तेन॑ । \newline
35. आसी॒त् तेन॒ तेनासी॒ दासी॒त् तेने न्द्र॒ इन्द्र॒ स्तेनासी॒ दासी॒त् तेने न्द्रः॑ । \newline
36. तेने न्द्र॒ इन्द्र॒ स्तेन॒ तेने न्द्रो॒ वज्रं॒ ॅवज्र॒ मिन्द्र॒ स्तेन॒ तेने न्द्रो॒ वज्र᳚म् । \newline
37. इन्द्रो॒ वज्रं॒ ॅवज्र॒ मिन्द्र॒ इन्द्रो॒ वज्र॒ मुदुद् वज्र॒ मिन्द्र॒ इन्द्रो॒ वज्र॒ मुत् । \newline
38. वज्र॒ मुदुद् वज्रं॒ ॅवज्र॒ मुद॑यच्छ दयच्छ॒ दुद् वज्रं॒ ॅवज्र॒ मुद॑यच्छत् । \newline
39. उद॑यच्छ दयच्छ॒ दुदु द॑यच्छ॒द् विष्ण्व॑नुस्थितो॒ विष्ण्व॑नुस्थितो ऽयच्छ॒ दुदु द॑यच्छ॒द् विष्ण्व॑नुस्थितः । \newline
40. अ॒य॒च्छ॒द् विष्ण्व॑नुस्थितो॒ विष्ण्व॑नुस्थितो ऽयच्छ दयच्छ॒द् विष्ण्व॑नुस्थितः॒ स स विष्ण्व॑नुस्थितो ऽयच्छ दयच्छ॒द् विष्ण्व॑नुस्थितः॒ सः । \newline
41. विष्ण्व॑नुस्थितः॒ स स विष्ण्व॑नुस्थितो॒ विष्ण्व॑नुस्थितः॒ सो᳚ ऽब्रवी दब्रवी॒थ् स विष्ण्व॑नुस्थितो॒ विष्ण्व॑नुस्थितः॒ सो᳚ ऽब्रवीत् । \newline
42. विष्ण्व॑नुस्थित॒ इति॒ विष्णु॑ - अ॒नु॒स्थि॒तः॒ । \newline
43. सो᳚ ऽब्रवी दब्रवी॒थ् स सो᳚ ऽब्रवी॒न् मा मा ऽब्र॑वी॒थ् स सो᳚ ऽब्रवी॒न् मा । \newline
44. अ॒ब्र॒वी॒न् मा मा ऽब्र॑वी दब्रवी॒न् मा मे॑ मे॒ मा ऽब्र॑वी दब्रवी॒न् मा मे᳚ । \newline
45. मा मे॑ मे॒ मा मा मे॒ प्र प्र मे॒ मा मा मे॒ प्र । \newline
46. मे॒ प्र प्र मे॑ मे॒ प्र हार्॑. हा॒र् प्र मे॑ मे॒ प्र हाः᳚ । \newline
47. प्र हार्॑. हा॒र् प्र प्र हा॒ रस्त्यस्ति॑ हा॒र् प्र प्र हा॒ रस्ति॑ । \newline
48. हा॒ रस्त्यस्ति॑ हार्. हा॒ रस्ति॒ वै वा अस्ति॑ हार्. हा॒ रस्ति॒ वै । \newline
49. अस्ति॒ वै वा अस्त्यस्ति॒ वा इ॒द मि॒दं ॅवा अस्त्यस्ति॒ वा इ॒दम् । \newline
50. वा इ॒द मि॒दं ॅवै वा इ॒दम् मयि॒ मयी॒दं ॅवै वा इ॒दम् मयि॑ । \newline
51. इ॒दम् मयि॒ मयी॒द मि॒दम् मयि॑ वी॒र्यं॑ ॅवी॒र्य॑म् मयी॒द मि॒दम् मयि॑ वी॒र्य᳚म् । \newline
\pagebreak
\markright{ TS 2.4.12.5  \hfill https://www.vedavms.in \hfill}

\section{ TS 2.4.12.5 }

\textbf{TS 2.4.12.5 } \newline
\textbf{Samhita Paata} \newline

मयि॑ वी॒र्यं॑ तत् ते॒ प्र दा᳚स्या॒मीति॒ तद॑स्मै॒ प्राय॑च्छ॒त् तत् प्रत्य॑गृह्णा॒द् द्विर्मा॑ऽधा॒ इति॒ तद्-विष्ण॒वेऽति॒ प्राय॑च्छ॒त् तद्-विष्णुः॒ प्रत्य॑गृह्णाद॒स्मास्विन्द्र॑ इन्द्रि॒यं द॑धा॒त्विति॒ यद्दि॒वि तृती॑य॒मासी॒त् तेनेन्द्रो॒ वज्र॒मुद॑यच्छ॒द्-विष्ण्व॑नुस्थितः॒ सो᳚ऽब्रवी॒न्मा मे॒ प्रहा॒र्येना॒ह - [  ] \newline

\textbf{Pada Paata} \newline

मयि॑ । वी॒र्य᳚म् । तत् । ते॒ । प्रेति॑ । दा॒स्या॒मि॒ । इति॑ । तत् । अ॒स्मै॒ । प्रेति॑ । अ॒य॒च्छ॒त् । तत् । प्रतीति॑ । अ॒गृ॒ह्णा॒त् । द्विः । मा॒ । अ॒धाः॒ । इति॑ । तत् । विष्ण॑वे । अति॑ । प्रेति॑ । अ॒य॒च्छ॒त् । तत् । विष्णुः॑ । प्रतीति॑ । अ॒गृ॒ह्णा॒त् । अ॒स्मासु॑ । इन्द्रः॑ । इ॒न्द्रि॒यम् । द॒धा॒तु॒ । इति॑ । यत् । दि॒वि । तृती॑यम् । आसी᳚त् । तेन॑ । इन्द्रः॑ । वज्र᳚म् । उदिति॑ । अ॒य॒च्छ॒त् । विष्ण्व॑नुस्थित॒ इति॒ विष्णु॑ - अ॒नु॒स्थि॒तः॒ । सः । अ॒ब्र॒वी॒त् । मा । मे॒ । प्रेति॑ । हाः॒ । येन॑ । अ॒हम् ।  \newline


\textbf{Krama Paata} \newline

मयि॑ वी॒र्य᳚म् । वी॒र्य॑म् तत् । तत् ते᳚ । ते॒ प्र । प्र दा᳚स्यामि । दा॒स्या॒मीति॑ । इति॒ तत् । तद॑स्मै । अ॒स्मै॒ प्र । प्राय॑च्छत् । अ॒य॒च्छ॒त् तत् । तत् प्रति॑ । प्रत्य॑गृह्णात् । अ॒गृ॒ह्णा॒त् द्विः । द्विर् मा᳚ । मा॒ऽधाः॒ । अ॒धा॒ इति॑ । इति॒ तत् । तद् विष्ण॑वे । विष्ण॒वेऽति॑ । अति॒ प्र । प्राय॑च्छत् । अ॒य॒च्छ॒त् तत् । तद् विष्णुः॑ । विष्णुः॒ प्रति॑ । प्रत्य॑गृह्णात् । अ॒गृ॒ह्णा॒द॒स्मासु॑ । अ॒स्मास्विन्द्रः॑ । इन्द्र॑ इन्द्रि॒यम् । इ॒न्द्रि॒यम् द॑धातु । द॒धा॒त्विति॑ । इति॒ यत् । यद् दि॒वि । दि॒वि तृती॑यम् । तृती॑य॒मासी᳚त् । आसी॒त् तेन॑ । तेनेन्द्रः॑ । इन्द्रो॒ वज्र᳚म् । वज्र॒मुत् । उद॑यच्छत् । अ॒य॒च्छ॒द् विष्ण्व॑नुस्थितः । विष्ण्व॑नुस्थितः॒ सः । विष्ण्व॑नुस्थित॒ इति॒ विष्णु॑ - अ॒नु॒स्थि॒तः॒ । सो᳚ऽब्रवीत् । अ॒ब्र॒वी॒न् मा । मा मे᳚ । मे॒ प्र । प्र हाः᳚ । हा॒र्येन॑ । येना॒हम् । अ॒हमि॒दम् । \newline

\textbf{Jatai Paata} \newline

1. मयि॑ वी॒र्यं॑ ॅवी॒र्य॑म् मयि॒ मयि॑ वी॒र्य᳚म् । \newline
2. वी॒र्य॑म् तत् तद् वी॒र्यं॑ ॅवी॒र्य॑म् तत् । \newline
3. तत् ते॑ ते॒ तत् तत् ते᳚ । \newline
4. ते॒ प्र प्र ते॑ ते॒ प्र । \newline
5. प्र दा᳚स्यामि दास्यामि॒ प्र प्र दा᳚स्यामि । \newline
6. दा॒स्या॒मीतीति॑ दास्यामि दास्या॒मीति॑ । \newline
7. इति॒ तत् तदितीति॒ तत् । \newline
8. तद॑स्मा अस्मै॒ तत् तद॑स्मै । \newline
9. अ॒स्मै॒ प्र प्रास्मा॑ अस्मै॒ प्र । \newline
10. प्राय॑च्छ दयच्छ॒त् प्र प्राय॑च्छत् । \newline
11. अ॒य॒च्छ॒त् तत् तद॑यच्छ दयच्छ॒त् तत् । \newline
12. तत् प्रति॒ प्रति॒ तत् तत् प्रति॑ । \newline
13. प्रत्य॑गृह्णा दगृह्णा॒त् प्रति॒ प्रत्य॑गृह्णात् । \newline
14. अ॒गृ॒ह्णा॒द् द्विर् द्विर॑गृह्णा दगृह्णा॒द् द्विः । \newline
15. द्विर् मा॑ मा॒ द्विर् द्विर् मा᳚ । \newline
16. मा॒ ऽधा॒ अ॒धा॒ मा॒ मा॒ ऽधाः॒ । \newline
17. अ॒धा॒ इती त्य॑धा अधा॒ इति॑ । \newline
18. इति॒ तत् तदितीति॒ तत् । \newline
19. तद् विष्ण॑वे॒ विष्ण॑वे॒ तत् तद् विष्ण॑वे । \newline
20. विष्ण॒वे ऽत्यति॒ विष्ण॑वे॒ विष्ण॒वे ऽति॑ । \newline
21. अति॒ प्र प्रात्यति॒ प्र । \newline
22. प्राय॑च्छ दयच्छ॒त् प्र प्राय॑च्छत् । \newline
23. अ॒य॒च्छ॒त् तत् तद॑यच्छ दयच्छ॒त् तत् । \newline
24. तद् विष्णु॒र् विष्णु॒ स्तत् तद् विष्णुः॑ । \newline
25. विष्णुः॒ प्रति॒ प्रति॒ विष्णु॒र् विष्णुः॒ प्रति॑ । \newline
26. प्रत्य॑गृह्णा दगृह्णा॒त् प्रति॒ प्रत्य॑गृह्णात् । \newline
27. अ॒गृ॒ह्णा॒ द॒स्मा स्व॒स्मा स्व॑गृह्णा दगृह्णा द॒स्मासु॑ । \newline
28. अ॒स्मा स्विन्द्र॒ इन्द्रो॒ ऽस्मा स्व॒स्मा स्विन्द्रः॑ । \newline
29. इन्द्र॑ इन्द्रि॒य मि॑न्द्रि॒य मिन्द्र॒ इन्द्र॑ इन्द्रि॒यम् । \newline
30. इ॒न्द्रि॒यम् द॑धातु दधा त्विन्द्रि॒य मि॑न्द्रि॒यम् द॑धातु । \newline
31. द॒धा॒ त्वितीति॑ दधातु दधा॒ त्विति॑ । \newline
32. इति॒ यद् यदितीति॒ यत् । \newline
33. यद् दि॒वि दि॒वि यद् यद् दि॒वि । \newline
34. दि॒वि तृती॑य॒म् तृती॑यम् दि॒वि दि॒वि तृती॑यम् । \newline
35. तृती॑य॒ मासी॒ दासी॒त् तृती॑य॒म् तृती॑य॒ मासी᳚त् । \newline
36. आसी॒त् तेन॒ तेनासी॒ दासी॒त् तेन॑ । \newline
37. तेने न्द्र॒ इन्द्र॒ स्तेन॒ तेने न्द्रः॑ । \newline
38. इन्द्रो॒ वज्रं॒ ॅवज्र॒ मिन्द्र॒ इन्द्रो॒ वज्र᳚म् । \newline
39. वज्र॒ मुदुद् वज्रं॒ ॅवज्र॒ मुत् । \newline
40. उद॑यच्छ दयच्छ॒ दुदु द॑यच्छत् । \newline
41. अ॒य॒च्छ॒द् विष्ण्व॑नुस्थितो॒ विष्ण्व॑नुस्थितो ऽयच्छ दयच्छ॒द् विष्ण्व॑नुस्थितः । \newline
42. विष्ण्व॑नुस्थितः॒ स स विष्ण्व॑नुस्थितो॒ विष्ण्व॑नुस्थितः॒ सः । \newline
43. विष्ण्व॑नुस्थित॒ इति॒ विष्णु॑ - अ॒नु॒स्थि॒तः॒ । \newline
44. सो᳚ ऽब्रवी दब्रवी॒थ् स सो᳚ ऽब्रवीत् । \newline
45. अ॒ब्र॒वी॒न् मा मा ऽब्र॑वी दब्रवी॒न् मा । \newline
46. मा मे॑ मे॒ मा मा मे᳚ । \newline
47. मे॒ प्र प्र मे॑ मे॒ प्र । \newline
48. प्र हार्॑. हा॒र् प्र प्र हाः᳚ । \newline
49. हा॒र् येन॒ येन॑ हार्. हा॒र् येन॑ । \newline
50. येना॒ह म॒हं ॅयेन॒ येना॒हम् । \newline
51. अ॒ह मि॒द मि॒द म॒ह म॒ह मि॒दम् । \newline

\textbf{Ghana Paata } \newline

1. मयि॑ वी॒र्यं॑ ॅवी॒र्य॑म् मयि॒ मयि॑ वी॒र्य॑म् तत् तद् वी॒र्य॑म् मयि॒ मयि॑ वी॒र्य॑म् तत् । \newline
2. वी॒र्य॑म् तत् तद् वी॒र्यं॑ ॅवी॒र्य॑म् तत् ते॑ ते॒ तद् वी॒र्यं॑ ॅवी॒र्य॑म् तत् ते᳚ । \newline
3. तत् ते॑ ते॒ तत् तत् ते॒ प्र प्र ते॒ तत् तत् ते॒ प्र । \newline
4. ते॒ प्र प्र ते॑ ते॒ प्र दा᳚स्यामि दास्यामि॒ प्र ते॑ ते॒ प्र दा᳚स्यामि । \newline
5. प्र दा᳚स्यामि दास्यामि॒ प्र प्र दा᳚स्या॒मीतीति॑ दास्यामि॒ प्र प्र दा᳚स्या॒मीति॑ । \newline
6. दा॒स्या॒मीतीति॑ दास्यामि दास्या॒मीति॒ तत् तदिति॑ दास्यामि दास्या॒मीति॒ तत् । \newline
7. इति॒ तत् तदितीति॒ तद॑स्मा अस्मै॒ तदितीति॒ तद॑स्मै । \newline
8. तद॑स्मा अस्मै॒ तत् तद॑स्मै॒ प्र प्रास्मै॒ तत् तद॑स्मै॒ प्र । \newline
9. अ॒स्मै॒ प्र प्रास्मा॑ अस्मै॒ प्राय॑च्छ दयच्छ॒त् प्रास्मा॑ अस्मै॒ प्राय॑च्छत् । \newline
10. प्राय॑च्छ दयच्छ॒त् प्र प्राय॑च्छ॒त् तत् तद॑यच्छ॒त् प्र प्राय॑च्छ॒त् तत् । \newline
11. अ॒य॒च्छ॒त् तत् तद॑यच्छ दयच्छ॒त् तत् प्रति॒ प्रति॒ तद॑यच्छ दयच्छ॒त् तत् प्रति॑ । \newline
12. तत् प्रति॒ प्रति॒ तत् तत् प्रत्य॑गृह्णा दगृह्णा॒त् प्रति॒ तत् तत् प्रत्य॑गृह्णात् । \newline
13. प्रत्य॑गृह्णा दगृह्णा॒त् प्रति॒ प्रत्य॑गृह्णा॒द् द्विर् द्विर॑गृह्णा॒त् प्रति॒ प्रत्य॑गृह्णा॒द् द्विः । \newline
14. अ॒गृ॒ह्णा॒द् द्विर् द्विर॑गृह्णा दगृह्णा॒द् द्विर् मा॑ मा॒ द्विर॑गृह्णा दगृह्णा॒द् द्विर् मा᳚ । \newline
15. द्विर् मा॑ मा॒ द्विर् द्विर् मा॑ ऽधा अधा मा॒ द्विर् द्विर् मा॑ ऽधाः । \newline
16. मा॒ ऽधा॒ अ॒धा॒ मा॒ मा॒ ऽधा॒ इती त्य॑धा मा मा ऽधा॒ इति॑ । \newline
17. अ॒धा॒ इतीत्य॑धा अधा॒ इति॒ तत् तदित्य॑धा अधा॒ इति॒ तत् । \newline
18. इति॒ तत् तदितीति॒ तद् विष्ण॑वे॒ विष्ण॑वे॒ तदितीति॒ तद् विष्ण॑वे । \newline
19. तद् विष्ण॑वे॒ विष्ण॑वे॒ तत् तद् विष्ण॒वे ऽत्यति॒ विष्ण॑वे॒ तत् तद् विष्ण॒वे ऽति॑ । \newline
20. विष्ण॒वे ऽत्यति॒ विष्ण॑वे॒ विष्ण॒वे ऽति॒ प्र प्राति॒ विष्ण॑वे॒ विष्ण॒वे ऽति॒ प्र । \newline
21. अति॒ प्र प्रात्यति॒ प्राय॑च्छ दयच्छ॒त् प्रात्यति॒ प्राय॑च्छत् । \newline
22. प्राय॑च्छ दयच्छ॒त् प्र प्राय॑च्छ॒त् तत् तद॑यच्छ॒त् प्र प्राय॑च्छ॒त् तत् । \newline
23. अ॒य॒च्छ॒त् तत् तद॑यच्छ दयच्छ॒त् तद् विष्णु॒र् विष्णु॒ स्तद॑यच्छ दयच्छ॒त् तद् विष्णुः॑ । \newline
24. तद् विष्णु॒र् विष्णु॒ स्तत् तद् विष्णुः॒ प्रति॒ प्रति॒ विष्णु॒ स्तत् तद् विष्णुः॒ प्रति॑ । \newline
25. विष्णुः॒ प्रति॒ प्रति॒ विष्णु॒र् विष्णुः॒ प्रत्य॑गृह्णा दगृह्णा॒त् प्रति॒ विष्णु॒र् विष्णुः॒ प्रत्य॑गृह्णात् । \newline
26. प्रत्य॑गृह्णा दगृह्णा॒त् प्रति॒ प्रत्य॑गृह्णा द॒स्मा स्व॒स्मा स्व॑गृह्णा॒त् प्रति॒ प्रत्य॑गृह्णा द॒स्मासु॑ । \newline
27. अ॒गृ॒ह्णा॒ द॒स्मा स्व॒स्मा स्व॑गृह्णा दगृह्णा द॒स्मा स्विन्द्र॒ इन्द्रो॒ ऽस्मास्व॑गृह्णा दगृह्णा द॒स्मा स्विन्द्रः॑ । \newline
28. अ॒स्मा स्विन्द्र॒ इन्द्रो॒ ऽस्मा स्व॒स्मा स्विन्द्र॑ इन्द्रि॒य मि॑न्द्रि॒य मिन्द्रो॒ ऽस्मा स्व॒स्मा स्विन्द्र॑ इन्द्रि॒यम् । \newline
29. इन्द्र॑ इन्द्रि॒य मि॑न्द्रि॒य मिन्द्र॒ इन्द्र॑ इन्द्रि॒यम् द॑धातु दधा त्विन्द्रि॒य मिन्द्र॒ इन्द्र॑ इन्द्रि॒यम् द॑धातु । \newline
30. इ॒न्द्रि॒यम् द॑धातु दधा त्विन्द्रि॒य मि॑न्द्रि॒यम् द॑धा॒ त्वितीति॑ दधा त्विन्द्रि॒य मि॑न्द्रि॒यम् द॑धा॒त्विति॑ । \newline
31. द॒धा॒ त्वितीति॑ दधातु दधा॒ त्विति॒ यद् यदिति॑ दधातु दधा॒ त्विति॒ यत् । \newline
32. इति॒ यद् यदितीति॒ यद् दि॒वि दि॒वि यदितीति॒ यद् दि॒वि । \newline
33. यद् दि॒वि दि॒वि यद् यद् दि॒वि तृती॑य॒म् तृती॑यम् दि॒वि यद् यद् दि॒वि तृती॑यम् । \newline
34. दि॒वि तृती॑य॒म् तृती॑यम् दि॒वि दि॒वि तृती॑य॒ मासी॒ दासी॒त् तृती॑यम् दि॒वि दि॒वि तृती॑य॒ मासी᳚त् । \newline
35. तृती॑य॒ मासी॒ दासी॒त् तृती॑य॒म् तृती॑य॒ मासी॒त् तेन॒ तेनासी॒त् तृती॑य॒म् तृती॑य॒ मासी॒त् तेन॑ । \newline
36. आसी॒त् तेन॒ तेनासी॒ दासी॒त् तेने न्द्र॒ इन्द्र॒ स्तेनासी॒ दासी॒त् तेने न्द्रः॑ । \newline
37. तेने न्द्र॒ इन्द्र॒ स्तेन॒ तेने न्द्रो॒ वज्रं॒ ॅवज्र॒ मिन्द्र॒ स्तेन॒ तेने न्द्रो॒ वज्र᳚म् । \newline
38. इन्द्रो॒ वज्रं॒ ॅवज्र॒ मिन्द्र॒ इन्द्रो॒ वज्र॒ मुदुद् वज्र॒ मिन्द्र॒ इन्द्रो॒ वज्र॒ मुत् । \newline
39. वज्र॒ मुदुद् वज्रं॒ ॅवज्र॒ मुद॑यच्छ दयच्छ॒ दुद् वज्रं॒ ॅवज्र॒ मुद॑यच्छत् । \newline
40. उद॑यच्छ दयच्छ॒ दुदु द॑यच्छ॒द् विष्ण्व॑नुस्थितो॒ विष्ण्व॑नुस्थितो ऽयच्छ॒ दुदु द॑यच्छ॒द् विष्ण्व॑नुस्थितः । \newline
41. अ॒य॒च्छ॒द् विष्ण्व॑नुस्थितो॒ विष्ण्व॑नुस्थितो ऽयच्छ दयच्छ॒द् विष्ण्व॑नुस्थितः॒ स स विष्ण्व॑नुस्थितो ऽयच्छ दयच्छ॒द् विष्ण्व॑नुस्थितः॒ सः । \newline
42. विष्ण्व॑नुस्थितः॒ स स विष्ण्व॑नुस्थितो॒ विष्ण्व॑नुस्थितः॒ सो᳚ ऽब्रवी दब्रवी॒थ् स विष्ण्व॑नुस्थितो॒ विष्ण्व॑नुस्थितः॒ सो᳚ ऽब्रवीत् । \newline
43. विष्ण्व॑नुस्थित॒ इति॒ विष्णु॑ - अ॒नु॒स्थि॒तः॒ । \newline
44. सो᳚ ऽब्रवी दब्रवी॒थ् स सो᳚ ऽब्रवी॒न् मा मा ऽब्र॑वी॒थ् स सो᳚ ऽब्रवी॒न् मा । \newline
45. अ॒ब्र॒वी॒न् मा मा ऽब्र॑वी दब्रवी॒न् मा मे॑ मे॒ मा ऽब्र॑वी दब्रवी॒न् मा मे᳚ । \newline
46. मा मे॑ मे॒ मा मा मे॒ प्र प्र मे॒ मा मा मे॒ प्र । \newline
47. मे॒ प्र प्र मे॑ मे॒ प्र हार्॑. हा॒र् प्र मे॑ मे॒ प्र हाः᳚ । \newline
48. प्र हार्॑. हा॒र् प्र प्र हा॒र् येन॒ येन॑ हा॒र् प्र प्र हा॒र् येन॑ । \newline
49. हा॒र् येन॒ येन॑ हार्. हा॒र् येना॒ह म॒हं ॅयेन॑ हार्. हा॒र् येना॒हम् । \newline
50. येना॒ह म॒हं ॅयेन॒ येना॒ह मि॒द मि॒द म॒हं ॅयेन॒ येना॒ह मि॒दम् । \newline
51. अ॒ह मि॒द मि॒द म॒ह म॒ह मि॒द मस् म्यस्मी॒द म॒ह म॒ह मि॒द मस्मि॑ । \newline
\pagebreak
\markright{ TS 2.4.12.6  \hfill https://www.vedavms.in \hfill}

\section{ TS 2.4.12.6 }

\textbf{TS 2.4.12.6 } \newline
\textbf{Samhita Paata} \newline

-मि॒दमस्मि॒ तत् ते॒ प्रदा᳚स्या॒मीति॒ त्वी(3) इत्य॑ब्रवीथ् स॒न्धां तु सन्द॑धावहै॒ त्वामे॒व प्रवि॑शा॒नीति॒ यन्मां प्र॑वि॒शेः किं मा॑ भुञ्ज्या॒ इत्य॑ब्रवी॒त् त्वामे॒वेन्धी॑य॒ तव॒ भोगा॑य॒ त्वां प्रवि॑शेय॒मित्य॑ब्रवी॒त् तं ॅवृ॒त्रः प्रावि॑शदु॒दरं॒ ॅवै वृ॒त्रः क्षुत् खलु॒ वै म॑नु॒ष्य॑स्य॒ भ्रातृ॑व्यो॒ य - [  ] \newline

\textbf{Pada Paata} \newline

इ॒दम् । अस्मि॑ । तत् । ते॒ । प्रेति॑ । दा॒स्या॒मि॒ । इति॑ । त्वी(3) । इति॑ । अ॒ब्र॒वी॒त् । स॒धांमिति॑ सं - धाम् । तु । समिति॑ । द॒धा॒व॒है॒ । त्वाम् । ए॒व । प्रेति॑ । वि॒शा॒नि॒ । इति॑ । यत् । माम् । प्र॒वि॒शेरिति॑ प्र-वि॒शेः । किम् । मा॒ । भु॒ञ्ज्याः॒ । इति॑ । अ॒ब्र॒वी॒त् । त्वाम् । ए॒व । इ॒न्धी॒य॒ । तव॑ । भोगा॑य । त्वाम् । प्रेति॑ । वि॒शे॒य॒म् । इति॑ । अ॒ब्र॒वी॒त् । तम् । वृ॒त्रः । प्रेति॑ । अ॒वि॒श॒त् । उ॒दर᳚म् । वै । वृ॒त्रः । क्षुत् । खलु॑ । वै । म॒नु॒ष्य॑स्य । भ्रातृ॑व्यः । यः ।  \newline


\textbf{Krama Paata} \newline

इ॒दमस्मि॑ । अस्मि॒ तत् । तत् ते᳚ । ते॒ प्र । प्र दा᳚स्यामि । दा॒स्या॒मीति॑ । इति॒ त्वी(3) । त्वी(3) इति॑ । इत्य॑ब्रवीत् । अ॒ब्र॒वी॒थ् स॒न्धाम् । स॒न्धाम् तु । स॒न्धामिति॑ सम् - धाम् । तु सम् । सम् द॑धावहै । द॒धा॒व॒है॒ त्वाम् । त्वामे॒व । ए॒व प्र । प्र वि॑शानि । वि॒शा॒नीति॑ । इति॒ यत् । यन् माम् । माम् प्र॑वि॒शेः । प्र॒वि॒शेः किम् । प्र॒वि॒शेरिति॑ प्र - वि॒शेः । किम् मा᳚ । मा॒ भु॒ञ्ज्याः॒ । भु॒ञ्ज्या॒ इति॑ । इत्य॑ब्रवीत् । अ॒ब्र॒वी॒त् त्वाम् । त्वामे॒व । ए॒वेन्धी॑य । इ॒न्धी॒य॒ तव॑ । तव॒ भोगा॑य । भोगा॑य॒ त्वाम् । त्वाम् प्र । प्र वि॑शेयम् । वि॒शे॒य॒मिति॑ । इत्य॑ब्रवीत् । अ॒ब्र॒वी॒त् तम् । तं ॅवृ॒त्रः । वृ॒त्रः प्र । प्रावि॑शत् । अ॒वि॒श॒दु॒दर᳚म् । उ॒दरं॒ ॅवै । वै वृ॒त्रः । वृ॒त्रः क्षुत् । क्षुत् खलु॑ । खलु॒ वै । वै म॑नु॒ष्य॑स्य । म॒नु॒ष्य॑स्य॒ भ्रातृ॑व्यः । भ्रातृ॑व्यो॒ यः । य ए॒वम् \newline

\textbf{Jatai Paata} \newline

1. इ॒द मस् म्यस्मी॒द मि॒द मस्मि॑ । \newline
2. अस्मि॒ तत् तदस् म्यस्मि॒ तत् । \newline
3. तत् ते॑ ते॒ तत् तत् ते᳚ । \newline
4. ते॒ प्र प्र ते॑ ते॒ प्र । \newline
5. प्र दा᳚स्यामि दास्यामि॒ प्र प्र दा᳚स्यामि । \newline
6. दा॒स्या॒मीतीति॑ दास्यामि दास्या॒मीति॑ । \newline
7. इति॒ त्वी(3) त्वी(3) इतीति॒ त्वी(3) । \newline
8. त्वी(3) इतीति॒ त्वी(3) त्वी(3) इति॑ । \newline
9. इत्य॑ब्रवी दब्रवी॒ दिती त्य॑ब्रवीत् । \newline
10. अ॒ब्र॒वी॒थ् स॒न्धाꣳ स॒न्धा म॑ब्रवी दब्रवीथ् स॒न्धाम् । \newline
11. स॒न्धाम् तु तु स॒न्धाꣳ स॒न्धाम् तु । \newline
12. स॒न्धामिति॑ सं - धाम् । \newline
13. तु सꣳ सम् तु तु सम् । \newline
14. सम् द॑धावहै दधावहै॒ सꣳ सम् द॑धावहै । \newline
15. द॒धा॒व॒है॒ त्वाम् त्वाम् द॑धावहै दधावहै॒ त्वाम् । \newline
16. त्वा मे॒वैव त्वाम् त्वा मे॒व । \newline
17. ए॒व प्र प्रैवैव प्र । \newline
18. प्र वि॑शानि विशानि॒ प्र प्र वि॑शानि । \newline
19. वि॒शा॒नीतीति॑ विशानि विशा॒नीति॑ । \newline
20. इति॒ यद् यदितीति॒ यत् । \newline
21. यन् माम् मां ॅयद् यन् माम् । \newline
22. माम् प्र॑वि॒शेः प्र॑वि॒शेर् माम् माम् प्र॑वि॒शेः । \newline
23. प्र॒वि॒शेः किम् किम् प्र॑वि॒शेः प्र॑वि॒शेः किम् । \newline
24. प्र॒वि॒शेरिति॑ प्र - वि॒शेः । \newline
25. किम् मा॑ मा॒ किम् किम् मा᳚ । \newline
26. मा॒ भु॒ञ्ज्या॒ भु॒ञ्ज्या॒ मा॒ मा॒ भु॒ञ्ज्याः॒ । \newline
27. भु॒ञ्ज्या॒ इतीति॑ भुञ्ज्या भुञ्ज्या॒ इति॑ । \newline
28. इत्य॑ब्रवी दब्रवी॒ दिती त्य॑ब्रवीत् । \newline
29. अ॒ब्र॒वी॒त् त्वाम् त्वा म॑ब्रवी दब्रवी॒त् त्वाम् । \newline
30. त्वा मे॒वैव त्वाम् त्वा मे॒व । \newline
31. ए॒वे न्धी॑ये न्धीयै॒वैवे न्धी॑य । \newline
32. इ॒न्धी॒य॒ तव॒ तवे᳚ न्धीये न्धीय॒ तव॑ । \newline
33. तव॒ भोगा॑य॒ भोगा॑य॒ तव॒ तव॒ भोगा॑य । \newline
34. भोगा॑य॒ त्वाम् त्वाम् भोगा॑य॒ भोगा॑य॒ त्वाम् । \newline
35. त्वाम् प्र प्र त्वाम् त्वाम् प्र । \newline
36. प्र वि॑शेयं ॅविशेय॒म् प्र प्र वि॑शेयम् । \newline
37. वि॒शे॒य॒ मितीति॑ विशेयं ॅविशेय॒ मिति॑ । \newline
38. इत्य॑ब्रवी दब्रवी॒ दिती त्य॑ब्रवीत् । \newline
39. अ॒ब्र॒वी॒त् तम् त म॑ब्रवी दब्रवी॒त् तम् । \newline
40. तं ॅवृ॒त्रो वृ॒त्र स्तम् तं ॅवृ॒त्रः । \newline
41. वृ॒त्रः प्र प्र वृ॒त्रो वृ॒त्रः प्र । \newline
42. प्रावि॑श दविश॒त् प्र प्रावि॑शत् । \newline
43. अ॒वि॒श॒ दु॒दर॑ मु॒दर॑ मविश दविश दु॒दर᳚म् । \newline
44. उ॒दरं॒ ॅवै वा उ॒दर॑ मु॒दरं॒ ॅवै । \newline
45. वै वृ॒त्रो वृ॒त्रो वै वै वृ॒त्रः । \newline
46. वृ॒त्रः क्षुत् क्षुद् वृ॒त्रो वृ॒त्रः क्षुत् । \newline
47. क्षुत् खलु॒ खलु॒ क्षुत् क्षुत् खलु॑ । \newline
48. खलु॒ वै वै खलु॒ खलु॒ वै । \newline
49. वै म॑नु॒ष्य॑स्य मनु॒ष्य॑स्य॒ वै वै म॑नु॒ष्य॑स्य । \newline
50. म॒नु॒ष्य॑स्य॒ भ्रातृ॑व्यो॒ भ्रातृ॑व्यो मनु॒ष्य॑स्य मनु॒ष्य॑स्य॒ भ्रातृ॑व्यः । \newline
51. भ्रातृ॑व्यो॒ यो यो भ्रातृ॑व्यो॒ भ्रातृ॑व्यो॒ यः । \newline
52. य ए॒व मे॒वं ॅयो य ए॒वम् । \newline

\textbf{Ghana Paata } \newline

1. इ॒द मस् म्यस्मी॒द मि॒द मस्मि॒ तत् तदस्मी॒द मि॒द मस्मि॒ तत् । \newline
2. अस्मि॒ तत् तदस् म्यस्मि॒ तत् ते॑ ते॒ तदस् म्यस्मि॒ तत् ते᳚ । \newline
3. तत् ते॑ ते॒ तत् तत् ते॒ प्र प्र ते॒ तत् तत् ते॒ प्र । \newline
4. ते॒ प्र प्र ते॑ ते॒ प्र दा᳚स्यामि दास्यामि॒ प्र ते॑ ते॒ प्र दा᳚स्यामि । \newline
5. प्र दा᳚स्यामि दास्यामि॒ प्र प्र दा᳚स्या॒मीतीति॑ दास्यामि॒ प्र प्र दा᳚स्या॒मीति॑ । \newline
6. दा॒स्या॒मीतीति॑ दास्यामि दास्या॒मीति॒ त्वी(3) त्वी(3) इति॑ दास्यामि दास्या॒मीति॒ त्वी(3) । \newline
7. इति॒ त्वी(3) त्वी(3) इतीति॒ त्वी(3) इतीति॒ त्वी(3) इतीति॒ त्वी(3) इति॑ । \newline
8. त्वी(3) इतीति॒ त्वी(3) त्वी(3) इत्य॑ब्रवी दब्रवी॒ दिति॒ त्वी(3) त्वी(3) इत्य॑ब्रवीत् । \newline
9. इत्य॑ब्रवी दब्रवी॒ दिती त्य॑ब्रवीथ् स॒न्धाꣳ स॒न्धा म॑ब्रवी॒ दिती त्य॑ब्रवीथ् स॒न्धाम् । \newline
10. अ॒ब्र॒वी॒थ् स॒न्धाꣳ स॒न्धा म॑ब्रवी दब्रवीथ् स॒न्धाम् तु तु स॒न्धा म॑ब्रवी दब्रवीथ् स॒न्धाम् तु । \newline
11. स॒न्धाम् तु तु स॒न्धाꣳ स॒न्धाम् तु सꣳ सम् तु स॒न्धाꣳ स॒न्धाम् तु सम् । \newline
12. स॒न्धामिति॑ सं - धाम् । \newline
13. तु सꣳ सम् तु तु सम् द॑धावहै दधावहै॒ सम् तु तु सम् द॑धावहै । \newline
14. सम् द॑धावहै दधावहै॒ सꣳ सम् द॑धावहै॒ त्वाम् त्वाम् द॑धावहै॒ सꣳ सम् द॑धावहै॒ त्वाम् । \newline
15. द॒धा॒व॒है॒ त्वाम् त्वाम् द॑धावहै दधावहै॒ त्वा मे॒वैव त्वाम् द॑धावहै दधावहै॒ त्वा मे॒व । \newline
16. त्वा मे॒वैव त्वाम् त्वा मे॒व प्र प्रैव त्वाम् त्वा मे॒व प्र । \newline
17. ए॒व प्र प्रैवैव प्र वि॑शानि विशानि॒ प्रैवैव प्र वि॑शानि । \newline
18. प्र वि॑शानि विशानि॒ प्र प्र वि॑शा॒नीतीति॑ विशानि॒ प्र प्र वि॑शा॒नीति॑ । \newline
19. वि॒शा॒नीतीति॑ विशानि विशा॒नीति॒ यद् यदिति॑ विशानि विशा॒नीति॒ यत् । \newline
20. इति॒ यद् यदितीति॒ यन् माम् मां ॅयदितीति॒ यन् माम् । \newline
21. यन् माम् मां ॅयद् यन् माम् प्र॑वि॒शेः प्र॑वि॒शेर् मां ॅयद् यन् माम् प्र॑वि॒शेः । \newline
22. माम् प्र॑वि॒शेः प्र॑वि॒शेर् माम् माम् प्र॑वि॒शेः किम् किम् प्र॑वि॒शेर् माम् माम् प्र॑वि॒शेः किम् । \newline
23. प्र॒वि॒शेः किम् किम् प्र॑वि॒शेः प्र॑वि॒शेः किम् मा॑ मा॒ किम् प्र॑वि॒शेः प्र॑वि॒शेः किम् मा᳚ । \newline
24. प्र॒वि॒शेरिति॑ प्र - वि॒शेः । \newline
25. किम् मा॑ मा॒ किम् किम् मा॑ भुञ्ज्या भुञ्ज्या मा॒ किम् किम् मा॑ भुञ्ज्याः । \newline
26. मा॒ भु॒ञ्ज्या॒ भु॒ञ्ज्या॒ मा॒ मा॒ भु॒ञ्ज्या॒ इतीति॑ भुञ्ज्या मा मा भुञ्ज्या॒ इति॑ । \newline
27. भु॒ञ्ज्या॒ इतीति॑ भुञ्ज्या भुञ्ज्या॒ इत्य॑ब्रवी दब्रवी॒ दिति॑ भुञ्ज्या भुञ्ज्या॒ इत्य॑ब्रवीत् । \newline
28. इत्य॑ब्रवी दब्रवी॒ दिती त्य॑ब्रवी॒त् त्वाम् त्वा म॑ब्रवी॒ दिती त्य॑ब्रवी॒त् त्वाम् । \newline
29. अ॒ब्र॒वी॒त् त्वाम् त्वा म॑ब्रवी दब्रवी॒त् त्वा मे॒वैव त्वा म॑ब्रवी दब्रवी॒त् त्वा मे॒व । \newline
30. त्वा मे॒वैव त्वाम् त्वा मे॒वे न्धी॑ये न्धीयै॒व त्वाम् त्वा मे॒वे न्धी॑य । \newline
31. ए॒वे न्धी॑ये न्धीयै॒वैवे न्धी॑य॒ तव॒ तवे᳚ न्धीयै॒वैवे न्धी॑य॒ तव॑ । \newline
32. इ॒न्धी॒य॒ तव॒ तवे᳚ न्धीये न्धीय॒ तव॒ भोगा॑य॒ भोगा॑य॒ तवे᳚ न्धीये न्धीय॒ तव॒ भोगा॑य । \newline
33. तव॒ भोगा॑य॒ भोगा॑य॒ तव॒ तव॒ भोगा॑य॒ त्वाम् त्वाम् भोगा॑य॒ तव॒ तव॒ भोगा॑य॒ त्वाम् । \newline
34. भोगा॑य॒ त्वाम् त्वाम् भोगा॑य॒ भोगा॑य॒ त्वाम् प्र प्र त्वाम् भोगा॑य॒ भोगा॑य॒ त्वाम् प्र । \newline
35. त्वाम् प्र प्र त्वाम् त्वाम् प्र वि॑शेयं ॅविशेय॒म् प्र त्वाम् त्वाम् प्र वि॑शेयम् । \newline
36. प्र वि॑शेयं ॅविशेय॒म् प्र प्र वि॑शेय॒ मितीति॑ विशेय॒म् प्र प्र वि॑शेय॒ मिति॑ । \newline
37. वि॒शे॒य॒ मितीति॑ विशेयं ॅविशेय॒ मित्य॑ब्रवी दब्रवी॒ दिति॑ विशेयं ॅविशेय॒ मित्य॑ब्रवीत् । \newline
38. इत्य॑ब्रवी दब्रवी॒ दिती त्य॑ब्रवी॒त् तम् त म॑ब्रवी॒ दिती त्य॑ब्रवी॒त् तम् । \newline
39. अ॒ब्र॒वी॒त् तम् त म॑ब्रवी दब्रवी॒त् तं ॅवृ॒त्रो वृ॒त्रस्त म॑ब्रवी दब्रवी॒त् तं ॅवृ॒त्रः । \newline
40. तं ॅवृ॒त्रो वृ॒त्र स्तम् तं ॅवृ॒त्रः प्र प्र वृ॒त्र स्तम् तं ॅवृ॒त्रः प्र । \newline
41. वृ॒त्रः प्र प्र वृ॒त्रो वृ॒त्रः प्रावि॑श दविश॒त् प्र वृ॒त्रो वृ॒त्रः प्रावि॑शत् । \newline
42. प्रावि॑श दविश॒त् प्र प्रावि॑श दु॒दर॑ मु॒दर॑ मविश॒त् प्र प्रावि॑श दु॒दर᳚म् । \newline
43. अ॒वि॒श॒ दु॒दर॑ मु॒दर॑ मविश दविश दु॒दरं॒ ॅवै वा उ॒दर॑ मविश दविश दु॒दरं॒ ॅवै । \newline
44. उ॒दरं॒ ॅवै वा उ॒दर॑ मु॒दरं॒ ॅवै वृ॒त्रो वृ॒त्रो वा उ॒दर॑ मु॒दरं॒ ॅवै वृ॒त्रः । \newline
45. वै वृ॒त्रो वृ॒त्रो वै वै वृ॒त्रः क्षुत् क्षुद् वृ॒त्रो वै वै वृ॒त्रः क्षुत् । \newline
46. वृ॒त्रः क्षुत् क्षुद् वृ॒त्रो वृ॒त्रः क्षुत् खलु॒ खलु॒ क्षुद् वृ॒त्रो वृ॒त्रः क्षुत् खलु॑ । \newline
47. क्षुत् खलु॒ खलु॒ क्षुत् क्षुत् खलु॒ वै वै खलु॒ क्षुत् क्षुत् खलु॒ वै । \newline
48. खलु॒ वै वै खलु॒ खलु॒ वै म॑नु॒ष्य॑स्य मनु॒ष्य॑स्य॒ वै खलु॒ खलु॒ वै म॑नु॒ष्य॑स्य । \newline
49. वै म॑नु॒ष्य॑स्य मनु॒ष्य॑स्य॒ वै वै म॑नु॒ष्य॑स्य॒ भ्रातृ॑व्यो॒ भ्रातृ॑व्यो मनु॒ष्य॑स्य॒ वै वै म॑नु॒ष्य॑स्य॒ भ्रातृ॑व्यः । \newline
50. म॒नु॒ष्य॑स्य॒ भ्रातृ॑व्यो॒ भ्रातृ॑व्यो मनु॒ष्य॑स्य मनु॒ष्य॑स्य॒ भ्रातृ॑व्यो॒ यो यो भ्रातृ॑व्यो मनु॒ष्य॑स्य मनु॒ष्य॑स्य॒ भ्रातृ॑व्यो॒ यः । \newline
51. भ्रातृ॑व्यो॒ यो यो भ्रातृ॑व्यो॒ भ्रातृ॑व्यो॒ य ए॒व मे॒वं ॅयो भ्रातृ॑व्यो॒ भ्रातृ॑व्यो॒ य ए॒वम् । \newline
52. य ए॒व मे॒वं ॅयो य ए॒वं ॅवेद॒ वेदै॒वं ॅयो य ए॒वं ॅवेद॑ । \newline
\pagebreak
\markright{ TS 2.4.12.7  \hfill https://www.vedavms.in \hfill}

\section{ TS 2.4.12.7 }

\textbf{TS 2.4.12.7 } \newline
\textbf{Samhita Paata} \newline

ए॒वं ॅवेद॒ हन्ति॒ क्षुधं॒ भ्रातृ॑व्यं॒ तद॑स्मै॒ प्राय॑च्छ॒त्‌तत् प्रत्य॑गृह्णा॒त्-  त्रिर्मा॑ऽधा॒ इति॒ तद्-विष्ण॒वेऽति॒ प्राय॑च्छ॒त् तद्-विष्णुः॒ प्रत्य॑गृह्णाद॒स्मास्विन्द्र॑ इन्द्रि॒यं द॑धा॒त्विति॒ यत्त्रिः प्राय॑च्छ॒त् त्रिः प्र॒त्यगृ॑ह्णा॒त् तत् त्रि॒धातो᳚स्त्रिधातु॒त्वं ॅयद्-विष्णु॑र॒न्वति॑ष्ठत॒ विष्ण॒वेऽति॒ प्राय॑च्छ॒त् तस्मा॑दैन्द्रावैष्ण॒वꣳ ह॒विर्भ॑वति॒ ( ) यद्वा इ॒दं किञ्च॒ तद॑स्मै॒ तत् प्राय॑च्छ॒द् ऋचः॒ सामा॑नि॒ यजूꣳ॑षि स॒हस्रं॒ ॅवा अ॑स्मै॒ तत् प्राय॑च्छ॒त् तस्मा᳚थ् स॒हस्र॑दक्षिणं ॥ \newline

\textbf{Pada Paata} \newline

ए॒वम् । वेद॑ । हन्ति॑ । क्षुध᳚म् । भ्रातृ॑व्यम् । तत् । अ॒स्मै॒ । प्रेति॑ । अ॒य॒च्छ॒त् । तत् । प्रतीति॑ । अ॒गृ॒ह्णा॒त् । त्रिः । मा॒ । अ॒धाः॒ । इति॑ । तत् । विष्ण॑वे । अति॑ । प्रेति॑ । अ॒य॒च्छ॒त् । तत् । विष्णुः॑ । प्रतीति॑ । अ॒गृ॒ह्णा॒त् । अ॒स्मासु॑ । इन्द्रः॑ । इ॒न्द्रि॒यम् । द॒धा॒तु॒ । इति॑ । यत् । त्रिः । प्रेति॑ । अय॑च्छत् । त्रिः । प्र॒त्यगृ॑ह्णा॒दिति॑ प्रति – अगृ॑ह्णात् । तत् । त्रि॒धातो॒रिति॑ त्रि - धातोः᳚ । त्रि॒धा॒तु॒त्वमिति॑ त्रिधातु - त्वम् । यत् । विष्णुः॑ । अ॒न्वति॑ष्ठ॒तेत्य॑नु - अति॑ष्ठत । विष्ण॑वे । अति॑ । प्रेति॑ । अय॑च्छत् । तस्मा᳚त् । ऐ॒न्द्रा॒वै॒ष्ण॒वमित्यै᳚न्द्रा - वै॒ष्ण॒वम् । ह॒विः । भ॒व॒ति॒ ( ) । यत् । वै । इ॒दम् । किम् । च॒ । तत् । अ॒स्मै॒ । तत् । प्रेति॑ । अ॒य॒च्छ॒त् । ऋचः॑ । सामा॑नि । यजूꣳ॑षि । स॒हस्र᳚म् । वै । अ॒स्मै॒ । तत् । प्रेति॑ । अ॒य॒च्छ॒त् । तस्मा᳚त् । स॒हस्र॑दक्षिण॒मिति॑ स॒हस्र॑ - द॒क्षि॒ण॒म् ॥  \newline


\textbf{Krama Paata} \newline

ए॒वं ॅवेद॑ । वेद॒ हन्ति॑ । हन्ति॒ क्षुध᳚म् । क्षुध॒म् भ्रातृ॑व्यम् । भ्रातृ॑व्य॒म् तत् । तद॑स्मै । अ॒स्मै॒ प्र । प्राय॑च्छत् । अ॒य॒च्छ॒त् तत् । तत् प्रति॑ । प्रत्य॑गृह्णात् । अ॒गृ॒ह्णा॒त् त्रिः । त्रिर्मा᳚ । मा॒ऽधाः॒ । अ॒धा॒ इति॑ । इति॒ तत् । तद् विष्ण॑वे । विष्ण॒वेऽति॑ । अति॒ प्र । प्राय॑च्छत् । अ॒य॒च्छ॒त् तत् । तद् विष्णुः॑ । विष्णुः॒ प्रति॑ । प्रत्य॑गृह्णात् । अ॒गृ॒ह्णा॒द॒स्मासु॑ । अ॒स्मास्विन्द्रः॑ । इन्द्र॑ इन्द्रि॒यम् । इ॒न्द्रि॒यम् द॑धातु । द॒धा॒त्विति॑ । इति॒ यत् । यत् त्रिः । त्रिः प्र । प्राय॑च्छत् । अय॑च्छ॒त् त्रिः । त्रिः प्र॒त्यगृ॑ह्णात् । प्र॒त्यगृ॑ह्णा॒त् तत् । प्र॒त्यगृ॑ह्णा॒दिति॑ प्रति - अगृ॑ह्णात् । तत् त्रि॒धातोः᳚ । त्रि॒धातो᳚स्त्रिधातु॒त्वम् । त्रि॒धातो॒रिति॑ त्रि - धातोः᳚ । त्रि॒धा॒तु॒त्वं ॅयत् । त्रि॒धा॒तु॒त्वमिति॑ त्रिधातु - त्वम् । यद् विष्णुः॑ । विष्णु॑र॒न्वति॑ष्ठत । अ॒न्वति॑ष्ठत॒ विष्ण॑वे । अ॒न्वति॑ष्ठ॒तेत्य॑नु - अति॑ष्ठत । विष्ण॒वेऽति॑ । अति॒ प्र । प्राय॑च्छत् । अय॑च्छ॒त् तस्मा᳚त् । तस्मा॑दैन्द्रावैष्ण॒वम् । ऐ॒न्द्रा॒वै॒ष्ण॒वꣳ ह॒विः । ऐ॒न्द्रा॒वै॒ष्ण॒वमित्यै᳚न्द्रा - वै॒ष्ण॒वम् । ह॒विर् भ॑वति ( ) । भ॒व॒ति॒ यत् । यद् वै । वा इ॒दम् । इ॒दम् किम् । किम् च॑ । च॒ तत् । तद॑स्मै । अ॒स्मै॒ तत् । तत् प्र । प्राय॑च्छत् । अ॒य॒च्छ॒दृचः॑ । ऋचः॒ सामा॑नि । सामा॑नि॒ यजूꣳ॑षि । यजूꣳ॑षि स॒हस्र᳚म् । स॒हस्रं॒ ॅवै । वा अ॑स्मै । अ॒स्मै॒ तत् । तत् प्र । प्राय॑च्छत् । अ॒य॒च्छ॒त् तस्मा᳚त् । तस्मा᳚थ् स॒हस्र॑दक्षिणम् । स॒हस्र॑दक्षिण॒मिति॑ स॒हस्र॑ - द॒क्षि॒ण॒म् । \newline

\textbf{Jatai Paata} \newline

1. ए॒वं ॅवेद॒ वेदै॒व मे॒वं ॅवेद॑ । \newline
2. वेद॒ हन्ति॒ हन्ति॒ वेद॒ वेद॒ हन्ति॑ । \newline
3. हन्ति॒ क्षुध॒म् क्षुधꣳ॒॒ हन्ति॒ हन्ति॒ क्षुध᳚म् । \newline
4. क्षुध॒म् भ्रातृ॑व्य॒म् भ्रातृ॑व्य॒म् क्षुध॒म् क्षुध॒म् भ्रातृ॑व्यम् । \newline
5. भ्रातृ॑व्य॒म् तत् तद् भ्रातृ॑व्य॒म् भ्रातृ॑व्य॒म् तत् । \newline
6. तद॑स्मा अस्मै॒ तत् तद॑स्मै । \newline
7. अ॒स्मै॒ प्र प्रास्मा॑ अस्मै॒ प्र । \newline
8. प्राय॑च्छ दयच्छ॒त् प्र प्राय॑च्छत् । \newline
9. अ॒य॒च्छ॒त् तत् तद॑यच्छ दयच्छ॒त् तत् । \newline
10. तत् प्रति॒ प्रति॒ तत् तत् प्रति॑ । \newline
11. प्रत्य॑गृह्णा दगृह्णा॒त् प्रति॒ प्रत्य॑गृह्णात् । \newline
12. अ॒गृ॒ह्णा॒त् त्रि स्त्रि र॑गृह्णा दगृह्णा॒त् त्रिः । \newline
13. त्रिर् मा॑ मा॒ त्रि स्त्रिर् मा᳚ । \newline
14. मा॒ ऽधा॒ अ॒धा॒ मा॒ मा॒ ऽधाः॒ । \newline
15. अ॒धा॒ इती त्य॑धा अधा॒ इति॑ । \newline
16. इति॒ तत् तदितीति॒ तत् । \newline
17. तद् विष्ण॑वे॒ विष्ण॑वे॒ तत् तद् विष्ण॑वे । \newline
18. विष्ण॒वे ऽत्यति॒ विष्ण॑वे॒ विष्ण॒वे ऽति॑ । \newline
19. अति॒ प्र प्रात्यति॒ प्र । \newline
20. प्राय॑च्छ दयच्छ॒त् प्र प्राय॑च्छत् । \newline
21. अ॒य॒च्छ॒त् तत् तद॑यच्छ दयच्छ॒त् तत् । \newline
22. तद् विष्णु॒र् विष्णु॒ स्तत् तद् विष्णुः॑ । \newline
23. विष्णुः॒ प्रति॒ प्रति॒ विष्णु॒र् विष्णुः॒ प्रति॑ । \newline
24. प्रत्य॑गृह्णा दगृह्णा॒त् प्रति॒ प्रत्य॑गृह्णात् । \newline
25. अ॒गृ॒ह्णा॒ द॒स्मा स्व॒स्मा स्व॑गृह्णा दगृह्णा द॒स्मासु॑ । \newline
26. अ॒स्मा स्विन्द्र॒ इन्द्रो॒ ऽस्मा स्व॒स्मा स्विन्द्रः॑ । \newline
27. इन्द्र॑ इन्द्रि॒य मि॑न्द्रि॒य मिन्द्र॒ इन्द्र॑ इन्द्रि॒यम् । \newline
28. इ॒न्द्रि॒यम् द॑धातु दधा त्विन्द्रि॒य मि॑न्द्रि॒यम् द॑धातु । \newline
29. द॒धा॒ त्वितीति॑ दधातु दधा॒ त्विति॑ । \newline
30. इति॒ यद् यदितीति॒ यत् । \newline
31. यत् त्रि स्त्रिर् यद् यत् त्रिः । \newline
32. त्रिः प्र प्र त्रि स्त्रिः प्र । \newline
33. प्राय॑च्छ॒ दय॑च्छ॒त् प्र प्राय॑च्छत् । \newline
34. अय॑च्छ॒त् त्रि स्त्रि रय॑च्छ॒ दय॑च्छ॒त् त्रिः । \newline
35. त्रिः प्र॒त्यगृ॑ह्णात् प्र॒त्यगृ॑ह्णा॒त् त्रि स्त्रिः प्र॒त्यगृ॑ह्णात् । \newline
36. प्र॒त्यगृ॑ह्णा॒त् तत् तत् प्र॒त्यगृ॑ह्णात् प्र॒त्यगृ॑ह्णा॒त् तत् । \newline
37. प्र॒त्यगृ॑ह्णा॒दिति॑ प्रति - अगृ॑ह्णात् । \newline
38. तत् त्रि॒धातो᳚ स्त्रि॒धातो॒ स्तत् तत् त्रि॒धातोः᳚ । \newline
39. त्रि॒धातो᳚ स्त्रिधातु॒त्वम् त्रि॑धातु॒त्वम् त्रि॒धातो᳚ स्त्रि॒धातो᳚ स्त्रिधातु॒त्वम् । \newline
40. त्रि॒धातो॒रिति॑ त्रि - धातोः᳚ । \newline
41. त्रि॒धा॒तु॒त्वं ॅयद् यत् त्रि॑धातु॒त्वम् त्रि॑धातु॒त्वं ॅयत् । \newline
42. त्रि॒धा॒तु॒त्वमिति॑ त्रिधातु - त्वम् । \newline
43. यद् विष्णु॒र् विष्णु॒र् यद् यद् विष्णुः॑ । \newline
44. विष्णु॑ र॒न्वति॑ष्ठता॒ न्वति॑ष्ठत॒ विष्णु॒र् विष्णु॑ र॒न्वति॑ष्ठत । \newline
45. अ॒न्वति॑ष्ठत॒ विष्ण॑वे॒ विष्ण॑वे॒ ऽन्वति॑ष्ठता॒ न्वति॑ष्ठत॒ विष्ण॑वे । \newline
46. अ॒न्वति॑ष्ठ॒तेत्य॑नु - अति॑ष्ठत । \newline
47. विष्ण॒वे ऽत्यति॒ विष्ण॑वे॒ विष्ण॒वे ऽति॑ । \newline
48. अति॒ प्र प्रात्यति॒ प्र । \newline
49. प्राय॑च्छ॒ दय॑च्छ॒त् प्र प्राय॑च्छत् । \newline
50. अय॑च्छ॒त् तस्मा॒त् तस्मा॒ दय॑च्छ॒ दय॑च्छ॒त् तस्मा᳚त् । \newline
51. तस्मा॑ दैन्द्रावैष्ण॒व मै᳚न्द्रावैष्ण॒वम् तस्मा॒त् तस्मा॑ दैन्द्रावैष्ण॒वम् । \newline
52. ऐ॒न्द्रा॒वै॒ष्ण॒वꣳ ह॒विर्. ह॒वि रै᳚न्द्रावैष्ण॒व मै᳚न्द्रावैष्ण॒वꣳ ह॒विः । \newline
53. ऐ॒न्द्रा॒वै॒ष्ण॒वमित्यै᳚न्द्रा - वै॒ष्ण॒वम् । \newline
54. ह॒विर् भ॑वति भवति ह॒विर्. ह॒विर् भ॑वति । \newline
55. भ॒व॒ति॒ यद् यद् भ॑वति भवति॒ यत् । \newline
56. यद् वै वै यद् यद् वै । \newline
57. वा इ॒द मि॒दं ॅवै वा इ॒दम् । \newline
58. इ॒दम् किम् कि मि॒द मि॒दम् किम् । \newline
59. किम् च॑ च॒ किम् किम् च॑ । \newline
60. च॒ तत् तच् च॑ च॒ तत् । \newline
61. तद॑स्मा अस्मै॒ तत् तद॑स्मै । \newline
62. अ॒स्मै॒ तत् तद॑स्मा अस्मै॒ तत् । \newline
63. तत् प्र प्र तत् तत् प्र । \newline
64. प्राय॑च्छ दयच्छ॒त् प्र प्राय॑च्छत् । \newline
65. अ॒य॒च्छ॒ दृच॒ ऋचो॑ ऽयच्छ दयच्छ॒ दृचः॑ । \newline
66. ऋचः॒ सामा॑नि॒ सामा॒ न्यृच॒ ऋचः॒ सामा॑नि । \newline
67. सामा॑नि॒ यजूꣳ॑षि॒ यजूꣳ॑षि॒ सामा॑नि॒ सामा॑नि॒ यजूꣳ॑षि । \newline
68. यजूꣳ॑षि स॒हस्रꣳ॑ स॒हस्रं॒ ॅयजूꣳ॑षि॒ यजूꣳ॑षि स॒हस्र᳚म् । \newline
69. स॒हस्रं॒ ॅवै वै स॒हस्रꣳ॑ स॒हस्रं॒ ॅवै । \newline
70. वा अ॑स्मा अस्मै॒ वै वा अ॑स्मै । \newline
71. अ॒स्मै॒ तत् तद॑स्मा अस्मै॒ तत् । \newline
72. तत् प्र प्र तत् तत् प्र । \newline
73. प्राय॑च्छ दयच्छ॒त् प्र प्राय॑च्छत् । \newline
74. अ॒य॒च्छ॒त् तस्मा॒त् तस्मा॑ दयच्छ दयच्छ॒त् तस्मा᳚त् । \newline
75. तस्मा᳚थ् स॒हस्र॑दक्षिणꣳ स॒हस्र॑दक्षिण॒म् तस्मा॒त् तस्मा᳚थ् स॒हस्र॑दक्षिणम् । \newline
76. स॒हस्र॑दक्षिण॒मिति॑ स॒हस्र॑ - द॒क्षि॒ण॒म् । \newline

\textbf{Ghana Paata } \newline

1. ए॒वं ॅवेद॒ वेदै॒व मे॒वं ॅवेद॒ हन्ति॒ हन्ति॒ वेदै॒व मे॒वं ॅवेद॒ हन्ति॑ । \newline
2. वेद॒ हन्ति॒ हन्ति॒ वेद॒ वेद॒ हन्ति॒ क्षुध॒म् क्षुधꣳ॒॒ हन्ति॒ वेद॒ वेद॒ हन्ति॒ क्षुध᳚म् । \newline
3. हन्ति॒ क्षुध॒म् क्षुधꣳ॒॒ हन्ति॒ हन्ति॒ क्षुध॒म् भ्रातृ॑व्य॒म् भ्रातृ॑व्य॒म् क्षुधꣳ॒॒ हन्ति॒ हन्ति॒ क्षुध॒म् भ्रातृ॑व्यम् । \newline
4. क्षुध॒म् भ्रातृ॑व्य॒म् भ्रातृ॑व्य॒म् क्षुध॒म् क्षुध॒म् भ्रातृ॑व्य॒म् तत् तद् भ्रातृ॑व्य॒म् क्षुध॒म् क्षुध॒म् भ्रातृ॑व्य॒म् तत् । \newline
5. भ्रातृ॑व्य॒म् तत् तद् भ्रातृ॑व्य॒म् भ्रातृ॑व्य॒म् तद॑स्मा अस्मै॒ तद् भ्रातृ॑व्य॒म् भ्रातृ॑व्य॒म् तद॑स्मै । \newline
6. तद॑स्मा अस्मै॒ तत् तद॑स्मै॒ प्र प्रास्मै॒ तत् तद॑स्मै॒ प्र । \newline
7. अ॒स्मै॒ प्र प्रास्मा॑ अस्मै॒ प्राय॑च्छ दयच्छ॒त् प्रास्मा॑ अस्मै॒ प्राय॑च्छत् । \newline
8. प्राय॑च्छ दयच्छ॒त् प्र प्राय॑च्छ॒त् तत् तद॑यच्छ॒त् प्र प्राय॑च्छ॒त् तत् । \newline
9. अ॒य॒च्छ॒त् तत् तद॑यच्छ दयच्छ॒त् तत् प्रति॒ प्रति॒ तद॑यच्छ दयच्छ॒त् तत् प्रति॑ । \newline
10. तत् प्रति॒ प्रति॒ तत् तत् प्रत्य॑गृह्णा दगृह्णा॒त् प्रति॒ तत् तत् प्रत्य॑गृह्णात् । \newline
11. प्रत्य॑गृह्णा दगृह्णा॒त् प्रति॒ प्रत्य॑गृह्णा॒त् त्रि स्त्रि र॑गृह्णा॒त् प्रति॒ प्रत्य॑गृह्णा॒त् त्रिः । \newline
12. अ॒गृ॒ह्णा॒त् त्रि स्त्रि र॑गृह्णा दगृह्णा॒त् त्रिर् मा॑ मा॒ त्रि र॑गृह्णा दगृह्णा॒त् त्रिर् मा᳚ । \newline
13. त्रिर् मा॑ मा॒ त्रि स्त्रिर् मा॑ ऽधा अधा मा॒ त्रि स्त्रिर् मा॑ ऽधाः । \newline
14. मा॒ ऽधा॒ अ॒धा॒ मा॒ मा॒ ऽधा॒ इतीत्य॑धा मा मा ऽधा॒ इति॑ । \newline
15. अ॒धा॒ इतीत्य॑धा अधा॒ इति॒ तत् तदित्य॑धा अधा॒ इति॒ तत् । \newline
16. इति॒ तत् तदितीति॒ तद् विष्ण॑वे॒ विष्ण॑वे॒ तदितीति॒ तद् विष्ण॑वे । \newline
17. तद् विष्ण॑वे॒ विष्ण॑वे॒ तत् तद् विष्ण॒वे ऽत्यति॒ विष्ण॑वे॒ तत् तद् विष्ण॒वे ऽति॑ । \newline
18. विष्ण॒वे ऽत्यति॒ विष्ण॑वे॒ विष्ण॒वे ऽति॒ प्र प्राति॒ विष्ण॑वे॒ विष्ण॒वे ऽति॒ प्र । \newline
19. अति॒ प्र प्रात्यति॒ प्राय॑च्छ दयच्छ॒त् प्रात्यति॒ प्राय॑च्छत् । \newline
20. प्राय॑च्छ दयच्छ॒त् प्र प्राय॑च्छ॒त् तत् तद॑यच्छ॒त् प्र प्राय॑च्छ॒त् तत् । \newline
21. अ॒य॒च्छ॒त् तत् तद॑यच्छ दयच्छ॒त् तद् विष्णु॒र् विष्णु॒ स्त द॑यच्छ दयच्छ॒त् तद् विष्णुः॑ । \newline
22. तद् विष्णु॒र् विष्णु॒ स्तत् तद् विष्णुः॒ प्रति॒ प्रति॒ विष्णु॒ स्तत् तद् विष्णुः॒ प्रति॑ । \newline
23. विष्णुः॒ प्रति॒ प्रति॒ विष्णु॒र् विष्णुः॒ प्रत्य॑गृह्णा दगृह्णा॒त् प्रति॒ विष्णु॒र् विष्णुः॒ प्रत्य॑गृह्णात् । \newline
24. प्रत्य॑गृह्णा दगृह्णा॒त् प्रति॒ प्रत्य॑गृह्णा द॒स्मा स्व॒स्मा स्व॑गृह्णा॒त् प्रति॒ प्रत्य॑गृह्णा द॒स्मासु॑ । \newline
25. अ॒गृ॒ह्णा॒ द॒स्मा स्व॒स्मा स्व॑गृह्णा दगृह्णा द॒स्मा स्विन्द्र॒ इन्द्रो॒ ऽस्मा स्व॑गृह्णा दगृह्णा द॒स्मा स्विन्द्रः॑ । \newline
26. अ॒स्मा स्विन्द्र॒ इन्द्रो॒ ऽस्मा स्व॒स्मा स्विन्द्र॑ इन्द्रि॒य मि॑न्द्रि॒य मिन्द्रो॒ ऽस्मा स्व॒स्मा स्विन्द्र॑ इन्द्रि॒यम् । \newline
27. इन्द्र॑ इन्द्रि॒य मि॑न्द्रि॒य मिन्द्र॒ इन्द्र॑ इन्द्रि॒यम् द॑धातु दधा त्विन्द्रि॒य मिन्द्र॒ इन्द्र॑ इन्द्रि॒यम् द॑धातु । \newline
28. इ॒न्द्रि॒यम् द॑धातु दधा त्विन्द्रि॒य मि॑न्द्रि॒यम् द॑धा॒ त्वितीति॑ दधात्विन्द्रि॒य मि॑न्द्रि॒यम् द॑धा॒ त्विति॑ । \newline
29. द॒धा॒ त्वितीति॑ दधातु दधा॒ त्विति॒ यद् यदिति॑ दधातु दधा॒ त्विति॒ यत् । \newline
30. इति॒ यद् यदितीति॒ यत् त्रि स्त्रिर् यदितीति॒ यत् त्रिः । \newline
31. यत् त्रि स्त्रिर् यद् यत् त्रिः प्र प्र त्रिर् यद् यत् त्रिः प्र । \newline
32. त्रिः प्र प्र त्रि स्त्रिः प्राय॑च्छ॒ दय॑च्छ॒त् प्र त्रि स्त्रिः प्राय॑च्छत् । \newline
33. प्राय॑च्छ॒ दय॑च्छ॒त् प्र प्राय॑च्छ॒त् त्रि स्त्रि रय॑च्छ॒त् प्र प्राय॑च्छ॒त् त्रिः । \newline
34. अय॑च्छ॒त् त्रि स्त्रि रय॑च्छ॒ दय॑च्छ॒त् त्रिः प्र॒त्यगृ॑ह्णात् प्र॒त्यगृ॑ह्णा॒त् त्रि रय॑च्छ॒ दय॑च्छ॒त् त्रिः प्र॒त्यगृ॑ह्णात् । \newline
35. त्रिः प्र॒त्यगृ॑ह्णात् प्र॒त्यगृ॑ह्णा॒त् त्रि स्त्रिः प्र॒त्यगृ॑ह्णा॒त् तत् तत् प्र॒त्यगृ॑ह्णा॒त् त्रि स्त्रिः प्र॒त्यगृ॑ह्णा॒त् तत् । \newline
36. प्र॒त्यगृ॑ह्णा॒त् तत् तत् प्र॒त्यगृ॑ह्णात् प्र॒त्यगृ॑ह्णा॒त् तत् त्रि॒धातो᳚ स्त्रि॒धातो॒ स्तत् प्र॒त्यगृ॑ह्णात् प्र॒त्यगृ॑ह्णा॒त् तत् त्रि॒धातोः᳚ । \newline
37. प्र॒त्यगृ॑ह्णा॒दिति॑ प्रति - अगृ॑ह्णात् । \newline
38. तत् त्रि॒धातो᳚ स्त्रि॒धातो॒ स्तत् तत् त्रि॒धातो᳚ स्त्रिधातु॒त्वम् त्रि॑धातु॒त्वम् त्रि॒धातो॒ स्तत् तत् त्रि॒धातो᳚ स्त्रिधातु॒त्वम् । \newline
39. त्रि॒धातो᳚ स्त्रिधातु॒त्वम् त्रि॑धातु॒त्वम् त्रि॒धातो᳚ स्त्रि॒धातो᳚ स्त्रिधातु॒त्वं ॅयद् यत् त्रि॑धातु॒त्वम् त्रि॒धातो᳚ स्त्रि॒धातो᳚स्त्रिधातु॒त्वं ॅयत् । \newline
40. त्रि॒धातो॒रिति॑ त्रि - धातोः᳚ । \newline
41. त्रि॒धा॒तु॒त्वं ॅयद् यत् त्रि॑धातु॒त्वम् त्रि॑धातु॒त्वं ॅयद् विष्णु॒र् विष्णु॒र् यत् त्रि॑धातु॒त्वम् त्रि॑धातु॒त्वं ॅयद् विष्णुः॑ । \newline
42. त्रि॒धा॒तु॒त्वमिति॑ त्रिधातु - त्वम् । \newline
43. यद् विष्णु॒र् विष्णु॒र् यद् यद् विष्णु॑ र॒न्वति॑ष्ठता॒ न्वति॑ष्ठत॒ विष्णु॒र् यद् यद् विष्णु॑ र॒न्वति॑ष्ठत । \newline
44. विष्णु॑ र॒न्वति॑ष्ठता॒ न्वति॑ष्ठत॒ विष्णु॒र् विष्णु॑ र॒न्वति॑ष्ठत॒ विष्ण॑वे॒ विष्ण॑वे॒ ऽन्वति॑ष्ठत॒ विष्णु॒र् विष्णु॑ र॒न्वति॑ष्ठत॒ विष्ण॑वे । \newline
45. अ॒न्वति॑ष्ठत॒ विष्ण॑वे॒ विष्ण॑वे॒ ऽन्वति॑ष्ठता॒ न्वति॑ष्ठत॒ विष्ण॒वे ऽत्यति॒ विष्ण॑वे॒ ऽन्वति॑ष्ठता॒ न्वति॑ष्ठत॒ विष्ण॒वे ऽति॑ । \newline
46. अ॒न्वति॑ष्ठ॒तेत्य॑नु - अति॑ष्ठत । \newline
47. विष्ण॒वे ऽत्यति॒ विष्ण॑वे॒ विष्ण॒वे ऽति॒ प्र प्राति॒ विष्ण॑वे॒ विष्ण॒वे ऽति॒ प्र । \newline
48. अति॒ प्र प्रात्यति॒ प्राय॑च्छ॒ दय॑च्छ॒त् प्रात्यति॒ प्राय॑च्छत् । \newline
49. प्राय॑च्छ॒ दय॑च्छ॒त् प्र प्राय॑च्छ॒त् तस्मा॒त् तस्मा॒ दय॑च्छ॒त् प्र प्राय॑च्छ॒त् तस्मा᳚त् । \newline
50. अय॑च्छ॒त् तस्मा॒त् तस्मा॒ दय॑च्छ॒ दय॑च्छ॒त् तस्मा॑ दैन्द्रावैष्ण॒व मै᳚न्द्रावैष्ण॒वम् तस्मा॒ दय॑च्छ॒ दय॑च्छ॒त् तस्मा॑ दैन्द्रावैष्ण॒वम् । \newline
51. तस्मा॑ दैन्द्रावैष्ण॒व मै᳚न्द्रावैष्ण॒वम् तस्मा॒त् तस्मा॑ दैन्द्रावैष्ण॒वꣳ ह॒विर्. ह॒वि रै᳚न्द्रावैष्ण॒वम् तस्मा॒त् तस्मा॑ दैन्द्रावैष्ण॒वꣳ ह॒विः । \newline
52. ऐ॒न्द्रा॒वै॒ष्ण॒वꣳ ह॒विर्. ह॒वि रै᳚न्द्रावैष्ण॒व मै᳚न्द्रावैष्ण॒वꣳ ह॒विर् भ॑वति भवति ह॒वि रै᳚न्द्रावैष्ण॒व मै᳚न्द्रावैष्ण॒वꣳ ह॒विर् भ॑वति । \newline
53. ऐ॒न्द्रा॒वै॒ष्ण॒वमित्यै᳚न्द्रा - वै॒ष्ण॒वम् । \newline
54. ह॒विर् भ॑वति भवति ह॒विर्. ह॒विर् भ॑वति॒ यद् यद् भ॑वति ह॒विर्. ह॒विर् भ॑वति॒ यत् । \newline
55. भ॒व॒ति॒ यद् यद् भ॑वति भवति॒ यद् वै वै यद् भ॑वति भवति॒ यद् वै । \newline
56. यद् वै वै यद् यद् वा इ॒द मि॒दं ॅवै यद् यद् वा इ॒दम् । \newline
57. वा इ॒द मि॒दं ॅवै वा इ॒दम् किम् कि मि॒दं ॅवै वा इ॒दम् किम् । \newline
58. इ॒दम् किम् कि मि॒द मि॒दम् किम् च॑ च॒ कि मि॒द मि॒दम् किम् च॑ । \newline
59. किम् च॑ च॒ किम् किम् च॒ तत् तच् च॒ किम् किम् च॒ तत् । \newline
60. च॒ तत् तच् च॑ च॒ तद॑स्मा अस्मै॒ तच् च॑ च॒ तद॑स्मै । \newline
61. तद॑स्मा अस्मै॒ तत् तद॑स्मै॒ तत् तद॑स्मै॒ तत् तद॑स्मै॒ तत् । \newline
62. अ॒स्मै॒ तत् तद॑स्मा अस्मै॒ तत् प्र प्र तद॑स्मा अस्मै॒ तत् प्र । \newline
63. तत् प्र प्र तत् तत् प्राय॑च्छ दयच्छ॒त् प्र तत् तत् प्राय॑च्छत् । \newline
64. प्राय॑च्छ दयच्छ॒त् प्र प्राय॑च्छ॒ दृच॒ ऋचो॑ ऽयच्छ॒त् प्र प्राय॑च्छ॒ दृचः॑ । \newline
65. अ॒य॒च्छ॒ दृच॒ ऋचो॑ ऽयच्छ दयच्छ॒ दृचः॒ सामा॑नि॒ सामा॒न्यृचो॑ ऽयच्छ दयच्छ॒ दृचः॒ सामा॑नि । \newline
66. ऋचः॒ सामा॑नि॒ सामा॒न्यृच॒ ऋचः॒ सामा॑नि॒ यजूꣳ॑षि॒ यजूꣳ॑षि॒ सामा॒न्यृच॒ ऋचः॒ सामा॑नि॒ यजूꣳ॑षि । \newline
67. सामा॑नि॒ यजूꣳ॑षि॒ यजूꣳ॑षि॒ सामा॑नि॒ सामा॑नि॒ यजूꣳ॑षि स॒हस्रꣳ॑ स॒हस्रं॒ ॅयजूꣳ॑षि॒ सामा॑नि॒ सामा॑नि॒ यजूꣳ॑षि स॒हस्र᳚म् । \newline
68. यजूꣳ॑षि स॒हस्रꣳ॑ स॒हस्रं॒ ॅयजूꣳ॑षि॒ यजूꣳ॑षि स॒हस्रं॒ ॅवै वै स॒हस्रं॒ ॅयजूꣳ॑षि॒ यजूꣳ॑षि स॒हस्रं॒ ॅवै । \newline
69. स॒हस्रं॒ ॅवै वै स॒हस्रꣳ॑ स॒हस्रं॒ ॅवा अ॑स्मा अस्मै॒ वै स॒हस्रꣳ॑ स॒हस्रं॒ ॅवा अ॑स्मै । \newline
70. वा अ॑स्मा अस्मै॒ वै वा अ॑स्मै॒ तत् तद॑स्मै॒ वै वा अ॑स्मै॒ तत् । \newline
71. अ॒स्मै॒ तत् तद॑स्मा अस्मै॒ तत् प्र प्र तद॑स्मा अस्मै॒ तत् प्र । \newline
72. तत् प्र प्र तत् तत् प्राय॑च्छ दयच्छ॒त् प्र तत् तत् प्राय॑च्छत् । \newline
73. प्राय॑च्छ दयच्छ॒त् प्र प्राय॑च्छ॒त् तस्मा॒त् तस्मा॑ दयच्छ॒त् प्र प्राय॑च्छ॒त् तस्मा᳚त् । \newline
74. अ॒य॒च्छ॒त् तस्मा॒त् तस्मा॑ दयच्छ दयच्छ॒त् तस्मा᳚थ् स॒हस्र॑दक्षिणꣳ स॒हस्र॑दक्षिण॒म् तस्मा॑ दयच्छ दयच्छ॒त् तस्मा᳚थ् स॒हस्र॑दक्षिणम् । \newline
75. तस्मा᳚थ् स॒हस्र॑दक्षिणꣳ स॒हस्र॑दक्षिण॒म् तस्मा॒त् तस्मा᳚थ् स॒हस्र॑दक्षिणम् । \newline
76. स॒हस्र॑दक्षिण॒मिति॑ स॒हस्र॑ - द॒क्षि॒ण॒म् । \newline
\pagebreak
\markright{ TS 2.4.13.1  \hfill https://www.vedavms.in \hfill}

\section{ TS 2.4.13.1 }

\textbf{TS 2.4.13.1 } \newline
\textbf{Samhita Paata} \newline

दे॒वा वै रा॑ज॒न्या᳚-ज्जाय॑माना-दबिभयु॒-स्तम॒न्तरे॒व सन्तं॒ दाम्ना ऽपौ᳚म्भ॒न्थ् स वा ए॒षोऽपो᳚ब्धो जायते॒ यद्-रा॑ज॒न्यो॑ यद्वा ए॒षोऽन॑पोब्धो॒ जाये॑त वृ॒त्रान् घ्नꣳ श्च॑रे॒द्यं का॒मये॑त राज॒न्य॑मन॑पोब्धो जायेत वृ॒त्रान् घ्नꣳ श्च॑रे॒दिति॒ तस्मा॑ ए॒तमै᳚न्द्रा बार्.हस्प॒त्यं च॒रुं निर्व॑पेदै॒न्द्रो वै रा॑ज॒न्यो᳚ ब्रह्म॒ बृह॒स्पति॒ र्ब्रह्म॑णै॒वैनं॒ ( ) दाम्नो॒ऽप्ॐभ॑नान् मुञ्चति हिर॒ण्मयं॒ दाम॒ दक्षि॑णा सा॒क्षादे॒वैनं॒ दाम्नो॒ऽप्ॐभ॑नान् मुञ्चति ॥ \newline

\textbf{Pada Paata} \newline

दे॒वाः । वै । रा॒ज॒न्या᳚त् । जाय॑मानात् ।   अ॒बि॒भ॒युः॒ । तम् । अ॒न्तः । ए॒व । सन्त᳚म् ।   दाम्ना᳚ । अपेति॑ । औ॒भं॒न्न् । सः । वै । ए॒षः । अपो᳚ब्ध॒ इत्यप॑- उ॒ब्धः॒ । जा॒य॒ते॒ । यत् । रा॒ज॒न्यः॑ । यत् । वै । ए॒षः । अन॑पोब्ध॒ इत्यन॑प - उ॒ब्धः॒ । जाये॑त । वृ॒त्रान् । घ्नन्न् । च॒रे॒त् । यम् । का॒मये॑त । रा॒ज॒न्य᳚म् । अन॑पोब्ध॒ इत्यन॑प - उ॒ब्धः॒ । जा॒ये॒त॒ । वृ॒त्रान् । घ्नन्न् । च॒रे॒त् । इति॑ । तस्मै᳚ । ए॒तम् । ऐ॒न्द्रा॒बा॒र्.॒ह॒स्प॒त्यमित्यै᳚न्द्रा-बा॒र्.॒ह॒स्प॒त्यम् । च॒रुम् । निरिति॑ । व॒पे॒त् ।   ऐ॒न्द्रः । वै । रा॒ज॒न्यः॑ । ब्रह्म॑ । बृह॒स्पतिः॑ । ब्रह्म॑णा । ए॒व । ए॒न॒म् ( ) । दाम्नः॑ । अ॒प्ॐभ॑ना॒दित्य॑प-उभं॑नात् । मु॒ञ्च॒ति॒ । हि॒र॒ण्मय᳚म् । दाम॑ । दक्षि॑णा । सा॒क्षादिति॑ स-अ॒क्षात् । ए॒व । ए॒न॒म् । दाम्नः॑ । अ॒प्ॐभ॑ना॒दित्य॑प - उभं॑नात् । मु॒ञ्च॒ति॒ ॥  \newline


\textbf{Krama Paata} \newline

दे॒वा वै । वै रा॑ज॒न्या᳚त् । रा॒ज॒न्या᳚ज्जाय॑मानात् । जाय॑मानादबिभयुः । अ॒बि॒भ॒यु॒स्तम् । तम॒न्तः । अ॒न्तरे॒व । ए॒व सन्त᳚म् । सन्त॒म् दाम्ना᳚ । दाम्नाऽप॑ । अपौ᳚म्भन्न् । औ॒म्भ॒न्थ् सः । स वै । वा ए॒षः । ए॒षोऽपो᳚ब्धः । अपो᳚ब्धो जायते । अपो᳚ब्ध॒ इत्यप॑ - उ॒ब्धः॒ । जा॒य॒ते॒ यत् । यद् रा॑ज॒न्यः॑ । रा॒ज॒न्यो॑ यत् । यद् वै । वा ए॒षः । ए॒षोऽन॑पोब्धः । अन॑पोब्धो॒ जाये॑त । अन॑पोब्ध॒ इत्यन॑प - उ॒ब्धः॒ । जाये॑त वृ॒त्रान् । वृ॒त्रान् घ्नन्न् । घ्नꣳश्च॑रेत् । च॒रे॒द् यम् । यं का॒मये॑त । का॒मये॑त राज॒न्य᳚म् । रा॒ज॒न्य॑मन॑पोब्धः । अन॑पोब्धो जायेत । अन॑पोब्ध॒ इत्यन॑प - उ॒ब्धः॒ । जा॒ये॒त॒ वृ॒त्रान् । वृ॒त्रान् घ्नन् । घ्नꣳश्च॑रेत् । च॒रे॒दिति॑ । इति॒ तस्मै᳚ । तस्मा॑ ए॒तम् । ए॒तमै᳚न्द्राबार्.हस्प॒त्यम् । ऐ॒न्द्रा॒बा॒र्॒.ह॒स्प॒त्यं च॒रुम् । ऐ॒न्द्रा॒बा॒र्॒.ह॒स्प॒त्यमित्यै᳚न्द्रा - बा॒र्॒.ह॒स्प॒त्यम् । च॒रुम् निः । निर् व॑पेत् । व॒पे॒दै॒न्द्रः । ऐ॒न्द्रो वै । वै रा॑ज॒न्यः॑ । रा॒ज॒न्यो᳚ ब्रह्म॑ । ब्रह्म॒ बृह॒स्पतिः॑ । बृह॒स्पति॒र् ब्रह्म॑णा । ब्रह्म॑णै॒व । ए॒वैन᳚म् । ए॒नं॒ दाम्नः॑ । दाम्नो॒ ऽपोम्भ॑नात् । अ॒पोम्भ॑नान् मुञ्चति । अ॒पोम्भ॑ना॒दित्य॑प - उम्भ॑नात् । मु॒ञ्च॒ति॒ हि॒र॒ण्मय᳚म् । हि॒र॒ण्मयं॒ दाम॑ । दाम॒ दक्षि॑णा । दक्षि॑णा सा॒क्षात् । सा॒क्षादे॒व । सा॒क्षादिति॑ स - अ॒क्षात् । ए॒वैन᳚म् ( ) । ए॒नं॒ दाम्नः॑ । दाम्नो॒ ऽपोम्भ॑नात् । अ॒पोम्भ॑नान् मुञ्चति । अ॒पोम्भ॑ना॒दित्य॑प - उम्भ॑नात् । 
मु॒ञ्च॒तीति॑ मुञ्चति । \newline

\textbf{Jatai Paata} \newline

1. दे॒वा वै वै दे॒वा दे॒वा वै । \newline
2. वै रा॑ज॒न्या᳚द् राज॒न्या᳚द् वै वै रा॑ज॒न्या᳚त् । \newline
3. रा॒ज॒न्या᳚ज् जाय॑माना॒ज् जाय॑मानाद् राज॒न्या᳚द् राज॒न्या᳚ज् जाय॑मानात् । \newline
4. जाय॑माना दबिभयु रबिभयु॒र् जाय॑माना॒ज् जाय॑माना दबिभयुः । \newline
5. अ॒बि॒भ॒यु॒ स्तम् त म॑बिभयु रबिभयु॒ स्तम् । \newline
6. त म॒न्त र॒न्त स्तम् त म॒न्तः । \newline
7. अ॒न्त रे॒वैवान्त र॒न्त रे॒व । \newline
8. ए॒व सन्तꣳ॒॒ सन्त॑ मे॒वैव सन्त᳚म् । \newline
9. सन्त॒म् दाम्ना॒ दाम्ना॒ सन्तꣳ॒॒ सन्त॒म् दाम्ना᳚ । \newline
10. दाम्ना ऽपाप॒ दाम्ना॒ दाम्ना ऽप॑ । \newline
11. अपौ᳚म्भन् नौम्भ॒न् नपापौ᳚म्भन्न् । \newline
12. औ॒म्भ॒न् थ्स स औ᳚म्भन् नौम्भ॒न् थ्सः । \newline
13. स वै वै स स वै । \newline
14. वा ए॒ष ए॒ष वै वा ए॒षः । \newline
15. ए॒षो ऽपो॒ब्धो ऽपो᳚ब्ध ए॒ष ए॒षो ऽपो᳚ब्धः । \newline
16. अपो᳚ब्धो जायते जाय॒ते ऽपो॒ब्धो ऽपो᳚ब्धो जायते । \newline
17. अपो᳚ब्ध॒ इत्यप॑ - उ॒ब्धः॒ । \newline
18. जा॒य॒ते॒ यद् यज् जा॑यते जायते॒ यत् । \newline
19. यद् रा॑ज॒न्यो॑ राज॒न्यो॑ यद् यद् रा॑ज॒न्यः॑ । \newline
20. रा॒ज॒न्यो॑ यद् यद् रा॑ज॒न्यो॑ राज॒न्यो॑ यत् । \newline
21. यद् वै वै यद् यद् वै । \newline
22. वा ए॒ष ए॒ष वै वा ए॒षः । \newline
23. ए॒षो ऽन॑पो॒ब्धो ऽन॑पोब्ध ए॒ष ए॒षो ऽन॑पोब्धः । \newline
24. अन॑पोब्धो॒ जाये॑त॒ जाये॒ता न॑पो॒ब्धो ऽन॑पोब्धो॒ जाये॑त । \newline
25. अन॑पोब्ध॒ इत्यन॑प - उ॒ब्धः॒ । \newline
26. जाये॑त वृ॒त्रान् वृ॒त्रान् जाये॑त॒ जाये॑त वृ॒त्रान् । \newline
27. वृ॒त्रान् घ्नन् घ्नन् वृ॒त्रान् वृ॒त्रान् घ्नन्न् । \newline
28. घ्नꣳ श्च॑रेच् चरे॒द् घ्नन् घ्नꣳ श्च॑रेत् । \newline
29. च॒रे॒द् यं ॅयम् च॑रेच् चरे॒द् यम् । \newline
30. यम् का॒मये॑त का॒मये॑त॒ यं ॅयम् का॒मये॑त । \newline
31. का॒मये॑त राज॒न्यꣳ॑ राज॒न्य॑म् का॒मये॑त का॒मये॑त राज॒न्य᳚म् । \newline
32. रा॒ज॒न्य॑ मन॑पो॒ब्धो ऽन॑पोब्धो राज॒न्यꣳ॑ राज॒न्य॑ मन॑पोब्धः । \newline
33. अन॑पोब्धो जायेत जाये॒ता न॑पो॒ब्धो ऽन॑पोब्धो जायेत । \newline
34. अन॑पोब्ध॒ इत्यन॑प - उ॒ब्धः॒ । \newline
35. जा॒ये॒त॒ वृ॒त्रान् वृ॒त्रान् जा॑येत जायेत वृ॒त्रान् । \newline
36. वृ॒त्रान् घ्नन् घ्नन् वृ॒त्रान् वृ॒त्रान् घ्नन्न् । \newline
37. घ्नꣳ श्च॑रेच् चरे॒द् घ्नन् घ्नꣳ श्च॑रेत् । \newline
38. च॒रे॒दितीति॑ चरेच् चरे॒दिति॑ । \newline
39. इति॒ तस्मै॒ तस्मा॒ इतीति॒ तस्मै᳚ । \newline
40. तस्मा॑ ए॒त मे॒तम् तस्मै॒ तस्मा॑ ए॒तम् । \newline
41. ए॒त मै᳚न्द्राबार्.हस्प॒त्य मै᳚न्द्राबार्.हस्प॒त्य मे॒त मे॒त मै᳚न्द्राबार्.हस्प॒त्यम् । \newline
42. ऐ॒न्द्रा॒बा॒र्॒.ह॒स्प॒त्यम् च॒रुम् च॒रु मै᳚न्द्राबार्.हस्प॒त्य मै᳚न्द्राबार्.हस्प॒त्यम् च॒रुम् । \newline
43. ऐ॒न्द्रा॒बा॒र्॒.ह॒स्प॒त्यमित्यै᳚न्द्रा - बा॒र्॒.ह॒स्प॒त्यम् । \newline
44. च॒रुम् निर् णिश्च॒रुम् च॒रुम् निः । \newline
45. निर् व॑पेद् वपे॒न् निर् णिर् व॑पेत् । \newline
46. व॒पे॒ दै॒न्द्र ऐ॒न्द्रो व॑पेद् वपे दै॒न्द्रः । \newline
47. ऐ॒न्द्रो वै वा ऐ॒न्द्र ऐ॒न्द्रो वै । \newline
48. वै रा॑ज॒न्यो॑ राज॒न्यो॑ वै वै रा॑ज॒न्यः॑ । \newline
49. रा॒ज॒न्यो᳚ ब्रह्म॒ ब्रह्म॑ राज॒न्यो॑ राज॒न्यो᳚ ब्रह्म॑ । \newline
50. ब्रह्म॒ बृह॒स्पति॒र् बृह॒स्पति॒र् ब्रह्म॒ ब्रह्म॒ बृह॒स्पतिः॑ । \newline
51. बृह॒स्पति॒र् ब्रह्म॑णा॒ ब्रह्म॑णा॒ बृह॒स्पति॒र् बृह॒स्पति॒र् ब्रह्म॑णा । \newline
52. ब्रह्म॑णै॒वैव ब्रह्म॑णा॒ ब्रह्म॑णै॒व । \newline
53. ए॒वैन॑ मेन मे॒वैवैन᳚म् । \newline
54. ए॒न॒म् दाम्नो॒ दाम्न॑ एन मेन॒म् दाम्नः॑ । \newline
55. दाम्नो॒ ऽपोम्भ॑ना द॒पोम्भ॑ना॒द् दाम्नो॒ दाम्नो॒ ऽपोम्भ॑नात् । \newline
56. अ॒पोम्भ॑नान् मुञ्चति मुञ्च त्य॒पोम्भ॑ना द॒पोम्भ॑नान् मुञ्चति । \newline
57. अ॒प्ॐभ॑ना॒दित्य॑प - उंभ॑नात् । \newline
58. मु॒ञ्च॒ति॒ हि॒र॒ण्मयꣳ॑ हिर॒ण्मय॑म् मुञ्चति मुञ्चति हिर॒ण्मय᳚म् । \newline
59. हि॒र॒ण्मय॒म् दाम॒ दाम॑ हिर॒ण्मयꣳ॑ हिर॒ण्मय॒म् दाम॑ । \newline
60. दाम॒ दक्षि॑णा॒ दक्षि॑णा॒ दाम॒ दाम॒ दक्षि॑णा । \newline
61. दक्षि॑णा सा॒क्षाथ् सा॒क्षाद् दक्षि॑णा॒ दक्षि॑णा सा॒क्षात् । \newline
62. सा॒क्षा दे॒वैव सा॒क्षाथ् सा॒क्षा दे॒व । \newline
63. सा॒क्षादिति॑ स - अ॒क्षात् । \newline
64. ए॒वैन॑ मेन मे॒वैवैन᳚म् । \newline
65. ए॒न॒म् दाम्नो॒ दाम्न॑ एन मेन॒म् दाम्नः॑ । \newline
66. दाम्नो॒ ऽपोम्भ॑ना द॒पोम्भ॑ना॒द् दाम्नो॒ दाम्नो॒ ऽपोम्भ॑नात् । \newline
67. अ॒पोम्भ॑नान् मुञ्चति मुञ्च त्य॒पोम्भ॑ना द॒पोम्भ॑नान् मुञ्चति । \newline
68. अ॒प्ॐभ॑ना॒दित्य॑प - उंभ॑नात् । \newline
69. मु॒ञ्च॒तीति॑ मुञ्चति । \newline

\textbf{Ghana Paata } \newline

1. दे॒वा वै वै दे॒वा दे॒वा वै रा॑ज॒न्या᳚द् राज॒न्या᳚द् वै दे॒वा दे॒वा वै रा॑ज॒न्या᳚त् । \newline
2. वै रा॑ज॒न्या᳚द् राज॒न्या᳚द् वै वै रा॑ज॒न्या᳚ज् जाय॑माना॒ज् जाय॑मानाद् राज॒न्या᳚द् वै वै रा॑ज॒न्या᳚ज् जाय॑मानात् । \newline
3. रा॒ज॒न्या᳚ज् जाय॑माना॒ज् जाय॑मानाद् राज॒न्या᳚द् राज॒न्या᳚ज् जाय॑माना दबिभयु रबिभयु॒र् जाय॑मानाद् राज॒न्या᳚द् राज॒न्या᳚ज् जाय॑माना दबिभयुः । \newline
4. जाय॑माना दबिभयु रबिभयु॒र् जाय॑माना॒ज् जाय॑माना दबिभयु॒ स्तम् त म॑बिभयु॒र् जाय॑माना॒ज् जाय॑माना दबिभयु॒ स्तम् । \newline
5. अ॒बि॒भ॒यु॒ स्तम् त म॑बिभयु रबिभयु॒ स्त म॒न्त र॒न्त स्त म॑बिभयु रबिभयु॒ स्त म॒न्तः । \newline
6. त म॒न्त र॒न्त स्तम् त म॒न्त रे॒वैवान्त स्तम् त म॒न्त रे॒व । \newline
7. अ॒न्त रे॒वैवान्त र॒न्त रे॒व सन्तꣳ॒॒ सन्त॑ मे॒वान्त र॒न्त रे॒व सन्त᳚म् । \newline
8. ए॒व सन्तꣳ॒॒ सन्त॑ मे॒वैव सन्त॒म् दाम्ना॒ दाम्ना॒ सन्त॑ मे॒वैव सन्त॒म् दाम्ना᳚ । \newline
9. सन्त॒म् दाम्ना॒ दाम्ना॒ सन्तꣳ॒॒ सन्त॒म् दाम्ना ऽपाप॒ दाम्ना॒ सन्तꣳ॒॒ सन्त॒म् दाम्ना ऽप॑ । \newline
10. दाम्ना ऽपाप॒ दाम्ना॒ दाम्ना ऽपौ᳚म्भन् नौम्भ॒न् नप॒ दाम्ना॒ दाम्ना ऽपौ᳚म्भन्न् । \newline
11. अपौ᳚म्भन् नौम्भ॒न् नपापौ᳚म्भ॒न् थ्स स औ᳚म्भ॒न् नपापौ᳚म्भ॒न् थ्सः । \newline
12. औ॒म्भ॒न् थ्स स औ᳚म्भन् नौम्भ॒न् थ्स वै वै स औ᳚म्भन् नौम्भ॒न् थ्स वै । \newline
13. स वै वै स स वा ए॒ष ए॒ष वै स स वा ए॒षः । \newline
14. वा ए॒ष ए॒ष वै वा ए॒षो ऽपो॒ब्धो ऽपो᳚ब्ध ए॒ष वै वा ए॒षो ऽपो᳚ब्धः । \newline
15. ए॒षो ऽपो॒ब्धो ऽपो᳚ब्ध ए॒ष ए॒षो ऽपो᳚ब्धो जायते जाय॒ते ऽपो᳚ब्ध ए॒ष ए॒षो ऽपो᳚ब्धो जायते । \newline
16. अपो᳚ब्धो जायते जाय॒ते ऽपो॒ब्धो ऽपो᳚ब्धो जायते॒ यद् यज् जा॑य॒ते ऽपो॒ब्धो ऽपो᳚ब्धो जायते॒ यत् । \newline
17. अपो᳚ब्ध॒ इत्यप॑ - उ॒ब्धः॒ । \newline
18. जा॒य॒ते॒ यद् यज् जा॑यते जायते॒ यद् रा॑ज॒न्यो॑ राज॒न्यो॑ यज् जा॑यते जायते॒ यद् रा॑ज॒न्यः॑ । \newline
19. यद् रा॑ज॒न्यो॑ राज॒न्यो॑ यद् यद् रा॑ज॒न्यो॑ यद् यद् रा॑ज॒न्यो॑ यद् यद् रा॑ज॒न्यो॑ यत् । \newline
20. रा॒ज॒न्यो॑ यद् यद् रा॑ज॒न्यो॑ राज॒न्यो॑ यद् वै वै यद् रा॑ज॒न्यो॑ राज॒न्यो॑ यद् वै । \newline
21. यद् वै वै यद् यद् वा ए॒ष ए॒ष वै यद् यद् वा ए॒षः । \newline
22. वा ए॒ष ए॒ष वै वा ए॒षो ऽन॑पो॒ब्धो ऽन॑पोब्ध ए॒ष वै वा ए॒षो ऽन॑पोब्धः । \newline
23. ए॒षो ऽन॑पो॒ब्धो ऽन॑पोब्ध ए॒ष ए॒षो ऽन॑पोब्धो॒ जाये॑त॒ जाये॒ता न॑पोब्ध ए॒ष ए॒षो ऽन॑पोब्धो॒ जाये॑त । \newline
24. अन॑पोब्धो॒ जाये॑त॒ जाये॒ता न॑पो॒ब्धो ऽन॑पोब्धो॒ जाये॑त वृ॒त्रान् वृ॒त्रान् जाये॒ता न॑पो॒ब्धो ऽन॑पोब्धो॒ जाये॑त वृ॒त्रान् । \newline
25. अन॑पोब्ध॒ इत्यन॑प - उ॒ब्धः॒ । \newline
26. जाये॑त वृ॒त्रान् वृ॒त्रान् जाये॑त॒ जाये॑त वृ॒त्रान् घ्नन् घ्नन् वृ॒त्रान् जाये॑त॒ जाये॑त वृ॒त्रान् घ्नन्न् । \newline
27. वृ॒त्रान् घ्नन् घ्नन् वृ॒त्रान् वृ॒त्रान् घ्नꣳ श्च॑रेच् चरे॒द् घ्नन् वृ॒त्रान् वृ॒त्रान् घ्नꣳ श्च॑रेत् । \newline
28. घ्नꣳ श्च॑रेच् चरे॒द् घ्नन् घ्नꣳ श्च॑रे॒द् यं ॅयम् च॑रे॒द् घ्नन् घ्नꣳ श्च॑रे॒द् यम् । \newline
29. च॒रे॒द् यं ॅयम् च॑रेच् चरे॒द् यम् का॒मये॑त का॒मये॑त॒ यम् च॑रेच् चरे॒द् यम् का॒मये॑त । \newline
30. यम् का॒मये॑त का॒मये॑त॒ यं ॅयम् का॒मये॑त राज॒न्यꣳ॑ राज॒न्य॑म् का॒मये॑त॒ यं ॅयम् का॒मये॑त राज॒न्य᳚म् । \newline
31. का॒मये॑त राज॒न्यꣳ॑ राज॒न्य॑म् का॒मये॑त का॒मये॑त राज॒न्य॑ मन॑पो॒ब्धो ऽन॑पोब्धो राज॒न्य॑म् का॒मये॑त का॒मये॑त राज॒न्य॑ मन॑पोब्धः । \newline
32. रा॒ज॒न्य॑ मन॑पो॒ब्धो ऽन॑पोब्धो राज॒न्यꣳ॑ राज॒न्य॑ मन॑पोब्धो जायेत जाये॒ता न॑पोब्धो राज॒न्यꣳ॑ राज॒न्य॑ मन॑पोब्धो जायेत । \newline
33. अन॑पोब्धो जायेत जाये॒ता न॑पो॒ब्धो ऽन॑पोब्धो जायेत वृ॒त्रान् वृ॒त्रान् जा॑ये॒ता न॑पो॒ब्धो ऽन॑पोब्धो जायेत वृ॒त्रान् । \newline
34. अन॑पोब्ध॒ इत्यन॑प - उ॒ब्धः॒ । \newline
35. जा॒ये॒त॒ वृ॒त्रान् वृ॒त्रान् जा॑येत जायेत वृ॒त्रान् घ्नन् घ्नन् वृ॒त्रान् जा॑येत जायेत वृ॒त्रान् घ्नन्न् । \newline
36. वृ॒त्रान् घ्नन् घ्नन् वृ॒त्रान् वृ॒त्रान् घ्नꣳ श्च॑रेच् चरे॒द् घ्नन् वृ॒त्रान् वृ॒त्रान् घ्नꣳ श्च॑रेत् । \newline
37. घ्नꣳ श्च॑रेच् चरे॒द् घ्नन् घ्नꣳ श्च॑रे॒दितीति॑ चरे॒द् घ्नन् घ्नꣳ श्च॑रे॒दिति॑ । \newline
38. च॒रे॒दितीति॑ चरेच् चरे॒दिति॒ तस्मै॒ तस्मा॒ इति॑ चरेच् चरे॒दिति॒ तस्मै᳚ । \newline
39. इति॒ तस्मै॒ तस्मा॒ इतीति॒ तस्मा॑ ए॒त मे॒तम् तस्मा॒ इतीति॒ तस्मा॑ ए॒तम् । \newline
40. तस्मा॑ ए॒त मे॒तम् तस्मै॒ तस्मा॑ ए॒त मै᳚न्द्राबार्.हस्प॒त्य मै᳚न्द्राबार्.हस्प॒त्य मे॒तम् तस्मै॒ तस्मा॑ ए॒त मै᳚न्द्राबार्.हस्प॒त्यम् । \newline
41. ए॒त मै᳚न्द्राबार्.हस्प॒त्य मै᳚न्द्राबार्.हस्प॒त्य मे॒त मे॒त मै᳚न्द्राबार्.हस्प॒त्यम् च॒रुम् च॒रु मै᳚न्द्राबार्.हस्प॒त्य मे॒त मे॒त मै᳚न्द्राबार्.हस्प॒त्यम् च॒रुम् । \newline
42. ऐ॒न्द्रा॒बा॒र्॒.ह॒स्प॒त्यम् च॒रुम् च॒रु मै᳚न्द्राबार्.हस्प॒त्य मै᳚न्द्राबार्.हस्प॒त्यम् च॒रुम् निर् णिश्च॒रु मै᳚न्द्राबार्.हस्प॒त्य मै᳚न्द्राबार्.हस्प॒त्यम् च॒रुम् निः । \newline
43. ऐ॒न्द्रा॒बा॒र्॒.ह॒स्प॒त्यमित्यै᳚न्द्रा - बा॒र्॒.ह॒स्प॒त्यम् । \newline
44. च॒रुम् निर् णिश्च॒रुम् च॒रुम् निर् व॑पेद् वपे॒न् निश्च॒रुम् च॒रुम् निर् व॑पेत् । \newline
45. निर् व॑पेद् वपे॒न् निर् णिर् व॑पेदै॒न्द्र ऐ॒न्द्रो व॑पे॒न् निर् णिर् व॑पेदै॒न्द्रः । \newline
46. व॒पे॒दै॒न्द्र ऐ॒न्द्रो व॑पेद् वपेदै॒न्द्रो वै वा ऐ॒न्द्रो व॑पेद् वपेदै॒न्द्रो वै । \newline
47. ऐ॒न्द्रो वै वा ऐ॒न्द्र ऐ॒न्द्रो वै रा॑ज॒न्यो॑ राज॒न्यो॑ वा ऐ॒न्द्र ऐ॒न्द्रो वै रा॑ज॒न्यः॑ । \newline
48. वै रा॑ज॒न्यो॑ राज॒न्यो॑ वै वै रा॑ज॒न्यो᳚ ब्रह्म॒ ब्रह्म॑ राज॒न्यो॑ वै वै रा॑ज॒न्यो᳚ ब्रह्म॑ । \newline
49. रा॒ज॒न्यो᳚ ब्रह्म॒ ब्रह्म॑ राज॒न्यो॑ राज॒न्यो᳚ ब्रह्म॒ बृह॒स्पति॒र् बृह॒स्पति॒र् ब्रह्म॑ राज॒न्यो॑ राज॒न्यो᳚ ब्रह्म॒ बृह॒स्पतिः॑ । \newline
50. ब्रह्म॒ बृह॒स्पति॒र् बृह॒स्पति॒र् ब्रह्म॒ ब्रह्म॒ बृह॒स्पति॒र् ब्रह्म॑णा॒ ब्रह्म॑णा॒ बृह॒स्पति॒र् ब्रह्म॒ ब्रह्म॒ बृह॒स्पति॒र् ब्रह्म॑णा । \newline
51. बृह॒स्पति॒र् ब्रह्म॑णा॒ ब्रह्म॑णा॒ बृह॒स्पति॒र् बृह॒स्पति॒र् ब्रह्म॑णै॒वैव ब्रह्म॑णा॒ बृह॒स्पति॒र् बृह॒स्पति॒र् ब्रह्म॑णै॒व । \newline
52. ब्रह्म॑णै॒वैव ब्रह्म॑णा॒ ब्रह्म॑णै॒वैन॑ मेन मे॒व ब्रह्म॑णा॒ ब्रह्म॑णै॒वैन᳚म् । \newline
53. ए॒वैन॑ मेन मे॒वैवैन॒म् दाम्नो॒ दाम्न॑ एन मे॒वैवैन॒म् दाम्नः॑ । \newline
54. ए॒न॒म् दाम्नो॒ दाम्न॑ एन मेन॒म् दाम्नो॒ ऽपोम्भ॑ना द॒पोम्भ॑ना॒द् दाम्न॑ एन मेन॒म् दाम्नो॒ ऽपोम्भ॑नात् । \newline
55. दाम्नो॒ ऽपोम्भ॑ना द॒पोम्भ॑ना॒द् दाम्नो॒ दाम्नो॒ ऽपोम्भ॑नान् मुञ्चति मुञ्च त्य॒पोम्भ॑ना॒द् दाम्नो॒ दाम्नो॒ ऽपोम्भ॑नान् मुञ्चति । \newline
56. अ॒पोम्भ॑नान् मुञ्चति मुञ्च त्य॒पोम्भ॑ना द॒पोम्भ॑नान् मुञ्चति हिर॒ण्मयꣳ॑ हिर॒ण्मय॑म् मुञ्च त्य॒पोम्भ॑ना द॒पोम्भ॑नान् मुञ्चति हिर॒ण्मय᳚म् । \newline
57. अ॒प्ॐभ॑ना॒दित्य॑प - उंभ॑नात् । \newline
58. मु॒ञ्च॒ति॒ हि॒र॒ण्मयꣳ॑ हिर॒ण्मय॑म् मुञ्चति मुञ्चति हिर॒ण्मय॒म् दाम॒ दाम॑ हिर॒ण्मय॑म् मुञ्चति मुञ्चति हिर॒ण्मय॒म् दाम॑ । \newline
59. हि॒र॒ण्मय॒म् दाम॒ दाम॑ हिर॒ण्मयꣳ॑ हिर॒ण्मय॒म् दाम॒ दक्षि॑णा॒ दक्षि॑णा॒ दाम॑ हिर॒ण्मयꣳ॑ हिर॒ण्मय॒म् दाम॒ दक्षि॑णा । \newline
60. दाम॒ दक्षि॑णा॒ दक्षि॑णा॒ दाम॒ दाम॒ दक्षि॑णा सा॒क्षाथ् सा॒क्षाद् दक्षि॑णा॒ दाम॒ दाम॒ दक्षि॑णा सा॒क्षात् । \newline
61. दक्षि॑णा सा॒क्षाथ् सा॒क्षाद् दक्षि॑णा॒ दक्षि॑णा सा॒क्षा दे॒वैव सा॒क्षाद् दक्षि॑णा॒ दक्षि॑णा सा॒क्षा दे॒व । \newline
62. सा॒क्षा दे॒वैव सा॒क्षाथ् सा॒क्षा दे॒वैन॑ मेन मे॒व सा॒क्षाथ् सा॒क्षा दे॒वैन᳚म् । \newline
63. सा॒क्षादिति॑ स - अ॒क्षात् । \newline
64. ए॒वैन॑ मेन मे॒वैवैन॒म् दाम्नो॒ दाम्न॑ एन मे॒वैवैन॒म् दाम्नः॑ । \newline
65. ए॒न॒म् दाम्नो॒ दाम्न॑ एन मेन॒म् दाम्नो॒ ऽपोम्भ॑ना द॒पोम्भ॑ना॒द् दाम्न॑ एन मेन॒म् दाम्नो॒ ऽपोम्भ॑नात् । \newline
66. दाम्नो॒ ऽपोम्भ॑ना द॒पोम्भ॑ना॒द् दाम्नो॒ दाम्नो॒ ऽपोम्भ॑नान् मुञ्चति मुञ्च त्य॒पोम्भ॑ना॒द् दाम्नो॒ दाम्नो॒ ऽपोम्भ॑नान् मुञ्चति । \newline
67. अ॒पोम्भ॑नान् मुञ्चति मुञ्च त्य॒पोम्भ॑ना द॒पोम्भ॑नान् मुञ्चति । \newline
68. अ॒प्ॐभ॑ना॒दित्य॑प - उंभ॑नात् । \newline
69. मु॒ञ्च॒तीति॑ मुञ्चति । \newline
\pagebreak
\markright{ TS 2.4.14.1  \hfill https://www.vedavms.in \hfill}

\section{ TS 2.4.14.1 }

\textbf{TS 2.4.14.1 } \newline
\textbf{Samhita Paata} \newline

नवो॑नवो भवति॒ जाय॑मा॒नोऽह्नां᳚ के॒तुरु॒षसा॑ मे॒त्यग्रे᳚ । भा॒गं दे॒वेभ्यो॒ विद॑धात्या॒यन् प्रच॒न्द्रमा᳚स्तिरति दी॒र्घमायुः॑ ॥ यमा॑दि॒त्या अꣳ॒॒शुमा᳚प्या॒यय॑न्ति॒ यमक्षि॑त॒-मक्षि॑तयः॒ पिब॑न्ति । तेन॑ नो॒ राजा॒ वरु॑णो॒ बृह॒स्पति॒रा प्या॑ययन्तु॒ भुव॑नस्य गो॒पाः ॥प्राच्यां᳚ दि॒शि त्वमि॑न्द्रासि॒ राजो॒तोदी᳚च्यां ॅवृत्रहन् वृत्र॒हाऽसि॑ । यत्र॒ यन्ति॑ स्रो॒त्यास्त - [  ] \newline

\textbf{Pada Paata} \newline

नवो॑नव॒ इति॒ नवः॑ - न॒वः॒ । भ॒व॒ति॒ । जाय॑मानः । अह्ना᳚म् । के॒तुः । उ॒षसा᳚म् । ए॒ति॒ । अग्रे᳚ ॥ भा॒गम् । दे॒वेभ्यः॑ । वीति॑ । द॒धा॒ति॒ । आ॒यन्नित्या᳚ - यन्न् । प्रेति॑ । च॒न्द्रमाः᳚ । ति॒र॒ति॒ । दी॒र्घम् । आयुः॑ ॥ यम् । आ॒दि॒त्याः । अꣳ॒॒शुम् । आ॒प्या॒यय॒न्तीत्या᳚ - प्या॒यय॑न्ति । यम् । अक्षि॑तम् । अक्षि॑तयः । पिब॑न्ति ॥ तेन॑ । नः॒ । राजा᳚ । वरु॑णः । बृह॒स्पतिः॑ । एति॑ । प्या॒य॒य॒न्तु॒ । भुव॑नस्य । गो॒पा इति॑ गो - पाः ॥ प्राच्या᳚म् । दि॒शि । त्वम् । इ॒न्द्र॒ । अ॒सि॒ । राजा᳚ । उ॒त । उदी᳚च्याम् । वृ॒त्र॒ह॒न्निति॑ वृत्र - ह॒न्न् । वृ॒त्र॒हेति॑ वृत्र - हा । अ॒सि॒ ॥ यत्र॑ । यन्ति॑ । स्रो॒त्याः । तत् ।  \newline


\textbf{Krama Paata} \newline

नवो॑नवो भवति । नवो॑नव॒ इति॒ नवः॑ - न॒वः॒ । भ॒व॒ति॒ जाय॑मानः । जाय॑मा॒नो ऽह्ना᳚म् । अह्ना᳚म् के॒तुः । के॒तुरु॒षसा᳚म् । उ॒षसा॑मेति । ए॒त्यग्रे᳚ । अग्र॒ इत्यग्रे᳚ ॥ भा॒गम् दे॒वेभ्यः॑ । दे॒वेभ्यो॒ वि । वि द॑धाति । द॒धा॒त्या॒यन्न् । आ॒यन् प्र । आ॒यन्नित्या᳚ - यन्न् । प्र च॒न्द्रमाः᳚ । च॒न्द्रमा᳚स्तिरति । ति॒र॒ति॒ दी॒र्घम् । दी॒र्घमायुः॑ । आयु॒रित्यायुः॑ ॥ यमा॑दि॒त्याः । आ॒दि॒त्या अꣳ॒॒शुम् । अꣳ॒॒शुमा᳚प्या॒यय॑न्ति । आ॒प्या॒यय॑न्ति॒ यम् । आ॒प्या॒यय॒न्तीत्या᳚ - प्या॒यय॑न्ति । यमक्षि॑तम् । अक्षि॑त॒मक्षि॑तयः । अक्षि॑तयः॒ पिब॑न्ति । पिब॒न्तीति॒ पिब॑न्ति ॥ तेन॑ नः । नो॒ राजा᳚ । राजा॒ वरु॑णः । वरु॑णो॒ बृह॒स्पतिः॑ । बृह॒स्पति॒रा । आ प्या॑ययन्तु । प्या॒य॒य॒न्तु॒ भुव॑नस्य । भुव॑नस्य गो॒पाः । गो॒पा इति॑ गो - पाः ॥ प्राच्या᳚म् दि॒शि । दि॒शि त्वम् । त्वमि॑न्द्र । इ॒न्द्रा॒सि॒ । अ॒सि॒ राजा᳚ । राजो॒त । उ॒तोदी᳚च्याम् । उदी᳚च्यां ॅवृत्रहन्न् । वृ॒त्र॒ह॒न् वृ॒त्र॒हा । वृ॒त्र॒ह॒न्निति॑ वृत्र - ह॒न्न्॒ । वृ॒त्र॒हा ऽसि॑ । वृ॒त्र॒हेति॑ वृत्र - हा । अ॒सीत्य॑सि ॥ यत्र॒ यन्ति॑ । यन्ति॑ स्रो॒त्याः । स्रो॒त्यास्तत् । तज्जि॒तम् \newline

\textbf{Jatai Paata} \newline

1. नवो॑नवो भवति भवति॒ नवो॑नवो॒ नवो॑नवो भवति । \newline
2. नवो॑नव॒ इति॒ नवः॑ - न॒वः॒ । \newline
3. भ॒व॒ति॒ जाय॑मानो॒ जाय॑मानो भवति भवति॒ जाय॑मानः । \newline
4. जाय॑मा॒नो ऽह्ना॒ मह्ना॒म् जाय॑मानो॒ जाय॑मा॒नो ऽह्ना᳚म् । \newline
5. अह्ना᳚म् के॒तुः के॒तु रह्ना॒ मह्ना᳚म् के॒तुः । \newline
6. के॒तु रु॒षसा॑ मु॒षसा᳚म् के॒तुः के॒तु रु॒षसा᳚म् । \newline
7. उ॒षसा॑ मेत्ये त्यु॒षसा॑ मु॒षसा॑ मेति । \newline
8. ए॒त्यग्रे॒ अग्र॑ एत्ये॒ त्यग्रे᳚ । \newline
9. अग्र॒ इत्यग्रे᳚ । \newline
10. भा॒गम् दे॒वेभ्यो॑ दे॒वेभ्यो॑ भा॒गम् भा॒गम् दे॒वेभ्यः॑ । \newline
11. दे॒वेभ्यो॒ वि वि दे॒वेभ्यो॑ दे॒वेभ्यो॒ वि । \newline
12. वि द॑धाति दधाति॒ वि वि द॑धाति । \newline
13. द॒धा॒ त्या॒यन् ना॒यन् द॑धाति दधा त्या॒यन्न् । \newline
14. आ॒यन् प्र प्रायन् ना॒यन् प्र । \newline
15. आ॒यन्नित्या᳚ - यन्न् । \newline
16. प्र च॒न्द्रमा᳚ श्च॒न्द्रमाः॒ प्र प्र च॒न्द्रमाः᳚ । \newline
17. च॒न्द्रमा᳚ स्तिरति तिरति च॒न्द्रमा᳚ श्च॒न्द्रमा᳚ स्तिरति । \newline
18. ति॒र॒ति॒ दी॒र्घम् दी॒र्घम् ति॑रति तिरति दी॒र्घम् । \newline
19. दी॒र्घ मायु॒ रायु॑र् दी॒र्घम् दी॒र्घ मायुः॑ । \newline
20. आयु॒रित्यायुः॑ । \newline
21. य मा॑दि॒त्या आ॑दि॒त्या यं ॅय मा॑दि॒त्याः । \newline
22. आ॒दि॒त्या अꣳ॒॒शु मꣳ॒॒शु मा॑दि॒त्या आ॑दि॒त्या अꣳ॒॒शुम् । \newline
23. अꣳ॒॒शु मा᳚प्या॒यय॑ न्त्याप्या॒यय॑ न्त्यꣳ॒॒शु मꣳ॒॒शु मा᳚प्या॒यय॑न्ति । \newline
24. आ॒प्या॒यय॑न्ति॒ यं ॅय मा᳚प्या॒यय॑ न्त्याप्या॒यय॑न्ति॒ यम् । \newline
25. आ॒प्या॒यय॒न्तीत्या᳚ - प्या॒यय॑न्ति । \newline
26. य मक्षि॑त॒ मक्षि॑तं॒ ॅयं ॅय मक्षि॑तम् । \newline
27. अक्षि॑त॒ मक्षि॑तयो॒ अक्षि॑तयो॒ अक्षि॑त॒ मक्षि॑त॒ मक्षि॑तयः । \newline
28. अक्षि॑तयः॒ पिब॑न्ति॒ पिब॒ न्त्यक्षि॑तयो॒ अक्षि॑तयः॒ पिब॑न्ति । \newline
29. पिब॒न्तीति॒ पिब॑न्ति । \newline
30. तेन॑ नो न॒ स्तेन॒ तेन॑ नः । \newline
31. नो॒ राजा॒ राजा॑ नो नो॒ राजा᳚ । \newline
32. राजा॒ वरु॑णो॒ वरु॑णो॒ राजा॒ राजा॒ वरु॑णः । \newline
33. वरु॑णो॒ बृह॒स्पति॒र् बृह॒स्पति॒र् वरु॑णो॒ वरु॑णो॒ बृह॒स्पतिः॑ । \newline
34. बृह॒स्पति॒रा बृह॒स्पति॒र् बृह॒स्पति॒रा । \newline
35. आ प्या॑ययन्तु प्यायय॒न्त्वा प्या॑ययन्तु । \newline
36. प्या॒य॒य॒न्तु॒ भुव॑नस्य॒ भुव॑नस्य प्याययन्तु प्याययन्तु॒ भुव॑नस्य । \newline
37. भुव॑नस्य गो॒पा गो॒पा भुव॑नस्य॒ भुव॑नस्य गो॒पाः । \newline
38. गो॒पा इति॑ गो - पाः । \newline
39. प्राच्या᳚म् दि॒शि दि॒शि प्राच्या॒म् प्राच्या᳚म् दि॒शि । \newline
40. दि॒शि त्वम् त्वम् दि॒शि दि॒शि त्वम् । \newline
41. त्व मि॑न्द्रे न्द्र॒ त्वम् त्व मि॑न्द्र । \newline
42. इ॒न्द्रा॒स्य॒ सी॒न्द्रे॒ न्द्रा॒सि॒ । \newline
43. अ॒सि॒ राजा॒ राजा᳚ ऽस्यसि॒ राजा᳚ । \newline
44. राजो॒तोत राजा॒ राजो॒त । \newline
45. उ॒तोदी᳚च्या॒ मुदी᳚च्या मु॒तोतो दी᳚च्याम् । \newline
46. उदी᳚च्यां ॅवृत्रहन् वृत्रह॒न् नुदी᳚च्या॒ मुदी᳚च्यां ॅवृत्रहन्न् । \newline
47. वृ॒त्र॒ह॒न् वृ॒त्र॒हा वृ॑त्र॒हा वृ॑त्रहन् वृत्रहन् वृत्र॒हा । \newline
48. वृ॒त्र॒ह॒न्निति॑ वृत्र - ह॒न्न् । \newline
49. वृ॒त्र॒हा ऽस्य॑सि वृत्र॒हा वृ॑त्र॒हा ऽसि॑ । \newline
50. वृ॒त्र॒हेति॑ वृत्र - हा । \newline
51. अ॒सीत्य॑सि । \newline
52. यत्र॒ यन्ति॒ यन्ति॒ यत्र॒ यत्र॒ यन्ति॑ । \newline
53. यन्ति॑ स्रो॒त्याः स्रो॒त्या यन्ति॒ यन्ति॑ स्रो॒त्याः । \newline
54. स्रो॒त्या स्तत् तथ् स्रो॒त्याः स्रो॒त्या स्तत् । \newline
55. तज् जि॒तम् जि॒तम् तत् तज् जि॒तम् । \newline

\textbf{Ghana Paata } \newline

1. नवो॑नवो भवति भवति॒ नवो॑नवो॒ नवो॑नवो भवति॒ जाय॑मानो॒ जाय॑मानो भवति॒ नवो॑नवो॒ नवो॑नवो भवति॒ जाय॑मानः । \newline
2. नवो॑नव॒ इति॒ नवः॑ - न॒वः॒ । \newline
3. भ॒व॒ति॒ जाय॑मानो॒ जाय॑मानो भवति भवति॒ जाय॑मा॒नो ऽह्ना॒ मह्ना॒म् जाय॑मानो भवति भवति॒ जाय॑मा॒नो ऽह्ना᳚म् । \newline
4. जाय॑मा॒नो ऽह्ना॒ मह्ना॒म् जाय॑मानो॒ जाय॑मा॒नो ऽह्ना᳚म् के॒तुः के॒तु रह्ना॒म् जाय॑मानो॒ जाय॑मा॒नो ऽह्ना᳚म् के॒तुः । \newline
5. अह्ना᳚म् के॒तुः के॒तु रह्ना॒ मह्ना᳚म् के॒तु रु॒षसा॑ मु॒षसा᳚म् के॒तु रह्ना॒ मह्ना᳚म् के॒तु रु॒षसा᳚म् । \newline
6. के॒तु रु॒षसा॑ मु॒षसा᳚म् के॒तुः के॒तु रु॒षसा॑ मेत्ये त्यु॒षसा᳚म् के॒तुः के॒तु रु॒षसा॑ मेति । \newline
7. उ॒षसा॑ मेत्ये त्यु॒षसा॑ मु॒षसा॑ मे॒त्यग्रे॒ अग्र॑ एत्यु॒षसा॑ मु॒षसा॑ मे॒त्यग्रे᳚ । \newline
8. ए॒त्यग्रे॒ अग्र॑ एत्ये॒त्यग्रे᳚ । \newline
9. अग्र॒ इत्यग्रे᳚ । \newline
10. भा॒गम् दे॒वेभ्यो॑ दे॒वेभ्यो॑ भा॒गम् भा॒गम् दे॒वेभ्यो॒ वि वि दे॒वेभ्यो॑ भा॒गम् भा॒गम् दे॒वेभ्यो॒ वि । \newline
11. दे॒वेभ्यो॒ वि वि दे॒वेभ्यो॑ दे॒वेभ्यो॒ वि द॑धाति दधाति॒ वि दे॒वेभ्यो॑ दे॒वेभ्यो॒ वि द॑धाति । \newline
12. वि द॑धाति दधाति॒ वि वि द॑धा त्या॒यन् ना॒यन् द॑धाति॒ वि वि द॑धा त्या॒यन्न् । \newline
13. द॒धा॒ त्या॒यन् ना॒यन् द॑धाति दधा त्या॒यन् प्र प्रायन् द॑धाति दधा त्या॒यन् प्र । \newline
14. आ॒यन् प्र प्रायन् ना॒यन् प्र च॒न्द्रमा᳚ श्च॒न्द्रमाः॒ प्रायन् ना॒यन् प्र च॒न्द्रमाः᳚ । \newline
15. आ॒यन्नित्या᳚ - यन्न् । \newline
16. प्र च॒न्द्रमा᳚ श्च॒न्द्रमाः॒ प्र प्र च॒न्द्रमा᳚ स्तिरति तिरति च॒न्द्रमाः॒ प्र प्र च॒न्द्रमा᳚ स्तिरति । \newline
17. च॒न्द्रमा᳚ स्तिरति तिरति च॒न्द्रमा᳚ श्च॒न्द्रमा᳚ स्तिरति दी॒र्घम् दी॒र्घम् ति॑रति च॒न्द्रमा᳚ श्च॒न्द्रमा᳚ स्तिरति दी॒र्घम् । \newline
18. ति॒र॒ति॒ दी॒र्घम् दी॒र्घम् ति॑रति तिरति दी॒र्घ मायु॒ रायु॑र् दी॒र्घम् ति॑रति तिरति दी॒र्घ मायुः॑ । \newline
19. दी॒र्घ मायु॒ रायु॑र् दी॒र्घम् दी॒र्घ मायुः॑ । \newline
20. आयु॒रित्यायुः॑ । \newline
21. य मा॑दि॒त्या आ॑दि॒त्या यं ॅय मा॑दि॒त्या अꣳ॒॒शु मꣳ॒॒शु मा॑दि॒त्या यं ॅय मा॑दि॒त्या अꣳ॒॒शुम् । \newline
22. आ॒दि॒त्या अꣳ॒॒शु मꣳ॒॒शु मा॑दि॒त्या आ॑दि॒त्या अꣳ॒॒शु मा᳚प्या॒यय॑ न्त्याप्या॒यय॑ न्त्यꣳ॒॒शु मा॑दि॒त्या आ॑दि॒त्या अꣳ॒॒शु मा᳚प्या॒यय॑न्ति । \newline
23. अꣳ॒॒शु मा᳚प्या॒यय॑ न्त्याप्या॒यय॑ न्त्यꣳ॒॒शु मꣳ॒॒शु मा᳚प्या॒यय॑न्ति॒ यं ॅय मा᳚प्या॒यय॑ न्त्यꣳ॒॒शु मꣳ॒॒शु मा᳚प्या॒यय॑न्ति॒ यम् । \newline
24. आ॒प्या॒यय॑न्ति॒ यं ॅय मा᳚प्या॒यय॑ न्त्याप्या॒यय॑न्ति॒ य मक्षि॑त॒ मक्षि॑तं॒ ॅय मा᳚प्या॒यय॑ न्त्याप्या॒यय॑न्ति॒ य मक्षि॑तम् । \newline
25. आ॒प्या॒यय॒न्तीत्या᳚ - प्या॒यय॑न्ति । \newline
26. य मक्षि॑त॒ मक्षि॑तं॒ ॅयं ॅय मक्षि॑त॒ मक्षि॑तयो॒ अक्षि॑तयो॒ अक्षि॑तं॒ ॅयं ॅय मक्षि॑त॒ मक्षि॑तयः । \newline
27. अक्षि॑त॒ मक्षि॑तयो॒ अक्षि॑तयो॒ अक्षि॑त॒ मक्षि॑त॒ मक्षि॑तयः॒ पिब॑न्ति॒ पिब॒ न्त्यक्षि॑तयो॒ अक्षि॑त॒ मक्षि॑त॒ मक्षि॑तयः॒ पिब॑न्ति । \newline
28. अक्षि॑तयः॒ पिब॑न्ति॒ पिब॒ न्त्यक्षि॑तयो॒ अक्षि॑तयः॒ पिब॑न्ति । \newline
29. पिब॒न्तीति॒ पिब॑न्ति । \newline
30. तेन॑ नो न॒ स्तेन॒ तेन॑ नो॒ राजा॒ राजा॑ न॒ स्तेन॒ तेन॑ नो॒ राजा᳚ । \newline
31. नो॒ राजा॒ राजा॑ नो नो॒ राजा॒ वरु॑णो॒ वरु॑णो॒ राजा॑ नो नो॒ राजा॒ वरु॑णः । \newline
32. राजा॒ वरु॑णो॒ वरु॑णो॒ राजा॒ राजा॒ वरु॑णो॒ बृह॒स्पति॒र् बृह॒स्पति॒र् वरु॑णो॒ राजा॒ राजा॒ वरु॑णो॒ बृह॒स्पतिः॑ । \newline
33. वरु॑णो॒ बृह॒स्पति॒र् बृह॒स्पति॒र् वरु॑णो॒ वरु॑णो॒ बृह॒स्पति॒रा बृह॒स्पति॒र् वरु॑णो॒ वरु॑णो॒ बृह॒स्पति॒रा । \newline
34. बृह॒स्पति॒रा बृह॒स्पति॒र् बृह॒स्पति॒रा प्या॑ययन्तु प्यायय॒न्त्वा बृह॒स्पति॒र् बृह॒स्पति॒रा प्या॑ययन्तु । \newline
35. आ प्या॑ययन्तु प्यायय॒न्त्वा प्या॑ययन्तु॒ भुव॑नस्य॒ भुव॑नस्य प्यायय॒न्त्वा प्या॑ययन्तु॒ भुव॑नस्य । \newline
36. प्या॒य॒य॒न्तु॒ भुव॑नस्य॒ भुव॑नस्य प्याययन्तु प्याययन्तु॒ भुव॑नस्य गो॒पा गो॒पा भुव॑नस्य प्याययन्तु प्याययन्तु॒ भुव॑नस्य गो॒पाः । \newline
37. भुव॑नस्य गो॒पा गो॒पा भुव॑नस्य॒ भुव॑नस्य गो॒पाः । \newline
38. गो॒पा इति॑ गो - पाः । \newline
39. प्राच्या᳚म् दि॒शि दि॒शि प्राच्या॒म् प्राच्या᳚म् दि॒शि त्वम् त्वम् दि॒शि प्राच्या॒म् प्राच्या᳚म् दि॒शि त्वम् । \newline
40. दि॒शि त्वम् त्वम् दि॒शि दि॒शि त्व मि॑न्द्रे न्द्र॒ त्वम् दि॒शि दि॒शि त्व मि॑न्द्र । \newline
41. त्व मि॑न्द्रे न्द्र॒ त्वम् त्व मि॑न्द्रास्यसीन्द्र॒ त्वम् त्व मि॑न्द्रासि । \newline
42. इ॒न्द्रा॒स्य॒सी॒न्द्रे॒ न्द्रा॒सि॒ राजा॒ राजा॑ ऽसीन्द्रे न्द्रासि॒ राजा᳚ । \newline
43. अ॒सि॒ राजा॒ राजा᳚ ऽस्यसि॒ राजो॒तोत राजा᳚ ऽस्यसि॒ राजो॒त । \newline
44. राजो॒तोत राजा॒ राजो॒तोदी᳚च्या॒ मुदी᳚च्या मु॒त राजा॒ राजो॒तो दी᳚च्याम् । \newline
45. उ॒तोदी᳚च्या॒ मुदी᳚च्या मु॒तोतो दी᳚च्यां ॅवृत्रहन् वृत्रह॒न् नुदी᳚च्या मु॒तोतो दी᳚च्यां ॅवृत्रहन्न् । \newline
46. उदी᳚च्यां ॅवृत्रहन् वृत्रह॒न् नुदी᳚च्या॒ मुदी᳚च्यां ॅवृत्रहन् वृत्र॒हा वृ॑त्र॒हा वृ॑त्रह॒न् नुदी᳚च्या॒ मुदी᳚च्यां ॅवृत्रहन् वृत्र॒हा । \newline
47. वृ॒त्र॒ह॒न् वृ॒त्र॒हा वृ॑त्र॒हा वृ॑त्रहन् वृत्रहन् वृत्र॒हा ऽस्य॑सि वृत्र॒हा वृ॑त्रहन् वृत्रहन् वृत्र॒हा ऽसि॑ । \newline
48. वृ॒त्र॒ह॒न्निति॑ वृत्र - ह॒न्न् । \newline
49. वृ॒त्र॒हा ऽस्य॑सि वृत्र॒हा वृ॑त्र॒हा ऽसि॑ । \newline
50. वृ॒त्र॒हेति॑ वृत्र - हा । \newline
51. अ॒सीत्य॑सि । \newline
52. यत्र॒ यन्ति॒ यन्ति॒ यत्र॒ यत्र॒ यन्ति॑ स्रो॒त्याः स्रो॒त्या यन्ति॒ यत्र॒ यत्र॒ यन्ति॑ स्रो॒त्याः । \newline
53. यन्ति॑ स्रो॒त्याः स्रो॒त्या यन्ति॒ यन्ति॑ स्रो॒त्या स्तत् तथ् स्रो॒त्या यन्ति॒ यन्ति॑ स्रो॒त्या स्तत् । \newline
54. स्रो॒त्या स्तत् तथ् स्रो॒त्याः स्रो॒त्या स्तज् जि॒तम् जि॒तम् तथ् स्रो॒त्याः स्रो॒त्या स्तज् जि॒तम् । \newline
55. तज् जि॒तम् जि॒तम् तत् तज् जि॒तम् ते॑ ते जि॒तम् तत् तज् जि॒तम् ते᳚ । \newline
\pagebreak
\markright{ TS 2.4.14.2  \hfill https://www.vedavms.in \hfill}

\section{ TS 2.4.14.2 }

\textbf{TS 2.4.14.2 } \newline
\textbf{Samhita Paata} \newline

-ज्जि॒तं ते॑ दक्षिण॒तो वृ॑ष॒भ ए॑धि॒ हव्यः॑ ॥ इन्द्रो॑ जयाति॒ न परा॑ जयाता अधिरा॒जो राज॑सु राजयाति । विश्वा॒ हि भू॒याः पृत॑ना अभि॒ष्टीरु॑प॒सद्यो॑ नम॒स्यो॑ यथाऽस॑त् ॥ अ॒स्येदे॒व प्ररि॑रिचे महि॒त्वं दि॒वः पृ॑थि॒व्याः पर्य॒न्तरि॑क्षात् । स्व॒राडिन्द्रो॒ दम॒ आ वि॒श्वगू᳚र्तः स्व॒रिरम॑त्रो ववक्षे॒ रणा॑य ॥ अ॒भि त्वा॑ शूर नोनु॒मोऽदु॑ग्धा इव धे॒नवः॑ । ईशा॑न- [  ] \newline

\textbf{Pada Paata} \newline

जि॒तम् । ते॒ । द॒क्षि॒ण॒तः । वृ॒ष॒भः । ए॒धि॒ । हव्यः॑ ॥ इन्द्रः॑ । ज॒या॒ति॒ । न । परेति॑ । ज॒या॒तै॒ । अ॒धि॒रा॒ज इत्य॑धि - रा॒जः । राज॒स्विति॒ राज॑ - सु॒ । रा॒ज॒या॒ति॒ ॥ विश्वाः᳚ । हि । भू॒याः । पृत॑नाः । अ॒भि॒ष्टीः । उ॒प॒सद्य॒ इत्यु॑प - सद्यः॑ । न॒म॒स्यः॑ । यथा᳚ । अस॑त् ॥ अ॒स्य । इत् । ए॒व । प्रेति॑ । रि॒रि॒चे॒ । म॒हि॒त्वमिति॑ महि - त्वम् । दि॒वः । पृ॒थि॒व्याः । परीति॑ । अ॒न्तरि॑क्षात् ॥ स्व॒राडिति॑ स्व - राट् । इन्द्रः॑ । दमे᳚ । एति॑ । वि॒श्वगू᳚र्त॒ इति॑ वि॒श्व-गू॒र्तः॒ । स्व॒रिः । अम॑त्रः । व॒व॒क्षे॒ ।    रणा॑य ॥ अ॒भीति॑ । त्वा॒ । शू॒र॒ । नो॒नु॒मः॒ । अदु॑ग्धाः । इ॒व॒ । धे॒नवः॑ ॥ ईशा॑नम् ।  \newline


\textbf{Krama Paata} \newline

जि॒तम् ते᳚ । ते॒ द॒क्षि॒ण॒तः । द॒क्षि॒ण॒तो वृ॑ष॒भः । वृ॒ष॒भ ए॑धि । ए॒धि॒ हव्यः॑ । हव्य॒ इति॒ हव्यः॑ ॥ इन्द्रो॑ जयाति । ज॒या॒ति॒ न । न परा᳚ । परा॑ जयातै । ज॒या॒ता॒ अ॒धि॒रा॒जः । अ॒धि॒रा॒जो राज॑सु । अ॒धि॒रा॒ज इत्य॑धि - रा॒जः । राज॑सु राजयाति । राज॒स्विति॒ राज॑ - सु॒ । रा॒ज॒या॒तीति॑ राजयाति ॥ विश्वा॒ हि । हि भू॒याः । भू॒याः पृत॑नाः । पृत॑ना अभि॒ष्टीः । अ॒भि॒ष्टीरु॑प॒सद्यः॑ । उ॒प॒सद्यो॑ नम॒स्यः॑ । उ॒प॒सद्य॒ इत्यु॑प - सद्यः॑ । न॒म॒स्यो॑ यथा᳚ । यथा ऽस॑त् । अस॒दित्यस॑त् ॥ अ॒स्येत् । इदे॒व । ए॒व प्र । प्र रि॑रिचे । रि॒रि॒चे॒ म॒हि॒त्वम् । म॒हि॒त्वम् दि॒वः । म॒हि॒त्वमिति॑ महि - त्वम् । दि॒वः पृ॑थि॒व्याः । पृ॒थि॒व्याः परि॑ । पर्य॒न्तरि॑क्षात् । अ॒न्तरि॑क्षा॒दित्य॒न्तरि॑क्षात् ॥ स्व॒राडिन्द्रः॑ । स्व॒राडिति॑ स्व - राट् । इन्द्रो॒ दमे᳚ । दम॒ आ । आ वि॒श्वगू᳚र्तः । वि॒श्वगू᳚र्तः स्व॒रिः । वि॒श्वगू᳚र्त॒ इति॑ वि॒श्व - गू॒र्तः॒ । स्व॒रिरम॑त्रः । अम॑त्रो ववक्षे । व॒व॒क्षे॒ रणा॑य । रणा॒येति॒ रणा॑य ॥ अ॒भि त्वा᳚ । त्वा॒ शू॒र॒ । शू॒र॒ नो॒नु॒मः॒ । नो॒नु॒मोऽदु॑ग्धाः । अदु॑ग्धा इव । इ॒व॒ धे॒नवः॑ । धे॒नव॒ इति॑ धे॒नवः॑ ॥ ईशा॑नम॒स्य \newline

\textbf{Jatai Paata} \newline

1. जि॒तम् ते॑ ते जि॒तम् जि॒तम् ते᳚ । \newline
2. ते॒ द॒क्षि॒ण॒तो द॑क्षिण॒त स्ते॑ ते दक्षिण॒तः । \newline
3. द॒क्षि॒ण॒तो वृ॑ष॒भो वृ॑ष॒भो द॑क्षिण॒तो द॑क्षिण॒तो वृ॑ष॒भः । \newline
4. वृ॒ष॒भ ए᳚ध्येधि वृष॒भो वृ॑ष॒भ ए॑धि । \newline
5. ए॒धि॒ हव्यो॒ हव्य॑ एध्येधि॒ हव्यः॑ । \newline
6. हव्य॒ इति॒ हव्यः॑ । \newline
7. इन्द्रो॑ जयाति जया॒तीन्द्र॒ इन्द्रो॑ जयाति । \newline
8. ज॒या॒ति॒ न न ज॑याति जयाति॒ न । \newline
9. न परा॒ परा॒ न न परा᳚ । \newline
10. परा॑ जयातै जयातै॒ परा॒ परा॑ जयातै । \newline
11. ज॒या॒ता॒ अ॒धि॒रा॒जो अ॑धिरा॒जो ज॑यातै जयाता अधिरा॒जः । \newline
12. अ॒धि॒रा॒जो राज॑सु॒ राज॑ स्वधिरा॒जो अ॑धिरा॒जो राज॑सु । \newline
13. अ॒धि॒रा॒ज इत्य॑धि - रा॒जः । \newline
14. राज॑सु राजयाति राजयाति॒ राज॑सु॒ राज॑सु राजयाति । \newline
15. राज॒स्विति॒ राज॑ - सु॒ । \newline
16. रा॒ज॒या॒तीति॑ राजयाति । \newline
17. विश्वा॒ हि हि विश्वा॒ विश्वा॒ हि । \newline
18. हि भू॒या भू॒या हि हि भू॒याः । \newline
19. भू॒याः पृत॑नाः॒ पृत॑ना भू॒या भू॒याः पृत॑नाः । \newline
20. पृत॑ना अभि॒ष्टी र॑भि॒ष्टीः पृत॑नाः॒ पृत॑ना अभि॒ष्टीः । \newline
21. अ॒भि॒ष्टी रु॑प॒सद्य॑ उप॒सद्यो॑ अभि॒ष्टी र॑भि॒ष्टी रु॑प॒सद्यः॑ । \newline
22. उ॒प॒सद्यो॑ नम॒स्यो॑ नम॒स्य॑ उप॒सद्य॑ उप॒सद्यो॑ नम॒स्यः॑ । \newline
23. उ॒प॒सद्य॒ इत्यु॑प - सद्यः॑ । \newline
24. न॒म॒स्यो॑ यथा॒ यथा॑ नम॒स्यो॑ नम॒स्यो॑ यथा᳚ । \newline
25. यथा ऽस॒ दस॒द् यथा॒ यथा ऽस॑त् । \newline
26. अस॒दित्यस॑त् । \newline
27. अ॒स्ये दिद॒स्यास्येत् । \newline
28. इदे॒वैवे दिदे॒व । \newline
29. ए॒व प्र प्रैवैव प्र । \newline
30. प्र रि॑रिचे रिरिचे॒ प्र प्र रि॑रिचे । \newline
31. रि॒रि॒चे॒ म॒हि॒त्वम् म॑हि॒त्वꣳ रि॑रिचे रिरिचे महि॒त्वम् । \newline
32. म॒हि॒त्वम् दि॒वो दि॒वो म॑हि॒त्वम् म॑हि॒त्वम् दि॒वः । \newline
33. म॒हि॒त्वमिति॑ महि - त्वम् । \newline
34. दि॒वः पृ॑थि॒व्याः पृ॑थि॒व्या दि॒वो दि॒वः पृ॑थि॒व्याः । \newline
35. पृ॒थि॒व्याः परि॒ परि॑ पृथि॒व्याः पृ॑थि॒व्याः परि॑ । \newline
36. पर्य॒न्तरि॑क्षा द॒न्तरि॑क्षा॒त् परि॒ पर्य॒न्तरि॑क्षात् । \newline
37. अ॒न्तरि॑क्षा॒दित्य॒न्तरि॑क्षात् । \newline
38. स्व॒रा डिन्द्र॒ इन्द्रः॑ स्व॒राट् थ्स्व॒रा डिन्द्रः॑ । \newline
39. स्व॒राडिति॑ स्व - राट् । \newline
40. इन्द्रो॒ दमे॒ दम॒ इन्द्र॒ इन्द्रो॒ दमे᳚ । \newline
41. दम॒ आ दमे॒ दम॒ आ । \newline
42. आ वि॒श्वगू᳚र्तो वि॒श्वगू᳚र्त॒ आ वि॒श्वगू᳚र्तः । \newline
43. वि॒श्वगू᳚र्तः स्व॒रिः स्व॒रिर् वि॒श्वगू᳚र्तो वि॒श्वगू᳚र्तः स्व॒रिः । \newline
44. वि॒श्वगू᳚र्त॒ इति॑ वि॒श्व - गू॒र्तः॒ । \newline
45. स्व॒रि रम॑त्रो॒ अम॑त्रः स्व॒रिः स्व॒रि रम॑त्रः । \newline
46. अम॑त्रो ववक्षे ववक्षे॒ अम॑त्रो॒ अम॑त्रो ववक्षे । \newline
47. व॒व॒क्षे॒ रणा॑य॒ रणा॑य ववक्षे ववक्षे॒ रणा॑य । \newline
48. रणा॒येति॒ रणा॑य । \newline
49. अ॒भि त्वा᳚ त्वा॒ ऽभ्य॑भि त्वा᳚ । \newline
50. त्वा॒ शू॒र॒ शू॒र॒ त्वा॒ त्वा॒ शू॒र॒ । \newline
51. शू॒र॒ नो॒नु॒मो॒ नो॒नु॒मः॒ शू॒र॒ शू॒र॒ नो॒नु॒मः॒ । \newline
52. नो॒नु॒मो ऽदु॑ग्धा॒ अदु॑ग्धा नोनुमो नोनु॒मो ऽदु॑ग्धाः । \newline
53. अदु॑ग्धा इवे॒ वादु॑ग्धा॒ अदु॑ग्धा इव । \newline
54. इ॒व॒ धे॒नवो॑ धे॒नव॑ इवे व धे॒नवः॑ । \newline
55. धे॒नव॒ इति॑ धे॒नवः॑ । \newline
56. ईशा॑न म॒स्यास्येशा॑न॒ मीशा॑न म॒स्य । \newline

\textbf{Ghana Paata } \newline

1. जि॒तम् ते॑ ते जि॒तम् जि॒तम् ते॑ दक्षिण॒तो द॑क्षिण॒त स्ते॑ जि॒तम् जि॒तम् ते॑ दक्षिण॒तः । \newline
2. ते॒ द॒क्षि॒ण॒तो द॑क्षिण॒त स्ते॑ ते दक्षिण॒तो वृ॑ष॒भो वृ॑ष॒भो द॑क्षिण॒त स्ते॑ ते दक्षिण॒तो वृ॑ष॒भः । \newline
3. द॒क्षि॒ण॒तो वृ॑ष॒भो वृ॑ष॒भो द॑क्षिण॒तो द॑क्षिण॒तो वृ॑ष॒भ ए᳚ध्येधि वृष॒भो द॑क्षिण॒तो द॑क्षिण॒तो वृ॑ष॒भ ए॑धि । \newline
4. वृ॒ष॒भ ए᳚ध्येधि वृष॒भो वृ॑ष॒भ ए॑धि॒ हव्यो॒ हव्य॑ एधि वृष॒भो वृ॑ष॒भ ए॑धि॒ हव्यः॑ । \newline
5. ए॒धि॒ हव्यो॒ हव्य॑ एध्येधि॒ हव्यः॑ । \newline
6. हव्य॒ इति॒ हव्यः॑ । \newline
7. इन्द्रो॑ जयाति जया॒तीन्द्र॒ इन्द्रो॑ जयाति॒ न न ज॑या॒तीन्द्र॒ इन्द्रो॑ जयाति॒ न । \newline
8. ज॒या॒ति॒ न न ज॑याति जयाति॒ न परा॒ परा॒ न ज॑याति जयाति॒ न परा᳚ । \newline
9. न परा॒ परा॒ न न परा॑ जयातै जयातै॒ परा॒ न न परा॑ जयातै । \newline
10. परा॑ जयातै जयातै॒ परा॒ परा॑ जयाता अधिरा॒जो अ॑धिरा॒जो ज॑यातै॒ परा॒ परा॑ जयाता अधिरा॒जः । \newline
11. ज॒या॒ता॒ अ॒धि॒रा॒जो अ॑धिरा॒जो ज॑यातै जयाता अधिरा॒जो राज॑सु॒ राज॑ स्वधिरा॒जो ज॑यातै जयाता अधिरा॒जो राज॑सु । \newline
12. अ॒धि॒रा॒जो राज॑सु॒ राज॑ स्वधिरा॒जो अ॑धिरा॒जो राज॑सु राजयाति राजयाति॒ राज॑ स्वधिरा॒जो अ॑धिरा॒जो राज॑सु राजयाति । \newline
13. अ॒धि॒रा॒ज इत्य॑धि - रा॒जः । \newline
14. राज॑सु राजयाति राजयाति॒ राज॑सु॒ राज॑सु राजयाति । \newline
15. राज॒स्विति॒ राज॑ - सु॒ । \newline
16. रा॒ज॒या॒तीति॑ राजयाति । \newline
17. विश्वा॒ हि हि विश्वा॒ विश्वा॒ हि भू॒या भू॒या हि विश्वा॒ विश्वा॒ हि भू॒याः । \newline
18. हि भू॒या भू॒या हि हि भू॒याः पृत॑नाः॒ पृत॑ना भू॒या हि हि भू॒याः पृत॑नाः । \newline
19. भू॒याः पृत॑नाः॒ पृत॑ना भू॒या भू॒याः पृत॑ना अभि॒ष्टी र॑भि॒ष्टीः पृत॑ना भू॒या भू॒याः पृत॑ना अभि॒ष्टीः । \newline
20. पृत॑ना अभि॒ष्टी र॑भि॒ष्टीः पृत॑नाः॒ पृत॑ना अभि॒ष्टी रु॑प॒सद्य॑ उप॒सद्यो॑ अभि॒ष्टीः पृत॑नाः॒ पृत॑ना अभि॒ष्टी रु॑प॒सद्यः॑ । \newline
21. अ॒भि॒ष्टी रु॑प॒सद्य॑ उप॒सद्यो॑ अभि॒ष्टी र॑भि॒ष्टी रु॑प॒सद्यो॑ नम॒स्यो॑ नम॒स्य॑ उप॒सद्यो॑ अभि॒ष्टी र॑भि॒ष्टी रु॑प॒सद्यो॑ नम॒स्यः॑ । \newline
22. उ॒प॒सद्यो॑ नम॒स्यो॑ नम॒स्य॑ उप॒सद्य॑ उप॒सद्यो॑ नम॒स्यो॑ यथा॒ यथा॑ नम॒स्य॑ उप॒सद्य॑ उप॒सद्यो॑ नम॒स्यो॑ यथा᳚ । \newline
23. उ॒प॒सद्य॒ इत्यु॑प - सद्यः॑ । \newline
24. न॒म॒स्यो॑ यथा॒ यथा॑ नम॒स्यो॑ नम॒स्यो॑ यथा ऽस॒दस॒द् यथा॑ नम॒स्यो॑ नम॒स्यो॑ यथा ऽस॑त् । \newline
25. यथा ऽस॒दस॒द् यथा॒ यथा ऽस॑त् । \newline
26. अस॒दित्यस॑त् । \newline
27. अ॒स्ये दिद॒स्या स्ये दे॒वैवे द॒स्यास्ये दे॒व । \newline
28. इदे॒वैवे दिदे॒व प्र प्रैवे दिदे॒व प्र । \newline
29. ए॒व प्र प्रैवैव प्र रि॑रिचे रिरिचे॒ प्रैवैव प्र रि॑रिचे । \newline
30. प्र रि॑रिचे रिरिचे॒ प्र प्र रि॑रिचे महि॒त्वम् म॑हि॒त्वꣳ रि॑रिचे॒ प्र प्र रि॑रिचे महि॒त्वम् । \newline
31. रि॒रि॒चे॒ म॒हि॒त्वम् म॑हि॒त्वꣳ रि॑रिचे रिरिचे महि॒त्वम् दि॒वो दि॒वो म॑हि॒त्वꣳ रि॑रिचे रिरिचे महि॒त्वम् दि॒वः । \newline
32. म॒हि॒त्वम् दि॒वो दि॒वो म॑हि॒त्वम् म॑हि॒त्वम् दि॒वः पृ॑थि॒व्याः पृ॑थि॒व्या दि॒वो म॑हि॒त्वम् म॑हि॒त्वम् दि॒वः पृ॑थि॒व्याः । \newline
33. म॒हि॒त्वमिति॑ महि - त्वम् । \newline
34. दि॒वः पृ॑थि॒व्याः पृ॑थि॒व्या दि॒वो दि॒वः पृ॑थि॒व्याः परि॒ परि॑ पृथि॒व्या दि॒वो दि॒वः पृ॑थि॒व्याः परि॑ । \newline
35. पृ॒थि॒व्याः परि॒ परि॑ पृथि॒व्याः पृ॑थि॒व्याः पर्य॒न्तरि॑क्षा द॒न्तरि॑क्षा॒त् परि॑ पृथि॒व्याः पृ॑थि॒व्याः पर्य॒न्तरि॑क्षात् । \newline
36. पर्य॒न्तरि॑क्षा द॒न्तरि॑क्षा॒त् परि॒ पर्य॒न्तरि॑क्षात् । \newline
37. अ॒न्तरि॑क्षा॒दित्य॒न्तरि॑क्षात् । \newline
38. स्व॒रा डिन्द्र॒ इन्द्रः॑ स्व॒राट् थ्स्व॒रा डिन्द्रो॒ दमे॒ दम॒ इन्द्रः॑ स्व॒राट् थ्स्व॒रा डिन्द्रो॒ दमे᳚ । \newline
39. स्व॒राडिति॑ स्व - राट् । \newline
40. इन्द्रो॒ दमे॒ दम॒ इन्द्र॒ इन्द्रो॒ दम॒ आ दम॒ इन्द्र॒ इन्द्रो॒ दम॒ आ । \newline
41. दम॒ आ दमे॒ दम॒ आ वि॒श्वगू᳚र्तो वि॒श्वगू᳚र्त॒ आ दमे॒ दम॒ आ वि॒श्वगू᳚र्तः । \newline
42. आ वि॒श्वगू᳚र्तो वि॒श्वगू᳚र्त॒ आ वि॒श्वगू᳚र्तः स्व॒रिः स्व॒रिर् वि॒श्वगू᳚र्त॒ आ वि॒श्वगू᳚र्तः स्व॒रिः । \newline
43. वि॒श्वगू᳚र्तः स्व॒रिः स्व॒रिर् वि॒श्वगू᳚र्तो वि॒श्वगू᳚र्तः स्व॒रिरम॑त्रो॒ अम॑त्रः स्व॒रिर् वि॒श्वगू᳚र्तो वि॒श्वगू᳚र्तः स्व॒रिरम॑त्रः । \newline
44. वि॒श्वगू᳚र्त॒ इति॑ वि॒श्व - गू॒र्तः॒ । \newline
45. स्व॒रिरम॑त्रो॒ अम॑त्रः स्व॒रिः स्व॒रिरम॑त्रो ववक्षे ववक्षे॒ अम॑त्रः स्व॒रिः स्व॒रिरम॑त्रो ववक्षे । \newline
46. अम॑त्रो ववक्षे ववक्षे॒ अम॑त्रो॒ अम॑त्रो ववक्षे॒ रणा॑य॒ रणा॑य ववक्षे॒ अम॑त्रो॒ अम॑त्रो ववक्षे॒ रणा॑य । \newline
47. व॒व॒क्षे॒ रणा॑य॒ रणा॑य ववक्षे ववक्षे॒ रणा॑य । \newline
48. रणा॒येति॒ रणा॑य । \newline
49. अ॒भि त्वा᳚ त्वा॒ ऽभ्य॑भि त्वा॑ शूर शूर त्वा॒ ऽभ्य॑भि त्वा॑ शूर । \newline
50. त्वा॒ शू॒र॒ शू॒र॒ त्वा॒ त्वा॒ शू॒र॒ नो॒नु॒मो॒ नो॒नु॒मः॒ शू॒र॒ त्वा॒ त्वा॒ शू॒र॒ नो॒नु॒मः॒ । \newline
51. शू॒र॒ नो॒नु॒मो॒ नो॒नु॒मः॒ शू॒र॒ शू॒र॒ नो॒नु॒मो ऽदु॑ग्धा॒ अदु॑ग्धा नोनुमः शूर शूर नोनु॒मो ऽदु॑ग्धाः । \newline
52. नो॒नु॒मो ऽदु॑ग्धा॒ अदु॑ग्धा नोनुमो नोनु॒मो ऽदु॑ग्धा इवे॒ वादु॑ग्धा नोनुमो नोनु॒मो ऽदु॑ग्धा इव । \newline
53. अदु॑ग्धा इवे॒ वादु॑ग्धा॒ अदु॑ग्धा इव धे॒नवो॑ धे॒नव॑ इ॒वादु॑ग्धा॒ अदु॑ग्धा इव धे॒नवः॑ । \newline
54. इ॒व॒ धे॒नवो॑ धे॒नव॑ इवे व धे॒नवः॑ । \newline
55. धे॒नव॒ इति॑ धे॒नवः॑ । \newline
56. ईशा॑न म॒स्यास्येशा॑न॒ मीशा॑न म॒स्य जग॑तो॒ जग॑तो अ॒स्येशा॑न॒ मीशा॑न म॒स्य जग॑तः । \newline
\pagebreak
\markright{ TS 2.4.14.3  \hfill https://www.vedavms.in \hfill}

\section{ TS 2.4.14.3 }

\textbf{TS 2.4.14.3 } \newline
\textbf{Samhita Paata} \newline

-म॒स्य जग॑तः सुव॒र्दृश॒मीशा॑नमिन्द्र त॒स्थुषः॑ ॥त्वामिद्धि हवा॑महे सा॒ता वाज॑स्य का॒रवः॑ । त्वां ॅवृ॒त्रेष्वि॑न्द्र॒ सत्प॑तिं॒ नर॒स्त्वां काष्ठा॒स्वर्व॑तः ॥ यद्द्याव॑इन्द्रतेश॒तꣳ श॒तं भूमी॑रु॒त स्युः । न त्वा॑ वज्रिन्थ् स॒हस्रꣳ॒॒ सूर्या॒ अनु॒ न जा॒तम॑ष्ट॒ रोद॑सी ॥ पिबा॒ सोम॑मिन्द्र॒ मन्द॑तु त्वा॒ यन्ते॑ सु॒षाव॑ हर्य॒श्वाद्रिः॑ । \newline

\textbf{Pada Paata} \newline

अ॒स्य । जग॑तः । सु॒व॒र्दृश॒मिति॑ सुवः - दृश᳚म् । ईशा॑नम् । इ॒न्द्र॒ । त॒स्थुषः॑ ॥ त्वाम् । इत् । हि । हवा॑महे । सा॒ता । वाज॑स्य । का॒रवः॑ ॥ त्वाम् । वृ॒त्रेषु॑ । इ॒न्द्र॒ । सत्प॑ति॒मिति॒ सत् - प॒ति॒म् । नरः॑ । त्वाम् । काष्ठा॑सु । अर्व॑तः ॥ यत् । द्यावः॑ । इ॒न्द्र॒ । ते॒ । श॒तम् । श॒तम् । भूमीः᳚ । उ॒त । स्युः ॥ न । त्वा॒ । व॒ज्रि॒न्न् । स॒हस्र᳚म् । सूर्याः᳚ । अन्विति॑ ।   न । जा॒तम् । अ॒ष्ट॒ । रोद॑सी॒ इति॑ ॥ पिब॑ । सोम᳚म् । इ॒न्द्र॒ । मन्द॑तु । त्वा॒ । यम् । ते॒ । सु॒षाव॑ । ह॒र्य॒श्वेति॑ हरि - अ॒श्व॒ । अद्रिः॑ ॥  \newline


\textbf{Krama Paata} \newline

अ॒स्य जग॑तः । जग॑तः सुव॒र्दृश᳚म् । सु॒व॒र्दृश॒मीशा॑नम् । सु॒व॒र्दृश॒मिति॑ सुवः - दृश᳚म् । ईशा॑नमिन्द्र । इ॒न्द्र॒ त॒स्थुषः॑ । त॒स्थुष॒ इति॑ त॒स्थुषः॑ ॥ त्वामित् । इद्धि । हि हवा॑महे । हवा॑महे सा॒ता । सा॒ता वाज॑स्य । वाज॑स्य का॒रवः॑ । का॒रव॒ इति॑ का॒रवः॑ ॥ त्वां ॅवृ॒त्रेषु॑ । वृ॒त्रेष्वि॑न्द्र । इ॒न्द्र॒ सत्प॑तिम् । सत्प॑ति॒म् नरः॑ । सत्प॑ति॒मिति॒ सत् - प॒ति॒म् । नर॒स्त्वाम् । त्वाम् काष्ठा॑सु । काष्ठा॒स्वर्व॑तः । अर्व॑त॒ इत्यर्व॑तः ॥ य द्यावः॑ । द्याव॑ इन्द्र । इ॒न्द्र॒ ते॒ । ते॒ श॒तम् । श॒तꣳ श॒तम् । श॒तम् भूमीः᳚ । भूमी॑रु॒त । उ॒त स्युः । स्युरिति॒ स्युः ॥ न त्वा᳚ । त्वा॒ व॒ज्रि॒न्न्॒ । व॒ज्रि॒न्थ् स॒हस्र᳚म् । स॒हस्रꣳ॒॒ सूर्याः᳚ । सूर्या॒ अनु॑ । अनु॒ न । न जा॒तम् । जा॒तम॑ष्ट । अ॒ष्ट॒ रोद॑सी । रोद॑सी॒ इति॒ रोद॑सी ॥ पिबा॒ सोम᳚म् । सोम॑मिन्द्र । इ॒न्द्र॒ मन्द॑तु । मन्द॑तु त्वा । त्वा॒ यम् । यम् ते᳚ । ते॒ सु॒षाव॑ । सु॒षाव॑ हर्यश्व । ह॒र्य॒श्वाद्रिः॑ । ह॒र्य॒श्वेति॑ हरि - अ॒श्व॒ । अद्रि॒रित्यद्रिः॑ । \newline

\textbf{Jatai Paata} \newline

1. अ॒स्य जग॑तो॒ जग॑तो अ॒स्यास्य जग॑तः । \newline
2. जग॑तः सुव॒र्दृशꣳ॑ सुव॒र्दृश॒म् जग॑तो॒ जग॑तः सुव॒र्दृश᳚म् । \newline
3. सु॒व॒र्दृश॒ मीशा॑न॒ मीशा॑नꣳ सुव॒र्दृशꣳ॑ सुव॒र्दृश॒ मीशा॑नम् । \newline
4. सु॒व॒र्दृश॒मिति॑ सुवः - दृश᳚म् । \newline
5. ईशा॑न मिन्द्रे॒ न्द्रेशा॑न॒ मीशा॑न मिन्द्र । \newline
6. इ॒न्द्र॒ त॒स्थुष॑ स्त॒स्थुष॑ इन्द्रे न्द्र त॒स्थुषः॑ । \newline
7. त॒स्थुष॒ इति॑ त॒स्थुषः॑ । \newline
8. त्वा मिदित् त्वाम् त्वा मित् । \newline
9. इ द्धि हीदि द्धि । \newline
10. हि हवा॑महे॒ हवा॑महे॒ हि हि हवा॑महे । \newline
11. हवा॑महे सा॒ता सा॒ता हवा॑महे॒ हवा॑महे सा॒ता । \newline
12. सा॒ता वाज॑स्य॒ वाज॑स्य सा॒ता सा॒ता वाज॑स्य । \newline
13. वाज॑स्य का॒रवः॑ का॒रवो॒ वाज॑स्य॒ वाज॑स्य का॒रवः॑ । \newline
14. का॒रव॒ इति॑ का॒रवः॑ । \newline
15. त्वां ॅवृ॒त्रेषु॑ वृ॒त्रेषु॒ त्वाम् त्वां ॅवृ॒त्रेषु॑ । \newline
16. वृ॒त्रे ष्वि॑न्द्रे न्द्र वृ॒त्रेषु॑ वृ॒त्रे ष्वि॑न्द्र । \newline
17. इ॒न्द्र॒ सत्प॑तिꣳ॒॒ सत्प॑ति मिन्द्रे न्द्र॒ सत्प॑तिम् । \newline
18. सत्प॑ति॒म् नरो॒ नरः॒ सत्प॑तिꣳ॒॒ सत्प॑ति॒म् नरः॑ । \newline
19. सत्प॑ति॒मिति॒ सत् - प॒ति॒म् । \newline
20. नर॒ स्त्वाम् त्वाम् नरो॒ नर॒ स्त्वाम् । \newline
21. त्वाम् काष्ठा॑सु॒ काष्ठा॑सु॒ त्वाम् त्वाम् काष्ठा॑सु । \newline
22. काष्ठा॒ स्वर्व॑तो॒ अर्व॑तः॒ काष्ठा॑सु॒ काष्ठा॒ स्वर्व॑तः । \newline
23. अर्व॑त॒ इत्यर्व॑तः । \newline
24. यद् द्यावो॒ द्यावो॒ यद् यद् द्यावः॑ । \newline
25. द्याव॑ इन्द्रे न्द्र॒ द्यावो॒ द्याव॑ इन्द्र । \newline
26. इ॒न्द्र॒ ते॒ त॒ इ॒न्द्रे॒ न्द्र॒ ते॒ । \newline
27. ते॒ श॒तꣳ श॒तम् ते॑ ते श॒तम् । \newline
28. श॒तꣳ श॒तम् । \newline
29. श॒तम् भूमी॒र् भूमीः᳚ श॒तꣳ श॒तम् भूमीः᳚ । \newline
30. भूमी॑ रु॒तोत भूमी॒र् भूमी॑ रु॒त । \newline
31. उ॒त स्युः स्यु रु॒तोत स्युः । \newline
32. स्युरिति॒ स्युः । \newline
33. न त्वा᳚ त्वा॒ न न त्वा᳚ । \newline
34. त्वा॒ व॒ज्रि॒न्॒. व॒ज्रि॒न् त्वा॒ त्वा॒ व॒ज्रि॒न्न् । \newline
35. व॒ज्रि॒न् थ्स॒हस्रꣳ॑ स॒हस्रं॑ ॅवज्रिन्. वज्रिन् थ्स॒हस्र᳚म् । \newline
36. स॒हस्रꣳ॒॒ सूर्याः॒ सूर्याः᳚ स॒हस्रꣳ॑ स॒हस्रꣳ॒॒ सूर्याः᳚ । \newline
37. सूर्या॒ अन्वनु॒ सूर्याः॒ सूर्या॒ अनु॑ । \newline
38. अनु॒ न नान्वनु॒ न । \newline
39. न जा॒तम् जा॒तम् न न जा॒तम् । \newline
40. जा॒त म॑ष्टाष्ट जा॒तम् जा॒त म॑ष्ट । \newline
41. अ॒ष्ट॒ रोद॑सी॒ रोद॑सी अष्टाष्ट॒ रोद॑सी । \newline
42. रोद॑सी॒ इति॒ रोद॑सी । \newline
43. पिबा॒ सोमꣳ॒॒ सोम॒म् पिब॒ पिबा॒ सोम᳚म् । \newline
44. सोम॑ मिन्द्रे न्द्र॒ सोमꣳ॒॒ सोम॑ मिन्द्र । \newline
45. इ॒न्द्र॒ मन्द॑तु॒ मन्द॑त्विन्द्रे न्द्र॒ मन्द॑तु । \newline
46. मन्द॑तु त्वा त्वा॒ मन्द॑तु॒ मन्द॑तु त्वा । \newline
47. त्वा॒ यं ॅयम् त्वा᳚ त्वा॒ यम् । \newline
48. यम् ते॑ ते॒ यं ॅयम् ते᳚ । \newline
49. ते॒ सु॒षाव॑ सु॒षाव॑ ते ते सु॒षाव॑ । \newline
50. सु॒षाव॑ हर्यश्व हर्यश्व सु॒षाव॑ सु॒षाव॑ हर्यश्व । \newline
51. ह॒र्य॒श्वाद्रि॒ रद्रि॑र्. हर्यश्व हर्य॒श्वाद्रिः॑ । \newline
52. ह॒र्य॒श्वेति॑ हरि - अ॒श्व॒ । \newline
53. अद्रि॒रित्यद्रिः॑ । \newline

\textbf{Ghana Paata } \newline

1. अ॒स्य जग॑तो॒ जग॑तो अ॒स्यास्य जग॑तः सुव॒र्दृशꣳ॑ सुव॒र्दृश॒म् जग॑तो अ॒स्यास्य जग॑तः सुव॒र्दृश᳚म् । \newline
2. जग॑तः सुव॒र्दृशꣳ॑ सुव॒र्दृश॒म् जग॑तो॒ जग॑तः सुव॒र्दृश॒ मीशा॑न॒ मीशा॑नꣳ सुव॒र्दृश॒म् जग॑तो॒ जग॑तः सुव॒र्दृश॒ मीशा॑नम् । \newline
3. सु॒व॒र्दृश॒ मीशा॑न॒ मीशा॑नꣳ सुव॒र्दृशꣳ॑ सुव॒र्दृश॒ मीशा॑न मिन्द्रे॒ न्द्रेशा॑नꣳ सुव॒र्दृशꣳ॑ सुव॒र्दृश॒ मीशा॑न मिन्द्र । \newline
4. सु॒व॒र्दृश॒मिति॑ सुवः - दृश᳚म् । \newline
5. ईशा॑न मिन्द्रे॒ न्द्रेशा॑न॒ मीशा॑न मिन्द्र त॒स्थुष॑ स्त॒स्थुष॑ इ॒न्द्रेशा॑न॒ मीशा॑न मिन्द्र त॒स्थुषः॑ । \newline
6. इ॒न्द्र॒ त॒स्थुष॑ स्त॒स्थुष॑ इन्द्रे न्द्र त॒स्थुषः॑ । \newline
7. त॒स्थुष॒ इति॑ त॒स्थुषः॑ । \newline
8. त्वा मिदित् त्वाम् त्वा मिद्धि हीत् त्वाम् त्वा मिद्धि । \newline
9. इद्धि हीदिद्धि हवा॑महे॒ हवा॑महे॒ हीदिद्धि हवा॑महे । \newline
10. हि हवा॑महे॒ हवा॑महे॒ हि हि हवा॑महे सा॒ता सा॒ता हवा॑महे॒ हि हि हवा॑महे सा॒ता । \newline
11. हवा॑महे सा॒ता सा॒ता हवा॑महे॒ हवा॑महे सा॒ता वाज॑स्य॒ वाज॑स्य सा॒ता हवा॑महे॒ हवा॑महे सा॒ता वाज॑स्य । \newline
12. सा॒ता वाज॑स्य॒ वाज॑स्य सा॒ता सा॒ता वाज॑स्य का॒रवः॑ का॒रवो॒ वाज॑स्य सा॒ता सा॒ता वाज॑स्य का॒रवः॑ । \newline
13. वाज॑स्य का॒रवः॑ का॒रवो॒ वाज॑स्य॒ वाज॑स्य का॒रवः॑ । \newline
14. का॒रव॒ इति॑ का॒रवः॑ । \newline
15. त्वां ॅवृ॒त्रेषु॑ वृ॒त्रेषु॒ त्वाम् त्वां ॅवृ॒त्रेष्वि॑न्द्रे न्द्र वृ॒त्रेषु॒ त्वाम् त्वां ॅवृ॒त्रेष्वि॑न्द्र । \newline
16. वृ॒त्रेष्वि॑न्द्रे न्द्र वृ॒त्रेषु॑ वृ॒त्रेष्वि॑न्द्र॒ सत्प॑तिꣳ॒॒ सत्प॑ति मिन्द्र वृ॒त्रेषु॑ वृ॒त्रेष्वि॑न्द्र॒ सत्प॑तिम् । \newline
17. इ॒न्द्र॒ सत्प॑तिꣳ॒॒ सत्प॑ति मिन्द्रे न्द्र॒ सत्प॑ति॒म् नरो॒ नरः॒ सत्प॑ति मिन्द्रे न्द्र॒ सत्प॑ति॒म् नरः॑ । \newline
18. सत्प॑ति॒म् नरो॒ नरः॒ सत्प॑तिꣳ॒॒ सत्प॑ति॒म् नर॒ स्त्वाम् त्वाम् नरः॒ सत्प॑तिꣳ॒॒ सत्प॑ति॒म् नर॒ स्त्वाम् । \newline
19. सत्प॑ति॒मिति॒ सत् - प॒ति॒म् । \newline
20. नर॒ स्त्वाम् त्वाम् नरो॒ नर॒ स्त्वाम् काष्ठा॑सु॒ काष्ठा॑सु॒ त्वाम् नरो॒ नर॒ स्त्वाम् काष्ठा॑सु । \newline
21. त्वाम् काष्ठा॑सु॒ काष्ठा॑सु॒ त्वाम् त्वाम् काष्ठा॒स्वर्व॑तो॒ अर्व॑तः॒ काष्ठा॑सु॒ त्वाम् त्वाम् काष्ठा॒स्वर्व॑तः । \newline
22. काष्ठा॒स्वर्व॑तो॒ अर्व॑तः॒ काष्ठा॑सु॒ काष्ठा॒स्वर्व॑तः । \newline
23. अर्व॑त॒ इत्यर्व॑तः । \newline
24. यद् द्यावो॒ द्यावो॒ यद् यद् द्याव॑ इन्द्रे न्द्र॒ द्यावो॒ यद् यद् द्याव॑ इन्द्र । \newline
25. द्याव॑ इन्द्रे न्द्र॒ द्यावो॒ द्याव॑ इन्द्र ते त इन्द्र॒ द्यावो॒ द्याव॑ इन्द्र ते । \newline
26. इ॒न्द्र॒ ते॒ त॒ इ॒न्द्रे॒ न्द्र॒ ते॒ श॒तꣳ श॒तम् त॑ इन्द्रे न्द्र ते श॒तम् । \newline
27. ते॒ श॒तꣳ श॒तम् ते॑ ते श॒तम् । \newline
28. श॒तꣳ श॒तम् । \newline
29. श॒तम् भूमी॒र् भूमीः᳚ श॒तꣳ श॒तम् भूमी॑ रु॒तोत भूमीः᳚ श॒तꣳ श॒तम् भूमी॑रु॒त । \newline
30. भूमी॑ रु॒तोत भूमी॒र् भूमी॑ रु॒त स्युः स्युरु॒त भूमी॒र् भूमी॑ रु॒त स्युः । \newline
31. उ॒त स्युः स्यु रु॒तोत स्युः । \newline
32. स्युरिति॒ स्युः । \newline
33. न त्वा᳚ त्वा॒ न न त्वा॑ वज्रिन्. वज्रिन् त्वा॒ न न त्वा॑ वज्रिन्न् । \newline
34. त्वा॒ व॒ज्रि॒न्॒. व॒ज्रि॒न् त्वा॒ त्वा॒ व॒ज्रि॒न् थ्स॒हस्रꣳ॑ स॒हस्रं॑ ॅवज्रिन् त्वा त्वा वज्रिन् थ्स॒हस्र᳚म् । \newline
35. व॒ज्रि॒न् थ्स॒हस्रꣳ॑ स॒हस्रं॑ ॅवज्रिन्. वज्रिन् थ्स॒हस्रꣳ॒॒ सूर्याः॒ सूर्याः᳚ स॒हस्रं॑ ॅवज्रिन्. वज्रिन् थ्स॒हस्रꣳ॒॒ सूर्याः᳚ । \newline
36. स॒हस्रꣳ॒॒ सूर्याः॒ सूर्याः᳚ स॒हस्रꣳ॑ स॒हस्रꣳ॒॒ सूर्या॒ अन्वनु॒ सूर्याः᳚ स॒हस्रꣳ॑ स॒हस्रꣳ॒॒ सूर्या॒ अनु॑ । \newline
37. सूर्या॒ अन्वनु॒ सूर्याः॒ सूर्या॒ अनु॒ न नानु॒ सूर्याः॒ सूर्या॒ अनु॒ न । \newline
38. अनु॒ न नान्वनु॒ न जा॒तम् जा॒तम् नान्वनु॒ न जा॒तम् । \newline
39. न जा॒तम् जा॒तम् न न जा॒त म॑ष्टाष्ट जा॒तम् न न जा॒त म॑ष्ट । \newline
40. जा॒त म॑ष्टाष्ट जा॒तम् जा॒त म॑ष्ट॒ रोद॑सी॒ रोद॑सी अष्ट जा॒तम् जा॒त म॑ष्ट॒ रोद॑सी । \newline
41. अ॒ष्ट॒ रोद॑सी॒ रोद॑सी अष्टाष्ट॒ रोद॑सी । \newline
42. रोद॑सी॒ इति॒ रोद॑सी । \newline
43. पिबा॒ सोमꣳ॒॒ सोम॒म् पिब॒ पिबा॒ सोम॑ मिन्द्रे न्द्र॒ सोम॒म् पिब॒ पिबा॒ सोम॑ मिन्द्र । \newline
44. सोम॑ मिन्द्रे न्द्र॒ सोमꣳ॒॒ सोम॑ मिन्द्र॒ मन्द॑तु॒ मन्द॑त्विन्द्र॒ सोमꣳ॒॒ सोम॑ मिन्द्र॒ मन्द॑तु । \newline
45. इ॒न्द्र॒ मन्द॑तु॒ मन्द॑त्विन्द्रे न्द्र॒ मन्द॑तु त्वा त्वा॒ मन्द॑त्विन्द्रे न्द्र॒ मन्द॑तु त्वा । \newline
46. मन्द॑तु त्वा त्वा॒ मन्द॑तु॒ मन्द॑तु त्वा॒ यं ॅयम् त्वा॒ मन्द॑तु॒ मन्द॑तु त्वा॒ यम् । \newline
47. त्वा॒ यं ॅयम् त्वा᳚ त्वा॒ यम् ते॑ ते॒ यम् त्वा᳚ त्वा॒ यम् ते᳚ । \newline
48. यम् ते॑ ते॒ यं ॅयम् ते॑ सु॒षाव॑ सु॒षाव॑ ते॒ यं ॅयम् ते॑ सु॒षाव॑ । \newline
49. ते॒ सु॒षाव॑ सु॒षाव॑ ते ते सु॒षाव॑ हर्यश्व हर्यश्व सु॒षाव॑ ते ते सु॒षाव॑ हर्यश्व । \newline
50. सु॒षाव॑ हर्यश्व हर्यश्व सु॒षाव॑ सु॒षाव॑ हर्य॒श्वा द्रि॒ रद्रि॑र्. हर्यश्व सु॒षाव॑ सु॒षाव॑ हर्य॒श्वाद्रिः॑ । \newline
51. ह॒र्य॒श्वा द्रि॒ रद्रि॑र्. हर्यश्व हर्य॒श्वाद्रिः॑ । \newline
52. ह॒र्य॒श्वेति॑ हरि - अ॒श्व॒ । \newline
53. अद्रि॒रित्यद्रिः॑ । \newline
\pagebreak
\markright{ TS 2.4.14.4  \hfill https://www.vedavms.in \hfill}

\section{ TS 2.4.14.4 }

\textbf{TS 2.4.14.4 } \newline
\textbf{Samhita Paata} \newline

सो॒तुर्बा॒हुभ्याꣳ॒॒ सुय॑तो॒ नार्वा᳚ ॥ रे॒वती᳚र्नः सध॒माद॒ इन्द्रे॑ सन्तु तु॒विवा॑जाः । क्षु॒मन्तो॒ याभि॒र्मदे॑म ॥ उद॑ग्ने॒ शुच॑य॒स्तव॒>1 , वि ज्योति॒षो>2,दु॒ त्यं जा॒तवे॑दसꣳ>3स॒प्त त्वा॑ ह॒रितो॒ रथे॒ वह॑न्ति देव सूर्य । शो॒चिष्के॑शं ॅविचक्षण ॥ चि॒त्रं दे॒वाना॒मुद॑गा॒दनी॑कं॒ चक्षु॑र्मि॒त्रस्य॒ वरु॑णस्या॒ऽग्नेः । आऽप्रा॒ द्यावा॑पृथि॒वी अ॒न्तरि॑क्षꣳ॒॒ सूर्य॑ आ॒त्मा जग॑तस्त॒स्थुष॑ - [  ] \newline

\textbf{Pada Paata} \newline

सो॒तुः । बा॒हुभ्या॒मिति॑ बा॒हु - भ्या॒म् । सुय॑त॒ इति॒ सु - य॒तः॒ । न । अर्वा᳚ ॥ रे॒वतीः᳚ । नः॒ । स॒ध॒माद॒ इति॑ सध - मादः॑ । इन्द्रे᳚ । स॒न्तु॒ । तु॒विवा॑जा॒ इति॑ तु॒वि - वा॒जाः॒ ॥ क्षु॒मन्तः॑ । याभिः॑ । मदे॑म ॥ उदिति॑ । अ॒ग्ने॒ । शुच॑यः । तव॑ । वीति॑ । ज्योति॑षा । उदिति॑ । उ॒ । त्यम् । जा॒तवे॑दस॒मिति॑ जा॒त - वे॒द॒स॒म् । स॒प्त । त्वा॒ । ह॒रितः॑ । रथे᳚ । वह॑न्ति । दे॒व॒ । सू॒र्य॒ ॥ शो॒चिष्के॑श॒मिति॑ शो॒चिः - के॒श॒म् । वि॒च॒क्ष॒णेति॑ वि -  च॒क्ष॒ण॒ ॥ चि॒त्रम् । दे॒वाना᳚म् । उदिति॑ । अ॒गा॒त् । अनी॑कम् । चक्षुः॑ । मि॒त्रस्य॑ । वरु॑णस्य ।   अ॒ग्नेः ॥ एति॑ । अ॒प्राः॒ । द्यावा॑पृथि॒वी इति॒ द्यावा᳚ - पृ॒थि॒वी । अ॒न्तरि॑क्षम् । सूर्यः॑ । आ॒त्मा ।   जग॑तः । त॒स्थुषः॑ ।  \newline


\textbf{Krama Paata} \newline

सो॒तुर् बा॒हुभ्या᳚म् । बा॒हुभ्याꣳ॒॒ सुय॑तः । बा॒हुभ्या॒मिति॑ बा॒हु - भ्या॒म् । सुय॑तो॒ न । सुय॑त॒ इति॒ सु - य॒तः॒ । नार्वा᳚ । अर्वेत्यर्वा᳚ ॥ रे॒वती᳚र् नः । नः॒ स॒ध॒मादः॑ । स॒ध॒माद॒ इन्द्रे᳚ । स॒ध॒माद॒ इति॑ सध - मादः॑ । इन्द्रे॑ सन्तु । स॒न्तु॒ तु॒विवा॑जाः । तु॒विवा॑जा॒ इति॑ तु॒वि - वा॒जाः॒ ॥ क्षु॒मन्तो॒ याभिः॑ । याभि॒र् मदे॑म । मदे॒मेति॒ मदे॒म ॥ उद॑ग्ने । अ॒ग्ने॒ शुच॑यः । शुच॑य॒स्तव॑ । तव॒ वि । वि ज्योति॑षा । ज्योति॒षोत् । उदु॑ । उ॒ त्यम् । त्यम् जा॒तवे॑दसम् । जा॒तवे॑दसꣳ स॒प्त । जा॒तवे॑दस॒मिति॑ जा॒त - वे॒द॒स॒म् । स॒प्त त्वा᳚ । त्वा॒ ह॒रितः॑ । ह॒रितो॒ रथे᳚ । रथे॒ वह॑न्ति । वह॑न्ति देव । दे॒व॒ सू॒र्य॒ । सू॒र्येति॑ सूर्य ॥ शो॒चिष्के॑शं ॅविचक्षण । शो॒चिष्के॑श॒मिति॑ शो॒चिः - के॒श॒म् । वि॒च॒क्ष॒णेति॑ वि - च॒क्ष॒ण॒ ॥ चि॒त्रम् दे॒वाना᳚म् । दे॒वाना॒मुत् । उद॑गात् । अ॒गा॒दनी॑कम् । अनी॑क॒म् चक्षुः॑ । चक्षु॑र् मि॒त्रस्य॑ । मि॒त्रस्य॒ वरु॑णस्य । वरु॑णस्या॒ग्नेः । अ॒ग्नेरित्य॒ग्नेः ॥ आऽप्राः᳚ । अ॒प्रा॒ द्यावा॑पृथि॒वी । द्यावा॑पृथि॒वी अ॒न्तरि॑क्षम् । द्यावा॑पृथि॒वी इति॒ द्यावा᳚ - पृ॒थि॒वी । अ॒न्तरि॑क्षꣳ॒॒ सूर्यः॑ । सूर्य॑ आ॒त्मा । आ॒त्मा जग॑तः । जग॑तस्त॒स्थुषः॑ ( ) । त॒स्थुष॑श्च \newline

\textbf{Jatai Paata} \newline

1. सो॒तुर् बा॒हुभ्या᳚म् बा॒हुभ्याꣳ॑ सो॒तुः सो॒तुर् बा॒हुभ्या᳚म् । \newline
2. बा॒हुभ्याꣳ॒॒ सुय॑तः॒ सुय॑तो बा॒हुभ्या᳚म् बा॒हुभ्याꣳ॒॒ सुय॑तः । \newline
3. बा॒हुभ्या॒मिति॑ बा॒हु - भ्या॒म् । \newline
4. सुय॑तो॒ न न सुय॑तः॒ सुय॑तो॒ न । \newline
5. सुय॑त॒ इति॒ सु - य॒तः॒ । \newline
6. नार्वा ऽर्वा॒ न नार्वा᳚ । \newline
7. अर्वेत्यर्वा᳚ । \newline
8. रे॒वती᳚र् नो नो रे॒वती॑ रे॒वती᳚र् नः । \newline
9. नः॒ स॒ध॒मादः॑ सध॒मादो॑ नो नः सध॒मादः॑ । \newline
10. स॒ध॒माद॒ इन्द्र॒ इन्द्रे॑ सध॒मादः॑ सध॒माद॒ इन्द्रे᳚ । \newline
11. स॒ध॒माद॒ इति॑ सध - मादः॑ । \newline
12. इन्द्रे॑ सन्तु स॒न्त्विन्द्र॒ इन्द्रे॑ सन्तु । \newline
13. स॒न्तु॒ तु॒विवा॑जा स्तु॒विवा॑जाः सन्तु सन्तु तु॒विवा॑जाः । \newline
14. तु॒विवा॑जा॒ इति॑ तु॒वि - वा॒जाः॒ । \newline
15. क्षु॒मन्तो॒ याभि॒र् याभिः॑ क्षु॒मन्तः॑ क्षु॒मन्तो॒ याभिः॑ । \newline
16. याभि॒र् मदे॑म॒ मदे॑म॒ याभि॒र् याभि॒र् मदे॑म । \newline
17. मदे॒मेति॒ मदे॑म । \newline
18. उद॑ग्ने अग्न॒ उदुद॑ग्ने । \newline
19. अ॒ग्ने॒ शुच॑यः॒ शुच॑यो अग्ने अग्ने॒ शुच॑यः । \newline
20. शुच॑य॒ स्तव॒ तव॒ शुच॑यः॒ शुच॑य॒ स्तव॑ । \newline
21. तव॒ वि वि तव॒ तव॒ वि । \newline
22. वि ज्योति॑षा॒ ज्योति॑षा॒ वि वि ज्योति॑षा । \newline
23. ज्योति॒षो दुज् ज्योति॑षा॒ ज्योति॒षोत् । \newline
24. उदु॑ वु॒ वुदुदु॑ । \newline
25. उ॒ त्यम् त्य मु॑ वु॒ त्यम् । \newline
26. त्यम् जा॒तवे॑दसम् जा॒तवे॑दस॒म् त्यम् त्यम् जा॒तवे॑दसम् । \newline
27. जा॒तवे॑दसꣳ स॒प्त स॒प्त जा॒तवे॑दसम् जा॒तवे॑दसꣳ स॒प्त । \newline
28. जा॒तवे॑दस॒मिति॑ जा॒त - वे॒द॒स॒म् । \newline
29. स॒प्त त्वा᳚ त्वा स॒प्त स॒प्त त्वा᳚ । \newline
30. त्वा॒ ह॒रितो॑ ह॒रित॑ स्त्वा त्वा ह॒रितः॑ । \newline
31. ह॒रितो॒ रथे॒ रथे॑ ह॒रितो॑ ह॒रितो॒ रथे᳚ । \newline
32. रथे॒ वह॑न्ति॒ वह॑न्ति॒ रथे॒ रथे॒ वह॑न्ति । \newline
33. वह॑न्ति देव देव॒ वह॑न्ति॒ वह॑न्ति देव । \newline
34. दे॒व॒ सू॒र्य॒ सू॒र्य॒ दे॒व॒ दे॒व॒ सू॒र्य॒ । \newline
35. सू॒र्येति॑ सूर्य । \newline
36. शो॒चिष्के॑शं ॅविचक्षण विचक्षण शो॒चिष्के॑शꣳ शो॒चिष्के॑शं ॅविचक्षण । \newline
37. शो॒चिष्के॑श॒मिति॑ शो॒चिः - के॒श॒म् । \newline
38. वि॒च॒क्ष॒णेति॑ वि - च॒क्ष॒ण॒ । \newline
39. चि॒त्रम् दे॒वाना᳚म् दे॒वाना᳚म् चि॒त्रम् चि॒त्रम् दे॒वाना᳚म् । \newline
40. दे॒वाना॒ मुदुद् दे॒वाना᳚म् दे॒वाना॒ मुत् । \newline
41. उद॑गा दगा॒ दुदु द॑गात् । \newline
42. अ॒गा॒ दनी॑क॒ मनी॑क मगा दगा॒ दनी॑कम् । \newline
43. अनी॑क॒म् चक्षु॒ श्चक्षु॒ रनी॑क॒ मनी॑क॒म् चक्षुः॑ । \newline
44. चक्षु॑र् मि॒त्रस्य॑ मि॒त्रस्य॒ चक्षु॒ श्चक्षु॑र् मि॒त्रस्य॑ । \newline
45. मि॒त्रस्य॒ वरु॑णस्य॒ वरु॑णस्य मि॒त्रस्य॑ मि॒त्रस्य॒ वरु॑णस्य । \newline
46. वरु॑णस्या॒ग्ने र॒ग्नेर् वरु॑णस्य॒ वरु॑णस्या॒ग्नेः । \newline
47. अ॒ग्नेरित्य॒ग्नेः । \newline
48. आ ऽप्रा॑ अप्रा॒ आ ऽप्राः᳚ । \newline
49. अ॒प्रा॒ द्यावा॑पृथि॒वी द्यावा॑पृथि॒वी अ॑प्रा अप्रा॒ द्यावा॑पृथि॒वी । \newline
50. द्यावा॑पृथि॒वी अ॒न्तरि॑क्ष म॒न्तरि॑क्ष॒म् द्यावा॑पृथि॒वी द्यावा॑पृथि॒वी अ॒न्तरि॑क्षम् । \newline
51. द्यावा॑पृथि॒वी इति॒ द्यावा᳚ - पृ॒थि॒वी । \newline
52. अ॒न्तरि॑क्षꣳ॒॒ सूर्यः॒ सूर्यो॒ ऽन्तरि॑क्ष म॒न्तरि॑क्षꣳ॒॒ सूर्यः॑ । \newline
53. सूर्य॑ आ॒त्मा ऽऽत्मा सूर्यः॒ सूर्य॑ आ॒त्मा । \newline
54. आ॒त्मा जग॑तो॒ जग॑त आ॒त्मा ऽऽत्मा जग॑तः । \newline
55. जग॑त स्त॒स्थुष॑ स्त॒स्थुषो॒ जग॑तो॒ जग॑त स्त॒स्थुषः॑ । \newline
56. त॒स्थुष॑श्च च त॒स्थुष॑ स्त॒स्थुष॑श्च । \newline

\textbf{Ghana Paata } \newline

1. सो॒तुर् बा॒हुभ्या᳚म् बा॒हुभ्याꣳ॑ सो॒तुः सो॒तुर् बा॒हुभ्याꣳ॒॒ सुय॑तः॒ सुय॑तो बा॒हुभ्याꣳ॑ सो॒तुः सो॒तुर् बा॒हुभ्याꣳ॒॒ सुय॑तः । \newline
2. बा॒हुभ्याꣳ॒॒ सुय॑तः॒ सुय॑तो बा॒हुभ्या᳚म् बा॒हुभ्याꣳ॒॒ सुय॑तो॒ न न सुय॑तो बा॒हुभ्या᳚म् बा॒हुभ्याꣳ॒॒ सुय॑तो॒ न । \newline
3. बा॒हुभ्या॒मिति॑ बा॒हु - भ्या॒म् । \newline
4. सुय॑तो॒ न न सुय॑तः॒ सुय॑तो॒ नार्वा ऽर्वा॒ न सुय॑तः॒ सुय॑तो॒ नार्वा᳚ । \newline
5. सुय॑त॒ इति॒ सु - य॒तः॒ । \newline
6. नार्वा ऽर्वा॒ न नार्वा᳚ । \newline
7. अर्वेत्यर्वा᳚ । \newline
8. रे॒वती᳚र् नो नो रे॒वती॑ रे॒वती᳚र् नः सध॒मादः॑ सध॒मादो॑ नो रे॒वती॑ रे॒वती᳚र् नः सध॒मादः॑ । \newline
9. नः॒ स॒ध॒मादः॑ सध॒मादो॑ नो नः सध॒माद॒ इन्द्र॒ इन्द्रे॑ सध॒मादो॑ नो नः सध॒माद॒ इन्द्रे᳚ । \newline
10. स॒ध॒माद॒ इन्द्र॒ इन्द्रे॑ सध॒मादः॑ सध॒माद॒ इन्द्रे॑ सन्तु स॒न्त्विन्द्रे॑ सध॒मादः॑ सध॒माद॒ इन्द्रे॑ सन्तु । \newline
11. स॒ध॒माद॒ इति॑ सध - मादः॑ । \newline
12. इन्द्रे॑ सन्तु स॒न्त्विन्द्र॒ इन्द्रे॑ सन्तु तु॒विवा॑जा स्तु॒विवा॑जाः स॒न्त्विन्द्र॒ इन्द्रे॑ सन्तु तु॒विवा॑जाः । \newline
13. स॒न्तु॒ तु॒विवा॑जा स्तु॒विवा॑जाः सन्तु सन्तु तु॒विवा॑जाः । \newline
14. तु॒विवा॑जा॒ इति॑ तु॒वि - वा॒जाः॒ । \newline
15. क्षु॒मन्तो॒ याभि॒र् याभिः॑ क्षु॒मन्तः॑ क्षु॒मन्तो॒ याभि॒र् मदे॑म॒ मदे॑म॒ याभिः॑ क्षु॒मन्तः॑ क्षु॒मन्तो॒ याभि॒र् मदे॑म । \newline
16. याभि॒र् मदे॑म॒ मदे॑म॒ याभि॒र् याभि॒र् मदे॑म । \newline
17. मदे॒मेति॒ मदे॑म । \newline
18. उद॑ग्ने अग्न॒ उदुद॑ग्ने॒ शुच॑यः॒ शुच॑यो अग्न॒ उदुद॑ग्ने॒ शुच॑यः । \newline
19. अ॒ग्ने॒ शुच॑यः॒ शुच॑यो अग्ने अग्ने॒ शुच॑य॒ स्तव॒ तव॒ शुच॑यो अग्ने अग्ने॒ शुच॑य॒ स्तव॑ । \newline
20. शुच॑य॒ स्तव॒ तव॒ शुच॑यः॒ शुच॑य॒ स्तव॒ वि वि तव॒ शुच॑यः॒ शुच॑य॒ स्तव॒ वि । \newline
21. तव॒ वि वि तव॒ तव॒ वि ज्योति॑षा॒ ज्योति॑षा॒ वि तव॒ तव॒ वि ज्योति॑षा । \newline
22. वि ज्योति॑षा॒ ज्योति॑षा॒ वि वि ज्योति॒षोदुज् ज्योति॑षा॒ वि वि ज्योति॒षोत् । \newline
23. ज्योति॒षोदुज् ज्योति॑षा॒ ज्योति॒षोदु॑ वु॒ वुज् ज्योति॑षा॒ ज्योति॒षोदु॑ । \newline
24. उदु॑ वु॒ वुदुदु॒ त्यम् त्य मु॒ वुदुदु॒ त्यम् । \newline
25. उ॒ त्यम् त्य मु॑ वु॒ त्यम् जा॒तवे॑दसम् जा॒तवे॑दस॒म् त्य मु॑ वु॒ त्यम् जा॒तवे॑दसम् । \newline
26. त्यम् जा॒तवे॑दसम् जा॒तवे॑दस॒म् त्यम् त्यम् जा॒तवे॑दसꣳ स॒प्त स॒प्त जा॒तवे॑दस॒म् त्यम् त्यम् जा॒तवे॑दसꣳ स॒प्त । \newline
27. जा॒तवे॑दसꣳ स॒प्त स॒प्त जा॒तवे॑दसम् जा॒तवे॑दसꣳ स॒प्त त्वा᳚ त्वा स॒प्त जा॒तवे॑दसम् जा॒तवे॑दसꣳ स॒प्त त्वा᳚ । \newline
28. जा॒तवे॑दस॒मिति॑ जा॒त - वे॒द॒स॒म् । \newline
29. स॒प्त त्वा᳚ त्वा स॒प्त स॒प्त त्वा॑ ह॒रितो॑ ह॒रित॑ स्त्वा स॒प्त स॒प्त त्वा॑ ह॒रितः॑ । \newline
30. त्वा॒ ह॒रितो॑ ह॒रित॑ स्त्वा त्वा ह॒रितो॒ रथे॒ रथे॑ ह॒रित॑ स्त्वा त्वा ह॒रितो॒ रथे᳚ । \newline
31. ह॒रितो॒ रथे॒ रथे॑ ह॒रितो॑ ह॒रितो॒ रथे॒ वह॑न्ति॒ वह॑न्ति॒ रथे॑ ह॒रितो॑ ह॒रितो॒ रथे॒ वह॑न्ति । \newline
32. रथे॒ वह॑न्ति॒ वह॑न्ति॒ रथे॒ रथे॒ वह॑न्ति देव देव॒ वह॑न्ति॒ रथे॒ रथे॒ वह॑न्ति देव । \newline
33. वह॑न्ति देव देव॒ वह॑न्ति॒ वह॑न्ति देव सूर्य सूर्य देव॒ वह॑न्ति॒ वह॑न्ति देव सूर्य । \newline
34. दे॒व॒ सू॒र्य॒ सू॒र्य॒ दे॒व॒ दे॒व॒ सू॒र्य॒ । \newline
35. सू॒र्येति॑ सूर्य । \newline
36. शो॒चिष्के॑शं ॅविचक्षण विचक्षण शो॒चिष्के॑शꣳ शो॒चिष्के॑शं ॅविचक्षण । \newline
37. शो॒चिष्के॑श॒मिति॑ शो॒चिः - के॒श॒म् । \newline
38. वि॒च॒क्ष॒णेति॑ वि - च॒क्ष॒ण॒ । \newline
39. चि॒त्रम् दे॒वाना᳚म् दे॒वाना᳚म् चि॒त्रम् चि॒त्रम् दे॒वाना॒ मुदुद् दे॒वाना᳚म् चि॒त्रम् चि॒त्रम् दे॒वाना॒ मुत् । \newline
40. दे॒वाना॒ मुदुद् दे॒वाना᳚म् दे॒वाना॒ मुद॑गा दगा॒ दुद् दे॒वाना᳚म् दे॒वाना॒ मुद॑गात् । \newline
41. उद॑गा दगा॒ दुदु द॑गा॒ दनी॑क॒ मनी॑क मगा॒ दुदु द॑गा॒ दनी॑कम् । \newline
42. अ॒गा॒ दनी॑क॒ मनी॑क मगा दगा॒ दनी॑क॒म् चक्षु॒ श्चक्षु॒ रनी॑क मगा दगा॒ दनी॑क॒म् चक्षुः॑ । \newline
43. अनी॑क॒म् चक्षु॒ श्चक्षु॒ रनी॑क॒ मनी॑क॒म् चक्षु॑र् मि॒त्रस्य॑ मि॒त्रस्य॒ चक्षु॒ रनी॑क॒ मनी॑क॒म् चक्षु॑र् मि॒त्रस्य॑ । \newline
44. चक्षु॑र् मि॒त्रस्य॑ मि॒त्रस्य॒ चक्षु॒ श्चक्षु॑र् मि॒त्रस्य॒ वरु॑णस्य॒ वरु॑णस्य मि॒त्रस्य॒ चक्षु॒ श्चक्षु॑र् मि॒त्रस्य॒ वरु॑णस्य । \newline
45. मि॒त्रस्य॒ वरु॑णस्य॒ वरु॑णस्य मि॒त्रस्य॑ मि॒त्रस्य॒ वरु॑णस्या॒ग्ने र॒ग्नेर् वरु॑णस्य मि॒त्रस्य॑ मि॒त्रस्य॒ वरु॑णस्या॒ग्नेः । \newline
46. वरु॑णस्या॒ग्ने र॒ग्नेर् वरु॑णस्य॒ वरु॑णस्या॒ग्नेः । \newline
47. अ॒ग्नेरित्य॒ग्नेः । \newline
48. आ ऽप्रा॑ अप्रा॒ आ ऽप्रा॒ द्यावा॑पृथि॒वी द्यावा॑पृथि॒वी अ॑प्रा॒ आ ऽप्रा॒ द्यावा॑पृथि॒वी । \newline
49. अ॒प्रा॒ द्यावा॑पृथि॒वी द्यावा॑पृथि॒वी अ॑प्रा अप्रा॒ द्यावा॑पृथि॒वी अ॒न्तरि॑क्ष म॒न्तरि॑क्ष॒म् द्यावा॑पृथि॒वी अ॑प्रा अप्रा॒ द्यावा॑पृथि॒वी अ॒न्तरि॑क्षम् । \newline
50. द्यावा॑पृथि॒वी अ॒न्तरि॑क्ष म॒न्तरि॑क्ष॒म् द्यावा॑पृथि॒वी द्यावा॑पृथि॒वी अ॒न्तरि॑क्षꣳ॒॒ सूर्यः॒ सूर्यो॒ ऽन्तरि॑क्ष॒म् द्यावा॑पृथि॒वी द्यावा॑पृथि॒वी अ॒न्तरि॑क्षꣳ॒॒ सूर्यः॑ । \newline
51. द्यावा॑पृथि॒वी इति॒ द्यावा᳚ - पृ॒थि॒वी । \newline
52. अ॒न्तरि॑क्षꣳ॒॒ सूर्यः॒ सूर्यो॒ ऽन्तरि॑क्ष म॒न्तरि॑क्षꣳ॒॒ सूर्य॑ आ॒त्मा ऽऽत्मा सूर्यो॒ ऽन्तरि॑क्ष म॒न्तरि॑क्षꣳ॒॒ सूर्य॑ आ॒त्मा । \newline
53. सूर्य॑ आ॒त्मा ऽऽत्मा सूर्यः॒ सूर्य॑ आ॒त्मा जग॑तो॒ जग॑त आ॒त्मा सूर्यः॒ सूर्य॑ आ॒त्मा जग॑तः । \newline
54. आ॒त्मा जग॑तो॒ जग॑त आ॒त्मा ऽऽत्मा जग॑त स्त॒स्थुष॑ स्त॒स्थुषो॒ जग॑त आ॒त्मा ऽऽत्मा जग॑त स्त॒स्थुषः॑ । \newline
55. जग॑त स्त॒स्थुष॑ स्त॒स्थुषो॒ जग॑तो॒ जग॑त स्त॒स्थुष॑श्च च त॒स्थुषो॒ जग॑तो॒ जग॑त स्त॒स्थुष॑श्च । \newline
56. त॒स्थुष॑श्च च त॒स्थुष॑ स्त॒स्थुष॑श्च । \newline
\pagebreak
\markright{ TS 2.4.14.5  \hfill https://www.vedavms.in \hfill}

\section{ TS 2.4.14.5 }

\textbf{TS 2.4.14.5 } \newline
\textbf{Samhita Paata} \newline

श्च ॥ विश्वे॑ दे॒वा ऋ॑ता॒वृध॑ ऋ॒तुभि॑र्. हवन॒श्रुतः॑ । जु॒षन्तां॒ ॅयुज्यं॒ पयः॑ ॥ विश्वे॑ देवाः शृणु॒तेमꣳ हवं॑ मे॒ ये अ॒न्तरि॑क्षे॒ य उप॒ द्यवि॒ष्ठ । ये अ॑ग्निजि॒ह्वा उ॒त वा॒ यज॑त्रा आ॒सद्या॒स्मिन् ब॒र्॒.हिषि॑ मादयद्ध्वं ॥ \newline

\textbf{Pada Paata} \newline

च॒ ॥ विश्वे᳚ । दे॒वाः । ऋ॒ता॒वृध॒ इत्यृ॑त - वृधः॑ । ऋ॒तुभि॒रित्यृ॒तु-भिः॒ । ह॒व॒न॒श्रुत॒ इति॑ हवन - श्रुतः॑ ॥ जु॒षन्ता᳚म् । युज्य᳚म् ।    पयः॑ ॥ विश्वे᳚ । दे॒वाः॒ । शृ॒णु॒त । इ॒मम् । हव᳚म् । मे॒ । ये । अ॒न्तरि॑क्षे । ये । उपेति॑ । द्यवि॑ । स्थ ॥ ये । अ॒ग्नि॒जि॒ह्वा इत्य॑ग्नि - जि॒ह्वाः । उ॒त । वा॒ । यज॑त्राः । आ॒सद्येत्या᳚ - सद्य॑ । अ॒स्मिन्न् । ब॒र्॒.हिषि॑ । मा॒द॒य॒द्ध्व॒म् ॥  \newline


\textbf{Krama Paata} \newline

चेति॑ च ॥ विश्वे॑ दे॒वाः । दे॒वा ऋ॑ता॒वृधः॑ । ऋ॒ता॒वृध॑ ऋ॒तुभिः॑ । ऋ॒ता॒वृध॒ इत्यृ॑त - वृधः॑ ॥ ऋ॒तुभि॑र्. हवन॒श्रुतः॑ । ऋ॒तुभि॒रत्यृ॒तु - भिः॒ । ह॒व॒न॒श्रुत॒ इति॑ हवन - श्रुतः॑ ॥ जु॒षन्तां॒ ॅयुज्य᳚म् । युज्य॒म् पयः॑ । पय॒ इति॒ पयः॑ ॥ विश्वे॑ देवाः । दे॒वाः॒ शृ॒णु॒त । शृ॒णु॒तेमम् । इ॒मꣳ हव᳚म् । हव॑म् मे । मे॒ ये । ये अ॒न्तरि॑क्षे । अ॒न्तरि॑क्षे॒ ये । य उप॑ । उप॒ द्यवि॑ । द्यवि॒ ष्ठ । स्थेति॒ स्थ ॥ ये अ॑ग्निजि॒ह्वाः । अ॒ग्नि॒जि॒ह्वा उ॒त । अ॒ग्नि॒जि॒ह्वा इत्य॑ग्नि - जि॒ह्वाः । उ॒त वा᳚ । वा॒ यज॑त्राः । यज॑त्रा आ॒सद्य॑ । आ॒सद्या॒स्मिन्न् । आ॒सद्येत्या᳚ - सद्य॑ । अ॒स्मिन् ब॒र्॒.हिषि॑ । ब॒र्॒.हिषि॑ मादयद्ध्वम् । मा॒द॒य॒द्ध्व॒मिति॑ मादयद्ध्वम् । \newline

\textbf{Jatai Paata} \newline

1. चेति॑ च । \newline
2. विश्वे॑ दे॒वा दे॒वा विश्वे॒ विश्वे॑ दे॒वाः । \newline
3. दे॒वा ऋ॑ता॒वृध॑ ऋता॒वृधो॑ दे॒वा दे॒वा ऋ॑ता॒वृधः॑ । \newline
4. ऋ॒ता॒वृध॑ ऋ॒तुभिर्॑. ऋ॒तुभिर्॑. ऋता॒वृध॑ ऋता॒वृध॑ ऋ॒तुभिः॑ । \newline
5. ऋ॒ता॒वृध॒ इत्यृ॑त - वृधः॑ । \newline
6. ऋ॒तुभि॑र्. हवन॒श्रुतो॑ हवन॒श्रुत॑ ऋ॒तुभिर्॑. ऋ॒तुभि॑र्. हवन॒श्रुतः॑ । \newline
7. ऋ॒तुभि॒रित्यृ॒तु - भिः॒ । \newline
8. ह॒व॒न॒श्रुत॒ इति॑ हवन - श्रुतः॑ । \newline
9. जु॒षन्तां॒ ॅयुज्यं॒ ॅयुज्य॑म् जु॒षन्ता᳚म् जु॒षन्तां॒ ॅयुज्य᳚म् । \newline
10. युज्य॒म् पयः॒ पयो॒ युज्यं॒ ॅयुज्य॒म् पयः॑ । \newline
11. पय॒ इति॒ पयः॑ । \newline
12. विश्वे॑ देवा देवा॒ विश्वे॒ विश्वे॑ देवाः । \newline
13. दे॒वाः॒ शृ॒णु॒त शृ॑णु॒त दे॑वा देवाः शृणु॒त । \newline
14. शृ॒णु॒ते म मि॒मꣳ शृ॑णु॒त शृ॑णु॒ते मम् । \newline
15. इ॒मꣳ हवꣳ॒॒ हव॑ मि॒म मि॒मꣳ हव᳚म् । \newline
16. हव॑म् मे मे॒ हवꣳ॒॒ हव॑म् मे । \newline
17. मे॒ ये ये मे॑ मे॒ ये । \newline
18. ये अ॒न्तरि॑क्षे अ॒न्तरि॑क्षे॒ ये ये अ॒न्तरि॑क्षे । \newline
19. अ॒न्तरि॑क्षे॒ ये ये अ॒न्तरि॑क्षे अ॒न्तरि॑क्षे॒ ये । \newline
20. य उपोप॒ ये य उप॑ । \newline
21. उप॒ द्यवि॒ द्यव्युपोप॒ द्यवि॑ । \newline
22. द्यवि॒ ष्ठ स्थ द्यवि॒ द्यवि॒ ष्ठ । \newline
23. स्थेति॒स्थ । \newline
24. ये अ॑ग्निजि॒ह्वा अ॑ग्निजि॒ह्वा ये ये अ॑ग्निजि॒ह्वाः । \newline
25. अ॒ग्नि॒जि॒ह्वा उ॒तोता ग्नि॑जि॒ह्वा अ॑ग्निजि॒ह्वा उ॒त । \newline
26. अ॒ग्नि॒जि॒ह्वा इत्य॑ग्नि - जि॒ह्वाः । \newline
27. उ॒त वा॑ वो॒तोत वा᳚ । \newline
28. वा॒ यज॑त्रा॒ यज॑त्रा वा वा॒ यज॑त्राः । \newline
29. यज॑त्रा आ॒सद्या॒ सद्य॒ यज॑त्रा॒ यज॑त्रा आ॒सद्य॑ । \newline
30. आ॒सद्या॒स्मिन् न॒स्मिन् ना॒सद्या॒ सद्या॒स्मिन्न् । \newline
31. आ॒सद्येत्या᳚ - सद्य॑ । \newline
32. अ॒स्मिन् ब॒र्॒.हिषि॑ ब॒र्॒.हि ष्य॒स्मिन् न॒स्मिन् ब॒र्॒.हिषि॑ । \newline
33. ब॒र्॒.हिषि॑ मादयद्ध्वम् मादयद्ध्वम् ब॒र्॒.हिषि॑ ब॒र्॒.हिषि॑ मादयद्ध्वम् । \newline
34. मा॒द॒य॒द्ध्व॒मिति॑ मादयद्ध्वम् । \newline

\textbf{Ghana Paata } \newline

1. चेति॑ च । \newline
2. विश्वे॑ दे॒वा दे॒वा विश्वे॒ विश्वे॑ दे॒वा ऋ॑ता॒वृध॑ ऋता॒वृधो॑ दे॒वा विश्वे॒ विश्वे॑ दे॒वा ऋ॑ता॒वृधः॑ । \newline
3. दे॒वा ऋ॑ता॒वृध॑ ऋता॒वृधो॑ दे॒वा दे॒वा ऋ॑ता॒वृध॑ ऋ॒तुभिर्॑. ऋ॒तुभिर्॑. ऋता॒वृधो॑ दे॒वा दे॒वा ऋ॑ता॒वृध॑ ऋ॒तुभिः॑ । \newline
4. ऋ॒ता॒वृध॑ ऋ॒तुभिर्॑. ऋ॒तुभिर्॑. ऋता॒वृध॑ ऋता॒वृध॑ ऋ॒तुभि॑र्. हवन॒श्रुतो॑ हवन॒श्रुत॑ ऋ॒तुभिर्॑. ऋता॒वृध॑ ऋता॒वृध॑ ऋ॒तुभि॑र्. हवन॒श्रुतः॑ । \newline
5. ऋ॒ता॒वृध॒ इत्यृ॑त - वृधः॑ । \newline
6. ऋ॒तुभि॑र्. हवन॒श्रुतो॑ हवन॒श्रुत॑ ऋ॒तुभिर्॑. ऋ॒तुभि॑र्. हवन॒श्रुतः॑ । \newline
7. ऋ॒तुभि॒रित्यृ॒तु - भिः॒ । \newline
8. ह॒व॒न॒श्रुत॒ इति॑ हवन - श्रुतः॑ । \newline
9. जु॒षन्तां॒ ॅयुज्यं॒ ॅयुज्य॑म् जु॒षन्ता᳚म् जु॒षन्तां॒ ॅयुज्य॒म् पयः॒ पयो॒ युज्य॑म् जु॒षन्ता᳚म् जु॒षन्तां॒ ॅयुज्य॒म् पयः॑ । \newline
10. युज्य॒म् पयः॒ पयो॒ युज्यं॒ ॅयुज्य॒म् पयः॑ । \newline
11. पय॒ इति॒ पयः॑ । \newline
12. विश्वे॑ देवा देवा॒ विश्वे॒ विश्वे॑ देवाः शृणु॒त शृ॑णु॒त दे॑वा॒ विश्वे॒ विश्वे॑ देवाः शृणु॒त । \newline
13. दे॒वाः॒ शृ॒णु॒त शृ॑णु॒त दे॑वा देवाः शृणु॒ते म मि॒मꣳ शृ॑णु॒त दे॑वा देवाः शृणु॒ते मम् । \newline
14. शृ॒णु॒ते म मि॒मꣳ शृ॑णु॒त शृ॑णु॒ते मꣳ हवꣳ॒॒ हव॑ मि॒मꣳ शृ॑णु॒त शृ॑णु॒ते मꣳ हव᳚म् । \newline
15. इ॒मꣳ हवꣳ॒॒ हव॑ मि॒म मि॒मꣳ हव॑म् मे मे॒ हव॑ मि॒म मि॒मꣳ हव॑म् मे । \newline
16. हव॑म् मे मे॒ हवꣳ॒॒ हव॑म् मे॒ ये ये मे॒ हवꣳ॒॒ हव॑म् मे॒ ये । \newline
17. मे॒ ये ये मे॑ मे॒ ये अ॒न्तरि॑क्षे अ॒न्तरि॑क्षे॒ ये मे॑ मे॒ ये अ॒न्तरि॑क्षे । \newline
18. ये अ॒न्तरि॑क्षे अ॒न्तरि॑क्षे॒ ये ये अ॒न्तरि॑क्षे॒ ये ये अ॒न्तरि॑क्षे॒ ये ये अ॒न्तरि॑क्षे॒ ये । \newline
19. अ॒न्तरि॑क्षे॒ ये ये अ॒न्तरि॑क्षे अ॒न्तरि॑क्षे॒ य उपोप॒ ये अ॒न्तरि॑क्षे अ॒न्तरि॑क्षे॒ य उप॑ । \newline
20. य उपोप॒ ये य उप॒ द्यवि॒ द्यव्युप॒ ये य उप॒ द्यवि॑ । \newline
21. उप॒ द्यवि॒ द्यव्युपोप॒ द्यवि॒ ष्ठ स्थ द्यव्युपोप॒ द्यवि॒ ष्ठ । \newline
22. द्यवि॒ ष्ठ स्थ द्यवि॒ द्यवि॒ ष्ठ । \newline
23. स्थेति॒स्थ । \newline
24. ये अ॑ग्निजि॒ह्वा अ॑ग्निजि॒ह्वा ये ये अ॑ग्निजि॒ह्वा उ॒तोता ग्नि॑जि॒ह्वा ये ये अ॑ग्निजि॒ह्वा उ॒त । \newline
25. अ॒ग्नि॒जि॒ह्वा उ॒तोता ग्नि॑जि॒ह्वा अ॑ग्निजि॒ह्वा उ॒त वा॑ वो॒ता ग्नि॑जि॒ह्वा अ॑ग्निजि॒ह्वा उ॒त वा᳚ । \newline
26. अ॒ग्नि॒जि॒ह्वा इत्य॑ग्नि - जि॒ह्वाः । \newline
27. उ॒त वा॑ वो॒तोत वा॒ यज॑त्रा॒ यज॑त्रा वो॒तोत वा॒ यज॑त्राः । \newline
28. वा॒ यज॑त्रा॒ यज॑त्रा वा वा॒ यज॑त्रा आ॒सद्या॒ सद्य॒ यज॑त्रा वा वा॒ यज॑त्रा आ॒सद्य॑ । \newline
29. यज॑त्रा आ॒सद्या॒ सद्य॒ यज॑त्रा॒ यज॑त्रा आ॒सद्या॒स्मिन् न॒स्मिन् ना॒सद्य॒ यज॑त्रा॒ यज॑त्रा आ॒सद्या॒स्मिन्न् । \newline
30. आ॒सद्या॒स्मिन् न॒स्मिन् ना॒सद्या॒ सद्या॒स्मिन् ब॒र्॒.हिषि॑ ब॒र्॒.हिष्य॒स्मिन् ना॒सद्या॒ सद्या॒स्मिन् ब॒र्॒.हिषि॑ । \newline
31. आ॒सद्येत्या᳚ - सद्य॑ । \newline
32. अ॒स्मिन् ब॒र्॒.हिषि॑ ब॒र्॒.हिष्य॒स्मिन् न॒स्मिन् ब॒र्॒.हिषि॑ मादयद्ध्वम् मादयद्ध्वम् ब॒र्॒.हिष्य॒स्मिन् न॒स्मिन् ब॒र्॒.हिषि॑ मादयद्ध्वम् । \newline
33. ब॒र्॒.हिषि॑ मादयद्ध्वम् मादयद्ध्वम् ब॒र्॒.हिषि॑ ब॒र्॒.हिषि॑ मादयद्ध्वम् । \newline
34. मा॒द॒य॒द्ध्व॒मिति॑ मादयद्ध्वम् । \newline
\pagebreak


\end{document}